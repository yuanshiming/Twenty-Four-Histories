\article{志三十三}

\begin{pinyinscope}
地理五

△江蘇

江蘇:禹貢揚及徐、豫三州之域。明為南京。清順治二年改江南省,設布政使司,置兩江總督轄江南、江西,駐江寧。又設淮揚總督,尋裁。及江寧巡撫。治蘇州。又設鳳廬安徽巡撫,尋裁。十八年,分府九:安慶、徽州、寧國、池州、太平、廬州、鳳陽、淮安、揚州,直隸州四:徐、滁、和、廣德,屬安徽,江南左布政使領之。康熙元年,安徽設巡撫。三年,分江北按察使往治。五年,揚州、淮安、徐州復隸江南。六年,江南更今名,改左布政使為安徽布政使司,駐江寧。右布政使為江蘇布政使司,治蘇州。統江寧、蘇州、常州、松江、鎮江、揚州、淮安府七,徐州直隸州一。雍正二年,升太倉、邳、海、通四州為直隸州。十一年,徐州升府,邳還為州,屬之。乾隆二十五年,移安徽布政使司安慶,增設江寧布政使司,析江寧、淮安、徐、揚四府,通、海二直隸州屬之,與江蘇布政使司對治。三十二年,增海門直隸,屬江寧。光緒三十年,又設江淮巡撫,駐清江浦。尋復故。廣九百五十里,袤千一百三十里,積三十七萬二千五十四方里。北極高三十一度五分至三十五度十分。京師偏東五分至五度三分。宣統三年,編戶三百二十一萬三千四百八十三,口九百三十五萬六千七百五十五。領府八,直隸州三,直隸一,州三,四,縣六十。

江寧府:沖,繁,難。隸江寧道。明,應天府。江寧布政、交涉、提學三使,江安糧儲、江南勸業、巡警、鹽法四道,江寧將軍、副都統,織造兼督龍江西新稅關駐。順治初,因明制,縣八。雍正八年,改溧陽屬鎮江。北距京師二千四百四十五里。廣二百里,袤三百里。北極高三十二度四分。京師偏東二度二十八分。領縣七。上元沖,繁,難。倚。附郭有清涼山、師子山、富貴山。北:紫金山、幕府山。東:烏龍山、聖游山。有硃湖洞,道書三十一洞天。清江門內有小倉山。石城門內冶城山。南:大江自安徽當塗入,受秦淮河水,為草鞋夾。左與江浦分岸,得觀音山水。有燕子磯。秦淮河上承句容赤山湖水,合廬山水,逕通濟門,一入江寧,一入城。又西北流,至下關入江。新開河東北,乾隆四十五年濬,賜名便民。有市曰石埠橋。又東為黃天蕩。鎮四:淳化、靖安、土橋、石步。草鞋夾、燕子磯、棲霞街、湖熟有汛。一驛:金陵。淳化巡司。有鐵路。商埠:下關。光緒二十一年馬關條約四埠之一。江寧沖,繁,難。倚。南:聚寶山雨花臺設砲臺。大江西逕下關鎮。港七:銅井、烈山,北曰河口、綠新墅,又北大勝關,古新林浦也,西北曰北河,曰下關,分受秦淮河水。鎮三:江寧、秣陵、金陵。大勝、秣陵有汛。有驛。江寧、秣陵巡司二。有鐵路。句容沖,難。府東九十里。縣有句容山,以此名。北:華山。東北:銅山。東南:茅山。大江西來。港二:羅絲溝、下蜀港。赤山湖出絳巖山,秦淮水源於此,亦曰絳巖湖。匯亭水、黃堰河、蒲裏溪,曰南源,與北源合於白米湖,又西入上元。鎮五:白土、常寧、東陽、下蜀、龍潭。龍潭巡司。有驛。溧水簡。府東南一百四十里。南:芝山、中山一曰獨山。東:廬山,秦淮水別源所出。石臼湖西南,逕城北流入秦淮,明故運道也,今淤。一驛:孔家。江浦沖。府西北四十里。東北:十三公山、九連山。西:龍洞山。大江西南自安徽和州入,右與江寧分岸。為口四:曰烏江,曰老西江,曰新河,曰老河。受浦子口河,東北入六合。滁水右瀆自安徽滁州入,亦曰後河,東與來安分岸,復盡入境,曰前河,右出支津至東葛鎮,又東北逕岔河集,會沙河入六合。鎮三:高望、香泉、葛城。二驛:江淮、東葛。浦口巡司一。江淮有驛丞,裁。有鐵路。六合簡。府北一百二十里。南:瓜步山。西南:晉王山。大江西南自江浦入,右與上元分岸。折東南為通江集口支津,北抵城隍湖。有沙洲圩砲臺。又東劃子口。滁河西自江浦入,逕皁河口,北為汊河,又南屈曲流入江。稅課局大使駐。鎮四:瓜步、長蘆、宣化、竹鎮。有堂邑驛丞,裁。瓜步巡司一。高淳簡。府東南二百四十里。東:大游山。東北:荊山。南:固城湖,又東播為胥河。西:丹陽湖,北接石臼湖。有水自蕪湖東入丹陽湖,又東南入固城湖。或云禹貢中江也。鎮三:廣通、固城、水陽。廣通巡司一。

淮安府:沖,繁,疲,難。隸淮揚海道。順治初,因明制,州二,縣九。雍正二年,升海、邳為直隸州,贛榆、沭陽屬海,宿遷、睢寧屬邳。九年,析山陽、鹽城地置阜寧。南距省治五百里。廣三百八十里,袤二百七十里。北極高三十三度三十二分。京師偏東二度五十二分。領縣六。山陽沖,繁,疲,難。倚。漕標副將駐。北運河南流,烏沙、澗河諸水注之。東六草蕩,南白馬湖,匯洪澤湖水,與寶應錯,東北會於運河。北黃河故道。咸豐三年徙,今堰存。河所經南北岸,設同知、管河縣丞、主簿、巡檢,弁官廢置不常。咸豐十年裁。板徬鎮有鈔關巡司一。鎮二:北神、廟灣。菱陵、高堰、楊家廟有汛。驛一:淮陰。驛丞裁。阜寧繁,疲,難。府東北一百六十里。雍正九年置。東北:大海。有堰曰範公堤。射陽湖上承苔大縱湖水,匯淮水為湖,又東流,會諸水入海。運鹽河受射陽湖水,逕城南流,循範公堤入鹽城。西有黃河故道。鎮三:馬邏、北沙、蒙龍。草堰巡司一。鹽城繁,難。府東南二百四十里。東:大海。港二:新洋、鬥龍。有新興、五佑鹽場,鹽課大使駐。運鹽河自草堰口環城流,至便倉鎮入興化。苔大縱湖西南與興化錯。縣西諸水所匯。有天妃徬,徬官裁。小關、劉莊、新陽、沙溝有汛。鎮九:上岡、大岡、沙溝、岡門、新河、安豐、清溝、喻口、新興。上岡、沙溝巡司二。清河沖,繁,疲,難。淮揚道治所。江北提督、總兵駐。舊置總河,後省入總漕。自府城徙此,光緒三年裁。里河同知及河庫道均先後裁。府西北三十五里。北:清江浦。明陳瑄開,宋沙河也。運河西北自桃源入,歧為鹽河。又東為中河口,水經謂之中瀆水,出山陽白馬湖。又東迤南至清口屈而東,逕三徬,與清江浦合,東南入山陽,是為淮南運河。南:六塘河自桃源入,東北逕劉家莊入沭陽。鹽河東北流,逕西壩,淮安分司運判駐,乾隆二十八年移海州。又東至周莊入安東。西南:洪澤湖,西有黃河故道。鎮十:王家營、洪澤、老子、西壩、漁溝、官亭、大河口、澗橋、馬頭、周橋。王家營、馬頭、河北、漁溝有汛。一驛:清口。有驛丞,裁。澗橋巡司一。安東繁,疲,難。府東北六十里。西南鹽河自清河入,貫縣境,入海州,與六塘河合。東北:一帆河自海州入,南至旗桿村。水經,淮水東左右各合一水,至淮浦入海。東北:黃河故道。淮海河務兵備道駐,咸豐十年裁。鎮三:太平、長樂、魚場口。五港、佃湖有汛。佃湖巡司一。桃源沖,繁,難。府西北一百二十里。運河自宿遷南來,逕古城驛,入清河,歧為六塘河,一曰北鹽河,東北流入沭陽。洪澤湖西南與清河錯。西北有黃河故道。鎮七:三義、河北、崔鎮、眾興、張泗沖、白洋河、赤鯉湖。崔鎮、洋河、三義有汛。二驛:桃源、古城。驛丞裁。有巡司。

揚州府:沖,繁,疲,難。隸淮揚海道。兩淮鹽運使駐。順治初,因明制,州三,縣七。康熙十一年,海門圮于海,並通州。雍正三年,通州升直隸州,以如皋、泰興往屬。九年,析江都置甘泉。乾隆三十二年,析泰州置東臺。西南距省治二百十里。廣三百五十里,袤二百三十里。北極高三十二度二十七分。京師偏東二度五十六分。領州二,縣六。江都沖,繁,疲,難。倚。大江西自六合歷揚子入,東逕七濠口。監制同知駐。又東逕裕民洲,為夾江,歧為二。又東為三江口,東南流,與江合。三江口與天福洲對,設砲臺,守備駐。裁鹽務巡道。又東逕揚子港,入泰興。運河北入,環城南,逕新河灣,分流,西入揚子。又南流至瓜洲口,有砲臺。總兵駐。又東逕連城洲,分入江。鹽河導運河水東北入泰州,白塔龍兒河水注之。有榷關。鎮三:瓜洲、萬壽、宜陵。瓜洲、大橋、馬橋、沙洲有汛。廣陵驛丞裁。瓜洲、萬壽巡司二。甘泉沖,繁,疲,難。倚。雍正九年置。西北:蜀岡、甘泉山。北:邵伯湖,與高郵錯。運河合湖水南流,至壁虎橋入江都,綠洋湖、喬墅蕩分流入之。鎮三:邵伯、上官、大儀。一驛:邵伯。有汛。上官、邵伯二巡司。揚子沖,繁。府西南七十里。明為儀真。雍正二年,改「真」為「徵」。宣統元年,復曰揚子。西北有銅山、界墩山。南濱大江。西自六合入,有里世洲、沙漫洲二水自林家橋、王家壩北來注之。又西分流至泗源溝入江。稅課大使駐。新河出月塘集,西南流,亦入江。一鎮:新城。有水驛驛丞。清江芒稻河徬官,裁。青山、舊港、黃泥港有汛。舊江口巡司一。高郵州沖,繁。府北一百二十里。西南:神居山。運河北逕稅務橋,鹽河西流注之。又逕車邏壩,南澄子河注之,南匯為綠洋湖。馬霓河東南流,入於清水潭,受運河北洩諸水,東積為草蕩,三陽河南來注之。高郵湖西北,一曰甓社湖,北接界首湖,南赤岸湖,與甘泉錯。水高、永南有汛。二驛:界首、孟城。界首、時保巡司二。興化疲,難。府東北一百六十五里。東:大海,有堤。鹽河並堤流,西受界河、海溝、橫涇諸水,東出為大團河、八灶、七灶河,東北會鬥龍港,入於海。有劉莊、草堰、丁溪三場,鹽課大使駐。北有吳公湖、苔大蹤湖,與鹽城、寶應錯。石、白駒三徬,有徬官。鎮三:安豐、陵亭、芙蓉。安豐巡司一。寶應沖,繁。府北二百四十里。運河北自山陽入,逕八口鋪,東溢為瓦溝溪。又南流,逕汜水鎮,至界首,有界首湖,入高郵。其西寶應湖,匯淮流下瀦之水。苔大蹤湖東北,周二百里,分支入運河。衡陽有汛。一驛:安平,有驛丞,裁。衡陽、槐樓巡司二。泰州繁,疲,難。府東一百二十里。鹽河西自江都入,夾城東流,一曰里下河,有溱潼水注之。至白米鎮,左通串場河,右出支津,入泰興。又東逕海安鎮,左歧為界河,東南入如皋。鹽河東北自東臺入,西南流,逕淤溪達鰍魚港,又西南與之合。有泰壩,泰州分司運判駐。鮑湖東北。鎮四:海安、安鄉、斗門、樊汊。海安、曲塘有汛。海安、安鄉巡司二。東臺繁,疲。府東二百四十里。乾隆三十二年置。東:大海,有堤。鹽場七:東臺、何垛、梁垛,安豐、富安、角斜、拼茶。鹽課大使駐。又小海場大使,裁。里下河自泰州環城北流,又東溢為支河入海。鹽河出縣西海道徬,西南流,錯出復入,至淤溪入泰州。水利同知駐東臺場。草堰四徬有徬官。一鎮:西溪。巡司裁。王家港有汛。

徐州府:要,沖,繁,難。隸淮徐道。徐州鎮總兵駐。順治初為直隸州。領蕭、碭山、豐、沛。雍正十一年,升府。置銅山縣。又以降直隸邳州來隸,及所領宿遷、睢寧。東南距省治七百三十里。廣三百二十里,袤一百八十里。北極高三十四度五分。京師偏東五十八分。領州一,縣七。銅山沖,繁,難。倚。雍正十一年置。東北有銅山,故名。微山湖,東北出為荊山河,即引河,一曰徐州河,承湖水至卞塘入邳州,與運河合。資河一曰奎河,東南流入蕭縣。黃河故道西北。一鎮:卞塘。鄭集三堡有汛。利國、東岸二驛,驛丞裁。雙溝、利國巡司二。蕭簡,難。府西五十里。南:丁公山。西:岱山。西北為岱山湖。又東南有龍山河。資河自銅山入,東南逕軸山西,左出支津入靈壁,正渠入宿州。其西望川湖,逕大海子東,亦入宿。鎮二:白土、永安。一驛:桃山。張山店巡司一。碭山沖,繁,疲,難。府西北一百六十里。東北:芒碭山。利民溝一曰小神湖,東南流,屈而西,入永城。西沙河,西南逕鼎新集,入河南夏邑。城北為黃河故道。周家寨、蟠龍集有汛。豐簡。府西北一百五十里。東南:華山。新開河北流逕章固鎮,又北入魚臺。舊濬以導黃河,今堤存。豐水一曰泡河,班志泡水也,入泗,湮。一鎮:吳康。沛沖。府西北一百二十里。西:七山。有棲山圩,乾隆四十六年河決,縣沒,徙此。四十七年建城。咸豐元年河決,城復沒,遷夏鎮。十一年仍還舊治。東:微山湖。西有聶莊鋪小河口。運河自滕入,屈曲流入湖。泗水自山東魚臺入,亦曰南清河,受金溝水,為金溝渡,東合三河口水,自此入運。有彭口、楊莊二徬。徬官裁。夏鎮、棲山圩有汛。夏陽巡司一。邳州沖,難。府東北一百五十里。舊治下邳。康熙二十八年遷治艾山南。七年河決,移今治。雍正三年升直隸州。十一年來屬。南:葛嶧山,即距山。北:艾山、石埠山。西北:黃石山。運河自嶧錯入,逕泇口,岔河東北注之。至徐塘口合徐川河水,又南合沂水,入宿遷。武河,古武水,一曰治水,左通沂河,右入武原水,復出數支津,與燕子、柴溝等並入運。武原水即泇河,自蘭山入,東會沂水,達宿遷之黃墩湖,入黃河。城南有黃河故道。鎮三:直河、新安、泇口。姚灣、泇口有汛。舊城巡司一。宿遷沖,繁,難。府東一百里。北:司峿山、馬陵山。東:五華峰。南:斗山。運河自邳州入,南合六塘河水,入桃源。西北:駱馬湖,匯沂河、山澗諸水為巨浸。北:沭河自郯城入,南得桃花澗水,再錯沭陽,折東北,逕燕集圩仍入之。西南:故黃河,有堤。鎮三:白洋河、小河口、邵店。順河、司峿有汛。二驛:鍾吾、司峿。鍾吾有驛丞,裁。司峿巡司一。睢寧簡。府東南一百二十里。有池山、官山。西:九頂山。西南:峰山、荊山、英公山。東南:池山。白塘河出小李集,東南流,合沈家河,即今涸沙河,東入宿遷。又潼水,水經注所謂潼陂水入睢者,湮。鎮二:高紹、辛安。

通州直隸州:繁,難。隸常鎮通海道。順治初,因明制,屬揚州府。縣一,海門。康熙十一年,縣省。雍正二年,升直隸州,割揚州府之如皋、泰興來屬。西距省治五百三十里。廣三百里,袤百三十里。北極高三十二度三分。京師偏東四度十一分。領縣二。東:軍山、劍山。西:黃泥山、馬鞍山,五峰並峙。東北:天竺山。南:狼山,設砲臺。狼山鎮總兵駐。東北:大海,產鹽,置場五:呂四、餘東、餘西、金沙、石港,鹽課大使駐。又馬塘、餘中二場,乾隆元年裁。西亭場,三十三年裁。通州分司運判駐石港,稅課大使亦駐。南:大江西自如皋入,東行達老洪港,會於海。鹽河自如皋西入江,東分流,循城而南,又東入於海。鎮二:狼山、石港。石港、金沙、餘東、呂四有汛。狼山巡司裁。呂東巡司一。如皋繁,難。州西北一百二十里。東,瀕海。鹽場二:豐利、掘港。鹽課大使駐。大江西自靖江入,又東入通州,北通運鹽河。河西北自泰州入,循城南,分為二。一南流入江。一東逕丁堰,又分流,至岔河,為鹽場諸水。又南流,逕白蒲鎮入通州。鎮四:丁堰、掘港、豐利、白蒲。馬塘、豐利有汛。主簿駐掘港。西場、石莊巡司二。泰興疲,難。州西二百四十里。大江西北自江都入,右與丹徒分岸,為廟港。納李薛河,又南與丹陽分岸,東至界港。界河自靖江緣界而西入之,又東入靖江,分支為老龍河,至黃橋,折南注界河。黃橋有汛。口岸、黃橋、印莊,巡司三。

海州直隸州:繁,難。隸淮揚海道。順治初,因明制,屬淮安府。縣一。雍正二年升直隸州,又割淮安府之沭陽來屬。西南距省治八百二十里。廣百七十五里,袤百九十里。北極高三十四度二十三分。京師偏東二度五十六分。領縣二。東北:雲臺山,瀕海。東:高公島。西:金墅汛,設砲臺。北:鴨島、竹島。東北:鷹游山。鹽場三:中正、臨興、板浦。鹽課大使駐。又白駒、莞瀆二場,乾隆元年裁。海州分司運判駐板浦,有太平局、中富局、大義甿、富民甿、中興甿鹽垣。鹽河自安東入,逕新安鎮,合南北六塘河入海,其東支津與海通。西南:青伊湖、碩頃湖,北播為薔薇河。南有一帆河,受鹽河水入安東。鎮五:板浦、高橋、莞瀆、石湫、新壩。板浦、房山、吳家集有汛。高橋、惠澤巡司二。贛榆難。州北八十里。北:吳山。西:徐山、界山。東:蘭山。南:泊船山、武強山。東,瀕海,自山東日照入,有秦山望海墩,設砲臺。大沙河自郯城、青口河自莒,南流入海。興莊河水出西北吳山中。鎮四:臨洪、青口、荻水口、中岡站。青口巡司一。乾隆十六年,省荻水司改。沭陽難。州西南一百二十里。西北:張倉山。東北:韓山、萬山。沭河,古漣水,自宿遷入,東流為新挑河。後河循城東北入青伊湖,又南與沙礓河合,逕陽溝,六塘水注之,達於海。鎮六:湯溝、侯鎮、華沖、高流、陰平、劉莊。吳家集有汛。縣丞駐高流。

海門直隸:沖,繁。隸常鎮通海道。舊本沙洲。乾隆初,設沙務同知。三十三年,割通州之安慶、南安十九沙,崇明之半洋、富民十一沙,及天南沙,置。移蘇州府海防同知來治。西距省治五百七十里。廣一百四十里,袤三十七里。北極高三十一度五十五分。京師偏東四度四十五分。東南,瀕海。西,大江,西南自通州入,右與昭文分岸。又東錯崇明,折東北,由界洪復入,東南至蓼角嘴入海。白茆口為江海潮所會。界河承海水西流,環治而南,入於江。

蘇州府:最要。沖,繁,疲,難。分巡蘇州道治所。江蘇布政、提學、提法三使,巡警、勸業二道,織造兼督滸墅榷關駐。雍正八年,按察使自江寧移此。宣統二年改提法使。順治初,因明制,州一,縣七。雍正二年,升太倉為直隸州,割崇明、嘉定屬之。又析長洲置元和,昆山置新陽,常熟置昭文,吳江置震澤。乾隆元年,又設太湖。光緒三十年,設靖湖,隸府。北距京師二千七百里。廣二百里,袤二百四十里。北極高三十一度二十三分。京師偏東四度一分。領二,縣九。太湖府西南九十八里。乾隆元年置,移吳江同裏撫民同知來駐,治洞庭東山。東山一曰胥母山,有莫釐峰。太湖環治,積三萬六千頃。天目山水西南自浙之臨安、餘杭合苕、霅溪水,至大錢口;其西合宣、歙諸山水,逕長興箬溪,至小梅口,與宜興、荊溪諸縣水,西北匯為湖。又東北,播為吳淞江。又東為澱山湖,達黃浦入於海。甪頭、下揚灣村巡司二。靖湖簡。光緒三十年置,設撫民通判,治洞庭西山。有縹緲峰。吳沖,繁,疲,難。倚。南:橫山。西:皋峰、姑蘇靈巖山。東南:香山。西南有天平、楞伽、靈巖、穹窿、鄧尉諸山。西北:運河自浙江秀水歷吳江、元和入,受太湖水,自胥口東逕木瀆,與光福塘、箭涇諸水會,又逕跨塘至胥門,越來溪注之。北出為橫塘,與縣南占魚口水並入運河。商埠,城南青陽場,馬關條約四埠之一。鎮三:橫塘、橫涇、木瀆。縣丞駐木瀆。光福巡司一。乾隆十一年,省木瀆司改。長洲沖,繁,疲,難。倚。西:高景山。西北:卑猶山。西:運河自吳入,有寒山汛。西北逕射瀆,會金墅港水,又西北入無錫。射瀆水東出為長蕩,滸墅、烏角、白鶴諸溪並與運河合。婁江支津自元和緣界入,東北,左溢為尚澤蕩,右陽城、西湖,北後湖,逕南蕩,逕陸港折東入新陽。滸墅有榷關。鎮三:陸墓、蠡口、望亭。滸墅、黃埭有汛。吳塔巡司一。有鐵路。元和沖,繁,疲,難。倚。雍正二年置。東北:維亭山。西有虎丘。唐白居易鑿渠南達運河,今謂之山塘。東南:江寧山。吳淞江自吳江北迤東入新陽。運河亦自其縣入。其南:澄湖,溢為蕭澱湖,又東南為長白諸蕩。尹山湖,縣東南。其北:獨墅湖,有黃天蕩。又陽城湖東北西湖跨長洲。中湖、東湖俱與新陽錯。鎮二:甪直、維亭。沙河、周莊、章練塘有汛。二驛:姑蘇、望亭。縣丞二,駐甪直、章練塘。周莊巡司一。有鐵路。昆山疲,難。府東七十里。吳淞江東逕三江口,屈曲流入青浦。南有澱山湖,北溢為棋盤蕩、陳墓蕩,又北白蓮湖,歧為商羊潭、楊氏田湖,逕直港與吳淞合。致和塘水自元和環城流,東會新洋江入太倉。鎮三:安亭、泗橋、蓬閬。石浦巡司一。有鐵路。新陽疲,難。府東七十里。雍正二年置。西北;昆山、綽墩山。吳淞江自元和東入,復錯出。新洋江一曰新陽江,納吳淞水,北入致和塘。有傀儡湖、鰻鯉湖、巴城湖、雉城湖。巴城、雉城今湮。一鎮:兵墟。大王廟有汛。巴城巡司一。有鐵路。常熟繁,疲,難。府北九十里。蘇松糧儲道駐。乾隆三十二年移省。北:大江。福山與隔江狼山對,設砲臺,總兵駐。西北:崇德山、河陽山。西南:宛山。北:大江自江陰入,左與通州分岸,有捍海塘。元和塘水即運河,自長洲入,北逕福山塘。又黃泗浦水西北流,並入江。東北:大海。有塘。東南:昆承湖,一名隱湖,與尚湖相對,亦曰八字湖。鎮二:慶安、福山。鹿苑、唐市有汛。黃泗浦巡司一。昭文繁,難。府北九十里。雍正二年,析常熟東境置。東北:大江自常熟入,又東入太倉。其港口以許浦、白茆為大。白茆受吳中諸水。許浦北海舶出入長江道,此為深水。針路、白茆、許浦,及茜涇、下張七鴉,宋為昆山、常熟五大浦。自白茆嶽廟起,北至周涇口入江,長二千九百丈,亦名里睦塘。鎮二:梅李、許浦。薛家沙、支塘、徐六涇有汛。白茆巡司一。吳江沖,繁,難。府南四十里。北:吳淞江,占魚口水北流入之。運河二源,一南塘河,一官塘河,東匯為諸蕩,與汾湖合。龐山湖東受太湖水,溢為九里湖,又東同里湖,其南為葉澤湖,有元鶴、韓郎蕩。鶯脰湖,縣南。鎮三:簡村、八赤、盛澤。同里有汛。一驛:松陵。縣丞駐盛澤。汾湖、同裏巡司二。震澤繁,難。府南四十里。雍正二年置。東臨運河,自吳江入,至平望鎮,西塘河來會。西臨太湖,合諸港漾水注唐家湖,東入吳江。橫塘西導烏程諸水,歧為三,東與鶯脰湖會。橫塘之西曰震澤塘,東曰梅堰塘,為孔道。鎮二:平望、嚴幕。震澤有汛。平望、震澤巡司二。

松江府:要,繁,疲,難。隸蘇松太道。江南提督駐。順治初,因明制,縣三。十二年,析華亭置婁縣。雍正二年,又析華亭置奉賢,析上海置南匯,析青浦置福泉,改金山衛為縣。乾隆八年,福泉省。嘉慶十年,又析上海南匯地設川沙,隸府。西北距省治一百六十里。廣一百六十里,袤一百四十里。北極高三十一度。京師偏東四度二十七分。領一,縣七。川沙繁,疲,難。府東南二百四十里。故明川沙堡。乾隆二十四年,改董漕同知為川沙海防同知。嘉慶十年,析置為撫民同知。東:大海。有捍海塘三,曰外圩塘、欽公塘、東護塘。夾護塘河二。鹽河逕界濱入寶山。其左御寇河,椿樹浦水引黃浦東入,與鹽河合。三尖嘴、海中、曹家路有汛。華亭繁,疲,難。倚。東南有柘山、金山。海中有捍海塘。松江上承太湖,東逕笠澤,與東江、婁江而三。今婁江塞,而東江合松江出海,祗一江耳。黃浦江為吳淞支津,首受泖、澱諸水,屈曲流,大洋涇水會之。春申塘水東引黃浦支流,合千步涇,會於北俞塘。又分流逕顓橋入黃浦。柘林營東南,水利通判駐。有鹽場曰袁浦,大使駐。鎮五:亭林、葉謝、曹涇、柘林、沙岡。柘林、亭林、張澤有汛。都司駐柘林,縣丞駐曹涇。亭林巡司一。有鐵路。婁疲,難。倚。順治十二年置。西北有橫雲山、機山、天馬山。南:泖湖,源出華亭谷,與青浦金山錯,古三泖也。斜塘上承泖湖,自青浦入,東歧,合古浦塘及支津,貫城至華亭界,為南俞塘。其北出者為通波塘。斜塘東南合秀州塘、大蒸塘,入金山,為黃浦,又東入上海。有橫浦鹽場,大使駐。一鎮:楓涇。天馬鎮、泗涇、楓涇有汛。縣丞駐白龍潭。小徵巡司一。有鐵路。奉賢疲,難。府東九十里。明於華亭置青村所守御、千戶,隸金山衛。雍正二年析置。南,瀕海,有塘。有青村鹽場,大使駐。青村港,縣西,有汛。南橋塘水上游望河涇,自華亭引黃浦水東入姚涇,又東會蕭塘,為南橋塘,左得金匯塘,上承南匯界河水,又東為青村港。西有龍泉港,亦受望河涇,錯出復入,逕阮港鎮,折東抵柘林營而止。鎮三:陶宅、南橋、四團。縣丞駐四團。南橋巡司一。金山疲,難。府南七十二里。雍正二年置。故明金山衛,屬華亭。初治衛城,尋徙洙涇鎮。東南:秦山、查山。海中有金山,縣以此名。今隸華亭。東北:泖港,橫潦涇西流入之,匯平湖諸水,曰三秀塘。納秀州塘,逕城南,東達掘撻涇,南匯諸水合泖港入黃浦。南有鹽河,循衛城西溢為黃姑塘,歧為里界河、黃浦界河,並北流而合,至大泖港與橫潦涇會,又北為黃浦。折東與婁分岸,入華亭。有浦東鹽場,大使駐。典史駐衛城。一鎮:洙涇。張堰巡司一。洙涇、張堰、呂港有汛。上海沖,繁,疲,難。府東北九十里。蘇松太道駐。黃浦江自華亭入,夾城流,東北至虹口。吳淞江西北來與之合,又東北入於海。吳淞江自嘉定入,納盤龍浦水、橫瀝水,逕新涇,又東為古滬瀆,逕新閘北、泥城橋、老閘會黃浦江。西堧歐、美各國互市租界,道光二十三年英約五口通商之一。吳淞岸東北四十五里,光緒二十四年開為商埠,海舶殷輳,利盡東南。租界有會審公堂,理華、洋獄訟。有海關,蘇松太道監督。又南洋軍械制造局,西南。鎮四:吳淞、烏涇、吳會、閔行。塘橋、引翔港、閔行有汛。黃浦、吳淞巡司二。有鐵路。南匯繁,疲,難。府東一百二十里。雍正二年置。故明南匯守御所。東:大海。捍海塘二:內東護塘;外外護塘,即欽公塘。西黃浦江自華亭入,逕閘港,折北,左與上海分岸。縣西縱河曰鶴坡塘,在新陽鎮。會南七灶諸港水,至分水墩,是為港閘。西會金匯塘,入奉賢。縣號窮海,獨饒鹽。東護塘內有運鹽河,南自奉賢入。一鎮:下沙。置鹽場三,鹽課大使駐。周浦有汛。縣丞駐泥城。三林莊巡司一。青浦繁,疲,難。府西北五十里。東:鉶山、佘山。東南:鳳凰山、薛山。北:福泉山。西:盧山、辰山。北:吳淞江。澱山湖西受太湖水,播為諸蕩,南與泖湖合。北會硃家港水入於江。有趙屯浦、大盈浦、顧會浦、盤龍浦,俱分受吳淞水,入黃浦。鎮六:泗涇、金澤、硃家角、趙屯、七寶、白鶴江。北鉶山、小徵有汛。縣丞駐七寶。澱山、新涇巡司二。

太倉直隸州:繁,疲,難。隸蘇松太道。順治初,因明制,屬蘇州府,縣一。雍正二年,升直隸州,析州置鎮洋縣,又割蘇州府之嘉定屬之,析其地置寶山,同隸州。西南距省治一百二十里。廣一百五十里,袤一百四十里。北極高三十一度二十九分。京師偏東四度二十五分。領縣四。北有穿山。東北:大海,有塘。七鴉口設砲臺。一鎮:雙鳳。璜涇有汛。州同駐劉河鎮。七浦巡司一。昔太倉之水八百五十。南路之水,婁江獨任之。北路之水,七浦、楊林分任之。故七浦以輔婁江,楊林又以輔七浦。楊林南有湖川塘。湖川南硃涇,為古婁江北道。又貫南北者,有鹽鐵塘,南出吳淞入海。北道白茆達江。雍正中,發帑疏濬兩江,兼治白茆,以補三江之缺。鎮洋繁。倚。雍正二年置。東:大海。縣東劉河口,一曰婁河口,有汛。婁江入海處。禹貢中江也。「劉」即「婁」,聲近字。上承致河塘,自新陽入,為太倉塘。自城南南馬頭東合新塘港,又東入海。新塘港即舊湖川塘,逕小塘子入劉河。南:鹽鐵塘水環城流,西北與七浦塘合。有徬官,裁。茜涇河西抵漕塘河,東逕花雙入海。茜涇城,乾隆三年築。鎮二:沙頭、茜涇。甘草巡司一。崇明沖,繁。州東北五十七里。東:金鼇山、茶山。東北:海中設汛。海環縣治,港沙綺錯。有望海臺,當沙港南,與崇寶沙對,設砲臺,總兵駐。施翹河水西南夾城流,又東與十欃口合,入於海。東:鹽灘,有場,巡鹽大使駐。雍正八年,於縣設太通巡道。乾隆五年移通州,六年裁。鎮三:新鎮、豹貔、楊家河。上沙、中沙、外沙、下沙有汛。縣丞駐五欃。大安有廢巡司。崇海巡司一。嘉定疲,難。州南三十六里。初屬蘇州府。雍正二年來隸。東南:鶴槎山。吳淞江東入,緣界流,北為鹽鐵塘水,入鎮洋。縣北劉河,古婁江也。橫瀝水北流逕縣城,又東與之合。練祁塘水承吳淞西來,環城流,逕羅店,入寶山。鎮三:外岡、安亭、南翔。縣丞駐南翔,有汛。諸翟巡司一。有鐵路。寶山繁,疲,難。州東九十里。雍正二年置。故嘉定縣吳淞所,明寶山。東南有寶山故城。山北設汛。東瀕大海,有塘。南為吳淞口,黃浦江入海處,設砲臺,控扼東南,為軍港要塞。崇寶沙,海中,與崇明對。蘊藻濱水自嘉定逕陳行鎮,界涇水西北逕羅店,合練祁塘水會之。歧為二,東至胡巷口,南至虹口,並入黃浦。又北泗塘水引蘊藻濱水南迤東環城流,西有採綯港。鎮四:高橋、江灣、大場、羅店。舊砲臺、胡巷口、楊行、江灣、月浦有汛。縣丞駐高橋。有鐵路。

常州府:沖,繁,疲,難。隸常鎮通海道。順治初,因明制,縣五。雍正二年,總督查弼納以蘇、松、常賦重事繁,疏請太倉等十三州縣各析為二,析武進置陽湖,無錫置金匱,宜興置荊溪。東南距省治二百八十里。廣一百六十里,袤一百八十里。北極高三十一度五十二分。京師偏東三度二十四分。領縣八。武進沖,繁,疲,難。倚。府西偏。西北:黃山、固山。毗陵江西北自丹陽入,東南至桃花港入江陰。運河循城流,逕奔牛鎮入丹陽。滆湖北受運河,西受壇、溧、洮湖諸水,匯為湖,又西溢為大圩蕩,南與湖塘河會,入宜興。鎮三:奔牛、青城、阜通。西埠、孟河、魏村有汛。一驛:毗陵。奔牛、孟河巡司二。有鐵路。陽湖繁,難。倚。府東偏。雍正二年置。以縣東陽湖名。東:芳茂山。東北:舜山。南:太湖,有馬跡山,舊置寨,有汛。運河自無錫逕丁堰、戚墅堰,北商河水合舜河水東西分流入之。戚墅港合宋建湖,至白蕩歧為三,一東入無錫達閭江,一黃堰河達百跡,一薛堰河達下埠,並入太湖。其武進支津曰宜荊漕河,一曰西蠡河,西南流,會滆湖水,並湖行入宜興。一鎮:橫林。馬跡巡司一。有鐵路。無錫沖,繁。府東南九十里。北:九龍山。西:舜山、錫山。其東惠山,有泉。太湖,西南。又東溢為五里湖,南出為長廣溪,西逕吳塘門,仍入太湖。運河東南自長洲入,夾城流,東納漕河,即白塘圩,支津出江陰,首受大江,北流,逕高橋,與之合。鎮一:潘葑。一驛:錫山。清寧有汛。高橋巡司一。有鐵路。金匱繁,難。府東九十里。雍正二年置。以城內金匱山名。東北有斗山、膠山。北:橫山。南:夾山、前山。運河東南自長洲入,常昭漕河首受太湖,東緣長洲界,左與無錫分岸,環城入之。又分流,南北入常熟、江陰。又自東亭屈而西為百瀆港,東流會於鵝真蕩,與長洲錯。一鎮:望亭。黃埠墩有汛。巡司一。有鐵路。江陰繁,疲,難。府東七十里。江蘇學政駐,光緒三十一年裁。北:君山。東北:綺山、定山、黃山。東:馬鞍山。隔江與天生港對,有砲臺。北:大江西自武進入,漕河首受江水,逕四河口入無錫。應天河分漕河水,屈曲流,逕華墅東南,為南長河。橫河,城東至泗港北入江。有青草、壽星諸沙。鎮三:楊舍、夏港、申浦。沙洲、楊舍有汛。顧山巡司一。宜興疲,難。府南一百二十里。西北:有山、東北:羊山、金鵝、羅科山。西:大坯山。北:滆湖,與武進、陽湖錯,受長蕩湖水。其支津湖塘河自武進入,歧為二,至吳瀆口入於太湖。縣東有東氿、西氿,金壇、溧陽諸水會之。漕河北與二氿合,匯為羊山諸蕩。又東北為橫蕩,逕百瀆港入太湖。一鎮:楊港。和橋有汛,縣丞駐。鍾溪、下邾巡司二。荊溪疲,難。府南百二十一里。雍正二年置。南:荊溪,縣以此名。南:白雲、茗嶺、君山、啄木嶺。西:芙蓉山、國山。三國吳天璽元年封禪為中嶽,有摩崖,右群峰相繆不一名。東:銅官山。西南:章山。東南:茶山、蘭山。瀕太湖東西二氿,與宜興錯。楊港河、文定港水分流入之。其南沙河自溧陽戴步流並瀦焉。東南:蜀山河,合川步水,東歧為施塘,並注之。又東至大浦口,其南蓮花蕩自湖汊匯諸山水,至烏溪口,並入太湖。徐舍有汛。湖汊、張渚巡司二。靖江難。府東一百五十里。東北:孤山。南濱大江,西自泰興入,東:張黃港。右與江陰分岸,又東逕縣南入如皋。港口八。迤東歧為界河,折南至張黃港復合。港南紫氣河,漩洑深洪,海舶入江處。界河自港北環縣流,西達界港入於江。西有團河。鎮三:陳阜、生祠、新豐市。新港巡司一。

鎮江府:最要,沖,繁,疲,難。常鎮通海道治所。長江水師提督、京口副都統駐。順治初設鎮海將軍,乾隆二十八年裁。順治初,因明制,縣三。雍正八年,以江寧府之溧陽來屬。光緒三十年,又設太平,隸府。東南距省治三百七十里。廣二百十里,袤一百三十六里。北極高三十二度十二分。京師偏東二度五十七分。領一,縣四。太平簡。府東七十里。光緒三十年置,設撫民同知,治太平洲,江中。丹徒沖,繁,疲,難。倚。西北:金山,臨江,有中泠泉。北:北固山。焦山,江中,南北與象山、連城洲對,又東圌山、五峰山,隔江與高橋對,皆設砲臺。大江,城北逕孩溪,復南繞圌山,分支為大小夾江,有寶晉、天福、補沙諸洲。運河南自丹陽入,逕雩山西、洪山東,折西環城北流,所謂南運河。糧艘渡江入伊婁河,至邗溝,為北運河,並入於江。橫越徬有徬官。西:高資河,東西與新開河合。河為乾隆四十五年巡撫吳壇濬,出排灣西經高資入句容。商埠,縣北二里,外國互市租界,咸豐十年英法條約長江三口之一。有新關,常鎮道監督。鎮五:丹徒、高資、諫壁、大港、新豐。硃家圩有汛。二驛:京口、炭渚。京口有驛丞,裁。高資、安港、丹徒巡司三。有鐵路。丹陽繁,疲,難。府東南七十里。東北有九齡山。大江北自丹徒播為夾江,逕姚家橋入,東與江合。運河東南逕七里橋,漕河會之。又西南播為香草河。簡瀆河環城流,入於江。包港東北導運河水與夾江合。北有練湖。鎮二:呂城、延陵。一驛:雲陽。呂城、包港巡司二。有鐵路。金壇疲,難。府南一百六十里。西:茅山,一曰三茅峰。南:長蕩湖,與溧陽錯,古洮湖也。漕河環城為濠,南會於白龍蕩,又南受湖水入溧陽。薛步水出薛步鎮,東流分為二,一入漕河,一南與漕河遇,入於湖。東有錢資蕩。湖溪巡司一,裁。溧陽繁,疲。府南二百四十里。雍正八年來隸。西:曹姥山、鐵冶山一曰鐵峴。北:涪山,峙洮湖中,湖與金壇錯。三塔蕩西南溢為升平蕩。前馬蕩水出溧水廬山,合高淳諸水,東逕為蕩入中河,東南流與漕河合,古中江也。五代楊行密築五堰,江自是不復東,禹跡中湮矣。鎮三:舉善、甓橋、廣道。


\end{pinyinscope}