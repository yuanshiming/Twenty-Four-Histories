\article{列傳二百十五}

\begin{pinyinscope}
秦定三郝光甲鄭魁士傅振邦邱聯恩黃開榜

陳國瑞郭寶昌

秦定三,字竹坡,湖北興國人。道光六年武進士,授二等侍衛。出為廣西桂林營游擊,洊擢貴州鎮遠鎮總兵。三十年,平湖南李沅發之亂,賜號愨勇巴圖魯。咸豐元年,率貴州、雲南兵赴廣西剿匪,克武宣三里墟賊營。進剿象州,以賊竄逸,坐褫花翎,降三級留任。尋連破賊馬鞍山、竹園村,復之。偕副都統烏蘭泰破賊新墟,又奪雙髻山、豬仔峽要隘,被嘉獎。又擊賊於永安州,力戰受傷。二年,破水竇賊壘,賊棄永安潰圍走,擒賊首洪大泉。賊趨桂林,定三偕烏蘭泰追之。急不暇結營而戰,定三止之,勿聽,烏蘭泰以傷歿。定三代將其軍,克花礄。桂林尋解圍,以保守省城被優敘。追賊入湖南,破賊於道州桃花井、五里亭、龍安橋,進援長沙。總兵和春營妙高峰,為賊所圍,定三分兵襲賊營,得解。尋賊竄岳州,定三坐不能遏賊,革職留任。進援武昌,戰於洪山。三年,賊浮江東下,向榮率大軍由陸路追之,令和春及定三為前鋒。甫至九江,而江寧已陷。逾月大軍始至,迭戰城下,賊堅壁以拒。

四年,賊分黨陷廬州,和春疏調定三及鄭魁士率所部往助剿。時廬州久為賊踞,旁縣並陷,定三連戰破賊,復六安,屯三角井。會江寧賊分黨入安徽,圖北犯,以援畿南竄匪,道經舒城;賊首羅大綱、石達開、胡以晄、秦日昌等合眾數萬,四路來撲。定三所部僅二千,堅守十餘日,陣斬羅大綱,賊始挫,引去。定三集團勇攻舒城,悉破城外賊壘,又伺賊出截擊,連破之。圍之數月,六年,賊營火藥自焚,乘其亂,薄城奮攻,梯而登,遂復舒城,殲賊四千餘,予騎都尉世職。進屯軍鋪,賊自廬江、桐城分路來犯,定三往來馳擊,大破之,復五河、廬江二縣。進規桐城,奪小關、下關、白河嶺諸隘,屯陳家鋪。是年冬,賊由安慶來援,定三血戰十八日,賊乃退。又破賊於桐城北門外,毀其城樓。

捻匪擾河南,詔定三赴蒙城、亳州會剿,以鄭魁士代任桐城軍事。巡撫福濟疏言定三圍攻方得手,留之。改以魁士援北路,而魁士軍已至。定三初與魁士同列,及和春赴江南督師,魁士會辦安徽軍務,權位出定三上,又因爭餉,定三心不平,上疏劾之。福濟所恃惟兩軍,難左右袒,軍饑且渙。七年春,賊又陷廬江,進犯桐城。官軍為所圍,不戰而潰,坐褫翎頂。文宗知定三頻年苦戰,敗非其罪,原之,故薄譴,命赴江南大營,隸和春軍,屯句容。大軍方攻鎮江,令移駐溧水以遏援賊。尋卒於軍,詔念前勞,依例賜恤,謚恭武。

郝光甲,直隸任丘人。道光十八年一甲一名武進士,授頭等侍衛。出為山東撫標中軍參將,巡撫李僡薦之,超擢陜安鎮總兵。咸豐三年,率陜、甘兵援山東,從解懷慶圍。追賊至山西,破之於平陽。賊入畿輔,光甲從勝保追剿,陜甘總督舒興阿剿賊河南,互相爭調,光甲以擅自移營褫職。尋隨舒興阿援安徽,其軍改隸秦定三。戰舒城,迭破賊,詔予三品頂戴,署陜安鎮總兵。從克廬州,復舒城,復總兵頂戴,賜花翎。尋調赴河南剿捻,誤往徐州,被劾,革職留營。擊潁州捻匪於江集,擒捻首王鳳林。復以調赴蒙城遲延,降二級。七年,援桐城,兵敗,歿於陣。詔復原官,依總兵賜恤,予騎都尉世職,謚武節。

鄭魁士,直隸萬全人。由行伍洊擢湖南提標守備。道光三十年,平李沅發之亂,擢鎮筸鎮標都司。從提督向榮赴廣西剿匪,屢捷,賜花翎。擢湖南九谿營游擊,以參將升用。咸豐二年,守桂林,援長沙,擢副將,賜號沙拉瑪巴圖魯。援武昌,遂從向榮追賊沿江東下。以違軍令被劾,褫職留營。尋戰江寧有功,給都司翎頂。四年,提督和春調率所部赴廬州,進攻屢捷,復其職。尋署安徽壽春鎮總兵。廬州數縣皆陷,府城賊眾糧足,殊死守。和春一軍倚魁士及秦定三二人,定三分兵攻舒城;而廬州軍事專恃魁士。圍攻歷年餘,安慶、江寧援賊屢來援,皆擊走。至五年冬,攻愈急,魁士潛至城下以雲梯登城克之。被優敘,加提督銜。六年春,追賊至三河,焚其巢,而捻匪日熾。魁士率兵赴宿州擊破之,乃分路竄入河南境。巡撫英桂疏請魁士赴援永城,和春方倚辦皖賊,疏留,令往來策應。於是迭擊捻匪於懷遠茅塘集、河溜等處,擒其酋褚澱等四十餘人。又破之於蒙城,焚其積聚。駐守懷遠賊分隊來犯,魁士被圍,力戰,身被二十餘創,卒破賊,解圍去,詔嘉其勇,賜黃馬褂。又督團練敗賊於太和。會和春督師江南,詔安徽軍務以魁士繼任,會同巡撫福濟督辦,實授壽春鎮總兵。迭克舒城、廬江、無為,下部優敘,頒賜御用衣服及珍物。又以魁士躬冒鋒鏑,被創甚劇,特詔嘉獎,賜藥調治。先後分兵復和州、潛山。

先是秦定三攻桐城,賊堅守不下,魁士往會剿,迭戰,並擊退援賊。時悍賊石達開往來桐城、安慶,勢甚張;又勾通捻匪,蔓延皖、豫之間。詔秦定三移兵蒙城剿捻,尋又留攻桐城,以魁士代之,會同河南巡撫英桂節制三省剿捻之兵;而桐城兵事方棘,福濟復疏留不遣。值歲荒餉匱,定三軍原取給地方捐給,魁士兵至,悉取轉供。定三疏爭,福濟一無措置,兩軍遂成水火。詔促魁士速赴蒙城,亦迄未行。

七年春,廬江、潛山連陷,賊由安慶大舉來犯,城賊突出,官軍饑疲不相顧,不戰潰圍而走。於是詔褫魁士翎頂,罷其剿捻會辦,歸福濟節制。退保廬州,粵、捻各匪會合來犯,魁士迎擊挫之,復翎頂。尋克桃鎮、派河,進扼全椒、滁州以杜北竄。八年,調赴江南大營,授浙江提督,督辦寧國軍務。九年,克灣沚,進剿貴池、南陵。尋命駐防高淳、東壩。

十年,以傷病乞假,詔斥屢次退卻,以總兵降補。從漕運總督袁甲三剿賊,授甘肅寧夏鎮總兵。十一年,以病罷。尋召來京候簡。同治五年,捻匪北犯,命赴直隸東路協剿。六年,署直隸提督。八年,乞病歸。十二年,卒。大學士李鴻章疏陳魁士久於軍事,堅苦剛毅,疊受重傷,詔依例賜恤,謚忠烈。

傅振邦,山東昌邑人。道光十六年武進士,授三等侍衛。二十三年,出為湖南長沙協中軍都司,署鎮筸游擊。三十年,從平新寧土匪李沅發,受槍傷,賜花翎,實授游擊。咸豐二年,赴援桂林。三年,從向榮追賊抵江南,擢湖南撫標中軍參將。以圍攻江寧功,賜號綽克托巴圖魯。四年,擢貴州定廣協副將,署江蘇徐州鎮總兵。賊由蕪湖犯東壩,陷高淳。向榮令迎擊敗之,復其城。又偕鄧紹良克太平府,偪秣陵關,破賊於採石磯。六年,蒞徐州署任。捻酋張洛行、夏白、任乾圍宿州,振邦敗之夾溝、符離,解城圍。再敗張洛行於瓦子口,毀其巢。擊退蒙城賊於灘口。又偕伊興額破捻酋紀學中、王得六於永城鐵佛寺,毀柳集、臨渙集賊巢,擒紀學中,實授徐州鎮總兵。

會江南大營失利,命振邦馳援。偕總兵明安泰、秦如虎破賊東壩,進攻溧水。七年,克之。又破賊湖墅,追至龍都,偕張國樑克句容,加提督銜。八年,援寧國,拔灣沚、黃池,郡城解嚴。四月,回軍徐州,命幫辦袁甲三軍務。時捻匪蜂起,振邦馳逐江北、皖、豫之間,擒石得珍於山套;覆李大喜於符離;蹙孫葵心於茨河,歸德、陳州均肅清,以提督記名。九年,命代袁甲三督辦三省剿匪事,副都統伊興額副之。尋復命幫辦欽差大臣勝保軍,仍留督辦三省剿匪事。

澮北捻渠劉添福糾眾三萬圍團練苗沛霖營,振邦馳救,毀賊壘二十四。乘勝攻澮南,陣斬賊酋任乾,夷其圩,授雲南提督。蒙城王家圩諸圩聞任乾死,俱乞降,獨淝南板橋集賊陸連科負隅久抗。振邦設計招降黃家圩,李華東為內應,擒陸連科誅之,淝南北六十餘圩悉就撫。六月,賊陷定遠,振邦馳援,破賊於宿州。賊竄固鎮,破之於方家坎渡口。孫葵心竄唐家寨,窺濟寧,截擊之,賊退走。

十年,詔袁甲三代勝保為欽差大臣,振邦專任徐、宿剿匪事。捻匪屢窺徐、宿,其老巢袁、徐兩圩跨澮南、河北,振邦進剿,連破其沖要臨渙、韓村、趙家海、張圩,餘多自拔就撫。遂渡澮河攻袁圩。捻酋劉添福自豫回竄,擊敗之,再破之褚莊、邱家圩、檀城,五戰皆捷,殲賊六千有奇,擒其酋任護、任大牛。東路捻匪擾宿遷、睢寧,振邦戰於苗村,大破之。閏三月,偕田在田克閻圩,擒任虎、鄧三摩等誅之。復破援賊,擒李大喜。四月,連克澮南解溝、五溝、任圩賊巢,斬賊目李四喜、任友得三十餘名,收撫童亭、藕池四十二圩。五月,會攻袁圩。捻酋劉添祥等大舉來援,分軍擊之;而永城捻萬餘直趨童亭,窺孫甿大營,振邦令副將龔耀倫擊敗其眾,擒捻首趙學煥等。七月,拔蒙城西洋集賊圩十四。潁、亳捻首姜臺凌等北竄澮南,扼險截擊,擒賊目百餘。尋因傷發,請假回籍醫治,允之。十一年,命督練民團防堵登、萊、青三府,振邦病未已,疏辭防堵,請專任團練,報可。是年冬,命來京候簡。

同治元年,勝保奏調振邦幫辦皖、豫軍,為山東所留,不果行。二年,僧格林沁調統前軍,從攻淄川、白蓮池,援蒙城。三年,從破捻酋張總愚於湖北隨州。四年,以疾告歸,未幾疾愈,留督軍青、萊,移扼張秋河防。六年,會剿直隸梟匪,賊降復叛,褫翎頂。尋破賊夏津,復之。五年,西捻平,補直隸提督。光緒六年,調湖北。九年,以傷發回籍,未幾,卒於家,賜恤,謚剛勇。

邱聯恩,字偉堂,福建同安人,浙江提督良功子。襲男爵,授乾清門侍衛。道光二十三年,出為直隸通州協副將,調河間協。咸豐四年,從勝保剿粵匪於靜海李家莊,擊敗之,又破梁頭、孫家莊賊營,擢南陽鎮總兵。剿光州捻匪,擒其渠丁心田,賜花翎。五年,捻首李世林敗死,其黨易添富糾汝陽、息縣諸匪,戕烏龍集州判,陷息縣,據光山。聯恩督兵圍攻,賊宵遁,追擊,大破之,斃賊千餘,擒斬王黨、黃五雷等。

六年,皖捻首張洛行、龔瞎子等擾歸德,聯恩間道赴援,甫至,賊三路來撲,擊走之。尋以進剿遲延,革職留營。連破賊於穀熟集、界溝集,殲斃甚眾。進剿亳州五馬溝,大破之,殲賊千餘,擒賊目三十餘人,復原官。其冬,襄樊土匪起,入河南,陷鄧州、內鄉,聯恩馳擊,復其城,殲賊渠硃中立等,轄境得安。七年春,張洛行擁眾掠光州、固始,分據洪河南北。勝保大軍扼北岸,聯恩率兵千餘擊南岸,進攻方家集賊巢。五月,諸軍合擊,聯恩直搗賊壘,破圩而入,乘勝追殺,焚洪河橋,兩岸賊皆潰,殲斃三千餘。是役功最,賜號圖薩蘭巴圖魯。九月,剿角子山捻匪,都統德楞阿敗賊確山,聯恩乘勝躡擊。賊竄沁陽、嵩縣諸山中,搜捕數月,賊氛始清。八年,回軍援固始,圍尋解。粵匪犯湖北,陷麻城。聯恩扼沙窩坊、虎頭關,防光山、商城一路。十月,捻首孫葵心竄周家口,聯恩破之槐店。

九年春,張洛行、龔瞎子復擾歸德,聯恩馳援,連破之。追至五溝營,賊分為二,其東竄者分兵擊潰於商水南,而自躡其西,孤軍獨進。巡撫恆福劾其追賊遲延,革職留營。賊犯西華,進擊解其圍。追至舞陽北舞渡,日已暮,人馬皆未食,遇賊奮戰,進至殺虎橋,賊騎四面兜圍。聯恩身被重創,馬僕,步戰,手殺十餘賊,力竭,死之。詔復原官,依提督陣亡例優恤,予騎都尉並一雲騎尉世職,謚武烈。南陽、同安並建專祠。無子,以族子嗣,炳忠襲男爵,炳義襲世職。

黃開榜,湖北施南人。初入湘軍,從塔齊布戰武、漢、蘄、黃間,累擢至都司。咸豐七年,從勝保剿捻匪,克正陽關,擢游擊。八年,偕副都統穆騰阿戰馬頭,開榜失利,褫翎頂。復六安,加副將銜。九年正月,會豫軍毀潁上南照集賊巢,率水師攻蚌埠、長淮衛,戰七晝夜,獲賊船百餘,斃賊千餘,又焚賊舟糧,破懷遠水路諸卡,毀文昌閣賊壘,殺賊甚眾,賜號勤勇巴圖魯。合諸軍擊退援賊,直抵懷遠城下,先登,復懷遠,擢副將。十年,袁甲三圍鳳陽,開榜會攻爐橋,捻首張洛行來援,會諸軍夾擊破之。賊酋鄧正明潛乞降,覘府城虛實,開榜請聚師城外,示以兵威。總兵張得勝誘擒賊首張隆,令縛獻賊酋悍黨十四人,磔於市。開榜梟張隆首示城賊,賊眾縛獻其酋乞降,誅悍者三百餘人,餘遣散歸業。功最,以總兵記名。偕總兵田在田等破賊王家營,復清江浦,遂駐防。江寧大營潰,降賊薛成良叛入邵伯湖,開榜偕副將劉成元等毀賊船三百餘,殲賊殆盡,成良赴水死。加提督銜,授江西九江鎮總兵。十一年,攻天長,疊平賊壘。

同治元年,捻匪竄寶應,開榜督砲船擊走之,又敗賊於山陽、汊河。偕道員張富年破賊宿州觀音寺、仁和集,擒賊酋王春玉於邳州,拔貓兒窩賊柵。僧格林沁劾開榜飾詞冒功,下漕運總督吳棠按究,得白,薦統徐、宿軍,兼節制水師。二年,攻長城賊堡,克之,收撫附近諸墟。破郜家花園、孫甿賊巢,以提督記名。

粵匪渡江北犯,開榜扼高郵,賊掠船渡湖犯天長,開榜往援,焚賊筏,軍於堤上。賊列陣以拒,開榜令副將龔云福由陸路迎擊,參將陳浚家率砲劃潛出小河口,轉戰而前,與長城兵夾擊,破賊於三汊河,天長圍解。提督楊岳斌復江浦、浦口,開榜破七里洲賊壘,焚船六十餘艘。助攻九洑洲,拔之。開榜奉調赴臨淮,偕總兵普承堯平七里河岸賊壘。三年,率所部師船防通州,江寧平。四年,赴九江鎮任。十年,卒,謚剛愍。

陳國瑞,字慶雲,湖北應城人。年十餘歲陷賊中,出投總兵黃開榜,收為義子,冒姓黃氏。在軍每戰沖鋒。咸豐九年,從攻懷遠,率七人夜渡河攀堞先登,擲火燔譙樓,斬悍賊十餘人,師畢登,遂克懷遠,自是以勇聞。欽差大臣袁甲三進圍定遠,捻首李光等來援,國瑞陷陣,脅中槍,裹創力戰,賊闢易,乘勝破二圩,賜號技勇巴圖魯。奉檄援壽州,中途聞賊犯鳳陽,回軍夜往,連破賊壘,立解圍,超擢游擊。十一年,江、皖賊合眾窺揚州,國瑞馳剿湖西,屢破賊,加副將銜。

同治元年春,捻匪犯淮安,國瑞率五百人繞出賊後,與總兵龔耀倫夾擊,賊驚潰,馬賊悉遁,步賊萬餘回拒,國瑞偕總兵王萬清合戰破之。再破賊黨李城於版徬。賊由眾興集撲清江浦,擊走之。以砲船三十遏運河,夜襲桃源北岸,破賊圩四,直取眾興,拔十餘壘,擢副將。三月,率步卒八百敗賊於涇河,轉戰至新河,賊逼堤而陣。國瑞麾隊猛進,手燃砲殪執旗賊目,斬級千餘,以總兵記名。進剿泗州捻首韓老萬,敗之。四月,戰於邳州新村,捻眾亙三十里,國瑞分三路迎擊,斬賊渠王春玉,擲其首賊陣中,賊駭亂,夜冒雨襲破其三營。別賊趨救,昏暗不辨,自相殺,乘勢蹙之,殲數千。捻勢遂衰。

時山東棍、幅各匪麕集郯城,漕運總督吳棠檄國瑞進剿,連克數圩,斃悍酋孫化祥,餘黨多就撫。五月,會攻兗州鳳凰山,約副將郭寶昌、參將康錦文分路設伏,躬率小隊抵白蓮池,誘賊出,伏發,截賊隊為二,擒悍匪劉雙印。緣崖先登,諸軍繼之,克鳳凰山,戮逆首宋維鵬等,賜黃馬褂、頭品頂戴。國瑞呈請歸宗,復陳姓。

會苗沛霖叛,僧格林沁移剿,檄國瑞先發,漕運總督吳棠奏請國瑞幫辦軍務。國瑞至蒙城,先襲破紅裡賊圩以通糧道,繼克王圩,越重壕進逼賊巢。皖軍總兵宋慶會攻,國瑞以賊壘連屬不易下,密令郭寶昌自全家集鳧水支浮橋,宋慶守之,親引軍渡河焚賊糧屯,連破數壘。沛霖夜遁,為人所殺。淮甸平,以提督記名。三年,授浙江處州鎮總兵,屯正陽關。

僧格林沁剿捻湖北不利,檄國瑞赴援,坐遷延,降三級調用,奪所部隸郭寶昌。國瑞觖望,人言其將反。八月,國瑞率千餘人謁僧格林沁於光山,請為前鋒,偕翼長成保等剿柳林大小諸寨。深入失利,國瑞力戰兩晝夜,始突圍出。追賊蘄水、蘄州、羅田、廣濟,屢捷。賊竄英山、霍山,合諸軍戰於土漠河,殲斃數千,生擒數百。時群賊因江寧已克,降散過半。敘功,復原官。四年正月,翼長恆齡追賊至魯山,遇伏,與副都統舒倫保等同日陣亡,國瑞力扼橋口,餘眾得還。

賊犯襄城,國瑞乘夜大雪,出賊不意,火其壘,賊潰定。時賊被剿急,來往飄忽,僧格林沁率騎軍窮追,國瑞步隊從其後。三月,遇賊於確山,與諸軍合擊,大破之。賊僅餘馬隊,由遂平、西平直走睢州,過舊黃河,入山東境。僧格林沁以國瑞與郭寶昌戰最力,奏賞所部軍士各銀五千兩,又請獎寶昌遇提督簡放。詔謂國瑞確山之戰最出力,命酌量保奏。賊從臺莊渡運河,遂趨江北,國瑞躡之,屯沭陽。

四月,賊復折入山東,僧格林沁戰於曹州,兵挫遇害。詔罪諸將不能救護,國瑞以受傷免議。素恃功桀驁,自僧格林沁外,罕聽節制。曾國籓奉命督師,諭戒甚切,飭赴援歸德。至濟寧,與劉銘傳交惡,發兵爭鬥,殺傷甚多,踞長溝相持不下,詔嚴斥之,亦未加之罪。國籓疏論:「曹州之役,國瑞與郭寶昌分統左右兩翼,寶昌革職拏問,國瑞不應幸免。」遂撤去幫辦軍務,褫黃馬褂,暫留處州鎮戴罪立功。尋養病淮安,益縱恣不法,欲殺義子振邦。漕運總督吳棠劾其病癲,褫職,押送回籍,收其鹽本、田產充公;存銀二萬五千兩儲湖北官庫,分年付貲生計,毋令失所,俟其病痊奏聞。既而病痊,疆吏張之萬、譚廷襄等交章論薦,召至京,予頭等侍衛。

六年春,捻匪張總愚猝犯畿南,命率師迎擊。國瑞兩晝夜馳抵保定,詔嘉之。數敗賊,追至河南境。行軍輒自由,不聽節制,所部尤無紀律,屢被彈劾。擊賊於濟陽、德平,皆捷。洎捻平,悉復原職、黃馬褂、勇號,予雲騎尉世職。以傷發,乞假居揚州。

李世忠與有嫌,相閧,世忠縛諸舟,將斃之。曾國籓劾世忠,革職,國瑞降都司,勒令回籍。國瑞復潛至揚州,因總兵詹啟綸毆斃胡士禮獄,牽連論罪,戍黑龍江。逾數年,朝廷猶念舊功,以詢大學士李鴻章,鴻章謂其情性未改,精力已衰,遂不復用。光緒八年,歿於戍所。給事中鄧承修、山東巡撫福潤、安徽巡撫沈秉成、湖廣總督張之洞先後疏陳戰績,詔允復官,並於立功諸省建專祠。

郭寶昌,安徽鳳陽人。投效臨淮軍中,從戰數有功。尋改隸陳國瑞楚勝軍。咸豐十一年,國瑞擊捻匪於高郵、寶應,寶昌率驍健十八人為前鋒,陷陣得捷,又率兵三百破賊於天長龍崗,擢守備,賜花翎。同治元年正月,捻酋李成、任柱等犯清江浦,楚勝軍御之,戰車橋鎮。賊分眾劫淮關,寶昌追截,奪還所劫稅銀數萬兩。賊奔還眾興集,寶昌潛師夜襲,連破二十餘壘,賊引去,擢游擊,賜號卓勇巴圖魯。捻黨劉添福等糾餘匪擾泗洲,山東棍匪亦響應,寶昌連破之汊河、沙浦莊,匪勢漸衰。二年,匪首孫化祥就擒。積功洊擢副將,楚勝軍名益著。

僧格林沁調令助剿白蓮池、鳳凰山,從陳國瑞迭出奇兵力戰,生擒賊首劉雙印,斬其黨劉金春等於陣。任柱糾棍匪、教匪諸黨來援,並擊走之。白蓮池平,論功,以總兵記名。移軍剿苗沛霖。寶昌偕陳國瑞先至,攻破王家圩,渡河築三壘,與賊對峙,斷其餽運。賊悉銳來爭,擊卻之,賊氣奪。大軍至,諸圩以次下,沛霖走死,加提督銜。

三年,調援湖北,與陳國瑞分軍,名曰卓勝營,始獨當一面。八月,粵、捻諸匪由湖北入安徽,至英山東北,寶昌合諸軍敗之黑石渡。賊首馬融和擁眾十萬,議投誠,未決。寶昌單騎入其營,曉譬禍福,融和即日降。事聞,賜黃馬褂。四年,從僧格林沁轉戰楚、豫之交,功多,特奏保提督記名。尋以曹州之敗,詔斥不能救護主將,革職遣戍新疆。五年,曾國籓、喬松年奏請免發遣,留營效力。六年,從喬松年赴陜西,偕提督劉松山剿回匪於臨平,克之。

捻魁張總愚率眾萬餘犯富平,寶昌縱間伺隙出奇襲之。令部將宋朝儒等設伏村墟,自率親軍挑戰,伏起夾擊,斬馘數千,又敗賊於大荔大濠,復原官、勇號。進復綏德州,授安徽壽春鎮總兵。七年春,捻匪由山西、河南直犯畿輔,寶昌馳援,日行百餘里,抄出賊前抵保定。賊至,見官軍盛,引去,晉號法凌阿巴圖魯。躡賊入河南,敗之封丘、黃河套。張總愚匿村舍中,寶昌單騎獨出,突遇賊,受傷墮馬,部將宋朝儒翼之出。事聞,予假兩月調理,賜尚方珍藥。未幾,捻匪平,復黃馬褂,以提督簡放,予騎都尉世職。命從左宗棠赴陜西剿回匪。

八年,傷愈,西行,破賊於宜川,平綏德州叛卒。回匪東趨,命赴山西防河。九年,河西土匪起,寶昌渡河擊破其眾。奉檄搜捕北山土匪,悉平。十年,赴壽春鎮任。十一年,霍丘蔡家集土匪李六倡亂,率輕騎百人往剿,誅渠魁而還。事定,加頭品頂戴。光緒二年,平永城、渦陽土匪,被優敘。寶昌剿捻功最多,鎮壽春先後三十年,淮北恃為保障。調廣東南韶鎮,未任,尋還故官。俄羅斯、法蘭西、日本三次開兵釁,調防南北,事定仍回本任。二十六年,卒於官,賜恤。

論曰:秦定三、鄭魁士並向榮得力之將,和春克廬州,悉賴二人,遂與皖事終始。桐城之潰,由於爭餉不和,亦疆臣無調度之方以致之。傅振邦老於軍事,持重無失。邱聯恩名將之子,在豫軍中最號忠勇。陳國瑞勇足冠軍,剽悍不受繩尺,不能以功名終。郭寶昌戰績亦與並稱,材武不及,而器量差勝焉。


\end{pinyinscope}