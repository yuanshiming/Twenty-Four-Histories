\article{列傳三十七}

\begin{pinyinscope}
李霨孫廷銓杜立德馮溥王熙弟燕吳正治黃機

宋德宜子駿業伊桑阿子伊都立阿蘭泰子富寧安徐元文弟秉義

李霨,字坦園,直隸高陽人,明大學士國縉子。少孤,劬學自厲。順治三年,成進士,選庶吉士,授檢討,進編修。十年,世祖親試習國書翰林,霨列上等,擢中允。累遷秘書院學士。時初設日講官,霨與學士麻勒吉、胡兆龍,侍讀學士折庫納,洗馬王熙,中允方懸成、曹本榮等並入直。尋充經筵講官。十五年,拜秘書院大學士。內三院改內閣,以霨為東閣大學士,兼工部尚書,加太子太保。以票擬疏誤,鐫四秩。未幾,復官,任事如故。偕大學士巴哈納等校定律例。

十八年,聖祖即位,復內三院,以霨為弘文院大學士。時四大臣輔政,決機務,或議事齟,霨輒默然,既乃出片言定是非,票擬或未當,不輕論執。每於談笑間婉言曲喻,徐使更正。其間調和匡救,保護善類,霨有力焉。

康熙八年夏,旱,奉詔清刑獄,釋系囚,多所平反。明年,復內閣,霨以保和殿大學士兼戶部尚書。與修世祖實錄,充總裁官。十一年,書成,賜銀幣、鞍馬,晉太子太傅。未幾,三籓叛,繼以察哈爾部作亂。上命將出征,凡機密詔旨,每口授霨起草,退直嘗至夜分,或留宿閣中。所治職務,出未嘗告人,忠謹慎密,始終匪懈。二十一年,重修太宗實錄成,進太子太師。

臺灣初定,提督施瑯請設官鎮守,廷議未決。有謂宜遷其人、棄其地者,上問閣臣,霨言:「臺灣孤懸海外,屏蔽閩疆。棄其地,恐為外國所據;遷其人,慮有奸宄生事。應如瑯議。」上韙之。二十三年,卒,謚文勤。

霨弱冠登第,大拜時年裁三十有四,風度端重,內介外和。久居相位,尤嫺掌故,眷遇甚厚。四十九年,上追念前勞,超擢其孫工部主事敏啟為太常寺少卿。

孫廷銓,初名廷鉉,字枚先,山東益都人。明崇禎進士,任永平推官。順治元年,授天津推官。二年,以巡撫雷興薦,擢吏部主事,歷郎中。與曲沃衛周祚同官文選司,有聲於時。累遷左通政。十年,擢戶部侍郎。以大學士洪承疇薦,召對。尋坐事,罰俸,論告歸。還朝,改兵部,擢尚書。

十三年,調戶部。廷銓以歲會無總錄,無以劑盈絀之宜,殫心綜覈,錢穀舊隸諸部者,各還所司,條貫釐然。歲會之成自此始。十四年,疏言:「山東、河南荒田,請招民墾闢。其已熟者,清釐賦額,無使隱漏。」上從其言。

十五年,調吏部,加太子太保。十六年,諭獎其勤勞,加少保。廷銓疏請復學道升補舊制,下所司集議,如廷銓請。時吏部銓除,一事數例,吏胥因緣為奸。給事中楊雍建、胡爾愷。黏本盛、孫際昌、王啟祚,御史許劭昕,交章發其弊,且劾廷銓因循為所蔽,奪加銜,罰俸。十七年,疏言:「新闢邊疆員缺,督撫委用,即予實授,與部選之員,一體遷轉。蒞事未久,輒移內地,請定為試署二年,乃予實授。」又言:「司道不宜輕易,非大計處分及貪酷被糾者,遇降革,仍留任。」皆從之。又因旱,疏請寬考成,興屯政。上命兵部議屯政,而詢廷銓請寬考成議中有云「積資累薦,棄以一眚」語,何所指?廷銓言:「積疲州縣,久累人材,宜稍寬減觀後效,非為處分人員求免。」

世祖崩,二十七日制滿。廷銓發議尊皇太后為太皇太后,上所生母為皇太后,率九卿上請舉大禮疏。及議大行皇帝謚號,廷銓曰:「大行皇帝龍興中土,混一六合,功業同於開創。宜謚為高皇帝。眾皆和之,而輔臣鼇拜持異議,遂定謚章皇帝。時太祖謚武皇帝,故廷銓議如是。時論頗歸之。

康熙二年,拜秘書院大學士。奉職勤慎,終歲未嘗休沐。逾年,以父母年老,解職歸養,閉戶卻掃,不與外事。十三年,卒,謚文定。

杜立德,字純一,直隸寶坻人。明崇禎進士。順治元年,以順天巡撫宋權薦,授中書科中書。二年,考選戶科給事中。疏陳:「治平之道有三:一曰敬天。君為天之子,當修省以迓天休。今秦、晉、燕畿水旱風雹,天心示警。凡開誠布公,懋德敦行,皆敬天事也。一曰法古。古者事之鑒,是非定於一時,法則昭於百代。故合經而後能權,遵法而後能創。凡建學明倫,立綱陳紀,皆法古事也。一曰愛人。自大臣以至百姓,宜一視同仁。且無論新舊,悉存棄短取長之心。凡親賢納諫,尚德緩刑,皆愛人事也。」上以其有裨治理,深嘉納之。又累疏言:「牧民之官,宜久任以驗成功。凡遇賑蠲,宜分別款項,豫行頒示,使小民咸喻,胥吏不能為奸。」「條編法簡易便民。軍興草豆無定額,宜敕部定價值,使民先事為備。」皆下部議行。累遷戶科都給事中。疏言:「漕運叢弊,今漕臣庫禮搜獲運官使費冊三十本送部。請敕窮究,以釐奸弊。」再遷吏科都給事中。八年,疏請舉行經筵,擇廷臣經明行修者為講官,以裨聖治;又請定朝期,肅禁地,杜加派。上甚韙之。

初,睿親王多爾袞攝政,給事中許作梅,御史吳達、李森先、桑蕓等交章劾大學士馮銓奸貪狀,疏上旬日,未下廷議。立德請令滿、漢大臣集議,以伸公論,鼓直言之氣;並及馬士英、阮大鋮、宋企郊等,在前朝或納賄招權,或煽惡流毒,今並逋逃,宜急捕誅,以彰法紀。下刑部,以事在赦前,寢其議。世祖親政,銓既黜,立德因言作梅等前以劾銓為所切齒,又僉都御史趙開心素為銓所忌,相繼構陷去官,乞矜察。由是開心等俱起用。

立德尋遷太常寺少卿,超擢工部侍郎,調兵部。畿輔水災,奉詔賑濟大名,全活甚眾。再調吏部,以父憂去。坐兵部任詿誤,鐫秩調用。服闋,除太僕寺卿,擢刑部侍郎。十六年,加太子少保銜。領侍衛內大臣額爾克岱青家奴縛侍衛誣訴,部議罪侍衛,下內大臣索尼等察實,立德奪加銜。十六年,擢尚書。

立德治獄仁恕,上聞其用法平,深嘉之。嘗入對,既出,上顧左右曰:「此新授刑部尚書杜立德也!不貪一錢,亦不妄殺一人。」康熙元年,調戶部。考滿,復加太子少保。三年,調吏部。八年,拜國史院大學士。聖祖親政,乾清宮成,擇日臨御,欽天監奏吉神在隅,不宜從中門入。立德言:「紫微帝星所在,吉神拱向。皇上遷正新宮,臣庶觀瞻,應從中門入。監臣所奏非是。」上從其言。九年,改保和殿大學士,兼禮部尚書,進太子太傅。

三籓事起,立德與李霨、馮溥參預機務。從容整暇,中外相安。廣東平,所司具正雜賦稅之數以聞。立德言:「廣東雜稅多尚之信所加,為民間大累,非朝廷正額。今變亂甫定,宜與民休息。其除之便。」上從之。十八年,自陳乞休。其秋地震,復請罷,詔輒慰留。雲南平,議頒恩赦,立德告病未與議,遣大臣持詔旨就其家諮詢,俟還奏乃下詔。一日,上顧閣臣,謂在廷諸臣誰堪大用者,立德面疏數人以對。比退,人訝其不稍引嫌,答曰:「自筮仕以來,惟此心可邀帝鑒。他非所計也。」

二十一年夏,復乞休,上許之,賜禦制詩及「怡情洛社」篆章,馳驛遣行人護歸。太宗實錄成,進太子太師,賜銀幣、鞍馬。二十六年,太皇太后喪,立德詣京師哭臨,上念其老病不任拜起,命學士張英扶掖以行,慰勞甚至。三十一年,卒,年八十一,上聞,諭大學士曰:「杜立德秉性厚重,行事正大。直言敷奏,不肯茍隨同列。可謂賢臣!」賜祭葬如禮,謚文端。三十九年,帝南巡,其子恭俊迎駕三河,上問立德葬所,手書「永言惟舊」四字賜之,命揭諸阡。恭俊官廣信知府,好義,善濟人急。

馮溥,字孔博,山東益都人。順治三年進士,選庶吉士,授編修。累遷秘書院侍讀學士,直講經筵。世祖幸內院,顧大學士曰:「朕視馮溥乃真翰林也!」十六年,擢吏部侍郎。會各省學道缺,部郎不足,以知府補之。已,會禮部議奏,時尚書孫廷銓、侍郎石申並乞假;給事中張維赤因劾溥徇私,溥疏辨。上曰:「朕知溥不為也!」置勿問。明年,京官三品以上自陳,忽嚴旨黜滿尚書科爾坤及兩侍郎,獨留漢官在部。溥與廷銓疏言:「部事滿、漢同治,今滿臣得罪,漢臣安得免,乞並黜。」詔供職如故。

康熙初,停各省巡按,議每省遣大臣二人廉察督撫。吏部尚書阿思哈、侍郎泰必圖議設公廨,頒冊印。溥謂:「國家設督撫,皆重臣。今謂不可信,復遣兩大臣監之。權既太重,勢復相軋,保無屬吏仰承左右啟隙端?」泰必圖性暴伉,聞溥言,恚,瞋目攘臂起。溥徐曰:「會議也,獨不容吾兩議耶?且可否自有上裁,豈敢專主?」疏入,上然溥言,事遂寢。御史李秀以考績黜,後夤緣得復官,劾溥為故相劉正宗黨,主銓時違例徇私,溥疏辨,嚴旨責秀誣訐。六年,遷左都御史。內閣有紅本,已發科鈔,輔臣鼇拜取回改批。溥抗言:「本章既批發,不便更改。」鼇拜欲罪之,上直溥,戒輔臣詳慎。盛京工部侍郎缺,已會推,奉旨以規避者多,不旬日三易其人。溥疏言:「王言不宜反汗,當慎重於未有旨之先,不當更移於已奉旨之後。」首輔班布爾善寢其奏,上聞,取溥疏覽之,稱善,飭部施行。

八年夏,旱,應詔陳言,請省刑薄稅。略謂:「古者罪人不孥,今一事牽連佐證,或數人,或數十人。往往本犯尚未審明,而被累致死者已多。且或遲至七八年尚未結案,遂致力穡供稅之人,拋家失業。請敕部嚴禁。百姓之財,不過取之田畝。今正月已開徵,舊稅之逋甫償,新歲之田未種,錢糧從何辦納?請敕部酌議。自後徵賦,緩待夏秋。」下戶、刑二部議。刑部議,承審強盜、人命重案,限一年速結,不得牽累無辜,督撫及承審官隱漏遲延皆有罰。戶部議,春季兵餉不能待至夏秋,仍舊例便。得旨,俟國用充足,戶部奏請更定。戶部吏陳一魁冒領清苑等縣錢糧事發,溥言:「錢糧者百姓之脂膏也,其已輸在官,則朝廷之帑藏也。若任胥吏侵盜,職掌謂何?請嚴定所司處分,懲前毖後。」擢刑部尚書。十年,拜文華殿大學士。疏言:「直隸、山東、河南、山西、陜西米麥豐收,穀價每斗值銀三四分。當此豐稔之時,宜廣積貯,以備兇年。」

先是,溥以衰病累疏乞休,上曰:「卿六十四歲,未衰也,俟七十乃休耳。」自吳三桂反,軍事旁午,乃不敢復言。十四年,建儲禮成,內閣議恩赦,滿大臣以八旗逃人應不赦,溥不可,遂兩議以進。詔下閣臣畫一奏聞,有謂當從滿大臣議者,溥持之力,仍以兩議進,上卒從之。十七年,福建平,溥以年屆七十,復申前請,上仍慰留。二十一年秋,詔許致仕,遣官護行馳驛如故事。比將歸,詣闕謝,賜游西苑,內侍攜酒果,所至坐飲三爵。臨發,疏請清心省事,與民休息,言甚切,溫旨報聞。賜禦制詩及「適志東山」篆章,命講官牛鈕、陳廷敬傳諭曰:「朕聞山東仕於朝者,彼此援引,造為議論,務有濟於私,又居鄉多擾害地方,朕審知其弊。馮溥久居禁密,可教訓子孫,務為安靜。」太宗實錄成,加太子太傅。三十年,卒,年八十三,謚文毅。

溥居京師,闢萬柳堂,與諸名士觴詠其中。性愛才,聞賢能,輒大書姓名於座隅,備薦擢。一時士論歸之。

王熙,字子雍,順天宛平人。父崇簡,明崇禎十六年進士。順治三年,以順天學政曹溶薦,補選庶吉士,授檢討。累遷禮部尚書,加太子少保。嘗疏請賜血⼙明季殉難範景文、蔡懋德等二十八人,又議帝王廟罷宋臣潘美、張浚從祀,北嶽移祀渾源,皆用其議。十八年,引疾解職。康熙十七年,卒,謚文貞。

熙,順治四年進士,選庶吉士,授檢討。累遷右春坊諭德。召直南苑。譯大學衍義,充日講官,進講稱旨。累擢弘文院學士。時崇簡方任國史院學士,上曰:「父子同官,古今所罕。以爾誠恪,特加此恩。」十五年,擢禮部侍郎,兼翰林院掌院學士。考滿,加尚書銜。時崇簡為尚書,父子復同官。十八年正月,上大漸,召熙至養心殿撰遺詔,熙伏地飲泣,筆不能下,上諭勉抑哀痛,即御榻前先草第一條以進。尋奏移乾清門撰擬,進呈者三,皆報可。是夕上崩,聖祖嗣位,熙改兼弘文院學士。

康熙五年,遷左都御史。時三籓擁兵逾制,吳三桂尤崛強,擅署官吏,浸驕蹇,萌異志。子應熊,以尚主居京師,多聚奸人,散金錢,交通四方。熙首疏請裁兵減餉,略言:「直省錢糧,半為雲、貴、湖廣兵餉所耗。就云、貴言,籓下官兵歲需俸餉三百餘萬,本省賦稅不足供什一,勢難經久。臣以為滇、黔已平,綠旗額兵亟宜汰減,即籓下餘丁,亦宜散遣屯種,則勢分而餉亦裕。」復疏言:「閩、廣、江西、湖廣等省官吏,挾貲貿易,與民爭利。或指稱籓下,依勢橫行。宜飭嚴禁。」又言:「近例招民百家送至盛京,得授知縣。不肖奸人,借資為市,貽害地方,宜改給散秩。現任官吏捐輸銀米,博取議敘,名出私橐,實取諸民,宜一切報罷。」上俱從之。

七年夏,旱,金星晝見,詔求直言。熙疏言:「世祖章皇帝精勤圖治,諸曹政務,皆經詳定。數年來有因言官條奏改易者,有因各部院題請更張者,有會議興革者,則例繁多,官吏奉行,任意輕重。請敕部院諸司詳察現行事例,有因變法而滋弊者,悉遵舊制更正。其有從新例便者,亦條晰不得不然之故,裁定畫一。」上命各部院條議,遵舊制,刪繁例,凡數十事。遷工部尚書。

十二年,調兵部。是年冬,三桂反,京師聞變,都城內外一夕火四起,皆應熊黨為之也。明年三月,用熙言誅應熊。尋命熙專管密本。漢臣與聞軍機自熙始。十七年,以父憂去。二十一年,即家拜保和殿大學士,兼禮部尚書。時三籓既平,熙以和平寬大,宣上德意,與民休息。造次奏對,直陳無隱,上每傾聽。太祖實錄成,加太子太傅。三十一年,以疾累疏乞休,溫旨慰留。四十年,詔許致仕,晉少傅。明年上元節,賜宴其家,遣官齎手敕存問。四十二年,卒,上命皇長子直郡王允禔、大學士馬齊臨喪,行拜奠禮,舉哀酹酒,恩禮有加,謚文靖。

熙持大體,有遠慮。平定三籓後,開方略館。一日,上諭閣臣:「當三桂反時,漢官有言不必發兵,七旬有苗格者。」又其時漢官多移妻子回家,顧學士韓菼曰:「汝為朕載之!」菼退而皇恐。熙乃昌言閣中曰:「『有苗格』乃會議時魏象樞語。告者截去首尾,遂失其本意。然如其言,豈非誤國?移家偶然耳,日久何從分別,其移者豈非背主?漢官負此兩大罪,何顏立朝?」翌日入見,執奏如閣中語,上許之。

熙子克善、克勤,皆世祖命名。克善能文,熙不令與試,遇鄉、會典試,熙輒注假,以聖祖方惡漢人師生之習,故尤慎之。二十七年,典會試,蓋特命也。雍正中。入祀賢良祠。

弟燕,字子喜,以父廕,任戶部郎中。出為鎮江知府,擢江蘇按察使,治獄稱平。遷湖廣布政使,巡撫貴州,建學設官,減賦稅,教養兼施,善拊循苗人,頒條教,飭州縣無縱奸人詭索土司。撫黔三年,移疾歸,卒。

吳正治,字當世,湖北江夏人。順治六年進士,選庶吉士,授國史院編修。丁母憂,服闋,起故官。遷右庶子。十五年,特簡翰林官十五人外用,正治與焉,得江西南昌道。遷陜西按察使。所至以清廉執法著稱。十七年,內擢工部侍郎,調刑部。平亭疑獄,釋江南逋賦無辜諸生二百餘人。疏論奉行赦款宜速,丈量田地宜停,禁狀外指扳,嚴婦女私嫁,皆著為令。

康熙八年,以父憂去。起兵部督捕侍郎,充經筵講官。十二年,遷左都御史。疏言:「緝逃事例,首嚴窩隱。一有容留,雖親如父子,即坐以罪,使小民父子視若仇讎。伏讀律有親屬容隱之條,惟叛逆者不用此律。逃人乃旗下家人之事,與叛逆輕重相懸。請自今有父子窩逃,被人舉發者,逃犯治罪,免坐窩隱。若容留逾旬,父子首報者,逃犯依自首例減罪。則首報者多,逃人易獲。朝廷之法與天性之恩,兩不相悖矣。」又言:「今歲雨澤愆期,方事祈禱。近因直隸多盜,廷議於玉田、灤州、霸州、雄縣增設駐防旗兵,構建營房,勞民動眾,應暫停止。俟農隙時酌行。」疏入,下部議,俱如所請。先是睿親王多爾袞當國,嚴旗下逃人之禁,鰲拜繼之,禁益嚴。株連窮治,天下囂然,而圈地建營房,凡涉旗務,漢大臣莫敢置喙。自正治疏出,逃人禁稍寬,營房亦罷建,世多以是稱之。

尋遷工部尚書,調禮部。十八年,自陳乞休,詔嘉其端勤誠慎,慰留之。二十年,拜武英殿大學士。時修太祖實錄、聖訓、會典、方略、一統志,俱充總裁官,加太子太傅。

正治守成法,識大體。一日,聖祖閱朝審冊,有以刃刺人股致死而抵法者,上曰:「刺股傷非致命,此可寬也。」正治對曰:「當念死者之無辜。」他日,又閱冊,有囚當死,上問此囚尚可活否,眾皆以情實對。正治曰:「皇上好生之德,臣等敢不奉行。」退而細勘,得可矜狀,遂從末減。二十六年,復疏乞休,詔許原官致仕。三十年,卒,謚文僖。

黃機,字次辰,浙江錢塘人。順治四年進士,選庶吉士,授弘文院編修。世祖幸內院,詢機裏籍官職,命與侍講法若真、修撰呂宮、編修程芳朝撰柳下惠不以三公易其介論,上覽畢,賜茶。授左中允,尋遷弘文院侍讀。

十二年,機疏言:「自古仁聖之君,必祖述前謨,以昭一代文明之治。今纂修太祖、太宗實錄告成,乞敕諸臣校定所載嘉言嘉行,仿貞觀政要、洪武寶訓諸書,輯成治典,頒行天下。尤原萬幾之暇,朝夕省覽。法開創之維艱,知守成之不易,何以用人而收群策之效?何以納諫而宏虛受之風?何以理財而裕酌盈劑虛之方?何以詳刑而無失出失入之患?力行身體,則動有成模,紹美無極。」上俞之,詔輯太祖、太宗聖訓,以機充纂修官。累遷國史院侍讀學士,擢禮部侍郎。

康熙六年,進尚書。疏言:「民窮之由有四:雜捐私派,棍徒哧詐,官貪而兵橫。請嚴察督撫,舉劾當否,以息貪風、甦民命。各省籓王、將軍、提、鎮有不法害民之事,許督撫糾劾。請飭破除情私,毋更因循,貽誤地方。」七年,調戶部,再調吏部。機以疏通銓法、議降補官對品除用,為御史季振宜所劾。既而給事中王曰溫劾故庶吉士王彥即機子黃彥博,欺妄,應罷黜。機以彥與彥博姓名不同,且彥博死已久,疏辨,得免議。尋以遷葬乞假歸,而論者猶不已。

十八年,特召還朝,以吏部尚書銜管刑部事。御史張志棟言機老成忠厚,然衰邁,恐誤部事,應令罷歸。上以志棟言過當,命機供職如故。明年,授吏部尚書。以年老請告,詔慰留。二十一年,拜文華殿大學士,兼吏部。逾年,復乞休,許以原官致仕,遣官護行馳驛如故事。二十五年,卒,謚文僖。

宋德宜,字右之,江南長洲人。父學硃,明御史,巡按山東,死於難。德宜年十七,伏闕請恤,與兄德宸、弟德宏並著文譽。順治十二年,成進士,選庶吉士,授編修。累遷國子監祭酒,嚴立條教,六館師生咸敬憚之。聖祖親政,釋奠太學,御彞倫堂,命德宜東鄉坐,講周易乾卦辭,稱旨。遷翰林院侍讀學士,擢內閣學士。

德宜風度端重,每奏事,輒當上意。康熙十一年,扈蹕塞外,上從容詢及江南逋賦之由,德宜極言蘇、松賦役獨重,民力凋敝,上為動容。詔明年蠲蘇、松四府錢糧之半。遷戶部侍郎,發龍江關大使李九官餽遺,上嘉其不私,褫九官職。尋調吏部。

十五年,擢左都御史。時陜、甘、閩、粵漸已底定,惟吳三桂未平。德宜疏言:「三桂所恃,不過槍砲,槍砲專藉硝黃。硝黃產自河南、山西,必奸民圖利私販,請飭嚴禁。」上以督、撫、提、鎮稽察不嚴,下兵、刑二部嚴定處分。德宜又疏言:「頻年發帑行師,度支不繼。皇上允廷臣之請,開例捐輸。三年所入,二百萬有餘。捐納最多者,莫如知縣,至五百餘人。始因缺多易得,踴躍爭趨。今見非數年不得選授,徘徊觀望。請敕部限期停止,慎重名器。」又疏言:「沿海居民,以漁為生。佐賦稅,備災荒,而利用通商,又立市舶之制。本朝以海氛未靖,立禁甚嚴。近者日就蕩平,宜及此時招攜撫恤。沿海居民,以捕魚為業。商人通販海島,皆許其造船出海,官給印票,仿舊例輸稅。人口商貨,往來出入,咸稽核之。」事並下所司議行。

十七年,疏言:「自三桂煽亂,各路統兵大將軍以下,亦有玩寇殃民,營私自便。或越省購買婦女,甚者掠奪民間財物,稍不如意,即指為叛逆。今當剋期滅賊,尤恐借端需索。請嚴飭。」上下王大臣申禁。山東提督柯永蓁縱兵鼓言喿,德宜劾奏,上命逮治。

孝昭皇后崩,德宜上疏請秉禮節哀,並言;「宵旰憂勤,天顏清減。昔唐太宗銳意勤學,劉洎諫以多記損心。宋儒程頤亦曰:『帝王之學,與儒生不同。』伏原紬繹篇章,略方名象數之繁,擇其有關政治、裨益身心者而討論之。稍節耳目之勞,用葆中和之德。」上嘉納焉。遷刑部尚書,調兵部。

四川初定,大軍糗糧皆運自陜西,出棧道,顛踣相望,陜西民大困。工部侍郎趙璟、金鼐疏上陳,德宜因言:「大軍下雲、貴,需餉孔亟。秦、蜀互相推諉,皆由總督分設。川、陜設一總督,則痛癢相關,隨地調發,可以酌劑均平。」詔如議行。靖逆將軍張勇以甘肅防邊事重,請緩裁前此添設官兵,部臣議如所請,德宜獨謂:「當日河東有兵事,添設官兵,事平應即裁汰。將軍標下前以步兵二千名改為馬兵,今宜復原,定經制馬六步四。惟以防邊添設之兵,無可議裁。」上遣尚書折爾肯往會勇等閱核,留河州、寧夏添設兵,餘仍復原定經制,如德宜議。迨三籓平,軍中俘獲婦女,並籍旗下。德宜言宜聽收贖,所釋甚眾。

調吏部。左都御史魏象樞、副都御史科爾昆等劾德宜會推江西按察使事失當,德宜疏辨,部議降五級。上以會推原令各出所見,免德宜處分。二十三年,拜文華殿大學士。重修太宗實錄成,加太子太傅。

德宜嚴毅木訥,然議國家大事,侃侃獨攄所見。居官廉謹,未仕時有宅一區,薄田數頃;既貴,無所增益,門巷蕭然。二十六年,卒,謚文恪。

子駿業,自副貢授翰林院待詔,直御書處,歷兵科給事中。康熙四十一年,疏劾湖廣總督郭琇、提督林本植、巡撫金璽、總兵雷如等辦理苗疆剿撫失宜,鞫實,琇等降革有差。終兵部侍郎。

伊桑阿,伊爾根覺羅氏,滿洲正黃旗人。順治九年進士,授禮部主事。累擢內閣學士。康熙十四年,遷禮部侍郎,擢工部尚書,調戶部。時吳三桂踞湖南,廷議創舟師,自岳州入洞庭,斷賊餉道,命伊桑阿赴江南督治戰艦。明年,復命偕刑部侍郎禪塔海詣茶陵督治戰艦。

二十一年,黃河決,命往江南勘視河工,以布政使崔維雅隨往,維雅條上治河法,與靳輔議不合。伊桑阿因請召輔面詢,上以維雅所奏無可行,寢之。尋疏陳黃河兩岸堤工修築不如式,奪輔職,戴罪督修。復命籌海運,疏言:「黃河運道,非獨輸輓天庾,即商賈百貨,賴以通行,國家在所必治。若海運,先需造船,所費不貲;且膠、萊諸河久淤,開濬匪易。」上是之。是年冬,俄羅斯犯邊,命往寧古塔造船備徵調。再調吏部。

二十三年夏,旱,偕王熙等清刑獄。其秋,扈蹕南巡,命閱視海口。疏言車路、串場諸河及白駒、草堰、丁溪諸口,宜飭河臣疏濬,引流入海。歷兵、禮二部尚書。二十七年,拜文華殿大學士,兼吏部,充三朝國史總裁。三十六年,上親征噶爾丹,命往寧夏安設驛站,事平,與大學士阿蘭泰充平定朔漠方略總裁官。

居政府十五年,尤留意刑獄,每侍直勾本,上有所問,輒能舉其詞,同列服其精詳。上嘗御批本房,伊桑阿與大學士王熙、吳琠及學士韓菼等以折本請旨,上曰:「人命至重,今當勾決,尤宜詳慎。爾等茍有所見,當盡言。」伊桑阿乃舉可矜疑者十餘人,皆得緩死,上徐曰:「此等所犯皆當死,猶曲求其可生之路,不忍輕斃一人。因念淮、揚百姓頻被水害,死者不知凡幾。河患不除,朕不能暫釋於懷也!」伊桑阿陳災民困苦狀,上曰:「百姓既被水害,必至流離轉徙。田多不耕,賦安從出?今當預免明年田賦,俾災黎於水退時思歸故鄉,粗安生業。」伊桑阿等皆頓首,遂下詔免淮、揚明年田賦。

三十七年,以年老乞休。上諭阿蘭泰曰:「伊桑阿厚重老成,宣力年久。爾二人自任閣事,推誠布公,不惟朕知之,天下無不知者。伊桑阿雖年老求罷,朕不忍令去也。」四十一年,復以病告,詔許原官致仕。逾年卒,謚文端。乾隆中,入祀賢良祠。

子伊都立,自舉人任內務府員外郎,歷刑部侍郎,巡撫山西。坐事奪職。雍正七年,命赴大將軍傅爾丹軍治糧餉,授額外侍郎。十三年,以侵蝕軍糧事覺,褫職下獄,論大闢。乾隆七年,赦釋。

阿蘭泰,富察氏,滿洲鑲藍旗人。性敏慎。初授兵部筆帖式。康熙初,累遷職方郎中。三籓事起,專司軍機文檄。議政王大臣以勤勞詳慎疏薦,得旨以三品卿用。二十年,擢光祿寺卿,遷內閣學士,充平定三逆方略副總裁,兼充明史總裁。二十二年,遷兵部侍郎,兼管佐領。擢左都御史。上閱方略,以敘事多舛錯,諭閣臣曰:「平逆始末,阿蘭泰知之甚詳,可與酌改,務期紀載得實。」遷工部尚書。累調吏部。二十八年,上以雨澤愆期,命偕尚書徐元文慮囚,奏減罪可矜疑者四十五人。是年拜武英殿大學士。陜西饑,命阿蘭泰與河督靳輔議運江、淮糧米自黃河溯西安,以備積儲。

三十四年,上出古北口巡歷塞外,命留京綜閱章奏。明年,上親征噶爾丹,阿蘭泰仍留京,與尚書馬齊、佛倫宿衛禁城。其秋,隨駕出歸化城,駐蹕黃河西界,經畫軍務。以扈從勞,賜內廝馬。厄魯特臺吉丹濟拉來降,上駐蹕翰特穆爾嶺,召入見,阿蘭泰及郎中阿爾法引之入御幄,上屏左右,令阿蘭泰等出,獨與丹濟拉語良久。及退,召阿蘭泰諭曰:「爾偕降人入,以防不測,意甚善。朕令爾出,欲推誠示不疑耳。」

三十七年,與伊桑阿俱以年老善忘奏解閣務,上曰:「大學士重任,必平坦雍和、任事謹慎者方為稱職。至於記事,可令學士任之。」明年,卒。方病劇,上欲臨視,遣皇子先往,而阿蘭泰已卒。上為輟朝一日,遣皇子及內大臣奠醊,贈太子太保,加贈少保,謚文清。

阿蘭泰操行清謹,處政府遠權勢,人莫敢干以私,以是為上所重。後上與大學士論內閣舊臣,稱阿蘭泰能強記,且善治事云。

子富寧安,初襲其從祖尼哈納拜他喇布勒哈番世職。自侍衛歷官正黃旗漢軍都統,改授左都御史,遷吏部尚書。富寧安內行修篤,事親至孝,聖祖亟稱之,又嘗諭廷臣曰:「富寧安自武員擢用,人皆稱其操守,是以授為吏部尚書。今部院中欲求清官甚難,當於初為筆帖式時,即念日後擢用,可為國家大臣,自立品行也。」

五十四年,策妄阿喇布坦侵哈密,命富寧安赴西寧視師,許以便宜調遣。賊旋遁,詔緩進兵,回駐肅州,經理糧馬。五十六年,授靖逆將軍,駐軍巴里坤,與將軍傅爾丹等分路規賊。旋率兵襲擊厄魯特邊境,進屯烏魯木齊,屢敗賊。五十九年,進兵烏蘭烏蘇,遣侍衛哲爾德等分道襲擊,斬獲甚眾;別遣散秩大臣阿喇納等諭降闢展回人,進擊吐魯番,降其酋長,獲駝馬無算。時策妄阿喇布坦挾所屬吐魯番回人偕徙,中道多遁歸,命富寧安收撫其眾。未幾,賊復來犯,遣將援剿,自率兵進駐伊勒布爾和碩,調遣策應。會阿喇納連敗賊,竄走,乃還駐巴里坤。六十一年,疏言:「嘉峪關外、布隆吉爾之西,為古瓜、沙、燉煌地。昔吐魯番建城屯種,遺址猶存,若駐兵屯牧,設總兵官一人統之,可扼黨色爾騰之路。」又請專遣大臣領屯田糧儲及牧駝運糧事,上可其奏。

世宗即位,授武英殿大學士,管軍務如故。雍正四年,還朝,賜御用冠服、雙眼花翎、黃轡鞍馬,並諭王大臣:「富寧安端方廉潔,年來領兵將軍聲名無出其右者。」授世襲侯爵。尋進一等侯,加太子太傅,署西安將軍。六年,坐事奪爵,仍留大學士任。是年卒於西安,謚文恭,與父阿蘭泰同祀賢良祠。

徐元文,字公肅,江西昆山人。初冒姓陸,通籍後復姓。少沉潛好學,與兄乾學、弟秉義有聲於時,稱為「三徐」。

元文舉順治十六年進士第一,世祖召見乾清門,還啟皇太后曰:「今歲得一佳狀元。」賜冠帶、蟒服,授翰林院修撰。從幸南苑,賜乘御馬。嘗奉命撰孚齋說,孚齋,世祖讀書所也,上覽之稱善,命刊行。康熙初,江南逋賦獄起,元文名麗籍中,坐謫鑾儀衛經歷,事白,復原官。丁父憂,居喪行古禮。起補國史院修撰,累遷國子監祭酒,充經筵講官。

元文閒雅方重,音吐宏暢,進講輒稱旨。元文疏請「敕直省學臣間歲一舉優生,鄉試仍復副榜額,俱送監肄業」。並著為令。復請永停納粟,章下所司。居國學四年,端士習,正文體,條教大飭。其後上語閣臣:「徐元文為祭酒,規條嚴肅。滿洲子弟不率教者,輒加撻責,咸敬憚之,後人不能及也。」十三年,遷內閣學士,改翰林院掌院學士,充日講起居注官,教習庶吉士。

先是熊賜履在講筵,累稱說孔、孟、程、硃之道,上欲博覽前代得失之由,命詞臣以通鑒與四書參講。元文因取硃子綱目,擇其事之系主德、裨治道者,採取先儒之說,參以臆斷,演繹發揮,按期進講。尋以母憂歸。十八年,特召監修明史,疏請徵求遺書,薦李清、黃宗羲、曹溶、汪懋麟、黃虞稷、姜宸英、萬言等,徵入史館,不至者,錄所著書以上。尋補內閣學士。時有議遣大臣巡方者,元文言於閣中曰:「巡方向遣御史,以有臺長約束,故僨事者鮮。若遣大臣,或妄作威福,誰能禁之?」因入告,事得寢。

明年,擢左都御史。會師下雲南,吳三桂之徒多率眾歸附,耗餉不貲。元文疏言:「三桂遺孽,旦夕伏誅。凡脅從之眾,恩許自新。若仍留本土,既非永久之規;移調他方,亦多遷徙之費。統以別將,則猜疑未化,終涉危嫌;攝之歸旗,則放恣既久,猝難約束。請以武職及入伍者,與綠旗一體錄用。餘俱分遣為民,以裕餉需。至耿精忠、尚之信、孫延齡舊隸將弁,尤宜解散,勿仍籓旗名目。」又請「革三籓虐政,在粵者五:曰鹽埠,曰渡稅,曰總店,曰市舶,曰魚課;在閩者四:曰鹽稅,曰報船,曰冒擾驛夫,曰牙行渡稅;在滇者四:曰勛莊,曰圈田,曰礦廠,曰冗兵。」疏入,俱下所司議行。

初,御史劉安國請察隱占田畝,州縣利有升敘,多捏報累民。元文力言其弊,謂名為加稅,實耗糧戶。請飭督撫檢舉,復條列近時督撫四弊。時部例捐納官到任三年後稱職者,具題升轉;不稱職者,罷之。既,復令捐銀者免其具題,又生員得捐納歲貢。元文言捐納事例,系一時權宜,請於收復滇南之日,降詔停止,言甚剴切。

雲南平,告廟肆赦,廷臣多稱頌功德。元文獨言:「聖人作易,於泰、豐、既濟諸卦,垂戒尤切。景運方新,原皇上倍切咨儆。兼諭大小臣工,洗心滌慮,毗贊大業。勿狃目前之淺圖,務培國家之元氣。振紀綱以崇大體,核名實以課吏材,崇清議以定國是,厲廉恥以正人心,端教化以圖治本,抑營競以儆官邪,敦節儉以厚風俗,正名分以絕奸萌,並當今急務。」上俞之。

時方嚴窩逃之禁,杭州將軍馬哈達以民間多匿逃人,請自句攝,勿移有司。元文曰:「是重擾民也。無已,當令督撫會同將軍行之。」京師奸人,多掠平民賣旗下,官吏豫印空契給之,屢發覺,元文疏請禁止。又八旗家人投水、自經,報部者歲及千人,疏請嚴定處分。上俱從之。京察計典罷官者,謀入貲捐復,元文力持不可,遂罷議。先後疏劾福建總督姚啟聖縱恣譎詐,杭州副都統高國相縱兵虐民,兩淮巡鹽御史堪泰徇庇貪官,御史蕭鳴鳳居喪蔑禮,俱讞鞫得實,惟啟聖辨釋。二十二年,以會推湖北按察使,坐所舉不實,鐫三秩調用。尋命專領史局。二十七年,復代其兄乾學為左都御史,遷刑部尚書,調戶部。二十八年,拜文華殿大學士,兼掌翰林院事。

上南巡,幸蘇州,以江南浮糧太重,有旨詢戶部。元文考宋、元以來舊額官田、民田始末及前明歷代詔書以聞。元文在內閣,上復諭及之,元文頓首曰:「聖明及此,三吳之福也。」因下九卿議,有力尼之者,事遂寢。

元文兄乾學,豪放,頗招權利,坐論罷;而元文謹禮法,門庭肅然。二十九年,兩江總督傅拉塔劾乾學子侄交結巡撫洪之傑,招權競利,詞連元文,上置不問,予元文休致回籍。舟過臨清,關吏大索,僅圖書數千卷,光祿饌金三百而已。家居一年卒。乾學自有傳。

弟秉義,字彥和,舉康熙十二年進士第三,授編修,遷右中允。乞假歸。乾學卒,召補原官。累遷吏部侍郎。命偕刑部侍郎綏色克如陜西,讞糧鹽道黃明受賄,擬罪失當,左遷詹事。擢內閱學士,乞歸。上南巡,賜御書「恭謹老成」榜額。五十年,卒。

論曰:康熙初葉,主少國疑,滿、漢未協,四輔臣之專恣,三籓之變亂,臺灣海寇之趒蕩,措置偶乖,皆足以動搖國本。霨、廷銓、立德、溥當多事之日,百計匡襄;熙預顧命,參軍謀;正治等入閣,值事定後,從容密勿,隨事納忠;伊桑阿、阿蘭泰推誠布公,受知尤深。康熙之政,視成、宣、文、景駕而上之,諸臣與有功焉。


\end{pinyinscope}