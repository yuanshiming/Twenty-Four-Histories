\article{列傳八}

\begin{pinyinscope}
諸王七

高宗諸子

定安親王永璜端慧太子永璉循郡王永璋

榮純親王永琪哲親王永琮儀慎親王永璇

成哲親王永瑆貝勒永?慶僖親王永璘

仁宗諸子

穆郡王惇恪親王綿愷惇勤親王奕脤瑞懷親王綿忻

惠端親王綿愉

宣宗諸子

隱志郡王奕緯順和郡王奕綱慧質郡王奕繼

恭忠親王奕醇賢親王奕枻鍾端郡王奕硉

孚敬郡王奕譓

文宗子

憫郡王

高宗十七子:孝賢純皇后生端慧太子永璉、哲親王永琮,皇后納喇氏生貝勒永?、永璟,孝儀純皇后生永璐、仁宗、第十六子、慶僖親王永璘,純惠皇貴妃蘇佳氏生循郡王永璋、質莊親王永瑢,哲憫皇貴妃富察氏生定安親王永璜,淑嘉皇貴妃金佳氏生履端親王永、儀慎親王永璇、第九子、成哲親王永瑆,愉貴妃珂裏葉特氏生榮純親王永琪,舒妃葉赫納喇氏生第十子。永出為履懿親王允祹後,永瑢出為慎靖郡王允禧後。永璟、永璐、第九子、第十子、第十六子皆殤,無封。

定安親王永璜,高宗第一子。乾隆十三年,上南巡,還蹕次德州,孝賢純皇后崩,永璜迎喪,高宗斥其不知禮,切責之。十五年三月,薨。上諭曰:「皇長子誕自青宮,齒序居長。年逾弱冠,誕毓皇孫。今遘疾薨逝,朕心悲悼,宜備成人之禮。」追封定親王,謚曰安。

子綿德,襲郡王。坐事,奪爵。弟綿恩,襲。五十八年,進封親王。嘉慶四年正月,封其子奕紹為不入八分輔國公。八年閏二月,有陳德者,匿禁門,犯蹕,諸王大臣捍禦。論功,賜綿恩御用補褂,進奕紹貝子。二十年,授御前大臣。道光二年,薨,賜銀五千治喪,謚曰恭。子奕紹,先以上六十萬壽進貝勒,至是襲親王。十五年,奕紹年六十,封其子載銓為輔國公。十六年,奕紹薨,賜銀治喪,謚曰端。載銓襲。

載銓初封二等輔國將軍,三進封輔國公,授御前大臣、工部尚書、步軍統領,襲爵。道光末,受顧命。文宗即位,益用事。咸豐二年六月,給事中袁甲三疏劾:「載銓營私舞弊,自謂『操進退用人之權』。刑部尚書恆春、侍郎書元潛赴私邸,聽其指使。步軍統領衙門但準收呈,例不審辦;而載銓不識大體,任意顛倒,遇有盜案咨部,乃以武斷濟其規避。又廣收門生,外間傳聞有定門四配、十哲、七十二賢之稱。」舉所繪息肩圖朝官題詠有師生稱謂為證。上諭曰:「諸王與在廷臣工不得往來,歷聖垂誡周詳。恆春、書元因審辦案件,趨府私謁,載銓並未拒絕。至拜認師生,例有明禁,而息肩圖題詠中,載齡、許誦恆均以門生自居,不知遠嫌。」罰王俸二年,所領職並罷。九月,仍授步軍統領。三年,加親王銜,充辦理巡防事宜。二月,疏請申明會議舊章,報可。四年九月,病作,詔以綿德曾孫溥煦為後。是月,薨。追封親王,賞銀五千兩治喪,謚曰敏。

溥煦襲郡王。光緒三十三年,薨,謚曰慎。子毓朗,襲貝勒。光緒末,授民政部侍郎、步軍統領。宣統二年七月,授軍機大臣。三年四月,改授軍諮大臣。

端慧太子永璉,高宗第二子。乾隆三年十月,殤,年九歲。十一月,諭曰:「永璉乃皇后所生,朕之嫡子,聰明貴重,氣宇不凡。皇考命名,隱示承宗器之意。朕御極後,恪守成式,親書密旨,召諸大臣藏於乾清宮「正大光明」榜後,是雖未冊立,已命為皇太子矣。今既薨逝,一切典禮用皇太子儀注行。」旋冊贈皇太子,謚端慧。

循郡王永璋,高宗第三子。乾隆二十五年七月,薨。追封循郡王。四十一年,以永瑆子綿懿為後,襲貝勒。卒,子奕緒,襲貝子。卒,子載遷,襲鎮國公。

榮純親王永琪,高宗第五子。乾隆三十年十一月,封榮親王。永琪少習騎射,嫺國語,上鍾愛之。三十一年三月,薨,謚曰純。子綿億,四十九年十一月,封貝勒。嘉慶四年正月,襲榮郡王。綿億少孤,體羸多病,特聰敏,工書,熟經史。十八年,林清變起,綿億方扈蹕,聞警,力請上速還京師,上即日回鑾,因重視之,寵眷日渥。逾年,薨,謚曰恪。子奕繪,襲貝勒。卒,子載鈞,襲貝子。卒,子溥楣,襲鎮國公。

哲親王永琮,高宗第七子,與端慧太子同為嫡子。端慧太子薨,高宗屬意焉。乾隆十二年十二月,以痘殤,方二歲。上諭謂:「先朝未有以元後正嫡紹承大統者,朕乃欲行先人所未行之事,邀先人不能獲之福,此乃朕過耶!」命喪儀視皇子從優,謚曰悼敏。嘉慶四年三月,追封哲親王。

儀慎親王永璇,高宗第八子。乾隆四十四年,封儀郡王。嘉慶四年正月,進封親王,總理吏部。二月,罷。諭曰:「六卿分職,各有專司,原無總理之名,勿啟專權之漸。」十三年正月,諭曰:「內廷行走諸王日入直,儀親王朕長兄,年逾六十,冬寒無事,不必進內。」十四年正月,封其子綿志為貝勒。十七年,以武英殿刻高宗聖訓,誤書廟諱,罷王俸三年。

十八年,林清變起,賊入禁城,綿志從宣宗發鳥槍殪賊。仁宗褒其奮勇,加郡王銜,加俸歲千兩。永璇亦以督捕勤勞,免一切處分。二十年七月,命祭裕陵,阻雨,還京,坐降郡王,並奪綿志郡王銜及加俸,仍罰王俸五年。二十四年正月,復綿志郡王銜,賜三眼孔雀翎。七月,坐刺探政事,上諭曰:「朕兄儀親王年已七十有四,精力漸衰。所領事務甚多,恐有貽誤,探聽尚有可原。朕不忍煩勞長兄,致失頤養。嗣後止留內廷行走,平日不必入直。」六月,綿志坐縱妾父冒職官詐贓,奪郡王銜,罰貝勒俸四年。

二十五年七月,宣宗即位,諭儀親王不必遠迎,又諭召對宴賚無庸叩拜。道光三年正月,綿志復郡王銜,加俸。八年正月,命在紫禁城乘轎,並加賞俸銀五千,示親親敬長之意。十一月,復諭朝賀免行禮。十年十月,永璇詣圓明園視大阿哥,徑入福園門,諭罷綿志官。十一年,諭壽皇殿、安佑宮當行禮時,於府第內行禮。又諭元旦暨正月十四日宗親筵宴,均免其入宴,別頒果殽一席。十二年八月,薨,年八十八。賜銀五千治喪,親臨賜奠,謚曰慎。綿志襲郡王,薨,謚曰順。子奕絪,襲貝勒,加郡王銜。卒,曾孫毓?,襲貝子。卒,弟毓岐,襲鎮國公。

成哲親王永瑆,高宗第十一子。乾隆五十四年,封成親王。永瑆幼工書,高宗愛之,每幸其府第。嘉慶四年正月,仁宗命在軍機處行走,總理戶部三庫。故事,親王無領軍機者,領軍機自永瑆始。二月,儀親王永璇罷總理吏部,並命永瑆俟軍務奏銷事畢,不必總理戶部。三月,和珅以罪誅,沒其園第,賜永瑆。七月,永瑆辭總理戶部三庫,允之。八月,編修洪亮吉上書永瑆,譏切朝政,永瑆上聞,上治亮吉罪。語在亮吉傳。十月,上諭曰:「自設軍機處,無諸王行走。因軍務較繁,暫令永瑆入直,究與國家定制未符。罷軍機處行走。」

永瑆嘗聞康熙中內監言其師少時及見董其昌以前三指握管懸腕作書,永星廣其說,作撥鐙法,推論書旨,深得古人用筆之意。上命書裕陵聖德神功碑,並令自擇書跡刻為詒晉齋帖,以手詔為序。刻成,頒賞臣工。

十八年,林清變起,永瑆在紫禁城內督捕,上嘉其勤勞,免一切處分及未完罰俸。二十四年正月,加其子不入八分輔國公綿懃郡王銜。五月,祭地壇,終獻時,贊引誤,永瑆依以行禮。上以永瑆年老多病,罷一切差使,不必在內廷行走,於邸第閉門思過,罰親王半俸十年。綿懃亦罷內大臣,居家侍父。二十五年六月,綿懃卒,贈郡王。有司請謚,以非例斥之,著為令。

仁宗崩,有旨免迎謁。語見儀親王傳。十月,命曾孫載銳襲貝勒。道光二年十月,上還自行在,永瑆進食品十六器,以非例卻之。三年三月,薨,年七十二,賜銀五千治喪,謚曰哲。載銳襲郡王。綿懃及載銳父奕綬並追封如其爵。咸豐九年,薨,謚曰恭。子溥莊,襲貝勒,加郡王銜。卒,子毓橚,襲貝子。

貝勒永?,高宗第十二子。乾隆四十一年,卒。嘉慶四年三月,追封貝勒。以成親王子綿偲為後,初封鎮國將軍,再進封貝子。道光十八年正月,諭曰:「綿偲逮事皇祖,昔同朕在上書房讀書者只綿偲一人。」進貝勒。二十八年,卒,子奕縉,襲貝子。卒,弟奕繕,襲鎮國公。

慶僖親王永璘,高宗第十七子。乾隆五十四年,封貝勒。嘉慶四年正月,仁宗親政,封惠郡王,尋改封慶郡王。三月,和珅誅,沒其宅賜永璘。五年正月,以祝穎貴太妃七十壽未奏明,命退出乾清門,留內廷行走。二十一年正月朔,乾清宮筵宴,輔國公綿?就席遲,奕紹推令入座,拂墮食?,永璘告內奏事太監。得旨:「諸王奏事不得逕交內奏事太監。」罰永璘俸。二十五年三月,永璘疾篤,上親臨視,命進封親王。尋薨,謚曰僖。命皇子往奠,上時謁陵歸,復親臨焉。

子綿?,襲郡王。綿?奏府中有毗盧帽門口四座、太平缸五十四件、銅路鐙三十六對。上諭曰:「慶親王府第本為和珅舊宅,凡此違制之物,皆和珅私置。嗣後王、貝勒、貝子當依會典,服物寧失之不及,不可僭逾,庶幾永保令名。」府置諳達二,亦命裁汰。道光三年正月,賜綿?三眼孔雀翎,管雍和宮、中正殿。十六年十月,薨,賜銀四千治喪,謚曰良。上命再襲郡王一次。

以儀順郡王綿志子奕採為後,襲郡王。十七年正月,命在御前行走。二十二年十月,奕採以服中納妾,下宗人府議處。奕採行賕請免,永璘第六子輔國公綿性亦行賕覬襲王爵,事發,奕採奪爵,綿性戍盛京。以永璘第五子不入八分鎮國公綿悌奉永璘祀。旋又坐事,降鎮國將軍。二十九年,卒。

以綿性子奕劻為後。三十年,襲輔國將軍。咸豐二年正月,封貝子。十年正月,上三十萬壽,進貝勒。同治十一年九月,大婚,加郡王銜,授御前大臣。光緒十年三月,命管理總理各國事務衙門。十月,進慶郡王。十一年九月,會同醇親王辦理海軍事務。十二年二月,命在內廷行走。十五年正月,授右宗正。大婚,賜四團正龍補服,子載振頭品頂帶。二十年,太后六十萬壽,懿旨進親王。二十六年七月,上奉太后幸太原,命奕劻留京會大學士李鴻章與各國議和。二十七年六月,改總理各國事務衙門為外務部,奕劻仍總理部事。十二月,加載振貝子銜。二十九年三月,授奕劻軍機大臣,仍總理外務部如故。尋命總理財政處、練兵處,解御前大臣以授載振。

載振赴日本大?觀展覽會歸,請振興商務,設商部,即以載振為尚書。十月,御史張元奇劾載振宴集召歌妓侑酒。上諭:「當深加警惕,有則改之,無則加勉。」旋請開缺,未許。三十年三月,御史蔣式瑆奏:「戶部設立銀行,招商入股。臣風聞上年十一月慶親王奕劻將私產一百二十萬送往東交民巷英商匯豐銀行收存。奕劻自簡任軍機大臣以來,細大不捐,門庭如市。是以其父子起居、飲食、車馬、衣服異常揮霍,尚能儲蓄鉅款。請命將此款提交官立銀行入股。」命左都御史清銳、戶部尚書鹿傳霖按其事,不得實,式瑆回原衙門行走。

三十一年,充日、俄修訂東三省條約全權大臣。三十二年,遣載振使奉天、吉林按事。改商部為農工商部,仍以載振為尚書。三十三年,命奕劻兼管陸軍部事。東三省改設督撫,以直隸候補道段芝貴署黑龍江巡撫。御史趙啟霖奏:「段芝貴善於迎合,上年貝子載振往東三省,道經天津,芝貴以萬二千金鬻歌妓以獻,又以十萬金為奕劻壽,夤緣得官。」上為罷芝貴,而命醇親王載灃、大學士孫家鼐按其事,不得實,奪啟霖官。載振復疏辭御前大臣、農工商部尚書,許之。三十四年十一月,命以親王世襲。

宣統三年四月,罷軍機處,授奕劻內閣總理大臣,大學士那桐、徐世昌協理大臣。八月,武昌兵起,初命陸軍部尚書廕昌視師,奕劻請於朝,起袁世凱湖廣總督視師。世凱入京師,代奕劻為內閣總理大臣,授奕劻弼德院總裁。十二月,詔遜位,奕劻避居天津。後七年薨,謚曰密。

仁宗五子:孝淑睿皇后生宣宗,孝和睿皇后生惇恪親王綿愷、瑞懷親王綿忻,恭順皇貴妃鈕祜祿氏生惠端親王綿愉,和裕皇貴妃劉氏生穆郡王。

穆郡王,未命名,仁宗第一子。二歲,殤。宣宗即位,追封。

惇恪親王綿愷,仁宗第三子。嘉慶十八年,林清變起,綿愷隨宣宗捕賊蒼震門,得旨褒嘉。二十四年,封惇郡王。宣宗即位,進親王。子奕纘,封不入八分公。道光三年正月,命綿愷內廷行走。旋以福晉乘轎徑入神武門,坐罷,罰王俸五年。上奉太后幸綿愷第,仍命內廷行走,減罰王俸三年。七年,坐太監張明得私相往來,復匿太監苑長青,降郡王。八年十月,追?蒼震門捕賊,急難禦侮,復親王,諭加意檢束。十三年五月,綿愷以議皇后喪禮引書「百姓如喪考妣,四海遏密八音」,於義未協,退出內廷,罰王俸十年。十八年五月,民婦穆氏訴其夫穆齊賢為綿愷所囚,命定郡王載銓按實,復降郡王,罷一切職任。十二月,薨,復親王。上親臨奠,謚曰恪。奕纘前卒,追封貝勒,命賜福晉郡王半俸。

二十六年,以皇五子奕脤為綿愷後,襲郡王。文宗即位,命在內廷行走。奕脤屢以失禮獲譴。咸豐五年三月,降貝勒,罷一切職任,上書房讀書。六年正月,復封惇郡王。十月,進親王。穆宗即位,諭免叩拜稱名。同治三年,江寧克復,封其子載濂不入八分鎮國公,載津賜頭品頂帶。四年六月,授宗令。七年正月,捻匪逼近畿,奕脤陳防守之策。八年十一月,醇郡王奕枻劾王自授宗令,藉整頓之名,啟攬權之漸,詔兩解之。十一年,大婚,賜紫內大臣班及帶豹尾槍。載濂進輔國公。十三年十二月,賜親?禁城乘四人肩輿,並免進領侍王雙俸。光緒五年六月,普祥峪吉地工竣,復賜食雙俸。十三年,上親政,免帶領引見。十五年正月,薨,上奉太后臨奠,謚曰勤。

子八,有爵者五:載濂、載漪、載瀾、載瀛、載津。載濂,奕脤第一子。初封一等輔國將軍,累進輔國公,襲貝勒,加郡王銜。二十五年,子溥偁,賜頭品頂帶。二十六年,載濂以庇義和拳,奪爵,弟載瀛,襲。載瀛,奕脤第四子。初封二等鎮國將軍,加不入八分輔國公銜,襲貝勒。載漪,奕脤第二子。出為瑞郡王奕志後。獲罪,奪爵,歸宗。語在瑞懷親王綿忻傳。載瀾,奕脤第三子。初封三等輔國將軍,再進封不入八分輔國公。以庇義和拳,奪爵,戍新疆。載津,奕脤第五子。封二等鎮國將軍,加不入八分輔國公銜。

瑞懷親王綿忻,仁宗第四子。嘉慶二十四年,封瑞親王。道光三年,命在內廷行走。八年七月,薨,謚曰懷。子奕約甫晬,上命定親王奕紹檢察邸第官吏,內務府大臣敬徵治家政。十月,奕約襲郡王,予半俸。尋更名奕志。三十年五月,薨,謚曰敏。無子。賜綿忻福晉郡王半俸。咸豐三年,福晉薨,復賜奕志福晉郡王半俸。

十年,命以惇親王子載漪為奕志後,襲貝勒。同治十一年,大婚,命食貝勒全俸。光緒十五年,加郡王銜。十九年九月,授為御前大臣。二十年,進封端郡王。循故事,宜仍舊號;更曰端者,述旨誤,遂因之。載漪福晉,承恩公桂祥女,太后侄也。二十四年,太后復訓政。二十五年正月,賜載漪子溥?頭品頂帶。十二月,上承太后命,溥?入為穆宗後,號「大阿哥」,命在弘德殿讀書,以承恩公尚書崇綺、大學士徐桐為之傅。明年元旦,大高殿、奉先殿行禮,以溥?代。都下流言將下詔禪位,大學士榮祿與慶親王奕劻以各國公使有異同,諫止。

二十六年,義和拳亂起,載漪篤信之,以為義民,亂遂熾。五月,命充總理各國事務大臣。義和拳擊殺日本使館書記杉山彬,復及德國使臣克林德,圍攻東交民巷使館。八月,諸國聯軍自天津逼京師,上奉太后出狩,載漪及溥?皆從。次大同,命載漪為軍機大臣,未逾月罷。命奕劻與大學士李鴻章議和,諸國目載漪為首禍。十二月,奪爵,戍新疆。二十七年十月,上奉太后還京師。次開封,諭:「載漪縱義和拳,獲罪祖宗,其子溥?不宜膺儲位,廢『大阿哥』名號。」賜公銜俸,歸宗。

二十八年六月,別以醇賢親王奕枻子鎮國公載洵為奕志後,襲貝勒。宣統間,為海軍部尚書。改海軍部大臣,加郡王銜。

惠端親王綿愉,仁宗第五子。嘉慶二十五年七月,宣宗即位,封惠郡王,在內廷行走,上書房讀書。故事,親、郡王未及歲,食半俸。道光九年,命食全俸。十九年,進親王。文宗即位,諭:「惠親王為朕叔父,內廷召對及宴賚賞賜宜免叩拜,章奏免書名。」咸豐三年,賜御用龍褂。

洪秀全之徒北擾近畿,命為奉命大將軍,頒銳捷刀,統健銳、火器、前鋒、護軍、巡捕諸營,及察哈爾兵,哲裏木、卓索圖、昭烏達東三盟蒙古兵,與科爾沁郡王僧格林沁督辦防剿。僧格林沁出駐涿州,綿愉留京師。九月,會奏頒行銀錢鈔法。時秀全兵至深州,請發哲裏木盟馬隊一千及熱河、古北口兵各五百赴涿州助防;復奏請發蒙古兵三千,以德勒克色楞為將,督兵進擊。

四年正月,命朝會大典外悉免叩拜。尋與恭親王奕、定郡王載銓疏請鑄鐵錢為大錢輔,上令王詳議以行。五年四月,北路肅清,行凱撤禮,上奉命大將軍印。十二月,以鑄鐵錢有效,下宗人府議敘。八年五月,以奏保耆英,罷中正殿、雍和宮諸職任。九年,罷鐵錢局。

十年七月,英、法二國兵至天津,命至通州與僧格林沁辦防,並諭綿愉及怡親王載垣、鄭親王端華、尚書肅順、軍機大臣等籌商交涉。同治二年,穆宗典學,太后以綿愉行輩最尊,品行端正,命在弘德殿專司督責,並令王子奕詳、奕詢伴讀。三年十二月,薨,上親臨奠,賜銀五千治喪,謚曰端。

子六,有爵者三:奕詳、奕詢、奕謨。奕詳,綿愉第五子。初封不入八分輔國公。賜三眼孔雀翎,進鎮國公,襲郡王。穆宗大婚,加親王銜。十三年,命食親王俸。光緒十年十月,太后萬壽,命食親王全俸。十一年六月,授內大臣。十二年正月,薨,謚曰敬。子載潤,襲貝勒。奕詢,綿愉第四子。初封不入八分輔國公,進封鎮國公。卒,無子,以愉恪郡王允潖五世孫載澤為後,襲輔國公,進鎮國公,加貝子銜。光緒末,授度支部尚書。奕謨,綿愉第六子。初封不入八分鎮國公,再進封貝子,加貝勒銜。卒,以醇賢親王奕枻孫溥佶為後,襲鎮國公。

宣宗九子:孝全成皇后生文宗,孝靜成皇后生順和郡王奕綱、慧質郡王奕繼、恭忠親王奕,莊順皇貴妃生醇賢親王奕枻、鍾端郡王奕硉、孚敬郡王奕譓,和妃納喇氏生隱志郡王奕緯,祥妃鈕祜祿氏生惇勤親王奕脤。奕脤出為惇恪親王綿愷後。

隱志郡王奕緯,宣宗第一子。嘉慶二十四年,封貝勒。道光十一年四月,薨,以皇子例治喪,進封隱志貝勒。文宗即位,進郡王。無子,以貝勒綿懿子奕紀為後,襲貝勒。卒,謚恭勤。子溥倫,襲貝子,進貝勒;溥侗,授一等鎮國將軍。

順和郡王奕綱,宣宗第二子。二歲,殤。文宗即位,進封謚。

慧質郡王奕繼,宣宗第三子。三歲,殤。文宗即位,追封謚。

恭忠親王奕,宣宗第六子。與文宗同在書房,肄武事,共制槍法二十八勢、刀法十八勢,宣宗賜以名,槍曰「棣華協力」,刀曰「寶鍔宣威」,並以白虹刀賜奕。文宗即位,封為恭親王。咸豐二年四月,分府,命仍在內廷行走。

內大臣辦理巡防,命仍佩白虹刀。十月,?三年九月,洪秀全兵逼畿南,以王署領侍命在軍機大臣上行走。四年,迭授都統、右宗正、宗令。五年四月,以畿輔肅清,予優?。七月,孝靜成皇后崩,上責王禮儀疏略,罷軍機大臣、宗令、都統,仍在內廷行走,上書房讀書。七年五月,復授都統。九年四月,授內大臣。

十年八月,英吉利、法蘭西兵逼京師,上命怡親王載垣、尚書穆廕與議和,誘執英使巴夏禮,與戰,師不利。文宗幸熱河,召回載垣、穆廕,授王欽差便宜行事全權大臣。王出駐長辛店,奏請飭統兵大臣激勵兵心,以維大局。克勤郡王慶惠等奏釋巴夏禮,趣王入城議和。英、法兵焚圓明園。豫親王義道等奏啟城,許英、法兵入。王入城與議和,定約,悉從英、法人所請,奏請降旨宣示,並自請議處。上諭曰:「恭親王辦理撫局,本屬不易。朕深諒苦衷,毋庸議處。」十二月,奏通商善後諸事。初設總理各國事務衙門,命王與大學士桂,邊防空虛,?良、侍郎文祥領其事。王疏請訓練京師旗兵,並以吉林、黑龍江與俄羅斯相議練兵籌餉。上命都統勝保議練京兵,將軍景淳等議練東三省兵。

十一年七月,文宗崩,王請奔赴,兩太后召見,諭以贊襄政務王大臣載垣、端華、肅順等擅政狀。穆宗侍兩太后奉文宗喪還京師,譴黜載垣等,授議政王,在軍機處行走,命王爵世襲,食親王雙俸,並免召對叩拜、奏事書名。王堅辭世襲,尋命兼宗令、領神機營。

同治元年,上就傅,兩太后命王弘德殿行走,稽察課程。三年,江寧克復。上諭曰:「恭親王自授議政王,於今三載。東南兵事方殷,用人行政,徵兵籌餉,深資贊畫,弼亮忠勤。加封貝勒,以授其子輔國公載澂,並封載濬輔國公、載水瑩不入八分輔國公。」四年三月,兩太后諭責王信任親戚,內廷召對,時有不檢,罷議政王及一切職任。尋以惇親王奕脤、醇郡王奕枻及通政使王拯、御史孫翼謀、內閣學士殷兆鏞、左副都御史潘祖廕、內閣侍讀學士王維珍、給事中廣誠等奏請任用,廣誠語尤切。兩太后命仍在內廷行走,管理總理各國事務衙門。王入謝,痛哭引咎,兩太后復諭:「王親信重臣,相關休戚,期望既厚,責備不得不嚴。仍在軍機大臣上行走。」

七年二月,西捻逼畿輔,命節制各路統兵大臣。授右宗正。十一年九月,穆宗大婚,復命王爵世襲。十二年正月,穆宗親政,十三年七月,上諭責王召對失儀,降郡王,仍在軍機大臣上行走,並奪載澂貝勒。翌日,以兩太后命復親王世襲及載澂爵。十二月,上疾有間,於雙俸外復加賜親王俸。旋復加劇,遂崩。德宗即位,復命免召對叩拜、奏事書名。

光緒元年,署宗令。十年,法蘭西侵越南,王與軍機大臣不欲輕言戰,言路交章論劾。太后諭責王等委靡因循,罷軍機大臣,停雙俸。家居養疾。十二年十月,復雙俸。自是國及甲數,歲時祀事賜神糕,節序輒有賞賚,以為常。二十年,日本侵朝鮮,兵?有慶屢增護事急,太后召王入見,復起王管理總理各國事務衙門,並總理海軍,會同辦理軍務,內廷行走;仍諭王疾未愈,免常川入直。尋又命王督辦軍務,節制各路統兵大臣。十一月,授軍機大臣。二十四年,授宗令。王疾作,閏三月增劇,上奉太后三臨視,四月薨,年六十七。上再臨奠,輟朝五日,持服十五日。謚曰忠,配享太廟,並諭:「王忠誠匡弼,悉協機宜,諸臣當以王為法。」

子四:載澂,貝勒加郡王銜,卒,謚果敏;載瀅,出為鍾端郡王奕硉後,襲貝勒,坐事奪爵歸宗;載濬,與載瀅同時受封;載潢,封不入八分輔國公。載澂、載濬、載潢皆前王卒。王薨,以載瀅子溥偉為載澂後,襲恭親王。

醇賢親王奕枻,宣宗第七子。文宗即位,封為醇郡王。咸豐九年三月,分府,命仍在內大臣,管?內廷行走。穆宗即位,諭免宴見叩拜、奏事書名。迭授都統、御前大臣、領侍神機營。同治三年,加親王銜。四年,兩太后命弘德殿行走,稽察課程。十一年,進封醇親王。十二年,穆宗親政,罷弘德殿行走。

德宗即位,王奏兩太后,言:「臣侍從大行皇帝十有三年,昊天不吊,龍馭上賓。仰瞻遺容,五內崩裂。忽蒙懿旨下降,擇定嗣皇帝,倉猝昏迷,罔知所措。獨犯舊有肝疾,委頓成廢。惟有哀懇矜全,許乞骸骨,為天地容一虛糜爵位之人,為宣宗成皇帝留一庸鈍無才之子。」兩太后下其奏王大臣集議,以王奏誠懇請罷一切職任,但令照料菩陀峪陵工,從之。命王爵世襲,王疏辭,不許。光緒二年,上在毓慶宮入學,命王照料。五年,賜食親王雙俸。

十年,恭親王奕罷軍機大臣,以禮親王世鐸代之,太后命遇有重要事件,與王商辦。時法蘭西侵越南,方定約罷兵,王議建海軍。十一年九月,設海軍衙門,命王總理,節制沿海水師,以慶郡王奕劻、大學士總督李鴻章、都統善慶、侍郎曾紀澤為佐。定議練海軍自北洋始,責鴻章專司其事。十二年三月,賜王與福晉杏黃轎,王疏辭,不許。鴻章經畫海防,於旅順開船塢,築砲臺,為海軍收泊地。北洋有大小戰艦凡五,輔以蚊船、雷艇,復購艦英、德,漸次成軍。五月,太后命王巡閱北洋,善慶從焉,會鴻章自大沽出海至旅順,歷威海、煙臺,集戰艦合操,?視?臺、船塢及新設水師學堂,十餘日畢事。王還京,奏?諸將吏及所聘客將,請太后御書榜懸大沽海神廟。

太后命於明年歸政,王疏言:「皇帝甫逾志學,諸王大臣籥懇訓政,乞體念時艱,俯允所請,俟及二旬,親理庶務。至列聖宮廷規制,遠邁前代。將來大婚後,一切典禮,咸賴訓教。臣愚以為諸事當先請懿旨,再於皇帝前奏聞,俾皇帝專心大政,承聖母之歡顏,免宮闈之劇務。此則非如臣生深宮者不敢知,亦不敢言也。」太后命毋庸議。十三年正月,上親政。四月,太后諭預備皇帝大婚,當力行節儉,命王稽察。十四年九月,王奏:「太平湖賜第為皇帝發祥地。世宗以潛邸升為宮殿,高宗諭子孫有自籓邸紹承大統者,應用其例。」太。葺?後從之,別賜第,發帑十萬葺治。十五年正月,大婚禮成,賜金桃皮鞘威服刀,增護治邸第未竟,復發帑六萬。並進封諸子:載灃鎮國公,載洵輔國公,載濤賜頭品頂帶、孔雀翎。

二月,河道總督吳大澂密奏,引高宗御批通鑒輯覽,略謂:「宋英宗崇奉濮王,明世宗崇奉興王,其時議者欲改稱伯叔,實人情所不安,當定本生名號,加以徽稱」;且言:「在臣子出為人後,例得以本身封典貤封本生父母,況貴為天子,天子所生之父母,必有尊崇之典,請飭廷臣議醇親王稱號禮節。」特旨宣示。上即位逾年,王密奏:「臣見歷代繼承大統之君,推崇本生父母者,備載史書。其中有適得至當者焉,宋孝宗不改子偁秀王之封是也。有大亂之道焉,宋英宗之濮議、明世宗之議禮是也。張璁、桂之儔,無足論矣。忠如韓琦,乃與司馬光議論牴牾,其故何歟?蓋非常之事出,立論者勢必紛沓擾攘,乃心王室,不無其人;而以此為梯榮之具,迫其主以不得不視為莊論者,正復不少。皇清受天之命,列聖相承,十朝一脈,詎穆宗毅皇帝春秋正盛,遽棄臣民。皇太后以宗廟社稷為重,特命皇帝入承大統,復推恩及臣,以親王世襲罔替。渥叨異數,感懼難名。原不須更生過慮,惟思此時垂簾聽政,簡用賢良,廷議既屬執中,邪說自必潛匿。倘將來親政後,或有草茅新進,趨六年拜相捷徑,以危言故事聳動宸聰,不幸稍一夷猶,則朝廷滋多事矣。仰懇皇太后將臣此摺,留之宮中。俟皇帝親政,宣示廷臣世賞之由及臣寅畏本意,千秋萬載,勿再更張。如有以治平、嘉靖之說進者,務目之為奸邪小人,立加屏斥。果蒙慈命嚴切,皇帝敢不欽遵,不但臣名節得以保全,而關乎君子小人消長之機者,實為至大且要。」太后如王言,留疏宮中。大澂疏入,諭曰:「皇帝入承大統,醇親王奕枻謙卑謹慎,翼翼小心,十餘年來,殫竭心力,恪恭盡職。每優加異數,皆涕泣懇辭,前賜杏黃轎,至今不敢乘坐。其秉心忠赤,嚴畏殊常,非徒深宮知之最深,實天下臣民所共諒。光緒元年正月初八日,王即有豫杜妄論一奏,請俟親政宣示,俾千秋萬載,勿再更張。自古純臣居心,何以過此?當歸政伊始,吳大澂果有此奏,特明白曉諭,並將王原奏發鈔,俾中外咸知賢王心事,從此可以共白。闞名希寵之徒,更何所容其覬覦乎?」

十六年正月,以上二十萬壽,增護軍十五、藍白甲五十,授載濤二等鎮國將軍。十一月,王疾作,上親詣視疾。丁亥,王薨,年五十一。太后臨奠,上詣邸成服。定稱號曰皇帝本生考,稱本生考,遵高宗御批;仍原封,從王志也。謚曰賢,配享太廟。下廷臣議:上持服期年,縞素、輟朝十一日;初祭、大祭,奉移前一日,親詣行禮,御青長袍褂,摘纓;期年內御便殿,用素服;葬以王,祭以天子,立廟班諱。十八年,葬京師西山妙高?。宣統皇帝即位,定稱號曰皇帝本生祖考。

子七:德宗,其第二子也;載洸,初封不入八分輔國公,進鎮國公;載灃,襲醇親王,宣統皇帝即位,命為監國攝政王;載洵,出為瑞郡王奕志後;載濤,出為鍾郡王奕硉後。宣統間,載洵為海軍部大臣,載濤為軍諮府大臣,主軍政。三年十月,並罷。十二月,遜位。

鍾端郡王奕硉,宣宗第八子。文宗即位,封為鍾郡王。穆宗即位,命免宴見叩拜、奏事書名。同治三年,分府,仍在內廷行走。七年十一月,薨,謚曰端。無子,以恭忠親王奕子載瀅為後,襲貝勒。坐事,奪爵,歸宗。又以醇賢親王奕枻子載濤為後,襲貝勒,加郡王銜。

孚敬郡王奕譓,宣宗第九子。文宗即位,封孚郡王。穆宗即位,命免宴見叩拜、奏事書名。同治三年,分府,仍在內廷行走,命管樂部。十一年,授內大臣,加親王銜。德宗即位,復命免宴見叩拜、奏事書名。光緒三年二月,薨,謚曰敬。無子,以愉恪郡王允潖四世孫奕棟子載沛為後,襲貝勒。卒,又以奕瞻子載澍為後,襲貝勒,坐事奪爵歸宗;又以貝勒載瀛子溥伒為後,封貝子。

文宗二子:孝欽顯皇后生穆宗,玫貴妃徐佳氏生憫郡王。

憫郡王,生未命名,殤。穆宗即位,追封。

論曰:莊親王佐太祖建業,將出師,登?而謀,策定馳而下,黃道周亟稱其驍勇;太祖崩,諸子嗣業,未有成命,禮烈親王擁立太宗,親為捍禦邊圉,夏允彞以為行事何減聖賢。蓋雄才讓德,雖在敵國,不能掩也。睿忠親王手定中原,以致於世祖,求之前史,實罕其倫。徒以執政久,威福自專,其害肅武親王,相傳謂因師還賜宴拉殺之,又或謂還至郊外遇伏死,死處即今葬地。傳聞未敢信,然其慘酷可概見矣。身後蒙謗,僅乃得雪,亦有以取之也。

聖祖遇諸宗人厚,遺詔猶以禮親王、饒餘親王子孫安全,拳拳在念。然當用兵時,諸王貝勒為帥,小違律必議罰,且不得以功掩。義以行法,仁以睦親,固不相悖也。雍正中,允禩、允禟之獄,世宗後亦悔之。怡賢親王特馴謹,渥加寵榮,示非寡恩。誠以尺布斗粟,相逼笮過甚,恂勤郡王嘗握兵柄,非母弟亦豈得幸生耶?時去開國未遠,以尚武為家法,其失則獷。

太宗屢諭諸子弟當讀書,?厚公承其教,彬彬有東丹王之風。高宗諸子多擅文學,尤以成哲親王為最,詞章書翰,無愧古人。恭忠親王繼以起,綢繆宮府,定亂綏疆,罷不生懟,用不辭勞,有純臣之度焉。醇賢親王尊為本生親,乾乾翼翼,靡間初終,預絕治平、嘉靖之議,載在方策,彰彰邁前代遠甚。迨時移勢易,天方降割,乃以肺腑之親,寄腹心之重,漠然不知陰雨之已至,一發而不可復收。天歟人歟,亡也忽諸,尤足為後來之深鑒矣!


\end{pinyinscope}