\article{列傳九十一}

\begin{pinyinscope}
張照甘汝來陳★華王安國劉吳龍楊汝穀張泰開

秦蕙田彭啟豐夢麟

張照,字得天,江南婁縣人。康熙四十八年進士,改庶吉士,授檢討,南書房行走。雍正初,累遷侍講學士。聖祖訓士民二十四條,世宗為之註,題曰聖諭廣訓,照疏請下學官,令學童誦習。復三遷刑部侍郎。十一年,授左都御史,遷刑部尚書,疏請更定律例數事。

大學士鄂爾泰初為雲貴總督,定亂苗,稍收其地,置流官。既而苗復叛,揚威將軍哈元生、副將軍董芳討之,不以時定。上責鄂爾泰措置不當,照素忤鄂爾泰,因請行。十三年五月,上命照為撫定苗疆大臣。照至貴州,議劃施秉以上為上游,用雲南、貴州兵,專屬元生;以下為下游,用湖廣、廣東兵,專屬芳:令諸軍互易地就所劃。元生、芳遂議村落道路皆別上下界,文移辨難。照致書元生等,令劾鄂爾泰。會高宗即位,召照還,以湖廣總督張廣泗往代。上怒照挾私誤軍興,廣泗復劾照謬妄,元生等並發照致書令劾鄂爾泰事,遂奪職逮下獄。乾隆元年,廷議當斬,上特命免死釋出獄,令在武英殿修書處行走。

二年,起內閣學士,南書房行走。五年,復授刑部侍郎。照言:「律例新有更定,校刻頒行諸行省,期以一年。舊輕新重者,待新書至日遵行,不必駁改;舊重新輕者,刑部即引新書更正。庶一年內薄海內外早被恩光。」特旨允行。上以朝會樂章句讀不協節奏,慮壇廟樂章亦復如是,命莊親王允祿及照遵聖祖所定律呂正義,考察原委。尋合疏言:「律呂正義編摩未備,請續纂後編。壇廟朝會樂章,考定宮商字譜,備載於篇,使律呂克諧,尋考易曉。民間俗樂,亦宜一體釐正。」下部議行。七年,疏請矜恤軍流罪人妻孥,罪人發各邊鎮給旗丁為奴,其在籍子孫到配所省視,旗丁不得並沒為奴。

尋擢刑部尚書,兼領樂部。民間貸錢徵息,子母互相權,謂之「印子錢」。雍正間,八旗佐領等有以印子錢朘所部旗丁者,世宗諭禁革。都統李禧因請貸錢者得自陳,免其償,並治貸者罪。至是,照言印子錢宜禁,如止重利放債,依違禁取利本律治罪,禧所議宜罷不用,從之。九年十二月,父匯卒於家,照方有疾,十年正月,奔喪,上勉令節哀,毋致毀瘠。至徐州,卒,加太子太保、吏部尚書,謚文敏。

照敏於學,富文藻,尤工書。其以苗疆得罪,高宗知照為鄂爾泰所惡,不欲深罪照,滋門戶恩怨。重惜照才,復顯用。及照卒,見照獄中所題白雲亭詩意怨望,又指照集憤嫉語,諭諸大臣以照已死不追罪。後數年,一統志奏進,錄國朝松江府人物不及照,上復命補入,謂:「照雖不醇,而資學明敏,書法精工,為海內所共推,瑕瑜不掩,其文採風流不當泯沒也。」

甘汝來,字耕道,江西奉新人。康熙五十二年進士,以教習授知縣,補直隸淶水知縣。淶水旗丁與民雜居,汝來至,請罷雜派,以火耗補之。禁莊田無故增租易佃。旗丁例不得行笞,汝來請以柳梃約束。三等侍衛畢裏克調鷹至淶水,居民家,僕捶民幾斃,訴於汝來。畢裏克率其僕閧於縣庭,汝來逮畢裏克,械其僕於獄。事聞,下刑部議,奪汝來職,畢裏克罰俸,聖祖命奪畢裏克職,汝來無罪。汝來自是負循吏名。移知新安縣,鑿白楊澱堤,溉田數千頃。又移知雄縣,懲奸吏,復請罷雜派。雍正初,授吏部主事,擢廣西太平府知府,三遷至廣西巡撫。五年,遷都察院左副都御史。

汝來為按察使時,李紱為巡撫,奉議州土司羅文剛糾眾阻塘汛,吏請兵捕治,紱與汝來持不許。事聞,世宗命紱、汝來如廣西捕文剛。廣西巡撫韓良輔如雲南,與總督鄂爾泰計事,上令汝來署巡撫。泗城府土司岑映宸所部民相仇,汝來與鄂爾泰、良輔、紱設謀縶映宸,隸其土流官。汝來請於鎮安土府置學官,上以非苗疆急務,責其沽名。又以汝來謝恩疏言曲賜寬容,上詰之曰:「人君持國法,當行直道,曲則不直,汝來語何意?」召還京。六年,良輔獲文剛,汝來坐疏縱奪職,在咸安宮官學行走。山東巡撫費金吾議濬濟寧、嘉祥、沛縣等處水道,命汝來效力。九年,起直隸霸昌道。丁母憂,令在任守制。

再遷禮部侍郎。高宗即位,議行三年喪,諮於諸大臣,汝來曰:「三年之喪,無貴賤,一也。皇上法堯、舜之道,宜行周、孔之禮,立萬年彞倫之極。」或言二十七月中朝祭大典若有所妨,汝來曰:「墨縗視事,越紼以祭,禮固言之,夫何疑?」乃考載籍,上儀制,援古證今,具有條理。

遷兵部尚書,疏言:「廣東海濱微露灘形,民間謂之『水坦』。漸生青草,謂之『草坦』。徐成耕壤,謂之『沙坦』。坦初見,沿海民報圍築者,當先令立標定四至,毋於圍築後爭控。民有田十頃以上,毋許圍築,以杜豪占。即貧民圍築,限五頃。其出工本牛種助他人圍築量取租息者,聽。陸地開墾例六年升科,海田浮脃,當寬至十年。潮大至坦沒,蠲一歲糧。圍毀則免升科原額。」疏入,敕廣東督撫議行。復疏言:「海濱居民單桅船採捕魚蝦,例不輸稅。近聞各海關監督與雙桅船同令領牌納鈔,又閩、廣間貧民有置𥭋取魚者,有就埠育鴨者,吏或按𥭋按埠私徵稅,請通行嚴禁。」從之。乾隆三年,調吏部尚書,仍兼領兵部,加太子少保。

四年七月,汝來方詣廨治事,疾作,遂卒。大學士訥親領吏部,與共治事,親送其喪還第。至門,訥親先入,嫗縫衣於庭,納親謂曰:「傳語夫人,尚書暴薨於廨矣!」嫗愕曰:「汝誰也?」訥親具以告,嫗汪然而泣,始知即汝來妻也。訥親因問有餘貲否,嫗曰:「有。」持囊出所餘俸金,訥親為感泣。奏上,上獎其寒素,賜銀千兩,命吏經紀其喪,謚莊恪。

嘉慶間,汝來曾孫紹烈應順天鄉試,以懷挾得罪,仁宗猶念汝來居官持正,宥紹烈,命仍得原名應試。

陳德華,字云倬,直隸安州人。雍正二年一甲一名進士,授修撰,再遷侍讀學士。提督廣東肇高學政,旋調廣韶學政。遭母喪歸,未終制,召充一統志館副總裁官。乾隆元年,遷詹事,上書房行走,再遷刑部侍郎。四年,遷戶部尚書。七年,調兵部尚書。八年,以弟德正為陜西按察使,讞獄用酷刑,為巡撫塞楞額所劾。德正具密摺擬揭部科,為書告德華,德華沮之,未奏聞。上以德華既知德正事非是,當奏聞,乃為隱匿,非大臣體,且曰:「父為子隱,子為父隱,直在其中。朕非不知以此風天下。然君臣之倫,實在弟兄之上。」下部議奪職,命左遷兵部侍郎。十二年,以議處江西總兵高琦武備廢弛,違例邀譽,奪職。十四年,起為左副都御史,上書房行走。以督諸皇子課怠,屢詰責奪俸。二十二年,遷工部侍郎。二十三年,遷禮部尚書。二十九年,致仕。三十六年,皇太后萬壽,詔繪九老圖,以德華入致仕九老中。四十四年,卒,年八十三。

德華性篤儉,縕袍蔬食,蕭然如寒素。立身循禮法,而不自居道學。嘗謂:「士大夫之患,莫大於近名。求以立德名,則必有迂怪不情之舉而實行荒;求以立言名,則必有異同勝負之論而正理晦;求以立功名,則必務見所長,紛更舊制。立一法反生一弊,而實行無所裨。」方為尚書時,京師富民俞民弼死,諸大臣皆往吊。上聞,察未往者,德華與焉。

王安國,字春圃,江南高郵人。雍正二年一甲二名進士,授編修,再遷侍講。提督廣東肇高學政,復再遷左僉都御史。乾隆二年,疏請禁官吏居喪詣省會謁大吏,下部議行。復三遷左都御史。五年,兩江總督馬爾泰論廣東巡撫王謩徇縱,命安國往按,即命以左都御史領廣東巡撫。安國曰:「吾奉命勘事而即得其位,古所譏蹊田奪牛者非歟?」疏力辭,上不許。廣東俗奢靡,安國事事整肅,倉有餘粟。故事,自總督以下皆有分,安國獨以非制,止之。九年正月,就遷兵部尚書,尋遭父喪。廣州將軍策楞疏言安國孤介廉潔,歸葬無貲,與護理巡撫託庸等具賻歸之,報聞。

十年,召為兵部尚書,調禮部。安國疏乞終喪,居廬營葬。服闋,乃入朝。十四年六月,安國入對,言諸行省方科試,諸學臣尚有未除積弊。上令具疏陳,安國疏言:「上科鄉試後,頗聞諸學臣因錄科例嚴,轉開僥幸。或於省會書院博督撫之歡,或於所屬義學徇州縣之請,或市恩於朝臣故舊,或縱容子弟家人乘機作弊,致取錄不甚公明。」上召安國詢所論諸學臣姓名,安國舉尹會一、陳其凝、孫人龍、鄧釗等。上以會一、釗已物故,其凝、人龍皆坐事黜,因責安國瞻徇,手詔詰難。二十年,遷吏部尚書。二十一年,疏乞假為父改葬。上以來年當南巡,諭俟期扈行。冬,病作,予假治疾。二十二年春,卒,賜白金五百治喪,謚文肅。

安國初登第,謁大學士硃軾,軾戒之曰:「學人通籍後,惟留得本來面目為難。」安國誦其語終身。至顯仕,衣食器用不改於舊。深研經籍,子念孫,孫引之,承其緒,成一家之學,語在儒林傳。

劉吳龍,字紹聞,江西南昌人。雍正元年進士,授庶吉士。二年,以硃軾薦,改吏部主事。六遷至光祿寺少卿。嘗視讞牘,有以欲劫行舟定罪者,吳龍曰:「欲劫二字,豈可置人於死?」論釋之。十一年,出為安徽按察使。十三年,內遷光祿寺卿,命管理北路軍需。乾隆元年,召還,疏言:「北路軍需,有輸送科布多截留察漢廋爾諸處,應就車駝戶追繳腳價。尚有逋負,請量予豁除。」上從其議。三遷左都御史,疏言:「步軍統領衙門番役,私用白役,生事害民,宜令具冊考覈,有所追捕,官畀差票,詣有司呈驗。步軍統領鞫囚,旗人會本旗都統,民人會順天府尹、巡城御史,互相覺察。」疏入,議行。又疏言諸行省州縣董理訟獄,其有舛誤,小民無所申訴,宜令督撫遣監司按行稽考,以申民隱。旋劾罷浙江巡撫盧焯,論如律。遷刑部尚書。七年,卒,賜白金五百治喪,謚清愨。

吳龍簡重,不茍言笑。為政慎密持重,得大體。督學直隸、江蘇,士循其教。乾隆初,楊汝穀、張泰開與吳龍先後為左都御史,皆以篤謹被上眷。

楊汝穀,字令貽,江南懷寧人。康熙三十九年進士,授浙江浦江縣知縣。行取,授禮部主事。三遷監察御史。河南南陽鎮標兵以知府沈淵禁博,劫淵,圍諸教場三日。汝穀論劾,上遣尚書張廷樞等往按,譴總兵高成誅標兵之首事者。別疏言:「選人待缺,輒言出為人後,或值遠缺,報治喪,冀更選。請飭選人具三代,已選,復稱出為人後,報治喪,以不孝論。」下部議行。六遷兵部侍郎,兼署左副都御史。疏言直隸被水災,請運關東米十萬石至天津,留南漕十萬石存河間、保定適中地,分貯備賑。下部議行。高宗即位,調戶部侍郎,疏言:「河南滎澤地濱黃河,康熙三十六年河勢南侵,縣地多傾陷。民困虛糧,流亡遠徙。」上命河南巡撫察議,刪賦額。尋遷左都御史。乾隆三年,以老乞休,命本省布政使給俸。五年,卒,年七十六,謚勤恪。

張泰開,字履安,江南金匱人。乾隆七年進士,改庶吉士,命上書房行走。旋自編修五遷禮部侍郎。十九年,國子監學錄缺員,泰開舉同部侍郎鄒一桂子志伊。上責其瞻徇,部議奪職,予編修,仍在上書房行走。二十年,內閣學士胡中藻為詩謗朝政,坐誅,泰開為詩序,授刻,部議奪官治罪,上特宥之,仍在上書房行走。尋復授編修。二十二年,擢通政使。三遷左都御史。三十一年,授禮部尚書。三十二年,復授左都御史。三十三年,以老乞休,上獎其勤慎,加太子少傅,賦詩餞其行。三十九年,卒,年八十六,謚文恪。

秦蕙田,字樹峰,江南金匱人。祖松齡,順治十二年進士,官左春坊左諭德。本生父道然,康熙四十八年進士,官禮部給事中,與貝子允禟善,為其府總管。允禟得罪,逮下獄,蕙田往來省視。世宗貸道然死,而獄未解。乾隆元年一甲三名進士,授編修,南書房行走。乃上疏言:「臣本生父道然身罹重罪,蒙恩曲宥;以追銀未完,系獄九年,年已八十,衰朽不堪。本年五六月間,浸染暑濕,瘧癘時作,奄奄一息,幾至瘐斃。情關骨肉,痛楚難忍。臣雖備官禁近,還顧臣父,老病拘幽,既無完解之期,更無生存之望,方寸昏迷,不能自主。誠不忍昧心竊祿,內慚名教。伏惟皇上矜慎庶獄,一線可原,概予寬釋。當此聖明孝治天下,惟有乞恩,★H0臣父八十垂死之年,得以終老牖下。臣原奪職效奔走以贖父罪。」高宗命宥道然,並免所追銀。

蕙田累遷禮部侍郎,丁本生父憂,服將闋,命仍起禮部侍郎。二十二年,遷工部尚書,署刑部尚書。二十三年,調刑部尚書,仍兼領工部,加太子太保。疏請諸行省流★H0遞籍編甲收管,上諭曰:「蕙田所奏甚是,為清獄訟、弭盜賊之良法。但此輩展轉流徙,城市村落,所在皆有。必一一收捕傳送,令原籍保甲監察,事理繁瑣,不若就所在地察禁。當令有司遇流★H0強悍不法,即時捕治。」二十九年,以病乞休,上不允。再請,上命南還謁醫,不必解任。九月,卒於途,謚文恭。明年,上南巡,幸無錫,賦詩猶及蕙田。

蕙田通經能文章,尤精於三禮,撰五禮通考,首採經史,次及諸家傳說儒先所未能決者,疏通證明,使後儒有所折衷。以樂律附吉禮,以天文歷法、方輿疆理附嘉禮。博大閎遠,條貫賅備。又好治易及音韻、律呂、算數之學,皆有著述。

子泰鈞,乾隆十九年進士,翰林院編修。

彭啟豐,字翰文,江南長洲人。祖定求,康熙十五年,會試、殿試皆第一,官至翰林院侍講。啟豐,雍正五年會試第一,殿試置一甲第三,世宗親拔第一。授翰林院修撰,南書房行走。三遷右庶子。乾隆六年,充江西鄉試副考官,再遷左僉都御史。疏言:「臣驛路經宿州,宿州方被水,蒙恩賑恤。知州許朝棟任甲長胥吏索費,饑民戶籍登記不以實。鳳陽知府梅毓健不親詣察覈。」下兩江總督那蘇圖嚴察。七年,遷通政使,督浙江學政。三遷刑部侍郎,疏言:「浙省吏民占官湖為田,餘杭南湖發源天目,下注苕溪,溉杭、嘉、湖三郡。自巡撫硃軾濬治,今已沙淤。其他會稽、餘姚、慈谿等湖,皆僅存其名,請敕次第開濬。江南漕米,每石收錢五十四,半給運丁,半歸州縣為公使錢。杭、嘉、湖運丁有漕截,而州縣無漕費,石米私加一二升至五六升,請敕如江南例,石米收錢二十四,為州縣修倉鋪墊費,而禁其浮收。浙江額設均平夫銀供差徭,差簡可以敷用,差繁每苦賠墊,本省官吏來往,任意多索,請敕部按官吏尊卑、差役繁簡,定人夫名額,俾為成例。浙省黃巖、太平地多斥鹵,民家稍有餘鹽,兵弁藉以婪索。婪索不遂,指為私鹽,甚或以數家數人之鹽合並誣報,請敕文武大臣申禁。」下部議行。尋以憂去。

十五年,授吏部侍郎。十八年,調兵部侍郎。二十年,疏乞養母,允之。二十六年,復授吏部侍郎。二十七年,以京察注考,吏部郎中阿敏爾圖諸尚書、侍郎皆列一等,啟豐獨列二等,上責其示異市名。旋遷左都御史。二十八年,遷兵部尚書。三十一年,上以史奕昂為侍郎,入對,諭加意部事。奕昂遂自恣,面斥啟豐,不稱尚書,侍郎期成額以是訐奕昂。上詰啟豐,啟豐力言無之。詢侍郎鍾音,鍾音對如期成額。啟豐語乃塞。上為罷奕昂,因謂:「啟豐學問尚優,治事非所長。今乃巽心耎模棱,奏對不以實,失大臣體。」即降侍郎。三十三年,命原品休致。四十一年,上東巡,迎駕,予尚書銜。四十九年,卒,年八十四。

子紹升,語在文苑傳。孫希濂,乾隆四十九年進士,官至刑部右侍郎,左遷福建按察使。曾孫蘊章,自有傳。

夢麟,字文子,西魯特氏,蒙古正白旗人,尚書憲德子。乾隆十年進士,改庶吉士,授檢討。十五年,遷侍講學士,再遷祭酒,提督河南學政。十六年,授內閣學士。十七年,湖北羅田民據天堂寨謀亂,夢麟以河南商城鄰羅田,馳往捕治,上嘉之。疏言:「商城界江、楚,峻嶺深巖,易藏奸宄,請增兵巡察。」下河南巡撫議,移駐守備,增兵百。十八年,署戶部侍郎,充江南鄉試考官,即命提督江蘇學政。二十年,授工部侍郎,代還,調署兵部,兼鑲白旗蒙古副都統。二十一年,命在軍機處學習行走。大臣在軍機處,資望少淺者曰「學習行走」,自夢麟始。

是歲,河決孫家集。二十二年,河道總督白鍾山奏請開荊山橋河,命夢麟馳勘,趣即興工,工竟,議敘。上南巡閱河,以六塘河以下積潦,桃源、宿遷、清河諸縣卑成浸,令夢麟勘治。尋奏:「六塘河上承駱馬湖,至清河分兩派,由武障、義澤等河匯潮河入海,長三百餘里,中間淤淺數十處,已令速疏濬南北兩堰。並去年水壞宿遷堰工,及諸缺口,俱加修築。諸縣積水,開溝十五,設涵洞五,建閘四,俾得宣洩。」工既竟,又奏:「荊山橋河道經銅、沛、邳、睢四州縣,分設四汛;黃水自丁家樓匯入蘇家閘,荊山橋正當其沖,應令堵築。微山湖至荊山橋河下游王母山,紆長灣曲,每歲霜降後應令疏濬。居民就灣築堰壩捕魚,渡口疊石為步,皆阻河道,應令嚴禁。」上命如所議行。

山東巡撫鶴年奏金鄉、魚臺、濟寧諸州縣水患,命侍郎裘曰修偕夢麟馳往相度,合疏言:「諸縣久為微山湖水所浸,當籌分洩之路。韓莊閘南伊家河至江南梁旺城入運,今已久淤,當開濬引積水東注。」從之。兩江總督尹繼善以沂水入運為害,奏建湖口閘,命夢麟與在工諸臣分任其責。合疏言:「沂水自盧口傍洩,淹民田,阻運河。當築壩堵截,使不得入運,毋礙微山諸湖入河歸海之路。六塘河在駱馬湖下游,為沂水疏洩要道,宿遷、桃源諸水自沭入漣歸海,並宜疏治宣通。兼濬六塘河出口,使無淺阻。此治沂水之概要也。夏邑、永城諸水,自睢河下注洪澤湖,出清口會黃入海。近歲河道多淤,董家溝諸地尤宜急治,兼濬洪澤湖出口。清口束水二壩,遵旨撤除。各閘口門亦宜加寬。此治睢河之概要也。」疏入,上許為頗得要領。調戶部。冬,工竟,還京師。二十三年,復調工部,署翰林院掌院學士。卒,賜祭葬。

論曰:照絀於盤錯,而優於詞翰,高宗知之審矣。汝來以清節著,德華等以文學庸,而安國博辨群書,好學深思,自為家法。蕙田治禮,綜歷代政事學術,貫串會通,體大思精,尤彬彬名世之大業也。夢麟早歲負清望,參大政,方駕遽稅,惜哉!


\end{pinyinscope}