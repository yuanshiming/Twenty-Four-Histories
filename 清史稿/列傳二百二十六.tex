\article{列傳二百二十六}

\begin{pinyinscope}
景廉額勒和布許庚身錢應溥廖壽恆

榮慶那桐戴鴻慈

景廉,字秋坪,顏札氏,隸滿洲正黃旗。父彥德,官綏遠城將軍。景廉,咸豐二年進士,由編修五遷至內閣學士。典福建鄉試,擢工部侍郎,賜奠朝鮮。八年,授伊犁參贊大臣。故事,哈薩克貿易訖即行。後以貨滯鬻,許二三人守以度歲,漸成聚落,周二里許。景廉謂禍伏肘腋,毀之便,將軍憚不敢發。會將軍卒,景廉攝任,疏陳利害,請以便宜從事,卒毀之。詔下,如所請。塔爾巴哈臺參贊大臣英秀、阿克蘇辦事大臣綿性、葉爾羌參贊大臣英蘊先後以貪暴被劾,皆命景廉往鞫,得實,降革有差。

十一年,調葉爾羌參贊大臣,其城為南路八城之首,漢、回雜處。安集延常擾邊,俄人復於西南徼往來窺伺,哈薩克各部落多貳於俄。景廉籌餉練兵,持以鎮靜,八城以安。嚴禁綠營兵以重利侵奪回民貲產,人心大悅。同治二年,坐事落職,男婦數千哭於札爾瑪。札爾瑪者,回部棲神之所,意欲禱神阻其行也。景廉既去官,遣往寧夏軍營效力,將軍都興阿檄參戎幕。適安徽巡撫翁同書卒於軍,復檄景廉代領其眾,防剿後路。

五年,授頭等侍衛,充哈密幫辦大臣。募勇千餘,騎不滿百,糧乏,冰雪中殭僕相屬。景廉勉以忠義,夜支單帳,燃馬矢,席地坐,時出撫循,以是兵心固結。肅州賊沿南山西竄,景廉遣總兵張玉春敗之黃花營。賊擾安西州,又大敗之。景廉以安西玉門為新疆門戶,巴里坤雖天險可守,然力單不足恃,疏請駐安西,布置防務輓運,得旨報可。賊撲敦煌,景廉陽令副將蔣富山邀擊南乾溝,而伏勁旅橋灣三水梁。賊果取道三水梁南戈壁,伏起,追擊敗之。捷聞,得旨嘉獎。賊復撲安西,景廉戒守將堅壁毋浪戰,伺其懈擊之,而設伏要其歸路,賊大創,遁。景廉謂敦煌重鎮,當守以重兵,因移鎮敦煌,留兵安西、玉門相犄角。建堅壁清野之計,完城浚壕,擇要區築空心墩臺,守具畢備。復以商團民練輔翼官兵,隱寓保甲之法,賊擄掠之計遂沮。招徠土著三千六百餘戶,勸募雜糧二萬餘石,立轉運局馬蓮井,官民咸稱便焉。

時烏魯木齊回酋妥得璘勾結漢、回、纏頭萬八千餘東犯,潛約哈密回子郡王為內應。王素騃,其母福晉邁哩巴紐賢明有才略,以逆書呈官軍,誓效力守。景廉遣使獎慰,復令富山率兵會辦事大臣文麟、裨將孔才擊賊,連戰六晝夜,大敗之。論功,升擢有差。旋授烏魯木齊都統。時古牧地偽元帥馬明屢詐言降,復假貿易分布逆黨於濟木薩、木壘河。景廉偵知,密檄孔才、金永清等一夕殲之。俄人挾蒙古、哈薩克入境求通商,景廉言地方未靖,不任保護,以兵衛之出。自是終景廉任,俄人不言通商事。

穆宗親政,景廉以為政治在乎始基,上崇正學、開言路、慎牧令、簡軍實、重農桑、弭異端六事。移軍古城,疏請以副都統吉爾洪額、領隊大臣沙克都林札布任軍事。陜回白彥虎糾西寧回萬餘,將奔烏魯木齊,賊勢梟悍,破哈密回城,游騎越天山,擾巴里坤,兩城告急。會妥得璘死,安集延酋帕夏合烏魯木齊、古牧地等漢、回撲沙山子,遙應白彥虎。景廉急檄孔才嚴備濟木薩各要隘,黑龍江營總依勒和布援沙山子,吉爾洪額等援哈密,而景廉坐鎮古城,飲酒習射,若無事然。依勒和布與游擊徐學功率騎五百敗賊沙棗園,擒斬無算。帕夏遁歸吐魯番,遂解沙山子之圍。吉爾洪額等抵巴里坤,連戰皆捷,遂度天山,敗賊哈密泥基頭。城中聞援軍至,大呼突出,賊敗,巴里坤肅清。是役也,論者謂新疆治亂一大關鍵也。白彥虎竄唐朝渠,將入瑪納斯,學功偵得賊口號,選精騎四百,偽為瑪納斯人,迎之龔家瀧,握手慰勞,賊不之疑,益前進,前臨大河。官軍從後起,賊大驚,白彥虎引四十餘騎逸去,餘盡殲焉。學功者,烏魯木齊農家子,沉勇多智略。軍興,集鄉勇自衛。或離合於妥得璘、帕夏之間,為以賊攻賊之計。景廉招之來,推誠待之,遂原效死,至是果得其力。奏請破格錄用,報可。

景廉以憂勤致疾,再乞解職,溫旨慰留。十三年,授欽差大臣,督辦新疆軍務。於是景廉奏請通籌全局,命伊犁將軍金順取道古牧地,提督張曜由天山南取吐魯番,領隊大臣沙克都林札布、錫綸由沙山子取瑪納斯,三路齊舉,使賊不相顧。奇臺、古城為哈密、巴里坤屏蔽,命副都統額爾慶額、孝順、福珠哩駐西湖,防賊逸入北路。烏魯木齊之南俗呼搭板城者,實通吐魯番要路,賊以重兵守之,宜潛師攻擾以搤其吭。並請飭陜甘總督左宗棠總司後路糧臺。移甘肅民千戶實奇臺、古城屯田,購蒙古駝數千隻,借撥部款六十萬兩。疏上,悉蒙嘉納,而忌者尼之,未竟所施。改正白旗漢軍都統。俄召回京,遷左都御史。

光緒二年,命入軍機,兼總理各國大臣。授工部尚書,調戶部。坐事降二級,仍留軍機。補內閣學士,再遷兵部尚書。時言路尚激烈,或不平,景廉曰:「政府如射之有的,言者期其中耳,於我輩何憾?且詆政府者率無罪,未必非大臣之福也。」人服其量。新疆勘定,將軍金順上言景廉前勞,請獎勵。景廉謂邊帥推功樞臣,恐開迎合之漸,請勿許,時論與之。十年,硃諭景廉循分供職,經濟非所長,降二級調用。明年,補內閣學士。八月,卒於官,年六十二。子治麟,國子監司業,見孝友傳。

額勒和布,字筱山,覺爾察氏,滿洲鑲藍旗人。咸豐二年繙譯進士,改庶吉士,用戶部主事。累遷理籓院侍郎。同治三年,熱河土默特貝勒旗老頭滋事,額勒和布奉命查辦得實,請將貝勒議處,其佐領、章京等降革有差,事遂定。由蒙古副都統調補滿洲。旋授盛京戶部侍郎,兼奉天府府尹。直隸總督劉長佑率師防剿熱河及奉天馬賊,額勒和布籌給軍食。賊酋周榮糾黨回竄,擾及昌圖,所在告警。額勒和布遣將率馬隊迎擊開原,而以步隊扼其後,賊遂潰散。六年,請酌抽鹽釐充練兵經費,增設海防同知駐營口,均議行。於賑務尤盡力捐募。署盛京將軍,調察哈爾都統。新疆用兵,額勒和布經紀糧運,並調八旗官兵助剿,擢烏里雅蘇臺將軍,屢卻悍賊。

光緒三年,因病乞休。六年,起鑲白旗漢軍都統,調蒙古。歷熱河都統、理籓院尚書、戶部尚書、內務府大臣。十年,命直軍機,協辦大學士。奏請允開滇、越邊界礦務,又奏光緒四年以前直省錢漕積欠者,請予蠲免。司業潘衍桐建言特開藝學科,以額勒和布持不可,寢其議。十一年,授體仁閣大學士,轉武英殿。歷充閱卷大臣等差。二十年,免直軍機。二十二年,致仕。逾四年,卒於家,謚文恭。

額勒和布木訥寡言,時同列漸攬權納賄,獨廉潔自守,時頗稱之。

許庚身,字星叔,浙江仁和人。咸豐初,由舉人考取內閣中書。嘗代同官夜直,一夕,票二百簽,署名牘背。文宗閱本,心識之,以詢侍郎許乃普,乃普為其諸父行也,遂命充軍機章京。故事,大臣子弟不得入直,是命蓋異數云。十年,車駕獮木蘭,召赴行在。是時肅順方怙權勢,數侵軍機事,高坐直廬,有所撰擬,輒趣章京往屬草。庚身以非制,不許,使者十數至,卒弗應。肅順慚且懟,欲中以危法,未得間。穆宗纘業,特賜金以旌其風節,命隨大臣入直。

同治元年,成進士,自請就本官,補侍讀。累遷鴻臚寺少卿。母憂歸,服竟,遷內閣侍讀學士,入直如故。進春秋屬辭,被嘉獎。補光祿寺卿。典試貴州,督江西學政,頗以天算、輿地諸學試士。光緒四年,授太常寺卿。擢禮部侍郎,調戶部、刑部。十年,法越事起,充軍機大臣,兼總理各國事務,晉頭品服。時樞府孫毓汶最被眷遇,庚身以應對敏練,太后亦信仗之。十四年,晉兵部尚書。十九年,卒,謚恭慎。

庚身自郎曹至尚侍,直樞垣垂三十年,與兵事相終始,為最久云。

錢應溥,字子密,浙江嘉興人。拔貢生,朝考一等,用七品小京官,分吏部,直軍機。咸豐十年,粵寇連陷浙東西郡縣,應溥父海寧州學訓導泰吉,質行樸學,老儒也,時已罷官,州人留主講書院。應溥聞警,亟請歸奉親,轉徙經年,須發為白。

曾國籓治兵安慶,招入幕,工為文檄,敏捷如夙構。國籓屢欲特薦,皆力辭。同治三年,奏加五品卿銜。大軍征捻,駐周家口。捻宵至,守卒僅千人,眾駭懼,應溥鎮靜若無事然。於是國籓堅臥不起,捻卒不敢犯。晉四品卿銜,國籓深倚重之,其督兩江,有大興革,上奏辭皆囑應溥具草。

光緒初,養親事畢,乃入都,重直軍機,擢員外郎。恭忠親王、醇賢親王相繼秉政,皆嘉其諳練。每承旨繕詔,頃刻千言,曲當上意。累遷禮部侍郎。偕尚書昆岡按事河南,自巡撫裕寬以下降黜有差。朝鮮事起,廷議主戰,應溥造膝敷陳,多人所不敢言。旋任軍機大臣,再遷工部尚書。謝病歸。二十八年,卒,謚恭勤。子駿祥,翰林院侍讀。

廖壽恆,字仲山,江蘇嘉定人。同治二年進士,授編修。出督湖南學政。光緒二年,再擢侍講。近畿旱災,壽恆應詔陳言,以為:「吏治壞則民情鬱,以其愁苦之氣薄陰陽之和而災祲生,應天以實不以文。原皇上審敬怠,明是非,覈功罪,信賞罰,勿徒視為具文。」語甚切至。尋以內務府開支失實,請嚴飭,以為浮濫者戒。再督河南學政,累遷內閣學士,仍留視學。坐疏察生員欠考,下部議處。

九年,法人侵據越南安定,壽恆疏言:「法以傳教為事,今乃思闢商務,取徑越南。越固我籓屬,萬無棄而不顧之理。臣愚以謂今日有必戰之勢,而後有可和之局。李鴻章威望最隆,北洋勁旅,非他人所能統御。宜飭鴻章仍回北洋大臣本任,坐鎮天津,以衛畿輔,而飭署督張樹聲還督兩廣。樹聲忠勇宿將,必能相機進討,以伸保護屬國之義。兩督臣各還本任,事屬尋常,可不啟外人之疑;而進戰退守,能發能收。彼若悔禍,自可轉圜。若必並吞越南,則是兵端自彼而開,不得謂為不修鄰好。」

法越和議成,壽恆復上疏言:「風聞法使至天津,稱越南既議款,因以分界撤兵事要約李鴻章,鴻章拒不允,擬即來都磋商譯署。論者謂當虛與委蛇。不知法據越南,去我之屬國;逐黑旗,撤我之籓籬;通紅江,奪我滇江之大利。先機已失,不可不圖挽回。為今之計,直宜以欺陵小弱之罪,布告列邦,折以公法,令改削所立條約。河內、安定,一律讓還,然後緩議法越通商之約。現聞津海防務,已飭備嚴整,軍容改觀。臣謂仍當選派知兵大員,率兵輪駛赴越都,以觀動靜。又飛檄廣西防軍援助劉永福,增兵制械,迅拔河內,以扼敵沖。河內既下,北圻乃安。蓋我不與法構兵,永福不能不為越守土,故邇來陰助黑旗,屢戰皆捷。法人不得已,乃託言保護。永福忿懣填胸,茍奉詔書,無不一以當百。如此,則滇、粵之邊患稍紓,越、法之兵端可戢。」壽恆又以:「根本之計,責在宸躬。跬步不離正人,乃可薰陶德性。擬請皇太后、皇上,御前太監務取厚重樸實之人,其有年紀太輕、性情浮動者,屏勿使近。並請懿旨時加訓飭,凡一切淺俗委瑣之言,勿許達於宸聽。庶幾深宮居息,無往非崇德之端,或可補毓慶宮課程所不及。至於宮廷土木之工,內府傳辦之件,事屬尋常,最易導引侈念。伏原皇太后崇儉黜奢,時以民生為念,俾皇上知稼穡之艱難,目染耳濡,聖功自懋。如是,則慈闈教育,更勝於典樂命夔。」疏入,上為之動容。

十年,行走總理衙門。遷兵部侍郎,調禮部、戶部、吏部侍郎,屢典試事。偕都御史裕德查辦四川鹽務,劾罷鹽茶道蔡逢年,遣戍。二十三年,遷左都御史,入軍機。明年,調禮部尚書。太后訓政,命出軍機。以疾乞休。二十九年,卒。

榮慶,字華卿,鄂卓爾氏,蒙古正黃旗人。光緒九年,會試中式。十二年,成進士,以編修充鑲藍旗管學官。累遷至侍讀學士、蒙古學士。遷轉遲滯,榮慶當引見,或諷以乞假,謝曰:「窮達命也,欺君可乎?」居三年,擢鴻臚卿,轉通政副使。簡山東學政,丁母憂。二十七年,擢大理卿,署倉場侍郎。以剝船盜米,改由火車逕運,並倉廒,增經費,杜領米弊端,裁稽查倉務御史,皆如所請行。和議成,奉命會辦善後事宜,兼政務處提調。二十八年,授刑部尚書。大學堂之創立也,命榮慶副張百熙為管學大臣。百熙一意更新,榮慶時以舊學調濟之。尋充會試副考官、經濟特科閱卷大臣。調禮部尚書,復調戶部。拜軍機大臣、政務大臣。

榮慶既入政地,尤汲汲於厲人才,厚風俗。嘗疏陳:「國家取才,滿、漢並重。請飭下閣部,將所屬滿員嚴加考試,設館課之:一、掌故之學,二、吏治之學,三、時務之學。尤以禦制勸善要言、人臣儆心錄、性理精義、上諭八旗諸書,為居官立身之大本。均令分門學習,劄記大綱,以覘其才識。」疏入,報聞。

三十一年,協辦大學士。是冬,改學部尚書。明年,充修訂官制大臣。尋罷軍機,專理部務。德宗上賓,充恭辦喪禮大臣。宣統元年,以疾乞休,溫旨慰留。調禮部尚書。孝欽後奉安,充隨入地宮大臣,恭點神牌,晉太子少保。三年,裁禮部,改為弼德院副院長。旋充顧問大臣、德宗實錄館總裁。國變後,避居天津。卒,年五十八,謚文恪。

榮慶持躬謹慎。故事,軍機大臣無公費,率取給餽贐。榮慶始入直,深以為病,語同列合辭上請,乃得支養廉銀二千,而御前諸臣亦援例增給有差。

那桐,字琴軒,葉赫那拉氏,內務府滿洲鑲黃旗人。光緒十一年舉人,由戶部主事歷保四品京堂,授鴻臚寺卿,遷內閣學士。二十六年,兼直總理各國事務衙門,晉理籓院侍郎。

拳匪肇釁,各國聯兵來犯,令赴豐臺御之。外兵入京,誤以東壩為匪窟,欲屠之,力解乃免。兩宮西巡,命充留京辦事大臣,隨李鴻章議和。約成,專使日本謝罪,又派赴日觀博覽會。二十九年,擢戶部尚書,調外務部,兼步軍統領,管工巡局事,創警務,繕路政。平反王維勤冤獄,商民頌之。三十一年,晉大學士,仍充外務部會辦大臣。歷兼釐訂官制、參預政務、變通旗制,署民政部尚書。

宣統元年,命為軍機大臣。丁母憂,請終制,不許。出署直隸總督,請撥部款修鳳河。尋還直。三年,改官制,授內閣協理大臣,旋辭,充弼德院顧問大臣。國變後,久臥病。卒,年六十有九。

戴鴻慈,字少懷,廣東南海人。光緒二年進士,改庶吉士,以編修督學山東。父憂歸,服除,督學云南。後復充雲南鄉試正考官。二十年,大考一等,擢庶子。日韓啟釁,我軍屢挫。鴻慈連疏劾李鴻章調遣乖方,遷延貽誤,始終倚任丁汝昌,請予嚴懲;並責令速解汝昌到部治罪,以肅軍紀:均不報。和議成,鴻慈奏善後十二策:一,審敵情以固邦交;二,增陪都以資拱衛;三,設軍屯以實邊儲;四,築鐵道以省漕運;五,開煤鐵以收利權;六,稅煙酒以佐度支;七,行抽練以簡軍實;八,廣鑄造以精器械;九,簡使才以備折沖;十,重牧令以資治理;十一,召對群僚以勵交修;十二,變通考試以求實用。遷侍講學士。督學福建,再遷內閣學士。學政報滿,假歸省墓。擢刑部侍郎。

赴西安行在,上陳治本疏;又請建兩都,分六鎮,以總督兼經略大臣,得闢幕僚,巡撫以下咸受節制。是年冬,隨扈還京,轉戶部侍郎。時各省教案滋多,鴻慈請設宣諭化導使,以學政兼充。編輯外交成案,頒發宣講。又請就翰林院創立報局,各省遵設官報,議格不行。時設會議政務處,有奉旨交議事件,三品京堂以上與議。鴻慈請推行閣部、九卿、翰林、科道皆得各抒所見,屬官則呈堂代遞,可以收群策、勵人才。下政務處採擇。

三十一年,命五大臣出使各國考求政治,鴻慈與焉。將發,黨人挾炸藥登車狙擊,從者或被創,人情惶懼。鴻慈從容詣宮門取進止,兩宮慰諭,至泣下,遂行。歷十五邦,凡八閱月,歸國。與載澤、端方、尚其亨、李盛鐸等裒輯列國政要百三十三卷、歐美政治要義十八章,會同進呈。並奏言:「各國治理大略,以為觀其政體:美為合眾,而專重民權;德本聯邦,而實為君主;奧、匈同盟,仍各用其制度;法、義同族,不免偏於集權;唯英人循秩序而不好激進,其憲法出於自然之發達,行之百年而無弊。反乎此者,有憲法不聯合之國,如瑞典、挪威則分離矣;有憲法不完全之國,如土耳其、埃及則衰弱矣;有憲法不平允之國,如俄羅斯則擾亂無已時矣。種因既殊,結果亦異。故有雖革改而適以召亂者,此政體之不同也。覘其國力,陸軍之強莫如德,海軍之強莫如英,國民之富莫如美,此國力之不同也。窺其政略,則俄、法同盟,英、日同盟,德、奧、義同盟,既互相倚助以求國勢之穩固;德、法摩洛哥之會議,英、俄東亞之協商,其對於中國者,德、美海軍之擴張,美、法屯軍之增額,又各審利害以為商業之競爭。蓋列強對峙之中,無有一國孤立可以圖存者,勢使然也。況人民生殖日繁,智識日開,內力亦愈以澎漲。故各國政策,或因殖民而造西伯利亞之鐵路,或因商務而開巴拿馬之運河,或因國富而投資本於世界,均有深意存焉。此政略之不同也。驗其民氣,俄民志偉大而少秩序,其國失之無教;法民好美術而流晏逸,其國失之過奢;德民性倔強而尚武勇,其國失之太驕;美民喜自由而多放任,其國失之衣復雜;義民尚功利而近貪詐,其國失之困貧;惟英人富於自治自營之精神,有獨立不羈之氣象,人格之高,風俗之厚,為各國所不及。此民氣之不同也。臣等觀於各國之大勢既如此,又參綜比較,窮其得失之源,實不外君臣一心,上下相維,然後可收舉國一致之益。否則,名實相懸,有可以斷其無效者,約有三端:一曰,無開誠之心者國必危。西班牙苛待殖民,致有斐律賓、古巴之敗。英鑒於美民反抗,而於澳洲、坎拿大兩域予人民以自治之權,致有今日之強盛,開誠故也。俄滅波蘭而用嚴法以禁其語言,今揭竿而起要求權利者,即波蘭人也。又於興學練兵,皆以專制為目的,今滿洲之役,不戰先潰。莫斯科、聖彼得堡之暴動,即出於軍人與學生也。防之愈密,而禍即伏於所防之中,患更發於所防之外,不開誠故也。二曰,無慮遠之識者國必弱。俄以交通之不便,而用中央集權,故其地方之自治,日以不整。美以疆域之大,而用地方分權,故其中央與地方之機關,同時進步。治大國與治小國固不侔也。德以日爾曼法系趨於地方分權,雖為君主之國,而人民有參與政治之資格。法以羅馬法系趨於中央集權,雖為民主之國,而政務操之官吏之手,人民反無自治之能力。兩相比較,法弱於德,有由來矣。三曰,無同化之力者國必擾。美以共和政體,重視人民權利,雖人種衣復雜,而同化力甚強,故能上下相安於無事。土耳其一國之中,分十數種族,語言宗教各不相同,又無統一之機關,致有今日之衰弱。俄則種族尤雜,不下百數,語言亦分四十餘種,其政府又多歧視之意見,致有今日之紛亂。奧、匈兩國雖同戴一君主,而兩族之容貌、習尚、語言、性情迥殊,故時起事端,將來恐不免分離之患。蓋法制不一,畛域不化,顯然標其名為兩種族之國,未有能享和平、臻富強者矣。此考察各國所得之實在情形也。竊惟學問以相摩而益善,國勢以相競而益強。中國地處亞東,又為數千年文化之古國,不免挾尊己卑人之見,未嘗取世界列國之變遷而比較之。甲午以前,南北洋海陸軍制造各廠同時而興,聲勢一振。例之各省,差占優勝矣。然未嘗取列國之情狀而比較之也。故比較對於內,則滿盈自阻之心日長;比較對於外,則爭存進取之志益堅。然則謀國者亦善用其比較而已。」

又奏:「臣等曠觀世界大勢,深察中國近情,非定國是,無以安大計。國是之要,約有六事:一曰舉國臣民立於同等法制之下,以破除一切畛域;二曰國是採決於公論;三曰集中外之所長,以謀國家與人民之安全發達;四曰明宮府之體制;五曰定中央與地方之權限;六曰公布國用及諸政務。以上六事,擬請明降諭旨,宣示天下以定國是,約於十五年或二十年頒布憲法,召集國會,實行一切立憲制度。」又奏:「實行立憲,既請明定期限,則此十數年間,茍不先籌預備,轉瞬屆期,必至茫無所措。今欲廓清積弊,明定責成,必先從官制入手。擬請參酌中外,統籌大局,改定全國官制,為立憲之預備。」均奉俞旨採納,遂定立憲之議。

先是鴻慈奉使在途,已擢禮部尚書;及還,充釐定官制大臣,轉法部尚書。充經筵講官、參預政務大臣。時法部初設,與大理院畫分權責,往復爭議,又改並部中職掌。於是京外各級審判次第設矣。又採英、美制創立京師模範監獄。三十四年,疾作,乞解職,溫旨慰留。兩宮升遐,力疾視事。

宣統元年,賞一等第三寶星,充報聘俄國專使大臣。禮成返國,奏言:「道經東三省,目擊日、俄二國之經營殖民地不遺餘力。非急籌抵制,無以固邊圉;非振興實業擴其自然之利,無以圖富強。請速辦墾殖、森林二端。俟財力稍裕,再籌興學、路礦、兵屯各事,以資捍衛。」臚陳辦法。得旨,下所司議行。是年八月,命入軍機,晉協辦大學士。二年,卒,加太子少保,謚文誠。

論曰:樞臣入對,序次有定,後列者非特詢不得越言。晚近領以尊親,勢尤禁隔,旅進旅退而已。景廉多戰績,額勒和布有清操,庚身、應溥通達諸諳練,壽恆有責難之言,鴻慈負知新之譽,榮慶謹慎持躬,那桐和敏解事,皆庶幾大臣之選者歟?


\end{pinyinscope}