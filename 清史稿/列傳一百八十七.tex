\article{列傳一百八十七}

\begin{pinyinscope}
何桂珍徐豐玉張汝瀛金雲門唐樹義嶽興阿易容之

溫紹原金光箸李孟群趙景賢

何桂珍,字丹畦,雲南師宗人。道光十八年進士,選庶吉士,年甫冠,乞假歸娶。散館授編修,督貴州學政。入直上書房,授孚郡王讀。文宗在潛邸,即受知。桂珍鄉試出倭仁門,與唐鑒、曾國籓為師友,學以宋儒為宗。及文宗即位,以所撰大學衍義芻言奏進,優詔嘉納。數上疏論時政得失,言琦善、牛鑒僨軍之將,不宜任兵事。咸豐三年,出為福建興泉永道。巡防大臣賈楨等奏請開缺,留京隨辦城守事宜。

四年,畿輔解嚴,授安徽徽寧池太廣道。安慶久陷,巡撫福濟駐廬州之店埠。桂珍所治在江南,阻於賊,遂留江北。檄募勇從征,餉無所出,久之,得二百人,至霍山,號召鄉團,增為三千人,激以忠義,破捻匪李兆受於霍城,追擊至麻埠,進逼流波甿;檄商城、固始鄉團截其北,金寨練勇拒其東,自率所部遏其西,兆受大懼,與其黨馬超江等同降,解散脅從萬計,民歡呼載道,饋糗糧不絕。福濟令桂珍援廬江,檄至,城已陷,馳救不及,劾罷職。是年,曾國籓破賊田家鎮,進圍九江,桂珍通牒言戰狀,國籓以聞。袁甲三軍臨淮,欲資桂珍兵西與楚師會,至蘄水而九江軍失利,武昌再陷。國籓入江西,文報不相聞。桂珍乃提孤軍轉戰潛、霍間。五年春,克蘄水、英山,殲賊首田金爵。和春上其功,予六品頂戴,留駐英山。自桂珍受事,至是八閱月,僅支餉銀三百兩。民團相從者踵至,益以李兆受降眾,餓不得食,五月,師遂潰。

兆受之降也,桂珍請福濟羈以官,不聽,不能無觖望。未幾,馬超江被殺,兆受乞拘仇,弗獲,則大恚,議為超江復仇,設位受吊,捻黨大集。於是安徽、河南皆以兆受復叛入告,兆受詣桂珍自陳,撫慰之,稍定。會福濟密書囑先發絕其患,書由驛遞,為兆受所得,謂桂珍賣己。十月,陽置酒,伏兵英山小南門外,桂珍遂遇害,左右四十餘人皆從死。事聞,依道員陣亡例賜恤,贈光祿寺卿,予雲騎尉世職。同治初,江南平,曾國籓疏言桂珍率鄉團剿賊,饑餓艱難,歷人間未有之苦,機事不密,為叛人所戕,天下冤之。詔晉世職為騎都尉,予謚文貞,建祠英山縣。

徐豐玉,字石民,安徽桐城人。父鏞,嘉慶十四年進士,官至太僕寺卿。豐玉少應科舉不遇,捐納銓授貴州平遠知州。署威寧,捕斬大盜,總督林則徐嘉異之,調黃平。苗寨盜魁保禾日聚眾剽掠。豐玉清保甲,理屯軍,請兵會剿。巡撫喬用遷慮激變,不許。既而苗益恣,從知府胡林翼往剿,保禾遁。時廣西匪起,蔓及貴州境。豐玉練民兵,入山搜捕,多得盜魁,誅之。雲南巡撫張亮基過黃平,悉其狀,密疏薦。遷郎岱同知,署思州府。

咸豐二年,擢湖北黃州知府。甫蒞任,而張亮基調湖南,奏調豐玉往襄軍事,助守長沙。尋從總督徐廣縉赴岳州,武昌已陷,豐玉請廣縉速移鎮黃州,截賊下竄。廣縉不能用,得罪去,張亮基代之。三年,擢湖北督糧道,署漢黃德道。廣濟民變,戕縣令。黃州知府邵綸及新令鮑開運往撫,均遇害。豐玉偕按察使江忠源往剿,捕斬數百人,乃定。

會粵匪由江寧分竄上游,忠源率師援江西,亮基令豐玉統湖北防軍駐田家鎮。鎮當江北岸,後有大山曰黃金塔,小山曰磨盤,下有河直入江中,與南岸半壁山接。山塹水湍,舟行必循湍繞河乃得過,最據形勢。豐玉列營諸山,於河上聯筏作城,列砲以守。半壁山背倚湖,湖通興國,入湖處曰富池口。豐玉欲分營半壁山上而兵單,僅遣兵弁了望而已。九月,賊由南昌退九江,遂上犯田家鎮。豐玉偕總兵楊昌泗憑墻砲擊沉賊船,又斃陸路撲營賊,乘勝追壓乃退。次日,賊船擁至,分三路迎擊,斃賊甚眾,毀其大船。賊由富池口分船數百犯興國,會江忠源由江西回援,賊復由興國會於富池口。荊門知州李榞輕軍襲之,豐玉遣兵夾擊,敗挫,榞陣歿。忠源聞田家鎮危急,調九江兵馳援,未達,忠源獨挈親兵數十人至。見賊眾兵單,驚曰:「不可守矣!」次晨,大風作,賊連檣驟至,環撲我營。豐玉偕漢黃德道張汝瀛督戰,筏城被焚,營壘皆不守。豐玉手佩刀殺賊,遂自剄,汝瀛同殉焉。忠源親隨僅存數人,收集餘眾,退駐廣濟。事聞,予騎都尉世職。光緒中,大學士李鴻章疏陳豐玉政績、死事狀,予謚勇烈,建專祠。

張汝瀛,山東樂陵人。道光元年舉人。官廣西知縣,歷貴縣、蒼梧,以剿匪功洊升知府,亦為張亮基所薦拔。咸豐三年,擢漢黃德道。甫抵任,偕豐玉同守田家鎮,歿於陣,予騎都尉世職,追謚勇節。

金雲門,安徽休寧人。道光十三年進士,官浙江雲和知縣。改湖北,歷天門、崇陽、隨州。以擒崇陽匪首鍾人傑功,晉知州。洊擢安陸知府,署糧儲道,護按察使,調署黃州。自田家鎮失利,賊遂進陷黃州,雲門死之,贈太僕寺卿,予騎都尉世職。後京山士民以政績卓越請建祠,謚果毅。

唐樹義,貴州遵義人。嘉慶二十一年舉人,官湖北咸豐、監利、江夏知縣,洊擢湖北布政使。以病歸,在籍辦團練。張亮基奏調湖北,署按察使。及田家鎮軍事亟,率兵防江北陵路,駐廣濟。既而黃州、漢陽相繼陷,樹義剿賊德安,進軍灄口。咸豐四年,戰失利,褫職留任,率舟師禦賊金口,船破,死之。予騎都尉世職,謚威恪。

岳興阿,博爾濟吉特氏,滿洲正藍旗人。考授內閣中書,出為河南南陽知府,洊擢湖北布政使。四年,武昌陷,死之。予騎都尉世職,謚剛節。

易容之,廣東鶴山人。捐納銓授湖北德安知府。四年,德安陷,罵賊死之,予騎都尉世職。李榞自有傳。

溫紹原,字北屏,湖北江夏人。少負奇略。入貲為兩淮鹽運司經歷,改知縣。咸豐二年,署六合,減賦役,蠲苛法,民戴之。

粵匪陷武昌東下,紹原以六合為南北要沖,勸民積穀儲群堡,修城垣,治守具。團練四鄉,合為一氣,別募壯勇訓練。三年春,江寧陷,賊游騎至境,輒殲之。既而大至,御於龍池,以兵單失利,練總徐琳、達成榮戰死,紹原退保南關。會日暮,賊營火,乘亂攻之,斬偽丞相一、偽統制四,餘眾殲焉。紹原益增守要隘,浚品字坑伏地雷。守備秦淮陽,千總夏定邦、王家幹,皆能戰,賊屢至,隨機御之,每擒斬過當,賊懼之,不敢逼。欽差大臣向榮、總督怡良先後上其功,以知府升用,賜花翎,特詔嘉獎;並以紳民深明大義,蠲免六合一年丁漕,增廣學額,以示旌異。

四年,賊屯九洑洲,結簰置砲,翼以戰艦,順流下,至八卦洲,紹原夜以小舟襲之,縱火焚簰幾盡,偕總兵武慶、江浦知縣曾勉禮,分路進攻九洑洲。天大霧,架浮橋襲賊營,大破之,平其壘,被議敘。

五年,署江寧知府,在縣設治,督辦府屬團練事宜。賊屢糾悍黨自浦口來撲,皆不得逞。六年,大軍攻鎮江、瓜洲急,賊數路來援。其自蕪湖來者,紹原要之於江,七戰皆捷,進劃南岸七里洲賊壘,毀其舟。賊乃出陸路,竄踞高資港、下蜀街,巡撫吉爾杭阿檄紹原赴援。紹原令其弟溫綸率千人往戰,數有功。江北托明阿軍潰,揚州陷。紹原由儀徵往援,而賊陷江浦,犯浦口,踞六合葛塘集,偕張國樑馳擊於龍池,大破之;又破之於盤城集,連復江浦、浦口。捷聞,擢道員。未幾,賊再陷江浦,進犯六合,紹原合水陸擊走之。

時軍事分隸江南、江北兩大營。六合地居江北,紹原以孤城為保障,且數出境渡江助大軍攻剿立功,向榮深推重,令充南軍翼長。德興阿督北軍,意嗛之。七年,天長、來安土匪起,遣兵破之。列上所部戰績,德興阿謂越境邀功,置勿錄,紹原力爭,遂以乾頂保舉疏劾褫職,仍留六合帶勇防堵。尋有旨命兼管江寧、江浦團練。總督何桂清疏言:「紹原以一縣倡募水陸各勇,激勵紳團,屢殲賊眾,出奇制勝。且餘力上搤江浦,下救儀徵,北援來安,江北大營得免西顧之憂。自來安至廬州,尚有一線運道可通者,亦惟紹原是賴。才足匡時如紹原者,實不多見。請復原官,以維系眾心。」詔允開復知府。八年,從大軍克來安,加鹽運使銜。

悍酋李秀成、陳玉成大舉援江寧,先陷江浦。德興阿退六合,三戰皆敗,又退揚州。賊久憾紹原,合力圍攻。文宗恐其有失,詔促德興阿、勝保速援,皆不至。紹原堅守幾及一月,力竭城陷,死之。張國樑既克揚州,即日馳赴,於城陷次日始至,聞者莫不嗟悼。詔嘉紹原「六載守城,久為江北重鎮。援師未集,力竭捐軀,深為憫惜」,贈布政使銜,予騎都尉世職,於六合建專祠,謚壯勇。

夏定邦,六合人;王家幹,睢寧人。從紹原守御,及八卦洲、九洑洲、江浦諸戰,皆有殊績,並擢官都司。城陷,同死難。

金光箸,字濂石,直隸天津人。捐納通判,分甘肅,署巴燕戎格,改安徽知縣。青陽民因歲荒抗徵,幾釀變,光箸奉檄單騎諭解之。補建平,調定遠。定遠多盜,巡緝無間,捕土匪陳小喚子置之法。又調壽州。

咸豐三年春,粵匪連陷安慶、江寧,皖北盜蜂起,光箸集民團備戰守。陸遐齡者,定遠巨猾,系安慶獄。城陷,賊令歸結黨為北路應,擾定遠、壽州、合肥,勢甚張。巡撫周天爵兵少不能制,令光箸圖之。先布間諜,散其黨羽,然後進攻莊木橋。光箸設奇計,親率勇士擒遐齡父子及其黨四十餘人,戮之。天爵特疏薦,晉秩知府,賜花翎。

四月,賊由江寧、揚州分股北竄臨淮,擾及鳳陽、懷遠。光箸於兩河口立水營,八公山雜張旗幟為疑兵,列砲要隘。獲賊諜逃兵,並斬之以徇,壽州獲安。五月,賊復由六合撲正陽關,光箸調練勇千,屯三十里鋪及兩河口迎擊,殲賊二百餘人,乃引去。招降附近土匪談家寶、張茂等黨眾數千,皆效用。是年冬,粵匪陷廬州。四年,六安繼陷,北路捻匪日猖獗。和春督大軍規廬州,不暇北顧。袁甲三剿捻,徬徨於皖、豫之交。正陽為要沖,距州城六十里。光箸扼關以御,捻黨數來犯,五戰皆捷。季學盛踞於家圍,而馬四、馬五、王亮彩、鄧三虎等諸捻黨出沒州境,先後平之。廬州大軍無後顧憂者,光箸之力也。

五年,大軍克廬州,光箸署知府,撫流亡,嚴斥候,數殲伏匪。六年,遂實授。尋巡撫福濟疏列其治行上聞,以道員記名,署廬鳳道。時和春移督江南大軍,袁甲三再起軍臨淮,捻勢南趨。光箸甫出兵,捻首張洛行已破周鎮、王莊,犯三十里鋪。光箸背水為陣,令曰:「有進無退!」分三路擊之,以八百人破賊數萬。七年春,捻匪龔德等掠正陽關,光箸偕副都統德勒格爾渡河襲擊,斃賊八百餘,追七十里。將搗其巢,聞六安復為粵匪所陷,回保壽州。粵匪驟至,圍城。破其地雷,夜乘霧出城,分三路襲賊營,鄉團應之。賊驚潰,追擊,斃賊千餘,圍立解。捷入,加按察使銜。乘勝合水陸進剿,毀賊營四十餘處,克正陽關,賜號鏗色巴圖魯。閏五月,捻匪復踞正陽關,欽差大臣勝保率兵至八里垛,光箸請夾擊於沫河口,建浮橋先渡馬隊。賊忽由後路鈔來,光箸立船頭督戰,左腿中槍,猶指揮進擊,纜斷溜急,舟覆,沒於河。詔贈布政使銜,依贈官賜恤,予騎都尉世職,謚剛愍,於壽州建專祠。

光箸吏治戰績為安徽第一。嘗言:「大兵宜攻不宜守。郡縣吏宜守四境,不宜守孤城。」皖北倚為保障。及其歿後,捻氛乃益熾,人尤思之云。

李孟群,字鶴人,河南光州人。父卿穀,道光二年舉人,四川長寧知縣,累擢湖北督糧道,署按察使。咸豐四年,粵匪陷武昌,巡撫青麟走湖南,卿穀守城殉難,贈布政使,予騎都尉世職,謚愍肅。

孟群,道光二十七年進士,廣西即用知縣。歷署靈川、桂平,以剿匪功擢南寧同知。咸豐元年,匪首洪秀全犯盤龍河,孟群手執藤牌督戰殺賊,鏖戰連日,賊不得渡。擢知府,調赴永安軍營。二年,授泗城知府。賊犯桂林,孟群赴援,連戰北門外及古牛山、五里墟、夾山口、睦鄰村,迭挫賊鋒。圍解,加道銜。進平潯州艇匪,擢道員,署潯州知府。三年,調江西九江府,仍留廣西剿賊。

四年,曾國籓在籍治水師,聞孟群名,奏調率千人往偕楊載福、彭玉麟東下,攻拔城陵磯,克岳州,調廣西平樂府。賊陷武昌,孟群聞父殉難,誓滅賊復仇,仍請終制,詔留軍。國籓屯金口,塔奇布進扼洪山,定三路攻武昌之策。孟群偕載福、玉麟中流直下,艦分二隊,前隊沖鹽關出賊背,後隊自上擊下,毀賊船二百餘艘。會諸軍剷沿江木柵,破漢關及金沙洲、白沙洲,抵占魚套,西渡攻漢陽朝宗門。賊揚帆下竄,尸蔽江。毀晴川閣下木柵、大別山下木壘,武昌、漢陽同日收復。孟群奔赴父死所慟哭收殮,一軍感動。捷聞,加按察使銜,賜號珠爾杭阿巴圖魯。

於是國籓進規江西,孟群率水師抵九江,戰兩岸及湖口皆捷。五年春,師挫於湖口,賊溯江上犯,陷漢陽,武昌大震。孟群回援,偕彭玉麟敗賊漢陽。署湖北按察使,以在憂辭,詔不允。武昌尋為賊陷,從胡林翼屯金口,改統陸師。五月,合擊賊,四戰皆捷。七月,賊糾黨撲金口,孟群拒戰失利,陸營潰。詔以眾寡不敵原之,命攻漢陽。六年,從總督官文迭進攻,十一月,孟群據龜山俯擊,總兵王國才攻西南各門,城中賊亂,遂克漢陽,加布政使銜,以布政使遇缺題奏。

七年,安徽北路捻匪方熾,粵匪自桐城進陷六安、英山、霍山,廬州危急。巡撫福濟請援,孟群率陸師二千五百人赴之,途次授安徽布政使。進兵克英山、霍山,攻獨山,駐軍麻埠。霍山復為賊陷,尋復之。八年,粵匪由潛山、太湖竄擾河南固始。孟群自六安赴援,偕勝保力戰解圍,被獎敘。剿商城匪黨,平之,回軍克六安。七月,福濟卒於軍,暫攝巡撫,未十日,廬州為粵匪所陷,褫職,留軍效力。收集潰軍,駐廬州西官亭、長城一帶。

皖北赤地千里,協餉不至,所部號四千,饑疲已甚。湘軍李續賓方克桐城、舒城,飛書乞援,而續賓戰歿於三河,勢益孤危。九年二月,六安復陷,賊六七萬逼長城,營壘被圍,死守十餘日。壘破,手刃數賊,受傷被執,擁至廬州,賊首陳玉成優禮之,絕糧不食,賦詩四章書於絹,付人使出報大營,遂死之。

勝保等先已疏陳孟群殺賊陣亡,詔復原官,賜恤,謚武愍。十年,巡撫翁同書以尋獲遺骸入奏,命送回籍。袁甲三復奏孟群死事實跡,詔於廬州建專祠,依巡撫例優恤,予騎都尉兼雲騎尉世職。穆宗即位,以孟群父子殉節,忠烈萃於一門,與賜祭死事諸臣之列焉。

趙景賢,字竹生,浙江歸安人。父炳言,嘉慶二十二年進士,授刑部主事,歷官湖南巡撫。

景賢,道光二十四年舉人,誤註烏程籍,被黜。捐復,授宣平教諭,改內閣中書。豪邁有大略。咸豐三年,在籍倡團練,以勸捐鉅款,晉秩知府,分發甘肅,未往。十年,尚書許乃普薦之,命從團練大臣邵燦治事。聞粵匪陷廣德,自蘇州馳歸,籌布守城。總兵李定太、參將周天孚先後來援失利。景賢收集潰兵,為戰守計。偵知江南援軍至,出城夾擊,擒斬數千,立解城圍。從張玉良復杭州,克長興、德清、武康。既而賊擾嘉興,景賢分兵屯南潯,扼其沖。四月,賊由太湖、夾浦犯湖州。道員蕭翰慶來援,戰歿,招其潰兵入伍,出北門擊賊,血戰數晝夜,賊遁。五月,率砲船進攻平望鎮,與楚軍合擊,克之。會賊酋陳玉成由溧水竄浙境,景賢回救,合民團要擊走之,賜號額爾德木巴圖魯,以道員用。六月,進復廣德,交軍機處記名簡放。十月,賊犯杭州,景賢馳援。湖州告警,速回師,賊已至南門外峴山。副將劉仁福率廣勇來援,有通賊狀,誘擒仁福,斬之以徇。賊奪氣,分擾四鄉,旋犯西門。合水陸擊退,盡破附近諸山賊壘,圍復解,加按察使銜。

十一年,復長興。尋賊踞洞庭東、西兩山,長興不能守,郡北七十二漊時被擾。景賢於大錢口增駐水師,聯絡民團,分顧各路,屢戰皆捷。五月,賊踞菱湖鎮。率水師進攻,毀賊舟,又破之於澉山溪。九月,賊又逼郡城,鏖戰五晝夜,追奔出境。時杭州久被困,景賢率兵滾營前進,連破賊卡二十餘處。賊復乘虛襲大錢口,景賢且戰且退,掩擊之,賊遁。聞杭州再陷,嘆曰:「湖郡孤注,惟當效死弗去,以報國恩耳!」是年冬,授福建督糧道。同治元年春,詔念景賢殺賊守城,於團練中功稱最,特加布政使銜。自賊氛逼城,僅大錢口可通太湖糧道。會大雪湖凍,賊由洞庭東山履冰來犯,大錢遂為所踞。

賊以屢戰傷亡多,恨景賢次骨,掘其父墓,戒不與戰,但斷絕糧道以困之。景賢迭出戰不利,密寄帛書至上海與其叔炳麟訣,誓以死守。朝廷惜其才,命曾國籓、左宗棠設法傳諭輕裝出赴任,景賢益感奮,選壯士三千人,分出斫賊營,奪其糧而還。被圍既久,兵日給米二合五勺,官民皆食粥糜,道殣相望。五月,城陷。

景賢冠帶見賊,曰:「速殺我,勿傷百姓。」賊首譚紹洸曰:「亦不殺汝。」拔刀自刎,為所奪,執至蘇州,誘脅百端,皆不屈。羈之逾半載,李秀成必欲降之,致書相勸。景賢復書略曰:「某受國恩,萬勿他說。張睢陽慷慨成仁,文信國從容取義,私心竊向往之。若隳節一時,貽笑萬世,雖甚不才,斷不為此也。來書引及洪承疇、錢謙益、馮銓輩,當日已為士林所不齒,清議所不容。純皇帝御定貳臣傳,名在首列。此等人何足比數哉?國家定制,失城者斬。死於法,何若死於忠。泰山鴻毛,審之久矣。左右果然見愛,則歸我者為知己,不如殺我者尤為知己也。」秀成赴江北,戒紹洸勿殺。景賢計欲伺隙手刃秀成,秀成去,日惟危坐飲酒。二年三月,紹洸聞太倉敗賊言景賢通官軍,將襲蘇州,召詰之,景賢謾罵,為槍擊而殞。

自湖州陷,屢有旨問景賢下落。至是死事上聞,詔稱其「勁節孤忠,可嘉可憫」,加恩依巡撫例優恤,於湖州建專祠,宣付史館為立特傳,予騎都尉世職,謚忠節。長子深彥,年十二,在湖南,聞湖州陷,即自酖死。先被旌,附祀景賢祠。次子濱彥,賜官主事;溱彥、淶彥皆以通判用。

論曰:何桂珍儒臣出為監司,以忠義激勵饑軍,竟撫悍寇;誤於庸帥,倉猝殞身。徐豐玉才裕匡濟,兵單致敗。溫紹原守六合,金光箸守壽州,並以彈丸一邑,出奇制勝,砥柱狂瀾,其有關於江淮全局者大矣。李孟群戰功卓著,至皖北兵食俱絀,卒不復振,父子繼死國事,為世所哀。趙景賢以鄉紳任戰守,殺敵致果,繼以忠貞。當時團練遍行省,自湖湘之外,收效者斯為僅見。諸人不幸以節烈終,未竟其勛略,惜哉!


\end{pinyinscope}