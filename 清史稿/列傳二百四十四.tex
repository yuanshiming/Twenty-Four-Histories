\article{列傳二百四十四}

\begin{pinyinscope}
蔣東才劉廷李承先李南華兄子得勝董履高董全勝

牛師韓曹德慶馬復震程文炳

方耀鄭紹忠鄧安邦

蔣東才,字軼眾,安徽亳州人。咸豐初,捻酋張洛行圍城,築砲臺高阜,俯擊城中,東才兄遇害,憤甚,乃創義團,為官軍前驅。會城中糧盡,東才殺馬饗士,與同邑劉廷、李承先夜縋城出襲,毀之,寇遁。

四年,投豫軍,英翰器其才,俾充哨長。戰亳北,被巨創,卒擒其渠。南道團練大臣毛昶熙檄領東震營,累勛至守備。商丘寇之據金樓也,東才築土為山頓其上。寇來襲,輒敗去,縱兵乘之,遇伏,砲石雨坌。東才方解衣激戰,寇突出襲我後,東才回矛決蕩,大殲其眾,寨拔。同治二年,規汝寧。夜獲邏卒,東才乃服寇裝,效口號,奪門入,諸軍踵之,夷巨壘。乘勢下南陽、息縣,又敗之商丘大周集。數遷至副將。七年,張總愚北犯,東才攻以火,殪寇千餘。又截擊任柱等黃河、徒駭間,晉總兵,賜號威勇巴圖魯,徙守運河。捻平,擢提督。明年,赴陜征甘肅竄匪,並敗退波羅營以西馬賊,更勇號額騰額。十二年,從克肅州,賞黃馬褂。事寧,假歸。

光緒初,統領豫軍。先後疏浚賈魯河、京師內外城河。除甘肅涼州鎮總兵,仍留豫。十三年,黃水暴漲,力護鄭州以下堤工,救難民二千餘。風雨罷勞,遘疾困篤,俄卒於工次。優詔賜恤,予開封、亳州建祠。

廷既解亳圍,旋奪西境兩河口,補千總,從宋慶駐守宋集。同治間,從攻懷遠,平高丘,積功至參將。任柱等掘滎澤將圖北,又從慶迎擊。夜率壯士襲其營,寇南走,廷截之光州,誅其酋張顯。復破張總愚於饒陽、臨邑,擢總兵。西捻平,晉提督,賜號額騰依巴圖魯。八年,入陜平綏德,賜秩頭品。寧夏既寧,賞黃馬褂。光緒四年,卒於洛陽,祀亳州。

承先,字光前。少英敏,好讀明戚繼光書。亳平,赴歸德擊高州匪,拔汝寧寨,與有功。同治間,攻張岡,匪首孫葵心來援,圍承先數匝,冒圍出,裹創力戰,敗之,遷都司。進解光州圍,連敗之上蔡、祥符。守黃河,降中牟寇馮增,再遷副將。張總愚竄畿南,又從慶敗之饒陽,賜號節勇巴圖魯。長驅玉林鎮,戰良久,中矛,浴血陷陣,大捷。逐北濟陽,直蹙之黃河,晉號志勇,擢總兵。錄守運河功,晉提督。

光緒十四年,河工成,遣散夫役近數萬,為奸民所惑,嘯聚硃仙鎮。提督董明禮被圍,巡撫倪文蔚議剿,承先止之曰:「用兵必有潰擾,歸、陳各屬不能安枕矣!且河工夫役數十萬,設有牽動,患更大。」乃單騎往撫,杖其前者數人,餘皆愕錯,受部勒。十七年,署河北鎮總兵,自同治八年至是凡三攝矣。尋補歸德鎮。四月,卒。亳民感之,建祠以祀。

李南華,字孟莊,安徽蒙城人。咸豐初,粵寇陷江寧,淮北捻蜂起。南華治團衛鄉里,擊捻數獲勝,累勛至守備。捻入境,率死士百人拒之,斬悍賊百餘,進討群捻,七戰皆捷,遷游擊,賜號猛勇巴圖魯。

同治改元,平澮北。先是,苗練沛霖跨有長淮,既輸款發、捻,大誅練之異己者,群練帖伏。獨南華與抗,翦除其黨,沛霖怒,遣張建猷等圍蒙。南華破之馬家店,再至再敗之,尋就撫。明年,復叛,築壘蒙城東南,斷我糧運,南華會總兵王才秀擊卻之。沛霖深塹長濠,謀久困。南華誓死守,數出戰,負重創,力疾攻之,尸山積。會糧絕,令眾潛取之以為食,一夕皆盡,寇大駴。出奇兵夜襲之,奪其輜重以歸,斬馘不可稱計。僧格林沁入城,見家食人肉,南華竟體創痕,深嘆異之。唐訓方上其功,超擢總兵。未幾,統全軍駐守懷遠。三年,徙臨淮、壽州。聞任柱入蒙境,亟還軍,而捻又竄豫,巡撫喬松年移撫陜,奏自隨。張總愚擾關中,率師馳擊之。陜事定,稱疾去。

久家居,慷慨好義,值歲饑嗛,毀家紓難,誦聲如沸焉。光緒二十四年,土寇牛世修倡亂渦陽。南華聞警,率練勇赴援,會各軍擊退之。明年,巡撫鄧華熙疏薦其才,令綜鳳、潁、六、泗團練,參皖北軍事。數獲劇盜,萑蒲斂跡。調赴豫,權河北鎮總兵,尋補福建汀州鎮。二十八年,卒。鄉人思其德,籥建蒙城專祠,報可。

其兄子得勝,佐治鄉團,亦頗力。沛霖之亂,戰常陷堅。累遷參將,賜號奮勇巴圖魯。蒙圍解,改練為軍,俾得勝領之。轉戰直、魯、蘇、豫,頗有功。克宿遷、郯城,擢總兵,補安慶協副將。直、東平,晉提督,更勇號曰剛安。移軍江寧,平土寇胡志瑞亂,仍歸於亳。十七年,卒,恤如制,附祀英翰祠。

董履高,字仰之,安徽合肥人。咸豐末,粵寇亂,治練衛鄉里。同治元年春,李鴻章治軍上海,履高隸戲下,從援北新涇、四江口。師攻常熟弗克,履高率敢死士數百先登,拔之。連克昭文。歷遷至守備。二年,從克江陰、無錫、金匱,移師浙江。時寇麕集嘉善,江、浙道梗。西塘勢險奧,寇據為嘉善犄角,殊死眾。履高率眾泅濟,直薄壘下,砲彈掠肩過,弗少顧,譟而上,手刃數悍賊,奪纛而舞,氣百倍,寇驚亂,拔西塘。旋克嘉善,江、浙師始相應。四年,師復宜興、荊溪、嘉定、溧陽,履高每戰必克。追擊金壇寇,斬馘數千,餘黨星散。五年,援湖州,破廣德,晉游擊。

鴻章征捻,檄履高出淮城,次車橋鎮,遇寇,擊敗之。寇截淮關稅銀,一日夜馳數百里奪還。時捻酋張總愚竄陜西,任柱、賴文光竄山東,履高東西馳逐,夷阬谷,拔鹵莽,當者輒靡。捻集麻城、光山、固始間,編木為城,實土其中,燃砲俯擊,眾莫敢偪。履高率千人,夜掘隧,曳木入,衷擊之,盡殪,擢副將。事寧,假歸省親。九年,募淮軍赴晉防河,以功晉總兵。

光緒三年,蒙古草地馬賊蜂起,移師防歸化、包頭。沙漠平衍,寇騎梠疾,日嘗數遇,以寡擊眾,月餘,討平之。母憂去。五年,起署潯州協副將。鬱林大竹根故盜藪,官軍莫能制。履高至,潛易裝詗其地,選勁卒數百,距寇巢十里外而軍,佯示怯,寇易之,不戒備。忽大風雨,履高銳師宵加之,擊殺數百人,寇奔遁。

九年,法越肇釁,移頓南寧、龍州備策應。明年,再署潯州協。思恩革生莫夢弼構苗匪,廣、黔各匪,據五崗以叛。五月,深入苗疆,擒夢弼誅之,事遂定。擢提督,賜號奇車伯巴圖魯。調署新太協,仍駐龍州。十月,徙屯枚,與法軍血戰數晝夜,左足中砲幾斷,當軸遽易之,諒山遂陷。年餘,創平,除慶遠協。尋謝病歸。十五年,補廣西左江鎮總兵,嚴軍紀,能捕劇盜,鄉民感之,為立主生祀焉。

二十五年,調直隸正定。時拳民始萌蘗,月朔望說法愚民。履高督兵擒其渠,繩以法,餘皆股慄,匪卒不敢入境。明年,畿輔大亂,獨正定晏然。歷江蘇淮揚鎮、貴州安義鎮,袁世凱疏留北洋練軍。三十一年,除壽春鎮。淮流盛漲,城不沒者數尺。履高晨夜徼循,修補救護,城得無虞。三十二年冬,巡視泗州防營,墜馬,舊創發,假歸。越二年,卒。詔優恤,予建祠。

董全勝,字凱臣,江蘇銅山人。同治初,以把總隸李鴻章軍,充馬隊官。攻無錫,全勝率敢死士為軍先,擒偽潮王黃子隆,城遂克。復金匱、宜興、荊溪、溧陽、常州、嘉興,皆有功,累擢守備。移剿捻,賊擾福山、寧海諸地,全勝分防吳家閘,潛出賊背夾擊,斃無算。賊南竄,復敗之莒州、沭陽,追抵海州境,賊創亡略盡。贛榆六塘河之戰,斃賊尤夥。東捻平,擢游擊,賞花翎。張總愚竄畿南,全勝敗之安平。賊偷渡滹沱河,全勝追剿偽懷王邱得才一支殆盡。賊趨天津,全勝冒雨急馳,繞截賊前。賊回竄高唐,南走陵縣、臨邑,適黃、運漲,賊既困於水,又屢受巨創,不能軍。西捻肅清,擢升參將。駐津沽管練軍營,率所部開濬陳家溝,抵北塘咸河百餘里,歲溉稻田無數。

光緒十四年,以塞永定河決口功,升副將。北運河紅廟漫口,全勝率軍堵塞,詔以總兵記名。二十二年,王文韶督直隸,擢充天津練軍翼長,兼帶中營砲隊。二十五年,卒,年六十,恤如例。

全勝治軍四十年,與士卒同甘苦,故臨陣咸為效命;又善以寡擊眾,身經數百戰,未嘗一挫敗。鴻章恆稱之。

牛師韓,安徽渦陽人。父斐然,官知府,在鄉治團練。師韓隨父擊賊,數挫之,稱「牛家團練營」。咸豐八年,投皖軍,破趙家海、檀城集,收撫各圩。十一年,發、捻各寇竄擾睢寧。師韓以少擊眾,克周堂,積勛至守備。同治二年,苗沛霖據鳳臺,與捻首張洛行互犄角,數百里寇寨林立。蒙城被圍久,士卒無現糧,城幾潰。英翰方牧宿州,亟赴援,而悍黨斜趨西南,將襲我後。適師韓率騎旅至,戰卻之,又出奇兵通運道。已而英翰以鳳潁道統蒙、亳諸軍,與捻相持數月。師韓常以騎兵摧寇鋒,援師續至,復選卒潰圍會援軍,躪寇壘數十,飛彈傷額,裹創力戰,寇黨殲焉,圍解。

先是,英翰計擒洛行,及其子喜、義子王宛兒,夜獻僧格林沁軍,先遣師韓單騎詣大營,乞兵迎解,穿寇壘而過。比寇覺,馳劫之,不及,張酋竟駢誅,時師韓年甫冠也,再遷至游擊。嗣從英翰剿發寇,戰霍山黑石渡,大敗之。未幾,陳得才、藍長春等構黨號十萬,游弋英、霍、潛、太間。師韓請英翰剿撫兼施,不及旬日,降者踵接。得才窮蹙自裁,而長春猶嵎負。師韓苦戰,嬰十餘創,屹不動。旋藍逆伏誅,餘眾悉平。張總愚合賴文光、任柱窺蒙、亳,圍雉河集。師韓聞警,率師直薄寇營,內外夾擊,遂解重圍,擢參將,賜號信勇巴圖魯。

六年,任柱竄山東,截之於郯城,又擊退宿遷、運河悍賊。東捻平,超擢總兵,更勇號曰達春。七年,西捻竄直、豫間,英翰請馳兵汴梁,入衛畿輔,檄師韓率騎旅三千會援。尋命駐黃河以南備守御。師韓日與豫捻鏖戰,所向輒捷,長驅山東境,復與諸軍截之恩縣。捻驚走,躡至鹽山、海豐,馳四晝夜抵高唐。捻湧至,將犯運河。會天大風,師韓趨上風邀擊,寇大潰。西捻平,賞黃馬褂。英翰疏稱其好謀能斷,堪勝提鎮任。捻酋宋景詩逋誅,復以計擒僇之,晉提督,賜秩頭品。

光緒元年,授河南歸德鎮總兵。十五年,調河北鎮,遭父憂去。二十年,日韓構釁,授甘肅寧夏鎮,命入衛,駐軍榆關。事定,還本官。二十一年,河湟回蠢動,師韓赴之。次西寧,聞平戎驛被圍久,亟入。大峽距驛四十里,悍回數千恃險負命。師韓以四百人制之,血戰竟日,賊敗潰,復大峽,其小峽寇亦遁。旋創發,卒於軍。當其赴援時,陰雨彌旬,山逕聳巘,行帳無所用,士卒有假居旅舍者。提督董福祥劾之,議奪職,師韓未及知而已疾終。事聞,詔復故官。總督周馥狀其績以上,予原籍建祠。

曹德慶,安徽廬江人。粵寇蹂皖,練團保境。嗣從官軍克柘皋、三河,被重創。改隸淮軍吳長慶麾下。同治改元,李鴻章督兵上海,檄德慶探賊,盡得其虛實,大破賊新橋。時總兵程學啟被圍,復從長慶疾擊之,圍解。自是官軍連下十餘城。德慶戰常陷堅,積勛至守備。水陸軍規蘇州,德慶一軍為游兵。蘇城既下,從克無錫、金匱,移師援浙,助擊平湖、乍浦、海鹽,據寇棄城走。興城寇來犯,迎擊敗之,彈貫右臂,裹創克嘉善,攻嘉興。從劉銘傳克常州,徇下宜、荊、溧、太、嘉諸邑,晉參將。再從郭松林援浙,克湖州;援閩,克潭、浦。東捻平,擢總兵。防直、東運河,銘傳困西捻黃、運間,德慶領所部橫擊之。西捻平,晉提督,賜號烈勇巴圖魯。師旋,駐守江蘇,歷揚州,徙浦口。會天旱,天長、盱眙鹽梟煽亂,擒其渠陳紅慶誅之,解遣脅從數萬人,發粟賑饑。駐江陰,建議築鵝梟嘴及下關砲臺。

光緒二年,統淮揚水師,疏濬赤山湖埂,蕩金陵諸河道。十年,法越釁起,移軍防吳淞,增築南石塘、獅子林砲臺。曾國籓疏薦其設防要隘,不避艱險,授狼山鎮總兵,留防如故。皖北饑,輸巨金助賑,詔旌之。十六年,罷戍,赴本官。時通海裏下河縱橫數百里,梟寇出沒,民苦之。德慶盡法懲治,奸宄浸息。二十七年,卒,恤如制,從祀長慶祠。

馬復震,字心楷,安徽桐城人。曾祖宗梿、祖瑞辰、父三俊,均見儒林傳。復震年十六,襲雲騎尉。以祖若父均死於賊,誓欲殺寇,投詩曾國籓行營。國籓奇其才,遂檄令增募兵,號淮勇。初,國籓治團練長沙,號湘勇。李鴻章募兵皖北,以淮勇繼之,然初不稱淮勇。淮勇之名,實自復震始。

國籓困祁門,復震扼祁門櫸根嶺。次年,會軍御寇石門橋。又從攻徽州,拔統領唐義訓於重圍。迭克黟縣、徽郡,又大捷屯溪、巖市,以解徽州圍;大捷孔靈,以克績溪、祁門。復震性剛,不能下人,人或讒之國籓,國籓稍稍戒飭之。復震頗責望國籓,謂:「不當用人言戒我,乃不我知也。」會左宗棠率師征浙,調復震從攻餘杭,比戰皆捷。餘杭既克,追寇至遂安、開化、馬金。湖州既克,追寇至鉛山縣坊湖鎮。常為諸軍選鋒,積功至副將。宗棠奏其血性過人,膽識堅定,又好學知書,請改文職,格於例,以總兵隨宗棠剿捻陜西。

復震自初入軍,即誓死滅賊,捻平,年三十,始歸娶。事母孝,友愛諸弟甚至。生有奇姿,骯髒不平,往往至於大醉泣下,輒歌詩以自遣。海疆日益多事,朝廷圖自強,創造火輪兵船。鴻章任湖廣總督,遂委復震管帶操江船,則益研求西國水師兵法。鴻章督直隸,調巡北洋,時國籓為兩江總督,仍令往來南北,且合疏薦復震沉毅有為,足勝海疆專閫。光緒三年,簡授陽江鎮總兵,已前卒月餘,年未四十。於是鴻章念其積勞久,且興淮軍及海上兵船,復震皆首其事,乃奏請優恤。著有莪園詩鈔;又嘗從寇中攜父殘稿出,展轉兵間,卒請宗棠序而刊之,為馬徵君遺集。

程文炳,字從周,安徽阜陽人。初結鄉團自衛。年十八,投袁甲三軍,領馬隊為選鋒,戰輒冠其曹,洊升至守備。從克定遠,破湖溝寇圩,補潛山營游擊。同治二年,率所部二千人駐蒙城。會苗沛霖構捻來犯,相持八閱月,大小百十戰,數獲勝。已而捻酋葛小年擁眾可數萬,殊死鬥,蒙圍益急,與布政使英翰內外夾擊,大敗之。僧格林沁軍至,誅沛霖。文炳會諸軍擒小年等駢僇之,皖北始稍靖。

四年夏,任柱、賴文光復入皖。英翰頓雉河集,與寇相持五十餘日,餉糈不繼,兵疲饉,文炳邀擊之,軍士戰稍卻,語所部曰:「此生死呼吸之際,汝輩尚不力耶?軍令在,不汝恕!」策馬陷陣,將弁繼之,呼聲震天,寇披靡。追戰至夜分,左臂中矛傷,裹創力戰,寇憚之。援至,因大破虜。論功,擢總兵。五年,補貴州清江協副將,駐軍皖北。

朝命英翰撫皖。初,文炳以軍事與英翰不相能,至是稱疾不出。英翰之母賢,諸將自史念祖以下均母事之。英翰以母命召文炳,至則拜床下,誓捐前隙共生死。比出,即檄統前敵師干。是時,捻騎飄忽成流寇,李鴻章既定圈河策,文炳統皖軍萬五千人,與總兵張得勝等進擊。東捻勢蹙,任柱死,其黨四散,大呼文炳名求降。鴻章逮降卒問故,僉曰:「昔皖北善後,程公以身家保鄉人。今我輩窮而乞憐,必能拯我。」其信義孚人如此。英翰上其功,擢提督。

六年,西捻張總愚北犯,詔文炳率師入直會剿。逾歲,敗之滹沱河。各軍至,捻狂奔,爭先渡河,棄騾馬貲糧河干。文炳下令軍中曰:「速濟追賊,敢取物者斬!」於是皖軍先渡,躡賊而南,斬馘無算。西捻平,賞黃馬褂,還駐亳。十二年,授江西九江鎮總兵。光緒二年,移疾去。明年,秦、晉大饑,捐巨貲佐袁保恆辦賑濟,民獲甦。五年,起署壽春鎮,旋補官南贛。九年,擢湖北提督。綠營廢弛久,文炳既受事,實行加餉抽練法,軍容一振。蒞官十載,遭本生繼母憂,終喪。會中日戰事起,詔趣赴京。至則命統皖軍駐守張家灣,尋授福建提督。

二十五年,入覲,假歸。明年,拳亂作,詔福建、江南、浙江、安徽、江西勤王軍受節度,赴彰、衛、懷備守御。又明年,提督長江水師,目睹船械窳敝,乃牒商劉坤一、張之洞改用快槍;調師船二百,編為游擊備策應。又以師船舊砲不能擊遠,與緣江各省籌易快砲,增餉益師,軍威始壯。宣統二年,卒,年七十有七。先是,詔疆閫諸臣條陳時政,文炳洞見新軍癥結,具疏未上。俄病篤,命繕入遺摺中,分編兵籍、節餉糈、增額缺、造器械、變操法五事。上嘉其老成謀國,下所司行。優詔褒恤,予本籍及立功省分建祠,謚壯勤。

方耀,字照軒,廣東普寧人。咸豐初,隨其父原治鄉團,所部多悍勇。嗣投官軍,征土匪有功,補把總。自是連克清遠、廣寧、德慶,截擊連州竄匪。總督黃宗漢疏薦謀勇冠軍,敘都司,賜號展勇巴圖魯。九年,發寇陳四虎侵廣寧,土匪四應。耀入自英德,會水師抵三峽,沉賊船,水路始通。進解陽山圍,擊退婆逕、黃陂各匪,匪奔韶州,復大破之。十年,從克仁化、南雄。總督勞崇光倚以破賊,令援贛,連下安遠、平遠。十一年,援閩疆,下武平、永定。時偽興王陳金缸陷信宜,數犯高州。耀還軍助擊,大敗之。

同治二年,肇、羅寇氛熾,客匪眾至十餘萬。耀與副將卓興以所部八千人夾擊之,迭破巨壘,焚其屯糧。其黨鄭金斬金缸以降,鄭金即鄭紹忠也。高州平,晉副將。三年,赴平遠八尺墟,坐縣城失守、進兵遲誤,暫褫職。時發寇丁太陽分據武平,耀自平遠進逼,奮擊退之。又設伏誘敵,乘勝徑斫賊營,大潰,城賊亦驚走,遂克武平;而丁賊猶據永定,負固不下,耀進圍之,詗知賊將赴金砂,隱卒以待。賊至,伏起,賊返奔,追襲之,奪城外砲樓土壘,俯瞰城中,日夜下襲,賊尸山積,啟東門遁,復故官。四年,偽康王汪海洋竄大埔,耀還軍扼守,遇偽侍王李世賢,血戰三晝夜,以少擊眾,大敗之。復與紹忠會師入閩,連克平和、詔安、長樂、鎮平,而餘匪嘯聚和平者勢猶盛。耀以無備,再褫職。旋收嘉應,復官。

七年,授南韶連鎮總兵,調署潮州。潮俗故悍,械斗奪敓以為常,甚且負嵎築寨,拒兵抗糧。耀以為積匪不除,民患不息,乃創為選舉清鄉法,先辦陸豐斗案,明正其罪。潮人始知有官法。陳獨目結會戕官,謝奉章恃險擅命,並捕治之,潮民遂安堵。暇輒釐占產,徵逋賦,丈沙田,潮稅歲增鉅萬。又御水患以保農田,建書院以育俊秀,士民頌之。總督瑞麟狀其績以上,賞黃馬褂。

光緒三年,調署陸路提督。五年,還本官,治潮州、南澳、碣石軍事。九年,法越構兵,充海防全軍翼長,改署水師提督。越二年,實授。嘗率師出搗盜穴,廣、惠安謐。十七年,卒,恤如制。

耀身矯捷,履山險若平地,眼有異光,暮夜擊槍靡弗中,以故粵中諸匪咸憚之。

紹忠,籍三水。始隨金缸為寇,既自贖,提督昆壽許領其眾為一營,號安勇。克廣西岑溪,賞都司銜,始更名。永定、大埔之役,與有功。數遷至副將,權羅定協。寇據嘉應,其黨譚光明等殊死戰。紹忠扼守長沙墟,寇至,擊卻之。城寇悉眾出,並力追擊,擒渠率,城拔。以次征肇慶、思平諸匪,賜號敢勇巴圖魯。平五坑客匪,更勇號額騰伊。自是察匪所向,捕之。不二年,擢提督,補潮州鎮總兵。光緒二年,搜治欽州、靈山積匪,晉秩頭品。五年,攻克瓊州、儋臨,賞黃馬褂。十年,權陸路提督。粵故多匪,紹忠頗善治之。攻剿遍粵境,轉戰閩、桂,匪為斂跡。十五年,授湖南提督。十七年,還綜廣東水師。二十年,加尚書銜。越二年,卒,恤如制。

鄧安邦,廣東東莞人。以勇目積功至守備。同治三年,從耀等克武平。四年,汪海洋陷鎮平,圍平遠。安邦赴援,抵城下,饑疲甚,雜食薯芋,卒解城圍。再敗賊大柘、超竹。嘉應陷,與諸軍截殲之,晉游擊,賜號銳勇巴圖魯,遷參將。光緒三年,補清遠營游擊。明年,匪首歐就起襲據佛岡,安邦約紹忠內外合攻,復其城,獲就起,置之法。十二年,授湖州鎮總兵。十四年,卒。

論曰:自發、捻起,各省興團練,淮、皖為盛,實淮勇之始也。東才以下諸人,初皆起鄉團,其後或隸豫軍,或隸淮軍,皆先後著戰績,為時所稱。方耀以粵團歸官軍,善戰兼謀勇,尤善治盜,民多感頌,茲故並著之。


\end{pinyinscope}