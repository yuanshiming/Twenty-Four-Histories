\article{列傳二百七}

\begin{pinyinscope}
韓超田興恕曾璧光席寶田

韓超,字南溪,直隸昌黎人。道光十四年,副貢。二十二年,天津治海防,超詣軍門獻策,事平,獎敘州判。尋以府經歷揀發貴州,歷署三角屯州同、獨山知州。獨山多盜,號難治。超募勇訓練,用土民為鄉導,擒其渠。胡林翼守黎平,深倚重之,言之巡撫蔣霨遠,超由是知名。

咸豐元年,烏沙苗倡亂,超從林翼進剿,馳風雪中,先後斬獲數百人,餘黨悉平。論功,以知縣用。二年,署清江通判。知黔將亂,捐俸豢勇士八十人,練成勁卒。四年,獨山土匪結粵匪內犯,超率兵練迎擊,分軍出賊後,攻其不備,擒賊首楊元保,復深入廣西南丹州境,擊諸匪平之,加同知銜,賜花翎。桐梓匪楊鳳竄永寧,合黔西匪王三扎巴連陷數城,圍遵義,超馳至,敗賊南關,陣斬王三扎巴,立解城圍。復追敗諸葛章司河,擒楊鳳斬之,餘黨盡殲,擢知府。五年,苗亂蔓延,超馳援臺拱,解黃平、平越圍;轉戰至施秉、鎮遠,賊塹山斷道,以阻官軍。超以孤軍馳突其間,大小數十戰,補石阡知府。

超性剛直,有膽略,每與上官爭執是非,上官責以剿賊而靳其餉,饑師轉戰,往往求協助於鄰省。四川總督駱秉章、湖北巡撫胡林翼交章論薦,侍郎王茂廕亦疏薦之,詔下巡撫問狀,以道員記名。九年,授貴州糧儲道。時苗、教各匪連陷諸郡縣,駐軍工⼙水汛,扼其中,使苗、教不得合,且遏其下竄湖南之路。賊出全力撲之,超約楚軍夾擊,賊大潰。剿思州響鼓坪,施秉土地坪,鎮遠金鼎、鋒嚴、唐家營諸賊巢,擒賊目張東山、歐光義等,鎮遠所屬皆平。

民團舊以十戶養一壯丁,超因其意稍變通之,官募士而民輸糧;又籍叛產分授降眾、流人,以田代餉;行之二年,得兵三千人。自軍興,協餉不至,地方官吏爭抽取釐金以為補助。超建議釐金統一,一抽之後,不復再抽,商無滯累,餉用差給。十年,命幫辦貴州剿匪事。十一年,署按察使。提督田興恕疏陳超前後戰功,加布政使銜,賜號武勇巴圖魯。詔予二品頂戴,署貴州巡撫。田興恕方以欽差大臣督辦軍務,超久在行間,亦以肅清全黔為己任。

同治元年,田興恕罷,乃命超辦理防剿事宜。時尚大坪、玉華山兩處為賊巢,遵義、安順、思南、大定、銅仁、石阡諸府所在皆賊,五月,回匪陷興義,雲南叛回潰勇擾境,粵匪亦由川竄至正陽、廟堂並桐梓、松坎諸地。超令總兵吳安康進剿,用內應夜縱火攻破賊巢,擒匪首倪老帽斬之,出難民二千餘人。六月,閔家場踞賊糾集苗、教諸黨偪江口,天柱匪首亦糾合土匪攻陷縣城,分股竄湖南晃州、高寨,陷工⼙水、青谿兩城,謀截楚軍糧道。超令總兵羅孝連、道員趙國澍進攻安順仲匪,夷其壘,擒斬賊酋韋登鳳等。尚大坪賊復約苗、教分掠江內,超令孝連斷其歸路,國澍等馳軍迎擊,復令副將趙德元出冷水河、梯子巖進襲尚大坪,立破之,卬水汛城同時克復。進平玉華山賊巢,攻拔瓦寨,復天柱縣城,特詔嘉獎。道員鄧爾巽、總兵李有恆,破王家苗寨、夾馬洞諸賊巢,獲其酋李玉榮等。黃、白號,教匪竄遵義,知府李德莪擊破之於三臺山,奪五里坎諸隘口。副將周宏順進攻石阡,毀老王賊巢,諸就撫。

石阡、銅仁苗匪攻毀鎮遠營壘,工⼙水戍軍亦潰,遂南掠松桃,北攻天柱。湖南援師至,賊始引去。詔斥超專恃援軍,有負疆寄。雲南方議撫回,巡撫徐之銘咨會停剿,而回匪益恣,竄陷安南、興義,分擾郎岱、永寧、歸化,詔原其誤信撫議,免議處。石達開自川回竄,分三路,一走遵義,一走黔西,一走桐梓。遣沈宏富、李有恆、余祖凱擊之。田興恕以教案獲譴去官,黔軍益單。二年,乞病回籍。光緒四年,卒於家,年七十有九。詔念前功,賜恤,謚果靖。貴州請建專祠,並附祀胡林翼祠。

田興恕,字忠普,湖南鎮筸人。年十六,充行伍,隸鎮筸鎮標。咸豐二年,從守長沙。賊屯湘江西岸,軍中募敢死士夜驚賊營,興恕請行,夜浮小舟往,潛燔賊營,賊騎數百追之,泅水免。巡撫駱秉章奇之,委充哨官。五年,從克郴州。六年,領五百人號虎威營。從蕭啟江援江西,克萬載、袁州。七年,戰上高英岡嶺,深入被圍,左手受創,亡馬,步戰,他將馳救,得免。是役以少擊眾,斃賊千數。進攻臨江,掘地道轟城,先登,再被創,賊死拒未下。援賊大至,啟江議暫退,興恕不可,曰:「兵在精不在多,原為前鋒。」率所部直貫賊陣,賊張左右翼圍之,後軍望見興恕旗指東麾西,賊皆披靡,夾擊,賊大敗竄走,遂復臨江。八年,克崇仁、樂安、宜黃、南豐,積功至副將,加總兵銜,賜號尚勇、摯勇兩巴圖魯。

貴州苗、教匪熾,黎平府被圍久。興恕奉檄赴援,至即攻破賊營,連戰三日而圍解,進克古州、永從,署古州鎮總兵。九年,石達開圍寶慶,興恕率四千五百人赴援,扼九鞏橋,無日不戰,歷月餘,糧藥將罄,選死士欲以一戰決勝負,會李續宜援軍至,內外夾擊,毀附城營三,連日攻下,勢如破竹。達開竄廣西,遂移軍靖州防黔邊,命署貴州提督,督辦貴州軍務,增軍盈二萬。十年,道銅仁,取印江,分軍略思南、石阡,進克貓貓山賊巢。

石達開由廣西入貴州,連陷數縣,省城大震。巡撫劉源灝趣赴援,興恕奏言:「黔省上游道路分歧,賊若以一軍擾黔,一軍入蜀,道遠兵單,斷難兼顧。已檄韓超防鎮遠,沈宏富守湄潭,劉義方進松桃,臣駐石阡,居中調度。賊如上竄,則親會川軍以攻之;窺楚,即馳還靖州。」時興恕已實授提督,詔授欽差大臣,命援省城。師至,部署省防,督軍赴定番迎剿,賊棄城而走。

十一年,兼署巡撫。時回、仲、苗、教諸匪分擾,上下游幾無完土。興恕分兵援剿,戰屢捷。招撫匪首唐天佑、賈福保、陳大六、柳天成等,克復歸化、荔波、定番、廣順、獨山諸城,疏通驛路,軍威漸振。興恕年甫二十有四,驟膺疆寄,恃功而驕,又不諳文法,左右用事,屢被論劾,乃罷兼職,以韓超代之。

同治元年,罷欽差大臣。會法國教士文乃爾傳教入黔,因事齟,興恕惡其倔強,殺之,坐褫職,赴四川聽候查辦。經遵義旺超,值雲貴總督勞崇光為賊所困。興恕驟馬沖入,大呼:「田某在此!」賊驚潰,翼崇光出。尋論罪遣戍新疆,行至甘肅,總督左宗棠奏請留防秦州。十二年,釋歸。光緒三年,卒於家。

曾璧光,字樞垣,四川洪雅人。道光三十年進士,選庶吉士,授編修,記名御史。入直上書房,授恭親王奕、醇郡王奕枻讀。咸豐九年,出為貴州鎮遠知府。同治元年,署貴東道。二年,剿平銅仁踞賊蕭文魁,賜花翎。雲貴總督勞崇光薦其才,迭署糧道、按察使、布政使。

六年,予二品頂戴,署貴州巡撫。七年,實授。貴州地瘠亂久,北境接四川,東境接湖南,軍事悉倚鄰援,本省餉既艱窘,將多驕蹇。總兵林自清劾罷後,戕興義縣令,率所部萬人擾川境。八年,璧光密遣提督陳希祥擒斬之,令吳宗蘭剿青山餘匪,克普安、安南。時席寶田軍已由東路進規臺拱,省城附近諸匪糅雜,出沒無常。九年,周達武調任貴州提督,率川軍至貴陽,漸次勘定。自軍興鄉試久停,至是年始補行,人心益定。與達武議增兵扼要駐守,令道員蹇訚破遵義賊,擒其酋吳三;令提督劉士奇克都勻,殪其酋吳章。

十年,令提督鍾有思等進剿上游,克永寧、威寧,下游諸軍擒悍賊潘得洪,收復八寨等城。又收復上江、下江、三腳各城,平上游鎮寧、歸化賊巢,殪永城踞賊侯大五,斬郎岱金家硐踞賊金大七,盤江北岸肅清。又破畢節、威寧諸匪,清八寨、三角餘賊,毀其巢。令總兵何世華擊斬安南賊酋潘么,進克貞豐,西路悉平。十一年,周達武率所部會楚軍定苗疆,詔嘉調度有方,予優敘。

十二年,會滇軍克新城老巢,全省肅清,加太子少保、頭品頂戴,予雲騎尉世職。尋新城防軍索餉譁變,匪首何玉亭攻新城,遣其黨黎正關攻興義,分軍馳剿,捕誅其渠,事旋定。光緒元年,卒於官,追贈太子太保,依總督例賜恤,謚文誠。四川、貴州請建專祠。

席寶田,字研薌,湖南東安人。諸生。咸豐二年,率鄉團殺賊,復縣城,獎敘訓導。六年,劉長佑援江西,招參軍事,遂從轉戰,積功累擢同知直隸州。九年,石達開由廣西犯湖南,寶田從解寶慶圍,擢知府。十年,駱秉章令募千人號精毅營,防湖南邊。廣東賊犯郴州、桂陽,擊走之。同治元年,石達開復由廣西入境,連敗之於黔陽,克來鳳,以道員記名,加按察使銜。

二年,粵匪黃文金大舉犯江西,命提督江忠義赴援,寶田副之,戰饒州桃溪渡,大破之;又迭破之於湖口、洋塘、石門、青山橋,賊引去,趨池州,圍青陽。寶田襲石嶺,破賊卡,分軍遮其前,命水陸夾擊,文金遁走,遂解青陽圍,累功賜號業鏗額巴圖魯,加布政使銜。江忠義卒於軍,寶田代領其眾,留防江西。三年,李世賢、黃文金復合犯江西,將以遙掣江寧之師,寶田逆擊白沙關,奪見橋要隘,鈔擊於大濟關、泥嶺關,賊竄山谷,復金谿,以按察使記名,授雲南按察使。

時楊岳斌初至江西督辦軍務,檄寶田援南豐,坐遷延被劾,降知府,留軍。會大軍克江寧,群賊擁洪秀全子福瑱逸出,由開化犯玉山,走瀘溪,寶田邀擊於新城,進至石城楊家牌,擒洪仁玕、洪仁政、黃文英等。福瑱匿山谷中,捕得之,檻送南昌,伏誅。詔復寶田原官,予雲騎尉世職,賜黃馬褂,授貴州按察使。時餘賊汪海洋等走廣東,四年,寶田自平遠邀擊,降萬餘人,又扼鐵石嶺,降者二萬,諸軍合擊於嘉應州,全數蕩平。論功,江西軍以寶田為第一,詔以布政使記名,遇缺題奏。軍事既定,請回籍終養,允之。

貴州苗、教諸匪構亂十有餘年,東路素倚湖南援軍,自粵匪平後,議大舉剿平。先是授兆琛為貴州布政使,偕總兵周洪印率師往,積歲無功。李元度圍荊竹園,亦久不下。巡撫李瀚章、劉昆先後劾罷兆琛、洪印,元度亦鐫級,薦起寶田招集舊部萬人入貴州,總統東路諸軍。

六年冬,進軍石阡,荊竹園為教匪老巢,寶田審視地勢高峻,匪砦環列,惟北面平夷可掩入。七年元旦,進攻,部將黃元果先登,諸將肉薄壘下,一日平十八砦,克荊竹園,擒斬匪首蕭桂盛、何瑞堂,其旁三十六砦相繼攻下。捷聞,被珍賚。夏,進規寨頭。寨頭為苗疆門戶,諸苗帑賄資糧所萃,連拔東西三屯,陣斬苗酋桂金保,破援賊張臭迷,攻下臺笠、丁耙塘諸砦,遂克寨頭。分軍克天柱,斬其酋陳大六。

會丁繼母憂,回籍治喪,提督榮維善暫領其眾,尋詔奪情趣赴軍,進攻臺拱。臺拱苗最強,踞清江、鎮遠二城為犄角。寶田請增兵萬人,按察使黃潤昌、道員鄧子垣領之出晃州為北路,寶田自當南路,令榮維善用雕剿法,轉戰山谷間,破諸苗砦,漸近鎮遠。潤昌、子垣由思州進攻鎮遠府城,克之。八年二月,維善連破董敖、公鵝兩隘,遂克清江城。兩軍合趨黃飄,山地狹峻,人行頂趾相接,遇伏。維善軍疾行先出險,潤昌軍誤以為陷伏中,爭道相擠,為賊所乘,潤昌、子垣皆戰歿。維善聞變,率二百人馳救,被圍,為苗所擒,遇害。於是苗氛復熾。

張秀眉犯巴冶,寶田親督軍擊走之,進克稿米,令龔繼昌、蘇元春破苗寨,擊走張臭迷等,分軍守鎮遠、施秉。時以寶田軍苦戰年餘,尚未深入,議罷其軍,劉昆仍主專任,復增兵萬人,分三路進。九年,會攻施洞,克之。苗走九股河,白洗苗來援,擊敗之。進攻臺拱,破革夷諸砦,薄臺拱城下,苗棄城走,克之,加頭品頂戴。進軍九股河,分別剿撫,凡平黑苗砦二百餘所。雞講、丹江苗皆請歸化。十年,進攻凱里,一鼓而下。苗潰走雷公山,麕眾六七萬人,黃茅嶺、雷口坪、九眼塘、燕子窩諸寨皆絕險,寶田督諸軍冒暑入山,合擊張臭迷,斬馘三萬,燔其廬舍,剿洗一空。駐軍施洞口,寶田遽病風痺,乞假醫療,命部將龔繼昌、蘇元春、唐本有、謝蘭階分統其軍。軍事進止機宜,仍稟命於寶田。

十一年,三路進兵,凱北以北悉定。合攻烏雅坡,諸酋皆在,以長圍困之。迭戰,斬九大白、巖大五於陣,先後降者數萬。四月,擒張秀眉、楊大六、金大五等,檻送長沙,伏誅。張臭迷先逸,捕得戮之。諸酋或降或斬,無脫者。苗疆平,詔晉寶田騎都尉世職,家居養痾。光緒十二年,詔以寶田前擒洪福瑱功,命曾國荃繪其像以進。十五年,卒,贈太子少保,優恤。原籍及江西、貴州建專祠。

論曰:貴州之匪,總名有六:曰苗匪、教匪,曰黃號、白號,其小者曰槓匪、仲匪,其他濫練、游勇、逆回、悍夷,揭竿踵起,不可悉數。始於咸豐四年,無兵無餉,不能制也。韓超有辦賊之才,久屈下僚,事權不屬。田興恕入黔,兵威始振,超亦驟起,未久相繼去。張亮基治黔數年,亦僅補苴。中原大定,曾國籓乃議以湖南兵力、餉力為平黔根本,而駱秉章亦令劉岳昭剿黔北以保川邊。後專倚席寶田,戡定苗疆。自周達武以川兵、川餉濟黔之不及,曾璧光賴之以竟全功。蓋閱二十年而後大定。古云:「蠻夷之人,先叛後服。」蓋以地勢使然。然使若韓超者早膺疆寄,其延禍或不致如是之甚。弭亂之道,在得其人,用人之道,必盡其才,固古今不爽者爾。


\end{pinyinscope}