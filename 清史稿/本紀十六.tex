\article{本紀十六}

\begin{pinyinscope}
仁宗本紀

仁宗受天興運敷化綏猷崇文經武孝恭勤儉端敏英哲睿皇帝,諱顒琰,高宗第十五子也。母魏佳氏,追尊孝儀皇后。乾隆二十五年十月初六日生。五十四年,封嘉親王。六十年九月,策立為皇太子,高宗將傳位焉,以明年為嘉慶元年。

嘉慶元年丙辰春正月戊辰朔,舉行內禪,上侍高宗遍禮於堂子、奉先殿、壽皇殿。高宗御太和殿,授璽。上即位,尊高宗為太上皇帝,訓政。頒詔天下,賜宴宗籓。庚戌,立皇后喜塔拉氏。寧壽宮舉行千叟宴,太上皇帝蒞焉。九十以上者,召至御座,賜卮酒如故事。辛酉,祈穀於上帝。癸亥,上奉太上皇帝賜廷臣宴於正大光明殿。凡賜宴皆如之。辦理苗疆大學士福康安等奏攻克朗坡,進攻平隴。湖北枝江、宜都教匪起。

二月丁丑朔,釋奠先師孔子。戊寅,祭社稷。庚辰,初舉經筵。辛巳,敕甘肅貴德建文廟。戊子,春分,朝日於東郊。己丑,上禦乾清門聽政,園居則御勤政殿,以為常。己亥,湖北當陽教匪起,戕官。西安將軍恆瑞率兵二千剿之。辛丑,祭歷代帝王廟。丙午,湖北巡撫惠齡奏獲教匪聶傑人。

三月庚戌,停四川續徵軍需銀兩。辛亥,上耕耤田,四推。壬子,上奉太上皇帝謁陵。丁卯,車駕還京。己巳,皇后祀先蠶。癸酉,恆瑞奏收復湖北竹山。壬申,留保住免。以烏爾圖納遜為理籓院尚書,富銳為綏遠城將軍,永慶為蒙古都統。

夏四月丙子朔,時享太廟。命宜綿、永保、恆瑞、孫士毅等分剿湖北教匪。辛巳,常雩,祀天於圜丘。以剿來鳳功,晉四川總督孫士毅三等男。敕伊犁貢馬由草地行。丁酉,上侍太上皇帝祈雨黑龍潭。是日,雨。庚子,賜趙文楷等一百一人進士及第出身有差。

五月戊申,詔額魯特來京有出痘者,嗣後由草地赴熱河覲見。辛酉,祭地於方澤。壬戌,上奉太上皇帝避暑木蘭。乙丑,以富綱為漕運總督。壬申,大學士、貝子福康安卒於軍。

六月乙亥朔,日有食之。以魁倫為閩浙總督,硃珪為兩廣總督。以紀昀為兵部尚書,金士松為禮部尚書,沈初為左都御史。丙子,調福昌為福州將軍。以明亮署廣州將軍。丁丑,除山西代州三州縣水沖田賦。戊寅,和琳奏獲苗匪石三保,解京誅之。癸巳,江南豐汛河決。

秋七月辛亥,明亮奏剿平孝感縣教匪。大學士、四川總督、三等男孫士毅卒於軍。

八月丙子,以雨停秋獮。壬寅,和琳卒於軍,命明亮、鄂輝接統軍務。

九月乙巳,車駕還京。

冬十月戊寅,上萬壽節,詣太上皇帝行禮。禮成,受廷臣賀。己卯,以董誥為大學士。王傑以足疾疏辭軍機處、南書房、禮部事,允之。命沈初為軍機大臣。辛巳,贈征苗陣亡提督花連布太子少保,予世職。丙戌,調沈初為兵部尚書,以紀昀為左都御史。

十一月庚戌,豐汛河工合龍復決。予湖北死事巡檢王翼孫、訓導甘杜、典史浦寶光世職。甲子,冬至,祀天於圜丘。乙丑,江西巡撫陳淮有罪,逮問遣戍。己巳,以湖北教匪偷渡滾河入秦,褫永保職逮問,以惠齡統其軍。

十二月戊子,湖南苗匪平,封明亮伯爵,額勒登保侯爵,及德楞泰等世職有差。庚子,祫祭太廟。辛丑,上奉太上皇帝御太和殿,賜宴朝正外籓。

是歲,免順天、江蘇、山西、湖南、福建等省三十九州縣災賦逋各賦有差。會計天下民數二萬七千五百六十六萬二千四十四名口,穀數三千七百二十萬六千五百三十九石一升二合七勺。朝鮮入貢。

二年丁巳春正月丁卯,貴州南籠仲苗夷婦王囊仙作亂,命總督勒保剿之。庚午,觀成奏四川教匪徐添德侵擾達州、東鄉,命總兵硃射鬥等剿之。

二月癸酉,上御經筵。江南豐汛復報合龍。戊寅,皇后崩,奉太上皇帝誥,素服七日,不摘纓。廷臣如之,近臣常服不掛珠。辛巳,敘景安剿擒教匪功,晉三等伯。戊戌,冊謚大行皇后曰孝淑皇后。惠齡奏獲匪首劉起等,解京誅之。

三月戊申,上謁西陵。丁巳,還京。癸亥,以劉墉為大學士,調沈初為吏部尚書,硃珪為兵部尚書。以福長安、慶桂為滿洲都統,德楞泰為漢軍都統。巴克坦布、慶成奏,由應山追賊入豫,查明賊首李全、王廷詔、姚之富均在其內。諭令擒捕。

夏四月壬申,設湖南鎮筸鎮總兵官,改保靖土縣為流官。辛巳,追贈侍郎奉寬太師、禮部尚書,上受書師也。

五月戊辰,上奉太上皇帝避暑木蘭。已巳,惠齡奏教匪姚之富等由白馬石搶渡漢江入川。詔罷總統慶成、恆瑞等,各降官,以宜綿為總統,明亮、德楞泰為幫辦。

六月癸酉,勒保奏,剿辦南籠仲苗,迭克水煙坪、卡子河等處。得旨:亟將苗首仙姑等擒獲。

閏六月庚子,吉慶奏克西隆州亞稿苗寨。丙午,勒保奏進克普坪,搶斃匪首,解南籠圍。詔獎紳民堅守危城,深明大義,改南籠府為興義府。勒保續報解黃草壩圍,滇、黔路通。壬戌,軍機章京吳熊光、戴衢亨均加三品卿銜,與侍郎傅森一體在軍機大臣上學習行走。

秋七月己巳,永定河決。己卯,命喀什噶爾、英吉沙爾二回城儲糧備荒。癸未,都統巴克坦布卒於軍。乙酉,免四川運送軍糈奉節六州縣明年額賦。

八月甲辰,永定河合龍。丙辰,範宜恆卒,調沈初為戶部尚書,紀昀為禮部尚書。己未,大學士誠謀勇公阿桂卒。丙寅,上奉太上皇帝還京。

九月戊辰,勒保奏攻克仲苗賊巢,獲賊首王囊仙等,解京誅之。封勒保三等侯。丁丑,上臨奠故大學士阿桂。甲申,以蘇凌阿為大學士,李奉翰為兩江總督。庚寅,詔宜綿、勒保、奉承恩、景安等分募鄉勇入伍剿賊。癸巳,詔曰:「聞賊每逼平民入夥,迎拒官軍。官軍報捷,所稱殺賊,多系平民,非真賊也。故日久無功。領兵大員尚其設法解散,勿令玉石俱焚。」甲午,以湖北恩施、利川,四川奉節士民奮勇殺賊,再免一年錢糧。

冬十月戊戌,明亮、德楞泰請廣修民堡,以削賊勢。詔斥其迂緩。丙辰,乾清宮交泰殿災。辛酉,命勒保總統四川軍務。

十一月丙寅朔,予陣亡散秩大臣佛住、護軍統領阿爾薩朗世職。

十二月戊申,以康基田為江南河道總督,司馬騊為東河河道總督。予陣亡總兵明安圖,副將曾攀桂、伊薩納等世職。甲子,祫祭太廟。

是歲,免順天、湖廣、陜西、雲南、甘肅等省五十七州縣災賦有差。朝鮮、琉球、暹羅入貢。

三年戊午春正月庚午,以梁肯堂為兵部尚書,胡季堂為直隸總督。甲申,調勒保為四川總督。乙丑,額勒登保奏獲賊首覃加耀。上責其遲延,奪額勒登保爵職。並以疏防奪明亮、德楞泰爵職,奪舒亮、穆克登阿職,籍其家,均隨軍自效。

二月丁未,上釋奠文廟,臨雍講學。以鄂奇泰為黑龍江將軍,慶霖為江寧將軍。辛亥,柯籓、烏爾圖納遜坐縱陜賊漢入楚,褫職。壬子,以吳省欽為左都御史。乙卯,命內閣學士那彥成在軍機處學習行走。

三月丁丑,德楞泰奏,追剿賊首齊王氏、姚之富,投崖死。予明亮副都統銜。己丑,以剿賊遲延,褫觀成、劉君輔職。以富成為成都將軍。

夏五月丙寅,免福建全省遠年逋賦。己巳,截留江西漕糧,接賑山東曹縣等十三州縣被水災民。甲戌,上奉太上皇帝避暑木蘭。

六月己酉,以剿賊遲延,盡奪德楞泰爵職,予副都統銜自效。甲寅,雲貴總督、三等男鄂輝卒。

秋七月庚午,富楞泰卒。以德勒格楞貴為寧夏將軍。以雨停秋獮。

八月,以獲教匪王三槐功,晉勒保及和珅公爵,福長安侯爵。己酉,張誠基奏江西西寧州教匪作亂,剿平之。

九月癸亥,上奉太上皇帝還京。己卯,祀明總制袁崇煥於賢良祠。

冬十月庚子,新建乾清宮交泰殿成。

十一月丁亥,左都御史舒常卒。

十二月乙巳,惠齡奏獲賊首羅其清、羅其書。戊午,祫祭太廟。

是歲,免陜西、貴州等省四十八州縣災賦有差。朝鮮、琉球、暹羅入貢。

四年己未春正月壬戌,太上皇帝崩,上始親政。丁卯,大學士和珅有罪,及尚書福長安俱下獄鞫訊。晉儀郡王永璇親王,貝勒永璘為慶郡王,綿億封履郡王,奕綸、奕紳在上書房讀書,綿志等各封賞有差。詔:「中外陳奏直達朕前,不許副封關會軍機處。」命成親王永瑆、大學士董誥、尚書慶桂在軍機處行走。沈初免直。成親王永瑆管戶部。丁丑,和珅賜死於獄,福長安論斬。己卯,特詔申明軍紀。命勒保為經略,明亮、額勒登保為參贊,並查詢劉清居官,具實保奏。吳省欽免,以劉權之為左都御史。以保寧為大學士,仍管伊犁將軍,慶桂協辦大學士,書麟為吏部尚書,松筠為戶部尚書。敘斬賊首冉文儔功,獎敘惠齡、德楞泰。丙戌,宜綿解任,以恆瑞為陜甘總督。丁亥,贈原任御史曹錫寶副都御史,廕一子。召前內閣學士尹壯圖來京。

二月己丑,以松筠為陜甘總督,布彥達賚為戶部尚書。辛卯,詔曰:「自教匪滋事以來,迫脅良民,焚毀田舍。民非甘心從賊,欲逃無歸,歸亦無食。亟宜招撫解散,而非空言所能收效。應如何綏輯安插,令勒保詢之劉清及其他良吏,籌議良法,俾可施行,速具以聞。」甲午,弛私售和闐玉禁。辛丑,秦承恩以貽誤軍事,褫職逮問。李奉翰卒,以費淳為兩江總督。乙巳,復宗室鄉會試例,增部院郎官宗室額缺。壬子,釋回徐述夔、王錫侯子孫緣坐發遣者。丁巳,錄用故大學士硃軾、孫嘉淦子孫。

三月己未朔,蘇凌阿免,以慶桂為大學士,成德為刑部尚書,傅森為左都御史。庚申,戶部尚書沈初卒,以範建中為戶部尚書。癸亥,以書麟為閩浙總督、協辨大學士。甲子,調慶霖為福州將軍,福昌為江寧將軍。戊辰,許直省道員密摺上奏。庚午,解景安任,以倭什布為湖廣總督,吳熊光為河南巡撫。丙子,額勒登保奏剿滅教匪蕭占國、張長更,上嘉之,予二等男。敘獎裨將硃射鬥、楊遇春等。戊寅,定侍衛軍政。壬午,追贈皇四兄履端郡王永為親王,皇七兄悼敏皇子永琮為哲親王,皇十二兄永璣為貝勒。癸未,勒保奏剿滅教匪冷天祿。得旨:「旬日之內,連翦三酋,深為可嘉,額勒登保晉一等男。」免河南被匪之鄧州二十州縣新舊額賦。甘肅布政使廣厚奏剿斃賊目張世龍。

夏四月己丑朔,欽天監言四月朔日,日月合璧,五星聯珠。上曰:「躔度偶逢,兵戈未息,何足言瑞。」予尹壯圖給事中,準回籍養親。丙申,恭上大行皇帝尊謚,禮成,頒詔覃恩。丁酉,免陜西被賊之孝義等三十五州縣新舊額賦。己亥,免四川被賊之奉節等三十六州縣新舊額賦。辛酉,詔遵奉皇考敕旨,於庚申、辛酉舉鄉會恩科。癸丑,賜姚文田等二百二十人進士及第出身有差。丙辰,以慶成為成都將軍。

五月戊午朔,停本年秋決。甲子,免湖北被賊之孝感等四十七州縣衛新舊額賦。庚午,江蘭罷,以初彭齡為雲南巡撫。庚辰,以傅森為兵部尚書,阿迪斯為左都御史。辛巳,克勤郡王恆謹以不謹削爵。甲申,以董誥為大學士。丁亥,敕費淳訪劾貪吏。詔免伯德爾格回民增金錢及葡萄折價。

六月己丑,增設步軍統領左右翼總兵官。庚寅,詔曰:「朕聞湖北隨州未被賊擾,因民人掘溝壘山,足資捍禦。民間村堡,侭可照辦。勒保、松筠、吳熊光即曉諭百姓知之。」辛卯,吳熊光、吳琠請加徵河工稭料運費銀。得旨申飭,下部議處。庚戌,恤陜西陣亡總兵官保興等世職。

秋七月辛酉,調山西兵三千赴湖北,盛京兵二千,額勒亨額統之,赴四川剿賊。癸亥,勒保奏獲賊首包正洪,予硃射鬥騎都尉世職。壬申,經略勒保以玩誤軍務奪職逮問,以明亮為經略,魁倫為四川總督。乙亥,削景安伯爵,遣戍伊犁。免甘肅被賊隴西等四十八州縣新舊額賦。辛巳,停中秋節貢。

八月己丑,富俊免,以興奎為烏魯木齊都統。壬辰,調盛京兵二千,吉林、黑龍江兵各一千,赴湖北剿賊。癸巳,以長麟為雲貴總督。乙未,勒保奏德楞泰生擒賊目龔文玉,給騎都尉世職。癸卯,罷明亮經略,命額勒登保以都統銜為經略。乙巳,命修撰趙文楷、中書李鼎元冊封琉球國王尚溫。己酉,慶成、永保以督軍不力逮問,命那彥成往陜西督辦。癸丑,編修洪亮吉致書成親王私論國政,遣戍伊犁。

九月丙辰朔,恤陣亡貴州副將孫大猷世職。丙寅,怡親王永瑯薨。庚午,大行梓宮發引,上恭送啟鑾。庚午,葬高宗純皇帝於裕陵。癸酉,還京。甲戌,高宗純皇帝、孝賢純皇后、孝儀純皇后升祔太廟,頒詔覃恩。辛巳,故湖廣總督畢沅坐濫用軍需削世職,奪廕官。壬午,明亮以剿賊不力罷參贊,褫都統,予副都統剿賊。

冬十月壬辰,調硃珪為戶部尚書,劉權之為吏部尚書,範建中為左都御史。丁酉,明亮奏獲賊首張漢潮。湖北道員胡齊侖以侵盜錢糧處斬。壬寅,德楞泰奏獲賊首高均德、高二。予德楞泰二等男。丁未,成親王永瑆免值軍機處。命傅森仍為軍機大臣。辛亥,命廷臣保舉賢良。壬子,勒保論斬,解京監候。

十一月甲子,故超勇公海蘭察子公安祿於四川剿賊陣亡,詔優恤之,名其子恩特赫默扎拉芬,襲超勇公。癸酉,免直隸積年逋賦。戊寅,興肇、慶成以帶兵不力遣戍。賞額勒登保銀一萬兩,德楞泰銀五千兩。庚辰,冬至,祀天於圜丘,奉高宗純皇帝配享,頒詔覃恩。

十二月壬辰,漕運總督蔣兆奎以率請加賦濟運罷。恤陣亡副將丁有成、德亮等世職。甲午,福寧以殺降報捷,景安以縱賊殃民,俱褫職逮問。丙申,額勒登保奏獲教匪王登廷。辛丑,姜晟奏獲湖南苗匪吳陳受。得旨嘉獎,加太子少保。壬子,祫祭太廟。

是歲,免河南、湖北被兵六十七州縣新舊額賦,徵兵經過直隸、河南、湖北田賦。又除江蘇、湖北各一縣坍田額賦,吉林三姓、黑龍江、雲南石屏州災賦。普免天下積年逋賦。朝鮮、暹羅入貢。

五午庚申春正月甲寅朔,上謁陵。丙辰,詣裕陵行初期祭禮。庚申,上還京。命額勒登保剿辦陜西教匪,德楞泰、魁倫剿辦四川教匪。辛酉,以松筠為伊犁將軍,仍留陜西剿賊。調長麟為陜甘總督,以玉德為閩浙總督,阮元為江蘇巡撫。壬戌,詔清查庫款,從容彌補,勿以嚴急而致累民。金士松卒,以張若渟為兵部尚書。辛未,祈穀於上帝,奉高宗純皇帝配享。解倭什布任,以姜晟為湖廣總督,移松筠剿湖北賊。戊寅,以景熠為黑龍江將軍。

二月丁亥,命那彥成參贊甘肅軍務。辛卯,以汪承霈為左都御史。癸巳,敕新疆鑄乾隆錢。壬寅,恤四川陣亡副將關聯升等世職。丁未,追論縱賊諸臣,秦承恩、宜綿戍伊犁。庚戌,予告大學士蔡新卒。

三月庚申,上謁陵。辛酉,解七十五任,逮京治罪。甲子,清明節,上行敷土禮。乙丑,阿迪斯以擁兵玩誤逮問,起勒保護成都將軍。丁卯,上幸南苑。德楞泰奏截剿渡江教匪,獲匪首冉添元,晉三等子。壬申,上謁西陵。乙亥,還京。辛巳,甄錄賢良祠大臣後裔。以縱賊渡嘉陵江,復過潼河,奪魁倫職逮問。以勒保署四川總督,起明亮藍翎侍衛從軍。

夏四月癸未朔,日有食之,乙酉,阿迪斯遣戍伊犁,以德楞泰為成都將軍。庚子,雲南惈夷平,加書麟太子太保。

閏四月甲寅,命刑部查久禁官犯及禁錮子孫與久戍者寬減之。丙午,上步禱祈雨。乙卯,釋洪亮吉回籍。丙辰,釋安南人黎屌等於獄,安置火器營,給月餼。是日,雨。丙寅,恤四川陣亡提督達三泰世職。戊辰,以那彥成不任戎務,罷直軍機處,召回京。

五月壬戌朔,夏至,祀地於方澤,奉高宗純皇帝配享。己丑,經略額勒登保以剿辦匪目劉允恭等功,晉三等子。丙午,那彥成到京,奏對無狀,降為翰林院侍講。

六月壬戌,額勒登保奏獲賊首楊開甲。丁卯,以張若渟為刑部尚書,汪承霈為兵部尚書,馮光熊為左都御史。甲戌,賜魁倫自盡,戍其子扎拉芬於伊犁。

秋七月辛卯,命右翼總兵長齡統吉林、黑龍江兵赴湖北協剿教匪。瑯玕奏青苗楊文泰作亂,剿平之。馬慧裕奏獲傳教首犯劉之協,解京誅之。丙申,禮部尚書德明卒,以達椿為禮部尚書。己酉,額勒登保奏獲賊目陳傑。

八月丙辰,固原提督王文雄剿賊陣亡,予三等子。

九月壬午,上謁東陵。戊子,還京。丁未,恤四川陣亡副將李錫命世職。

冬十月戊辰,胡季堂卒,以姜晟為直隸總督,書麟為湖廣總督,瑯玕為雲貴總督。

十一月乙酉,睿親王淳穎薨。己亥,恤陣亡革職將軍富成等世職。

十二月甲寅,陜西教匪徐添德竄湖北,湖北教匪冉學勝竄陜西,降責德楞泰、勒保等。丁巳,德楞泰奏獲教匪楊開第等。丙子,祫祭太廟。

是歲,免順天、江蘇、四川、雲南、甘肅等省七十州縣災賦,及兵差經過、坍田額賦各有差。朝鮮、琉球入貢。

六年辛酉春正月壬午,以傅森為戶部尚書,明安為步軍統領。辛卯,遣少卿窩星阿、裘行簡犒額勒登保、德楞泰軍。丁酉,德楞泰以剿山陽教匪功,復一等子。甲辰,德楞泰奏獲賊首高二、王儒。乙巳,勒保奏獲黃、藍、白三號賊目徐萬富等。

二月乙卯,勒保奏獲賊首王士虎。丙辰,書麟奏明亮獲賊目卜興昂。戊午,賜賢良後裔尚書魏象樞六世孫煜、尚書楊名時曾孫景曾、巡撫徐士林孫從旭舉人。戊辰,上謁陵,行敷土禮。壬申,上還京。改湖廣提督為湖南提督。置湖北提督,駐襄陽。改襄陽鎮總兵為鄖陽鎮總兵。癸酉,傅森卒,以成德為戶部尚書、軍機大臣。乙亥,額勒登保奏獲賊首王廷詔。

三月庚辰,詔:「被賊裹脅匪徒多系良民,凡投出者悉貸其死。軍前大臣仰體朕意,廣為宣示,務使周知。」恤陣亡總兵多爾濟扎布、李紹祖等世職。丁酉,賜賢良後裔大學士李光地四世孫維翰、尚書湯斌四世孫念曾舉人,巡撫傅弘烈六世孫縣丞徵瓏知縣。己亥,詔:「朕將謁陵,春苗申昜發,令大臣監護民田,勿許踐踏禾苗。」敘江西士民協剿教匪劉聯登功,改江西寧州為義寧州。辛丑,上謁陵。乙巳,行釋服禮。

夏四月丁未朔,上還京。己未,以四川民人輸資急公,免遂寧等八十六州縣明年額賦。辛酉,冊立皇后鈕祜祿氏。壬戌,協辦大學士、湖廣總督書麟卒,以吳熊光為湖廣總督。德楞泰奏獲賊首張允壽。丙寅,以獲王廷詔、高二、馬五功,晉額勒登保二等子,楊遇春騎都尉。戊辰,以兩廣總督吉慶協辦大學士。辛未,賜顧皋等二百七十五人進士及第出身有差。

五月己卯,賜賢良後裔大學士王熙曾孫元洪舉人。甲申,上祭文昌廟,始命列入祀典。乙酉,恤四川陣亡總兵硃射鬥視提督,予世職。丙戌,命總兵官輪班人覲。奉天府丞視學政,三年更任。乙巳,以額勒登保為理籓院尚書。

六月壬子,大雨。永定河決,分遣卿員撫恤被水災民。以水災停本年秋獮。姜晟免,發永定河效力。起陳大文署直隸總督。丙辰,復雨。西安將軍恆瑞卒。辛未,上步禱社稷壇祈晴。是日,晴。勒保奏東鄉青、藍號匪悉數殲除。

七月庚辰,特發在京兵丁口糧一月。甲申,命那彥寶、巴寧阿修築永定河工。勒保奏獲匪目徐添壽、王登高。戊戌,賑熱河水災。

八月丁巳,額勒登保奏獲匪首王士虎、冉添泗。勒保奏七十五獲賊目劉清選、湯步武等。甲子,勒保奏獲賊首冉學勝等,封三等男。

九月己丑,續修大清會典。

冬十月丙午,永定河合龍。癸丑,額勒登保奏獲賊首辛斗。德楞泰奏斃賊首龍紹周。癸亥,詔甄敘川、陜軍勞,晉額勒登保三等伯,德楞泰二等伯,賽沖阿騎都尉,溫春雲騎尉。

十一月甲申,貴州巡撫伊桑阿以驕黷欺罔賜死。癸巳,詔曰:「軍務即日告蕆,安插鄉勇為善後要事。其通籌詳議以聞。」乙未,額勒登保奏獲賊首高見奇。戊戌,七十五以縱賊,奪職逮問。己亥,升四川達州為綏定府,太平營為太平協。

十二月癸卯朔,慶成奏獲茍文明股匪。丁未,詔曰:「前奉皇考特旨,查考本朝殉節諸臣未得世職者,業經查出一百四十餘員,補給恩騎尉世職。茲又續查得九百九十餘員,開單呈覽,均系抗節效忠之臣。其子孫俱即給與恩騎尉世職,支給俸饟。除投標當差外,有原應試者,準作文武生員,一體應試。」癸亥,詔獎劉清,特授四川建昌道。壬申,額勒登保奏剿辦通江賊匪,斃匪目茍朝獻。辛未,祫祭太廟。

是歲,免直隸、山西、浙江、安徽、四川、雲南、甘肅等省二百三十一州縣衛額賦有差。朝鮮、暹羅入貢。

七年壬戌春正月癸酉朔,上謁裕陵,行三期祭禮。賜所過貧民棉衣。甲戌,定祭社稷壇用上戊。戊寅,上還京。壬午,以松筠為伊犁將軍。甲午,額勒登保奏獲首逆辛聰,餘黨悉平。吳熊光奏獲匪首張允壽子得貴,撲滅藍號賊股。明安以貪黷褫職,遣戍伊犁。以祿康為步軍統領,解刑部尚書。額勒登保以疏防茍文明竄渡漢江,降男爵。庚子,上御經筵。

二月癸卯,以茍文明竄南山老林,飭領兵大臣堵剿,地方官嚴密查拏,勿令蔓延。丁未,釋奠先師孔子。壬戌,優恤陣亡副將韓自昌與其弟副將韓加業,飭地方官為建雙烈祠,賜其母銀三百兩。丙寅,額勒登保奏劉清獲賊首李彬、辛文,加按察使銜花翎。

三月癸酉,勒保奏獲賊首張添倫、魏學盛、陳國珠。丁丑,德楞泰奏獲匪首龔其堯、李世漢、李國珍,餘黨悉平。壬午,上謁泰陵。庚寅,還京。壬辰,成德卒,以祿康為戶部尚書。

夏四月戊申,以顏檢為直隸總督。乙丑,賜吳廷琛等二百四十八人進士及第出身有差。丁卯,慶成奏獲賊首魏洪升、張喜、白庸。

五月己卯,睿親王寶恩薨。瑯玕奏獲惈匪首逆臘者布。壬午,勒保奏獲匪首庹向瑤、徐添陪、張思從。甲午,慶成奏搜捕餘匪,獲康二麻、張昌元,加太子太保。

六月己酉,德楞泰奏教匪樊人傑溺水死,俘其妻孥,餘匪殲盡,晉封三等侯。甲寅,命劉權之、德瑛為軍機大臣。乙卯,達椿卒,以長麟為禮部尚書。命保寧管理兵部。以祿康、恭阿拉為漢軍都統。

秋七月辛未,勒保奏剿殲黃、白、青、藍四號賊匪,晉一等男。庚辰,陜西貢生何泰條陳黜奢崇儉,挽回風氣。得旨可採,賞大緞二匹。甲申,大學士王傑致仕,加太子太傅,在籍食俸。戊子,上秋獮木蘭。癸巳,詔曰:「廣東博羅監犯越獄一案,經朕硃諭查詢,始據該督撫據實陳奏。則天下事之不發覺者多矣,殊堪感嘆,更深懍畏。除分別懲治外,尚其大法小廉,用副澂敘官方至意。」以興奎為西安將軍,明亮為烏魯木齊都統。甲午,額勒登保奏獲逆首茍文明。諭:「適到木蘭,便聞捷音。教匪起事諸犯,只餘此賊。今既授首,不難肅清。額勒登保晉一等伯,楊遇春以下,各優予敘賚。」張若渟卒,以熊枚為刑部尚書。轉汪承霈為左都御史,戴衢亨為兵部尚書。

八月己亥朔,日有食之。詔曰:「月朔日食,月望月食,天象示儆,兢惕時深。朕躬有闕失歟?剿捕邪匪,餘孽未盡,其應靖以兵威,或迪以德化歟?政事有不便於民者,或一時行之,日久則滋流弊歟?其各讜言無隱。至月食修刑,惟當於明法敕罰,力求詳慎,所當與內外諸臣交勉焉。」以硃珪協辦大學士。癸卯,以嵇承志為東河河道總督,以劉清為四川按察使。乙卯,上行圍。越南農耐、阮福映率屬內附,繳前籓敕印。詔許其入貢。辛酉,德楞泰奏獲賊首蒲添寶。

九月庚辰,上回鑾。戊子,上謁陵。辛卯,還京。丙申,吳熊光奏斃黃號匪首唐明萬。

冬十月己酉,杭州將軍弘豐卒,以張承勛為杭州將軍。壬子,勒保奏獲白號賊首張簡、藍號賊首湯思蛟。丁巳,德楞泰奏斃賊首戴四,獲賊目趙鑒。

十一月戊辰朔,德楞泰奏獲賊首陳傳學。庚午,詔以吉慶辦理傅羅會匪,奏報不實,免協辦大學士,命那彥成查辦。尋解總督,敕瑚圖理署理。丙戌,額勒登保奏獲賊首景英,晉三等侯。

十二月戊戌朔,安徽宿州盜匪作亂,費淳等討平之。癸丑,詔額勒登保、德楞泰、勒保、惠齡、吳熊光會報川、陜、楚教匪蕩平。封額勒登保、德楞泰一等侯,勒保一等伯,明亮一等男,賽沖阿、楊遇春以次封賚。並推恩成親王永瑆等、軍機大臣慶桂、董誥等。乙丑,祫祭太廟。

是歲,免直隸、陜西、江西、四川等省五十六州縣災賦。除江蘇、福建、山東十縣衛坍田額賦。朝鮮入貢。

八年癸亥春正月庚午,以倭什布為兩廣總督。丁丑,命伊犁廣開民田。張誠基以剿辦義寧州土匪陳奏不實,論絞。乙酉,賜貧民棉衣。甲午,上御經筵。

二月己未,上謁東陵。

閏二月戊寅,上還駐圓明園。乙酉,還宮,入順貞門,奸人陳德突出犯駕。定親王綿恩、額駙拉旺多爾濟及丹巴多爾濟等擒獲之,交廷臣嚴鞫。獎賚綿恩等有差。丁亥,祀先農,上親耕耤田。己丑,詔曰:「陳德之事,視如猘犬,不必窮鞫。所慚懼者,德化未昭,始有此警予之事耳。即按律定擬。」是日,陳德及其二子伏誅。予告大學士王傑陛辭,賜玉鳩杖,御書詩章,馳驛回籍。庚寅,嚴申門禁。

三月丙申,御試翰林。甲辰,甘肅提督穆克登布以剿捕餘匪陣亡,贈二等男。恤湖北陣亡總兵王懋賞等世職。庚申,皇后行躬桑禮。

四月丙戌,上祈雨。丁亥,雨。

五月乙未,建宗室、覺羅住房。癸丑,以富俊為吉林將軍。

六月戊子,尚書彭元瑞乞休,允之,仍總裁高宗實錄。以費淳為兵部尚書,陳大文為兩江總督。己丑,封阮福映為越南國王。

秋七月乙巳,以那彥成為禮部尚書。丁未,以三省餘匪肅清,優★額勒登保、德楞泰及軍機大臣。壬申,上巡幸木蘭。

八月壬午,調富俊為盛京將軍。以停止行圍回鑾。辛卯,上還京。

九月戊申,致仕尚書、前協辦大學士彭元瑞卒。

冬十月壬申,瑯玕奏獲首犯恆乍綱,枲僳匪平。癸未,葬孝淑皇后於山陵。

十一月戊戌,硃珪等請磨敬一亭明代碑文,上不許。

十二月己丑,祫祭太廟。

是歲,免直隸、山東、河南、江蘇、安徽、陜西、湖北、四川、雲南、甘肅等省四百十八州縣衛災賦逋賦有差。朝鮮、越南入貢。

九年甲子春正月丁未,調興奎為寧夏將軍,賽沖阿為西安將軍。

二月壬戌,上御經筵。癸亥,上臨幸翰林院,賜宴,賦柏梁體詩。戊子,上謁東陵。

三月壬辰,幸盤山。壬寅,詣明陵,尊酒長陵。甲辰,上還京。

夏四月己巳,上閱健銳營兵。丙子,召嵇承志來京,以徐端署河東河道總督。

五月甲午,上祈雨黑龍潭。丁酉,雨。丁未,鐵保奏進八旗詩一百三十四卷,賜名熙朝雅頌集。

六月壬戌,玉德等奏海盜蔡牽擾及鹿耳門,突入汕大寨。得旨:追擒務獲。戊辰,以祿康協辦大學士,明亮為工部尚書,長麟為刑部尚書,費淳為吏部尚書。德瑛罷直軍機處,以那彥成、英和為軍機大臣。乙亥,惠齡卒,以那彥成為陜甘總督。恤捕海盜陣亡總兵胡振聲,贈提督,予世職,錄用其子。

秋七月丙午,上巡幸木蘭。庚子,初彭齡以誣參吳熊光褫職。癸丑,以歲周水夾甲,停本年決囚。

八月己未,清查湖北濫支軍需,追罰福康安、和琳之子並畢沅等。丁丑,上回鑾謁陵。

九月庚寅,上幸南苑行圍。辛卯,以搜捕三省餘匪凈盡,甄敘額勒登保以次有差。甲午,上還京。

冬十月癸酉,廣西武緣知縣孫廷標匿傷縱兇,特旨處絞,臬司公瓘遣戍烏魯木齊。己卯,上御惇敘殿,賜宴宗室諸王。

十一月戊申,調那彥成為兩廣總督,倭什布為陜甘總督。

十二月丁卯,調徐端為江南河道總督。庚辰,大學士劉墉卒。甲申,祫祭太廟。

是歲,免直隸、湖北、四川等省二十一州縣災賦有差。朝鮮、暹羅入貢。

十年乙丑春正月乙未,予告大學士王傑因賜壽來京卒,優詔恤贈。辛亥,以硃珪為大學士,紀昀協辦大學士,以鐵保為兩江總督。詔內務府大臣嚴行約束內監,稽其出入,纂入宮史,著為令。

二月己未,上御經筵。己巳,禮親王永恩薨,子昭梿襲。協辦大學士紀昀卒,調劉權之禮部尚書、協辦大學士。

三月己丑,上幸南苑行圍。己亥,上謁泰陵。丙午,回鑾,閱健銳營兵。戊申,上還京。以弘康為廣州將軍。

夏四月辛巳,御史蔡維鈺疏請查禁西洋人刻書傳教。得旨:一體查禁。戊寅,賜彭浚等二百四十三人進士及第出身有差。

五月甲申朔,詔內務府大臣管理西洋堂,未能嚴切稽查,任令傳教,下部議處。其經卷檢查銷毀,習教之佟瀾等罪之。戊申,追敘削平教匪清野功,加勒保太子太保,明亮一等子。

六月庚申,顏檢以失察虧帑黜免,調吳熊光為直隸總督,百齡為湖廣總督。丁丑,永定河決。

閏六月癸未,劉權之免,以費淳協辦大學士,秦承恩為左都御史。戊戌,永定河合龍。乙巳,以清安泰為浙江巡撫。

秋七月壬辰,上詣盛京謁陵啟鑾。

八月丙戌,上祭北鎮廟。乙未,上謁永陵。丙申,行大饗禮。閱吉林官兵射。庚子,上謁福陵,行大饗禮。辛丑,上謁昭陵,行大饗禮。臨奠克勤郡王岳託、武勛王揚古利、弘毅公額亦都、直義公費英東墓。上駐蹕盛京,詣寶冊前行禮。甲辰,詣天壇、地壇行禮。乙巳,上御崇政殿受賀。御前大臣、三等公額勒登保卒,建祠京師。以慶成為成都將軍。丙午,上御大政殿,賜扈從王大臣及朝鮮陪臣宴。禦制盛京頌八章。賜朝鮮國王李松御書匾額。戊申,上回鑾。

九月己巳,上謁東陵。壬申,還京。丙子,臨奠額勒登保。

冬十月甲午,命戴均元馳赴南河勘工。丙申,英吉利國王入貢,賜敕並文綺。辛丑,那彥成免,調吳熊光為兩廣總督,裘行簡署直隸總督。癸卯,以賽沖阿為廣州將軍。

十一月丙辰,百齡免,以全保為湖廣總督。己未,以慶溥為湖北提督。

十二月丁未,祫祭太廟。

是歲,免直隸、山西、陜西等省三十四州縣災賦及兩淮十一場額課有差。會計天下民數三萬三千二百一十八萬一千四百三名口,穀數二千九百四十一萬一千九百九十九石七升三合二勺。朝鮮、英吉利入貢。

十一年丙寅春正月壬子,海盜蔡牽陷鳳山縣,命玉德剿辦,調廣州將軍賽沖阿馳往督辦。丙子,那彥成以在署演戲,濫收海盜,奪職,戍伊犁。

二月癸未,上御經筵。辛卯,上謁東陵。甲辰,上幸南苑行圍。戊申,還京。

三月己丑,臺灣總兵愛新泰克復鳳山縣,予世職。

夏四月辛卯,上閱健銳營兵。癸巳,李亨特免,以吳璥為河東河道總督。丙申,續編皇清文穎。

五月丙寅,玉德罷,以阿林保為閩浙總督。

六月戊寅,調姜晟為工部尚書,秦承恩為刑部尚書。庚辰,慶成以奏對失實削職,戍黑龍江。以特清額為成都將軍。庚寅,以戴均元為江南河道總督,徐端為副總河。庚子,命德楞泰管理兵部。

秋七月癸亥,寧陜鎮新兵陳逢順糾黨戕官,陷洋縣,擾及寧羌。命德楞泰統巴圖魯侍衛、索倫兵剿之。丁卯,上巡幸木蘭。

八月庚寅,上行圍。甲辰,李長庚奏剿殲蔡牽匪黨多名,蔡牽逸。

九月乙巳,發巴圖魯侍衛、索倫等兵赴陜西。癸丑,論直隸失察侵帑案,顏檢戍烏魯木齊,降姜晟、陳大文、熊枚四品京堂。起初彭齡為安徽巡撫。庚申,起劉權之為左都御史。癸亥,上還京。

冬十月丁丑,德楞泰奏剿平洋縣叛兵。甲申,以全保為陜甘總督,汪志伊為湖廣總督,曹振鏞為工部尚書。丁亥,以溫承惠為直隸總督。起阮元署福建巡撫,以病辭。調張師誠為福建巡撫,金光悌為江西巡撫。癸巳,以和寧為烏魯木齊都統。大學士保寧乞休,優詔致仕,予食公俸。

十一月庚申,以祿康為大學士,長麟協辦大學士,文寧為步軍統領。詔以德楞泰剿辦叛兵,寬大受降,切責之,降楊遇春寧陜鎮總兵,楊芳遣戍伊犁,即押降兵赴戍。

十二月戊寅,大學士硃珪卒。己卯,上臨第賜奠。庚辰,特詔旗民力求節儉。辛丑,祫祭太廟。

是歲,免直隸、四川等省三十五州縣災賦有差。朝鮮、琉球入貢。

十二年丁卯春正月丙午,以費淳為大學士,戴衢亨協辦大學士。癸亥,詔曰:「從前剿辦邪匪,鄉勇過多。迨事平遣散為難,多令入伍充兵。今陜之寧陜,川之綏定,迭報新兵滋事,隨時剿平。此等獷悍之徒,必須隨時懲創,勿令別生事端。」戊辰,陜西瓦石坪新兵滋事,討平之。

二月甲戌,上御經筵,戊子,積拉堪罷,削爵。壬辰,上謁東陵。

三月壬辰,上幸南苑行圍。辛亥,謁西陵。甲寅,還京。丁巳,高宗實錄、聖訓成。辛巳,上祈雨。甲子,雨。

夏四月丙戌,上閱健銳營兵。庚子,上祈雨。

五月己丑,雨。己未,以長齡為陜甘總督,薩彬圖為漕運總督。丙寅,增定河工料價。雍正以來,常年工費率六十萬。自此馴增百六十萬。

六月乙未,禁督撫幕友矇保入官。

秋七月乙巳,命編修齊鯤、給事中費錫章冊封琉球國王。戊午,上巡幸木蘭。

八月乙酉,上行圍。

九月丙午,上還駐木蘭。暹羅私招商人貿易,降敕訓止之。辛亥,上回鑾。甲寅,閱古北口兵。丙辰,還京。

冬十月乙未,令武鄉、會試內場罷策論,默寫武經。

十一月辛丑,塞陳家浦壩口,導黃河由故道入海。

十二月癸未,調清安泰為河南巡撫,以阮元為浙江巡撫。癸巳,祫察太廟。

是歲,免直隸、江蘇、四川、甘肅等省四十七州縣災賦鹽課。除江蘇、福建、山西五縣水沖坍田額賦。朝鮮、琉球、南掌入貢。

十三年戊辰春正月戊午,浙江提督李長庚追擊海盜,卒於軍,贈伯爵。以部將王得祿為浙江提督。

二月丁卯,命皇次子釋奠先師孔子。庚午,上御經筵。丙子,予告大學士保寧卒。戊寅,特詔獎敘湖南辰沅永靖道傅鼐,加按察使銜。

三月庚子,上謁東陵。壬午,上巡閱天津長堤。丙辰,以徐端為南河河道總督。己未,上閱天津鎮兵。丙寅,上幸南苑行圍。命長麟、戴衢亨勘察南河。

夏四月戊辰,上還京。辛卯,賜吳信中等二百六十一人進士及第出身有差。

五月癸卯,長麟、戴衢亨奏查勘河工,請用一百三十餘歲張姓老民指出靳輔舊於天然閘東建閘二座,驗有壩基,擬請修復。得旨照準,賞老民銀緞。庚申,修闕里孔廟。

閏五月壬午,湖南提督仙鶴翎以表賀生皇長孫失辭,罷。

六月甲辰,禦制耕織圖詩,刊於授時通考。乙巳,秦承恩免,以吳璥為刑部尚書。

秋七月庚辰,上巡幸木蘭。

八月己酉,上行圍。甲寅,恤廣東捕盜被戕總兵林國良世職。

九月己卯,上還京。

冬十月癸巳朔,日有食之。

十一月壬午,吳熊光罷,以永保為兩廣總督。庚寅,以興肇為杭州將軍。

十二月壬辰朔,命皇次子詣大高殿祈雪。己亥,上祈雪。乙巳,雪。以周興岱為左都御史。己未,祫祭太廟。

是歲,免直隸、四川等省十三州縣災賦逋賦。除直隸、江蘇、浙江、福建、雲南、甘肅等省十一州縣沖田額賦,浙江、福建二場坍地額課。朝鮮、琉球入貢。

十四年己巳春正月辛酉朔,上五旬萬壽節,頒詔覃恩。加封儀親王永璇子綿志、成親王永瑆孫奕綸為貝勒,加恩籓臣、廷臣有差。丁卯,以百齡為兩廣總督。壬申,廣興有罪處斬,子蘊秀戍吉林,籍其家。緣以降黜者多人,長齡戍伊犁。以和寧為陜甘總督。

二月壬辰,上御經筵。壬寅,上制崇儉詩、義利辨,頒示廷臣。丁未,上謁東陵。丁巳,福建總兵許松年殲斃海盜硃濆,予世職。己未,上還京。

三月癸亥,上謁西陵。丙子,還京。西安將軍、三等公德楞泰卒。己卯,松筠奏遣戍叛兵蒲大芳、馬友元等一百餘人在戍不法,均分起誅訖。上責其濫殺,奪職。以晉昌為伊犁將軍,興肇為荊州將軍。

夏四月甲寅,賜洪瑩等二百四十一人進士及第出身有差。吳熊光戍伊犁,百齡劾之也。孫玉庭罷。

五月丁丑,特詔切責廷臣洩沓。戊寅,巡漕御史英綸以貪婪卑污處絞。

六月乙未,倉場黑檔盜米事發,責黜歷任侍郎有差。丁未,以松筠為陜甘總督。

秋七月戊辰,詔停本年秋決。江蘇查賑知縣李毓昌為山陽知縣王伸漢毒斃,下部鞫實,王伸漢立斬,知府王轂立絞,家丁李祥等均極刑,總督鐵保奪職遣戍,巡撫汪日章奪職。上制憫忠詩,賜其嗣子李希佐舉人、控訴得申武生李清泰武舉。調阿林保為兩江總督,以方維甸為閩浙總督。壬申,給事中花傑以參劾軍機大臣戴衢亨徇私不得直降官。乙亥,詔曰:「朕恫在抱,每直省報災,無不立霈恩施,多方賑恤。乃督撫不加查察,致有冒賑之事。如近日寶坻、山陽二案,竟謀斃持正委員,豈可不加以懲治,非有所靳惜也。御史周鉞因請報災之處,另委道府詳查。不知道府又安盡賢能。現在寶坻一案,該管東路同知歸恩燕即曾索銀三千兩。山陽一案,該管知府王轂收銀二千兩。設遇此類道府,又可信乎!道府亦不能遍歷村莊,仍委之委員,益不足憑矣。其要惟在督撫得人耳。至若以查災為難,因而相率諱災,則其咎更重矣。將此通諭知之。」壬午,上巡幸木蘭。

八月庚戌,浙江學政、侍郎劉鳳誥以監臨舞弊褫職,戍黑龍江。巡撫阮元以徇隱奪職。

九月己未,以慶成為福州將軍。庚申,上還京。己巳,張師誠疏報王得祿、邱良功合剿海盜蔡牽,緊逼賊船,沖斷船尾,蔡牽落海淹斃。予王得祿子爵,邱良功男爵。壬申,百齡疏請粵鹽改陸運,從之。

冬十月癸巳,上萬壽節,御太和殿受賀,賜宴。庚戌,阿林保疏請漕糧加折收納,上嚴斥之。

十一月壬辰,以松筠為兩江總督,那彥成為陜甘總督。

十二月戊戌,以失察工部書吏冒領戶部、內務府官銀,祿康、費淳以次降黜。甲寅,祫祭太廟。

是歲,免直隸、江蘇等省二十四州縣災賦。除順天文安窪地、浙江錢清場、湖南茶州坍地田賦。朝鮮、琉球、暹羅、越南、南掌入貢。

十五年庚午春正月丙子,以劉權之為協辦大學士。

二月己丑,上御經筵。壬辰,長麟以疾免,以瑚圖禮為刑部尚書,托津為工部尚書。丙申,召勒保來京,以常明為四川總督。丙子,詔以鴉片煙戕生,通飭督撫斷其來源。

三月甲子,上謁東陵。戊寅,上幸南苑行圍。癸未,還京。

夏四月丁酉,上閱健銳營兵。

五月癸亥,勒保以不奏匿名書,罷大學士,降工部尚書。復以祿康為大學士,明亮協辦大學士。以戴衢亨為大學士,費淳為工部尚書。

六月戊戌,改熱河副都統為都統,以積拉堪補授。壬子,百齡以擒解海盜烏石二功,予輕車都尉世職。

秋七月甲寅,永定河溢。壬申,上巡幸木蘭。辛巳,以徐端為南河河道總督。修改雲梯關海口,命馬慧裕督辦。

八月戊戌,上行圍。壬子,以皁保為蒙古都統。設廣東水師提督,陽江鎮水師總兵。

九月己未,以汪志伊為閩浙總督,馬慧裕為湖廣總督,恭阿拉為工部尚書。甲子,永定河漫口合龍。己巳,上還京。乙亥,增南河稭料價銀。

冬十月甲午,江南高堰、山盱兩堤決壩。丁酉,定部院直日例。

十一月壬戌,前吉林將軍秀林以盜用葠銀,賜死。

十二月丙申,廣西疏報壽民藍祥一百四十二歲,特賜禦制詩章、御書匾額、六品頂戴、銀五十兩。丁酉,馬慧裕奏雲梯關大工合龍,河歸正道入海。得旨嘉獎。己亥,以陳鳳翔為江南河道總督。壬寅,調興肇察哈爾都統。己酉,祫祭太廟。

是歲,免直隸七州縣災賦。除江蘇丹徒、上海坍田,安徽無為州廢田田賦。朝鮮、暹羅入貢。

十六年辛未春正月戊午,以雲梯關馬港新築長堤,增設淮海道,海安、海阜二同知。癸酉,以百齡為刑部尚書,松筠調兩廣總督,勒保為兩江總督。

二月壬午,上御經筵。丁亥,釋奠先師孔子。詔曰:「朕因連年南河河工糜費至四千餘萬,特命托津、初彭齡前往查察。茲據奏覆,查勘工帳銀款出入尚屬相符,而工程未盡堅固。此實歷任河臣之咎,吳璥、徐端俱降革有差。在工人員一並斥革。其未發銀六十萬,並著停發。」

三月丙寅,上謁西陵。壬午,謁陵禮成,西巡五臺山。乙亥,工部尚書費淳卒,贈大學士。以肅親王永錫為蒙古都統。

閏三月庚辰,上駐蹕五臺山。乙酉,上回鑾。丙申,上謁堯母陵、帝堯廟行禮。戊戌,上閱直隸綠營兵,幸蓮池書院,遣官祭明臣楊繼盛祠。癸卯,上還京。

夏四月戊申,大學士戴衢亨卒。甲子,上祈雨。致仕協辦大學士長麟卒。壬申,賜蔣立鏞等二百三十七人進士及第出身有差。以福慶為漢軍都統,崇祿為蒙古都統。

五月辛巳,以劉權之為大學士,鄒炳泰協辦大學士,劉鐶之兵部尚書。丁亥,上再詣天神壇祈雨。庚寅,雨。

六月壬午,明亮以覆奏不實,降副都統。以松筠為協辦大學士。癸丑,祿康以覆奏不實,降副都統。以勒保為大學士,管理吏部,吉綸為工部尚書、步軍統領。乙丑,湖南按察使傅鼐卒,贈巡撫,許建專祠。

秋七月戊寅,命光祿寺少卿盧廕溥入直軍機處,加四品卿銜。壬辰,禁西洋人潛居內地。丙申,上巡幸木蘭。癸丑,江南李家樓河決。乙巳,興肇以老免,起貢楚克扎布為察哈爾都統。

八月壬戌,上行圍。

九月己卯,建興安大嶺神祠,春秋致祀,戊子,上回鑾。乙未,以松筠為吏部尚書,蔣攸金舌為兩廣總督。丁酉,上謁陵。庚子,上還京。辛丑,四川十二支嶺夷向化,改土歸流。

十一月庚子,敕改運河邳、宿工程復歸河員管理。

十二月癸丑,以和寧為盛京將軍。癸酉,祫祭太廟。

是歲,免順天、江蘇、河南等省八州縣災賦。除甘肅逋賦,又除喀什噶爾回莊田賦。朝鮮、琉球、暹羅、緬甸入貢。

十七年壬申春正月壬午,時享太廟,命皇次子行禮。

二月甲辰朔,上御經筵。

三月丙子,上謁東陵。己丑,上幸南苑行圍。辛卯,以明亮為西安將軍。壬辰,上御晾鷹臺,大閱八旗官兵。丙申,上還京。

夏四月甲辰,詔曰:「八旗生齒日繁,亟宜廣籌生計。朕聞吉林土膏沃衍,地廣人稀。柳條邊外,葠場移遠,其間空曠之地,不下千有餘里,多屬腴壤,流民時有前往耕植。應援乾隆年間拉林成案,將閒散旗丁送往吉林,撥給地畝,或耕或佃,以資養贍。農暇仍可練習騎射,以備當差,教養兩得其益。該將軍等盡心籌畫,區分棲止,詳度以聞。」丙辰,上閱健銳營兵。癸亥,護軍統領扎克塔爾卒,予銀三百兩。

五月戊子,溫承惠奏灤州拏獲金丹、八卦邪教董懷信等。得旨:從嚴懲辦。

六月乙巳,移閑散宗室於盛京居住,築室給田給銀。

秋七月戊子,上巡幸木蘭。

八月壬子,陳鳳翔以不職免,以黎世序為江南河道總督。甲寅,以阮元為漕運總督。丙辰,上行圍。

九月戊子,上還京。甲午,慶桂以年老罷,以松筠為軍機大臣。

冬十月丁卯,以恭阿拉為禮部尚書。

十一月辛未,以景安為理籓院尚書兼漢軍都統。

十二月壬子,以鐵保為禮部尚書,潘世恩為工部尚書。甲寅,以興肇為江寧將軍。

是歲,免順天、奉天、直隸、河南、安徽等省二十七州縣災賦、逋賦、旗租,臺灣噶瑪蘭水沖田賦。朝鮮、暹羅入貢。

十八年癸酉春正月乙亥,軍機大臣松筠罷為御前大臣,以勒保為軍機大臣。

二月庚子,上御經筵。

三月丁丑,上幸南苑行圍。丙戌,上謁西陵。丙申,上還京。

夏四月己亥,以明亮為蒙古都統。甲寅,上祈雨。癸亥,以富俊為黑龍江將軍。

五月庚辰,上祈雨。壬辰,雨。

六月乙卯,賜進書生員鮑廷博舉人。庚申,以松筠為伊犁將軍。

秋七月甲戌,申嚴販運鴉片煙律,食者並罪之。丁丑,御史馮大中疏言中外臣工辦事遲延怠緩,請旨稽覈,上是之。壬午,上巡幸木蘭。

八月庚戌,上行圍。

九月甲子,上以陰雨減圍。癸酉,上回鑾。乙亥,河南睢州河溢。河南滑縣八卦教匪李文成糾眾謀逆,知縣強克捷捕系獄。其黨馮克善、牛亮臣陷縣城,克捷死之。直隸長垣、山東曹縣賊黨咸應。上命高杞、同興防堵,溫承惠佩欽差大臣關防剿之。召楊遇春統兵北上。賊黨徐安幗陷長垣,戕知縣趙綸。金鄉知縣吳階捕賊崔士俊等。戊寅,上行次棽髻山。是日,奸人陳爽數十人突入紫禁城,將逼內宮,皇次子用槍殪其一人。一賊登月華門墻,執旗指揮,皇次子再用槍擊之墜,貝勒綿志續殪其一。王大臣率健銳、火器營兵入,盡捕斬之。己卯,詔封皇次子為智親王,綿志郡王銜。論捕賊功,各予獎敘。奪吉綸職,以英和為步軍統領。庚辰,詔曰:「朕紹承大統,不敢暇逸,不敢為虐民之事。自川、楚教匪平後,方期與吾民共享承平之福,乃昨九月十五日,大內突有非常之事。漢、唐、宋、明之所未有,朕實恧焉。然變起一朝,禍積有素。當今大患,惟在因循怠玩。雖經再三誥誡,舌敝筆禿,終不足以動諸臣之聽,朕惟返躬修省耳。諸臣原為忠良,即盡心力,匡朕之咎,正民之志,切勿依前尸位,益增朕失。通諭知之。」命那彥成為欽差大臣,剿賊河南。以提督楊遇春、副都統富僧德、總兵楊芳帶兵協剿。辛巳,首逆林清就擒。壬午,上還京。癸未,以松筠、曹振鏞為大學士,托津、百齡協辦大學士,鐵保、章煦為吏部尚書。丙戌,首逆林清、通逆內監劉進亨等伏誅。

冬十月丙申,祖之望免,以韓崶為刑部尚書。癸卯,山東鹽運使劉清大破賊於扈家集,侍衛蘇爾慎復定陶、曹縣。御史張鵬展疏陳,百姓不敢出首邪匪,由於地方官規避處分,不為受理,或反坐誣。上是之。己酉,那彥成奏各路調兵,再行進剿。上嚴斥之。甲寅,命托津往督河南軍務,桂芳入直軍機處。丁巳,恤禁城拒賊傷亡侍衛那倫等世職。己未,祿康、裕瑞失察屬人從逆,發盛京禁錮。辛酉,謫降漢軍籍、直隸籍之科道官。壬戌,以明亮為兵部尚書。

十一月甲子朔,那彥成奏攻克道口賊巢,進圍滑城。丙寅,敕刪減公罪則例。壬申,通逆都司曹綸伏誅。戊子,那彥成奏楊芳等攻克司寨山賊寨,殲斃首犯李文成。

十二月丙申,命松筠、長齡籌議新疆經費。丙午,那彥成奏攻克滑城,賊渠宋元成等伏誅,生擒牛亮臣等。予那彥成三等男,楊遇春等以次獎敘有差。命托津留辦長垣賊匪。

是歲,免直隸、河南、湖南等省二十六州縣災賦。除江蘇、河南、湖南廢田田賦。朝鮮、琉球、越南、暹羅入貢。

十九年甲戌春正月壬午,以吳璥為河東河道總督。

二月甲午,上御經筵。乙未,以晉昌為盛京將軍。壬寅,成都將軍賽沖阿以剿陜西賊匪苗小一等,予三等男,長齡輕車都尉,楊遇春晉一等男。壬子,以富俊為吉林將軍,特依順保為黑龍江將軍。丙辰,鐵保免,以英和為吏部尚書,奕紹為漢軍都統。以戴均元為左都御史。

閏二月甲子,以和寧為禮部尚書。己丑,予死事滑縣知縣強克捷、教諭呂秉鈞、巡檢劉斌等世職。

四月乙亥,上閱健銳營兵。豫親王裕豐失察屬人祝現入教,謀逆已發覺,不入奏,削爵。以其弟裕興襲封。以興肇為漢軍都統。壬午,漕運總督桂芳卒。丙戌,賜龍汝言等二百二十六人進士及第出身有差。

五月癸亥,以和寧為熱河都統。

六月庚申朔,日有食之。庚辰,以劉鐶之為戶部尚書,初彭齡為兵部尚書。署江蘇巡撫。

八月甲子,上御經筵。辛未,大學士、威勤伯勒保再乞致仕,許之,命食伯俸。以托津為大學士,明亮協辦大學士。戊寅,上謁陵。甲申,上還京。

九月乙未,以景安為戶部尚書。

冬十月乙丑,以慶溥為左都御史。己巳,江西巡撫阮元以擒捕土匪,加太子少保。

十一月癸丑,命開墾伊犁、吉林荒地。

十二月癸未,百齡罷協辦大學士,以章煦為協辦大學士。乙酉,祫祭太廟。

是歲,免直隸二縣、河南二縣、黑龍江各城災賦。除奉天岫巖、浙江西安四縣廢田田賦。朝鮮、琉球入貢。

二十年乙亥春正月甲午,時享太廟,命智親王行禮。

二月己未,上御經筵。

三月庚寅,上謁東陵。戊申,上還京。甲午,初彭齡以參劾百齡不實,又代茅豫乞病,降官。旋經百齡查覆參奏,奪職。己酉,兩廣總督蔣攸金舌疏陳查禁鴉片★章程。得旨:「洋船到澳門時,按船查驗,杜絕來源。官吏賣放及民人私販者,分別治罪。」

夏四月己巳,上閱健銳營兵。壬午,上制官箴二十六章,宣示臣工。

五月丁亥,刑部疏,審明知府王樹勛即僧明心,矇混捐保職官。得旨:枷號兩個月,遣戍黑龍江。入教侍郎蔣予蒲褫職。

六月戊辰,上制勤政愛民論,宣示中外。己卯,常明奏中瞻對土番洛布七力滋事,改委總兵羅思舉由下瞻對前往剿辦。其剿辦不力之總兵羅聲皋及★★之都司圖棠阿均褫職逮問。

秋七月甲午,總兵羅思舉剿辦瞻對土番洛布七力竣事,下部議敘。癸卯,上巡幸木蘭。

八月戊辰,上行圍。百齡以捕獲編造逆詞首犯方榮升功,晉三等男。

九月己亥,上還京。

冬十月庚申,召松筠來京,以長齡為伊犁將軍。癸亥,命侍郎那彥寶往勘山西地震災。

十一月丁亥,禮親王昭梿以刑比佃丁欠租,削爵圈禁,以麟趾襲。

十二月己卯,祫祭太廟。

是歲,免直隸寧晉二縣災賦。除江蘇寶山、靖江,山西靜樂廢田田賦。會計天下民數三萬二千六百五十七萬四千八百九十五名口,存倉穀數三千八十萬二千八百六十九石九斗一升七合五勺。朝鮮、琉球、暹羅入貢。

二十一年丙子春正月丙戌,特詔諸親王、郡王勿令內監代為奏事,致開交結之端。

二月壬子,上御經筵。甲戌,上謁東陵。庚辰,上還京。

三月庚寅,上謁西陵。辛丑,上臨故大學士硃珪墓賜奠。丁未,上還宮。

夏四月丙子。張師誠以父疾具奏,不候旨即回籍,罷。以胡克家為江蘇巡撫。

五月辛卯,以馬慧裕為左都御史,孫玉庭為湖廣總督。丁未,以鄂勒哲依圖為御前大臣。

六月丁丑,休致大學士慶桂卒。戊寅,那彥成緣事褫職逮問,以方受疇為直隸總督。

閏六月戊戌,釋昭梿於禁所。壬寅,以戴均元為吏部尚書。

秋七月乙卯,和世泰、穆克登額、蘇楞額以帶領英吉利國使臣,不諳事體,不克入覲,俱黜降。以松筠為滿洲都統,和寧為工部尚書。乙丑,上巡幸木蘭。

八月壬辰,上行圍。九月戊午,上回鑾。閱古北口兵。壬戌,上還京。

冬十月戊子,命松筠署兩江總督,章煦為軍機大臣。

十一月壬子,百齡卒,調孫玉

庭為兩江總督,阮元為湖廣總督。丙辰,以綿志為領侍衛內大臣。

十二月癸卯,祫祭太廟。

是歲,免直隸、河南、浙江、湖南等省五十六州縣災賦有差。朝鮮、琉球、英吉利入貢。

二十二年丁丑春正月壬申,上御經筵。

二月丁丑,釋奠先師孔子。癸未,以長齡為陜甘總督,晉昌為伊犁將軍,富俊為盛京將軍。

三月甲辰朔,以董教增為閩浙總督。戊申,增設天津水師營總兵官,專轄水師兩營。壬子,上謁東陵。己巳,上還京。辛未,章煦免,以戴均元協辦大學士,盧廕溥為兵部尚書,汪廷珍為左都御史。

夏四月丁亥,上閱健銳營兵。庚寅,停伊犁仲夏進馬。辛卯,雲南夷匪平,加伯麟太子少保。戊戌,賜吳其濬等二百五十五人進士及第出身有差。

五月辛酉,上祈雨。壬戌,雨。以玉麟為駐藏大臣。丁卯,福建布政使李賡蕓被誣自縊,遣熙昌、王引之鞫其事,得實。奉旨:總督汪志伊、巡撫王紹蘭俱奪職。壬申,上制望雨省愆說。

六月甲戌,松筠疏請停止明年奉謁祖陵。奉旨嚴斥,罷大學士,黜為察哈爾都統。以明亮為大學士,伯麟協辦大學士,和寧為兵部尚書。以賽沖阿為御前大臣,德寧阿為成都將軍。

秋七月庚申,上巡幸木蘭。以蘇楞額為工部尚書,和世泰為理籓院尚書。

八月丁亥,上行圍。壬辰,積拉堪罷,以毓秀為杭州將軍。

九月癸丑,常明卒,以蔣攸金舌為四川總督,阮元為兩廣總督,慶保為湖廣總督。庚申,上還京。庚午,上制諫臣論,頒都察院。

冬十月辛未朔,日有食之。

十一月乙丑,以伊沖阿為熱河都統。

十二月甲戌,免雲南銅廠逋銀。丁酉,祫祭太廟。

是歲,免直隸八縣、黑龍江三城災賦。除奉天承德,直隸定州,江蘇丹徒、江陰,江西豐城,河南孟縣,福建侯官等縣水沖、河壓田賦。朝鮮、琉球、越南入貢。

二十三年戊寅春正月戊申,特詔松筠勿沽名市惠,以保桑榆。甲寅,詔明亮年逾八旬,宜節勞頤養,勿庸常川入直,並免帶領引見承旨。

二月庚午,命戴均元、和寧為軍機大臣。大學士董誥致仕,命食全俸。庚辰,上御經筵。己丑,上閱火器營兵。

三月庚子,上謁西陵。庚戌,以章煦為大學士,汪廷珍為禮部尚書,吳芳培為左都御史。戊午,上還京。

四月戊辰朔,日有食之。乙亥,風霾。丙子,詔曰:「昨日酉初三刻,暴風自東南來,塵霾四塞,燃燭始能辨色。其象甚異。朕心震懼惕,思上蒼示警之因,稽諸洪範咎徵,蒙恆風若之義,皆朕蒞事不明、用人不當之所致也。有言責者,體朕遇災而懼之心,剴切論列,無有所隱。即下民有冤抑者,亦可據事代為直陳,以副朕修德弭災之意。」給事中盧浙疏言,風沙示警,請禁員弁貪功妄捕,擾累平民。得旨:「所奏甚是。林清案內逸犯飭緝,承緝員弁輒以他犯塞責。番役兵丁,乘機肆虐,誣陷索手虜,無所不至。比到官審明,業已皮骨僅存,貲產蕩盡,甚有因而殞命者。冤苦莫訴,宜致斯災。所有次要五十餘犯,概令停緝。即祝現等六犯,亦只交刑部存記,獲日辦理。嗣後捕役有犯前情,該管官嚴刑重懲,以其家產付諸被誣之家,庶可儆惡習而安良懦。」己卯,欽天監疏言:「謹按天文正義,天地四方昏濛,若下塵雨,名曰霾。故曰天地霾,君臣乖,大旱,又主米貴。」得旨:「初八日之事,正與正義之象相同。惟朕恪遵成憲,日日召見臣工,前席周諮,似不致於乖離。但此其跡也,其實與朕同心望治,有幾人哉!不敢面諍,退有後言,貌合而情暌,是即乖也。其於同僚,不為君子之和而為小人之同,是亦乖也。我君臣其交儆焉。」庚辰,上祈雨。戊子,上再祈雨。辛卯,雨。

五月戊戌,詔曰:「館臣呈進敕修明鑒,於萬歷、天啟載入先朝開創之事,又加按語頌揚,於體例均為未合。副總裁侍郎秀寧降為侍衛,前往新疆換班。正總裁曹振鏞等各予薄罰,另行纂輯。」

六月壬申,武陟沁河溢,旋報合龍。

七月甲子,上東巡啟鑾。

八月丁卯朔,詔以取道民田,免經過奉天承德四州縣額賦。戊子,頒行皇朝通禮。壬午,上祭北鎮。辛卯,謁永陵,行大饗禮。

九月丙申朔,謁福陵。丁酉,謁昭陵,均行大饗禮,詣寶冊前行禮。上制再舉東巡慶成記。臨奠克勤郡王岳託、武勛王揚古利、弘毅公額亦都、直義公費英東墓。加恩額亦都後裔五人,費英東後裔一人。庚子,上詣天壇、堂子行禮。辛亥,上回鑾。丁巳,以富俊為吉林將軍,賽沖阿為盛京將軍。

冬十月庚午,上駐蹕興隆寺。辛未,萬壽節,行宮受賀。癸酉,上謁東陵。丙子,上還京。辛巳,予告大學士董誥卒,上臨第賜奠。

十一月戊申,以奕灝為蒙古都統。辛亥,詔曰:「國家臨御年久,宜加意於人心風俗。而人心之正,風俗之醇,則系於政教之得失。其間消息甚微,系於國ャ甚重,未可視為迂圖也。天下事有萬殊,理歸一是。從嚴、從寬,必準諸理。施行所及,乃能大畏民志。民志定,民心正矣。凡我君臣,當以憂盛危明之心,不為茍且便安之計。其於風俗之淳薄,尤當時時體察,潛移默化,整綱飭紀,正人心以正風俗。亮工熙績,莫重於斯。期與內外臣工交勉之。」

十二月戊辰,上祈雪。戊子,以八十六為廣州將軍,松筠為禮部尚書,以劉鐶之為左都御史。壬辰,祫祭太廟。

是歲,免順天、直隸、山東、安徽、甘肅、雲南等省七十九州縣災賦有差。朝鮮、琉球入貢。

二十四年己卯春正月甲午朔,上六旬萬壽,頒詔覃恩,賜廷臣宴。封皇三子綿愷為惇親王,皇四子綿忻為瑞親王,皇長孫奕緯為貝勒。晉封綿志、奕紹等有差。丁巳,和寧免直軍機,以侍郎文孚為軍機大臣。

二月甲子,上御經筵。

三月己亥,上謁東陵。壬子,上幸南苑行圍。己未,上謁西陵。

夏四月甲子,上還京。庚辰,上閱健銳營兵。丙戌,賜陳沆等二百二十四人進士及第出身有差。戊子,罷鳳陽、九江兩關監督,由巡道兼理。己巳,上祈雨。庚寅,以松筠為內大臣。

閏四月己酉,上詣天神壇祈雨。是日,雨。

五月乙酉,成親王永瑆以告祭禮愆,罷職削俸歸第。以英和、和世泰俱為滿洲都統。

六月癸卯,調松筠為工部尚書。

秋七月壬戌,以鄭親王烏爾恭阿為漢軍都統。庚申,上巡幸木蘭。壬午,永定河決,命吳璥、那彥寶勘築。

八月辛卯,河南蘭陽北岸河溢。予告大學士、威勤伯勒保卒,贈一等侯。

九月壬戌,上還京。癸酉,罷松筠御前大臣為盛京將軍。

冬十月乙未,萬壽節,上御太和殿受賀。侍郎周系英因參劾湖南客民焚殺,兼致私書,革職,並斥革其子舉人。

十一月乙巳,晉封明亮三等侯。

十二月庚子,吳邦慶以奏覆湖南客民焚殺案不實,降官。丙午,董教增疏請洋船準販茶葉,得旨斥駁。丙辰,祫祭太廟。

是歲,免直隸、浙江、湖南等省三十九州縣衛災賦旗租有差。除江蘇川沙、寶山縣廢地田賦。朝鮮、琉球、越南、暹羅、南掌入貢。

二十五年庚辰春正月壬申,詔優恤老臣明亮、和寧等,毋庸來園帶領引見。

二月己丑,上御經筵。癸卯,章煦以疾致仕,以戴均元為大學士,吳璥協辦大學士。戊申,上閱火器營兵。乙卯,慶郡王永璘有疾,上臨視,晉封親王。

三月甲子,上謁東陵。兵部遺失行印,事聞,明亮以次罰降有差。乙丑,上詣明成祖、宣宗、孝宗陵奠酒。己巳,慶親王永璘薨。戊寅,上還京。臨故慶親王第賜奠,命其子綿(敏心)襲郡王。

夏四月甲午,上詣八里莊慶僖親王殯所賜奠。庚戌,賜陳繼昌等二百四十六人進士及第出身有差。

六月癸卯,禁王公私設諳達及買民女為妾。松筠黜為驍騎校。

秋七月壬申,上巡幸木蘭。方受疇等疏呈嘉禾。戊寅,駐蹕避暑山莊。己卯,上不豫,鄉夕大漸。宣詔立皇次子智親王為皇太子。日加戌,上崩於行宮,年六十有一。

八月乙巳,奉移梓宮還京。十月甲辰,恭上尊謚曰受天興運敷化綏猷崇文經武孝恭勤儉端敏英哲睿皇帝,廟號仁宗。道光元年三月癸酉,葬昌陵。

論曰:仁宗初逢訓政,恭謹無違。迨躬蒞萬幾,鋤奸登善。削平逋寇,捕治海盜,力握要樞,崇儉勤事,闢地移民,皆為治之大原也。詔令數下,諄切求言。而籲咈之風,未遽睹焉,是可嘅已。


\end{pinyinscope}