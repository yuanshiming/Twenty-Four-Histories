\article{志四十二}

\begin{pinyinscope}
地理十四

△湖北

湖北:禹貢荊州之域。明置湖廣等處承宣布政使司。旋設湖廣巡撫及總督。清康熙三年,分置湖北布政司,始領府八:武昌,漢陽,黃州,安陸,德安,荊州,襄陽,鄖陽。並設湖北巡撫。雍正六年,升歸州為直隸州。十三年,升夷陵州為宜昌府,降歸州直隸州為州屬焉。以恩施縣治置施南府。乾隆五十六年,升荊門州為直隸州。光緒三十年,升鶴峰州為直隸。東至安徽宿松;五百五十里。南至湖南臨湘;四百里。西至四川巫山;千八百九十里。北至河南羅山。二百八十里。廣二千四百四十里,袤六百八十里。面積凡五十八萬九千一百一十六方里。北距京師三千一百五十五里。宣統三年,編戶五百五萬五千九十一,口二千三百九十一萬七千二百二十八。共領府十,直隸州一,直隸一,縣六十。驛道:自武昌西北渡江、漢達河南淅川;自襄陽西渡江達湖南灃州。電線:漢口東通九江,西通成都,南通長沙,北通鄭州。鐵路:京漢南段,粵漢北段。航路:自漢口以下,黃州、黃口港、蘄州、武穴,漢口以上,金口、寶塔州、新碮、城陵磯、沙巿、宜昌,皆江輪艤泊處。

武昌府:要,沖,難。隸鹽法武昌道。明為湖廣布政使司治。康熙三年為湖北布政司治。湖廣總督及湖北巡撫、布政使、按察使、督糧道駐。廣五百三十二里,袤四百七十二里。北極高三十度三十三分,京師偏西二度十四分。領州一,縣九。江夏要,沖,繁,難。倚。其名山:荊山、內方、大別。城內黃鵠山,亦名黃鶴山。與城北隅鳳凰山,俱置砲臺。東有洪山,一名東山,百戰地也。西臨大江,雄據東南上游,有險無蔽。大江自嘉魚,逕城西北入武昌,光緒中,沿江岸建紡紗、織布、繅絲、制麻各局。西南:金水,一曰塗水,自咸寧來,匯為斧頭湖,北至金口入江。有金口鎮巡司。南:南湖,通大江,今為軍屯重地。西南:占魚套、南山坡二巡司。東北:滸黃洲廢司。西:長江關。有將臺、東湖、山坡、土橋四驛。武昌,難。府東一百八十里。南:黃龍。西:樊山。東:石門。有長港,即樊港,納縣南諸湖水,逕樊口入江。大江自江夏來,東北流,逕縣北入大冶。西:蘆洲。東:安樂磯。西南:金牛鎮,縣東金子磯巡司移駐此。西:白湖鎮巡司,後移葛仙鎮。有華容驛。嘉魚簡,難。府西南一百五十里。城西魚嶽山。東南:陰山。東北:赤壁。南:白雲。大江自湖南臨湘入,右逕陸口入江夏。陸水自南來會,逕嘉魚口,有太平、岳公諸湖水流合焉。簰洲、石頭口二鎮,有巡司。有驛。蒲圻沖,難。府西南三百六十里。城內疊秀山。西:茅山。東:黃葛。西北:竹山、茗山。西南:羊樓洞。白鹿山,荊港水出焉。南:陸水,一曰蒲圻河。東:赤馬港,與荊港俱入焉。西南:新店。古大嶓水通黃蓋湖,下流至嘉魚石頭港入江。有港口巡司。羊樓洞廢司。港口、官塘二驛。鳳山廢驛。鐵路。咸寧沖。府東南二百四十里。東:東高山。西:鈷鉧。東南:相山。南:桃花尖山。塗水出其西桃花泉,曰白枵港,與西源浚水嶺減河會,又西北逕城南金鐙山,始為塗水,西北合諸湖港水,入江夏。其北麓東水出,下流入武昌之樊港。又東楊埠橋水,源出石川畈,流合東水,即東水別源也。西有咸寧驛。雍正六年徙入。崇陽繁,疲,難。府南三百六十里。西:巖頭。北:葛仙。東:雨山。西南:龍泉山。陸水自通城縣流入,曰崇陽河,右合梓木港,左合桂口港,至莎塘鋪,流逕花山。至城南,又東折,西北逕仰蓮山,又西北至壺頭山,為崇陽洪,入蒲圻。有桂口巡司。通城難。府西南五百里。九嶺。西南:白面。東南:幕府山,陸水所出,一曰俊水,納秀水入崇陽。南:黃龍山,新安港水出。東有鯉港,源出蓑荷洞,流合新安港,西入之。興國州繁。府東南三百八十里。南:闔閭。東:大坡。西:黃姑。北:大銀山。西南:龍山。東北:大江自大冶入,又東入江西瑞昌境。富水自西流入,謂之富池口。有富池鎮巡司。西南:龍港,北與富水合,移州東北黃顙口巡司駐之。有驛。大冶難。府東南一百五十里。順治二年自興國州改隸。東:圍爐。北:鐵山。西南:銅綠。北:白雉山。光緒間鑛政大興,鐵冶之利甲於全省。東北:磁湖山,產磁石。東:西塞山,下有道士洑,磯臨大江。舊設巡司,後移縣北。大江,西北自武昌入,為黃石港,東南流入興國州。有驛。通山難。府南一百八十里。順治二年自興國州改隸。南:九宮山。順治初,李自成為鄉人擊斃於此。東:沈水。西南:白羊,古青湓山,窩水出,亦曰通羊港,合湄港,自南流入。東南:黃梨山,寶石河出,合桐港,自西流入,東北至興國州合流,謂之富水。東有黃泥壟舊司,後裁。

漢陽府:沖,繁,疲,難。隸漢黃德道。順治初,沿明制,屬湖廣布政司。康熙三年,屬湖北布政司。東北距省治十里。廣二百七十里,袤四百七十里。北極高三十度三十三分。京師偏西二度二十一分。領州一,一,縣四。漢陽繁,疲,難。倚。西南:九真。東北:大別,即魯山,光緒間建鐵政局於山下。漢水自漢川入,逕北麓。大江自嘉魚來,環城而東合焉。西南:太白湖,接沔陽界,匯江、漢支流及諸湖澤,東洩於沌水。出江即沌口也。有沌口鎮巡司,後移下蒲潭。又西蔡店鎮,西南新灘口鎮,漢口鎮,仁義、禮智四巡司。光緒二十四年,移禮智司屬夏口。縣及蔡甸二驛。漢川沖,繁。府西北一百二十里。東南:小別,俗名甑山。西南:陽臺山,康熙間更名採芝。漢水自沔陽天門來,入漢陽。東:溳水自雲夢、應城入,淪河、西河、汊水皆注之,下流至夏口入漢,謂之溳口。有劉家堛、小里潭二巡司。縣及田兒河二驛。孝感沖,繁,疲,難。府北一百四十里。雍正七年自德安府來屬。北:黃茅嶺。東北:大悟,一名上界山。又北有澴水,自河南信陽州流入。南:淪河,即澴水下流也,上通溳水,東會灄水入江。太平、雙橋二鎮。縣丞駐東南馬溪河巡司,後移東北楊店驛。又北小河溪巡司,嘉慶十一年改駐灄口。有九里關,一名黃峴關,義陽三關之一也。縣及小河溪、楊店三驛。鐵路。黃陂沖,繁,難。府北一百二十里。雍正七年自黃州府來屬。東北:大陂山。東:魯臺山。宋二程夫子於此築臺望魯,因名。東南:大江自漢陽入,澴河注之,又東入黃岡界。縣河即灄水,南合淪河。又武湖即武口水也,承灄水分流,皆入焉。灄口、河口二鎮。大成潭、灄口鎮二巡司。縣及雙廟、灄口三驛。沔陽州繁,疲,難。府西南三百二十里。明屬承天府。順治間屬安陸。乾隆二十八年來屬,析州分置文泉縣。三十年省入。東南:黃蓬山、烏林磯。大江自監利來,入嘉魚境。漢水南派自天門來,入漢川境。又南長夏河,一曰夏水,江水支流也。又有襄水,為漢水支流,即沱潛也。自襄河澤口分流,逕監利縣入境,右合夏水,東匯於陽名、太白諸湖。西南:漕河,即玉帶河,西北通順、洛江、恩江等河俱自潛水分流入沔,今皆淤。州判駐仙桃鎮。鍋底灣、沙鎮二巡司。有驛。夏口沖,繁,疲,難。府治北。光緒二十四年析漢陽縣漢水以北地分置,治漢口鎮。自咸豐八年闢商埠,設江漢關。漢黃德道自黃州徙駐。西北:柏泉山。城東大江自漢陽來,至南岸嘴,合漢水,入黃岡界。漢水自漢川緣界會溳水,曰溳口。又東南來會,為漢口。古夏口,亦沔口,其故道襄河口。又東北入黃陂,為灄口。北有鐵路自大智門北經黃陂、孝感、應山等縣,與河南信陽州路接,為京漢鐵路南段。新溝市汛。禮智司巡司。

黃州府:沖,繁,難。隸漢黃德道。順治初,沿明制,屬湖廣布政司。康熙三年,屬湖北布政司。西北距省治一百八十里。廣六百六十五里,袤四百八十里。北極高三十度二十六分。京師偏西一度四十一分。領州一,縣七。黃岡沖,繁,難。倚。故城西北黃岡山。城東北大崎。北:淘金山。西:武磯。大江自黃陂來,入蘄水界。東巴水,西舉水,自麻城入,會道觀河、沙河南入之,謂之舉口,亦謂三江口。西北新生洲,與武昌白鹿磯相對。但店、團風、陽邏三鎮,與倉子埠四巡司。有齊安驛,李坪、陽邏廢驛。黃安簡。府西北三百二十里。西北:老山。東北:張家。東南:五雲。西南:似馬。北:雞公。紫潭河源出白沙關,合境內諸水南流,至黃岡入江。西:灄水源出北仙居山,下流入黃陂。東南有中和鎮、黃陂站二巡司。西北有金局關,一名金山關,相近有大城關,即麻城五關之一也。蘄水沖,繁,難。府東南一百十里。東:斗方。北:茶山。東北:張家。東南:仙女。西南濱江。大江自黃岡來,入蘄州,浠水、巴河自羅田來注之。有蘭溪鎮、巴河鎮二巡司。巴水、浠川二驛。羅田簡。府東一百六十里。南:望江。北:雞籠、桂家。東北:鹽堆山,巴水所出,一名平湖鄉河。有尤河合湯河、北峰河,逕城東至蘄水界,折西北,合石源河來會。東南有浠水,源出安徽英山,緣界右合樂秋、王家、觀音諸河,入蘄水。東北有多雲鎮巡司。又北有慄子關。東北有甕門關。西北又有平湖、同羅、松子等關。麻城繁,難。府東北一百八十里。西:大安。西北:羚羊。東南:白臬。舉水出縣境龜峰、黃蘗諸山,受閻家、柏塔、麻溪、白臬、浮橋諸河,下流至黃岡入江。又木樨二里河與東義州河,並南流,亦至黃岡入巴水。東北:殷山畈,上有陰山關,相近有虎頭關巡司。又北木陵山,上有木陵關,與黃土、虎頭、白沙暨黃安之大城,為麻城五關。又西鵝籠山巡司,一名鐵壁關,後移縣西南宋埠。同知駐岐亭鎮。有驛。蘄州沖,繁,難。府東一百八十里。東北:百家冶。西北:靈★。西南:空石。大江自蘄水來,入廣濟。蘄水源出大浮山,西南流,合三十六水及鈷鉧潭入赤東湖,至州西入江,謂之掛口,一曰蘄陽口。有茅山鎮、大同鎮二巡司。西河驛。又蘄陽水驛。廣濟沖,繁。府東二百五十里。明隸蘄州領屬。順治初改屬。東:大闔。東南:太平。西南:積布,即古高山也。大江自蘄州入,東南流,入黃梅。東有梅川,下流入午山湖。西南有馬口湖。通江有馬口巡司,後移武穴鎮。東南龍坪鎮巡司。西南田家鎮,水利同知駐。鎮對半壁山,束江流最狹處,咸豐中置砲臺。有廣濟、雙城二驛。黃梅沖,繁。府東三百五十里。明隸蘄州領屬。順治初改屬。西北:黃梅山,縣以此名。東南:礦山。東北:馮茂。西南:蔡山。大江在南,自廣濟入,東南流,逕清江鎮,入宿松。縣河在縣東,即隆斗河,及縣西雙城河,會諸湖港水,至黃連嘴合流,出急水溝入之。東北縣丞駐清江廢鎮。新開口、亭前、孔壟三鎮巡司,均有驛。

安陸府:沖,繁。隸安襄鄖荊道。明,承天府,屬湖廣布政司。順治三年更名。康熙三年,屬湖北布政司。東南距省治五百三十五里。廣五百二十里,袤七百四十五里。北極高三十一度十二分。京師偏西三度五十九分。領縣四。鍾祥繁,疲,難。倚。城南:樠木山。東北:純德。北:九華。西北:馬鞍。西:漢水自宜城來,入荊門州。又東聊屈山,臼水出,即左傳所謂成臼。東北:黃仙洞山,敖水出,下流曰直河。西南權水、北豐樂河,皆入於漢。豐樂有驛,設巡司。又麗陽驛,乾隆三十二年自荊門州割隸,移仙居口巡司於此。又石城、郢東二驛。石牌鎮,縣丞駐。京山繁,難。府東一百五十里。東:京山。西北:大洪。南:子陵。西南:寶香。西:漢水自鍾祥入,逕丁口潭,又東南緣潛江界入之。溾水俗名回河,逕城南,皆匯縣境諸水入之。中源曰縣河,南流入天門界。又富水一曰撞河,源出大洪山,東南會小富水,為雙河口,合石板河,入應城。又聊屈山,臼水出,古成臼也,與澨水小河會,曰南河,東南入天門。其東楊水、巾水並從之。有宋河鎮巡司,乾隆二十九年自荊門州新城移此。有驛。潛江難。府南二百二十里。漢水自京山來,逕縣北,分入天門、沔陽界。東有潛水,一名蘆洑河,自漢水別出,南有沱水,自江陵流入,在縣西沱埠淵合流,為江、漢會通故道,後淤。東南縣河、班灣河、沙口河,皆潛水下流,亦淤。西:夜汊河,上承漢水,舊由大澤口分流,亦謂策口。咸豐時改由吳灘潰口,即吳家改口。西南高家場巡司,有驛。天門沖,繁,難。府東南二百二十里。明為景陵縣,隸沔陽州,屬承天府。順治三年直屬今府。雍正四年更名。西北有天門山,漢水北派自潛江西南逕縣南,下流合南派,入漢川界。又澨水自京山流入,合楊水、巾水,曰三汊河,一曰汊水。禹貢「過三澨」,即此。至城西分二派,合於城東,北通楊桑湖,東通三臺湖,至漢川注于松湖,分流入漢。南嶽口市,縣丞駐。乾灘鎮巡司。有驛。

德安府:沖。隸漢黃德道。順治初,沿明制,屬湖廣布政司。康熙三年,屬湖北布政司。東南距省治三百二十里。廣三百八十里,袤三百八十里。北極高三十一度十八分。京師偏西二度五十五分。領州一,縣四。安陸沖。倚。東:章山,即豫章山。西:太平。西北:壽山。溳水亦曰府河,即清發水,左傳「吳敗楚於柏舉,從之及於清發」是也。自隨州應山流入,會洑水、瀖水、石河水,至兩河口,與楊家河合。南高竅鎮,有廢巡司。西北漴陽鎮。有驛。雲夢沖,難。府東南六十里。溳水自安陸來,東南流,入漢川界。北岸有溳河堤,康熙五年重築,其支津由白河口南分流而東,為縣河,會鄭家河,入孝感界,通澴河。東興安、南隔蒲潭、北利塘三鎮。興安有廢巡司。有驛。應城難。府南八十里。東南:高樓山。東臨溳水。西北有西河,即富水也。富水自京山入,又南,左納省港,至桂口,右歧為小河,注三臺、五當,納五龍河。又東南,金梁湖為金盆,入漢川。右潼水出縣西北潼山,自縣南又東,逕安陸入溳,湮。東長江埠,巡司自崎山鎮移此。有驛。隨州疲,難。府西北一百三十里。北:厲山,一名烈山。西南:大洪山,溳水出焉。西北:溠水,源出栲栳山,南流注之。又左受水厥水、溧水,右受支水、浪水,下流至夏口入漢。西南有章水,東南流,經安陸、應城縣界入溳,亦曰楊家河。祝林鎮,州同駐。唐縣鎮,州判駐。環潭,梅丘巡司駐。高城鎮,總巡司駐。又有合河店巡司、唐縣鎮巡司,嘉慶十五年裁。應山沖,繁。府北九十里。左:孔山。西:洞庭。東北:黃茅。西北:瞞箭山,漻水出,西南入隨達溳。二水又東南緣界合徐家河,入安陸。東:黃沙河,亦曰環河,出縣東北雞頭山,有東河會簸箕港水流合焉,南入孝感。西北有三里店巡司,雍正十年自平里市鎮移,後遷平靖關,俗名恨這關,即古之冥★也。又禮山關即武勝關,一名武陽關,京漢鐵路所經之義陽三關,此其二也。廣水、馬平港、龍泉、太平四鎮。縣城、平靖關、觀音店、廣水鎮四驛。

荊州府:沖,繁,疲,難。隸荊宜道。將軍,左右翼副都統均駐。順治初,沿明制,屬湖廣布政司。康熙三年,屬湖北布政司。東距省治八百里。廣七百二十五里,袤二百十里。北極高三十度二十六分。京師偏西四度二十八分。領縣七。江陵沖,繁,疲,難。倚。西北:龍山。北:紀山。大江自松滋西來,逕城南,入公安界。沮水自當陽縣合漳水南來注之。西南:虎渡河,自大江分流,下注澧水,入洞庭湖,即禹貢「導江東至於澧」也。東南:夏水,即沱江,為大江支津。又有湧水,則夏水支流也,通江處謂之湧口。漕河在城東北,名草市河,經沙市,名沙市河。又東瓦子湖,一名長湖,匯諸湖水,下流俱達於沔。萬城堤在縣西南,雍正中築,乾隆五十三年修,歲遣大臣駐防。沙市,通判駐,有巡司。光緒二十一年開為商埠。與龍灣市、虎渡口巡司三。郝穴口有廢司。荊南、丫角廟二驛。公安沖,繁。府西南一百四十里。順治八年,由斗湖堤徙祝家岡。同治十二年復徙唐家岡。東:太歲。東南:黃山。大江在北,自江陵東入石首界。西:油水,舊由油河口入江,今淤。虎渡河自江陵縣南流入境,至黃金口,分一支為東河,合吳達河諸水達薦祖溪。正流南經港口,會孫黃河,東南流,至泗水口,均入湖南注澧水。東北:孱陵鎮巡司。東有塗郭市、東南有孟家溪市。有孫黃驛,後裁。石首簡。府東南一百八十里。東:石首山。東南:石門。西:陽岐山。大江自公安入,逕縣北,入監利界。其支津由藕池口分數道南達洞庭湖。又東焦山河,亦其支流也,自調弦口經焦山,亦達洞庭。南:黃金堤市。西:楊林市。監利繁,疲,難。府東少南二百四十里。東南:獅子山。西:大江自石首來,逕縣西南,入湖南華容。東:魯洑江,即夏水也,自江陵流入。東為大馬長川,周環二百餘里,與林長、分鹽、龍潭、三汊等河均至沔陽州合於沔水。白螺磯、分鹽所、窯圻鎮三巡司。硃家河廢司。松滋簡。府西一百二十里。東:竺園。南:金羊。西:九岡。西南:巴山。大江自枝江入,逕縣北,亦曰川江。岷江至此分為三派,下流復合為一,達於江陵。洈水源出西南之起龍山,即古洈山,逕樟木山,右合隔沙河,左天木河,逕文公山,又東曰紙廠河,入公安達洞庭湖。東南磨盤洲巡司,紅崖子砦巡司,後廢。有縣城、涴市二驛。枝江簡。府西一百八十里。南有紫山。西:金紫。西南:官木。大江自宜都來,逕縣北,入松滋。江中有百里洲。南為外江,北為內江,即江與沱也。東北沮水,又西北白水港,合群溪水注之。有江口巡司。董市鎮。宜都簡。府西北一百八十里。順治初屬夷陵州。雍正十三年改屬。南:羊腸。西南:大梁。東北:石羊。西北:荊門山,對岸即虎牙山。大江自東湖來,逕其間,為絕險處,東南流入枝江。西北清江,即夷水也,自長陽流入,東南流,會漢洋河,至清江口入江。東北滄茫溪,一名瑪瑙河,亦入江。白洋在江北岸,順治初僑治於此,尋復故。東北:普通關廢司。西南:聶家河市。北:安福市。又虎腦背市,即古猇亭。

襄陽府:沖,繁,難。隸安襄鄖荊道。順治初,沿明制,屬湖廣布政司。康熙三年,屬湖北布政司。東南距省治六百八十里。廣六百七十里,袤二百七十里。北極高三十二度五分。京師偏西四度二十分。領州一,縣六。襄陽沖,繁,難。倚。東南:鹿門。西南:虎頭。南:峴山。西:隆中山。漢水自穀城來,逕城北,入宜城。城四周有堤,謂之襄陽城堤。對岸即樊城,古重鎮也。東北淯水,自河南唐縣入,名唐河。合濁水,名唐白河。縣別有白水,自東來會。又西北清泥河,東淳河,皆入漢。同知、丞駐樊城。呂堰、雙溝二巡司。油房廢司。漢江、呂堰二驛。漢江今移城中。宜城沖。府東南一百二十里。西:石梁。東南:赤山。南:太山。漢水自襄陽會潼水入,逕縣東,入鍾祥。西南蠻水,一曰鄢水,又曰夷水,合瀰水,與其支津木里溝、長渠皆入漢。又沶水自漢中來,合於蠻水,謂之沶口。北疏水,亦名襄水,土人呼涑水,亦自★口入漢。東南樓子汊、南康坡汊、北羊祜汊,皆漢水之旁出也。西:田家集巡司。南:鄢城驛。南漳簡。府西南一百二十里。西南有八疊山,一名柤山,又名沮山,吳硃然、諸葛瑾北出沮中,即此。西有荊山,左氏傳所云「荊山九州之險」是也。漳水出焉,下流至當陽會沮水入江。其北深溪河,蠻水入,曰榨洛河,逕大鴻山,至城南,入宜城界。有方家堰巡司,後移保安鎮。南有雞頭關,東北有石河鋪。棗陽沖,繁,難。府東北一百四十里。東:霸山。東南:資山。南:瀴源山,瀴水所出。東:大阜山,白水所出。又東南昆水,西南濜水,合白水下流入于淯水,至襄陽入漢。西南蔡水,西流亦入淯水。有湖河、鹿頭、雙河、太平諸鎮。穀城簡。府西北一百四十里。湖北提督駐。西北:穀城山,縣以此名。西南:薤山。南:金斗。東北:漢水自光化入,亦曰穀水。南:築水,一名南河,東入沔,謂之築口。北:汎水,一名古羊河,或曰北河,至城東,與築水合。有花石街、張家集二巡司。張家集後移駐太平店。光化簡。府西北一百八十里。西北:三夫山。漢水自均州入,逕城西,有涓水流入,歷上涓、中涓、下涓三口入穀城。又黑水、排子、硃寨諸河下流皆入焉。東南舊有茨湖,今湮。有左旗營巡司,後徙縣南老河口。均州簡。府西北三百九十里。南:武當,一曰太和,亦曰篸上山,明時尊為「太岳」,浪河、曾水並出焉。漢水自鄖遠河口入,又東為禹貢滄浪之水。其由浪河入者,有殷家河、蕭河,其由曾水入者,有黃沙、小芝、水磨、篤河。又均水自州南流,至光化之小江口,亦入之。有草店、浪河諸鎮。光緒四年置孫家灣巡司。

鄖陽府:繁,疲,難。隸安襄鄖荊道。總兵駐。順治初,沿明制,屬湖廣布政司,並設撫治、都御史。康熙三年,屬湖北省布政司。六年,罷撫治。東南距省治一千二百五十里。廣七百十里,袤四百里。北極高三十二度四十九分。京師偏西五度四十二分。領縣六。鄖難。倚。北:兜鍪。西:錫穴。西北:老砦。西南:白馬。漢水自鄖西入均州。堵水自縣南流入焉,謂之堵口。又將軍河、曲遠河、神定河、龍門河、遠河俱來入。滔河自陜西商州流入,經縣東北,會丹水,入河南淅川。西:黃龍鎮巡司。雷峰堊鎮、青桐關二巡司,裁。有驛。房簡。府西南三百一十里。西南:房山。南:景山,一名雁塞山,沮水出焉。又東,汛水,今名八渡河。北:築水,源出楊子山。東北有粉水,俱流逕保康入穀城注漢。有三岔口、九道梁二巡司。竹山難。府西南三百六十里。東南:方城,又名望楚山。西南:白馬塞山。西:丫角山。南有堵水,一名陡河,源出陜西平利,自竹谿東流入境。右會官渡河、章落河、霍河,左受苦桃河、上元水、嶔峪河、對峙河,又東北流,經房北,至鄖入漢。同知駐白河堡。官渡河堡巡司。黃茅關、吉陽關二廢司。竹谿簡。府西南五百九十里。東:誥軸。西南:峒崎山,有砦最險隘。西北:竹谿河,流合縣河,為堵水上源。南秦坪河,東南白沙河,會柿河注堵水。有尹店社、白土關二廢巡司。東:縣河鎮。保康簡。府東南三百四十里。東:岮峪山。西有湯浹河,一名湯洋河,水溫可療疾。西北:粉水,東流與築水會,名曰南河。西南板倉河,北來注之。東南有常平堡廢巡司。鄖西簡。府西北一百三十里。順治十六年,以西北上津縣省入。西:礦山。西北:十八盤山。南有漢水,緣界合仙河、白河,又東逕金蘭山,甲河自山陽來會,入鄖西。天河源出縣西北牛頭山,激浪河、麥峪河流入焉。西北:上津堡廢巡司。西:江口鎮。

宜昌府:沖。隸荊宜道。總兵官駐。順治初,沿明制,為夷陵州,屬荊州府。雍正十三年升為府,更名,屬湖北布政司,置東湖為治。鶴峰、長樂,降歸州及所屬長陽、興山、巴東來隸。光緒三十年,析荊宜施道為施鶴道,升鶴峰為隸之。東距省治一千八十里。廣五百九十里,袤四百十里。北極高三十度四十九分。京師偏西五度十五分。領州一,縣五。東湖沖,繁,難。倚。舊為夷陵縣。明省入夷陵州。雍正十三年復置,更名。光緒二年闢城南為宜昌商埠。東:對馬。北:豐寶。南:高笄。東北:方山。西北:黃牛峽,亦稱黃牛山。北:西陵峽,一名夷山,古所謂「三峽」之二。大江自歸州來經之,至縣西,始出峽就平地,東入宜都。東南:虎牙山,對岸為宜都之荊門山,下臨虎牙灘。更有流頭、使君、鹿角、狼尾等灘,皆奇險。北:黃柏河,下流為長橋溪,由長橋入江。西北:南沱巡司。又南津、西津、白虎諸關。有驛。歸州簡。府西北三百五里。明屬荊州府。雍正六年升為直隸州。十三年復降為州來屬。大江自巴東來,東入東湖界。香溪源出興山縣南入之,曰香溪口。江中有新灘、叱灘及石碣、達洞、獨石諸灘,又有馬肝、白狗、空舲三峽,皆險處也。有南邏口、牛口巡司。州城及建坪二驛。長陽簡。府西南七十六里。明隸夷陵州,屬荊州府。雍正六年屬直隸歸州,十三年來屬。北:宜陽。西北:佷山。西:資丘。清江自巴東入,逕武落鍾離山,一名難留城山,五姓蠻所從出也。清江俗名長陽河,合招徠河,又東逕金紫山,合平樂河、丹水,逕城南,又東入宜都。西有舊關堡、蹇家園二廢巡司。有資丘鎮。古捍關。興山簡。府北三百十里。明隸歸州,屬荊州府。康熙中,直屬荊州府。雍正六年屬直隸歸州。十三年來屬。西北:神龍、茅麓。北:羅鏡。東:仙侶。西:萬朝。城南香溪一名縣前河,建陽、南陽兩河入之,合白沙、九沖河,至城南,始為香溪。又南合大里溪,至峽門口,會大峽水。又西南入歸州。有關口堊、青林堊,貓兒關諸隘。又有■D1葉塢,出鄖、襄間道也。西北:高雞寨廢巡司。巴東沖,難。府西四百二十五里。明隸歸州,屬荊州府。康熙中,直屬荊州府。雍正六年屬直隸歸州,十三年來屬。東:鐵峰。北:青銅。南:巴山,一名金字山。西南:安居。大江自四川巫山來,由巫峽流入,逕城北,出東湖西陵峽,下流至黃梅,入安徽宿松縣界。三灞河源出縣西北九府坪,支流三,其一入西瀼溪,東曰東瀼溪,逕城北,又東逕牛口山。西南:清江,自建始入,下流入歸。野山關巡司,後移駐縣南勸農亭。縣城、火峰口二驛。長樂簡。府南一百九十一里。明為五峰司,隸容美宣撫司,屬施州衛。雍正十三年置縣,以石梁、水濜、長茅三司,及長陽、松滋、枝江、宜都,與湖南石門等縣邊地益之,來屬。西北:金雞。南:壺坪。西:五峰山。長茅河經縣北,會縣河入清江。東:漢洋河,源出東北山中,東經百年關北、漁陽關南,下流至宜都,亦入清江。南:白溪河,即渫水之上源。西:南灣潭,縣丞駐焉。

施南府:簡,難。隸施鶴道。明,施州衛,屬湖慶都司。康熙中,因明制,為施州衛,屬荊州府。雍正六年,改為恩施縣,屬直隸歸州。十三年升為府,更名,屬湖北布政司,增宣恩、來鳳、咸豐、利川。乾隆元年,割四川夔州、建始來隸。東距省一千九百八十里。廣四百二十八里,袤四百九十四里。北極高三十度十六分。京師偏西七度二分。領縣六。恩施繁,難。倚。明為施州衛。雍正六年置縣,更名。十三年建府治,遂屬焉。以原屬支羅等地分入他縣。西北:都亭。東北:捍山。東:連珠,一名五峰山,下有五峰關。北:清江,源出四川石龍關東諸山,一曰夷水,又曰鹽水。後漢書「廩君乘土船從夷水至鹽陽」,即此。經縣東,有忠建河及麒麟、金印諸溪水注之,下流入於大江。崔家壩巡司。木貢,縣丞駐。宣恩簡。府東南八十里。明為施南宣撫司,屬施州衛。康熙時為施南土司。雍正六年屬恩施縣。十三年置縣屬府。以忠峒、高羅、木冊、東鄉、忠建、石虎七司地益之。北:墨達山。南:將軍山,白水河出焉,一曰車溪,又曰酉溪,入來鳳,下流謂之漫水。漊水出縣東北鶯嘴荒,一曰九谿河,澧之北源也。忠建河在城東,名玉帶溪,自咸豐發源,北入清江。有獅子、東門二關,乾灞巡司。東有東鄉鎮巡司,後裁。來鳳簡。府南二百七十里。明,散毛宣撫司,屬施州衛。康熙時為散毛土司。雍正六年屬恩施縣。十三年置縣,治司屬之桐子園,隸以蠟壁、大旺、東流、卯峒、漫水五司地,屬府。東南:翔鳳。西北:三尖。西:佛山,與雀兒峰對峙,高峒河水出焉。東有佛塘崖,下有佛塘河,即宣恩之白水河也,流合眾川,逕峒東流,入於辰河。有卯峒巡司,滴水、老鴉二關。咸豐簡。府西南二百二十五里。明為大田軍民千戶所,屬施州衛。康熙時改設巡司。雍正六年屬恩施縣。十三年置縣,以唐崖、龍潭、金峒三司地益之。城內角樓山。東:小關。西北:龍潭河,一曰唐崖河,自利川入,西南入四川黔江,謂之黔水。西:龍嘴河,亦自利川入,逕萬頃湖南入彭水。西南:張家坪巡司。利川簡。府西一百七十八里。明,施南司屬之官渡灞粗石地。雍正十三年置縣,以忠路、忠孝、沙溪三司及恩施屬之支羅、南坪堡等處益之。東:金字。西:桂子、七藥山,前江出焉,南與後江合流,謂之龍嘴河,即中清河上源也。北有清江。境內水多伏流。有南坪、建南二巡司。建始簡。府東北一百二十里。初因明制,屬四川夔州府。乾隆元年改屬。南:文山。西:石乳。東:州基山。西:石乳山,上有關,馬水河出,西南右合桐木溪、木瓜河,逕祿山,右會廣潤河,至撒毛,入恩施。南有清江,入逕麻根塔口、景陽河,又東入巴東。有龍駒河。大巖嶺鎮,縣丞駐。

荊門直隸州:沖,繁,疲,難。隸安襄鄖荊道。明屬承天府。順治初,沿明制,為安陸府屬州。乾隆五十六年,升直隸州,屬湖北布政司。東南距省治六百里。廣二百六十里,袤三百二十五里。北極高三十一度四分。京師偏西四度十六分。領縣二。南三十里荊門山。西北:武陵。東:伯夷。西:象山。東南:章山,即內方山。漢水來逕城東,亦曰沔水,東南入潛江。濱漢為堤,亦自內方山達潛江,為五邑保障。西權水,北象河,東南有直江,下流均入漢。又建水一名建陽河,上流古通漢,今淤,下流至江陵匯為湖。有建陽鎮、石橋鎮二巡司,俱有驛。又城有荊山驛。舊設新城鎮、仙居口二巡司,及荊門所、宜門所,後均廢。東南有沙洋鎮。當陽簡。州西南一百二十里。明屬州,隸承天府。順治初,屬安陸府。乾隆五十六年還屬。東南:紫蓋。東北:綠林。玉泉山,玉泉水出焉。北:沮水,自遠安來,東南流,合鞏河、玉泉水,至麥城南,與漳水會,下流入江。有河溶鎮巡司。北:百寶砦。東:淯溪鎮。有驛。遠安簡。州西一百四十里。明屬夷陵州,隸荊州府。雍正十三年直屬府。乾隆五十六年來屬。西北:鳴鳳。北:神馬。沮水自南漳來,逕縣東南流,合福河溪、通天樓河、石洋河、白龍溪、泥水溪、青溪諸水,入當陽界。西北:黃柏河。北:南襄堡,西北:砦洋坪汛。

鶴峰直隸:沖,繁,疲,難。光緒三十年,析荊宜施道為施鶴道,隸之。順治初,因明制,為容美土司,屬施州衛。雍正六年,屬恩施縣。十三年置州,以五星坪、北佳坪益之,屬宜昌府。光緒三十年,升直隸,屬湖北布政司。東距省治一千五百五十里。廣一百九十五里,袤三百四十五里。北極高三十度。京師偏西六度三十分。東南:柘雞。東:平山。北:印山。南:天星。西北:巴子山。南:八峰山。有山河,即漊水之上源,東南流,大典河入焉。東北:咸盈河,逕巴東入於清江。有奇峰、鄔陽、大崖諸關。山羊隘舊屬湖南慈利,雍正時來屬,設巡司駐之,後移白果坪。


\end{pinyinscope}