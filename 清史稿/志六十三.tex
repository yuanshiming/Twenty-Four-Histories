\article{志六十三}

\begin{pinyinscope}
禮七(嘉禮一)

登極儀授受儀太后垂簾儀親政儀大朝儀常朝儀御門聽政附

太上皇帝三大節朝賀儀太皇太后皇太后皇后三大節朝賀儀

大宴儀上尊號徽號儀尊封太妃太嬪儀附冊立中宮儀冊紀嬪儀附

冊皇太子儀太子千秋節附冊諸王儀冊公主附

二曰嘉禮。屬於天子者,曰朝會、燕饗、冊命、經筵諸典。行於庶人者,曰鄉飲酒禮。而婚嫁之禮,則上與下同也。周官「以嘉禮親萬民」,體國經野,罔不繇此。茲舉其大者,附以儀之同者,著於篇。

登極儀清初太祖創業,建元天命,正月朔即位,貝勒、群臣集殿前,按翼序立。皇帝御殿,皆跪。八大臣出班,跪進上尊號表,侍臣受,跪御前宣讀。帝降座,焚香告天,率貝勒、群臣行禮,三跪九叩,畢,復座,貝勒等各率旗屬慶賀。太宗踐阼亦如之。

天聰十年,改元崇德,建國號曰大清。前期誓戒三日,築壇,備鹵簿。屆日,帝率群臣詣天壇祗告。禮成,奉御寶官先行,帝自中階登壇升座,貝勒等三跪九叩。畢,眾跪,貝勒分左右列。奉寶官跪獻,帝受寶,轉授內院官,群臣行禮如初。畢,皆跪,宣讀官奉滿、蒙、漢三體表文立壇東,以次畢讀,群臣行禮訖,復位,奏樂,駕還宮。翼日帝御殿,群臣表賀,三跪九叩,次執事官行禮如前儀。於是賜宴,頒赦詔。八年,世祖嗣服,遣官告壇、廟如初禮,唯不設鹵簿,不作樂,不賜宴。

順治元年十月朔,定鼎燕京,先期太常官除壇壝,司禮監設座案。屆日,遣官告廟、社,備大駕鹵簿,帝御祭服,出大清門,詣南郊,告天地。禮成,導入天壇東幄次易禮服。御座,群臣跪,禮部尚書引大學士一人升自東階,正中北面跪,學士一人自案上奉寶授大學士,祗受,致辭云:「皇帝君臨萬國,諸王文武群臣不勝歡忭。」訖,轉授學士,學士跪受,陳於案,復位。群臣禮畢,駕還宮。鴻臚寺官設御案皇極門中,簷東設表案,王、貝勒等序立內金水橋北,文武官序立橋南,俱東西鄉。樂作,帝御座則止。鳴鞭。執事官階上行禮畢,就位。王率群臣進表,行禮畢,鳴鞭,駕還宮。越九日甲子,頒詔如制。

聖祖纘業,分遣官祭告天地、宗社,帝衰服詣幾筵行三跪九叩禮,祗告受命。御側殿易禮服,詣太皇太后、皇太后兩宮,各行三跪九叩禮。遂乘輿出乾清門,御中和殿,內大臣等執事官行禮。復御太和殿,王公百官上表行禮如儀。不宣讀,不作樂,不設宴。王公入,賜茶畢,還宮。反喪服,就苫次,頒詔。世宗承大統,一如前儀,惟罷賜茶。高宗以後,儲宮嗣立者並同。

授受儀古內禪儀。初高宗享國日久,嘗諭年至八十六歲即歸政。逮乾隆六十年,詔曰:「自古帝王內禪,非其時怠荒,即其時多故,倉猝授受,禮無可採。今國家全盛,其詳議典禮以聞。」於是諏吉定儲位,以明年為嗣皇帝元年。禮臣上儀注。先期遣官祭告廟、社,屆日所司設御座太和殿。左右幾二,正中寶案,稍南東西肆;東楹詔案,西楹表案,南北肆;黃案居丹陛中。檻內敷嗣皇帝拜褥。殿前陳鹵簿,門外步輦。午門外五輅、馴象、仗馬、黃蓋、雲盤,簷下設中和韶樂,門外丹陛大樂。內閣學士奉傳位詔陳東案,禮部官陳賀表西案,大學士等詣乾清門請寶陳左幾,大學士二人分立兩簷下,王公百官序立。朝鮮、安南、暹羅、廓爾喀使臣列班末。欽天監官詣乾清門報時,嗣皇帝朝服出毓慶宮,時後扈內大臣二人率侍衛二十人集乾清門外,導引禮部長官二人立門階下,前引大臣十人立殿後階下。太上皇帝禮服乘輿出,嗣皇帝從諸臣前引後扈。午門鳴鐘鼓,至殿後降輿。太上皇帝御中和殿升座,嗣皇帝殿內西鄉立,鴻臚寺官引執事大臣按班,不贊,行九叩禮。侍班者趨出,就外朝位,中和韶樂作,奏元平章。太上皇帝御太和殿,嗣皇帝侍立如初。樂止,階下鳴鞭三,丹陛大樂作,奏慶平章。嗣皇帝詣拜位立,王公立丹陛上,百官及陪臣立丹墀下,鳴贊官贊「跪」,嗣皇帝率群臣跪。贊「宣表」,宣表官入,奉表至簷下正中跪,大學士二人左右跪,展表,樂止。宣訖,還奉原案,退。贊「興」,嗣皇帝退立左旁,西鄉,大學士二人導近御前跪。左大學士請寶,跪奉太上皇帝,太上皇帝親授嗣皇帝,嗣皇帝跪受,右大學士跪接,陳右幾。嗣皇帝詣拜位,樂作,贊「跪,叩,興」,率群臣行九叩禮。贊「退」,樂止,禮成。鳴鞭如初。中和韶樂作,奏和平章。太上皇帝還宮。內監豫設樂懸,太上皇帝御內殿,公主,福晉,暨皇孫、皇曾元孫未錫爵者,行禮慶賀。

嗣皇帝易禮服,祗俟保和殿暖閣,內閣學士豫奉傳位詔及御寶陳太和殿中案,禮部官奉登極賀表陳東案,扈引者集保和殿外。欽天監報時,嗣皇帝御中和殿,執事者按班行禮,不贊。禮畢,嗣皇帝御太和殿登極。作樂,止樂,宣表,行禮,悉準前式。禮畢,退,復位。大學士進,奉詔,出中門,授禮部尚書。尚書跪受,興,奉置黃案,行三叩禮。復奉詔陳雲盤,儀制司一人跪受,興,自中道出。禮成,俱退,嗣皇帝還宮。大學士等詣乾清門送寶,禮部恭鐫詔書頒行。

垂簾儀咸豐十一年,文宗崩,穆宗幼沖嗣位。御史董元醇奏請皇太后暫權朝政,稱旨,命王大臣等議垂簾儀制。議上,懿旨猶謂「垂簾非所樂為,唯以時事多艱,王大臣等不能無所稟承,姑允所請」云。於是仲冬月朔,帝奉兩宮皇太后御養心殿聽政,王公大臣集殿門外,行禮如儀。凡召見內外臣工,兩宮皇太后、皇帝同御養心殿,太后前垂簾。或召某臣進見,議政王、御前大臣番頒之。引見外官,則御養心殿前殿,議政王、御前大臣率侍衛等按班分立,太后前垂簾設案,進各員銜名,豫擬諭旨,分別錄注。皇帝前設案,各長官依例進綠頭簽,議政王等奉陳案上,引見如常儀。皇太后簡單內某名鈐印,已,授王大臣傳旨。其臣工請安摺,並具三分以進。各省、各路軍事摺報,凡應降諭旨者,議政王等請旨繕擬後,次日呈閱頒行。唯撰擬文句,仍本帝意,宣示臣工,宜書曰「朕」。

同治十三年,德宗入繼文宗,王公大臣復請兩宮皇太后垂簾,悉準同治初成式。光緒六年,慈安皇太后薨,慈禧皇太后始專垂簾,制十三年歸政,德宗以時艱尚棘,凡召見、引見,仍升座訓政,設紗屏以障焉。

親政儀同治十二年正月,兩宮皇太后歸政,穆宗行親政典禮,先期遣告天、地、廟、社,屆日陳皇太后儀駕、皇帝法駕鹵簿,設表案慈寧宮門,檻內敷皇帝拜褥,太和殿內東旁設詔案,東次表案,丹陛中案各一。午門外設龍亭、香亭,內閣學士奉皇帝慶賀表文納諸櫝,捧出。大學士從至永康左門外,大學士接櫝,至慈寧門,升東階,陳案上,退。內侍舉案入,庋慈寧宮寶座東,內閣學士奉詔陳殿中黃案,禮部官奉王公百官賀表陳東次黃案。凡將軍、提、鎮賀表置龍亭內。鴻臚寺官引和碩親王以下,入八分公以上暨蒙古王公等集隆宗門外,不入八分公以下二品大臣以上集長信門外,三品以下集午門外。欽天監報時,帝御禮服乘輿出隆宗門,至永康左門外降,王以下隨行,至慈寧門,帝升東階,及門左,西鄉立。日講官四人在西階,東鄉立。前引大臣率侍衛在儀駕末,分左右立。皇太后出御慈寧宮,中和樂作,奏豫平章,升座,樂止。帝就拜位,丹陛樂作,奏益平章。王公大臣侍衛等循次鄉上立,贊「拜跪」,帝率群臣三跪九拜。時西楹下置御史二,鳴贊官二。儀駕末及午門外御史、禮部官、鳴贊官各二,藉以侍儀。永康左門及諸門內外並置鳴贊官,接續外傳。午門外各官隨同行禮,鳴贊官贊「禮成」,帝復位。王大臣各復位立,皇太后還宮,禮部尚書奏「禮成」,然後帝還宮。俄復出御中和殿,執事官行禮畢,趨出就外朝立,帝御太和殿,樂作,升座,樂止,鳴鞭三,王公百官行禮。其宣表、頒詔並如前制。光緒十三年德宗親政仿此。

大朝儀天命元年,始行元旦慶賀,制朝儀。天聰六年,行新定朝儀,此班朝所繇始,崇德改元,定元旦進表箋及聖節慶賀儀。順治八年,定元旦、冬至、萬壽聖節為三大節。康熙八年,定正朝會樂章,三大節並設。大朝行禮致慶,王以下各官、外籓王子、使臣咸列班次,所司陳鹵簿、樂懸如制。太和殿東具黃案。質明,王、貝勒、貝子集太和門,不入八分公以下官集午門外。禮部奉表置亭內,校尉舁行至午門外陳兩旁,奉表入太和殿列案上。鴻臚卿引王、貝勒等立丹陛。鳴贊官引群臣暨進表官入兩掖門,序立丹墀。朝鮮、蒙古諸臣自西掖門入,立西班末。糾儀御史立西簷下東鄉者二人,丹陛、丹墀東西相鄉者各四人,東西班末八人,鳴贊官立殿簷者四人,陛、墀皆如之。丹陛南階三級,鑾儀衛官六人司鳴鞭。欽天監報時,皇帝出御中和殿,執事官行禮畢,趨外朝視事。駕出,前導、後扈如儀。午門鳴鐘鼓,中和樂作,御太和殿,樂止。內大臣分立前後,侍衛又次其後護守之。起居注官四人立西旁金柱後,大學士,學士,講、讀學士,正、少詹事立東簷下。御史、副僉都御史立西簷下,鑾儀衛官贊「鳴鞭」,鳴贊官贊「排班」,王公百官就拜位立跪。宣表官奉表出,至殿下正中北鄉跪,大學士二人展表,宣表官宣訖,置原案,丹陛樂作,群臣皆三跪九叩。退,就立原次。鴻臚寺官引朝鮮等使臣,理籓院官引蒙古使臣就拜次,三跪九叩,丹陛樂作,禮畢,樂止,退立如初。賜坐,群臣暨外臣皆就立處一跪三叩,序坐。賜茶畢,復鳴鞭三,中和樂作,駕還宮。樂止,群臣退。

初制,外官元日朝覲,集保和殿前行禮,康熙二十六年後罷。乾隆六年,定行在聖節朝賀行禮。二十四年,定大朝百官班次,設立紅漆木牌。五十四年,增置都察院長官二人,科、道三十六人,分立品級山旁整朝序。又高宗初年,文三品、武二品以上賜茶,餘惟記注官、外國使臣與焉。嘉慶二年罷賜茶。令甲,元旦、萬壽節午時設宴,冬至節次日受賀。萬壽節先謁太廟,次詣皇太后宮行禮,畢,受賀。直省文武官值三大節,俱設香案,朝服望闕行禮,滿、蒙、漢軍分兩翼,漢官分文東武西。

常朝儀太祖丙辰建元後,益勤國政,五日一視朝,焚香告天,宣讀古來嘉言懿行及成敗興廢所由,訓誡臣民,然未垂為定制也。崇德初,始定儀注,設大駕鹵簿,王以下各官朝服,俟帝出宮,樂作。御殿,升座,樂止。賜坐,諸臣各依班次,一叩就座。部、院官出班奏事畢,駕還宮。順治九年,給事中魏象樞言:「故事有朔、望朝,有早朝、晚朝、內朝、外朝,今縱不能如往制,請一月三朝,以副厲精圖治至意。」楊簧亦言:「舊例百官每月十一朝,似太繁數,今每日入朝奏事,較十一朝不為少,應定每月初五、十五、二十五日行朝參禮。」自是遂定逢五視朝制。尋定見朝、辭朝、謝恩各官,俱常朝日行禮。帝御太和殿,引見畢,賜坐賜茶,悉準常儀。如是日不御殿,各官行禮午門外。外籓來朝暨貢使,亦常朝日行禮,如速返,則不拘朝期,即赴午門行禮,外官應速赴任者亦然。

又定常朝御殿,王公入殿中旁坐如次。康熙八年,定公、侯、伯以下各官為六班,按次列坐,後復改為九班。九年,諭都察院糾察王大臣失儀。二十年,置常朝糾儀御史及司員。雍正二年,遣侍衛四人監察朝班,定視朝日天未明,鴻臚寺官二人引左右翼官入西掖門依班坐。鼓嚴,起立聽贊,自仗南引進,整齊班列,行禮如儀。乾隆初,敕大小各官依內廷官例,黎明坐班。十六年,諭部院大臣董率庶僚,常朝按期赴班,毋曠闕。

光緒九年,更定朝制,凡新除授各官,鴻臚寺列銜名交內閣,屆日禮部尚書、鴻臚卿請駕御殿,導各官謝恩行禮,王公百官侍鹵簿後。不御殿,文武官則坐班午門外。其時刻,春冬以辰正,夏秋以卯正,遇雨雪及國忌則免。坐班日,鴻臚寺官按翼定位,王公集太和門外,東西各二班,百官集午門外,東西各九班,糾儀御史吏、禮司員各四人,分列班首末,並西面北上。屆時吏、禮司員受職名,糾儀官環班稽察,復位坐。有間,以次出。

御門聽政儀,清初定制,每日聽政,必御正門,九卿科道齊集啟奏,率以為常。雍正初,始定御門典禮,凡部院所進本有未經奉旨者,摺本下內閣,積若干,傳旨某日御門辦事。是日,乾清門正中設御榻、黼扆、本案一。黎明,部院奏事大臣暨陪奏官屬畢集庭內。帝升座,侍衛左右立,記注官升西階,部院官升東階,各就列跪,尚書前,侍郎後,陪奏官又後。尚書一人奉本匣折旋而進,詣本案前,跪陳於案,興,少退,趨東楹,轉入班首。跪,口奏某事,畢,興,少退,率屬循階左降。其奏事次序,戶、禮、兵、工四部輪班首上,三法司直第三班,吏部直第六班,宗人府則列部院前,翰詹科道及九卿會奏則居部院後,各依班進奏如初。至吏部奏事,兼帶領各部番直司員八人,引見畢,始退。內閣侍讀學士二人升東階,詣案前跪,舉本匣,興,退。翰詹科道暨侍衛俱退。時欽派讀本滿學士一人,奉摺本匣升東階,折旋而退,大學士從,依班次跪。記注官少進東鄉立,奉匣學士詣案前跪啟匣,取摺本依次啟奏,帝降旨宣答。大學士等承旨訖,興,自東階降,記注官自西階降。駕還宮。奏事時,令翰林官記注,自順治二年始。

先是奏事春夏以卯正,秋冬以辰初。康熙二十一年,命展御門晷刻,春夏改辰初,秋冬辰正。越二年,御史衛執蒲請以五日或二三日為期,聖祖諭:「政治務在精勤,始終不宜有間。」二十五年,置科道各二人侍班,列起居注官上。二十七年,省起居注官,其侍班翰林,令啟奏摺本時即退。雍正初,復設起居注官,增二人。又令編檢四人侍班,列科道上。乾隆二年,命修撰、編、檢依科道例,懸數珠,肅朝儀。嘉慶十八年,諭宣本承旨時,御前大臣及侍衛毋退,著為令。

太上皇帝三大節朝賀儀嘉慶元年,高宗傳位仁宗,尊為太上皇帝,定朝賀儀。屆日陳法駕、鹵簿、樂懸如授受儀,太和殿設三案,表亭舁至午門,慶賀表文陳東案,筆硯陳西案。質明,王公百官朝服,外國使臣服本國服,集闕下。皇帝禮服,俟保和殿暖閣。太上皇帝乘輿出,至太和殿北階降,中和韶樂作,奏元平章,御殿升座,樂止。帝殿內西鄉立,鳴鞭三,贊「排班」,丹陛大樂作,奏慶平章。帝就拜位,北鄉,時鴻臚官分引群臣暨外使肅班立,贊「進」,贊「跪,叩,興」。帝率群臣行三跪九叩禮。畢,帝旋位立,眾退,復班次,樂止。鳴鞭,中和韶樂作,奏和平章。太上皇帝還宮,樂止。帝御殿,群臣進表行禮如儀。

太皇太后、皇太后、皇后三大節朝賀儀順治八年,定元旦慈寧宮階下設皇太后儀杖、樂器,皇太后御宮,樂作。升座,樂止。帝率內大臣、侍衛詣宮行三跪九叩禮。畢,公主、福晉以下,都統、子、尚書命婦以上,行六肅三跪三叩禮。作樂如初,大設筵宴。冬至、聖壽節同,唯冬至罷宴。康熙八年,定元日太皇太后、皇太后儀駕、中和韶樂、丹陛大樂全設。帝率王公大臣、侍衛暨都統、子、尚書以上官,先朝太皇太后宮,次詣皇太后宮,行禮如儀。畢,皇后率公主、福晉、命婦行禮亦如之。二十一年,諭京、外進表官集午門外行禮。尋置糾儀御史,分列宮門外、午門外儀駕末,嚴監視。

乾隆十二年,定慶賀皇太后許二品命婦入班,尋諭世爵朝賀增入男爵。嘉慶二十五年,諭值皇太后三大節,將軍、督、撫、提、鎮具表慶賀,罷遞黃摺祝文。道光元年元旦,大學士先進皇帝慶賀表文,帝始率群臣詣宮行禮。同治元年,皇太后、皇帝同御慈寧宮受賀,明年,改御養心殿。王、公、二品以上官,集慈寧門外,三品以下集午門外,朝鮮使臣列西班末,按班行禮,不贊。冬至、聖壽節同。唯遇大慶年,俟皇太后升殿後,增用宣表例。光緒二年,皇太后聖壽,皇帝親進表文,餘儀同。

皇後向無受群臣賀儀,順治間,定元旦慶賀,儀仗全設。皇后詣皇太后宮行禮畢,還宮,自公主及命婦俱詣皇后宮朝賀。冬至、千秋節同。康熙時,定皇后先詣太皇太后宮,次皇太后宮行禮,還宮升座,自公主迄鎮國將軍夫人,公、侯迄尚書命婦,咸朝服行禮。雍正六年,始令皇后千秋節王公百官咸蟒袍補服,後準此行。攝六宮事皇貴妃千秋節,儀同皇后。

大宴儀凡國家例宴,禮部主辦,光祿寺供置,精膳司部署之。建元定鼎宴,崇德初,太宗改元建號,設宴篤恭殿。順治元年,定鼎燕京,設筵宴、設寶座皇極門正中,帝升座,賜百官坐,賜茶、進酒,俱一跪一叩。宴畢謝恩如初禮。是日賜宴,有內監數輩先行拜舞,諭:「朝賀大典,內監不得沿明制入班行禮。」裁抑宦官自此舉始。

元日宴,崇德初,定制,設宴崇政殿,王、貝勒、貝子、公等各進筵食牲酒,外籓王、貝勒亦如之。順治十年,令親王、世子、郡王暨外籓王、貝勒各進牲酒,不足,光祿寺益之,禦筵則尚膳監供備。康熙十三年罷,越數歲復故。二十三年,改燔炙為肴羹,去銀器,王以下進肴羹筵席有差。

雍正四年,定元旦宴儀,是日巳刻,內外王、公、臺吉等朝服集太和門,文武各官集午門。設御筵寶座前,內大臣、內務府大臣、禮部、理籓院長官視設席。丹陛上張黃幔,陳金器其下,鹵簿後張青幔,設諸席。鴻臚寺官引百官入,理籓院官引外籓王公入。帝御太和殿,升座,中和韶樂作,王大臣就殿內,文三品、武二品以上官就丹陛上,餘就青幔下,俱一叩,坐。賜茶,丹陛大樂作,王以下就坐次跪,復一叩。帝飲茶畢,侍衛授王大臣茶,光祿官授群臣茶,復就坐次一叩。飲畢,又一叩,樂止。展席冪,掌儀司官分執壺、爵、金卮,大樂作,群臣起。掌儀司官舉壺實酒於爵,進爵大臣趨跪,則皆跪。掌儀司官授大臣爵,大臣升自中陛,至御前跪進酒。興,自右陛降,復位,一叩,群臣皆叩。大臣興,復自右陛升,跪受爵,復位,跪。掌儀司官受虛爵退,舉卮實酒,承旨賜進爵大臣酒。王以下起立,掌儀司官立授卮,大臣跪受爵,一叩,飲畢,俟受爵者退,復一叩,興,就坐位,群臣皆坐。樂止,帝進饌。中和清樂作,分給各筵食品,酒各一卮,如授茶儀。樂止,蒙古樂歌進。畢,滿舞大臣進,滿舞上壽。對舞更進,樂歌和之。瓦爾喀氏舞起,蒙古樂歌和之,隊舞更進。每退俱一叩。雜戲畢陳。訖,群臣三叩。大樂作,鳴鞭,韶樂作,駕還宮。

冬至宴,順治間制定如元旦儀,後往往停罷。元會宴,凡元正朝會,歲有常經,遇萬壽正慶,或十年國慶,特行宴禮。乾隆三十五年、五十五年,聖制元會作歌,宴儀如前。惟行酒後,慶隆舞進,司章歌作,司舞飾面具,乘禺馬,進揚烈舞。司弦箏阮節抃者,以次奏技。喜起舞,大臣入,行三叩禮,循歌聲按隊起舞,歌闋,笳吹進,番部合奏進,內府官引朝鮮俳,回部、金川番童陳百戲,為稍異耳。

千秋宴,為康熙五十二年創典,設暢春園。凡直省現官、致仕漢員暨士庶等,年六十五以上至九十者咸與。遣子孫、宗室執爵授飲,分給食品,諭毋起立,以示優崇。乾隆五十年,設宴乾清宮,自王、公訖內、外文、武大臣暨致仕大臣、官員、紳士、兵卒、耆農、工商與夫外籓王、公、臺吉,回部、番部土官、土舍,朝鮮陪臣,齒逾六十者,凡三千餘人。其大臣七十以上,餘九十以上者,子孫得扶掖入宴。年最高者,如百五歲司業銜郭鍾岳等,得隨一品大臣同趨黼座,親與賜觴。宴罷,頒賞珍物有差。嘉慶初元再舉,設宴皇極殿,與宴者三千五十六人,邀賞者五千人。上自榑槐,下逮袀襏,以至蒙、回、番部、朝鮮、安南、暹羅、廓爾喀陪價,略其年甲,咸集丹墀,誠盛典也。

大婚宴,順治八年,大婚禮成,設宴如元旦儀。並進皇太后筵席牲酒,嗣後仿此。

耕耤宴,順治十一年舉行,命曰「勞酒」。

凱旋宴,自崇德七年始。順治十三年定制,凡凱旋陛見獲賜宴。乾隆中,定金川,宴瀛臺;定回部,宴豐澤園;及平兩金川,錫宴紫光閣。其時所俘番童有習鍋莊及甲斯魯者,番神儺戲,亦命陳宴次,後以為常。道光八年,回疆奠定,錫宴正大光明殿,是日大將奉觴上壽,帝親賜酒,命侍衛頒從征大臣酒,餘如常儀。

宗室宴,乾隆十一年,設宴瀛臺,賜宗室王公,遵旨長幼列坐,行家人禮,並引至淑清院流杯亭游覽,賜酒果。四十八年,設宴乾清宮,命皇子、王、公等暨三、四品頂戴宗室千三百有八人入宴。其因事未與宴者咸與賞,都凡二千人。嘉慶九年,設筵惇敘殿,略同瀛臺宴。

外籓宴,歲除日設保和殿,賜蒙古王、公等,凡就位、進茶、饌爵、行酒、樂舞、謝恩,並如元會儀。其來朝進貢,送親入覲,或御賜恩宴,或宴禮部,取旨供備。至諸國朝貢,如朝鮮、安南、琉球、荷蘭遣使來京,亦有例宴。乾隆間,緬甸使臣陪宴萬樹園,以其國樂器五種合奏。厥後凡遇筵宴,備陳準部、回部、安南、緬甸、廓爾喀樂。

又順治中,定制鄉試宴順天府,會試及進士傳臚宴禮部。餘如臨雍、經筵、修書、初舉日講、臨幸翰林院、繕寫神牌,亦賜宴如例。衍聖公、正一真人來朝,纂實錄、會典皆於禮部設宴雲。

上尊號徽號儀清初太祖、太宗建元,群臣皆上尊號,其禮即登極儀也。康熙中,臣民合辭擬上尊號。至六旬聖壽,復籥請。聖祖諭言無裨治道,皆不允行。迄高宗敉定邊陲,王大臣猶以上尊號請,亦未俞納。惟新君踐阼,奉母後為皇太后、皇太后為太皇太后,則上尊號。國家行大慶,則上徽號,或二字、或四字,遞進以致推崇。

順治八年,上孝莊皇后尊號,其徽號曰「昭聖慈壽」。先期祭告,帝躬上奏書。屆期太和殿陳皇帝法駕,慈寧宮陳皇太后儀駕,供設咸備。王公集太和門,大臣集右翼門,各官集午門,分翼立。帝升殿,中和韶樂作,奏海上蟠桃章,帝閱冊、寶畢,執事官分置亭內,鑾儀校舁行,前冊亭,後寶亭。帝率群臣從駕至慈寧門,入宮立陛東,禮部侍郎、內閣學士奉冊、寶入,大學士奉宣讀冊、寶文入,侍立左旁,帝就拜位,王公百官依班位序立。皇太后御宮,中和韶樂作,奏豫平章,升座,樂止。贊「跪」,帝率群臣跪。奏「進冊」,大學士右旁跪進,興,退,帝受冊,恭獻,大學士左旁跪接,興,陳中案。奏「進寶」,如前儀。贊「宣冊」,宣冊官至案前北面跪,啟函宣讀訖,仍納之,興,退。贊「宣寶」同,仍置原案。女官四人舉案陳宮階上。丹陛大樂作,奏益平章,帝率群臣三跪九叩。午門外各官承傳隨班行禮。禮成,皇太后起座,中和韶樂作,奏履平章,還宮。皇后率六宮、公主以下詣宮慶賀。翼日,帝御太和殿,王公百官上表慶賀,頒詔如制。是歲大婚禮成,加上徽號禮亦如之。

康熙初元,加上徽號,時以諒陰,不奏書,不行禮,不朝賀。凡大婚、親政、冊立皇后、武功告成、皇太后大慶、上徽號並如常儀。

乾隆四十一年,金川平,上徽號,皇太后諭帝春秋高,不宜過勞,令豫陳冊寶,至時行禮,罷宣讀表文,後仿此。

道光九年,平回疆,上皇太后徽號,緬甸國王遣使進金葉賀表,緬王進表自此始。

尊封太妃進冊寶如前儀,唯內監舉案陳太妃座前,帝行禮,太妃起避立座旁。次日御殿受賀同。若遣官將事,禮部尚書朝服詣內閣,冊寶舁出,偕大學士送之,至宮門外,內監入獻太妃、太嬪,受訖,禮成。冊寶初制用金,康、乾時兼用嘉玉,道光後專以玉為之。凡尊封皇貴妃、貴太嬪,並用冊寶,太妃用冊印,太嬪用冊。

冊立中宮儀崇德初元,孝端文皇后以嫡妃正位中宮,始行冊立禮。是日設黃幄清寧宮前,幄內陳黃案,其東冊寶案。王公百官集崇政殿,皇帝御殿閱冊寶。正、副使二人持節,執事官舉冊寶至黃幄前,皇后出迎。使者奉冊寶陳案上,西鄉立,宣讀冊文,具滿、蒙、漢三體,以次授右女官,女官跪接獻皇后,後以次跪受,轉授左女官,亦跪接,陳黃案。次宣寶、受寶亦如之。使者出,復命,皇后率公主、福晉、命婦至崇政殿御前六肅三跪三叩。畢,還宮升座,妃率公主等行禮,王公百官上表慶賀,賜宴如常儀。

康熙十六年,冊立孝昭仁皇后,前期補行納採、大徵如大婚禮。親詣奉先殿告祭,天地、太廟後殿則遣官祭告。至日設節案太和殿中,東西肆;左右各設案一,南北肆。帝御殿閱冊寶,王公百官序立,正、副使立丹陛上,北鄉,宣制官立殿中門左。宣制曰:「某年月日,冊立妃某氏為皇后,命卿等持節行禮。」於是正、副使持節前行,校尉舁冊寶亭出協和門,至景運門,以冊寶節授內監,奉至宮門,皇后迎受。行禮畢,內監出,還節使者,使者復命,帝率群臣詣太皇太后、皇太后宮行禮。翼日,皇后禮服詣兩宮及帝座前行禮。

乾隆二年,冊立孝賢純皇后,如常儀。命頒詔,著為家法。

嘉慶元年,立孝淑睿皇后,冊命日,會太上皇帝千秋宴訖還宮,帝、後詣前行禮。帝御殿,正、副使持節,禮成,先詣太上皇宮門前復命,餘如常儀。

冊封妃、嬪,亦自崇德初元始,四妃同日受封,屆時命使持節冊封如禮。妃等率公主、福晉、命婦詣帝前六肅三跪三叩,後前亦如之,妃前則行四肅二跪二叩,妃等相對各二肅一跪一叩。康熙時,貴妃、七嬪與中宮同日封,諸嬪有冊無寶。乾隆十三年,定皇妃攝六宮事,體制宜崇,祭告如冊中宮儀。次日朝皇太后,拜跪甬路左旁。道光三年,諭嗣後封嬪罷祭告,即與妃同日受封亦然,著為令。

冊立皇太子儀康熙十四年,立嫡子允礽為皇太子,先期祭告,玉帛香版,皆皇帝躬視。屆日御殿傳制,與冊立中宮同。正使授冊,副使授寶。行禮畢,正、副使復命。帝率皇太子祭告奉先殿,皇太子拜褥敷檻外,並詣帝、後宮行禮。翼日,帝御殿受賀、頒詔如常儀。王公進箋皇太子前致慶,皇太子詣武英殿與親、郡王等行禮。外省文武官並箋賀如儀。

遇太子千秋節,太子先詣奉先殿致祭,隨詣皇帝前行禮,還毓慶宮,旋御惇本殿受賀。王公百官二跪六叩,畢,還宮,群臣退。

厥後允礽廢立,迄晚年儲位未定。五十年後,大學士王掞七上密疏,請建國本,六十年,復申前請,觸聖怒。至乾、嘉後,始明宣不立儲貳諭旨,開國固未嘗有也。

冊封諸王儀崇德元年,定冊封日,王、貝勒序立崇政殿前,內院官奉制冊、印陳於案,俟旨授封。諸王等皆跪,宣冊官、奉冊官並立案東,次第宣畢,奉冊、印授諸王等。王等祗受,轉授從官,復位。禮畢,隨奉冊官赴清寧宮,詣帝、後前行禮,三跪九叩。遂出大清門,諸王等互賀,俱二跪六叩。還邸,福晉、夫人各行慶賀。府僚致賀諸王,二跪六叩,貝勒僚屬一跪三叩。

康熙十二年定制,凡冊封,簡正、副使二人,前一日,殿堂上設節案,香案,冊寶案,堂前儀衛、樂懸備陳。屆期,正、副使詣太和殿奉節出,校尉舁冊寶亭赴王府,王率府僚跪迎門外。正、副使奉冊寶節分陳各案,立節案東,王立案西。行禮畢,王詣香案前跪,聽宣制冊,使者授冊寶,王祗受,復位,行禮如初。使者奉節復命,王率府僚跪送,迎送俱用樂。封親王曰寶,郡王曰印,貝勒有制冊無印。行禮謝恩並同。初制,封親王世子用金冊,郡王鍍金銀冊,貝勒授誥命,旋改用紙制冊。咸豐十年,諭冊封親王用銀質鍍金,以恭親王奕王爵世襲,仍制金冊。

冊封公主,封使至,公主率侍女迎儀門右,使者奉制冊入,陳門前黃案上,移置堂前幄內。公主升西階,六肅三跪三叩,宣訖,授侍女,公主跪受,行禮如初。使者復命,仍送儀門外。是日帝升殿,公主至御前,次入後宮,並六肅三跪三叩。又次詣諸妃前,各四肅二跪二叩,還府,府屬慶賀,餘如封親王儀。凡固倫公主、和碩公主,同輩者封長公主,長者封大長公主,並給金冊云。


\end{pinyinscope}