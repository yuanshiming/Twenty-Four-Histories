\article{列傳六十}

\begin{pinyinscope}
李率泰趙廷臣袁懋功徐旭齡郎廷佐弟廷相郎永清永清子廷極

佟鳳彩麻勒吉阿席熙瑪祜施維翰

李率泰,字壽疇,漢軍正藍旗人,永芳子。初名延齡,年十二,入侍太祖,賜今名。年十六,以宗室女妻之。弱冠,從太宗徵察哈爾、朝鮮及明錦州,又從貝勒阿巴泰征山東,並有功,洊擢梅勒額真。

順治元年,命以刑部參政兼任,率師駐防錦州。四月,從睿親王多爾袞入關,破李自成;又率兵徇山東、河南,斬自成將趙應元,降其眾萬人。二年,從豫親王多鐸破自成兵潼關。移師南征,克揚州,下江寧,分兵定蘇州、松江諸郡。江陰典史閻應元拒守,督兵攻破之。豫親王令駐防蘇州。會明將吳志葵、黃蜚等來犯,時城兵僅千餘,率泰使繞城張幟為援兵狀。志葵等斬關入,勁騎突起截擊,盡殲之。

三年,從端重親王博洛平浙江、福建,敘功,授世職二等阿達哈哈番兼拖沙喇哈番。五年,鄭彩犯福建漳、泉諸郡,詔率泰與靖南將軍陳泰協剿,斬獲甚眾。復長樂、連江二縣。彩走,復擒斬所署總督顧世臣等,遂克興化。寇攻福州十四月,圍始解。民食盡,江西盜郭天才自杉關長驅至福州,載米麥江上,誘民出就食。率泰師次建寧,檄守吏嚴備,乃夜焚洪山橋遁。巡按御史周世科虐刑婪賄,率泰疏劾,置諸法。六年,從征大同叛將姜瓖,下保德州,擒瓖黨牛化麟等。敘功,復加拖沙喇哈番。

初定官制,改參政為侍郎,率泰仍以刑部侍郎兼梅勒額真。八年,調吏部,拜弘文院大學士。條奏請懲貪酷官吏,給滿洲兵馬草料,酌量營造工程次第,上從之。未幾,與大學士陳泰坐誤增恩詔赦款,並罷任,降世職為拜他喇布勒哈番。九年,特進三等阿思哈尼哈番。

十年,用大學士洪承疇薦,授兩廣總督。時明桂王硃由榔居安隆,其將李定國擁兵廣西,土寇廖篤增等應之。十一年,率泰遣兵進剿,斬篤增於玉版巢。十二年,定國犯廣東,率泰禦之,敗其將高文貴。會靖南將軍珠瑪喇率禁旅至,合兵夾擊,大破之。復高、雷二郡。

十三年,加太子太保,調閩浙總督。率泰有方略,善用兵,與士卒同甘苦。時鄭成功據臺灣,數入寇。率泰疏請增設水師三千,造哨船百餘艘,招降海盜,散其羽翼。又言成功父芝龍不宜徙寧古塔,其地近海,恐乘間遁歸,為患滋大。世祖悉用其言。以破定國功,進世職一等。考滿,加少保。十五年,招撫成功將唐邦傑、林翀、葉祿等,降者數萬人。十五年,成功攻溫州,陷平陽、瑞安,率泰調江寧滿洲兵助剿,成功敗走。是年,詔分閩浙總督為二:以都統趙國祚督浙江,駐溫州;而以率泰專督福建,駐福州。未幾,成功據南安嶺窺福州,其黨陳斌既降復叛,率眾據羅星塔。率泰檄兵燔其巨艦千餘,成功遁。斌復降,奏誅之。十六年,坐事奪世職,任總督如故。

康熙元年,率泰以漳州為福建門戶,奏增設水師二千。尋與靖南王耿繼茂擊走定海小埕諸寇,復與提督馬得功平萬安所,擊走成功將楊宣。是年成功死,其子錦拒命如故,部下漸攜貳。於是率泰復招降其將林俊奇、陳輝、何義、魏明等三百餘人,兵二千有奇。統建寧、延平、邵武三路士卒剿內地山寇,獲其渠王鐵佛,斬之。既,錦率其將周全斌以五百餘人自梁山內犯,率泰遣總兵王進加、參將折光秋夾擊,大破之;復與靖南王耿繼茂統舟師搗廈門,取浯嶼、金門二島,錦宵遁。三年,降其將林國樑,進兵八尺門,降其將翁求多;夜半渡海拔銅山,斬級三千有奇,其將黃廷等率兵民三萬餘人來降,獲敵艦、軍械無算。錦僅以數十艘遁入臺灣。敘功,加秩正一品。

尋以病累疏乞休,詔輒慰留。五年,卒官。遺疏言:「海賊遠竄臺灣,奉旨撤兵,與民休息。第將眾兵繁,撤之驟,易致驚疑;遲,又恐貽患。今當安反側之心,後須防難制之勢。紅毛夾板船雖已回國,然往來頻仍,異時恐生釁。至數年以來,令沿海居民遷移內地,失其故業。宜略寬界限,俾獲耕漁,庶甦殘喘。」上聞,優詔褒恤,贈兵部尚書,復世職,謚忠襄。

趙廷臣,字君鄰,漢軍鑲黃旗人。順治二年,自貢生授江蘇山陽知縣,遷江寧同知,有政聲。坐催徵逾限,免。十年,大學士洪承疇經略湖廣,薦廷臣清幹,題授下湖南道副使,屢平冤獄。十三年,調督糧道。

十五年,從定貴州,遂擢授巡撫。甫至官,察民間疾苦,定賦蠲賑,懲貪橫,禁吏卒驛騷。疏言:「貴州古稱鬼方,自城市外,四顧皆苗。其貴陽以東,苗為夥,而銅苗、九股為悍;其次為革老,曰羊黃,曰八番子,曰土人,曰侗人,曰蠻人,曰冉家蠻,皆黔東苗屬也。自貴陽以西,羅羅為夥,而黑羅為悍;其次曰仲家,曰米家,曰蔡家,曰龍家,曰白羅,皆黔西苗屬也。專事鬥殺,馭之甚難。臣以為教化無不可施之地。請自後應襲土官年十三以上者,令入學習禮,由儒學起送承襲。其族屬子弟原入學讀書者,亦許其仕進,則儒教日興而悍俗漸變。土官私襲,支系不明,爭奪易起,釀成變亂,令歲終錄其世次籍上布政司達部。有爭襲者,按籍立辨,豫杜釁端。」並下部議行。

十六年,擢雲貴總督。土寇馮天裕陷湄潭,犯甕安,調兵擊卻之。疏請改馬乃、曹滴諸土司為流官。又言:「貴州曩被寇,改衛為府,改所為縣,法令紛更,民苦重役,今應復舊制。雲南田土荒蕪,當招民開墾。沖路州縣,請以順治十七年秋糧貸為春種資。」並下部議行。吳三桂貢象五,世祖命免送京,廷臣因乞概停邊貢,允之。十八年,以平土酋龍吉兆功,加兵部尚書。是年調浙江。敘云南墾荒勞,加太子少保。

康熙二年,疏言:「浙江逋賦不清,由徵解繁雜,請以一條鞭法令各州縣隨徵隨解,布政司察明註冊,至為簡易。」又疏言:「徵糧之法不一,茍能寓撫字於催科,即百姓受其福。急公好義,人情皆然。有司止以箠楚為能,民安得不重利借債,減價賣產?錢糧完,地方壞矣。茍能得廉有司,禁革火耗,天平不欺天,法馬不違法,又禁絕差擾,一酒一飯無不為民節省,民未有不交納恐後者。徵糧之能,在人不在法,然不得其人而循法行之,亦得半之道也。實徵冊籍立實在戶名,以杜詭卸;流水紅簿送本府印發,以防侵蝕;易知由單遍散窮山深谷,以絕橫索。臣於浙屬立法通行,催徵得法之吏,請敕部酌議,許題請獎勵。」又疏請移海島投誠官兵分插內地,杜其煽誘;定水師提鎮各營兵制,以備水戰。杭、嘉、湖三郡毗連太湖,易藏奸宄,請增造快號船,撥兵巡哨。詔並從之。時鄭成功死,廷臣招明魯王所署將軍阮美、都督鄭殷、侍郎蔡昌登等,皆率眾來降;惟張煌言散兵居定海山中,執而殺之。

四年,疏請崇節儉,維風俗。又言用人宜寬小眚,請敕部分別罣誤降革人員,量才錄用。又言民人鬻身旗下,宜令有司給與印契,並曉諭鄰里,後或逃歸,有容留者,乃可坐以窩逃。並議行。時錢滯不行,疏請令外省收銅開鑄,準寶泉、寶源兩局法式,去各省分鑄之名,以天下之錢供天下之用。上命復各省二十四監鑄錢。浙東初平,叛獄屢起,廷臣平情讞鞫,全活甚眾。時海濱尚多餘孽,聞廷臣寬大,多解甲來歸。六年,以病乞休,詔慰留之。八年,巡海自福建還,至奉化,病卒,謚清獻。

廷臣為政寬靜而善折獄。有瞽者入屠者室,攫其簝中錢,屠者逐之,則曰:「欺吾瞽,奪吾錢。」廷臣令投錢水中,見浮脂,以錢還屠者。有殺人獄已誣服,廷臣察傷格,曰:「傷寸而刃尺,必冤也!」更求之,得真殺人者。旱,山中人言魃見,入人家輒失財物。廷臣曰:「盜也!」令吏捕治之。

袁懋功,字九敘,順天香河人。順治二年進士,授禮科給事中。疏請慎簡學官,磨勘文體,釐定禮制。又以前明廢官援恩詔踵至,請敕吏部會都察院嚴覈才品。累擢戶部侍郎。十七年,世祖諭懋功才品敏練,授雲南巡撫。時雲南初定,懋功令降卒入籍歸農,墾無主之田。編保甲,以時稽察。奏減屯田糧額,請停派部員履勘田畝。撫雲南九載,政績大著。以父憂去。服除,起山東巡撫。康熙十年,濟南五十六州縣衛新墾地被淹,懋功疏請展限一年起科,部格不行,上特允之。調浙江,未行,卒,謚清獻。

徐旭齡,字元文,浙江錢塘人。順治十二年進士,除刑部主事,再遷禮部郎中。康熙六年,授雲南道御史。裁缺,改湖廣道。迭疏請汰額外衙役,核州縣贖鍰,降調官百姓保留敕督撫核實,皆下部議行。命偕御史席特納巡視兩淮鹽政,疏陳積弊,請嚴禁斤重不得逾額,部議如所請勒石。又疏請停止豫徵鹽課,部議不允。遷太常寺少卿,累擢左僉都御史,請裁軍興以後增設道員。二十二年,授山東巡撫。二十三年,遷工部侍郎。復出為漕運總督,疏請釐三害,籌三便,革隨漕增、裁運耗二項,及民間幫貼盤費腳價,各省給軍款項,改由州縣逕發運丁,行月糧改入現運項下撥給,並合並漕船幫次,皆下九卿議行。二十六年,卒,亦謚清獻。

郎廷佐,字一柱,漢軍鑲黃旗人,世籍廣寧。父熙載,明諸生。太祖克廣寧,熙載來歸,授防禦,以軍功予世職游擊。崇德元年卒,長子廷輔嗣。廷佐,其次子也。自官學生授內院筆帖式,擢國史院侍讀。順治三年,從肅親王豪格徇四川,平張獻忠。六年,從英親王阿濟格討叛將姜瓖。遷秘書院學士。

十一年,授江西巡撫。江西自明末洊遭兵亂,逋賦鉅萬。廷佐累疏請蠲緩蘇民困,詔允行。土寇洪國柱等掠饒州、廣信,遣兵剿平之。十二年,擢江南江西總督。江南逋賦至四百餘萬,廷佐覈賦籍,曰:「此非盡民困不能輸也,必有官吏侵蝕而詭稱民逋者。民困可矜,官吏弊不可不革。」乃籍之為三:曰官侵,曰吏蝕,曰民逋。責右布政使按籍督催,而令左布政使稽徵新賦,以除新舊牽混之弊。並疏請官吏徵賦未完者,令戴罪留任催徵,於是宿弊頓革。師行取估舶以濟,商民交困。廷佐疏請視江西例,發帑造船備用,上韙其言,命議行。

十六年,巡閱江海,因密疏言:「鄭成功屯聚海島,將犯江南。江南汛兵無多,水師舟楫未備,請調發鄰省勁兵防禦。」事格不行。未幾,成功陷鎮江,襲瓜洲,遂窺江寧,城守單弱。會梅勒額真噶楚哈、瑪爾賽自貴州旋師,廷佐與駐防總管喀喀木邀入城共禦敵,挫其前鋒,得舟二十餘。成功兵大至,戰艦蔽江,廷佐登埤固守。提督管效忠、總兵梁化鳳等水陸夾擊,焚敵艦五百餘,擒斬無算,成功遁入海。捷聞,詔嘉獎。十八年,分江南江西總督為二,以廷佐專督江南。康熙四年,復舊制,仍兼江西。七年,以疾解任。致仕大學士金之俊家居,得匿名書帖,詆其曾降李自成,之俊訴廷佐,令有司窮治。上聞,慮株連無辜,責之俊違例妄訴,廷佐俟病痊起用,鐫二秩。

十三年,耿精忠反,授廷佐福建總督。廷佐奏言:「臣孫為耿氏婿,臣與精忠有連。然誓不與賊俱生,原力疾前驅,殲除叛寇。」上嘉之,賜鞍馬、甲胄以寵其行。廷佐至浙江,從大將軍康親王傑書治軍,駐金華。疏陳精忠句結海寇,宜剿撫兼施。上曰:「海寇當撫,精忠當用剿,或用間。」廷佐頗有規畫,未及行,十五年,卒於軍,賜祭葬。江南、江西俱祀名宦。

弟廷相,字鈞衡。初授欽天監筆帖式。累官四川左布政使。四川屢經兵燹,廷相蒞任,百廢俱興,民不知擾。康熙八年,授河南巡撫。廷佐卒,上即擢廷相為福建總督。會精忠降,餘黨紀朝佐、張八等尚抗拒,廷相剿撫兼用,旬月悉平。鄭錦及山寇硃寅屢犯郡縣,遣兵分剿,屢卻之,擒斬甚眾。十七年,錦窺漳州,據玉州等寨,分擾石碼、江東橋。廷相請援,詔康親王督兵協剿。時寇勢甚張,上責廷相庸懦不能殄賊,命解任。二十七年,卒。

郎永清,字定庵。初授禮部筆帖式。出知山西渾源州,招民開墾,豁逋賦萬餘。姜瓖黨高山等竄伏山谷間為盜,永清簡丁壯,親率搜捕,多斬獲。事平,擢江西贛州知府,平反冤獄,居官有聲。師討李定國,議牧馬贛州,民譁言兵且入城,爭竄避。永清度城外地為牧場,區畫八旗駐營,具芻茭,兵不入城,贛民安堵。師還,徵民夫數千挽舟,灘水湍激,永清慮民夫無食且逃,以大艦載米尾其後,軍行無滯。民德之,為立像祠焉。

從子廷佐巡撫江西,永清例回避,調山西汾州。遷山東東昌道副使,轉湖廣下荊南道。李自成黨踞房、竹間,官軍分路會剿,餽餉俱取給鄖、襄,陸路挽運,議徵民夫數萬。永清疏水道,仿古轉搬法,安塘遞運,軍得無匱。累遷湖南布政使。衡、永、寶三郡苦食粵鹽,灘險道遠,商民交病。永清申請改食淮鹽,民便之。康熙十二年,調河南。師討吳三桂,議養馬南陽,永清請移牧湖廣。河南協濟湖廣軍米十萬石,申巡撫題請改於江南、江西採運。在官十二年,課最。二十五年,擢山東巡撫。未幾,卒官,祀湖南名宦。永清子廷極、廷棟。

廷極,字紫衡。初授江寧府同知,遷雲南順寧知府,有政聲。累擢江西巡撫。江西多山,州縣運糧盤兌,民間津貼夫船耗米五斗三升,載賦役全書,歲分給如法。戶部初議駁減,總督範承勛以請,得如故。至是戶部復議停給,並追前已給者,廷極累疏爭之。尋兼理兩江總督。五十一年,擢漕運總督。卒,謚溫勤。廷棟,字樸齋。官湖南按察使。

佟鳳彩,字高岡,漢軍正藍旗人,養性從孫也。初授國史院副理事官。外改順天香河知縣,內擢山西道御史,出視河東鹽政。順治七年,巡按湖南。八年,外轉湖廣武昌道參議,遷廣西右布政使。時師征雲南,道廣西,供億浩繁,鳳彩籌濟無匱。調江西左布政使。十七年,擢四川巡撫。四川經張獻忠亂,城邑殘破,勸官吏捐輸,修築成都府城,葺治學宮,濬都江大堰。以祖母憂去官。

康熙六年,起貴州巡撫。疏言:「驛站累民,而貴州尤苦。層山峻嶺,俗言『地無三里平』。行一站,馬則蹄瘤脊爛,夫則足破肩穿。應於重安江、楊老堡、黃絲鋪、盤江坡、江西坡、輭轎坡等六處增置腰站,設夫馬如額。」復言:「黔省田土多奇零,國初隸版圖,州縣衛所等官不諳賦役,任意牒報。戶部以明季賦役全書發黔訂正,原報多者不復更改,少者照數增添。臣蒞任,酌定繇單規式,飭所司填給花戶,以杜私派。嗣各屬造報,此多彼縮,不能照則填給。且田地名色甚多,錢糧輕重不一。現飭所司清釐,更正賦役全書,以垂永久。」詔並允行。丁母憂。

十一年,起河南巡撫。彰德舊有萬金渠,康熙七、八年水患三至,鳳彩奏請修濬,以弭民害。尋疏言:「豫省歲修黃河,用夫多或至萬餘,俱按畝起派,雇直年需三四十萬,小民重困。請改為官雇,按通省地畝等則派銀,刊明繇單。若遇意外大工,再具題請旨。」上以派銀雇夫仍屬累民,命並免之。十二年,鳳彩疏言:「均平里甲,直省通行。河南雖有里甲之名,其實多者每里或五六百頃,少者止一二百頃,或寥寥數頃。有司止知照例編差,里小田少,難以承役,愈增苦累。今飭州縣按徵糧地畝冊,如一州縣有地一千頃,原分為十里者,每里均分一百頃;一里之中各分十甲,每甲均分十頃。遇有差徭,按里甲分當,則豪強無計規避,貧弱不致偏枯。」又言:「豫省民間栽柳供河工採辦,歲需百餘萬束。自康熙七年以後,協濟江南河工已二百七十餘萬束。去歲陽武險工,無柳可用,將民間桃、李、梨、杏盡行斫伐,方事堵御。是修防本省河工尚屬不敷,實難協濟外省。且黃河渡船裝柳止二三百束,至無船之地,官吏束手,若非亟圖變通,必至誤運。向例本省河工運柳,每束給銀五分,今遠運江南千里之外,止給銀四分五釐,民安得不賠累?乞敕河臣於江南雇船到豫,使民止備柳束輓運江幹。嗣後就江南鄰近無河患處,酌派協濟。留河南有餘不盡之柳,以備本省河患,庶百姓稍得蘇息,大工不致遲誤。」疏入,並下九卿科道議行。河南民稱均里甲、蠲夫柳為利民二大疏。

吳三桂反,河南當通衢,鳳彩悉心調度,民不知擾。十三年,以疾乞休,許之,士民赴闕籥留。左都御史姚文然疏言,鳳彩撫豫數載,民所愛戴,宜令力疾視事,命仍留任。十六年,卒官,謚勤僖。河南、四川、貴州並祀名宦。

麻勒吉,瓜爾佳氏,滿洲正黃旗人。先世居蘇完,有達邦阿者,當太祖時來歸,麻勒吉其曾孫也。順治九年,滿、漢分榜,麻勒吉以繙譯舉人舉會試第一,殿試一甲第一,授修撰,世祖器之。十年,諭麻勒吉兼通滿、漢文,氣度老成,擢弘文院侍講學士。十一年,擢學士,充日講官,教習庶吉士,編纂太祖、太宗聖訓副總裁,經筵講官。

明將孫可望詣經略洪承疇軍降,封義王,命麻勒吉為使,學士胡兆龍、奇徹伯副之,齎敕印授之,即偕詣京師。麻勒吉初與直隸總督張玄錫同官學士,使還,玄錫迎於順德,麻勒吉訶辱之,玄錫憤,自剄不殊。巡撫董天機以玄錫手書遺疏上聞,上遣學士折庫納、侍郎霍達往按。玄錫復疏言:「麻勒吉於迎候時面斥失儀,又責以前此南行不出迎,且云:『在南方洪經略日有饋遺,何等盡禮!』奇徹伯又索臣騾駝。臣因賄賂干禁,不與。」上責麻勒吉等逼迫大臣,任意妄行,下九卿會勘。玄錫,直隸清苑人,明庶吉士。順治初授原官,自檢討累遷至學士。上稱其勤敏,擢宣大總督,移督直隸、河南、山東。至是,以聽勘詣京師,居僧寺,自縊。九卿議麻勒吉等當奪官籍沒,上寬之,削加級、奪誥敕而已。

十六年,以雲南初定,發帑金三十萬,命麻勒吉偕尚書伊圖、左都御史能圖往賑,並按大將軍貝勒尚善縱兵擾民狀,麻勒吉為奏辨。尋安親王岳樂覆勘,尚善兵入永昌掠民婦事實,麻勒吉坐徇庇,奪官。十八年,命以原銜入直。上大漸,召麻勒吉與學士王熙撰擬遺詔,付內廷侍衛賈卜嘉進奏。上命麻勒吉懷詔草,俟上更衣畢,與賈卜嘉奏知皇太后,宣示諸王貝勒。是夕上崩,麻勒吉遵旨將事。旋授秘書院學士。

康熙五年,擢刑部侍郎。七年,授江南江西總督。時蘇州、松江頻遭水患,布政使慕天顏議濬吳淞江、劉河口,麻勒吉因與巡撫瑪祜疏請以各府漕折銀十四萬充工費。淮、揚被水坍沒田地,請永免歲賦。詔並允行。鎮江駐防兵訐將軍李顯貴、知府劉元輔侵冒錢糧,遣學士折爾肯等往按得實,麻勒吉坐不先舉發,並械系至京聽勘。給事中姚文然疏言麻勒吉罪狀未定,宜寬鎖系,上然之。尋命復任。十二年,大計,左遷兵部督捕理事官。

吳三桂反,定南王孔有德壻孫延齡及提督馬雄以廣西叛應之。十六年,命赴簡親王喇布軍,招撫延齡。比至桂林,延齡已為三桂所殺,其部將劉彥明等率眾降。十八年,詔麻勒吉赴廣西護諸軍,時雄已死,其子承廕降,授招義將軍,封伯爵。已,部兵以餉匱譁,麻勒吉上言:「承廕與黃明、葉秉忠皆賊帥歸誠,今承廕授高爵,而明、秉忠未授官,故陰嗾兵士為變。秉忠年老無異志,惟明強悍,為柳州官兵所懾服,若不調用他所,終恐為害。」乃授明總兵官。明復叛,詔麻勒吉與偏沅巡撫韓世琦會剿,尋報為苗人所殺。十九年,巡撫傅弘烈剿賊至柳州,承廕復叛,弘烈遇害,命麻勒吉兼攝巡撫事。時柳州再變,民多逃竄,田荒賦淆,麻勒吉招撫流亡,令歸故業,葺學宮,振興文教,頗著治績。二十一年,撤故定南王所部,分隸八旗漢軍,麻勒吉率以還京。

二十三年,授步軍統領。二十八年,卒。三十七年,兵部奏黃明為貴州參將上官斌等所擒,麻勒吉追坐妄報。奪官。江南民為麻勒吉立碑雨花臺紀績,祀名宦。

阿席熙,瓜爾佳氏,滿洲鑲紅旗人。自兵部筆帖式四遷光祿寺卿。考滿,輔政大臣鼇拜等令解任,隨旗行走,復坐事奪官。聖祖親政,鑒其無罪,命以郎中用。七年,超擢陜西布政使。舉卓異,擢巡撫。康熙十二年,遷江南江西總督。耿精忠叛,窺江西,阿席熙發兵赴剿,並檄援浙江。未幾,精忠陷廣信、建昌、饒州,參將陳九傑等應之。阿席熙遣兵防徽州,賊陷績溪、婺源,擾及徽州,迭克之。簡親王喇布率師至江寧,以阿席熙參贊軍務。十七年,疏報江南清出隱漏田地一萬四千餘頃、山八百餘里,加兵部尚書。尋坐瞻徇巡撫慕天顏奏銷浮冒,罷任。卒。阿席熙居官廉潔,江南士民德之,祀名宦。

瑪祜,哲柏氏,滿洲鑲紅旗人。順治九年繙譯進士。授佐領,兼刑部員外郎。遷欽天監監正。康熙八年,江寧巡撫缺,命議政大臣等會推滿洲郎中以上、學士以下通漢文有才能者備擢用,舉奏皆不當上意,特以命瑪祜。九年夏,淮安、揚州二府久雨,田廬多淹,詔發帑賑濟。瑪祜疏請蠲免桃源等縣積欠賦銀,及六、七兩年未完漕米,部議漕米無蠲免例,上特允其請,並蠲減蘇、松、常三府被災歲賦。

十年,疏言:「蘇、松二府額賦最重,由明洪武初以張士誠竊據其地,遷怒於民,取豪戶收租籍,付有司定賦額,較宋多七倍、元多三倍,是以民力困竭,積逋遂多。自康熙元年至八年,民欠二百餘萬,催徵稍急,逃亡接踵,舊欠仍懸,新逋復積。請敕部覈減二府浮糧,以期歲賦清完。」疏下部議,以科則久定,報寢。時布政使慕天顏請濬吳淞江、劉河,瑪祜與總督麻勒吉請以漕折十四萬充費。給事中柯聳疏言,東南水利宜乘此興工,盡疏各支河。下瑪祜覆勘。瑪祜言各州縣支河皆已疏通,吳江縣長橋乃太湖洩水耍道,應令開濬。未幾,以京口將軍李顯貴等侵餉事覺,坐不先舉發,罣吏議,當左遷,命留任。十二年,黃、淮水漲,清水潭石堤決,高郵等十八州縣衛所被災,瑪祜奏請發帑賑濟。十五年,霪雨久不霽,以憂卒。遺疏極陳水災民困,無一語及私。詔褒惜,謚清恪。

施維翰,字及甫,江南華亭人。順治九年進士,授江西臨江推官,清漕弊,善折獄,奸頑斂跡。巡撫郎廷佐奏其治行,舉卓異,內擢兵部主事。改山東道御史,疏言:「察吏首重懲貪,尤宜先嚴大吏。各督撫按露章彈劾,宜及監司,勿僅以州縣塞責。」又言;「糾舉之法,密於文,疏於武。鎮帥擁重兵,有庸碌衰憊、緩急難恃者,有縱恣婪贓、肆虐軍民者,督撫按徇隱弗糾,事發同罪。」詔並議行。十七年,出按陜西。聖祖即位,裁巡按,維翰乞假歸。

康熙三年,復授江南道御史,疏言:「直省錢糧,每委府佐協徵,所至鋪設供給,不免擾民。甚或縱容胥役,橫肆誅求。請概行禁止,以專責成、杜擾害。」下部飭禁。巡鹽河東,徵課如額。八年,疏劾偏沅巡撫周召南徇庇貪吏。十一年,疏劾福建總督劉斗徇情題建故靖南王耿繼茂祠。召南、斗並坐譴。十二年,內升,以四品服俸仍留御史任。疏言:「設登聞鼓,原以伸士民冤抑,故使科道共與其事。然每收訴狀,必待科道六十餘員集議,輒致稽延。請用滿、漢科道各一員司之,半年更易。」從之。

遷鴻臚寺少卿,累遷左副都御史。浙江巡撫陳秉直薦舉學道陳汝璞,為左都御史魏象樞所劾,秉直應降調,以加級抵銷。維翰言:「秉直與汝璞見聞最近,乃徇情妄舉,非尋常詿誤可比。請敕部定議,凡保舉非人坐降調者,不許抵銷。」上然之,因著為列。給事中李宗孔繼劾秉直,坐左遷。

十八年,授山東巡撫。會歲祲,民多流亡,維翰疏請賑恤,並截留漕米五萬石發濟南倉存貯,散給饑民。又疏言:「青、萊等府距臨清倉遠,辦解甚艱。請永行改折,以息轉輸。」民大悅服。二十一年,代李之芳為浙江總督。之芳按治軍士鼓譟,系累二百餘人。維翰至,即日定讞,多平反。二十二年冬,調福建,未上官,二十三年春,卒,謚清惠。

論曰:李率泰鎮福建,御鄭成功父子,趙廷臣督浙江,執張煌言,有功於戡定。郎廷佐釐逋賦。佟鳳彩均里甲、蠲夫柳,為民袪害。麻勒吉初奉使迫張玄錫至死,聖祖諭斥其縱恣。然於江南有惠政,阿席熙、瑪祜清望尤過之。施維翰在臺敢言,出持疆節,措置得大體。皆康熙初賢大吏也。愷悌君子,屏籓王國,厥績懋矣!


\end{pinyinscope}