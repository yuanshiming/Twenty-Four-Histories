\article{志三十七}

\begin{pinyinscope}
地理九

△河南

河南:禹貢豫及冀、揚三州之域。明置河南布政使司。清初為河南省,置巡撫。雍正二年,升陳、許、禹、鄭、陜、光六州為直隸州。十二年,升陳、許為府,鄭、禹仍屬州。乾隆九年,許復直隸。光緒末,鄭復直隸。宣統初,淅川直隸。領府九,直隸州五,直隸一,州五,縣九十六。東至江蘇蕭縣;六百五里。西至陜西潼關縣;一千三十里。南至湖北黃陂縣;一千一十里。北至直隸磁州。五百八十里。廣千六百三十里,袤千三百九十里。宣統三年,編戶四百六十六萬一千五百六十六,口二千六百八十九萬四千九百四十五。其名山:嵩高、三崤、熊耳、太行。其大川:河水、淮、汴、洛、潁、汝、白、丹、衛、漳。其鐵路:京漢,開鄭,道澤。其電線:東北達濟南、京師;西,長安。

開封府:沖,繁,疲,難。巡撫,布政、提學、提法司,鹽、糧、開歸陳許鄭、兵備、巡警、勸業道駐。明洪武元年,以元汴梁路改。清初,河南省治,仍領州四,縣三十。雍正二年,陳、許、鄭、禹直隸,割縣十四隸之。延津、原武屬衛輝、懷慶。乾隆中,禹及密、新鄭還隸;河陰省;陽武、封丘屬懷慶、衛輝;儀封為,後亦省。北至京師千五百八十里。廣三百七十里,袤三百六十里。北極高三十四度五十一分。京師偏西一度五十五分。領州一,縣十一。祥符沖,繁,難。倚。城東北隅:夷山。東北:赤岡。河水自元至元中始,盡歷府境,自中牟緣封丘界,逕黑岡、柳園口入,東入陳留。其賈魯河入,逕硃仙鎮入尉氏,即蔡故瀆,上游一曰沙水。水經注渠水實鴻溝,而浚水堙。其惠濟河入,逕府治南,亦入陳留。宋都四渠及五丈、白溝河亦堙。吹臺,縣丞駐。陳橋鎮。大梁驛。鄭汴鐵路。陳留沖。府東少南五十里。東北:潘岡。河水自祥符入,逕小黃故城北,又東入蘭儀。北:惠濟河,自祥符入,逕城北,水經注「沙水逕牛首亭東,魯渠出焉」者。其東,桃河、古渙水,又東,睢水,並汴支津,東南入杞,觀省陂在焉。縣驛一。杞沖,繁,難。府東百里。西北:惠濟河自陳留入。水經注「逕陽樂城南、鳴雁亭北」。睢水亦自陳留入,逕高陽城,合桃河為橫河,實古澮水,並東南入睢。西南:青岡。河自通許緣界入太康。河水舊逕縣北,故有漢堤、隋堤,自大梁至灌口,即老鸛河也。雍丘驛。通許簡。府東南九十里。吳召岡、李大岡諸岡綿亙縣境,河流環之。東南:青岡。河出縣西北,下流為燕城河,入太康。北:雙溝,蔡故渠。水經注「沙水逕裘氏亭西,澹臺子羽塚東」者,半截河出焉,西南入尉氏。縣驛一。尉氏沖。府西南九十里。城內:尉繚子臺。東:錦被岡。西南:三亭岡。城南:五鳳山。東北:賈魯河自祥符入,右合康溝及大溝新河。水經注「長明溝逕向城北、尉氏故城南,三分」者。至白潭鎮,左納半截河,東南入扶溝。縣驛一。洧川簡。府西南百五十里。東:東里。東南:赤阪岡。西:雙洎河,即洧水,自長葛入,左合蟄龍復受清河、大沼水,逕新汲故城考升廟北,大隧澗在焉,迤東入鄢陵。縣驛一。鄢陵難。府西南百九十里。北:彭祖岡。東北:彪岡。雙洎河自洧川入,逕彭祖岡,東南入扶溝。南:艾城河自臨潁緣界,右會石梁為流潁河,逕陶城入西華。城南:文水,又南,三道河,達太丘城。縣驛一。中牟沖,繁。府西七十里。北:牟山。西南:馬陵。西北:圃田澤。河水自鄭入,逕楊橋口,又東,黃練集。賈魯河入,合龍須溝,隋志鄭水。又東,右合鴨陂水,至縣西。乾隆六年濬為惠濟河。正渠又東逕官渡城,又東南,右合糞陂,古末水,丈八溝,焦城在焉,古清池水,並入祥符。自周定王五年河南徙,邑沮洳。明萬歷中,令陳幼學濬渠百九十有六。縣境瀕河,有管河上汛縣丞、下汛縣丞駐。曲遇聚、白沙、東張、楊橋四鎮。城驛一。鄭汴鐵路。蘭封沖,繁。府東北九十里。明,蘭陽。道光四年改蘭儀。同治二年省儀封入。宣統元年復諱改。東北:黃陵岡。西北:河水自陳留入,舊入考城。咸豐五年決銅瓦廂,改東北逕龍門口入直隸長垣。舊賈魯七河堙。陽封,管河縣丞駐。管城驛。禹州沖,繁。府西南二百九十里。明初鈞州,後改。雍正二年升,十二年降屬許州府。乾隆六年還隸。北:大騩山。西南:九山。西北:荊山,小洪河出,入長葛;崆峒、鐵母。潁水自登封入,逕康城陽關聚,左合書堂麻地川。右湧水,逕城南,一曰褚河,入襄城,其西土爐河,下流並達之。水經注「故瀆逕三封山,有嵎水」。今泉二:上棘、小韓。清潁一驛。密簡。府南二百八十里。清初自禹來隸。雍正二年復屬禹。乾隆六年復。南:密岵山。西北:開陽。東南:洧水,源出登封馬嶺,東北流,逕縣東南,綏水注之。又東流,溱水注之。又東入新鄭。東南:大騩山,潩水出,其玉女陂從之。東北:聖水峪,聖水出,入鄭鄶城。縣驛一。新鄭沖。府西二百里。清初自禹來隸。雍正二年復屬禹。乾隆六年復。東南:大騩。潩水自密出山,逕風後頂,又東南,逕陘山入長葛。其北,洧水自密會溱入,曰雙洎河,至城南為洧淵,又東南,逕土城,左合黃水,右梨園河,亦入長葛,梅從之。水經注「長明溝水出苑陵故城西北,東即古制澤、西瑣澤,合龍淵泉、白雁陂」者。永新、郭店二驛。鄭汴鐵路。

歸德府:沖,繁,難。隸開歸陳許鄭道。總兵駐。西距省治二百八十里。廣四百七十里,袤三百二十里。北極高三十四度三十二分。京師偏西三十五分。沿明制,領州一,縣八。商丘沖,繁,難。倚。商丘,城西南三里。又城南四十里,穀丘。河水自宋開寶四年至康熙四年決入郡境者以十數,府治與為轉徙,南北不恆。咸豐五年後,故道淤。豐樂河出焉,東南入夏邑。古汴水一曰護水,其支津澮河,即睢水上源,湮。今首縣西北,俗名沙河,歧為三。北岔入永城。正渠及南岔,與其支苞河、其西陳兩河,自寧陵入,右合沙家窪、冀家河,左合古宋河,並入安徽亳州。沙為馬尚,南岔為武家,而陳兩為清河。大蒙,古景亳。小蒙側有漆溝、孟諸澤。濟陽、葛驛二鎮。縣驛一。寧陵沖。府西六十里。西:甘露嶺。東北:河水故道,淤。其自睢入西南者曰張公河,逕漢已吾故城東入柘城。西北:陳梁沙河,俗名陳兩河。長安一鎮。寧城一驛。鹿邑繁,疲,難。府南百二十里。故城,縣西,古鳴鹿,縣丞駐。東:陰靈山、隱山。西南:橫嶺。西北:惠濟河自柘城入,逕賈灘南。渦水自太康入,錯淮寧復入。南:清水河,渦支津,舊自淮寧入,今首虞詡墓北,逕匯城東南,為練溝,並入安徽亳州。其清水,南出偃王陂者,茨刺河,右合瀖水,會西明河。水經注「自陳城百尺溝東逕寧平故城南」者,入太和,東明河亦入之。穀陽一鎮。縣驛一。夏邑沖,難。府東百二十里。清河自虞城入。左合橫河。西北:豐樂河自商丘入,為響河,及虯龍河、歧河,並東南入永城。分防夏商永、縣丞駐。會亭一驛。永城沖,繁,難。府東南百八十里。北:碭山。巴清河即減水溝,自夏邑入,東南入江蘇蕭縣。東洪溝,自蕭入,仍從之。響河逕太丘故城,合虯龍溝、歧河,為巴溝河,逕城北,東南入安徽宿州。南:澮河自商丘入,逕建平、酂、費故城北,右合北岔沙河。又東,包河自安徽亳州入,並從之。新興、保安二鎮。太丘一驛。虞城沖,繁。府東北七十里。東北:柱岡、黎丘。河水故道自商丘入,東入江蘇碭山,即古汴渠。水經注「逕周塢側」者,橫河出焉。南惠民溝,並入夏邑。治平一鎮。石榴堌一驛。睢州沖,繁。府西百里。城西:駱駝岡。北:黃河故道自考城入。明嘉靖十九年決野雞岡,南流者為張弓河,入寧陵。西:惠濟河自杞入,左合橫河,即擅其故道,東南入柘城。橫即睢,睢即渙。水經注「逕承匡城,又東逕襄邑故城南」者。歸化、重華二鎮。五橋集,州判駐。葵丘一驛。考城簡。府西北百二十里。乾隆四十九年改隸衛輝。光緒元年復。南:葛岡。河水故道舊自蘭封入,東入山東曹縣。咸豐五年北徙。舊有戴水,並堙。斜城、葵丘有驛。柘城簡。府西南九十里。城東北隅:廓山。河水故道二。西北:惠濟河自睢入,逕心悶寺,水經注「睢水歷傿縣北」者,舊納渦支津。北:張弓河自寧陵入,逕牛鬥城,會於東南磚橋,東南入鹿邑。又東劉家河,古穀水,即渙水,水經注「逕鄫城北」者。又古泓水,縣西,並堙。八橋一鎮。縣驛一。

陳州府:繁,難。隸開歸陳許鄭道。清初沿明制,為開封屬州,領縣四。雍正二年,升直隸州。十二年,升府,並割太康、扶溝來隸,增附郭。西北距省治三百里。廣一百九十里,袤二百十五里。北極高三十度四十七分。京師偏西一度二十六分。領縣七。淮寧繁,疲,難。倚。明省宛丘入州。雍正十二年,析改為府治。西北:西銘山、杏岡。北:鞍子嶺,西明河出,逕漢新平故城北。東北:渦水自太康入,並入鹿邑。西南:沙河自商水緣界會賈魯河入,逕趙牛口,納柳涉河,逕新站集,又東南,左納西蔡河,又東南入項城。汾河自商水流入縣西南,又東入項城。東南:東蔡河,入沈丘。周家口在縣西南,賈魯河、沙河交匯於此。縣驛一。商水簡。府西南七十里。西北:沙河,古渡水,自西華入,逕鄧城,又東,右會潁水,逕叢臺,至周家口。南綰汝、蔡,北轂陳、汴,通判駐。左會賈魯河,逕灌溉城、潁歧渡,緣淮寧界入之。西有汾河,舊自西華入,逕扶蘇城,左合枯河,東逕範臺,右納界溝河,入淮寧。穀陽一鎮。縣驛一。西華難。府西北百八十里。南:宜山。西:廟陵岡。西南:沙河自郾城入,東逕小陶、夏亭城入商水。渚河即潁,右合土爐河,又東北,左納其支津流潁為合河口,逕叢桑村,又東,左納大浪溝從之。西南:洪河,自上蔡錯入,仍入之。又賈魯河,西北自扶溝入,逕護當城,側城東南入淮寧。柳涉河源自縣東,東南入淮寧。常社一鎮。縣驛一。項城簡。府南百二十里。河水故道即今沙河,自淮寧入,逕公路城入沈丘。汾河西北入,逕後魏平鄉諸陂,水經注「逕南頓故城南」者。西有泥河,即蔡河,自上蔡入,錯汝南復入,逕石橋,並東入沈丘。縣驛一。沈丘難。府東南百三十里。北:大沙河自項城入,左納東蔡河,逕其北。汾河入為小沙,左右合谷河、泥河,逕城南,入安徽太和。紙店一鎮。縣驛一。太康繁,疲,難。府北五十里。北:石山。東北:長白。西北:青岡。河自通許入,為燕城河,渦水冒為源,匯白洋諸溝,逕城南,又東南,左合河水故渠,逕馬廠集入鹿邑。槐店,縣丞駐。崔橋一鎮。縣驛一。扶溝簡。府西北百二十里。西北:雕陵岡。賈魯河自尉氏入,至張單口,左會雙洎河,水經注「洧水逕桐丘城西」,其孟亭故道堙。所謂小扶亭、洧溝,縣氏焉。側城東南,逕大扶城,古渦水出焉。又東南入西華。其西,文水河自鄢陵入,右合三道河,為大浪溝,逕鴨岡,洧西南故道逕新汲故城西、匡城南,左迤為鴨子陂者亦入之。白亭、洧陽、固城、呂潭四鎮。縣驛一。

許州直隸州:沖,繁。隸開歸陳許鄭道。清初沿明制,為開封屬州。雍正二年升,仍所領。十二年為府。乾隆六年復。東北距省治二百五十里。廣九十里,袤百二十里,北極高三十四度五分。京師偏西二度二十五分。領縣四。西南:熊耳山。渚河,今潁水,自襄城緣界入,逕潁陽故城,古許國,東南入臨潁。其古潁水支津石梁河。西北自禹入,左納暖泉河,逕城西。又東南,右合椹澗,其東澮河,即潩水,水經注「逕射犬城」,自長葛入,東至秋湖,曰艾城河。其東洧倉城、其西岸亭,並從之。椹澗、石固二鎮。縣驛一。鐵路。臨潁沖。府東南六十里。潁水自州入。水經注「逕繁昌故城北」,有鍋壅口,東則棗祇河故瀆出焉。又東南,逕澤城北,古皋鼬,緣郾城界錯西華復入,入西華。其東支津石梁河亦自州入,逕大陵城南、御龍城南,左會艾城河,右合五里河左瀆,入鄢陵。西南:土爐河自襄城緣界並達之。繁城一鎮。縣驛一。襄城沖,繁。府西南九十里。城南:首山。汝水西自郟入,左合汜河,水經注「逕西不羹城南」,右納湛河、輝河,入舞陽。東北:潁水自禹入,逕汾丘城,緣州界入之。東北:土爐河自禹州入,逕李膺墓、白草原,匯為硃湖潭,一曰扊勒河,左瀆入臨潁。其南瑪瑙河,出縣東,東南入郾城。襄城一驛。郾城沖。州東南百二十里。東:召陵岡。城南:陘亭。西北:潁水,自臨潁緣界,逕青陵城東入西華。其土爐河入逕襄城時曲柵,右合瑪瑙河,出扊勒橋從之。西有沙河,即汝水,自舞陽入,逕道州城,至城南,右合澧河、唐河,曰大溵河。東南歧為洄曲河,逕沱口鎮五溝營。其故渠自西平入,左合淤泥河來會,入上蔡。正渠折東北,一曰螺灣河,亦入西華。縣驛一。長葛簡。府西北五十里。西北:延秀岡。雙洎河,即洧水,自新鄭入,左合梅河,屈東北入洧川。澮河在縣西,上游曰潩水,自新鄭入,後河自西注之。又東南,入許州,曰艾城河。暖泉河自禹入,逕城西南隅,東南入州。鎮五:董村、石相、和尚橋、會河、後河。縣驛一。

鄭州直隸州:沖,繁,疲,難。隸開歸陳許鄭道。明屬開封。雍正二年升,並割其縣四。十二年並還隸。乾隆三十年,省河陰入滎澤。東北距省治百四十里。廣五十三里,袤六十五里。北極高三十四度四十九分。京師偏西二度三十四分。領縣三。西南:梅山。南:泰山。西北:河水自滎澤入,逕花園口,又東入中牟。須索河入,會京水,東逕衍南、祭城北,右合鄭水為沙河,一曰賈魯河,右合潮河從之。古汴水,禹貢曰灉,春秋曰邲,秦鴻溝,漢蒗蕩渠,東流曰官渡水,曰陰溝,曰浚儀渠。管城一驛。京漢,鄭洛,鄭汴鐵路。滎澤沖,繁。州西四十里。乾隆三十年省河陰入為鄉,巡司駐。西北:河水自汜水入,逕敖山,又東廣武滎澤口,又東入州。西南:索水自滎陽入,逕故城,踐土營在焉,右會須水,為須索河,逕平桃城。其京水緣州界從之。廣武一驛。鄭洛鐵路。滎陽沖。州西七十里。東南:嵩渚山,一名大周山,水經注謂之黃堆山。其西有萬山、賈峪山、靈源、檀山。諸山皆與中嶽聯體,而嵩渚為尊。索水,古旃然水,出其麓,轉北逕城東。東南:京故城。西:索氏。所謂「楚、漢戰滎陽南京索間」,屈折東北入滎澤,須水從之。京水達之。索亭一驛。鄭洛鐵路。汜水沖,州西百一十里。城北:太和山。東南:五雲。西北:河水自鞏入,逕成皋縣北,即虎牢。春秋北制所謂東虢。側有黃馬關。其南,方山,山海經「浮臝,記水出」,左納玉仙水,北逕城西入焉。爾雅「水決復入汜」。又東,板渚,入滎澤。縣驛一。鄭洛鐵路。

河南府:沖,繁。隸河陜汝道。糧捕、水利通判駐。清初沿明制,領州一,縣十三。雍正二年,陜升直隸州。靈寶、閿鄉、盧氏先後割屬。東距省治三百八十里。廣三百六十里,袤五百十五里。北極高三十四度。京師偏西四度二分。洛陽沖,繁,難。倚。城北:北邙山。東南:大石。南:周山。西南:秦山。洛水自宜陽入,右合甘水,至王城西南。澗水,即穀水,自新安入,逕穀城故城東,合孝水、金谷水來會。又東,逕王城南,至城南,瀍水亦自孟津來會。所謂「澗水東、瀍水西,惟洛食」。南有伊水,自伊陽入,右納江左河,古大狂水,又北,左合土溝、板橋、厭澗,右納小狂水,古來需水,逕前亭、伊闕口,其左龍門,右香山,左合靈巖寺水,逕右枝津,左枝渠故瀆從之。龍門、彭婆、翟莊、白沙四鎮。周南一驛。偃師沖。府東少北七十里。古西薄。縣西,帝嚳及湯所都。城北:北邙山。東南:轘轅。西:首陽。南:緱氏、景山。古陽渠、穀水故道,堙。洛水自洛陽入,伊河注之,又折東北流入鞏。伊河亦自洛陽入,逕縣西南,又東北注于洛。又合水、劉水、休水、鄩水皆注于洛。府店一鎮。首陽一驛。宜陽簡。府西南七十里。南:錦屏山、萬安城。西南:石墨。西:熊耳。洛水自永寧入,水經注:東合白馬谿、昌澗、杜陽澗。又東,左合渠谷、厭梁、黃中澗、祿泉、共、臨亭川水,又東逕九曲南,注豪水,右合黑澗、虢水,又東北出散關南,又東,枝瀆左出焉,惠水注之,入洛陽。韓城鎮,縣丞駐。又福昌、三鄉二鎮。縣驛一。新安沖。府西七十里。東南:瞻諸山。西南:鬱山。北:慕容山。南:密山。西北:隊山。河水自澠池入,逕匡口渡,合畛水。山海經「出青要山」。水經注:彊山俗名彊山水,又東入孟津,橫水從之。山海經:正回水出騩山。穀水逕爛柯山,又東逕闕門,合廣陽川,右石默谿、宋水,逕城南,又東逕函谷關,東入特阪,右合皁澗、爽慈澗水,入洛陽達之。慈澗即婁涿山。少水出瞻諸山,實亂流合澗水。白石山陂水,古澗水正源,水經注意主山海經,而並列四澗,則郭注誤之耳。匡口、楊寺、倉頭、石寺、北冶、石井、慈澗、闕門八鎮。西關一驛。鞏沖。府東北百二十里。周鞏伯邑。後東周君居。有轘轅山、九山。東南:天陵,山海經霍山,以其西宋諸陵改焉。南:侯山。西北:萯山。河水自孟津入,為裴峪渡,古小平津,右合鮪水,又東五社津、神尾山。西南:洛水自偃師合休水,逕鄩城、訾城,右合羅水、明谿泉。又東北,黑石渡,右合黃水、康水、石子河,逕城北,右合市河、魏氏河,又東神堤渡,右合任村水,為洛口,亦洛汭,入汜水,石城河從之。黑石渡、青泥、回郭三鎮。洛口一驛。孟津簡。府東北四十里。城南:邙山。西:柏崖。西北:河水自新安入,合正回水,又東合滽滽水為河清渡,後魏峽石津。又東逕漢平陰,合五曲九水,逕光武陵,至城北。又東,古孟津,逕平縣故城北,合浿水,入鞏。西南:穀城山,瀍水出,其任嶺從之。長泉、舊縣、雙槐、油房四鎮。縣驛一。登封簡。府東南百十里。北:太室山。漢置嵩高以奉,是為中嶽,古外方。其西少室,休水出,合大穴山水入偃師。其西南,大熊,山海經大虘,地理志陽乾。潁水出潁谷,是為右潁,左會中潁、左潁,逕城南,又東,左合少陽谿、五渡水,逕陽城故縣南,左合石淙水,古平洛谿,又東南入禹。其北,陽城山,洧水出,東逕陽子臺入密。西南:大虘口,狂水出,水經注「西逕綸氏故城南,左與倚薄山水合,八風谿水注之。又西得三交水口,逕缶高山北,與湮水合,又西逕湮陽城南」,入洛陽,來需水從之。縣驛一。永寧簡。府南百九十里。崤山,縣北,漢回谿阪在焉。東北:熊耳。東南:天柱。西南:金門。洛水自盧氏入,左合大溝河。水經注「東逕高門城南,東與高門水合」者。又東,松陽谿水,逕黃亭南,合黃亭谿水。又東得鵜鶘水口,右元滬山水、荀公澗口,逕檀山南,庫谷水注之。又逕僕谷亭北,左合北水。又東,侯穀水,逕龍驤城北,左合宜陽北山水,又東,右廣由澗水、直谷水,左蠡縣西塢水,又東過蠡城縣南,右會金門谿水,左合款水,黍良谷水入焉。又東,右太陰谷水、白馬谿,又東,左合北谿,入宜陽。昌澗水、杜陽谿水、西度水並從之。縣驛一。澠池沖。府西百六十里。東:大媚山。北:韶山、石門。東北:天壇、白石。西北:河水自陜入,為槐耙渡,逕桓王山,合五龍潭,又東,濟民渡,合金陵澗,入新安。西南:馬頭山躡陜。穀水出穀陽谷,逕土壕,合熊耳北阜水,水經注澠池川。又東逕俱利城,左合羊耳河,至城南,又東,左合北溪,搭泥鎮千秋亭,雍谷水、晉水從焉。崤店一鎮。南村巡司。義昌、蠡城二驛。嵩難。府西南百六十里。東北:三塗山、鳴泉。北:介立。西北:陸渾。東:惠明。西南:臥雲。伊水自盧氏入,逕郭落山北,水經注,左合滽滽水。又東北,南屈為淵潭,右合太陽谷水、鮮水、左蠻水,又東,北歷崖口,左合七谷水,逕嵩縣南,左合蚤谷水,又東北逕陸渾嶺,東,溫泉水、焦澗水、明水、洧陽水、馬懷穚水,右大戟水,左吳澗水,又東北入伊陽。伊闕前溪水從之。乾隆中,令康基淵濬新故渠二十有一。南:伏牛山,汝水出,其分水嶺石柏谷。水經注:東北逕太和城,歷長白沙口,狐白谿水注之,東入伊陽。又西北,離山,淯水出,俗名白河,東入南召。舊縣鎮,巡司駐。縣驛一。

陜州直隸州:沖,繁。河陜汝道治所。州隸之。清初沿明制,為河南府屬州,領縣二。雍正二年升。十二年,割盧氏來隸。東距省治六百八十里。廣三百三十里,袤五百四十里。北極高三十四度四十六分。京師偏西五度二十分。領縣三。東:崤山。南:常烝。西:虢山。河水自靈寶入,合橋頭溝、藏龍、青龍澗。水經注:安陽溪及譙水、橐水、崤水匯焉。有太陽津。又東逕城北為茅津渡,又東三門山,過砥柱入澠池,穀水從焉。曲沃、張茅、石壕、上村、乾壕五鎮。硤石一驛。靈寶沖,繁。州西六十里。南:秦山。西南:地肺、石城、浮山。東南:峴山、鹿氾。南:女郎。西北:河水自閿鄉入,合柏谷水、稠桑河,又東逕函谷關,合宏農澗,古門水。及燭水、田渠水,逕城北,又東合曹水。菑水入州。虢略一鎮。桃林一驛。閿鄉沖,難。州西北百二十里。南:荊山、秦山。其支閿山,其東皇天原,又西桃原,古桃林,瑕城在焉。河水自陜西潼關入,為風陵渡,逕黃卷阪,合玉溪澗,又合泉鳩澗為浢津渡,又東逕曹公壘,合石姥峪、誇父山水,即湖水,為西關渡,逕城北,又東入靈寶,稠桑河從之。關東一鎮。鼎湖一驛。盧氏簡。州西南百四十里。盧氏山,西北。西:小青。洛水自陜西雒南入,其南熊耳,禹所導。東逕城北入永寧。其支蔓渠,俗名悶頓嶺,伊水出,東北逕欒川鎮入嵩。西南:湯水,俗名黃沙五渡,入內鄉。水經注:出盧氏大嵩山。硃陽一鎮。縣驛一。

汝州直隸州:繁,難。隸南汝光道。糧捕、水利州同駐。東北距省治四百九十里。廣袤各二百二十里。北極高三十四度十三分。京師偏西三度三十六分。沿明制,領縣四。西南:崆峒山。東北:風穴山。其石樓、鹿臺、望雲、檀樹、狼皋、鑾駕諸山,皆中嶽熊耳之支脈也。西北:永安河入伊陽,逕楊家樓。水經注「趨狼皋山東出峽,謂之汝厄。東歷麻解城北,逕周平城南,又東與廣成澤水合。又東得魯公水口,合霍陽山水」者。又東逕城西南,左納洗耳河,又東,左合趙洛河,逕成安故城北,又東,黃水注之,即承休水,入郟、寶豐。楊家樓,州同駐。趙洛、臨汝二鎮。縣驛一。魯山難。州西南百二十里。東:魯山。南:簸箕。東南:商餘。西北:堯山,水經「滍水出」,故汝支津,今出西百七十里吳大嶺,俗名沙河。水經注「與波水合,又東逕魯陽故城南,右合魯陽關水,又東北合牛蘭水,又東逕應城南,彭水注之」者。又東緣寶豐界入。葉犨水從之。趙家村巡司。縣驛一。郟難。州東南九十里。北:綠石山。東南:紫雲。西北:大劉、扈陽。汝水自州緣界合扈澗水,納青龍河,入逕城南,右納石河,又東,左納藍水。水經注「逕化民城西、黃阜東」者。又東逕摩陂入襄城。長橋、黃道二鎮。縣驛一。寶豐難。州東九十里。東南:香山、扁鵲。西:鋸齒嶺。汝水自州緣郟界之西北。石河,古養水,源出三堆山,東南流,有柏河來會,又東南入郟。柏河有二源,皆出縣西山中,東流而合,又東南注石河。水蚩河即沙河,在縣東南,自魯山入,東入葉。應水一名瀴河,又名石渠,源出北峙山,東南注滍河。東:湛水,東南流入葉。宋村、曹二鎮。縣驛一。伊陽簡。州西南九十里。東南:雲夢山。南:霍陽。東北:連珠。西北:篩子垛。伊水自嵩緣界合杜水,納永定河,入洛陽。西南:汝水自嵩緣界入,逕城南,右合馬藍河,逕紫邏口,左合練溪入州。上店一鎮。縣驛一。

彰德府:沖,繁。隸河北道。糧捕通判駐。清初沿明制,領州一,縣六。雍正中,割直隸大名之內黃來隸,以磁隸廣平。南距省治三百六十里。廣三百二十里,袤二百里。北極高三十六度六分。京師偏西二度。領縣七。安陽沖,繁,疲。倚。西南:蒙賚山。西北:銅山、藍嵯、魯山、清涼山。漳水自涉入,逕邯鄲故城,緣直隸磁州界,又東逕豐樂鎮入臨漳。東南:湯水自湯陰緣界合羑水,及南萬金渠、防水,又東逕伏恩村。西有洹水自林伏入,至善應山北復出,其西龍山,合虎澗水,右歧為南、北、中三萬金渠,又北逕河亶甲城,左合珍珠泉,折東逕殷墟,韓陵山故瀆右出焉,又東南先後來會,又東入內黃。豐樂鎮,縣丞駐。鄴城一驛。鐵路。臨漳繁。府東北七十里。河故道在縣界,今已南徙。滏水、汙水並在縣西,今為漳、汙所經。漳河南自安陽、磁州入,側城西南,分二派,東至大名,並注衛河。鸕甪陂為境內蒲魚之利。三臺在鄴城內西北隅,講武城在西。漳水上曹操疑塚在焉。冰井、銅雀、金鳳。隆、鄴二鎮。縣驛一。鐵路。湯陰沖,繁。府西南四十五里。西:五巖山、柏尖。西南:淇水自林緣界,衛河自濬緣界,北逕五陵,其西鸴城。又北,普濟河出焉,緣內黃界入之。西:牟山,水經注石尚蕩水出,唐改湯,逕城北,至岳王墳東。宜師溝出西南黑山,一曰永通河,北逕高暯橋注之。又東北抵安陽界,左合羑水入之。鎮二:鶴壁、宜溝。縣驛一。鐵路。林繁。府西南百十里。林慮山,西二十里,太行支。其異目:西黃華、天平、玉泉,西南谼峪、棲霞,西北魯般門、倚陽,皆林慮之異名者也。濁漳自山西潞城入,緣涉界,左會清漳為漳水,東入河內。水經注所謂「逕葛公亭、磻陽城北合滄溪」者。其南,洹水自黎城伏入,復出為大河頭,逕城北,左合史家河、陵陽河,至龍頭山復伏。西南:淇水自輝入,逕石城、淇陽城,右會淅水,入湯陰。縣驛一。武安繁。府西北百六十里。南:鼓山。西:龍虎頭。西南:磁山、閼與。西北:摩天嶺、三門。有磨盤,南洺河出,屈東北,逕粟山,合玉帶及紫金河。其天井,北洺河出,逕儒山,合於紫金山,西入直隸永年。縣驛一。涉簡。府西北二百二十里。城北:龍山。南:熊耳。東:韓五。西南:風洞。東北:符山。東南:青頭。西北:石鼓、毛嶺口。清漳水自山西遼州入,逕城南,一曰涉河,縣以是名。又東南,濁漳自黎城緣林界來會,為合漳口,入安陽。索堡一鎮。縣驛一。內黃繁,難。府東百十里。明屬大名。雍正二年來隸。東:博望岡。河水故瀆在焉,有金堤。西南:衛河,自安陽緣界逕牽城入,左合湯水、洹水,逕繁陽城,折東楚王鎮,右合柯河,入直隸清豐。衛實淇水,水經注「過內黃縣南為白溝,逕並陽城為黃澤,逕戲陽城東」。地理志清河水。隋,永濟渠。高堤一鎮。縣驛一。

衛輝府:沖,繁。隸河北道。上北河,衛糧通判駐。清初沿明制,領縣六。雍正中,割開封之延津、直隸大名之濬、滑來隸,胙城省。乾隆中,割開封之封丘、歸德之考城來隸。光緒初,考城仍還隸。東南距省治百六十里。廣三百九十里,袤百七十八里。北極高三十五度二十七分。京師偏西二度十二分。領縣九。汲沖,繁。倚。西北:霖落、蒼峪、壇山。西:仙翁。北:華蓋。並太行支脈也。東南:河故瀆。北:衛河自新鄉入,一曰清水河,右納孟姜女河,逕府治北、比干墓南,又東北,右納滄河,緣淇界入之。銅關、杏園、淇門三鎮。驛一:衛源。鐵路。新鄉沖,繁。府西五十里。北:寺兒山、五陵岡。西南:黃、沁故瀆。東北:衛河自獲嘉入,右合小丹河及沙河,有合河鎮,又東北入汲。驛一:新中。獲嘉沖,繁。府西南九十里。東北:同盟山。南:黃、沁故瀆。西:小丹河自修武入;其新河會重泉注之,東逕三橋,左納峪河,即清水河。其西北,太白陂,春秋大陸。又東入新鄉。北流河自輝入為沙河,從之。崇寧、亢村二驛。丞兼巡司。鐵路。淇沖。府北五十里。東北:浮山。西北:靈山。西:朝陽。東南:衛河自汲合滄河,緣界納斮脛河,所謂肥泉,又東北會淇水入濬。早生、青龍二鎮。淇門一驛。輝繁。府西六十里。西:太行。其支,東北:方山。北:九山。西北:蘇門,衛河出焉,曰百泉。詩「毖彼泉水」。匯卓水、白沙、蓮花、萬泉,歷閘五,入新鄉,下至山東臨清會汶,行九百二十三里。其西:沙河,匯丁公、清濂、焦泉,又西,峪河、清水,匯梅竹、重泉,並入獲嘉。重泉,水經注長泉,逕鄧城東,又謂白屋水。淇山,西北。山海經沮洳。淮南子大號。淇水出東北,入林。縣驛一。延津沖,疲。府南七十里。雍正二年,自開封來隸。五年,省胙城入。西南:酸棗山。北:河水故瀆。西北:孟姜女河,東北流,至汲注衛河。濮水、酸水、延津、棘津、文石津,並堙,惟烏巢澤存。沙門一鎮。驛一,曰廩延。濬沖,繁。府東北百十里。城西南隅:浮丘山。東南:大伾,即黎陽山,其支,紫金、鳳皇。有禹二渠。白馬津西即遮害亭,又西,衛河。古泉源水自汲會淇入衛。詩所謂「在右」。淇口,古宿胥口。魏遏淇入白溝,所謂枋頭,即今之淇門渡也,東北逕雍榆城南,又北逕白祀山、頓丘故城。道口鎮,縣丞駐。縣驛一。滑繁,難。府東九十里。東北:白馬山、鮒鰅城。西北:狗脊、天臺,河故瀆在焉。有瓠子堤、金堤。滑水,堙。西北:衛河自濬錯緣界仍入之。老岸一鎮,巡司駐。縣驛一。封丘繁。府東南百里。南有河水自陽武入,緣祥符界入之。城北:黑山。東北:淳于岡、青陵臺,圮。古濮渠,堙。潘店、中欒二鎮。有驛。

懷慶府:沖,繁。隸河北道。河北鎮總兵、黃沁同知駐。清初沿明制,領縣六。後割開封之原武、陽武來隸。東南距省治三百里。廣三百九十里,袤百三十里。北極高三十五度六分。京師偏西三度二十七分。領縣八。河內沖,繁。倚。北:太行山。沁水自濟源入,左傳少水,水經注「東逕小沁亭北,右合小沁、倍澗水、邘水,逕野王故城北」者。其水逕柏香鎮、絺城為豬龍河,合豐稔南支,南入孟。其支津東北貫城,合利仁河,東出合廣濟支津注之。左會丹水,又東逕武德鎮,古州邑,入武陟。丹水自山西鳳臺入,為丹口,逕鄈城、苑鄉城,釃為十九渠,古光溝、界溝、長明溝故瀆在焉,並注沁。而小丹河為大,合白馬溝,逕清化鎮。廣濟河及北支豐稔自濟源入,並絕濟。廣濟復歧為二支津,並入溫。鎮七:崇義、柏香、邘臺、萬善、清化、尚香、武德。驛二:覃懷、萬善。濟源難。府西七十里。西:王屋、天壇。王屋,志稱「天下第一洞天」。天臺,道書所謂「清虛小有洞天」也。西北:析城、秦嶺、陵山。北:盤谷。東北:孔山、熊山。西南:河水自山西垣曲入,納濝水。又東,河清渡、馬渚合柴河。水經注「湛水逕向城、湛城東」者。又東入孟。★C5水源出西北山,東南流,逕城東南注溴河,逕琮山口,至勛掌村,淤。故水經注,溴出原山勛掌穀,俗謂之白澗水。側城東南,其南源姑嫂、五指、秦嶺三山水自右來會,又東南,左合濟支渠。濟出王屋西麓太乙池,為沇水,伏九十里,至共山南,復出於東丘,為濟瀆。東西二源亂流,其支南注溴。又東入河內,為豬龍河。東北:沁水自山西鳳臺入,為枋口,東南,右歧為廣濟河,古秦渠。水經注硃溝,元為廣濟河,明為二十四堰。在永福堰者利仁渠,在廣福堰者豐稔南北渠,古奉溝,與正渠並入河內。在永利堰者永利渠,又歧為二,一南注為支,一東南為餘,入。邵源鎮,巡司駐。縣驛一。原武難。府東百八十里。明屬開封。雍正二年來隸。東北:黑洋山,古漯水出。西南:河水自滎澤入,又東入中牟,天然渠從之。下至扶溝,長七十五里。縣驛一。修武沖,繁。府東北百十里。北:太行山。西北:天門。西南:小丹河自武陟入,一曰預河,逕習村,側城東北,又東入獲嘉。新河上承靈泉、劉公河,至城東北,匯皇母諸泉,入獲嘉。待王、承恩二鎮。縣驛一。武陟沖,繁。府東百里。河北道治。西南:清風嶺。河水自溫入,納廣濟河,沁河水注之,又東入滎澤。沁河自河內入,逕故懷城木欒店,側城東南,又東逕詹店入原武。廣濟河自河內入,逕縣西南注黃河。小丹水亦自河內入,逕縣西北入修武。永橋、寧郭二鎮。武陟、寧郭二驛。孟沖,繁。府南五十里。城西:紫金山。西北:五龍臺嶺。山下至梁村,古溴梁。其東,馬吉嶺。西南:河水自濟源入,逕宋河清故城,為白坡渡,古治阪津,其下吉利沾,古高渚。又東合軹陽河,其下楊樹沾,古淘渚。又東逕野戍鎮,為河陽渡,古孟津,其下郭沾:所謂「河陽三城」。古河中渚,合衡磵,又東順磵至城南,其渡小平津,又東逕沇水鎮入溫。西北:溴水自濟源入,逕冶城,右合同水,逕古安國城,合青龍澗,又南逕穀旦鎮,至無鼻城,左合餘濟南支。又南,孟港。東,豬龍河自河內緣界,合豐稔南支及餘濟北支,並從之。沇河、白陂二鎮。驛一:河陽。溫繁。府東南五十里。西:太平山。西南:河水自孟入,至小營西北。濟水自河內入,為豬龍河,緣界合豐稔北支。又有大墊水,至上浣村,仍曰沇水,逕虢公臺南,會溴水入焉,逕城南。又東至平泉西,大豐及長濟及興隆堰水亦自河內入焉,又東入武陟。趙堡一鎮。縣驛一。陽武繁。府東北九十里。西南:河水自原武入,逕官渡東入祥符。天然渠逕黃練集,東北入封丘。其河、濟故瀆西北。河自山西垣曲入郡境,凡行六百四十六里。太平、延州二鎮。縣驛一。

南陽府:沖,繁,難。隸南汝光道。南陽鎮總兵駐。清初沿明制,領州二,縣十一。道光中,淅川升。東北距省治六百十里。廣五百八十里,袤三百四十里。北極高三十三度六分。京師偏西三度五十五分。領州二,縣十。南陽沖,繁,難。倚。西北:精山、紫山。東北:豐山、蒲山。淯水俗名白河,自南召入,逕其北。水經注「逕博望西鄂故城,又南逕豫山宛城東,梅溪水注之」者。至府治南,支津南出為溧河。又西南,右合木溝、十二里河,逕淯陽城,並入新野。潦河緣鎮平界從之。東有唐河,自裕緣唐界入,桐河從之。石橋一鎮。賒旗店巡司。博望驛驛丞。林水驛驛丞。又宛城一驛。南召難。府西北百二十里。順治十七年省入南陽。雍正十二年復。南:百重山、天子望山。西:香爐。西南:燕尾、壺山。西北:伏牛、聖人。白河自嵩入,逕其東,右合獅子、黃洋河,左五路山水,至十里岡,右合留山及空山、雞子河。留即丹霞,其河即魯陽關水,水經注「逕皇后城西」者,關南三鴉水。有雉衡山,地理志醴水出,東入葉。李青店巡司。縣驛一。唐繁,難。府東百二十里。城南:天封、百里、唐子山、紫玉、午峰、花山。西北:富春。東南:孤山、馬武。東北:唐河自南陽緣界入,左會沘水及馬仁陂水,右合桐河,側城西南。左納澧河及江河、秋河,逕湖陽故城西、謝城北,合謝水、湖河,逕蒼苔鎮,緣新野界入湖北襄陽。蒼苔鎮,縣丞駐。明陽、桐河二鎮。縣驛一。泌陽簡。府東二百里。北:虎頭腦山。東:萬千。東南:祝家衡。東北:大胡,沘水出,譌「泌」,縣氏焉。左會小銅山水,逕城南,又西,比陽故城南,左合蔡水,右澳水。水經注「出磐石、茈丘二山」者,入唐,馬仁陂水從之。其支江河,與出磐石紅崖河,並入桐柏。西北扶予,潕水出,東北中陽,瀙水出,合為沙河,東入遂平。古路、饒良、羊柵三鎮。縣驛一。桐柏簡。府東南三百里。東:石門山、映山。西:天木。桐柏山在縣西南,與熊耳、伏牛聯體。其支大復、胎簪、黃山、石柱,通目之。淮水、澧水出。淮東北匯水簾洞、太陽城諸水,伏,至陽口復出,東逕尖山,東南逕復陽、義陽故城,左合月河,入湖北隨州。慄樹河從之。地理志,東南至淮陵入海,過郡四,行三千四百二十里。澧西北匯紅泥、三家,右納紅崖,逕平氏故城東,入唐。西南秋河,西北江河,自泌陽緣界自隨州入,並從之。吳城一鎮。縣驛一。鎮平簡。府西七十里。東:遮山。西北:歧棘。潦河出其東麓,緣南陽界入之,下注淯。照河,出嬌女朵,俗十二里河,匯東西三里淇河,及其西嚴陵河,並達之。縣驛一。鄧州繁,難。府西南百二十里。南:析隈山。西:五隴。西北:靈山、永青。湍水自內鄉入,逕臨湍、冠軍故城,右合得子河,側城東南至槃灘,左納趙河及嚴陵河。水經注「又逕穰縣為六門陂,又東南逕魏武故城西南白牛邑,安眾故城南,涅水注之」者,漢東陽涅陽城在焉。入新野,與淯會,為白河。其西,刁河自內鄉入,逕紅崖山,右合朝水,東南逕紫金山,為鉗盧陂,又南,黃渠河並從之。西南:禹山,茱萸河出,合排子河入湖北光化。板橋、槃灘、千金、張村、穰東五鎮。縣驛一。內鄉繁,難。府西百九十里。北:老君山。其南:秋林、夏館。山海經,翼望山,湍水出,會青山河,逕赤眉城,右合長城。又螺螄河,水經注「東南逕南陽酈故城東,菊水注之」者。逕城東又南,右合黃水,丹水故城在焉。又南,左合墨河。西北:霄山,刁河出,並入鄧。西北:熊耳山。淅水自盧氏入,逕修陽故城,一曰湯河,俗名黃沙五渡。逕菊潭,至西峽口,曰三渡河,又東南入淅川,與丹水會。丹水復逕順陽川,緣界入湖北光化。西峽口巡司。馬尾一鎮。縣驛一。新野沖。府東南五十里。北:蔓荊山。白河自南陽入,逕岡頭鎮,又西南,右合潦河,會湍水,合城東北,又西南,右納刁河,其支樊陂,折東南,逕新店鎮,左納支津漂河,復右納黃渠河。東南:唐河自其縣入,逕蒼苔鎮,右合小澗河,古安仁陂水,並入湖北襄陽。湍城一驛。裕州沖,難。府東北百二十里。東北:黃石山、方城山。東:中封。北:七峰,拐河出,醴河舊自南召入合之,今淤。東逕牛心山,洪河上游潕別源賈河出,分流東南逕小乘山復合,折東北,並入葉。西北:酈鳴山,唐河北源趙河出,南逕賒旗店,三里河即堵水,合清河、潘河、呂河注之,入唐。平臺一鎮。赭陽一驛。舞陽簡。府東北百七十里。南:牛腦山、蘇家寨、鐵山。東南:瞻山。西:馬鞍。西:千江河自葉入,逕城南,曰三里河,右合八里河,東入西平,滾河從之。北:汝水自葉入,錯襄城,有湛河來注,又東南注沙河。沙河自葉入,有輝河、澧河,亦自葉來注,又東入郾城。唐河源出城東北,東流至郾城注澧。縣驛一。葉沖。府北百三十里。西南:方城、黃城。西北:北渡。滍、汝同源,俗名沙河,自寶豐入。逕河山,至臥羊山北為汝墳,東入舞陽。北:湛河,亦自寶豐入,逕平頂山,緣襄城界。其南輝河,古昆水,水經注,出魯陽縣唐山,逕昆陽故城西。又南拐河,即醴水,自裕入,逕王喬墓南。又南,賈河自裕入,曰千江河,古潕水,自泌陽入與會,通目之。滍水、保安二驛。保安,縣丞駐。

汝寧府:沖,繁,難。隸南汝光道。汝南分防通判、新息分防通判駐。清初沿明制,領州二,縣十二。雍正二年,光州直隸。光山、固始、息、商城割隸。北距省治四百六十里。廣二百四十里,袤五百九十里。北極高三十三度一分。京師偏西二度九分。領州一,縣八。汝陽繁,難。倚。城北:天中山。北汝,汝正源。西汝,潕及澺。南汝,瀙。元季,汝溢病蔡,自舞陽堨故瀆,則潕及西平、雲莊諸山水擅之。明嘉靖中涸,則遂平灈、瀙擅之。汝源凡三易,今北汝自上蔡合澺,通曰洪河。右合硃馬、馬常,左茅河,逕廟灣鎮,右合荊河,其故道蔡埠河入會。南汝右納黃酉、吳桂橋河,左迤為懸瓠池,右慄渚,側城東南,右合半截河,納溱水,錯正陽復入,並入新蔡。廟灣鎮巡司。黃岡、陽埠、射子、寒凍四鎮。縣驛一。正陽繁。府南百二十里。明真陽。雍正二年改。西:橫山。東北:南汝河自汝陽錯入,右合固城港、陳家溝,仍入之。水經注,首受慎水於慎陽故城南陂,注七陂,東入汝。南有淮水,自信陽緣界入息。西南:閭河、清水港並自確山入,又東從之。汝南埠,通判駐。縣驛一。上蔡繁,難。府北七十里。東:蔡岡。西北:北汝自郾城入,西汝、潕水右自西平會澺來注,遂通曰洪河,東南絕蔡河入汝陽,茅河、硃馬、馬常河從之。其故道自西洪橋右出納流堰為硃里河,通目之。復納石洋河,為蔡埠河,其西水親水即南汝,自遂平入,右合清水河,亦並入汝陽。蔡河,澺支津,水經注「東南流為練溝,至上蔡西岡,北為黃陵陂,於上蔡岡東為蔡塘」者。又東為包河,入項城。北:華陂集,界溝河出,東緣商水界入之。邵店一鎮。縣驛一。新蔡簡。府東南百四十里。南汝,瀙,即汝水,洪河,澺,並自汝陽入,合於城東五里三汊口,又東南入息。又安徽阜陽谷水,即鮦水,從之;延河亦入焉。水經注「汝水逕櫟亭北,又東南逕新蔡故城南,又東南,左會澺水,逕壺丘故城北,澺水逕平輿故城南,左迤為葛陂」者。漢葛陵故城在焉。縣驛一。西平沖,繁。府西北百二十里。西:九頂山。潕水舊自舞陽入,逕故城。水經注,其西有呂墟,至合水鎮,匯諸石、雲莊諸山水。逕城北,又東歧為二,左支合周家泊水,古澺水。水經注「上承汝水,別流於奇額城東」者,今淤。泥河,緣郾城界,復合右支,會流堰河,並入上蔡。潕即西汝,自元季於舞陽鍋河堨之,今雲莊諸山水擅其故瀆。又會澺水,因通曰洪河。重渠、蔡砦、儀封三鎮。縣驛一。鐵路。遂平沖,繁。府西北九十里。西:奧崍山、嵖岈。南汝上游沙河,古瀙水,自泌陽入,逕金山,左合楊奉河。水經注「東過吳房縣南,又東過灈陽縣南」者,入上蔡。其逕城南支津,東北出為新河,會石洋河。河古灈,出西北嵢峰垛,水經注興山。逕吳家橋東南,清水河自確山入,並從之。縣驛一。確山沖,繁。府西南九十里。確山,城東南二里。又東南,朗陵、佛光。城南:蟠山。西南:平頂。西北:樂山,練水出,俗名黃酉河。秀山,吳桂橋河出。西有溱水自泌陽入,俗名石河,又東曰吳砦,逕確山故城。水經注謂「溱出浮石嶺北青衣山」,又東北逕獨山,並入汝陽。東南,閭河塘、下溝河、清水港,並入正陽。西北,清水河,入遂平。姬家堰。毛城、竹溝、明港三鎮。縣驛一。信陽州沖,繁,難。府西南二百七十里。東南:鍾山。南:士雅、峴山。西南:董奉。西:卓斧、堅山。西北:淮水自湖北隨州入,左合明港河,屈東緣信陽界入羅山。水經注「逕平春城陽鍾武故城南」。其溮水入合油水、三灣河、九渡水,逕城南從之。平昌關,州判駐。楊家堂巡司。信陽、明港二驛。京漢鐵路。羅山繁,難。府南二百三十里。羅山,城南十里。又南:獨山、鵲山。西南:黃神、霸山。皆桐柏支麓也。西北:淮水自信陽入,逕謝城合溮水,又東逕縣北。西南:六斗山,竹竿河出。水經注谷水,合黑龍池、小黃河、古瑟水,緣光山界注之,入息。大勝關,巡司駐。縣驛一。

光州直隸州:繁,疲,難。隸南汝光道。鹽捕、水利通判駐。清初沿明制,為汝寧屬州。雍正二年升直隸州。北距省治八百里。廣二百四十五里,袤二百里。北極高三十二度十三分。京師偏西一度二十八分。領縣四。州,古黃國。故城,西十二里。東:鳳皇山,為州左翼。西:浦口岡,為州右翼。東南:彭山。南:車谷。西北:淮水自光山入,合寨河,古壑水,又東北逕鄭家店,復合黃水。水經注「逕弋陽郡東,又東入固始」。其雙輪河入為白鷺河,古渒水。及春河自商城緣界,古詔虞水,並從之。州驛一。光山繁,難。州西南四十里。古弦子國。縣境大半山區,自西北而來,綿亙近二百里。其最著者,老君山、天臺、春風嶺、黑石諸山。老君山之北,雲臺、仙居、馬鞍、守軍、浮光諸山,皆桐柏支脈也。地理志弋山,西有淮水自羅山合竹竿河,緣界逕軑縣故城至其麓。又東入州。西南:黃茅腦,寨河出。水經壑水。會馬鞍山水為清流河,又合牢山龍潭、沖水、泥河,其東黃水,至花石山為三道河。右合梅林河,逕塔山,右合潑陂河。水經注木陵關水。左合晏家河,逕黃川西陽故城,至城南為官渡河,逕天賜山,水經注渒水。又東雙輪河,並從之。中渡、牛山二鎮。長潭一驛。固始繁,疲,難。州東瑾四十里。東:大山。南:獨山、木賊、青峰嶺。西北:淮水自息入,逕棗林岡、安寧、期思。古蔣國,亦浸丘故城,其左岸會汝水,至硃皋鎮,納白鷺及春河。又東,往流集,巡司駐。至三河尖,決水、灌水入焉。決自商城入,為史河,左合長江河,右歧為泉河,古陽泉水水經注,自雩婁東北逕雞備亭,過安豐故城,邊城郡治。又逕茹陂。陂今龍潭口。右歧為清河,合勝湖,又西北逕史家故城,左納羊行河、急流澗,逕城東而北,古蓼國在焉。灌自商城入為曲河。淮南子「孫叔敖決期思之水,以灌雩婁之野」。又西北,逕蓼潭,至城北來會,為兩河口。東魏澮州在焉。又東北,右歧為堪河,迤為七里岡,復與清泉二支津合。又北入淮。淮水又東入霍丘。硃皋、期思二鎮。縣驛一。息繁,疲,難。州西北九十里。西有淮水自羅山入,又東逕白公城,至城南。又東,新息故城。分流,左納清水港,合泥河,復合閭河,自正陽入,蓋慎水故瀆,逕褒信長陵故城注之。水經注申陂水。又東逕烏龍集入州。其白鷺河入逕期思集。西北:汝水自汝陽入,入新蔡,復緣安徽阜陽界逕固城汛,並達之玉梁渠。楊莊一鎮。縣驛一。商城難。州東南百二十里。東南:大蘇山,古大別。南:花陽、馬頭。東北:青山。西南:牛山,決水出。水經注「出廬江雩婁縣南大別山」。東合八仙臺、黃昏山五關水。又東北曰寨河,左合麻河,逕金家寨,其西北則長江、石槽、沙河。西南:黃柏,灌水出,北合木廠、盛家店、九水河,逕城西,亦曰龍潭河,並入固始。西北:熊山,春河出。水經詔虞水。亦緣固始界入州。牛食畈巡司。縣驛一。

淅川直隸:繁,難。明復析內鄉置縣。道光十二年為。宣統元年升,改南汝光道為南汝光淅道。西:岵山。西北:簧鎖里。丹水自陜西商南緣界逕荊子關,其北葛花山,其南丹崖。又東南,逕凌老龍山,其黑漆河入為淇河,逕花園關、岞客、獨阜山注之。至城西南納滔河,逕石杯、雷山至於村保,古商於三戶城在焉。左會淅水。又東南逕太白、玉照山,緣內鄉界入湖北均州。水經注「丹水自三戶城逕丹水故城南、南鄉縣北,右合汋水」。汋即均,形之誤。荊子關,縣丞駐。峽口一鎮。驛一。


\end{pinyinscope}