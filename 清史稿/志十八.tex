\article{志十八}

\begin{pinyinscope}
災異四

洪範曰:「金曰從革。」金不從革,則為咎徵。凡恆暘、詩妖、毛蟲之孽、犬禍、金石之妖、白眚、白祥皆屬之於金。

順治元年八月,蒼梧旱。三年,平樂、永安州大旱,二月至八月始雨;臺州自三月不雨至於五月;紹興府自四月至八月不雨;金華府屬旱;東陽自四月至九月不雨;浦江旱;南昌各府自五月至十月不雨,大旱。秋,萍鄉、萬載大旱。四年夏,通州旱。秋,開化、江山旱。五年夏,饒平旱。六年,吉州自春徂夏旱。七年夏,萬泉旱。八年,甘泉、延長、安定自四月至九月不雨;崖州不雨,逾年三月乃雨。九年春,銅陵、無為、廬江、蕪湖、當塗旱。五月,上海亢旱。九月,武強旱。十年夏,樂亭旱。秋,海寧、高郵旱。十一年四月,天臺大旱。七月,襄垣、沁州旱。十一月,武強旱。十二年正月,順德大旱。四月,金華屬五州旱。五月,鄒平旱,遂安自夏徂秋不雨。八月,昌樂、曲江、湖州、衢州、龍門、開化、江山大旱,禾盡枯。十月,揭陽、全椒旱。十三年春,章丘、潞城、高平、沁水旱。九月,揭陽大旱,深潭俱竭。十四年五月,蕭縣、太湖旱,湖井盡涸。八月,涇陽、商南旱。十五年八月,昌樂大旱。十一月,龍州旱,逾年四月乃雨。十六年五月,惠來、思州、玉屏、安南旱。十七年,三水春旱不雨,至小滿乃雨。秋,鎮海、惠州、天臺旱。十八年,寧波、東陽自夏徂秋不雨,南籠府、海鹽、壽昌、江陰、東阿、蒲州旱。八月,餘姚、臨安、嚴州、桐鄉旱。

康熙元年九月,昌黎旱。二年二月,東莞、鄆城旱,六月始雨。四月,江陰旱。五月,萬載、黃州旱。六月,懷來旱。八月,保安、羅田、蕭縣旱。三年春,交河、邢臺、內丘、揭陽旱。夏,長山、平原、禹城、臨道、武定、阜陽、鄒縣、費縣、定陶、莘縣、華陽、寧海旱。四年春,朝城、城武、恩縣、堂邑、夏津、萊州、東明、靈壽、武邑大旱;高密自三月至次年四月不雨,大旱。夏,登州府屬大旱。七月,文水、平定、壽陽、孟縣、代州、蒲縣旱。八月,兗州府、濟寧州旱。五年二月,揭陽旱。三月,三水旱。五月,鍾祥、大冶旱。六月,寧海、衡州旱。秋,宣平、松陽大旱,至次年四月始雨。六年春,廣州、惠州、海豐、惠來旱。四月,黃州府屬旱。五月,應山、黃安、蘄水、羅田大旱,萬載自夏徂秋不雨。七年六月,黃安、羅田、懷安、西寧、龍門旱。七月,靜海旱。八年七月,臨海旱。九年春,開州、東明、蠡縣、廣平、任縣、武清、大城、景州、慶雲、靈壽、沙河、磁州、元城大旱無麥。夏,東陽、羅田旱。冬,棗陽、安陸、德安大旱。

十年春,霸州、公安、石首旱。四月,龍山、黃安、麻城、廣濟大旱,金華府屬六縣,五月不雨至於九月;湖州大旱,自五月至九月不雨,溪水盡涸;桐鄉大旱,地赤千里。六月,鄞縣、象山、寧海、天臺、仙居、烏程、蘭溪旱。七月,齊河、東明、邢臺、廣平、江浦、蘇州、鎮洋、任縣、成安旱。八月,太湖、新城、唐山、西寧、懷安旱。九月,紹興屬八縣大旱。十一年春,芮城、解州旱。四月,福山旱。五月,高密大旱。八月,臨朐旱。十二年,揭陽春、秋旱,惠來春旱。夏,陽信旱。九月,高明、興寧大旱。十三年春,樂陵、許州、剡城、費縣旱。四月,濟南府屬旱。六月,高郵、★陶、恩縣旱。七月,鄖陽、黃安、麻城、羅田旱。十四年六月,海寧旱。七月,黃安、羅田旱。十五年春,興寧旱。十六年,湖州、萬載自五月至七月不雨,大旱。十七年春,東流、壽州、全椒、五河、泰安旱。夏,桐鄉、嘉定、黃岡旱。八月,金華、寧州、高淳旱。十八年春,滿城旱。四月,杭州旱。萬安、羅田、宜都、麻城、公安自五月至八月不雨,大旱。蘇州、昆山、上海、青浦、陽湖、宜興大旱,溪水涸。六月,萊州、平度旱。七月,合肥、廬江、巢縣、無為、舒城、當塗大旱。九月,臨縣大旱。十九年夏,蠡縣旱。秋,開建、連州、翁源旱。十一月,萬全大旱。

二十年春,安丘旱。夏,溫州、寧波旱。井泉涸;奉化秋冬無雨,井竭;黃巖、仙居、太平、義烏旱,井泉涸。二十一年五月,連平旱。九月,博田、北流旱。二十二年,揭陽自正月至四月不雨。三月,黃縣、惠來、普寧旱。夏,汶上、鄒縣、兗州、曲沃旱。七月,太平旱。二十三年,彭水、壁山自五月至八月不雨。六月,蓬州、鄰水、興安、漢陽、安邑、洵陽、綏德州、秦州旱。秋,邢臺、棗強、獲鹿、井陘、酆都、遂寧、巫山旱,井涸。二十四年春,安定旱;瑞安、曲江、樂昌春夏不雨,井泉竭。二十五年,恭城自五月至八月不雨。六月,沁州、普州、★城、饒陽旱。七月,孝感、黃安、麻城旱。二十六年四月,樂昌旱;嚴州自五月至八月不雨,禾苗盡槁;鄞陽夏秋大旱。七月,開建、鶴慶、海豐旱。二十七年,瑞安自夏徂秋不雨;湖州、寧州旱。二十八年,羅田、石首、枝江旱,自五月至九月不雨;宣平自夏徂秋旱,井泉涸。六月,萬全、景州、清苑、新安、獻縣、東光、普州、曲陽、武強、沙河旱。秋,開建、應城旱,河水涸。二十九年四月,湖北全境旱。六月,樂平旱,竹溪自夏徂秋旱。

三十年春,開平、揭陽、化州旱;陽春自正月至四月不雨。五月,介休旱。七月,邢臺、懷安旱。三十一年三月,臨潼旱。夏,孝感旱。九月,青浦旱。三十二年,杭州、嘉興、海鹽自春徂夏大旱,禾盡槁。六月,桐鄉旱。七月,震澤、昆山、嘉定、青浦、丹陽大旱,河水涸。三十三年秋,黃岡、蘄水、黃安、廣濟、江夏、武昌、興國、大冶旱。三十四年夏,長寧、馬邑旱。秋,永寧州、臨縣旱。三十五年四月,臺州旱。五月,靜樂、衢州旱。秋,永安州、平樂、蒼梧旱。三十六年春,陽江、陽春、永安州、平樂旱。六月,順德旱。八月,桐廬、松陽旱。三十七年四月,豐樂旱。五月,銅陵旱。三十八年三月,黃陂旱。夏,杭州、桐鄉旱。秋,武昌、陽湖旱。三十九年二月,湖州旱。五月,沙河旱。秋,常山旱。

四十年五月,堂邑旱。六月,蘭州、河州旱。九月,瓊州旱。四十一年,高郵大旱。四十二年三月,宣平旱。五月,橫州旱。六月,連州旱。四十三年春,青浦、沛縣、沂州、樂安、臨朐旱。五月,靜寧州、衢州旱。六月,絳縣旱。八月,永平旱。四十四年春,朝陽旱。四月,羅田、上海旱。九月,鉅鹿旱。四十五年春,瓊州旱。五月,黃巖旱。四十六年夏,池州、石門、湖州、海鹽、桐鄉旱,河港皆涸。秋,臨江府屬、當塗、蕪湖、東流、含山、歷城旱。四十七年夏,東平、平原、霑化、臨朐旱。秋,黃岡、恩縣、茌平、臨清旱。四十八年四月,溧水旱。秋,武進、滿城旱。冬,湖州旱。四十九年二月,揭陽、澄海旱。五月,臨朐、新城、武強旱。秋,湖州、臺州、仙居旱。

五十年七月,應城、枝江、德安、羅田旱。五十一年五月,固安、定州、井陘、清苑旱。九月,崖州旱。五十二年夏,臺州、常山旱,至十月不雨。秋,五河大旱。五十三年春,臨朐旱。五月,宣平、東明、元氏旱。六月,臺州、蘇州、震澤、陽湖旱,景州夏、秋旱。五十四年春,翼城、陽江、解州旱。六月,銅陵、合肥旱。七月,鶴慶旱,惠來自八月歷冬不雨。五十五年二月,海豐、朝陽旱。五月,揭陽、福山、密雲、懷柔旱,常山夏、秋旱。五十六年,福山旱。五十七年春,南昌旱。四月,臨朐旱。秋,崇陽、寧陽旱。五十八年二月,曲阜旱。夏,福山、常山、縉雲、峽江旱。八月,義烏旱。五十九年夏,東平、岳陽、曲沃、臨汾、湖州、桐鄉、石門旱。秋,臨朐、沁州旱。六十年春,興寧、全州、安州、臨安、登州、西安、延安、鳳翔旱;懷柔自春不雨至五月,二麥無收;鶴慶春、秋旱;慶遠府大旱,自正月至七月不雨,田禾盡槁;桐廬自五月至七月不雨,禾盡枯;橫州自六月至九月不雨;昌化、桐鄉、海寧旱,河涸。七月,宣平、嵊縣、寧都、黃岡、房縣旱。八月,夏津旱。六十一年二月,濟南旱。六月,武進、無為、含山、青城、海寧、湖州、寧津旱;祁州夏、秋旱;松陽、鍾祥、江陵、荊門旱。

雍正元年春,元氏旱。夏,海寧、湖州、桐鄉、井陘、武進、祁州、莒州、蒙陰、東昌旱。秋,雞澤、嘉興、蘇州、高淳、昆山大旱,河水涸。二年,鶴慶自二月至八月不雨。夏,海寧、嘉興旱。七月,景州、景津、全州旱,井泉涸。三年春,霑化、莒州旱,河涸。夏津春、夏旱。七月,全州、丘縣旱。四年春,壽光旱。五月,英山旱。五年六月,慶陽府屬旱。六年五月,洛川旱;興安自七月旱至次年二月始雨,竹木盡枯。七年春,元氏旱。八年八月,東光、滄州旱。九月,邢臺、平鄉、沙河、揭陽、長治旱。十年春,平原、曲阜、莒州、北鄉旱;沂州自正月至六月不雨。六月,臨清、福山旱。十一年春,同官、常山旱。八月,濟南府屬旱。十二年春,膠州旱。六月,同官、甘泉旱。十三年五月,夏津、壁山、池州旱,湖水涸。七月,蒲圻、鍾祥、當陽、宜都、江夏、崇陽、蘄水旱。

乾隆元年,潮陽旱。二年三月,會寧、東安旱,無麥;玉田春、夏大旱。六月,漢陽、黃陂、孝感、黃岡、麻城旱。九月,獲鹿、欒城、平山旱。三年,鹽城自二月至六月不雨,大旱,赤地千里。夏,震澤、清河旱。九月,武進、鹽城旱。四年春,蘄水、高郵旱。夏,通州、潛山、銅陵、合肥、廬江、青浦、無為、東流旱。秋,漢陽、黃陂、孝感、鍾祥、京山、天門、武昌旱。五年六月,全州旱。六年,嘉應、崖州春、夏旱。七年春,廣寧、鶴慶、龍川、潮陽、饒平、普寧旱,陽江春、夏旱。八年春,壽州旱,新安自春徂夏不雨。四月,銅陵旱。閏四月,★城旱。六月,德州、武強、正定、河間、寧津、衡水旱。冬,武昌府屬旱。九年四月,西清、慶平、高邑、寧河旱。七月,武定府屬旱。

十年五月,三河旱。秋,元氏、邢臺、棗強、懷來、正定、無極、★城、樂平、代州旱,晚禾皆★。十一年,雲都自五月至七月不雨。十二年春,即墨、平度旱。夏,文登旱。秋,高密、安邑、垣曲旱。十三年三月,臨安旱。五月,嘉興、石門旱。六月,芮城、懷來旱。十四年十

月,大同府屬旱。十五年春,惠來旱。五月,交河、蘄城旱。秋,連州旱。十六年七月,溧水、連州、惠來旱;建德、遂安、淳安、壽昌、桐廬、分水夏、秋不雨,禾苗盡槁。十七年春,房縣旱,解州自五月至七月不雨。秋,海寧、富陽、餘杭、臨安、杭州、雷州、諸城、寧鄉旱。十八年,桐廬春、夏旱,禾苗枯,井泉涸;廣靈自五月至九月不雨。秋,唐山、樂清、平陽旱。十九年,荊門州大旱,至二十一年始雨。

二十年三月,普寧旱。五月,梧州旱。七月,黃縣旱。十一月,武進旱。二十一年,金華春、夏旱。五月,桐鄉、天門旱。二十二年春,龍川大旱,惠來自春徂秋不雨。夏,石門、梧州、桐鄉旱。二十三年三月,東平旱。六月,慶陽旱。二十四年,平定、樂平、盂縣春、夏大旱。六月,枝江、高郵、太原旱。秋,代州、翼城、寧州、寧鄉、安邑、絳縣、垣曲、潞安、河津、應州、大同、懷仁、山陰、靈丘、豐鎮、甘泉、新樂旱。二十七年夏,會寧、湖州旱。二十八年,武昌旱。二十九年夏,寧津、東光旱。

三十年夏,洛川旱。三十一年秋,文登、榮成旱。三十二年,湖州旱。三十三年四月,陽湖、高郵旱。六月,日照、石門、嘉善旱,連州夏、秋大旱。七月,孝感、安陸、雲夢、應城、應山、武昌、鍾祥、棗陽旱。八月,泰州大旱,河竭。三十四年六月,高淳旱。三十五年夏,臨潼、珙縣旱。七月,常山旱。三十六年二月,即墨旱。夏,五河旱。冬,瑞安、當陽、宜城旱。三十七年春,文登旱。秋,宣平旱。三十八年夏,洛川旱。七月,壽光、宣平、天津、青縣、靜海、武清、東光、寧津旱。三十九年七月,鍾祥、荊門州、應城、黃安旱。八月,秦州、鎮番、慶雲、南樂、霸州旱。

四十年六月,杭州旱,九月兼旬不雨;房縣、溧水、武進、高郵、文登、榮成旱。四十一年秋,平定、樂平旱。四十二年夏,洛川、穀城、歸州旱。八月,吳川、武寧、宣平旱。四十三年,太原自正月至五月不雨,諸城旱。夏,嘉興、石門、東平旱,河涸。秋,江夏、武昌、崇陽、黃陂、漢陽、鍾祥、潛江、保康、枝江旱。冬,九江武寧旱。四十四年六月,湖州、武城、安丘、泰安、潛山旱。四十五年五月,應城旱。四十六年四月,宣平旱。六月,金華、新城旱。四十七年春,文登旱。五月,黃縣旱。六月,羅田旱。秋,綏德州旱。四十八年二月,文登、榮成、綏德州旱。秋,菏澤旱。四十九年二月,寧陽、菏澤旱。三月,大名府屬七州縣旱。五月,應城旱。秋,寧陜大旱,長安河涸。

五十年二月,江夏、武昌旱,濟南、菏澤自春徂夏不雨。夏,鄒平、臨邑、東阿、肥城、滕縣、寧陽、日照、嘉善、桐鄉、宣平、蘇州、高淳、武進、甘泉皆大旱,河涸。秋,太平、觀城、沂水、壽光、安丘、諸城、博興、昌樂、黃縣旱。五十一年春,東平旱。五月,洮州旱。七月,荊門州、松滋旱。五十二年三月,黃縣、博興旱。夏,滕縣大旱,微山湖涸。五十三年三月,黃縣復旱。五十四年,宜都大旱。自三月至五月不雨。五十六年五月,應山大旱。五十七年,歷城、霑化、黃縣春早。秋,順德、武強、南宮、慶雲、靜海、望都、蠡縣、樂亭旱。五十八年,陵川自二月至三月不雨,保定、大名、元城、東光春旱。七月,德平旱。五十九年三月,文登、榮成旱。秋,黃縣不雨至冬。六十年春,鄒平、壽光、昌樂、諸城旱。五月,蓬萊、黃縣、棲霞、江山、溪陽旱。秋,文登不雨。

嘉慶元年春,浦江旱。五月,穀城、麻城旱。夏,洛州、懷遠旱。秋,漁陽旱。二年五月,江陵旱。三年四月,黃安旱。五月,青浦旱。六月,文登、榮成旱。四年夏,江山大旱。五年春,枝江旱。夏,安康旱。六年春,章丘旱;榮成夏秋大旱,草木盡枯。七年四月,京師旱。五月,金華、江山、常山旱。六月,武昌、漢陽、黃川、德安、咸寧、黃岡、安陸旱。八月,宣平、嵊縣、南昌、臨江旱。八年,江山自春徂夏不雨。九年二月,臨朐旱。夏,漢陽旱。秋,定平旱。十年六月,章丘大旱。十一年夏,泰州旱。十二年二月,武進、黃縣旱。四月,樂清旱。五月,崇陽、石首旱。七月,宣平旱。八月,灤州不雨。十三年春,樂清不雨,黃安春、夏旱。十四年四月,邢臺、應山旱。十五年,安丘春、夏大旱。十六年春,黃縣旱。四月,京師、臨榆、撫寧旱。五月,永嘉、麗水、縉雲、景州、嵊縣、鍾祥、房縣、江陵、宜都旱。六月,曲陽、蓬萊、招遠、寧海、文登、即墨旱。秋,觀城、臨朐旱。十七年春,東阿、滕縣、高唐旱。十八年春,東平、東阿、濟寧、曹縣旱。夏,保康旱。八月,鄖縣、麻城、鍾祥、襄陽、棗陽旱。九月,樂清、寧津、南樂、清苑、邢臺、廣宗、井陘、清豐、武邑、唐山、望都、南宮旱。十九年春,應城、鄖縣、蘄水、羅田旱。夏,嘉興、新城、湖州、石門、鍾祥、武進、臨朐、定遠、泰州、通州皆大旱,河盡涸。七月,青浦、蘇州、高淳旱。二十年六月,嘉興旱。七月,灤州旱。二十一年九月,麗水大旱。二十二年四月,曲陽旱。秋,長清、觀城、博興、蘇州、定州、諸苑、固安、武強、涿州、清苑、無極、廣宗旱。二十四年六月,貴陽、湖州、石門旱。八月,應山、麻城旱。九月,黃陂旱。二十五年,新城自二月至七月不雨。五月,黃梅大旱。八月,縉雲、麗水、嵊縣、南昌、建昌、臨江、贛州、袁州、武昌、咸寧、崇陽、金華、常山旱。

道光元年秋,黃巖、龍泉旱。二年春,宜都、日照旱。夏,嘉興、湖州旱。三年夏,滕縣大旱。四年,宜城自四月至六月不雨,曹縣、房縣、麻城旱。秋,章丘、榮成旱。五年六月,應山旱。七月,歷城、黃縣旱。六年春,諸城、東阿旱。六月,永豐、萬安旱。七年七月,內丘大旱。九年,湖州夏、秋旱,宜城八月不雨至於十月。十年夏,湖州旱。秋,武強、唐山旱。十二年春,昌平大旱,六月始雨;內丘、懷來、萬全、望都旱。夏,嘉興、湖州、嵊縣旱。七月,東光、靜海旱。九月,陵縣、臨邑、鄒平、新城、博興旱。十三年春,武清旱。夏,皋蘭、狄道州旱。十四年春,孝義旱。秋,定海旱。十五年春,元氏、臨朐、枝江、宜都、宜昌旱,黃巖自五月至七月不雨,縉雲自五月至八月不雨。夏,湖州、永嘉、麗水、嵊縣、宜城、穀城旱。七月,房縣、黃州、安陸旱。冬,太平、玉山、武昌旱。十六年春,登州府屬旱。夏,應城、皋蘭、狄道州、孝義旱。十七年,臨朐自正月不雨至於五月。六月,雷州旱。七月,元氏、阜城、邢臺旱。十八年夏,常州、應山、靖遠旱。八月,阜陽二十一州縣旱。十九年三月,武強、懷來旱,望都春、夏無雨。秋,莊浪大旱。二十年,皋蘭、狄道州、金縣旱。二十一年九月,寧陽旱。二十三年七月,湖州旱。二十四年,光化秋、冬旱。二十五年六月,青田旱。七月,縉雲、雲和旱。二十六年六月,藍田、三原大旱。二十七年夏,宜城大旱。秋,麗水大旱。二十八年春,永嘉旱。秋,昌平旱。二十九年七月,莊浪大旱。三十年夏,嵊縣、太平、寧陽、皋蘭旱。

咸豐二年,定海、常山旱。四年五月,麗水旱。七月,咸寧、保康旱。五年正月,皋蘭旱,四閱月不雨。四月,青縣旱,武昌夏、秋旱。六年,宜城、安陸自夏徂秋不雨,樹木多枯死。五月,咸寧、桐鄉、黃陂、鍾祥、潛江大旱,河水涸。閏五月,隨州大旱,至九月始雨。六月,嘉興、蘇州、青浦旱。七月,武進、羅田、通州、肥城、陵縣旱,河水竭。七年春,昌平、唐山、望都旱。夏,清苑、元氏、無極、武邑、永清、廣宗、柏鄉旱。八年夏,青縣旱。九年春,即墨旱。夏,臨朐、濱州、黃縣旱。七月,元氏、灤州旱。十年春,清豐、蓬萊、皋蘭旱。六月,青縣大旱。十一年,青縣春、夏不雨。七月,太平旱。八月,皋蘭、通渭、秦安大旱。

同治元年二月,青縣旱。六月,孝義、皋蘭旱。七月,葦縣、棲霞、咸寧、江夏旱。二年,嵊縣旱。三年夏,常山旱。秋,崇陽、撫寧旱。四年春,蘄水大旱荒,民有鬻子女者。秋,麻城旱,高鄉自冬至次年夏不雨。五年夏,江夏、江山旱。九月,崇陽、漢陽旱。六年夏,昌平、玉田、黃陂、荊門、德州旱。秋,邢臺、懷來、武昌、黃州旱。七年春,皋蘭旱。冬,陵縣旱。八年春,青縣旱。九年春,新樂、黃縣旱。十年春,清苑大旱,無麥。十一年,皋蘭春、夏旱。十二年五月,公安、枝江旱。十三年三月,江陵、公安、枝江旱。秋,均州旱。

光緒元年,青縣夏、秋旱。二年春,望都、蠡縣、灤州、臨榆旱。五月,肥城旱。八月,★城旱。三年四月,武進、霑化、寧陽、南樂、唐山旱,應山夏、秋大旱。四年春,東平、三原旱。七月,內丘、井陘、順天、唐山、平鄉、臨榆旱。八月,京山旱。六年秋,甘泉、魚臺、邢臺旱。八年六月,均州、雲夢、鶴★州旱。十一年秋,東光旱。十三年七月,靖遠、東光旱。十六年,皋蘭春、夏旱。十七年,靜寧、合水旱。十八年六月,皋蘭、金華、靜寧、通渭、洮州、安化旱。十九年五月,太平旱。二十年,太平自七月至十月不雨,大旱。二十一年六月,太平旱。二十四年九月,寧津旱。二十六年六月,涇州、皋蘭、平涼、莊浪、固原、洮州旱。閏八月,南樂、邢臺旱。二十七年春,皋蘭、平涼、莊浪、固原、洮州大旱。三十三年,皋蘭旱。三十四年八月,蘭州、靜寧大旱。

宣統元年,甘肅全省亢旱。

順治元年十一月十二日,鹽亭山頂崩一大石,如數間房,橫截路口,是夕大風雨,居民避張獻忠者得脫大半。先是有童謠云「入洞數,鉆巖怪,沿山走的後還在」,至是果應。

康熙十四年,籓王尚可喜於■秀山築壘,土中得一石碑,其碑文云:「抱破老龍傷■秀,八風吹箭入陀城,種柳昔年曾有恨,看花今日豈無情?殘花已自傷零落,折柳何須關廢興,可憐野鬼黃沙跡,直待劉終班馬鳴。」似詩似讖,未有能識者。五十七年八月初一,鍾祥火災,先是有童謠云:「八月初一火龍過」,至是果應。

乾隆六年,知州林良銓改修諸葛忠武祠,掘地得二石人,一背銘字云「守土守三分辛苦」,一背金雋字云「遇隆則興。」

光緒五年,文縣有童謠云「兩個土地會說話,兩個石人會撻架」,未幾即山崩地震。

順治七年正月朔,衢州黑熊入城,是年多火災。

康熙二年十一月,平度民間獲兔,八足、四耳、兩尾。二十七年十二月十六日,有黃熊鳴於合浦西門,十七夜復鳴。

乾隆十八年,畢節熊入城,傷二人。

嘉慶七年,陸川有熊傷人。二十三年七月,黃縣有熊走入荏苒村,土人以槍殺之。

順治八年,泰山元君廟鐘鼓自鳴。

康熙十八年正月,六安州金鐵出火。三十九年十一月,海陽馬王廟鐘自鳴,越三日復鳴。

雍正九年五月,七姑廟鐘自鳴。

乾隆三十八年十二月除夕,黃縣叢林、冶基、寶塔三寺鐘鼓自鳴。

嘉慶十四年冬,泰州雨箭。

咸豐八年四月,魚臺兵器夜吐火光。

同治三年九月,東嶽廟鐘自鳴。八年九月,彭澤長嶺酒店釜鳴,聲聞數里許,月餘方止。

順治四年,崇明民家犬生六足。七年,商州民李旺家有犬坐坑上,作人言:「老的忒老,小的忒小。」縛而殺之。十三年,鄒平民生子,犬頭猴身,能吠。

康熙四十五年二月,蕭縣民家犬作人言。

乾隆二年,利津民家犬生一畜,一首二尾七足。

咸豐十一年,來鳳民家犬作人言。

同治十一年,大埔民家犬生六足。

順治元年二月,興國寺前出白氣一道。六年三月,江陰白氣■天,彌月始滅。七年正月二十六夜,昆山西方有白氣如練,十餘日始滅;蕭縣白氣見西方,二十餘日始滅。六月甲申,泰安見白氣■天,益都見白氣■天。十二月三十日,蕭縣見白氣如練數十條,寒光射人。十八年十二月十二日,棲霞白氣■天。

康熙二年夏,萊陽有白氣沖天。七年正月,廣平見白氣■天,西出指東,越二十日方滅;內丘夜見白氣如銀河,經五六日方滅;溫江有白氣,自西直■數十丈,下銳上闊,光如銀,形如竹,經四晝夜方散;威縣見白氣■天。二月,廣州有白氣如槍,長十餘丈,四十日乃滅;武邑夜白氣■天,夜半始散;唐山見白氣■天。七月,高邑夜見白氣如疋布,■西方。九年三月乙丑,廬陵白氣現自西方。十一月,通渭夜見白氣如虹,自南而北。十一年七月十四日夜,交河有白氣自西南向東北,其疾如飛,聲如風。十六年七月壬申夜,盧龍有白氣如霓,自東向日。十八年六月二十四日,武定見白氣貫天。十一月,玉田有白氣自西南來。十九年十月,全椒見白氣於西方,月餘始滅。十一月朔,滄州有白氣如帚,自西南向東北,水夾旬方滅;盧龍有白氣如云,長■向東,越數夕色淡,而高起如帚芒狀;絳縣夜見白氣如虹。初二日,鎮洋西方見白氣■天,長數丈,移時乃滅;臨淄見白氣自西而東。初四日,溫州夜見白氣如練,長十餘丈,月餘始滅。二十年六月二十一日夜,望江見白氣■天,至八月十一日方滅。十一月,山陽見白氣■天,一月始滅;漢中西方見白氣■天如練。二十二年五月己未夜,清河有白氣數道如虹。三十九年九月,江夏見白氣如練,六七日始滅。四十一年二月,沛縣見白氣於西方。六十年十一月十九日,遵化有白氣如練,聚於西南,移時方滅。六十一年六月十四日,嘉定有白氣■天。

雍正九年閏五月二十七日夜,南宮有白氣一道南行有聲。

乾隆十八年九月癸丑,東流有氣如虹著天,色紫白,久而沒。三十五年七月二十八日,肥城有白氣十三道,至夜半乃退。

嘉慶二十年五月,武定有白氣■天向西,長數丈。

道光十三年四月十八日,棲霞有白氣■天。二十年,昌黎夜見白氣■天,逾月乃滅。二十二年春,莘縣有白氣如練數丈,月餘乃滅。冬,玉田有白氣■天。二十三年三月,黃州有白氣如練,斜指西南,經月始散。四月,滕縣有白氣■天,月餘乃滅。二十四年夏,登州有白氣■天。二十五年春,即墨有白氣西北■天。二十六年秋,寧津夜有白氣長竟天。

咸豐七年秋,黃安有白光如電,燭暗室,有聲。十一年六月,棲霞有白光如疋練,橫■西北,十餘日始滅。

同治七年九月十五日,玉田有火光至空際化為白氣,長丈許,其中有聲如鼓。

光緒元年秋,海陽有白氣突起,移時始滅。


\end{pinyinscope}