\article{列傳一百二十七}

\begin{pinyinscope}
王傑董誥硃珪

王傑,字偉人,陜西韓城人。以拔貢考銓藍田教諭,未任,遭父喪,貧甚,為書記以養母。歷佐兩江總督尹繼善、江蘇巡撫陳宏謀幕,皆重之。初從武功孫景烈游,講濂、洛、關、閩之學;及見宏謀,學益進,自謂生平行己居官得力於此。

乾隆二十六年,成進士,殿試進呈卷列第三。高宗熟視字體如素識,以昔為尹繼善繕疏,曾邀宸賞,詢知人品,即拔置第一。及引見,風度凝然,上益喜。又以陜人入本朝百餘年無大魁者,時值西陲戡定,魁選適得西人,禦制詩以紀其事。尋直南書房,屢司文柄。五遷至內閣學士。三十九年,授刑部侍郎,調吏部,擢左都御史。四十八年,丁母憂,即家擢兵部尚書。車駕南巡,傑赴行在謝,上曰:「汝來甚好。君臣久別,應知朕念汝。然汝儒者,不欲奪汝情,歸終制可也。」服闋,還朝。五十一年,命為軍機大臣、上書房總師傅。次年,拜東閣大學士,管理禮部。臺灣、廓爾喀先後平,兩次圖形紫光閣,加太子太保。

傑在樞廷十餘年,事有可否,未嘗不委曲陳奏。和珅勢方赫,事多擅決,同列隱忍不言,傑遇有不可,輒力爭。上知之深,和珅雖厭之而不能去。傑每議政畢,默然獨坐。一日,和珅執其手戲曰:「何柔荑乃爾!」傑正色曰:「王傑手雖好,但不能要錢耳!」和珅赧然。嘉慶元年,以足疾乞免軍機、書房及管理部事,允之。有大事,上必諮詢,傑亦不時入告。

時教匪方熾,傑疏言:「賊匪剿滅稽遲,由被賊災民窮無倚賴,地方官不能勞來安輯,以致脅從日眾,兵力日單而賊焰日熾。此時當安良民以解從賊之心,撫官兵以勵行間之氣。三年之內,川、楚、秦、豫四省殺傷不下數百萬,其幸存而不從賊者,亦皆鋒鏑之餘,男不暇耕,女不暇織。若再計畝徵輸,甚至分外加派,胥吏因緣勒索,艱苦情形無由上達聖主之前。祈將被賊地方錢糧蠲免,不令官吏舞弊重徵,有來歸者概勿窮治,賊勢或可漸孤矣。至於用兵三載未即成功,實由將帥有所依恃,怠玩因循,非盡士卒之不用命也。乞頒發諭旨,曲加憐恤,有驕惰不馴者,令經略概行撤回,或就近更調召募,申明紀律,鼓行勵戎,庶幾人有挾纊之歡,眾有成城之志。」又言:「教匪之蔓延,其弊有二:一由統領之有名無實。勒保雖為統領,而統兵大員名位相等,人人得專摺奏事,於是賊至則畏避不前,賊去則捏稱得勝。即如前歲賊竄興安,領兵大員有『匪已渡江五日,地方官並不稟報』之奏,此其畏避情形顯而易見。又如去歲賊擾西安城南,殺傷數萬,官兵既不近賊,撫臣一無設施;探知賊去已遠,然後虛張聲勢,名為追賊,實未見賊。近聞張漢潮蔓延商、雒,高均德屯據洋縣,往來沖突,如入無人之境。秦省如此,川省可知。實由統領不專、賞罰不明之所致也。一由領兵大員專恃鄉勇。鄉勇陣亡,無庸報部,人數可以虛捏;藉鄉勇為前陣,既可免官兵之傷亡,又可為異日之開銷,此所以耗國帑而無可稽核也。臣以為軍務緊要,莫急於去鄉勇之名而為召募之實,蓋有五利:一,民窮無依,多半從賊,茍延性命,募而為兵,即有口糧,多一為兵之人,即少一從賊之人;一,隔省徵調,曠日持久,就近召募,則旬日可得;一,徵兵遠來,筋力已疲,召募之人,不須跋涉;一,隔省之兵,水土不習,路徑不諳,就近之人,則不慮此;一,鄉勇勢不能敵,則逃散無從懲治,召募之兵退避,則有軍法。具此五利,何不增募,一鼓而殲賊?如謂兵多費多,獨不思一萬兵食十月之糧,與十萬兵食一月之糧,其費相等而功可早奏也。」疏入,並被採用。

二年,復召直軍機,隨扈熱河。未幾,因腿疾,詔毋庸入直,先行回京。三年秋,川匪王三槐就擒,封賞樞臣,詔:「傑現雖未直軍機,軍興曾有贊畫功,並予優敘。」

洎仁宗親政,傑為首輔,遇事持大體,竭誠進諫,上優禮之。五年,以衰病乞休,溫詔慰留,許扶杖入朝。七年,固請致仕,晉太子太傅,在籍食俸。八年春,瀕行上疏,略謂:「各省虧空之弊,起於乾隆四十年以後,州縣營求餽送,以國帑為夤緣,上司受其挾制,彌補無期。至嘉慶四年以後,大吏知尚廉節,州縣仍形拮據,由於苦樂不均,賢否不分,宜求整飭之法。又,舊制,驛丞專司驛站,無可誅求。自裁歸州縣,濫支苛派,官民俱病。宜先清驛站,以杜虧空。今當軍務告竣,朝廷勤求治理,無大於此二者。請睿裁獨斷,以挽積重之勢。」所言切中時弊,上嘉納之。陛辭日,賜高宗御用玉鳩杖、御制詩二章,以寵其行,有云:「直道一身立廊廟,清風兩袖返韓城。」時論謂足盡其生平。既歸,歲時頒賞不絕,每有陳奏,上輒親批答,語如家人。

九年,傑與妻程並年八十,命巡撫方維甸齎禦制詩、額、珍物,於生日就賜其家。傑詣闕謝,明年正月,卒於京邸。上悼惜,賜金治喪,贈太子太師,祀賢良祠,謚文端。

傑體不逾中人,和靄近情,而持守剛正,歷事兩朝,以忠直結主知。當致仕未行,會有陳德於禁城驚犯乘輿,急趨朝請對曰:「德庖廚賤役,安敢妄蓄逆謀?此必有元奸大憝主使行明張差之事,當除肘腋之患。」至十八年林清逆黨之變,上思其言,特賜祭焉。

孫篤,道光二年進士,歷編修、御史,出為汀州知府、廣東督糧道,署鹽運使。時林則徐為按察使,治海防,甚倚之。募廣州游手精壯者備守御,以機敏稱。擢山東布政使,署巡撫。失察家人、屬官受賂,連降罷職歸,襄理西安城工。卒,贈布政使銜。

董誥,字蔗林,浙江富陽人,尚書邦達子。乾隆二十八年進士,殿試進呈卷列第三,高宗因大臣子,改二甲第一。選庶吉士,即預修國史、三通、皇朝禮器圖。散館,授編修。三十二年,命入懋勤殿寫金字經為皇太后祝嘏。次年,大考翰詹,因寫經未與試,特加一級。尋擢中允,丁父憂。三十六年,服闋,入直南書房。初,邦達善畫,受高宗知。誥承家學,繼為侍從,書畫亦被宸賞,尤以奉職恪勤為上所眷注。累遷內閣學士。四十年,擢工部侍郎,調戶部,歷署吏、刑兩部侍郎,兼管樂部。充四庫館副總裁,接辦全書薈要,命輯滿洲源流考。四十四年,命為軍機大臣。五十二年,加太子少保,擢戶部尚書。臺灣、廓爾喀先後底定,並列功臣,圖形紫光閣。

嘉慶元年,授受禮成,詔硃珪來京,將畀以閣務,仁宗賀以詩。屬稿未竟,和珅取白高宗曰:「嗣皇帝欲市恩於師傅。」高宗色動,顧誥曰:「汝在軍機、刑部久,是於律意云何?」誥叩頭曰:「聖主無過言。」高宗默然良久,曰:「汝大臣也,善為朕輔導之。」乃以他事罷珪之召。時大學士懸缺久,難其人。高宗謂劉墉、紀昀、彭元瑞三人皆資深,墉遇事模棱,元瑞以不檢獲愆,昀讀書多而不明理,惟誥在直勤勉,超拜東閣大學士,明詔宣示,俾三人加愧勵焉。命總理禮部,仍兼管戶部事。二年,丁生母憂,特賜陀羅經被,遣御前侍衛、額駙豐紳殷德奠醊。

誥既以喪歸,川、楚兵事方亟,高宗欲召之,每見大臣,數問;「董誥何時來?」逾年,葬母畢,詣京師,和珅遏不上聞。會駕出,誥於道旁謝恩,高宗見之,喜甚,命暫署刑部尚書,素服視事,不預典禮,專辦秋讞及軍營紀略,且曰:「誥守制已逾小祥,不得已用人之苦心,眾當共諒。」尋以王三槐就擒,與軍機大臣同被議敘。四年春,高宗崩,和珅伏誅,命誥復直軍機,晉太子太保。既,服闋,授文華殿大學士,兼刑部尚書如故。高宗山陵禮成,命題神主,晉太子太傅。七年,三省教匪平,予騎都尉世職。十二年,高宗實錄告成,詔以誥在館八年,始終其事,特加優獎,賜其父邦達入祀賢良祠。十四年,萬壽慶典,晉太子太師。充上書房總師傅。十七年,晉太保。

十八年,扈從秋獮。林清逆黨突入禁城,時回鑾,中途聞變,有議俟調大兵成列而後進者,誥曰:「是滋亂也,獻俘者行至矣!」即日扈駕進次,人心乃定。窮治邪教,誥謂:「燒香祈福,愚民無知,率所常有。惟從逆者不可貸。」凡論上,皆以是定讞。林清既誅,滑縣逆匪尋平,論功,迭被優敘,賜子淳為郎中。二十年,因病請致政,溫詔慰留,改管兵部。未幾,復命管刑部。二十三年,再疏乞休,許致仕食全俸。是年十月,卒,贈太傅。上親奠,入祀賢良祠,賜金治喪,禦制詩輓之,嘉其父子歷事三朝,未嘗增置一畝之田、一椽之屋,命刻詩於墓,以彰忠藎。謚文恭。

誥直軍機先後四十年,熟於朝章故事,有以諮者,無不悉。凡所獻納皆面陳,未嘗用奏牘。當和珅用事,與王傑支柱其間,獨居深念,行處幾失常度,卒贊仁宗殲除大憝。及林清之變,獨持鎮定,尤為時稱云。

硃珪,字石君,順天大興人。先世居蕭山,自父文炳始遷籍。文炳官盩厔知縣,曾受經於大學士硃軾。珪少傳軾學,與兄筠同鄉舉,並負時譽。乾隆十三年成進士,年甫十八,選庶吉士,散館授編修。數遇典禮,撰進文冊。高宗重其學行,累遷侍讀學士。二十五年,出為福建糧驛道。擢按察使,治獄平恕,以父憂去。三十二年,補湖北按察使。會緬甸用兵,以部署驛務詳慎,被褒獎。

調山西,就遷布政使,署巡撫。疏請歸化、綏遠二城穀二萬餘石搭放兵糧,以省採買、免紅朽;又免土默特蒙古私墾罪,以所墾牧地三千餘頃,許附近兵民認耕納租,歲六千餘兩,增官兵公費;又太僕寺牧地苦寒,改徵折色,以便民除弊;皆下部議行。珪方正,為同僚所不便,按察使黃檢奏劾讀書廢事。

四十年,召入覲,改授侍講學士,直上書房,侍仁宗學。四十四年,典福建鄉試。次年,督福建學政。瀕行,上五箴於仁宗:曰養心,曰敬身,曰勤業,曰虛己,曰致誠。仁宗力行之,後親政,嘗置左右。五十一年,擢禮部侍郎,典江南鄉試,督浙江學政。還朝,調兵部。五十五年,典會試。出為安徽巡撫。皖北水災,馳驛往賑,攜僕數人,與村民同舟渡,賑宿州、泗州、碭山、靈壁、五河、盱眙餘災,輕者貸以糧種。築決堤,展春賑,並躬蒞其事,民無流亡。五十九年,調廣東。尋署兩廣總督,授左都御史、兵部尚書,仍留巡撫任。嘉慶元年,授總督,兼署巡撫。珪初以文學受知,洎出任疆寄,負時望,將大用。和珅忌之,授受禮成,珪進頌冊,因加指摘,高宗曰:「陳善納誨,師傅之職宜爾,非汝所知也。」會大學士缺,詔召珪,卒為和珅所沮。以廣東艇匪擾劫閩、浙,責珪不能緝捕,寢前命,左遷安徽巡撫。皖北復災,親治賑,官吏無侵蝕。三省教匪起,安徽亦多伏莽。珪曰:「疑而索之,是激之變。」親駐界上籌防禦,遍蒞潁、亳所屬,集鄉老教誡之,民感化,境內迄無事。明年,授兵部尚書,調吏部,仍留巡撫任。

四年正月,高宗崩,仁宗即馳驛召珪,聞命奔赴。途中上疏,略曰:「天子之孝,以繼志述事為大。親政伊始,遠聽近瞻,默運乾綱,雱施渙號。陽剛之氣,如日重光,惻怛之仁,無幽不浹。修身則嚴誠欺之界,觀人則辨義利之防。君心正而四維張,朝廷清而九牧肅。身先節儉,崇獎清廉,自然盜賊不足平,財用不足阜。惟原皇上無忘堯、舜自任之心,臣敢不勉行義事君之道。」至京哭臨,上執珪手哭失聲。命直南書房,管戶部三庫,加太子少保,賜第西華門外。時召獨對,用人行政悉以諮之。珪造膝密陳,不關白軍機大臣,不沽恩市直,上傾心一聽,初政之美,多出贊助。

尋充上書房總師傅,調戶部尚書。詔清漕政,禁浮收。疆吏以運丁苦累,仰給州縣,州縣不得不取諸民,於是安徽加贈銀,江蘇加耗米,珪謂小民未見清漕之益,先受其害,力爭罷之,令曹司凡事近加賦者皆議駁。長蘆鹽政請加增鹽價,駁曰:「蘆東因錢價賤,已三加價矣,且免積欠三百六十萬兩,餘欠展三年,商力已寬,無庸再議加價。」廣東請濱海沙地升賦,駁曰:「海沙淤地,坍漲靡常,故照下則減半賦之。今視上、中田增賦,是與民計微利,非政體。且民苦加賦,別有漲地,將不敢報墾,不可行。」倉場請預納錢糧四五十倍,準作義監生,駁曰:「國家正供有常經,名實關體要。於名不正,實必傷,斷不可行。」凡駁議每自屬稿,奏上,皆韙之。五年,兼署吏部尚書。

先是彭元瑞於西華門內墜馬,珪呼其輿入舁之,為御史周栻所劾。尋有珪輿人毆傷禁門兵,忌者嗾護軍統領訐之。詔:「珪素恪謹,造次不檢,特申戒。」坐褫宮銜,解三庫事,鐫級留任。七年,協辦大學士,復太子少保。尋兼翰林院掌院學士,晉太子少傅。九年,上幸翰林院,聯句賜宴,御書「天祿儲才」額刻懸院中,以墨書賜珪家。十年,拜體仁閣大學士,管理工部。上以是命遵高宗諭,遣詣裕陵謝。逾歲,年七十六,以老乞休,溫詔慰留,賜玉鳩杖;命天寒,間二三日入直。

未幾,召對乾清宮,眩暈,扶歸第,數日卒。上親奠,哭之慟。贈太傅,祀賢良祠,賜金治喪。詔:「珪自為師傅,凡所陳說,無非唐、虞、三代之言,稍涉時趨者不出諸口,啟沃至多。揆諸謚法,足當『正』字而無愧,特謚文正。又見其門庭卑隘,清寒之況,不減儒素。」命內府備筵,遣皇子加奠。啟殯日,遣慶郡王永璘祖奠目送。逾年,上謁西陵,珪墓近蹕路,遣官賜奠。高宗實錄成,特賜祭,擢長子錫經為四品京堂。二十年,復因謁陵回鑾,親奠其墓,恩禮始終無與比。

珪文章奧博,取士重經策,銳意求才。嘉慶四年典會試,阮元佐之,一時名流搜拔殆盡,為士林宗仰者數十年。學無不通,亦喜道家,嘗曰:「硃子注參同契,非空言也。」

論曰:君子小人消長之機,國運系焉。王傑、董誥、硃珪皆高宗拔擢信任之臣,和珅一再間沮,卒不屈撓。一旦共、驩伏法,眾正盈朝,攄其忠誠,啟沃新主,殄寇息民,苞桑永固。天留數人,弼成仁宗初政之盛,可謂大臣矣。


\end{pinyinscope}