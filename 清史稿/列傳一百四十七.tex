\article{列傳一百四十七}

\begin{pinyinscope}
司馬騊王秉韜嵇承志康基田吳璥徐端

陳鳳翔黎世序

司馬騊,字云皋,江蘇江寧人。乾隆中,大學士高晉為兩江總督,闢佐幕司章奏。習河事,以從九品留工效用,授山陽主簿。累遷淮安同知,仍兼幕職。從晉塞河,屢有功。薩載繼任總督,亦倚之。五十年,奏擢江南河庫道。道庫歲修六十萬,溢額則俟上聞,遇險工,員借帑,久輒因緣為弊,騊從容籌補,公私具舉。五十五年,遷江西按察使,在官七年,巡撫簠簋不飭,被劾多所牽連,騊以謹慎獲免。嘉慶元年,遷山西布政使。二年,調山東,兼管河務。是年秋,曹州河溢,命騊偕兩江總督李奉翰、南河總督康基田、前山東巡撫伊江阿同任堵塞。冬,擢河東河道總督。曹工尋合龍。三年春,西壩蟄,革職留任。疏言豫東兩岸堤工卑薄,請擇要增高,以御汛漲。詔以下游不能深通,徒事加堤,斥其不揣本而齊末,曹工之蟄,由於堵築不堅,罰騊等賠修,奪翎頂,所議工事仍允行。九月,睢州河溢,詔免治罪,責速塞。四年正月,工竣,復頂戴,議敘,免其代賠帑銀。尋卒於工次,賜恤。

王秉韜,字含谿,漢軍鑲紅旗人。由舉人授陜西三原知縣,累遷河南光州直隸州知州。緣事降浙江按察司經歷,改雲南知縣。累遷山西保德知州,有政聲。乾隆五十五年,擢安徽潁州知府,因讞獄遲延罷職,詔以原官發江蘇,補淮安。嘉慶二年,復調潁州。會教匪犯河南,去潁州甚近。秉韜慨然曰:「同為守土臣,豈可以畛域遺害乎?」與壽春鎮總兵定柱團結鄉勇數千,勵以忠義,助糧餉,戰於境上,破賊走之。時大學士硃珪為安徽巡撫,器其才。未幾,擢廣西左江道。復以在潁州失察逸犯,罣議,鐫級去官,留治江南豐、碭河工。尋署廬鳳道。洎仁宗親政,硃珪薦之,擢奉天府尹,遷河南布政使。五年,擢河東河道總督。

秉韜老於吏事,治河主節費,堤埽單薄者擇要修築,不以不急之工擾民。河北道羅正墀信用劣幕舞弊,曹考通判徐鼐張皇糜費,並劾治之。薪料如額採買,河員濫報輒駮斥,使多積土以備異漲,於是浮冒者不便其所為,言官遽論劾,詔慰勉,戒勿偏於節省。七年,防汛,卒於工次。

秉韜性方正,不沽名。時疆吏中長麟、汪志伊並以廉著,秉韜不愜其為人,嘗曰:「長三,汪六皆名過其實,奚足貴?」繼其任者為嵇承志。

承志,大學士璜子。由舉人官內閣中書,累遷長蘆鹽運使。乾隆五十九年,天津海河溢,築堤守御。高宗以承志無守土責,能盡力,特詔嘉之。尋病歸。嘉慶六年,從侍郎那彥寶治永定河,復授長蘆鹽運使。七年,署河東河道總督。承志年已老,上特以其家世習河事,故任之。八年,河決封丘衡家樓,次年,塞決工竣。召還京,授大理寺少卿。十年,遷順天府尹。尋卒。

康基田,字茂園,山西興縣人。乾隆二十二年進士,授江蘇新陽知縣,調昭文。為令幾十年,遷廣東潮州通判。以獲盜功,晉秩同知。累遷河南河北道,調江南淮徐道,治河有聲。五十二年,擢江蘇按察使。命每年大汛赴淮、徐襄河務。六月,河南睢州河溢,基田奉檄馳往堵築。次年,遷江寧布政使,兼河務如故。五十四年,署江南河道總督,尋回任。六月,基田防汛睢南,值周家樓河溢,上游魏家莊大埽翻陷,基田壓焉,援救得生。詔嘉其奮勉,特加恩賚。五十五年,護理安徽巡撫。以高郵糧胥偽造印串,巡撫閔鶚元被嚴譴,褫基田頂戴。復以陳奏不實,革職逮問,遣戍伊犁。尋許贖罪,以南河同知用。五十六年,仍授淮徐道。五十九年,力守豐汛曲家莊堤,特詔褒獎。擢江蘇按察使,調山東,仍兼黃、運兩河事。

嘉慶元年,南河豐汛河溢,基田赴工襄治,遷布政使。命回山東,疏消漫水,撫恤災民,基田遂往來其間。次年春,豐工竣,賜花翎。擢江蘇巡撫。秋,河溢碭山楊家壩,命馳視。山東曹縣河亦溢,復命往襄同堵築。授河東河道總督,尋調南河。三年,曹工合而復蟄,部議革職,詔寬免。疏言:「口門深逾十丈,擬就二壩前河勢灣處開引河,別築一壩,即以舊西壩改作挑水壩,俟秋後興工。」詔責其延玩,褫翎頂。尋命專任下游挑河事。九月,河南睢州河復溢,水入渦、濉諸河,正河斷流。大工旋合。次年春,睢工亦竣,河歸故道,引河通暢,復翎頂。時有條奏治海口及復舊制混江龍者,基田疏言:「治河之法,首在束水攻沙。自曹工漫溢,溜或旁趨,遂致正河淤墊。因上決而下淤,非先淤而後決。今睢工、曹工既竣,連年黃水漫衍,所在停沙,比至清江會淮,已成清水。海口刷滌寬三百數十丈,毋庸疏濬。混江龍助水之力甚微,不若束水攻沙、以水治水之力大而功倍。」仁宗嘉納之。

秋,河溢邵家壩。十二月,堵合未旬日,壩復蟄,滲水,責基田賠帑。五年正月,壩工失火,積料盡焚,革職,留工效力。基田馭下素嚴,督率將卒守堤,動以軍法從事,稽延者杖枷不貸,人多怨之。又官吏積弊懼揭,陰縱火以掩其跡。帝亦知基田性剛守潔,惟責其苛細,仍命隨辦要工,欲復用之。及邵家壩工竣,以知州用,補江蘇太倉直隸州。逾年,擢廣東布政使,調江西,又調江寧。十一年,因貴州鉛船遲滯,降調,授戶部郎中。

十三年,從協辦大學士長麟、戴衢亨察視南河,基田請修復天然閘迤東十八里屯二石閘,靳輔所建也,足以減黃濟運,且山石夾峙,無奪溜沖決之患,據以入告。帝嘉其留心河務,加道銜,賜花翎。尋予太僕寺少卿職銜,稽核南河要工錢糧。十六年,以年逾八旬,乞休,允之,命來京就養,以示優恤。後議改建山盱五壩,特命與議。基田疏陳:「舊制盡善,不宜輕改。今仁、義、禮三壩石底損壞,跌成深塘,不得已為變通之計。請將仁、義二壩先改其一,俟大汛果見順利,再議添所建。擬禮壩先築草壩,非湖水大漲,不可輕放。」奏入,報聞。十八年,鄉舉重逢,賜三品卿銜,與鹿鳴宴。尋卒。

吳璥,字式如,浙江錢塘人,吏部侍郎嗣爵子。乾隆四十三年進士,選庶吉士,授編修。大考擢侍講學士,典陜西鄉試。五十四年,督安徽學政。召見,高宗因其父曾為總河,詢以河務,所對稱旨,即日授河南開歸陳許道。累遷布政使。五十九年,巡撫出視賑,璥充鄉試監臨,聞河水暴漲,即出闈馳防,帝嘉之。六十年,署巡撫。

嘉慶二年,楚匪齊王氏犯河南,擊走之,復剿息縣匪,賜花翎。母憂留任。四年,署河東河道總督,尋實授。請增河工料價,歸地糧攤徵,詔斥其病民,革職留任。五年,調南河,堵合邵家壩漫口,加太子少保。八年秋,河決衡家樓,命豫籌來年漕運,請疏邳州、宿遷諸閘,於宿遷、桃源交界築束水草壩,濬淤淺,依議行。又言徐州一帶河水寬深而未消落,乃海口壅塞所致,詔相度治之。尋疏陳:「雲梯關海口暗灘,尚非全被阻遏。請於黃泥嘴開引河,並挑吉家浦、於家港、倪家灘、宋家尖諸灘。」允之。九年秋,洪湖水漲未消,請緩築仁、智兩壩,以保堰、盱堤工。時東河衡工甫合,清江浦河口水淺阻糧船,上謂清水力弱,由啟放仁、智等壩所致,命侍郎姜晟往會籌蓄黃濟運。璥與合疏請堵二壩及惠濟閘之鉗口壩,使湖水全力東注,刷通河口,並啟李工口門,減掣黃水,從之。上終以璥多病,治河不力,雖宥其罪,命解職。十年,授兵部侍郎,調倉場侍郎。

十一年,復授河東河道總督。因料物例價不敷,請依南河按時價折銷,允之。復請歲料幫價歸地糧攤徵,被嚴斥,革職留任。尋又以堤堰工需並入衡工善後題銷,上切責之。十三年,召回京,授刑部尚書。命偕侍郎托津赴江蘇鞫獄,並勘議海口改道,請仍復故道,接築雲梯關外大堤,從之。復授江南河道總督。十四年,疏陳:「海口應濬,而大堤不堅,旁洩必淤;蓄清為要,而堤壩不復,遇漲必潰。今閘壩無減黃之路,五壩無節宣之方,皆宜急為救治。」詔韙之,令盡心經理。是冬,以海口挑復正河,費用浩繁,不及於次年桃汛前舉工,請權宜仍濬北潮河以通去路。十五年春,偕兩江總督松筠合疏請修復正河,詔允行;而斥璥無定見,前後矛盾,責其認真督治,不得以事由松筠主持為推諉之地。尋因病乞假,詔解職,俟病痊以六部尚書用。

璥既去任,松筠疏論河工積弊,謂璥與徐端治理失宜,用人不當,墊款九十餘萬,恐有冒捏。又兩淮鹽政阿克當阿劾揚河通判繆元淳浮冒工款,稱:「璥路過揚州,與言員營弁不肖者多,往往虛報工程,且有無工借支。前在任六七年,用帑一千餘萬,今此數年,竟至三四千萬。」詔斥璥知而不奏,命尚書托津等往南河按之,劾璥失察誤工;又濬淮北鹽河,未經奏陳,濬後復淤,詔切責,降四級調用,與徐端分賠鹽河工款,命璥赴南河襄辦王營減壩及李家樓漫口。十七年,補光祿寺卿,累遷吏部侍郎。

十八年,睢州河溢,命赴南河察勘湖河。十九年,授河東河道總督,督治睢工。次年,遷兵部尚書,工竣回京,歷刑部、吏部,協辦大學士。上以璥練習河務,無歲不奉使出勘河。二十一年,協防東河秋汛。二十二年,勘睢工及山東運河,南河蕭南民堰,清江浦禦黃、束清諸壩。二十三年,築沁河漫口。二十四年,築河南蘭陽、儀封及武涉馬營壩決口。二十五年,勘南河束清、御黃諸壩及洩水事宜。其間再署河南巡撫,一署河東河道總督。道光元年,以病免。二年,因侍郎那彥寶治河不職降黜,追論璥與同罪,雖已致仕家居,褫其翎頂。尋卒。

徐端,字肇之,浙江德清人。父振甲,官江蘇清河知縣。端少隨任,習於河事。入貲為通判。乾隆中,河決青龍岡。振甲知涉縣,分挑引河,端佐役,大學士阿桂督工,見而器之,留東河任用,授蘭儀通判。尋升缺為同知,調睢寧,又調開封下南河。

嘉慶三年,署山東沂曹道。睢州河決,端預築曹州堤,得無害。四年,擢江西饒州知府,未之任,調江蘇淮安。七年,擢淮徐道,丁父憂,與假治喪,仍回任。九年,加三品頂戴,護理東河河道總督。時衡家樓甫塞決,詔以前官王秉韜惜費,嵇承志年衰,修防多疏,責端通籌全河為未雨綢繆之計。端疏陳臨河埽工固緊要,無工之地尤須慎防,仁宗韙之。冬,清口水淺阻漕船,端偕尚書姜晟等往視,請展引河,啟祥符五瑞壩,分河水入洪湖助清敵黃,清口乃通。尋授江南河道總督。十年,請疏治雲梯關沙淤,培築桃源以下堤工;又請移建河口束清壩於迤南湖水匯出之處,以資節制;挑清壩外築束清東壩,對岸張家莊增築西壩,留口門二十丈,視湖水大小為束展:詔允行。秋,築義壩。時命侍郎戴均元會籌蓄黃濟運,端與合疏請濬王營減壩以下鹽河,遇盛漲,相機啟放,庶黃減淮強,湖水暢出,堰工亦免著重,從之。

十一年,洪湖異漲,高堰賴新築子堰抵御,不為害。俄黃水並漲,決鹽河民堰,運河東岸荷花塘亦決。以功過相抵,免議。舊制,南河設正副總河,後裁其副;至是授戴均元為河道總督,端副之。秋,河決周家樓,上游郭家房堤蟄,命端專治郭家房堵口,四閱月工竣。時黃水由減壩六塘河入海,正河斷流,群議改道,上頒示御制黃河改道記,命端視察海口。尋以六塘河下游水勢散漫,難施工作,復頒示御制治舊河記,命端專駐減壩督工。十二年春,工竣,河循故道,加太子少保。秋,海潮上漾,河由陳家浦旁溢入射陽湖歸海,請於黃泥嘴建壩,擇要疏淤,俾仍故道。

十三年,署正總河。先是端屢言河淤由於海口流緩,宜接築雲梯關外長堤,束水攻沙,未及舉。至是兩江總督鐵保疏申前議,並請培高堰土坡,修補智、禮二壩,以備湖漲;復毛城鋪石堤、王營減壩,以節宣黃水:端贊其議。命協辦大學士長麟、戴衢亨察視,惟輟毛城鋪壩工,改建徐州十八里屯雙閘,餘依原議行。夏,湖水漲,端啟智、信二壩,不敷宣洩,壞磚工百餘丈,褫翎頂,降三級留任。尋堵合,復之。時黃水由馬港口分流,經灌河口歸海,命尚書吳璥、侍郎托津會勘,以荷花塘壩工垂成復蟄,降端為副總河。十五年,復授河道總督,裁副總河。端始終主復舊海口堵馬港,命尚書馬慧裕會同督治。兩江總督松筠劾端於河流逢灣取直,以致停淤,上不直其奏,端疏辨,詔松筠無預河務,責端與慧裕速施工,勿游移。尋以洪湖風汛,壞高堰、山盱兩工甚鉅,革職留任。松筠復密陳端祗知工程,不曉機宜,糜帑千萬,迄無成功,且恐有浮冒之弊。詔斥端不勝河督之任,革職留工,專任堵築義壩。十六年,命以通判用,復命治李家樓引河。十七年,工甫竣,病卒。

端治南河七年,熟諳工作。葦柳積堤,一過測其多少。與夫役同勞苦,廉不妄取。河工積弊,端知之,憚於輕發,欲入覲面陳而終不得,以至於敗。繼之者為陳鳳翔,河事遂益敝。

陳鳳翔,字竹香,江西崇仁人。謄錄,議敘授縣丞,發直隸河工,累遷永定河道。嘉慶六年,畿輔大水,河決者四,鳳翔從侍郎那彥寶塞決,為仁宗所知。逾年,丁父憂,賜金治喪。後復授永定河道。

十四年,擢河東河道總督,逾年,調南河。時南河敝壞已久,河湖受病日深,詔以蓄清敵黃為急務,其要在修復高堰之堤,責鳳翔克期程工,尤以借黃濟運為戒。十六年,疏陳急治河口及運河各工,高堰二堤亦次第興辦。尋偕兩江總督勒保奏報堵合御黃、鉗口兩壩,疏末微言:「海口北岸無人煙之地,面面皆水,俟秋間水落,相機辦理。」上以上年堵築馬港,兩岸皆新堤,北岸地勢尤高,明是新決諱飾,責令據實奏聞。適王營減壩土堤又決,詔切責,革職留任。尋奏:「王營減壩旁注,由海口逼緊,水無他路,致有漫溢。請俟水落,修築減壩海口,但保南岸,勿築北岸,以免水逼。」援引高宗諭旨雲梯關外勿與水爭地,詔以「從前瀕海沙灘無居民,今則馬港口外現有村落,非昔可比。且水勢散漫,河緩沙停,弊不勝言。又鳳翔等所繪海口圖無村落地名,與十三年吳璥所呈圖說不同,河形曲直亦異。」斥鳳翔意存朦混,恃才妄作:「前稱雲梯關外溜勢暢達,未挑處刷深至十餘丈,可見海口非高仰;鳳翔既未身歷其境,今因北岸漫溢,束手無策,反言從前挑築皆屬非計,以相抵塞。」特簡百齡為兩江總督,與鳳翔同勘海口。鳳翔謂海口不能暢,下壅故上潰,諉為淮海道黎世序所言;而世序實謂下壅在倪家灘新堤上下,非在海口。及百齡至,親勘海口深通,惟中段涸成平陸,乃去歲挑河積土河灘,春水漫刷,仍歸河內。又攔潮壩放水時,壩根起除未凈,阻水停淤,世序屢請籌辦,鳳翔視為緩圖,詔斥因循貽誤。會上游綿拐山、李家樓兩處漫口,革職留任。

十七年春,禮壩又決,百齡劾:「鳳翔急開遲閉,壩下沖動,不早親勘堵築,用帑二十七萬兩有奇;而壩工未竣,清水大洩,下河成災。」嚴詔斥鳳翔貽誤,革職,罰賠銀十萬兩,荷校兩月,遣戍烏魯木齊。尋鳳翔訴辨,命大學士松筠、府尹初彭齡按訊,得百齡與鳳翔同時批準開壩狀;鳳翔又訐百齡信任鹽巡道硃爾賡額督辦葦蕩柴料,捏報邀功:譴百齡等,鳳翔免枷,仍赴戍,未行,病歿。

黎世序,初名承惠,字湛溪,河南羅山人。嘉慶元年進士,授江西星子知縣,調南昌。擢江蘇鎮江知府。十六年,遷淮海道。與河督陳鳳翔爭堵倪家灘漫口,由是知名。

十七年,調淮陽道。尋鳳翔黜,詔加世序三品頂戴,署南河河道總督,俟三年後果稱職,始實授。疏言:「自上年大濬,千里長河,王營減壩及李家樓漫口堵合,雲梯關外水深二三丈至四五丈,為近年所未有。而清江浦至雲梯關一帶,較之河底深通時尚高八九尺。此非人力所能猝辦,計惟竭力收蓄湖水,以期暢出。敵黃蓄清之法,在堰、盱二堤,有旨緩辦;今年禮壩跌損,宣洩路少,二是尤應急築,以資捍衛。」允之。

十八年,以仁、義、禮三壩基壞,請於蔣家壩附近山岡移建三壩,挑引河三道,詔令詳議,並飭填實舊壩。尋如議行。因全漕渡黃較早,議敘。疏請加高徐州護城石工,添築越堤,於清江浦汰黃堤外加重堤,又於駱馬湖尾閭五壩迤下添碎石滾壩,並允之。先是百齡擬於清江浦石馬頭築圈堤,其灣處對王營,上起御黃壩,下屬貼心壩,河寬千餘丈,至此陡束為二百丈,論者以為不便,得不行;世序卒成之。是年秋,睢南薛家樓、桃北丁家莊漫水壞堤,世序躍入河者再。會上游河南睢州決口奪溜,河水陡落,睢、桃兩工得補築無事,詔以世序不能先事預防,降一級留任。睢州決口久未合,黃水全入洪湖。世序力籌宣洩,濬順清河於清口淤窄處,自束清壩起至御黃壩止,挑引河三,束清、鉗口各壩一律闢展,智、仁兩壩及蔣壩以南,新挑仁、義兩壩引河,並為分減之路。至十九年霜降,安瀾,詔嘉世序修防得宜,加二品頂戴。

二十年,疏言:「徐州十八里屯舊有東西兩閘,金門寬三丈五尺,不足減水。其西南虎山腰兩山對峙,凹處寬二十餘丈,山根石腳相連,可作天然滾壩。北面臨河,即十八里屯,山岡淤於土中,剝平山頂,改作臨河滾壩。以虎山腰為重門擎托,可期穩固。」允之。夏,洪湖盛漲,拆展束清,禦黃兩壩,啟山盱引河滾壩,清水暢出,會黃東注,刷河益深,特詔嘉獎,賜花翎。

世序治河,力舉束水對壩,課種柳株,驗土埽,稽垛牛,減漕規例價。行之既久,灘柳茂密,土料如林,工修河暢。南河歲修三百萬兩為率,每年必節省二三十萬。碎石坦坡,自靳輔始用之於高堰,後蘭第錫、吳璥、徐端偶一用之;世序始用之於通工,謗言四起,世序力持,卒獲其效。二十一年,京察,議敘。二十二年,因禦黃壩刷深不能施工,束清壩掣溜太急,亦難穩立,請於舊二壩水淺處添築重壩,又於束清壩外添建一壩,以為重門鉗束,於是比歲安瀾,奏減料價一成。

道光元年,入覲,宣宗嘉其勞勩,加太子少保,開復一切處分,賜詩以寵之。二年,京察,復予議敘。四年,卒於官,優詔褒恤,加尚書銜,贈太子太保,謚襄勤,入祀賢良祠。江南請祀名宦建專祠,帝追念前勞,禦制詩一章,命勒石於墓。賜其子學淳,主事;學淵,舉人;學澄,副榜貢生。

自乾隆季年,河官習為奢侈,帑多中飽,浸至無歲不決;又以漕運牽掣,當其事者,無不蹶敗。世序澹泊寧靜,一湔靡俗。任事十三年,獨以恩禮終焉。幕僚鄒汝翼,無錫人,世序倚如左右手,欲援陳潢故事,薦之於朝,力辭而止。涇縣包世臣號知河事,世序多用其說,惟築圈堰一事論不合。及創虎山腰滾壩,世臣阻之曰:「河以無溜為至險,攻大埽不與焉;湖以淤底為至險,掣石工不與焉。公謂減黃入湖,為化險為平。黃緩湖高,吾坐見其積平成險也。兩險交至,其禍甚烈。公意在及身,然以憂患貽後世已。」世序初奏亦謂壩成遇不得已乃啟,然後實無歲不啟。洎嘉慶二十五年,上游河南睢州馬營兩口既合,閱歲大汛至,清河、安東、阜寧三縣境內河水常平堤,而中泓無溜。世序心知其害,憂瘁而卒。後數月,高堰竟決。

論曰:仁宗銳意治河,用人其慎。然承積弊之後,求治愈殷,窟穴於弊者轉益譸張以為嘗試。海口改道之說起,紛紜數載而後定。康基田、徐端等皆諳習河事,程功亦僅。至黎世序宣勤久任,南河乃安;而減黃病湖,遂遺隱患。得失之故,具於斯焉。


\end{pinyinscope}