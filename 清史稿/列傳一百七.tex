\article{列傳一百七}

\begin{pinyinscope}
三寶永貴蔡新程景伊梁國治英廉彭元瑞

紀昀陸錫熊陸費墀

三寶,伊爾根覺羅氏,滿洲正紅旗人。乾隆四年繙譯進士,授內閣中書。襲世管佐領。遷內閣侍讀。出為湖北驛鹽道。入補戶部郎中。師征準噶爾,命赴北路董達什達瓦游牧。擢直隸布政使。二十六年,上幸熱河,坐蹕路不修,命以道銜駐哈密。二十九年,起四川布政使,更湖北、湖南、貴州諸省。三十七年,擢山西巡撫。明年,移浙江。四十二年,擢湖廣總督。閱兵,衡州協副將海福、沅州協副將洪昌運皆衰老,三寶請以海福內授旗員,昌運令休致。上以偏護滿洲,顯分軒輊,拒不允。四十四年,授東閣大學士,兼禮部尚書,督湖廣如故。

旋移閩浙總督。浙江海塘自老鹽倉以上皆柴塘,上南巡,諭改築石塘。三寶疏言:「時方大汛,未宜更動。當於柴塘內下椿築石,而以柴塘為外護。」會上亦降旨令留柴塘為重關保障,與三寶議合。旋命入閣治事。巡撫王亶望以贓敗,三寶坐未舉劾,部議當奪職,上命留任。尋復令在上書房總師傅上行走。四十九年,扈蹕熱河,以疾還京師。卒,謚文敬。

三寶喜讀宋諸儒書,大節不茍。為直隸布政使時,高宗幸熱河,至密雲,值大霖雨,水盛漲。上欲策騎亂流渡,三寶諫曰:「千金之子,坐不垂堂。今以萬乘輕狎波濤,使御駟有失,臣等雖萬段,何可追悔?」上曰:「滿洲舊俗宜親習勞勩,顧不可耶?」三寶復曰:「上方奉太后乘輿同臨幸,即上渡河安便,不識奉太后何所?」上動容,為之回轡。其為上書房總師傅,輯古今儲貳事曰春華日覽,授諸皇子,論者謂其得師保之體云。

永貴,字心齋,拜都氏,滿洲正白旗人。父布蘭泰,自雲騎尉世職授理籓院員外郎。雍正間,為江西巡撫,治嚴刻,世宗召還京師面詰之,對曰:「臣治事從嚴,待上改正,俾恩出自上。」世宗不懌,奪職。尋復起,至古北口提督。卒,謚愨僖。

永貴,自筆帖式授戶部主事。乾隆初,累遷郎中。出為湖南辰沅永靖道。擢雲南布政使。移浙江,署巡撫。前總督李衛領鹽政,發帑收餘鹽,名曰「帑鹽」;令武職任緝私,其制未善。永貴條上八事,俾文武互任其責,下部議行。居三年,命真除。溫、臺諸縣旱,永貴令知府金洪銓治賑,不稱職。永貴論劾,請休致。總督喀爾吉善再劾,上為奪洪銓職。御史範廷楷因劾永貴瞻徇,上難其代,命寬之。永貴請留本省及江蘇漕八十萬,借撥江蘇等省米五十五萬,又請開事例,補倉儲。上責其張皇,既又聞永貴陳災狀有所諱飾,乃命奪職,赴北路軍董理糧餉。居三年,賜按察使銜,署甘肅臨洮道,仍赴巴里坤主餉。

二十一年,加副都統銜,兼參贊大臣。是歲冬,厄魯特宰桑達什策凌等為亂,定邊右副將軍兆惠駐伊犁辦賊。永貴既抵巴里坤,具以軍事上聞,上嘉其奮勉,予三等輕車都尉世職,令從兆惠自額林沁畢爾罕進兵。命署西安巡撫,未之任,令赴魯克察克屯田。二十三年,以侍郎銜留軍,因授刑部侍郎,董屯田。烏魯木齊、闢展、托克三、哈喇沙爾、昌吉、羅克倫皆駐兵營墾,秋穫得穀三萬五千八百餘石。是時兆惠兵次葉爾羌,命永貴駐阿克蘇主餽軍。

二十四年,還至庫車,布政使德舒為嗎哈沁所戕。永貴與護軍統領努三協殲逆眾,回部平。移倉場侍郎。擢左都御史。二十六年,命赴克什噶爾辦事。旋授禮部尚書、鑲紅旗漢軍都統,仍駐克什噶爾。疏請疏溝渠,興耕稼,議自赫色勒河東南浚渠四十餘里,引水入赫色勒布伊,材托庸河湍急,宜增堤壩,鑿山石,弱水勢。召還京師。

三十年,烏什回人為亂,復命赴哈什哈爾。事平,移駐烏什。三十三年,署伊犁將軍。移吏部,再移禮部。坐厄魯特兵盜哈薩克馬轉誣哈薩克,辦事大臣巴爾品斷獄未得其實,永貴論劾,語有所諉飾。又以涼州、莊浪滿洲兵損馬當償,誤扣熱河兵餉,召還京師,命授左都御史,命不得用翎頂。旋移禮部尚書,得用頂帶,仍不得戴翎。四十二年,命署大學士,題孝聖憲皇后神主。尋補吏部尚書,在阿哥總諳達處行走,賜花翎。初,山東民王倫為亂,給事中李漱芳陳奏饑民釀釁,坐妄言,左授禮部主事。及是,吏部請以漱芳升授員外郎。上責永貴市恩,削職奪花翎,令以三品頂帶赴烏什辦事。詔詰責甚至,且言:「永貴回烏什,如不實心任事,必在彼處正法。」先是葉爾羌辦事大臣侍郎高樸役回民採玉,並婪取金珠,為諸伯克所訟。永貴如葉爾羌,訊得實,聞上。上為誅高樸,手詔嘉永貴持正,並謂:「永貴罪不至貶。今命西行,適以發高樸之奸,潛銷禍萌,此天啟朕衷也!」仍授吏部尚書,賜花翎。尋授參贊大臣。四十四年,召還京師,授鑲藍旗滿洲都統。四十五年,協辦大學士。四十八年,卒,謚文勤。

永貴端謹。初直軍機處,與阿桂齊名,時稱「二桂」。其撫浙江,有廉聲。

子伊江阿,官至山東巡撫。高宗崩,伊江阿因奏事附書和珅勸節哀。和申已下獄,仁宗得其書,詔詰責,奪職。既,又追論在山東日佞佛寬盜,命戍伊犁。尋授藍翎侍衛、古城領隊大臣。卒。

蔡新,字次明,福建漳浦人,贈尚書世遠族子。乾隆元年進士,選庶吉士,授編修。入直上書房。試御史第一,辭,授侍講。累遷工部侍郎,移刑部。十八年,以母老請歸省,賜其母貂緞;旋乞終養,允之。即家命為上書房總師傅,辭,高宗諭之曰:「非令汝即來供職,待後日耳。」二十五年,上五十壽,入京師祝嘏。二十六年,南巡,覲行在。母喪終,授刑部侍郎。三十二年,擢工部尚書。三十八年,移禮部。四十五年,命以吏部尚書協辦大學士。四十六年,乞假修墓。四十八年,還朝。拜文華殿大學士,兼吏部尚書。五十年,與千叟宴。上臨雍講學,新以大學士領國子監,講易「天行健,君子以自強不息」,賜茶並文綺。

新操履端謹,言行必衷於禮法。上眷之厚,賦臨雍詩,注謂:「今群臣孰可當三老五更?獨新長朕四歲,或可居兄事。然恐其局促勿敢當,舉王導對晉元帝語以謝耳。」新上疏乞致仕,語切至,上許其歸,加太子太師,三賦詩以餞。既歸,上每制文,屢以寄新,且曰:「在朝無可與言古文者。不可阿好徒稱頌。」五十五年,上八十壽,詣京師祝嘏,賜宴同樂園,賜人葠一斤。及歸,命歸途所經,有司具舟車護行。上仍以詩文寄新,諭將以驗學詣,戒詩毋和韻。五十七年,重赴鹿鳴宴。六十年,上御極六十載,諭新不必入賀。新奏言上九旬萬壽,冀再詣闕祝嘏。上諭之曰:「覽奏,字字出誠心,我君臣共勉之。若天恩得符所原,實佳話也!」嘉慶元年,新年九十,賜額曰「綠野恆春」,侑以諸珍物。四年,高宗崩,奔赴,至福州,病不能進。巡撫汪志伊以聞,溫詔止其行。是冬,卒,贈太傅,謚文端。

新學以求仁為宗,以不動心為要。嘗輯先儒操心、養心、存心、求放心諸語,曰事心錄。直上書房四十二年,培養啟迪,動必稱儒先。高宗以新究心根柢,守世遠家法,深敬禮之。既歸,福建督撫坐貪黷、虧倉庫得重譴,上責「新知而不言,自比寒蟬,無體國公忠之意」。新上疏請下吏議,卒以篤老寬之。嘉慶初,海盜方肆,新子本俊官京師,御史宋樹疏言新家書及海盜事,不以聞。上為詰本俊,本俊言新已具疏令謄真入奏,上亦不之責,仍諭新毋畏。新家居謙慎,遇丞尉執禮必恭。或問之,曰:「欲使鄉人知位至宰相,亦必敬本籍官吏,庶心有所不敢,犯法者鮮耳。」著有緝齋詩文集。

程景伊,字聘三,江南武進人。乾隆四年進士,改庶吉士,授編修。再遷侍讀學士,命在上書房行走。復三遷兵部侍郎。景伊致人書,言:「承乏中樞,晨夕內廷多曠廢。今秋未與木蘭之役,稍得專心職業。」為上聞,責其躭逸,解上書房行走。歷禮、工諸部。三十四年,擢工部尚書,歷刑、吏諸部。三十八年,協辦大學士。四十一年,上東巡回鑾,駐蹕黃新莊。景伊與在京王大臣迎駕,未召見即退班,命奪職,仍留任。四十四年,授文淵閣大學士。四十五年,上南巡,命景伊留京治事。上還京師,入對,以景伊病後衰弱,命安心調理,勿勉強行走。七月,卒,謚文恭。

梁國治,字階平,浙江會稽人。乾隆十三年一甲一名進士,授修撰。遷國子監司業。充廣東鄉試正考官。復命,奏對稱旨,命以道員發廣東待缺。旋除惠嘉潮道,移署糧驛道。卓異引見,擢署左副都御史。遷吏部侍郎。廣東總督楊廷璋等追論國治署糧驛道時失察家人舞弊,讞實,奪職。起授山西冀寧道。三遷湖北巡撫。三十四年,命署湖廣總督,兼荊州將軍。時湖北頻歲水旱,治賑,缺倉穀四十八萬餘石。國治議發司庫白金二十萬,俟秋穫易穀,來歲春夏間出糶,石溢銀一錢。行之數年,倉穀得無缺。三十六年,移湖南巡撫。師征金川,治軍械,造藥彈,費不給。國治請以司庫儲備軍興白金十餘萬,照一年應扣各糧通行借給,仍分三年扣還歸款。國治又以出征將弁,例軍中升用,本營缺出,仍系照常拔補。循資按格者,轉得坐致升遷;冒敵沖鋒者,專待軍營缺出,無以鼓勵戎行。請嗣後本營缺出,與出征將弁一體論升。皆從其請。三十八年,召還京師,命在軍機處行走,並直南書房。三十九年,授戶部右侍郎。四十二年,遷尚書。四十七年,加太子少傅。四十八年,命協辦大學士。五十年,晉授東閣大學士,兼戶部尚書。五十一年,卒,加太子太保,謚文定。

國治父文標,官刑部司獄,恤囚有惠政。國治篤孝友,與兄孿生,兄蚤卒,終生不稱壽,事嫂如母。治事敬慎縝密。生平無疾言遽色,然不可以私干。門下士有求入按察使幕主刑名者,戒之曰:「心術不可不慎!」其人請改治錢穀,則曰:「刑名不慎,不過殺一人,所殺必有數,且為人所共知。錢穀厲人,十倍刑名,當時不覺。近數十年,遠或數百年,流毒至於無窮,且未有已!」卒不許。著有敬思堂集。

英廉,字計六,馮氏,內務府漢軍鑲黃旗人。雍正十年舉人。自筆帖式授內務府主事。乾隆初,命往江南河工學習,補淮安府外河同知。累遷永定河道。河決,總督方觀承劾英廉淤溝鑲埽,沖陷水上月堤,匿不以聞,遂誤要工。奪職,逮治,英廉抗辨。逾年讞未決,觀承請遣大臣蒞其事。上命尚書舒赫德會鞫,言英廉申報不以實,且未將淤溝先事預防,堵築經費,當責出私財以償。上諭言:「英廉上官未及兩月,淤溝失防,咎實在前政。然觀承以總督劾屬吏,不敢率意入罪,讞逾年未定,請遣大臣蒞其事。是其心有所警畏,亦朕明慎庶政之效。仍從其請。」未幾,命在高梁橋迤西稻田廠效力。尋復自筆帖式授內務府主事。累遷內務府正黃旗護軍統領。外授江寧布政使,兼織造。英廉以父老,乞留京師,賜二品銜,授內務府大臣、戶部侍郎。

三十四年,徵緬甸,師行,命與尚書托庸等董其事。遷刑部尚書,仍兼戶部侍郎、正黃旗滿洲都統。三十九年,侍郎高樸劾左都御史觀保,侍郎申保、倪承寬、吳壇交內監高雲從,洩道府記載。上問英廉,英廉謝不知。詔詰責,命奪職,從寬留任。京師商人投呈皇六子,有所陳請,事下內務府。上召內務府諸大臣,問:「收呈者誰也?」英廉、金簡皆謝不知。邁拉遜乃言「六阿哥收呈」。上責英廉、金簡隱諱,下部議,命寬之,仍註冊。

四十二年,協辦大學士。四十四年,暫署直隸總督。四十五年,大學士於敏中卒,上以英廉本漢軍,協辦有年,特授漢大學士。漢軍授漢大學士自英廉始。尋授東閣大學士,仍領戶部。四十六年,復署直隸總督,疏請清州縣虧帑。四十七年,加太子太保。復署直隸總督。直隸災,治賑,疏請以截存漕米補各倉儲穀,又疏請蠲未完耗羨三萬餘兩,皆從其請。尋以病乞罷,命以大學士還京師養痾。卒,賜白金五千治喪,祀賢良祠,謚文肅。

彭元瑞,字蕓楣,江西南昌人。乾隆二十二年進士,改庶吉士。散館授編修,直懋勤殿。大考,以內直不與。遷侍講。擢詹事府少詹事。直南書房。遷侍郎,歷工、戶、兵、吏諸部。高宗六十壽,次聖教序為贊以進,上嘉之。上制全韻詩,元瑞重次周興嗣千字文為跋。上手詔獎諭,稱為「異想逸材」,賜貂裘、硯、墨。敕撰寧壽宮、皇極殿鐙聯,稱旨,賜以詩。闢雍成,釋奠講學,又繼以耕耤。上三大禮賦。擢尚書,歷禮、兵、吏三部。五十五年,上八十壽,以歲陽在庚,進八庚全韻詩。上以庚立字數奇,易首句用韻去一聯,末句乃諧律,親為裁定。尋加太子少保、協辦大學士。五十六年,以從孫冒入官,御史初彭齡論劾,左授禮部侍郎,命仍直南書房。尋復授工部尚書。嘉慶四年,高宗奉安禮成,元瑞撰祝文,仁宗嘉其得體,加太子太保。元瑞子翼蒙,官江南鹽巡道,坐事免,元瑞自劾,又坐誤舉編修繆晉,下吏議,上皆寬之。修高宗實錄,命充總裁。八年,以疾乞罷,慰留,久之乃許。命仍領實錄總裁。旋卒,贈協辦大學士,謚文勤。

元瑞以文學被知遇。內廷著錄藏書及書畫、彞鼎,輯秘殿珠林、石渠寶笈、西清古鑒、寧壽鑒古、天祿琳瑯諸書,元瑞無役不與。和章獻頌,屢荷褒嘉。所著有經進槁、知聖道齋跋尾諸書。高宗實錄成,推恩賜祭,並祀賢良祠,官翼蒙員外郎。

紀昀,字曉嵐,直隸獻縣人。乾隆十九年進士,改庶吉士。散館授編修。再遷左春坊左庶子。京察,授貴州都勻府知府。高宗以昀學問優,加四品銜,留庶子。尋擢翰林院侍讀學士。前兩淮鹽運使盧見曾得罪,昀為姻家,漏言奪職,戍烏魯木齊。釋還,上幸熱河,迎鑾密雲。試詩,以土爾扈特全部歸順為題,稱旨,復授編修。三十八年,開四庫全書館,大學士劉統勛舉昀及郎中陸錫熊為總纂。從永樂大典中搜輯散逸,盡讀諸行省所進書,論次為提要上之,擢侍讀。上復命輯簡明書目。坐子汝傳積逋被訟,下吏議,上寬之。旋遷翰林院侍讀學士。建文淵閣藏書,命充直閣事。累遷兵部侍郎。四庫全書成,表上。上曰:「表必出昀手!」命加賚。遷左都御史。再遷禮部尚書。復為左都御史。畿輔災,饑民多就食京師。故事,五城設飯廠,自十月至三月。昀疏請自六月中旬始,廠日煮米三石,十月加煮米二石,仍以三月止,從之。復遷禮部尚書,仍署左都御史。疏請鄉會試春秋罷胡安國傳,以左傳本事為文,參用公、穀,從之。嘉慶元年,移兵部尚書。復移左都御史。二年,復遷禮部尚書。疏請婦女遇強暴,雖受污,仍量予旌表。十年,協辦大學士,加太子少保。卒,賜白金五百治喪,謚文達。

昀學問淵通。撰四庫全書提要,進退百家,鉤深摘隱,各得其要指,始終條理,蔚為巨觀。懲明季講學之習,宋五子書功令所重,不敢顯立異同;而於南宋以後諸儒,深文詆諆,不無門戶出入之見雲。

陸錫熊,字健男,江蘇上海人。乾隆二十六年進士。召試,授內閣中書。累遷刑部郎中。與昀同司總纂,旋並授翰林院侍讀。五遷左副都御史。旋以書有譌謬,令重為校正,寫官所費,責錫熊與昀分任。又令詣奉天校正文溯閣藏書,卒於奉天。

陸費墀,字丹叔,浙江桐鄉人。陸費為衣復姓。墀,乾隆三十一年進士,改庶吉士,授編修。充四庫全書館總校,用昀、錫熊例,擢侍讀。累遷禮部侍郎。書有譌謬,上謂昀、錫熊、墀專司其事,而墀咎尤重。文瀾、文匯、文宗三閣書面葉木匣,責墀出資裝治。仍下吏議,奪職。旋卒。上命籍墀家,留千金贍其孥,餘充三閣裝治之用。

論曰:乾隆中年後,多以武功致臺鼎。若三寶、永貴、國治、英廉,皆先陟外臺,易又歷著聲績。國治直樞廷十餘年,先後與於敏中、和珅未嘗有所阿。新、元瑞、昀起侍從,文學負時望。新謹厚承世遠之教。昀校定四庫書,成一代文治,允哉,稱其位矣!


\end{pinyinscope}