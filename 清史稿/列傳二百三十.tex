\article{列傳二百三十}

\begin{pinyinscope}
孫家鼐張百熙唐景崇於式枚沈家本

孫家鼐,字燮臣,安徽壽州人。咸豐九年一甲一名進士,授修撰。歷侍讀,入直上書房。光緒四年,命在毓慶宮行走,與尚書翁同龢授上讀。累遷內閣學士,擢工部侍郎。江西學政陳寶琛疏請以先儒黃宗羲、顧炎武從祀文廟,議者多以為未可,家鼐與潘祖廕、翁同龢、孫詒經等再請,始議準。十六年,授都察院左都御史、工部尚書,兼順天府尹。

二十年,中日事起,朝議主戰,家鼐力言釁不可啟。二十四年,以吏部尚書協辦大學士。命為管學大臣。時方議變法,廢科舉,興學校,設報編書,皆特交核覆,家鼐一裁以正。嘗疏謂:「國家廣集卿士以資議政,聽言固不厭求詳,然執兩用中,精擇審處,尤賴聖知。」其所建議,類能持大體。及議廢立,家鼐獨持不可。旋以病乞罷。

二十六年,乘輿西狩,召赴行在,起禮部尚書。還京,拜體仁閣大學士。歷轉東閣、文淵閣,晉武英殿。充學務大臣,裁度規章,折衷中外,嚴定宗旨,一以敦行實學為主,學風為之一靖。議改官制,命與慶親王奕劻、軍機大臣瞿鴻示幾總司核定。御史趙啟霖劾奕劻及其子貝子載振受賄納優,命醇親王載灃與家鼐往按,啟霖坐污衊親貴褫職,而載振尋亦乞罷兼官。資政院立,命貝子溥倫及家鼐為總裁,一持正議不阿。時詔諸臣輪班進講,家鼐撰尚書四子書講義以進。三十四年二月,以鄉舉重逢,賞太子太傅。歷蒙賜「壽」,頒賞御書及諸珍品,賜紫韁,紫禁城內坐二人暖輪,恩遇優渥。宣統元年,再疏乞病,溫詔慰留。尋卒,年八十有二,贈太傅,謚文正。

家鼐簡約斂退,生平無疾言遽色。雖貴,與諸生鈞禮。閉門齋居,雜賓遠跡,推避權勢若怯。嘗督湖北學政,典山西試,再典順天試,總裁會試,屢充閱卷大臣,獨無所私。拔一卷廁二甲,同列意不可,即屏退之,其讓不喜競類此。器量尤廣,庚子,外人請懲禍首戮大臣,編修劉廷琛謂失國體,責宰輔不能爭,家鼐揖而引過。其後詔舉御史。家鼐獨保廷琛,謂曩以大義見責,知忠鯁必不負國,世皆稱之。

張百熙,字埜秋,長沙人。同治十三年進士,授編修。督山東學政,典試四川。命直南書房,再遷侍讀。

光緒二十年,朝鮮釁起,朝議多主戰。百熙疏劾李鴻章陽作戰備,陰實主和,左寶貴、聶士成皆勇敢善戰之將,以餉械不繼,遂致敗績,咎在鴻章;又劾禮親王世鐸筦樞務,招權納賄,戰事起,一倚鴻章,貽誤兵機:皆不報。時值太后萬壽,承辦典禮者猶競尚華飾,百熙奏罷之。復偕侍講學士陸寶忠等合彈樞臣朋比誤國十大罪。未幾,孫毓汶引疾歸,恭親王奕復入軍機,而百熙亦出督廣東學政。累遷內閣學士。二十四年,坐濫舉康有為,革職留任。二十六年,授禮部侍郎,擢左都御史,充頭等專使大臣。拳匪亂定,下詔求言,百熙抗疏陳大計,請改官制,理財政,變科舉,建學堂,設報館。明年,遷工部尚書,調刑部,充管學大臣。

京師之有大學堂也,始於中日戰後。侍郎李端棻奏請立學,中旨報可,而樞府厭言新政,請緩行。迄戊戌,乃奉嚴旨,促擬學章,命孫家鼐為管學大臣。及政變,惟大學以萌芽早得不廢。許景澄繼管學,坐論義和團被誅。兩宮西幸,百熙詣行在,以人望被斯任,於是海內欣然望興學矣。百熙奏加冀州知州吳汝綸五品卿銜,總教大學。汝綸辭不應,百熙具衣冠拜之,汝綸請赴日本察視學務。大學教職員皆自聘,又薪金優厚,忌嫉者眾,蜚語浸聞。汝綸返國,未至京,卒;而百熙所倚以辦學者,門人沈兆祉亦受讒構。大學既負時謗,言官奏稱本朝定制,部官大率滿、漢相維,請更設滿大臣主教事,乃增命榮慶為管學大臣。旋別設學務處,以張亨嘉為大學總監督,百熙權益分。始議分建七科大學,又選派諸生游學東西洋。榮慶意不謂可,而百熙持之堅,親至站送諸生登車。各省之派官費生自此始。值張之洞入覲,命改定學章,及還鎮,復命家鼐為管學大臣。凡三管學,百熙位第三矣。百熙擬建分科大學,以絀於貲而止,惟創醫學及譯學館、實業館,遽謝學務。賞黃馬褂、紫禁城騎馬。後歷禮部、戶部、郵傳部尚書,政務、學務、編纂官制諸大臣。卒,贈太子少保,謚文達。

唐景崇,字春卿,廣西灌陽人。父懋功,舉人,有學行。景崇,同治十年進士,授編修。由侍讀四遷至內閣學士。光緒二十年,典試廣東。明年,主會試。歷兵部、禮部侍郎,權左都御史,出督浙江學政,母憂歸。拳禍起,命督辦廣西團練。二十九年,以工部侍郎典試浙江,督江蘇學政,三十一年,詔罷科歲試,學政專司考校學務。景崇條上十事。明年,罷學政,還京供職。疏陳立憲大要四事。

時兩廣疆臣建議廣西省會移治南寧,京朝官皆持異議。景崇奏陳:「遷省之議,以越南逼近龍州,法人時蓄狡謀,桂林距離遠,聲氣難通,不若改建南寧之便。臣謂不然,今我兵力尚不能經營邕州,扼北海水陸沖要,徒虛張聲勢,招外人疑忌何為?且遷徙締造之費,桂林善後之費,練兵設防之費,皆非巨款不辦。方今俄居西陲,英窺南徼,蒙、藏、川、滇勢均岌岌,非獨一法人之可畏。以大局論,決不能竭全力事廣西之一隅;以廣西論,亦不能竭全力事南寧之一隅:明矣。故為今之計,誠能簡重臣駐龍州,於對汛邊地二千里,相度土宜,興辦樹藝、屯墾、畜牧、開礦諸端,俟地利漸興,人齒漸繁,再以兵法部勒。此上策也。至目前應變之方,莫如迅設龍州電線,移提督駐南寧,增募十營,暫停廣西應解賠款,飭各省欠解廣西協餉,分年攤解,用抵賠款。一轉移間,餉足則兵強,可紓朝廷南顧之憂。若遷省之舉,勞民費財,無益於治。」事得寢。

調吏部侍郎,充經筵講官。景崇以績學端品受主知,屢司文柄。迨科舉罷,廷試游學畢業生,皆倚景崇校閱。宣統元年,戴鴻慈卒,遺疏薦景崇堪大用。二年,擢學部尚書。明年,詔設內閣,改學務大臣。是時學說紛歧,景崇力謀溝通新舊,慎擇教科書。兼任弼德院顧問大臣。武昌變起,袁世凱總理內閣,仍命掌學務。引疾去。越三年,卒,謚文簡。

景崇博覽群書,通天文算術,尤喜治史。自為編修時,取新唐書為作注,大例有三:曰糾繆,曰補闕,曰疏解,甄採書逾數百種。家故貧,得秘籍精本,輒典質購之。殫精畢世,唯缺地理志內羈縻州及藝文志,餘均脫稿。

於式枚,字晦若,賀縣人。博聞強記,善屬文。光緒六年進士,以庶吉士,散館用兵部主事。李鴻章疏調北洋差遣,歷十餘年,奏牘多出其手。性不樂為外吏,又格於例不得保升京秩,久之不遷。二十二年,鴻章賀俄皇加冕,因歷聘德、法、英、美諸國,式枚充隨員。俄選授禮部主事,由員外郎授御史,遷給事中。贊辛丑和約,賞五品京堂。充政務處幫提調、大學堂總辦、譯學館監督。三十一年,以鴻臚寺少卿督廣東學政,改提學使,疏辭,命總理廣西鐵路。三十三年,擢郵傳部侍郎。

當是時,政潮激烈,有詔預備立憲,舉朝競言西法,無敢持異議者。於是式枚奉命出使德國,充考察憲政大臣。瀕行,疏言:「憲政必以本國為根據,採取他國以輔益之,在求其實,不徒震其名。我朝道監百王,科條詳備,行政皆守部章,風聞亦許言事,刑賞予奪,曾不自私。有大政事、大興革,內則集廷臣之議,外或待疆吏之章。勤求民隱,博採公論,與立憲之制無不符合。上有教誡無約誓,下有遵守無要求。至日久官吏失職,或有奉行之不善,海國開通,又有事例之所無,自可因時損益,並非變法更張。惟人心趣向各異,告以堯、舜、周、孔之孔,則以為不足法;告以英、德、法、美之制度,而日本所模仿者,則心悅誠服,以為當行。考日本維新之初,即宣言立憲之意。後十四年,始發布開設國會之敕諭,二十年乃頒行憲法。蓋預備詳密遲慎如此。今橫議者自謂國民,聚眾者輒雲團體,數年之中,內治外交,用人行政,皆有干預之想。動以立憲為詞,紛馳電函,上廑宸慮。蓋以立憲為新奇可喜,不知吾國所自有。其關於學術者,固貽譏荒陋,以立憲為即可施行,不審東洋之近事。關於政術者,尤有害治安。惟在朝廷本一定之指歸,齊萬眾之心志,循序漸進。先設京師議院以定從違,舉辦地方自治以植根本,尤要在廣興教育,儲備人才。凡與憲政相輔而行者,均當先事綢繆者也。臣前隨李鴻章至柏林,略觀大概。今承特簡,謹當參合中、西同異,歸極於皇朝典章,庶言皆有本而事屬可行。是臣區區之至原。」

明年,調禮部侍郎。時新黨要求實行立憲,召集國會日亟。式枚上言:「臣遍考東西歷史,參校同異,大抵中法皆定自上而下奉行,西法則定自下而上遵守。惟日本憲法,則纂自日臣伊藤博文,雖西國之名詞,仍東洋之性質。其採取則普魯士為多,其本原則德君臣所定,名為欽定憲法。夫國所以立曰政,政所以行曰權,權所歸即利所在。定於一則無非分之想,散於眾則有競進之心。行之而善,則為日本之維新;行之不善,則為法國之革命。法國當屢世苛虐之後,民困已深,欲以立憲救亡,而適促其亂。日本當尊王傾幕之時,本由民力,故以立憲為報,而猶緩其期。中國名義最重,政治最寬,國體尊嚴,人情安習,既無法國之怨毒,又非日本之改造。皇上俯順輿情,迭降諭旨,分定年期,自宜互相奮勉,靜待推行。豈容欲速等於取償,求治同於論價?至敢言監督朝廷,推倒政府,胥動浮言,幾同亂黨。欲圖補救之策,惟在朝廷舉錯一秉至公,不稍予以指摘之端,自無從為煽惑之計。至東南各省疆吏,當慎擇有風力、知大體者鎮懾之。當十年預備之期,為大局安危所系。日皇所謂『組織權限,為朕親裁』,德相所謂『法定於君,非民可解』。故必正名定分,然後措正施行。臣濫膺考察,斷不敢附會時趨,貽誤國家,得罪名教。」章下所司。尋調吏部侍郎。

上海政聞社法部主事陳景仁等電請定三年內開國會,罷式枚謝天下,嚴旨申飭,褫景仁職。式枚復奏言:「德皇接受國書,答言憲政紛繁,慮未必合中國用,選舉法尤未易行。又昔英儒斯賓塞爾亦甚言憲法流弊,謂美國憲法本人民平等,行之久而治權握於政黨,平民不勝其苦。蓋歐人言憲法,其難其慎如此。今橫議遍於國中,上則詆政府固權,下則罵國民失職,專以爭競相勸導。此正斯賓塞爾所云政黨者流,與平民固無與也。伊藤博文論君臣相與,先道德而後科條。君民何獨不然?果能誠信相接,則普與日本以欽定憲法行之至今;如其不然,則法蘭西固民約憲法,何以革命者再三,改法者數十而猶未定?臣愚以為中國立憲,應以日本仿照普魯士之例為權衡,以畢士麥由君主用人民意見制定,及伊藤博文先道德後科條之言為標準,則憲法大綱立矣。」章下所司。又以各省諮議局章程與普國地方議會制度不符,大恉謂:「改革未定之時,中央政權唯恐少統一堅強之力,而國民識政體知法意者極少。驟以此龐大政權之地方議會,橫亙政府與國民之間,縱使被選者不皆營私武斷,而一國政權落於最少數人之手,劫持中外大臣,後患何可勝言?」因證以普制,逐條駁議。先後譯奏普魯士憲法全文、官制位號等級,暨兩議院新舊選舉法。式枚以三十三年冬行,宣統元年六月返國,以疾乞假。張之洞遺疏薦式枚堪大用。轉吏部侍郎,改學部侍郎,總理禮學館事、修訂法律大臣、國史館副總裁。國變後,僑居青島。未幾,卒,年六十三,謚文和。

式枚生而隱宮,精力絕人,夜倚枕坐如枯僧。內介而外和易。論事謇諤,頗有聲公卿間云。

沈家本,字子惇,浙江歸安人。少讀書,好深湛之思,於周官多創獲。初援例以郎中分刑部,博稽掌故,多所纂述。光緒九年,成進士,仍留部。補官後,充主稿,兼秋審處。自此遂專心法律之學,為尚書潘祖廕所稱賞。十九年,出知天津府,治尚寬大,奸民易之,聚眾鬥於市,即擒斬四人,無敢復犯者。調劇保定,甘軍毀法國教堂,當路懾於外勢,償五萬金,以道署舊址建新堂,侵及府署東偏。家本據府志力爭得直。拳匪亂作,家本已擢通永道、山西按察使,未及行,兩宮西幸。聯軍入保定,教士銜前隙,誣以助拳匪,卒無左驗而解。因馳赴行在,授光祿寺卿,擢刑部侍郎。

自各國互市以來,內地許傳教,而中外用律輕重懸殊,民、教日齟。官畏事則務抑民,民不能堪,則激而一逞,往往焚戮成巨禍。家本以謂治今日之民,當令官吏普通法律。然中律不變而欲收回領事審判權,終不可得。會變法議起,袁世凱奏設修訂法律館,命家本偕伍廷芳總其事;別設法律學堂,畢業者近千人,一時稱盛。補大理寺卿,旋改法部侍郎,充修訂法律大臣。宣統元年,兼資政院副總裁,仍日與館員商訂諸法草案,先後告成,未嘗以事繁自解。其所著書,有讀律校勘記、秋讞須知、刑案匯覽、刺字集、律例偶箋、歷代刑官考、歷代刑法考、漢律摭遺、明大誥竣令考、明律目箋,他所著非刑律者又二十餘種,都二百餘卷。卒,年七十四。

論曰:自變法議興,凡新政特設大臣領之。百熙管學務,家本修法律,並邀時譽。景崇之主教育,謀溝通新舊;式枚之論憲政,務因時損益。而大勢所趨,已莫能挽救。家鼐儒厚廉謹,常以資望領新政,每參大計,獨持正不阿。賢哉,不愧古大臣矣!


\end{pinyinscope}