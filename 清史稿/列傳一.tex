\article{列傳一}

\begin{pinyinscope}
后妃

顯祖宣皇后繼妃庶妃

太祖孝慈高皇后元妃繼妃大妃壽康太妃太祖諸妃

太宗孝端文皇后孝莊文皇后敏惠恭和元妃

懿貴妃康惠淑妃太宗諸妃

世祖廢後孝惠章皇后孝康章皇后孝獻皇后貞妃

淑惠妃世祖諸妃

聖祖孝誠仁皇后孝昭仁皇后孝懿仁皇后孝恭仁皇后

敬敏皇貴妃定妃通嬪惇怡皇貴妃?惠皇貴妃聖祖諸妃

世宗孝敬憲皇后孝聖憲皇后敦肅皇貴妃

純?皇貴妃世宗諸妃

高宗孝賢純皇后皇後烏拉納喇氏孝儀純皇后

慧賢皇貴妃純惠皇貴妃慶恭皇貴妃哲憫皇貴妃

淑嘉皇貴妃婉貴太妃高宗諸妃

仁宗孝淑睿皇后孝和睿皇后恭順皇貴妃

和裕皇貴妃仁宗諸妃

宣宗孝穆成皇后孝慎成皇后孝全成皇后孝靜成皇后

莊順皇貴妃彤貴妃宣宗諸妃

文宗孝德顯皇后孝貞顯皇后孝欽顯皇后

莊靜皇貴妃玫貴妃端恪皇貴妃文宗諸妃

穆宗孝哲毅皇后淑慎皇貴妃

莊和皇貴妃敬懿皇貴妃榮惠皇貴妃

德宗孝定景皇后端康皇貴妃恪順皇貴妃

宣統皇后淑妃

太祖初起,草創闊略,宮闈未有位號,但循國俗稱「福晉」。福晉蓋「可敦」之轉音,史述後妃,後人緣飾名之,非當時本稱也。崇德改元,五宮並建,位號既明,等威漸辨。世祖定鼎,循前代舊典。順治十五年,採禮官之議:乾清宮設夫人一,淑儀一,婉侍六,柔婉、芳婉皆三十;慈寧宮設貞容一、慎容二,勤侍無定數;又置女官。循明六局一司之制,議定而未行。

康熙以後,典制大備。皇后居中宮;皇貴妃一,貴妃二,妃四,嬪六,貴人、常在、答應無定數,分居東、西十二宮。東六宮:曰景仁,曰承乾,曰鍾粹,曰延禧,曰永和,曰景陽;西六宮:曰永壽,曰翊坤,曰儲秀,曰啟祥,曰長春,曰咸福。諸宮皆有宮女子供使令。每三歲選八旗秀女,戶部主之;每歲選內務府屬旗秀女,內務府主之。秀女入宮,妃、嬪、貴人惟上命。選宮女子,貴人以上,得選世家女;貴人以下,但選拜唐阿以下女。宮女子侍上,自常在、答應漸進至妃、嬪,後妃諸姑、姊妹不赴選。帝祖母曰「太皇太后」,母曰「皇太后」,居慈寧、壽康、寧壽諸宮。先朝妃、嬪稱太妃、太嬪,隨皇太后同居,與嗣皇帝,年皆逾五十,乃始得相見。諸宮殿設太監,秩最高不逾四品,員額有定數,廩給有定量,分領執事有定程。此其大較也。

二百數十年,壼化肅雍,詖謁蓋寡,內鮮燕溺匹嫡之嫌,外絕權戚蠹國之釁,彬彬盛矣。追尊四代,惟宣皇后著氏族,且有繼室,託始於是。歷朝居正號者,謹而次之,並及妃、嬪有子若受後朝尊封者。世祖以漢女為妃,高宗以回女為妃,附書之,以其僅見也。

顯祖宣皇后,喜塔臘氏,都督阿古女。歸顯祖為嫡妃。歲己未,太祖生。歲己巳,崩。順治五年,與肇祖原皇后、興祖直皇后、景祖翼皇后同時追謚。子三:太祖、舒爾哈齊、雅爾哈齊。女一,下嫁噶哈善哈斯虎。

繼妃,納喇氏,哈達部長萬所撫族女。遇太祖寡恩,年十九,俾分居,予產獨薄。子一,巴雅喇。庶妃,李佳氏。子一,穆爾哈齊。

太祖孝慈高皇后,納喇氏,葉赫部長楊吉砮女。太祖初起兵,如葉赫,楊吉砮以後許焉。楊吉砮為明總兵李成梁所殺,子納林布祿繼為貝勒,又為成梁擊破。歲戊子秋九月,以後來歸,上率諸貝勒、大臣迎之,大宴成禮。是歲,後年十四。歲壬辰冬十月,太宗生。歲癸卯秋,後病作,思見母,上遣使迎焉,納林布祿不許。九月庚辰,後崩,年二十九。

後莊敬聰慧,詞氣婉順,得譽不喜,聞惡言,愉悅不改其常。不好諂諛,不信讒佞,耳無妄聽,口無妄言。不預外事,殫誠畢慮以事上。及崩,上深悼之,喪斂祭享有加禮,不飲酒茹葷者逾月。越三載,葬赫圖阿拉尼雅滿山岡。天命九年,遷葬東京楊魯山。天聰三年,再遷葬沈陽石嘴頭山,是為福陵。崇德元年,上謚孝慈昭憲純德真順承天育聖武皇后。順治元年,祔太廟。康熙元年,改謚。雍正、乾隆累加謚,曰孝慈昭憲敬順仁徽懿德慶顯承天輔聖高皇后。子一,太宗。

元妃,佟佳氏。歸太祖最早。子二:褚英、代善。女一,下嫁何和禮。

繼妃,富察氏。歸太祖亦在孝慈皇后前。歲癸巳,葉赫諸部來侵,上夜駐軍,寢甚酣,妃呼上覺曰:「爾方寸亂耶,懼耶?九國兵來攻,豈酣寢時耶?」上曰:「我果懼,安能酣寢?我聞葉赫來侵,以其無期,時以為念。既至,我心安矣。我若負葉赫,天必厭之,安得不懼?今我順天命,安疆土,彼糾九國以虐無咎之人,天不佑也!」安寢如故。及旦,遂破敵。天命五年,妃得罪,死。子二:莽古爾泰、德格類。女一,名莽古濟,下嫁鎖諾木杜棱。

大妃,納喇氏,烏喇貝勒滿泰女。歲辛丑,歸太祖,年十二。孝慈皇后崩,立為大妃。天命十一年七月,太祖有疾,浴於湯泉。八月,疾大漸,乘舟自太子河還,召大妃出迎,堡,上崩。辛亥,大妃殉焉,年三十七。同殉者,二庶妃。妃子三?入渾河。庚戌,舟次靉:阿濟格、多爾袞、多鐸。順治初,多爾袞攝政,七年,上謚孝烈恭敏獻哲仁和贊天儷聖武皇后,祔太廟。八年,多爾袞得罪,罷謚,出廟。

壽康太妃,博爾濟吉特氏,科爾沁郡王孔果爾女。太祖諸妃中最老壽。順治十八年,聖祖即位,尊為皇曾祖壽康太妃。康熙四年,薨。

太祖諸妃稱側妃者四:伊爾根覺羅氏,子一,阿巴泰,女一,下嫁達爾漢;納喇氏,孝慈皇后女弟,女一,下嫁固爾布什;其二皆無出。稱庶妃者五:兆佳氏,子一,阿拜;鈕祜祿氏,子二,湯古代、塔拜;嘉穆瑚覺羅氏,子二,巴布泰、巴布海,女三,下嫁布占泰、達啟、蘇納;西林覺羅氏,子一,賴慕布;伊爾根覺羅氏,女一,下嫁鄂托伊。

太宗孝端文皇后,博爾濟吉特氏,科爾沁貝勒莽古思女。歲甲寅四月,來歸,太祖命太宗親迎,至輝發扈爾奇山城,大宴成禮。天聰間,後母科爾沁大妃屢來朝,上迎勞,錫賚有加禮。崇德元年,上建尊號,後亦正位中宮。二年,大妃復來朝,上迎宴。越二日,大妃設宴,上率後及貴妃、莊妃幸其行幄。尋命追封后父莽古思和碩福親王,立碑於墓,封大妃為和碩福妃,使大學士範文程等冊封。世祖即位,尊為皇太后。順治六年四月乙巳,崩,年五十一。七年,上謚。雍正、乾隆累加謚,曰孝端正敬仁懿哲順慈僖莊敏輔天協聖文皇后。女三,下嫁額哲、奇塔特、巴雅思祜朗。

孝莊文皇后,博爾濟吉特氏,科爾沁貝勒寨桑女,孝端皇后侄也。天命十年二月,來歸。崇德元年,封永福宮莊妃。三年正月甲午,世祖生。世祖即位,尊為皇太后。順治十一年,贈太后父寨桑和碩忠親王,母賢妃。十三年二月,太后萬壽,上制詩三十首以獻。上承太后訓,撰內則衍義,並為序以進。聖祖即位,尊為太皇太后。

康熙九年,上奉太后謁孝陵。十年,謁福陵、昭陵。十一年,幸赤城湯泉,經長安嶺,上下馬,扶輦;至坦道,始上馬以從。還,度嶺,正大雨,仍下馬,扶輦。太后命騎從,上不可,下嶺,乃乘馬傍輦行。吳三桂亂作,頻年用兵,太后念從征將士勞苦,發宮中金帛加犒。聞各省有偏災,輒發帑賑恤。布爾尼叛,師北征,太后以慈寧宮庶妃有母年九十餘,居察哈爾,告上誡師行毋擄掠。

國初故事,後妃,王、貝勒福晉,貝子、公夫人,皆令命婦更番入侍,至太后始命罷之。宮中守祖宗制,不蓄漢女。上命儒臣譯大學衍義進太后,太后稱善,賜賚有加。太后不預政,朝廷有黜陟,上多告而後行。嘗勉上曰:「祖宗騎射開基,武備不可弛。用人行政,天子以一身臨其上,?務敬以承天,虛公裁決。」又作書以誡曰:「古稱為君難。蒼生至得國之道,使四海咸登康阜,綿歷數於無疆,惟休。汝尚?生養撫育,莫不引領,必深思得寬裕慈仁,溫良恭敬,慎乃威儀,謹爾出話,夙夜恪勤,以祗承祖考遺緒,俾予亦無疚於厥心。」十九年四月,上撰大德景福頌進太后。

二十年,上復奉太后幸湯泉。雲南平,上詣太后宮奏捷。二十一年,上詣奉天謁陵,途次屢奏書問安,使獻方物,奏曰:「臣到盛京,親網得鰱、,浸以羊脂,山中野燒,自落榛實及山核桃,朝鮮所進柿餅、松、慄、銀杏,附使進上,伏乞俯賜一笑,不勝欣幸。」二十二年夏,奉太后出古北口避暑。秋,幸五臺山,至龍泉關。上以長城嶺峻絕,試輦不能陟,奏太后。次日,太后輦登嶺,路數折不可上,太后乃還龍泉關,命上代禮諸寺。二十四年夏,上出塞避暑,次博洛和屯,聞太后不豫,即馳還京師,太后疾良已。

二十六年九月,太后疾復作,上晝夜在視。十二月,步禱天壇,請減算以益太后。讀祝,上泣,陪祀諸王大臣皆泣。太后疾大漸,命上曰:「太宗奉安久,不可為我輕動。況我心戀汝父子,當於孝陵近地安厝,我心始無憾。」己巳,崩,年七十五。上哀慟,欲於宮中持服二十七月,王大臣屢疏請遵遺誥,以日易月,始從之。命撤太后所居宮移建昌瑞山孝陵近地,號「暫安奉殿」。二十七年四月,奉太后梓宮詣昌瑞山。自是,歲必詣謁。雍正三年十二月,世宗即其地起陵,曰昭西陵。

世祖親政,上太后徽號,國有慶,必加上。至聖祖以雲南平,奏捷,定徽號曰昭聖慈壽恭簡安懿章慶敦惠溫莊康和仁宣弘靖太皇太后,初奉安上謚。雍正、乾隆累加謚,曰孝莊仁宣誠憲恭懿至德純徽翊天啟聖文皇后。子一,世祖。女三,下嫁弼爾塔哈爾、色布騰、鏗吉爾格。

敏惠恭和元妃,博爾濟吉特氏,孝莊皇后姊也。天聰八年,來歸。崇德元年,封關睢宮宸妃。妃有寵於太宗,生子,為大赦,子二歲而殤,未命名。六年九月,太宗方伐明,聞妃病而還,未至,妃已薨。上慟甚,一日忽迷惘,自午至酉始瘥,乃悔曰:「天生朕為撫世安民,豈為一婦人哉?朕不能自持,天地祖宗特示譴也。」上仍悲悼不已。諸王大臣請出獵,遂獵蒲河。還過妃墓,復大慟。妃母和碩賢妃來吊,上命內大臣掖輿臨妃墓。郡王阿達禮、輔國公扎哈納當妃喪作樂,皆坐奪爵。

懿靖大貴妃,博爾濟吉特氏,阿霸垓郡王額齊格諾顏女。崇德元年,封麟趾宮貴妃。四年,額齊格諾顏及其妻福晉來朝,妃率諸王、貝勒迎宴。次日,上賜宴清寧宮,福晉入見,稱上外姑。順治九年,世祖加尊封。康熙十三年,薨,聖祖侍太后臨奠。子一,博穆博果爾。女一,下嫁噶爾瑪索諾木。又撫蒙古女,嫁噶爾瑪德參,濟旺子也。

康惠淑妃,博爾濟吉特氏,阿霸垓塔布囊博第塞楚祜爾女。崇德元年,封衍慶宮淑妃。撫蒙古女,上命睿親王多爾袞娶焉。順治九年,加尊封,前懿靖大貴妃薨。

太宗諸妃:元妃,鈕祜祿氏,弘毅公額亦都女,子一,洛博會;繼妃,烏拉納喇氏,子二,豪格、洛格,女一,下嫁旺第。稱側妃者二:葉赫納喇氏,子一,碩塞;扎魯特博爾濟吉特氏,女二,下嫁誇扎、哈尚。稱庶妃者六:納喇氏,子一,高塞,女二,下嫁輝塞、拉哈;奇壘氏,察哈爾部人,女一,下嫁吳應熊;顏札氏,子一,葉布舒;伊爾根覺羅氏,子一,常舒;其二不知氏族,一生子,韜塞;一生女,下嫁班第。

世祖廢後,博爾濟吉特氏,科爾沁卓禮克圖親王吳克善女,孝莊文皇后侄也。後麗而慧,睿親王多爾袞攝政,為世祖聘焉。順治八年八月,冊為皇后。上好簡樸,後則嗜奢侈,又?,積與上忤。

十年八月,上命大學士馮銓等上前代廢后故事,銓等疏諫,上嚴拒,諭以「無能,故當廢」,責諸臣沽名。即日奏皇太后,降後為靜妃,改居側宮,下禮部,禮部尚書胡世安、侍郎呂崇烈、高珩疏請慎重詳審,禮部員外郎孔允樾及御史宗敦一、潘朝選、陳棐、張璟、杜果、聶玠、張嘉、李敬、劉秉政、陳自德、祖永傑、高爾位、白尚登、祖建明各具疏力爭。允樾言尤切,略言:「皇后正位三年,未聞失德,特以『無能』二字定廢嫡之案,何以服皇后之心?何以服天下後世之心?君後猶父母,父欲出母,即心知母過,猶涕泣以諫;況不知母過何事,安忍緘口而不為母請命?」上命諸王、貝勒、大臣集議,議仍以皇后位中宮,而別立東西兩宮。上不許,令再議,並責允樾覆奏,允樾疏引罪,諸王大臣再議,請從上指,於是後竟廢。

孝惠章皇后,博爾濟吉特氏,科爾沁貝勒綽爾濟女。順治十一年五月,聘為妃,六月,冊為後。貴妃董鄂氏方幸,後又不當上恉。十五年正月,皇太后不豫,上責後禮節疏闕,命停應進中宮箋表,下諸王、貝勒、大臣議行。三月,以皇太后旨,如舊制封進。

聖祖即位,尊為皇太后,居慈仁宮。上奉太皇太后謁孝陵,幸盛京,謁福陵、昭陵,出古北口避暑,幸五臺山,皆奉太后侍行。康熙二十二年,上奉太皇太后出塞,太后未侍行,中途射得鹿,斷尾漬以鹽,並親選榛實,進太后。二十六年,太皇太后不豫,太后朝夕奉侍。及太皇太后崩,太后悲痛。諸妃主入臨,太后慟甚,幾僕地。上命諸王大臣奏請太后節哀回宮,再請乃允。歲除,諸王大臣請太后諭上回宮,上不可。二十七年正月,行虞祭,上命諸王大臣請太后勿往行禮,太后亦不可。二十八年,建寧壽新宮,奉太后居焉。

三十五年十月,上北巡,太后萬壽,上奉書稱祝。駐麗蘇,太后遣送衣裘,上奉書言:「時方燠,河未冰,帳房不須置火,俟嚴寒,即歡忭而服之。」三十六年二月,上親征噶爾丹,駐他喇布拉克。太后以上生日,使賜金銀茶壺,上奉書拜受。噶爾丹既定,?臣請上加太后徽號壽康顯寧,太后以上不受尊號,亦堅諭不受。三十七年七月,奉太后幸盛京謁陵,道喀喇沁。途中以太后父母葬發庫山,距蹕路二百里,諭內大臣索額圖擇潔地,太后遙設祭。十月,次奇爾賽畢喇,值太后萬壽,上詣行宮行禮,敕封太后所駐山曰壽山。

三十八年,上奉太后南巡。三十九年十月,太后六十萬壽,上制萬壽無疆賦,並奉佛像,珊瑚,自鳴鐘,洋鏡,東珠,珊瑚、金珀、御風石,念珠,皮裘,羽緞,哆羅呢,沈、檀、蕓、降諸香,犀玉、瑪瑙、赩、漆諸器,宋、元、明名畫,金銀、幣帛;又令膳房數米萬粒,號「萬國玉粒飯」,及肴饌、果品以獻。四十九年,太后七十萬壽,亦如之。

五十六年十二月,太后不豫。是歲,上春秋六十有四,方有疾,頭眩足腫,聞太后疾輿詣視,跪?下,捧太后手曰:「母後,臣在此!」太后張目,畏明,?甚,以帕?足,乘障以手,視上,執上手,已不能語。上力疾,於蒼震門內支幄以居。丙戌,太后崩,年七十七。上號慟盡禮。五十七年三月,葬孝陵之東,曰孝東陵。初上太后徽號,國有慶,必加上。至雲南平,定曰仁憲恪順誠惠純淑端禧皇太后。及崩,上謚,大學士等初議誤不系世祖謚,上令至太廟、奉先殿瞻禮高皇后、文皇后神位,大學士等引罪;又以所擬謚未多留徽號字,命更議。雍正、乾隆累加謚,曰孝惠仁憲端懿慈淑恭安純德順天翼聖章皇后。

孝康章皇后,佟佳氏,少保、固山額真佟圖賴女。後初入宮,為世祖妃。順治十一年春,妃詣太后宮問安,將出,衣裾有光若龍繞,太后問之,知有?,謂近侍曰:「朕?皇帝實有斯祥,今妃亦有是,生子必膺大福。」三月戊申,聖祖生。聖祖即位,尊為皇太后。康熙二年二月庚戌,崩,年二十四。初上徽號曰慈和皇太后。及崩,葬孝陵,上謚。雍正、乾隆累加謚,曰孝康慈和莊懿恭惠溫穆端靖崇文育聖章皇后。後家佟氏,本漢軍,上命改佟佳氏,入滿洲。後族?旗自此始。子一,聖祖。

孝獻皇后,棟鄂氏,內大臣鄂碩女。年十八入侍,上眷之特厚,寵冠後宮。順治十三年八月,立為賢妃。十二月,進皇貴妃,行冊立禮,頒赦。上皇太后徽號,鄂碩本以軍功授一等精奇尼哈番,進三等伯。十七年八月,薨,上輟朝五日。追謚孝獻莊和至德宣仁溫惠端敬皇后。

上親制行狀,略曰:「後兒靜循禮,事皇太后,奉養甚至,左右趨走,皇太后安之。事朕,晨夕候興居,視飲食服御,曲體罔不悉。朕返蹕晏,必迎問寒暑,意少■H3,則曰:『陛下歸晚,體得毋倦耶?』趣具餐,躬進之,命共餐,則辭。朕值慶典,舉數觴,必誡侍者,室無過燠,中夜罝罝起視。朕省封事,夜分,未嘗不侍側。諸曹循例章報,朕輒置之,後曰:『此雖奉行成法,安知無當更張,或有他故?奈何忽之!』令同閱,起謝:『不敢干政。』覽廷讞疏,握筆未忍下,後問是疏安所云,朕諭之,則泣曰:『諸闢皆愚無知,豈盡無冤?宜求可矜宥者全活之!』大臣偶得罪,朕或不樂,後輒請霽威詳察。朕偶免朝,則諫毋倦勤。日講後,與言章句大義,輒喜。偶遺忘,則諫:『當服膺默識。』蒐狩,親騎射,則諫:『毋以萬邦仰庇之身,輕於馳驟。』偶有未稱旨,朕或加譙讓,始猶自明無過;及聞姜后脫簪事,即有宜辯者,但引咎自責而已。後至節儉,不用金玉。誦四書及易已卒業;習書,未久即精。朕喻以禪學,參究若有所省。後初病,皇太后使問安否,必對曰:『安。』疾甚,朕及今後、諸妃、嬪環視之,後曰:『吾殆將不起,此中澄定,亦無所苦,獨不及酬皇太后暨陛下恩萬一。妾歿,陛下宜自愛!惟皇太后必傷悼,奈何?』既又令以諸王賻施貧乏,復屬左右毋以珍麗物斂。歿後,皇太后哀之甚。」行狀數千言,又命大學士金之俊別作傳。是歲,命秋讞停決,從後志也。

時鄂碩已前卒,後世父羅碩,授一等阿思哈尼哈番。及上崩,遺詔以後喪祭逾禮為罪己之一。康熙二年,合葬孝陵,主不祔廟,歲時配食饗殿。子一,生三月而殤,未命名。

貞妃,棟鄂氏,一等阿達哈哈番巴度女。殉世祖。聖祖追封為皇考貞妃。

淑惠妃,博爾濟吉特氏,孝惠皇后妹也。順治十一年,冊為妃。康熙十二年,尊封皇考淑惠妃。妃最老壽,以五十二年十月薨。

同時尊封者:浩齊特博爾濟吉特氏為恭靖妃,阿霸垓博爾濟吉特氏為端順妃,皆無所出;棟鄂氏為寧?妃,在世祖時號庶妃,子一,福全。又恪妃,石氏,灤州人,吏部侍郎申女。世祖嘗選漢官女備六宮,妃與焉。居永壽宮。康熙六年薨,聖祖追封皇考恪妃。

又在三妃前,世祖庶妃有子女者,又有八人:穆克圖氏,子永幹,八歲殤;巴氏,子鈕鈕,為世祖長子,二歲殤,女二,一六歲殤,一七歲殤;陳氏,子一,常寧;唐氏,子一,奇授,七歲殤;鈕氏,子一,隆禧;楊氏,女一,下嫁納爾杜;烏蘇氏,女一,八歲殤;納喇氏,女一,五歲殤。

內大臣噶布喇女。康?聖祖孝誠仁皇后,赫舍里氏,輔政大臣、一等大臣索尼孫領侍熙四年七月,冊為皇后。十三年五月丙寅,生皇二子允礽,即於是日崩,年二十二。謚曰仁孝皇后。二十年,葬孝東陵之東,即景陵也。雍正元年,改謚。乾隆、嘉慶累加謚,曰孝誠恭肅正惠安和淑懿恪敏儷天襄聖仁皇后。子二:承祐,四歲殤;允礽。

孝昭仁皇后,鈕祜祿氏,一等公遏必隆女。初為妃。康熙十六年八月,冊為皇后。十七年二月丁卯,崩。二十年,與仁孝皇后同葬。上每謁孝陵,輒臨仁孝、孝昭兩後陵奠醊。乾隆、嘉慶累加謚,曰孝昭靜淑明惠正和安裕端穆欽天順聖仁皇后。

孝懿仁皇后,佟佳氏,一等公佟國維女,孝康章皇后侄女也。康熙十六年,為貴妃。二十年,進皇貴妃。二十八年七月,病篤,冊為皇后。翌日甲辰,崩。謚曰孝懿皇后。是冬,葬仁孝、孝昭兩後之次。雍正、乾隆、嘉慶累加謚,曰孝懿溫誠端仁憲穆和恪慈惠奉天佐聖仁皇后。女一,殤。

孝恭仁皇后,烏雅氏,護軍參領威武女。後事聖祖。康熙十七年十月丁酉,世宗生。十八年,為德嬪。二十年,進德妃。世宗即位,尊為皇太后,擬上徽號曰仁壽皇太后,未上冊。雍正元年五月辛丑,崩,年六十四。葬景陵。上謚,曰孝恭宣惠溫肅定裕慈純欽穆贊天承聖仁皇后。子三:世宗,允祚,允。允祚六歲殤。女三:其二殤,一下嫁舜安顏。

敬敏皇貴妃,章佳氏。事聖祖為妃。康熙三十八年,薨。謚曰敏妃。雍正初,世宗以其子怡親王允祥賢,追進封。妃又生女二,下嫁倉津、多爾濟。

定妃,萬琉哈氏。事聖祖為嬪。世宗尊為皇考定妃。就養其子履親王允祹邸。高宗朝,歲時伏臘,輒迎入宮中上壽,然未進尊封。薨年九十七。

通嬪,納喇氏。事聖祖為貴人。雍正二年,世宗以其?喀爾喀郡王策棱功,尊封。乾隆九年,薨。子二:萬黼,五歲殤;允禶,二歲殤。女一。

惇怡皇貴妃,瓜爾佳氏。事聖祖為和妃。世宗尊為皇考貴妃。高宗尊為皇祖溫惠皇貴太妃。乾隆三十三年,薨,年八十六。謚曰惇怡皇貴妃。葬景陵側皇貴妃園寢。女一,殤。聖祖諸妃,妃薨最後。

乾隆初,同時尊封者:?惠皇貴妃,佟佳氏,孝懿皇后妹。事聖祖為貴妃。世宗尊為皇考皇貴妃。高宗尊為皇祖壽祺皇貴太妃。薨,謚曰?惠皇貴妃。順懿密太妃,王氏。初為密嬪,自密妃尊封。子三:允潖、允祿、允祄,允祄八歲殤。純裕勤太妃,陳氏。初為勤嬪,自勤妃尊封。子一,允禮。襄嬪,高氏。自貴人尊封。子一,允禕。女一,殤。謹嬪,色赫圖氏。自貴人尊封。子一,允祜。靜嬪,石氏。自貴人尊封。子一,允祁。熙嬪,陳氏。自貴人尊封。子一,允禧。穆嬪,陳氏。自貴人卒後追尊封。子一,允祕。

其卒於康熙中及雖下逮雍正、乾隆而未尊封者,又有:溫僖貴妃,鈕祜祿氏,孝昭皇后妹。子一,允示我。女一,殤。惠妃,納喇氏。子二:承慶,殤;允禔。宜妃,郭絡羅氏。當聖祖崩時,妃方病,以四人舁軟榻詣喪所,出太后前,世宗見之,又傲,世宗為詰責宮監。子三:允祺、允禟、允禌,允帟十二歲殤。榮妃,馬佳氏。子五:承瑞,為聖祖長子,四歲殤;賽音察渾,長華,長生皆殤;允祉。女一,下嫁烏爾滾。成妃,戴佳氏。子一,允祐氏。子一,允禩。平妃,赫舍里氏,孝誠皇后妹。子一,允禨,殤。端嬪,董氏?。良妃,。女一,殤。貴人,兆佳氏。女一,下嫁噶爾臧。郭絡羅氏,宜妃妹。子一,允騕,殤。女一,下嫁敦多布多爾濟。袁氏,女一,下嫁孫承運。陳氏,子一,允■H4,殤。庶妃,鈕祜祿氏,女一;張氏,女二;王氏,女一;劉氏,女一:皆殤。

世宗孝敬憲皇后,烏喇那拉氏,內大臣費揚古女。世宗為皇子,聖祖冊后為嫡福晉。雍正元年,冊為皇后。九年九月己丑,崩。時上病初愈,欲親臨含斂,諸大臣諫止。上諭曰:「皇后自垂髫之年,奉皇考命,作配朕躬。結褵以來,四十餘載,孝順恭敬,始終一致。朕調理經年,今始全愈,若親臨喪次,觸景增悲,非攝養所宜。但皇后喪事,國家典儀雖備,而朕禮數未周。權衡輕重,如何使情文兼盡,其具議以聞。」諸大臣議,以明會典皇后喪無親臨祭奠之禮,令皇子朝夕奠,遇祭,例可遣官,乞停親奠,從之。謚孝敬皇后。及世宗崩,合葬泰陵。乾隆、嘉慶累加謚,曰孝敬恭和懿順昭惠莊肅安康佐天翊聖憲皇后。

孝聖憲皇后,鈕祜祿氏,四品典儀凌柱女。後年十三,事世宗潛邸,號格格。康熙五十年八月庚午,高宗生。雍正中,封熹妃,進熹貴妃。高宗即位,以世宗遺命,尊為皇太后,居慈寧宮。高宗事太后孝,以天下養,惟亦兢兢守家法,重國體。太后偶言順天府東有廢寺當重修,上從之。即召宮監,諭:「汝等嘗侍聖祖,幾曾見昭聖太后當日令聖祖修蓋廟宇?嗣後當奏止!」宮監引悟真庵尼入內,導太后弟入蒼震門謝恩,上屢誡之。上每出巡幸,輒奉太后以行,南巡者三,東巡者三,幸五臺山者三,幸中州者一。謁孝陵,獮木蘭,歲必至焉。遇萬壽,率王大臣奉觴稱慶。

乾隆十六年,六十壽;二十六年,七十壽;三十六年,八十壽:慶典以次加隆。先期,日進壽禮九九。先以上親制詩文、書畫,次則如意、佛像、冠服、簪飾、金玉、犀象、瑪瑙、水晶、玻璃、琺瑯、彞鼎、赩器、書畫、綺繡、幣帛、花果,諸外國珍品,靡不具備。太后為天下母四十餘年,國家全盛,親見曾玄。

四十二年正月庚寅,崩,年八十六。葬泰陵東北,曰泰東陵。初尊太后,上徽號。國有慶,屢加上,曰崇德慈宣康惠敦和裕壽純禧恭懿安祺寧豫皇太后。既葬,上謚。嘉慶中,再加謚,曰孝聖慈宣康惠敦和誠徽仁穆敬天光聖憲皇后。子一,高宗。

敦肅皇貴妃,年氏,巡撫遐齡女。事世宗潛邸,為側福晉。雍正元年,封貴妃。三年十一月,妃病篤,進皇貴妃。並諭妃病如不起,禮儀視皇貴妃例行。妃薨逾月,妃兄羹堯得罪死。謚曰敦肅皇貴妃。乾隆初,從葬泰陵。子三:福宜、福惠、福沛,皆殤。女一,亦殤。

純?皇貴妃,耿氏。事世宗潛邸,為格格。雍正間,封裕嬪,進裕妃。高宗時,屢加尊為裕皇貴太妃。乾隆四十九年,薨,年九十六。謚曰純?皇貴妃。葬妃園寢,位諸妃上。子一,弘晝。

世宗諸妃,又有:齊妃,李氏。事世宗潛邸,為側室福晉。雍正間,封齊妃。子三:弘盼、弘昀,皆殤;弘時。女一,下嫁星德。謙妃,劉氏。事世宗潛邸,號貴人。雍正間,封謙嬪。高宗尊為皇考謙妃。子一,弘適。懋嬪,宋氏。事世宗,號格格。雍正初,封懋嬪。女二,皆殤。

高宗孝賢純皇后,富察氏,察哈爾總管李榮保女。高宗為皇子,雍正五年,世宗冊后為嫡福晉。乾隆二年,冊為皇后。後恭儉,平居以通草絨花為飾,不御珠翠。歲時以鹿羔沴毧制為荷包進上,仿先世關外遺制,示不忘本也。上甚重之。十三年,從上東巡,還蹕,三月乙未,後崩於德州舟次,年三十七。上深慟,兼程還京師,殯於長春宮,服縞素十二日。

初,皇貴妃高佳氏薨,上謚以慧賢,後在側,曰:「吾他日期以『孝賢』,可乎?」至是,上遂用為謚。並制述悲賦,曰:「易何以首乾坤?詩何以首關睢?惟人倫之伊始,固天儷之與齊。念懿后之作配,廿二年而於斯。痛一旦之永訣,隔陰陽而莫知。昔皇考之命偶,用掄德於名門。俾逑予而尸藻,定嘉禮於渭濱。在青宮而養德,即治壼而淑身。縱糟糠之未歷,實同甘而共辛。乃其正位坤寧,克贊乾清。奉慈闈之溫凊,為九卿之儀型。克儉於家,爰始繅品而育繭;克勤於邦,亦知較雨而課晴。嗟予命之不辰兮,痛元嫡之連棄。致黯然以內傷兮,遂邈爾而長逝。撫諸子如一出兮,豈彼此之分視?值乖舛之疊遘兮,誰不增夫怨封心?況顧予之傷悼兮,更怳悢而切意。尚強歡以相慰兮,每禁情而制淚。制淚兮淚滴襟,勞,促歸程兮變故遭。登畫?強歡兮歡匪心。聿當春而啟轡,隨予駕以東臨。抱輕疾兮念舫兮陳翟褕,由潞河兮還內朝。去內朝兮時未幾,致邂逅兮怨無已。切自尤兮不可追,論生平兮定於此。影與形兮難去一,居忽忽兮如有失。對嬪嬙兮想芳型,顧和敬兮憐弱質。望湘浦兮何先徂,求北海兮乏神術。循喪儀兮愴徒然,例展禽兮謚孝賢。思遺徽之莫盡兮,詎兩新昌而增?字之能宣。包四德而首出兮,謂庶幾其可傳。驚時序之代謝兮,屆十旬而迅如。慟兮,陳舊物而憶初。亦有時而暫弭兮,旋觸緒而欷歔。信人生之如夢兮,了萬事之皆虛。嗚呼,悲莫悲兮生別離,失內位兮孰予隨?入淑房兮闃寂,披鳳幄兮空垂。春風秋月兮盡於此已,夏日冬夜兮知復何時?」

十七年,葬孝陵西勝水峪,後即於此起裕陵焉。嘉慶、道光累加謚,曰孝賢誠正敦穆仁惠徽恭康順輔天昌聖純皇后。子二:永璉、永琮。女二:一殤,一下嫁色布騰巴爾珠爾。

皇后,烏喇那拉氏,佐領那爾布女。後事高宗潛邸,為側室福晉。乾隆二年,封嫻妃。十年,進貴妃。孝賢皇后崩,進皇貴妃,攝六宮事。十五年,冊為皇后。三十年,從上南巡,至杭州,忤上旨,後剪發,上益不懌,令後先還京師。三十一年七月甲午,崩。上方幸木蘭,命喪儀視皇貴妃。自是遂不復立皇后。子二,永?、永璟。女一,殤。

四十三年,上東巡,有金從善者,上書,首及建儲,次為立後。上因諭曰:「那拉氏本朕青宮時皇考所賜側室福晉,孝賢皇后崩後,循序進皇貴妃。越三年,立為後。其後自獲過愆,朕優容如故。國俗忌剪發,而竟悍然不顧,朕猶包含不行廢斥。後以病薨,止令減其儀文,並未削其位號。朕處此仁至義盡,況自是不復繼立皇后。從善乃欲朕下詔罪己,朕有何罪當自責乎?從善又請立後,朕春秋六十有八,豈有復冊中宮之理?」下行在王大臣議從善罪,坐斬。

孝儀純皇后,魏佳氏,內管領清泰女。事高宗為貴人。封令嬪,累進令貴妃。乾隆二十五年十月丁丑,仁宗生。三十年,進令皇貴妃。四十年正月丁丑,薨,年四十九。謚曰令懿皇貴妃,葬勝水峪。六十年,仁宗立為皇太子,命冊贈孝儀皇后。嘉慶、道光累加謚,曰孝儀恭順康裕慈仁端恪敏哲翼天毓聖純皇后。後家魏氏,本漢軍,?入滿洲旗,改魏佳氏。子四:永璐,殤;仁宗;永璘;其一殤,未命名。女二,下嫁拉旺多爾濟、札蘭泰。

慧賢皇貴妃,高佳氏,大學士高斌女。事高宗潛邸,為側室福晉。乾隆初,封貴妃。薨,謚曰慧賢皇貴妃。葬勝水峪。

純惠皇貴妃,蘇佳氏。事高宗潛邸。即位,封純嬪。累進純皇貴妃。薨,謚曰純惠皇貴妃。葬裕陵側。子一,永瑢。女一,下嫁福隆安。

慶恭皇貴妃,陸氏。初封慶嬪。累進慶貴妃。薨。仁宗以嘗受妃撫育,追尊為慶恭皇貴妃。

哲憫皇貴妃,富察氏。事高宗潛邸。雍正十三年,薨。乾隆初,追封哲妃,進皇貴妃。謚曰哲憫皇貴妃,葬勝水峪。子一,永璜,為高宗長子。女一,殤。

淑嘉皇貴妃,金佳氏。事高宗潛邸,為貴人。乾隆初,封嘉妃,進嘉貴妃。薨,謚曰淑嘉皇貴妃,葬勝水峪。子四:永,永璇,永瑆;其一殤,未命名。

婉貴太妃,陳氏。事高宗潛邸。乾隆間,自貴人累進婉妃。嘉慶間,尊為婉貴太妃。壽康宮位居首。薨,年九十二。穎貴太妃,巴林氏。亦自貴人累進穎貴妃。尊為穎貴太妃,亦居壽康宮。薨,年七十。

貴人:西林覺羅氏,柏氏,皆自常在進尊為貴人。晉太妃,富察氏。事高宗為貴人。逮道光時,猶存。宣宗尊為皇祖晉太妃。

高宗諸妃有子女者:忻貴妃,戴佳氏,總督那蘇圖女。女二,皆殤。愉貴妃,珂裏葉特氏。子一,永琪。舒妃,葉赫那拉氏。子一,殤,未命名。惇妃,汪氏。嘗笞宮婢死,上命降為嬪。未幾,復封。女一,下嫁豐紳殷德。

又有容妃,和卓氏,回部臺吉和札賚女。初入宮,號貴人。累進為妃。薨。

仁宗孝淑睿皇后,喜塔臘氏,副都統、內務府總管和爾經額女。仁宗為皇子,乾隆三十九年,高宗冊后為嫡福晉。四十七年八月甲戌,宣宗生。仁宗受禪,冊為皇后。嘉慶二年二月戊寅,崩,謚曰孝淑皇后,葬太平峪,後即於此起昌陵焉。道光、咸豐累加謚,曰孝淑端和仁莊慈懿敦裕昭肅光天佑聖睿皇后。子一,宣宗。女二:一殤,一下嫁瑪尼巴達喇。

孝和睿皇后,鈕祜祿氏,禮部尚書恭阿拉女。後事仁宗潛邸,為側室福晉。仁宗即位,封貴妃。孝淑皇后崩,高宗敕以後繼位中宮。先封皇貴妃。嘉慶六年,冊為皇后。二十五年八月,仁宗幸熱河崩,後傳旨令宣宗嗣位。宣宗尊為皇太后,居壽康宮。道光二十九年十二月甲戌,崩,年七十四。宣宗春秋已高,方有疾,居喪哀毀,三十年正月,崩於慎德堂喪次。咸豐三年,葬後昌陵之西,曰昌西陵。初尊皇太后,上徽號。國有慶,累加上,曰恭慈康豫安成莊惠壽禧崇祺皇太后。逮崩,上謚。咸豐間加謚,曰孝和恭慈康豫安成欽順仁正應天熙聖睿皇后。子二:綿愷、綿忻。女一,殤。

恭順皇貴妃,鈕祜祿氏。嘉慶初,選入宮,為如貴人。累進如妃。宣宗尊為皇考如皇妃,居壽安宮。文宗尊為皇祖如皇貴太妃。薨,年七十四,謚曰恭順皇貴妃。子一,綿愉。女二,殤。

和裕皇貴妃,劉佳氏。事仁宗潛邸。嘉慶初,封諴妃。進諴貴妃。宣宗尊為皇考諴禧皇貴妃。薨,謚曰和裕皇貴妃。子一,未命名,殤。女一,下嫁索特納木多布齋。

仁宗諸妃有子女者:華妃,侯佳氏。事仁宗潛邸。嘉慶初,封瑩嬪。改進封。女一,殤。簡嬪,關佳氏;遜嬪,沈佳氏:皆事仁宗潛邸,號格格。嘉慶初,追封。女各一,皆殤。

仁宗嬪御至宣宗朝尊封者,又有信妃,劉佳氏;恩嬪,烏雅氏;榮嬪,梁氏:皆自貴人進。安嬪,蘇完尼瓜爾佳氏,自常在進。

宣宗孝穆成皇后,鈕祜祿氏,戶部尚書、一等子布顏達賚女。宣宗為皇子,嘉慶元年,仁宗冊后為嫡福晉。十三年正月戊午,薨。宣宗即位,追冊謚曰孝穆皇后。初葬王佐村,移寶華峪,以地宮浸水,再移龍泉峪,後即於此起慕陵焉。咸豐初,上謚。光緒間加謚,曰孝穆溫厚莊肅端誠恪惠寬欽孚天裕聖成皇后。

孝慎成皇后,佟佳氏,三等承恩公舒明阿女。宣宗為皇子,嫡福晉薨,仁宗冊後繼嫡福晉。宣宗即位,立為皇后。道光十三年四月己巳,崩,謚曰孝慎皇后,葬龍泉峪。咸豐初,上謚。光緒間加謚,曰孝慎敏肅哲順和懿誠惠敦恪熙天詒聖成皇后。女一,殤。

、一等男頤齡女。後事宣宗,冊全嬪。累進全貴妃?孝全成皇后,鈕祜祿氏,二等侍。道光十一年六月己丑,文宗生。十三年,進皇貴妃,攝六宮事。十四年,立為皇后。二十年正月壬寅,崩,年三十三。宣宗親定謚曰孝全皇后,葬龍泉峪。咸豐初,上謚。光緒間加謚,曰孝全慈敬寬仁端?安惠誠敏符天篤聖成皇后。子一,文宗。女二:一殤,一下嫁德穆楚克扎布。

孝靜成皇后,博爾濟吉特氏,刑部員外郎花良阿女。後事宣宗為靜貴人。累進靜皇貴妃。孝全皇后崩,文宗方十歲,妃撫育有恩。文宗即位,尊為皇考康慈皇貴太妃,居壽康宮。咸豐五年七月,太妃病篤,尊為康慈皇太后。越九日庚午,崩,年四十四。上謚,曰孝靜康慈弼天撫聖皇后,不系宣宗謚,不祔廟。葬慕陵東,曰慕東陵。穆宗即位,祔廟,加謚。光緒、宣統累加謚,曰孝靜康慈懿昭端惠莊仁和慎弼天撫聖成皇后。子三:奕綱、奕繼、奕。女一,下嫁景壽。

莊順皇貴妃,烏雅氏。事宣宗,為常在。進琳貴人,累進琳貴妃。文宗尊為皇考琳貴太妃。穆宗尊為皇祖琳皇貴太妃。同治五年,薨,命王公百官持服一日,謚曰莊順皇貴妃,葬慕東陵園寢。德宗朝,迭命增祭品,崇規制,上親詣行禮。封三代,皆一品。子三,奕枻、交硉、奕譓。女一,下嫁德徽。

彤貴妃,舒穆嚕氏。事宣宗,為彤貴人。累進彤貴妃。復降貴人。文宗尊為皇考彤嬪。穆宗累尊為皇祖彤貴妃。女二,一下嫁扎拉豐阿,一殤。

宣宗諸妃有子女者:和妃,納喇氏。初以宮女子,事宣宗潛邸。嘉慶十三年,子奕緯生。仁宗特命為側室福晉。道光初,封和嬪。進和妃。祥妃,鈕祜祿氏。事宣宗,為貴人。進嬪,復降。文宗尊為皇考祥妃。穆宗追尊為皇祖祥妃。子一,奕脤。女二,一殤,一下嫁恩醇。

他無子女而受尊封者:佳貴妃,郭佳氏;成貴妃,鈕祜祿氏:皆事宣宗,為貴人,進嬪,復降。歷咸豐、同治二朝進封:常妃,赫舍哩氏,以貴人進封;順嬪,失其氏,以常在進封。恆嬪,蔡佳氏;豫妃,尚佳氏;貴人李氏,那氏:以答應進封。

文宗孝德顯皇后,薩克達氏,太常寺少卿富泰女。文宗為皇子,道光二十七年,宣宗冊后為嫡福晉。二十九年十二月乙亥,薨。文宗即位,追冊謚曰孝德皇后。權攢田村,同治初,移靜安莊,旋葬定陵,上謚。光緒、宣統屢加謚,曰孝德溫惠誠順慈莊恪慎徽懿恭天贊聖顯皇后。

孝貞顯皇后,鈕祜祿氏,廣西右江道穆揚阿女。事文宗潛邸。咸豐二年,封貞嬪,進貞貴妃。立為皇后。十年,從幸熱河。十一年七月,文宗崩,穆宗即位,尊為皇太后。是時,孝欽、孝貞兩宮並尊,詔旨稱「母後皇太后」、「聖母皇太后」以別之。十一月乙酉朔,上奉兩太后御養心殿,垂簾聽政。同治八年,內監安得海出京,山東巡撫丁寶楨以聞,太后立命誅之。十二年,歸政於穆宗。十三年,穆宗崩,德宗即位,復聽政。光緒七年三月壬申,崩,年四十五,葬定陵東普祥峪,曰定東陵。初尊為皇太后,上徽號。國有慶,累加上,曰慈安端康裕慶昭和莊敬皇太后。及崩,上謚。宣統加謚,曰孝貞慈安裕慶和敬誠靖儀天祚聖顯皇后。

孝欽顯皇后,葉赫那拉氏,安徽徽寧池廣太道惠徵女。咸豐元年,後被選入宮,號懿貴人。四年,封懿嬪。六年三月庚辰,穆宗生,進懿妃。七年,進懿貴妃。十年,從幸熱河。十一年七月,文宗崩,穆宗即位,與孝貞皇后並尊為皇太后。

是時,怡親王載垣、鄭親王端華、協辦大學士尚書肅順等以文宗遺命,稱「贊襄政務王大臣」,擅政,兩太后患之。御史董元醇請兩太后權理朝政,兩太后召載垣等入議,載垣等以本朝未有皇太后垂簾,難之。侍郎勝保及大學士賈楨等疏繼至。恭親王奕留守京師,聞喪奔赴,兩太后為言載垣等擅政狀。九月,奉文宗喪還京師,即下詔罪載垣、端華、肅順,皆至死,並罷黜諸大臣預贊襄政務者。授奕議政王,以上旨命王大臣條上垂簾典禮。

十一月乙酉朔,上奉兩太后御養心殿,垂簾聽政。諭曰:「垂簾非所樂為,惟以時事多艱,王大臣等不能無所?承,是以姑允所請。俟皇帝典學有成,即行歸政。」自是,日召議政王、軍機大臣同入對。內外章奏,兩太后覽訖,王大臣擬旨,翌日進呈。閱定,兩太后以文宗賜同道堂小璽鈐識,仍以上旨頒示。旋用御史徐啟文奏,令中外臣工於時事闕失,直言無隱;用御史鍾佩賢奏,諭崇節儉,重名器;用御史卞寶第奏,諭嚴賞罰,肅吏治,慎薦舉。命內直翰林輯前史帝王政治及母後垂簾事跡,可為法戒者,以進。同治初,寇亂未弭,兵連不解,兩太后同心求治,登進老成,倚任將帥,粵、捻蕩平,滇、隴漸定。十二年二月,歸政於穆宗。

十三年十二月,穆宗崩,太后定策立德宗,兩太后復垂簾聽政。諭曰:「今皇帝紹承大統,尚在沖齡,時事艱難,不得已垂簾聽政。萬幾綜理,宵旰不遑,矧當民生多蹙,各省水旱頻仍。中外臣工、九卿、科道有言事之責者,於用人行政,凡諸政事當舉,與時事有裨而又實能見施行者,詳細敷奏。至敦節儉,袪浮華,宜始自宮中,耳目玩好,浮麗紛華,一切不得上進。」「封疆大吏,當勤求閭閻疾苦,加意撫恤;清訟獄,勤緝捕。辦賑積穀,飭有司實力奉行;並當整飭營伍,修明武備,選任賢能牧令,與民休息。」用御史陳彞奏,黜南書房行走、侍講王慶祺;用御史孫鳳翔等奏,黜總管內務府大臣貴寶、文錫;又罪宮監之不法者,戍三人,杖四人。一時宮府整肅。

光緒五年,葬穆宗惠陵。吏部主事吳可讀從上陵,自殺,留疏乞降明旨,以將來大統歸穆宗嗣子。下大臣王議奏,王大臣等請毋庸議,尚書徐桐等,侍讀學士寶廷、黃體芳,司業張之洞,御史李端棻,皆別疏陳所見。諭曰:「我朝未明定儲位,可讀所請,與家法不合。皇帝受穆宗付託,將來慎選元良,纘承統緒,其繼大統者為穆宗嗣子,守祖宗之成憲,示天下以無私,皇帝必能善體此意也。」

六年,太后不豫,上命諸督撫薦醫治疾。八年,疾愈。孝貞皇后既崩,太后獨當國。十年,法蘭西侵越南。太后責恭親王奕等因循貽誤,罷之,更用禮親王世鐸等;並諭軍機處,遇緊要事件,與醇親王奕枻商辦。庶子盛昱、錫珍,御史趙爾巽各疏言醇親王不宜參豫機務,諭曰:「自垂簾以來,揆度時勢,不能不用親籓進參機務。諭令奕枻與軍機大臣會商事件,本專指軍國重事,非概令與聞。奕枻再四懇辭,諭以俟皇帝親政,再降諭旨,始暫時奉命。此中委曲,諸臣不能盡知也。」是年,太后五十萬壽。

十一年,法蘭西約定。醇親王奕枻建議設海軍。十三年夏,命會同大學士、直隸總督李鴻章巡閱海口,遣太監李蓮英從。蓮英侍太后,頗用事。御史硃一新以各直省水災,奏請修省,辭及蓮英。太后不懌,責一新覆奏。一新覆奏,言鴻章具舟迎王,王辭之,蓮英乘以行,遂使將吏迎者誤為王舟。太后詰王,王遂對曰:「無之。」遂黜一新。

太后命以次年正月歸政,醇親王奕枻及王大臣等奏請太后訓政數年,德宗亦力懇再三,太后乃許之。王大臣等條上訓政典禮,命如議行。請上徽號,堅不許。十五年,德宗行婚禮。二月己卯,太后歸政。御史屠仁守疏請太后歸政後,仍披覽章奏,裁決施行。太后不可,諭曰:「垂簾聽政,本萬不得已之舉。深宮遠鑒前代流弊,特飭及時歸政。歸政後,惟醇親王單銜具奏,暫須徑達。醇親王密陳:『初裁大政,軍國重事,定省可以?承。』並非著為典常,使訓政永無底止。」因斥仁守乖謬,奪官。

同治間,穆宗議修圓明園,奉兩太后居之,事未行。德宗以萬壽山大報恩延壽寺,高宗奉孝聖憲皇后三次祝釐於此,命葺治,備太后臨幸,並更清漪園為頤和園,太后許之。既歸政,奉太后駐焉。歲十月十日,太后萬壽節,上率王大臣祝嘏,以為常。十六年,醇親王奕枻薨。二十年,日本侵朝鮮,以太后命,起恭親王奕。是年,太后六十萬壽,上請在頤和園受賀,仿康熙、乾隆間成例,自大內至園,蹕路所經,設彩棚經壇,舉行慶典。朝鮮軍事急,以太后命罷之。二十四年,恭親王奕薨。

上事太后謹,朝廷大政,必請命乃行。顧以國事日非,思變法救亡,太后意不謂然,積相左。上期以九月奉太后幸天津閱兵,訛言謂太后將勒兵廢上;又謂有謀圍頤和園劫太后者。八月丁亥,太后遽自頤和園還宮,復訓政。以上有疾,命居瀛臺養?。二十五年十二月,立端郡王載漪子溥?繼穆宗為皇子。

二十六年,義和拳事起,載漪等信其術,言於太后,謂為義民,縱令入京師,擊殺德意志使者克林德及日本使館書記,圍使館。德意志、奧大利亞、比利時、日斯巴尼亞、美利堅、法蘭西、英吉利、義大利、日本、和蘭、俄羅斯十國之師來侵。七月,逼京師。太后率上出自德勝門,道宣化、大同。八月,駐太原。九月,至西安。命慶親王奕劻、大學士總督李鴻章與各國議和。二十七年,各國約成。八月,上奉太后發西安。十月,駐開封。時端郡王載漪以庇義和拳得罪廢,溥?以公銜出宮。十一月,還京師。上仍居瀛臺養?。太后屢下詔:「母子一心,勵行新政。」三十二年七月,下詔預備立憲。

三十四年十月,太后有疾。上疾益增劇。壬申,太后命授醇親王載灃攝政王。癸酉,上崩於瀛臺。太后定策立宣統皇帝,即日尊為太皇太后。甲戌,太后崩,年七十四,葬定陵東普陀峪,曰定東陵。初尊為皇太后,上徽號國有慶,累加上曰慈禧端佑康頤昭豫莊誠壽恭欽獻崇熙皇太后。及崩,即以徽號為諡。子一,穆宗。

莊靜皇貴妃,他他拉氏。事文宗,為貴人,累進麗妃。穆宗尊封為皇考麗皇貴太妃。薨,諡曰莊靜皇貴妃。女一,下嫁符珍。

玫貴妃,徐佳氏。事文宗,為貴人,進玫嬪。穆宗尊為皇考玫貴妃。子一,未命名,殤。

端恪皇貴妃,佟佳氏。事文宗,為祺嬪。同治間尊為皇考祺貴妃。宣統初,尊為皇祖祺貴太妃。薨,諡曰端恪皇貴妃。

文宗諸妃,未有子女而同治、光緒兩朝尊封者,婉貴妃、璷妃、吉妃、禧妃、慶妃、雲嬪、英嬪、容嬪、璹嬪、玉嬪,皆自貴人進封。婉貴妃,索綽絡氏。雲嬪,武佳氏。英嬪,伊爾根覺羅氏。餘不知氏族。

穆宗孝哲毅皇后,阿魯特氏。戶部尚書崇綺女。同治十一年九月立為皇后。十三年十二月,穆宗崩,德宗即位。以兩太后命封為嘉順皇后。光緒元年二月戊子崩,梓宮暫安隆福寺。二年五月,御史潘敦儼因歲旱上言,請更定諡號,謂:「後崩在穆宗升遐百日內,道路傳聞,或稱傷悲致疾,或雲絕粒霣生,奇節不彰,何以慰在天之靈?何以副兆民之望?」太后以其言無據,斥為謬妄,奪官。五年三月,合葬惠陵,上諡。宣統加謚,曰孝哲嘉順淑慎賢明恭端憲天彰聖毅皇后。

淑慎皇貴妃,富察氏。穆宗立後,同日封慧妃。進皇貴妃。德宗即位,以兩太后命,封為敦宜皇貴妃。進敦宜榮慶皇貴妃。光緒三十年,薨。謚曰淑慎皇貴妃。

莊和皇貴妃,阿魯特氏,大學士賽尚阿女,孝哲毅皇後姑也。事穆宗,為珣嬪,進妃。光緒間,進貴妃。宣統皇帝尊為皇考珣皇貴妃。孝定景皇后崩未逾月,妃薨。謚曰莊和皇貴妃。敬懿皇貴妃,赫舍里氏。事穆宗,自嬪進妃。光緒間,進貴妃。宣統間,累進尊封。榮惠皇貴妃,西林覺羅氏。事穆宗,自貴人進嬪。光緒間,進妃。宣統間,累進尊封。

德宗孝定景皇后,葉赫那拉氏,都統桂祥女,孝欽顯皇后侄女也。光緒十四年十月,孝欽顯皇后為德宗聘焉。十五年正月,立為皇后。二十七年,從幸西安。二十八年,還京師。三十四年,宣統皇帝即位。稱「兼祧母後」,尊為皇太后。上徽號曰隆裕。宣統三年十二月戊午,以太后命遜位。越二年正月甲戌,崩,年四十六。上謚曰孝定隆裕寬惠慎哲協天保聖景皇后,合葬崇陵。

端康皇貴妃,他他拉氏。光緒十四年,選為瑾嬪。二十年,進瑾妃。以女弟珍妃忤太后,同降貴人。二十一年,仍封瑾妃。宣統初,尊為兼祧皇考瑾貴妃。遜位後,進尊封。歲甲子,薨。

恪順皇貴妃,他他拉氏,端康皇貴妃女弟。同選,為珍嬪。進珍妃。以忤太后,諭責其習尚奢華,屢有乞請,降貴人。逾年,仍封珍妃。二十六年,太后出巡,沈於井。二十七年,上還京師。追進皇貴妃。葬西直門外,移祔崇陵。追進尊封。

宣統皇后,郭博勒氏,總管內務府大臣榮源女。遜位後,歲壬戌,冊立為皇后。淑妃,額爾德特氏。同日冊封。

創業之難,而樹委裘之主,政出王大臣?論曰:世祖、聖祖皆以沖齡踐祚,孝莊皇后,當時無建垂簾之議者。殷憂啟聖,遂定中原,克底於升平。及文宗末造,孝貞、孝欽兩皇后躬收政柄,內有賢王,外有名將相,削平大難,宏贊中興。不幸穆宗即世,孝貞皇后崩,孝欽皇后聽政久,稍稍營離宮,修慶典,視聖祖奉孝莊皇后、高宗奉孝聖皇后不逮十之一,而世顧竊竊然有私議者,外侮迭乘,災祲屢見,非其時也。不幸與德宗意恉不協,一激而啟戊戌之爭,再激而成庚子之亂。晚乃壹意變法,怵天命之難諶,察人心之將渙,而欲救之以立憲,百端並舉,政急民煩,陵土未乾,國步遂改。綜一代之興亡,系於宮闈。嗚呼!豈非天哉?豈非天哉?


\end{pinyinscope}