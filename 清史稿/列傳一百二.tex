\article{列傳一百二}

\begin{pinyinscope}
高天喜鄂實三格和起唐喀祿阿敏道滿福

豆斌端濟布諾爾本

高天喜,甘肅西寧人。天喜本準噶爾人,雍正中為我師所俘。高氏撫為子,因從其族籍。從軍,累擢保寧堡守備。乾隆二十二年,副將軍兆惠擊伊犁,天喜從參將邁斯漢赴援。遇噶勒雜特賊百餘,擊殺之,獲其駝馬。既,聞兆惠被困濟爾哈朗,議馳救,邁斯漢怯不進。巴里坤辦事大臣雅爾哈善以聞,上即奪邁斯漢官以命天喜。尋遷金塔協副將。再遷西寧鎮總兵,授領隊大臣。二十三年十月,師攻葉爾羌,兆惠議出間道襲取賊輜重,渡黑水。天喜督兵修橋渡師,未及半,賊大至。天喜聞兆惠陷賊陣,舍橋亟赴之,奮與賊戰,與鄂實、三格、特通額俱沒於陣。上賦詩惜之。謚果義,又賜其家白金千。

鄂實,西林覺羅氏,滿洲鑲藍旗人,大學士鄂爾泰第二子。出為叔父鄂禮後。自廕生授三等侍衛。累遷本旗副都統、左翼前鋒統領。兄鄂容安死阿睦爾撒納之亂,鄂實請從軍,授參贊大臣,佐定邊將軍成袞扎布,出西路。二十二年夏,成袞扎布令逐捕扎那噶爾布,鄂實以地險馬疲,中道引還。上手詔詰責曰:「若謂地險,賊何以能行?若謂馬疲,賊馬何獨能壯健?」左授藍翎侍衛。是冬,鄂實逐扎哈沁賊,斬一百四十餘級,獲牲械。上謂:「今當大雪,馬力應疲乏,尚能剿賊。彼時鄂實為參贊大臣,有事但諉諸將軍。茲以負罪,乃直前剿賊,朕知其隱矣。」量遷三等侍衛。死事,上令仍視前鋒統領賜恤,謚果壯。

三格,棟鄂氏,滿洲正白旗人。自諸生授藍翎侍衛。累遷黑龍江副都統。命將索倫、巴爾呼兵三千,佐參贊大臣策楞出西路,為領隊大臣。策楞以怯懦逮,三格亦坐奪官。旋復授正白旗蒙古副都統。攻呼爾璊臺吉賽音伯勒克等,再戰,掠其牧地,予三等輕車都尉世職。二十二年春,定邊將軍成袞扎布令逐捕扎那噶爾布,未得。秋,師至博羅和羅,遇叛黨額林沁達瓦等百餘戶,三格與戰。會布魯古特臺吉琿齊、呼爾璊臺吉達瓦斬扎那噶爾布偽請降,並請招額林沁達瓦,三格信之,遽引師還,琿齊等旋遁去。坐奪官,並削世職,以兵伍自效。死事,上命仍視副都統賜恤,謚剛勇。

天喜、鄂實、三格並祀昭忠祠,予騎都尉兼雲騎尉世職。回部平,圖形紫光閣。特通額,策楞子也,附見策楞傳。

和起,馬佳氏,滿洲鑲藍旗人。其先世阿音布,國初以軍功授拜他喇布勒哈番世職。和起襲職,授盛京協領。累擢寧夏副都統。乾隆十九年,命與侍衛海福將千人佐定西將軍永常討達瓦齊,遷寧夏將軍。永常劾和起兵不及額,而和起先疏言將九百人以往,留百人護輜重,上得永常疏,不之罪也。尋又命偕提督豆斌為巴里坤辦事大臣,策楞代永常為定西將軍,復劾和起送兵馬遲誤,當奪官,留任。旋復官,授欽差大臣關防,召詣京師諮軍事。達什達瓦所屬宰桑訥默庫、曼集、烏達瑚們都等在軍私還游牧,命和起嚴鞫得實,以降人請予寬典,上不許,命正軍法。

二十一年十一月,輝特臺吉巴雅爾叛,掠扎哈沁五百餘戶。定邊右副將軍兆惠令和起將索倫兵百人往按,檄吐魯番伯克莽阿里克等集闢展,而噶勒雜特宰桑哈薩克錫喇、布魯特臺吉尼瑪陰應巴雅爾,詭以兵五百會。和起望兵至,疑之。令莽阿里克詗之,紿告曰:「我兵也!」逾時,尼瑪等操戈前,莽阿里克自後譟,賊眾蜂集。和起所將兵僅百人,負重創,手刃數賊,股中槍,徒步轉戰,至夜力盡。和起垂死,命索倫侍衛努古德、彰金布突圍出,以所戴孔雀翎為識報兆惠,遂死之。謚武烈,追封一等伯,以一等子世襲,祀賢良、昭忠二祠。二十三年,師還,獲尼瑪及其子檻送京師,命戮於和起墓前。子和隆武,自有傳。

唐喀祿,他塔喇氏,蒙古正藍旗人。自筆帖式再遷理籓院員外郎。乾隆十九年,賜副都統銜,命赴北路軍董理新降輝特臺吉阿睦爾撒納、班珠爾等游牧地。唐喀祿疏言:「班珠爾所屬多老稚不能耕,慮饑餒。」上以距耕時尚遠,責其瑣屑,命撤還。扎薩克林丕勒多爾濟初命同董理游牧,將軍別有指揮,唐喀祿疏請留。上責其不當,左遷理籓院筆帖式。尋復授員外郎,命送濟隆呼圖克圖自巴林赴伊犁,董理定邊右副將軍薩喇爾游牧。復賜副都統銜,授領隊大臣,將駐防扎布堪兵千人,從定邊右副將軍哈達哈赴哈薩克,逐捕阿睦爾撒納。賊渠固爾班和卓遁入烏梁海,唐喀祿報哈達哈督兵擒之,賜孔雀翎。阿睦爾撒納令其徒達瓦藏布入掠,唐喀祿令索倫總管鄂博什將五百人御之,降其眾三百。尋命屯科布多。授理籓院侍郎、鑲藍旗蒙古副都統。

唐喀祿行按諸部,輝特降人屯扎克賽,每自相劫奪,請移屯呼倫貝爾、齊齊哈爾諸地;喀爾喀俘獲扎哈沁、特楞古特、奇爾吉斯、烏爾罕濟蘭諸部人萬餘,請以扎哈沁人移駐卡倫內;特楞古特、奇爾吉斯、烏爾罕濟蘭人給東三省兵丁為奴;杜爾伯特游牧請移烏蘭固木:上並從其請。師出西路擊哈薩克錫喇,命唐喀祿屯額爾齊斯為聲援。阿睦爾撒納敗走,唐喀祿詗知杜爾伯特貝勒巴圖博羅特、臺吉阿喇善等潛與相結;遣兵攻之輝巴朗山,擒阿喇善等,並戮烏梁海五十餘戶,遂赴塔爾巴哈臺逐捕阿睦爾撒納及哈薩克錫喇,賜御用荷包、鼻煙壺。師至塔爾巴哈臺,糧罄馬乏,唐喀祿引師退,疏言遵旨撤兵,上怒,左授藍翎侍衛,佐定邊左副將軍納穆扎爾出北路。降人和碩齊,上擢用至散秩大臣,至是令護哈薩克來使入邊,上命納穆扎爾遣唐喀祿將二百人迎之。阿睦爾撒納竄俄羅斯,上命唐喀祿偕和碩齊駐額爾齊斯偵御。

二十三年三月,土爾扈特舍棱等謀走俄羅斯,上命偕和碩齊逐捕。四月,師次布固圖河,獲舍棱弟勞章扎卜。勞章扎卜詭為兄乞降,唐喀祿未敢信,和碩齊遽縱之還。越日,舍棱詭約降,獻酒,和碩齊飲之,邀唐喀祿過其營,賊噪而起。唐喀祿及侍衛富錫爾、穆倫保、佛爾慶額力戰,均遇害,和碩齊更衣降。事聞,賜騎都尉世職,祀昭忠祠。富錫爾、佛爾慶額,皆滿洲鑲黃旗人;穆倫保,滿洲正白旗人:皆賜雲騎尉世職。

阿敏道,圖爾格期氏,蒙古鑲紅旗人,世居察哈爾。父阿吉斯,康熙間討噶爾丹,以員外郎從軍,中道糧匱,兵苦饑。阿吉斯言於眾曰:「我等官兵世受國恩,甘斃道路。誓竭力前進。」眾皆諾。於是有昭莫多之勝。聖祖嘉其能,予拖沙喇哈番世職。卒。

阿敏道,襲職。雍正初,累遷二等侍衛。九年,命將巴里呼兵百人自固爾班塞堪赴巴爾坤佐軍,又命偕侍讀學士查克丹調喀爾喀兵三千率之往。尋復偕護軍統領費雅思哈赴烏爾輝音扎罕練兵。乾隆元年,準噶爾乞和,撤軍,阿敏道還京,授鑲藍旗察哈爾總管。十九年,師收烏梁海,將察哈爾兵以從,加副都統銜。二十年,遷所獲巴爾沁人等於齊拉罕。師定伊犁,定北將軍班第奏以阿敏道督臺站。是年,阿睦爾撒納叛,班第陷賊。阿巴噶斯、哈丹附逆肆掠,臺站中斷。阿敏道輒督兵巡徼,使驛遞恆得相續。會定西將軍永常自木壘退駐烏爾圖布拉克,撤阿敏道還。上奪永常官,以策楞代將。命阿敏道將精騎詣伊犁求班第消息。策楞不即遣,上詰責之。尋將千人捕阿巴噶斯、哈丹賊眾。

二十一年,授鑲藍旗蒙古副都統。時回酋布拉呢敦、霍集占有異志,定邊右副將軍兆惠詗知之,遣阿敏道將索倫兵百、厄魯特兵三千赴葉爾羌、喀什噶爾慰撫,且使致二渠。至庫車,霍集占布在焉,閉城拒我師。阿敏道斬游騎四十餘,圍之。城人詭言曰:「厄魯特吾仇,慮為害。撤還即納降。」阿敏道遂命厄魯特兵退,僅留索倫兵百。或慮有變,阿敏道曰:「吾招撫回眾,惟期於國有濟,何暇他慮?」遂入,為霍集占所執。

二十二年,上諭諸將檄霍集占送阿敏道還,不從,謀加害。庫車伯克呼岱巴爾以告,阿敏道謀脫歸,不克,死之。二等男署察哈爾營總旺扎勒及諸裨將繃科、耨金吹、扎木蘇

七、巴克薩拾,並索倫兵百人,皆從死。事平,諸有功者圖形紫光閣,阿敏道列後五十功臣,加世職為騎都尉兼一雲騎尉,祀昭忠祠。旺札勒加雲騎尉,繃科等皆予雲騎尉世職。

滿福,瓜爾佳氏,滿洲鑲藍旗人。自世管佐領累擢拉林副都統。乾隆二十二年,遷都統,駐巴里坤。命將吉林兵千人屯吐魯番,尋授領隊大臣。定邊將軍成袞扎布出珠勒都斯,令滿福將三百人巡視阿勒輝至烏納哈特十三臺站,搜剿嗎哈沁。沙拉斯、嗎唬斯既降復叛,掠臺站,上命滿福自阿勒輝往剿,又令巴里坤辦事大臣阿里袞帥師與會。阿里袞未至,滿福師次肯色嶺,與賊遇,擊之,賊敗走,偽遣人乞降,且言賊渠已就縛,請除道迎。滿福信之,行次哈喇和落,徑險林密,下臨深溝。滿福悟為賊所紿,急麾前隊返。賊千餘突自林中出,圍我師。滿福厲聲督兵力戰,被創墜溝,死之。上以滿福雖為賊所愚,愍其捐軀,命如陣亡例議恤,謚武毅,祀昭忠祠,圖形紫光閣。

豆斌,陜西固原人。初以馬兵入提標,累遷肅州鎮標中營守備。雍正間,從征準噶爾。力戰受創,賜白金四百。遷川陜督標前營游擊。準噶爾犯科舍圖,率兵擊走之。乾隆初,累遷提督,自廣東移廣西。疏言:「各營鳥槍,舊式大小參差,坐臥倚伏,不能應手;又質薄易熱,難收實用。請照陜西威字號纏絲槍式改制。」下兩廣總督議行。俄,調還固原。又命以提督銜領湖北宜昌鎮總兵事。尋復歷甘肅、安西提督。命討準噶爾,帥將標兵出駐巴里坤,以輸軍馬後時,下吏議。旋乞病,罷。

居數月,復授安西提督,仍令赴巴里坤兆惠師。師攻霍集占於庫車,命斌將所部從,充領隊大臣,徼巡魯克察克、闢展、庫車諸地驛路。兆惠被圍黑水,斌從副將軍富德自阿克蘇兼程赴援。師次呼爾璊,霍集占以五千人迎戰,我師分兩翼,賊據高岡,斌率中軍火器進攻。賊知我師馬力乏,擁眾相偪。阿里袞解馬至,斌偕眾將夾擊,脅中創,仍力戰,賊大敗。創甚遂卒,謚壯節,祀昭忠祠,予騎都尉兼雲騎尉。上制詩惜之。回部平,圖形紫光閣。孫澍,襲世職,官至山東登州鎮總兵。

端濟布,瓜爾佳氏,滿洲鑲黃旗人。自前鋒累遷頭等侍衛、鑲黃旗察哈爾總管。乾隆二十二年,上令選兵千佐定邊將軍兆惠出西路。自硃爾圖斯赴瑪納斯,獲得木齊鄂羅斯,並所部三百人、馬駝牛羊二千餘。扎哈沁頭人巴哈曼集叛走,端濟布偕侍衛奎瑪岱追捕,至小衛和勒津,降所部二百戶,又得掠臺站賊札木布。師捕治厄魯特頭人噶爾藏多爾濟、扎那噶爾布等,布魯古特臺吉琿齊、呼爾璊臺吉達瓦斬扎那噶爾布,詣端濟布軍請降。端濟布遽引師還,琿齊、達瓦復叛去。上懲端濟布惟事姑息,命靖逆將軍雅爾哈善按治。師至羅克倫孟古圖嶺,獲噶爾藏多爾濟宰桑羅卜札尼瑪、得木齊敦多克,檻送巴里坤。上聞,命貸端濟布罪。

扎哈沁得木齊哈勒拜等謀掠臺站,參贊大臣哈寧阿檄端濟布往捕,至瑪納斯,得間諜十餘。渡河至美羅托山,賊遁,收其游牧牲畜。師圍庫車,端濟布將吉林、厄魯特兵以從。霍集占將三千人自賽裏木來援,屯高阜。端濟布偕侍衛順德納等奮擊,斬二千餘級。師攻葉爾羌,霍集占築臺城東北。端濟布及侍衛諾爾本將右翼後隊攻之,賊拒戰,復斬二千餘級。兆惠被圍於黑水,端濟布從定邊左副將軍富德赴援,十餘戰,至呼爾璊,與兆惠軍會,賜三等輕車都尉世職,授鑲紅旗滿洲副都統。

師逐賊,戰於阿爾楚爾,再戰於伊西洱庫爾淖爾,端濟布將二百人截賊逃路。偵山有賊寨,越嶺攻之,被創,賜號塔什巴圖魯。師還,圖形紫光閣,列前五十功臣。卒,贈都統,謚壯節,祀昭忠祠。諭以「端濟布力戰受傷,與陣亡者無異也」。

諾爾本,吳機格忒氏,滿洲正藍旗人。以前鋒從軍。富德獲宰桑烏巴什,遣諾爾本送兆惠軍。道遇賊,力戰,賜號克籌巴圖魯。師圍庫車,霍集占來援。諾爾本偕公袞楚克,侍衛齊凌札卜、齊努渾等擊賊右翼,賊敗走,逐之六十餘里,至鄂根河口,斬獲甚眾;賊逃入蘇巴什山,復偕齊努渾入山搜戮:溫詔嘉焉。師攻葉爾羌,偕端濟布戰城東,敗賊。師還,命在乾清門行走,賚銀帛,賜騎都尉加一雲騎尉世職,圖形紫光閣。擢頭等侍衛,從明瑞徵緬甸,擊賊被創。尋令將兵屯騰越。還京,擢圍場總管,加副都統銜。卒。

論曰:高天喜驍勇善戰,與鄂實、三格奮鬥破陣,死事為最烈。和起等倉卒為賊陷,慷慨授命。斌與端濟布以力戰受創,得與戰死者同其血食。旌勇勵忠,當如是也。


\end{pinyinscope}