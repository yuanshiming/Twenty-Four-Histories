\article{志一百二十一}

\begin{pinyinscope}
○藝文二

史部十六類:一曰正史類,二曰編年類,三曰紀事本末類,四曰別史類,五曰雜史類,六曰詔令奏議類,七曰傳記類,八曰史鈔類,九曰載記類,十曰時令類,十一曰地理類,十二曰職官類,十三曰政書類,十四曰目錄類,十五曰金石類,十六曰史評類。

正史類

明史三百三十六卷。康熙十八年敕撰,乾隆四年書成表進。遼金元三史國語解四十六卷。乾隆四十六年敕撰。史記補注一卷。方苞撰。史記疑問一卷。邵泰衢撰。史記考證七卷。杭世駿撰。史記志疑三十六卷。梁玉繩撰。讀史記十表十卷。汪越撰,徐克範補。史記天官書補目一卷,考證十卷。孫星衍撰。史記律歷天官書正譌三卷。王元啟撰。史記三書釋疑三卷。錢塘撰。史記功比說一卷。張錫瑜撰。史記毛本正誤一卷。丁晏撰。校刊史記札記五卷。張文虎撰。史漢箋論十卷。楊於果撰。史漢駢枝一卷。成蓉鏡撰。漢書辨疑二十二卷。錢大昭撰。漢書拾遺一卷。劉臺拱撰。漢書疏證三十六卷。沈欽韓撰。漢書注校補五十六卷。周壽昌撰。漢書管見四卷。硃一新撰。漢書補注一百卷。王先謙撰。漢初年月日表一卷。姚文田撰。漢書律歷志正譌二卷。王元啟撰。漢書地理志稽疑六卷。全祖望撰。漢書地理志補注一百三卷。吳卓信撰。新注漢書地理志十六卷。錢坫撰。漢書地理志校注二卷。王紹蘭撰。漢書地理志校本二卷。汪遠孫撰。漢志水道疏證四卷。洪頤煊撰。漢書地理志水道圖說七卷。陳澧撰。漢志釋地略漢志志疑一卷。汪士鐸撰。漢書地理志集釋十四卷,西域傳補注二卷。徐松撰。漢西域圖考七卷。李光廷撰。漢書古今人表考九卷。梁玉繩撰。人表考校補一卷,續補一卷。蔡云撰。漢書正誤四卷。王峻撰。漢書刊誤一卷。石韞玉撰。漢書注考證一卷。何若瑤撰。兩漢朔閏表二卷,附漢太初以前朔閏表一卷。張其(曾羽)撰。兩漢舉正五卷。陳景雲撰。後漢書補注二十四卷。惠棟撰。後漢書辨疑十一卷,續後漢書辨疑九卷,後漢書補表八卷,補續漢書藝文志一卷,後漢郡國令長考一卷。錢大昭撰。後漢書疏證三十卷。沈欽韓撰。後漢書補注續一卷,補後漢書藝文志四卷。侯康撰。後漢書注補正八卷。周壽昌撰。後漢書注又補一卷。沈銘彞撰。後漢書儒林傳補二卷。李聿修撰。後漢書補逸二十一卷。姚之駰撰。後漢書注刊誤一卷,後漢公卿表一卷。練恕撰。後漢三公年表一卷。華湛恩撰。後漢書注考證一卷。何若瑤撰。三國志舉正四卷。陳景雲撰。三國志考證八卷。潘眉撰。三國志補注六卷。杭世駿撰。三國志續考證一卷。盧文弨撰。三國志辨疑三卷。錢大昭撰。三國志注補六十五卷。趙一清撰。三國志補注十六卷。沈欽韓撰。三國志旁證三十卷。梁章鉅撰。三國志證聞二卷。錢儀吉撰。三國紀年表一卷。周嘉猷撰。補三國疆域志三卷。洪亮吉撰。三國職官表三卷。洪飴孫撰。三國志注續一卷,補三國藝文志四卷。侯康撰。三國志注證遺四卷。周壽昌撰。晉書地理志新補正五卷。畢沅撰。東晉疆域志四卷。洪亮吉撰。晉書補傳贊一卷。杭世駿撰。補晉書兵志一卷。錢儀吉撰。晉書校勘記四卷。周云撰。晉書校勘記三卷。勞格撰。補晉書藝文志四卷,晉書校文五卷。丁國鈞撰。晉宋書故一卷,補宋書刑法志一卷,食貨志一卷。郝懿行撰。宋書州郡志校勘記一卷。成蓉鏡撰。補梁疆域志四卷。洪飴孫撰。魏書校勘記一卷。王先謙撰。北周公卿表一卷。練恕撰。南北史識疑四卷。王懋竑撰。補南北史表七卷。周嘉猷撰。補南北史志十四卷。汪士鐸撰。隋書經籍志考證十三卷。章宗源撰。隋書地理志考證九卷。楊守敬撰。新舊唐書互證二十卷。趙紹祖撰。舊唐書疑義四卷。張道撰。舊唐書校勘記六十六卷。羅士琳、陳立、劉文淇、劉毓崧同撰。唐學士年表一卷。錢大昕撰。五代史志疑四卷。楊陸榮撰。五代史纂誤補四卷。吳蘭庭撰。五代史纂誤續補六卷。吳光耀撰。五代史纂誤補續一卷。周壽昌撰。舊五代史考異二卷。邵晉涵撰。新五代史注七十四卷。彭元瑞、劉鳳誥同撰。五代紀年表一卷。周嘉猷撰。五代史地理考一卷。練恕撰。補五代史藝文志一卷。顧櫰三撰。五代學士年表一卷。錢大昕撰。宋史地理志校勘記一卷。成蓉鏡撰。宋史藝文志補一卷。倪燦撰。宋中興學士年表一卷,宋修唐書史臣表一卷。錢大昕撰。遼史拾遺二十四卷,補五卷。厲鶚撰。遼史拾遺續三卷。楊復吉撰。金史詳校十卷,金源劄記二卷。施國祁撰。元史本證五十卷,元史證誤二十三卷。汪輝祖撰。元史氏族表三卷,補元史藝文志四卷。錢大昕撰。元史譯文證補三十卷。洪鈞撰。宋遼金元四史朔閏考二卷,遼金元三史拾遺五卷。錢大昕撰。補遼金元三史藝文志一卷。倪燦撰。補遼金元三史藝文志一卷。金門詔撰。明史考證攟逸四十二卷。王頌蔚撰。二十二史考異一百卷,諸史拾遺五卷。錢大昕撰。十七史商榷一百卷。王鳴盛撰。二十二史劄記三十六卷,補遺一卷。趙翼撰。四史發伏十二卷。洪亮吉撰。讀史舉正八卷。張熷撰。諸史然疑一卷。杭世駿撰。諸史考異十八卷。洪頤煊撰。歷代史目表一卷。洪飴孫撰。

宋薛居正等舊五代史一百五十卷、目錄二卷,宋吳縝五代史記纂誤三卷。以上乾隆時奉敕輯。漢書音義三卷、補遺一卷。臧鏞堂輯。

編年類

太祖實錄十三卷。崇德元年敕纂,康熙二十一年聖祖重修,雍正十二年敕加校訂。太宗實錄六十八卷。順治九年敕纂,康熙十二年聖祖重修,雍正十二年敕加校訂。世祖實錄一百四十七卷。康熙六年敕纂,雍正十二年敕加校訂。聖祖實錄三百三卷。康熙六十一年敕纂。世宗實錄一百五十九卷。雍正十三年敕纂。高宗實錄一千五百卷。嘉慶四年敕纂。仁宗實錄三百七十四卷。道光四年敕纂。宣宗實錄四百七十六卷。咸豐二年敕纂。文宗實錄三百五十六卷。同治元年敕纂。穆宗實錄三百七十四卷。光緒五年敕纂。德宗實錄五百六十一卷。宣統時敕纂。御批通鑒輯覽一百十六卷,附明唐桂二王本末三卷。乾隆三十二年傅恆等奉敕撰。御定通鑒綱目三編四十卷。乾隆四十年敕撰。開國方略三十二卷。乾隆三十八年敕撰。竹書統箋十二卷。徐文靖撰。竹書紀年集證五十卷。陳逢衡撰。考定竹書十三卷。孫之騄撰。竹書紀年校正十四卷。郝懿行撰。校正竹書紀年二卷。洪頤煊撰。竹書紀年集注二卷。陳詩撰。竹書紀年校補二卷。張宗泰撰。考訂竹書紀年十四卷,竹書紀年義證四十卷。雷學淇撰。竹書紀年補證四卷。林春溥撰。資治通鑒後編一百八十四卷。徐乾學撰。續資治通鑒後編校勘記十五卷。夏震武撰。續資治通鑒三百二十卷。畢沅撰。續資治通鑒長編拾補六十卷。秦緗業撰。續資治通鑒長編拾遺六十卷。黃以周撰。通鑒胡注舉正一卷。陳景雲撰。通鑒注辨正二卷。錢大昕撰。通鑒注商十八卷。趙紹祖撰。通鑒刊本識誤三卷,通鑒補略一卷。張敦仁撰。通鑒校勘記七卷。張瑛撰。通鑒地理今釋十六卷。吳熙載撰。綱目訂誤四卷。陳景雲撰。綱目分注補遺四卷。芮長恤撰。通鑒綱目釋地糾繆六卷,釋地補注六卷。張庚撰。綱目志疑一卷。華湛恩撰。讀通鑒綱目條記二十卷。李述來撰。明鑒前紀二卷。齊召南撰。明通鑒一百卷。夏燮撰。明紀六十卷。陳鶴撰。周季編略九卷。黃式三撰。古史紀年十四卷,古史考年異同表二卷,戰國紀年六卷,附年表一卷。林春溥撰。國策編年一卷。顧觀光撰。小腆紀年附考二十卷。徐鼒撰。東華錄三十二卷。蔣良驥撰。十朝東華錄四百二十五卷。王先謙撰。咸豐朝東華續錄六十九卷。潘頤福撰。光緒東華錄二百二十卷。硃壽朋撰。滇云歷年傳十二卷。倪蛻撰。

宋李燾續資治通鑒長編五百二十卷,宋不著撰人兩朝綱目備要十六卷,宋王益之西漢紀年三十卷,宋熊克中興小紀四十卷。以上乾隆時敕輯。陸機晉紀一卷,干寶晉紀一卷,習鑿齒漢晉春秋一卷,鄧粲晉紀一卷,孫盛晉陽秋一卷,劉謙之晉紀一卷,徐廣晉紀一卷,檀道鸞續晉陽秋一卷,劉道薈晉起居注一卷。以上黃奭輯。晉紀五卷,晉陽秋五卷,漢晉春秋四卷,三十國春秋十八卷。以上湯球輯。

紀事本末類

平定三逆方略六十卷。康熙二十一年,勒德洪等奉敕撰。親征平定朔漠方略四十八卷。康熙四十七年,溫達等奉敕撰。平定金川方略三十二卷。乾隆十三年,來保等奉敕撰。平定準噶爾方略前編五十四卷,正編八十五卷,續編三十三卷。乾隆三十七年,傅恆等奉敕撰。臨清紀略十六卷。乾隆四十二年,於敏中等奉敕撰。平定兩金川方略一百五十二卷。乾隆四十六年,阿桂等奉敕撰。蘭州紀略二十卷。乾隆四十六年敕撰。石峰堡紀略二十卷。乾隆四十九年敕撰。臺灣紀略七十卷。乾隆五十三年敕撰。安南紀略三十二卷。乾隆五十六年敕撰。廓爾喀紀略五十四卷。乾隆六十年敕撰。巴布勒紀略二十六卷。乾隆時敕撰。平苗匪紀略五十二卷。嘉慶二年,鄂輝等奉敕撰。剿平三省邪匪方略前編三百六十一卷,續編三十六卷,附編十二卷。嘉慶十五年,慶桂等奉敕撰。平定教匪紀略四十二卷。嘉慶二十一年,托津等奉敕撰。平定回疆剿捦逆裔方略八十卷。道光九年,曹振鏞等奉敕撰。剿平粵匪方略四百二十卷。同治十一年敕撰。剿平捻匪方略三百二十卷。同治十一年敕撰。平定陜甘新疆回匪方略三百二十卷。光緒二十二年敕撰。平定雲南回匪方略五十卷。光緒二十二年敕撰。平定貴州苗匪紀略四十卷。光緒二十二年敕撰。繹史一百六十卷。馬驌撰。左傳紀事本末五十三卷。高士奇撰。通鑒本末紀要八十一卷。蔡毓榮撰。遼史紀事本末四十卷,金史紀事本末五十二卷。李有棠撰。明史紀事本末八十卷。谷應泰撰。續明紀事本末十八卷。倪在田撰。明朝紀事本末補編五卷。彭孫貽撰。三籓紀事本末四卷。楊陸榮撰。四籓始末四卷。錢名世撰。綏寇紀略十二卷。吳偉業撰。滇考二卷。馮甦撰。皇朝武功紀盛四卷。趙翼撰。聖武記十四卷。魏源撰。平定羅剎方略四卷。不著撰人氏名。平臺紀略一卷,附東征集六卷。藍鼎元撰。平定粵匪紀略十卷,附記四卷。杜文瀾撰。湘軍志十六卷。王闓運撰。湘軍記二十卷。王定安撰。平浙紀略十六卷。秦緗業、陳鍾英同撰。吳中平寇記八卷。錢勖撰。淮軍平捻記十二卷。周世澄撰。豫軍紀略十二卷。尹耕云撰。山東軍興紀略二十二卷。不著撰人氏名。霆軍紀略十六卷。陳昌撰。平定關隴紀略十三卷。易孔昭、胡孚駿同撰。粵東剿匪紀略五卷。陳坤撰。平回志八卷。楊毓秀撰。剿定新疆記八卷。魏光燾撰。浙東籌防錄四卷。薛福成撰。國朝柔遠記十八卷。王之春撰。中西紀事二十四卷。夏燮撰。普法戰紀二十卷。王韜撰。中東戰紀本末八卷。蔡爾康撰。

別史類

歷代紀事年表一百卷。康熙五十一年,王之樞等奉敕撰。續通志五百二十七卷。乾隆三十二年敕撰。逸周書補注二十二卷,補遺一卷。陳逢衡撰。汲塚周書輯要一卷。郝懿行撰。逸周書集訓校釋十卷,逸文一卷。硃右曾撰。逸周書集訓校釋增校一卷。硃駿聲撰。逸周書管箋十六卷。丁宗洛撰。逸周書王會篇箋釋三卷。何秋濤撰。校輯世本二卷。雷學淇撰。世本輯補十卷。秦嘉謨撰。帝王世紀考異一卷。宋翔鳳撰。帝王世紀地名衍四卷。迮鶴壽撰。春秋戰國異詞五十六卷,通表二卷,摭遺一卷。陳厚耀撰。春秋紀傳五十一卷。李鳳雛撰。尚史一百七卷。李鍇撰。後漢書補逸二十一卷。姚之駰撰。季漢書九十卷。章陶撰。季漢書九十卷。湯成烈撰。季漢五志十二卷。王復禮撰。後漢書十四卷。王廷璨撰。晉記六十八卷。郭倫撰。晉略六十卷。周濟撰。西魏書二十四卷。謝啟昆撰。續唐書七十卷。陳鱣撰。宋史翼四十卷。陸心源撰。元史新編九十五卷。魏源撰。元祕史注十五卷。李文田撰。元史備志五卷。王光魯撰。續宏簡錄四十二卷。邵遠平撰。明書一百七十一卷。傅維鱗撰。明史稿三百十卷。王鴻緒撰。明史稿二十卷,續二卷。湯斌撰。擬明史列傳二十四卷。汪琬撰。擬明史傳不分卷。姜宸英撰。明史分稿殘編二卷。方象英撰。明史擬傳六卷,藝文志五卷,外國志五卷。尤侗撰。國史考異六卷。潘檉章撰。開闢傳疑二卷。林春溥撰。歷代甲子考一卷。黃宗羲撰。二十一史年表十卷。顧炎武撰。歷代史表五十九卷。萬斯同撰。二十一史四譜五十四卷,歷代世系紀年編一卷。沈炳震撰。歷代帝王年表三卷。齊召南撰。歷代帝王廟謚年諱譜一卷。陸費墀撰。紀元要略二卷。陳景雲撰。歷代建元考十卷。鍾淵映撰。元號略四卷,補遺一卷。梁玉繩撰。紀元通考十二卷。葉維庚撰。列代建元表十卷,建元類聚考二卷。錢東垣撰。紀元編三卷。李兆洛撰。歷代統紀表十三卷。段承基撰。

漢劉珍東觀漢記二十四卷,元郝經續後漢書九十卷。乾隆時敕輯。世本一卷。孫馮翼輯。漢宋衷世本注五卷。張澍輯。七家後漢書二十一卷。汪文臺撰。重訂謝承後漢書補逸五卷。孫志祖輯。薛瑩後漢書一卷,華嶠後漢書注一卷,謝沈後漢書一卷,袁山松後漢書一卷,張璠後漢記一卷,虞預晉書一卷,硃鳳晉書一卷,何法盛晉中興書一卷,謝靈運晉書一卷,臧榮緒晉書一卷,眾家晉書一卷。以上黃奭輯。九家舊晉書三十七卷。湯球輯。

雜史類

蒙古源流八卷。蒙古小徹辰薩囊臺吉撰。乾隆四十二年敕譯。國語韋昭注疏十六卷。洪亮吉撰。國語校文一卷。汪中撰。國語補注一卷。姚鼐撰。國語補校一卷。劉臺拱撰。國語補韋四卷。黃模撰。國語三君注輯存四卷,國語考異四卷,國語發正二十一卷。汪遠孫撰。國語翼解六卷。陳彖撰。國語釋地三卷。譚澐撰。國語正義二十一卷。董增齡撰。戰國策去毒二卷。陸隴其撰。戰國策釋地二卷。張琦撰。國策地名考二十卷。程恩澤撰,狄子奇箋。讀戰國策隨筆一卷。張尚瑗撰。戰國策札記三卷。顧廣圻撰。武王克殷日記一卷,滅國五十考一卷。林春溥撰。考信錄提要二卷,補上古考信錄二卷,唐虞考信錄四卷,夏考信錄二卷,商考信錄二卷,豐鎬考信錄八卷,豐鎬別錄三卷,考古續說二卷,考信附錄二卷。崔述撰。熹廟諒陰記一卷,聖安本紀六卷,明季實錄六卷。顧炎武撰。南宋六陵遺事一卷,庚申君遺事一卷。萬斯同撰。見聞隨筆二卷。馮甦撰。安南使事記一卷。李仙根撰。建文帝后紀一卷。邵遠平撰。武宗外紀一卷,後鑒錄七卷。毛奇齡撰。烈皇勤政記一卷,思陵典禮記四卷。孫承澤撰。三朝野紀七卷。李遜之撰。弘光日錄四卷,永歷實錄二十五卷,行朝錄十二卷,汰存錄一卷,贛州失事記一卷,紹武爭立記一卷,舟山興廢記一卷,四明山寨記一卷,沙州定亂記一卷,賜姓始末一卷,鄭成功傳一卷,滇考一卷,日本乞師記一卷。黃宗羲撰。永歷實錄二十六卷。王夫之撰。魯春秋一卷。查繼佐撰。偽東宮偽後及黨禍記略一卷,榆林城守記略一卷,保定城守記略一卷,揚州城守記略一卷。戴名世撰。二申野錄八卷。孫之騄撰。遜代陽秋二十八卷。餘美英撰。復社記事一卷。吳偉業撰。社事始末一卷。杜登春撰。啟禎野乘十六卷,二集八卷。鄒漪撰。蜀難★略一卷。沈荀蔚撰。金陵野鈔十四卷。顧苓撰。甲申傳信錄十卷。錢士馨撰。史外八卷。汪有典撰。明季北略二十四卷,南略十八卷。計六奇撰。東南紀事十二卷,西南紀事十二卷。邵廷寀撰。南疆逸史三十卷,恤謚錄八卷,摭遺十八卷。溫睿臨撰。南疆繹史五十八卷。李瑤撰。海東逸史十八卷。不著撰人氏名。爝火錄三十卷。李本撰。小腆紀傳六十五卷。徐鼒撰。補遺五卷,考異一卷。徐承禮撰。閩事紀略二卷。華廷獻撰。平定耿逆記一卷。李之芳撰。平閩記十三卷。楊捷撰。嘯亭雜錄十卷,續錄三卷。禮親王昭梿撰。養吉齋叢錄二十二卷。吳振棫撰。郎潛記聞初筆十四卷,二筆十六卷,三筆十二卷。陳康祺撰。聖德紀略一卷,儤直紀略一卷,恩遇紀略一卷,舊聞紀略一卷。瞿鴻禨撰。

宋不著撰人咸淳遺事二卷,大金吊伐錄四卷,元王鶚汝南遺事四卷。乾隆時敕輯。國語賈注一卷。蔣曰豫輯。鄭眾國語解詁一卷,賈逵國語注一卷,唐固國語注一卷,王肅國語章句一卷,孔晁國語注一卷,孔衍春秋後語一卷,陸賈楚漢春秋一卷,伏侯古今注一卷,王粲英雄記一卷、司馬彪戰略一卷、九州春秋一卷,傅暢晉諸公贊一卷,荀綽晉後略一卷,盧綝晉八王故事一卷,晉四王遺事一卷。以上黃奭輯。

詔令奏議類

太祖高皇帝聖訓四卷。康熙二十五年敕編。太宗文皇帝聖訓六卷。順治時敕編,康熙二十六年告成。世祖章皇帝聖訓六卷。康熙二十六年敕編。親政綸音不分卷。順治時敕編。聖祖仁皇帝聖訓六十卷。雍正九年敕編。庭訓格言不分卷。世宗御編。聖諭廣訓不分卷。雍正二年敕刊。上諭內閣一百五十九卷。雍正七年敕刊,乾隆時續刊。硃批諭旨三百六十卷。雍正十年敕編,乾隆三年告成。上諭八旗十三卷,上諭旗務議覆十二卷,諭行旗務奏議十三卷。雍正九年敕編。訓飭州縣條規二十卷。雍正八年敕刊。世宗憲皇帝聖訓三十六卷。乾隆五年敕編。高宗純皇帝聖訓三百卷。嘉慶十二年敕編。仁宗睿皇帝聖訓一百十卷。道光四年敕編。宣宗成皇帝聖訓一百三十卷。咸豐六年敕編。文宗顯皇帝聖訓一百十卷。同治五年敕編。穆宗毅皇帝聖訓一百六十卷。光緒五年敕編。明名臣奏議二十卷。乾隆四十六年奉敕編。息齋疏草五卷。金之俊撰。龔端毅奏議八卷,附錄一卷。龔鼎孳撰。孟忠毅公奏議二卷。孟喬芳撰。趙忠襄奏疏存稿六卷。趙良棟撰。張襄壯奏疏六卷。張勇撰。兼濟堂奏議四卷。魏裔介撰。寒松堂奏議四卷。魏象樞撰。文襄公奏疏十五卷。李之芳撰。撫虔奏議一卷。佟國器撰。平岳疏議一卷,平海疏議一卷。萬正色撰。郝恭定集五卷。郝惟訥撰。中山奏議四卷。郝浴撰。靳文襄奏疏八卷。靳輔撰。乾清門奏對記一卷。湯斌撰。撫浙奏議一卷,督閩奏議一卷。範承謨撰。撫浙疏草五卷。硃昌祚撰。撫吳封事八卷,撫楚封事一卷,撫黔封事一卷,撫漕封事一卷,輯瑞陳言一卷。慕天顏撰。於山奏牘七卷。於成龍撰。清忠堂奏疏不分卷。硃宏祚撰。西臺奏議一卷,京兆奏議一卷,附曲徙錄一卷。楊素蘊撰。楊黃門奏疏不分卷,撫黔奏疏八卷。楊雍建撰。華野疏稿五卷。郭琇撰。河防疏略二十卷。硃之錫撰。西陂奏疏六卷。宋犖撰。督漕疏草二十二卷。董訥撰。奏疏稿不分卷。江蘩撰。撫豫宣化錄四卷。田文鏡撰。防河奏議十二卷。嵇曾筠撰。平蠻奏疏一卷。鄂爾泰撰。張公奏議二十四卷。張鵬翮撰。條奏疏稿二卷。蔣廷錫撰。奏疏十卷。高其倬撰。望溪奏疏一卷。方苞撰。尹元孚奏議十卷。尹會一撰。裘文達奏議一卷。裘曰修撰。那文毅奏議八十卷。那彥成撰。兩河奏疏不分卷。嚴烺撰。思補齋奏稿偶存一卷。潘世恩撰。恭壽堂奏議十二卷。韓文綺撰。楚蒙山房奏疏五卷。晏斯盛撰。東溟奏稿四卷。姚瑩撰。林文忠政書三卷。林則徐撰。陶雲汀先生奏議三十二卷。陶澍撰。耐菴奏議存稿十二卷。賀長齡撰。吳文節遺集八十卷。吳文鎔撰。張大司馬奏稿四卷。張亮基撰。駱文忠奏議十六卷。駱秉章撰。李文恭奏議二十二卷。李星沅撰。李尚書政書八卷。李宗羲撰。王侍郎奏議十卷。王茂廕撰。臺垣疏稿一卷。丁壽昌撰。張文毅奏稿八卷。張芾撰。曾文正奏稿三十二卷。曾國籓撰。胡文忠奏稿五十二卷。胡林翼撰。左文襄奏疏初編三十八卷,續編七十六卷,三編六卷。左宗棠撰。曾忠襄奏疏六十一卷。曾國荃撰。沈文肅政書十二卷。沈葆楨撰。李忠武奏議一卷。李續賓撰。劉中丞奏稿八卷。劉昆撰。劉中丞奏議二十卷。劉蓉撰。劉武慎奏稿十六卷。劉長佑撰。彭剛直奏議八卷。彭玉麟撰。郭侍郎奏疏十二卷。郭嵩燾撰。岑襄勤奏稿三十卷。岑毓英撰。丁文誠奏議二十六卷。丁寶楨撰。毛尚書奏稿十六卷。毛鴻賓撰。曾惠敏奏議六卷。曾紀澤撰。出使奏疏二卷。薛福成撰。養雲山莊奏稿四卷。劉瑞芬撰。錢敏肅奏疏七卷。錢鼎銘撰。黎文肅奏議十六卷。黎培敬撰。許太常奏稿一卷。許乃濟撰。豸華堂奏議十二卷。金應麟撰。水流雲在館奏議二卷。宋晉撰。吳柳堂奏疏一卷。吳可讀撰。王文敏奏疏稿一卷。王懿榮撰。袁太常戊戌條陳一卷。袁昶撰。諫垣存稿四卷。安維峻撰。李文忠政書一百六十五卷。李鴻章撰。張宮保政書十二卷。張之洞撰。端忠敏奏議十六卷。端方撰。三賢政書十八卷。湯斌、宋犖、張伯行撰。嘉定長白二先生奏議四卷。徐致祥、寶廷撰。

宋陳次升讜論集五卷。乾隆時敕輯。

傳記類

宗室王公功績表傳十二卷。乾隆四十六年敕撰。蒙古王公功績表傳十二卷。乾隆四十四年敕撰。八旗滿洲氏族通譜八十卷。乾隆九年敕撰。勝朝殉節諸臣錄十二卷。乾隆四十一年敕撰。滿漢名臣傳八十卷,貳臣傳八卷,逆臣傳二卷。乾隆時敕撰。史傳三編五十六卷。硃軾撰。歷代忠臣義士卓行錄八卷。戴作銘撰。歷代名臣言行錄二十四卷。硃桓撰。廣群輔錄六卷。徐汾撰。臣鑒錄二十卷。蔣伊撰。歷代黨鑒五卷。徐賓撰。續高士傳五卷。高兆撰。續補高士傳三卷。魏裔介撰。孝史類編十卷。黃齊賢撰。元祐黨人傳十卷。陸心源撰。明名臣言行錄四十五卷。徐開仕撰。崇禎五十宰相傳一卷,年表一卷。曹溶撰。明儒言行錄十卷,續錄十卷。沈佳撰。東林列傳二十四卷,留溪外傳十八卷。陳鼎撰。復社姓氏傳略十卷。吳山嘉撰。國朝耆獻類徵初編七百二十卷,編目十九卷。李桓撰。碑傳集一百六十卷。錢儀吉撰。續碑傳集八十六卷。繆荃孫撰。國朝先正事略六十卷。李元度撰。中興將帥別傳三十卷,一作咸同以來功臣別傳,一作中興名臣事略,一作續先正事略。續編六卷。硃孔彰撰。大清名臣言行錄一卷。留保撰。文獻徵存錄十卷。錢林撰。從政觀法錄三十卷。硃方曾撰。初月樓聞見錄十卷,續錄十卷。吳德旋撰。學統五十六卷。熊賜履撰。雒閩淵源錄十九卷。張夏撰。聖學知統錄二卷,聖學知統翼編二卷。魏裔介撰。道統錄二卷,附錄一卷,道南源委六卷,伊洛淵源續錄二十卷。張伯行撰。儒林宗派十六卷。萬斯同撰。理學宗傳二十六卷。孫奇逢撰。理學宗傳辨正十六卷。劉廷詔撰。宋元學案一百卷。黃宗羲原本,全祖望補編。明儒學案六十二卷。黃宗羲撰。明儒林錄十九卷。張恆撰。國朝學案小識十五卷。唐鑒撰。國朝經學名儒記一卷。張星鑒撰。國朝宋學淵源記二卷,附記一卷。江籓撰。國朝儒林文苑傳四卷。阮元撰。康熙己未詞科錄十二卷。秦瀛撰。鶴徵錄八卷。李集、李富孫、李遇孫同撰。詞科掌錄十七卷,餘話二卷。杭世駿撰。鶴徵後錄十二卷。李富孫撰。疇人傳四十六卷。阮元撰。續疇人傳六卷。羅士琳撰。疇人傳三編七卷。諸可寶撰。國朝名家詩鈔小傳二卷。鄭方坤撰。畿輔人物志二十卷。孫承澤撰。洛學編四卷。湯斌撰。中州人物考八卷。孫奇逢撰。中州道學編二卷,補編一卷。耿介撰。關學編十卷。廉偉然撰。東越儒林後傳一卷,文苑後傳一卷。陳壽祺撰。閩中理學淵源考九十二卷,閩學志略十七卷。李清馥撰。粵東名儒言行錄二十四卷。鄧淳撰。豫章十代文獻略五十卷。王模撰。金華徵獻略二十卷。王崇炳撰。嘉禾獻徵錄四十六卷。盛楓撰。松陵文獻錄十五卷。潘檉章撰。海州文獻錄十六卷。許喬林撰。吳門耆舊記一卷。顧承撰。列女傳補注八卷,附敘錄一卷,校正一卷。閨秀王照圓撰。列女傳校注八卷。閨秀梁端撰。列女傳集注八卷。閨秀蕭道管撰。廣列女傳二十卷。劉開撰。勝朝彤史拾記六卷。毛奇齡撰。賢媛類徵初編十二卷。李桓撰。越女表微錄五卷。汪輝祖撰。

宋不著撰人慶元黨禁一卷,京口耆舊傳九卷,元辛文房唐才子傳八卷。以上乾隆時奉敕輯。魏嵇康聖賢高士傳一卷,後魏常景鑒戒象贊一卷。以上馬國翰輯。趙岐三輔決錄一卷,劉向孝子傳一卷,蕭廣濟孝子傳一卷,師覺授孝子傳一卷。以上黃奭輯。

以上傳記類總錄之屬

晏子春秋音義一卷。孫星衍撰。晏子春秋校正一卷。盧文弨撰。晏子春秋校勘一卷。黃以周撰。周公年表一卷。牟廷相撰。孔子年譜五卷。楊方晃撰。孔子年譜輯注一卷。江永撰,黃定宜輯注。孔子編年注五卷。胡培翬撰。至聖編年世紀二十四卷。李灼、黃晟同撰。先聖生卒年月考二卷。孔廣牧撰。孔子世家考二卷,仲尼弟子列傳考一卷。鄭環撰。宗聖志十二卷。孔允植撰。闕里文獻考一百卷。孔繼汾撰。孔子世家補訂一卷,孔門師弟年表一卷,孔孟年表一卷,孟子列傳纂一卷,孟子時事年表一卷。林春溥撰。孔子編年四卷,孟子編年四卷。狄子奇撰。洙泗考信錄四卷,餘錄一卷,孟子事實錄二卷。崔述撰。孔子弟子門人考一卷,孟子弟子門人考一卷。硃彞尊撰。孟子年譜一卷。黃玉蟾撰。孟子生卒年月考一卷。閻若璩撰。孟子游歷考一卷。潘眉撰。三遷志十二卷。孟衍泰、王特選、仲蘊錦同撰。從祀名賢傳六卷。常安撰。劉更生年表一卷。梅毓撰。許君年表一卷。陶方琦撰。鄭司農年譜一卷。孫星衍撰。漢鄭君晉陶靖節魏陳思王唐陸宣公年譜四卷。丁晏撰。鄭康成紀年一卷。袁鈞撰。鄭學錄四卷。鄭珍撰。諸葛忠武故事五卷。張澍撰。忠武志八卷。張鵬翮撰。王右軍年譜一卷。魯一同撰。安定言行錄一卷。丁寶書撰。濂溪周夫子志十五卷。吳大鎔撰。增訂歐陽文忠年譜一卷。硃文藻撰。胡少師年譜一卷。胡培翬撰。王荊公年譜二十五卷,雜錄二卷,附錄一卷。蔡上翔撰。米海嶽年譜一卷。翁方綱撰。考訂硃子世家一卷。江永撰。硃子年譜四卷,考異四卷,附錄二卷。王懋竑撰。重訂硃子年譜一卷。褚寅亮撰。別本硃子年譜二卷,附錄一卷。黃中撰。陸象山年譜二卷。李紱撰。楊文靖年譜二卷。張夏撰。洪文惠年譜一卷,洪文敏年譜一卷,陸放翁年譜一卷,王伯厚年譜一卷。錢大昕撰。王深寧年譜一卷。張大昌撰。謝皋羽年譜一卷。徐沁撰。元遺山年譜三卷。翁方綱撰。元遺山年譜二卷。凌廷堪撰。元遺山年譜一卷。施國祁撰。周文襄公年譜二卷。周仁俊撰。李文正公年譜一卷。法式善撰。王文成集傳本二卷。毛奇齡撰。王弇州年譜一卷。錢大昕撰。歸震川年譜一卷。孫岱撰。楊升庵年譜一卷。李調元撰。周忠介公遺事一卷。彭定求撰。繆文貞公年譜一卷。繆之鎔撰。袁督師事跡一卷。不著撰人氏名。倪文正公年譜一卷。倪會鼎撰。黃忠端公年譜二卷。黃炳垕撰。左忠毅年譜二卷。左宰撰。張忠烈公年譜一卷。趙之謙撰。劉子行狀二卷。黃宗羲撰。蕺山年譜二卷。劉均撰。顧亭林年譜一卷。吳映奎撰。顧亭林年譜四卷。張穆撰。黃黎洲年譜二卷。黃炳垕撰。孫夏峰年譜二卷。湯斌撰。李二曲歷年紀略二卷。惠嗣撰。楊園先生年譜四卷。陳梓撰。楊園先生年譜一卷。蘇惇元撰。顏習齋先生年譜二卷。李恭撰。李恕谷先生年譜五卷。馮辰撰。申鳧盟先生年譜一卷。申涵煜、申涵盼同撰。寧海將軍固山貝子功績錄一卷。不著撰人氏名。漁洋山人自訂年譜注一卷。惠棟撰。施愚山年譜四卷。施念曾撰。陸清獻年譜一卷。羅以智撰。陸稼書年譜二卷。吳光酉撰。閻潛丘年譜四卷。張穆撰。硃文端公行述一卷。硃必階撰。阿文成年譜二十四卷。那彥成撰。錢文端公年譜三卷。錢儀吉撰。王述庵年譜二卷。嚴榮撰。孫文靖年譜一卷。孫惠惇撰。黃昆圃年譜一卷。黃叔琳撰。黃蕘圃年譜一卷。江標撰。戴東原年譜一卷。段玉裁撰。洪北江年譜一卷。呂培撰。焦理堂事略一卷。焦廷琥撰。寄圃老人自記年譜一卷。孫玉庭撰。思補老人自訂年譜一卷。潘世恩撰。石隱山人自訂年譜一卷。硃駿聲撰。彭文敬自訂年譜一卷。彭蘊章撰。翁文端年譜一卷。翁同龢撰。駱文忠年譜一卷。駱天保撰。曾文正年譜十二卷。黎庶昌撰。曾文正公大事記四卷。王定安撰。吳柳堂孤忠錄三卷。傅巖霖撰。豫章先賢九家年譜九卷,四朝先賢六家年譜七卷。楊希閔撰。四史疑年錄七卷。阮元撰。歷代名人年譜十七卷。吳榮光撰。疑年錄四卷。錢大昕撰。續疑年錄四卷。吳修撰。補疑年錄四卷。錢椒撰。疑年賡錄二卷。張鳴珂撰。三續疑年錄十卷。陸心源撰。

以上傳記類名人之屬

史鈔類

史緯三百三十卷。陳允錫撰。讀史蒙拾一卷。王士祿撰。廿一史約編十卷。鄭元慶撰。漢書蒙拾三卷,後漢書蒙拾二卷。杭世駿撰。漢書古字類一卷。郭夢星撰。國志蒙拾二卷。郭麟撰。宋書瑣語一卷。郝懿行撰。兩晉南北集珍六卷。陳維崧撰。南史識小錄八卷,北史識小錄八卷。沈名孫、硃昆田同撰。南北史識小錄補正二十八卷。張應昌撰。南北史捃華八卷。周嘉猷撰。新舊唐書合鈔二百六十卷。沈炳震撰。

載記類

吳越春秋校文一卷。蔣光煦撰。吳越春秋校勘記一卷,逸文一卷。顧觀光撰。讀吳越春秋一卷,讀越絕書一卷。俞樾撰。越絕書札記一卷,逸文一名。錢培名撰。增訂吳越備史五卷,補遺一卷。錢時鈺撰。補華陽國志三州郡縣目錄一卷。廖寅撰。華陽國志校勘記一卷。顧觀光撰。十六國疆域志十六卷。洪亮吉撰。十六國春秋輯補一百卷,十六國春秋纂錄校本十卷。湯球撰。十六國年表一卷。張愉曾撰。十六國年表三十二卷。孔尚質撰。西秦百官表一卷。練恕撰。十國春秋一百十四卷。吳任臣撰。拾遺一卷,備考一卷。周昂撰。南漢書十八卷,考異十八卷,叢錄二卷,文字略二卷。梁廷相撰。南漢紀五卷,地理志一卷,金石志一卷。吳蘭修撰。南唐拾遺記一卷。毛先舒撰。西夏國志十六卷。洪亮吉撰。西夏書事四十二卷。吳廣成撰。西夏紀事本末三十六卷。張鑒撰。西夏書十卷。周春撰。西夏事略十六卷。陳昆撰。

晉陸翽鄴中記一卷,唐樊綽蠻書十卷,宋不著撰人江南餘載二卷。乾隆時奉敕輯。

時令類

月令輯要二十四卷,圖說一卷。康熙五十四年,李光地等奉敕撰。古今類傳歲時部四卷。董穀士、董炳文同編。時令匯紀十六卷,餘日事文四卷。硃濂撰。月日紀古十二卷。蕭智漢撰。節序同風錄十二卷。孔尚任撰。七十二候考一卷。曹仁虎撰。月令粹編二十四卷。秦嘉謨撰。二十四史日月考二百三十六卷。汪曰楨撰。古今冬至表四卷。譚澐撰。

唐韓鄂四時纂要一卷。馬國翰輯。

◎地理類

皇輿表十六卷。康熙四十三年,喇沙裏等奉敕撰。方輿路程考略不分卷。康熙時,汪士鋐等奉敕撰。大清一統志三百四十卷。乾隆八年敕撰。大清一統志五百卷。乾隆二十九年敕撰。皇朝職貢圖九卷。乾隆十六年,傅恆等奉敕撰。歷代疆域表三卷,沿革表三卷。段長基撰。歷代地理沿革表四十七卷。陳芳績撰。東晉南北朝輿地表二十一卷。徐文範撰。輿地沿革表四十卷。楊丕復撰。周末列國所有郡縣考一卷,古國都今郡縣合考一卷。閔麟嗣撰。戰國地輿一卷。林春溥撰。楚漢諸侯疆域志三卷。劉文淇撰。歷代郡國考略三卷。葉澐撰。今古地理述二十卷。王子音撰。歷代地理沿革圖一卷,輿地圖一卷,歷代地理志韻編今釋二十卷,皇朝輿地韻編二卷。李兆洛撰。王會新編一百四十五卷。茹鉉撰。乾隆府州縣志五十卷。洪亮吉撰。皇朝輿地全圖不分卷。董祐誠撰。大清一統輿圖三十卷。胡林翼撰。皇朝輿地韻編一卷,輿地略一卷。嚴德撰。郡縣分韻考十卷。黃本驥撰。肇域志一百卷,天下郡國利病書一百二十卷。顧炎武撰。讀史方輿紀要一百三十卷,形勢紀要九卷。顧祖禹撰。太平寰宇記補缺二卷。陳蘭森撰。山河兩戒考十四卷。徐文靖撰。

晉太康三年地記一卷,王隱晉書地道記一卷,唐濮王泰等括地志一卷。以上黃奭輯。

以上地理類總志之屬

滿洲源流考二十卷。乾隆四十二年,阿桂等奉敕撰。熱河志八十卷。乾隆四十六年,和珅等奉敕撰。日下舊聞考一百二十卷。乾隆三十九年敕撰。日下舊聞四十二卷。硃彞尊撰。盛京通志一百二十卷。乾隆四十四年,阿桂等奉敕撰。新疆識略十三卷。道光元年,汪廷珍等奉敕撰。盛京通志四十八卷。雷以誠等修。畿輔通志一百二十卷。李衛等修。畿輔通志三百卷。李鴻章等修。江南通志二百卷。趙宏恩等修。安徽通志二百六十卷。陶澍修。安徽通志三百五十卷。劉坤一等修。江西通志二百六卷。白璜等修。江西通志一百六十二卷。謝旻等修。江西通志一百八十卷。劉坤一等修。浙江通志二百八十卷。嵇曾筠等修。福建通志七十八卷。郝玉麟修。福建通志二百七十八卷。吳棠等修。湖廣通志八十卷。徐國相等修。湖廣通志一百二十卷。邁柱等修。湖北通志一百卷。吳熊光等修。湖南通志一百七十卷。陳宏謀等修。湖南通志二百二十八卷。巴哈布等修。湖南通志三百十五卷。裕祿等修。河南通志八十卷。王士俊等修。續河南通志八十卷。阿思喀等修。山東通志三十六卷。岳濬等修。山東通志六十四卷。錢江等修。山西通志二百三十卷。覺羅石麟等修。山西通志一百八十四卷。張煦等修。山西志輯要十卷。雅德撰。陜西通志一百卷。劉於義等修。甘肅通志五十卷。許容等修。甘肅通志一百卷。長庚等修。四川通志四十七卷。黃廷桂等修。四川通志二百二十六卷。楊芳燦等修。廣東通志六十四卷。郝玉麟等修。廣東通志三百三十四卷。阮元等修。廣西通志一百二十八卷。金鉷等修。廣西通志二百八十卷。吉慶等修。雲南通志三十卷。鄂爾泰等修。續雲南通志稿一百九十四卷。王文韶等修。貴州通志四十六卷。鄂爾泰等修。吉林通志一百二十二卷。長順等修。順天府志一百三十卷。李鴻章修。保定府志八十卷。李振祜修。承德府志六十卷。海忠修。永平府志七十二卷。游智開修。河間府志二十卷。周嘉露修。天津府志四十卷。李梅賓修。天津府志五十四卷。李鴻章修。正定府志五十卷。鄭大進修。順德府志十六卷。徐景曾修。廣平府志二十四卷。吳穀修。大名府志二十二卷。李煐修。大名府志六卷。武蔚文修。宣化府志四十二卷。王畹修。江寧府志五十六卷。呂燕昭修。江寧府志十五卷。蘇啟勛修。蘇州府志八十卷。習巂撰。蘇州府志一百六十卷。石韞玉撰。蘇州府志一百五十卷。馮桂芬撰。松江府志八十四卷。宋如林修。松江府志四十卷。博潤修。常州府志三十八卷。於琨修。淮安府志三十二卷。顧棟高撰。揚州府志四十卷。張萬壽修。揚州府志七十二卷。張世浣修。揚州府志三十卷。晏端書撰。徐州府志三十卷。王峻修。安慶府志三十二卷。張楷修。徽州府志八卷。鄭交泰修。寧國府志三十八卷。魯銓修。池州府志五十八卷。張士範修。太平府志四十四卷。硃肇基修。廬州府志五十四卷。張祥雲修。鳳陽府志二十一卷。馮煦修。潁州府志十卷。王斂福修。南昌府志七十六卷。黃良棟修。饒州府志三十六卷。黃家遴修。廣信府志二十六卷。康基淵修。南康府志十二卷。廖文英修。九江府志二十二卷。胡宗虞修。建昌府志三十四卷。姚文光修。撫州府志四十五卷。張四教修。臨江府志十六卷。施潤章撰。瑞州府志二十四卷。黃廷金修。袁州府志十五卷。陳喬樅撰。吉安府志七十六卷。盧松修。贛州府志七十八卷。李本仁修。南安府志二十卷。陳奕禧撰。南安府志三十二卷。黃鳴珂修。杭州府志四十卷。馬鐸修。杭州府志一百十卷。鄭枟修。嘉興府志十六卷。吳永芳修。嘉興府志八十卷。伊湯安修。嘉興府志九十卷。許瑤光修。湖錄一百五卷。鄭元慶撰。湖州府志十二卷。程量修。湖州府志四十八卷。李堂修。湖州府志九十六卷。宗源瀚撰。寧波府志三十六卷。曹秉仁修。紹興府志六十卷。鄒尚周修。紹興府志八十卷。李亨特修。臺州府志十八卷。馮甦修。金華府志三十卷。張藎修。衢州府志三十五卷。楊廷望修。嚴州府志三十五卷。吳士進修。溫州府志三十卷。汪爌修。處州府志二十卷。曹掄彬修。處州府志三十二卷。潘紹貽修。福州府志七十六卷。高景崧修。泉州府志七十六卷。章倬標修。建寧府志四十八卷。張琦修。延平府志四十六卷。徐震耀修。汀州府志四十五卷。曾曰煐修。邵武府志三十卷。王琛修。邵武府志二十四卷。張鳳孫修。漳州府志五十卷。沈定均修。福寧府志三十卷。李紱修。臺灣府志二十六卷。六十七修。武昌府志十二卷。裴天錫修。漢陽府志五十卷。陶士偰修。安陸府志三十六卷。張尊德修。襄陽府志四十卷。陳諤修。鄖陽府志十卷。王正常修。鄖陽府志三十八卷。楊廷耀修。德安府志二十四卷。傅鶴祥修。黃州府志二十卷。王勍修。荊州府志五十八卷。施廷樞修。宜昌府志十六卷。聶光鑾修。施南府志三十卷。松林修。長沙府志五十卷。呂肅高修。岳州府志三十卷。黃凝道修。寶慶府志一百五十七卷。黃宅中修。衡州府志三十二卷。饒佺修。常德府志四十八卷。應光烈撰。辰州府志十一卷。畢本烈修。沅州府志四十卷。張官五修。永州府志十八卷。宗績辰撰。永順府志十二卷。張天如修。開封府志四十卷。管竭忠修。陳州府志三十卷。崔應楷修。歸德府志三十六卷。陳錫輅修。彰州府志三十二卷。湯康業修。衛輝府志五十五卷。德昌修。懷慶府志三十二卷。杜悰修。河南府志一百十六卷。施誠修。南陽府志六卷。孔傳金修。汝寧府志三十卷。德昌修。濟南府志七十二卷。王贈芳修。泰安府志三十二卷。成城修。武定府志三十八卷。李熙齡修。兗州府志三十二卷。陳顧修。沂州府志二十三卷。李希賢修。曹州府志二十二卷。周尚質修。東昌府志五十卷。白嵩修。青州府志六十四卷。毛永相修。登州府志六十九卷。賈瑚修。萊州府志十六卷。嚴有禧修。太原府志六十卷。沈樹聲修。平陽府志三十六卷。章廷珪修。蒲州府志二十四卷。周景柱修。潞安府志四十卷。張淑渠修。汾州府志三十六卷。孫和相修,戴震撰。澤州府志五十二卷。硃樟修。大同府志三十二卷。吳輔宏修。寧武府志十二卷。周景柱修。朔平府志十二卷。劉士銘修。西安府志八十卷。嚴長明撰。同州府志三十四卷。李思繼修。鳳翔府志十二卷。達靈阿修。漢中府志三十二卷。嚴如煜撰。興安府志三十卷。葉世倬修。延安府志八十卷。張蕙修。榆林府志五十卷。李熙齡修。蘭州府志四卷。陳如稷修。西寧志七卷。蘇銳修。甘州府志十六卷。鍾賡起修。保寧府志六十二卷。史觀修。重慶府志九卷。王夢庚修。夔州府志三十六卷。恩成修。雅州府志二十卷。陳鈞修。廣州府志六十卷。沈廷芳修。肇慶府志二十一卷。何夢瑤撰。韶州府志十六卷。唐宗堯修。惠州府志二十卷。呂應奎修。惠州府志四十五卷。劉溎年修。潮州府志四十二卷,廉州府志二十卷。周碩勛修。高州府志十六卷。黃安濤撰。雷州府志二十卷。雷學海修。瓊州府志四十四卷。張岳崧撰。平樂府志四十卷。清桂修。潯州府志三十九卷。魏篤修。鎮安府志八卷。傅聚修。雲南府志三十卷。張毓修。大理府志三十卷。黃元治修。臨安府志二十卷。江濬源修。楚雄府志十卷。張嘉穎修。澂江府志十六卷。柳正芳修。廣南府志四卷。何愚修。順寧府志十卷。劉靖修。曲靖府志八卷。程封修。麗江府志二卷。萬咸燕修。永昌府志二十六卷。宣世濤修。永北府志二十八卷。陳奇典修。東川府志二十卷。方桂修。思州府志八卷。蔣深修。鎮遠府志二十卷。蔡宗建修。銅仁府志十一卷。徐訚修。黎平府志四十一卷。劉宇昌修。遵義府志四十八卷。鄭珍、莫友芝同撰。遵化直隸州志十二卷。劉靖修。易州直隸州志十八卷。張登高修。冀州直隸州志二十卷。範清曠修。趙州直隸州志十卷。祝萬祉修。深州直隸州風土記二十二卷。吳汝綸撰。定州直隸州志四卷。王榕吉修。口北三志十八卷。黃可潤修。川沙志十四卷。俞樾撰。海州直隸州志三十二卷。唐仲冕撰。通州直隸州志十五卷。王宜亨修。廣德直隸州志五十卷。周廣業修。滁州直隸州志三十卷。敦泰修。和州直隸州志二十四卷。夏煒修。六安直隸州志五十卷。周廣業修。泗州直隸州志十八卷。莫之幹修。蓮花志十卷。李其昌修。寧州直隸州志三十二卷。劉丙修。定南志八卷。賴勛修。定海直隸志三十卷。陳重威、黃以周同撰。玉環志四卷。張坦龍修。玉環志十五卷。呂鴻燾修。廈門志十六卷。周凱修。永春直隸州志十六卷,龍巖直隸州志十六卷。鄭一崧修。噶嗎蘭志八卷。董正官修。淡水志十五卷。陳培桂修。荊門直隸州志十二卷。黃昌輔修。鶴峰直隸志十四卷。吉鍾穎修。澧州直隸州志二十八卷。魏式曾修。桂陽直隸州志二十七卷。陳延棨修。鳳皇直隸志二十卷。黃應培修。永綏直隸志十八卷。周玉衡修。乾州直隸志十六卷。趙文在修。晃州直隸志四十四卷。俞光振修。靖州直隸州志十二卷。汪尚文修。郴州直隸州志四十三卷。硃偓修。鄭州直隸州志十二卷。張鉞修。許州直隸州志十六卷。段汝舟修。陜州直隸州志二十卷。龔崧林修。淅川直隸志九卷。徐光第修。汝州直隸州志十卷。錢福昌修。濟寧直隸州志三十四卷。周永年、盛百二同撰。臨清直隸州志十一卷。硃度修。膠州直隸州志八卷。於智修。平定直隸州志十卷。金明源修。忻州直隸州志四十二卷。方戊昌修。代州直隸州志六卷。吳重光修。保德直隸州志十二卷。王秉韜修。霍州直隸州志二十五卷。崔允臨修。解州直隸州志十八卷。言如泗修。絳州直隸州志二十卷。張成德修。沁州直隸州志十卷。雷暢修。商州直隸州志十四卷。王如玖修。潼關志九卷。楊端本修。定遠志二十六卷。餘修鳳修。留壩志十卷。賀仲瑊修。漢陰志十卷。錢鶴年修。鄜州直隸州志十卷。吳鳴捷修。涇州直隸州志二卷。張延福修。階州直隸州志二卷。林忠修。秦州直隸州志十二卷。任其昌修。肅州直隸州志不分卷。黃文煒修。循化志八卷。龔景瀚撰。資州直隸州志三十卷。劉蜅修。綿州直隸州志五十四卷。範紹泗修。茂州直隸州志四卷。楊迦懌修。馬邊志六卷。周斯才修。敘永直隸志四十六卷。周偉業修。江北志八卷。宋煊修。酉陽直隸州志二十四卷。馮世瀛修。忠州直隸州志八卷。呂鋧麟撰。石砫直隸志十二卷。王槐齡修。眉州直隸州志十九卷。徐長發修。眉州直隸州志四十六卷。吳鞏修。連州直隸州志十二卷。單興詩修。連山直隸志一卷。姚柬之修。南雄直隸州志三十四卷。黃其勤修。嘉應直隸州志十二卷。王之正修。欽州直隸州志十二卷。硃椿年修。陽江直隸州志八卷。胡璟修。崖州直隸州志十卷。宋錦修。景東直隸志二十八卷。羅含章修。廣西府志二十六卷。周埰修。元江直隸州志四卷。廣裕修。蒙化直隸志六卷。徐時行修。永北府志二十八卷。陳奇典修。鎮邊撫彞直隸志八卷。謝體仁修。永清縣志二十四卷。章學誠撰。遷安府志二十卷,撫寧縣志十二卷。史夢蘭撰。靈壽縣志十卷。陸隴其撰。上元江寧縣志三十卷。莫友芝、甘紹盤同撰。高淳縣志二十八卷。張裕釗撰。吳江縣志四十六卷。郭琇撰。黎里志十六卷。徐達源撰。崇明縣志十八卷。李聯琇撰。華亭縣志二十四卷。姚光發、張文虎撰。婁縣志三十卷。陸錫熊撰。上海縣志二十卷。李林松撰。南匯縣志二十二卷。張文虎撰。青浦縣志四十卷。王昶撰。武進陽湖縣志三十卷。湯成烈撰。無錫金匱縣志四十卷。秦緗業撰。宜興荊溪縣志十卷。吳德旋撰。荊溪縣志四卷。唐仲冕撰。丹徒縣志六十卷。呂耀鬥撰。寶應圖經六卷。劉寶楠撰。邳州志二十卷,清河縣志二十四卷。魯一同撰。山陽縣志二十一卷。何紹基、丁晏同撰。合肥縣志三十六卷。左輔撰。鳳臺縣志十二卷。李兆洛撰。弋陽縣志十四卷,宜春縣志十五卷,分宜縣志十五卷,萬載縣志十八卷。陳喬樅撰。海昌備志十六卷。錢泰吉撰。海鹽縣續圖經七卷。王為珪撰。南潯鎮志四十一卷。汪曰楨撰。黃巖縣志四十卷。王詠霓撰。羅源縣志三十卷。林春溥撰。臺灣縣志十七卷。王禮撰。黃岡縣志二十四卷。劉恭冕撰。麻城縣志五十六卷。潘頤福撰。東湖縣志三十卷。王柏心撰。湘陰縣志三十六卷。郭嵩燾撰。武陵縣志三十一卷。楊丕復、楊彞珍同撰。龍陽縣志三十一卷。黃教鎔撰。杞紀二十二卷。張楨撰。孟縣志十卷。馮敏昌撰。偃師縣志三十卷。孫星衍撰。登封縣志二十八卷。洪亮吉撰。新城縣志十四卷。王士禛撰。曲阜縣志二十六卷。孔毓琚撰。聊城縣志四卷。傅以漸撰。靈石縣志十二卷。王志瀜撰。澄城縣志二十一卷。洪亮吉、孫星衍同撰。武威縣志一卷,鎮番縣志一卷,永昌縣志一卷,古浪縣志一卷,平番縣志一卷。張弨美撰。什邡縣志五十四卷。紀大奎撰。羅江縣志十卷。李調元撰。遂寧縣志六卷。張鵬翮撰。新會縣志十四卷。黃培芳、曾釗同撰。師宗州志二卷。夏治元撰。彌勒州志二十七卷。王緯撰。祿勸州志二卷。李廷宰撰。永寧州志十二卷。沈毓蘭撰。

以上地理類都會郡縣之屬

盤山志二十一卷。乾隆十九年,蔣溥等奉敕撰。清涼山新志十卷。康熙間敕撰。萬山綱目二十一卷。李誠撰。長白山錄一卷,補遺一卷。王士禛撰。萬歲山考證一卷,昌平山水記二卷,岱岳記一卷。顧炎武撰。泰山志二十卷。金棨撰。泰山道里記一卷。聶鈫撰。岱覽三十二卷。唐仲冕撰。泰山述記十卷。宋思仁撰。說嵩三十二卷,嵩嶽廟史十卷。景日钁撰。南嶽志八卷。高自位撰。岳麓志八卷。趙寧撰。華嶽志八卷。李榕撰。恆岳志三卷。張崇德撰。恆山志五卷。桂敬順撰。攝山志八卷。陳毅撰。寶華山志十五卷。劉名芳撰。山志八卷。顧云撰。茅山志十四卷。笪重光撰。北固山志二卷。釋了璞撰。金山志略四卷。釋行海撰。焦山志二十六卷。吳雲撰。虎丘山志二十四卷。顧詒祿撰。慧山記續編四卷。邵涵初撰。黃山志七卷。閔麟嗣撰。九華紀勝二十三卷,齊山巖洞志二十六卷。陳蔚撰。廬山小志二十四卷。蔡瀛撰。青源山志略十三卷。施潤章撰。四明山志九卷。黃宗羲撰。普陀山志十五卷。硃謹、陳璿同撰。西天目祖山志八卷。釋廣賓撰。天臺山全志十六卷。張聯元撰。廣雁蕩山志三十卷。曾唯撰。天竺山志十二卷。管廷芳撰。武夷山新志二十四卷。董天工撰。麻姑山丹霞洞天志十七卷。羅森撰。鼓山志十二卷。僧元賢撰。大別山志十卷,黃鵠山志十二卷。胡鳳丹撰。蓮峰志五卷。王夫之撰。洛陽龍門志一卷。路朝霖撰。太嶽太和山紀略八卷。王概撰。瓘眉山志十八卷。蔣超撰。羅浮山志會編二十二卷。宋廣業撰。西樵志六卷。馬符籙撰。桂鬱巖洞志一卷。賈敦臨撰。雞足山志十卷。範承勛撰。水經注集釋訂譌四十卷。沈炳巽撰。水經注釋四十卷,刊誤十二卷,附錄一卷。趙一清撰。水經注校三十卷,水地記一卷。戴震撰。水經注校正四十卷,補遺一卷,附錄一卷。全祖望撰。水經注釋地四十卷,水道直指一卷,補遺一卷。張匡學撰。水經釋地八卷。孔繼涵撰。水經注疏證四十卷。沈欽韓撰。水經注圖說殘稿四卷。董祐誠撰。水經注西南諸水考三卷。陳澧撰。水經注洛涇二水補一卷。謝鍾英撰。水經注圖二卷。汪士鐸撰。合校水經注四十卷,附錄二卷。王先謙撰。河源紀略三十六卷。乾隆四十七年,紀昀、陸錫熊等奉敕撰。今水經一卷。黃宗羲撰。水道提綱二十八卷。齊召南撰。江源記一卷。查拉吳麟撰。導江三議一卷。王柏心撰。長江圖說十二卷。黃翼升撰。淮流一勺二卷。範以煦撰。昆侖河源考一卷。萬斯同撰。黃河全圖五卷。吳大澂、倪文蔚同撰。中國黃河經緯度圖一卷。梅啟照撰。歷代黃河變遷圖考四卷。劉鶚撰。東西二漢水辨一卷。王士禛撰。漢水發源考一卷。王筠撰。直隸河渠志一卷。陳儀撰。二渠九河圖考一卷。孫彤撰。永定河志三十二卷。李逢亨撰。西域水道記五卷。徐松撰。關中水道記一卷。孫彤撰。蜀水考四卷。陳登龍撰。汴水說一卷。硃際虞撰。漳水圖經一卷。姚柬之撰。山東全河備考四卷。葉方恆撰。山東運河備覽十二卷。陸燿撰。揚州水道記四卷。劉文淇撰。太湖備考十六卷。金友理撰。新劉河志一卷,婁江志一卷。顧士梿撰。章水經流考一卷。硃際虞撰。浙江圖考三卷。阮元撰。洞庭湖志十四卷。萬年淳撰。兩河清匯八卷。薛鳳祚撰。河紀二卷。孫承澤撰。居濟一得八卷。張伯行撰。治河奏績書四卷。靳輔撰。畿輔水利輯覽一卷,水利營田圖說一卷,畿輔河道管見一卷,水利私議一卷。吳邦慶撰。河防芻議六卷。崔維雅撰。畿輔水利四案四卷,附錄一卷。潘錫恩撰。畿輔安瀾志十卷。王履泰撰。畿輔水利議一卷。林則徐撰。北河續記八卷。閻廷謨撰。行水金鑒一百七十五卷。傅澤洪撰。續行水金鑒一百五十六卷。黎世垿撰。五省溝洫圖說一卷。沈夢蘭撰。西北水利議一卷。許承宣撰。東南水利八卷。沈愷曾撰。明江南治水記一卷。陳士礦撰。三吳水利條議一卷。錢中諧撰。江蘇水利圖說二十一卷。陶澍撰。江蘇水利全案正編四十卷,附編十二卷。李慶雲撰。浙西水利備考八卷。王鳳生撰。西湖水利考一卷。吳農祥撰。蕭山水利書七卷。來鴻雯、張文瑞、張學懋同撰。湘湖水利志三卷。毛奇齡撰。海塘新志六卷,兩浙海塘通志二十卷。方觀承撰。海塘攬要十二卷。楊鑅撰。捍海塘志一卷。錢文瀚撰。海塘錄二十六卷。翟均廉撰。海道圖說十五卷。金約撰。

元沙克什河防通議二卷,王喜治河圖略一卷。以上乾隆時奉敕輯。

以上地理類山川河渠之屬

西域圖志五十二卷。乾隆二十一年,劉統勛等奉敕撰。籓部要略十八卷,西陲要略四卷,西域釋地一卷,西域行程記一卷,萬里行程記四卷。祁韻士撰。蒙古游牧記十六卷。張穆撰。漢西域圖考七卷。李光廷撰。西陲總統事略十二卷。松筠撰。西域聞見錄八卷。七十一撰。衛藏圖志五十卷。盛繩祖撰。西藏通考八卷。黃沛翹撰。康輶紀行十六卷。姚瑩撰。金川瑣記六卷。李心衡撰。西游記金山以東釋一卷。沈■J0撰。朔方備乘八十五卷。何秋濤撰。三州輯略九卷。和寧撰。蠻司合志十五卷。毛奇齡撰。楚南苗志六卷。段汝霖撰。苗防備覽二十二卷,三省邊防備覽十六卷。嚴如煜撰。苗蠻合志二卷。曹樹翹撰。楚峒志略一卷。吳省蘭撰。雲緬山川志一卷。李榮升撰。臺灣紀略一卷。林謙光撰。澎湖紀略十二卷。胡建偉撰。澳門記略二卷。印光任、張汝霖同撰。海防述略一卷。杜臻撰。海防備覽十卷。薛傳源撰。防海輯要十八卷,圖一卷。俞昌會撰。洋防輯要二十四卷。嚴如煜撰。

以上地理類邊防之屬

西湖志纂十二卷。乾隆十六年,梁詩正奉敕撰。歷代帝王宅京記二十卷。顧炎武撰。歷代陵寢備考五十卷,宗廟附考八卷。硃孔陽撰。帝陵圖說四卷。梁份撰。唐兩京城坊考五卷。徐松撰。宋東京考二十卷。周城撰。圓明園記一卷。黃凱鈞撰。南宋古跡考二卷。周春撰。北平古今記十卷,建康古今記十卷,營平二州地名記一卷,山東考古錄一卷,譎觚一卷。顧炎武撰。關中勝跡圖志三十二卷。畢沅撰。江城名跡二卷。陳宏緒撰。潞城考古錄二卷。劉錫信撰。兩浙防護錄不分卷。阮元撰。西湖志四十六卷。傅玉露撰。先聖廟林記一卷。屈大均撰。闕里廣志二十卷。宋際、李慶長同撰。闕里述聞十四卷。鄭曉如撰。倉聖廟志一卷。祝炳森撰。梅里志四卷。吳存禮撰。伍公廟志六卷。金志章撰。臥龍岡志二卷。羅景星撰。鸚鵡洲志四卷。胡鳳丹撰。蘭亭志一卷。王復禮撰。南嶽二賢祠志八卷。尹繼隆撰。濂溪志七卷。周誥撰。岳廟志略十卷。馮培撰。於忠肅公祠墓錄十二卷。丁丙撰。平山堂小志十二卷。程夢星撰。滄浪小志二卷。宋犖撰。竹垞小志五卷。阮元撰。白鹿書院志十九卷。毛德琦撰。鵝湖講舍匯編十二卷。鄭之僑撰。明道書院紀績四卷。章秉法撰。東林書院志二十二卷。高(桂)、高、高廷珍、高陛、許獻同撰。毓文書院志八卷。洪亮吉撰。學海堂志一卷。林伯桐撰。文瀾閣志二卷。孫樹禮等撰。

以上地理類古跡之屬

宸垣識略十六卷。吳長元撰。天府廣記四十四卷。孫承澤撰。金鰲退食筆記二卷,松亭行紀二卷,塞北小鈔一卷,東巡扈從日錄一卷,西巡扈從日錄二卷。高士奇撰。都門紀略四卷。楊靜亭撰。盛京疆域考六卷。楊同桂、孫宗瀚同撰。遼載前集二卷。林本裕撰。吉林外紀十卷。薩英額撰。黑龍江外紀四卷。西清撰。龍江述略六卷。徐宗亮撰。龍沙紀略一卷。方式濟撰。寧古塔紀略一卷。吳桭臣撰。柳邊紀略五卷。楊賓撰。封長白山記一卷。方象瑛撰。畿輔地名考三卷。王灝撰。顏山雜記四卷。孫廷銓撰。津門雜記三卷。張燾撰。江南星野辨一卷。葉燮撰。三吳採風類記十卷。張大純撰。百城煙水九卷。徐崧、張大純同撰。白下瑣言十卷。甘熙撰。清嘉錄十二卷。顧祿撰。具區志十六卷。翁澍撰。林屋民風十二卷。王維德撰。廣陵通典三十卷。汪中撰。廣陵事略七卷。姚文田撰。揚州畫舫錄十八卷。李鬥撰。邗記六卷,北湖小志五卷。焦循撰。淮壖小記六卷。範以煦撰。桃溪客語五卷。吳騫撰。太倉風俗記一卷。程穆衡撰。雲間第宅志一卷。王澐撰。皖省志略四卷。硃雲錦撰。皖游紀略二卷。陳克劬撰。姑孰備考八卷。夏之符撰。杏花村志十二卷。郎遂撰。二樓小志四卷。程元愈撰,汪越、沈廷璐補。潯陽蹠醢六卷。文行遠撰。東鄉風土記一卷,鵝湖書田志四卷。吳嵩梁撰。浙江通志圖說一卷。沈德潛撰。杭志三詰三誤辨一卷。毛奇齡撰。武林志餘三十二卷。張暘撰。西湖夢尋五卷。張岱撰。西湖覽勝志十四卷。夏基撰。增修雲林寺志八卷。厲鶚撰。武林第宅考一卷。柯汝霖撰。東城雜記四卷。厲鶚撰。北隅掌錄二卷。黃士珣撰。清波小志二卷。徐逢吉撰。南湖紀略槁六卷。邱峻撰。龍井見聞錄六卷。汪志鋗撰。定鄉小志十六卷。張道撰。湖壖雜記一卷,北墅瑣言一卷。陸次云撰。唐棲景物略二卷。張半菴撰。乍浦九山補志十二卷。李確撰。峽石山水志一卷。蔣宏任撰。濮川所聞錄六卷。金淮、濮璜同撰。海昌外志不分卷。談遷撰。石柱記箋釋五卷。鄭元慶撰。四明談助十六卷。徐兆昺撰。越中觀感錄一卷。陳錦撰。蕭山縣志刊誤三卷。毛奇齡撰。偁陽雜錄一卷。章大來撰。甌江逸志一卷。勞大與撰。江心志十二卷。釋元奇撰。閩越巡視紀略六卷。杜臻撰。閩小紀四卷。周亮工撰。續閩小紀一卷。黎定國撰。臺海使槎錄八卷。黃叔璥撰。東槎紀略五卷。姚瑩撰。中州雜俎三十五卷。汪價撰。鄢署雜鈔十二卷。汪為熹撰。光緒湖北輿地記二十四卷。不著撰人氏名。漢口叢談六卷。範鍇撰。監利風土記一卷。王柏心撰。湖南方物志八卷。黃本驤撰。浯溪考二卷。王士禛撰。澧志舉要三卷,補一卷。潘相撰。海岱史略一百三十卷。王馭超撰。濟寧圖記二卷。王元啟撰。海岱日記一卷。張榕端撰。雲中紀程二卷。高懋功撰。高平物產記二卷。鄒漢勛撰。河套志六卷。陳履中撰。延綏鎮志六卷。譚吉璁撰。陜西南山谷口考一卷。毛鳳梧撰。三省山內風土雜記一卷。嚴如熤撰。新疆大記六卷。闞鳳樓撰。伊犁日記二卷,天山客話二卷。洪亮吉撰。荷戈紀程一卷。林則徐撰。輪臺雜記二卷。史善長撰。蜀徼紀聞四卷,隴蜀餘聞一卷。王士禛撰。蜀典十二卷。張澍撰。蜀都碎事六卷。陳祥裔撰。錦江脞記十二卷。戴璐撰。廣東新語二十六卷。屈大均撰。羊城古鈔八卷。仇巨川撰。廣州游覽志一卷。王士禛撰。嶺南雜記二卷。吳方震撰。韓江聞見錄十卷。鄭昌時撰。南粵筆記十六卷。李調元撰。嶺海見聞四卷。錢以塏撰。粵行紀事三卷。瞿昌文撰。嶺南風物記一卷。吳綺撰。連陽八排風土記八卷。李來章撰。惠陽山水紀勝四卷。吳騫撰。星餘筆記一卷。王鉞撰。粵西偶記一卷。陸祚蕃撰。桂游日記三卷。張維屏撰。滇系四十卷。師範撰。雲南備徵志二十一卷。王崧撰。滇南雜志二十四卷。曹樹翹撰。滇海虞衡志十三卷。檀萃撰。洱海叢談一卷。釋同揆撰。滇黔土司婚禮記一卷。陳鼎撰。黔書二卷。田雯撰。續黔書八卷。張澍撰。黔記四卷。李宗昉撰。黔話二卷。吳振棫撰。黔軺紀程一卷。黎培敬撰。淮西見聞記一卷。俞慶遠撰。

唐劉恂嶺表錄異三卷,元訥新河朔訪古記二卷。以上乾隆時奉敕輯。

以上地理類雜志之屬

海國聞見錄二卷。陳倫炯撰。坤輿圖志二卷。西洋南懷仁撰。異域錄一卷。圖理琛撰。八紘譯史四卷,紀餘四卷,八紘荒史一卷。陸次云撰。海錄二卷。楊炳南撰。瀛寰志略十卷。徐繼畬撰。海國圖志一百卷。魏源撰。朝鮮史略六卷。不著撰人氏名。朝鮮載記備編二卷,朝鮮史表一卷。周家祿撰。奉使朝鮮日記一卷。柏葰撰。朝鮮箕田考一卷。韓百謙撰。越史略三卷。不著人氏名。海外紀事六卷。釋大汕撰。安南史事記一卷。李仙根撰。安南紀游一卷。潘鼎珪撰。越南世系沿革略一卷,越南山川略一卷,中外交界各隘卡略一卷。徐延旭撰。中山沿革志二卷。汪楫撰。中山傳信錄六卷。徐葆光撰。琉球志略十五卷。周煌撰。續琉球志略五卷。費錫章撰。中山見聞辨異二卷。黃景福撰。記琉球入學始末一卷。王士禛撰。琉球入學見聞錄四卷。潘相撰。琉球朝貢考一卷。王韜撰。緬述一卷。彭崧毓撰。緬事述聞一卷。師範撰。緬甸瑣記一卷。傅顯撰。征緬紀聞一卷。王昶撰。從征緬甸日記一卷。周裕撰。滇緬邊界紀略一卷。不著撰人氏名。暹邏考略一卷。龔柴撰。暹邏別記一卷。李麟光撰。游歷日本圖經三十卷。傅云龍撰。日本國志四十卷。黃遵憲撰。日本新政考二卷。顧厚焜撰。東槎聞見錄四卷。陳家麟撰。使東雜記一卷。何如璋撰。東游叢錄四卷。吳汝綸撰。使俄羅斯行程錄一卷。張鵬翮撰。綏服紀略一卷。松筠撰。俄羅斯國紀要一卷。林則徐撰。俄游匯編十二卷。繆祐孫撰。俄羅斯疆界碑記一卷。徐元文撰。吉林勘界記一卷。吳大澂撰。中俄交界圖不分卷。洪鈞撰。西北邊界俄文譯漢圖例言一卷,帕米爾圖說一卷。許景澄撰。東三省韓俄交界表一卷。聶士成撰。使俄草八卷。王之春撰。英吉利考略一卷。汪文臺撰。英政概一卷,英籓政概四卷,法政概一卷。劉錫彤撰。法國志略二十四卷。王韜撰。英法德俄四國志略四卷。沈敦和撰。美利加圖經三十二卷。傅云龍撰。初使泰西記一卷。宜厚撰。乘槎筆記一卷。斌春撰。使西紀程一卷。郭嵩燾撰。奉使英倫記一卷。黎庶昌撰。英軺私記一卷。劉錫鴻撰。西軺紀略四卷。劉瑞芬撰。出使英法日記二卷。曾紀澤撰。出使英法義比日記六卷,續十卷。薛福成撰。出使美日秘日記十六卷。崔國因撰。使德日記一卷。李鳳苞撰。李傅相歷聘歐美記二卷。蔡爾康編。三洲日記八卷。張廕桓撰。游歷巴西圖經十卷,游歷圖經餘記十五卷。傅云龍撰。使美紀略一卷。陳蘭彬撰。四述奇十六卷。張德彞撰。環游地球新錄四卷。李圭撰。西史綱目二十卷。周維翰撰。邊事匯鈔十二卷,續鈔七卷。硃克敬撰。

宋趙汝適諸蕃志二卷。乾隆時奉敕輯。

以上地理類外志之屬

職官類

詞林典故八卷。乾隆九年,鄂爾泰等奉敕撰。皇朝詞林典故六十四卷。嘉慶十年,硃珪等奉敕撰。國子監志六十二卷。乾隆四十三年,梁國治等奉敕撰。歷代職官表六十三卷。乾隆四十五年敕撰。刑部則例二卷。康熙十八年敕撰。工部則例五十卷。乾隆十四年,史貽直等奉敕撰。工部續增則例九十五卷。乾隆二十四年,史貽直奉敕撰。吏部則例六十六卷。乾隆三十七年,傅恆等奉敕撰。戶部則例一百二十卷。乾隆四十一年,於敏中等奉敕撰。戶部則例一百卷。同治十二年,潘祖廕等奉敕撰。禮部則例一百九十四卷。乾隆四十九年,德保等奉敕撰。兵部處分則例三十九卷。道光五年,明亮等奉敕撰。金吾事例十卷。咸豐三年,步軍統領衙門奉敕撰。內務府則例四卷。光緒十年,福錕等奉敕撰。宗人府則例二十卷。光緒十四年,世鐸等奉敕撰。理籓院則例六十四卷。光緒十七年,松森等奉敕撰。光祿寺則例九十卷。官本。古官制考一卷。王寶仁撰。歷代官制考略二卷。葉枟撰。漢官答問五卷。陳樹鏞撰。漢州郡縣吏制考一卷。強汝詢撰。唐折沖府考四卷。勞經撰。樞垣紀略十六卷。梁章鉅撰。重修樞垣紀略二十六卷。硃智等撰。中書典故考八卷。王正功撰。槐載筆二十卷,清祕述聞十六卷。法式善撰。清祕述聞續十六卷。王家相撰。國朝翰詹源流編年二卷,館選爵里謚法考二卷。吳鼎雯撰。南臺舊聞十六卷。黃叔璥撰。春曹儀注一卷。王士禛撰。南省公餘錄八卷。梁章鉅撰。

宋程俱麟臺故事五卷,陳騤南宋館閣錄十卷,不知撰人續錄十卷。以上乾隆時敕輯。

以上職官類官制之屬

人臣儆心錄一卷。順治十二年,世祖御撰。朋黨論一卷。雍正三年,世宗御撰。訓飭州縣條規二十卷。雍正八年敕撰。政學錄五卷。鄭端撰。為政第一編八卷。孫鋐撰。百僚金鑒十二卷。牛天宿撰。臣鑒錄二十卷。蔣伊撰。大臣法則八卷。謝文洊撰。學仕遺規八卷,在官法戒錄四卷。陳宏謀撰。居官日省錄六卷。烏爾通阿撰。居官寡過錄六卷。李元春撰。臨民金鏡錄一卷。趙殿成撰。從政餘談一卷。王定柱撰。學治臆說二卷,續說一卷,說贅一卷。汪輝祖撰。庸吏庸言二卷,庸吏餘談二卷,蜀僚問答一卷。劉衡撰。牧令書二十三卷。徐棟撰。勸諭牧令文一卷。黃輔辰撰。勸戒淺語一卷。曾國籓撰。吏治輯要一卷。倭仁撰。福惠全書三十二卷。黃六鴻撰。道齊正軌二十卷。鄒鳴鶴撰。圖民錄四卷。袁守定撰。富教初桄錄二卷。宗源瀚撰。宦海慈航一卷。周埴撰。

不著撰人州縣提綱四卷。乾隆時敕輯。

以上職官類官箴之屬

政書類

大清會典二百五十卷。起崇德元年迄康熙二十五年,聖祖敕撰。自康熙二十六年至雍正五年,世宗敕撰,雍正十年刊。大清會典一百卷,會典則例一百八十卷,乾隆二十六年,履親王允祹奉敕撰。大清會典八十卷,圖一百三十二卷,事例九百二十卷。嘉慶二十三年敕撰。大清會典一百卷,圖二百七十卷,事例一千二百二十卷。光緒二十五年敕撰。續通典一百四十四卷。乾隆三十二年敕撰。續文獻通考二百五十二卷。乾隆十二年敕撰。皇朝通典一百卷。乾隆三十二年敕撰。皇朝通志二百卷。乾隆三十二年敕撰。皇朝文獻通考二百六十六卷。乾隆十二年敕撰。元朝典故編年考十卷。孫承澤撰。明會要八十卷。紀文彬撰。

宋李攸宋朝事實二十卷。乾隆時敕輯。

以上政書類通制之屬

幸魯盛典四十卷。康熙二十三年,孔毓圻編。萬壽盛典一百二十卷。康熙五十二年,王原祁等編。南巡盛典一百二十卷。乾隆三十一年,高晉等編。八旬萬壽盛典一百二十卷。乾隆五十四年,阿桂等編。西巡盛典二十四卷。嘉慶十六年,董誥等編。大清通典四十卷。乾隆元年敕撰。皇朝禮器圖式二十八卷。乾隆二十四年敕撰。滿洲祭神祭天典禮六卷。乾隆四十二年敕撰。國朝宮史三十六卷。乾隆七年敕撰。宮史續編一百卷。嘉慶六年敕撰。大清通禮五十四卷。道光四年敕撰。廟制圖考一卷。萬斯同撰。壇廟祀典三卷。方觀承撰。壇廟樂章一卷。張樂盛撰。萬壽衢歌樂章六卷。彭元瑞撰。北郊配位議一卷,辨定嘉靖大禮議二卷。毛奇齡撰。北岳恆山歷祀上曲陽考一卷。劉師峻撰。盛京典制備考八卷。崇厚撰。滿洲四禮考四卷。索寧安撰。太常紀要十五卷,四譯館考十五卷。江蘩撰。學典三十卷。孫承澤撰。國學禮樂錄二十四卷。李周望、謝履忠同撰。泮宮禮樂全書十六卷。張安茂撰。聖門禮樂統二十四卷。張行言撰。文廟祀典考五十卷。龐鍾璐撰。直省釋奠禮樂記六卷。應寶時撰。醴陵縣文廟丁祭譜四卷。藍錫瑞撰。文廟從祀先賢先儒考一卷。郎廷極撰。孔廟從祀末議一卷。閻若璩撰。家塾祀典一卷。應手為謙撰。大清通禮品官士庶儀纂六卷。劉師陸撰。吾學錄初編二十四卷。吳榮光撰。國朝謚法考一卷。王士禛撰。皇朝大臣謚法錄四卷。邵晉涵撰。皇朝謚法考五卷。鮑康撰。

漢衛宏漢官舊儀一卷,補遺一卷,不著撰人廟學典禮四卷。以上乾隆時敕輯。

以上政書類典禮之屬

學政全書八十卷。乾隆三十九年,素爾納等奉敕撰。磨勘簡明條例二卷,續二卷。乾隆時奉敕撰。科場條例六十卷。光緒十四年奉敕撰。奏定學堂章程不分卷。光緒二十九年,管學大臣奉敕撰。吏部銓選則例十七卷。嘉慶十年敕撰。吏部處分則例五十二卷,驗封司則例六卷,稽勛司則例八卷。道光十年敕撰,光緒十三年重修。歷代銓選志一卷。袁定遠撰。銓政論略一卷。蔡方炳撰。登科記考三十卷。徐松撰。國朝貢舉年表三卷。陳國霖、顧錫中同撰。國朝貢舉考略三卷。黃崇簡撰。歷科典試題名錄一卷,考官試題錄四卷。黃崇簡、饒玉成同撰。國朝鼎甲考一卷,狀元事考一卷。饒玉成撰。制義科瑣記四卷,續記一卷,淡墨錄十六卷。李調元撰。國朝右文掌錄一卷。宗源瀚撰。制科雜錄一卷。毛奇齡撰。匯征錄一卷。不著撰人氏名。歷代武舉考一卷。譚吉璁撰。

以上政書類銓選科舉之屬

賦役全書一百卷。順治間敕撰。孚惠全書六十四卷。乾隆六十年,彭元瑞奉敕撰。辛酉工賑紀事三十八卷。嘉慶六年敕撰。戶部漕運全書九十六卷。光緒二年敕撰。官田始末考一卷。顧炎武撰。蘇松歷代財賦考一卷。不著撰人氏名。杭州府賦役全書一卷。不著撰人氏名。浙江減賦全案十卷。楊昌濬編。大元海運記二卷。胡敬撰。明漕運志一卷。曹溶撰。丁漕指掌十卷。王大經撰。海運芻言一卷。施彥士撰。江蘇海運全案十二卷。賀長齡撰。浙江海運全案十卷。黃宗漢編。江北運程四十卷。董恂編。錢幣芻言一卷。王鎏撰。泉刀匯纂不分卷。邱峻撰。錢錄十二卷。張端本撰。大錢圖錄一卷。鮑康撰。長蘆鹽法志二十卷,附編十卷。珠隆阿修。河東鹽法志十卷。石麟等修。山東鹽法志二十四卷,附編十卷。崇福等修。山東鹽法續增備考六卷。王定柱編。兩淮鹽法志四十卷。吉慶修。兩淮鹽法志五十六卷。佶山修。兩淮鹽法志一百二十卷。劉坤一修。淮南鹽法紀略十卷。龐際雲撰。淮鹽備要十卷。李澄撰。淮鹽問答一卷。周濟撰。淮南調劑志略四卷。不著撰人氏名。淮北票鹽續略十二卷。許寶書撰。兩浙鹽法續纂備考十二卷。楊昌濬編。兩廣鹽法志三十五卷。阮元等修。粵鹽蠡測編一卷。陳銓撰。鹽法議略一卷。王守基撰。歷代征稅紀一卷。彭寧和撰。續纂淮關統志十四卷。元成編。北新關志十六卷。許夢閎撰。粵海關志三十卷。豫堃編。荒政叢書十卷,附錄二卷。俞森撰。救荒備覽四卷。勞潼撰。荒政輯要十卷。汪志伊撰。康濟錄六卷。倪國璉撰。籌濟編三十二卷。楊景仁撰。捕蝗考一卷。陳芳生撰。捕蝗匯編一卷。陳僅撰。伐蛟說一卷。魏廷珍撰。畿輔義倉圖六卷。方觀承撰。左司筆記二十卷。吳璟撰。己庚編六卷。祁韻士撰。石渠餘紀六卷。王慶雲撰。光緒會計錄三卷。李希聖撰。

以上政書類邦計之屬

八旗通志初集二百五十卷。雍正五年,鄂爾泰奉敕撰。八旗通志三百五十四卷。乾隆三十七年,福隆安等奉敕撰。八旗則例十二卷。乾隆三十七年,福隆安等撰。軍器則例二十四卷。嘉慶十九年敕撰。綠營則例十六卷。官本。中樞政考三十二卷。嘉慶二十年,明亮等奉敕撰。中樞政考續纂七十二卷。道光九年,長齡等奉敕撰。杭州駐防八旗志略二十五卷。張大昌撰。荊州駐防八旗志十六卷。希元撰。駐粵八旗志二十四卷。長善撰。馬政志一卷。蔡方炳撰。保甲書四卷。徐棟撰。鄉守外編輯要十卷。許乃釗撰。

以上政書類軍政之屬

督捕則例二卷。乾隆二年,徐本等奉敕撰。大清律例四十七卷。乾隆五年,三泰等奉敕撰。大清律續纂條例總類二卷。乾隆二十五年敕撰。五軍道里表四卷。乾隆四十四年,福隆安等奉敕撰。三流道里表四卷。乾隆四十九年,阿桂等奉敕撰。刪除律例附商律不分卷。光緒三十一年,沈家本奉敕撰;商律,三十二年,商部奉敕撰。清現行刑律三十六卷,秋審條款一卷。光緒時,沈家本等奉敕撰。禁煙條例一卷。光緒時,善耆等奉敕撰。蒙古律例十二卷。官本。刑部奏定新章四卷。官本。刑部比照加減成案三十二卷。許梿、熊義同撰。刑案匯覽六十卷,卷首一卷,卷末一卷,拾遺備考一卷,續編十卷。祝慶祺撰。駁案新編三十九卷。全士潮等編。秋審比較匯案續編八卷。不著撰人氏名。清律例歌括一卷。不著撰人氏名,丁承禧注。重修名法指掌圖四卷。徐灝撰。法曹事宜四卷。不著撰人氏名。

以上政書類法令之屬

乘輿儀仗做法二卷。乾隆十三年奏刊。工程做法七十四卷。雍正十二年,果親王允禮等撰。物料價值則例二百二十卷。乾隆三十三年,陳宏謀等奉敕撰。武英殿聚珍板程式一卷。乾隆三十八年,金簡等奉敕撰。內廷工程做法八卷,簡明做法無卷數。工部會同內務府撰。圓明園工部則例不分卷。不著撰人氏名。城垣做法冊式一卷。官本。工部軍器則例六十卷。嘉慶十六年,劉權之等奉敕撰。戰船則例內河五十八卷,外海四十卷。官本。重訂鐵路簡明章程一卷。光緒二十九年,商部撰。河工器具圖式四卷。麟慶撰。浮梁陶政志一卷。吳允嘉撰。築圩圖式一卷。孫峻撰。

以上政書類考工之屬

目錄類

天祿琳瑯書目十卷。乾隆四十年敕撰。天祿琳瑯書目後編二十卷。嘉慶二年敕撰。四庫全書總目提要二百卷。乾隆三十七年,紀昀等奉敕撰。簡明目錄二十卷。乾隆三十九年,紀昀等奉敕撰。抽毀書目一卷。官本。禁書目錄一卷。官本。違礙書目一卷。乾隆五十三年,官刻頒行。四庫全書考證一百卷。王太嶽、曹錫寶等撰。四庫簡明目錄標注二十卷。邵懿辰撰。四庫全書提要纂稿一卷。邵晉涵撰。四庫未收書目提要五卷。阮元撰。四庫闕書目一卷。徐松撰。國子監書目一卷。不著撰人氏名。徵刻唐宋人祕本書目三卷。黃虞稷、周在浚同編。傳是樓宋元板書目一卷。徐乾學撰。靜惕堂宋元人集書目一卷。曹溶撰。汲古閣珍藏祕本書目一卷。毛扆編。藝蕓書舍宋元本書目一卷。汪士鐘撰。古泉山館宋元板書序錄一卷。瞿中溶撰。滂喜齋宋元本書目一卷。潘祖廕撰。宋元舊本書經眼錄三卷,附錄一卷。莫友芝撰。宋元本行格表二卷。江標撰。崇文總目輯釋五卷,補佚一卷。錢東垣撰。通志堂經解目錄一卷。翁方綱撰。全上古三代秦漢三國六朝文編目百三卷。嚴可均撰。天一閣書目四卷。汪本撰。天一閣見在書目六卷。薛福成撰。絳雲樓書目一卷。錢謙益撰。述古堂藏書目四卷。錢曾撰。千頃堂書目三十二卷。黃虞稷撰。傳是樓書目八卷。徐乾學撰。培林堂書目二卷。徐秉義撰。含經堂書目四卷。徐元文撰。潛採堂書目四卷,曝書亭宋元人集目一卷。硃彞尊撰。青綸館藏書目錄三卷。宋筠撰。季滄葦藏書目一卷。季振宜撰。楝亭書目三卷。曹寅撰。孝慈堂書目不分卷。王聞遠撰。佳趣堂書目二卷。陸漻撰。百歲堂書目三卷。惠棟撰。小山堂藏書目二卷。趙一清撰。好古堂藏書目四卷。姚際恆撰。文瑞樓書目十二卷。金檀撰。塾南書庫目錄六卷。王昶撰。稽瑞樓書目一卷。陳揆撰。振綺堂書目六卷。汪諴撰。抱經樓書目一卷。盧沚撰。清吟閣書目四卷。瞿瑛撰。環碧山房書目一卷。汪輝祖撰。瞑琴山館藏書目四卷。範楷撰。別下齋書目一卷。蔣光堉撰。樂意軒書目四卷。吳成佐撰。石研齋書目四卷。秦恩復撰。竹崦盦傳鈔書目一卷。趙魏撰。孫氏祠堂書目內編四卷,外編三卷。孫星衍撰。績溪金紫胡氏所箸書目二卷。胡培系撰。鑒止水齋書目一卷。許宗彥撰。津逮樓書目十八卷。甘福撰。結一廬書目四卷。硃學勤撰。帶經堂書目五卷。陳徵芝撰。海源閣書目一卷。楊以增撰。持靜齋書目五卷。丁日昌撰。郘亭知見傳本書目十六卷。莫友芝撰。行素草堂目睹書目十卷。硃記榮撰。讀書敏求記四卷。錢曾撰。熏習錄二十卷。吳焯撰。廉石居藏書記二卷,平津館鑒賞記三卷,補遺一卷,續編一卷。孫星衍撰。士禮居藏書題跋記四卷,續錄一卷,百宋一廛錄一卷。黃丕烈撰。拜經樓藏書題跋記六卷。吳壽暘撰。愛日精廬藏書志三十六卷。張金吾撰。鐵琴銅劍樓藏書目二十四卷。瞿鏞撰。皕宋樓藏書志一百二十卷,續志四卷。陸心源撰。滂喜齋藏書記三卷。潘祖廕撰。善本書室藏書志四十卷,附錄一卷。丁丙撰。楹書偶錄五卷,續編四卷。楊紹和撰。經義考三百卷。硃彞尊撰。經義考補正十二卷。翁方綱撰。古今偽書考一卷。姚際恆撰。歷代載籍足徵錄一卷。莊述祖撰。知聖道齋讀書跋尾二卷。彭元瑞撰。校訂存疑十七卷。硃文藻撰。竹汀先生日記鈔三卷。何元錫編。經籍跋文一卷。陳鱣撰。經籍舉要一卷。龍翰臣撰。曝書雜記三卷,可讀書齋校書譜一卷。錢泰吉撰。群書答問二卷,補遺一卷。凌曙撰。書目答問七卷。張之洞撰。群書提要一卷,皇清經解提要一卷,皇清經解淵源錄一卷。沈豫撰。半氈齋題跋二卷。江籓撰。東湖叢記六卷。蔣光煦撰。開有益齋讀書志六卷,續一卷。硃緒曾撰。木居士書跋二卷。瞿中溶撰。鄭堂讀書日記不分卷。周中孚撰。儀顧堂題跋十六卷,續跋十六卷。陸心源撰。浙江採輯遺書總錄十一卷。三寶等撰。關右經籍考十一卷。邢澍撰。長河經籍考十卷。田雯撰。毗陵經籍志四卷。盧文弨撰。武林藏書錄三卷。丁申撰。日本訪書志十七卷。楊守敬撰。汲古閣題跋初集二卷,續一卷。毛鳳苞編。汲古閣校刻書目一卷,補遺一卷,刻板存亡考一卷。鄭德懋編。金山錢氏家刻書目十卷。錢培蓀編。勿菴歷算書目一卷。梅文鼎撰。嘉定錢氏藝文略三卷。錢師璟撰。廬江錢氏藝文略一卷。錢儀吉撰。流通古書約一卷。曹溶撰。藏書紀要一卷。孫慶增撰。百宋一廛賦一卷。顧廣圻撰。藏書紀事詩六卷。葉昌熾撰。靈隱書藏紀事一卷。潘衍桐撰。焦山藏書約一卷,書目一卷,續一卷。梁鼎芬撰。藝文待訪錄一卷。羅以智撰。國朝箸述未刊書目一卷。鄭文焯撰。國朝未刻遺書志略一卷。硃記榮編。

漢劉向七略別錄一卷。馬國翰輯。

金石類

西清古鑒四十卷。乾隆十四年,梁詩正等奉敕編。西清續鑒甲編二十卷,附錄一卷。乾隆五十八年敕編。校正淳化閣帖釋文十卷。乾隆三十四年,金簡奉敕編。積古齋藏器目一卷。阮元撰。清儀閣藏器目一卷。張廷濟撰。竹崦盦藏器目一卷。趙魏撰。嘉廕簃藏器目一卷。劉喜海撰。平安館藏器目一卷。葉志詵撰。雙虞壺館藏器目一卷。吳式芬撰。懷米山房藏器目一卷。曹載奎撰。簠齋藏器目一卷。陳介祺撰。木庵藏器目一卷。程振甲撰。梅花草盦藏器目一卷。丁彥忠撰。選青閣藏器目一卷。王錫棨撰。愛吾鼎齋藏器目一卷。李璋煜撰。石泉書屋藏器目一卷。李佐賢撰。兩罍軒藏器目一卷。吳雲撰。䦛齋藏器目一卷。吳大澂撰。天壤閣藏器目一卷。王懿榮撰。䦛齋集古錄二十六卷,恆軒吉金錄不分卷,度量權衡實驗說不分卷。吳大澂撰。匋齋吉金錄八卷,續二卷。端方撰。焦山鼎銘考一卷。翁方綱撰。周無專鼎銘考一卷。羅士琳撰。齊侯罍銘通釋二卷。陳慶鏞撰。盤亭小錄一卷。劉銘傳撰。京畿金石考二卷。孫星衍撰。畿輔金石記殘稿不分卷。沈濤撰。畿輔碑目二卷。樊彬撰。常山貞石志二十四卷。沈濤撰。趙州石刻錄一卷。陳鍾祥撰。江寧金石記八卷,待訪錄二卷。嚴觀撰。江左石刻文編四卷。韓崇撰。江寧金石待訪錄四卷。孫彤撰。吳郡金石目一卷。程祖慶撰。吳中金石記一卷。顧沅撰。徐州金石記一卷。方駿謨撰。崇川金石志一卷。馮雲鵬撰。安徽金石略十卷,涇川金石記一卷。趙紹祖撰。山左金石志二十四卷。畢沅、阮元同撰。山左訪碑錄十三卷。法偉堂撰。山左碑目四卷。段赤苓撰。山左南北朝石刻存目一卷。尹彭壽撰。至聖林廟碑目六卷。孔昭薰撰。孔林漢碑考一卷。顧仲清撰。濟州金石志八卷。徐宗幹撰。濟州學碑釋文一卷。張弨撰。濟南金石記四卷。馮云鵷撰。歷城金石考二卷。周永年撰。諸城金石略二卷。李文藻撰。益都金石記四卷。段赤苓撰。山右金石志一卷。夏寶晉撰。山右金石記八卷。楊篤撰。山右石刻叢編四十卷。胡聘之撰。中州金石記五卷。畢沅撰。中州金石考八卷。黃叔璥撰。中州金石目四卷,補遺一卷。姚晏撰。中州金石目錄八卷。楊鐸撰。安陽金石錄十三卷。武億、趙希璜同撰。河陽金石記三卷。馮敏昌撰。河內金石記二卷,補遺一卷。方履籛撰。嵩洛訪碑日記一卷。黃易撰。嵩陽石刻集記二卷。葉封撰。登封縣金石志一卷。洪亮吉撰。偃師金石記四卷,偃師金石遺文補錄二卷,郟縣金石志一卷,寶豐金石志五卷,魯山金石志三卷。武億撰。孟縣金石志三卷。馮敏昌撰。濬縣金石錄二卷。熊象階撰。關中金石記八卷。畢沅撰。雍州金石記十卷。硃楓撰。關中金石附記一卷。蔡汝霖撰。陜西得碑目二卷,長安獲古編二卷,補遺一卷。劉喜海撰。關中金石文字存佚考十二卷。毛鳳枝撰。唐昭陵石跡考五卷。林侗撰。昭陵六駿贊辨一卷。張弨撰。昭陵碑考十三卷。孫三錫撰。扶風金石錄二卷,郿縣金石遺文錄二卷,興平金石志一卷。張員撰。寶雞縣金石志一卷。鄧夢琴撰。武林金石刻記十卷。倪濤撰。武林金石記殘稿不分卷。丁敬撰。兩浙金石志十八卷,補遺一卷。阮元撰。吳興金石志十六卷。陸心源撰。墨妙亭碑目考二卷。張鑒撰。越中金石記十二卷。杜春生撰。東甌金石錄十二卷。戴咸弼撰。臺州金石錄十三卷,闕訪二卷。黃瑞撰。括蒼金石志十二卷,續四卷。李遇孫撰。括蒼金石志補遺四卷。鄒柏森撰。湖北金石存佚考二十二卷。陳詩撰。湖北金石詩二卷。嚴觀撰。永州金石略一卷。宗稷辰撰。三巴★H1古志不分卷。劉喜海撰。蜀碑補記十卷。李調元撰。粵東金石略十八卷。阮元撰。高要金石略四卷。彭泰來撰。粵西金石略十五卷。謝啟昆撰。粵西得碑記一卷。楊翰撰。滇南古金石錄一卷。阮福撰。和林金石錄一卷。李文田撰。高麗碑全文八卷。葉志詵撰。海東金石苑四卷,海東金石考存一卷。劉喜海撰。日本金石志二卷。傅云龍撰。兩漢金石記二十二卷。翁方綱撰。隋唐石刻拾遺二卷。黃本驥撰。南漢金石志二卷。吳蘭修撰。元刻偶存一卷。陸增祥撰。元碑存目一卷。黃本驥撰。寰宇訪碑錄十二卷。孫星衍、邢澍同撰。訪碑續錄一卷。嚴可均撰。訪碑後錄三卷。黃本驥撰。補寰宇訪碑錄五卷。趙之謙撰。攟古錄二十卷。吳式芬撰。天一閣碑目一卷,潛研堂金石文字目錄八卷。錢大昕撰。小蓬萊閣金石目一卷。黃易撰。平安館碑目八卷。葉志詵撰。玉雨堂碑目四冊。韓泰華撰。式訓堂碑目三卷。章壽康撰。求古錄一卷,金石文字記六卷。顧炎武撰。來齋金石考三卷。林侗撰。觀妙齋金石文考略十六卷。李光暎撰。金石續錄四卷。劉青藜撰。金石經眼錄一卷。褚峻摹圖,牛運震補說。金石錄補二十七卷,續七卷,金石小箋一卷。葉奕苞撰。金薤琳瑯補遺一卷。宋振譽撰。平津館讀碑記八卷,續記一卷,再續一卷,三續二卷。洪頤煊撰。潛研堂金石文字跋尾二十五卷。錢大昕撰。金石三跋十卷,金石文字續跋十四卷。武億撰。古泉山館金石文跋不分卷。瞿中溶撰。鐵橋金石跋四卷。嚴可均撰。古墨齋金石文跋六卷。趙紹祖撰。寶鐵齋金石跋尾三卷。韓崇撰。石經閣金石跋文一卷。馮登府撰。攀古小廬古器物銘釋文不分卷,碑跋不分卷。許瀚撰。清儀閣題跋一卷。張廷濟撰。枕經堂金石題跋三卷。方朔撰。求是齋金石跋尾四卷。丁紹基撰。宜祿堂金石記六卷。硃士端撰。簠齋金石文字考釋一卷,筆記一卷。陳介祺撰。開有益齋金石文字記一卷。硃緒曾撰。十二硯齋金石過眼錄十八卷。汪鋆撰。金石萃編一百六十卷。王昶撰。金石萃編補目三卷。黃本驥撰。金石續編二十一卷,目一卷。陸耀遹撰。金石萃編補略二卷。王言撰。金石萃編補正四卷。方履籛撰。八瓊室金石補正一百三十卷。陸增祥撰。匋齋藏石記四十四卷。端方撰。金石表一卷。曹溶撰。金石存十六卷。吳玉搢撰。金石索十二卷。馮雲鵬、馮云鵷撰。金石品二卷,金石存十五卷。李調元撰。金石契四卷。張燕昌撰。金石屑四卷。鮑昌熙撰。金石摘十卷。陳善墀撰。香南精舍金石契二卷。覺羅崇恩撰。金石遺文錄十卷。陳奕禧撰。金石文釋六卷。吳穎芳撰。古志石華三十卷。黃本驥撰。金石文鈔八卷。趙紹祖撰。碑錄二卷。硃文藻撰。績語堂碑錄不分卷。魏錫曾撰。金石圖二卷。褚峻摹圖,牛運震補說。求古精舍金石圖四卷。陳經撰。小蓬萊閣金石文字不分卷。黃易撰。古均閣寶刻錄一卷。許梿撰。平安館金石文字不分卷。葉名灃撰。隨軒金石文字八種無卷數。徐渭仁撰。二銘草堂金石聚十六卷。張得容撰。淇泉摹古錄一卷。趙希璜撰。漢碑篆額不分卷。何澂撰。紅崖碑釋文一卷。鄒漢勛撰。漢武梁祠畫象考證二卷。沈梧撰。漢射陽石門畫象匯考一卷。張寶德撰。華山碑考四卷。阮元撰。石門碑醳一卷,郙閣銘考一卷。王森文撰。天發神讖碑釋文一卷。周在浚撰。國山碑考一卷。吳騫撰。漢魏碑刻記存一卷。謝道承撰。北魏鄭文公碑考一卷。諸可寶撰。龍門造象釋文一卷。陸繼煇撰。瘞鶴銘辨一卷。張弨撰。瘞鶴銘考一卷。汪士鋐撰。瘞鶴銘考一卷。吳東發撰。瘞鶴銘考補一卷。翁方綱撰。山樵書外紀一卷。張開福撰。唐尚書省郎官石柱題名考二十六卷,唐御史臺精舍題名考三卷,附錄一卷。趙鉞、勞格同撰。楚州石柱考一卷。範以煦撰。邠州石室錄三卷。葉昌熾撰。石魚文字所見錄一卷。姚覲元撰。九曜石刻錄一卷。周中孚撰。蒼玉洞題名石刻一卷。劉喜海撰。翠微亭題名考一卷。釋達受撰。龍興寺經幢題跋一卷。羅矩撰。金天德鐘款識一卷。丁晏撰。鐵券銅塔考三卷。錢泳撰。岳廟彞器銘一卷。不著撰人氏名。分隸偶存二卷。萬經撰。碑文摘奇一卷。梁廷枬撰。碑別字五卷。羅振鋆撰。金石要例一卷。黃宗羲撰。碑版廣例十卷。王芑孫撰。志銘廣例二卷。梁玉繩撰。金石例補二卷。郭麟撰。金石綜例四卷。馮登府撰。金石訂例四卷。鮑振方撰。金石稱例五卷,續一卷。梁廷枬撰。漢石例六卷。劉寶楠撰。漢魏六朝墓銘纂例四卷。吳鎬撰。唐人志墓例一卷。徐朝弼撰。金石學錄四卷。李富孫撰。金石學錄補四卷。陸心源撰。金石札記四卷,袪偽一卷。陸增祥撰。語石六卷。葉昌熾撰。閒者軒帖考一卷。孫承澤撰。淳化祕閣法帖考正十二卷。王澍撰。淳化閣帖考證十卷。吳有蘭撰。淳化閣跋一卷。沈蘭先撰。淳化閣帖源流考一卷。周行仁撰。法帖釋文十卷。徐朝弼撰。南村帖考四卷。程文榮撰。鳴野山房帖目四卷。沈復粲撰。禊帖綜聞一卷。胡世安撰。蘇米齋蘭亭考八卷。翁方綱撰。定武蘭亭考一卷。王灝撰。鳳墅殘帖釋文二卷。錢大昕撰。惜抱軒法帖題跋三卷。姚鼐撰。蘇米齋題跋二卷。翁方綱撰。竹雲題跋四卷。王澍撰。鐵函齋書跋六卷。楊賓撰。芳堅館題跋四卷。郭尚先撰。錢錄十六卷。乾隆十六年敕撰。泉神志七卷。李世熊撰。泉志校誤四卷。金嘉採撰。錢志新編二十卷。張崇懿撰。琴趣軒泉譜一卷。黃灼撰。廣錢譜一卷。張延世撰。歷代古錢目一卷。硃煒撰。泉布統志九卷。孟麟撰。選青小箋十卷。許原愷撰。虞夏贖金釋文一卷。劉師陸撰。古今待問錄六卷。硃楓撰。吉金所見錄十六卷。初尚齡撰。古今錢略三十四卷。倪模撰。貨布文字考四卷。馬昂撰。泉寶所見錄十六卷。沈巍皆撰。歷代鍾官圖經八卷。陳萊孝撰。吉金志存四卷。李光廷撰。癖談六卷,附錄四卷。蔡云撰。運甓軒錢譜四十卷。呂佺孫撰。癖泉臆說六卷。高煥撰。古泉叢話三卷,藏泉記一卷。戴熙撰。觀古閣泉說一卷,叢稿二卷,續稿一卷,三編二卷。鮑康撰。論泉絕句二卷。劉喜海撰。古泉匯六十卷,續十四卷,補遺二卷。李佐賢撰。齊魯古印攗四卷,續一卷。高慶齡撰。集古官印考證七卷。瞿中溶撰。兩罍軒印考漫存九卷。吳雲撰。秦漢瓦當文字二卷,續一卷。程敦撰。浙江磚錄不分卷。馮登府撰。百磚考一卷。呂佺孫撰。千甓亭磚錄六卷,續四卷,古專圖釋二十卷。陸心源撰。匋齋藏專記二卷。端方撰。秋景庵主印譜四卷。黃易撰。龍泓山人印譜八卷。丁敬撰。訒葊集古印存三十二卷。汪啟淑撰。求是齋印譜四卷。陳豫鍾撰。吳讓之印存二卷。吳廷颺撰。楊聾石印存二卷。楊澥撰。選集漢印分韻二卷,續二卷。袁日省撰。楊嘯村印集二卷。楊大受撰。胡鼻山人印集二卷。胡震撰。觀自得齋印集十六卷。徐子靜撰。秦漢印選六卷。石潛撰。二金蜨堂印譜四卷。趙之謙撰。封泥考略十卷。吳式芬、陳介祺同撰。

宋不著撰人寶刻類編八卷。乾隆時敕輯。宋歐陽棐集古錄目五卷。黃本驥輯。

史評類

御批通鑒綱目五十九卷,通鑒綱目前編一卷,外紀一卷,舉要三卷,通鑒綱目續編二十七卷。康熙四十六年御撰。評鑒闡要十二卷。乾隆三十六年,劉統勛等奉敕編。古今儲貳金鑒六卷。乾隆四十六年敕撰。承華事略補圖六卷。元王惲撰,光緒時徐郙等奉敕補圖。史記評注十二卷。牛運震撰。史漢發明五卷。傅澤鴻撰。讀通鑒論三十卷。王夫之撰。宋論十五卷。王夫之撰。史論五答一卷。施國祁撰。明史評二卷。納蘭常安撰。明史十二論一卷。段玉裁撰。讀通鑒札記二十卷。章邦元撰。通鑒評語五卷。申涵煜撰。看鑒偶評四卷。尤侗撰。鑒語經世編二十七卷。魏裔介撰。唐鑒偶評四卷。周池撰。綱目通論一卷,歷代通論一卷。任兆麟撰。讀史雜記一卷,讀宋鑒論三卷。方宗誠撰。鑒評別錄六十卷。黃恩彤撰。閱史卻視四卷,續一卷。李恭撰。讀史管見一卷。湯斌撰。午亭史評二卷。陳廷敬撰。茗香堂史論四卷。彭孫貽撰。史見二卷。陳遇夫撰。史評一卷。謝濟世撰。四鑒十六卷。尹會一撰。中山史論二卷。郝浴撰。十七朝史論一得一卷。郭倫撰。石溪史話八卷。劉鳳起撰。史學提要箋釋五卷。楊錫祐撰。讀書任子自鏡錄二十二卷。胡季堂撰。史林測義三十八卷。計大受撰。讀史大略六十卷,附錄一卷。沙張白撰。味俊齋史義二卷。周濟撰。讀史筆記十二卷。吳烜撰。讀史提要錄十二卷。夏之蓉撰。史論五種十一卷。李祖陶撰。史說一卷。黃式三撰。史說略四卷。黃以周撰。讀史臆說五卷。楊琪光撰。史通二十卷。周悅讓撰。救文格論一卷。顧炎武撰。炳燭偶鈔一卷。陸錫熊撰。南江書錄一卷。邵晉涵撰。讀史劄記一卷。盧文弨撰。文史通義八卷,校讎通義三卷,文史通義補編一卷。章學誠撰。史通通釋二十卷。浦起龍撰。史通訓故補二十卷。黃叔琳撰。史通校正一卷。盧文弨撰。史通削繁四卷。紀昀撰。

宋曹彥約經幄管見四卷,李心傳舊聞證誤四卷。以上乾隆時奉敕輯。


\end{pinyinscope}