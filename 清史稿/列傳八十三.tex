\article{列傳八十三}

\begin{pinyinscope}
岳鍾琪季父超龍超龍子鍾璜鍾琪子濬策棱子成袞扎布車布登扎布

岳鍾琪,字東美,四川成都人。父升龍,初入伍,授永泰營千總。康熙十二年,吳三桂反,永泰營游擊許忠臣受三桂劄。升龍使詣提督張勇告變,密結兵民,執忠臣殺之。十四年,從西寧總兵王進寶克蘭州,先登被創,遷莊浪守備;從克臨洮,平關隴,加都督僉事銜。累擢天津總兵。三十五年,上親征噶爾丹,升龍將三百騎護糧。上命升龍及馬進良、白斌,副將以次有違令退怯者,得斬之乃聞。昭莫多之捷,授拖沙喇哈番,擢四川提督。初,西藏營官入駐打箭爐,上使勘界。四川巡撫於養志言營官司貿易,不與地方事。居數年,營官喋吧昌側集烈發兵據瀘河東諸堡,升龍以五百人防化林營。養志反劾升龍擅發兵,升龍亦訐養志。上使勘讞,養志坐斬,升龍亦奪官。喋吧昌側集烈擊殺明正土司蛇蠟喳吧,傷官兵,提督唐希順討之,上命升龍從軍。事定,希順以病解任,仍授升龍提督。四十九年,乞休。升龍本貫甘肅臨洮,以母年逾九十,乞入籍四川,許之。逾二年,卒。雍正四年,追謚敏肅。

鍾琪,初入貲為同知。從軍,請改武職,上命以游擊發四川,旋授松潘鎮中軍游擊。再遷四川永寧協副將。五十八年,準噶爾策妄阿喇布坦遣其將策凌敦多卜襲西藏,都統法喇督兵出打箭爐,撫定里塘、巴塘。檄鍾琪前驅,至里塘,第巴不受命,誅之。巴塘第巴懼,獻戶籍。乍丫、察木多、察哇諸番目皆順命。五十九年,定西將軍噶爾弼師自拉里入,仍令鍾琪前驅。鍾琪次察木多,選軍中通西藏語者三十人,更衣間行至洛隆宗,斬準噶爾使人,番眾驚,請降。噶爾弼至軍,用鍾琪策,招西藏公布,以二千人出降。鍾琪遂督兵渡江,直薄拉薩,大破西藏兵,擒喇嘛為內應者四百餘人。策凌敦多卜敗走,西藏平。六十年,師還,授左都督,擢四川提督,賜孔雀翎。命討郭羅克番部,鍾琪率師並督瓦斯、雜穀諸土司兵自松潘出邊。郭羅克番兵千餘出拒,鍾琪擊破之,取下郭羅克吉宜卡等二十一寨,殲其眾。乘夜督兵進至中郭羅克納務寨,番兵出拒,鍾琪奮擊,未終日,連克十九寨,斬三百餘級,獲其渠駿他爾唪索布六戈。復督兵進攻上郭羅克押六寨,番目旦增縛首惡假磕等二十二人以降。郭羅克三部悉定,予拜他喇布勒哈番世職。六十一年,討平羊峒番,於其地設南坪營。

雍正元年,師討青海,撫遠大將軍年羹堯請以鍾琪參贊軍事。鍾琪將六千人出歸德堡,撫定上寺東策卜、下寺東策卜諸番部。南川塞外郭密九部屢盜邊,而呈庫、活爾賈二部尤橫。鍾琪移師深入搗其巢,盡平之。二年,授奮威將軍,趣進兵。郭隆寺喇嘛應羅卜藏丹津為亂,鍾琪會諸軍合擊,殲其眾,毀寺,擒戮其渠達克瑪胡土克圖。羅卜藏丹津居額穆納布隆吉爾,其大酋阿爾布坦溫布、吹拉克諾木齊分屯諸隘,鍾琪與諸將分道入。鍾琪及侍衛達鼐出南路,總兵武正安出北路,黃喜林、宋可進出中路,副將王嵩、紀成斌搜山。師進至哈喇烏蘇,方黎明,番眾未起,即縱擊,斬千餘人,番眾驚走,逐之,一晝夜至伊克喀爾吉,獲阿爾布坦溫布。復進次席爾哈羅色,遣兵攻噶斯;復進次布爾哈屯,薄額穆納布隆吉爾,羅卜藏丹津西竄,鍾琪逐之,一晝夜馳三百里。其酋彭錯等來降,鍾琪令守備劉廷言監以前驅,鍾琪繼其後。其酋吹因來降,言羅卜藏丹津所在距師百五六十里。鍾琪令暫休,薄暮復進,黎明至其地。羅卜藏丹津之眾方散就水草,即縱擊,大破之,擒諸臺吉,並羅卜藏丹津母阿爾泰哈屯及女弟阿寶,羅卜藏丹津易婦人服以遁。廷言等亦得吹拉克諾木齊等。鍾琪復進至桑駝海,不見虜乃還。出師十五日,斬八萬餘級。大酋助羅卜藏丹津為亂者皆就擒。青海平,上授鍾琪三等公,賜黃帶。

莊浪邊外謝爾蘇部土番據桌子、棋子二山為亂,納硃公寺、朝天堂、加爾多寺諸番與相糾合。羹堯遣鍾琪等督兵分十一路進剿,凡五十餘日,悉討平之。命兼甘肅提督。三年,復命兼甘肅巡撫。四月,解羹堯兵柄,改授杭州將軍,命鍾琪亦上奮威將軍印,署川陜總督,盡護諸軍。河州、松潘舊為青海蒙古互市地,羹堯奏移於那喇薩喇。鍾琪奏言青海部長察罕丹津等部落居黃河東,請仍於河州、松潘互市。額爾德尼額爾克托克托鼐等部落居黃河西,請移市西寧塞外丹噶爾寺。蒙古生業,全資牲畜,請六月後不時交易。四川雜穀、金川、沃日諸土司爭界,羹堯令金川割美同等寨畀沃日,致仇殺不已。鍾琪奏請還金川,而以龍堡三歌地予沃日,上皆許之。

尋真除川陜總督。疏言:「土司承襲,文武吏往往索費,封其印數年不與,致番目專恣仇殺。請定限半年,仍令應襲者先行署理。土司有外支循謹能治事者,許土官詳督撫給職銜,分轄其地,多三之一,少五之一,使勢相維、情相安。」入覲,加兵部尚書銜。疏言:「察木多外魯隆宗察哇、坐爾剛、桑噶、吹宗、袞卓諸部,距打箭爐遠,不便遙制。請宣諭達賴喇嘛,令轄其地。中甸、里塘、巴塘及得爾格特、瓦舒霍耳諸地,並歸內地土司。」又言:「巴塘隸四川,中甸隸雲南,而巴塘所屬木咱爾、祁宗、拉普、維西諸地偪近中甸,總會於阿墩子,實中甸門戶。請改隸雲南,與四川里塘、打箭爐互為犄角。」下王大臣議,如所請。四年春,請選西安滿洲兵千人駐潼關。冬,請以陜、甘兩省丁銀攤入地畝徵收,自雍正五年始,著為定例。逾年,復疏言甘屬河東糧輕丁多,河西糧多丁少,請將二屬各自均派:河東丁隨糧辦,河西糧照丁攤。下部議行。四川烏蒙土知府祿萬鍾擾雲南東川,鎮雄土知府隴慶侯及建昌屬冕山、涼山諸苗助為亂。上命鍾琪與雲貴總督鄂爾泰會師討之。五年春,擒萬鍾,慶侯亦降。烏蒙、鎮雄皆改土歸流。冕山、涼山亦以次底定。

鍾琪督三省天下勁兵處,疑忌眾。成都訛言鍾琪將反,鍾琪疏聞,上諭曰:「數年以來,讒鍾琪者不止謗書一篋,甚且謂鍾琪為岳飛裔,欲報宋、金之仇。鍾琪懋著勛勞,朕故任以要地,付之重兵。川、陜軍民,受聖祖六十餘年厚澤,尊君親上,眾共聞知。今此造言之人,不但謗大臣,並誣川、陜軍民以大逆。命巡撫黃炳、提督黃廷桂嚴鞫。」尋奏湖廣人盧宗寄居四川,因私事造蜚語,無主使者,論斬。

六年,疏請以建昌屬河西、寧番兩土司及阿都、阿史、紐結、歪溪諸地改土歸流,河東宣慰司以其地之半改隸流官,升建昌為府,領三縣,並釐定營汛職制,及善後諸事。下部議,如所請。定新設府曰寧遠,縣曰西昌、冕寧、鹽源,又請改岷州兩土司歸流。尋分疏請升四川達州,陜西秦、階二縣為直隸州。七年,又分疏請升甘肅肅州為直隸州,陜西子午谷隘口增防守官兵,里塘、巴塘諸地,置宣撫、安撫諸司至千百戶,視流官例題補。俱議行。雷波土司為亂,遣兵討平之。

靖州諸生曾靜遣其徒張熙投書鍾琪,勸使反。鍾琪與設誓,具得靜始末,疏聞。上褒鍾琪忠,遣侍郎杭奕祿等至湖南逮鞫治,語詳杭奕祿傳。

羅卜藏丹津之敗也,走投準噶爾,其酋策妄阿喇布坦納之。策妄阿喇布坦死,子噶爾丹策零立,數侵掠喀爾喀諸部。上命傅爾丹為靖邊大將軍,屯阿爾泰山,出北路;鍾琪為寧遠大將軍,屯巴里坤,出西路:討之。加鍾琪少保,以四川提督紀成斌等參贊軍務。鍾琪率師至巴里坤,築東西二城備儲胥,簡卒伍為深入計。八年五月,召鍾琪及傅爾丹詣京師授方略,鍾琪請以成斌護大將軍印。科舍圖嶺者,界巴密、巴里坤間,鍾琪設牧廠於此。準噶爾聞鍾琪方入覲,乘間以二萬餘人入犯,盡驅駝馬去。成斌使副參領查廩以萬人護牧廠,寇至不能御,走過總兵曹勷壘呼救;勷以輕騎往赴,戰敗亦走。總兵樊廷及副將冶大雄等將二千人,轉戰七晝夜。總兵張元佐督所部夾擊,拔出兩卡倫官兵,還所掠駝馬強半。成斌欲罪查廩,既而釋之,以捷聞。上已遣鍾琪還鎮,上謂當於卡倫外築城駐兵,出游兵擊敵,俾不敢深入,令鍾琪詳議。尋諭獎廷、大雄、元佐功,賜金予世職,遣內務府總管鄂善齎銀十萬犒師。立祠安西,祀陣亡將士。上以酒三爵遙酹,亦俾鄂善齎往設祭。

九年春,鍾琪請移兵駐吐魯番、巴爾庫爾,為深入計。上諭曰:「鍾琪前既輕言長驅直入,又為敵盜駝馬,既恥且憤,必欲進剿,直搗巢穴,能必勝乎?」九年正月,鍾琪部兵有自敵中脫歸者,言噶爾丹策零將移駐哈喇沙爾,以大隊赴西路,而令其將小策零敦多卜犯北路。鍾琪以聞,並言敵將自吐魯番侵哈密,擾安西、肅州邊界。我軍眾寡莫敵,當持重堅壁固守,告北路遣兵應援,並調兵自無克克嶺三面夾擊。上諭曰:「前以鍾琪軍寡,諭令持重堅守,今已有二萬九千人。樊廷馬步二千,敵彼二萬,轉戰七晝夜,猶足相當。乃以二萬九千人而雲眾寡莫敵,何懦怯至此?且前欲直搗伊犁,豈有賊至數百里內轉堅壁而不出乎?賊果至巴爾庫爾,即敗逃,亦從科舍圖直走伊爾布爾和邵而遁。無克克嶺相去二三百里,安所得夾擊?鍾琪於地勢軍機,茫然不知,朕實為煩憂。」

三月,準噶爾二千餘犯吐魯番,成斌遣廷將四千人赴援,敵引退。四月,又以千餘人犯吐魯番,別以二百餘人犯陶賴卡倫。六月,又以二千餘人圍魯谷慶城。吐魯番回目額敏和卓等率所部奮擊,殺二百餘人。鍾琪議令元佐、勷及張存孝將三千人赴援。提督顏清如將二千人屯塔庫,成斌將四千人防陶賴,俟我軍進擊烏魯木齊,移回民入內地。上諭鍾琪:「今年秋間襲擊,是第一善策。援吐魯番,乃不得已之舉。若但籌畫應援,而不計及襲擊,是舍本而逐末也。」

魯谷慶城圍四十餘日不下,準噶爾移攻哈喇火州城,以梯登,回民擊殺三百餘人。元佐等兵將至,敵引退。七月,準噶爾大舉犯北路,傅爾丹之師大敗於和通腦兒,鍾琪請乘虛襲擊烏魯木齊。上諭鍾琪:「賊既得志於北路,今冬仍往西路,且增添賊眾,更多於侵犯北路,俱未可知。當先事圖維,臨時權變,勿貪功前進,勿坐失機宜。」並令略行襲擊,即撤兵回營。鍾琪自巴爾庫爾經伊爾布爾和邵至阿察河,遇敵,擊敗之。逐至厄爾穆河,敵踞山梁以距。鍾琪令元佐將步兵為右翼,成斌將馬兵為左翼,勷及總兵王緒級自中路上山,參將黃正信率精銳自北山攻敵後,諸軍奮進,奪所踞山梁,敵敗走。諜言烏魯木齊敵帳盡徙,乃引兵還。疏聞,上獎鍾琪進退遲速俱合機宜。

十二月,上追舉科舍圖之役,責成斌怠忽,降沙州副將。十年正月,鏡兒泉邏卒遇敵,殺其二,掠其一以去。鍾琪劾副將馬順,上並以鍾琪下部察議。俄,準噶爾三千餘人犯哈密,鍾琪令勷、成斌將五千人自回落兔大阪,總兵紀豹將二千人自科舍圖嶺,分道赴援。又令副將軍石云倬、常賚,鎮安將軍卓鼐分地設伏,待敵占天生圈山口,顏清如屯塔爾那沁,遣參將米彪、副將陳經綸分道御戰,敵引去。勷等將至二堡,遇準噶爾五千餘人,即縱兵奮戰一晝夜。敵登山,勷督兵圍山,力戰至午,敵潰遁。勷自二堡至柳拊泉,與經綸及副將焦景竑軍會,乘夜追剿。鍾琪使告云倬等,遣兵至無克克嶺待敵,疏聞,上獎慰之。鍾琪議城穆壘駐軍,並命乘勝興工。雲倬等至無克克嶺,鍾琪令速赴梯子泉阻敵歸路,卓鼐繼其後。雲倬遲發一日,敵自陶賴大阪西越向納庫山遁去。師至敵駐軍處,餘火猶未息,云倬又令毋追襲。鍾琪劾云倬僨事,奪官,逮京師治罪,以張廣泗代為副將軍。上諭曰:「岳鍾琪素諳軍旅,本非庸才,但以懷游移之見,致戰守乖宜。前車之鑒,非止一端。嗣後當痛自省惕,壹號令,示威信,朕猶深望之!」大學士鄂爾泰等劾鍾琪專制邊疆,智不能料敵,勇不能殲敵。降三等侯,削少保,仍留總督銜,護大將軍印。六月,鍾琪疏報移軍穆壘。尋召鍾琪還京師,以廣泗護印。廣泗劾鍾琪調兵籌餉、統馭將士,種種失宜。穆壘形如釜底,不可駐軍。議分駐科舍圖、烏蘭烏蘇諸地。上命還軍巴爾庫爾,盡奪鍾琪官爵,交兵部拘禁。

十一年,以查郎阿署大將軍,又論鍾琪驕蹇不法,且劾成斌、元佐疏防,上命斬成斌,元佐降調。又劾勷縱賊,上命斬勷。十二年,大學士等奏擬鍾琪斬決,上改監候。乾隆二年,釋歸。十三年,師征大金川,久無功。三月,高宗命起鍾琪,予總兵銜。至軍,即授四川提督,賜孔雀翎。時經略大學士訥親視師,而廣泗以四川總督主軍事。大金川酋莎羅奔居勒烏圍,其兄子郎卡居噶拉依。鍾琪至軍,訥親令攻黨壩。上以軍事諮鍾琪,鍾琪疏言:「黨壩為大金川門戶,碉卡嚴密,漢、土官兵止七千餘。臣商諸廣泗,請益兵三千,廣泗不應。廣泗專主自昔嶺、卡撒進攻。此二處中隔噶拉依,距勒烏圍尚百餘里。黨壩至勒烏圍僅五六十里,若破康八達,即直搗其巢。臣商諸廣泗,廣泗不謂然,而廣泗信用土舍良爾吉及漢奸王秋等,恐生他虞。」訥親亦劾廣泗老師糜餉,詔逮治;亦罷訥親大學士,傅恆代為經略。鍾琪奏請選精兵三萬五千,萬人出黨壩及瀘河,水陸並進;萬人自甲索攻馬牙岡、乃當兩溝,與黨壩軍合,直攻勒烏圍;卡撒留兵八千,俟克勒烏圍,前後夾攻噶拉依;黨壩留兵二千護糧,正地留兵千防瀘河,餘四千往來策應。期一年擒莎羅奔及郎卡。臣雖老,請肩斯任。」命傅恆籌議,傅恆用其策。

鍾琪自黨壩攻康八達山梁,大破賊。師進戰塔高山梁,復屢破賊。鍾琪初佐年羹堯定西藏,莎羅奔以土目從軍;及為總督,以羹堯所割金川屬寨還莎羅奔,且奏給印信、號紙,莎羅奔以是德鍾琪。師入,莎羅奔懼,遣使詣鍾琪乞降。鍾琪請於傅恆,以十三騎從入勒烏圍開諭。莎羅奔請奉約束,頂經立誓,次日,率郎卡從鍾琪乘皮船出詣軍前降。上諭獎鍾琪,加太子少保,復封三等公,賜號曰威信。入覲,命紫禁城騎馬,免西征追償銀七十餘萬,官其子沺、淓侍衛,賜詩褒之。尋命還鎮。十五年,西藏珠爾默特為亂,鍾琪出駐打箭爐,事旋定。十七年,雜穀土司蒼旺為亂,鍾琪遣兵討擒之。十九年,重慶民陳琨為亂,鍾琪力疾親往捕治,還,卒於資州,賜祭葬,謚襄勤。上以所封公爵不世襲,予一等輕車都尉,令其子瀞襲。

鍾琪沈毅多智略,御士卒嚴,而與同甘苦,人樂為用。世宗屢獎其忠誠,遂命專征。終清世,漢大臣拜大將軍,滿洲士卒隸麾下受節制,鍾琪一人而已。既廢復起,大金川之役,傅恆倚以成功。高宗御制懷舊詩,列五功臣中,稱為「三朝武臣巨擘」云。

超龍,升龍弟,初冒劉姓,名曰傑。入伍,屢遷建昌左營守備。引見,聖祖垂詢,乃復本姓名,超擢東川營游擊。以避鍾琪,改西寧左營。雍正二年,授河州協副將,剿定鐵布等寨亂番。又以避鍾琪,改張家口協。六年,遷天津總兵。八年,擢湖廣提督。烏蒙亂,超龍令總兵蘇大有率副將何勉、參將毋椿齡討平之。尋遣兵分防貴州界,上以深合機宜嘉之。十年,卒。

鍾璜,超龍子。雍正七年,以鍾琪奏赴西路軍效力,授藍翎侍衛,除鑾儀衛治儀正。乾隆初,擢四川威茂營參將。再遷總兵,歷建寧、南贛、開化、昭通諸鎮。擢廣西提督,鍾琪卒,代為四川提督。疏言:「松潘總兵例出塞化番,三年一度。番性多猜,調集守候,彼此互防,甚非所原。又見小道遠費鉅,託病不至,惟附近土司領賞,有名無實。請停止,以節勞費。」上從之。金川土舍郎卡侵革布什咱土司,革布什咱合九土司兵攻金川,相持數年未決,郎卡乞令罷兵。鍾璜率兵出塞,至拉必斯滿安營,召郎卡出,令還所侵地及所掠穆爾津岡諸土司番民。九土司之兵悉罷。旋卒,賜祭葬,謚莊恪。

濬,鍾琪子。以二品廕生授西安同知,擢口北道,再擢山東布政使。雍正六年,調山西,署山東巡撫。鍾琪出師,命濬送至肅州。八年,召鍾琪詣京師,命濬就省。乾隆元年,請免郯城、蘭山諸縣水沖地應徵丁米。尋調江西。三年,請免南昌府屬浮糧三萬七千餘兩,復疏請發帑修築豐城江堤,濬江關河口,議行社倉,皆允所請。兩江總督楊超曾劾濬與屬吏朋比納賄,坐奪官。六年,授光祿寺卿,出為福建按察使。再遷廣東巡撫,調雲南。兩廣總督陳大受劾濬誤舉糧道明福以婪贓敗,又採木修堤,任屬吏作弊,召還京師。十八年,授鴻臚寺少卿,轉通政使參議,卒。濬在巡撫任虧庫項,鍾琪請以公俸按年扣還,上特命免之。

策棱,博爾濟吉特氏,蒙古喀爾喀部人。元太祖十八世孫圖蒙肯,號班珠爾,興黃教,西藏達賴喇嘛賢之,號曰賽音諾顏。其第八子丹津生納木扎勒,納木扎勒生策棱。康熙三十一年,丹津妻格楚勒哈屯自塔密爾攜策棱及其弟恭格喇布坦來歸,聖祖授策棱三等阿達哈哈番,賜居京師,命入內廷教養。四十五年,尚聖祖女和碩純愨公主,授和碩額駙。尋賜貝子品級,詔攜所屬歸牧塔密爾。五十四年,命赴推河從軍,出北路防御策妄阿喇布坦。五十九年,師征準噶爾,策棱從振武將軍傅爾丹出布拉罕,至格爾額爾格,屢破準噶爾,獲其宰桑貝坤等百餘人,俘馘甚眾。戰烏蘭呼濟爾,焚敵糧。師還,道遇準噶爾援兵,復擊敗之,授扎薩克。

策棱生長漠外,從軍久,習知山川險易。憤喀爾喀為準噶爾凌藉,銳自磨厲,練猛士千,隸帳下為親兵。又以敵善馳突而喀爾喀無紀律節制,每游獵及止而駐軍,皆以兵法部勒之,居常欽欽如臨大敵。由是賽音諾顏一軍雄漠北。

雍正元年,世宗特詔封多羅郡王。二年,入覲,命偕同族親王丹津多爾濟駐阿爾泰,並授副將軍,詔策棱用正黃旗纛。五年,偕內大臣四格等赴楚庫河,與俄羅斯使薩瓦立石定界,事畢,陳兵鳴砲謝天,議罪當削爵,上命改罰俸。九年,從靖邊大將軍順承郡王錫保討噶爾丹策零,偵賊自和通呼爾哈諾爾窺圖壘、茂海、奎素諸界,偕翁牛特部貝子羅卜藏等分兵擊卻之。準噶爾諸酋有大策零敦多卜、小策零敦多卜,皆噶爾丹策零同族,最用事。噶爾丹策零遣大策零敦多卜將三萬人入掠喀爾喀,聞錫保駐察罕廋爾,振武將軍傅爾丹軍科布多,乃遣其將海倫曼濟等將六千人取道阿爾泰迤東,分擾克魯倫及鄂爾海喀喇烏蘇,留餘眾於蘇克阿勒達呼為聲援。策棱偕丹津多爾濟迎擊,至鄂登楚勒,遣臺吉巴海將六百人宵入敵營,誘之出追,伏兵突擊,斬其驍將,餘眾驚潰,大策零敦多卜及海倫曼濟等遁去。詔進封和碩親王,賜白金萬。尋授喀爾喀大扎薩克。

十年六月,噶爾丹策零遣小策零敦多卜將三萬人自奇蘭至額爾德畢喇色欽,策棱偕將軍塔爾岱青御於本博圖山。未至,準噶爾掠克爾森齊老,分兵襲塔密爾,掠策棱二子及牲畜以去。策棱不及援,侍郎綽爾鐸以轉餉至,語策棱曰:「王速率兵遏敵歸路,當大破敵。」策棱還軍馳擊,距敵二日程。初,招丹津多爾濟赴援,不至。準噶爾兵趨額爾德尼昭,八月,策棱率兵追敵,十餘戰,敵屢敗。小策零敦多卜據杭愛山麓,逼鄂爾坤河而陣;策棱令滿洲兵陣河南,而率萬人伏山側,蒙古諸軍陣河北,遂戰。敵見滿洲兵背水陣,兵甚弱,意輕之,越險進。滿洲兵卻走,準噶爾兵逐之,策棱伏起自山下,如風雨至,斬萬餘級,谷中尸為滿,獲牲畜、器械無算。小策零敦多卜以餘眾渡河,蒙古兵待其半渡擊之,多入水死,河流盡赤。錫保馳疏告捷,首表策棱功,上嘉悅,賜號超勇,錫黃帶。諭:「此次軍功非尋常勞績可比,隨征兵弁,著從優加倍議敘。」上以策棱牧地被寇,賚馬二千、牛千、羊五千、白金五萬,賑所屬失業者,並命城塔密爾,建第居之。十二月,進固倫額駙,時純愨公主已薨,追贈固倫長公主。

十一年,定邊大將軍平郡王福彭統軍駐烏里雅蘇臺,詔策棱佩定邊左副將軍印,進屯科布多,尋授盟長。十二年五月,召來京諮軍務。六月,移軍察罕廋爾。十三年,準噶爾乞和,請以哲爾格西喇呼魯蘇為喀爾喀游牧界,上諮策棱。策棱謂:「向者喀爾喀游牧尚未至哲爾格西喇呼魯蘇,此議可許。惟準噶爾游牧,必以阿爾泰山為界,空其中為甌脫。」準噶爾不從。乾隆元年,師還,命策棱將喀爾喀兵千五百人駐烏里雅蘇臺,分防鄂爾坤。上以策棱母居京師,策棱在軍久,不得朝夕定省,命送歸游牧,並賜白金五千治裝。二年,噶爾丹策零貽書策棱,稱為車臣汗,申前請。策棱以聞,上命策棱以己意為報書,書曰:「阿爾泰為天定邊界。爾父琿臺吉時,阿爾泰迤西初無厄魯特游牧。自滅噶爾丹,我來建城,駐兵其地,眾所共知。其不令爾游牧者,原欲以此為隙地,兩不相及,以息爭端。今臺吉反云難以讓給,試思阿爾泰為誰地,誰能讓給?爾誠遵上指定議,我必不為禍始,亦不復居科布多。又謂我等哨兵逼近阿爾泰,宜向內撤。哨兵乃聖祖時舊例,即定界,豈能不設?臺吉其思之!」冬,準噶爾使達什博爾濟奉表至,命策棱偕詣京師。

三年春,至京師。噶爾丹策零表請喀爾喀與準噶爾各照現在駐牧。上召達什博爾濟入見,諭曰:「蒙古游牧,冬夏隨時遷徙。必指定山河為界,彼此毋得逾越。」遣侍郎阿克敦等使準噶爾,與達什博爾濟偕往。冬,噶爾丹策零復使哈柳從阿克敦等奉表至,請循布延圖河,南以博爾濟昂吉勒圖、無克克嶺噶克察諸地為界,北以遜多爾庫奎、多爾多輝庫奎至哈爾奇喇博木、喀喇巴爾楚克諸地為界,準噶爾人不越阿爾泰山,蒙古居山前,亦止在扎卜堪諸地,兩不相接。並乞移托爾和、布延圖二卡倫入內地。上以所議準噶爾不越阿爾泰山定界已就範,惟移托爾和、布延圖二卡倫不可許。四年春,賜敕遣還。哈柳詣策棱,哈柳曰:「額駙游牧部屬在喀爾喀,何弗居彼?」策棱答曰:「我主居此,予惟隨主居。喀爾喀特予游牧耳!」哈柳又曰:「額駙有子在準噶爾,何不令來京?」答曰:「予蒙恩尚公主,公主所出乃予子,他子無與也。即爾送還,予必請於上誅之。」冬,噶爾丹策零使哈柳復奉表至,始定議準噶爾不過阿爾泰山梁,不復言徙卡倫事。自雍正間與準噶爾議界,策棱三詣京師,準噶爾憚其威重,卒如上指。上獎策棱忠,子陷準噶爾,不復以為念,乃用宗室親王例,封其子成袞扎布世子。五年,命勘定喀爾喀游牧,毋越扎布堪、齊克慎、哈薩克圖、庫克嶺諸地,與準噶爾各守定界。六年,上以策棱老,命移軍駐塔密爾。初,喀爾喀凡三部;及是,土謝圖汗十七旗滋息至三十八旗,乃分二十旗與策棱,為賽音諾顏部。以鄂爾昆河西北烏里雅蘇河為游牧,為三部屏蔽。自此喀爾喀為四部。十五年,病篤,上遣其次子車布登扎布還侍,使侍衛德山等往存問。尋卒,遺言請與純愨公主合葬。喪至京師,上親臨奠,命配享太廟,謚曰襄,禦制詩輓之。

子八,最著者長子成袞扎布,次子車布登扎布。

成袞扎布,初授一等臺吉。乾隆元年,封固山貝子。四年,封世子,賜杏黃轡。十五年,襲扎薩克親王兼盟長,授定邊左副將軍。十七年,入覲。十八年,杜爾伯特臺吉車凌等內附,成袞扎布遣兵赴烏里雅蘇臺防準噶爾追兵。準噶爾宰桑禡木特以二百人追入邊,上命毋縱使還。禡木特逸去,詔以責成袞扎布。十九年,命移軍烏里雅蘇臺。尋罷定邊左副將軍,命赴額爾齊斯督屯田。二十年,師定伊犁,屯田兵撤還,仍駐烏里雅蘇臺。二十一年,和托輝特青袞咱卜謀為亂,成袞扎布發其謀。八月,亂作,仍授定邊左副將軍,率師討之,賜三眼孔雀翎。十二月,獲青袞咱卜,賜杏黃帶。二十二年,輝特巴雅爾為亂,正月,授定邊將軍,率師赴巴里坤捕治。十二月,入覲,復授定邊左副將軍,駐烏里雅蘇臺。二十六年,以準噶爾及回部悉平,請展喀爾喀汛界,下軍機大臣議,以附近烏魯木齊四汛,令索倫、綠旗兵駐防;自蘇伯昂阿至烏拉克沁伯勒齊爾十一汛,令成袞扎布督理。二十八年,入覲。二十九年,以烏里雅蘇臺城圮,請築城,舊址外立木柵,內實以土,引水環之,報聞。三十六年,卒。

子七,獲青袞咱卜,封其第四子占楚布多爾濟為世子,代掌扎薩克。卒,命其長子輔國公額爾克沙喇代掌扎薩克。卒,命次子輔國公伊什扎卜楚代掌扎薩克。及成袞扎布卒,以第七子拉旺多爾濟襲扎薩克親王。拉旺多爾濟,尚高宗女固倫和靜公主,授固倫額駙。從征臨清、石峰堡有功。嘉慶八年閏二月,仁宗乘輿入順貞門,有陳德者伏門側突出,侍衛丹巴多爾濟御之,被三創,拉旺多爾濟捘其腕,乃獲而誅之,賜御用補褂,封其子巴彥濟爾噶勒輔國公。

車布登扎布,初授一等臺吉。額爾德尼昭之役,力戰被創,封輔國公,賜雙眼孔雀翎。十七年,成袞扎布請析所部授車布登扎布自為一旗,上允之,別授扎薩克。十九年,督兵剿撫烏梁海,獲準噶爾宰桑,賜貝子品級。二十年,師征伊犁,車布登扎布將三百騎自察罕呼濟爾疾馳至集賽,擒宰桑齊巴漢,偵達瓦齊所在,奪舟渡伊犁河,逐達瓦齊,封多羅貝勒。阿睦爾撒納謀以伊犁叛,車布登扎布首發其奸,密以告將軍班第。師還,命招降烏梁海部落,即以隸焉。二十一年,烏梁海酋郭勒卓輝譌言哈薩克汗阿布賚與阿睦爾撒納連合,上命率師討之。有宰桑固爾班和卓者,攜千餘戶赴烏梁海謀偕遁,車布登扎布麾兵捕治,殲其眾。遂進兵哈薩克界,會尚書阿里袞自伊什勒諾爾轉戰至汗扎爾會,斬獲無算,封多羅郡王。

成袞扎布討青袞咱卜,詔車布登扎布還烏里雅蘇臺為佐。二十二年,代成袞扎布署定邊左副將軍。尋命兆惠代成袞扎布為定邊將軍,而以車布登扎布為之副。二十三年正月,授定邊右副將軍,從兆惠出巴里坤,遣兵赴哈什崆格斯搜逸寇。尋命赴博囉塔拉,捕布庫察罕、哈薩克錫喇等。哈薩克部人擒布庫察罕,哈薩克錫喇及宰桑鄂哲特等走和落霍斯,車布登扎布督兵逐之,哈薩克錫喇度不得脫,悉★據高岡拒戰。部將以兵寡,請待其走擊之,車布登扎布持不可,麾兵急進,擒鄂哲特,哈薩克錫喇僅以身免,詔以其父超勇號賜之。鄂哲特械至京師,言車布登扎布身先士卒,所向無前,上益嘉嘆,賜金黃帶。

車布登扎布進次阿布勒噶爾,哈薩克縛布庫察罕以獻,因請赴阿克蘇與將軍兆惠會。上命還伊犁,進親王品級。尋以在軍久,令歸游牧休息。二十四年,令佐將軍兆惠進葉爾羌討霍集占,旋復命還伊犁。二十七年,使西藏。三十六年,代成袞扎布為定邊左副將軍,授盟長。以牟利被訐,罷左副將軍,擅請展牧界,削親王品級,命以郡王兼扎薩克世襲。四十七年,卒。子三丕勒多爾濟,襲。

論曰:世傳鍾琪長身赬面,隆準而駢脅。臨陣挾二銅鎚,重百餘斤,指麾嚴肅不可犯。軍西陲久,番部皆讋其名。其受莎羅奔降也,傅恆升幄坐,鍾琪戎服佩刀侍。莎羅奔出語人曰:「我曹仰岳公如天人,乃傅公儼然踞其上,天朝大人誠不可測也!」策棱白晰微髭,善用兵,所部多奇士。有脫克渾者,日行千里,登高張兩手,若雕鼓翼,詗敵,敵不之察。事定,策棱欲官之,辭,賚以千金,酌酒勞之。脫克渾請出侍姬舞,起而歌,慷慨,策棱大悅,即以姬及所乘馬賜之。載籍言名將,往往舉其狀貌及其軼事,使讀者慕焉。鍾琪忠而毅,策棱忠而勇,班諸衛、霍、郭、李之倫,毋謂古今人不相及也。


\end{pinyinscope}