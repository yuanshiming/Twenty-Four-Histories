\article{志二十七}

\begin{pinyinscope}
時憲八

凌犯視差新法上道光中,欽天監秋官正司廷棟所撰,較舊法加密,附著卷末,以備參考。

求用時

推諸曜之行度,皆以太陽為本;而太陽之實行,又以平行為根。其推步之法,總以每日子正為始,此言子正者,乃為平子正,即太陽平行之點臨於子正初刻之位也。今之推步時刻,雖以兩子正之實行為比例,而所得者亦皆平行所臨之點,則實行所臨之點,自有進退之殊。設太陽在最卑後實行大於平行,則太陽所臨之點必在平行之東,以時刻而言,乃為未及。若太陽過最高後實行小於平行,則太陽所臨之點必在平行之西,以時刻而言,乃為已過。故以應加之均數變時為應減之時差,應減之均數變時為應加之時差,此因太陽有平行實行之別,以生均數時差也。然太陽所行者黃道,時刻所據者赤道,因黃道與赤道斜交,則同升必有差度。如二分後赤道小於黃道,其差應減,在時刻為未及。二至後赤道大於黃道,其差應加,在時刻為已過。故以正弧三角形法求得黃赤升度差,變為時分,二分後為加,二至後為減,此因經度有黃道赤道之分,以生升度時差也。按本時之日行自行所生之二差,各加減於平時而得用時,由用時方可以推算他數,故交食亦必以推用時為首務,即日月食之第一求也。其法理圖說已載於考成前編,講解最詳,其圖分而為二,且均數時差圖系用小輪。至考成後編求均數改為橢圓法,其法理亦備悉於求均數篇內,然未言及時差。今依太陽實行所臨黃道之點,以均數之分取得黃道上平行點,即以平實二點依過二極、二至經圈作距等圈法,引於赤道,可使二差合為一圖。其太陽之經度所臨之時刻及二時差之加減,皆可按圖而稽矣。

如道光十二年壬辰三月初六日癸丑戌正二刻十一分,月與司怪第四星同黃道經度,是為凌犯時刻。本日太陽引數三宮三度五十五分,太陽黃道經度三宮十五度五十三分,求用時。如圖甲為北極,乙丙丁戊為赤道,乙甲丁為子午圈,乙為子正,丁為午正,己庚辛壬為黃道,丙甲戊為過二極二至經圈,己為冬至,辛為夏至,庚為春分,壬為秋分。子為太陽實行之點,當赤道於丑,則醜點即太陽實臨之用時。卯為太陽平行之點,而當赤道於辰。其卯子之分,即應加之均數一度五十五分四十五秒,試自卯子二點與丙甲戊過極至經圈平行作卯午、子未二線,即如距等圈,將太陽平行、實行之度皆引於赤道,則庚午必與庚卯等,庚未必與庚子等,其赤道之午未亦必與卯子均數等。變時得七分四十三秒,為赤道午未之分,即均數時差也。次用庚醜子正弧三角形求庚丑弧,此形有醜直角,有庚角黃赤交角二十三度二十九分,有庚子弧太陽距春分後黃道度十五度五十三分。乃以半徑為一率,庚角之餘弦為二率,庚子弧之正切為三率,求得四率為庚丑弧之正切,檢表得庚丑弧十四度三十七分三十六秒,為太陽距春分後赤道度。乃與庚子黃道弧相等之庚未弧相減,得醜未弧一度十五分二十四秒,為應減之黃赤升度差。變時得五分二秒,即升度時差也。蓋太陽平行卯點,距春分之庚卯弧與庚午弧等,則午點乃為平時,即今之凌犯時刻。而太陽實行子點,距春分之庚子與庚未弧等,則午未為平行與實行之差。如以太陽右旋而言之,為實行已過平行,然以隨天左旋而計之,為實行未及平行,是未點轉早於午點,故必減午未均數時差,乃得未點時刻,此太陽在黃道虛映於赤道之時刻也。然子點太陽實當赤道之醜,則醜未為黃道與赤道之差。若以經度東行而言之,為赤道未及黃道,茲以時刻西行而計之,為赤道已過黃道,是醜點復遲於未點,故必加丑未升度時差,方得醜點時刻,即太陽在黃道實當於赤道之時刻也。其兩時差既為一加一減,而所減者又大於應加之分,故先以兩時差相減,得醜午時分二分四十一秒,而為時差此因兩時差加減異號故相減,若同號則相加,所謂兩數通為一數也。又因減數大於加數,故仍從減,若加數大者則從加矣。乃減於午點凌犯時刻戌正二刻十一分,即得醜點戌正二刻八分十九秒,為凌犯用時也。

一率半徑

二率庚角餘弦

三率庚子弧正切

四率庚丑弧正切

圖形尚無資料

又設凌犯時刻醜正一刻,太陽引數三宮十三度二十九分,黃道實行三宮二十五度三十四分,求用時。如子為太陽實行之點,當赤道於丑,其醜點即所臨之用時。卯為太陽平行之點,當赤道於辰,其子卯為應加之均數一度五十二分二十五秒,亦自卯子二點與過極至經圈平行作卯丑、子未二距等圈,其平行卯點映於赤道,恰與實行當赤道之醜點合,是由平行所得之時刻,已合實行實臨赤道之用時,遇此可無庸求其時差也。然何以知之,蓋兩時差之數相等,必減盡無餘,即無時差之總數矣。今試按法求之,既作卯丑、子未二線,其庚醜與庚卯等,庚未與庚子等,則醜未必與卯子均數等,變時得七分三十秒,即赤道上應減之均數時差。次用庚醜子正弧三角形,求得庚丑弧赤道度,與庚子弧黃道度相等之庚未弧相減,得醜未弧,黃赤升度差恰與均數等。變時亦得七分三十秒,即赤道上應加之升度時差。其時差一為加、一為減,而兩數相等,乃減盡無餘,既無時差之總數,則其凌犯時刻即為用時可知矣。此法以醜點凌犯時刻減去均數時差,得未點實行虛映之時刻,而復加相等之升度時差,所得用時,固仍在丑點之位,蓋因太陽平行距春分後黃道度等於太陽實行距春分後赤道度故也。又如太陽正當本天之最卑或最高,乃無平行實行之差,自無均數時差,止加減升度時差一數。設太陽當本天最卑,又當子正,如太陽在黃道之子點,則庚乙與庚子等,以庚醜子正弧形求得丑乙黃赤升度差。變時減於乙點時刻,即得醜點用時,乃在乙點子正之前也。若太陽當本天最高,又當午正,如太陽在黃道之午點,則壬丁與壬午等,以壬寅午正弧形求得寅丁黃赤升度差,變時減於丁點時刻,即得寅點用時,乃在丁點午正之前也。

圖形尚無資料

又如太陽實行正當冬、夏至或正當春、秋分,此四點皆無黃道赤道之差,自無升度時差,止加減均數時差一數。設太陽實行六宮初度為正當夏至,在黃道之辛點,當赤道於戊,而平行卯點,當赤道於辰,自卯點與丙甲戊過極至經圈平行作卯午距等圈,則午點為凌犯時刻,其戊午與辛卯均數等,變時得均數時差。減於午點而得戊點,即用時也。

圖形尚無資料

求春分距午時分、黃平象限宮度及限距地高

推算太陰凌犯視差,固依後編求日食三差之法,而其為用不同。蓋日食之東西差為求視距弧,而南北差為求視緯,其視距弧、視緯則為求視相距及視行之用。緣太陰行於白道,是必以白平象限為準焉。若五星之距恆星、五星之互相距,皆以黃道同經度之時為相距時刻,而較黃緯南北相距之數為其上下之分也。至月距五星、月距恆星,亦皆以黃道經度相同之時為凌犯時刻,不更問白道經度,其於白平象限又何與焉?然其以東西差定視時之進退,以南北差判視緯之大小,以定視距之遠近者,其差皆黃道經緯之差,故必以黃平象限之宮度為準。黃平象限者,地平上黃道半周適中之點也。顧黃道與赤道斜交,地平上赤道半周適中之點,恆當子午圈,而地平上黃道半周適中之點,則時有更易。蓋黃極由負黃極圈每日隨天左旋,繞赤極一周,如黃極在赤極之南,則冬至當午正,其黃道斜升斜降;若黃極在赤極之北,則夏至當午正,其黃道正升正降,而黃平象限亦皆恰當子午圈;設黃極在赤極之西,則春分當午正,其黃道之勢斜倚,出自東北而入西南,黃平象限乃在午正之東;設黃極在赤極之東,則秋分當午正,其黃道出自東南而入西北,黃平象限乃在午正之西。是則黃道之向,隨時不同,故以黃道之逐度,推求黃平象限及限距地高以立表。

先設太陽正當春分點,黃道實行為三宮初度,求午正初刻黃平象限宮度及限距地高度分。如圖甲乙丙丁為子午圈,甲為天頂,丙丁為地平,乙為北極,乙丙為京師北極出地,高三十九度五十五分,戊己庚為赤道,交於地平之己點,其戊點當午正,為地平上赤道半周適中之點,戊丁為赤道距地高五十度五分,當戊己丁角,辛子壬為負黃極圈,子為黃極,乙子己丑為過極至經圈,戊丑庚為黃道,而交地平於寅點,庚為秋分,醜為冬至,戊為春分,即太陽之所在,臨於午正,乃無春分距午之時分。試自黃極子點出弧線過天頂作子甲卯黃道經圈,為本時黃平象限,其辰點為地平上黃道半周適中之點,而在正午之東,即黃平象限宮度也。辰寅卯角為黃道與地平相交之角,而當辰卯弧,即本時限距地高之度也。法用戊辰甲正弧三角形求戊辰、甲辰二弧,此形有辰直角,有戊甲弧赤道距天頂,與乙丙北極高度等。以赤道交子午圈之戊直角九十度內減己戊丑角黃赤交角二十三度二十九分,得寅戊丁角六十六度三十一,為黃道交子午圈角;亦名黃道赤經交角。與辰戊甲角為對角,其度等。乃以半徑為一率,戊角黃道赤經交角之餘弦為二率,戊甲弧赤道距天頂,亦即太陽距天頂其正切為三率,求得四率,為黃平象限距午之正切,檢表得十八度二十六分十四秒,為戊辰弧黃平象限距午正之黃道度。與戊點春分三宮相加,因黃平象限在午東,故加。得辰點三宮十八度二十六分十四秒,即本時黃平象限之經度也。又以半徑為一率,戊角黃道赤經交角之正弦為二率,戊甲弧太陽距天頂之正弦為三率,求得四率,為黃平象限距天頂之正弦,檢表得三十六度三分九秒,為甲辰弧黃平象限距天頂。與甲卯象限九十度相減,得辰卯弧五十三度五十六分五十一秒,即本時限距地高,而當辰寅卯角之度也。

一率半徑

二率戊角餘弦

三率戊甲弧正切

四率戊辰弧正切

一率半徑

二率戊角正弦

三率戊甲弧正弦

四率甲辰弧正弦

圖形尚無資料

又設太陽正當秋分點,黃道實行為九宮初度,求午正初刻春分距午時分並黃平象限及限距地高,即以秋分當於正午之戊,則庚未戊為黃道,交地平於寅,庚為春分,未為夏至,子乙未己為過極至經圈,亦自黃極子點出弧★過天頂,作子甲卯弧黃平象限,而地平上黃道適中之辰點,在正午之西。先以春分距午西之庚戊赤道半周變十二時為春分距午之時分,次仍用戊辰甲正弧三角形求戊辰、甲弧二弧,此形有辰直角,有戊甲赤道距天頂。以戊直角內減己戊未角黃赤交角,得辰戊甲角黃道赤經交角,亦六十六度三十一分,求得戊辰弧黃平象限距午正之黃道度,亦十八度二十六分十四秒。與戊點秋分九宮相減,因黃平象限在午西,故減。得辰點八宮十一度三十三分四十六秒,即本時黃平象限之經度。又求得甲辰弧8888與甲卯象限相減,得辰卯弧,亦為五十三度五十六分五十一秒,即本時限距地高,而當辰寅卯角之度也。

又設太陽距春分後三十度,黃道實行為四宮初度,求午正初刻黃平象限諸數。乃以黃道經度四宮初度當午正如辛點,即太陽之所在,辛壬癸為黃道,交地平於寅。丑為冬至,壬為春分,乙子丑為過極至經圈。仍自黃極子點過天頂甲點作子甲卯弧黃平象限,其黃道適中之辰點,在午正之東。求法先用辛戊壬正弧三角形求壬戊、辛戊二弧及壬辛戊角,此形有戊直角,有壬角黃赤交角,有壬辛太陽距春分後黃道弧三十度。乃以半徑為一率,黃赤交角之餘弦為二率,黃道弧之正切為三率,求得四率,為赤道弧之正切,檢表得二十七度五十四分一十秒,為壬戊弧赤道同升度,亦即本時春分距午後赤道度。變時得一時五十一分三十七秒,即本時春分距午時分。又以半徑為一率,黃赤交角之正弦為二率,黃道弧之正弦為三率,求得四率,為黃赤距度之正弦,檢表得十一度二十九分三十三秒,為辛戊弧太陽距赤道北緯度。又以黃道弧之餘弦為一率,黃赤交角之餘切為二率,半徑為三率,求得四率,為黃道交子午圈角之正切,檢表得六十九度二十二分五十一秒,為壬辛戊角黃道交子午圈角,即黃道赤經交角。次用辛辰甲正弧三角形求辛辰、甲辰二弧,此形有辰直角,有辛角,與壬辛戊角為對角,其度等。以甲戊弧赤道距天頂內減辛戊黃赤距度,得甲辛弧二十八度二十五分二十七秒,為本時太陽距天頂。乃以半徑為一率,辛角黃道赤經交角之餘弦為二率,甲辛弧太陽距天頂之正切為三率,求得四率,為黃平象限距午之正切,檢表得十度四十七分二十八秒,為辛辰弧黃平象限距午正之黃道度。與辛點四宮初度相加,因黃平象限在午東,故加。得辰點四宮十度四十七分二十八秒,即本時黃平象限之經度也。又以半徑為一率,辛角黃道赤經交角之正弦為二率,甲辛弧太陽距天頂之正弦為三率,求得四率,為黃平象限距天頂之正弦,檢表得二十六度二十七分二十秒,為甲辰弧黃平象限距天頂。與甲卯象限九十度相減,得辰卯弧六十三度三十二分四十秒,為本時限距地高,即當辰寅卯角之度也。

一率半徑

二率壬角餘弦

三率壬角弧正切

四率壬戊弧正切

一率半徑

二率壬角正弦

三率壬辛弧正弦

四率辛戊弧正弦

一率壬辛弧餘弦

二率壬角餘切

三率半徑

四率辛角正切

一率半徑

二率辛角餘弦

三率甲辛弧正切

四率辛辰弧正切

一率半徑

二率辛角正弦

三率甲辛弧正弦

四率甲辰弧正弦

又設太陽距秋分前三十度,黃道實行為八宮初度,求午正初刻黃平象限諸數。乃以辛點太陽實行當正午,其申點為秋分,而在午東,壬為春分,未為夏至,子乙未為過極至經圈,亦自黃極子點過天頂,作子甲卯弧本時黃平象限,而在午西。法用辛戊申正弧三角形,此形戊為直角,申角為黃赤交角,申辛黃道弧亦為三十度,求得申戊赤道同升度,亦為二十七度五十四分一十秒。乃與壬申赤道之半周相減,得壬戊弧五宮二度五分五十秒,為本時春分距午後赤道度。變時得十時八分二十三秒,即本時春分距午時分也。次用辛辰甲正弧三角形,辰為直角,其辛角黃道赤經交角及甲辛弧太陽距天頂,皆與前圖之度等。求得辛辰弧黃平象限距午正黃道度,亦為十度四十七分二十八秒。與辛點八宮初度相減,因黃平象限在午西,故減。得辰點七宮十九度十二分三十二秒,即本時黃平象限之經度。又求得甲辰弧與甲卯象限相減,得辰卯弧,亦為六十三度三十二分四十秒,即本時限距地高,亦當辰寅卯角之度也。

又設太陽當正午實行距春分前三十度為二宮初度,乃以辛點太陽當午正,則春分壬點在午正之東,申為秋分,醜為冬至,乙子丑為過極至經圈,其子甲卯本時黃平象限亦在午正之東。法用辛戊壬正弧三角形,有戊直角,有壬角黃赤交角,有壬辛黃道弧三十度。求得壬戊赤道弧,亦為二十七度五十四分一十秒。乃與赤道全周相減,得十一宮二度五分五十秒,為本時春分距午後赤道度。變時得二十二時八分二十三秒,即本時春分距午時分也。又求得辛戊弧亦為十一度二十九分三十三秒,為太陽距赤道南緯度,並求得壬辛戊角亦為六十九度二十二分五十一秒,為本時黃道赤經交角。次用辛辰甲正弧三角形,此形有辰直角,有辛角,以甲戊赤道距天頂與辛戊黃赤距度相加,得甲辛弧太陽距天頂五十一度二十四分三十三秒。乃以半徑為一率,辛角之餘弦為二率,甲辛弧之正切為三率,求得四率,為黃平象限距午之正切,檢表得二十三度四十八分四十秒,即辛辰弧黃平象限距午正之黃道度。與辛點二宮初度相加,得辰點二宮二十三度四十八分四十秒,即本時黃平象限之經度也。又以半徑為一率,辛角之正弦為二率,甲辛弧之正弦為三率,求得四率,為甲辰弧黃平象限距天頂之正弦,檢餘弦表得四十二度五十九分一秒,即卯辰弧本時限距地高之度也。

一率半徑

二率辛角餘弦

三率甲辛弧正切

四率辛辰弧正切

一率半徑

二率辛角正弦

三率甲辛弧正弦

四率甲辰弧正弧

又設太陽當午正實行距秋分後三十度為十宮初度,乃以辛點太陽當午正,則申點秋分在午正後,而春分必在午正前,未為夏至,子乙未為過極至經圈,其子甲卯本時黃平象限在午正之西。求法仍用辛戊申正弧三角形,此形邊角之度與前圖之辛戊壬形同,惟申戊弧所變之一時五十一分三十七秒,乃秋分距午後之時分,是以加赤道半周之十二時,得十三時五十一分三十七秒,始為本時春分距午時分也。次用辛辰甲正弧三角形,此形邊與角之度亦與前圖之辛辰甲形同,惟因辰點在辛點之西,是以十宮初度內減辛辰弧二十三度四十八分四十秒,得九宮六度十一分二十秒,即本時黃平象限之經度。其辰卯弧限距地高四十二度五十九分一秒,亦與前數相同也。由此則逐度皆以距春、秋分前後各相對之度推之,其求午正太陽距天頂之加減,則以緯南、緯北而分。求黃平象限宮度之加減,則以冬至、夏至為斷。蓋冬至過午西,黃平象限恆在午正之東,夏至過午西,黃平象限恆在午正之西,此加減所由定也。

今設太陽黃道經度三宮十六度四十四分,用時為戌正二刻八分十九秒,求春分距午時分及黃平象限宮度、限距地平高度。如申辛壬癸為黃道,交地平於寅,壬為春分,醜為夏至,申為秋分,子乙丑亥為過二極二至經圈。乃自黃極子點過天頂甲點作子甲卯黃道經圈,其黃道適中之辰點,乃在午正之西。今太陽在春分後之未點,當赤道之午點,自子正計之,即用時之時刻。先用未午壬正弧三角形求壬午弧,此形午為直角,有壬角黃赤交角二十三度二十九分,有壬未弧太陽距春分後黃道度十六度四十四分,求得壬午弧十五度二十四分五十八秒,為太陽距春分後赤道度。變時得一小時一分四十秒,與午點用時相加,得二十一小時三十九分五十九秒,為壬點春分距子正後之時分。內減十二時,得九小時三十九分五十九秒,即壬戊弧本時春分距午時分。次用甲戊辛正弧三角形,因壬戊春分距午後之度已過象限,故用申戊辛正弧形。求辛角及辛戊、辛申二弧。此形戊為直角,有申角黃赤交角,有申戊弧秋分距午前時分所變之赤道度三十五度零十五秒,求得戊辛弧十三度五十九分四十秒,為本時正午之黃赤距度。求得申辛戊角七十度五十六分五十八秒,為黃道交子午圈角,即黃道赤經交角。與甲辛辰角為對角,其度等。求得申辛弧三十七度二十一分五十秒,為秋分距午正前黃道度。與申點秋分九宮相減,得七宮二十二度三十八分一十秒,即辛點正午黃道經度。次用甲辰辛正弧三角形求辛辰、甲辰二弧,此形辰為直角,有辛角黃道赤經交角。以甲戊弧京師赤道距天頂三十九度五十五分,內減辛戊正午黃赤距度,得甲辛弧二十五度五十五分二十秒,為本時正午黃道距天頂度,求得辛辰弧九度零五十三秒,為黃平象限距午西之黃道度。與辛點正午黃道經度相減,得辰點七宮十三度三十七分十七秒,即本時黃平象限之經度,並求得甲辰弧二十四度二十四分二十四秒,為黃平象限距天頂之度。與甲卯象限相減,得辰卯弧六十五度三十五分三十六秒,為本時黃平象限距地平之高度,即當辰寅卯角之度也。

求距限差

距限差者,乃月距黃平象限之差度也。蓋舊法月距限以九十度為率,因黃道麗天,其向隨時不同,而出於地平之上者,恆為半周,其適中之點,距地平東西皆九十度。故以九十度之限,以察月在地平之上下,若月距限逾九十度者,為在地平下,遂不入算,然此以黃道為立算之端也。顧白道與黃道斜交,月行白道,不無距黃道南北之緯度。緯南者早入遲出,月當地平時,其距黃平象限不及九十度;緯北者早出遲入,月當地平時,其距黃平象限已過九十度;是則九十度之率未足為據也。於是立法以求其差,猶五星伏見距日限度有距日加減差之義也。其法以限距地平之高及月距黃道之緯,依正弧三角形法求之。蓋黃道之勢,隨天左旋,其升降正斜,時時不同。正升正降者,京師限距地高至七十三度餘,高度大,則月緯所當之距限差轉小;斜升斜降者,京師限距地高只二十六度餘,高度小,則月緯所當之距限差轉大。若值月緯最大,其差可至十度有奇,此距限差之不可不立也。故依京師黃平象限距地平高度,逐度求其太陰黃道實緯度所當距限差以立表。

設京師限距地平高度三十四度,太陰距黃道實緯度南北各五度,求距限差。如圖甲為天頂,乙丙為地平,丁為黃極,甲丁乙丙為黃道經圈,戊己庚為黃道,交地平於己點,其戊點即黃平象限。戊丙為限距地高三十四度,與甲丁黃極距天頂之度等,而當戊己丙角與乙己庚角為對角,其度亦等。如月恰在正交或中交,合於黃道之己點,正當地平,則戊己為月距限九十度,若過九十度,自必在地平之下。今設月在黃道南五度,則辛壬癸為黃道距等圈,月在地平時為壬點,當於黃道之卯,其戊卯月距限乃不及九十度。又設月距黃道北五度,則子丑寅為黃道距等圈,月在地平時為醜點,當於黃道之辰,其戊辰月距限乃已過九十度,故必求其差數以加減之。法用己卯壬正弧三角形求己卯弧,此形有卯直角,有己角,當限距地高,有卯壬弧月距黃道緯度。乃以己角之正切為一率,半徑為二率,卯壬弧之正切為三率,求得四率,為距限差度之正弦,檢表得七度四十二分,即己卯弧為所求之距限差,而與己辰弧之度分等,蓋己辰丑正弧三角形與己卯壬形同用己角,而辰丑弧月距黃道緯度,亦與卯壬等是兩正弧形為相等形,故所得之己卯弧必與己辰弧相等無疑矣。既得己卯距限差,與戊己九十度相減,得八十二度十八分,即戊卯距限,而與距等圈辛壬之度相應,為月在緯南之地平限度。以己辰距限差與戊己九十度相加,得九十七度四十二分,即戊辰距限,而與距等圈子丑之度相應,為月在緯北之地平限度也。

一率己角正切

二率半徑

三率卯壬弧正切

四率己卯弧正弦

圖形尚無資料

求黃經高弧交角及月距天頂

舊法推日食三差,原以黃平象限為本。自考成前編謂三差並生於太陰,而太陰之經緯度為白道經緯度,用白道較之用黃道為密,故求三差則按月距白平象限之度,以白道高弧交角及太陰高弧為據。後編變通其法,乃以白經高弧交角及日距天頂以求三差,而求白經高弧交角,系赤經高弧交角加減赤白二經交角而得,並不求月距白平象限之度,是法較前頗為省算。今推視差者,乃求其星月黃道同經之視距視時,故三差應由黃平象限而定也。是則其法原可仿於後編不求黃平象限而竟求黃經高弧交角之術,即黃道高弧交角之餘度。然非月距黃平象限度與地平限度相較,其月在地平之上下無由可知。故今求交角,乃先求得月距黃平象限之東西、黃平象限去地之高下、太陰距黃極之遠近,然後按後編用斜弧形求赤經高弧交角日距天頂之法,則黃經高弧交角及月距天頂之度可得矣。

設星、月黃道經度同為申宮二十六度二十二分十一秒,月距正交前四十三度四十八分五十六秒,黃白交角五度四分一十秒,黃平象限七宮十三度三十七分十七秒,限距地高六十五度三十五分三十六秒,求太陰實緯黃經高弧交角月距天頂。如圖甲為天頂,甲乙丙丁為子午圈,丙丁為地平,乙為北極,戊己庚為赤道,戊為午正,己為酉正,庚為子正,卯為黃極,辛壬癸子為黃道,壬為春分,癸為夏至,午為黃道交地平之點。午未弧為九十度,其未點即黃平象限,宮度為七宮十三度三十七分十七秒。未辰弧當午角為六十五度三十五分三十六秒,即限距地高度,而與甲卯黃極距天頂之度等。巳寅丑為白道,寅為正交,寅角為黃白交角五度四分一十秒,申為太陰當黃道於酉,申寅為月距正交前白道度四十三度四十八分五十六秒,申酉為月距黃道緯度,其酉點為星月所當之黃道經度五宮二十六度二十二分十一秒,與未點黃平象限宮度相減,得未酉弧四十七度十五分六秒,為月距黃平象限西之度。乃當未卯酉角,甲申戌為高弧,卯申甲角為黃經高弧交角,甲申為月距天頂。求法,先用寅酉申正弧三角形,此形酉為直角,有寅角黃白交角,有寅申弧月距正交前白道度,求得申酉弧三度三十分二十七秒,即太陰距黃道南實緯度。與卯酉象限相加,得卯申弧九十三度三十分二十七秒,為月距黃極。次用甲卯申斜弧三角形,此形有甲卯邊黃極距天頂,有申卯邊月距黃極,有申卯甲角當酉未弧月距限度為所夾之角,求申角及甲申邊。乃自天頂作甲亥垂弧,分為甲亥卯、甲亥申兩正弧三角形。先用甲亥卯正弧三角形,此形亥為直角,有卯角,有甲卯邊,求得卯亥弧五十六度十四分十五秒,為距極分邊。與申卯弧月距黃極相減,得申亥弧三十七度十六分十二秒,為距月分邊。次用甲亥申正弧三角形,此形亥為直角,有申亥邊,兼甲亥卯正弧三角形之亥卯邊及卯角。用合率比例法,求得申角五十六度二分五十一秒,即黃經高弧交角。仍以甲卯申斜弧形,用對邊對角法,求得甲申弧五十三度四十三分二十四秒,即月距天頂之度也。

圖形尚無資料

求太陰距星及凌犯視時

太陰距地平上之高弧,自地心立算者為實高,在地面所見者為視高,其相差之分,即地半徑差也。月當地平時,距天頂為九十度,其相差之數最大,而角之正弦即當地之半徑。迨月上升,則距地漸高,距地愈高,則差數愈小,其所差之分,皆與本時月距天頂之正弦相應,故用比例法而得本時高下差也。夫高下既差,則有視經、視緯之別。其視經、實經之差者,東西差也;視緯、實緯之差者,南北差也。今求三差,乃依後編日食求三差法用直線三角形算之。然後編三差圖乃寫渾於平,今則用以渾測渾之圖,求其三差,其所得之南北差,與本時太陰實緯之度相較,而得視緯。得以視緯與星緯相較,觀其緯之南北而定相距之上下也。其所得之東西差,與一小時之太陰實行為比例,而得用時距視時之距分。辨其月距限之東西加減凌犯用時,而得凌犯之視時也。

前求得道光十二年壬辰三月初六日癸丑,月距司怪第四星凌犯用時戌正二刻八分十九秒,黃經高弧交角五十六度二分五十一秒,月距天頂五十三度四十三分二十四秒,本日太陰最大地半徑差六十分七秒,太陰黃道實緯度南三度三十分二十七秒,司怪第四星黃道緯度南三度十一分四十四秒,一小時太陰實行三十六分三十三秒,求星月相距分秒凌犯視時。如圖甲為天頂,甲未辰巳為黃道經圈,辰午巳為地平,卯為黃極,未午辛為黃道,未點即黃平象限宮度,未辰弧即限距地高,與卯甲黃極距天頂之度等。申點為太陰,子點為司怪第四星,同當黃道於酉。其酉點即月與星之黃道經度,酉未弧即月距限西之度,子酉為星距黃道南緯度三度十一分四十四秒,申酉為太陰距黃道南實緯度三度三十分二十七秒,申卯弧即月距黃極,甲申戌為高弧,申甲為月距天頂度五十三度四十三分二十四秒,卯申甲角為黃經高弧交角五十六度二分五十一秒,而與戌申亥角為對角,其度等。此皆自地心立算之實度也。然人居地面高於地心,故視高常低於實高,而月當地平時,其地半徑差為最大,今乃六十分七秒。於是依後編求本時高下差之法,以半徑與甲申弧正弦之比同於最大地半徑差與本時高下差之比,得本時高下差四十八分二十八秒。如申火之分,其火點即太陰之視高,自火點與黃道平行,作火木線,遂成申木火直角三角形。因弧度甚小,乃作直線算,與後編求日食三差之理同。此形木為直角,有申角黃經高弧交角,有申火邊本時高下差,求得木火邊四十分十二秒為東西差,求得申木邊二十七分四秒為南北差,加於申酉太陰實緯,得木酉太陰視緯三度五十七分三十一秒。內減子酉星緯,得子木弧四十五分四十七秒,為人目仰視太陰距司怪第四星月在星下之分也。夫星、月同當酉點之經度,固為相距。今太陰視高在火,其視緯雖差至木,而距星之子點尚在一度內,其火點當黃道之視經度則差至土,是用時時星經度雖在酉,而太陰視經度之土點乃在其西,是為未及。然土酉之分與火木等,故以一小時太陰實行與火木東西差為比例,得距分一時六分,為月行火木之時分。加於月視高臨火點之用時,得亥初二刻十四分十九秒,即人目視太陰臨於木點與星,同當酉點經度之視時也。

圖形尚無資料

求視時月距限

視時月距限,必大於用時月距限,因其視經差所當之距分既有加減,則太陰與星隨天西移自有進退也。蓋太陰以地半徑差由高而變下,則視經之差於實經、視緯之差於實緯必矣。茲據黃平象限在天頂南之地面而言之,視緯恆差而南,如實緯北者,視緯常小於實緯,其差為減;實緯南者,視緯常大於實緯,其差為加。故緯南之星、月實距雖在一度內,而視距轉在一度外者有之;緯北之星、月實距雖在一度外,而視距轉在一度內者有之。南北相距一度外者不入凌犯之限,故不取用。至若視經之差,所當月行距分之最大者或至二小時,而二小時之際,諸曜隨天左旋,幾至一宮,故視經之差,關於月行之進退矣。如月在黃平象限西者,視經度差之而西,視時必遲於用時;月在黃平象限東者,視經度差之而東,視時必早於用時。以致用時星、月未入地平,而視時星、月已入地平者有之,或用時星、月已出地平,而視時星、月未出地平者有之。是故於求用時之後,即以月距黃平象限與地平限度相較,可知斯時月在地平之上下。月距限小於地平限度者,為月在地平上;大於地平限度者,為月在地平下。如遇月距限微小於地平限度者,用時星、月必在地平上,視時星、月或在地平下,其所差者,即視經之差當月行距分之諸曜左旋度。今取最小實經、視經之差所當左旋之度為視經差,法見下卷求地平限度節下。減於地平限度,所得視地平限度,而與月距限度考之。如月距限小於地平限度而大於視地平限度者,則為用時月雖在地平上,視時月必在地平下矣;既知月必在地平下,故遇此者去之。如月距限小於視地平限度者,則為視時月在地平之上。夫猶有不然者,以視經差所取皆最小之數也。若知月行實跡非由視時,再推月距限度,則其時月果在地平之上下,未可得其確準。故今於既得視時之後,必詳察太陰實緯及用時月距限度。如實緯南月距限過六十度,或實緯北月距限過七十度者,用時月距限在此限度內者,視時月必在地平之上。皆以視時復求月距黃平象限之度。如其度大於地平限度者,乃視時月在地平之下,仍不取用。必其度小於地平限度,始為視時月必在地平之上,而可證諸實測。此視差之所以必逐細詳推,然後可得而取用也。


\end{pinyinscope}