\article{列傳一百六十三}

\begin{pinyinscope}
辛從益張鱗顧皋沈維鐈硃為程恩澤吳傑

辛從益,字謙受,江西萬載人。乾隆五十五年進士,選庶吉士,授編修。遷御史,以母老陳請終養。嘉慶十七年,起復補原官。會京畿多雨,詔發廩平糶,從益在事,釐剔弊端,實惠及民,時稱之。疏請飭督撫詳慎甄別以澄吏治,略曰:「外省甄別,與京員不同。京師耳目甚密,稍有徇私,難逃聖明洞鑒。外省督撫權勢既尊,操縱甚易,豈知州縣有當切責之處,亦有當體恤之處,偏私則是非倒置,刻覈則下情不通。臣以為大吏必持廉法之大綱,略趨承之末節;務幹事之勤能,責安民之實效;揣時勢之難易,量才分之優絀;而又常存敬畏之心,然後能愛惜人才,澄清吏治。」遷給事中。

十八年,滑縣匪平,軍中多攜養難民子女,從益疏請遣送歸家,如議行,並譴領兵大員。又面奏:「正教昌明,邪說自息,小民不識大義,故易為邪教煽惑。而選人得官,不問風俗淳澆,祗計缺分肥瘠,何以教民?欲厚風俗,宜先責成牧令。」歷光祿寺少卿、通政司參議、內閣侍讀學士、光祿寺卿、太常寺卿。道光初,山西學政陳官俊鐫級回京,仍直上書房,從益疏劾曰:「上書房為教胄諭德之地,視學政為尤重,宜慎選德行敦厚、器識宏達之儒臣,使皇子有所觀法,薰陶養其德性。陳官俊在學政任,不能遠色避嫌,懲忿窒欲,性行之駁,器識之褊,不宜仍居授讀之任。」

二年,遷內閣學士。宣宗溫諭曰:「爾甚樸忠,無所希冀,亦無所揣摩。有所聞見,直言無隱,朕無忌諱也。」命偕尚書文孚赴陜西讞獄。渭南富民柳全璧殺其傭硃錫林,賄知縣徐潤得免死,巡撫硃勛庇之,獄久不決。從益等鞫得其狀,論如法。覆命,陳陜西馬政之害,地方官春秋計裡買馬,實則民不得直,而官亦不需馬,第指馬索賕以為民病,請禁革。三年,擢禮部侍郎,督江蘇學政。於是巡撫陶澍奏禁紳衿包漕,橫索漕規,下學政稽查懲治。從益上疏曰:「江蘇漕額本重,豈堪浮收無節?州縣自應調劑,閭閻尤宜體恤。久懸定額,尚肆苛求;明語浮收,必滋流弊。撫臣之意,謂控漕之人即包漕之人,臣以為未必盡然。官之收漕,必用吏役,吏役貪狠,必圖肥己。官既浮收,吏又朘削,不特小民受害,即循謹生監,亦被其累,激而上控,此中固有不得已者。撫臣又稱生監需索漕規,地方官費無所出,乃取償於純謹小民。臣伏思吏役貪得無厭,縱生監悉循循守法,而小民追呼徵比之煩,亦斷不能為之少減。吏役倚官府為城社,倘違例浮收,無人控訴,將何術以治之?夫劣衿律所不宥,苛政亦法所必裁。矯枉勢必過正,創法宜防流弊。管見所及,不敢不以上聞。」

從益廉靜坦白,遇非理必爭,不為權要詘。八年,卒於學政任所。著有奏疏、詩文內外集、公孫龍子注。

張鱗,字小軒,浙江長興人。嘉慶四年進士,選庶吉士。習國書,授檢討。仁宗臨幸翰林院,鱗獻詩冊,被恩賚。十七年,大考二等,遷贊善。歷侍講、庶子。二十年,選翰林官入直懋勤殿,纂輯秘殿珠林、石渠寶笈,鱗與焉。歷侍講學士、國子監祭酒。二十四年,典江西鄉試。尋以齋戒未至齋所,降授太常寺少卿。遷通政使司副使、太僕寺卿。道光元年,命偕太常寺少卿明安泰赴楊村挑驗剝船,遂赴東光、盧龍兩縣訊鞫京控獄,各論如律;並劾承審官濫刑,巡道徇庇,褫黜有差。三年,轉太常寺卿,督安徽學政,擢內閣學士。七年,以繼母憂歸,服闋,補原官。擢兵部侍郎,督福建學政。十三年,補戶部,又調吏部。福建縣丞秦師韓控訐總督程祖洛,侍郎趙盛奎偕鱗同案鞫,白其誣,師韓遣戍。

鱗清廉儉素,杜絕干謁。兩為學政,卻陋規,拔寒畯,閩人尤頌之。衡文力矯通榜之習。十五年,典會試,以校閱勞致疾,出闈,卒。福建士民請祀名宦祠。

顧皋,字歅齊,江蘇無錫人。嘉慶六年一甲一名進士,授修撰。九年,督貴州學政,釐剔弊竇,奏改黎平、開泰學額,士林頌之。超擢國子監司業。二十一年,直懋勤殿,編輯秘殿珠林、石渠寶笈。歷翰林院侍讀、左右庶子、侍講學士、侍讀學士。典陜甘鄉試。二十四年,入直上書房,甚被仁宗眷注。二十五年,扈蹕熱河。上升遐之日,御筆擢皋詹事。次日,宣宗即位,執皋手大慟。道光元年,遷內閣學士,擢工部侍郎,兼管錢法堂。二年,調戶部。連典順天、浙江鄉試,管理國子監事務。

皋在戶部,不為激亢之行,考覈利病,慎稽出納,不可干以私。嘗曰:「學期見諸實用。吾久回翔於文學侍從。及任經世理物之責,未能壹志專慮,以求稱職,為自愧耳。」八年,以病乞歸。十一年,卒。

沈維鐈,字子彞,浙江嘉興人。嘉慶七年進士,選庶吉士,授編修。歷司業、洗馬。與修全唐文、西巡盛典、一統志,入直懋勤殿,纂輯秘殿珠林、石渠寶笈。二十一年,督湖北學政,禁習邪教,以端士風。累遷侍讀學士。道光二年,典福建鄉試,留學政。疏陳州縣私設班館之弊,請飭嚴禁,並禁監生充緝捕、催科諸役。四年,遷大理寺少卿。八年,督順天學政,轉太僕寺卿。任滿,遷宗人府丞,署副都御史,尋實授。十二年,督安徽學政,奏請增建壽州考棚,與鳳陽分試。瀕江水災,偕疆吏會籌賑撫,士民頌之。維鐈居官廉,屢視學,所至弊絕風清,振拔多知名士,宣宗知之,期滿連任。擢工部侍郎。十七年,請回籍營葬,詔予假三月,毋庸開缺,事竣回京。十八年,以耳疾許免職,命病痊以聞。逾年,卒於家。

維鐈學以宋儒為歸,謂典章制度與夫聲音訓詁當宗漢人,而道理則備於程、硃,務為身心有用之學。校刊宋儒諸書以教士,時稱其醇謹焉。祀鄉賢祠。

硃為弼,字右甫,浙江平湖人。嘉慶十年進士,授兵部主事,遷員外郎。道光元年,授御史,遷給事中。疏請整頓京師緝捕,劾倉場覆奏海運倉豆石霉變情形不實,命大臣按鞫,侍郎和桂、張映漢並被譴。又疏陳江蘇海口壅塞,浙江上游均受其害,請疏濬太湖下游劉河、吳淞諸水,為一勞永逸之計,如所議行。四年,擢順天府府丞,遷府尹。有蝗孽,單騎馳視,卻屬官供張,曰:「吾為蝗來,乃以我為蝗耶?」六年,復降授府丞。歷通政司副使、太常寺卿、宗人府府丞、都察院左副都御史。十三年,擢兵部侍郎,權倉場侍郎,尋實授。

十四年,出為漕運總督。時漕船水手恣橫,廬州幫在東昌械斗,傷斃多命,下為弼查辦,疏言:「漕督例隨幫尾,在前者無從遙制。請責成押運官弁會同地方官拏辦。」並定頭柁十家聯保,舉發徇隱賞懲之法,奏陳剔弊速漕章程八事,下所司議行。十五年,以病乞免,允之。二十年,卒。

為精揅金石之學,佐阮元纂鐘鼎彞器款識,所著有蕉聲館詩文集。

程恩澤,字春海,安徽歙縣人。父昌期,乾隆四十五年一甲三名進士,累官至侍講學士,直上書房。恩澤勤學嗜奇,受經於江都凌廷堪,廷堪勖之曰:「學必天人並至,博而能精,所成乃大。」嘉慶十六年,成進士,選庶吉士,授編修。道光元年,入直南書房,宣宗曰:「汝父蘭翹先生昔年在上書房,朕敬其品學。汝之聲名,亦所深悉,宜更守素行。」典試四川。三年,督貴州學政,勸民育慄蠶,其利大行。重刊岳珂五經以訓士。鄭珍有異才,特優異之,餉以學,卒為碩儒。六年,調湖南學政。任滿回京,洊擢國子監祭酒。命充春秋左傳纂修官,推本賈、服,不守杜氏一家之言。母憂歸。十一年,服闋,仍直南書房。未補官,特命典試廣東。知南海曾釗名,冀得之。釗未與試,榜發,大失望。所得多知名士。改直上書房,授惠親王讀。遷內閣學士。十四年,授工部侍郎,調戶部。以部務繁,罷直書房。十七年,卒,上甚惜之,優詔賜恤,賜其子德威舉人。

恩澤博聞強識,於六藝九流皆深思心知其意,天象、地輿、壬遁、太乙、脈經莫不窮究。謂近人治算,由九章以通四元,可謂發明絕學,而儀器則罕傳,欲修復古儀器而未果。詩古文辭皆深雅。時乾、嘉宿儒多徂謝,惟大學士阮元為士林尊仰,恩澤名位亞於元,為足繼之。所欲著書多未成,惟國策地名考二十卷、詩文集十卷傳於世。

吳傑,字梅梁,浙江會稽人。少能文,為阮元所知。以拔貢生應天津召試,二等,充文穎館謄錄,書成,授昌化教諭。嘉慶十九年,成進士,選庶吉士,授編修,遷御史。道光二年,督四川學政,疏請以唐陸贄從祀文廟,下部議行。遷給事中,出為湖南嶽常澧道,歷貴州按察使、順天府丞。

十三年,川南叛夷犯邊,師久無功,傑疏言:「川夷作亂,提督桂涵連戰克捷,生擒首逆,清溪近邊遂無夷。楊芳繼任,用兵之區僅瓘邊一處,夷寇不過數部落,當易獲勝。惟夷巢跬步皆山,夏令河水盛漲,徒涉尤難。楊芳自抵瓘邊,頓兵三月。臣思其故,必逆夷退伏老巢,水潦既降,不易深入。楊芳不敢以軍情入告,但稱督兵進剿,實皆游移觀望之辭。曠日持久,邊事所關非細,請敕總督鄂山體察確奏,毋得徇隱。」

又疏言:「馭夷長策,當先剿后撫。未剿遽撫,良莠不分。兵至,相率歸誠;兵退,復出焚掠。層巒疊嶂,我師轉運為艱。夷族因利伺隙,倏起★L1伏,使我猝不及防。國家既厚集兵力,自當掃穴犁庭,除惡務盡,使諸夷望風震懾,一勞永逸。自古馭夷之法,討伐易而安撫難。善後之舉,至要者二:一曰除內奸。游手無業之徒,潛居夷地,為之謀主,教以掠人勒贖,聚眾焚殺,及避火器敵官軍之策。夷悍而愚,得之乃如虎傅翼,必應名捕,盡法懲治。良民亦驅使回籍,毋任逗留異域;宣諭土司,不得容留漢民;營伍邏詰,絕其潛入之路,則奸人無繇手冓煽矣。一曰分疆界。夷族愚惰,不諳農事,漢民租地,耕作有年,既漸闢磽鹵為膏腴,群夷涎其收穫,復思奪歸,構釁之原,不外於此。今當勘丈清釐,凡漢民屯種夷地,強占者勒令退還,佃種者悉令贖歸。無主之田,墾荒已久,聚成村落,未便遷移,畫為漢界,禁其再行侵占,庶爭端永息。」又奏:「越巂設撫民通判,止治漢民,而熟夷皆受治土司,通判無專責,且營伍非其所轄,呼應不靈,每以細故釀為大釁。請改為撫夷通判,千把總以下皆受節制。」疏上,下鄂山議行。

遷內閣學士。十五年,擢工部侍郎,連典順天鄉試及會試。十六年,卒。

論曰:宣宗最重文學廉謹之臣,辛從益直言獻納,張鱗廉介絕俗,沈維鐈服膺理學,程恩澤博物冠時,皆負清望。顧皋、硃為、吳傑並以雅材回翔卿貳,亦足紀焉。


\end{pinyinscope}