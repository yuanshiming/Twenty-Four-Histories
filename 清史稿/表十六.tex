\article{表十六}

\begin{pinyinscope}
軍機大臣年表一

軍機處名不師古,而絲綸出納,職居密勿。初祗秉廟謨商戎略而已,厥後軍國大計,罔不總攬。自雍、乾後百八十年,威命所寄,不於內閣而於軍機處,蓋隱然執政之府矣。今詳著其人,庶後之考心腹股肱之佐,而究其時政化隆?消長之跡者,以覽觀焉。作軍機大臣表。

雍正七年己酉六月,始設軍機房。

怡親王允祥六月癸未,命密辦軍需一應事宜。十月,賜加儀仗一倍。

張廷玉六月癸未,以太子太保、保和殿大學士,命密辦軍需一應事宜。十月,晉少保。

蔣廷錫六月癸未,以文華殿大學士,命密辦軍需事宜。十月,加太子太傅。

八年庚戌

怡親王允祥三月,病。五月辛未,薨。

張廷玉十月,以贊襄機務周詳妥協,賜一等阿達哈哈番世職。

蔣廷錫十月,以贊襄機務周詳妥協,賜一等阿達哈哈番世職。

馬爾賽五月丁卯,以世襲一等公、武英殿大學士,命與張廷玉、蔣廷錫詳議軍行事宜。十月,以贊襄機務周詳妥協,賜一等阿達哈哈番世職。

九年辛亥

馬爾賽三月,晉襲一等忠達公。七月甲戌,授撫遠大將軍。八月,啟行。出。

張廷玉

蔣廷錫

十年壬子三月,改軍機房稱辦理軍機處。

張廷玉

蔣廷錫閏五月,病。七月,卒。

鄂爾泰二月,以少保、三等男、保和殿大學士辦理軍機事務。旋晉一等伯。七月,命往肅州經略西路軍務。

哈元生十月,以召覲貴州提督在辦理軍機處行走。旋命回籍省親。十一月,貴州苗叛,命回任。出。

十一年癸丑

鄂爾泰正月,轉命經略北路軍務。六月,還。

張廷玉十月,給假還籍。

馬蘭泰內大臣、蒙?二月己未,以一等英誠侯、領侍古都統在辦理軍機處行走。四月戊午,仍命往軍前督兵操演。出。

平郡王福彭四月,以右宗正在辦理軍機處行走。七月戊子,授定邊大將軍。出。

訥親十一月甲辰,以三等果毅公、御前大臣、鑾儀使在辦理軍機處行走。

班第十一月,以理籓院右侍郎在辦理軍機處行走。

十二年甲寅

鄂爾泰

張廷玉在假。二月,還。

訥親

班第

十三年乙卯十月,罷辦理軍機處,由總理事務處兼理。

鄂爾泰五月,命兼值辦理苗疆事務處。七月乙卯,降三等男,解職。八月己丑,起原官,命總理事務。十月,晉

-6232-

一等子。甲午,裁辦理軍機處。

張廷玉五月,命兼值辦理苗疆事務處。八月,命總理事務。十月甲午,裁辦理軍機處。

訥親八月,授滿洲都統。十月,授領侍衛內大臣。甲午,裁辦理軍機處,命協辦總理事務。

班第八月庚寅,改在總理事務處差委辦事。

索柱以內閣學士辦理軍機事務。八月庚寅,命改在總理事務處差委辦事。

豐盛額以一等英誠公、都統辦理軍機事務。十月甲午,裁辦理軍機處,命回本任。

海望以內大臣、戶部左侍郎辦理軍機事務。九月,遷戶部尚書。十月甲午,裁辦理軍機處,命協辦總理事務。

莽鵠立以兼管理籓院侍郎、滿洲都統辦理軍機事務。十月甲午,裁辦理軍機處,命回本任。

納延泰以理籓院左侍郎辦理軍機事務。十月甲午,裁辦理軍機處,命在總理事務處差委辦事。

徐本十月辛巳,以協辦大學士、刑部尚書在辦理軍機處行走。甲午,裁辦理軍機處,命協辦總理事務。

乾隆元年丙辰總理事務處。

鄂爾泰

張廷玉

訥親

班第

二年丁巳十一月,復辦理軍機處。

鄂爾泰十一月辛巳,仍以少保、一等子、保和殿大學士為辦理軍機大臣。十二月,晉三等伯。

張廷玉十一月辛巳,仍以少保、三等子、保和殿大學士為辦理軍機大臣。十二月,晉三等伯。

訥親十一月辛巳,仍以一等果毅公、兵部尚書為辦理軍機大臣。

海望十一月辛巳,仍以戶部尚書為辦理軍機大臣。

納延泰十一月辛巳,仍以刑部左侍郎為辦理軍機大臣。

班第十一月辛巳,仍以理籓院左侍郎為辦理軍機大臣。

三年戊午

鄂爾泰

張廷玉

訥親十二月,轉吏部尚書。

海望

納延泰四月,遷理籓院尚書。

班第四月,轉兵部右侍郎。

徐本是年仍以東閣大學士為辦理軍機大臣。

四年己未

鄂爾泰五月,晉太保。

張廷玉五月,晉太保。

徐本五月,加太子太保。

訥親五月,加太子太保。

海望五月,加太子少保。

納延泰

班第七月丙寅,授湖廣總督。出。

五年庚申

鄂爾泰

張廷玉

徐本

訥親

海望

納延泰

六年辛酉

鄂爾泰

張廷玉

徐本

訥親

海望

納延泰

班第正月乙酉,復以原任湖廣總督在軍機處行走。三月,授兵部尚書。

七年壬戌

鄂爾泰

張廷玉

徐本

訥親

海望

班第

納延泰

八年癸亥

鄂爾泰

張廷玉

徐本

訥親

海望

班第

納延泰

九年甲子

鄂爾泰

張廷玉

徐本六月己酉,致仕。

訥親正月,差赴江、浙、魯、豫勘事。七月,還。

海望

班第

納延泰

十年乙丑

鄂爾泰正月,病。三月,晉太傅。四月,卒。

張廷玉

訥親三月,協辦大學士。五月,授保和殿大學士。

海望十二月乙卯,以精力漸衰罷。

班第

納延泰

傅恆六月己酉,以戶部右侍郎在軍機處行走。

汪由敦十月戊午,以刑部尚書在軍機處行走。

高斌十二月乙卯,以太子太保、協辦大學士、吏部尚書在軍機處行走。

蔣溥十二月乙卯,以吏部右侍郎在軍機處行走。

十一年丙寅

訥親

張廷玉

高斌二月,差赴南河勘事。十一月,還。

班第三月,差赴四川辦理軍務。七月,差赴鳳皇城勘界。九月,命署山西巡撫。十二月,召還。

汪由敦

納延泰

溥恆

蔣傅

十二年丁卯

訥親四月,差赴山西勘案。六月,還。

張廷玉

高斌三月,授文淵閣大學士。四月,差赴江南勘河。九月,差赴浙江鞫案。

班第

汪由敦

納延泰八月,差赴蘇尼特給賑。

傅恆三月,遷工部尚書。

蔣溥

十三年戊辰

訥親正月,差赴浙江審案。四月,命往金川經略軍務。九月庚辰,革職。

張廷玉

高斌三月,命轉赴山東勘事。閏七月丙辰,授江南河道總督。出。

班第正月己亥,差赴金川辦理軍務。出。

傅恆四月,加太子太保協辦大學士。九月,命經略金川軍務。十月,授保和殿大學士。十二月,晉太保。

汪由敦

納延泰

蔣溥四月,遷戶部尚書。丁卯,命專辦部務。罷。

陳大受四月丁卯,以太子少保、兵部尚書在軍機處行走。旋協辦大學士。

舒赫德九月己卯,以戶部侍郎、漢軍都統在軍機處行走。十月,遷兵部尚書。十一月,轉戶部尚書。

來保九月辛巳,以太子太保、武英殿大學士在軍機處行走。

尹繼善十一月己巳,以太子少保、協辦大學士、戶部尚書在軍機處行走。甲戌,授陜甘總督。出。

十四年己巳

傅恆經略金川軍務。正月,封一等忠勇公。三月,還。

張廷玉八月,晉三等勤宣伯。十一月戊辰,致仕。

來保二月,晉太子太傅。

陳大受二月,晉太子太保。七月,命署直隸總督。十月,還。十一月,病假。

汪由敦二月,加太子少師。十一月,署協辦大學士。十二月,革署協辦大學士,仍留刑部尚書。

納延泰二月,加太子少保。

舒赫德正月,授金川參贊大臣。二月,加太子太保。十二月庚寅,復轉兵部尚書。以職務繁多,命罷。

十五年庚午

傅恆

來保

陳大受正月丁未,授兩廣總督。出。

汪由敦七月,降兵部侍郎。

納延泰

劉綸正月壬戌,以工部右侍郎在軍機處行走。

兆惠四月庚辰,以刑部侍郎在軍機處行走,轉戶部侍郎。十一月,差赴西藏會辦善後事宜。

舒赫德十一月丙辰,復以太子太保、兵部尚書在軍機處行走。十二月,差往江南審案。

十六年辛未

傅恆

來保

舒赫德

納延泰

汪由敦八月,轉戶部右侍郎。

劉綸九月壬申,以父憂免。

兆惠八月,命署山東巡撫。

十七年壬申

傅恆

來保

舒赫德正月,差赴北路軍營。

納延泰

汪由敦九月,遷工部尚書。

兆惠

班第九月辛巳,復以都統銜在軍機處行走。旋授漢軍都統。

劉統勛十一月甲子,以刑部尚書在軍機處行走。

十八年癸酉

傅恆

來保

舒赫德九月,差勘南河。十二月,差往北路辦理軍務。

劉統勛七月,差勘河工。

汪由敦

納延泰

班第正月,命署兩廣總督。

兆惠二月,差赴西藏會辦事件。

劉綸八月,以服制將闋,召來京補戶部右侍郎。尋復入直。

十九年甲戌

傅恆

來保

舒赫德在差。七月甲辰,以安置準部降人失策,革職。

劉統勛正月,差勘海口。四月,加太子太傅。五月,命馳往西安協辦總督事。

汪由敦四月,晉太子太傅。

納延泰

班第署兩廣總督。四月,內召。七月甲辰,授兵部尚書,署定邊左副將軍。出。

兆惠三月,差往北路協辦軍務。出。

劉綸

覺羅雅爾哈善六月壬甲,以署戶部左侍郎在軍機處行走。十月,補兵部右侍郎。

阿蘭泰八月戊申,以召覲盛京將軍暫在軍機處行走。壬子,命赴軍營帶兵。出。

二十年乙亥

傅恆

來保

劉統勛協辦西安總督事。九月丙申,革職。

汪由敦九月,轉刑部尚書。

納延泰

劉綸十二月,差赴浙江審案。

覺羅雅爾哈善十月,命往北路參贊軍務。

二十一年丙子

傅恆四月,命往西路經理軍務。旋止行。

來保

汪由敦六月,轉工部尚書。

納延泰八月癸卯,差赴北路軍營。出。

劉綸四月癸亥,命回部辦事,罷直。

覺羅雅爾哈善三月,召還。四月癸亥,命回部辦事,罷直。

阿里袞四月甲寅,以戶部尚書暫在軍機處行走。五月癸酉,差往西路軍營領隊。出。

裘曰修四月癸亥,以吏部左侍郎在軍機處行走。

劉統勛六月癸丑,起授原官,仍入直。九月,差勘銅山漫工。十月,命署江南河道總督。十一月,內召。

夢麟八月癸卯,以工部右侍郎在軍機處學習行走。

二十二年丁丑

傅恆

來保

劉統勛四月,差赴徐州督修石壩。五月,轉差雲南勘獄。十一月,差赴山西勘獄。十二月,加太子太保。

汪由敦正月,轉吏部尚書。

裘曰修

夢麟正月,差赴江南、山東勘事。九月,轉戶部右侍郎。尋還直。

二十三年戊寅

傅恆

來保

劉統勛正月,轉吏部尚書。五月,內召。

汪由敦正月甲寅,卒。

裘曰修十二月癸亥,以事免。

夢麟四月,仍轉工部右侍郎。八月,卒。

三泰正月己酉,以吏部左侍郎在軍機處行走。四月,轉戶部左侍郎。七月己巳,授庫車參贊大臣。出。

劉綸正月己酉,復以戶部左侍郎在軍機處行走。

二十四年己卯

傅恆

來保

劉統勛正月,協辦大學士。二月,差赴西安勘獄。六月,差赴山西勘獄。

劉綸閏六月,遷左都御史。

二十五年庚辰

傅恆

來保

劉統勛八月,差赴江南勘事。十月,轉赴江西勘事。十二月,內召。

劉綸

富德內大臣、?二月乙巳,以一等成勇靖遠侯、領侍都統在軍機處行走。三月,授理籓院尚書。

兆惠二月,仍以一等武毅謀勇公、戶部尚書入直。

阿里袞七月甲寅,仍以襲封一等果毅公、兵部尚書入直。

於敏中八月己亥,以戶部右侍郎在軍機處行走。十一月,轉左侍郎。

二十六年辛巳

傅恆

來保

劉統勛五月,授東閣大學士。八月,命督辦河南楊橋漫工。十月,內召。

兆惠七月,協辦大學士。

阿里袞

劉綸五月,轉兵部尚書。

富德

於敏中

二十七年壬午

傅恆

來保

劉統勛三月,差勘高、寶河入江水道。四月,轉勘德州運河。

兆惠

阿里袞

劉綸

富德九月丁亥,革職,削爵。

於敏中

二十八年癸未

傅恆

來保

劉統勛

兆惠十月,加太子太保。

阿里袞六月,命署陜西巡撫。十月,加太子太保。

劉綸五月,轉戶部尚書。六月,協辦大學士。十月,加太子太保。

於敏中

阿桂月正壬申,以工部尚書在軍機處行走。四月,差赴歸化城、西寧等處勘事。十月,加太子太保。

二十九年甲申

傅恆

來保三月,卒。

劉統勛

兆惠十一月,卒。

劉綸

阿里袞六月,還直。十一月,轉戶部尚書、協辦大學士。

阿桂三月,署四川總督。十二月,還。

於敏中

三十年乙酉

傅恆

劉統勛

劉綸正月癸丑,憂免。

阿里袞

阿桂閏二月,以烏什回亂,命往伊?辦事。出。

於敏中正月,遷戶部尚書。

尹繼善九月,復以太子太保、文華殿大學士入直。

三十一年丙戌

傅恆

尹繼善

劉統勛

阿里袞

於敏中

三十二年丁亥

傅恆

尹繼善

劉統勛

阿里袞

於敏中

劉綸三月,服闋。五月,仍以太子太保、協辦大學士入直。

三十三年戊子

傅恆二月,命往雲南經略徵緬軍務。未行。

尹繼善

劉統勛四月,差勘江南清口疏濬事宜。

阿里袞正月壬子,命往雲南參贊軍務。出。

劉綸

於敏中八月,加太子太保。

福隆安二月丙戌,以和碩額駙、兵部尚書在軍機處學習行走。四月,轉工部尚書。

索琳十一月癸卯,以署戶部右侍郎在軍機處行走。

三十四年己丑

傅恆二月,往雲南經略軍務。

尹繼善

劉統勛九月,差勘挑濬運河事宜。

劉綸

於敏中

福隆安

索琳二月,補戶部右侍郎。

三十五年庚寅

傅恆經略徵緬軍務。三月,還。七月,卒。

尹繼善

劉統勛

劉綸

於敏中

福隆安七月,穿孝給假。十月,襲封一等忠勇公。

索琳十二月,差赴土默特鞫獄。

溫福閏五月己未,以吏部侍郎在軍機處行走。七月,遷理籓院尚書。

豐升額八月丙戌,以襲封一等果毅公、署兵部尚書在軍機處學習行走。

三十六年辛卯

尹繼善四月,卒。

劉統勛

劉綸二月,授文淵閣大學士。

於敏中二月,協辦大學士。

福隆安

溫福五月己巳,命往雲南署定邊右副將軍。出。

豐升額

索琳三月癸卯,降為軍機司員。免。

桂林四月甲戌,以戶部右侍郎在軍機處學習行走。九月己酉,命往四川會辦軍務。出。

慶桂九月癸卯,以理籓院侍郎在軍機處學習行走。

三十七年壬辰

劉統勛

劉綸

於敏中

福隆安五月,差赴四川勘事。尋還直。

豐升額三月,命往四川參贊軍務。出。

慶桂

福康安五月辛丑,以戶部侍郎在軍機處學習行走。十二月癸酉,命往四川領隊。出。

三十八年癸巳

劉統勛十一月辛未,卒。

劉綸六月,卒。

於敏中

福隆安四月,加太子太保。

慶桂四月辛亥,授伊?參贊大臣。出。

索琳四月庚戌,復以署禮部侍郎在軍機處學習行走。十月,補內閣學士。旋差赴歸化城勘事。出。

舒赫德七月甲子,復以太子太保、武英殿大學士入直。

袁守侗九月丙子,以刑部左侍郎在軍機處學習行走。十月,差赴浙江勘事。

梁國治十一月壬申,以湖南巡撫內召,在軍機處行走。十二月,署禮部左侍郎。

三十九年甲午

於敏中

舒赫德九月,命赴山東剿賊。尋還直。

福隆安

袁守侗二月,差赴四川勘事。十月,差赴貴州勘事。十二月,轉吏部右侍郎。

梁國治六月,補戶部左侍郎。

阿思哈七月乙亥,以左都御史在軍機處行走。九月,差赴山東剿賊。

四十年乙未

於敏中

舒赫德

福隆安

阿思哈十月,差赴青海勘事。

袁守侗八月,差赴貴州勘事。

梁國治

四十一年丙申

於敏中正月,賜世職。

舒赫德

福隆安正月,轉兵部尚書。

阿思哈正月庚寅,授漕運總督。出。

袁守侗三月,遷戶部尚書。

梁國治

和珅三月庚子,以戶部右侍郎在軍機處行走。

阿桂四月辛亥,復以太子太保、一等誠謀英勇公、協辦大學士、吏部尚書在軍機處行走。

豐升額四月,還,仍以太子少保、一等果毅公、戶部尚書入直。

福康安四月,還,仍以三等嘉勇男、戶部左侍郎入直。

明亮十二月丙午,以入覲一等襄勇伯、成都將軍暫在軍機處行走。旋命還四川本任。出。

四十二年丁酉

於敏中

舒赫德四月丁巳,卒。

阿桂正月,命赴雲南受降。五月,授武英殿大學士。七月,還。

福隆安十一月,差赴盛京勘事。十二月,還。

豐升額十月,卒。

袁守侗十一月,轉刑部尚書。

梁國治十一月,遷戶部尚書。

和珅六月,轉戶部左侍郎。十月,兼步軍統領。

福康安六月乙卯,授吉林將軍。出。

四十三年戊戌

於敏中

阿桂

福隆安

袁守侗

梁國治

和珅

李侍堯六月癸巳,以入覲太子太保、二等昭信伯、武英殿大學士、雲貴總督暫在軍機處行走。尋還總督本任。出。

四十四年己亥

於敏中十二月丁巳,卒。

阿桂正月,差勘南河壩工。

福隆安三月,差赴真定勘事。尋還直。

袁守侗四月戊寅,授山東河道總督。出。

梁國治

和珅八月,授御前大臣。

董誥十二月甲寅,以戶部左侍郎在軍機處行走。

四十五年庚子

阿桂四月,還。十二月,差勘浙江海塘。

福隆安

梁國治

和珅正月,差赴雲南勘事。三月,遷戶部尚書。五月,還。

董誥

福長安正月丙午,以署工部右侍郎在軍機處學習行走。二月,授戶部右侍郎。

四十六年辛丑

阿桂三月,命轉赴甘肅剿叛回。八月,命回途赴豫勘河。十月,命赴浙讞獄。十二月,還。

福隆安

梁國治

和珅三月,差赴甘肅剿逆回。五月,還。

董誥

福長安

四十七年壬寅

阿桂

福隆安

梁國治八月,加太子少傅。

和珅四月,差赴山東勘事。八月,加太子太保。

董誥

福長安四月,差赴奉天勘事。九月,差赴浙江勘事。十二月,還。

四十八年癸卯

阿桂正月,差勘河工。四月,還。

福隆安

梁國治七月,協辦大學士。

和珅

董誥

福長安七月,轉戶部左侍郎。

福康安五月庚戌,復以太子太保、三等嘉勇男、署工部尚書在軍機處行走。十二月,差赴廣東勘事。

四十九年甲辰

阿桂五月,命討固原叛回。八月,還。尋差督辦河工。十二月,還。

梁國治

福隆安三月己酉,卒。

和珅七月,轉吏部尚書、協辦大學士。九月,封一等男。

福康安閏三月,轉兵部尚書。五月戊辰,授陜甘總督。出。

福長安

董誥

慶桂五月丁巳,復以工部尚書在軍機處行走。旋轉兵部尚書。十一月,差赴山東等處勘事。

五十年乙巳

阿桂八月,差赴江南勘河。十一月,還。

梁國治五月,授東閣大學士。

和珅

慶桂九月己酉,命署陜甘總督。

福長安

董誥

五十一年丙午

阿桂四月,差赴江南籌辦河工。十月,還。

梁國治十二月壬子,卒。

和珅閏七月,授文華殿大學士。

慶桂九月,內召。十二月,還。

福長安閏七月,遷戶部尚書。

董誥

王傑十二月壬子,以兵部尚書在軍機處行走。

五十二年丁未

阿桂六月,差赴睢州籌辦河工。十月,命轉勘江南高堰河工。

和珅

慶桂十一月,差赴湖北勘事。十二月,命署盛京將軍。

福長安十二月,轉工部尚書。

王傑正月,授東閣大學士。

董誥正月,遷戶部尚書。

五十三年戊申

阿桂正月,還。七月,差赴荊州勘水。十月,還。

和珅二月,晉三等忠襄伯。

王傑

慶桂十月,命署吉林將軍。

董誥

福長安

五十四年己酉

阿桂四月,差赴荊州勘工。八月,還。

和珅

王傑

慶桂四月,命署烏里雅蘇臺將軍。

董誥

福長安

孫士毅六月庚午,以太子太保、兵部尚書在軍機處行走。十一月癸巳,命署四川總督。出。

五十五年庚戌

阿桂

和珅正月,賜用黃帶。

王傑十一月,加太子太保。

慶桂

董誥十一月,加太子少保。

福長安十一月,加太子少保。

五十六年辛亥

阿桂

和珅

王傑

慶桂三月,丁母憂,給假。

董誥

福長安十月,轉戶部尚書。

五十七年壬子

阿桂

和珅

王傑

福長安

慶桂十二月,差赴浙江鞫案。

董誥

五十八年癸丑

阿桂

和珅

王傑

福長安

慶桂四月己卯,授荊州將軍。出。

董誥

松筠四月庚寅,以戶部左侍郎在軍機處行走。九月,差送英吉利貢使馬嘎爾呢赴■。

五十九年甲寅

阿桂

和珅

王傑

福長安

董誥

松筠正月丁酉,差赴盛京勘案。旋命署吉林將軍。出。

六十年乙卯

阿桂

和珅

王傑

福長安

董誥

臺布九月庚申,以內閣學士在軍機處學習行走。旋遷工部左侍郎。

嘉慶元年丙辰

阿桂

和珅

王傑十月,病假。

福長安

董誥十月,授東閣大學士。

臺布六月,轉戶部右侍郎。十一月,差赴浙江、江西勘事。

沈初十月己卯,以左都御史在軍機處學習行走。旋遷兵部尚書。

二年丁巳

阿桂八月丁巳,卒。

和珅

王傑閏六月壬戌,罷。

董誥二月,憂免。

福長安

沈初三月,轉吏部尚書。八月,轉戶部尚書。

臺布正月丙午,命署江西巡撫。出。

傅森閏六月壬戌,以兵部右侍郎在軍機處學習行走。十月,轉戶部右侍郎。

戴衢亨閏六月壬戌,以侍講學士加三品卿銜,在軍機處學習行走。

吳熊光閏六月壬戌,以通政使司參議加三品卿銜在軍機處學習行走。十二月壬戌,授直隸布政使。出。

三年戊午

和珅八月,晉一等忠襄公。

福長安八月,封侯。

沈初

傅森二月乙卯,命回部辦事。罷直。

戴衢亨正月,遷內閣學士。二月,遷禮部右侍郎。七月,轉戶部右侍郎。

那彥成二月乙卯,以內閣學士在軍機處學習行走。五月,遷工部右侍郎。

四年己未

和珅正月丁卯,革職,逮獄。

福長安正月丁卯,革職,逮獄。

沈初正月丁卯,以年老罷直。

戴衢亨正月丁卯,申命仍留軍機處行走。

那彥成正月丁卯,申命仍留軍機處行走。旋轉戶部右侍郎,遷工部尚書。八月,加欽差大臣,赴陜西督辦

軍務。

成親王永瑆正月丁卯,命在軍機處行走。旋署戶部尚書。十月丁未,以非祖制罷直。

董誥正月丁卯,復以太子少保、前任大學士、署刑部尚書在軍機處行走。二月,晉太子太保。五月,授文華殿大學士。九月,晉太子太傅。

慶桂正月丁卯,復以兵部尚書在軍機處行走。旋轉刑部尚書、協辦大學士。二月,加太子太保。三月,授文淵閣大學士,晉太子太傅。

傅森十月丁未,復以兵部尚書在軍機處行走。

五年庚申

慶桂

董誥

傅森三月,差赴盛京勘事。四月,還。

那彥成督辦陜西軍務。閏四月戊辰,以辦賊不力,免直。

戴衢亨正月,轉戶部左侍郎。

六年辛酉

慶桂

董誥

傅森正月,轉戶部尚書。二月,卒。

戴衢亨

成德二月癸酉,以戶部尚書在軍機處學習行走。

七年壬戌

慶桂十二月,賜世職。

董誥十二月,賜世職。

成德三月,卒。

戴衢亨七月,遷兵部尚書。十二月,加太子少保,賜世職。

劉權之六月甲寅,以吏部尚書在軍機處學習行走。

德瑛六月甲寅,以刑部尚書在軍機處學習行走。

八年癸亥

慶桂

董誥

劉權之

戴衢亨六月,轉工部尚書。

德瑛

九年甲子

慶桂

董誥

劉權之六月,轉兵部尚書。

戴衢亨

德瑛正月,差赴山東勘事。六月戊辰,轉吏部尚書,命專管部務。罷直。

那彥成六月戊辰,復以禮部尚書在軍機處行走。乙亥,命署陜西總督。出。

英和六月戊辰,以太子少保、戶部左侍郎在軍機處學習行走。

十年乙丑

慶桂

董誥

劉權之二月,轉禮部尚書、協辦大學士。六月辛巳,降級,免直。

戴衢亨正月,轉戶部尚書。

英和六月辛巳,以事革宮銜,降級,免直。

托津閏六月壬午,以吏部左侍郎在軍機處學習行走。九月,差赴湖北、廣東勘事。

十一年丙寅

慶桂

董誥

戴衢亨

托津正月,轉戶部左侍郎。四月,差赴河南讞獄。十二月,差赴天津讞獄。

十二年丁卯

慶桂三月,賜用紫韁。

董誥

戴衢亨五月,協辦大學士。

托津七月,差赴熱河讞獄。

十三年戊辰

慶桂

董誥

戴衢亨三月,差赴南河勘工,並給假歸籍省墓。

托津十月,差勘南河海口。

英和閏五月丙寅,復以工部左侍郎暫在軍機大臣上學習行走。尋罷直。

十四年己巳

慶桂正月,晉太子太師。

董誥正月,晉太子太師。

戴衢亨正月,晉太子少師。七月,轉工部尚書。

托津正月,差赴江蘇讞獄。八月,差赴浙江按事。

十五年庚午

慶桂

董誥

戴衢亨五月,授體仁閣大學士。

托津正月,差赴山西勘事。二月,遷工部尚書。旋差赴四川勘事。五月,轉戶部尚書。六月,還。十一月,差赴揚州勘事。

十六年辛未

慶桂

董誥

戴衢亨四月,卒。

托津正月,暫署兩江總督。六月,加太子少保。

方維甸四月己酉,召原任閩浙總督為軍機大臣。以母病不至。癸酉,許在籍終養。

盧廕溥七月戊寅,以光祿寺少卿加四品卿銜在軍機大臣上學習行走。旋遷通政司副使。

十七年壬申

慶桂正月,晉太保。九月甲午,以年老罷直。

董誥正月,晉太保。

托津

盧廕溥十一月,轉通政司正使。十二月,遷內閣學士。

松筠九月甲午,復以太子少保、協辦大學士、吏部尚書在軍機大臣上行走。十月,差赴南河勘事。

十八年癸酉

董誥

松筠正月乙亥,罷直。

托津九月,協辦大學士。十月,差赴河南勘事。十二月,還。

盧廕溥三月,遷兵部右侍郎。八月,轉左侍郎。九月,轉戶部左侍郎。

勒保正月乙亥,以太子太保、一等威勤伯、武英殿大學士在軍機大臣上行走。十月,病假。

桂芳十月甲寅,以戶部右侍郎暫在軍機大臣上學習行走。

十九年甲戌

董誥

勒保閏二月甲子,乞病,罷直。

托津八月,授東閣大學士。九月,晉太子太保。十一月,差赴江南勘事。

盧廕溥九月,差赴河南勘事。十一月,還。

桂芳閏二月,差往廣西勘事。三月癸卯,授漕運總督。出。

英和十一月丁未,復以吏部尚書暫在軍機大臣上行走。尋罷直。

二十年乙亥

董誥

托津

盧廕溥

二十一年丙子

董誥

托津六月,差赴天津勘事,暫署直隸總督。旋還。

盧廕溥六月,轉戶部右侍郎。

章煦十月己亥,以太子少保、協辦大學士、禮部尚書在軍機大臣上行走。十一月,轉刑部尚書。

二十二年丁丑

董誥

托津

章煦二月,病假。三月辛未,罷。

盧廕溥三月,遷禮部尚書,轉兵部尚書。六月,加太子少保。九月,轉戶部尚書。

二十三年戊寅

董誥二月乙亥,致仕。

托津

盧廕溥

戴均元二月辛未,以太子少保、協辦大學士、吏部尚書在軍機大臣上學習行走。

和瑛二月辛未,以太子少保、兵部尚書在軍機大臣上學習行走。三月,差赴保定勘事。

二十四年己卯

托津正月,賜用紫韁。

戴均元十月,差赴河南勘事。

盧廕溥

和瑛正月丁巳,轉刑部尚書,命專任部務。罷直。

文孚正月丁巳,以刑部右侍郎在軍機大臣上學習行走。六月,差赴江南勘事。

二十五年庚辰

托津九月庚申,以撰遺詔錯誤免直。

戴均元二月,授文淵閣大學士,晉太子太保。九月庚申,以撰遺詔錯誤免直。

盧廕溥九月,以撰遺詔錯誤降級留任,仍在軍機大臣上行走,轉工部尚書。

文孚二月,差赴甘肅勘事。三月,轉戶部左侍郎。九月,以撰遺詔錯誤降級留任,仍在軍機大臣上行走,轉工部右侍郎。十一月,遷左都御史。

曹振鏞九月庚申,以太子太保、體仁閣大學士在軍機大臣上行走。

黃鉞九月庚申,以太子少保、戶部尚書在軍機大臣上行走。

英和九月庚申,復以吏部尚書在軍機大臣上行走。十月,轉戶部尚書。十二月乙巳,以言事忤旨免直。


\end{pinyinscope}