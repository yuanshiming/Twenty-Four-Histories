\article{志一}

\begin{pinyinscope}
天文一

歷代天文志,自史記天官書後,唯晉、隋兩志,備述天體、儀象、星占,唐、宋加詳,皆未盡也。至元,景測益精明,占候較密,然疆宇所囿,聲教未宏,齊政窺璣,尚多略焉。有清統一區夏,聖聖相承。聖祖親釐象數,究極精微,前後制新儀七,測日月星辰,則窮極分秒;度輿圖經緯,則穹歷幅隕。世宗復以歲久積差,準監臣改用橢圓術。高宗又以舊記星紀,間有疏漏,禦制璣衡撫辰儀,重加測候。迨平定回疆及兩金川,復令重度裡差,增入時憲。理明數確,器精法密,自古以來,所未有也。今為天文志,備載推驗之法,其天象昭垂,見於歷朝實錄及所司載記者,亦悉書之。乾隆六十年以後,國史無徵,則從闕焉。

天象地體裡差

天象歷象考成天象篇云:「楚辭天問曰:『圜則九重,孰營度之?』後世歷家,謂天有十二重,非天實有如許重數,蓋言日月星辰運轉於天,各有所行之道,即楚辭所謂圜也。欲明諸圜之理,必詳諸圜之動,欲考諸圜之動,必以至靜不動者準之,然後得其盈縮。蓋天道靜專者也,天行動直者也。至靜者自有一天,與地相為表裏,故群動者運於其間而不息。若無至靜者以驗至動,則聖人亦無所成其能矣。人恆在地面測天,而七政之行無不可得者,正為以靜驗動故也。

「十二重天,最外者為至靜不動;次為宗動,南北極赤道所由分也。次為南北歲差;次為東西歲差;此二重天,其動甚微,歷家姑置之而不論焉。次為三垣二十八宿,經星行焉。次為填星所行;次為歲星所行;次為熒惑所行;次則太陽所行,黃道是也。次為太白所行;次為辰星所行;最內者則太陰所行,白道是也。要以去地之遠近而為諸天之內外,然所以知去地之遠近者,則又從諸曜之掩食及行度之遲疾而得之。蓋凡為所掩食者必在上,而掩之食之者必在下。月體能蔽日光而日為之食,是日遠月近之徵也。月能掩食五星,而月與五星又能掩食恆星,是五星高於月而卑於恆星也。五星又能互相掩食,是五星各有遠近也。

「又宗動天以渾灝之氣挈諸天左旋,其行甚速。故近宗動天者,左旋速而右移之度遲。漸遠宗動天,則左旋較遲而右移之度轉速。今右移之度,惟恆星最遲,土木次之,火又次之。日、金、水較速而月最速,是又以次而近之證也。」

考成後編日躔歷理云:「西法自多祿某以至第谷,立為本天高卑、本輪、均輪諸說,近世刻白爾、噶西尼等,又以本天為橢圓。」月離歷理云:「自西人創為橢圓之法,日距月天最高有遠近,則太陰本天心有進退。地心與天心相距,兩心差有大小。」合觀諸論,天象備矣。

恆星天無地半徑差及次輪消息,故志土星以下七天距地心數,著考測之詳焉。

諸天距地心數:

土星最高一十一又一百零四萬二千六百分之三十五萬二千六百日天半徑;

木星最高六又一百九十二萬九千四百八十分之一百三十萬五千九百日天半徑;

火星最高二又六百三十萬二千七百五十分之五百五十五萬二千二百五十日天半徑;

日均輪術最高一千一百六十二地半徑,橢圓術最高二萬零九百七十五地半徑;

金星最高高於日一千萬分日天半徑之七百五十四萬五千六百四十四,最下下於日如之;

水星最高高於日一千萬分日天半徑之四百五十三萬二千一百五十五,最下下於日如之;

月均輪術最高朔望時五十八又百分之一十六地半徑,橢圓術最高六十三又百分之七十七地半徑。

地體渾天家謂天包地如卵裹黃,內經:「黃帝曰:『地之為下否乎?』岐伯曰:『地為人之下,太虛之中也。』曰:『憑乎?』曰:『大義舉之也。』」大戴禮:「單居離問於曾子曰:『天圓而地方,誠有之乎?』曾子曰:『如誠天圓而地方,則是四角之不掩也。參嘗聞諸夫子曰:「天道曰圓,地道曰方。」』」宋儒邵子曰:「天何依?依乎地;地何附?附乎天。天地何所依附?自相依附。自相依附,天依形,地附氣。」程子曰:「據日景以三萬里為中,若有窮,然有至一邊已及一萬五千里,而天地之運蓋如初。然則中者亦時中耳。」又曰:「今人所定天體,只是且以眼定,視所極處不見,遂以為盡。然向曾有於海上見南極下有大星十,則今所見天體蓋未定。日月升降,不過三萬里中,然而中國只到鄯善、沙車,已是一萬五千里。若就彼觀日,尚只是三萬里中也。伯淳在澤州,嘗三次食韭黃,始食懷州韭,次食澤州,次食並州,則知數百里間,氣候已爭三月矣。若都以此差之,則須爭半歲。如是,則有在此冬至、在彼夏至者,只是一般為冬夏而已。」硃子天問注云:「天之形圓如彈丸,其運轉者亦無形質,但如勁風之旋。地則氣之渣滓聚成形質者,但以其束於勁風旋轉之中,故得以兀然浮空甚久而不墮耳。」西人謂地體渾圓,四面皆有人,冬夏互異,晝夜相反,與內經、戴記及宋儒之言若合符節。今以天周三百六十度徵之,南行二百里,則北極低一度;北行二百里,則北極高一度。東西當赤道下行二百里,則見月食之早晚亦差一度。其在赤道南北緯圈下行,雖廣狹不同,然莫不應乎渾象。則知地之大周皆三百六十度,東西南北皆周七萬二千里,以古尺八寸計之,則周九萬里;以圍三徑一率之,則徑三萬里;亦與古三萬里為中之說相符。然則地體渾圓,無疑義矣。距緯應大周里數不同,為志其要。

赤道南北距緯東西每度相距里數:

距緯一度,一百九十九里三百四十步;

距緯五度,一百九十九里八十步;

距緯十度,一百九十六里三百四十步;

距緯十五度,一百九十三里六十步;

距緯二十度,一百八十七里三百二十步;

距緯二十五度,一百八十一里八十步;

距緯三十度,一百七十三里六十步;

距緯三十五度,一百六十三里二百八十步;

距緯四十度,一百五十三里八十步;

距緯四十五度,一百四十一里一百二十步;

距緯五十度,一百二十八里二百步;

距緯五十五度,一百一十四里二百四十步;

距緯六十度,九十九里三百四十步;

距緯六十五度,八十四里二百步;

距緯七十度,六十八里一百四十步;

距緯七十五度,五十一里二百四十步;

距緯八十度,三十四里一百六十步;

距緯八十五度,一十七里八十步;

距緯八十九度,三里一百六十步。

裡差者,因人所居有南北東西之不同,則天頂地平亦異,可以計里而定,故名裏差,其所關於仰觀甚鉅。蓋恆星之隱見,晝夜之永短,七曜之出沒,節氣之早晚,交食之深淺先後,莫不因之而各殊。惟得其所差之數,則各殊之故,皆可豫知,不致詫為失行而生飾說矣。新法算書所載各省北極高及東西偏度,大概據輿圖道里定之,多有未確。今以康熙年間實測各省及諸蒙古高度、偏度,並乾隆時憲所增省分,與回疆部落、兩金川土司等,晝夜永短,節氣早晚,推得高度、偏度備列焉。

北極高度:

京師高三十九度五十五分;

盛京高四十一度五十一分;

山西高三十七度五十三分三十秒;

朝鮮高三十七度三十九分十五秒;

山東高三十六度四十五分二十四秒;

河南高三十四度五十二分二十六秒;

陜西高三十四度十六分;

江南高三十二度四分;

四川高三十度四十一分;

湖廣高三十度三十四分四十八秒;

浙江高三十度十八分二十秒;

江西高二十八度三十七分十二秒;

貴州高二十六度三十分二十秒;

福建高二十六度二分二十四秒;

廣西高二十五度十三分七秒;

雲南高二十五度六分;

廣東高二十三度十分;

布壟堪布爾嘎蘇泰高四十九度二十八分;

額格塞楞格高四十九度二十七分;

桑錦達賚湖高四十九度十二分;

肯特山高四十八度三十三分;

克嚕倫河巴爾城高四十八度五分三十秒;

圖拉河汗山高四十七度五十七分十秒;

喀爾喀河克勒和碩高四十七度三十四分三十秒;

杜爾伯特高四十七度十五分;

鄂爾坤河額爾得尼昭高四十六度五十八分十五秒;

崆格扎布韓堪河高四十六度四十二分;

扎賚特高四十六度三十分;

推河高四十六度二十九分二十秒;

科爾沁高四十六度十七分;

郭爾羅斯高四十五度三十分;

阿嚕科爾沁高四十五度三十分;

翁吉河高四十五度三十分;

薩克薩克圖古里克高四十五度二十三分四十五秒;

烏硃穆沁高四十四度四十五分;

浩齊特高四十四度六分;

固爾班賽堪高四十三度四十八分;

巴林高四十三度三十六分;

扎嚕特高四十三度三十分;

阿巴哈納爾高四十三度二十三分;

阿巴噶高四十三度二十三分;

奈曼高四十三度十五分;

克什克騰高四十三度;

蘇尼特高四十三度;

哈密高四十二度五十三分;

翁牛特高四十二度三十分;

敖漢高四十二度十五分;

喀爾喀高四十一度四十四分;

四子部落高四十一度四十一分;

喀喇沁高四十一度三十分;

茂明安高四十一度十五分;

烏喇特高四十度五十二分;

歸化城高四十度四十九分;

土默特高四十度四十九分;

鄂爾多斯高三十九度三十分;

阿拉善山高三十八度三十分。

右康熙年間實測。

雅克薩城高五十一度四十八分;

黑龍江高五十度一分;

三姓高四十七度二十分;

伯都訥高四十五度十五分;

吉林高四十三度四十七分;

甘肅高三十六度八分;

安徽高三十度三十七分;

湖南高二十八度十三分;

越南高二十二度十六分;

阿勒坦淖爾烏梁海高五十三度三十分;

汗山哈屯河高五十一度十分;

唐努山烏梁海高五十度四十分;

烏蘭固木杜爾伯特高四十九度二十分;

額爾齊斯河高四十八度三十五分;

齋桑淖爾高四十八度三十五分;

阿勒臺山烏梁海高四十八度三十分;

阿勒輝山高四十八度二十分;

科布多城高四十八度二分;

烏里雅蘇臺城高四十七度四十八分;

哈薩克高四十七度三十分;

塔爾巴哈臺高四十七度;

布勒罕河土爾扈特高四十七度;

巴爾噶什淖爾高四十七度;

烏隴古河高四十六度四十分;

赫色勒巴斯淖爾高四十六度四十分;

和博克薩哩土爾扈特高四十六度四十分;

扎哈沁高四十六度三十分;

齋爾土爾扈特高四十六度十分;

哈布塔克高四十五度;

吹河高四十四度五十分;

博羅塔拉高四十四度五十分;

拜達克高四十四度四十三分;

晶河土爾扈特高四十四度三十五分;

庫爾喀喇烏蘇土爾扈特高四十四度三十分;

安濟海高四十四度十三分;

哈什高四十四度八分;

伊犁高四十三度五十六分;

塔拉斯河高四十三度五十分;

穆壘高四十三度四十五分;

濟木薩高四十三度四十分;

巴里坤高四十三度三十九分;

崆吉斯高四十三度三十三分;

烏魯木齊高四十三度二十七分;

珠勒都斯高四十三度十七分;

吐魯番高四十三度四分;

塔什幹高四十三度三分;

和碩特高四十三度;

那林山高四十三度;

特穆爾圖淖爾高四十二度五十分;

魯克沁高四十二度四十八分;

烏沙克塔勒高四十二度十六分;

哈喇沙爾高四十二度七分;

庫爾勒高四十一度四十六分;

布爾古高四十一度四十四分;

賽哩木高四十一度四十一分;

納木干高四十一度三十八分;

庫車高四十一度三十七分;

布嚕特高四十一度二十八分;

安集延高四十一度二十八分;

霍罕高四十一度二十三分;

阿克蘇高四十一度九分;

烏什高四十一度六分;

鄂什高四十度十九分;

喀什噶爾高三十九度二十五分;

巴爾楚克高三十九度十五分;

英吉沙爾高三十八度四十七分;

葉爾羌高三十八度十九分;

斡罕高三十八度;

色埒庫勒高三十七度四十八分;

喀楚特高三十七度十一分;

哈喇哈什高三十七度十分;

克里雅高三十七度;

和闐高三十七度;

伊里齊高三十七度;

博羅爾高三十七度;

三珠高三十六度五十八分;

玉隴哈什高三十六度五十二分;

鄂囉善高三十六度四十九分;

什克南高三十六度四十七分;

巴達克山高三十六度二十三分;

三雜谷高三十二度一分;

黨壩高三十一度五十六分;

綽斯甲布高三十一度五十三分;

金川勒烏圍高三十一度三十四分;

金川噶拉依高三十一度十九分;

瓦寺高三十一度十七分;

革布什咱高三十一度八分;

布拉克底高三十一度四分;

小金川美諾高三十一度;

巴旺高三十度五十八分;

沃克什高三十度五十六分;

明正高三十度二十八分;

木坪高三十度二十五分;

右乾隆時憲所增。

東西偏度:

盛京偏東七度十五分;

浙江偏東三度四十一分二十四秒;

福建偏東二度五十九分;

江南偏東二度十八分;

山東偏東二度十五分;

江西偏西三十七分;

河南偏西一度五十六分;

湖廣偏西二度十七分;

廣東偏西三度三十三分十五秒;

山西偏西三度五十七分四十二秒;

廣西偏西六度十四分四十秒;

陜西偏西七度三十三分四十秒;

貴州偏西九度五十二分四十秒;

四川偏西十二度十六分;

雲南偏西十三度三十七分;

朝鮮偏東十度三十分;

郭爾羅斯偏東八度十分;

扎賴特偏東七度四十五分;

杜爾伯特偏東六度十分;

扎嚕特偏東五度;

奈曼偏東五度;

科爾沁偏東四度三十分;

敖漢偏東四度;

阿祿科爾沁偏東三度五十分;

喀爾喀河克勒和邵偏東二度四十六分;

巴林偏東二度十四分;

喀喇沁偏東二度;

翁牛特偏東二度;

烏硃穆秦偏東一度十分;

克什克騰偏東一度十分;

蒿齊忒偏東三十分;

阿霸哈納爾偏東二十八分;

阿霸垓偏東二十八分;

蘇尼特偏西一度二十八分;

克魯倫河巴拉斯城偏西二度五十二分;

四子部落偏西四度二十二分;

歸化城偏西四度四十八分;

土默特偏西四度四十八分;

喀爾喀偏西五度五十五分;

毛明安偏西六度九分;

吳喇忒偏西六度三十分;

肯忒山偏西七度三分;

鄂爾多斯偏西八度;

圖拉河韓山偏西九度十二分;

翁機河偏西十一度;

固爾班賽堪偏西十一度;

布龍看布爾嘎蘇泰偏西十一度二十二分;

阿蘭善山偏西十二度;

厄格塞楞格偏西十二度二十五分;

鄂爾昆河厄爾德尼招偏西十三度五分;

推河偏西十五度十五分;

桑金答賴湖偏西十六度二十分;

薩克薩圖古里克偏西十九度三十分;

空格衣扎布韓河偏西二十度十二分;

哈密城偏西二十二度三十二分。

右康熙年間實測。

三姓偏東十三度二十分;

黑龍江偏東十度五十八分:

吉林偏東十度二十七分;

伯都訥偏東八度三十七分;

安徽偏東三十四分;

雅克薩城偏西十七分;

湖南偏西三度四十二分;

越南偏西十度;

甘肅偏西十二度三十六分;

烏里雅蘇臺城偏西二十二度四十分;

巴里坤偏西二十三度;

扎哈沁偏西二十三度十分;

唐努山烏梁海偏西二十四度二十分;

哈布塔克偏西二十四度二十六分;

拜達克偏西二十五度;

穆壘偏西二十五度三十六分;

烏蘭固木杜爾伯特偏西二十五度四十分;

魯克沁偏西二十六度十一分;

吐魯番偏西二十六度四十五分;

濟木薩偏西二十六度五十二分;

科布多城偏西二十七度二十分;

烏魯木齊偏西二十七度五十六分;

布勒罕河土爾扈特偏西二十八度十分;

烏沙克塔勒偏西二十八度二十六分;

阿勒臺山烏梁海偏西二十八度三十五分;

阿勒坦淖爾烏梁海偏西二十八度四十分;

汗山哈屯河偏西二十九度;

烏隴古河偏西二十九度十五分;

赫色勒巴斯淖爾偏西二十九度十五分;

哈喇沙爾偏西二十度十七分;

庫爾勒偏西二十九度五十六分;

塔爾巴哈臺偏西三十度;

珠勒都斯偏西三十度五十分;

安濟海偏西三十度五十四分;

和碩特偏西三十一度;

和博克薩哩土爾扈特偏西三十一度十五分;

庫爾喀喇烏蘇土爾扈特偏西三十一度五十六分;

崆吉斯偏西三十二度;

布古爾偏西三十二度七分;

額爾齊斯河偏西三十二度二十五分;

齋桑淖爾偏西三十二度二十五分;

哈什偏西三十三度;

齋爾土爾扈特偏西三十三度;

博囉塔拉偏西三十三度三十分;

晶河土爾扈特偏西三十三度三十分;

庫車偏西三十三度三十二分;

克里雅偏西三十三度三十三分;

伊犁偏西三十四度二十分;

賽哩木偏西三十四度四十分;

哈薩克偏西三十四度五十分;

玉隴哈什偏西三十五度三十七分;

和闐偏西三十五度五十二分;

伊里齊偏西三十五度五十二分;

哈喇哈什偏西三十六度十四分;

阿勒輝山偏西三十六度五十分;

阿克蘇偏西三十七度十五分;

三珠偏西三十七度四十七分;

巴爾噶什淖爾偏西三十八度十分;

烏什偏西三十八度二十七分;

特穆爾圖淖爾偏西三十九度二十分;

巴爾楚克偏西三十九度三十五分;

葉爾羌偏西四十度十分;

英吉沙爾偏西四十一度五十分;

吹河偏西四十二度;

喀什噶爾偏西四十二度二十五分;

色埒庫勒偏西四十二度二十四分;

喀楚特偏西四十二度三十二分;

鄂什偏西四十二度五十分;

博羅爾偏西四十三度三十八分;

巴達克山偏西四十三度五十分;

塔拉斯河偏西四十四度;

布嚕特偏西四十四度三十五分;

安集延偏西四十四度三十五分;

什克南偏西四十四度四十六分;

那林山偏西四十五度;

斡罕偏西四十五度九分;

鄂囉善偏西四十五度二十六分;

納木干偏西四十五度四十分;

霍罕偏西四十五度五十六分;

塔什乾偏西四十七度四十三分;

瓦寺偏西十二度五十八分;

木坪偏西十三度三十七分;

沃克什偏西十三度五十一分;

三雜穀偏西十三度五十六分;

小金川美諾偏西十四度七分;

布拉克底偏西十四度二十二分;

金川噶拉依偏西十四度二十九分;

黨壩偏西十四度二十九分;

金川勒烏圍偏西十四度三十四分;

巴旺偏西十四度三十四分;

綽斯甲布偏西十四度四十四分;

明正偏西十四度四十九分;

革布什咱偏西十四度五十一分。

右乾隆時憲所增。


\end{pinyinscope}