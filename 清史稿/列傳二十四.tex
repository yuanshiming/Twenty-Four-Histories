\article{列傳二十四}

\begin{pinyinscope}
洪承疇夏成德孟喬芳張文衡張存仁

洪承疇,字亨九,福建南安人。明萬歷四十四年進士。累遷陜西布政使參政。崇禎初,流賊大起,明莊烈帝以承疇能軍,遷延綏巡撫、陜西三邊總督,屢擊斬賊渠,加太子太保、兵部尚書,兼督河南、山、陜、川、湖軍務。時諸賊渠高迎祥最強,號闖王,李自成屬焉,承疇與戰,敗績。莊烈帝擢盧象升總理河北、河南、山、陜、川、湖軍務,令承疇專督關中,復與自成戰臨潼,大破之,迎祥就俘。自成號闖王,分道入四川,承疇與屢戰輒勝。自成還走潼關,承疇使總兵曹變蛟設伏邀擊,自成大敗,以十八騎走商洛。關中賊略盡。是歲為崇德三年。

太宗伐明,師薄明都,莊烈帝徵承疇入衛。明年春,移承疇總督薊、遼軍務,帥秦兵以東,授變蛟東協總兵、王廷臣遼東總兵、白廣恩援剿總兵,與山海馬科、寧遠吳三桂二鎮合軍;復命宣府楊國柱、大同王樸、密雲唐通各以其兵至:凡八總兵,兵十三萬,馬四萬,咸隸承疇。太宗師下大凌河,祖大壽入錦州為明守,松山、杏山、塔山三城相與為犄角。承疇至軍,莊烈帝遣職方郎中張若麒趣戰,乃進次松山,國柱戰死,以山西總兵李輔明代。

六年八月,太宗自將御之。上度松山、杏山間,自烏忻河南山至海,當大道立營。承疇及遼東巡撫邱民仰率諸將駐松山城北乳峰山,步兵分屯乳峰山至松山道中為七營,馬兵分屯松山東、西、北三方,戰敗,移步兵近松山城為營,復戰又敗。上誡諸將曰:「今夕明師其遁!」命諸軍當分地為汛以守,敵遁,視其眾寡,遣兵追擊,至塔山而止;分遣諸將截塔山、杏山道及桑噶爾寨堡,又自小凌河西直抵海濱,絕歸路。是夜三桂、樸、通、科、廣恩、輔明皆率所部循海引退,為我師掩殺,死者不可勝計。承疇、民仰率將吏入松山城守,上移軍松山,議合圍。變蛟夜棄乳峰山寨,悉引所部馬步兵犯鑲黃旗汛地者一,犯正黃旗汛地者四,直攻上營,殊死戰,變蛟中創,奔還松山。三桂、樸引餘兵入杏山。上遣諸將為伏於高橋及桑噶爾寨堡,明兵自杏山出奔寧遠,遇伏,殪強半。三桂、樸僅以身免。承疇師十三萬,死五萬有奇,諸將潰遁,惟變蛟、廷臣以殘兵萬餘從。

城圍既合,上以敕諭承疇降。九月,上還盛京,命貝勒多鐸等留護諸軍。承疇悉眾突圍,攻鑲黃旗擺牙喇阿禮哈超哈,戰敗,不能出。十月,命肅郡王豪格、公滿達海駐松山。十二月,承疇聞關內援師且至,復遣將以兵六千夜出攻正紅旗擺牙喇阿禮哈超哈及正黃旗蒙古營,戰敗,城閉不得入,強半降我師。餘眾潰走杏山,道遇伏,死。莊烈帝初以楊繩武督師援承疇,繩武卒,以範志完代,皆畏我師強,宿留不進。承疇被圍閱六月,食且盡。明年二月,松山城守副將夏成德使其弟景海通款,以子舒為質。我師夜就所守堞樹雲梯,阿山部卒班布里、何洛會部卒羅洛科先登,遂克其城,獲承疇、民仰、變蛟、廷臣及諸將吏,降殘卒三千有奇。時為崇德七年二月壬戍。上命殺民仰、變蛟、廷臣,而送承疇盛京。

上欲收承疇為用,命範文程諭降。承疇方科跣謾罵,文程徐與語,泛及今古事,梁間塵偶落,著承疇衣,承疇拂去之。文程遽歸,告上曰:「承疇必不死,惜其衣,況其身乎?」上自臨視,解所御貂裘衣之,曰:「先生得無寒乎?」承疇瞠視久,嘆曰:「真命世之主也!」乃叩頭請降。上大悅,即日賞賚無算,置酒陳百戲,諸將或不悅,曰:「上何待承疇之重也!」上進諸將曰:「吾曹櫛風沐雨數十年,將欲何為?」諸將曰:「欲得中原耳。」上笑曰:「譬諸行道,吾等皆瞽。今獲一導者,吾安得不樂?」

居月餘,都察院參政張存仁上言:「承疇歡然幸生,宜令薙發備任使。」五月,上御崇政殿,召承疇及諸降將祖大壽等入見。承疇跪大清門外,奏言:「臣為明將兵十三萬援錦州,上至而兵敗。臣入守松山,城破被獲,自分當死,上不殺而恩育焉。今令朝見,臣知罪,不敢遽入。」上使諭曰:「承疇言誠是。爾時與我交戰,各為其主,朕豈介意?且朕所以戰勝明兵,遂克松山、錦州諸城,皆天也。天道好生,故朕亦恩爾。爾知朕恩,當盡力以事朕。朕昔獲張春,亦嘗遇以恩,彼不能死明,又不能事朕,卒無所成而死,爾毋彼若也!」承疇等乃入朝見,命上殿坐,賜茶。上語承疇曰:「朕觀爾明主,宗室被俘,置若罔聞。將帥力戰見獲,或力屈而降,必誅其妻子,否亦沒為奴。此舊制乎,抑新制乎?」承疇對曰:「舊無此制。邇日諸朝臣各陳所見以聞於上,始若此爾。」上因嘆謂:「君闇臣蔽,遂多枉殺。將帥以力戰沒敵,斥府庫財贖而還之可也,奈何罪其孥?其虐無辜亦甚矣!」承疇垂涕叩首曰:「上此諭真至仁之言也!」上還宮,命宴承疇等於殿上。宴畢,使大學士希福等諭曰:「朕方有元妃之喪,未躬賜宴。爾等勿以為意!」承疇等復叩首謝。莊烈帝初聞承疇死,予祭十六壇,建祠都城外,與邱民仰並列。莊烈帝將親臨奠,俄聞承疇降,乃止。承疇既降,隸鑲黃旗漢軍,太宗遇之厚。然終太宗世,未嘗命以官。

順治元年四月,睿親王多爾袞帥師伐明,承疇從。既定京師,命承疇仍以太子太保、兵部尚書兼右副都御史,同內院官佐理機務。旋與同官馮銓啟睿親王,復明內閣故事,題奏皆下內閣擬旨,分下六科,鈔發各部院。九月,上至京師,與銓及謝升奏定郊廟樂章。

二年,豫親王多鐸師下江南。閏六月,命承疇以原官總督軍務,招撫江南各省,鑄「招撫南方總督軍務大學士」印,賜敕便宜行事。是時明唐王聿鍵稱號福建,其大學士黃道周率師道廣信、衢州向徽州,左僉都御史金聲家休寧,募鄉兵十餘萬屯績溪;諸宗姓高安王常淇保徽州,蘄水王術肸子常闒自號樊山王屯潛山、太湖間,由揾號金華王據饒州,誼石號樂安王、誼泐號瑞安王分屯溧陽、金壇、興化諸縣;荊本徹以舟師駐太湖,敗,復入崇明:皆為明守。承疇至官,招撫江南寧國、徽州,江西南昌、南康、九江、瑞州、撫州、饒州、臨江、吉安、廣信、建昌、袁州諸府。十月,遣提督張天祿,總兵卜從善、李仲興、劉澤泳等攻破績溪。十二月,進破道周於婺源,聲、道周見獲,皆不屈,送江寧殺之;總兵李成棟破崇明,本徹走入海,殺其將李守庫、徐君美。三年二月,遣總兵馬得功、卜從善等擊破司空寨,斬守寨石應璉、應璧等五人,獲常闒。

既,誼石、誼泐合兵二萬犯江寧。承疇先事誅內應西溝池萬德華、郭世彥、尤琚等八十餘人。誼石等攻神策門,我分兵出朝陽、太平二門,截誼石等後,乃啟神策門出城兵奮擊,破之,追及攝山,斬馘無算。承疇疏請還京,以江南未大定,不允,賜其妻白金百、貂皮二百。八月,征南大將軍貝勒博洛克金華,獲誼石。九月,誼泐復犯江寧,承疇出御,追獲誼泐及所置經略韋爾韜、總兵楊三貫、夏含章。十二月,天祿搜婺源嚴杭山,獲常淇及所置監軍道江於東、職方司許文玠等。四年二月,從善及總兵黃鼎攻宿松,獲誼泐弟瑞昌王誼貴及所置軍師趙正;下饒州,獲由揾及其族人常洊、常沘、常涫:並請命斬之。江南眾郡縣以次定。

明魯王以海轉徙浙、閩海中,號監國,明諸遺臣猶密與相聞。是年四月,明給事中陳子龍家華亭,陰受魯王官,謀集太湖潰兵舉事。承疇遣章京索布圖往捕,子龍投水死。是月,柘林游擊陳可獲諜者謝堯文,得魯王敕封承疇國公,江寧巡撫土國寶為侯;又得魯王將黃斌卿與承疇、國寶書;鎮守江寧昂邦章京巴山、張大猷以聞。上獎巴山等嚴察亂萌,命與承疇會鞫諜者,別敕慰諭承疇。

粵僧函可者,為故明尚書韓日纘子,日纘於承疇為師生。函可將還里,乞承疇畀以印牌護行出城,守者譏察笥中,得文字觸忌諱。巴山、張大猷以聞,承疇疏引咎,部議當奪職,上命貰之。

承疇聞父喪,請解任守制,上許承疇請急歸,命治喪畢入內院治事。五年四月,還京師。六年,加少傅兼太子太傅,疏請定會推督、撫、提、鎮行保舉連坐法。得旨:「自後用督、撫、提、鎮,內院九卿咸舉所知。得人者賞,誤舉者連坐。」

八年閏二月,命管都察院左都御史。尋甄別諸御史為六等,魏琯等二十二人差用,陳昌言等二人內升,張煊等十一人外轉,王世功等十七人外調,降黜有差。煊疏劾吏部尚書陳名夏,因及承疇嘗與名夏及尚書陳之遴集火神廟,屏左右密議逃叛;承疇又嘗私送其母歸里。疏入,上方狩塞外,巽親王滿達海居守,集諸王大臣會鞫。承疇言:「火神廟集議,即議甄別諸御史定等差,非有他也。」並以送母未請旨引罪。名夏亦列辨,因坐煊誣奏,論死。未幾,上雪煊冤,黜名夏。因諭:「承疇火神廟集議,事雖可疑,難以懸擬;送母歸原籍未奏聞,為親甘罪,情尚可原。留任責後效。」九年五月,承疇聞母喪,命入直如故,私居持服,賜其母祭葬。九月,達賴喇嘛來朝,上將幸代噶,待喇嘛至入覲。承疇及大學士陳之遴疏諫,上為罷行,並遣內大臣索尼傳諭曰:「卿等以賢能贊密勿,有所見聞,當以時入告。朕生長深宮,無自洞悉民隱。凡有所奏,可行即行;縱不可行,朕亦不爾責也。」

十月正月,調內翰林弘文院大學士。明桂王由榔稱號肇慶,頻年轉戰,兵朁地蹙,至是居安隆所,雲南、貴州二省尚為明守。諸將李定國、孫可望等四出侵略,南攻湖南南境諸州縣,東陷桂林,西據成都,兵連不得息。五月,上授承疇太保兼太子太師、內翰林國史院大學士、兵部尚書兼都察院右副都御史,經略湖廣、廣東、廣西、雲南、貴州等處地方,總督軍務兼理糧餉。敕諭撫鎮以下咸聽節制,攻守便宜行事。滿兵當留當撤,即行具奏。命內院以特假便宜條款詳列敕書,宣示中外;並允承疇疏薦,起原任大學士李率泰督兩廣。以江西寇未盡,命承疇兼領,鑄「經略大學士」印授之。臨發,賜蟒朝衣、冠帶、鞾衣蔑、松石嵌撒袋、弓矢、馬五、鞍轡二,諸將李本深等八十七人朝衣、冠帶、撒袋、弓矢、刀馬、鞍轡有差。

承疇至軍,疏言:「湖南駐重兵足備防剿,而各郡窵遠,兵力所不及。郝搖旗、一隻虎等竊伏湖北荊、襄諸郡,倘南窺澧、岳,則我軍腹背受敵。臣與督臣、議臣宜往來長沙四應調度。督臣率提標兵駐荊州,別遣兵增武昌城守,以壯聲援。」又疏言:「桂林雖復,李定國軍距桂林僅二百里,滿洲援剿官兵豈能定留?克復州縣,何以分守?又使孫可望詗我兵出援,潛自靖、沅截粵西險道,則我首尾難顧。置孤軍於徼外,其危易見。臣已分兵馳赴,俾佐戰守,且當親歷衡、永,察機宜以聞。」十二月,上授固山額真陳泰為寧南靖寇大將軍,及固山額真藍拜、濟席哈,擺牙喇纛章京蘇克薩哈等率師鎮湖南;十一年二月,命靖南王耿繼茂率所部自廣州移鎮桂林:皆承疇疏發之也。

是歲孫可望劫桂王,殺大學士吳貞毓等,方內訌。十二年六月,可望遣劉文秀攻常德,分兵使盧明臣、馮雙禮攻武昌、岳州。承疇、陳泰遣蘇克薩哈迎擊,破之。明臣墮水死。文秀、雙禮皆走貴州。陳泰旋卒於軍,以固山額真阿爾津為寧南靖寇大將軍,率固山額真卓羅、祖澤潤等分駐荊州、長沙。十三年,考滿,加太傅,仍兼太子太師。李定國奉明桂王入雲南,湖廣無兵事。阿爾津議以重兵駐辰州,謀自沅、靖入滇、黔,承疇與異議。上召阿爾津還京師,以宗室羅託代。十四年,可望叛其主,舉兵攻雲南,與定國戰而敗;十一月,詣長沙降。時上已允承疇解任還京師養痾,至是命承疇留任,督所部與羅託等規取貴州,並命平西大將軍吳三桂自四川、征南將軍卓布泰自廣西分道入。

十五年正月,復命信郡王多尼為安遠靖寇大將軍,帥師南征,於是承疇與羅託會師常德,道沅州、靖州入貴州境,克鎮遠。卓布泰招南丹、那地、撫寧諸土司,下獨山州,會克貴陽。三桂亦自重慶取遵義進攻開州、桐梓,以其師來會。承疇上疏籌軍食,言:「貴州諸府、州、縣、衛、所僅留空城,即有餘糧,兵過輒罄。惟省倉存米七千餘石、穀四千餘石,足支一月糧。臣所部兵,令分駐鎮遠、偏橋、興隆、清平、平越諸處。降兵暫駐三五日,改屯天柱、會同、黔陽諸縣及湖南沅州。四川兵駐遵義,廣西兵駐獨山,使分地就糧。聞信郡王大兵自六月初發荊州,需糧多且倍蓰。貴州山深地寒,收穫皆在九月。臣方遣吏勸諭軍民須納今歲秋糧之半,並檄下沅州運糧儲鎮遠,又令常德道府具布囊、椶套、木架、繩索,思南、石阡諸府、州、縣、衛、所及諸土司募夫役,具工糈,以赴軍興。」九月,授武英殿大學士。

信郡王多尼師至,駐平越楊老堡,承疇、三桂、卓布泰皆會,議多尼軍出中路,經關嶺鐵索橋至雲南省城,行一千餘里;三桂軍自遵義經七星關,凡一千五百餘里,先中路十日行;卓布泰以南寧方有寇,自貴州、廣西邊境平浪、永順壩、威透山,出安隆所、黃草壩、羅平州,凡一千八百餘里,先四川兵十五日行。既定議,承疇還貴陽,與羅託駐守,遣提督張勇等從多尼軍。明將李定國等拒戰皆敗,明桂王奔永昌。十六年正月乙未,三路師會,克雲南省城,明桂王奔緬甸。承疇如雲南,疏言:「雲南險遠,請如元、明故事,以王公坐鎮。」上以命三桂。

三月,承疇至雲南,疏言:「信郡王令貝子尚善及三桂等追剿至永昌、騰越。明將賀九義、李成爵、李如碧、廖魚、鄒自貴、馬得鳴輩收集潰兵,分遁元江、順寧、雲龍、瀾滄、麗江,處處窺伺。民間遭兵火,重以饑饉,近永昌諸處被禍更烈,周數百里杳無人煙,省城米價石至十三兩有奇。諸軍就糧宜良、富民、羅次、姚安、賓川、臨安、新興、澂江、陸涼諸處。上明察萬里,自有宸斷,俾邊臣得以遵奉。」疏入,上命戶部發帑三十萬,以十五萬賑兩省貧民,十五萬命承疇收貯,備軍餉不給。

八月,承疇疏言:「兵部密咨令速攻緬甸。臣受任經略,目擊民生彫敝,及土司降卒尚懷觀望,以為須先安內,乃可剿外。李定國等竄伏孟艮諸處,山川險阻,兼瘴毒為害,必待霜降始消,明年二月青草將生,瘴即復起,其間可以用師不過四月,慮未能窮追。定國等覬自景東、元江復入廣西,要結諸土司,私授劄印,歃血為盟。若聞我師西進,必且避實就虛,合力內犯。我軍相隔已遠,不能回顧;省城留兵,亦未遑堵御:致定國等縱逸,所關非細。臣審度時勢,權其輕重,謂今歲秋冬宜暫停進兵,俾雲南迤西殘黎,稍藉秋收以延餘喘;明年盡力春耕,漸圖生聚。我軍亦得養銳蓄威,居中制外,俾定國等不能窺動靜以潛逃,諸土司不能伺間隙以思逞。絕殘兵之勾結,斷降卒之反側,則饑飽勞逸皆在於我。定國等潛藏邊界,無居無食,瘴癘相侵,內變易生,機有可俟。是時芻糧輳備,苗、蠻輯服,調發將卒,次第齊集,然後進兵,庶為一勞永逸、安內剿外長計。」疏下議政王、貝勒、大臣會議,如所請暫停進兵。

十月,以目疾乞解任,命回京調理。明年,三桂進兵攻緬甸,獲明桂王以歸。語見三桂傳。聖祖即位,承疇乞致仕,予三等阿達哈哈番世職。康熙四年二月,卒,謚文襄。子士欽,順治十二年進士,官至太常寺少卿。

夏成德,廣寧人。既,以松山降,隸正白旗漢軍。順治初,授三等昂邦章京。其弟景海,授一等甲喇章京。出為山東沂水總兵,嘗疏請收沂州明大學士張四知等財產,又越職乞頒方印,皆不得請。旋以縱所部越境暴掠,與青州道韓方昭互揭,還京師,卒。乾隆初,定封三等子。

孟喬芳,字心亭,直隸永平人。父國用,明寧夏總兵官。喬芳仕明為副將,坐事罷,家居。

天聰四年,太宗克永平,喬芳及知縣張養初、家居兵備道白養粹、罷職副將楊文魁、游擊楊聲遠等十五人出降,命以養粹為巡撫,養初為知府,喬芳、文魁仍為副將,率降兵從諸貝勒城守。上移軍向山海關,諸貝勒率喬芳、文魁、聲遠入謁行營,上召三人者酌以金卮,且諭之曰:「朕不似爾明主,凡我臣僚,皆令侍坐,吐衷曲,同飲食也。」喬芳使詗陽和,而明總兵祖大壽亦使詣喬芳詗我師,喬芳縛以獻。五月,明兵取灤州,貝勒阿敏棄永平出塞。瀕發,屠城民,諸降官養粹、養初等死者十一人,喬芳、文魁、聲遠及郎中陳此心得免。喬芳從師還,隸烏真超哈為牛錄額真。五年七月,置六部,以喬芳為刑部漢承政,授世職二等參將。

崇德三年,更定官制,改左參政。四年,烏真超哈析置八旗四固山,以喬芳兼領正紅、鑲紅兩旗梅勒額真。七年,從伐明,克塔山城。烏真超哈八旗復析置八固山,改鑲紅旗梅勒額真,遂為漢軍鑲紅旗人。八年,或訴貝勒羅洛渾家人奪金,喬芳置不問,坐瞻徇,降世職三等甲喇章京。旋以從克前屯衛、中後所二城,加半個前程。

順治元年,入關,改左侍郎。從諸軍西討。二年四月,以兵部右侍郎兼右副都御史,總督陜西三邊。時張獻忠尚據四川,關中群盜並起,叛將賀珍躪漢中、興安諸府。是年冬,武大定叛固原,徒黨甚眾。初,上命內大臣何洛會帥師鎮西安,至是就拜定西大將軍,遣固山額真巴顏、李國翰將禁旅濟師。三年,復敕靖遠大將軍肅親王豪格帥師督諸將自漢中、興安入四川取獻忠,喬芳於其間亦分遣所部四出捕治。初上官,長安民胡守龍者挾左道惑民,妄改元清光,將為亂,喬芳遣副將陳德捕誅守龍,散其脅從。是年春,賀珍與其徒孫守法、胡向化等以七萬人攻西安。何洛會主城守,喬芳遣德軍西門,副將任珍軍北門,往來沖突,會李國翰師至,賀珍敗走。三年十月,肅親王豪格師既入川,喬芳亦遣總兵官範蘇等攻獻忠部眾,為伏柷溪第溝子,戰白水、青川,屢破之;復以反間殺其渠況益勤等,遂收龍安。

四年五月,喬芳帥師出駐固原,討大定之黨,分遣諸將任珍擊斬白天爵等;劉芳名攻寧夏,俘王元、焦浴;陳德攻鎮原,降姬蛟、王總管:於是固原西北悉定。復遣珍、德及副將馬寧、王平徇興安,討賀珍之黨,戰蕎麥山,再戰板橋,斬胡向宸;困椒溝,破藥箭寨,斬孫守法;破漫營山寨,擒米國軫、李世英:於是興安定。是年秋,馬德亂寧夏,復遣馬寧會寧夏兵共討之。戰亂麻川,逐至河兒坪,斬德。又遣張勇、劉友元攻鐵角城,復戰安家川,擒賀宏器;攻李明義寨,擒明義:於是環慶亦定。乃益遣陳德、王平等招降青嘴寨渠折自明,三十六寨渠王希榮,轆轤寨渠高一祥,擊斬天峰寨渠張貴人,於是關中群盜垂盡。五年四月,流賊一朵雲、馬上飛等攻西鄉,喬芳遣任珍等討之,斬所署監軍許不惑,凡千餘級,生致其渠。

河西回米喇印、丁國棟挾明延長王識駉為亂,既陷甘、涼,渡河東,殘岷、蘭、洮、河諸州,薄鞏昌。喬芳帥師出駐秦州,遣趙光瑞、馬寧等赴援,城兵出,夾擊,斬百餘級。寧等復戰廣武坡,逐北七十餘里,斬三千餘級,鞏昌圍解。喇印、國棟之黨數百人,分擾臨洮、岷州內官營。喬芳部勒諸將,令張勇、陳萬略向臨洮,馬寧、劉友元取內官營,趙光瑞、佟透徇岷、洮、河三州。勇等敗賊馬韓山,斬級七百,進復臨洮。光瑞等敗賊梅嶺,得其渠丁光射,斬級三千。岷、洮、河三州皆下。寧等直擊內官營,斬級八百。喇印、國棟之徒退據蘭州。閏四月,喬芳與侍郎額塞率師自鞏昌薄蘭州。勇敗賊馬家坪,獲識駉,與寧、光瑞會師蘭州城下,攻拔之。別遣光瑞克舊洮州,其渠丁嘉升走死,師渡河。七月,定涼州。八月,攻甘州,喬芳遣張勇夜攻城,而與昂邦章京傅喀禪及寧、光瑞等為繼。喇印等食盡,皆出降。

六年,徵諸道兵下四川。喇印降後授副將,在蘭州軍中,覘鎮兵憚遠征,因惎中軍參將蔣國泰,戕甘肅巡撫張文衡等,據甘州以叛。國棟亦攻陷涼、肅二州。喬芳帥師自蘭州渡河而西,與傅喀禪等會師合圍,攻不下,深溝堅壘以困之。喇印等食盡夜遁,喬芳遣兵追及之水泉,擊殺喇印。國棟復與纏頭回土倫泰等據肅州,號倫泰王,而國棟自署總兵官,城守,出掠武威、張掖、酒泉。會平陽盜渠虞允、韓昭宣等應大同叛將姜瓖為亂,以三十萬人陷蒲州,上命喬芳與額塞還軍御之。喬芳留勇、寧等圍肅州,率師遂東。八月,師自潼關濟,督協領根特、副將趙光瑞等克蒲州,斬級七千。進次寧晉,瓖將白璋等六千人往攻榮河,光瑞等擊破之,斬級二千有奇。璋北走,師從之,迫河,賊多入水死,遂擊斬璋。餘賊入孫吉鎮,殲焉。復進向猗氏,行十餘里,瓖所置監軍道衛登方以數千人依山拒我師,其將張萬全又以四千人助戰。光瑞等擊斬萬全,乃還攻,獲登方,斬其將王國賢等三十餘人、級三千有奇。又令章京杜敏等攻解州,破其渠邊王張五、黨自成等。榮河、猗氏、解州皆下。杜敏等殲餘寇。根特等又破所置都督郭中傑於侯馬驛。九月,光瑞等進克運城,斬允、昭宣。瓖之徒內犯者皆盡誅。十一月,勇、寧克肅州,誅國棟、倫泰及其黨黑承印等,斬五千餘級。河西平。

七年三月,論功,加兵部尚書,進世職一等阿達哈哈番。十二月,喬芳遣任珍擊斬興安寇何可亮。是年,遣趙光瑞等討北山寇劉宏才,戰保安,擒其軍師苗惠民;戰合水,擒斬宏才。八年,遣游擊陳明順等擊敗雒南寇何柴山,游擊仰九明詗紫陽山寇孫守全;復令光瑞等會興安鎮兵擊斬守全及其徒翹興寧、趙定國、謝天奇等,犁其寨。

喬芳督陜西十年,破滅群盜,降其脅從,前後十七萬六千有奇。獎拔諸將,不限資格,如張勇、馬寧、趙光瑞、陳德、狄應魁、劉友元輩,皆自偏裨至專閫。諸寇既殄,疏言:「陜西寇劇,多荒田,請蠲其賦。分兵徠民,行屯田法。」乃遣諸將白士麟等分屯延慶、平固諸地,歲得粟四萬二千石有奇,以佐軍糈。恩詔累進三等阿思哈尼哈番,加太子太保。

十年二月,命兼督四川兵馬錢糧,疏言:「陜西七鎮及督撫各標為兵九萬八千有奇,合滿洲四旗及平西王吳三桂、固山額真李國翰兩軍,歲餉三百六十萬而弱,而陜西賦入一百八十六萬,不足者殆半,後將難繼。甘肅處邊遠興安界,三省兵當循舊額。延綏、寧夏、固原、臨鞏四鎮鎮留三千人,慶陽協五百人,餘五千五百人可省也。漢羌既駐三桂、國翰兩軍,宜裁總兵官。興鎮置副,留千人,陽平關、黑水峪、漢陰縣各五百人,餘二千五百人可省也。提督駐省會,留二千人,餘二千人亦可省也。各道標兵悉令屯田,延鎮、定邊、神木三道無屯田,止用守兵,計所省又二千餘人。都省兵一萬二千人,省餉歲三十一萬。今四川未定,當令右路總兵官馬寧率精兵三千駐保寧,以步兵五千分駐保寧迤北廣元、昭化間,以屯田為持久。三桂駐漢中,相為犄角,規取四川。」既,復疏言:「師進取四川,宜隨在留兵駐防,以樹幹城,謀生聚。師行,當人給馬三、伴丁一,攜甲仗,以利攸往。」上褒其謀當。

十月,西寧回謀為亂,遣狄應魁捕治,得其渠祁敖、牙固子等以歸。喬芳屢乞退,至是以疾告,加少保,召還京師。十二月,命未至而喬芳卒,謚忠毅。太宗拔用諸降將,從入關,出領方面,喬芳績最顯,張存仁亞焉。聖祖嘗誡漢軍諸官吏,因曰:「祖宗定鼎初,委任漢軍諸官吏,與滿洲一體。其間頗有宣猷效力如喬芳、存仁輩,朝廷亦得其用」云。

張文衡,遼東開平衛人。明諸生。天聰八年閏八月,太宗自將伐明,入宣府。文衡自大同詣軍前求見,言在明為代王參謀。明諸臣方尚貪酷,虐民罔上,必有聖主應天而興,故徒步上謁。旋疏言:「大同城小而堅,師攻當先關而後城,攻關宜穴地。宣府城大破碎,宜決洋河灌之。」九年正月,復疏策進取,言:「明文武將吏皆以賂得,無謀無勇;又以貪故,餉減器窳,兵不用命。所以能拒我者,不過畏殺、畏掠、畏父母妻子離散,乃倚火器以死御我。今宣、大新被兵,山、陜、川、湖陷於流賊。賊半天下,兵亦半天下。惟東南無事,又困於新餉。上不及此時進兵,明不恆弱,我不恆強,節短勢險,人有鼎立之志。豈非自失其機,反貽異日憂乎?原上毋負天生上之心也。」疏入,上曰:「待朕思之。」二月,遣貝勒多爾袞帥師收察哈爾。文衡又言:「宜率蒙古入偏岢,略太原,假中國物力以富蒙古;且張軍威,並可近招流賊,並力並進。」上授文衡秘書院副理事官,賜田宅、銀幣,以大臣雅希禪女妻焉。隸鑲黃旗漢軍。

順治元年,出為山東青州知府。初上官,總兵官柯永盛以戍青州之兵徇高密,而侍郎王鰲永以招撫至,主餉。趙應元者,從李自成為旗鼓,覸青州兵寡,陽就鰲永降,請置孥於城。既入,遂執殺鰲永。文衡見應元,為好語,具疏請留鎮。應元喜,攫庫金,群酗。會梅勒額真和託、李率泰率禁旅略登、萊,道青州,營城西北。文衡惎應元出謁和託等,慰勞遣還,密令兵從入,夜起戮應元及其黨數十人。青州平。

二年,移淮安府知府。豫親王多鐸下揚州,道淮安。文衡請禁將吏毋擾市,糗糧芻秣應期立辦。三遷,巡撫甘肅。五年二月上官,逾月而遘米喇印之亂。變未作,喇印詭言要文衡造其家集議。文衡行未至,賊環射殺之。總兵劉良臣,副將毛鑌、潘雲騰,游擊黃得成、金印,都司王之俊,守備胡大年、李廷試、李承澤、陳九功皆死。參將翟大有與戰,沒於陣。賊挾西寧道副使林維造至北關,手益殺之。越日,陷涼州,戕西寧道參議張鵬翼。賊四出侵掠,破鞏昌,戕臨鞏兵備道李絮飛;破岷州,戕知州杜懋哲、王札;破蘭州,戕同知趙沖學,知州趙翀,訓導白旗、國學錦;破臨洮,戕同知徐養奇;破渭源,戕知縣李淐;戰通渭,圍子山,知縣周盛時被創死。事平,皆贈恤如例。

張存仁,遼陽人。明寧遠副將,與總兵祖大壽同守大凌河。天聰五年,太宗自將攻大凌河,從大壽出降,仍授副將。六年正月,存仁與副將張洪謨、參將高光輝、游擊方獻可合疏請乘時進取,參將姜新別疏請令副將祖可法、劉天祿先取松、杏二城,則錦州自下。七年五月,新復請進兵,洪謨等及新皆大凌河降將也。

崇德元年五月,始設都察院,班六部上。以存仁為承政,並授世職一等梅勒章京。越數日,存仁上言:「臣自歸國,默察諸臣賢否,政事得失,但不敢出位妄論列。今上創立此官,而以命臣。臣而正直,後之人正直必有過於臣者;臣而邪佞,後之人邪佞亦必有甚於臣者。所慮臣本心而行事,人不敢彈劾而臣彈劾之,人不敢更張而臣更張之,舉國必共攻臣,使臣上無以報主恩,下無以伸己志,獲戾滋甚。臣雖愚,豈不知隨眾然諾,其事甚易;發奸擿伏,其事甚難。誠見不如此,不足以盡職。敢於受任之始,瀝誠以請:如臣茍且塞責,畏首畏尾,請以負君之罪殺臣;如臣假公行私,瞻顧情面,請以欺君之罪殺臣;如臣貪財受賄,私家利己,請以貪婪之罪殺臣。茍臣無此三罪,而奸邪誣陷,亦原上申乾斷,以儆讒嫉。」上曰:「此或知有其人而為是言。朕素不聽讒,惟親見者始信之。且朕志定於上,而諸臣蒙澤於下,縱有奸邪,孰能售其術哉?」越數日,以阿什達爾漢為都察院滿承政,尼堪為蒙古承政,並增置祖可法為漢承政。上御清寧宮,阿什達爾漢等前奏事,上因諭曰:「朕有過,親王以下壞法亂紀,民左道惑眾,皆當不時以聞。若舉細而遺大,非忠直也。」可法對曰:「臣等惟上是懼,他復何忌?有聞必以奏。」存仁曰:「可法言非是。臣誠忠直為國,上前且犯顏直諫,況他人乎?」上曰:「然。人果正直,天地鬼神不能搖動,人主焉得而奪之?」是歲,都察院劾刑部承政郎球貪汙,論罪;劾工部奪民居授降人,復別造宅償民,勞民非制。上以諸臣多未更事,事事加罪,反令惶惑,但誡毋更違令。

三年正月,可法、存仁疏言:「禮部行考試,令奴僕不得與。上前歲試士,奴僕有中式者,別以人畀其主。今忽更此制,臣等竊謂奴僕宜令與試,但限以十人為額。茍十人皆才,何惜以十人易之?」上曰:「昔取遼東,良民多為奴僕。朕令諸王下至民家,皆察而出之,復為良民。又許應試,少通文藝,拔為儒生。今滿洲家奴僕,非先時濫占者比。或有一二諸生,非攻城破敵血戰而得,即以戰死被賚。昨歲克皮島,滿洲官兵爭效命,漢官兵坐視不救。此行所得之人,茍無故奪之,彼死戰之勞,捐軀之義,何忍棄也?若別以人相易,易者無罪,強令為奴,獨非人乎?爾等但愛漢人,不知惜滿洲有功將士及見易而為奴者也。」可法、存仁引罪謝。既,復論戶部承政韓大勛盜帑,大勛坐奪職。四月,疏請敕戶部立四柱年冊,再疏請誅大勛,又劾吏部、刑部復用贓吏違旨壞法,皆與可法合疏上,上皆嘉納之。七月,更定官制,可法、存仁皆改都察院右參政。漢軍旗制定,隸鑲藍旗。

大壽既降,復入錦州為明守,攻數年不克。五年正月,存仁疏請屯兵廣寧,扼寧遠、錦州門戶。四月,又疏言:「臣睹今日情勢,錦州所必爭。但略地得利易,圍城見功難。原上振軍心,與之堅持。截彼詗察,禁我逃亡。遠不過一歲,近不過一月,當有機可乘。兵法全城為上,蓋貴得人得地,不貴得空城也。我師壓境,彼必棄錦州,保寧遠;再急,彼必棄寧遠,保山海關。大壽跋扈畏罪,豈肯輕去其窟?事緩則計持久,事急則慮身家。大壽背恩失信,人皆以為無顏再降。臣深知其心無定,惟便是圖,急則悉置不顧。況彼所恃者蒙古耳,今蒙古多慕化而來,彼必疑而防之。防之嚴則思離,離則思變。伏原以屯耕為本務,率精銳薄城,顯檄蒙古,縱俘宣諭,未有不相率出降者。此攻心之策,得人得地之術也。」十二月,復言:「兵事有時、有形、有勢,三者變化無定,而用之在人。松山、杏山、塔山三城,乃錦州之羽翼,寧遠之咽喉。塔山城倚西山之麓,自其巔發砲俯擊,城易破也。既得此城,羽翼折,咽喉塞矣。兵法困堅城者,必留其隙。錦州雖不甚堅,當留山海關以為之隙。錦州遼兵少,西兵多,一人負箭入,群驚而思遁。能善用巧,山海關可下。」疏末並言烏真超哈每遇番上,輒令奴僕代,上為申禁。

六年,師屢破明兵松山、杏山間,存仁復疏請相機度勢,以時進兵。七年,既克錦州,存仁請招吳三桂降。上頒禦劄撫諭,並命存仁遺以私書,略言:「明運將終,重臣大帥就俘歸命。將軍祖氏甥,雖欲逃罪,無以自明。大廈將傾,一木不能支。縱茍延歲月,智竭力窮,終蹈舅氏故轍。何若未困先降,勛名俱重?」六月,烏真超哈四旗始分置八固山,授存仁鑲藍旗梅勒額真。八年,從鄭親王濟爾哈朗取前屯衛、中後所,加半個前程。

順治元年,從入關,與固山額真葉臣率師徇山西,下府六、州二十四、縣一百三十一,遂克太原。又從豫親王多鐸略河南,下江南,督所部以砲戰,屢有克捷。二年六月,從貝勒博洛定浙江,以存仁領浙江總督。兵後民流亡,存仁集士紳使撫諭,民復其所。七月,疏言:「近有薙發之令,民或假此號召為逆。若反形既著,重勞大兵,莫若速遣提學,開科取士,下令免積逋,減額賦,使讀書者希仕進,力田者逭追呼,則莫肯相從為逆矣。」得旨,謂「誠安民急務也」,令新定諸行省皆準恩詔施行。

十一月,授兵部右侍郎,兼都察院右副都御史,總督浙江、福建。時明魯王以海保紹興,號「監國」,其將方國安鎮嚴州。故明福王由崧倚大學士馬士英,用以亡國,士英走依國安。是歲九月,國安自富陽渡錢塘江窺杭州,存仁遣副將張傑、王定國率師御之,斬四千餘級。國安退保富陽。又令定國出屯餘杭,遇國安兵,與戰,自關頭至小嶺,逐北二十里,斬國安子士衍。十月,士英復以兵至,去杭州十里為壘五。平南大將軍貝勒勒克德渾帥師赴之,未至,士英引去,存仁與總兵官田雄追擊之,斬五百餘級。十一月,士英、國安復以兵至,存仁與梅勒額真季什哈及雄等帥師擊之,敵溺江死者無算。十二月,士英、國安屯赭山,掠硃橋、範村諸處。存仁與梅勒額真硃瑪喇及雄、傑等分兵與戰,國安所將水師數萬人殲焉,餘眾俘馘殆盡。三年二月,有姚志卓者,為亂於昌化,與國安相應。存仁遣傑等擊走志卓,復昌化。五月,敘功,進三等昂邦章京。六月,遣副將張國勛等破敵太湖,獲士英等,戮之。十一月,存仁請設水師五千,備錢塘江御海寇。四年五月,遣副將滿進忠等收福州鎮東衛,破海寇周鶴芝;遣副將李繡援浦城,逐鶴芝黨岑本高。十二月,遣副將馬成龍等破敵處州,克景寧、雲和、龍泉三縣。五年正月,明宜春王議衍率眾自江西入福建,保汀州山寨,總兵官於永紱擊破之。二月,分兵克連城、順昌、將樂三縣,獲明侍郎趙士冕、總兵黃鍾靈等。存仁自至浙江,屢以疾乞休,至是始得請,受代以去。

六年八月,起授兵部尚書,兼右副都御史,總督直隸、山東、河南三行省,巡撫保定諸府,提督紫金諸關,兼領海防。盜發榆園,為大名諸縣害。存仁聞歸德侯方域才,貽書咨治盜策,方域具以對。存仁用其計,盜悉平。七年,上令疆吏考校諸守令,以文藝最高下。存仁出按諸府縣,廉能吏有一二語通曉,即注上考;非然者,文雖工亦乙之。監司請其故,存仁曰:「我武臣也,上命我校文,我第考實,文有偽,實難欺也。況諸守令多從龍之士,未嘗教之,遽以文藝校短長,不寒廉能吏心乎?」屢遇恩詔,進一等精奇尼哈番兼拖沙喇哈番。九年,卒,贈太子太保,謚忠勤,祀直隸、山東、河南、浙江、福建五行省名宦。乾隆初,定封三等子。

存仁弟子瑞午,康熙間為福建邵武府知府。耿精忠叛,徇諸郡邑,瑞午不為下,死之。子颺、瑛、珍、珖、玳、瑜,子婦王、李皆從死。事定,贈瑞午太僕寺卿。存仁孫璲,康熙間以佐領從軍,鄭成功將劉國軒攻海澄,戰死,贈拖沙喇哈番。

論曰:國初諸大政,皆定自太祖、太宗朝。世謂承疇實成之,誣矣。承疇再出經略,江南、湖廣以逮滇、黔,皆所勘定;桂王既入緬甸,不欲窮追,以是罷兵柄。孟喬芳撫綏隴右,在當日疆臣中樹績最烈。張存仁通達公方,洞達政本。二人皆明將。明世武臣,未有改文秩任節鉞者,而二人建樹顧如此。資格固不足以限人歟,抑所遭之時異也?


\end{pinyinscope}