\article{列傳一百二十一}

\begin{pinyinscope}
馬全牛天畀阿爾素納張大經曹順敦住烏爾納

科瑪佛倫泰達蘭泰薩爾吉岱常祿保瑪爾占庫勒德穆哈納

國興巴西薩扎拉豐阿觀音保李全王玉廷珠魯訥

許世亨子文謨尚維升張朝龍李化龍邢敦行

臺斐英阿阿滿泰花連布明安圖

馬全,字具堂,山西陽曲人,初名瑔。乾隆十七年一甲三名武進士。自二等侍衛出為福建撫標右營游擊,與同官爭言,奪職。更名,寄籍大興。二十五年,會試再中式,上御紫光閣校閱,見全識之,問曰:「爾馬瑔耶?」全叩頭謝罪,遂成一甲一名武進士,授頭等侍衛。二十七年,扈上南巡,命署江西南昌鎮總兵,賜孔雀翎。疏陳校閱各營操練,赴禁山隘口巡查,防奸民闌入。上褒其奮勉,授江蘇蘇松鎮總兵。擢江南提督。請改歸原籍。調甘肅提督,陛見,賜黑狐褂。

三十八年,命從征金川,為領隊大臣。將軍溫福駐軍木果木,全偕都統海蘭察分攻昔嶺,奪碉二,賊大至,鏖戰冰雪中一畫夜,卒敗賊。會日暮撤兵,賊後尾追,為伏擊敗之。搜山麓逸賊,建柵數十為聲援。木果木大營潰,全殿後,戰竟夜,死之,事聞,上曰:「提督馬全乃國家出力有用之人,今力戰死事,實堪軫惜!」謚壯節,予騎都尉兼雲騎尉世職。同時死事諸將有戰績者,牛天畀、阿爾素納、張大經。

天畀,山西太谷人。以武進士授藍翎侍衛,累遷四川川北鎮總兵。征金川,天畀率兵赴木坪,佐提督董天弼進剿。師自達木巴宗分三道趨資哩,天畀偕侍衛阿爾素納擊賊於瑪爾瓦爾濟山巔,戰三畫夜,克卡十,與大軍會,賜孔雀翎。師圍資哩,天畀攻南山,參贊五岱攻北山,未下。上以阿喀木雅地當孔道,得此可破資哩,手敕諭諸將。天畀偕侍衛烏什哈達將四百人覓路,伏箐中,誘阿喀木雅守賊出寨,擊之,賊敗匿。天畀列兵山麓截賊援,賊四百餘突出寨,援賊二百自得爾蘇山至,天畀擊之,斬五十餘級。參贊大臣阿桂代五岱攻北山,賊不支,天畀自南山夾擊,遂克資哩,阿喀木雅、得爾蘇賊皆潰。天畀捕治餘賊,巖洞箐林,搜戮殆盡,自得爾蘇山巔下至河岸訖北山麓,皆屬我師。攻喇卜楚克山巔,賊守甚密。副都統富勒渾出山後,奪卡四;天畀自前登,奪卡一。賊自林中出,天畀督兵冒槍石,縱火焚賊卡;又偕章京德保等進攻布朗郭宗,取德木達碉寨三、石卡七,與大軍會,遂克之。進取底木達,俘澤旺。三十八年,師攻功噶爾拉,天畀與副都統烏什哈達、總兵張大經冒雪陟山前二峰,奪其碉,賊自山後至,擊之走。定邊將軍溫福疏陳天畀戰功,請署貴州提督。木果木大營潰,天畀力戰死之,謚毅節,予騎都尉兼雲騎尉世職。子敬一,自陳文生不習弓馬,賜舉人。

阿爾素納,祿葉勒氏,吉林滿洲鑲黃旗人。乾隆時,以前鋒隨征西域、緬甸,累遷二等侍衛,賜號額騰伊巴圖魯。金川叛,從征,攻巴朗拉,與侍衛額森特先登;攻資哩、阿喀木雅、美美卡、兜烏諸地,均有功,擢一等侍衛,加副都統銜,授領隊大臣。隨大軍移營木果木,屢克碉卡,授鑲白旗蒙古副都統。大營陷,率滿洲兵退,行至大壩溝,遇賊,力戰死,贈都統銜,予騎都尉兼雲騎尉世職。

大經,山西鳳臺人。乾隆時,由武進士歷官陜西興漢鎮總兵。三十六年,率西寧、陜西兵各千人從征金川。師圍資哩,大經出中路,進攻兜烏。大經以兵千駐阿喀木雅,旋移駐木闌壩鄂克什舊寨,從攻明郭宗,克之。復從攻底木達,俘澤旺。三十八年,溫福進駐木果木,大經將五百人分駐簇拉角克。上以其地在功噶爾拉袨口之北,形勢險要,諭增兵協防。四月,偕烏什哈達等攻達扎克角山,擊敗伏箐賊;沿山下攻得斯東寨,賊棄寨遁。木果木大營潰,參贊大臣海蘭察檄大經撤兵出,遇賊於乾海子,路險不能騎,徒步力戰,死,予騎都尉世職。

諸將死事皆祀昭忠祠,全、天畀、阿爾素納並圖形紫光閣:全列前五十功臣,天畀、阿爾素納皆列後五十功臣。

曹順,四川閬中人。入伍。從將軍溫福征金川。師攻固卜濟山梁,賊為柵阻木闌壩路,匿柵內發槍石,其渠啟柵門出,順斬之;奪門入,焚柵,殲柵內賊,賜孔雀翎。從攻明郭宗,自木雅山至木爾古魯山麓,奪賊寨下,進克嘉巴,賜號扎親巴圖魯。順與頭等侍衛烏什哈達督兵至功噶爾拉,攻昔嶺;又與司轡托爾托保率瓦寺鄂克什土兵先逼卡,殺賊數十,賜緞二匹。攻昔嶺第五碉,與副都統巴朗、普爾普等分兵攀登,溝內伏賊起,迎擊,斬其渠,順面中石傷。先後敘功,遷湖南衡州協副將。阿桂策督諸軍攻宜喜,先攻木思工噶克及得式梯,綴賊使不相應,令書麟等攻袨口碉卡,賊赴援,順攻峰右碉,克之。師自康薩爾進據袨口山峰,賊悉力拒,退復進者七,順與侍衛穆哈納等迎擊,群賊悉殪,遂克擦庸碉寨。師分道斷賊後路,順督土兵縱火,與參贊大臣豐升額為犄角,並進,賊不能支,穴寨後竄,順奮擊,迫賊墜箐死,取石碉十二,遂克遜克爾宗,擢甘肅肅州鎮總兵。四十年閏十月,攻西里山麓黃草坪,順跨木柵指麾,賊於暗中發槍,被創,沒於陣。金川平,與福建建寧鎮總兵敦住、陜西延綏鎮總兵烏爾納並祀昭忠祠,圖形紫光閣,同列前五十功臣。

敦住,瓜爾佳氏,滿洲正黃旗人,昭勛公圖賴四世孫。圖賴曾孫馬爾薩事聖祖,自佐領擢至本旗都統。雍正初,授內大臣,佐靖邊大將軍傅爾丹駐和通呼爾。哈諾爾賊來犯,馬爾薩力戰,殺千餘人,大風雨,渡哈爾噶河,戰沒,予騎都尉兼雲騎尉世職。敦住,其從子也。乾隆初襲職,累遷頭等侍衛。從征金川,三十九年,令署總兵。攻宜喜,冒雨克達爾圖、俄坡諸碉。十一月,攻日旁,自木克什進,短兵搏戰,沒於陣。

烏爾納,納喇氏,滿洲鑲藍旗人。自護軍累遷至甘肅蘭州城守營參將。從征金川,克沙壩山,賜孔雀翎。攻遜克爾宗,攻甲爾納,皆力戰,中槍;攻榮噶爾博,敗援賊:再遷總兵。復克邁過爾,進屯凱立葉。從攻木思工噶克、勒吉爾博、得式梯諸地,累有功。師攻勒烏圍,烏爾納從攻轉經樓,盡下諸城寨。師征大金川,攻西里,烏爾納督兵造甲爾日磉浮橋,賊至,擊敗之;力戰至科布曲,率前隊渡河,克其第四碉。四十一年,從攻噶喇依。二月,噶喇依既克,喇嘛寺火起,延及火藥房。烏爾納往救,藥轟石躍,中傷死。上以烏爾納轉戰甚力,功成身殞,深嗟惜焉。議恤,順予世職騎都尉兼雲騎尉,敦住進世職三等輕車都尉,烏爾納官其子都司。

科瑪,敖拉氏,滿洲正黃旗人。以三等侍衛從征金川。師攻克邦甲山梁,科瑪自翁克爾壟力戰至美諾,奪碉寨,賜號納親巴圖魯。攻當噶爾拉山梁,科瑪督兵斧斫柵,逼碉,毀其垣以入,殺賊。從克美諾、拉約,將六百人取卡卡角,繞出山後仰攻,殲守賊。副將軍明亮攻斯第,科瑪將三百人陟西岡;又克達爾圖第六碉。累擢頭等侍衛,授領隊大臣。將六百人攻穀爾提,獲頭人索爾甲、木達爾甲等。督兵攻沙壩,擲火彈爇賊寨二百餘,加副都統銜。乾隆四十年四月,自得楞力戰至基木斯丹當噶,深入賊陣,中槍死。

佛倫泰,庫雅拉氏,滿洲正白旗人。亦以三等侍衛從師克巴朗拉,賜號扎勒丹巴圖魯。攻資哩,沖入石卡,殺賊四十餘,俘十二,遂克之,將五百人取咱贊及溝東諸寨。攻美美卡,佛倫泰自西山下,多斬獲。從攻路頂宗、底木達、達爾圖、日旁、凱立葉,皆有功。攻遜克爾宗,兩目受石傷。攻康薩爾,克其碉,加副都統銜,授領隊大臣。四十年四月,師攻基木斯丹當噶,科瑪戰死,佛倫泰自薩克薩穀進至榮噶爾博,力戰,亦沒於陣。

達蘭泰,薩克達氏,滿洲鑲藍旗人。以護軍從征緬甸,戰新街、老官屯,有勞。征金川,命選年壯得力將士,達蘭泰與焉。攻明郭宗、昔嶺奪據達扎克角泉水。師攻羅博瓦山,賊來援,達蘭泰迎擊,賊潰;督兵殺賊,上駐軍山峰,賜號額依巴爾巴圖魯,累擢二等侍衛。攻甲爾納來珠寨,賊出我軍後,自山梁下;達蘭泰設伏射賊,賊負創遁。四十年五月,擊賊達撒谷,被數創,卒。

薩爾吉岱,博和爾氏,齊齊哈爾鑲紅旗人。以藍翎侍衛從克馬奈、日旁;再進,攻該布達什諾、色淜普,薩爾吉岱沖入賊陣,力戰,盡克其碉卡,賜號善巴巴圖魯。從克默格爾、凱立葉,授三等侍衛。攻格魯克古丫口,賊負險據寨,槍石並發;薩爾吉岱奮登丫口,射賊殪,賊引退,我師從之,越山溝五,奪碉五十、寨卡三百餘。攻達瑪噶朗,陟山梁,克其碉。師臨勒烏圍,分道攻轉經樓,賊來援,薩爾吉岱伏兵橫擊,賊潰。師自達烏達圍向當噶克底,薩爾吉岱為前鋒,冒雨拔柵以登,擊守碉賊盡殪。四十年閏十月,擊賊阿穰曲,麾士卒倚柵射賊,中槍死。

金川平,科瑪、佛倫泰、達蘭泰、薩爾吉岱並圖形紫光閣,列前五十功臣。

常祿保,赫舍哩氏,滿洲鑲藍旗人。其先有德祿者,以軍功予騎都尉世職。常祿保襲職,自三等侍衛屢遷四川提標左營游擊。從征金川,擢成都城守營參將。副都統海蘭察等攻得拉密色欽山梁,賊潛伏林內,常祿保往來搜擊,進攻明郭宗,取旁近山梁。師進攻路頂宗所屬喀木色爾寨,常祿保從海蘭察自南山大澗潛越山頂,克之;復進取博爾根山,仰攻,克木城,受石傷。溫福等上其功,賜孔雀翎。又從副都統阿爾素納等分路進攻昔嶺大碉,賊百餘從旁沖出,常祿保督兵橫擊敗之,進駐日壟。旋擢甘肅河州協副將。定西將軍阿桂等攻克羅博瓦,常祿保駐山巔,賊九百餘乘雪夜分兩隊劫營,四面環攻,勢甚迫,常祿保督兵力戰御之,被槍石傷,賊竄入卡內者皆殲焉。副都統烏什哈達等先後赴援,常祿保督兵夾攻,賊敗竄,賜號西爾努恩巴圖魯、白金百。尋擢廣東高廉鎮總兵。分攻菑則大海諸碉,賊掘壕,排松,簽鹿角,備禦甚嚴。常祿保分兵出賊後,合攻各碉卡,同時皆下。又偕總兵官達色合攻雅木賊碉,克之。乾隆三十八年十一月,師攻科布曲山梁,賊死拒,槍石交下,常祿保被創,歿於陣。

事平,錄死事諸將,圖形紫光閣,功稍次者為後五十功臣,常祿保及侍衛瑪爾占、庫勒德、穆哈納,參將國興,佐領巴西薩皆與焉。

瑪爾占,巴爾汗氏,察哈爾正白旗人。自準噶爾來降。以三等侍衛從軍,攻日旁,馬蹶,傷,仍請從軍。擢二等侍衛,命創愈仍從軍。攻凱立葉,力戰,賜戰拉布巴爾巴圖魯,遷頭等侍衛,授領隊大臣。攻克該布達什諾木城及色淜普前碉,先登,又被創,予副都統銜。三十九年,攻康薩爾大碉,戰沒。

庫勒德,沃埒氏,滿洲正藍旗人。以藍翎侍衛從軍,攻昔嶺及達扎克角木柵,累遷二等侍衛。攻克默格爾山梁,賜號朗親巴圖魯。攻遜克爾宗、康薩爾,被創。四十年四月,攻木思工噶克,戰死。

穆哈納,瓜爾佳氏。以護軍校從軍,攻克默格爾山梁及凱立葉碉寨,遷三等侍衛。攻木思工噶克丫口,直前奪其碉,賊潰;攻巴木通,正濃霧,督兵分道擊賊,賊伏深箐中,皆殲焉,盡克其碉卡:賜號巴爾丹巴圖魯。四十年八月,攻勒烏圍,力戰死。

國興,貴州大定人。以千總從貴州威寧鎮總兵王萬邦征金川,攻巴朗拉。溫福疏言貴州綠營將士功多。攻資哩北山,興為前鋒。進攻墨壟溝、甲爾木,再進攻東瑪,我師為木卡,興將三百人為守。賊夜至,興滅火以待;賊逼卡,發槍砲,賊盡殪。又從阿桂攻勒烏圍,賜孔雀翎,號圖多布巴圖魯。累遷朗洞營參將。四十年四月,攻木思工噶克,興持斧斫木城,率眾擁入,克其碉。賊來攻,興督兵射賊,賊散復聚者七,卒不能陷。興負創,越日卒。

巴西薩,布拉穆氏,索倫正紅旗人。以佐領從軍,攻羅博瓦山,山甚峻,巴西薩督兵攀登,射賊殪,遂取山梁,諸碉卡皆下,賜孔雀翎,號塔爾濟巴圖魯。四十年,攻康薩爾,攻碉迫懸崖,賊無路,殊死戰,巴西薩死焉。

扎拉豐阿,赫舍里氏,滿洲正黃旗人,前鋒統領定壽孫。襲二等輕車都尉,授三等侍衛,累遷御前侍衛。從討霍集占,師次陽阿里克,扎拉豐阿將五百人捉生,俘三十餘。師還,賜西朗阿巴圖魯名號,進一等輕車都尉,圖形紫光閣,擢正白旗漢軍副都統。出為烏里雅蘇臺參贊大臣,旋令赴科布多經理屯田。定邊左副將軍成袞札布入覲,令署將軍印。召還京,以正白旗護軍統領從明瑞出師,授領隊大臣。次蠻結,戰破賊,加都統銜。賊圍小猛育,中槍死,謚昭節,進封一等男。子春寧襲爵,官至綏遠城將軍。

觀音保,瓜爾佳氏,滿洲正黃旗人。初授健銳營前鋒藍翎長,再遷前鋒參領。從副將軍兆惠戰濟爾哈朗,從參贊大臣雅爾哈善攻庫車,戰甚力,擢正白旗蒙古副都統予騎都尉世職,圖形紫光閣。出為伊犁領隊大臣。從明瑞攻烏什,負創奮進,克其城,賜卓里克圖巴圖魯名號。遷鑲藍旗護軍統領,署云南楚雄鎮總兵。從明瑞出師,為領隊大臣,戰於蠻結,日昳大霧,賊出林中。扎拉豐阿率眾薄賊壘,觀音保當賊沖,殺賊二百餘,乘霧深入,破木砦。師至小猛育,賊圍急,觀音保發數矢,輒殪賊,箙僅餘一矢,欲復射,驟策馬向草深處,以其鏃射喉死,予二等輕車都尉。

李全,山西陽曲人。自行伍拔山西撫標把總,累遷雲南永昌鎮總兵。從征,戰蠻結,與扎拉豐阿據東山梁,張犄角,破象陣;至天生橋,乘霧破賊壘。至蠻化,賊大至,中槍,數日卒。

王玉廷,甘肅武威人。自行伍累遷雲南臨元鎮總兵。從征,攻老官屯,賊據木城拒守,玉廷親發砲乘霧督攻,中槍傷股,戰益力。賊敗,匿不出;復自力督戰,創發卒,謚勤義。玉廷初從討達瓦齊,援將軍兆惠黑水營之圍;佐雅爾哈善圍庫車;又從兆惠攻喀什噶爾:皆有戰功。至是,與全同予騎都尉又一雲騎尉世職。

珠魯訥,那爾氏,滿洲鑲白旗人。繙譯舉人,授筆帖式,充軍機處章京。再遷戶部顏料庫員外郎。出為荊州副都統,入授禮部侍郎,調工部,兼署兵部。明瑞出師,授參贊大臣,駐雅爾。移軍木邦,土司甕團降,請於清水河招商復業,遣兵監焉。擺夷環歇等五十輩偽降,斬以徇。奏設木邦至阿瓦臺站凡五,分兵防衛,上嘉之。緬甸兵自東、西二山來犯,遣裨將分御。俄,賊焚游擊福珠營,夜圍珠魯訥,珠魯訥具遺奏,遣筆帖式福祿突圍出,遂自戕。上責珠魯訥怯懦,以其情亦可愍,賜祭葬,祀昭忠祠。

許世亨,四川新都人,先世出回部。初為騎兵。從征金川、西藏,並有勞。旋以武舉授把總,累遷守備。復從征金川,從四川總督阿爾泰攻約咱東、西山梁,進攻扎口、阿仰、格藏、達烏諸地,連拔碉寨。復攻甲爾木山梁及岳魯、登達諸地,拔木城、石卡、又克多功山坡及日木城碉寨。進擊古魯碉,賊夜劫營,世亨率兵百餘御戰,至曙,度賊且去,開壁奮呼追擊,殺賊無算,遂克古魯碉寨,賜孔雀翎,加勁勇巴圖魯。尋累擢參將。從參贊大臣、副都統明亮攻當噶爾拉山梁,拔第五碉。又從參贊大臣富德自墨壟溝進兵,克甲爾木、日赤爾丹思、僧格宗諸寨。又從定邊將軍明亮自底旺至馬奈,克拉窠、絨布、根扎葛木、卡卡角、思底、喀咱普諸碉寨。又從明亮自宜喜攻達爾圖山梁,擒頭人丹巴阿太,奪俄坡、木克什、格木勺諸碉卡。又從領隊大臣奎林攻木克什西南山寨。又從副都統三寶攻西郭洛,進駐得爾巴克山梁。又從明亮攻得楞山梁,拔數碉,進擊基木思丹當噶及薩穀諸山梁,毀其碉,俘馘無算。克額爾替第一碉,殺賊四十餘,又克第二碉;又克石真噶、沙爾尼、瑯谷、烏岳、斯當安諸碉寨。凡七戰,皆勝。進攻扎烏古,時賊踞山巔,碉卡連亙。世亨冒石矢率兵直上,拔數碉卡;又克碾占山、阿爾古山及平壩諸寨。又克達撒穀大山梁,毀其碉寨。又克獨古木上、下寨,進踞布吉魯達那兩道山梁。又克甲雜官寨獨松隘口。奪獲大小寨落數十,並獲賊渠雍中旺爾結。遂西至噶拉依,與南路馬爾邦軍會。乾隆四十一年,金川平,擢雲南騰越鎮總兵。

四十九年,甘肅回亂,世亨奉命往安定捕逸回,獲二百餘。事竣,補貴州威寧鎮總兵。

五十二年,臺灣林爽文叛,世亨率黔兵二千餘赴剿,攻克集集堡,俘斬甚眾,獲偽印、器械、旗幟。進攻小半天,賊奔潰,追襲至老衢峙,俘爽文,並頭人何有志。又從參贊成都將軍鄂輝自大武隴進攻南路水底藔,手殺頭人一。時莊大田等敗竄瑯嶠,眾尚數千,世亨率黔兵與諸軍分隊,水陸合攻,擒大田並諸賊目。臺灣平,改賜堅勇巴圖魯名號,圖形紫光閣,列前二十功臣。

五十三年二月,擢浙江提督,未至,調廣西提督。安南有大酋曰阮惠,攻其國都,逐其君黎維祁。兩廣總督孫士毅主用兵,世亨諫不聽。師行,將兩廣綠旗兵八千人,與總兵尚維升、張朝龍等從出關入安南境,至其國都,有大川三:北曰壽昌江,南曰市球江,又南曰富良江。十一月辛未,師渡壽昌江。甲戌,師次市球江。惠兵據南岸山,守甚固。朝龍兵自上游渡,世亨亦力戰,殺賊數千,賜御用玉搬指、大小荷包。越三日丁丑,黎明,師次富良江,南岸即黎城,黎城者安南國都,以王姓名其城也。惠兵盡伐濱江竹木,斂舟泊對岸。循江岸得小舟,載兵百餘,夜分至江心奪惠軍舟,世亨等親率二百餘人先渡,復掠小舟三十餘,更番渡兵,分搗惠軍,惠軍潰,焚其舟十餘,俘其將數十。戊寅旦,師畢濟,黎氏宗族及安南民出迎,世亨從士毅入城安撫。求維祁,承制立為王。捷聞,封一等子,疏辭,弗許。

阮惠有分地曰廣南,去黎城二千餘里。方議進討,請益兵籌餉。上欲罷兵,世亨亦謂士毅曰:「我兵深入重地,惠未戰遽退,事叵測。及時振旅入關,上計也。」士毅不納。五十四年正月戊午朔,士毅召諸將置酒高會。己未,維祁告惠兵至,士毅倉皇奪圍出,渡富良江,浮橋斷,世亨與維升、朝龍率數百人戰橋南,陣沒。士毅初奏言:「惠兵至,臣與世亨督兵決戰;賊眾圍合,臣與世亨不相見,乃奪圍出。」上猶冀世亨全師而還;既聞其戰死,命予恤。副將廣成自軍中還,見上,言:「當惠兵攻黎城,士毅與世亨退據富良江拒惠。士毅欲渡江與惠戰,不利,以身殉。世亨力諫,以大臣系國重輕,不可輕入,令慶成護士毅還師。又命千總薛忠挽士毅馬以退。世亨督諸將渡江陷陣,力戰死。」上愍世亨知大體,進封三等壯烈伯,祀昭忠祠,謚昭毅。福康安師至,惠更名光平,乞降。立祠黎城祀死事諸將,世亨居首列。

子文謨,自武舉襲爵,命在頭等侍衛上行走。期滿,以湖廣參將用,並賜孔雀翎。嘉慶元年,枝江教匪聶人傑為亂,湖北巡撫惠齡令文謨捕治,有勞,賜繼勇巴圖魯名號,擢副將。賊黨鄧之學詐降,詗知之,俟其入壘將半,文謨突起擒斬。從總兵慶溥防賊黃柏山,又從副都統德楞泰擊冉文儔等大神山,遷四川建昌鎮總兵。又與總兵德齡、副將褚大榮擊賊陳家場,德齡戰敗,文謨馳救,殺賊二百餘;又戰大竹、梁山、忠州,屢敗賊,擒其渠陳隴光等四十餘,防嘉陵江,遏賊不令渡:加提督銜。復督兵捕治川北餘匪,擢廣東提督。尋調福建水師提督。海盜蔡牽為亂,文謨渡海討之,並焚毀竹園尾、太史宮莊諸賊巢,再調浙江提督。卒,謚壯勇。

尚維升,漢軍鑲藍旗人,平南王可喜四世孫。自官學生授鑾儀衛整儀尉,五遷廣西右江鎮總兵。五十三年,隨兩廣總督孫士毅出師,十一月辛未,維升與副將慶成以兵千餘至壽昌江,阮惠軍保南岸,我兵乘之,浮橋斷,皆超筏直上,惠軍霧中自相格殺,我兵遂盡渡,大破賊,渡市球江,乘筏奪橋,奮勇直進,賜孔雀翎。渡富良江,斬獲甚眾,從士毅入黎城,士毅敗退,維升戰死,謚直烈。

張朝龍,山西大同人,寄籍貴州。以馬兵從征緬甸,戰老官屯,槍傷左額。又從征金川,攻阿喀爾布里、布朗郭宗。又從參贊大臣海蘭察自大板昭進剿,克喇穆喇穆、色淜普,朝龍先登。攻遜克爾宗,復先登,被槍傷。攻康薩爾山,戰勒吉爾博,攻達佳布唵吉,皆有功。又從攻勒烏圍,克之,賜藍翎。攻西里、阿穰曲,克木城十餘。又攻雅瑪朋、格隆古、索隆古諸地碉寨,克之。金川平,敘功,賜孔雀翎。累擢廣東撫標中軍參將。五十二年,臺灣林爽文為亂,朝龍率廣東兵進剿,多所斬獲,賜誠勇巴圖魯名號。進攻大里杙,槍傷右肩,爽文就擒。朝龍復與諸軍合攻莊大田於瑯嶠,擒之。臺灣平,圖形紫光閣,列後三十功臣。擢福建南澳鎮總兵。五十三年,從討安南,師渡壽昌江。朝龍以別軍破阮惠軍於柱石,進臨市球江,江寬,南岸群山綿亙,惠軍據險列砲,我師不能結筏。諸將督兵陽運竹木造浮橋示且渡,而朝龍以兵二千循上游二十里,求得流緩處,小舟宵濟。諸將乘筏薄南岸,方與惠軍相持,朝龍自上游繞出惠軍後,乘高下擊,惠軍潰。復進薄富良江,奪艦渡河,入黎城。士毅敗退,朝龍戰死,謚壯果。

李化龍,山東齊東人。自武進士授藍翎侍衛,擢貴州銅仁協都司。從大學士傅恆討緬甸,師次老官屯,化龍以大砲殺賊。乾隆三十七年,又從將軍溫福討金川,克固卜濟、瑪爾迪克諸碉卡。嗣進攻路頂宗、明郭宗等處,化龍皆力戰有功。明年三月,師次昔嶺,化龍射賊渠殪。征小金川,克阿噶爾布里、別斯滿諸地。從都統海蘭察克兜烏山梁,復連克路頂宗、明郭宗諸地,旋收美諾。征大金川,從海蘭察攻克喇穆喇穆諸地,被石傷,賜綿甲。先後攻克遜克爾宗、格魯古、群尼、木思工噶克諸地山梁,被槍傷,賜孔雀翎。金川平,累遷廣東左翼總兵。林爽文為亂,率廣東兵赴剿,至鹿仔港,總兵普吉保令化龍留守。爽文攻諸羅急,化龍密令游擊穆騰額率兵自番仔溝至大肚溪為疑兵,而親率游擊裴起鰲等自八卦山抵柴坑,賊聚拒,化龍督兵力戰,賊潰。五十三年,從討安南,師渡市球江,阮惠軍拒戰,化龍督兵發砲擊賊,造浮橋,與張朝龍等率兵徑渡,入黎城。士毅敗退,至市球江,令化龍先渡,渡浮橋,落水死。

邢敦行,直隸安州人。乾隆四十三年一甲一名武進士。自頭等侍衛累遷廣東三江口協副將。阮惠攻黎城,戰死。敦行事母孝,將出戰,解衣付其僕,使歸告母。

予恤,維升、朝龍三等輕車都尉,化龍、敦行騎都尉。諸裨將同時死者二十一人。師還,經富良江,惠軍追至,戰死者九人。又有參將鄧永亮、都司盧文魁,以出師時戰死。

臺斐英阿,庫雅拉氏,滿洲正白旗人。自護軍補司轡長,授乾清門藍翎侍衛。乾隆三十九年,從征金川,命為領隊。與內大臣海蘭察等攻喇穆喇穆山梁,破碉,毀木城,復循山梁逐賊至其麓。進攻該布達什諾,奪賊碉;再進,圍遜克爾宗,毀碉二百餘;再進,克默格爾以西及凱立葉前山梁諸碉卡;擢三等侍衛。復自羅卜克鄂博逾溝攻格魯克古丫口,破沙木拉渠革什式圖諸寨;復從領隊大臣福康安攻勒吉爾博山脊,克兩碉,進攻薩克薩谷山梁及舍圖柱卡,再進攻克覺拉喇嘛寺,及所屬卦爾沙巴等寨;賜號拉布凱巴圖魯。又偕海蘭察攻章噶山峰,進攻托古魯,潛師自山嶺涉險攀援而上,盡破之。再進,遂克勒烏圍。師自達烏達圍攻達思裏,海蘭察分兵七隊,臺斐英阿領其一,自懸崖下,夜半抵達烏達圍,奪碉一。及旦,至當噶克底,乘霧薄碉,賊眾皆就戮。從攻阿穰曲,克大碉、木城各二。進攻布魯木山峰,連克舍勒固租魯、瓦喇占、薩爾克爾、古什拉斯等諸寨。又從福康安攻雍中喇嘛寺,盡降其喇嘛,擢二等侍衛。金川平,圖形紫光閣,列前五十功臣。

四十六年,從剿撒拉爾叛回,敗賊龍尾山梁;登華林山,殲賊無算。賊平,擢頭等侍衛。從剿甘肅石峰堡叛回,以功加副都統銜,補公中佐領,擢御前侍衛。旋授正藍旗滿洲副都統,擢正紅旗護軍統領,調鑲黃旗。

五十六年,徵廓爾喀,從福康安分攻擦木,克之。進攻濟嚨,率索倫勁騎沖擊,轉戰至東覺山,克賊寨十一,砲殪賊目二,俘七十有六。加都統銜,授散秩大臣。進逼甲爾古拉山,賊三道來犯,臺斐英阿射斃紅衣賊目二,突中槍,卒於陣,謚果肅,賜白金千。廓爾喀平,再圖形紫光閣,列前十五功臣,予騎都尉又一雲騎尉世職。

阿滿泰,郭佳氏,滿洲正白旗人,本黑龍江達呼爾披甲。從征回疆,攻喀什噶爾城,逐賊自阿拉楚爾至巴達克山,獲其渠,令入旗充護軍。乾隆三十八年,授藍翎侍衛。從征金川,攻當噶爾拉山梁,賊自庚額特山出,阿滿泰與前鋒參領巴克坦布據險要殪賊。攻達爾旺山梁,克之。攻格木勺,截甲索賊來路。與侍衛阿蘭保等攻科拉木達,撲碉,勝援賊。擢三等侍衛,賜號扎努恩巴圖魯。攻扎烏孤山梁、加雜肚、絨布、巴魯坦諸處,皆有功。金川平,圖形紫光閣,列後五十功臣,擢副護軍參領。蘭州回為亂,從軍攻華林山,殲賊百餘,身被創,擢護軍參領。攻石峰堡,偵賊底店,奪卡,擢頭等侍衛。從征廓爾喀,自中路破擦木隘口,出濟嚨,破其官寨;進破賊熱索橋,渡河至雅爾寨,登博爾東拉山巔,破木城三、石卡七:授鑲紅旗蒙古副都統。進至堆補木,自帕朗古攻橫河大橋,我師臨北岸,賊據南岸禦。阿滿泰先登,師從之。渡橋,阿滿泰中槍,落水死,水深,戰方急,求其尸不可得。賜騎都尉世職,祀昭忠祠。廓爾喀平,再圖形紫光閣,列前十五功臣。

花連布,額爾德特氏,蒙古鑲黃旗人。性質直。少讀書,習論語、左傳。充健銳營前鋒,累遷火器營委署鳥槍護軍參領。以參將發湖廣,授武昌城守營參將,累遷貴州安籠鎮總兵。乾隆六十年,福康安徵貴州亂苗,令將精兵三千為前驅,通松桃、銅仁兩路餉道;援永綏,釋正大營圍:賜孔雀翎。軍自啞喇塘經阿寨營、安靜關轉戰而入,經巖板橋,收諸碉寨。又經上下麻洲、高陂塘、上下長坪,自嗅腦至松桃,平緣道苗卡,填坑谷過大軍。上以花連布奮勇,賜號剛安巴圖魯,賚白金百。又戰卡落塘,擊梁帽寨,且戰且前。時永綏被圍已八十餘日,花連布軍至,方戰,圍始解。苗皆烏合,未見大敵,相驚為神兵。花連布著豹皮戰裙督戰,因呼為花老虎。又擊賊小排吾,攻巴茅汛、鴨酉、黃瓜諸寨。自滾牛坡循崖下攻臘夷寨,槍傷左腋。上手詔獎其勇,問創已愈未。復自葫蘆坪攻克黨槽、三家廟諸寨,焚上下竹排。再進破桿子坳,屯軍古哨營山梁。上錄花連布功,授貴州提督。

福康安軍至,令結壘大營前,悉以兵事屬之,日置酒高會。苗詗知福康安持重不戰,一日數至,花連布力禦之,晝夜徼循,苗屢敗,頗畏憚。福康安益易視之,苗益掠焚無忌。頭人吳半生集群苗拒戰,花連布與額勒登保會總兵那丹珠等合軍攻爆木林,克苗寨十餘。深入,自成光寨至上下狗腦坡,山峻險,冒矢石,援藤葛,直陟山巔,苗漸卻。分兵下攻,福康安焚附坡諸苗寨;花連布督兵伐竹木,薰窒大小巖峒,死者枕藉。又自貓頭進克茶林碉、上下麻沖諸寨。下黃毛山坡,遇苗兵數千,額勒登保迎戰,花連布出賊後夾擊,大破之。再進,克馬腦、豬革、殺苗坪、竹子諸寨。分兵攻巖板井、瀼水沱、溪頭、綠樹沖、關鑲坪諸隘,皆下。吳半生亡匿高多寨,與諸軍分道入,環攻之,生得半生。又有頭人吳八月據平隴,自稱吳三桂後,糾黨轉盛。福康安令花連布引兵攻鵝洛等二十四寨,皆下。進攻龍角峒,奮戰,自辰至酉,乃克之。附近諸苗寨皆降。又克大坡腦等三十餘卡。攻鴨保,去平隴七十里。時已昏,風大作,山木動搖,崖高溝窄。花連布督兵攀越,縱火痛擊,破木城七、石卡五。旁收垂藤、董羅諸寨,遂擒八月。其子廷禮、廷義猶據險,乘勝克小、中、大三天星寨。再進,取黃沖口等十三寨,得盤、木營兩山梁。歲暮雨雪,進圍地良坡,收八荊、桃花諸寨。轉戰經連雲山、猴子山、蛇退嶺、壁多山、高吉陀,下貴道嶺等四十餘卡。抵長吉山,圍石城,未至平隴三十里所。

詔責復乾州。時福康安感瘴卒,和琳代將,令花連布率兵攻全壁嶺,自馬鞍山入,山蔽城,下瞰大河。將濟,懼苗涉水相襲,花連布分兵剿旁近諸寨翼大軍,遂復乾州。會和琳亦卒,上諭湖南巡撫姜晟以軍事諮花連布。貴州清溪民高承德以邪術糾眾為亂,戕縣吏,花連布督軍捕治,克槐花坪四寨。進攻小竹山,破其寨,殲承德及戕縣吏賊;再進攻大小鬼,戮餘賊。嘉慶元年九月丁卯日加已,賊攻夏家沖,花連布令副將海格、參將施縉張兩翼擊賊,賊數千拒戰。花連布出其中逐賊,賊見攻急,據坡擲石,花連布方上坡,中石,自巖墮深澗,罵賊,賊欲鉤出之;自力轉入巖下,頸折死。諸將爭殺賊,賊卻,出花連布尸,顱骨寸寸折,失一臂。上愍其死事烈,加太子少保,賜騎都尉兼雲騎尉世職,賚白金八百,謚壯節。

明安圖,博爾濟吉特氏,蒙古正紅旗人。以雲騎尉授三等侍衛,累遷湖南保靖營游擊。從征金川,大小戰五十有四,敘功,累遷鎮筸鎮總兵。貴州、湖北苗石柳鄧、石三保等糾眾為亂,明安圖督兵禦戰,永綏協副將伊薩納赴援,同戰死。苗攻滾牛坡,劫我軍饋運,雲南鶴麗鎮中營游擊永舒、四川阜和協左營都司班第共擊之,沒於陣。

論曰:師再徵金川,歷四年,大小數百戰,將士夷傷眾矣。全、順等平時力戰功最,死事尤凜凜。扎拉豐阿等死緬甸,與明瑞並烈。世亨等死安南,以全孫士毅,賞尤厚。臺斐英阿死廓爾喀,福康安因以受降還師。花連布善戰,死,不欲為群苗得,糜軀矢節,其狀視諸死事者尤慘,烈矣哉!


\end{pinyinscope}