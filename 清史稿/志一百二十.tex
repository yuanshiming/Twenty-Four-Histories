\article{志一百二十}

\begin{pinyinscope}
○藝文一

清起東陲,太宗設文館,命達海等繙譯經史。復改國史、祕書、弘文三院,編纂國史,收藏書籍,文教始興。世祖入定中原,命馮銓等議修明史,復詔求遺書。聖祖繼統,詔舉博學鴻儒,修經史,纂圖書,稽古右文,潤色鴻業,海內彬彬向風焉。高宗繼試鴻詞,博採遺籍,特命輯修四庫全書,以皇子永瑢、大學士於敏中等為總裁,紀昀、陸錫熊等為總纂,與其事者三百餘人,皆極一時之選,歷二十年始告成。全書三萬六千冊,繕寫七部,分藏大內文淵閣,圓明園文源閣,盛京文溯閣,熱河文津閣,揚州文匯閣,鎮江文宗閣,杭州文瀾閣。命紀昀等撰全書總目,著錄三千四百五十八種,存目六千七百八十八種,都一萬二百四十六種。復命於敏中、王際華擷其精華,別為四庫薈要,凡一萬二千冊,分繕二部,藏之大內摛藻堂及御園味腴書屋。又別輯永樂大典三百八十五種,交武英殿以聚珍版印行。時大典儲翰林院者尚存二萬四百七十三卷,合九千八百八十一冊。其宋、元精槧,多儲內府,天祿琳瑯,備詳宮史。經籍既盛,學術斯昌,文治之隆,漢、唐以來所未逮也。各省先後進書,約及萬種,阮元既補四庫未收書四百五十四種,復刊經解一千四百十二卷,王先謙又刊續經解一千三百十五卷,而各省督撫,廣修方志,郡邑典章,粲然大備。其後曾國籓倡設金陵、蘇州、揚州、杭州、武昌官書局,張之洞設廣雅書局,延聘儒雅,校刊群籍,私家亦輯刻日多,叢書之富,曩代莫京。及至晚近,歐風東漸,競譯西書,道藝並重。而敦煌寫經,殷墟龜甲,奇書祕寶,考古所資,其有裨於學術者尤多,實集古今未有之盛焉。藝文舊例,胥列古籍,茲仿明史為志,凡所著錄,斷自清代。唯清人輯古佚書甚夥,不可略之,則附載各類之後。

經部十類:一曰易類,二曰書類,三曰詩類,四曰禮類,五曰樂類,六曰春秋類,七曰孝經類,八曰四書類,九曰經總義類,十曰小學類。

易類

易經通注九卷。順治十三年,傅以漸等奉敕撰。日講易經解義十八卷。康熙二十二年,牛鈕等奉敕撰。周易折中二十二卷。康熙五十四年,李光地等奉敕撰。周易述義十卷。乾隆二十年,傅恆等奉敕撰。易圖解一卷,周易補注十一卷。簡親王德沛撰。易翼二卷。孫承澤撰。讀易大旨五卷。孫奇逢撰。周易稗疏四卷,考異一卷,周易內傳六卷,發例一卷,周易大象解一卷,周易外傳七卷。王夫之撰。易學象數論六卷。黃宗羲撰。周易象辭二十一卷,尋門餘論二卷,圖書辨惑一卷。黃宗炎撰。讀易筆記一卷。張履祥撰。周易說略四卷。張爾岐撰。易酌十四卷。刁包撰。易聞十二卷。歸起先撰。田間易學十二卷。錢澄之撰。大易則通十五卷,閏一卷,易史一卷。胡世安撰。周易疏略四卷。張沐撰。易學闡十卷。黃與堅撰。讀易緒言二卷。謝文洊撰。易經衷論二卷。張英撰。讀易日鈔六卷。張烈撰。周易通論四卷,周易觀彖大指二卷,周易觀彖十二卷。李光地撰。周易淺述八卷。陳夢雷撰。周易定本一卷。邵嗣堯撰。易經識解五卷。徐秉義撰。易經筮貞四卷。趙世對撰。周易明善錄二卷。徐繼發撰。易原就正十二卷。包儀撰。周易通十卷,周易辨正二十四卷。浦龍淵撰。合訂刪補大易集義粹言八十卷。納喇性德撰。周易筮述八卷。王弘撰撰。周易應氏集解十三卷。應手為謙撰。仲氏易三十卷,推易始末四卷,春秋占筮書三卷,易小帖五卷,太極圖說遺議一卷。河圖洛書原舛編一卷。毛奇齡撰。喬氏易俟十八卷。喬萊撰。大易通解十卷。魏荔彤撰。周易本義蘊四卷,周易蘊義圖考二卷。姜兆錫撰。周易傳注七卷,周易筮考一卷。李恭撰。學易初津二卷,易翼宗六卷,易翼說八卷。晏斯盛撰。周易劄記二卷。楊名時撰。易經詳說不分卷。冉覲祖撰。易經辨疑七卷。張問達撰。周易傳義合訂十二卷。硃軾撰。易宮三十六卷,讀易管窺五卷。吳隆元撰。讀易觀象惺惺錄十六卷,讀易觀象圖說二卷,太極圖說二卷,周易原始一卷,天水答問一卷,羲皇易象二卷,羲皇易象新補二卷。李南暉撰。孔門易緒十六卷。張德純撰。易圖明辨十卷。胡渭撰。身易實義五卷。沈廷勱撰。先天易貫五卷。劉元龍撰。易互六卷。楊陸榮撰。周易玩辭集解十卷,易說一卷。查慎行撰。易說六卷。惠士奇撰。周易函書約存十八卷,約注十八卷,別集十六卷。胡煦撰。易箋八卷。陳法撰。周易觀象補義略不分卷。諸錦撰。索易臆說二卷。吳啟昆撰。周易孔義集說二十卷。沈起元撰。陸堂易學十卷。陸奎勛撰。易經揆一十一卷,易學啟蒙補二卷。梁錫興撰。易經詮義十五卷,易經如話十五卷。汪紱撰。周易本義爻徵二卷。吳日慎撰。周易圖說正編六卷。萬年茂撰。易翼述信十二卷。王又樸撰。周易原始六卷。範咸撰。周易淺釋四卷。潘思★撰。易學大象要參四卷。[二]林贊龍撰。周易解翼十卷。上官章撰。東易問八卷。魏樞撰。周易洗心九卷。任啟運撰。空山易解四卷。牛運震撰。周易剩義二卷。童能靈撰。周易匯解衷翼十五卷。許體元撰。易象援古不分卷。申爾宣撰。豐川易說十卷。王心敬撰。周易粹義五卷。薛雪撰。周易圖說六卷。蔡新撰。讀易別錄三卷。全祖望撰。周易經言拾遺十四卷。徐文靖撰。易象大意存解一卷。任陳晉撰。周易集解纂疏三十六卷。李道平撰。周易圖書質疑二十四卷。趙繼序撰。易象通義六卷。秦篤輝撰。易深八卷。許伯政撰。易說存悔二卷。汪憲撰。卦氣解一卷,八卦觀象解二卷,彖傳論一卷,彖象論一卷,系辭傳論二卷。莊存與撰。易例舉要五卷,十家易象集說九十卷。吳鼎撰。周易大衍辨一卷。吳鼐撰。周易井觀十二卷,周大樞撰。周易注疏校正一卷。盧文弨撰。易守三十二卷。葉佩蓀撰。周易二閭記三卷,周易小義二卷。茹敦和撰。周易輯要五卷。硃瓚撰。易卦私箋二卷。蔣衡撰。易經明洛義六卷。孫慎行撰。易卦圖說一卷。崔述撰。周易章句證異十一卷。翟均廉撰。周易考占一卷。金榜撰。易經貫一二十二卷。金誠撰。周易辨畫四十卷。連鬥山撰。大易擇言三十六卷,程氏易通十四卷,易說辨正四卷。程廷祚撰。周易懸象八卷。黃元御撰。周易本義注六卷。胡方撰。周易略解八卷。馮經撰。周易述二十三卷,易漢學八卷,易例二卷,易微言二卷,易大誼一卷,周易本義辨證五卷,增補周易鄭注一卷,周易鄭注爻辰圖一卷,易說六卷。惠棟撰。觀象居易傳箋十二卷。汪師韓撰。周易述翼五卷。黃應騏撰。周易述補五卷。李林松撰。孫氏周易集解十卷。孫星衍撰。卦本圖考一卷。胡秉虔撰。周易虞氏義九卷,虞氏消息二卷,虞氏易禮二卷,虞氏易事二卷,虞氏易言二卷,虞氏易候一卷,虞氏易變表二卷,周易鄭氏義二卷,周易荀氏九家義一卷,易義別錄十四卷,易圖條辨一卷,易緯略義三卷。張惠言撰。易大義補一卷。桂文燦撰。學易討原一卷。姚文田撰。易說十二卷,易說便錄二卷。郝懿行撰。易經衷要十二卷。李式穀撰。易章句十二卷,易通釋二十卷,易圖略八卷,周易補疏二卷,易餘籥錄二十卷,易話二卷,易廣記二卷。焦循撰。易經異文釋六卷,李氏集解賸義三卷,校異二卷。李富孫撰。易問四卷,觀易外編六卷。紀大奎撰。周易指三十八卷,易例一卷,易圖五卷,易斷辭一卷。端木國瑚撰。卦氣解一卷,周易考異二卷。宋翔鳳撰。古易音訓二卷。宋咸熙撰。周易倚數錄二卷,圖一卷。楊履泰撰。周易虞氏略例一卷。李銳撰。周易學三卷。沈夢蘭撰。周易述補四卷。江籓撰。六十四卦經解八卷,易鄭氏爻辰廣義二卷,易經傳互卦卮言一卷,易章句異同一卷,易消息升降圖二卷,學易札記四卷。硃駿聲撰。易經述傳二卷,周易訟卦淺說一卷,周易解詁一卷,易經象類一卷。丁晏撰。周易姚氏學十六卷,周易通論月令二卷,易學闡元一卷。姚配中撰。虞氏易消息圖說一卷。胡祥麟撰。易確十二卷。許桂林撰。易漢學考二卷,易漢學師承表一卷,易彖傳大義述一卷,易爻例一卷。吳翊寅撰。周易附說一卷。羅澤南撰。周易舊疏考證一卷。劉毓崧撰。讀易叢記二卷。葉名灃撰。周易舊注十二卷。徐鼐撰。鄭氏爻辰補六卷。戴棠撰。周易爻辰申鄭義一卷。何秋濤撰。諸家易學別錄一卷,虞氏易學匯編一卷,周易卦象集證一卷,周易互體詳述一卷,周易卦變舉要一卷。方申撰。周易故訓訂一卷。黃以周撰。易例輯略五卷。龐大堃撰。易貫五卷,玩易篇一卷,艮宧易說一卷,邵易補原一卷,卦氣直日解一卷,易窮通變化論一卷,八卦方位說一卷,卦象補考一卷,周易互體徵一卷。俞樾撰。陳氏易說四卷,讀易漢學私記一卷。陳壽熊撰。易釋四卷。黃式三撰。讀易筆記二卷。方宗誠撰。周易釋爻例一卷。成蓉鏡撰。易解說二卷。吳汝綸撰。易經通論一卷。皮錫瑞撰。

唐史徵周易口訣義六卷。宋司馬光溫公易說六卷。宋邵伯溫易學辨惑一卷。宋李光讀易詳說十卷。宋鄭剛中周易窺餘十五卷。宋都絜易變體義十二卷。宋程大昌易原八卷。宋趙善譽易說四卷。宋徐總幹易傳燈四卷。宋馮椅厚齋易學五十二卷。宋蔡淵易象意言一卷。宋李杞周易詳解十六卷。宋俞琰讀易舉要四卷。宋丁易東周易象義十六卷。元吳澄易纂言外翼八卷。元解蒙易精蘊大義十二卷。元曾貫易學變通六卷。以上均乾隆三十八年王際華等奉敕輯。周卜氏易傳一卷。漢孟喜周易章句一卷。漢京房周易章句一卷。漢馬融周易傳一卷。漢荀爽周易注一卷。漢鄭玄周易注三卷,補遺一卷。漢劉表周易章句一卷。漢宋衷周易注一卷。魏董遇周易章句一卷。魏王肅周易注一卷。蜀範長生周易注一卷。吳陸績周易述一卷。吳姚信周易注一卷。吳虞翻周易注十卷。晉王廙周易注一卷。晉張璠周易集解一卷。晉向秀周易義一卷。晉干寶周易注一卷。晉翟玄周易義一卷。齊劉巘周易義疏一卷。以上均孫堂輯。連山一卷。歸藏一卷。漢蔡景君易說一卷。漢丁寬易傳二卷。漢韓嬰易傳二卷。漢劉安周易淮南九師道訓一卷。漢施讎周易章句一卷。漢梁丘賀周易章句一卷。漢費直易注一卷。易林一卷。周易分野一卷。古五子易傳一卷。不著時代薛虞周易記一卷。魏王肅周易音一卷。魏何晏周易解一卷。晉鄒湛周易統略一卷。晉楊乂周易卦序論一卷。晉張軌周易義一卷。晉黃穎周易注一卷。晉徐邈周易音一卷。晉李軌周易音一卷。晉孫盛易象妙於見形論一卷。晉桓玄周易系辭注一卷。宋荀柔之周易系辭注一卷。齊明僧紹周易系辭注一卷。齊沈驎士周易要略一卷。梁武帝周易大義一卷。梁伏曼容周易集解一卷。梁褚仲都周易講疏一卷。陳周弘正周易義疏一卷。陳張譏周易講疏一卷。後魏盧景裕周易注一卷。後魏劉昞周易注一卷。隋何妥周易講疏一卷。隋侯果周易注三卷。不著時代姚規周易注一卷。崔覲周易注一卷。王凱沖周易注一卷。王嗣宗周易義一卷。傅氏周易注一卷。莊氏易義一卷。唐崔憬周易探元三卷。唐李淳風周易元義一卷。唐陰弘道周易新論傳疏一卷。唐徐勛周易新義一卷。唐僧一行易纂一卷。以上均馬國翰輯。齊劉巘乾坤義一卷。黃奭輯。漢京房易飛候一卷。晉郭璞易洞林一卷。以上均王謨輯。

書類

日講書經解義十三卷。康熙十九年,庫勒納等奉敕編。書經傳說匯纂二十四卷。康熙六十年,王頊齡等奉敕撰。書經圖說五十卷。光緒二十九年奉敕撰。尚書近指六卷。孫奇逢撰。書經稗疏四卷,尚書引義六卷。王夫之撰。書經筆授三卷。黃宗羲撰。尚書體要六卷。錢肅潤撰。尚書埤傳十七卷,禹貢長箋十二卷。硃鶴齡撰。尚書集解二十卷,九州山川考三卷,洪範經傳集義一卷。孫承澤撰。書經衷論四卷。張英撰。尚書解義一卷,尚書句讀一卷,洪範說一卷。李光地撰。古文尚書考一卷。陸隴其撰。古文尚書疏證八卷。閻若璩撰。古文尚書冤詞八卷,尚書廣聽錄五卷,舜典補亡一卷。毛奇齡撰。古文尚書辨一卷。硃彞尊撰。禹貢錐指二十卷,圖一卷,洪範正論五卷。胡渭撰。書經蔡傳參議六卷。姜兆錫撰。禹貢解八卷。晏斯盛撰。尚書地理今釋一卷。蔣廷錫撰。尚書質疑八卷。王心敬撰。禹貢譜二卷。王澍撰。尚書質疑二卷。顧棟高撰。今文尚書說三卷。陸奎勛撰。書經詮義十二卷。汪紱撰。尚書約註四卷。任啟運撰。禹貢會箋十二卷。徐文靖撰。尚書注疏考證一卷。齊召南撰。尚書既見三卷,尚書說一卷。莊存與撰。晚書訂疑三卷。程廷祚撰。尚書注疏校正三卷。盧文弨撰。尚書質疑二卷,尚書異讀考六卷。趙佑撰。尚書後案三十卷,附後辨一卷。王鳴盛撰。尚書小疏一卷。沈彤撰。尚書釋天六卷。盛百二撰。禹貢三江考三卷。程瑤田撰。古文尚書考二卷。惠棟撰。古文尚書辨偽二卷。崔述撰。尚書義考二卷。戴震撰。古文尚書撰異三十二卷。段玉裁撰。古文尚書正辭三十三卷。吳光耀撰。尚書讀記一卷。閻循觀撰。尚書今古文疏證七卷。莊述祖撰。禹貢川澤考二卷。桂文燦撰。大雲山房十二章圖說一卷。惲敬撰。尚書今古文注疏三十卷,古文尚書馬鄭注十卷,尚書逸文二卷。孫星衍撰。禹貢地理古注考一卷。孫馮翼撰。尚書訓詁一卷。王引之撰。尚書敘錄一卷。胡秉虔撰。尚書集注音疏十二卷,尚書經師系表一卷。江聲撰。尚書周誥考辨二卷。章謙存撰。禹貢鄭注釋二卷,尚書補疏二卷。焦循撰。書說二卷。郝懿行撰。尚書略說二卷,尚書譜二卷。宋翔鳳撰。逸湯誓考六卷。徐時棟撰。尚書隸古定釋文八卷,附經文二卷。李遇孫撰。書經異文釋八卷。李富孫撰。尚書今古文集解三十一卷,書序述聞一卷。劉逢祿撰。古文尚書私議二卷。張崇蘭撰。召誥日名考一卷。李銳撰。尚書古注便讀四卷。硃駿聲撰。禹貢集釋三卷,禹貢錐指正誤一卷,禹貢蔡傳正誤一卷,尚書餘論一卷。丁晏撰。太誓答問一卷。龔自珍撰。禹貢正字一卷。王筠撰。尚書伸孔篇一卷。焦廷琥撰。尚書通義二卷,尚書傳授異同考一卷。邵懿辰撰。尚書沿革表一卷。戴熙撰。禹貢舊疏考證一卷。劉毓崧撰。尚書今文二十八篇解。楊鍾泰撰。禹貢鄭注略例一卷。何秋濤撰。尚書後案駁正二卷。王劼撰。考正胡氏禹貢圖一卷。陳澧撰。今文尚書經說考三十二卷,尚書歐陽夏侯遺說考一卷。陳喬樅撰。虞書命羲和章解一卷。曾釗撰。書傳補商十七卷。戴鈞衡撰。書古微十二卷。魏源撰。達齋書說一卷,生霸死霸考一卷,九族考一卷。俞樾撰。禹貢說一卷。倪文蔚撰。書傳補義一卷。方宗誠撰。尚書歷譜二卷,禹貢班義述三卷。成蓉鏡撰。尚書故三卷。吳汝綸撰。尚書古文辨惑十八卷,釋難二卷,析疑一卷,商是一卷。洪良品撰。書經通論一卷,今文尚書考證三十卷。皮錫瑞撰。尚書孔傳參正三十六卷。王先謙撰。尚書大傳考異補遺一卷。盧文弨撰。別本尚書大傳三卷,補遺一卷。孫之騄撰。尚書大傳注四卷。孔廣林撰。尚書大傳注五卷,五行傳注三卷。陳壽祺撰。

宋胡瑗洪範口義二卷。宋毛晃禹貢指南四卷。宋程大昌禹貢論五卷,後論一卷,山川地理圖一卷。宋史浩尚書講義二十卷。宋夏僎尚書詳解二十六卷。宋傅寅禹貢說斷四卷。宋楊簡五誥解四卷。宋袁燮絜齋家塾書鈔十二卷。宋黃倫尚書精義五十卷。宋錢時融堂書解二十卷。宋趙善湘洪範統一一卷。以上均乾隆三十八年王際華等奉敕輯。今文尚書一卷。古文尚書三卷。漢歐陽生尚書章句一卷。漢夏侯建尚書章句一卷。漢馬融尚書傳四卷。魏王肅尚書注二卷。晉徐邈古文尚書音一卷。晉範甯尚書舜典注一卷。隋劉焯尚書義疏一卷。隋劉炫尚書述義一卷。隋顧彪尚書疏一卷。以上均馬國翰輯。漢伏勝尚書大傳四卷。漢張霸百兩篇一卷。漢劉向五行傳二卷。以上均王謨輯。漢鄭玄尚書注九卷,尚書五行傳注一卷,尚書略說注一卷。以上均袁鈞輯。

詩類

詩經傳說匯纂二十卷,序二卷。康熙六十年,王鴻緒等奉敕撰。詩義折中二十卷。乾隆二十年,傅恆等奉敕撰。詩經稗疏四卷,詩經考異一卷,詩廣傳五卷。王夫之撰。田間詩學十二卷。錢澄之撰。詩說簡正錄十卷。提橋撰。詩經通義十二卷。硃鶴齡撰。毛詩稽古篇三十卷。陳啟源撰。詩問一卷。汪琬撰。毛詩日箋六卷。秦松齡撰。詩所八卷。李光地撰。毛硃詩說一卷。閻若璩撰。毛詩寫官記四卷,詩札二卷,國風省篇一卷,詩傳詩說駁義五卷,續詩傳鳥名三卷,白鷺洲主客說詩一卷。毛奇齡撰。詩蘊四卷。姜兆錫撰。詩識名解十五卷。姚炳撰。毛詩國風繹一卷,讀詩隨記一卷。陳遷鶴撰。詩傳名物集覽十二卷。陳大章撰。詩說三卷,附錄一卷。惠周惕撰。詩經劄記一卷。楊名時撰。陸堂詩學十二卷。陸奎勛撰。讀詩質疑三十一卷,附錄十五卷。嚴虞惇撰。硃子詩義補正八卷。方苞撰。詩經測義四卷。李鍾僑撰。毛詩類說二十一卷,續編三卷。顧棟高撰。詩疑辨證六卷。黃中松撰。毛詩說二卷。諸錦撰。詩經詮義十五卷。汪紱撰。毛詩名物圖說九卷。徐鼎撰。詩經正解三十卷。姜文燦撰。毛詩說四卷。莊存與撰。詩細十二卷,毛詩草木鳥獸蟲魚疏校正二卷。趙佑撰。虞東學詩十二卷。顧鎮撰。三家詩拾遺十卷,詩水審二十卷。範家相撰。詩序補義二十四卷。姜炳章撰。讀風偶識四卷。崔述撰。毛詩廣義不分卷。紀昭撰。毛鄭詩考正四卷,詩經補注二卷。戴震撰。詩經小學四卷,毛詩故訓傳三卷。段玉裁撰。童山詩說四卷。李調元撰。邶風說一卷。龔景瀚撰。詩志八卷。牛運震撰。詩考異字箋餘十四卷。周邵蓮撰。韓詩內傳徵四卷,敘錄二卷。宋綿初撰。韓詩外傳校注十卷。周廷寀撰。毛詩考證四卷,周頌口義三卷。莊述祖撰。毛詩證讀不分卷,讀詩或問一卷。戚學標撰。三家詩補遺三卷。阮元撰。毛詩天文考一卷。洪亮吉撰。韓詩遺說二卷,訂譌一卷。臧庸撰。詩古訓十卷。錢大昭撰。詩譜補亡後訂一卷。吳騫撰。毛詩傳箋異義解十六卷。沈鎬撰。毛詩通說三十卷,補遺一卷。任兆麟撰。毛詩補疏五卷,毛詩地理釋四卷,陸璣毛詩疏考證一卷。焦循撰。三家詩遺說考一卷。陳壽祺撰。詩經補遺一卷。郝懿行撰。詩說二卷,待問二卷。郝懿行妻王照圓撰。詩氏族考六卷。李超孫撰。詩經異文釋十六卷。李富孫撰。詩序辨正八卷。汪大任撰。毛詩紬義二十四卷。李黼平撰。毛詩後箋三十卷。胡承珙撰。山中詩學記五卷。徐時棟撰。三家詩異文疏證六卷,補遺三卷,續補遺二卷。馮登府撰。重訂三家詩拾遺十卷。葉鈞撰。多識錄九卷。石韞玉撰。毛鄭詩釋四卷,鄭氏詩譜考正一卷,詩考補注二卷,補遺一卷,毛詩陸疏校正二卷,詩集傳附釋一卷。丁晏撰。讀詩札記八卷,詩章句考一卷,詩樂存亡譜一卷,硃子詩集傳校勘記一卷。夏炘撰。毛詩通考二十卷,毛詩識小三十卷,鄭氏詩譜考正一卷。林伯桐撰。毛詩禮徵十卷。包世榮撰。齊詩翼氏學二卷。迮鶴壽撰。讀詩小牘二卷。焦廷琥撰。詩古微二十卷。魏源撰。毛詩傳箋通釋三十二卷。馬瑞辰撰。三家詩遺說考四十九卷,毛詩鄭箋改字說四卷,四家詩異文考五卷,齊詩翼氏學疏證二卷,詩緯集證四卷。陳喬樅撰。詩經集傳拾遺二卷。吳德旋撰。詩名物證古一卷,達齋詩說一卷,讀韓詩外傳一卷。俞樾撰。詩毛氏傳疏三十卷,鄭氏箋考徵一卷,釋毛詩音四卷,毛詩說一卷,毛詩傳義類一卷。陳奐撰。詩小學三十卷。吳樹聲撰。毛詩多識二卷。多隆阿撰。詩學詳說三十卷,正詁五卷。顧廣譽撰。詩地理徵七卷。硃右曾撰。詩本誼一卷。龔橙撰。詩經異文四卷,韓詩輯一卷。蔣曰豫撰。毛詩序傳三十卷,毛詩讀三十卷。王劼撰。毛詩異文箋十卷。陳玉樹撰。毛詩譜一卷。胡元儀撰。詩經通論一卷。皮錫瑞撰。詩三家義集疏二十九卷。王先謙撰。

宋楊簡慈湖詩傳二十卷。宋戴溪續呂氏家塾讀詩記三卷。宋袁燮絜齋毛詩經筵講義四卷。宋林岊毛詩講義十二卷。元劉玉汝詩纘緒十八卷。以上均乾隆三十八年王際華等奉敕輯。漢申培魯詩故三卷。漢後蒼齊詩傳三卷。漢韓嬰詩故二卷,詩內傳一卷,詩說一卷。漢薛漢韓詩章句二卷。漢侯苞韓詩翼要一卷。漢馬融毛詩注一卷。魏劉楨毛詩義問一卷。魏王肅毛詩注一卷,毛詩義駁一卷,毛詩奏事一卷,毛詩問難一卷。魏王基毛詩駁一卷。吳韋昭、硃育毛詩答難問一卷。吳徐整毛詩譜暢一卷。晉孫毓毛詩異同評三卷。晉陳統難孫氏毛詩評一卷。晉郭璞毛詩拾遺一卷。晉徐邈毛詩音一卷。齊劉瓛毛詩序義一卷。宋周續之毛詩周氏注一卷。梁簡文帝毛詩十五國風義一卷。梁何胤毛詩隱義一卷。梁崔靈恩集注毛詩一卷。不著時代舒瑗毛詩義疏一卷。不著時代、撰人毛詩草蟲經一卷,毛詩提綱一卷。後周沈重毛詩義疏二卷。後魏劉芳毛詩箋音義證一卷。隋劉炫毛詩述義一卷。唐施士丐詩說一卷。以上均馬國翰輯。漢轅固齊詩傳一卷。魏王基毛詩申鄭義一卷。均黃奭輯。漢鄭玄毛詩譜一卷。王謨輯。

◎禮類

周官義疏四十八卷。乾隆十三年,鄂爾泰等奉敕撰。周官筆記一卷。李光地撰。周禮述注二十四卷。李光坡撰。高註周禮二十卷。高愈撰。周官辨非一卷。萬斯大撰。周禮問二卷。毛奇齡撰。周禮訓纂二十一卷。李鍾倫撰。周禮節訓六卷。黃叔琳撰。周官集注十二卷,周官析疑三十六卷,考工記析義四卷,周官辨一卷。方苞撰。周官翼疏三十卷。沈淑撰。周禮輯義十二卷。姜兆錫撰。禮說十四卷。惠士奇撰。周官記六卷,周官說二卷,周官說補三卷。莊存與撰。周禮疑義舉要七卷。江永撰。周禮精義十二卷。連鬥山撰。周官祿田考三卷。沈彤撰。周官祿田考補正三卷。倪景曾撰。考工記圖注二卷。戴震撰。周禮軍賦說四卷。王鳴盛撰。周禮漢讀考六卷。段玉裁撰。田賦考一卷。任大椿撰。考工記論文一卷。牛運震撰。周禮故書考一卷。程際盛撰。周禮故書疏證六卷。宋世犖撰。車制圖考一卷。阮元撰。車制考一卷。錢坫撰。周官肊測六卷,敘錄一卷。孔廣林撰。周禮學二卷。王聘珍撰。周官故書考四卷。徐養原撰。周禮畿內授田考實一卷。胡匡衷撰。周官禮鄭氏注箋十卷。莊綬甲撰。周禮學一卷。沈夢蘭撰。周禮釋注二卷。丁晏撰。考工輪輿私箋二卷。鄭珍撰。圖一卷。珍子知同撰。周官注疏小箋五卷。曾釗撰。考工記考辨八卷。王宗涑撰。周禮補注六卷。呂飛鵬撰。周官參證二卷。王寶仁撰。周禮正義八十六卷。孫詒讓撰。

宋王安石周官新義十六卷,附考工記解二卷。宋易祓周官總義三十卷。元毛應龍周官集傳十六卷。以上均乾隆三十八年王際華等奉敕輯。漢鄭興周禮解詁一卷。漢鄭眾周禮解詁六卷。漢杜子春周禮注二卷。漢賈逵周禮解詁一卷。漢馬融周官傳一卷。漢鄭玄周禮音一卷。晉干寶周禮注一卷。晉徐邈周禮音一卷。晉李軌周禮音一卷。晉陳邵周官禮異同評一卷。不著時代劉昌宗周禮音二卷。聶氏周禮音一卷。後周沈重周官禮義疏一卷。陳戚袞周禮音一卷。以上均馬國翰輯。

以上禮類周禮之屬

儀禮義疏四十八卷。乾隆十三年,鄂爾泰等奉敕撰。儀禮鄭注句讀十七卷,附監本正誤一卷。張爾岐撰。讀禮通考一百二十卷。徐乾學撰。儀禮述注十七卷。李光坡撰。儀禮商二卷,附錄一卷。萬斯大撰。喪禮吾說篇十卷,三年服制考一卷。毛奇齡撰。喪服翼注一卷。閻若璩撰。儀禮章句十七卷。吳廷華撰。儀禮節要二十卷。硃軾撰。儀禮析疑十七卷,喪禮或問一卷。方苞撰。儀禮經傳內編二十三卷,外編五卷。姜兆錫撰。饗禮補亡一卷。諸錦撰。朝廟宮室考十三卷,肆獻祼饋食禮三卷。任啟運撰。禮經本義十七卷。蔡德晉撰。儀禮釋宮增注一卷,儀禮釋例一卷。江永撰。儀禮小疏一卷。沈彤撰。儀禮管見十七卷。褚寅亮撰。喪服文足徵記十卷。程瑤田撰。儀禮注疏詳校十七卷。盧文弨撰。儀禮漢讀考一卷。段玉裁撰。儀禮集編四十卷。盛世佐撰。儀禮今古文疏證二卷。宋世犖撰。禮經釋例十三卷,目錄一卷。凌廷堪撰。儀禮圖六卷,讀儀禮記二卷。張惠言撰。冕服考四卷。焦廷琥撰。儀禮今古文異同疏證五卷。徐養原撰。儀禮校正十七卷。黃丕烈撰。禮經宮室答問二卷。洪頤煊撰。儀禮經注一隅一卷。硃駿聲撰。儀禮釋官九卷,鄭氏儀禮目錄校正一卷。胡匡衷撰。儀禮學一卷。王聘珍撰。儀禮今古文疏義十七卷。胡承珙撰。喪禮經傳約一卷。吳卓信撰。儀禮正義四十卷。胡培翬撰。儀禮宮室提綱一卷。胡培系撰。儀禮經注疏正譌十七卷。金曰追撰。儀禮禮服通釋六卷。凌曙撰。儀禮釋注二卷。丁晏撰。儀禮私箋八卷。鄭珍撰。讀儀禮錄一卷。曾國籓撰。喪服會通說四卷。吳嘉賓撰。士昏禮對席圖一卷,喪服私論一卷。俞樾撰。昏禮重別論對駁義二卷。劉壽曾撰。

宋李如圭儀禮集釋三十卷,儀禮釋宮一卷。以上均乾隆三十八年王際華等奉敕輯。蔡氏月令二卷。蔡云輯。漢戴德喪服變除一卷。漢何休冠禮約制一卷。漢鄭眾昏禮一卷。漢馬融喪服經傳注一卷。漢鄭玄喪服變除一卷。漢劉表新定禮一卷。魏王肅喪經傳注一卷,喪服要記一卷。吳射慈喪服變除圖一卷。晉杜預喪服要集一卷。晉袁準喪服經傳注一卷。晉孔倫集注喪服經傳一卷。晉劉智喪服釋疑一卷。晉蔡謨喪服譜一卷。晉賀循喪服譜一卷,葬禮一卷,喪服要記一卷。晉葛洪喪服變除一卷。晉孔衍兇禮一卷。不著時代陳銓喪服經傳注一卷。謝徽喪服要記注一卷。宋裴松之集注喪服經傳一卷。宋雷次宗略注喪服經傳一卷。宋崔凱喪服難問一卷。宋周續之喪服注一卷。齊王儉喪服古今集記一卷。齊王逡之喪服世行要記一卷。以上均馬國翰輯。

以上禮類儀禮之屬

日講禮記解義六十四卷。乾隆元年敕編。禮記義疏八十二卷。乾隆十三年敕撰。禮記章句四十九卷。王夫之撰。深衣考一卷。黃宗羲撰。禮記纂編六卷。李光地撰。禮記述注二十八卷。李光坡撰。禮記偶箋三卷。萬斯大撰。廟制圖考四卷。萬斯同撰。陳氏禮記集說補正三十八卷。納喇性德撰。曾子問講錄四卷,檀弓訂誤一卷。毛奇齡撰。禮記章義十卷。姜兆錫撰。禮記疑義十八卷。吳廷華撰。禮記析疑四十六卷。喪禮或問一卷。方苞撰。戴記緒言四卷。陸奎勛撰。禮記章句十卷,或問四卷。汪紱撰。禮記章句十卷。任啟運撰。檀弓疑問一卷。邵泰衢撰。禮記訓義擇言八卷,深衣考誤一卷。江永撰。學禮闕疑八卷。劉青蓮撰。續衛氏禮記集說一百卷。杭世駿撰。禮記注疏考證一卷。齊召南撰。禮記注疏校正一卷。盧文弨撰。祭法記疑二卷。王元啟撰。明堂大道錄八卷,禘說二卷。惠棟撰。深衣釋例三卷,弁服釋例八卷。任大椿撰。撫州本禮記鄭注考異二卷。張敦仁撰。釋服二卷。宋綿初撰。明堂考三卷。孫星衍撰。明堂億一卷。孔廣林撰。禮記鄭讀考四卷,禮記天算釋一卷。孔廣牧撰。盧氏禮記解詁一卷,蔡氏月令章句一卷。臧庸撰。禮記集解六十一卷。孫希旦撰。七十二候考一卷。曹仁虎撰。禮記補疏三卷。焦循撰。禮記說八卷。楊秉杷撰。禮記異文釋八卷。李富孫撰。禮記箋四十九卷。郝懿行撰。禮記宮室答問二卷。洪頤煊撰。燕寢考三卷。胡培翬撰。禮記經注正譌六十三卷。金曰追撰。禮記訓纂四十九卷。硃彬撰。禮記釋注四卷,投壺考原一卷。丁晏撰。檀弓辨誣三卷。夏炘撰。禮記鄭讀考六卷。陳喬樅撰。禮記質疑四十九卷。郭嵩燾撰。禮記異文箋一卷,禮記鄭讀考一卷,七十二候考一卷。俞樾撰。禮記集解補義一卷。方宗誠撰。禮記淺說二卷。皮錫瑞撰。

宋張虙月令解十二卷。宋袁甫蒙齋中庸講義四卷。以上均乾隆三十八年王際華等奉敕輯。漢馬融禮記注一卷。漢盧植禮記注一卷。漢荀爽禮傳一卷。漢蔡邕月令章句一卷,月令問答一卷。魏王肅禮記注一卷。魏孫炎禮記注一卷。不著時代謝氏禮記音義隱一卷。晉範宣禮記音一卷。晉徐邈禮記音一卷。不著時代劉昌宗禮記音一卷。宋庾蔚之禮記略解一卷。梁何胤禮記隱義一卷。梁賀瑒禮記新義疏一卷。梁皇侃禮記義疏四卷。後魏劉芳禮記義證一卷。後周沈重禮記義疏一卷。後周熊安生禮記義疏四卷。唐成伯興禮記外傳一卷。以上均馬國翰輯。唐明皇月令注釋一卷。黃奭輯。吳射慈禮記音義隱一卷。漢蔡邕明堂月令論一卷。漢崔寔四民月令一卷。以上均王謨輯。

以上禮類禮記之屬

夏小正解一卷。徐世溥撰。曾子問天員篇一卷。梅文鼎撰。夏小正註一卷。黃叔琳撰。夏小正詁一卷。諸錦撰。夏小正輯注四卷。範家相撰。夏小正考注一卷。畢沅撰。曾子注釋四卷。阮元撰。大戴禮記補注十三卷,敘錄一卷。孔廣森撰。夏小正經傳考釋十卷。莊述祖撰。夏小正傳校正三卷。孫星衍撰。大戴禮解詁十三卷,敘錄一卷。王聘珍撰。大戴禮記正誤一卷。汪中撰。夏小正分箋四卷,異義二卷。黃謨撰。大戴禮記箋證五卷。胡培系撰。大戴禮記補注十三卷,目錄一卷,附錄一卷。汪照撰。大戴禮記考一卷。吳文起撰。夏小正傳箋四卷,公符篇考一卷。王謨撰。夏小正補注四卷。任兆麟撰。夏小正補傳三卷。硃駿聲撰。夏小正經傳通釋四卷。梁章鉅撰。夏時考五卷。安吉撰。夏時考一卷。劉逢祿撰。夏小正經傳考二卷,本義四卷。雷學淇撰。夏小正集解四卷,校錄一卷。顧鳳藻撰。孔子三朝記七卷,目錄一卷。洪頤煊撰。夏小正疏義四卷,附釋音異字記一卷。洪震煊撰。夏小正正義四卷。王筠撰。夏小正箋疏四卷。馬徵麟撰。夏小正集說四卷。程鴻詔撰。夏時考一卷。鄭曉如撰。夏小正戴氏傳訓解四卷,考異一卷,通論一卷。王寶仁撰。夏小正私箋一卷。吳汝綸撰。

以上禮類大戴禮之屬

學禮質疑二卷,宗法論一卷。萬斯大撰。讀禮志疑十三卷,禮經會元疏解四卷。陸隴其撰。郊社禘祫問一卷,北郊配位尊西向義一卷,昏禮辨正一卷,大小宗通繹十卷,明堂問一卷,廟制折衷一卷,學校問一卷。毛奇齡撰。參讀禮志疑二卷。汪紱撰。釣臺遺書四卷。任啟運撰。禮經質疑二卷。杭世駿撰。稽禮辨論一卷。劉凝撰。三禮鄭注考一卷。程際盛撰。禮箋三卷。金榜撰。禮學卮言六卷。孔廣森撰。五服異同匯考二卷。崔述撰。禘祫觿解篇一卷。孔廣林撰。三禮義證十卷。武億撰。白虎通闕文一卷。莊述祖撰。三禮圖三卷。孫星衍、嚴可均同撰。禮說四卷。凌曙撰。鄭氏三禮目錄一卷。臧庸撰。禘祫答問一卷。胡培翬撰。禮堂經說二卷。陳喬樅撰。三禮陳數求義三十卷。陳喬廕撰。四禘通釋三卷。崔適撰。白虎通疏證十二卷。陳立撰。三禮通釋二百八十卷。林昌彞撰。求古錄禮說十六卷,補遺一卷。金鶚撰。求古錄禮說校勘記三卷。王士駿撰。學禮管釋十八卷,三綱制服述義三卷。夏炘撰。佚禮扶微五卷。丁晏撰。禮經通論一卷。邵懿辰撰。積石禮說三卷。張履撰。禮說二卷。吳嘉賓撰。鄭康成駁正三禮考一卷,玉佩考一卷。俞樾撰。禮書通故一百卷,禮說略三卷。黃以周撰。經述三卷。林頤山撰。

漢戴聖石渠禮論一卷,漢鄭玄魯禮禘祫志一卷,三禮圖一卷,魏董勛問禮俗一卷,晉盧諶雜祭法一卷,晉範汪祭典一卷,晉干寶後養義一卷,晉範甯禮雜問一卷,晉範宣禮論難一卷,晉吳商禮雜議一卷,宋顏延之逆降義一卷,宋徐廣禮論答問一卷,宋何承天禮論一卷,宋任豫禮論條牒一卷,齊王儉禮義答問一卷,齊荀萬秋禮論鈔略一卷,梁賀述禮統一卷,梁周舍禮疑義一卷,梁崔靈恩三禮義宗四卷,後魏李謐明堂制度論一卷,不著時代梁正三禮圖一卷,唐張鎰三禮圖一卷,唐元行沖釋疑論一卷。以上均馬國翰輯。漢叔孫通禮器制度一卷,漢鄭玄三禮目錄一卷,晉孫毓五禮駁一卷。以上均王謨輯。漢鄭玄答臨碩難禮一卷。袁鈞輯。

以上禮類總義之屬

硃子禮纂五卷。李光地撰。辨定祭禮通俗譜五卷,家禮辨說十六卷。毛奇齡撰。讀禮偶見二卷。許三禮撰。呂氏四禮翼一卷。硃軾撰。禮學匯編七十卷。應手為謙撰。禮樂通考三十卷。胡掄撰。禮書綱目八十五卷。江永撰。六禮或問十二卷。汪紱撰。四禮寧儉編一卷。王心敬撰。五禮通考二百六十二卷。秦蕙田撰。五禮經傳目五卷。沈廷芳撰。冠昏喪祭儀考十二卷。林伯桐撰。三禮從今三卷。黃本驥撰。四禮榷疑八卷。顧廣譽撰。

以上禮類通禮之屬

樂類

律呂正義五卷。康熙五十二年御撰。律呂正義後編一百二十卷。乾隆十一年敕撰。詩經樂譜三十卷,樂律正俗一卷。乾隆五十三年敕撰。樂律二卷。薛鳳祚撰。大成樂律一卷。孔貞瑄撰。古樂經傳五卷。李光地撰。聖諭樂本解說二卷,皇言定聲錄八卷,竟山樂錄四卷。毛奇齡撰。古樂書二卷。應手為謙撰。李氏學樂錄二卷。李恭撰。昭代樂章恭紀一卷。張廷玉撰。易律通解八卷。沈光邦撰。樂律古義二卷。童能靈撰。樂經律呂通解五卷,樂經或問三卷。汪紱撰。樂律表微八卷。胡彥升撰。琴旨三卷。王坦撰。律呂新論二卷,律呂闡微十卷。江永撰。律呂考略一卷。孔毓焞撰。大樂元音七卷。潘士權撰。律呂古義六卷。錢塘撰。燕樂考原六卷,晉泰始笛律匡謬一卷。凌廷堪撰。樂懸考二卷。江籓撰。樂譜一卷。任兆麟撰。律呂臆說一卷,荀勖笛律圖註一卷,管色考一卷。徐養原撰。古律經傳附考六卷。紀大奎撰。樂志輯略三卷。倪元坦撰。音分古義二卷,附一卷。戴煦撰。聲律通考十卷。陳澧撰。律呂通今圖說一卷,律易一卷,音調定程一卷。繆闐撰。

元熊朋來瑟譜六卷,元餘載韶舞九成樂補一卷,元劉瑾律呂成書二卷。以上均乾隆三十八年王際華等奉敕輯。漢陽城子長樂經一卷,漢劉向樂記一卷,漢劉德樂元語一卷,漢揚雄琴清英一卷,梁武帝樂社大義一卷、鍾律緯一卷,陳僧智匠古今樂錄一卷,後魏信都芳樂書一卷,後周沈重樂律義一卷,不著時代、撰人樂部一卷,琴歷一卷,隋蕭吉樂譜集解一卷,唐趙惟暕琴書一卷。以上均馬國翰輯。漢劉歆鍾律書一卷,漢蔡邕琴操一卷。以上均黃奭輯。

春秋類

左傳讀本三十卷。道光三年,英和等奉敕編。左傳杜解補正三卷。顧炎武撰。續春秋左氏傳博議四卷。王夫之撰。讀左日鈔十二卷,補錄二卷。硃鶴齡撰。左傳事緯十二卷,附錄八卷。馬驌撰。春秋地名考略十四卷。高士奇撰。春秋國都爵姓考一卷,補一卷。陳鵬撰。春秋分年系傳表一卷。翁方綱撰。春秋左傳事類年表一卷。顧宗瑋撰。春秋長歷十卷,春秋世族譜一卷。陳厚耀撰。春秋識小錄九卷。程廷祚撰。春秋左傳補注六卷。惠棟撰。春秋地理考實四卷。江永撰。讀左補義五十卷。姜炳璋撰。春秋左傳小疏一卷。沈彤撰。春秋左傳古經十二卷,附五十凡一卷。段玉裁撰。春秋左傳會要四卷,左傳官名考二卷。李調元撰。春秋左傳詁五十卷,春秋十論一卷。洪亮吉撰。春秋列國官名異同考一卷。汪中撰。左通補釋三十二卷。梁履繩撰。春秋左傳分國土地名二卷,春秋列國職官一卷,春秋器物宮室一卷。沈淑撰。左傳劉杜持平六卷。邵英撰。春秋名字解詁二卷。王引之撰。春秋左氏補疏五卷。焦循撰。讀左卮言一卷。石韞玉撰。左氏春秋考證二卷。劉逢祿撰。春秋左傳補注三卷。馬宗梿撰。左傳識小錄一卷。硃駿聲撰。春秋左傳補注十二卷,考異十卷,左傳地名補注十二卷。沈欽韓撰。春秋左氏古義六卷,臧壽恭撰。左傳賈服注輯述二十卷。李貽德撰。左傳杜注辨正六卷。張總咸撰。春秋國都爵姓續考一卷。曾釗撰。左傳舊疏考正八卷。劉文淇撰。春秋名字解詁補義一卷。俞樾撰。春秋世族譜拾遺一卷。成蓉鏡撰。春秋名字解詁駁一卷。胡元玉撰。補春秋僖公事闕書一卷。桑宣撰。

晉杜預春秋釋例十五卷,宋呂祖謙春秋左氏傳續說十二卷。以上均乾隆三十八年王際華等奉敕輯。漢劉歆春秋左氏傳章句一卷,漢賈逵春秋左氏傳解詁二卷、春秋左氏傳長經章句一卷,漢服虔春秋左傳解誼四卷,漢彭汪左氏奇說一卷,漢許淑春秋左傳注一卷,魏董遇春秋左氏經傳章句一卷,魏王肅春秋左傳注一卷,魏嵇康春秋左傳音一卷,晉孫毓春秋左氏傳義注一卷,晉干寶左氏傳函義一卷,陳沈文阿春秋左氏經傳義略一卷,陳王元規續春秋左氏經傳義略一卷,不著時代蘇寬春秋左氏傳義疏一卷。以上均馬國翰輯。隋劉炫左氏傳述義一卷。黃奭輯。漢鄭玄春秋傳服氏注十二卷。袁鈞輯。

以上春秋類左傳之屬

春秋正辭十一卷,春秋舉例一卷,春秋要指一卷。莊存與撰。公羊墨史二卷。周拱辰撰。春秋公羊通義十一卷,敘一卷。孔廣森撰。公羊何氏釋例十卷,公羊何氏解詁箋一卷,發墨守評一卷,箴膏肓評一卷,穀梁廢疾申何二卷。劉逢祿撰。公羊補注一卷。馬宗梿撰。公羊禮疏十一卷,公羊禮說一卷,公羊答問二卷,春秋繁露注十七卷。凌曙撰。春秋決事比一卷。龔自珍撰。公羊義疏七十六卷。陳立撰。公羊注疏質疑二卷。何若瑤撰。公羊歷譜十一卷。包慎言撰。公羊逸禮考徵一卷。陳奐撰。

漢董仲舒春秋決事一卷,漢嚴彭祖公羊春秋一卷,漢顏安樂春秋公羊記一卷,漢何休春秋公羊文謚例一卷。以上均馬國翰輯。

以上春秋類公羊之屬

穀梁釋例四卷。許桂林撰。穀梁大義述三十卷。柳興恩撰。穀梁禮證二卷。侯康撰。穀梁經傳補注二十四卷。鍾文烝撰。

漢尹更始春秋穀梁傳章句一卷,漢劉向春秋穀梁傳說一卷,魏糜信春秋穀梁注一卷,晉徐邈春秋穀梁傳注義一卷、音一卷,晉範甯薄叔元問穀梁義一卷,晉鄭嗣春秋穀梁傳說一卷。以上均馬國翰輯。晉範甯穀梁傳例一卷。黃奭輯。

以上春秋類穀梁之屬

春秋傳說匯纂三十八卷。康熙三十八年,王掞等奉敕撰。日講春秋解義六十四卷。雍正七年敕撰。春秋直解十六卷。乾隆二十三年,傅恆等奉敕撰。春秋稗疏二卷,春秋家說三卷,春秋世論五卷。王夫之撰。春秋平義十二卷,春秋四傳糾正一卷。俞汝言撰。春秋傳議四卷。張爾岐撰。學春秋隨筆十卷。萬斯大撰。春秋大義、春秋隨筆共一卷,春秋毀餘四卷。李光地撰。春秋毛氏傳卷。毛奇齡撰。春秋集解十二卷,校補春秋集解緒餘一卷,春秋提要補遺一卷。應手為謙撰。春秋參義十二卷,春秋事義慎考十四卷,公穀匯義十二卷。姜兆錫撰。春秋管窺十二卷。徐庭垣撰。春秋三傳異同考一卷。吳陳琰撰。春秋遵經集說二十八卷。邵鍾仁撰。三傳折諸四十四卷。張尚瑗撰。春秋闕如編八卷,小國春秋一卷。焦袁熹撰。春秋宗硃辨義十二卷。張自超撰。春秋通論四卷,春秋義法舉要一卷,春秋比事目錄四卷,春秋直解十二卷。方苞撰。半農春秋說十五卷。惠士奇撰。春秋義十五卷。孫嘉淦撰。春秋大事表五十卷,輿圖一卷,附錄一卷。顧棟高撰。春秋七國統表六卷。魏翼龍撰。春秋義存錄十二卷。陸奎勛撰。春秋日食質疑一卷。吳守一撰。春秋集傳十六卷,首、末各一卷。汪紱撰。空山堂春秋傳十二卷。牛運震撰。春秋原經二卷。王心敬撰。春秋深十九卷。許伯政撰。春秋一得一卷。閻循觀撰。三正考二卷。吳鼐撰。春秋三傳定說十二卷。張甄陶撰。春秋夏正二卷。胡天游撰。春秋究遺十六卷。葉酉撰。春秋隨筆二卷。顧奎光撰。春秋三傳雜案十卷,讀春秋存稿四卷。趙佑撰。三傳補注三卷。姚鼐撰。春秋三傳比二卷。李調元撰。春秋疑義二卷。華學泉撰。公穀異同合評四卷。沈赤然撰。春秋經傳朔閏表二卷。姚文田撰。春秋說略十二卷,春秋比二卷。郝懿行撰。春秋目論二卷。鄧顯撰。三傳異同考一卷。陳萊孝撰。春秋經傳比事二十二卷。林春溥撰。春秋三家異文覈一卷,春秋亂賊考一卷。硃駿聲撰。春秋三傳異文釋十三卷。李富孫撰。春秋屬辭辨例編六十卷。張應昌撰。春秋上律表四卷。範景福撰。春秋至朔通考四卷。張冕撰。駮正朔考一卷。陳鍾英撰。春秋三傳異文箋四卷。趙坦撰。春秋新義十二卷,春秋歲星表一卷,日食星度表一卷,日表一卷。硃兆熊撰。春秋釋地韻編五卷。徐壽基撰。春秋述義拾遺九卷,春秋規過考信九卷。陳熙晉撰。春秋古經說二卷。侯康撰。達齋春秋論一卷,春秋歲星考一卷,春秋古本分年考一卷。俞樾撰。春秋朔閏異同考三卷。羅士琳撰。春秋鉆燧四卷。曹金籀撰。春秋經傳朔閏表發覆四卷,推春秋日食法一卷。施彥士撰。春秋日月考四卷。譚澐撰。春秋朔閏日食考二卷。宋慶雲撰。春秋釋一卷。黃式三撰。春秋測義三十五卷。強汝詢撰。春秋說一卷。陶正靖撰。春秋傳正誼四卷。方宗誠撰。春秋日南至譜一卷。成蓉鏡撰。春秋說二卷。鄭杲撰。

宋劉敞春秋傳說例一卷,宋蕭楚春秋辨疑四卷,宋崔子方春秋經解十二卷、春秋例要一卷,宋張大亨春秋通訓六卷,宋葉夢得春秋考十六卷、春秋讞二十二卷,宋高閌春秋集注四十卷,宋戴溪春秋講義四卷,宋洪咨夔春秋說三十卷,元程端學春秋三傳辨疑二十卷。以上均乾隆三十八年王際華等奉敕輯。春秋大傳一卷,漢鄭眾春秋牒例章句一卷,漢馬融春秋三傳異同說一卷,漢戴宏解疑論一卷,漢穎容春秋釋例一卷,晉劉兆春秋公羊穀梁傳解詁一卷,晉江熙春秋公羊穀梁二傳評一卷,晉京相璠春秋土地名一卷,後魏賈思同春秋傳駁一卷,隋劉炫春秋述義一卷、春秋規過一卷、春秋攻昧一卷,不著時代、撰人春秋井田記一卷,唐啖助春秋集傳一卷,唐趙匡春秋闡微纂類義統一卷,唐陸希聲春秋通例一卷,唐陳嶽春秋折衷論一卷。以上均馬國翰輯。漢嚴彭祖春秋盟會圖一卷,晉樂資春秋後傳一卷。以上均黃奭輯。漢鄭玄箴膏肓一卷、起廢疾一卷、發墨守一卷。以上均王復、武億同輯。

以上春秋類通義之屬

孝經類

孝經注一卷。順治十三年御撰。孝經集注一卷。雍正五年敕撰。欽定繙譯孝經一卷。雍正五年敕撰。孝經全注一卷。李光地撰。孝經問一卷。毛奇齡撰。孝經類解十八卷。吳之騄撰。孝經正文一卷,內傳一卷,外傳一卷。李之素撰。孝經集注二卷。陸遇霖撰。孝經詳說二卷。冉覲祖撰。孝經注三卷。硃軾撰。孝經三本管窺三卷。吳隆元撰。孝經章句一卷,或問一卷。汪紱撰。孝經章句一卷。任啟運撰。孝經通義一卷。華玉淳撰。孝經約義一卷。汪師韓撰。孝經外傳一卷,孝經中文一卷。周春撰。孝經音義考證一卷。盧文弨撰。孝經通釋十卷。曹庭棟撰。孝經鄭注補證一卷。洪頤煊撰。孝經義疏補九卷。阮福撰。孝經述注一卷,孝經徵文一卷。丁晏撰。孝經曾子大孝一卷。邵懿辰撰。孝經指解補正一卷,辨異一卷。伊樂堯撰。孝經今古文傳注輯論一卷。吳大廷撰。孝經十八章輯傳一卷。汪宗沂撰。孝經鄭注疏二卷。皮錫瑞撰。

明項霦孝經述注一卷。乾隆三十八年,王際華等奉敕輯。周魏文侯孝經傳一卷,漢後蒼孝經說一卷,漢張禹孝經安昌侯說一卷,漢長孫氏孝經說一卷,魏王肅孝經解一卷,吳韋昭孝經解贊一卷,晉殷仲文孝經注一卷,晉謝萬集解孝經一卷,齊永明諸王孝經講義一卷,齊劉瓛孝經說一卷,梁武帝孝經義疏一卷,梁嚴植之孝經注一卷,梁皇侃孝經義疏一卷,隋劉炫古文孝經述義一卷,隋魏真己孝經訓注一卷,唐元行沖御注孝經疏一卷。以上均馬國翰輯。漢鄭玄孝經注一卷。袁鈞輯。

四書類

日講四書解義二十六卷。康熙十六年,庫勒納奉敕撲。繙譯四書集注二十九卷。乾隆二十年敕譯。四書近指二十卷。孫奇逢撰。大學講義一卷,中庸講義二卷。硃用純撰。孟子師說二卷。黃宗羲撰。四書訓義三十八卷,讀四書大全說十七卷,四書稗疏一卷,四書考異一卷。王夫之撰。四書反身錄十四卷,續錄二卷。李顒撰。四書翊注四十二卷。刁包撰。四書講義困勉錄三十七卷,續困勉錄六卷,松陽講義十二卷,三魚堂四書大全四十卷。陸隴其撰。大學古本說一卷,中庸章段一卷,中庸餘論一卷,讀論語劄記二卷,讀孟子劄記二卷。李光地撰。四書述十九卷。陳詵撰。四書貫一解十二卷。閔嗣同撰。論語稽求篇七卷,四書賸言四卷,補二卷,大學證文四卷,四書改錯二十二卷,四書索解四卷,大學知本圖說一卷,大學問一卷,中庸說五卷,逸講箋三卷。毛奇齡撰。四書釋地一卷,續一卷,又續二卷,三續二卷,孟子生卒年月考一卷。閻若璩撰。四書硃子異同條辨四十卷。李沛霖、李楨撰。四書諸儒輯要四十卷。李沛霖撰。大學傳注四卷,中庸傳注一卷,論語傳注二卷,傳注問一卷。李恭撰。四書劄記四卷,闢雍講義一卷,大學講義二卷,中庸講義二卷。楊名時撰。四書講義四十三卷。呂留良撰。大學困學錄一卷,中庸困學錄一卷。王澍撰。成均講義不分卷。孫嘉淦撰。大學翼真七卷。胡渭撰。此木軒四書說九卷。焦袁熹撰。大學說一卷。惠士奇撰。四書詮義十五卷。汪紱撰。中庸解一卷。任大任撰。四書錄疑三十九卷。陳綽撰。四書本義匯參四十五卷。王步青撰。論語說二卷。桑調元撰。四書約旨十九卷。任啟運撰。論語隨筆二十卷。牛運震撰。論語附記二卷,孟子附記二卷。翁方綱撰。四書溫故錄十一卷。趙佑撰。四書逸箋六卷。程大中撰。四書注說參證七卷。胡清(取火)撰。鄉黨圖考十卷。江永撰。魯論說三卷。程廷祚撰。四書考異總考三十六卷,條考三十六卷。翟灝撰。論語補注三卷。劉開撰。論語駢枝一卷。劉臺拱撰。孟子字義疏證三卷。戴震撰。論語後錄五卷。錢坫撰。論語餘說一卷。崔述撰。中庸注一卷。惠棟撰。四書摭餘說七卷。曹之升撰。四書偶談二卷。戚學標撰。四書考異句讀一卷。武億撰。四書拾義五卷。明紹勛撰。孟子四考四卷。周廣業撰。孟子七國諸侯年表一卷。張宗泰撰。論語偶記一卷。方觀旭撰。論語俟質三卷。江聲撰。孟子時事略一卷。任兆麟撰。論語古訓十卷。陳鱣撰。論語異文考證十卷。馮登府撰。論語補疏三卷,論語通釋一卷,孟子正義三十卷。焦循撰。讀論質疑一卷。石韞玉撰。四書瑣語一卷。姚文田撰。論語說義十卷,孟子趙注補正六卷,四書釋地辨證二卷,大學古義說二卷。宋翔鳳撰。論語魯讀考一卷。徐養原撰。大學舊文考證一卷,中庸舊文考證一卷。硃曰佩撰。論語旁證二十卷。梁章鉅撰。論語類考二十卷,孟子雜記四卷。陳士元撰。四書拾遺五卷,孟子外書補證四卷。林春溥撰。論語孔注辨偽二卷。沈濤撰。鄉黨正義一卷。金鶚撰。六書叚借經徵四卷。硃駿聲撰。孟子音義考證二卷。蔣仁榮撰。論語述何二卷,四書是訓十五卷。劉逢祿撰。論語古解十卷。梁廷枬撰。孟子學一卷。沈夢蘭撰。四書地理考十一卷。王鎏撰。四書釋地補一卷,續補一卷,又續補一卷,三續補一卷。樊廷枚撰。四書典故覈三卷。凌曙撰。大學臆古一卷,附古今文附證一卷,中庸臆測二卷。王定柱撰。四書說略四卷。王筠撰。論語集注附考一卷。丁晏撰。讀孟子劄記二卷。羅澤南撰。孟子班爵祿疏證十六卷,正經界疏證六卷。迮鶴壽撰。論語正義二十卷。劉寶楠撰。大學質疑一卷,中庸質疑二卷。郭嵩燾撰。論語古注集箋十卷,考一卷。潘維城撰。論語古注擇從一卷,論語鄭義一卷,何邵公論語義一卷,續論語駢枝一卷,論語小言一卷,孟子古注擇從一卷,孟子高氏義一卷,孟子纘義一卷,四書辨疑辨一卷。俞樾撰。論語注二十卷。戴望撰。何休注訓論語述一卷。劉恭冕撰。論語後案二十卷。黃式三撰。讀孟子質疑二卷,孟子外書集證五卷。施彥士撰。論語集解校補一卷。蔣曰豫撰。讀大學中庸筆記二卷,讀論孟筆記二卷,補記一卷。方宗誠撰。硃子論語集注訓詁考二卷。潘衍桐撰。

宋餘允文尊孟辨三卷,續辨二卷,別錄一卷。以上乾隆三十八年王際華等奉敕輯。古論語十卷,齊論語一卷,漢孔安國論語訓解十一卷,漢包咸論語章句二卷,漢周氏論語章句一卷,漢馬融論語訓說一卷,漢鄭玄論語注十卷、論語孔子弟子目錄一卷,魏陳群論語義說一卷,魏王朗論語說一卷,魏王肅論語義說一卷,魏周生烈論語義說一卷,魏王弼論語釋疑一卷,晉譙周論語注一卷,晉衛瓘論語集注一卷,晉繆播論語旨序一卷,晉繆協論語說一卷,晉郭象論語體略一卷,晉欒肇論語釋疑一卷,晉虞喜論語贊注一卷,晉庾翼論語釋一卷,晉李充論語集注二卷,晉範甯論語注一卷,晉孫綽論語集解一卷,晉梁凱論語注釋一卷,晉袁喬論語注一卷,晉江熙論語集解二卷,晉殷仲堪論語解釋一卷,晉張憑論語注一卷,晉蔡謨論語注解一卷,宋顏延之論語說一卷,宋僧慧琳論語說一卷,齊沈驎士論語訓注一卷,齊顧歡論語注一卷,梁武帝論語注一卷,梁太史叔明論語注一卷,梁褚仲都論語義疏一卷,不著時代沈峭論語說一卷,熊埋論語說一卷,不著時代、撰人論語隱義注一卷,漢趙岐孟子章指二卷、篇敘一卷,漢程曾孟子章句一卷,漢高誘孟子章句一卷,漢劉熙孟子注一卷,漢鄭玄孟子注一卷,晉綦母邃孟子注一卷,唐陸善經孟子注一卷,唐張鎰孟子音義一卷,唐丁公著孟子音一卷。以上馬國翰輯。逸論語一卷。趙在翰輯。逸語十卷。曹庭棟輯。逸孟子一卷。李調元輯。

經總義

繙譯五經五十八卷。乾隆二十年敕譯。五經翼二十卷。孫承澤撰。墨庵經學不分卷。沈起撰。經問十八卷,經問補三卷。毛奇齡撰。松源經說四卷。孫之騄撰。七經同異考三十四卷,韋庵經說一卷。周象明撰。此木軒經說匯編六卷。焦袁熹撰。十三經義疑十二卷。吳浩撰。經義雜記三十卷。臧琳撰。經稗六卷。鄭方坤撰。經玩二十卷。沈淑撰。硃子五經語類八十卷。程川撰。經咫一卷。陳祖範撰。經言拾遺十四卷。徐文靖撰。考信錄三十六卷,讀經餘論二卷。崔述撰。古經解鉤沈三十卷。餘蕭客撰。易堂問目四卷。吳鼎撰。九經說十七卷。姚鼐撰。群經補義五卷。江永撰。群經互解一卷。馮經撰。十三經札記二十二卷。硃亦棟撰。經學卮言六卷。孔廣森撰。經傳小記三卷,漢學拾遺一卷。劉臺拱撰。九經古義十六卷。惠棟撰。經考五卷。戴震撰。通藝錄四十八卷。程瑤田撰。群經釋地六卷。呂吳撰。五經小學述二卷。莊述祖撰。群經識小八卷。李惇撰。經義知新記一卷。汪中撰。詩書古訓八卷。阮元撰。浙士解經錄五卷。阮元編。周人經說四卷,王氏經說六卷。王紹蘭撰。九經學三卷。王聘珍撰。五經異義疏證三卷,左海經辨二卷。陳壽祺撰。邃雅堂學古錄七卷。姚文田撰。經義述聞三十二卷,經傳釋詞十卷。王引之撰。五經要義一卷,五經通義一卷。宋翔鳳撰。群經宮室圖二卷。焦循撰。頑石廬經說十卷。徐養原撰。經義未詳說五十四卷。徐卓撰。十七史經說十二卷。張養吾撰。經義叢鈔三十卷。嚴傑編。鳳氏經說三卷。鳳韶編。介庵經說十卷。雷學淇撰。十三經詁答問六卷。馮登府撰。說緯六卷。王崧撰。安甫遺學三卷。江承之撰。實事求是齋經說二卷。硃大韶撰。讀經說一卷。丁晏撰。玉函山房目耕帖三十一卷。馬國翰撰。漢儒通義七卷。陳澧撰。娛親雅言六卷。嚴元照撰。經傳考證八卷。硃彬撰。十三經客難五十五卷。龔元玠撰。一鐙精舍甲部稿五卷。何秋濤撰。群經平議三十五卷,茶香室經說十五卷,詁經精舍自課文二卷,經課續編八卷,群經賸義一卷,達齋叢說一卷。俞樾撰。開有益齋經說五卷。硃緒曾撰。讀書偶志十一卷。鄒漢勛撰。貴陽經說一卷,經說殘稿一卷。劉書年撰。巢經巢經說一卷,鄭學錄三卷。鄭珍撰。儆居經說四卷。黃式三撰。愚一錄十二卷。鄭獻甫撰。鏺經筆記一卷。陳倬撰。隸經賸義一卷。林兆豐撰。鄭志考證一卷。成蓉鏡撰。漢碑徵經一卷。硃百度撰。漢孳室經說一卷。陶方琦撰。經說略二卷。黃以周撰。操齋遺書四卷。管禮耕撰。經窺四卷。蔡以盛撰。九經誤字一卷,五經同異三卷。顧炎武撰。助字辨略五卷。劉淇撰。十三經注疏正字八十一卷。沈廷芳撰。注疏考證六卷。齊召南撰。九經辨字瀆蒙十二卷。沈炳震撰。經典釋文考證三十卷。盧文弨撰。經典文字考異一卷。錢大昕撰。群經義證八卷,經讀考異八卷,補一卷,句讀敘述二卷,補一卷。武億撰。經典文字辨正五卷。畢沅撰。十三經注疏校勘記二百十七卷,孟子音義校勘記一卷,釋文校勘記二十五卷。阮元撰。群經字考四卷。曾廷枚撰。十經文字通正書十四卷。錢坫撰。經苑不分卷。錢儀吉撰。七經異文釋五十卷。李富孫撰。群經字考十卷。吳東發撰。經典釋文補條例一卷。汪遠孫撰。經典異同四十八卷。張維屏撰。十三經注疏校勘記識語四卷。汪文臺撰。漢書引經異文錄證六卷。繆祐孫撰。授經圖四卷。明硃睦原本,黃虞稷、龔翔麟重編。十三經注疏姓氏一卷。翁方綱撰。建立伏博士始末二卷。孫星衍撰。傳經表一卷,通經表一卷。洪亮吉撰。西漢儒林傳經表二卷。周廷寀撰。漢西京博士考二卷。胡秉虔撰。兩漢五經博士考三卷。張金吾撰。兩漢傳經表二卷。蔣曰豫撰。國朝漢學師承記七卷,附經義目錄一卷,隸經文四卷。江籓撰。古文天象考十二卷,附圖說一卷。雷學淇撰。經書算學天文考一卷。陳懋齡撰。學計一得二卷。鄒伯奇撰。石經考一卷。顧炎武撰。石經正誤一卷。張爾岐撰。漢魏石經考一卷,唐宋石經考一卷。萬斯同撰。石經考異二卷。杭世駿撰。漢石經殘字考一卷。翁方綱撰。魏石經毛詩殘字一卷。王昶撰。蜀石經毛詩考異二卷。陳鱣撰。石經考文提要十三卷。彭元瑞撰。魏三體石經殘字考二卷。孫星衍撰。石經儀禮校勘記四卷。阮元撰。漢石經殘字證異二卷。孔廣牧撰。唐石經校文十卷。嚴可均撰。石經補考十二卷。馮登府撰。北宋汴學篆隸二體石經記一卷。丁晏撰。唐開成石經圖考一卷。魏錫曾撰。

漢劉向五經通義一卷,漢鄭玄六藝論一卷,鄭記一卷,不著時代雷氏五經要義一卷,魏王肅聖證論一卷,晉譙周五經然否論一卷,晉束晰五經通論一卷,晉楊芳五經鉤沈一卷,晉戴逵五經大義一卷,後魏常爽六經略注一卷,後魏邯鄲綽五經析疑一卷,後周樊文深七經義綱一卷,漢石經尚書一卷,魯詩一卷,儀禮一卷,公羊傳一卷,論語一卷,魏三字石經尚書一卷,春秋一卷。以上均馬國翰輯。漢鄭玄駁五經異義一卷,補遺一卷,魏鄭小同鄭志三卷,補遺三卷。以上均王復、武億同輯。

小學類

爾雅補注六卷。姜兆錫撰。爾雅補郭二卷。翟灝撰。爾雅正義二十卷,音義三卷。邵晉涵撰。爾雅補注四卷。周春撰。爾雅漢注三卷。臧庸撰。爾雅釋文補三卷。錢大昭撰。爾雅義疏二十卷。郝懿行撰。爾雅釋地以下四篇注四卷,爾雅古義二卷。錢坫撰。爾雅古義二卷。胡承珙撰。爾雅小箋三卷。江籓撰。爾雅古義十二卷。黃奭撰。爾雅注疏本證誤五卷。張宗泰撰。爾雅匡名二十卷。嚴元照撰。爾雅補注殘本一卷。劉玉麟撰。爾雅詁二卷。徐孚吉撰。爾雅郭注補正三卷。戴瑩撰。爾雅經注集證三卷。龍啟瑞撰。爾雅正郭三卷。潘衍桐撰。爾雅古注斠三卷。閨秀葉蕙心撰。續方言二卷。杭世駿撰。方言校正十三卷。盧文弨撰。方言補校一卷。劉臺拱撰。方言疏證十三卷。戴震撰。續方言補證一卷。程際盛撰。方言箋疏十三卷。錢繹撰。續方言疏證二卷。沈齡撰。釋名疏證八卷,補遺一卷,續釋名一卷。江聲撰。廣釋名二卷。張金吾撰。釋名補證一卷。成蓉鏡撰。廣雅疏義二十卷。錢大昭撰。廣雅疏證十卷。王念孫撰。小爾雅約注一卷。硃駿聲撰。小爾雅訓纂六卷。宋翔鳳撰。小爾雅義證十三卷。胡承珙撰。小爾雅疏八卷。王煦撰。小爾雅疏證五卷。葛其仁撰。補小爾雅釋度量衡一卷。鄒伯奇撰。字詁一卷。黃生撰。越語肯綮錄一卷。毛奇齡撰。連文釋義一卷。王言撰。別雅五卷。吳玉搢撰。經籍撰詁一百六卷,附補遺一百六卷。阮元撰。比雅十九卷。洪亮吉撰。釋繒一卷。任大椿撰。通詁二卷。李調元撰。越言釋二卷。茹敦和撰。釋廟一卷,釋車一卷,釋帛一卷,釋色一卷,釋詞一卷,釋農具一卷。硃駿聲撰。釋服一卷。宋翔鳳撰。釋穀一卷。劉寶楠撰。釋人注一卷。孫馮翼撰。釋祀一卷。董蠡舟撰。拾雅二十卷。夏味堂撰,夏紀堂注。駢字分箋二卷。程際盛撰,駢雅訓纂十六卷。魏茂林撰。周秦名字解詁補一卷。王萱齡撰。疊雅十三卷。史夢蘭撰。別雅訂五卷。許瀚撰。

漢郭舍人爾雅注三卷,漢劉歆爾雅注一卷,漢樊光爾雅注一卷,漢李巡爾雅注三卷,魏孫炎爾雅注三卷、音一卷,晉郭璞爾雅音義一卷、圖贊一卷,梁沈旋集注爾雅一卷,陳施乾爾雅音一卷,陳謝嶠爾雅音一卷,陳顧野王爾雅音一卷,唐裴瑜爾雅注一卷。以上馬國翰輯。吳韋昭辨釋名一卷。黃奭輯。

以上小學類訓詁之屬

康熙字典四十二卷。康熙五十五年,張玉書等奉敕撰。字典考證三十六卷。道光十一年,王引之奉敕撰。急就章考異一卷。孫星衍撰。急就章姓氏補注一卷。吳省蘭撰。急就章音略一卷,音略考證一卷。王紹蘭撰。急就章考證一卷。鈕樹玉撰。急就篇統箋一卷,急就姓氏考一卷。陳本禮撰。急就篇考異一卷。莊世驥撰。說文廣義三卷。王夫之撰。說文引經考二卷。吳玉搢撰。說文系傳考異四卷,附錄一卷。汪憲撰。說文答問一卷。錢大昕撰。六書通十卷。閔齊汲撰。說文偏旁考二卷。吳照撰。說文舊音一卷,音同字異辨一卷。畢沅撰。六書轉注古義考一卷。曹仁虎撰。說文解字段氏注三十卷,六書音韻表五卷,汲古閣說文訂一卷。段玉裁撰。惠氏讀說文記十五卷。惠棟撰。說文解字通正十四卷。潘奕俊撰。王氏讀說文記一卷,說文解字校勘記一卷。王念孫撰。說文補考一卷,漢學諧聲二十四卷,古音論一卷,附錄一卷。戚學標撰。說文古籀疏證六卷。莊述祖撰。說文古語考二卷。程際盛撰。六書轉注錄十卷。洪亮吉撰。說文解字義證五十卷,說文段注鈔案一卷,補一卷。桂馥撰。說文段注訂補十四卷。王紹蘭撰。說文徐氏新附考證一卷,說文統釋序注一卷。錢大昭撰。說文解字斠詮十四卷。錢坫撰。說文述誼二卷。毛際盛撰。說文字原集注十六卷,表一卷,說一卷。蔣和撰。席氏讀說文記十五卷。席世昌撰。說文管見三卷。胡秉虔撰。六書說一卷。江聲撰。說文校義三十卷。姚文田、嚴可均同撰。說文聲系十四卷,說文解字考異十四卷,偏旁舉略一卷。姚文田撰。說文翼十六卷,說文聲類二卷,說文訂訂一卷。嚴可均撰。說文五翼八卷。王煦撰。說文辨字正俗八卷。李富孫撰。說文解字群經正字二十八卷。邵英撰。說文通訓定聲十八卷,補遺一卷,柬韻一卷,說雅一卷,小學識餘四卷。硃駿聲撰。說文經字考一卷。陳壽祺撰。說文檢字二卷,補遺一卷。毛謨撰。說文雙聲疊韻譜一卷,鄧廷楨撰。形聲類編五卷。丁履恆撰。說文段注札記一卷。徐松撰。讀說文證疑一卷。陳詩庭撰。小學說一卷。吳夌云撰。說文古字考十四卷。沈濤撰。說文說一卷。孫濟世撰。說文系傳校錄三十卷,說文釋例二十卷,說文補正二十卷,說文解字句讀三十卷,句讀補正三十卷,說文韻譜校五卷,新附考校正一卷,正字略一卷,文字蒙求四卷。王筠撰。說文諧聲譜九卷,張成孫撰。說文段注訂八卷,說文新附考六卷,續考一卷,說文解字校錄三十卷,說文玉篇校錄一卷。鈕樹玉撰。說文釋例二卷,說文音韻表十八卷。江沅撰。說文段注匡謬八卷。徐承慶撰。說文辨疑一卷。顧廣圻撰。說文段注札記一卷。龔自珍撰。許氏說音四卷。許桂林撰。說文引經考異十六卷。柳榮宗撰。說文疑疑二卷。附錄一卷。孔廣居撰。說文拈字七卷,補遺二卷。王玉樹撰。說文校定本二卷。硃士端撰。說文答問疏證六卷。薛傳均撰。說文新附考六卷,說文逸字二卷,附錄一卷。鄭珍撰。說文聲讀考七卷,說文聲訂二卷,說文建首字讀一卷。苗夔撰。六書轉注說二卷。夏炘撰。說文諧聲孳生述一卷。陳立撰。說文引經考證八卷,說文舉例一卷。陳彖撰。讀說文記一卷。許梿撰。唐寫本說文木部箋異一卷。莫友芝撰。諧聲補逸十四卷,附札記一卷。宋保撰。六書系韻二十四卷,檢字二卷。李貞撰。說文雙聲二卷,說文疊韻二卷。劉熙載撰。兒笘錄四卷。俞樾撰。印林遺著一卷。許瀚撰。說文段注撰要九卷。馬壽齡撰。說文外編十六卷,說文引經例辨三卷。雷浚撰。說文揭原二卷,說文發疑六卷,汲古閣說文解字校記一卷。張行孚撰。說文解字索隱一卷,補例一卷。張度撰。說文系傳校勘記三卷。承培元、夏灝、吳永康撰。說文引經證例二十四卷。承培元撰。說文古籀補十四卷,補遺一卷,附錄一卷,字說一卷。吳大澂撰。說文本經答問二卷,說文淺說一卷。鄭知同撰。說文重文本部考一卷。曾紀澤撰。古籀拾遺三卷,附宋政和禮器文字考一卷,名原二卷。孫詒讓撰。說文引群說故二十七卷。鄭文焯撰。說文解字引漢律令考二卷,附錄一卷。王仁俊撰。水交民遺文一卷。孫傳鳳撰。小學考五十卷。謝啟昆撰。九經字樣疑一卷,五經文字疑一卷。孔繼涵撰。汗簡箋正七卷。鄭珍撰。隸釋刊誤一卷。黃丕烈撰。復古編校正一卷,附錄一卷。葛鳴陽撰。古音駢字續編五卷。莊履豐、莊鼎鉉同撰。繆篆分韻五卷,補一卷。桂馥撰。篆隸考異二卷。周靖撰。隸辨八卷。顧藹吉撰。隸法匯纂十卷。項懷述撰。漢隸拾遺一卷。王念孫撰。漢隸異同六卷。甘揚聲撰。隸通二卷。錢慶曾撰。隸篇十五卷,續十五卷,補十五卷。翟雲升撰。金石文字辨異十二卷。邢澍撰。鐘鼎字源五卷。汪立名撰。積古齋鐘鼎彞器款識十卷。阮元撰。筠清館金文五卷。吳榮光撰。從古堂款識學十六卷。徐同柏撰。攟古錄金文九卷。吳式芬撰。兩罍軒彞器圖釋十二卷。吳雲撰。攀古樓彞器款識二卷。潘祖廕撰。石鼓然疑一卷。莊述祖撰。石鼓文考釋一卷。任兆麟撰。石鼓文讀七種一卷。吳東發撰。石鼓文定本十卷。沈梧撰。續字匯補十二卷。吳志伊撰。字貫提要四十卷。王錫侯撰。字學辨正集成四卷。姚心舜撰。

倉頡篇三卷,續一卷,補二卷。孫星衍原輯,任大椿續輯,陶方琦補輯。小學鉤沈十八卷。任大椿輯。字林考逸八卷,補一卷。任大椿原輯,陶方琦補輯。周太史籀篇一卷,秦李斯等倉頡篇一卷,漢司馬相如凡將篇一卷,漢揚雄訓纂篇一卷,漢杜林倉頡訓詁一卷,漢服虔通俗文一卷,漢衛宏古文官書一卷,漢蔡邕勸學篇一卷,漢郭顯卿雜字指一卷,魏張揖埤蒼一卷、古今字詁一卷、雜字一卷,魏周成雜字解詁一卷,吳硃育異字一卷,吳項峻始學篇一卷,晉索靖草書狀一卷,晉衛恆四體書勢一卷,晉葛洪要用字苑一卷,晉束晰發蒙記一卷,晉顧愷之啟蒙記一卷,晉李彤字指一卷,附單行字一卷,宋何承天纂文一卷,宋顏延之庭誥一卷、纂要一卷、詁幼一卷,梁元帝纂要一卷,梁阮孝緒文字集略一卷,梁庾儼默演說文一卷,梁樊恭廣蒼一卷,後魏楊承慶字統一卷,後魏江式古今文字表一卷,隋曹憲文字指歸一卷,隋諸葛穎桂苑珠叢一卷,不著時代、撰人分毫字樣一卷。以上均馬國翰輯。後魏宋世良字略一卷,不著時代陸善經新字林一卷,字書一卷,唐開元文字音義一卷、小學一卷。以上均黃奭輯。

以上小學類字書之屬

易音三卷,詩本音十卷。顧炎武撰。詩葉韻辨一卷。王夫之撰。易韻四卷。毛奇齡撰。詩經葉音辨譌八卷。劉維謙撰。九經韻證一卷。吳廷華撰。十三經音略十三卷。周春撰。詩音表一卷。錢坫撰。詩音辨二卷。李調元撰。詩聲類十二卷,詩聲分例一卷。孔廣森撰。詩經韻讀四卷,群經韻讀一卷,先秦韻讀一卷。江有誥撰。詩聲衍一卷。劉逢祿撰。毛詩雙聲疊韻說一卷。王筠撰。毛詩韻訂十卷。苗夔撰。三百篇原聲七卷。夏味堂撰。爾雅直音二卷。王祖源撰。唐韻正二十卷,補正一卷。顧炎武撰。廣韻正四卷。李因篤撰。唐韻考五卷。紀容舒撰。唐韻四聲正一卷。江有誥撰。九經補韻考正一卷。錢繹撰。集韻考正十卷。方成珪撰。廣韻說一卷。吳夌云撰。集韻校誤四卷,群經音辨校誤一卷。陸心源撰。音論三卷,古音表二卷。顧炎武撰。古今通韻十二卷。毛奇齡撰。古今韻考四卷。李因篤撰。聲韻叢說一卷,韻問一卷。毛先舒撰。古音通八卷。柴紹炳撰。古今韻略五卷。邵長蘅撰。古音正義一卷。熊士伯撰。聲韻圖譜一卷。錢人麟撰。古韻標準四卷。江永撰。聲韻考四卷,聲類表十卷,轉語二十章。戴震撰。聲類四卷,音韻問答一卷。錢大昕撰。漢魏音四卷。洪亮吉撰。古音諧八卷。姚文田撰。古韻論三卷。胡秉虔撰。廿一部諧聲表一卷,入聲表一卷。江有誥撰。古今韻準一卷。硃駿聲撰。歌麻古韻考四卷。苗夔撰。五音論二卷。鄒漢勛撰。述韻十卷。夏燮撰。古韻通說四卷。龍翰臣撰。劉氏遺箸一卷。劉禧延撰。韻府鉤沈四卷。雷浚撰。欽定葉韻匯輯五十八卷。乾隆十五年,梁詩正等奉敕撰。榕村韻書五卷。李光地撰。韻歧四卷。江昱撰。詩韻析五卷,附錄二卷。汪紱撰。官韻考異一卷。吳省欽撰。韻辨附文五卷。沈兆霖撰。詩韻辨字略五卷。黃倬撰。韻詁五卷,補遺一卷。方濬頤撰。欽定音韻闡微十八卷,韻譜一卷。康熙五十四年,李光地等奉敕撰。欽定同文韻統六卷。乾隆十五年,莊親王允祿等奉敕撰。欽定音韻述微三十卷。乾隆三十八年敕撰。類音八卷,潘耒撰。等切元聲十卷。熊士伯撰。四聲切韻表四卷,音學辨微一卷。江永撰。沈氏四聲考二卷。紀昀撰。四聲韻和表五卷。洪榜撰。四聲易知錄四卷。姚文田撰。等韻叢說一卷。江有誥撰。字母辨一卷。黃廷鑒撰。四聲切韻表補正三卷。汪曰楨撰。劉氏碎金一卷,中州切音論贅論一卷。劉禧延撰。四聲定切四卷。劉熙載撰。切韻考六卷,外篇三卷。陳澧撰。翻切簡可篇二卷。張燮承撰。

宋司馬光切韻指掌圖二卷,附撿例一卷。以上乾隆三十八年王際華等奉敕輯。魏李登聲類一卷,晉呂靜韻集一卷,北齊陽休之韻略一卷,唐僧神珙四聲五音九弄反鈕圖一卷。以上均馬國翰輯。宋李概音譜一卷,聲譜一卷,唐孫愐唐韻二卷,唐顏真卿韻海鏡源一卷,唐李舟切韻一卷。以上均黃奭輯。

以上小學類韻書之屬

西域同文志二十四卷。乾隆二十八年,傅恆等奉敕撰。霮增訂清文鑒三十二卷、補編四卷、總綱八卷、補總綱二卷。乾隆三十六年,傅恆等奉敕撰。清漢對音字式一卷。乾隆三十七年敕撰。滿洲蒙古漢字三合切音清文鑒三十三卷。乾隆四十四年,阿桂等奉敕撰。清文匯書十二卷。李延基撰。清文補匯八卷。宗室宜興撰。清文備考六卷。戴穀撰。清文啟蒙四卷。舞格撰。三合便覽十二卷。不著撰人名氏。清文總匯二卷。不著撰人名氏。

以上小學類清文之屬

[一]按:藝文志序,關外一次本與此相同,而關內本與此詳略互異,附錄於後,以資參考。

清代肇基東陲,造創伊始,文教未宏。太宗首命大學士希福等譯遼、金、元三史,逮世祖譯史告成,二年又有議修明史之詔。惟其時區宇未寧,日不暇給,是以石渠之建,猶未遑焉。聖祖繼統,詔舉博學鴻儒,繼修明史,復纂諸經解、圖書集成等書,以網羅遺逸,拔擢英才,宏獎斯文,潤色鴻業,馴致太平之治,而海內彬彬靡然向風矣。世宗嗣位,再舉鴻詞,未行而崩。

高宗初元,繼試鴻博,採訪遺書。乾隆三十七年諭曰:「朕稽古右文,聿資治理,幾餘典學,日有孜孜。因思策府縹緗,載籍極博,其鉅者羽翼經訓,垂範方來,固足稱千秋法鑒,即在識小之徒,專門撰述,細及名物象數,兼綜條貫,各自成家,亦莫不有所發明,可為游藝養心之助。然或逸在名山,未登柱史,正宜及時採集,匯送京師,以彰千古同文之盛。其令直省督撫、學政加意購訪,量為給價,家藏鈔本,錄副呈送。庶幾藏在石渠,用儲乙覽,四庫、七略益昭美備,稱朕意焉。」於是安徽學政硃筠條奏明永樂大典內多古書,請開局纂輯,繕寫各自為書。時永樂大典儲翰林院,已有殘缺,原書為卷二萬二千九百三十七,缺二千四百四卷,存二萬四百七十三卷,為冊九千八百八十一。高宗下筠議,大學士於敏中力贊其說。明年,詔設四庫全書館,以皇子永瑢、大學士於敏中等為總裁,侍郎紀昀、大理寺卿陸錫熊等為總纂,其纂修等官則有戴震、邵晉涵、莊存與、任大椿、王念孫、姚鼐、翁方綱、硃筠等,與事者三百餘人,皆博選一時之俊。歷二十年,始繕寫告成。先後編輯之書三百八十五種,以聚珍版印行百餘種。三十九年,催繳直省藏書,四方競進秘籍甚眾,江、浙督撫採進者達四五千種,浙江鮑士恭、範懋柱、汪啟淑,江蘇馬裕家藏之籍,呈進者各六七百種,周厚育、蔣曾瑩、吳玉墀、孫仰曾、汪汝瑮等亦各進書百種以上。至是天府之藏,卓越前代,特命紀昀等撰四庫全書總目,著錄三千四百五十八種,存目著錄六千七百八十八種,都一萬二百四十六種。復以總目提要卷帙浩繁,學子繙閱匪易,又命紀昀就總目之書別纂四庫簡明目錄,其存目之書不預焉。

先是高宗命擷四庫精華,都四百六十四部,繕為薈要,藏諸摛藻堂,以備御覽。

當是時,四庫寫書至十六萬八千冊,詔鈔四分,分庋京師文淵、京西圓明園文源、奉天文溯、熱河文津四閣,復簡選精要,命武英殿刊版頒行。四十七年,詔再寫三分,分貯揚州大觀堂之文匯閣、鎮江金山寺之文宗閣、杭州聖因寺玉蘭堂之文瀾閣,令好古之士欲讀中秘書者,任其入覽。用是海內從風,人文炳蔚,學術昌盛,方駕漢、唐。後文源載籍燼於英法聯軍,文匯、文宗毀於洪楊之亂,文瀾亦有散佚。獨文淵、文溯、文津三閣之書,巍然具存,書皆鈔本,其宋、元精刊,多儲大內天祿琳瑯等處,載諸宮史;而外省督撫,禮聘儒雅,廣修方志,郡邑典章,粲然大備。阮元補四庫未收書四百五十四種,復刊學海堂經解一千四百十二卷,王先謙續刊一千三百十五卷,甄採精博,一代經學人文萃焉。曾國籓督兩江,倡設金陵、蘇州、揚州、浙江、武昌官書局,張之洞督粵,設廣雅書局,皆慎選通儒,審校群籍,廣為剞劂,以惠士林,而私家校勘,精鏤亦夥,叢書之富,曩代莫京。

清之末葉,歐風東漸,科學日昌。同治初,設江南制造局,始譯西籍。光緒末,復設譯書局,流風所被,譯書競出,憂世俊英,群研時務。是時敦煌寫經,殷墟龜甲,異書秘寶,胥見塏壤,實足獻納藝林,宏裨學術,其間碩學名儒,各標宗派,故鴻篇鉅制,不可殫紀。

藝文舊例,胥列古籍,清代總目,既已博載,茲志著錄,取則明史,斷自清代。四部分類,多從總目,審例訂譌,間有異撰。清儒箸述,總目所載,捊採靡遺,存目稍蕪,斠錄從慎。乾隆以前,漏者補之,嘉慶以後,缺者續之,茍有纖疑,則從蓋闕。前朝群書,例既弗錄,清代輯佚,異乎斯旨,裒纂功深,無殊撰述,故附載焉。


\end{pinyinscope}