\article{列傳二百九十一}

\begin{pinyinscope}
藝術三

王時敏族侄鑒子撰孫原祁原祁曾孫宸

陳洪綬崔子忠禹之鼎餘集改琦費丹旭

釋道濟髡殘硃耷弘仁王翬吳歷楊晉黃鼎方士庶

惲格馬元馭王武沈銓龔賢趙左項聖謨查士標

高其佩李世倬硃倫瀚張鵬翀

唐岱焦秉貞郎世寧張宗蒼餘省金廷標丁觀鵬繆炳泰

華嵒高鳳翰鄭燮金農羅聘奚岡錢杜方薰

王學浩黃均

王時敏,字遜之,號煙客,江南太倉人,明大學士錫爵孫。以廕官至太常寺少卿。時敏系出高門,文採早著。鼎革後,家居不出,獎掖後進,名德為時所重。明季畫學,董其昌有開繼之功,時敏少時親炙,得其真傳。錫爵晚而抱孫,彌鍾愛,居之別業,廣收名跡,悉窮秘奧。於黃公望墨法,尤有深契,暮年益臻神化。愛才若渴,四方工畫者踵接於門,得其指授,無不知名於時,為一代畫苑領袖。康熙十九年,卒,年八十有九。

鑒,字圓照,明尚書世貞曾孫。與時敏同族,為子侄行,而年相若。崇禎中,官廉州知府,甫強仕,謝職歸。就弇園故址,營構居之,蕭然世外。與時敏砥礪畫學,以董源、巨然為宗,沈雄古逸,雖青綠重色,書味盎然。後學尊之,與時敏匹。康熙十六年,卒,年八十。

孫原祁,字茂京,號麓臺。幼作山水,張齋壁,時敏見之,訝曰:「吾何時為此耶?」問知,乃大奇曰:「此子業且出我右!」康熙九年成進士,授任縣知縣。行取給事中,尋改中允,直南書房。累擢戶部侍郎,歷官有聲。時海內清晏,聖祖右文,幾餘怡情翰墨,常召入便殿,從容奏對。或於御前染翰,上憑幾觀之,不覺移晷。命鑒定內府名跡,充書畫譜總裁、萬壽盛典總裁,恩禮特異。五十四年,卒於官,年七十四。

原祁畫為時敏親授,於黃公望淺絳法,獨有心得,晚復好用吳鎮墨法。時敏嘗曰:「元季四家,首推子久,得其神者,惟董宗伯;得其形者,予不敢讓;若形神俱得,吾孫其庶幾乎?」王翬名傾一時,原祁高曠之致突過之。每畫必以宣德紙,重毫筆,頂煙墨,曰:「三者一不備,不足以發古雋渾逸之趣。」或問王翬,曰「太熟」;復問查士標,曰「太生」。蓋以不生不熟自居。中年後,供奉內廷,乞畫者多出代筆,而自署名。每歲晏,與門下賓客畫,人一幅,為制裘之需,好事者緘金以待。弟子最著者黃鼎、唐岱,並別有傳。

原祁曾孫宸,字子凝,號蓬心。乾隆二十五年舉人,官湖南永州知府。原祁諸孫,多以畫世其家,惟宸最工。枯毫重墨,氣味荒古。愛永州山水,自號瀟湘子,有終焉之志。罷官後,貧不能歸,畢沅為總督,遂往依之武昌。以詩畫易酒,湖湘間尤重其畫。著繪林伐材十卷,王昶稱為「畫史總龜」云。

陳洪綬,字章侯,浙江諸暨人。幼適婦翁家,登案畫關壯繆像於素壁,長八九尺,婦翁見之驚異,扃室奉之。洪綬畫人物,衣紋清勁,力量氣局,在仇、唐之上。嘗至杭州,摹府學石刻李公麟七十二賢像,又摹周昉美人圖,數四不已,人謂其勝原本,曰:「此所以不及也。吾畫易見好,則能事猶未盡。」嘗為諸生,崇禎間,游京師,召為舍人,摹歷代帝王像,縱觀御府圖畫,藝益進。尋辭歸。鼎革後,混跡浮屠間,初號老蓮,至是自號悔遲。縱酒不羈,語及亂離,輒慟哭。後數年卒。子字,號小蓮。畫亦有名。

洪綬在京師與崔子忠齊名,號「南陳北崔」云。

子忠,一名丹,字道母,別號青蚓,山東萊陽人,寄籍順天。為諸生,負異才。作畫意趣在晉、唐之間,不屑襲宋、元窠臼。人物士女尤勝,董其昌稱之,謂非近代所有。以金帛請者不應,家居常絕食。史可法贈以馬,售得金,呼友痛飲,一日而金盡。為詩古文,奧博奇崛。遭亂,走居土室中,遂窮餓以死。其後畫人物士女最著者,曰禹之鼎、餘集、改琦、費丹旭。

之鼎,字尚吉,號慎齋,江蘇江都人。幼師藍瑛,後出入宋、元諸家,尤擅人物,繪王會圖傳世。其寫真多白描,不襲李公麟之舊,而用吳道子蘭葉法,兩顴微用脂赭染之,彌復古雅。康熙中,授鴻臚寺序班。愛洞庭山水,欲居之,遂歸。朝貴名流,多屬繪圖像,世每傳之。

集,字秋室,浙江錢塘人。乾隆三十一年進士。工畫士女,時稱曰「余美人」,廷試,當得大魁,因此抑之。尋充四庫全書纂修,薦授翰林院編修,累擢侍讀。所作風神靜朗,無畫史氣,為世所重,比諸仇、唐遺跡。

琦,字伯蘊,號七薌,先世為西域人,壽春鎮總兵光宗孫,因家江南,居華亭。琦通敏多能,工詩詞。嘉、道後畫人物,琦號最工。出入李公麟、趙孟頫、唐寅及近代陳洪綬諸家。花草蘭竹小品,迥出塵表,有惲格遺意。

丹旭,字子苕,號曉樓,浙江烏程人。工寫真,如鏡取影,無不曲肖。所作士女,娟秀有神,景物布置皆瀟灑,近世無出其右者。

釋道濟,字石濤,明楚籓裔,自號清湘老人。題畫自署或曰大滌子,或曰苦瓜和尚,或曰瞎尊者,無定稱。國變後為僧,畫筆縱恣,脫盡窠臼,而實與古人相合。晚游江、淮,人爭重之。著論畫一卷,詞議玄妙。與髡殘齊名,號「二石」。

髡殘,字石溪,湖南武陵人。幼孤,自翦發投龍三三家菴。遍游名山,後至江寧,住牛首,為堂頭和尚。畫山水奧境奇闢,緬邈幽深,引人入勝。道濟排奡縱橫,以奔放勝;髡殘沉著痛快,以謹嚴勝;皆獨絕。

硃耷,字雪個,江西人,亦明宗室。崇禎甲申後,號八大山人,嘗為僧。其書畫題款「八大」二字每聯綴,「山人」二字亦然,類「哭」類「笑」,意蓋有在。畫簡略蒼勁,生動盡致,山水精密者尤妙絕,不概見。慷慨嘯歌,世以狂目之。

弘仁,字漸江,安徽休寧人,姓江,字亦奇。明諸生,亦甲申後為僧。工詩古文,畫師倪瓚,新安畫家皆宗之。然弘仁所作層崖陡壑,偉俊沈厚,非若世之以疏竹枯株摹擬高士者比。歿後,墓上種梅數百本,因稱梅花古衲雲。

自道濟以下,皆明之遺民,隱於僧,而以畫著。其後畫僧,上睿、明中、達受最有名。

上睿,字目存,吳人。嘗從王翬游,得其指授。

明中,字大恆,浙江桐鄉人。晚主杭州南屏凈慈。高宗南巡,賜紫衣。山水得元人法。

達受,字六舟,浙江海寧人。故名家子。耽翰墨,書得徐渭、陳道復縱逸之致。善別古器。精摹搨,或點綴折枝於其間,多古趣。阮元呼曰「金石僧」。

王翬,字石谷,號耕煙,江南常熟人。太倉王鑒游虞山,見其畫,大驚異,索見,時年甫冠。載歸,謁王時敏,館之西田。盡出唐以後名跡,俾坐臥其中,時敏復挈之游江南北,盡得觀收藏家秘本。如是垂二十年,學遂成。康熙中詔徵,以布衣供奉內廷。繪南巡圖,集海內能手,逡巡莫敢下筆,翬口講指授,咫尺千里,令眾分繪而總其成。圖成,聖祖稱善,欲授官,固辭,厚賜歸。公卿祖餞,賦詩贈行。翬天性孝友,篤於風義,時敏、鑒既歿,歲時猶省其墓。康熙五十六年,卒,年八十六。翬論畫曰:「以元人筆墨,運宋人丘壑,而澤以唐人

氣韻,乃為大成。」稱之者曰:「古今筆墨之齟不相入者,翬羅而置之筆端,融冶以出。畫有南、北宗,至翬而合。」

吳歷,又名子歷,字漁山,號墨井道人,亦常熟人。學畫於王時敏,心思獨運,氣韻厚重沈鬱,迥不猶人。晚年棄家從天主教,曾再游歐羅巴。作畫每用西洋法,雲氣綿渺凌虛,迥異平時。康熙五十七年,卒,年八十七。當時或言其浮海不歸,後於上海南郭得其墓碣,題曰「天學修士」云。翬初與友善,後絕交。王原祁論畫。右歷而左翬,曰:「邇時畫手,惟吳漁山而已。」世以時敏、鑒、翬、原祁、歷及惲格,並稱為六大家。同縣又有楊晉、黃鼎。

晉,字子鶴。翬弟子,山水清秀,尤以畫牛名。翬作圖,凡有人物與轎駝馬牛羊,皆命晉寫之。從翬繪南巡圖,因摹內府名跡進御。

鼎,字尊古。學於王原祁,而私淑翬,得其意。臨摹古人輒逼真,尤擅元王蒙法。遍游名山,號獨往客。論者謂翬看盡古今名畫,下筆具有淵源;鼎看盡九州山水,下筆具有生氣。常客宋犖家,梁、宋間其跡獨多。

方士庶,字循遠,號小師道人,安徽歙縣人,家於揚州。鼎弟子,早有出藍之目。年甫逾四十,卒,論者惜之。翬畫派為一代所宗,世比之王士禎之詩,當時門弟子甚盛,傳衍其法者益眾,附著其尤者。

惲格,字壽平,後以字行,改字正叔,號南田,江南武進人。父日初,見隱逸傳。格年十三,從父至閩。時王祈起兵建寧,日初依之。總督陳錦兵克建寧,格被掠,錦妻撫以為子。從游杭州靈隱寺,日初偵遇之,紿使出家為僧,乃得歸。格以父忠於明,不應舉,擅詩名,鬻畫養父。畫出天性,山水學元王蒙。既與王翬交,曰:「君獨步矣!吾不為第二手。」遂兼用徐熙、黃筌法作花鳥,天機物趣,畢集豪端,比之天仙化人。畫成,輒自題詠書之,世號「南田三絕」。雖自專意寫生,間作山水,皆超逸,得元人冷淡幽雋之致。王時敏聞其名,招之,不時至。至,則時敏已病,榻前一握手而已。家酷貧,風雨常閉門餓,以金幣乞畫者,非其人不與。康熙二十九年,卒,年五十四。子不能具喪,王翬葬之。

從父向,字道生。自明季以畫著,山水得董源法,格少即師之。及格負重名,群從子弟多工畫。其著者源濬,字哲長,官天津縣丞。能傳徐熙法,筆有生氣。族曾孫鍾廕之女曰冰,尤有名,詳列女傳。

其弟子尤著者:馬元馭,字扶曦,常熟人。家貧,好讀書。幼即工畫,王翬稱之。後學於格,得其逸筆,頗稱入室。孫女荃,傳其學,名與惲冰相匹。元馭嘗以畫法授同縣蔣廷錫,後廷錫宮禁近,以書招之,謝不往。

格人品絕高,寫生為一代之冠,私淑者眾,然不能得其機趣神韻。惟乾隆中華嵒號為繼跡。後改琦亦差得其意云。

王武,字勤中,吳縣人。畫花草,流麗多風,王時敏亦稱為妙品,學者宗之。及格出,遂掩其上。

沈銓,字南蘋,浙江德清人。工寫花鳥,專精設色,妍麗絕人。雍正中,日本國王聘往授畫,三年乃歸,故其國尤重銓畫,於格為別派。

龔賢,字半千,江南昆山人。寓江寧,結廬清涼山下,葺半畝園,隱居自得。性孤僻,詩文不茍作。畫得董源法,埽除蹊徑,獨出幽異,自謂前無古人,後無來者。

同時與樊圻、高岑、鄒喆、吳弘、葉欣、胡造、謝蓀號「金陵八家」。圻,字會公;造,字石公,與蓀,皆江寧人。岑,字蔚生,杭州人。喆。字方魯,吳人。弘,字遠度,金谿人。欣,字榮木,華亭人。諸家皆擅雅筆,負時譽,要以賢為稱首。

清初畫學蔚盛,大江以南,作者尤多,各成派別,以婁東王時敏為大宗。若金陵、雲間、嘉禾、新安,皆聞人迭起。

趙左,字文度,華亭人。畫出於宋旭,為雲間派之首,吳、松間多宗之。

項聖謨,字孔彰,嘉興人,元汴之孫。初學文徵明,後益進於古,董其昌稱其與宋人血戰,又得元人氣韻。子奎,字東井,世其學。

同縣李琪枝,字雲連,日華之孫。山水淡逸,傳世者梅竹為多。項、李皆名族,濡染有緒,群從多以畫名。

其後雍、乾中錢綸光妻陳書,花鳥人物並工,詳列女傳。錢氏子孫及閨秀傳其法者眾,更盛於項、李二家。

張庚,字浦山,亦嘉興人。學於書,深通畫理,著畫徵錄及續錄,自明末至乾、嘉中,所載四百餘人。

查士標,字二瞻,號梅壑,安徽歙縣人。明諸生,後棄舉子業,專精書畫。家饒於貲,多藏鼎彞古器,及宋、元名跡。初學倪瓚,後參以吳鎮、董其昌法,稱逸品。晚益以幽淡為宗,疏嬾罕接賓客,蓋託以逃世。與同縣孫逸,休寧汪之瑞、釋弘仁,號「新安四家」。久寓揚州,康熙三十七年,卒,年八十四。

逸,字無逸。流寓蕪湖,曾繪歙山二十四圖。

之瑞,字無瑞。豪邁自喜,渴筆焦墨,酒酣揮灑如風雨。

時當塗蕭雲從,字尺木。與逸齊名,山水不專宗法,兼長人物。於採石太白樓下四壁畫五嶽圖,又畫太平山水及離騷圖,好事者並鐫刻以傳。

高其佩,字韋之,號且園,奉天遼陽人,隸籍漢軍。父殉耿籓之難,其佩以廕官至戶部侍郎。畫有奇致,人物山水,並蒼渾沉厚,衣紋如草篆,一袖數折。尤善指畫,嘗畫黃初平叱石成羊,或已成羊而起立,或將成而未起,或半成而未離為石,風趣橫生。畫龍、虎,皆極其態。世既重其指墨,晚年以便於揮灑,遂不復用筆。其筆畫之佳,幾無人知之。雍正十二年,卒。甥李世倬、硃倫瀚皆學於其佩。

世倬,字漢章,總督如龍子。官至右通政。少至江南,從王翬游,得其傳。後官山西,觀吳道子水陸道場圖,悟人物之法。花鳥寫生,得其佩指墨之趣,易以筆運,各名一家。

倫瀚,字涵齋,明裔也,隸籍漢軍。官至都統,直內廷。指畫師其佩,丘壑奇而正,色淡味厚。喜作巨障,元氣淋漓。指上生有肉錐,故作人物,須眉尤有神,出於天授。其後傳其佩法者,有傅雯、瑛寶。

雯,字凱亭。奉天布衣,為諸王邸客,京師多其遺跡。

瑛寶,字夢禪,滿洲人,大學士永貴子。以疾辭廕不仕,詩畫自娛。指墨以簡貴勝,深自矜許。

張鵬翀,字天飛,自號南華山人,江蘇嘉定人。雍正五年進士,入翰林,官至詹事府詹事。天才超邁,詩畫皆援筆立就,瀟酒自適,類其為人。高宗愛其才,不次拔擢。進奉詩文,多寓規於頌。畫無師承,自然入古。雖應制之作,蕭散若不經意,愈見神韻。繪春林澹靄圖,題詩進上,上賜和,鵬翀即於宮門疊韻陳謝。嘗從駕西苑液池,一渡之頃,得詩八首。屢敕御舟作畫,賜御筆枇杷折枝及松竹雙清圖,又賜雙清閣書額,迭拜筆硯、文綺之賜無算。乾隆十年,乞假歸,卒於途次。上眷之,久不忘,對群臣輒曰:「張鵬翀可惜!」

自康熙至乾隆朝,當國家全盛,文學侍從諸臣,每以藝事上邀宸眷。大學士蔣廷錫及子溥,董邦達及子誥,尚書錢維城,侍郎鄒一桂,與鵬翀為尤著。

廷錫以逸筆寫生,奇正、工率、濃淡,一幅間恆間出,無不超脫。源出於惲格,而不為所囿。邦達山水源於董源、巨然、黃公望,墨法得力於董其昌,自王原祁後推為大家。久直內廷,進御之作,大幅尋丈,小冊寸許,不下數百。溥、誥各承其家法。維城山水蒼秀,花卉傅色尤有神採。一桂以百花卷被宸賞,世謂惲格後罕匹者。諸人所繪並入石渠寶笈,御題褒美,傳為盛事。

嘉慶中,尚書黃鉞由主事改官翰林,入直,畫為仁宗所賞。道、咸以後,侍郎戴熙、大學士張之萬,並官禁近,以畫名。然國家浸以多故,視承平故事稍異焉。

唐岱,字毓東,滿洲人。康熙中,以廕官參領。從王原祁學畫,丘壑似原祁。供奉內廷,聖祖品題當時以為第一手,稱「畫狀元」。歷事世宗、高宗。高宗在潛邸,即喜其畫,數有題詠,後益被寵遇。唐岱專工山水,以宋人為宗。少時名動公卿。直內廷久,筆法益進,人間傳播者轉稀。著繪事發微行世。

清制,畫史供御者無官秩,設如意館於啟祥宮南,凡繪工、文史及雕琢玉器、裝潢帖軸皆在焉。初類工匠,後漸用士流,由大臣引薦,或獻畫稱旨召入,與詞臣供奉體制不同。間賜出身官秩,皆出特賞。高宗萬幾之暇,嘗幸館中,每親指授,時以為榮。其畫之精美者,一體編入石渠寶笈、秘殿珠林二書。嘉慶中,編修胡敬撰國朝院畫錄,凡載八十餘人,其尤卓著可傳者十餘人。

焦秉貞,山東濟寧人。康熙中,官欽天監五官正。工人物樓觀,通測算,參用西洋畫法,剖析分刌,量度陰陽向背,分別明暗,遠視之,人畜、花木、屋宇皆植立而形圓。聖祖嘉之,命繪耕織圖四十六幅,鐫版印賜臣工。自秉貞創法,畫院多相沿襲。

其弟子冷枚,膠州人,為最肖。與繪萬壽盛典圖。

陳枚,江蘇婁縣人。官內務府郎中。初法宋人,折衷唐寅,後亦參西洋法。寸紙尺縑,圖群山萬壑,人物胥備。

郎世寧,西洋人。康熙中入直,高宗尤賞異。凡名馬、珍禽、琪花、異草,輒命圖之,無不奕奕如生。設色奇麗,非秉貞等所及。

艾啟蒙,亦西洋人。其藝亞於郎世寧。

張宗蒼,字默存,江蘇吳縣人。學畫於黃鼎。初官河工主簿。乾隆十六年南巡,獻冊,受特知,召入直。數年,授戶部主事,以老乞歸。宗蒼山水,氣體深厚,多以皴擦取韻,一洗畫院甜熟之習,被恩遇特厚。所畫著錄石渠者,百十有六,多荷禦題。

弟子徐揚、方琮最得其法,亦邀宸賞,賜揚舉人,授內閣中書。

餘省,字曾三,江蘇常熟人。善寫生,能得花外之趣。同時楊大章,亦賦色修潔,可與鄒一桂頡頏,花鳥以二人為最工。

金廷標,字士揆,浙江桐鄉人。南巡進白描羅漢,稱旨,召入祇候。廷標畫不尚工緻,以機趣傳神。高宗題所作琵琶行圖曰:「唐寅舊圖,有琵琶伎在別船,廷標祇繪白居易一人側耳而聽,別有會心。古人畫意為先,非畫院中人所及。」會愛烏罕進四駿,郎世寧繪之,復命廷標別作,仿李公麟法,增寫執靮人,古趣出彼上。及廷標卒,上命舊黏殿壁者悉付裝池,收入石渠寶笈。

丁觀鵬,工人物,效明丁云鵬,以宋人為法,不尚奇詭。畫仙佛神像最擅長,著錄獨多。

時有嚴弘滋者,南巡兩次獻畫,所作三官神像,秀發飛揚,稱為絕作,屢命畫院諸人摹之。

姚文瀚,亦以人物仙佛名,亞於觀鵬。

繆炳泰,字象賓,江蘇江陰人。初以國子監生召繪御容。南巡,應召試,賜舉人,授中書,官至兵部郎中。乾隆五十年以後御容,皆出所繪。又命繪紫光閣功臣像,人人逼肖,寫真之最工者。

畫院盛於康、乾兩朝,以唐岱、郎世寧、張宗蒼、金廷標、丁觀鵬為最,宗蒼所作,尤有士氣,道光以後無聞焉。至光緒中,孝欽皇后喜藝事,稍復如意館舊規,畫史皆凡材,無可紀者。

華嵒,字秋嶽,號新羅山人,福建臨汀人。慕杭州西湖之勝,家焉。畫山水、人物、花鳥、草蟲無不工,脫去時蹊,力追古法。有時過求超脫,然其率略處,愈不可及。工詩,有離垢集,古質清峭。書法脫俗,世稱「三絕」,可繼惲格。僑居揚州最久,晚歸杭州,卒年近八十。

乾、嘉之間,浙西畫學稱盛,而揚州游士所聚,一時名流競逐。其尤著者,為高鳳翰、鄭燮、金農、羅聘、奚岡、黃易、錢杜、方薰等。

鳳翰,字西園,山東膠州人。雍正初,以薦得官,署安徽績溪知縣,被劾罷。久寓江、淮間,病偏痺,遂以左手作書畫,縱逸有奇氣。嘗登焦山觀瘞鶴銘,尋宋陸游題名,親埽積蘚,燃燭捫圖,以敗筆清墨為圖,傳為傑作。性豪邁不羈,藏硯千,手自鐫銘,著硯史。又藏司馬相如玉印,秘為至寶。盧見曾為兩淮運使,欲觀之,長跪謝不可,其癖類此。

燮,字板橋,江蘇興化人。乾隆元年進士,官山東濰縣知縣,有惠政。辭官鬻畫,作蘭竹,以草書中豎長撇法為蘭葉,書雜分隸法,自號「六分半書」。詩詞皆別調,而有摯語。慷慨嘯傲,慕明徐渭之為人。

燮同縣李鱓,字復堂。舉人。官山東滕縣知縣。花鳥學林良,多得天趣。

陳撰,字楞山,浙江鄞縣人,亦居揚州。舉鴻博,不就試。與鱓齊名,寫梅尤雋逸。

農,字壽門,號冬心,浙江仁和人。布衣,薦鴻博,好學癖古,儲金石千卷。中歲,游跡半海內,寄居揚州,遂不歸。分隸小變漢法,又師禪國山及天發讖兩碑。截毫端,作擘窠大字。年五十,始從事於畫。初寫竹,師石室老人,號稽留山民。繼畫梅,師白玉蟾,號昔耶居士。又畫馬,自謂得曹、韓法。復畫佛,號心出家盦粥飯僧。其點綴花木,奇柯異葉,皆意為之。問之,則曰:「貝多龍窠之類也。」性逋峭,世以迂怪目之。詩亦鑱削苦硬。無子,晚手錄以付其女。歿後,羅聘搜輯雜文編為集。

聘,字兩峰,江都人。淹雅工詩,從農游,稱高足弟子,畫無不工。躭禪悅,夢入招提曰花之寺,仿佛前身,自號花之寺僧。多摹佛像,又畫鬼趣圖,不一本。游京師,跌宕詩酒,老而益貧。曾燠為兩淮運使,資之歸,未幾卒。妻方婉儀,亦工詩畫,好禪,號白蓮居士。

岡,字鐵生,號蒙泉,舊為歙縣人,居錢塘,遂隸籍。負奇,不得志,寄於詩畫。山水取法婁東,自成逸韻;竹石花木,超雋得元人意;四十後名益噪。曾游日本,海外估舶,懸金購其畫。徵孝廉方正,辭不就。

岡與同縣黃易齊名。易父樹穀,亦工書畫。易詳文苑傳,篤嗜金石,每以訪碑紀游作圖,為世所重。畫境簡淡,山左多宗之。

杜字,叔美,號松壺,仁和人。屈於下僚,曾官云南經歷,足跡逾萬里。深揅畫學,摹趙伯駒、孟頫、王蒙皆神似。間為金碧雲山,妍雅絕俗。畫梅疏冷出趙孟堅。兼擅詩名。著松壺畫贅、畫憶,多名論。

從兄東,字袖海,畫近惲格,名亞於杜。

薰,字蘭坻,浙江石門人。父,故善畫,薰幼從父游吳、越間,多見名跡,接耆宿,遂兼眾長。論畫曰:「寫生以意勝形似。」又曰:「不拘難易,須雅馴。」著山靜居論畫,以布衣終。

王學浩,字椒畦,江蘇昆山人。乾隆五十一年舉人。幼學畫於同縣李豫德,豫德為王原祁外孫,得南宗之傳。學浩溯源倪、黃,筆力蒼勁。論畫曰:「六法,一寫字盡之。寫者,意在筆先,直追所見,雖亂頭粗服,而意趣自足。或極工麗,而氣味古雅,所謂士大夫畫也。否則與俗工何異?」又曰:「畫以簡為上,雖煙客、麓臺,猶未免繁碎,如大癡,真未易到。大癡法固在荒率蒼古中求之,尤須得其不甚著力處。」時論學浩用墨,能入絹素之骨,比人深一色。晚好用破筆,脫盡窠臼,畫格一變。著南山論畫。卒,年七十九。學浩享大年,道光之季,畫苑推為尊宿。館吳中寒碧山莊劉氏,壇坫甚盛。其時吳、越作者雖眾,足繼前哲名一家者,蓋寥寥焉。

黃均,字穀原,元和人。守婁東之法,盡其能事。游京師,法式善、秦瀛為之延譽,得官,補湖北潛江主簿,未之任。於武昌臙脂山麓築小園,居之二十年,以吏為隱。畫晚而益工,於吳中稱後勁。

清畫家聞人多在乾隆以前,自道光後,卓然名家者,惟湯貽汾、戴熙二人,並自有傳。昭文蔣寶齡著墨林今話,繼張庚畫徵錄之後,子茝生為續編,至咸豐初,視庚錄數幾倍之。其後光緒中,無錫秦祖詠著桐陰論畫,論次一代作者,分三編,評騭較嚴,稱略備焉。今特著其尤工者,寶齡、祖詠畫亦並有法。


\end{pinyinscope}