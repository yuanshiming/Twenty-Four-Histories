\article{列傳二百八十二}

\begin{pinyinscope}
忠義九

宗室奕功札隆阿等覺羅清廉等松林文炘等

崇壽韓紹徽韓培森馬鍾祺董瀚譚昌祺莊禮本馮福疇

宮玉森景善等宋春華馬福祿楊福同

吳德潚子仲韜成肇麟

宗室奕功,歷官奉宸苑卿,至御前侍衛。光緒二十六年,拳匪肇禍,各國聯軍破京都,德宗奉孝欽顯皇后西狩,奕功以世受國恩,未能隨扈,引火自焚。妻祥佳氏、子載捷等,闔家投井殉節。

先後被難者,宗室有奉恩將軍札隆阿,子樸誠等;奉恩將軍緝御,子培善,孫存德、存厚等;文舉人恩煦,子繼勛、懋勛,從子啟勛、世勛等;掌江西道監察御史德籓,戶部員外郎恩斝,戶部主事謹善,宗人府經歷訥欽,頭等侍衛德潤,帶隊官鈺璋,及奕鑫、載袍、恕誠、聯德、恩溥、松達、善章、國文、松根、景璋、承惠、和桂、鳳喜、吉辰、海明,覺羅有清廉、年瑞、德潤、榮綿。

松林,巴雅爾氏,滿洲鑲黃旗人。由筆帖式累擢給事中。出知臨安府,升雲南糧儲道,晉山東按察使、布政使。內擢順天府府尹,病解任,起為內閣侍讀學士。聯軍犯京師,分守東直門,親指揮砲火中,抵禦甚力。俄中砲死,尸不可辨。

時陣亡者,前敵有世管佐領文炘,騎都尉玉廕、奎齡,筆帖式寶善,前鋒校榮春,護軍校玉連,驍騎校鍾安、德昌,前鋒舒元、明順,護軍秀亮、雙祿、瑞升、文福、成福、恩啟、常貴、成秀,把總文通,隊官全成,隊長全興,領催崇寬、貴斌、崇歡、慶祥、廣升、奎秀、永順、暇安、恩慶、廣立,馬甲成恆、瑞喜、慶山、倭克金布、世昌、玉興、恩隆、德勝、祥瑞、賡音布、董連元、保麟、裕安、長泉、保玲、王永立、保祥、李景瑞、田應時、張桂祥、李永福、清華、吉順、全立德、玉崇、喜保、林長玉、布克坦、全保、喜壽、海寬、延祿、玉山、成昌、長福、松齡、柯永、文斌、徐培田、文達、慶連、興瑞、李燁、保慶、清海、長春、恩常、保順、廣禧、廣海、崇福、鳳齡、成棨、雙全、玉岑、汪恆吉,養育兵明祿、玉海、玉存、景立、關喜、慶祿、色勒、連貴、雙壽、文奎、奎茂、齊德森、明保、永順、泳全、常來、吉祿、萬善、立得、長桂、松樑、德成、長安,閒散全興、松澤、德祿、連升、保盛阿、玉慶、德祿、廣成、連山、倭克金泰、立海、德緒、富森、廣海、崇福、榮羲、國安、祥桂、富順、延茂、德全、恩隆、楊德福,槍甲吉慶、連魁、李長升、景英、文海,槍兵崇昆,砲甲吉安、文、景瑞、張啟茂、劉龍、富琪、全奎、全保、德鳳、增銳、增輝、周奎斌,練兵桂普,隊兵光輝、林慶。

東直門有護軍參領賢普,世管佐領德續,公中佐領松鶴、錫昌、連秀,筆帖式榮山,驍騎校惠斌、倭什洪額、瀛緒、連桂、常浩、銘綸、鳳啟,護軍瑞斌、常福、春安、普惠、德謙、恆有、兆芳、隨善、同廣、崇敬、恆斌、桂祿、三多,隊官英璞、惠斌,領催德緒、常慶、成山、富順、常全、雙印、文森、松、雙奎、廣義,馬甲喬齡、錫瑞、田德貴、奎秀、廣喜、寶庚、廣祿、富通、明喜、廣林、文印、德林、永山、錫連、榮和、永霈、長安、李忠、春元、得林、興順、福貞、文芳、文普、玉芳、烏雲珠、達崇阿、德貴、明安、世達、黃培長、貴普、英玉、錫祿、文華、德本、春倫、成祐、崇慶、雙奎、雙海、立福、德保、潤秀、奎秀、順立、志亨、志隆、銘榮、崇喜、恩順、連敬,養育兵慶林、雙祿、隆福、宜緒、濟堃、長奎、德文、長清、得隆、景立、得保、明增、成林、福祥、寶瑞、恩佑,閒散榮喜、崇儀、順福、吉昆、長山、英振、阿炳、阿均、廣成、連山、世瑞、承英、錫保、雙興、德玉、治得、和森、廣立、李斌、世山、永利、長齡、鐵壽、定坤、龍泰、鳳林、鳳祥、景珍、崇錫、存德、延齡、錫光、寶忠、得虎、奎福,砲兵恆安、國安、承萬、吉恆、玉森、善溥、盛濂,隊兵凌貴、伊立布。

崇文門有護軍校富亮,驍騎校德瑞,筆帖式潤普,七品官薩斌圖,監生福壽,隊官彤勛,護軍慶升、定昆、世喜、富山,領催玉山、連英、國棟、文通,馬甲志福、鐵升、桂安、清海、巨泰、烏林、興海、聚泰、玉保、成喜、恩沾、全順、恩保、輔廷、達英、張仲蘭,養育兵永祿、文斌、隆興、德存、富寬、常壽、全祿、海玉、英鋆、松山、連升、存德,閒散文成、文亮、崇林、松山、常林、秀斌、松玉、忠福、巴克坦布、奎榮、崇海、緒順、德清,槍兵文海,隊兵恩保、德祿、隆興,幼丁劉長立。

朝陽門有雲騎尉富珠倫,恩騎尉連福,護軍校富亮,驍騎校續魁,鳥槍藍翎長松春,護軍海秀、常福、烏林泰、萬玉斌,前鋒吉昌,領催常興、保昌,馬甲永安、福山、雙喜、保勛、德福、鐵升、興海、長瑞、玉安、巴揚阿、烏林保,養育兵貴全、凌山、恩啟、保春、湧澂、德順、裕泰、玉厚、成玉、趙文忠、閏福、文瑞、榮德,閒散長緒、文立、多太、誠堃、恆立、常興、伊三布、文祿、常林、瑞申、恩錫、連升、松山、厚寬、張勛、松山、忠福。

東便門有游擊韓萬鍾、弟韓萬祿,千總慶餘,把總金鈺,戰兵王壽、李永福,馬兵梁坤、張德輿。

德勝門有副參領祥存、世管佐領承瑞,驍騎校崇桂、領催柏銘、容剛、文惠,馬甲錫連、桂啟,養育兵常海,隊兵榮喜。

安定門有筆帖式增俊,馬甲立貴、長慶、德閏、盧檢貴、恩壽、德平、長存、松祿、趙俊雙、恆山、莊立、玉明、劉殿臣、長壽、榮桂、合海、袁明林、楊有春、文愈、文茂、文毓、連順、施彬、文福、王玉鳳、線長海、全英、煜祥、鍾銘、傅合、連升、馬玉和,養育兵恩緒、奎元、二立、文浩,閒散清聯、德謙,武生長緒。

齊化門有護軍校連瑞。

西直門有養育兵烏什哈,閒散全桂。

阜成門有敖爾布鍾珊。

永定門有閒散長泰、玉泰、春祥。

正陽門有閒散清林、奎連、德勝。

宣武門有砲甲林廣明,藍翎長祥瑞,領催常連、景緒,馬甲榮福、崇善、德斌、全順、定保、榮慶、維明,砲手慶煥,養育兵松長,閒散英緒、續順、崇海。

大清門有前鋒玉興。

天安門有護軍參領玉山,副護軍參領雙福,護軍校花連布,侍衛潤志,前鋒岐俊,護軍永壽、文瑞、瑞升、承通、林安、玉慶、春喜、祥林、松桂、永壽、文祿、常升、常海、松惠、海全、桂升、雙壽。

午門有副護軍參領鳳齡,前鋒崇祥、桂豐,護軍玉壽、德凱。

東安門有公中佐領松壽,步軍校文通,領催延壽。

東華門有副護軍參領長年,副令官英寬,藍翎長富升,隊官玉昌,護軍恩秀、奎英、成光、忠明、貴慶、昆連、松群、玉山、阿杭阿、玉壽、恩秀、奎俊、成英、文廣、托克托虎、常山、廣慶、希拉布、他克布、連德,馬甲長山,養育兵存山,閒散德元,技勇兵全貴。

西安門有養育兵永順、德福。

西華門有馬甲春明。

地安門有虎神營營總昆明,副護軍參領恆謙,護軍營管理祥瑞,護軍隊官凌魁,隊長彥祿,護軍常瑞、薩圖布、永安、常山、雙壽、興斌,馬甲文海、福山,養育兵崇恩、全苓、順喜、閒散德祥。

紫禁城內有護軍參領海忠,親軍校文玉。

守陴者有世管佐領德潤,馬甲錫秀。

巷戰者有驍騎校多倫布,藍翎長德英額、雙貴,前鋒鳳玉、希拉奔、崇安、文英、榮昆,護軍德玉、崇貴、崇福、崇興,領催鶴鳴,馬甲雙福、長海、慶裕、桂保、長升、恩立、興岱、存桂、常泰,養育兵英厚、文志、德成、俊成,幼丁元成、全祥、世增、烏凌阿、廣林、廣俊、松廕、松祺、松立、延尉、成明、廣瑞,閒散全順、頤霈、多山、慶祿,外委王文志、聞廷標、王灝、高玉、常存,百總郭立奎,管隊張海、金松林,把總王洪銘,馬兵彭玉恩、全祥,戰兵李逢春、戴永福、彭玉堂、孟祿,守兵王政樞、劉永安、季茂軒,砲甲祥通,砲手白萬泰。

死事者:寧壽宮員外郎誠年、筆帖式福臻在內值宿,七月二十一日巳刻,聞兩宮西狩,即赴各殿封鎖,至斂禧門外值房投井死。太廟五品官富亮,值班上香,洋兵突進,拒之,槍死。織工張繼福,在綺華館被戕。左營參將王長廕守署不去,以獨力難持,投井死。護軍連升值班端門;護軍崇連,神機營呈遞公事步軍校賡音布、常福、勝喜,領催雙喜,馬甲存林、恩明,外委孫國瑞,技勇兵常有、隆祥、萬昭,均在值班;領催榮鈐,養育兵定成,隊兵布興泰,均看守軍庫;南城正指揮項同壽,在署辦公;戶部書吏高世祥,總理衙門供事沈鵬儀、徐伯興、洪瑞汶,均在署值班,與於難。

在先陣亡者:把總李鍾山,外委李鍾林,七月十七日,在張家灣禦敵,不克,死。

先後被難者:游擊王燮,五月二十五日在東便門彈壓拳匪,被戕,並毀其尸;採育營部司楊光第,於閏八月二十九日聞洋兵至,衣冠坐營中,被槍死;把總張進志擁護同死。

均經留京辦事大臣昆岡上聞,贈恤有差。

崇壽,溫徹亨氏,滿洲鑲黃旗人。光緒十六年進士,入翰林,累擢翰林院侍讀。變作時,不勝憂憤,仰藥死。詔以「見危授命」褒之,謚文貞。

韓紹徽,字筱珊,貴州貴陽人。光緒二十年進士,授主事,分刑部,勤於所職。拳亂初起,嘗走同官,涕泣誓身殉。七月二十一日,自經於陜西司司堂。

掌江西道御史韓培森,巡城積勞,城破,絕食死。內閣中書堃厚,手書「見危授命」四字,與妻同縊死。

馬鍾祺,字維春,隸漢軍鑲黃旗。少為諸生,以襲一等子,例不得與試,授三等侍衛,擢二等,有文武才。初服膺陸、王之學,繼參以程、硃、張、呂,不主一家。為人伉爽有奇氣,慕孫白穀之為人,好與朝野賢士游,與語或不合,輒哦詩亂之,以此得狂名。光緒二十年,日本爭朝鮮,廷議出師,鍾祺上書請自效,遂從戎奉天。盛京將軍依克唐阿器之,使統鎮邊馬隊。會和議定,遂歸。二十五年,李秉衡奉旨巡視長江,親訪於家,疏請從行。拳匪禍作,冒鋒火而北,秉衡殉難,鍾祺護其喪歸。歸三日,京師破,鍾祺自縊死。著五倫大義、馬氏日記若干卷。

候選縣丞董瀚,於城破日與弟候補巡檢徵曰:「我等職雖微末,既讀聖賢書,惟有以身殉國而已。」同時自縊。

涿州附生譚昌祺,聞城陷,懷藥哭諸聖廟,仰藥死。

舉人莊禮本,留京讀書。拳匪初起,即以為憂。洋兵入城,痛哭不食,後以一慟而絕。

州同銜馮福疇,在通州署辦刑名事。七月十六日,敵入署,守護案牘,不屈,被戕。

東城司吏目、練勇局委員宮玉森,洋兵攻局,其女請避,怒投其女於井,拔刀出戰。傷數處,自知不免,亦投井死。

時同被難者,為原品休致禮部侍郎景善,前奉天府尹福裕,蒙古副都統耆齡,前察哈爾副都統明秀,冠軍使文琭,工科給事中恩順,刑部郎中汪以莊,兵部員外郎薩德賀、趙寶書,吏部主事鍾傑,戶部主事陶見曾、李慕、鐵山,刑部主事毛煥樞、王者馨,工部主事白慶、恆昌,理籓院主事英順,光祿寺署丞多文,國子監助教柏山,候選道鄭錫敞,前紹興府知府繼恩,分省知縣王朝鐀等,見冊報者千餘人。

全家焚溺服毒自經以盡節者眾,騎都尉候選員外郎陳鑾,住東便門二閘,於七月十九日洋人攻城,勢急,與諸弟率眷屬僕婢三十二名,一時自盡,尤為慘烈云。

宋春華,字實菴,陜西三原人。光緒十二年武進士,授藍翎侍衛。出為天津鎮標右營守備,與士卒共甘苦,所部為天津綠營冠。聯軍內犯,總督裕祿檄春華守城南門。城東南制造軍械所不守,春華集其眾曰:「軍械所存亡,天津生死系之。不奪歸不可,膽勇者盍隨吾出城!」皆應曰:「諾!」率百餘人夜半潛出,及庫垣,春華先登,眾隨之。槍中春華左股,眾欲退,春華負創大呼曰:「今夕之事,有進無退!」眾爭奪敵,死傷甚眾,卒以守堅,退歸城。已而敵兵日集,守土官多棄城走,春華慨語其妻陳曰:「城孤兵單,終恐不守。汝當以吾子出求生,吾誓與城存亡矣!」語畢,登陴督戰不少息。城既陷,身被數傷,猶死守不退。或勸少避,春華曰:「城不守,死自吾分。汝曹各有父母妻子,歸可也,俱死無益!」眾感其義,無退者。敵畢登城,乃仰天嘆曰:「吾志不遂,負國恩矣!然自接戰以來,殺敵過當,今日之死,亦無所恨。」以首觸陴,腦出,死,年三十五。

馬福祿,字壽三,甘肅河州人。光緒六年武進士,用衛守備,歸河南鎮標,以終養告歸。二十年,循化撒拉回族以爭教叛,固原提督雷正綰檄福祿往崔家峽、樊家峽協防,戰輒勝。河、湟回匪繼起,復助官兵獲大捷。累功至記名總兵。

二十一年,河州諸回變,福祿本回教,回以福祿助官軍,欲加害。福祿在城,人亦以回教為疑,獨正綰信之。時河州鎮總兵湯彥和遠駐起乍堡,命福祿率騎兵迎入河州城鎮之,彥和猶豫不果行,叛回周七十乃糾眾據山巔下擊。福祿戰二日,以失地利無功。彥和復潛走,軍無統帥,賊益蹙之。福祿乃突圍出南番境,至蘭州乞師。沿路拔出難民數千,難民德之,狀總督楊昌濬,昌濬以福祿孚眾望,乃檄與蘭州道黃雲由北路援河州。時喀什噶爾提督董福祥奉旨赴甘肅協剿,由狄道進兵。福祿率師至蓮花渡,與賊隔岸相持,為福祥軍犄角,卒解河州圍。時韓文秀亦作亂,河湟提督李培榮、總兵牛師韓軍失利,陜西巡撫魏光燾與福祥會白塔寺,議進兵。福祿入謁,陳亂事顛末,及前後戰狀,福祥奇之,檄剿叛回冶主麻於米拉溝。剿未盡,馬營土豪馬採哥應之,福祥部將石堯臣等告敗,福祿復分道往援,首先陷陣,斬採哥,聚而殲之。冶主麻收餘燼由黑山趨米拉,復還兵破之,斬無算,用是有驍將名。

拳匪倡亂,福祥奉旨入都,檄福祿統馬步七營、旗防山海關,尋移永平府,福祥入衛京師,檄隨行。五月,各國聯軍躪楊村而西,偕漢中鎮總兵姚旺等赴黃村御之。抵廊坊,兩軍相接,乃令騎兵下設七覆,步兵張兩翼,敵近始發槍,倒者如僕墻。敵彈落如雨,騎兵以散處少傷,兩翼左右復包抄其後,短兵相接,敵不支,遽卻,為庚子之役第一惡戰。六月,福祥檄令攻使館,中彈歿於陣,猶子耀圖、兆圖亦死,同殉者百餘人。

楊福同,直隸清苑人。同治七年,投軍,累擢游擊,從討朝陽教匪。嗣以副將駐營大名,專力緝捕,以功記名總兵,分統練軍左翼馬隊,兼統天津馬步隊各營。近畿拳匪蜂起,淶水尤甚,總督裕祿檄福同率隊往。至史家莊,伏匪邀擊,力禦之,擒數人。次日,又敗匪於石亭鎮,擒首要梁修。福同不忍多誅,令限日解散,留馬隊三十人鎮之。無何,匪以千餘眾攻留隊,福同率步兵數十馳援。將及石亭,群匪自溝中突出,白刃交下,創甚,猶格殺數人,力盡死之。從弁孫裕清、盧興璠俱力戰死,賜恤如例。

吳德潚,字筱村,四川達縣人。性至孝。博極群書,以進士用知縣。庚子年,任浙江,西安、北京拳亂起,江山縣土匪以仇教為名,連陷江山、常山,縣人咸欲應之,德潚謂北事未定,洋人必不宜殲。有羅楠者,素健訟,德潚嘗嚴懲之,久含恨。結都司周之德,挾眾指德潚袒洋教,劫德潚縛道署轅門,盡鑷須發,以利刃攢刺,洞腹死,德潚罵不絕口。子仲韜馳哭尸下,又殺之,並入縣署殺全家四十餘口。事定,恤如例。

成肇麟,江蘇華亭人。父孺,諸生,列儒林傳。肇麟由舉人官直隸知縣,遷直隸州知州,署滄州靜海,補靈壽,所至有績。光緒二十七年,京師和議梗,聯軍西上,覃及邑境,責供牲畜糗糧甚厲,肇麟壹弗應。俄而布政使廷雍檄至,令迎犒,肇麟自念:「不迎犒,無以全民命;迎犒,則以中國臣子助攻君父;事處兩難,守土之義無可避,惟有一死耳!」乃繕遺牒遣人間道達府,媵之以詩曰:「屈體全民命,捐軀表素懷。」李鴻章狀死事以上,謂其能伸大義,降敕褒嘉,贈太僕寺卿,謚恭恪,予世職。明年,允直督請,建直隸省城專祠。


\end{pinyinscope}