\article{本紀二十四}

\begin{pinyinscope}
德宗本紀二

二十一年乙未春正月癸酉朔,停筵宴。乙亥,日兵寇威海。丁丑,我海軍與戰於南岸,敗績。己卯,吳大澂始出關視師。辛巳,威海陷,守將戴宗騫死之。改命聶士成統兵入關。丁亥,詔責李鴻章。庚寅,劉公島陷,水師熸,丁汝昌及總兵劉步蟾死之。諭張之洞、松椿防海、贛、清江水陸要沖,保清、淮通運。辛卯,授李鴻章為頭等全權大臣,使日本。壬辰,見各國使臣於文華殿。陶模言喀什噶爾、莎車、和闐等屬戶民,英印度部收買為奴,應由公家贖放,從之。丙申,葉志超、龔照興俱論斬。己亥,日本陷文登、寧海,偪煙臺。宋慶等及日人戰於太平山,敗績,走。

二月乙巳,宋慶、吳大澂敗日人於亮甲山,參將劉雲桂、守備趙雲奇戰死。賑錦州、寧遠災民。丁未,命聶士成總統津、沽海口防軍。乙酉,日兵薄遼陽,長順、唐仁廉擊卻之。庚戌,日兵陷牛莊,吳大澂退走,日人遂襲營口。癸丑,馬玉昆敗日人於田莊臺。甲寅,復戰,敗績。丙辰,日兵陷田莊臺。吳大澂奔錦州,宋慶退雙臺。丁巳,以吳大澂師徒撓敗,切責之。戊午,恭親王等奏撤海軍署。免上元、江寧等處,淮安等衛賦課。賑直隸水災。庚申,分神機營兵駐喜峰口。癸亥,命吳大澂解軍務幫辦來京,湘、鄂諸軍以魏光燾領之。乙丑,撥庫帑十萬加賑薊州等處災民。戊辰,知州徐慶璋集民團固守遼陽,命裕祿濟饟械。己巳,賑玉田、灤州、樂亭水災。日人狙擊李鴻章,彈傷其頰。庚午,日人犯澎湖。

三月壬申朔,命吳大澂回湖南巡撫任。癸酉,濟陽高家紙坊河決。乙亥,日兵陷澎湖。戊子,褫提督蔣希夷職,逮問。癸巳,命郭寶昌隨同劉坤一辦防務。己亥,李鴻章與日本全權伊藤博文、陸奧宗光馬關會議。和約成,定朝鮮為獨立自主國,割遼南地、臺灣、澎湖各島,償軍費二萬萬,增通商口岸,任日本商民從事工藝制造,暫行駐兵威海。

夏四月戊申,撥京倉米石備順天平糶。己酉,天津海溢,王文韶自請罷斥,不許。諭曰:「非常災異,我君臣惟當修省惕厲,以弭天災。」甘肅撒回叛,陷循化,雷正綰剿之。庚戌,命道員聯芳、伍廷芳赴煙臺與日本換約。乙卯,諭曰:「和約定議,廷臣交章謂地不可棄,費不可償,當仍廢約決戰。其言固出忠憤,而未悉朝廷苦衷。自倉卒開釁,戰無一勝。近者情事益迫,北可逼遼、瀋,南可犯畿疆。沈陽為陵寢重地,京師則宗社攸關。況慈闈頤養廿餘年,使徒御有驚,藐躬何堪自問?加以天心示警,海嘯成災,戰守更難措手。一和一戰,兩害兼權,而後幡然定計。其萬難情事,言者所未及詳,而天下臣民皆當共諒者也。茲將批準定約,特宣示前後辦理緣由。我君臣惟期堅苦一心,痛除積弊。」戊午,諭軍機大臣及諸臣工,和局已成,勿再論奏。留山東運糧十萬石備寧河等處賑。命裕祿接濟寧、錦等屬賑需。己未,賞前宿松縣知縣孫葆田五品卿銜。辛酉,達賴喇嘛受戒畢,賚哈達、念珠等物。癸亥,撥湖北漕米三萬石,備寧、錦等屬賑。乙丑,京師平糶。命李經方為臺灣交地全權委員。丙寅,賜駱成驤等二百八十二人進士及第出身有差。丁卯,召唐景崧來京。

五月辛未朔,賑臨漳等縣水災。庚辰,蔣希夷論斬。乙酉,見俄使喀希尼、法使施阿蘭於文華殿。壬辰,日本歸我遼南地。丁酉,免湖南新化,雲南阿迷、保山、昆明上年被災田賦。賑長武等縣水災雹災。庚子,唐景崧休致。

閏五月辛丑朔,撥山東庫帑二萬助賑奉天。壬寅,撫恤江、浙運漕稽候船戶萬餘人。甲辰,大學士福錕致仕。乙巳,命直隸提督聶士成總統淮軍駐津、沽,江西布政使魏光燾總統浙軍駐山海關,四川提督宋慶總統毅軍駐錦州,俱聽北洋大臣調度。癸丑,吳大澂罷。戊午,予惠潮嘉道裕庚四品京堂,充出使日本大臣。丁卯,諭曰:「近中外臣工條陳時務,如修鐵路,鑄鈔幣,造機器,開礦產,折南漕,減兵額,創郵政,練陸軍,整海軍,立學堂,大抵以籌餉練兵為急務,以恤商惠工為本源,皆應及時興舉。至整頓釐金,嚴覈關稅,稽察荒田,汰除冗員,皆於國計民生多所裨補。直省疆吏應各就情勢,籌酌辦法以聞。」

六月甲戌,孫毓汶以疾免。丁丑,賑熱河饑民。乙酉,軍機大臣徐用儀罷。以麟書為武英殿大學士,昆岡以禮部尚書協辦大學士。命錢應溥為軍機大臣,翁同龢、李鴻藻均兼總理各國事務衙門行走。戊子,賑鍾祥等處水災。

秋七月甲辰,沁河決。乙巳,滎澤河決。丁未,詔李鴻章入閣辦事。授王文韶直隸總督兼北洋大臣。戊申,賑商州、清澗等處水災雹災。己酉,予宋儒呂大臨從祀文廟。壽張、齊東河決。豐升阿遣戍軍臺。戊午,賑鎮安等縣水災。辛酉,江西巡撫德馨有罪褫職。色勒庫爾地震。壬戌,以回眾猖獗,褫總兵湯彥和職,楊昌濬、雷正綰並褫職留任。丁卯,已革提督黃仕林論斬。

八月壬申,賑富川、容縣水災。丙子,賑階、文、西寧等州縣水災。己卯,四川總督劉秉璋以不能保護教堂褫職。丙戌,命工部郎中慶常以五品京堂充出使法國大臣。癸巳,免雲南威遠被災田賦。

九月庚子,賑梧州府屬火災。留山東新漕備瀕河諸縣災賑。乙巳,留湖北冬漕三萬石備鍾祥等縣賑需。丁未,命魏光燾統軍援甘肅。戊申,免望都差徭,及退出圈地額賦五成,著為令。己酉,免陜西前歲民欠,暨華州開渠占地錢糧。壬子,見英使歐格訥於文華殿。乙卯,賑甘肅被擾各地難民。戊午,賑臨湘蛟災。撥帑三萬購倉穀,備常德、衡州旱災。壬戌,見和使克羅伯於文華殿。癸亥,命宗人府府丞吳廷芬兼總理各國事務衙門行走。丙寅,後藏班禪額爾德尼來京謁陵,進方物。揭陽、潮陽、普寧等縣地震。

十月辛未,楊昌濬罷,以陶模署陜甘總督。辛巳,李鴻章與日使互換歸遼條約。甲申,長麟、汪鳴鑾並以召對妄言褫職。己丑,初設新建陸軍,命溫處道袁世凱督練。丙申,免江川被災田賦二年。賑鶴慶等州縣水旱災。

十一月乙酉朔,山東趙家口合龍。丁未,免盛京被淹官莊額賦。戊申,留河南漕折八萬備內黃等縣工賑。己酉,以湖北布政使王之春充俄皇加冕賀使。庚戌,免奉天被兵各屬旗民兩年田賦,並積年逋賦。癸丑,劉永福免。癸亥,甘肅提督李培榮以赴援西寧逗留褫職。乙丑,調董福祥為甘肅提督,仍總統甘軍,前敵諸將均歸節制。賑保山蛟災。

十二月戊寅,壽張決口合龍。庚辰,撥庫帑六萬備湖北春賑。癸巳,改命李鴻章使俄,邵友濂副之。是月,免陜西前歲逋賦、奉天上年葦稅及官莊稅糧。賑盛京、萍鄉災。發帑各十萬,賑湖南、雲南、陜西各屬災。

二十二年丙申春正月丙申朔,停筵宴。丁酉,以特遣李鴻章使俄,諭止邵友濂、王之春毋往。己亥,賑長沙各府水旱災。乙卯,見各國公使於文華殿。庚申,命馮子材仍回廣東,督辦欽、廉防務。

二月庚午,移塔爾巴哈臺額魯特領隊大臣駐布倫布拉克,伊犁察哈爾營領隊大臣駐博羅塔拉。壬申,始議郵政與各國聯會。開龍州鐵路。劉銘傳卒。丁亥,戶部火。

三月戊戌,額勒和布致仕。癸卯,開杭州商埠。丁未,命王文韶、張之洞督辦蘆漢鐵路。辛酉,回匪窺珠勒都斯。癸亥,命董福祥駐西寧,專任剿撫,魏光燾還駐河州,尋命回陜西巡撫任。

夏四月壬申,五臺山菩薩頂災。乙亥,免昆明、丘北被災夏糧。辛巳,命榮祿往天津閱新建陸軍。戊子,授昆岡體仁閣大學士,榮祿以兵部尚書協辦大學士。

五月丁酉,諭李秉衡查州縣糧賦,浮收者覈減之。免恩安被災額賦。辛丑,鄭州文廟災。是月,上數奉皇太后臨醇王邸視福晉疾。癸卯,醇賢親王福晉葉赫那拉氏薨,輟朝十一日,上奉皇太后臨邸視殮,越日復往奠祭。懿旨,醇賢親王福晉薨逝,應稱曰「皇帝本生妣」。乙巳,上成服。壬子,免安徽歷年逋賦。甲子,緩鄂倫春牲丁進貂貢。

六月丙寅,諭奎順撫恤青海蒙民。丁卯,河決利津。戊辰,免浙江歷年各場積欠灶課鹽課。庚午,賑浙江風災。壬申,醇賢親王福晉金棺奉移,上躬詣臨送。甲戌,上奉皇太后如醇王園寓臨奠福晉金棺。己卯,諭整頓長江水師。壬午,命裕祿兼充船政大臣。丙戌,松潘番亂,官軍剿平之。丁亥,允王大臣請,神機營練兵處仿西制練兵。辛卯,永定河溢。是月,賑大東溝海溢災,安徽、湖北蛟災。

秋七月甲午朔,日有食之。丁酉,順天東南各屬水,命孫家鼐等速籌賑需。乙巳,留南漕十萬石於天津備賑。

八月乙丑,以關內外回匪漸平,諭陶模、董福祥安輯降眾,搜捕餘匪。己巳,川軍剿瞻對,疊克要隘,進逼中瞻。庚辰,諭鹿傳霖:「瞻對用兵,乃暫時辦法。事定後仍設番官否,當再審詳。不得因此苛責喇嘛,轉生他釁,慎勿鹵莽而行。」己丑,諭刑部訊獄應速結,毋任延宕。壬辰,禁各省濫用非刑。

九月丙申,福錕卒。免陜西前歲逋賦。己亥,東陵蟲災。丙午,賞盛宣懷四品京堂。先是,王文韶、張之洞請立招商輪船總公司,舉盛宣懷督辦。至是,旨下,並準專奏。大學士張之萬致仕。丁未,見德使海靖、比使費葛於文華殿。庚戌,命李鴻章在總理各國事務衙門行走。癸丑,李秉衡言勘明黃河尾閭,擬由舊黃河東岸挑濬新河,仍導歸舊河入海。諭以大舉興辦,務期一勞永逸,以副委任。

是秋,賑河南、奉天、湖北、安徽、山東、山西、吉林、黑龍江水災,湖南蛟災,及陜、甘水災雹災,新疆蝗災雹災,廣東洋面風災。

冬十月壬戌朔,賑湖北江、漢水災。癸亥,辦河州冬賑。甲子,增設蘇州、杭州、沙市、思茅四關。丙寅,諭陶模選廉明賢吏,和輯漢、回,偶有爭執,專論是非,準情理以劑其平,並分別撫恤被兵區域。論平回功,予董福祥騎都尉世職,授陶模陜甘總督,饒應祺新疆巡撫,予奎順、魏光燾優敘,其餘甄敘有差。甲戌,永定河決口合龍。戊寅,定朝鮮設領事,不立條約,不遣使,不遞國書,以總領事一人駐其都城。庚辰,命左都御史楊儒充出使俄奧荷大臣,道員羅豐祿充出使英義比大臣,黃遵憲充出使德國大臣,伍廷芳充出使美日祕大臣。癸未,免武清等州縣秋賦雜課。乙酉,賑華州等處水災。己丑,以徐桐為體仁閣大學士,李鴻藻以禮部尚書協辦大學士。

十一月戊申,冬至,祀天於圜丘。己酉,免朝賀。辛亥,免河、洮等處被災賦課。丁巳,命工部侍郎許景澄充出使德國大臣。是月,賑山東、四川水災。

十二月乙丑,初,鹿傳霖屢奏瞻對宜剿,擬收回後改設漢官。上慮失達賴心,命鹿傳霖、文海等詳議。至是,疏陳瞻民向化,藏番震懾各情。因諭剴切勸導達賴,期於保藏、保川兩無窒礙。賑四川東鄉等屬災。丙子,免遼陽各村屯糧賦,綏德等州縣逋糧。

二十三年丁酉春正月辛卯朔,停筵宴。丁酉,免山東光緒初年逋賦。辛亥,留湖北漕米充工賑。乙卯,見美、法、英、德、荷、比、俄、義、日本及日、奧諸國公使於文華殿。

二月壬戌,命戶部侍郎張廕桓使英。庚午,河決歷城、章丘。己卯,命崇禮、許應騤在總理各國事務衙門行走。

三月癸巳,詔汰冗兵。甲辰,懿旨發內帑十萬賑四川,五萬賑湖北,並以庫帑十萬加賑四川夔、綏、忠三屬。辛亥,免銅仁、青谿被水田賦。丁巳,初設海參葳委員。

夏四月乙亥,李秉衡奏減山東錢漕。

五月丙申,詔棍噶札拉參胡圖克圖嘉穆巴圖多普準轉世為八音溝承化寺胡圖克圖。甲辰,張之萬卒,贈太保。丁未,上詣本生妣醇賢親王福晉園寢,周年釋服。壬子,予呂海寰四品京堂,充出使德荷二國大臣。

六月己卯,賑崇陽等縣水災。

是夏,見奧使齊幹、俄使烏爾他木斯科、英使竇納樂、日使矢野文雄於文華殿。

秋七月庚寅,李鴻藻卒。丙申,命廖壽恆在總理各國事務衙門行走。辛丑,復故陜西固原提督雷正綰原官。甲辰,免岷州衛二十四寺進騾,並展緩馬貢。甲寅,平遙普洞村山陷入地中。

八月己巳,靖西地震。壬申,命翁同龢以戶部尚書協辦大學士。癸未,弛科布多札哈沁寶爾吉礦禁,許蒙、漢民人開採。乙酉,以鹿傳霖於德爾格忒土司措理失宜,罷改土歸流議,釋土司昂翁降白仁青暨其家屬,仍回德爾格忒管土司事。

九月戊子,鹿傳霖罷。己丑,命德爾格忒撤兵。戊戌,見挪威使柏固於文華殿。甲辰,達賴喇嘛請還瞻對地。諭恭壽等會商以聞。丙午,利津決口合龍。乙卯,復故陜甘總督楊昌濬官。

是秋,賑陜西雹災水災,湖南北、江西、廣東、安徽、雲、貴水災,新疆蝗災。

十月戊午,廣西巡撫史念祖坐事褫職。壬申,曹州匪戕害德國教士,命李秉衡察勘之。戊寅,德以兵輪入膠澳。壬午,免樂亭等州縣被災額賦。是月,賑廣東風災,陜西雹災,湖南、江南水災。

十一月辛卯,撥江北漕米三萬石,備徐、海各屬賑。甲午,詔罷三瞻改土歸流議,仍隸達賴喇嘛。辛丑,諭安撫江蘇各屬饑民。丁未,英使竇納樂入見。癸丑,冬至,祀天於圜丘。甲寅,免朝賀。昭烏達盟旗匪平。

十二月甲子,利津河決。己巳,免安州澇地租。乙亥,三巖野番就撫,改設土千戶,隸巴塘。罷硃窩、章谷兩土司歸流議。戊寅,詔各省保護教堂教士。免狄道、巴燕戎格等處額賦。

二十四年戊戌春正月乙酉朔,日有食之。元旦受禮改於乾清宮,停宗親宴。戊子,詔各省大吏定議籌餉練兵,速覆以聞。庚寅,定經濟特科及歲舉法。命中外保薦堪與特科者。乙未,免建水被旱夏糧。己酉,見各國公使於文華殿。壬子,免石屏、昆明夏糧。

二月甲子,命廖壽恆在軍機大臣上學習行走。丙寅,免清海阿里克番族馬貢銀。乙巳,留江北漕米一萬石賑徐州災。丁丑,命神機營選練先鋒隊。庚辰,詔武科改試槍砲,停默寫武經。

三月丁亥,詔立義倉。戊子,俄使巴布羅福覲見。乙巳,除新化被水額賦。是月,開直隸北戴河至秦王島、湖南岳州、福建三都澳口岸。

閏三月乙卯,召張之洞來京。丙辰,麟書卒。庚申,以德人入即墨文廟,毀聖賢像,下總署察問。乙丑,臨恭親王邸視疾。甲戌,上侍皇太后幸外火器營教場,閱火器、健銳、神機三營及武勝新隊操,凡三日。丁丑,以湖北沙市焚毀教堂,諭張之洞回任。續賑徐、海災。戊寅,見德親王亨利於玉瀾堂。己卯,還宮。免新興被旱額賦。庚辰,見法使畢勝於文華殿。壬午,安徽鳳、潁、泗災。

是春,以膠州灣租借於德意志,旅順口、大連灣、遼東半島租借於俄羅斯。

夏四月壬辰,恭親王奕薨,輟朝五日,素服十五日,臨邸賜奠。懿旨特謚曰忠。守衛園寢增設丁戶,每祭祀官經理之。孫貝勒溥偉襲。甲午,懿旨,恭忠親王功在社稷,應配饗太廟。詔中外臣工當法恭忠親王,各攄忠悃,共濟時艱。己亥,授榮祿為文淵閣大學士,剛毅為兵部尚書協辦大學士。乙巳,詔定國是,諭:「中外大小諸臣,自王公至於士庶,各宜發憤為雄。以聖賢義理之學植其根本,兼博採西學之切時勢者,實力講求,以成通達濟變之才。京師大學堂為行省倡,尤應首先舉辦。軍機大臣、王大臣妥速會議以聞。」丙午,詔各省立商務局。賜夏同龢等三百四十二人進士及第出身有差。己酉,翁同龢罷。選派宗室王公出洋游歷。近支王、貝勒等,上親察之;公以下及閒散人員,由宗人府保薦。召王文韶來京。裁督辦軍務處。庚戌,召見工部主事康有為,命充總理各國事務衙門章京。辛亥,前藏達賴喇嘛貢方物。

五月癸丑朔,詔陸軍改練洋操,令營弁學成者教練,於北由新建陸軍,於南由自強軍派往。各疆臣限六閱月,舉並餉練兵及分駐地,妥議以聞。其軍械槍砲,各省機器局酌定格式,精求制造。甲寅,賑棲霞火災。丁巳,詔自下科始,鄉、會、歲、科各試,向用四書文者,改試策論。命孫家鼐以吏部尚書協辦大學士,王文韶以戶部尚書為軍機大臣兼總理各國事務衙門行走。授榮祿直隸總督兼北洋大臣。庚申,趣盛宣懷蘆漢鐵路刻日興工,並開辦粵漢、寧滬各路。甲子,詔以經濟歲舉歸並正科,歲、科試悉改策論,毋待來年。丁卯,詔立京師大學堂,命孫家鼐管理。賞舉人梁啟超六品銜,辦理譯書局。戊辰,詔興農學。諭曰:「振興庶務,首在鼓勵人材。各省士民著有新書,及創新法,成新器,堪資實用者,宜懸賞以勸。或試之實職,或錫之章服。所制器給券,限年專利售賣。其有獨力創建學堂,開闢地利,興造槍砲廠者,並照軍功例賞勵之。」辛未,免祿勸被水田糧。癸酉,詔八旗兩翼諸營,均以其半改習洋槍、抬槍。以奕劻等管理驍騎營,崇禮等管理護軍營。甲戌,詔改直省各屬書院為兼習中西學校,以省書院為高等學,郡書院為中等學,州、縣書院為小學。其地方義學、社學亦如之。乙亥,命裕祿為軍機大臣。丙子,諭各省州縣實力保護教堂。丁丑,命三品以上京堂及各省督撫、學政舉堪與經濟特科者。頒士民著書,制器暨創興新政獎勵章程。命中外舉制造、駕駛、聲光化電人材。戊寅,詔各省保護商務。免海康、遂谿上年被災額賦。賑長安等州縣水災雹災。

六月癸未朔,詔改定科舉新章。丙戌,賑徐、海災。己丑,詔頒張之洞著勸學篇,令直省刊布。命康有為督辦官報。壬辰,命榮祿會同張之洞督辦蘆漢鐵路。鬱林、梧州土匪、會匪相結為亂,陷容、興業、陸川三縣,官軍剿平之。丙申,饒應祺進回部貢金。丁酉,命翰詹、科道輪班召對。部院司員條列時事,堂官代陳。士民得上書言事。設礦務鐵路總局於京師,王文韶、張廕桓專理之。庚子,湖南設制造槍砲兩廠。辛丑,賑寧羌火災,洵陽等縣水災雹災。癸卯,命伍廷芳賑古巴華民。乙巳,諭曰:「時局艱難,亟須圖自強之策。中外臣工墨守舊章,前經諭令講求時務,勿蹈宋、明積習,訓誡諄諄。惟是更新要務,造端宏大,條目繁多,不得不廣集眾長,折衷一是。諸臣於交議之事,當周諮博訪,詳細討論。毋緣飾經術,附會古義,毋膠執成見,隱便身圖。倘面從心違,致失朝廷實事求是本旨,非朕所望也。朕深惟窮變通久之義,創建一切,實具萬不得已之苦衷。用申諭爾諸臣,其各精白乃心,力除壅蔽,上下一誠相感,庶國是以定,而治道蒸蒸矣。」諭南北洋大臣籌辦水師及路礦學堂。諭各省廣開通商口岸。命黃遵憲以三品京堂充駐朝鮮大臣。

是夏,廣東九龍半島、山東威海衛俱租借於英吉利。

秋七月甲寅,詔停新進士朝考,並罷試詩賦。賑奉天被賊各縣災。丙辰,詔於京師設農工商總局,以端方、徐建寅、吳懋鼎督理,並加三品卿銜。命出使大臣設僑民學堂於英、美、日本各國。丁巳,河決山東上中游,濟陽等六縣同時並溢。己未,詔定於九月十五日奉皇太后幸天津閱兵。移沙市關監督、荊宜施道、江陵縣並駐沙市鎮。壬戌,賑南陽水災。乙丑,詔裁詹事府,通政司,大理、光祿、太僕、鴻臚諸寺,歸並其事於內閣、禮、兵、刑部兼理之。裁湖北、廣東、雲南巡撫,以總督兼管之。河東河道總督並於河南巡撫。兼裁各省糧道、鹽道。庚午,以抑格言路,首違詔旨,奪禮部尚書懷塔布、許應騤,侍郎堃岫、徐會灃、溥頲、曾廣漢等職。賞上書主事王照四品京堂。辛未,頒曾國籓州縣清訟事宜及功過章程於各省,並增道府功過。諭疏導京師河道溝渠,平治道塗。諭各省實行團練。賞內閣侍讀楊銳、中書林旭、刑部主事劉光第、江蘇知府譚嗣同並加四品卿銜,參預新政。賑建水水災。癸酉,罷李鴻章總理各國事務衙門行走。以裕祿為禮部尚書,在總理各國事務衙門行走。乙亥,置三、四、五品卿,三、四、五、六品學士。丙子,賑泰和水災。丁丑,召袁世凱來京。諭曰:「國家振興庶政,兼採西法,誠以為民立政,中西所同,而西法可補我所未及。今士大夫昧於域外之觀,輒謂彼中全無條教。不知西政萬端,大率主於為民開智慧,裕身家。其精者乃能淑性延壽。生人利益,推擴無遺。朕夙夜孜孜,改圖百度,豈為崇尚新奇。乃眷懷赤子,皆上天所畀,祖宗所遺,非悉使之康樂和親,未為盡職。加以各國環相陵逼,非取人之所長,不能全我之所有。朕用心至苦,而黎庶猶有未知。職由不肖官吏與守舊士夫不能廣宣朕意。乃至胥動浮言,小民搖惑驚恐,山谷扶杖之民,有不獲聞新政者,朕實為嘆恨。今將變法之意,布告天下,使百姓咸喻朕心,共知其君之可恃。上下同心,以成新政,以強中國,朕不勝厚望焉。」諭各省撤驛站,設郵政。嚴米糧出口禁。

八月壬午朔,命戶部編定歲出入表頒行之。諭出使大臣徵送僑民歸國備任使。命袁世凱以侍郎候補,專任練兵事宜。丙戌,見日本侯爵伊藤博文、署使林權助於勤政殿。賑射洪等縣水災,略陽等縣水災雹災。丁亥,皇太后復垂簾於便殿訓政。詔以康有為結黨營私,莠言亂政,褫其職,與其弟廣仁皆逮下獄。有為走免。戊子,詔捕康有為與梁啟超。庚寅,戶部侍郎張廕桓、翰林院侍讀學士徐致靖、御史楊深秀暨楊銳、林旭、劉光第、譚嗣同並坐康有為黨逮下獄。辛卯,上稱疾,徵醫天下。召榮祿來京。命逮文廷式,捕孫文。壬辰,詔復設詹事府,通政司,大理、光祿、太僕、鴻臚諸寺。禁官民擅遞封章。罷時務官報。各省祠廟毋改學堂。命吏部侍郎徐用儀在總理各國事務衙門行走。癸巳,撥江漕八萬石改折,備徐、海賑。賑高州水災。甲午,楊深秀、楊銳、林旭、劉光第、譚嗣同、康廣仁俱處斬。謫張廕桓新疆。徐致靖禁錮。命榮祿為軍機大臣。以裕祿為直隸總督兼北洋大臣。乙未,以康有為大逆不道,構煽陰謀,頒硃諭宣示臣下。罷巡幸天津閱操。命榮祿管兵部事,兼節制北洋諸軍及宋慶軍。丁酉,籍康有為、梁啟超家。命趙舒翹會同王文韶督辦礦路總局。諭蘇、浙新漕運京,罷改折議。留山東新漕米石備賑。戊戌,賞袁昶三品京堂,在總理各國事務衙門行走。庚子,李端棻以濫保褫職,戍新疆。褫王照職,籍其家,逮捕。辛丑,賞前御史文悌知府。壬寅,黃遵憲以疾免,賞李盛鐸四品京堂充出使日本大臣。陳寶箴以濫保奪湖南巡撫任。癸卯,詔疆臣飭吏治,培人才,開財源,修武備,舉劾牧令,整齊營規。詔言責諸臣指陳國計得失,其淆亂是非事攻訐者罪之。乙巳,懿旨復鄉、會試及歲、科考舊制,罷經濟特科,罷農工商局。丙午,端方進所編勸善歌,詔頒行。懿旨命疆臣保衛民生,慎選循良,整飭保甲團練。凡水利蠶桑,及制造販運,資民間利賴者,以時教導之。申聯名結會之禁。授榮祿為欽差大臣。己酉,命裕祿會辦蘆漢等處鐵路。設上海、漢口水利局。

九月辛亥朔,懿旨,一切政治關國計民生者,無論新舊,仍次第推行。建言諸臣章奏務裨時局,毋妄意揣摩。癸丑,發內帑二十萬賑山東水災。甘肅、新疆地震。丁巳,廣西匪平。己未,命軍機大臣會大學士及部院議治河之策。辛酉,初,強劫盜案,不分首從。至是,命樞臣暨法司詳議區別。代州地震。壬戌,免陜西咸寧等處逋課。戊辰,復武鄉、會試及童試舊制,惟營用武進士及投標武舉令習槍砲。復置湖北、廣東、雲南巡撫,河東河道總督。免裁糧道等缺。己巳,命許景澄在總理各國事務衙門行走。甲戌,復刑名解勘舊制,除軍務省分及情事重大者,仍就地正法,餘不準行。丙子,命胡燏棻在總理各國事務衙門行走。己卯,權停福州船廠制造。庚辰,命李鴻章往勘山東黃河。是月,賑直、陜、川、鄂、蘇、滇、晉、新等各省災。

冬十月辛巳朔,享太廟,禮親王世鐸攝行,是後郊廟祀典皆遣代,至辛丑冬自西安還京,始親詣。丙戌,命道員張翼督辦直隸暨熱河礦務,立公司。賑順天各屬災。丙申,賑韓城等縣災。己亥,命戶部撥帑八萬備安徽賑。辛丑,追奪翁同龢職。前湖南巡撫吳大澂坐事褫職。濟陽決口合龍。壬寅,懸賞購捕康有為、梁啟超、王照。甲辰,允榮祿請,以宋慶、聶士成、袁世凱、董福祥所部分立四軍,別募萬人為中軍。乙巳,見俄使格爾思於勤政殿。命胡燏棻督辦津鎮鐵路,以張翼副之。丁未,賑羅平水災雹災。

十一月癸丑,諭張汝梅辦山東災賑。賞桂春三品京堂,命在總理各國事務衙門行走。甲寅,命啟秀為軍機大臣,趙舒翹、聯元並在總理各國事務衙門行走。丁巳,留河南漕折於滑縣備賑。撥庫帑二十萬於江蘇備賑。己巳,命溥良察山東賑。庚午,命裕庚在總理各國事務衙門行走。辛未,命疆臣均兼總理各國事務大臣銜。壬申,賑吐魯番等處水災蝗災。丁丑,以稱疾停年節升殿筵宴。戊寅,罷直隸練軍。

十二月丙戌,湖北巡撫曾鉌坐事免。癸巳,命馬玉昆往河南辦理防剿。罷胡燏棻津蘆路督辦,以許景澄代之。丁酉,免漢陽等州縣被災額賦。壬寅,改湖北漢口同知為夏口撫民同知。戊申,發內帑五萬於清、淮備賑。

二十五年己亥春正月庚戌,撫恤豫、皖被賊州縣災民。丙辰,詔清庶獄。庚申,免渦陽等州縣被賊稅糧。辛酉,止各國駐京公使覲賀。壬戌,再撥部帑五萬於安徽備賑。丙寅,召李秉衡來京。

二月甲申,申諭各省辦積穀、清訟、團練、保甲。丁亥,命武勝新隊名曰虎神營。舉行京師保甲。戊戌,膠州灣德兵藉詞護教,入沂州境。命呂海寰告德國外部,止其進兵。以新建陸軍訓練有效,予袁世凱優敘。庚子,命副都統壽山募練十六營,為鎮邊新軍。甲辰,德兵至蘭山。丁未,陷日照城。

三月乙卯,諭有漕各州縣,自今冬始,改徵本色運京師。丁丑,召蘇元春來京。

夏四月癸未,諭曰:「近因時事艱難,朝廷孜孜求治,疊諭疆臣整頓一切。旋據覆陳練兵、籌餉、保甲、團練、積穀各事,雖匪空言,尚虛確效。用再諭令所籌諸務,速即興辦。仍將有無成效,據實以聞。」申諭疆臣切實校閱營伍。又諭察勘荒田,勸導民墾,勿任吏胥訛擾,亦毋遽擬升科。義人以兵艦來,圖登三門灣,諭嚴戒備。己丑,命剛毅往江南諸省覈庫藏出納實數。癸巳,命聶士成軍馬步四營駐熱河,實邊防。丙申,諭劉坤一等集重兵為備,義兵登陸,即迎擊之。丁酉,命按察使李光久督辦浙江防剿,長順往吉林稽察練兵。乙巳,詔:「關稅、釐、鹽諸課,歲有常經,疆吏瞻徇,不能力除積弊。大學士、軍機大臣其詳覈會議以聞。」

五月壬子,命吳廷棻在總理各國事務衙門行走。甲寅,神機營兵廠藥庫火。乙卯,命太僕少卿裕庚充出使法國大臣。乙丑,命正定鎮總兵楊玉書統練軍駐熱河。除安化、武岡、新寧被水田賦。己巳,岳州開商埠,移嶽常澧道駐之,兼岳州關監督。

六月戊子,免迪化等屬逋賦。丁酉,諭整飭海軍,除積弊。庚子,賑廬陵等縣水災。

秋七月庚戌,以法人租借廣州灣,命蘇元春往會勘。乙卯,訂朝鮮通商條約。丁巳,開秦皇島商埠。己巳,命剛毅往廣東清釐財政。庚午,命蘇元春赴淮、徐練兵,聽榮祿節制。

八月丁亥,甘肅海城回亂,官軍剿平之。己亥,詔各省宣講聖諭廣訓。甲辰,錦州、廣寧匪亂,剿平之。

九月丁未,以旱詔求直言。庚戌,詔清訟獄,緩徵輸。諭疆吏整躬率屬,持公道,順輿情。己未,副都統壽長以廢弛營務,褫職謫戍,榮和褫職逮問。辛酉,命李徵庸充督辦四川商礦大臣。甲戌,義人兵艦續至,諭直、魯、江、浙嚴防。

是秋,賑浙江、湖南、甘肅水災,陜西旱災。

冬十月庚寅,命李秉衡巡閱長江水師。丙申,命李鴻章為通商大臣,考察商埠。壬寅,免陜西咸寧等處前歲逋賦。

十一月癸丑,命太僕寺卿徐壽朋充出使韓國大臣。甲寅,廖壽恆罷軍機大臣,命趙舒翹在軍機大臣上學習行走。免北流被賊上年逋賦。壬戌,再暴康有為、梁啟超罪狀,懸賞嚴捕。戊辰,孫家鼐以疾免。己巳,以戶部尚書王文韶協辦大學士。

十二月甲戌朔,詔停年節升殿筵宴。丙子,舉行察典,敕毋冒濫。乙酉,免榆林等處被災田糧。己丑,罷蘇元春江南練兵,回廣西提督任。乙未,命陳澤霖募勇駐江北操練,為武衛先鋒右軍。丁酉,詔以端郡王載漪之子溥俊為穆宗嗣,封皇子。命崇綺直弘德殿,授皇子溥俊讀。壬寅,詔來年三旬壽辰,停朝賀筵宴,止文武大吏來京祝嘏。特舉恩科,明年庚子鄉試,次年辛丑會試。其正科鄉、會試,遞推於辛丑、壬寅年舉行。

是冬,賑山西、雲南、陜西、甘肅、山東等屬災。

是歲,廣州灣租借於法郎西,並開滇越鐵路。

二十六年庚子春正月甲辰朔,詔以三旬慶辰,加宗支近臣恩賚。己酉,命醇親王載灃直內廷,命侍講寶豐直弘德殿。停本年秋決。壬子,先是,知府經元善聯名上書諫立嗣。至是,詔嚴捕治罪,尋籍其家。戊子,詔大索康有為、梁啟超,毀所著書,閱其報章者並罪之。壬戌,三嚴夷平,增設巴塘等處土官各職。癸亥,總署與法人議廣州灣租約,訂期九十九年。甲子,留南漕三萬石賑河北災民。是月,拳匪起山東,號「義和拳會」,假仇教為名,劫殺相尋,蔓延滋害。

二月丙子,河決濱州。乙酉,免昆明等州縣被災額賦。戊戌,定墨國條約。

三月戊申,命李盛鐸使日本,賀其太子聯姻;呂海寰使德,賀其太子加冠。壬子,濱州決口合龍。癸丑,以旱詔中外慮囚。甲寅,賞高賡恩四品京堂,直弘德殿。丁巳,命內閣學士桂春充出使俄國大臣,尋命兼使奧國。撥部帑十萬賑山東、貴州各屬水災。己未,靖遠夷就撫,置諸夷土官。壬戌,命袁世凱集新兵二十營,增立一軍,名為武衛右軍先鋒隊。

夏四月乙酉,善聯罷,以許應騤兼管船政大臣。庚寅,義和拳入京師,詔步軍統領等會議防禁以聞。辛卯,免宣威、嵩明被水秋賦。丙申,賑重慶等處水旱災。丁酉,總署言拳會造言煽惑,人心浮動,易肇釁端。諭所司妥議。授李鴻章兩廣總督。庚子,免新化等州縣被水額賦。是月,拳匪焚毀保定鐵路,副將楊福同往鎮攝,行及淶水,被戕。

五月癸卯,拳匪毀琉璃河、長辛店車站局廠。命聶士成護蘆保、津蘆兩路,防禦之。甲寅,命載漪管總理各國事務衙門,啟秀、溥興、那桐同時兼行走,罷廖壽恆。乙卯,拳匪殺日本使館書記杉山彬於永定門外。丁巳,諭令馬玉昆赴京西剿拳匪。大沽戒嚴。己未,拳匪擾五城,坊市流血。詔步軍統領神機營、虎神營、武衛中軍會巡,大臣巡察街陌,分駐九門監啟閉。召李鴻章、袁世凱入衛。庚申,榮祿以武衛中軍護各國使館。命李端遇、王懿榮為京師團練大臣。召李秉衡及馬玉昆統兵來京。是夕,拳匪焚正陽門城樓,閭市灰燼。庚申,詔剛毅、董福祥募拳民精壯者成軍,自餘遣散。辛酉,詔各省以兵入衛。外軍攻大沽口,提督羅榮光不能御,走天津,死之,大沽遂陷。裕祿以捷聞,詔發內帑十萬犒師。壬戌,命徐桐、崇綺會同奕劻、載濂等商軍務。癸亥,命許景澄、那桐往告各國公使速出京。自庚申至於是日,皇太后連召王大臣等入見,諮眾論。載漪持戰議甚堅。載勛、載濂、載瀾、徐桐、崇綺、啟秀、溥良、徐承煜等,更相附和。榮祿依違其間。獨許景澄、袁昶言匪宜剿,釁不可開,殺使臣,悖公法,辭殊切直。故有是命。甲子,拳匪戕德使克林德於崇文門內。乙丑,詔以中外釁啟,飭戰備。罷崇禮步軍統領。以載勛代之。發倉米開糶濟民食。庚午,召鹿傳霖來京。

六月辛未朔,諭順天府五城平糶,瘞教民暴骸。癸酉,命倉場侍郎劉恩溥往天津募水會強壯者,編立成軍,與通州、武清、東安團民駐直隸,濟之餉械。發倉於通州開糶。長萃等言津通道阻,請暫停漕運,不許。乙亥,諭各省護教士回國,教民悔悟自首者許自新。己卯,南漕運阻,命清江浦置局,採買運京。壬午,調李鴻章為直隸總督兼北洋大臣,趣兼程來京。乙酉,詔展緩本年恩科鄉試,明年三月八日舉行,會試八月八日舉行,庚子正科鄉試及會試以次遞推。外兵襲天津,聶士成戰於八里臺,死之。戊子,以呂本元為直隸提督。天津陷,裕祿、宋慶、馬玉昆並退守北倉。庚寅,命顧璜、張仁黼會辦河南團防。下戶部尚書立山於獄。辛卯,詔緝戕害德使兇犯。額勒和布卒。丙申,上三旬萬壽,東華門不啟,群臣朝賀皆自神武門入。免疏附、拜城被災額賦。賑福建水災。

秋七月庚子朔,命李秉衡幫辦武衛軍事,張春發、陳澤霖、萬本華、夏辛酉諸軍並聽節制。壬寅,殺吏部侍郎許景澄、太常寺卿袁昶。乙巳,調馬玉昆為直隸提督。丁未,命榮祿以兵護各國公使往天津。己酉,外兵據北倉。庚戌,陷楊村,直隸總督裕祿自殺。壬子,授李鴻章全權大臣,與各國議停戰。外兵襲蔡村。癸丑,李秉衡戰於蔡村,敗績。外兵進占河西塢。甲寅,增祺言蓋平、熊岳先後失守。丙辰,殺戶部尚書立山、兵部尚書徐用儀、內閣學士聯元。李秉衡戰敗於張家灣,死之。丁巳,外兵陷通州。命剛毅幫辦武衛軍事。己未,德、奧、美、法、英、義、日、俄八國聯兵陷京師。庚申,上奉皇太后如太原,行在貫市。壬戌,次懷來。命榮祿、徐桐、崇綺留京辦事。癸亥,廣東布政使岑春煊率兵入衛,遂命扈蹕。甲子,次沙城堡。懿旨命岑春煊督理前路糧臺。丁丑,次雞鳴驛,下詔罪己,兼誡中外群臣。丙寅,次宣化。命萬本華、孫萬林、奇克伸布軍聽馬玉昆節制,駐後路。丁卯,詔求直言。免蹕路所過宛平、昌平等處錢糧一年。

八月庚午朔,次左衛。辛未,次懷安。壬申,次天鎮。詔奕劻還京,會李鴻章議和。癸酉,次陽高。甲戌,次聚樂鎮。太監張天順騷擾驛站,處斬。乙亥,次大同。命劉坤一、張之洞會議和局。以載漪為軍機大臣。戊寅,賞隨扈王公暨大小臣工津貼銀兩。己卯,次懷仁。命京師部、院、卿寺堂官暨內廷行走者,除留京外,均率司員赴行在。辛巳,次廣武鎮。命程文炳統軍駐潼關。壬午,次陽明堡。諭榮祿收集整理武衛中軍。癸未,次原平鎮。諭廷雍督剿直屬拳匪。甲申,次忻州。丙戌,次太原,御巡撫署為行宮。免蹕路所過天鎮、陽高等州縣今歲額賦。丁亥,西安等府旱。戊子,諭榮祿約束武衛中軍。癸巳,詔有司勸教民安業,拳民被脅者令歸農。乙未,賑四川各屬災。

閏八月庚子朔,賑麗水等縣水災。辛丑,追悼德使克林德,命昆岡往奠之。論庇拳啟釁罪,削莊親王載勛、怡親王溥靜、貝勒載濂、載瀅爵。罷載漪、載瀾、剛毅、趙舒翹、英年職,並下府部議。命鹿傳霖為軍機大臣。壬寅,以日書記杉山彬被戕,遣那桐使日本致祭賻。毓賢罷。乙巳,詔幸西安。丁未,啟蹕。是日,次徐溝。戊申,次祈縣。己酉,次平遙。庚戌,次介休。辛亥,次靈石。壬子,次霍州。召榮祿赴行在。甲寅,詔改陜西巡撫署為行宮。乙卯,次平陽。丙辰,次史村驛。諭北五省嚴捕自立會黨。戊午,次聞喜。己未,詔以西幸,陵寢壇廟久疏對越,命奕劻遴近支王貝勒代享太廟及祭東西陵,太常寺派員祭壇廟。尋令今歲除夕、來歲元旦祀典,並遣代行。趣近省解京餉給在京官弁俸糧。授奕劻全權大臣,會李鴻章議和約,劉坤一、張之洞仍會商。辛酉,次臨晉。癸亥,次蒲州。諭江蘇等省解款百萬濟京城俸饟。免蹕路所過太原、陽曲等屬今歲額賦。乙丑,次潼關。賑福州水災。丁卯,次華陰。命敬信、溥興管理虎神營。戊辰,次華州。

九月己巳朔,次渭南。壬申,至西安府,御巡撫署為行宮。甲申,以裕鋼為駐藏辦事大臣。丙子,予殉難祭酒王懿榮世職,並旌其妻謝氏、子婦張氏。乙卯,李鴻章奏誅附匪逞亂道員譚文煥。壬午,德人陷紫荊關,布政使升允退軍浮圖億。尋奏德兵退易州,上以其張皇,切責之。己丑,罷保德貢黃河冰魚。庚寅,削載漪爵,與載勛、溥靜、載瀅並交宗人府圈禁。載瀾、英年金雋級。趙舒翹奪職留任。剛毅病故,免議。毓賢戍極邊。壬辰,予闔家自焚黑龍江將軍延茂、祭酒熙元、侍讀寶豐、崇壽等恤。乙未,賑陜西荒。丙申,免陜西咸寧等縣逋賦。戊戌,免雲南各州縣暨土司被災逋賦。

冬十月戊申,皇太后聖壽節,停筵宴。庚戌,詔董福祥不諳外情,遇事鹵莽,奪提督,仍留任。辛亥,發內帑四十萬賑陜西饑民,趣江、鄂轉漕購糧以濟。癸丑,授王文韶為體仁閣大學士,崇禮、徐郙並協辦大學士。丁巳,諭廓爾喀、前後藏及各土司暫勿貢獻。癸亥,開秦、晉實官捐例賑旱災。

十一月壬申,免長安額賦十之五。乙亥,清平苗匪王老九等作亂,官軍剿擒之。庚辰,命楊儒為全權大臣,與俄議交收東三省事。辛巳,以長沙等府旱災,開賑捐事例。壬午,免蹕路所經山西各州縣額賦十之二。癸未,命盛宣懷為會辦商務大臣。乙酉,命徐壽朋赴京隨辦商約。癸巳,安徽開籌餉捐例。丙寅,增祺坐擅與俄人立交還奉天暫行約,予嚴議,尋褫職。

十二月甲辰,詔免明年元旦禮節。丁未,詔議變法,軍機大臣、大學士、六部、九卿、出使大臣、直省督撫參酌中西政要,條舉以聞。庚戌,諭直省大小官吏保護外僑,違者重譴。嚴立會仇教之禁,犯者問死刑。壬子,命左都御史張百熙充專使英國大臣。甲寅,留京大臣奏京師盜風甚熾,權用重典,允之。庚申,賞張佩綸編修,隨李鴻章辦交涉。壬戌,詔復冤陷諸臣立山、徐用儀、許景澄、聯元、袁昶職。再論縱匪肇亂首禍諸臣罪,奪載瀾爵職,與載漪並謫新疆禁錮。褫剛毅職。英年、趙舒翹並褫職論斬。追褫徐桐、李秉衡職。啟秀、徐承煜褫職聽勘。董福祥褫職解任。癸亥,下詔自責。以當時委曲苦衷示天下。並誡中外諸臣激發忠誠,去私心,破積習,力圖振作。

二十七年辛丑,行在西安。春正月戊辰朔,詔以救濟順直兵災,開實官捐例。罷多倫諾爾歲貢海龍諸皮。庚午,賜載勛自盡。辛未,毓賢處斬。癸酉,英年、趙舒翹並賜自盡。剛毅、徐桐、李秉衡並論斬,以前沒免。乙亥,啟秀、徐承煜處斬。庚辰,免仁和等縣荒廢田糧。辛巳,免新會貢橙。

二月己亥,撥部帑百萬於山西備賑。壬子,廣東郎中黎國廉等進方物,升敘有差。

三月戊辰,免蹕路所過暨被災陜西咸寧等處稅糧。己巳,詔立督辦政務處,奕劻、李鴻章、榮祿、昆岡、王文韶、鹿傳霖並為督理大臣,劉坤一、張之洞遙為參預。甲戌,免雲南臨安等處逋糧。丁丑,論拳匪仇教保護不力罪,奪已故總督裕祿、駐藏大臣慶善原職,褫浙江巡撫劉樹棠職,布政使榮銓、副都統晉昌褫職戍極邊,道員鄭文欽、知縣白昶、都司周之德並處斬,餘褫謫有差。撥山東漕米五萬石賑直隸災民。壬午,諭免自京來行在各署司員停補扣資。

夏四月丁酉,賞在京王公百官半俸,旗、綠營兵丁一月錢糧。辛丑,命馬玉昆剿近畿餘匪,瞿鴻禨在軍機大臣上學習行走。丁未,命瞿鴻禨兼政務處大臣。己酉,賑直隸旱災。壬子,詔開經濟特科,命中外舉堪與試者。免各省例貢,除茶葉藥材及關祭品者,一切食物悉罷之。癸丑,命載灃充德國專使大臣。庚申,詔從各國議,停順天、奉天、黑龍江、直隸、山西、河南、陜西、浙江、江西、湖南諸省考試五年。壬戌,命張百熙等修京師蹕路。癸亥,停吉林今歲貢。

五月乙丑,命那桐充日本專使大臣。展山西本年恩、正兩科鄉試。癸未,賞道員蔡鈞四品京堂,充出使日本大臣。甲午,賑墨爾根等處災。

六月丙申,命副都統廕昌充出使德國大臣,尋命為荷蘭兼使。賞知府許臺身道員,充出使韓國大臣。庚子,萬壽節,停朝賀筵宴。癸卯,詔置外務部,以總理各國事務衙門改設之,命奕劻總理,王文韶為會辦大臣,瞿鴻禨任尚書並會辦大臣,徐壽朋、聯芳為侍郎。庚戌,各國聯軍去京師。壬子,發內帑五萬於江西備賑。賑棲霞火災。

秋七月甲子朔,命鄧增節制隨扈諸軍。免陜西、河南、直隸蹕路所過地額賦。乙丑,詔除漕務積弊,河運海運並改徵折色,在京倉採運收儲。世鐸罷直軍機。己巳,河決章丘、惠民。己卯,詔改科舉自明年始,罷時文試帖,以經義、時務策問試士,停武科。予羅豐祿三品京堂,充出使俄國大臣。戊子,全權大臣奕劻、李鴻章與十一國公使議訂和約十二款成。己丑,展陜西鄉試於明年舉行。壬辰,詔永罷實官捐例。諭各省建武備學堂。癸巳,諭各省裁兵勇,改練常備、續備、警察等軍。

八月甲午朔,以回鑾有日,遣官告祭西嶽、中嶽。蹕路所經名山大川、古帝王陵寢、先儒名臣祠墓,並由疆吏遣官致祭。乙未,詔直省立學堂。戊申,廢內外各署題本,除賀本外,均改為奏。壬子,命盛宣懷為辦理商稅大臣。癸丑,詔以變法圖強示天下,並以劉坤一、張之洞條奏命各疆吏舉要通籌。丁巳,車駕發西安。己未,升允奏臨潼知縣夏良材誤供應,請褫職。皇太后命從輕議。升允自請處分,原之。

九月己酉,李鴻章卒,贈太傅,晉一等侯爵。命王文韶署全權大臣,袁世凱署直隸總督兼北洋大臣。

是秋,發帑十五萬賑陜西、安徽災,留漕款十萬、漕米六萬石備安徽、江蘇賑。又賑兩湖、安徽、雲南水災,江蘇潮災。

冬十月癸巳朔,日有食之。甲午,次開封。惠民決口合龍。丙申,賞道員張德彞三品卿銜,充出使英國大臣,旋命兼使義比。壬寅,皇太后聖壽節,停朝賀。壬子,懿旨撤溥俊皇子名號。丙辰,詔展會試於癸卯年。其明年順天鄉試及癸卯科會試,權移河南貢院舉行。

十一月丙子,特予故大學士李鴻章建祠京師。戊子,命貽穀督晉邊墾務。章丘決口合龍。庚寅,上奉皇太后至自西安。辛卯,詔以珍妃上年殉節宮中,追晉貴妃。命翰、詹、科道及各署司員,按日預備召見。

十二月癸巳朔,命王文韶仍督辦路礦,瞿鴻禨副之,袁世凱督辦關內外鐵路事宜,胡燏棻會辦。丙申,申諭中外臣工,重邦交,安民教。以比匪誤國,附和權貴,褫左副都御史何乃瑩、侍講學士彭清藜、編修王龍文、知府連文沖、曾廉職。丁酉,賑蹕路所過三十里內貧民。己亥,祀天於圜丘。自戊戌年八月至於是月,始親詣。庚子,祭大社、大稷。遣睿親王魁斌等告祭方澤、朝日壇、夕月壇,恭親王溥偉、貝子溥倫詣東西陵告祭。壬寅,命袁世凱參預政務處。甲辰,命鎮國將軍載振充英國專使,賀其君加冕,尋晉貝子銜。免蹕路所過河南州縣額賦十之三。賑廣西火災。辛亥,兩宮見各國公使於乾清宮。免雲南銅廠積年民欠。甲寅,以瞿鴻禨為軍機大臣。授孫家鼐體仁閣大學士。乙卯,兩宮見各國公使暨其夫人等於養性殿。丁巳,免山西州縣歷年逋賦倉穀。庚申,祫祭太廟。辛酉,上始復御保和殿,筵宴蒙古王公暨文武大臣。免浙江仁、錢等州縣,杭嚴、嘉湖二衛未墾地畝糧賦。

二十八年壬寅春正月庚午,享太廟。辛未,祈穀於上帝。癸酉,四川提督宋慶卒,晉封三等男爵。丁丑,命張翼總辦路礦事宜,王文韶、瞿鴻禨為督理,呂海寰會盛宣懷議商約。戊寅,罷河東河道總督。命各省大吏清釐屯地,改屯食襄為丁糧,撤衛官歸營,屯丁、運軍並罷。諭各省立農工學堂。戊子,罷詹事府、通政使司。

二月壬辰朔,命張德彞充日斯巴尼亞專使,賀其君加冕。癸巳,諭各省亟立學堂暨武備學堂,開館編纂新律。甲午,廣西游匪戕法兵官,剿辦之。丁酉,釋奠於先師。戊戌,祭大社、大稷。庚戌,劉坤一乞疾,慰留。

三月辛酉朔,交收東三省條約成。甲子,見義使嘎釐納於乾清宮。乙丑,祀先農,親耕耤田。丙寅,上奉皇太后謁東陵,免蹕路所過州縣額賦十之三。己巳至庚午,謁諸陵。甲戌,幸南苑,駐蹕團河行宮。壬午,至自東陵。癸未,皇后祀先蠶。

是春,免宣威、昆明及齊齊哈爾、墨爾根旗屯災賑。免榆林等處逋賦,西安等縣秋糧十之二。

夏四月壬辰,見俄使雷薩爾於乾清宮。甲午,常雩祀天。丙申,命沈家本、伍廷芳參訂現行法律。戊戌,李經羲以陳奏失辭,免雲南巡撫,下部議。壬寅,命許鎯充出使義國大臣,吳德章充出使奧國大臣,楊兆鋆充出使比國大臣。癸卯,皇后躬桑。甲辰,裁銀、緞匹、顏料三庫,罷管庫大臣。乙卯,免灤平被災地課。

五月壬戌,授袁世凱直隸總督兼北洋大臣。免雙城逋賦。甲子,見各國公使等於樂壽堂。丙寅,廣西匪陷廣南之皈朝,雲南官軍擊走之,復其城。丙子,夏至,祭地於方澤。戊寅,見美使康格等於乾清宮。

六月己丑朔,免鶴慶、賓川被災雜賦。丙申,命孫寶琦充出使法國大臣,胡惟德充出使俄國大臣,梁誠充出使美日祕大臣。庚戌,見美使康格及博覽會長巴禮德於乾清宮。辛亥,命張之洞為督辦商務大臣。癸丑,賑四川南充、簡等屬災。

秋七月庚午,頒行學堂章程。

八月甲申,移雲南迤西道駐騰越,兼監督關務。戊戌,袁世凱請裁陋規加公費,命他省仿行。癸卯,河決利津、壽張等處。己酉,見德使葛爾士等於仁壽殿。庚戌,河復決惠民。

九月癸巳,兩江總督劉坤一卒,追封一等男,贈太傅。命張之洞署兩江總督兼南洋大臣。免天津被兵新舊額賦。丁酉,見法使賈斯那等於仁壽殿。甲辰,見各國公使於仁壽殿。壬子,命袁世凱充督辦商務大臣,伍廷芳副之,兼議各國商約。

是秋,發庫帑三十萬,續撥義賑十二萬,並於四川備賑。又賑山東、廣東、雲南、福建、貴州等屬水災。

冬十月戊子,中英商約成。己丑,湖南都司劉長儒坐不保護教士處斬。是月,賑山、陜各屬災。雲南劍川、鶴慶州,新疆疏勒等縣俱地震。

十一月戊午,詔自明年會試始,凡授編、檢及改庶常與部屬中書用者,胥肄業京師大學堂,俟得文憑,始許散館及奏留。分省知縣亦各入課吏館學習。己未,以有泰為駐藏大臣。辛酉,發內帑、部帑各五萬於山東備賑。壬戌,調魏光燾為兩江總督兼南洋大臣。丙寅,免臨潼被水地課五年。庚辰,冬至,祀天於圜丘。是月,見法使呂班、美使康格於乾清宮。

十二月癸卯,命袁世凱充督辦電務大臣。辛亥,旌殉親異域使俄大臣楊儒子錫宸孝行。是月,免江、浙各州縣衛額賦,宜良被水租糧。

二十九年癸卯春正月丁巳朔,停筵宴。以明歲皇太后七旬聖壽,詔開慶榜,本年為癸卯恩科鄉試,來年為甲辰恩科會試,其正科鄉、會試並於下屆舉行。乙丑,見美使康格等於乾清宮。丁卯,命榮慶同管大學堂事。己巳,見各國公使等於養性殿。丁亥,免鎮西、疏附被災糧賦。

二月壬子,惠民決口合龍。

三月丙辰朔,日有食之。庚申,見德親王亨利、公使葛爾士等於乾清宮。詔以謁陵取道鐵路,禁攤派差徭,扈從並免供給。辛酉,裁官學滿、漢總裁及教習。癸亥,祀先農,親耕耤田。上奉皇太后謁西陵。乙丑,幸保定府駐蹕,免蹕路所過州縣額賦十之三。己巳,榮祿卒,贈太傅,晉一等男。罷印花稅及一切苛細雜捐,科派侵漁者論如律。庚午,命奕劻為軍機大臣。癸亥,幸南苑。甲戌,幸團河駐蹕。庚辰,命奕劻、瞿鴻禨會戶部整理財政。立銀錢鑄造總廠於京師。命載振、袁世凱、伍廷芳參訂商律。辛巳,至自南苑。是月,免陜西庚子年逋賦。

夏四月己亥,見各國公使於仁壽殿。己酉,雲南匪陷臨安府城。庚戌,免蹕路所過州縣旗租。辛亥,命崇禮為東閣大學士,敬信協辦大學士。

五月癸亥,命鐵良會袁世凱練京旗兵。戊辰,戶部火。甲戌,命楊樞充出使日本大臣。乙亥,雲南惈夷平。壬午,賜王壽彭等三百一十五人進士及第出身有差。

閏五月甲申朔,命馮子材會岑春煊辦理廣西軍務。丙戌,命張之洞會張百熙、榮慶釐定大學堂章程。庚寅,滇軍復臨安府城,石屏匪首周云祥伏誅。壬辰,自四月不雨,至於是日雨。丙申,廣西巡撫王之春、提督蘇元春並褫職,以柯逢時為廣西巡撫,劉光才為廣西提督。己亥,御試經濟特科人員於保和殿。壬寅,命馬玉昆巡緝近畿盜賊。甲辰,中英續訂商約成。

六月壬戌,予考取特科袁嘉穀等升敘有差。癸亥,逮蘇元春下獄。丁卯,世鐸等請加上皇太后徽號。懿旨以廣西兵事方殷,民生困苦,不許。丁丑,河決利津。是月,見日使內田康哉等、義使嘎釐納等於仁壽殿。山東煙臺水災,賑之。

秋七月乙酉,開廈門、鼓浪嶼為各國公地。辛卯,賞鄭孝胥四品京堂,督辦廣西邊防,得專奏。昆岡致仕。戊戌,初置商部,以載振為尚書。

八月壬子朔,王公百官豫請來年皇太后七旬萬壽報效廉俸申祝,懿旨止之。癸丑,免靈州瀕河地糧。丁卯,日本商約成。庚寅,見各國公使於仁壽殿。壬申,以敬信為體仁閣大學士,裕德協辦大學士。丁丑,見法使呂班、德使穆默於仁壽殿。

九月丙申,命榮慶在軍機大臣上學習行走。調那桐為外務部尚書兼會辦大臣。丁酉,命那桐與奕劻、瞿鴻禨整理戶部財政,榮慶充政務處大臣。戊戌,命孫家鼐、張百熙並充政務處大臣。

是秋,賑湖北、陜西等屬水災,懷柔雹災,雲南各屬水旱災雹災,鎮西、綏來蝗災凍災。

十月辛亥朔,見荷使希特斯於乾清宮。戊午,以英秀接收阿勒臺借地,率議展緩,命瑞洵往按之。丙寅,置練兵處,命奕劻總理,袁世凱、鐵良副之。甲戌,命岑春煊總統廣西諸軍。乙亥,賞楊晟四品卿銜,充出使奧國大臣。丙子,袁世凱劾張翼擅售開平煤礦暨秦王島口岸於外人。詔褫職,責令收回。

十一月丙午,諭曰:「興學育才,當務之急。據張之洞同管學大臣會訂學章所稱,學堂、科舉合為一途,俾士皆實學,學皆實用。著自丙午科始,鄉、會中額,及各省學額,逐科遞減。俟各省學堂辦齊有效,科舉學額分別停止,以後均歸學堂考取。」丁未,改管學大臣為學務大臣,以孫家鼐任之。

十二月丙辰,廣西匪首覃志發等伏誅。戊午,詔內務府再減宮廷用費,罷一切不急工作。己巳,置翰林學士撰文,並增員缺,更定品級。丙子,以日、俄手冓兵,中國守局外中立例,宣諭臣民。己卯,授榮慶軍機大臣。是月,免安州被澇、昆明被旱地畝租糧。

是冬,賑甘肅、雲南各州縣水災,南州、新化蛟災,瀘州火災。

三十年甲辰春正月癸未,移廣西鹽道駐梧州,兼關監督。河決利津王莊。甲申,見美、英、法、德、日、義、比、荷、葡各使康格等於乾清宮。己丑,雲南提督張春發有罪,褫職戍軍臺。甲午,以皇太后七旬聖壽,上御太和殿,頒詔天下,覃恩有差。已亥,雲南普洱鎮總兵高德元坐玩寇殃民處斬。己酉,詔停本年秋決。

二月庚戌朔,日有食之。己未,見葡使白朗穀於乾清宮。丙寅,利津決口合龍。

三月庚辰朔,見德使穆默等於乾清宮。癸未,御史蔣式瑆以疏劾奕劻語無根據,責還本官。戊子,下王照於獄。庚寅,免榆林等州縣逋課。丁未,張德彞與英訂保工條約成。

夏四月辛亥,見德親王阿拉拜爾、公使穆默於乾清宮。乙亥,蘇元春戍新疆。是月,免鄧川上年災糧,新化被蛟,呼蘭、綏化等屬被兵逋賦。

五月辛巳,命道員袁大化辦理安徽礦務。乙酉,熱河行宮災。丙戌,懿旨特赦戊戌黨籍,除康有為、梁啟超、孫文外,褫職者復原銜,通緝、監禁、編管者釋免之。戊戌,廣西叛勇陷柳城,斬統領祖繩武於軍前。己亥,旌九世同居邢臺貢生範鳳儀。癸卯,賜劉春霖等二百七十三人進士及第出身有差。乙巳,懿旨,本年七旬壽節停筵宴,將軍、督撫等毋來京祝嘏,並免進獻。罷粵海、淮安關監督,江寧織造。

六月己酉,諭曰:「時艱民困,官吏壅蔽,下情不通。甚至州縣錢糧浮收中飽,以完作欠,百弊叢生,大負朝廷恤民之意。各督撫速將糧額幾何,實徵幾何,正耗收米或折色幾何,具列簡明表冊。此外有無規費,一一登明聲敘,毋飾毋漏,據實奏聞。」壬子,命鐵良往江南考求制造局廠,籌畫所宜,並察出入款目,及各司庫局所利弊。戊午,趣岑春煊赴桂、柳督師。癸亥,青海住牧盟長車琳端多布等,請藉年班齎貢物赴京祝嘏。懿旨嘉獎,仍卻之。癸酉,永定河決。丙子,河決利津薄莊。

秋七月戊寅,見比使葛飛業於乾清宮。罷福建水師提督,歸並於陸路提督,移駐廈門。甲申,永定河北下汛復決。戊子,發內帑十萬賑四川水旱災。壬辰,英兵入藏境,達賴逃,褫其名號,命班禪額爾德尼攝之。甲午,甘肅黃河決,皋蘭被災,命崧蕃賑濟。乙未,停九江進瓷器。丙申,命李興銳署兩江總督兼南洋大臣。是月,賞湯壽潛四品卿銜,督辦浙江鐵路。

八月丁未朔,裁並內務府司員。癸亥,賞唐紹儀副都統銜,往西藏查辦事件。辛未,見義使嘎尼納於仁壽殿。癸酉,見墨使酈華於乾清宮。

九月丙子朔,見英使薩道義於乾清宮。癸未,敬信以疾免。己亥,李興銳卒,命周馥署兩江總督兼南洋大臣。以英兵入藏,達賴求救,命德麟安撫之。英兵旋退。敕唐紹儀為議約全權大臣。癸卯,改湖北糧道為施鶴兵備道。

是秋,免吉林被兵、雲南水旱兵災逋賦,武威、金州額賦。賑雲南、順天、福建、甘肅、江西水災,山西、浙江、廣東等處災。

冬十月丙午,呂海寰續訂中葡商約成。以裕德為體仁閣大學士,世續協辦大學士。庚戌,見奧、美、德、俄、比諸使齊幹等於皇極殿。永定河決口合龍。壬子,上奉皇太后御仁壽殿,賜近支宗籓等宴,率王、貝勒、貝子、公等進舞。甲寅,皇太后聖壽節,上詣排雲殿進表賀。辛酉,見英、日、法、韓諸使薩道義等於皇極殿。丙寅,懿旨禁各省藉新政巧立名目,苛細私捐。一切學堂工藝有關教養者,當官為勸導,紳民自籌,毋滋苛擾。除浙江墮民籍,準入學堂,畢業者予出身。

十一月乙亥朔,命廕昌仍充出使德國大臣,曾廣銓充出使韓國大臣。四川打箭爐地震。丁丑,見義使巴樂禮於乾清宮。壬午,廣西匪首陸亞發伏誅。戊子,定新軍官制。甲辰,諭增祺賑撫東三省難民。

十二月戊申,見義使巴樂禮、荷使希特斯、葡使阿梅達等於皇極殿。甲寅,裁江安糧道,改江南鹽道為鹽糧道。丁巳,發內帑三十萬賑奉天難民。壬戌,直隸始行公債票。丙寅,罷漕運總督,置江淮巡撫。丁卯,立貴胄學堂。戊辰,置黑龍江巡道兼按察使銜,蘭綏海兵備道,呼蘭、綏化二府。辛未,修四川都江堰。

是冬,裁湖北、雲南巡撫,湖南、陜西糧道。免石屏、趙州秋糧,陳留等州縣逋賦,朝邑被水額賦。

三十一年乙巳春正月丁丑,見德、英、日本、法、荷、比、義、日、葡、墨、美、韓、奧諸使於乾清宮。達賴喇嘛請於庫倫建廟諷經,不許。命仍還藏,善撫眾生。癸巳,鐵良言察閱諸省營伍,湖北陸軍為最優,詔嘉獎。江南各軍統領懲罰有差。命唐紹儀充出使英國大臣。

二月乙巳,懿旨發內帑三十萬撫恤東三省難民。庚戌,命長庚、徐世昌考驗改編三鎮新軍。丙寅,景陵隆恩殿災。庚午,見美使康格於海晏堂。壬申,賑阿拉善游牧。癸酉,免陜西前歲逋糧。

三月乙亥,奉天饑。俄兵入長春,據之。丙子,巴塘番人焚毀法國教堂,駐藏幫辦鳳全剿捕,遇伏死。飭四川提督馬維騏剿之。命柯逢時管理八省土膏統捐事宜。丁丑,見德親王福禮留伯、公使穆默於乾清宮。己卯,詔督撫舉堪勝提鎮官者。己丑,雲南省城開商埠。庚寅,罷新置江淮巡撫,改淮揚總兵為江北提督。癸巳,諭更定法律。死罪至斬決止,除凌遲、梟首、戮尸等刑。斬、絞、監候者以次遞減。緣坐各條,除知情外,餘悉寬免。刺字諸例並除之。甲午,以禁止刑訊,變通笞、杖,清查監獄羈所,諭督撫實力奉行。乙未,犍為匪徒作亂,官軍剿平之。丙申,命周馥往江北籌畫吏治、海防、河工、捕務。

夏四月甲辰,以俄艦至南洋,諭所在預防,並禁商人運煤接濟。更定竊盜條款。凡應擬笞、杖者改罰工作。乙巳,諭各省府州縣立罪犯習藝所。丙午,賞劉永慶侍郎銜,署江北提督,鎮、道以下歸節制。丁未,裁廣東糧道,置廉欽兵備道。己酉,命程德全署黑龍江將軍。壬子,德兵艦突至海州測量,飭嚴詰。

五月丁亥,見日使賈思理、美使柔克義於乾清宮。癸巳,見墨使胡爾達於皇極殿。庚子,王文韶罷軍機大臣,命徐世昌在軍機大臣上學習行走,兼政務處大臣,鐵良、徐世昌會辦練兵事。

六月丙午,見俄使璞科第於仁壽殿。免中牟等州縣逋賦。甲寅,予考試留學生金邦平等進士舉人出身有差。命載澤、戴鴻慈、徐世昌、端方往東西洋各國考察政治。戊午,詔置盛京三陵守護大臣。裁盛京戶、禮、兵、刑、工五部侍郎。己未,以世續為體仁閣大學士,那桐協辦大學士。癸亥,裁廣東巡撫。庚午,黔匪陷都勻之四寨,官軍克復之。

七月丙子,罷御史巡視五城及街道,改練勇為巡捕。乙酉,續派紹英為出洋考察政治大臣。己丑,以巴塘兵事,開實官捐一年。丙申,賞廷傑侍郎,往奉天辦墾荒事務。常德、湘潭開商埠。丁酉,命鐵良在軍機大臣上學習行走,尋兼政務處大臣。

八月壬寅,諭:「各省工商抵制美約,既礙邦交,亦損商務。疆吏當剴切開導,以時稽察之。」甲辰,詔廢科舉。丙午,裁奉天府尹、府丞,改置東三省學政。命劉式訓充出使法日大臣,黃誥充出使義國大臣,周榮曜充出使比國大臣。榮曜旋罷,改任李盛鐸。丁未,免奉天北路被兵額賦。辛亥,發內帑三萬於江蘇備急賑。癸丑,詔各省學政專司考校學堂,嗣後學政事宜,歸學務大臣考核。戊午,新疆巡撫潘效蘇坐侵款褫職,戍軍臺。己未,命袁世凱、鐵良校閱新軍秋操。壬戌,命汪大燮充出使英國大臣,楊晟充出使德國大臣,李經邁充出使奧國大臣。甲子,開海州商埠。乙丑,改命李經方為商約大臣。丁卯,載澤等啟行,甫登車,有人猝擲炸彈。事上,詔嚴捕重懲。己巳,巴塘亂平,匪首喇嘛阿澤、隆本郎吉等伏誅。

九月丙子,以三品京堂周榮曜舊充關書,侵盜鉅帑,褫職逮治,籍其貲。庚辰,初置巡警部,以徐世昌為尚書。庚寅,北新倉火。辛卯,論肅清廣西功,晉岑春煊太子少保銜,李經羲優獎。丙申,見德使穆默於勤政殿。戊戌,命尚其亨、李盛鐸會同載澤等赴各國考察政治。

是秋,賑貴州、雲南各屬水災,太康風災,鎮番暨巴燕戎格雹災風災。

冬十月癸卯,見日本公使內田康哉等於勤政殿。置吉林哈爾濱道。丙辰,蘆漢鐵路成。英兵入藏,索賠款一百二十餘萬。諭國家代給,以恤番艱。壬戌,訂鑄造銀幣及行用章程。乙丑,以陸徵祥充出使荷國大臣,兼理海牙和平會事。戊辰,置考察政治館,擇各國政法宜於中國治體者,斟酌損益,纂訂成書,取旨裁定。詔:「近有不逞之徒,造為革命排滿之說,假借黨派,陰行叛逆。各疆臣應嚴禁密緝。首從各犯,論如謀逆例。」

十一月庚午朔,陜、洛會匪平。辛未,裕德卒。丙子,罷駐韓使臣,改置總領事。己卯,詔置學部,以國子監歸並之,調榮慶為尚書。乙未,中日新約成。

十二月辛亥,授那桐體仁閣大學士,榮慶協辦大學士。癸亥,置京師內外城巡警總。罷工巡局。命徐世昌、鐵良並為軍機大臣。是月,免盛京各旗、陜西各屬被兵逋賦,安州積澇、韓城水沖地租。

是冬,賑會澤潦災,荊州水災,英吉沙爾水災雹災。

三十二年丙午春正月丙子,緩布特哈貢貂。丁丑,見德、英、法、美、日本、荷、義、俄、奧、比、葡、墨諸使穆默等於乾清宮。丁亥,漳浦匪首張嬰伏誅。壬辰,徐郙以察典罷。甲午,命瞿鴻禨協辦大學士。

二月戊辰,詔各省保護教堂及外人身家。乙丑,見德使穆默等於勤政殿。是月,頒帑十萬助賑日本災。

三月戊辰朔,以忠君、尊孔、尚公、尚武、尚實五大綱為教育宗旨,宣詔天下。庚午,罷選八旗秀女。丙子,命汪大燮往賀日君婚禮。丙戌,開江蘇通州商埠。丁酉,美國舊金山地震,頒帑十萬賑華民。是月,奧使顧新斯基、義使巴樂禮等、德使穆默等、法使呂班先後覲見。

是春,免浙江仁和等場與各州縣,杭、嚴、衢三所灶課及荒地山塘雜課,雲南、湖南、新疆災糧,陜西逋賦。

夏四月戊戌朔,命陸徵祥往瑞士議紅十字會公約。己亥,裁各省學政,改置提學使。庚子,見日本公使內田康哉於勤政殿。癸丑,命鐵良充督辦稅務大臣,唐紹儀副之。丁巳,發湖南庫帑十萬賑水災。

閏四月丙戌,以暘雨失時,偏災屢告,懿旨飭君臣上下交儆。戊子,唐炯以衰疾解云南礦務。

五月戊戌,發庫帑五萬賑廣東水澇災。癸卯,河南沁河溢,賑災民。是月,見法使巴思德、義親王費爾迪安德等於乾清宮。

六月丁卯,德國減直隸駐兵,歸我廊坊、楊村、北戴河、秦王島、山海關地。庚辰,沅陵匪首覃加位伏誅。

是夏,免浪穹舊逋,莎車復荒額賦,甘肅、雲南被災逋賦。賑武陟水災,朝陽火災。

秋七月戊戌,置川滇邊務大臣,以趙爾豐任之,賞侍郎銜。沁河決口合龍。庚子,江蘇水陸各營旗防軍改編巡防隊。辛丑,考察政治大臣載澤等還京,上封事。命醇親王、軍機政務處大臣、大學士、北洋大臣公閱,取進止。乙巳,奉天開商埠大東溝,置海關,以東邊道兼監督。戊申,諭曰:「載澤等陳奏,謂國勢不振,由上下相暌,內外隔閡。官不知所以保民,民不知所以衛國。而各國所由富強,在實行憲法,取決公論。時處今日,惟有仿行憲政,大權統於朝廷,庶政公諸輿論。預備立憲基礎,內外臣工切實振興。俟數年後規模粗具,參用各國成法,再定期限實行。」己酉,諭立憲預備,須先釐定官制,命大臣編纂,奕劻、孫家鼐、瞿鴻禨總司核定,取旨遵行。調端方為兩江總督兼南洋大臣。甲子,發江蘇庫儲十萬賑徐、海、淮西水災。

八月丁亥,除臨川水沖地額賦。庚寅,見日本王爵博恭、公使林權助於仁壽殿。是月,賑安徽水災,廣東風災,湖州澇災。

九月癸卯,見各國公使等於仁壽殿。丙午,賜游學畢業陳錦濤等各科進士、舉人出身有差。甲寅,詔更定官制。內閣、軍機處、外務、吏、禮、學部、宗人府、翰林院等仍舊。改巡警部為民政部,戶部為度支部,兵部為陸軍部,刑部為法部,工部並入商部為農工商部,理籓院為理籓部。各設尚書一人,侍郎二人,不分滿、漢。都察院都御史一人,副都御史二人。改六科給事中為給事中,大理寺為大理院。增設郵傳部、海軍部、軍諮府、資政院、審計院。以財政處歸度支部,太常、光祿、鴻臚三寺歸禮部。太僕寺、練兵處歸陸軍部。各部尚書俱充參預政務處大臣。命世續為軍機大臣,林紹年軍機大臣上學習行走,鹿傳霖、榮慶、徐世昌、鐵良並罷軍機,專理部務。乙卯,發廣東庫儲十萬賑香港及潮、高、雷、欽、廉屬風災。丁巳,改政務處為會議政務處。戊午,命曾廣銓以三品京堂充出使德國大臣。

冬十月癸酉,皇太后聖壽節,停筵宴。癸未,見英使硃邇典、比使柯霓雅於乾清宮。乙酉,裁並廣東陸路、水路提督為廣東提督。丁亥,見日本公使林權助等於勤政殿。戊子,瀏陽、醴陵匪巫王永求、陳顯龍倡亂,官軍擒斬之。己丑,撥漕折三十萬備賑江蘇。辛卯,立官報局於京師。

十一月己亥,留廣東京餉十萬備賑。壬寅,免廣西銻礦出井稅。甲辰,撥陜西官帑八萬助賑江蘇。戊申,詔升孔子為大祀,所司議典禮以聞。癸丑,詔各省議幣制。丁卯,建曲阜學堂,發內帑十萬濟工。是月,見墨使胡爾達於勤政殿,德使雷克司、法使巴思德、英使硃邇典於乾清宮。

十二月癸亥朔,日有食之。丁卯,加京官養廉。甲戌,改駐各國公使為二品實官。

是冬,賑普寧、趙州、羅平、師宗災,江寧、揚州水災。免灤平、安州澇災糧賦,永城額賦,陜西咸寧等處逋賦,永平、太和、昆明災地欠糧。

三十三年丁未春正月甲辰,見各國公使於乾清宮。庚戌,裁各部小京官。

二月甲子,有泰以貽誤藏事褫職,謫戍軍臺。壬申,留蘇漕十五萬備賑。

三月丙申,見日本公使林權助等於勤政殿。戊戌,長春、哈爾濱闢商埠。己亥,改盛京將軍為東三省總督,裁吉林、黑龍江將軍,改置奉、吉、黑三巡撫,授徐世昌欽差大臣,為東三省總督。壬寅,命府尹孫寶琦充出使德國大臣。壬子,命天津道梁敦彥充出使美墨祕古大臣。丙申,命陸徵祥充保和會專使大臣,李經方充出使英國大臣,錢恂充出使荷國大臣。丁巳,昆岡卒。

是春,免中衛被水及榆林等屬逋賦,雲南旱災等州縣逋糧及額賦。

四月甲子,裁各省民壯捕役,改設巡警。綏來地震。乙丑,御史趙啟霖坐污衊親貴褫職。辛未,更定東三省官制,奉天、吉林、黑龍江各設行省公署,以總督為長官,巡撫為次官,置左右參贊,分領承宣、諮議兩,分設交涉、旗務、民政、提學、度支、勸業、蒙務七司,各置司使,及提法使、督練處等官。己卯,祈雨。辛巳,以江北水災,嚴米糧出口禁。丁亥,定陸海軍官制,陸軍部設兩十司,軍諮處五司,海軍部六司。戊子,命衍聖公孔令貽稽察山東學務。

五月癸巳,巴塘等屬喇嘛脅河西蠻作亂,官軍討平之。乙未,命王士珍以侍郎銜署江北提督。丙申,西陵禁山火。丁酉,瞿鴻禨罷。己亥,授鹿傳霖軍機大臣。命醇親王直軍機。辛丑,王文詔罷,命張之洞協辦大學士。癸卯,崇禮卒。丁巳,改按察使為提法使,置巡警、勸業道,裁分守、分巡各道,酌留兵備道,及分設審判,備司法獨立,增易佐治員,備地方自治,期十五年內通行。戊午,詔:「憲法,官民均有責任,凡知所以預備之方、施行之序者,許各條舉,主者甄採以聞。」安徽候補道徐錫麟刺殺巡撫恩銘,錫麟捕得伏誅。

六月辛酉,命李家駒充出使日本大臣。丙寅,復御史趙啟霖官。壬申,自四月不雨至於是日雨。授張之洞體仁閣大學士,鹿傳霖協辦大學士。乙酉,停萬壽筵宴。永定河決。

是夏,免新化被水額賦,伊通被賊逋課,雲南旱災等州縣銀米。賑雲南饑及直隸水災。

秋七月辛卯,詔中外臣工議化除滿、漢。甲午,改考查政治館為憲政編查館。其軍機大臣、大學士、參預政務大臣會議事,於內閣行之。壬寅,懿旨遣楊士琦赴南洋各埠考察,獎勵華僑。免趙州、祿豐被災額賦。賑順天等屬災民,及瀏陽、邵陽蛟災。甲辰,詔以匪徒謀逆,往往假革命名詞,巧為煽誘。各督撫當設法解散。獲犯擬罪,分別叛逆、盜匪科論,被脅及家屬不知情者勿株連。命張廕棠為全權大臣,與英人議藏約。敬信卒。己酉,定限年編練陸軍三十六鎮。丙辰,命張之洞、袁世凱並為軍機大臣,以袁世凱為外務部尚書。丁巳,命楊士驤署直隸總督兼北洋大臣。戊午,李經邁以母病免,命雷補同充出使奧國大臣。己未,河決孟縣。

八月辛酉,上不豫,詔各省薦精通醫理者。命汪大燮使英國,達壽使日本,於式枚使德國,俱充考察憲政大臣。壬戌,置京師高等審判。己巳,置總檢察。庚申,立資政院,以貝子溥倫、孫家鼐為總裁。乙亥,命伍廷芳充出使美國大臣,薩廕圖充出使俄國大臣。己卯,詔以各省駐防習為游惰,命各將軍等授田督耕,歸農後,一切歸有司治理。庚辰,裁奉天驛站,設文報局。壬午,詔中外臣工研究君主立憲政體。諭定自治章程。甲辰,見德使雷克司、日使阿部守太郎於仁壽殿。諭神機營衛隊及官兵歸陸軍部管轄。

九月辛卯,詔議定滿、漢禮制、刑律,考定度量權衡畫一制度章程。是日,以煙習未除,敕責莊親王載功、睿親王魁斌、都御史陸寶忠、副都御史陳名侃解職,迅速戒斷。並諭內外文武,限三月凈盡,否即嚴懲。癸巳,命沈家本、俞廉三、英瑞充修訂法律大臣。己亥,命各省立諮議局,公舉議員,並籌設州、縣議事會。壬寅,日本以水災來告糴,輸江、皖、浙、鄂諸省米糧六十萬石濟之。甲辰,命各省立調查局,各部、院設統計處。予游學畢業生章宗元等進士、舉人出身有差。戊申,湖北按察使梁鼎芬言挽回時局,莫亟於禁賄賂,絕請託,劾奕劻、袁世凱等夤緣比附,貪私誤國。廷旨以有意沽名,斥之。是月,免雲南旱傷等州縣稅糧。賑懷寧等縣水災。

冬十月乙丑,命派孫家鼐、榮慶、陸潤庠、張英麟、唐景崇、寶熙、硃益籓進講經史及國朝掌故。永定河合龍。戊辰,皇太后聖壽節,停筵宴。壬申,見日使林權助等於勤政殿。丙戌,哲布尊巴丹胡圖克圖進方物。

十一月庚寅,廣西匪踞南關砲臺,責張鳴岐督剿,尋復之。戊申,嚴聚眾開會演說之禁。諭各省整頓學堂,增訂考核勸戒法。壬子,見俄使璞科第等於乾清宮。以內外臣工條議幣制,用兩用元,互有利害,諭各督撫體察籌議以聞。發帑五十萬濟廣西軍。

十二月戊午朔,復分置廣東陸路提督、水師提督。癸亥,裁吉林副都統,置交涉、民政、度支三司使暨提法使、勸業道。予進士館游學畢業學員楊兆麟等進敘有差。壬申,裁山東糧道,置巡警、勸業二道。甲戌,諭熱河圍場辦屯墾,裁駐防官兵。乙亥,命呂海寰充督辦津浦鐵路大臣。丙子,那桐兼督辦稅務大臣。辛巳,賞總稅務司赫德尚書銜。丙戌,再停布特哈貢貂一年。

是冬,免雲南被旱、直隸被潦暨陜西逋賦。賑雲南等屬蛟災,四川水災,廣東風災水災。

三十四年戊申春正月丁亥朔,授醇親王載灃為軍機大臣。庚寅,見各國公使於乾清宮。己亥,以京師銀價驟高,物直踴貴,發帑五十萬,命順天府尹貶價收錢,並令各省廠鑄當十銅元,定額外加鑄三成一文新錢,以資補救。甲寅,建蘭州黃河鐵橋。丙午,見奧使顧新斯基於勤政殿。是月,免雲南昆明等縣逋賦,浙江仁和等場灶課,湖南邵陽額賦。

二月戊午,祭大社、大稷,是後祀典不克親行,皆遣代。庚申,賞趙爾豐尚書銜,為駐藏大臣,仍兼邊務大臣。癸亥,詔增給滿大臣暨各旗官十成養廉,更定御前大臣以下等員津貼。丙寅,諭京、外清庶獄。甲戌,京師勸工陳列所災。丙子,諭以「禁煙議成,英人許分年減運,見已實行遞減。相約試行三年,限滿再為推減。轉瞬期至,其何以答友邦。民政、度支二部迅訂稽核章程,責成督撫飭屬將減種、減食,切實舉辦以聞」。丁丑,召達壽還,命李家駒為考察憲政大臣,胡惟德充出使日本大臣。壬午,黃誥罷,調錢恂為出使義國大臣,以陸徵祥為出使荷國大臣。

三月壬辰,命恭親王溥偉、鹿傳霖、景星、丁振鐸充禁煙大臣,立戒煙所,專司查驗。丙午,廓爾喀貢方物。甲寅,以湛深經術,賜湘潭舉人王闓運檢討。是月,免雲南被災新舊額賦。

夏四月丙辰,見日使林權助等於勤政殿。綏遠城將軍貽穀有罪褫職,逮下刑部獄,尋籍其家。命信勤充墾務大臣,兼署綏遠城將軍。丁巳,裁安徽安廬滁和道。己未,越匪陷河口,命劉春霖以三品京堂幫辦雲南邊防,前敵諸軍歸節制。戊辰,裁貴州糧道、貴西道。己巳,見各國公使希特斯等於仁壽殿,賜宴。庚辰,雲南官軍敗匪於田房,復四隘,旋克大小南溪及河口,發帑犒軍。

五月乙酉朔,滇匪平。丁亥,裁巴塘、里塘土司,置流官。壬辰,上疾復作,命直省薦精通醫理者。癸巳,錄中興功臣多隆阿、向榮、江忠源、羅澤南、駱秉章、張國樑、李續賓、彭玉麟、楊岳斌、鮑超、李孟群、程學啟、劉松山、馮子材等後,升敘有差。甲午,修曲阜孔子廟。癸卯,襄河決,颶風為災。庚戌,郎中曹元弼進所著禮經校釋,賞編修。癸丑,廣東大雨,東西北三江並溢,沖決圍堤。

六月丁巳,前祭酒王先謙進所著尚書孔傳參正、漢書補注、荀子集解、日本源流考,賞內閣學士銜。甲子,免廣西礦地出井、出口兩稅五年。庚午,予進士館游學畢業學員黎湛枝等進敘有差。甲戌,命張之洞兼督辦粵漢鐵路。乙亥,允達賴喇嘛入覲。丙子,以美國減收賠款,命唐紹儀充專使致謝,兼赴東西洋考察財政。議免釐加稅。乙卯,授楊士驤直隸總督兼北洋大臣。辛巳,法部主事陳景仁等請三年後開國會。詔以景仁倡率生事,褫職交管束。

是夏,免雲南水旱雹災,奉天水災荒地額賦。賑江蘇風雹災,湖北水災。發部帑五萬賑察哈爾蒙旗及兩翼牧群災,又部帑十萬賑廣東廣州、肇慶、陽江各屬水災。

秋七月壬辰,裁黑龍江愛琿等三副都統,增置愛琿道、呼倫貝爾道。丙申,釋蘇元春回。己亥,免鐵路公私稅三年。庚子,以各省設政聞社斂財結黨,陰擾治安,諭所在嚴禁。辛丑,修浙江海塘。癸卯,廣西巡勇叛變,戕殺統將,張人駿督剿之。丙午,命三品卿銜胡國廉總理瓊、崖墾礦事。庚戌,置雲南交涉使。是月,山東、安徽蝗。

八月甲寅朔,憲政編查館、資政院上憲法、議院、選舉各綱要,暨逐年籌備事宜。詔頒行京、外官署,依限舉辦,每六閱月,臚列成績以聞。辛未,命姜桂題總統武衛左軍。戊寅,見俄使廓索維慈、荷使希特斯等於仁壽殿。己卯,命廕昌、端方校閱秋操。庚辰,馬玉昆卒,晉二等輕車都尉。辛巳,命廕昌充出使德國大臣。壬午,遣御前大臣博迪蘇往保定迎勞達賴喇嘛。

九月癸未朔,予先儒顧炎武、王夫之、黃宗羲從祀文廟。乙酉,美軍艦游歷至廈門,遣貝勒毓朗、梁敦彥往勞問。己丑,開寧夏渠墾田。庚寅,達賴喇嘛至京,尋於仁壽殿覲見。癸巳,頒畫一幣制。丙申,允郵傳部請,試辦本國公債。戊戌,予進士館畢業陳云誥等敘進有差。庚子,見英使硃邇典等於仁壽殿。癸卯,予游學畢業陳振先等出身,進士館畢業葉光圻等敘進各有差。己酉,裁四川成釂龍茂道,增置巡警、勸業兩道。辛亥,詔以前籌備憲政事宜尚有未盡,諭各部院依前格式,各就職司所系,分期臚列奏明,交編查館覆核,取旨遵行。

是秋,免雲南會澤被水逋賦,楚雄等縣及湖南漵浦被水額糧。發帑十萬賑湖北、湖南災民。復賑甘肅災,廣東風災水災,廣西、浙江、黑龍江、福建水災。

冬十月甲寅,見日使伊集院彥吉於勤政殿。廣州、肇慶等屬颶風為災,諭施急賑。戊午,賜達賴宴於紫光閣。壬戌,皇太后聖壽節,停筵宴。達賴祝嘏,進方物,懿旨加封誠順贊化西天大善自在佛,歲賜廩餼萬金,遣歸藏。

壬申,上疾甚。懿旨,醇親王載灃之子溥儀在宮中教養,復命載灃監國為攝政王。癸酉,上疾大漸,崩於瀛臺涵元殿,年三十有八。遺詔攝政王載灃子溥儀入承大統,為嗣皇帝。皇太后懿旨,命嗣皇帝承繼穆宗為嗣,兼承大行皇帝之祧。宣統元年正月己酉,上尊謚曰同天崇運大中至正經文緯武仁孝睿智端儉寬勤景皇帝,廟號德宗,葬崇陵。

論曰:德宗親政之時,春秋方富,抱大有為之志,欲張撻伐,以湔國恥。已而師徒撓敗,割地輸平,遂引新進小臣,銳意更張,為發奮自強之計。然功名之士,險躁自矜,忘投鼠之忌,而弗恤其罔濟,言之可為於邑。洎垂廉再出,韜晦瀛臺。外侮之來,釁自內作。卒使八國連兵,六龍西狩。庚子以後,怫鬱摧傷,奄致殂落,而國運亦因此而傾矣。嗚呼,豈非天哉!


\end{pinyinscope}