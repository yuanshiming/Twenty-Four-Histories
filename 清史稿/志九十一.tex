\article{志九十一}

\begin{pinyinscope}
○職官三外官

順天府奉天府總督巡撫學政布政使按察使鹽運使

道府州縣儒學巡檢驛丞庫倉稅課河泊各大使徬官

醫學陰陽學僧綱司道紀司

順天府兼管府尹事大臣,漢大學士、尚書、侍郎內特簡。尹,正三品。丞,正四品。俱各一人。其屬:治中,正五品。通判,正六品。經歷司經歷,從七品。照磨所照磨,司獄司司獄,並從九品。俱各一人,並漢員。儒學教授,正七品。訓導,從八品。滿、漢各一人。所轄四路,正五品。二十州、縣,正七品。各一人。在京者大興、宛平二縣知縣各一人,正六品。縣丞正七品。四人,大興一人,宛平三人。巡檢從九品。七人,大興三人,宛平四人。典史,閘官,崇文門副使,俱未入流。副使後隸監督。各一人。

尹掌清肅邦畿,布治四路,帥京縣頒政令條教。歲立春,迎春東郊。天子耕耤,具耒耜絲鞭,奉青箱播種,禮畢,率庶人終畝。田賦出納,以時勾稽,上其要於戶部。治鄉飲典禮。鄉試充監臨官。丞掌學校政令,鄉試充提調官。治中掌貳府事,紀綱眾務,兼鄉會試場務。通判掌主牙稅,平禁爭偽。經歷、照磨掌出納文書。司獄掌罪囚籍錄。儒學掌京畿黌序,文武生月課其藝射,不帥教者戒飭之,三歲報優劣於學政。大興、宛平二縣各掌其縣之政令,與五城兵馬司分壤而治,品秩服章視外縣加一等。

初,世祖奠鼎燕京,建順天府,置尹一人,丞一人,兼提督學政銜。乾隆五十八年停。別置學政。丞止申送童生。治中三人,通判三人,順治六年留管糧一人。省馬政、軍匠各一人。經歷、照磨、司獄,各一人,推官、知事、並從六品,檢校、從九品。以上三員俱康熙六年省。遞運所大使、康熙三十八年省。庫大使、康熙三十九年省。張家灣宣課司大使,康熙四十年省。以上三員俱未入流。各一人。儒學漢教授一人,訓導六人。順治二年省四人。康熙四年俱省,五年復置一人。京衛武學漢教授一人,訓導二人。順治二年省。康熙十五年復置一人。轄大興、宛平二縣,知縣各一人,縣丞各一人,雍正四年增宛平管河一人。嘉慶十三年復增宛平管河一人。巡檢七人,主簿、順治三年省。典史、閘官,詳內務府。各一人。順治六年,省治中二人。康熙十五年,始以昌平等十九州、縣來隸。二十七年,置東、西、南、北四路同知。雍正元年,特簡大臣領府事,號兼尹。三年,改京衛武學為府武學。明年,省武學教授、訓導官;增府儒學教授、訓導,滿洲各一人。乾隆八年,定為二十四州、縣隸府。嘉慶十八年,定所屬官吏歸尹考察。光緒元年,省治中。別設驛巡道。宣統二年,罷兼尹。

奉天府兼管府事大臣一人。盛京五部侍郎內特簡,後歸將軍兼管。尹,滿洲一人;丞,漢一人。其屬:治中,圍場通判,庫大使,經歷,司獄,巡檢兼司獄,府學教授,各一人。所轄海防同知,軍糧同知,各一人。承德縣知縣,典史,各一人。

尹掌留都治化與其禁令,小事決之,大事以聞。丞掌主學校,兼稽宗室、覺羅官學、義學。治中以次各官所掌視順天府。

初建盛京,順治十年,設遼陽府。十四年,更名奉天府,置尹一人,經歷、教授、訓導,康熙三年省。各一人。康熙二年,置丞一人;治中、通判、推官,六年省。各一人。設承德縣附郭,置知縣、典史,各一人。巨流河巡檢一人。乾隆四十二年省。七年,增府司獄一人。二十八年,定府丞主奉天考試事。乾隆二十七年,詔府尹受將軍節度。明年,增興京理事通判一人。光緒二年省。三十年,始以侍郎為兼尹,著為令。光緒二年,省治中,別設驛巡道。改命將軍兼管;加兼尹總督銜,府尹二品銜,以兵部侍郎、右副都御史行巡撫事。三十一年,改行省,罷尹丞,置知府。宣統元年,省教佐各官。越明年,省承德縣。

總督從一品。掌釐治軍民,綜制文武,察舉官吏,修飭封疆。標下有副將、參將等官。巡撫從二品。掌宣布德意,撫安齊民,修明政刑,興革利弊,考覈群吏,會總督以詔廢置。標下有參將、游擊等官。其三年大比充監臨官,武科充主試官,督、撫同。

初沿明制,督、撫系右都御史、右副都御史、右僉都御史銜,無定員。順治十年,諭會推督、撫,不拘品秩,擇賢能者具題。康熙元年,停巡撫提督軍務加工部銜。不置總督省分,兼轄副將以下等官。十二年復故,並設撫標左、右二營。三十一年,定總督加銜制。由各部左、右侍郎授者,改兵部左、右侍郎;由巡撫授者,升兵部右侍郎兼都察院右副都御史。乾隆十三年,定大學士兼管總督者仍帶原銜。明年,改授右都御史銜,其兵部尚書銜由吏部疏請定奪。嘉慶十四年,定以二品頂戴授者兼兵部侍郎銜,俟升品秩再加尚書銜。光緒三十二年,更名陸軍部尚書銜。宣統二年停。七年,定山陜督、撫專用滿員。雍正元年,定巡撫加銜制。由侍郎授者,改兵部右侍郎兼右副都御史銜;由學士、副都御史及卿員、布政使等官授者,俱為右副都御史;由左僉都御史或四品京堂、按察使等官授者,俱為右僉都御史。乾隆十四年,定巡撫不由侍郎授者,俱兼右副都御史;其兵部侍郎銜,疏請如總督。光緒三十二年,更名陸軍部侍郎銜。宣統二年停。時西安有同署巡撫者,山東、山西並有協辦巡撫之目,非制也。是歲,諭山陜督、撫參用蒙古、漢軍、漢人,纂為令甲。乾隆十八年,以漕運、河道總督無地方責,授銜視巡撫。嘉慶十二年,定由尚書授者,應否兼兵部尚書銜,疏請如總督。光緒二十四年,加總理各國事務衙門大臣銜,尋罷。三十二年,定闢除掾屬、分曹治事制。條為十科:曰交涉、曰吏、曰民、曰度支、曰禮、曰學、曰軍政、曰法、曰農工商、曰郵傳,各置參事、秘書,是為幕職。宣統二年,充會辦鹽政大臣兼職,尋亦罷。

初,河南、山東、山西等省專置巡撫,無統轄營伍權,以提督為兼銜。直隸、四川、甘肅等省專置總督,吏治歸其考覈,以巡撫為兼銜。而巡撫例受總督節度,浸至同城巡撫僅守虛名。即分省者,軍政民事亦聽總督主裁。文宗蒞政,命浙江、安徽、江西、陜西、湖南、廣西、貴州各巡撫節制鎮、協武職;總督兼轄省分,由巡撫署考會題,校閱防剿,定為專責,職權漸崇。光緒季年,裁同城巡撫,其分省者,權幾與總督埒,所謂兼轄,奉行文書已耳。宣統間,軍政、鹽政厚集中央,督、撫權削矣。

總督東三省等處地方兼管三省將軍、奉天巡撫事一人。康熙元年置將軍。詳武職。光緒二年,兼管兵、刑二部及府尹,以兵部尚書、都察院右都御史銜行總督事。三十二年,建行省,改將軍曰總督,授為欽差大臣,隨時分駐三省行臺。宣統二年,兼奉天巡撫事。初建行省,置巡撫一人,至是省。

總督直隸等處地方提督軍務、糧饟、管理河道兼巡撫事一人。順治五年,置直隸山東河南三省總督,駐大名。十五年,改為直隸巡撫。十七年,徙真定。明年,復置總督於大名。康熙三年,仍為三省總督。八年省,移巡撫還駐保定。五十四年,加巡撫以總督銜,不為例。雍正元年,詔嘉李維鈞勤慎,特授總督,自是為永制。四年,以禮部右侍郎協理總督,不為常目。乾隆十四年,令兼河道。二十八年,詔依四川例,兼管巡撫事。咸豐三年,兼管長蘆鹽政。同治九年,加三口通商事務,授為北洋通商大臣,駐天津。冬令封河,還駐保定。初置有宣大總督,順天、保定、宣府三巡撫。順治八年省宣府巡撫,以宣大總督兼其事。十三年省宣大總督,令順天巡撫兼之。十八年省順天巡撫,歸保定巡撫兼管。後亦省。

總督兩江等處地方提督軍務、糧饟、操江、統轄南河事務一人。順治二年,以內閣大學士洪承疇總督軍務,招撫江南各省。尋改應天府為江寧,罷南直隸省府尹。四年,置江南江西河南三省總督,駐江寧。九年,徙南昌,時號江西總督;已,復駐江寧。十八年,江南、江西分置總督。康熙元年,加江南總督操江事務。初置鳳廬巡撫,駐淮安,以操江管巡撫事領之。六年省歸漕督。至是始來隸。四年,復並為一。十三年,復分置。二十一年仍合。尋定名兩江總督。雍正元年,以綜治江蘇、安徽、江西三省,加兵部尚書兼都察院右都御史銜。道光十一年,兼兩淮鹽政。同治五年,加五口通商事務,授為南洋通商大臣,與北洋遙峙焉。

總督陜甘等處地方提督軍務、糧饟、管理茶馬兼巡撫事一人。順治元年,置陜西總督,駐固原,兼轄四川。十四年,徙漢中。康熙三年,更名山陜總督,兼轄山西,還駐西安。十四年,改為陜甘總督。時山西別置總督。十九年,仍改陜甘為山陜,省山西總督入之。轄四川如故。雍正元年,以綜治陜西、甘肅、四川三省,加兵部尚書兼都察院右都御史銜。三年,授兵部尚書岳鍾琪為總督。先是定為滿缺,參用漢人自此始。九年,諭仍專轄陜、甘。十四年,復轄四川,更名川陜甘總督。乾隆十三年,西陲用兵,仍置陜西總督。十九年,省甘肅巡撫,移陜甘總督駐蘭州,兼甘肅巡撫事。二十四年,別置甘肅總督,兼轄陜西,駐肅州;移川陜總督駐四川。尋復定名陜甘總督,還駐蘭州,仍兼巡撫事。光緒八年,新疆建行省,復兼轄之。

總督閩浙等處地方提督軍務、糧饟兼巡撫事一人。順治二年,置福建總督,駐福州,兼轄浙江。五年,更名浙閩總督,徙衢州,兼轄福建。十五年,兩省分置總督,福建總督駐漳州,浙江總督駐溫州。康熙十一年,移福建總督駐福州。明年,省浙江總督。二十六年,改福建總督為福建浙江總督。雍正五年,特授李衛總督浙江,整飭軍政吏治,並兼巡撫事;郝玉麟以浙閩總督專轄福建。十二年,復省浙江總督,仍合為一。乾隆元年,詔依李衛例,特授嵇曾筠為浙江總督,郝玉麟仍專轄福建。三年,嵇曾筠入閣,郝玉麟仍總督閩、浙如故。閩、浙或分或合,至是始為永制。光緒十一年,省福建巡撫,並兼巡撫事。

總督湖北湖南等處地方提督軍務、糧饟兼巡撫事一人。順治元年,置湖廣總督,駐武昌。康熙七年省,九年復置。十九年,改川湖總督復為湖廣總督,還駐武昌。二十六年,更名湖北湖南總督。光緒三十年,兼湖北巡撫事。

總督四川等處地方提督軍務、糧饟兼巡撫事一人。順治元年,置四川巡撫,駐成都,不置總督。十年,以川省兵馬錢糧皆從陜西調發,詔陜西總督孟喬芳兼督四川。十四年,停陜督兼轄,專置四川總督,駐重慶。康熙七年,更名川湖總督,徙荊州。九年,還駐重慶。十三年,四川省會別置總督一人。十九年,省隸陜甘總督,其川湖總督省歸湖廣總督兼理。雍正九年復置,駐成都。十三年又省。乾隆十三年,以金川用兵,始定為專缺,兼管巡撫事。二十四年,兼轄陜西,尋停兼轄。宣統元年,以將軍所轄松潘、建昌二鎮,阜和協所屬各營,建昌、松茂二道府、、州、縣、改隸之。

總督兩廣等處地方提督軍務、糧饟兼巡撫事一人。順治元年,置廣東總督,駐廣州,兼轄廣西。十二年,徙梧州。康熙二年,別置廣西總督,移廣東總督駐廉州。三年,復並為一,駐肇慶。雍正元年,復分置。明年仍合。七年,以苗患,令云貴總督兼轄廣西。十二年,仍隸廣東。光緒三十一年,兼廣東巡撫事。

總督云貴等處地方提督軍務、糧饟兼巡撫事一人。順治十六年,置經略,尋改總督,兩省互駐。康熙元年,分置雲南總督,駐曲靖;貴州總督,駐安順。三年,復並為一,徙貴陽。十二年,仍分置,尋復故。二十六年,徙雲南府。雍正十年,上嘉鄂爾泰才,以雲貴總督兼制廣西,給三省總督印。十二年,仍轄兩省,以經略苗疆,授張廣泗為貴州總督兼巡撫事,尹繼善為雲南總督,專轄雲南。十二年復故。光緒三十一年,兼雲南巡撫事。

總督漕運一人。掌治漕輓,以時稽覈催趲,綜其政令。標下官同總督。順治元年,遣御史巡漕,尋置總督,駐淮安。四年,以滿洲侍郎一人襄治漕務。八年省。十三年復置,十八年又省。六年,兼鳳廬巡撫事。十六年,停兼職。康熙二十一年,定糧艘過淮,總漕隨運述職。咸豐十年,令節制江北鎮、道各官。光緒三十年,以淮、徐盜警,改置巡撫。明年省。

河道總督,江南一人,山東河南一人。直隸河道以總督兼理。掌治河渠,以時疏濬堤防,綜其政令。營制視漕督。順治元年,置總河,駐濟寧。康熙十六年,移駐清江浦。二十七年,還駐濟寧,令協理侍郎開音布等駐其地。三十一年,總河並駐之。三十九年,省協理。四十四年,兼理山東河道。雍正二年,置副總河,駐武陟,專理北河。七年,改總河為總督江南河道,駐清江浦,副總河為總督河南山東河道,駐濟寧,分管南北兩河。八年,增置直隸正、副總河,為河道水利總督,駐天津。自是北河、南河、東河為三督。九年,置北河副總河,駐固安,並置東河副總河,移南河副總河駐徐州。十二年,移東河總督駐兗州。乾隆二年,省副總河。厥後省置無恆。十四年,省直隸河道總督。咸豐八年,省南河河道總督。光緒二十四年,省東河河道總督,尋復置。二十八年又省,河務無專官矣。

巡撫江蘇等處地方提督軍務兼理糧饟一人。順治元年,置江南巡撫,駐蘇州,轄江寧、蘇州、松江、常州、鎮江五府。十八年,江南分省,更名江蘇巡撫。

巡撫安徽等處地方提督軍務、節制各鎮兼理糧饟一人。順治元年,置操江兼巡撫安徽徽、寧、池、太、廣,駐安慶。康熙元年,省操江,所部十二營改隸總督,始置安徽巡撫。嘉慶八年,以距壽春鎮窵遠,加提督銜。

巡撫山東等處地方提督軍務、糧饟兼理營田一人。順治元年置,駐濟寧。時海防巡撫駐登州,九年省。康熙四十四年,管理山東河道。五十三年,兼臨清關務。乾隆八年,依山西、河南例,加提督銜。

巡撫山西等處地方提督軍務兼理糧饟一人。順治元年置巡撫,駐太原,提督雁門等關。雍正十二年,管理提督事務,通省武弁受節度。

巡撫河南等處地方提督軍務、糧饟兼理河道、屯田一人。順治元年置,駐開封。康熙十七年,定管理河南歲修工程。雍正四年,加總督銜,不為例。尋省。十三年復置。乾隆五年,以盜警,加提督銜。

巡撫陜西等處地方提督軍務、節制各鎮兼理糧饟一人。順治元年置,駐西安,定為滿缺。雍正九年,以兵部尚書史貽直署巡撫,參用漢人自此始。

巡撫新疆等處地方提督軍務兼理糧饟一人。順治元年,置甘肅巡撫,駐甘州衛。雍正二年改衛為府。五年,徙蘭州。康熙元年,移駐涼州衛。後亦改府。五年,還駐蘭州,尋改駐鞏昌。十九年,仍回蘭州。四十四年,兼管茶馬事。乾隆十九年省,移陜甘總督來駐,兼巡撫事。光緒十年,新疆建行省,置甘肅新疆巡撫,駐烏魯木齊。初置有延綏巡撫、寧夏巡撫各一人,康熙間俱省。

巡撫浙江等處地方提督軍務、節制水陸各鎮兼理糧饟一人。順治元年置,駐杭州。雍正五年,改總督。十三年,仍為巡撫,兼總督銜。乾隆元年,復置總督。三年復故。

巡撫江西等處地方提督軍務、節制各鎮兼理糧饟一人。順治元年置,駐南昌,轄十一府。康熙三年,兼轄南安、贛州。初置南贛巡撫,至是省入。乾隆十四年,加提督銜。

巡撫湖南等處地方提督軍務、節制各鎮兼理糧饟一人。順治元年,置偏沅巡撫,駐偏橋鎮。同時置撫治鄖陽都御史,駐沅州,以控湘、蜀、豫、晉之交,十八年省。康熙十五年,以盜警復置。十九年又省。康熙三年,湖南分省,移駐長沙。雍正二年,更名湖南巡撫,令節制各鎮。

巡撫湖北等處地方提督軍務兼理糧饟一人。順治元年,置湖廣巡撫,駐武昌。康熙三年,更名湖北巡撫。光緒二十四年省,尋復置。三十二年又省。

巡撫廣東等處地方提督軍務兼理糧饟一人。順治元年置,駐廣州。雍正二年,兼太平關務。光緒二十四年省,尋復置。三十一年,以廣西軍務平,又省。

巡撫廣西等處地方提督軍務兼理糧饟加節制通省兵馬銜一人。順治元年置,駐桂林。六年,省鳳陽巡撫標兵來隸。雍正九年,令節制通省兵馬。

巡撫雲南等處提督軍務兼理糧饟一人。順治元年置,駐雲南府。雍正四年,命江蘇布政使鄂爾泰為巡撫,兼總督事。十年,升總督,兼巡撫事。張廣泗繼之,亦兼巡撫。乾隆十二年,始授圖爾炳阿為巡撫。光緒二十四年省,尋復置。三十年又省。

巡撫貴州等處地方提督軍務兼理糧饟加節制通省兵馬銜一人。順治十五年置。十八年,停提督軍務。乾隆十二年,以苗患復之。明年,加愛必達節制通省兵馬銜。十八年,著為例。

巡撫臺灣等處地方提督軍務兼理糧饟一人。順治元年,置福建巡撫,駐福州。光緒元年,移駐臺北。十一年,臺灣建行省,改福建巡撫為臺灣巡撫,兼學政事,其福建巡撫事歸閩浙總督兼管。二十一年,棄臺灣,省巡撫。

提督學政,省各一人。以侍郎、京堂、翰、詹、科、道、部屬等官進士出身人員內簡用。各帶原銜品級。掌學校政令,歲、科兩試。巡歷所至,察師儒優劣,生員勤惰,升其賢者能者,斥其不帥教者。凡有興革,會督、撫行之。

初,各省並置督學道,系按察使僉事銜,各部郎中進士出身者補用。惟直隸差督學御史一人,後稱順天學政。順治十年改用翰林編、檢、中、贊、講、讀並差。乾隆以來多用卿貳。江南、江北二人,順治十年改用翰林官,明年仍用僉事。康熙元年省並為一,二十四年復用翰林官。雍正三年,析置江蘇、安徽各一人。稱學院。順治七年,定學道考選部屬制。由內閣與吏、禮二部會考選,禮部二人,戶、兵、刑、工各一人。十六年停。十五年,省宣大學政歸山西學道兼理。康熙元年,並湖北、湖南提學道為一,更名湖廣提學道。雍正二年復分置。明年,命奉天府丞主考試事,省陜西臨鞏學政改歸西安學道兼理。二十三年,停督學論俸補授例,並定浙江改用翰林官,依順天、江南北例稱學院,其各省由部屬、道、府任者,仍為學道。三十九年,定翰林與部屬並差。雍正四年,各省督學並更名學院,凡部屬任者,俱加編修、檢討銜,自是提學無道銜矣。明年,命巡察御史兼理臺灣學政。乾隆十七年改臺灣道兼理。光緒十二年,巡撫兼學政事。七年,改廣東學政為廣韶學政,增置肇高學政一人。乾隆十六年,復並為一。光緒二年,增置甘肅學政一人。先是甘肅歲、科試由陜西學政兼理,至是始置。三十一年,省奉天府丞,增置東三省學政一人。是歲罷科舉,興學校,改學政為提學使。詳新官制。初置,有提督滿洲、蒙古糸番譯學政,以滿洲侍讀、侍講充。雍正元年省。

承宣布政使司布政使,省各一人。從二品。其屬:經歷司經歷,正六品。都事,從七品。照磨所照磨,從八品。理問所理問,從六品。庫大使,正八品。倉大使,從九品。各一人。布政使掌宣化承流,帥府、州、縣官,廉其錄職能否,上下其考,報督、撫上達吏部。三年賓興,提調考試事,升賢能,上達禮部。十年會戶版,均稅役,登民數、田數,上達戶部。凡諸政務,會督、撫議行。經歷、都事掌出納文移。照磨掌照刷案卷。理問掌推勘刑名。庫大使掌庫藏籍帳。倉大使掌稽倉庾。

初,直隸不置布政使,置口北道一人司度支,兼山西布政使銜。雍正二年,改從直隸布政使銜。各省置左、右布政使一人,貴州事簡,不置右布政使。左、右參政、參議,因事酌置。守道並兼參政、參議銜。所屬經歷,江寧、蘇州、湖南、甘肅不置。都事,福建、河南各一人。照磨,浙江、福建、湖北、山西、四川、甘肅各一人。檢校,正九品。雍正二年省。理問,副理問,從七品。康熙三十八年省。庫大使,倉大使,寶源局大使,正九品。康熙三十八年省。因時因地,省置無恆。順治三年,罷南直隸舊設部院遣侍郎,滿、漢各一人,駐江寧治事,至是省,定置左、右布政使各一人。十八年,江南分省,右布政使徙蘇州,左仍駐江寧。康熙二年,陜西分省,右布政使徙鞏昌,分治甘肅。明年,湖廣分省,右布政使徙長沙,分治湖南。六年,改江南右布政使為江蘇布政使,左為安徽布政使;陜西左布政使為西安布政使,右為鞏昌布政使;湖廣左布政使為湖北布政使,右為湖南布政使。並定山東、山西、河南、江蘇、安徽、江西、福建、浙江、湖北、湖南、四川、廣東、廣西、雲南、貴州各一人,陜西二人,罷左、右系銜,名曰守道。七年,定山西、陜西、甘肅為滿洲缺。雍正元年,授胡期恆陜西布政使。明年,授高成齡山西布政使。又明年,授孔毓璞甘肅布政使。參用漢人自此始。八年,置直隸守道一人,綜司度支;改西安布政使為陜西布政使;徙鞏昌布政使駐蘭州,為甘肅布政使。雍正二年,改直隸守道為布政使。乾隆十八年,停各省守道兼布政使、參政、參議銜。二十五年,以江寧錢穀務劇,增置布政使一人,析江、淮、揚、徐、通、海六府、州隸之;蘇、松、常、鎮、太五府仍隸蘇州布政使;其安徽布政使回治安慶。光緒十年,新疆建行省,增置甘肅新疆一人,駐烏魯木齊。十三年,臺灣建行省,增置福建臺灣一人,駐臺北。二十一年棄臺灣,乃省。三十年,命江寧布政使兼理江淮布政使事,尋罷。宣統二年,各省設財政公所,或名度支公所。分曹治事,以布政使要其成,間省經歷等官。

提刑按察使司按察使,省各一人。正三品。其屬:經歷司經歷,正七品。知事,正八品。照磨所照磨,正九品。司獄司司獄,從九品。各一人。按察使掌振揚風紀,澄清吏治。所至錄囚徒,勘辭狀,大者會籓司議,以聽於部、院。兼領闔省驛傳。三年大比充監試官,大計充考察官,秋審充主稿官。知事掌勘察刑名。司獄掌檢察系囚。經歷、照磨所司視籓署。

初,直隸不置按察使,置大名巡道兼河南按察使銜,通永天津巡道兼山東按察使銜,霸昌井陘巡道兼山西按察使銜。雍正二年改直隸按察使銜。各省置按察使一人。副使、僉事,因事酌置。巡道並兼副使、僉事銜。所屬經歷、安徽、湖南、甘肅、貴州不置。知事,江西、福建、山西、廣東、廣西各一人。照磨,安徽、福建、浙江、湖南、甘肅、貴州各一人。檢校、康熙六年定江西、福建、山西、陜西各一人。三十九年省。司獄,因時因地,省置無恆。順治三年,增置江寧按察使一人。康熙三年,增置江北按察使,駐泗州;湖廣按察使,駐長沙;甘肅按察使,駐鞏昌。六年,定江蘇、安徽、湖北、湖南、陜西、甘肅、浙江、江西、福建、山東、山西、河南、四川、廣東、廣西、雲南、貴州各一人,名曰巡道,徙安徽按察使駐安慶。七年,定山西、陜西、甘肅為滿洲缺。雍正元年,授高成齡山西按察使。二年,授費金吾陜西按察使,張適甘肅按察使。參用漢人自此始。八年,增置直隸巡道一人,綜司刑名。徙甘肅按察使駐蘭州。雍正二年,改直隸巡道為按察使。八年,江蘇按察使徙蘇州。江蘇隸此。乾隆十八年,停各省巡道兼按察使副使、僉事銜。咸豐三年,加安徽徽寧池太廣道按察使銜。後改皖南道。同治五年,加奉天奉錦山海道按察使銜。後改錦新營口道。光緒十三年,福建臺灣道、甘肅新疆道並加按察使銜。三十年,加江蘇淮揚海道按察使銜。福建臺灣道後省,餘並改提法使銜。宣統三年,更名提法使,間省經歷等官。

都轉鹽運使司鹽運使,從三品。奉天、直隸、山東、兩淮、兩浙、廣東、四川各一人。鹽法道,江南、江西、福建、湖北、湖南、河南、山西、陜西、四川、廣西、雲南各一人,甘肅二人。兼分守地方者二,分巡地方者六。詳道員。運同,從四品。長蘆、山東、廣東分司各一人。運副,從五品。兩浙分司一人。監掣同知,正五品。山西、河東、兩淮、淮南、淮北各一人。鹽課提舉司提舉,從五品。雲南三人,分司石膏、黑鹽、白鹽三井。運判,從六品。直隸薊永分司、兩淮海州通州泰州分司、兩浙嘉松分司各一人。鹽課司大使,正八品。直隸、場凡八:曰越支、曰巖鎮、曰蘆臺、曰豐財、曰石碑、曰歸化、曰濟民、曰海豐。山東場凡八:曰王家岡、曰永阜、曰永利、曰富國、曰濤雒、曰石河、曰官臺、曰西繇。各八人,山西三人,曰東場、曰西場、曰中場。兩淮二十有三人,曰板浦、曰臨興、曰中正、曰金沙、曰呂四、曰餘西、曰掘港、曰豐利、曰石港、曰角斜、曰拼茶、曰廟灣、曰劉莊、曰新興、曰伍佑、曰富安、曰安豐、曰梁垛、曰河垛、曰草偃、曰丁溪、曰東臺,場各一人。福建十有六人,內西河、浦下驗掣大使各一人。其場曰福清、曰詔安、曰莆田、曰下里、曰浯州、曰福興、曰潯美、曰石馬、曰惠安、曰祥豐、曰蓮河。又有江陰西場、漳浦南場、前江團場。兩浙三十有二人,內崇明巡鹽大使一人。其場曰仁和、曰三江、曰錢清、曰曹娥、曰穿山、曰石堰、曰鳴鶴、曰清泉、曰大嵩、曰雙穗、曰長林、曰長亭、曰黃巖、曰下沙、曰下沙頭、曰杜瀆、曰西路、曰許村、曰海沙、曰鮑郎、曰蘆瀝、曰橫浦、曰袁浦、曰永嘉、曰青村、曰浦東、曰龍頭、曰玉泉、曰黃灣、曰東江、曰金山。四川五人,曰青是渡、曰庸家渡、曰牛華溪、曰雲陽、曰大寧,場各一人。廣東十有二人,曰白石、曰博茂、曰大洲、曰招收、曰淡水、曰小靖、曰石橋、曰茂暉、曰隆井、曰東界、曰敢白、曰電茂,場各一人。雲南七人。曰黑鹽井、曰白鹽井、曰石膏井、曰阿陋井、曰按板井、曰大井、曰麗江井,場各一人。鹽引批驗所大使,正八品。直隸、分駐小直沽、長蘆。山東、分駐雒口、蒲臺。兩淮分駐儀徵、淮安。各二人,四川三人,重慶、嘉定府經歷各兼一人。遂寧縣丞兼一人。兩浙四人,杭州、紹興、松江、嘉興各一人。廣東一人。駐西匯關。庫大使,從八品。長蘆、兩淮、兩浙、山東、廣東、隸鹽運使。山西、福建、四川、雲南隸鹽法道。各一人。經歷,從七品。長蘆、兩淮、兩浙、山東、廣東、隸鹽運使。山西隸鹽法道。各一人。知事,從八品。兩淮、廣東各一人。巡檢,從九品。長蘆一人,駐張家灣。兩淮、分駐白塔河、烏沙河。山西分鹽池駐長樂。各二人。

運使掌督察場民生計,商民行息,水陸輓運,計道里,時往來,平貴賤,以聽於鹽政。長蘆、兩淮各一人。其福建、四川、廣東,總督兼之。兩浙、山西、雲南,巡撫兼之。沿革詳下。鹽法道亦如之。運同,運副,運判,掌分司產鹽處所,輔運使、鹽道以治其事。同知掌掣鹽政令。提舉治事如分司。場大使掌治鹽場、池、井,分轄於運同、運判,統轄於運使或鹽法道。

初差御史巡視鹽課,長蘆、咸豐十年省歸直隸總督兼理。河東、雍正二年省歸川陜總督兼理,明年復故。乾隆四十三年省歸山西巡撫兼理。嘉慶十二年改隸河東道。兩淮、道光十一年省歸兩江總督兼理。兩浙雍正三年省歸浙江巡撫兼理。乾隆五十八年改織造為鹽政。嘉慶二十五年仍歸巡撫。各一人。十年停差巡鹽御史,十二年復故。康熙六年,定各部郎員並差滿、漢各一人。八年仍改御史。十年定差一人。十一年俱歸各省巡撫兼理。十二年復差。後兼差內府員司。並稱鹽政。置都轉鹽運使,長蘆、山東、河東、乾隆五十七年省。兩淮、兩浙、康熙四十九年改驛鹽道。乾隆五十八年復故。福建、雍正四年改驛鹽道,十二年更名鹽法。兩廣尋改驛鹽道。康熙三十二年復故。各一人,雲南鹽法道一人。其各省行銷事務,並守巡道兼之。運同,長蘆、山東、俱康熙十六年省,明年復置。兩淮、康熙六十年省。兩浙康熙十六年省。明年復置。四十三年又省。河東、康熙十六年省。雍正二年復置。乾隆五十七年又省。兩浙康熙十六年省。三十二年復置。各一人,副使各一人。順治十三年省兩淮一人。康熙十六年俱省。明年復置兩浙一人。運判,兩淮四人,康熙三十八年省一人。長蘆、康熙十七年省。乾隆四十六年復置。山東、河東、俱雍正二年省。嘉慶十二年復置。十七年又省。兩浙各一人。提舉,廣東一人,康熙五年省巿舶提舉七人,歸鹽提舉兼理。三十二年省。雲南三人。吏目,從九品。廣東、康熙三十二年省。雲南雍正十年省。各一人。經歷,知事,並所轄各場鹽課司,鹽引批驗所,庫倉大使,巡檢,省置無恆。順治三年,置江南驛鹽道一人。十三年省。康熙十三年置二人,分駐江寧、安慶。二十一年省安慶一人。七年,置湖北驛鹽道一人。改屯田水利、驛傳二道置。康熙七年省,十三年復置。五十八年又省。雍正元年復置。乾隆四十四年改分守武昌鹽法道。明年,置甘肅慶陽鹽課同知一人。尋省。康熙四年,以廣西桂平梧鬱道兼鹽法。明年,置江西驛鹽道一人。十七年,置福建運同一人。四十三年省。三十年,差巡鹽御史,兩廣、三十二年停。五十七年差廣東一人。五十九年改歸兩廣總督兼理。福建雍正元年改隸閩浙總督。十二年改歸鹽法道。各一人。雍正四年,置山西鹽捕同知一人。嘉慶十二年省。明年,置四川驛鹽道一人。先是歸糧道兼理。二十五年專司鹽茶。十一年,置江蘇鹽務巡道,乾隆六年省。兩廣運判,乾隆七年省。各一人。十二年,改陜西驛傳道為驛鹽,專司鹽法。乾隆五十九年改置分巡鳳邠道。並置湖南驛鹽道一人。兼轄常、寶。十三年,改河南開歸道為分守糧驛鹽道。先是歸大梁道兼理。乾隆元年,置廣西梧州運同一人。七年省。二十四年,定淮南、淮北監掣同知二人。揀員兼攝。明年定為額缺。嘉慶十一年,定陜西鳳邠道、宣統元年省歸巡警道兼理。甘肅寧夏道兼鹽法。明年,復設山西鹽署,以河東道兼鹽法,置監掣同知一人。宣統二年,增置奉天運使一人,復改四川鹽茶道為運使。明年,改各省運使為鹽務正監督,增福建、雲南、山東、河東各一人。省鹽法道,改置副監督,定淮南、江岸、皖岸、西岸、鄂岸、湘岸、淮北、四川、滇黔邊計、濟楚、廣西、甘肅,各一人。統轄於鹽政大臣。

道員正四品。糧道。江南、蘇松、江安、浙江、雲南各一人。其山東、湖北、湖南、廣東、貴州,俱光緒、宣統間省。江西兼巡南撫建、福建兼巡福寧、陜西兼守乾鄜,並省。河道。直隸永定河道駐固安。山東運河道、江蘇河庫道,俱光緒季年省。各道兼河務者詳後。海關道。津海關道駐天津。兼關務者詳後。巡警道。勸業道。省各一人,均駐省。詳新官制。分守道:山東濟東泰武臨道,兼驛傳、水利,駐省。山西雁平道,駐代州。宣統元年省。冀寧道,兼水利,駐省。宣統二年省。湖北武昌道,廣西桂平梧道;俱鹽法道兼,駐省。其帶兵備者,黑龍江興東道,兼營務、墾務、木植、礦產,駐內興安嶺。山西河東道,鹽法道兼,駐運城。陜西潼商道,駐省城。福建興泉永道,兼海政、驛傳,駐廈門。湖北安襄鄖荊道,兼水利,駐襄陽。湖南衡永郴桂道;兼驛傳,駐衡州。整飭兵備道,直隸口北道,駐宣化,定為滿缺。後參用漢人。甘肅甘涼道。駐涼州。分巡道:直隸清河道,兼河務,駐省。霸昌道,駐昌平。光緒三十年省。河南河陜汝道,兼水利、驛傳,駐陜州。福建延建邵道,駐延平。浙江金衢嚴道,兼水利,駐衢州。湖南嶽常澧道,兼驛傳、商埠、關務,駐澧州。四川川南道,駐瀘州。廣東廣肇羅道,兼水利,駐肇慶。雲南臨安開廣道;兼關務,駐蒙自。其帶兵備者,奉天洮昌道,兼蒙旗事,駐遼源州。臨長海道,駐臨江。錦新營口道,兼關務,駐營口。興鳳道,駐安東。吉林東南路道,兼關務,駐琿城。東北路道。兼關務,駐三姓。西路道,專司交涉,駐長春。黑龍江呼倫道,駐呼倫。璦琿道,駐璦琿。以上五員並加參領銜。直隸通永道,兼河務、海防、屯田,駐通州。天津道,兼河務,見前。大順廣道,兼河道、水利,駐大名。蘇州道,糧道兼,並司水利,見前。蘇松太倉道,兼水利、漁業、關務,駐上海。常鎮通海道,兼河道、關務,駐鎮江。淮揚海道,兼鹽法、漕務、海防,加提法使銜,駐淮安。徐州道,兼河務,駐宿遷。安徽安廬滁和道,駐省城。光緒三十三年省。皖南道,省寧太池廣道改置,兼關務,加提法使銜,駐蕪湖。皖北道,省鳳潁六泗道改置,駐鳳陽。山東兗沂曹濟道,兼驛傳、河務、水利,駐兗州。山西歸綏道,兼關務、驛傳及蒙旗事,駐綏遠。初定為滿缺,後參用漢人。河南開歸陳許鄭道,兼河務,駐省。河北道,兼河務、水利,駐武陟。南汝光道,兼水利,駐信陽州。陜西陜安道,兼水利,駐漢中。鳳邠道,鹽法道兼。宣統元年省。甘肅平慶涇固化道,鹽法道兼,駐平涼。蘭州道,兼屯田、茶馬,駐省城。宣統二年省。阿克蘇道,兼水利、屯政,撫馭蒙部,稽查卡倫,駐本城。喀什噶爾道,兼水利、屯墾、通商,撫馭布魯特,稽查卡倫,駐本城。福建汀漳龍道,駐漳州。臺灣道,光緒二十一年棄臺灣,省。浙江杭嘉湖道,兼海防,駐嘉興。寧紹臺道,兼水利、海防,駐寧波。溫處道,兼水利、海防,駐溫州。江西瑞南臨道,鹽法道兼,駐萍鄉。撫建廣饒九南道,兼關務、水利、窯務,駐九江。吉南贛寧道,兼關務、水利、驛傳,駐贛州。湖北漢黃德道,兼水利,駐漢口。上荊南道,兼關務、水利,駐沙巿。施鶴道,兼轄文武,駐施南。湖南辰沅永靖道,兼界亭,鎮苗疆,駐鳳凰營。四川成綿龍茂道,兼水利,駐省城。光緒三十四年省。建昌上南道,兼驛傳,撫土司,駐雅州。川東道,兼驛傳,駐重慶。川北道,駐保寧。康安道,駐巴安,加提法使銜。邊北道,駐登科。以上二員,宣統二年置,隸川滇邊務大臣。廣東南韶連道,兼水利,駐韶州。惠潮嘉道,駐惠州。廉欽道,駐欽州。高雷陽道,駐高州。瓊崖道,駐瓊州。廣西左江道,駐南寧。右江道,駐柳州。太平思順道,駐龍州。以上二員,並控制漢、土。雲南迤東道,兼驛傳,駐曲靖。迤西道,兼驛傳、關務,駐大理。迤南道,兼驛傳,駐普洱。貴州貴東道,兼驛傳,鎮苗疆,駐古州。貴西道;駐安順。宣統二年省。整飭兵備道,直隸熱河道,加提法使銜,駐本城。江南江寧道,鹽法道兼,並司水利,駐省。山東登萊青道,兼海防、水利,駐登州。陜西延榆綏道,兼鹽茶,駐榆林。甘肅寧夏道,兼鹽法、水利,駐寧夏。鞏秦階道,兼茶馬、屯田,駐秦州。新疆鎮迪道,兼驛傳,加提法使銜,駐省。伊塔道;兼水利、屯田,稽查卡倫,駐寧遠。撫治兵備道,甘肅西寧道,兼治蒙、番,駐西寧。乾隆間定為滿、蒙缺,後參用漢人。嘉慶間復舊制,後仍參用。安肅道。兼屯田,駐肅州。各掌分守、分巡,及河、糧、鹽、茶,或兼水利、驛傳,或兼關務、屯田;並佐籓、臬覈官吏,課農桑,興賢能,勵風俗,簡軍實,固封守,以帥所屬而廉察其政治。其雜職有庫大使,從九品。倉大使,關大使,俱未入流,詳後雜職。皆因地建置,不備設。

布、按二司置正、副官。尋改置布政使左、右參議,是為守道;按察使副使、僉事,是為巡道。時道員止轄一府,或數道同轄一府也。順治十六年,諭各道兼帶布、按二司銜,著為例。康熙六年,省守、巡道百有八人,厥後漸次復置,有統轄闔省者,有分轄三、四府州者,省置無恆,銜額靡定,均視其升補本職為差。如由京堂等官補授者為參政道,掌印給事中、知府補授者為副使道,由科道補授者為參議道,郎中、員外郎、主事、同知補授者為僉事道,守、巡皆同。乾隆十八年,罷參政、參議、副使、僉事諸銜,特峻其品秩。初制,參政道從三品,副使道正四品,參議道從四品,僉事道正五品。至是俱定正四品。嗣是守、巡諸道先後加兵備者,八十餘人。四十一年,詔道員署布、按二司者,許上封奏。嘉慶四年,以道員職司巡察,詔復雍正間舊制,許言事。德宗以降,別就省會置巡警、勸業二道,分科治事,議省守、巡道,酌留一二帶兵備者,未果。又初制有山東、安徽、浙江、江西、湖北、湖南興屯道,浙江、江蘇海防道,福建巡海道,江蘇江防道,馬政道,後俱省。

府知府一人。初制正四品。乾隆十八年改從四品。同知,正五品。通判,正六品。無定員。其屬:經歷司經歷,正八品。知事,正九品。照磨所照磨,從九品。司獄司司獄,從九品。各一人。又江蘇檢校、貴州長官司吏目,各二人。知府掌總領屬縣,宣布條教,興利除害,決訟檢奸。三歲察屬吏賢否,職事修廢,刺舉上達,地方要政白督、撫,允乃行。同知、通判,分掌糧鹽督捕,江海防務,河工水利,清軍理事,撫綏民夷諸要職。其直隸布政使者,全國二十有二,制同直隸州,或隸將軍與道員,各因地酌置。經歷、知事、照磨、司獄,所掌如兩司首領官。自同知以下,事簡者不備。

初制,知府秩正四品,區三等,多用漢員,時滿洲郎、員外轉布、按不占府缺。康熙初始參用。並置推官康熙六年省。及掛銜推官。順治三年省。督捕左、右理事官康熙三十八年省。各一人。康熙元年,以委署州、縣專責知府,行保舉連坐法。五十一年,允御史徐樹庸請,引見督、撫特舉人員。自是知府授官,引見時觀敷奏,報最時課治績,著為令甲。雍正元年,諭督、撫甄別知府,厥後府與同知且許言事。後停。十二年,以府職重要,援引古誼,思復久任制。部議以遷擢為鼓勵,止於限年升調。仁宗親政,以知府為承上接下要職,嚴諭各督、撫考覈。宣宗時猶然。文、穆而下,古轍浸遠矣。宣統之季,省各府附郭縣,以知府領其事。自江南、陜西、湖廣分省,奉天、吉林、黑龍江、新疆建省,四川、雲南改土歸流,各以府隸之,計全國府二百十有五。

州知州一人。初制從五品。乾隆三十五年改直隸州知州正五品。州★,從六品。州判,從七品。無定員。其屬:吏目一人。從九品。知州掌一州治理。屬州視縣,直隸州視府。唯無附郭縣。州同、州判,分掌糧務、水利、防海、管河諸職。吏目掌司奸盜、察獄囚、典簿錄。

初制,州置知州一人。嗣後因地制宜,省析並隨時更易,佐貳亦如之。計全國直隸州七十有六,屬州四十有八。

縣知縣一人。正七品。縣丞一人。正八品。主簿無定員。正九品。典史一人。未入流。知縣掌一縣治理,決訟斷闢,勸農賑貧,討猾除奸,興養立教。凡貢士、讀法、養老、祀神,靡所不綜。縣丞、主簿,分掌糧馬、征稅、戶籍、緝捕諸職。典史掌稽檢獄囚。無丞、簿,兼領其事。

初制,縣置知縣一人。順治十二年,諭吏部參酌州、縣制,區三等。先是臺諫需人,依明往例,行取知縣。聖祖親政,以親民官須諳利弊,命督、撫舉賢能。康熙二十九年,復諭九卿察廉吏。清苑知縣邵嗣堯等十二人擢置憲府,錚然有聲。高宗猶亟稱之。自部議防太驟,俾回翔曹司間,其途稍紆矣。乾隆十六年,停止行取升部員,其賢能者仍得題擢也。嘉慶十五年,刊欽定訓飭州縣規條一書,頒示各省。文宗時,軍書旁午,民生凋敝,申諭督、撫隨時嚴察。顧其時雜流競進,廉能者寡。穆宗厲精圖治,諭各省甄別捐納、軍功人員,尋以招流亡、墾地畝課第殿最。同治七年,復命設局刊牧令諸書,猶存振厲至意。光緒間,督、撫違例更調州、縣官,視同傳舍。二十四年,議復久任制。三十一年,定考覈州、縣章程,詳考績。制亦少密焉。計全國縣凡千三百五十有八。

儒學府教授、正七品。訓導,從八品。州學正、正八品。訓導,縣教諭、正八品。訓導,俱各一人。教授、學正、教諭,掌訓迪學校生徒,課藝業勤惰,評品行優劣,以聽於學政。訓導佐之。例用本省人。同府、州者否。江蘇、安徽兩省通用。初沿明制,府、、州、縣及各衛武學並置學官。康熙三年,府、州及大縣省訓導,小縣省教諭。十五年復置,自是教職分正復。厥後開俊秀監生捐納教職例。三十年,允江南學政許汝霖請,凡捐學正、教諭者改為縣丞,訓導改為主簿,繇是唯生員始得入貲,教授必由科目。三十二年,省各衛武學訓導。三十九年,頒學宮聖諭十六條,月朔望命儒學官集諸生宣讀。四十一年,頒禦制訓飭士子文,命學宮鑱石。四十二年,定教職,學各二人。雍正元年,允云南土人、四川建昌番夷、湖南永綏等處建立義學,嗣是改土歸流,塞外荒區漸次俱設儒學。明年,置雲南井學訓導,井學自此始。又明年,省都司儒學、京衛武學教授,滿洲生員並歸漢官月課。十三年,定府、州、縣儒學官品秩。如前所列。光緒三十年後,科舉既罷,各省教職缺出不補。時議改置文廟官,不果。

巡檢司巡檢,從九品。掌捕盜賊,詰奸宄。凡州縣關津險要則置。隸州者,專司河防。

驛驛丞,未入流。掌郵傳迎送。凡舟車夫馬,廩糗庖饌,視使客品秩為差,支直於府、州、縣,籍其出入。雍正六年,定滿人不得為驛丞。典史同。

庫大使一人。隸布政使者正八品,運使、鹽法道、各道從九品,鹽茶道及各所俱未入流。掌主庫藏。

倉大使一人。隸布政使及各府從九品。州、縣未入流。掌主倉庾。

稅課司大使一人。隸道、府者從九品。州、縣未入流。掌主稅事。凡商賈、儈屠、雜巿俱有常徵,以時榷之,輸直於道、府若縣。

徬官一人。未入流。掌瀦洩啟閉。

河泊所大使一人。未入流。掌徵魚稅。

醫學府正科,州典科,縣訓科,各一人。俱未入流。由所轄有司遴諳醫理者,咨部給劄。宣統元年,奉天模範監獄成,置醫務所所長,省府正科。

陰陽學府正術,州典術,縣訓術,各一人。俱未入流。由所轄有司遴行端者,咨部給劄。雍正七年,令兼轄星學。

府僧綱司都綱、副都綱,州僧正司僧正,縣僧會司僧會,各一人。府道紀司都紀、副都紀,州道正司道正,縣道會司道會,各一人。俱未入流。遴通曉經義,恪守清規者,給予度牒。


\end{pinyinscope}