\article{列傳一百六十二}

\begin{pinyinscope}
白鎔孫桓史致儼那清安升寅李宗昉姚元之

何汝霖季芝昌

白鎔,字小山,順天通州人。嘉慶四年進士,選庶吉士,授編修,典福建鄉試。十八年,大考二等,擢贊善。督安徽學政,詔密詢地方利弊,疏言:「安徽錢糧,惟鳳陽、泗州遭湖、河之害,積逋較巨。遇豐稔之年,循例帶徵舊額。在小民以一年而輸數年之賦,雖樂歲不免拮據;而官吏懼譴,規避多方,積重難返。與其存徵之名,致小民日受追呼,國計依然無補,何如核徵之實,使官吏從容措理,舊額尚可漸清。請嗣後二屬錢糧,每年祗帶徵一年,儻遇歉收,再行遞緩,民力漸紓,催科者自顧考成,行之必有效。」詔允行。

青陽有孝子曰徐守仁,幼孤,事母孝。母沒,廬墓三年,鎔造廬贈賻,題請旌表。訪求明臣左光鬥遺裔,取列縣庠。按試所至,集士人講學,以正人心厚風俗為本。累遷少詹事。道光元年,督廣東學政。歷詹事、內閣學士。七年,擢工部侍郎,調吏部。九年,偕尚書松筠赴直隸按外委白勤被誣冤斃獄,護理總督屠之申以下降黜有差。督江蘇學政。尋偕侍郎寶興勘視南河垛料,舉實以聞。十一年,擢左都御史,召還京,未至,命查勘江南災賑。時尚書穆彰阿、硃士彥亦奉命勘湖、河汎溢狀,穆彰阿先回京,鎔遂偕士彥履勘沿河閘壩工程,與總督陶澍定議以工代賑。赴安徽,周歷太平、寧國、池州、安慶、廬州各郡,先後疏劾飾災侵賑諸弊。次年,回京,署翰林院掌院學士,典順天鄉試。十三年,擢工部尚書,典武會試。故事,武闈雙好不足額,始取單好。是科雙好不盡取中,坐降大理寺卿。十九年,乞病歸,卒於家,年七十四。

鎔事母孝,教子弟嚴。宣宗嘗嘉其家法之善,以勉朝臣云。

孫桓,字建侯。同治二年進士,授吏部主事。累遷郎中。掌選,清嚴慎密,吏不能欺,為時所稱。光緒中,洊擢兵部侍郎,綜覈一如為司官時。十七年,因病乞休,尋卒。

史致儼,字容莊,江蘇江都人。家酷貧。甫冠,為諸生,學政謝墉器其才,給膏火,居尊經閣讀書。薦預召試,未與選。嘉慶四年,成進士,選庶吉士,授編修。督四川學政。累遷右庶子。二十一年,督河南學政。自滑縣匪平,猶有伏莽,密詔偵察。疏陳彰、衛二郡民間習邪教猶眾,州縣編查保甲,有名無實,撰敦俗篇,刊布以化導之。商丘廩生陳忠錦以不濫保被毆,知府、經歷受賕,反加斥責,忿而自經。疏劾,譴罪有差。

道光元年,典湖北鄉試。累遷內閣學士。三年,擢刑部侍郎,調禮部。五年,督福建學政。奏分臺灣舉人中額,增所屬四縣學額。漳、泉諸郡習械斗,諸生與者,屏不與試,悍風稍息。九年,偕侍郎鍾昌赴山西鞫獄,平定知州故出人罪,鞫實,論兇犯如律,褫知州恆傑職。調刑部,歷左都御史,遷禮部尚書。兩典順天鄉試。調工部,又調刑部。勤於其職,竟日坐堂上閱案牘,揅析論難,視司員如弟子。任刑部凡四年,京察,以刑名詳慎,被議敘。十八年,乞解職。尋卒,年七十九,贈太子太保,祀鄉賢及名宦祠。

那清安,字竹汀,葉赫納喇氏,滿洲正白旗人。嘉慶十年進士,授戶部主事,遷翰林院侍講。累遷內閣學士。二十四年,授禮部侍郎,歷刑部、工部。道光元年,命赴直隸讞獄,擢左都御史,管光祿寺事,兼都統。尋遷兵部尚書,調刑部。四年,出為熱河都統,偕左都御史松筠等赴土默特讞獄,事竣,疏言:「蒙古惡習,常有移尸訛詐,為害滋甚。蒙古律例,凡軍流徒犯,罪止折枷,情重法輕。請嗣後遇有假捏人命詐財者,所擬軍流徒罪即行實發,不準折枷,以懲刁惡。」下所司議行。六年,召授左都御史。逾年,復任熱河都統,召對,詢知其母年老,命仍還左都御史任。十一年,復授兵部尚書,典順天鄉試及會試。十四年,以疾乞解職,允之。尋卒,贈太子太保,謚恭勤。

那清安工為館體應制詩,時皆誦習。因與穆彰阿同榜成進士,晚乃受宣宗知,迭秉文衡。既卒,會兵部以慶廉送武會試有殘疾,為監試御史所劾。先是那清安為監射大臣,曾以慶廉殘疾扣除,上追念其持正,予其子全慶加二級。全慶,光緒初官大學士,自有傳。

升寅,字賓旭,馬佳氏,滿洲鑲黃旗人。拔貢,考授禮部七品小京官。舉嘉慶五年鄉試。累遷員外郎,改御史。疏言學校為人才根本,請嚴課程,務實用,戒奢靡;又疏陳防禁考試八旗生懷挾冒替諸弊:從之。改右庶子,累遷副都御史。二十一年,授盛京禮部侍郎,署盛京將軍。調刑部,召為工部侍郎,又調刑部。道光六年,出為熱河都統。以蒙古各旗招內地游民開採煤礦,往往生事械斗,疏請諭禁,從之。八年,命赴甘肅偕總督鄂山按寧夏將軍慶山、副都統噶普唐阿互劾事,罷慶山,即以升寅代之。歷成都、綏遠城將軍。命鞫鄂爾多斯京控獄,奏言:「蒙古京控日繁,請自後各部落封禁地樹立界牌,以杜私墾;蒙古阿勒巴圖禁止餽贈,以息爭端;扎薩克王、貝勒等毋用內地書吏,以免教唆;各旗協理臺吉,會同盟長選舉,以昭慎重;盟長會盟需用烏拉,應明定限制,以免浮索:庶積弊清而獄訟息。」

十一年,召授左都御史,兼都統。十二年,署工部尚書。京畿旱,疏請發米,設十廠煮粥以濟災民,從之。十三年,偕侍郎鄂順安按西安將軍徐錕貪縱,得實,議褫職。十四年,命閱兵山東、河南,就鞫桐柏知縣寧飛濱故出人罪,治如律。命赴廣東、湖南按事,授禮部尚書,未至,卒於途。優詔賜恤,稱其老成清介,贈太子太保,謚勤直。

子寶琳,直隸保定知府,濬定州洿澤,有治績;寶珣,同治中,官兵部侍郎、山海關副都統。孫紹祺,咸豐六年進士,由編修官至理籓院尚書;紹諴,光緒中,山西布政使,從治鄭州河工,終駐藏大臣;紹英,宣統初,度支部侍郎,內務府大臣。

李宗昉,字芝齡,江蘇山陽人。嘉慶七年一甲二名進士,授編修,典陜甘鄉試。大考二等,擢贊善。督貴州學政,累遷侍讀學士,督浙江學政。歷詹事、內閣學士。道光元年,授禮部侍郎。次年,典會試,又典江西鄉試,留學政。值大水,歲饑,與巡撫籌賑務,多所全活。調戶部侍郎。初,宗昉督學貴州時,巡撫議丈全省田為增賦計,民情惶駭,會檄學官徵集圖書,得御史包承祚奏疏,乾隆初,學政鄒一桂請丈田,而承祚奏駁之,極言黔中山多平地少,民每虛占不毛之地,胥吏高下其手,以丈高下不可準之田,賦未必增,民受其害。部議停止,宗昉持以示巡撫曰:「此事學臣嘗奏之,被駁。今必解其所駁乃可。」巡撫亦悟,事得寢。至是,官戶部,署巡撫麟慶因復奏上其事,部援故事詳覆之,乃定議不行。歷工部、吏部侍郎,兼管國子監、順天府尹事。自七年至十年,典順天鄉試二,會試一,浙江鄉試一,得士稱盛。擢左都御史、禮部尚書。二十四年,以疾乞休。二十六年,卒,依例賜恤。

姚元之,字伯昂,安徽桐城人。嘉慶十年進士,選庶吉士,授編修,典陜甘鄉試。入直南書房。給事中花傑劾戴衢亨、英和援引,詔元之文字本佳,斥傑詆訐,尋亦罷元之入直。十七年,大考一等,擢侍講。復以武英殿刊刻聖訓有誤,仍降編修。十九年,督河南學政,疏禁坊刻類典等書以杜剿襲;又密陳河南與安徽、湖北交界地多捻匪,陳州、汝寧鹽運迥殊,土匪把持:並嘉納之。累遷內閣學士。

道光十三年,授工部侍郎。疏陳臺灣營務積弊,窩娼聚賭,械鬥殺人,操演雇人替代,詔下閩督嚴察整頓。調戶部,又調刑部。迭典順天、江西鄉試。督浙江學政,未滿,十八年,擢左都御史,召回京。尋以南昌知府張寅為江西巡撫裕泰劾罷,元之為寅疏辯,臚陳政績,請查辦,詔斥冒昧,降二級調用。二十一年,海防方亟,疏陳廣東形勢,豫籌戰守,下靖逆將軍奕山等採行。授內閣學士。二十三年,京察,以年衰休致。

元之學於族祖鼐,文章爾雅,書畫並工。習於掌故,館閣推為祭酒。愛士好事,穆彰阿素重之。後以論洋務不合,乃被黜。咸豐二年,卒。

何汝霖,字雨人,江蘇江寧人。拔貢,考授工部七品小京官。中式道光五年舉人,充軍機章京,累遷郎中。歷內閣侍讀學士、大理寺少卿。偕侍郎恩桂按事浙江,查勘南河料垛。命在軍機大臣上行走,歷宗人府丞、副都御史。二十二年,授兵部侍郎,調戶部。偕大學士敬徵勘東河工程。二十五年,擢兵部尚書。值太后七旬萬壽,汝霖母丁年九十,五世同堂,賜御書扁額,尋以母憂歸。江蘇大水,命在籍襄治賑務。先是,總督陶澍於江寧立豐備倉以備荒,縣令虧挪穀價,大吏許以他款抵。汝霖曰:「倉穀以備兇。今荒象如此,汝霖不敢欺朝廷,當各為奏上。」乃以給賑用。服闋,命以一品頂戴署禮部侍郎,尋署戶部尚書,仍直軍機處,授禮部尚書。

汝霖久襄樞務,資勞己深,尚書陳孚恩由章京躋大臣,駸用事,厭汝霖居其前。汝霖年逾七十,一日在直,觸火爐幾僕。孚恩笑曰:「人當避爐,爐豈能避人?」汝霖知其諷己,咸豐二年,以足疾乞罷直,許之。未幾,卒,謚恪慎,祀鄉賢。子兆瀛,浙江鹽運使。

季芝昌,字仙九,江蘇江陰人。父麟,直隸鉅鹿知縣,居官慈惠。嘉慶十八年,捕邪教,焚其籍,免株連數千人。坐捕匪不力,戍伊犁。

芝昌年逾四十,成道光十二年一甲三名進士,授編修,散館第一。未幾,大考第三,擢侍讀,督山東學政。十九年,大考復第三,擢少詹事,晉詹事,典江西鄉試,督浙江學政。母憂歸,服闋,擢內閣學士。二十三年,授禮部侍郎,督安徽學政,調吏部,又調倉場。二十八年,命偕定郡王載銓籌辦長蘆鹽務,清查天津倉庫,疏陳:「蘆鹽積累,各商憚於承運,懸岸至四十餘處。請將河南二十四州縣仿淮南例改票鹽,先課後引。直隸二十四州縣限半年招商招販,無商販即責成州縣領運,或由鹽政遴員官運。支銷浮費及官役陋規,永遠裁汰。每年應完帑利,灘及通綱額引,與正課一律徵收。其協濟補欠充公等項加價名目,概行革除。並於各引鹽加斤免課,每斤準其減價敵私。」詔依議行。

二十九年,偕大學士耆英赴浙江閱兵,並清查倉庫,籌辦鹽務。途經東河、南河,查詢節浮費、裁冗員事宜,奏減東河正款二十萬兩,裁泉河通判、歸河通判,南河每年用款以三百萬兩為率,減省五六十萬兩,並揚運通判於江防,改為江運同知,裁丹陽縣丞、靈壁主簿、呂梁洪巡檢,從之。耆英病留清江浦,芝昌獨赴浙江,疏陳變通鹽務章程七事:杭、嘉、紹三所引鹽,分別加斤,止令完交正課;松所引鹽,酌裁科則;虛懸口岸,選商接辦,並籌款收鹽;緝私責成官商,由運司審覈;緝獲私鹽,分別充賞,及補課作正配銷;禁革引地陋規;覈裁巡驗浮費。尋查州縣倉庫,統計實虧之數,多至三百九十餘萬,請將虧數最多之員,革職,勒追;不足,則由原任上司按成分賠,或由本省各官分成提補;其有欠在胥吏者,尤嚴補追,毋任幸免:並從之。

授山西巡撫,未一月,召署吏部侍郎,命在軍機大臣上行走。尋授戶部侍郎。三十年,擢左都御史。咸豐元年,出為閩浙總督。艇匪在浙洋劫掠山東兵船,被剿遁閩洋,遣水師截擊,賊眾畏罪投誠,分別安置。二年,兼署福州將軍。疏請停罷捐納舉人、附生之例;又奏禁鹽商代銷官運,以杜取巧:並從之。尋以疾乞休。

芝昌以文字受宣宗特達之知,嘗曰:「汝為文,行所無事,譬之於射,五矢無一失。」及查辦長蘆、兩浙鹽務稱旨,遂驟進膺樞務。甫數月,宣宗崩,文宗猶欲用之,畀以外任。未一歲,謝職歸。久之,卒於家,未予恤典。光緒初,署閩浙總督文煜奏陳政績,追謚文敏。子念詒,道光三十年進士,官編修。孫邦楨,同治十二年進士,官至福建布政使。

論曰:承平,士大夫平進而致列卿,或以恪謹稱,或以文學顯,固不能盡有所建樹;或餘澤延世,子孫復繼簪纓,若白鎔、那清安、升寅諸人是也。季芝昌晚遭殊遇,已值宣宗倦勤之年,暫任兼圻,奉身而退,其見幾知止者耶?


\end{pinyinscope}