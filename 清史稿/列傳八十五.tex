\article{列傳八十五}

\begin{pinyinscope}
噶爾弼法喇查克丹欽拜常賚哈元生子尚德董芳

查弼納達福定壽素圖

噶爾弼,納喇氏,滿洲鑲紅旗人。父額爾德赫,為敬謹親王尼堪長史,屢從征伐。順治十六年,署護軍統領。偕安南將軍達素等師下廈門,擊鄭成功。額爾德赫將右翼,獲其將周序。命署鎮海將軍。康熙元年,還京,尋卒。雍正間,以噶爾弼疏乞補謚,謚果毅。

噶爾弼,初授前鋒參領,累遷鑲紅旗護軍統領。準噶爾策妄阿喇布坦遣策零敦多卜襲據西藏,康熙五十八年,命噶爾弼馳赴四川佐總督年羹堯治軍事。噶爾弼詗知策零敦多卜與其副三濟不睦,謂其隙可乘,疏請招策零敦多卜降。五十九年二月,上命平逆將軍延信自青海入西藏,而授噶爾弼定西將軍,偕都統武格將四川、雲南兵出拉里。策零敦多卜自將拒延信,而遣其黨春丕勒宰桑將二千六百人自章米爾戎拒噶爾弼。噶爾弼取間道至莫珠貢喀,集皮船渡河,直趨西藏,八月,克之。噶爾弼集西藏大小第巴、頭目及諸寺喇嘛宣上指安撫,封達賴喇嘛倉庫,遣兵守隘,截準噶爾糧道,擒斬策零敦多卜所署置總管喇嘛五。策零敦多卜為延信所破,遁走。西藏平。

捷聞,上諭曰:「噶爾弼等遵朕指行師絕域,各自奮勵,撫定唐古特人民,命優敘。」延信留駐西藏,六十年,以病召還,命噶爾弼佩定西將軍往代。尋授鑲藍旗蒙古都統。行至瀘定橋,託病不行。年羹堯以聞,命奪官;逗遛不敢詣京師,論斬。雍正元年,世宗貰其罪,賜都統銜從軍。迭署固原提督、布隆吉爾副將軍,授鑲紅旗漢軍都統。三年,擢奉天將軍。五年,疏言:「奉天金、銀、銅、鉛諸礦,雖開採有禁,而竊掘尚多。惟杯犀湖產鐵,為民間農器所需。遼陽黃波羅峪、開原打金廠,請視錦州大悲嶺例,永禁開採。」下部議行。旋卒。

法喇,那木都魯氏,滿洲正白旗人。父敦泰,從達素擊鄭成功,戰死。母喜塔臘氏,守節撫孤。法喇,初授筆帖式。康熙十三年,以護軍從討吳三桂,自廣東下雲南。三十五年,以署驍騎參領從征噶爾丹,累擢鑲白旗蒙古都統、護軍統領。

準噶爾策妄阿喇布坦遣其族兄策零敦多卜攻西藏,四川提督康泰率師次黃勝關,兵譁潰。上命法喇馳赴四川佐年羹堯治軍事,並按提督標兵譁潰狀。法喇察知泰偏信守備汪文藻克餉,請斬文藻及倡亂兵以徇,上從之,並奪泰官。五十七年,策零敦多卜戕拉藏汗,幽達賴喇嘛,遂據有其地。法喇遣員外郎巴特瑪等赴里塘傳諭,又令前鋒參領伍林葩、化林協副將趙宏基將滿、漢兵五百與之偕。疏言:「西藏資茶養生,應令松潘禁茶出口。里塘、巴塘番寨所需,當開具戶口,定數買運。」下所司議行。五十八年,命法喇出駐打箭爐,令副將岳鍾琪率師徇里塘,番酋達瓦喇扎木巴、第巴塞卜騰阿珠不從命,縛送法喇軍,斬以徇。進次巴塘,第巴喀木布等請降,命法喇進駐巴塘。五十九年,年羹堯請授噶爾弼定西將軍,率師入西藏,令法喇還駐打箭爐。

六十年,還京師。尋以護軍有自戕者,不以實奏,坐奪官。六十一年,與千叟宴,賜復原銜。雍正十三年,卒。

查克丹,博爾濟吉特氏,滿洲正黃旗人,奉義公恩格德爾曾孫。自官學生襲三等阿達哈哈番,授頭等侍衛。累遷正黃旗護軍統領、鑲藍旗蒙古都統。雍正三年,署甘州將軍。準噶爾使至,守備馬德仁等供應失時,查克丹疏劾,並陳花馬池至甘州驛馬疲羸狀,命總督年羹堯嚴察。四年,還京師,授正黃旗滿洲都統。五年,命率番代兵出北路。九年,振武將軍順承郡王錫保出北路討準噶爾,命查克丹參贊軍務,授內大臣。十年,準噶爾將小策零敦多卜入邊,掠喀爾喀諸部。查克丹偕額駙策棱等赴奔博圖山,敵越察罕廋爾入掠杭愛山,師逐之,至額爾德尼昭,大戰破敵。查克丹督兵奮擊,敵自推河遁走;復追至察罕托輝,斬馘殆盡。以功進二等阿達哈哈番。錫保代傅爾丹為靖邊大將軍,仍以查克丹參贊軍務。十三年,還京師,調正紅旗蒙古都統。乾隆四年,以病再疏乞休,命致仕。十一年,卒,賜祭葬,謚敏恪。

欽拜,瓜爾佳氏,滿洲鑲紅旗人。曾祖羅璧,勞薩弟也,偕來歸。有功,授一等阿達哈哈番,以兼襲兄子程尼世職,合為一等公。欽拜改襲一等伯,授頭等侍衛。累遷正黃旗蒙古副都統。雍正元年,授兵部侍郎。四年,以引見失儀,上詰責,巧辯,奪官,戍軍臺。九年,召還,復官。撫遠大將軍馬爾賽出北路討噶爾丹,命欽拜將右衛兵以從,參贊軍務,授內大臣,駐扎克拜達里克。十年,振武將軍順承郡王錫保駐察罕廋爾,奏請移欽拜相佐。上諭曰:「馬爾賽治事甚不愜朕意,扎克拜達里克軍中恃欽拜一人,當仍留北路。」準噶爾將小策零敦多卜等自推河走,欽拜等力請追擊,馬爾賽聽敵過,師乃出。既至博木喀喇,令欽拜將七百人進,不及敵而還。欽拜等疏聞,上誅馬爾賽。尋署綏遠將軍。十一年,復署建勛將軍。平郡王福彭代為定邊大將軍,命軍事諮於欽拜。乾隆元年,還京師。出為青州將軍。還,在內大臣上行走。十二年,卒,賜祭葬,謚肅敏。

常賚,納喇氏,滿洲鑲白旗人,鎮安將軍瑪奇子。事世宗雍邸。雍正元年,授工部員外郎,遷郎中。二年,調戶部。三年,授廣東布政使。四年,擢福建巡撫。廣東巡撫楊文乾言福建倉庫虧空,上命文乾清理,即移常賚署廣東巡撫。疏言:「廣東地卑苦,夏秋潦漲,廣州、肇慶二府尤甚。請以廣州通判管南海、三水堤工,肇慶通判管高要、高明、四會堤工,歲冬督堤長修築,定保固賞罰。水漲護防,仍以鴨埠、魚諸稅充用。」尋赴福建。六年,調雲南。

常賚在廣東,盜竊奏摺匣鎖鑰,令工私制;將軍標兵匿盜,徇不治;電白、從化盜發,隱不奏;又與將軍石禮哈等訐文乾。上諭曰:「常賚朕籓邸微員,以其謹慎,擢至巡撫。乃盜失摺匣鑰匿不奏,尚得謂無欺乎?且與石禮哈等黨同伐異,其罪不可貸!奪官,赴廣東待鞫。」論斬,上推瑪奇下雲南舊功,特赦之,令從尚書查弼納往陜西治餉。八年,授刑部侍郎,署寧夏將軍。九年,授鎮安將軍,將肅、甘、涼三州兵五千人自為一隊,備聲援。尋授西路副將軍。

十年,準噶爾侵哈密,常賚與都統良敦、總兵張存孝將滿、漢兵三千二百,駐無克克嶺御之。旋授內大臣。從大將軍岳鍾琪移軍穆壘,復從護大將軍張廣泗移軍巴爾庫爾。十一年冬,署大將軍查郎阿奏方冬雪深,請分兵駐防,廣泗將萬人駐北山,常賚將九千人駐南山。十三年,命統綠旗兵萬人駐巴爾庫爾,提督顏清如、尚書馬會伯為副。準噶爾乞和,乾隆元年,率兵還京師。五年,以疾致仕,予半俸。十一年,卒,賜祭葬。

哈元生,直隸河間人。康熙間入伍,授把總。累遷建昌路都司。坐失察私木過關,奪官。雍正二年,命引見,發直隸以守備用,補撫標右營守備。貴州威寧總兵石禮哈請以元生從剿仲家苗,有勞,三年,補威寧鎮中軍游擊。烏蒙土知府祿萬鍾侵東川,鎮雄土知府隴慶侯助為亂。鄂爾泰檄元生會四川兵討賊,賊據險拒戰,元生冒矢石奪攻克之。鄂爾泰上其功,上獎元生取仲家苗、克烏蒙能效力,命以副將、參將題用,尋授尋霑營參將。

六年,米貼苗婦陸氏為亂,鄂爾泰令元生往剿,破險設伏,搗其巢,獲陸氏。率師赴阿驢,破雷波土司,以其助陸氏劫糧也。賚白金四千。遷元江副將。師還,阿驢夷目從,坐事,元生鞭之,其人大譟,圍元生。元生率游擊卜萬年等與戰兩晝夜,賊敗卻,元生督兵奪據赤衣臺。鶴麗總兵張耀祖赴援,元生出小溜筒江,搜斬餘賊,阿驢人空寨遁。拉金、者呢諸寨助為亂,並討平之。鄂爾泰具以聞,上諭曰:「野夷性反覆,即無鞭責事,亦未必帖然。元生效力多,功過相當。置不議。」

七年,調黎平副將,擢安籠總兵。八年,烏蒙復為亂,鄂爾泰令元生督兵出威寧,破賊數萬,射殪其渠黑寡、暮末,連躪賊壘八十里,遂克烏蒙。賜孔雀翎及冠服,賚白金萬。九年,擢雲南提督。上以元生母逾八十,予封誥。尋調貴州。十年,召詣京師,入對,解御衣以賜,命在辦理軍機處行走。旋令回籍省親。

貴州九股苗為亂,命還貴州督剿。遭母喪,賜祭,令在任守制。率兵攻九股苗,獲悍苗百餘,俘斬甚眾,餘悉請降。十二月,進新闢苗疆圖志,命巡撫元展成勘訂。十三年,古州苗為亂,擾黃平,元生遣兵擊之,總督尹繼善奏調湖廣、廣西兵會剿。上授元生為揚威將軍,統兵進討,而以湖廣提督董芳為之副。尋遣尚書張照為撫定苗疆大臣,元生與之忤。乃議劃施秉以上為上游,用雲南、貴州兵,隸元生;施秉以下為下游,用湖廣、廣西兵,隸芳。元生與芳議界,詳逮村莊道路,文移辨論,日久師無功。經略張廣泗至,劾元生徒事招撫,奪官逮京師,坐貽誤軍機論斬。乾隆元年,上命貸其死,賜副將銜,赴西路軍營效力。三年,卒,上深惜之,加總兵銜,賜祭葬。

子尚德,初從元生至雲南,入伍,授千總。烏蒙既克,齎疏奏捷,上命以游擊題補,補雲南鶴麗右營游擊,遷奇兵營參將。乾隆元年,廣泗奏尚德奉檄從征,因父獲譴,黽勉自效。擢貴州清江協副將,調定廣協。三年,討平定番州屬姑盧寨苗。以父憂歸,起湖南辰州副將。遷總兵,歷宜昌、涼州、臨元、古州諸鎮。十三年,討大金川,命從軍。尋為總督張允隨劾擾民虐兵,坐奪官。二十二年,賜副將銜,赴西路軍營效力。以送羊赴軍多斃,奪官責償,遣回籍。卒。

董芳,陜西咸寧人。初入伍,隸督標。中式武舉,補千總。雍正二年,師征青海,從副都統達鼐等追獲丹津琿臺吉及其孥,並羅卜藏丹津女兄。四年,超授三等侍衛,出為直隸正定鎮標游擊,累遷雲南臨元鎮總兵。十一年,思茅土酋刁興國等為亂,芳與提督蔡成貴等率師討之,擒興國及助亂土目楊昌祿等,斬三千六百餘人,降四萬二千六百餘人。總督高其倬留芳搜餘黨,悉平之。十二年,擢湖廣提督。

十三年,貴州九股苗為亂,授雲南提督哈元生揚威將軍,芳副將軍,率師討之。尋命尚書張照總理撫定苗疆,亂未定,高宗即位,授張廣泗為經略,視師。廣泗劾芳駐軍八弓,依附張照,與元生互訐,師集數月,剿撫初無端緒。奪芳官,逮京師。乾隆元年,王大臣會鞫,擬發邊遠充軍,上命寬之,以副將發雲南。遭父憂,服除,署劍川協副將。擢總兵,歷楚姚、昭通二鎮。遭母憂,十三年,召赴京師,賜孔雀翎。

命從征大金川,即授四川重慶總兵。經略訥親檄芳助總兵莽阿納等攻克普瞻左梁及阿利山梁碉卡。又從提督岳鍾琪攻木耳金岡,奪土卡三、水卡一。十四年,大金川事定,芳赴鎮,疏陳考察營汛,修補器械,並以地當黔、楚要沖,密訪侂嚕邪教,復發存庫米折借濟貧兵,上命諸事盡心料理。尋調建昌鎮。敘平大金川功,加左都督。十五年,西藏硃爾墨特、那木扎爾謀叛,既誅,其黨羅布藏扎什等為亂,總督策楞、提督岳鍾琪師入藏,命芳督兵策應。十九年,調松潘鎮,擢貴州提督。二十二年,卒。

查弼納,完顏氏,滿洲正黃旗人。祖愛音布,事世祖為戶部理事官,考滿,授拖沙喇哈番。以其孫觀音保襲,恩詔進三等阿達哈哈番。查弼納,觀音保弟也,襲世職,管佐領。康熙四十七年,授吏部郎中,三遷兵部侍郎。六十一年,授江南江西總督。雍正元年,臺灣硃一貴餘黨溫上貴糾江西棚民掠萬載、新昌。亂定,大學士白潢、尚書張廷玉並疏議安輯棚民,下查弼納詳議。查弼納奏:「江西界連福建、湖廣、廣東諸省,地曠山深,民無力開墾,招流民藝麻種靛。以其棚居,名曰『棚民』。安業日久,驅令回籍,必且生事。當編保甲,千戶以上,駐將吏稽察。編冊后,續到流移,不得容隱。其讀書向學及有膂力者,得入籍應試。」下部議行。二年,奏言私鹽責所在州縣嚴捕,停駐防兵巡緝。又奏言太湖跨數郡為盜藪,請移參將駐洞庭東山,周村、鐵橋、占魚口、馬跡山、黿山、東山、鳳山、吳溜設汛駐兵。又奏言江南賦重事繁,請改六安、太倉、潁、泗、廬、邳、海、通諸州為直隸州,蘇、松、常三府增設元和、震澤、昭文、新陽、寶山、鎮洋、奉賢、金山、福泉、南匯、陽湖、金匱、荊溪諸縣。

上既譴廉親王允禩,以貝勒蘇努、尚書隆科多等結黨亂政,詢查弼納。詔八至,查弼納不以實奏。四年,召詣京師,上親詰之,猶堅執不肯言。命奪官,下王大臣會鞫,乃具言蘇努與阿靈阿、揆敘、鄂倫岱、阿爾松阿結黨,欲戴允禩致大位,及隆科多交結揆敘、阿靈阿狀。王大臣擬查弼納罪斬,上諭曰:「查弼納本後進,畏附權勢。朕昨言及聖祖,查弼納痛器不止,尚有良心,可免其罪。」尋授內務府總管、鑲紅旗漢軍都統,擢吏部尚書,協理兵部。五年,以濫保郎中舒伸,降級。旋授兵部尚書。

七年,師征準噶爾,靖邊大將軍傅爾丹出北路,寧遠大將軍岳鍾琪出西路,查弼納赴肅州督西路軍需。八年,召入覲,授副將軍,佐傅爾丹出北路。九年六月,噶爾丹策零大舉入犯,傅爾丹中敵間,欲及敵未集先發,查弼納亦頗信之。師進,查弼納偕傅爾丹督兵繼之,至庫列圖嶺,入谷遇敵伏,師敗績。移軍和通呼爾哈諾爾,師大潰。查弼納與傅爾丹及副將軍巴賽收餘兵四千,設營護輜重,且戰且行,渡哈爾噶納河。敵追至,查弼納躍馬舞刀潰圍出,與傅爾丹相失,慮以陷帥得罪,曰:「吾罪當死,蒙恩幸得生。頒白之年,豈可復對獄吏?」遂復入陣,死。巴賽亦求傅爾丹不得,趨敵力戰死。巴賽,鄭親王濟爾哈朗孫也,敵旌其黃帶以示師。參贊馬爾薩至紅石巖遇敵,中槍死。

達福,瓜爾佳氏,滿洲鑲黃旗人,鼇拜孫也。康熙五十二年,聖祖追錄鼇拜戰功,賜一等阿思哈尼哈番。達福襲職,管佐領。累擢正藍旗滿洲副都統。雍正五年,世宗以鼇拜功多,復一等公,仍以達福襲,授散秩大臣、前鋒統領。七年,師將出,上召廷臣議,達福力諫。上問故,達福曰:「噶爾丹策零狡黠,能得諸酋心為捍禦。主少則諫易,將強則制專。我數千里轉餉,攻彼效死之士,臣未見其可。」辭益堅,上曰:「今使汝副傅爾丹以行,汝尚敢辭?」達福乃叩首出。師至邊,傅爾丹令達福將二千人駐庫卜克爾。九年,傅爾丹出師,使達福偕定壽領第一隊,及移軍和通呼爾哈諾爾,晝夜力戰,殺敵千餘。敵益大集,軍方移,達福殿,敵三萬餘環攻之,力戰,復殺敵千餘,沒於陣。

定壽,赫舍里氏,滿洲正黃旗人。初襲三等阿達哈哈番世職,授三等侍衛。累遷正黃旗蒙古副都統。康熙五十六年,以傅爾丹為振武將軍,出阿爾泰討策妄阿喇布坦,定壽將盛京、吉林兵千人當前鋒,屢破賊博囉布爾哈蘇、烏魯木齊。雍正二年,授鑲黃旗蒙古都統。策妄阿喇布坦使乞和,定壽率兵還駐巴爾庫爾。部議阿爾泰當駐軍,授定壽阿爾泰駐防將軍。尋改命穆克登,而令定壽以都統銜參贊軍務。四年,率兵往扎布罕,召偕穆克登還京師。定壽奏留察罕蒐勒軍中自效。七年,大將軍傅爾丹自北路出師,命定壽仍以都統銜為軍營前鋒統領。八年,傅爾丹令定壽以二千人駐伊克斯諾爾,護阿濟必濟卡倫。九年,傅爾丹將出師,集諸將議,定壽曰:「噶爾丹策零聞我師至,斂兵觀變,是有謀也。不可信俘言輕進。」傅爾丹責其懦,侍郎永國、副都統覺羅海蘭皆持不可,弗聽,師遂行。以定壽領第一隊,至扎克賽河,獲準噶爾兵二千餘;及至庫列圖嶺,攻不克,將移軍和通呼爾哈諾爾。呼爾哈諾爾,華言大澤也。定壽詰傅爾丹曰:「違眾陷師,誰執其咎?」傅爾丹默不語,定壽曰:「言在先,敢辭死乎?」軍甫移,敵大至,定壽督兵奮擊,所向披靡,乘勝入敵陣,風驟起,雨雹並至,師大敗。敵圍定壽數重,定壽中鳥槍,猶力戰,相持竟夜。敵欲生致之,拔刀自剄,死於陣。副都統西爾賴令索倫兵赴援,兵潰,亦自殺。

素圖,富察氏,滿洲正黃旗人,費雅斯哈孫,素丹子也。素圖初名福列,襲二等阿達哈哈番,改名。授護軍參領。康熙五十四年,策妄阿喇布坦侵哈密,素圖與都統新泰率烏拉兵屯阿爾泰。五十九年,從征西將軍祁里德出布勒罕,深入,斬敵伏四百餘。次鏗爾河,其宰桑色布騰據山拒,素圖督兵奮擊,大破之,色布騰以二千人降。六十年,移軍巴爾庫爾,赴吐魯番督築城屯田。雍正元年,從副將軍阿喇納駐布隆吉爾。二年,準噶爾犯邊,偕總兵孫繼宗擊之,敵敗走,乃城布隆吉爾。復從副都統達鼐逐羅卜藏丹津至花海子,獲臺吉丹津及其妻子,並招降臺吉噶斯等。上以方冬冰凍草枯,師奮勇遠征,下詔褒勉。擢寧夏左翼副都統。時素丹為寧夏將軍,年已老,上命素圖協理將軍。尋命率西安滿洲兵二千從傅爾丹出北路,授參贊大臣。及庫列圖嶺之戰,素圖與副都統岱豪殺敵四百餘。移軍和通呼爾哈諾爾,素圖與定壽及副都統常祿等據山梁之東,敵大至,素圖、常祿與歸化城副都統馬爾齊力禦之,沒於陣。侍郎永國、副都統覺羅海蘭、岱豪帳中自經死。

時諸將惟副都統德祿、承保從傅爾丹得出。伯都訥副都統塔爾岱中槍穿脛,蒙古醫蒙以羊皮,三日始蘇。上令還伯都訥,塔爾岱言:「原從軍剿賊雪恥。若負罪而還,何顏見七十有七之老母?」上深嘉之,並賜塔爾岱及其母各白金千。參贊都統陳泰屯科布多河岸,聞敵至,退駐扎布韓,上命斬之。議恤查弼納、馬爾薩、素圖、覺羅海蘭,皆授拜他喇布勒哈番兼拖沙喇哈番;達福、岱豪、西彌賴、常祿、定壽、永國授拜他喇布勒哈番;餘並授拖沙喇哈番。查弼納、達福、定壽、素圖舊有世職,查弼納合為三等阿達哈哈番,定壽、素圖皆合為三等阿思哈尼哈番,達福以其孫別襲巴賽,追封簡親王,見鄭親王濟爾哈朗傳。

論曰:西藏之師,噶爾弼深入奮戰,而功獨歸主將,番代遠戍,怏怏不欲行,殆以此歟?查克丹與額爾德尼昭之戰,常賚佐巴爾庫爾之師,元生、芳屢定亂苗,而元生尤著,卒以牽制坐使遷延。查弼納易又歷已久,晚乃從軍,和通腦兒之敗,一軍盡覆,而主將獨逭重誅,抑又何也?


\end{pinyinscope}