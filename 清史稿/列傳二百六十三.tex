\article{列傳二百六十三}

\begin{pinyinscope}
循吏一

白登明湯家相任辰旦於宗堯宋必達陸在新張沐張塤陳汝咸繆燧陳時臨姚文燮黃貞麟駱鍾麟崔宗泰祖進朝趙吉士張瑾江皋張克嶷賈樸邵嗣堯立鼎高廕爵靳讓?崔華周中鋐劉棨陶元淳廖冀亨佟國瓏陸師龔鑒

清初以武功定天下,日不暇給。世祖親政,始課吏治,詔嚴舉劾,樹之風聲。聖祖平定三籓之後,與民休息,拔擢廉吏,如於成龍、彭鵬、陳瑸、郭琇、趙申喬、陳鵬年等,皆由縣令洊歷部院封疆,治理蒸蒸,於斯為盛。世宗綜覈名實,人知奉法。乾隆初政,循而勿失。國家豐亨豫大之休,蓋數十年吏治修明之效也。及後權相用事,政以賄成,蠹國病民,亂萌以作。仁宗矯之,冀滌瑕穢。道、咸以來,軍事興而吏治疏。同治中興,疆吏賢者猶能激揚清濁,以彌縫其間。然保舉冒濫,捐例大開,猥雜不易爬梳。末造財政紊亂,新令繁興,簿書期會,救過之不遑。又遷調不時,雖有潔己愛民者,亦不易自舉其職。論者謂有清一代,治民寬而治吏嚴,其敝也奉行故事,實政不修,吏道媮而民生益蹙。迨紀綱漸隳,康、雍澄清之治,邈焉不可見。觀此,誠得失之林也。明史所載,以官至監司為限,今從之。尤以親民為重,其非由守令起家者不與焉。

白登明,字林九,奉天蓋平人,隸漢軍鑲白旗。順治二年拔貢,五年,授河南柘城知縣。時大兵之後,所在萑苻嘯聚。登明治尚嚴肅,擒諸盜魁按以法,境內晏然。憫遺黎荒殘,多方招撫,停止增派河夫,設條以勸耕讀。十年,考最,擢江南太倉知州。釐賦稅,除耗羨,雪諸冤獄,訪察利弊,所摘發輒中。鄰境有冤抑,赴愬上官,輒原下州為理。海濱居民因亂蕩析,登明召民開墾,復成聚落。是年九月,海寇犯劉河堡,登明盡力守御,寇不得逞,遂退。十六年,海寇破鎮江,由江寧敗走,急攻崇明。巡撫蔣國柱治兵策應,欲遣告師期,莫敢前。登明獨駕一艘夜半往,絙城入,眾知援至,守益力,寇乃遁。

劉河北支有硃涇者,宋範仲淹新塘遺跡也,久淤塞。登明請於上官,疏鑿五十里。巡按李森先知其能,復令大開劉河六十里,於是震澤在北諸水悉導入海,旱潦有備,為一郡利。先是寇急時,需餉無出,以雲南協餉應之,卒為大吏所劾落職。州民列治狀請留,弗得,坐廢二十餘年。

康熙十八年,會臺灣用兵,福建總督姚啟聖、巡撫吳興祚素知登明,代為入貲,疏薦,起授高郵知州。值歲旱蝗,繼而大水,湖漲。決清水潭,築堤御之。嚴禁胥吏克減,役者踴躍從事。次年復災,再請蠲賑,勸富民分食,全活無算。時三籓初平,軍檄猶繁。登明與民約,凡供億驛夫,聞吹笳而至,免奪民時。上官有所徵調,不輕給,然皆諒其清廉,亦無相督過者。以積勞卒官,貧無餘貲,州人醵金以殮。入祀名宦祠,鄉民多肖像立祠私祀焉。

時江南以良吏稱者,湯家相、任辰旦、於宗堯,皆與登明相先後云。

家相,字泰瞻,山西趙城人。順治六年進士。八年,授常熟知縣。潔己愛民,釐剔耗蠹,撫恤凋殘,善政具舉。前令被劾逮問,家相左右之,力白其誣,以是忤巡按御史。時江南逋賦數百萬,嚴旨奪各官職,家相坐免。士民爭先輸納,不逾宿而額足,且以治狀訴大吏,請留,勿獲。既而給事中周之桂疏上其事,十三年,起授湖北南漳縣。縣居萬山中,寇盜窟穴,時出肆掠,戕官,人咸危之。家相下車,即令堅壁清野。寇大至,家相謂同城守備曰:「寇眾我寡,當效羅士信破盧明月法,可勝。」密授方略,寇果墮伏中,遂擒其魁黨馬成、孫信輩,斬首數百級。寇大創,遠遁。於是招流亡,修學校,教養兼施,墾田六百餘頃。築永泉、八觀諸堰,民賴其力,邑以大治。疆吏交章薦之,以病乞歸。

辰旦,字千之,浙江蕭山人。順治十三年進士。康熙初,授上海知縣。清苦自勵,敏於聽斷,數決疑獄,豪猾斂跡。催科以時,不大用鞭樸,百姓感其仁,輸納恐後。瀕海防軍將撤,密請行期,故邀軍主歡飲,宣言期須少緩,次日令下,促急行。乃厚其牛酒,道上勞軍,軍無敢遷延他顧,居民帖然。黃龍浦為吳淞江入海要口,建閘屢圮。故事,修閘必築壩,費不貲。辰旦仿浙人為梁法,度基廣狹,約丈尺伐石,識其甲乙,下之水,使善泅者厝之,悉中程。復廣左右護堤,約水就道,十閱月而工成。不病役,不糜帑,邑人頌之。縣田沒水者六千畝,賦額未除,輸者率破家。前令屢勘虛實貿亂,至是巡撫慕天顏疏請復勘。辰旦喜曰:「是吾志也。」日往來泥沙中,按舊冊履丈,釐其荒者,閱二月,費皆自辦,俸不足,出銀釧棉布償之。籍上,得減除額徵有差。十八年,舉博學鴻儒,放還故官。復以良吏薦,入為給事中,論事切直,改大理寺丞。母憂歸。旋以前廷推事詿誤落職,卒於家。

宗堯,字二巍,漢軍正白旗人,廣西總督時曜子。以廕入監讀書。康熙七年,授常熟知縣,年甫十九。興利除弊,勇於為治,老於吏事者勿逮也。時漕政積弊,糧皆民運,往往破家。宗堯議定官收官兌之法,重困得甦。其徵糧則戒期令各自輸,胥吏莫由上下其手,民便之。興文教,戢豪強,救荒療疫,皆以誠懇肫摯出之,四年如一日。以勞致疾,卒於官,年二十有三耳。民為罷市,醵金發喪,遂葬之虞山南麓,題其阡曰「萬民留葬」。

宋必達,字其在,湖北黃州人。順治八年進士,授江西寧都知縣。土瘠民貧,清泰、懷德二鄉久罹寇,民多遷徙,地不治。請盡蠲逋賦以徠之,二歲田盡闢。縣治瀕河,夏雨暴漲,城且沒。禱於神,水落,按故道疏治之,自是無水患。

康熙十三年,耿精忠叛,自福建出攻掠旁近地,江西大震,群賊響應。寧都故有南、北二城,南民北兵。必達曰:「古有保甲、義勇、弓弩社,民皆可兵也。王守仁破宸濠嘗用之矣。」如其法訓練,得義勇二千。及賊前鋒薄城下,營將邀必達議事,曰:「眾寡食乏,奈何?」必達曰:「人臣之義,有死無二。賊本烏合,掩其始至,可一鼓破也。」營將遂率所部進,賊少卻,必達以義勇橫擊之,賊奔。已而復率眾來攻,巨砲隳雉堞,輒壘補其缺,隨方備禦益堅。會援至,賊解去。或言於巡撫,縣堡砦多從賊,巡撫將發兵,必達刺血上書爭之,乃止。官軍有自汀州還者,婦女在軍中悲號聲相屬,自傾橐計口贖之,詢其姓氏里居,護之歸。

縣初食淮鹽,自明王守仁治贛,改食粵鹽,其後苦銷引之累,必達請以粵額增淮額,商民皆便。卒以粵引不中額,被論罷職,寧都人哭而送之,餞貽皆不受,間道赴南昌,中途為賊所得,脅降不屈,系旬有七日。忽夜半有數十人持兵逾垣入,曰:「宋爺安在?吾等皆寧都民。」擁而出,乃得脫。

既歸里,江西總督董衛國移鎮湖廣,見之,嘆曰:「是死守孤城者耶?吾為若咨部還故職,且以軍功敘。」必達遜謝之。既而語人曰:「故吏如棄婦,忍自媒乎?」褐衣蔬食,老於田間,寧都人歲時祀之。越數年,滇寇韓大任由吉安竄入寧都境,後令萬蹶生踵必達鄉勇之制御之,卒保其城雲。

陸在新,字文蔚,江南長洲人。康熙五年,以策論取士,在新夙講經濟,遂得舉,除松江府學教授,教諸生以質行為先,其以金贄者卻之,用不足,知府魯某分俸助之。巡撫湯斌察其廉勤,以卓異薦。是歲江南七府一州諸長吏被薦者獨在新一人,時以此服斌之知人。二十五年,擢江西廬陵知縣,嚴重有威,境內貼然。誓不以一錢自污,錢穀耗羨,革除都盡。傍水設五倉,便民輸納。建問苦亭於衙西,訪求民隱。時裹糧歷山谷間,勞苦百姓,軫其災患而導之於善。召諸生,考德論藝,如為校官時。設四門義學,刻孝經、小學頒行之。二十六年,江溢,民多溺。在新急出錢募民船往救,躬自倡率,出入洪濤中,全活無算。以受前官虧帑盈萬無所抵,憂卒。初赴官時,子孔奐在京師,蹙然曰:「吾父此行,必殉是官矣。」亟從之。卒之日,鬻書數篋以斂。廬陵人為罷市三日,請祀名宦祠,長洲人亦以鄉賢祀之。

張沐,字仲誠,河南上蔡人。順治十五年進士。康熙元年,授直隸內黃知縣。縣苦賦役不均,沐令田主自首,不丈而清。嚴行十家牌法,奸宄斂跡。大旱,自八月不雨至明年九月,民饑甚。沐力籌賑,捐貲為倡,勸富民貸粟,官為書其數,俟秋穫取償,人爭應之,民免轉徙。沐為政務德化,令民各書「為善最樂」四字於門以自警。著六諭敷言,俾人各誦習,反覆譬喻,雖婦孺聞之,莫不欣欣鄉善。五年,坐事免。十八年,以左都御史魏象樞薦,起授四川資陽縣,途出內黃,民遮道慰問,日行僅數里。既抵任,值吳三桂據瀘州,相去數百里,羽檄如織。城中人戶不滿二百,沐入山招撫,量為調發,供夫驛不缺。滇事平,以老乞休。

沐自幼勵志為聖賢,初官內黃,講學明倫堂,請業恆數百人。湯斌過境,與語大悅,遺書孫奇逢,稱其任道甚勇,求道甚切。沐因以禮幣迎奇逢至內黃講學,俾多士有所宗仰。及在資陽,供億軍興之暇,猶進諸生誨導不倦。退休後,主講汴中,兩河之士翕然歸之,多所成就。年八十三,卒。沐之自內黃罷歸也,值登封令張塤興書院,偕耿介同講學,為文紀其事,一時稱盛。

塤字牖如,江蘇長洲人。以官學教習議敘知縣。康熙十七年,授登封縣,單騎之任。途中與登封吏同宿逆旅,吏不知也。至縣三日,拜岳,誓不取一錢,不枉一人。衙前樹巨石鐫曰「永除私派」。設櫃,民自封投,無羨折。招集流亡,督之耕種,相其土宜,課植木棉及諸果實。大修學宮,復嵩陽書院,宋四大書院之一也,延耿介為之師。導諸生以程、硃之學。自縣治達郊鄙,立學舍二十一所。課童子,以時巡閱,正句讀,導之以揖讓進退之禮。間策蹇驢歷諸郊問所苦,有小爭訟,輒於阡陌間決之。西境有呂店者,俗好訟。塤察里長張文約賢,舉為鄉約,俾行化導,澆風一變。里長申爾瑞負課且受杖,路拾人輸稅金,返之,寧受責,不利人財,塤義之,旌其門。鄉民高鵬舉死,妻孟年少,舅欲強嫁之,孟哭夫墓將自縊,塤適微行,問其故,給以銀米勸還家而免其徭,歲時存問,俾終其節。縣故多胥役,時獄訟日尟,奸偽無所容,諸胥多自引去。其更番執事者,退則操耒耜為農,以在官無所得錢也。開萼嶺二百里,復古轘轅路。建古賢令祠,修鄢公墓,崇禎末為令守城抗賊死者也。在官五年,民知向方,生聚日盛,大書「官清民樂」於門。耿介嘗嘆曰:「年來嵩、洛間,別一世界矣!」二十二年,以卓異薦,擢廣西南寧通判。去之日,民遮道哭,立祠於四鄉,肖像祀焉,榜曰「天下清官第一」。至南寧,未幾,乞歸。母喪,服除,赴京師,卒。

陳汝咸,字華學,浙江鄞縣人。少隨父錫嘏講學證人社,黃宗羲曰:「此程門之楊迪,硃門之蔡沈也。」康熙三十年,會試第一,成進士,選庶吉士,散館授福建漳浦知縣。民好訟,嚴懲訟師,無敢欺者。縣中賦役故責戶長主辦,版籍混淆,吏緣為奸。汝咸躬自編審人丁,各歸現籍。糧戶自封投納,用滾單法輪催,以三百戶為一保,第其人口多寡供役。五年一編丁,而役法平。吏胥以不便撓之,大吏搖惑,汝咸毅然不回,奸人無所施技。民樂輸將,賦無逋負。

俗輕生,多因細故服斷腸草死,挾以圖財,力懲其弊,令當刑者掘草根贖罪。禁舁神療病,曉示方證,自制藥以濟貧者。毀學宮伽藍祠,葺故儒陳真晟、周瑛、高登諸人所著書表章之。歸誠書院,乃黃道周講學地,為僧據,逐而新之。無為教者,男女群聚茹蔬禮佛,籍其居為育嬰堂。西洋天主教要大吏將於漳浦開堂,卻止之。修文廟,造祭器,時會邑中士紳於明倫堂講經史性理諸書。設義學,延諸生有學行者為之師。修硃子祠。教養兼施,風俗為之一變。會大水驟漲,幾及城堞,輿錢登城,多為木筏,渡一人與錢三十,人皆以錢助拯,活者數千。多方撫恤,雖災不害。

土寇伏七里洞,將入海,發兵擊之,走山中。密招賊黨,誘擒其渠曾睦等,餘黨悉散。又擒海盜徐容,盡得賊中委曲,赦其罪,責以招撫。諸盜歸誠,海氛遂清。汝咸任漳浦凡十有八年,大吏因南靖多盜,調使治之,縣民請留不得,手冓生祠曰月湖書院,歲時祀之。汝咸至南靖,諸盜自首就撫,開示威信,頌聲大作。

四十八年,內遷刑部主事,擢御史。疏言:「商船出海,掛號無益,徒以滋累。」又言:「海賊入內地,必返其家。下海劫掠,責之巡哨官;未下海之蹤跡,責之本籍縣令;當力行各澳保甲。」會海盜陳尚義乞降,汝咸自請往撫。聖祖命郎中雅奇偕汝咸所薦阮蔡生往,尚義率其黨百餘人果就撫,擢通政使參議。五十二年,奉使祭炎帝神農、帝舜陵,並頒賚駐防兵。遍歷苗疆,審度形勢撫馭之策。歷鴻臚寺少卿、大理寺少卿。五十三年,命赴甘肅賑荒,徒步窮鄉,感疫,卒於固原。漳浦士民聞之,奔哭於月湖書院,醵金置田,歲祀不絕。著有兼山堂遺稿、漳浦政略諸書。

繆燧,字雯曜,江蘇江陰人。貢生,入貲為知縣。康熙十七年,授山東沂水縣。時山左饑,朝使發賑,將購米濟南。燧以路遠往返需日,且運費多,不便。請以銀給民自買,當事以違旨勿聽。燧力爭以因地制宜之義,代草疏奏請,得允。既而帑金不足,傾囊以濟之。洊饑之後,民多流亡,出私錢為償逋欠,購牛種,招徠復業。因捕劇盜已獲復逸,被議歸。尋復官。

三十四年,授浙江定海縣,故舟山也,設治未久,百度草創。海水不宜穀,築塘岸以御咸蓄淡,修復塘碶百餘所,田日增闢。繕城濬濠,葺學宮,建祠廟,役繁而不擾。地瘠民貧,完賦不能以時,逾限者先為墊解,秋後輸還。舊有塗稅,出自漁戶網捕之地,後漁塗被占,苦賠累,為請罷之。地故產鹽,無灶戶,鹽運使屢檄設廠砌盤,官為收賣。燧持不可,請仿江南崇明縣計丁銷引,歲完鹽稅銀四十二兩有奇,著為例。學額多為外籍竄冒,援宣平縣例,半為土著,半令他縣人認墾入籍以充賦。又以土著不能副額,擴建義學,增廩額以鼓舞之,文教興焉。民間日用所需,多航海市諸郡城,關胥苛索,請永禁,立石海關。海嶼為盜藪,隨監司歷勘,凡羊巷、下八、盡山、花腦、玉環、半邊、牛韭諸島,權度要害措置之,盜風頓戢。同歸域者,海上死事諸人瘞骨處,捐貲修葺,建成仁祠,以勸忠義。

歷權慈谿、鎮海、鄞縣及寧波府事,皆有惠政。擢杭州府同知,未任。五十六年,卒於定海。士民援唐王漁、宋趙師旦故事,留葬衣冠,奉祀於義學,名之曰蓉浦書院,蓉浦,燧自號也。遺愛久而不湮,光緒中復請祀名宦祠。燧任定海前後二十二年,賜四品頂戴,賜御書。後雖擢官,迄未離任。時朝廷重守令,循良多久於其職。陳汝咸治漳浦十有八年,陳時臨治汝陽亦二十年。一邑利病,無所不知,視如家事,故吏治蒸蒸日上云。

時臨,字二咸,浙江鄞縣人。少從陳錫嘏學,得聞證人書院之教。家貧,游京師。三籓之變,從軍敘功,授湖南城步知縣。父憂歸,廬墓三年。康熙三十年,起授河南汝陽縣,兵亂之後,風俗大壞,民不知喪禮。時臨為斟酌古今所可通行者,衰絰聚飲之風以息。楊埠有支河,久淤,濬復其舊,民獲灌溉之利。河南諸縣多食蘆鹽,獨汝寧一郡食淮鹽,蘆商欲並之,時臨謂:「蘆鹽計口而授,不問其所需之多寡,以成額給之,是厲民也。吾不能為河南盡革其害,反徇商人意以害境內乎?」力爭得止。巡撫徐潮亟稱之,於是前後諸大吏皆以為循吏當令久任,數報最,數留之。時臨亦與民相安於無事。後擢兵部主事,宦橐蕭寥,臨行,百姓扶老載弱相送數十里。逾年,以病乞歸,卒。

姚文燮,字經三,安徽桐城人。順治十六年進士,授福建建寧府推官。建寧俗號獷悍,以睚眥仇殺者案山積,文燮片言立剖,未數月囹圄為空。有方秘者,殺方飛熊,前令已讞定大闢。文燮鞫得飛熊初為盜,嘗殺秘一家,既就撫,秘乃乘間復仇,不可與殺平人等,秘得活。大吏謂文燮明允,凡疑獄輒委決之。有武弁被殺,株連眾,文燮僅坐數人罪。大吏駭曰:「此叛案,何遽輕率?」文燮曰:「某所據初報文及盜供也。」蓋鄉民逐盜,弁適遇之,從騎未至,為盜所殺而盜逸,營中執為民叛殺弁。文燮檢得初報文,而盜亦獲,自供殺弁,故得其情。

時耿氏建籓,其下多怙勢虐民,貸民錢而奪其妻女。文燮悉使訐發,為捐募代償,贖歸百數。奉檄主丈田事,建寧環郡皆山,民依山鑿田,每陡峻不能施弓繩,文燮授吏勾股法,計田廣狹,增減為畝,區畫悉當。值邊海修戰船,或擬按戶口出錢,文燮上陳疾苦,籌款以代,民乃安。秩滿,報最。康熙六年,詔裁各府推官,去職。

八年,改直隸雄縣知縣。渾河泛溢,浸城,文燮修城築堤,造橋利涉者。邑貢狐皮為民累,條上其弊,獲免。地近京畿,膏腴多圈占為旗產,文燮為民爭之。旗人請於戶部,遣司官至,牽繩量地,繩所及,民不得有。文燮拔刀斷繩,司官見其剛直,詞稍遜。未幾,有旨退地還民。團練屯丁,以資守望,盜賊屏跡。報墾地,蠲耗羨,減鹽引,恤驛政,拊循瘡痏,民慶更生。

擢雲南開化府同知,攝曲靖府阿迷州事。吳三桂叛,文燮陷賊中。密與建義將軍林興珠有約,為賊所覺,被系,乘隙遁,謁安親王岳樂軍中。王以聞,召至京,賜對,詢軍事甚悉。滇寇平,乃乞養歸。

黃貞麟,字振侯,山東即墨人。順治十二年進士。十八年,授安徽鳳陽推官,嚴懲訟師,閤郡懍然。大旱,禱雨未應,貞麟曰:「得無有沈冤未雪,上干天和乎?」於禱雨壇下,立判諸大獄,三日果雨。江南逋賦案興,蒙城、懷遠、天長、盱眙各逮紳民百餘人系獄候勘。獄不能容,人皆立,貞麟曰:「彼逋賦皆未驗實,忍令殭死於獄乎?」悉還其家。及訊,則或舞文吏妄為注名,或誤報,或續完,悉原而釋之,保全者五百家。

河南優人硃虎山,游食太和,發長數寸,土猾範之諫與昝姓有隙,誣以藏匿故明宗室謀不軌。事發,江寧推官不敢問,以委貞麟,貞麟力白其誣。逮至京師復勘,刑鞫無異,乃釋昝姓而治之諫罪。潁州民吳月以邪教惑眾,株連千餘人,貞麟勘多愚民無知,止坐月及為首者。捕人索財於水姓,不得,指為月黨,追至新蔡殺之。鄉人來救,並誣為月黨。撫鎮發兵圍之,系其眾至鳳陽。貞麟廉得實,懲捕而盡釋新蔡鄉人。其理枉活人類如此。旋以他事解官,得白。

康熙九年,改授直隸鹽山知縣,地瘠而多盜,立法牌甲互相救護。有警,一村中半守半援,盜日以息。清里役,逃亡者悉與豁除,不期年,流民復業數百家。十二年,旱,謂父老曰:「大吏使勘災者至,供給惟官是責,不費民一錢。」及秋徵,吏仍以舊額進。貞麟曰:「下輸上易,上反下難。待準蠲而還之,反覆間民必受損。」立令除之。又永革雜派陋例,民皆感惠。內擢戶部山西司主事,山西聞喜丁徭重,力請減之。監督京左、右翼倉,因失察侵盜罷職,卒於家。

駱鍾麟,字挺生,浙江臨安人。順治四年進士副榜,授安吉學正。十六年,遷陜西盩厔知縣。為政先教化,春秋大會明倫堂,進諸生迪以仁義忠信之道。增刪呂氏士約,頒學舍。朔望詣里社講演,訪耆年有德、孝弟著聞者,見與鈞禮,歲時勞以粟肉。立學社,擇民間子弟授以小學、孝經。飭保伍,修社倉。蒞獄明決,所案治即勢豪居間莫能奪,人畏而愛之。縣城去渭不十里,鍾麟行河畔,知水勢將南浸,議自覽家寨迤東開復故道,眾難之。康熙元年夏,大雨,渭南溢,且及城,齋沐臨禱,自跪水中,幸雨止,水頓減,徙而北流者數里。兼攝興平、鄠兩縣,興平豪右分為部黨,前令不能治,廉得其狀,收案以法。奏最,內遷北城兵馬司指揮,復出為西安府同知。

八年,擢江南常州知府。常州、縣賦重,科條繁多,吏緣為奸。鍾麟立法鉤稽清逋,吏受成事而已。屬邑歲例餽漕羨三千金,鍾麟曰:「利若金,如吾民何?」峻卻之。諸漕卒咸斂手奉法。

初,鍾麟在盩厔以師禮數造李顒廬,至是創延陵書院,迎顒講學,率僚屬及薦紳學士北面聽。問為學之要,顒曰:「天下之治亂在人心,人心之邪正在學術。人心正,風俗移,治道畢矣。」鍾麟書其言,終身誦之。已而江陰、靖江、無錫諸有司爭禮致顒,顒為發明性善之旨,格物致知之說,士林蒸蒸向風,吏治亦和。

九年,大水,發倉廩,勸富人出粟賑,民無荒亡。十年夏,大旱,葛衣草履,步禱不應,責躬籥天,言知府不德累民,涕泣並下。尋丁母憂,士民乞留,不可。既歸,連遭父喪,以毀卒。郡人論賢有司知治體必首推鍾麟。先鍾麟守常州者,祖重光、崔宗泰,皆有名。其後有祖進朝,政聲尤著。重光官至天津巡撫。

宗泰,奉天人。順治初,授松江府同知,以敏幹稱。擢常州知府,政尚嚴厲,善鉤距,吏民驚為神明。十三年,大兵征閩,過郡久駐,人情恇擾,宗泰先期儲偫,纖悉備具。有游騎入村落,逐婦女溺水死,宗泰夜叩營門,白將軍縛置之法。時時單騎巡行,遇小有剽奪,隸傳呼「崔太守來」,皆引避去,民得安堵。令甲,府漕以推官監兌,推官懦而衛弁橫。宗泰自請於漕督,檄之監兌,盛騶從,帶刀建臨倉,弁卒悚懼,竟事無譁。尋以事左遷福建延平府同知。後乞免歸。

進朝,亦奉天人。以廕監起家。康熙二十三年,由部郎擢授常州知府,有惠政,以失察鐫級去,士民呼籥於巡撫湯斌,請留進朝。斌上疏言:「進朝履任未一載,操守廉、治事勤,臣私心重之。頃緣失察法寶事降調,常州五縣士民輒號泣罷市,赴臣請留,日不下數千人。臣諭以保留例已久停,士民謂常州四十年未有愛民如進朝者,其減繇輕耗,興學正俗,戢奸除暴,息訟安民,窮鄉僻壤,盡沾惠澤。朝廷軫念東南,如江寧府知府於成龍,特恩超擢,吏治丕變。進朝操守才幹可與成龍頡頏,而獨以一眚被謫,士民攀留,言之泣下,臣不知進朝何以感人之深如此。臣受事四日始獲法寶,是受事之日,已為失察之日,且當候處分,何敢代人瀆奏?惟臣蒙恩簡畀封疆大任,屬吏之敗檢者得糾劾之,廉能者不能為之一言,非公也。民情皇皇如是,而不為之解慰安輯,非仁也。畏罪緘默而使輿情不上聞,非忠也。敢據情陳奏。」章下部議,格不行。聖祖諭曰:「設官原以養民,湯斌保奏祖進朝清廉,百姓同聲懇留,可從所請,以勸廉吏。」進朝復任。未幾,以老疾乞免,民恆思之不置雲。

趙吉士,字天羽,安徽休寧人,寄籍杭州。順治八年舉人。康熙七年,授山西交城知縣。縣居萬山中,地產馬,饒灌木,時禁民間牧馬,停南堡村木廠,民困,往往去為盜。武弁路時運貪而擾民,民殺時運作亂,與大同叛將姜瓖合,連破諸邑。及瓖誅,餘盜匿山中。吉士到官,定先撫後剿之策,有投撫者,給示令招其黨。詗知群盜陰事,選鄉兵,得技優者百人。令紳戶家出一丁,與民均役。分夕巡城,行保甲法,匿賊者連坐,鄰盜相戒不入境。

時交城多抗賦,河北都者賦倍他都。吉士往諭朝廷德意,勖以力耕勿為盜,眾悚息。日暮寢陶穴中聽訟,左右多賊黨,吉士陽若勿知,詰朝深入,察其形勢。最險者曰三坐崖,東西兩葫蘆川繞其下。塞葫蘆口,則官軍不得登。吉士默識之而還。交山賊楊芳林、芳清等時出肆掠,九年春,吉士入山勸農,撫姜瓖舊卒惠崇德,詢得二楊所在,命二卒立擒至,杖系之。賊渠任國鉉、鍾斗等糾眾尾之不敢發。會有陜西叛弁黃某入葫蘆川與國鉉合,吉士謀間之,遣山民持書付國鉉等,偽誤投黃所,黃得書疑國鉉等,率眾去。國鉉等既失黃弁,無所恃,有投誠意。靜樂盜李宗盛踞周洪山,遣其黨趙應龍劫清源,吉士遣惠崇德入山說國鉉等,令獻趙應龍可免罪。國鉉與宗盛紿應龍縛付崇德,應龍恨為所賣,盡發諸盜陰謀。吉士會兵剿宗盛,復遣崇德往說國鉉等使無動,遂擒宗盛,賊黨益渙。

十年,廷旨下總督治群盜,期盡剿絕。吉士曰:「交山劇賊不過十餘人,其它率烏合,一聞盡剿,恐山中向化之民畏罪自疑,反為賊用。今靖安堡初復,請協兵三百以駐防為名,克期入山,可一戰擒也。」靖安堡者,近葫蘆口三十里,昔以屯兵,吉士就廢壘新築之。守備姚順率兵至縣,吉士約期進屯。先期七日置酒大享客,夜半,席未散,吉士上馬會師,疾驅四十里至水泉灘。分三隊,一襲東葫蘆,一襲西葫蘆,自偕姚順進駐東坡底,為兩葫蘆要道。東西賊援並絕,國鉉等為內應,呼曰:「官兵入山矣!」兩葫蘆賊皆走上三坐崖。吉士遣人至崖下語之曰:「汝等良民,毋為賊脅,官且按戶稽丁,不在即以賊論。」眾乃稍稍去,僅存二百餘人。分兵要賊去路,賊四竄,被獲頗眾。分搜巢穴,縱降賊,質其妻子,俾捕他賊以自贖。入山旬有六日,盜悉平。乃召山中民始終不附賊者三十七家,賚以羊酒,立為約正;其素不與徭役者千四百三十家,編其籍入都圖。自後交山無賊患。吉士初患山路險阻,命每都具一圖,鱗比為大圖,召父老詢徑途曲折注之,以次及永寧、靜樂鄰縣諸山。每獲賊,善遇之,因得諸賊蹤跡。上官知其能,不拘以文法,用卒成功。

治交城五年,百廢俱舉,內遷戶部主事,監揚州鈔關,擢戶科給事中。忌者劾其父子異籍被黜,尋補國子監學正。四十五年,卒,祀交城名宦祠。

張瑾,字去瑕,江南江都人。康熙二年舉人。十九年,授雲南昆明知縣。時吳三桂初平,故軍衛田隸籓府者,徵租量豐歉收之,事平沿為額,民不能供。又軍興後官司府署器用皆里下供應,而取給於縣,故昆明之徭,尤重於賦。瑾請於大吏,奏減其賦,不可;乃疆畫荒地,招流亡,給牛種,薄其徵以濟軍衛之賦。一年墾田千三百餘畝,三年得萬餘畝。又均其徭,里蠹無科派,奸民無包收,諸侵漁弊皆絕。民舊供縣公費日十金,瑾曰:「吾食祿於君,不食傭於民。」革之。總督曰:「陳仲子之廉,能理劇乎?」又問:「今家幾何人?」對曰:「子一,客與僕各二。」瞷之,信,皆驚異。自公費除而上之取給者亦減。

昆明池受四山之水,夏秋暴漲,怒流入閘河。沙石壅塞,水乃溢。浸瀕池田,歲勞民力濬之。晉寧州境毗於昆明,受東南諸箐之水,舊跡有河道入江,上官議鑿之以通閘河。瑾按地勢為圖白之曰:「閘河獨受昆明之水,已不能吐納,沙石旁溢為害,豈可更受晉寧水乎?且其地高若建瓴,沙石犖確尤甚,殆不可治。」臺司持之堅,則指圖爭曰:「高下在目,何忍陷民於死!」總督範承勛曰:「令言是也。」議遂寢。

縣有止善、春登、利城諸里田,坳垤錯出,不旱則潦。瑾廉得旁近有白沙、馬裊、清水三河,可資蓄洩,年久湮塞,率民濬治。三月河復,田以常稔。大小東門外舊皆市,兵後為墟,盜賊窟其中。為創造室廬,以居流亡,移城中騾、馬、羊諸市實之。貨廛牧場相比,盜遂絕跡。安阜園者,故籓囿也,請耕之以食孤貧廢疾而無告者。

是時上官多賢者,每倚信瑾。兵備道欲以流民所墾田牧馬,求之期年,不與,久亦稱其直。將軍僕殺人,按察使置酒為請,陽諾之,退而正其罪。巡撫僕子謀奪士人聘妻,即縣庭令士人行合巹禮,判曰:「法不得娶有夫之婦,婦乘我輿,壻乘我馬,役送之歸,有奪者治其罪。」時人作歌詩以傳之。初至,滯獄以百數,斷訖皆當。後一省疑獄輒付瑾治,屢有平反。居三年,病卒。士民圖其像藏之,請祀名宦祠。

江皋,字在湄,安徽桐城人。順治十八年進士,觀政刑部。父病,乞養歸。喪除,授江西瑞昌知縣。故事,歲一巡鄉堡、校戶籍,斂輿馬費,皋罷之。縣城近河,壖岸善崩,屢決改道,環城無隍,民病汲。皋出俸金,率先效力,築堅堤,濬壅塞。水復其故,形勢益壯,民居遂蕃。三籓叛,縣界連湖南,土寇乘間起。皋曰:「吾民緣饑寒出此,迫之則走藉寇」。飭鄉、保長開諭撫安,而密督丁壯巡查,屢擒其魁,盜遂息。居七歲,考最,遷九江府同知,尋擢甘肅鞏昌知府。大軍入蜀,治辦軍需。值歲除,檄徵騾馬千匹,茭芻器具,取具倉猝。皋策畫便宜,供應無缺。士卒驕悍,所過漁奪百姓,皋遇,輒縛送軍主,斬以徇,繇是肅然。

越四歲,調廣西柳州。時新收嶺西,兵猶留鎮。軍中多掠婦女,皋白大吏,檄營帥,籍所掠送郡資遣,凡數百人。軍餉不繼,士譁噪將變,皋馳諭緩期,趣臺司發餉,應期至,軍乃戢。郡民王纘緒,故官家子,經亂,產為四奴所據,只身寄食僧舍。皋詰得之,悉逮捕諸奴。奴懼,納二千金乞免,佯受之。訊伏罪,乃出金授纘緒,命奴從歸,盡還其產,柳人歌誦之。太和殿大工興,使者採木,民大恐。長老言故明採木於此,殭僕谿谷,橫藉不可數。皋曰:「上命也,何敢匿諱!」使者至,令民前導,自控騎偕使者往視。巨木森挺絕巘,下臨深谷。下騎,掖使者攀援以登,崖益峻,無側足所。使者咋舌曰:「是不可取。」還奏免役。民讙呼,戴上恩德。

尋被薦提學四川,以母喪解官。服闋,補陜西平慶道副使,遷福建興泉道參政。以事左遷,旋以恩復職,卒於家。皋於廣西聲績最著。其後稱張克嶷、賈樸。

克嶷,字偉公,山西聞喜人。康熙十八年進士,選庶吉士,改刑部主事,累遷郎中。有獄連執政族人,諸司莫敢任,克嶷請獨任之。內務府以其人出使為辭,克嶷鉤提益急。牒問奉使何地、歸何期,力請部長入告。事雖格,聞者肅然。出為廣西平樂知府,瑤、僮雜居,盜不可詰。克嶷至浹月,以信義服苗酋。獲巨盜二人,斃其一,宥其一,責以偵緝,終其任盜不敢窺。調廣東潮州,屬縣賊蜂起,或稱明裔,聚眾千餘人。克嶷疾馳至其地,命吏士速據白葉祁山,設疑兵,賊不敢逼。會夜半,大風起,簡健卒二百斫其營,呼曰:「大兵至矣!」城中鼓噪出兵以助之,賊奔祁山,要擊之,斬其渠魁三人,眾散乞降。巡撫將上其功,克嶷曰:「此盜耳,而稱明裔,興大獄,株連多,恐轉生變。」乃以盜案結。郡有大豪戕親迎者於路而奪其妻,克嶷微行跡而得之。獄成,當大闢。監司以督撫命為之請,曰:「稍遼緩之,當有以報。」克嶷曰:「吾官可罷,獄不可鬻也。」卒寘諸法。或假親王命以開礦,縛執之。其人出龍牌,克嶷命系之獄,以牌申大府。情既得,立杖殺之。丁父憂歸,遂不出。年七十六,卒。

樸,字素庵,直隸故城人。貢生。康熙二十三年,授廣西柳州同知,有政聲。思明土屬負固抗官,大吏知其能,調任思明治之。夜遣健卒潛入山,焚賊寨,遂出降。署思明知府,土田州岑氏母子相爭,土目陸師等構之以為利,殺人千餘。樸至切諭,母子俱感泣。師等聚眾謀不軌,先懾以兵,單騎往,曉以禍福,乃聽命。建明倫堂,設義學,代完寒士逋糧。民立生祠奉之。擢貴州平越知府,罣誤去官。樸在廣西,嘗條上邊事,巡撫彭鵬奇其才。四十年,詔舉廉吏,鵬特疏薦,授江南蘇州知府。與吏民相見以誠,屏絕請託,政聲大起。四十六年,聖祖南巡,幸蘇州,嘉其清廉為吳中最,擢江常鎮道,吳民數千人遮道請留賢守,御書「宜民」匾額賜之。調蘇松常鎮太糧儲道、布政使參政,仍兼管蘇州府事,從民原也。革四府徵糧例規,積弊一清。忤總督噶禮,摭事劾之,四十九年,去官。留吳門三年,歸里卒。

邵嗣堯,字子昆,山西猗氏人。康熙九年進士,授山東臨淄知縣。有惠政,以憂去。十九年,服闋,補直隸柏鄉。興水利,減火耗,禁差擾,民安之。縣人大學士魏裔介為嗣堯會試座主,家人犯法,嚴治之,不少貸。又有旗丁毒毆子錢家,入縣庭,勢洶洶。嗣堯不稍屈,系之獄,移文都統訊主者,主者不敢承,具論如法。值歲饑,或言勒積粟家出粟,嗣堯曰:「人惟不積粟,故歲饑則束手,吾方蘄令積粟家獲厚利,何勒為?」已而蠲粟者眾,歲不為災。有言開滏陽河通舟楫者,巡撫於成龍使嗣堯往相度,嗣堯力持不可,謂:「此河旱潦不常,未可通舟楫。即或能通,恐舟楫之利歸商賈,挑濬之害歸窮民矣。」事遂寢。

盜殺人於縣界,立捕至,置之法。或毀於上官,以酷刑奪職。尚書魏象樞奉命巡視畿輔,民為申訴,事得白。於成龍復薦之,補清苑。嗣堯益感奮自勵,屢斷疑獄,人以包孝肅比之。二十九年,尚書王騭薦嗣堯清廉慈惠,行取,擢御史。三十年,出為直隸守道,持躬清介,苞苴杜絕。遇事霆發機激,勢要憚之。所屬州縣,肅然奉法。

三十三年,江南學政缺,聖祖諭曰:「學政關系人材,朕觀陸隴其、邵嗣堯操守學問俱優,若以補授,必能秉公校士,革除積弊。」時隴其已卒,遂命嗣堯以參議督學江南。既蒞事,虛衷衡校,論文宗尚簡質,著四書講義,傳示學者。甫試三郡,以積勞遘疾卒。身無長物,同官斂貲致賻乃得歸葬。士民思之,為立祠肖像以祀焉。

聖祖澄清吏治,拔擢廉明,近畿尤多賢吏,如彭鵬、陸隴其及嗣堯,當時皆循名上達,聞於天下。鵬及隴其自有傳。又有衛立鼎、高廕爵、靳讓,治績亦足媲美。

立鼎,字慎之,山西陽城人。康熙二年舉人,授直隸盧龍知縣。地當兩京孔道,驛使旁午,供張糗崿,悉自營辦,不以擾民。先是縣中徵糧,勺杪以下,皆用升合量。納草以銀代,仍抑價買諸民間。立鼎令輸戶含納奇零,統歸斛斗,徵草則以本色輸,民甚便之。興行教化,獎拔士類,丕變其俗,尤以清廉著稱。尚書魏象樞及侍郎科爾坤奉命巡畿內,至盧龍,已治具,不肯食,僅啜一甌。曰:「令飲盧龍一杯水耳,吾亦飲令一杯水。」諸大獄悉以咨之,立鼎引經準律,象樞大稱善。於成龍之巡撫直隸也,嘗迎駕於霸州,奏舉循吏,以立鼎、陸隴其並稱。嗣巡撫格爾古德以事至盧龍,謂立鼎曰:「令之苦,無異秀才時。秀才徒自苦,今令苦而百姓樂,非苦中之樂乎?」疏薦立鼎治行第一,靈壽令陸隴其次之。內遷戶部郎中,秩滿授福建福州知府,以年老致仕歸。教授鄉里,以倡論道學為事。年七十有六,卒。

廕爵,字子和,奉天鐵嶺人,隸漢軍。康熙初,謁選,授直隸蠡縣知縣。縣多旗屯,居民田之半,佃者倚勛貴為奸利,持吏長短。河數決孟嘗村,歲比不登,民大饑。廕爵至,曰:「吾未暇理他政,且活民。」倉有粟二萬石,請發以賑。牘再上,不許;請解官,乃許之五千石。廕爵曰:「若今歲又惡,民不能償,二萬石、五千石等死耳,吾且活吾民。」乃盡發之。更出帑五百金貸民種麥。夏旱,蝗起,捕蝗盡。秋又大霖雨,河暴溢,率吏民冒風雨捍禦,堤完而歲大熟,民乃安。某甲以財雄諸佃,多為不法,誣諸生為奴,而籍其田。按治得實,置之法。豪猾心習服,莫敢犯令。於是設義倉,置鄉學,尊禮賢士,民大和悅。調三河,一以簡易為治。或問之,曰:「前令已治矣,何紛更為?」前令,彭鵬也。聖祖校獵至三河,問父老:「高令與彭令孰賢?」對曰:「彭廉而毅,高廉而和。」上稱善,擢順天府南路同知。於成龍問以捕盜方略,條上三事,略言:「盜以旗屯為逋逃藪,請嚴保甲首實之令,使無所匿,而平日能使之衣食粗足,則可不至為盜。」成龍韙之。會丁父艱歸。成龍總督南河,築界首堤,以屬廕爵。堤成,上南巡閱工,召見,賜克食。起復補湖北德安府同知,累擢四川松茂道、直隸口北道,皆有惠政,卒於官。子其倬,官至大學士,自有傳。

讓,字益庵,河南尉氏人。康熙十八年進士,授浙江宣平知縣。旱災,請蠲甚力,巡撫張鵬翮以為賢。父憂去,服闋,授山西汾西。會親征漠北,供張杜絕擾累,民力不足,請以正賦辦治。行取,擢御史,數上疏言察吏安民,實行教養。聖祖諭曰:「朕御極四十年,惟冀天下黎庶盡獲安全,邊疆無事。如靳讓所言,必令家給人足,無一人凍餒,此非朕所可必者,恐其不過徒為大言。曩者錢鎯、衛既齊亦曾為此言,及後用為大吏,皆不能自踐其語。靳讓曾為縣令,其所為能如是乎?通州驛馬事繁,著調為通州知州,果能如所言,朕即超用。」上意欲試之也,許其便宜啟奏。讓布衣羸馬之官,皇莊、旗莊恣肆病民,繩以法,不少貸。私錢、私鑄悉禁止。時禁河捕魚,誣累平民,讓分別治之。奸商藉權貴勢,謀專賣麥豆及設姜肆牟利,並拒絕。上聞,皆韙之。會學政更替,命九卿舉所知。上曰:「朕亦舉一人。」命以僉事督學廣西。逾年,調浙江,除弊務盡,教士先德行而後文藝。值南巡,召對,褒獎曰:「汝不負朕舉,朕將用汝為巡撫。」讓以母老乞終養,賜御書「天庥堂」額以榮其母。尋母喪,以毀卒。

崔華,字蓮生,直隸平山人。順治十六年進士。康熙六年,授浙江開化知縣。政務寬平,建塾校藝,士爭鄉學。縣舊有里總,主賦稅,橫派滋擾,除之。又以虛糧為累,請豁於上官,未竟其事。十三年,耿籓亂作,縣南墾戶多閩人,豎旗以應,城守千總吳正通賊,陷城,露刃相逼。華從間道出,檄召十六都義勇鄭大來、夏祚等,涕泣開諭,立聚萬人,躬冒矢石,閱五日,城遂復。總督李之芳上其事,詔嘉之。

時閩寇方熾,分三路犯浙。衢州當中路之沖,縣城再陷,慘掠尤甚,民無叛志。華率兵退保遂安,圖恢復,時出有所擒斬。大兵扼衢州,久與賊持。十五年春,始遣將由遂安復開化,至秋,大破賊軍。浙境漸清,流亡初集,積逋尤多。華圖上遺黎困苦狀,乞為請命,盡蠲十三年至十六年額賦。贖民之流徙者,俾得完聚。疫癘盛行,廣施藥餌,全活無算。

先後論功,十九年,擢江南揚州知府。值湖、河並漲,屬縣被災者眾,華加意撫恤。二十三年,命九卿舉中外清廉之吏,廷推七人,外吏居其三,華為首焉。擢署兩淮鹽運使,軍興商困,乃權宜變通,令先行鹽、後納課,務與休息,商力甦而賦亦無缺。先是湖南諸府因兵蠲引三十九萬有奇,至是有請補行蠲引者。華以兩淮浮課重,又帶加斤,若補蠲引,必致額售者滯銷誤課,力言不便,事得寢。三十一年,遷甘肅莊涼道,未行,卒。淮商祠祀之。

周中鋐,字子振,浙江山陰人。康熙中為江南崇明縣丞。崇明故重鎮,兵籍千人,欲預取軍食於官,不獲,彀刃譁噪。官吏咸避匿,中鋐獨挺身前,宣布順逆利害,感切聳動,眾皆投械散。擢華亭知縣,民有被誣殺人久系獄,中鋐立出之,而坐其實殺人者。提標兵庇盜,前令莫敢問,中鋐捕治置諸法,境內乂安。四十三年秋,大霪雨以風,海水驟溢,漂數縣。乃具衣糗棺槥救恤之,又為請賑蠲租,活民甚眾。雍正四年,以催科不及格罷,縣民萬數遮言,上官聞於朝,得復職。

"

時左都御史硃軾被命修海塘,知中鋐賢,悉以事付之。塘成,丁母憂,民復籥留,中鋐先已擢松江知府,至是予假治喪,還視府事。五年,議濬淞、婁諸水,以中鋐署太倉知州,董其役。六年二月,築壩於陳家渡,一再潰,與千總陸某晝夜冒險指揮,倉卒覆其舟,既歿而築合。事聞,贈太僕寺少卿。

當中鋐令華亭時,奉賢猶隸境內,後析為縣,中鋐適為知府,至是民懷其澤,奉以為奉賢城隍之神,歲時祈報,著靈異,長洲王芑孫為廟碑紀其事。道光七年,巡撫陶澍復濬吳淞江,疏請立廟江幹。

劉棨,字弢子,山東諸城人。康熙二十四年進士。三十四年,授湖南長沙知縣,以廉明稱。時訛言裁兵,撫標千人環轅門大噪,棨為開陳大義,預給三月餉,示無裁意,眾乃定。總督吳琠以循良薦之。三十七年,擢陜西寧羌知州。關中大饑,漢南尤甚。州無宿儲,介萬山中,艱於輓運。棨請貸鄰邑倉粟,約民能負一斗至者予三升,不十日輓三千石。大吏下其法賑他邑,咸稱便。又奉檄賑洋縣,移粟沿漢而下。棨先遍歷審勘,克期給發,數日而畢。謂洋令曰:「此粟貸之官,倘民不能償,吾兩人當代任。」比秋大熟,洋縣民相勉還粟,不煩催督。

始寧羌地苦凋瘵,棨為均田額,完逋賦,補棧道,修旅舍。安輯招徠,期年而廬舍萃集。山多槲葉,民未知蠶,遣人旋鄉里,齎蠶種,募善蠶者教之,人習其利,名所織曰「劉公綢」。士苦無書,為召賈列肆,分購經籍,建義塾,親為講解。

四十一年,擢甘肅寧夏中路同知,未赴,母憂去。以代民完賦,負累不能行,囑弟代售遺產,不足,弟並以己產易金償負。民聞之,爭輸金為助,卻不受。服闋,補長沙府同知。入覲,奉溫旨,試文藝於乾清門,即日擢山西平陽知府。裁汰陋例,蠲除煩苛,訟牘皆立剖決之。四十八年,九卿應詔舉廉能吏,以知府被舉者,惟棨與陳鵬年二人。

四十九年,擢直隸天津道副使,迎駕澱津,詔許從官恭瞻親灑宸翰。棨因奏兄果昔官河間知縣,奉「清廉愛民」之褒,乞賜御書「清愛堂」額,上允之。歷江西按察使、四川布政使。五十五年,上詢九卿,本朝清介大臣數人,求可與倫比者。九卿舉四人,棨與焉。駕幸湯泉,又以棨治狀語諸從臣,會廷推巡撫,共薦棨,上嘉納之。以四川用兵,未輕調。五十七年,卒於官。

兄果,官山西太原府推官,有聲。改河間知縣,康熙八年,駕幸河間,問民疾苦,父老陳果治狀,召見褒之。卒,祀名宦。棨子統勛、孫墉、曾孫鐶之,並為時名臣,自有傳。

陶元淳,字子師,江蘇常熟人。康熙中舉博學鴻詞,以疾不與試。二十七年,成進士,廷對,論西北賦輕而役重,東南役均而賦重,原減浮額之糧,罷無益之費。閱者以其言戇,置二甲。三十三年,授廣東昌化知縣,到官,首定賦役,均糧於米,均役於糧。裁革雜徵,自坊裏供帳始,相率以力耕為業。縣隸瓊州,與黎為界,舊設土舍,制其出入,吏得因緣為奸,元淳立撤去。一權量,定法度,黎人便之。城中居人,舊不滿百家,至此戶口漸蕃。元淳時步行閭里間,周咨疾苦,煦嫗如家人。

瓊郡處海外,軍將多驕橫,崖州尤甚。元淳嘗署州事,守備黃鎮中以非刑殺人,游擊餘虎縱不問;且貪,索黎人獻納。元淳廉得其狀,列款以上,虎私以金賄之不得,造蜚語揭之。總督石琳下瓊州總兵會訊,元淳申牘曰:「私揭不應發審,鎮臣不應侵官,必挫執法之氣,灰任事之心。元淳當棄官以全政體,不能蒲伏武臣,貽州縣羞也。」初鞫是獄,鎮中令甲士百人佩刀入署,元淳據案怒叱曰:「吾奉命治事,守備敢令甲士劫持,是藐國法也。」鎮中氣懾,疾揮去,卒定讞,論罪如律。崖人為語曰:「雖有餘虎,不敵陶公一怒。」而總督卒因元淳倔彊,坐不檢驗失實,會赦免。復欲於計典黜之,巡撫蕭永藻初授事,曰:「吾初下車,便劾廉吏,何以率屬?」為言於總督,乃已。

元淳自奉儉約,在官惟日供韭一束。喜接諸生,講論至夜分不倦。屢乞病未果,竟以勞卒於官。昌化額田四百餘頃,半淪於海,賦不及二千,浮糧居三之一,民重困。元淳為浮糧考,屢請於上官,乞豁除,無應者。乾隆三年,元淳子正靖官御史,疏以入告,竟獲俞旨免焉。

廖冀亨,字瀛海,福建永定人。康熙二十九年舉人,四十七年,授江蘇吳縣知縣。值歲旱,留漕賑饑,不足,自貸金易米以濟。士人感其誠,相率捐助,賑以無乏。吳中賦額甲天下,縣尤重,冀亨減火耗,用滾單,民皆稱便。知收漕弊多,拘不法者重治之,凡留難、勒索、蹋斛、淋尖、高颺、重篩諸害,埽除一清。太湖中有蘆洲,或墾成田,或種蓮養魚,官吏輒假清丈增糧名以自利。冀亨曰:「湖蕩偶爾成田,未可久持,今增其賦,朝廷所得幾何,而民累無盡期。」一無所問。初,冀亨蒞任時,有吳人語之曰:「吳俗健訟,然其人兩粥一飯,肢體薄弱,凡訟宜少準、速決,更加二字曰『從寬』。」冀亨悚然受之。收詞不立定期,民隱悉達。嘗自謂訟貴聽,聽之明,乃能速決而無冤抑。在吳三年,非奸盜巨猾,行杖無過二十,蓋守此六字箴也。

有庠生授徒鹽商家,自刎死,勘得實。或有謗其受賄者,冀亨無所避,卒釋鹽商勿罪。東山巡檢報鄉人弒父屠嫂,未遂,自盡。冀亨方秉二燭閱其詞,燭無風齊滅,知有冤。克日渡湖往驗,大風,舟幾覆,從者色變。冀亨曰:「縣官伸冤理枉而來,神必佑之,何懼!」須臾抵岸。訊得父故殺狀,巡檢得賄誣報,俱論如律。

冀亨既有聲於吳,他縣疑獄,往往令推治。會有宜興知縣誣揭典史故勘平民為盜,刑夾致死,冀亨奉檄按驗。知縣者總督噶禮之私人也,或告宜少假借,冀亨不為動。檢踝骨無傷,原揭皆誣。獄上,噶禮屢駁詰。再三審,卒如冀亨議,以是忤總督。時巡撫張伯行以清廉著,深契冀亨,布政使陳鵬年尤重之;而噶禮不懌於伯行,尤惡鵬年。四十九年,鵬年被劾,並及冀亨,以虧帑奪職。逾年,噶禮敗,冀亨始復原官,以病不赴選。及卒,吳人祀之百花書院。

冀亨歿後,家留於吳,入籍嘉定。曾孫文錦,嘉慶十六年進士,由翰林出為河南衛輝知府,有惠政,祀名宦。文錦子惟勛,道光十三年進士,亦由翰林為貴州鎮遠知府,撫苗有法,終貴陽府。

佟國瓏,字信侯,奉天人,隸漢軍籍。康熙三十年,由筆帖式授山東文登知縣。縣俗愚悍,有勸治宜嚴峻者。國瓏曰:「為政在誠心愛民,興利除害,化導之而已,嚴峻非民之福也。」副將某以暱妓蝕餉,軍大噪,夜半斬關出屯東郊。國瓏聞變,單騎往諭曰:「吾與軍民同疾苦,有冤當訴我,何妄動為?」眾猶洶洶,國瓏當砲立,曰:「吾不忍見爾曹族誅,請先試若砲。」眾動色,曰:「公廉明,軍何敢犯,然事已至此,奈何?」國瓏力任保全。究其故,得實。縛妓手失之,眾泣拜而散,副將尋被劾去。

歲饑,奸民騷動,國瓏歷村墟,給賑撫諭,捕治兇渠,民賴以安。邑豪宋某以鄰婦貸錢不償息殺之。吏役得賂,皆為豪掩,又以千金賄國瓏。國瓏怒,覆驗婦有重傷,鞫得其情,置豪於法。邑故瀕海,副將林某縛商舶之泊島嶼者數千人,指為寇,國瓏訊釋之,別捕誅真盜四十餘人。

五十年,擢山西澤州知州。歲祲,發常平倉以貸民,克期輸還無爽。又減耗羨,革陋規,省徭役,平物價,民情大悅。國瓏嘗以論事忤太原知府某,某嗾人誣揭之,坐罷任。州民鳴鐘鼓罷市,欲詣闕。既而得白,留原任。時平陽民變,巡撫檄國瓏以兵往,國瓏曰:「是速之亂也。」單騎馳赴,民皆額手曰:「佟公至,吾屬無慮矣!」乃安堵受撫。五十九年,以疾乞免。後以所屬高平令虧帑被逮,責償萬金,民感其惠,捐金投州庫代償其半雲。

陸師,字麟度,浙江歸安人。少負文名。康熙四十年進士,授河南新安知縣。修學校,集諸生治經,童子能應試者免其徭,民興於學。響馬賊季國玉為患久,捕誅之。巡鹽使者下縣,取鹽犯四十人。師曰:「律以人鹽並獲始為犯,今勘犯止二人,何濫為?」父憂歸,在途,有六七騎挾弓矢,驅牛車,載婦女三十餘人,言歸德饑民,某將軍買以歸者也。師叱止之,令官還婦女於其家,白將軍收其騎卒。或謂已去官胡忤將軍,師曰:「知縣一日未出境,忍以饑民婦女媚將軍耶?」

服闋,補江蘇儀徵。有盜引良民為黨,師親馳往捕,見壞器滿地,言有暴客食此不償值,因而鬥毀。詰其人,狀與盜肖,事得白。春徵,勸富戶先輸,秋則減其耗,令自封投櫃。故事,驛夫臨時取給鋪戶,倉猝滋擾。一切禁革,但令戶日賦一錢歸驛,商賈以安。揚州五縣饑,大吏令縣各以五千金糴穀備賑,具舟車往,則虛而歸。師察知府意欲縣官借補所虧也,力爭,於是五縣皆得穀以賑。

卻鹽商例餽,固請,乃籍其入以修學宮,具祭器樂舞,浚泮池,植桃李其上。修宋文天祥祠,又以其餘建倉廒,潔治囹圄。質庫書票,故有月無日,勿論久近,皆取一月息。師辭其歲餽,令視他縣月讓五日。舊有豬稅,下令蠲除之。

課最,行取擢吏部主事,升員外郎。掌選,有要人求官,力持不可。督山東礦務,條上開採無益,罷其役。還,擢御史,巡河、讞獄皆稱職。康熙六十一年,河督陳鵬年疏請以師為山東兗沂曹道,未到官,卒。祀名宦祠。

龔鑒,字明水,浙江錢塘人。早與同郡杭世駿齊名。雍正初,以拔貢就選籍,授江蘇甘泉知縣。縣新以江都析置,故脂膏之地,鑒恥為俗吏,一以子惠黎元、振興文教為己任。故某侍郎子與有舊,入謁,有所囑,拒之。有同城官為大吏所暱,令伺察屬吏者,有挾而請,又拒之;巨室延飲,又拒之。於是大江南北盛傳甘泉令不近人情,鑒益自刻苦,無一長物。

縣境邵伯埭受高、寶諸湖之水,地卑下。鑒謂當於農隙運土築高埂沿堤為防,以徐議溝洫。堤上即植桑,興蠶事。其西境地高,浹旬不雨即龜坼,宜每一里為水塘以蓄之。如是則高下之田俱無患。大吏韙之,然不能行。邵伯埭下有芒稻河,設閘洩水尤要。值大水泛溢,鑒冒雨至,呼閘官洩之。閘官以鹽漕為言,不可。會總河嵇曾筠視河至,鑒直陳,厲聲訶閘官,曾筠即令啟閘。又用鑒言,定鹽漕船過湖需水不過六尺,過即啟閘,無得藉口蓄水,為民田患。每歲晏,江都之鰥寡孤獨多入甘泉部中。

西湖聖因寺僧明慧者,恃前在內廷法會恩寵,干求遍於江、浙。一日以書幣關白,鑒杖其使而遣之。事流傳,上聞。世宗召明慧還京,錮不許出。當是時,甘泉令聲聞天下。在任六年,以父憂去官,貧,至無以葬。河南巡撫尹會一故為揚州守,雅與鑒善,招之,欲使主大梁書院,以修脯助葬。遂卒於河南。

鑒湛深經術,能摘先儒之誤,顧書多未成。所成者毛詩疏說,闡明李光地之說為多。


\end{pinyinscope}