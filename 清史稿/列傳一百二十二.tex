\article{列傳一百二十二}

\begin{pinyinscope}
富僧阿伊勒圖胡貴俞金鼇尹德禧剛塔

富僧阿,舒穆祿氏,滿洲正黃旗人。雍正初,授拜唐阿,累遷頭等侍衛。出為副都統,歷成都、三姓、寧古塔諸地。擢將軍,自荊州移黑龍江。黑龍江北鄰俄羅斯,康熙二十九年與定界。歲久,將吏憚行邊,道里不能詳。富僧阿遣副都統瑚爾起等分探諸水源,皆至興堪山還報。乃上疏言:「副都統瑚爾起探格爾畢齊河源,自黑龍江至格爾畢齊河口,水程一千六百九十七里;自河口行陸路二百四十七里,至興堪山:其間無人跡。協領納林布探精奇哩江源,自黑龍江入精奇哩江,北行至托克河口,水程一千五百八十七里;自河口行陸路二百四十里至興堪山:地苦寒,無水草禽獸。協領偉保探西里木第河源,自黑龍江經精奇哩江入西里木第河口,復過英肯河,水程一千三百五里;自英肯河口行陸路一百八十里至興堪山:地苦寒,無水草禽獸。協領阿迪木保探鈕曼河源,自黑龍江入鈕曼河,復經西里木第河入烏默勒河口,水程一千六百十五里;自河口行陸路四百五十六里至興堪山。諸地俱無俄羅斯偷越。臣按呼倫貝爾有額爾古訥河,西為俄羅斯界,東屬我國。自此至珠爾特,處處設卡。今復自珠爾特至莫哩勒克河口,設卡二,索博爾罕增立鄂博,逐日巡查。俄羅斯、鼐瑪爾斷難偷越。黑龍江與俄羅斯接壤,興堪山延亙至海。嗣後請飭打牲總管每歲六月遣章京、驍騎校、兵丁,自水路與捕貂人同至托克、英肯兩河口,及鄂勒希、西里木第兩河間,巡察還報;三年遣副總管、佐領、驍騎校於冰解後,自水路至興堪山巡察還報;黑龍江官兵每歲巡察格爾畢齊河口,三年亦至興堪山巡察還報:歲終報部。」上從之。

富僧阿治事嚴,嘗疏請罪人予官兵為奴,並其妻子皆令為奴;又以遣犯脫走,出巡並將校婪索,皆請逮送刑部:上不許。移西安將軍,西安、寧夏移駐滿洲兵,復分駐巴里坤,富僧阿議定規制,皆如所請。乾隆四十年三月,卒官。

伊勒圖,納喇氏,滿洲正白旗人。乾隆初,以世管佐領授三等侍衛,累遷鑲紅旗蒙古副都統。出駐烏魯木齊,移阿克蘇。三十二年,授伊犁參贊大臣,移喀什噶爾。內擢理籓院尚書,外授伊犁將軍。三十四年,師征緬甸,授副將軍,從經略大學士傅恆分道進軍,緬甸人拒戛鳩江,築寨。伊勒圖偕參贊大臣阿里袞與戰,奪寨三,殺賊五千餘。師還,授兵部尚書。復外授伊犁將軍。土爾扈特汗渥巴錫、臺吉策伯克多爾濟等率所部三萬餘戶來歸,先期使至伊犁,具書通款。伊勒圖以聞,高宗命加意撫綏,俾得所。於是土爾扈特部悉內附,哈薩克、布魯特兩部厄魯特降者日眾。伊勒圖請增置佐領,俾領其眾,從之。三十六年,左授參贊大臣,駐烏什,移塔爾巴哈臺。三十八年,復授伊犁將軍。兵部議禁鳥槍,伊勒圖以土爾扈特部新歸附,牧馬御豺虎恃鳥槍,不當一體收禁。四十八年,加太子太保,賜雙眼花翎。五十年七月,卒,謚襄武,封一等伯,祀賢良祠。發帑金千,遣侍衛豐伸濟倫如伊犁賜奠。

伊勒圖在邊二十餘年,諸所經畫,縝密垂久遠。其在塔爾巴哈臺受代去,上諭繼任參贊大臣慶桂循其規制。鎮伊犁尤久,伊犁屯田,請兵得攜妻子。於塔爾奇溝口外烏可爾博蘇克、東察罕烏蘇、霍爾果斯、巴彥岱諸地築城堡,水足地厚,俾得久屯。設寶伊局鑄錢,採哈爾哈圖銅礦,三年得九千餘斤,令加鑄,於烏什鑄普爾。烏什及庫車、哈喇沙爾諸城與伊犁錢並用,普爾,回錢名也。又於崆郭羅鄂博諸地採煤,聽商人充窯戶,徵其稅。都統海祿請令遣犯皆入鐵廠,與罪人畀官兵為奴者同例。伊勒圖請仍如舊制,使遣犯與為奴者有別。其卒,上稱其鎮靜妥協,各部落皆心服,封恤特厚。

胡貴,字爾恆,福建同安人。少有智略。入伍,稍遷水師提標右營千總。雍正六年,齎奏入都,世宗召入見。再遷後營游擊。監修戰艦,出巡海,坐誤工,吏議當左授,上特宥之。累遷江南蘇松鎮總兵。督運漕糧十萬轉海賑福建,道溫州鳳凰洋,颶作,損米五百餘,請出私財以償。高宗諭曰:「冒險已可嘉,豈有復令出私財償米之理?」命罷勿償。旋坐廢弛當奪職,復特宥之。疏言:「本鎮春、秋兩哨,中營游擊司糧餉,奇兵營游擊職城守,例不出巡。惟既任水師,當知海道,應從眾出巡。陸路將士原改水師者,先令出海演試,如有膽略,量為改補。」並從所請。崇明海漲,沒民廬。召縣吏議賑,吏言當待請。貴曰:「民死在頃刻,豈能俟報?有譴吾任之。」即發倉以賑,令所屬為助,眾有難色,貴曰:「設官非以衛民乎?賑不周,生它變,豈能免患?」疏請發帑金十八萬、倉穀二十八萬,並留漕米續賑,上深嘉之。歷廣東潮州、瓊州諸鎮,擢提督。增城民王亮臣為亂,貴勒兵馳赴,分遣所屬防隘,扼賊走路。總督阿里袞軍亦至,分道捕治,諸賊皆就擒。以失察自劾,貸勿問,仍敘勞。入覲,賜花翎。移福建水師提督,復自浙江還廣東。乾隆二十五年,卒,謚勤愨。子振聲,附李長庚傳。

俞金鼇,字厚菴,直隸天津人。乾隆七年武進士,授藍翎侍衛。以守備發山東,累遷甘肅肅州鎮總兵。命如伊犁董理屯田,歲豐,伊犁將軍伊勒圖奏綠營兵二千二百名,人穫米二十八石有奇。得旨,敘勞。移巴里坤總兵,擢烏魯木齊提督,仍領屯田事。奏請移沙州副將駐安西,巴里坤迤西至瑪納斯,擇有水草地設墩塘,皆議行。時令移軍戍烏魯木齊及瑪納斯,得挈妻子以往,謂之「眷兵」。金鼇請具一歲糧,亦從其請。歷江南、福建、甘肅諸省提督。固原回李化玉與河州回田五糾眾為亂,攻靖遠,金鼇與涼州副都統圖桑阿合軍討之,逐賊馬營街,固原提督剛塔亦以師來會,多所斬獲。土司楊宗業以土兵助戰,賊憑山設拒,土兵敗走,金鼇擊賊退。賊夜走石峰堡,糾會寧諸回,勢復張,副都統明善戰死。金鼇進次烏家坪,擊賊,斃頭人三,擒二十有九。轉戰至秦安土鼓山,賊敗竄蓮花城,師從之,至於雙峴,從總督李侍堯自中路進攻,敗之。福康安督兵剿石峰堡,令金鼇防底店護運道。

回亂定,移湖廣。復移直隸,未行,鳳凰苗石滿宜糾眾為亂,金鼇聞報馳赴,令鎮筸鎮總兵尹德禧督軍破賊寨,生致其渠。上以金鼇習苗疆事,命仍留湖廣。臺灣林爽文為亂,命德禧將湖北兵二千以往,金鼇出駐鳳凰鎮苗疆。旋入覲,命在乾清門行走,賜紫禁城騎馬。引疾乞罷,上以金鼇有勞,下總督畢沅察病狀,乃加左都督,允解官歸。旋卒。

金鼇嘗預千叟宴,高宗賜之酒,命賦詩記事,金鼇辭不能詩。上顧笑曰:「汝為香樹妻弟,又從受業,豈不能詩者?」香樹,錢陳群字也。官湖廣,和珅已柄政,欲納交焉。金鼇謝不可。

尹德禧,鑲黃旗包衣人,初名色喀通額。以領催從征伊犁,遷至防禦。開戶出旗,更姓名,改籍直隸密雲縣。從征金川,復六遷至總兵。石滿宜據句捕砦為亂,德禧破砦獲滿宜,賜花翎。上詰德禧:「當苗亂,何不專摺奏?」德禧請罪,命貸之。搜捕滿宜餘黨,苗疆悉定。其出師臺灣,師至,爽文已就俘,福康安令德禧屯竹仔港防賊逸。臺灣定,召入見,令署湖南提督。卒,遺言請還旗籍,復隸鑲黃旗包衣。

剛塔,烏濟克忒氏,滿洲正藍旗人。初充前鋒,從征準噶爾,授雲騎尉世職。三遷直隸泰寧鎮中營游擊。從克臨清,山東巡撫楊景素奏留山東。四遷直隸提督,兼領馬蘭鎮總兵。移陜西固原提督。乾隆四十九年,鹽茶小山回田五糾眾為亂,攻破安西州。剛塔督兵逐賊,殺賊數十,射殪乘馬賊渠,賜上用玉韘、大小荷包。復逐賊至浪山,田五戰被創,自殺。其徒竄據馬家堡,剛塔督兵合圍,賊夜出堡逾山遁,環壘樹木桿,懸衣帽其上,紿官軍,官軍逼壘,乃知賊已走。剛塔督兵逐賊,戰於馬家灣,剛塔中矢。復進至馬營街,殺賊數十,得級二十五。賊攻陷通渭,其徒分據石峰堡。西安副都統明善攻之,沒於陣。上以師無功,令大學士阿桂、尚書福康安出視師。上謂馬營街、石峰堡皆通渭地,剛塔方逐賊馬營街,通渭陷不赴援,明善又以攻石峰堡戰死,詔詰責。剛塔疏言:「獲賊言將自通渭道伏羌、秦州天潼關。」上責剛塔信賊妄語搖軍心,令福康安傳諭,奪剛塔職,逮送京師。上方幸熱河,留京王大臣等讞當斬,上以剛塔殲賊渠田五,戰馬家灣身被創,貸死,戍伊犁。卒。

論曰:富僧阿鎮黑龍江,察國界,定巡徼之制。伊勒圖鎮伊犁,徠屬部,著拊循之績。建威銷萌,邊帥之職舉矣。貴定增城,金鼇、剛塔攻石峰堡,名位顯晦殊,要不可謂無功也,故類次焉。


\end{pinyinscope}