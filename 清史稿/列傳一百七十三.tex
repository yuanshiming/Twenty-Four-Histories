\article{列傳一百七十三}

\begin{pinyinscope}
文慶文祥寶鋆

文慶,字孔修,費莫氏,滿州鑲紅旗人,兩廣總督永保之孫也。道光二年進士,選庶吉士,授編修。五遷至詹事。歷通政使、左副都御史、內閣學士。十二年,授禮部侍郎,兼副都統。十三年,總理孝慎皇后喪儀,會奏軍民薙發及停止宴會期限疏中,誤引「百姓如喪考妣,四海遏密八音」語,下諸臣嚴議。宣宗以文慶翰林出身,隨聲附和,獨重譴,褫副都統,降三品頂戴。尋復之,歷吏部、戶部侍郎。十六年,偕尚書湯金釗赴陜西、四川按劾巡撫楊名颺、布政使李羲文,並下嚴議,尋復按名颺被訐事,褫其職。金釗留署陜西巡撫。文慶又按河南武陟知縣趙銘彞貪婪狀,劾褫職。調戶部侍郎。十七年,命在軍機大臣上學習行走,兼右翼總兵。命赴熱河,偕都統耆英按歷任總管虧短庫款,褫職追繳。十九年,以查辦熱河虧空案內擬罪未晰,召問,奏對失實,下部議,罷直軍機。二十年,典江南鄉試,以上下江中額有誤,又私攜湖南舉人熊少牧入闈閱卷,議褫職。

二十二年,予三等侍衛,充庫倫辦事大臣。二十三年,召授吏部侍郎、內務府大臣,連擢左都御史、兵部尚書。二十五年,命赴四川,偕總督、將軍按前任駐藏大臣孟保、鍾芳等濫提官物,劾罷之。二十七年,復命為軍機大臣,解內務府事務。尋署陜甘總督,道經河南,命察賑務,劾玩誤之知縣四人。

二十八年,召授吏部尚書,兼步軍統領、內務府大臣,罷直軍機處、兼翰林院掌院學士。三十年,充內大臣。薛執中者,甘肅河州人,以符咒惑眾。至京師,藉術醫病,朝貴多與往來。遂妄議時政,談休咎,行蹤詭秘,為巡城御史曹楙堅捕治,中外大臣牽連被譴者眾。文慶曾延治病,文宗斥其身為步軍統領,不能立時捕究,有乖職守,褫職。咸豐元年,予五品頂戴,辦理昌陵工程。二年,起授內閣學士,尋擢戶部尚書,復為內大臣、翰林院掌院學士。五年,復為軍機大臣、協辦大學士。題孝靜皇后神主,加太子太保,拜文淵閣大學士,晉武英殿大學士,管理戶部,充上書房總師傅。

文慶醇謹持大體,宣宗、文宗知之深,屢躓屢起,眷倚不衰。時海內多故,粵匪猖熾,欽差大臣賽尚阿、訥爾經額先後以失律被譴。文慶言:「當重用漢臣,彼多從田間來,知民疾苦,熟諳情偽。豈若吾輩未出國門、懵然於大計者乎?」常密請破除滿、漢畛域之見,不拘資格以用人。曾國籓初任軍事,屢戰失利,忌者沮抑之。文慶獨言國籓負時望,能殺賊,終當建非常之功。曾與胡林翼同典試,深知其才略,屢密薦,由貴州道員一歲之間擢至湖北巡撫,凡所奏請,無不從者。又薦袁甲三、駱秉章之才,請久任勿他調,以觀厥成。在戶部,閻敬銘方為主事,當採用其議,非所司者亦諮之。後卒得諸人力以戡定大難。端華、肅順漸進用事,皆敬憚其嚴正焉。

六年,卒。遺疏言各省督撫如慶端、福濟、崇恩、瑛棨等,皆不能勝任,不早罷,恐誤封疆。文宗深惜之,優詔賜恤,嘉其人品端粹,器量淵深,辦事精勤,通達治體,贈太保,賜金治喪。及親奠,見其遺孤幼穉,特詔加恩入祀賢良祠,命其子善聯俟及歲引見;弟文玉,以罪遣戍,即釋回。予謚文端。善聯,官至福州將軍。

文祥,字博川,瓜爾佳氏,滿洲正紅旗人,世居盛京。道光二十五年進士,授工部主事,累遷郎中。咸豐六年,京察,記名道府,因親老,乞留京職。歷太僕寺少卿、詹事、內閣學士,署刑部侍郎。八年,命在軍機大臣上行走,授禮部侍郎,歷吏部、戶部、工部侍郎,兼副都統、左翼總兵。

十年,英法聯軍犯天津,僧格林沁密疏請幸熱河。文祥以搖動人心,有關大局,且塞外無險可扼,力持不可,偕廷臣言之,復請獨對;退偕同直侍郎匡源、杜翰具疏請罷所調車馬,明詔宣示中外。八月,敵氛益熾,車駕遽行,命文祥署步軍統領,司留守。從恭親王奕議和,出入敵營,於非分之求,侃侃直言,折之以理。尋以步軍統領難兼顧,疏辭,改署正藍旗護軍統領。十月,和議成,疏請回鑾,以定人心。偕恭親王等通籌全局,疏上善後事宜,於是設立總理各國事務衙門,恭親王領之,滿、漢大臣數人,文祥任事最專。

時和局甫定,發、捻猶熾,兵疲餉竭,近畿空虛。文祥密疏請選練八旗兵丁,添置槍砲,於是始立神機營,尋命管理營務。又疏言僧格林沁兵力單薄,勝保所部新募未經行陣。既恃僧格林沁保障畿輔,必得良將勁卒為贊助,薦副都統富明阿、總兵成明隸其軍;又薦江西九江道沈葆楨、湖北候補知縣劉蓉堪大用。疏上,並嘉納焉。

十一年,文宗崩於熱河行在,穆宗即位,肅順等專政,文祥請解樞務,不許。十月,回鑾,偕王大臣疏請兩宮皇太后垂簾聽政。同治元年,連擢左都御史、工部尚書,兼署兵部尚書,為內務府大臣,兼都統。二年,管理籓院事務。東南軍事以次戡定,江蘇、浙江省城克復,議加恩樞臣,固辭。三年,江寧復,首逆就殲,捷至,加太子太保,予侄凱肇員外郎。四年,署戶部尚書,辭內務府大臣,允之。

是年秋,馬賊入喜峰口,命文祥率神機營兵防護東陵,督諸軍進剿,賊遁水欒陽。疏陳:「地方官豢賊釀患,請除積弊,清盜源。馬賊巢穴多在奉天昌圖八面城、熱河八溝哈達等處。請購線偵察,調兵掩捕,庶絕根株。」事定,回京。文宗奉安山陵,賜其子熙聯員外郎。尋以母病請假三月,回旗迎養。奉天馬賊方熾,命率神機營兵往剿,增調直隸洋槍隊出關,約東三盟蒙古王公由北路夾擊,破賊於錦州東井子。諜知賊將劫奉天獄,約期攻城,兼程馳援,賊退踞城東南,圍撫順;令總兵劉景芳夜擊破之,賊遁出邊。遣軍趨吉林,五年春,解長春圍,追賊至昌圖朝陽坡,分三路進擊,十數戰皆捷,擒斬三千餘。賊首馬傻子窮蹙乞降,磔之;留兵餉授將軍都興阿,俾清餘孽。請蠲奉天地丁銀米,停鋪捐。回京,調吏部尚書。文宗實錄成,賜子熙治員外郎。

八年,丁母憂,特賜諭祭。百日假滿,病未出。天津教案起,力疾還朝。十年,以吏部尚書協辦大學士。十一年,拜體仁閣大學士。文祥自同治初年偕恭親王同心輔政,總理各國事務,以一身負其責。洋情譸幻,朝論紛紜,一以忠信持之,無諉卸。洎穆宗親政,臚陳歷年洋務情形,因應機宜甚備,冀有啟悟。既而恭親王以阻圓明園工程忤旨斥罷,文祥涕泣,偕同列力諫,幾同譴。恭親王尋復職,而自屢遭挫折後,任事不能如初。文祥正色立朝,為中外所嚴憚,朝局賴以維持,不致驟變。十三年,病久不瘉,在告,會日本窺臺灣,強出籌戰守。疏請:「敕下戶部、內務府寬籌餉需,裁減浮用,停不急之工作,謀至急之海防,俾部臣、疆臣皆得專力圖維。皇上憂勤惕厲,斯內外臣工不敢蹈玩洩之習。否則狃以為安,不思變計,恐中外解體,人心動搖,其患有不可勝言者。」言甚切至。

是年冬,穆宗崩,德宗繼統即位,晉武英殿大學士。以久病請罷,溫詔慰留,解諸兼職,專任軍機大臣及總理各國事務。時國家漸多故,文祥深憂之,密陳大計疏曰:「洋人為患中國,愈久愈深,而其窺伺中國之間,亦愈熟愈密。從前屢戰屢和,迄無定局,因在事諸臣操縱未宜。及庚申定約,設立衙門專司其事,以至於今,未見決裂。就事論事,固當相機盡心辦理,而揣洋人之用心,求馭外之大本,則不系於此,所系者在人心而已矣。溯自嘉慶年間,洋人漸形強悍,始而海島,繼而口岸,再及內地,蓄力厲精習機器,以待中國之間,一逞其欲。道光年間,肆掠江、浙,自江寧換約以後,覬覦觀望。直至粵匪滋事,以為中國有此犯上作亂之事,人心不一,得其間矣。於是其謀遂洩,闖入津門,雖經小挫,而其意愈堅,致有庚申之警。然其時勢局固危,民心未二,勤王之師雖非勁旅,而聞警偕來;奸細之徒雖被誘脅,而公憤同具,以是得受羈縻,成此和局。十餘年來,仰賴皇太后、皇上勵精圖治,宵旰勤勞,無間隙之可尋;在事諸臣始得遇事維持,未至啟釁,偶有干求,尚能往返爭持,不至太甚,非洋務之順手,及在事者折沖之力,皆我皇太后、皇上朝乾夕惕,事事期符民隱,人心固結,有以折外族之心,而杜未形之患也。然而各國火器技藝之講求益進,彼此相結之勢益固。使臣久駐京師,聞我一政之當則憂,一或不當則喜,其探測愈精。俄人逼於西疆,法人計占越南,緊接滇、粵,英人謀由印度入藏及蜀,蠢蠢欲動之勢,益不可遏。所伺者中國之間耳,所惎者中國大本之未搖,而人心之難違耳。說者謂各國性近犬羊,未知政治,然其國中偶有動作,必由其國主付上議院議之,所謂謀及卿士也;付下議院議之,所謂謀及庶人也。議之可行則行,否則止,事事必合乎民情而後決然行之。自治其國以此,其觀他國之廢興成敗亦以此。儻其國一切政治皆與民情相背,則各國始逞所欲為,取之恐後矣。如土耳其、希臘等國,勢極弱小,而得以久存各大國之間者,其人心固也。強大如法國,而德國得以勝之者,以法王窮侈任性,負國債之多不可復計,雖日益額餉以要結兵心,而民心已去,始有以乘其間也。夫人必自侮而後人侮之,物必先自腐而後蟲生焉。理之所在,勢所必至。中國之有外國,猶人身之有疾病,病者必相證用藥,而培元氣為尤要。外國無日不察我民心之向背,中國必求無事不愜於民心之是非。中國天澤分嚴,外國上議院、下議院之設,勢有難行,而義可採取。凡我用人行政,一舉一動,揆之至理,度之民情,非人心所共愜,則急止勿為;事系人心所共快,則務期於成。崇節儉以裕帑需,遇事始能有備,納諫諍以開言路,下情藉以上通。總期人心永結,大本永固,當各外國環伺之時,而使之無一間可乘,庶彼謀不能即遂,而在我亦堪自立。此為目前猶可及之計,亦為此時不能稍緩之圖。若待其間之既開,而欲為斡旋補苴之法,則和與戰俱不可恃。即使仍可茍安,而大局已不堪復問,則何如預防其間之為計也。咸豐六年王茂廕奏陳夷務,謂:『海外諸國日起爭雄,自人視之,雖有中外之分,自天視之,殆無彼此之意。』引書言『皇天無親,惟德是輔』,及大學平天下章三言得失,首人心、次天命、而終以君心為證。何其言之危且切歟!欲戢夷心,莫要於順民心,能順民心,斯足以承天心,固不待蓍蔡而昭然若睹耳。臣受恩最重,辦理洋務最久,實有見於洋人居心積慮之處,而現時尤為迫切緊要之關。外國之求間在此,中國之彌間亦在此。在事諸臣,僅謀其末,我皇上實操其本。用敢直陳,伏乞俯鑒芻言,將此摺時置左右,力求端本之治,以回隱患之萌。天下幸甚!」

先是,當臺灣事平,文祥即偕恭親王議興海防,條上六事:曰練兵,曰簡器,曰造船,曰籌餉,曰用人,曰持久。各具條目,敕下中外大臣會議。至光緒二年,疆臣覆奏,將復下廷議。文祥已病不能出,自知且不起,乃密疏上曰:「馭外之端,為國家第一要務。現籌自強之計,為安危全局一大關鍵。臣衰病侵尋,心長智短,知不能永效犬馬以報主知。恐一旦填溝壑,則平生欲言未言之隱,無以上達宸聰,下資會議,何以對陛下?此心耿耿,有非總理衙門原奏所能盡者,敢竭誠吐赤,為我皇上敬陳之。夫敵國外患,無代無之,然未有如今日之局之奇、患之深、為我敵者之多且狡也。果因此患而衡慮困心,自立不敗,原足作我精神,惺我心志,厲我志氣,所謂生於憂患者正在於此。至此而復因循洩沓;一聽諸數而莫為之籌,即偶一籌念而移時輒忘,或有名無實,大局將不堪設想,而其幾不待智者而決矣。從前夷患之熾,由於中外之情相隔,和戰之見無定,疆吏又遇事粉飾,其情形不能上達於朝廷。坐是三失,而其患遂日久日深,無所底止。泰西各國官商一氣,政教並行,各商舶遠涉重洋,初至中華,處處受我侮抑,事事被我阻塞,其情鬱而不能不發者,勢也。繼而見中國官之阻之者可以通,抑之者可以伸,必不可破之格,或取勝於兵力之相迫而卒無不破,此中國之為所輕而各國漸敢恣肆之機也。迨至立約通商已有成議,而內無深知洋務之大臣,在外無究心撫馭之疆吏,一切奏牘之陳,類多敷衍諱飾。敵人方桀驁而稱為恭順,洋情方怨毒而號為懽忭,遂至激成事端,忽和忽戰;甚且彼省之和局甫成,此省之戰事又起,賠款朝給,捷書暮陳。乘遭風之船以為勝仗,執送信之酋以為擒渠,果至兩軍相交,仍復一敗不可收拾。於是夷情愈驕,約款愈肆,中外大臣皆視辦理洋務為畏途,而庚申釁起,幾至無可措手。自設立總理衙門,其事始有責成,情形漸能熟悉,在事諸臣亦無敢推諉。然其事非在事諸臣之事,而國家切要之事也。既為國家切要之事,則凡為大清臣子者,無人不應一心謀畫,以維大局。況和局之本在自強,自強之要在武備,亦非總理衙門所能操其權盡其用也。使武備果有實際,則於外族要求之端,持之易力,在彼有顧忌,覦覬亦可潛消,事不盡屬總理衙門,而無事不息息相關也。乃十數年來,遇有重大之端,安危呼吸之際,事外諸臣以袖手為得計;事甫就緒,異議復生,或轉託於成事不說;不問事之難易情形若何,一歸咎於任事之人。是從前之誤以無專責而仔肩乏人,今日之事又以有專屬而藉口有自。設在事諸臣亦同存此心,爭相諉謝,必至如唐臣杜甫詩中所謂『獨使至尊憂社稷』矣。夫能戰始能守,能守始能和,宜人人知之。今日之敵,非得其所長,斷難與抗,稍識時務者,亦詎勿知?乃至緊要關鍵,意見頓相背,往往陳義甚高,鄙洋務為不足言,抑或茍安為計,覺和局之深可恃。是以歷來練兵、造船、習器、天文、算學諸事,每興一議而阻之者多,即就一事而為之者非其實。至於無成,則不咎其阻撓之故,而責創議之人;甚至局外紛紛論說,以國家經營自立之計,而指為敷衍洋人。所見之誤,竟至於此!今日本擾臺之役業經議結,日本尚非法、英、俄、美之比,此事本屬無名之師,已幾幾震動全局,費盡筆爭舌戰,始就範圍。若泰西強大各國環而相伺,得中國一無理之端,藉為名義,構兵而來,更不知如何要挾,如何挽回?言念及此,真有食不下咽者,則自強之計尚可須臾緩哉?此總理衙門奏請飭令會議諸條,實為緊要關系,不可不及早切實籌辦者也。今計各疆吏遵旨籌議,指日將依限上陳,如飭下廷議,非向來會議事件可比,應由各王大臣期定數日,詳細籌商,將事之本末始終,一律貫澈,利害之輕重,條議之行止,辦法切實,折中定見,無蹈從前會議故習。如今日議之行之,而異日不能同心堅持,則不如不辦。如事雖議行,而名是實非,徒為開銷帑需,增益各省人員差使名目,亦不如不辦。度勢揆時,料敵審己,實有萬萬不能不辦之勢,亦實有萬萬不可再誤之機。一誤即不能復更,不辦即不堪設想。總理衙門摺內所謂『必須上下一心,內外一心,局中局外一心,且歷久永遠一心』,即此意也。而大本所在,尤望我皇上切念而健行之。總理衙門承辦之事,能否維持,全視實力之能否深恃。必確有可戰可守之實,庶可握不戰之勝。惟我皇上念茲在茲,則在事諸臣之苦心,自能上邀宸鑒。凡百臣工亦人人有求知此事共籌此事之心,其才識智力必有百倍於臣者。否則支持既難,變更不免,變而復合,痛心之端,必且百倍今日,非臣之所忍言矣。」疏上,未幾卒。溫詔賜恤,稱其「清正持躬,精詳謀國,忠純亮直,誠懇公明,為國家股肱心膂之臣」,贈太傅,予騎都尉世職,入祀賢良祠,賜銀三千兩治喪,遣貝勒載澂奠醊,謚文忠,歸葬盛京,命將軍崇實往賜祭。十五年,皇太后歸政,追念前勞,賜祭一壇。

文祥忠勤,為中興樞臣之冠。清操絕人,家如寒素。謀國深遠,當新疆軍事漸定,與俄國議交還伊犁,大學士左宗棠引以自任,文祥力主之,奏請專任。文祥既歿,後乃遣侍郎崇厚赴俄國,為所迫脅,擅允條款,朝論譁然。譴罪崇厚,易以曾紀澤往,久之乃定議,幸免大釁。法越事起,和戰屢更,以海防疏,不能大創敵,遷就結局。及興海軍,未能竭全力以成之,卒挫於日本。皆如文祥所慮,而朝局數變,日以多事矣。子熙治,以員外郎襲騎都尉世職。

寶鋆,字佩蘅,索綽絡氏,滿洲鑲白旗人,世居吉林。道光十八年進士,授禮部主事,擢中允。三遷侍讀學士。咸豐二年,粵匪竄兩湖,寶鋆疏請鄰近諸省力行堅壁清野之策。四年,命往三音諾顏部賜奠,謝絕餽贐,外籓敬之。擢內閣學士。五年,遷禮部侍郎,兼正紅旗蒙古副都統,調戶部。八年,典浙江鄉試,以廣額加中官生一名,坐違制,鐫一級留任,文宗諭「寶鋆素以果敢自命,亦同瞻徇」,特嚴斥焉。

十年,命赴天津驗收海運漕糧,復赴通州察視,迭疏請定杜弊章程,並劾監督貽誤,如所請行。任總管內務府大臣,署理戶部三庫事務,會辦京城巡防。時英法聯軍內犯,車駕幸熱河,既至,命提庫帑二十萬兩修葺行宮。寶鋆以國用方亟,持不可。上怒,欲加嚴譴,會所管三山被掠,詔切責,降五品頂戴。逾月後,以巡防勞勩,復之,兼鑲紅旗護軍統領,復兼署正紅旗漢軍都統、左翼前鋒統領。十一年,文宗崩於行在。十月,穆宗回京,命在軍機大臣上行走,並充總理各國事務大臣。

同治元年,擢戶部尚書。二年,奏劾壽莊公主府首領太監張玉蒼出言無狀,嚴旨逮訊,玉蒼治如律。三年,命大臣輪班進講治平寶鑒,寶鋆與焉。江寧克復,以翊贊功,加太子少保,賜花翎。四年,命佩帶內務府印鑰。尋以樞務事繁,請解內務府大臣職,允之。自設立總理各國事務衙門,始求通知外國語言文字,置同文館,肄習西學,廷臣每以為非。六年,都察院代奏職員楊廷熙上書請撤同文館,語涉恭親王及寶鋆等專擅挾持,於是寶鋆偕恭親王請罷直候查辦,溫詔慰留,勉以不避嫌怨,勿因浮言推諉。七年,直東捻匪肅清,加軍功二級。十一年,調吏部。穆宗大婚禮成,加太子太保。十二年,兼翰林院掌院學士,以吏部尚書協辦大學士。尋調兵部,拜體仁閣大學士,管理吏部。光緒三年,晉武英殿大學士。四年,回疆肅清,被優敘。

寶鋆自同治初年預樞務,偕文祥和衷翊贊,通達政體,知人讓善,恭親王資其襄助,至是朝列漸分門戶。文祥既歿,議論益紛,編修何金壽因旱災劾樞臣不職,請加訓責,詔斥恭親王、寶鋆等目擊時艱,毫無補救,嚴議革職,加恩改留任。五年,以題穆宗神主,加太子太傅,復以實錄告成,推恩其子景灃晉秩郎中,侄景星賜舉人。七年,庶子陳寶琛以星變陳言,專劾寶鋆,請仿漢災異策免三公故事,立予罷斥。詔曰:「寶鋆在軍機大臣上行走有年,尚無過失。陳寶琛謂其畏難巧卸,瞻徇情面,亦不能確有所指。惟既有此奏,自平時與王大臣等議事未能和衷共濟,致啟人言。該大學士受恩深重,精力尚健,自當恪矢公忠,勉圖報稱,務宜殫精竭慮,力戒因循積習,用副委任。」

十年三月,軍機大臣自恭親王以下同日斥罷,詔:「寶鋆入直最久,責備宜嚴,姑念年老,特錄前勞,全其末路,以原品休致。」十二年,皇太后懿旨加恩,改以大學士致仕,賞食半俸。寶鋆退休後,時偕恭親王居西山游覽唱和。年逾八十,恩賚猶及。十七年,卒。遺疏入,詔褒其「忠清亮直,練達老成」,贈太保,祀賢良祠,擢子景灃四品京堂,賜孫廕桓舉人,遣貝勒載瀅奠叕,飾終之典,視在位無所減,謚文靖。子景灃,官至廣州將軍,卒,謚誠慎。孫廕桓,光緒二十四年進士,歷官國子監司業,改乾清門頭等侍衛。

論曰:咸、同之間,內憂外患,岌岌不可終日。文慶倡言重用漢臣,俾曾國籓、胡林翼等得展經猷,以建中興之業,其功甚偉。文祥、寶鋆襄贊恭親王,和輯邦交,削平寇亂。文祥尤力任艱鉅,公而忘私,為中外所倚賴,而朝議未一,猶不能盡其規略;晚年密陳大計,於數十年馭外得失,洞如觀火,一代興亡之龜鑒也。寶鋆明達同之,貞毅不及,遂無以鎮紛囂而持國是。如文祥者,洵社稷臣哉!


\end{pinyinscope}