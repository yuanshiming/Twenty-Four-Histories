\article{本紀二十}

\begin{pinyinscope}
文宗本紀

文宗協天翊運執中垂謨懋德振武聖孝淵恭端仁寬敏顯皇帝,諱奕蟭,宣宗第四子也,母孝全成皇后鈕祜祿氏,道光十一年六月初九日生。二十六年,用立儲家法,書名緘藏。

三十年正月丁未,宣宗不豫,宣召大臣示硃筆,立為皇太子。宣宗崩,己未,上即位,頒詔覃恩,以明年為咸豐元年。尊皇貴妃為孝慈皇貴妃。追封兄貝勒奕緯、奕綱、奕繹為郡王。封弟奕恭親王,奕枻醇郡王,奕硉鍾郡王,奕言惠孚郡王。定縞素百日,素服二十七月。

二月戊辰,命左都御史柏葰、內務府大臣基溥營建昌西陵,為孝和皇后山陵。初宣宗遺詔,毋庸升配、升祔。交廷臣議。議上。詔曰:「先帝謙讓,所不敢從。曲體先懷,宜定限制。即以三祖五宗為斷,嗣後不復舉行。」湖南土匪李沅發作亂。詔:「惠親王系朕之叔,免叩拜禮,示敬長親親。」庚辰,敕沿海整頓水師,認真巡緝。壬辰,大理寺卿倭仁應詔陳言,上嘉其直諫。

三月癸巳朔,保昌卒,以柏葰為兵部尚書,花沙納為左都御史。壬寅,通政使羅惇衍應詔陳言,上優詔答之。癸卯,左副都御史文瑞疏陳四事,並錄進乾隆元年故大學士孫嘉淦三習一弊疏,禮部侍郎曾國籓疏陳用人三事,均嘉納之。辛亥,濬江蘇白茅河,移建海口石徬於老徬橋。壬戌,禮親王全齡薨,子世鐸襲。

夏四月乙丑,俄羅斯請於塔爾巴哈臺通商,允之。己巳,內閣學士車克慎疏陳敬天繼志、用人行政凡十條,優詔答之。癸酉,戶部疏陳整頓財政,臚陳各弊,得旨:實力革除。庚辰,英吉利國船至江蘇海口遞公文,卻之。乙酉,船至天津。

五月丙申,起碇南旋。丁酉,詔曰:「州縣親民之官,責任綦重。近年登進冒濫,流品猥雜,多倚胥吏而朘閭閻,民生何賴焉。督撫大吏其加意考察,薦進廉平,鋤斥貪茸,庶民困漸蘇,以副朕望。」獲湖南逆首李沅發,解京誅之。詔鄭祖琛「廣西會匪四起,應時捕剿,疏報勿得諱飾。」辛亥,改山東登州鎮為水師總兵,兼轄陸路。癸丑,詔東南兩河勘籌民堰。甲寅,以固慶為吉林將軍。

六月癸亥,永定河溢。大學士潘世恩致仕,食全俸。以祁俊藻為大學士,杜受田協辦大學士,孫瑞珍為戶部尚書,王廣廕為兵部尚書,季芝昌為左都御史。甲戌,除甘肅民、番升科畸零地銀。甲申,敕督撫舉劾屬員,臚列事實,勿以空言。是月,廣東花縣人洪秀全在廣西桂平縣金田起事。

秋七月辛卯,敕沿海督撫籌防海口。丙辰,尚書文慶坐延請妖人薛執中治病,免。

八月丁卯,洪秀全竄修仁、荔浦,敕鄭祖琛剿之。調向榮為廣西提督剿賊。甲申,詔曰:「各省糾眾滋事,重案層見疊出,該地方官所司何事?即如河南捻匪結黨成群,甚至擾及鄰省,橫行劫掠,自應合力捕治,凈絕根株。若封疆大吏玩縱於前,復諱飾於後,以致釀成鉅患,朕必將該督撫從重治罪。凜之!」

九月丙申,以廣西賊勢蔓延,調湖南、雲南、貴州兵各二千赴剿,並勸諭紳民舉辦團練。辛丑,命林則徐為欽差大臣,剿賊廣西。甲辰,以廣東游匪滋事,命徐廣縉剿之。丙午,大行梓宮發引。辛亥,暫安宣宗成皇帝於隆恩殿。

冬十月壬午,以彌縫釀患,奪鄭祖琛職,林則徐署廣西巡撫。甲子,永定河漫口合龍。丙戌,詔曰:「大學士穆彰阿柔佞竊位,傾排異己,沮格戎機,罔恤國是,即行褫職。協辦大學士耆英無恥無能,降員外郎。頒示中外。」以賽尚阿協辦大學士。

十一月戊戌,以奕山為伊犁將軍。庚子,欽差大臣林則徐道卒,以周天爵署廣西巡撫,命前兩江總督李星沅為欽差大臣,赴廣西剿賊。乙巳,敕各省籓庫積存雜款,撥充軍需,暫緩開捐。劉韻珂免,以裕泰為閩浙總督,程矞採為湖廣總督,吳文鎔為雲貴總督。獲廣西匪首鍾亞春,誅之。

十二月己巳,孝德皇后冊謚禮成,追封后父富泰為三等公。敕奕山酌定俄羅斯通商條例以聞。庚午,敕江蘇四府漕糧暫行海運。甲戌,向榮剿賊橫州,敗之。己卯,恤廣西陣亡副將伊克坦布等世職。丙戌,祫祭太廟。

是歲,免直隸、浙江、湖南等省六十七州縣災賦有差。朝鮮、琉球入貢。

咸豐元年辛亥春正月戊子朔,御太和殿受朝賀。詔直省查明道光三十年以前正耗錢糧實欠在民者,開單請旨。命賽尚阿為大學士。壬寅,上謁慕陵,行周年大祭禮。庚戌,還京。辛亥,詔翰、詹諸臣分撰講義進呈。給事中蘇廷魁疏請推誠任賢,慎始圖終。上嘉納之。

二月乙丑,詔免直省民欠錢糧巳入奏銷者,及於江蘇民欠漕糧,悉予蠲免。杜受田疏陳整軍威、募精勇、勸鄉團、察地形四事,發軍前大臣。庚午,李星沅奏剿賊金田獲勝。己卯,詔曰:「今年節過春分,寒威未解。朕返躬內省,未能上感天和。因思去冬禮部匯題烈婦一本,內閣票擬雙簽,遂用不必旌表之簽發下。該烈婦等舍生取義,足激薄俗而重綱常,所有烈婦彭氏等三十七口,準其一體旌表,以慰貞魂。」命廣州副都統烏蘭泰帶所制軍械赴廣西剿賊。

三月丙申,命大學士賽尚阿佩欽差大臣關防,馳往湖南辦理防堵,都統巴清德、副都統達洪阿隨往。庚子,上御紫光閣閱射。辛丑,御拱辰殿步射,閱大臣、侍衛射。己酉,河南巡撫潘鐸奏拿獲捻匪姚經年二百餘名。庚戌,調廣東、湖南、四川兵赴廣西助剿。壬子,發內帑銀一百萬兩備廣西軍儲,發四川倉穀碾運湖南。

夏四月戊午,命賽尚阿馳赴廣西接辦軍務。己未,命戶部左侍郎舒興阿為軍機大臣。庚申,上禦乾清門聽政。恤廣西中伏陣亡副將齊清阿等世職。詔以李星沅等毫無成算,中賊奸計,切責之。以鄭祖琛養癰貽患,遣戍伊犁。丙寅,周天爵奏洪秀全等眾皆散處,山險路熟,伺間沖突,即敗不足以大創。此時兵力不足,專飭防守。須兵有餘力,乃可連營偪剿。得旨:「務當嚴防,勿令竄逸。」賽尚阿師行,賜遏必隆刀,命天津鎮總兵長瑞、涼州鎮總兵長壽從征。庚午,免直隸道光三十年民欠錢糧。周天爵奏劾右江鎮總兵惠慶、右江道慶吉剿賊不力,均奪職。丙子,李星沅奏剿滅上林墟會匪。癸未,李星沅卒。烏蘭泰奏,四月初三日,抵武宣軍營。查詢賊勢,類皆烏合。惟武宣東鄉會匪有眾萬餘,蓄發易服,有偽王、偽官名目,實廣西腹心之患。得旨:「賊情狡獪,務當持重。」

五月戊子,周天爵奏,武宣東鄉逸賊竄入象州。詔切責之,各予薄譴。詔湖南提督餘萬清協同堵剿。庚寅,卓秉恬奏請行堅壁清野之法,下賽尚阿及督撫知之。甲午,周天爵奏剿平泗城股匪,陳亞等投誠,追賊入合浦。丁酉,烏蘭泰奏,四月十七日,馳至象州,堵截逸賊。甲辰,陜甘總督琦善以剿辦番族,率意妄殺,奪職逮問。乙巳,以季芝昌為閩浙總督,以戶部尚書裕誠協辦大學士。己酉,詔停中外一切工程。命工部右侍郎彭蘊章為軍機大臣。乙卯,上詣大高殿祈雨。

六月丁巳,賽尚阿報抵長沙。詔曰:「象州之賊,宜重兵合圍。分竄南寧、太平之賊,應分兵追剿。其尚審度地勢人材,聯絡布置。糧臺尤關緊要,並宜分置,以利轉輸。」丙寅,烏蘭泰奏,五月初十日,賊陷貴州兵營,當日奪回。其南山屯集之賊,亦經迎擊南竄。陣亡官十五員,兵二百餘名,附單請恤。首先敗退之貴州參將佟攀梅等褫職。辛未,撥江海關稅銀十五萬兩,解備湖北過境兵差。乙亥,賽尚阿奏,六月初四日,馳抵桂林,通籌全局。上嘉其均合機宜。丁丑,河南南陽捻匪四出滋擾,詔所司捕之。辛巳,西寧番匪搶掠,敕薩迎阿遣將剿捕。

秋七月丙戌,賽尚阿奏,賊由象州回竄東鄉,派兵堵剿。庚寅,御史焦友瀛疏言吏治因循,宜綜覈名實。得旨:「如果牧令得人,何至奸宄潛聚,釀成巨患?嗣後有似此者,惟督撫是問。」庚子,賽尚阿奏,進剿新墟賊匪,七戰皆捷。賞還烏蘭泰、秦定三花翎。命湖廣、四川督撫嚴查會匪、教匪。丁未,敕南河歲修工程,以三百萬為率。己酉,賽尚阿奏:「查明軍將功過,烏蘭泰先勝後敗,由於猛追中伏,賊人壅流設伏,後軍死流湍者百餘。向榮初到桂時,連獲勝仗,每勝賞兵銀人各一兩。李星沅既至,減為三錢。眾兵譁然,誓不出戰。現巳分別汰除,務知持重。」安徽巡撫蔣文慶奏,壽州匪犯程六麻與合肥捻匪高四八作亂。庚戌,調鮑起豹為湖南提督,榮玉材為雲南提督,重綸為貴州提督。

八月乙卯,賽尚阿奏,進剿新墟賊巢,奪占豬綌峽、雙髻山。得旨嘉獎。乙丑,山東巡撫陳慶偕奏,登州水師船被賊手虜,副將落水。得旨:「速往追剿。」並敕奕興、訥爾經額嚴防海口。禮部尚書惠豐卒。

閏八月甲申朔,新墟眾首洪秀全陷永安州,踞之,僭號太平天國。陸建瀛奏請禁天主教。得旨:「與外夷交涉,當慎之於始。原約所有者,仍應循守舊章。」戊子,程矞採奏,陽山賊匪竄撲宜章、乳源,飭總兵孫應照往剿。予廣西殉難巡檢馮元吉世職,建祠,其子澍溥附祀。甲午,南河豐北三堡河決。庚子,定考試軍機章京例。壬寅,賽尚阿奏新墟賊翻山竄出,陷永安州。詔切責之,下部議處。己酉,命河北鎮總兵董光甲、鄖陽鎮總兵邵鶴齡馳赴廣西剿賊。庚戌,常大淳奏盜船在石浦肆劫,知府羅鏞擊走之。辛亥,以舒興阿為陜甘總督。

九月庚午,賽尚阿奏巴清德、向榮託病諉卸,進兵遲延。得旨,均奪職自效。丙子,詔議河海並運漕米章程。

冬十月戊戌,敕建定海陣亡總兵葛云飛、鄭國鴻專祠。

十一月己卯,葉名琛奏,剿辦英德賊匪凈盡。加太子少保。

十二月丁酉,賽尚阿奏,向榮進扎龍眼塘。己酉,陸建瀛奏,海盜布興有繳械投誠,撥營安插。庚戌,祫祭太廟。

是歲,普免道光三十年以前民欠錢糧。又免直隸六十一州縣民欠旗租,浙江五十一州縣帶徵銀米。又免奉天十五州縣,吉林四城,黑龍江一城,湖南七州縣災賦。又免浙江、福建鹽場欠課。又免廣西被賊八十六州縣額賦。朝鮮、琉球入貢。

二年壬子春正月壬子朔,封奕劻貝子,奉慶親王永璘祀。乙卯,以裕誠為大學士,訥爾經額協辦大學士,禧恩為戶部尚書。壬戌,賽尚阿奏,距永安州城三里安營督戰。辛未,命侍郎全慶、副都統隆慶冊封朝鮮國王妃。

二月丁亥,陳慶偕病免,以李僡為山東巡撫。辛丑,上詣西陵。

三月壬子,大葬宣宗成皇帝於慕陵。丁巳,上還京,恭奉宣宗成皇帝,孝穆、孝慎、孝全三皇后神牌升祔太廟,頒詔覃恩。庚申,鄒鳴鶴奏永安踞匪全數東竄,烏蘭泰追賊不利,總兵長瑞、長壽、董光甲、邵鶴齡均死之。得旨,賽尚阿等下部議處,敕程矞採派兵在湖南防堵,恤長瑞等四總兵世職,建祠。廣州副都統烏蘭泰卒於軍,贈都統,照陣亡例賜恤。丙子,恤廣西死事副將阿爾精阿等世職。庚辰,內閣學士勝保疏言:「游觀之所,煥然一新。小民竊議,有累主德。」上優容之。

夏四月壬午,常雩,祀天於圜丘,恭奉宣宗成皇帝配享。甲申,府尹王慶雲疏陳河東鹽務,永禁簽商,可募鉅款。下部議行。丙戌,上謁慕陵,行釋服禮。命徐廣縉為欽差大臣,接辦廣西軍務。辛卯,程矞採奏郴州匪徒劉代偉作亂,參將積拉明捕誅之。癸巳,常大淳奏,鹽梟拒捕,戕斃副將張蕙、知縣德成,經提督善祿、知府畢承昭派兵攻擊,斬擒百餘,餘匪逃散。予張蕙、德成世職。太僕寺少卿徐繼畬疏陳釋服之後,宜防三漸:一、土木之漸,一、宴安之漸,一、壅蔽之漸。得旨:「置諸座右,時時省覽。」己亥,減乾隆朝所增名糧兵六萬六千餘名。庚子,程矞採奏,洪秀全撲全州,進撲永州,分股竄永福、義寧,檄提督鮑起豹、劉長清分御之,並咨照賽尚阿一同堵御。辛丑,特登額免,以桂良為兵部尚書。乙巳,賜章鋆等二百三十九人進士及第出身有差。琦善遣戍吉林。丙午,鄒鳴鶴以留兵守城,不令追賊,奪職。以勞崇光為廣西巡撫。己酉,命截留漕米六十萬石,分運江蘇、山東備賑。

五月辛亥,布彥泰奏,庫存回布四十萬匹,請變通折征,允之。甲寅,夏至,祭地於方澤,恭奉宣宗成皇帝配享。庚申,賊陷湖南道州。賽尚阿留守桂林,檄江忠源、張國樑移兵湖南。

六月甲申,查辦山東賑務。杜受田、怡良疏言漕船入東,先行起卸,以資散放。丙戌,命賽尚阿赴湖南督辦軍務,徐廣縉接辦廣西軍務。丁亥,策立皇后鈕祜祿氏。癸巳,僧格林沁奏劾御前大臣鄭親王端華修改大考侍講學士保清試卷,阻止不聽,驕矜亢愎,難與共事。詔端華退出御前大臣,保清褫職。戊戌,以慧成為河東河道總督。

秋七月己未,廣東羅鏡凌十八股匪剿平,上嘉獎之。烏什辦事大臣春熙奏,回匪鐵完庫里霍卓竄擾烏什,官兵擊退。詔參贊詳查以聞。甲子,詔軍務未竣,需材孔亟,其有知兵之人,所在保舉錄用。詔直省修整城垣。丙寅,協辦大學士杜受田卒。丁卯,羅繞典奏,行抵長沙,聞知賊由道州竄出江華、永明、桂陽、嘉禾,誠恐衡郡有失,省垣亦應預防。得旨,即妥籌辦理。戊辰,給事中袁甲三劾定郡王載銓、尚書恆春、侍郎書元,迭查有跡,各予譴責,其題詠載銓息肩圖各員,並下部議處。庚午,奕山、布彥泰奏,回匪倭里罕糾約布魯特突入卡倫,官兵擊卻之。壬申,洪秀全攻陷郴州。甲戌,常大淳奏岳州宜籌防堵,詔徐廣縉撥兵前往。以麟魁為刑部尚書。

八月己卯朔,向榮以稱病規避奪職,遣戍新疆,尋留軍自效。以福興為廣西提督。癸未,初舉經筵。甲申,詔湖廣督撫:「湖南之洞庭湖、湖北之大江,均有捕魚小船及經商大船數千百隻,亟宜收集,免為賊用。其各船水手習於風濤,堪充水勇,其即留心招集。」己丑,羅繞典、駱秉章奏,賊匪陷安仁、攸縣,進圖省城。敕賽尚阿速解省圍。庚寅,命廷臣會籌軍儲。調常大淳為山西巡撫,以羅繞典為湖北巡撫,張芾署江西巡撫。甲辰,命暫免四川、江西商販運往湖北米稅。調福建、浙江兵一千名赴江西防堵。

九月己酉,詔賽尚阿視師無功,貽誤封疆,褫職逮問,籍其家。辛亥,以載銓為步軍統領,以訥爾經額為大學士,禧恩協辦大學士。甲寅,獲西寧番賊阿里克公住,斬之。命駱秉章暫留湖南會辦。戊午,上謁東陵。恤湖南陣亡總兵福誠等世職。己未,常大淳奏賊將北竄,防禦兵單。命徐廣縉撥兵赴岳州助防。丁卯,上還京。

冬十月辛巳,上臨贈大學士杜受田第賜奠,加其父杜堮禮部尚書銜。甲申,黃宗漢奏請浙江新漕改由海運,從之。壬辰,季芝昌免,以吳文鎔為閩浙總督。

十一月丁未朔,日有食之。丁巳,賊陷岳州。戊午,起琦善署河南巡撫。辛酉,詔徐廣縉分兵防守武昌、漢陽、荊州,陸建瀛、蔣文慶各就地勢扼要嚴防。癸亥,以賊近湖北,敕琦善嚴防河南邊境,詔張芾嚴防沿江要隘。甲子,以文慶為戶部尚書。癸酉,賊陷漢陽,命陸建瀛馳赴上游防堵。乙亥,復向榮提督銜。詔在籍侍郎曾國籓督辦團練。調福珠洪阿為江南提督。

十二月丁丑,敕各省紳士在籍辦理團練。命四品京堂勝保從軍河南。癸巳,賊陷武昌,巡撫常大淳死之。上切責督軍大臣不籌全局,擁兵自衛,逮徐廣縉治罪。以向榮為欽差大臣,督辦軍務,張亮基署湖廣總督。以葉名琛為兩廣總督,柏貴為廣東巡撫。癸卯,向榮奏賊連陷武、漢,搭有浮橋,必須多備砲船,將橋焚毀,方可進剿。得旨:「刑部郎中盧應翔所帶砲船,曾在長沙擊賊,即迅赴軍前。」甲辰,吉林、黑龍江徵兵到京。得旨:「每起間二日起行,帶兵官嚴守紀律,不得多索車輛,騷擾驛站。」祫祭太廟。

是歲,免直隸四十二州縣、山西一府災賦,浙江四十八州縣緩徵銀米各有差。朝鮮、暹羅入貢。

三年癸丑春正月丁未,調青州副都統常青兵移防豫、楚。戊申,張亮基奏,賊目蕭朝貴實在長沙城外轟斃,起獲尸身,驗明梟剉。己酉,蔣文慶奏城薄兵單,移調東西梁山兵勇來城防禦。癸丑,向榮奏,武昌踞賊抬砲上船,意欲逃竄。陸應穀奏,偵得賊匪開年有東竄安慶、江寧之信。敕向榮多方偵備,迎擊兜剿。甲寅,敕步軍統領、前鋒統領整備軍實,盤詰奸宄。甲子,賊陷九江,陸建瀛退守江寧。賽尚阿論斬,革其子崇綺等官職。丁卯,命工部左侍郎呂賢基回安徽辦防,加周天爵侍郎銜,會辦防務。壬申,陸建瀛褫職逮問,以祥厚為欽差大臣。癸酉,以山西、陜西、四川三省紳民捐輸軍饟,加鄉試中額、生員學額。甲戌,賊陷安慶,蔣文慶死之,命周天爵署安徽巡撫。予江西陣亡總兵恩長世職。

二月丙子朔,詔:「京師八旗營兵十五萬之多,該管大臣勤加訓練。」贈恤湖北殉難學政馮培元加侍郎,謚文介,布政使梁星源謚敏肅,按察使瑞元謚端節,及知府以下官各予世職、建專祠,提督雙福、總兵官王錦繡附常大淳祠。丁丑,釋奠先師孔子。遣少卿雷以諴、侍講學士晉康往南河,少詹事王履謙赴東河,會辦防務。癸未,上臨雍講學,加衍聖公孔繁灝太子太保。丁亥,敕文臣三品以上養廉以四成、武臣二品以上以二成充軍饟。戶部請辦商捐、戶輸,上不許。壬辰,賊陷江寧,將軍祥厚、提督福珠洪阿等死之。以怡良為兩江總督,命慧成馳赴江南防剿。調托明阿為江寧將軍,文斌為綏遠城將軍,瑞昌為杭州將軍,鄧紹良為江南提督。丙申,命琦善會防淮揚。敕湖北行鹽暫用川鹽二千引。敕李僡查拿山東兗、沂、曹三府捻匪。命內閣學士勝保幫辦江北防務。

三月乙巳,賊陷鎮江、揚州。丙午,孝和睿皇后升祔太廟。辛亥,上耕耤田。壬子,命湖北按察使江忠源幫辦江南軍務。丙辰,敕侍郎奕經統密雲兵赴山東會防。丁巳,敕各省團練格殺土匪勿論。以駱秉章復為湖南巡撫。敕江寧布政使陳啟邁在徐州設立糧臺。庚申,向榮擊賊於江寧,敗之。以施得高為福建水師提督。壬戌,以廬州為安徽省會。周天爵剿賊妥速,琦善進攻連獲勝仗,均嘉賚之。敕直隸、奉天備防海口。丙寅,向榮奏迭勝城賊,進據鍾山。上優獎之。命奕經、托明阿赴清江防剿。命瑞昌統盛京兵赴淮、徐會防,恩華統吉林兵駐防直隸。辛未,敕廣東招募紅單船,遴將帶赴江南剿賊。以福濟為漕運總督。

夏四月庚辰,日見黑暈。己丑,賊陷浦口、滁州。甲午,命琦善統制江北諸軍。逮治楊文定。庫倫喀爾喀蒙古哲布尊丹巴喇嘛進馬三千匹,及西林盟長進馬,均溫諭止之。己亥,賜孫如僅等二百二十二人進士及第出身有差。癸卯,賊陷鳳陽。安徽捻匪竄擾蒙城。

五月戊申,始制銀鈔。壬子,王懿德奏海澄會匪陷同安、安谿、廈門,嚴飭之。周天爵奏收復鳳陽。癸丑,李嘉端奏金陵賊船上竄。得旨,此與向榮疏報不同,令確切查探。駱秉章奏,江西上猶縣匪首劉洪義聚眾在桂東滋擾,毗連廣東、湖南。得旨,三省會剿。丙辰,陸應穀奏亳州失守,賊撲汴梁。敕江忠源馳赴河南剿賊。王懿德奏漳州鎮、道被賊戕害,永安、沙縣先後失守。丁巳,命勝保統兵馳赴河南。戊午,釋賽尚阿、徐廣縉於獄,從軍自效,楊殿邦、但明倫均留清江浦辦防。周天爵奏鳳陽逸匪竄擾而西,即日赴援。得旨:「周天爵素稱勇敢,所保臧紆青練勇可當一面,獨不能與賊決一死戰耶?」陸應穀、恩華奏竄賊由曹河搶渡,犯及山東。得旨,調陜西兵應援,仍令固守潼關門戶。賊陷歸德。己未,賊復陷安慶。詔江忠源防守九江。徵蒙古兵及其所進馬五千匹集於熱河。壬戌,詔以賊匪北竄,勸諭北地紳民練團自衛,如能殺賊出力,並與論功。命僧格林沁、花紗納、達洪阿、穆廕督辦京城巡防。癸亥,以許乃普為刑部尚書,翁心存為工部尚書。甲子,以河南兵民固守省城,優詔嘉勉。丁卯,命訥爾經額防守河北。桂良赴保定辦理防守。己巳,開封解嚴,賊南竄中牟、硃仙鎮,敕托明阿等追之。辛未,始鑄當十大錢。

六月乙亥,福建紳商克復漳州,優詔嘉之,查明給獎。戊寅,河南賊犯汜水,分股渡河陷溫縣。托明阿擊之,復汜水。己卯,金陵賊船上陷南康,進圍南昌。辛巳,溫縣紳勇敗賊,復其城,復會官軍敗賊於武陟。命納爾經額為欽差大臣,督辦河南、河北軍務,恩華、托明阿副之。黃河再決豐北。甲申,雲南東川回匪作亂。福建臺灣土匪作亂。戊子,美國使人求入覲,詔止之。河南賊圍懷慶。官軍解許州圍,賊走羅山。福建官軍收復永安、沙縣。托明阿等敗賊於懷慶。乙未,鎮江官軍失利,奪提督鄧紹良職,以和春署江南提督。戊戌,優恤揚州攻城傷亡總兵雙來世職銀兩。廣西全州土匪作亂。

秋七月甲辰朔,廣西土匪陷興安、靈川,分撲桂林,官軍敗之,復靈川、興安。丙午,敕慧成回清江浦防剿。丁未,命勝保幫辦河南軍務。丙辰,敕東南河臣收撤渡船,防賊偷渡。恤江西陣亡總兵馬濟美世職。丁巳,詔江西、湖廣新漕折價解京。辛酉,賊竄湖北、安徽。敕怡良於上海設關收稅。癸亥,恤提督福珠洪阿世職。甲子,詔紳士辦團禦賊捐軀者,一體恩恤。乙丑,福建官軍復尤谿。

八月丙子,官軍解懷慶圍,賊竄山西。戊寅,調吳文鎔為湖廣總督,裕瑞為四川總督,樂斌為成都將軍。庚辰,賊陷垣曲。癸未,李僡卒,以張亮基為山東巡撫,駱秉章授湖南巡撫。甲申,江西賊陷饒州郡城,吉安土匪遙應之。丙戌,賊陷絳縣、曲沃,進圍平陽。哈芬免,以恆春為山西巡撫。庚寅,賊陷平陽,勝保兵至,敗之,復平陽。賊由洪洞東竄。癸巳,命勝保為欽差大臣,賜神雀刀,恩華、托明阿副之。丁酉,托明阿敗賊於陳留。

九月癸卯朔,再敗之潞城、黎城,賊竄直隸,入臨洺關。奪訥爾經額職逮問,以桂良為直隸總督。丙午,賊陷柏鄉。江西南昌圍解,賊復竄踞安慶。丁未,調魁麟為禮部尚書,花沙納為工部尚書,以勝保為漢軍都統。江蘇土匪陷青浦、寶山,官軍復之。戊申,命截留漕糧備山東災賑。以軍務方急,緩修豐北河工。辛亥,命惠親王為奉命大將軍,賜銳捷刀,科爾沁郡王僧格林沁為參贊大臣,賜訥庫尼素刀,恭親王奕、定郡王載銓、內大臣壁昌會辦巡防。乙卯,賊由趙卅、城陷深卅。命於河間、涿卅、通州設防。辛酉,李嘉端罷,以江忠源為安徽巡撫。甲子,僧格林沁復深州。丙寅,陸應穀罷,以英桂為河南巡撫。己巳,周天爵卒於軍。辛未,賊陷獻縣、交河、滄州,進撲天津,知縣謝子澄督帶練勇迎擊,死之,所部敗賊三十里。特贈謝子澄布政使,並建祠,優獎練勇。警聞,京師戒嚴,僧格林沁駐軍於武清。

冬十月甲戌,命曾國籓督帶練勇赴湖北剿賊。丙子,賊陷黃州,漢黃德道徐豐玉死之,連陷漢陽,進圍武昌。丁丑,賊踞獨流鎮,勝保督軍至,連擊敗之。戊寅,命恭親王奕在軍機處行走,解麟魁軍機大臣,以瑞麟、穆廕為軍機大臣。乙卯,加給事中袁甲三三品卿銜,剿辦安徽捻匪。壬辰,武昌解嚴,江忠源赴皖。命署臬司唐樹義江面剿賊。癸巳,賊陷桐城。戊戌,豫征山西、陜西、四川三省糧賦,尋止之。

十一月壬寅朔,以王慶雲為陜西巡撫。丙午,福建官軍克復廈門。安徽賊陷舒城,辦團大臣侍郎呂賢基死之。庚戌,賊陷儀徵。癸丑,命侍郎曾國籓督帶水師剿賊安徽。丁卯,勝保剿賊獨流,不利,陣歿副都統佟鑒,贈將軍賜恤。

十二月甲戌,揚州賊潰圍出,官軍復其城,琦善、慧成等均褫職從軍。乙亥,詔以黃州賊宗麕集,飭吳文鎔出省剿賊。戊子,琦善復儀徵。己丑,賊陷廬州,江忠源死之。以福濟為安徽巡撫,邵燦為漕運總督。丙申,以侍郎杜翰為軍機大臣。翁心存罷,以趙光為工部尚書。己亥,祫祭太廟。

是歲,免奉天、直隸、山東、山西、浙江、湖北、湖南、廣西、雲南、甘肅等省三百四十四州縣衛災賦。又免甘肅中衛地震銀糧、草束各有差。朝鮮、琉球、暹羅、越南、緬甸、南掌入貢。

四年甲寅春正月辛丑朔,蒙古各盟長親王、郡王迭次報效軍需銀兩,溫旨嘉獎,均卻還之。乙巳,撥內庫銀三十萬兩解赴勝保軍營。庚戌,官軍克獨流鎮,踞匪回竄。壬子,張芾罷,以陳啟邁為江西巡撫。王履謙疏陳河南吏治廢弛,軍需浮冒,河工糜費。下英桂查覆。丙辰,浙江海運漕米改由劉河口放洋,命江蘇派員設局。己未,命福濟經理淮北鹽務。以王懿德為閩浙總督,呂佺孫為福建巡撫。辛酉,袁甲三疏請事關籌饟,由軍機處徑交所司,勿發內閣,從之。乙丑,命廣東購辦夷砲運赴武昌。丙寅,賊踞束城村,嚴詔僧格林沁、勝保迅速剿擒。丁卯,湖北進攻黃州兵潰,總督吳文鎔,署按察使、前布政使唐樹義死之。戶部議覆四川學政何紹基捐廉疏上違式用駢文,上責祁俊藻曰:「當閱何紹基疏時,卿亦議其迂拘,何為尤而效之?大學士管部,乃不能動司官稿一字乎!」賊竄獻縣東城莊,僧格林沁、勝保合軍擊之。賊竄陷阜城,分股竄山東。己巳,江蘇六合縣紳團力保危城,詔嘉之,免一年錢糧。

二月丁丑,上御經筵。己卯,許乃普罷直南書房,降內閣學士。以硃鳳標為刑部尚書,周祖培為左都御史。起翁心存為吏部左侍郎。辛巳,以臺湧為湖廣總督。壬午,曾國籓奏統帶水陸師萬七千人,自衡州起程馳赴湖北。癸巳,奕興罷,以英隆為盛京將軍。曾國籓疏請前巡撫楊健之孫楊江捐銀二萬兩,準楊健入祀鄉賢祠。得旨:「楊健系休致之員,鄉賢鉅典,非可以捐納得之。曾國籓不應遽為陳請,下部議處。」軍興以來,饟空事棘,而帝於名器猶慎之如此。予殉難安徽布政使劉裕珍世職,謚勤壯。癸未,前協辦大學士湯金釗、兵部尚書特登額重宴鹿鳴,加宮銜,賜御書匾額。丙戌,張亮基奏獲戕害大員之賊目王小湧,摘心遙祭。得旨,即傳知佟鑒、謝子澄家屬告祭。命托明阿幫辦僧格林沁軍務。癸巳,以青麟為湖北巡撫,崇綸丁憂,仍同守城。戊戌,張亮基奏捻賊渡河由豐縣竄入單縣,官兵迎擊獲勝,復陷金鄉。

三月庚子朔,張亮基奏賊陷鉅野、鄆城。辛丑,命載齡帶兵一千駐防河間,桂齡、臺祿帶馬步兵千五百駐防德州。駱秉章奏賊陷岳州,曾國籓回省防堵,留候補道胡林翼楚南剿賊。壬寅,賊陷陽穀,知縣文穎蒞任五日,死之,優恤建祠。甲辰,賊由陽穀、冠縣竄至清河之小灘,又分竄至臨清之李官莊。乙巳,命勝保迎擊山東竄賊,布政使崇恩奏帶兵扼守臨清州。辛亥,上耕耤田。丁巳,賊陷臨清。越十日,官軍復之,潰匪南竄,勝保追擊。曾國籓奏剿賊岳州失利,回守長沙。下部議處。

夏四月庚辰,順承郡王春山薨。阜城賊竄連鎮,僧格林沁追擊圍之。壬午,勝保奏馬隊追剿臨清潰匪,全數殄滅。得旨嘉獎,加太子少保,德勒克色楞、善祿黃馬褂。己丑,予告大學士潘世恩卒。內大臣壁昌卒。辛卯,鮑起豹罷,以塔齊布署湖南提督,曾國籓奪職剿賊。曾國籓克復湘潭,塔齊布、彭玉麟、楊載福剿賊大勝,靖港賊退。

五月己亥朔,葛云飛祠成,賜御書匾額。廓爾喀國王表請出兵剿賊。溫詔止之。辛丑,孫瑞珍免,以硃鳳標為戶部尚書,趙光為刑部尚書,彭蘊章為工部尚書。副都統綿洵追賊於豐縣,敗之,賜巴圖魯勇號。乙巳,連鎮賊首李開方竄陷高唐州,勝保督兵追之。壬申,上祈雨大高殿。丁巳,祈雨天神壇。庚申,荊州將軍官文奏官軍收復監利縣、宜昌府城。敕塔齊布統軍赴湖北剿賊。前湖北巡撫崇綸以託病奪職。壬戌,雨。癸亥,和春、福濟奏收復安徽六安州城。

六月戊辰朔,賜臨清、冠縣被賊難民一月口糧。江西賊竄湖北德安。庚辰,許乃釗免,以吉爾杭阿為江蘇巡撫。詔曰:「中國海口,除通商五口外,夷船向不駛入。近日乃有闌入金陵、鎮江之事,意欲何為?葉名琛即向各國夷酋正言阻止。」辛巳,詔直省團練殺賊者,建立總坊,入祀忠義祠,婦女遇難捐軀者,入祀節孝祠。癸未,賊陷武昌。臺湧罷,以楊霈為湖北巡撫,署總督。命曾國籓由岳州進剿,英桂赴信陽防堵。副都統達洪阿卒於軍,贈都統。辛卯,敕葉名琛剿捕廣東會匪盜船。鑄鐵錢、鉛錢。

秋七月辛丑,湖北賊陷岳州,連陷常德。壬子,詔:「青麟棄城逃走,遠赴長沙,飭官文傳旨正法。」副都統特爾清額卒於軍。庚申,湖南水師克復岳州,予革職侍郎曾國籓三品銜。命道員胡林翼攻剿常德。壬戌,楊霈奏克復沔陽,賊陷安陸。

閏七月戊辰,湖北官軍克復安陸。丁丑,欽差大臣琦善卒於軍,以托明阿為欽差大臣,督辦揚州軍務。庚辰,楊霈奏克復京山、孝感、天門、黃陂、麻城等城。向榮奏官軍收復高淳。丙申,和春奏克復太平。

八月庚子,官文奏連復嘉魚、蒲圻。癸卯,廣東土匪陷肇慶,調湖南、福建兵剿之。甲寅,湖南官軍由城陵磯進攻通城。癸亥,英、美二國兵船抵天津海口,命桂良蒞事。

九月辛未,湖北、湖南官軍攻克武昌、漢陽。授楊霈湖廣總督,曾國籓以二品銜署湖北巡撫,塔齊布賜黃馬褂,李孟群、羅澤南、李續賓並升敘有差。殉難布政使嶽興、署按察使李卿穀均予謚建祠。壬午,湖北官軍克復黃州。命曾國籓以兵部侍郎銜會塔齊布督軍東下。甲申,裕瑞罷,以黃宗漢為四川總督,何桂清為浙江巡撫。戊子,安徽官軍收復廬江。乙未,魏元烺卒,以翁心存為兵部尚書。

冬十月丙辰,以花沙納為吏部尚書,全慶為工部尚書,領國子監。調文慶為滿洲都統,奕興為漢軍都統,奕山為內大臣。丁巳,曾國籓奏水陸軍攻半壁山賊,斃賊萬餘。戊午,以扎拉芬泰為伊犁將軍。甲子,曾國籓等奏攻克田家鎮,予楊載福、彭玉麟升敘。湖北軍收復蘄州。

十一月丁丑,上詣大高殿祈雪。庚辰,楊霈奏克復廣濟、黃梅。戊子,羅繞典卒,以恆春為雲貴總督,王慶雲為山西巡撫,吳振棫為陜西巡撫。綏遠城將軍善祿卒於軍。庚寅,大學士、軍機大臣祁俊藻致仕。以賈楨為大學士,翁心存為吏部尚書,周祖培為兵部尚書,許乃普為左都御史。癸巳,湖北賊陷安徽英山。安慶賊竄九江、湖口,及於吳城。

十二月乙未,曾國籓奏攻克小池口,上嘉獎之,賜狐腿黃馬褂。戊戌,和春奏克復英山。以克復英、霍兩縣均資民力,免三年漕糧。辛丑,袁甲三奏舉人臧紆青進攻桐城,力竭陣亡,贈三品銜,予世職。乙卯,封奕紀之子載中貝勒,嗣隱志郡王,改名載治。貴州官兵擊賊,敗之,解興義城圍。辛酉,安徽官軍克復含山。僧格林沁奏攻毀西連鎮賊巢。癸亥,祫祭太廟。

是歲,免河南、山東、山西、福建、湖南、廣西等省一百二十九州縣,又廣西土州縣十二災賦有差。朝鮮、琉球入貢。

五年乙卯春正月己巳,四川官軍克復貴州桐梓。壬申,貴州官軍剿匪雷臺山,擒匪首陳良模。甲戌,以江、浙漕米不敷京倉支放,命怡良開辦米捐解京。戊寅,吉爾杭阿奏克復上海縣城。詔嘉獎之。辛巳,湖北賊由黃梅回竄漢口,楊霈退守德安,奪職,仍留任。癸未,江西官軍克復武寧。乙酉,僧格林沁奏攻克連鎮,首逆林鳳祥就擒。封僧格林沁親王,移軍山東,攻剿高唐踞匪。欽差大臣勝保師久無功,褫職逮問。丙戌,浙江樂清土匪滋事,剿平之。敘連鎮功,西凌阿、瑞麟、慶祺、綿洵、拉木棍布扎布、棍楚克林沁各予優賚。

二月甲午朔,王懿德奏夷商來閩販茶,租賃民房久居,藉收茶稅,從之。以法將剌尼樂助攻上海,賚綢四端、銀一萬兩,從吉爾杭阿請也。己亥,上御經筵。僧格林沁奏克復高唐州,餘匪竄入馮官屯。辛丑,福建匪徒作亂,剿平之。戊午,鄂賊北竄,敕僧格林沁調撥馬步兵三四千赴河南助防。

三月甲子,廣東官軍復海豐。皖賊陷徽州。乙丑,上謁西陵。賊陷武昌,巡撫陶恩培死之,以胡林翼署湖北巡撫。辛未,上還京。辛卯,貴州匪首楊鳳捕誅,餘匪平。

夏四月乙未,安徽官軍收復婺源。以額駙景壽為御前大臣。丁未,江西賊陷廣信。庚戌,僧格林沁等奏攻克馮官屯賊巢,擒獲首逆李開芳,餘匪盡殲。得旨:欣慰,僧格林沁即以親王世襲,許乘肩輿,德勒克色楞加貝勒銜,餘各升敘。江西官軍復弋陽。浙江賊陷開化。己未,西安將軍扎拉芬在湖北剿賊陣亡,優恤之。褫楊霈職,以官文為湖廣總督,綿洵為荊州將軍,瑞麟為西安將軍。以西凌阿為欽差大臣,赴湖北剿賊。庚申,江西官軍復饒州、廣信及興安。辛酉,廣東官軍剿匪獲勝,水陸股匪悉平。

五月丙寅,恤福建陣亡知縣高鴻飛,入祀京師昭忠祠,並於臺灣建祠。丁卯,向榮奏剿賊三山,勝之。戊辰,廣東官軍復河源等縣,殲賊於三水。辛未,上禦乾清門,奉命大將軍惠親王綿愉、參贊大臣親王僧格林沁恭繳大將軍印、參贊關防。壬申,詔曰:「興辦團練,原以保衛鄉閭。而河南迭有抗糧、抗官之事。似此相率效尤,流弊甚大。各督撫其尚加意整頓,勿令日久釀患。」是時,山東已有黑團之害,尚未上聞。其後卒以兵力平之。乙亥,以柏葰為熱河都統。戊寅,楊霈軍復隨州。癸未,河南軍收復光山。丁亥,胡林翼奏分督水陸各軍力攻武、漢,四戰四勝。得旨,迅圖克復。詔曰:「朕聞雲南回民易滋事端,屢有聚眾抗糧之事。恆春、舒興阿務將首要各犯懲處,勿令日久蔓延。」以李鈞為東河河道總督。

六月乙未,江西賊陷義寧。丁酉,提督鄧紹良克復休寧。乙巳,廣東官軍收復封川,殲賊於虎門洋面。丙辰,河南蘭陽河溢。己未,敕安徽徽寧池廣道照臺灣道專摺奏事。辛酉,官文奏官軍克復云夢、應城。

秋七月壬戌朔,尊皇貴太妃為康慈皇太后。廣東賊陷湖南郴州、宜章。癸亥,陳啟邁奪職,以文俊為江西巡撫。己巳,向榮奏克復蕪湖。庚午,皇太后崩。丁丑,西凌阿進剿德安賊匪不利,退守隨州。命都興阿自馮官屯移軍剿之。辛巳,恭親王奕罷直軍機,回上書房讀書。以文慶為軍機大臣。癸未,廣東官軍收復肇慶府、德慶州。甲申,山西陽城土匪滋事,剿平之。丁亥,官文奏克復漢川。

八月辛卯朔,胡林翼督軍攻克漢鎮,進圍漢陽。甲午,英桂奏邱聯恩擒獲捻首易添富、王黨等誅之。己亥,湖南提督塔齊布卒於軍,贈將軍。庚子,上大行皇太后尊謚曰孝靜康慈皇后。喀什噶爾回匪入卡,倭什琿布派兵逐出之。戊申,廣東官軍連復連州、三江、連山,解永安城圍。

九月甲子,大學士卓秉恬卒。乙丑,以劉鉦為漢軍都統。庚午,命文慶、葉名琛協辦大學士。癸酉,發內帑十萬兩續賑直隸、山東災民。壬午,四川馬邊夷匪滋事,官軍剿平之。癸未,捻首張洛行由歸德南竄,命提督武隆額剿之。乙酉,命官文為欽差大臣,督辦湖北軍務。浙軍克復安徽休寧、石埭。戊子,調鄧紹良為固原提督。

冬十月丁酉,和春、福濟奏克復廬州府城。得旨嘉獎,賜和春黃馬褂,福濟太子少保,免合肥三年額賦。辛丑,貴州苗匪陷都江。壬寅,官文奏克復德安。戊申,石達開回竄湖北,胡林翼堵剿之。壬子,永免河南攤徵河工加價銀四十萬兩。

十一月甲子,胡林翼奏,羅澤南、李續賓迎擊石達開、韋俊於羊樓峒,敗之;請購洋砲擊賊。敕葉名琛採購洋砲六百尊,由湖南水運湖北應用。辛未,廓爾喀夷人占踞後藏濟嚨。德興卒,調麟魁為刑部尚書,以瑞麟為禮部尚書。戊子,官文奏攻克咸寧、金口,並報江西賊陷義寧,檄飭羅澤南回剿。得旨:「羅澤南正在攻剿,武漢契緊,不可回剿。」詔令曾國籓等遣周汝筠前赴崇、通,為羅澤南後路援應。和春等奏捻匪李兆受竄踞英山,道員何桂珍密謀會捕,不克,死之。

十二月辛卯,上詣大高殿祈雪。丙申,江西賊陷臨江、瑞州,敕曾國籓撥兵剿之。戊戌,留江蘇漕米二十萬石濟江南軍。癸卯,廣西官軍收復興安。貴州賊徐廷傑陷鎮筸,分陷思南。乙巳,命文慶、葉名琛為大學士,桂良、彭蘊章協辦大學士,柏葰為戶部尚書,奕湘為盛京將軍,英隆為熱河都統。丙午,以鄭親王端華為滿洲都統,奕山為黑龍江將軍。命西凌阿赴河南防剿。庚戌,捻匪張洛行回竄歸德。癸丑,命英桂督剿豫、東、皖三省捻匪。景淳奏陳防夷情形,上嘉獎之。駐藏大臣赫特賀奏馳抵後藏籌御大略。得旨:「江孜、定日汛、馬布加各地,均屬中道要害,即宜扼守。噶布倫中擇其為夷情信仰者,令協同辦事,以輔兵力之不及。生擒夷人,暫留營中,令來往通信,以示羈縻。樂斌等所擬六條,下該大臣知之。」丁巳,祫祭太廟。

是歲,免直隸、山東、湖北、廣西、貴州等省二府一百五十八州縣,又廣西三十八土州縣災賦,江蘇鹽場場課各有差。朝鮮、琉球入貢。

六年丙辰春正月己未朔,惇郡王奕脤復親王。以奕山為御前大臣,貝勒載治御前行走。壬戌,楊以增卒,以庚長為江南河道總督。壬申,賊擾湖南晃州、麻陽,官軍擊走之,斬賊首何祿。乙亥,詔駱秉章檄知府劉長佑赴江西剿賊。戊寅,廣東提督昆壽剿歸善賊,平之。辛巳,提督秦定三攻克舒城。

二月壬辰,詔湖南苗弁剿匪出力,準其留營序補。戊戌,上御經筵。辛丑,順天府尹蔣琦淳疏進克己、復禮二箴,上嘉納之。丙午,英、美二國求改條約,下葉名琛知之。丁未,調吉林、黑龍江、察哈爾、綏遠城兵赴山東、河南剿賊。己酉,酌增直省文員減成養廉。壬子,命福興幫辦江南軍務。丙辰,廓爾喀請罷兵。丁巳,貴州官軍攻克銅仁。

三月己未,瓜州賊出竄運河,托明阿追剿之。奕湘免,以慶祺為盛京將軍。壬戌,湖南官軍克復永明、江華。劉長佑軍入江西,復萍鄉。江西賊陷吉安。癸亥,上耕耤田。甲子,江南賊再陷揚州,奪托明阿、雷以諴職,授德興阿欽差大臣,少詹事翁同書副之。乙丑,石達開陷瑞州,詔廣東堵剿。丁卯,釋賽尚阿、訥爾經額於戍所。乙亥,提督鄧紹良力攻揚州,克之,命幫辦德興阿軍務。賊竄江浦。丁丑,羅澤南力攻武昌,陣亡,贈巡撫,賜恤予謚。戊寅,賊陷江西建昌。命浙江學政萬青藜、布政使晏端書督辦三衢防務。庚辰,穆宗生母懿嬪那拉氏晉封懿妃。曾國籓攻賊樟樹失利,下部議處。癸未,恆春奏軍務省分督撫,請許單銜奏事,從之。丙戌,張國樑軍攻克浦口。

夏四月戊子,粵賊復陷儀徵,官軍尋復之。甲午,貴州軍復郎岱。丙申,雲南楚雄漢、回構釁。己亥,江西軍復進賢。辛丑,奉天金州地震。癸卯,安徽賊陷寧國。丙午,前協辦大學士、致仕光祿寺卿湯金釗卒,贈尚書。辛亥,賜翁同龢等二百一十六人進士及第出身有差。丙辰,德興阿奏官軍攻賊三汊河,毀其巢。

五月辛酉,以穆克德訥為廣州將軍,都興阿為江寧將軍。壬戌,湖北通城官軍失利,道員江忠濟死之。江蘇巡撫吉爾杭阿擊賊鎮江之黃泥州,不勝,死之,贈總督。以趙德轍署江蘇巡撫。甲子,江南賊撲九華山營盤,陷之。河南軍復光州。復西凌阿都統。袁甲三復三品卿。丁丑,賊陷溧水。

六月丙戌朔,金陵賊撲陷大營,官軍退守丹陽,奪向榮、福興職。戊子,以按察使徐宗幹幫辦安徽防務。命怡良雇募火輪船入江剿賊。敕河南、廣東撥兵,和春、傅振邦赴援江南。丁未,葉名琛奏英、美、法各國公使以定約十二年,請赴京重修條約。詔酌允變通,阻止來京。辛亥,永定河溢。江西賊陷饒州。

秋七月辛酉,廣東援軍連復江西上猶、雩都,解贛州城圍。王懿德呈進美國國書,得旨:「更換條約,難以準行,仍令回廣東商訂。」丁卯,命總兵張國梁幫辦向榮軍務。壬申,江西官軍連復南康、饒州。癸酉,欽差大臣向榮卒於軍。丙子,甘肅撒拉回匪滋事,官軍剿平之。命和春馳赴丹陽剿賊,鄭魁士接辦安徽軍務。湖北援軍克復江西新昌、上高。赫特賀奏廓爾喀與唐古忒和成,撤回戍兵。

八月戊子,黃宗漢罷,以吳振棫為四川總督,譚廷襄為陜西巡撫。癸巳,命舒興阿嚴辦回匪,舉行鄉團。癸卯,廣西官軍復上思州、貴縣。丁未,貴州賊陷都勻、施秉,進陷古州。戊申,安徽官軍攻克三河。己酉,江西會匪攻陷廣昌、南豐、新昌、瀘溪。

九月乙卯朔,日有食之。戊午,京師米貴,開五城飯廠,並撥倉穀制錢賑固安六州縣饑民。己巳,雲南土匪陷浪穹。庚午,江南官軍攻高淳,克之。癸酉,安徽官軍復無為州。丁丑,文慶等疏進孟保繙譯大學衍義,命校刊頒行。壬午,西寧黑番族滋事,提督索文剿平之。易棠病免,以樂斌為陜甘總督,有鳳為成都將軍,東純為福州將軍。

冬十月丙戌,貴州賊陷臺拱、黃平。庚寅,官文剿襄陽匪徒,平之。甲午,命英桂、秦定三會剿渦河、蒙城捻匪。丁酉,安徽官軍克復和州。雲南大理回匪戕官踞城。壬寅,襄樊賊犯鄧州。河南賊由夏邑趨擾徐州。甲辰,浙江官軍再復休寧。予前巡撫張芾三品卿。丁未,廣西右江鎮標兵變,勞崇光討平之。壬子,何桂清奏浙軍進克黟縣,徽州肅清。邵燦奏官軍擊退捻匪,徐州解圍。以常清為伊犁將軍。

十一月乙卯朔,宣宗實錄成。以彭蘊章為大學士,翁心存協辦大學士,許乃普為工部尚書,硃嶟為左都御史。辛酉,雲南官軍克復姚州。乙丑,升文昌入中祀。命鄭魁士移軍會英桂剿捻匪,秦定三會福濟剿皖匪。丙寅,命勝保赴安徽軍營。辛未,大學士文慶卒。英人在廣東以查船構釁,放砲攻城。紳團憤擊之,殲數百人。敕葉名琛相機辦理。壬申,命柏葰為軍機大臣。乙亥,江西賊陷撫州。戊寅,楚軍道員劉長佑連復江西袁州、分宜,加按察使銜,予其父母三品封典,予巡撫駱秉章花翎。英桂奏攻破雉河集賊巢。庚辰,上臨大學士文慶第賜奠。壬午,胡林翼克復武昌,癸未,官文克復漢陽,均得旨嘉獎。貴州軍攻克都勻。

十二月乙酉,湖北官軍攻克老河口。丙戌,上祈雪。戊子,以肅親王華豐為內大臣。己丑,詔曰:「湖北累為賊踞,小民兵燹餘生,瘡痍可念。現在武、漢既復,亟宜援救民瘼。錢糧分別蠲緩,災黎作何撫恤,其速籌議以聞。」湖北官軍連復武昌縣、黃州府城。甲午,胡林翼奏陳湖北兵政吏治。得旨:「既能確有所見,即當實力舉行。」丙申,官文奏剿辦隨州土匪,匪首就擒。續報官軍連復興國、大冶、蘄水、蘄州、廣濟。辛丑,皖、浙官軍克復寧國,賜何桂清花翎。癸卯,以湖南官軍剿除湖北崇、通賊匪,加候選道王珍按察使銜。甲辰,官文奏官軍在九江焚毀賊船。詔曾國籓激厲將士,由湖出江,以便合剿。戊申,山東官軍剿斃捻首王方云。湖北官軍克復黃梅。己酉,命桂良為大學士,柏葰協辦大學士。以譚廷襄為直隸總督,曾望顏為陜西巡撫。壬子,祫祭太廟。

是歲,免直隸、江蘇、山東、山西、河南、湖南、貴州等省一百六十五州縣被災、被賊額賦,又免江蘇六場鹽課各有差。朝鮮入貢。

七年丁巳春正月庚午,怡良奏傅振邦克復高淳,張國樑進取句容。何桂清奏浙省援剿,內防本境,外保鄰封。得旨嘉獎。調全慶為兵部尚書,文彩為工部尚書,肅順為左都御史。廣西太平府土匪平。丙子,召西凌阿、崇安回京。加勝保副都統銜,幫辦剿匪事宜。王履謙回籍,命李鈞接辦河防。己卯,葉名琛奏防剿英夷獲勝。得旨:「控制外夷,非內地可比。定海前事,可取為鑒。其務操縱得宜,勿貽後悔,朕不為遙制也。下蘇、直、閩、浙各督撫知之。」

二月乙酉,曾國籓奏克復建昌。丙戌,上御經筵。辛卯,湖北官軍收復宜昌。甲午,雲南賓川回匪作亂。甲辰,湖北賊陷遠安、荊門,官軍擊走之。丁未,安徽賊匪上犯黃梅,都興阿擊敗之。安徽匪陷六安。壬子,英桂、勝保奏剿辦捻匪,奪回烏龍集,進規固始。

三月癸巳朔,曾國籓丁父憂,給假治喪,命楊載福暫統水軍,彭玉麟副之。丙辰,湖北官軍唐訓方、巴揚阿剿南彰匪徒,敗之,賊首劉尚義降。貴州提督孝順兵潰於都勻,死之。己未,襄樊賊陷河南內鄉,官軍擊復之。詔怡良「密查張國樑是否與和春意見不合。軍中統帥,全在能得人心,倘駕馭無方,使健將不肯出力,貽誤非輕」。癸亥,上耕耤田。丁卯,以耆齡為江西巡撫。庚午,敘克復武、漢功,協領多隆阿以副都統用。辛未,恆春奏回匪滋擾,將領乏員,請調鄖陽鎮總兵王國材來滇協剿,從之。壬申,江西官軍攻景德鎮,不利,都司畢金科戰歿,劉長佑復敗於新喻。辛巳,廣西橫州土匪滋事,廣東官軍剿平之。葉名琛奏英船退出省河。得旨:「總宜弭此釁端,不可使生邊患。」

夏四月甲申,恆春奏迤西回匪降。德勒克多爾濟奏俄國請遣使來京,詔止之。丁亥,江西賊竄福建,陷邵武、光澤。癸巳,怡良以病免,命何桂清為兩江總督。乙未,貴州賊陷永從。丁酉,湖南援軍劉長佑攻克江西新喻。

五月丙辰,薩迎阿卒,以劉鉦署西安將軍。湖北官軍克復江西奉新、靖安、安義。癸亥,李孟群奏赴援廬州,克復英山。福建賊陷汀州。丙子,德勒克多爾濟奏俄使由天津來京,敕譚廷襄羈縻之。

閏五月甲申,和春奏克復溧水。乙酉,曾國籓奏請終制,溫旨留之,仍令迅赴江西視師。庚寅,雲南武定州回匪滋事,官軍剿平之。李孟群奏擊敗霍丘竄賊,得旨嘉獎。丁酉,勝保攻正陽關,不利,道員金光箸死之,贈布政使。庚子,俄人以兵至海蘭泡,建營安砲,要求通商。命奕山拒之。辛丑,何桂清奏請知府溫紹原復官,辦理六合鄉團。詔吉林、黑龍江兵久勞於外,酌量撤回。壬寅,慶英奏浩罕勾結回匪,占踞英吉沙爾城,集兵剿之。以張國樑為湖南提督。癸卯,福建官軍收復光澤、汀州,踞匪出竄連城,擊敗之。

六月壬子,召舒興阿來京,以桑春榮為雲南巡撫。癸丑,福建官軍收復邵武。乙卯,江南官軍克復句容,加和春太子少保,賜張國樑黃馬褂。辛酉,王珍援江西吉安,連戰勝之,賜巴圖魯勇號。丁卯,河南南陽土匪平。癸酉,福建官軍收復泰寧、建寧。俄夷至天津遞國書,命文謙卻之。永定河決。乙亥,雲南回匪犯省城,恆春自盡。事聞,調吳振棫為雲貴總督,以王慶雲為四川總督,恆福為山西巡撫。丙子,江西官軍收復龍泉。戊寅,命許乃釗幫辦江南軍務,張亮基予五品銜,幫辦雲南剿匪事宜。

秋七月乙酉,李孟群奏收復霍山。己丑,河南官軍收復鄧州。癸巳,命奕山會集俄使勘定黑龍江兩岸邊界。甲午,貴州官軍收復錦屏。湖北官軍攻剿黃梅大勝,總兵王國材力戰陣歿,贈提督,賜恤建祠。甲辰,命都興阿幫辦官文軍務。

八月己酉朔,日有食之。壬子,福建官軍收復寧化。癸丑,江西官軍克復瑞州。丁丑,法福理奏克復英吉沙爾回城,解漢城圍。戊寅,官文、胡林翼奏湖北全境肅清。得旨:「胡林翼親督所部攻克小池口賊城,即乘此聲威規復九江,以振全局。」先是,林翼密奏欲保鄂省而復金陵,惟有先取九江,次復安慶,始握要領,故明詔從之。

九月庚辰,湖南援贛道員王珍卒於軍,贈布政使。壬午,勝保奏克復正陽關,又奏鳳臺生員苗沛霖藉團聚眾。得旨:「正當示之不疑,藉消反側。」丙戌,法福理奏收復喀什噶爾回城。庚寅,湖北賊陷舒城。河南捻匪陷南陽。丙申,江西官軍克復東鄉。丁未,湖南援黔官軍克復黎平。

冬十月戊申朔,官文、胡林翼奏,李續賓等水陸齊進,攻克江西湖口縣城。勝保、袁甲三奏,總兵硃連泰、史榮椿等攻剿捻匪,平毀韓圩賊巢。蔣霨遠、佟攀梅奏,剿辦苗匪、教匪,斬擒多名,都勻賊退。河南官軍敗賊於南召,進剿裕州、泌陽餘匪。己未,李孟群剿捻匪於獨山,不利,兵潰。乙丑,湖北援軍李續賓等攻克彭澤。廣西官軍收復南寧。戊辰,胡林翼奏漕糧積弊,請改章徵收,以濟軍需,從之。庚午,河南賊入武勝關,直撲商南,陜西官軍擊走之。甲戌,以楊載福為福建陸路提督。以李續賓為浙江布政使。

十一月戊寅朔,英桂奏德楞泰敗賊於盧氏,邱聯恩敗賊於淅川。安徽賊陷和州、霍山。楊載福克復望江、東流、銅陵。乙酉,駱秉章奏蔣益澧、江忠濬援剿廣西,連戰獲勝,進圍平樂。戊子,胡林翼疏薦布衣萬斛泉、宋鼎、鄒金粟等。甲午,廓爾喀奉表輸誠,頒賞珍物。丙申,德興阿等奏克復瓜州。得旨嘉勉,賜雙眼花翎、騎都尉世職。翁同書以侍郎用,鞠殿華加提督銜。戊戌,和春奏同張國樑克復鎮江,賜和春雙眼花翎、輕車都尉世職,張國樑騎都尉世職,何桂清太子太保。庚子,英桂奏敗賊於汝州,豫西肅清。辛丑,永定河合龍。

十二月辛亥,耆齡奏曾國荃攻克吉水。駱秉章、勞崇光會奏官軍攻克平樂。廣西賊陷慶遠。丙辰,督辦三省剿匪副都統勝保奏請皖兵悉歸節制。得旨:「勝保尚屬勇敢,若平其躁氣,斂其驕心,可為有用之材,何庸自行★J6請。」庚申,英人入廣東省城,劫總督葉名琛以去。詔革名琛職,以黃宗漢為兩廣總督,柏貴署理。乙亥,李孟群奏粵、捻合股東竄,偪近商、固。命勝保嚴防之。丙子,祫祭太廟。

是歲,免直隸、江蘇、山東、山西、河南、陜西、湖南、廣西等省二百三十五州縣衛,廣西四土縣被災、被賊額賦有差。朝鮮、琉球入貢。

八年戊午春正月己卯,佟攀梅罷,以蔣玉龍為貴州提督。丙戌,敕王懿德籌備海防。庚寅,江西官軍收復臨江。

二月庚午,官軍克復秣陵關,進圍金陵,加和春太子太保,張國樑雙眼花翎,陣亡總兵虎坤元優恤世職。

三月丁丑朔,勝保奏剿賊獲勝,固始解圍。得旨嘉獎。戊寅,俄船至天津。敕譚廷襄防堵。癸未,江北官軍克復江浦,道員溫紹原復官。庚寅,福濟奏收復和州。貴州賊陷都勻,前提督佟攀梅死之。

夏四月丙午朔,譚廷襄奏俄人不守興安舊約,請以烏蘇里河、綏芬河為界,使臣仍請進京。得旨:「分界已派大員會勘,使臣非時不得入京,駁之。」丁未,江西賊竄入福建,陷政和、松谿。戊申,俄人請由陸路往來,英人、法人請隔數年進京一次,詔不許。勝保奏捻首李兆受乞降,許之。己酉,安徽賊陷麻城,另股陷蒙、亳、懷、宿,詔袁甲三剿之。詔許俄之通商,不許進京。庚戌,賊陷和州。雲南大理回匪陷順寧。戊申,詔譚廷襄告知英人、法人,減稅增市,俟之粵事結日,彼時再議來京。庚戌,江西賊陷常山、開化,命總兵周天受加提督銜,專辦浙防,道員饒廷選防守衢州。辛亥,譚廷襄呈進美國國書,詔許減稅率、增口岸,仍不許入京。乙卯,英、法兵船入大沽,官軍退守。命僧格林沁備兵通州。戊午,江西官軍復雩都、樂安、崇化、宜黃。辛酉,英、法船抵津關。命大學士桂良、尚書花沙納往辦夷務。江西賊竄浙江,陷處州及永康。壬戌,湖北官軍克復九江,加官文、胡林翼太子少保,李續賓加巡撫銜。乙丑,英、法兵退三汊河,與俄、美來文,請求議事大臣須有全權便宜行事,始可開議。桂良等以聞,詔許便宜行事。丙寅,命僧格林沁佩帶欽差大臣關防,辦理防務。戊辰,勝保奏克復六安。乙巳,敕各省軍營挑練馬隊。庚午,命和春兼辦浙江軍務。英船開出大沽。桂良等奏英人之約於鎮江、漢口通商,長江行輪,擇地設立領事,國使駐京。上久而許之。

五月丙子,皖匪陷湖北黃安。桂良、花沙納奏,英使堅偪立約,不見耆英。耆英請回京,詔止之。戊寅,捻匪陷懷遠。己卯,奕山奏請黑龍江左岸曠地割畀俄人。甲申,桂良等奏俄允代轉圜,先允俄人陸行。丁亥,命廷臣集議和戰二者,兩害取其輕。戊子,桂良等奏英人謂我徒事遷延,即棄和言戰。大學士裕誠卒,上親臨賜奠。庚寅,桂良等奏進英、法訂約五十一款,並請先訂俄、美條約。壬辰,湖北官軍復黃安、麻城。福建官軍復光澤。廣東官軍復廣西梧州。敕耆齡檄調蕭啟江、張運蘭、王開化各軍由祁門進援浙江。癸巳,耆英擅回京,賜自盡。太傅杜堮卒,上親臨賜奠。乙未,命曾國籓辦理浙江軍務。丁酉,桂良、花沙納奏進俄、美、英、法四國條約。得旨:「既巳蓋用關防,今復硃批依議,宜示四國照此辦理。至通商稅則,在上海議之。」庚子,江北官軍克復江浦、來安。甲辰,夷船全數退出內河。命吏部侍郎匡源、內閣學士文祥在軍機大臣上行走。

六月己酉,命桂良、花沙納、侍郎基溥、武備院卿明善前往江蘇會議通商稅則。江西官軍復新城、金谿。癸丑,福建匪陷建寧。福興罷,以周天受統其軍赴援福建。召桑春榮來京,以張亮基為雲南巡撫。甲寅,廣西軍復象州。丁巳,浙江賊陷壽昌,官軍尋復之。福濟以不職,奪宮銜,解任。以翁同書為安徽巡撫。庚申,論天津失事狀,譚廷襄解任,提督張殿元遣戍。以慶祺為直隸總督,玉明為盛京將軍。丁卯,福建道員趙印川剿匪,死之。浙江官軍復常山、開化。江西援軍復浙江武義、永康、衢州,紹興城圍解嚴。瑞麟請籌款修築天津營壘砲臺,下僧格林沁辦理。辛未,俄人請停辦驛站羊隻,詔庫倫大臣援舊事拒之。壬申,賞刑部員外郎段承實五品卿銜,幫辦會議稅則。曾國籓奏由九江登陸赴浙,詔嘉勉之。浙江軍復縉雲。

秋七月甲戌朔,奕山、景淳奏俄人闖越黑河口,欲入松花江,於烏蘇里建屋安砲。詔勘明吉、黑地界,據理拒絕。乙亥,以李孟群署安徽巡撫。丁丑,從法福理請,升喀什噶爾領隊大臣為辦事大臣。周天受攻復浙江處州,移軍福建。癸未,詔曾國籓衢、嚴肅清,改援福建。乙酉,楊載福收復安徽建德。癸巳,湖北巡撫胡林翼丁母憂,詔在任守制,給假、給銀治喪。丙申,賊陷廬州,李孟群奪職留軍,以勝保為欽差大臣,督辦安徽軍務,袁甲三援剿三省捻匪。丁酉,福建軍復建陽、光澤,賊陷寧化。庚子,召晏端書來京,以胡興仁為浙江巡撫。壬寅,張芾軍復龍泉,賜花翎。

八月癸卯朔,復設天津水師。甲辰,福建軍復政和、松谿。勝保奏發逆偽英王陳玉成竄店埠、梁園,直撲定遠。庚戌,李定太剿賊玉山,勝之,解其圍。辛亥,蔣益澧援軍復廣西慶遠,擢按察使。丙辰,周天受援福建,克復浦城,進克寧化。捻匪陷豐縣。辛酉,捻匪竄山東,陷單縣。調英桂為山西巡撫,恆福為河南巡撫。乙丑,官軍復豐縣。捻匪陷曹縣,尋復之。何桂清請以海關盈餘用充軍饟,允之。壬申,江北軍在浦口失利,奪德興阿、鞠殿華職。和春奏:「浦口失利,已飛調援浙之師徑赴六合。探聞閩省回竄之賊,將由寧、太以援金陵,明系城賊圍急,令其部眾到處竄擾,以分我兵力。請飭各路自行援剿,勿致掣動全局。」上是之。

九月癸酉朔,湖北官軍多隆阿克復太湖。乙亥,詔以「天長、儀徵相繼失陷,六合危急,溫紹原雖素得民心,日久亦恐難支。即調周天培一軍分援六合、德安,一軍前往援應。」辛巳,官文、胡林翼奏李續賓、都興阿分路克復桐城、潛山,多隆阿進攻石牌,鮑超力攻雷公埠,均屬得手。詔令聯絡水師進規安慶。湖南官軍克復吉安,予同知曾國荃等升敘有差。壬午,明誼奏俄案議結,互換文憑,開辦通商。賊陷揚州,奪德興阿世職。命柏葰、翁心存為大學士,官文、周祖培協辦大學士。調瑞麟為戶部尚書,肅順為禮部尚書,硃鳳標為戶部尚書,陳孚恩為兵部尚書,瑞常為理籓院尚書,綿森為左都御史。敕總兵毛三元、成明幫辦德興阿軍務。甲午,張國樑攻克揚州,續復儀徵。慶端奏攻克邵武,閩省肅清。戊戌,荊州將軍綿洵卒,調都興阿為荊州將軍,和春為江寧將軍,張國樑為江南提督。己亥,賊陷六合,知縣溫紹原死之。紹原孤守危城,數年百餘戰,力竭而陷。上悼惜之,贈布政使,優恤,建祠予謚。

冬十月癸卯朔,浙江寧海土匪滋事,提督阿麟保剿平之。乙巳,勝保奏克復天長,李兆受在事出力。得旨:「李兆受賜名李世忠,予三品銜、花翎,以參將補用。」己酉,御史孟傳金奏劾舉人平齡硃墨不符,派載垣、端華認真查辦。丁巳,僧格林沁奏天津砲臺工竣。上嘉之,賜御服。己未,江南官軍復溧水。壬戌,命李續賓幫同勝保辦理安徽軍務。戊辰,詔本年鄉試主考、同考官荒謬已極,覆試應議之卷,竟有五十本之多,正考官柏葰先革職,副考官硃鳳標、程庭桂暫行解任,聽候查辦。命莊親王奕仁在御前大臣上學習行走。

十一月壬申朔,移吉林馬隊益袁甲三軍。乙亥,袁甲三請於山東東三府抽釐助饟,許之。己卯,徐澤醇卒,以硃嶟為禮部尚書,張祥河為左都御史。乙酉,援閩、浙軍復浦城、順昌,予周天培提督銜。丙戌,恆福奏官軍剿捻大勝,豫境肅清,總兵傅振邦擢提督,編修袁保恆賜巴圖魯勇號。丁酉,內閣副本庫被盜。己亥,吳振棫以病免,以張亮基為雲貴總督,徐之銘為雲南巡撫。庚子,予陣亡提督鄧紹良優恤建祠。

十二月丁未,以宋丞相陸秀夫從祀文廟。庚辰,提督李朝斌收復安徽東流、建德,賜巴圖魯勇號。永州鎮總兵樊燮以乘肩輿劾免。丙辰,以鄭魁士為浙江提督,督辦寧國軍務。己未,李續賓進剿安徽,敗績於三河集,死之,贈總督,建祠予謚。同知曾國華贈道員,予謚。丁卯,以何桂清為欽差大臣,辦理通商事宜。趙德轍免,以徐有壬為江蘇巡撫。庚午,以瑞麟為大學士,調肅順為戶部尚書,麟魁為禮部尚書,瑞常為刑部尚書。祫祭太廟。

是歲,免直隸、安徽、福建、湖北、貴州等省九十二州縣被災、被賊額賦,又免江蘇六場鹽課各有差。朝鮮、琉球入貢。

九年己未春正月壬申朔,桂良等奏英人藉口廣東有事,罷議回粵。乙亥,召袁甲三來京,以傅振邦督剿三省捻匪,伊興阿副之。壬午,江西官軍復瑞金,解安遠圍,別賊陷南安。桂良等奏和約四事。敕俟英使回滬妥議。庚寅,福建匪周灴爔等降,遂復連城。乙未,安徽官軍復建德。丁酉,敕湖北採買馬匹訓練馬隊。戊戌,桂良等奏英使堅欲進京。敕僧格林沁嚴防海口。辛丑,都興阿請假,以多隆阿接統其軍。詔海運漕船探避夷輪。

二月丁未,捻匪薛之元舉江浦降,會李世忠攻克浦口,賜名薛成良,予花翎、三品銜,擢李世忠副將。癸丑,築奉天沿海砲臺。鄭魁士攻克灣沚、黃池賊壘。甲寅,上召廷臣宣示戊午科場舞弊罪狀,依載垣、端華所擬,主考官大學士柏葰坐家人掉換中卷批條,處斬。同考官浦安坐聽從李鶴齡賄屬,羅鴻繹行賄得中,均處斬。乙卯,張芾奏官軍攻克婺源,賊目張淙相等乞降。丁巳,翁同書奏賊陷六安。慶祺卒,以恆福為直隸總督,瑛棨為河南巡撫。癸亥,張國樑奏攻克揚州、儀徵,回軍連克溧水。特詔嘉獎,予輕車都尉世職,李若珠賜黃馬褂。乙丑,曾國籓奏軍抵南康,蕭啟江克復南安。得旨嘉獎,予蕭啟江巴圖魯勇號。詔編修李鴻章交伊興阿差委。

三月辛未朔,前布政使李孟群兵潰於官亭,死之,復官予恤。甲戌,奕山、景淳奏俄人徑至烏蘇里江、綏芬河擇地建屋,並請會勘,詔不許。丙子,捻匪犯河南西華、舞陰,前總兵邱聯恩死之,贈提督,予恤。丁丑,桂良等奏英使兵船北上,阻止不聽。己卯,四川里塘頭人作亂,恩慶討平之,誅其夷目鄧珠。甲申,上祈雨。庚寅,以旱求言。辛卯,李鈞卒,以黃贊湯為東河河道總督。乙未,俄人在黑龍江通商,許免征稅,不許闌入烏蘇里、綏芬。

夏四月辛丑朔,勝保奏克復六安。伊興阿解幫辦,以關保幫辦傅振邦軍務。壬寅,調王慶雲為兩廣總督,黃宗漢為四川總督。江西賊竄湖南郴州、桂陽,劉長佑擊走之。癸卯,勝保奏捻匪張元龍降,收復鳳陽府縣,並復臨淮關。築寧河砲臺。戊申,浙江餘姚土匪作亂,討平之。甲寅,俄使賽善由察哈爾陸路入京,請助槍砲,致於恰克圖。丙辰,上再祈雨。己未,邵燦病免,以袁甲三署漕運總督。調勞崇光為廣東巡撫,兼署總督。賊陷天長,前提督德安死之,復官予恤。辛酉,奕山奏俄船由黑龍江入松花江東駛入海。得旨,不許入綏芬,令特普欽派員阻之。壬戌,王懿德免,以慶端為閩浙總督,羅遵殿為福建巡撫。癸亥,雨。乙丑,賜孫家鼐等一百八十人進士及第出身有差。戊辰,廣東官軍復嘉應,竄賊擾連平,陷樂昌。

五月丙子,詔駱秉章仍令田興恕回援貴州,兆琛一軍撤回。己卯,敕奕山更正俄人條約。辛巳,敕慶昀密查張家口、白城居住俄人。壬午,以周天受督辦寧國軍務。甲申,俄人請赴三姓貿易。詔責奕山辦理輭弱,革副都統吉拉明阿職,枷號烏蘇里地方。庚寅,官文奏探聞石達開將犯四川,詔曾國籓移軍夔州。辛卯,桂良、花沙納奏英酋於本月十三日起碇入京,桂良等即日馳驛回京。大學士翁心存乞休,允之。復以賈楨為大學士。調許乃普為吏部尚書,張祥河為工部尚書,沈兆霖為左都御史。癸巳,駱秉章奏石達開竄湖南,劉長佑、江忠義、田興恕諸軍擊走之。丙申,僧格林沁奏英船鳴砲闖入大沽,我軍開砲轟擊,擊沈多船,並有步隊上岸搦戰,我軍徑前奮擊,擊斃數百名,其兵頭赫姓並被砲傷。我軍亦傷亡提督史榮椿、副將龍汝元等。夷船即時出口。得旨:「將弁齊心協力,異常奮勇,先獎賞銀五千兩,並查明保奏。」戊戌,詔夷人雖經懲創,仍宜設法撫馭,即派恆福專辦撫局,僧格林沁仍辦防務。

六月己亥朔,賜僧格林沁御用珍服。庚子,捻匪陷盱眙,官軍尋復之。壬寅,特普欽奏俄人在三姓者,倔強不肯折回。命景淳前往查辦。癸卯,廣西官軍復上林,匪陷賓州。甲辰,張亮基奏回匪馬凌漢伏誅。丙午,恆福奏美人進京換約,許之。癸巳,英、法兵船全數開行。庚申,以李若珠為福建陸路提督。辛酉,何桂清奏英、法陸續回滬。乙丑,陳玉成陷定遠。丙寅,和春奏水師剿賊獲勝。

秋七月庚午,曾國籓奏克景德鎮,復浮梁。戊寅,勝保奏翁同書潰敗情形。得旨:「汝為統帥,只知炫己之長,不原援人之失。日日聚訟,庸何濟乎!」己卯,美使華若翰遞國書,和約用寶,在北塘交換。庚辰,詔曰:「朕聞勝保專以招降為能事,降眾未盡薙發,張元漋且四外打糧。又報克復盱眙,該縣並無城池,賊因無糧退出,虛報邀功。此次姑不深究。即約束反側,力改前非。凜之!」癸未,御史趙元模奏黃河北流,涸出濱河田畝三四千頃,請辦屯田,寓兵於農,較勝團練。上是之,下袁甲三、庚長議奏。乙酉,詔曰:「王大臣續陳審明科場舞弊之大員父子,及遞送關節之職員,分別定擬。此案程炳採於伊父程庭桂入闈後,接收關節,令家人轉遞場內,程庭桂並不舉發。程炳採處斬,程庭桂免死,遣戍軍臺。謝森墀、潘祖同、潘敦儼等俱免死,發遣新疆。」己丑,駱秉章奏石達開圍寶慶,李續宜援之,立解城圍。癸巳,命李若珠幫辦江南軍務。

八月戊戌朔,崇恩罷,以文煜為山東巡撫。己亥,上御經筵。乙巳,敕恆祺留辦廣東通商。勝保奏李世忠剿賊獲勝,解定遠、滁州圍。詔擢李世忠總兵。廣東官軍復連山、開建。庚戌,命曾國籓駐軍湖口。命都興阿蒞江寧將軍視事,多隆阿接統所部,總理前敵事務。甲寅,景淳奏俄人船在三姓者,現令折回。在烏蘇里者,未肯聽命。詔體察輿情,妥為辦理。己未,美人請先開市,以英、法約議未定,卻之。辛酉,駱秉章奏石達開南陷江華、永門,將入廣西。現飭劉長佑統軍追剿。得旨,田興恕一軍援黔,李續宜一軍回湖北備調。壬戌,發逆、捻匪會攻壽州,官軍擊卻之。御史陳慶松奏科場案內大員子弟陳景彥等贖罪太驟,請仍發遣,嚴旨斥之。甲子,廣東官軍復靈山。

九月戊辰,安徽賊陷霍山、盱眙,勝保擊退之。勝保丁母憂,留營視軍。甲戌,胡興仁罷,調羅遵殿為浙江巡撫。戊寅,王慶雲病免,以勞崇光為兩廣總督。庚辰,官文、胡林翼奏多隆阿攻破安徽石牌,擊破援賊,獲賊目霍天燕石廷玉,得旨嘉獎。己丑,傅振邦奏追剿捻匪,敗之。甲午,曹澍鍾奏石達開圍廣西省城,蕭啟江、蘇鳳文會合蔣益澧分途剿擊,敗之,立解城圍。

冬十月丁酉朔,時享太廟,上親詣行禮。駱秉章奏賊中投出難民,給予免死護照,資遣回籍,原效力者,準其留營,得旨,各省均可照辦。戊戌,雲南官軍克復嵩明,陣斬賊首孫漢鼎。庚子,以曾望顏署四川總督,譚廷襄署陜西巡撫。辛丑,以袁甲三為欽差大臣,督辦安徽軍務。以侍郎匡源、內閣學士文祥為軍機大臣。癸卯,河南捻匪陷蘭儀,圍考城、通許,擾尉氏,分竄直隸、山東。戊申,命總兵田在田幫辦傅振邦軍務。乙卯,授袁甲三漕運總督。丙辰,勝保克復懷遠。江蘇官軍剿六合失利,奪李若珠職。戊午,美使請開潮州、臺灣通商口岸。庚申,河南官軍剿平鄢陵捻匪,西路肅清。壬戌,以明誼為烏里雅蘇臺將軍,景廉為伊犁參贊大臣,崇實為駐藏大臣。乙丑,命官文、曾國籓、胡林翼妥籌四路規皖。

十一月戊辰,滇匪犯敘州,奪萬福職,以皁升為四川提督。辛未,何桂清奏,探聞英、法明春必來尋釁。恆祺奏英兵續行至粵。詔僧格林沁加意津防。丁丑,賊陷浦口,總兵周天培死之,予世職。癸未,特普欽奏俄人在黑龍江左岸占踞五十餘屯,請調西丹墨爾根、布特哈兵交那爾胡善訓練,聯絡旗民參夫,有事抵御,從之。丙戌,命張芾督辦皖南軍務。己丑,曾國籓奏韋志俊以池州降。滇匪陷敘州,另股陷酉陽、秀山。庚寅,四川官軍復筠連、慶符、高縣。乙未,戶部災。

十二月丙申朔,蔣霨遠奏石達開糾黨十餘萬由桂犯黔,將以窺蜀。詔田興恕剿之。戊戌,上詣大高殿祈雪。雲南丘北土匪滋事戕官,官軍討平之。庚子,和春奏官軍攻破江浦賊壘,揚州西界肅清。壬寅,吏部尚書花沙納卒。丙午,何桂清報英、法兵船到滬。以田興恕為貴州提督。辛未,援黔湘軍攻復鎮遠。庚申,景淳奏請招集流民參夫,給地設卡,以助邊防,從之。壬戌,袁甲三奏攻克臨淮關,得旨嘉勉,下部議敘,穆騰阿加都統銜。甲子,祫祭太廟。

是歲,免直隸、河南、山東、浙江、貴州等省一百五十七州縣被災、被賊額賦有差。朝鮮、琉球入貢。

十年庚申春正月丙寅,上三旬萬壽,頒詔覃恩。詔先朝壽節有告祭之禮,升殿之儀,本年勿庸舉行,外吏、外籓並停來京祝嘏。加恩親籓,惇郡王奕脤晉親王,貝子奕劻晉貝勒,餘各封賚,及於廷臣、疆臣。戊辰,前寧夏將軍托云保卒。己巳,解勝保欽差大臣,專辦河南剿匪,袁甲三專辦安徽。丁丑,瑛棨以遲解京饟降官,以慶廉為河南巡撫。己丑,刑部主事何秋濤呈進所纂北徼匯編八十卷,上嘉與之,賜名朔方備乘,入直懋勤殿。壬辰,有鳳免,以全亮為成都將軍,占泰為四川提督。甲午,御史白恩佑言津防重大,請預籌後路,以保萬全。得旨:「所奏固是,然駐兵籌饟,甚覺為難。現在津防周備,可勿庸議。」特普欽奏請召集鄂倫春人入伍。從之。扎拉芬泰奏請與俄、廓合攻印度。上曰:「俄非和好也。廓豈英敵?」

二月丁酉,上御經筵。庚子,以劉源灝為貴州巡撫。袁甲三奏克復鳳陽,賜黃馬褂。辛丑,何桂清奏上海英人經華商開導,索兵費一百萬。津約不能更易,入京換約。如不見許,即開船北駛。詔僧格林沁嚴備津防後路。海運漕糧,暫緩放洋。丙午,湖南官軍克復貴州鎮遠。庚戌,捻匪陷桃源,上竄清江,庚長退守淮安。壬子,援桂湘軍克復柳州、柳城,加道員劉坤一按察使銜。甲寅,張芾奏官軍復建德,匪陷涇縣、旌德,連陷太平。己巳,以倭什琿布為禮部尚書,春佑為熱河都統。辛酉,詔和春分兵援浙。

三月乙丑朔,袁甲三奏官軍復清江。庚子,命提督張玉良統軍援浙。丙子,賊陷杭州,巡撫羅遵殿死之。越六日,將軍瑞昌復其城。重賚瑞昌、張玉良等。以王有齡為浙江巡撫。丁亥,上耕耤田。辛卯,浙江官軍克復長興、臨安、孝豐。甲午,何桂清奏夷船北犯。

閏三月癸卯,四川官軍克復蒲江,賊陷名山。丙午,命曹澍鍾督軍四川,以劉長佑為廣西巡撫。丁未,賊陷溧水,連陷句容。以張玉良為廣西提督,留蘇督軍,尋令折回杭州。庚申,和春等奏陳玉成率眾突犯大營,城賊出而合犯,官軍力不能支,退守鎮江。壬戌,以王夢齡為漕運總督。

夏四月丙寅,以明儒曹端從祀文廟。癸酉,賊陷丹陽,張國樑死之,和春走常州。戊寅,詔直省舉辦團練。命都興阿督辦江北軍務。癸未,詔兩江總督何桂清屢失城池,褫職逮問。以曾國籓署兩江總督。擢兵部郎中左宗棠四品京堂,襄辦曾國籓軍務。乙酉,賊犯常州,和春迎戰受傷,卒。以魁玉署江寧將軍,會巴棟阿扼守鎮江。辛卯,賊陷建平,張玉良兵潰於無錫。壬辰,賜鍾駿聲等一百八十三人進士及第出身有差。癸巳,賊陷蘇州,巡撫徐有壬死之。

五月甲午朔,以薛煥為江蘇巡撫,暫署總督。己亥,江蘇常熟縣知縣周沐潤招募沙勇,克復江陰。辛丑,賊陷浙江長興,圍湖州,蕭翰慶赴援失利,死之。甲辰,曾國籓奏陳三路進兵,規蘇保浙,並調沈葆楨差遣。上嘉允之。以東純兼署四川總督。丙午,賊陷吳江、昆山及浙之嘉興。玉明奏金州、岫巖海口有洋船六十餘停泊,劫掠牲畜。庚戌,敕王夢齡督同喬松年開辦江北糧臺。辛亥,恤殉難在籍侍郎戴熙,贈尚書,予世職,建專祠,謚文節。甲寅,命毛昶熙辦河南團練,杜辦山東團練。戊午,李若珠奏薛成良投誠復叛,捕誅之。己未,曾國籓奏隨調鮑超、硃品隆進駐祁門,鄂軍不宜再調。從之。玉明奏洋船到金州海面一百餘艘,文煜奏英、法兵到煙臺者約有萬人,探聞有由海豐大山北犯之說,均下僧格林沁知之。

六月癸亥朔,敕準巴爾虎旗人一體考試。甲子,英船駛入北塘。丙寅,賊陷青浦、松江。己巳,劉長佑奏復慶遠,石達開南竄。庚午,瑞昌奏復廣德。辛未,萬壽節,御殿受賀。壬申,大學士彭蘊章罷直軍機。命邵燦、劉繹、晏端書、龐鍾璐各在原籍舉辦團練。戊寅,王有齡奏在籍道員趙景賢克復湖州。薛煥奏克復松江。庚辰,英、法兵登岸,遂踞北塘。裁南河河道總督暨淮海道各官。壬午,僧格林沁奏英、法勢大志驕,難望議和。得旨,以撫事責之恆福,以顧大局。丙戌,命曾國籓為欽差大臣,實授兩江總督。己丑,夷人犯新河,官軍退守塘沽。命駱秉章馳赴四川督辦軍務。辛卯,手詔僧格林沁曰:「握手言別,倏逾半載。大沽兩岸危急,諒汝憂心如焚。惟天下大本在京師不在海口。若有挫失,總須退保津、通,萬不可寄身命於砲臺,為一身之計。握管淒愴,汝其勉遵!」敕西凌阿固守天津,瑞麟、伊勒東阿赴通州防堵。

秋七月癸巳,命巴棟阿援金壇。戊戌,大沽砲臺失守,提督樂善死之,優血⼙建祠。庚子,僧格林沁退守通州。辛丑,英人陷天津。浙江賊陷臨安、餘杭。四川賊陷邛、蒲、新津。甲辰,江蘇賊復陷松江。丁未,以崇實署四川總督。己酉,裕瑞奏浩罕請依前通商,許之。以常清為伊犁將軍。辛酉,金壇陷,知縣李淮守三年,援兵不至,力竭死之,紳民從死者逾千人。命勝保督馬隊守通州。

八月癸亥,洋兵至通州,載垣誘擒英使巴夏禮解京。戊辰,瑞麟等與洋兵戰於八里橋,不利。命恭親王奕為欽差大臣,辦理撫局。己巳,上幸木蘭,自圓明園啟鑾。丁丑,上駐蹕避暑山莊。李世忠以擒叛將薛成良擢授江南提督。戊寅,詔曰:「江南提督張國樑謀勇兼優,忠義奮發。在軍十年,戰功卓著,東南半壁,倚為長城。本年大營潰散,回援擊賊,受傷沒水。先後奏報,朕猶冀其不確。迄今數月,其為效死捐軀無疑。若使張國樑尚在,蘇、常一帶,何至糜爛若此。追念藎勞,益深愴惻。贈太子太保,入祀昭忠祠,分建專祠。子孫幾人,送部錄用。」己卯,命都興阿帶兵入衛,從官文請也。命玉明、成凱、樂斌、文煜、英桂督兵入衛。辛巳,命恆福駐古北口備防,吳廷棟接轉文報。壬午,浙江官軍克復平湖、嘉善。廣東官軍克復樂昌、仁化癸。未,江蘇賊陷常熟。圓明園災,常嬪薨,內務府大臣、尚書文豐死之。庚寅,恭親王奏請還巴夏禮於英軍。薛煥奏劾馮子材赴援金壇,擁兵不進,致令城陷。詔薄譴之。

九月壬辰,命勝保為欽差大臣,總統援軍。敕恭親王奕照會英人,勿修城北砲臺,速行議約。甲午,英使、法使入城。大學士彭蘊章、尚書許乃普以病乞免,許之。己亥,命慶廉、英桂兵駐直備調。辛丑,賊陷寧國,周天受死之。甲辰,命左宗棠督辦浙江軍務。乙巳,撫局成。恭親王奕奏請宣示中外,如約遵行。許俄人駐烏蘇里、綏芬。停各省援兵。敕英桂來京。議西巡。戊申,李若珠奏克復江陰。辛亥,賊陷徽州,守城道員李元度棄城走。癸丑,直隸、山東、河南賊匪並起,命僧格林沁討之。庚申,恭親王奕奏洋人退至天津,籲請回鑾。

冬十月辛酉朔,詔天氣漸寒,暫緩回鑾。以田興恕為貴州提督。予陣亡提督周天受、周天培世職,建祠予謚,附祀道員福咸等。壬戌,以劉源灝為雲貴總督,鄧爾恆為貴州巡撫。甲子,敕文謙、恆祺辦理通商事宜,吳廷棟督辦防務。以文安為湖南提督。以馮子材督辦鎮江軍務。丙寅,恭親王奕奏換俄人和約,請用御寶,從之。辛未,俄羅斯致槍砲。癸酉,敕樂斌、英桂回任。庚辰,以嚴樹森為河南巡撫,毛昶熙督辦河南捻匪。辛巳,命都興阿督辦江北軍務,李若珠副之。以總兵田在田接辦徐、宿剿匪,淮徐道吳棠幫同辦理。

十一月辛卯,勝保奏以大順廣道聯英專辦河防,準其奏報,從之。癸巳,翁同書奏陳謹天戒,固邦本,收人才,練京營,爭形勢。得旨:「收人才一條,利少弊多。餘留覽。」甲午,浙江賊陷新城、臨安、富陽。乙未,王夢齡奏剿賊獲勝,三河肅清,並請節制黃開榜水師,從之。庚子,曾國籓奏鮑超等克復黟縣。辛丑,李若珠乞養親,以曾秉忠代之。癸卯,以杭州解嚴,優賚瑞昌、王有齡等。瑞昌奏陳慶端力保浙疆,請加優獎。得旨:「不分畛域,皆爾大吏分內之事。甄敘督撫,出自朝旨,非汝所得擅請。」戊申,命成琦會景惇查勘俄羅斯東界。癸卯,浙軍張玉良攻復嚴州。甲寅,官文、胡林翼奏陳玉成圖犯懷、桐,多隆阿會合李續宜迎剿,大敗之,殺賊萬餘。多隆阿賜黃馬褂,李續宜加二品銜。

十二月辛酉,命西凌阿、國瑞幫辦僧格林沁軍務。丙寅,命張亮基留辦雲南軍務。己巳,始置總理各國通商事務衙門,命恭親王奕、桂良、文祥管理。以崇厚充三口通商大臣,薛煥兼辦上海等處通商事務。準旗人學習外國語言文字。己巳,以田興恕為欽差大臣,督辦貴州軍務。丙子,左宗棠奏督軍克復江西德興、安徽婺源,予三品京卿。乙酉,以官文、周祖培為大學士,肅順協辦大學士,沈兆霖為戶部尚書,硃鳳標為兵部尚書。戊子,祫祭太廟。

是歲,免江蘇、浙江、安徽三省額賦逋賦,又直隸、山東、河南、江西、湖北、湖南、福建、廣西等省四百一州縣衛被災、被賊額賦有差。會計天下民數二萬六千零九十二萬四千六百七十五名口,存倉穀數五百二十三萬一千九百二十石四斗六升五合一勺。朝鮮入貢。

十一年辛酉,上在木蘭。春正月庚寅朔,上御綏成殿受賀。辛酉,詔二月十三日回鑾。乙未,曾國籓奏楊載福剿賊,克都昌,解南陵圍。田在田奏捻匪犯碭山,擊走之,加提督銜。丙申,召翁同書來京,以李續宜為安徽巡撫。丁酉,以福濟為成都將軍。辛丑,賊陷孝豐,杭州戒嚴。壬寅,詔:「紀年開秩,應予特赦,非常赦所不原者咸減除之。」癸卯,左宗棠兵復饒州暨都梁。乙巳,恆福以病免,以文煜為直隸總督,譚廷襄為山東巡撫,鄧爾恆為陜西巡撫,何冠英署貴州巡撫。丁未,僧格林沁奏捻匪竄入山東,派隊追剿,及於菏澤,失利。得旨:「僧格林沁督帶重兵,北地倚為屏障。乃以饑疲之卒,追方張之寇,旁無援應,宜其敗也。勇往有餘,未能持重。尚其汰兵選將,扼要嚴防,謀定後動,勿再輕進。」戊申,詔袁甲三等:「捻匪裹脅良民,未便概行誅戮,可剴切曉諭,設法解散。投誠者免罪,殺賊者敘功。並傳知李世忠一體招撫。」辛亥,貴州官軍克復獨山。壬子,翁同書奏陳撫練苗沛霖劫擾壽州,跋扈異常。詔李續宜酌辦。河南捻匪竄擾東明、長垣。

二月己未朔,雲南官軍克復晉寧。壬戌,復置奉天金州水師營。丙寅,詔準山東抵還法國教堂地基,並敕直省遇有交涉,即行酌辦請旨,勿許推諉。丁卯,張玉良軍克復江山、常山。庚午,曾國籓奏左宗棠敗賊於景德鎮,鮑超敗賊於石門洋塘。壬申,浙軍克復富陽。熱河朝陽盜匪陷城,命克興阿剿之。命明誼、明緒會勘俄界,英蘊、奎英辦理俄人通商。捻匪撲汶河,副都統伊興阿、總兵滕家勝逆戰陣歿。乙亥,陳玉成糾合捻匪由英山犯湖北蘄水,詔胡林翼回兵擊之。庚辰,詔曰:「前經降旨,訂日回鑾。旬日以來,體氣未復。綏俟秋間再降諭旨。」壬午,朝鮮國王遣使朝覲行在。溫諭止之,頒賜文綺、珍器,及其使臣。癸未,詔挑選兵丁演習俄國送到槍砲。甲申,裁撤黑龍江團丁歸農。

三月己丑朔,詔派辦約大臣崇綸、崇厚給與全權便宜行事。敕侍郎成琦赴興凱湖會勘俄人分界事宜。予道員聯捷四品京卿,辦理河防。壬辰,恭親王奕請赴行在祗叩起居。上手詔答之曰:「別經半載,時思握手而談。惟近日欬嗽不止,時有紅痰,尚須靜攝,未宜多言。且俟秋間再為面話。」丙申,詔皇長子於四月七日入學,以李鴻藻充師傅。戊戌,都興阿奏鎮、揚水師船隻年久損壞,請飭廣東購運紅單船應用,從之。庚子,命勝保督辦直隸、山東剿匪。以賈臻署安徽巡撫。庚戌,英、法兩國兵退出廣東省城。辛亥,以前大學士彭蘊章署兵部尚書。甲寅,浙江賊陷海鹽、平湖、乍浦,副都統錫齡阿死之。丙辰,廣西土匪陷太平府、養利州。

夏四月己未朔,嚴樹森奏賊犯汝寧,道員張曜擊走之。戊辰,山東捻匪、教匪連陷館陶七縣。僧格林沁入滕縣固守,詔勝保分兵援之。甲戌,詔曰:「朕聞各處辦捐,有指捐、借捐、砲船捐、畝捐、米捐、饟捐、堤工捐、船捐、房捐、鹽捐、板捐、活捐,名目滋多,員司猥雜。其實取民者多,歸公者寡。近年軍錡浩繁,不得已而借資民力商力。然必涓滴歸公,撙節動用,始得實濟。若似此徵求無藝,朘薄民生,尚復成何政體。各大臣、督撫,尚其嚴密稽查,剔除奸蠹,以副朕意。」乙亥,左宗棠敗賊於樂平。庚辰,山東教匪撲圍大名,聯捷擊走之。癸未,皖賊復竄浙江,陷常山、江山,進偪衢州。

五月癸巳,田在田奏苗練犯符離,敕僧格林沁分兵援之。甲午,鄧爾恆被戕於曲靖。飭劉源灝查辦。以瑛棨為陜西巡撫。庚子,勝保奏克復館陶。辛丑,命賈臻、李世忠幫辦袁甲三軍務。甲辰,命多隆阿幫辦官文、胡林翼軍務。乙巳,賊陷浙江壽昌、金華、龍游、湯谿、長興,進陷蘭谿、武義。詔催左宗棠赴援。

六月戊午朔,日有食之。庚申,曾國籓、胡林翼奏:「安慶省城自我軍長圍,逆酋陳玉成率黨回援安慶,於集賢關外赤岡嶺堅築四壘。經鮑超、成大吉會合多隆阿馬隊奮力進剿,晝夜轟擊。五月初一日,三壘俱降。釋去脅從,將長發老賊概行正法。其踞第一壘之賊劉滄琳,乘夜潛遁。經鮑超殲於馬踏石,餘為水師斬戮殆盡,並將劉滄琳驗明支解梟示。」得旨嘉獎。布魯斯亞國換約通商。辛酉,許俄人在庫倫、恰克圖通商。乙丑,欽天監奏八月初一日,日月合璧,五星聯珠。得旨,不必宣付史館。甲戌,賊陷浙江遂昌、松陽、永康。丙子,回匪撲擾喀什噶爾。詔景廉赴阿克蘇防剿。丙戌,浙江官軍克復長興。

秋七月丁亥,詔每年秋間王公致祭兩陵,如遇山水漲發,可在途守候道路通時,即行前往。屆期不到,由守護之貝勒、公等行禮。甲午,曾國籓奏收復安徽徽州。戊戌,予四川陣亡侍衛昭勇侯楊炘建祠。

辛丑,上不豫。壬寅,上大漸,召王大臣承寫硃諭,立皇長子為皇太子。癸卯,上崩於行宮,年三十一。十月,奉移梓宮至京。十二月,恭上尊謚。同治四年九月,葬定陵。

論曰:文宗遭陽九之運,躬明夷之會。外強要盟,內孽競作,奄忽一紀,遂無一日之安。而能任賢擢材,洞觀肆應。賦民首杜煩苛,治軍慎持馭索。輔弼充位,悉出廟算。鄉使假年禦宇,安有後來之伏患哉?


\end{pinyinscope}