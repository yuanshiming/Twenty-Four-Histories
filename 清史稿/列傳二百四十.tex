\article{列傳二百四十}

\begin{pinyinscope}
榮全喜昌升泰善慶柏梁恩澤銘安恭鏜

慶裕長庚文海鳳全增祺貽穀信勤

榮全,關佳氏,滿洲正黃旗人,一等威勇侯那銘嗣子。咸豐元年,襲爵,授二等侍衛。從征山東,以功遷頭等,還直乾清門。十一年,出為塔爾巴哈臺領隊大臣,歷喀喇沙爾辦事大臣、伊犁參贊大臣。同治五年,以鑲紅旗蒙古副都統署伊犁將軍。明年,調烏里雅蘇臺參贊大臣。時纏回襲陷伊犁,俄乘機遣兵入,藉口代為收復。榮全內籌守御,外示羈縻。又以索倫、蒙古被兵,民多亡入俄境,為請擇地安插,分部護之。

八年,朝旨以新疆各城多與俄接壤,命榮全會俄官,依三年勘辦西北界約記,建設烏屬界牌鄂博。先是,塔城和約兩國分界,自恰克圖西北逾烏梁海,首沙濱達巴哈,訖浩罕邊界,繪畫地圖,識以紅線。至是,集議烏克克卡倫仍依舊界,惟自東北沙濱達巴哈至西南賽留格木山柏郭蘇克壩補牌博八,明定界限,所謂烏里雅蘇臺界約是也。九年,坐烏魯木齊城陷,褫職留任。十年,俄遣柯福滿將軍占領庫爾札,聲收烏魯木齊,詔榮全赴伊犁收回城池。榮全遂自烏城西進至霍博克賽里,直抵塔爾巴哈臺。會天大雪,止舍。逾歲,與俄官布呼策勒傅斯奇集議色爾賀鄂魯勒,榮全向之索還。俄官陽言請命本國,而陰遣兵襲取瑪納斯,駸駸欲東犯。榮全不獲已,返塔城。是時,俄人據伊犁可千餘人,滋驕橫,索倫、錫伯苦之。十二年,錫伯窘益甚,榮全濟以銀,俄官反出阻之。榮全曰:「為我屬地,我自濟之。與俄奚涉焉?」牒駁之,俄官詞屈。上聞而嘉之。

會回構安集延擾動,上命榮全進攻瑪納斯綴寇勢,遂復其官。十三年,白彥虎犯上馬橋,榮全遣軍敗之沙子山。光緒二年,師克瑪納斯南、北二城。榮全數有功,尋召入京,歷兼護軍統領、右翼前鋒統領。五年,卒,恤如制。

喜昌,字桂亭,葛濟勒氏,滿洲鑲白旗人,世居吉林。亦以防俄著。初從軍征捻,累功至協領。河內之役,以少勝眾,功尤盛,晉副都統。西捻平,賜頭品秩,充西寧辦事大臣,光緒六年,調烏里雅蘇臺參贊大臣。時中俄有違言,俄軍窺吉林邊壤。朝命喜昌佐防務,因上言琿春為兵沖要地,宜練馬隊二千、步隊八千資守御。逾歲,抵琿春,相度地勢,乃專囑伊克唐阿防守事,而自率所部頓磨夷石,扼雙城、紅土巖來路,上韙之。和議成,授庫倫辦事大臣,條上邊防六事,尋謝病歸。十七年,卒,予易州建祠。

升泰,字竹珊,卓特氏,蒙古正黃旗人。入貲為員外郎,銓戶部。出知山西汾州府,有政聲。回寇擾境,錄守城功,晉道員,除河東道。歷浙江按察使、雲南布政使。光緒七年,賞副都統銜,充伊犁參贊大臣,尋授內閣學士。明年,署烏魯木齊都統,與俄羅斯定阿爾泰山邊界。俄人遇事齟,升泰執原議不稍讓。始受約束。

十三年,改充駐藏幫辦大臣。時藏人築卡隆阿,為印度所敗。上命辦事大臣文碩令藏人撤卡。文碩謂為藏地,無可撤,嚴旨責焉,以升泰代之。而藏人誓復仇,頓兵帕克里,將痛擊印軍。升泰搜集乾隆五十三年舊檔,哲孟雄受偪廓爾喀,達賴以日納宗給之,以雅拉、木支兩山為界,持示藏人。藏人曰:「地雖予哲,今哲通英,宜收回。」升泰數止之,不從。英使願媾和,朝旨令升泰赴邊界與印官議約。十四年,印軍收哲全境。藏兵又敗咱利,亞東、朗熱並失。隙愈深,群思報復,升泰數嚴止之,仍不從。會天寒,印官趣升泰赴議,而藏人請代索哲孟雄、布魯克巴侵地,否則傾眾一戰。升泰仍百計諭藏僧,戒藏番毋妄動。及至邊,布部長遣兵千七百人護衛。升泰慮為英口實,謝去,並乞印綬封典,升泰允代請諸朝。既與英政務司保爾會於納蕩,索藏償兵費。升泰曰:「哲為藏屬,索費無名。」英人又在布境及後藏乾壩修路,藏人益大震。英官要求甚奢,升泰力折之,藏人漸就範。升泰數要英撤兵,英不可。升泰以大雪封山,運糧無所,退駐仁進岡。英人既掠哲地,復羈其部長土尕朗思,置之噶倫繃,招印度、廓爾喀游民墾荒。廷議以哲事無從挽救,慮梗藏議,諭升泰勿問。

藏、哲舊界,本在雅拉、支木。後商人往來咱利,為新闢捷徑。升泰議以咱利山分藏、哲界,以符前案。其印、哲界在日喜曲河,擬約中註明。哲部長母挈兩孫赴升泰營泣訴,丐中朝作主,升泰無如何。英人又欲易置其部長,升泰力阻之。土尕朗思謂願棄此居春丕,升泰弗許,慮英責言也。

十五年春,藏兵撤退。升泰請總署達英使,電印軍速撤。逮既撤,而英人猶久不訂約。升泰上疏略謂:「聞藏人言:『與有仇之英議和,不若與無仇之俄通好。』設藏番果與通款,英、俄必互相猜忌,後患方長。乞告英使,電趣印督速定藏約。」又言:「與英初次會議,英人欲至藏貿易。告以番情疑詐,始許退至江孜。力言再四,又許退至帕隘。臣力諭藏番,通商萬不能免,始出結遵辦。今英慮他國援以為請,忽議中止。在藏人固所深願,在俄人亦不能有所干求。惟日後防範宜嚴,未可再涉疏懈。入夏至今,曠日持久,請敕總署牒英使速議結。」

十六年,以升泰為全權大臣,與印督定約八款,自布坦交界支莫摯山起,至廓爾喀邊界止,分藏、哲界,哲境歸英保護,所謂藏印條款是也,語詳邦交志。十八年,卒於仁進岡。事聞,優詔賜恤。

善慶,張佳氏,滿洲正黃旗人,黑龍江駐防。初從勝保征捻,積勛至協領,賜號濟特固勒忒依巴圖魯。克鳳陽,擢副都統。論復定遠功,晉頭品服。同治元年,追捻至靈壁,平宿州寇墟。創發,乞病去。逾歲,朝旨以捻事棘,命選吉林、黑龍江騎旅赴皖。軍抵河南,張之萬疏留,連敗粵寇於南陽及湖北陽邳灘鮮花鎮。坐所部兵馬疲瘠褫職,仍留軍。四年,授吉林雙城堡總管。以戰功復故官,即於軍前除杭州副都統。再坐營馬侵踏民田褫職,追擊竄賊大同集,被宥。

六年,與劉銘傳剿東捻,敗之濰縣松樹山。捻奔贛榆,追及之。銘傳自當賴文光,而令善慶當任柱。任柱殊死鬥,善慶令騎旅下馬結陣疾擊之,尸山積,猶進不止。會大霧,窈冥不見人。銘傳分軍襲其後,善慶率隊大呼沖殺,槍砲雨坌,降人潘貴升斬任酋於陣。善慶乘勢追擊,斬馘千餘級。論功,賞黃馬褂。賴酋勢益蹙,阻瀰河弗能達,乃據壽光王胡城。銘傳等分左右進,善慶與溫德勒克西拒之。追至鳳凰臺,為他將所敗,就縛於揚州,予騎都尉世職。七年,西捻平,張總愚自沉於河,餘匪為善慶等所殲,晉二等輕車都尉,赴本官。

擢杭州將軍。杭州駐防自克復後,昆壽規復營制,連成重建營墻。善慶至,籌設漸備。光緒改元,調綏遠城,歷寧夏、江寧。召還,授正紅旗漢軍副都統,駐師通州。十一年,充御前侍衛,佐海軍事務。十三年,出為福州將軍。次年,卒,予建祠,謚勤敏。

柏梁,字研香,瓜爾佳氏,滿洲正白旗人,杭州駐防。父麟瑞,咸豐末陣亡乍浦,見忠義傳。柏梁少從其叔父鳳瑞出,隸李鴻章軍,轉戰江、浙。攻太倉州,柏梁自南門先登。復攻蘇州,戰於黃天蕩,陣斬悍目。攻嘉興、宜興、江陰、金壇,柏梁皆有功。改隸勝保軍,戰江北,屢捷,累保至協領,賞花翎。杭州克復,調歸駐防,補協領。承歷任將軍辦理營務,善慶尤倚任之。光緒中,駐防初設洋槍隊,以柏梁充全營翼長,兼掌兵司。規畫營制,均照新軍式訓練,紀律肅然。敘勞,以副都統記名。入覲,奏對稱旨。以曉暢戎機、訓練出力,賞頭品服。駐防舊有旗倉,久為兵燹,柏梁請撥款重建。旋授乍浦副都統。乍浦駐防營毀於粵亂,副都統駐杭州。柏梁蒞任,歲至乍浦巡視海防。以勞卒。賜恤如制。

恩澤,字雨三,噶奇特氏,蒙古鑲藍旗人,荊州駐防。光緒初,以佐領從金順出關,克黃田,復烏魯木齊諸城,擢協領。其秋,回寇奔呼圖壁,追擊之,大潰,又扼之頭屯河,白彥虎益窘。進攻瑪納斯,轟潰城垣數丈,恩澤先登,諸軍繼之,城拔,晉副都統。歷權巴里坤、烏魯木齊領隊大臣。以劉錦棠薦。除吉林副都統,移琿春。

二十年,日本敗盟,與將軍長順籌戰守。乃治團練,築臺壘,設疑兵,敵知有備,引兵去。尋署將軍。其時東山馬賊猖獗,伯都訥、烏拉教匪乘機竊發,竄擾官街、白旗屯。恩澤聞警,率師分擊之,夷其堅堡。又遣提督云春等,搜東山逸匪。明年,調黑龍江,督邊防。先後疏請改練洋操,招墾荒地,賑恤窮乏。俄而鬍匪據觀音山南北圍,謀劫金廠。恩澤詗知之,嚴備以待。已,寇果至,營官王槐林等迎擊,大敗之。別遣將大搜山林,自是首觀音山訖烏蘇里滿卡,千餘里無寇蹤。又以撓力溝素窟匪,留兵鎮攝之。上以為能,降敕褒嘉。二十五年,卒於官,予黑龍江及立功省分建祠。

銘安,字鼎臣,葉赫那拉氏,內務府滿洲鑲黃旗人。咸豐六年進士,選庶吉士,授編修,除贊善。累遷內閣學士,歷泰陵總兵、倉場侍郎。同治十三年,調盛京刑部。德宗纘業,充頒詔朝鮮正使。光緒二年,勘事吉林,條上四事,曰:剿馬賊、禁賭博、設民官、稽荒地,上韙之,命署將軍。吉省武備久弛,寇盜充斥。銘安蒞任,嚴治盜。復募獵戶為砲勇,號吉勝營。先後檄統領穆隆阿、協領金福,分道追剿,斬馘甚眾。益練西丹步隊八百,入山窮搜,寇勢漸蹙。已,復捕治東山逸匪,擒誅金廠黨魁,軍威大振。默念吉省幅瀼四五千里,斷非十數委員能濟事;且旗員未諳民治,請破積習,調用漢官,部臣尼之,銘安抗疏力爭,始俞允。

五年,實授。又言盜賊雖平,餘孽未靖,亟宜增置民官,畫疆分治。先後奏改伯都訥同知、長春通判,理事,為撫民,置知府、巡道各一,賓州、五常同知二,雙城通判、伊通知州、敦化知縣各一,並請無分滿、漢。又奏弛秧蓡禁,免山獸貢,增各旗義學,士民利賴之。東北與俄接壤,舊設卡倫,無兵駐守。乃遣將分扼要塞,並築營伯力、紅土崖、雙城子,守以重兵,因上安內攘外方略,稱旨。長春號難治,銘安稔知鍾彥才,奏請除通判,部臣以違例請下吏議,銘安盛氣抗辯,上兩解之。然銘安終不自安,引疾去。尋坐失察屬吏受賄,降三級。二十三年,上以治吉有功,部民感念,復故官。明年,鄉舉重逢,加太子太保。宣統三年,卒,年八十四,詔優恤,謚文肅。

恭鏜,字振魁,博爾濟吉特氏,滿洲正黃旗人,大學士琦善子。以任子授吏部主事。累遷郎中,兼內務府銀庫員外郎,充總理各國事務衙門章京,出為湖北荊宜施道。論捕獲江陵教匪功,加按察使銜。同治十年,擢奉天府府尹,坐事降。光緒三年,賞二等侍衛,充烏魯木齊領隊大臣。越二年,遷都統。

先是,陜回阿渾妥明客參將索煥章家。煥章者,前甘州提督索文子也,素蓄異志。戍卒硃小桂告變,提督業普沖惑煥章言,誣斬小桂。及煥章反,烏城陷,業普沖被害。至是恭鏜廉得實,請予平反。奪索文榮典,分別恤小桂、業普沖及赴援殉難諸臣,人心稱快,賜頭品秩。九年,除西安將軍,病免。十二年,署黑龍江將軍。疏請舉辦漠河金礦,杜俄人覬覦。又建議墾荒十利,曰:儲國帑、濟民食、嚴保衛、便輯綏、裕經費、富徵收、集商賈、益釐稅、廣生聚、實邊備,詔不許。十四年,實授。明年,移杭州,入覲,道卒天津,詔優恤。子瑞澂,自有傳。

慶裕,字蘭圃,喜塔臘氏,滿洲正白旗人。以繙譯生員考取內閣中書,充軍機章京,兼總理各國事務衙門行走。從文祥赴奉天剿匪,還補侍讀。出知湖北鄖陽府。追錄平捻功,晉道員。光緒元年,擢奉天府府尹。歷遷至漕運總督,調河東河道。九年,除盛京將軍。明年,法越構釁,慶裕巡視沒溝營、旅順口、大連灣,諭示居民曰:「有能殺敵立功,擒獲奸細者賞。」又遵旨增練蘇拉千人、食餉旗兵五百,上言:「整頓旗營,兼顧海防。今日多一兵,即有一兵之用;異日補旗兵,即可裁客兵之餉。所費者少,所系者重。」詔嘉許之。朝鮮亂作,檄提督黃仕林等扼隘口。以營口為兵沖要地,運石塞海口,設電線達省城。建議籌邊籌餉機宜,附陳宜變通者三事:一,道府年終加考;一,推廣薦舉卓異;一,崇府尹品秩,行巡撫事,議行。

十一年,安東十二州縣告災,慶裕籌賑撫恤,民獲甦。是秋霪雨,遼河、大凌河暴漲,田禾被淹。發倉以濟。設粥廠牛莊、田莊臺收養之。明年,金州蝗,旱魃為虐。又明年,興京水祲,賑如初。十九年,授熱河都統。道孫河、半壁店,上流民被災就食狀,並請變通盜案、稅額章程。又使吏捕平泉黑役為害鄉里者,頗著政聲。二十年,調福州將軍。閩海關沿襲舊規,吏胥因緣為奸,上敕其整理。既至,鉤稽糾剔,蠲苛息煩,弊風盡革。其秋,卒於官,恤如制。

長庚,字少白,伊爾根覺羅氏,滿洲正黃旗人。以縣丞保知縣。伊犁將軍榮全調充翼長。時白彥虎糾西寧回匪寇烏垣,進圍哈密。安集延酋帕夏並偽元帥馬明眾,合烏魯木齊、古牧地、昌吉、瑪納斯、呼圖壁漢回,撲犯沙山子,與為遙應,勢張甚。長庚奉榮全檄,領練勇赴援。而烏魯木齊都統景廉所遣黑龍江營總伊勒和布兵亦至。兩軍夾擊,殲擒殆盡,卒解沙山子圍。旋贊都統金順戎幕,總理營務,積勛至道員。光緒六年,授巴彥岱領隊大臣。未幾,丁母憂。服闋,入覲,上召見,垂詢西北情形。長庚手繪輿圖,奏陳邊事,以阿爾泰山宜設防守,伊犁邊防宜籌布置,纏金等境宜開屯山,漠北草地宜善撫綏,及哈薩克應仿例編為佐領等條以對。遷伊犁副都統。

十四年,命充駐藏大臣。行次里塘,值瞻對番族叛。長庚暫往碩般多,廉知釁由番官肆虐釀成,遴員授以機宜,調集漢、土官兵,聲罪致討,殲渠宥脅,嚴懲藏官,事乃就緒。議者遂欲收其地,仍歸川轄。長庚以瞻對自乾隆以來,叛服靡常,勞師糜餉。同治初年,西藏底定,奉旨將瞻對劃歸達賴喇嘛,派堪布管理。今若蹊田奪牛,使朝廷失信於衛藏,恐所得小而所失大。乃為詳定善後章程,與將軍岐元、川督劉秉璋等同上。藏亂遂定。

擢伊犁將軍。時伊犁當大亂後,萬端待理。長庚至,多所規畫。蔥嶺西有帕米爾者,即唐之波謎羅也,東距疏勒約一千四百里。乾隆二十四年,將軍富德窮追回酋,一至其地,立碑記焉,然稱之為葉什勒庫爾,未明言帕米爾三字。嘉、道以來,久未顧問,碑亦湮沒。咸、同後,俄人遽以哈薩克右中各部與浩罕八部,設土耳其斯坦、斜米七河、費爾干等省,甚至塔城西之舊雅爾城、阿克蘇之察林河卡倫,同就淪胥。蔥嶺東有坎巨提者,一名乾竺特,其都城曰棍雜,與哪咯耳隔水相望,在莎車州西南約二千里。其西北可通帕米爾。坎民貧而多盜,其酋縱掠鄰郡。英人責言,牒告我政府。坎酋又交通俄人。英使臣以割分帕地請,政府恐啟俄爭,拒弗許。時英、俄各以兵壓境。長庚致書新疆巡撫陶模,謂:「屬地當爭,邊地當守,兵釁萬不可開。況能戡土匪之將士,未足以御強敵;軍中所資。仰給內地及濱江海各省,數月乃達。而俄境鐵軌已至薩瑪爾干,英屬鐵軌已至北印度之勞爾,遲速迥殊。又新疆南北路與俄地犬牙相錯者幾五千餘里,雖兵倍加,不敷防守。且俄若以輕兵由齊桑斯克走布倫托海犯鎮西、哈密,即可梗我咽喉。當此民窮財匱之時,尤不可輕戰。只能備豫不虞,徐圖轉圜。毋以小忿遂起大釁,增兵徒增民困。」陶模以為然,卒如長庚議。

又伊、塔之間,有巴爾魯克山者,西連俄界,南逼精河,西南與博羅塔拉接壤,為伊、塔要道,泉甘土沃,久為俄人垂涎。自借與俄後,俄人視為己有。先是,北路劫盜多窟此山,擾行旅。前副都統額爾慶額請租借期滿索回。總署以俄使有續借之請,函詢情形。長庚詳陳利弊,謂此山關系重大,急應收回。隨遣員赴塔城與俄領事會商,堅持人隨地歸之約,卒收回。二十年,甘回作亂,官軍兜剿。賊不能得志於甘,欲循白彥虎故事,西竄新疆,由伊犁遁俄境。長庚諜知,遣兵扼守珠勒都斯等地,賊不能越,遂就擒於羅布淖爾。二十二年,命兼鑲藍旗漢軍都統。二十六年,拳匪肇亂,俄人調兵入伊。長庚與俄領事交涉,凡教堂及俄人財產,力任保護,諭令退兵,人心乃定。調成都將軍,未之任,奉電旨飭赴阿爾泰山查勘界址。旋內召,授兵部尚書。

三十一年,復授伊犁將軍。疏陳伊犁應辦事宜,並言籌餉練兵,必合新疆全省籌畫。將軍事權不屬,莫若裁去新疆巡撫、伊犁將軍,增設總督兼管巡撫事宜,庶呼應靈而事權一。又籌擬北方興屯、置省事宜,請築西安至蘭州、歸化至包頭、包頭至古城各鐵路,皆不果行。

宣統元年,遷陜甘總督。三年,武昌事起,西安等處繼之。前陜甘總督升允奉命督辦軍務,事略定。遜位旨下,長庚乃將總督印交布政使趙惟熙而去。越四年卒,謚恭厚。

文海,字仲瀛,費莫氏,滿洲鑲紅旗人。以繙譯舉人考取內閣中書,充軍機章京,遷侍讀。光緒九年,轉御史。建言培養人才,宜令中外大臣杜徇情,勵廉恥,以植其本,上嘉納焉。十二年,巡視北城。以兄文治授詹事,依例回避,調戶部郎中。十四年,出知貴州安順府,調貴陽。所蒞有聲。

二十二年,數遷至按察使,尋加副都統,充駐藏辦事大臣。既至,即上言叛番雖靖,餘孽猶存,兵未可罷,願自任剿辦。二十五年,呼圖克圖第穆構康巴喇嘛用邪術咒詛達賴。文海曰:「此關風化,不可不有以懲之也。」乃奏請奪其名號。已而野番出掠博窩,地為川、藏孔道,行旅苦之。官軍入昂多往捕,彼即扼縮隆岡來路,崛強莫能制。文海率眾進擊,別遣通番語者繞道叩其壁,宣播朝威,反覆開喻。於是上博窩業魯第巴宿木宗,中博窩雨茹寺,下博窩蒲隆、瓊多諸寺,皆相率乞款附,數月而事定,賜頭品服。未幾遘疾,請入川療治,卒於塗。依尚書例賜恤,予入城治喪。

鳳全,字茀堂,滿洲鑲黃旗人,荊州駐防。以舉人入貲為知縣,銓四川。光緒二年,權知開縣,至則使吏捕仇開正。開正故無賴,痛以重法繩之,卒改為善。李氏為邑豪族,其族人倚勢,所為多不法。鳳全直法行治,雖豪必夷,以故人人惴恐。歷成都、綿竹,補蒲江,署崇慶州,一如治開。舉治行第一,擢工⼙州直隸州。二十三年,調資州。大足縣餘蠻子亂起,其黨唐翠屏等構眾入境。鳳全乃治城防,設間諜,練鄉勇,聯客軍,謀定寇至,亟遣軍間道襲擊。戰太平場,捕斬略盡。復越境搜治餘黨,不兩月而事寧。州屬患水祲,民多失業,設法賑濟之,全活甚眾。再以治行聞,調署瀘州。二十八年,權知嘉定府。緣江會匪嘯聚,既蒞事,舉團練,嚴治通匪土豪,居民莫敢玩法。無何,拳匪延入蜀,嘉定當水陸沖,郡中一夕數驚。鳳全內固人心,外嚴拒守。嘗提一旅師四出游弋,匪不敢近。故鄰境多破碎,惟嘉郡差全,各國僑民多樂就之,繇是名大著。岑春煊性嚴厲,喜彈劾,屬吏鮮當意,獨亟賞鳳全,一再論薦。遷成綿龍茂道,特加副都統。

三十年,充駐藏幫辦大臣。行抵巴塘,見土司侵細民,喇嘛尤橫恣,久蔑視大臣。鳳全以為縱之則滋驕,後且嬰患,因是有暫停剃度、限定人數之議。喇嘛銜之深,遂潛通土司,嗾番匪播流言,阻墾務,漸至戕營勇,燔教堂,勢洶洶。鳳全率衛兵五百人往,至紅亭子,伏突起,戰良久,被害。事聞,予建祠,謚威愍。繼室李佳氏留成都,聞變,率子忠順馳入打箭爐辨遺骸,隨喪歸省垣。祠既成,乃觴將軍、總督以下官及文武士紳,告靈安主,慨然曰:「吾可以見先夫於地下矣!」事畢,夜赴荷池死,獲附祀。

鳳全清操峻特,號剛直,然性忭急,少權變,不能與番眾委蛇,故終及難云。

增祺,字瑞堂,伊拉里氏,滿洲鑲白旗人,密雲駐防。以佐領調黑龍江,佐練兵事,歷至齊齊哈爾副都統。光緒二十年,署將軍。二十四年,擢福州將軍,充船政大臣,兼署閩浙總督,旋遷盛京將軍。奉天自中日戰後,副都統榮和、壽長編練仁字、育字兩軍,營務廢弛,增祺奏請派員查辦,上命李秉衡往查,奪二人職,交部治罪,軍制肅然。

二十六年,拳匪亂作,副都統晉昌率眾附和,增祺不能阻,遂啟戰釁。奉省自日還遼南,旅順、大連既轉歸俄租,復築鐵道,沿路皆駐俄兵。戰累挫,蓋平、熊岳先後失守。增祺先以敵強兵脆,大局不支,連電上達,並照會旅順俄水師提督、營口俄領事,磋商停戰,不果。俄兵遂抵省城,諸軍皆潰。增祺奏請恭奉盛京大內尊藏聖容、太廟冊寶出城。俄兵至,招增祺還,商議善後。增祺往旅順,與俄議訂奉天交地暫約九條,以荒謬交嚴議,詔革職,尋仍留任。諭楊儒向俄外部商改,以吏治兵權不失自主為要。二十八年,交收東三省條約始成。俄兵駐奉數年,遇事強橫,無復公理,增祺隱忍周旋,憂勞備至,至是駐兵始退。

未幾,復有俄日之戰,朝旨守中立。增祺嚴飭文武官吏堅明約束,並告兩國主兵者勿得犯中立。日兵迫省亟,勸俄兵先退,日兵官始入城,省城幸免戰禍。

三十年冬,諭增祺賑撫東三省難民,並發內帑三十萬賑之。明年,懿旨復發內帑三十萬賑恤。增祺招集流亡,商民復業。頗留意吏治,先後增設洮南、海龍、遼源、開通、靖安、西安、西豐等府縣。凡牧廠、圍場及蒙荒,逐漸放墾。奉省財政素絀,徵榷一切,向無定章,咸豐後始辦貨釐,光緒初始辦鹽釐。增祺銳意清理,籌辦糧、酒、煙、藥各稅,明年規章,變通鹽法,就廠徵稅,歲入漸增。尤嚴治盜,以增官設治為弭盜清源之本。三十一年,以憂免。三十三年,授寧夏將軍,改正白旗蒙古都統。宣統元年,遷廣州將軍,兼署兩廣總督。三年,調京,仍授都統,兼弼德院顧問大臣,旋去職。越八年,卒,謚簡愨。

貽穀,字藹人,烏雅氏,滿洲鑲黃旗人。光緒元年舉人。以主事分兵部,晉員外郎。十八年,成進士,選庶吉士,授編修,累遷內閣學士。兩宮西幸,貽穀聞警,步行追及宣化,流涕入對,隨扈西安。授兵部左侍郎,屢召詢時政,直言無隱,上皆嘉納。明年,扈駕還京。兵部公署已毀,假柏林寺為廨舍。貽穀昕夕蒞事,如在行在時。

是冬,山西巡撫岑春煊奏晉邊察哈爾左右翼及西北烏蘭察布、伊克昭兩盟荒地甚多,請及時開墾,派大員督辦。詔以貽穀為督辦蒙旗墾務大臣。貽穀有經濟才,艱貞自勵。既奉命,銳以籌邊殖民為己任。其督墾地界,綿延直、晉、秦、隴、長城、河套,凡數千里。統籌全局,擬陳開墾大綱,規畫至詳。疏入報可,並加理籓部尚書銜,節制秦、晉、隴沿邊各州縣。旋復授綏遠城將軍,事權始一。

貽穀首重官墾。立墾務局,設東路公司,官商合辦。初辦察哈爾右翼,改舊設押荒局為豐寧墾務局,旋分為豐鎮、寧遠兩局。清查舊墾,招闢生荒,派員丈勘繪圖,酌留蒙員隨缺地畝及公共牧廠,其餘乃悉開放之。牛羊群地,錯處左右翼間,直隸、山西民戶,頻年互爭,貽穀親往勘之,由固爾班諾爾中分界址,其爭始息。繼放察哈爾左翼地,為留牧廠、隨缺,與右翼同。移正黃旗牛羊兩群於商都牧群,又移騸馬群於騍馬群,籌撥直、晉邊學田。烏蘭察布、伊克昭兩盟夾河套為部落,烏拉特三公,杭錦、達拉特數旗,尤逼近套。其地恃河渠灌之,自元、明以還,渠盡湮廢,或並古道不存。貽穀躬蒞其地相度,修通長濟、永濟兩大幹渠,又疏濬塔布河、五加河、老郭諸渠,增鑿枝渠數十、子渠三百餘道,水利始興。先後六年,始自察哈爾兩翼八旗,而推之二盟十三旗,以及土默特、綏遠右衛與駐防馬廠各地,凡墾放逾十萬頃,東西二千餘里。絕塞大漠,蔚成村落,眾皆稱之。

復以時創設陸軍,置槍砲器械,築營壘,興警察,立武備陸軍學校及中小蒙學校數十所,創工藝局、婦女工廠。資送綏遠學生出洋,或就北洋學堂肄業。建設興和、陶林、武川、五原、東勝五。練巡防馬步十營,修繕綏遠城垣,濬城外溝渠,建築蒙地村屯,植樹造林,勸課園圃果實蔬菜。暇輒就田間耕夫婦豎問疾苦,或策單騎馳營壘,召士卒申儆之,教之以習勤崇儉,戒嗜好,勤勤如訓子弟,不率者乃罰譴之。方其治河套墾地,蒙人時起抗阻,臺吉丹丕爾攘其旗主地,戕文武官吏,貽穀請於朝誅之,眾始帖伏。

三十四年,貽穀劾歸化城副都統文哲琿侵吞庫款,而文哲琿先以敗壞邊局、蒙民怨恨劾貽穀。朝命軍機大臣鹿傳霖等往查,傳霖以已革布政使樊增祥等為隨員,奏覆,衹貽穀職,逮京,下法部勘問,三年不能決,卒坐誅丹丕爾事,譴戍川邊。宣統三年赴戍,方經鄂,武昌變起,直隸總督陳夔龍奏請進止,詔改易州安置。國變後,嘗自嘆曰:「昔姜埰譴戍宣城衛,自號『宣城老兵』。吾其終此矣!即死,必葬於是。」丙寅年,卒。晉邊官紳念其德,請昭雪,釋處分,遂葬易州白楊村,成其志。

信勤,字懷民,鈕祜祿氏,滿洲鑲黃旗人。以廕生累至浙江布政使,署巡撫,代貽穀為綏遠城將軍。督辦墾務,踵其遺規。益勤遠略,頗禮致賢才,思有所建樹,功未竟而遽罷。辛亥後,久病,卒。

論曰:將軍、都統,職視專圻,西北邊疆大臣與之並重。非才足當一面者不能任也。榮全、升泰以下諸人,或多戰績,或著邊功,或勤旗務,或兼民治,所至皆能盡其職,多有可稱,故並著於篇。


\end{pinyinscope}