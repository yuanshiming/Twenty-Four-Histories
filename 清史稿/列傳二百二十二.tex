\article{列傳二百二十二}

\begin{pinyinscope}
華爾勒伯勒東法爾第福戈登日意格德克碑赫德帛黎

華爾,美國紐約人。嘗為其國將弁,以罪廢來上海,國人欲執之。會粵匪陷蘇州,上海籌防,謀練精兵。蘇松太道吳煦識其才,言於美領事,獲免,以是德之,原效力,俾領印度兵。既撤,自陳原隸中國。咸豐十年,粵匪陷松江,煦令募西兵數十為前驅;華人數百,半西服、半常裝,從其後。華爾誡曰:「有進無止,止者斬!」賊迎戰,槍砲雨下,令伏,無一傷者。俄突起轟擊之,百二十槍齊發,凡三發,斃賊數百。賊敗入城,躡之同入,巷戰,斬黃衣賊數人。賊遁走,遂復松江,華爾亦被創。

先是煦與華爾約,城克,罄賊所有以予。至是入賊館,空無所得,以五千金酬之。令守松江,又募練洋槍隊五百,服裝器械步伐皆效西人。同治元年,賊又犯松江富林、塘橋,眾數萬,直偪城下。華爾以五百人御之,被圍,乃分其眾為數圓陣,陣五重,人四鄉,最內者平立,以次遞俯,槍皆外指。華爾居中吹角,一響眾應,三發,死賊數百。逐北辰山,再被創,力疾與戰,賊始退。遂會諸軍搗敵營,殺守門者,爭先入毀之。是役也,以寡敵眾,稱奇捷。時浦東賊據高橋,偪上海,華爾約英、法兵守海濱,而自率所部進擊,賊大敗,加四品翎頂。

會李鴻章帥師至滬,乃隸戲下,令立常勝軍,益募兵三千俾教練,參將李恆嵩副之,餉倍發。賊據王家寺,與英提督何伯等合攻。華爾賈勇先入,大斬虜首,進偪南翔,賊亦悉眾轟拒,何伯負傷。華爾冒煙直進,立毀其營,生獲八百餘人,遂復嘉定。規取青浦,華爾略東門,城潰;英、法兵自西入,華爾為承。賊奔,爭赴水死。攻奉賢,法提督卜羅德遇害,詔賞貂皮採絨,恤其家。時恆嵩扼趙屯港、四江口,屢失利,嘉、青復危。華爾方議直搗金山衛,聞敗,還守青浦。而富林、泗涇又相繼失,乃棄青浦,簡壯士五百襲天馬山,破之。入城挈守軍出,並力守松江,登陴轟擊兩晝夜不絕,賊宵遁,圍解。官軍圖青浦,華爾攻南門,駕輪舶入濠,毀城十餘丈,麾眾登堞,賊鬥且走,追敗之白鶴江黃渡,復其城,晉副將銜,降敕褒賞。俄偽慕王譚紹光復來犯,薄西門,與總兵黃翼升各軍擊之,賊潰,奔北岸,華爾毀其七營。逾月,會西兵再復嘉定。

其秋,賊十萬復犯上海,華爾自松江倍道應赴,與諸軍擊卻之。時寧波戒嚴,巡道史致諤乞援,鴻章遣華爾偕往。值廣艇與法兵構釁,引賊寇新城,從姚北紆道犯慈谿。華爾約西兵駕輪舶三,一泊灌浦,一泊赭山,一自丈亭駛入太平橋、餘姚四門鎮,而自率軍數百至半浦。平旦薄城,方以遠鏡了敵,忽槍丸洞胸,遽踣地,舁回舟。餘眾悉力奮攻,賊啟北門走。華爾至郡城,猶能叱其下恤軍事,越二日始卒。以中國章服斂,從其志也。鴻章請於朝,優恤之,予寧波、松江建祠。初,喪歸,煦檢其篋,得金陵城圖,賊所居處及城垣丈尺方位纖悉畢具,論者頗稱其機密雲。

勒伯勒東德加理尼阿爾伯依都額爾,法國加爾衣蔑多人。初為本國水師參將。咸豐十一年,來上海。時寇據寧波,西人惡之,益兵戍守,遣勒伯勒東乘輪泊三江口。同治元年,從官軍克府城,募壯丁千五百為洋槍隊,自陳願隸。明年,權授浙江總兵,受巡撫、寧波道節度。時上虞賊犯泗門、馬渚,勒伯勒東軍餘姚以待。尋與同知銜謝採嶂直搗賊屯,賊赴水死者千餘,乘銳毀其卡,薄城先登,擊殺守陴悍賊,餘宵遁,城克。赴蟶浦,略紹興,以賊遺土砲往,巡道張景渠止之,不聽,未幾,砲果裂,負傷而死,賜優恤。以法參將法爾第福為江蘇副將,領其軍,退守百官。

法爾第福,又名買忒勒,頗讀華書。後攻紹興,焚西郭門。次日復戰,潰十餘丈,麾眾登城,賊殊死鬥,別有黑種人數十助之,遂遇害。優恤之。

戈登,英國人。同治二年,李鴻章檄領常勝軍二千攻常州、福山營。別遣呂宋兵乘小舟薄賊壘,支木橋,伏死士城墻下。日中,港東西賊營皆破,緣墻入,痛殲之,遂奪福山石城。圍解,權授江蘇總兵。進攻太倉,毀南門賊卡,戈登轟潰二石壘,官軍繼進,克之。規取昆山,與總兵程學啟度地勢,以環昆多水,惟西南通進義,策先斷其歸路。遂與駕輪舶以偏師繞而西,賊不虞其至也,即時敗奔,奪其四壘。譚紹光構悍賊來爭,與諸軍大破之,薄昆城,偕李恆嵩夾擊,賊酋偽朝將先期逸去。逾月,學啟攻東城,戈登自果浦河奄至,扼守西路,分道疾攻。賊奪西門走,阻水,殲焉。遂留駐昆城,策應各路。移師攻花涇港,知賊必不誡,率眾擊北門,毀城外賊壘。次日,賊降,收吳江、震澤而還。

以事謁鴻章於上海。先是白齊文閉松城索餉,既撤,潛通賊,領二百人入蘇州。戈登詗知之,亟返昆山為備。旋攻蘇城,率軍三千,與學啟俱力爭要害,稍剪城外賊壘。偽忠王李秀成聞警赴援,屢敗;而紹光所部每戰猶致死,自偽納王郜雲官以下,皆萌貳志,詣營乞降。乃與學啟乘單舸會雲官等於洋澄湖,令斬秀成、紹光以獻,學啟與誓,戈登證之。未幾,秀成遁,雲官殺紹光,開齊門迎降,賞頭等功牌、銀幣,並犒其軍。助攻宜興、溧陽,並擊退楊舍賊。進規常州,轟破南門,合諸軍掘壕築墻以敗之。敘功,賞黃馬褂、花翎,賜提督品級章服。

初,戈登與學啟為昆弟交,每戰必偕。及誅降酋,頗不直其所為,捧雲官首而哭,誓不與見。嗣聞學啟卒,悲不自勝,乞其戰時大旗二,攜歸國為遺念。戈登歸後,埃及亂,督師討之,遇害。朝廷遣使往吊焉。戈登嘗言:「中國人民耐勞易使,果能教練,可轉弱為強。」又曰:「中國海軍利於守,船砲之制,大不如小。」當時稱其將略云。

日意格,法國人。嘗為其國參將,駐防上海。同治元年,改調稅務司。徙寧波,復郡城,與有功。官軍攻慈谿,遣法兵馳往策應。會餘姚四門鎮陷,遂與前護提督陳世章勒兵往討,逾月,直搗上虞。賊緣道築卡樹柵,悉奪毀之,薄城,並力轟擊,賊殊死戰,賈勇直前,被創,眾軍繼進,斬級千,賊始渡曹娥江去。進攻奉化,與諸軍克之。攻安吉思溪、雙福橋,駕小輪舶赴荻港,毀袁家匯賊壘,浙江平。左宗棠令與德克碑討測西邦制造,仿造小輪舶試行。五年,宗棠創福州船政局,充正監督,度地募工,殫心所事;復籌設繪事院、小鐵廠。七年,加提督銜,賞花翎。十三年,以船政教導勞賞銀幣。光緒年,卒。

德克碑,法參將。初,助攻奉化有功。旋奉其公使檄,將受代歸,謁左宗棠,宗棠撫諭之。德克碑感服,願易服色受節度。令駐守蕭山。蔣益澧攻杭城,檄助戰,游擊何文秀攻雞籠山,德克碑從寶塔嶺登岸,攻倚城賊壘。會天大霧,賊手冓嘉興援賊自萬松嶺偪都司張志公營,勢張甚。德克碑率眾助擊,敗之。益澧督水陸軍並進,連破九壘,令總兵高連升據其五,德克碑據其二,屯饅頭山。轟潰城數丈,毀鳳山門,官軍為承,城遂復。賊潰,奔湖州。攻安吉思溪,德克碑率所部助之,轟擊雙福橋,不克,駕小舟泊河汊,火八角亭,支木橋以濟。賊阻兵中流不得進,德克碑賈勇偪岸,所部遇伏卻走,改趨荻港,越壕入,克三壘。事寧,撤兵還上海。五年,充船政局副監督。七年,馬尾設船廠,督役興工,賞花翎。九年,宗棠平回亂,檄調甘肅,隸麾下。十三年,錄經始船政勞,膺獎賞。後卒。

赫德,字鷺賓,英國倍爾發司人。咸豐四年,來中國,充寧波領事署繙譯官,調廣州。又充香港督署書記官。九年,改任粵海關副稅務司。十一年,總稅務司李泰國奉令購戰艦,以赫德權代之,赴長江新開各口岸置新關。同治二年,李泰國去職,赫德實授,徙駐上海。三年,置臺灣南北新關。還駐京,加按察使銜。八年,晉布政使銜,赴緣海各地度置鐙樓塔表。光緒二年,佐定砲臺條約。十年,赴金陵與法使議越南案。會巡船置鐙樓臺灣洋,為法虜,乃遣駐英稅務司金登幹赴巴黎申理,乘機與議停戰草約,還。未幾,其國授為清、韓駐使,不就。逾年,賞花翎、雙龍二等第一寶星。

十二年,赴香港、澳門,條議洋藥稅釐並徵,並置關九龍、拱北。十三年,葡使來華,與訂澳門草約。十五年,藏兵寇哲孟雄,英兵乘勢闌入,赫德遣其弟稅務司赫政馳往,與駐藏大臣會籌劃界諸事。十九年,賞三代一品封典。二十五年,與德使籌置膠海新關。明年,各國聯軍入京,贊襄和議,晉太子少保。二十八年,召入覲,賜「福」字。三十一年,與德使更議膠關章程,改行無稅區地法。尋與日使籌置大連灣新關,徵榷一如膠海。三十三年,東三省度地置關。逾年,謝病歸,詔許之,加尚書銜。

赫德官中國垂五十年,頗與士大夫往還。嘗教其子習制藝文,擬應試,未許。總署嘗擬請授總海防司,道員薛福成以其陰鷙專利,常內西人而外中國,上書鴻章力爭之,議始寢。辛亥後,病卒,賜優恤。

帛黎,法國人。同治八年,來中國,充福州船政學校教員。十二年,賞五品銜,予雙龍獎牌。明年,調充江海關稅務幫辦,歷鎮江、北海、甌海、臨海、粵海諸關。光緒十九年,晉三品銜,調北京,遷稅務司。二十二年,朝議行郵政,以赫德兼領其事,帛黎實參治之。凡都會、省城、通商口岸,漸次置局,命曰「大清郵政」。尋徙拱北。二十六年,還京。明年,遷郵政總辦,晉二品銜。置代辦局於蕪湖。二十九年,河南、山東、山西、貴州復置副總局,自是內地城鄉村鎮,街郵遍設。時尚未入萬國郵政公會,即已與日本及英屬印度、香港聯約試行。三十年,賞雙龍三等第一寶星。與法、德及英屬那達商定聯郵章程。先後成郵政六百餘局,代辦四千二百餘所。宣統三年,改隸郵傳部,設總局,尚書盛宣懷疏薦之,遂被命為總辦,郵局置官自此始。越二年,乞病歸。未幾,卒。

論曰:華爾、戈登先後領常勝軍,立功江、浙,世稱「洋將」,時傳其戰略。日意格初亦參防戰,繼以船政著勞。赫德久總稅務,兼司郵政,頗與聞交涉,號曰「客卿」,皆能不負所事。茲數人者,受官職,易冠服,或原隸國籍。食其祿者忠其事,實有足多,故並著於篇。


\end{pinyinscope}