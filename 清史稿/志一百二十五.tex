\article{志一百二十五}

\begin{pinyinscope}
○交通二

△輪船

自西人輪船之制興,有兵輪,有商輪。其始僅往來東西洋各國口岸而已。中國自開埠通商而後,與英吉利訂江寧條約,而外輪得行駛海上矣。續與訂天津條約,而外輪得行駛長江矣。商旅樂其利便,趨之若鶩。於時內江外海之利盡為所占。

同治十一年,直隸總督李鴻章建議設輪船招商局,論者謂妨河船生計。鴻章謂當咸豐間河船三千餘艘,今僅存四百艘。及今不圖,將利權盡失。請破群議力行之。十三年,鴻章又疏言:「同治間曾國籓、丁日昌在江蘇督、撫任,迭據道員許道身、同知容閎創議華商造船章程,分運漕米,兼攬客貨。曾經寄請總理衙門核準,飭由江海關道曉諭各口商人試辦。日久因循,未有成局。同治七年,僅借用夾板船運米一次,旋又中止。本年夏間,臣於驗收海運之暇,遵照總理衙門函示,商令浙局總辦海運委員知府硃其昂酌擬輪船章程。嗣以現在官造輪船內並無商船可領,各省在滬殷商,或自置輪船行駛各埠,或挾資本依附西商之籍。若中國自立招商局,則各商所有輪船股本必漸歸官局,似足順商情而強國體。擬請先行試辦招商,為官商浹洽地步。俟商船造成,即可隨時添補,推廣通行。又海運米石,本屆江浙沙寧船不敷,應請以商船分運,以補沙寧之不足。將來米數愈增,可無缺船之患。請照戶部核準練餉制錢借給蘇、浙典章,準商等借領二十萬緡,以作設局商本,仍預繳息錢助賑。所有盈虧,全歸商認,與官無涉。當令硃其昂回滬設局招商。商人爭先入股,現已購集堅捷輪船三艘。經臣咨商浙江督撫臣飭撥明年漕米二十萬石,由招商輪船運津,其水腳耗米等項,悉照沙寧船定章。至攬載貨物,報關納稅,仍照新關章程,以免藉口。若從此輪船暢行,庶使我內江外海之利不致為洋人占盡,其關於國計民生者實非淺鮮。」疏入,報可。

先是閩廠專為制造兵輪而設。學士宋晉言糜款過鉅,議請罷之。事下,鴻章力持不可。略言:「歐洲諸國闖入中國邊界腹地,無不款關而求互市。海外之險,有兵船巡防,而我與彼可共分之。長江及各海口之利,有輪船轉運,而我與彼亦共分之。或不至讓洋人獨擅其利與險,而浸至反客為主也。」又言:「沿江沿海各省,不準另行購雇西洋輪船。若有所需,令其自向閩、滬兩廠商撥訂制。至載貨輪船,與兵船規制迥異。閩廠現造之船,商船皆不合用。曾國籓前飭滬廠造兵船外,另造商船四五艘。閩廠似亦可間造商船,以資華商雇領。現與曾國籓籌議,中國殷商每不原與官交涉,且各口岸生意已被洋商占盡。華商領官船,另樹一幟,洋人勢必挾重貲以傾奪,則須華商自立公司,自建行棧,自籌保險,本鉅用繁,初辦恐亦無利可圖。若行之既久,添造與租領稍多,乃有利益。聞華商原領者,必準其兼運漕糧,方有專門生意,不至為洋商排擠。將來各廠商船造有成數,再請敕下總理衙門,商飭各省籌辦。」疏上,下所司議行。

是年冬,招商局成立,以知府硃其昂主其事,道員盛宣懷佐之。其昂以道員胡光墉、李振玉等招徠商股,入貲者極為踴躍,宣懷亦援粵人唐廷樞、徐潤董局事。購船、設械、立埠,次第經營,悉屬商本,規模觕具。光緒元年,鴻章奏獎其昂等有差。三年,增購旗昌船艦,始假用直隸、江蘇、江西、湖北、東海關官款百九十萬兩有奇。初擬購旗昌輪船,宣懷持之最力,需銀二百數十萬兩。商本無幾,不足以應。宣懷以國防大計、江海利源之說,力陳於江督沈葆楨。葆楨為所動,撥銀百萬以濟,論者咸謂是舉為失計,至以「旗昌棄垂敝之裘,得值另制新衣,期於適體」為喻。事後募集商股,應者寥寥,僅得銀四萬者以此也。御史董俊翰言:「招商局每月虧至五六萬兩。致虧之由,因置船過多,輪車行駛,經費過鉅,必須一船得一船之用,方可無虞折耗。聞商局各船攬載之資,不敷經費,船多貨少。刻下既未能遽赴外洋各國,以廣收貿易之利,祗宜量為變置,使所出之數不至浮於所入也。」六年,祭酒王先謙請整頓招商局務,語涉宣懷。疏下江督劉坤一,言宣懷於購旗昌輪船時,聲言有商款百餘萬,實無所有,有意欺謾,冀獲酬金,請奪宣懷職。復請以官款概作官股,以其贏餘作海防經費。疏入,均不報。

招商局所假官帑,至光緒六年,應分期拔還。乃償已逾半,復假洋債。鴻章言兼籌並顧,招商局力有未逮,請先償洋債,後及官帑,格於部議。嗣以遞年清還,而商股尚達四百萬兩焉。當招商開辦之初,僅輪船三艘。嗣承領閩、滬兩廠,購之英國,增至十二艘。迨購入旗昌輪船十八艘,遂與英商太古、怡和並稱三公司。貲本過鉅,收入轉微。

是年,以言官劾奏招商局辦理毫無實濟,請飭認真整頓,諭李鴻章及江督吳元炳澈查。鴻章等奏言:「輪船招商局之設,乃各商集股,自行經理,已於創辦之初奏明,盈虧全歸商認,與官無涉。輪船商務牽涉洋務,更不便由官任之,與他項設立官局開支公款者,迥不相同。惟此舉為收回中國利權,事體重大,故須官為扶持,並酌借官帑,以助商力之不足。光緒三年冬曾將商局事宜籌畫整頓復奏,並飭江海、津海兩關道,於每年結帳,就近分赴滬洋各局清查帳目,如有隱冒,據實奏請參賠。數年以來,雖有英商太古、怡和洋行極力傾擠,而局事尚足相持,官帑漸可拔還。復先承運京倉漕米、各省賑糧,不下數百萬石,徵兵調餉、解送官物軍械者,源源不絕,豈得謂於國事毫無實濟?其攬載客貨,以及出入款目,責成素習商業之道員唐廷樞、徐潤總理其事,每年結帳後,分晰開列清冊,悉聽入本各商閱看稽查。若局中稍有弊端,則眾商不待官查,必已相率追控。而自開辦至今,並無入股商人控告者。現值漕運攬載吃緊之時,若紛紛調簿清查,不特市面徒滋搖惑,生意難以招徠;且洋商嫉忌方深,更必乘機傾擠,冀遂其把持專利之謀,殊於中國商務大局有礙。總之,商局關系國課最重,而各關各納稅課,絲毫無虧,所借官款由商局運漕水腳分年扣還,公款已歸有著,其各商股本盈虧,應如前奏,全歸商認,與官無涉。應俟每年結帳時,照案由滬、津兩關道就近清查,以符定章。」疏入,報聞。

十一月,學士梅啟照言:「招商局自歸並旗昌輪船,各國輪船之利漸減,然祗在香港、福州、寧波、上海、天津、牛莊、長江等處碼頭,不如推廣,竟令其赴東西洋各國。請飭南北洋大臣,督令局員,酌派豐順、保大等船,先赴東洋試行。行之有效,漸及於西洋,則貿遷有無之利,中外分之。」明年,祭酒王先謙亦以為言。均下所司核議。先是招商局船駛往新嘉坡、小呂宋、日本等處,不足與外輪競利,尋即停罷。嗣遣和眾船往夏威仁國之檀香山、美之舊金山兩埠,華人麕集,航業頗振。因復遣美富船往。而各國商業,英為巨擘。七年,粵人梁雲漢等設肇興公司於倫敦,船政大臣黎兆棠實倡斯議。鴻章疏言:「西洋富強之策,商務與船政互相表裏。以兵船之力衛商船,必先以商船之稅養兵船,則整頓尤為急務。邇者各國商船爭赴中國,計每年進出口貨價約銀二萬萬兩以外。洋商所逐什一之利,已不下數千萬兩,以十年計之,則數萬萬兩。此皆中國之利,有往無來者也。故當商務未興之前,各國原可閉關自治。逮風氣大開,既不能拒之使不來,惟有自擴利源,勸令華商出洋貿易,庶土貨可暢銷,洋商可少至,而中國利權亦可逐漸收回。前此招商局輪船嘗駛往新嘉坡、小呂宋、越南等埠攬載。近年和眾、美富等船分駛夏威仁國之檀香山、美國之舊金山,載運客貨,究止小試其端,尚未厚集其力。英國倫敦為通商第一都會,並無華商前往。黎兆棠志在匡時,久有創立公司之議,盡心提倡,力為其難。現既粗定規模,自當因勢利導,期於必成。」報聞。

十年,法人來擾,海疆不靖,股商洶懼,局船慮為劫奪,以銀五百二十五萬兩暫售之旗昌行主。事平收回,復增置江新、新昌、新康、新銘各艦。而沈沒朽敝者,不一而足,其後共達二十九艘云。十二年,鄂督張之洞遣總兵王榮和至南洋,籌辦捐船護商事項。宣統三年,設商船學校於吳淞。凡此皆為擴充航業之張本,而局船行駛外洋之利,終不能與各國爭衡也。

招商局之設,本為挽回江海已失航利。開辦之始,即知為洋商所嫉,而彌補之策,首在分運蘇、浙漕米,嗣更推之鄂、贛、江、安。而滇之銅斤,蜀之燈木,江、浙之採辦官物,直、晉之賑糧,胥由局船經營其事。光緒十一年,道員葉廷眷復條上扶持商局運鄂茶、鄂鹽,增加運漕水腳諸策。事下直督李鴻章。先是局船運漕,石銀五錢有奇。嗣英、美人攬運,故廉其值,商局運費因之減少,勢益不支。鴻章請稍增益之,格部議,不果行。蓋招商局自開辦以來,局中之侵蝕與局外之傾擠,所有資力頗虞虧耗。商股不足,貸及官款,繼以洋債。當事者日言維持補救之策,裨益實鮮,而以用款浮濫,復屢為言官所劾。至是部臣疏言:「三代之治法,貴本而抑末,重農而賤商,從古商務未嘗議於朝廷。海上互市以來,論者乃競言商政。竊謂商者逐什一之利,以厚居積、權子母為事者也。厚居積,必月計之有餘;權子母,必求倍入之息。若計存本則日虧,問子母則無著,甚且稱貸乞假以補不足,猶號於眾曰『此吾致富之術也』,有是理乎?嘗見富商大賈,必擇忠信之人以主會計。其入有經,其出有節。守餘一餘三之法,核實厚積,乃能久遠。若主計不得其人,生之者寡,食之者眾,取之無度,用之無節,不旋踵而終窶。用人理財之道,與政通矣。前者李鴻章、沈葆楨創立此局,謀深慮遠,實為經國宏謀,固為收江海之利,與洋商爭衡,轉貧為富、轉弱為強之機,盡在此舉。乃招商局十餘年來,不特本息不增,而官款、洋債,欠負累累,豈謀之不臧哉?稽之案牘,證之人言,知所謂利權,上不在國,下不在商,盡歸於中飽之員紳。如唐廷樞、硃其昂之被參於前,徐潤、張鴻祿之敗露於後,皆其明證。主計之不得其人,出入之經,不能講求撙節,又安得以局本虧折,諉之於海上用兵耶?商局既撥有官款,又津貼以漕運水腳,減免於貨稅,其歲入歲出之款,即應官為稽察。請飭下南北洋大臣,將局中現存江海輪若干只,碼頭幾處,委員商董銜名,及運腳支銷,分別造報。此後總辦如非其人,原保大臣應即議處。」報可。然管理招商局之權,始終屬之直隸總督,部臣無從過問。迨三十三年,商局與英商怡和、太古訂利益均享之約,始免互相傾擠,而其利漸著。此招商局辦理之大略情形也。

招商輪船航行各埠,悉自滬始。駛行長江者曰江輪,駛行海洋者曰海輪。停泊口岸,大小不一,惟商務殷闐之所,設貨棧焉。以故上海設總棧,而蘇之鎮江、南京,皖之蕪湖,贛之九江,鄂之漢口,浙之寧波、溫州,閩之福州、汕頭,粵之廣州、香港,魯之煙臺,奉之營口,直之塘沽、天津,皆設行棧,而通州以漕運所關,亦設棧焉。江輪、海輪,時統名之為大輪。其與大輪並行於內江外海,或駛行大輪所不能達之處,則有小輪。光緒初,商置小輪之行駛,僅限於通商口岸。十年,明申禁令,小輪不得擅入內河。官商雇用,須江海關給照乃可。然祗限於蘇杭之間。其輸運客貨、駛入江北內河者,皆在所禁。

十六年,詹事志銳疏請各省試行小輪。總署王大臣議以為不可。護湘撫沈晉祥言:「湘民沿河居住,操舟為業者,實繁有徒。自上海通商以後,僅有淮鹽一項,尚可往來裝運,其餘貨物,多由輪船載送,湘省民船祗能行抵江、漢而止,舵工水手失業者多。今再加以小輪行駛內河,誠如總理衙門原奏所云,必至奪民船之利,有礙小民生計。」江督劉坤一亦言小輪行駛內河,流弊滋多,礙民生,妨國課,病地方,請嚴禁之。俱如所請。

初,外輪行駛長江,由滬至漢口而止。二十一年,馬關約成,許日輪一自漢口達宜昌,更溯江上至重慶,一自上海入運河以抵蘇、杭,於時朝旨始許華商小輪於蘇杭間行駛。而江督張之洞更推廣其航行之路於鎮江、江寧、清江浦及贛之鄱陽。二十四年,長江通商約成,而通州蘆涇港、泰興天星橋、湖北荊河口悉定為洋輪上下搭客處,而桂之西江、直之白河、沈之遼河、松花江,亦先後許外輪行駛。迨中英馬凱約成,更及於粵之北江、東江。與英、日訂內港行輪章程,凡內地水道,外輪悉攫得行駛之權,於是向之華商小輪不得行駛各地,始一律弛禁焉。江、浙、閩、粵輪船公司次第設立,轉輸客貨,人稱便捷。特以洋商創始於前,華商瞠乎其後,而跌價傾擠,時有所聞,欲求贏利,蓋綦難矣。

三十年,商部參議王清穆言:「植商業之基礎,莫如內河航政一事。凡鐵路之尚未通者,可藉航路控接之,凡軌路所不能達者,可由航路轉輸之。江、鄂諸省,若漢湘,若九南,若鎮揚、鎮浦、蘇杭、蘇滬、常鎮各航路,四通八達,往往為外人所經營,其公司多不過數萬金,視軌路之動需千百萬者,難易迥殊,華商之力尚能興辦,洵為今日切要之舉。請飭各省有航路處所,於華商輪船公司亟予保護。未設者,提倡籌辦。」報可。自是小輪公司漸推漸廣,閩、粵濱海之區,輪檣如織,隨處可通。直則有往來安東、天津、大連、營口、牛莊、煙臺、龍口、義馬島、威海衛、海參崴之小輪,蘇則有往來鎮江、清江浦、通州、海門、上海、蘇、杭、江寧、揚州、六合之小輪,皖則有往來蕪湖、廬州、安慶、寧國、巢縣之小輪,贛則有往來南昌、九江、吳城、湖口、豐城、樟樹鎮、吉安、饒州之小輪,湘、鄂則有往來漢口、黃州、沙市、宜昌、武昌、嘉魚、長沙、株州、常德、咸寧、岳州、湘潭、益陽、仙桃鎮、老河口之小輪,桂則有往來梧州、南寧、貴縣、柳州之小輪,浙則有往來寧波、溫州、穿山、定海、象山、寧海、臺州、海門、沈家門、普陀山、餘姚、西塢、瑞安、平望、震澤、南潯之小輪,川則有往來宜昌、重慶、嘉定、敘府之小輪,各公司盈虧不一,而航路四達,商旅便之,實與江海大輪有相輔而行之利。此外則有各省官用小輪暨專用小輪,是又於商輪之外特設者也。

三十一年,修撰張謇醵銀五十萬,設大達輪步公司於上海。宣統三年,吉林巡撫陳昭常創辦吉林圖長航業公司,自滬越日本長崎達圖們江,以滬商硃江募貲為之。此皆於招商局外別樹一幟者也。


\end{pinyinscope}