\article{列傳三百十四}

\begin{pinyinscope}
屬國二

○越南

越南先稱安南。順治初,安南都統使莫敬耀來歸,未及授爵而卒,尋授其子莫元清為安南都統使。

十六年八月,經略大學士洪承疇始奏言安南國遣吏目玉川伯鄧福綏、朝陽伯阮光華,齎啟赴信郡王軍前抒誠納款。十七年九月,黎維祺始自稱國王,奉表貢方物,帝嘉之,賜文綺、白金。十八年,敕曰:「朕惟修德來遠,盛代之弘謨;納款歸仁,人臣之正誼。既輸誠而向化,用錫命以宣恩。褒忠勸良,典至重也。爾安南國王黎維祺,僻處炎方,保有厥眾。乃能被服聲教,特先遣使來歸,循覽表文,悃忱可見。古稱識時俊傑,王庶幾有之。用錫敕獎諭,仍賚爾差官金義仁根銀幣衣服等事,遣通事序班一員伴送至廣西,沿途撥發兵馬導之出疆。爾受茲寵命,其益勵忠節,永作屏籓,恪守職貢,丕承無斁。欽哉!」未幾,維祺卒,子維禔嗣。尋又卒,子維禧嗣。

康熙二年十一月,維禧遣黎斅等表謝,附貢方物。三年二月,遣內院編修吳光、禮部司務硃志遠,諭祭故王維祺、維禔。五年五月,維禧繳送故明王永歷敕、印,遣內國史館翰林學士程方朝、禮部郎中張易賁冊封維禧為安南國王,賜鍍金駝鈕銀印。六年,維禧奪都統使莫元清高平地,元清奔雲南,上疏陳訴,帝命安置南寧。維祺亦上疏言興兵復仇本末。

初,明正德十一年,社堂燒香官陳暠殺其王莫晭自立,晭臣都力士莫登庸討殺暠,立晭兄子譓。嘉靖元年,登庸逐譓自立,譓子黎平據清華自為一國。後莫氏漸衰,但保高平一郡,勢益弱。至是,帝遣內院侍讀李仙根、兵部主事楊兆傑,齎敕諭維禧,將高平土地人民歸莫元清:「各守其土,盡爾籓職。」初,安南定為三年一貢。七年,維禧疏請六年兩貢並進,帝如所請。八年,使臣李仙根等齎回維禧覆疏,言遵旨將高平府石林、廣原、上瑯、下瑯土地人民歸莫元清,因奏稱黎維禧所歸土地,尚有保樂、七源二州,昆侖、金馬等十二總社未還,請再敕諭全還,帝不許。

是年,黎維禧薨,弟維權理國事。十三年正月,維以訃告,遣陪臣胡士揚等進康熙八年、十一年歲貢,疏言:「先王世守安南,為逆臣莫登庸篡弒,賴輔政鄭檍之祖剿除恢復。莫逆遺孽篡據高平,乍臣乍叛。至莫元清懼臣討罪,潛入內地投誠。康熙八年,奉命令還高平,臣維禧欽奉君命,敢不■遵。但莫元清為臣不共之仇,高平為世守之土,叛逆竊據,禍在蕭墻。叩■天恩,仍令高平屬歸本國。且莫元清尚有誓辭及祭伊父莫敬耀文,內有『圖逆天朝』之語,今謹敬呈,並貢方物。」事下部議。尋議:「前維禧退還莫元清高平,取有復相和好印結。今維雖言收得誓書、祭文,但此文年久,誓辭系莫敬耀名,或得自敬耀存時,或得自元清今日,殊難懸擬,應飭維查明具題再議。」從之。

十四年,黎維卒,弟維正權理國事。十六年,帝諭維正曰:「逆賊吳三桂,值明季闖賊之變,委身從賊,以父死賊手,窮竄來歸,念其投誠,錫之王爵,方且感恩圖報,殫竭忠誠。詎意以梟獍之資,懷狙詐之計,陰謀不軌,自啟釁端,藉請搬移,輒行叛逆,煽惑奸宄,塗炭生靈。朕連年遣兵征討,秦、隴底定,閩、粵蕩平,惟吳三桂竊據一隅,茍延旦夕。今大兵雲集,恐其挺走,潛竄嶺南。茲以王累世屏籓,效忠天國,亂臣賊子,諒切同仇。今已遣諸軍大張撻伐,平定粵西,進取滇、黔。爾國壤地相屬,素諳形勢,王其遴選將士,協力殲除,懋賞榮褒,朝有令典。欽哉,無負朕命!」十八年十一月,維正慶賀大捷,疏言:「逆賊吳三桂,變亂數年,阻臣貢路,且再三脅誘,迫令服從,區區愚忠,罔敢易節。乃有逆臣莫元清與三桂密相締結,潛入高平,圖為掩襲。今原仗天威,追擒逆黨,明正其罪,以固屏籓。」許之。

二十一年九月,維正遣陪臣甲全等表賀閩、粵肅清,並進歲貢方物;又為故王維請恤,議恤如例。時所貢金銀器皿與本內不符,詔免深求,其餘貢物酌減白絹、降真香、中黑線香等物。二十二年四月,遣翰林院侍讀明圖、翰林院編修孫卓冊封黎維正為安南國王,御書「忠孝守邦」四字賜之。同時遣翰林院侍讀鄔黑、禮部郎中周燦諭祭故王維禧、維。時莫元清已故,其弟敬光為黎氏所敗,率眾來奔,帝命發回安南。尋敬光病歿泗城土府,莫氏遂絕。

二十五年,增賜安南國王表裏五十,著為例。三十六年,維正奏言牛馬、蝴蝶、浦園三處為鄰界土司侵占,請給還。帝問云南巡撫石文晟,知其地屬開化府已三十餘年,並非安南故地,移文責之。五十七年十月,黎維正薨,嗣子維示匋以訃告,請襲封,附貢方物。五十八年二月,遣內閣中書鄧廷喆、翰林院編修成文諭祭故王黎維正,兼冊封維祹為安南國王。

雍正二年,維祹遣陪臣表賀登極,附貢方物,賜御書「日南世祚」四字。三年,雲南總督高其倬奏言:「雲南開化府與安南接界,自開化府馬伯汛外四十里至鉛廠山下小河內有逢春裏六寨,冊載秋糧十二石零。康熙二十八年,入於安南。又云南通志載自開化府文山縣南二百四十里至賭咒河與安南為界。今自開化府至現在之馬伯汛,止一百二十里,即至鉛廠山下小河,亦止一百六十里,是鉛廠山小河外尚有八十里,內設都龍、南丹兩廠,為雲南舊境。雖失在前明,但封疆所系,均應一並清查,委勘立界。」帝諭:「都龍、南丹等處明季已入安南,是侵占非始於我朝。安南入我朝以來,累世恭順,不宜與爭尺寸之地。」維示匋尋疏辯。

嗣總督鄂爾泰疏請於鉛廠山下小河離馬伯汛四十里立界,維祹復激詞陳訴。五年,諭維祹曰:「朕統馭寰區,凡茲臣庶之邦,莫非吾土,何必較論此區區四十里之地。但分疆定界,政所當先,侯甸要荒,事同一體。今遠籓蒙古,奉諭之下,莫不欽承,豈爾國素稱禮義之邦,獨違越於德化之外哉?王不必以侵占內地為嫌,拳拳申辯,此乃前人之誤,非王之過也。王惟祇遵諭旨,朕不深求,儻意或遲回,失前恭順,則自取咎戾,懷遠之仁,豈能幸邀?王其祇哉,無替朕命!」維祹感悔奏謝。帝因以馬伯汛外四十里賜維祹,仍以馬伯汛之小賭咒河為界。六年三月,遣副都御史杭奕祿、內閣學士任蘭枝往安南宣諭,略云:「王今自悔執迷,情詞恭謹,朕特沛殊恩,即將馬伯汛外四十里之地,仍賜國王世守之。」尋諭鄂爾泰曰:「朕既加恩外籓,亦當俯從民便。此四十里內人民,若有原遷內地者,可給貲安插滇省,毋使失所。其原居外籓屬安南管轄者,亦聽其便。」

十一年十一月,黎維祹薨,王嗣子維祜以訃告,請襲封,附貢方物。十二年二月,遣翰林院侍讀春山、兵科給事中李學裕諭祭故王維祹,冊封維祜為安南國王。十三年,黎維祜薨,弟維禕權理國事。乾隆二年,維禕以訃告,請襲封。遣翰林院侍讀嵩壽、修撰陳倓諭祭故王維祜,冊封維禕為安南國王。三年九月,維禕遣使奉表賀登極,並貢方物。

九年九月,兩廣總督馬爾泰奏:「粵西奸民葉蓁私出外夷,誘教為匪,安南饑民流入寧明諸處。」帝命滇、粵界接安南關隘嚴行稽查,毋釀事端。嗣兩廣總督馬爾泰、廣西署撫托庸、提督豆斌奏言:「南寧府屬遷隆土峒之板蒙等隘,太平府屬思陵土州之川荒等隘,鎮南府屬下雷土州之下首等隘,共三十餘口岸,俱逼近安南,宜疊石建柵,添卡撥兵,各土司帶領土勇,扼險守巡,並飭地方官每年冬月查修通報。安南驅驢地方為貨物聚集之所,最與由隘相近。從由隘出入,向設閉禁,開之實便商民。應設客長,稽商民往來,並責地方官慎察查。至平而、水口兩關,通太源、牧馬等地,宜設立鐵鍊橫江攔截,逢五、十日開一面以通商。」從之。初,廣西思陵州沿邊與安南接壤,巡撫舒輅請栽竹以杜私越。憑祥、思陵土目有乘機侵安南地者,交人不甘,恆與爭閧。十六年,總督蘇昌奏聞,帝諭舒輅下部察議。

安南瑤匪盤道鉗、鄧成玉等謀亂,造黃袍、黃旗、木印,句結內地民夷何聖烈等,散劄招匪,謀攻都龍、安北、宜經等處,為安南兵目偵知,獲何聖烈等,盤道鉗等竄匿山箐間。十九年,安南八寶河沙目黃國珍誘獲盤道鉗、鄧成玉,雲貴總督碩色訊得實,奏聞正法。初,廣東土匪李文光與順化土豪阮姓謀踞祿賴、桐狔等處為亂,番官捕獲系諸獄。二十一年,械送李文光十六人於福建,閩浙總督喀爾吉善奏言:「安南僻處蠻陬,不敢將李文光擅自加誅,送歸請示,足徵懷服之忱。應將李文光等照交結外國例,分別處治。」從之。二十二年六月,安南番船失風,飄泊永寧汛,撥兵守護,給貲送歸,並收貯其軍械,歸時給還。帝諭:「收械貯庫,殊為非體,可頒諭沿海提鎮知之。」二十五年,閩浙總督愛必達奏言:「安南邊境沙匪與交目蘇由為難,闌入漫卓、馬鹿二寨,搶掠滋事,已咨其國王擒解矣。」帝以平日巡防不嚴,臨時追捕不力,切責之。

二十六年,黎維禕薨,王嗣子維以訃告,請襲封,遣翰林院侍讀德保、大理寺少卿顧汝修諭祭故王維禕,冊封維為安南國王。維欲以彼國五拜事天之禮受封,德保等執不可,隨如儀,禮成。顧汝修既出境,以安南王送迎儀節未周,遺書責之,廣西巡撫熊學鵬以聞,汝修坐革職。二十七年三月,帝諭禮臣曰:「安南世為屬國,凡遇朝使冊封至其國,自應遵行三跪九叩頭禮。乃國王狃於小邦陋見,與冊使商論拜跪儀注,德保、顧汝修指示成例,始終恪遵。外籓不諳體制,部臣應預行宣示。嗣後遇安南冊封等事,即將應行典禮並前後遵行拜跪儀節告知正副使,令其永遠遵循,著為令。」三十四年,安南莫氏後黃公纘居南掌猛天寨,黎氏逼之,率屬內投,維請索回處治,移檄責之。

四十三年,安南解竄匪入關,賜維緞匹。四十六年,維遣使謝恩,貢方物。帝命收受,下次正貢著減一半,並命嗣後陳謝表奏,毋庸備禮。五月,諭禮部:「本年安南國貢使到京,命堂官一人帶往熱河瞻覲。」四十九年,帝南巡,安南陪臣黃仲政、黎有容、阮堂等迎覲南城外,賜幣帛有差,特賜國王「南交屏翰」扁額。

五十一年,安南阮氏變作。初,明嘉靖中,安南王黎維潭復國,實其臣鄭氏、阮氏之力,自是世為左右輔政。後右輔政乘阮死幼孤,兼攝左輔政以專國事,而出阮氏於順化,號廣南王。阮、鄭世仇構兵。及黎維,權益下移,僅同守府。輔政鄭棟遂殺世子,據金印,謀篡國,而忌廣南之強,乃誘其土酋阮岳、阮惠,共攻廣南王,滅之於富春。阮惠自為泰德王,鄭棟自為鄭靖王,兩不相下,維無如何也。

安南所都曰東京,即古交州,唐安南都護治所;而以廣南、順化二道為西京,即古日南、九真地。黎維潭起兵之所,與東京中隔海口,世為廣南阮氏所據,兵強於安南。至是,鄭棟死,阮惠以鄭姓專國,人心不附,乃藉除鄭氏為名,攻破黎城,擊滅鄭棟之子鄭宗,阮氏復專國,維犒以兩郡,且妻以女。五十二年,維卒,嗣孫維祁立,阮惠盡取象載珍寶歸廣南,使鄭氏之臣貢整留鎮都城。貢整思扶黎拒阮,乃以王命率兵奪回象五十,而阮嶽亦於廣南要奪其輜重。阮惠歸,治城池於富春,使其將阮任以兵數萬攻貢整於國都。整戰死,維祁出亡,阮任遂據東京,四守險要,有自王之志。五十三年夏,阮惠復以兵誅阮任於東京,而請維祁復位。維祁知其叵測,不敢出。惠知民心不附,盡毀王宮,挾子女玉帛舟回富春,留兵三千守東京。

有高平府督阮輝宿者,護維祁母妻宗族二百口由高平登舟遠遁至博淰溪河,廣西太平府龍州邊也,冒死涉水登北岸,其不及渡河者,盡為追兵所殺。兩廣總督孫士毅、廣西巡撫孫永清先後以聞,且言:「推固予奪,惟上所命。」帝以黎氏守籓奉貢百有餘年,宜出師問罪,以興滅繼絕。先置其家於南寧,遣其陪臣黎屌、阮廷枚回國,密報嗣孫。時安南疆域,東距海,西接老撾,南與占城隔一海口,北連廣西、雲南。有二十二府,其二府為土司所居,實止二十府,共分十三道。此時未陷者,清華道四府十五縣,宣光道三州一縣,興化道十州二縣;又上路未陷、下路已陷者,安邦道四府十二縣,山西道五府二十四縣,京北道四府二十縣,太源道三州八縣;其上路已陷、下路未陷者,山南道九府三十六縣,海陽道四府十九縣。惟廣南、順化二道,本阮酋巢穴,又據高平道一府四州,諒山道一府七縣,以捍遏內地。

帝命孫士毅移檄安南諸路,示以順逆,早反正。時維祁弟維、維昬皆外出避難,維死宣光城,維昬由京北波篷廠來投。孫士毅以維昬有才氣,欲令權攝國事。帝慮其兄弟日後嫌疑,不許,乃令土田州岑宜棟護維昬出口,號召義兵。會阮廷枚等以嗣孫復書至,乞轉奏。於是安南國土司及未陷各州官兵爭縛偽黨獻地圖,而關外各廠義勇亦皆乞餉團練,請為鄉導。時阮惠兄弟亦叩關請貢,以其國臣民表至,言黎維祁不知存亡,請立故王維之子翁皇司維主國事,並迎其母妃回國。帝知阮惠欺維愚懦易與,狡計緩師,命孫士毅嚴斥之。

安南進兵路三:一,出廣西鎮南關為正道;一,由廣東欽州泛海,過烏雷山至安南海東府,為唐以前舟師之道;一,由雲南蒙自縣蓮花灘陸行至安南之洮江,乃明沐晟出師之道。孫士毅及提督許世亨率兩廣兵一萬出關,以八千直搗王京,以二千駐諒山為聲援。其雲南提督烏大經以兵八千取道開化府之馬白關,逾賭咒河,入交趾界千有百里而至宣化鎮,較沐晟舊路稍近。雲貴總督富綱請行,帝以一軍不可二帥,命駐關外都龍督餉運。

十月末,粵師出鎮南關。詔以安南亂後,勞瘠不堪供億,運餉由內地滇、粵兩路,設臺站七十餘所,所過秋亳無犯。孫士毅、許世亨由諒山分路進,總兵尚維升、副將慶成率廣西兵,總兵張朝龍、李化龍率廣東兵。時土兵義勇皆隨行,聲言大兵數十萬,各守隘賊望風奔遁,惟扼三江之險以拒。十一月十三日,尚維升、慶成率兵千餘,五鼓抵壽昌江。賊退保南岸,我兵乘之,浮橋斷,皆超筏直上。時天大霧,賊自相格殺,我兵遂盡渡,大破之。張朝龍亦破賊柱石。十五日,進兵市球江。江闊,且南岸依山,高於北岸,賊據險列砲,我兵不能結筏。諸軍以江勢繚曲,賊望不及遠,乃陽運竹木造浮橋,示必渡,而潛兵二千於上游二十里溜緩處用小舟宵濟。十七日,乘筏薄岸相持。適上游兵已繞出其背,乘高大呼下擊,聲震山谷。賊不知王師何自降,皆驚潰。

十九日,薄富良江,江在國門外,賊盡伐沿江竹木,斂舟對岸。然遙望賊陣不整,知其眾無固志,乃覓遠岸小舟,載兵百餘,夜至江,復奪小舟三十餘,更番渡兵二千,分搗賊營。賊昏夜不辨多寡,大潰,焚其十餘艘,獲總兵、侯、伯數十。黎明,大軍畢濟。黎氏宗族、百姓出迎伏道左,孫士毅、許世亨入城宣慰而出。城環土壘,高不數尺,上植叢竹,內有磚城二,則國王所居,宮室已蕩盡矣。而黎維祁匿民村,是夜二鼓始出詣營見孫士毅,九頓首謝。捷聞。初,王師之出也,帝慮事成後,冊封往返稽時,致王師久暴露於外,先命禮部鑄印,內閣撰冊,郵寄軍前。孫士毅遂以二十二日宣詔冊封黎維祁為安南國王,並馳報孫永清歸其家屬。維祁表謝,請於乾隆五十五年詣京祝八旬萬壽。帝命俟安南全定,維祁能自立,許來朝。是役也,乘思黎舊民與各廠義勇先驅鄉導,又許世亨、張朝龍等新自臺灣立功,皆善戰之將,故得以兵萬餘長驅深入,不匝月而復其都,時雲南烏大經之兵尚未至也。詔封孫士毅一等謀勇公,許世亨一等子,諸將士賞賚有差。

時阮惠已遁歸富春,孫士毅謀造船追討。孫永清奏言:「廣南距黎都又二千里,用兵萬人,設糧站需運夫十萬,與鎮南關至黎城等。」帝以安南殘破空虛,且黎氏累世孱弱,其興廢未必非運數也。既道遠餉艱,無曠日老師代其搜捕之理,詔即班師入關。而孫士毅貪俘阮為功,師不即班,又輕敵,不設備,散遣土軍義勇,懸軍黎城月餘。阮氏諜知虛實,歲暮傾巢出襲國都,偽為來降者,士毅等信其誑詞,晏然不知也。五十四年正月朔,軍中置酒張樂,夜忽報阮兵大至,始倉皇禦敵。賊以象載大砲沖我軍,眾寡不敵,黑夜中自相蹂躪。黎維祁挈家先遁,滇師聞砲聲亦退走,孫士毅奪渡富良江,即斬浮橋斷後,由是在岸之軍,提督許世亨、總兵張朝龍,官兵夫役萬餘,皆擠溺死。時士毅走回鎮南,盡焚棄關外糧械數十萬,士馬還者不及半。其雲南之師,以黎臣黃文通鄉導得全返。黎維祁母子復來投。奏聞,帝以士毅不早班師,而又漫無籌備,致挫國威、損將士,乃褫職來京待罪,以福康安代之。

阮惠自知賈禍,既懼王師再討,又方與暹羅構兵,恐暹羅之乘其後也,於是叩關謝罪乞降,改名阮光平,遣其兄子光顯齎表入貢,懇賜封號。略言守廣南已九世,與安南敵國,非君臣。且蠻觸自爭,非敢抗中國,請來年親覲京師,並於國內為死綏將士築壇建廟,請頒官銜謚號,立主奉祀。又聞暹羅貢使將入京,恐受其媒孽,乞天朝勿聽其言。福康安先後以聞。

帝以維祁再棄其國,並冊印不能守,是天厭黎氏,不能自存;而阮光平既請親覲,非前代莫、黎僅貢代自金人之比。且安南自五季以來,曲、矯、吳、丁、李、陳、黎、莫互相吞噬,前代曾郡縣其地,反側無常,時憂南顧。乃允其請,即封阮光平為安南國王,冊曰:「朕惟王化遐覃,伐罪因而舍服,侯封恪守,事大所以畏天。鑒誠悃於荒陬,貰其既往,沛恩膏於屬國,嘉與維新,賁茲寵命之頒,勖以訓行之率。惟安南地居炎徼,開十三道之封疆,而黎民臣事天朝,修百餘年之職貢,每趨王會,舊附方輿。自遭難以流離,遂式微而控愬。方謂興師復國,字小堪與圖存,何期棄印委城,積弱仍歸失守,殆天心厭其薄德,致世祚訖於終淪。爾阮光平起自西山,界斯南服,向匪君臣之分,浸成婚媾之仇。釁啟交訌,情殊負固。抗顏行於倉卒,雖無心而難掩前愆,悔罪咎以湔除,原革面而自深痛艾。表箋籥請,使先猶子以抒忱,琛獻憬來,躬與明年之祝嘏。自非仰邀封爵,榮藉龍光,曷由下蒞民氓,妥茲鳩集。況王者無分民,詎在版章其土宇,而生人有司牧,是宜輯寧爾邦家。爰布寵綏,俾憑鎮撫,今封爾為安南國王,錫之新印。於戲!有興有廢,天子惟順天而行,無貳無虞,國王咸舉國以聽。王其懋將丹款,肅矢冰兢,固圉以長其子孫,勿使逼滋他族,悉心以勤於夙夜,罔令逸欲有邦,益敬奉夫明威,庶永承夫渥典。欽哉,毋替朕命!」其黎維祁賞三品銜,令同屬下人戶來京,歸入漢軍旗下,即以維祁為佐領。又令阮光平訪問維祁親屬,護送進關。其前安插內地之西南夷人,有系懷故土者,並令阮光平善為撫綏,以示矜全。

五十五年,阮光平來朝祝釐,途次封其長子阮光纘為世子。七月,入覲熱河山莊,班次親王下、郡王上,賜禦制詩章,受冠帶歸。其實光平使其弟冒名來,光平未敢親到也,其譎詐如此。五十六年,擊敗黎維昬及萬象國之師來獻捷,帝優賞之。五十七年,議定安南貢期,舊例三年一貢者,定為兩年,六年遣使來朝一次者,定為四年。

九月,阮光平在義安病故,世子阮光纘權國事,以訃告。五十八年正月,遣廣西按察使成林諭祭,加謚忠純,並頒賜禦制詩,於墓道勒碑,以表恭順。封光纘為安南國王。帝以阮邦新造,人心未定,阮光纘尚幼,且阮岳尚在廣南,吳文楚久握兵柄,主少國疑,恐有變,特調福康安總督云、貴備邊,並令成林密偵其國。成林旋以國事觕定聞,乃止。

八月,署兩廣總督郭世勛奏安南添立花山市。先是安南通市,平而、水口兩關商人在其國之高憑鎮牧馬立市,由隘商人在諒山鎮之驅驢立市,分設太和、豐盛二號,並置廒長、市長各一人,保護、監當各一員。而從平而關出口之商,必由水路先抵花山,計程僅二百餘里。且花山附近村莊稠密,至是添設行鋪,其市長、監當各員,即於驅驢額內派往。客民中有由陸路前赴牧馬者,仍聽其便。

嘉慶元年,福州將軍魁倫、兩廣總督吉慶先後奏言,獲烏艚船海盜,有安南總兵及封爵敕命、印信等物。初,阮氏據廣南,以順化港為門戶,與占城、真臘、暹羅皆接壤,西南瀕海。有商舶飄入海者,阮氏輒沒入其貨,即中國商船,亦倍稅沒其半,故紅毛、占臘、暹羅諸國商船,皆以近廣南灣為戒。阮光平父子既以兵篡國,國用虛耗,商船不至,乃遣烏艚船百餘、總兵十二人,假採辦軍餉,多招中國沿海亡命,啖以官爵,資以器械船隻,使鄉導入寇閩、粵、江、浙各省。時浙師御海盜,值大風雨,雨中有火爇入賊舟,悉破損。參將李成隆率兵涉水取賊砲,並搜獲安南敕文、總兵銅印各四。敕稱「差艚隊大統兵進祿侯倫貴利」,而教諭王鳴珂獲三賊,一詭為瘖者,一名王貴利,訊,云即倫貴利也。同時閩中獲艇賊安南總兵範光喜,供述:「阮光平既代黎氏,光平死,傳子光纘,時與舊阮構兵,而軍費又苦不給,其總督陳寶玉招集粵艇肆掠於洋。繼而安南總兵黃文海與賊官伍存七有隙,以二艇投誠於閩,今閩中造船用其式也。倫貴利者,廣東澄海人,投附安南,與舊阮戰有功,封侯。以巡海,私結閩盜來閩、浙劫掠。安南艇七十六艘,分前、中、後支,倫貴利統帶後支。其銅印凡四,貴利自佩其一,餘三印,三總兵曰耀、曰南、曰金者佩之,耀已擒斬,南、金則均溺斃於海」云。巡撫阮元磔貴利,而以供辭入奏。

帝命軍機大臣字寄兩廣總督,照會安南國王。冬十二月,阮光纘呈覆,略曰:「小番世蒙天朝恩庇,曠格逾涯,無能酬報,思以慎守疆宇,永作屏翰。祗以本國極南沿海農耐地初,有賊渠阮種,竊據其地,嘯聚齊桅盜夥,數為海患。本國整飭海防,間收艙客,以離賊黨,且助海面帆柁之役。倫貴利者,前居本國,隨同商伴巡防。詎料伊包藏禍心,私瞞小番,竟敢潛約匪船,越赴內洋,肆行劫掠。又擅造印劄,轉相誑誘,情罪重大,實為法律所不容。小番不能先燭其奸,疏於鈐束。仰蒙聖慈普鑒,洞悉肫誠,訓誨有加,天日垂照。恭繹聖諭,且感且悚。謹當遵奉彞訓,靖守籓封,令本國巡海人員,嚴加警飭,密施鈐勒,斷不容結同匪夥,越境作非,務期桂海永清,以上副聖天子懷柔之至德,是所自勉也。」帝以國王不知,赦之。二年,兩廣總督奏稱,安南國王阮光纘差委官弁丁公雪等,帶領兵船,拿獲盜犯黃柱、陳樂等六十餘名,解送內地。帝降敕褒賜,並頒賜如意、玉山、蟒錦、紗器,以示優獎。

初,阮光平既攻滅廣南王阮某,阮某為黎王婿,妻黎氏有娠,逃於農耐,農耐為水真臘舊都,即嘉定省,今之西貢也。黎氏生子曰阮福映,本名種,潛匿民間。及長,奔暹羅。暹羅王故與阮光平夙仇,乃以女弟歸福映,助之兵,攻克農耐,據之,勢漸強,號「舊阮」,而稱阮光平父子為「新阮」,亦曰「西阮」。舊阮以復仇為辭,奪其富春舊都,時嘉慶四年也。六年十一月,安南偽總兵陳天保攜眷內投,始知安南與農耐兵爭事。七年八月,農耐攻升隆城,阮光纘敗走被擒。八月,阮福映縛送莫觀扶等三名來粵,並獻其攻克富春時所獲阮光纘封冊、金印,奉表投誠。莫觀扶等皆中國盜犯,受安南招往投順,封東海王及總兵偽職者。帝以「從前阮光平款闕內附,恩禮有加,阮光纘嗣服南交,復頒敕命,俾其世守勿替。乃藪奸窩盜,肆毒海洋,負恩反噬,莫此為甚!且印信名器至重,輒行舍棄潛逃,罪無可逭!其命兩廣總督吉慶赴鎮南關備邊,俟阮福映攻復安南全境以聞。」十二月,阮福映滅安南,遣使入貢,備陳構兵始末,為先世黎氏復仇;並言其國本古越裳之地,今兼並安南,不忘世守。乞以「南越」名國。帝諭以「南越」所包甚廣,今兩廣地皆在其內,阮福映全有安南,亦不過交趾故地,不得以「南越」名國。八年,改安南為越南國。六月,命廣西按察使齊布森往封阮福映為越南國王。蓋自阮光平篡黎氏十九年,復滅於阮福映,嗣後修職貢者為舊阮子孫矣。

九年,遣編置佐領及安插江寧、熱河、張家口、奉天、黑龍江、伊犁等處安南人回國,賚銀有差,並許黎維祁歸葬。十一年,越南興化鎮目請以臨安府所屬六猛地方外附,檄諭王自懲之。阮光纘遺族阮如權避捕投內地,兩廣總督吳熊光奏請發交阮福映。帝嫌其為屬籓擒送逋逃,不許,亦不許其逼留內地。十四年,阮福映遣員至諒山,齎送乾隆六十年錫封南掌國王敕印,帝嘉獎之。

阮福映之得國也,藉嘉定、永隆兵力居多,乃取二省為年號,曰嘉隆。在位十七年而薨,子福皎嗣。道光元年,遣廣西按察使潘恭辰齎敕印往封阮福皎為越南國王。九年,越南使臣請改貢道由廣東水路,部議駁之。十九年,帝諭向來越南國二年一貢,四年遣使來朝一次,合兩貢並進,嗣後改為四年遣使來貢一次,其貢物照兩貢並進之數減其半。福皎改元明命,在位二十一年。嘗以兵奪高蠻國河仙一帶地,分通境為三十省:曰富春,國都也;廣南、廣義二省為右圻;廣治、廣平二省為左圻;平順、富安、廣和、邊和、嘉定、安江、河仙、永隆、定祥九省為南圻;河靜、海陽、廣安、清化、乂安、南定、廣平、興安、河內、北寧、諒山、高平、太原、山西、宣光、興化十六省為北圻。後又以廣義、廣治各省過小,改為道。疆域較歷世為大。惟宣光省西北直廣西鎮安府之南,有地曰保樂州,其酋農姓,系黎氏舊臣,仍念故主,不服新王,越南僅羈縻處之。黎維昬子孫逃居老撾深山中,時思聚眾復國,所謂黎王後也。其餘黎氏疏族,好滋事,俱安置平順以南各省。又自鄙其國文教之陋,奏請頒發康熙字典。其取士則用元制,以經義、詩賦考試。

道光二十一年,阮福皎薨,遣使告哀,詔停進貢方物,命廣西按察使寶清往封其子福巿為越南國王。福巿改元紹治,在位七年。道光二十八年,薨,子福時嗣。凡朝使冊封,歷世只在河內。河內即東京,其國建都處也。及阮福映得國,以東京屢毀於兵,而其先人世居嶺南,遂遷都於富春省,改東京為河內省。封使至其國,仍循例駐節於此。阮福時嗣位年幼,奏乞天使至其國都,由是廣西按察使勞崇光至富春冊封焉。

三十年,鄭祖琛奏越南國王阮福時因先後奉到孝和睿皇后,宣宗成皇帝遺詔,擬請遣使恭進香禮,並進香品祭物,又齎遞表文、貢物慶賀登極。帝諭孝和睿皇后、宣宗成皇帝梓宮均已奉移陵寢,止其遠來進香。其慶賀登極方物,亦無庸呈進。咸豐二年,論越南國明年例貢著於咸豐三年五月內到京。六年,諭越南國王阮福時以丁巳年正貢屆期,咨呈勞崇光奏請於何月進關。現在用兵諸省分尚未肅清,越南國此次例貢,著緩至下屆兩貢並進。

八年,法蘭西奪取越南國西貢。先是,明季有法蘭西天主教徒布教來安南。康熙五十九年,法兵艦俄羅地號泊交趾,士官三人登陸至平順省,土人縛而獻之王。艦長與教師商,以重金贖歸。此為法、越交涉之始。乾隆十四年,法王路易十五命皮易甫亞孛爾者為全權大臣,至順化府謀通商,國王不許。乾隆十八年,越人大戮天主教徒。五十一年,越內亂,阮岳自稱王,阮光平使其子景叡詣法國乞援。翌年,遂訂法越同盟之約,割昆侖島之茶麟港於法。未幾,爽約。嘉慶二十五年,法艦來越南測量海口,國人激王殺法人狄亞氏。道光二十七年,法人以兵艦至茶麟港,大敗越軍,至是年遂徑奪西貢,越南第一都會也。

咸豐十年,諭內閣:「劉長佑奏越南國入貢屆期,現在廣西軍務未竣,道路不寧,其丁巳、辛酉兩屆例貢,暫行展緩。」同治元年,法國拿破侖第三以海軍大舉伐越南,奪茶麟港,約割下交趾邊和、嘉定、定祥三省,開通商三口,賠償二千萬佛郎,許其和。嘉定省即西貢所在也。二年,越南國王阮福時因奉到文宗顯皇帝遺詔,咨請遣使進香、表賀登極、貢方物,卻之。三年,越南乙丑例貢及上二屆兩貢仍命展緩。

六年冬,廣西太平、鎮安兩府土匪蜂起,官軍擊之,敗遁越南。七年,國王咨乞廣西巡撫蘇鳳文代奏請兵援剿,帝命提督馮子材率三十營討之。八年七月二十一日,華軍由鎮南關進發。八月,賊酋吳鯤戰北寧,傷於銃,飲孔雀血死,諸賊大懼,大兵至,遂乞降。冬,賊酋梁天錫西奔宣光,投歸河陽賊首黃崇英。是年,法人割取越南國安江、河仙、永隆三省,自是下交趾六省悉隸法版。九年,興化省保勝賊首劉永福、太原省蘇街賊首鄧志雄皆來降。夏四月,黃崇英遁入保樂州白苗界內,提督溤子材班師。

七月,師次龍州,而黃崇英復踞河陽,劉永福復踞興化之保勝,鄧志雄復踞太原之蘇街。十月,降賊蘇國漢乘夜襲陷諒山省城,北圻總統段壽死之。時廣西候補道徐延旭因事至諒山城外驅驢,調兵助越攻城,不克。十一月,賊酋阮四、陸之平、張十一等復踞高平省,越王復懇出師,帝命馮子材再督軍出關,廣西巡撫李福泰請以廣東候補道華廷傑襄辦軍事。十年夏,馮子材次龍州。四月二十一日,總兵劉玉成督諸將出關次北寧。九月,欽州知州陳某誘擒蘇國漢,解送兩廣總督瑞齡,誅之,其子蘇亞鄧遁入海,踞狗頭山。道員華廷傑旋回廣東。十一年,廣西巡撫劉長佑檄道員覃遠璡率勇十營辦太平、鎮安二府邊防,馮子材亦調回防邊。

十二年,華軍將撤,法人突以兵船至河內省。國王咨稱華總兵陳得貴派隊押令放入。劉長佑據情奏聞,朝命革職提訊。法人遂招中國散勇及雲南邊境不逞之徒攻越南各省,其守臣多降。至太原省,守臣招劉永福相助。法兵至,永福設伏敗之,擒其帥安鄴,法人敗退河內省,與王和。王遣其臣阮文祥與議,法人遂建館河內,並於白藤海口設關收稅。初,賊首黃崇英為吳鯤中表,劉永福亦吳鯤之黨。吳鯤死,其弟吳鯨合家自殺。黃崇英、劉永福素不相能,永福降,越南王授以三省提督之職,黃崇英踞河陽為盜自若。十三年,劉長佑遣劉玉成將左軍十營,道員趙沃將右軍十營,由鎮安府出關討黃崇英。是年,法人逼令越南王公布天主教及紅河通航二事,紅河即富良江也。旋又以保商為名,派兵駐守河內、海防諸地,且求開採紅河上流礦山。光緒元年,趙沃連克底定縣、襄安府各處,保樂州土民及白苗皆約降。崇英率眾來拒,旋遁去。趙沃督諸軍攻克河陽老巢,賊黨陳亞水降。七月,擒黃崇英戮之。二年春,班師。

七年,劉長佑移督云、貴,知法人志在得越南以窺滇、粵,上疏略曰:「邊省者,中國之門戶,外籓者,中國之籓籬。籓籬陷則門戶危,門戶危則堂室震。越南為滇、粵之脣齒。泰西諸國,自印度及新加坡、檳榔嶼設立埠頭以來,法國之垂涎越南久矣。開市西貢,據其要害,復通悍賊黃崇英,規取東京,聚兵謀渡洪江以侵諒山諸處,又欲割越南、廣西邊界地六百里為駐兵之所。臣時任廣西巡撫,雖兵疲餉絀,立遣將卒出關往援。法人不悅,訐告通商衙門,謂臣包藏禍心,有意敗盟。賴毅皇帝察臣愚忠,乃得出助剿之師,內外夾擊。越南招用劉永福,以折法將、沙酋之鋒。廣西兩軍,左路則提督劉玉成趨太原、北寧,右路則道員趙沃由興化、宣光分擊賊黨,直抵安邊、河陽,破崇英巢穴,殲其渠魁。故法人寢謀,不敢遽肆吞並者,將逮一紀。然臣每詳詢邊將,知法人之志在必得越南,以窺滇、粵之郊而通楚、蜀之路,狡焉思啟,禍近切膚。乃入秋以來,法國增加越南水師經費,其下議院議借二百五十萬佛郎,經理東京海灣水師。其海軍卿格羅愛逐日籌畫東京兵事,俟突尼斯案一結,即可進行,竊嘆法人果蓄志而潛謀,嗜利而背約也。竊聞造此謀者為伯朗手般,在越南西貢為巡檢司。開埠之後,招入土夷、客民眾至百萬,民情漸洽,物產日增。柬埔寨所招商民,亦逾百萬。運米出洋,歲百萬石,所徵賦稅入西貢庫藏者,歲計佛郎二百五十萬。柬埔本荒藪,開成通衢,車路方軌,溝渠修濬,東埔人感法恩德,至原以六百萬口獻地歸附,故伯朗手般以越南情形告其總統。富良江一帶,法已駛船開市,議上溯以達瀾滄江通中國之貨,結楢方諸夷以窺滇、粵邊境,築西貢至柬埔寨鐵路,以避海道之迂繞。越南四境皆有法人之跡,政治不修,兵賦不足,勢已危如累卵。今復興兵吞噬,加以柬埔之叛民,勢必摧敗不可支拄。同治十三年,法提督僅鳴砲示威,西三省已入於法人之手,而紅海通舟,地險復失。所立條約,惟不肯與以東京,國勢岌岌,恃此為犄角。若復失其東京,即不窮極兵力圖滅富春,已無能自立矣。臣以為法人此舉,志吞全境。既得之後,必請立領事於蒙自等處,以攘山礦金錫之利,或取道川蜀以通江海,據列邦通商口岸之上游。況滇南自同治以後,平定逆回,其餘黨桀黠者,或潛竄越南山谷,或奔洋埠役於法人,軍情虛實,邊地情形,盡行洩漏,故時有夷人闌入滇以觀形勢。儻法覆越南,逆黨又必導之內寇,逞其反噬之謀。臣受任邊防,密邇外寇,不敢聞而不告。」奏入,不報。

時駐英法使臣曾紀澤以越事迭與法廷辨詰,福建巡撫丁日昌亦疏法、越事以聞。帝命與北洋大臣李鴻章籌商辦法,並諭沿江沿海督撫,密為籌辦。八年二月,法人以兵艦由西貢駛至海陽,謀取東京,直督張樹聲以聞,帝諭滇督相機因應。三月,移曾國荃督兩廣。法攻東京,破之,張樹聲奏令滇、粵防軍嚴守城外,以剿辦土匪為名,藉圖進步,並令廣東兵艦出洋遙為聲援。五月,滇督劉長佑遣道員沈壽榕帶兵出境,與廣西官軍連絡聲勢,保護越南。並奏言:「探聞法人破東京後,退駐輪船,日日添兵,增招群盜,懸賞萬金購劉永福,十萬金取保勝州。又法領事破城後,劫掠商政衙門,傳示各商,出入貨稅另有新章,現仍調取陸軍趕造拖船,為西取保勝之計。越王派其兵部侍郎陳廷肅接署河內總督,遣吏部尚書阮正等抵山西與黃佐炎等籌商禦敵之策。各省巡撫、布、按大半與黃佐炎、劉永福同原決一死戰。嗣後統領防軍提督黃桂蘭報稱劉永福馳赴山西,道經諒山,來見。比曉以忠義,感激奮發,據稱分兵赴北寧助守保勝,萬不使法人得逞,但兵力不足,望天朝為援。其河內探報云,法人恐援兵猝至,當釋所獲之河內巡撫,交還城池倉庫。巡撫不受,稱法人違約弄兵,以死自誓,乃轉交按察使。宗室阮霸復以火藥轟毀東京,以免越人復聚,且省兵力分守。其輪船或東下海陽,或分駛廣南、西貢,俟添兵既集,從事上游。伏查法人焚掠東京,狡謀叵測,越南諸臣決計主戰。山西為上通云南要地,越軍能悉力抵御,微特滇、粵邊防可保,即越南大局,亦尚有振興之期。而粵督與總署所議以滇、粵、桂三省兵力合規北圻一策,更可乘勢早圖,以杜窺伺。然越國受制法人已久,人心恇怯,此次決戰山西,期於必勝,稍有撓敗,則大局不堪設想。蓋山西有失,則法人西入三江口,不獨保勝無復障蔽,而滇省自河底江以下,皆須步步設防,益形勞費。以事機而論,中國有萬難坐視之處,且不可待山西有失,始為事後之援。」旋召長佑入覲,以岑毓英署滇督。

劉永福者,廣西上恩州人。咸豐間廣西亂,永福率三百人出鎮南關。時粵人何均昌據保勝,永福逐而去之,遂據保勝,所部旗皆黑色,號「黑旗軍」。永福既立功,越南授三省提督職,時時自備餉械剿匪,而黃佐炎皆匿不上聞,越臣亦多忌之,永福積怨於佐炎。佐炎為越南駙馬,以大學士督師,督撫均受節制。馮子材為廣西提督時,佐炎以事來見,子材坐將臺,令以三跪九叩見,佐炎銜之次骨。越難已深,國王阮福時憤極決戰,責令佐炎督永福出師,六調不至。法軍忌永福,故越王始終倚任之。

先是,劉長佑命籓司唐炯率舊部屯保勝,曾國荃至粵,命提督黃得勝統兵防欽州,提督吳全美率兵輪八艘防北海,廣西防軍提督黃桂蘭、道員趙沃相繼出關,所謂三省合規北圻也。時法人要中國會議越事,諭滇、粵籌畫備議。法使寶海至天津,命北洋大臣會商越南通商分界事宜。吏部主事唐景崧自請赴越南招撫劉永福,帝命發雲南岑毓英差遣。九年正月,景崧乃假道越南入滇,先至粵謁曾國荃,韙其議,資之入越。見永福,為陳三策,言:「越為法逼,亡在旦夕,誠因保勝傳檄而定諸省,請命中國,假以名義,事成則王,此上策也;次則提全師擊河內,驅法人,中國必助之餉,此中策也;如坐守保勝,事敗而投中國,此下策也。」永福曰:「微力不足當上策,中策勉為之。」

三月,法軍破南定。帝諭廣西布政使徐延旭出關會商,黃桂蘭、趙沃籌防。李鴻章丁憂,奪情回北洋大臣任,鴻章懇辭。至是,命鴻章赴廣東督辦越南事宜,粵、滇、桂三省防軍均歸節制。鴻章奏擬赴上海統籌全局。法使寶海在天津議約久不協,奉調回國,以參贊謝滿祿代理。劉永福與法人戰於河內之紙橋,大破法軍,陣斬法將李成利,越王封永福一等男。徐延旭奏留唐景崧防營效用,並陳永福戰績。帝促李鴻章回北洋大臣任,並詢法使脫利古至滬狀,令鴻章定期會議。脫利古詢鴻章:「是否助越?」鴻章仍以邊界、剿匪為辭,而法兵已轉攻順化國都,迫其議約。鴻章與法新使德理議不就,法兵聲言犯粵,廣東戒嚴。總署致法使書,言:「越南久列籓封,歷經中國用兵剿匪,力為保護。今法人侵陵無已,豈能蔑視?倘竟侵我軍駐扎之地,惟有決戰,不能坐視。」帝諭徐延旭飭劉永福相機規復河內,法軍如犯北寧,即令接戰。命滇督增兵防邊,唐蜅迅赴前敵備戰,並濟永福軍餉。旋命岑毓英出關督師。

法兵破越之山西省,粵勢愈急,以彭玉麟為欽差大臣督粵師。彭玉麟奏:「法人逼越南立約,欲中國不預紅河南界之地,及許在雲南蒙自縣通商,顯系圖我滇疆,冀專五金之利。不特滇、粵邊境不能解嚴,即廣東、天津,亦須嚴備。」時越南王阮福時薨,無子,以堂弟嗣。法人乘越新喪,以兵輪攻順化海口,入據都城。越南嗣君在位一月,輔政阮說啟太妃廢之,改立阮福升。至是乞降於法,立約二十七條,其第一條即言中國不得干預越南事,此外政權、利權均歸法人,逼王諭諸將退兵,重在逐劉永福也。

滇撫唐炯屢促永福退兵,永福欲退駐保勝,黑旗將士皆憤怒。副將黃守忠言:「公可退保勝,請以全軍相付,守山西。有功,公居之,罪歸末將。」永福遂不復言退。徐延旭奏言:「越人倉卒議和,有謂因故君未葬權顧目前者,有謂因廢立之嫌,廷臣植黨構禍者。迭接越臣黃佐炎等鈔寄和約,越誠無以保社稷,中國又何以固籓籬?越臣輒以俟葬故君即行翻案為詞,請無撤兵。劉永福仍駐守山西,嗣王阮福升嗣位,具稟告哀,並懇準其遣使詣闕乞封。越國人心渙散,能否自立,尚未可知。」並將法越和約二十七款及越臣黃佐炎來稟錄送軍機處。

兩江總督左宗棠請飭前籓司王德榜募勇赴桂邊扼扎。十一月,法人破興安省,拘巡撫、布政、按察至河內槍斃之。進攻山西,破之,劉團潰,永福退守興化城。十二月,嗣王阮福升暴卒,或云畏法逼自裁,國人立前王阮福時第三繼子為王,輔政阮說之子也。徐延旭奏報山西失守,北寧斷無他虞,帝責其誇張。十年,唐景崧在保勝上樞府書,言:「滇、桂兩軍偶通文報,為日甚遲,聲勢實不易連絡。越南半載之內,三易嗣君,臣庶皇皇,類於無主。欲培其根本以靖亂源,莫如遣師直入順化,扶翼其君,以定人心而清匪黨,敵焰庶幾稍戢,軍事亦易於措手。若不為籓服計,北圻沿邊各省,我不妨直取,以免坐失外人。否則首鼠兩端,未有不歸於敗者也。」

劉永福謁岑毓英於家喻關,毓英極優禮之,編其軍為十二營。法軍將攻北寧,毓英遣景崧率永福全軍赴援。桂軍黃桂蘭、趙沃方守北寧,山西之圍,桂蘭等坐視不救,永福憾之深,景崧力解之,乃赴援。景崧勸桂蘭離城擇隘而守,桂蘭不從。二月,法兵攻扶良,總兵陳得貴乞援,北寧師至,扶良已潰,法兵進逼北寧,黃桂蘭、趙沃敗奔太原,劉永福亦坐視不救。徐延旭老病,與趙沃有舊,偏信之。趙沃庸懦,其將黨敏宣奸,欺蔽延旭。敵犯北寧,敏宣先遁。陳得貴為馮子材舊部;驍勇善戰,子材曾劾延旭,延旭怨之,並怨得貴。及北寧陷,乃奏戮之,敏宣亦正法。延旭調度失宜,帝命革職留任。三月,命湖南巡撫潘鼎新辦廣西關外軍務,接統徐延旭軍,黃桂蘭懼罪仰藥死。帝諭:「徐延旭株守諒山,僅令提督黃桂蘭、道員趙沃駐守北寧,遇敵先潰,殊堪痛恨!徐延旭革職拿問,黃桂蘭、趙沃潰敗情形,交潘鼎新查辦。」以王德榜署廣西提督,德榜辭不拜。唐炯革職拿問,以張凱嵩為雲南巡撫。北寧敗後,徐延旭以唐景崧護軍收集敗殘,申明約束。時唐仁廉署廣西提督。法軍由北寧進據興化,別以兵艦八艘駛入中國海,窺廈門及上海吳淞口,沿海戒嚴,於是中、法和議起。

四月,李鴻章與法總兵福祿諾在天津商訂條款,諭滇、桂防軍候旨進止。鴻章旋以和約五款入告,大略言:「中國南界毗連北圻,法國任保護,不虞侵占。中國應許於毗連北圻之邊界,法、越貨物聽其運銷,將來法與越改約,決不插入傷中國體面之語。」朝旨報可,予鴻章全權畫押。既而法公使以簡明條約法文與漢文不符相詰,帝責鴻章辦理含混,輿論均集矢鴻章,指為「通夷」。法使既藉端廢約,帝令關外整軍嚴防,若彼竟求犯,即與交綏。命岑毓英招劉永福率所部來歸。潘鼎新奏:「法兵分路圖犯穀松、屯梅二處,桂軍械缺糧乏,恐不可恃。」帝以其飾卸,責之。法兵欲巡視諒山,抵觀音橋,桂軍止之,令勿入。法將語無狀,遂互擊,勝之。奏入,諭進規北寧,責法使先行開砲,應認償。令告法外部止法兵,並諭我軍:「如彼不來犯,不宜前進。」法使續請和議,帝諭桂軍回諒山,滇軍回保勝,不得輕開釁。

法將孤拔欲以兵艦擾海疆,法使巴德諾逗留上海,不肯赴津,乃改派曾國荃全權大臣,陳寶琛會辦,邵友濂、劉麟祥隨同辦理。諭言:「兵費、恤款萬不能允。越南須照舊封貢。劉永福一軍,如彼提及,須由我措置。分界應於關外空地作為甌脫。雲南通商應在保勝,不得逾值百抽五。」六月,法將孤拔以兵監八艘窺閩海,欲踞地為質,挾中國議約,何璟、張佩綸以聞。法艦攻臺灣之基隆砲臺,臺撫劉銘傳拒守。曾國荃、陳寶琛與法使議約於上海,國荃許給撫血⼙費五十萬,奉旨申斥。約議久不就,乃一意主戰。諭岑毓英令劉永福先行進兵,規復北圻,岑毓英、潘鼎新關內各軍陸續進發。以法人失和,不告各國。

七月,法公使謝滿祿下旗出京,帝乃宣諭曰:「越南為我封貢之國,二百餘年,載在史冊,中、外咸知。法人先據南圻各省,旋又進據河內,戮其人民,利其土地,奪其賦稅。越南闇懦,私與立約,並未奏聞,挽回無及。越亦有罪,是以姑與包涵,不加詰問。光緒八年,法使寶海在天津與李鴻章議約三條,當與總理各國事務衙門會商妥籌,法人又撤使翻覆。越之山西、北寧等省,為我軍駐扎之地,清查越匪,保護屬籓,與法國絕不相涉。本年二月間,法兵竟來撲犯,當經降旨宣示,正擬派員進取,忽據伊國總兵福祿諾先向中國議和。其時法國因埃及之事岌岌可危,中國明知其勢處迫逼,本可峻詞拒絕,而仍示以大度,許其行成,特命李鴻章與議簡明條約五款,互相畫押。諒山、保勝等軍,應照議於定約三月後調回,迭經諭飭各防軍扼扎原處,不準輕動開釁。諸軍將士,奉令維謹。乃法國不遵定約,忽於閏五月初一、初二等日,以巡邊為名,直撲諒山防營,先行開砲轟擊,我軍始與之接仗,互有殺傷。法人違背條約,無端開釁,傷我官兵,本應以干戈從事。因念訂約通好二十餘年,亦不必因此盡棄前盟,仍準各國總理事務衙門與在京法使往返照會,情喻理曉,至再至三。閏五月二十四日,復明降諭旨,照約撤兵。昭示大信,所以保全和局者,實屬仁至義盡。法人乃竟始終怙飾,橫索兵費,恣意要挾,輒於六月十五日占據臺北基隆山砲臺,經劉銘傳迎剿獲勝。本月初三日,何璟等甫接本領事照會開戰,而法兵已自馬尾先期攻擊,傷壞兵商各船。雖經官軍焚毀法船,擊壞雷艇,並陣斃法國兵官,尚未大加懲創。若再曲予含容,何以伸公論而順人心?用特揭其無理情節,布告天下。」

八月,諭岑毓英督飭劉永福及在防各營規復北圻,並諭潘鼎新飭各軍聯絡聲勢,分路並進。提督蘇元春與法軍戰於陸岸縣,敗之。十月,內閣學士周德潤奏:「官軍進取越南,宜以正兵牽制河內之師,別用奇兵由車里趨老撾,走哀牢,以暗襲順化,募用滇邊土人,必能得力。」得旨交滇督詳察籌辦。是月,蘇元春與法人戰於紙作社,陣斬法兵官四人。十一月,王德榜軍大敗於豐穀,蘇元春不往援,唐景崧與劉永福、丁槐軍攻宣光,力戰大捷,優詔褒之。十二月十九日,法兵攻穀松,王德榜以豐穀之敗怨蘇軍不救,至是亦不往援,蘇軍敗退威坡,諒山戒嚴。帝命馮子材幫辦廣西關外軍務。二十九日,法軍攻諒山,據之,潘鼎新等退駐鎮南關,龍州大震。唐景崧、劉永福、丁槐攻宣光,月餘不能下。諒山失守,岑毓英慮景崧等軍斷後援,令勿拚孤注,景崧不可。馮子材與法軍戰於文淵,互有殺傷。

十一年正月初九日,法兵攻鎮南關,轟毀關門而去,提督楊玉科戰歿。潘鼎新退駐海村,帝命戴罪立功。元春退駐幕府。王德榜自負湘中宿將,屢催援不至,鼎新劾之,落職,所部歸元春轄。法軍攻劉永福於宣光,永福軍潰。唐景崧退駐牧馬,欽、廉防急。彭玉麟請調馮子材軍防粵,朝旨令鼎新議,鼎新素不協於子材,乃命子材行。子材以關外防緊,不肯退,玉麟乃令專顧桂防。鼎新師久無功,褫職,以李秉衡護理廣西巡撫,蘇元春督辦廣西軍務。法兵既毀鎮南關,逃軍難民蔽江而下,廣西全省大震。子材至,乃力為安輯。

子材久駐粵西,素有威惠,桂、越民懷之,人心始定。乃於關內十里之關前隘,跨東西兩嶺間,築長墻三里餘,外掘深塹,為扼守計,自率所部駐之,而令王孝祺勒軍屯其後為犄角。法兵揚言某日犯關,子材逆料其必先期至,乃議先發制敵,鼎新止之,子材力爭,徑率王孝祺軍夜犯敵壘,殺敵甚多。法起諒山之眾撲鎮南關,子材誓眾曰:「法再入關,吾有何面目見粵人?必死拒之!」士氣皆憤。法攻長墻,急砲猛烈,子材勒諸統將屹立接戰,遇退後者手刃之。戰酣,子材自開壁率兩子相榮、相華直沖敵軍,諸軍以子材年七十,奮身陷敵,皆感憤,殊死戰。王孝祺、陳嘉率部將潘瀛、張春發等隨其後,王德榜軍旁至,夾擊之,斃法兵無算。鏖戰兩日,法軍子彈盡,大敗潰遁。子材率兵攻文淵,法軍棄城走。諸軍三路攻諒山,孝祺、德榜戰尤力,連戰皆捷。二月十三日,遂克諒山,法悉眾遁。子材進軍克拉木,逼攻郎甲,王孝祺進軍貴門關,盡復昔年所駐邊地。越民立忠義五大團;二萬餘人,皆建馮軍旗幟。西貢亦聞風通款。自海通以來,中國與外國戰,惟是役大捷,子材之功也。

法兵六千犯臨洮府,復分兩隊;一北趨珂嶺、安平,一南趨緬旺、猛羅。滇督岑毓英命岑毓寶、李應珍等扼北路,王文山扼南路,而自率軍當中路,皆有斬獲。法軍遂合趨臨洮府,滇軍拒戰南北路,回軍夾攻之,陣斬法將五人,法軍大潰。

時法兵艦據臺灣之澎湖。諒山既大捷,法人力介英人赫德向李鴻章議和,言法人交還基隆、澎湖,彼此撤兵,不索兵費。鴻章奏言:「澎湖既失,臺灣必不可保,當藉諒山一勝之威,與締和約,則法不至再事要求。」朝廷納其議,立命停戰。臨洮之戰,乃在停戰後電諭未達前也。鴻章遽請簽約,令諸將皆退還邊界,將士扼腕痛憤,不肯退,彭玉麟、張之洞屢電力爭。帝以津約斷難失信,嚴諭遵辦。法人要求逐劉永福於越南,張之洞乃擬令永福駐思、欽,永福堅不肯行,唐景崧危詞脅之,朝旨嚴切,乃勉歸於粵,授總兵。馮子材奉督辦廉、欽邊防之命。約既成,越南遂歸法國保護焉。


\end{pinyinscope}