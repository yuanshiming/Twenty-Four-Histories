\article{列傳二百六十四}

\begin{pinyinscope}
循吏二

陳?榮芮復傳蔣林閻堯熙王時翔藍鼎元

葉新施昭庭陳慶門周人龍童華黃世發李渭

謝仲?李大本牛運震張甄陶邵大業

周克開鄭基康基淵言如泗周際華汪輝祖茹敦和硃休度

劉大紳吳煥彩紀大奎邵希曾

陳德榮,字廷彥,直隸安州人。康熙五十一年進士,授湖北枝江知縣。修百里洲堤,除轉餉雜派。雍正三年,遷貴州黔西知州,父憂歸。服闋,署威寧府。未幾,威寧改州,補大定知府。烏蒙土司叛,東川、鎮雄附之,德榮赴威寧防守。城陴頹圮,倉猝聚米桶,實土石,比次甃築,墉堞屹然。賊焚牛衛鎮,去城三十里,德榮日夜備戰,賊不敢逼。總兵哈元生援至,賊敗走。尋以母憂去官。服闋,授江西廣饒九南道。九江、大孤兩關錮弊盡革之。

乾隆元年,經略張廣泗疏薦,擢貴州按察使。時群苗交煽,軍事方殷,古州姑盧硃洪文諸叛案,德榮治鞫,詳慎重輕,咸稱其情,眾心始安。及苗疆漸定,駐師與屯將吏多以刻急見能。二年,貴陽大火,德榮謁經略曰:「天意如此,當竭誠修省,苗亦人類,曷可盡殺?」廣泗感動,戒將吏如德榮言。

四年,署布政使,疏言:「黔地山多水足,可以疏土成田。小民難於工本,不能變瘠為腴。山荒尤多,流民思墾,輒見撓阻。桑條肥沃,亦不知蠶繅之法。自非牧民者經營而勸率之,利不可得而興也。今就鄰省雇募種棉、織布、飼蠶、紡績之人,擇地試種,設局教習,轉相仿效,可以有成。應責各道因地制宜,隨時設教。一年必有規模,三年漸期成效。」詔允行。乃給工本,築壩堰,引山泉,治水田,導以蓄洩之法。官署自育蠶,於省城大興寺繅絲織作,使民知其利。六年,疏陳課民樹杉,得六萬株。七年,貴築、貴陽、開州、威寧、餘慶、施秉諸州、縣報墾田至三萬六千畝。開野蠶山場百餘所,比戶機杼相聞。德榮據以入告,數被溫旨嘉獎。又大修城郭、壇廟、學舍。廣置棲流所,收行旅之病者。益囚糧。冬寒,恤老疾嫠孤之無衣者。親課諸生,勖以為己之學。設義學二十四於苗疆,風氣丕變。十一年,遷安徽布政使,賑鳳、潁水災,流移獲安。十二年,卒於官。

德榮在貴州興蠶桑,為百世之利。時遵義知府陳玉畐,山東歷城人,到郡見多檞樹,土人取為薪炭。玉畐曰:「此青萊樹也,吾得以富吾民矣。」乃購歷城山蠶種,兼以蠶師來,試育五年,而蠶大熟,獲繭八百萬,自是遵綢之名大著。正安州吏目徐階平,亦自浙江購繭種,仿玉畐行之正安,亦大食其利。遵義鄭珍著樗繭譜,以傳玉畐遺法。

芮復傳,字衣亭,順天寶坻人,原籍江蘇溧陽。康熙四十八年進士,授浙江錢塘知縣。悉除諸無名錢,曰:「官足給饔飧而已。」有金三者,交通上官署,為奸利,立逮杖斃之,一時大快。五十八年,大旱,復傳勘實上狀,上官欲寢之,固爭曰:「律有捏災、匿災並當劾,某今日請受捏災罪。」時同城仁和民千人,跣走圍署,曰:「錢塘為民父母,仁和獨不父母我耶?」上官感動,竟以災聞。開倉行賑,復傳設粥廠二十有七。微行覘視,治胥吏之侵擾者,帑不費而賑溥。駐防營卒馳躪民田,便宜懲治,輒縛而鞭之。

治績上聞,世宗特召引見,擢溫州知府。故事,貢柑,歲期至。織造封園,民以為累。復傳第取足供貢,不使擾民。府境私鹽充斥,設三團,集灶戶,平其直,私販息,官鹽不督自行。天臺山東南有山曰玉環,在海中,總督李衛欲開田設治,檄復傳往勘,以徒費無益,陳請罷之。衛怒,檄他吏往,意必行。時山中田僅二萬畝,乃割天臺、樂清兩縣民田隸玉環,經費不足,則捐通省官俸,又加關津一切雜稅以給之。弛山禁,漁者往來並稅,曰塗稅。既而漁者不入,山者度關納稅,亦徵其塗稅。復傳爭曰「是重稅也」,是牘凡七上。衛益怒,以為阻撓玉環墾田事,蜚語頗聞。劉統勛奉使視海塘,過溫州,語之曰:「君與李宮保,兩雄不相下,不移不屈,君之謂乎?」

尋擢溫處道。會銅商積弊敗露,復傳持法,又揭劾知府尹士份不職,士份反誣以阻商誤銅,大吏故嫉之,遂並劾復傳。解任,總督趙弘恩質訊,坐失察關吏舞弊奪職。會高宗登極,詔仍留浙江辦銅,事竣,例得復官,以親喪歸,遂不出。家居三十餘年,卒,年九十有四。

蔣林,字元楚,廣西全州人。康熙五十四年進士,選庶吉士,授檢討。直南書房,十年不遷。大將軍年羹堯欲闢為幕僚,林急告歸。尋調戶部郎中,出為福建邵武知府,以事解職,詔發浙江,歷杭州、嚴州、金華三府。在杭州,值織造隆升建議改海門尖山海口,別開河以固海塘。林極言不可,曰:「能使海不潮,則役可興。否則勞民傷財,萬無成理。」上書督撫,俱不省。雍正十二年三月二十五日夜,牒下,索杭夫萬五千人,合旁郡無慮數萬人。期三日集海上。林又爭曰:「田蠶方亟,期會迫,萬一勿戢,奈何?必不得已,俟蠶功畢。」隆升怒,督益急,以抗旨脅之。四月,送役往,面詰以工不可成狀。隆升益怒,留林督役以困之。冒雨撫循,泥深沒脛,役人感其誠,咸盡力。隆升復虐使,動以捶撻,眾屢譁噪。微林,事幾殆。役迄無成,隆升得罪去。乾隆初,召至京,入對,即日擢長蘆鹽運使。曩時院司歲各費數萬緡,林率以儉,歲費百緡而已,羨餘悉歸公。居四年,以親老乞養。高宗曰:「世乃有不原久為長蘆運使者耶?」久之,卒於家。

閻堯熙,字涑陽,河南夏邑人,原籍山西太原。康熙四十五年進士,五十二年,授直隸槁城知縣。滹沱常以秋溢,築堤樹木椿,以捍其沖,夾岸種柳,堤固,水不為患。雍正元年,調南宮,擢晉州知州。州瀕滹沱河,河決徙道,蕩析民居。堯熙為籌安集,民免於患,扶攜老稚來謝。堯熙曰:「此朝廷恩,我何與?」令望闕拜,人給百錢,以資裹糧,散錢十萬,咸感泣曰:「真父母也!」怡賢親王奉使過境,聞其名,奏循良第一。擢山東青州知府,未之官,改授浙江嘉興。俗健訟,良懦不得直。訟府,下縣,或不理,奸猾益無忌。堯熙始至,日受狀三百。比對簿,自請息者二百餘,庭折數十,各得其情。豪民張某稔惡,訊實,杖殺之,民皆稱快。屬縣賦重,名目糾紛,里胥因緣為奸。民完如額,官不知,民亦不自知,官累以缺賦課殿去。堯熙巡行清理,民始知額,歲無逋賦。

海鹽縣塘工不就,總督李衛聽浮言,欲開引河洩潮。堯熙言:「滷水入內河,田皆傷,非等壞廬舍、糜帑金已也。」議遂罷。營弁緝私鹽,縱其梟,持他人抵罪。堯熙言其誣,總督不聽,庭爭再三,總督乃自勘,釋之,愈以賢堯熙。累擢湖北按察使、四川布政使,皆持大體,有惠政。乾隆七年,卒於官。

堯熙質直,好面折人過,雖上官不少避。然勇於從善,在川籓多得成都知府王時翔之助,人兩賢之。

時翔,字皋謨,江蘇鎮洋人。為諸生,績學未遇。雍正六年,世宗重選守令,命中外官各舉一人,同州人沈起元,官興化知府,以時翔應詔,即授福建晉江知縣。時福建吏治頹廢,遣使按視,多更諸守令有司,頗尚操切。晉江民好訟,時翔至,曰:「此吾赤子,忍以盜賊視乎?」一以寬和為治。坐堂皇,呴呴作家人語。曲直既判,令兩造釋忿,相對揖,由是訟者日衰。觀風整俗使劉師恕按泉州,委時翔鞫疑獄二十餘事,語人曰:「晉江長者,決獄又何精敏也!」尋調政和,又調甌寧。

擢漳州府同知,駐南勝。南勝民族居峒中,多械斗。有賴唱者,糾眾奪犯,匿險自固。時翔親入山諭之曰:「汝諸賴萬人,奈何庇一人而以死殉耶?為我縛唱來即無事。」唱不得已自縛出,始如律。瀨子坑民葉揚煽亂,時翔謂緩之可一紙定,或張其事,大吏檄入山剿之。事平,意不自得,乞病歸。

乾隆元年,以薦起山西蒲州府同知,擢成都知府。以廉率屬,善審機要。錢價騰,布政使榜平其直,市大譁。時翔方在假,召成都、華陽二令曰:「市直當順民情,抑之,錢閉不出,奈何?」言於布政撤其榜,錢價尋平。

議徙涼州兵於成都,拓駐防城,當奪民居二千家。時翔檢故牘,請曰:「城故容兵三千,現兵一千五百,尚虛其半。第出現所侵地足矣,奚拓為?」已而涼州兵亦不果徙。成都當康熙時,人稀穀賤,旗兵利得銀。至雍正以後,生聚多,穀貴,又原得穀。或徇其意,令民受銀,購穀給兵。未幾,漢兵亦欲仿行,時翔曰:「旗兵例不出城,語言與土人殊,故代購。漢兵皆土著,奚代為?」二事亦賴布政力主其議得止。

至七年,江南、湖廣災,巡撫奏運蜀米四十萬石濟之。湖廣急米,來領運,江南則否。巡撫乃檄下縣餽運,舳艫蔽江,商賈不通,成都薪炭俱絕。時翔謂江南運可緩,徒病蜀。請獨運楚,而聽商人自運江南。時堯熙既沒,竟無用其言者。時翔在成都,屢雪疑獄,時稱神明。九年,卒。

藍鼎元,字玉霖,福建漳浦人。少孤力學,通達治體,嘗泛海考求閩、浙形勢。巡撫張伯行器之,曰:「藍生經世之良材,吾道之羽翼也。」

康熙六十年,臺灣硃一貴倡亂,鼎元從兄南澳鎮總兵廷珍率師進討,多出贊畫,七日臺灣平。復從廷珍招降人,殄遺孽,撫流民,綏番社,歲餘始返。著論言治臺之策,大意謂:「土地有日闢、無日蹙,經營疆理,則為戶口貢賦之區;廢置空虛,則為盜賊倡亂之所。山高地肥,最利墾闢。利之所在,人所必趨。不歸之民,則歸之番與賊。即使內亂不生,寇自外來,將有日本、荷蘭之患,不可不早為措置。」時議者謂臺灣鎮當移澎湖,鼎元力言不可,大吏採其說,見諸施行。鼎元復為臺灣道條十九事,曰「信賞罰、懲訟師、除草竊、治客民、禁惡俗、儆吏胥、革規例、崇節儉、正婚嫁、興學校、修武備、嚴守御、教樹畜、寬租賦、行墾田、復官莊、恤澎民、撫土番、招生番。」後之治臺者,多以為法。

雍正元年,以選拔入京師,分修一統志。六年,大學士硃軾薦之,引見,奏陳時務六事,世宗善之。尋授廣東普寧知縣,在官有惠政,聽斷如神。集邑士秀異者講明正學,風俗一變。調權潮陽縣事,歲薦饑,多逋賦,減耗糧,除苛累,民爭趨納。妖女林妙貴惑眾,寘之法。籍其居,建棉陽書院。以忤監司罷職,總督鄂彌達疏白其誣,徵詣闕。逾年,命署廣州知府,抵官一月,卒。

鼎元尤善治盜及訟師,多置耳目,劾捕不稍貸,而斷獄多所平反,論者以為嚴而不殘。志在經世,而不竟其用。著鹿洲集、東征集、平臺紀略、棉陽學準、鹿洲公案傳於世。

葉新,字惟一,浙江金華人。康熙五十一年,順天舉人。從蠡縣李恭受業,立日譜自檢,尤嚴義利之辨。雍正五年,以知縣揀發四川,授仁壽縣。有與鄰縣爭地界者,當會勘,鄉保因閽人以賄請,新怒,悉下之獄。勘畢,各按其罪,由是吏民斂手奉法。

署嘉定州,故有沒水田,多逋賦。新視曠土可耕者,召民墾闢,以新科抵賦額,舊逋悉免。時仁壽採木,部匠倚官為暴,民勿堪,糾眾相抗,縣以變告,檄新往治之,抵匠頭及首糾眾者於法,餘釋不問。遷工⼙州知州,再遷夔州府同知,署龍安及成都知府。又署瀘州知州,訟者至,立剖決,滯獄一空。治瀘兩載,俗一變焉。新自授夔州同知,閱五載,始一蒞任。尋又署保寧、順慶兩府,擢雅州知府,母憂歸。

乾隆十年,服闋,補江西建昌。修盱江書院,招引文士與講論學術。復南城黃孝子祠,以勵民俗。十三年,南豐令報縣民饒令德謀反,令德好拳勇,令以風聞遣役往偵,誤探其仇,謂謀反有據,遂往逮令德,適他往,乃逮其弟系獄。令德歸,自詣縣,受刑誣服,雜引親故及鄰境知識為同謀,追捕蔓及旁郡。新得報,集諸囚親鞫,株連者已七十餘人,言人人殊。新詰縣役捕令德弟狀,役言初至其家,獲一篋,疑有金寶匿之。及發視,無所有,棄之野。令聞,意篋有反跡,訊以刑。妄稱發篋得簿劄,納賄毀之矣。令謂實然,遂逼令德誣服。新於是盡釋七十餘人縲線,命隨往南昌。戒之曰:「有一逋者,吾代汝死矣。」及至,七十餘人則皆在。謁巡撫,具道所以,巡撫愕不信,集才能之吏會勘,益雜逮諸所牽引,卒無據,而巡撫已於得報時遽上奏。朝命兩江總督委官就讞,新為一一剖解得白,所全活二百餘人。

十七年,調贛州,有贛縣搶奪拒捕之獄,值改例,新舊輕重懸殊。新謂事在例前,當依舊比,爭之不得。復以寧都民獄事,與同官持異同,不得直,謝事閉門候代。上官慰喻,不從,遂以任性被劾免歸。欣然曰:「今而後可無疚於心矣!」家居十餘年,卒。

施昭庭,字筠瞻,江蘇吳縣人。康熙五十四年進士,授江西萬載知縣。地僻多山,客民自閩、粵來,居之累年,積三萬餘人,號曰「棚民」。溫尚貴者,臺灣逸盜也,亦處山中。雍正元年,福建移捕盜黨急,尚貴謀為變。始昭庭之至也,以棚民為慮,厚禮縣人易廉野使偵之。廉野積粟貸棚民,不取息,或免償,得棚民心。其才者嚴林生、羅老滿,從廉野游,盡得山中要領。尚貴將舉事,廉野以聞,昭庭、林生、老滿率勇敢三百人待之。尚貴有眾二千肆掠,昭庭曰:「賊易破也,然慮其擾傍縣。」撫賊諜使誑尚貴趨萬載。乃張疑兵於山徑,賊不敢入,由官道來。預設伏叢棘中,伺賊過,突出擊殺。賊數中伏,疑駭,逆擊之,一戰獲尚貴。尚貴起二日而敗,又二日而撫標兵至。

初,棚民與市人積嫌,事起,道路洶洶,指目棚民。昭庭以免死帖與諸降者,取棚民不從賊者結狀,兵至搜山,不戮一人。巡撫初到官,張其事入奏,既見縣申狀不合,欲改之,昭庭不可。又謂棚民匿盜從亂,今雖赦之,必驅歸本籍。昭庭曰:「棚民種植自給,非刀手老瓜賊之比。歷年多,生齒眾,間與居民爭訐細故,不必深懲。今亂由臺灣逸盜,而平盜悉資棚民。」力請:「覈戶口,編保甲,泯其主客之形,寬其衣食之路,長治久安,於計便。」總督查弼納許之,巡撫尋亦悟,如昭庭策,棚民乃安。事聞,世宗諭九卿曰:「知縣以數年心力辦賊,巡撫到官幾日,豈得有其功耶?」獨下總督疏,議敘,以主事知州用。尋引疾歸,卒於家。

陳慶門,字容駟,陜西盩厔人。雍正元年進士。從鄠王心敬講學,養親不仕。母王趣之,乃謁選。七年,授安徽廬江知縣,修建文廟,規制悉備。大濬城壕。置義田二百畝有奇。贍養煢獨,立社倉四所,積穀以貸平民。縣民舊習,止知平疇種稻,高阜皆為棄壤。因市牛具,仿北方種植法,躬督墾闢,遂享其利。

尋署無為州事。州瀕江,上下二百里,率當水沖,前人築壩四,常沒於水。慶門於鮑魚橋、匋魚口二處,樹椿編竹,實土為坦坡;又取亂石填擲水中,水停沙淤,久而成洲,民免墊溺之患。又署六安州,舊有水塘,議者欲墾塘以為田,將絕灌溉之利。慶門力言於上官,事乃寢。

十一年,擢亳州知府,俗悍,好群鬥,倚蠹役,表裏為奸。慶門廉得其魁黨,先後杖遣數百人。又好訟,仿古鄉約法,使之宣導排解。勤於聽斷,日決數十事。不數月,澆風一變。州瀕湖,地窪下,用秦中收澱之犁法,督民挑濬,地下者漸高,水歸其壑,農田賴焉。母憂歸。

乾隆元年,服闋,以大臣薦,補四川達州知州。境環萬山,歲常苦旱,教民種旱稻,始無艱食之憂。鄰郡巴州,桑柘素饒,乃買桑遍植,教以分繭繅絲之法,獲利與巴州等。時川東多流民,官廩不給,遂釐剔腴田之被隱占者,為義產以贍之,全活甚眾。建宣漢書院,聘名流教授,文風漸振。未幾,乞病歸。著仕學一貫錄,世以儒吏稱之。

周人龍,字雲上,直隸天津人。康熙四十八年進士,授山西屯留知縣。興學賑荒,有聲。調清源,境內洞渦、嶑峪諸河入汾,常有水患,濬渠築堰,民賴之。歷忻州直隸州知州、蒲州知府。蒲郡瀕黃河,河水遷徙無常。山、陜兩省民隔河爭地,訟數十年不結。人龍請於大吏曰:「臨河灘地,當以河為界。河東遷,則山西無地之糧歸陜西;河西遷,則陜西無地之糧歸山西。糧隨地起,不缺正賦。因地納糧,無累民生。山、陜沿河二千餘里,凡兩省湮沒之地,令地方官照糧查地,按地過糧。除鹵咸者照例題請免徵,其餘水退之地,招令沿河民認糧承種,庶事無偏枯,爭訟可息。」大吏從其議,至今便之。

雍正初,有言丁糧歸地,便於無力之丁,不便於有田之家。人龍駁之,略曰:「有田者,尚以輸納為艱,豈無田者反易?君子平其政,焉得人人而悅之?今不悅者,不過紳衿富戶;而大悅者,乃在煢煢無告之小民。若因其控告而不行,則豪強得志,而窮民終於無告。此議在當日未行則已耳,今行之數年,勢難中止。窮民狃於數年樂利,必不安於一旦變更。且富民少而窮民多,不當以彼易此。」議上,事乃定。以憂去官。

服闋,補湖北安陸。數月,擢江西督糧道,未行,江水決鍾祥三官廟堤及天門沙溝垸,招集鄰縣民,諭以利害,同築御。踴躍荷畚鍤至者數萬人,親冒風雨,率以施工。或勸其「已遷官,何自苦」,人龍曰:「助夫由我招至,我去即散矣。伏汛一至,民何以堪?」閱兩月工成,安陸人尸祝之。江西漕糧徵運素多弊,嚴立規條,宿蠹一清。乾隆十年,乞病歸,卒。

童華,字心樸,浙江山陰人。年未冠為諸生,長習名法家言,出佐郡邑治。雍正初,入貲為知縣。時方修律例,大學士硃軾薦其才,世宗召見,命察賑直隸。樂亭、盧龍兩縣報饑口不實,華倍增其數。怡賢親王與硃軾治營田水利,至永平,問灤河形勢,華對甚晰,王器之。尋授平山知縣,邑災,不待報,遽出倉粟七千石貸民。擢真定知府,權按察使。以前在平山發粟事,部議免官,特詔原之。

怡賢親王奏以華理京南局水利,華度真定城外得泉十八,疏為渠,溉田六百畝,先後營田三百餘頃。滏陽河發源磁州,州民欲獨擅其利。自春徂秋,閉閘蓄水,下游永年、曲周滴涓不得。時改州歸直隸,以便控制。華建議仿唐李泌、明湯紹恩西湖三江兩閘遺規,計板放水,數縣爭水之端永息。華又以北人不食稻,請發錢買水田穀運通倉,省漕費,民得市稷黍以為食,從之。

調江蘇蘇州,會清查康熙五十一年以來江蘇負課千二百餘萬,巡撫督責急,逮捕追比無虛日,華固請寬之。巡撫怒曰:「汝敢逆旨耶?」對曰:「華非逆旨,乃遵旨也,上知有積欠,不命嚴追而命清查,正欲晰其來歷,查其委曲,或在官,或在役,或在民,或應徵,或應免,了然分曉,奏請上裁,乃稱詔書意。今奉行者不顧名思義,徒以十五年積欠立求完納,是暴徵,非清查也。今請寬三月限,當部居別白,分牒以報。」巡撫從其請,乃盡釋獄系千餘人,次第造冊請奏。時朝廷亦聞江南清查不善,下詔切責,如華言。

浙江總督李衛嘗捕人於蘇,華以無牒不與,衛怒,蜚語上聞。世宗召見,責以沽名干譽,對曰:「臣竭力為國,近沽名;實心為民,近干譽。」上意解,命往陜西以知府用。署肅州,佐經略鄂爾泰屯田事,鑿通九家窯五山,引水穿渠,溉田萬頃。以忤巡撫被劾罷官。乾隆元年,起福州知府,調漳州。頗好長生術,招集方士,習丹家言,復劾罷歸。數年卒。

華剛而忤時,屢起屢蹶。在蘇州,民德之尤深,以比明知府況鍾。當世宗治畿輔營田時,所用者多一時賢守令,有黃世發,名與華相媲云。

世發,字成憲,貴州印江人。康熙三十五年舉人,授直隸肅寧知縣。舊例,錢糧加一二作耗銀,世發亦收之而不自用,雜派畝銀三四錢悉除之。縣有役事,若修學校、繕城垣及上官別有攤派,即以耗銀應。河間府檄修府城,親齎餱糧,出錢雇役,不以擾社甲。視民如家人,教以生計。坑鹼荒地,令穿井耕種。綠城植桑柳樹萬株,凡水車、蠶箔、糞灌、紡績,悉為經畫。復闢護城廢地,穿池種稻以導之。建社學,教以孝親敬長,贖官田九十餘畝,以其租為學者膏火。旬三日集諸生講學會文,士有自鄰縣來學者。雍正三年,水災,大吏遣官履勘,世發不能得其意,被劾罷。士民呼籥挽留,特詔復官,加四品銜。已,晉授按察使兼直隸營田觀察使,巡行勸民農桑,察水利可興者。所至剴切宣諭,民多興起。修堤墾田,變汙下為沃壤。最後開易州水峪田,經營年餘,以勞卒。

李渭,字菉涯,直隸高邑人。父兆齡,康熙中官福建閩清知縣,以廉能稱。渭,康熙六十年進士,授內閣中書,遷刑部主事。雍正二年,出為湖南岳州知府,詔許密摺奏事。忤大吏,左遷武昌府同知,未之任,丁母憂。服闋,授四川嘉定知府,復以爭冤獄忤上官。渭曰:「吾官可棄,殺人媚人不為也。」奉檄賑重慶水災,多所全活。父憂歸。

後補河南彰德,萬金渠源出善應山,環府城,入洹河,灌田千數百頃,山水暴發易淤。渭履勘濬治,增開支河,建閘啟閉,定各村分日用水,歲以有秋。漳河當孔道,舊設草橋於臨漳,道回遠,移於豐樂鎮,行旅便之。雪武安民班某誣殺族兄獄。林縣富室毆人死,賂尸屬以病死報。渭驗尸腿骨盡碎,治如律。舉卓異。

乾隆九年,擢山東鹽運使,時議增鹽引,渭以增引則商不能賠,必增鹽價,商、民且兩病,持不可。十二年,山東大水,大吏檄渭勘災,至益都、博興、樂安諸縣,餓莩載途,而有司先以未成災報,已入告,難之;乃請以借作賑,異日免追,民乃蘇。十三年,就遷按察使,折獄平。嘗曰:「古人言求其生而不得,今俗吏移易獄詞,何求生不得之有?然如死者何!此婦寺之仁,非持法之正。」

尋遷安徽布政使,禁革徵糧長單差催法,以杜詭寄。調山東,墾荒,令客民帶完舊欠,免鄰保代賠逃戶之累,民便之。為政持大體,不吝出納,不輕揭一官,馭吏嚴而不念舊過。十九年,卒於官。子經芳,乾隆中官至湖北施南知府,亦廉謹守其家風。

謝仲坃,字孔六,廣東陽春人。雍正元年舉人,登明通榜。初官長寧教諭,乾隆初,擢授湖南常寧知縣,峻卻餽遺。履鄉自裹行糧,嚼生萊菔供饌。月兩課士,以節行相勸勉。調平江,再調衡陽。前令李澎徵漕米浮收斛面,糧儲道謝濟世發其奸。時巡撫許容方以浮收誣劾濟世,總督孫嘉淦亦徇巡撫意,故濟世與澎並免。言官論奏,朝命侍郎阿里袞往按。署糧道倉德又因布政使函囑改換衡陽浮收詳文,據以上揭,詔責切究。事急,澎則盡出賄贈簿以脅上官,阿里袞重興大獄,欲出澎浮收罪,與濟世俱復官。仲坃乃重治澎丁役,以決罰過當被劾罷官。逾年,特起為衡山知縣。以讞巴陵獄,巡撫與按察使互奏,奉旨引見,擢荊州府通判。又以歸州縱盜冤良之獄,自巡撫按察以下皆被重譴,仲坃承審時,堅不會印,特旨召對。擢常德府同知,歷署襄陽、寶慶、宜昌、武昌、永順、岳州、永州七府知府,護衡永郴桂道。正躬率屬,屏絕請託,暇輒延耆士論學不倦。

仲坃官湖南先後三十年,長於折獄,大吏倚重。歷奉檄鞫獄二百餘,多所平反,以直戇名。乾隆三十七年,在永州議改淮引食粵鹽,格於例不行,遂以目疾請告。解組日,貧如故,卒於家。

李大本,字立齋,山東安丘人。雍正十三年舉人。乾隆九年,銓授湖北棗陽知縣,改湖南益陽。居官自奉儉約,勤於吏事。益陽人不知蠶,大本教之樹桑,後賴其利。調長沙,遷寶慶府理瑤同知。所隸通水峒有苗僧行賈臨桂,知縣田志隆見之,意為賊黨。吳方曙者,從馬朝桂謀叛,時方繪圖懸購者也。僧畏刑誣服,又訊朝桂所在,妄言在峒中。廣西巡撫定長立上奏,率兵出,命大本從行。大本曰:「僧言真偽不可知,大兵猝至,苗必駭,且生變,請潛訪之。」既而白僧言實妄,巡撫疑未釋,復欲以兵往,大本力諫乃止。後廷訊苗僧果誣如大本言。

橫嶺峒苗乏食,籥官求粟,大本多方賑之。復為苗民籌生計,請於上官曰:「橫嶺峒自逆渠授首,安插餘苗,因惡其人,故薄其產,每口授田才三十欑至四十欑。每欑上田穫米六升,中田五升,下田四升,得米無多。又峒田稍腴者盡與堡卒,極惡者方畀苗民,歲入不足,男則斫柴易米,女則★L9蕨為粉,給口食。年來生齒日繁,材木竭,米價益昂,饑餓愁嘆,深可憐憫,恐不可坐視而不為之所。現有入官苗田一千三百四十八畝,舊募漢民佃種,出租供餉,奸良不一,屢經淘汰。請視苗民家貧丁眾者書諸簿,有漢佃應除者,即書簿之苗丁次第受種,出租如故,則苗民得食而餉亦無虧,乃補救之一端。」議上,不許。後巡撫陳宏謀見之,曰:「此識時務之言也。」將陳其事,會他遷,未果。二十一年,題請升授知府,因病足歸,卒於家。

牛運震,字階平,山東滋陽人。雍正十一年進士。乾隆元年,召試博學鴻詞,不遇。尋授甘肅秦安知縣,開九渠,溉田萬畝。縣北玉鍾峽山崩塞河,水溢為災。運震率丁夫開濬,凡四日夜,水退。緣山步行,以錢米給災戶。縣聚曰西固,去治二百餘里,輸糧苦運艱,多積逋。運震許以銀代納,民便之。先是巡檢某誣馬得才兄弟五人為盜,前令弗察,得才自刎死。其兄馬都上控,令又誘而斃之獄。其三人者將解府,運震鞫得其情,昭雪之。又清水縣某令冤武生杜其陶父子謀殺罪,上官檄運震覆治,驗死者得自刎狀,以移尸罪其陶而釋其子。他訟獄多所平反。

官秦安八年,惠農通商。暇則行視郊野,鑄農具,教民耕耨。稱貸販褐戶,不求其息。設隴川書院,日與諸生講習,民始向學。兼攝徽縣,又攝兩當縣,舍於三縣之中,曰大門鎮,以聽訟。徽縣多虎,募壯士殺虎二十六,道始通。調平番,值縣境五道峴告饉,捐粟二百石以賑,民感之。人輸一錢,制衣銘德,運震受衣返幣。固原兵變四掠,督撫皆至涼州,檄召運震問方略。運震請勿以兵往,但屯城外為聲援,令城內捋出亂者。游擊某執三百餘人,眾忷懼,運震請釋無辜,入城慰喻。斬三人,監候四人,餘予杖徒有差,反側遂安。有忌者摭前受萬民衣事,劾免官。貧不能歸,留主皋蘭書院,教學得士心。及歸,有走千里送至灞橋者。

運震居官,不假手幕下,事輒自治。所至嚴行保甲,鬥爭訟獄日即於少。遇人乾訟,必嚴懲。治盜尤嚴,曰:「邊鄙風俗疵悍,不如此,則法不立;令不行,民不可得而治。且與其輕刑十人,不如重處一人而九人畏,是懲一而恕九也。」罷官歸後,閉門治經,搜考金石,所著經義、史論、文集及金石圖,皆行於世。嘗主晉陽、河東兩書院,所造多名雋士,世稱「空山先生」。

張甄陶,字希周,福建福清人。舉鴻博,補試未合格罷。大學士硃軾、侍郎方苞薦修三禮,辭,而請受業於苞。乾隆十年,成進士。時方許極言直諫,甄陶對策,困極陳時務。選庶吉士,授編修,尋改授廣東鶴山知縣。歷香山、新會、高要、揭陽,皆劇邑,所至有聲。疆田疇,修堤圩,弛戶蠔蜆之禁,增建書院、社倉,平反冤獄,詰捕盜賊,為政務無怫逆於民。以憂去官,服除,起授雲南昆明,弗獲於上官,坐事免。主講五華書院,尹壯圖、錢灃皆其弟子。復移掌貴州貴山書院,課士有法。總督劉藻疏薦,詔加國子監司業銜。晚以病歸閩,主鼇峰書院。以經義教閩士,於是咸通漢、唐注疏之學。在滇時著經解百餘卷。方甄陶之補外,人咸惜之。大學士陳世倌贈以明呂坤呻吟語,甄陶讀其實政錄而慕之,在粵作學實政錄,見其書者,咸曰:「循吏之言也。」

邵大業,字在中,順天大興人,舊籍浙江餘姚。雍正十一年進士,乾隆元年,授湖北黃陂知縣。初到官,投訟牒者坌至,不移晷,決遣立盡。吏人一見問姓名,後無不識,眾莫敢弄以事。有兄弟爭產訟,皆頒白,貌相類。令以鏡鏡面,問曰:「類乎?」曰:「類。」則進與為家人語曰:「吾新喪弟,獨不得如爾兩人白首相保也。」二人感動罷去。蛟水壞城,當壞處立,誓以殉,水驟止,拯溺餔饑,完堤岸,民得免患。總督以其名上聞,會父憂去。

服闋,授河南禹州知州,調睢州。頻澇,請糶請賑,民以免患。濬惠濟河,以俸錢更直,擢江南蘇州知府。松江盜獄久不決,株連瘐斃者眾,奉檄鞫治。見群犯皆斷脛折踝,蹙然曰:「爾等亦人子,迫饑寒至此,猶茹刑顛倒首從,誣連非罪人,何益於爾?」有盜幡然曰:「官以人類待我,我不忍欺。」獄辭立具。

兼署蘇松太道,尋攝布政使事,大吏交章薦。十六年,高宗南巡,御舟左右挽行,名嘏須纖。大業語從臣,除道增纖必病民,非所以宣上德意,遂改單纖。會積雨,治吳江帳殿未就,總督劾大業觀望。及乘輿至,則供備已具,然大業卒因左遷。

尋授河南開封知府,屬縣封丘民被控侵占田畝,及勘丈,非侵占,而畝浮於額。大業考志乘,河南賦則,自明萬歷改並,中地十畝,作上地七畝;下地十畝,作上地三畝。上官以昔為下則,今則膏腴,議加賦。大業曰:「此河沖淤積,百姓以墳墓田廬所易之微利也。今日為退灘淤地,異日即可為沙壓水沖。冬春播種,夏秋之收穫不可知。上年河決,屋宇未盡葺,流亡未盡復,遽增歲額,何以堪?」旋從部議試種三年,次年果沒入水,乃止。未幾,以河溢,降江南六安州知州,又以盜案鐫級。引見,再還江南,署江寧府。

二十八年,授徐州知府,府城三面瀕黃河,西北隅尤當沖,雖有重堤,恃韓家山埽為固。大業按視得蘇公舊堤,起城西雲龍山,迄城北月堤,長三里,湮為民居,復其舊。越歲,韓家山埽幾潰,民恃此堤以無恐。復濬荊山橋河,於水利宣洩,規畫盡善。治徐七年,間有水患,不病民。三十四年,坐妖匪割辮事罷職,謫戍軍臺,數年卒。

大業所至以勸學為務,因黃陂二程子祠建義學,葺睢州洛學書院,集諸生親為之師焉。

周克開,字乾三,湖南長沙人。乾隆十二年舉人。十九年,以明通榜授甘肅隴西知縣。調寧朔,縣屬寧夏府,並河有三渠,曰漢來、唐延、大清,皆引河入渠灌田。唐延渠所經地多沙易漫,克開治之使深狹,又頗改其水道,渠行得安。渠有石竇,洩水於河,以備旱澇,民謂之暗洞。時暗洞崩塞,渠水不行,上官欲填暗洞而竭唐延入漢來,以便寧夏縣之引河,寧夏利而寧朔必病。克開恐夏、秋水盛無所宣洩,時新水將至,不可待。克開請五日為期,取故渠及廢閘之石,晝夜督工,五日而暗洞復,兩縣皆利。大清渠長三十餘里,鑿自康熙間,久而石門首尾壞,民失其利,克開亦修之,皆費省而工速。再以卓異薦,擢固原知州,父憂去。服闋,補洮州。

尋擢貴州都勻知府。從總督吳達善、侍郎錢維城治貴州逆苗獄,用法有失當者,力爭無少遜。調貴陽,亦以強直忤巡撫宮兆麟,因公累解職。引見,復授山西蒲州知府,調太原。清釐積獄,修復風峪山堤堰,障山潦,導之入汾,民德之。擢江西吉南贛寧道,署布政使,以王錫侯書案被議。高宗知其賢,發江南,以同知用。會南巡,克開署江寧府,迎駕,授江西九江知府,尋擢浙江糧儲道。

時巡撫王亶望貪黷,屬吏多重徵以奉上官。克開至,誓不取一錢,請於巡撫,約與之同心。亶望姑應之,心厭克開,乃奏克開才優,請移治海塘,於是調杭嘉湖道。會改建海岸石塘,總督欲徙柴塘近數百丈以避潮,克開曰:「海與河異,讓之則潮必益侵,無益也。」乃止。年餘,以督工勞瘁卒。

克開在寧朔治水績最著,生平治獄多平反。禮儒士,嘗以私錢興書院。歿無餘貲,天下稱清吏。當時守令以興水利著者,又有鄭基、康基淵、言如泗,後有周際華。

基,字築平,廣東香山人。以諸生入貲為知縣。乾隆間銓授安徽鳳臺縣,東鄉有通川三:曰黑濠,曰濕泥,曰裔溝。匯潁上、蒙城諸縣水以達淮,歲久盡湮,秋潦輒成巨浸。侍郎裘曰修奉使治淮、潁諸水,獨不及鳳臺。基具牘陳利害及工事甚悉,曰修允其請。基察土宜,穿故渠,三河交暢。釃上游諸水以通淮流,不逾時工成。魯松灣地遠淮而卑,頻患潦,捐俸倡築堤障,遂成膏腴。調定遠,舉卓異,擢壽州知州。安豐塘,古芍陂也。塘圮,基審覈舊制,繕復之,為水門三十六,為閘六,為橋一。其旁則為堨、為堰、為圩,啟閉以時,汙萊盡闢。嘗循行阡陌,見沙地磽確多不治,教民種薯蕷,佐菽麥,俾無曠土。壽州不知蠶織,而地多椿樗,可飼蠶。購蠶種,教民飼之,農桑並興。其後遇旱,獨鳳臺、壽州秋成稔於他縣,以水利修也。遷泗州直隸州知州。賑水災,饑而不害。擢江蘇淮安知府,淮安為眾水所聚,於城東濬澗市河,於北開漁濱山字河,於西開護城河,壅滯悉通,民便之。

基博覽前史,於河渠水利圖經,丹鉛殆遍,施行輒有成效。乾隆四十一年,擢江南守巡道,命甫下而卒。

基淵,字靜溪,山西興縣人。乾隆十七年進士,歸班銓授河南嵩縣知縣。舊傍伊水有渠十一,久湮絕。基淵按行舊址,勸民修復。山澗諸流可引溉者,皆為開渠。渠身高下不一者,分段設閘以蓄洩之。田高渠下者,則教為水車引溉。凡開新、舊渠十八,灌田六萬二千餘畝。巡撫上其事,優詔議敘,尋以憂去。服闋,授甘肅鎮原,調皋蘭,擢肅州直隸州知州。洪水渠岸峻易崩,基淵度勢於南石岡引鑿渠口,以避沖陷之害。野豬溝有荒田,無水久廢。基淵詢訪耆舊,加寬柳樹閘龍口,別開子渠。界荒田為七區,招民佃種,區取租十二石,給各社學,名曰新文渠。州東南九家窯,鑿山後渠開屯田,舊駐州判主之,久之田益薄瘠,民租入不足支官役;基淵請汰州判,改屯升科,為籌歲修費,民於是有恆產。

基淵治官事如家事,博求利病。在嵩縣,植桑教蠶,出絲甲於他邑。以無業之地,建社學三十二所。在肅州,開郊外廢灘,種楊十餘萬株。遍諭鄉堡種樹,薪樵取給,建社學二十一所。又於金佛、清水兩鄉建倉,以免徵糧借囤民房之累。革番、民採買需索,皆有實惠。四十四年,擢江西廣信知府,卒於官。

如泗,字素園,江蘇昭文人,言子七十五世孫。乾隆三年,高宗臨雍,如泗以賢裔陪祀,賜恩貢生,充正黃旗官學教習。十四年,銓授山西垣曲知縣,城濱黃河,修石堤以捍水。亳河故有數渠,復於上游濬之,分以溉田,民稱「言公渠」。調聞喜,涑水湍急,舊渠多圮,別濬新渠,食其利者五村。舉卓異,擢保德直隸州知州。新疆軍興,徵調過境,值歉歲,如泗經畫曲當,民無所累。陜西巡撫明德聞其能而薦之,乞養歸。父喪除,補解州。白沙河在城南,地如建瓴,南決則害鹽池,北決則壞城。如泗請於大吏,用鹽帑修築兩岸石堰,長五里。又姚暹渠本以護鹽池,民田不能灌溉。故事,商民分修,商盡諉之於民,力爭,乃仍舊貫。二十九年,擢湖北襄陽知府。如泗愛士恤民而治盜嚴,在解州,民間夜不閉戶。襄陽素為盜藪,聞其至,盜皆遠遁。三十四年,因失察屬員罷職。尋以皇太后萬壽祝嘏復原官,遂不出。嘉慶十一年,卒於家,年九十一。光緒中,祀名宦。

際華,字石籓,貴州貴築人。嘉慶六年進士,授內閣中書,親老乞改教職。歷遵義、都勻兩府教授,以薦擢知縣。道光六年,授河南輝縣。百泉出縣北蘇門山,衛河之源也。其西諸山水經縣南入衛,曰峪河;其北諸澗水歷縣東入新河,曰東石河。新河者,自縣北鑿渠引衛河,至縣南復入衛,又稱玉帶河,皆資疏水曳、利灌溉。時並淤塞,遇水輒苦漂溺。際華履視溝、渠,出俸錢率民醵貲濬峪河,修紅石堰,疏新河。鑿東石河六十餘丈,堅築其岸。諸渠綺交脈注,潦患以息。課民種桑四萬株,教之育蠶,他樹亦十五萬株,於是邑有絲絮、材木之利。蘇門故多名賢祠宇,咸新之,修明祀事,以勵風教焉。

署陜州直隸州知州。自澠池入陜,道硤石五十餘里,險惡為行旅所苦。際華別開平道,往來者便之。回避,改授江蘇興化縣。當里下河之下游,水患尤急。際華議開攔江壩以洩湖、河之水,鹽官及商皆力爭,以為壩開則水南下溜急,於鹽舟牽挽不便。際華曰:「彼所爭者,十四里牽挽之勞,以較揚州東七縣田廬場灶之漂溺,蠲免賑恤之煩費,輕重何如?」總督林則徐韙其議。調江都,兼署泰州,毀淫祠百餘區,改為義學。則徐疏薦之,尋告歸,卒於家。

先是輝縣及興化民皆不習織,際華輒自出貲置織器教之,轉相授,於是二縣有衣被販貿之利,至今賴之。輝縣請祀名宦祠。

汪輝祖,字龍莊,浙江蕭山人。少孤,繼母王、生母徐教之成立。習法家言,佐州縣幕,持正不阿,為時所稱。乾隆二十一年成進士,授湖南寧遠知縣。縣雜瑤俗,積逋而多訟,前令被訐去,黠桀益肆挾持;又流丐多強橫。輝祖下車,即捕其尤,驅餘黨出境。民納賦不及期,手書諭之曰:「官民一體,聽訟責在官,完賦責在民。官不勤職,咎有難辭;民不奉公,法所不恕。今約每旬以七日聽訟,二日較賦,一日手辦詳。較賦之日亦兼聽訟。若民皆遵期完課,則少費較賦之精力,即多聽訟之功夫。」民感其誠,不逾月而賦額足。

治事廉平,尤善色聽,援據比附,律窮者,通以經術,證以古事。據漢書趙廣漢傳鉤距法,斷縣民匡學義獄;據唐書劉蕡傳斷李、蕭兩氏爭先隴獄;判決皆曲當,而心每欿然。遇匪人當予杖,輒呼之前曰:「律不可逭,然若父母膚體,奈何行不肖虧辱之?」再三語。罪人泣,亦泣。或對簿者,反代請得免,卒改行為善良。每決獄,縱民觀聽。又延紳耆問民疾苦、四鄉廣狹肥瘠、人情良莠,皆籍記之。

寧遠例食淮鹽,直數倍於粵鹽,民食粵私,大吏遣營弁偵捕,輝祖白上官,以鹽愈禁則值愈增,私不可縱,而食淡可虞,請改淮引為粵引。未及報,輝祖即張示:「鹽不及十斤者聽。」偵弁謂其縱私,輝祖揭辨,總督畢沅嘉之,立弛零鹽禁,時偉其議。兩署道州,又兼署新田縣,皆有惠政。以足疾請告,時大吏已疏調輝祖善化,又檄鄰邑獄,因足疾久不赴,疑其規避,奪職。歸里,閉戶讀書,不問外事。值紹興西江塘圮,巡撫吉慶強輝祖任其事,帑節工堅,時稱之。舉孝廉方正,固辭免。

輝祖少尚氣節,及為令,持論挺特不屈,而從善如轉圜。所著學治臆說、佐治藥言,皆閱歷有得之言,為言治者所宗。初通籍在京師待銓,主同郡茹敦和,論治最契。同時硃休度並以慈惠稱。

敦和,字三樵,浙江會稽人。初嗣婦翁李為子,占籍廣東。乾隆十九年成進士,歸本宗,授直隸南樂知縣。慎於折獄,於片紙召兩造,立剖曲直,當笞者薄責之,民輒感悔自新。擇清白謹願者充社長、里正,令密陳利弊,以次行之。縣當豬龍河之沖,察河源委,於開州、清豐之間審地形高下,因勢利導,水不為患。地多茅沙鹽咸,教以土化之法,廣植雜樹。鄉民以麥稭編笠為生,敦和勸種桑。

調大名,漳水患劇,旁有渠河,敦和謀開渠以殺其勢。適內遷大理寺評事,不及上請。乃手書揭城門,勸民刻期集河干,親為指示,民具畚鍤來者以萬計。經旬而渠成,後利賴之。尋復出為湖北德安府同知,署宜昌知府,緣事降秩。卒,祀直隸名宦祠。子棻,以一甲一名進士官至兵部尚書。

休度,字介斐,浙江秀水人。乾隆十八年舉人,官嵊縣訓導,以薦授山西廣靈知縣。值大荒疫,流亡過半,休度安撫招徠。糧籍舊未清,履勘勸耕,一年而荒者墾,三年而無曠土。糧清賦辦,獲優敘。尤善決獄,劉杷子妻張,以夫出,饑欲死,易姓改嫁郭添保。疑郭為略賣,詰朝手刃所生子女二而自剄。休度詣驗,婦猶未絕,目郭作聲曰:「販,販!」察其無他情,讞定,杷子乃歸。眾曰:「汝欲知婦所由死,問硃爺。」休度語之狀,並及其家某事某事。杷子泣曰:「我歸愆期至此,勿怨他人矣。」稽首去。薛石頭偕妹觀劇,其友目送之。薛怒,刃傷其左乳,死。自承曰:「早欲殺之,死無恨。」越日,復詰之曰:「一刃何即死也?」薛曰:「刃時不料即死。」曰:「何不再刃?」薛曰:「見其血出不止,心惕息,何忍再刃?」遂以誤殺論,減戍。休度嘗曰:「南方獄多法輕情重,北方獄多法重情輕,稍忽之,失其情矣。」待人以誠,人亦不忍欺。周知民情,訴曲直者,數語處分,民皆悅服。數年囹圄一空,舉卓異。嘉慶元年,引疾歸,縣人懇留不得,乞其「壺山垂釣」小像勒諸石。歿後,祀名宦。

休度博聞通識,尤深於詩,以其鄉硃彞尊、錢載為法。任校官時,採訪遺書,得四千五百餘種,撰總目上諸四庫。大學士王傑為學政,任其一人以集事,時盛稱焉。

劉大紳,字寄庵,雲南寧州人。乾隆三十七年進士,四十八年,授山東新城知縣。連三歲旱,大紳力賑之。調曹縣,代者至,民數千遮道乞留,大吏為留大紳三月。及至曹縣,旱災更重於新城。大紳方務與休息,河督檄修趙王河決堤,集夫萬餘人,以工代賑,兩月竣事,無疾病逃亡者。既又檄辦河工稭料三百萬,大紳以時方收斂,請緩之。大吏督責益急,將按以罪,請限十日,民聞,爭先輸納,未即期而數足。一日巡行鄉間,有於馬後議穀賤銀貴開徵期迫者,大紳顧語之曰:「俟穀得價再輸未遲也。」語聞於大吏,怒其擅自緩徵,遣能吏代之。民慮失大紳,爭輸賦,代者至,已畢完。大吏因責徵累年逋,久倘不足,終以代者受事。民益恐,晝夜輸將,不數日得三萬餘兩。初,大紳以忤上官意,自劾求去,民環署泣留,相率走訴大吏。適大吏有事泰山,路見而諭止之,不得去。至是密自申請,民知之,已無及,乃得引疾歸。

五十八年,病起,仍發山東,補文登。值新城修城,大吏徇士民請,檄大紳督工,逾年始竣,尋以曹縣舊獄被議,罷職遣戍。新城、曹縣民為捐金請贖,得免歸。嘉慶五年,有密薦者,詔以大紳操守廉潔、兼有才能,辦理城工、渡船二事,民情愛戴,引見,復發山東,攝福山,補朝城。大水,大紳以災報,大吏駁減其分數,民感大紳,雖未獲減徵,亦無怨謗者。大紳又力以病求去,移攝青州府同知,尋擢武定府同知。捕蝗查賑,並著勞勩。以母老終養歸,遂不出。卒,祀名宦祠。

大紳素講學能文章,在官公暇,輒詣書院課士。嘗訓諸生曰:「硃子小學,為作聖階梯,入德塗軌。必讀此書,身體力行,庶幾明體達用,有益於天下國家之大。」於是士知實學,風氣一變。

吳煥彩,字蘊之,福建安定人。乾隆二十五年進士,授山東範縣知縣。民苦充牌頭。吏列多名進,以次需索,煥彩革其弊。清河水溢為災,其岸左高右卑,因開五頃窪,以瀉其東南;築福金堤,以防其西北;歲得麥田四萬畝。啇地民苦納租,欲請免而格於例,代輸租之半,教之種番薯,民困乃紓。三十九年,壽張逆匪王倫作亂,距範縣四十里,煥彩修城籌守御,力清保甲,凡村落大小,人民賢愚可指數。有孟興璧者,與黃昌吉等有隙,上變列三十餘人,朝命侍郎高樸與巡撫往察治。使者出牒示,煥彩曰:「某已死,某為某之父,某之子皆良民,呼之即至。」使者欲以兵往,煥彩曰:「兵至,愚民非死即走,無可訊,咎將誰執?」煥彩夜抵村中呼告之,皆呼冤。煥彩曰:「惟無其事,必出就訊,亟從我去。不然,禍立至。」民皆裹糧從。使者按籍,少二人,煥彩曰:「一已死,一外出,已命其兄招之。」言未畢,有跪門外者,則已來矣。訊之皆誣,遂坐告變者。巡撫曰:「知縣者,知一縣事,君可謂之知縣矣。知縣者,民之父母,君可謂之民之父母矣。」以卓異薦,擢湖北鶴峰知州。地本苗疆,改流未久,奸宄雜居。煥彩勤於聽訟,積弊一清。土司族裔,每借祖墳詐人財,懲治之,澆風自息。民樸陋不知書,設義塾,資以膏火,至五十三年,始有舉於鄉者。後以病歸,鶴峰請祀名宦,範縣亦為建生祠。年逾八十,卒。

紀大奎,字慎齋,江西臨川人。乾隆四十四年舉人,充四庫館謄錄。五十年,議敘知縣,發山東,署商河。會李文功等倡邪教,誘民為亂,訛言四起。大奎集縣民,諭以禍福,皆驚悟。鄰郡惑者聞之,亦相率解散。補丘縣,歷署昌樂、棲霞、福山、博平,民皆敬而親之。父憂歸。嘉慶中復出,授四川什邡縣。或謂:「什邡俗強梗,宜示以威。」答曰:「無德可懷,徒以威示,何益?」奸民吳忠友據山中聚眾積粟,講清涼教。大奎躬率健役,夜半搗其巢,獲忠友,餘眾驚散。下令受邪書者三日繳,予自新,民遂安。擢合州知州,道光二年,引疾歸。年八十,卒,祀合州名宦。

邵希曾,字魯齋,浙江錢塘人。乾隆五十四年舉人,嘉慶中,官河南知縣。歷權通許、盧氏、鄢陵、西華、沈丘、太康、扶構、淮寧、新鄉,皆有聲。滑縣教匪之役,司糧臺。及匪平,訊鞫俘虜,治餘匪,凡良民被脅者皆得釋,保全甚眾。晚授桐柏,民苦盜,令村集建棚巡更,鄉數家出一人為門夫,有警環集,無事歸業。訪捕強暴者繩以法,積匪率遠徙。慎於折獄,皆速結,訟日以稀。朔望蒞學,集諸生講論,增書院膏火,親課之如師。道光六年,邑人王四傑始登進士第,自明初以來所未有。募錢萬緡,建義學。凡經塾三,蒙塾十五。擇其秀者入書院肄業,文教興而悍俗漸化。在任十年,民安之。老病,大吏不令去,卒於官。


\end{pinyinscope}