\article{志三十六}

\begin{pinyinscope}
地理八

△山東

山東:禹貢青、徐及兗、豫四州之域。明置山東承宣布政使司。清初因之。雍正二年,升濟南府之泰安、武定、濱,兗州府之濟寧、曹、沂等六州為直隸州。八年,濟寧州仍屬兗州府。十二年,升武定、沂二州為府,濱州改屬武定。十三年,升泰安、曹二州為府。乾隆四十一年,仍升濟寧、臨清為直隸州。凡領府十,州二,散州八,縣九十六。在京師之南。八百里。東至大海;一千三百里。西至直隸元城縣界;三百四十里。南至江南沛縣界;五百七十里。北至直隸寧津縣界。二百四十里。廣一千六百四十里,袤八百里。北極高三十四度三十五分至三十八度二十分。京師偏西一度二十五分至偏東六度四十分。宣統三年,編戶五百三十七萬七千八百七十二,口三千一百三萬六千九百四十四。

濟南府:沖,繁,難。巡撫,布政、提學、提法、鹽運司,濟東泰武臨、巡警、勸業道駐。初沿明制為省治,領州四,縣二十六。雍正中,武定、泰安、濱直隸,割陽信、萊蕪、利津等九縣屬之。北至京師八百里。廣三百六十里,袤二百八十里。北極高三十六度四十五分。京師偏東四十一分。領州一,縣十五。歷城沖,繁,難。倚。城南:歷山。東:華不注。東南:長城嶺,玉水出焉。錦雲川水入長清注黃河。河即大清、濟水故道,其右瀆入。西北逕藥山,北至濼口鎮。又東北逕鵲山,其南新小清河入,逕城北,右受大明湖,東逕虞山,納巨合及關盧、武原水,並入章丘。今大明湖在城內,惟匯珍珠、濯纓諸泉,乃宋西湖,非唐以前遺址矣。有巡司:申公集。主簿:譚城。龍山驛。津浦、膠濟鐵路。章丘繁,疲,難。府東百十里。南:長白山、東陵、平頂。西南:危山;雞山,水經注「巨合水出西北」,逕榆科泉莊,及武原、關盧水,入歷城,注新小清河。新小清復入西北,右合繡江,即溝河,注百脈水,出土鼓故城,逕陽丘故城黃巾固。小清河又東北逕賈莊入齊東。其北黃河右瀆自歷城入,緣濟陽界。小清昔源濼水,今源獺河,別源出東南野狐嶺,西北逕青龍山,土鼓、寧戚故城,折東入鄒平,即小清故道。有普濟鎮。鄒平疲。府東北百六十五里。南:長白山。東南:黃山。西南:九龍。北:新小清河,自齊東入,東入長山。西北:獺河,自章丘入,逕滸山濼,東北為清河溝,並入長山。東南:孝婦河,即瀧水,自長山錯入。左合白條溝、沙河,逕伏生墓,屈東北,仍入之。其故道逕梁鄒故城。有孫家鎮。淄川簡。府東少南二百二十里。東南:原山,淄水出焉。南有豬龍河,俗呼孝婦,即瀧水,自博山入,逕城西南,右合般水,又北逕浮山。獺水河出黌山,逕昌國故城,會浸泗河自左注之,乃古德會水。左得萌水口,入長山。鐵路。長山簡。府東百九十里。西南:長白山。南:鳳山。西北:新山。清河自鄒平入,折北入高苑。清河溝東入新城。東南:孝婦河自淄川入,左得魚子溝,錯鄒平仍入。有周村鎮。新城簡。府東北二百十六里。東南:商山。西北:清河溝自長山入,右迤為青沙泊,淤。東有烏河,即時水,逕西安故城,左納澇淄河,折北,逕會城湖入博興,西通麻大湖。西北:孝婦河自長山入,右會鄭黃溝及系水,時枝津匯焉。齊河沖,繁,難。府西五十里。西南:黃河左瀆自長清入,逕城南,又東入歷城。北:徒駭,自禹城入。西南趙牛河自長清入,歧為岔河,並入禹城,而經流復入,合倪倫河自右注之,又東逕梁家莊入臨邑。劉宏鎮。有驛曰晏城。齊東疲,難。府東北百五十里。西北:黃河右瀆自濟陽入,逕延安鎮,又東緣惠民界入青城。南:新小清,自章丘入,逕臨濟故城,又東入鄒平。減水河、壩河,堙。臨河鎮。濟陽疲,難。府東北七十里。黃河左瀆自禹城入,逕城南,又東北緣齊東界入之。西北:徒駭及商河,並自臨邑入,屈北入商河鎮。鎮二:回河、新市。禹城沖。府西北百十里。西南:徒駭河自高唐入,少淤,逕三岔口,右會漯河及管氏、趙牛、岔河,逕城西北,又東北入齊河。趙牛河錯入,右合溫聰、刁強河,仍入注之。禹城橋,縣丞駐。新安鎮。劉普驛。鐵路。臨邑簡。府北百四十里。南:徒駭河自禹城、東南商河自齊河入,並東入濟陽。西鉤盤自陵緣界東北仍入之,並淤。長清沖,繁,疲,難。府西南七十里。南:磧湑山。東南:方山,有行宮。東北:瓘眉。西南:孝堂,古巫山。黃河自肥城入,逕城西北,右合南沙河。水經注「出南格馬山賓谿谷,北逕盧縣故城北與中川水合」者。又東北緣齊河界納玉水,其南新小清,並入歷城。西南:趙牛河,自茌平入,右納趙王河,北入齊河。張夏鎮,縣丞駐。二驛:崮山、長城。陵簡。府西北二百里。故城與德互徙。明永樂七年,西新鬲津河自德入,環城又東,右合篤馬、趙王,又東鉤盤自禹城入,並入德平,並涸。滋博鎮。德州沖,繁,難。府西北二百六十里。糧道駐。南有運河自恩入,北逕城西。其南支四女寺減河入檀。老黃河故道逕城東北哨馬營,是為北支,並入直隸。東南:馬頰河,即篤馬,自平原入,水經注「逕臨齊城南」,今邊臨鎮,州判駐,東北入德平。其新鬲津,東北入陵,淤。有鬲故城,古瀆在焉。老黃河故屯氏瀆,篤馬其別河南瀆。桑園、安德驛。又良店、梁家莊水驛二。鐵路。德平難。府北二百五十里。西:馬頰河自德入,右會小河,東北入樂陵及直隸寧津,下至海豐月河口入海。自直隸元城入,凡行山東境六百四十八里。南有鉤盤自陵入,東北入商河,涸。有懷仁鎮。平原沖。府西北百八十里。西:馬頰河,自夏津緣界合蒲河,其東新鬲津故道,並北入德。東:篤馬溝,舊合趙王河入陵,涸。有水務鎮。桃源驛。鐵路。

東昌府:沖,繁。隸濟東泰武臨道。初沿明制,領州三,縣五。雍、乾間,濮、臨清直隸,割範、觀城、朝城、夏津、丘。東距省治二百二十里。廣二百二十里,袤二百八十里。北極高三十六度三十三分。京師偏西十八分。領州一,縣九。聊城沖,繁,難。倚。南有運河,即會通,自陽穀入,右播為趙牛、湄河,入茌平。王官鎮。崇武水驛。堂邑沖,疲。府西四十里。東北:運河,自聊城入,逕梁鄉閘。西南:馬頰,自冠入,逕張家堂,絕運,並入博平。侯固鎮有城。博平沖,繁。府東北四十里。西南:運河,自堂邑入,逕土橋閘,西北至田家口。其西馬頰,自堂邑入,西北復絕。西:徒駭,自聊城入,逕鄧家橋,右納湄河,東北擅濕故瀆,入茌平,涸。茌平沖,繁,難。府東北六十里。管氏河首李莊,匯小馮新河,東北入禹城。西北:徒駭,自博平錯入,入高唐。西:湄河,自聊城錯入,入博平。

南:趙牛河,自聊城入,錯東阿復入,入長清。西:古黃河。有四瀆口。廣平鎮。清平沖,繁。府北少東七十里。西南:運河,自堂邑入,北逕魏家灣,有巡司,入臨清。西:馬頰,自博平再入,入高唐。古黃河,西北自臨清入,入夏津。水驛。莘簡。府西一百里。西北:馬頰,宋六塔、二股河所逕,自朝城入,東北入冠。東有古濕河,自朝城入,入聊城;分支入陽穀。鎮:馬橋。冠沖,繁,難。府西百里。東南:弇山,馬頰自莘入,入堂邑。其西,古黃河,逕西北二十里冉子墓,東入館陶。宋北流故瀆,乾隆、道光兩次決入馬頰,至府境入運為患,其故道循陳公堤北入館陶、臨清,絕運,至舊城外,曰沙河,入直隸吳橋。館陶簡。府西北百二十里。南:館陶鎮。有廢巡司。陶山,西南四十里。衛河,隋永濟渠,自直隸元城入,左合漳水,逕喬亭城東北,右會古黃河,入臨清。其北,屯氏。高唐沖,繁,難。府東北百十里。雍正八年直隸,十二年削所領禹城、平原、臨邑、陵。高唐山,東北五里。東南:徒駭,自茌平入,東北入禹城。地理志「河水自靈縣別出為鳴犢河」。其故瀆西南馬頰自清平入,北入夏津,並涸。固河鎮。魚丘驛。恩沖,繁,疲,難。府東北百八十里。西北:衛河,自武城錯入。西南古黃河,南馬頰,左瀆並自夏津入,入德。四女寺,縣丞駐。太平驛。

泰安府:沖,繁,難。隸濟東泰武臨道。初沿明制,為濟南屬州。雍正二年直隸,領新泰、萊蕪、長清。長清尋換肥城。十三年為府,增附郭,降東平與其所領東阿、平陰來隸。北距省八十里。廣四百三十里,袤百七十里。北極高三十六度十五分。京師偏東五十二分。領州一,縣六。泰安沖,繁,疲,難。倚。泰山,北五里,東嶽,亦曰岱宗,周略百六十里,高四十里,有行宮。南:介石、石閭、亭亭、梁父。東南:龜山、徂徠。西南:社首、高里。其峰南有汶水,自萊蕪入,右合天津水,左合牟汶,西南逕博故城,右合北汶、泮河及石汶、環水。其北汶北出者沙河,入長清。又西南,逕陽關、龍故城。東南:淄水,自寧陽入,逕岱山、梁父、柴故城,曰柴汶,右河、仙源河自左注之。又西南,大汶口,緣界逕汶陽故城南,合西濁,古蛇水,入東平。西北:黃山,肥河出,西入肥城。濟運泉六十有六。靜封鎮。婁德舊置巡司,改通判。安駕莊,主簿駐。肥城簡。府西七十里。雍正十三年自濟南來隸。西:金牛山。西北:陶山、巫山、黃崖。東南:瀑布。南:馬頭,沙河出,亦曰小會肥河,入東南,右合孤山河、黃河、趙王河,入西北,並東北入長清。範公河,堙。濟運泉十有二。石橫鎮。清泉水驛。新泰沖,繁。府東南百五十里。東南:山,西南:龜山。西北:新甫。北:青沙峴。東北:龍堂,小汶即牟汶出,西南逕敖陽鎮,右合平陽、西周、蘇莊、羊流,左廣明河,逕靈查保入泰安。濟運泉四十。上四莊巡司。敖陽驛。萊蕪簡。府東百二十里。西南:冠山。北:陰涼。東南:大石。南:安期。西北:羊丘。東北:杓山;原山,地理志「淄水出其陰,東入博山」。周禮「幽州,其浸菑時」。汶水出其陽,所謂嬴汶,屈折西南,匯黑虎、辛興、魚池諸泉,逕嬴縣故城入泰安。東南:牟汶,自蒙陰入,逕牟縣故城,匯響水灣、海眼泉、孝義河水,至城南,又西,左合司馬河,從之。濟運泉四十有九。東平州沖,繁,難。府西百四十里。明屬兗州。雍正八年直隸。十三年降削所領東阿、平陰、陽穀、壽張。北:蠶尾山、瓠山,龍山即危山。東有汶水自泰安入,右納匯河,明築戴村壩閼之,西南入汶上。其溢而西者,奪漆溝逕龍堌北,亦曰大清河。其逕堌南者,小清合龍拱河,所謂「城南二汶」,夾城至馬口而合。又西北,黃河西自壽張入奪之。運河即元會通河,後自汶上入,即梁山濼,逕安氏山東絕之,並入東阿。安山湖、赤河,並堙。濟運泉三十有五。東阿沖,繁。府西北二百十里。東北:穀城山。東南:雲翠。西北:曲山。西南:黃河自東平奪大清入,逕魚山,水經注馬頰口在焉。又西北,右合狼溪,入平陰。西北趙王河從之,其正渠趙牛河,自茌平錯入,並涸。而古黃河、瓠子堙。運河自東平絕黃河入,逕陶城鋪入陽穀。舊運河,淤。四鎮:楊劉、安平、南穀、新橋。二驛:舊縣、銅山。平陰簡。府西北百九十里。明屬東平。雍正十二年來隸。西:邿山。西北:榆山。西南:黃河自東阿入,右合錦水,逕城西北,又東入肥城。其肥河入東南,合柳溝泉,折南,入東平。西北:趙王河,自東阿入,分入肥城、長清。古黃河,堙。鎮:滑口。

武定府:繁,難。隸濟東泰武臨道。初沿明制,永樂初改金棣州曰樂安,宣德初,平漢庶人,改曰武安。國制為濟南屬州,領縣三。雍正二年直隸。十二年為府,置附郭,降濱並所領利津、霑化、蒲臺及濟南之青城、商河來隸。西南距省治二百里。廣二百八十里,袤二百七十里。北極高三十七度三十四分。京師偏東一度十三分。領州一,縣九。惠民繁,疲,難。倚。明初省入州。雍正十二年復。南有黃河左瀆自濟陽入,逕清河口,西入濟南。徒駭自商河東逕聶索鎮,與清河經永利、支角西亦入濱。又北,沙河、商河分入而合,通曰沙河,逕鍾家營、陽信,右得惠民溝,又西北,鉤盤入。青城簡。府東南六十里。雍正十二年自濟南來隸。黃河右瀆自齊東入,逕董溝東入濱。有田鎮。陽信疲。府東北四十里。西南:鉤盤自惠民入,逕紅廟莊,一曰信河,縣氏焉,東北入霑化,涸。沙河錯入仍入之。有欽風鎮。海豐簡。府東北六十里。西北:騮山。海,東北百五十里,為大沽河口,與直隸鹽山接。有巡司。鬲津河逕馬谷山入,又東南,月河口。馬頰河自慶雲逕街東鎮入。其故道堙。今馬頰,唐所開。又東南,石橋口。至霑化,鉤盤錯入仍入之。有分水鎮。樂陵疲,難。府西北九十里。西北:鬲津河,自直隸寧津入,錯南皮復入,入鹽山。西南:馬頰,自德平入,東北逕興隆鎮入慶雲。古鉤盤,水經注屯氏別河,逕樂陵故城北,北瀆逕重合定縣故城南,並堙。舊縣鎮,明置巡司,廢。商河繁,疲,難。府西南九十里。古商河,北十五里。水經注「逕朸縣故城南」,實沙河,今圖誤。鉤盤自德平入,涸,而沙南徙,西首臨邑界,逕城南。南有徒駭,即古黃河,及其支津商河,自濟陽入,並東入惠民,而商與沙合。寬河鎮。濱州繁,難。府東九十里。雍正二年直隸。十二年復將所領蒲臺、利津、霑化削。西南:黃河,自惠民、青城入,屈東北,錯蒲臺,復緣界故道左出,與合右瀆仍入之。徒駭自惠民入,左合沙河,涸。迤東北入霑化。利津繁。府東百五十里。海,東北百六十里。西與霑化接。西南:黃河,自濱、蒲臺入,側城東北,入為牡蠣嘴。豐國鎮有巡司。霑化難。府東北七十里。海,北少東百里。西與海豐接。西有鉤盤河自陽信入,錯海豐復入,又東南大洋口。西南:徒駭,自濱入。鎮:永豐,久山有廢巡司。蒲臺疲,難。府東少南百二十里。雍正十二年自濱來隸。西有黃河自濱入,逕城南,又東北,左瀆仍緣濱合故道入利津。龍湖。鎮:龍混。

臨清直隸州:沖,繁,難。隸濟東泰武臨道。初沿明制,為東昌屬州,領縣二。乾隆四十一年直隸,割武城、夏津、丘還隸,而館陶還東昌。東南距省治百十里。廣百五十二里,袤百三十里。北極高三十六度五十七分。京師偏西三十六分。領縣三。東有運河自清平入,逕城南,西南衛河自館陶來會,是為南運河,亦通曰衛河,貫城而北,擅屯氏故瀆,入直隸清河。古黃河,東北自館陶入,歧為沙河,並入夏津。王家淺巡司。清源、渡河水驛。武城沖,繁。府北少東一百里。南有衛河自夏津入,逕城西,折東北,復會沙河,錯恩復入,入直隸故城。舊有一字、黃蘆、五溝諸水。縣卑淖,金末艾家凹水濼,廣數十里,深一丈。漢復陽故城,今饒陽鎮。甲馬營巡司,又驛。夏津疲。府東少北四十里。北:孫生鎮。西有衛河自州入,再錯直隸清河入,入武城。舊有沙河自州入,東有古黃河、馬頰,並自清平入,入恩。丘簡。府西南八十里。漳河,此順治九年自廣平平固店直注者,非古漳故道。二並自直隸曲周入,一逕城西,至宋八甿仍入之,一逕城東,至柳甿入清河。

兗州府:沖,繁,難。總兵,兗沂曹濟道治所。初沿明制,領州四,縣二十三。雍正中,沂、曹、濟寧、東平,濟寧直隸。先後割縣十三分隸,而東平降割泰安。東北距省治三百二十里。廣五百十里,袤二百六十里。北極高三十五度四十二分。京師偏東三十四分。領縣十。滋陽沖,繁,難。倚。嵫山,西北三十里。東有泗水,自曲阜入,至金口壩,歧為府河,貫城而出,左得十四泉,入濟寧,實珠水正渠,又南,左合沂水、蓼水入鄒。西北:漕河,自寧陽入,左合漢馬,右洸河,亦南入濟寧,涸。故城驛二:昌平、新嘉。鐵路。曲阜簡。府東三十里。東:防山。又東:戈山。泗水自其縣入,右合嶮水及石門山水。東南:沂水、蓼水,自鄒入。沂逕城西南而分流,得右洙、左雩,復合蓼,逕北店村而分,一仍入鄒,一與泗、沂並西入滋陽。古者洙北、泗南,今互易,蓋自後魏亂流始。泗故道,孔林夫子墓南。濟運泉二十有八。鐵路。寧陽簡。府北五十里。寧山,北十八里。西:水牛。西北:鶴山。東北:告山、壽山。南:鳳山,淄水出,北逕魯成邑北,入泰安注汶。汶復緣界逕漢汶陽至剛故城,洸水出焉,所謂「汶為闡」,與漕河、漢白馬並南入滋陽。正渠復逕春城口入東平。青川驛。鐵路。鄒沖,繁。府東南五十里。東南:繹山。相近紅山,實鳧山。北沙河出,與白水河並入滕。南:昌平,而岱脈南馳寧陽、曲阜。入東北六里,曰尼山,其西南昌平有鄉,孔子生焉,故屬曲阜。今山南長莎村相近四基西麓,孟子墓在焉。沂水導源尼山,西與蓼水並入曲阜,注泗。泗復自滋陽入,西北錯濟寧復入,入魚臺。蓼水亦復入,會溪湖水,為白馬河,合大沙、紅溝河,從之。咸丘。縣丞駐辛莊。驛二:邾城、界河。鐵路。泗水簡。府東少北九十里。東北:歷山、龜山、陪尾。有桃墟,泗水出,其北關山,洙水出,逕卞故城而合。南,姑[B167]郚城。又西,左合黃淵、右金線諸河,入曲阜。濟運泉八十有七。滕沖,繁,疲,難。府東南百四十里。東南:桃山。狐駘自嶧入,逕微山湖,右有許由泉,自嶧入,為南明河。薛河逕昌慮南,石橋泉逕薛城注之,再錯出江蘇沛縣。漷水出東北述山,逕藍陵、祝其、合鄉故城,合南梁水、趵突入沛注之,復入,右合三里河,北沙河自鄒入,夾休城,其白馬河入合界河為鬱郎淵注之,又西北入魚臺。東北:小沂水,入費,而昭陽湖堙。鎮:安平、南穀、陶陽,別有夏鎮。泇河,通判駐。驛二:滕陽、臨城。嶧沖,繁,難。府東南二百六十里。北:君山。相近車梢峪,[C051]水出焉,曰滄浪淵,會許池泉,逕葛嶧山,合金注河,其南茅茨、仙人河。東南,運河自江蘇邳州入合之,又北逕微山湖、南陽湖入滕。乾隆中濬伊家河。濟運泉十有四。萬家莊水驛。汶上沖,繁,難。府西北九十里。東北:太白山、坦山。西南:趙王河,自鄆城入,入嘉祥。北有汶水,自東平入,受濼澢諸泉、蒲灣泊水,曰魯溝,西南擅鵝河故瀆,注南旺湖,濟運運河,遂東南入嘉祥達濟寧,西北入東平達臨清。湖東接蜀山湖,北馬踏湖,並水經注汶左二水,逕東平陸故城北,漢縣,古厥國,入茂都澱。柴城鎮。南旺、馬村二集,並縣丞駐。新橋、開河二水驛。陽穀沖,繁,疲,難。府西北三百里。南有黃河,自範入,錯壽張入。運河入,逕東阿故城,有阿澤,又北入聊城。西:徒駭河故道,自範入,錯朝城,莘,再入,入聊城。鎮:安樂。阿城,縣丞駐。壽張疲,難。府西北二百四十里。東南:梁山,舊有濼。西南:黃河,自陽穀入,仍錯出復入,入東阿。其北,運河,自東阿錯入仍入之。有張秋鎮。其南沙灣、沮河並入鄆城。有竹口鎮。

沂州府:沖,繁,難。隸兗沂曹濟道。初沿明制,為兗州屬州,領縣二。雍正二年直隸。十二年為府,置附郭,降莒及所屬蒙陰、沂水、日照來隸。西北距省治六百六十里。廣五百二十里,袤五百十里。北極高三十五度九分。京師偏東二度十二分。領州一,縣六。蘭山沖,繁,難。倚。西南:寶山。東南:馬陵。北:大柱。沂水自其縣入。蒙山河入為汶水,逕鐵角山,納小河,又南逕魯中丘故城王祥墓右,合孝感河,至府治東北,右納小沂及胭脂河,又涑支津,其正渠逕勃莊南,為蘆塘河。沂水又南逕龍塘口,右歧為武河,其東白馬河。東北:沭水,自莒入,右合溫泉水,右武陽溝,再入郯城,而武河、蘆塘河錯郯城復入而合。又東,西泇河自費入,並南入江蘇邳州,而西泇右納別源巨梁水,復歧為夾山河,分入嶧芙蓉湖。鎮:長江、羅滕。青駝寺巡司。楊家莊、徐公店二驛。府屬沿海墩臺二十有八。郯城沖,繁,難。府東南百二十里。東北:羽山、蒼山。沭河自蘭山再入,逕城東北,右得墨河故瀆焉,環城南入宿遷。又南,逕馬陵山,西至紅花埠,又西,白馬河入逕城西,並入江蘇宿遷。又西,沂水入,入邳州。又西支津武水入,又西,蘆塘河入,合燕子河,右歧為鴨蛋河,入邳。正渠又南與武水仍入之。大興鎮,通判駐。舊置沂郯海贛同知,乾隆三十八年改。又磨山有廢巡司。紅花埠驛。道平、解村廢驛。費簡。府西北九十里。西北:蒙山。西南:南城。西北:聰山,浚水出,即地理志「南武陽冠石山治水」,應劭曰「武水南逕古顓臾」。右納小淮水,一曰小沂,又東南逕萬松山,右合祊水,至鐘山。左合洪塔、蒙陽、紅衣諸河。東南:旗山,東泇出。西南:抱犢崮。南:大沭崮,涑水出,西泇水出,並入蘭山。鎮:毛陽。關陽、平邑二巡司。莒州簡。府東北九十里。雍正二年直隸。十三年降削所領沂水、日照、蒙陰。北:七寶山。東:觀山、盧山。南:焦原。西:浮丘,有莒子墓,或誤浮來。有寺曰定林,因譌為二山。西北:洛山。水經注濰水導源濰山,東南逕屋山,漢箕縣故城。今合南源瓦屋山水,折東北,逕仲固山,右合析泉水。其東浯水,地理志、說文,出靈門壺山,逕漢姑幕故城,並入諸城。其西沭水,自沂水入,合華洛、袁公水,側城東南,左納鶴水、潯水,右合黃華水、馬溝河,至道口入蘭山。東南:石河,地理志夜頭水,入日照。其西柘汪、硃汪、青口三河入,並入江蘇贛榆,達於海。葛陂水,堙。十字路鎮。石埠集有巡司。沂水沖,繁。府東北百二十里。西:龍山。西南:靈山。東:峨。西北:雹山。北:沂山。其連麓大弁,沭水出,東南逕楊家城子注邳鄉南。左合大小峴、箕山水,逕孟母墓。莒西北,沂水自蒙陰入,左合螳螂河,東南逕蓋故城,水經注左合連綿、浮來、小沂水,至城西北,左納雪山、英山河,右合閭山、時密水,至河陽集,右納東汶水,入蘭山。其西蒙山河,自蒙陰入,復分支入費。縣丞駐東里店。垛莊巡司,又驛。蒙陰沖。府西北二百里。南:蒙山。東北:盧崮、具山。西北:敖山。北:兩縣。沂水三源,鄭氏主中源,出沂山,班氏主東源,出臨樂山,桑氏主西源,出艾山,逕龍洞山而合,東入沂水。其北魯山,螳螂水出。西南:五女山,桑泉水出,屈北逕城南,左會巨圍、堂阜,右髇崮水,又東,俗曰汶河,右合桃墟河,古蒙陰水,逕鐵城東北,盧川水會金星梓水,再錯沂水復入,合著善河,自左注之,又東南入沂水。黑龍寨。紫金關有巡司。日照簡。府東二百四十里。西北:昆山。西南:矮岐。北:會稽、白石。南:觀山、堙臺。海,東南五十里。東北自諸城以南為石臼口,納潮河。又南,夾倉口,傅甿河合伐莊河、固河入。又南,濤雒口、漲雒口、嵐頭山口。折西為狄水口,納石河。故安東衛在焉,有巡司。西有潯水,水經注「出巨公山」,俗黃墩河,其北鶴水,並西入莒。北有濰水,自莒錯入,東入諸城,洪陵河從之。鎮三:濤洛、夾倉、石臼。巨峰寨。

曹州府:繁,疲,難。隸兗沂曹濟道。總兵駐。初沿明制,為兗州屬州。雍正二年直隸,仍領縣二。八年,鉅野、嘉祥自兗割隸。十三年為府,置附郭。降濮並所領縣三,又割兗之單、城武、鄆城來屬;而嘉祥還舊隸。東北距省治

五百八十五里。廣百九十五里,袤二百八十里。北極高三十五度二十分。京師偏西五十一分。領州一,縣十。菏澤繁,疲,難。倚。黃河自直隸開州入,其南瓠子故道。水經注,東至濟陰句陽為新溝。城南:灉河。又南:北渠、河。水經注,北東北逕煮棗城南、冤朐北、莒都南。冤水,今大禰溝。有沙土集巡司。單繁,疲,難。府南一百五十里。明洪武二年降單州為縣,屬濟寧。十八年改屬兗州。雍正十三年改隸。東:棲霞山。西南:大陵山。南:黃河,自河南儀封縣界流入縣南,東流入江南碭山縣界。東:古淶河流入金鄉縣,湮。鉅野繁,疲,難。府東一百四十里。明洪武初,縣屬濟寧。十八年屬兗州府濟寧州。雍正二年分屬濟寧州。八年改隸。東南:高平山,山出蜂石,石片上結成形,有酷肖者。其東北:白馬山。東南:獨山、麟山。鉅野澤在縣北五里,亦曰巨澤,濟水故瀆所入也。元末為沙水所決,遂涸。東北:運河。東南:會通河。西南有故黃河,堙。有通濟閘閘官。鄆城簡。府東北一百二十里。明洪武十八年屬兗州府濟寧州。雍正二年分屬濟寧州。八年屬兗州府。十三年改隸。東北:獨孤山。東:金線嶺。黃河東北流,逕鄆城西二十五里,有黃河故道。灉河自直隸東明縣流入,東北經鄆城,西南入壽張縣界。古濟二水合流,北逕鄆城,南流入東平州界。城武簡。府東南一百十里。明洪武四年屬濟寧府,尋改屬兗州府,以城武為武成。雍正十三年改隸。明正德十四年,縣城圮于黃河,後河決多在城武。南有黃河故道,後堙。黃水自河南考城縣流入,東逕城武縣,南入江南豐縣界。城武東北有黃水枝溝。曹繁,疲,難。府東南四十里。東:青山。東南:景山。北:曹南山,即禹貢陶丘。古氾水出,與南、古黃河、賈魯河並堙,今惟南泡水河,首河南商丘界,東南至青涸集,歧為淶河,又八里河,入單,而西北柳林沙及南堤、夏月湖、白花諸河並淤。安陵、盤石鎮。縣丞駐劉家口。定陶簡。府東南四十里。東:菏澤,古南、北匯焉。菏水出西陶丘西南七里,南今南渠,中渠今氾故道,逕仿山,北合北渠為瀦水河,並入鉅野,南渠之南柳河,沙河並自曹入,入城武,並涸。濮州繁,疲,難。府北百二十里。雍正八年直隸,領範、觀城、朝城。十三年降,所領削。東:古濮水,堙。西南:黃河。北:金堤河。並自直隸開州入,東北入範。東南:趙王河,古灉水,自鉅野入,入鄆城。古清河即水,今黃河即魏河,實濮渠,亦北,其所合西無名洪河,亦堙。有瓠河鎮。範簡。府東北百六十里。範水,堙。西南:黃河,自濮入。水經注「逕範秦亭西」,春秋築臺於秦。又東逕委粟津,其金堤入逕城南,並入陽穀。鎮:安定。觀城簡。府北百七十里。南北引河,首直隸清豐界,逕城東,分入朝城,涸。夾堤河,首縣西馬陵堤下古龍潭,入杜家河,東北至櫻桃園入範,下至朝城入漯。沙河首角四池,北逕馬廠入朝城,下至莘入漯。浮河自直隸開州入,至朝城入河,堙。有武鄉鎮。朝城簡。府北二百十里。古漯水,亦武水故道,自西南楊家陂逕雁翎鋪入陽穀,合夾堤河、石人陂水,入莘合沙河,下至聊城入運,堙。南北引河舊自觀城入,夾城東西,分入陽穀、莘。馬頰自直隸元城入,此唐篤馬,非禹跡也,東北入莘,下至堂邑入運,並涸。

濟寧直隸州:沖,繁,難。隸兗沂曹濟道。運河道駐。明,兗州屬州。雍正二年直隸,仍領嘉祥、鉅野、鄆城。八年仍降屬兗州。乾隆四十一年復,割汶上、魚臺並嘉祥來隸。尋以嘉祥易汶上。東北距省治百八十里。廣百四十里,袤百八十里。北極高三十五度三十三分。京師偏東二十八分。領縣三。南:承注山。西南:縉雲。西北:運河,自嘉祥入,左受蜀山湖、馬場湖,府、洸二河自滋陽分入匯焉。逕城南,又東南逕南陽湖,左納泗水,入魚臺。泗復歧為新泗,錯鄒復入,合白馬河。西:趙王河,自嘉祥入,東南逕王貴屯橋為牛頭河,又合長澹,納蔡河,並入魚臺。趙王,古黃河北,長澹,河水南。有魯橋鎮。金鄉簡。府西南百里。明屬兗州。乾隆四十五年來隸。金鄉山,西北三十七里,隸鉅野陽山。西:萬福河,自鉅野入,右會西溝,屈東北,左合柳林河,又東逕蘇家橋,右通淶河自單入,合東溝,並東入魚臺。左通蔡河自嘉祥入,入州。柳林,古菏水,即也。嘉祥簡。府西五十里。城南:澹臺山。東南:武翟。西南:遂山。東北:南旺湖,運河自汶上入逕之,入州兩躡焉。其趙王河入西北逕萬善橋,左合牛頭河。西南:南清河自鉅野入,左合姚河,至城南為澹臺河,東南並入州。其金山河入為蔡河,東入金鄉,涸。魚臺沖,繁,難。府南百十里。東北:黃山、平山、獨山。有湖一,曰南陽。運河自州入,左合新河,受之。牛頭河入。西北,淶河、柳林河,自金鄉入,合為新開河來會。又東南,並入滕。而牛頭支津南入後,又有東支、西支河,自豐入焉。南陽鎮。河橋水驛。

登州府:沖,繁。總兵駐。登萊青膠道;今徙煙臺。明,領州一,縣七。雍正十三年,裁所轄四衛,置榮成、海陽。西南距省治九百二十里。廣五百六十里,袤三百五十里。北極高三十七度四十八分。京師偏東四度三十六分。領州一,縣九。蓬萊沖,繁。倚。東:硃高山、九日。東南:羽山、龍山、金果、馬山、丘山。府治三面環海,運舶駛焉。西北自黃迤東為欒家口、西山口,又東丹崖山,古蓬萊島,水城環之,黑水入。又東抹直口,沙河入。灣子口,安香河入。迤東南解宋營口、平暢口,至福山界,時家河從之。西南:崮山河入黃。其欒家口西北:大黑山島、北沙河島,東北:長山島,南北隍城島。有驛。黃繁。府西南六十里。黃山,南二十里。又南,石城。東南:萊山、掇芝。西南:盧山。北、西際海,西自招遠迤東北,界首河、隋家甿河入,至龍口,納呂家甿河。其西,母屺島。又東,納潁門及南欒河。其外,依島、桑島。又東,黃河營口,納榆林及莊頭河,至蓬萊界。馬停鎮。黃山館巡司。龍山、黃山館二驛。福山沖,繁。府東少南二百三十里。福山,西北五里。又西北,磁山,古牟山。東南:蛤蠦。西南:迷雞、青石。海,北十餘里。自蓬萊迤東為八角口、浮瀾口,其時家河又東,古縣河入,至縣北,納清洋及大姑、道平河。之罘島即轉附。又東,煙臺,明奇山所,今東海關,同治二年,登萊青道徙駐。奇山所巡司,孫夼鎮廢司。商埠。棲霞簡。府東南百五十里。西北:艾山。北:白山。東:岠嵎,即書嵎夷,地理志「矱有居上山,聲洋、丹水所出」。今靈山,丹今清陽,西逕翠屏山,屈東北,左納清漣水。其東大姑河自萊陽入,右會安濬河從之,入福山。聲洋今楊攕河,南逕釜甑山,會西源郭落山水。其西方山,縣河出,左會觀裏河,其東蛇窩河、陶漳河,並南入萊陽。西北:榆林河,入黃。招遠簡。府西南百四十四里。東北:羅山。潁門河出雲屯。西南:齊山。北:烏喙城。東:滾泉。西北際海,自掖迤東,萬盛河入。又東,東良河口,界河入,至黃界,其東徐家甿,潁門及南欒河並從之。其南:會仙山,大沽河亦古冶水,入萊陽。西南:萬歲河,入掖。萊陽沖,繁。府南二百五十里。地理志「長廣有萊山」,今旌旗北三十里。東南:昌山。西:長清。東:倉山、福阜。西南:高麗。東北:三螺,大姑河出,入棲霞。南際海,西南自即墨迤東為五龍口。東北:陶漳河,自棲霞入,至城東南,右會楊攕、蛇窩、觀裏河,又南,右合九里,左會昌水河,為五龍河,山以氏焉。折東南,逕浮山入之,東至海陽界。西北:大沽河,自招遠入,右合夼里河,左合平南、東良河,西南分流,其東吳姑河,其西小沽自掖緣界並入即墨。縣丞駐姜山集。寧海州沖,繁。府東少南二百六十里。順治十六年省寧海衛入之。東:盧山、九佛。東南:大昆侖。西南:鐵官。東北:金山。南北際海。西北有福山;迤東為龍門港口,辛安、七里河入。其外栲栳島。又東,戲山口,沁水河入,至小河口,龍泉河入,至文登界。西南自海陽迤東為浪暖廢口,黃壘河入,亦至文登界。西南:安濬河,入棲霞。水厥港、夏村局河並入海陽。湯泉鎮。文登沖,繁。府東南三百三十里。雍正十二年省威海、靖海二衛入之。城東:文登山。西:紫金、綠山。東南:斥山、石門、牛仙。西南:馬鞍、岊山。南北際海,西北自寧海迤東為鹿門口,羊亭河入。又東,楮島。其內,威海衛。衛東,劉公島。折南至榮成界,招阜河從之。西南自寧海迤東為姚山口,木渚河匯送駕、古橋諸河入。古橋合小河,古昌水,漢故昌陽在焉。又東,望海口,高村河入。其南,靖海衛。衛北,鐵槎山。其西,五壘島。其東南,蘇門島。東北,延真、琵琶島。溫泉鎮。文登營。威海、靖海二巡司。榮成簡。府東四百六十里。明洪武置成山衛及尋山所。順治十二年,所省入。雍正十二年改置。成山,東三十里。其麓召石,即朝鳷。南:龍山。西南:潯山。三面際海,北自文登而東為渤海青島,納柘埠河,不夜河亦入焉。有雞鳴島。又北,東海驢島,為龍口、崖口。迤西南,榮盛澳。西南:尋山所,納沽河為卸口。又西南,寧津所。其南,鏌邪島,至文登界。鹽灘石島巡司。租界。海陽簡。府東南二百二十里。明洪武三十一年置大嵩衛及海洋所。順治十二年省入。雍正十二年改置。東:岠嵎山。西:昌山,北:嵩山、林寺。西北:觀山。海,城南二里,自萊陽而東為沙家口,白沙河入。有魯島、牙島。又東,紀甿河入。有泥島、土阜島。至城南,為老龍頭,納水厥港河入,為劉格莊河。其東,草島嘴。其南,千里島。又東北,大小竹島、小青島。乳山口納夏村河、局河。又東,綿花島。又東北,宮家島、腰島,至寧海界。西北:觀山,古觀水出,俗廢。發城河,今入萊陽。行村寨有巡司。

萊州府:沖,繁。舊隸登萊青膠道。明,領州二,縣五。光緒三十一年,膠直隸,割高密、即墨。西距省治六百八十里。廣二百九十里,袤四百三十里。北極高三十七度十分。京師偏東三度四十二分。領州一,縣三。掖沖,繁。倚。掖山,東二十里,今大基,掖水出,逕城南,合三里河。又二十里,崮山。西北:斧山。府北際海,西北自昌邑迤東為海滄口,其膠萊北河入,有土山,迤東浞河入,古過國在焉。又東,白沙、英村、果村河、掖水、淇水、蘇郭河入。又東,太平灣口,龍王河入。有小石島。其西,芙蓉島、古傅巖。其東,三山島口,納萬歲河。其西岸,萬里沙。又東,硃橋河入,至招遠界。地理志「曲城陽丘山,冶水出」。左傳尤水即小姑。其西,硃東河,並入平度。沙丘城。海滄鎮。縣丞駐硃橋。柴胡、滄海二廢司。飛霜驛。平度州簡。府南百里。雍正十二年削所領濰、昌邑。東:六曲山。北:公沙、天柱、大澤。明堂,白沙河出,南逕魚脊山,至分水口,東為膠萊南河,左合雲河、落藥河,入州。東北:小沽河,自掖緣界合硃東河,墨水從之。西為膠萊北河,緣界合現河、龍王、韓村、藥石河,西北入掖。二鎮:亭口、灰埠,州同駐,又驛。濰沖,繁,難。府南少西二百五十里。西南:程符山。西:黑山。海,北百里,自壽光而東,堯丹河入。其桂河入,會大於、白狼,及孝義河入。又東至昌邑界。東南:塔山,溉水出。又東,寒浞河、瀑沙、浮塘、張固河。又東,濰水,自安丘緣界,左合汶河,並入昌邑。固底鎮巡司。古亭驛。昌邑沖,繁。府南百十里。城東:東山。南:陸山。海,北五十里,自濰而東,其寒浞、瀑沙、浮塘、張固四河並入焉。又東,安丘、濰水入。又東,膠萊北河自平度緣界,逕密阜。其西,漢故下密密鄉在焉。又北,逕狗塚山,合媒河入,至平度界。夏店驛。鐵路。

青州府:沖,繁,難。登萊青膠道治所。副都統駐。領安東衛,州一,縣十三。雍正中,莒直隸,割蒙陰、沂水、日照,尋降並屬沂,增置博山。乾隆七年衛省。西距省治三百三十里。廣二百七十五里,袤三百九十里。北極高三十六度四十五分。京師偏東二度十二分。領縣十一。益都沖,繁,難。倚。東:箕山。西:金山。西北:堯山。西南:淄水自博山入,右合仁河,東北逕稷山,其西時水,並入臨淄。又西,澇淄河,入新城。西南:石膏山,與城南雲門並,即逢山。水經注「洋水出其東南,入臨朐」者。石溝水亦曰石膏,其東北貫城者曰南陽水,右合建德水,東南巨洋水,今洱河,自臨朐入合之。折東北,右納康浪水、洗耳河、堯水,其西躍龍河,地理志「為山,濁水出」,俗北陽河,逕高柳村,並入壽光。縣丞駐金嶺鎮。青社驛。博山簡。府西南百八十里。明兵備副使治所。雍正二年改置,割淄、萊地益之。博山東南五十里,岳陽城。東:荊山。西南:原山;長城嶺,隴水出,水經注「古袁水」,合白洋河,北逕城西,合倒流泉、沙溝河入淄川。南有淄河,自萊蕪入,東逕石馬山、萊蕪谷,迤北,右合泉河,聖水,出金雞山口,入益都鎮。臨淄簡。府西北五十五里。西南:弇中峪。西:葵丘。南有淄水自益都入,水經注,逕牛山西,其北營丘,東得天齊水口,又北逕管仲墓,至城東,逕雪宮西高敬仲墓東入樂安。西南:時水自益都分入而合,北逕杜山,右合澅水及系水、京水,又北折西,一曰烏河,即乾時,入新城。西北:澠水,亦漢溱,分入樂安、博興。周禮「其浸菑時」。東南:鼎足山,地理志「菟頭,女水出」,北逕故酅亭,伏,至漢東安平故城復出。博興簡。府西北百十里。西南:小清河、支脈溝自高苑入,並淤。今自馬踏湖納新城澇淄水,左得小清故道,故亦曰小清河。又東為會城泊,水經注「平州坑右納漢溱」,即澠水,出為預備河,並入樂安。冶城河,堙。有純化鎮。高苑簡。府西北百五十里。南有小清河,自新城入,至軍張閘,右得故道,左為支脈溝,俗岔河,東入博興。田鎮,橫所居。樂安沖,繁。府北少東九十里。海,東北百三十里,自利津迤東南為淄河口,有小清、支脈溝自博興入之。一故道至壽光界,今小清入。西北其富民河亦入焉,右會淄水,緣其界,其女水入,分流折東並入之。西南:澠水自臨淄錯入,亦仍入之。樂安、高家港鎮。塘頭寨。壽光沖,繁,難。府東北六十里。海,東北百四十里,自樂安迤東南為淄河門。西南,清水泊,古鉅定,有鹽城,匯益都躍龍、王欽、北陽河,臨淄女水、小清古瀆。又東南,洱河口水,南自益都入,逕劇故城,古紀國,又北逕黑家泊,又東南至濰。東南:堯河,亦自益都入,逕故樂城。南丹河自昌樂入,逕斟灌國。其桂河入,逕故樂望。廣陵、侯鎮。臨朐簡。府南少東四十五里。朐山,東二里。西:逢山。東南:大弁。西南:八旗、嵩山、大峴。其關一,曰穆陵,有巡司。地理志,硃虛東泰山,今沂山,汶水出其東,東北左合英山水,入安丘。其北虛水、西丹水並從之。巨洋水出其西北,逕月明崖,右合龍門、南丹,左略逯、冶泉,逕城東,又北逕委粟山,左納石膏水,入益都。其東康浪水、洗耳河、堯水從之。安丘繁,難。府東南百六十里。安丘山,西南十五里,即今牟山,所謂牟婁,古牟夷國。又西南:劉山、峿山、書院。東北:擔山。南有峿水,自莒入,左合淇河、雹泉河,東入諸城。東南:濰水,自諸城入,逕礪阜山,左合小峿水,側有蓋公山,又北緣昌邑界復入,逕岞山入濰。汶水西自臨朐入,左合金山,右合牛沭山水,水經注「東北逕漢故郚城北、管寧塚東、孫嵩墓東、柴阜山西」,今右合靈水,又側城東北逕漢故淳于,從之。鎮:李丈。景芝,縣丞駐。昌樂沖,繁。府東七十里。東:弧山。南:喬山。東南:叢角,小汶河出,入安丘。塔山,水經注覆甑,溉水出。其西,孝義河。西南:擂鼓,白狼河出,東逕後魏故營陵。東南:方山,虞河出。並入濰。其北麓,桂河出,其西麓,東丹水出,西丹水自朐來入會,逕北郝集,丹硃墓在焉。其西:堯水,自益都緣界,並北入壽光。丹河鎮。鐵路。諸城沖,繁,難。府東南二百八十里。東南:瑯邪山、雲母、烽火。南:黃山。西南:馬耳;九仙山,潮河出,會北源石河峪水,逕故梁鄉,入日照,達於海。海又東為宋家河口,距城百二十里,黃山河入。又東,徐家港,紀裏河納白馬河入。又東,崔家溜口,東南橫河東源自膠入,會西源入。其外,沐官島。又東,鴨島。迤北,齋堂島。又北,龍潭口;瑯邪臺在焉。至莒界,濰水入。西南:涓水,納白納河,自其右注之,左納西商溝河。又東逕白玉山,右合扶淇水,至城北。折北,右合盧水,地理志「橫故山,久臺出」。密水,逕巴山入高密。其東五龍河,其西長幹溝,又西浯水自安丘入,逕漢故平昌,合荊水,並從之。信陽、龍灣、普慶、芝盤鎮。南信巡司。藥溝驛。

膠州直隸州:沖,繁,難。明,萊州,領縣二。雍正中降,省靈山衛入之。光緒三十一年直隸。西距省治百里。仍所領。南:艾山、珠山、崆峒。東暨南際海,自即墨迤西南為麻灣口。北有膠萊南河,自平度入,西南膠河自諸城入,右合西源望蕩山水,逕漢故祝茲,錯復入,逕金梁鄉鎮,漢祓侯國,即東黔陬城。又逕西黔陬,左合周陽河,錯高密,合張奴水復入,逕都濼。又東南,右合碧溝,至夾河套,左會沽河。又南守風灣、雲溪河、洋水,又南黃山島、淮子口,迤西薛家島、靈山島。其北岸靈山衛,衛北徐山,又西柴湖蕩口、湘子門口,至諸城界麻灣、女姑口,外為膠州灣。光緒二十三年德人租之。鎮:古鎮、逢猛。夏河寨。靈山巡司。鐵路。高密簡。府西南一百二十里。南:王子山。膠河自州入,右合張奴水,逕都濼仍入之。北:膠萊北河自平度緣界。納五龍河仍入之。側有百脈湖,涸。西南:濰水自諸城入,左合張洋河,左長幹溝,又西北逕礪阜,鄭康成墓在焉。左納浯水,入安丘。鐵路。即墨沖,繁。府東南二百五十里。東南:不其山、勞山。西南:天室。西:高鞍。東及南際海,東北自萊陽入,為周甿口。其內鼇山廢衛。迤東南,栲栳島。巡司二。又南,田橫島、峗山口。又西,女姑口匯海口,遠西河入。其外膠澳。又西,赤島。西南:青島,至州界。北:孟沙河入平度注姑河。姑河復緣界入州,流浩河從之。


\end{pinyinscope}