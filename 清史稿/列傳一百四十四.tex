\article{列傳一百四十四}

\begin{pinyinscope}
吳熊光汪志伊陳大文熊枚裘行簡

方維甸董教增

吳熊光,字槐江,江蘇昭文人。舉順天鄉試,乾隆三十七年,登中正榜,授內閣中書,充軍機章京。累遷刑部郎中,改御史。當罷直,大學士阿桂素倚之,請留直如故。阿桂屢奉使出剿匪、治河、閱海塘、讞獄,熊光輒從。累遷通政司參議。

嘉慶二年,高宗幸熱河,夜宣軍機大臣,未至,命召章京,熊光入對稱旨,欲擢任軍機大臣。和珅稱熊光官五品,不符體制,因薦學士戴衢亨,官四品,在軍機久,用熊光不如用衢亨,詔同加三品卿銜入直。居政府六閱月,和珅忌之,出為直隸布政使。四年,高宗崩,仁宗親政,和珅伏誅。熊光言和珅管理各部日久,多變舊章以營私,大憝雖除,猾吏仍可因緣為奸,亟宜更正,上韙之。

擢河南巡撫。教匪逼境,熊光駐防盧氏,張漢潮竄商州,分掠藍田,疏請截留山東兵赴明亮軍協剿;復以張天倫竄近鄖陽江岸,謀犯豫南,調直隸正定標兵備剿。上以所見與合,詔嘉獎。尋漢潮趨雒南,遣總兵張文奇、田永桐擊走之。令南汝光道陳鍾琛扼襄河要隘,糧道完顏岱率滿營兵協防,撥壽春鎮兵五百駐樊城。請召募練兵五千,並以開封練勇千名改為撫標新兵,從之。

五年,楚匪自均州、鄖縣窺渡襄河,賴預防擊退。上念河南兵單,命直隸、山西遣兵赴援,又命添募鄉勇,熊光疏言:「河南盧、淅一帶,原有鄉勇萬餘,而賊竄自如。凡游民應募,賊至先逃,反搖兵心。是以上年撤勇添兵,賊未敢肆,此兵勝於勇之明驗。今有直隸等省官兵,擇要駐守,已足策應,無庸募勇。」七月,殲寶豐、郟縣潰匪於彭山,教首劉之協遁葉縣就擒,予議敘。

六年,擢湖廣總督。途遇協防陜西兵二百餘人,逃回本營,廉得其缺餉狀,杖首謀者二人,餘釋不問。房縣鄉勇糾搶民寨,縛送三十餘人,立誅之。提督長齡、巡撫全保率師防剿,迭敗湯思蛟、劉朝選等。川匪擾興山、竹谿、房縣,分兵追剿,殲獲甚眾。平樊人傑餘匪,俘賊首崔宗和。上以熊光調度供支,迭詔褒獎。新設湖北提督,改移鄖陽鎮協,添兵三千五百名,即以無業鄉勇充之。又奏定稽查寨勇章程,略言:「寨勇習於戰鬥,輕視官兵,流弊不可不慮。今將寨堡戶口、器械逐一登記,陽資其力以助此日之軍威,默挈其綱以弭將來之民患。」上韙其言。七年,三省匪平,加太子少保。遣撤鄉勇,以叛產變價給賞,詔嘉其撙節。

九年,劾湖南巡撫高杞違例調補知縣,杞坐降調。未幾,侍郎初彭齡劾熊光受沔陽知州秦泰金,及兩淮匣費,上詰彭齡,以得自高杞對。命巡撫全保按驗無跡,彭齡、杞俱獲譴。傳諭熊光返躬自省,平心辦事,戒勿躁妄。

十年,調直隸。時兩廣總督那彥成與湖廣總督百齡互訐,命偕侍郎托津赴湖北按之。百齡被訐,事有跡。方鞫治,未定讞,那彥成亦以倡撫洋盜逮京,調熊光兩廣總督。會直隸官吏勾通侵帑事發,歷任總督籓司俱獲譴。上以熊光任籓司無虛收,任總督無失察,特詔嘉之。

十三年八月,英吉利兵船十三艘泊香山雞頸洋,其酋率兵三百擅入澳門,占踞砲臺,兵艦駛進黃埔。熊光以英人志在貿易,其兵費出於商稅,惟封關足以制其死命;若輕率用兵,彼船砲勝我數倍,戰必不敵,而東南沿海將受其害,意主持重。逾月始上聞,言已令停止開艙,俟退出澳門,方準貿易。上以熊光未即調兵,故示弱,嚴詔切責。洋舶遷延至十月始陸續去。下吏議,褫職,效力南河。百齡代其任,疏言熊光葸懦,上益怒,遣戍伊犁。逾年,召還,授兵部主事,引疾歸。道光八年,重與鹿鳴宴,加四品卿銜。十三年,卒於家,年八十四。

熊光嘗曰:「刑賞者,聖主之大權,而以其柄寄於封圻大吏。若以有司援案比例,求免駁斥之術處之,舛矣。刑一人,賞一人,而有益於世道人心,雖不符於例,所必及也。不得請,必再三爭,乃為不負。若憂嫌畏譏,隨波逐流,其咎不止溺職而已。」當調直隸,入覲,上曰:「教匪凈盡,天下自此太平。」熊光曰:「督撫率郡縣加意撫循,提鎮率將弁加意訓練,百姓有恩可懷,有威可畏,太平自不難致。若稍懈,則伏戎於莽,吳起所謂舟中皆敵國也。」及東巡返,迎駕夷齊廟,與董誥、戴衢亨同對。上曰:「道路風景甚佳!」熊光越次言曰:「皇上此行,欲稽祖宗創業艱難之跡,為萬世子孫法,風景何足言耶?」上有頃又曰:「汝蘇州人,朕少扈蹕過之,其風景誠無匹。」熊光曰:「皇上所見,乃剪採為花。蘇州惟虎丘稱名勝,實一墳堆之大者!城中河道逼仄,糞船擁擠,何足言風景?」上又曰:「如汝言,皇考何為六度至彼?」熊光叩頭曰:「皇上至孝,臣從前侍皇上謁太上皇帝,蒙諭『朕臨御六十年,並無失德。惟六次南巡,勞民傷財,作無益害有益。將來皇帝如南巡,而汝不阻止,必無以對朕』。仁聖之所悔,言猶在耳。」同列皆震悚,壯其敢言。後熊光告人,「墳堆」、「糞船」兩語,乃乾隆初故相訥親奏疏所言,重述之耳。

熊光晚年著伊江別錄、春明補錄、葑溪筆錄三書,紀所聞名臣言行,多可法云。

汪志伊,字稼門,安徽桐城人。乾隆三十六年舉人,充四庫館校對,議敘,授山西靈石知縣。除徵糧擾累,刻木為皁隸書裏分糧數,以次傳遞,民遵輸納。調榆次,遷霍州直隸州知州。代州民孟木成殺人,已定讞情實,其弟代呼冤,巡撫勒保檄志伊往按,平反之。承審者護前失,不決,命大臣臨鞫,重違眾議,志伊堅執與爭,孟木成竟得免死。志伊以此負強項名。

擢江蘇鎮江知府,調蘇州,連擢蘇松糧道、按察使。五十八年,遷甘肅布政使,調浙江。江、浙漕重積弊,由官吏規費多。志伊歷任,皆先除規費之在官者,然後以次裁革,嚴設科條。嘉慶元年,以杭州、乍浦駐防營養贍錢三月未放,被劾,議降二級調用,詔以志伊平日操守尚好,加恩授江西按察使。二年,遷福建布政使,未數月,就擢巡撫。

時海盜方張,仁宗於閩事特加意。志伊屢疏陳水師人材難得,請寬疏防處分,變通選補章程,副參以上,兼用本省之人;以下,兩省通融撥用。又州縣徵糧處分過嚴,升調要缺難得合例,請人地相需者,不拘俸滿參罰。皆允行。詔飭嚴懲會匪及械斗惡習。

五年,疏報漳、泉一帶,匪徒節經剿捕,均知斂跡。諭曰:「滋事不法,有犯必懲,不可無事滋擾。責以鎮靜,不可姑息養奸,亦不可持之太蹙。」尋奏龍溪、詔安、馬港、海澄四縣,遴員治理,民不械斗。諭曰:「一經良有司整飭,改除積習,是小民不難化導,要在親民之官得人。當於平日遴選賢員,俾實心任事,為正本清源之道。」志伊薦閩縣知縣王紹蘭,上素知其人,詔嘉志伊能留心察吏。既而偕總督玉德,疏請泉州知府錢學彬改京職,上斥疏語矛盾。尋究得學彬任聽家人舞弊婪贓事,坐察吏不明,議革任,特寬之。六年,病,請解職。

八年,起署副都御史、刑部侍郎,授江蘇巡撫。給事中蕭芝請就產米之鄉採買,由海運京,下議,志伊言其不便,罷之。九年,清江浦淤淺,糧船停滯。上慮京倉缺米,詔志伊預籌,請碾常平倉穀三千石備撥。以新漕減運,命酌量採買,志伊疏言:「安徽民田有一歲兩收者,各令七月完納漕糧,九十月可運通。江西、湖廣亦如之。」上以一歲兩徵近加賦,且來歲仍屬短絀,斥為迂繆。尋奏採米十二萬石搭運,報聞。時江北淮、揚水災,徐、海苦旱。志伊手編荒政輯要,頒屬吏為賑濟之法。蘇州人文薈萃,增設正誼書院課士。奏請頒禦制詩文集於江南各書院,上勿許,曰:「朕之政治即文章,何必以文字炫長耶?」

十一年,擢工部尚書。未幾,授湖廣總督。川、楚餘匪散匿洞庭湖,環湖數府州多盜。志伊多選幹吏偵訪,檄下分捕,盜無所匿。濱江地自乾隆末大水湮沒,民田未復。親駕小舟,歷勘疏塞,建二閘於第江口、福田寺,以時啟閉。

十六年,調閩浙總督。先是湖北應山民喻春謀殺人,其母以刑求誣服,控於京,命志伊提鞫。同知劉曜唐等誘供翻案,以無辜之葉秀承兇,而無左證。巡撫同興為之平反,奏劾。至是入覲召對,為劉曜唐等剖辯,原代認處分。上斥其偏執,嚴議革職,改留任。捕誅海盜黃治,其黨吳屬乞降。時降盜多授官,志伊曰:「是獎盜也!」仍依律遣戍。

舊有天地等會匪熊毛者,創立仁義會,授張顯魯傳煽。事覺,顯魯伏誅,毛遁,募寧化生員李玉衡捕殺之,奏賜玉衡舉人。布政使李賡蕓,廉吏也,為志伊所薦舉至監司。會龍溪知縣硃履中以不職劾,因訐賡蕓婪索,遽劾訊。履中已自承誣告,志伊固執駁詰,福州知府塗以輈迎合逼供,賡蕓自經死,輿論大譁。二十二年,命侍郎熙昌、副都御史王引之往按,得其狀,詔斥志伊衰邁謬誤,褫職永不敘用。逾年,卒。

志伊矯廉好名,自峻崖岸。仁宗初甚鄉用,時論毀譽參半焉。卒以偏執獲咎。

陳大文,河南杞縣人,原籍浙江會稽。乾隆三十七年進士,授吏部主事。典廣東鄉試,累遷郎中。四十八年,出為廣西南寧知府,擢雲南迤東道。歷貴州、安徽按察使,江寧布政使,皆有聲。父憂歸,服闋,補廣東布政使。總督硃珪薦大文操守廉潔,化其偏僻,可倚用,詔人才難得,命珪加以勸迪,俾成有用才。

嘉慶二年,擢巡撫。海盜方熾,大文以運鹽為名,集商船載鄉勇出洋,擊沉盜船六,斬獲二百餘人,賜花翎;屬縣不職者,列案劾治。詔嘉其捕盜察吏皆有實心,予議敘。尋兼署總督。

四年,調山東巡撫。濟、曹兩府水災,興工代賑,州縣玩視者立劾;有拙於催科而輿情愛戴者,疏請留任;禁漕幫旂丁陋規。五年,丁母憂。自乾隆末,山東大吏多不得人,吏治日弛。大文性深嚴,見屬吏溫顏相對,使盡言,然後正色戒之曰:「汝某事賄若干,吾悉知。不速改,彈章已具草矣!」人莫不畏之。尤銳剔漕弊,杜浮收,官吏被告發劾治者三十餘人。及去任時,其摘印在系未經奏劾者,尚七八人。事上聞,詔布政使分別省釋。

六年,畿輔大水。大文服將闋,特召署直隸總督。疏請大賑提早一月,以救災黎。劾查災開賑遲緩之縣令二人,以儆其餘。逾年,因病自乞京職,歷署吏部侍郎、工部尚書。八年,授兩江總督。劾按察使珠隆阿喜事株累,士民多怨,調珠隆阿內用。江蘇昭文浮收漕糧,江西樂平勒折重徵,縣民並走訴於京,先後下大文鞫實,劾府縣官,褫職究治。詔嘉大文秉公,不徇庇屬員,使小民含冤得白,奸胥猾吏不致幸逃法網,訓責各督撫力改積習。

九年,召授左都御史,未至,擢兵部尚書。大文赴京,病於途,詔遣侍衛率醫往視,久不痊,賜尚書銜回籍。既而因在直隸失察屬吏侵挪,部議革職,詔俟病痊以四品京堂用,遂不出。二十年,卒於家。

熊枚,字存甫,江西鉛山人。乾隆三十五年,舉鄉試第一,次年,成進士,授刑部主事。斷獄平。左翼護軍給餉誤用白片,懼責,私補印,其長當以盜印罪;枚謂知誤更正,與盜用異,改緩。宜城縣吏毆斃社長,賄改病死,擬緩;枚謂鬥毆情輕,舞文情重,改實。在部八年,多所持議,遷員外郎。尚書英廉薦其才,出為甘肅平涼知府,母憂去,服闋,補河南汝寧府。汝陽有殺人獄,已得實,控不止,枚訊鞫時,忽熟視旁吏曰:「此汝所教也!」吏色變,刑之,則稱將嫁禍某富家,咸以為神。丁生母憂,代者未至,米價騰漲,枚於喪次諭縣令治居奇者,運米接濟,民乃安。服闋,補直隸順德府,擢山東泰武臨道。

五十八年,遷江蘇按察使。逮治博徒馬修章及竹堂寺僧恆一,皆稔惡骫法者。吳江太湖濱淫祠三郎神,奸民所祀,其黨結胥吏擾民。枚廉知,值賽祠,舟集鶯脰湖,密捕得三十八人,或以誣良訴,尾其舟,得盜贓,並逮劇盜九人,毀三郎像火之,盜遂息。教匪劉之協傳彌勒教,入教者給命根錢。安徽民任梓家供彌勒像,有簿記六十人奉錢數,官吏捕得,指為匪,巡撫已上聞,逮至江南,枚親訊,六十人皆任梓戚友賀婚嫁者,乃得釋。六十年,遷雲南布政使,以治劉河工未竣,留署江蘇布政使。開蘇州城河,集銀六萬兩,擇郡紳董其役,不使縣令與工事。嘉慶二年,調安徽,尋擢刑部侍郎。

六年,直隸大水,總督姜晟以辦賑延緩免,命枚署總督。截留漕糧六十萬石儲天津北倉,枚請分儲鄭家口、泊頭諸水次,便災區輓運。條上賑恤事宜,災戶仿保甲造冊,省覆查,杜刁控,酌量變通賑期,捐賑者分別旌賞,各學貧生給口糧,綠營兵丁給修房價,修災縣監獄,以工代賑,並如議行。偕侍郎那彥寶築永定河決口,既而調陳大文為總督,詔枚受代後專任查賑,巡閱數十州縣,舉者五人,劾四人。玉田令倪為德清而戇,枚初至,怒之,明日詰賑事,指畫悉中,即首薦。上嘉枚勤事,擢左都御史。時有劾枚擾驛需索供應者,命陳大文察訪,白其誣,且言枚盡心賑務,特詔褒之。

七年,回京典會試,復署直隸總督,授刑部尚書。調左都御史,管理三庫。十年,授工部尚書,復命署直隸總督,率布政使裘行簡清查虧空。部議各省販鐵,官為定額,疏上。枚面陳鐵為民間日用所需,不能預定多寡,官為查辦,恐滋流弊。上俞其說,而斥枚隨同畫諾,召對忽有異詞,年老重聽,不宜部務,復調左都御史。未幾,有山東民婦京控應奏,枚意未決,左副都御史陳嗣龍劾枚模棱,且言枚聲名平常,詔斥嗣龍見枚左遷,揣測妄劾,終以枚不能和衷,鐫級留任。直隸籓司書吏偽印虛收庫銀事覺,坐失察,議褫職,詔以四品京堂用,補順天府丞。次年,充鄉試提調官,冊券遲誤,降五品職銜休致。十三年,卒。

裘行簡,字敬之,江西新建人,尚書曰修子。乾隆四十年,賜舉人,授內閣中書,充軍機章京,遷侍讀。四十九年,從大學士阿桂剿甘肅石峰堡回匪,復從察治河南睢州河工。五十年,出為山西寧武知府,調平陽,因親老,自請改京秩,補戶部員外郎,仍直軍機。累遷太僕寺少卿。

嘉慶六年,命赴陜西犒軍,時經略額勒登保駐略陽,行簡疏言:「川、陜兵宜扼沖嚴守,使陜匪不入川,川匪不入陜,然後逼使東竄,經略以大兵蹙之,可計日梟縛。」又言自寶雞至褒城,棧道卡兵宜復設。且於要害設大營,隔賊路,通糧運。又以額勒登保方引嫌,自請舉劾止及於麾下,行簡疏請五路將士皆聽舉劾,移書川督勒保,陳廉、藺相下之義,兩帥大和。途次,進太僕寺卿,賜花翎。尋出為河南布政使,丁母憂,服闋,補福建布政使。

自乾隆末授受禮成,恩免廢員,各州縣錢穀出入,益滋糾葛,行簡銳事清帑,司冊目十有一,創增子目,支解毫黍皆見,吏不能欺。九年,入覲,會仁宗欲清釐直隸倉庫,嘉其成效,特以調任。行簡澈底清覈,逐條覆奏,略曰:「直隸州縣,動以皇差為名,藉口賠累。自乾隆十五年至三十年,四舉南巡,兩幸五臺,六次差務,何以並無虧空?四十五年至五十七年,兩舉南巡,三幸五臺,差務較少,而虧空日增。由於地方大吏,貪黷營私,結交餽送,非差務之踵事增華,實上司之借端需索。近年一不加察,任其藉詞影射,相習成風。試令州縣捫心自問,其捐官肥己之錢,究從何出?此臣不敢代為寬解者也。分年彌補,則有二難:直隸驛務繁多,所有優缺,祗可調劑沖途,又別無陋規可提,此為難一也。現任虧空,革留勒限,彼必愛惜官職,賣田鬻產,亦思全完。若責以代前任按年彌補,焉肯解囊,勢必取給倉庫。前欠未清,後虧復至,此為難二也。州縣虧項無著,例應道府分賠;道府賠項無著,例應院司攤賠。今直隸未申明定例,請於兩次清查應行監追者,再限一年。如財產實屬盡絕,著落上司分別賠繳。嘉慶十年以後,交代虧缺,惟有執法從事,不得混入清查,致有寬縱。」疏入,上嘉其明晰,下部議行。尋命以兵部侍郎銜署直隸總督。

十一年,察出籓司書吏假印虛收解款二十八萬有奇,遣使按訊,歷任總督、布政使議譴有差。行簡任內虛收之數少,詔以事由行簡立法清查,始得發覺,寬之。是年秋,赴永定河勘工,途次感疾,卒。上深惜之,優詔賜恤依一品例,謚恭勤,賜子元善舉人。

方維甸,字南耦,安徽桐城人,總督觀承子。觀承年逾六十,始生維甸。高宗命抱至御前,解佩囊賜之。乾隆四十一年,帝巡幸山東,維甸以貢生迎駕,授內閣中書,充軍機章京。四十六年,成進士,授吏部主事,歷郎中。五十二年,從福康安征臺灣,賜花翎。遷御史,累擢太常寺少卿。又從福康安征廓爾喀。歷光祿寺卿、太常寺卿,授長蘆鹽政。嘉慶元年,坐事奪職。吏議遣戍軍臺,詔寬免,降刑部員外郎,仍直軍機。遷內閣侍讀學士。從尚書那彥成治陜西軍務。

五年,授山東按察使,遷河南布政使。時川、楚教匪未靖,維甸率兵六千防守江岸。疏言:「大功將蕆,裁撤鄉勇,最為要務。宜在撤兵之前,預為籌議。俟陜西餘匪殄盡,酌移河南防兵以易勇,可節省勇糧。」上韙之。

八年,調陜西,就擢巡撫。督捕南山零匪,籌撤鄉勇,覈治糧餉,並協機宜,復賜花翎。十一年,寧陜新兵叛,維甸亟令總兵楊芳馳回,偕提督楊遇春進山督剿。會德楞泰奉命視師,賊竄兩河,將趨石泉,維甸遣總兵王兆夢擊之,勸民修寨自衛,賊無所掠。未幾,叛兵乞降,德楞泰請以蒲大芳等二百餘人仍歸原伍。上責其寬縱,命維甸按治,疏陳善後六事,如議行。

十四年,擢閩浙總督。蔡牽甫殲,硃渥乞降,遣散餘眾。臺灣嘉義、彰化二縣械斗,命往按治,獲犯林聰等,論如律。疏言:「臺灣屯務廢弛,派員查勘,恤番丁苦累,申明班兵舊制,及歸並營汛地,以便操防;約束臺民械斗,設約長、族長,令管本莊、本族,嚴禁隸役黨護把持;又商船貿易口岸,牌照不符,定三口通行章程,杜丁役句串舞弊。」詔皆允行。以臺俗民悍,命總督、將軍每二年親赴巡查一次,著為例。

十五年,入覲,以母老乞終養,允之。會浙江巡撫蔣攸銛疏劾鹽政弊混,命維甸按治。明年,召授軍機大臣。維甸疏陳母病,請寢前命,允其留籍侍養。十八年,丁母憂,遣江寧將軍奠醊。未幾,教匪林清謀逆,李文成據滑縣,奪情起署直隸總督,維甸自請馳赴軍營剿賊,會那彥成督師奏捷,允維甸回籍守制。二十年,卒於家。上以維甸忠誠清慎,深惜之,贈太子少保,謚勤襄,賜其子傳穆進士。

董教增,字益甫,江蘇上元人。乾隆四十五年,南巡,召試舉人,授內閣中書。五十一年,成一甲三名進士,授編修,散館改吏部主事,累遷郎中。嘉慶四年,以道員發四川,明年,授按察使。瓘眉、雷波二銅鉛各廠,毗連夷地。奸民與爭界,焚夷巢,惈夷糾涼山生番為變,教增率兵往,議者多主剿,教增不可,廉得漢奸構釁者十一人,夷匪首事者六人,集眾誅之,夷情帖然。仁宗以教增不煩兵力,而遠夷心服,諭獎有加。尋調貴州。九年,遷四川布政使。

十二年,擢安徽巡撫。寧國、池州、廣德各屬,舊有棚民,植雜糧為業。戶部慮妨民田,議遣回籍。教增言:「棚民既立室家,難復遷徙。且所種多隙壤,於民田無損,於民食有益,第約束之而已。」從之。又言:「徽、寧等府巨室,向有世僕,出戶已久,告訐頻仍,請嚴杜妄訟,凡世僕以現在是否服役為斷;其出戶及百年者,雖有據亦開豁為良。」得旨允行,著為例。

十五年,調陜西。興安七屬,舊食河東引鹽。乾隆間,課攤地丁,其後復歸商運。地介川、楚,土鹽侵礙,運艱費重,引課多虧。教增請循鳳翔例,改食花馬池鹽,引歸民運,課按丁攤,以恤商力。又榆林、綏德、吳堡、米脂四州縣,向食土鹽,官給票銷售。前撫方維甸請用部引,以二百斤為率,凡萬一千三百餘引,民力難勝。教增規復其舊,由州縣頒發小票,每票五十斤,民皆便之。時南山善後倚漢中知府嚴如熤,能盡其才,不拘文法,歲歉請賑,逾限破例,上陳得允。

十八年,調廣東。先是百齡銳意滅海寇,曾貽教增詩云:「嶺南一事君堪羨,殺賊歸來啖荔支。」既而張保仔就撫,教增報書曰:「詩應改一字為『降』賊歸來也。」百齡愧之;至是承其後,諸降人桀驁,為閭閻害,懲治甚力,然未嘗妄殺。廣州府有死囚,值赦減等改軍而逃,獲之,論重闢,按察使持之堅,教增以律不當死,齗齗與辯,此囚卒免死。

二十二年,擢閩浙總督。先是海寇未平,禁商民造船高不得逾一丈八尺,小不任重載,難涉風濤,沿海多失業。教增以寇平已久,請免立禁限,以從民便,允之。福清武生林彌高者,健訟包糧,阻眾不納,邑令躬緝,為其黨邀奪,官役並傷,令文武往捕獲,彌高嗾其黨劫持,通縣抗徵。教增親鞫得彌高罪狀,立斬以徇,諸郡心習懼,強宗悍族抗欠者,皆輸納如額。奏入,詔嘉其能。臨海民糾眾毆差,致釀大獄。巡撫楊頀坐褫職,命教增兼權浙撫,鞫治之。漳、泉兩郡多械鬥殺人,官吏往往不能制。龍溪令姚瑩捕渠魁五人,杖斃之。巡撫疑其違制,教增曰:「刑亂國宜用重典。」優容之,悍俗稍戢。張保仔就撫後,改名寶,官至澎湖副將,時論猶指斥。教增責令捕盜,奔走海上,盜平而寶亦死。二十五年,入覲,乞病未允,道光元年,乃得請歸。二年,卒,賜恤,謚文恪。

教增有識量,強毅不阿。官四川時,力矯豪奢,崇節儉,宴集不設劇。總督勒保以春酒召,聞樂而返;亟撤樂,乃至,盡歡。嘗言「刻於己為儉,儉於人為刻」,時嘆為名言。

論曰:吳熊光忠讜任重,有大臣風。汪志伊、陳大文矜尚廉厲,或矯或偏。熊枚勤於民事,晚誚模棱。名位雖皆不終,要為當時佼佼。裘行簡、方維甸,名父之子,特被恩知。董教增有為有守,建樹閎達,蓋無間然。


\end{pinyinscope}