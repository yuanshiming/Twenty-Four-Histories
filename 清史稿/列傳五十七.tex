\article{列傳五十七}

\begin{pinyinscope}
郝浴子林楊素蘊郭琇

郝浴,字雪海,直隸定州人。少有志操,負氣節。順治六年進士,授刑部主事。八年,改湖廣道御史,巡按四川。時張獻忠將孫可望、李定國等降明,為桂王將,據川南為寇,師討之,郡縣吏率軍前除授,恣為貪虐。浴至,嚴約束,廉民間疾苦,將吏始斂跡。九年,平西王吳三桂與固山額真李國翰分兵復成都、嘉定、敘州、重慶。已而兩路兵俱敗,三桂退駐綿州。浴在保寧監臨鄉試,可望將數萬人薄城,浴飛檄邀三桂,激以大義,謂「不死於賊,必死於法」。逾月,三桂乃赴援,可望等引去。

浴在圍城中,上詔詢收川方略,疏言:「秦兵苦轉餉,川兵苦待哺,故必秦不助川而後秦可保;川不冀秦助而後川可圖。成都地大且要,灌口一水,襟帶三十州縣。若移兵成都,照籍屯田,開耕一年,可當秦運三年。所難者牛種,倘令土司出牛,撫臣與立券,豐年還其值,當無不聽命。嘉定據上游,饒茶、鹽,令暫易穀種,則牛、種俱不難辦也。臣故謂開屯便。川所患者滇寇也,滇寇所恃,不過皮兜、布鎧、鳥銃、濩刀,善於騰山逾嶺。蜀中土官土兵,其技尤嫻於此。若拔其精銳為前茅,以滿洲驍騎為後勁,疾雷迅霆,賊必鳥獸散。臣故謂用土兵便。」上以其言可採,下部議。部議謂戰守事當聽三桂主之,遂報寢。浴又言:「土賊投誠,給劄授官,恣行劫掠為民害。請嗣後原歸伍者歸伍,原為民者,令有司造冊編丁,免牛租,除雜派,就熟地開徵,俾有定額。」疏議行。

三桂入四川,浸驕橫,部下多不法,憚浴嚴正,輒禁止沿路塘報。浴上言:「臣忝司朝廷耳目,而壅閼若此,安用臣為?」及保寧圍解,頒賞將士,三桂以冠服與浴,浴不受。疏言:「平賊乃平西王責。臣司風憲,不預軍事,而以臣預賞,非黨臣則忌臣也。」因陳三桂擁兵觀望狀,三桂深銜之。浴劾永寧總兵柏永馥臨陣退縮,廣元副將胡一鵬驕悍不法,並命奪官逮治。降將董顯忠等以副將銜題授司道,恣睢虐民,浴復疏劾,改原職。三桂嗾顯忠等入京陳辨,浴坐鐫秩去。

十一年,大學士馮銓、成克鞏、呂宮等交章薦浴,三桂乃摭浴保寧奏捷疏有「親冒矢石」語,指為冒功,論劾,部議當坐死,上命寬之,流徙奉天。大學士馮銓、成克鞏、呂宮皆以薦浴罣吏議。浴至戍所,益潛心義理之學,嗜孟子及二程遺書,以「致知格物」顏其廬,刻苦厲志。康熙十年,聖祖幸奉天,浴迎謁道左,具陳始末,上為動容,慰勞良久。

十二年,三桂反,尚書王熙、給事中劉沛先薦浴,為部議所格。十四年,侍郎魏象樞復疏言:「浴血性過人,才守學識,臣皆愧不及。使在西蜀操尺寸之權,豈肯如羅森輩俯首從逆?臣子立朝,各有本末。當日參浴者三桂也,使三桂始終恭順,方且任以腹心。浴一書生耳,即老死徙所,誰復問之?今三桂叛矣,天下無不恨三桂,即無不憐浴。浴當三桂身居王爵,手握兵柄,不畏威,不附勢,致為所仇。三桂之所仇,正國家之所取,何忍棄之?」上乃召浴還,復授湖廣道御史。

時陜西提督王輔臣叛應三桂,浴疏言:「大兵進剿平涼,宜於西安、潼關用重兵屯駐,以待策應。用鄖陽之兵攻興安,調河南之兵入武關,直取漢中,逆賊計日可擒。」上然之,下其疏諸帥。復請禁苛徵,恤民困,止督、撫、提、鎮坐名題補之例。章十數上,皆中時弊。十六年,命巡視兩淮鹽政,嚴剔宿蠹,增課六十餘萬。淮、揚大饑,發倉米賑救,全活甚眾。十七年,擢左僉都御史,遷左副都御史。

十九年,授廣西巡撫。廣西新經喪亂,民生凋瘵,浴專意撫綏,疏陳調劑四策,請裁兵、汰馬、防要害、簡精銳;復請停鼓鑄,改米徵銀,復南寧、太平、思恩諸府縣行鹽舊制:上輒報可。時南疆底定,滿洲兵撤還京師。浴疏言撫標兵不宜裁減,下部議,留其半。又請為死事巡撫馬雄鎮、傅弘烈建祠桂林,知府劉浩、知縣周岱生為孫延齡所戕,疏請予恤。二十二年,卒官。喪歸,士民泣送者數千里不絕。

初,傅弘烈以軍事急,移庫金七萬有奇、米七千餘石供餉,浴請以庫項扣抵。及卒,布政使崔維雅署巡撫,劾浴侵欺,命郎中蘇赫、陳光祖往按,如維雅言。部議奪官追償。上知浴廉,諭所動錢糧非入己,從寬免追。二十五年,子林訟父冤,復原官,賜祭葬。

林,字中美。康熙二十一年進士,授中書科中書,歷吏部郎中,亦以廉正稱。累遷禮部侍郎,加尚書銜。致仕,卒。

楊素蘊,字筠湄,陜西宜君人。順治九年進士,授直隸東明知縣。東明當河決後,官舍城垣悉敗,民居殆盡,遺民依丘阜,僅數十家。素蘊至,為繕城郭,招集流亡,三年戶增至萬餘。山東群盜任鳳亭等剽掠旁郡,擾及畿南。素蘊設計降其渠,散其脅從。十七年,舉卓異,行取,授四川道御史。疏言:「臣言官也,宜以言為事。然今天下所患,正在議論多而成功少。國家建官分職,各有所事。誠使司舉劾,籌財用,任封疆,理刑獄,各舉其職,則平天下無餘事。更原皇上推誠御物,肅大閑,寬小眚,俾人人得展其才,尤端本澄源之要也。」

時吳三桂鎮雲南,郡縣吏得自闢署,謂之「西選」。漸乃題用朝臣,無復顧忌。素蘊疏言:「三桂以上湖南道胡允等十員題補雲南各道,並有奉差部員在內,深足駭異。爵祿者人主之大柄,綱紀者朝廷之大防,柄不可移,防不可潰。前此經略用人,特命二部不得掣肘,亦惟以軍前效用及所轄五省各官酌量題請,從未聞敢以他行省及現任京官坐缺定銜者也。且疏稱求於滇省既苦索駿無從,求於遠方又恐叱馭不速,則湖南、四川距雲南猶近,若京師、山東、江南相去萬里,不知其所謂遠者更在何方?皇上特假便宜,不過許其就近調補。若盡天下之官,不分內外,不論遠近,皆可擇而取之,何如歸吏部銓授,尤為名正言順。縱或云、貴新經開闢,料理乏人,諸臣才品為籓臣所素知,亦宜請旨令吏部簽補;乃徑行擬用,不亦輕名器而褻國體乎?人臣忠邪之分,起於一念之敬肆。籓臣易又歷有年,應知大體。此舉為封疆計,未必別有深心,然防微杜漸,當慎於幾先。祈申飭籓臣,嗣後惟力圖進取,加意撫綏,一切威福大權,俱宜稟自朝廷,則君恩臣誼兩盡其善。」疏下部。

十八年,聖祖即位,輔臣柄政,出素蘊為川北道。三桂見素蘊前奏,惡之,具疏辨,並摘「防微杜漸」語,謂意含隱射,語伏危機。詔責素蘊回奏,素蘊言:「防微杜漸,古今通義。臣但期籓臣每事盡善,為聖世純臣,非有他也。」下部議,坐素蘊巧飾,當降調,罷歸。

居十年,三桂反。尚書郝惟訥、冀如錫,侍郎楊永寧交章請起用,惟訥詞尤切,略言:「素蘊首劾三桂,云當防微杜漸。在當日反狀未形,似屬杞憂。由今觀之,則素蘊先見甚明,且為國直陳,奮不自顧,其剛腸正氣,實有大過人者!亟宜優錄。」乃命發湖廣軍前,以原品用。會丁父憂,服闋,乃赴軍前。總督蔡毓榮題補湖廣提學道,部議當以現辦軍務參議道題補。康熙十七年,題補下荊南道。時襄陽總兵楊來嘉、副將洪福等叛應三桂。大軍運餉,自襄至房、保路險■C7,舟車不通,歲調襄陽、安陸、德安三郡丁夫擔負,餉苦不繼。素蘊訪知穀城有小溪可通舟,乃按行山谷開餉道,由是水運通利,省丁夫什九,軍乃無乏。遷山西提學道。二十四年,任滿,薦舉擢通政司參議,累遷順天府尹。二十六年,授安徽巡撫。會歲饑,上疏請賑。甫拜疏,即檄州縣開倉賑給,全活甚眾。

尋調湖廣巡撫。夏逢龍亂初定,脅從尚眾,人情恇擾,一夕數驚。素蘊首嚴告訐之禁,反側以安。二十八年,大旱,疏請蠲免武昌等屬三十二州縣錢糧,上遣戶部郎中舒淑等會督撫勘災。舒淑至武昌,素蘊適患暑疾,令布政使於養志從總督丁思孔往勘。尋稱病乞休,上疑其託疾,奪官。命甫下而素蘊已卒。

先是,湖北郡縣疾苦最甚者,如沔陽、江陵、漢陽、嘉魚濱江地陷未蠲賦額,咸寧、黃陂、景陵穀折,江夏、崇陽、武昌、通城、漢陽、漢川、雲夢、孝感、應城穀田科重,監利一年兩賦,為民害數十年。素蘊得其實,條為兩疏。未及上而病革,口授入遺疏,曰:「此疏行,吾目瞑矣!」

郭琇,字華野,山東即墨人。康熙九年進士。十八年,授江南吳江知縣。材力強幹,善斷疑獄。徵賦行版串法,胥吏不能為奸。居官七年,治行為江南最。二十五年,巡撫湯斌薦琇居心恬淡,蒞事精銳,請遷擢。部議以琇徵賦未如額,寢其奏,聖祖特許之,行取,授江南道御史。時河督靳輔請停濬下河,築高家堰重堤,清丈堤外田畝以為屯田,謂可增歲收百餘萬。巡撫於成龍議不合,上令尚書佛倫往勘,主輔議。下九卿覈奏,尚書張玉書、左都御史徐乾學力言屯田擾民。二十七年,琇疏劾輔治河無功,偏聽幕客陳潢阻濬下河。上禦乾清門,召諸大臣,下琇疏,令會同察議。尋輔入覲,復召諸大臣與議。琇申言屯田害民,輔坐罷,而擢琇僉都御史。

大學士明珠柄政,與餘國柱比,頗營賄賂,權傾一時,久之為上所覺。琇疏劾明珠與國柱結黨行私,詳列諸罪狀,並及佛倫、傅拉塔與輔等交通狀,於是明珠等降黜有差。琇直聲震天下。遷太常寺卿,再遷內閣學士。二十八年,復遷吏部侍郎,充經筵講官,擢左都御史。疏劾少詹事高士奇與原任左都御史王鴻緒植黨為奸,給事中何楷、修撰陳元龍、編修王頊齡依附壞法,士奇等並休致回籍。

未幾,御史張星法劾山東巡撫錢鎯貪黷,鎯奏辨,因及琇嘗致書囑薦即墨知縣高上達等,卻之,遂挾嫌使星法誣劾,下法司訊。獄未具,琇疏言:「左都御史馬齊於會訊時多方鍛鍊,必欲實以指使誣劾罪。」詔責琇疑揣。尋法司奏琇請託事實,當奪官。上以琇平日鯁直敢言,改降五級調用。二十九年,吏部推琇通政司參議,上命改令予琇休致。江寧巡撫洪之傑以吳江縣虧漕項,事涉琇,牒山東追琇赴質。時佛倫為山東巡撫,因劾琇違例逗留希進用,請奪官逮治;又劾琇世父郭爾印乃明季御史黃宗昌家奴,琇父郭景昌原名爾標,嘗入賊黨伏法,琇私改父名請誥封,應追奪。部議如所請,逮赴江寧勘治。坐侵收運船飯米二千三百餘石,事發彌補,議遣戍,詔寬之。

三十八年,上南巡,琇迎駕德州。既還京師,諭大學士阿蘭泰等曰:「原任左都御史郭琇,前為吳江令,居官甚善,百姓感頌至今。其人有膽量,可授湖廣總督,令馳驛赴任。」琇上官,疏言:「黃州、武昌二府兵米二萬七千有奇,運給荊州、鄖陽汛地,懸隔千里,輓輸費不貲,請改折色。江夏等十三州縣有故明籓產,田瘠賦重,數倍民糧,請一律減徵。江夏、嘉魚、漢陽三縣瀕江地,水齧土陊,有賦無田者三百餘頃,請豁免。」皆允行。

三十九年,入覲,因奏言:「臣父景昌,即墨縣諸生,有冊可稽。邑匪郭爾標本無妻室,安得有子?不知佛倫何所據,誣臣並及臣父。」時佛倫為大學士,上詰之,以舛錯對,命仍予誥軸。琇陛辭,奏請清丈地畝,並言湖南地廣人稀,恐清丈後賦當差減。上問:「當減幾何?」琇言:「當減十分之三。」上曰:「果益民,雖倍於此,亦不惜也!」尋條陳三事:一,嚴定築堤處分;一,停造無用糧船;一,通融調補苗疆官吏。又疏禁徵賦諸弊政。上嘉其實心除弊,並允行。時紅苗就撫,琇陳善後之策,請頒詔敕,令勒石永遵。

四十年,以病乞休,上曰:「琇病甚,思一人代之不可得,能如琇者有幾人耶?」給事中馬士芳劾湖廣布政使任風厚久病,巡撫年遐齡徇庇不以聞。遐齡奏風厚實無病。風厚入覲,上見其未衰,因曰:「任風厚若不堪任使,郭琇豈肯徇庇耶?」未幾,琇以病劇再疏求罷,仍慰留。黃梅知縣李錦催科不力,琇委員摘印。錦得民心,民閉城拒之,乞留錦。御史左必蕃劾琇,部議當奪官,上以清丈未畢,緩之。

四十一年,鎮筸諸生李定等叩閽奏紅苗殺掠,總督、巡撫匿不以聞;而給事中宋駿業亦劾琇向騖虛聲,近益衰廢,持祿養癰。乃命侍郎傅繼祖、甘國樞,浙江巡撫趙申喬往按。會琇報清丈畢,乞罷任。上責其清丈稽延,與前奏不合,行不顧言;並及匿報紅苗殺掠與黃梅拒命事。琇自陳老病失察,請治罪。初紅苗犯鎮筸,游擊沈長祿往剿,至大梅山,守備許邦垣、千總孫清俱陷賊,長祿私贖之歸,諱不報;而副將硃紱報苗已就撫,琇據以入告。繼祖等勘得狀,琇與提督林本植並奪官。五十四年,卒。尋祀鄉賢,並祀吳江名宦。

論曰:郝浴、楊素蘊秉剛正之性,抗論強籓,曲突徙薪,防禍未形,甘竄逐而不悔。郭琇抨擊權相,有直臣之風,震霆一鳴,僉壬解體。蓋由聖祖已悟其奸,而琇遂得行其志。然以浴之廉,蒙議於身後;素蘊居官愛民,不終於位;琇則橫被誣陷,廢置十年,始獲申雪。得君如聖祖,猶不克善全,直道難行,不其然哉?


\end{pinyinscope}