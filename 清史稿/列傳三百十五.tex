\article{列傳三百十五}

\begin{pinyinscope}
屬國三

○緬甸暹羅南掌蘇祿

緬甸,在雲南永昌府騰越邊外,而順寧、普洱諸邊皆與緬甸界。順治十八年,李定國挾明桂王硃由榔入緬,詔公愛星阿偕吳三桂以兵萬八千人臨之。李定國走孟艮,不食死。緬酋莽應時縛由榔以獻,遂班師。緬自是不通中國者六七十年。

雍正九年,緬與景邁交閧,景邁使至普洱求貢,乞視南掌、暹羅,雲貴總督鄂爾泰疑而卻之。緬密遣人至車裏土司,探知景邁貢被卻,則大喜,揚言緬來歲亦入貢。旋興兵二萬攻景邁,而貢竟不至。

緬地亙數千里,其酋居阿瓦城。城西瀕大金沙江。江發源野人番地,縱貫其國中,南注於海。沿海富魚鹽,緬人載之,溯江上行十餘日,抵老官屯、新街、蠻暮粥市,邊內外諸夷人皆賴之。而江以東為孟密,有寶井,產寶石。又有波龍者,產銀,江西、湖廣及雲南大理、永昌人出邊商販者甚眾,且屯聚波龍以開銀礦為生,常不下數萬人。自波龍迤東有茂隆廠,亦產銀。乾隆十年,葫蘆酋長以廠獻,遂為內地屬,然其地與緬犬牙相錯。十八年,廠長吳尚賢思挾緬自重,說緬入貢,緬酋麻哈祖乃以馴象、塗金塔遣使叩關,雲南布政司等議卻之,而巡撫圖爾炳阿遽以聞。帝下禮部議,如他屬國入貢例。

其冬,緬使還至順寧,聞白古部酋撒翁起兵攻緬,緬兵敗,麻哈祖逃至約提朗,為白古所得,沉之江。撒翁據阿瓦五年,而緬屬之木梳頭目甕藉牙復起兵攻走白古,自據其地,令頭目播定鮓等以兵脅諸部役屬之。既而甕藉牙死,子懵洛立。未幾,亦死,弟懵駁立。

貴家者,隨永明入緬之官族也,其子孫自相署曰「貴家」,據波龍廠採銀。其酋宮裏雁不附於甕藉牙,約木邦酋攻之。兵敗,逃入孟連,而孟連土司刀派春奪其孥賄,為宮酋妻囊占所襲殺。雲貴總督吳達善誘宮裏雁至,則坐以擾邊罪,肆諸市。而木邦酋罕莽底亦兵敗走死,懵駁立其弟罕黑。由是緬人益無忌。

明萬歷時,巡撫陳用賓因永昌府近緬,設八關控之。八關者,萬仞、巨石、神護、銅壁、鐵壁、虎踞、天馬、漢龍也。其實八關皆無險厄可守,山箐間小徑往往通人行。自永昌迤邐而南為順寧,又南為普洱,其邊袤亙蓋二千餘里。永昌之盞達、隴川、猛卯、芒市、遮放,順寧之孟定、孟連、耿馬,普洱之車里,數土司外,又有波龍、養子、野人、根都、佧佤、濮夷雜錯而居,非緬類,然多役於緬。土司亦稍致餽遺,謂之「花馬禮」,由來久矣。暨緬人內訌,禮遂廢。甕藉牙父子欲復其舊,諸土司弗應,乃遣兵擾其地,而普洱獨先有事。

二十八年,劉藻為雲南巡撫,額爾格圖為提督。是年冬,緬人先遣刀派先之兄刀派新自阿瓦還至孟連,徵索幣貨,又遣頭目卜布拉、木邦罕黑至耿馬責其禮。普洱之十三板納者,本車裏土司地。雍正七年,鄂爾泰總督云南,招降之,始割其地置府。至是,緬人亦來索米。永順鎮總兵田允中、普洱鎮總兵劉德成、知府達成阿檄土司各率兵御之,殺其頭目卜布拉、召罕標等,餘眾潰走。

孟艮本緬屬,距普洱千餘里,土司召孟容與弟召孟必不相能。召孟必之子召散譖召孟容於緬,緬人執之,其子召丙走南掌。尋入居於十三板納之孟遮,召散因令素領散聽、素領散撰、素領黨阿烏弄等犯打樂,分侵九龍江橄欖壩,車裏土司遁去,賊入據其城。總督劉藻檄大理順寧營兵七千往剿,游擊司邦直先進,為賊人所圍。會參將劉明智至,夾攻破之,乘勝復車裏土司城。進攻猛籠、猛歇、猛混、猛遮諸壘,連破之,然賊往往竄伏屯聚,未肯即退。藻議益以曲尋、楚姚兵二千,未至,而參將何瓊詔、游擊明浩等聞猛阿為賊所攻,遽率兵過滾弄江,束器械以行,不設備,入山遇賊,兵敗,詔論斬。時乾隆三十年也。

三十一年正月,詔大學士楊應琚自陜甘移督云南,降劉藻湖北巡撫,藻自刎死。是月己亥,應琚至雲南,楚姚鎮總兵華封已平打樂,猛臘參將哈國興已平大猛養,合剿孟艮,召散遁,官軍得其城。而劉得成與提督達啟及參將孫爾桂攻整欠,亦克之。普洱邊外悉平。

叭先捧者,車裏土司之所屬,蓋微者也。顧與其妻咸以從軍自效,斬素領散撰於小猛侖,素領散聽亦為其妻殺死。應琚乃請以召丙居孟艮,叭先捧居整欠,均授以指揮,使守其地。時提督李勛方至雲南,應琚令往孟艮、整欠正經界,定賦稅,附入版圖,為久遠計。然召丙為人懦,不能安輯其人;叭先捧不敢至整欠,退棲於猛搿。四月,召散之黨召猛烈、召猛養以次被獲,其弟僧召龍亦自投首,惟召散逋逃未得。

應琚見夷人之易於摧殄也,遂上奏云:「臣兩月以來,訪問召散跡,逃往阿瓦,已飭土司繕寫緬文索取,不獻,當即興師問罪。臣查緬甸連年內亂,篡奪相尋,實有可乘之會。臣謹選人潛往阿瓦,將地方之廣狹,道路之險夷,詳悉繪圖,探明奏報。現已備可調之兵,布置練習,密修戎器,以待進行。」疏入,帝諭曰:「應琚久任邊疆,必不至輕率喜事。如確有把握,自可乘時集事,剋日奏功。倘勞師耗餉,稍致張皇,轉非慎重籌邊之道。務須熟計兼權,期於妥善,以定行止。」

是時諸將希應琚意,爭言內附。李勛以猛勇、猛散告,劉德成以猛龍、補哈告,華封以整賣、景線、景海告,率侈言夷地廣輪或二千里,或二千餘里,為邊外大都。應琚一一奏聞,以其頭目為千總、守備。緬寧通判富森言木邦人殺緬立土司罕黑,奉線甕團為主,原求內屬。永昌知府陳大呂亦言蠻暮土司被緬殘虐,久原歸誠,請發兵為助。應琚乃往駐永昌,而遣副將趙宏榜將永順、騰越兵三百餘人出鐵壁關屯新街,為蠻暮捍蔽。宏榜抵關,遇大呂所遣使,羈之,而自受蠻暮土司瑞團降。大呂恚,訴應琚,應琚曲解之。是時騰越知州陳廷獻招猛育、猛英、猛密,陳元震招戛鳩、允帽、結此夕,富森招佧佤,而宏榜又招孟養、乃壩竹、孟岳十六寨諸夷,先後遣人來約降應琚又為文檄緬,侈言天朝有陸路兵三十萬,水路兵二十萬,陳於境以待速降,不然則進討。緬聞,乃大出兵。緬人素不養兵,有事則於所屬土司諸寨籍戶口多寡出夫,名曰「門戶兵」。自甕藉牙據阿瓦,蓄勝兵萬人,一人給以餉四十兩,其餘派夫如故。每戰則以所派土司濮夷居前,勝兵督其後,而以馬兵為左右兩翼。戰既合,兩翼分繞而進,往往以此取勝。若自度不可勝,則急樹柵自固,而發連環槍砲蔽之。比煙開則柵木已立,入而拒守。其兵法如此。

九月,賊先以兵出落卓攻木邦,線甕團不能守,入居遮放,又以兵溯江而上,抵新街。宏榜相持兩日,勢不支,燒其器械輜重及傷病之兵,退回鐵壁關駐守,而蠻暮土司亦偕其母走入內地。

應琚憂甚,痰疾遽作,詔兩廣總督楊廷璋赴滇,代治應琚軍,並廉宏榜兵敗狀。又遣侍衛傅靈安挾御醫診應琚病,又命其子江蘇按察使重英、湖南寶慶知府重穀赴滇省視之。

應琚所調兵一萬四千名將集,令永順鎮總兵烏爾登額駐宛頂進剿木邦,永北鎮總兵硃侖由鐵壁關進駐新街,而令提督李時升在杉木籠山居中調度。侖至楞木,突遇賊,戰四晝夜,賊退走,追擊之。懵駁之弟卜坑及其舅莽聶渺節速詭求和,言原頂經吃咒水。頂經者,以經加於首,咒水者,取水咒之,分與其眾飲,蓋夷人盟誓之禮也。議未定,賊已擁眾越神護、萬仞關,入掠盞達,圍游擊馬拱垣於盞達江上,分兵入戶撒,游擊邵應泌亦被圍。劉德成在乾崖有兵二千人,坐視不救。時升因檄侖還守鐵壁。又聞賊欲從庫弄河出關後,侖復引兵卻,駐守隴川。賊勢張甚,應琚數以檄促德成,始擊賊於銅壁關下,破之。賊自西而東趨隴川,德成亦由戶撒擊其後;時升又檄烏爾登額帥宛頂兵至邦中山,以助聲勢;於是軍威稍振。賊人見大兵之集也,復來乞降,侖以報應琚,命許之。

賊伺我軍懈,遂走犯猛卯。猛卯與木邦親,木邦之降,猛卯實左右焉。賊怨,故蹂躪之。時三十二年正月丙寅朔也。副將哈國興帥兵二千五百人趨猛卯,比至,見賊勢盛,乃入城與土司堅守。賊攻城,緣梯而上,城上矢砲交發,賊不敢近。圍八日,癸酉,副將陳廷蛟、游擊雅爾姜阿各以兵至,城中出合擊之,賊大潰;而烏爾登額久不至,故賊得浮猛卯江而逸。硃侖乃造浮橋過宿養渡,由景陽、暮董偕烏爾登額進剿木邦。是月丁丑,楊廷璋至軍,見賊未易遽平,遂奏言應琚病已痊,臣當歸粵。帝召廷璋還京師。

時賊入關侵擾,應琚皆不以聞,僅言硃侖殺賊幾萬人,賊震懼,乞降,欲以新街、蠻暮與之;而時升亦言猛卯之捷,誅其大頭目播定鮓、皮魯布。奏入,帝視應琚所進地圖,用藍筆分中外界,而猛卯、隴川均在藍線內,疑之,以為如果殲賊萬餘及大頭目,賊當遁走不暇,何以硃侖展轉退卻,賊敢蔓延內地土司之境?降旨駁詰。而傅靈安先奉詔廉訪軍事,具言趙宏榜棄新街,硃侖退守隴川,及李時升未經臨敵情事,與帝所駁詰者悉合。應琚復劾劉德成、烏爾登額逗留貽誤。於是逮李時升、硃侖、劉德成、烏爾登額、趙宏榜,而晉楊寧為提督。且以應琚欺罔乖謬不能任事,乃召明瑞於伊犁,以將軍督軍雲南,遣額爾景額為參贊大臣,徙巡撫湯聘於貴州,以鄂寧代之。

上年冬,緬人已據整賣、景線,召散遂率以攻孟艮,召丙懼,出奔,賊延入打樂,思茅同知黑光以聞。時湯聘未聞上命,楊重英方至自江蘇,乃偕赴普洱,奏言總兵華封、寧珠安坐普洱,失剿御,請革職治罪。奏入,華封、寧珠與游擊權恕、司邦直,都司甘其卓皆被逮,調開化鎮總兵書敏總統進剿。頃之,鄂寧亦至普洱,奏言:「上年九龍江外兵馬以瘴死者不可勝數,官弁夫役死亦大半。此時正盛瘴發生,湯聘乃稱嚴飭將卒,克日進剿,懷詐塞責,實無誠款。」奏入,湯聘以革職逮治。應琚見前所招撫土司復陰附緬,其土司頭目夷人千百為群,皆蕩析離居,而緬賊時出沒為患,邊事日棘。鄂寧復奏應琚貪功啟釁,為硃侖等諱飾,又不令湯聘、傅靈安與聞邊務,及隱沒游擊班第、守備江紀陣亡各狀。應琚懼,乃奏請是秋大舉征緬,調兵五萬,五路並進,兼約暹羅夾攻。帝下其議,廷臣皆斥之。詔逮應琚至京,賜死。

四月,明瑞至永昌。時楊寧壁軍木邦,餉道為賊所斷,潰退滿河。永北鎮總兵索柱及烏爾登額亡其印信。明瑞以聞,楊寧亦被逮,調譚五格為提督,詔派八旗兵三千、四川兵八千、貴州兵一萬、雲南兵四千,赴邊進討。綠營馬匹皆本營預備,惟八旗兵三千人,每兵例需馬三匹,合官員所用,計馬幾萬匹。明瑞議撥廣西馬一千、廣東馬八百、四川馬五千八百、貴州馬六千、湖南馬二千,每兵裹兩月糧,計六斗,馱以一馬。馬、驢少,購牛代之。糧不足,可殺牛以抵。共用驢、馬、牛八百餘。其糧於大理、鶴慶、蒙化三府撥六萬石,又於永昌、順寧買三萬石。兵行之道,自宛頂、木邦進者為正兵,明瑞身統之。烏爾登額、譚五格則由猛密分進。至新街,水路,時方暑雨,難造舟,宜削木★O2沿江流下,疑賊以牽其勢。奏入,帝嘉之,悉從其議。

九月,諸路兵皆至永昌,馬、牛亦集。甲寅,明瑞率軍啟行。值大雨,潞江舟少,以次待渡,而溝路陰仄,輜重壅塞於道,軍士立雨中竟夕。十月甲申,抵帕兒,帝復遣參贊大臣珠魯訥至軍,而參贊大臣額爾景額、楚姚鎮總兵國柱相繼病歿。賊偵知,毀津渡橋梁,且伐大樹撲之。又雨多道壞,軍行遲滯,明瑞乃選銳兵一半,帥以先驅。領隊大臣觀音保由孟谷出木邦之右。十一月丙戌,抵木邦城。賊先挾夷民以去,獲其糧貯,留珠魯訥以兵四千守之。進至錫箔江,江寬,架橋以渡。行四日,至天生橋,橋南有賊砦相偪。會商人馬子團言橋之東三十里水淺可涉,且岸頗平,乃以兵繞出其後。賊復棄砦去,遂進至蠻結。賊依山立十六柵以待。明瑞抵柵下,親冒槍砲督兵進攻。觀音保麾眾先據山左。哈國興等三路登山,俯薄之,呼而逼其壘。貴州步兵王連睨柵左有積木,躤之以登,躍入柵內,八十餘人繼之。賊恇亂,莫知所措,多被殺,遂破其一柵。旋復攻破三柵,而十二柵之賊悉乘夜潛遁。捷聞,晉封明瑞誠嘉毅勇公,以恩澤侯與其弟奎林,特擢王連為游擊,餘俱交部敘功。

然夷境益峭險,其草率綠竹、王芻之屬,馬乏食,多致斃,而牛行遲滯,箠之以登,死者尤眾。賊燒其村寨,斂積貯而窖埋之,掠食無所得,軍糧垂竭。進至象孔,迷失道。明瑞度不能至阿瓦,約烏爾登額等軍由猛密入。其地近孟籠,有緬屯糧,且可與猛密軍相合,乃議向孟籠,果大獲糧;而烏爾登額等趨猛密,出虎踞關,聞老官屯有賊,意輕之,先率眾往攻。賊固守,弗能下,軍士多傷亡,陜西興漢鎮總兵王玉廷亦中槍卒。

珠魯訥守木邦,有夷數十人來降,疑其偽,悉誅之,而遣索柱等往錫箔江設臺站,以通明瑞軍信息。索柱等至蒲卡,聞賊至,以兵少,退守錫箔,賊躡之,戰歿。賊遂附木邦城下,絕營南水道,糧運之從宛頂來者,賊又截之,軍士皆饑渴,火藥亦盡,賊審其困,佯為好語求和,珠魯訥不得已,遣楊重英及守備王呈瑞往報,賊人留之,且誘軍士出汲,斷其後,皆不得還。三十三年正月,益兵攻城。丁未夜,兵亂,珠魯訥自剄死,普洱鎮總兵胡大猷亦歿。賊之圍木邦也,珠魯訥屢促鄂寧救援,而永昌兵盡行,無可調發。已而促之急,始令游擊袁夢麟等率駐宛頂兵三百人以往,遇賊,皆不知所之。知府陳元震、郭鵬翀持參贊印先三日逸出,鄂寧捕得之,磔死。

明瑞既就糧孟籠,諜知烏爾登額未至猛密,而諜者報大山土司瓦喇遣弟羅旺育來迎,且率其子阿隴從軍;而緬自去冬象孔改道後,獲官軍病卒,知糧盡,不向阿瓦,即悉眾躡官軍後。官軍且戰且行,每日先以一軍拒敵,即以軍退至數里外成列,待軍至,則成列者復迎戰。明瑞及觀音保、哈國興更番殿後,步步為營,每日行不三十里。正月丙午,至蠻化,營於山巔,賊即營山半。明瑞曰:「賊輕我甚矣,不一痛創之不可!」時賊識官軍軍號,每晨吹波倫者三而起行,賊亦起。次日五鼓復吹波倫三,乃盡出營伏箐中以待。賊聞波倫聲爭上山來追,萬槍突出,四面兜擊,賊潰墜者趾頂相藉,坑穀皆滿,殺四千餘人。

明瑞休軍蠻化數日,取所得牛馬犒士。又自蠻化至邦邁、虎布、蠻移、小天生橋、僮子壩,大小數十戰,永順鎮總兵李全歿於陣。又稍稍聞木邦失守。明瑞恥是役之無功也,二月己未,至猛育,距宛頂糧臺二百里,賊蝟集數萬。明瑞乃令軍士乘夜出,而自與領隊大臣及巴圖魯侍衛數十人率親兵數百斷其後。及晨,血戰萬賊中,無不一當百。俄,明瑞槍傷於肋,呼從者取水至,飲水少許而絕。觀音保、扎拉豐阿皆戰死,死者凡千餘人。是夕也,星隕如雨,餘軍先後潰歸宛頂。

明瑞自蠻結破賊後,懸軍深入。帝久不得報,命戶部尚書果毅公阿里袞以參贊大臣赴邊援應。又聞木邦被困,命明瑞旋軍,而敕烏爾登額撤老官屯之圍,往援木邦。賊覺,扼馬膊子嶺,烏爾登額幾不得出;而自旱塔抵猛密,木邦有袤徑頗近,烏爾登額以馬盡糧乏,紆道入虎踞關,經猛卯,至宛頂,復駐軍。明日而明瑞陣亡之信已至,鄂寧劾其有心玩誤,詔逮至京磔之,並誅譚五格於市,而厚恤明瑞。其後阿里袞募人至猛育,求其尸,歸於京師以葬。是為征緬前一役。

明瑞之死也,緬人不知,震其餘威,懼再討。五月,縱所獲兵許爾功等八人自木邦持緬書來,且使楊重英、王呈瑞等言:「懵駁之母得罪天朝,欲使懵駁內附。」重英恐緬書繙譯誤,乃譯清、漢字各一通,益以木邦臘戌頭目苗溫之書。苗溫者,緬人守土官之稱。臘戌在木邦南。木邦殘破,而臘戍城在嶺下,險可守,故苗溫徙居於此。緬書云:「暹羅國、得楞國、得懷國、白古國、一勘國、罕紀國、結此夕國、大耳國及金銀寶石廠,飛刀、飛馬、飛人、有福好善之王殿下掌事官拜書領兵元帥。昔吳尚賢至阿瓦,敬述大皇帝仁慈樂善,我緬王用是具禮致貢,蒙賜緞帛、玉器諸物,自是商旅相通,初無仇隙。近因木邦、蠻暮土司播弄是非,興兵兆釁,致彼此人馬互有傷亡。茲特投文敘明顛末,請循古禮,貢賜往來,永息干戈,照舊和好。」阿里袞以聞。帝念明瑞軍入關者尚逾萬,所喪亡不過十之一二,然將帥親臣皆捐軀異域,而緬夷求款未親遣頭目,非大舉無以雪忠憤,命絕之勿報。自後緬人數以書與隴正野人及遮放土司訪問許爾功狀,皆置不答;而以楊重英偷生阿瓦,籍其家,並置其子於理。

時大學士公傅恆自請督師,乃命為經略;阿桂、阿里袞皆為副將軍,明德為總督,哈國興為提督。八月,阿桂詣熱河行在,奏言:「緬賊愍不畏死。臣至滇,當相度時勢,以正天誅,不敢鹵莽滅裂,誤軍國大事。」帝頷之。既陛辭,至襄陽,會守備程轍前從楊寧軍陷於賊,至是密以書來告,言緬人方與暹羅仇殺,可約以夾攻。帝遣人馳問阿桂,奏言:「官軍會合暹羅,必赴緬地。若由廣東往,則遠隔重洋,相去萬餘里,期會在數月之後,恐不能如期。」帝以為然。蓋自明陳用賓有要暹羅攻緬之說,楊應琚、楊廷璋先後奏上,延議雖斥之,不能釋然也。因詔兩廣總督李侍堯詢察之。侍堯奏言:「聞暹羅為花肚番殘破,國主詔氏竄跡他所,餘地為屬下甘恩敕、莫士麟分據。」花肚番者,緬人以膝股為花,故云。由是約暹羅之議始寢。

是年冬,帝念明瑞所統旗兵勞苦,命回京,復選旗兵五千人赴滇,合荊州、貴州、四川兵一萬三千人。阿里袞乃令副都統綿康、曲尋鎮總兵常青帥二千人駐隴川,侍衛海蘭察、烏爾圖納遜帥二千人駐盞達,領隊大臣豐安、鶴麗鎮總兵德福帥二千人駐遮放,侍衛興兆、巴朗帥一千人駐芒市,侍衛玉林、普爾普帥五百人亦駐盞達,侍衛恆山保、永順鎮總兵常保柱帥三千人駐永昌,廣東右翼鎮總兵樊經文帥一千人駐緬寧,荊州將軍永瑞、四川副都統雅朗阿、提督五福帥六千人駐普洱,而騰越兵一千令綿康兼轄之。防守嚴密,邊以無事。帝以緬人狡惡,思出偏師疑之,使其疲於奔命。欲出九龍江及舊小,皆不果。阿里袞乃議剿戛鳩。十一月,阿桂至永昌,聞信馳往會師討之。十二月,出關,焚數寨,殲其眾數百人,止丹山。濮夷團五卒者,率四十餘戶來降,遷之盞達。

三十四年二月,經略傅恆發京師,帝御太和殿授以敕印。或告傅恆曰:「元伐緬,由阿禾、阿昔二江以進。今其跡不可考,意其為大金沙江無疑。前鄂寧言騰越之銀江,下通新街,南甸之檳榔江,流注蠻暮,兩江皆從萬山中行,石磡層布,舟楫不可施。若於近江地為舟具,使兵扛運至江滸,合成之以入於江,下阿瓦,既速且可免運糧,而師期亦較早一二月,緬人必不暇設備。又以一隊渡江而西,覆其木梳舊巢。如此,緬不足平也。」傅恆然其言。四月丙辰,至永昌,條奏進兵事宜,皆如所議。遂遣護軍統領伍三泰、左副都御史傅顯及哈國興,率夷人賀丙往銅壁關外相視造舟地。還報野牛壩山勢爽塏,樹木茂密,且距蠻暮河一百餘里,於入江為宜。乃令常青等率兵三千人,督湖廣工匠四百六十餘,馳往造辦。又使賀丙潛行招撫。賀丙者,戛鳩頭目賀洛子也。

是役也,續遣滿洲、索倫、鄂倫春、吉林、西僰、厄魯特、察哈爾,及自普洱調赴騰越之滿洲兵,共萬餘人;又福建、貴州、本省昭通鎮兵,共五萬餘人。河南、陜西、湖廣與在省曲靖各府飼養之馬,凡六萬餘匹。益以四川工咒術之喇嘛,京城之梅針箭、沖天砲、贊叭喇、鳥槍,河南之火箭,四川之九節銅砲,湖南之鐵鹿子,廣東之阿魏,雲南省城制造之鞍屜、帳幙、旗纛、火繩、鉛藥,及鉛鐵、灰油、麻枲諸船料物,悉運往以資軍實。

乃議分路進:傅恆由江西戛鳩路,阿桂由江東猛密路,阿里袞以肩瘡未愈,由水路,都計新舊調兵二萬九千人。其由戛鳩路者,滿州兵一千五百人,護軍統領伍三泰,侍衛玉麟、納木札、五福、鄂寧、烏爾袞保,參領滿都虎、德保領之;吉林兵五百人,護軍統領索諾木策凌、侍衛占坡圖領之;索倫兵二千人,副都統呼爾起、奎林、莽克察,侍衛塔尼、布克車德、受菩薩,參領占皮納領之;鄂倫春兵三百人,侍衛成果領之;厄魯特兵三百人,侍衛鄂尼、積爾噶爾領之;綠營兵四千人,提督哈國興,開化鎮總兵永平及德福領之。其由猛密路者,滿洲兵二千人,副都統綿康、豐安、常保柱,侍衛海蘭察、瑪格、喬蘇爾、興兆、普爾普領之;索倫兵一千人,散秩大臣葛布舒,侍衛額森退領之;厄魯特兵三百人,侍衛巴朗領之;綠營兵四千人,曲尋鎮總兵常青,永北鎮總兵馬彪,楚姚鎮總兵於文煥領之。其由水路者,健銳營兵五百人,侍衛烏爾圖納遜、奈庫納領之;吉林水師五百人,副都統明亮,侍衛豐盛額領之;福建水師兵二千人,福建提督葉相德,福建建寧鎮總兵依昌阿領之。又令副都統鐵保,侍衛永瑞領成都滿洲兵一千二百人,侍衛富興、蒙古爾岱、鄂蘭、必拉爾海領西僰兵一千人,提督本進忠、臨元鎮總兵吳士勝領綠營兵二千二百人,分守驛站。又令侍衛諾爾奔領滿洲兵五百人,永順鎮總兵孫爾桂領綠營兵一千人,屯宛頂,以牽制木邦之賊。又令雅朗阿領荊州滿洲兵二千人,普洱鎮總兵喀木齊布領綠營兵一千五百人,駐守普洱。

分置略定,而賀丙往戛鳩招撫孟拱,挾其頭目脫烏猛以來。其言曰:「上年懵駁遣頭目盞拉機以千人守猛戛,需索煩重,土司畏其偪,避往戶工。孟拱人苦緬人魚肉久矣,聞大軍來,皆呀呷忻喜。請由戛鳩濟江出孟拱。孟拱米穀多,可以佐軍食。頭目歸,當集舟於江以待。」傅恆上言:「孟拱遣大頭目來,稱歸備舟以候官兵過渡。臣思野牛壩造舟之役,賊早有見聞,若於西岸設伏沿江拒我,未易渡也。今忽由戛鳩過江,先從陸路據蠻暮西岸,已出賊意計之外。且自戛鳩渡後,可將舟楫順流放至蠻暮,添備東岸官軍過渡。如造舟處有緩急,我兵在西岸,乘舟往來策應亦最便利。臣傅恆謹先統兵進發,阿里袞、阿桂偕往野牛壩督辦船工。」

癸卯,次盞達,分道行,阿里袞固請從傅恆。庚申,出萬仞關。八月癸丑,次允帽。允帽,江滸也。賀丙、脫烏猛以舟三十餘來迎。丙子,次孟拱。土司渾覺竄往節東,蹤跡之,獲其小妻並頭目興堂札,原往尋渾覺,縱之,即日偕以來,獻象四。傅恆令其人持大纛騎以先,夷人望見皆驚駭。而予渾覺銀萬兩,市牛數千頭,米數千石,以給軍。

時阿桂以七月戊申次野牛壩。舟工畢,八月乙酉,進次蠻暮。初,官兵之裹糧兩月也,議以進剿為始;而督工時仍令內地饋運,總督明德面諾之,不為具。及是,移檄往促,始令騰越州發運。泥深道遠,經月不能至。乃奏糧運遲誤狀,降明德江蘇巡撫,以阿思哈代之。九月壬辰,阿桂由蠻暮進至新街。舟成,將出江口,賊人從猛戛來逆戰,阿桂伏兵甘立寨。賊至,水陸奮擊,發巨砲,沉其舟,譟而從之,笳鼓競作,賊大沮,退走。

先是傅恆在江西,文報越兩三日輒一至,自孟拱而南,信益稀。阿桂聞蒼浦、蠻岡間有伏戎,乃募夷間道以書往訊。及伊犁將軍伊勒圖、總督阿思哈奉命皆至軍中,乃以兵二千屬伊勒圖渡江迎傅恆,並令玉麟、哈青阿率兵據西岸以待。伊勒圖渡江遇賊,擊走之,柵賊一夕皆遁去。

傅恆率十八騎,以是月戊申抵哈坎。是時緬人列船江岸,且於沙洲及林莽間樹柵以守。十月戊午,傅恆及阿桂督水師擊之,侍衛阿爾蘇納首先乘小舟沖入,眾繼進,奪其柵,獲旗纛器械無算,殲頭目賓啞得諾;而阿里袞、伊勒圖攻西岸諸柵,賊皆棄而走。丙寅,傅恆、阿桂循江東岸,伍三泰、常青循江西岸,阿里袞、伊勒圖率水師並進。丁卯,阿里袞以瘡甚卒於舟。

伊勒圖領其眾已抵老官屯。賊柵徑圍三里許,柵尾迤邐屬於江中,瀦水可泊船。柵以巨木深入土中,外周三壕,壕外橫臥大樹,銳其枝末外向,蓋其大頭目布拉莽儻所居也。西岸頭目得楞孝楞率船一百三十、兵三千,起兩柵。及夕,柵木杪皆懸火。有頃,鼓登登,雜以管籥侏離之歌,傳呼以達於江西,遠近相和,竟曉乃輟,而老官屯南巴窪、章薄賊,皆築柵以為應援。庚午,進攻其柵,經略將軍親摩壘。總兵德福中槍,逾日卒。乃令舟師絕兩柵中,下泊於柵南,斷賊江中援救。發威遠大砲,砲重三千斤,子三十餘斤,聲如奔雷,遇木輒洞以過,柵不為塌。又改用火攻之法,先以桿牌御槍砲,眾挾膏薪隨之,百牌齊進,逾壕抵柵;而江自四更大霧起,迄平旦始息,柵木沾潤不能爇,兼值反風,遂卻。又取生革為長鐶鉤之,力急鐶輒斷;乃伐箐中數百丈老藤,夜往鉤其柵,役數千指曳之,輒為賊斧斷。總兵馬彪乃闕隧窌藥其中,深數十丈,藥發,柵突高起丈餘,賊號駭;俄柵忽落平地,又起又落者三,遂不復動。蓋柵坡迤下,而地道平進,故土厚不能迸裂也。賊自巴窪、章薄來鉛丸、火藥、糧米,卒不得斷絕,是以無逃志。

然懵駁聞新街之敗,大懼,而攻圍日久,死傷者多。十一月己丑,布拉莽儻乃遣使求罷兵。明日,復以懵駁書至。傅恆、阿桂召諸將問可否,諸將皆言懵駁從阿瓦致書,非震悚誠切不出此,可借此息兵。壬辰,作檄答之,言:「汝國欲貸天討,必繕表入貢,還所拘縶官兵,永不犯邊境。如撤兵背約,明年復當深入,不汝貸也。」癸巳,緬十三頭目來議事,乃遣明亮、海蘭察、哈青阿、明仁、哈國興、常青、馬彪、依常阿、於文煥、雅爾姜阿等會議,申諭所約三事,頭目皆拱手聽命。哈國興曰:「汝國僻在海裔,不知籓臣典禮,汝入貢當具表文,文首行書『緬甸王臣某奉表大皇帝陛下』,與安南、高麗各外籓等。」其管五營頭目得勒溫曰:「謹受教。」目左右具書以歸。丁酉,陳錦布、毯百餘端,獻經略將軍,而進魚鹽犒軍。於是焚舟鎔巨砲,奏聞,以己亥班師。甲辰,進虎踞關,緬人遣頭目率六十餘人送至關上。是日奉旨以緬地瘴癘,命貰其罪,令渾覺還孟拱,而以所進四象送京師。伊勒圖、傅恆先後還京。

木邦、蠻暮兩土司走入內地後,線甕團居緬寧之海臘,丁山、瑞團居盞達之壩築,其猛密頭人線官猛亦率眾居綿川戶南山,餘遷徙無常處。及是,移線甕團於蒙化,移瑞團、線官猛於大理,各取官莊租贍之;而賀丙則從其請,居於萬仞關外之南底壩。其後又以召丙、叭先捧等分置於寧洱縣之蕨箕壩,而大山之侄阿隴、允帽頭目之女老安皆屬縣官,予以廩給。猛勇頭目召工、整欠頭目召教、景海頭目召別,咸原輸誠進獻。

三十五年二月,因緬人貢使不至,帝令毋許奸商挾貨貿遷以利緬,且漏內地消息。時阿桂還至省城,命核所用軍裝馬匹,又命總督彰寶檄斥緬入貢使遲滯狀,使都司蘇爾相持至老官屯,布拉莽儻留之。阿桂回至永昌察賊狀,三十六年三月,阿桂奏言:「蠻暮、木邦、猛密三土司外,始有緬人村落,距邊已二千餘里,偏師不可深入。若出近邊,則所殲乃濮夷野人,與緬無損。不如休息數年,外約暹羅同時大舉。」帝以大舉非計,乃罷阿桂,以溫福代之。明年,金川反,溫福、阿桂皆赴四川。而緬亦方用兵暹羅,於是暹羅滅於緬。

四十一年,金川平。時緬甸先遣孟遮等五人以書呈雲南總督圖思德,總督縶之歸京師。及是,命赴市曹觀狀,且告之故,乃縱使歸緬,而令阿桂以大學士赴永昌備邊。緬懼,請入貢,原出楊重英、蘇爾相,求開關互市。明年,出蘇爾相,而楊重英不至。

四十三年,暹羅遺民起兵逐緬人復國。五十一年,詔封鄭華為暹羅國王,於是緬益懼。五十二年,耿馬土司罕朝瑗報言:「滾弄隔岸即緬甸木邦,緬酋孟雲遣大頭目葉渺瑞洞、細哈覺控、委盧撤亞三名,率小頭人從役百餘人,齎金葉表文,金塔及馴象八、寶石、金箔、檀香、大呢、象牙、漆盒諸物,絨氈、洋布四種,懇求進貢。譯其文,稱孟雲乃甕藉牙第四子,幼為僧,懵駁其長兄也。懵駁死,子贅角牙立。孟雲次兄孟魯,以甕藉牙有兄終弟及之諭,懵駁死而子襲,非約,乃戕殺贅角牙,欲自立,國人不服,亦殺孟魯,迎孟雲立之。孟雲深知父子行事錯謬,感大皇帝恩德,屢欲投誠進貢,因與暹羅構釁,且移建城池,未暇備辦。今緬甸安寧,特差頭目遵照古禮進表納貢。」總督富綱等以聞,帝允所請,賚其使而歸之,且賚孟云佛像、文綺、珍玩器皿。五十四年,孟雲遣使賀八旬萬壽,乞賜封,又請開關禁以通商旅,帝皆從之,封為緬甸國王,賜敕書、印信,及御制詩章、珍珠手串,遣道員、參將齎往其新都蠻得列,定十年一貢。自是西南無緬患。

六十年,緬王遣使祝釐,進緬石長壽佛、貝葉緬字經、福字鐙、金海螺、銀海螺、金鑲緬刀、金柄麈尾、黃緞傘、貼金象轎、洋槍、馬鞍、象牙、犀角、孔雀、木化石、玄猴皮、各色呢、各色花布,都十有八種。時有三緬盜逸入印度,緬人以五千人追之,突入印度之勢他加境,英人領土也。英守將爾斯根詰緬人,以盜付之。嘉慶元年,緬王復遣使朝貢。總督勒保以緬使甫經回國,不宜數來,檄雲南司道拒勿納。事聞,帝諭曰:「緬甸國王以本年國慶,特遣使臣齎表備物申虔稱賀,勒保不據實奏聞,遽行拒絕,致令使臣徒勞跋涉,殊失柔遠綏懷之意。勒保交部嚴議。」命軍機大臣擬旨曉諭緬王,頒賜蟒錦四端。五年,緬甸入貢。十年冬,緬甸復遣使叩關求入貢,以是年暹羅伐緬,有敕諭暹羅罷兵故也。帝以非貢期,卻之。

時緬甸雖失暹羅,國勢猶盛。其疆域南盡南海,北迄孟拱,西包阿拉干,東聯麻爾古。又有拈人之地環其東境,舊稱九十九國,多為領屬,地廣兵強。既東失暹羅,乃西覬印度之富,時思襲取。緬西北有曼尼坡部,又西有阿薩密部,緬嘗以兵攻二部,漸有從西黑特旁侵入英領之勢。西黑特居阿薩密南,為印度孟加東北境,過此即克車部,英人所保護也。緬人恃其習戰,蔑視英人,後果侵英邊,殺英戍兵,擄其人民。又南侵入勢他加,英人以少兵守內府河口之刷浦黎島。道光三年,緬人攻守島英兵,英以眾寡不敵而潰,亡數人。英人來責言,緬置不答,益輕英。

明年,英人伐緬,水師副提督喀姆稗兒率師進厄勒瓦諦江,即大金沙江也。次仰光,緬人御諸海口而敗,英軍遂登陸攻仰光、克曼庭村寨。緬兵懼,每戰輒奔潰,然去必毀其積貯,堅壁清野以待,英人野無所掠,糧運又不繼,遂大困。緬王乘其敝,自阿瓦遣大隊圍攻之,英軍固守不動,緬人不能勝。英軍尋以巨砲反攻緬,緬軍潰。逾數月,喀姆稗兒乘間攻克艾報、墨爾階兩城,與瀕海地那悉林之地,然英軍傷病相屬,其強壯能勝戰者僅三千人,乃移病卒休養於艾報諸城,勢復振。進攻擺古河口之悉林工場,與葡萄牙所築舊堡,悉取之。又克馬爾達般省。

緬人懼,徵鎮守阿拉干長勝軍回援,其帥班都拉,健將也。班都拉既至,急突英軍,不得入,乃退而集師。十一月,班都拉以眾六萬攻仰光及克曼廷村寨,不克。還至丹阿卜,掘地營而守,喀姆稗兒於是進攻普羅美,其地西距厄勒瓦諦江約三里許。明年,英軍分水陸進,將軍可敦將水師,喀姆稗兒將陸軍,會於丹阿卜,合力奪地營,緬將班都拉中砲死,遂長驅入普羅美城。時值大雨,約各休兵一月,以九月十七日為期。入夏以來,英別將馬立生攻克阿拉干部,並逐阿薩密北部緬人,進駐克車。

十月,緬軍三路攻普羅美,英守將僅有歐人三千,印人二千,緬軍不能入。十二月,英人分擊緬軍,緬軍沿厄勒瓦諦江敗退,各以一萬二千人分入米投、麥龍,築壘堅守。未幾,米投破,餘兵奔麥龍,緬人力竭,求成於英,英將允之,遣人議和款,要以四事:一,割阿拉干、艾報、墨爾階與意愛各城歸英轄;二,阿薩密部與各小部,緬人毋得干預其治權;三,賠軍費一千萬羅比;四,應準各國代理人駐扎緬京,且得以兵五十名為衛,英艦之入緬港者,毋得勒令繳槍彈船舵。

議員簽押呈緬王署押,緬王不允,飭整戰備。英將偵知緬王無和意,明年一月十九日,攻克麥龍城,緬人復遣使議和,且徵蒲甘兵衛京城。英將知非王本意,進攻不已,緬廷乃使美士迫拉意斯持前署押約章,並羅比二百五十萬至英軍乞止兵,英乃撤兵去。時道光六年也。

約成,緬國遂失西偏沿海地數部。然緬國上下均不服此約。迨緬王弗極道為其弟撒拉瓦第所篡,撒拉瓦第素主排英,尤蔑視前約。先是英使臣軍佐白奈駐阿瓦,與緬王齟而去,兩國交遂破,英政府撤回駐緬職事人。是後緬人遇英人頗暴厲,英艦至緬者,緬人常與其水手閧,英廷遣使詰責緬廷,且護以水師。比英使至仰光,謁其督臣,語不合,英使遂以兵艦封其港,責償前英船所受損失費,要緬廷禮接英使,仰光督臣在英使前謝罪。時緬王蒲甘曼嗣立,執不允。於是英、緬再失和,而修職貢於中國如故。

咸豐三年十一月,羅繞典奏緬國貢使入京,請變通辦理。帝諭軍機大臣曰:「朕念緬甸國王久列籓封,貢使遠道輸誠,具徵忱悃。惟其國貢使向取道貴州、湖南、湖北進京。現在粵匪未平,若令繞道而行,殊非所以示體恤。即傳旨其使臣,此次無庸來京,仍優予犒賞,委員護送回國。」

是年,緬、英再開戰,南方嚴城要地盡入於英,前所交還擺古部亦為英擾。英將道好西宣言以擺古隸英版圖。適緬親王曼同下王於獄,自立為王,遣使說印督道好西索還擺古,英廷命軍佐雅實勿里為擺古行政長官,且充使以報。偕雅實勿里行者為參贊亨利幼兒、地質學家倭爾罕,挾緬王立永讓擺古之約,緬王拒焉。久之,至同治元年始定約,英乃於緬甸海岸設官分部,稱「英領緬甸」,即擺古、厄勒瓦諦、阿拉干、地那悉林也。以厄勒瓦諦江東支海口為會城,即所謂仰光鎮,以溫個那職視巡撫。

初,英人欲覓一自英領緬甸通中國商路,苦為緬隔。後緬王許英人威廉游歷緬境,北抵八募,又溯厄勒瓦諦江而上,至江上游之山峽。同治六年,緬廷與英人結通航緬境之約,又命英人代收八募與其他口岸商稅。次年,緬王曼同薨,子錫袍嗣位,復命旅於仰光之英工程師威廉、生物理學學士愛迭生、水師兵官暴厄爾與司忒華德、白恩諸人探訪運路,而以軍佐斯賴登率之行,且諭八募守臣以兵五十人護行。於是安抵八募東北之中國騰越境。八年,緬始開厄勒瓦諦江航路,上通八募,命水師兵官斯討拉爾駐八募,理其事。緬王頗注重商務,凡克亨山一帶危險地,皆設官防護,英人交口譽之。然緬王戇而多忌,廢斥舊臣,誅鋤兄弟親戚殆盡。外官雖有四千六百餘土司,皆祿無常俸,專朘民膏,百姓恆產,任意抄沒,緬、英雖交好,而猜忌尤深。

光緒九年,法蘭西由下安南進踞北圻,暹羅亦命官分駐老撾土酋各部,英據南緬既久,洞知上緬寶藏之區,甲於南海,且慮法人由北圻西趨,蔓及緬甸。十一年十月三日,英首相侯爵沙力斯伯里值倫敦府尹大宴時,宣布伐緬意,假判斷木商歇業為名,由印度派兵進攻,入蠻得勒,擒其王,流之於印度孟買海濱拉德乃奇黎島。初,緬與法蘭西、意大利立私約,損自主權利,英弗善也。至是欲存緬祀,則私約不能廢,遂決計滅之,並取所屬拈人地。南緬地區部為四:曰擺古部,曰阿拉干部,曰厄勒瓦諦部,曰地那悉林部。北緬地區部為六:曰北部,曰中部,曰拉歇山嶺部,曰南部,曰東部,曰喀倫尼山嶺部,各部皆設行政長官,而隸於印度總督,緬甸自是遂亡。

時出使大臣曾紀澤駐英,帝以屬國故,命與英外部會商緬事。初議立君存祀,俾守十年一貢之例,不可得。旋議由英駐緬大員按期遣使齎送儀物,其界務、商務兩事,則擬先定分界,再議通商。英人自以驟闢緬甸全境,所獲已多,有稍讓中國展拓邊界之意。英外部侍郎克蕾稱:「英廷原將潞江以東之地,自雲南南界之外起,南抵暹羅北界,西濱潞江,即洋圖所謂薩爾溫江,東抵瀾滄江下游,其中北有南掌國,南有拈人各種,或留為屬國,或收為屬地,聽中國自裁。」曾紀澤轉咨總理衙門,言:「南掌本中華貢國,英人果將潞江以東讓我,宜即受之,將拈人、南掌均留為屬國,責其按期朝貢,並將上邦之權明告天下,方可防後患而固邊圉。」

紀澤又向英外部索還八募。八募即蠻幕之新街。昔時蠻幕土司地甚大,後悉並於緬,其商貨匯集之區謂之新街,洋圖譯音則為八募,距騰越邊外百數十里,在大金沙江上游之東,龍川江下游之北,檳榔江下游之南,向為滇、緬通商巨鎮。英人以其為全緬菁華所萃,不許。爭論久之,克蕾始云,英廷已飭駐緬英官勘驗一地,以便允中國立埠,且可在彼設關收稅。參贊官馬格裏言八募雖不可得,其東二三十里舊有八募城,似肯讓與中國,日後貿易亦可大興。且允將大金沙江為兩國公共之江,如此,則利益與彼分之,其隱裨大局,尤較得潞東之地為勝。議未定,紀澤旋回國。

十二年六月,總署與英使歐格訥議約五條:第一,申明十年呈進方物之例;第三,中緬邊界應由中、英兩國派員會同勘定,其邊界通商事宜另立專章。約成,遷延者五年。

十七年,出使大臣薛福成始申前議,奏言:「英人所稱原讓潞東之地,南北將及千里,東西亦五六百里,果能將南掌與拈人收為屬國,或列為甌脫之地,誠系綏邊保小之良圖。惟查南掌即老撾之轉音。臣閱外洋最新圖說,似老撾已歸屬暹羅。若徒受英人之虛惠,終不能實有其地,非計之得者。南掌、拈人本各判為數小國,分附緬甸、暹羅。宜先查明南掌入暹羅之外,是否尚有自立之國,以定受與不受。其向附緬甸之拈人,地實大於南掌,稍能自立,且素服中國之化。若收為我屬,則普洱、順寧等府邊徼皆可鞏固矣。至曾紀澤所索八募之地,雖為英人所不肯舍,其曾經默許之舊八募者,亦可為通至大金沙江張本。若將來竟不與爭,或爭而不得,竊有五慮焉。夫天下事不進則退。從前展拓邊界之論,非謂足增中國之大也。臣聞乾隆年間,緬甸恃強不靖,吞滅滇邊諸土司,騰越八關之外,形勢不全。西南一隅,本多不甚清晰之界,若我不求展出,彼或反將勘入。一慮也。我不於邊外稍留餘地,彼必築鐵路直接滇邊,一遇有事,動受要挾。二慮也。長江上源為小金沙江,最上之源由藏入滇,距邊甚近,洋圖即謂之揚子江。我若進分大金沙江之利,尚可使彼離邊稍遠。萬一能守故界,則彼窺知江源伊邇,或浸圖行船,徑入長江以爭通商之利。三慮也。夫英人經營商埠,是其長技。我稍展界,則通商在緬甸,設關收稅,亦可與之俱旺。我不展界,則通商在滇境,將來彼且來擇租界、設領事,地方諸務不能不受其牽制。四慮也。我得大金沙江之利,則迤西一路之銅,可由輪船遵海北上,運費當省倍蓰。否則彼獨據運貨之利,既入滇境,窺知礦產之富,或且漸生狡謀。五慮也。凡此五慮,皆在意計之中。又查中、英所定緬約第一條內,緬甸每屆十年,向有派員呈進方物成例。英國允由緬甸最大之大臣,每屆十年派員循例舉行,所派之人應選緬甸國人等語。當時中外注意專在申明成例,惟緬甸何年入貢,並未計及,所以但有此約,而英之駐緬大員尚未舉行。竊恐久不催問,此約即成虛設。臣查成案,緬甸向系十年一貢。自道光二十三年入貢後,道路不通,至光緒元年始復入貢一次。計截至光緒十一年,正應緬甸入貢之期。若不按時理論,彼亦斷不過問。此與勘界各為一事,未便受其牽制,臣擬再加查訪,即行文外部,請其知照駐緬大員,補進光緒十一年應呈方物,俟光緒二十一年,再按定例辦理。萬一彼謂必俟駐緬十年始呈方物,則經此一番考覈,彼於光緒二十一年之期斷難宕緩矣。」

既而英人不認允曾紀澤三端之說,謂普洱外邊南掌、拈人諸地,及大金沙江為公用之江,與八募設關也。十九年七月,福成奏言:「英人自翻前議,雖以公法為解,實亦時勢使然。前議三端,既不可恃,則展拓邊界之舉,毫無把握。前歲英兵游弋滇邊,以查界為名,闌入界內。常駐之地,則有神護關外之昔董,暨鐵壁關外之漢董。雲貴督臣王文韶迭經電達總理衙門。臣承總理衙門急電,照會外部,斥其違理,責令退兵。又屢赴外部爭論,英兵稍自撤退,滇邊至今靜謐。臣又查野人山地,綿亙數千里,不在緬甸轄境之內。曾紀澤曾照會外部,請以大金沙江為界,江東之境,均歸滇屬,英人堅拒不納。其印督至進兵盞達邊外之昔馬,攻擊野人,以示不原分地之意。臣相機理論,稍就範圍,於是有就滇境東南讓我稍展邊界之說。據稱已與印督商定於孟定橄欖壩西南邊外讓我一地曰科干,在南丁河與潞河中間,蓋即孟艮土司舊壤,計七百五十英方里。又自孟卯土司邊外包括漢龍關在內,作一直線,東抵潞江麻慄壩之對岸止,悉劃歸中國,約計八百英方里。又有車里、孟連土司,轄境甚廣,向隸雲南版圖,近有新設鎮邊一,系從孟連屬境分出。英人以兩土司昔嘗入貢於緬,並此一爭為兩屬,今亦原以全權讓我,訂定約章,永不過問。至滇西老界與野人山地毗連之處,亦允我酌量展出。其駐兵之昔董大寨,雖未肯讓歸中國,原以穆雷江北現駐英兵之昔馬歸我,南起坪隴峰,北抵薩伯坪峰,西逾南嶂至新陌,計三百英方里;又自穆雷江以南、既陽江以東有一地,約計七八十英方里。是彼於野人山地亦稍讓矣。其餘均依滇省原圖界線劃分。外部於三月二十三日行文照會前來,臣先行文外部,訂定大局。惟騰越八關界趾未清,尚須理論。外部請待印督所寄地圖,又值外部諸員避暑在外,稍有停頓。前據督臣王文韶電稱漢龍關自前明已淪於緬,天馬關亦久為野人所占踞,則八關僅存六關。現經再三爭論,此二關亦可歸中國。又前年英兵所駐之漢董,本在界線之外,因其扼我形勢,逼處堪虞,向彼力索。外部亦原退讓,以表格外睦誼。刻下界務已竣,商務本不似界務之繁重,且已先將大意議明,無甚爭論。現正商訂條款,計可刻期蕆事。」尋福成議定商約,續爭回鐵壁、虎踞二關,時二關皆英兵占據也。

二十年正月,訂滇緬新約十九條,劃定自尖高山起,向西南行至江洪抵湄江之界線,大金沙江許中國任便行船,刪去八募設關一條。於是緬事粗結。惟十年進呈方物之例,英外部初許待至光緒二十三年照約舉行;繼稱英廷已豫備光緒二十年第一次派員赴中國,至是又聲請展緩,迄未實行雲。

暹羅,在雲南之南,緬甸之東,越南之西,南瀕海灣。順治九年十二月,暹羅遣使請貢,並換給印、敕、勘合,允之。自是奉貢不絕。

康熙二年,暹羅正貢船行至七洲海面,遇風飄失護貢船一,至虎門,仍令駛回。三年七月,平南王尚可喜奏暹羅來餽禮物,卻不受。其年,議準暹羅進貢,正貢船二艘,員役二十名,補貢船一艘,員役六名,來京,並允貿易一次。明年十一月,國王遣陪臣等齎金葉表文,文曰:「暹羅國王臣森列拍臘照古龍拍臘馬虖陸坤司由提呀菩埃誠惶誠恐稽首,謹奏大清皇帝陛下。伏以新君御世,普照中天,四海隸帡幪,萬方被教化。卑國久荷天恩,傾心葵藿,今特竭誠朝貢,敬差正貢使握坤司吝喇耶邁低禮、副貢使握坤心勿吞瓦替、三貢使握坤司敕博瓦綈、大通事揭帝典,辦事等臣,梯航渡海,齎上金葉表文、方物進獻,用伸拜舞之誠,恪盡遠臣之職。伏冀俯垂天聽,寬宥不恭,微臣不勝瞻天仰聖戰慄屏營之至,謹具表以聞。御前方物:龍涎香、西洋閃金緞、象牙、胡椒、黃、豆蔻、沉香、烏木、大楓子、金銀香、蘇木、孔雀、六足龜等;皇后前半之。」帝錫國王緞、紗、羅各六;金緞、紗、羅各四,王妃各減二。正副使等賞賚有差。定暹羅貢期三年一次,貢道由廣東,常貢外加貢無定額。貢船以三艘為限,每艘不許逾百人,入京員役二十名,永以為例。

十二年,貢使握坤司吝喇耶邁低禮等至,具表請封。四月,冊封暹羅國王,賜誥命及駝鈕鍍金銀印,令使臣齎回。誥曰:「來王來享,要荒昭事大之誠;悉主悉臣,國家著柔遠之義。朕纘承鴻緒,期德教暨於遐陬,誕撫多方,使屏翰躋於康乂。彞章具在,渙號宜頒。爾暹羅國森烈拍臘照古龍拍臘馬虖陸坤司由提呀菩埃秉志忠誠,服躬禮義,既傾心以向化,乃航海而請封。礪山帶河,克荷維籓之寄;制節謹度,無忘執玉之心。念爾悃忱,朕甚嘉尚。今封爾為暹羅國王,錫之誥命,爾其益矢忠貞,廣宣聲教,膺茲榮寵,輯乃封圻。於戲!保民社而王,纂休聲於舊服;守共球之職,懋嘉績於侯封。欽哉,無替朕命!」

二十三年,王遣正使王大統、副使坤孛述列瓦提,齎金葉表入貢。帝諭暹羅進貢員役,有不能乘馬者,官給夫轎,從人給舁夫。先是貢船抵虎跳門,守臣查驗後,進泊河干,封貯貨物,俟禮部文到,方準貿易。至是疏請嗣後貢船到廣,具報即準貿易,並請本國採買器用,乞諭地方官給照置辦,允之。頒賞暹羅之鞾,始折絹。貢使回國,禮部派司官、筆帖式各一人伴送。二十四年,議定暹羅國王原賞緞三十四,今加十六,共表裏五十。四十七年,貢馴象二、金絲猴二。是年,禮官議準暹羅貢船壓艙貨物在廣東貿易,免其徵稅。

六十一年,部議暹羅入貢照安南國例,加賜國王緞八、紗四、羅八、織金紗羅各二;王妃緞、織金緞、紗、織金紗、羅、織金羅各二。是年,國王奏稱彼國有紅皮船二,前被留禁,請令廣東督撫交貢使帶回。帝可其請,並諭禮部曰:「暹羅米甚豐足,若運米赴福建、廣東、寧波三處各十萬石貿易,有裨地方,免其稅。部臣與暹羅使臣議定,年運三十萬石,逾額米糧與貨物照例收稅。

雍正二年十月,廣東巡撫年希堯陳暹羅運米並進方物,詔曰:「暹羅不憚險遠,進獻穀種、果樹及洋鹿、獵犬等物,恭順可嘉。壓船貨物概免徵稅,用獎輸心向化之誠。」六年,帝諭暹羅商船運來米穀永遠免稅。七年,常貢內有速香、安息香、袈裟、布疋等,帝以無必須之物,免其入貢,著為例。時貢使呈稱「京師為萬國景仰,國王欲令觀光上國,遍覽名勝,歸國陳述,以廣見聞」。帝命賢能司官帶領游覽,並賞銀一千兩,遇所喜物購買。使臣復稱本國產馬甚小,國王命購數匹帶歸,允之,命馬價向內庫支給。復賜國王御書「天南樂國」扁額、緞二十五、玉器八、琺瑯器一、松花石硯二、玻璃器二、瓷器十四。貢使赴廣採買京弓、銅線等物,復詔賞給。

乾隆元年六月,國王遣陪臣朗三立哇提等齎表及方物來貢,增馴象一隻,金緞二疋、花幔一條,並言昔賜蟒龍袍藏承恩亭上,歷世久遠,難保無虞,懇再賜一二襲。帝特賞蟒緞四疋。禮部奏暹羅照丕雅大庫呈稱伊國造福送寺需銅,懇弛禁,議弗許,帝特賞八百斤。八年,詔暹羅商人運米來閩、粵諸省貿易,萬石以上免船貨稅銀十之五,五千石以上免十之三。其米照市價公平發糶。若民間米多,官為收買,以補常平社倉,或散給沿海標營兵糧之用。十三年,入貢方物外,附黑熊一、鬥雞十二、太和雞十六、金絲白肚猿一。十四年,國王遣陪臣朗呵派提等入貢,錫御書「炎服屏籓」四字。十六年,帝諭閩督喀爾吉善等籌辦官運暹羅米法。疏陳非便,並言不如獎勵商人赴暹羅運米至二千石以上者,予議敘給頂戴,從之。十八年,國王遣使入貢,懇賜人葠、纓牛、良馬、象牙、及通徹規儀內監。禮臣不可,帝加賜人葠四斤,特飭使臣歸國曉諭國王「恪守規制,益勵敬恭」。二十二年,入貢,特賜其王蟒緞、錦緞各二,閃緞、片金緞各一,絲緞四,玉器、瑪瑙各一,松花石硯二,琺瑯器十有三,瓷器百有四。三十一年,暹羅入貢,賜與前同。

頃之,兩廣總督李侍堯奏暹羅為花肚番所破,繳還原頒賜物。花肚番即緬甸也。當其時,緬甸攻暹羅,進圍其國都阿由提亞,三月陷之,殺其王,暹羅遂亡。

緬甸酋懵駁既破暹羅,恃強侵雲南邊,高宗疊遣將軍明瑞、大學士傅恆、將軍阿桂、阿里袞等征之,緬甸調徵暹羅軍自救。阿由提亞之陷也,暹羅守長鄭昭方率軍有事柬埔寨,聞都城陷,旋師赴援,疊與緬甸戰,構兵數年。既以緬甸困於中國,鄭昭乘其疲敝擊破之,國復。昭,中國廣東人也。父賈於暹羅,生昭。長有才略,仕暹羅。既破緬軍,國人推昭為主,遷都盤谷,鎮撫綏輯,國日殷富。四十六年,鄭昭遣使朗丕彩悉呢、霞握撫突等入貢,奏稱暹羅自遭緬亂,復土報仇,國人以詔裔無人,推昭為長,遵例貢獻。帝嘉之,宴使臣於山高水長。所貢方物,收象一頭、犀角一石,餘物準在廣東出售,與他貨皆免稅。特賜國長蟒緞、珍物如舊制。

四十七年,昭卒,子鄭華嗣立。華亦材武,屢破緬,緬酋孟隕不能敵,東徙居蠻得勒。五十一年,華遣使入貢御前方物:龍涎香、金鋼鉆、沉香、冰片、犀角、孔雀尾、翠皮、西洋氈、西洋紅布、象牙、樟腦、降真香、白膠香、大楓子、烏木、白豆蔻、檀甘密皮、桂皮、螣黃,外馴象二。中宮前無象,物半之。並請封。十二月戊午,封鄭華為暹羅國王,如康熙十二年之例。制曰:「我國誕膺天命,統御萬方,聲教覃敷,遐邇率服。暹羅國地隔重洋,向修職貢,自遭緬亂,人民土地悉就摧殘,實堪憫惻!前攝國事長鄭昭,當舉國被兵之後,收合餘燼,保有一方,不廢朝貢。其嗣鄭華,克承父志,遣使遠來,具見忱悃。朕撫綏方夏,罔有內外,悉主悉臣,設暹羅舊王後嗣尚存,自當擇其嫡派,俾守世封。茲聞舊裔遭亂淪亡,鄭氏攝國長事,既閱再世,用能保其土宇,輯和人民,闔國臣庶,共所推戴。用是特頒朝命,封爾鄭華為暹羅國王,錫之誥印,尚其恪修職事,慎守籓封,撫輯番民,勿替前業,以副朕懷柔海邦、興廢繼絕之至意。」是年,粵督穆騰額奏定暹羅正副貢船各一免稅,餘船按貨征榷,以杜奸商取巧。

先是緬甸憚國威內附,後屢為暹羅所敗,五十三年,來貢,乞諭暹羅罷兵。五十四年正月,帝賜鄭華敕曰:「朕惟自古帝王功隆丕冒,典重懷柔,凡航海梯山重譯而至者,無不悉歸涵育,咸被恩膏。爾暹羅國王鄭華遠處海隅,因受封籓職,遣使帕使滑里遜通亞排那赤突等恭齎方物,入貢謝恩,具徵忱悃。朕念爾國與緬甸接壤,往者懵駁、贅角牙相繼為暴,侵陵爾國,興師構怨,匪爾之由。今緬甸孟雲新掌國事,悔罪輸誠,籥求內附,已於其使臣回國時諭令孟雲與爾國重修和好,毋尋干戈。爾亦宜盡釋前嫌,永弭兵釁,同作籓封,共承恩眷。茲特賜國王絲、幣等物,尚其祇受嘉命,倍篤忠忱,仰副眷懷,長膺天寵。欽哉!」

明年,鄭華咨稱:「乾隆三十一年,烏肚構兵,國破君亡。其父鄭昭光復故物,十僅五六。舊有丹荖氏、麻叨、塗懷三城,仍被占據,懇諭令烏肚歸還,以復國土之舊。」粵督郭世勛以聞。帝念暹羅所稱之「烏肚番」即緬甸。前緬甸與暹羅詔氏構兵,系已故緬酋懵駁,非今王孟雲之事。丹荖氏等三城,亦系詔氏在國時被緬甸侵占,非鄭氏國土。相安年久,自應各守疆界。今暹羅已經易世,暹羅又系異姓繼立為王,更不當爭論詔氏舊失疆土。命軍機大臣代世勛擬檄諭止之。是年,入貢,因慶祝萬壽,加進壽燭、沉香、紫膠香、冰片、燕窩、犀角、象牙、通大海、哆囉呢九種,帝亦加賜國王御筆「福」字。六十年,暹羅破柬埔寨,取阿可耳及破丁篷二地。

嘉慶元年,暹羅遣使進太上皇帝、皇帝漢、番字金葉表文並方物。正月,命使臣與寧壽宮千叟宴,賜正使聖制千叟宴詩一章。二年,遣使賀歸政及登極,貢龍涎香、冰片等二十四種。帝奉太上皇帝命賜鄭華敕曰:「九服承風,建極著會歸之義,三加錫命,樂天廣怙冒之仁。舊典維昭,新綸用沛。爾暹羅國王鄭華屢供王會,久列籓封。茲於嘉慶二年,復遣使臣奉表入貢,鑒其忱悃,允荷褒揚。至以天朝疊慶重熙,倍呈方物,具見輸誠效順,弗懈益虔。國家厚往薄來,字小柔遠,自有定制。第念爾國僻處海陬,梯航遠涉,其所備物若從擯卻,勞費轉多,特飭收受,加賜文綺等物。嗣後止宜照常進呈一分,以示體恤。王其祇承眷顧,益懋忠純,永膺蕃庶之恩,長隸職方之長。欽哉!」三年,召暹羅使臣宴重華宮。五年,國王遣使齎祭文、儀物,詣高宗純皇帝前進香,並獻方物,廣東巡撫遵旨令使臣毋庸來京,悉將方物齎回。六年,副貢使怕窩們孫哖哆呵叭病歿廣東,?地方官妥為照料,賞銀三百兩,先行回國。

十年,暹羅貢表,言與緬甸戰獲捷,有詔和解之。十二年九月,帝諭鄭華:「不許違例用中國人駕船,代運貨物往來,以免奸商隱匿,致啟訟端。倘有違背,奸商治罪,國王亦難辭其咎。特申禁令,以嚴逾越之防。爾國王其凜遵毋忽!」

十四年,遣使祝嘏,加賞正副使筵宴重華宮。秋,鄭華卒,世子鄭佛繼立。遣使入貢請封,遭風沉失貢物九種,帝諭不必補進。十五年,封鄭佛為暹羅王,給誥命、駝鈕鍍金銀印,交使齎回。十八年冬,總督蔣攸銛奏暹羅正貢船在洋焚毀,僅副貢船抵粵,副使唧拔察哪丕汶知突有疾,聞正貢船遭焚,驚懼,益劇,不能即赴都。帝命副使留粵調治,所存貢物十種,派員送京,失物毋庸補備。且諭曰:「暹羅國王抒忱納贐,沿海申虔,即與到京齎呈無異。例賞物件及敕書,交兵部發交兩廣總督頒給。」明年,暹羅王聞貢船焚毀,補備方物入貢,遇颶風,船漂散。二十年秋,正副貢船先後抵粵,蔣攸銛以聞。仁宗嘉其恭順,諭曰:「暹羅向系三年一貢,明年又屆入貢之期。此次方物,可作二十一年例貢。」暹羅王復表請準用內地水手駕駛,部議駁之。

道光元年,暹羅遠征馬來半島開泰州,懸軍深入,破沙魯他軍,南下服派拉克,進與色蘭格耳國戰,以軍疲,由新格拉而還。三年,遣使入貢賀萬壽。四年,鄭佛在位十五年,傳位其子鄭福。明年,遣使入貢請封,舟毀,貢物沉沒。帝免補進,仍封鄭福為暹羅王。福朝貢益恭。十九年三月,宣宗以暹羅服事之勤,諭曰:「暹羅三年一貢,其改為四年。」

咸豐元年,鄭福卒,弟蒙格克托繼立,中國稱曰鄭明者是也。明奉孝和睿皇后、宣宗成皇帝遺詔,遺使進香並齎遞表文、方物,慶賀登極。又因例貢屆期,請將貢物一並呈進。文宗命兩廣總督徐廣縉傳知臣毋庸來京,儀物、方物悉令齎回。至應進例貢,現當國制,二十七月之內不受朝賀,並停止筵宴,俟嗣王請封時再行呈遞。二年,徐廣縉奏:「暹羅國王遣使補進例貢,並請敕封,現已行抵粵東。」帝命於封印前伴送來京;應給嗣王誥命,俟貢使抵都發給齎回。靠粵匪亂熾,貢使竟不能至,入貢中國亦於此止。此後暹羅遂為自主之國矣。

鄭明通佛學,善英語,用歐人改制度,行新政,國治日隆,稱皇帝。復與英、法諸國訂約,遣使分駐各國。同治七年,鄭明卒,子抽拉郎公繼立,廢奴隸,行立憲。北部亂賊蜂起,討平之。法既吞越南,復迫暹羅割湄江東地。光緒十九年,國王派軍防守。法藉口暹羅侵越南,出兵占孔格沙丹格、托倫格二地,復進據老撾之加核蒙隆拍拉朋。暹軍敗退湄河西岸,法復以海軍攻盤谷海港,暹人懼,乞和。既,英人疾法日盛,不益於己,乃與法立約,保證湄南屬暹羅,暹羅賴以少安,致力內政,日蒸富強。宣統二年,卒,子馬活提路特立。

暹羅版圖,北緯六度至二十度,東經九十七度至一百七度。官制,設外務、內務、財政、陸軍、海軍、司法、教育、農務、交通九部,佐國王管理國政。另設樞密院,國王選親貴勛臣充之,國之大事皆諮詢而行。中央稱畿甸省。全國分十七州,置總督。州下有縣、郡、村。人口八百萬,中國人占三分之一。軍備仿德國徵兵制,常備軍三萬人,戰時可增十倍。海軍有砲艦、水雷艇數艘。制造槍砲廠、造船所皆備。暹羅疊出英君,政治修明,故介於英、法諸大國屬地,而能自保其獨立也。

南掌,舊稱老撾。雍正七年,雲貴總督鄂爾泰疏言:「南掌國王島孫遣使奉銷金緬字編蒲表文一道、馴象二隻,求入貢。」帝嘉獎,其貢道命由普洱府入,沿途護送,從厚支給。八年二月,遣使表貢,並請定貢期,命五年一貢。賜之敕諭並文綺等物,令使臣齎捧回國。九年六月,表謝頒敕諭恩。

乾隆元年,賜國王島孫彩緞、文綺。八年二月,帝以南掌遠道致貢,改為十年一次。十四年正月,貢馴象。二十六年二月,國王準第駕公滿奏言:「臣母喃瑪喇提拉同臣遣使奉表,進馴象二隻,慶賀皇上五旬萬壽,皇太后七旬萬壽。」準第駕公滿又別備表文一、貢象二,宴賞如例。六月十三日,禮臣議:「嗣後各省巡撫值南掌、琉球、蘇祿、安南等國貢使到境,遴委同知、通判中一員,武弁守備一員,伴行長送至京,並知照經過各省添派妥員護送,按省更替;貢使回國,亦一例辦理。」從之。又奏:「南掌外籓入貢使臣,俱於陳設鹵簿之日,帶領道旁瞻仰天顏,備觀儀典。今國王準第駕公滿遣使叭哩細哩門遮昆來京,擬於七月初八日聖駕起鑾之期,帶領大東門道旁叩見。」

四十七年,國王召翁遣使臣叭整哄等四人入貢,帝於山高水長連日賜茶果,又賜宴於紫光閣、三無私殿。五十五年,國王表貢馴象祝釐,並附進例貢。帝諭云貴總督富綱派員護送。南掌貢使定於七月二十日至熱河行在,與蒙古王公、各外籓貢使同預壽筵。五十八年,諭免例進貢象。明年,國王召溫猛遣使請封,特頒誥敕,並駝鈕鍍金銀印,交使臣齎回。六十年,國王奉表祝釐,進長生經一卷、阿魏二十斤、象牙四十、夷錦四十。時召溫猛已播遷越南昭晉州地,既受敕印,仍未能返國。

嘉慶四年,國王遣使齎表,懇求赴京進香。帝諭止之,令云貴督臣由驛進呈金葉表文,所貢檀香三枝交太常寺。十二年,國王遣使進馴象四隻、象牙四百斤、犀角三十斤、土絹一疋,帝賞賚有加。十四年,越南國王阮福映遣使恭繳南掌敕印。帝諭曰:「南掌國王召溫猛心耎懦不振,流徙越南,遺棄敕印,朕念其流離,不加聲責,豈能復掌國事?聽其在越南居住可也。其國事以其伯召蛇榮代辦。」二十四年,召蛇榮子召蟒塔度臘虔修職貢,籥懇再頒敕印。禮臣覆稱前繳印信字畫完好,毋庸另鑄,準於頒給敕印外,再給誥命一道,交召蟒塔度臘祇領。道光二十二年,遣使齎敕封召喇嘛呢呀宮滿為南掌國王。

咸豐三年,南掌國長召整塔提拉宮滿遣使叩關,請入貢。帝以南掌貢使向由貴州、湖南、湖北、河南取道進京,惟現在粵匪未盡殲除,命云貴督臣吳文鎔等即傳諭南掌使臣,此次毋庸來京,仍優與犒賞,俾先行回國。貢物象只即由督臣派員送京。然自是雲南回匪亂起,貢道遂絕。時南掌兼貢越南之順化,暹羅之曼穀。嗣越南衰,南掌入暹羅,號為暹羅屬國。光緒十一年,法人得越南全境,以南掌地居湄公江中間,為傳教通商孔道,復設法保護之,於是南掌又折入於法矣。

南掌國都曰隆勃剌邦,據湄公江左岸,江東折南流,南岡江自東來會,曲注如玦環,城在山下,當南岡江會流處,水穿城而過。王宮在城之北,背山建屋,規制壯麗。佛墓寺塔森立城市中。瀕江兩岸多花園。居民大半老撾種,或喀木種。老撾種人俗同暹羅,不文身雕題,性愚而嬾。奉佛教,好生惡殺。務耕種、畜牧,能鑄造、紡織。其狀貌短小,鼻寬而脣厚,膚色紅紫,剪發留頂,不蓄須。男子衣飾,橫布一幅圍腰至膝,富貴者以紬緞為之。婦人下裳似裙,上服摺蓋於胸,發黝黑,鬢垂於後項,耳手足皆帶環圈,以金銀銅為飾。其房屋率用藤竹縛造。富室官衙則用堅木,極壯麗。常食睟米,雜以秔稻。中國人教以制酒醴、養蠶絲之法。家畜象、牛,供耕田馱貨。其物產有五金各礦,稻則有秔有睟,多包穀,少粟麥,有靛青、漆、藤、竹、麻、棉、椰葉、桄榔、甘蔗、檳榔、豆蔻、煙葉、芝麻、花生,而松木、楢木尤多。其貨幣或用暹羅之體格,或印度之魯卑,皆銀錢也。此外或用銅錢、用鐵錢,或用銀錠、用海貝。然用錢頗少,以貨易者為多。天氣溫和,自二月至八月多東風、多雨,九月至正月多北風、多晴雲。

蘇祿,南洋島國也。雍正四年,蘇祿國王毋漢未母拉律林遣使奉表,貢方物。五年六月,貢使至京,貢珍珠、玳瑁、花布、金頭牙薩白幼洋布、蘇山竹布、燕窩、龍頭、花刀、夾花標槍、滿花番刀、藤席、猿十二種。賜宴賚賞,頒敕諭一道,令使臣齎回。定期五年一貢,貢道由福建。十一年六月,國王奉表謝恩,並奏:「伊祖東王於明永樂間入朝,歸至德州病故。帝命有司營葬,勒碑墓道,謚曰『恭定』,留妻妾傔從十人守墓。畢三年喪,遣歸。今事隔三百餘年,所有墳墓及其子孫存留周恤之處,懇請修理給復。」禮臣議覆:「蘇祿國東王巴都噶叭哈答歿,長子都馬含歸國襲封。次子安都祿,三子溫哈喇,留居守塋,其子孫以祖名分為安、溫二姓,應如所請。飭查王墓所有神道享亭、牌坊,修理整飭,於安、溫二姓中各遴一人給頂戴奉祀。著為例。」帝允之。乾隆五年八月,蘇祿國王麻喊味呵稟朥寧遣番丁護送遭風商人回內地。八年,貢使馬明光奏請三年後復修朝貢,帝命仍遵雍正五年所定五年一貢之例。十九年,蘇祿國王麻喊味安柔律噒遣使貢方物,並貢國土一包,請以戶口人丁編入中國圖籍。帝諭:「蘇祿國傾心向化,其國之土地人民即在統御照臨之內,毋庸復行齎送圖冊。」二十八年,國王遣使貢方物。自後遂不復至。

蘇祿本巫來由番族,悍勇善鬥。西班牙既據呂宋,欲以蘇祿為屬國,蘇祿不從,西人以兵攻之,為所敗。獨慕義中國,累世朝貢不絕。其國小,有巉巖之嶺,其極南為石崎山、犀角嶼、珠池,因島環繞。海內有珍珠,土人與華商市易,大者利數十倍。此外土產則蘇木、豆蔻、降香、藤條、蓽茇、鸚鵡之類。戶口繁多。地磽瘠,食不足,常糴於別島。土人奉回教。與婆羅洲芒佳瑟民結為海盜云。


\end{pinyinscope}