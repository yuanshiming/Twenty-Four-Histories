\article{志一百三十三}

\begin{pinyinscope}
○邦交六

△日本

日本久通中國。明季以寇邊禁互市,清興始復故。康熙十二年,平南王尚可喜致書於長崎奉行,請通商舶。閩、粵商人往者益眾,雜居長崎市。初有船百八十艘,後由七十艘迭減至二十餘艘。貨運中國歲限八千貫,置奉行三人譏察之,榷其稅。然日本方嚴通海之禁,其國人或潛來臺灣及各口貿易,事發輒罪之。三十二年,廣東廣西總督石琳奏,日本船避風至陽江縣。詔資以衣食,送浙江,具舟遣歸。

雍正六年,浙江總督李衛以日本招集內地人,教習弓矢技藝,制造戰船,慮為邊患,奏明:「密飭沿海文武營縣,及各口稅關員役,嚴行稽查,水師兵船不時哨巡,以為有備無患之計。」上覽奏,諭曰:「昔聖祖遣織造烏林達麥爾森陽為商人,往覘其國。比復命,盛言國小民巽,開洋之舉繼此而起。朕數諭閩、廣督撫留意考察。聞日本近與朝鮮交親,往來無間。夫安內攘外之策,以固本防患為先。其體朕前諭無怠。」並頒諭沿海諸省防海。兩廣總督孔毓珣疏請沿海練舟師、置火器、增砲臺,並自赴廈門、虎門諸口巡察。上不欲啟外人疑懼,但令飭備而已。李衛復奏稱:「日本貿易不能遽絕,請於洋商中擇殷富老成者,立八人為商總,責其分處稽察,互相繩舉,庶免日久弊生之慮。」報可。乾隆四十六年,戶部奏請頒江海關則例,定東洋商船出口貨稅律。嘉慶元年,上諭:「日本商人每遇風暴,漂至沿海,情殊可憫。其令有司送乍浦,附商船歸國。」著為令。

初,日本專主鎖港,通華商而禁西洋諸國。及明治維新,始與各國開港通商。後以各國咸在中華互市,同治元年,長崎奉行乃遣人至上海,請設領事,理其國商稅事。通商大臣薛煥不許。三年,日本商船介英領事巴夏禮以求通。七年,長崎奉行河津又致書江海關道應寶時,言其國人往來歐洲,時附西舶經行海上,或赴內地傳習學術,經營商業,皆有本國符,乞念鄰誼保護。許之。

九年,日本遣外務權大丞柳原前光齎外務卿書致總理各國事務署,略曰:「方今文化大開,交際日盛。我近與泰西十四國訂盟。鄰如貴國,宜先通情好、結和親;而內國多故,遷延至今,信誼未修,深以為憾。茲令前光等詣臺下,豫商通信,以為他日遣使修約之地,幸取裁焉。」前光至天津,三口通商大臣成林、直隸總督李鴻章達其書總署,議允通商而拒其立約。前光謁鴻章曰:「西人脅我立約,彼此相距十萬里,尚遣公使、領事遠來保其僑民。中、日脣齒相依,商賈往還,以無約故,反託外人代理,聽其約束,喪失國權,莫此為甚。今特使人遠輸誠意,而其來也,西人或交尼之;若不得請,是重吾恥也,前光雖死,不敢奉命。」鴻章復為請於朝,下廷議。兩江總督曾國籓等疏言:「日本二百年來,與我無嫌。今援西國之例,詣闕陳辭,其理甚順。自宜一視同仁,請與明定規約,分條詳列,不載比照泰西總例一語,致啟利益均霑之心。」上韙其議,允前光請,命總署答書,詔鴻章豫籌通商事。

十年,日本以大藏卿藤原宗臣為專使來聘,命授李鴻章欽差大臣,應寶時、陳欽副之,與議條款。日使初請照西約辦理。久之,始訂條約十八款,通商章程三十三款,互遣使臣,設領事,以上海等十五口與日本橫濱等八口通商,而禁其私入內地,微異西國。諸約既成,宗臣來獻儀物,期來年換約。十一年,日本罷宗臣官,遣柳原前光詣北洋大臣李鴻章交日本外務卿副島照會,謂來歲與歐西諸國改修條約,欲酌改所議事件,與歐西一律,豫擬條款請商。鴻章答以去秋甫經立約,尚未互換,此時遽行改議,殊非信守。特令津海關道陳欽等與商,均俟換約後照約商辦。

十二年四月,日本使臣副島種臣來京換約,遣其隨員柳原前光、繙譯官鄭永寧詣總署詢三事:一詢澳門是否中國管轄,抑由大西洋主張?一詢朝鮮諸凡政令,是否由朝鮮自主,中國向不過問?一詢臺灣生番戕害琉球人民,擬遣人赴生番處詰問等語。王大臣等當與辯正。尋命李鴻章為換約大臣,與之互換。副島種臣並致國書,慶賀大婚及親政大典。時各國因請覲,報可,副島種臣亦請面遞國書,許之。尋進賀儀方物,答以禮,並給璽書。副島種臣照會,使事畢回國。李鴻章以日本換約時,其上諭內僅蓋用太政官印,未用國璽,駮令換用。繙譯官鄭永寧謂:「本國向與西洋各邦換約,均鈐用太政官印。」鴻章謂:「見爾國副本,聲明鈐用國璽,又上海道抄送總領事井田讓等敕書,亦用國璽。」鄭永寧允回國換寄。時日本未設駐京公使,交涉事託俄使倭良嘎哩代辦。

十三年三月,日本兵船至廈門,聲稱赴臺灣查辦生番。李鴻章致書總署,謂:「各國興兵,必先有文函知會,因何起釁。臺灣生番一節,並未先行商辦,豈得遽爾稱兵?」既聞美人李讓禮帶領陸軍,又雇美國水師官領兵船,欲圖臺灣。李鴻章復致總署,謂:「此事如果屬實,不獨日本悖義失好,即美人幫助帶兵,雇商船裝載弁兵軍械,均屬違背萬國公法,且與美約相助調處之意不符。應請美使遵照公法,撤回李讓禮等,嚴禁商船應雇裝載弁兵。日本既無文函知會,僅將電信抄送上海道。雲派員往臺灣查問,難保不乘我不備,闖然直入閩省,應先派兵輪水師,往臺灣各港口盤查了望,另調得力陸軍數千,即用輪船載往鳳山、瑯軿附近一帶,擇要屯扎,為先發計。」乃日本兵船忽犯臺灣番社,以兵船三路進攻,路各五六百人。生番驚竄,牡丹、高士佛、加芝來、竹仔各社咸被焚。其時尚有兵輪船泊夏門。於是臺灣戒嚴,命船政大臣沈葆楨渡臺設防。葆楨密疏聯外交、儲利器、儲人才、通消息四事。閩浙總督李鶴年亦陳臺灣地利,並遣水路各營分往鳳山、澎湖等處屯扎。

是月日本攻生番網索、加芝來等社,移兵脅龜仔角社,社番誓不降。帝命福建布政使潘霨赴臺灣會商設防。五月,沈葆楨、潘霨率洋將日意格、斯恭塞格至臺灣,奏陳理諭、設防、開禁等事,皆報可。初八日,潘霨偕臺灣兵備道夏獻綸及洋將日意格、斯恭塞格等,乘輪船由安平出海抵瑯軿。詣日營,晤中將西鄉從道,示以葆楨照會,略云:「生番土地隸中國者二百餘年,殺人者死,律有明條,雖生番豈能輕縱。然此中國分內應辦之事,不當轉煩他國勞師糜餉。乃聞貴中將忽然以船載兵,由不通商之瑯軿登岸。臺民惶恐,謂不知開罪何端,使貴國置和約於不顧?及觀貴中將照會閩浙總督公文,方知為牡丹社生番戕害琉球難民而起。無論琉球雖弱,侭可自鳴不平。即貴國專意恤鄰,亦何妨照會總理衙門商辦。乃積累年之舊案,而不能候數日之回文,此中曲直是非,想亦難逃洞鑒。今牡丹社已殘毀矣,而又波及於無辜之高士佛等社。來文所稱殛其兇首者,謂何也?所稱往攻其心者,謂何也?幫辦潘布政使自上海面晤貴國柳原公使,已商允退兵,以為必非虛語。乃聞貴中將仍扎營牡丹社,且有將攻卑南社之謠。夫牡丹社戕琉球難民者也。卑南社救貴國難民者也。以德為怨,想貴中將必不其然。第貴中將知會閩浙總督公文,有佐藤利八至卑南番地亦被劫掠之語,誠恐謠傳未必無因。夫鳧水逃生者,有餘資可劫,天下有劫人之財,肯養其人數月不受值者耶?即謂地方官所報難民口供不足據,貴國謝函俱在,並未涉及劫掠一言。貴國所賞之陳安生,即卑南社生番頭目也。所賞之人即所誅之人,貴國未必有此政體。兩國和誼,載在盟府,永矢弗諼。本大臣敢不開誠布公,以效愚者之一得,惟高明裁察見覆。」霨復造其營,從道辭以病。霨及獻綸遂遣人傳各社番目,至者凡十五社,譯傳大意,皆求保護。因諭令具狀,原遵約束,不敢劫殺。霨等宣示國家德意,加以犒賞。番目等咸求設官經理,永隸編氓。霨等因從道不出,將還。從道復來謁,堅以生番非中國版圖為詞。及示以臺灣府志所載生番歲輸番餉之數,與各社所具結狀,日將始婉謝。請遣人附我輪船,一至上海,致書柳原前光,一請廈門電報本國,暫止添兵。霨等遂返。

初,日本逐牡丹社番踞其地。旋有輪船二先後至,一逕往後山射藔港,一載兵二百、婦人十餘泊射藔港,攜食物什具農器,及花果草木各種,分植龜潭、後灣,為久居計。窺我兵力不厚,仍肆要求。沈葆楨請派水師提督彭楚漢率師來臺灣。日旋增兵駐風港。沈葆楨急飭營將王開俊由東港進駐枋藔,以戴德一營由鳳山駐東港為後應。日人水野遵入豬朥索、高士佛諸社,又自後灣開道達龜山巔,其風港之營將分駐平埔為援應。因遣其通事彭城中平至瑯軿,謁委員周有基,訊中國四處布兵何意。有基以巡察應之。葆楨照會日將,勸令回兵。時李鴻章亦深慮臺地兵單。及沈葆楨請借撥洋槍隊,即奏以提督唐定奎統軍赴臺灣助防。葆楨亦奏稱:「澎湖為臺、廈命脈所關,守備單弱,非大枝勁旅,仍無以壯民氣而戢戎心。請催迅速前來,庶臺、澎氣脈藉以靈通,金、廈諸防亦資鞏固。」奉旨俞允。潘霨又偕前署鎮曾元福等赴鳳山舊城募土勇,並勵鄉團。因親履海口之打鼓山等處,踏勘要隘,建立兵柵,以待淮軍分駐。

是月柳原前光入京先謁李鴻章,鴻章遣道員孫士達往答拜,屬以到京後勿言兵費及請覲兩事。日本又遣大久保利通入京。美領事畢德格復出任調停,說鴻章仍允照柳原原議三條,並加撫恤賠命。

初,日人劉穆齋在花蓮港遭風,破船失銀,稱社番盜劫。沈葆楨命夏獻綸集訊其地居人及船戶,查無劫掠失銀之事。惟日人欲從生番租地,給有洋銀,番目來益不受而止,並繳出日本前給旗物。葆楨因奏言:「日本和約第三條,禁商民不準誘惑土人;第十四條,約沿海未經指定口岸,不準駛入;第二十七條,船只如到不準通商口岸私作買賣,準地方官查拿。今臺後歧萊地方,中國所轄,並非通商口岸。此次前赴歧萊之成富清風等,攜游歷執照,勾引土番,均違和約。現已確查歧萊各社並無竊盜銀物。其繳出旗、扇各件,當即發交蘇松太道,轉給駐滬日本領事收回,將游歷執照追銷。其違約妄為之處,應由彼國自行查辦。並錄民、番供結,咨呈總署,牒其外務省,轉飭日本領事照章辦理,以弭釁端。」從之。命速修安平砲臺,及籌辦鐵甲船。續諭:「日本雖未啟兵端,然日久相持,終非了局。現淮軍續抵鳳山,羅大春業抵蘇澳、滬尾、雞籠等口,調兵扼扎。」葆楨於是設防益嚴,日人乃謀撤兵。而西鄉從道仍遷延不即退,欲牡丹社賠給兵費。

柳原前光既至京,先遞照會有「臺灣生番為無主野蠻,本不必問之中國」之語。先請覲見。總署責以:「臺灣生番系中國地,不應稱為『無主野蠻』。迭次來京,並未與中國商明,何以捏稱中國允許日本自行辦理?」柳原前光答辯。久之,始議定三條,給撫恤銀十萬,再給修道建房費四十萬兩,定期撤兵付銀,互換條約。於是大久保往瑯軿,命領事福島九成謁沈葆楨陳五事:一,請派人受代;一,請撤銷兩國大臣來往公文;一,請被害遺骸於收埋處建碑表墓,並許日人以後登岸掃祭;一,請以後臺灣交涉事件,由中國官交廈門領事。葆楨以撫局已成,允之。惟於登岸掃祭一節,覆以須有領事官鈐印執照,祭畢即歸。遂各遣員交代。事訖,西鄉從道率兵去。

光緒元年八月,日本署公使鄭永寧牒中國,請補正前約。李鴻章令津、滬兩道詳議,復將各條逐加查核,因致總署云:「通商章程第二十八款,進出口稅未便一例,及日本進口稅則第八十三條(毛曷)布類,又日入至日出不準開封鎖艙,應行更正補載等事,可以照準。但換定之約,不便改寫,祗可由總署另給照覆,附刊章程之後。至鴉片嚴定罰款一條,彼國既有各國貿易通例,或可權宜照辦,無須補列。查曾國籓預籌日本議約奏內亦云,彼國嚴禁傳教與鴉片,中國犯者即由中國駐員懲辦,或解回本省審辦,而鄭署使照會末段,華民歸彼地方官照料,是中國遣理事官一端,實有難再從緩之勢。查橫濱、長崎、神戶三處華民最多,總理事官駐最要之口,各口即選各幫公正司事,俾為副理事官,遇事妥商辦理,實與中外大局有裨,應主持早辦。」總署亦以為然。會日使議改章,欲於鴉片進口照西例加倍嚴罰,李鴻章亦援西例與爭。議久不決。

是秋,日本派使臣帶兵船往朝鮮攻毀砲臺,以朝鮮砲擊日船,特遣森有禮為駐華公使,要求總署發給護照,派人前往,又欲代遞文信。總署堅拒。李鴻章謂宜由總署致書朝鮮政府,勸其以禮接待,或更遣使赴日本報聘,辨明開砲轟船原委,以釋疑怨,為息事寧人之計。總署即派辦理大臣往問朝鮮政府。朝鮮政府頗不原與日本通商往來,而日使森有禮往謁李鴻章,則以高麗非中國屬邦為詞。因提出條件三:一,高麗以後接待日本使臣;一,日本或有被風船隻,代為照料;一,商船測量海礁,不要計較。鴻章答以高麗系中國屬國。事既顯違條約,中國豈能不問,森使急求與高麗通好,鴻章請徐之。

二年八月,始命直隸候補道許鈐身出使日本,擬設理事、副理事各員。日使森有禮詣李鴻章,謂中國商民向由日本地方官管理。中國若派領事官前往,恐日本不肯承認。鴻章答以同治十年修好條規第八條云,兩國指定各口,彼此均可設理事官。茲照約選派理事,日本何能不認?日本自訂約後,在上海、廈門、天津設立領事,中國無不照約招待。彼此一例,何能稍有區別?森使乃不復言。

是年,日本屯兵琉球。福建巡撫丁日昌以琉球距臺北雞籠,水程不過千里,請統籌全局以防窺伺,報可。三年三月,日本因內亂,來借士乃得槍子百萬,政府以十萬應之。五月,琉球國王密遣陪臣齎咨赴閩,訴日本阻貢物。閩浙總督何璟等以聞,並出使日本大臣何如璋。如璋乃往日本外務寺島宗則商議,並照會其外務卿,延不答覆。五年正月,日人驅遣琉球官員之在日本者,令回琉球,並派內務大丞松田往琉球,廢琉球為郡縣,並令改用紀元。如璋函報總署,復親往見其內務卿伊藤博文及外務卿,皆不得要領。時有美前總統格蘭忒者,游歷來華,又將有日本之行。鴻章因以琉球事相託,格蘭忒慨然以調處自任。及至日本,以琉球各島本分三部,商擬將中部歸球立君復國,中、東兩國各設領事保護,其南部近臺灣,為中國屬地,割隸中國,北部近薩摩島,為日本屬地,割隸日本,冀可息事。而日本總稱琉球為己屬國,改球為縣,系其內政。格蘭忒請另派大員會商。李鴻章因達總署,請照會日本外務省,請其另派大員來華會商。而日本則欲中國另派大員前往東京,或如光緒二年在煙臺會議。李鴻章執不許。

會俄因廢約事,與中國肇釁。詹事府左庶子張之洞奏:「俄人恃日本為後路,宜速聯絡日本。所議商務,可允者早允,但得彼國兩不相助,俄事自沮。」政府得奏,因徇日使戶璣之請,以南部宮古、八重山二島歸中國,而加入內地通商照各國利益均霑之條。戶璣又以本國現與西洋各國商議增加關稅、管轄商民兩事,美國已允,請一並加入條約。總署以日本既與各國商議,俟日本與各國訂定後,再彼此酌議,暫不並加入約。已定議矣,而右庶子陳寶琛以俄事垂定,球案不宜遽結,日約不可輕許上言。兩江總督劉坤一、出使日本大臣黎庶昌、內閣學士黃體芳各有建議,皆不果行。八年十二月,李鴻章復與總署議球案,欲就前議中國封貢議結,仍不決。

十年九月,日本公使榎本武揚請於登州、牛莊二口運豆餅。政府以非條約所有。李鴻章謂:「同治元年總署徇英使之請,暫弛豆禁,而已開竟難禁止。同治八年,滬上洋商雇用輪船徑從牛莊裝豆運往長崎,當經總署飭總稅司查禁議罰,不果。以後豆石漸多流入東洋,旋值中、日訂約,其時豆禁開已十年。日使援例為請,但允以通商別口買運,至登、牛兩處,仍堅持不許。榎使所請,僅豆餅一項。中日通商章程載明年限屆滿,兩國方可會商酌改。今尚未訂改期,若婉辭以緩,至重修商辦,似無不可。如仍嘵瀆,應予通融,聲明原約其餘各款照舊信守,庶於羈縻之中,仍寓限制之義。」

會朝鮮亂,日本進兵,以保護使館為名,又以中國兵槍傷日本兵為口實,十一年正月,派參議伊藤博文為全權大臣,來華議事,並遞國書,進謁李鴻章。初日本敕書內有「議辦前日案件,妥商善後方法」之語,李鴻章以為隱括朝案宗旨。伊藤開議要求三事:一,撤回華軍;二,議處統將;三,償恤難民。鴻章以撤兵一節尚可商議,議處統將、償恤難民,力爭不許。函致總署,謂議處、償血⼙兩層,縱不能悉如所請,須求酌允其一。但我軍入宮保護,名正言順,交戰亦非得已,斷無再加懲處之理。伊藤強請三事皆允,鴻章只允撤兵,並要同撤,伊藤亦允。吳大澂擬四條,送交伊藤:一,一同撤兵;二,練兵各營,須有中國教習武弁若干人,定立年限,年滿再行撤回;三,以後朝鮮與日本商民爭端,日本派員查辦,不得帶兵,中國亦然;四,朝鮮如有內亂,朝王若請中國派兵,自與日本無涉,事定亦即撤兵,不再留防。伊藤不以為然,自出所擬條款:一,議定將來中、日兩國永不派兵駐朝;二,前約款仍與中、日兩國戰時之權無干,若他國與朝鮮或有戰爭,或朝鮮有叛亂,亦不在前條之例;三,將來在朝鮮如有中、日兩國交涉,或一國與朝鮮交涉,兩國各派員商辦;四,朝鮮教練兵士,宜由朝鮮選他國武弁一員或數員教練;五,兩國駐朝兵,於畫押蓋印後四個月限盡撤。鴻章以伊藤所擬五條,意在將來彼此永不派兵駐朝,辨駮不允。旋奉旨:「撤兵可允,永不派兵不可允;至教練兵士一節,亦須言定兩國均不派員為要。」鴻章奉旨後,與伊藤會議,因議將前五條改為三條:一,議定兩國撤兵日期;二,中、日均勿派員在朝教練;三,朝鮮若有變亂重大事件,兩國或一國要派兵,應先互行文知照。遂定議,而於議處、償恤仍不許。惟因當時日兵實被我軍擊敗傷亡,鴻章因牒日本致惋惜,並自行文戒飭官兵,以明出自己意,與國家不相干涉。三月初四日,立約畫押,是為中日天津會議專條。

十二年五月,日本公使鹽田議修約,李鴻章以為宜緩,因致總署,謂:「日廷現與歐、美各國改約,應俟彼商定後,我再與議,庶可將西國所訂各款參酌辦理。又球案亦當並商妥結,免致彼此久存芥蒂。請總署酌奪。」旋因長崎兵捕互斗案出,暫置未議,而琉球遂屬於日,不復議及矣。

十三年正月,鹽田因崎案已結,請催修約,總署仍令李鴻章核覆。鴻章謂:「原約分修好條規、通商章程為二。條規首段聲明彼此信守,歷久弗渝。通商章程第三十二款則聲明現定章程十年重修。是章程可會商酌改,條規並無可改之說。至通商章程,大致本與西約無甚懸殊。惟第十四、五款,不準日人運洋貨入內地、赴內地買土貨,為最要關鍵。當時伊藤與柳原前光為此兩款力爭,鴻章堅持不改。今日稿第一款內,一曰遵守彼國通商章程,再曰遵守清國與各與國所締通商章程,固寓一體均霑之意,實欲將十四、五款刪除,關系甚大,請緩議。」時日本伊藤博文新秉政,仍欲中國派全權商議,卒不果。

二十年三月,朝鮮東學黨亂作,乞援於中國,中國派兵前往,日本旋亦以兵往。李鴻章電駐日公使汪鳳藻,與日本政府抗議,日仍陸續出兵。及事平,駐韓道員袁世凱牒日本駐韓公使大鳥圭介,援約同時撤兵。日本外務省提出三項:一,中、日兩國兵協同平定韓國內亂;二,亂定後,兩國各設委員於京城,監督財政及吏治;三,募集公債,以為朝鮮改革經費。總署電令汪鳳藻答覆,略謂朝鮮內政,應由朝鮮自由改革,不應干預。日本政府覆鳳藻,謂朝鮮缺獨立資格,日本為鄰邦交誼,不能不代謀救濟。既又提出二條件,謂無論中國政府贊成提案與否,日本軍隊決不撤回。中國主撤兵再議,日本則要求議定再撤兵,持久不下。

七月,日本遂宣戰,誤擊沈高升英船。時日本寓華商民,屬美領事保護,中國寓日商民,亦託美保護,美使調停無效。及戰事起,提督葉志超、衛汝貴守平壤牙山,先潰,左寶貴陣亡,海軍繼敗。於是日軍渡鴨綠江,九連城、鳳凰城、金州、海城、大連、旅順、蓋平、營口、登州次第失守,又破威海衛,襲劉公島,降提督丁汝昌,海軍艦盡熸。

初,日人志在朝鮮,至是並欲中國割地賠費,指索臺灣,又提出四條件:一,派大員往東洋議約;二,賠兵費五萬萬;三,割旅順及鳳凰城以東地;四,韓為自主之邦。二十一年正月,命張廕桓、邵友濂赴日本議和,拒不納,乃再以李鴻章為全權。鴻章至日本,日本派伊藤博文、陸奧宗光為全權大臣,與鴻章會議於馬關,月餘不決。鴻章旋為日本刺客所傷,又命其子李經芳為全權幫辦,卒訂約十一款:認朝鮮獨立,割遼南及臺灣,賠款二萬萬,且許以內地通商、內河行輪、制造土貨等事,暫行停戰。

張之洞、劉坤一等聞之,亟電力爭。俄國亦約法、德勸日讓還遼南。日索交臺灣益亟,朝旨命臺灣巡撫唐景崧交臺,臺民洶洶欲變,並引公法力爭。政府不得已,又因王文韶、劉坤一電阻,乃諭之曰:「新定和約,讓地兩處,賠款二萬萬,日人堅執非此不能罷兵。連日廷臣來奏,皆以和約為必不可準。目前事機至迫,和戰兩事,利害攸關,即應主斷。」命直陳。又命李鴻章覆電伊藤展期。鴻章以原議批準電知,若改約另議,適速其決裂,請暫行批換。乃派道員伍廷芳、聯元等往煙臺換約。初限期四月十四日。及伍廷芳等至煙臺,日使伊東美久治請速換約,限十四日申刻。廷芳駮以停戰至十四夜子刻為止,乃聽稍緩。亥刻換訖,伊東美久治即行。會臺灣民變,將劫唐景崧、劉永福守臺,別求各國查照公法,從公剖斷。於是日派水師提督樺山資紀赴臺,限日交割。政府乃派李經芳為交付臺灣大臣。經芳之澎湖,與樺山指交於舟次。自是臺灣屬日矣。

尋議還遼,日派林董為全權,與李鴻章議商,辯論久不決。嗣定議分為六款:一,還遼南地;二,償兵費三千萬;三,交款三個月以內撤兵;四,寬貸日本軍隊占踞之間所有關涉日本之中國臣民;五,漢文、日本文遇有解譯不同之處,以英文為憑;六,兩國批準自署名蓋印之日起,遂在北京互換。復訂專條,於定議五日內互相達知,以期迅速。是為中日遼南條約。

先是中日新約第六款所列各條,如蘇州、杭州、重慶、沙市等處添設口岸,聽其任便往來;第二條,日本輪船得駛入各口搭客運貨;第三條,日本臣民得在中國內地購買經工貨件若自生之物;第四條,日本臣民得在中國制造各項工藝,又得將各項機器裝運進口,止交進口稅,日本在中國制造一切貨物,即照日本運入中國貨物一體辦理等節:朝廷因損失利權,欲挽救之。又值通商行船章程將開議,乃命中外臣工籌議。廖壽豐、譚繼洵、鹿傳霖均有論奏,而張之洞言尤切直,並擬辦法十九條,電總署代奏:「一,寧波口岸並無租界名目,洋商所居地在江北岸,即名曰洋人寄居之地,其巡捕一切,由浙海關道出費雇募洋人充當。今日本新開蘇、杭、沙市三處口岸,系在內地,與海口不同,應照寧波章程,不設租界名目,但指定地段縱橫四至,名為通商場。其地方人民管轄之權,仍歸中國,其巡捕、緝匪、修路一切,俱由地方官出資募人辦理,不準日人自設巡捕,以免侵我轄地之權。二,制造貨物,自系單指通商口岸而言,華文有含混內地之意,須更正。『任便』兩字太寬,宜議定限制。三,出示曉諭產貨地方,須先完坐賈釐捐,方準售賣。無論洋商、華商,一律辦理。日本人在內地購買土貨,只可暫行租棧存放,不準自行開行,及自向散戶收買,以免奪我產貨地方坐賈釐稅,且杜華商影射洋票漏釐。四,內地收買土貨,準其租棧暫存,不準購買房地、懸掛招牌。所買土貨,務須運載出口,不得在內地轉售。洋貨運入內地,須大宗販賣,不準零售。租棧應給地方公舉費用,須照華民房屋一律攤派。五,日本人在內地制造土貨,出廠後即完正稅一道,運出通商地界,無論行銷內地及運出外洋,均須再完半稅一道。六,通商章程善後條約第二款所載各項器用食物進口,通商各口皆準免稅,原為洋商在各口岸自用。若作貨物轉售,應照值百抽五納稅,不得藉口家用雜物蒙混免稅。七,日本輪船不準販運食鹽。八,米穀、銅錢不準販運出洋。九,軍火禁販,非有官買執照,不準進口。十,日本輪船不準拖帶民船,免致影射漏釐。十一,日本行內河輪船,尺寸大小、時刻早晚,須有限制,以免傷礙民船。十二,日本輪船隻準到指定口岸裝卸人貨,不準沿途起卸搭載。十三,內河輪船應收船鈔,須較長江加多,以備修理河道之費。十四,日本人入內地辦貨賣貨,不準薙發改為華裝,違者查出即作華人照奸細治罪。十五,雇用華民工作,須按日給值,聽其自原,不得立約限期,抑勒作工,鞭撻虐待。十六,裝運機器,制造各物,須無傷民命,方能照準,不得以『任便』兩字藉口。十七,船隻非日本商人購置,行戶藉日本商資本不得懸掛日本旗,若有冒名包庇,查出即行充公。十八,制造各廠,如有藏匿犯法華人,一面由地方官知照領事,一面即派人到廠緝拏,廠主不得袒庇。如廠主確知為好人,須照洋例存銀作保,到審訊日交出候審。十九,廠內如有華工滋鬧,毀傷機器廠屋,地方官只能辦犯,不能賠償。若僅罷工細故,應由廠主自行調停,官不與聞。」於是派張廕桓為全權大臣,與日本使臣林董議商約。林董交約稿四十款,之洞致總署請駮辯,即由全權另擬約本與林董議,屢延不決。是年開蘇州商埠,日人欲即行船,總署以租界未定,稅關未設,行船不便。日本又欲於租界設巡捕、立工程等局,總署援寧波章程,復不允。

二十二年正月,商約開議,張廕桓將日使原稿駮刪九款,駮改七款。惟第三十四款,日本官商財產,遇有辦理案件,均照相待最優之國一律;第三十五款,日本商民所有事件,均照中國臣民、中國船、中國貨並相待最優之國臣民、船貨一律相待;第三十六款,他國國家官員、船貨、人民得有利益,日本一律同獲其美:此三款日本舊約皆不得與各國均霑,不能過拒,乃照英約第二十四款,改作一條,刪此三款。遂定議。初,馬關約準開四口,本有均照向開海口及內地鎮市章程辦理之言。中國欲以寧波辦法為程,日本欲取法上海章程專管租界之條,乃不得不允矣。

是年開四口租界。初開沙市租界,因地窪下,要中國築堤,中國以與各國通例不符,卻之。又索漢口城外德國租界起沿江之地長三百丈作租界,中國以所索地在中國興辦鐵路應用限內,不許,惟許在德界千丈以外,偪近鐵路,讓給租界三百丈。因聲明兩條入條款:「一,偪近鐵路江岸,日本一年須自築堤岸,以資保障;二,所給界內軌道穿過之處,已為鐵路購用,若干方數內,應仍歸鐵路總公司管業,兩不相礙」等語。二十四年三月,日使至總署,請沙市租界未定以前,日商運貨暫免釐金,許之。

是月僑寓沙市湘人,因與招商局起釁,延燒日本領事館,駐沙日領事永瀧訴於日本公使矢野,要求五事;已,復提四條:一,索賠一萬八千兩;二,以八萬六千餘兩作沿江堤費,兩國各半;三,專界內道路免價豁租;四,界內租地價酌行核減。張之洞即電總署,謂:「一條索賠一萬八千兩一節,擬允給一萬兩。第二條以八萬六千餘兩作沿江堤費兩國各半一節,彼此兩益,事屬可行,當照允。第三條專界內道路免價豁租一節,其租可免,地價未便不給。第四條界內租地價酌行核減一節,可行,當照允。」案旋結。五月,準中國商民居住日本專界,援德界例也。六月,駐沙日領事請地價減一半,道路溝渠地價認十分之一,許之。七月,命派學生游學日本。十月,日使矢野又請中國南北洋、湖北三處各派武備學生前往肄業。

二十六年春,拳匪起,連戕日本使館書記生杉山彬、德使克林德,各國皆出兵。日本福島正安統兵赴津。六月,與各國聯軍攻天津城。七月二十日,入京師。時政府已特召李鴻章,未至而京師陷,兩宮出狩。日本外部電告李鴻章等維持中國善後。福島正安請速奕劻返京,奕劻遂有全權大臣之命,與李鴻章同議和。適盛京將軍增祺與俄擅定暫約,日本外部謂公約未定,不應立私約,俄約應歸公議,與英、德同。然勸俄訖不應。時禍首已懲辦,公約亦定,朝廷因日本使館書記生杉山彬被害,特簡戶部侍郎那桐為專使,赴日本道歉,所得償款四百五十兆,日本應得三千七百九十三萬一千兩,惟以俄不退東三省、俄約不歸公議為言。

二十八年三月,日本領事小田切奉其政府命詣張之洞:一,告阻止俄約情形;二,勸中國收買洋藥;三,勸江、鄂會奏改東三省官制章程;四,欲與中國商人合開銀行;五,欲與招商局合辦推廣江海輪船。既又談商約三條:一曰美使不原加稅,日本意與美同;二曰長沙、常德開口岸;三曰米穀出洋。張之洞分別答辨,並將所言致書商約大臣呂海寰等核議。未幾,日本商稅使日置益、小田切又送新約十款,大抵皆抽稅、免釐、行輪、開埠、居住、合股等特殊利益。時方議英約,中國只欲於英約已允者照辦,未允者不允。屢議不行。九月,改派伍廷芳充商約大臣,並派袁世凱會議。日本於加稅免釐,仍不允如英約加至十二五,僅允值百抽十,並欲將由日本運進中國之煤炭、棉紗及一切棉貨概不加稅,尤與英約相背。中國不允。惟第三款川江設施拖攬,第四款內港行輪及修補章程,第七款中、日商民合股經營,第八款保護商牌,第九款改定國幣,均為英約所有,允之。又於商牌款內議增保護版權一事,內港行輪款後議增照會聲明,往來東三省輪船亦系照內港章程辦理,不能駮拒。此外第五款索開各處口岸,第六款口岸城鎮任便居住,第九款第二節釐飭度量權衡,第十款請運米穀出口,均駮拒不允。日使內田康哉赴部晤商,又提出北京開埠、加稅免釐、米穀出口三條,欲在京與張之洞議,餘仍歸滬定。時之洞在京,外務部答以不能兩處分議,遂暫停。

十月,漢口因議給比利時界增日本租界。初,日本索租界三百方丈,止允給一百方丈,留二百方丈備中國公司之用。當時日使言明,日界外地如別有餘地讓給他國,日本仍須照原議添索二百方丈。茲議給比界中僅餘地約三百丈,擬添給日本租界一百五十丈,仍畫留約一百五十丈立作華業公司地界,以備中國官商自用。日本猶爭不許。日本議設兩湖輪船公司,欲華洋合股,不果。是月,撤駐水扈日兵。

尋復議約。日使內田康哉與張之洞在京會議,研商數月,始漸就緒。即致總署,謂我所索允者三事:一,照各國一律加稅;一,查禁違礙書報;一,中國人民在日本者,極力優待。駮辯刪去者三事:一,請運米穀出口;一,口岸城鎮任便居住;一,常德府等九處口岸。以要索為抵制者一事:各國護路護館兵隊全撤後,北京方能開埠。因有益於中國商民,可除積弊,而許其入約者,度量權衡一款;照滬議原文增改字句者,改定國幣一款,內港行輪一款,川江設施拖攬一款;因英已有而許其入約定議者,長沙通商一款。餘皆仍照滬議原文。又致外務部及呂海寰等,謂日約東三省開埠,言明悉照美約文法,惟安東縣改大東溝,緣大東溝系日本原議所索。嗣增索安東縣,再三商駮,內田始允仍將安東縣刪去。遂定議,於二十九年八月十八日在滬畫押,是為日本商約。是年與日使議索還前借漢口大阪馬頭,仍未還。又盛宣懷與日本立漢冶礦石借款合同,數三百萬元,息六釐,預定三十年還清,不還現銀,以礦價扣還。

三十年,日商三井在漢廠購生鐵一萬六千頓,值日俄戰起,中國慮於局外中立有礙,擬阻止。會日本領事永瀧來函,謂訂運生鐵,不在戰時禁貨之列,日使亦來函聲明,作為商工制造之用,不得以禁貨論,遂許運。三十一年,日戰勝俄,兩國議和,政府令外務部照會日、俄,謂關涉中國之事,若中國不與聞者,中國將來斷不承認。是年十一月二十六日,外務部慶親王奕劻與日本大使小村壽太郎、公使內田康哉訂新約。正約三款:一,凡俄國允讓之利益,中政府悉承諾之;二,凡中、俄所訂借地造路等項,日本悉照約履行;三,此約簽字即便施行。附約十二款:一,中國將東三省自行開闢商埠;二、三,撤兵事宜;四,日本允將所占公私產業,在撤兵前後交還;六、七、八,安奉、南滿鐵道建築事宜;九,另訂奉天日本租界辦法;十,鴨綠江右岸設中日木植公司;十一、十二,中、日彼此以最優國相待遇。

三十二年,日人設立南滿洲鐵道株式會社,並於關東州置都督府,另設領事五人,總領事駐奉天。安奉鐵道外有間島領土權,撫順炭坑、新法鐵道、營口支線、新奉、吉長兩鐵道借款諸事,經東三省總督趙爾巽、徐世昌及外務部尚書袁世凱先後與日使爭議,久不決。

三十三年三月,外務部大臣那桐與日本駐京公使林權助訂中日新奉吉長鐵路協約七條:一、二,中國以日金一百六十六萬元收買日本所已造之新奉鐵路,其續造遼河以東一段及自造吉長鐵路需款,均向南滿洲公司籌借半款。三,除還清期限外,均照山海關內外鐵路借款合同辦理。其主要事務,又開列六條:甲,借款還清期限,遼河以東十六年,吉長二十五年,限前不得還清;乙,借款以鐵路產業及進款作保,未還清以前,不得以此作他項借款之抵保物,中國自行籌款建築他路,與南滿洲公司無涉;丙,借款本息,由中國政府作保,到期爽約,應由政府代還,或將產業交公司暫管;丁,在借款期內,總工程師應用日本人,並添派鐵路日帳房一員;戊,如遇軍務、賑務,政府在各路運送兵食,均不給價;己,各路進款,應存日本國銀行。四,與南滿洲鐵路公司訂立關於遼河以東之借款合同,及吉長鐵路借款合同。五,中國奉新、吉長鐵路,均應與南滿洲鐵路聯絡,派員會訂章程。六,借款實收價值,照中國最近與他國借款酌定。此約結後,日人又要求吉長鐵路延長至延吉南境,以與韓國會寧鐵道相聯,且照吉長鐵道例,於南滿鐵道會社借資本之半數築之。政府不允,遂成懸案。

三十四年,日使忽提出安奉鐵道案,要求解決。先是滿洲善後協約之附約,允安奉鐵道仍歸日本經營,改為工商業鐵道,規定自此路竣工日起,以十五年為限。至是復提議。郵傳部乃派委員與日本委員會勘改良之新路線。日政府又要求勘定路線即行收買地基。東三省總督錫良祗許按舊線改築,要求日本撤退鐵道守備兵與警察等事,日本不允,令鐵道會社自由起工,海陸皆作警備。乃命錫良會同奉天巡撫程德全與日本奉天總領事締結安奉鐵道協約,此宣統元年七月事也。協約要目如左:一,中國確認前次兩國委員勘定之路線,陳相屯至奉天一段,由兩國再協議決定;二,軌道與京奉鐵道同樣;三,此約調印之當日,即協議購買土地及一切細目;四,此約調印之翌日,即行急進工事;五,沿鐵道之中國地方官,關於施行工事,應妥為照料。

未幾,間島之爭議又起。先是,康熙年間,政府與朝鮮劃定國境,於鴨綠江、圖們江水源之長白山上樹立界碑,規定西以鴨綠江、東以圖們江為兩國國境。因圖們江中有江通灘,地面不及二千畝,因地居江間,四面環水,故以「間島」呼之。此島向屬吉林,惟皇室以長白山一帶為發祥之地,不許人民移居,因之吉林東部所在人煙稀少,間島愈形荒僻。同治間,朝鮮鐘城歲饑,其民多渡圖們江移居間島,按年納地租於我國光霽峪經歷署。光緒初年,朝鮮人忽請免納地租,政府以主權攸關,令朝鮮人退出間島,不果,乃置延吉以治之,間島仍準朝鮮人民居住,按納地租。

日俄戰後,日本伊藤統監命齋藤中佐率兵據之。政府與日使交涉,日使謂光霽峪以東為東間島,和龍峪一帶為西間島,系兩國未定之界。且謂長白山上界碑載土門江為界,朝鮮人稱海蘭河為「土門河」,圖們江系豆滿江,非「土門江」,中、韓國境實為海蘭河。中國以「土門」、「豆滿」、「圖們」均系一音之轉,圖們江北岸界碑矻立,鑿鑿可據。且光緒十三年,朝鮮王致北洋大臣書,聲明鴨綠江、豆滿江為兩國境界,是豆滿江即土門江無疑,執不許。至是,日使伊集院彥吉與外務部尚書梁敦彥重提舊案,締間島條約:一,中、日兩國協約以圖們江為中、韓兩國國境,其江源地方以界碑為起點,依石乙水為界;二,中國準外國人居住龍井村、局子街、頭道溝、百草溝等處貿易,日本於此等地方得設置領事館;三,中國準韓國人民在圖們江北之墾地居住;四,圖們江墾地之韓人,服從中國法權,歸中國地方官管轄及裁判,中國官吏於此等韓人與中國人一律待遇,所有納稅及其他一切行政上處分,亦同於中國人;五,韓人訴訟事件,由中國官吏按中國法律秉公辦理,日本領事或委員可任便到堂聽審,惟人命重案,則須先行知日領事到堂,如中國有不按法律判斷之處,日領事可請覆審;六,圖們江雜居區域內韓人之財產,中國地方官視同中國人民財產,一律保護,該江沿岸,彼此人民得任便往來,惟無護照公文,不得持械過境;七,中國將吉長鐵道延長至延吉南邊界,與朝鮮會寧鐵道聯絡,一切辦理與吉長鐵道同;八,本協約調印後,日本統監府派出所及文武人員於兩月內完全撤退。是約既成,政府以吳祿貞為延吉邊務大臣。

嗣議五案協約,即新法鐵道,營口支線,撫順、煙臺炭礦,安奉鐵道沿線及南滿鐵道幹路沿線之礦務是也。新法鐵道者,新民屯至法庫門之鐵道,政府欲借英款築造此路,以分南滿鐵道之勢力,日本謂系南滿鐵道競爭線,極力抗議。營口支線者,光緒二十五年東清鐵道會社規定築造旅順、哈爾濱間之鐵道,得設營口支線,以運送材料,俟鐵道落成後拆去。日俄戰爭後,南滿鐵道歸日本,政府要求日本拆此支線,日本不允。撫順炭礦,距奉天城東六十里,日公使以此地炭礦為東清鐵道附屬品,利權應歸日本。政府以炭山在東清鐵道三十里外,不認為附屬財產,日使不允;並煙臺炭礦均成懸案。因安奉鐵道交涉,定約如下:一,中國如築新法鐵道時,當先與日本商議;二,中國允日本營口支路,俟南滿鐵道期限滿,同時交還,並允將該支線延長至營口新市街;三,中國承認日本有開採撫順、煙臺兩處炭礦之權,日本承認該兩處開採之煤斤納稅與中國,惟稅率應按照中國他處最輕煤稅之例,另行協定,其礦界及一切章程,亦另委員定之;四,安奉鐵道沿線及南滿洲鐵道幹路沿線之礦務,除撫順、煙臺外,應按照光緒三十三年東三省督撫與奉天日本總領事議定之大綱,歸中、日合辦;五,京奉鐵道沿長至奉天城根一節,日本無異議。自此南滿洲大勢遂一變矣。

吉長、新奉兩路借款細目,旋亦議定。其後錦齊鐵道、渤海漁權與領海、鴨綠江架橋、南滿鐵道附屬電線、收買日本遼東方面軍用電線及旅順芝罘間海底電線諸交涉,次第起焉。錦齊鐵道者,即自錦州經洮南至齊齊哈爾之鐵道也。日本原允中國自修,惟要求昌圖洮南間之鐵道歸日本築造。及滿洲諸協約成,英、美爭錦齊鐵道借款,迭與中國交涉,事皆中阻。渤海漁業與領海交涉,自光緒三十二年,中國課關東漁業團漁稅,迭經日本領事要求住關東之日本人有滿洲沿岸漁業權,日本漁團因避稅,全出距海岸三海裏外海面。東督錫良通告日本領事,謂三海裏外之海面系中國領海,應準中國漁業規則課稅。日本領事以三海裏外為公海,反抗之。鴨綠江駕橋,聯絡滿、韓,議定依安奉鐵道契約,十五年後賣還中國。南滿鐵道附屬電線,原中國所設,日本占有之,後取供公用,中國抗爭無效。又日俄戰爭時,日本在南滿洲所設軍用電線,戰局終,應歸中國收買,日本初起反抗,後始歸中國收買。旅順芝罘之海底電線,系俄國布設,戰時皆斷絕。至此,日本要求依該海底電線直通芝罘之日本電線局,為中國所拒。卒以距芝罘海岸七里半以內之一部歸中國,餘盡屬諸日本。其後復有日俄協約之議,於是東三省大勢又一變矣。


\end{pinyinscope}