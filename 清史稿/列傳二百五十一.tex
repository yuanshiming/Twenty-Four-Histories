\article{列傳二百五十一}

\begin{pinyinscope}
李端棻徐致靖子仁鑄陳寶箴黃遵憲曾鉌

楊深秀楊銳劉光第譚嗣同唐才常林旭

康廣仁

李端棻,字苾園,貴州貴築人。同治二年進士,選庶吉士,授編修,為大學士倭仁、尚書羅敦衍所器。十年,出督云南學政。值回寇亂後,荒服道亙,前使者試未遍,端棻始一一按臨,文化漸振。光緒五年,轉御史,以叔父朝儀官京尹,回避,改故官。累擢內閣學士。十八年,遷刑部侍郎。越六年,調倉場。前後迭司文柄,四為鄉試考官,一為會試副總裁,喜獎拔士類。典試廣東,賞梁啟超才,以從妹妻之,自是頗納啟超議,娓娓道東西邦制度。

維時康有為上書請變法,兼及興學。二十二年,端棻遂疏請立京師大學,凡各省府、州、縣遍設學堂,分齋講習;並建藏書樓、儀器院、譯書局,廣立報館,選派游歷生。二十四年,密薦康有為及譚嗣同堪大用。又以各衙門則例,語涉紛歧,疏請刪訂,上尤善之,詔趣各長官定限期革前敝。擢禮部尚書。未幾,有為等敗,端棻自疏檢舉,詔褫職,戍新疆。中道遘疾,留甘州。二十七年,赦歸,主講貴州經世學堂。三十三年,卒。宣統元年,從雲南、貴州京朝官請,復官。

徐致靖,字子靜,江蘇宜興人,寄籍宛平。光緒二年進士,選庶吉士,授編修。累遷侍讀學士。父憂服闋,二十三年,起故官。致靖嘗憂外患日迫,思所以為獻納計。

子仁鑄,時以編修督湘學,倡新學,書告致靖舉康有為。致靖遂上言:「國是未定,請申乾斷示從違。」藉以覘上意。未幾,詔果求通才,於是致靖奏有為堪大用,並及梁啟超、黃遵憲等。又連上書請廢制藝,改試策論,省冗官,酌置散卿。復以邊患棘,宜練重兵,力薦袁世凱主軍事。上皆然其言,敕依行。罷斥禮部尚書許應騤等阻遏言路,遂命致靖權右侍郎。二十四年八月,太后復出訓政,參預新政諸臣皆獲罪。致靖褫職坐系,尋定永遠監禁,仁鑄亦罷官。庚子,聯軍陷京師,致靖始出獄待罪,詔赦免。卒,年七十五。

陳寶箴,字右銘,江西義寧人。少負志節,詩文皆有法度,為曾國籓所器。以舉人隨父偉琳治鄉團,御粵寇。已而走湖南,參易佩紳戎幕,軍來鳳、龍山間。石達開來犯,軍饑疲,走永順募糧,糧至不絕,守益堅,寇稍稍引去。寶箴之江西,為席寶田畫策殲寇洪福瑱,事寧,敘知府,超授河北道。創致用精舍,遴選三州學子,延名師教之。遷浙江按察使,坐事免。湖南巡撫王文韶薦其才,光緒十六年,召入都,除湖北按察使,署布政使。二十年,擢直隸布政使,入對,時中東戰亟,見上形容憂悴,請日讀聖祖御纂周易,以期變不失常。他所陳奏語甚多,並稱旨。上以為忠,命治糈臺,專摺奏事。馬關和約成,泣曰:「殆不國矣!」

明年,以榮祿薦,擢湖南巡撫。撫幕有任驎者,植黨私利,至即重治之。直隸布政使王廉為關說,據以上聞,廉獲譴。覆按史念祖被劾事,盡暴其任用非人狀,念祖遂褫職。繇是有伉直聲。湘俗故闇僿,寶箴思以一隅致富強,為東南倡,先後設電信,置小輪,建制造槍彈廠,又立保衛局、南學會、時務學堂。延梁啟超主湘學,湘俗大變。又疏請釐正學術及練兵、籌款諸大端,上皆嘉納,敕令持定見,毋為浮言動,並特旨褒勵之。是時張之洞負盛名,司道咸屏息以伺。寶箴初綰鄂籓,遇事不合,獨與爭無私撓,之洞雖不懌,無如何也。久之,兩人深相結,凡條上新政皆聯銜,而鄂撫譚繼洵反不與。

會康有為言事數見效。寶箴素慕曾、胡薦士,因上言楊銳、劉光第、譚嗣同、林旭佐新政。上方詔求通變才,遽擢京卿,參新政,於是四人上書論時事無顧忌。寶箴又言四人雖才,恐資望輕,視事過易,原得厚重大臣如之洞者領之。疏上而太后已出訓政,誅四京卿,罪及舉主,寶箴去官,其子主事三立亦革職,並毀湘學所著學約、界說、劄記、答問諸書。

初,寧鄉已革道員周漢,以張揭帖攻西教為總督所治。寶箴至,漢復刊帖傳布,寶箴令毀之,漢毆毀帖者,寶箴怒,下之獄。舊黨恨次骨,然喜新之士,亦以此翕然稱之。寶箴既去,諸所營構便於民者,雖效益已著,皆廢毀無一存云。卒,年七十。

黃遵憲,字公度,嘉應州人。以舉人入貲為道員。充使日參贊,著日本國志上之朝。旋移舊金山總領事。美吏嘗藉口衛生,逮華僑滿獄。遵憲徑詣獄中,令從者度其容積,曰:「此處衛生顧右於僑居邪?」美吏謝,遽釋之。歷湖南長寶鹽法道,署按察使。時寶箴為巡撫,行新政,遵憲首倡民治於眾曰:「亦自治其身,自治其鄉而已。由一鄉推之一縣、一府、一省,以迄全國,可以成共和之郅治,臻大同之盛軌。」於是略仿西國巡警之制,設保衛局,凡與民利民瘼相麗,而為一方民力能舉者,悉屬之,領以民望,而官輔其不及焉。尋解職,奉出使日本之命,未行而黨禍起,遂罷歸。著有人境廬詩草等。

曾鉌,字懷清,喜塔臘氏,滿洲正白旗人。父慶昀,寧夏將軍。以任子為工部主事,累遷郎中,充軍機章京,轉御史。光緒九年,出為陜西督糧道。西、同各屬農民納糧例繳省倉,道塗艱遠,多弊竇,設法清釐之,民稱便。三輔士風樸僿,藝事苦窳,延長安柏景偉、咸陽劉光蕡主關中書院,督課實學,士論翕然。又設蠶桑局,聘織師教以煮湅織染法,歲出絲帛埒齊、豫。十三年,遷按察使。明年,母憂解職。服除,起故官,俄遷甘肅布政使。二十四年,調直隸,回避,留本任。擢湖北巡撫,慨然曰:「時艱至此,猶可拘成法不變耶?」於是假陜甘總督印上陳補官、掣簽、度支、訟獄四事,宜變通成例,厚植國本。侍讀學士貽穀、光祿寺少卿張仲炘彈其亂政,詔褫職。始,曾鉌官京朝,家綦貧,僦居陋室。及任外臺,孜孜民事,不顧問有無。既閒廢,出入皆徒步,陜民恆歲醵金濟之。後益困,至敝衣鬻卜都市。未幾,卒。宣統改元,總督端方為奏復原官。

楊深秀,字儀村,本名毓秀,山西聞喜人。少穎敏,諳中西算術。同治初,以舉人入貲為刑部員外郎。假歸,值晉大饑,閻敬銘銜命籌賑,深秀條上改革差徭法,困少蘇。光緒十五年,成進士,就本官遷郎中,轉御史。嘗言:「時勢危迫,不革舊無以圖新,不變法無以圖存。」

二十四年,俄人脅割旅順、大連灣。深秀力請聯英、日拒之,詞甚切直。時朝廷銳意行新政,而大臣恆多異議。深秀乃與徐致靖先後疏請定國是,又以取士之法未善,請參酌宋、元、明舊制,釐正文體,下其議於禮部,尚書許應騤心非之,未奏也。會議經濟特科務減額,於是深秀合宋伯魯彈其阻撓。上令應癸自陳,奏上,劾康有為夤緣要津,請罷斥,詞連深秀,上不之詰也。御史文悌劾深秀傳布有為所立保國會,並暴有為交通內外狀,德宗責以代人報復,反獲咎。深秀益感奮,連上書請設譯書局,派王公游歷各國,並定游學日本章程,皆報可。又請試庶官,日番二十人,料簡貞實,而汰其庸愚罷老不諳時務者,繇是廷臣益側目。湖南巡撫陳寶箴圖治甚急,中蜚語,深秀為剖辨之,上以特旨褒寶箴,寶箴乃得行其志。

八月,政變,舉朝惴惴,懼大誅至,獨深秀抗疏請太后歸政。方疏未上時,其子黻田苦口諫止,深秀厲聲叱之退。俄被逮,論棄市。

深秀性鯁直,嘗面折人過,以此叢忌。官臺諫十閱月,封事二十餘上,稿不具存,惟獄中詩三章流傳於世。著有虛聲堂稿、聞喜縣新志。

楊銳,字叔嶠,四川綿竹人。少俊慧,督學張之洞奇其才,招入幕。肄業尊經書院,年最少,嘗冠其曹。優貢朝考得知縣。之洞督兩廣,從赴粵。光緒十一年,舉順天鄉試,考取內閣中書。

二十四年,之洞薦應經濟特科。又以陳寶箴薦,與劉光第、譚嗣同、林旭並加四品卿,充軍機章京,參新政。召見,銳面陳興學、練兵為救亡策,稱旨。七月,禮部主事王照上封事,尚書許應騤等格不奏。上聞,震怒,盡褫尚書侍郎六人革職,朝臣皆不自安。上手詔密諭銳云:「近日朕仰觀聖母意旨,不欲退此老耄昏庸大臣而進英勇通達之人,亦不欲將法盡變。朕豈不知中國積弱不振,非力行新政不可?然此時不惟朕權力所不及,若強行之,朕位且不能保。爾與劉光弟、譚嗣同、林旭等詳悉籌議,必如何而後能進用英達,使新政及時舉行,又不致少拂聖意,即具奏,候朕審擇,不勝焦慮之至!」銳復奏言:「太后親挈大位授之皇上,皇上宜以孝先天下,遇事將順。變法宜有次第,進退大臣不宜太驟。」上是之。

已而太后再訓政,諸言新政者皆予重誅。銳既下獄,自揣實無罪,謂即訊不難白,次日,遽詔與光第等同棄市。宣統改元,銳子慶昶繳手詔於都察院,請代奏,始傳於世。

劉光第,字裴村,四川富順人。光緒九年進士,授刑部主事。治事精嚴,因讞獄忤長官,遂退而閉戶勤學,絕跡不詣署。家素貧,而性廉介,非舊交,雖禮饋弗受。獨與楊銳善。通周官、禮及大小戴禮記。其應召也,亦以陳寶箴薦,然非其素志,將具疏辭,川人官京朝者力勸之。一日,召見,力陳時危民困,外患日迫,亟宜虛懷圖治,上稱善。惟時言路宏啟,臣民奏事日數百計,光第竟日批答,簽識可否,以待上裁。退語所親曰:「吾終不任此,行當亟假歸矣!」未一月而禍作,光第自投獄。臨刑,協辦大學士剛毅監斬,光第詫曰:「未訊而誅,何哉?」令跪聽旨,光第不可,曰:「祖制,雖盜賊,臨刑呼冤,當復訊。吾輩縱不足惜,如國體何!」剛毅默不應,再詢之,曰:「吾奉命監刑耳,他何知?」獄卒強之跪,光第崛立自如。楊銳呼曰:「裴村,跪!跪!遵旨而已。」乃跪就戮。著有介白堂詩文集。

譚嗣同,字復生,湖南瀏陽人。父繼洵,湖北巡撫。嗣同少倜儻有大志,文為奇肆。其學以日新為主,視倫常舊說若無足措意者。繼洵素謹飭,以是頗見惡。嗣同乃游新疆劉錦棠幕,以同知入貲為知府,銓江蘇。陳寶箴撫湖南,嗣同還鄉佐新政。梁啟超倡辦南學會,嗣同為之長。屆會期,集者恆數百人,聞嗣同慷慨論時事,多感動。

光緒二十四年,召入都,奏對稱旨,擢四品卿、軍機章京。四人雖同被命,每召對,嗣同建議獨多。上欲開懋勤殿,設顧問官,令嗣同擬旨,必載明前朝故事,將親詣頤和園請命太后。嗣同退謂人曰:「今乃知上絕無權也!」時榮祿督畿輔,袁世凱以監司練兵天津。詔擢世凱侍郎,召入覲。嗣同嘗夜詣世凱有所議。明日,世凱返天津。越晨,太后自頤和園還宮,收政權。啟超避匿日本使館,嗣同往見之,勸嗣同東游。嗣同曰:「不有行者,無以圖將來;不有死者,無以酬聖主。」卒不去。未幾,斬於市。著有仁學及莽蒼蒼齋詩集等。

唐才常,字佛塵。少與嗣同齊名,稱「瀏陽二生」,兩湖學堂高材生也。聞嗣同死,憂憤,屢有所謀,每言及德宗,常泣下。二十六年,兩宮出狩,才常陰結富有會謀舉事,號勤王,將攻武、漢。被獲,慷慨言無所隱,請就死,遂殺之。

林旭,字暾谷,福建侯官人。年十九,舉本省鄉試第一。後試禮部,值中日構釁,糾同試者上書論時事,不報。入貲為內閣中書。時康有為倡言變法,先於京師立粵學會,以振厲士氣,而蜀學、浙學、陜學、閩學諸會繼之。旭為閩學會領袖,又充保國會會員。榮祿先為福州將軍,雅好閩士,及至天津,延旭入幕。俄以奏保人才召見,操土語,上不盡解。退繕摺,上稱善,遂命與譚嗣同等同參機務,詔諭多旭起草。及變起,同戮於市,年二十有四。著有晚翠軒詩集。妻沈葆楨孫女,聞變,仰藥不死,以毀卒。

康廣仁,名有溥,以字行,有為弟。少從兄學。有為上書請改革,廣仁謂當先變科舉,庶人才可出。其後罷鄉會試、制藝,而歲科試未變,廣仁激勵言官抗疏論之,得旨俞允。於是廣仁語有為:「今科舉既廢,宜且南歸興學專教育,俟養成多數有用才,數年後乃可云改革也。」有為不忍去。及初聞變,廣仁復趣有為歸。有為走,廣仁被逮。在獄言笑自若,臨刑猶言曰:「中國自強之機在此矣!」

論曰:戊戌變法,德宗發憤圖強,用端棻等言,召用新進。百日維新,中外震仰,黨爭遽起,激成政變。銳、光第、嗣同、旭及深秀、廣仁同日被禍,世稱「六君子」,皆悲其志。內爭不已,牽及外交。其後遂釀庚子排外之亂,終致危亡。此亦清代興衰一大關鍵也。


\end{pinyinscope}