\article{列傳一百十八}

\begin{pinyinscope}
海蘭察子安祿奎林珠勒格德和隆武額森特普爾普

海蘭察,多拉爾氏,滿洲鑲黃旗人,世居黑龍江。乾隆二十年,以索倫馬甲從征準噶爾。輝特臺吉巴雅爾既降,復從阿睦爾撒納叛,師索之急,遁入塔爾巴哈臺山中,海蘭察力追及之,射墜馬,生獲以歸,敘功,賜號額爾克巴圖魯。累擢頭等侍衛,予騎都尉兼雲騎尉世職,圖形紫光閣。三十二年,以記名副都統從征緬甸,師出虎踞關,海蘭察率輕騎先驅,至罕塔,遇賊,殪三人,俘七人,遂攻老官屯,馘二百;設伏,殲賊四百,賊自猛密出襲我師,援擊卻之。三十三年,再出師,度萬仞關,敗賊戛鳩江,毀江岸賊居,授鑲黃旗蒙古副都統。師薄老官屯,攻賊於錫箔,毀其木柵,賊來攻,急擊之,追戮其強半,縛二人以歸。既還師,命留軍防邊。移鑲白旗蒙古副都統。

三十六年,師征金川,命自雲南赴四川與師會。三十七年六月,參贊大臣豐升額方攻美美寨,賊御戰甚力。海蘭察師至,合力奮擊,克之;乘勝毀賊寨十三,克木城,師屯其旁山岡,築卡以守。七月,敗賊策卜丹。八月,賊出貢噶山左,謀截糧,海蘭察設四伏,斬級百餘。十月,進攻路頂宗及喀木色爾,破碉卡三百餘,殲賊數百,詔嘉獎,擢正紅旗蒙古都統。十一月,進至格實迪,自色木僧格山後取瑪覺烏大寨,仰攻布喇克及扎喀爾寨,得碉卡九十。十二月,進攻明郭宗,突入寨門,焚轉經樓,直搗美諾。

小金川既定,進討大金川,授參贊大臣,從將軍溫福出西路,自功噶爾拉入。三十八年二月,趨昔嶺,道經蘇克奈,奪卡二,據木果木後山,與領隊大臣額森特軍合戰,得碉卡五,鑿冰開道,一日而至固木卜爾山。山接昔嶺麓,昔嶺多賊碉,當道碉凡十,我師遇賊碉,若山峰縱橫並列,往往為之次第,便指目。海蘭察與額森特計分兵為六隊,力攻第九、第十二碉,先下,進取第七、第八兩碉,力戰冰雪中。及暮,陽撤兵,賊下追,伏起,殪二百人。第五碉尤堅厚,海蘭察運砲轟擊,晝夜無稍休,碉乃破。移軍攻達扎克角山梁,奪獲得斯東寨。上按地圖示諸將形勢,海蘭察復移軍攻功噶爾拉山口。五月,還攻昔嶺,造砲臺高與山齊,痛殲守賊。六月,後路賊攻陷底木達,進據登春。海蘭察還御,戰正力,俄聞木果木大營有警,疾馳。次日大營陷,將軍溫福歿於陣。海蘭察令領隊大臣富興整兵出,而為之殿。夜半,至功噶爾拉總兵牛天畀營,度功噶爾拉亦不可守,合軍引退,令額森特等為前導,與富興、普爾普及天畀殿。是日暮,屯崇德。次日至美諾,與領隊博清額、五岱、和隆武合軍,馳奏請罪。上諭以「鎮靜,鼓士氣,圖恢復」。與五岱共守美諾,賊屢來攻,均戰退。

時當新敗,綠營兵多潰散。海蘭察請遣回怯卒,毋使搖亂新兵,上從其請。尋詗知阿桂方駐軍當噶爾拉,乃分兵千人,令額森特自南山往迎;又令普爾普將三百人巡鄂克什諸隘口。七月,賊大至,美諾、明郭宗俱失守,海蘭察退保日隆。上責其不能禦賊,命阿桂按治。阿桂至日隆,奏:「海蘭察當兵潰時,前後攔截,未與懦卒同潰。惟平日不能申明軍律,咎不能辭。」命左授領隊大臣,停俸。十月,命以阿桂為定西將軍,謀再舉,海蘭察偕領隊常清等將八千人自達木巴宗北山取道分三路進,奪別斯滿大小十餘寨。復與富興等攻取帛噶爾角克、底木達、布朗郭宗諸寨,師復克美諾。上嘉海蘭察奮勉,命支俸。

三十九年正月,阿桂令海蘭察將五千人自明郭宗進谷噶山擊賊,又令與保寧將二千人自喇穆喇穆橫梁繞八十餘里,攻登古山。登古山在諸山最峻,羅博瓦山與對峙,亦賊中奇險處。二月,令普爾普順山梁進,海蘭察出山後,自石罅躍登,搏賊酣戰;額森特、保寧至,合力擊賊,賊少卻;復分隊冒死沖突,射之,殪數十人,餘賊負矢遁。乃還取羅博瓦前山,攻第三、第四峰,而額森特攻第二峰,普爾普攻第一峰,俱克之。上諗羅博瓦為賊險要門戶,海蘭察力攻功最,授內大臣。

三月,從第四峰下,進攻得斯東寨,克之。四月,賊乘霧雨於山坡立兩碉,海蘭察率兵毀之。五月,於喇穆喇穆山後築柵,賊屢自林中來犯,與額森特合擊,賊披靡走。六月,攻色淜普岡,賊設大碉六,互相應。額森特克左兩碉,烏什哈達克右一碉,海蘭察獨克中三碉及附近卡寨。七月,抵色淜普,南崖石壁陡滑,督兵手足攀援上,殲東西峰守碉賊殆盡。又自喇穆喇穆山麓乘勝攻日則丫口,取碉卡百餘,賊堅守該布達什諾木城。師循山溝,海蘭察出其左,額森特出其右,官達色出中路,三道並進,遂逼遜克爾宗。上嘉海蘭察為諸將倡,屢克險要,賜號綽爾和羅科巴圖魯,並賚白金三百。

八月,偕額森特自遜克爾宗峰脊分左右翼仰攻,登碉頂,縱火毀碉卡二百餘;又旁出遜克爾宗西,逼賊寨,督兵躍進。賊穴地匿,不敢出。九月,取遜克爾宗水碉,斷賊汲道。乘勝攻官寨,賊槍石如雨,督兵奮進,額森特取其右第一寨;海蘭察左頰傷,裹創力戰,克第二寨。軍中目賊渠所居大寨為「官寨」,亦曰「正寨」,示與他碉卡別也。上以海蘭察傷甫平,即督兵攻奪堅碉,手敕嘉獎。十月,克默格爾山梁及密拉噶拉木,得大寨一、石碉四,山後凱立葉官寨亦下,復授參贊大臣。又自默格爾西進攻布拉克森及格思巴爾,焚寨落數百,於是凱立葉附近碉卡皆盡。命在御前侍衛上行走。

十一月,夜度山溝,進格魯克古丫山,崖磡壁立,督兵揉登,天明,登者六百人,賊並力拒,奪二碉,循山梁下攻桑噶斯瑪特;別遣兵自陡烏當噶山進克沙木拉渠什爾德諸寨,復督兵攻克革什戎岡及作固頂。賊寨橫越諸山,下溝上梁,鼓勇徑度,盡克諸碉寨,與丹壩軍合。十二月,抵桑噶斯瑪特山,賊於碉外設木城為護。師自柵隙發矢,或拔柵木撞之,城立毀。四十年正月,自康薩爾分路進剿,據山溝碉寨。二月,克甲爾納沿河諸寨。進攻勒吉爾博寨,海蘭察克山麓碉二。賊自噶爾丹寺來援,擊敗之。四月,將軍阿桂令往宜喜,會明亮詗兵入道,約期合攻。上賜緞二端。

尋分兵千人偕福康安赴宜喜,先取甲索賊碉,進攻得楞山岡,皆下,焚薩克薩穀大小寨落數百,西北兩路兵合。五月,攻上、下巴木通大碉,並克色爾外、安吉、達佳布諸寨,焚噶爾丹寺。六月,自榮噶爾博山梁攻巴占寨落,賊恃險拒攻,未下;紆道繞舍圖枉卡以入。海蘭察督兵進據昆色爾山梁,克果克多碉,進至拉枯喇嘛寺。再進經菑則大海,又攻章噶上下十餘寨,盡克之。合諸路兵逼勒烏圍,海蘭察自托古魯逾溝直上山梁。八月,取隆斯得寨三,分地設伏,遂克勒烏圍。

九月,整軍進攻噶拉依。初自達思裏正路入,慮賊防密,改自達烏達圍進。海蘭察繞至莫魯古上,連奪噶克底、綽爾丹諸寨,又克西里山梁並科布曲諸碉。十月,攻達噶,自中路入,分兵張兩翼出旁徑,克兩堅碉,下攻雅瑪朋寨。閏十月,據黃草坪,築柵斷賊援。賊起木城,海蘭察督兵陟山,自上壓下,克之。十一月,分道攻奔布魯木,夜迫山下,焚賊木城,遂據西里正寨。又克舍勒固租魯寨四。進攻雅瑪朋正寨,從中路設伏,偕普爾普等盡克附近寨落。十二月,克勒隈勒木通石碉,築柵至科布曲。海蘭察冒槍石進,乘勝克索隆古、得木巴爾、們都斯諸寨。賊又於布哈爾下積木設伏拒師,海蘭察分兵三道並進,立時攻破,遂取奇石磯;又遣兵悉收庫爾納、額木里多諸寨,及巴斯科官寨。四十一年正月,克舍齊、雍中兩寺。海蘭察屯兵噶拉依河岸,扼要隘。尋偕福康安、普爾普等截噶拉依右路,克大石卡,移砲進擊扎木什克寨。二月,大金川酋索諾木就縛,金川平,封海蘭察一等超勇侯,賜雙眼花翎。師還,郊勞,賜御用鞍轡馬一。飲至,賜緞二十端、白金千。圖形紫光閣,列前五十功臣。授領侍衛內大臣。補公中佐領。

四十六年三月,甘肅撒拉爾回蘇四十三爭立新教為亂,破河州,據華林山。命大學士阿桂視師,疏請以海蘭察自佐。上已命為領隊,馳驛詣軍前。四月,抵蘭州,督兵攻龍尾山,賊伏穴中守。阿桂至,令海蘭察盡護諸軍。五月,偕明亮、額森特等分左右翼陟山殺賊。復逾水磨溝,猝上華林山,賊駭,傾穴出;師陽退,賊來逐,還兵擊之,殲賊甚眾。賊被創鉅,望見海蘭察乘馬出陣,輒先驚竄。閏五月,將阿拉山馬兵繞出華林山江南潛伏,候賊至,突出壕殺賊;又督屯練兵取賊卡四,步戰中槍傷。上憫其勞,諭阿桂撫慰。賊據大卡負嵎,海蘭察單騎至五泉山審度,還向華林山暫伏壕中,詗賊還,急起猛攻,遂克之。入賊營,焚所居板屋。賊退保華林寺,督兵逼寺立柵,殲賊眾,馘渠傳示各回民。賊平,上諭獎海蘭察功,官其子安祿三等侍衛。四十九年四月,甘肅回復私起新教,聚眾滋事。命尚書福康安視師,授海蘭察參贊大臣。賊屯靜寧底店,海蘭察督巴圖魯侍衛等進逼賊巢,設伏痛殲之,遂破石峰堡,擒賊渠張文慶等。擢安祿二等侍衛,予騎都尉世職。

五十二年,臺灣林爽文為亂,命將軍福康安視師,仍授海蘭察參贊大臣。十月,渡鹿仔港,登岸後三日,率巴圖魯二十人至彰化八卦山察地勢。賊方於山上築卡,海蘭察躍馬登,賊擁至,發箭殪數賊,餘驚遁。上以其能用少擊眾,諭獎之。十一月,自笨港開道,同福康安援嘉義,分隊五,沿途搜剿,自侖仔頂、侖仔尾逼至牛稠山,賊萬餘阻溪守。海蘭察越溪徑上山梁,攻克賊柵,賊遁,追至大排竹,盡焚賊藔,嘉義圍解。上嘉海蘭察身先士卒,勇略過人,進二等超勇公,賜紅寶石頂、四團龍補褂。

十二月,剿城西大侖莊及海岸賊,又焚城東興化店、員林賊莊,督兵直剿北路。時賊屯中林,尤剽悍,海蘭察冒槍石馳剿,克之。大埔林、大埔尾諸莊賊俱潰。收斗六門,抵水沙連,賊已遁。尋蹤搜捕,見賊渠方乘馬執幟,射墜馬,獲以歸。進攻大里杙,林爽文起事地也,殲賊目數十、賊黨二百。林爽文逃入番社,即自內山平砦仔逐賊至集集埔。賊砦前阻大溪,海蘭察策馬逕渡,盡殲砦中賊,追十餘里,至浩淮角,焚草藔千。進剿小半天山寨,海蘭察遍歷東勢角山峰獅子頭、打鐵藔、叚骨、合歡諸社,至極北炭窯,捕治餘賊。五十三年正月,得爽文於老衢崎,檻送京師。上念海蘭察功,解佩囊賜之。二月,還兵至南路,自彎里社至極南瑯嶠,執賊渠莊大田,磔於市。臺灣平,賜紫韁、金黃辮珊瑚朝珠,再圖形紫光閣。

五十六年,廓爾喀侵後藏,仍以福康安為將軍,海蘭察為參贊大臣,率巴圖魯侍衛及索倫兵千人往討。出西寧,明年三月,抵後藏。閏四月,抵第哩浪古。與福康安分往絨轄、聶拉木察地勢,定策自濟嚨進兵。海蘭察偕阿滿泰出中路,賊兩碉前後相輔,師奪前碉,賊守後碉不出;督兵毀旁垣入,短兵接,殺賊目三、賊兵二百,進屯擦木。乘勝克瑪噶爾轄爾甲山梁,賊渠率眾陟山,我兵暫伏,賊至山半,橫擊之,賊且戰且退,海蘭察疾馳下擊賊,斬賊渠七、賊二百餘,俘三十。海蘭察馬足中槍,上聞,戒以「接仗時宜持重,毋輕冒險」。

師進攻濟嚨官寨,海蘭察與臺斐英阿督索倫兵往來沖擊,自丑至亥,克之,斬賊六百,俘二百。自濟嚨進至索喇拉山,山下有石卡。師直攻之,賊棄卡奔。逐至熱索橋,賊撤橋,攻之不及。海蘭察密令阿滿泰等東越峨綠山,自上流潛渡,賊駭奔,墜河者甚眾。師悉渡,遂據熱索橋,進至密哩頂,越崇山數重,抵旺噶爾,深入八百七十里,不見賊。旺噶爾西南有大川橫亙,北曰旺堆,南曰協布魯,迤東為克堆寨,賊各築卡以守。師至旺堆,賊扼河抵御,不得渡,乃留兵牽賊;密從上游縛木以濟,出賊不意,直薄克堆寨,大敗之。六月,督兵自協布魯進,由噶多東南越雅爾賽拉山,晝夜行,至博爾東拉前山。賊築木城三、石卡七,據要隘,乃轉從山巔下臨賊卡,與阿滿泰上下夾擊,諸城卡盡下;乘勝逐賊至瑪木拉,殺伏賊百餘人。師屯雍雅山,廓爾喀乞降,拒不許。七月,進攻噶勒拉山,三道皆勝。逐賊至堆補木山,奪其卡。山下為帕朗古橫河,賊扼橋以拒。官兵奪橋渡,馳上甲爾古拉山;別兵從上游潛渡,抵集木集山,合軍。賊來侵,往來迎擊,戰兩日夜,越大山二,克木城四、大小石卡十一,戮賊目十三,斃賊六百,俘十七。廓爾喀渠畏懼,力請降,詔許之,進海蘭察一等公。

五十八年三月,卒,謚武壯。復圖形紫光閣,甫成,上制贊嗟惜,諭曰:「海蘭察以病卒,例不入昭忠祠。念其在軍奮勉,嘗受多傷,加恩入祀。」

子安祿,襲公爵,授頭等侍衛。嘉慶四年,佐經略勒保征四川教匪,戰屢有功。賊渠茍文明等窺開縣,安祿與總兵硃射鬥合軍逐剿,賊不敢東竄。十一月,與射鬥逐賊枯草坪,乘雨登汪家山殺賊,賊多墜崖死。安祿望見數十賊匿山溝,率數騎逐之,賊潰散,獨策馬從其後,數賊自林中出,安祿倉卒中矛死。謚壯毅,賜白金千治喪,加騎都尉世職,合前賜騎都尉為三等輕車騎尉。是時奎林子惠倫亦戰沒。上以二人皆名將子,與烏合亂民戰,沒於行陣,深致惜焉。

奎林,字直方,富察氏,滿洲鑲黃旗人,承恩公傅文子也。自拜唐阿襲雲騎尉,擢雲麾使,襲承恩公爵,授御前侍衛。累遷鑲白旗護軍統領,管理健銳營。

乾隆三十七年,授領隊大臣,從副軍阿桂征金川,與侍衛和隆武攻納圍山梁,攻當噶爾拉。木果木師潰,命阿桂為定西將軍,召奎林入咨軍事。旋命佐副將軍明亮出南路,自墨壟溝進攻得里。賊築碉山嶺,奎林率兵晝伏夜行,至其側,突擊破之。攻拉約,夜渡河,鼓譟,克賊壘,遂抵僧格宗,連破石碉,獲軍糧火藥。時阿桂復美諾,明亮遣奎林往會師。復從明亮攻斯第,奎林率第一隊兵先占班得古水泉,與賊持兩晝夜,涉險鏖戰,飛石傷脊。兩賊握利刃突前,侍衛珠勒格德射之,殪,餘賊驚逸。上諭嘉奎林勇猛。攻達爾圖,賊碉綿亙數里,奎林冒雨先登,立拔第一碉。官軍乘勢疾擊,克碉十五,俘賊目八,獲糧械無算。復自木克什山梁進克賊碉一,中槍傷頂,上諭曰:「奎林平日戰甚力,今頂傷中要害。」時富德軍於馬爾那,令奎林代防,即以富德佐明亮擊賊。旋授鑲紅旗漢軍都統。

傷愈,復從明亮攻宜喜。阿桂遣領侍衛內大臣海蘭察會奎林度地勢,約兩軍隔河夾擊,直搗勒烏圍。勒烏圍、噶拉依,兩金川渠所居地也。奎林分攻甲索,又自薩克薩穀攻得楞,賊棄碉竄,乘勝追躡,墮崖死者相枕籍。攻基木斯丹當噶,奪碉二、卡九,又奪茹寨麥田十餘里,賜繃武巴圖魯名號。復趨噶西喇嘛寺,拔沙爾尼溝碉卡。阿桂破勒烏圍,奎林偕明亮、和隆武等攻扎烏古山,未克,請益兵。上諭奎林、和隆武:「毋以勇往好勝,愧激輕進。雖云『不入虎穴,焉得虎子』;亦當番度機要,權利害而行,不可冒昧。」旋自什扎古進兵,偕和隆武自山溝潛行,登其巔,碉內賊無一脫者。上諭明亮、奎林、和隆武:「宜黽勉立勛,毋讓西路專美!但當度利害,不可但知輕進。」進克扎烏古山梁。再進據納木迪、斯底葉安,奪三十餘寨。又自耳得穀下擊賊碉卡,斃賊百餘。復自碾占進攻,達撒谷,拔碉卡三十,斃賊百。趨獨古木思得,賊潰,平山上下八十餘寨。師經乃當,降其渠。攻甲雜,俘賊酋,降其眾千餘。克卡拉爾,抵舍斯滿,賊出降。奎林繞山巔行三百里,至底角河沿,撫定寨落數百,遂與阿桂軍合圍噶拉依。上加奎林一等男,命其子崇倫承襲,並賜雙眼花翎。遂俘金川酋索諾木。師還,凱旋,上郊勞,賜文綺十二、銀五百、御用鞍轡馬一。圖形紫光閣,列前五十功臣。授右翼前鋒統領,擢理籓院尚書。

四十五年,出為烏魯木齊都統。驍騎校常福杖斃披甲多羅,奎林論劾,上以多羅不孝,罪當死,責奎林誤劾。改授烏里雅蘇臺將軍。坐在烏魯木齊失察各州縣浮報糧值,命以公爵畀其叔傅玉承襲。復授烏魯木齊都統。遷伊犁將軍。

奎林貴戚有軍功,耆酒躁急。五十二年,參贊海祿疏劾,上命烏魯木齊都統永鐸勘奏。逮至京師,命諸皇子、軍機大臣會刑部按治,獄成,奎林坐擅殺罪人,擬杖;海祿所劾不盡實,亦有罪,坐誣告,死罪,未決,擬流;帝以奎林孝賢皇后侄,而祿海所論劾不盡虛,擬罪乃反重,失平,命俱奪職,在上虞備用處拜唐阿上效力。旋授奎林藍翎侍衛,再遷臺灣鎮總兵。時林爽文亂甫平,多盜,為民害。上欲嚴懲之,諭奎林:「勿拘泥,勿姑息,有犯必懲。」奎林屢捕治劇盜,復論誅裨將坐贓及營兵之為盜者,稱上旨,加提督銜。五十六年,擢福建水師提督。師征廓爾喀,改授成都將軍、參贊大臣,帥師入藏。五十七年,行至江卡,疽發於頂,遂卒,謚武毅。

珠勒格德,鈕祜祿氏,滿洲正白旗人。以三等侍衛從軍。其救奎林也,上命擢一等侍衛,賜號扎克博巴圖魯。戰於木克什,據水卡,斷賊汲道,設伏以待。賊乘霧分道來犯,守碉兵御之,伏起;賊復自山下援,珠勒格德突入陣,刃三人,大敗之,遂克木克什山下碉。復與都統和隆武等襲取日旁山後碉十餘,日旁近勒烏圍,賊碉寨相望,後路必爭地也。授正紅旗蒙古副都統。奎林攻什扎古,珠勒格德與和隆武設伏瑯谷,奎林兵至,夾擊,破木城;進攻扎烏古,克賊碉四、卡八。自日新滿至巴扎木,賊碉林立,珠勒格德與和隆武分兵進,連克賊碉十七。金川平,圖形紫光閣,禦制贊猶及救奎林事。尋卒。

和隆武,馬佳氏,滿洲正黃旗人,寧夏將軍和起子也。初隸鑲藍旗,以和隆武功,高宗命以本佐領抬入正黃旗。凡抬旗,或以功,或以恩,或以佐領,或以族,或以支,皆出特命。和隆武襲一等子爵,授三等侍衛。

乾隆三十七年,從護軍統領明亮征金川,自墨壟溝攻甲爾木山梁。師分道而進,和隆武為領隊侍衛,明亮攻美諾喇嘛寺,和隆武傍水夾攻,賊潰而復聚,盡殲之,夜克美諾諸碉寨,復分攻納圍正面山梁,敗賊於鳩寨,奪碉五十餘,遷鑲藍旗蒙古副都統。旋收僧格宗。從富德攻克絨布寨北沃什山、摩格、孟格、里格、穆圖德宗,進攻卡角。從奎林等取斯第,賊迎戰,和隆武麾眾蕩決,矢盡,以矛斗,被創,賜玉搬指、荷包。進攻克木克什第一碉,賜黃馬褂。師攻日旁,和隆武自周叟繞出其後,突入碉,賊驚潰,槍石不及施,短刃相搏,循山逐賊碉十餘,隳二百餘,日旁賊殲焉。復偕珠勒格德攻穀爾堤諸地碉寨,盡克之。上屢詔嘉眾,授正藍旗蒙古都統。進攻得楞以南碉卡,又進攻額爾替山梁,殺賊甚眾。賊據石真噶,和隆武與奎林乘勝運砲,軍甚囂,分隊突出攻據之,賊奔潰。四十年七月,阿桂師逼勒烏圍,而和隆武與明亮、奎林合軍出北路,自扎烏古山進。語已具奎林傳。

四十一年,金川平,進和隆武三等果勇侯,賜雙眼花翎。師還,賜御用鞍轡馬一,並賚銀幣。圖形紫光閣,列前五十功臣。出為寧夏將軍,移吉林將軍。卒,謚壯毅。

額森特,臺褚勒氏,滿洲正白旗人。以前鋒馬甲從征伊犁。右部哈薩克與塔什罕相攻,參贊大臣富德使額森特諭哈薩克內附,使入覲,額森特護至京師。擢藍翎侍衛。遷二等侍衛。乾隆三十四年,從經略大學士傅恆征緬甸,攻老官屯,賊出戰,額森特率索倫兵擊敗之。

三十六年,從定邊右副將軍溫福征小金川,攻巴朗拉,奪其東山峰,毀碉,賜號丹巴巴圖魯。師取達木巴宗,額森特由別道出山北,連破碉卡。至資哩,合師,奪北山。賊乘夜築卡,將兵邀擊,賊數百踵至,三卻三進,額森特中槍,力擊敗之,遂克資哩。復策取普爾瑪寨。攻東瑪,連戰敗賊,擢頭等侍衛。賊分兩道出戰,伏兵逆擊,賊大敗;薄其碉,身被創,大呼殺賊,遂克東瑪。進克美美卡,拔路頂宗山碉,授鑲黃旗蒙古副都統。至博爾根,奪山巔大寨。夜渡水,仰攻納拉覺山,克碉十二、卡十五。擊格實迪,破公雅山。逾木爾古山麓,取溝內寨卡,據嘉巴山,授領隊大臣。

小金川平,復從將軍溫福至功噶爾拉山。功噶爾拉者兩金川接壤要隘也,峰陡絕,積雪封徑,賊碉扼險。額森特督兵直上,副都統烏什哈達繼之,漸克旁碉,戰於固木卜爾山,敗賊。從溫福移營木果木,會攻昔嶺,賊碉密布,與海蘭察合攻,冰雪中相持數十日,木果木軍潰。副將軍阿桂在當噶爾拉,全師撤駐翁古爾壟。上命阿桂為定邊將軍,再進,額森特與總兵海祿奪北山橋卡。總兵成德至,三路合攻阿喀木雅山,乘勝取木蘭壩,平鄂克什官寨。師至路頂宗,額森特越山攀堞躍入,刃賊數十,墮崖死。進攻明郭宗,遂復美諾,授正紅旗護軍統領,賜御用黑狐冠。

偕海蘭察至穀噶山下,有橫梁曰喇穆喇穆,峰勢峻險。海蘭察與侍衛公保寧從旁進,額森特當其前,夜乘雪影穿箐越險,直前奮擊,轉戰至黎明,已二十餘里,始見高峰列大碉九,繚石墻。俄雪又作,乘晦抵碉趾,賊不敢出,乃攻取其左、右山梁及附近儹巴拉克山峰。夜擊梁東色依谷山,與海蘭察兵合。海蘭察據登古山,與羅博瓦山相對,險特甚。共率兵由石罅躍登,林中砲石如雨,及第三峰麓,賊數百分隊迎擊,卒敗之,攻克第二峰碉。上獎其奮勉,授散秩大臣。進剿得斯東寨,斫寨門,縱火,賊出,殺之。雪夜,賊劫副將常祿保營,額森特聞槍聲赴援,賊敗走。賊乘雨霧建二碉於羅博瓦山,額森特與海蘭察率兵八百,夜雨中薄碉,毀墻入,賊驚竄,平其碉。賊夜劫烏什哈達營,追擊敗之。

賊於羅博瓦峰下色淜普大岡置大碉六,左右相應援。海蘭察克其中三碉,額森特克其左二,烏什哈達克其右一,山砦皆平,上嘉之,制詩紀事。額森特於大雨中攻色淜普左偏,砍柵進,克二木城,遙見該布達什諾各砦煙起,知海蘭察兵至,遂乘機奪筆郎納克、該筆達烏諸砦,改墨爾根巴圖魯,賜白金二百。

師圍遜克爾宗,額森特與海蘭察毀平房、碉卡二百餘。克水碉,攻官寨,自叢木中驟逼寨墻,賊死戰,額森特傷鼻及足;撲第三寨,賊舉槍折其弓弰,傷指,易弓,連斃數賊。上以額森特被傷能易弓射賊,手詔嘉獎,賜貂冠、猞猁猻褂。攻默格爾山,與海蘭察共攻克密拉噶拉木碉及凱立葉官寨。敗勒烏圍援賊,馘百餘,授參贊大臣。乘勝取布拉克森及格斯巴爾二山,毀山下羅卜克鄂博溝口七碉,於是凱立葉上下及附近寨落皆平。上獎其奮勉超群,命在乾清門行走。

復與海蘭察分隊乘月黑度山溝,入格普古丫口,得碉卡十二。抵桑噶斯瑪特,破石城、木柵,奪擦庸、群尼二寨。攻上下巴木通,克之。下寨落百餘,賊不敢復拒。至直古腦山頂,與福康安兵合,直趨勒烏圍賊巢。賊負高阻深,力戰克之。額森特負傷不能乘馬,上命駐守勒烏圍。額森特隔河見明亮兵攻阿爾古,發砲助之。上聞,曰:「額森特不分畛域,無愧為參贊!」額森特望見攻西里官兵得捷,率保寧、常祿保等攻西里山麓,克其木城。勒烏圍前山曰克爾古什拉斯者,取噶拉依正道也。賊於山上城碉密布,額森特攻克之。乘勝取格隆古。師將逼賊巢,賊恃布哈爾、則朗噶克為門戶,斫木塞道。額森特率諸將烏爾納、那木扎、彰靄等進攻,賊伏積木中,發槍如雨。額森特乘柵以登,設伏兵夾擊,賊遂驚潰。進克喀爾巴山後,毀附近寨落,遂薄噶拉依。上嘉額森特勇,封一等嫺勇男,世襲。金川平,賜御用鞍馬、緞二十端、白金千。圖形紫光閣,列前五十功臣。

四十六年,循化回蘇四十三因爭立新教為亂,破河州,命從大學士阿桂討之,額森特與海蘭察、明亮等分攻華林山,力戰被傷。賊平,進三等子。四十七年,卒。

普爾普,額爾特肯氏,蒙古正黃旗人。父巴圖濟爾噶爾,本額魯特杜爾伯特部宰桑。來降,隸蒙古正黃旗。從征準噶爾,討霍集占,皆有功。官至內大臣,賜騎都尉世職,圖形紫光閣。

普爾普自閒散再遷三等侍衛。從征緬甸,擢御前侍衛,授公中佐領。乾隆三十七年,命率額魯特兵詣金川,從定邊右將軍溫福進討。師攻達克蘇,普爾普奪賊卡,斷賊來路。從參贊大臣豐升額攻明郭宗,命為領隊侍衛,偕巴雅爾取明郭宗南寨,加副都統銜。進攻噶爾拉,經丫口,盡得賊卡寨。偕副都統海蘭察攻昔嶺,克要路碉二。普爾普與海蘭察、額

前山曰克爾古什拉斯者,取噶拉依正道也。賊於山上城碉密布,額森特攻克之。乘勝取格隆古。師將逼賊巢,賊恃布哈爾、則朗噶克為門戶,斫木塞道。額森特率諸將烏爾納、那木扎、彰靄等進攻,賊伏積木中,發槍如雨。額森特乘柵以登,設伏兵夾擊,賊遂驚潰。進克喀爾巴山後,毀附近寨落,遂薄噶拉依。上嘉額森特勇,封一等嫺勇男,世襲。金川平,賜御用鞍馬、緞二十端、白金千。圖形紫光閣,列前五十功臣。

四十六年,循化回蘇四十三因爭立新教為亂,破河州,命從大學士阿桂討之,額森特與海蘭察、明亮等分攻華林山,力戰被傷。賊平,進三等子。四十七年,卒。

普爾普,額爾特肯氏,蒙古正黃旗人。父巴圖濟爾噶爾,本額魯特杜爾伯特部宰桑。來降,隸蒙古正黃旗。從征準噶爾,討霍集占,皆有功。官至內大臣,賜騎都尉世職,圖形紫光閣。

普爾普自閒散再遷三等侍衛。從征緬甸,擢御前侍衛,授公中佐領。乾隆三十七年,命率額魯特兵詣金川,從定邊右將軍溫福進討。師攻達克蘇,普爾普奪賊卡,斷賊來路。從參贊大臣豐升額攻明郭宗,命為領隊侍衛,偕巴雅爾取明郭宗南寨,加副都統銜。進攻噶爾拉,經丫口,盡得賊卡寨。偕副都統海蘭察攻昔嶺,克要路碉二。普爾普與海蘭察、額森特、巴雅爾、烏什哈達、馬全、阿爾納素戰尤力。復與諸將攻斯達克拉、阿噶爾布里、碩藏噶爾山梁,克之。進攻色布色爾山梁,得賊碉十餘。羅博瓦者,金川渠所恃為門戶者也,師進,悉據其諸峰,授散秩大臣。賊劫副將常祿保,援擊敗之。與海蘭察合攻喇穆喇穆,射殺紅衣賊渠。又拔該布達什諾木城二,賜御用黑狐冠。賊劫我軍所置卡,與烏什哈達赴援,賊潰。攻遜克爾宗,中創,復攻舍圖旺,斷遜克爾宗去路。偕臺斐英阿等攻章噶,得賊寨二十餘。又克隆斯得寨,賊貯鉛丸火藥處也,遂偕臺斐英阿等克勒烏圍,賜什勒瑪咳巴圖魯名號。進攻阿穰曲強達巴,克大碉三、木城四。仰攻西里山峰,賊越碉竄,普爾普逐捕,所殺傷過當。攻舍勒圖租魯,得碉一;攻開布智章,得寨一。又克薩爾歪,阿結占賊寨,據勒隈勒木通、科布曲山梁,斬獲甚眾。四十一年正月,合諸軍圍噶拉依,普爾普出其右,與海蘭察築壘逼賊巢,遂克之。金川平,封三等奮勇男,世襲。圖形紫光閣,列前五十功臣。

師還,上郊勞,賜御用鞍轡馬一。授正紅旗護軍統領,正白旗滿洲副都統,賜雙眼花翎。四十三年,扈蹕謁泰東陵。離營住宿,坐奪雙眼花翎。林爽文之亂,授領隊大臣,命從將軍福康安赴臺灣援嘉義,解圍,克大里杙。爽文逃小半天山頂,同海蘭察進攻,賊拒戰,山路險惡,普爾普率廣東兵及屯練降番攀木柵先登,賊潰,遂擒爽文。進軍瑯嶠,追剿賊目莊大田,賊來劫營,普爾普於大武壟隘口沖殺,敗之。諭於臺灣嘉義建生祠。事見福康安傳。大田就擒,臺灣平,再圖形紫光閣,晉封二等男,襲一次,以三等男世襲。五十五年,卒。

論曰:海蘭察勇而有智略。每戰,微服策馬觀敵,察其瑕,集兵攻之,輒勝。平生惟服阿桂知兵,福康安禮先焉,乃為盡力,師所向有功。奎林亦孝賢皇后諸侄,剛而不撓,勛名與群從並。和隆武、額森特、普爾普皆以克敵功最受封爵。乾隆中多將材,此尤其魁傑也。


\end{pinyinscope}