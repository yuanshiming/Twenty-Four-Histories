\article{列傳二百七十一}

\begin{pinyinscope}
文苑一

魏禧兄際瑞弟禮禮子世效世儼李騰蛟邱維屏曾燦林時益

梁份侯方域王猷定陳宏緒徐士溥歐陽斌元申涵光張蓋

殷岳吳嘉紀徐波錢謙益龔鼎孳吳偉業曹溶宋琬嚴沆

施閏章高詠鄧漢儀王士祿弟士祜田雯曹貞吉顏光敏

王蘋張篤慶徐夜陳恭尹屈大均梁佩蘭程可則方殿元

吳文煒王隼馮班宗元鼎劉體仁吳殳胡承諾賀貽孫唐甄

阿什垣劉淇金德純傅澤洪汪琬計東吳兆騫顧我錡

彭孫遹硃彞尊李良年謤吉璁尤侗秦松齡曹禾李泰來

陳維崧吳綺徐釚潘耒倪燦嚴繩孫徐嘉炎方象瑛

萬斯同錢名世劉獻廷邵遠平吳任臣周春陳鱣

喬萊汪楫汪懋麟陸葇兄子奎勛龐塏邊連寶

陸圻丁澎柴紹炳毛先舒孫治張丹吳百朋沈謙虞黃昊

孫枝蔚李念慈丁煒林侗林佶黃任鄭方坤

黃與堅王昊顧湄吳雯陶季梅清梅庚馮景邵長蘅

姜宸英嚴虞惇黃虞稷性德顧貞觀項鴻祚蔣春霖

文昭蘊端博爾都永忠書誠永諲裕瑞趙執信葉燮馮廷櫆

黃儀鄭元慶查慎行弟嗣瑮查升史申義周起渭張元臣潘淳

顧陳垿何焯陳景雲景雲子黃中戴名世

清代學術,超漢越宋。論者至欲特立「清學」之名,而文學並重,亦足於漢、唐、宋、明以外別樹一宗,嗚呼盛已!明末文衰甚矣!清運既興,文氣亦隨之而一振。謙益歸命,以詩文雄於時,足負起衰之責;而魏、侯、申、吳,山林遺逸,隱與推移,亦開風氣之先。康、乾盛治,文教大昌。聖主賢臣,莫不以提倡文化為己任。師儒崛起,尤盛一時。自王、硃以及方、惲,各擅其勝。文運盛衰,實通世運。此當舉其全體,若必執一人一地言之,轉失之隘,豈定論哉?道、咸多故,文體日變。龔、魏之徒,乘時立說。同治中興,文風又起。曾國籓立言有體,濟以德功,實集其大成。光、宣以後,支離龐雜,不足言文久矣。茲為文苑傳,但取詩文有各能自成家者,匯為一編,以著有清一代文學之盛。派別異同,皆置勿論。其已見大臣及儒林各傳者,則不復著焉。

魏禧,字冰叔,寧都人。父兆鳳,諸生。明亡,號哭不食,翦發為頭陀,隱居翠微峰。是冬,筮離之乾,遂名其堂為易堂。旋卒。

禧兒時嗜古,論史斬斬見識議。年十一,補縣學生。與兄際瑞、弟禮,及南昌彭士望、林時益,同邑李騰蛟、邱維屏、彭任、曾燦等九人為易堂學。皆躬耕自食,切劘讀書,「三魏」之名遍海內。禧束身砥行,才學尤高。門前有池,顏其居曰勺庭,學者稱勺庭先生。為人形幹修頎,目光射人。少善病,參術不去口。性仁厚,寬以接物,不記人過。與人以誠,雖見欺,怡如也。然多奇氣,論事每縱橫排奡,倒注不窮。事會盤錯,指畫灼有經緯。思患豫防,見幾於蚤,懸策而後驗者十嘗八九。流賊起,承平久,人不知兵,且謂寇遠猝難及。禧獨憂之,移家山中。山距城四十里,四面削起百餘丈。中徑坼,自山根至頂若斧劈然。緣坼鑿磴道梯而登,因置閘為守望。士友稍稍依之。後數年,寧都被寇,翠微峰獨完。喜讀史,尤好左氏傳及蘇洵文。其為文凌厲雄傑。遇忠孝節烈事,則益感激,摹畫淋漓。

年四十,乃出游。於蘇州交徐枋、金俊明,杭州交汪渢,乍浦交李天植,常熟交顧祖禹,常州交惲日初、楊瑀,方外交藥地、槁木,皆遺民也。當是時,南豐謝文洊講學程山,星子宋之盛講學髻山,弟子著錄者皆數十百人,與易堂相應和。易堂獨以古人實學為歸,而風氣之振,由禧為之領袖。僧無可嘗至山中,嘆曰:「易堂真氣,天下無兩矣!」無可,明檢討方以智也。友人亡,其孤不能自存,禧撫教安業之。凡戚友有難進之言,或處人骨肉間,禧批郤導窾,一言輒解其紛。或訝之,禧曰:「吾每遇難言事,必積誠累時,待其精神與相貫注,夫然後言。」康熙十八年,詔舉博學鴻儒,禧以疾辭。有司催就道,不得已,舁疾至南昌就醫。巡撫舁驗之,禧蒙被臥稱疾篤,乃放歸。後二年卒,年五十七。妻謝氏,絕食殉。著有文集二十二卷、日錄三卷、詩八卷、左傳經世十卷。

際瑞,原名祥,字善伯,禧兄。明亡後,禧、禮並謝諸生。際瑞嘆曰:「吾為長子,祖宗祠墓,父母尸饔,將誰責乎?」遂出就試。順治十七年歲貢生。寧都民亂,贛軍進討,索餉於山砦。際瑞身冒險阻,往來任其事,屢瀕於死。際瑞重信義,翠微峰諸隱者暨族戚倚際瑞為安危者三十餘年。康熙十六年,滇將韓大任踞贛,當事議撫之。大任曰:「非魏際瑞至,吾不信也!」時際瑞館總鎮哲爾肯所,遂遣之。家人泣勸毋往,際瑞曰:「此鄉邦宗族所關也,吾不行,恐禍及。行而無成,吾自當之。」遂往。甫入營。官兵遽從東路急攻。大任疑賣己,因拘留之。大任變計走降閩,際瑞遂遇害,年五十八。子世傑殉焉。際瑞篤治古文,喜漆園、太史公書。著有文集十卷、五雜俎五卷。

禮,字和公,禧弟。少魯鈍,受業於禧。禧嘗笞詈之,禮弗憾,曰:「兄固愛弟也!」禧喜過望。方九歲,父將析產,持一田券躊躇曰:「與祥,則禮損矣。奈何?」禮適在旁,應聲曰:「任損我,毋損伯兄。」父笑曰:「是固魯鈍者耶?」禮寡言,急然諾,喜任難事,以鬱鬱不得志,乃益事遠游。所至必交其賢豪,物色窮巖遺佚之士。年五十,倦游返,於翠微左幹之巔構屋五楹。是時伯叔踵逝,石閣、勺庭久虛無人。諸子各散處,不復居易堂。禮獨身率妻子居十七年,未他徙。卒,年六十六。著有詩文集十六卷。子世效、世儼。

世效,字昭士。生二十餘月,母口授九歌,輒能成誦。稍長,從仲父禧讀。性狷急,勇於任事。禧嘗謂其文一如其人,鋒銳所及,往往有沒羽之力。以多病不應試。遍游燕、楚、吳、越,一至嶺南。適王士禎使粵,見所作,原折節與交。著有耕廡文稿十卷。

世儼,字敬士。善病如其兄,然不廢翰墨。與世傑、世效時稱「小三魏」。著有為穀文稿八卷。

李騰蛟,字咸齋,亦寧都人。諸生。於易堂中年最長,諸子皆兄事之,嚴敬無敢斁。後居三巘峰,以經學教授。著周易賸言。年六十,卒。

邱維屏,字邦士,寧都人,三魏姊壻也。明諸生。為人高簡率穆。讀書多玄悟,禧嘗從之學。晚為歷數、易學及泰西算法。僧無可與布算,退語人曰:「此神人也!」彭士望與維屏交三十餘年,未嘗見其毀一人。然維屏獨推服禧,嘗貽禧書曰:「拒諫飾非者大惡也,不拒諫而嘗自拒諫,不飾非而嘗自飾非,尤惡之惡也。足下敢於自信,自處有故,而持之以堅,拒諫飾非,蓋有如此者!」禧得之痛服。維屏教授弟子,手批口講,日夜不輟業。康熙十八年,卒,年六十六。垂歿,示子曰:「食有菜飯,穿可補衣,無譎戾行,堪句讀師。」士望服其言。著有周易剿說十二卷、松下集十二卷、邦士文集十八卷。

曾燦,字青藜,亦寧都人,給事中應遴仲子。歲乙酉,楊廷麟竭力保南贛。應遴以閩嶠山澤間有眾十萬,命燦往撫之。既行,而應遴病卒,贛亦破,乃解散。尋祝發為僧,游閩、浙、兩廣間。大母及母念燦成疾,乃歸寧都。以大母命受室,築六松草堂,躬耕不出者數年。後僑居吳下二十餘年,客游燕市以卒。著有六松草堂文集、西崦草堂詩集。

林時益,本明宗室,名議敘,字確齋,南昌人。與彭士望同里。兩人謀居。士望與魏禧一見定交,極言金精諸山可為嶺北耕種處,乃攜家偕士望往。僑居十餘年,與魏氏昆弟相講習。康熙七年,詔明故宗室子孫眾多,竄伏山林者還田廬,復姓氏。時益久客寧都,弗樂歸。卜居冠石,結廬傭田,非其力不食。冠石宜茶,時益以意制之,香味擬陽羨,所謂林茶者也。晚好禪悅。著有冠石詩集五卷、確齋文集。

梁份,字質人,南豐人。少從彭士望、魏禧游,講經世之學。工古文辭。嘗只身游萬里,西盡武威、張掖,南極黔、滇,遍歷燕、趙、秦、晉、齊、魏之墟,覽山川形勢,訪古今成敗得失,遐荒軼事,一發之於文。方苞、王源皆重之。其論山海關,謂:「關自明洪武間始設,隋置臨榆於西,唐為榆關。東北古長城,燕、秦所築,距關遠,皆不足輕重。金之伐遼,自取遷民始。李自成席卷神京,敗石河而失之。天之廢興,人之成敗,而決於山海一隅。荒榛千百年之上,偏重於三百年間。天下定則山海安,山海困則天下舉困,其安危之重如此。」生平以未游山海為憾。為人樸摯強毅,守窮約至老不少挫。卒,年八十九。著有懷葛堂文集十五卷、西陲今略八卷。

侯方域,字朝宗,商丘人。父恂,明戶部尚書;季父恪,官祭酒:皆以東林忤閹黨。

方域師倪元璐。性豪邁不羈,為文有奇氣。時太倉張溥主盟復社,青浦陳子龍主盟幾社,咸推重方域,海內名士爭與之交。方恂之督師援汴也,方域進曰:「大人受詔討賊,廟堂議論多牽制。今宜破文法,取賜劍誅一甲科守令之不應徵辦者,而晉帥許定國師噪,亟斬以徇。如此則威立,軍事辦,然後渡河收中原土寨團結之眾,以合左良玉於襄陽,約陜督孫傳庭犄角並進,則汴圍不救自解。」恂叱其跋扈,不用,趣遣之歸。

方域既負才無所試,一放意聲伎,流連秦淮間。閹黨阮大鋮時亦屏居金陵,謀復用。諸名士共檄大鋮罪,作留都防亂揭,宜興陳貞慧、貴池吳應箕二人主之。大鋮知方域與二人善,私念因侯生以交於二人,事當已,乃囑其客來結驩。方域覺之,卒謝客,大鋮恨次骨。已而驟柄用,將盡殺黨人,捕貞慧下獄。方域夜走依鎮帥高傑,得免。順治八年,出應鄉試,中式副榜。十一年,卒,年三十七。

方域健於文,與魏禧、汪琬齊名,號「國初三家」。有壯悔堂集。

同時江西以文名者,南昌王猷定,新建陳宏緒、徐士溥、歐陽斌元。

猷定,字於一。選拔貢生。父時熙,進士,官太僕卿,名在東林。猷定好奇,有辯口,文亦如之。著四照堂集。

宏緒,字士業。父道亨,進士,官兵部尚書。疏救楊漣,罷歸。藏書萬卷。宏緒不仕,輯宋遺民錄以見志,有石莊集。

士溥,字巨源。父良彥,進士。忤崔、魏削籍,戍清浪。溧陽陳名夏聞士溥善古文,手書招之,拒不納。有榆溪集。

斌元,字憲萬。嘗為南司馬呂大器草奏劾馬士英二十四大罪,又佐史可法幕府。有文集十二卷。

申涵光,字孚孟,號鳧盟,永年人,明太僕寺丞佳胤子。年十五,補諸生。文名藉藉,顧不屑為舉子業。日與諸同志論文立社,載酒豪游為樂。萬歷六年亂起,議城守,出家貲四百金、錢二十萬犒士。甲申,奉母避亂西山,誅茅廣羊絕頂。與鉅鹿楊思聖,雞澤殷岳、殷淵,定患難交。京師破,佳胤殉國難,涵光痛絕復蘇。因渡江而南,謁陳子龍、夏允彞、徐石麟諸名宿,為父志、傳。歸里,事親課弟,足跡絕城市。日與殷岳及同里張蓋相往來酬和,人號為「廣平三君」。

清初,詔訪明死難諸臣。柏鄉魏裔介上褎忠疏,列佳胤名,格於部議。涵光徒跣赴京師,踔泥水中,幾瀕於死。麻衣絰帶,號哭東華道上,觀者皆飲泣。裔介再疏爭之,卒與祀恤如例。一時士大夫高其行,皆傾心納交,宴游贈答無虛日。

涵光為詩,吞吐眾流,納之爐治。一以少陵為宗,而出入於高、岑、王、孟諸家。嘗謂:「詩以道性情,性情之真者,可以格帝天,泣神鬼。若專事附會,寸寸而效之,則啼笑皆偽,不能動一人矣。」尚書王士禎稱涵光開河朔詩派。學士熊伯龍謂今世詩人吾甘為之下者,鳧盟一人而已。

嘗謁孫奇逢,執弟子禮。奇逢恨得之晚,以聖賢相敦勉。自是始聞天人性命之旨,究心理學,不復為詩。順治十七年,詔郡縣舉孝行,有司以涵光應,力辭之。再舉隱逸之士,堅辭不就。嘗自悔為名累,謝絕交游。晚年取諸儒語錄昕夕研究。作性習圖、義利說及荊園小語、進語諸書。嘗曰:「主靜不如主敬,敬,自靜也。硃、陸同適於道,硃由大路,雖遲而穩;陸由便徑,似捷而危:在人自擇耳。」奇逢謂其苦心積慮,閱歷深而動忍熟。裔介則贊之曰:「年少文壇,老來理路,聖賢之所謂博文而約禮也。」其推重如此。康熙十六年,卒,年五十九。

涵光又解琴理。書法顏魯公,尤工漢隸。間作山水木石,落落有雅致。著有聰山詩集八卷,文集四卷,說杜一卷。

蓋,字覆輿。明亡後,謝諸生,悲吟侘傺,遂成狂疾。嘗游齊、晉、楚、豫間,歸自閉土室中,雖妻子不得見。唯涵光、岳至則延入,談甚洽。其詩哀憤過情,恆自毀其稿。卒後,涵光為刊遺詩,曰柿葉集。

岳,字宗山,雞澤人。舉人。京師陷,入西山,與其弟淵謀舉義。事洩,淵被害,岳匿涵光家得免。其為詩自魏、晉以下屏不觀,尤不喜律詩,所作唯古體,莽莽然肖其為人。有留耕堂集。

吳嘉紀,字賓賢,泰州人。布衣。家安豐鹽場之東淘。地濱海,無交游。自名所居曰陋軒。貧甚,雖豐歲常乏食。獨喜吟詩,晨夕嘯詠自適,不交當世。郡人汪楫、孫枝蔚與友善,時稱道之,遂為王士禎所知。尤賞其五言清冷古淡,雪夜酌酒,為之序,馳使三百里致之。嘉紀因買舟至揚州謁謝定交,由是四方知名士爭與之倡和。

嘉紀工為危苦嚴冷之詞,嘗撰今樂府,淒急幽奧,能變通陳跡,自為一家。所著陋軒集多散佚,友人復裒集之為四卷。其詩風骨頗遒,運思亦復劖刻。由所遭不偶,每多怨咽之音,而篤行潛修,特為一時推重雲。

徐波,字元嘆,吳縣人。少任俠。明亡後,居天池,構落木菴,以枯禪終。詩多感喟,虞山錢謙益與之善,贈以詩,頗推重之。有謚簫堂、染香菴等集。

錢謙益,字受之,常熟人。明萬歷中進士,授編修。博學工詞章,名隸東林黨。天啟中,御史陳以瑞劾罷之。崇禎元年,起官,不數月至禮部侍郎。會推閣臣,謙益慮尚書溫體仁、侍郎周延儒並推,則名出己上,謀沮之。體仁追論謙益典試浙江取錢千秋關節事,予杖論贖。體仁復賄常熟人張漢儒訐謙益貪肆不法。謙益求救於司禮太監曹化淳,刑斃漢儒。體仁引疾去,謙益亦削籍歸。

流賊陷京師,明臣議立君江寧。謙益陰推戴潞王,與馬士英議不合。已而福王立,懼得罪,上書誦士英功,士英引為禮部尚書。復力薦閹黨阮大鋮等,大鋮遂為兵部侍郎。順治三年,豫親王多鐸定江南,謙益迎降,命以禮部侍郎管秘書院事。馮銓充明史館正總裁,而謙益副之。俄乞歸。五年,鳳陽巡撫陳之龍獲黃毓祺,謙益坐與交通,詔總督馬國柱逮訊。謙益訴辨,國柱遂以謙益、毓祺素非相識定讞。得放還,以箸述自娛,越十年卒。

謙益為文博贍,諳悉朝典,詩尤擅其勝。明季王、李號稱復古,文體日下,謙益起而力振之。家富藏書,晚歲絳雲樓火,惟一佛像不燼,遂歸心釋教,著楞嚴經蒙鈔。其自為詩文,曰牧齋集,曰初學集、有學集。乾隆三十四年,詔毀板,然傳本至今不絕。

龔鼎孳,字孝升,合肥人。明崇禎七年進士,授兵科給事中。李自成陷都城,以鼎孳為直指使,巡視北城。及睿親王至,遂迎降,授吏科給事中。改禮科,遷太常寺少卿。順治三年,丁父憂,請賜恤典。給事中孫自齡疏言:「鼎孳辱身流賊,蒙朝廷擢用,曾不聞夙夜在公,惟飲酒醉歌,俳優角逐。聞訃仍復歌飲留連,冀邀非分之典,虧行滅倫,莫此為甚!」部議降二級。尋遇恩詔獲免,累遷左都御史。

先是大學士馮銓被劾,睿親王集科道質訊。鼎孳斥銓閹黨,為忠賢義兒。銓曰:「何如逆賊御史?」鼎孳以魏徵歸順太宗自解,王笑曰:「惟無瑕者可以戮人。奈何以闖賊擬太宗!」遂罷不問。坐事降八級調用,補上林苑丞,旋罷。康熙初,起左都御史,遷刑部尚書。卒,謚「端毅」。乾隆三十四年,詔削其謚。

鼎孳天才宏肆,千言立就。世祖在禁中見其文,嘆曰:「真才子也!」嘗兩典會試,汲引英雋如不及。硃彞尊、陳維崧游京師,貧甚,資給之。傅山、閻爾梅陷獄,皆賴其力得免。臨歿,以徐釚囑梁清標曰:「負才如虹亭,可使之不成名耶?」釚後以清標薦試鴻博,入史館。自謙益卒後,在朝有文藻負士林之望者,推鼎孳云。著有定山堂集。

吳偉業,字駿公,太倉人。明崇禎四年進士,授編修。充東宮講讀官,再遷左庶子。弘光時,授少詹事,乞假歸。順治九年,用兩江總督馬國柱薦,詔至京。侍郎孫承澤、大學士馮銓相繼論薦,授秘書院侍講,充修太祖、太宗聖訓纂修官。十三年,遷祭酒。丁母憂歸。康熙十年,卒。

偉業學問博贍,或從質經史疑義及朝章國故,無不洞悉原委。詩文工麗,蔚為一時之冠,不自標榜。性至孝,生際鼎革,有親在,不能不依違顧戀,俯仰身世,每自傷也。臨歿,顧言:「吾一生遭際,萬事憂危,無一時一境不歷艱苦。死後斂以僧裝,葬我鄧尉、靈巖之側。墳前立一圓石,題曰『詩人吳梅村之墓』。勿起祠堂,勿乞銘。」聞其言者皆悲之。著有春秋地理志、氏族志,綏寇紀略及梅村集。

曹溶,字鑒躬,嘉興人。明崇禎十年進士,官御史。清定京師,仍原職。尋授順天學政。疏薦明進士王崇簡等五人,又請旌殉節明大學士苑景文、尚書倪元璐等二十八人,孝子徐基、義士王良翰等及節婦十餘人。試竣,擢太僕寺少卿。坐前學政任內失察,降二級。久之,稍遷左通政,上言:「通政之官職在納言,請嗣後凡遇挾私違例章疏即予駮還,仍許隨事建議。」又言:「王師入關,各處駐兵,乃一時權宜。今當歸並於盜賊出沒險阻之地,則兵不患少。其閒散無事之兵,遇缺勿補,遇調即遣,則餉不虛糜。且當裁提鎮,增副將,以專責成。」又言:「諸司職掌無成書,請以近年奉旨通行者,參之前朝會典,編為簡明則例,以重官守。」擢左副都御史。疏請時御便殿,召大臣入對,賜筆札以辨其才識,有切中利弊者,即飭力行,勿概下部議,帝並嘉納。擢戶部侍郎,出為廣東布政使,降山西陽和道。康熙初,裁缺歸里。十八年,舉鴻博,丁憂未赴,學士徐元文薦修明史。又數年,卒,有倦圃詩集。

宋琬,字玉叔,萊陽人。父應亨,明天啟中進士。令清豐,有惠政,民為立祠。崇禎末殉節,贈太僕寺卿。

琬少能詩,有才名。順治四年進士,授戶部主事,累遷吏部郎中。出為隴西道,過清豐,民遮至應亨祠,款留竟日,述往事至泣下。琬益自刻厲,期不墜先緒。調永平道,又調寧紹臺道,皆有績。十八年,擢按察使。時登州於七為亂。琬同族子懷宿憾,因告變,誣琬與於七通,立逮下獄,並系妻子。逾三載,下督撫外訊。巡撫蔣國柱白其誣,康熙三年放歸。十一年,有詔起用,授四川按察使。明年,入覲,家屬留官所。值吳三桂叛,成都陷,聞變驚悸卒。

始琬官京師,與嚴沆、施閏章、丁澎輩酬倡,有「燕臺七子」之目。其詩格合聲諧,明靚溫潤。既構難,時作淒清激宕之調,而亦不戾於和。王士禎點定其集為三十卷。嘗舉閏章相況,目為「南施北宋」。歿後詩散佚,族孫邦憲綴輯之為六卷。

沆,字子餐,餘杭人。順治十二年進士,官至戶部侍郎。性退讓,或譏彈其詩,輒應時改定。有皋園集。

施閏章,字尚白,號愚山,宣城人。祖鴻猷,以儒學著。子姓傳業江南,言家法者推施氏。

閏章少孤,事叔父如父。從沈壽民游,博綜群籍,善詩古文辭。順治六年進士,授刑部主事,以員外郎試高等。擢山東學政,崇雅黜浮,有冰鑒之譽。秩滿,遷江西參議,分守湖西道。屬郡殘破多盜,遍歷山谷撫循之,人呼為施佛子。嘗作彈子嶺、大阬嘆等篇告長吏,讀者皆曰:「今之元道州也。」尤崇獎風教,所至輒葺書院,會講常數百人。新淦民兄弟忿戾不睦,一日聞講禮讓孝弟之言,遂相持哭,詣堦下服罪。峽江患虎,制文祝之,俄有虎墮深塹,患遂絕。歲旱,禱雨輒應。康熙初,裁缺歸。民留之不,得,乃醵金創龍岡書院祀之。初,閏章駐臨江,有清江環城下,民過者咸曰:「是江似使君。」因改名使君江。及是傾城送江上,又送至湖。以官舫輕,民爭買石膏載之,乃得渡。十八年,召試鴻博,授翰林院侍講,纂修明史,典試河南。二十二年,轉侍讀,尋病卒。

閏章之學,以體仁為本。置義田,贍族好,扶掖後進。為文意樸而氣靜,詩與宋琬齊名。王士禎愛其五言詩,為作摘句圖。士禎門人問詩法於閏章,閏章曰:「阮亭如華嚴樓閣,彈指即見。予則不然,如作室者,瓴甓木石,一一就平地築起。」論者皆謂其允。著有學餘堂集、矩齋雜記、蠖齋詩話,都八十餘卷。

閏章與同邑高詠友善,皆工詩,主東南壇坫數十年,時號「宣城體」。

詠,字阮懷。幼稱神童。祖維嶽,知興國州,清介無長物。詠食貧勵學,屢躓名場,年近六十,始貢入太學。詞科之舉,詠與焉,授檢討。閏章稱其討優入古人。兼工書畫,有遺山堂、若巖堂集。

時同舉鴻博又有泰州鄧漢儀,字孝威。以年老授中書舍人。亦工詩。游跡所至,輒以名集,逐年編紀,凡七集。詩家咸推重之。

王士祿,字子底,濟南新城人。少工文章,清介有守。弟士祜、士禎從之學詩。士禎遂為詩家大宗,官尚書,自有傳。士祿,順治九年進士。投牒改官,選萊州府教授,遷國子監助教,擢吏部主事。康熙二年,以員外郎典試河南,磨勘罣吏議下獄。久之得雪,免歸。居數年,起原官。學士張貞生、御史李棠先後建言獲咎,力直之,人以為難。尋又免歸。母喪,以毀卒,年四十有八。其文去雕飾,詩尤閒澹幽肆。有西樵、十笏山房諸集。

士祜,字子測。十歲時,客或疑焦竑字弱侯何耶?坐客未對,即應聲曰:「此出考工記,『竑其幅廣以為之弱』也。咸驚其夙慧。康熙初,第進士,未仕卒。士禎輯其詩為古缽山人遺集。

當是時,山左詩人王氏兄弟外,有田雯、顏光敏、曹貞吉、王蘋、張篤慶、徐夜皆知名。

雯,字紫綸,號山姜,德州人。康熙三年進士,授中書。先是中書以貲郎充,是年始改用進士,遂為例。累遷工部郎中。督江南學政,所取士多異才。每按試,從兩騾,二僕隨之,戒有司勿供張。授湖廣督糧道,遷光祿寺卿,巡撫江寧,調貴州。時苗、仲猖獗,粵督議會剿,雯謂:「制苗之法,犯則治之,否則防之而已,無庸動眾勞民也。」議遂寢。丁憂,起補刑部侍郎,調戶部,以疾歸。康熙中,士禎負海內重名,其論詩主風調。雯負其縱橫排奡之氣,欲以奇麗抗之。有古懽堂集。

貞吉,字升六,安丘人。與雯同年進士,禮部郎中。詩格遒練,有實庵詩略。兼工倚聲,吳綺選名家詞,推為壓卷。

光敏,字遜甫,曲阜人,顏子六十七世孫也。康熙六年進士,除國史院中書舍人。帝幸太學,加恩四氏子孫,授禮部主事,歷吏部郎中。其為詩秀逸深厚,出入錢、劉。吳江計東謂足以鼓吹休明。雅善鼓琴,精騎射蹋鞠。嘗西登太華,循伊闕,南浮江、淮,觀濤錢塘,溯三衢。所至輒命工為圖,得金石文恆懸之屋壁。有樂圃集、舊雨堂集。

蘋,字秋史,歷城人。少落拓不偶,人目為狂。雯見其詩,為延譽。嘗賦「黃葉」句絕工,人稱為王黃葉。康熙四十五年進士,當為令,以母老改成山衛教授。閉門耽吟,介節彌著。有二十四泉草堂集。

篤慶,字歷久,淄川人。拔貢生。早受知施閏章。會徵鴻博,有欲薦之者,辭不應。詩以盛唐為宗,有昆侖山房集。

夜,字東癡,新城人,本名元善。舉鴻博,不赴。有詩集。

陳恭尹,字元孝,順德人。父邦彥,明末殉國難,贈尚書。恭尹少孤,能為詩,習聞忠孝大節。棄家出游,賦姑蘇懷古諸篇,傾動一時。留閩、浙者七年。一日,父友遇諸塗,責之曰:「子不歸葬,奈何徒欲一死塞責耶!」恭尹泣謝之,乃歸。既葬父增城,遂渡銅鼓洋訪故人於海外。久之歸,主何衡家。與陶窳、梁無技及衡弟絳相砥礪,世稱「北田五子」。已,復游贛州,轉泛洞庭,再游金陵,至汴梁,北渡黃河,徘徊大行之下。於是南歸,築室羊城之南以詩文自娛,自稱羅浮布衣。

恭尹修髯偉貌,氣幹沉深。其為詩激昂頓挫,足以發其哀怨之思。自言平生文辭多取諸胸臆,僕僕道塗,稽古未遑也。卒,年七十一。著獨漉堂集。王隼取恭尹詩合屈大均、梁佩蘭共刻之,為嶺南三家集。

大均,字介子,番禺人。初名紹隆,遇變為僧,中年返初服。工詩,高渾兀奡,有翁山詩文集。

佩蘭,字芝五,南海人。童時日記數千言。順治十四年鄉試第一,又三十一年始成進士,年六十矣。佩蘭夙負詩名。既選庶吉士,館中推為祭酒。不一年假歸,里居十五載。會詔飭詞臣就職,復入都。逾月散館,以不習國書罷歸。結蘭湖社,與同邑程可則,番禺王邦畿、方殿元及恭尹等稱「嶺南七子」。有六瑩堂集。

可則,字周量。順治九年會試第一。以磨勘停殿試歸,益恣探經史。十七年,始應閣試,授內閣中書,累遷兵部郎中。出知桂林府,以敏幹稱。其官都下,與宋琬、施閏章、王士祿、士禎、陳廷敬、沈荃、曹爾堪輩為文酒之會,吳之振合刻八家詩選。可則詩曰海日堂集。

殿元,字蒙章。康熙三年進士。歷知剡城、江寧等縣。置祭田以贍兄弟,而自攜長子還、次子朝僑寓蘇州。父子皆有詩名。所稱「嶺南七子」,並其二子數之也。殿元著九穀集;還,靈州集;朝,勺園集。

佩蘭之友又有南海吳文煒,字山帶。十歲工詩,兼善繪事。時初效長吉體,務為險語取快。康熙三十二年舉人。計偕,卒於旅舍。有金茅山堂集,恭尹為之序。

王隼,字蒲衣,番禺人。父邦畿,明副貢生。隱居羅浮,嶺南七子之一也。有耳鳴集。隼七歲能詩。慕道術,早歲棄家入丹霞,尋入匡廬,居太乙峰,六七年始歸。性喜琵琶,終日理書卷,生事窘不顧,惟取琵琶彈之。琵琶聲急,即其窘益甚。著大樗堂集。妻潘,女瑤湘,並工詩。

馮班,字定遠,常熟人。淹雅善持論,顧性不諧俗。說詩力牴嚴羽,尤不取江西宗派,出入義山、牧之、飛卿之間。書四體皆精。著鈍吟集。趙執信於近代文家少許可者,見班所著獨折服,至具衣冠拜之。嘗謁其墓,寫「私淑門人」刺焚塚前。其為名流所傾仰類此。

宗元鼎,字定九,江都人。七歲詠梅,遠近傳誦其句。堂有古梅一株,人謂之「宗郎梅」。性狷而孝,釜甑屢空,未嘗以貧告人。康熙初,貢太學,銓注州同知。未仕卒。元鼎與從弟元豫、觀,從子之瑾、之瑜皆工詩,有「廣陵五宗」之目。

劉體仁,字公霝,潁州人。順治中進士。有家難,棄官從孫奇逢講學。後官考功郎中。體仁喜作畫,鑒識其精,又工鼓琴。與汪琬、王士禎友善,著七頌堂集。士禎稱其詩似孟東野;又言今日善學才調集者無如元鼎,學西昆體者無如吳殳。

殳,字修齡,原名喬,亦常熟人也。著圍爐詩話,云:「意喻則米,炊而為飯者文,釀而為酒者詩乎?」又曰:「詩之中須有人在。」執信嘆為知言。

胡承諾,字君信,天門人。崇禎時舉人。明亡後,隱居不仕,臥天門巾、柘間。順治十二年,部銓縣職。康熙五年,檄徵入都,六年,至京師,未幾告歸。構石莊於西村,窮年誦讀,著繹志二十餘萬言。繹志者,繹己所志也。原本道德,切近人事,為有體有用之學。其吏治篇曰:「古之人不敢輕言變法也。必有明哲之德,於精粗之理無所不昭,不獨精者為之地,即粗者亦為之地,有和悅之氣,於異同之見無所不容,不獨同者樂其然,即異者亦樂其然;然後可奪其久安之法,授以更新之制,而民不驚顧不讙譁也。」租庸篇曰:「欲富國者,當使君民之力皆常有餘。民之餘力,生於君之約取;君之餘力,生於民之各足。」他篇準此。承諾自擬其書於徐幹中論、顏氏家訓。或頗譏其掇拾群言,未能如古人自成一家之說,然大體必軌於正。又有讀書錄,則鱗雜細碎,殆繹志取材之餘矣。二十六年,卒,年七十五。

同時篤志撰述,其學與承諾相上下者,又有賀貽孫,字子翼,永新人;唐甄,字鑄萬,達州人。

貽孫九歲能屬文。明季社事盛行,貽孫與萬茂先、陳士業、徐巨源、曾堯臣輩結社豫章。及明亡,遂不出。順治初,學使者慕其名,特列貢榜,避不就。巡按御史笪重光欲舉應鴻博,書至,貽孫愀然曰:「吾逃世而不逃名,名之累人實甚。吾將從此逝矣!」乃翦發衣緇,結茅深山,無復能蹤跡之者。晚年窮益甚。著有易觸、詩觸、詩筏、騷筏,又著水田居激書。激書者,備名物以寄興,紀逸事以垂勸,援古鑒今,錯綜比類。言之不足,故長言之,長言之不足,故危悚惕厲,必暢所欲言而後已,激濁揚清。始自貴因,終於空明,凡四十一篇。

甄性至孝,父喪,獨棲殯室三年。以世亂不克還葬,遂葬父虎丘。順治十四年舉人。選長子令,下車,即導民樹桑凡八十萬本,民利賴焉。未幾,坐逃人詿誤去官。僦居吳市,炊煙屢絕,至採枸杞葉為食,衣敗絮,著述不輟。始志在權衡天下,作衡書,後以連蹇不遇,更名潛書。分上下篇,上篇論學,始辨儒,終博觀,凡五十篇;下篇論政,始尚治,終潛存,凡四十七篇。上觀天道,下察人事,遠正古跡,近度今宜,根於心而致之行,非虛言也。寧都魏禧見而嘆之曰:「是周、秦之書也,今猶有此人乎!」卒,年七十五。

阿什坦,字金龍,完顏氏,滿洲正白旗人。順治九年進士,授刑科給事中。初繙譯大學、中庸、孝經諸書,詔刊行。阿什坦上言;「學者宜以聖賢為期,經史為導,此外無益雜書當屏絕。」又請嚴旗人男女之別,定部院九品之制,俱報可。康熙初,罷職家居。鰲拜專政,欲令一見終不往。嗣以薦起,聖祖召問節用愛人,對曰:「節用莫要於寡欲,愛人莫先於用賢。」聖祖顧左右曰:「此我朝大儒也!」著有大學中庸講義及奏稿。孫留保,以掌院學士充明史總裁,附王蘭生傳。

劉淇,字武仲,漢軍鑲白旗人。弟汶,舉人。受知世宗,時有二難之目。著周易通說、禹貢說、助字辨略、堂邑志、衛園集。

金德純,字素公,漢軍正紅旗人。著旗軍志。

傅澤洪,字育甫,漢軍旗人。累官江南淮揚道。著行水金鑒百七十五卷。

汪琬,字苕文,長洲人。少孤,自奮於學,銳意為古文辭。於易、詩、書、春秋、三禮、喪服咸有發明。性狷介。深嘆古今文家好名寡實,鮮自重特立,故務為經世有用之學。其於當世人物,褒譏不少寬假。順治十二年進士,授主事,再遷刑部郎中。坐累降兵馬司指揮,能舉其職,不以秩卑自沮。任滿,稍遷戶部主事,民送之溢衢卷。榷江寧西新關,以疾假歸。結廬堯峰山,閉戶撰述,不交世事,學者稱堯峰先生。以宋德宜,陳廷敬薦博學鴻儒科,試列一等。授編修,纂修明史,棘棘爭議不阿。在館六十日,再乞病歸。歸十年而卒,年六十七。

初,聖祖嘗問廷敬今世誰能為古文者,廷敬舉琬以對。及琬病歸,聖祖南巡駐無錫,諭巡撫湯斌曰:「汪琬久在翰林,有文譽。今聞其居鄉甚清正,特賜御書一軸。」當時榮之。琬為文原本六經,疏暢類南宋諸家,敘事有法。公卿志狀,皆爭得琬文為重。嘗自輯詩文為類稿、續稿各數十卷,又簡其尤精者,囑門人林佶繕刻之。

計東,字甫草,吳江人。少負經世才,自比馬周、王猛。遭世變,著籌南五論,持謁史可法,可法奇之,弗能用也。順治十四年,舉順天鄉試,旋以江南奏銷案被黜。嘗從湯斌講學,又從汪琬受歐、曾古文義法,故其為文具有本原,而一出以和平溫雅。既廢不用,貧無以養,縱游四方,所至交其豪傑。過鄴城,尋明詩人謝榛葬處,得之南門外二十里,為修墓立石,請有司禁樵牧。又憩順德逆旅,念歸有光昔嘗佐郡,集中有壁記,求其遺址不得,乃即署旁廢圃中設瓣香,再拜流涕而去,觀者駭其狂。

東外若不羈,內行謹,事母至孝。同邑友人吳兆騫流徙出關,為恤其家,且以女許配其弱子。大學士王熙素重東,屢欲薦之,未果。會詔舉鴻博,而東已前一年卒,深悼惜焉。

初游河南,見商丘宋犖,輒引重。其後東歿二十餘年,犖至江蘇巡撫,為序其遺文,曰改亭集,刊行之。

兆騫,字漢槎。亦十四年舉人。以科場蜚語逮系,遣戍寧古塔。兆騫與弟兆宜皆善屬文,居塞上二十年,侘傺不自聊,一發之於詩。已而友人顧貞觀言於納蘭成德、徐乾學,為納鍰,遂於康熙二十年赦還。著秋笳集。兆宜嘗注徐、庾二集,韓偓詩集,又注玉臺新詠、才調集,並行於世。

同邑顧我錡,廩生。鄂爾泰任江蘇布政,試古學,得士五十三人,刻南邦黎獻集,推我錡為冠。乾隆丙辰開詞科,鄂爾泰惜我錡前卒,不獲舉,人謂其遇與東同。有湘南詩集。

彭孫遹,字駿孫,海鹽人。父期生,明唐王時官太僕卿,死贛州。長子孫貽以毀卒,孫遹其少子也。順治十六年進士,授中書。素工詞章,與王士禎齊名,號曰「彭王」。康熙十八年,開博學鴻儒科,詔中外諸臣廣搜幽隱,備禮敦勸,無論已仕未仕,徵詣闕下,月餼太倉米。明年三月朔,召試太和殿。發賦、詩題各一,學士院給官紙,光祿布席,賜宴體仁閣下。於是天子親擢孫遹一等一名,授編修。

自孫遹外,其籍隸浙江者,又有錢塘汪霦,秀水徐嘉炎、硃彞尊,平湖陸葇,海寧沈珩,仁和沈筠、吳任臣、邵遠平,遂安方象瑛、毛升芳,蕭山毛奇齡,鄞陳鴻績,凡十三人。江蘇二十三人,曰:上元倪燦,寶應喬萊,華亭王頊齡、吳元龍,無錫秦松齡、嚴繩孫,武進周清原,宜興陳維崧,長洲馮勛、汪琬、尤侗、範必英,吳錢中諧,儀真汪楫,淮安邱象隨,吳江潘耒、徐釚,太倉黃與堅,常熟周慶會,山陽李鎧、張鴻烈,上海錢金甫,江陰曹禾。直隸五人,曰:大興張烈,東明袁佑,宛平米漢雯,獲鹿崔如嶽,任丘龐塏。安徽三人,曰:宣城施閏章、高詠,望江龍燮。江西二人,曰:臨川李來泰,清江黎騫。陜西一人,曰富平李因篤。河南一人,曰睢州湯斌。山東一人,曰諸城李澄中。湖北一人,曰黃岡曹宜圃。凡五十人,皆以翰林入史館。其列二等者,亦多知名之士,稱極盛焉。

孫遹歷官吏部侍郎,充經筵講官。明史久未成,特命為總裁,賜專敕,異數也。年七十,致仕歸,御書「松桂堂」額賜之,遂以名其集。

硃彞尊,字錫鬯,秀水人,明大學士國祚曾孫。生有異秉,書經目不遺。家貧客游,南逾嶺,北出雲朔,東泛滄海,登之罘,經甌越。所至叢祠荒塚、破爐殘碣之文,莫不搜剔考證,與史傳參校同異。歸里,約李良年、周筼、繆泳輩為詩課,文名益噪。

康熙十八年,試鴻博,除檢討。時富平李因篤、吳江潘耒、無錫嚴繩孫及彞尊皆以布衣入選,同修明史。建議訪遺書,寬期限,毋效元史之迫時日。辨方孝孺之友宋仲珩、王孟縕、鄭叔度、林公輔諸人咸不及於難,則知從亡、致身錄謂誅九族,並戮其弟子朋友為一族不足據,所謂九族者,本宗一族也。又言東林不皆君子,異乎東林者,亦不皆小人。作史者未可存門戶之見,以同異分邪正。二十年,充日講起居注官。典試江南,稱得士。入值南書房,賜紫禁城騎馬。數與內廷宴,被文綺、時果之賚,皆紀以詩。旋坐私挾小胥入內寫書被劾,降一級,後復原官。三十一年,假歸。聖祖南巡,迎駕無錫,御書「研經博物」額賜之。

當時王士禎工詩,汪琬工文,毛奇齡工考據,獨彞尊兼有眾長。著經義考、日下舊聞、曝書亭集。又嘗選明詩綜,或因人錄詩,或因詩存人,銓次為最當。卒,年八十一。子昆田,亦工詩文,早卒。孫稻孫,舉乾隆丙辰鴻博,能世其家。

彞尊所與為詩課者,李良年,字武曹,同邑人。與兄繩遠、弟符並著詩名。試鴻博,罷歸。有秋錦山房集。譚吉璁,字舟石,嘉興人,彞尊姑之子也。少遇寇,以身蔽父,寇舍之去。後以諸生試國子監第一,授弘文院撰文中書舍人,出為延安同知。吳三桂叛,守榆城獨完,論功加一級。舉應鴻博,報罷。遷知登州府。卒。有嘉樹堂集。

尤侗,字展成,長洲人。少補諸生,以貢謁選。除永平推官,守法不撓。坐撻旗丁鐫級歸。侗天才富贍,詩文多新警之思,雜以諧謔,每一篇出,傳誦遍人口。康熙十八年,試鴻博列二等,授檢討,與修明史。居三年告歸。聖祖南巡至蘇州,侗獻詩頌。上嘉焉,賜御書「鶴棲堂」額,遷侍講。

初,世祖於禁中覽侗詩篇,以才子目之。後入翰林,聖祖稱之曰「老名士」。天下羨其榮遇。侗喜汲引才雋,性寬和,與物無忤。兄弟七人甚友愛,白首如垂髫。卒,年八十七。著西堂集、鶴棲堂集,凡百餘卷。

秦松齡,字留仙,無錫人。順治十二年進士,官檢討,罷歸。後舉鴻博,復授檢討。典江西鄉試,歷左贊善,以諭德終。松齡為庶常,召試詠鶴詩,有句云:「高鳴常向月,善舞不迎人。」世祖拔置第一,示閣臣曰:「是人必有品!」及告歸,里居二十餘年,專治毛詩。仿黃氏日鈔之例,著毛詩日箋六卷。自為詩文曰蒼峴山人集。

曹禾,字頌嘉,江陰人。康熙三年進士。選鴻博,授檢討,官至祭酒。與田雯、宋犖、汪懋麟、顏光敏、王又旦、謝重輝、曹貞吉、丁澎、葉封齊名,稱詩中十子。

同時江西選鴻博一等者,李泰來,字石臺,臨川人。順治九年進士。嘗督江南學政,除蘇松常道,以疾歸。試詞科,授侍講。古文博奧,詩以和雅稱。有石臺集。

陳維崧,字其年,宜興人。祖於廷,明左都御史。父貞慧,見遺逸傳。維崧天才絕艷,十歲,代大父撰楊忠烈像贊。比長,侍父側,每名流宴集,援筆作序記,千言立就,瑰瑋無比,皆折行輩與交。補諸生,久之不遇。因出游,所在爭客之。嘗由汴入都,與硃彞尊合刻一稿,名硃陳村詞,流傳至禁中,蒙賜問,時以為榮。逾五十,始舉鴻博,授檢討,修明史。在館四年,病卒。

維崧清臞多須,海內稱陳髯。平生無疾言遽色,友愛諸弟甚。游公卿間,慎密,隨事匡正,故人樂近之,而卒莫之狎。著湖海樓詩集、迦陵文集。時汪琬於同輩少許可者,獨推維崧駢體,謂自唐開、寶後無與抗矣。詩雄麗沉鬱,詞至千八百首之多,尤前此未有也。

順、康間,以駢文稱者,又有吳綺,字次,江都人。維崧導源庾信,泛濫於初唐四傑,故氣脈雄厚。綺則追步李商隱,才地視維崧為弱,而秀逸特甚。順治十一年拔貢生,薦授中書舍人。奉詔譜楊繼盛樂府,遷兵部主事,即以繼盛官官之也。出知湖州府,有吏能。人謂其多風力,尚風節,饒風趣,稱為「三風太守」。未幾,罷歸。貧無田屯,購廢圃以居。有匄詩文者,以花木潤筆,因顏其圃曰種字林。著林蕙堂集。詞最有名,婦孺皆能習之。以有「把酒祝東風,種出雙紅豆」之句,又稱「紅豆詞人」。

徐釚,字電發,吳江人。應鴻博,授檢討。會當外轉,遽乞歸。後起原官,不就。卒,年七十三。著南州草堂集、本事詩。又嘗刻菊莊樂府。昆山葉方靄稱其綿麗幽深,耐人尋繹。朝鮮貢使以兼金購之。釚既工倚聲,因輯詞苑叢談,具有裁鑒。

潘耒,字次耕,吳江人。生而奇慧,讀書十行並下,自經史、音韻、算數及宗乘之學,無不通貫。康熙時,以布衣試鴻博,授檢討,纂修明史。上書總裁,言要義八端:「宜搜採博而考證精;職任分而義例一;秉筆直而持論平;歲月寬而卷帙簡。」總裁善其說,令撰食貨志,兼他紀傳。尋充日講起居注官,修實錄、聖訓。嘗應詔陳言,謂:「建言古無專責,梅福以南昌尉言外戚,柳伉以太常博士言程元振,陳東以太學生攻六賊,楊繼盛以部曹劾嚴嵩。本朝舊制,京官並許條陳。自康熙十年憲臣奏請停止,凡非言官而言事為越職。夫人主明目達聰,宜導之使言。今乃禁之,豈盛世事?臣請弛其禁,俾大小臣工各得獻替,庶罔上行私之徒,有所忌而不敢肆。於此輩甚不便,於國家甚便也。其在外監司守令,遇地方大利弊,許其條奏。水旱災荒,州縣官得上聞。如此,則民間疾苦無不周知矣。」更請許臺諫官得風聞言事,有能奮擊奸回者,不次超擢,以作敢言之氣。二十三年,甄別議起,坐浮躁降調,遂歸。

耒有至性,初被徵,辭以母老,不獲命,乃行。既徐官,三牒吏部以獨子請終養,卒格於議不果歸。逮居喪,哀毀骨立。少受學同郡徐枋、顧炎武。枋歿,周恤其孤孫,而刻炎武所著書,師門之誼甚篤焉。四十二年,聖祖南巡,復原官。大學士陳廷敬欲薦起之,力辭而止。平生嗜山水,登高賦詠,名流折服。有遂初堂集。又因炎武音學五書為類音八卷。炎武復古,耒則務窮後世之變雲。

當時詞科以史才稱者,硃彞尊、汪琬、吳任臣及耒為最著。又有倪燦,字闇公,上元人。以舉人授檢討,撰藝文志序,與姜宸英刑法志序並推傑構。書法詩格秀出一時,有雁園集。

嚴繩孫,字蓀友,無錫人,明尚書一鵬孫。六歲能作擘窠大書。試日,目疾作,第賦一詩,亦授檢討,撰明史隱逸傳。典試江西,尋遷中允,假歸。有秋水集。子泓曾,亦善畫工詩。

徐嘉炎,字勝力,秀水人,明兵部尚書必達曾孫。幼警敏,強記絕人。既,試鴻博,授檢討。康熙二十年,王師收滇、黔,嘉炎仿鐃歌鼓吹曲,撰聖人出至文德舞二十四章以獻;又四年元夕,聖祖於南海大放燈火,縱臣民使觀,嘉炎復應制撰記:皆稱旨。嘗侍直,命背誦咸有一德,終篇不失一字。至「厥德靡常」數語,則斂容讀之,帝為悚異。又嘗問宋元祐黨人是非,嘉炎舉諸人姓名始末,及先儒評騭語其悉。特賜御臨蘇軾詩一卷,廷臣拜賜御書自此始也。累擢內閣學士,兼禮部侍郎,充三朝國史及會典、一統志副總裁。有抱經齋集。

方象瑛,字渭仁,遂安人。康熙六年進士。試鴻博,授編修,典試蜀中。尋告歸。象瑛性簡靜,早慧,十歲作遠山凈賦,驚其長老。致仕家居,望益重。邑有大利弊,則岳岳爭言,歲省脂膏萬計,邑人建思賢祠祀之。著健松齋集、封長白山記、松窗筆乘。

萬斯同,字季野,鄞縣人。父泰,生八子,斯同其季也。兄斯大,儒林有傳。性彊記,八歲,客坐中能背誦揚子法言。後從黃宗羲游,得聞蕺山劉氏學說,以慎獨為宗。以讀書勵名節與同志相劘切,月有會講。博通諸史,尤熟明代掌故。康熙十七年,薦鴻博,辭不就。

初,順治二年詔修明史,未幾罷。康熙四年,又詔修之,亦止。十八年,命徐元文為監修,取彭孫遹等五十人官翰林,與右庶子盧君琦等十六人同為纂修。斯同嘗病唐以後史設局分修之失,以謂專家之書,才雖不逮,猶未至如官修者之雜亂,故辭不膺選。至三十二年,再召王鴻緒於家,命偕陳廷敬、張玉書為總裁。陳任本紀,張任志,而鴻緒獨任列傳。乃延斯同於家,委以史事,而武進錢名世佐之。每覆審一傳,曰某書某事當參校,顧小史取其書第幾卷至,無或爽者。士大夫到門諮詢,了辯如響。

嘗書抵友人,自言:「少館某所,其家有列朝實錄,吾默識暗誦,未敢有一言一事之遺也。長游四方,輒就故家耆老求遺書,考問往事。旁及郡志、邑乘,私家撰述,靡不搜討,而要以實錄為指歸。蓋實錄者,直載其事與言,而無可增飾者也。因其世以考其事,覈其言而平心察之,則其人本末可八九得矣。然言之發或有所由,事之端或有所起,而其流或有所激,則非他書不能具也。凡實錄之難詳者,吾以他書證之。他書之誣且濫者,吾以所得於實錄者裁之。雖不敢具謂可信,而是非之枉於人者蓋鮮矣。昔人於宋史已病其繁蕪,而吾所述將倍焉。非不知簡之為貴也,吾恐後之人務博而不知所裁,故先為之極,使知吾所取者有所捐,而所不取,必非其事與言之真,而不可溢也。」又以:「馬、班史皆有表,而後漢、三國以下無之。劉知幾謂得之不為益,失之不為損。不知史之有表,所以通紀、傳之窮者。有其人已入紀、傳而表之者,有未入紀、傳而牽連以表之者。表立而後紀、傳之文可省,故表不可廢。讀史而不讀表,非深於史者也。」嘗作明開國訖唐、桂功臣將相年表,以備採擇。其後明史至乾隆初大學士張廷玉等奉詔刊定,即取鴻緒史槁為本而增損之。鴻緒槁,大半出斯同手也。

平生淡於榮利,脩脯所入,輒以以周宗黨。故人馮京第死義,其子沒入不得歸,為醵錢贖之。尤喜獎掖後進。自王公以至下士,無不呼曰萬先生。李光地品藻人倫,以謂顧寧人、閻百詩及萬季野,此數子者,真足備石渠顧問之選。而斯同與人往還,其自署則曰「布衣萬某」,未嘗有他稱也。卒,年六十。著歷代史表,創為宦者侯表,大事年表二例。又著儒林宗派。

名世,字亮工。康熙四十二年一甲進士,授編修。夙負文譽,王士禎見其詩激賞之。鴻緒聘修明史,斯同任考核,付名世屬辭潤色之。官至侍讀,坐投詩諂年羹堯奪職。

劉獻廷,字繼莊,大興人,先世本吳人也。其學主經世,自象緯、律歷、音韻、險塞、財賦、軍政、以逮岐黃、釋老之書,無所不究習。與梁谿顧培、衡山王夫之、南昌彭士望為師友,而復往來昆山徐乾學之門。議論不隨人後。萬斯同引參明史館事,顧祖禹、黃儀亦引參一統志事。獻廷謂諸公考古有餘,實用則未也。

其論方輿書:「當於各疆域前,測北極出地,定簡平儀制度,為正切線表,而節氣之後先,日食之分秒,五星之凌犯占驗,皆可推矣。諸方七十二候不同,世所傳者本之月令。乃七國時中原之氣候,與今不合,則歷差為之。今宜細考南北諸方氣候,取其核者詳載之,然後天地相應,可以察其遷變之微矣。燕京、吳下,水皆南流,故必東南風而後雨,衡、湘水北流,故必北風而後雨。諸方山水向背分合,皆紀述之,而風土之剛柔,暨陰陽燥濕之徵,可次第而求矣。」

其論水利,謂:「西北乃先王舊都,二千餘年未聞仰給東南。何則?溝洫通,水利修也。自劉、石雲擾,以訖金、元,千餘年未知水利為何事,不為民利,乃為民害。故欲經理天下,必自西北水利始矣。西北水利,莫詳於水經酈注。雖時移勢易,十猶可得六七。酈氏略於東南,人以此少之。不知水道之當詳,正在西北。」於是欲取二十一史關於水利農田戰守者,考其所以,附以諸家之說,為之疏證。凡獻廷所撰著,類非一人一時所能成,故卒不就。

又嘗自謂於華嚴字母悟得聲音之道,作新韻譜,足窮造化之奧。證以遼人林益長之說,益自信。其法先立鼻音二,各轉陰、陽、上、去、入之五音共十聲,而不歷喉齶舌齒脣之七位。故有橫轉,無直送,則等韻重疊之失去。次定喉音四,為諸韻之宗,從此得半音、轉音、伏音、送音、變喉音。又以二鼻音分配之,一為東北韻宗,一為西南韻宗,八韻立,而四海之音可齊。於是以喉音互相合,得音十七;喉音鼻音互相合,得音十;又以有餘不盡者三合之,得音五:共三十二音,為韻父,而韻歷二十二位,為韻母。橫轉各有五子,而萬有不齊之聲攝於此矣。

同時吳殳盛稱其書。他所著多佚。歿後,弟子黃宗夏輯錄之,為廣陽雜記。全祖望稱為薛季宣、王道父一流雲。

邵遠平,字戒三,仁和人。康熙三年進士,選庶吉士。歷戶部郎中,出為江西學政,擢光祿寺少卿。試鴻博,授侍讀,至少詹事,致仕歸。以書史自娛,於世務泊如也。聖祖南巡,賜御書「蓬觀」額,因自號蓬觀子。遠平高祖經邦,明正德中進士,刑部員外郎。以建言獲罪。著弘簡錄,起唐迄宋,附以遼、金,未遑及元也。遠平循其例續之,刊除舊史衣復重不雅馴者,入制誥於帝紀,採著作於儒林,而文苑分經學、文學、藝學三科,十三志則分載於紀傳,名曰元史類編。硃彞尊稱其書非官局所能逮也。別著史學辨誤,京邸、粵行等集。

同邑吳任臣,字志伊。志行端愨,強記博聞,為顧炎武所推。以精天官、樂律試鴻博,入翰林,承修明史歷志。著周禮大義、禮通、春秋正朔考辨、山海經廣注、託園詩文集,而十國春秋百餘卷尤稱淹貫。其後如謝啟昆之西魏書,周春之西夏書,陳鱣之續唐書,義例皆精審,非徒矜書法,類史鈔也。

謝啟昆,字蘊山,南康人。乾隆二十五年進士。由編修簡鎮江知府,後至廣西巡撫,卒官。嘗築湘、漓二江之堤,詳見本傳。又修廣西通志,阬元言可為省志法。啟昆以魏書專主東魏,不載西魏四主,北史亦無糾正,乃作西魏書十二篇。

周春,字芚兮,海寧人。乾隆十九年進士,選岑溪令,父憂去。民懷其澤,合前令山陽劉信嘉、金壇於烜共祀之,曰岑溪三賢祠。重宴鹿鳴,加六品銜。卒,年八十七。撰述甚多,而西夏書為最著。

春同州陳鱣,字仲魚。強於記誦,喜聚書。州人吳騫拜經樓書亦富,得善木互相鈔藏。嘉慶改元,舉孝廉方正。又明年,中式舉人。計偕入都,從錢大昕、翁方綱、段玉裁游。後客吳門,與黃丕烈定交。精校勘之學。嘗以硃梁無道,李氏既系賜姓,復奉天祐年號,至十年立廟太原,合高祖、太宗、懿宗、昭宗為七廟,唐亡而實存焉;南唐為憲宗五代孫建王之玄孫,祀唐配天,不失舊物,尤宜大書年號,以臨諸國:於是撰續唐書七十卷。又有論語古訓、石經說、經籍跋文,恆言廣證諸書。卒,年六十五。

喬萊,字石林,寶應人。父可聘,明末為御史,有聲。萊,康熙六年進士,授內閣中書,乞養歸。十八年,試鴻博,授編修,與修明史。典廣西鄉試,充實錄館纂修官,遷侍讀。時御史奏濬海口,瀉積水,而河道總督靳輔言其不便,請於邵伯、高郵間置閘洩水,復築長堤抵海口束之,使水勢高則趨海易,廷議多主河臣言。適萊入直,詔問萊,疏陳四不可行,略謂:「開河築堤,勢必壞隴畝,毀村落,不可行一。淮、揚地卑,多積潦,今取濕土投深淵,工安得成?不可行二。築丈六之堤,束水高一丈,秋雨驟至,勢必潰;即當未潰,瀦水屋廬之上,豈能安枕?不可行三。至於七州縣之田,向沒於水,今更束河使高,則田水豈復能涸?不可行四。」帝是之,議乃寢。二十六年,罷歸。久之,召來京。旋卒。

萊著易俟,雜採宋、元諸家易說,推求人事,參以古今治亂得失,蓋誠齋易傳之支流。詩文有應制、直廬、使粵、歸田諸集。孫億,亦工詩。

汪楫,字舟次,江都人,原籍休寧。性伉直,意氣偉然。始以歲貢生署贛榆訓導。應鴻博,授檢討,入史館。言於總裁,先仿宋李燾長編,匯集詔諭、奏議、邸報之屬,由是史材皆備。二十一年,充冊封琉球正使,宣布威德。瀕行,不受例餽,國人建卻金亭志之。歸撰使琉球錄,載禮儀暨山川景物。又因諭祭故王,入其廟,默識所立主,兼得琉球世纘圖,參之明代事實,詮次為中山沿革志。出知河南府,置學田,嵩陽書院聘詹事耿介主講席。治行為中州最。擢福建按察使,遷布政使。楫少工詩,與三原孫枝蔚、泰州吳嘉紀齊名。有悔齋集、觀海集。

同里汪懋麟,字季甪,並有詩名,時稱「二汪」。康熙六年進士,授內閣中書。舉鴻博,持服不與試。服闋,復用徐乾學薦,以刑部主事入史館為纂修官。懋麟績學有幹才。為中書時,楚人硃方旦挾邪說動公卿,懋麟作辨道論詆之。熊賜履見其文,與定交。及居刑曹,勤於職事。有武某乘車宿董之貴家,之貴利其貲,殺之。車載而棄於道,鞭馬使馳。武父得車馬劉氏之門,訟劉殺其子。懋麟曰:「殺人而置其車馬於門,非理也。」乃微行,縱其馬,馬至之貴門,駭躍悲鳴。因收之貴,一訊得實,置於法。其發奸摘伏多類此。懋麟從王士禎學詩,而才氣橫逸,視士禎為別格。有百尺梧桐閣集。

陸葇,字次友,平湖人。幼時值大軍收平湖,父被執,葇詣軍前乞代父。軍將手詩★M3示之曰:「兒能讀是耶?吾赦汝父。」葇朗誦「收兵四解降王縛,教子三升上將臺」,曰;「此宋人贈曹武惠王詩也。將軍不嗜殺,即今之武惠王矣!」將軍喜,挾與北行,善育之,為議婚。以先問名於楊,辭歸。補諸生,入國學,試授中書。康熙六年進士,管內秘書院典藉。再試鴻博,授編修,分纂明史,命直南書房。三十三年,召試翰詹諸臣豐澤園,聖祖親置第一,謂曰:「連試詩文。無出汝右者。」一歲七遷,至內閣學士。長至,奏句決本,請出矜疑二十餘人。後一年告歸。葇性孝友,兄南雄知府世楷前卒,葇教養遺孤,俾成立,有名於時。年七十,卒。著雅坪詩文槁。

奎勛,字聚侯,世楷子也。少隨葇京師,以學行為公卿所推重,顧久困諸生中。康熙末,年幾六十,始成進士,授檢討,充明史纂修官。匄疾歸,主廣西秀峰書院。奎勛篤於經學,忘饑渴寒暑。著陸堂易學,謂說卦一篇,足該全易。其詩學與明何楷詩世本古義相近。尚書說,惟解伏生今文二十八篇、戴禮緒言,糾正漢人穿鑿附會之失。春秋義存錄,則凡經、傳、子、緯所載孔子語盡援為據,力主春秋非以一字褒貶。奎勛說經務新奇,使聽者忘倦。最後撰古樂發微,未成而卒。

龐塏,字霽公,任丘人。生有至性。七歲時,父緣事被逮,母每夕禱天。塏即隨母泣拜,無或間也。稍長,工為文。康熙十四年舉人,試鴻博,授檢討,分修明史。明都御史某諂附魏忠賢,其裔孫私餽金,匄閹黨傳諱其事勿書,力拒之。大考降補中書,洊擢戶部郎中,出知建寧府。浦城民以令嚴苛激變,夜焚冊局,殺吏胥,罷市,令懼而逃。塏聞變即馳至浦城,集士民明倫堂,曉喻禍福,戮一人而事定。民感其德,立書院祀之。九仙山多盜,至掠人索贖。掩捕數十人,境內帖然。未幾,告歸。

塏嗜吟詠,與同裏邊汝元以詩學相劘切。其所作醇雅,以自然為宗。有叢碧山房集。

汝元子連寶,字趙珍。世其家學。以諸生貢成均,廷試第一。應乾隆元年博學鴻詞科,不中選。十四年,復薦經學,辭不赴。或勸之行,曰:「吾自審不能如漢伏勝、董仲舒,安敢幸取哉?」著有隨園集。

陸圻,字麗京,錢塘人。少與弟堦、培以文學、志行見重於時,稱曰「三陸」。所為詩號西陵體。性穎異,善思誤書。嘗讀韓非子「一從而咸危」,曰:「是『一徙而成邑』也。」戲令他人射覆,不得,惟弟廷中之。平生不喜言人過,有語及者,輒曰:「吾與汝,姑自淑。」莊廷鑨史禍作,圻坐逮。以先嘗具狀自陳,事得白,嘆曰:「今幸得不死,奈何不以餘年學道耶!」親歿,遂棄家遠游,不知所終。子寅,成進士。往來萬里,尋父不得,竟悒悒以死,時稱其孝。培死甲申之難。

丁澎,字飛濤,仁和人。有雋才。嗜飲,一石不亂,弟景鴻、溁並能文,時有「三丁」之目。澎,順治十二年進士,官禮部郎中。嘗典河南鄉試,得一卷奇之。同考請置之乙,澎曰:「此名士也!」榜發,乃廬陽李天馥,出語人曰:「吾以世目衡文,幾失此士。」坐事謫居塞上五載,躬自飯牛,吟嘯自若。所作詩多忠愛,無怨誹之思。有扶荔堂集。

先是陳子龍為登樓社,圻、澎及同里柴紹炳、毛先舒、孫治、張丹、吳百朋、沈謙、虞黃昊等並起,世號「西泠十子」。

紹炳,字虎臣。在十子中文名最著。持躬尤端謹。有省軒集。

先舒,字稚黃。嘗從劉宗周講學。其詩音節瀏亮,有七子餘風。著思古堂集。

治,字宇臺。篤友誼,陸培死,以孤女託為擇婿,得吳任臣。及立嗣,又以甥女嫁焉。有鑒菴集。

丹,字綱孫。美須髯。淡靜不樂交游,而嗜山水。其詩悲涼沉遠,曰秦亭集。

百朋,字錦雯。以舉人令南和,有異政,百姓祠祀之。有襆庵集。

謙,字去矜。工詩,初喜溫、李,後乃循漢、魏以窺盛唐。有東江草堂集。謙與紹炳、先舒皆精韻學。紹炳作古韻通,先舒作韻學通指、南曲正韻,謙作東江詞韻。陸圻嘆曰:「恨孫偭、周德清曾無先覺。」

黃昊,字景明。十歲即善屬文。薄柳州乞巧,更作辭巧文,識者知其遠到。康熙中舉人,終教諭。

孫枝蔚,字豹人,三原人。少遭闖賊亂,結邑里少年擊賊,墮坎埳,幸不死。乃走江都,習賈,屢致千金,輒散之。既乃折節讀書,僦居董相祠,高不見之節。王士禎官揚州,以詩先,遂定交,稱莫逆焉。時左贊善徐乾學方激揚士類,才俊滿門,枝蔚弗屑也。以布衣舉鴻博,自陳衰老,乞還山,遂不應試,授內閣中書。著溉堂集,詩詞多激壯之音,稱其高節。

李念慈,字屺瞻,涇陽人。順治十五年進士,以河間府推官改知新城縣。坐逋賦罷。會有荊襄之役,敘運餉勞,再起,補天門。與枝蔚同舉鴻博,試不中選。喜游,好吟詠。有谷口山房集。施閏章稱其雄爽之氣勃勃眉宇,蓋秦風而兼吳、楚者。

丁煒,字瞻汝,晉江人。諸生。工詩,有吏才。順治十二年,定遠大將軍濟度統師取漳州,詔便宜置郡縣吏,得試士幕下,拔煒第一。授漳平教諭,遷知直隸獻縣,內擢戶部主事。時議稅閩鹽,煒力陳不可,事得寢。由郎中出為贛南分巡道。閩人佃贛者乘亂劫略,號「田賊」,捕治之,民情大洽。遷湖北按察使,脫重囚為盜誣者二十餘人於獄。尋坐事謫官,居武昌,未發,武昌卒夏包子作亂,脅使署。巡撫以死拒,東走安慶,乞師巡撫楊素蘊。事平,降補知府雲南。會素蘊移撫湖廣,以煒事聞,復按察職。俄以疾歸。

煒論詩,以為詩貴合法,然法勝則離;貴近情,然情勝則俚。故其為詩,力追三唐、漢、魏。無詭薄之失。有問山集。

林侗,字同人,閩人也。縣貢生。喜金石。卒,年八十八。弟佶,字吉人。康熙五十二年進士,官中書,工楷法。文師汪琬,詩師陳廷敬、王士禎。此三人集皆佶手繕付雕,精雅為世所重。家多藏書,徐乾學輯經解,硃彞尊選明詩,皆就傳鈔。有樸學齋集。

黃任,字莘田,永福人。工書。口辯若懸河。有硯癖,以舉人令四會,罷官歸,惟硯石壓裝。詩清新刻露,有香草齋集。乾隆二十七年,重宴鹿鳴。卒,年八十餘。

鄭方坤,字則厚,建安人。雍正元年進士。為令邯鄲,屢擢至山東兗州知府。時禁人口出海,抵奉天而未入籍者,悉勒還本土。方坤適知登州,以為司牧者但當嚴奸宄之防,不得閉其謀生之路,為白大吏,弛其禁。調武定,能盡心賑務。兗州饑,復移治之。方坤記誦博,詩才凌厲,與兄方城齊名。有蔗尾集,又著經稗、五代詩話、全閩詩話、國朝詩人小傳。

黃與堅,字廷表,太倉人。幼有奇慧,八歲,酷好唐人詩,錄小本,懷袖中諷誦之。已而究心經術,遍讀周、秦古書。性落落,與人交有終始。順治十六年進士,後舉鴻博,授編修,遷贊善,分修明史及一統志。寓居委巷,寂寞著書,如窮愁專一之士。有忍菴集。

吳偉業選「婁東十子」詩,以與堅為冠。十子者,周肇、許旭、王撰、王攄、王昊、王揆、王忭、王曜升、顧湄也。肇詩曰東岡集,旭曰秋水集,撰曰三餘集,攄曰蘆中集。

昊,為世貞後,有文藻,下筆如宿構。康熙十八年,召試,授官正字。所著曰碩園集。揆,順治中進士,所著曰芝廛集。忭曰健菴集,曜升曰東皋集。

湄,字伊人,亦太倉人。事母以孝聞,父夢麟,長於毛、鄭之學,湄傳其業。尤工詩,清麗婉約,陳瑚以為過元人。其詩曰水鄉集。

吳雯,季天章,蒲州人,原籍遼陽。父允升,任蒲州學政,卒官,遂家焉。雯少朗悟,記覽甚博,尤長於詩。游京師,父執劉體仁、汪琬皆激賞之。王士禎目為仙才。嘗與葉方靄同直,誦其警句,方靄下直即趨訪,名大噪。大學士馮溥出扇索詩,雯大書二絕句答之,其坦率類是。卒以不遇,不悔也。試鴻博不中選。後居母憂,以毀卒。雯著蓮洋集,詩體峻潔,有其鄉人元好問之風。據名山記蓮洋村在華嶽下,取以名集。

陶季,寶應人。初名澂,字季深,以字行,復去其一,稱曰陶季。負異才,鋒穎踔厲。游燕、趙、齊、魯之郊,逾太行,浮湘、沅,所至皆有詩。士禎刪定其客滇南、閩中諸詩,以高、岑、龍標相況。先是詔舉鴻博,公卿爭欲薦,季辭不就,以布衣終。有湖邊草堂集及舟車集。

梅清,字瞿山,宣城人,宋梅堯臣後也。清英偉豁達,自力於學,以淹雅稱。順治十一年舉人,試禮部不第。朝士爭與之交,王士禎、徐元文尤傾倒焉。詩凡數變,自訂天延閣前後集。年七十餘,復合編瞿山詩略。書法仿顏真卿、楊凝式。畫尤盤薄多奇氣。嘗作黃山圖,極煙雲變幻之勝,為當時所重。同族有梅庚者,生後於清。善八分書,亦工詩畫,與清齊名。

庚,字耦長。少孤,承其祖鼎祚、父朗中之傳,益昌大之。施閏章見其詩,引為忘年交。康熙二十年舉人,為硃彞尊所得士。性狷介,客游京師,不妄投一刺。士禎主禮闈,庚復被黜,士禎贈詩引為恨也。後知泰順縣,有惠政,民德之。

馮景,字山公,錢塘人。國子監生。善屬文,千言立就。康熙時游京師,侍郎項景襄、金鼐皆遣子弟從受學。會營宮室,求楠木梁不得,有請以他木易國子監彞倫堂梁者。景上書尚書魏象樞,極陳不可,事得寢。由是馮太學生之名盛傳京師。大學士索額圖召欲見之,謝不往。歸館淮安邱象隨家垂十年。宋犖撫江蘇,禮致幕府,或納金求為緩頰,峻卻之,人益欽其品。景篤師友風義,與仁和汪煜、湯右曾交最篤。二人為給事中,多所論列,亦由景數責善有以激厲之也。王士禎轉左都御史,景以受知士禎,冀其大有匡濟,為書諷之。景雖布衣,不求仕進,而未嘗忘當世之務。在淮安時,有水患,湯斌奉詔北上,作書陳災狀及所以致患之由,斌見書嗟賞,又嘗稱其文為不朽。其著述多佚,今存者解舂集。

邵長蘅,字子湘,武進人。十歲補諸生,因事除名,旋入太學。工詩,尤致力古文辭,陶鍊雅正。與景同客犖幕,長蘅亦觥觥持古義,無所貶損,時論賢之。著有青門稿。

姜宸英,字西溟,慈谿人,明太常卿應麟曾孫。父晉珪,諸生,以孝聞。宸英績學工文辭,閎博雅健。屢躓於有司,而名達禁中。聖祖目宸英及硃彞尊、嚴繩孫為海內三布衣。侍讀學士葉方靄薦應鴻博,後期而罷。方藹總裁明史,又薦充纂修,食七品俸,分撰刑法志。極言明詔獄,廷杖,立枷,東、西廠之害,辭甚愷至。尚書徐乾學領一統志事,設局洞庭東山,疏請宸英偕行。久之,舉順天鄉試。三十六年,成進士。廷對李蟠第一,嚴虞惇第二,帝識宸英手書,親拔置第三人及第,授編修,年七十矣。明年,副蟠典試順天,蟠被劾遣戍,宸英亦連坐。事未白,卒獄中。

宸英性孝友。與人交,坦夷而不阿。祭酒翁叔元劾湯斌偽學,遽移書責之。著湛園集、葦間集。書法得鍾、王遺意,世頗重之。

虞惇,字贊成,常熟人。幼能背誦九經、三史。既官翰林,館閣文字多出其手。科場獄興,虞惇諸子是科獲雋,考官蟠、宸英皆其同年友。用是罣吏議鐫級,閒居數年。起大理寺寺副,平反內務府殺人移獄被誣者,累遷太僕寺少卿,卒官。著有讀詩質疑。江南人刻其文曰嚴太僕集,以繼明歸太僕云。

黃虞稷,字俞邰,上元人,本籍晉江。七歲能詩。以諸生舉鴻博,遭母喪,不與試。左都御史徐元文薦修明史,又修一統志,皆與宸英同。家富藏書。著千頃堂書目,為明史藝文志所本。

性德,納喇氏,初名成德,以避皇太子允礽嫌名改,字容若,滿洲正黃旗人,明珠子也。性德事親孝,侍疾衣不解帶,顏色黧黑,疾愈乃復。數歲即習騎射,稍長工文翰。康熙十四年成進士,年十六。聖祖以其世家子,授三等侍衛,再遷至一等。令賦乾清門應制詩,譯禦制松賦,皆稱旨。俄疾作,上將出塞避暑,遣中官將御醫視疾,命以疾增減告。遽卒,年止三十一。嘗奉使塞外有所宣撫,卒後,受撫諸部款塞。上自行在遣中官祭告,其眷睞如是。

性德鄉試出徐乾學門。與從揅討學術,嘗裒刻宋、元人說經諸書,書為之序,以自撰禮記陳氏集說補正附焉,合為通志堂經解。性德善詩,尤長倚聲。遍涉南唐、北宋諸家,窮極要眇。所著飲水、側帽二集、清新秀雋,自然超逸。嘗讀趙松雪自寫照詩有感,即繪小像,仿其衣冠。坐客期許過當,弗應也。乾學謂之曰:「爾何似王逸少!」則大喜。好賓禮士大夫,與嚴繩孫、顧貞觀、陳維崧、姜宸英諸人游。貞觀友吳江吳兆騫坐科場獄戍寧古塔,賦金縷曲二篇寄焉,性德讀之嘆曰:「山陽思舊,都尉河梁,並此而三矣!」貞觀因力請為兆騫謀,得釋還,士尤稱之。

貞觀,字梁汾,無錫人。康熙十一年舉人,官內閣中書。工詩,自定集僅五言三十餘篇,清微婉篤,上睎韋、柳;而世特傳其詞,與維崧及硃彞尊稱詞家三絕。清世工詞者,往往以詩文兼擅,獨性德為專長,仁和譚獻嘗謂為詞人之詞。性德後,又得項鴻祚、蔣春霖三家鼎立。

鴻祚,字蓮生,錢塘人。道光十二年舉人。善詞,上溯溫、韋,下逮周密、吳文英。擷精棄滓,以自名其家。屢應禮部試不第。卒,年三十八。自序憶雲詞,有曰:「不為無益之事,何以遣有涯之生!」學者誦而悲之。

春霖,字鹿潭,江陰人,寄籍大興。咸豐中,官東臺場鹽大使。工詞。時方亂離,傍徨沉鬱,高者直逼姜夔。困於卑官,孤介忤時,益侘傺。舟經吳江,一夕暴卒。春霖慕性德飲水、鴻祚憶雲,自署水雲樓,即以名其詞。

宗室文昭,字子晉,饒餘親王阿巴泰曾孫,鎮國公百綬子。辭爵讀書,從王士禎游。工詩,才名藉甚。王式丹稱其詩以鮑、謝為胚胎,而又兼綜眾有,擷百家之精華,其味在酸咸之外。著有薌嬰居士集、紫幢詩鈔。

又宗室以詩名者,蘊端,初名嶽端,字正子,號紅蘭主人,多羅安郡王岳樂子。封貝子。有玉池生稿。

博爾都,字問亭,號東皋漁父,恪僖公拔都海子,蘊端從弟。封輔國將軍。有問亭詩集。

永忠,字良輔,又字臞仙,多羅貝勒弘明子。輔國將軍。有延芬室集。詩體秀逸,書法遒勁,頗有晉人風味。常不衫不履,散步市衢。遇奇書異籍,必買之歸,雖典衣絕食不顧也。

書諴,字實之,號樗仙,鄭獻親王濟爾哈朗六世孫,輔國將軍長恆子。奉國將軍。有靜虛堂集。性慷慨,不欲嬰世俗情。年四十,即託疾去官。邸有餘隙地,盡種蔬果,手執畚鎛,從事習勞以為樂。

永諲,字嵩山,康修親王崇安子。鎮國將軍。詩宗盛唐,書法趙文敏。晚年獨居一室,不與人接。詩多散佚。

裕瑞,字思元,豫通親王多鐸裔。封輔國公。工詩善畫,通西番語。常畫鸚鵡地圖,即西洋地球圖。又以佛經自唐時流入西藏,近日佛藏皆出一本,無可校讎。乃取唐古特字譯校,以復佛經唐本之舊,凡數百卷。著有思元齋集。

趙執信,字仲符,益都人。從祖進美,官福建按察使,詩名甚著。執信承其家學,自少即工吟詠。年十九,登康熙十八年進士,授編修。時方開鴻博科,四方雄文績學者皆集輦下,執信過從談宴,一座盡傾。硃彞尊、陳維崧、毛奇齡尤相引重,訂為忘年交。出典山西鄉試,遷右贊善。二十八年,坐國恤中宴飲觀劇,為言者所劾,削籍歸。卒,年八十餘。

執信為人峭峻褊衷,獨服膺常熟馮班,自稱私淑弟子。娶王士禎甥女,初頗相引重。後求士禎序其詩,士禎不時作,遂相詬厲。嘗問詩聲調於士禎,士禎靳之,乃歸取唐人集排比鉤稽,竟得其法,為聲調譜一卷。又以士禎論詩,比之神龍不見首尾,雲中所露一鱗一爪而已,遂著談龍錄,云:「詩以言志,詩之中須有人在,詩之外尚有事在。」意蓋詆士禎也。說者謂士禎詩尚神韻,其弊也膚;執信以思路劖刻為主,其失也纖。兩家才性不同,實足相資濟云。執信所著詩文曰飴山堂集。

當是時,海內以詩名者推士禎,以文名者推汪琬。而嘉興葉燮,字星期,其論文亦與琬不合,往復論難,互譏嘲焉。及琬歿,慨然曰:「吾失一諍友矣!今誰復彈吾文者?」取向所短汪者悉焚之。燮父紹袁,明進士,官工部主事,國亡後為僧。燮生四歲,授以楚辭,即成誦。康熙九年進士,選授寶應令。值三籓亂,又歲饑,民不堪苦。累以伉直失上官意,坐累落職。時嘉定知縣陸隴其亦被劾,燮以與隴其同罷為幸。性喜山水,縱游宇內名勝幾遍。年七十六,猶以會稽、五洩近在數百里獨未游為憾。復裹糧往,歸遂疾。逾年卒。寓吳時,以吳中論詩多獵範、陸皮毛,而遺其實,著原詩內外篇,力破其非。吳士始而訾謷,久乃更從其說。著已畦詩文集。士禎謂其鎔鑄往昔,獨立起衰。

馮廷櫆,字大木,德州人。康熙二十一年進士,授中書。幼有奇童之目,讀書一覽輒記,尤長於詩。嘗充湖廣副考官,試畢,登黃鶴樓,俯江、漢之流,南望瀟湘、洞庭,慨然遠想,賦詩百餘篇,識者以為騷之遺也。平生深契者惟執信,其詩孤峭亦相類,歿後散佚。其孫德培搜輯得五百篇,名馮舍人遺詩。

黃儀,字六鴻,常熟人。精輿地之學。嘗以班固地志所載諸川,第詳水出入,其中間經歷之地,備著於水經,然讀者非繪圖不能了,乃反覆尋究,每水各為一圖。凡都邑建署沿革、山川險易皆具焉,條縷分析,各得其理。閻若璩見之,嘆曰:「酈道元千古後一知己也!」若璩嘗問儀:「後漢志溫縣濟水出,王莽時大旱,遂枯絕。是河南無濟矣,何酈氏言之詳也?」儀曰:「新莽時雖枯,後復見,酈氏所謂其後水流逕通,津渠勢改,尋梁脈水,不與昔同是也。杜君卿乃不信水經,專憑彪志,竊以彪特紀一時災變耳,非謂永不截河南過也。」徐乾學修一統志,儀與若璩、胡渭、顧祖禹任分纂,皆地學專家。儀又訂正晉書地理志。兼工詩詞,著有紉蘭集。

鄭元慶,字芷畦,歸安人。通史傳,旁及金石文字。李紱、張伯行雅重其學,欲薦於朝未得也。顏魯公書湖州石柱記,元慶為之箋釋,甚博贍。又著湖錄百二十卷,七易槁而後成,自謂平生精力殫於是書。平生慕鄭子真之為人,自號鄭谷口。晚更治經,其著書處名魚計亭。著有周易集說、詩序傳異同、禮記集說參同、官禮經典參同、家禮經典參同、喪服古今異同考、春王正月考、海運議。

查慎行,字悔餘,海寧人。少受學黃宗羲。於經邃於易。性喜作詩,游覽所至,輒有吟詠,名聞禁中。康熙三十二年,舉鄉試。其後聖祖東巡,以大學士陳廷敬薦,詔詣行在賦詩。又詔隨入都,直南書房。尋賜進士出身,選庶吉士,授編修。時族子升以諭德直內廷,宮監呼慎行為老查以別之。帝幸南苑,捕魚賜近臣,命賦詩。慎行有句云:「笠簷蓑袂平生夢,臣本煙波一釣徒。」俄宮監傳呼「煙波釣徒查翰林」。時以比「春城寒食」之韓翃云。充武英殿書局校勘,乞病還。坐弟嗣庭得罪,闔門就逮。世宗識其端謹,特許於歸田里,而弟嗣瑮謫遣關西,卒於戍所。

嗣瑮,字德尹。康熙三十九年進士,官至侍講。性警敏,數歲即解切韻諧聲。詩名與慎行相埒。慎行著敬業堂集、周易玩辭集解,又補注蘇詩,行於世。嗣瑮著查浦詩鈔、音類通考。

升,字仲韋。康熙二十七年進士。官少詹事。詩筆清麗。尤工書,似董其昌。有澹遠堂集。

史申義,字叔時,江都人。少工詩,與同里顧圖河齊名,稱維揚二妙。康熙二十七年進士,授編修。充雲南鄉試考官,改御史、禮科給事中,乞病歸。王士禎以風雅詔後進,嘗謂申義及湯右曾足傳己衣缽,人稱「王門二弟子」。在翰林時,聖祖以後進詩人詢大學士陳廷敬,廷敬舉申義、周起渭對,故又有「翰苑兩詩人」之目。

起渭,字漁塘,貴陽人。康熙三十三年進士,由檢討累遷詹事府詹事。詩才雋逸,尤肆力於蘇軾、元好問、高啟諸家。貴州自明始隸版圖,清詩人以起渭為冠,而銅仁張元臣、平遠潘淳亦並有詩名。

元臣,字志伊。康熙三十六年進士,由檢討累遷左諭德。有豆村詩鈔。

淳,字元亮。康熙五十四年進士,官檢討。文安陳儀與同榜,一時咸推潘詩陳筆。有椽林詩集。

顧陳垿,字玉停,鎮洋人。少有文名,嘗得徐光啟歷書,精求一月,通其術。康熙五十四年舉人,以薦入湛凝齋修書。書成,議敘行人司行人。時外廷送算學三百餘員候試,聖祖親策之,得七十二人,陳垿為冠。又充樂館纂修。雍正元年,出使山東、浙江,還督通州倉。三年,以目疾乞歸,閉門撰述,四方走書幣乞文者踵至。性耿介,敦於內行。居喪不飲酒食肉,不處內。沈起元官河南,延主大梁書院,引範文正憂中掌學睢陽以勸;陳垿執象山責東萊故事,謝不往也。乾隆元年,詔起官,又舉鴻博,及六年設樂部,復以洞曉音律宣召,皆辭不赴,時論高之。年七十,卒。

陳垿精字學、算學、樂律,時稱三絕。嘗造八矢注守圖說,謂字學居六藝之末,聲音,樂也,形體,書也,而口出耳入,手運目存,則皆有數焉。學士惠士奇、通政孫勷得其書,置酒延陳垿請其說。陳垿為言經聲緯音開發收閉之旨,及每矢實義,一矢未發,則聲不能出,字有所避,八矢盡而音定字死矣。二人嘆為天授。少與同里王時翔為性命交,並工詩。婁東詩人大率宗吳偉業,陳序晚出,乃自闢町畦。著洗桐集、抱桐集。

何焯,字屺瞻,長洲人。通經史百家之學。藏書數萬卷,得宋、元舊槧,必手加讎校,粲然盈帙。學者稱義門先生,傳錄其說為義門讀書記。

康熙四十一年,直隸巡撫李光地以草澤遺才薦,召入南書房。明年,賜舉人,試禮部下第,復賜進士,改庶吉士。仍直南書房,授皇八子讀,兼武英殿纂修。連丁內外艱。久之,復以光地薦,召授編修。尚書徐乾學、翁叔元爭延致焯。尋遘讒,與乾學失歡,而叔元劾湯斌,焯上書請削門下籍,天下快之。聖祖幸熱河,或以蜚語上聞,還京即命收系。盡籍其卷冊文字,帝親覽之,曰:「是固讀書種子也!」無失職觖望語,又見其草槁有手簡吳縣令卻金事,益異之。命還所籍書,解官,仍參書局。六十一年,卒,年六十一。帝深悼惜,特贈侍講學士。贈金,給符傳歸喪,命有司存恤其孤。

焯工楷法,手所校書,人爭傳寶。門人著錄者四百人,吳江沈彤、吳縣陳景雲為尤著。

景雲,字少章。博聞彊識,能背誦通鑒。年十七,湯斌撫吳,試士拔第一。應京兆試,不遇。館籓邸三年,以母老辭歸,遂不出,以諸生終。少從焯游,焯歿,獨系吳中文獻幾二十年。著有讀書紀聞及綱目、通鑒、兩漢書、三國志、文選、韓、柳集皆有訂誤,共三十餘卷。文集四卷,亦簡嚴有法。

子黃中,字和叔。諸生。父子皆長史學,而黃中尤以才略自負。舉乾隆元年博學鴻詞,入都上書,論用人、理財、治兵三端。大學士陳世倌韙其言。頃之,詔求骨鯁之士,如古馬周、陽城者,世倌欲薦之,謝不應。胡天游傲睨群士,獨推服黃中。示以文,每發其瑕璺,未嘗有忤也。嘗病宋史蕪雜,別撰紀傳表百七十卷。又著國朝謚法考、閣部督撫年表。其卒也貧不能葬,或賻以金,妻張氏固卻之,曰:「奈何以貧故,傷夫子義!」遂賣所居宅以營葬。

戴名世,字田有,桐城人。生而才辨雋逸,課徒自給。以制舉業發名廩生,考得貢,補正藍旗教習。授知縣,棄去。自是往來燕、趙、齊、魯、河、洛、吳、越之間,賣文為活。喜讀太史公書,考求前代奇節瑋行。時時著文以自抒湮鬱,氣逸發不可控御。諸公貴人畏其口,尤忌嫉之。嘗遇方苞京師,言曰:「吾非役役求有得於時也,吾胸中有書數百卷,其出也,自忖將有異於人人。然非屏居深山,足衣食,使身無所累,未能誘而出之也。」因太息別去。康熙四十八年,年五十七,始中式會試第一,殿試一甲二名及第,授編修。又二年而南山集禍作。

先是門人尤雲鶚刻名世所著南山集,集中有與餘生書,稱明季三王年號,又引及方孝標滇黔紀聞。當是時,文字禁網嚴,都御史趙申喬奏劾南山集語悖逆,遂逮下獄。孝標已前卒,而苞與之同宗,又序南山集,坐是方氏族人及凡掛名集中者皆獲罪,系獄兩載。九卿覆奏,名世、雲鶚俱論死。親族當連坐,聖祖矜全之。又以大學士李光地言,宥苞及其全宗。申喬有清節,惟興此獄獲世譏云。名世為文善敘事,又著有孑遺錄,紀明末桐城兵變事,皆毀禁,後乃始傳云。


\end{pinyinscope}