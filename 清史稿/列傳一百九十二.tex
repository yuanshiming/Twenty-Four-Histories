\article{列傳一百九十二}

\begin{pinyinscope}
曾國籓

曾國籓,初名子城,字滌生,湖南湘鄉人。家世農。祖玉屏,始慕鄉學。父麟書,為縣學生,以孝聞。

國籓,道光十八年進士。二十三年,以檢討典試四川,再轉侍讀,累遷內閣學士、禮部侍郎,署兵部。時太常寺卿唐鑒講學京師,國籓與倭仁、吳廷棟、何桂珍嚴事之,治義理之學。兼友梅曾亮及邵懿辰、劉傳瑩諸人,為詞章考據,尤留心天下人材。

咸豐初,廣西兵事起,詔群臣言得失。奏陳今日急務,首在用人,人才有轉移之道,有培養之方,有考察之法。上稱其剴切明辨。尋疏薦李棠階、吳廷棟、王慶雲、嚴正基、江忠源五人。寇氛益熾,復上言:「國用不足,兵伍不精,二者為天下大患。於歲入常額外,誠不可別求搜刮之術,增一分則民受一分之害。至歲出之數,兵餉為鉅,綠營兵額六十四萬,常虛六七萬以資給軍用。自乾隆中增兵議起,歲糜帑二百餘萬。其時大學士阿桂即憂其難繼,嘉、道間兩次議裁,不及十之四,仍宜汰五萬,復舊額。自古開國之初,兵少而國強,其後兵愈多則力愈弱,餉愈多則國愈貧。應請皇上注意將才,但使七十一鎮中有十餘鎮足為心腹,則緩急可恃矣。」又深痛內外臣工諂諛欺飾,無陳善責難之風。因上敬陳聖德預防流弊一疏,切指帝躬,有人所難言者,上優詔答之。歷署刑部、吏部侍郎。二年,典試江西,中途丁母憂歸。

三年,粵寇破江寧,據為偽都,分黨北犯河南、直隸,天下騷動,而國籓已前奉旨辦團練於長沙。初,國籓欲疏請終制,郭嵩燾曰:「公素具澄清之抱,今不乘時自效,如君父何?且墨絰從戎,古制也。」遂不復辭。取明戚繼光遺法,募農民樸實壯健者,朝夕訓練之。將領率用諸生,統眾數不逾五百,號「湘勇」。騰書遐邇,雖卑賤與鈞禮。山野材智之士感其誠,莫不往見,人人皆以曾公可與言事。四境土匪發,聞警即以湘勇往。立三等法,不以煩府縣獄。旬月中,莠民猾胥,便宜捕斬二百餘人。謗讟四起,自巡撫司道下皆心誹之,至以盛暑練操為虐士。然見所奏輒得褒答受主知,未有以難也。一日標兵與湘勇閧,至闌入國籓行臺。國籓親訴諸巡撫,巡撫漫謝之,不為理,即日移營城外避標兵。或曰:「曷以聞?」國籓嘆曰:「大難未已,吾人敢以私憤瀆君父乎?」

嘗與嵩燾、忠源論東南形勢多阻水,欲剿賊非治水師不可,乃奏請造戰艦於衡州。匠卒無曉船制者,短橈長槳,出自精思,以人力勝風水,遂成大小二百四十艦。募水陸萬人,水軍以褚汝航、楊載福、彭玉麟領之,陸軍以塔齊布、羅澤南領之。賊自江西上竄,再陷九江、安慶。忠源戰歿廬州,吳文鎔督師黃州亦敗死。漢陽失,武昌戒嚴,賊復乘勢擾湖南。國籓銳欲討賊,率水陸軍東下。舟師初出湖,大風,損數十艘。陸師至岳州,前隊潰退,引還長沙。賊陷湘潭,邀擊靖港,又敗,國籓憤投水,幕下士章壽麟掖起之,得不死。而同時塔齊布大破賊湘潭,國籓營長沙高峰寺,重整軍實,人人捓揄之。或請增兵,國籓曰:「吾水陸萬人非不多,而遇賊即潰。岳州之敗,水師拒戰者惟載福一營;湘潭之戰,陸師塔齊布、水師載福各兩營:以此知兵貴精不貴多。故諸葛敗祁山,且謀減兵損食,勤求己過,非虛言也。且古人用兵,先明功罪賞罰。今世亂,賢人君子皆潛伏,吾以義聲倡導,同履危亡。諸公之初從我,非以利動也,故於法亦有難施,其致敗由此。」諸將聞之皆服。

陸師既克湘潭,巡撫、提督上功,而國籓請罪。上詰責提督鮑起豹,免其官,以塔齊布代之。受印日,士民聚觀,嘆詫國籓為知人,而天子能明見萬里也。賊自岳州陷常德,旋北走,武昌再失。國籓引兵趨岳州,斬賊梟將曾天養,連戰,下城陵磯。會師金口,謀取武昌。澤南沿江東岸攻花園寇屯,塔齊布伏兵洪山,載福舟師深入寇屯,士皆露立,不避鉛丸。武昌、漢陽賊望見官軍盛,宵遁,遂復二郡。國籓以前靖港敗,自請奪官,至是奏上,詔署湖北巡撫,尋加兵部侍郎銜,解署任,命督師東下。

當是時,水師奮厲無前,大破賊田家鎮,斃賊數萬,至於九江,前鋒薄湖口。攻梅家洲賊壘不下,駛入鄱湖。賊築壘湖口斷其後,舟不得出,於是外江、內湖阻絕。外江戰船無小艇,賊乘舴艋夜襲營,擲火燒坐船,國籓跳而免,水師遂大亂。上疏請罪,詔旨寬免,謂於大局無傷也。五年,賊再陷武漢,擾荊襄。國籓遣胡林翼等軍還援湖北,塔齊布留攻九江,而躬至南昌撫定水師之困內湖者。澤南從征江西,復弋陽,拔廣信,破義寧,而塔齊布卒於軍。國籓在江西與巡撫陳啟邁不相能,澤南奔命往來,上書國籓,言東南大勢在武昌,請率所部援鄂,國籓從之。幕客劉蓉諫曰:「公所恃者塔、羅。今塔將軍亡,羅又遠行,脫有急,誰堪使者?」國籓曰:「吾計之熟矣,東南大局宜如是,俱困於此無為也。」嵩燾祖餞澤南曰:「曾公兵單,奈何?」澤南曰:「天茍不亡本朝,公必不死。」九月,補授兵部侍郎。

六年,賊酋石達開由湖北竄江西,連陷八府一州,九江賊踞自如,湖南北聲息不相聞。國籓困南昌,遣將分屯要地,羽檄交馳,不廢吟誦。作水陸師得勝歌,教軍士戰守技藝、結營布陳之法,歌者咸感奮,以殺賊敢死為榮。顧眾寡,終不能大挫賊。議者爭請調澤南軍,上以武漢功垂成,不可棄。澤南督戰益急,卒死於軍。玉麟聞江西警,芒鞋走千里,穿賊中至南昌助守。林翼已為湖北巡撫,國籓弟國華、國葆用父命乞師林翼,將五千人攻瑞州。湖南巡撫駱秉章亦資國荃兵援吉安,兄弟皆會行間。而國籓前所遣援湖北諸軍,久之再克武漢,直下九江,李續賓八千人軍城東。續賓者,與弟續宜皆澤南高第弟子也。載福戰船四百泊江兩岸,江寧將軍都興阿馬隊、鮑超步隊駐小池口,凡數萬人。國籓本以憂懼治軍,自南昌迎勞,見軍容甚盛,益申儆告誡之。而是時江南大營潰,督師向榮退守丹陽,卒。和春為欽差大臣,張國樑總統諸軍攻江寧。

七年二月,國籓聞父憂,逕歸。給三月假治喪,堅請終制,允開侍郎缺。林翼既定湖北,進圍九江,破湖口,水師絕數年復合。載福連拔望江、東流,揚風過安慶,克銅陵泥汊,與江南軍通。由是湘軍水師名天下。林翼以此軍創始國籓,楊、彭皆其舊部,請起國籓視師。會九江克復,石達開竄浙江,浸及福建,分股復犯江西,朝旨詔國籓出辦浙江軍務。

國籓至江西,屯建昌,又詔援閩。國籓以閩賊不足慮,而景德地沖要,遣將援贛北,攻景德。國荃追賊至浮梁,江西列城次第復。時石達開復竄湖南,圍寶慶。上慮四川且有變,林翼亦以湖北餉倚川鹽,而國籓又久治兵,無疆寄,乃與官文合疏請國籓援蜀。會賊竄廣西,上游兵事解,而陳玉成再破廬州,續賓戰歿三河,林翼以群盜蔓廬、壽間,終為楚患,乃改議留國籓合謀皖。軍分三道,各萬人。國籓由宿松、石牌規安慶,多隆阿、鮑超出太湖取桐城,林翼自英山鄉舒、六。多隆阿等既大破賊小池,復太湖、潛山,遂軍桐城。國荃率諸軍圍安慶,與桐城軍相犄角。安慶未及下,而皖南賊陷廣德,襲破杭州。

李秀成大會群賊建平,分道援江寧,江南大營復潰,常州、蘇州相繼失,咸豐十年閏三月也。左宗棠聞而嘆曰:「此勝敗之轉機也!江南諸軍,將蹇兵疲久矣。滌而清之,庶幾後來可藉手乎?」或問:「誰可當者?」林翼曰:「朝廷以江南事付曾公,天下不足平也。」於是天子慎選帥,就加國籓兵部尚書銜,署理兩江總督,旋即真,授欽差大臣。是時江、浙賊氛熾,或請撤安慶圍先所急。國籓曰:「安慶一軍為克金陵張本,不可動也。」遂南渡江,駐祁門。江、浙官紳告急書日數十至,援蘇、援滬、援皖、援鎮江詔書亦疊下。國籓至祁門未數日,賊陷寧國,陷徽州。東南方困兵革,而英吉利復失好,以兵至。僧格林沁敗績天津,文宗狩熱河,國籓聞警,請提兵北上,會和議成,乃止。

其冬,大為賊困,一出祁門東陷婺源;一出祁門西陷景德;一入羊棧嶺攻大營。軍報絕不通,將吏惵然有憂色,固請移營江幹就水師。國籓曰:「無故退軍,兵家所忌。」卒不從,使人間行檄鮑超、張運蘭亟引兵會。身在軍中,意氣自如,時與賓佐酌酒論文。自官京朝,即日記所言行,後履危困無稍間。國籓駐祁門,本資餉江西,及景德失,議者爭言取徽州通浙米。乃自將大軍次休寧,值天雨,八營皆潰,草遺囑寄家,誓死守休寧。適宗棠大破賊樂平,運道通,移駐東流。多隆阿連敗賊桐城,鮑超一軍游擊無定居,林翼復遣將助之。十一年八月,國荃遂克安慶。捷聞,而文宗崩,林翼亦卒。穆宗即位,太后垂簾聽政,加國籓太子少保銜,命節制江蘇、安徽、江西、浙江四省。國籓惶懼,疏辭,不允,朝有大政,咨而後行。

當是時,偽天王洪秀全僭號踞金陵,偽忠王李秀成等犯蘇、滬,偽侍王李世賢等陷浙杭,偽輔王楊輔清等屯寧國,偽康王汪海洋窺江西,偽英王陳玉成屯廬州,捻首苗霈霖出入潁、壽,與玉成合,圖竄山東、河南,眾皆號數十萬。國籓與國荃策進取,國荃曰:「急搗金陵,則寇必以全力護巢穴,而後蘇、杭可圖也。」國籓然之。乃以江寧事付國荃,以浙江事付宗棠,而以江蘇事付李鴻章。鴻章故出國籓門,以編修為幕僚,改道員,至是令從淮上募勇八千,選良將付之,號「淮軍」。同治元年,拜協辦大學士,督諸軍進討。於是國荃有搗金陵之師,鴻章有徵蘇、滬之師,載福、玉麟有肅清下游之師;大江以北,多隆阿有取廬州之師,續宜有援潁州之師;大江以南,鮑超有攻寧國之師,運蘭有防剿徽州之師,宗棠有規復全浙之師:十道並出,皆受成於國籓。

賊之都金陵也,堅築壕壘,餉械足,猝不可拔。疾疫大作,將士死亡山積,幾不能軍。國籓自以德薄,請簡大臣馳赴軍,俾分己責,上優詔慰勉之,謂:「天災流行,豈卿一人之咎?意者朝廷政多缺失,我君臣當勉圖禳救,為民請命。且環顧中外,才力、氣量無逾卿者!時勢艱難,無稍懈也。」國籓讀詔感泣。時洪秀全被圍久,召李秀成蘇州,李世賢浙江,悉眾來援,號六十萬,圍雨花臺軍。國荃拒戰六十四日,解去。三年五月,水師克九洑洲,江寧城合圍。十月,鴻章克蘇州。四年二月,宗棠克杭州。國籓以江寧久不下,請鴻章來會師,未發,國荃攻益急,克之。江寧平,天子褒功,加太子太傅,封一等毅勇侯,賞雙眼翎。開國以來,文臣封侯自是始。朝野稱賀,而國籓功成不居,粥粥如畏。穆宗每簡督撫,輒密詢其人,未敢指缺疏薦,以謂疆臣既專征伐,不當更分黜陟之柄,外重內輕之漸,不可不防。

初,官軍積習深,勝不讓,敗不救。國籓練湘軍,謂必萬眾一心,乃可辦賊,故以忠誠倡天下。其後又謂淮上風氣勁,宜別立一軍。湘勇利山徑,馳騁平原非所長,且用武十年,氣亦稍衰矣,故欲練淮士為湘勇之繼。至是東南大定,裁湘軍,進淮軍,而捻匪事起。

捻匪者,始於山東游民相聚,其後剽掠光、固、潁、亳、淮、徐之間,捻紙燃脂,故謂之「捻」。有眾數十萬,馬數萬,蹂躪數千里,分合不常。捻首四人,曰張總愚、任柱、牛洪、賴文光。自洪寇、苗練嘗糾捻與官軍戰,益悉攻鬥,勝保、袁甲三不能御。僧格林沁征討數年,亦未能大創之。國籓聞僧軍輕騎追賊,一日夜三百餘里,曰:「此於兵法,必蹶上將軍。」未幾而王果戰歿曹州,上聞大驚,詔國籓速赴山東剿捻,節制直隸、山東、河南三省,而鴻章代為總督,廷旨日促出師。國籓上言:「楚軍裁撤殆盡,今調劉松山一軍及劉銘傳淮勇尚不足。當更募徐州勇,以楚軍之規模,開齊、兗之風氣;又增募馬隊及黃河水師,皆非旦夕可就。直隸宜自籌防兵,分守河岸,不宜令河南之兵兼顧河北。僧格林沁嘗周歷五省,臣不能也。如以徐州為老營,則山東之兗、沂、曹、濟,河南之歸、陳,江蘇之淮、徐、海,安徽之廬、鳳、潁、泗,此十三府州責之臣,而以其餘責各督撫。汛地有專屬,則軍務乃漸有歸宿。」又奏:「扼要駐軍臨淮關、周家口、濟寧、徐州,為四鎮。一處有急,三處往援。今賊已成流寇,若賊流而我與之俱流,必致疲於奔命。故臣堅持初議,以有定之兵,制無定之寇,重迎剿,不重尾追。」然督師年餘,捻馳突如故。將士皆謂不苦戰而苦奔逐,乃起張秋抵清江築長墻,憑運河御之,未成而捻竄襄、鄧間,因移而西,修沙河、賈魯河,開壕置守。分地甫定,而捻沖河南汛地,復突而東。時議頗咎國籓計迂闊,然亦無他術可制捻也。

山東、河南民習見僧格林沁戰,皆怪國籓以督兵大臣安坐徐州,謗議盈路。國籓在軍久,益慎用兵。初立駐軍四鎮之議,次設扼守黃運河之策。既數為言路所劾,亦自以防河無效,朝廷方起用國荃,乃奏請鴻章以江督出駐徐州,與魯撫會辦東路;國荃以鄂撫出駐襄陽,與豫撫會辦西路:而自駐周家口策應之。或又劾其驕妄,於是國籓念權位不可久處,益有憂讒畏譏之心矣。匈病假數月,繼請開缺,以散員留軍效力;又請削封爵:皆不許。

五年冬,還任江南,而鴻章代督軍。時牛洪死,張總愚竄陜西,任柱、賴文光竄湖北,自是有東西捻之號。六年,就補大學士,留治所。東捻由河南竄登、萊、青,李鴻章、劉長佑建議合四省兵力堵運河。賊復引而西,越膠、萊、河南入海州。官軍陣斬任柱,賴文光走死揚州。以東捻平,加國籓雲騎尉世職。西捻入陜後,為松山所敗。乘堅冰渡河竄山西,入直隸,犯保定、天津。松山繞出賊前,破之於獻縣。諸帥勤王師大至,賊越運河竄東昌、武定。鴻章移師德州,河水盛漲,扼河以困之。國籓遣黃翼升領水師助剿,大破賊於荏平。張總愚赴水死,而西捻平。凡防河之策,皆國籓本謀也。是年授武英殿大學士,調直隸總督。

國籓為政務持大體,規全勢。其策西事,議先清隴寇而後出關;籌滇、黔,議以蜀、湘二省為根本。皆初立一議,後數年卒如其說。自西人入中國,交涉事日繁。金陵未下,俄、美、英、法皆請以兵助,國籓婉拒之。及廷議購機輪,置船械,則力贊其成,復建議選學童習藝歐洲。每定約章,輒詔問可許不可許,國籓以為爭彼我之虛儀者可許,其奪吾民生計者勿許也。既至直隸,以練兵、飭吏、治河三端為要務,次第興革,設清訟局、禮賢館,政教大行。

九年四月,天津民擊殺法領事豐大業,毀教堂,傷教民數十人。通商大臣崇厚議嚴懲之,民不服。國籓方病目,詔速赴津,乃務持平保和局,殺十七人,又遣戍府縣吏。國籓之初至也,津民謂必反崇厚所為,備兵以抗法。然當是時,海內初定,湘軍已散遣,天津咫尺京畿,民、教相閧,此小事不足啟兵端,而津民爭怨之。平生故舊持高論者,日移書譙讓,省館至毀所署楹帖,而國籓深維中外兵勢強弱,和戰利害,惟自引咎,不一辯也。丁日昌因上奏曰:「自古局外議論,不諒局中艱苦,一唱百和,亦足以熒上聽,撓大計。卒之事勢決裂,國家受無窮之累,而局外不與其禍,反得力持清議之名,臣實痛之!」

國籓既負重謗,疾益劇,乃召鴻章治其獄,逾月事定,如初議。會兩江缺出,遂調補江南,而以鴻章督直隸。江南人聞其至,焚香以迎。以亂後經籍就熸,設官書局印行,校刊皆精審。禮聘名儒為書院山長,其幕府亦極一時之選,江南文化遂比隆盛時。

國籓為人威重,美須髯,目三角有棱。每對客,注視移時不語,見者竦然,退則記其優劣,無或爽者。天性好文,治之終身不厭,有家法而不囿於一師。其論學兼綜漢、宋,以謂先王治世之道,經緯萬端,一貫之以禮。惜秦蕙田五禮通考闕食貨,乃輯補鹽課、海運、錢法、河堤為六卷;又慨古禮殘闕無軍禮,軍禮要自有專篇,如戚敬元所紀者。論者謂國籓所訂營制、營規,其於軍禮庶幾近之。晚年頗以清靜化民,俸入悉以養士。老儒宿學,群歸依之。尤知人,善任使,所成就薦拔者,不可勝數。一見輒品目其材,悉當。時舉先世耕讀之訓,教誡其家。遇將卒僚吏若子弟然,故雖嚴憚之,而樂為之用。居江南久,功德最盛。

同治十三年,薨於位,年六十二。百姓巷哭,繪像祀之。事聞,震悼,輟朝三日。贈太傅,謚文正,祀京師昭忠、賢良祠,各省建立專祠。子紀澤襲爵,官至侍郎,自有傳;紀鴻賜舉人,精算,見疇人傳。

論曰:國籓事功本於學問,善以禮運。公誠之心,尤足格眾。其治軍行政,務求蹈實。凡規畫天下事,久無不驗,世皆稱之,至謂漢之諸葛亮、唐之裴度、明之王守仁,殆無以過,何其盛歟!國籓又嘗取古今聖哲三十三人,畫像贊記,以為師資,其平生志學大端,具見於此。至功成名立,汲汲以薦舉人才為己任,疆臣閫帥,幾遍海內。以人事君,皆能不負所知。嗚呼!中興以來,一人而已。


\end{pinyinscope}