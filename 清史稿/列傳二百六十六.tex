\article{列傳二百六十六}

\begin{pinyinscope}
循吏四

徐臺英牛樹梅何曰愈吳應連劉秉琳陳崇砥夏子齡蕭世本

李炳濤俞澍硃根仁鄒鍾俊王懋勛蒯德模林達泉方大湜陳豪

楊榮緒林啟王仁福硃光第冷鼎亨孫葆田柯劭憼

塗官俊陳文黻李素張楷王仁堪

徐臺英,字佩章,廣東南海人。道光二十一年進士,授湖南華容知縣。俗好訟,臺英謂訟獄糾纏,由於上下不通。與民約,傳到即審結,胥役需索者痛懲之。一日,閱呈詞,不類訟師胥吏筆,鞫之,果諸生也。拘至,試以詩、文,文工而詩劣。諭曰:「詩本性情,汝性情卑鄙,宜其劣。念初犯,姑宥,其改行!」其人感泣去。規復沱江書院,月自課之。曰:「陸清獻作令,日與諸生講學。吾不曉講學,若教人作文,因而誘之讀書立品,是吾志也。」縣田有圻田、埦田、山田之分。瀕湖地,旱少潦多,埦、圻例有蠲緩,田無底冊,影射多。書役墊徵,官給空票。花戶糧數,任其自注。役指為欠者,拘而索之,官不知所徵之數。保戶包納漕米,相沿以為便,挾制浮收,無過問者。積欠數萬,官民交病。臺英知其弊,乃清田冊,注花戶糧數、姓名、住址,立碑埦上,使冊不能改。應緩、應徵者可親勘,而影射之弊絕。申糧隨業轉之例,即時過割,而飛灑之弊絕。收漕分設四局,俾升合小戶,就近輸納,免保戶之加收,而包納之弊絕。埦田舊有堤修費,出田主。有挪埦田作圻田,冀免堤費者;有賣田留稅,派費賠累者;有賣稅留田,派費不至者:堤費不充。一埦堤潰,他埦同希豁免。凡借帑修堤者,久無償,相率亡匿。臺英丈田均費,低窪者許減派,不許匿畝。其人戶俱絕,歸宗祠管業承費。巨族有抗者罪之。行之期年,堤工皆固,逋賦盡輸。

調耒陽。耒陽徵糧,由櫃書裏差收解,取入倍於官。刁健之戶輕。良善之戶重,民積忿。有楊大鵬者,以除害為名,欲揭竿為亂。事平,臺英遂盡革裏差。時上官欲命舉甲長以代裡差,仍主包收包解。臺英以甲長之害,與裏差同。因集鄉紳問之曰:「巡撫命汝等舉甲長,何如?」曰:「無人原充。」臺英曰:「甲長所慮在不知花戶住址,汝等所慮在甲長包收。吾今並戶於村,分村立冊。以各村糧數合一鄉,以四鄉糧數合一縣。各村納糧,就近投櫃,糧入串出,胥吏不得預。甲長祗任催科,無昔日包收之害。此可行否?」眾皆拜曰;「諾。」臺英曰;「隱匿何由核?」眾曰:「取清冊磨對,有漏,補入可耳。」曰:「虛糧何由墊?」曰:「虛糧無幾,有則按畝勻攤可耳。」數月而清冊成,糧法大定。大鵬之亂,誘脅者多。臺英禁告訐,一縣獲安。以憂去官。同治元年,詔起用,發浙江,署臺州知府,未任,卒。

牛樹梅,字雪橋,甘肅通渭人。道光二十一年進士,授四川彰明知縣,以不擾為治。決獄明慎,民隱無不達,咸愛戴之。鄰縣江油匪徒何遠富糾眾劫中壩場,地與彰明之太平場相近。樹梅率民團御之,匪言我不踐彰明一草一木也。迨官軍擊散匪眾,遠富匿下莊白鶴洞,恃險負隅。遙呼曰:「須牛青天來,吾即出。」樹梅至,果自縛出。擢茂州直隸州知州,尋署寧遠知府。地大震,全城陷沒,死傷甚眾。樹梅壓於土,獲生。蜀人謂天留牛青天以勸善。樹梅自咎德薄,不能庇民,益修省。所以賑恤災黎甚厚,民愈戴之。父憂去官。

咸豐三年,尚書徐澤醇薦其樸誠廉幹,詔參陜甘總督舒興阿軍事。八年,湖廣總督官文薦循良第一,發湖北,病未往。同治元年,四川總督駱秉章復薦之,擢授四川按察使,百姓喜相告曰:「牛青天再至矣!」三年,內召,以老病不出,主成都錦江書院。

時甘肅回匪尚熾,樹梅眷念鄉里,遺書當事,論剿回宜用土勇。略云:「軍興以來,劇寇皆南勇所掃蕩。今金積堡既平,河州水土猶惡。若參用本省黑頭勇,其利有六:飽粗糲,耐冰霜,一也;有父母兄弟妻子之仇,有田園廬墓之戀,二也;給南勇半餉,即樂為用,三也;無歸之民,收之,不致散為賊,四也;久戰狄、河一帶,不費操練,五也;地勢熟習,設伏用奇,無意外虞,六也。」後總督左宗棠採其說,主用甘軍,卒收其效。光緒初,歸里,卒,年八十四。

何曰愈,字云亹,廣東香山人。父文明,河南洧川知縣,有惠政。曰愈少隨父宦,讀書勵志,有幹材。道光初,授四川會理州吏目。土司某桀驁,所部夷人殺漢民,知州檄曰愈往驗,以賄乞免,卻之。乃率眾來劫,不為動,卒成驗而還。獄上,大吏廉得直,曰愈由是知名。捐升知縣,以習邊事,辦西藏糧臺,三載,還補岳池縣。不畏強御,豪右斂戢。練鄉團,繕城郭,庀器械。逾數年,滇匪犯岳池,後令賴所遺械以拒賊,時比張孟談之治晉陽雲。調署平山,以母憂去。

咸豐六年,服闋,寧遠府野夷出巢焚掠,大吏檄曰愈參建昌鎮軍事。川西惈夷凡數十支,自雷波、瓘邊,滇南二十四塞,頻年肆擾。值西昌縣告變,曰愈馳至,眾大譁,曰:「夷傷吾人。」曰愈曰:「若等平日欺夷如鹿豕,使無所控告,故釀禍。今且少息,吾為若治之。」乃集兵練出不意搗夷巢,夷皆匍匐聽約束。漢民屋毀粟罄,夷請以山木供屋材,並貸穀為食。曰愈諭民曰:「此見夷人具有天良,若等毋再生釁。」漢、夷遂相安。曰愈既益悉夷、番之情偽,山川之險隘,擬綏邊十二策,格不得上。

未幾,滇匪韓登鸞糾眾入會理州境,聲言與回民尋仇。回民疑漢民召匪,因焚民居。曰愈率一旅往,聞流言奸細伏城內,乃下令毋閉城。三日後,按戶搜查,容奸細者從軍法。越三日,城內外賊黨悉遁。曰愈曰:「吾不閉門、不遽搜者,正開其逃路耳。」眾皆服。遣人持榜文諭登鸞,遵示釋怨退去。復持諭回民,回民曰:「昔日被水災,田廬盡沒。何公一騎渡水賑我,又為我濬河,至今無水患。戴德未忘,今敢不遵諭!違者誅之。」自是回民亦不擾州境。事定,鎮府上其功,會有攘之者,遂不敘。比粵匪犯蜀,曰愈數陳機宜,當事不能用。退居灌縣,後歸,卒於家。子璟,官至閩浙總督。

吳應連,江西南城人。道光元年舉人,以知縣揀發四川。歷署天全、涪州、永川、安岳、蒲江、新津、綿竹、仁壽諸州縣。補石泉,調彭縣。宦蜀先後二十年,所至修塘堰,濬河渠,平治水陸道塗,捕盜賊、土豪,撫災民,皆有實政。咸豐初,蜀匪漸熾,應連在彭縣,編團儲械,以備不虞。四年,卒於官。未幾,悍匪迭來犯,賴鄉勇保全危城,民思遺績,留殯於城內三忠祠旁,歲時祀之。涪州、安岳、永川、石泉、仁壽先後請祀名宦祠。

劉秉琳,字昆圃,湖北黃安人。咸豐二年進士,授順天寶坻知縣。持躬清苦,恤孤寡,懲豪猾,悉去雜派及榷酤贏餘者。索倫兵伐民墓樹,縱馬躪田禾,反誣村民縶其馬,秉琳力爭得直。蝗起,督民自捕,集貲購之,被蝗者得錢以代賑,且免踐田苗。遷宛平京縣。十年,英法聯軍犯京師,秉琳奉檄赴營議犒,納刀鞾中,慮以非禮相加,義不受辱。抗論無少屈,犒具皆如議。尋引疾歸。

穆宗登極,有密薦者,復至直隸,署任丘。民以驛車為累,籌貲招雇,永除其害。擢深州直隸州知州。七年,捻匪張總愚竄畿輔,且至。人勸其眷屬可避,秉琳曰:「吾家人皆食祿者,義不可去。」授兵登陴,鄉民及鄰境聞之,咸挈入保,至十餘萬人。嬰城四十餘日,賊圍之,不破。秉琳上書統帥,言賊入滹沱,河套勢益蹙,宜兜圍急擊,緩將偷渡東竄。卒如其言。寇平,優敘。州地多斥鹵,民以鹽為恆產,課與常賦埒,水旱不得報災,非漉鹽無以應正供。秉琳議官銷法,以杜私販,民悅服。

九年,擢正定知府。滹沱溢,發所儲兵米以賑。築曹馬口、回水、斜角三堤,水不齧城,民用安集。郡與山西接壤,固關守弁,苛稅煤鐵,商販委物於路,聚眾上訴。秉琳往解散,除其重徵。鎮將獲盜三,已誣服,秉琳鞫之,乃兵挾負博嫌,栽贓刑逼,以成其獄,釋三人者而重懲其兵。

光緒元年,擢天津河間道,兼轄南運河工。請復歲修銀額,河兵口食足,乃無偷減工料之弊。築中亭河北堤,涸出腴田千餘頃。時方旱,流民集天津,設粥廠,躬親其事,所活甚眾。嘗太息曰:「哺饑衣寒,救荒末策也。本計當於河渠書、農桑譜中求之。」四年,乞病歸,數年卒。同治初年,軍事漸定,始課吏治。大學士曾國籓為直隸總督,下車即舉賢員,如李文敏、任道鎔、李秉衡,後並至巡撫。

秉琳及陳崇砥、夏子齡、蕭世本諸人,治行皆卓著,當時風氣為之一振云。

崇砥,字亦香,福建侯官人。道光二十五年舉人,咸豐三年,大挑知縣,發直隸,授獻縣。盜賊充斥,嚴緝捕,渠魁多就擒。治鄉團十六區,合千五百人,分班輪值,邑以有備。捻匪張錫珠擾畿輔,崇砥開城納逃亡,誓眾效死。縣境臧家橋為通衢,河間守欲毀橋阻賊,崇砥謂:「方宜安集難民,遙為聲援,豈可夷險示弱?且委東鄉於賊,非計也。」竟不毀橋,賊旋引去。大學士祁俊藻疏薦之,擢保定府同知,筦水利。崇砥以府河港汊紛歧,苦易淤。設水志,增夫役、器具,以時汰淤。商船打壩阻水,為設壩船,給板椿,過淺構橋咸稱便。

同治八年,署大名知府,兵亂時,民多築寨堡自衛,後事定,浸至藏奸抗官。崇砥親履勘,收繳軍械,易正紳司之,澆風漸息。畿南久苦旱,賑難普及,崇砥議有田十畝以上者不賑;極貧,大口錢千,小口半之,壯者不給。先編保甲,造細冊,不曰賑而曰貸。事畢,奏請蠲貸,民安之。南樂縣民抗徭聚眾,令告變。崇砥輕騎往,平其輕重,眾歡然輸納。副將駐兵獻縣,兵不戢,鄉團疑其匪也,戕副將。既而知誤,畏罪,眾聚不散。檄崇砥往治,令縛首禍者,脅從皆免之。

調署順德府,尋擢河間知府。河間素多訟,崇砥剋期審結,數決疑獄,期年而清。滹沱下游為災,崇砥請築古洋河堤,自獻縣至肅寧六十里。於蔡家橋作堤防支流,開溝六千丈,以資宣洩。自馮家村至高旦口,造橋建徬,防子牙河暴漲。於是古洋通流,近地皆大稔。光緒元年,卒於官,祀名宦。

夏子齡,字百初,江蘇江陰人。道光十六年,會試第一,成進士。初官禮部主事,任事果決,尚氣節。庫丁賄當事,請準捐考,力持駁議,時稱之。改授河南汲縣知縣,勤聽訟,嚴治盜,遇事持大體。咸豐初,詔求人才,巡撫潘鐸特薦之,會母憂去官。

服闋,授直隸深澤,調饒陽。比歲旱蝗,盜劫肆擾,選健役百人,教以技擊,更番直。有事,雖午夜立率以出,捕劇盜幾盡。分境內團練為八區,輪期會操,久之皆可用。十年,英法聯軍入京師,畿南土匪蜂起,冀州王洛悅,河間劉四、賈漋等,各麕集千人,連擾郡邑。子齡率團勇迎擊境上,斬獲數百。劉四受創遁,王洛悅聞風驚潰。劉四等尋於他縣被擒伏法,王洛悅亦就撫。事平,優敘。

縣舊為滹沱所經,北徙已久。十一年,上游決溢,水驟至,近郊為澤國。訪尋故道,濬老澗溝,上接安平境,下入獻縣之廉頗窪,以資宣洩。次年,水復至,暢流不為患。城西官道沖刷成河,建長橋五十丈,民便之。遷宛平京縣。

擢易州直隸州知州。西陵在州境,故事,護陵俸餉及祭品、牛羊、芻豆,州領帑給之。陵員與州吏因緣為侵蝕,數煩朝使察治。子齡與守陵大臣議訂章程,弊去泰甚,始相安焉。歲旱,奸民聚眾擾大戶,立杖斃煽眾者。勸捐賑恤,災不為害。

同治六年,河北馬賊起,擾及鄰境,募勇治團如饒陽時,匪懾其名不敢犯。次年,捻匪竄擾畿輔,守要隘,清內匪,防軍久駐,有淫掠者,立斬以徇,闔境肅然。論功,晉秩知府。美利堅教會私購民居為耶蘇堂,執條約與爭。以其無游歷執照,購屋未先告,州境附近陵寢,有關風水,皆與約背,竟退價撤契,且杜其後至。尋請離任,以知府候補。未幾,卒。易州、饒陽並祀名宦祠。子詒鈺,官永年知縣,亦以廉平稱,有治績。

世本,字廉甫,四川富順人。同治二年進士,選庶吉士,散館授刑部主事,改直隸知縣。先在籍治團練有聲,曾國籓蒞直隸,闢為幕僚。九年,天津民、教相閧,斃法國領事,幾肇大釁。遂以世本署天津縣,尋實授。天津民悍好鬥,鍋夥匪動為地方害,世本嚴懲之。地為通商大埠,訟獄殷繁,世本手批口鞫,斷決如神。逾年,父憂去。服闋,仍補天津。歲旱,災

黎就食萬數,給粥、施醫無失所。調清苑,擢遵化直隸州知州,復以母憂去。服闋,以知府候補,筦天津守望局。捕誅大盜王洛八、謝昆,海道肅清。倡修運河堤,以免水患。疏瀦龍河故道,開範家堤及石碑河、宣惠河、金沙嶺下水道四十餘里。皆藉賑興工,民利賴之。署天津、正定兩府。十三年,卒。附祀曾國籓祠。

李炳濤,字秋槎,河南河內人。咸豐中,就職州判,謁曾國籓於軍中,尋佐皖軍營務。能調和將士,積功晉同知,留安徽。同治四年,國籓北征捻匪,炳濤上書言四事:「一,專責防堵,以嚴分竄;一,聯絡民團,以孤賊勢;一,設局開荒,以資解散;一,多備火器,以奪賊長。」國籓頗採其言。檄查亳州圩,炳濤微服出入,盡得諸匪徒姓名及蠹役胡採林通匪虐民狀,誘採林誅之,竿其首,一州驚歡。自是訟獄者咸取決於炳濤。按圩查閱,立條教,別良莠,戮悍賊二百,予自新者三千。期年而俗變,無盜竊者。五年,捻匪竄州境,曉諸圩以大義,雖與寇有親故者,無敢出應,捻匪引去。

六年,署蒙城縣。蒙、亳接壤,瘠苦尤甚。炳濤耡強梗,撫良懦,振興書院,弦誦聲作。捻匪餘黨解散及各軍凱撤還鄉者數千人,彈壓安輯,民用晏然。巡撫英翰疏陳炳濤治行為安徽第一,被詔嘉獎。十年,調署亳州。

尋擢廬州知府。廬州故劇郡,中興以來,元勛宿將相望,豪猾藉倚聲勢為不法,官吏莫敢誰何,炳濤嚴治之,稍戢。無為州江堤,官督民修,炳濤禁胥吏索規費,工必覈實。府東施河口為沖途,冬涸,商船以數牛牽挽始行。時值旱災,以工代賑,濬河深通,運賑者皆至,商民便之。西洋人欲於城內立教堂,成有日矣。炳濤諭地主曰:「爾不聞寧國之變耶?他日民、教有爭端,爾家首禍。」其人懼,事得寢。光緒二年,大江南北訛言有妖術剪人發者,民情洶洶,奸民藉以倡團立卡,多苦行旅。炳濤遍示城邑無妄動,誅一真匪,其疑似者悉不問,人心旋定。三年,母憂去官。皖南興辦保甲墾荒,大吏奏調炳濤主其事。五年,卒於寧國。

炳濤機警,善斷獄。在蒙城,營馬為賊所劫。乃傳諭,詰旦城但啟一門。見有馬奔出,有鞍而無轡,命羈之。俄一人手持一封,將出城,回顧者再,縛之。發其封,則轡與劫物皆在,其人伏罪。在亳州,田父報子夜投井死,驗無傷,井旁有汲水器。炳濤念夜非取水時,既原死,何暇持器。詢其婦,無戚容。偵其平日與鄰婦往來,拘鄰婦鞫之,果得狀。蓋鄰婦弟與婦通,欲害其夫。適其夫以事忤父,鄰婦邀醉以酒而投之井。置汲器者,欲人信其取水投井也,於是皆伏法。

時皖北被兵久,撫輯遺黎,多賴良吏,炳濤為最。又有俞澍、硃根仁、鄒鍾俊、王懋勛,並為時所稱。

澍,直隸天津人。以縣丞發安徽,襄壽春鎮軍事。咸豐六年,署蒙城知縣。時縣城初復,人煙寥落,招集流亡,以大義激紳民,築城籌守御,趨工者踴躍,不費公家一錢。捻渠苗沛霖,反側叵測,窺縣城十餘次,不能破城。有內應賊者,捕斬三人而賊退。七年,攻賊於酆墟,擒其酋徒成德等。八年,攻克龍元賊壘。捻酋孫葵心來犯,出奇計擊走之。附近捻墟,懾於聲威,往往反正受約束。九年,實授。先後敘功,晉同知直隸州。在官數年,潔己愛民。及歿,民皆痛哭,送其柩二千里歸葬。詔贈道銜,建專祠。

根仁,字禮齋,江蘇常熟人。以州判從軍,晉秩知縣,留安徽。同治三年,署定遠。兵燹初定,徵調尚繁。前令試辦開徵,根仁以民不堪命,請緩之。籌備供億,民無所擾。捕巨猾雍秀春未獲,得黨羽名冊,根仁曰:「我何忍興大獄以博能名?喪亂未平,民氣未固,激之生變,可勝誅乎?」遂火其冊,聞者為之改行。跕雞岡周姓聚族居,有從逆者已死,里人利其田廬,致周族人於獄,根仁一訊釋之。後再署定遠,捻匪擾境,根仁修城濬隍,聚糧固守。暇輒輕騎巡鄉,勸民修復陂堰,十家治一井,田二頃闢一塘,旱不為災。歷署阜陽、懷寧,捕阜陽積匪程黑,置之法。補全椒,興水利,有實政。光緒四年,卒。

鍾俊,字雋之,江蘇吳縣人。同治中,以州判官安徽,積勞晉秩知縣,補太平。平反冤獄,慈祥而人不欺。墾荒勸農,蒿萊盡闢,不追呼而賦辦。邑行淮鹽,與浙引接界,屢以緝私釀大獄,乃請以官牒領鹽,試辦分銷,民始安。修復水利,興書院,儲書七萬卷。輯儒先格言,曰人生必讀書。訓士敦本行,旌節孝,修祠祀,舉行賓興鄉飲酒禮。在任五年,以興養立教為務。調太和,歷署懷寧、六安、阜陽、蕪湖、渦陽,所至有聲。光緒中,乞休,卒於家。清貧如故。子嘉來,官至外務部尚書,守其家法焉。

懋勛,字弼丞,湖北松滋人。咸豐中,以議敘縣丞,發安徽,從軍有功,晉知縣。歷署潁上、合肥、亳州、泗州。補六安直隸州知州,因事去職。尋因籌賑捐,獎以知府候補。懋勛先後官安徽近五十年,任亳州、泗州皆三次。初至亳,捻匪苗沛霖初平,清查戶口,收繳軍械,平毀寨堡數百,民始復業。懲械斗,清積案,釐學產,復書院,士民戴之。以父憂去,會巡撫過境,州人萬眾乞留懋勛,巡撫許以俟服闋重任,後如其言,夾道歡迎。光緒初,洊饑,煮粥以賑。河南、山西、陜西饑民流轉入境,留養資遣,全活無算。泗州瀕洪澤湖,為匪藪,捕誅劇盜數十,閭閻得安。治獄無株連,禁差保擾民。勸農事,勵風化,親歷鄉曲,民隱悉達。最後至泗,距前已二十餘年,盜賊聞風遠竄,奸胥皆避歸田野。宣統元年,卒。

蒯德模,字子範,安徽合肥人。咸豐末,以諸生治團練,積功洊保知縣,留江蘇。同治三年,署長洲。時蘇州新復,盜日數發,德模偵之輒獲。有匿鎮將營者,親往擒以歸,置之法。車渡民聚眾抗租,或欲懾以兵。德模曰:「是激之變也。」扁舟往,治首惡,散脅從,事立平。治有天主堂,雍正間鄂爾泰撫蘇,改祠孔子,泰西人伊宗伊以故址請。德模曰:「某官可罷,此祠非若有也。」卒不行。奸人誘買良家女,倚勢豪為庇,德模挈女親屬往出之,豪亦屈服,其不畏強禦類此。常周行鄉陌,田夫走卒相酬答,周知民隱。馭下嚴而恤其私,胥役奉法,不敢為蠹。附郭訟獄故繁,日坐堂皇判決,間用俳語鉤距發摘,豪猾屏息。然執法平,不為覈刻。上官遇疑獄,輒移鞫治,多所平反。治長洲四年,判八百餘牘,盡愜民意,或播歌謠焉。

江北大水,災民麕集,德模請於大吏,分各縣留養,三萬餘人無失所。民有為饑寒偷竊者,設化莠室,給衣食,使習藝,藝成遣歸。為滸墅關營籌芻秣費,永免比閭供役。修望亭塘,為橋二十八,以利行旅。兵祲之後,百廢待舉,壇廟、倉庾、書院、善堂、祠宇及先賢祠墓,率先修復;不足,則斥俸助之。徵漕,舊有淋尖、踢斛、花邊、樣米、捉豬諸色目,又有截串、差追諸弊,一皆革除,不追呼而賦辦。惟大小戶均一,便於民而不便於紳,御史硃鎮以浮收劾奏,事下按治,總督曾國籓、巡撫郭柏廕奏雪之。詔以「是非倒置」切責原奏官。旋署太倉直隸州知州、蘇州知府。

九年,調署鎮江,時天津民擊斃法蘭西領事豐大業,沿江戒嚴。德模至,則葺外城,浚甘露港,召還居民之聞警遠徙者,人心始定。

調署江寧,未幾,擢四川夔州知府。府城濱江,屢圮於水,修築輒不就。德模自出方略,築保坎十三道,甃以方丈大石,層累而上。捐萬金以倡其役,不二年遂成。附郭有臭鹽磧,盛漲則沒水,水落,貧民相聚煎鹽。嗣為雲陽灶戶所持,請封禁,然冬令私煎如故,聚眾抗捕無如何。德模請弛禁,官買其鹽,運銷宜昌。不奪奉節貧民之業,不侵雲陽銷引之岸,遂著為令。勸民種桑,奉節一縣二十二萬株,他邑稱是。在夔四年,卒於官。長洲、太倉、夔州皆祠祀之。

林達泉,字海巖,廣東大埔人。咸豐十一年舉人,江蘇巡撫丁日昌闢佐幕府。留心經濟,每論古今輿圖、武備及海外各國形勢,歷歷如指掌,日昌雅重之。同治三年,粵匪餘孽竄廣東,達泉歸里練鄉勇,籌防禦,大埔得無患。敘績,以知縣選用。七年,隨剿山東捻匪有功,晉直隸州知州,發江蘇。八年,署崇明知縣。亂後彫敝,達泉革陋規,清積獄,修城垣,浚河渠,建橋梁,置義塚,增書院膏火,設同仁育嬰堂。利民之政,知無不為。及去任,父老遮道攀留。其後兵部侍郎彭玉麟巡閱過境,見老者饑踣於道,與之食,曰:「若林公久任於此,吾邑豈有饑人哉?」

十一年,署江陰。城河通江潮,又縣境東橫河關,農田十餘萬畝,灌溉之利,亂後皆淤塞,大浚之。建義倉,勸捐積穀。所定章程,歷久遵守。光緒元年,授海州。達泉先奉檄勘海、沭鹽河,請以工代賑,下車次第舉辦。浚甲子河及玉帶河,復橋路,增堤防,民咸稱便。州地瘠民貧,素為盜藪。達泉時出巡,擒巨憝,置之法。土宜棉,設局教民紡績,廣植桐柏雜樹於郭外錦屏山,所規畫多及久遠。

時方經營臺灣,船政大臣沈葆楨疏薦達泉器識宏遠,潔己愛民,請調署新設之臺北府。格於部議,特詔從之。達泉至,陳治臺諸策。議建置,減徵收,整飭防軍,招民墾荒,皆因地制宜,事事草創,積勞致疾。四年,丁父憂,以毀卒。

方大湜,字菊人,湖南巴陵人。咸豐五年,以諸生從巡撫胡林翼軍中,洊保知縣,授廣濟縣。清保甲,治團練,盜賊屏息。築盤塘石堤,下游數縣皆免水患。十年,土匪何致祥等謀結皖賊,襲攻官軍,大湜偕員外郎閻敬銘馳往擒之。十一年,皖賊竄湖北,黃州、德安諸屬縣先後陷,廣濟亦被擾。大湜被吏議,革職留任。調署襄陽,飛蝗遍野,大湜躡屩持竿,躬率農民撲捕,三日而盡。濬城南襄水故道,渠成,涸復田數萬畝。同治初,巡撫嚴樹森疏陳大湜政績優異,復原職。

八年,擢宜昌知府。九年,大水,難民避高阜,絕食兩日。大湜捐貲煮粥糜,又為餺飥數萬賑之。諭米商招民負米,日致數十石,計口散給,災戶無失所。攝荊宜施道。十年,調武昌。樊口有港蜿蜒九十餘里,外通江,內則重湖環列,周五百里。江水盛漲,由港倒灌,近湖居者苦之。僉請築壩樊口,以禦江水。大湜謂閉樊口則湖水無所洩,環湖數縣受其害,上下江堤亦危,力持不可。光緒五年,再署荊宜施道,尋擢安襄鄖荊道,歷直隸按察使、山西布政使。八年,開缺,另候簡用,遂乞病。為言者所劾,鐫級歸。

大湜生平政績,多在為守令時。所至興學校,課蠶桑,事必親理,胥吏無所容奸,民親而信之。時周歷民間,一吏一擔夫自隨,即田隴間判訟。守武昌時,勘堤過屬縣,暮宿民家,已去而縣官猶不知。嚴義利之辨,嘗曰:「以利誘者,初皆在可取不可取之間。偶一為之,自謂無損,久則顧忌漸忘。自愛者當視為冘毒,饑渴至死,不可入口。」又曰:「居官廉,如婦人貞節,不過婦道一端。若恃貞節,而不孝、不敬、不勤、不慎,豈得謂賢乎?」公暇輒讀書,所著平平言及蠶桑、捕蝗、修堤、區田諸書,皆自道所得。歸田後,謂所親曰:「官至兩司,不如守令之與民親,措置自如也。」遂不出,卒於家。

陳豪,字藍洲,浙江仁和人。同治九年優貢,以知縣發湖北,光緒三年,署房縣。勤於聽訟,每履鄉,恆提榼張幕,憩息荒祠,與隸卒同甘苦。會匪柯三江謀亂,立擒置之法。置匭縣門,諭脅從自首,杖而釋之。徵米斗斛必平,不留難,不挑剔,民大悅,刁紳感而戢訟。禁種鶯粟,募崇陽人教之植茶,咸賴其利。歷署應城、蘄水。

授漢川,頻年襄河溢,修築香花垸、彭公垸、天興垸諸堤,疏濬茶壺溝、縣河口,以工代賑。新溝者,毗漢陽,冬涸舟澀。江口奸民輒恃眾索詐,捕治,諭禁之。因病乞休沐,將去任,有淹訟久未決,慮貽後累,舁胡床至事判定,兩造感泣聽命。值年饑,發賑,大吏知豪得民心,強起,力疾往,民夾道歡呼。賑未半,復以疾去。

尋署隨州,素多盜,豪如治房縣時,置匭令自首。選賢紳,行保甲,盜風頓戢。俗多自戕圖詐,豪遇訟,實究虛坐,不稍徇,澆風革焉。立輔文社,選才雋者親教之,多所成就。治隨二年,瀕行,聞代者好殺,竭數晝夜之力,凡獄情可原者,悉與判決免死。後因養母,乞免,歸。浙中大吏輒諮要政,多所匡益。家居十餘年,卒。豪在隨州,重修季梁祠。及卒,隨人思其德,於西偏為建遺愛祠祀之。

楊榮緒,字黼香,廣東番禺人。咸豐三年進士,選庶吉士,授編修,擢御史。英法聯軍犯京師,駕幸熱河,榮緒與同官抗疏請回鑾,又劾參贊國瑞★法營私,風裁頗著。

同治二年,出為浙江湖州知府。粵匪據湖州四年,時甫克復,荒墟白骨,闃無人煙。榮緒置善後局,規畫庶政,安集流亡,閭閻漸復。屬縣糧冊無存,榮緒招來墾闢,試辦開徵,歲有起色。湖蠶利甲天下,經亂,桑盡伐,課民復種,貧者給以桑苗,絲業復興。

郡稱澤國,匯天目諸山之水入太湖,烏程、長興境內舊有漊港,各三十六,以為宣洩,亂後多淤塞。五年,榮緒奉檄開濬,至八年粗畢,烏程漊港尤易淤,賴設閘以禦湖水之倒灌。九年,重修諸閘,因經費不充,頻年經營,猶未盡也。十年,內閣侍讀學士鍾佩賢疏陳其事,朝命大加濬治,時榮緒舉卓異入覲,宗源瀚代攝郡,源瀚亦能事,規畫舉工。及榮緒回任,集絲捐,得鉅款,以資興作。屏去傔從,輕舟巡驗,常駐湖濱,逾年工始竣。以漊港旋開旋淤,議定分年疏濬之法及鏟蘆、撈淺、閘版啟閉章程,數十年遵守不輟。又開碧浪湖,疏北塘河及城河。葺學校,建考舍,修書院,建倉庫,造橋梁,復育嬰堂,百廢具舉。

鞫獄詳審,吏胥立侍相更代,終日無倦容。親受訟牒,指其虛謬,曰:「勿為胥吏所用也。」手書牒尾,輒數百言,剖析曲直,人咸服之。訟以日稀,刑具朽敝。隸役坐府門,賣瓜果自活。客坐無供張,儉素如布衣時,遠近頌為賢守。在任十年,嗣為人所譖,遂求去。捐升道員,離任。尋卒。郡人思之,請祀名宦祠。

林啟,字迪臣,福建侯官人。光緒二年進士,選庶吉士,授編修。督陜西學政,馭士嚴正。任滿,遷御史,直言敢諫,稽察祿米倉,不受陋規,為時所稱。十九年,出為浙江衢州知府,多惠政。二十二年,調杭州,除衙蠹,通民隱,禁無名苛稅。餘杭巨猾楊乃武,因奸通民婦葛畢氏,興大獄。刑部訊治,幸免重罪。歸則益橫,攬訟事,挾制官吏,莫敢誰何。啟捕治之,乃武控京師,不為動,卒論如法。尤以興學為急務,時各行省學堂猶未普立,杭郡甫建求是書院,啟復養正書塾,並課新學。舊有東城講舍,益振興之。兼經義、治事,陰主程、硃之說,而變其面目。誘諸生研尋義理,以成有用,一時優秀之士皆歸之。又以浙中蠶業甲天下,設蠶學館於西湖,講求新法,成效頗著。遇國外交涉事,持正無遷就,遠人亦心服。治杭四年,剛直不阿,喜接布衣,士民翕然頌之。卒官,葬於孤山林處士墓側,杭人歲設祭焉,號曰林社,久而勿輟。啟之治杭,得友高鳳岐為之助,後官廣西梧州知府,亦有聲。歿而杭人附祀於林社云。

王仁福,字竹林,江蘇吳縣人。少誠愨,勇於任事。祖宦河南,歿後,仁福扶柩歸葬。道經徐州,遇捻匪,徒步率廝役出入烽火,肩行四十里,竟免。尋入貲為東河同知。粵匪犯開封,城壕沙淤如平地,仁福奉檄督工濬治,剋期蕆事而賊至,城守賴之。同治五年,署祥河同知。黃河自北徙,中原多故,工帑大減。頻年軍事亟,發帑復不以時。歲修不敷,堤埽殘缺,料無宿儲。祥河汛地當沖,險工迭出,人皆視為畏途。仁福盡力修守,不避艱危。六年秋,汛水驟漲,掣埽去如削木★H9。仁福奔走風雨泥淖中,搶護歷七晝夜。款料俱竭,堤岌岌將破。居民蟻附堤上,仁福對之流涕,曰:「我為河官,擠汝等於死,我之罪也,當身先之!」躍立埽巔。風浪卷埽,走入大溜沉沒。河聲如吼,堤前水陡落。風止浪定,大溜改趨,殘堤得保。眾咸驚為精誠所格,令善泅者覓其尸,不得,乃以衣冠斂。事聞,詔依陣亡例賜恤,附祀河神祠。

硃光第,字杏簪,浙江歸安人。少孤貧,幕游江南,奉汪輝祖佐治藥言為圭臬。咸豐末,捻匪方熾,佐蕭縣令籌防禦,屢破賊。都統伊興額上其功,累晉秩知州,分發河南,佐讞局,治獄平。光緒中,補鄧州。在任三年,大祲之後,壹意休養。善治盜,民戴之。王樹汶者,鄧人,為鎮平盜魁胡體安執爨。鎮平令捕體安急,乃賄役以樹汶偽冒,致之獄。既定讞,臨刑呼冤。重鞫,則檄光第逮其父季福為驗。開歸陳許道任愷先守南陽,嘗讞是獄,馳書阻毋逮季福。且誘怵之。光第曰:「吾安能惜此官以陷無辜?」竟以季福上,則樹汶果其子。巡撫李鶴年袒愷,持初讞益堅。河南官科道者,交章論其事。命東河總督梅啟照覆訊,樹汶猶不得直,眾論大譁。刑部提鞫,乃得實。釋樹汶,自鶴年、啟照以次譴黜有差,而光第已先為鶴年摭他事劾去官,貧不能歸,卒於河南。後鄧州士民請祀名宦,以子祖謀官禮部侍郎,格於例,不行。

冷鼎亨,字鎮雄,山東招遠人。同治四年進士,即用知縣,發江西,署瑞昌。地瘠而健訟,鄉愚輒因之破家。捕訟師及猾吏數人,繩以法。因事詣鄉,使胥役盡隨輿後,返則令居前而己殿之,未嘗以杯勺累民。調署德化,懲防軍之陵民者,境內肅然。修瀕江堤塘,費省工速。德化、瑞昌、黃梅三邑民爭蘆洲,累歲相鬥殺。鼎亨諭解之,建臺於斗所,官吏誓不私,民皆悅服。白鶴鄉人叔與侄爭田,即樹下諭解,遂悔悟如初。旱,蝗起,徒步烈日中,掩捕經月,露宿禱神,得雨,蝗皆死。歷署新昌、彭澤,皆有實政。

上官以為賢,調補新建。附省首邑,官斯者多昕夕伺上官,不遑治民事。鼎亨先與上官約,屏酬應,親聽斷,民歌頌之。尋調鄱陽,值大水,發賑親勘給印票,盡除侵蝕舊習。次年,復災,跣足立沮洳中,濕疾遍體,十閱月。常小舟行駭浪中,屢瀕於危,深夜返署理訟牘。侍郎彭玉麟巡江過境,寄書巡撫曰:「某所至三江五湖數千里,未見堅剛耐苦如冷知縣者也。」

歷官十年,食無兼味,妻子衣履皆自制。以廉率下,胥吏幾無以為生。俸入輒捐為地方興利,訓士以氣節為先。鄱陽俗好鬥,鼎亨曰:「化民有本,未教而殺之,非義也。」以孝經證聖祖聖諭廣訓為淺說,婦孺聞之皆感動。治教案必持平,屢遇民、教齟事,桀黠者欲借以鼓眾毀教堂,慮遺禍好官而止,蓋有以感之。光緒十年,擢南昌府同知,巡撫潘霨疏薦入覲,遂乞歸,卒於家。

孫葆田,字佩南,山東榮成人。同治十三年進士,授刑部主事,改知縣,銓授安徽宿松。勤政愛民,日坐堂皇,妻紡績,室中蕭然如寒士。調合肥,大學士李鴻章弟子之傔人橫於鄉,以逼債毆人死。葆田檢驗尸傷,觀者數萬人,恐縣令為豪強迫脅驗不實。葆田命仵作曰:「敢欺罔者論如律。」得致命狀,人皆歡噪,謂包龍圖復出,讞遂定。有御史劾葆田誤入人死罪,詔巡撫陳彞按之,卒直原讞。葆田遂自免歸,名聞天下。逾數年,安徽將清丈民田,巡撫福潤疏調葆田主其事,辭不赴。貽書當事,言清丈病民,陳:「清賦之要,熟地報荒者,當寬其既往,限年墾復。平歲報災者,當警其將來,分年帶徵。弊自可除,無事紛擾。」時以為名言。

葆田故從武昌張裕釗受古文法,治經,實事求是,不薄宋儒。歷主山東、河南書院,學者奉為大師。巡撫張曜疏陳其學行,賜五品卿銜。中外大臣迭薦之,詔徵,不出。宣統元年,卒,年七十。

柯劭憼,字敬儒,山東膠州人。光緒十五年進士,即用知縣。亦官安徽,署貴池,補太湖。貴池自粵匪亂後,地丁冊為吏所匿,託言已毀。徵賦由吏包納,十不及四五,而浮收日甚,民苦之。劭憼知其弊,令花戶自封投櫃,吏百計撓之,不為動。民輸將恐後,增收銀二萬餘兩,民所節省數且倍。巡撫鄧華熙初聽浮言將奏劾,總督劉坤一曰:「柯令,皖中循吏,奈何登於彈章?」華熙悟,遂疏薦送覲,晉秩直隸州。劭憼為治清簡,斷獄明決,所至民愛戴。亦績學,善為古今體詩。時與葆田並稱儒吏。

塗官俊,字劭卿,江西東鄉人。光緒二年進士,截取知縣,發陜西,署富平、涇陽、長安諸縣。補宜君,山邑地瘠民樸,官此者多不事事。官俊勸農桑,興水利,成稻田數百畝。躬巡阡陌,與民絮語如家人。調涇陽,歷官皆有聲。凡兩任涇陽,政績尤著。初至,值回亂後,清積訟千餘,庶政以次規復,期年而改觀。龍洞渠,故白渠也,官俊倡言開濬,眾議以工鉅為難,獨毅然為之。由梯子關而下,水量增三分之一,復於清冶河畔修復廢渠二,水所不至者,勸民鑿井以濟之。先後增井五百有餘,無旱憂。

涇民多逐末,不重蓋藏,義倉無實儲。官俊謂積穀備荒,莫善於年出年收。躬詣各鄉勸諭捐穀,嚴定收放之法,民感其誠,輸納恐後,倉皆充實。十九年,旱荒,全活凡數萬人。編保甲,捕盜賊,地方靖謐。官俊故績學,立賓興堂,置性理、經濟有用之書,日與諸生講習。增義塾,定課程,親考校之。凡有利於民者,為之無不力。二十年,卒。疾篤時,猶強起治事,捐俸千金以恤孤貧。民為祠,歲時祀之。

陳文黻,湖南長沙人。以諸生入貲為通判。同治間,從軍,積功晉同知,留陜西。光緒七年,署鄠縣知縣,以教化為先,政平訟理。九年,授留壩同知。獄舊有棗茨,經費歲徵之民,文黻革之。境內無質庫,貧民稱貸,盤剝者要重息。文黻設裕民公所,貸民錢,息以十一,取其贏以備公用,民便之。境山多於田,無物產以資生。乃周歷山谷,辨其土宜,作種橡說及山蠶四要,遍諭鄉民。頒給樹秧蠶種,募工導之。絲成,制機教織,設局收買,重其值以招之。又購紫陽茶種,課之樹藝,於是地無棄利。俗素樸陋,歲科試附鳳縣額,每試或不得一人。建書院、義塾,置書籍,延高才者為之師。數年之後,橫舍彬彬,遂請奏設學,建官置額。

谿河多壅閼,橫溢為患。陳開河策,未果行,值水猝發,已逾報災例限,便宜開倉賑之。跋涉沮洳,勞疾不輟。煮粥賑近郊,多所全活。久之,流民坌集,復申開河議,以工代賑,不得請。則因其眾治道路,濬溝渠,出私錢給值,負累至數千緡,民感其德。介萬山中,林谷深阻,奸民狙伏行劫,或掠婦孺賣境外。文黻密圖其處示捕役,時復微服跡之,多就擒治。實行保甲,於民戶職業、田產、丁口、年歲、婚嫁,載冊不厭煩瑣。及賑饑,稽之冊,如家至戶覿,訴訟亦莫敢欺,事益簡焉。民有殺子婦匿其尸者,母家以無左驗,不得直。文黻偶行山徑,群鴉噪於前,索而得之,一訊具服,人以為神。十八年,調署潼關,未任,卒。

李素,字少白,雲南保山人。同治六年舉人。光緒初,授陜西商州直隸州知州。值州境歉收,饑民聚掠。時山西大祲,商州為轉運要沖。素招民運賑糧,使饑者得食。集貲數萬緡,購籽糧散給。設粥廠十餘所,災後倉儲一空,捐穀萬石。六年,大水,加意撫恤,災不為害。州城濱丹河,遇盛漲則負郭田廬漂沒,城中亦半為澤國。素創築石堤二百餘丈,城門月堤十餘丈,遂無水患。開州東隸花河山路三十餘里、州西麻蒦嶺山路二十餘里,行旅便之。擴充商山書院,延碩儒課士,設義塾三十餘區,弦誦聞於比戶。陋規病民者悉除之。每歲寒冬,出私錢給孤寡。緝捕籌經常之費。綠營餉薄,歲資助之。凡賑饑、積穀、築堤、修城、興學,莫不以鉅貲倡。一署同州知府。先後在官十八年,兩舉卓異。以病免歸,卒。士民感之,多私祠祀焉。

張楷,字仲模,湖北蘄水人。同治十年進士,選庶吉士,授編修,累遷至侍講。光緒初,疏論伊犁事,又請撤銷總兵周全有恤典,為時所稱。八年,出為浙江金華知府。永康山中七堡、八堡,地險僻,盜藪也。楷設方略,捕誅匪首蔣元地,移縣丞駐山麓,獷俗一變。父憂去,服闋,補山西汾州。汾陽、平遙兩縣瀕河,鄉民冬令攔河築堰,引水灌田,水不得暢流。夏秋漲溢,各築護堤。以鄰為壑,輒械斗蔓訟。楷禁築攔河堰,濬引渠以洩水,患紓而訟息。以南方戽水法導民,使開稻田,植桑課蠶。有山曰黑煙,與交山葫蘆峪相連,匪徒窟穴其間,偵其姓名,掩捕盡獲之。治汾州七年,考績為山西最。調太原,未任,母憂去。服闋,補河南府。鞏、洛之間素多盜,捕治巨魁,椎埋斂跡。治獄多平反。調開封。二十五年,畿輔拳匪亂起,大河南北,群情洶洶,大吏持重不敢決。楷力陳邪教不可信,外釁不可開。揭示:「義和團既號義民,謂能避槍砲。令詣城外空營候試,以槍擊果不入,編伍充兵。」奸民不得逞。聯軍入都,潰兵南下,楷創議守河。自汜水迄蘭儀,嚴稽渡口,凡持械之士,悉阻之不令入城,屬境安堵。論者謂微楷之堅定,中原禍未艾也。事定,開缺,以道員候補。三十年,卒。

王仁堪,字可莊,福建閩縣人,尚書慶雲之孫。光緒三年一甲一名進士,授修撰。督山西學政,歷典貴州、江南、廣東鄉試,入直上書房。時俄羅斯索伊犁,使臣崇厚擅定條約,仁堪與修撰曹鴻勛等合疏劾之。太和門災,復與鴻勛應詔陳言,極論時政。其請罷頤和園工程,謂:「工費指明不動正款,夫出之筦庫,何非小民膏血?計臣可執未動正款之說以告朝廷,朝廷何能執未動正款之說以謝天下?」言尤切直。

十七年,出為江蘇鎮江知府。甫下車,丹陽教案起,由於教堂發見孩尸。仁堪親驗孩尸七十餘具,陳於總督劉坤一曰:「名為天主教堂,不應有死孩骨。即兼育嬰局,不應無活嬰兒。傳教約本無準外國人育嬰之條,教士於約外兼辦育嬰,不遵奏行章程,使地方官得司稽察,禍由自召。請曲貸愚民之罪,以安眾心;別給撫恤之費,以贍彼族。」坤一迂之,卒定犯罪軍流有差。時外使屢責保護教堂,仁堪請奏定專律,謂:「條約無若何懲辦明文,每出一事,任意要挾。宜明定焚毀教堂,作何賠償;殺傷教士,作何論抵;以及口角鬥毆等事,有定律可遵。人心既平,訛言自息。」英人梅生為匪首李鴻購軍火,事覺,領事坐梅生罪僅監禁,仁堪上書總理各國事務衙門論之。又洋人忻愛珩遍謁守令,募捐義學,無游歷護照。仁堪請關道送領事查辦,復議無照私入內地,應按中國律法科罪。雖皆未果行,時論韙之。

郡地多岡壟,旱易成災,仁堪以設渠塘為急務,不欲擾民,捐廉為倡。馳書乞諸親舊,商富感而輸助,得錢三萬緡,開塘二千三百有奇,溝渠閘壩以百計。

十八年秋,丹陽大祲,恩賑之外,勸紳商捐貲,全活甚眾。又假官錢於民,使勿賣牛,名曰牛賑。濬太平港、沙腰河、練湖、越瀆、蕭河、香草、簡瀆之屬,凡二十餘所,支溝別渠二百三十有奇。又鑿塘四千六百,以蓄高原之水。皆以工代賑,東西百餘里間,水利畢舉。次年春,賑畢,餘四萬金,生息備積穀。牛賑餘錢,仿社倉法創社錢,按區分儲,為修溝洫、廣義塾之用。郡西鄉僻陋不知學,立榛思文社以教之。出私錢於府治前建南畾學舍。在任兩年,於教養諸端,盡力為之。

調蘇州,已積勞致疾,日坐讞局清積案,風採動一時。甫三閱月,猝病卒,時論惜之。鎮江士民列政績,籥請大吏上聞,謂其「視民事如家事,一以扶植善類、培養元氣為任,卓然有古循吏風」。詔允宣付史館立傳,以表循良。自光緒初定制,官吏歿後三十年,始得請祀名宦。於是疆臣率徇眾意,輒請宣付立傳表章,曠典日致猥濫,仁堪為不愧雲。


\end{pinyinscope}