\article{列傳六十一}

\begin{pinyinscope}
楊雍建姚締虞硃弘祚子綱王騭宋犖陳詵

楊雍建,字自西,浙江海寧人。順治十二年進士,授廣東高耍知縣。時方用兵,總督駐高耍。師行徵民夫,吏慮其逃,縶之官廨。當除夕,雍建命徙廊廡,撤餚饌畀之。師中索榕樹枝制繩以燃砲,軍吏檄徵,語不遜,雍建笞之。總督王國光以是稱雍建方剛,特疏薦。蒞官甫一年,擢兵科給事中。

十六年春,世祖幸南苑,雍建疏言:「昨因聖體違和,傳諭孟春饗太廟,遣官致祭。至期皇躬康豫,仍親廟祀,此敬修祀典之盛心也。乃回宮未幾,復幸南苑,寒威未釋,陟歷郊原,恐不足以慎起居。且古者蒐苗獮狩,各有其時。設使獸起於前,馬逸於後,驚屬車之清塵,豈能無萬一之慮?」疏入,上甚怒,宣雍建入,諭以閱兵習武之意。雍建奏對不失常度,上意亦解。

時平南王尚可喜、靖南王耿繼茂並鎮廣東,雍建疏陳廣東害民之政八:委吏太濫,雜派太繁,里役無定例,用夫無定制,鹽埠日橫,私稅日盈,伐薪採木,大肆流毒,均宜亟為革除。且兩籓並建,供億維繁。今川、貴底定,請移一籓鎮撫其地,俾粵民甦息。上尋命繼茂移鎮福建,雍建發之也。十七年,疏言:「朋黨之患,釀於草野。欲塞其源,宜嚴禁盟社,請飭學臣查禁。」從之。轉吏科給事中。聖祖即位,輔臣秉政,奏事者入見,皆長跪,雍建獨立語。比退,輔臣目之曰:「此南苑上書諫獵者也。」自是奏事者見輔臣皆不跪。

康熙三年,彗星見。雍建奏言:「天心仁愛,垂象示警。乞齋心修省,廣求直言,詳詢利病,並飭內外臣工,滌慮洗心,共修職業,」上優旨褒答。四年,疏言:「治化未醇,由於臣職未盡。比者部臣以推諉為卸責,明為本部應議之事,或請咨別部,或請飭督撫,致一案之處分,因一人之口供未到而更待另議;一事之行止,因一時文卷小誤而重俟行查;至地方利弊所關,憚於釐正,輒云已經題定,無庸再議。如此,則一二胥吏執定例以駁之足矣,不知滿、漢堂司各官所司為何事也。督撫以蒙蔽為茍安,民苦於差徭,而額外之私徵,未聞建長策以除積困;吏橫於貪暴,而有司之掊克,不過摘薄罪以引輕條。向日行考滿之法,則題報者皆稱職,曾無三等以下之劣員;平時上彈劾之章,則特糾者僅末僚,不及道府以上之大吏。凡此推諉蒙蔽之習,請嚴飭內外臣工各圖報稱,儻仍蹈故轍,立予罷斥,以儆官常。」疏入,報聞。尋自刑科都給事中累擢左副都御史。

十八年,典會試,授貴州巡撫。疏請立營制,減徭役,招集流亡,禁革私派。土司謁巡撫,故事,必鳴鼓角,交戟於門,俾拜其下。雍建悉屏去,引至座前問疾苦,予以酒食,土司咸輸服。始,貴陽斗米值錢五千,雍建請轉餉以給。既,令民翦荒茅,教以耕種。比三年,稻田日闢,民食以裕。二十三年,召授兵部侍郎。尋以親老乞終養,許之。四十三年,卒,賜祭葬。子中訥,進士,官右中允。

姚締虞,字歷升,湖廣黃陂人。順治十五年進士,授四川成都府推官。四川殘民多聚為盜,互告訐,釀大獄。締虞平恕讞鞫,輒得其情,審釋叛案株連獄囚十七人。總督苗澄、巡撫張德地薦廉能,舉卓異,會裁缺,改陜西安化知縣。行取,康熙十五年,授禮科給事中。疏請嚴選庶吉士,考覈翰林,報聞。十七年,典試江西,還,奏:「江西被賊殘破州縣在丁缺田荒案內者,請敕督撫酌量輕重,限三年或五年勸墾,以漸升科。全省逋賦二百二十萬,歷年追比,僅報完三萬。此二百十餘萬,雖敲骨吸髓,勢必不能復完。請早予蠲免,俾小民得免死亡。」

十八年,地震,求言。締虞上疏曰:「科道乃朝廷耳目之官,原期知無不言,有聞則告。自故憲臣艾元徵請禁風聞條奏,自此言路氣靡,中外多所顧忌。臣請皇上省覽世祖朝諸臣奏議,如何謇諤;今者相率以條陳為事,輭熟成風。蓋平時無以作其敢言之氣,一旦欲其慷慨直陳,難矣。乞敕廷臣會議,嗣後有矢志忠誠、指斥奸佞者,即少差謬,亦賜矜全。如或快意恩仇,受人指使,章奏鈔傳,眾目難掩,縱令彈劾得實,亦難免於徇私之罪。如此,則言官有所顧忌,不敢妄言;中外諸臣有所顧忌,不敢妄為。」疏下九卿科道會議。越日,召廷臣等集中左門,上問:「締虞疏如何定議?」吏部尚書郝惟訥等暨給事中李宗孔等俱言風聞之例,不宜復開。上問:「締虞,爾意如何?」締虞對曰:「皇上明聖,從未譴罪言官。但有處分條例在,言官皆生畏懼。」上曰:「如汝言,條例便當廢耶?」締虞對曰:「科條雖設,當辨公私誠偽。」上意稍解。諭言:「官宜敷陳國家大事,如有大奸大貪,糾劾得實,法在必行,決不姑貸。且魏象樞彈奏程汝璞,亦是風聞,已鞫問得實,原未嘗有風聞之禁也。」上宣締虞前,指內閣所呈世祖時章奏示之曰:「汝以朕為未閱此乎?」締虞對曰:「惟久經聖覽,臣故不憚盡言。」上命以所言宣付史館。次日,復命締虞入起居注,授筆札記之。尋轉工科掌印給事中。上考察科道,黜孫緒極、傅廷俊、和鹽鼎三人,而嘉締虞與王曰溫、李迥稱職。二十一年,疏論外吏積習,視事偷惰,公務沉閣,文移遲緩;僚屬宴會,游客酬酢,廢時糜費。請敕部禁飭。累擢左僉都御史。

二十四年,授四川巡撫。締虞先為推官有聲,百姓喜其來。締虞至,榜上諭於事,嚴約束,禁私徵雜派,杜絕餽遺,屬吏憚之。疏言:「四川迭經兵火,荒殘已極。官戶鄉紳,多流寓外省,雖令子弟復業,迨入學鄉舉登仕版後,仍棄本籍他往。百姓見其如此,亦裹足不歸。若招回鄉宦一家,可抵百姓數戶。紳宦既歸,百姓亦不招而自至。今察明各屬流寓外省紳衿,請敕部移行,飭令復業。」從之。蜀人困於採木,締虞陛辭,首陳其害。會松威道王騭入覲,亦舉是以奏,詔特免之。復請免運白蠟,停解鐵稅,皆獲施行。二十七年,卒官,賜祭葬。

硃弘祚,字徽廕,山東高唐人,昌祚弟。弘祚自舉人授江南盱眙知縣,有惠政,舉卓異。康熙十四年,行取御史,以昌祚子紱官大理寺卿回避,改刑部主事。再遷兵部督捕郎中,出為直隸天津道僉事,調直隸守道參議。

二十六年,超擢廣東巡撫。入見,奏對稱旨,賜帑金千,及內廝鞍馬。過庾嶺,察知夫役苦累,首禁革之。復牒兵部,凡使者過境,有驛站供億,不得更有所役。廣東軍興後,無藝之徵,浮於正供,悉罷免。劾墨吏尤者數人,餘悉奉法。鹽法為籓下奸民所亂,據引地莫敢譙訶。弘祚疏陳整飭鹽政數事,如議行。

高州屬縣吳川,瓊州屬縣臨高、澄邁,戶少田蕪,積逋十二萬兩有奇,疏請豁免。衛所屯田歲輸糧三斗,額重多逃亡。弘祚言:「民糧重,則每畝八升八合起科,今屯田浮三之二,非恤兵之道,當比例裁減。」事皆允行。逆亂方定,奸民告訐無已,疏請嚴妄首株連之例,略謂:「當定南分鎮,聞風投冒倚藉聲勢者,實繁有徒,迨經平定,籓下人應歸旗者,悉已簿錄解京;籍內無名者,釋放為民。嗣有旨:『籓下官兵、奴僕及貿易人等,除實系遼東舊人及價買人外,逐一清查,發出為民。』臣尋繹詔意,原以諸人皆朝廷赤子,不忍株累。且十餘年來,或補伍,或歸農,或死亡遷徙,無籍可稽。乃奸宄之徒,蔓引株連,或在部呈首,或向有司告訐;及事白省釋,而官民之被累已深。請敕部嚴議。」從之。

三十一年,擢福建浙江總督。值大計,弘祚疏言「福建地瘠民佻」,上責弘祚失言,謂:「賢才不擇地而生。四川巡撫張德地署延綏巡撫,言『延綏邊地,無可舉博學鴻詞者』;少詹事邵遠平奏『南方人輕浮不可用』。朕心甚不愜,因皆罷斥。今弘祚又以謬言陳奏,下部議降調。」三十九年,命修高家堰河工,病卒。

子絳,官至廣東布政使;綱,初授兵部主事,累官湖南布政使,雍正間,擢雲南巡撫,疏劾署巡撫楊名時徇隱廢弛,籓庫借支未清款項至十九萬有奇,名時坐是得罪。尋調撫福建,卒,謚勤恪。

王騭,字辰岳,山東福山人。順治十二年進士,授戶部主事。康熙五年,典試廣東。歷刑部郎中。十九年,出為四川松威道。時征雲南,騭督運軍糧,覆舟墜馬,屢經險阻,師賴以濟。二十四年,壘溪大定堡山後生番出掠,巡撫韓世琦檄兵追剿,令騭駐茂州,與總兵高鼎議剿撫。騭赴堡開諭,番族據巴豬寨,陽就撫,負嵎如故。騭招撫附近諸寨,遣兵自廟山進,圍寨,斬獲無算。追至黑水江,賊渠挖子被焚死,山後番眾悉降。調直隸口北道,未行。

時以太和殿工,命採蜀中柟木。騭入覲,疏言:「四川大半環山巉巖,惟成都稍平衍。巨材所生,必於深林窮壑,人跡罕到,斧斤難施,所以久存。民夫入山採木,足胝履穿,攀藤側立,施工既難;而運路自山抵江,或百餘里,或七八十里,深澗急灘,溪流紆折,經時歷月,始至其地。木在溪間,必待暴水而出,故陸運必於春冬,水運必於夏秋,非可一徑而行,計日而至,其艱如此。且四川禍變相踵,荒煙百里。臣當年運糧行間,滿目瘡痍。自蕩平以後,休養生息。然計通省戶口,仍不過一萬八千餘丁,不及他省一縣之眾。就中抽撥五千入山採木,衣糧器具,盈千累百,遣發民夫,遠至千里,近亦數百里,耕作全廢,國賦何徵?請敕下撫臣,親詣採柟處察勘,量材取用,其必不能採運者,奏請上裁。」疏入,上諭曰:「四川屢經兵火,困苦已極,採木累民。塞外松木,取充殿材,足支數百年,何必柟木?令免採運。」未幾,吏部循例疏請司道內擢京堂,騭未與,特命內升。尋授光祿寺少卿,累遷太常寺卿。

二十六年,授江西巡撫。陛辭,上諭曰:「大吏以操守為耍,大法則小廉,百姓蒙福。」騭對曰:「臣向在四川,不取民間粒米束草,日費取給於家。」上曰:「身為大臣,日費必取給於家,勢有所不能。但操守廉潔,念念愛民,便為良吏,且亦須安靜。貪污屬吏,先當訓誡;不悛,則糾劾。」瀕行,賜帑金千。二十七年,擢閩浙總督。疏言:「江西自蕩平後,積年蠲免銀米二百萬有奇,民生漸裕。然徵收之弊,尚為民累,錢糧明加火耗,暗加重戥,部院司道府皆有解費。臣入境之初,火耗已減,解費尚存,即揭示剔除積弊,盡革官役上下大小雜費。南昌、新建二縣漕糧尚仍民兌,俱行革除,漕運積年陋規,搜剔無遺。但在民則省費,在官則失利。恐臣去後,空言無用,乞天語嚴禁,不致前弊復生。」下所司知之。

時湖廣叛卒夏逢龍據武昌,陷黃州。騭次邵武,聞警,恐蔓及江西,奏撥福建兵協剿。自海禁既弛,奸民雜入商販,出洋劫掠。騭既上官,即檄溫州總兵蔣懋勛、黃巖總兵林本植、定海總兵董大本以舟師出洋搜捕。懋勛、本植得賊舟七、大本於白沙灣獲巨艦一,斬盜渠楊仕玉等十六輩,釋被擄難民百十一人。二十八年,上幸浙江,賜騭御用冠服。諭曰:「爾任總督,實心任事,浙、閩黎庶稱爾清廉,故特加優賚。」未幾,召拜戶部尚書,以老病累疏乞休,詔輒慰留。

三十三年,召大學士、九卿及河督於成龍入對,上責成龍排陷靳輔,並及騭與左都御史董訥、內閣學士李應薦附和成龍,騭等具疏引罪,訥、應薦並奪官,騭原品休致。三十四年,卒於家,賜祭葬。

宋犖,字牧仲,河南商丘人,權子。順治四年,犖年十四,應詔以大臣子列侍衛。逾歲,試授通判。康熙三年,授湖廣黃州通判。以母憂去。十六年,授理籓院院判,遷刑部員外郎,榷贛關,還遷郎中。二十二年,授直隸通永道。二十六年,遷山東按察使。再遷江蘇布政使,察司庫虧三十六萬有奇,犖揭報督撫,責前布政使劉鼎、章欽文分償。戶部採銅鑄錢,定值斤六分五釐,犖以江蘇不產銅,採自他省,值昂過半,牒巡撫田雯,疏請停採。下部議,改視各關例,斤一錢。

二十七年,擢江西巡撫。湖廣叛卒夏逢龍為亂,徵江西兵赴剿,次九江,挾餉缺幾譁變。犖行次彭澤,聞報,檄發湖口庫帑充行糧,兵乃進。至南昌受事,舊裁督標兵李美玉、袁大相糾三千餘人,謀劫倉庫,應逢龍以叛。犖詗知之,捕得美玉、大相,眾恟恟。犖令即斬以徇,諭眾受煽惑者皆貸不問,眾乃定。

江西採竹木,饒州供紫竹,南康、九江供檀、柟諸木,通省派供貓竹,名雖官捐,實為民累,犖疏請動支正帑採買。上命歲終巡撫視察布政司庫,犖疏請糧驛道庫,布政使察覈;府庫,道員察覈。漢軍文武官吏受代,家屬例當還旗,經過州縣,點驗取結。犖曰:「是以罪人待之也。」疏請自贓私斥革並侵挪帑項解部比追外,止給到京定限咨文,俾示區別。皆下部議行。

三十一年,調江蘇巡撫。蘇州濱海各縣遇颶,上元、六合諸縣發山水,淮、揚、徐屬縣河溢,疏請視被災輕重,蠲減如例。發江寧、鳳陽倉儲米麥散賑。別疏請除太湖傍坍地賦額,戶部以地逾千畝,令詳察。犖再疏上陳,上特允之。

犖在江蘇,三遇上南巡,嘉犖居官安靜,迭蒙賞賚,以犖年逾七十,書「福」、「壽」字以賜。四十四年,擢吏部尚書。四十七年,以老乞罷,瀕行,賜以詩。五十三年,詣京師祝聖壽,加太子少師,復賜以詩,還里。卒,年八十,賜祭葬。

陳詵,字叔大,浙江海寧人。康熙十一年舉人,授中書科中書舍人。二十八年,考授吏科給事中,乞養歸。三十六年,起補原官。輔刑科掌印給事中。疏言:「淮、黃自古不兩行。邇者修歸仁堤,開胡家溝,出睢湖之水;閉六壩,加築高家堰,出洪澤湖之水。此借淮敵黃不易之理。然淮水入運者多,則敵黃仍弱。舊設天妃閘,自淮、黃交會處至清江浦,凡為五閘,重運到時,更迭啟閉,過即下板鎖斷,是以全淮注黃。其引入運河者,不過暫資濟運。自改建草壩,淮、黃盡趨運河,清江浦民居可危。宜復天妃閘舊制,使淮易敵黃,有裨大工。」疏下河督張鵬翮議行。尋疏劾山東蒲臺知縣俞宏聲以赦前細故,拘系監生王觀成,迫令自殺;巡撫王國昌僅以杖責解役結案,玩視民命。命侍郎吳涵偕詵往按,宏聲坐奪官,國昌等議處。授鴻臚寺卿,再遷左副都御史。

四十三年,授貴州巡撫。疏言:「貴州田地俱在層岡峻嶺間,土性寒涼,收成歉薄,人牛種蓺維艱。前撫臣王蓺因合屬田地荒蕪十之四五,減輕舊則,招徠開墾成熟,六年後起科。有續報者亦如之。」疏下部,如所請。四十七年,調湖北。疏劾布政使王毓賢虧帑,命解任。尋以盤驗已完,奏免其罪。五十年,擢工部尚書。五十二年,調禮部。五十八年,乞休,命致仕。六十一年,卒,賜祭葬,謚清恪。子世倌,自有傳。

論曰:當三籓亂時,雲、貴、閩、粵,其發難地也;蹂躪所及,湖南北、江西、四川,受害最甚。伊闢、王繼文撫雲南,從師而南,參與軍畫,其事已別見;雍建於貴州,締虞於四川,弘祚於廣東,騭於江西,犖承騭,詵遙繼雍建,兵後撫綏甚勤。大亂方定,起衰救弊,出水火,登衣任席,偉哉諸人之功歟!


\end{pinyinscope}