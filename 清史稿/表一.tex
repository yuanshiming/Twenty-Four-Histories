\article{表一}

\begin{pinyinscope}
皇子世表一

自周室★建同姓,穆屬維城;炎漢以降,帝王之子,靡不錫以王爵。考帝系者,於以見親親之誼焉。清初封爵之制,未嘗釐定,武功、慧哲、宣獻諸王,皆以功績而獲崇封。崇德元年,定九等爵。順治六年,復定為親、郡王至奉恩將軍凡十二等,有功封,有恩封,有考封。惟睿、禮、鄭、豫、肅、莊、克勤、順承八王,以佐命殊勛,世襲罔替。其他親、郡王,則世降一等,有至鎮國公、輔國公而仍延世賞者。若以旁支分封,則降至奉恩將軍,迨世次已盡,不復承襲。蓋自景祖以上子孫謂之「覺羅」,與顯祖以下子孫謂之「宗室」者,親疏攸別,爵秩亦殊,數傳而後,僅得子、男。原夫錫爵之本意,酬庸為上,展親次之,故有皇子而僅封貝勒、貝子、公者。揆諸前禩,至謹極嚴。雍正後,惟怡賢親王以公忠體國,恭忠親王以贊襄大政,醇賢親王以德宗本生考,皆世襲罔替。至末年,而慶親王奕劻乃亦膺茲懋賞矣。自餘宗潢繁衍,非國有大慶,不得恩封;非嫻習騎射,不得考封。而入關二百餘年,習尚文勝,無復開國勇健之風,每■歲終,與於選者益★。此盛衰強弱之原歟?今自肇祖以下子孫,列為世表,本諸瑤牒,支別派分。其不列於十二等之封者,謂之「閒散宗室」,則從略焉。作皇子世表。

天命八年六月戊辰,太祖御八角殿,集諸公主、郡主,訓之曰:「朕仰體天心,勸善懲惡,雖貝勒、大臣,有罪必治。汝曹茍犯吾法,詎可徇縱?朕擇賢而有功之人,以汝曹妻焉。汝曹當敬謹柔順,茍陵侮其夫,恣為驕縱,惡莫大焉!法不汝容。譬如萬物依日光以生,汝曹亦依朕之光以安其生可也。」復語皇妹曰:「汝其以婦道訓諸女!有犯,朕必罪之。」越數日,又諭歸附蒙古諸貝勒曰:「有娶我諸女者,茍見陵侮,必以告。」太祖初起,諸女但號「格格」,公主、郡主,亦史臣緣飾云爾。厥後始定:中宮出者,為「固倫公主」;自妃、嬪出者,及諸王女

育宮中者,為「和碩公主」。然開國初,有皇女僅得縣君、鄉君者。康熙以後,有妃、嬪若諸王女封固倫公主者,則恩澤有隆殺也。終清之世,為主婿者,前有何和禮,後有策棱,賢而有功,斯為最著。他若拉多爾濟,宮門殄逆,承祖澤;索特納木多爾澤,親臣受遺勛,皆克由禮,而諸主亦皆循循?,集於其子:與國同休戚,稱肺腑之誼。餘則奉朝請,參宿孝謹,太祖之教遠矣。後漢書以公主附后妃後,南齊檀超議為帝女立傳,王儉駁之,乃寢。新唐書始用其例,明史仍之。而遼、元二史則改次為表,詳略得中,今效為之,主婿無傳者,附見其事跡焉。


\end{pinyinscope}