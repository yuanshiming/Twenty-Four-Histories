\article{本紀五}

\begin{pinyinscope}
世祖本紀二

八年春正月己酉朔,蒿齊忒部臺吉噶爾馬撒望、儲護爾率所部來歸。辛亥,以布丹為議政大臣。甲寅,和碩英親王阿濟格謀亂,幽之。其黨郡王勞親降貝子,席特庫等論死。乙卯,以蘇克薩哈、詹岱為議政大臣。丙辰,罷漢中歲貢柑及江南橘、河南石榴。戊午,罷諸處織造督進官役及陜西歲貢羢褐皮革。命和碩睿親王多爾袞子多爾博襲爵。己未,罷臨清歲造城磚。庚申,上親政,御殿受賀,大赦。詔曰:「朕躬親大政,總理萬幾。天地祖宗,付託甚重。海內臣庶,望治甚殷。自惟涼德,夙夜祗懼。天下至大,政務至繁,非朕躬所能獨理。凡我諸王貝勒及文武群臣,其各占忠盡職,潔己愛人,利弊悉以上聞,德意期於下究。百姓亦宜咸體朕心,務本樂業,共享泰寧

之慶。」孔有德克桂林,斬故明靖江王及文武官四百七十三人,餘黨悉降。壬戌,罷江西歲進龍碗。丙寅,以夏一鶚為江西巡撫。丁卯,升祔孝端文皇后於太廟。追尊故攝政王多爾袞為成宗義皇帝,祔於太廟。移內三院於禁城。己巳,以伊圖為議政大臣。免安州芝棉稅。丁丑,復封端重郡王博洛、敬謹郡王尼堪為和碩親王。以鞏阿岱、鰲拜為議政大臣。戊寅,以巴圖魯詹、杜爾瑪為議政大臣。

二月庚辰,進封滿達海為和碩巽親王,多尼為和碩信親王,羅可鐸為多羅平郡王,瓦克達為多羅謙郡王,傑書為多羅康郡王。更定錢制,每百文準銀一錢。辛巳,免朔州、渾源、大同荒賦。癸未,羅什、博爾惠有罪,論死。上欲宥其死,群臣執奏不可,遂伏誅。戊子,上昭聖慈壽皇太后尊號。己丑,大赦。免汶上等五縣六、七兩年災賦。辛卯,罷邊外築城之役,加派錢糧準抵八年正賦,官吏捐輸酌給議敘並免之。癸巳,蘇克薩哈、詹岱、穆濟倫首告故攝政王多爾袞逆節皆實,籍其家,誅其黨何洛會、胡錫。甲午,免山西荒賦。戊戌,封貝勒岳樂為多羅安郡王。己亥,暴多爾袞罪於中外,削其尊號及母妻追封,撤廟享。庚子,調陳泰為吏部尚書,以韓岱為刑部尚書。辛丑,上幸南苑。壬寅,命孔有德移駐桂林。癸卯,上還宮。乙巳,封和碩肅親王豪格子富壽為和碩顯親王。

閏二月戊申朔,湖南餘寇牛萬才率所部降。庚戌,封和碩鄭親王濟爾哈朗子濟度為多羅簡郡王,勒度為多羅敏郡王。甲寅,諭曰:「國家紀綱,首重廉吏。邇來有司貪污成習,百姓失所,殊違朕心。總督巡撫,任大責重,全在舉劾得當,使有司知所勸懲。今所舉多冒濫,所劾多微員,大貪大惡乃徇縱之,何補吏治?吏部其詳察以聞。」調黨崇雅為戶部尚書,金之俊為兵部尚書,劉餘祐為刑部尚書,謝啟光為工部尚書。免祥符等六縣七年災賦。乙卯,進封碩塞為和碩承澤親王。諭曰:「榷關之設,國家藉以通商,非苦之也。稅關官吏,擾民行私,無異劫奪。朕灼知商民之苦。今後每關設官一員,悉裁冗濫,並不得妄咨勤勞,更與銓補。」丙辰,諭督撫甄別有司才德並優兼通文義者擢之,不識文義任役作奸者黜之,吏部授官校試文義不通者除名。己未,總兵官許爾顯克肇慶、羅定,徐成功克高州。禁喇嘛貢佛像、銅塔及番犬。壬戌,幽阿濟格於別室,籍其家,削貝子勞親爵為庶人。乙丑,大學士馮銓、尚書謝啟光等以罪免。諭曰:「國家設官,必公忠自矢,方能裨益生民,共襄盛治。朕親政以來,屢下詔令,嘉與更始。乃部院諸臣因仍前弊,持祿養交。朕親行黜陟,與天下見之。自今以後,其淬礪前非,各盡厥職。若仍上下交欺,法必不貸。」丙寅,諭曰:「各省土寇,本皆吾民,迫於饑寒,因而為亂。年來屢經撲剿,而管兵將領,殺良冒功,真盜未殲,民乃荼毒,朕深痛之。嗣後各督撫宜剿撫並施,勿藉捕擾民,以稱朕意。」丁卯,孔有德克梧州、柳州。戊辰,大學士洪承疇兼都察院左都御史,陳之遴為禮部尚書,張鳳翔為工部尚書。己巳,裁江南、陜西督餉侍郎,淮安總理漕運侍郎。庚午,固山額真阿喇善等剿山東賊。壬申,免涿、良鄉等十三州縣圈地。乙亥,定阿附多爾袞諸臣罪,剛林、祁充格俱坐罪。丁丑,諭曰:「故明宗籓,前以恣行不軌,多被誅戮,朕甚憫焉。自後有流移失所甘心投誠者,有司禮送京師,加恩畜養。鎮國將軍以下,即其地占籍為民,各安厥業。」免宛平災賦。

三月壬午,端重親王博洛、敬謹親王尼堪以罪降郡王。癸未,命諸王、貝勒、貝子分管六部、理籓院、都察院事。乙酉,湖南保、靖、永順等土司來歸。丙戌,免武強上年災賦。己丑,以希福為弘文院大學士,陳泰為國史院大學士。改李率泰為弘文院大學士,寧完我為國史院大學士。以噶達渾為都察院承政,硃瑪喇為吏部尚書,雅賴為戶部尚書,譚布為工部尚書,藍拜為鑲藍旗滿洲固山額真。辛卯,定王公朝集例。壬辰,定襲爵例。癸巳,諭曰:「御史巡方,職在安民察吏。向來所差御史,苞苴請託,身已失檢,何由察吏?吏不能察,民何以安?今後各宜洗濯自新,務盡職事,並許督撫糾舉,都察院考覈以聞。」癸卯,定齋戒例。丙午,許滿洲、蒙古、漢軍子弟科舉,依甲第除授。

夏四月庚戌,詔行幸所過,有司不得進獻。遣官祭嶽鎮海瀆、帝王陵寢、先師孔子闕里。土賊羅榮等犯虔州,副將楊遇明討擒之。乙卯,幸沙河。辛酉,次赤城。以王文奎總督漕運。甲子,次上都。丙寅,翁牛特部杜棱郡王等來朝。己巳,次俄爾峒。庚午,免朝鮮歲貢柑、柚、石榴。巴林部固倫額駙色布騰郡王等來朝。命故靖南王耿仲明子繼茂襲爵。辛未,還次上都河。壬申,次俄爾峒河。

五月丁丑朔,次謨護里伊札里河。夏一鶚擊明唐王故將傅鼎銓等,追入福建,擒鼎銓等斬之。辛巳,次庫爾奇勒河。壬午,烏硃穆秦部貝勒塞棱額爾德尼等來朝。乙酉,次西喇塔。調噶達渾為戶部尚書。以覺善為都察院承政,綽貝為鑲白旗蒙古固山額真。壬辰,次孫河。癸巳,還宮。丙申,免英山五年至七年荒逋賦。庚子,復博洛、尼堪親王爵。甲辰,御史張煊以奏劾尚書陳名夏論死。

六月丙午朔,幸南苑。官軍破陜西賊何柴山等於雒南。丁巳,阿喇善擊山東盈河山賊,平之。壬戌,罷太和山貢符篆、黃精。乙丑,定諸陵壇廟祀典。庚午,諭曰:「朕以有司貪虐,命督撫察劾。乃閱四五月之久而未奏聞。毋乃受賕徇私,為有司所制,或勢要挾持,不敢彈劾歟?此盜賊所由滋,而黎民無起色也。其即奉行前詔,直陳無隱。」辛未,詔故明神宗陵如十二陵,以時致祭,仍設守陵戶。廣東官軍復廉州及永安等十二縣。壬申,命修繕祖陵,設守戶,定祭禮,復朝日、夕月禮。

秋七月丙子朔,諭曰:「比者投充漢人,生事害民,朕甚恨之。夫供賦役者編氓也,投充者奴隸也。今反厚奴隸而薄編氓,如國家元氣及法紀何?其自朕包衣牛錄,下至王公諸臣投充人,有犯法者,嚴治其罪,知情者連坐。前有司責治投充人,至獲罪譴。今後與齊民同罰,庶無異視。使天下咸知朕意。」又諭曰:「大小臣工,皆朝廷職官,待之以禮,則朝廷益尊。今在京滿、漢諸臣犯罪,有未奉旨革職輒提取審問者,殊乖大體。嗣後各衙門遇官員有犯,或被告訐,皆先請旨革職,然後送刑部審問,毋得徑行提審,著為令。」戊子,大學士陳泰、李率泰以罪免。以雅秦為內國史院大學士,杜爾德為議政大臣。乙未,幸南苑。己亥,以陳名夏為內弘文院大學士。

八月丙午朔,上還宮。丁未,科爾沁卓禮克圖親王吳克善來朝。己酉,副將許武光請括天下藏金充餉。上曰:「帝王生財之道,在節用愛民。掘地求金,自古未有。」命逐去之。乙卯,以趙開心為左都御史。定順天鄉試滿洲、蒙古為一榜,漢軍、漢人為一榜,會試、殿試如之。戊午,冊立科爾沁卓禮克圖親王吳克善女博爾濟錦氏為皇后。壬戌,更定馬步軍經制。吏部尚書譚泰有罪,伏誅,籍其家。乙酉,大婚禮成,加上太后尊號為昭聖慈壽恭簡皇太后。丙寅,御殿受賀,頒恩赦。戊辰,追復肅親王豪格爵。己巳,詔天下歲貢物產不便於民者悉罷之。癸酉,陳錦、金礪等追故明魯王於舟山,獲其將阮進。

九月庚辰,定朝儀。壬午,命平西王吳三桂征四川。陳錦、金礪克舟山,故明魯王遁走。丙戌,雅賴、譚布、覺善免,以卓羅為吏部尚書,車克為戶部尚書,藍拜為工部尚書,俄羅塞臣為都察院左都御史,趙國祚為鑲紅旗漢軍固山額真。封阿霸垓部都司噶爾為郡王。固山額真噶達渾征鄂爾多斯部多爾濟。丁亥,除永平四關荒屯賦。壬辰,改承天門為天安門。癸巳,上獵於近郊。辛丑,還宮。癸卯,喀爾喀部土謝圖汗、車臣汗、塞臣汗等來貢。

冬十月己酉,以和碩承澤親王碩塞、多羅謙郡王瓦克達為議政王。辛亥,免宣府災賦。丁巳,以額色黑為國史院大學士。庚申,賜阿濟格死。辛酉,李國翰會吳三桂征四川。以馬光輝為直隸山東河南總督。甲子,免諸王三大節進珠、貂、鞍馬及衍聖公、宣、大各鎮歲進馬。乙丑,封肇祖、興祖陵山曰啟運山,景祖、顯祖陵山曰積慶山,福陵山曰天柱山,昭陵山曰隆業山。是日,啟運山慶雲見。

十一月乙亥朔,皇第一子牛鈕生。丙子,於大海率所部至夷陵請降。丙戌,尚可喜克雷州。乙未,免平陽、潞安二府,澤、遼、沁三州上年災賦。戊戌,以伊爾德為正黃旗滿洲固山額真,佟圖賴為正藍旗漢軍固山額真。庚子,免陽曲等四縣上年災賦。壬寅,免寧晉荒賦。

十二月丙午,免桐城等四縣上年荒賦。丁卯,以周國佐為江寧巡撫。

是年,朝鮮,厄魯特部額爾德尼臺吉、昆都倫吳巴什、阿巴賴,喀爾喀部土謝圖汗、車臣汗、塞臣汗、顧實汗、臺吉吳巴什,達賴喇嘛俱來貢。

九年春正月癸酉朔,上幸南苑。辛巳,以陳泰為禮部尚書。壬午,大學士陳名夏以罪免。雪張煊冤,命禮部議恤。京師地震。乙酉,以陳維新為廣西巡撫。壬寅,皇第一子牛鈕薨。

二月丁未,以祜錫布為鑲紅旗滿洲固山額真。噶達渾等討鄂爾多斯部多爾濟等於賀蘭山,殲之。戊申,和碩巽親王滿達海薨,追封和碩簡親王。庚戌,頒六諭臥碑文於天下。庚申,加封鄭親王濟爾哈朗為叔和碩鄭親王。辛酉,以陳之遴為弘文院大學士,孫茂蘭為寧夏巡撫。

三月乙亥,以王鐸為禮部尚書,房可壯為左都御史。贈張煊太常寺卿,仍錄其子如父官。庚辰,定官員封贈例。丙戌,罷諸王、貝勒、貝子管理部務。追降和碩豫親王多鐸為多羅郡王。丁亥,和碩端重親王博洛薨,追封和碩定親王。己丑,以陳泰為鑲黃旗滿洲固山額真。癸巳,以遏必隆、額爾克戴青、趙布泰、賴塔庫、索洪為議政大臣,覺羅郎球、胡世安為禮部尚書。鞏阿岱、錫翰、西訥布庫、冷僧機以罪伏誅,籍其產。拜尹圖免死,幽系。戊戌,多羅順承郡王勒克德渾薨,追封多羅恭惠郡王。己亥,賜滿洲、蒙古貢士麻勒吉,漢軍及漢貢士鄒忠倚等進士及第出身有差。

夏四月丙午,以蔡士英為江西巡撫。丁未,裁登萊、宣府巡撫。乙卯,以韓岱為吏部尚書,藍拜為刑部尚書,星訥為工部尚書,阿喇善為都察院左都御史。戊午,孔有德克廣西南寧、慶遠、思恩,故明將陳邦傅以潯州降。己未,免府州縣官入覲。庚申,定諸王以下官名輿服之制。乙丑,允禮部議,一月三朝,春秋一舉經筵。設宗人府官。

五月丁丑,詔京察六年一舉行。己卯,免江陰、青浦牛稅。壬午,以喀喀木為昂邦章京,鎮守江寧。庚子,幸南苑。

六月丁未,裁並直隸諸衛所。戊申,上還宮。庚戌,以和碩敬謹親王尼堪掌宗人府事,貝勒尚善、貝子吳達海為左右宗正。官軍討肇慶、高州賊,平之。丁巳,詔軍政六年一舉行。丙寅,設詹事府官。追謚圖爾格為忠義公,圖賴為昭勛公,配享太廟。

秋七月癸酉,故明將孫可望陷桂林,定南王孔有德死之。丙子,名皇城北門為地安門。浙閩總督陳錦征鄭成功,至漳州,為其下所殺。庚辰,免淮安六年、七年牙行逋稅。甲申,以和碩敬謹親王尼堪為定遠大將軍,征湖南、貴州。定滿官喪制。丁亥,以巴爾處渾為鑲紅旗滿洲固山額真。免磁、祥符等八州縣及懷慶衛上年災賦。吳三桂、李國翰定漳臘、松潘、重慶。遣梅勒章京戴都圍成都,故明帥劉文秀舉城降。己丑,免臨邑四縣荒徭賦。辛卯,天全六番、烏思藏等土司來降。戊戌,以祖澤遠為湖廣四川總督。

八月乙巳,更定王公以下婚娶禮。丙午,多羅謙郡王瓦克達薨。丁巳,命尼堪移師討廣西餘寇。

九月庚午朔,以硃孔格、阿濟賴、伊拜為議政大臣。辛巳,更定王以下祭葬禮。癸未,以纛章京阿爾津為定南將軍,同馬喇希徵廣東餘寇。甲申,以劉清泰為浙江福建總督,王來用為順天巡撫。辛卯,幸太學釋奠。癸巳,賚衍聖公、五經博士、四氏子孫、祭酒、司業等官有差。敕曰:「聖人之道,如日中天,上之賴以致治,下之資以事君。學官諸生當共勉之。」

冬十月庚子,免沛縣六年至八年災賦。尚可喜、耿繼茂克欽州、靈山,故明西平王硃聿★縛賊渠李明忠來降,高、雷、廉、瓊諸郡悉平。壬寅,官軍復梧州。癸卯,以歲饑,詔所在積穀,禁遏糴,旌輸粟。丙午,免三水等三縣六年災賦。壬子,以劉餘祐為戶部尚書。癸丑,免霸州、東安、文安荒賦。甲寅,孫可望寇保寧,吳三桂、李國翰大敗之。以希福、範文程、額色黑、車克、覺羅郎球、明安達禮、濟席哈、星訥為議政大臣,巴哈納為刑部尚書,藍拜罷。戊午,命和碩鄭親王世子濟度,多羅信郡王多尼,多羅安郡王岳樂,多羅敏郡王勒都,貝勒尚善、杜爾祜、杜蘭議政。辛酉,以阿爾津為安西將軍,同馬喇希移鎮漢中。丙寅,以李化熙為刑部尚書。丁卯,尊太宗大貴妃為懿靖大貴妃,淑妃為康惠淑妃。

十一月庚午,以卓羅為靖南將軍,同藍拜等征廣西餘寇。己丑,祀天於圜丘。庚寅,故明將白文選寇辰州,總兵官徐勇、參議劉升祚、知府王任杞死之。辛卯,尼堪抵湘潭,故明將馬進忠等遁寶慶,追至衡山,擊敗之,又敗之於衡州。尼堪薨於軍。追封尼堪為和碩莊親王。乙未,免忻、樂平等州縣災賦。

十二月辛丑,免太原、平陽、汾州、遼、沁、澤災賦。壬寅,詔還清苑民三百餘戶所撥投充人地,仍免地租一年。官軍復安福、永新。丙午,撤卓羅等軍回京。庚戌,幸南苑。戊午,還宮。廣東賊犯香山,官軍討平之。己未,復命阿爾津為定南將軍,同馬喇希等討辰、常餘寇。甲子,免長武災賦。

是年,達賴喇嘛來朝。朝鮮,厄魯特部顧實汗、巴圖魯諾顏,喀爾喀部土謝圖汗下戴青諾顏、喇嗎達爾達爾漢諾顏,索倫部索郎阿達爾漢及班禪胡土克圖、第巴、巴喀胡土克圖喇嘛俱來貢。厄魯特顧實汗三至。

十年春正月庚午,諭曰:「朕自親政以來,但見滿臣奏事。大小臣工,皆朕腹心。嗣凡章疏,滿、漢侍郎、卿以上會同奏進,各除推諉,以昭一德。」辛未,諭:「言官不得捃摭細務,朕一日萬幾,豈無未合天意、未順人心之事。諸臣其直言無隱。當者必旌,戇者不罪。」癸酉,免莊浪、紅城堡、洮州衛災賦。丁丑,改洪承疇為弘文院大學士,陳名夏為祕書院大學士。庚辰,以貝勒吞齊為定遠大將軍,統征湖南軍,授以方略。丙戌,以多羅額駙內鐸為議政大臣。詔三品以上大臣各舉所知,仍嚴連坐法。庚寅,調金之俊為左都御史,以劉昌為工部尚書。癸巳,更定多羅貝勒以下歲俸。丙申,幸內苑,閱通鑒。上問漢高祖、文帝、光武及唐太宗、宋太祖、明太祖孰優。陳名夏對曰:「唐太宗似過之。」上曰:「不然,明太祖立法可垂永久,歷代之君皆不及也。」

二月庚子,封蒿齊忒部臺吉噶爾瑪薩望為多羅郡王。壬子,大學士陳之遴免。甲寅,以陳之遴為戶部尚書。乙卯,以沈永忠為剿撫湖南將軍,鎮守湖南。己未,裁各部滿尚書之衣復者。庚申,以高爾儼為弘文院大學士,費揚古為議政大臣。辛酉,明安達禮、劉餘祐有罪,免。甲子,喀爾喀部土謝圖汗下賁塔爾、袞布、奔巴世希、扎穆蘇臺吉率所部來歸。

三月戊辰,幸南臺較射。上執弓曰:「我朝以此定天下,朕每出獵,期練習騎射。今綜萬幾,日不暇給,然未嘗忘也。」賜太常寺卿湯若望號通玄教師。免山西岢嵐、保德七十四州縣六年逋賦,代、榆次十二州縣十之七。己巳,封喀爾喀部賁塔爾為和碩達爾漢親王,袞布為卓禮克圖郡王,奔巴世希為固山貝子。免薊、豐潤等十一州縣九年災賦。庚午,幸南苑。甲戌,免五臺縣逋賦及八年額賦之半。己卯,免江西六年荒地逋賦。辛巳,設宗學,親王、郡王年滿十歲,並選師教習。乙酉,還宮。丙戌,濟席哈免。以噶達渾為兵部尚書。甲午,復以馮銓為弘文院大學士。

夏四月丁酉,親試翰林官成克鞏等。庚子,御太和殿,召見朝覲官,諭遣之。諭曰:「國家官人,內任者習知紀綱,外任者諳於民俗,內外★歷,方見真才。今親試詞臣,其未留任者,量予改授,照詞臣外轉舊例,優予司、道各官。」始諭吏部、都察院舉京察。甲辰,免湖南六年至九年逋賦、山西夏縣荒賦。丙午,以佟國器為福建巡撫。丁未,以圖海為弘文院大學士。壬子,以旱,下詔求直言,省刑獄。甲寅,命提學御史、提學道清釐學政。定學額,禁冒濫。改折民間充解物料,行一條鞭法。丁巳,定滿官離任持服三年例。己未,以成克鞏為吏部尚書。癸亥,免福州等六府九年以前荒賦三之一。

五月甲戌,停御史巡按直省。免祥符等七縣九年災賦,沔陽、潛江、景陵八年災賦。乙亥,封鄭芝龍為同安侯,子成功為海澄公,弟鴻逵為奉化伯。以喀喀木為靖南將軍,徵廣東餘寇。免歷城等六十九州縣八、九年災賦。丁丑,定旌表宗室節孝貞烈例。己卯,詔曰:「天下初定,瘡痍未復,頻年水旱,民不聊生,饑寒切身,迫而為盜。魁惡雖多,豈無冤濫,脅從沈陷,自拔無門。念此人民,誰非赤子,摧殘極易,生聚綦難,概行誅鋤,深可憫惻。茲降殊恩,曲從寬宥,果能改悔,咸與自新。所在官司,妥為安插,兵仍補伍,民即歸農,不原還鄉,聽其居位,勿令失所。咸使聞知。」庚辰,定熱審例。乙酉,追封舒爾哈齊為和碩親王,額爾袞、界堪、雅爾哈齊、祜塞為多羅郡王。免武昌、漢陽、黃州、安陸、德安、荊州、岳州九年災賦。庚寅,加洪承疇太保,經略湖廣、廣東、廣西、雲南、貴州。壬辰,以張秉貞為刑部尚書。甲午,免霸、保定等三十一州縣九年災賦。

六月乙未朔,追封塔察篇古、穆爾哈齊為多羅貝勒。丁酉,諭曰:「帝王化民以德,齊民以禮,不得已而用刑。法者天下之平,非徇喜怒為輕重也。往者臣民獲罪,必下部議,以士師之任,職在明允。乃或私心揣度,事經上發,則重擬以待親裁;援引舊案,又文致以流刻厲。朕群生在宥,臨下以寬。在饑寒為盜之民,尚許自首,遐方未服之罪,亦予招手巂。況於甿庶朝臣,豈忍陷茲冤濫?自後法司務得真情,引用本律,金句距羅織,悉宜痛革,以臻刑措。」大學士高爾儼免。癸卯,復秋決朝審例。乙巳,命祖澤遠專督湖廣,孟喬芳兼督四川。丙午,免慈谿等五縣八年災賦。辛亥,賜故明殉難大學士範景文、戶部尚書倪元璐等及太監王承恩十六人謚,並給祭田,所在有司致祭。改折天下本色錢糧,行一條鞭法。癸丑,貝勒吞齊等敗孫可望於寶慶。庚申,以李率泰為兩廣總督。慈寧宮成。辛酉,增置內三院漢大學士,院各二人。癸亥,諭曰:「唐、虞、夏、商未用寺人,至周僅具其職,司閽闥灑掃、給令而已。秦、漢以來,始假事權,加之爵祿,典兵干政,貽禍後代。小忠小信,固結主心;大憝大奸,潛持國柄。宮庭邃密,深居燕閒,淆是非以溷賢奸,刺喜怒而張威福,變多中發,權乃下移。歷覽覆車,可為鑒戒。朕酌古準今,量為設置,級不過四品。非奉差遣,不許擅出皇城。外官有與交結者,發覺一並論死。」

閏六月丙寅,以成克鞏為祕書院大學士,張端為國史院大學士,劉正宗為弘文院大學士。乙亥,以金之俊為吏部尚書。庚辰,諭曰:「考之洪範,作肅為時雨之徵,天人感應,理本不爽。朕朝乾夕惕,冀迓天休。乃者都城霖雨匝月,積水成渠,壞民廬舍,窮黎墊居艱食,皆朕不德有以致之。今一意修省,祗懼天戒。大小臣工,宜相儆息。」

秋七月甲午朔,上以皇太后諭,發節省銀八萬兩賑兵民潦災。辛丑,以宜永貴為南贛巡撫。庚戌,皇第二子福全生。辛酉,以安郡王岳樂為宣威大將軍,率師駐防歸化城。

八月壬午,以太宗十四女和碩公主下嫁平西王吳三桂子應熊。尚可喜克化州、吳川。甲申,定武職品級。丙戌,以雷興為河南巡撫。己丑,廢皇后為靜妃。辛卯,李定國犯平樂,府江道周永緒,知府尹明廷,知縣塗起鵬、華鍾死之。

九月壬子,復刑部三覆奏例。丙辰,耿繼茂、喀喀木克潮州。丁巳,孟喬芳討故明宜川王硃敬樕於紫陽,平之。

冬十月癸亥朔,命田雄移駐定海。乙丑,馬光輝等討叛將海時行於永城,時行伏誅。丙寅,遣濟席哈討山東土寇。乙酉,設粥廠賑京師饑民。免通、密雲等七州縣災賦。戊子,命大學士、學士於太和門內更番入直。

十一月甲午,祀天於圜丘。戊戌,鄭成功不受爵,優諭答之。戊申,以亢得時為河南巡撫。己酉,官軍討西寧叛回,平之。乙卯,硃瑪喇、金之俊免。丙辰,免江南災賦。戊午,劉清泰剿九仙山賊,平之。己未,免江西五十四州縣災賦。

十二月丙寅,以陳泰為寧南靖寇大將軍,同藍拜鎮湖南。丁卯,以呂宮為弘文院大學士,博博爾代為議政大臣,馮聖兆為偏沅巡撫。辛未,幸南苑。甲戌,免金華八縣九年災賦。癸未,設兵部督捕官。以羅畢為議政大臣。甲申,免開封、彰德、衛輝、懷慶、汝寧九年、十年災賦。丙戌,鄭成功犯吳淞,官軍擊走之。丁亥,還宮。是夜,地震有聲。

是年,朝鮮,琉球,喀爾喀部土謝圖汗下索諾額爾德尼、額爾德尼哈談巴圖魯,厄魯特部顧實汗、顧實汗下臺吉諾穆齊,索倫部巴達克圖,富喇村宜庫達,黑龍江烏默忒、額爾多科,烏思藏達賴喇嘛俱來貢。朝鮮再至。

十一年春正月辛丑,罷織造官。戊申,免江寧、安徽、蘇、松、常、鎮、廬、鳳、淮、徐、滁上年災賦。己酉,以袁廓宇為偏沅巡撫,胡全才撫治鄖陽。庚戌,廣東仁化月峒賊平。癸丑,鄭成功犯崇明、靖江、泰興,官軍擊走之。甲寅,以金礪為川陜三邊總督。乙卯,鄭成功犯金山。丁巳,免順德、廣平、大名、天津、薊州上年災賦。辛酉,官軍擊賊於桃源,誅偽總兵李陽春等。

二月癸亥,朝日於東郊。丙寅,諭曰:「言官為耳目之司,朕屢求直言,期遇綦切。乃每閱章奏,實心為國者少,比黨徇私者多,朕甚不取。其滌肺腸以新政治。」以金之俊為國史院大學士。庚午,甄別直省督撫,黜陟有差。丙子,始耕耤田。戊寅,免江西缺額丁賦。辛巳,命尚可喜專鎮廣東,耿繼茂移駐桂林。壬午,以馬鳴珮為宣大山西總督,耿焞為山東巡撫,陳應泰為山西巡撫,林天擎為湖廣巡撫,黃圖安為寧夏巡撫。癸未,官軍復平遠縣。甲申,諭曰:「比年以來,軍興未息,供億孔殷,益以水旱,小民艱食,有司失於拊循,流離載道。朕心惻然,不遑寢處。即核庫儲,亟圖賑撫。」己丑,免河南州縣衛所十年災賦。庚寅,以李廕祖為直隸山東河南總督。

三月壬辰,官軍擊桂東賊,擒其渠賴龍。戊戌,免湖廣襄陽、黃州、常德、岳州、永州、荊州、德安及辰、常、襄三衛,山東濟南、東昌十年災賦。辛丑,寧完我劾陳名夏罪,鞫實,伏誅。乙巳,以王永吉為左都御史。戊申,皇第三子玄燁生,是為聖祖。以蔣赫德為國史院大學士。乙卯,以多羅慧哲郡王額爾袞、多羅宣獻郡王界堪、多羅通達郡王雅爾哈齊配享太廟。以孟明輔為兵部尚書。

夏四月壬戌,賊渠曹志攀犯饒州,官軍擊敗之,志攀降。庚午,四川賊魏勇犯順慶,官軍擊敗之。壬申,地震。官軍擊故明將張名振等於崇明,敗之。癸酉,免洛南上年災賦三之一。己卯,幸南苑,賚所過農民金。乙酉,免保康等四縣上年被寇災賦。丁亥,以王永吉為祕書院大學士,秦世楨為浙江巡撫。戊子,江南寇徐可進、硃元等降。

五月壬辰,上還宮。甲午,幸西苑,賜大臣宴。庚子,以胡圖為議政大臣。甲辰,免平涼衛上年災賦。丙午,起黨崇雅為國史院大學士,以龔鼎孳為左都御史。丁未,遣官錄直省囚。庚戌,免興安、漢陰、平利等州縣上年災賦。辛亥,太白晝見。丙辰,以楊麒祥為平南將軍,駐防杭州。

六月己未朔,河決大王廟。丙寅,陜西地震。丁卯,以硃瑪喇為靖南將軍,徵廣東餘寇。甲戌,立科爾沁鎮國公綽爾濟女博爾濟錦氏為皇后。庚辰,大赦。

秋七月戊子朔,封琉球世子尚質為中山王。壬辰,免秦州、朝邑、安定災賦。戊申,免鎮原、廣寧二縣災賦。丙辰,以佟代為浙閩總督。

八月戊午朔,免延安府荒賦。己未,官軍剿瑞金餘寇,誅偽都督許勝可等。庚申,罷直省恤刑官,命巡撫慮囚。辛酉,免真寧縣十年災賦。壬戌,山東濮州、陽穀等縣地震有聲。甲戌,以張中元為江寧巡撫。丙子,以張秉貞為兵部尚書。庚辰,以傅以漸為祕書院大學士,任濬為刑部尚書。壬午,故明樂安王硃議水朋謀反,伏誅。

九月己丑,範文程以病罷。免西安、平涼、鳳翔三府十年災賦。庚寅,封線國安為三等伯。壬辰,申嚴隱匿逃人之禁。癸巳,免宣府、萬全右衛災賦。丙申,以董天機為直隸巡撫。壬子,以馮聖兆為延綏巡撫。

冬十月丁巳朔,享太廟。辛未,免廬、鳳、淮、揚四府,徐、滁、和三州災賦。丁丑,命重囚犯罪三法司進擬,仍令議政王、貝勒、大臣詳議。壬午,賑畿輔被水州縣。免祁陽等七縣逋賦。李定國陷高明,圍新會,耿繼茂請益師。

十一月丁亥,以陳泰為吏部尚書,阿爾津為正藍旗滿洲固山額真。尚可喜遣子入侍。壬寅,詔曰:「朕纘承鴻緒,十有一年,治效未臻,疆圉多故,水旱疊見,地震屢聞,皆朕不德之所致也。朕以眇躬託於王公臣庶之上,政教不修,瘡痍未復,而內外章奏,輒以『聖』稱,是重朕之不德也。朕方內自省抑,大小臣工亦宜恪守職事,共弭災患。凡章奏文移,不得稱『聖』。大赦天下,咸與更始。」癸卯,幸南苑。甲辰,耿繼茂遣子入侍。

十二月辛酉,和碩承澤親王碩塞薨。戊辰,免荊門、鍾祥等六州縣災賦。己巳,免磁、祥符等三十六州縣災賦。壬申,以濟度為定遠大將軍,徵鄭成功。尚可喜、耿繼茂、硃瑪喇敗李定國於新會,定國遁走。乙亥,鄭成功陷漳州,圍泉州。丁丑,命明安達禮徵羅剎。免西安五衛荒賦。江西賊霍武等率眾降。

是年,朝鮮,琉球,厄魯特部阿巴賴諾顏、諾門汗、額爾德尼達雲綽爾濟,索倫部索朗噶達爾漢,湯古忒部達賴喇嘛、諦巴班禪胡土克圖均來貢。

十二年春正月戊子,官軍敗賊於玉版巢,又擊藤縣賊,破之。庚寅,免東平、濟陽等十八州縣上年災賦。乙未,免直隸八府,河南彰德、衛輝、懷慶上年災賦。戊戌,詔曰:「親政以來,五年於茲。焦心勞思,以求治理,日望諸臣以嘉謨入告,匡救不逮。乃疆圉未靖,水旱頻仍,吏治墮汙,民生憔悴,保邦制治,其要莫聞。諸王大臣皆親見祖宗創業艱難,豈無長策,而未有直陳得失者,豈朕聽之不聰,虛懷納諫有未盡歟?天下之大,幾務之繁,責在一人,而失所輔導。朕雖不德,獨不念祖宗培養之澤乎!其抒忠藎,以慰朕懷。」辛丑,以韓岱為吏部尚書,伊爾德、阿喇善為都統。癸卯,以於時躍為廣西巡撫。甲辰,命在京七品以上,在外文官知府、武官副將以上,各舉職事及兵民疾苦,極言無隱。辛亥,修順治大訓。

二月庚申,復遣御史巡按直省。壬戌,大學士呂宮以疾免。癸亥,免成安等六縣上年災賦。己巳,賑旗丁。免平涼、漢陰二縣上年災賦。丙子,封博穆博果爾為和碩襄親王。免濱、寧陽等二十一州縣上年災賦。己卯,免滁、和二州上年災賦。庚辰,以陳之遴為弘文院大學士,王永吉為國史院大學士。癸未,耿繼茂、尚可喜敗李定國於興業。廣東高、雷、廉三府,廣西橫州平。

三月戊子,免湖廣石門縣上年災賦。以戴明說為戶部尚書。庚子,以佟國器為南贛巡撫,宜永貴為福建巡撫。壬寅,免鄖陽、襄陽二府上年被寇荒賦。甲辰,賜圖爾宸、史大成等進士及第出身有差。丁未,削續順公沈永忠爵。壬子,諭曰:「自明末擾亂,日尋干戈,學問之道,闕焉弗講。今天下漸定,朕將興文教,崇儒術,以開太平。直省學臣,其訓督士子,博通古今,明體達用。諸臣政事之暇,亦宜留心學問,佐朕右文之治。」癸丑,設日講官。

夏四月乙丑,免沈丘及懷慶衛上年災賦。丁丑,進封尼思哈為和碩敬謹親王,齊克新為和碩端重親王。癸未,詔修太祖、太宗聖訓。

五月乙酉,以圖海兼刑部尚書。辛卯,和碩鄭親王濟爾哈朗薨,輟朝七日。丁酉,以石廷柱為鎮海將軍,駐防京口。戊戌,以胡沙為鑲黃旗固山額真。庚子,以覺羅巴哈納為弘文院大學士。辛丑,靈丘縣地震有聲。乙巳,以覺羅郎球為戶部尚書。丙午,以李際期為兵部尚書。丁未,以恩格德為禮部尚書。己酉,以衛周祚為工部尚書。

六月甲寅,免杭州、寧波、金華、衢州、臺州災賦。丁卯,諭曰:「朕覽法司章奏,決囚日五、六人,或十餘人。念此愚氓,兵戈災祲之後,復罹法網,深可憫惻。有虞之世,民不犯於有司。漢文帝、唐太宗亦幾致刑措。今犯法日眾,豈風俗日偷歟?抑朝廷德教未敷,或讞獄者有失入歟?嗣後法司其明慎用刑,務求平允。」戊辰,免房山縣上年災賦。桂王將劉文秀寇常德,遣其黨犯岳州、武昌,官軍擊走之。己卯,封博果鐸為和碩莊親王。辛巳,命內十三衙門立鐵牌。諭曰:「中官之設,自古不廢。任使失宜,即貽禍亂。如明之王振、汪直、曹吉祥、劉瑾、魏忠賢輩,專權擅政,陷害忠良,出鎮典兵,流毒邊境,煽黨頌功,謀為不軌,覆敗相尋,深可鑒戒。朕裁定內官職掌,法制甚明。如有竊權納賄,交結官員,越分奏事者,凌遲處死。特立鐵牌,俾世遵守。」

秋七月癸未朔,日有食之。壬辰,復遣廷臣恤刑。辛亥,命直省繪進輿圖。

八月丙辰,免靈丘縣災賦。癸亥,以阿爾津為寧南靖寇大將軍,同卓羅駐防荊州,祖澤潤防長沙。乙丑,以多羅安郡王岳樂為左宗正,貝勒杜蘭為右宗正。癸酉,諭曰:「畿輔天下根本,部臣以運河決口,議徵逋賦。朕念畿內水旱相仍,人民荼苦,復供舊稅,其何以堪。今悉與蠲免。工築之費,別事籌畫。」免曹、城武等七州縣及臨清衛、齊河屯上年災賦。

九月癸未,免鳳陽災賦。壬寅,定武會試中式殿試如文進士。硃瑪喇、敦拜師還。丙午,頒禦制資政要覽、範行恆言、勸善要言、儆心錄,異姓公以下,文三品以上各一部。戊申,免兩當、寧遠二縣災賦。

冬十月辛亥朔,設尚寶司官。壬子,免蔚州及陽和、陽高二衛災賦。己未,免甘州、肅州、涼州、西寧災賦。辛酉,命每年六月慮囚,七月覆奏,著為令。癸亥,免磁、獲嘉等八州縣災賦。甲子,免隆平十一年以前逋賦、淄川等八縣災賦。丙寅,免宣府、大同災賦。戊辰,詔曰:「帝王以德化民,以刑輔治。茍律例輕重失宜,官吏舞文出入,政平訟理,其道曷由。朕覽讞獄本章,引用每多未愜。其以現行律例繕呈,朕將親覽更定之。」辛未,以祝世允為鑲紅旗滿洲固山額真。癸酉,以孫廷銓為兵部尚書。乙亥,修玉牒。丙子,龔鼎孳以罪免。

十一月壬午,免濱、堂邑等十三州縣災賦。癸未,鄭成功將犯舟山。乙酉,巡按御史顧仁坐納賄,棄市。丁亥,諭曰:「國家設督撫巡按,振綱立紀,剔弊發奸,將令互為監察。近來積習,乃彼此容隱。凡所糾劾止末員,豈稱設官之意。嗣有瞻顧徇私者,並坐其罪。」鄭成功將陷舟山,副將把成功降於賊。戊子,幸南苑。免鄖陽、襄陽逋賦,汲、淇、胙城等縣災賦。戊申,免臨漳災賦。

十二月丙辰,免耀州、同官、雒南災賦。癸亥,免安吉、仁和等十州縣,宣化八衛災賦。乙丑,頒大清滿字律。免臨清、齊河等十州縣,東昌衛災賦。丙寅,於時躍、祖澤遠平九團兩都瑤、僮一百九十二寨。己巳,多羅敏郡王勒度薨。癸酉,免涿、慶雲等三十三州縣,永平衛災賦。甲戌,以宜爾德為寧海大將軍,討舟山寇。以秦世禎為安徽巡撫,提督操江,陳應泰為浙江巡撫,白如梅為山西巡撫。免臨海等十八縣,祥符、蘭陽二縣,懷慶、群牧二衛災賦。

是年,喀爾喀部額爾德尼諾穆齊臺吉、門章墨爾根楚虎爾臺吉、伊世希布額爾德尼臺吉、額爾克戴青臺吉來朝。朝鮮,喀爾喀部畢席勒爾圖汗、俄木布額爾德尼、澤卜尊丹巴胡土克圖、丹津喇嘛、車臣汗、土謝圖汗、土謝圖汗下喇嘛塔爾達爾漢諾顏,厄魯特部杜喇爾渾津臺吉、都喇爾渾津阿里錄克三拖因、阿巴賴諾顏、鄂齊爾圖臺吉、噶爾丹霸,索倫部馬魯凱,訥墨禮河頭目伊庫達,黑龍江頭目庫拜,班禪胡土克圖,俄羅斯察幹汗遣使均來貢。朝鮮三至。厄魯特阿巴賴、鄂齊爾圖臺吉再至。

十三年春正月庚辰朔,幸南苑。癸未,諭修通鑒全書、孝經衍義。丙申,免漢中、鳳翔、西安上年災賦。己亥,鄭成功將犯臺州,副將馬信以城叛,降於賊。庚子,免廣德上年災賦十之一。甲辰,免富陽等六縣上年災賦。乙巳,免江西八年逋賦。

二月戊午,免荊州、安陸、常德、武昌、黃州上年災賦。庚申,免廣平上年災賦。丙寅,免岢嵐、五臺上年災賦。戊辰,命兩廣總督移駐梧州。官軍敗李定國於南寧。庚午,定部院滿官三年考滿、六年京察例。以李率泰為浙閩總督,王國光為兩廣總督。甲戌,以趙布泰為鑲黃旗固山額真。丙子,幸南苑,較射。免東平、濮、長山上年災賦。己卯,大學士馮銓致仕。

三月庚辰,幸瀛臺。癸未,免景陵等九縣上年災賦。癸巳,以費雅思哈為議政大臣,馬之先為川陜三邊總督。乙未,陳之遴有罪,以原官發盛京閒住。癸卯,諭曰:「朝廷立賢無方,比來罷譴雖多南人,皆以事論斥,非有所左右也。諸臣毋歧方隅,毋立門戶,毋挾忿肆誣,毋摭嫌苛訐,庶還蕩平之治。」丙午,諭曰:「朕親政以來,夙夜兢業,每期光昭祖德,蚤底治平,克當天心,以康民物。方睿王攝政,斥忠任奸,百姓怨嗟,望朕親政。乃者冬雷春雪,隕石雨土,所在見告。六載之中,康乂未奏,災祲時聞。是朕有負於百姓也。用是恐懼靡寧,冀昭告於上帝祖宗,實圖省戒,有司其涓日以聞。」

夏四月辛亥,廣西故明永安王硃華堧及土司等來降。乙卯,以災變祭告郊廟。辛酉,官軍破賊姚黃於夷陵。壬戌,太原陽曲地震。丁卯,以覺羅科爾坤為吏部尚書。庚午,免麟游荒賦。壬申,以梁清標為兵部尚書。丁丑,尚可喜復揭陽、普寧、澄海三縣。

五月辛卯,免大寧荒賦。癸巳,幸南苑。己亥,以羅託為鑲藍旗滿洲固山額真。覺羅郎球免。命明安達禮為理籓院尚書。以張懸錫為宣大總督。免荊門、京山等十一州縣,襄陽衛上年災賦。

閏五月戊申,幸瀛臺。丙辰,廣西都康等府土官來降。己未,乾清宮、坤寧宮、交泰殿及景仁、永壽、承乾、翊坤、鍾粹、儲秀宮成。以郎廷佐為江南江西總督,劉漢祚為福建巡撫。丙寅,以張朝璘為江西巡撫。

六月己丑,諭曰:「滿洲家人皆征戰所得,故立嚴法以儆逋逃。比年株連無已,朕心惻焉。念此僕隸,亦皆人子。茍以恩結,寧不知感。若任情因辱,雖嚴何益。嗣後宜體朕意。」壬辰,莒州地震有聲。庚子,免桃源上年荒賦。辛丑,容美土司田吉麟降。癸卯,命固山額真郎賽駐防福建。撤直省督催稅糧滿官。寧化賊帥黃素禾來降。

秋七月丁未朔,享太廟。戊申,官軍敗明桂王將龍韜於廣西,斬之。己酉,和碩襄親王博穆博果爾薨。庚戌,鄭成功將黃梧等以海澄來降。壬子,上初禦乾清宮。癸丑,大赦。戊午,以佟延年為甘肅巡撫。

八月戊寅,免廣信、饒州、吉安上年災賦。己丑,免莆田、仙游、興平衛十一、十二兩年災賦。辛卯,賑畿輔。壬辰,封黃梧為海澄公。停滿官榷關。癸巳,鄭成功軍陷閩安鎮,進圍福州,官軍擊卻之。丁酉,免順天比年災賦。己亥,免靖遠、洮岷等衛災賦。辛丑,命三年大閱,著為令。乙巳,免大同上年災賦。

九月丙午,官軍敗鄭成功將於夏關,又敗之於衡水洋,遂復舟山。癸亥,鄭成功將官顧忠來降。壬申,追封和碩肅親王豪格為和碩武肅親王。

冬十月丁丑,以蔣國柱為安徽巡撫,提督操江。戊寅,設登聞鼓。己卯,免宣府災賦,延綏鎮神木縣十之三。庚辰,四川賊帥鄧希明、張元凱率眾降。甲午,以胡全才為湖廣總督。乙未,幸南苑。丙申,以張尚撫治鄖陽。辛丑,官軍復辰州。壬寅,免和順縣災賦十之三。永順土司彭弘澍率所屬三州六司三百八十峒來降。癸卯,命陳之遴還京。

十一月丙午,還宮。丁未,興京陵工成。庚戌,祀天於圜丘。辛亥,幸南苑。申嚴左道之禁。戊午,免清水縣、鳳翔所災賦。丙寅,以張長庚為湖廣巡撫。免海州荒賦。辛未,免洛川災賦。

十二月己卯,冊內大臣鄂碩女董鄂氏為皇貴妃,頒恩赦。戊子,還宮。己丑,封盆挫監挫為闡化王。乙未,以李廕祖為湖廣總督。丁酉,加上皇太后尊號曰昭聖慈壽恭簡安懿章慶皇太后。戊戌,頒恩赦。

是年,土謝圖親王巴達禮、卓禮克圖親王吳克善、達爾漢巴圖魯郡王滿硃習禮、固倫額駙阿布鼐親王來朝。朝鮮,荷蘭,吐魯番,烏斯藏闡化王,喀爾喀部索特拔、宜爾登諾顏、喇嘛塔爾多爾濟達爾漢諾顏、車臣汗、土謝圖汗,土謝圖汗下丹津喇嘛、戴青、額爾德尼喇嘛,厄魯特部達賴吳巴什臺吉、訥穆齊臺吉、阿巴賴諾顏、察罕臺吉、馬賴臺吉、什虎兒戴青、額爾德尼臺吉、顧實汗下色棱諾顏,索倫部達爾巴均來貢。喀爾喀土謝圖汗、宜爾登諾顏再至。

十四年春正月辛亥,祈穀於上帝,以太祖武皇帝配。癸丑,以魏裔介為左都御史。甲寅,宜爾德師還。乙卯,以張懸錫為直隸山東河南總督。官軍敗鄭成功將於烏龍江,又敗之於惠安縣。戊午,諭曰:「制科取士,計吏薦賢,皆朝廷公典。臣子乃以市恩,甚無謂也。師生之稱,必道德相成,授受有自,方足當之。豈可攀援權勢,無端親暱。考官所得,及薦舉屬吏,輒號門生。賄賂公行,徑竇百出,鉆營黨附,相煽成風。朕欲大小臣工杜絕弊私,恪守職事,犯者論罪。」修金陵寢。庚申,以盧崇峻為宣大總督。甲子,諭曰:「我國家之興,治兵有法。今八旗人民,怠於武事,遂至軍旅隳敝,不及曩時。皆由限年定額,考取生童,鄉會兩試,即得錄用,及各衙門考取他赤哈哈番、筆帖式,徒以文字得官,遷轉甚速,以故人樂趨之。其一切停止。」丁卯,封猛峨、塔爾納為多羅郡王,多爾博為多羅貝勒,皇貴妃父鄂碩為三等伯。

二月戊寅,祭社稷。命儒臣纂修易經。癸未,故明崇陽王硃蘊鈐等來降。丁酉,祭歷代帝王廟。己亥,寬隱匿逃人律。以賽音達理為正白旗漢軍固山額真。壬寅,山西雲鎮地震有聲。癸卯,免沔陽、益陽上年災賦。

三月己酉,奉太宗文皇帝配享圜丘及祈穀壇。多羅郡王塔爾納薨。壬子,奉太祖武皇帝、太宗文皇帝配享方澤。癸丑,以配享禮成,大赦天下。甲寅,詔求遺書。丙辰,復孔子位號曰至聖先師。丁卯,定遠大將軍濟度師還。

夏四月甲戌,興寧縣雷連十二峒瑤官龐國安等來降。丁丑,流鄭芝龍於寧古塔。癸未,四川保寧府威、茂二州地大震。乙酉,以濟席哈為正紅旗滿洲都統。丁亥,以久旱,恤刑獄。辛卯,禱雨於郊壇,未還宮,大雨。丁酉,幸南苑。戊戌,置盛京奉天府。

五月癸卯朔,日有食之。丙午,以道喇為正紅旗蒙古固山額真。甲寅,封濟度為和碩簡親王。丁巳,以覺羅伊圖為兵部尚書。戊午,還宮。

六月辛巳,免彰德、衛輝二府上年災賦。壬午,免武陵縣上年災賦。辛丑,洪承疇以疾解任。

秋七月丙辰,削左都御史魏裔介職,仍戴罪辦事。庚申,以硃之錫為河道總督。

八月壬申,命敦拜為總管,駐防盛京。己丑,免山西荒地逃丁徭賦。丙申,鄭成功犯臺州,紹臺道蔡瓊枝叛,降於賊。丁酉,賚八旗貧丁。

九月辛丑,以亢得時為漕運總督,李國英為川陜三邊總督。丙午,初御經筵。以賈漢復為河南巡撫。癸丑,以高民瞻為四川巡撫。停直省秋決。丙寅,官軍復閩安鎮。丁卯,京師地震有聲。戊辰,詔曰:「自古變不虛生,率由人事。朕親政七載,政事有乖,致災譴見告,地震有聲。朕躬修省,文武群臣亦宜協心盡職。朕有闕失,輔臣陳奏毋隱。」

冬十月壬申,以開日講祭告先師孔子於弘德殿。免新樂上年災賦。癸酉,命固山額真趙布泰駐防江寧。丙子,皇第四子生。修賦役全書。辛巳,幸南苑。乙酉,閱武。丁亥,修孔子廟。戊子,還宮。庚寅,改梁化鳳為水師總兵官,駐防崇明。甲午,順天考官李振鄴、張我樸等坐受賄棄市。乙未,昭事殿、奉先殿成。

十一月壬寅,幸南苑。皇第五子常寧生。丙午,進安郡王岳樂為親王。庚戌,免吉水等八縣災賦。戊午,免霸、寶坻等二十八州縣,保安等四衛災賦。辛酉,荊州賊田國欽等來降。壬戌,明桂王將孫可望來降。固山貝子吞齊喀以罪削爵。

十二月癸酉,復命洪承疇經略五省,同羅託等取貴州。免新建、豐城災賦。甲戌,封孫可望為義王。癸未,命吳三桂自四川,趙布泰自廣西,羅託自湖南取貴州。丙戌,明桂王將譚新傳等降。丙申,以皇太后疾愈,賚旗兵,賑貧民。

是年,朝鮮,喀爾喀部畢席勒爾圖汗、冰圖臺吉、額爾德尼韋徵諾顏、吳巴什諾顏、土謝圖汗下完書克諾顏,厄魯特部敖齊爾圖臺吉子伊拉古克三、班第大胡土克圖、綽克圖臺吉、巴圖魯臺吉、達賴烏巴什臺吉,索倫部馬魯喀、虎爾格吳爾達爾漢,東夷託科羅氏、南迪歐,達賴喇嘛、班禪胡土克圖均來貢。朝鮮三至。

十五年春正月庚子,大赦。詔曰:「帝王孝治天下,禮莫大乎事親。比者皇太后聖躬違和,朕夙夜憂懼。賴荷天眷,今已大安。遘茲大慶,宜沛殊恩。其自王公以下,中外臣僚,並加恩賚。直省逋賦,悉與豁免。吏民一切詿誤,咸赦除之。」壬寅,停祭堂子。以多羅信郡王多尼為安遠靖寇大將軍,率師征雲南。戊午,祀圜丘,己未,祀方澤,辛酉,祀太廟社稷,以太后疾愈故。皇第四子薨。丙寅,以周召南為延綏巡撫。

二月甲戌,賑畿輔。甲申,免武清、漷上年災賦。己丑,減遼陽稅額。辛卯,川東賊帥張京等來降。甲午,命部院官各條陳事宜。乙未,御經筵。

三月辛丑,李定國黨閆維龍等陷橫州,官軍擊走之。甲辰,內監吳良輔以受賄伏誅。壬子,免襄陽、鄖陽荒賦。戊午,追封科爾沁巴圖魯王女為悼妃。甲子,追封皇第四子為和碩榮親王。

夏四月辛未,賜孫承恩等進士及第出身有差。丙子,官軍敗賊於合州,克重慶。癸未,免江夏等七縣十三年災賦。丙戌,較射於景山。辛卯,免淳化荒賦。大學士王永吉以罪免。壬辰,大學士陳之遴復以罪流盛京。

五月丁酉朔,日有食之。癸卯,調衛周祚為吏部尚書。戊申,以劉昌為工部尚書。更定銓選法。辛亥,鄭成功將犯澄海,游擊劉進忠以城叛,降於成功。壬子,免山東十一年以前灶丁逋課。己未,較射於景山。辛酉,裁詹事府官。壬戌,廣西賊將賀九儀犯賓州,官兵擊敗之。癸亥,以胡世安、衛周祚、李霨為內院大學士。甲子,官軍復沅靖,進取貴陽、平越、鎮遠等府,南丹、那地、獨山等州,撫寧土司俱降。

六月戊辰,吳三桂等敗李定國將劉正國於三坡,克遵義,拔開州。辛未,以趙廷臣為貴州巡撫。壬申,以佟國器為浙江巡撫,蘇弘祖為南贛巡撫。丙子,官軍敗海寇於白沙。辛巳,以李棲鳳為兩廣總督。甲申,以王崇簡為禮部尚書。壬辰,免靖、沅陵等十五州縣及平溪九衛所額賦。癸巳,鄭成功犯溫州,陷平陽、瑞安。

秋七月己亥,裁宣大總督。己酉,以潘朝選為保定巡撫。庚戌,沙爾虎達擊羅剎,敗之。改內三院大學士為殿閣大學士。設翰林院及掌院學士官。增各道御史三十人。己未,免桂陽、衡陽等十州縣上年災賦。甲子,以巴哈、費揚古、郭邁、屠祿會、馬爾濟哈、鄂莫克圖、坤巴圖魯、鄔布格德墨爾根袍、喀蘭圖、鄂塞、博洛塞冷、巴特瑪、巴泰俱為內大臣,趙國祚為浙江總督,李率泰專督福建。

八月癸酉,以李顯貴為鑲白旗漢軍固山額真。丙子,敕諭多尼等,授以方略。李定國將王興及水西宣慰使安坤等來降。癸巳,御經筵。

九月丁酉,以孫塔為鑲藍旗蒙古固山額真。庚戌,更定理籓院大闢條例。己酉,以能圖為左都御史。壬子,賜鑲黃、正黃、正白三旗官校金。甲寅,改內院大學士覺羅巴哈納、金之俊為中和殿大學士,額色黑、成克鞏為保和殿大學士,蔣赫德、劉正宗為文華殿大學士,洪承疇、傅以漸、胡世安為武英殿大學士,衛周祚為文淵閣大學士,李霨為東閣大學士。己未,免福州、興化、建寧三府,福寧州十二、十三兩年荒賦。癸亥,發帑賜出征軍士家。

冬十月壬午,以祖重光為順天巡撫。荊州、襄陽、安陸霪雨,江溢,漂沒萬餘人。

十一月甲午朔,海寇犯洛陽內港,官軍擊敗之。乙未,免鄖陽、襄陽荒賦。庚子,定宮中女官員額品級。辛丑,免林縣災賦十之三。江南考官方猶、錢開宗等坐納賄棄市。

十二月壬申,以索渾為鑲白旗滿洲固山額真。甲戌,免五臺災賦。壬午,故明宗室硃議滃率眾降。乙酉,以鄔赫為禮部尚書。免山陰等八縣上年災賦。戊子,以明安達禮為安南將軍,率師駐防貴州。己丑,諭曰:「川、湖、雲、貴之人,皆朕臣庶,寇亂以來,久罹湯火。今大軍所至,有來歸者,加意拊循,令其得所。能效力建功者,不靳爵賞。」

是年,朝鮮,喀爾喀部竇爾格齊諾顏、噶爾當臺吉、土謝圖汗、畢席勒爾圖汗、丹津喇嘛,厄魯特部阿巴賴諾顏,車臣臺吉下車臣俄木布、鄂齊爾圖臺吉,索倫部達把代,庫爾喀部塔爾善,使犬國頭目替爾庫,達賴喇嘛俱來貢。朝鮮、喀爾喀土謝圖汗、厄魯特阿巴賴諾顏再至。

十六年春正月甲午,桂王將譚文犯重慶,其弟譚詣殺之,及譚弘等來降。丁酉,以徐永正為福建巡撫。庚子,多尼克雲南,以捷聞。初,多尼、吳三桂、趙布泰會師於平越府之楊老堡,分三路取雲南。多尼自貴陽入,渡盤江至松嶺衛,與白文選遇,大敗之。三桂自遵義至七星關,不得進,乃由水西間道趨烏撒。趙布泰自都勻至盤江之羅顏渡,敗守將李成爵於山谷口,又敗李定國於雙河口,所向皆捷,遂俱抵雲南,入省城。李定國、白文選奉桂王奔永昌。癸卯,以林天擎為雲南巡撫。甲辰,以巴海為昂邦章京,駐防寧古塔。辛亥,賜外籓蒙古諸王貧乏者馬牛羊。癸丑,以趙廷臣為雲貴總督,卞三元為貴州巡撫。

二月丙寅,免潼關衛辛莊等屯上年災賦。丁卯,海寇犯溫州,官軍擊敗之。庚午,以雲、貴蕩平,命今秋舉會試。辛未,免荊州、潛江等九州縣及沔陽、安陸二衛上年災賦。丙子,命羅託等班師,明安達禮駐防荊州。壬午,以許文秀為山東巡撫。

三月丙申,以蔣國柱為江寧巡撫。己亥,以張仲第為延綏巡撫。戊申,以硃衣助為安徽巡撫。鄭成功犯浙江太平縣,官軍擊敗之。己酉,御經筵。甲寅,命吳三桂鎮雲南,尚可喜鎮廣東,耿繼茂鎮四川。丁巳,免襄陽等六縣災賦。

閏三月壬戌,大學士胡世安以疾解任。丁卯,定犯贓例,滿十兩者流席北,應杖責者不準折贖。甲申,免鍾祥縣上年災賦。圖海有罪,免。丙戌,封譚弘為慕義侯,譚詣為鄉化侯。丁亥,以張自德為陜西巡撫。

夏四月甲寅,多尼、吳三桂軍克鎮南州,白文選縱火燒瀾滄江鐵橋遁走。我軍進克永昌,李定國奉桂王走騰越,伏兵於磨盤山,我軍力戰,復克騰越。

五月壬戌,廣西南寧、太平、思恩諸府平。己巳,以劉秉政為寧夏巡撫。晉封滿硃習禮為和碩達爾漢巴圖魯親王。戊寅,官軍擊成功於定關,敗之,斬獲甚眾。辛巳,發內帑銀三十萬兩,以其半賑雲、貴窮黎,其半給徵兵餉。

六月庚子,朝鮮國王李淏薨。壬子,鄭成功陷鎮江府。

秋七月丁卯,以達素為安南將軍,同索洪、賴塔等率師征鄭成功。丙子,鄭成功犯江寧。庚辰,幸南苑。甲申,還宮。

八月己丑朔,江南官軍破鄭成功於高山,擒提督甘煇等,燒敵船五百餘艘。成功敗遁,我軍追至瓜州,敵兵大潰。先是,成功擁師十餘萬,戰艦數千,抵江寧城外,列八十三營,絡繹不絕,設大砲、地雷、雲梯、木柵,為久困之計,軍容甚盛。我軍噶褚哈、馬爾賽等自荊州以舟師來援,會蘇松水師總兵官梁化鳳及游擊徐登第、參將張國俊等各以軍至,總督郎廷佐合軍會戰,水陸並進,遂以捷聞。庚寅,御經筵。癸巳,幸南苑。以劉之源為鎮海大將軍,同梅勒章京張元勛等駐防鎮江。以蔡士英為鳳陽巡撫,總督漕運;宜永貴為安徽巡撫,提督操江。丙申,安南國都將武公恣遣使納款於洪承疇軍前。戊戌,還宮。甲辰,鄭成功復犯崇明,官軍擊敗之。乙巳,幸南苑。丙午,還宮。

九月庚申,免臺州四年至十年被寇稅賦。乙亥,賜陸元文等進士及第出身有差。丁丑,以杜立德為刑部尚書。戊寅,予故朝鮮國王李淏謚,封世子■B3為國王。庚辰,以海爾圖為鑲藍旗漢軍固山額真。辛巳,尊興京祖陵為永陵。甲申,幸南苑。

冬十月庚戌,洪承疇以疾解經略任。甲寅,奈曼部達爾漢郡王阿漢以罪削爵為庶人。

十一月己未,論故巽親王滿達海、端重親王博洛、敬謹親王尼堪前罪,削巽親王、端重親王爵,降其子為多羅貝勒。敬謹親王獨免。壬戌,以公渥赫、公樸爾盆為內大臣。丙寅,上獵於近畿。壬申,次昌平州,上酹酒明崇禎帝陵,遣學士麻勒吉祭王承恩墓。甲戌,遣官祭明帝諸陵,並增陵戶,加修葺,禁樵採。戊寅,皇第六子奇授生。己卯,次湯泉。甲申,次三屯營。追謚明崇禎帝為莊烈愍皇帝。丙戌,吳三桂取沅江。

十二月戊戌,還京。乙巳,定世職承襲例。庚戌,加公主封號。壬子,命耿繼茂移駐廣西。

是年,朝鮮,喀爾喀部丹津喇嘛、土謝圖汗、車臣汗、畢席勒爾圖汗、魯布臧諾顏、車臣濟農、昆都倫託音、土謝圖汗下多爾濟臺吉,厄魯特部阿布賴諾顏、達來吳霸西諾顏、俄齊爾圖臺吉,黑龍江能吉勒屯頭目韓批理,索倫部胡爾格烏爾達爾漢俱來貢。朝鮮,喀爾喀部土謝圖汗、丹津喇嘛再至。

十七年春正月丙寅,以硃國治為江寧巡撫。庚辰,京師文廟成。以能圖為刑部尚書。辛巳,詔曰:「自古帝王,統御寰區,治效已臻,則樂以天下;化理未奏,則罪在朕躬。敬天勤民,道不越此。朕續承祖宗鴻緒,兢兢圖治,十有七年。乃民生猶未盡遂,貪吏猶未盡除,滇、黔伏戎未靖,徵調時聞。反復思維,朕實不德,負上天之簡畀,忝祖宗之寄託,虛太后教育之恩,孤四海萬民之望。每懷及此,罔敢即安。茲以本年正月,祭告天地、太廟、社稷,抒忱引責。自今以後,元旦、冬至及朕壽令節慶賀表章,俱行停止。特頒恩赦,官民除十惡死罪外,悉減一等,軍流以下,咸赦除之。直省逋賦,概予豁免。有功者錄,孝義者旌。誕告中外,咸使聞知。」免洮州衛上年災賦。甲申,免莒、寧陽十二州縣上年災賦。

二月戊子,詔京官大學士、尚書自陳。其三品以下,親加甄別。吳三桂軍破賊於普洱。征南將軍趙布泰師還。壬辰,尚書劉昌自陳年老,致仕。癸巳,免貴陽等六府及土司上年災賦。復設鳳陽巡撫,駐泰州。戊戌,甄察直省督撫及京職三品以上漢官,石申、馮溥等錄敘黜降有差。壬寅,以林起龍為鳳陽巡撫。免淮、揚、鳳三府,徐州上年災賦。定每年孟春合祭天地日月及諸神於大享殿。癸卯,諭禮部:「向來孟春祈穀禮於大享殿舉行,今既行合祭禮於大享殿,以後祈穀禮於圜丘舉行。」壬子,免梁城所上年災賦。

三月癸亥,定平西、靖南二籓兵制。甲子,以史紀功為浙江巡撫。辛未,諭禮部:「朕載稽舊制,歲終祫祭之外,有奉先殿合祭之禮。自後元旦、皇太后萬壽及朕壽節,合祀於奉先殿。其詳議禮儀以聞。」論陷鎮江罪,革巡撫蔣國柱、提督管效忠職,免死為奴,協領費雅柱等棄市。甲戌,定固山額真漢稱曰都統,梅勒章京曰副都統,甲喇章京曰參領,牛錄章京曰佐領,昂邦章京曰總管。滿仍其舊。以袁懋功為雲南巡撫。丙子,御經筵。癸未,定王、貝勒、貝子、公妻女封號。甲申,更定民公、侯、伯以下,章京以上盔纓制。

夏四月丙戌,免寶坻、豐潤、武清上年災賦。甲午,以張長庚為湖廣總督。丙申,以劉祚遠為保定巡撫,張椿為陜西巡撫。辛丑,詔定匿災不報罪。癸卯,以白秉貞撫治鄖陽。丙午,皇第七子隆禧生。己酉,合祀天地於大享殿。

五月乙卯朔,以覺羅伊圖為吏部尚書。庚申,免綏德、膚施五州縣上年災賦。甲子,以阿思哈為兵部尚書,蘇納海為工部尚書。甲戌,以佟壯年為正藍旗漢軍都統,郭爾泰為鑲白旗蒙古都統。免沅州、鎮遠二衛災賦。己卯,詔曰:「前者屢詔引咎責躬,由今思之,皆具文而鮮實益。且十二、十三年間,時有過舉,經言官指陳,雖加處分,而此心介然未釋。今上天示儆,亢旱癘疫,災眚疊至。寇盜未息,民生困悴。用是深自刻責,夙夜靡寧。從前以言獲罪者,吏部列名具奏。凡國計民生利害,及朕躬闕失,各直言無隱。」庚辰,以張天福為正黃旗漢軍都統。壬午,覺羅巴哈納等以旱引罪自陳。上曰:「朕以旱災迭見,下詔責躬。卿等合辭引罪,是仍視為具文,非朕實圖改過意也。卿等職司票擬,僅守成規,未能各出所見,佐朕不逮。是皆朕不能委任大臣之咎。自後專加委任,其占力贊襄,秉公持正,以副朕懷。」多羅信郡王多尼師還。癸未,雲南土司那侖來降。

六月乙酉,始命翰林官於景運門入直。以阿思哈兼攝左都御史事。戊子,遣官省獄。以楊茂勛為湖廣巡撫。免澧、巴陵十二州縣及岳州等衛上年災賦。己丑,增祀商中宗、高宗、周成王、康王、漢文帝、宋仁宗、明孝宗於歷代帝王廟。罷遼太祖、金太祖、元太祖廟祀及宋臣潘美、張浚從祀。以蘇納海為兵部尚書。癸巳,以穆里瑪為工部尚書,白色純署河道總督。丙申,上以禱雨步至南郊齋宿。是日,大雨。戊戌,祀天於圜丘,又雨。己亥,大學士劉正宗、成克鞏、魏裔介以罪免。辛丑,命修舉天下名山大川、古帝王聖賢祀典。

秋七月甲寅朔,以霍達兼攝左都御史事。和碩簡親王濟度薨。戊午,編降兵為忠勇、義勇等十營,隸吳三桂,以降將馬寶等統之。丁卯,移祀北嶽於渾源州。己巳,免荊州、祁陽十三州縣及衡州等衛上年災賦。庚午,免均、保康七州縣及鄖、襄二衛上年荒賦。以楊義為工部尚書。丁丑,命耿繼茂移駐福建。寧古塔總管巴海敗羅剎於使犬部地,招撫費牙喀十五村一百二十餘戶。改徙席北流犯於寧古塔。庚辰,停遣御史巡按直省。壬午,以羅託為安南將軍,率師征鄭成功。癸未,能圖免。

八月丁亥,以彭有義為河南巡撫。己丑,免化、茂名四州縣及高州所上年災賦。庚寅,免武岡上年災賦。丙申,雲南車裏土司刀木禱來降。戊戌,以沈永忠為掛印將軍,鎮守廣東。辛丑,以愛星阿為定西將軍,徵李定國。壬寅,皇貴妃董鄂氏薨,輟朝五日。甲辰,追封董鄂氏為皇后。己酉,降將郝承裔叛,陷邛州,圍嘉定,官軍擊敗之。辛亥,以穆里瑪為鑲黃旗滿洲都統。

九月癸丑朔,安南國王黎維祺奉表來降。甲子,以佟鳳彩為四川巡撫。丁卯,偽將鄧耀據海康,官軍擊走之。壬申,以王登聯為保定巡撫。甲戌,免保昌六縣及南、韶二所十四年災賦。戊寅,幸昌平,觀故明諸陵。己卯,還宮。

冬十月丁亥,以覺羅雅布蘭為刑部尚書。戊子,罷朝鮮貢鷹。辛卯,幸近郊。甲午,還宮。己亥,以郭科為工部尚書。丁未,免睢、商丘十一州縣及歸德、睢陽二衛上年災賦。

十一月甲寅,免趙、柏鄉四州縣及真定衛上年災賦。乙卯,免寧、上饒四十六州縣上年災賦。丁巳,撤直省恤刑官。安南將軍明安達禮師還。辛酉,大學士劉正宗以罪免。壬戌,復遣御史巡按直省。乙丑,敬謹親王尼思哈薨。戊寅,免睢、虞城六州縣災賦。庚辰,免五河、安東上年災賦。

十二月癸巳,免邳、宿遷四州縣災賦。戊戌,免慶都災賦。甲辰,皇第八子永幹生。

是歲,朝鮮,喀爾喀部丹津喇嘛,土謝圖汗下萬舒克諾顏、七旗,厄魯特部鄂齊里汗,達賴喇嘛、班禪胡土克圖,阿里祿克山托因,虎爾哈部宜訥克,俄羅斯部察罕汗,使鹿索倫部頭目布勒、蘇定噶、索朗阿達爾漢子查木蘇來貢。朝鮮再至。

十八年春正月壬子,上不豫。丙辰,大漸。赦死罪以下。丁巳,崩於養心殿,年二十四。遺詔曰:「朕以涼德,承嗣丕基,十八年於茲矣。自親政以來,紀綱法度,用人行政,不能仰法太祖、太宗謨烈,因循悠忽,茍且目前。且漸習漢俗,於淳樸舊制,日有更張。以致國治未臻,民生未遂,是朕之罪一也。朕自弱齡,即遇皇考太宗皇帝上賓,教訓撫養,惟聖母皇太后慈育是依。隆恩罔極,高厚莫酬,朝夕趨承,冀盡孝養。今不幸子道不終,誠悃未遂,是朕之罪一也。皇考賓天,朕止六歲,不能服衰絰行三年喪,終天抱憾。惟侍奉皇太后順志承顏,且冀萬年之後,庶盡子職,少抒前憾。今永違膝下,反上廑聖母哀痛,是朕之罪一也。宗室諸王貝勒等,皆太祖、太宗子孫,為國籓翰,理宜優遇,以示展親。朕於諸王貝勒,晉接既疏,恩惠復鮮,情誼暌隔,友愛之道未周,是朕之罪一也。滿洲諸臣,或歷世竭忠,或累年效力,宜加倚託,盡厥猷為。朕不能信任,有才莫展。且明季失國,多由偏用文臣。朕不以為戒,委任漢官,即部院印信,間亦令漢官掌管。致滿臣無心任事,精力懈弛,是朕之罪一也。朕夙性好高,不能虛己延納。於用人之際,務求其德與己侔,未能隨才器使,致每嘆乏人。若舍短錄長,則人有微技,亦獲見用,豈遂至於舉世無才,是朕之罪一也。設官分職,惟德是用,進退黜陟,不可忽視。朕於廷臣,明知其不肖,不即罷斥,仍復優容姑息。如劉正宗者,偏私躁忌,朕已洞悉於心,乃容其久任政地。可謂見賢而不能舉,見不肖而不能退,是朕之罪一也。國用浩繁,兵餉不足。而金花錢糧,盡給宮中之費,未嘗節省發施。及度支告匱,每令諸王大臣會議,未能別有奇策,止議裁減俸祿,以贍軍餉。厚己薄人,益上損下,是朕之罪一也。經營殿宇,造作器具,務極精工。無益之地,糜費甚多。乃不自省察,罔體民艱,是朕之罪一也。端敬皇后於皇太后克盡孝道,輔佐朕躬,內政聿修。朕仰奉慈綸,追念賢淑,喪祭典禮,過從優厚。不能以禮止情,諸事太過,逾濫不經,是朕之罪一也。祖宗創業,未嘗任用中官。且明朝亡國,亦因委用宦寺。朕明知其弊,不以為戒。設立內十三衙門,委用任使,與明無異。致營私作弊,更逾往時,是朕之罪一也。朕性耽閒靜,常圖安逸,燕處深宮,御朝絕少。致與廷臣接見稀疏,上下情誼否塞,是朕之罪一也。人之行事,孰能無過?在朕日理萬幾,豈能一無違錯?惟聽言納諫,則有過必知。朕每自恃聰明,不能聽納。古云:『良賈深藏若虛,君子盛德,容貌若愚。』朕於斯言,大相違背。以致臣工緘默,不肯進言,是朕之罪一也。朕既知有過,每自刻責生悔。乃徒尚虛文,未能省改,過端日積,愆戾愈多,是朕之罪一也。太祖、太宗創垂基業,所關至重。元良儲嗣,不可久虛。朕子玄燁,佟氏妃所生,岐嶷穎慧,克承宗祧,茲立為皇太子。即遵典制,持服二十七日,釋服即皇帝位。特命內大臣索尼、蘇克薩哈、遏必隆、鰲拜為輔臣。伊等皆勛舊重臣,朕以腹心寄託。其勉矢忠藎,保翊沖主,佐理政務。布告中外,咸使聞知。」

三月癸酉,上尊謚曰體天隆運英睿欽文大德弘功至仁純孝章皇帝,廟號世祖,葬孝陵。累上尊謚曰體天隆運定統建極英睿欽文顯武大德弘功至仁純孝章皇帝。

論曰:順治之初,睿王攝政。入關定鼎,奄宅區夏。然兵事方殷,休養生息,未遑及之也。迨帝親總萬幾,勤政愛民,孜孜求治。清賦役以革橫徵,定律令以滌冤濫。蠲租貸賦,史不絕書。踐阼十有八年,登水火之民於衽席。雖景命不融,而丕基已鞏。至於彌留之際,省躬自責,布告臣民。禹、湯罪己,不啻過之。書曰:「亶聰明作元後,元後為民父母。」其世祖之謂矣。


\end{pinyinscope}