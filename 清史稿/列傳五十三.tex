\article{列傳五十三}

\begin{pinyinscope}
葉方藹沈荃勵杜訥子廷儀孫宗萬徐元珙許三禮王士禎

韓菼湯右曾

葉方藹,字子吉,江南昆山人。順治十六年一甲三名進士,授編修。江南奏銷案起,坐奪官。尋授上林苑蕃育署丞。事白,還故官。康熙十二年,充日講起居注官。十四年,遷國子監司業,再遷侍講。宴瀛臺,群臣皆進詩賦,方藹制八箴以獻,上甚悅,命撰太極圖論以進,賜貂裘、文綺。十五年,遷左庶子,再遷侍講學士。十六年,命充孝經衍義總裁,進講通鑒。上問:「諸葛亮何如伊尹?」方藹對曰:「伊尹聖人,可比孔子;諸葛亮大賢,可比顏淵。」上首肯。講中庸,上問:「知行孰重?」對曰:「宋臣硃熹之說,以次序言,則知先行後;以功夫言,則知輕行重。」上曰:「畢竟行重,若不能行,知亦虛知耳。」轉侍讀學士。十七年,充鑒古輯覽、皇輿表總裁,經筵講官,直南書房。上勤於典學,故事,以大臣二人日直,上特以屬方藹,兼掌院學士,兼禮部侍郎。

十八年,召試博學宏詞,命方藹閱卷,總裁明史。十九年,尚書講義成。上以講幄勞,加方藹尚書銜。上講易噬嗑卦辭,方藹與同官庫勒納進所撰乾坤二卦總論,上覽竟,諭曰:「卦爻義各有不同,即如噬嗑卦中四爻主用刑者言,初上二爻主受刑者言,必得總論發揮,庶全卦之義了然,諸卦可依此撰進。」二十年,授刑部侍郎。二十一年,卒,遣奠茶酒,賜白金二百。上以方藹久侍講幄,啟沃勤勞,命優恤,賜謚文敏。

方藹初釋褐,以文章受知世祖。家居時,有密陳其居鄉不法者,下其事江蘇巡撫田雯覈覆。雯以鄉評入告,上曰:「朕固知方藹不如是也!」其後事聖祖,直內廷,眷遇優渥。方藹故廉謹,其卒,以板扉為臥榻,支以四甕,布帳多補綴,無以為斂,見者以為難能。

沈荃,字貞蕤,江南華亭人。順治九年一甲三名進士,授編修。世祖擇翰林官外轉,荃出為大梁道副使。劇盜董天祿、牛光天剽掠許、潁間,荃督兵捕治,殲其渠,群盜皆散去。禹州盜倚竹園為巢,殺人越貨,荃遣吏卒收捕,發土得尸十餘,悉按誅之。尋署按察使,疏言:「師方南征,必經南陽、汝寧諸府,供應疲苦,亢村、郭店諸驛,官死夫逃,請敕均撥驛站銀兩。師既入楚,留馬彰德,役民飼秣,請敕以懷慶、衛輝、廣平、順德、大名諸府更番分駐。各縣常平倉蓄穀太寡,請敕定額:大縣五六百石,小縣三四百石。開封自河決後,城垣淤圮,官吏分駐各邑,鄉闈暫移輝縣。近奉旨修復汴城,請敕籌撥錢糧,督倡興工。河南土地,原有上中下等則,向因疆井混淆,一例派糧。今查勘漸定,請敕視萬歷年間則例,照地派糧。河南兵額一萬二千,奉旨缺額免補,有汰無增,駐防分汛,每苦不足,請敕仍許募補足額。」俱下部議行。

康熙元年,以憂歸。六年,授直隸通薊道,坐事左遷。九年,授浙江寧波同知。未上官,特旨召對,命作各體書,稱旨,詔以原品內用。十年,授侍講,直南書房。十一年,轉侍讀。十二年,充日講起居注官。十三年,擢國子監祭酒。十五年,遷少詹事。十六年,擢詹事。

十八年,旱,求直言。時更定新例,罪人當流者徙烏喇,下廷臣議。荃謂:「烏喇去蒙古三四千里,地極寒,人畜多凍死。今罪不至死者,乃遣流,而更驅之死地,宜如舊例便。」疏上,有旨令畫一,荃持前議益堅,且曰:「此議行,三日不雨者,甘服欺罔罪。」上改容納之。越二日,天竟雨,例得罷。十九年,上以講幄勞,加荃禮部侍郎銜。二十一年正月,乾清宮宴廷臣,賦柏梁體詩,荃與焉。二十三年,卒。上以荃貧,賜白金五百。

子宗敬,二十七年進士,改庶吉士,以編修入直,上命作書,因諭大學士李光地曰:「朕初學書,宗敬父荃指陳得失。至今作字,未嘗不思其勤也。」宗敬官至太常寺少卿。

勵杜訥,字近公,直隸靜海人。勵氏自鎮海北遷,訥以杜姓補諸生。康熙二年,纂世祖實錄,選善書之士,訥試第一,赴館繕錄。書成敘勞,授福建福寧州同,命留直南書房,食六品俸。十九年,授編修,充日講起居注官。二十一年,奏請復勵姓。聖祖方閱通鑒綱目,杜訥與學士張英侍,閱竟,杜訥請以御批宣示史館,下禮部翰林院會議,如所請。二十七年,遷贊善。二十九年,遷侍講,改光祿寺少卿。三十六年,遷通政司參議。三十七年,遷太僕寺卿,再遷宗人府府丞。

三十九年,遷左副都御史。疏言:「督撫大吏,朝廷畀以百餘城吏治、數千里民生,責任至重;若託詞鎮靜,漸成悠忽,不過以期會簿書忝封疆之寄。請敕各督撫年終匯奏若何察吏安民、興利除弊,以備清覽;不實,則治以欺罔之罪:庶時時警勉,不敢優游草率,貽誤地方。籓司專掌錢穀,臬司專掌刑名,州縣之錢糧有無虧空,定案之爰書有無駁審,詳實並列,則籓臬之優劣亦無遁情。」議如所請。又言:「提鎮保送將弁,時有騎射甚劣並年老之員,經特旨甄別。典戎要務,首在考察將弁,請敕部將各提鎮所屬引見不稱旨之員,匯冊呈覽,並定處分。」下詔所司飭行。四十二年,擢刑部侍郎。卒。

杜訥學行醇粹,直禁廷二十餘年,無纖芥過失。四十四年,上駐蹕靜海,敕獎杜訥謹慎勤勞,親定謚曰文恪,手書賜其家。雍正元年,贈禮部尚書。八年,祀賢良祠。高宗即位,加贈太子太傅。

子廷儀,字南湖。康熙三十九年進士,改庶吉士。四十一年,特命直南書房。四十三年,授編修,遭父喪,既終,充日講起居注官。累遷內閣學士,充經筵講官,擢翰林院掌院學士、兵部侍郎。雍正元年,遷刑部尚書。疏言各省常平倉穀,當責督撫覈實盤查,年終冊報;又請於古北口外設理事同知,檢察命、盜獄:並從之。二年,疏言各州縣團練民壯,當選習槍箭,勤加訓練,上韙之,下直省督撫實力奉行;又疏請分立內外監,內監居要犯,外監居輕犯,別為女監,另墻隔別:均報可。迭疏論監生考職,禁止私鹽,清查入官家產,各舉其叢弊所在,並下部議行。七年,加太子太傅,賜「矜慎平恕」榜。九年,調吏部,仍專管刑部事。十年,卒,謚文恭。

子宗萬,字滋大。康熙六十年進士,改庶吉士,授編修。雍正二年,命直南書房,充日講起居注官,督山西學政。六年,遷國子監司業,按試潞安。臨晉民解進朝詐稱御前總管,私書請託,宗萬疏發之,諭嘉獎,遷侍讀,命巡察山西。八年,巡撫石麟劾宗萬擾驛遞,並縱僕受賕,坐奪官。十年,起鴻臚寺少卿,仍直南書房。四遷至禮部侍郎,調刑部。乾隆元年,吏部劾宗萬保舉河員受請託,坐奪官。尋命直武英殿。七年,再起侍講學士,累遷通政使。直懋勤殿,纂秘殿珠林,遷左副都御史。擢工部侍郎,調刑部。十年,坐縱門客生事,復奪官,手詔詰責,命還里閉戶讀書。督撫那蘇圖劾宗萬縱弟占官地,命承修固安城工,免其罪。十六年,復起侍講學士,累遷光祿寺卿。二十四年,卒。

子守謙,嘉慶十年進士,官編修。

自杜訥以諸生受知遇,子孫繼起,四世皆入翰林。

徐元珙,字輯五,江南武進人。順治十二年進士,授刑部主事,遷員外郎。典廣西試,遷郎中。出為福建建寧道僉事,善治盜。移山西冀寧南道參議,遭母憂去。康熙十二年,起直隸口北道參議。時宣鎮未設府縣,但置同知分防。元珙和調將士,嚴斥堠,增亭障,葺城郭,修學舍,邊境晏然。入為光祿寺少卿,歷太僕寺卿、通政使。

二十四年,授太常寺卿。疏請釐正北郊配饗位次,略言:「本朝分祭南北郊。圜丘南鄉,三聖並配,甚鉅典也。獨方澤配位,臣不能無議。昭穆之位,分左右不分東西。圜丘南鄉,則東為左為昭,西為右為穆;地祇既北鄉,則西為左為昭,東為右為穆。蓋東西有定方,而左右無定位,從正位所鄉而殊。漢、唐地祇皆南鄉,至宋政和四年,引北牖答陰之義,始改北鄉,配位亦改焉。明嘉靖九年,建方澤壇,因宋制,地祇北鄉,而配位仍設於東,不應古禮。蓋其時禮官誤執以東為左,因循至今。然明配位止一太祖,或左或右,尚無越次之嫌。今三聖並配,左右易位,因之昭穆失序;況配位誤則從壇皆誤,即陵山從祀岳鎮者亦誤。揆諸典禮,實有未安,有待釐正。」疏入,下廷臣集議,學士徐乾學、韓菼皆韙元珙議,獨許三禮駁之,遂不行。語見三禮傳。

二十五年,遷左副都御史。疏請正北海祀典,略言:「唐望祭洛州,即今河南府。宋望祭孟州,即今懷慶府。明依宋制。說者謂懷慶屬濟源,潛通北海,故於此望祭焉。本朝定制,東海祀萊州,南海祀廣州,西海祀蒲州,皆為允當。獨北海仍祀懷慶,竊以岳鎮方位,當準皇都。往南祭北,於義未愜。謹按北鎮醫巫閭山在今奉天府境,山既為北鎮,川即可為北海,矧長白山水、黑龍、鴨綠諸江,悉朝宗於海。請更定北海之祭,就北鎮醫巫閭為便。或疑歷時已久,不可輒更。臣按北嶽祀恆山曲陽,積二千餘年,用科臣言改祀渾源州。嶽祭可更,何疑海祭?」疏入,議行。

二十六年,疏乞歸養。至家,父已前卒。二十七年,孝莊文皇后崩,赴闕哭臨。疾作,卒於京師,上聞而憫之,喪歸,許馳驛,恤如禮。

元珙尚風義,座主陳彩沒,妻妾繼逝,撫其一歲孤並其女,為營婚嫁,與己子無異。時論推其篤厚。

彩字美公,廣東順德人。順治九年進士,自編修出為江南常鎮道。康熙初,江南有大獄,諸生連染被逮,彩以輕刑全活之甚眾。

許三禮,字典三,河南安陽人。順治十八年進士,授浙江海寧知縣。海寧地瀕海,多盜,三禮練鄉勇,嚴保甲,擒盜首硃纘之等。益修城壕,築土城尖山、鳳凰山間,戍以土兵。築塘濬河,救災儲粟,教民以務本。立書院,延黃宗羲主講。在縣八年,聲譽甚美。

康熙八年,行取,授福建道御史。疏言:「漢儒董仲舒表章六經,其言道之大原出於天,與禪宗異學專主明心者不同。故宋儒程顥有儒道本天、釋教本心之辨。宜視宋時六大儒,從祀國學,進稱先賢。」下廷臣議,不果行。時雲、貴猶未定,三禮疏言蕩平後,察大吏宜嚴,蘇民困宜寬。

尋命巡視北城,太常寺卿徐元珙議北郊配位應改坐西鄉東,下廷臣集議,三禮曰:「陽生於子,極於巳,故祀天在冬至,位南郊南鄉;陰生於午,極於亥,故祀地在夏至,位北郊北鄉。答陰答陽,義各有取。配位者主道也,義在近尊者為上。故配天尚左,配地尚右,並居東。改之非是。」從之。尋疏請定武臣守制例,下廷臣集議,有謂本朝無此例者。三禮曰:「宋高宗紹興七年,岳飛聞母訃,解兵柄徒步歸廬山,廬墓三年。此往代守制例也。」遂定議武臣守制自此始。旋擢通政司右參議。二十七年,遷提督四譯館、太常寺少卿,再遷大理寺卿。

召對便殿,上曰:「河圖洛書,道治之原。一二三四五,六七八九十,忽金火易位何也?」對曰:「此即一陰一陽之道也。天地大德曰生,故河圖左旋,而相生為順數;洛書右轉,而相剋為逆數。一順一逆,位所由易也。」上曰:「既順何以逆?」對曰:「孤陽不生,獨陰不成。河圖自北而東,順以相生,木火土金水,就流行言;洛書自北而西,逆則相剋,上下四方中,就對待言。既五數在中,縱橫皆十五矣,惟剋乃所以生也。陰陽交則生變,變則生生不易。」上又問曰:「洪範九疇,皇建有極,謂人參三才,此說是乎?」對曰:「自天地開闢以來,賴有聖人,原治而不原亂者,天地之心;有治而不能無亂者,天地之數。數至則生聖人,撥亂而返之治,裁成輔相,以左右民,則聖人建極會極歸極之功也。聖人既能撥亂而返之治,始副天地長治之心,此人參三才之說,實理也,亦實事也。」上頗嘉美之。

遷順天府府尹。二十八年,遷右副都御史。再遷兵部督捕侍郎,以病告歸,未及行,卒。

三禮初師事孫奇逢,及在海寧,從黃宗羲游,官京師,有所疑,必貽書質宗羲。斅宋趙抃故事,旦晝所為,夜焚香告天,家居及在海寧,皆建告天樓。聖祖重道學,嘗以之稱三禮云。

王士禎,字貽上,山東新城人。幼慧,即能詩,舉於鄉,年十八。順治十二年,成進士。授江南揚州推官。侍郎葉成格被命駐江寧,按治通海寇獄,株連眾,士禎嚴反坐,寬無辜,所全活甚多。揚州鹺賈逋課數萬,逮系久不能償,士禎募款代輸之,事乃解。康熙三年,總督郎廷佐、巡撫張尚賢、河督硃之錫交章論薦,內擢禮部主事,累遷戶部郎中。十一年,典四川試,母憂歸,服闋,起故官。

上留意文學,嘗從容問大學士李霨:「今世博學善詩文者孰最?」霨以士禎對。復問馮溥、陳廷敬、張英,皆如霨言。召士禎入對懋勤殿,賦詩稱旨。改翰林院侍講,遷侍讀,入直南書房。漢臣自部曹改詞臣,自士禎始。上徵其詩,錄上三百篇,曰御覽集。

尋遷國子監祭酒,整條教,屏餽遺,獎拔皆知名士。與司業劉芳喆疏言:「漢、唐以來,以太牢祀孔子,加王號,尊以八佾、十二籩豆。至明嘉靖間,用張璁議,改為中祀,失尊崇之意。禮:祭從生者。天子祀其師,當用天子之禮樂。」又疏言:「自明去十哲封爵,稱冉子者凡三,未有辨別。宋周敦頤等六子改稱先賢,位漢、唐諸儒之上,世次殊有未安,宜予釐定。」又疏言:「田何受易商瞿,有功聖學,宜增祀。鄭康成注經百餘萬言,史稱純儒,宜復祀。」又疏言:「明儒曹端、章懋、蔡清、呂柟、羅洪先,並宜從祀。絳州貢生辛全,生際明末,以正學為己任,著述甚富,乞敕進遺書。」又請修監藏經史舊版。疏並下部議,以籩豆、樂舞、名號、位次,俟會典頒發遵循;增祀明儒及徵進遺書,俟明史告成覈定;修補南北監經史版,如所請行。

二十三年,遷少詹事。命祭告南海,父憂歸。二十九年,起原官,再遷兵部督捕侍郎。三十一年,調戶部。命祭告西嶽西鎮江瀆。三十七年,遷左都御史。會廷議省御史員額,士禎曰:「國初設御史六十,後減為四十,又減為二十四。天子耳目官,可增不可減。」卒從士禎議。

遷刑部尚書。故事,斷獄下九卿平議。士禎官副都御史,爭楊成獄得減等。官戶部侍郎,爭太平王訓、聊城於相元、齊河房得亮獄皆得減等,而衡陽左道蕭儒英,則又爭而置之法。徐起龍為曹氏所誣,則釋起龍而罪曹,案其所與私者,皆服罪。及長刑部,河南閻煥山、山西郭振羽、廣西竇子章皆以救父殺人論重闢,士禎曰:「此當論其救父與否,不當以梃刃定輕重。」改緩決,入奏,報可。

士禎以詩受知聖祖,被眷遇甚隆。四十年,乞假遷墓,上命予假五月,事畢還朝。四十三年,坐王五、吳謙獄罷。王五故工部匠役,捐納通判;謙太醫院官,坐索債毆斃負債者。下刑部,擬王五流徙,謙免議,士禎謂輕重懸殊,改王五但奪官。復下三法司嚴鞫,王五及謙並論死,又發謙囑託刑部主事馬世泰狀,士禎以瞻徇奪官。四十九年,上眷念諸舊臣,詔復職。五十年,卒。

明季文敝,諸言詩者,習袁宗道兄弟,則失之俚俗;宗鍾惺、譚友夏,則失之纖仄;斅陳子龍、李雯,軌轍正矣,則又失之膚廓。士禎姿稟既高,學問極博,與兄士祿、士祜並致力於詩,獨以神韻為宗。取司空圖所謂「味在酸咸外」、嚴羽所謂「羚羊掛角,無跡可尋」,標示指趣,自號漁洋山人。主持風雅數十年。同時趙執信始與立異,言詩中當有人在。既沒,或詆其才弱,然終不失為正宗也。

士禎初名士禛,卒後,以避世宗諱,追改士正。乾隆三十年,高宗與沈德潛論詩,及士正,諭曰:「士正績學工詩,在本朝諸家中,流派較正,宜示褒,為稽古者勸。」因追謚文簡。三十九年,復諭曰:「士正名以避廟諱致改,字與原名不相近,流傳日久,後世幾不復知為何人。今改為士禎,庶與弟兄行派不致淆亂。各館書籍記載,一體照改。」

韓菼,字元少,江南長洲人。讀書通五經,恬曠好山水。朋游飲酒,歡諧終日,而制行清嚴。特工制舉文。應順天鄉試,尚書徐乾學拔之遺卷中。康熙十二年,會試、殿試皆第一,授修撰,充日講起居注官。聖祖知其能文,命撰太極圖說以進,復諭進所作制舉文,召入弘德殿講大學。初世祖命纂孝經衍義未成,至是以菼專任纂修。十四年,典順天試。十五年,遷贊善。十六年,遷侍講。十七年,復典順天試。十八年,乞假歸。二十三年,起故官,尋轉侍讀。二十四年,上親試翰林,菼列第二,遷侍講學士。尋擢內閣學士。

二十六年,再假歸,築室西山。點勘諸經注疏,旁逮諸史。居八年,三十四年,召至京,命以原官總裁一統志。遷禮部侍郎,兼掌院學士。祭酒阿理瑚請以故大學士達海從祀文廟,下部議,菼謂:「從祀鉅典,論定匪易。達海造國書,一藝耳。」持不可。永定河工開事例,戶部請推廣,得捐納道府。菼謂道府不當捐納,御史鄭維孜疏言:「國子監生多江、浙人,有冒籍赴試者。請盡發原籍肄業。」菼曰:「京師首善地,遠人鄉化,方且聞風慕義而來。若因一二不肖,輒更定制,悉為驅除,太學且空,非國體。維孜言非是。」事得寢。三十九年,充經筵講官,授禮部尚書,教習庶吉士。四十一年,上疏乞解職,專意纂輯承修諸書,詔慰留之,並賜「篤志經學、潤色鴻業」榜。四十二年,再稱疾,上不悅,敕仍留原任。四十三年,再疏乞退,仍不允。是歲秋,卒,恤如禮。

菼負文章名,而立朝樹風概,敢言,與人有始終。其再假歸也,乾學方罷官家居,領書局洞庭山中。兩江總督傅臘塔構乾學,將興大獄,素交皆引去。菼旦暮造門,且就當事白其誣,乃已。其復起也,上遇之厚,嘗曰:「韓菼天下才,美風度,奏對誠實。」又曰:「菼學問優長,文章大雅,前代所僅有。所撰擬能道朕意中事。」會江寧布政使張萬祿蝕帑金三十餘萬金,總督阿山庇之,謂費由南巡。下廷臣議,有言阿山與有連,妄語罪當死。菼謂縱有連,情私而語公。忌者增益其語入告,上由是疏菼。及再謝病,詔責其教習庶吉士,每日率以飲酒多廢學;九卿集議,不為國事直言,惟事瞻徇。菼意不自得,病甚,飲不輟,至卒。乾隆十七年,高宗諭獎「菼雅學績文,湛深經術。所撰制義,清真雅正,開風氣之先,為藝林楷則」。追謚文懿。

子孝嗣,舉人;孝基,進士,官編修,菼卒,奉母不出十餘年。雍正初,召修明史。書成,移疾歸,年九十而終。

湯右曾,字西厓,浙江仁和人。康熙二十七年進士,改庶吉士,授編修。出典貴州試。三十九年,授刑科給事中。兩廣總督石琳疏言瓊州生黎以文武官吏婪索,激而為亂。上遣侍郎凱音布、學士邵希穆按治。右曾疏言:「揭帖言瓊州文武官往黎峒採取沈香、花梨致生釁,石琳及巡撫蕭永藻、提督殷化行平時絕不覺察,且黎亂在上年,遲且一載,始行題報,掩飾欺隱,請嚴加處分。」石琳等皆下吏議。四十年,疏請刊頒政治典訓及御制文集。

四十一年,轉戶部掌印給事中。初,以私錢多,改錢制輕小,使私鑄無所利,顧仍不止。上令仍鑄大錢,下廷臣議,改鑄大錢,其舊鑄小錢,期二年銷毀。右曾疏言:「改大錢宜遵聖諭,若毀小錢則民間必驚擾。且戶、工二部存錢八十四萬串,若議銷毀,工料耗折甚多。且二年中鑄出新錢不過一百萬串,豈能遍及各省?新錢無多,舊錢已毀,恐私鑄更繁,錢法愈壞。古者患錢重,則改輕而不廢重;患錢輕,則改重而不廢輕:使子母相權而行。新鑄重錢,每串作銀一兩;舊鑄輕錢作七錢:並聽行使。積久大錢流通,小錢自不行矣。」疏再下廷臣議,定新錢每重一錢四分,舊錢並行勿禁,如右曾議。

四十四年,提督河南學政。秩滿,巡撫汪灝疏言右曾取士公明。四十八年,遷奉天府府丞。四十九年,遷光祿寺卿。五十年,轉太常寺卿、通政使。五十一年,擢翰林院掌院學士。五十二年,授吏部侍郎。尚書富寧安、陳鵬翮皆廉辦有威棱,右曾貳之,銳意文案,糾剔是非。選人或挾大力以相要,必破其機紐,俾終不獲選。由是干進射利者,皆叢怨於吏部,而富寧安往蒞西師,鵬翮任事久,見知於上深,莫可搖動,遂爭為浮言撼右曾。六十年,命解右曾侍郎,仍專領掌院學士。六十一年,卒。

右曾少工詩,清遠鮮潤。其後師事王士禎,稱入室。使貴州後,風格益進,鍛鍊澄汰,神韻泠然。右曾朝熱河行在,上命進所為詩,右曾方詠文光果,即以進上。上為和詩,有句曰「叢香密葉待詩公」,右曾自定集,遂取是詩冠首。

論曰:方藹、荃、杜訥以文學直內廷,其結主知,尤在於廉謙。元珙、三禮議禮各申其所見,有當於經指。士禎以詩被遇,清和粹美,蔚為一代正宗。菼於文亦然,久而論定,並邀補謚,增文字之重。右曾師事士禎,繼以詩被遇。論者謂自明弘治、正德以後一百五十年,而文章復在臺閣,為聖祖崇儒右文之效云。


\end{pinyinscope}