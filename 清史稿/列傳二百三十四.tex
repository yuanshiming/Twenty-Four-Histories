\article{列傳二百三十四}

\begin{pinyinscope}
丁寶楨李瀚章楊昌濬張樹聲弟樹屏衛榮光

劉秉璋陳士傑陶模李興銳史念祖

丁寶楨,字稚璜,貴州平遠人。咸豐三年進士,選庶吉士。母喪里居,遵義楊隆喜反,斥家財募壯士八百捍鄉里,戰始不利,繼獲大勝。服闋,會苗、教蜂起,巡撫蔣霨遠奏留軍,特旨授編修,增募至四千人,復平越、獨山諸城。十年,除知嶽州府,始罷遣所募兵。虧餉巨萬,乃陳五百金案上,語眾曰:「與諸君共事久,今庫饋詘,徒手歸,奈何?」眾泣曰:「公毀家紓難,我等敢他求乎?」遂去。越歲,調長沙。有客軍數千,以無主將譁變,立請大府貸發三萬,斬五人,事遂定。

同治二年,擢山東按察使。會僧格林沁治兵魯、豫間,令擊河北宋景詩。旋劾其擅議招撫,部議降三級。又明年,遷布政使。僧格林沁戰歿曹州,坐法再幹議,皆得恩旨留任,於是言者復摭他款彈之,事下曾國籓,國籓白其無罪。巡撫閻敬銘夙高其能,至是乞休,舉以自代,遂拜巡撫之命。時捻趨海澨,李鴻章建議築墻膠萊河,寶楨會軍蹙之。六年,東捻走濰河,東軍王心安築壘方成,而堤墻未竣,捻長驅渡河,寶楨以聞。上怒,鴻章交部議,寶楨亦褫職留任。先是東軍守濰河,本皖將潘鼎新汛地。皖軍甫南移,而北路遽失。詔斬心安,寶楨抗辯,乃宥心安而責鴻章;寶楨復屢疏相詆,於是上益責鴻章忌刻縱寇矣。明年,西捻趨定州,近畿震動。寶楨聞警,即馳至東昌,率騎旅千、精卒三千,齎五日糧,倍道北援,捻遂南潰。是役也,朝廷遣宿衛之旅出國門備寇,統兵諸將帥皆獲譴讓,而上獨以寶楨一軍猝出寇前,轉戰雄、任、深、祁、高、肅間,復饒陽,功最盛,數降敕褒嘉,加太子少保。寶楨治軍善乘勢,不主畫疆自守,以故諸軍會集,東西二渠率皆就殲山東。

而其誅安得海事尤著人口。安得海者,以奄人侍慈禧太后,頗用事。八年秋,乘樓船緣運河南下,旗繒殊異,稱有密遣。所過招納權賄,無敢發者。至泰安,寶楨先已入告,使騎捕而守之。安得海猶大言,謂:「汝輩自速辜耳!」傳送濟南,寶楨曰:「宦豎私出,非制。且大臣未聞有命,必詐無疑。」奏上,遂正法。河決鄆城侯家林,運道梗,河臣議次年興工,寶楨謂宜及水涸時,力疾請自任。水齧堤,植立不退,費半功倍。又塞銅瓦廂決口,駐賈莊。聞日本構釁,遂密陳海防計,請築山東煙臺、威海、登州砲臺,設濼口制造機器局,從之。

光緒二年,代吳棠署四川總督。至即嚴劾貪墨吏,澄肅官方,建機器局,修都江堤,裁夫馬以恤民,革陋規以恤吏。又改鹽法,官運商銷,置總局瀘州,其井灶分置廠局,鹽岸分置岸局,歲增帑金百餘萬。而猾商奸吏不便所為,爭中以蜚語,於是臺諫交章糾奏。寶楨已坐堤毀鐫秩矣,而言者復劾停機器局,褫監工成綿龍茂道丁士彬、灌縣令陸葆德職,而尤齗齗爭鹽務。上以川鹽有成效,勿為動。已而成都將軍恆訓覈覆堤工,亦摭及鹽運病商民、流弊大,寶楨抗辯。上慮寶楨惑浮言,敕勿易初念。尋予實授。寶楨彌自警勖,益興積穀,嚴督捕。治蜀凡十年,初蒞事時,郭內月有盜劫,至是誅匪幾盡,聲為道不拾遺。十一年,卒官,贈太子太保,謚文誠,予山東、四川、貴州建祠。

寶楨嚴剛有威。其初至山東也,僧格林沁方蹙捻淄川,頗貴倨,見司道官不設坐。寶楨投謁,告材官啟王,坐則見,否則罷,左右皆大驚。王服其強,為改容加禮。敬銘聞之,大稱異,至之日,親迓於郊。自是事無大小,皆咨寶楨而後行。至今言吏治者,常與沈葆楨並稱,尤勵清操。喪歸,僚屬集賻,始克成行雲。子五人,體常尤著名,官廣東布政使。

李瀚章,字筱泉,安徽合肥人,大學士鴻章兄也。瀚章以拔貢生為知縣,銓湖南,署永定,調益陽,改善化。曾國籓出治軍,檄主餉運,累至江西吉南贛寧道,調廣東督糧道,就遷按察使、布政使。同治四年,擢湖南巡撫。時粵逆李世賢等聚福建,分犯贛南,窺兩楚,貴州苗匪、教匪又闌入楚界,而霆軍潰卒復竄湖、湘,三路告警。瀚章至,則遣前江蘇按察使陳士傑壁郴州防閩賊,前雲南按察使趙煥聯壁岳州防叛卒,閩賊旋引去。叛卒犯江西不得逞,則折入湘,犯攸縣,陷安仁、興寧、副將張義貴擊走之;士傑率軍會剿,遁入粵,卒就殲焉。先是瀚章遣總兵周洪印敗黔匪於邊界,又越境解銅仁圍,因奏言:「懸軍深入,兵家所忌,請敕新任貴州布政使兆琛緩赴任,專治軍事,與楚軍合。」從之。遂遣已革知府李元度進剿思南、石阡教匪,兆琛、洪印進剿清江、臺拱苗匪,所向克捷。苗、教復蟻結,連竄晃州鳳凰,各軍躡擊,皆大破之,黔匪遂不敢窺楚境。自盜起,國籓及胡林翼治師不主畫疆自守。瀚章久習楚軍,既受任,即出境討賊,亦其風類也。

六年,調撫江蘇。未至,署湖廣總督。七年,調浙江,再署湖廣總督,旋實授。光緒元年,調四川。明年,還督湖廣。瀚章性簡靜,更事久,習知民情偽,務與休息。其督湖廣最久,前後四至,皆與弟鴻章更迭受代,其母累年不移武昌官所,人以為榮。尋遭憂去官,家居六年,再起授漕運總督。未幾,移督兩廣。粵俗舊有闈姓捐,四成助餉,巡撫馬丕瑤議革之。會日本構釁,瀚章請循舊收繳備海防,時論大譁,遂以疾歸。又數年,卒,謚勤恪。子十人,經畬,翰林院侍講。

楊昌濬,字石泉,湖南湘鄉人。粵寇亂,以諸生從羅澤南治團練,出援湖北,連復廣濟、黃梅,敘訓導。從征贛、皖,戰楓樹嶺,下德興,戰高沙,下婺源,頻有功,遷知縣。同治元年,從左宗棠入浙,規江山,與劉典、劉璈分三路攻石門,破寇卡數重。進取花園港,縱火燔其棚,會天雨,止。其秋,規龍游,昌濬禦寇蓮塘,破之;又敗之孟塘,寇逸。李世賢聞警,遣悍黨赴救,中路寇方攻劉培元營,昌濬自山下擊,寇大潰,遷知衢州府。明年,師偪龍游城南,築三壘。寇夜奔,昌濬躡之湯溪。城拔,授糧儲道。與蔣益澧合兵萬三千戰餘杭城西北,寇益浚壕樹壘拒師。昌濬攻北門,寇出戰,會諸軍擊之,寇卻,昌濬連夷五卡。次日,攻林清塘,去城北十里,汪海洋老巢也,昌濬覘寇壘阻水,慮日暮為寇乘,乃退師。又明年,規武康,復其城。進略湖州,寇竄泗安、梅溪,昌濬自簰頭進桐嶺扼之,北攻安吉,追寇至孝豐,遇湖州敗寇,復與璈合攻之,降者七千餘人,輒解散。浙西平,遷鹽運使,累擢布政使。

九年,除巡撫。巡視鎮海海口,條具見聞,陳大恉,謂宜師敵伎,練勁旅,修築砲臺,上嘉納之。是時朝廷方銳意求治,詔舉賢才,昌濬以糧道如山四人應,力薦甘肅知縣陶模才器遠大,卒如所言。坐餘杭葛畢氏案褫職。光緒四年,起佐新疆軍事。數遷至漕運總督。十年,法人擾海疆,朝旨以閩事亟,命宗棠為欽差大臣,主軍務,昌濬與穆圖善佐之,張佩綸則會辦也。閩浙總督何璟自以不諳兵事,請解職,遂命昌濬代之。昌濬未至軍,而佩綸已遁,事下宗棠、昌濬。覆奏入,上責其袒護,移督陜甘,加太子太保。

昌濬性和巽,而務為姑息。督甘日,左右通回匪,莫能制,槍械反資寇,遂釀成湟中河、狄亂。昌濬檄各路募土勇助戰守,電令提督雷正綰往河州鎮懾,張永清往西寧策應,蘇員嶼往巴燕戎甘、都堂駐防,並具起事顛末以上。事聞,嚴旨責其庸瞶,乃罷官。二十三年,卒,釋處分。嗣以魏光燾請,予甘肅建祠。

張樹聲,字振軒,安徽合肥人。粵寇擾皖北,以稟生與其弟樹珊、樹屏治團殺賊。復越境出擊,連下含山、六安、英山、霍山、潛山、無為;而太湖一役,以五百人陷陣,擊退陳玉成眾數萬,功尤盛,復力行堅壁清野法。其時劉銘傳、周盛波、潘鼎新輩皆相繼築堡,聯為一氣,皖北破碎,獨合肥西鄉差全。曾國籓檄守蕪湖,調無為,遷知府。同治元年,從李鴻章援上海。鴻章立淮軍,與銘傳等分領其眾,從克江陰,晉道員。鴻章親視婁門程學啟軍,遣樹聲援蕩口,破謝家橋,逐北至齊門,又敗之黃埭,學啟遂偪城而軍,於是婁門寇道始絕。二年,攻無錫、金匱,擊寇芙蓉山,大破之,奪獲戰艦器械不可稱計,賜號卓勇巴圖魯,予三品服。樹聲乘勝趨常州。逾歲,攻河干二十餘營,盡破之。城拔,進復浙江湖州,詔以按察使記名。四年,署江蘇徐海道。尋授直隸按察使,赴大名督防務。

九年,調補山西。越二年,擢漕運總督,署江蘇巡撫,十三年,實授。遭繼母憂,歸。光緒三年,起授貴州巡撫。適廣東總兵李揚才據靈山,構匪擾越南,朝旨調樹聲撫廣西治之。事寧,擢總督,先後剿平西林苗匪、武宣積匪。八年,鴻章喪母歸葬,樹聲攝直督任。值朝鮮亂作,日使花房義質將兵五百入王京,迫朝議約,樹聲飛檄吳長慶等赴之,遂成約,尋盟而還。於是長慶等宵攻亂黨,悉殲其渠,亂乃定,樹聲奏令長慶暫戍朝,上嘉其能,加太子少保。明年,還督兩廣。會法越構兵,即以法人侵逼狀上聞。逮北寧陷,自請解總督職專治軍,報可。復坐按事不實,革職留任。未幾,病卒,謚靖達,予直隸、江蘇及本籍建祠。樹珊自有傳。

樹屏,以收復江蘇各州縣,積勛至副將。從征捻,駐周家口,戰數捷。捻平,擢提督,賜號額騰額巴圖魯。赴晉防河。光緒二年,徙守河曲、保德。會皖軍赴援烏魯木齊,甘肅流賊曹洪照竄後山,樹屏適奉檄詣省,聞警,乘大雪追擊之。事定,賜頭品服,授太原鎮總兵。移防包頭,調大同。十三年,乞休。既歿,鴻章狀其績以上,予優恤,太原建祠。

衛榮光,字靜瀾,河南新鄉人。咸豐二年進士,選庶吉士,授編修。九年,湖北巡撫胡林翼奏調赴軍,隨荊州將軍多隆阿攻剿黃州各郡,轉戰入安徽,平賊壘百餘,克太湖、潛山。捷入,以侍講待簡。林翼督師剿賊,榮光從,常以少擊眾。林翼卒,乃還京供職。道經新鄉,適山東竄匪入境,遂與知縣丁士選集團捍衛。同治元年,入都,補翰林院侍講。明年,擢侍講學士,疏陳剿匪、防河事宜。是年授濟東泰武臨道,署山東鹽運使、按察使。四年,捻首賴文光、張總愚竄山東,巡撫閻敬銘奏委榮光督辦河防。榮光以賊無現糧,利速戰,堅諭各軍嚴守困賊。賊乘夜偷渡,榮光燃砲擊之,諸軍繼進,賊大敗。六年,卸運使任,仍兼署按察使。時賊勢復振,巡撫丁寶楨督師出境,省城兵單餉竭。榮光募民團助守,賊屢逼城下,卒不能犯。旋以父憂歸。

十二年,起江安糧道,署按察使。光緒元年,授安徽按察使,遷浙江布政使,護理巡撫。母憂歸,服闋,授山西巡撫。八年,調江蘇。臺灣道劉璈被重劾,詔刑部尚書錫珍往按,復命榮光赴臺會鞫。榮光言:「璈總營務,開支浮冒,罪當死;然其治事疏節闊目,政頗便民,故臺地番民至今有尸祝者。請從寬典。」其持法嚴而能恕皆此類。十二年,調浙江巡撫,再調山西。以疾乞休。十六年,卒於家。

劉秉璋,字仲良,安徽廬江人。參欽差張芾軍,敘知縣。咸豐十年,成進士,選庶吉士,授編修。同治元年,李鴻章治兵上海,調赴營。洋將戈登所練常勝軍故駐滬,滋驕。淮軍初至,服陋械絀,西弁或侮笑之。秉璋語眾曰:「此不足病也,顧吾曹能戰否耳。」明年,從克常熟、太倉。鴻章使別募一軍圖嘉善分寇勢,遂提兵五千赴難,克楓涇、西塘,遷侍講。進攻張涇匯,約水師夾擊,彈丸貫胯下,不少卻,卒克之。規平湖,其酋陳殿選降,於是乍浦、海鹽、澉浦皆反正。又明年,與程學啟攻嘉興,秉璋入東門燔藥庫,寇駴亂,眾軍乘之,城拔。進取湖州,攻吳漊、南潯,所向摧靡。浙西平,賜號振勇巴圖魯。歷遷侍講學士。

四年,授江蘇按察使,從曾國籓討捻。時捻騎飆疾,國籓與鴻章皆主圈制策,秉璋力贊之,破捻豐、沛、宿遷南,追至倉家集,捻大潰。又敗之淮南,長驅蒙城,捻西走,自此捻分東、西。國籓令秉璋軍豫西,專剿東捻,與提督劉鼎勛俱。其冬,追入鄂。六年,除山西布政使。未上,捻自孝感小河溪竄河口鎮,與鼎勛軍追之,勛軍前鋒遇伏,總兵張遵道戰死,勢益熾,秉璋橫截之,始奔豫。七年,鴻章代國籓督師,議扼運蹙捻海隅。秉璋駐運西,捻撲濰河,將自沂、莒窺江淮。秉璋亟渡河詣桃源,會浙軍扼清江。亡何,賴酋率殘騎數千至,追破之淮城。事寧,被賞賚。父憂歸。服闋,起江西布政使。

光緒元年,擢巡撫。以母老再乞終養。六年,遭喪。至九年,再起撫浙。會法越構釁,緣海戒嚴,秉璋躬履鎮海,令緣岸築長墻,置地雷,悉所有兵輪五艘,輔以紅單師船,據險設防。十一年,法艦入蛟門,令守備吳傑轟拒之,傷其三艘。越數日,復入虎蹲山北,再敗之,法將迷祿中砲死。然猶浮小舟潛窺南岸,復令總兵錢玉興隱卒清泉嶺下突擊之,敵兵多赴水死。

逾歲,擢四川總督。川境窵遠,外接番、夷,內叢奸宄。秉璋曰:「盜賊蠻夷,何代蔑有?以重兵臨之,幸而勝,不為武;不幸而不勝,餉械轉資寇,是真不可為矣。」故督蜀八年,歷平萬縣、茂州、川北、秀山土寇,其大小涼山、拉布浪、瞻對各夷畔服靡恆,則用趙營平屯田法,數月間皆心習伏,加太子少保。御史鍾德祥劾提督錢玉興及道員葉毓榮不職狀,事下湖北巡撫譚繼洵,廉得實,秉璋坐濫舉罪罷。

初,丁寶楨督蜀,稱弊絕風清。秉璋承其後,難為繼,故世多病之。未受代而民教相閧,重慶先有教案,秉璋初至,捕教民羅元義、亂民石匯等寘之法。至是各屬繼起,教堂被毀者數十,教士忿,牒總署,指名奪秉璋職。朝廷不獲已,許之,秉璋遂歸。三十一年,卒。總督周馥及蘇紳惲彥彬等先後上其功,復官,予優恤,建祠。

陳士傑,字俊丞,湖南桂陽州人。以拔貢考取小京官,銓戶部,與閻敬銘同曹司,並以戇樸稱。遭父憂,歸。值粵寇亂,土匪竊發,集團勇得百餘人,平之。俄白水奸民陷永桂,新田告急,眾議拒之。士傑曰:「援新田乃所以自保也!」越境擊卻之。曾國籓治軍衡州,闢參戎幕。鮑超時為小校,坐法當斬,力請釋之。從援湖北,壁岳州城外,王珍軍次蒲圻,違國籓誡,敗退,入空城死守,國籓憤甚,將士莫敢為言,士傑獨請赴救,弗應,固請之,曰:「救之如何?」曰:「寇無戰船,宜遣水師傍岸舉砲為聲援。」珍因獲免於難,厥後鮑、王並為名將。

岳州既敗,寇遂略湘陰,陸走寧鄉,水斷靖港,進陷湘潭據之。國籓水師頓湘川,去寧鄉、靖港皆數十里。或請守省城,或請絕津逕奪寇艎,議未決。士傑謂宜援湘潭,即不利,猶得保衡、永,圖再舉。國籓如其言,果大捷。論功,遷主事。尋歸省,復出佐糧運。咸豐五年,永、桂土匪起,聞亂,單舸溯江歸,專治團練。亡何,連州匪構嶺南北奸民,眾十萬,陷郴州。與珍會師擊之,復其城,遂以南防屬之。留州賦充餉,改團為營,號廣武軍。

永、郴、桂陽邊地千里,廣武當其沖,數挫寇鋒,而以捍石達開功為盛。達開故黠猾,麾下號百萬,分七部,能檢勒之使毋擾。九年春,自贛而西,至桂陽,穿城北走。時廣武軍軍花園砦,有橋跨鍾水,曰斗下渡,其南兩山相崟,一逕中達,東西北皆環水。士傑遣一裨將領百人扼橋,寇夜至,大驚,不敢前。後來者欲退則隘塞,欲旁出則無路。平明,士傑率師轟擊之,自相蹈藉,墜死無算。是役也,士傑以數百人敗寇數十萬眾,達開襲省之計卒無所施,上嘉之,擢知府。嗣錄援藍山、嘉禾、寧遠功,晉道員。

同治元年,三吳軍事棘,以國籓薦,超授江蘇按察使。士傑慮石黨往來郴、永貽母憂,乞終養,以防遏上游為己任,數卻寇。四年,論功,加布政使銜。時江南既定,而霆軍所降寇復叛,自湖北金田入郴,數千里無與逆戰者。士傑要擊之,寇大潰,賜號剛勇巴圖魯。十年,母喪,服闋,除山東按察使。光緒元年,到官,多所平反。晉福建布政使。未上,會巡撫文格被劾,詞連士傑,罷免。尋以臺灣軍務,命署福建按察使。六年,遷布政使。明年,擢撫浙江。巡海口,增築鎮海笠山港及定海乍浦砲臺。八年,移山東,緣海設防。吳大澂會辦北洋防務,至登州、煙臺,見廣武軍壁壘,頗採其法而增損之,奏請頒行各海口。而忌者中以蜚語,至劾其海防草率,事下尚書延煦、左都御史祁世長,得白。海防軍罷,而士傑亦病矣,數請乞休,始允。十八年,卒於家,予省城及本籍建祠。

陶模,字方之,浙江秀水人。同治七年進士,改庶吉士。散館,授甘肅文縣知縣,調皋蘭。左宗棠為總督,方徵回,又創建貢院,兵工諸役並作,模躬自料量,民不知擾。遷秦州直隸州。歲旱,流徙饑民數十萬麕集,出積俸,並割公使銀四萬餘金設粥廠,不足,貸金益之。修養濟院,增義田,恤嫠婦。州南藉水齧城堙,模為築堤沼三百五十丈,植芙蕖楊柳,蓄鱗介,取其利,以時繕完。署甘州府知府,罷屬縣供億。宗棠奏模治行第一,調補迪化州。編修廖壽豐薦模器識宏遠,堪備閫寄。時回久亂,民戶寥落,模和輯漢、回,耕者復聚。時議定賦則,模謂經畫窮塞,當通周官一易再易之義,令民以二畝當一畝,徵其六緩其四。宗棠採其議,邊民始有久居志。歷署蘭州府、蘭州道、按察使,調直隸按察使、陜西布政使,護巡撫。

光緒十七年,授甘肅新疆巡撫。當蔥嶺西,有地曰帕米爾,乾隆間為我軍鋒所及,高宗嘗勒銘焉。蔥嶺東南有小部落曰坎巨提,歲納貢於我。模未至新疆,俄侵帕米爾,謀通印度,英攻破坎巨提。中外方議戰,模謂:「將士能戡土匪,未能禦強敵。軍資百物,運自內地,數月乃達。俄、英鐵軌,瞬息可至。新疆與俄相接幾五千里,增兵十倍未足固。當民窮財匱之時,不可輕言戰。惟當購機砲,擴電線,飭邊將嚴為備。羈坎巨提故酋無令北走,而撫其流民,與駐俄、英使臣合爭。」議未定,俄曰防英,英曰防俄,莫可究詰。明年,二國兵益進,將吏咸憤激請戰,終不許。於是奏請廢黜坎巨提故酋。會英人亦立其弟買賣提艾孜木,令鎮撫部民,歲納貢如故事,坎巨提事乃定。

而俄兵在帕米爾,意叵測。模以邊防無效,自請罷斥,不允。廷議將以帕米爾為三國甌脫,英垂諾,俄猶不可,陳兵相持。模取德意志兵法練邊軍,選幼童百餘,課以測算諸法,將徐推之各軍。見將佐必以惜勞苦、寶槍彈為戒。初,俄人借巴爾魯克山以處所屬哈薩克,期十年。山饒水泉林木,當塔城西北,廣袤數百里。至是期滿,無還意。模爭之,逾年乃如約。俄商及附英諸部至新疆皆不稅。模曰:「是獨苦吾民!」為奏請普免焉。

纏回文字語言不相通,漢民愚之,貸金輒取重息,至賣鬻妻子以償。模為之規定章條,令讀書習漢語,於是回族欣欣向化矣。羅布淖爾,古蒲昌海也,荒沙無垠,亙新疆中部。模議闢徑路,自新疆之南,青海、西藏之北,噶斯、烏蘭達布遜、阿耨達、托古茲尼蟒依諸大雪山之陰,迂回出入,分道測繪,得金鐵煤諸礦數十百計,欲開採利民,以絀於貲,工不克舉。乃於羅布淖爾北四百餘里築蒲昌城,南百四十里設屯防局,回民徙居成村落。其後設置營縣,實自模開之。

二十年,日本略朝鮮,朝議決戰,師屢敗。甘肅提督董福祥先以祝嘏在京,募兵備戰,河湟回族聞亂思蠢動。二十一年春,撒拉河州、西寧、大通諸回先後反。西寧回酋劉四伏尤悍,模遣將援巴燕戎格,與總督楊昌濬合疏請命福祥帥師西援。夏,平番回亦變,河西諸府東不能通省會,則西乞援新疆。模奏陳回亂日亟,部遣諸將羅平安戍哈密,牛允誠守安西、玉門,趙有正屯肅州,而於哈密置東防營務處,以道員潘效蘇護諸將。諸亂回遣其徒出關煽新疆回部。九月,綏來回發難,以有備,旋定。迪化回應之,模詗知莠民與牙役密相結,捕斬六人而亂弭。十月,回逼甘州,上罷昌濬,以模署陜甘總督,命入關剿撫。時福祥將甘軍渡洮,魏光燾將湘軍臨湟水。模策東路兵大集,回且西竄,乃遣兵分駐天山北迪化、鎮西為中權,而繕完防御天山以南諸要隘。後路既設備,乃將馬步八營馳入關,道經沙漠至吐魯番城,回王瑪木特來會,勖以大義。至哈密,校閱各軍,令纏回與焉。模以有正兵寡,戒毋輕出。有正喜功,出攻察漢俄博、永安二城皆下。二十二年元夕,薄北大通營,敗歸。模遣涼州戍軍赴援。二月,入關,群回斂聚山南,模至蘭州視事,令效蘇督諸將略北大通營,破所領十大莊堡,戮其酋,殲數千人,諸回氣奪。會光燾亦定西寧,諸回自水峽口西竄青海。模令效蘇等出塞,陳兵玉門諸山徑,毋縱賊出平地。青海蒙古積弱,久怵回悍,告急。朝議令光燾、福祥二軍追逐。模以師行絕域,糧芻車馱,重為民累,內地空虛,為禍滋大,奏寢其議。新疆將吏慮回更西竄,亦告急。朝議令提督鄧增出青海,張俊防北路。模策賊非至玉門、敦煌掠食,不能遽犯新疆,復請罷移軍議,而令增屯肅州為聲援。光燾將湘軍還陜西,以與福祥不相能也。賊自青海犯玉門,允誠等擊卻之。模令玉門軍赴安西。五月,賊大至,劉四伏奪路求食,諸將力戰,金蘭益匹馬陷陣,大敗賊於牛橋,降斬各數千人,饑凍死磧中者過半。四伏以千數騎遁,中道伏發,就擒。於是徙降回塔裏木河濱,計口授田。關內外悉平,論功,實授總督。

方日事之初起也,和戰議不決。模言:「國強弱視人才,人才不足,和戰皆不足恃,即戰勝亦無益。」因言:「天下事當變通者非一,如減中額,停捐例,汰冗員,令京官升遷不出本部,司員分類治事,刪棄舊案,破除旗兵積習,禁士大夫食鴉片,分設算學、藝學科目,廢武科,變操法,擇勛舊子弟游學各國,培植工藝。尤原皇上鑒天災之屢警,念民困之莫蘇,懍內政之宜修,知外患之難弭,毋始勤終怠,毋狃目前而忘遠慮。」時中外諸臣條奏,多言變法袪積習。模言:「推行宜漸,根本宜急。聚闒茸嗜利之輩以期富強,止於舊法外增一法,不得謂之變法;於積習外增一習,不得謂之袪積習。欲求富強,當先崇節儉,廣教化,恤農商。」其恉意大率類此。模督陜甘數年,銳欲開礦制械,興學廣教,皆以用不足,不能盡舉,累疏乞罷。

二十六年,述職入覲,道疾,留陜西。俄調補兩廣總督。兩宮西幸,迎謁蒲州,再乞休,不允,乃力疾上官。二十七年,疏請裁減宦官,略言:「宦官干政,史不絕書,我朝家法嚴明,從未有內監預聞政事。然除弊如除莠,留其芽蘗,終恐發生,宜大加裁汰。內廷差使悉可改用士人,定宮府一體之制,永不再選內監,非唯一時盛事,實亦千古美談。」別疏言:「變通政治,宜務本原。本原在朝廷,必朝廷實能愛國愛民,乃能以愛國愛民責百官;必朝廷先無自私自利,乃能以不自私不自利望天下。轉移之道,一曰除壅蔽,一曰去畛域,一曰務遠大。朝廷當以身作則,克己勝私,否則雖日言變通,無由獲變通之效。」

粵故多盜,模定清鄉章程,信賞必罰。凡練軍分屯,許所在州縣節制。一歲中捕斬名盜千餘人,欽、廉、肇、羅諸屬盜藪,皆次第削平。模謂民貧思亂,非殺可止,令府縣設勸工廠,囚不至死者令入廠教養。廣東名饒富,然取諸民者已重於他行省,歲不足五百餘萬,則取之賭規,仍不足,則貸之外人。模睹民力已屈,追呼不得寬,欲有所興革,皆坐中沮。迭疏請疾,甫受代,九月,卒於廣州,贈太子少保,謚勤肅。

模自為諸生,食貧力學,與平湖優貢生顧廣譽、震澤諸生陳壽熊、吳江舉人沈曰富以道義相勖。既通籍,大學士閻敬銘、總督楊昌濬皆嘗論薦,不以告模,模亦不謝也。儉約自將,不立崖岸,恂恂卑下,將吏爭為用,而無敢以私干者。卒後,蘭州、迪化皆允建專祠。

李興銳,字勉林,湖南瀏陽人。粵寇亂,以諸生治鄉團。曾國籓治軍東下,檄主軍糈,駐祁門。江南饑民就食者萬計,興銳慮為寇乘,先期結筏以濟,獲安全,敘知縣。數薦知府。同治四年,唐義訓、金國琛兩軍頓徽州,索餉譁變。興銳聞之,單騎叩其壁,諭之曰:「若輩不遠千里,從軍討賊,為富貴計耳,奈何自戕為?使寇知之而躡吾後,吾無焦類矣!餉不給,咎在臺。期以三日,逾期請殺我!」眾曰:「唯命!」乃潛訪主謀者三人,白國籓僇之,事定。金陵既克,儲平餘銀四十餘萬。目擊戎燼後殭尸蔽野,因出所餘購義塚一區,聚暴骨瘞之。

八年,調直隸,補大名府,洊升道員,乞終養。國籓再督兩江,檄綜營務,與彭玉麟規訂水師營制。國籓卒,李宗羲代督,亦頗信仗之。時日本窺臺灣,江海戒嚴。興銳言於宗羲,躬履江陰、狼山、吳淞、崇明,擇險設守,始倡緣海築砲臺議。光緒改元,綜辦上海機器制造局,博採西國新器,增建鐵船砲廠,鳩工庀材,閱十稔,規模略備。遭母喪去官,服竟,命偕鴻臚寺卿鄧承修往勘中越邊界。

十二年,充出使日本大臣。會遘疾,未上。居三年,補天津道,旋調山東東海關道。威海為日人所據,居民惶恐,興銳建議勘地分界,主客互守,閭市獲安堵。其辦交涉,獨條理精整,事可許者,一諾輒立辦;遇所不可,則抗辯廣坐,常服遠人。遷長蘆鹽運使,歷福建按察使、布政使。二十六年,擢撫江西。拳匪釁作,頑民相率不靖,旬日間毀教堂數十,掠教民財產,積案二千餘。興銳劾罷疏防官十餘人,限三月定讞,議償恤費八十餘萬,唯節餉以彌罅漏。和議成,償款累百萬,仍以節餉資挹注;猶不足,則取之土藥釐榷,絕不累民間毫末。署南贛鎮申道發統軍驕蹇不奉法,首劾罷之,軍紀始肅。興銳事國籓久,論治壹循軌跡,重實行。是時上方鄉新政,乃以十事上,曰:開特科,整學校,課官吏,設銀行,鑄銀幣,維圜法,立保險,修農政,講武備,而歸本於用人,為安內攘外之策,言至深切。旋移撫廣東。

二十九年,署閩浙總督。閩自軍興,局所林立,有善後、濟用、勸捐、稽覈、稅釐諸目,叢弊益甚。興銳受事,裁諸局所,並為財政局,事權始一。於是釐定常備軍制,汰虛冗,節浮費,而閩事稍稍振矣。逾歲,調署兩江。旋病卒,謚勤恪。

史念祖,字繩之,江蘇江都人,刑部尚書致儼孫。念祖幼穎異,好讀兵家言。逾冠,入貲為通判。從喬松年軍解蒙城圍,有功。僧格林沁戰歿曹州,捻益熾,皖北麋沸。念祖率師復英山,克高圩。雉河集者,張洛行老巢也,英翰守之,陷重圍,誓必死,念祖計出之,而自駐其地,期以二十日相見城下。乃為均糧法,數卻寇。嘗坐堞上彈琵琶,教士卒歌,寇出視,皆驚嘆。一日,聞槍砲聲,知援至,與寇戰,乃令居民登陴守,別選銳卒四千分道夾擊,縱橫掃蕩,寇大潰,謁英翰止逾二日雲。數保道員。

同治六年,移師鳳陽。時捻酋李允謀窺廬、鳳,詣五河就李世忠。念祖詗知之,計說世忠縛以獻,金巢送壽州寘之法,晉按察使。援滕縣,既捷,師還,寇逾萬躡其後,乃掘深溝,布機械,陰徙去,追騎多墜死,人服其智略。直東平,賜號捷勇巴圖魯。八年,除山西按察使,年未及三十也。上慮其資名輕,與直隸按察使張樹聲易官,令曾國籓察覆,稱念祖明爽,磨厲當成大器,宜稍緩任事,遂解職,留直差序。十年,左遷甘肅安肅道,主關內外糧運,給食不乏,征西軍倚以集事,頗見賞於左宗棠。

光緒四年,晉按察使。多所平反,理俞應鈞等殺降回讞忤宗棠意,再被劾去。十年,起雲南按察使。歷貴州,調補雲南布政使。時總督岑毓英督師出關,需餉亟,而巡撫張凱嵩與有卻。念祖為陳公私利害,請以地丁錢漕受巡撫指麾,釐金雜稅供總督兵餉,復為貸商款備糧械,毓英德之,密薦其賢。二十一年,授廣西巡撫。桂故多匪,至則選卒逐捕,痛繩以法,匪皆斂跡。坐失察贓罪,罷免。三十一年,賞加副都統銜,命赴奉天隨將軍趙爾巽治賑。尋督三省鹽務及財政局。奉省吏治不飭,冒憲黷貨,弊風相踵,念祖佐爾巽力抉其弊,蠲苛息煩,歲入倍蓰。期年奏績,上嘉之,晉記名副都統。爾巽移蜀,徐世昌代之,又劾罷。宣統二年,卒。爾巽先後上其功,復巡撫原官,恤如制。

論曰:寇亂初平,安民保土,自以吏治為先,然非負文武幹用如寶楨諸人,亦不易言效也。寶楨政尚威猛,瀚章治參清靜,而昌濬則不免於姑息。樹聲有智略,秉璋稱綜覈。榮光、士傑皆善於用兵,而疏於行政。興銳重實效,念祖好行權。模獨識議宏遠,能見本原。此十人中雖治績不必盡同,其賢者至今猶絓人口,庶幾不失曾、左之遺風歟。


\end{pinyinscope}