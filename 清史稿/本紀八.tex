\article{本紀八}

\begin{pinyinscope}
聖祖本紀三

四十一年壬午春正月壬寅,詔修國子監。丙午,詔系囚經緩決者減一等。以雅爾江阿襲封簡親王。庚戌,上巡幸五臺山。

二月庚申,次射虎川。士民請於菩薩頂建萬壽亭祝釐。不許。丁卯,上巡視子牙河。

三月壬午,上還京。以瓦爾岱為滿洲都統,吳達禪、馬思哈、滿丕為蒙古都統。丁亥,上御經筵。

夏四月甲戌,賜致仕大學士王熙御書匾對,傳旨曰:「卿先朝舊臣,其強餐食,慎醫藥,以慰朕念。」

五月癸巳,定發配人犯歸籍僉遣,流犯死配所,妻子許還鄉里。辛丑,顯親王丹臻薨,遣皇子及大臣治喪,賜銀萬兩,謚曰密,子衍璜襲。壬寅,先是,廉州府連山瑤人作亂,御史參奏,命都統嵩祝率禁旅會討,並命尚書範承勛勘狀。至是,嵩祝奏官兵一到,瑤人乞降,先後投出瑤人一萬九千餘名。獻出戕官黎貴等九人,即於軍前正法。降瑤安插,交總督料理。範承勛奏瑤人滋事,副將杜芳傷死,總兵劉虎先行退回,應擬斬,提督殷化行應革職。得旨:「殷化行有戰功,改原品致仕。劉虎免死。」丙午,召廷臣至保和殿,頒賜御書。

六月壬子,貴州葛彞寨苗人為亂,官軍討平之。戊午,上制訓飭士子文,頒發直省,勒石學宮。乙未,上奉皇太后幸熱河。乙丑,四川提督岳升龍疏報大涼山惈目馬比必率眾內附,請授土千戶,給印信。

閏六月辛丑,木鴉番民萬九千餘戶內附,請置安撫使、副使、土百戶等職,均從之。

八月庚辰朔,增順天、浙江、湖廣鄉試中額。戊申,上奉皇太后還宮。

九月辛亥,以李正宗、盧崇耀、馮國相為漢軍都統。壬子,定五經中式例。癸丑,停本年秋決。辛酉,以齊世、嵩祝為滿洲都統,莽喀為漢軍都統,車納福為蒙古都統。甲子,詔:「南巡閱河,所過停供張,禁科斂。官吏無相餽遺,百姓各守本業。督撫布告,使明知朕意。」己巳,以席哈納為大學士,敦拜為吏部尚書,席爾達為禮部尚書,溫達為左都御史,管源忠為廣州將軍。鎮筸諸生李豐等叩閽言紅苗殺人,有司不問。詔侍郎傅繼祖、甘國樞,巡撫趙申喬馳驛按問。癸酉,上南巡啟鑾。

冬十月壬午,次德州。皇太子胤礽有疾,上回鑾。癸卯,上還宮。丙午,以郭世隆為廣東廣西總督,金世榮為浙江福建總督。

十一月丙辰,詔免陜西、安徽明年額賦。甲子,大學士伊桑阿乞休,命致仕。壬申,廣西巡撫蕭永藻疏劾布政使教化新虧空倉穀,應令賠補。上曰:「米穀必有收貯之地,乃可經久。若無倉廒,積於空野,難免朽爛,況南方卑濕之地乎?其別定例以聞。」命修禹陵。

十二月壬辰,廷臣以明年五旬萬壽,請上尊號。上不許。戶部議駁奉天報災。上曰:「晴雨原無一定,始者雨水調和,其後被災,亦常事耳。可準其奏。」乙未,改趙申喬為偏沅巡撫,以趙弘燦為廣東提督,王世臣為浙江提督,孫徵灝為漢軍都統。壬寅,厄魯特丹津阿拉布坦來朝,厚賚之,封為郡王,賜地游牧。

是歲,免江南、河南、浙江、湖廣、甘肅等省十州縣災賦有差。朝鮮、琉球入貢。

四十二年癸未春正月壬子,大學士諸臣賀祝五旬萬壽,恭進「萬壽無疆」屏。卻之,收其寫冊。壬戌,上南巡閱河。丁卯,以俞益謨為湖廣提督。庚午,次濟南,觀珍珠泉,賦三渡齊河詩。壬申,次泰安,登泰山。詔免蹕路所經及歉收各屬去年逋賦。

二月丁丑,運漕米四萬石賑濟寧、泰安。閱宿遷堤工。己卯,自桃源登舟,遍閱河堤。甲申,渡江登金山。丙戌,次蘇州。遣官奠大學士宋德宜墓。庚寅,上駐杭州閱射。辛丑,次江寧。

三月戊申,上閱高家堰、翟家壩堤工。己酉,上閱黃河南龍窩、煙墩等堤。庚申,上還京。癸亥,萬壽節,上朝皇太后宮,免廷臣朝賀。頒恩詔,錫高年,蠲額賦,察孝義,恤困窮,舉遺逸,罪非常赦所不原者,咸赦除之。頒賜親王、郡王以下文武百官有差。庚午,以洞鄂襲封信郡王。辛未,上御經筵。賜內廷修書舉人汪灝、何焯、蔣廷錫進士,一體殿試。

夏四月辛巳,賜王式丹等一百六十三人進士及第出身有差。四川威州龍溪十八寨生番歸化納糧。丁亥,大學士熊賜履乞休,命解官食俸,留備顧問。傅繼祖等察審湖廣紅苗搶掠一案。得旨:「總督郭琇、提督杜本植隱匿不報,均革職。巡撫金璽降官。」以喻成龍為湖廣總督。癸巳,致仕大學士王熙卒,予祭葬,謚文靖。丙申,以陳廷敬為大學士兼吏部尚書。戊戌,詔原任侍郎任克溥年逾九十,洵為耆碩,加尚書銜。以李光地為吏部尚書,仍巡撫直隸。以莽喀為荊州將軍,諾羅布為杭州將軍,宗室愛音圖為漢軍都統,孫渣齊、翁俄里為蒙古都統。己亥,諭八旗人等:「朕不惜數百萬帑金為旗丁償逋贖地,籌畫生計。爾等能人人以孝弟為心,勤儉為事,則足仰慰朕心矣。倘不知愛惜,仍前游蕩飲博,必以嚴法處之。親書宣諭,其尚欽遵!」

五月壬子,裕親王福全有疾,上連日視之。癸亥,內大臣索額圖有罪,拘禁於宗人府。己巳,上巡幸塞外。

六月辛巳,恭親王常寧薨,命皇子每日齊集,賜銀一萬兩,遣官造墳立碑。壬寅,裕親王福全薨,上聞之,兼程回京。

秋七月乙巳朔,上臨裕親王喪,哭之慟,自蒼震門入居景仁宮。王大臣請還乾清宮,上曰:「居便殿乃祗遵成憲也。」居五日,命皇長子等持服,命御史羅占造墳建碑,謚曰憲。子保泰嗣爵。戊申,以山東大雨,遣官分賑。庚戌,上巡幸塞外。己巳,發帑金三十萬兩,截漕五十萬石賑山東。山東有司不理荒政,停其升轉。

八月癸巳,停本年秋審。

九月壬子,予故侍郎高士奇、勵杜訥祭葬。己巳,命尚書席爾達督辦紅苗。

冬十月癸未,上西巡啟鑾。命給事中滿普、御史顧素在後行,查僕從生事,即時鎖拿。庚寅,喇嘛請廣洮州衛廟,上曰:「取民地以廣廟宇,有△民生。其永行禁止。」癸巳,過井陘,次柏井驛。驛向乏泉,至是井泉湧溢。丁酉,駐太原。戊戌,詔免山西逋賦。百姓集行宮前■B6留車駕,上為再停一日。

十一月乙巳,上次洪洞。遣官祭女媧陵。壬子,渡黃河,次潼關。遣官祭西嶽。賜迎駕百歲老人白金。甲寅,次渭南。閱固原標兵射,賜提督潘育龍以下加一級。丙辰,上駐西安。丁巳,閱駐防官兵射。遣官祭周文王、武王,祭文書御名。遣官奠提督張勇、梁化鳳墓。己未,上大閱於西安,賜將軍博濟御用弓矢。賜官兵宴。軍民集行宮前■B6留,上為留一日。賜盩厔徵士李顒御書「操志清潔」匾額。免陜、甘逋賦。癸亥,上回鑾。己巳,次陜州。命皇三子胤祉往閱三門底柱。

十二月乙亥,上次修武。閱懷慶營伍不整,逮總兵官王應統入京論死。庚辰,次磁州。御書「賢哲遺休」額懸先賢子貢墓。庚寅,上還京。辛卯,定外任官在本籍五百里內者回避。封常寧子海善為貝勒。

是歲,免直隸、江南、山東、河南、陜西、浙江、湖廣等省九十一州縣災賦有差。朝鮮、琉球、安南入貢。

四十三年甲申春正月辛酉,詔曰:「朕諮訪民瘼,深悉力作艱難。耕三十畝者,輸租賦外,約餘二十石。衣食丁徭,取給於此。幸逢廉吏,猶可有餘。若誅求無藝,則民無以為生。是故察吏所以安民,要在大吏實心體恤也。」戊辰,詔漢軍一家俱外任者,酌改京員。己巳,上謁陵。

二月甲戌,封淮神為長源佑順大淮之神,御書「靈瀆安瀾」額懸之。癸巳,上還宮。以李基和為江西巡撫,能泰為四川巡撫。

三月辛丑,上御經筵。已酉,詔停熱審。辛酉,以吳洪為甘肅提督。資送山東饑民回籍。丙寅,以溫達為工部尚書。

夏四月癸酉,命侍衛拉錫察視河源。己卯,上幸棽髻山,遂閱永定河、子牙河。丙申,上還京。

五月辛酉,以於準為貴州巡撫。

六月乙亥,上巡幸塞外。

秋九月癸卯,詔督撫調員違例者罪之。侍郎常授招撫廣東海盜阿保位等二百三十七名,就撫為兵。戊午,刑部尚書王士禛以失出降官。癸亥,上還宮。丁卯,侍衛拉錫察視河源,還自星宿海,繪圖以進。

冬十月戊辰朔,濬楊村舊河。甲戌,詔免順天、河間二府及山東、浙江二省明年稅糧。庚辰,以李振裕為禮部尚書,徐潮為戶部尚書,屠粹忠為兵部尚書,王掞為刑部尚書,吳涵為左都御史。癸未,頒內制銅斗銅升於戶部,命以鐵制頒行。戊子,以趙弘燮為河南巡撫。己丑,命濬汾、渭、賈魯諸河。辛卯,上閱永定河。

十一月丁酉朔,日有食之。上還宮。上以儀器測驗與七政歷不符,欽天監官請罪,免之。郎中費仰嘏以貪婪棄市。辛亥,定吏部行取知縣例,停督撫保薦。戊午,湖廣巡撫劉殿衡建御書樓,上斥其糜費,並嚴禁藉修建侵帑累民者。四川陜西總督博霽疏參涼州總兵官魏勛年老,上曰:「魏勛前有軍功,兵民愛戴,與師帝賓、麥良璽、潘育龍俱系舊臣,難得,何可參耶?」壬戌,誡修明史史臣覈公論,明是非,以成信史。

十二月乙酉,天津總兵官藍理請沿海屯田,從之。甲午,以禦制詩集賜廷臣。

是歲,免直隸、江南、山東、湖廣、廣東等省一百九州縣災賦有差。朝鮮入貢。

四十四年乙酉春正月戊午,古文淵鑒成,頒賜廷臣,及於官學。癸亥,上幸湯泉。

二月乙丑朔,上還宮。癸酉,上南巡閱河。詔曰:「朕留意河防,屢行閱視,獲告成功。茲黃水申昜流,尚須察驗形勢,即循河南下。所至勿繕行宮,其有科斂累民者,以軍法治罪。」壬午,次靜海。遣官奠故侍郎勵杜訥墓,予謚文恪。

三月己亥,諭山東撫臣曰:「百姓歡迎道左者日數十萬人,計日回鑾,正當麥秀,其各務稼穡,毋致妨農。」乙巳,上駐揚州。授河臣張鵬翮方略。辛亥,上駐蘇州。命選江南、浙江舉、貢、生、監善書者入京修書。賜公福善,大學士張玉書、陳廷敬,在籍大學士張英,都統愛音圖白金。賜大學士馬齊等皇輿表。己未,次松江閱射。上書「聖跡遺徽」額賜青浦孔氏。賜故侍郎高士奇謚文恪。

夏四月丙寅,上駐杭州閱射。庚午,詔赦山東、江蘇、浙江、福建死罪減一等。戊寅,御書「至德無名」額懸吳太伯祠,並書季札、董仲舒、焦先、周敦頤、範仲淹、蘇軾、歐陽修、胡安國、米芾、宗澤、陸秀夫各匾額懸其祠。乙酉,上駐江寧。

閏四月癸卯,上閱高家堰堤工。辛酉,上還京。

五月戊寅,上親鞫郎中陳汝弼一案,原汝弼罪。刑部尚書安布祿、左都御史舒輅以失獄免職。庚辰,以貝和諾為雲南貴州總督。丙戌,上巡幸塞外。

六月甲午,命行取知縣非再任者不得考選科道。庚戌,停廣東開礦。丙辰,上駐蹕熱河。

秋七月壬申,河決清水溝、韓莊,命河臣察居民田舍以聞。

八月甲午,免八旗借支兵饟銀七十萬兩。戊午,喻成龍免,以石文晟為湖廣總督。庚申,上發博洛河屯,閱牧群。

九月己巳,進張家口。丙子還京。甲申,以希福納為左都御史,達佳為江寧將軍。

冬十月辛卯朔,重修華陰西嶽廟成,上制碑文。丙午,以富寧安為漢軍都統。

十一月辛酉,命蒙古公丹濟拉備兵推河,察視策旺阿拉布坦。己巳,以李光地為大學士,宋犖為吏部尚書,調趙弘燮為直隸巡撫。癸酉,詔免湖廣明年額賦及以前逋賦。甲戌,國子監落成,御書「彞倫堂」額。庚辰,以汪灝為河南巡撫。乙酉,上謁陵。巡幸近塞。戊子,設雲南廣南、麗江二府學官,許土人應試。

十二月壬寅,上臨裕親王福全葬。以阿靈阿兼理籓院尚書。己酉,上還宮。丙辰,以祖良璧為福州將軍。

是歲,免直隸、江南、湖廣、廣東等省四十六州縣災賦有差。朝鮮、琉球入貢。

四十五年丙戌春正月乙酉,命孫渣齊、徐潮督濬淮揚引河。順天考官戶部侍郎汪霦、贊善姚士藟以取士不公褫職。

二月癸巳,上巡幸畿甸。丁未,次靜海,閱子牙河。壬子,還駐南苑。以諸滿為江寧將軍。以王然為浙江巡撫。江南江西總督阿山劾江寧知府陳鵬年安奉聖訓不敬,部議應斬。先是,乙酉年南巡,陳鵬年遵旨不治行宮,阿山故假他事劾之。上命入京修書。戊午,上還宮。

三月庚申,上御經筵。辛巳,賜施雲錦等二百八十九人進士及第出身有差。詔直省建育嬰堂。

夏四月戊子朔,日有食之。加貴州提督李芳述鎮遠將軍。乙未,吳涵罷,以梅鋗為左都御史。

五月己未,以金世榮為兵部尚書。甲戌,詔免直隸、山東逋賦。丁丑,以梁鼐為福建浙江總督。戊寅,上巡幸塞外。

六月丁亥朔,詔修功臣傳。癸巳,命梅鋗、二鬲按容美土司田舜年獄。壬寅,命凡部寺咨取錢糧非由奏請者,戶部月會其數以聞。以藍理為福建陸路提督。辛亥,四川巡撫能泰疏報安樂鐵索橋告成,移化林營千總駐守。

秋七月庚申,上駐蹕熱河。甲子,以德昭嗣封信郡王。

八月壬辰,高家堰車邏壩澗河河堤告成。

九月己亥,上還京。

冬十月乙酉朔,敦拜罷,以溫達為吏部尚書,希福納為工部尚書。庚寅,武殿試。諭曰:「今天下承平日久,曾經戰陣大臣已少,知海上用兵者益少。他日臺灣不無可慮。朕甲子南巡,由江寧登舟,至黃天蕩,江風大作,朕獨立船頭射江豚,了不為意。迨後渡江,漸覺心動。去歲渡江,則心悸矣。皆年為之也。問之宿將亦然。今使高年奮勇效命,何可得耶?」壬寅,命大學士席哈納、侍郎張廷樞、蕭永藻覆按土司田舜年獄。丁未,以迓圖為滿洲都統。己酉,詔免山西、陜西、江蘇、安徽、江西、浙江、福建、湖北、湖南、廣東十省逋賦。

十一月癸酉,命尚書金世榮、侍郎巴錫、範承烈督濬清河。免八旗官兵貸官未歸銀三百九十五萬六千六百兩有奇。甲戌,以阿山為刑部尚書。庚辰,上謁陵。辛巳,以邵穆布為江南江西總督。癸未,以山東私鑄多,聽以小錢完正賦,責有司運京鼓鑄。甲申,上巡幸塞外。西藏達賴喇嘛卒,其下第巴匿之,又立偽達賴喇嘛。拉藏汗殺第巴而獻其偽喇嘛。西寧喇嘛商南多爾濟以聞。

十二月壬寅,上還宮。詔罪囚緩決至三四年者減一等。辛亥,郭世隆罷,以趙弘燦為廣東廣西總督。

是歲,免直隸、江南、福建、江西、湖廣等省三十二州縣災賦有差。朝鮮入貢。

四十六年丁亥春正月丁卯,詔:「南巡閱河,往返舟楫,不御室廬。所過勿得供億。」丁巳,梅鋗免,以蕭永藻為左都御史。

二月戊戌,次臺莊,百姓來獻食物。召耆老前,詳詢農事生計,良久乃發。癸卯,上閱溜淮套,由清口登陸,如曹家廟,見地勢毗連山嶺,不可疏鑿,而河道所經,直民廬舍墳墓,悉當毀壞。詰責張鵬翮等,遂罷其役,道旁居民驩呼萬歲。命別勘視天然壩以下河道。

三月己未,上駐江寧。乙巳,上駐蘇州。

夏四月甲申,上駐杭州。詔曰:「朕頃因視河,駐蹕淮上。江、浙二省官民■B6請臨幸,朕勉徇群情,涉江而南。方今二麥垂熟,百姓沿河擁觀,不無踐踏。其令停迎送,示朕重農愛民至意。」戊申,以鄂克遜為江寧將軍,殷泰為甘肅提督。

五月壬子朔,上次山陽,示河臣方略。癸酉,上還京。丙子,解阿山尚書,削張鵬翮宮保。戊寅,贈故河道總督靳輔太子太保,予世職。加福建提督吳英威略將軍。贈死難運司高天爵官,予謚忠烈。以達爾占為荊州將軍。

六月丁亥,上巡幸塞外。以巢可託為左都御史,起郭世隆為湖廣總督。

七月壬子,上駐蹕熱河。丁卯,車駕發喀拉河屯,巡幸諸蒙古部落。外籓來朝,各賜衣幣。

八月甲辰,次洮爾畢拉,賜迎駕索倫總管塞音察克、杜拉圖及打牲人銀幣。貴州三江苗人作亂,討平之。

九月癸亥,上駐和爾博圖噶岔。甲子,閱察哈爾、巴爾虎兵丁射。

冬十月辛巳,以江蘇、浙江旱,發帑市米平糶,截漕放賑,免逋賦。外籓獻駝馬,卻之。戊戌,上還京。己亥,戶部議增云南礦稅,命如舊額。庚子,金世榮免,以蕭永藻為兵部尚書。

十一月己酉朔,詔曰:「頃以江、浙旱災,隨命減稅、蠲逋、截漕。其江、浙兩省明年應出丁錢,悉予蠲免。被災之處,並免正賦。使一年之內,小民絕跡公庭,優游作息,副朕惠愛黎元至意。」己未,詔臺灣客民乏食,原歸者聽附公務船內渡。以汪悟禮為漢軍都統。己亥,詔江、浙諸郡縣興修水利備旱澇。

十二月丙戌,以溫達為大學士,馬爾漢為吏部尚書,耿額為兵部尚書,巢可託為刑部尚書,富寧安、王九齡為左都御史。丙午,賜親王以次內大臣、侍衛白金有差。

是歲,免直隸、江南、江西、福建、湖廣等省三十二州縣衛災賦有差。朝鮮、琉球入貢。

四十七年戊子春正月庚午,浙江大嵐山賊張念一、硃三等行劫慈谿、上虞、嵊縣,官兵捕平之。辛未,重修南嶽廟成,禦制碑文。以覺羅孟俄洛為奉天將軍。乙亥,詔截留湖廣、江西漕糧四十萬石,留於江南六府平糶。

二月庚寅,上御經筵。壬辰,遣侍郎穆丹按大嵐山獄,學士二鬲按紅苗獄。甲午,上巡畿甸。丙午,詔暹羅使臣挈帶土貨,許隨處貿易,免徵其稅。

三月丙辰,上還駐申昜春園。戊午,以希思哈、李繩宗為漢軍都統。

閏三月戊寅朔,重修北鎮廟成,禦制碑文。乙未,以施世驃為廣東提督,席柱為西安將軍。

夏四月己酉,宋犖罷,以徐潮為吏部尚書,以齊世武為四川陜西總督。戊午,山東巡撫趙世顯報捕獲硃三父子,解往浙江。上曰:「硃三父子游行教書,寄食人家。若因此捕拿,株連太多,可傳諭知之。」辛酉,湖廣提督俞益謨密請剿除紅苗。上以紅苗無大罪,不許。以阿喇衲為蒙古都統,李林盛為漢軍都統。內大臣明珠卒,命皇三子胤祉奠茶酒,賜馬四匹。

五月甲申,以王鴻緒為戶部尚書,富寧安為禮部尚書,穆和倫為左都御史。丙戌,上巡幸塞外。乙未,詔免大嵐山賊黨太倉人王昭駿伯叔兄弟連坐罪。

六月丁未,上駐蹕熱河。丁巳,九卿議覆大嵐山獄上,得旨:「誅其首惡者,硃三父子不可宥,緣坐可改流徙。巡撫王然、提督王世臣俱留任,受傷官兵俱議敘。」丁卯,清文鑒成,上制序文。

秋七月丁丑,諭刑部,免死流人在配犯罪者按誅之。癸未,平定朔漠方略成,上親制序文。壬辰,上行圍。二鬲奏按紅苗殺人之廖老宰等斬梟,擅自遣兵前往苗寨之守備王應瑞遣戍,從之。

八月甲辰朔,日有食之。壬戌,上回鑾,駐永安拜昂阿。

九月乙亥,上駐布爾哈蘇臺。丁丑,召集廷臣行宮,宣示皇太子胤礽罪狀,命拘執之,送京幽禁。己丑,上還京。丁酉,廢皇太子胤礽,頒示天下。

冬十月甲辰,削貝勒胤禩爵。乙卯,以王掞為工部尚書,張鵬翮為刑部尚書。辛酉,上幸南苑行圍。以辛泰為蒙古都統。

十一月癸酉朔,削直郡王胤禔爵,幽之。己卯,致仕大學士張英卒,予祭葬,謚文端。辛巳,副都御史勞之辨奏保廢太子,奪職杖之。丙戌,召集廷臣議建儲貳。阿靈阿、鄂倫岱、揆敘、王鴻緒及諸大臣以皇八子胤禩請。上不可。戊子,釋廢太子胤礽。己丑,王大臣請復立胤礽為皇太子。丙申,以宗室發度為黑龍江將軍。庚子,復胤禩貝勒。

十二月甲辰,褒恤死難生員嵇永仁、王龍光、沈天成、範承譜,附祀範承謨祠,承謨子巡撫範時崇請之也。丁巳,以陳詵為湖廣巡撫,蔣陳錫為山東巡撫,黃秉中為浙江巡撫,劉廕樞為貴州巡撫。

是歲,免山東、福建、湖廣等省六十州縣災賦有差。朝鮮入貢。

四十八年己丑春正月癸巳,召集廷臣問舉立胤禩,孰為倡議者。群臣皇恐莫敢對,乃進大學士張玉書而問之,對曰:「先聞之馬齊。」上切責之。次日,列馬齊罪狀,宥死拘禁。已而上徐察其誣,釋之。丙申,上幸南苑。己亥,命侍郎赫壽駐藏,協辦藏事。初拉藏汗與青海爭立達賴喇嘛,不決,特命大臣往監臨之。王鴻緒、李振裕免。

二月己酉,上巡幸畿甸。以宗室楊福為黑龍江將軍,覺羅孟俄洛為寧古塔將軍,王文義為貴州提督。戊午,以嵩祝署奉天將軍。戊辰,上還宮。庚午,以張鵬翮為戶部尚書,張廷樞為刑部尚書。

三月辛巳,復立胤礽為皇太子,昭告宗廟,頒詔天下。甲午,賜趙熊詔等二百九十二人進士及第出身有差。

夏四月甲辰,以富寧安為吏部尚書,穆和倫為禮部尚書,穆丹為左都御史。移禁胤禔於公所,遣官率兵監守。丁卯,上巡幸塞外。

五月甲戌,上駐蹕熱河。

六月戊午,康親王椿泰薨,謚曰悼,子崇安襲封。

秋七月庚寅,以殷泰為四川陜西總督,噶禮為江南江西總督,江琦為甘肅提督,師懿德為江南提督。戊戌,上行圍。

八月己亥朔,日有食之。加陜西提督潘育龍鎮綏將軍。

九月庚寅,上還京。以年羹堯為四川巡撫。

冬十月壬寅,詔福建、廣東督撫保舉深諳水性熟知水師者。戊午,冊封皇三子胤祉誠親王,皇四子胤禛雍親王,皇五子胤祺恆親王,皇七子胤祐淳郡王,皇十子胤蓪敦郡王,皇九子胤禟、皇十二子胤祹、皇十四子胤禵俱為貝勒。壬戌,詔免江蘇被災之淮、揚、徐,山東之兗州,河南之歸德明年地丁額賦。

十一月丙子,詔各省解部之款過多,可酌量截留,以備急需。安郡王馬爾渾薨,謚曰愨,子華颺襲。己卯,加漕運總督桑額太子太保。庚寅,上與大學士李光地論水ャ水源,泰、岱諸山自長白山來。水伏流,黃河未到積石亦是伏流,蒙古人有書言之甚詳。江源亦自昆侖來,至於岷山乃不伏流耳。遣張鵬翮、噶敏圖按江南宜思恭虧帑獄。

十二月己亥,上謁陵。己未,上還宮。命馬齊管鄂羅斯貿易事。刑部尚書巢可託免。

是歲,免直隸、江蘇、安徽、山東、河南、湖廣等省五十三州縣災賦有差。朝鮮、琉球入貢。

四十九年庚寅春正月庚寅,命修滿蒙合璧清文鑒。

二月丁酉,上巡幸五臺山。吏部尚書徐潮乞休,允之。

三月己巳,上還京。乙亥,命編纂字典。詔以故大學士李霨嫡孫主事李敏啟擢補太常寺少卿。戊寅,敕封西藏胡必爾汗波克塔為六世達賴喇嘛。辛巳,詔免浙江杭、湖二府未完漕米三萬九千餘石。

夏四月乙巳,調蕭永藻為吏部尚書,王掞為兵部尚書。

五月己酉朔,上巡幸塞外。癸酉,次花峪溝。閱吉林、黑龍江官兵。丁丑,上駐蹕熱河。

六月己亥,命諸皇子恭迎皇太后至熱河避暑。戊午,刑部尚書張廷樞免。

秋七月壬午,按事湖南尚書蕭永藻等疏報巡撫提督互訐案,查審俱實。得旨:「俞益謨休致,趙申喬革職留任。」

閏七月甲寅,上行圍。

八月乙亥,詔福建漳、泉二府旱,運江、浙漕糧三十萬石賑之,並免本年未完額賦。丙戌,上還駐熱河。庚寅,以範時崇為福建浙江總督,額倫特為湖南提督。

九月辛丑,上奉皇太后還宮。辛亥,希福納免。時戶部虧蝕購辦草豆銀兩事覺,積十餘年,歷任尚書、侍郎凡百二十人,虧蝕至四十餘萬。上寬免逮問,責限償完,希福納現任尚書,特斥之。以穆和倫為戶部尚書,貝和諾為禮部尚書。

冬十月甲子,詔曰:「朕臨御天下垂五十年,誠念民為邦本,政在養民。迭次蠲租數萬萬,以節儉之所餘,為渙解之弘澤。惟體察民生,未盡康阜,良由生齒日繁,地不加益。宜沛鴻施,藉培民力。自康熙五十年始,普免天下錢糧,三年而遍。直隸、奉天、浙江、福建、廣東、廣西、四川、雲南、貴州九省地丁錢糧,察明全免。歷年逋賦,一體豁除。其五十一年、五十二年應蠲省分,屆時候旨。地方大吏以及守令當體朕保乂之懷,實心愛養,庶幾升平樂利有可徵矣。文到,其刊刻頒布,咸使聞知。」丁卯,諭外籓已朝行在,勿庸朝正。丙子,以郭瑮為雲南貴州總督,以郭世隆為刑部尚書,鄂海為湖廣總督。癸未,諭大學士:「江南虧空錢糧多至數十萬兩,此或朕數次南巡,地方挪用。張鵬翮謂俸工可以抵補。牧令無俸,仍以累民,莫若免之為善。其會議以聞。」

十一月辛卯朔,詔凡遇蠲賦之年,免業主七分,佃戶三分,著為令。大學士陳廷敬以老乞休,溫旨慰諭,命致仕。乙巳,上謁陵。以蕭永藻為大學士,王掞為禮部尚書,徐元正為工部尚書。丁未,以孫徵灝為兵部尚書。乙卯,以桑額為吏部尚書。

十二月癸酉,以赫壽為漕運總督。戊寅,上還京。辛巳,詔曰:「朕因朝列舊臣漸次衰謝,順治年間進士去職在籍者,已無多人。王士禛、江皋、周敏政、葉矯然、徐淑嘉皆以公過屏廢,俱復還原官。」以趙申喬為左都御史。

是歲,免直隸、江南等省七州縣災賦有差。朝鮮、安南入貢。

五十年辛卯春正月癸丑,上巡畿甸,視通州河堤。

二月辛酉,以班迪為滿洲都統,善丹為蒙古都統。丁卯,閱筐兒港,命建挑水壩。次河西務,上登岸步行二里許,親置儀器,定方向,釘椿木,以紀丈量之處。諭曰:「用此法可以測量天地、日月交食。算法原於易。用七九之奇數,不能盡者,用十二、二十四之偶數,乃能盡之,即取象十二時、二十四氣也。」庚午,上還京。辛巳,上御經筵。

三月庚寅,王大臣以萬壽節請上尊號。自平滇以來,至是凡四請矣。上謙挹有素,終不之許。

夏四月庚申,徐元正養親回籍,以陳詵為工部尚書。庚辰,上奉皇太后避暑熱河。乙未,命禮部祈雨。庚子,大雨。丙午,留京大學士張玉書卒,上悼惜,賦詩一篇,遣官治喪,賜銀一千兩,加祭葬,謚文貞。己酉,詔免江蘇無著銀十萬兩有奇。丙辰,召致仕大學士陳廷敬入閣辦事。增鄉、會試五經中額。

六月戊辰,設廣西西隆州儒學訓導。

秋七月丙辰,上行圍。

八月庚午,高宗純皇帝生。以王原祁為掌院學士。設先賢子游後裔五經博士。

九月戊申,上奉皇太后還宮。藍理有罪免,以楊琳為福建陸路提督,馬際伯為四川提督。停本年秋決。

冬十月丙辰,詔免臺灣五十一年應徵稻穀。貝和諾免,以嵩祝為禮部尚書。戊午,詔前旨普免天下錢糧,五十一年輪及山西、河南、陜西、甘肅、湖北、湖南六省,地丁錢糧及逋欠俱行蠲免。庚午,以碩鼐為滿洲都統,瑚世巴、馬爾賽為蒙古都統。戊寅,免朝鮮白金豹皮歲貢。庚辰,詔舉孝義。辛巳,命張鵬翮置獄揚州,按江南科場案。壬午,鄂繕、耿額、齊世武、悟禮等有罪,褫職拘禁。趙申喬疏劾新科編修戴名世恃才放蕩,語多悖逆,下部嚴審。

十一月丙戌,以殷特布為漢軍都統,隆科多為步軍統領,張穀貞為雲南提督。丁未,上謁陵,賜守陵官役馬匹白金。

十二月癸酉,上還宮。癸未,祫祭太廟。

是歲,免直隸、安徽等省八州縣災賦有差。朝鮮、琉球入貢。丁戶二千四百六十二萬一千三百二十四,田地六百九十三萬三百四十四頃三十四畝,徵銀二千九百九十萬四千六百五十二兩八錢。鹽課銀三百七十二萬九千二百二十八兩。鑄錢三萬七千四百九十三萬三千四百有奇。

五十一年壬辰春正月丙午,擢編修張逸少為翰林院侍讀學士,故大學士張玉書之子也。壬子,命內外大臣具摺陳事。摺奏自此始。癸丑,上巡幸畿甸。詔右衛將軍宗室費揚古辦事誠實,供職年久,且系王子,可封為輔國公。

二月丁巳,詔宋儒硃子配享孔廟,在十哲之次。江蘇巡撫張伯行與總督噶禮互訐,俱解任,交張鵬翮、赫壽查審。福建浙江總督範時崇疏陳沿海漁船,只許單桅,不許越省行走,交地方文武鈐束。上曰:「此事不可行。漁戶並入水師營,則兵弁侵欺之矣。盜賊豈能盡除,竊發何地無之?只視有益於民者行之,不當以文法為捕具也。」戊寅,命卓異武官照文職引見。庚辰,上還京。壬午,詔曰:「承平日久,生齒日繁。嗣後滋生戶口,勿庸更出丁錢,即以本年丁數為定額,著為令。」

三月辛卯,諭大學士:「繙譯本章,甚有關系。昨見本內『假官』二字,竟譯作『偽官』,舛錯殊甚。其嚴飭之。」丁酉,上御經筵。

夏四月丁巳,賜王世琛等百七十七人進士及第出身有差。甲子,以康泰為四川提督。定會試分省取中例。壬申,諭:「故大學士熊賜履夙學舊臣,身歿以後,時軫於懷。聞其子已長成,可令來京錄用。」壬戌,予故一等待衛海青副都統銜,予祭葬,謚果毅。致仕大學士陳廷敬卒,命皇三子奠茶酒,御賦輓詩,命南書房翰林勵廷儀、張廷玉齎焚,予治喪銀一千,謚文貞。詔明年六旬萬壽,二月特行鄉試,八月會試。以嵩祝為大學士,黑碩咨為禮部尚書,滿篤為工部尚書,以王掞為大學士,陳詵為禮部尚書,起張廷樞為工部尚書。丙子,上奉皇太后避暑熱河,啟鑾。壬午,上駐蹕熱河。

五月壬寅,命有司稽察流民徙邊種地者。以穆丹為左都御史,鄂代為蒙古都統。

六月癸丑朔,日有食之。丁巳,命穆和倫、張廷樞覆按江南督撫互訐案。湖廣鎮筸紅苗吳老化率毛都塘等五十二寨內附。辛酉,以張朝午為廣西提督。

秋八月癸丑,上行圍。戊寅,詔朝鮮遇有中國漁船違禁至界汛,許拘執以聞。鎮筸苗民續內附八十三寨。

九月庚戌,上奉皇太后還宮。皇太子胤礽復以罪廢,錮於咸安宮。

冬十月壬戌,穆和倫等覆按江南獄上,上命奪噶禮職,張伯行復任。以揆敘為左都御史,赫壽為江南江西總督。

十一月乙酉,前福建提督藍理獄上,應死。上念征臺灣功,特原之。己亥,群臣以萬壽六旬請上尊號,不許。丁未,以復廢皇太子胤礽告廟,宣示天下。己酉,上謁陵,賜守陵大臣白金。

十二月甲戌,上還京。

是歲,免直隸、江南、山東、浙江等省二十三州縣災賦有差。朝鮮入貢。

五十二年癸巳春正月戊申,詔封後藏班禪胡土克圖喇嘛為班禪額爾得尼。

二月庚戌,趙申喬疏言太子國本,應行冊立。上以建儲大事,未可輕定,宣諭廷臣,以原疏還之。乙卯,上巡幸畿甸。編修戴名世以著述狂悖棄市。進士方苞以作序干連,免死入旗,旋赦出之。乙亥,上還駐申昜春園。

三月戊寅朔,諭王大臣:「朕昨還京,見各處為朕保釐乞福者,不計其數,實覺愧汗。萬國安,即朕之安,天下福,即朕之福,祝延者當以茲為先。朕老矣,臨深履薄之念,與日俱增,敢滿假乎?」又諭:「各省祝壽老人極多,倘有一二有恙者,可令太醫看治。朕於十七日進宮經棚,老人已得從容瞻覲。十八日正陽門行禮,不必再至龍棚。各省漢官傳諭知悉。」甲午,上還宮,各省臣民夾道俯伏歡迎,上駐輦慰勞之。乙未,萬壽節,上朝慈寧宮,御太和殿受賀,頒詔覃恩,錫高年,舉隱逸,旌孝義,蠲逋負,鰥寡孤獨無告者,官為養之,罪非殊死,咸赦除焉。壬寅,召直省官員士庶年六十五以上者,賜宴於申昜春園,皇子視食,宗室子執爵授飲。扶掖八十以上老人至前,親視飲酒。諭之曰:「古來以養老尊賢為先,使人人知孝知弟,則風俗厚矣。爾耆老當以此意告之鄉里。昨日大雨,田野霑足。爾等速回,無誤農時。」是日,九十以上者三十三人,八十以上者五百三十八人,各賜白金。加祝釐老臣宋犖太子少師,田種玉太子少傅。甲辰,宴八旗官員、兵丁、閒散於申昜春園,視食授飲、視飲賜金同前。是日,九十以上者七人,八十以上者一百九十二人。

夏四月甲寅,以鄂海為陜西四川總督,額倫特為湖廣總督,高其位為湖廣提督。四川提督岳升龍請入籍四川,許之。丁卯,遣官告祭山川、古陵、闕里。五月丙戌,上奉皇太后避暑熱河。調張廷樞為刑部尚書,王頊齡為工部尚書。頒賚蒙古老人白金。辛丑,詔停本年秋決。

閏五月乙卯,賚熱河老人白金。御史陳汝咸招撫海寇陳尚義入見,詢海上情勢及洋船形質,命於金州安置,置水師營。

六月丁丑,修律算書。

秋七月壬子,詔宗人削屬籍者,子孫分別系紅帶、紫帶,載名玉牒。丙寅,上行圍。

八月丁丑,蒙古鄂爾多斯王松阿拉布請於察罕託灰游牧,不許,命游牧以黃河為界,從總兵範時捷請也。

九月甲子,上奉皇太后還宮。辛未,以江南漕米十萬石分運廣東、福建平糶。

冬十月丙子,以張鵬翮為吏部尚書。乙酉,賜王敬銘等一百四十三人進士及第出身有差。

十一月己酉,詔免廣東、福建、甘肅二十一州縣衛明年稅糧。癸亥,上謁陵。

十二月己卯,以赫奕為工部尚書。辛卯,令文武科目原兼應者,許改試一科。壬辰,上還京。甲午,以五鬲為蒙古都統。辛丑,祫祭太廟。

是歲,免浙江十州縣災賦有差。朝鮮、琉球入貢。

五十三年甲午春正月己未,命修壇廟殿廷樂器。癸亥,戶部請禁小錢。上曰:「凡事必期便民,若不便於民,而惟言行法,雖厲禁何益。」戊辰,上巡幸畿甸。丁卯,以何天培為京口將軍。

二月甲戌,詔停今年秋審,矜疑人犯,審理具奏,配流以下,減等發落。乙酉,上還京。癸丑,命侍郎常泰、少卿陳汝咸赴甘肅賑撫災民。丁巳,前尚書王鴻緒進明史列傳二百八十卷,命付史館。

夏四月戊子,改師懿德為甘肅提督。辛卯,上奉皇太后避暑熱河。六月乙亥,詔:「拉藏汗年近六十,二子在外,宜防外患,善自為謀。」癸未,以炎暑免從臣晚朝。

秋七月辛卯,詔以江南又旱,浙江米貴,河南歉收,截漕三十萬石,分運三省平糶。

八月乙亥,上行圍。

九月丙寅,上奉皇太后還宮。

冬十月己巳朔,命張鵬翮、阿錫鼐往按江南牟欽元獄。己丑,命大學士、南書房翰林考定樂章。

十一月,敕戶部截漕三十餘萬石,於江南、浙江備賑。戊申,免甘肅靖邊二十八州縣衛明年額賦。誠親王胤祉等以禦制律呂正義進呈,得旨:「律呂、歷法、算法三書共為一部,名曰律歷淵源。」甲寅,冬至,祀天於圜丘,奏新樂。丙辰,上巡幸塞外。貝勒胤禩屬下人雅齊布有罪伏誅。遣何國棟測量廣東、雲南等省北極出地及日景。

十二月癸酉,上駐特布克,賜隨圍蒙古兵銀幣。己丑,上還京。辛卯,洮、岷邊外生番喇子等一十九族內附。

是歲,免江南、河南、甘肅、浙江、湖廣等省百二十二州縣災賦有差。朝鮮入貢。

五十四年乙未春正月甲子,停五經中式例。封阿巴垓臺吉德木楚克為輔國公。詔貝勒胤禩、延壽溺職,停食俸。

二月戊辰朔,張伯行緣事解任,交張鵬翮審理。己巳,以施世綸為漕運總督。辛未,上巡幸畿甸,諭巡撫趙弘燮曰:「去年臘雪豐盈,今年春雨應節,民田想早播種。但慮起發太盛,或有二疸之虞。可示農民蕓耨宜疏,以防風霾。」又諭:「朕時巡畿甸,見民生差勝於前。但誦讀者少,風俗攸關。宜令窮僻鄉壤廣設義學,勸令讀書。爾有司其留意。」甲午,以杜呈泗為江南提督,穆廷栻為福建陸路提督。

三月己亥,以蒙古吳拉忒等部十四旗雪災,命尚書穆和倫運米往賑,教之捕魚為食。庚子,以趙弘燮為直隸總督,任巡撫事。以睦森為寧古塔將軍。

夏四月庚午,賜徐陶璋等一百九十人進士及第出身有差。己卯,師懿德奏策旺阿拉布坦兵掠哈密,游擊潘至善擊敗之。命尚書富寧安、將軍席柱率師援剿,祁里德赴推河,諭喀爾喀等備兵。庚辰,徵外籓兵集歸化城,調打牲索倫兵赴推河。己丑,諭議政大臣:「朕曾出塞親征,周知要害。今討策旺阿拉布坦進兵之路有三:一由噶斯直抵伊里河源,趨其巢穴;一越哈密、吐魯番,深略敵境;一取道喀爾喀,至博克達額倫哈必爾漢,度嶺扼險。三路並進,大功必成。」壬午,漕運總督郎廷極卒,上稱其撫恤運丁,歷運無阻,予祭葬,謚溫勤。辛卯,上奉皇太后避暑熱河。乙未,命富寧安分兵戍噶斯口,總兵路振聲駐防哈密。

五月丙午,黑龍江將軍宗室楊福卒,賜銀一千兩,命侍衛尚崇義、傅森馳驛賜奠,謚襄毅,命其子三官保暫署父任。戊午,內閣侍讀圖理琛使於鄂羅斯,使備兵。

六月壬申,命都統圖斯海等赴湖灘河朔運糧。甲戌,富寧安、席柱疏報進兵方略。得旨,明年進兵。丁亥,兵部尚書孫徵灝卒,賜鞍馬二、散馬二、銀五百兩,謚清端。

秋七月甲午朔,命和托輝特公博貝招撫烏梁海。辛酉,命公傅爾丹往烏蘭古等處屯田。

八月辛未,大學士李光地乞假歸,上賦詩送之。癸酉,上行圍。壬辰,撤噶斯口戍兵還肅州。

九月己酉,博貝招撫烏梁海部來歸。

冬十月丙寅,上諭大學士:「朕右手病不能寫字,用左手執筆批答奏摺,期於不洩漏也。」辛巳,上奉皇太后還宮。詔順天、保定、河間、永平、宣化今歲雨溢,穀耗不登,所有五府應完五十五年稅糧,悉蠲除之。

十一月甲午,以範時崇為左都御史,覺羅滿保為浙江福建總督,宗室巴塞為蒙古都統。庚子,停京師決囚。辛丑,以宋臣範仲淹從祀孔廟。己未,冬至,祀天於圜丘,始用御定雅樂。

十二月己巳,以塔拜為杭州將軍。命護軍統領晏布帥師駐西寧。甲申,張伯行以疑贓誣參論罪應死,上原之,起為倉場侍郎。

是歲,免江南、湖南二省二十四州縣衛災賦有差。朝鮮、琉球入貢。

五十五年丙申春正月壬子,上幸湯泉。

二月乙丑,命副都統蘇爾德經理圖呼魯克等處屯田。癸酉,上還駐申昜春園。丙子,詔免安南歲貢犀角、象牙。己卯,上巡幸畿甸。庚寅,定丁隨地出例。

三月丁酉,恤贈廣西右江剿瑤傷亡參將王啟雲官廕。庚子,上還宮。乙巳,召席柱還,以晏布代之,路振聲參軍事。癸丑,蒙古圖爾胡特貝子阿拉布珠爾請從軍。命率蒙古兵戍噶斯口。貴州巡撫劉廕樞疏請罷兵,命乘傳詣軍周閱議奏。

閏三月癸亥,以額倫特署西安將軍,滿丕署湖廣總督。丁丑,以左世永為廣西提督。壬午,發京倉米二十萬石賑順天、永平。五城粥廠展期至秋。命禮部祈雨。

夏四月癸卯,上奉皇太后避暑熱河。

五月庚申,上駐熱河,齋居祈雨。起馬齊為大學士,穆和倫為戶部尚書。壬戌,發倉米平糶。預發八旗兵糧。甲子,雨。上曰:「宋儒云:『求雨得雨,旱豈無因。』此言可味也。」己巳,京師遠近雨足,上復常膳。乙酉,赫奕免,以孫渣齊為工部尚書。

六月丙辰,上幸湯泉。

秋七月辛未,命移噶斯口防軍分戍察罕烏蘇、噶順。癸未,上行圍。

八月乙卯,前奉天府尹董弘毅坐將承德等九州縣米豆改徵銀兩,致倉儲闕乏,黜官。

九月庚午,以蔣陳錫為雲南貴州總督。甲申,上奉皇太后還宮。

冬十月丁亥朔,詔刑部積歲緩決長系人犯,分別減釋之。停本年秋決。戊子,以托留為黑龍江將軍,趙弘燦為兵部尚書。癸巳,詔:「近以策旺阿拉布坦侵入哈密,徵兵備邊,一切飛芻輓粟經過邊境,不無借資民力。所有山西、陜西、甘肅四十八州縣衛應徵明年銀米穀草及積年逋欠,悉與蠲除。」丁酉,詔肅州與布隆吉爾毗連迤北西吉木、達里圖、金塔寺等處,招民墾種。以楊琳為廣東廣西總督。以宗室巴賽為滿洲都統,晏布為蒙古都統。丙午,策旺阿拉布坦執青海臺吉羅卜藏丹濟布,犯噶斯口,官兵擊走之。命額倫特駐師西寧,分兵戍噶斯口,布隆吉爾散秩大臣阿喇衲赴巴爾庫爾參贊軍事。

十一月乙丑,以傅爾丹、額爾錦為領侍衛內大臣。戊辰,上謁陵。甲申,上巡行塞外。盜發明陵,命置之法。

十二月己酉,上還京。詔免順天、永平三十五州縣明年地丁稅糧,其積年逋賦並除之。

是歲,免直隸、江南、山東、浙江、江西、湖廣等省六十三州縣災賦有差。朝鮮、安南入貢。

五十六年丁酉春正月丁卯,修周易折中成,頒行學宮。壬午,以徐元夢為左都御史,硃軾為浙江巡撫。

二月丙戌朔,上巡幸畿甸。乙未,徵奉天、吉林兵益祁里德軍。癸卯,上還駐申昜春園。丁未,定盜案法無可寬、情有可原例。順承郡王諾羅布薨,謚曰忠,子錫保襲封。左都御史揆敘卒,予祭葬,謚文端。

三月丁巳,上御經筵。戊寅,以富寧安為靖逆將軍,傅爾丹為振武將軍,祁里德為協理將軍,視師防邊。壬午,上巡視河西務堤。

夏四月乙酉,上還駐申昜春園。乙未,發通州倉米分貯直隸州縣備賑。丙申,碣石鎮總兵陳昂奏天主教堂各省林立,宜行禁止,從之。以孫柱、範時崇為兵部尚書。辛丑,上奉皇太后避暑熱河。

五月庚申,九卿議王貝勒差人出外,查無勘合,即行參究。

六月壬子,傅爾丹襲擊厄魯特博羅布爾哈蘇,斬俘而還。兵部尚書趙弘燦卒,予祭葬,謚清端。

秋七月丙辰,策旺阿拉布坦遣其將策零敦多布侵掠拉藏。癸亥,富寧安襲擊厄魯特於通俄巴錫,進及烏魯木齊,毀其田禾,還軍遇賊畢留圖,擊敗之。陣亡灰特臺吉扎穆畢,追封輔國公。

八月壬午朔,上行圍。

九月辛未,以路振揚署四川提督。河南奸民亢珽滋事,官兵捕之,珽走死。命尚書張廷樞、學士勒什布往鞫,得前巡撫李錫貪虐激變狀以聞。李錫褫職論死,賊黨伏誅。

冬十月乙酉,命侍郎梁世勛、海壽往督巴爾庫爾屯田。庚子,上奉皇太后還宮。乙巳,命內大臣公策旺諾爾布、將軍額倫特、侍衛阿齊圖等率師戍青海。以宗室公吞珠為禮部尚書,蔡升元為左都御史。

十一月壬子,命停決囚。乙丑,皇太后不豫,上省疾慈寧宮。辛未,詔曰:「帝王之治,必以敬天法祖為本。合天下之心以為心,公四海之利以為利,制治於未亂,保邦於未危,夙夜兢兢,所以圖久遠也。朕八齡踐祚,在位五十餘年,今年近七旬矣。當二十年時,不敢逆計至三十。三十年時,不敢逆計至四十。賴宗社之靈,今已五十七年矣,非涼德所能致也。齒登耆壽,子孫眾多。天下和樂,四海乂安。雖未敢謂家給人足,俗易風移,而欲使民安物阜之心,始終如一。占竭思慮,耗敝精力,殆非勞苦二字所能盡也。古帝王享年不永,書生每致譏評。不知天下事煩,不勝其勞慮也。人臣可仕則仕,可止則止,年老致仕而歸,猶得抱子弄孫,優游自適。帝王仔肩無可旁委,舜歿蒼梧,禹殂會稽,不遑寧處,終鮮止息。洪範五福,終於考終命,以壽考之難得也。易遯六爻,不及君主,人君無退藏之地也。豈當與臣民較安逸哉!朕自幼讀書,尋求治理。年力勝時,挽強決拾。削平三籓,綏輯漠北,悉由一心運籌,未嘗妄殺一人。府庫帑金,非出師賑饑,未敢妄費。巡狩行宮,不施採繢。少時即知聲色之當戒,佞倖之宜遠,幸得粗致謐安。今春頗苦頭暈,形漸羸瘦。行圍塞外,水土較佳,體氣稍健,每日騎射,亦不疲乏。復以皇太后違和,頭暈復作,步履艱難。倘一時不諱,不得悉朕衷曲。死者人之常理,要當於明爽之時,舉平生心事一為吐露,方為快耳。昔人每云帝王當舉大綱,不必兼綜細務。朕不謂然,一事不謹,即貽四海之憂;一念不謹,即貽百年之患。朕從來蒞事無論鉅細,莫不慎之又慎。惟年既衰暮,祗懼五十七年憂勤惕勵之心,隳於末路耳。立儲大事,豈不在念。但天下大權,當統於一,神器至重,為天下得人至難,是以朕垂老而惓惓不息也。大小臣工能體朕心,則朕考終之事畢矣。茲特召諸子諸卿士詳切言之。他日遺詔,備於此矣。」甲戌,免八旗借支銀二百萬兩。丙子,詔免直隸、安徽、江蘇、浙江、湖廣、陜西、甘肅等省積年逋賦,江蘇、安徽並免漕項銀米十分之五。

十二月甲申,皇太后病勢漸增,上疾七十餘日矣,腳面浮腫,扶掖日朝寧壽宮。丙戌,皇太后崩,頒遺誥,上服衰割辮,移居別宮。己酉,上還宮。

是歲,朝鮮入貢。

五十七年戊戌春正月乙卯,上有疾,幸湯泉。戊寅,賜防邊軍士衣二萬襲。

二月庚寅,拉藏乞師,命侍衛色楞會青海兵往援。癸卯,以路振聲為甘肅提督。檢討硃天保上疏請復立胤礽為皇太子,上於行宮親訊之曰:「爾何知而違旨上奏?」硃天保曰:「臣聞之臣父,臣父令臣言之。」上曰:「此不忠不孝之人也。」命誅之。丁未,上還宮。碣石鎮陳昂疏請洋船入港,先行查取大砲,方許進口貿易。部議不行。

三月癸丑,減大興、宛平門廠房稅。辛酉,上大行皇后謚曰孝惠仁憲端懿純德順天翊聖章皇后。丙寅,以顏壽為右衛將軍,黃秉鉞為福州將軍。戊辰,裁起居注官。甄別不職學政叢澍等七員,俱褫職。丁丑,命浙江南北新關稅交同知管理。戊寅,浙江巡撫硃軾請修海寧石塘。從之。

夏四月乙酉,葬孝惠章皇后於孝東陵。丁亥,賜汪應銓等一百七十一人進士及第出身有差。辛卯,上幸熱河。穆和倫免,以孫渣齊為戶部尚書。

五月癸丑,以徐元夢為工部尚書。丁巳,額倫特奏拉藏汗被陷身亡,二子被殺,達賴、班禪均被拘。己未,浙江福建總督滿保疏臺灣一郡有極沖口岸九處,次沖口岸十五處,派人修築,酌移員弁,設淡水營守備。從之。

六月壬辰,遣使冊封琉球故王曾孫尚敬為中山王。己丑,大學士李光地卒,命皇五子恆親王胤祺往奠茶酒,賜銀一千兩,徐元夢還京護其喪事,謚文貞。丁未,賜哈密軍士衣四百襲。

秋七月己未,打箭爐外墨裡喇嘛內附。甲戌,修省方盛典。

八月壬子,索倫水災,遣官賑之。孟光祖伏誅。戊子,上行圍。甲午,禮部尚書吞珠卒,予祭葬,謚恪敏。總兵官仇機有罪伏誅。

閏八月戊辰,詔曰:「夷虜跳梁,大兵遠駐西邊,一切征繕,秦民甚屬勞苦。所有陜西、甘肅明年地丁糧稅俱行蠲免,歷年逋賦亦盡除之。」

九月己卯,命都統阿爾納、總兵李耀率師赴噶斯口、柴旦木駐防。丙戌,以王頊齡為大學士,陳元龍為工部尚書。甲辰,上還京。將軍額倫特、侍衛色楞會師喀喇烏蘇,屢敗賊,賊愈進,師無後繼,矢竭力戰,歿於陣。

冬十月甲寅,停本年決囚。丙辰,命皇十四子貝子胤禵為撫遠大將軍,視師青海。命殉難總督甘文焜、知府黃庭柏建祠列祀。甲子,詔四川巡撫年羹堯,軍興以來,辦事明敏,即升為總督。命翰林、科道輪班入直。戊辰,上駐湯泉。命皇七子胤祐、皇十子胤蓪、皇十二子胤祹分理正黃、正白、正藍滿、蒙、漢三旗事務。

十一月丙子,上還駐申昜春園。福建巡撫陳瑸卒,贈禮部尚書,謚清端。以宜兆熊為漢軍都統。

十二月丙辰,上謁陵。己未,孝惠章皇后升祔太廟,位於孝康章皇后之左,頒詔天下。雲南撒甸苗人歸順。己巳,上還宮。

是歲,免江南、福建、甘肅、湖廣等省二十六州縣衛災賦有差。朝鮮、琉球、安南入貢。

五十八年己亥春正月甲戌朔,日有食之。詔曰:「日食三始,垂象維昭。宜修人事,以儆天戒。臣工其舉政事闕失以聞。」乙未,上幸湯泉。庚子,上還駐申昜春園。辛丑,詔立功之臣退閒,世職準子弟承襲。若無應襲之人,給俸終其身。壬寅,命截漕米四十三萬石,留江蘇、安徽備荒。

二月己巳,上巡幸畿甸。己卯,學士蔣廷錫表進皇輿全覽圖,頒賜廷臣。庚申,上還駐申昜春園。辛未,命都統法喇撫輯里塘、巴塘,護軍統領噶爾弼同理軍事。

三月乙未,侍郎色爾圖以運饟遲延罷,命巡撫噶什圖接管。

夏四月乙巳,命撫遠大將軍胤禵駐師西寧。癸丑,上巡幸熱河。

五月戊寅,以麥大熟,命民間及時收貯。庚辰,以揚都為蒙古都統。浙江正考官索泰賄賣關節,在籍學士陳恂說合,陳鳳墀夤緣中式,均論死,並罪其保薦索泰為考官者。南陽標兵執辱知府沈淵,總兵高成革職,游擊王洪道論死,兵處斬。

六月甲辰,以貝勒滿篤祜為滿洲都統。丁未,年羹堯、噶爾弼、法喇先後奏副將岳鍾琪招輯里塘、巴塘就撫。命法喇進駐巴塘,年羹堯撥兵接應。丙寅,以馬見伯為固原提督。

秋七月癸未,以宗查木為西安將軍。

八月庚戌,上行圍。庚申,振威將軍傅爾丹奏鄂爾齋圖二處築城設站。命尚書範時崇往董其役。

九月乙未,諭西寧現有新胡畢勒罕,實系達賴後身,令大將軍遣官帶兵前往西藏安禪。戊戌,安郡王華颺薨,謚曰節。

冬十月丁未,上還京。壬子,命蒙養齋舉人王蘭生修正音韻圖。甲寅,固原提督潘育龍卒,贈太子少保,予祭葬,謚襄勇。

十一月丙子,禮部尚書陳詵致仕。庚寅,增江西解額。

十二月壬寅,以蔡升元為禮部尚書,田從典為左都御史。戊申,西安將軍額倫特之喪至京,命皇五子恆親王胤祺、皇十二子貝子胤祹迎奠。庚申,命截湖廣漕糧十萬石留於本省備荒。辛酉,詔曰:「比年興兵西討,遠歷邊陲,居送行賚,民力勞瘁。所有沿邊六十六州縣衛所明年額徵銀米,俱行蠲免。」

是歲,免江蘇、安徽等省十三州縣災賦有差。朝鮮、琉球入貢。

五十九年庚子春正月丁酉,命撫遠大將軍胤禵移師穆魯斯烏蘇。以宗室延信為平逆將軍,領兵進藏,以公策旺諾爾布參贊軍務。命西安將軍宗查木駐西寧,平郡王訥爾素駐古木。

二月甲辰,上巡幸畿甸。癸丑,命噶爾弼為定西將軍,率四川、雲南兵進藏,冊封新胡畢勒罕為六世達賴喇嘛。辛酉,上還駐申昜春園。

三月己丑,命雲南提督張穀貞駐防麗江、中甸。丙申,命靖逆將軍富寧安進師烏魯木齊,散秩大臣阿喇衲進師吐魯番,祁里德領七千兵從布婁爾,傅爾丹領八千兵從布拉罕,同時進擊準噶爾。

夏四月戊申,上巡幸熱河。

五月辛巳,以旱求言。壬午,雨。

六月己亥,陜西饑,運河南積穀往賑。丙辰,保安、懷來地震,遣官賑之。

秋七月丙寅朔,日有食之。癸酉,富寧安擊賊於阿克塔斯、伊爾布爾和韶,敗之,擒其臺吉垂木拍爾。阿喇衲師至齊克塔木,遇賊,擊破之,盡虜其眾。進擊皮禪城,降之。師至吐魯番,番酋阿克蘇爾坦率眾迎降。丙戌,傅爾丹擊賊於格爾厄爾格,斬獲六百,陣擒寨桑貝肯,焚其積聚而還,貝肯送京。祁里德敗賊於鏗額爾河,降其寨桑色布騰等二千餘人。

八月戊戌,上行圍。庚子,琉球請令其陪臣子弟入國子監讀書,許之。癸丑,平逆將軍延信連敗賊眾於卜克河。丁巳,又敗賊眾於綽馬喇,賊將策零敦多布遁。定西將軍噶爾弼率副將岳鍾琪自拉里進兵。戊午,克西藏,執附賊喇嘛百餘,斬其渠五人,撫諭唐古特、土伯特,西藏平。以高其倬為廣西巡撫。

九月壬申,平逆將軍延信以兵送達賴喇嘛入西藏坐★床。富寧安兵入烏魯木齊,哈西哈回人迎降,軍回至烏蘭烏蘇。戊寅,雲貴總督蔣陳錫、巡撫甘國璧以饋饟後期褫職,仍令運米入藏。

冬十月癸卯,上還京。詔再以河南積穀運往陜西放賑。明年,河南漕糧照數補還倉穀,其餘漕糧留貯河南。甲辰,朝鮮國王李焞薨。詔曰:「李焞襲封五十年,奉籓恭謹,撫民慈愛。茲聞溘逝,惻悼實深,即令王子李昀襲封。所進貢物悉數帶回,仍查恤典具奏。」詔陜西、甘肅兩省康熙六十年地丁銀一百八十八萬兩零,通行蠲免。沿邊歉收,米價昂貴,兵力拮據,並豫發本年兵饟。賚進藏官兵。甲寅,戶部尚書趙申喬卒,予祭葬,謚恭毅。丁巳,詔撫遠大將軍胤禵會議明年師期。戊午,以陜西、甘肅歉收,命銀糧兼賑,以麥收為止。

十一月辛未,遣官致祭朝鮮國王李焞,特謚僖順,冊封世子李昀為朝鮮國王。戊寅,以田從典為戶部尚書,硃軾為左都御史,以楊名時為雲南巡撫。辛巳,詔:「大兵入藏,其地俱入版圖,山川名號番、漢異同,應即考訂明覈,傳信後世。」上因與大學士講論河源、江源,及於禹貢三危。庚寅,以隆科多為理籓院尚書,仍兼步軍統領。

十二月甲辰,廷臣再請行六十年慶賀禮。不允。壬子,授先賢子夏後裔五經博士。甲寅,以誠親王胤祉子弘晟、恆親王胤祺子弘升為世子。辛酉,祫祭太廟。

是歲,免直隸、江蘇、陜西、浙江、四川等省五十六州縣衛災賦有差。朝鮮、琉球入貢。

六十年辛丑春正月乙亥,上以御極六十年,遣皇四子胤禛、皇十二子胤示匋、世子弘晟告祭永陵、福陵、昭陵。

二月乙未,上謁孝莊山陵、孝陵、孝東陵,行告祭禮。遣官告祭郊廟社稷。乙卯,上還京。山東鹽徒王美公等作亂,捕斬之。己未,命公策旺諾爾布駐防西藏。論取藏功,封第巴阿爾布巴、康濟鼐為貝子,第巴隆布奈為輔國公。

三月乙丑,群臣請上萬壽節尊號,上不許,曰:「加上尊號,乃相沿陋習,不過將字面上下轉換,以欺不學之君耳。本朝家法,惟以愛民為事,不以景星、慶雲、芝草、甘露為瑞,亦無封禪改元之舉。現今西陲用兵,兵久暴露,民苦轉輸。朕方修省經營之不暇,何賀之有?」庚午,賜舉人王蘭生、留保進士,一體殿試。甲戌,先是,大學士王掞密疏復儲。至是御史陶彞、任坪、範長發、鄒圖雲、陳嘉猷、王允晉、李允符、範允、高玢、高怡、趙成簁、孫紹曾疏請建儲,上不悅,並掞切責之,命其子詹事王奕清及陶彞等十二人為額外章京,軍前效力。

夏四月甲午,以李麟為固原提督。乙未,賜鄧鍾岳等一百六十三人進士及第出身有差。丙申,詔釐定歷代帝王廟崇祀祀典。丁酉,命張鵬翮、陳鵬年赴山東閱河。以賴都為禮部尚書,托賴為刑部尚書。丙午,上幸熱河。戊午,命定西將軍噶爾弼駐藏。

五月壬戌,命撫遠大將軍胤禵移師甘州。丙寅,臺灣奸民硃一貴作亂,戕總兵官歐陽凱。癸酉,以署參將管永寧協副將岳鍾琪為四川提督。乙亥,改思明土州歸廣西太平府。戊寅,詔停本年進兵。以常授為理籓院額外侍郎,辦事西寧。乙酉,以年羹堯為四川陜西總督,賜弓矢。發帑金五十萬賑山西、陜西,命硃軾、盧詢董其事。

六月壬辰,改高其位為江南提督,魏經國為湖廣提督。丙申,詔曰:「平逆將軍延信,朕之侄也。統兵歷從古未到之煙瘴絕域,殲滅巨虜,平定藏地,允稱不辱宗支,可封為輔國公。」乙卯,吐魯番回人拖克拖麻穆克等來歸,命散秩大臣阿喇衲率兵護之。福建水師提督施世驃平臺灣,擒硃一貴解京。詔獎淡水營守備陳策固守功,超擢臺灣總兵。

閏六月庚申朔,日有食之。丙寅,令刑部弛輕系,戊辰,以噶爾弼為蒙古都統。

秋七月己酉,上行圍。

八月甲戌,命副都統莊圖率兵二千進駐吐魯番,益阿喇衲軍。丙戌,河決武陟入沁水。

九月辛卯,命副都統穆克登將兵二千赴吐魯番。甲午,噶爾弼以病罷,命公策旺諾爾布署定西將軍,駐藏,以阿寶、武格參軍事。丙申,策旺阿拉布坦犯吐魯番,阿喇衲擊走之。丙午,賑河南、山東、直隸水災。乙卯,上還京。丙辰,命副都御史牛鈕、侍講齊蘇勒、員外郎馬泰築黃河決口,引沁水入運河。丁巳,以阿喇衲為協理將軍。上制平定西藏碑文。

冬十月壬戌,置巡察臺灣御史。詔:「本年秋審俱已詳覽,其直省具題緩決之案,九卿已加核定,朕不忍覆閱,恐審求之或致改重也。」丙寅,召撫遠大將軍胤禵來京。辛未,詔:「大學士熊賜履服官清正,學問博通,朕久而弗忘,常令周恤其家。今其二子來京,觀其氣質,尚可讀書,宜加造就,可傳諭九卿知之。」以鍾世臣為浙江提督,姚堂為福水師提督,馮毅署廣東提督。

十一月辛卯,以陳鵬年署河道總督。戊戌,以馬武、伊爾哈岱為蒙古都統。己酉,上幸南苑。詔將軍額倫特、侍衛色楞、副都統查禮渾、提督康泰等,殺敵殉國,俱賜恤。

十二月壬申,四川提督岳鍾琪征郭羅克番人,平之。丁丑,上還駐申昜春園。遣鄂海、永泰往視吐魯番屯田。

是歲,免江南、河南、陜西、甘肅、福建、浙江、湖廣等省一百二十三州縣災賦有差,朝鮮、琉球、安南入貢。丁戶二千九百一十四萬八千三百五十九,又永不加賦後滋生人丁四十六萬七千八百五十,徵銀二千八百七十九萬零。鹽課銀三百七十七萬二千三百六十三兩零。鑄錢四萬三千七百三十二萬五千八百有奇。

六十一年壬寅春正月戊子,召八旗文武大臣年六十五以上者六百八十人,已退者咸與賜宴,宗室授爵勸飲。越三日,宴漢官年六十五以上三百四十人亦如之。上賦詩,諸臣屬和,題曰千叟宴詩。戊申,上巡幸畿甸。

二月庚午,以高其倬署云南貴州總督。丙子,上還駐申昜春園。

三月丙戌,以阿魯為荊州將軍。

夏四月甲子,遣使封朝鮮國王李昀弟昑為世弟。丁卯,上巡幸熱河。己巳,撫遠大將軍胤禵復蒞軍。癸未,福州駐防兵嘩,將軍黃秉鉞不能約束,褫職,斬為首者。

五月戊戌,施世綸卒,以張大有署漕運總督。

六月,以奉天連歲豐稔,弛海禁。暹羅米賤,聽入內地,免其稅。辛未,命直隸截漕二十萬石備賑。丙子,趙弘燮卒,以其兄子郎中趙之垣加僉都御史銜,署直隸巡撫。

秋七月丁酉,征西將軍祁里德上言烏蘭古木屯田事宜。請益兵防守。命都統圖拉率兵赴之。壬寅,命色爾圖赴西藏統四川防兵。戊申,以蔡珽為四川巡撫。予故直隸總督趙弘燮祭葬,謚肅敏。

八月丙寅,停今年決囚。故提督藍理妻子先以有罪入旗,至是,上念平臺灣功,貰還原籍,交★免追。己卯,上駐蹕汗特木爾達巴漢昂阿。賜來朝外籓銀幣鞍馬,隨圍軍士銀幣。

九月甲申,上駐熱河。乙酉,諭大學士曰:「有人謂朕塞外行圍,勞苦軍士。不知承平日久,豈可遂忘武備?軍旅數興,師武臣力,克底有功,此皆勤於訓練之所致也。」甲午,年羹堯、噶什圖請量加火耗,以補有司虧帑。上曰:「火耗只可議減,豈可加增?此次虧空,多由用兵。官兵過境,或有餽助。其始挪用公款,久之遂成虧空,昔年曾有寬免之旨。現在軍需正急,即將戶部庫帑撥送西安備用。」戊戌,上回鑾。丁未,次密雲,閱河堤。庚戌,上還京。

冬十月辛酉,命雍親王胤禛、弘升、延信、孫渣齊、隆科多、查弼納、吳爾占察視倉廒。壬戌,以覺羅德爾金為蒙古都統,安鮐為杭州將軍。辛未,以查弼納為江南江西總督。癸酉,上幸南苑行圍。以李樹德為福州將軍,黃國材為福建巡撫。

十一月戊子,上不豫,還駐申昜春園。以貝子胤祹、輔國公吳爾占為滿洲都統。庚寅,命皇四子胤禛恭代祀天。甲午,上大漸,日加戌,上崩,年六十九。即夕移入大內發喪。雍正元年二月,恭上尊謚。九月丁丑,葬景陵。

論曰:聖祖仁孝性成,智勇天錫。早承大業,勤政愛民。經文緯武,寰宇一統,雖曰守成,實同開創焉。聖學高深,崇儒重道。幾暇格物,豁貫天人,尤為古今所未覯。而久道化成,風移俗易,天下和樂,克致太平。其雍熙景象,使後世想望流連,至於今不能已。傳曰:「為人君,止於仁。」又曰:「道盛德至善,民之不能忘。」於戲,何其盛歟!


\end{pinyinscope}