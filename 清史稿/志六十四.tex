\article{志六十四}

\begin{pinyinscope}
禮八(嘉禮二)

大婚儀皇子婚儀王公婚禮附公主下嫁儀郡主以下於歸禮附

品官士庶婚禮視學儀臨雍附經筵儀日講附策士儀

頒詔儀迎接詔書附進書儀進表箋儀巡狩儀鄉飲酒禮

大婚儀清初太祖戊子年,葉赫國貝勒納林布祿送妹來歸,帝率貝勒等迎之,大宴,禮成,時猶未定儀注也。太宗即位後,行冊立禮。至順治八年,世祖大婚,始定納后儀。先期諏吉行納採禮,前一日,遣官祭告郊、社、太廟。屆日質明,設節案太和殿,禮物具丹陛上,陳文馬其下。正、副使俟丹墀東。鳴贊官口贊,使臣三跪九拜訖,升東階,立陛上。宣制官傳制,使臣跪。制曰:「茲納某氏某女為後,命卿等持節行納採禮。」大學士入,奉節出,授正使,正使受,偕副使興,前行降中階左。執事官納儀物採亭中。儀仗前導,衛士牽馬從,出太和中門,詣後邸。後父朝服跪迎門外道右。既入,使臣陳節中案,執事陳儀物左右案,陳馬於庭。使臣傳制納採,以次奉儀物授後父,後父跪受,興,率子弟望闕行禮。使臣出,跪送如初。前期一日,行納徵禮。所司具大徵儀物,遣使傳制,如納採儀。大婚前一日,復遣官祭告,屆期鹵簿、樂懸具。帝御太和殿閱冊、寶,制辭曰:「皇帝欽奉皇太后懿旨,納某氏為皇后。茲當吉月令辰,備物典冊,命卿等以禮奉迎。」遣使如冊后儀,使臣隨冊、寶亭出自協和門,駕還宮。

時皇后儀仗陳邸第,封使至,後父率親屬朝服迎門外,後禮服迎庭中,後母率諸婦咸朝服跪。使臣奉冊、寶入陳案上,後就案南北面跪,內院官西鄉立,讀冊、寶文,次第授左女官,女官跪接獻皇后,後祗受,轉授右女官,亦跪接,陳案上盝內。後興,六肅三跪三叩,禮畢,升輦。女官奉盝置採亭,鼓樂導前,次儀仗,次鳳輦。後父母跪送如跪迎儀。輦至協和門,儀駕止。女官奉盝前行置中宮,輦入自中門,至太和殿階下降輦入宮。

帝御中和殿,率諸王詣皇太后前行禮。畢,諸王退。帝御太和殿,賜后父及親屬宴,王公百官咸與。皇太后御位育宮,即保和殿,賜后母及親屬宴,公主、福晉、命婦咸與。越三日,帝復御太和殿,王公百官上表慶賀,頒詔如制。賜后父母兄弟服物有差。十一年大婚,越三日,後謁皇太后禮畢,始宴。康熙四年大婚,就後邸設納採宴,公主、輔臣命婦各三人,內大臣、侍衛及公以下、群臣二品以上咸與。

大徵亦如之。賜后祖父母、父母衣服,謝恩如儀。至日,使臣奉冊、寶至,後祗受畢,欽天監報時,後升輦。命婦四人導前,七人隨後,皆騎。內大臣、侍衛從,至太和殿階下退。後降輦,內監奉冊、寶導至中和殿,命婦退。執事命婦迎侍入宮,奉冊、寶內監授守寶內監,退。帝詣太皇太后、皇太后前行禮,御殿、賜宴如初。皇太后率輔臣命婦入宮,賜后母及親屬宴,公主、福晉不與。時加酉,宮中設宴,行合巹禮。翼日,後詣兩宮朝見,三日受賀,頒詔如常儀。

同治十一年,納採、大徵、發冊、奉迎,悉準成式。惟屆時後升輦,使臣乘馬先,內監扶,左右內大臣等騎從。至午門外,九鳳曲蓋前導,行及乾清門,龍亭止,使臣等退,禮部官奉冊、寶陳交泰殿左右案,退。輦入乾清宮,執事者俱退,侍衛合隔扇。福晉、命婦侍輦入宮,宮中開合巹宴,禮成。光緒十五年大婚,越六日,後始朝見皇太后,又越二日,帝受賀,餘儀同。

皇子婚儀先指婚,簡大臣命婦偕老者襄事。福晉父蟒服詣乾清門,北面跪,大臣西面傳旨:「今以某氏女作配皇子某為福晉。」福晉父三跪九拜,退。擇吉,簡內大臣、侍衛隨皇子詣福晉家行文定禮。福晉父採服迎門外,皇子升堂拜,福晉父答拜,三拜,興。見福晉母亦如之。辭出,福晉父送大門外。行納採禮,所司具儀幣,並備賜福晉父母服飾、鞍馬。以內府大臣、宮殿監督領侍充使。及門,福晉父迎入中堂,謝恩畢,與宴,大臣陪福晉父宴中堂,命婦、女官陪女眷宴內室,畢,使者還朝復命。婚前一日,福晉家齎妝具陳皇子宮,至日,皇子詣帝後前行禮,若為妃嬪出,則並詣焉。

吉時屆,鑾儀衛備採輿,內府大臣率屬二十、護軍四十詣福晉第奉迎。採輿陳堂中,女官告「升輿」,福晉升,父母家人咸送。內校舁行。女官從,出大門乘馬。至禁城門外,眾步行隨輿入,至皇子宮門降,女官導入宮。屆合巹時,皇子西鄉,福晉東鄉,行兩拜禮。各就坐,女官酌酒合和以進,皆飲,酒饌三行,起,仍行兩拜禮。於時宮所張幕、結採,設宴,福晉父母、親族暨大臣、命婦咸與,禮成。翌日皇子、福晉夙興,朝見帝、後,女官引皇子居左稍前,三跪九拜,福晉居右稍後,六肅三跪三拜。見所出妃嬪,皇子二跪六拜,福晉四肅二跪二拜。越九日,歸寧。巳宴,偕還,不逾午。

王公婚禮,崇德間定制,凡親王聘朝臣女為婚,納採日,府屬官充使,是日設宴,牲酒盛陳。婚日宴亦如之。給女父母服物鞍馬符例。若外籓親、郡王,貝勒,臺吉女,儀物視爵次為差。婚日宴,牲多少異宜。世子,郡王,貝勒,貝子,鎮、輔國公聘娶,儀物暨宴日牲酒,其數遞降,皆有差等。順治間,更婚制,限貝勒以下罷用珠緞。賜婚,王公詣中和殿或位育宮謝恩,其子未受封者,婚禮視其父,已受封則從其爵。康熙初,始令王公納採易布為緞,餘如故。

公主下嫁儀指婚日,額駙蟒服詣乾清門東階下,北面跪,襄事大臣西面立。宣制:「以某公主擇配某額駙。」祗受命,謝恩退。初定,諏日詣午門,進一九禮,即納採也。駝馬、筵席、羊酒如數。得旨分納所司。次日燕饗,額駙率族中人朝服謁皇太后宮,禮訖,集保和殿。帝升座,額駙等三跪九拜。禦筵既陳,進爵大臣跪進酒,帝受飲,還賜大臣酒,跪飲之。時額駙等行禮惟一拜。徹宴謝恩,一跪三拜。出至內右門外,三跪九拜,退。凡帝前謝恩皆贊,後宮前不贊。是日額駙眷屬詣皇太后、皇后宮筵宴如儀。釐降前一日,額駙詣宮門謝恩,內府官率鑾儀校送妝奩詣額駙第,內管領命婦偕女侍鋪陳。

至日,額駙家備九九禮物,如鞍馬、甲胄,詣午門恭納,燕饗如初定禮。吉時屆,公主吉服詣皇太后、帝、後暨所生妃、嬪前行禮。命婦翊升輿,下簾,內校舁出宮,儀仗具列,燈炬前引。福晉、夫人、命婦乘輿陪從,詣額駙第行合巹禮。其日設宴九十席,如下嫁外籓,但用牲酒。成婚後九日,歸宮謝恩。公主入宮行禮,額駙詣慈寧門外、乾清門外、內右門外行禮。

天命八年,太祖御八角殿,訓公主以婦道,毋陵侮其夫,恣意驕縱,違者罪之。時議謂王化所由始。厥後定制,額駙及其父母見公主俱屈膝叩安,有賚賜必叩首,尋遠古轍已。逮道光二十一年,宣宗以為非禮所宜,稍更儀注,額駙見公主植立申敬,公主立答之,舅、姑見公主正立致敬,公主亦如之。如餽物,俱植立,免屈膝,以重倫紀,著為令。

又定下嫁時停進九九禮,並罷筵宴,自後罷宴以為常。明年,改初定進羊九,繼此踵行。同治時,定公主歸寧,免額駙內右門行禮,餘如前儀。

郡主於歸禮,崇德間,定親王嫁女聘儀,鞍馬、甲胄十有五。如嫁外籓,親王以下納採用駝、馬、羊,準七九數。媵婢八,男、婦五戶。順治時,朝臣聘儀,鞍馬、甲胄各七。乾隆時,定郡王媵婢六,男、婦四戶。嫁朝臣聘用鞍馬七,外籓納採視崇德時為減。郡主以下,縣主、郡君、縣君、鄉君於歸禮,以次遞殺。康熙八年,定郡主、縣主歸寧,禁母家給滿洲人口,限用蒙、漢人八名,郡君至鄉君,蒙、漢人六名,將軍至宗室女,四名。

乾隆三十五年,罷朝臣進納採禮,外籓如故。不設宴。

品官士庶婚禮凡品官論婚,先使媒妁通書,乃諏吉納採。自公、侯、伯訖九品官,儀物以官品為降殺。主婚者吉服,命子弟為使,從者齎儀物至女氏第,主婚者吉服迎。從者陳儀物於庭,奉書致命,主婚者受書,告廟醴賓,賓退,送之門,使者還復命。是日設宴具牲酒,公、侯以下,數各有差。婚前一日,女氏使人奉箕帚往婿家,陳衾帷、茵褥、器用具。

屆日,婿家豫設合巹宴。婿吉服俟,備儀從。婿承父命親迎,以採輿如女氏第。女氏主婚者告廟,辭曰:「某第幾女某,將以今日歸某氏。」乃笄而命之。還醮女內室,父東母西。女盛服出,北面再拜,侍者斟酒醴女,父訓以宜家之道,母施衿結帨,申父命,女識之不唯。婿既至,入門再拜。奠雁,出。姆為女加景蓋首,出。婿揖降。女從姆導升輿,儀衛前導,送者隨輿後。婿先還。輿至門,婿導升西階,入室逾閾,媵布婿席東旁,御布婦席西旁,交拜訖,對筵坐。饌入,卒食,媵御取■D9實酒,分酳婿、婦,三酌用巹,卒酳,婿出。媵御施衾枕,婿入,燭出。是日具宴與納採同。

品官子未任職,禮視其父,受職者各從其品。士婚禮視九品官。庶民納採,首飾數以四為限,輿不飾採,餘與士同。婚三日,主人、主婦率新婦廟見,無廟,見祖、禰於寢,如常告儀。

雍正初,定制,漢人納採成婚,四品以上,綢緞、首飾限八數,食物限十品。五品以下減二,八品以下又減二,軍、民紬絹、果盒亦以四為限。品官婚嫁日,用本官執事,燈六、鼓樂十二人,不及品者,燈四、鼓樂八人。禁糜費,凡官民皆不得用財禮云。

視學儀順治建元,帝幸太學釋奠。先期衍聖公、五經博士至,聖裔五人,元聖及配、哲諸裔各二人,乘傳赴京。各氏子孫現列朝官者,各官學師生暨進士、舉、貢,咸與觀禮。內閣擬經、書,祭酒、司業撰講章進御。屆日,大成門東張大次,彞倫堂設黃幄御座,幄前置御案,左右講案二,祭酒等奉講章及進講副本,書左經右,陳於案。帝禮服乘輿詣學,祭酒、司業率官屬諸生跪迎成賢街右。駕入幄,詣大成殿釋奠。禮畢,出易袞服,幸彞倫堂,御講幄。升座,王公立階上,百官立階下,衍聖公率博士、各氏裔,祭酒等率官生就拜位,行三跪九叩禮。畢,自王公訖九卿以次賜坐,尋詣堂內跪,一叩。鴻臚官贊「進講」,祭酒、司業入,北鄉立,所司舉經案進御前。賜講官座,祭酒等一叩,坐。依次宣講。翰詹四品以下官,監官、師儒、博士、聖賢後裔、肄業諸生圜聽。講畢,退,聽講者咸退。復位序立,跪聆傳制。辭曰:「聖人之道,如日中天,講貫服膺,用資治理,爾師生勉之。」祭酒等三跪九叩,退。賜茶,群臣受飲。一叩,禮成。駕出,咸跪送。翼日,監官、博士暨諸生表謝,帝御太和殿,禮賜如常儀,並賜衍聖公、各官宴禮部。越三日,頒敕太學,詔諸生策勵,賚衍聖公冠服,監官、博士等衣一襲,助教、諸生白金有差。

康熙八年,聖祖釋奠太學,講經、悉準成式。

雍正二年,諭:「視學大典,稱幸非宜,嗣後更『幸』為『詣』。」

乾隆二年,命閔、冉、言、卜、顓孫、端木六氏博士陪祀觀禮,準五氏例行。明年,帝親視學,聖、賢各裔暨東野氏來觀禮者三十二人,送監求學,即召衍聖公等面諭之。謂:「既為聖賢後,當心聖賢心,非徒讀其書而已。必躬行實踐,事求無愧,方為不負所學。其務勤思勉勵,克紹心傳。」

三年三月上丁,帝親詣太學行釋菜禮。越六日,臨雍講學,王公大臣,聖賢後裔,以至太學諸生,環集橋門璧水間者以萬數。臨雍命下,既諏吉,所司設御幄大成門外,其闢雍殿階陳中和韶樂,太學門內陳丹陛大樂、清樂。殿內經書案、講案備具如前。帝釋奠畢、御彞倫堂,易袞服,臨闢雍。太學鳴鐘鼓,升座,樂奏,止有節。贊「齊班」,講官、侍班、糾儀各官就拜位,贊「跪,叩,興」,行二跪六叩禮,興。若衍聖公入覲,先進講,大學士以至諸生分班立,行禮訖,滿、滿講官入,一叩,就坐,講四書,帝闡發書義,宣示臣工,圜橋各官生跪聆畢,興。祭酒講經,帝闡經義如初禮。餘同視學儀。

先是御史曹學閔上言:「宜考古制,建闢雍於國子監。」格部議。至四十九年,新建國學成,明年將臨雍,命大臣規濬圜水,禮樂備舉。特旨獎學閔,並令朝鮮使臣隨班觀禮。禮成,賞賚有差。翼日加賚聖、賢各氏裔及諸生綢帛。

道光三年臨雍,命廕生豫聽宣講,諭監官曰:「化民成俗,基於學校,興賢育德,責在師儒。士先器識,漸摩濡染,厥有由來。爾監臣式茲多士,尚其端教術,正典型,毋即於華,毋鄰於固。入孝出弟,擇友親師。庶幾成風,紹休聖緒。」

令甲,車駕幸魯,展禮先師,講學闕里,豫選聖、賢裔二人直講,翰林官撰講章。前一日,張大次奎文閣,設御座詩禮堂。前置案,講案列西簷下。屆日,陳講章及副本於案,帝出行宮,衍聖公採服率五經博士暨各氏跪迎廟門右。帝入,詣大成殿祭孔子,如上丁儀。駕出,御詩禮堂,升座。衍聖公以下官隨至,序立庭中,行三跪九叩禮。訖,進講,直講者一跪三叩,興。講經書訖,俱退。駕謁孔林。翼日,賜衍聖公等帛、金、書籍有差。簡各氏弟子有文行者貢太學,凡登仕版,並進一階。

經筵儀初沿明制,閣臣例不兼經筵。順治九年,春、秋仲月一舉,始令大學士知經筵事。尚書、左都御史、通政使、大理卿、學士侍班,翰林二人進講。豫設御案、講官案,列講章及進講副本,左書右經,屆時,帝常服御文華殿,記注官立柱西,東面。講官等二跪六叩,興,序立左右,侍班官分立其後。糾儀官立東西隅。鳴贊官贊「進講」,直講官詣案前跪,三叩,興,分就左右案。先後講四書與經,復位。帝宣示清、漢文禦論,各官跪聆畢,大學士奏辭感悅。興,降階行二跪六叩禮。畢,帝臨文淵閣,賜坐、賜茶。禮成,還宮。賜宴本仁殿。宴畢,謝恩。

康熙十年舉經筵,命大學士熊賜履為講官,知經筵事。頃之,聖祖以春、秋兩講為期闊疏,遂諭日進講弘德殿。二十四年,定制,以大學士、左都御史、侍郎、詹事充經筵講官。二月,文華殿成,舉行典禮。世宗踐阼,居亮陰,未舉。

雍正三年八月吉日,詔言:「帝王御宇,咸資典學。朕承庭訓,時習簡編。味道研經,實敷政寧人之本。茲當釋服,亟宜舉行。」於是進講如儀。

乾隆五年,諭曰:「經筵之設,藉獻箴規。近進講章,辭多頌美,殊失咨儆古意。人君敷政,正賴以古證今,獻可替否。其務剴切敷陳,期裨政學,庶有當稽古典學實義。」

七年,經筵日雨,禮臣依例請改期。諭曰:「魏文侯出獵遇雨,尚不失信虞人。矧茲大典,復經祭告,詎宜改期?執事諸臣,可衣雨服列班,暫罷階下行禮、殿內賜茶諸儀。嗣後遇雨仿此。」

翰林院專司日講,冬、夏至前一日乃輟。十四年,以進呈經史,漸等具文,諭令停止。

五十一年,御經筵,賜宴禮臣隨侍者,分東西班,特命歌抑戒詩。

嘉慶中,張鵬展疏請翰林科道日進經義、奏議。詔責其迂。

文宗登極,曾國籓請復日講舊典,格部議。次年咸豐紀元,正月,遂奉特旨令翰詹諸臣番直,並躬制題目,俾撰講義,分日呈覽。迄光、宣之際,猶依此例云。

策士儀天聰間,始開科取士。順治初,會試中式舉人集天安門考試。十五年,改試太和殿丹墀,定臨軒策士制。先期一日,丹陛上正中,太和殿內東偏,分設黃案,東西閣簷下備試桌。屆日質明,內閣官朝服捧策題置殿內案上,帝御太和殿,王公百官侍立,鴻臚寺官引貢士詣丹陛下立。大學士取題授禮部官,跪受,置丹陛案上,三叩。舉案降左階,陳御道正中。讀卷官執事官各三跪九叩,諸貢士亦如之。畢,駕還宮。徙試桌丹墀左右,北鄉。禮部官散題,貢士跪受,三叩,就桌。對策訖,受卷、彌封諸官俟左廡簷下,收封盛入卷箱,收掌官送讀卷官校閱,不御殿,王以下官不會集,不陳鹵簿。閱卷三日畢,翼辰,前列十卷簽擬名次,緘封呈御覽。帝御養心殿西暖閣,閣畢,召讀卷官入,親定甲乙授之。出拆彌封,依次繕寫綠頭簽,引十人進乾清門,祗俟西階下。帝御宮,讀卷官捧簽入,跪呈。引班官引十人跪丹陛中,依次奏名籍,興,退。帝親定一甲三人,二甲七人,授簽讀卷官,跪受,興,退,率十人侍立西階下。駕還便殿。十人先出。讀卷官捧卷詣紅本房,填寫名次畢,交內閣題金榜。

傳臚日,設鹵簿,陳樂懸,王公百官列侍。貢士皆公服,冠三枝九葉頂冠,立班末。帝御太和殿,讀卷等官行禮如初,奉榜授受如奉策題儀。鴻臚寺官引貢士就位,跪聽傳。制曰:「某年月日,策試天下貢士,第一甲賜進士及第,第二甲賜進士出身,第三甲賜同進士出身。」贊「一甲一名某」,令出班前跪。贊二三名亦然。贊「二甲一名某等若干名,三甲某等若干名」,不出班,同行三跪九叩禮。退立。禮部官舉榜出中路,一甲進士從,諸進士出左右掖門,置榜龍亭,復行三叩禮。校尉舁亭,鼓樂前導,至東長安門外張之,三日後繳內閣。於是順天府備傘蓋、儀從送狀元歸第。越五日,狀元偕諸進士上表謝恩如常儀。

乾隆五十四年,殿試改保和殿舉行。自後為恆例。

頒詔儀清初詔書用滿、蒙、漢三體文。順治間,定制用滿、漢二體。頒詔日,太和殿前具鹵簿,丹墀內植黃蓋、雲盤、殿東設詔案,丹陛中設黃案。午門外備龍亭、香亭。天安門樓雉口中豫置朵雲金鳳,其東築宣詔臺。王公百官朝服集午門,內閣學士奉詔書至乾清門用寶訖,鋪黃案。帝御殿,王公以下行禮畢,大學士奉詔書詣殿簷下授禮部尚書,尚書跪受訖,陳丹陛案上。行禮畢,置詔書雲盤內,覆黃蓋。禮部官奉盤自中路出太和門,百官從至午門外,置龍亭。至天安門外橋南,奉詔書置高臺黃案上。各官按序北鄉立,宣讀官臺上西鄉立,眾跪聽宣。先宣滿文,次漢文,眾行三跪九叩禮。奉詔官取朵雲承詔書,系以採繩,自金鳳口中銜下。禮部官接受,仍置龍亭。出大清門,赴禮部,望闕列香案,尚書率屬行禮。詔書謄黃,刊頒各省。駕不御殿,百官祗俟天安門外橋南,餘儀同。

乾隆間,定制,凡詔書到日,有司備龍亭、旗仗郊迎。朝使降騎,奉詔書置龍亭,南鄉,守土官北鄉行禮。鼓樂前導,朝使騎以從。及公廨,眾官先入序立,龍亭至庭中,朝使東立。俟行禮訖,奉詔書授展讀官。跪受,眾官皆跪。宣讀畢,授詔朝使,復置龍亭,跪叩如初禮。退。長吏謄黃,分頒各屬。詔書所過,凡屬五里內府、州、縣、衛各官,咸出郭門迎送。

進書儀定制,纂修實錄、聖訓,擇吉進呈。帝御殿受書,王公百官表賀。玉牒、本紀次之。康熙十一年,世祖實錄成,前期一日,太和殿陛東設表案,階下列實錄案。至日具鹵簿,陳樂懸,監修官奉表陳表亭,纂修官奉實錄陳採亭,王公百官齊集行禮如儀。校尉分舁香亭、採亭出中道,表亭由左,監修各官從至太和殿丹墀,監修等奉實錄與表分陳案上。帝御殿,鴻臚官奏進實錄,樂作。禮部官舉實錄案自中道升,至殿門外,帝興座,樂止。舉案入,乃坐。設案保和殿正中,監修等立階下齊班,贊「跪」,則皆跪。贊「進表」,宣表官跪宣。畢,樂作,眾官三跪九叩,退立,樂止。眾復跪,宣表官代奏致詞云:「某親王臣某等暨文武群臣奏言,惟世祖皇帝神功聖德,纂述成書,光華萬世,群臣歡忭,禮當慶賀。」鴻臚卿宣制答云:「世祖皇帝功德配天,實錄纂成,朕心歡慶,與卿等同之。」宣訖,行禮如初。賜茶,俱一叩。駕還。監修等奉實錄至乾清門,交送大內,退。

雍正中,聖祖實錄與聖訓同進,後以為常。乾隆間,定實錄、聖訓歸皇史宬,遣監修等奉藏金匱,副本存內閣。嘉慶十二年,更定舉案、奉書,選貝子以下宗室官將事。自仁宗以來,帝仍詣皇史宬拈香,如往制。進玉牒,不上表,不傳制。監修等隨採亭入中和殿,置案上,展正中四篋。帝立閱,俟進全書覽畢,送皇史宬。十年一纂,或不御殿,則於宮中覽之。凡實錄、聖訓、玉牒,並送盛京尊藏。自乾隆年始進本紀,第諏吉藏皇史宬,方略則進二部,一藏史宬,一交禮部刊發。時憲書成,欽天監官歲以十月朔日進,並頒賜王公百官。午門行頒朔禮,頒到直省,督、撫受朔如常儀。

進表箋儀凡萬壽節及元日、長至,在京王公百官各進表文,在外將軍、都統、副都統、督、撫、提、鎮各進賀表、箋,匯齊驛遞送部。屆日設表案太和殿左楹。表文列採亭,舁至午門外,奉陳於案。帝御殿,宣表行禮訖,並表、箋送內閣收儲。皇太后聖壽、皇后千秋,王公暨內外文武表、箋,俱陳午門外。禮訖,亦送內閣。表文初用三體字式,後專用漢文,惟滿洲駐防用清文。先期內閣撰擬定式頒發,臨期恭進。慶賀三大節表式,在京稱「某親王臣某等」,「諸王貝勒文武官等」;在外稱「某官臣某等,誠歡誠忭,稽首頓首上言」,末云:「臣等無任瞻天仰聖,歡忭之至,謹奉表稱賀以聞。」進太皇太后、皇太后同。皇太子箋式,首具官同,末云:「臣等無任歡忭踴躍之至,謹奉箋稱賀以聞。」

初,元旦、冬至,直省文武五品以上各進賀表、箋,萬壽節祗進皇帝表文,並由長官匯進。督、撫不進表、箋,凡遇大典,具本慶賀。尋令各省表、箋通省用總火牌一,專遣齎奉。乾隆時,以布政使、副將不能專達章疏,停附進表、箋例。又定皇后千秋節暨元旦、冬至,永停箋賀。皇太子慶典,京朝官集賀,不具箋,外吏亦免箋賀。

六十年,高宗內禪,稱太上皇帝,具賀表式云:「子臣某率王公大臣等謹奏,某歲元旦,太上皇帝親授大寶,子臣敬承慈命,謹率同王公文武大臣等奉表賀者。」末云:「子臣及諸臣等曷勝欽悅慶忭之至,謹奉表稱賀以聞。」賀皇帝登極表式,惟「頓首」下云:「恭逢皇上受寶禮成,登極紀元,謹奉表慶賀者。」餘如前式。

巡狩儀皇帝省方觀民,特舉時巡盛典。既諏吉,帝禦征衣,乘輿出宮,領侍衛內大臣等率禁旅翊衛扈蹕,諸臣征衣乘騎以次發。鑾輅所經,禁隨駕官弁擾吏民、踐禾稼。辦治糧芻,悉用公帑。將入境,督、撫、提、鎮率屬迎道右,紳耆量遠近𧾷忌迎。已駐蹕,疆吏等朝行營門外。翼日,望秩方岳,祭昔帝王、先師,咸親詣。至名賢祠墓則遣官。官吏入覲,詢風土人情。臨視河防,指授方略。召試獻詞賦者,拔尤授官。閱方鎮兵,藉辨材武。經過州縣,賜復蠲租,存問高年,差給恩賚。

順治八年,定制,駕出巡幸,別造香寶攜行,並鑄扈從各印,加「行在」字。部院章奏,內閣匯齊,三日一送行在,所過禁獻方物。又定乘輿所經,百里內守土官道右迎送。

康熙二十三年,聖祖南巡,定扈從王公大臣及部院員限駕發按次隨行。厥後南巡江浙者五,至泰安躬祀岱嶽,渡河祠河神,詣江寧謁明太祖陵,四幸五臺,一幸西安,大率禁奢尚實,亟勤民事。乾隆間,數奉太后南巡,若河南,若五臺,若山東、天津,翠華所蒞,百姓蒙庥。六巡江浙,揆示工要,大建堤堰,雖糜巨萬帑金不恤也。嘉慶時,幸五臺清涼山,行慶施澤,如康熙故事。

鄉飲酒禮順治初元,沿明舊制,令京府暨直省府、州、縣,歲以孟春望日、孟冬朔日,舉行學宮。前一日,執事敷坐講堂習禮,以致仕官為大賓,位西北;齒德兼優為僎賓,位東北;次為介,位西南;賓之次為三賓;位賓、主、介、僎後;府、州、縣官為主人,位東南。若順天府則府尹為主人,司正一人主揚觶,教官任之。贊引、讀律各二人,生員任之。屆日執事牽牲具饌,主人率屬詣學,乃速賓。賓至,迓門外,主東賓西,三揖讓乃升,相鄉再拜。賓即席,延僎、介入,如賓禮。就位,贊「揚觶」,司正升自西階,北鄉立,賓主皆起立。贊「揖」,司正揖,賓、介以下答揖。執事舉冪酌酒於觶授司正,司正揚觶而語曰:「恭惟朝廷,率由舊章,敦崇禮教,舉行鄉飲。非為飲食,凡我長幼,各相勸勉。為臣盡忠,為子盡孝,長幼有序,兄友弟恭,內睦宗族,外和鄉黨。毋或廢墜,以忝所生。」讀畢,贊「飲酒」,司正立飲。贊「揖」,則皆揖。司正復位,賓、介皆坐。贊「讀律令」,生員就案北面立,咸起立旅揖。讀曰:「律令,凡鄉飲酒,序長幼,論賢良,別奸頑。年高德劭者上列,純謹者肩隨。差以齒,悖法偭規者毋俾參席,否以違制論。敢有譁譟失儀,揚觶者糾之。」讀畢復位。贊「供饌」,有司設饌。贊「獻賓」,則授主以爵,主受之,置賓席。少退,再拜,賓答拜。於僎亦如之。皆坐,有司遍酌,贊「飲酒」,酒三五行,湯三品,畢,徹饌。僎、主、僚屬居東,賓、介居西,皆再拜。贊「送賓」,各三揖,出,退。

雍正初元,諭:「鄉飲酒禮所以敬老尊賢,厥制甚古,順天府行禮日,禮部長官監視以為常。」乾隆八年,以各省鄉飲制不畫一,或頻年闕略不行。舊儀載圖有大賓、介賓、一賓、二賓、三賓,與一僎、二僎、三僎,名號紛歧。按古儀禮:「賓若有遵者,諸公大夫。」注云:「今文讀為僎,此鄉之人仕至大夫,來助主人樂賓,主人所榮而遵法者。」戴記:「坐僎於西北,以輔主人。」其言主人親速賓及介,拜至獻酬辭讓之節甚繁,無一言及僎,所謂「不干主人正禮」者也。嗣後鄉飲賓、介,有司當料簡耆紳碩德者任之,或鄉居顯宦有來觀禮者,依古禮坐東北,無則寧闕,而不立僎名。五十年,命歲時舉鄉飲毋曠。每行禮,奏御制補笙詩六章。其制,獻賓,賓酢主人後,酒數行。工升,鼓瑟,歌鹿鳴。賓主以下酒三行,司饌供羹,笙磬作,奏南陔,閒歌魚麗,笙由庚。司爵以次酌酒。司饌供羹者三,乃合樂,歌關睢。工告「樂備」,徹饌。賓主咸起立再拜。賓、介出,主人送門外,如初迓儀。初,鄉飲諸費取給公家,自道光末葉,移充軍饟,始改歸地方指辦。餘準故事行。然行之亦僅矣。


\end{pinyinscope}