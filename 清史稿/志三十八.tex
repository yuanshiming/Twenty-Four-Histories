\article{志三十八}

\begin{pinyinscope}
地理十

△陜西

陜西省:禹貢雍、梁二州之域。明置陜西等處左、右承宣布政使司,並治西安。清初因之,置巡撫,治西安,並置總督,兼轄四川,尋改轄山陜。雍正九年,專轄陜甘,治西安。十三年,復轄四川。乾隆十三年,罷兼轄。十九年,兼甘肅巡撫事。二十四年,改陜甘總督。二十九年,移駐甘肅蘭州,遂為定制。康熙二年,析臨洮、鞏昌、平涼、慶陽四府置甘肅省,移右布政使治之。雍正三年,升西安府之商、同、華、耀、乾、邠六州,延安府之鄜、綏德、葭三州,為直隸州。九年,改榆林衛為府。十三年,同州升府,華仍降州隸焉,耀並降州還舊隸。乾隆元年,葭仍降州隸榆林。四十八年,升興安州為府。東界河南閿鄉;三百五里。西界甘肅清水;六百三十里。南界四川太平;一千三十里。北界邊墻。一千三百九十六里。廣九百三十五里,袤二千四百二十六里。宣統三年,編戶一百六十萬一千四百四十四,口八百五萬四千四百七。領府七,直隸州五,七,州五,縣七十三。

西安府:沖,繁,疲,難。巡撫,布政、提學、提法三司,鹽法、巡警、勸業三道,提督,將軍,副都統駐。明,領州六,縣三十一。雍正三年,升商、同、華、耀、乾、邠為直隸州,割縣十七他屬。十三年,耀及同官還舊屬,白水改隸同州。乾隆四十七年,置孝義。嘉慶五年,置寧陜。東北距京師二千六百五十里。廣三百五里,袤四百三十八里。北極高三十四度十六分。京師偏西七度三十二分。領二,州一,縣十五。長安沖,繁,疲,難。倚。府西偏。西北:龍首山。西南:清華、圭峰。南:終南山,橫亙長安、咸寧、鄠、盩厔四縣境。渭水自西逕縣北,東入咸寧。西南:潏水,歧為二:一,西南合鎬水為東交河,灃水東北流來會,又北經咸陽入渭;一,北流為皁河,折東經咸寧入渭。南有漕渠。又西南有通濟渠。鎮三:杜角、秦杜、三橋。主簿駐斗門。行宮,城內。光緒二十六年,德宗西幸,改舊撫署駐焉。咸寧沖,繁,疲,難。倚。府東偏。南:樂游、少陵原。渭水逕縣北而東,灞水、滻水自東北合注之。又東逕高陵入臨潼。潏水即漕水,一名皁水,出東南石鱉穀。其西鎬水自寧陜入,右合白石、小庫諸水,左合梗梓水,入長安。明秦籓城在府城東北隅、縣治北。順治六年,改建滿城,將軍、都統駐。縣丞舊自灞橋移尹家衛,改駐縣北草灘。灞橋、渭橋、鳴犢三鎮。驛一:京兆。咸陽沖,繁,難。府西北五十五里。北:畢原。東:鮮原。東南:高陽。西南:短陰。南:渭水自興平入,納泥渠水,東北會二灃水為雞心灘,東入長安。東北:涇水,東入涇陽。鎮四:高橋、窯店、北賀、馬莊。驛一:渭水。興平沖,繁。府西百里。西:馬嵬坡。北:黃山。渭水自武功入,左納清黑、夾逮諸水,合新開河,東入咸陽。縣丞駐張店。鎮二:馬村、桑家。驛一:白渠。臨潼沖,繁,難。府東北六十里。東南:驪山,有溫泉。北:普陀原。東:鴻門阪。西南:坑儒穀。渭水自咸寧入,逕縣北,石川河合清穀水南流注之。西有潼水,東有戲水、零水,均北流注之,東入渭南。縣丞駐關山鎮。鎮五:新豐、零口、交口、廣陽、櫟陽。驛一:新豐。高陵簡。府東北七十里。西:降鶴山。南:奉政原。西南:渭自咸寧緣界逕鹿苑原,左合涇水,又東緣臨潼界入之。西北:白渠自涇陽入,播為二,曰昌運,曰高望。西南有毗沙鎮。鄠繁,難。府西南七十里。東南:紫閣峰。南:圭峰。東南:終南山。北有渭水自興平入,入咸陽。東南:灃水自長安緣界入,會澇水。澇水出縣南,合渼波水,東北入咸陽,注渭。鎮四:秦渡、趙王、澇店、大王。藍田簡。府東南九十里。北:橫嶺。南:秦嶺、七盤、嶢山、蕢山。東:藍田山,有關。灞水出縣東倒回穀,即藍田谷,逕南境,納藍水、輞水,逕城南,又西北合土膠河、猗水、注水,入咸寧。滻水出南山土門谷,西北流,為焦戴河,合湯谷水,均入咸寧。鎮三:藍橋、焦戴、新街。涇陽沖,繁,難。府西北七十里。北:嵯瓘山。西北:甘泉、仲山。涇水自醴泉緣界入,逕城南,東南入高陵。北:冶谷水自淳化入,會清水,入三原。西北:龍洞渠,逕縣北,歧為三:曰北白渠,入三原;中白渠,入高陵,下白渠,流數裏伏。又有冶清渠。冶峪,縣丞駐。鎮六:永樂、臨涇、石橋、雲陽、孟店、王橋。三原沖,繁,難。府北九十里。北:浮山。西北:嵯瓘、堯山。濁谷水自耀入,曰樓底河,東流,散入各渠。趙氏河即澗穀水,自富平錯入,仍入富平。清穀水自耀入,西北入涇陽,復經西境,合冶谷水,貫流南北二城中,東南入高陵。鎮四:陂西、王店、樓底、西陽。學政駐所。驛一:建忠。盩厔繁,難。府西南百六十里。南:秦嶺。東南:石樓。西南:安樂山。西:駱谷。竹谷水北緣郿界,仍逕清化入,一曰西清水河,合車谷、稻穀諸水,入武功注渭。渭水逕縣北而東,西南有黑水,即芒水,北流注之,又東入興平。東南:甘水亦北入興平。縣丞駐祖菴。鎮五:終南、尚村、啞柏、清化、臨川。渭南沖,繁,難。府東北百四十里。西南:石鼓。南:倒虎山。西:馬峪,泠水出,合駒兒嶺水,西北入臨潼注渭。渭水合杜化穀水,逕城北,古白渠在焉。西水,東赤水,俱北注之,又東入華州。縣丞駐下邽。鎮二:赤水、田市。驛一:豐原。富平繁,疲,難。府東北百二十里。西北:檀山、天乳、土門。西南:荊山。東北:頻山。石川河即漆沮,自耀入,下流自西北受金定河,一名趙氏河,即澗穀水,東南入臨潼。縣丞駐美原鎮。又東北,道賢鎮。醴泉沖。府西北百二十里。北:武將山。東北:九嵕山、芳山。涇水自永壽入,東北甘河自縣北東流注之,東南入涇陽。鎮二:叱千、甘北。驛一:張店。同官簡。府東北百八十里。明屬西安府。雍正三年改屬耀州。十三年還屬。西南:白馬、鐵龍。北:女回。又神女峽內有金鎖關。東:漆水出北高山,至城北,合同官川及雄同、雷平川,西南流,西有沮水,南流,俱入耀州。東北:大小石磐山水合北入宜君。其南烏泥川,東入蒲城。驛一:漆水。耀州簡。府東北百三十三里。明屬西安府。雍正三年升直隸州。十三年仍為州,還屬。東北:五臺山、磬玉。北:木門、大唐。西北:牛耳山。沮水上源姚渠川自宜君入,合銀耳坪、太子石水楊秀川,為宜君水,南合胡思泉,為沮水,東南逕城西,又東,左會漆水,入富平。澗谷山、清穀水、濁谷水均出西北,南入三原。鎮四:小丘、柳林、照金、廟灣。驛一:順義。孝義繁,難。府東南二百四十里。乾隆四十七年,析咸寧、藍田、鎮安三縣地置,設同知駐孝義川。嘉慶七年移駐舊縣關,即今治。北:秦嶺。東:大頂山。西南:車輪、天書。大峪河一名乾祐河,即柞水,出西北大峪嶺,西南流;東北金井河即甲水,東流;東社川河,東南流;西北洵河,南流:俱入鎮安。寧陜繁,難。府南五百二十里。明正德十六年,設柴家關、五郎壩二巡司。順治中廢。乾隆四十八年,移西安府水利通判駐五郎關。嘉慶五年,析長安、盩厔、洋、石泉、鎮安五縣地置,改設同知。東北:秦嶺。北:萬華山、子午谷。南:五臺山。洵河出紗羅嶺,西南至江口,左合江河,又南至孝義,澧河、日河並從之。西北:甘泉砭,文水出,匯東谷、西河諸水,屈西南入洋,蒲河從之。北:要竹嶺,長安河出,南逕城東,合東河、堤坪河入石泉。有四畝地、五郎關汛。主簿駐江口,嘉慶七年自長安斗門鎮移此。四畝地巡檢,嘉慶十三年移駐新城,十八年廢。

同州府:沖,繁,難。隸潼商道。明,同州屬西安府,領縣五。雍正三年,升直隸州。十三年升府,置附郭縣。耀、白水還隸,又降華州暨所屬之華陰、蒲城、潼關來隸。乾隆十二年,潼關升。西南距省治二百四十里。廣一百八十八里,袤二百九十里。北極高三十四度五十分。京師偏西六度三十七分。領一,州一,縣八。大荔繁,疲,難。倚。雍正十三年以同州地改置。西:黃堆山。北:商顏。南:沙苑。洛水自蒲城緣界逕其西,折東南至船舍渡入,逕西南,東流,渭水逕南界,東北流,並入朝邑。縣丞駐羌白鎮。又坊頭、船舍、潘驛三鎮。朝邑繁,難。府東三十里。明隸西安府。雍正三年來屬。黃河自郃陽入,逕東境而南,受金水,至趙渡南之望仙觀,為洛水入河故道。光緒三十四年,洛徙,至趙渡入之。又南三河口,渭水自大荔入,東北流注之,折東入潼關。主簿駐大慶關。有兩女、太奇、趙渡三鎮。郃陽難。府東北百十里。明隸西安府。雍正三年來屬。西北:梁山。東北:方山。黃河自韓城入,緣東界而南,受百良水。徐水西北、金水東南流,俱入朝邑。古洽水,亦瀵水,亦西南入朝邑。西北:大峪水,自澄城緣界,屈南仍入之。鎮五:百良坊、甘井、王村、黑地、路井。澄城簡。府北百里。明隸西安府。雍正十年來屬。北:界頭山、將軍。西北:壺梯、雲門山。西:洛水,受甘泉水,即縣西河,南入蒲城。東大谷河,南緣郃陽界從之。鎮九:寺頭、業善、韋莊、交通、窯頭、王莊、馮原、塔塚、良輔。韓城難。府東北二百二十里。明隸西安府。雍正三年來屬。東北:龍門山。西北:梁山。西南:韓原,即少梁。黃河緣東北自宜川入,合洽戶川,屈南得龍門口,禹跡存焉,南至官渡,合沮水及芝川,又南入郃陽。西北:神道嶺汛。薛峰、昝村二鎮。華州沖。府南百八十里。明隸西安府。雍正三年升直隸州。十三年仍為州,來隸。西南:五龍。南:少華山。渭水自渭南入,逕北境而東,納州南諸谷水,東北入華陰。鎮七:羅紋、柳子、臺頭、王宿、瓜坡、高唐、江村。驛一:華山。華陰沖,繁。府南百六十里。明隸西安府。雍正三年改屬華州,十三年來屬。南:太華山,即西嶽。河水自朝邑入。西北:渭水自華緣界合沈水,又東合敷水、黃酸水,諸谷水並注焉,又東入於河。鎮三:華嶽、泉店、敷水。蒲城疲,繁,難。府西八十里。明隸西安府。雍正三年改屬華州,十三年來隸。北:堯山,一名浮山。西北:豐山,一名蘇愚山。東北:金粟山。洛水自白水入,逕避難堡,左納甘泉水,合大峪河,入大荔。東北:永豐汛。鎮十:常樂、石表、渭原、孝同、興市、武店、漢底、車渡、荊桃、高陽。白水簡。府西北百三十里。明隸西安府。雍正三年改屬耀州,十三年來屬。東北:黃龍山。西北:秦山。洛水自宜春入,受鐵牛河,經縣北,受孔走河,又東南白水,即南河水,自南境東流注之,又南入蒲城。鎮十:馮雷、西故、南河、雷村、新村、新窯、鐵牛、雷衙、武莊、孔走。潼關沖,繁,難。府東南百里。潼關道治所。明置潼關衛。雍正二年廢。四年置潼關縣,屬華州,十三年來隸。乾隆十二年升。東:麒麟山。西:鳳山,倚以為城。黃河自華陰入,逕北,潼水自南貫城北流注之,東入河南閿鄉。巡司兼司獄駐風陵渡。驛一:潼關。

鳳翔府:沖,繁。鳳邠道治所。東南距省治三百六十里。廣四百二十里,袤三百四十里。北極高三十四度二十八分。京師偏西八度五十九分。領縣七,州一。鳳翔沖,繁。倚。西北:雍山,雍水出焉,南流經縣西,折東南與塔寺河合;又東有橫水,俱東南入岐山。汧水自汧陽緣界南入寶雞。鎮五:橫水、窯店、虢王、彪角、陳村。驛一:東河橋。岐山沖,繁。府東五十里。北:岐山,又有周原。南:秦嶺。北:武將山。西南:渭水自寶雞入,逕城南,東流,斜谷水出西南山,東北流,並入郿。西:湋水,即雍水,自鳳翔入,合橫水,逕縣南,東入扶風。畤溝河自扶風緣界仍入從之。鎮五:益店、龍尾、蔡家、高店、青化。驛一:岐周寶雞沖,繁,難。府西南九十里。秦嶺在南,亦名秦山。東南:陳倉山、石鼓山。西南:和尚原、大散嶺。渭水自秦州緣界入,逕城南而東,右合塔河、洛谷水,左合汧水,又東合潘溪,入岐山。東南:太白河、西南入留壩。上谷水、虢川河、西南凍河即故道水,並西入鳳。東北:利民渠。巡檢駐虢川鎮。又底店、陽平、馬營、益門四鎮。驛二:陳倉、東河。扶風沖,繁。府東百十里。北:岐山,吳雙。東北:梁山。南:飛鳳、賢山。西北:美山。東:茂陵、三畤原。東南渭水,南湋水,與東境漆水、美水合,並東入武功。鎮七:伏波、杏林、絳帳、午井、召公、天度、崇正。驛一:鳳泉。郿簡。府東南百十里。東:太白山,即禹貢惇物。西:馬塚山。西南:武功、斜谷,有五丈原。渭水自岐山入,右合斜谷水,中支磨渠,東支清水河,東南逕城北,又東入扶風。東井田、西南斜谷二渠。斜谷關汛。鎮五:槐芽、橫渠、青化、清湫、金渠。麟游簡。府北百十里。城內童山。西:天臺。東:石臼。南:箭括山。漆水出縣西青蓮山,東北合岐水,其西麻夫川、東雨亭河,並入甘肅靈臺。杜水出西北杜山,逕城南,受澄水,東入乾州。西良舍、西北招賢二鎮。汧陽沖。府西北七十里。東:圭山、龍泉。北:天臺:臥虎。南:箭括嶺。汧水自隴入,西北納草碧谷、暉川河,逕城南,納澗口河、界止河,東南入鳳翔。東縻隃澤。東黃理、西草碧二鎮。隴州沖。府西北百五十里。南:吳嶽。西北:隴山,即隴阪。又汧山,汧水出,合龍門、關山、蒲峪諸水,逕城南而東,受北河,又東南納八渡水,入汧陽。渭水自甘肅秦州逕西南,東入寶雞。西:關山汛。鎮十四:杜陽、東涼、新街、縣頭、八渡、神泉、馬鹿、長寧、赤延、故川、香泉、大松、通關河、溫水。驛一:長寧。

漢中府:沖,繁,疲,難。陜安道治所。總兵駐。明,領州一,縣八。乾隆三十八年,置留壩。嘉慶七年,置定遠。道光五年,置佛坪。東北距省治一千七百里。廣八百一十里,袤六百五十里。北極高三十三度。京師偏西九度十四分。領三,州一,縣八。南鄭沖,繁,難。倚。西南:旱山、黃牛。南:大巴山。東南:梁州。西:龍岡山。東北:武鄉谷、駱谷。沔水即漢水,自褒入,東受褒水中、東二支,及廉水、池水,東入城固。青石關,巡司駐。又西大壩關。鎮四:長柳、上水渡、沙河、彌勒院。驛一:漢陽。褒城簡。府西北四十里。北:七盤山,上為雞頭關。西北:連城。西:牛郎山。南:天池。褒谷在東北,自此入連雲棧。西北百五十里達留壩。沔水自其縣入,西南流,納華陽河,又東受褒水,入南鄭。西南:讓水,一名遜水。北馬道、虎頭、武曲,南松梁、米倉,西北漢陽、甘亭,七關。南:黃官嶺汛,巡司同駐。鎮四:宗營、褒城、長林、高臺壩。驛三:馬道、青橋、開山。城固簡。府東七十里。北:通關、九真、白雲。西北:斗山。漢水自南鄭入,逕胡城,左納文水,即文川,右納南沙河、小沙河,逕城南入洋。陰平、袁揚、原公、文川四鎮。洋簡。府東百二十里。東北:太白。東南:子午谷。西北:酆都。北:興勢山,又灙谷,即駱谷南口。東:赤阪、黃金穀。漢水自城固入,逕南境,左納灙水即鐵冶河、大龍河、酉水、金水河,右納東谷河、桃溪水,東南入西鄉。北:壻水,西經城固,復入西南境,注于漢水。北:華陽營。東北:茅坪汛。縣丞駐華陽鎮。又渭門、真符、謝村、壻水四鎮。西鄉繁,疲,難。府東二百四十里。西南:大巴。小巴。南:皁軍山。東北:饒風嶺。東南:子午山。漢水自洋入,左子午河,即椒溪,合寧陜紋河,西南流注之,牧馬河自城固入,逕城東南,合洋水、白鐵河、神溪,東北流注之。折東入石泉,高川從之。西南:菩提河,南入四川通江。北:司上汛。縣丞駐五里壩,嘉慶七年自大池壩移此。巡司駐大巴關。鹽場巡司,嘉慶七年廢。鎮二:茶溪、子午。鳳沖。府西北三百八十里。西北:紅崖。北:豆積。東北:黃牛寨山。故道水即嘉陵江上流,自寶雞入,逕東北,受三岔河,折西合黃花川、馬鞍山水,至雙石鋪,紅崖河自右注之,入甘肅兩當。野羊河自留壩入,逕城南,合東溝河,入略陽。西南:仙人關。東北:大散關,有漢鳳營駐防。東南:鐵爐川營。東北:黃牛堡汛。鎮四:南星,廟臺子、方石、白石。驛三:草涼、三岔、梁山。丞兼巡司駐三岔。寧羌州,沖,疲,難。府西南三百八十里。東南:龍頭。西北:雞鳴。東北:五丁山,有關。北:嶓塚山,漢水出焉,初名漾水,合五丁峽、黃銅鋪水,東北入沔。玉帶河出西南箭竹嶺,逕城北,受白巖水,為白巖河,亦北入沔。西漢水逕西境,納七道水,西南入四川廣元,為嘉陵江。西北:陽平關,州同駐。大安、黃壩二汛。西北:青鳥鎮。驛二:柏林、黃壩。沔沖。府西四百十里。北:鐵山。東南:定軍山。東北:天蕩、武興。西北:珈珂。漾水自西寧羌入,西南受白巖河,北沮水,西南流,逕略陽東境,復入縣西為黑河,南流注之,始名沔水,又逕城南,東入褒城。西北:黑河汛。鎮四:黃沙、舊州、元山、青羊。驛三:黃沙、順政、大要。略陽沖。府西北二百九十里。北:青泥嶺。西北:殺金嶺。東南:大丙山,丙穴在焉。故道水自甘肅徽縣入,東北合濁水,為白水江入。西:西漢水,即犀牛江,自甘肅成縣入,合石門河來會,是為嘉陵江。又西南,納八渡河,右納落索河,逕野豬山入寧羌。沮水逕東北合冷水河,東南復入沔。東北有白水江汛。峽口、石門二鎮。佛坪要。府東北四百里。嘉慶中設盩洋縣丞於袁家莊,屬西安府。道光五年析盩厔、洋二縣地置,省縣丞,設同知,來隸。南:冠山、鼇山。東:天華。西北:秦嶺、太白。西:楊家溝口,壻水出,馬黃溝水自寶雞南流注之,又南入洋。黑水出北扇子山,東北合蟒河、八斗河,入盩厔。椒溪河出東,東南入寧陜。東北:駱谷關,北口屬盩厔,南口屬洋,中貫境,有十八盤。有黃柏、厚軫子二汛。巡司駐袁家莊。定遠要。府東南四百里。嘉慶七年析西鄉地置,設同知。西:金竹。南:歸仁。西北:父子山。東:星子山,洋水出焉,即清涼川,逕城南,合小洋河、七里溝水,折西北入西鄉。東北楮河、東南雙北河,並東南入紫陽。東南漁水、西北巴水,並西南入四川通江。汛三:瓦石坪、漁渡壩、觀音堂。有漁渡壩、簡池壩二巡司。留壩沖,繁,難。府西北百四十里。本鳳縣地,明設巡司。乾隆十五年,移漢中捕盜通判駐之。三十年析置,職撫民。三十九年改置同知。西北:紫柏山,其東柴關嶺。西北:太白河,為褒水上游,自寶雞入,受紅巖河,為紫金河。虢川河亦自寶雞來注,逕東南,受文川河、青羊河,又南納武關河,入褒城。野羊河出紫柏山,西北入鳳。東北:西江口汛。巡司駐南星。武關巡司省。驛三:松林、留壩、武關。

興安府:繁,疲,難。隸陜安道。總兵駐。明曰興安州,領縣六。乾隆四十七年升府,置安康縣為府治,並省漢陰入之。五十五年,復置漢陰。北距省治六百八十里。廣七百六十里,袤六百二十里。北極高三十二度三十二分。京師偏西七度六分。領一,縣六。安康繁,疲,難。倚。明為興安州,新舊治均在漢南,萬歷十一年徙新治。順治四年還舊治。康熙四十六年復徙新治。乾隆四十七年州升府,改置。北:梅花、牛首。南:趙臺。西:鳳凰。東北:白雲山。西南:魏山。漢水自西紫陽緣界折北入,逕城北,右納大道河,左蒿坪河、月河、神灘河,東北入洵陽。東南:八仙河汛。通判、縣丞同駐西南磚坪。西:泰郊、衡口二鎮。平利簡。府東南百八十里。舊治在西北灌河口。嘉慶八年徙白土關,為今治。西北:女媧北。北:八里岡。西:錦屏。西南:石梁。嵐河出花池嶺,西有黃洋河,與灌河合,俱入安康,北流注漢。東:沖河,會秋河,北入洵陽,為壩河,注漢。東南:南江河,東入湖北竹山。縣丞駐鎮坪。洵陽簡。府東百二十里。北:羊山。東北:水銀、龍山。東南:紫荊山。南:將軍、女華。西北:廟埡,傅家河出,入安康,注漢。漢水自西逕城南,洵河合乾祐河、任河南流注之,又東納蜀河、仙河,入白河。南:七里關汛。白河簡。府東四百里。嘉慶二年,增築外城。南:龍岡山。東北:錫義山。漢水自洵陽入,西逕城北,右納冷水河、白石河,東入湖北鄖。紫陽簡。府西南二百四十里。東:三臺。南:三尖。東南:板廠。南:甕山,下有紫陽洞。又南,望夫山。漢水自漢陰入,逕其西,屈南,任河合紫溪河西南來注,又東逕城南,納汝河、洞河,東北入安康,蒿坪河從之。毛壩關,主簿駐。石泉簡。府西北二百七十里。東:馬嶺。南:銀洞。西:天池山。西:饒風嶺,舊有關。長安河自寧陜入,納汶水河,入西鄉注漢。漢水自西境折西南受珍珠河,又東逕城南,受江河、池河,東南入漢陰。富水河自西鄉入,東逕烏石梁,從之。漢陰繁,疲,難,簡。府西北百八十里。明,縣。乾隆四十七年省入安康,設鹽捕通判。五十五年復置為,改撫民。東南:梁門山。東北:朝陽山。南:文華、鳳天山。池河自寧陜入,合龍王溝,

又西南入石泉,注漢。漢水自西南逕城南,受富水河、木樨河,東南入紫陽。月河出西分水嶺,納花石河,東南入安康,合衡河,注漢。

延安府:繁,難。隸延榆綏道。明,領州三,縣十六。雍正三年,升鄜、綏德、葭三州為直隸州,以洛川、中部、宜君、米脂、清澗、吳堡、神木、府谷八縣分隸之。乾隆初,以榆林府之定邊、靖邊二縣來隸。南距省治七百四十里。廣四百八十里,袤三百九十里。北極高三十六度四十二分。京師偏西七度四分。領縣十。膚施簡。倚。西:鳳凰山,城跨其上。北:伏龍。東北:清涼。東南:嘉嶺。南:臥虎。延水自安塞入,西北而東,西川水東流注之,又東北,南河水北流注之,又曲折東北,左納豐林川、清化水,東入延長。南:石油泉。安塞簡。府北四十里。北:雲臺。東:天澤。西:龍安山。延水自保安入,西北納杏子河,逕城南,曲折東南入膚施。西南:洛水,南入甘泉。北有邊墻。甘泉簡。府南九十里。東北:伏陸山。南:秦冒、溫泉山。洛水自安塞入,右納自修川、北河、美水,左納清泉水、漫漲河水,南入鄜州。西南有甘泉,縣以此名。臨真鎮,縣丞駐。安定簡。府北百八十里。東:鵬山。西:祖師。南:祖師山。西北:高柏山,懷寧河出焉,亦名走馬水,又東北有東溝,並東入清澗。秀延水自安塞入,即北河,俗名縣河,逕城北,合根水、革班川,東南亦入清澗。南:清化水,南入膚施。保安簡。府西北二百二十里。東:艾蒿嶺。南:石樓臺山。西:九吾。洛水自靖邊入,逕城西,納梁家河、吳堡川、周水,東南入安塞。北:杏子河,亦自靖邊入從之。有沙家、靜遠二鎮。宜川簡。府東南二百八十里。東:鳳翅山。北:石關。西南:丹陽。東南:盤古山。黃河自延川入,南延水,逕東北來注之。又南過壺口,受雲巖河,經孟門,受銀川水,即西川,又東南入韓城。北有百直、交口鎮。延川簡。府東北百九十里。城西:西山。東:東峰。西北:青眉山。黃河自清澗入,至老龍口,秀延水合清平川、南站川諸水,東南流注之,又南入延長。西北:永平村,有石油井。延長簡。府東百五十里。東北:獨占。北:高奴山。西:延水自膚施入,逕城,右合關子口,左小鋪原水,又東逕翠屏山,納蘇家河,右安溝,東南入宜川。西北交口水,東至延川注延水。南:錦屏山,下舊有石油井。光緒三十二年,用新法鑿取,油旺質佳。附近膚施、延川、宜君數縣境均產石油。定邊沖,繁。府西北三百五十里。明正統二年置定邊營,屬延安鎮。雍正九年,以定邊、鹽場、磚井、安邊、柳樹澗五堡地置,屬榆林府。乾隆初來隸。東南:南梁山。西北:白露山,即白於山,洛水出焉。右合貝川水、郎兒溝,又東,左合吳倉坡水,東南入保安。南:三山水,一名耿家河,自甘肅靈州入,復合黃家泉,西南入甘肅環縣。北有邊墻,自甘肅花馬池入,東南至靖邊。西:鹽場堡,縣丞駐,後省。靖邊沖,難。府西北三百里。明成化十一置靖邊營,屬延綏鎮。順治初為靖邊所。雍正二年設同知,九年,以安邊、安塞、鎮羅、鎮靖、龍州五堡地置,屬榆林府。乾隆初來隸。西南:大白蓮花山。東:箭桿山。東南:蘆關嶺。西紅柳河,東荍麥河,至城北合流,北出邊墻,折東復入懷遠邊墻為[C104]水。東北:寺灣河、大理河,並東入懷遠。龍州堡、寧塞堡二汛。又寧條梁汛,巡司同駐。

榆林府:沖,繁,難。延榆綏道治所。初沿明制,置東、中、西三路道。康熙元年省西路入中。雍正九年改中為榆葭道,東為延綏鄜道。乾隆二十六年改。總兵同駐。明曰榆林衛。雍正九年,改置榆林府,並置榆林、懷遠、定邊、靖邊四縣。乾隆初,改定邊、靖邊屬延安府,葭降州,暨所隸神木、府穀二縣來隸。南距省治一千三百五十里。廣五百二十里,袤二百二十二里。北極高三十八度十八分。京師偏西七度六分。領州一,縣四。榆林沖,難。倚。本雙山、常樂、保安、歸德、魚河五堡地,明成化七年置榆林衛。雍正二年省入綏德,九年復置縣為府治。城東:駝山。北:紅山,上築墩。東南:石山。無定河自懷遠入,西逕城南而東。清水河一名西河,即榆林河,自邊入,西北納縣境諸水,東南流注之,又東南入米脂。東北:葭蘆川,一名沙河,東南入葭州。西北邊墻有魚河堡、常樂堡二汛。南:碎金鎮。驛二:榆林、魚河。懷遠沖。府西百六十里。明天順中置懷遠堡,屬榆林衛。雍正二年改屬綏德。九年以懷遠、波羅、響水、威武、清平五堡地置,來隸。南:火石山。東:五龍。西南:龍鳳山。無定河即生水,上流曰額圖渾河,一名奢延河,又名幌忽都河,自鄂爾多斯右翼入,逕城北而東,納硬地梁、黑水頭河、柿子河諸水,又東入榆林。西南:[C104]水自靖邊入,東北流,逕城北出邊墻,入無定河。南:大理河自靖邊入,合小理河,東入米脂,復經城東南入米脂。西北:邊墻。葭州疲,難。府東南百七十里。明屬延安府。雍正三年升直隸州。乾隆初,仍降州來隸。南:白雲。北:第一峰。西:西嶺。黃河自神木入,南禿尾河,即吐渾河,逕城北,東南流注之。葭蘆川自西南合五女川,東流來注之,又南受烏龍水、荷葉川入吳堡。神木沖,繁。府東北二百四十里。明屬葭州。乾隆初來隸。西:筆架。東南:天臺。東:龍眼山。東北:響石崖,石馬河出,入府穀注河。河水折西南入,受屈野河、芹河、泗滄河、大柏油河、柏林河諸水,西南入葭。禿尾河自邊入,合永利河從之。神木營,理事同知駐。西南:柏林堡汛。府穀沖。府東北二百里。明屬葭州。乾隆初來屬。北:高梁山。西南:又保。東:五龍山。黃河自鄂爾多斯左翼緣東界而南,受黃甫川、清水川,經南界,孤山川自西北合鎮羌水、麻家溝水、木瓜川,東南流注之,又西南,受石馬川,入神木。有孤山堡、木瓜園堡、清水堡三汛。巡司駐麻地溝。府谷、孤山、鎮羌廢驛。

乾州直隸州:沖,繁,難。隸西乾鄜道。明屬西安府。雍正三年,升直隸州。東南距省治一百六十里。廣九十五里,袤二百二十里。北極高三十四度三十三分。京師偏西八度十五分。領縣二。西北:梁山。東北:雞子堆。西:明月。北漠谷水,西北武水,一名武亭水,即杜水,均逕城西南入武功。東北:泔水,納甘溝,東入醴泉。鎮七:薛祿、陸陌、臨平、陽峪、馮市、陽洪、關頭。驛一:威勝。武功沖,繁。州西南六十里。東:東原。西:西原。西南:三畤原。渭水自扶風入,逕城南,嘉慶中北徙,東入興平。西北:武水自州入,逕城北,合漠谷水,又東南,湋水東流來匯,又南入興平。清水自盩厔入,東北流,逕城東南,又東至興平入渭。鎮六:魏公、游鳳、普集、大莊、楊陵、永安。驛一:邰陽。永壽沖。州西北九十里。西南:武陵山。北:分水嶺,泔水出,逕城東,漠谷水亦出之,逕城西,並南入州。武水出西南石牛山南,逕州西北,復逕縣南入州。西北:拜家河,東北入邠州,注太谷水。北:呂公渠,西南:趙家渠、李家渠、杜渠。鎮四:底窖、蒿店、監軍、儀並。驛一:永安。

商州直隸州:繁,疲,難。隸潼商道。明屬西安府。雍正三年,升直隸州。西北距省治三百里。廣四百六十里。袤四百三十里。北極高三十三度四十九分。京師偏西六度三十五分。領縣四。東南:商山。西:熊耳山。東:雞冠。北:金鳳、小華。西:西巖。西北:塚嶺,即秦嶺。丹水一名丹江,出其東麓,合黑龍峪水,東南流,受水道河、林岔河,經城南,受乳水,又東南受老君峪水,入山陽。有商洛、老君店、黃川、大荊、泉村、西巿、豐陽諸鎮。龍駒寨汛,州同同駐。又東,武關汛。鎮安繁,疲,難。州西南三百四十里。北:都家嶺、長陵、天書山。東南:石驢。東北:夢穀。金井河自孝義緣界入,合社川河,東南入山陽。北:乾祐河,逕城東南,納縣河、冷水河、西南洵河,合小任河,並東南入洵陽。又西南大任河,亦東南入洵陽,注于洵河。有鎮安營駐防。雒南簡。州東北九十里。北:云堂山。東北:陽華。東南:王喬。西:塚嶺山,洛水出焉,東南逕元扈山,北納文峪川,又東逕城北,合石門川,又東會縣河,故縣川,靈水、要水,逕熊耳山,北入河南盧氏。三要司,巡司駐。雞頭關汛。山陽簡。州東南百二十里。東南:天柱山。北:蓮花、元武。東:孤山。西:三鳳。西南:金井河,即甲河,自鎮安入,合花水河,至城南合河口。安武水即關柎水,逕城西,合縣河、桐峪河,又東受董家溝、箭河、漫川河諸水,南入湖北鄖西,注于漢。東:丹江,與銀花河並入商南。竹林關、漫川關二汛。商南簡。州東南二百五十里。南:商雒山。東:魚難。東南:青山。東北:角山。丹水自州入,西南受銀花河,為兩河,又東納武關河、清油河,逕城南,合縣河、湘河,東入河南淅川。有富水關汛。

邠州直隸州:簡。隸鳳邠道。明屬西安府。雍正三年,升直隸州。東南距省治三百二十里。廣二百九十里,袤九十五里。北極高三十五度四分。京師偏西八度二十三分。領縣三。南:豳山。西:無量。東:蒲澤谷。涇水自長武緣界入,西北而東,逕城北,合安化河、白土川即漆水,復合西河、南河,左納皇澗、過澗,又東南至斷涇渡,右納太峪河,緣永壽界入淳化。鎮七:高村、大峪、宜祿、停口、永樂、史店、白吉。驛一:新平。三水簡。州東北六十里。城東:翠屏。東南:石門山,七里川出,即姜嫄河,西南入淳化。東北:汃水,一名縣河,自宜賓入,受連家河、蒼耳溝水,逕城南,並西南入州。西北:大陵水,即皇澗,自甘肅正寧入,會羅川水,其南梁渠川,即過澗,並入州。又西北,馬嶺水,入甘肅寧州。鎮五:土橋、張洪、太羽、職田、底廟。淳化簡。州東南百四十里。東北:壽峰山。西北:甘泉山。西:涇水左瀆自涇入,受姜嫄河,逕城南入醴泉。冶谷水出縣北蠍子掌山,屈東,逕城東,匯甘泉、走馬水、胡盧河,東南入涇陽。東北:清水,自耀緣界,東南流,仍入之。鎮六:常實、大店、石橋、辛店、通潤、姜嫄。長武簡。州西北八十里。西:鶉觚原。北:神龍。南:宜山。涇水自甘肅涇州入,逕北界,受馬蓮河,折南逕城東,至回龍山北。西南黑水即芮水,與納水合,東南流注之,又東南入邠。鎮三:停口、冉店、窯店。驛一:宜祿。

鄜州直隸州:繁,疲,難。隸西乾鄜道。明屬延安府。雍正三年,升直隸州。南距省治五百五十里。廣三百五十里,袤三百八十五里。北極高三十六度四分。京師偏西七度十一分。領縣三。南:高奴山。東北:晉師。北:開元坡。北:洛水自甘泉入,南流,納採銅川、牛武川、逕城東南,廂西水合開撫水,自洛川會街子河來注之,又南入洛川。西北:華池水,即清水河,自甘肅合水入,逕城西,與黑水會,又南納直道河、三川水,西南入中部。州判駐王家角鎮。又交道、屯磨、張村、隆益、牛武五鎮。洛川簡。州東南七十里。舊治在東北。乾隆三十一年徙鳳棲堡,為今治。北:高廟山。東南:爛柯。南:鄜畤山。洛水自西北,南流,納杜家河,入中部。東仙宮河、黃梁河,逕城南,西南流注之,又南入中部。東南:聿津河,西南入宜君。又南川水,東入宜川。鎮十六:仙宮、白城、化石、土基、黃連、吳莊、興平、梁原、樂生、化莊、硃牛、漢寨、廂西、進蒙、永鄉、聿津。中部簡。州西南百四十里。城北:橋山。西北:石堂。洛水自洛川入,右受華池水、沮水、香川水、五交河,又南入宜君。鎮五:北谷、保安、孟家、蘆保、龍坊。驛一:翟道。宜君簡。州西南二百十里。東南:秦山。西北:太白。西南:青龍。洛水自東北,南流,右受石盤川,左受沙河,即聿津河,又南入白水。西南:纏帶水,合玉華川,東北流,入中部,注沮水。又馬蘭川,西南入三水。姚渠川,東南入同官。馬蘭鎮,巡司駐。又雷遠、五里、杏頭、石梯、偏橋、突泉六鎮。縣西姚曲村有石油井。

綏德州直隸州:沖,繁。隸延榆綏道。明屬延安府。領縣一。雍正三年,升直隸州,以延安府之清澗來隸。乾隆元年,以葭州之吳堡來隸。西南距省治一千一百里。廣二百七十里,袤二百四十五里。北極高三十七度三十七分。京師偏西六度二十五分。領縣三。城內:疏屬山。西南:雕陰。西:合龍。東:鳳凰山。黃河自吳堡入,南入清澗。無定河自米脂入,至城東北,右納大理河、懷寧河,東南入清澗。驛一:義合。米脂簡。州南百四十里。南:文屏。北:高家山。無定河自榆林入,逕城西,左納背川水,西南大理河自懷遠入,並南入州。驛一:[C104]川。清澗簡。州南百四十里。明屬延安府。雍正三年來隸。城內:草場山。西:筆架、烽臺。北:官山。黃河緣東界而南,東北無定河,東南流注之,又南入延川。西:秀延水,即辱水,一名清澗水,東流,納士子河,折東南,納坡底河,南入延川。西北:懷寧河,東北流入州境、驛二:奢延、石嘴。吳堡簡。州東百四十里。明屬葭州。乾隆初來隸。西北:高原砦山。南:龍鳳。北:大境。黃河自葭入,東北緣界,東南流,納黽洲水,又西南,納柳毫溝、相公泉、清水溝諸水,又東南入州。宋家川、川口、辛家溝鎮。驛一:河西。


\end{pinyinscope}