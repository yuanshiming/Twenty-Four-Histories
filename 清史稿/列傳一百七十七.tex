\article{列傳一百七十七}

\begin{pinyinscope}
賈楨周祖培硃鳳標單懋謙

賈楨,字筠堂,山東黃縣人。父允升,乾隆六十年進士,由檢討歷官兵部侍郎。

楨,道光六年一甲二名進士,授編修。十三年,大考一等,擢侍講。十六年,入直上書房,授皇六子讀。累擢侍講學士。十九年,大考翰詹,命免試。歷少詹事、內閣學士。二十一年,遷工部侍郎,調戶部。二十七年,連擢左都御史、禮部尚書,調吏部。咸豐二年,協辦大學士。三年,疏請山東籌辦團練,從之。題孝和睿皇后神主禮成,加太子太保。充上書房總師傅,兼管順天府尹。四年,兼翰林院掌院學士。順天府書吏範鶴等與戶部井田科銀庫書吏交結營私,以鈔票抵庫銀。楨察舉其弊,讞定,譴失察諸官有差。楨以發覺察議,拜體仁閣大學士,管理戶部。五年,兼管工部,晉武英殿大學士。

六年,丁母憂,命暫開缺,給假六月回籍治喪,假滿來京。楨疏言:「臣兄弟五人,諸昆疊故,臣幸僅存。今不能為母守制,是臣母有子而如無子,臣何以為子?」力求終制。時御史鄒焌傑亦疏請準其開缺守制,詔允之。八年,服闋,以大學士銜補吏部尚書,仍充上書房總師傅。尋復授體仁閣大學士,管理兵部,兼翰林院掌院學士。十年,充京城團防大臣。是年秋,英法聯軍犯京師,車駕幸熱河,命楨留守,日危坐天安門,阻外軍不令入。及與會議,慷慨不屈。十一年,復晉武英殿大學士,以病請開缺,不許。

穆宗回鑾,偕大學士周祖培,尚書沈兆霖、趙光上疏曰:「我朝從無皇太后垂簾聽政之典。前因御史董元醇條奏,特降諭旨甚明,臣等復有何異詞。惟是權不可下移,移則日替;禮不可稍渝,渝則弊生。皇上沖齡踐阼,欽奉先帝遺命,派怡親王載垣等八人贊襄政務。兩月以來,用人行政,皆經該王大臣擬定諭旨,每日明發,均用禦賞同道堂圖章,共見共聞,內外咸相欽奉。惟臣等詳慎思之,似非久遠萬全之策,不能謂日後之決無流弊。尋繹贊襄之義,乃佐助而非主持。若事無鉅細,皆由該王大臣先行定議,是名為佐助而實則主持。日久相沿,中外能無疑慮?為今日計,正宜皇太后親操出治威權,庶臣工有所稟承,命令有所咨決,不居垂簾之虛名,而收聽政之實效。準法前朝,憲章近代,不難折衷至當。伏查漢和熹鄧皇后、順烈梁皇后,晉康獻褚皇后,遼睿智蕭皇后皆以太后臨朝,史冊稱美。至如宋之章獻劉皇后,有今世任姒之稱,宣仁高太后有女中堯舜之譽。明穆宗皇后,神宗嫡母,上尊號曰仁聖皇太后;穆宗貴妃,神宗生母,上尊號曰慈聖皇太后,惟時神宗十歲,政事皆由兩宮抉擇,命大臣施行,亦未嘗居垂簾之名也。我皇上天亶聰明,不數年即可親政,而此數年間,外而寇難未平,內而洋人偪處,何以拯時艱?何以飭法紀?端以固結人心最為緊要。倘大權無所專屬,以致人心惶惑,是則大可憂者。請敕下廷臣會議皇太后召見臣工禮節,及一切辦事章程,或仍循向來軍機大臣承旨舊制;量為變通,條列請旨酌定,以示遵守。」疏入,命廷臣集議允行。

同治元年安徽降賊苗沛霖謀分兵:一由清江,一渡潁而西,聲稱赴陜西勝保軍營助剿,實有異圖。楨上疏言:「苗沛霖窮而就撫,仍復擁兵觀望,反覆無常。所部素無紀律,倘長驅入陜,何異引狼入室?由潁趨豫,尚為道所必經,繞道清江,則去之愈遠,意存窺伺。西犯山左,則北路門戶大開,固為腹心之患;東犯里下河,淮、揚通海,在在可虞。請飭下勝保嚴阻。」又疏言:「皖省軍情緊急,署撫臣李續宜回籍葬親,請勿拘百日定制,迅飭回任,以固疆圉。」並嘉納之。三年,文宗實錄、聖訓告成,以監修勞,賜花翎。六年,楨年七十,賜壽,恩禮甚渥。尋以病乞休,不許。七年,乃允致仕,食全俸,仍充團練大臣。十三年,卒,詔稱其「持躬端謹,學問優長」,依大學士例賜恤,晉贈太保,入祀賢良祠,謚文端。子致恩,官至浙江布政使。

周祖培,字芝臺,河南商城人。父鉞,嘉慶六年進士,歷官鴻臚寺少卿。

祖培,嘉慶二十四年進士,選庶吉士,授編修。五遷至侍講學士。道光十七年,督陜甘學政。歷侍讀學士、詹事、內閣學士。二十三年,擢禮部侍郎,調工部,又調刑部。二十六年,偕尚書賽尚阿查勘江南江防善後事宜,校閱江蘇、安徽、江西營伍。三十年,文宗即位,疏言:「我朝立政之要,用人之法,備載列聖實錄,請隨時披閱。利害所關,今昔同轍,容有昔之所利不盡利於今者,未有昔之所害不為害於今者;容有昔所欲除之害至今猶未盡除者,未有昔所應防之害至今轉可不防者。惟皇上成法在胸,以應幾務,庶利害了如指掌,而興廢可決於一心。並請責成大吏,力戒欺飾,考察屬吏;其徇隱庇護者,經言官彈劾,即嚴懲督撫,整頓營伍,責令捕盜,勿任推諉。」疏入,被嘉納,特詔飭行。咸豐元年,擢刑部尚書。二年,疏言:「戶部籌餉二十餘條,所議之款,緩不濟急。請照道光二十一年河南河工、城工捐輸章程,變通辦理。」又謂:「按戶派捐,先斂怨於民。請飭各督撫確查巨富之家,勸諭激發忠愛,力圖報效。」從之。

三年,要犯劉秋貴死於獄,承審官未得實情,祖培坐降三級調用,授左副都御史。疏言:「賊匪滋事以來,屢諭各省辦團練,築寨浚壕,仿嘉慶年間堅壁清野之法,行無實效,賊竄突靡定,各州縣毫無豫備,賊至即潰。請嚴飭督撫,責成賢能有司,會紳速辦;有怠玩從事,反滋擾累者,予參處。」從之。歷工部、吏部侍郎。四年,連擢左都御史、兵部尚書,兼管順天府尹。六年,宣宗實錄、聖訓成,加太子太保,調吏部。

八年,會辦五城團防,以吏部尚書協辦大學士,兼署戶部。九年,調戶部,兼署吏部。京師戒嚴,疏陳團防章程六條:曰查戶口以別良莠,勸保衛以聯眾志,任官紳以專責成,協營汛以聯臂指,設水會以備不虞,增幫辦以資助理。車駕幸熱河,命留京辦事,拜體仁閣大學士,管理戶部。十一年,文宗崩,命總理喪儀,兼辦定陵平安峪工程。及穆宗奉兩宮回鑾,祖培疏言怡親王載垣等擬定「祺祥」年號,意義重衣復,請更正,詔嘉其關心典禮。又言近畿各處抗糧拒捕成風,由於州縣不得其人,諭各督撫秉公遴選,毋稍徇隱。同治元年,調管刑部。四年,山陵告成,賜花翎。五年,文宗實錄、聖訓成,賜其子文龠員外郎,文令舉人。六年,卒,年七十五,優恤,謚文勤。

硃鳳標,字桐軒,浙江蕭山人。道光十二年一甲二名進士,授編修。十九年,大考二等,賜文綺,直上書房。尋督湖北學政。歷司業、侍講、庶子、侍講學士、侍讀學士。二十五年,授皇七子讀。連擢內閣學士、兵部侍郎,調戶部。二十八年,命赴天津驗收漕糧。尋偕大學士耆英查辦山東鹽務,疏劾歷任巡撫、運司收受程儀節壽,論譴有差。又言:「山東鹽政疲敝甚於他省,若求裕課暢銷,惟除弊、緝私最為先務。會議變通成法,請先課後鹽以重帑項。」下部議行。又查運庫出借銀七萬餘兩,責賠繳;籓庫積存減平及扣還軍需行裝等款三十萬兩,撥解部庫;通省倉庫正雜未完銀四十一萬兩,缺穀三十七萬石,命限八個月彌補。咸豐元年,擢左都御史,歷署工部、刑部、戶部尚書。

三年,粵匪陷江寧,復陷揚州,漕督楊殿邦退保淮安,廷議調山西、陜西兵七千赴援。鳳標與尚書文慶,侍郎全慶、王慶雲合疏,言:「淮安賊所必爭,萬一賊眾渡河,則河南、山東民情震動,撲滅愈難。請命山東巡撫李僡親往淮安扼賊北竄,並請敕直隸總督迅派布政使張集馨率兵扼要駐守,以為京師屏蔽。」疏入,如所請行。五月,賊陷河南歸德,鳳標與大學士賈楨、尚書翁心存等條擬防剿六事,多被採擇。未幾,悍賊林鳳祥等竄畿輔,復偕楨、心存等奏陳預籌守城事宜。疏入,報聞。四年,授刑部尚書。六年,宣宗實錄、聖訓告成,加太子少保。尋調兵部,復調戶部。

八年,典順天鄉試,因中式舉人平齡硃墨不符,為言官論劾,興大獄,大學士柏葰論大闢,鳳標亦解任聽勘。文宗原其無私,從寬坐失察革職。逾數月,命以翰林院侍講學士銜,仍直上書房,授醇郡王讀如故。歷大理寺少卿、通政使、左副都御史,署刑部侍郎。隨扈熱河,復擢兵部尚書。十一年,護送文宗梓宮回京,追錄扈從勞,加二級。調吏部,充上書房總師傅。同治七年,以吏部尚書協辦大學士,兼翰林院掌院學士。未幾,拜體仁閣大學士,管理吏部。十一年,以病乞休,命以大學士致仕,食全俸。十二年,卒於家,贈太子太保,謚文端。子其煊,工部郎中,官至山東布政使。

單懋謙,字地山,湖北襄陽人。道光十二年進士,選庶吉士,授編修。十七年,入直南書房。十九年,大考二等,以贊善升用。尋授司業,遷洗馬。二十年,督廣東學政,歷侍讀、庶子。以病歸,父喪服闋,請終母養。咸豐三年,粵匪擾湖北,懋謙方居母憂,命在籍治團練。六年,回京,仍直南書房,補原官。七年,督江西學政,歷侍讀學士、少詹事、內閣學士、工部侍郎,均留學政任。十一年,巡撫毓科、布政使慶廉為言官論劾,命懋謙按之,疏言:「毓科非應變之才,適當賊擾,省防尤重。本境兵勇不敷調遣,辦理未能悉合機宜。現雖全境肅清,善後急宜妥辦,籌備浙防,接濟皖餉,大局攸關,恐未能措理裕如。慶廉現未到任,無事跡可考,未敢妄陳。」疏入,報聞。任滿,回京,充實錄館副總裁。同治二年,調吏部,擢左都御史。三年,偕大學士瑞常等進講治平寶鑒,授工部尚書。

四年,命赴盛京偕侍郎志和等承修太廟、昭陵工程。時奉天馬賊猖獗,命懋謙就近查察,劾將軍玉明、府尹德椿,下部議處。回京,疏陳馬賊難防,請籌兵餉出邊會剿,以弭盜源。又請飭奉天所屬各州縣查勘市鎮鄉村應修堡寨之處,勸民作速興築,擇錄嘉慶年間龔景瀚所著堅壁清野議刊發各州縣,令遵照團練守禦之法,量為辦理。疏入,均得旨議行。六年,管戶部三庫事務。七年,調吏部。十年,管國子監事務。十一年,以吏部尚書協辦大學士,尋拜文淵閣大學士,兼管兵部。十三年,因久病請解職回籍,允之。光緒五年,卒於家,詔依例賜恤,有「學問優長,持躬端謹」之褒。贈太子太保,謚文恪。

論曰:自咸豐初軍事起,四郊多壘,廟堂旰食。京師舉辦團防,閣部重臣領之,賈楨、周祖培、硃鳳標皆預其事。其時用人猶循舊格,揆席多由資進。至穆宗踐阼,底定東南,漢閣臣多取勛望,六官中大拜者鮮,惟單懋謙獨由正卿入閣,時以為榮遇焉。


\end{pinyinscope}