\article{志二十八}

\begin{pinyinscope}
時憲九

△凌犯視差新法下

求均數時差

以本日太陽引數宮度分,滿三十秒進一分用。用後編日躔均數時差表,察其所對之數,得均數時差,記加減號。引數有零分者,用中比例求之。

求升度時差

以本日太陽黃道實行宮度分,滿三十秒進一分用。用後編日躔升度時差表,察其所對之數,得升度時差,記加減號。實行有零分者,用中比例求之。

求時差總

以均數時差與升度時差相加減,得時差總。兩時差同為加或同為減者,則相加得時差總,加亦為加,減亦為減。兩時差一為加一為減者,則相減得時差總,加數大為加,減數大為減。

求凌犯用時

置凌犯時刻,加減時差總,得凌犯用時。

求本時太陽黃道經度

以周日一千四百四十分為一率,本次日兩太陽實行相減帶秒減,足三十秒進一分用,有度化分。為二率,凌犯時刻化分為三率,求得四率與本日太陽實行相加,得本時太陽黃道經度。

求本時春分距午時分

以本時太陽黃道經度,滿三十分進一度用。察黃平象限表內右邊所列春分距午時分與凌犯用時相加,內減十二時,不足減,加二十四時減之。得本時春分距午時分。滿二十四時去之。

求本時黃白大距

以周日一千四百四十分為一率,本次日兩黃白大距相減為二率,凌犯時刻化分為三率,求得四率。加減本日黃白大距,本日黃白大距大相減,小相加。得本時黃白大距。

求本時月距正交

以周日一千四百四十分為一率,本次日兩月距正交相減化秒為二率,凌犯時刻化分為三率,求得四率。收作度分秒,與本日月距正交相加,得本時月距正交。

求太陰實緯

以半徑為一率,本時黃白大距正弦為二率,本時月距正交正弦為三率,如本時月距正交過三宮者,與六宮減,過六宮者減六宮;過九宮者,與十二宮減,用其餘。求得四率,為太陰實緯正弦,檢表得太陰實緯,記南北號。本時月距正交初宮至五宮為北,六宮至十一宮為南。如本時月距正交恰在初宮、六宮者,則無實緯。恰在三宮、九宮者,則本時黃白大距即實緯度,三宮為北,九宮為南。

求黃平象限及限距地高

以本時春分距午時分,察黃平象限表內,取其與時分相近者所對之數錄之,得黃平象限。隨看左邊之限距地高錄之,得限距地高。

求星經度

按所取之星,察儀象考成卷二十六表內所載本星之黃道經度,加入歲差,表以乾隆九年甲子為元,至道光十四年甲午,計九十年,應加歲差一度十六分三十秒,以後每年遞加歲差五十一秒。得本年星經度。

如求五星經度,則以周日一千四百四十分為一率,凌犯時刻化分為二率,一日星實行為三率,以本次日兩實行相減,得一日星實行。求得四率,為距時星實行。與本日星經度相加減,順行加,退行減。得本時星經度。

求星緯度

按所取之星,察儀象考成卷二十六表內所載本星之黃道緯度錄之,無歲差。記南北號。

如求五星緯度,則以周日一千四百四十分為一率,凌犯時刻化分為二率,一日星緯較為三率,本次日兩緯度同為南或同為北者,則相減得星緯較。一為南一為北者,則相加得星緯較。求得四率。與本日星緯度相加減,本日緯度大相減,本日緯度小相加。若相加為三率者,所得四率必與本日緯度相減,仍依本日南北號。如所得四率大於本日星緯,則以所得四率轉減本日星緯,其南北號應與次日同。得本時星緯度,記南北號。

求月距限

以星經度與黃平象限相減,得月距限,記東西號。星經度大為限東,小為限西。如星經度與黃平象限一在三宮內,一在九宮外,應將三宮內者加十二宮減之。所得月距限太陰實緯南在六十度內,實緯北在八十度內者,不必求地平限度。如緯南過六十度,緯北過八十度,則求地平限度。

求距限差

以限距地高及太陰實緯度分,察距限差表內縱橫所對之數錄之,得距限差,記加減號。太陰實緯南減北加。

求地平限度

置九十度,加減距限差,得地平限度。

以地平限度內減最小視經差八度五十五分一十七秒,得視地平限度,如月距限大於視地平限度者,為月在地平下,即不必算。因太陰距地最近,其視行隨時不同,故取最小視經差以定視限。乃按最小限距地高,月在黃道極南,求得最小黃經高弧交角二十六度六分二十四秒。以最小太陰地半徑差及最速月實行,求得最小距分三十七分八秒。變赤道度得九度一十七分,求其相當最小黃道度為八度三十一分三十四秒。再加最小東西差二十三分四十三秒,得最小視經差八度五十五分一十七秒。然月在最高時,地半徑差最小,而其月實行必遲,則距分轉大。今俱取其最小者,恐有遺漏耳。

求距極分邊

以半徑為一率,月距限餘弦為二率,限距地高正切為三率,求得四率,為距極分邊正切,檢表得距極分邊。

求月距黃極

置九十度,加減太陰實緯,南加北減。得月距黃極。

求距月分邊

以月距黃極內減距極分邊,得距月分邊。

求黃經高弧交角

以距月分邊正弦為一率,距極分邊正弦為二率,月距限正切為三率,求得四率,為黃經高弧交角正切,檢表得黃經高弧交角。若月距限為初度,是太陰正當黃平象限,則黃經與高弧合,無黃經高弧交角。

求本次日月實引

以本日月引數加減本日初均,得本日月實引,以次日月引數加減次日初均,得次日月實引。

求本時月實引

以周日一千四百四十分為一率,凌犯時刻化分為二率,本次日兩實引相減帶秒減,足三十秒進一分用,度化分。為三率,求得四率。收為度分,與本日月實引相加,得本時月實引。

求本時本天心距地

以周日一千四百四十分為一率,凌犯時刻化分為二率,本次日兩本天心距地數相減為三率,求得四率。與本日本天心距地數相加減,本日本天心距地數大相減,小相加。得本時本天心距地。

求距地較

以本時本天心距地內減距地小數,得距地較。

求月距天頂

以黃經高弧交角正弦為一率,限距地高正弦為二率,月距限正弦為三率,求得四率為月距天頂正弦,檢表得月距天頂。若無黃經高弧交角,則以月距黃極內減限距地高即得。

求太陰地半徑差

以本時月實引滿三十分,進一度用。及本時本天心距地,察後編交食太陰地半徑差表內所對之數,即太陰地半徑差。如本時本天心距地有遠近者,以距地較比例求之。

求本時高下差

以半徑為一率,月距天頂正弦為二率,太陰地半徑差為三率,若推凌犯五星,除土、木二星無地半徑差外,火、金、水三星皆有地半徑差。乃看星引數,自十宮十五度至一宮十五度,為最高限。自一宮十五度至四宮十五度,自七宮十五度至十宮十五度,為中距限。自四宮十五度至七宮十五度,為最卑限。以星引數所當之限,察其本星最大地半徑差,與太陰地半徑差相減,得星月地平高下差,為三率。求得四率,即本時高下差。

求東西差

以半徑為一率,黃經高弧交角正弦為二率,本時高下差為三率,求得四率,即東西差。如無交角,則無東西差,高下差即南北差,凌犯用時即凌犯視時。

求南北差

以半徑為一率,黃經高弧交角餘弦為二率,本時高下差為三率,求得四率,即南北差。

求太陰視緯

以太陰實緯與南北差相加減,得太陰視緯,記南北號。緯南相加仍為南,緯北相減仍為北,如南北差大,則反減變北為南。

求太陰距星

以太陰視緯與星緯相加減,得太陰距星,記月在上下號。如兩緯度同為北或同為南者則相減;月緯大,北為在上,南為在下;月緯小,北為在下,南為在上。兩緯度一為南一為北者則相加。月緯北為在上,月緯南為在下。若兩緯度相同,減盡無餘,為月掩星,凡相距在一度以內者用;過一度外者,為緯大,不用,即不必算。

求太陰實行

以本時月實引滿三十分,進一度用。及本時本天心距地,察後編交食太陰實行表內所對之數,得太陰實行。如本時本天心距地有遠近者,以距地較比例求之。

求距分

以太陰實行為一率,東西差為二率,一小時化作三千六百秒為三率,求得四率,即距分,記加減號。月距限東為減,月距限西為加。

求凌犯視時

置凌犯用時,加減距分,得凌犯視時,如凌犯用時不足減距分,加二十四時減之,所得凌犯視時為在前一日;如加滿二十四時去之,所得凌犯視時為在次日。時刻在日出前日入後者用;在日出後日入前者,即為在晝,不用。

如月在緯南,月距限過六十度,及月在緯北,月距限過七十度者,須用下法求之。

求視時春分距午時分

置本時春分距午,加減距分,得視時春分距午。如本時春分距午不足減距分者,加二十四時減之;若相加過二十四時者去之。

求視時黃平象限

以視時春分距午時分,察黃平象限表內,取其與時分相近者,所對之數錄之,即得視時黃平象限。

求視時月距限

置星經度,與視時黃平象限相減,得視時月距限,其度小於地平限度者用;若大於地平限度者,為月在地平下,不用。

黃平象限表

黃平象限表,按京師北極高度三十九度五十五分,黃赤大距二十三度二十九分,依黃道經度,逐度推得春分距午時分、黃平象限宮度、限距地高度分,三段列之。表名「春分距午」者,乃春分距午正赤道度所變之時分也。「黃平象限」者,乃本時黃平象限之宮度也。「限距地高」者,乃本時黃平象限距地平之高度也。表自三宮初度列起者,因太陽黃道經度三宮初度為春分,即春分距午之初也。

用表之法,以本時太陽黃道經度之宮度,察其所對之春分距午時分,加凌犯用時,得數內減十二時,不足減者加二十四時減之,得本時春分距午時分。依此時分,取其相近之春分距午時分所對之黃平象限宮度及限距地高度分,即得所求之黃平象限及限距地高也。設本時太陽經度一宮十五度,凌犯用時十九時四十五分,求春分距午及黃平象限★限距地高,則察本表黃道經度一宮十五度所對之春分距午為二十一時九分五十四秒。加凌犯用時十九時四十五分,內減十二時,餘過二十四時去之。得四時五十四分五十四秒,為所求之春分距午時分。乃以此時分察相近者,得四時五十四分五十一秒。其所對之黃平象限為五宮十六度五十九分二十七秒,即所求之黃平象限宮度。其所對之限距地高為七十二度四十九分五十八秒,即所求之限距地高也。若黃道經度有零分者,滿三十分以上則進為一度,不用中比例,因逐度所差甚微故也。

表略

○距限差表

距限差表,按限距地高度逐段列之,前列太陰實緯度分,中列黃道南北,自初度十分至五度十七分之距限差,緯南為減,緯北為加。

用表之法,以限距地高之度與太陰實緯度,察其縱橫相遇之數,即所求之距限差也。設限距地高二十八度,太陰距黃道南四度二十分,求距限差,則察限距地高二十八度格內橫對太陰實緯四度二十分之距限差為八度十二分,即所求之距限差。其緯在黃道南,是為減差也。限距地高以逐度為率,若限距地高有三十分以上者,進作一度,不及三十分者去之。太陰實緯以十分為率,若太陰實緯有零分者,五分以上進作十分,不足五分者去之。俱不用中比例,因逐度分之數所差甚微故也。

表略


\end{pinyinscope}