\article{志九十四}

\begin{pinyinscope}
○職官六新官制

內閣外務部出使大臣稅務處民政部內外巡警總度支部清理財政處大清銀行

造幣總廠學部國子監大學堂陸軍部海軍部法部修訂法律館大理院

京師各級審檢農工商部郵傳部軍諮府弼德院資政院鹽政院

典禮院禮學館提學使提法使外省各級審檢東三省各司禁★軍

督練公所軍制鎮制陸軍鎮監巡防隊海軍艦制

清初釐定官制,職儀粗具。中更六七作,存改洄沿,世不同矣。延及德宗,外患躡跡,譯署始立。繼改專部,商、警、學部接踵而設,並省府、寺,乃分十部。嗣議立憲,理籓改部,軍諮設處,復更巡警為民政,戶為度支,商為農工商,兵為陸軍,附立海軍處,刑為法,別立大理院,又取工部所司輪路郵電專設郵傳部。以今況往,洵稱多制。宣統紹述,合樞於閣,增海軍部,省吏部,改禮部為典禮院,鹽政處為鹽政院。猶慮閣權過重,設弼德院以相維系,資政院以為監督。增埤前事,取臬殊方,因事創名,官冗職雜,階資官品,肇域未區。簡奏咨補,故實斯在,輯而存之,具載後簡,亦得失之林也。

內閣總理大臣,協理大臣,各一人。特簡。國務大臣十人。各部大臣兼充。丞一人。承宣長,副長,各一人。制諮、敘官、統計、印鑄四局,局長各一人。丞以下俱請簡。其屬有:僉事,印鑄藝師,俱奏補。藝士,錄事俱咨補。各員。所轄法制院,院長,副院長,各一人。參議四人。俱請簡。參事,奏補。僉事,錄事,視事繁簡酌置。

總理掌參畫機要,締綸時務。法律詔令,會國務大臣尾署名銜。事涉一部或數部,會所司大臣署之。會議時充議長,協理佐之。丞掌主閣務,綜領眾局,方軌諸長。承宣掌布絲綸,守法典,司文書圖籍。制誥掌詔旨制敕,璽書冊命,起草進畫,稽頒寶星勛章,典領籓封勛級。敘官掌考功定課,匯覈履行。統計掌統一計表,刊行年鑒。印鑄掌編輯官報。餘依往制。詳禮部。法制院掌編纂法規,修明法令,擬上候裁。

光緒三十二年,改組內閣,設會議政務處,以各部尚書為內閣政務大臣。宣統三年,改責任內閣,以軍機大臣為總、協理大臣,並定內閣屬官制。如前所列。

外務部外務大臣,副大臣,各一人。特簡。承政左、右丞,參議左、右參議,各一人。俱請簡。參事四人。奏補。其屬:司務司務二人。咨補。和會、考工、榷算、庶務四司,郎中、員外郎、主事各二人。俱奏補。以上各部同。

大臣掌主交涉,昭布德信,保護僑人傭客,以慎邦交。副大臣貳之。丞掌機密文移,綜領眾務。參議掌審議法令,參事佐之。各部同。和會掌使臣覲見,盟約賞賚,兼司領事更替,司員敘遷。考工掌司鐵軌、礦產、電線、船政,凡制造軍火,聘用客卿,招工、游學諸事,各擅其職。榷算掌蕃貨海舶征榷貿易,綜典國債、郵政,勾檢本部暨出使度支。度務掌江海防務,疆域界址,凡傳教、游歷,賞恤、禁令,裁判獄訟,並按約以待。有丞、參上行走,額外司員,七品小京官。民政、郵傳、法部小京官定額缺。所轄:儲材館,提調、幫提調各一人。本部司員內遴派。文案、支應、庶務、俱派員分治其事。

雍正五年,定恰克圖市約,置辦理俄事大臣,見第五款。不為恆職。咸豐元年,改歸理籓院。十年,文宗北狩,特置專官辦理撫局。其冬,設總理各國事務衙門,命恭親王奕領之。司員統稱章京,置滿、漢各八人。時行分署治事制。戶部司員覈關稅,理籓部司員典文移,兵部司員治臺站驛遞,內閣人員主機密,俱隸總辦、幫辦。三年,改為英、法、俄、美四股。九年,增設海防股。後改俄、德、英、法、日本五股。宣統元年,合俄、德為一,增設秘書、機要二股。明年,置總辦四人,曰總辦章京。同治元年,增置額外章京,滿、漢各二人。二年各增六人。光緒九年各增四人。十年各減四人。二十三年各增二人。三年,設司務,置司務二人。光緒二十七年,辛丑和約成,更名外務部,班列各部上。置總理親王,會辦尚書,兼會辦左、右侍郎,各一人;改總辦為左、右丞,左、右參議各一人。並置郎中以次各官,不分滿、漢。三十二年改訂官制,意合滿、漢,而翰林、都察兩院仍依往制。是歲增置繙譯官十有五人。七、八、九品各五人,分股治事。宣統三年,新內閣成,省總理、會辦兼職,改尚書為大臣,侍郎為副大臣。省侍郎一缺,各部同。管部之制,至是遂廢。

頭等出使大臣,正一品。特簡。參贊,正三品。通譯官,正五品。俱奏補。無定員。有事權置,畢乃省。

二等出使大臣,正二品。特簡。參贊官,初制四品,後改從四品。奏補。各一人。英、俄、德、日本、奧、義、和、比各一人。法、日、葡各一人。美、墨、秘、古各一人。分館代使二等參贊官二人。日斯巴尼亞一人。葡萄牙一人。二等參贊兼總領事三人。墨西哥一人。秘魯一人。古巴一人。三等參贊八人。初制五品,後改正五品。奏補。英、法、德、俄各一人。美、日本各二人。二、三等通譯官。二等從五品。三等從六品。奏補。一、二等書記官。一等從五品。二等從六品。奏補。商務委員,正五品。奏補。武隨員,各使館俱一人。唯奧、義、和、比不置三等通譯官、武隨員。分館二等通譯官、書記官俱一人。總領事從四品。奏補。十有三人。新嘉坡、澳洲、南斐洲、坎拿大各一人,隸英使。海參葳一人,隸俄使。墨西哥、古巴、金山、小呂宋、美利濱、巴拿馬各一人,隸美使。橫濱、朝鮮各一人,隸日本使。爪哇一人,隸和使。領事正五品。奏補。十有四人。檳榔嶼、紐絲綸、仰光、溫哥埠各一人,隸英使。檀香山、嘉裏約各一人,隸美使。薩摩島一人,隸德使。神戶、長崎、仁川、釜山、新義州各一人,隸日本使。泗水、巴東各一人,隸和使。副領事從五品。奏補。二人。元山、甑南浦各一人,隸日本使。又有外國人兼代領事者。法,馬賽;義,米朗、納婆爾士;美,波士頓、費城諸處。

使臣掌國際交涉。參贊佐之。領事掌保護華僑。

康熙初,俄國通使,未垂為制。同治六年,始遣使辦理交涉,以道員志剛等及美使蒲安臣膺其選。光緒元年,定出使制,命侍郎郭嵩燾使英,翰林院侍講何如璋使日本,京卿陳蘭彬使美日秘國,俱置副使。別設秘、日分館,置金山、嘉裏約、古巴各總領事。後為自主國,改遣公使。二年,定使館參贊二人,糸番譯四人。十四年,復定繙譯、隨員二人或三人。分館參贊兼領事一人,糸番譯、隨員各一人,參贊如故。三十二年,定參贊以次各員額,如前所列。厥後聯翩四出,英使兼領義、比,俄使兼駐德,以奧、和隸之。四年,置新嘉坡領事,後改總領事。日本各口岸理事官。後改領事。明年,省副使,置檀香山領事。八年,置紐約領事。十三年,置小呂宋總領事,仰光領事,檳榔嶼副領事。後改領事。十七年,置南洋各島領事。二十一年,簡法國專使。二十三年,簡德國專使,和改隸之。並增置韓使。三十三年撤回,改總領事。二十六年,置韓國各口岸領事,及海參葳商務委員。後改總領事。二十八年,改駐法使臣兼使日國,駐美使臣兼使古巴,別設分館,並簡奧、義、比三國專使。明年,設墨分館。三十年,置南斐洲總領事。三十一年,簡和國專使兼理保和會事,並以法日使臣兼領葡使,尋設葡分館。三十四年,定使臣為二品專官,並參贊等官品秩。宣統元年,置美利濱、坎拿大、巴拿馬總領事。嗣是澳洲、溫哥埠、薩摩島、紐絲綸諸領事踵相躡。三年,置爪哇總領事,泗水、巴東領事。其秋置朝鮮新義州領事。

三等出使大臣,正三品。特簡。參贊臣,通譯官,無定員,不恆置。

保和會專使大臣一人。正二品。特簡。陸軍議員一人。武官諳西文者充之。光緒三十三年,罷和使兼職改置。

督辦稅務大臣,幫辦大臣,各一人。以大學士、尚書、侍郎充。後改大臣、副大臣充。掌主關稅,督率關吏。提調,幫提調,分股總辦,幫辦,俱各一人。外務部、度支部丞、參兼充。所轄:總稅務司,副總稅務司,各一人。稅務司四人,副稅務司六人,各關稅務司五十有九人,潮海五人。粵海、岳州、北海各三人。膠海、鎮江、東海、閩海、津海、金陵、蘇州、吉林各二人。江海、梧州、拱北、哈爾濱、山海、浙海、廈門、九龍、九江、亞東、長沙、大連、甌海、福海、三水、龍州、杭州、安東、沙市、重慶、江門、南寧、瓊海、宜昌、奉天、騰越、思茅、蒙自各一人。副稅務司三十有七人。江漢、粵海、江海、三水、津海、琿春各三人。大連、潮海、瓊海、九龍各二人。蘇州、南寧、龍州、重慶、奉天、杭州、廈門、閩海、哈爾濱、蕪湖、大通釐局各一人。以上俱外國人為之。初,海關置監督。各部俸深司員充之。旋改歸督、撫監督,名焉耳。自道光以來,海疆日闢,於是始置北洋、南洋通商大臣,關道及監督隸之。亦有將軍兼理者。津海歸直隸津海道管理,山海歸奉天奉錦山海道管理,東海歸山東登萊青道管理,俱隸北洋。鎮江歸江蘇常鎮通道管理,江海歸江蘇蘇松太道管理,蕪湖歸安徽皖南道管理,浙海歸浙江寧紹臺道管理,甌海歸浙江溫處道管理,江漢歸湖北漢黃德道管理,宜昌歸湖北荊宜施道管理,重慶歸四川川東道管理,俱隸南洋。閩海歸福州將軍管理。粵海、潮海、北海、瓊海、九龍、拱北,監督各一人。嘉峪歸甘肅安肅道管理,龍州歸廣西太平思順道管理,蒙自歸雲南臨安開廣道管理,隸本省督、撫。咸豐以後,聘用英人威妥瑪、美人斯密斯氏襄辦稅務,李泰國繼之。派為總稅務司;凡海關俱置稅務司、副稅務司,後沿江各埠,及內地陸路增開口岸,並屬海關。是為海關募用客卿之始。時管轄之權屬總理衙門。光緒二十三年,始設稅務處,總稅務司以次各官並受其節度。先是戶關、工關分隸戶、工兩部,至是始以常關標名。嗣外部與本處定常關分設稅局,五十里外者歸監督,五十里內者歸稅務司,此內、外常關名稱所由昉也。

民政部民政大臣,副大臣,左、右丞,左、右參議,各一人。承政員外郎,主事,小京官,各四人。參議參事二人。民治、警政、疆里、營繕、衛生五司,郎中八人,民治,警政、疆里各二人,餘各一人。員外郎十有六人,民治、警政、營繕各四人,餘各二人。主事十有八人,民治、警政各五人,營繕四人,餘各二人。小京官各一人。習藝所員外郎一人,兼充消防隊總理。主事二人,五品警官五人,消防隊三人。習藝所二人。六、七品警官各九人,消防隊各六人。習藝所各三人。八、九品警官各十有二人。消防隊各八人。習藝所各四人。以上俱隸警政司。六、七品藝師各一人。隸營繕司。六、七品醫官各一人。隸衛生司。自警官以下俱奏補。八品錄事二十人,九品錄事三十有二人。俱咨補。

大臣掌主版籍,整飭風教,綏靖黎物,以奠邦治。副大臣貳之。民治掌編審戶口,兼司保息鄉政。警政掌巡察禁令,分稽行政司法。疆里掌經界圖志,審驗官民土地。營繕掌陵寢工程,修治道路,並保守古跡祠廟。衛生掌檢醫防疫,建置病院。所轄:豫審所,後隸大理院。路工局,教養局,俱遴員分治之。

光緒三十年,設巡警部,置尚書,左、右侍郎,左、右丞,參議,各一人。警政、警法、警保、警學、警務五司,郎中五人,三十二年增二人。員外郎、主事各十有六人,三十二年增員外郎二人,主事四人。三十四年增營繕司一人。小京官四人,三十二年增五人。一、二、三等書記官各十人。仿七、八、九品筆帖式舊制。三十二年改為八、九品錄事。習藝所員外郎一人,主事二人。三十二年,更名民政部。設承政、參議兩,置參事二人。改設民治、疆里、營繕、衛生諸司,警政如故。宣統元年,定習藝所及消防隊員額。如前所列。三年,改尚書為大臣,侍郎為副大臣。

內、外城巡警總,丞各一人。初制正四品。光緒三十三年升從三品。請簡。掌徼循坊境,並典蹕路警衛。總務處總僉事各一人。從四品。奏補。行政、司法、衛生三處各僉事三人。正五品。俱奏補。五品警官各四人。六品警官十有九人。內城十人。外城九人。七品警官二十人。內城十有一人。外城九人。八品警官二十有七人。內城十有四人。外城十有三人。九品警官二十有八人。內城十有五人。外城十有三人。七品以上奏補。八品以下咨補。八、九品錄事各四人。委用。

光緒三十年,設京師內、外城巡警總,置丞各一人。設總務、警務、衛生三處,置參事各一人。正五品。三十三年改僉事。內城五分,外城四分,知事九人。正五品。三十二年,增司法處。改警務曰行政。升總務處僉事品秩為屬官首領。置五品以下各警官,無定員。八、九品錄事各四人。並內五分為中、左、右三,外四分為左、右二,省知事四人。設內城二十六區,外城二十區,置區官、六、七品警官充。尋改區長。區副八、九品警官充。尋改區員。各一人。三十四年,省內、外城區半之。宣統元年,裁分,省知事。

度支部度支大臣,副大臣,各一人。左、右丞,左、右參議,各一人。承政、參議兩,俱郎中三人,員外郎四人,主事三人。田賦、漕倉、稅課、筦榷、通阜、庫藏、廉俸、軍饟、制用、會計十司,郎中三十有一人,制用四人。餘各三人。員外郎四十有四人,制用六人。田賦、庫藏各五人。餘各四人。主事三十有五人。田賦、筦榷、通阜、廉俸、會計各四人,餘各三人。金銀庫,郎中一人,員外郎四人,主事二人。收發稽察處,督催所改。員外郎一人,主事二人。

大臣掌主計算,勾會銀行幣廠,土藥統稅,以經國用。副大臣貳之。田賦掌土田財賦,稽覈八旗內府莊田地畝。漕倉掌漕運覈銷,倉穀委積,各省兵米穀數,合其籍帳以聞。稅課掌商貨統稅,校比海關、常關贏絀。筦榷掌鹽法雜課,凡盤查道運,各庫賑斂,土藥統稅,並校其實。通阜掌礦政幣制,稽檢銀行幣廠文移。庫藏掌國庫儲藏,典守顏料、緞疋兩庫。廉俸掌覈給官祿,審計百司職錢餐錢。軍饟掌覈給軍糈,勾稽各省報解協饟。制用掌覈工銀,經畫京協各饟,兼司雜支例支。會計掌國用出納,審計公債外★,編列出入表式。金銀庫掌金帛期會。收發稽察處掌各司受事付事。所轄:幣制司,提調一人,幫提調二人。本部丞、參兼充。庶務處,調查、籌辦、稽覈、編譯各股,俱派員分治其事。

光緒三十二年,改戶部設,省財政處入之,置尚書,左、右侍郎,左、右丞,參議,各一人。並十四司為十司,改置郎中以次各官。如前所列。宣統三年,改尚書為大臣,侍郎為副大臣。

清理財政處,提調,幫提調,各二人。本部丞、參兼充。總辦,幫總辦,各一人。總覈坐辦科員無恆額。各省清理財政正監理官二十人,給三、四品卿銜,奉天、直隸、江蘇、安徽、山東、山西、河南、陜西、甘肅、新疆、福建、浙江、江西、湖北、湖南、四川、廣東、廣西、雲南、貴州各一人。副監理官二十有四人。奏派吉林、黑龍江、江寧、兩淮各一人,餘同正監理官。宣統元年置。

大清銀行,正監督,正三品。請簡。副監督,各一人。儲蓄銀行總辦一人。分行總辦二十人。津、滬、漢、濟、奉、營、庫、重、廣、贛、晉、汴、浙、閩、吉、秦、皖、湘、滇、寧各一人。以上由大臣奏派。光緒三十三年,設戶部銀行,置總監督,秩視左、左丞。尋更名正監督。明年改為大清銀行。

造幣總廠,正監督一人,正三品。請簡。副監督二人。分廠,總辦、奉天、江寧、廣州、四川、雲南,由清理財政正監理官兼充。幫辦江寧、武昌、廣州、四川、雲南,由副監理官兼充。各五人。光緒三十三年置。

學部學務大臣,副大臣,各一人。左、右丞,左、右參議,各一人。參事參事四人。司務司務二人。總務、專門、普通、實業、會計五司,郎中各二人,員外郎十有五人。總務五人,普通四人,餘各二人。主事十有八人。總務、普通各六人,餘各二人。一等書記官正七品。奏補。十有一人,二等正八品。十有七人,三等正九品。十有五人。二、三等俱咨補。

大臣掌勸學育材,稽頒各學校政令,以迪民智。副大臣貳之。總務掌機要文移,審覈圖書典藉。專門掌大學及高等學校,政藝專業,咸綜領之。普通掌師範、中、小學校,各以其法定規稽督課業。實業掌農工商學校,並審覈各省實業,為民興利。會計掌支計出入,典領器物,及教育恩給。其兼轄者,八旗學務處總理,協理,督學,調查圖書各局長,局員,編訂名詞館總纂,圖書館正副監督以次各員,俱擇人任使,不設專官。

光緒二十二年,置管理官書局大臣。先是京師設強學書局,詳練時務。至是改歸官辦。二十七年,更命尚書張百熙充管學大臣,管理大學堂事。二十九年,改學務大臣。三十二年,始設學部,置尚書,侍郎,左、右丞,參議,各一人;五司郎中各一人,員外郎十有二人,主事十有五人,視學官無恆額。定正五品。派司員暫充。明年,命大學士張之洞領部事,非永制。宣統元年,改視學官為差,增郎中五人、員外郎四人、主事三人。三年,改尚書為大臣,侍郎為副大臣。

國子監,丞一人。正四品。請簡。掌文廟闢雍典禮。典簿正七品。奏補。四人,掌祀典廟戶。典籍正八品。咨補。四人,掌祭器、樂器。文廟七、八、九品奉祀官各二人。咨補。正通贊官、從六品。奏補。副通贊官從八品。咨補。各二人。二、三等書記官各三人。光緒三十二年置。

大學堂,總監督一人。正三品。請簡。經、法、文、工、商五科監督各一人。奏派。教務、庶務、齋務各提調,俱延聘通曉學務者為之。光緒二十五年,創設京師大學堂,命大學士孫家鼐領之。三十二年,定總監督為專官。

陸軍部陸軍大臣,正都統。特簡。副大臣,副都統。特簡。各一人。參事四人。檢察官八人。部副官四人。各省調查員無恆額。俱正參領以次軍官充之。副參領以上請簡,協參領以下奏補,額外軍官、軍佐咨補。錄事二人。額外軍官及中、下士充之。下同。承政、軍制、軍衡、軍需、軍醫、軍法六司,各司長一人,副協都統、正參領充。處長同。副官一人。正、副軍校及相當文官充。科長十有六人,承政科四:曰秘書,曰典章,曰庶務,曰收支。軍制科七:曰蒐簡,曰步兵,曰馬兵,曰砲兵,曰工兵,曰輜重,曰臺壘。軍衡科四:曰考績,曰任官,曰賞賚,曰旗務。軍需科三:曰統計,曰糧服,曰建築。軍醫科二:、曰衛生,曰醫務。軍實科二:曰制造,曰保儲。科各一人。正、副參領充。一、二、三等科員百六十有二人。承政二十八人。軍制四十有一人。軍衡四十有七人。軍需三十人。軍醫十有四人。一等副協參領充。二等協參領、正軍校充。三等正、副軍校充。譯員四人,司電員三人,遞事官十有七人。隸承政司。繪圖員、藝師、藝士各一人。隸軍制司。以上陸軍官佐或學生充之。法規總編纂員二人,編纂員三人。隸軍需司。以文武相當人員充之。監長、協參領、正軍校充。監副正、副軍校充。各一人。司法官十有四人,看守官三人。隸軍法司。以學律軍官充之。審計處處長,副官,各一人。科長二人,綜察、核銷科各一人。科員二十有八人。各十四人。各司處錄事百三十有六人。其暫設者:軍實司司長,副官,各一人。科長二人,制造、保儲科各一人。科員十人。制造四人。保儲六人。軍牧司司長,副官,各一人。科長二人,均調、蕃殖科各一人。科員十有二人。科各六人。軍學處處長,副官,各一人。科長六人,教育,步、馬、砲,工程,輜重隊,科各一人。科員三十有四人。教育十二人,步隊八人,馬隊、砲兵、工程隊各四人。輜重隊二人。普通編輯員三人。兵事編輯員六人。繪圖員一人。屬輜重隊。

大臣掌主陸軍,稽頒營制饟章,以鞏陸防。副大臣貳之。參事掌法律章制。檢察官掌察軍隊、學校、局廠。部副官掌傳宣命令。承政掌出納文移,旌別員司功過。軍制掌編制徵調,凡軍械制造,交通建築,並審驗法式。軍衡掌班秩、階品、大將軍、將軍正一品,以正都統有積勞者充之。正都統從一品,副都統正二品,協都統從二品,正參領正三品,副參領從三品,協參領正四品,正軍校正五品,副軍校正六品,協軍校正七品,司務長、技士長正八品,上士從八品,中士正九品,下士從九品。階十有四。等級、共三等九級:上等一級正都統職,任總統官,秩視提督。二級副都統職,任統制官,秩視總兵。三級協都統職,任統領官,秩視副將。中等一級正參領職,任統制官,正參謀官,工隊參領官,總軍械官,護軍官;同正參領職,任總軍需官,總理醫官,總執法官,秩視參將。二級副參領職,任教練官,一等參謀官,正軍械官,中軍官;同副參領職,任正軍需官,正軍醫官,正執法官,總馬醫官,一等書記官,秩視游擊。三級協參領職,任管帶官,二等參謀官,副軍械官,參軍官;同協參領職,任副軍需官,副軍醫官,正馬醫官,二等書記官,秩視都司。次等一級正軍校職,任督隊官,隊官,三等參謀官,查馬長,軍械長,執事官;同正軍校職,任軍需長,軍醫長,稽查官,軍樂隊官,副馬醫官,三等書記官,秩視守備。二級副軍校職,任排長,掌旗官;同副軍校職,任司事生,醫生,司號官,軍樂排長,馬醫長,書記長,秩視千總;同協軍校職,任司號長,醫生,司書生,秩視把總。封贈、襲廕,凡軍官、軍佐並領其籍。軍需掌糧饟廩餼,兼司軍需人員教育。軍醫掌防疫、治療,兼司軍醫升遷教育。軍法掌審判、監獄,勾檢軍事條約。軍實所掌,視舊武庫司。軍牧所掌,視舊太僕寺。軍學掌學校教育,隊伍操演。審計掌預算、決算,審覈支銷。所轄:憲政籌備處,銀庫,捷報處,馬館,俱派員分治其事。

光緒三十二年改兵部設,省並練兵處入之。舊置總理親王一人,會辦、襄辦、提調各一人。軍政、軍令、軍學三司正、副使各一人。自親王以下俱請簡。考功蒐討糧饟,醫務、法律、器械隸軍政,運籌、鄉導、測繪、儲材隸軍令,繙譯、訓練、教育、水師隸軍學。十四科監督各一人,俱由總理遴委。置尚書,左、右侍郎,各一人。設承政、參議兩,置左、右丞,參議,各一人。一、二、三等諮議官、檢察官,簡文武官賢能者充之。正、副從事官,副協參領充。無定員。設軍衡、軍乘、軍計、軍實、軍制、軍需、軍學、軍醫、軍法、軍牧十司,職置司長各一人,科長三十有三人,一、二、三等科員二百有五人,承發官十有二人,承政二人。餘各一人。軍法未置。譯員五人,繪圖員、藝師、藝士各二人,錄事百十有六人。官置郎中十有六人,員外郎十有八人,主事二十有二人,筆帖式百有十人。以上統為部額,不系以司。正參領八人,同正參領四人。副參領十有二人,同副參領六人。協參領十有八人,同協參領八人。額視郎中、員外郎、主事。正軍校十有八人,同正軍校八人。副軍校二十有四人,同副軍校十有二人。協軍校三十有二人,同協軍校十有六人。額視七、八、九品筆帖式。以官分任各職。三十三年,命慶親王奕劻領部事,非恆制。宣統元年,修正陸軍官制,軍官自正參領以下,軍佐自副都統以下,並就所習科目,冠以各隊如馬、步、砲、工、輜、警察各隊,正、副協參領,正、副協軍校,司務長,及上士、中士、下士之類。專門如軍需、軍醫、制械副協都統,正、副協都統,正、副協軍校,馬醫、測繪正、副協參領,正、副協軍校,軍樂協軍校,測繪、軍樂司務長,上、中、下各士,會計、調護上、中、下各士。名稱削同字。二年,改尚書為大臣,侍郎為副大臣。省左、右丞,參議,諮議,承發各官。並兩十司為八司。增承政一司,省軍乘、軍計、軍學三司。設軍學、審計二處。明年,定陸軍官佐補充制,置部副官調查員,以軍實司省入軍制,改軍牧司、軍學處為暫設,冀樹軍馬總監、軍學院基礎也。三年,復定陸軍官佐充任制,如前所列。仍與舊司員參錯互用。

海軍部海軍大臣,正都統。副大臣,副都統。各一人。一等參謀官二人,二等四人。海軍學生充。參事官二人。秘書官六人。資格相當軍官,文官充。司電員,藝師,藝士,酌用海軍官佐或文官學生。錄事,酌用文官學生及額外軍官、軍佐。無恆額。軍制、軍政、軍學、軍樞、軍儲、軍法、軍醫七司,各司長一人。協都統、正參領充。科長二十有一人,軍制科五:曰制度,曰考覈,曰銓衡,曰駕駛,曰輪機。軍政科三:曰制造,曰建築,曰器械。軍學科五:曰教育,曰訓練,曰謀略,曰偵測,曰編譯。軍樞科三:曰奏咨,曰典章,曰承發。軍儲科三:曰收支,曰儲備,曰庶務。軍醫科二:曰醫務,曰衛生。科各一人。正、副參領充。下同。一、二、三等科員六十人。軍制、軍學各十有四人。軍樞、軍儲各十人。軍政八人。軍醫四人。充任視陸軍部。一等司法官二人,二、三等司法官,學習司法官八人。學律軍官充。主計處計長一人。正參領充。科長二人。會計、統計科各一人。各司處錄事四十有八人。

大臣掌主海軍,稽覈水師及司令部,以固海疆。副大臣貳之。參謀掌參訂改革。參事掌法律章制。秘書掌機密文移。軍制掌規制銓法,旌別水師人員,功過、封廕、賞恤並典領之。軍政掌營造船艦,檢校器械,兼司軍港工程。軍學掌學校教育,艦隊訓練。軍樞掌文牘典章,匯紀員司集課文簿。軍儲掌經營費用,稽覈糧廩服裝與其物用。軍法、軍醫、主計職掌視陸軍部。

光緒十一年,詔設海軍衙門,依軍機總署例,命醇親奕枻綜之,大學士李鴻章專司籌辦。十三年,北洋海軍成,置提督、總兵等官。甲午師熸。至三十三年,始議恢復,設海軍處,暫隸陸軍部。置正使,視協都統。副使,視正參領。各一人。承發官二人,錄事四人。設機要、船政、運籌三司,置司長、副官各一人。科長七人,機要科四:曰制度,曰籌械,曰駕馭,曰輪機。運籌科三:曰謀略,曰教務,曰測海。科各一人。船政不分科。承發官三人,司各一人。一、二、三等科員十有八人。機要十二人,運籌六人。考工官五人,船政司置。藝師三人,船政一人,運籌二人。藝士四人。船政運籌各二人。股長、股員,視事閑劇酌置。錄事十有八人。明年,改設海政、船政、籌備、儲蓄、醫務、法務六司。尋設主計處,置計長、副長各一人。宣統元年,命肅親王善耆等籌備海軍,設參贊,分秘書、庶務兩司,置一、二、三等參謀官,並設第一、第二、第三、第四四司,置司長以下各職。其夏,更命貝勒載濤等充籌辦海軍大臣,增設醫務司。二年,訂海軍暫行官制,改第一司曰軍制,第二司曰軍政,第三司曰軍學,第四曰軍防,醫務司曰軍醫,秘書司曰軍樞,庶務司曰軍儲;別設軍法一司,是為八司。省參贊各職。尋改處為部,省軍防司,置大臣、副大臣各一人。

法部司法大臣,副大臣,各一人。左、右丞,參議,各一人。參事四人。審錄、制勘、編置、宥恤、舉敘、典獄、會計、都事八司,郎中二十有五人,審錄四人,內宗室一缺。餘各三人。員外郎三十有四人,制勘、編置各五人,內宗室各一缺。餘俱四人。主事三十有三人。宥恤五人,內宗室一缺。餘俱四人。收發所員外郎、主事各二人。七品小京官二十有六人。內宗室二缺。八品錄事五十有三人,九品三十人。內宗室各二缺。

大臣掌主法職,監督大理院及京、外審判、檢察,以維法治。副大臣貳之。審錄掌朝審錄囚,覆覈大理院、審判刑名。兼稽雲南、貴州、廣東、廣西、察哈爾左翼案狀。制勘掌秋錄實緩,定科刑禁。兼稽四川、河南、陜西、甘肅、新疆、烏里雅蘇臺、科布多案狀。編置掌盜犯減等,定地編發。兼稽奉天、吉林、黑龍江、山東、山西、察哈爾右翼、綏遠城、歸化城案狀。宥恤掌恩詔赦典,清理庶獄。兼稽江蘇、安徽、江西、福建、浙江、湖北、湖南案狀。舉敘掌升遷調補,籍紀功罪,徵考法官、律師、書記。典獄掌修葺囹圄,嚴固扃鑰,習藝所俘隸簿錄並典司之。會計掌財用出入,勾稽罰鍰鈞金。都事掌糸番譯章奏,收發罪囚文移。所轄:司獄總管守長、正管守長各二人,副管守長六人,監醫正、正八品。監醫副正九品。各一人。

光緒三十二年改刑部設,置尚書,侍郎,左、右丞、參以次各官。並十七司為八司。設收發所。置員外郎、主事各官。明年,增置宗室郎中、主事各一人;員外郎,小京官,八、九品錄事,各二人。裁司務入都事司,司庫入會計司。司獄一職,改令典獄司小京官兼充,曰正管守長;八、九品錄事兼充,曰副管守長。舊設提牢,以典獄司員外郎、主事兼充,曰總管守長。三十四年,依提牢司獄往制,仍定為兼職。尋置監醫正、醫副各一人。宣統三年,改尚書為大臣,侍郎為副大臣。

修訂法律館大臣,無定員。特簡兼任。提調二人。總纂四人,纂修、協修各六人。庶務處總辦一人。譯員、委員無恆額。並以諳法律人員充之。光緒三十三年設。

大理院,正卿,正二品。少卿,正三品。俱特簡。各一人。刑科、民科推丞各一人。正四品。請簡。推事二十有八人。正五品。刑科、民科第一庭俱各四人,第二、三庭俱各五人。典簿都典簿一人,從五品。典簿四人。從六品。主簿六人,正七品。以上俱奏補。八、九品錄事三十人。咨補。

正卿掌申枉理讞,解釋法律,監督各級審判,以一法權。少卿佐之。推丞分掌民、刑案款,參議疑獄。刑科掌被旨推鞫宗室官犯,披詳刑事京控上訴法狀。民科掌宗室諍訟,披詳民事京控上訴法狀。都典簿掌簿籍罪囚。典簿掌出納文移。大理於重罪為終審。凡法庭審判,推事五人會鞫之,是為合議制。附設總檢察,掌綜司大理民、刑案內檢察事務,監督各級檢察,調度司法警察官吏。丞一人,從三品。請簡。檢察官六人,正五品。奏補。主簿二人,八、九品錄事四人。看守所所長一人,從五品。奏補。所官四人,正八品。奏補。九品錄事二人。

光緒三十二年,改大理寺設,置正卿、少卿各一人,推丞二人。刑事四庭,推事十有九人。民事二庭,推事九人。並置典簿以次各官。又總檢察丞一人,檢察官六人,主簿一人,錄事四人。設看守所,置所長各官。宣統元年,改刑科四庭為民科三庭,置推事各十有四人。三年,增置總檢察典簿一人,改錄事為八、九品各二人。

京師高等審判,丞一人,正四品。請簡。掌治務,監督下級審判。下同。刑科、民科推事十有二人。從五品。刑科、民科一二庭俱各三人。典簿典簿二人,正七品。主簿四人,從七品。以上俱奏補。九品錄事六人。於重罪為二審,輕罪為終審。審判會鞫視大理。檢察檢察長一人,正四品。請簡。掌糾正同級審判,監督下級檢察。下同。檢察官四人,從五品。奏補。典簿、主簿各一人,九品錄事二人。看守所所長、正六品。奏補。所官從八品。咨補。各一人,錄事六人。

光緒三十三年設。宣統三年,增置檢察典簿、主簿各一人,並置所長各官。

京師地方審判,丞一人。從四品。請簡。刑科、民科推事三十人。從五品。民、刑一二庭俱各六人,三庭俱各三人。典簿二人,正七品。主簿二人,正八品。以上俱奏補。錄事十有四人。於重罪為初審,輕罪為二審。推事三人會鞫之,亦合議制。檢察檢察長一人,正五品。奏補。檢察官五人,正六品。奏補。典簿、從七品。主簿、從八品。錄事各二人。看守所所長一人,從六品。奏補。所官二人。

光緒三十三年設。先是京城內外設豫審,掌主諍訟,隸民政部。至是省入,置丞一人。設民、刑各二庭,置推事二十有四人。典簿、主簿各二人,錄事十人。檢察檢察長一人,檢察官四人,典簿、主簿各一人。宣統元年,以獄訟煩興,增設民、刑各一庭,置推事各三人,錄事四人。檢察檢察官一人。三年,增檢察典簿、主簿各一人。

京師初級審判,區為五處。刑科、民科推事各一人。從六品。奏補。錄事二人。於輕罪為初審,推事一人訊斷之,是為單獨制。檢察檢察官二人,從六品。奏補。錄事一人。初級俱不置長官,由部揀資深者一人為監督。

農工商部農工商大臣,副大臣,各一人。左、右丞,左、右參議,各一人。農務、工務、商務、庶務四司,郎中十有二人,司各三人。員外郎十有六人,司各四人。主事十有八人。庶務六人,餘各四人。一、二等藝師,一等正六品,二等正七品。奏補。藝士,一等正八品,二等正九品。咨補。各二人。

大臣掌主農工商政令,專司推演實業,以厚民生。副大臣貳之。農務掌農桑、屯墾,樹藝、畜牧並隸,通各省水利,匯覈支銷。工務掌綜事訓工,制器尚象,並物占各省礦產,設法利導。商務掌埠市治教,勵民同貨,修訂專利保險約章,稽頒保護訴訟禁令。庶務掌章奏文移,計會本部收支,籍紀員司遷補。藝師、藝士掌治專門職業。所轄:農事試驗場,工藝局,勸工陳列所,化分礦質所,度量權衡局,商標局,商律館,俱遴顓業者分治其事。

光緒二十四年,設礦務鐵路總局,尋復設農工商總局,令大臣綜之。尋省。二十九年,設商部,省鐵路礦務總局入之。置尚書,左、右侍郎,左、右丞,參議,各一人。司務所司務二。三十二年,更名農工商部,改平均司為農務,以戶部農桑等事隸之。通藝司為工務,以鐵道等事劃歸郵傳部。保惠司為商務。增置郎中各一人,員外郎、主事各二人。並司務會計司為庶務,省司務二人,增郎中一人,員外郎二人,主事四人。宣統三年,改尚書為大臣,侍郎為副大臣。

郵傳部郵傳大臣,副大臣,各一人。左、右丞,左、右參議,各一人。承政、參議兩僉事,正五品。奏補。員外郎,主事,小京官,各二人。船政、路政、電政、郵政四司,郎中各二人,員外郎十人,船政、郵政各二人。路政、電政各三人。主事二十人,船政、郵政各四人。路政、電政各六人。小京官各二人。八、九品錄事無定員。

大臣掌主交通政令,汽行舟車,電達文語,靡所不綜,以利民用。副大臣貳之。船政掌議船律,兼司營闢廠塢,測量沙線。路政掌議路律,兼司釐定軌制,規畫路線。電政掌議電律,兼司官商局則例,海陸線規程。郵政掌議郵律,兼司郵局匯兌,郵盟條約。所轄:郵政總局局長,副大臣兼充。總辦,法國人充。各一人。鐵路總局提調二人。京漢路局總辦、提調各一人,南局、京局會辦各一人。京奉路局總辦二人,提調各一人。京張鐵路總辦、會辦,各一人。水扈寧路局總辦一人。吉長路局、廣九路局,總辦、提調各一人。張綏鐵路總辦、會辦各一人。萍株鐵路、正太路局、汴洛路局、道清路局,總辦各一人。電政總局局長一人,提調二人。分局總辦、幫辦、提調各一人。各省分局總辦各一人。電話局總辦、會辦各一人。天津、廣州、太原、煙臺總辦各一人。交通銀行總理、幫理各一人。北京總銀行,上海、漢口、廣州分銀行,總辦各一人。天津、營口管理各一人。差官三十有四人。提塘官十有三人。舊隸兵部。俱遴員分治其事。

光緒三十三年設。先是船政招商局隸北洋大臣,內地商船隸工部,郵政隸總稅務司,路政、電政別簡大臣領其事,至是俱並入。置尚書,左、右侍郎,左、右丞,參議,各一人,及承政、參議兩僉事各官。設船政、路政、電政、郵政、庶務五司,置郎中十人,員外郎十有二人,主事二十有四人,小京官十有四人。宣統元年,省庶務司郎中、員外郎、小京官各二人,主事四人。增承政、參議兩員外郎、主事各二人。三年,改尚書為大臣,侍郎為副大臣。

軍諮府軍諮大臣二人,特簡。掌秉承詔命,翼贊軍謨。總務軍諮使二人,副協都統、正參領充。掌綜領眾務。副官二人。協參領,正、副軍校充。下同。遞事長一人,遞事員五人。陸軍官佐充。第一、第二、第三、第四各長,協都統、正參領充。副官,俱一人。條為四科,科長各一人。正、副參領充。一等科員,副協參領、同副協領充。二、三等科員,協參領,正、副軍校及同協參領,同正、副軍校充。視事閒劇酌置。所轄:測地局,局長一人,第四長兼充。司務三人。三角、地形、制圖三股,各股長一人。第四各科長兼充。班長,班員,印刷所科員,藝士,司務,無心互額。軍事官報局,正、副局長各一人。庶務,文牘,收支,編纂,譯述,校對,無恆額。俱隸第四。唯第五別置編纂官三人,譯述一人。錄事六十有三人。額外軍官及中士、下士充。軍事參議官十有五人。直隸、江寧、江蘇、江北、安徽、江西、河南、湖北、湖南、山東、山西、福建、廣東、浙江、陜西各一人。協都統,正、副協參領充。

光緒三十三年,設軍諮處,置協都統一人充正使,正參領一人充副使,副參領六人,同副參領一人,協參領十人,同協參領二人,正軍校十人,同正軍校二人,副軍校十有二人,同副軍校三人,協軍校十有六人,同協軍校五人,分充各司長、科長,一、二、三等科員。第一、第二兩司,協、副參領充。測地司,同正、副參領充。十八科科長各一人,一、二司副參領充。測地司,同副、協參領充。第一司科員十有六人,第二司四人,正、副協軍校充。測地司六人,同正、副協軍校充。其承發官司各一人,譯員五人,屬第一司。藝師四人,藝士六人,屬測地司。並以陸軍官佐或學生充之。隸陸軍部。宣統元年,以立憲大綱皇帝統率海陸軍,別建軍諮處,命貝勒載濤等領之。設總務,置軍諮使二人。分設四,各置長一人,科長十有六人,科員無恆額。並定文官補充制。如前所列。尋削同字。詳上陸軍部。明年,設軍事會議處。三年,改稱府,令陸軍大臣領其事。

弼德院院長,副院長,各一人。特簡兼任。顧問大臣三十有二人。特簡兼任。掌參預密勿,朝夕論思,並審議洪疑大政。參議十人,請簡。掌纂擬章制。秘書秘書長一人,請簡。秘書官一、二等各三人,三等六人,俱奏補。分掌庶務。宣統三年設。

資政院總裁,王、公、大臣內特簡。副總裁,三品以上大臣內簡充。各一人。掌取決公論。凡歲入歲出,法典朝章,公債稅率,及被旨諮議者,經議員議決,會國務大臣上奏取裁。秘書秘書長一人,請簡。一、二、三等秘書官各四人,奏補。掌計會文牘。議員,宗室王、公世爵十有六人,滿、漢世爵十有二人,外籓王、公世爵十有四人,宗室、覺羅六人,各部院官三十有二人,碩學通儒納稅多額者各十人,俱欽選。各省諮議局六人。民選。

光緒三十三年設,置總裁二人。尋增協理四人。明年,復置幫辦、參議各三人。宣統元年,定秘書官制。二年,定總裁、副總裁各一人。

鹽政院鹽政大臣國務大臣內特簡兼任。一人。丞一人。總務長,參議,南鹽長,北鹽長,各一人。以上俱請簡。參事二人,一、二、三、四等僉事,俱奏補。一、二、三等錄事,咨補。視事閒劇酌置。

大臣掌主鹽政。丞掌佐理鹺綱。總務掌綜理庶務,典守機密。參議掌擬法制,僉事佐之。南鹽掌淮、浙、閩、粵鹽務,北鹽掌奉、直、潞、東鹽務。初沿明制,差御史巡視鹽課。後改鹽政。特旨兼充。都察院奏差者,亦以鹽政名之。由內務司官充者,仍帶御史銜。各省以督、撫綜理者,並因地制宜,定為永式。宣統元年,設督辦鹽政處,命鎮國公載澤充督辦大臣,產鹽行鹽各省督、撫俱充會辦。三年,以整理國稅,改處為院,特置鹽政專官。

典禮院掌院學士,副掌院學士,各一人。特簡。學士,直學士,各八人。請簡。總務長一人。簿正、典簿、司庫,俱奏補。無定員。禮制、祠祭、奉常、精膳四署署長各一人。一、二、三等僉事,鳴贊,俱奏補。序班,錄事,咨補。視事閒劇酌置。讀祝官、贊禮郎、陵寢各官如故。

掌院學士掌修明禮樂,典領朝會,虔肅明禋。副掌院學士佐之。學士、直學士掌討論參訂。總務掌綜理眾務。簿正掌庫儲收發,與其陳設,並司監牢事。典簿掌典守庫儲冊籍,兼稽覈出入。司庫掌典守各庫,並督率庫使,點驗庫兵。禮制掌朝會慶典。祠祭掌壇廟陵寢。奉常掌贊引儐導。精膳掌筵燕祭品。宣統三年改禮部設。凡涉行政,俱劃歸各部。

外省官制,變更略少,唯省會、司道別易新名,巡警、勸業兩道詳前。員額愈益。改學政為提學使。按察使為提法使,各級審檢隸之。故事,凡遇地方要政,籓、臬兩司得與督若撫議,議定稟仰施行,遇吏員升遷調補,亦會詳焉。至是,改稱為三司云。

提學使司提學使一人,正三品。掌教育行政,稽覈學校規程,徵考藝文師範。署設六科:曰總務,曰專門,曰普通,曰實業,曰圖書,曰會計。科長、科員分治之。遴諳學務者充之。別設學務公所,有議長、議紳以討論其事。奏充。光緒三十一年改置。增吉林、黑龍江、江寧、江蘇、舊置江南學政。新疆各一人,餘仍學政額。

提法使司提法使一人,正三品。掌司法行政,督監各級審判,調度檢察事務。署設三科:曰總務,曰民刑,曰典獄。科長各一人,正五品。一等科員各一人,正六品。二等科員正七品。無恆額。惟奉天置僉事科員。別有正司書,正八品;副司書,正九品。光緒三十三年,東三省各置提法使一人。宣統二年,改各省按察使為提法使,停轄驛傳。

高等審判,丞一人。從四品。商埠分,推事長代之。刑科、民科推事六人。正六品。典簿一人。正七品。主簿二人,正八品。錄事無定員。從九品。檢察檢察長一人,從四品。檢察官一人,正六品。錄事二人。

地方審判,推事長一人。從五品。刑科、民科推事六人。從六品。典簿、從七品。主簿從八品。事繁或二人,事簡不置。各一人,錄事無定員。檢察檢察長一人,從五品。檢察官一人,從六品。錄事二人。看守所所官一人,正九品。錄事無定員。

初級審判,推事二人。正七品。事繁或三、四人。錄事無定員。檢察檢察官一人,正七品。錄事二人。看守所所官一人。

管獄官一人,從五品。副管獄官一人。從六品。課長三人。正八品。文牘、守衛、庶務各一人。所長二人。正九品。教誨、醫務各一人。府管獄官一人。從七品。州、縣副管獄官一人。從八品。光緒三十四年,奉天設模範監獄,置正管獄官,省府司獄、縣典史。宣統二年,增置副管獄官。厥後各府、、州、縣有仿而行之者。時天津、保定、湖北監獄成,未置專官。

東三省地處邊要,自改建行省,變通例章,增置司道。提學、提法,各省通置,無庸贅述。今綜新設諸司詳左。初建行省,督署設承宣、諮議二,置左、右參贊各一人,從二品。僉事一人,一、二、三等科員佐之。旋省。

民政使司民政使一人,從二品。掌主民籍。僉事,從四品。科員,一等從五品,二等正六品,三等正七品。各司同。各有恆任。一、二等醫官無定員。一等正六品,二等正七品。光緒三十三年置,秩正三品。宣統元年,依布政使例,升從二品,主屬吏升遷調補。

交涉使司交涉使一人,正三品。掌主邦交。有僉事,科員,一、二等譯官佐之。一等正六品,二等正七品。光緒三十三年,奉天、吉林各置一人。宣統二年,直隸、江蘇、浙江、福建、湖北、廣東、雲南,並援奉天例續置。

度支使司度支使一人,正三品。掌主財賦。有僉事,科員,一、二等庫官佐之。一等正六品。二等正七品。光緒三十三年,三省各置一人。宣統元年,省黑龍江一人,隸民政司兼理。又光緒三十三年,奉天置旗務使司一人,僉事、科員如各司。宣統元年省。

甲午不競,當事者鑒於軍政未善,取則強邦,內自禁衛軍,外自督練公所,並遵新定章制,以漸從事。乃三軍、兩協方告成,而巨變作焉。爰就可考者著於篇。

禁衛軍訓練大臣三人,王大臣兼充。掌全軍政令。軍諮官六人,執事員十人,掌章奏文移,兼稽四科。協、標、營、隊執事佐之。書記員五人,一等一人,二、三等各二人。繪圖員二人,印刷、收支、庶務、遞事各一人。軍械、軍法、軍需、軍醫四科監督各一人,科員十有五人,軍械四人。軍需五人。餘各二人。俱遴員分治其事。協司令處統領官一人,協都統充。掌統帥全協。參軍官協參領充。掌贊畫機宜,副官正軍校充。掌綜理眾務,各一人。司號長一人。協軍校充。司書生二人。同上。標本署統帶官一人,正參領充。掌統轄全標。教練官,副參領充。副官,掌旗官,俱副軍校充。副軍械官,副軍需官,副軍醫官,俱協參領充。副馬醫官,正軍校充。司號長,各一人。司書生二人。步、馬、工程、輜重、交通、陸路砲、機關砲、警察各隊管帶官,協參領充。副官,軍需長,軍醫長,俱正軍校充。俱各一人。隊官正軍校充。俱各四人。排長俱各三人。副軍校一人,協軍校二人。原置步隊、機關砲隊各十有二人,馬隊八人,陸路砲隊九人,工程、輜重、交通、警察隊各六人。宣統三年改定如今制。司務長七十有九人。馬、步、機關砲隊各四人,陸路砲隊三人,工程、輜重、交通隊各二人。初以協軍校充。宣統元年改札補。軍械長四人。正軍校充。工程、交通、陸路砲、機關砲隊各一人。查馬長,正軍校充。馬醫長,副軍校充。各三人。司書生三十有五人。馬、步、機關砲隊六人,陸路砲隊五人,工程、輜重、交通隊各四人。藝師三人。隸交通隊。軍樂隊官,排長,各一人。

光緒三十四年,設禁衛軍,監國攝政王自領之,以貝勒載濤等司訓練。宣統元年,定訓練大臣三人,及軍諮官以次員額。先是各協、標、營置執事督隊諸官,至是俱改為副官,省協、標二等書記官及全協書記長。

督練公所督辦一人,督、撫、將軍、都統領之。掌整飭全省新舊營伍。軍事參議官一人,協都統、正參領充。掌綜領科、局。一等副官一人,協參領充。二等副官二人,正軍校充。分掌文移眾務。一、二、三等書記官五人,五、六、七品文官充。司書生十有六人。八、九品文官充。籌備、糧饟二科,科長各一人,分掌編練新軍,裁汰舊營,會計出納,服裝物品。軍械局局長一人,掌新舊軍槍砲彈藥。以上俱副軍校充。一等科員五人,籌備、糧饟科各二人。軍械一人。協參領充。二等十有一人,籌備四人。糧饟五人。軍械二人。正軍校充。三等十有二人。籌備五人。糧饟四人。軍械三人。協軍校充。測地分局,員闕。

光緒三十年,各省設督練公所,分兵備、參謀、教練三處,置總辦、幫辦、提調諸目。宣統三年,改設科、局,仿陸軍新制,任官授職。如前所列。

軍制總統一人,正都統充。掌全軍政令。總參謀官,協都統充。一等參謀官,正參領充。二等參謀官,協參領充。掌協贊號令,參畫機宜。一、二等各員佐之。工程隊參領官,掌佐本隊事務。護軍官掌理庶務,轄弁兵。砲隊協領官職掌如工程隊。總軍械官,總執法官,總軍需官,總軍醫官,詳禁衛軍。自工程隊以下,俱正參領充。總馬醫官,副參領充。俱各一人。司書生十有五人。副協軍校充。初,軍、鎮、協、標並置司事,後省。

鎮制統制官一人,副都統充。掌統帥全鎮。正參謀官,正參領充。二、三等參謀官,所司同軍制。執事官,俱正軍校充。中軍官,副協參領充。掌理庶務。正軍械官,正執法官,正軍需官,正軍醫官,俱副協參領充。正馬醫官,協參領充。司號長,副軍校充。俱各一人。司書生十有五人。其協、標、營制如禁衛軍。

光緒三十年,改練新軍,區為三十六鎮,定鎮、協、標、營官制。宣統元年,各省先後編混成等協,暫置執法官、司事生各一人,尋省。三年,報成鎮者二十有六,置總統一人。總參謀以下員闕。餘或成二協,或一協一標,鎮數未全。

陸軍鎮監,監長,協參領、正軍校充。監副,正、副軍校充。各一人。司書生二人。光緒三十四年,定監獄人員編制。

巡防隊分路統領官,事簡緩置。幫統官,書記官,會計官,執事官,各一人。馬、步隊管帶官一人。哨官、哨長各三人。書記長一人。以上各員,俱綠營將弁兼充。光緒三十三年,以防練舊營雜項隊伍章制不一,仿新軍成法,置統領以次各職。

海軍艦制巡洋長江艦隊統制一人。副都統加正都統銜。統領二人。協都統。海圻巡洋艦管帶,總管輪,正參領。一等參謀官,海籌、海琛、海容巡洋艦,南琛、鏡清、通濟練船,江元、江利、楚同、楚泰、楚有、江員砲船,保民運船諸管帶,副參領。飛鷹魚雷獵船,建威、建安魚雷砲船,江亨、楚謙、楚豫、聯鯨、楚觀、舞鳳砲船諸管帶,協參領。駐英威克斯阿摩士莊各船廠監造員,正參領。俱各一人。餘皆未補官。

同治十三年,朝議防海,購置兵輪都二十艘。光緒十年,法兵構釁,盡殲焉。越三年,編海軍經制,分為四軍,置提督一人為左翼,總兵二人為右翼,並置副將五人,參將四人,游擊九人,都司二十有七人,守備六十人,千總六十有五人,把總九十有九人,至是又復成軍。甲午一役又殲焉。宣統元年,設籌備處,復置海軍提督,仿陸軍等級,訂海軍官制。三年部成,先後除授如上制。


\end{pinyinscope}