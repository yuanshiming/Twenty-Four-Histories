\article{列傳一百八十一}

\begin{pinyinscope}
徐廣縉葉名琛黃宗漢

徐廣縉,字仲升,河南鹿邑人。嘉慶二十五年進士,選庶吉士,授編修,遷御史。道光十三年,出為陜西榆林知府,歷安徽徽寧池太道、江西督糧道、福建按察使。擢順天府尹,尋出為四川布政使。丁母憂,服闋,補江寧布政使。二十六年,擢雲南巡撫,調廣東。二十八年,擢兩廣總督,兼通商大臣。

自江寧定約五口通商,許廣州省城設立棧房,領事入城,以平禮相待。粵民堅執洋人不準入城舊制,聚眾以抗,官不能解。總督耆英既與英人議緩俟二年之後,尋內召,廣縉繼任。會黃竹岐鄉民毆殺英人六,領事德庇時要挾賠償保護,廣縉治殺人者罪,而拒其非理之求,戒諭人民毋暴動,事得解。德庇時回國,文翰代為領事,初至請謁。廣縉赴虎門閱砲臺,延見之,遂登其舟,示以坦白。二十九年,文翰以兩年入城之期已屆,要踐約,廣縉諭以耆英所許。乃姑為權宜之計,民情憤激,眾怒難犯,非官所能禁止。文翰則堅持成約,且以他省入城相詰難,揚言將駕兵船至天津訴諸京師,相持不下。

廣縉疏聞,自請嚴議。密詔許暫入城一次,以踐前言,不得習以為常。廣縉復疏言:「入城萬不可行。廣東民情剽悍,與閩、浙、江蘇不同。阻其入城而有事,則眾志成城,尚有爪牙之可恃;許其入城而有事,則人心瓦解,必至內外之交訌。明知有害無利,詎敢輕於一試。」卒堅拒之。英人乃集兵船三於香港,放小艇至海口各港測水探路,示恫喝。廣縉增兵守諸砲臺及要隘,嚴備以待。時民團號十萬,聲勢甚張。華商會議暫停各國貿易,密告美、法兩國領事,啟釁實由英人。於是諸洋商慮受擾累,將以損失歸領事負責。士紳聯名致文翰,為反覆陳利害甚切。文翰內受牽制,乃罷入城之議,乞照舊通商。與要約,停市開巿皆非由官令,不進城即通商,後有反覆,仍行停止。事既定,廣縉疏聞,宣宗大悅。詔曰:「洋務之興,將十年矣。沿海擾累,糜餉勞師,近雖略臻安謐,而馭之之法,剛柔未得其平,流弊因而愈出。朕恐瀕海居民或遭蹂躪,一切隱忍待之。昨英酋復伸入城之請,徐廣縉等悉心措理,動合機宜。入城議寢,依舊通商。不折一兵,不發一矢,中外綏靖,可以久安,實深嘉悅!」於是錫封廣縉一等子爵,賜雙眼花翎,是役商民一心,尤得紳士許祥光、伍崇曜之力為多,二人並被優擢。逾數月,文翰復言國王以進城未能如約,為人所輕,似覺赧顏,請為轉奏,廣縉以罷議進城之後貿易始復,豈可再申前說,拒之。三十年,文翰又遺書大學士穆彰阿、耆英,遣人至上海、天津投遞。文翰尋自赴上海,欲有所陳請,先後卻之;乃回香港,蓋覬覦未已也。

時兩廣盜賊蜂起,以廣西金田洪秀全為最悍。巡撫鄭祖琛柔懦縱賊,廣縉疏劾其養筴貽患,罷之。廣東韶州、廉州匪亦蔓延,廣縉遣軍扼梧州、肇慶。詔廣縉赴廣西剿辦,尋起林則徐督師,命廣縉剿捕廣東游匪。咸豐元年,出駐高州。匪首凌十八、陳二、吳三、何茗科踞羅鏡圩及信宜,與洪秀全聲勢相倚。廣縉遣兵進擊,殲吳三,追何茗科至貴縣擒之;又破廉州賊顏品瑤,擒李士青。二年春,乘勝進攻羅鏡圩,擒凌十八。捷聞,加太子太保。命馳赴梧州,而洪秀全大股已犯桂林,竄入湖南。賽尚阿以罪黜,授廣縉欽差大臣,署理湖廣總督。十月,至衡州,賊攻長沙甚急,駱秉章、張亮基力守,屢挫賊,乃下竄岳州。廣縉始抵長沙。未幾。岳州亦陷,直犯武昌。廣縉進駐岳州,而漢陽、武昌相繼陷。

詔斥廣縉遷延不進,調度失機,株守岳州,擁兵自眾,褫職逮問,籍其家,論大闢。三年夏,粵匪入河南境,釋廣縉,交巡撫陸應穀差遣,責令帶罪自效。率兵駐歸德,防剿捻匪有功。八年,命赴勝保軍營,尋予四品卿銜,留鳳陽從袁甲三剿捻匪。未幾,卒。

葉名琛,字昆臣,湖北漢陽人。道光十五年進士,選庶吉土,授編修。十八年,出為陜西興安知府。歷山西雁平道、江西鹽道、雲南按察使,湖南、甘肅、廣東布政使。二十八年,擢廣東巡撫。二十九年,英人欲踐入城之約,名琛偕總督徐廣縉堅執勿許,聯合民團,嚴為戒備。華商自停貿易以制之,英人始寢前議。論功,封一等男爵,賜花翎。三十年,平英德土匪,被優敘。咸豐元年,殲羅鏡會匪吳三,加太子少保。二年,廣縉赴廣西督師,命名琛接辦羅鏡剿捕事宜,出駐高州。是年秋,羅鏡匪首凌十八就殲,加總督銜,署總督,赴南、韶一帶督剿。尋實授兩廣總督,兼通商大臣。

時廣東盜賊蜂起,四年,廣州群匪擾及省城,遣將分路進剿,連戰皆捷。近省之佛山、龍門、從化、東莞、陽山、河源、增城、封川,韶州之海豐、開建,潮州之惠來,肇慶府城及德慶並陷,先後克復。鄰省軍務方亟,糧餉器械多賴廣東接濟,名琛籌供無缺,益得時譽。五年,以總督協辦大學士。六年,拜體仁閣大學士,仍留總督任。

名琛性木彊,勤吏事,屬僚憚其威重。初以偕徐廣縉拒英人入城被殊眷,因狃於前事,頗自負,好大言,遇中外交涉事,略書數字答之,或竟不答。會匪之逼廣州,或議借外國兵禦賊者,斥之退。匪既平,按察使沈棣輝功最多,列上官紳兵練出力者請獎,格不奏,兵練皆解體。又嚴治通匪餘黨,或藉捕匪仇殺,從賊逃不敢歸,其黠者投香港,勸英人攻廣州。會水師千總巡河,遇劃艇張英國旗,搜獲十三人,拔其旗。英領事巴夏禮索之不得,貽書名琛責問,謂捕匪當移取,不當擅執,毀旗尤非禮。名琛令送十三人於領事,不受,必欲並索千總,遂置之。未幾,遣通事來告:「越日日中不如約,即攻城。」至期,英兵果奪獵德、中流砲臺。名琛曰:「彼當自走。」令水師勿與戰,於是鳳皇山、海珠諸砲臺皆被踞,發砲擊省城,十月朔,毀城,既入復出。遣廣州知府往詰用兵之故,英人曰:「兩國官不晤,情不親。誤聽傳言,屢乖和好。請入城面議。」名琛勿許。請於城外會議,亦不許。兵練數萬來援,怵敵火器,不能力戰。民憤甚,焚英、法、美三國居室,凡昔十三行皆燼。英兵亦焚民居數千家,退泊大黃,各報其國。

英遣額羅金來粵,聚兵澳門、香港,貽書索償款。名琛以其言狂悖,不答。法、美兩國領事亦索賠償,且告英兵已決計攻城,原居間排解。名琛慮其合以脅我,亦不聽;且不設備。七年,英兵攻東莞,總兵董開慶與戰,軍潰。額羅金遣艇遞照會,名琛答以通商而外,概不能從。累疏言:「英國主厭兵,粵事皆額羅金等所為。臣始終堅持,彼窮當自伏。」密詔戒勿輕視,猶信其事有把握,仍褒勉之。九月,英兵驟至,法、美兵皆從。將軍司道商戰守,名琛惟恃通事張云同為內應,待敵窮蹙。民間見其夷然不驚,事皆秘不宣示,轉疑其陽拒陰撫,人心益渙。十一月,敵張榜城外,限二十四時破城,勸商民遷避。砲擊總督署,延燒市廛,城遂陷。巡撫柏貴檄紳士伍崇曜等議和,名琛猶持不許入城之議,夜避左都統署,英人大索得之,舁登舟。將軍、巡撫以聞,詔斥名琛剛愎自用,辦理乖謬,褫其職,英人遂踞省城,禁巡撫等官不得出,責以安民。民各集團練,設總局於佛山,相持數年。各國聯師赴天津,事乃益棘矣。

名琛既被虜,英人挾至印度孟加拉,居之鎮海樓上。猶時作書畫,自署曰「海上蘇武」,賦詩見志,日誦呂祖經不輟。九年,卒,乃歸其尸。粵人憾其誤國,為之語曰:「不戰、不和、不守,不死、不降、不走;相臣度量,疆臣抱負;古之所無,今之罕有。」

黃宗漢,字壽臣,福建晉江人。道光十五年進士,選庶吉士。散館改兵部主事,充軍機章京。歷員外郎、郎中,遷御史、給事中。二十五年,出為廣東督糧道,調雷瓊道,歷山東、浙江按察使。咸豐初,巡撫吳文鎔薦宗漢可重用,遷甘肅布政使。二年,擢雲南巡撫,未之任,調浙江。值試辦海運,湖郡漕船淺滯,改留變價,虧銀三十餘萬兩,布政使椿壽情急自縊。宗漢疏請原米隨新漕運京,允之。

三年,粵匪犯江寧,調浙江兵二千名赴援。江寧尋陷,宗漢赴嘉興、湖州籌防,疏言不可僅於本境畫疆而守。於是分兵赴江蘇、安徽境內協防,詔嘉其妥協。尋上海匪起陷城,請海運改於劉河受兌。時江南大營需餉甚鉅,宗漢貽書向榮,通盤籌算,請於江蘇、浙江、江西三省確定每月額數。榮據以上聞,文宗韙之。四年,特詔褒宗漢辦理防務、海運,及本境治匪、察吏,精詳無瞻顧,深堪嘉尚,特賜御書「忠勤正直」扁額,勉其慎終如始,以成一代良臣。

擢四川總督。給事中張修育疏言:「宗漢治浙,布置合宜,未可更易。」詔不允。會因數月未奏事,降旨詢問,以疾為言,詔斥之,議降三級調用,加恩降二品頂戴,仍留總督任。五年,馬邊夷匪為亂,平之。遵旨遣松潘鎮總兵德恩以兵二千援荊州,又調兵四千赴貴州剿苗,並協餉十萬兩。六年,復因久無奏報,命將軍樂斌查奏,以痰疾聞,下部議降調,命來京另候簡用。補內閣學士,兼署刑部侍郎、順天府尹。

廣東軍事起,葉名琛被擄,授宗漢兩廣總督,兼通商大臣。時廣州為英人所踞,巡撫柏貴在城中為所脅制。民團四起,文宗因徐廣縉等前拒英人入城,賴紳民之力,欲復用之,命在籍侍郎羅惇衍、京卿龍元僖、給事中蘇廷魁治團練。惇衍等號召鄉團,得數萬人,戒期攻城,卒無功;又禁華人不得受雇為洋人服役以困之。

八年春,各國遣人赴江蘇投書致京師大學士訴粵事,請遣大臣至上海會議;且言逾期即赴天津。詔仍回廣東候宗漢查辦,而英、俄兩國兵船已泊吳淞。宗漢過江蘇,總督何桂清堅留在上海開議,宗漢不可,遽去;取道浙、閩,調兵不可得。及至廣東,敵兵已犯天津。宗漢駐惠州,惟恃聯絡民團,出示空言激勵,為英人所禁格,不能遍及。既而天津和約成,俟償款六百萬兩分年交畢,始退出廣州,粵民愈憤。英領事宣布和議,新安鎮鄉勇殺其張示者數人,遂發兵陷新安。民團大舉攻城,初勝終挫,懸賞格購洋官首,亦僅時伺隱僻,有所殺傷而已。宗漢外怵強敵,內畏民嵒,不能有所措施。泊大學士桂良等至上海議稅則及換約事宜,將與商交還廣州,向宗漢詢近狀,輒不答。而英人以既議和,民團復相仇殺,來相詰問,且揭團紳告示載諭旨有異,必欲去宗漢及三團紳。桂良等疏聞,詔責宗漢捕偽造諭旨之人,罷其通商大臣,改授何桂清。英使額羅金猶不愜,遽率監赴廣東。九年,遂復有天津之役。

尋調宗漢四川總督,召至京,改以侍郎候補。十年,署吏部侍郎,尋實授。四川京官呈請飭赴四川督辦團練,不許。

宗漢與載垣、端華、肅順等交結。十一年,穆宗即位,載垣等獲罪。少詹事許彭壽疏劾宗漢與陳孚恩、劉昆並黨肅順等,蹤跡最密。詔曰:「黃宗漢本年春赴熱河,危詞力阻回鑾。迨皇考梓宮將回京,又以京城可慮,遍告於人,希冀阻止。其意存迎合載垣等,眾所共知。聲名品行如此,若任其濫廁卿貳,何以表率屬僚?革職永不敘用,以為大僚輭媚者戒。」並追奪前賜御書「忠勤正直」扁額。同治三年,卒。

論曰:當道、咸之間,海禁大開,然昧於外情,朝野一也。粵民身創夷患之深,目擊國威之墮,憤懼交乘,遂因拒入城一事,釀成大釁。朝廷誤信民氣可用,而不知虛聲之不足恃也。徐廣縉操縱有術,幸安一時;葉名琛狃於前事,驕愎致敗,宜哉。黃宗漢依違貽誤,終以依附權要被譴。廣縉在粵東剿平羅鏡匪有功,及代賽尚阿督師,軍事已壞,旁皇失措,咎無可辭焉。


\end{pinyinscope}