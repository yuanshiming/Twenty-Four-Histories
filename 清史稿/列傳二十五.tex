\article{列傳二十五}

\begin{pinyinscope}
蔣赫德額色赫車克覺羅巴哈納宋權傅以漸呂宮

成克鞏金之俊謝升胡世安王永吉黨崇雅衛周祚高爾儼

張端

蔣赫德,初名元恆,遵化人。天聰三年,太宗伐明,克遵化,選儒生俊秀者入文館,元恆與焉,賜名赫德。崇德元年,授秘書院副理事官,予四戶。漢軍旗制定,隸鑲白旗。

順治二年,擢國史館學士。九年,朝鮮國王李淏奏國內外奸徒謀不軌,巳伏其辜,命與侍郎伊勒都齎敕往慰問。十一年,擢國史院大學士。十二年,詔諸大臣陳時務,疏言:「察吏乃可安民,除害乃可興利。今百姓大害,莫甚於貪官蠹吏。懲治之法,惟恃督撫糾劾,以其確知屬吏之賢不肖也。近每見各督撫彈章,指事列款,贓跡累累;及奉旨勘讞,計贓科罪,不及十之二三。不曰『事屬子虛』,則曰『衙役作弊』。即坐衙役者,又多引雜犯律例,聽其贖免,何所懲憚而不肆行其志乎?其始官胥朋比,虐取瓜分;事敗,官嫁名於吏以覬燃灰,吏假貲於官以成展脫。究之官吏優游,兩獲無恙,糾劾雖行,竟成故事。請嚴飭各督撫,糾劾勘讞覆奏時,必全述原參疏語,某款不實,或開報虛構,或承問故縱,窮源質訊,是非不容並立;實系衙役詐騙,按律坐以應得之罪,不許折贖,則貪蠹清而民蘇矣。」得旨,下所司嚴飭行。旋加太子太保。

十五年,改文華殿大學士,兼禮部尚書。十六年,加少保。命齎冊封朝鮮國王李■H7,侍讀碩博輝副之。蔣赫德屢充殿試讀卷官、教習庶吉士。修輯明史、太宗實錄,充副總裁;太祖、太宗聖訓充總裁。譯三國志成,賜鞍馬。十七年,引疾乞休。康熙元年,起為弘文院大學士。二年,調國史院。九年,卒,謚文端。

蔣赫德初為明諸生,嘗應鄉試,夜聞明遠樓鼓聲,曰:「此頹敗之氣,國安能久?」不終試而去。遍游九邊,曰:「王氣在遼、沈,將有聖人出,吾蓄才以待可也。」旋為太宗賞拔,卒致通顯。

額色赫,富察氏,滿洲鑲白旗人,世居訥殷。祖莽吉圖,當太祖時,從其兄孟古慎郭和來歸。

額色赫事太宗,從征伐,自巴牙喇壯達授兵部理事官。天聰九年,從梅勒額真巴奇蘭伐黑龍江部,使還奏捷。崇德三年,擢秘書院學士。五年,睿親王多爾袞率師圍錦州,命額色赫賚敕諭機宜。會固山額真圖爾格敗明兵於木輪河,使還奏捷。六年,命與圖爾格及大學士範文程、剛林如錦州,按諸將離城遠駐,遣兵還家,睿親王以下坐降罰有差。明總督洪承疇以援師至,上又命額色赫詣軍前授諸將方略,還奏敵勢甚張,當益兵。上遂自將擊破明軍。既克錦州,又命宣諭慰撫祖大壽及同降諸將士。八年,從貝勒阿巴泰伐明,略山東,下兗州,同甲喇額真穆成格等奏捷。

順治元年,從入關,授世職牛錄章京,加半個前程。五年,遷刑部啟心郎。八年,擢國史院大學士,世職累進一等阿達哈哈番。十三年,命往朝鮮讞獄。十五年,改保和殿大學士。額色赫再主會試,修太宗實錄,輯太祖、太宗聖訓,纂資政要覽,並充總裁官,累加少師兼太子太師。十八年,卒,謚文恪。

車克,瓜爾佳氏,滿洲鑲白旗人,世居蘇完。祖克爾素,太祖時來歸。父席爾那,任牛錄額真,卒,車克嗣,兼巴牙喇轄。

天聰八年,從上伐明,自大同趨懷遠,薄左衛城,與巴牙喇纛章京圖魯什等設伏,敗明將曹文詔騎兵。略代州,至五臺山,還,遇明將祖大弼兵,擊敗之。崇德三年,授戶部副理事官。承政韓大勛私取庫金,事發,車克坐貯庫時未記檔,論死,命罰鍰以贖,仍留部。尋兼任甲喇額真。五年,從鄭親王濟爾哈朗圍錦州,令車克與噶布什賢噶喇依昂邦勞薩以三百人伏高橋北,坐縱敵,藉家財之半。六年,復從攻錦州,擊破明總督洪承疇步兵。

順治元年,從入關,擊李自成,授世職牛錄章京。考績,加半個前程。五年,擢戶部侍郎。從英親王阿濟格討姜瓖,師下大同,令車克援太原,與巡撫祝世昌謀,遣兵殲瓖將劉遷、萬鍊等。七年,兼任正白旗滿洲梅勒額真。世職累進二等阿達哈哈番。八年,改都察院參政。駐防河間,佐領碩爾對訐戶部給餉不均,事具巴哈納傳。車克亦坐降世職拖沙喇哈番。旋擢戶部尚書。十年,復世職。十一年,加太子太保。十二年,擢秘書院大學士,進少保。十三年,復進少傅兼太子太傅,領戶部尚書。十四年,考滿,加少師兼太子太師。十六年,命赴江南督造戰艦。十七年,命與安南將軍宗室羅託率師駐福建,防鄭成功。

聖祖即位,召還,調吏部尚書。有阿那庫者,與兄金布爭產,上命均分之。既,又與本旗佐領吉詹爭言,吉詹坐阿那庫違上旨。牒戶部,車克移刑部,坐阿那庫罪絞;阿那庫妻擊登聞鼓訟冤,命覆勘,車克當奪官,命削加銜。康熙元年,復授秘書院大學士。六年,以疾乞休。十年,卒,謚文端。

覺羅巴哈納,滿洲鑲白旗人,景祖第三兄索長阿四世孫也。年十七從軍,佐太宗征伐有功。天聰八年,授世職牛錄章京。九年,命免功臣徭役,分設牛錄,巴哈納與焉。崇德三年,授刑部理事官。四年,擢參政,兼正藍旗滿洲梅勒額真。七年,以刑部勘將佐功罪失平,奪世職。

順治元年,擢正藍旗滿洲固山額真。與固山額真石廷柱徇霸州、滄州、德州、臨清,皆下。移師山西,會固山額真葉臣,招降明總督李化熙等。師自汾州趨平陽,與廷柱擊破明兵,至黑龍關,降裨將三、卒六千餘,賚白金,進世職三等甲喇章京。三年,從肅親王豪格下四川,討張獻忠,分兵定遵義、夔州、茂州,斬所置吏數百,降卒數千,盡得其馬騾輜重。餘寇悉平。師還,以勘甲喇章京希爾根軍功失實,又肅親王欲以機賽為巴牙喇纛章京不當,巴哈納與索渾未阻止,且共為奏,議奪官,命降世職拜他喇布勒哈番。尋擢戶部尚書。

八年,世祖親政,巴哈納奏事畢,上問民間疾苦及國家無益之費,巴哈納舉臨清採磚及通州五閘運漕二事以對,上命即永行停止。尋兼正白旗滿洲固山額真。駐防河間牛錄額真碩爾對訐告戶部發餉不均,下法司鞫問,部議巴哈納阿附睿親王,厚白旗,薄黃旗。時方治睿親王獄,坐巴哈納罪至死,上命寬之,削世職,奪官,籍其家三之二。

九年,起授刑部尚書。十一年,同諸大臣分賑畿輔,賜敕印以行。累進少傅兼太子太傅。十二年,授弘文院大學士。十五年,改中和殿大學士。十八年,復設內三院,又改秘書院大學士。康熙元年,兼鑲白旗滿洲固山額真。五年,卒。時鰲拜擅政,巴哈納與不洽,恤不行。聖祖親政,其子巴什以請,贈少師兼太子太師,謚敏壯。

宋權,字元平,河南商丘人。明天啟五年進士。官順天巡撫,駐密雲。受事甫三日,李自成陷京師,權計殺自成將黃錠等。睿親王師入關,籍所部以降,命巡撫如故。權疏言:「舊主御宇十有七年,宵衣旰食,聲色玩好一無所嗜。不幸有君無臣,釀成大亂。幸逢聖主,殲亂復仇,祭葬以禮。倘蒙敕議廟號,以光萬世,則仁至義盡,天下咸頌,四海可傳檄而定。明朝軍需浩繁,致有加派,有司假公濟私,明徵外有暗徵,公派外有私派,民困已極。請照萬歷初年為正額,其餘加增悉予蠲免。勤求上理,宜育賢才。臣所知者,如王永吉、方大猷、楊毓楫、硃繼祚、葉廷桂等,均濟時舟楫,惟上召而用之。」得旨嘉納。尋又薦寶坻進士杜立德等十一人。

時權仍駐密雲,撫治二十餘州縣,兼領軍事。旋以遵化當沖要,命權移駐,先後擊降自成黨數千。豐潤盜起,權捕治,以未獲其渠,疏請罷斥,溫旨慰留。尋疏陳祖軍、民壯之害,言:「明制祖傳軍籍,隸在營路;選取民壯,隸在州縣。身故則勾子孫,子孫絕則勾宗族,宗族盡則勾戚屬,流離逃竄,亂由此階。請特沛恩綸,除茲秕政。」又有私刻順天巡撫印偽為糾舉咨文投部者,事覺,逮治。權疏言:「用舍者君人之權,黜陟者銓樞之政,薦劾者撫按之職。請飭各省撫按,有關用舍大典,必具疏請,不須以咨文從事,則百弊俱清。」疏入,並如所請,著為令。

畿輔既平,詔撥近京荒田及明貴戚內監廢莊,畫為旗地,民田錯雜,別給官田互易。權疏言:「農民甫得易換之田,廬舍無依,耕種未備,請蠲租三年。」又迭疏請蠲薊州田租一年,除密雲荒地逃丁派徵錢糧,興三協屯政,守兵一予田十畝。俱下部議行。有詔優恤綠旗陣亡兵家屬,權請特遣部臣蒞視散給,俾霑實惠。

三年,擢國史院大學士。五年,遭母喪,請終制,命如常入直,私居持服。六年,假歸葬親。尋加太子太保。七年,還朝。時議用明例,遺御史巡方,權力持以為不可。八年,條陳時政,又言宜復設巡按。給事中陳調元、王廷諫等劾權前後持兩端,且追劾其母喪未除,入闈主試,下部議,權老病宜罷歸,遂命致仕。九年,卒。部議權被論致仕,祭葬宜殺禮。上以權誅自成黨有功,賜祭葬如例,贈少保兼太子太保,謚文康。子犖,自有傳。

傅以漸,字於磐,山東聊城人。順治三年一甲一名進士,授弘文院修撰。八年,遷國史院侍講。九年,遷左庶子。十年,歷秘書院侍講學士、少詹事,擢國史院學士。十一年,授秘書院大學士。十二年,詔陳時務,條上安民三事。加太子太保,改國史院文學士。先後充明史、太宗實錄纂修,太祖、太宗聖訓並通鑒總裁。又命作資政要覽後序,撰內則衍義,覆核賦役全書。十四年,命以漸及庶子曹本榮修易經通注。十五年,偕學士李霨主會試。考官入闈,例得攜書籍,言官請申禁,以漸請仍如舊例,許之。入闈病咯血,請另簡,命力疾料理。尋加少保,改武英殿大學士,兼兵部尚書。旋乞假還里,累疏乞休。十八年,解任。康熙四年,卒。

呂宮,字長音,江南武進人。順治四年一甲一名進士,授秘書院修撰。九年,加右中允。十年二月,上幸內院,召宮與侍講法若真,編修程芳朝、黃機,命撰柳下惠不以三公易其介論。宮論有曰:「伊、周、衛、霍,爭介不介。」上喜曰:「此三公語。」列第一。尋諭吏部:「翰林升轉,舊例論資俸,亦論才品。呂宮文章簡明,氣度閒雅。遇學士員缺,即行推補。」尋授秘書院學士。閏六月,遷吏部侍郎。十二月,超授弘文院大學士。言官請禁江、浙簽富戶運白糧並織造報充機戶,部議已有例禁,宮復請嚴飭督撫察究。

大學士陳名夏得罪,十一年,給事中王士禎、御史王秉乾劾宮為名夏黨,宮引罪乞罷,上命省改。初,平西王吳三桂專鎮,漸跋扈。宮與名夏及大學士馮銓、成克鞏薦御史郝浴,命巡按四川。至是,浴露章劾三桂,三桂疏辨,上為罷浴,宮與銓、克鞏皆坐誤舉,鐫二級留任。

宮以病乞假,上遣醫療治,問病狀。疏言:「乞假已三月,稟體怯弱,人道俱絕,僅能殭臥兀坐。乞寬期調治。」御史姜圖南劾疏語褻嫚,楊義復劾其曠職,宮亦累疏乞罷。十二年,以修資政要覽書成,加太子太保。宮復疏申請,賜貂裘、蟒緞、鞍馬,命馳驛回籍,俟病痊召用。十三年,敕存問,賜羊酒。十七年,詔大學士、尚書自陳,宮不具疏,左都御史魏裔介劾宮「一病六年,聞問杳然,忘君負恩」。上以宮請告無自陳例,諭毋苛求。十八年,世祖崩,宮赴都哭臨,病益殆,還里。康熙三年,卒。

成克鞏,字子固,直隸大名人。父基命,明大學士。克鞏崇禎十六年進士,改庶吉士。避亂里居。

順治二年,以左庶子李若琳薦,授國史院檢討。五年,遷秘書院侍讀學士。尋擢弘文院學士。九年,遷吏部侍郎。十年,擢本部尚書。疏言:「臣部四司,分省設官,原以諮訪本省官評。請令各司人注一簿,詳列本省各官賢否,參以撫按舉劾,備要缺推選。督撫舊無考成,請令疏列事跡,消弭盜賊,開墾荒田,清理錢糧,糾除貪悍,定為四則,以別賞罰。文選推升,概從掣簽。但地方繁、簡、沖、僻不同,如江南蘇、松等郡積弊之區,非初任邑令所能振刷。請取卓異官,或升或調,通融補授。行之有效,即加優擢,亦於選法無礙。」章下所司。尋擢秘書院大學士。以薦御史郝浴失人,鐫二級。十二年,命還所降級。

十二年,加太子太保。左都御史缺員,命克鞏暫攝,並諭俟得其人,仍回內院。疏言:「用人為治平之急務,而大僚尤重。今通政使李日芳、甘肅巡撫周文葉、陜西巡撫陳極新皆衰老昏庸,亟當更易。財用困乏,宜定丈量編審之期。學校冒濫,宜嚴考貢入學之額。任樞密者,遇封疆失事,不得借行查以滋推諉。司刑憲者,於棍徒詐害,不得寬反坐以長刁風。又若修築河工,宜覈冒銷,杜侵帑。此數事皆當振刷,以圖實政。」上深韙之。

給事中孫光祀劾左通政吳達兄逵叛逆,下法司勘擬。克鞏疏論左都御史龔鼎孳與達同鄉,徇隱不舉,鼎孳疏辨不知逵為達弟,坐奪俸。尋命克鞏回內院。十五年,加少保,改保和殿大學士,兼戶部尚書。十六年,加少傅兼太子太傅。十七年,遵例自陳,諭不必求罷。

部推浙江布政參議李昌祚擢大理寺少卿。先是,揚州亂民李之春事發,其黨亦有名李昌祚者,克鞏與大學士劉正宗票擬未陳明;又在吏部時,薦周亮工,擢至福建布政使,坐贓敗:克鞏疏引罪。左都御史魏裔介劾正宗,語連克鞏,並及昌祚、亮工事,克鞏疏辨,上責其巧飾,下王大臣議,罪當奪官。世祖初以克鞏世家子,知故事,不次擢用,值講筵,命內臣將畫工就邸舍圖其像以進,居常或中夜出片紙作國書詢時事,克鞏占對惟謹;至是,諭責其依違附和,凡事因人,仍寬之,命任事如故。

十八年,聖祖即位,復為國史院大學士。康熙元年,調秘書院大學士。二年,乞休回籍。

克鞏迭主鄉、會試,稱得士,湯斌、馬世俊、張玉書、嚴我斯、梁化鳳等,皆出其門。歷充太宗實錄,太祖、太宗聖訓總裁,屢得優賚。二十六年,太皇太后崩,赴臨。三十年,卒,年八十四。子亮,編修;光,武昌守道。

金之俊,字豈凡,江南吳江人。明萬歷四十七年進士,官至兵部侍郎。睿親王定京師,命仍故官。疏請先蠲畿甸田租以慰民望,又言:「土寇率★降者,宜赦罪勿論。縛渠來獻,分別敘功。就撫之眾,宜編保甲,令安故業。無恆產者,別為區畫。」尋奏薦丁魁楚、丁啟睿、線國安、房可壯、左懋泰、郝絅等,又劾通州道鄭煇優游養寇、三關總兵郝之潤縱兵肆掠,俱宜罷斥;並請趣畿南北巡按及監司以下官赴任,禁止滿洲官役額外需索驛遞夫馬。疏入,皆採行。

順治二年,以京師米貴,疏言:「大兵直取江南,應令漕督及巡漕御史赴任。金陵底定,舉行漕政。」詔速議行。因復上漕政八事,疏下所司。尋調吏部侍郎。三年,疏請酌定進士銓選之制。五年,擢工部尚書。六年,乞假歸,加太子太保。七年,還朝。八年,調兵部,加少保兼太子太保。十年,調左都御史。疏言:「審擬盜犯,請用正律,不宜概行籍沒,致累無辜。」又疏言:「直省提學,例以僉事道分遣。畿輔為首善之區,江南人才之會,請以翰林官簡用。」均報可。尋遷吏部尚書,授國史院大學士。

十二年,之俊病,乞休,上不允,遣畫工就邸畫其像。十三年,諭諸大臣曰:「君臣之義,終始相維。爾等今後毋以引年請歸為念。爾等豈忍違朕,朕亦何忍使爾等告歸?昨歲之俊病甚,朕遣人圖其容。念彼已老,惟恐不復相見,不勝眷戀。朕簡用之人,欲皓首相依,不忍離也!」之俊泣謝。十五年,改中和殿大學士,兼吏部尚書。同校定律例。十六年,詔立明莊烈帝碑,命之俊撰文。尋加太保兼太子太師,復乞假歸。十七年,自陳乞罷,溫諭敦召,未至,加太傅。十八年,復改秘書院大學士。之俊自歸後,屢以衰老乞休,康熙元年,始允致仕。

之俊家居,有為匿名帖榜其門以謗之者,之俊白總督郎廷佐窮治之,牽累不決。事聞,上不直所為,以律禁收審匿名帖,鐫廷佐二級,之俊削太傅銜。九年,卒,謚文通。

謝升,山東德州人。明萬歷三十五年進士,官至建極殿大學士,兼吏部尚書,加少保兼太子太保。崇禎之季,明帝欲與我議和,升洩其語,罷歸里。李自成入京師,升與明御史趙繼鼎、盧世水隺逐自成所置吏,奉明宗室香河知縣師敔城守。尋奉表來歸,授師敔知州,命升以建極殿大學士管吏部尚書。升至京師,改命與諸大學士共理機務。順治二年,卒,贈太傅,謚清義。

胡世安,四川井研人。明崇禎元年進士,官至少詹事。順治初,授原官。四遷禮部尚書。十五年,授武英殿大學士,兼兵部尚書。聖祖即位,與之俊同改秘書院大學士。以疾乞休,累加少師兼太子太師。康熙二年,卒。

王永吉,字修之,江南高郵人。明天啟間進士,官至薊遼總督。順治二年,以順天巡撫宋權薦,授大理寺卿。四年,擢工部侍郎。永吉疏辭,上責其博虛名,特允之,並諭永不錄用。居數年,有詔起用廢員,復詣京師,吏部疏薦,八年,授戶部侍郎。條奏各衛所屯地分上、中、下三等,請撥上田給運丁;各項折色銀請仍令官收官解,本色物料動支折價採買;洲田丈量累民,請以蘆課並入州縣考成,五年一次丈量:皆見採擇。

永吉家居,究心黃河下游閼壅為害,嘗議修涇河閘,濬射陽湖。九年,疏言:「黃水自邳、宿下至清河口,淮、泗之水聚於洪澤湖,亦出清河口。二水交會,淮、泗弱勢,不能敵黃。折而南趨四百餘里,出瓜洲、儀真方能達江。一線運河,收束甚緊,即有大小閘洞宣洩,海口不開,下流壅滯,以致河堤十年九決。海口在興化、泰州、鹽城境內,輒為附近居民填塞。乞敕河、漕重臣相度疏濬,復其故道。淮、泗消則黃河勢亦減。」

時河以北諸省患水,而江以南又苦旱,屢詔蠲賑,而湖廣、四川、閩、廣諸鎮待餉甚急。永吉疏請下廷臣籌足餉救荒之策,上命永吉詳具以聞。永吉因言:「各省兵有罪革占冒,馬亦有老病弱斃,十汰其二。以百萬之餉計之,歲可省二十萬。即以裁省之項,酌定直省災傷分數,則兵清而賦亦減。」上嘉納之。

畿輔奸民,每藉投充旗下,橫行骫法。永吉疏陳其害,謂:「上干國法,下失人心,請敕禁王大臣濫收人投旗,以息諸弊。」十年,擢兵部尚書。十一年,與刑部尚書覺羅巴哈納等分賑直隸八府。轉都察院左都御史,擢秘書院大學士。

永吉在兵部,鞫德州諸生呂煌匿逃人行賄,讞未當,下王大臣詰問,永吉厲聲爭辨。事聞上,諭曰:「永吉破格超擢,當竭力為國,乃因詰問,輒至忿怒,豈欲效陳名夏故態耶?」左授倉場侍郎。十二年,仍授國史院大學士。尋加太子太保,領吏部尚書。

十四年夏,旱,疏請「下直省督、撫、按諸臣清釐庶獄,如有殊常枉屈,奏請上裁;贖徒以下,保釋寧家」。下所司議行。旋以地震具疏引咎,上復責其博虛名。十五年,以兄子樹德科場關節事發,左授太常寺少卿,遷左副都御史。十六年,卒。上以永吉勤勞素著,命予優恤,贈少保兼太子太保、吏部尚書,謚文通。

黨崇雅,陜西寶雞人。明天啟五年進士,官至戶部侍郎。順治元年,以天津總督駱養性薦,授原官,調刑部。疏言:「舊制,大逆大盜,決不待時,餘俱監候秋後處決,未嘗一罹死刑,輒棄於市。請凡罪人照例區別,以昭欽恤。新制未定,並乞暫用明律。俟新例頒行,畫一遵守。」二年,復疏言:「流寇暴虐,今剿滅殆盡。恐寇黨株連,下民未獲寧止。請速頒恩赦。督、撫、司、道及府、州、縣各官,簡用務在得人,庶可廣皇仁,布實政。」並得旨允行。駱養性被訐貪婪通賊,辭連崇雅,讞不實,免議。給事中莊憲祖劾崇雅衰庸,崇雅疏乞罷,留之。五年,擢尚書。六年,加太子太保。八年,調戶部,加少保。十年,引疾告歸,命仍支原俸。旋召還。十一年,授國史院大學士。十二年,復以老乞休,加少傅兼太子太傅。入謝,上見其老,賜御服,諭曰:「卿今還里,服朕賜衣,如見朕也!」臨行,復召見,賜茶,慰以溫語,命大學士車克送之。十三年,敕存問。康熙五年,卒。明福王時,定從賊案,崇雅與衛周祚、高爾儼皆與。

衛周祚,山西曲沃人。明崇禎進士,官戶部郎中。順治元年,授吏部郎中。再遷刑部侍郎,疏言:「各省逮捕土寇,坐輒數十人,請飾鞫訊得實,具獄詞解部。京師多訐訟,請嚴反坐罪。功臣犯法,請復收贖之令。」調吏部,疏言:「六部司屬,請每歲令堂官糾舉黜陟。」「疆圉新闢,招民百名,即授知縣,暫委各官,即予本職,乃一時權宜計。請試以文義,有不嫻者,招民改武職,暫委授佐雜。」皆下部議行。擢尚書,歷工、吏二部。十五年,授文淵閣大學士,兼刑部尚書,改國史院。以葬兄周胤乞假還。復起授保和殿大學士,兼戶部尚書。以疾乞休。康熙十四年,卒,謚文清。周祚居鄉謹厚,聖祖稱之。西巡,遣大臣酹其墓。

周胤,明崇禎七年進士,官御史。順治初,授原官。官至兵部侍郎。

高爾儼,直隸靜海人。明崇禎十二年進士,官編修。順治初,授秘書院侍講學士。遷侍郎,歷禮、吏二部,擢吏部尚書,加太子太保。九年,為御史吳達所論,乞罷。旋起補弘文院大學士。十二年,卒,贈少保,謚文端。

張端,山東掖縣人。父忻,明天啟五年進士,官至刑部尚書。端,明崇禎十六年進士,改庶吉士。李自成入京師,端從忻皆降。順治初,忻以養性薦,授天津巡撫。端亦以薦授弘文院檢討。三遷為禮部侍郎。十年,授國史院大學士。十一年,卒,贈太子太保,謚文安。忻以靜海土寇亂罷,後端卒。

養性,崇禎時官錦衣衛都指揮使,頗用事。大學士吳甡戍,周延儒死,皆有力。來降,授總督。尋坐事罷,仍加太子太傅、左都督,進太子太師。求自效,授浙江掌印都司。卒。

論曰:世祖既親政,銳意求治,諸臣在相位,宜有閎規碩畫足以輔新運者。如蔣赫德請懲貪蠹;權首請田賦循萬歷舊額,並罷祖軍、民壯;永吉議清兵額、恤災傷,痛陳投旗之害;之俊、崇雅鄭重斷獄:可謂能舉其大矣。若巴哈納以細事塞明問,以漸、宮以巍科虛特擢,及額色赫、車克輩,皆鮮所建白。要其謹身奉上,亦一代風氣所由始也。


\end{pinyinscope}