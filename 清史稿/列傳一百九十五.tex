\article{列傳一百九十五}

\begin{pinyinscope}
李續賓丁銳義曾國華李續宜王★弟開化劉騰鴻弟騰蔣益澧

李續賓,字迪庵,湖南湘鄉人。諸生,膂力過人,善騎射。羅澤南講學里中,折節受書。咸豐初,澤南募鄉勇殺賊,續賓奉父命往佐之,從平桂東土匪。三年,援江西,令將右營。澤南每戰,續賓皆從。歸湖南,屯衡州,復永興。

四年夏,從澤南規岳州,湘軍僅千人,戰於大橋,續賓率數騎駐山岡,賊至不動,俟兵漸集,親搏戰,馳斬賊目,奪其旗,追北十餘里。次日,塔齊布至戰地,服其勇,由是知名。連旬與賊戰,續賓曰:「賊不得擄掠,今且盡,可乘機薄其壘。」塔齊布從之。會風雨,奮擊,連破賊壘,賊乃棄岳州而遁。論功,累擢知縣。從澤南克崇陽、咸寧,規武昌,大戰於花園,及破占魚套賊營,功皆最。武漢復,擢直隸州知州,賜花翎。進攻田家鎮,賊水陸數萬,塔齊布阻於富池口,湘軍合寶勇僅二千六百人,咸色沮,續賓手刃逃者三人,軍心始固。大戰於半壁山,殺賊數千,焚其巢,遂平田家鎮。擢知府,賜號摯勇巴圖魯。尋授安慶知府。

於是從羅澤南、塔齊布連復廣濟、黃梅,破賊於翟港、孔壟,每戰率為軍鋒。進規九江,九江城堅,賊所聚合,攻不能下。議分兵剿湖口、梅家洲,從澤南屯盔山。十二月,水師失利,入彭蠡湖,為賊所扼。續賓憤甚,請於曾國籓,自率千人渡江攻小池口,塔齊布率二十人偕行。塔齊布與續賓皆恃勇,每合戰,逼賊,席地坐,槍彈如雨,不顧,忽躍起突陣,橫厲無前,習以為常。至是眾寡懸絕,戰竟日不能克,暮收隊,而塔齊布失蹤,欲再渡江入賊壘覓之,塔齊布旋自返。

五年春,粵匪由江寧大股上犯,武昌再陷。曾國籓頓兵江西,續賓偕澤南從之。尋分赴贛東攻剿,連復弋陽、廣信、德興、義寧,記名以道員用。是年秋,回援湖北,克通城、崇陽,分兵趨羊樓峒。策賊遠道赴援利速戰,堅守俟之。明日賊至,相持至暮,瞰其怠,突擊之,大潰。蒲圻、咸寧相繼復,加鹽運使銜。十一月,進攻武昌,破塘角賊壘,又敗賊於窯灣,屢戰皆捷,蹋平城外賊壘。六年二月,羅澤南以砲傷卒於軍,軍中新失帥,人情洶洶,賊復增壘抗拒。巡撫胡林翼奏以續賓代領其眾,軍勢復振,盡鏟平城外新壘,連於賽湖堤、小龜山、雙鳳山破城中出竄之賊。七月,石達開糾江南、江西各路賊七八萬來援,城賊將應之,續賓御之魯家港,旬日內大小二十餘戰,解散脅從萬餘,破賊二十餘壘,加布政使銜。賊閉城不出,乃開壕引江水灌入,為長圍困之。十一月,克武昌,記名以按察使用。

渡江克黃州,連復大冶、興國,直薄九江城下。九江賊首林啟榮堅守苦戰。續賓復用攻武昌法,濬長壕三十里。七年三月,壕成,湖口、安慶賊迭來援,皆擊走之。六月,賊犯蘄州、黃梅,續賓渡江迎擊於廣濟童司牌,大破之。合水師進攻小池口,毀其城。策九江賊恃湖口為犄角,不拔湖口,九江不可得。九月,令弟續宜攻梅家洲,自率師揚言往宿松,潛伏湖口後山。水師並至,分攻,賊方悉銳以拒。續賓率士卒攀蘿至山椒,破空下,賊大駭,盡殲其眾。立克湖口縣城,梅家洲賊亦遁,乘勝克彭澤及小姑洑。捷聞,授浙江布政使。於是賊援遂絕。八年四月,以地雷轟城百餘丈,梯而登,殄賊萬餘,擒林啟榮及李興隆等磔之。九江平,加巡撫銜,賜黃馬褂,許專摺奏事。

續賓既下九江,請假省親,抵湖北,陳玉成陷麻城、黃安,移兵擊走之。時續賓威望冠諸軍,浙人官京師者,合疏請飭援浙江。胡林翼議大舉進規安徽,詔將軍都興阿、總兵鮑超由宿松趨安慶,續賓由英山趨太湖。續賓乃留弟續宜屯武昌,自率八千人行,會起曾國籓視師,續賓復分所部千人與之,至太湖而署巡撫李孟群師潰廬州,改道赴援。八月至九月,克楓香鋪、小池驛、梅心驛,復太湖、潛山、桐城、舒城,賊望風潰走。軍無留行,進規廬州。

賊於三河鎮築城,外列九壘,憑河設險,我軍非得三河不能進。續賓克桐城、舒城後,各留守兵,所率臨敵僅五千人。十月,分三路攻賊,九壘皆下,殺賊七千餘,我軍傷亡亦逾千人。趣後軍未至,而陳玉成、李世賢糾合捻匪來援,眾十萬,連營十餘里。諸將議退守桐城,續賓不可。夜半,部勒各營,旦日迎擊,至樊家渡,天大霧,賊分隊包抄,我軍驚潰,副將劉祜山,參將彭友勝,游擊胡廷槐、鄒玉堂、杜廷光,皆戰死。續賓沖蕩苦戰,賊集愈多,營壘皆破。或勸突圍出,圖再振,續賓曰:「軍興十年,皆以退走損國威。吾前後數百戰,出隊即不望生還。今日必死,不原從者自為計。」諸將士皆曰:「原從公死!」日暮上馬,開壁擊殺數百人。總兵李續燾、副將彭祥瑞越壘沖出,賊踞其壘,決河堤,斷去路。續賓具衣冠望闕叩首,取所奉廷旨及批摺焚之,曰:「不可使宸翰污賊手。」躍馬馳入賊陣,死之。同知曾國華,知府何忠駿,知州王揆一,同知董容方,知縣楊德訚,從九品李續蓺、張溥萬,皆殉焉。道員孫守信、運同丁銳義猶守中右營,越三日營陷,同死之。是役文武官弁死者數百人,士卒數千人。

時方有旨命會辦安徽軍務,及死事上聞,文宗流涕,手敕曰:「惜我良將,不克令終。尚冀忠靈不昧,他年生申、甫以佐予也!」贈總督,入祀昭忠祠,立功地建專祠,謚忠武。賜其父一品封典,子光久、光令並賜舉人,予騎都尉世職。

續賓既歿,曾國籓疏上其生平戰績,略曰:「續賓隨羅澤南征剿,循循不自表異。岳州之戰,所將白旗,號為無敵,田家鎮以少勝眾。九江之敗,士卒多逃,獨所部依依不去,眾稱其能得士心。軍中人人以氣節相高,獨默然深藏。然忠果之色,見於眉宇。遠近上下,皆信其大節不茍。臣所立湘勇營制,行之既久,各營時有變更,獨續賓守法,始終不變。歷年節省餉項及廉俸,不寄家自肥,概留備軍中非常之需。量力濟人,不忍他軍饑而己軍獨飽。馭下極寬,而弁勇有罪,往往揮淚手刃之。至於臨陣,專以救敗為務。遇賊則讓人御其弱者,自當其悍者。分兵則以強者予人,而攜弱者自隨。弱者漸強,又易新營。軍中每言肯攜帶弱兵,肯臨陣救人者,前惟塔齊布,後惟續賓。三河之敗,亦由分兵所致。此軍民所由感泣不忘者也。」於是特詔嘉其有古名將風,以國籓疏宣付史館,用示褒異。洎江南平,軫念前勞,加二等輕車都尉,並為男爵,子光久襲。

丁銳義,字伯冕,長沙人。治鄉團有聲,咸豐四年,從胡林翼援湖北,募壯士百人,後增至千人,號義字營。戰武漢,以勇聞。六年,羅澤南傷殞,賊酋古隆賢率眾犯官軍後路。諸將以新失帥,皆主堅守。銳義曰:「我軍頓城下六閱月,求戰不得。今賊來乘我,出其不意,可一鼓滅。」林翼壯之,令與唐訓方、蔣益澧、孫守信等夜出掩擊,大破賊於豹子海。又戰葛店、華容,奪樊口賊舟,克武昌縣,圍黃州。會大水,退軍屯青山。武漢復,擢知縣。駐防蘄、黃間,屢與鄉團卻敵。八年,破黃泥畈、青天畈賊壘,擢同知。又破賊於南陽河、阿彌鎮,擢運同。遂從李續賓進剿安徽,破石牌賊壘,連下數縣。

將進攻三河,銳義諫曰:「孤軍深入,留兵四城,分力之半,死傷復多,士罷將驕,賊援將集,而貪進不已,此所謂強弩之末也。使賊斷絕我餉道,舒、桐、潛、太兵少,見勝則怠,見敗必潰,四城將並覆。乃令退師桐城,休息待援,僅可不敗耳。」續賓不聽,銳義乃馳書湖北請援。續賓讓之曰:「君嘗以千人破賊數萬,乃何怯耶!」及續賓軍敗,銳義率所部急救,身被數創。續賓突圍戰死,銳義偕孫守信堅守其壁。三日壘破,死之。銳義耳聾,喜論兵,戰每孤軍勇進。獨三河之役主持重,而說不見用。恤贈鹽運使,加太常寺卿、騎都尉世職。

孫守信,亦長沙人。由內閣供事敘從九品,官湖北,從軍積功,累擢道員。未嘗獨將,與銳義為友,臨危不去。同及於難。贈按察使,加太常寺卿、騎都尉世職。

曾國華,字溫甫,國籓弟。咸豐五年,國籓兵困於江西,國華請於父,赴湖北乞師。胡林翼令劉騰鴻,吳坤修、普承堯率五千人往援,以國華領其軍。攻克咸寧、蒲圻、通城、新昌、上高,以達瑞州。騰鴻戰城南,國華偕承堯戰城西北,屢破賊。國籓至,乃合圍,掘塹周三十里,斷賊接濟。會丁父憂,偕國籓奔喪去軍。與李續賓姻家,招佐軍事。當連克四縣,軍勢銳甚,國華以常勝軍家所忌,時與續賓深語,並書告國籓。及軍敗,從續賓力戰死,贈道銜,予騎都尉世職,謚愍烈。

李續宜,字希庵,續賓弟。同事羅澤南。以文童從軍,援江西、湖北,積功累擢同知,賜花翎。武昌、漢陽復,胡林翼疏陳續宜功多為續賓所掩,詔以知府選用。從續賓攻九江,賊由安徽上犯蘄、黃以牽我師。咸豐七年,續宜率兵千七百人回援湖北,戰於黃州壩崎山,分三路進,毀賊壘,次蘄水、黃岡界。上馬家河、火石港、郴柳灣賊壘林立,傾巢出撲,續宜伏兵山下,驟起突擊,賊大亂,譟,乘之,破壘四十,移屯蘄水。遇援賊於月山,誘至山角,發砲擊之,潰,直搗其巢,焚屯聚數十處,破偽城五。會克小池口,以道員用,賜號伊勒達巴圖魯,由是續宜之名與其兄相頡頏。

回軍江西,會攻梅家洲,克湖口。十月,賊酋韋俊率眾二萬復犯湖口。續宜駐蝘蚘山,分兵三路,一出馬影橋,一出流澌橋,一扼勞家渡,賊來,擊卻之。而賊由西洋橋、排龍口、二賢寺直趨蝘蚘山,續宜麾諸路奮擊,斬獲千餘。馳抵磨盤山,設伏破泰坪關援賊,賊乃遁。八年,九江既克,陳玉成由安徽竄蘄、黃,陷黃安。續宜馳援不利,續賓繼至,合擊。續宜攻北門,破其壘,賊夜遁,復黃安。進至麻城,賊不戰引去。續賓出師規安徽,胡林翼疏請留續宜固楚疆。洎三河師熸,續賓戰歿,續宜在黃州,收輯殘部,思鄉者遣歸,原留者歸伍,差汰罪將,簡用其良,申儆訓練,經歲軍氣始復振。

九年,授荊宜施道。石達開由江西竄入湖南,眾號三十萬,圍寶慶府城。胡林翼檄續宜率兵五千馳援,諸援軍悉歸統屬。時援軍三萬餘,城被圍兩月。賊眾,食且盡,野掠無所得,聞續宜至,攻愈急。續宜渡資江而軍,與劉長佑軍當賊沖,四戰而圍解,賊竄廣西境。詔嘉續宜赴援迅速,加布政使銜。

十年,遷安徽按察使。曾國荃方圍安慶,多隆阿攻桐城,續宜率萬人屯青草塥,在安慶、桐城之間。陳玉成以十萬眾來援,續宜與多隆阿夾擊於掛車河,盡破棠梨山、尊上庵、香鋪街、望鶴墩賊壘,斬馘無算,追奔二十餘里,玉成走廬江。捷聞,賜二品頂戴。十一年,擢安徽巡撫,疏言:「陳玉成圖解安慶之圍,悉銳西竄,以攻我之所必救。湖北為眾軍根本,臣宜提師回援,不能遽任皖撫之事。」比抵武昌,賊已陷黃州、德安兩府五縣,乃會彭玉麟水師夾攻孝感,乘夜縱火,復其城,進攻德安,穴地道克之。武昌、通城、咸寧、蒲圻諸縣相繼皆下,賜黃馬褂。胡林翼病歿,詔授續宜湖北巡撫,駐黃州督師。捻匪犯光化、穀城、均州及棗陽、襄陽,皆擊走之,調安徽巡撫。

同治元年,命幫辦欽差大臣勝保軍務。時苗沛霖叛服無常,勝保袒之。詔密詢續宜剿撫機宜,覆疏略謂:「苗沛霖官至道員,公犯不韙,圍撫臣於壽州,陷其城,屠其眾。乃復詭言求撫,此豈足信?不過假稱反正,號召近縣,養成羽翼。若正彼叛逆之名,人人得而誅之。寬其黨羽,使為我用,彼勢孤,終成擒耳。」上韙之。續宜駐臨淮,令提督成大吉、總兵蕭慶衍,渡淮援潁州,破捻匪張洛行於大橋集,潁州圍解。又令蔣凝學克霍丘,撫綏各圩,解散逆黨。沛霖懾湘軍兵威,請討捻自贖,而勝保終欲養沛霖以自重,轉嫉湘軍,勢不相下。會袁甲三以病請去,命續宜代為欽差大臣,督辦安徽全省軍務。續宜旋丁母憂,奪情留軍。三疏陳謝,舉唐訓方自代,允假百日。回籍病咯血,六次詔促起視師,不能赴,二年十一月,卒於家。詔加恩依總督軍營病故例賜恤,立功地方及原籍建專祠,謚勇毅。賜其父人蓡四兩,地方官以時存問。子光英,予官直隸州知州。

續宜治軍嚴整,與兄續賓同負重名。曾國籓嘗論其昆弟為人,續賓好蓋覆人過,續宜則嫉惡稍嚴。續賓戰必身先,驍果縝密,續宜則規畫大計,不校一戰之利,及其成功一也。

王珍,字璞山,湖南湘鄉人。諸生,從羅澤南學,任俠好奇。咸豐二年,粵匪犯長沙,上書縣令硃孫詒,請練鄉兵從澤南教練,屯馬垞埔,以團防勞敘縣丞。剿桂東土匪有功。廣東邊境匪犯興寧,率死士百人馳擊,殪賊甚多,累擢同知直隸州。

三年,羅澤南援江西,初戰多死傷。珍請於曾國籓,增募三千人,將往援,會南昌圍解。國籓議裁汰其軍,巡撫駱秉章見所募勇可用,留二千四百人防湖南。珍精於訓練,令士卒縛鐵瓦習超距。自以意為陣法,進退變動,異於諸軍。四年,粵匪踞岳州,珍由湘陰進攻,敗賊於杉木橋,乘勝克岳州,曾國籓率水陸軍並至。珍出境進剿,遇賊羊樓峒,失利,賊躡其後,岳州復陷。珍所部死者千人,坐輕進奪職,留營效力。既而羅澤南從國籓東征,珍收集散眾,留未遣,駱秉章令率五百人徇郴州。

時兩廣交界土匪蜂起,硃連英、胡有祿最強,各擁萬人,稱王號,時時擾湖南邊境,珍與參將周云耀協防江華,數擊走之。援道州,解其圍。策賊必乘虛襲江華,日馳百餘里,先至,待賊至迎擊,大破之。進搗桃州,出龍虎關,破恭城賊於慄木街,回軍解寧遠、藍山圍。別賊掠零陵,周云耀困於隘。珍率數十人馳進,令曰:「寇眾,退且死!」據險夾擊,逐北數十里,轉戰深入九嶷山,賊氛漸清,復原官,賜花翎。五年,土匪何賤茍勾結硃連英陷富川、江華,進犯永明。珍偕周云耀往剿,迭敗之。連州匪自龍虎關來犯,勢甚張,疾趨迎擊,殪賊二千,擢知府。餘賊走陷灌陽,復由全州襲陷東安。珍會廣西軍克灌陽,馳至東安城下,環攻兩月,始克之。賊竄出,合胡有祿,將入四明山。分路抄襲,擒有祿,焚山中賊巢,餘黨悉盡。時別賊何祿踞郴州,陳義和踞桂陽,分擾永興、茶陵、耒陽,窺衡州。珍增募至千五百人,分兵守耒陽,自率千人攻桂陽,再戰克之。賊聚糧於瓦蜜坪,火其屯,出奇兵攻郴州,賊遁走尚萬餘,合鄉團邀擊於黃沙堡,追至兩廣墟,賊方食,縱擊殲之。乘勝破永明、江華踞賊,窮追至連州,硃連英僅以身跳免。六年春,又破賊於陽山,賊遁英德。駱秉章上其功,迭詔嘉獎,予四品封典,以道員即選。

珍專辦南防凡二年,湘、粵間諸匪誅殄幾盡,軍士死亡亦多。請假將還,會羅澤南卒於武昌,李續賓代將其軍,粵匪石達開自江西窺湖北,續賓招珍助剿。遂進屯岳州,轉戰崇陽、通城、通山、蒲圻,復四縣,殲賊首張康忠、陳華玉等,興國、大冶匪眾亦解散。武昌尋克復,加按察使銜,以湖北道員記名簡放,仍駐軍岳州。

七年,調援江西,五月,抵吉安。先是官軍水陸合圍吉安,其攻臨江者,亦掘長壕困賊。賊渠胡壽階、何秉權率眾數萬來援,據水東,與城賊夾江相望。珍沿贛江而南,自三曲灘濟,結營水東東南山上。賊鼓噪乘之,珍登望樓,令士卒築壘不輟,毋許仰視,賊疑不進。俄山後一軍出賊背,鼓聲起,築壘者投畚大呼馳擊,左右伏起,陣斬秉權,蹙賊眾於水,餘走水東。軍中方具餐,珍曰:「不克水東不遑食!」揮軍搗賊壘。都司易普照,勇士也,先登中砲殞,眾憤,爭入壘,殺賊數千,壽階遁。珍渡江壁藤田,壽階自寧都、沙溪挾援眾來犯,珍分兵擊其左,自率百人搗其右,賊崩潰,蹙之瑤嶺,擒壽階,斬馘數千。是役悍賊俘斬殆盡。閏月,援賊復自寧都出永豐。珍以千二百人迎擊之,追至寧都之釣峰。賊背水以拒,既敗,盡沒於水。斬賊首蕭復勝等,拔難民萬餘;六月,再破新城賊於東山壩,斬賊首張宗相等。

時悍賊楊輔清憤屢敗,糾眾十萬踞廣昌頭陂,誓決死戰。珍笑曰:「賊聚此,可一鼓殲也!」勒兵大戰,先馳馬陷陣,眾從之,賊大潰,逐北六十里,斬馘無算。捷聞,詔嘉獎,稱其以寡敵眾,殲除鉅憝,賜號給什蘭巴圖魯。方拔樂安,進規撫、建兩郡,會周鳳山兵潰吉安,乞援。珍令鄉團張己幟趨建昌,而潛返藤田規吉水。楊輔清聞珍去,以七萬眾圍樂安。珍夜入城,誘賊至城下痛殲之。輔清屯林頭,珍進擊,賊以馬隊數千突陣,令火箭射之,藤牌兵俯首砍馬足。劉松山、易開俊左右合擊,自率精銳貫賊陣,斬級數千,獲馬三百匹,俘八百人,輔清遁走。珍感疾返樂安,九月,卒於軍,年僅三十有三。詔嘉獎紀律嚴明,身經數百戰,前後殺賊十餘萬,克復城池二十餘處,厥功甚偉,贈布政使銜,依二品從優議恤,予騎都尉世職。江西、湖南建專祠,謚壯武。

珍貌不逾中人,膽力沉鷙,用兵好出奇制勝,馭眾嚴而有恩。所著有練勇芻言、陣法新編,皆出心得。劉松山為湘軍後起名將,舊隸部下,後其軍皆用珍法。珍既歿,所部歸其弟開化及張運蘭分統之。

開化,年十七從珍軍中,南防剿匪功最多,累擢知縣。及援江西,寧都釣峰之戰,率伏兵潛襲賊營,遂大捷,無戰不與。駱秉章疏陳其功,擢知府。遂令分統珍軍,偕張運蘭攻吉安,連戰皆捷。八年,克樂安、宜黃、崇仁、南豐、建昌,擢道員,加按察使銜。病歸里。十年,左宗棠初出治軍,開化從之,戰鄱陽、樂平,皆有功。及宗棠大破李世賢於樂平,開化與劉典各當一路。是役官軍不及萬,破賊十萬,稱奇捷,加布政使銜。江西既平,從宗棠援皖南。十一年,卒於軍。開化在軍先後八年,勇毅亞於其兄。詔優恤,予騎都尉世職,謚貞介。

劉騰鴻,字峙衡,湖南湘鄉人。少讀書,未遇,服賈江湖間。咸豐三年,夜泊湘江,遇潰卒數十輩行掠,誘至湘潭,白縣令捕之,由是知名。

五年,巴陵土匪起,巡撫駱秉章令率五百人往戰於毛田,擒賊渠,又敗之於三林坳,散其黨,遂駐岳州。從羅澤南攻通城,攀堞登城,克之。參將彭三元等戰歿崇陽,澤南調騰鴻往,而石達開驅悍賊二萬來撲,騰鴻與游擊普承堯夾擊破之。蒲圻賊壘臨河,騰鴻由寶塔山截渡河賊,直抵城下,與普承堯循環攻擊,克蒲圻。連下咸寧,抵武昌。騰鴻偕蔣益澧為後隊,搜伏賊,殲斃甚眾。論功,以從九品選用。羅澤南愛其才,令增募五百人當前敵。騰鴻遂師事澤南,列弟子籍。攻克十字街、塘角賊壘,毀其船廠,進據小龜山。賊七八千由塘角沿湖而下,澤南自率中營出洪山西,令騰鴻出洪山東,夾擊,斃賊無算,蕩平賊壘。胡林翼奏騰鴻身先陷陣,七戰皆在諸軍前,超擢知縣。六年春,賊踞賽湖以阻官軍,騰鴻與戰於堤上,追及長虹橋,遇伏,賊七倍我,奮擊,殺賊五六百。羅澤南欲扼窯灣,賊出爭,大戰於小龜山,斬級六百,遂偕李續賓同駐其地。騰鴻所將號湘後營,樹黑幟,賊望見輒走。

會江西軍事棘,胡林翼令騰鴻率所部千人從曾國華赴援瑞州,道為賊梗,轉戰而前,連捷於羊樓峒、分水坳,擒斬偽總制三十餘人,克上高、新昌。七月,進攻瑞州,郡治有南北二城,中貫一河,聯以長橋。先拔南城,賊酋韋昌輝自臨江來援,軍容甚盛。騰鴻曰:「是羊質虎皮,不久見鞟。宜乘其敝攻之。」相持旬日,賊氣衰。乃從北岸渡兵抄其後,與南城兵夾擊,大敗之。偽指揮黃姓來援,列陣出岡,兩軍對峙。別賊馳截我後路,圖夾攻,俟其近,發劈山砲擊之,再至,皆擊退,追奔三十里。石達開適自九江來,勒賊復還,築五壘於東北。騰鴻曰:「不急破之,壘成則難制矣。」令楚軍防城賊,江軍進剿,自率死士三百督戰。賊見兵少,先犯之,三百人植立無聲,伺近乃發砲,凡沖突六次不為動,賊氣沮,諸營效力猛攻,賊大敗,盡平其壘。捷聞,擢直隸州知州,歸江西補用,賜號沖勇巴圖魯。

自克南城後,賊萃於北城。騰鴻欲斷其接濟,取南城磚石築壘造橋,賊來爭,且戰且築,又於北岸石封嶺築新城以逼之。七年春,曾國籓巡視瑞州,用騰鴻議,為長壕三十里,絕賊餉道。國籓尋以喪返湖南,囑騰鴻主南路軍事。先後遏賊於馬鞍嶺、陰岡嶺,戰皆捷,於是會諸軍克袁州、分宜、上高、新喻。劉長佑與賊戰於羅防,不利,騰鴻往援,擊敗之。七月,回攻瑞州。時李續賓進兵九江,胡林翼疏調騰鴻回湖北。騰鴻以功在垂成,先分兵應之,而攻城益力,奪南門砲臺,復撲東門,毀其城樓,身自督戰,中槍子五,臥不能起。次日,裹創舁往,城垂克,忽中砲,洞穿左脅,移時殞。語弟騰鶴曰:「城不下,無斂我!」一軍皆泣,冒砲火登城,斬殺悍賊過半,即夕克瑞州,迎騰鴻尸入城治喪。事聞,恤典加等,依道員例,予騎都尉世職,於瑞州建專祠,予其父母正四品封典。洎江南平,曾國籓追論前功,詔嘉其忠勇邁倫,加恩予謚武烈。

弟騰鶴,隨軍將中營。先數月,因攻城傷左臂,創甚。騰鴻命歸,不可。及騰鴻殞於陣,騰鶴號泣督戰,克竟厥功,遂代將其軍。進援臨江,復峽江。會攻吉安,當西南路,掘長壕久困之。八年秋,賊乘江漲突圍出,兩次皆擊退,尋拔其城。率所部窮追,斬馘過半。調防九江,屯彭澤。九年二月,戰牯牛嶺,進攻建德風雲嶺賊巢,破其二壘。賊大至,被圍,力戰死之,年二十有八。官候選知府,詔依道員例賜恤,予騎都尉世職,附祀兄祠。

蔣益澧,字薌泉,湖南湘鄉人。少不羈,不諧於鄉里,客游四方。湖南軍事起,從王珍攻岳州,以功敘從九品。復隸羅澤南部下,勇敢常先人,澤南異之,許列弟子籍。從克黃梅,擢縣丞。進剿九江,連敗賊於白水港、小池口。咸豐五年,進攻廣信。大軍駐城西烏石山,益澧屯山右。賊覷其壘未成,來攻。益澧堅壁不動,伺懈縱擊,斬賊首於陣。進逼城下,諸軍蟻附而登,復其城。進攻義寧,澤南潛師進鼇嶺,令益澧分駐乾坑。賊來爭,分數千人抄官軍後。益澧曰:「今以數百人當大敵,不死戰,將殞。」揮兵直薄之,當者披靡,遂會師鼇嶺,乘勝復義寧,擢知縣。

從澤南回援武昌。在軍與李續賓論事不相下,及澤南歿,續賓代將。益澧屯魯港,賊攻之急,請援,續賓置之。益澧大恚,憑壘死守,賊旋引去。益澧遂告歸,不待報而行。嗣武漢克復,仍論前功,擢知府,賜花翎。

益澧家居,悒悒不得志,會廣西匪熾,乞援於湖南,湖南宿將盡出征,駱秉章顧左右無可屬者,益澧請行,乃令率千六百人赴之。七年五月,連破賊於賣珠嶺、唐家市,復興安、靈川;艇匪踞平樂二塘墟、沙子街,進破之,焚賊艇,薄平樂,克之:擢道員,賜號額哲爾克巴圖魯,加按察使銜。巡撫勞崇光疏請留於廣西補用,八年,入屯桂林。時廣西兵食並絀,率藉招撫馭盜,兵賊相糅,橫行無忌,疆吏不能制。益澧至,乘兵威,悉按誅桀黠者,易置守軍,人心始定。駱秉章奏助益澧軍月餉二萬,造船六十艘,募水師以益其軍。省城既固,進規右江。賊踞柳州,連結洞砦,恃水師不能至。益澧具舟修仁,令軍士舁舢板陸行九十里,置洛青水中,載砲而下,遇賊洛垢墟,火賊舟。次日,賊水陸並集,力戰斬賊數千,進鷓鴣山,攻柳州克之,加布政使銜。偕右江道張凱嵩會剿慶遠,掘長壕斷賊出入,賊渡河竄,邀擊敗之。慶遠平,以按察使記名。

九年,石達開竄湖南,前隊掠全州,益澧分兵守柳州,自回援省城,授按察使,尋遷布政使。出剿恭城土匪,扼平樂。粵匪石國宗由全州、興安窺桂林,勢甚張。學政李載熙劾益澧失機及冒餉忌功等事,詔念益澧前勞,降道員,留廣西,並下疆臣察奏。會湖南遣劉長佑、蕭啟江率師來援,益澧與合剿,解桂林圍。駱秉章、曹澍鍾並為疏辨,得白。十年,賀縣匪分擾昭平、平樂,益澧擊走之。進破賊首陳金剛於大灣嶺,焚沙田賊寨,復布政使原銜。又會廣東援師破賊於竹洞嶺。十一年,復授廣西按察使,進駐平南。偕總兵李揚升復潯州,復布政使原官。

益澧年少戇急,曾國籓、胡林翼素不滿之,而左宗棠特器重。至是宗棠規浙江,疏請以益澧為助。同治元年,調浙江布政使。自湖南增募八千人,道廣東,總督勞崇光資以餉械。九月,至衢州,分兵復壽昌。賊酋李世賢屯裘家堰,按察使劉典兵先進,益澧繼之,降賊李世詳為內應,襲破之,悉毀賊壘。二年,克湯溪,被珍賚優敘。宗棠進屯嚴州,規富陽,援賊麕至,益澧渡江築壘新橋,分三路迎擊,大敗之。會游擊徐文秀等攻雞籠山,益澧自督戰,盡破十餘壘。八月,克富陽。自杭州至餘杭,賊營連數十里。益澧沿江下逼清波、鳳山兩門,據十里街、六和塔、萬松嶺,俯瞰城中,自駐東嶽廟,賊屢出犯,皆擊退。分兵會道員楊昌濬、總兵黃少春攻餘杭,敗賊城下,匿不出。又破鳳山門、清泰門賊壘,由錢江入西湖,奪賊舟。平湖、乍浦、海鹽皆下,海寧守賊蔡元吉、桐鄉守賊何紹章先後投誠效用。三年,令紹章扼烏鎮,元吉會蘇師復嘉興,賊勢日蹙。二月,饅頭山地雷發,壞城垣,諸軍擁入,戰竟日,悍賊多斃,餘夜遁,遂復杭州,餘杭亦下。詔嘉其功,賜黃馬褂,予雲騎尉世職。分軍克德清、石門,進攻湖州。蔡元吉深入,為賊所圍,益澧自往援之。轉戰而前,距元吉營隔一河未達。時偽幼王洪福瑱遁入湖州,悍酋黃文金眾尚十餘萬。七月,作浮橋通元吉營,出湖趺漾襲賊後。降賊譚侍友出太湖攻袁家匯,賊棄城走,邀擊之,解散數萬人。浙境肅清,晉騎都尉世職。

左宗棠追賊赴福建,益澧護理巡撫。疏陳善後事宜,籌閩餉,濬湖汊,築海塘,捕槍匪,又覈減漕糧,酌裁關稅,商農相率來歸。增書院膏火,建經生講舍,設義學,興善堂,百廢具舉。東南諸省善後之政,以浙江為最。逾歲,乃回本任。

五年,擢廣東巡撫,奏裁太平關稅陋規四萬兩,斥革丁胥,改由巡撫委員徵收;五坑客匪投誠,分別安插高、廣各府,另編客籍;設學額:並如議行。六年,以病乞休。尋為總督瑞麟疏劾,下閩浙總督吳棠按奏,坐任性不依例案,部議降四級,改降二級,以按察使候補,命赴左宗棠軍營差委。尋授廣西按察使,以病回籍。

十三年,日本窺臺灣,召至京。未及任用,病卒。太常寺卿周瑞清疏陳益澧廣西政績,詔復原官,依巡撫例賜恤。浙江巡撫楊昌濬、梅啟照先後疏言平浙功尤鉅,詔允建祠,謚果敏。

論曰:李續賓果毅仁廉,治軍一守羅澤南遺法,戡定武昌、九江,戰績為一時之冠。李續宜獨以持重稱,殆鑒於其兄之銳進不終而然耶?王珍、劉騰鴻皆出奇制勝,駿利無敵,惜早殞,未竟其功。蔣益澧經挫折而奮起,平浙、治浙,並著顯績,信乎能自樹立。諸人並湘軍之傑,不以名位論高下也。


\end{pinyinscope}