\article{志七十八}

\begin{pinyinscope}
○輿服二

皇帝冠服皇后冠服太皇太后皇太后附皇貴妃以下冠服

皇子親王以下冠服皇子親王福晉以下冠服

文武官冠服命婦冠服士庶冠服

崇德二年,諭諸王、貝勒曰:「昔金熙宗及金主亮廢其祖宗時冠服,改服漢人衣冠。迨至世宗,始復舊制。我國家以騎射為業,今若輕循漢人之俗,不親弓矢,則武備何由而習乎?射獵者,演武之法;服制者,立國之經。嗣後凡出師、田獵,許服便服,其餘悉令遵照國初定制,仍服朝衣。並欲使後世子孫勿輕變棄祖制。」乾隆三十七年,三通館進呈所纂嘉禮考,於遼、金、元各代冠服之制,敘載未能明晰。奉諭:「遼、金、元衣冠,初未嘗不循其國俗,後乃改用漢、唐儀式。其因革次第,原非出於一時。即如金代朝祭之服,其先雖加文飾,未至盡棄其舊。至章宗乃概為更制。是應詳考,以徵蔑棄舊典之由。衣冠為一代昭度,夏收殷冔,不相沿襲。凡一朝所用,原各自有法程,所謂禮不忘其本也。自北魏始有易服之說,至遼、金、元諸君浮慕好名,一再世輒改衣冠,盡去其純樸素風。傳之未久,國勢浸弱。況揆其議改者,不過雲袞冕備章,文物足觀耳。殊不知潤色章身,即取其文,亦何必僅沿其式?如本朝所定朝祀之服,山龍藻火,粲然具列,皆義本禮經,而又何通天絳紗之足雲耶?」蓋清自崇德初元,已釐定上下冠服諸制。高宗一代,法式加詳,而猶於變本忘先,諄諄訓誡。亦深維乎根本至計,未可輕革舊俗。祖宗成憲具在,所宜永守勿愆也。茲就乾隆朝增改之制,以類敘次,而仍以初定者附見於篇。

皇帝朝冠,冬用薰貂,十一月朔至上元用黑狐。上綴硃緯。頂三層,貫東珠各一,皆承以金龍四,餘東珠如其數,上銜大珍珠一。夏織玉草或藤竹絲為之,緣石青片金二層,裡用紅片金或紅紗。上綴硃緯,前綴金佛,飾東珠十五。後綴舍林,飾東珠七,頂如冬制。

吉服冠,冬用海龍、薰貂、紫貂惟其時。上綴硃緯。頂滿花金座,上銜大珍珠一。夏織玉草或藤竹絲為之,紅紗綢裡,石青片金緣。上綴硃緯。頂如冬吉服冠。

常服冠,紅絨結頂,不加梁,餘如吉服冠。

行冠,冬用黑狐或黑羊皮、青絨,餘俱如常服冠。夏織藤竹絲為之,紅紗裏緣。上綴硃氂。頂及梁皆黃色,前綴珍珠一。

端罩,紫貂為之。十一月朔至上元用黑狐。明黃緞里。左、右垂帶各二,下廣而銳,色與里同。

袞服,色用石青,繡五爪正面金龍四團,兩肩前後各一。其章左日、右月,萬壽篆文,間以五色雲。春、秋棉、袷,冬裘、夏紗惟其時。

朝服,色用明黃,惟祀天用藍,朝日用紅,夕月用月白。披領及袖皆石青,緣用片金,冬加海龍緣。繡文兩肩,前、後正龍各一,腰帷行龍五,衽正龍一,襞積前、後團龍各九,裳正龍二、行龍四,披領行龍二,袖端正龍各一。列十二章,日、月、星、辰、山、龍、華、蟲、黼黻在衣,宗彞、藻火、粉米在裳,間以五色雲。下幅八寶平水。十一月朔至上元,披領及裳俱表以紫貂,袖端薰貂。繡文兩肩,前、後正龍各一,襞積行龍六。列十二章,俱在衣,間以五色雲。

龍袍,色用明黃。領、袖俱石青,片金緣。繡文金龍九。列十二章,間以五色雲。領前後正龍各一,左、右及交襟處行龍各一,袖端正龍各一。下幅八寶立水,襟左右開,棉、袷、紗、裘,各惟其時。

常服褂,色用石青,花文隨所御,裾左右開。

行褂,色用石青,長與坐齊,袖長及肘。

常服袍,色及花文隨所御,裾四開。行袍同。

行裳,色隨所御。左右各一,前平,後中豐,上下斂。橫幅石青布為之,氈、袷惟時。冬用鹿皮或黑狐為里。

雨冠之制二:冬頂崇,前簷深;夏頂平,前簷敞。皆明黃色,月白緞里。氈及油綢、羽緞惟其時。

雨衣之制六,皆明黃色:一,如常服褂,而長與袍稱。自衽以下加博。上襲重衣。領下為襞積。無袖。斜帷相比,上斂,下遞豐。兩重俱加掩襟,領及鈕約皆青色。一,以氈及羽緞為之,月白緞里。不襲重衣。餘制同。領及鈕約如衣色,油綢為之,不加里。鈕約青色。一,如常服褂而加領,長與袍稱。氈羽緞為之,月白緞里。領及鈕約如衣色。一,如常服袍而袖端平,前施掩襠,油綢不加里。領用青羽緞,鈕約青色。外加袍袖如衣色。一,如常服褂,長與坐齊。氈、羽緞為之,月白緞里。領及鈕約如衣色。一,如常服袍而加領,長與坐齊。油綢為之,不加里。袖端平,前加掩襠,領用青羽緞,鈕約青色。

雨裳之制二,皆明黃色:一,左右幅相交,上斂下遞博。上前加淺帷為襞積。兩旁綴以紐約,青色。腰為橫幅,用石青布,兩末削為帶系之。一,前為完幅,不加淺帷,餘制同。

朝珠,用東珠一百有八,佛頭、記念、背雲、大小墜雜飾,各惟其宜,大典禮御之。惟祀天以青金石為飾,祀地珠用蜜珀,朝日用珊瑚,夕月用綠松石,雜飾惟宜。絳皆明黃色。

朝帶之制二,皆明黃色:一,用龍文金圓版四,飾紅藍寶石或綠松石,每具銜東珠五,圍珍珠二十。左右佩帉,淺藍及白各一,下廣而銳。中約鏤金圓結,飾寶如版,圍珠各三十。佩囊文繡、燧觿、刀削、結佩惟宜,絳皆明黃色,大典禮御之。一,用龍文金方版四,其飾祀天用青金石,祀地用黃玉,朝日用珊瑚,夕月用白玉,每具銜東珠五。佩帉及絳,惟祀天用純青,餘如圓版朝帶之制。中約圓結如版飾,銜東珠四。佩囊純石青,左觿、右削,並從版色。

吉服帶,用明黃色,鏤金版四,方圓惟便,銜珠玉雜寶各從其宜。左右佩帉純白,下直而齊。中約金結如版飾。餘如朝帶制,常服帶同。

行帶,色用明黃,左右佩系以紅香牛皮為之,飾金花文■J4銀鐶各三。佩帉以高麗布,視常服帶帉微闊而短,中約以香牛皮束,綴銀花文佩囊。明黃絳,飾珊瑚。結、削、燧、雜佩各惟其宜。初制,皇帝冠用東珠寶石鑲頂,束金鑲玉版嵌東珠帶。康熙二十三年,定凡大典禮祭壇廟,冠用大珍珠、東珠鑲頂,禮服用黃色、秋香色、藍色五爪、三爪龍緞。雍正元年,定禮服用石青、明黃、大紅、月白四色緞,花樣三色,圓金龍九,龍口珠各一顆。腰襴小團金龍九。周身五彩雲,下八寶平水、萬代江山。

皇后朝冠,冬用薰貂,夏以青絨為之,上綴硃緯。頂三層,貫東珠各一,皆承以金鳳,飾東珠各三,珍珠各十七,上銜大東珠一。硃緯上周綴金鳳七,飾東珠九,貓睛石一,珍珠二十一。後金翟一,飾貓睛石一,珍珠十六。翟尾垂珠,凡珍珠三百有二,五行二就,每行大珍珠一。中間金銜青金石結一,飾東珠、珍珠各六,末綴珊瑚。冠後護領垂明黃絳二,末綴寶石,青緞為帶。

吉服冠,薰貂為之,上綴硃緯。頂用東珠。

金約,鏤金雲十三,飾東珠各一,間以青金石,紅片金里。後系金銜綠松石結,貫珠下垂,凡珍珠三百二十四,五行三就,每行大珍珠一。中間金銜青金石結二,每具飾東珠、珍珠各八,末綴珊瑚。

耳飾,左右各三,每具金龍銜一等東珠各二。

朝褂之制三,皆石青色,片金緣:一,繡文前後立龍各二,下通襞積,四層相間,上為正龍各四,下為萬福萬壽文。一,繡文前後正龍各一,腰帷行龍四,中有襞積。下幅行龍八。一,繡文前後立龍各二,中無襞積。下幅八寶平水。皆垂明黃絳,其飾珠寶惟宜。

朝袍之制三,皆明黃色:一,披領及袖皆石青,片金緣,冬加貂緣,肩上下襲朝褂處亦加緣。繡文金龍九,間以五色雲。中有襞積。下幅八寶平水。披領行龍二,袖端正龍各一,袖相接處行龍各二。一,披領及袖皆石青,夏用片金緣,冬用片雲加海龍緣,肩上下襲朝褂處亦加緣。繡文前後正龍各一,兩肩行龍各一,腰帷行龍四。中有襞積。下幅行龍八。一,領袖片金加海龍緣,夏片金緣。中無襞積。裾後開。餘俱如貂緣朝袍之制。領後垂明黃絳,飾珠寶惟宜。

龍褂之制二,皆石青色:一,繡文五爪金龍八團,兩肩前後正龍各一,襟行龍四。下幅八寶立水。袖端行龍各二。一,下幅及袖端不施章採。

龍袍之制三,皆明黃色,領袖皆石青:一,繡文金龍九,間以五色雲,福壽文採惟宜。下幅八寶立水,領前後正龍各一,左右及交襟處行龍各一。袖如朝袍,裾左右開。一,繡文五爪金龍八團,兩肩前後正龍各一,襟行龍四。下幅八寶立水。一,下幅不施章採。

領約,鏤金為之,飾東珠十一,間以珊瑚。兩端垂明黃絳二,中貫珊瑚,末綴綠松石各二。

朝服朝珠三盤,東珠一,珊瑚二,佛頭、記念、背雲、大小墜珠寶雜飾惟宜。吉服朝珠一盤,珍寶隨所御。絳皆明黃色。

採帨,綠色,繡文為「五穀豐登」。佩箴管、縏袠之屬。絳皆明黃色。

朝裙,冬用片金加海龍緣,上用紅織金壽字緞,下石青行龍妝緞,皆正幅。有襞積。夏以紗為之。

太皇太后、皇太后冠服諸制與皇后同。初制,皇后冠服,凡慶賀大典,冠用東珠鑲頂,禮服用黃色、秋香色五爪龍緞、鳳皇翟鳥等緞。太皇太后、皇太后冠服,凡遇受賀諸慶典,冠用東珠鑲頂,禮服用黃色、秋香色五爪龍緞、繡緞、妝緞。

皇貴妃朝冠,冬用薰貂,夏以青絨為之。上綴硃緯。頂三層,貫東珠各一,皆承以金鳳,飾東珠各三,珍珠各十七,上銜大珍珠一。硃緯上周綴金鳳七,飾東珠各九,珍珠各二十一。後金翟一,飾貓睛石一,珍珠十六,翟尾垂珠,凡珍珠一百九十二,三行二就。中間金銜青金石結一,東珠、珍珠各四,末綴珊瑚。冠後護領垂明黃絳二,末綴寶石。青緞為帶。吉服冠與皇后同。

金約,鏤金雲十二,飾東珠各一,間以珊瑚,紅片金里。後系金銜綠松石結,貫珠下垂,凡珍珠二百有四,三行三就。中間金銜青金石結二,每具飾東珠、珍珠各六,末綴珊瑚。耳飾用二等東珠,餘同皇后。朝褂、朝袍、龍褂、龍袍、採帨、朝裙皆與皇后同。

領約,鏤金為之,飾東珠七,間以珊瑚。兩端垂明黃絳二,中貫珊瑚,末綴珊瑚各二。

朝服朝珠三盤,蜜珀一,珊瑚二。吉服朝珠一盤。絳明黃色。

貴妃冠服袍及垂絳皆金黃色,餘與皇貴妃同。

妃朝冠,頂二層,貫東珠各一,皆承以金鳳,飾東珠九,珍珠十七,上銜貓睛石。硃緯。上周綴金鳳五,飾東珠七,珍珠二十一。後金翟一,飾貓睛石一,珍珠十六,翟尾垂珠,凡珍珠一百八十八,三行二就。中間金銜青金石結一,飾東珠、珍珠各四,末綴珊瑚。冠後護領垂金黃絳二,末綴寶石。青緞為帶。吉服冠頂用碧■。餘同貴妃。

金約,鏤金雲十一,飾東珠各一,間以青金石,紅片金里。後系金銜綠松石結,貫珠下垂,凡珍珠一百九十七,三行三就。中間金銜青金石結二,每具飾東珠、珍珠各六,末綴珊瑚。耳飾用三等東珠。餘同貴妃。朝褂、朝袍、龍褂、龍袍、領約、朝裙、朝珠皆與貴妃同。

採帨,繡文為「雲芝瑞草」。餘與貴妃同。

嬪朝冠,頂二層,貫東珠各一,皆承以金翟,飾東珠九,珍珠十七,上銜子。硃緯。上周綴金翟五,飾東珠五,珍珠十九。後金翟一,飾珍珠十六,翟尾垂珠,凡珍珠一百七十二,三行二就。中間金銜青金石結一,飾東珠、珍珠各三,末綴珊瑚。冠後護領垂金黃絳二,末綴寶石。青緞為帶。吉服冠與妃同。

金約,鏤金雲八,飾東珠各一,間以青金石,紅片金里。後系金銜綠松石結,貫珠下垂,凡珍珠一百七十七,三行二就。中間金銜青金石結二,每具飾東珠、珍珠各四,末綴珊瑚。耳飾用四等東珠。餘與妃同。

朝褂,與妃同。龍褂,繡文兩肩前後正龍各一,襟夔龍四。餘同妃制。朝袍、龍袍俱用香色。餘與妃同。

朝服朝珠三盤,珊瑚一、蜜珀二。吉服朝珠一盤。絳用金黃色。領約、朝裙皆與妃同。採帨不繡花文。餘同妃制。初制,皇貴妃、貴妃、妃、嬪冠服,凡慶賀大典,皇貴妃、貴妃冠頂用東珠十二顆,妃冠頂用東珠十一顆。禮服用鳳凰、翟鳥等緞,五爪龍緞、妝緞、八團龍等緞。至黃色、秋香色,自皇貴妃以下,概不許服。嬪冠頂用東珠十顆,禮服用翟鳥等緞,五爪龍緞、妝緞、四團龍等緞。

皇子朝冠,冬用薰貂、青狐惟其時。上綴硃緯。頂金龍二層,飾東珠十,銜紅寶石。夏織玉草或藤竹絲為之。石青片金緣二層,裡用紅片金或紅紗。上綴硃緯。前綴舍林,飾東珠五。後綴金花,飾東珠四。頂如冬朝冠,吉服冠紅絨結頂。

端罩,紫貂為之,金黃緞里。左右垂帶各二,下廣而銳,色與里同。龍褂,色用石青。正面繡五爪金龍四團,兩肩前後各一,間以五色雲。

朝服之制二,皆金黃色:一,披領及裳俱表以紫貂。袖端薰貂。繡文兩肩前後正龍各一,襞積行龍六,間以五色雲。一,披領及袖俱石青,片金緣,冬加海龍緣。繡文兩肩前後正龍各一,腰帷行龍四,裳行龍八,披領行龍二,袖端正龍各一。下幅八寶平水。蟒袍亦金黃色,片金緣,繡文九蟒,裾左、右開。

朝珠不得用東珠,餘隨所用,絳皆金黃色。

朝帶,色用金黃,金銜玉方版四,每具飾東珠四,中銜貓睛石一,左右佩絳如帶色。吉服帶亦色用金黃,版飾惟宜,佩絳如帶色。

雨冠、雨衣、雨裳,均用紅色,氈、羽紗、油綢,各惟其時。初制,皇子冠服,凡慶賀大典,冠用東珠十三顆鑲頂,禮服用秋香等色,五爪、三爪龍緞,滿翠八團龍等緞,束金鑲玉嵌東珠帶。

親王朝冠,與皇子同。吉服冠,冬用海龍、薰貂、紫貂惟其時。夏織玉草或藤竹絲為之。紅紗綢里。石青片金緣。上綴硃緯。頂用紅寶石,曾賜紅絨結頂者,亦得用之。

端罩,青狐為之,月白緞里,若曾賜金黃色者,亦得用之。補服用石青色,繡五爪金龍四團,前後正龍,兩肩行龍。朝服、蟒袍藍及石青隨所用,若曾賜金黃色者,亦得用之。餘與皇子同。

朝珠、朝帶、吉服帶、雨冠、雨衣、雨裳,均與皇子同。崇德元年,定親王冠頂三層,上銜紅寶石,中嵌東珠八。前舍林,嵌東珠四。後金花,嵌東珠三。帶用金鑲玉版四片,嵌東珠四。順治九年,定冠頂共嵌東珠十,舍林、金花各增嵌東珠一。帶四片,每片嵌東珠四。服用五爪四團龍補、五爪龍緞、滿翠四補等緞。

親王世子朝冠,頂金龍二層,飾東珠九,上銜紅寶石。夏朝冠前綴舍林,飾東珠五。後綴金花,飾東珠四。吉服冠、端罩、補服、朝服、蟒袍、朝珠皆與親王同。

朝帶,色用金黃,金銜玉方版四,每具飾東珠三。左右佩絳如帶色。吉服帶與親王同。順治九年,定親王世子冠頂三層,共嵌東珠九。帶用金鑲玉版四片,每片嵌東珠三。服與親王同。

郡王朝冠,頂金龍二層,飾東珠八,上銜紅寶石。夏朝冠前綴舍林,飾東珠四。後綴金花,飾東珠三。吉服冠、端罩皆與親王世子同。

補服,用石青色,繡五爪行龍四團,兩肩前後各一。朝服、蟒袍、朝珠皆與親王世子同。

朝帶,色用金黃,金銜玉方版四,每具飾東珠二,貓睛石一。佩絳如帶色。吉服帶與親王世子同。崇德元年,定郡王冠頂三層,上銜紅寶石,中嵌東珠七。前舍林,嵌東珠三。後金花,嵌東珠二。帶用金鑲玉版四片,嵌綠松石四。順治九年,定冠頂共嵌東珠八,舍林、金花各增嵌東珠一。帶四片,每片嵌東珠二。服與親王同。

貝勒朝冠,頂金龍二層,飾東珠七,上銜紅寶石。夏朝冠前綴舍林,飾東珠三。後綴金花,飾東珠二。吉服冠、端罩皆與郡王同。

補服,色用石青,前後繡四爪正蟒各一團,朝服通繡四爪蟒文,蟒袍亦如之,均不得用金黃色,餘隨所用。朝珠絳用石青色。餘同郡王。朝帶色用金黃,金銜玉方版四,每具飾東珠二。佩絳皆石青色,吉服帶色用金黃,版飾惟宜。佩絳亦皆石青色。崇德元年,定貝勒冠頂三層,上銜紅寶石,中嵌東珠六。前舍林,綴東珠二。後金花,綴東珠一。帶用金鑲玉版四片,嵌寶石四。順治九年,定冠頂共嵌東珠七,舍林、金花各增嵌東珠一。帶四片,每片嵌東珠二。服用四爪兩團龍補及蟒緞、妝緞。

貝子朝冠,頂金龍二層,飾東珠六,上銜紅寶石。夏朝冠前綴舍林,飾東珠二。後綴金花,飾東珠一。吉服冠頂用紅寶石。皆戴三眼孔雀翎。孔雀花翎有三眼、雙眼、單眼之分,遇賞均得戴用。端罩制同貝勒。補服色用石青,前後繡四爪行蟒各一團。朝服、蟒袍、朝珠皆與貝勒同。

朝帶,色用金黃,金銜玉方版四,每具飾東珠一。吉服帶與貝勒同。崇德元年,定貝子冠頂二層。上銜紅寶石,中嵌東珠五。前舍林,後金花,各嵌東珠一。帶用金鑲玉版四片,每片嵌藍寶石一。順治九年,定冠頂共嵌東珠六,舍林增嵌東珠一,餘如舊。帶四片,每片嵌東珠一。服與貝勒同。

鎮國公朝冠,頂金龍二層,飾東珠五,上銜紅寶石。夏朝冠前綴舍林,飾東珠一。後綴金花,飾綠松石一。吉服冠,入八分公頂用紅寶石,未入八分公用珊瑚,皆戴雙眼孔雀翎。端罩紫貂為之,月白緞里。補服前後繡四爪正蟒方補。朝服、蟒袍、朝珠與貝子同。

朝帶,金銜玉方版四,每具飾貓睛石一。吉服帶與貝子同。崇德元年,定鎮國公冠頂二層,上銜紅寶石,中嵌東珠四。前舍林,嵌東珠一。後金花,嵌綠松石一。帶如貝子。順治九年,定冠頂共嵌東珠五,餘如舊。帶四片,每片嵌貓睛石一。服用四爪方蟒補。餘與貝勒同。

輔國公朝冠,頂金龍二層,飾東珠四,上銜紅寶石。餘皆如鎮國公。崇德元年,定輔國公冠頂二層,上銜紅寶石,中嵌東珠三。前舍林,嵌綠松石一。後金花,嵌寶石一。帶如鎮國公。順治九年,定冠頂共嵌東珠四,舍林、金花、帶、服色俱與鎮國公同。

鎮國將軍朝冠,頂鏤花金座,中飾東珠一,上銜紅寶石。吉服冠頂用珊瑚。補服前後繡麒麟。餘皆視武一品。崇德元年,定鎮國將軍冠頂上銜紅寶石,帶用金鑲圓版,嵌紅寶石四。順治九年,定冠頂中節嵌東珠,帶用金鑲方玉版,各嵌紅寶石一。補服繡麒麟,餘與鎮國公同。

輔國將軍朝冠,頂鏤花金座,中飾小紅寶石,上銜鏤花珊瑚。吉服冠頂亦用鏤花珊瑚。補服前後繡獅。餘皆視武二品。崇德元年,定輔國將軍冠頂上銜藍寶石,帶用圓金版。順治九年,定冠頂改銜紅寶石,中節嵌小紅寶石一。帶如鎮國將軍。補服繡獅。餘與鎮國公同。

奉國將軍朝冠,頂鏤花金座,中飾小紅寶石一,上銜藍寶石。吉服冠頂亦用藍寶石。補服前後繡豹。餘皆視武三品。崇德元年,定奉國將軍冠頂上銜水晶石,帶用玲瓏■J4金方鐵版。順治九年,定冠頂上銜紅寶石,中節嵌小藍寶石一。帶用起花金圓版。補服繡豹。餘與鎮國公同。

奉恩將軍朝冠,頂鏤花金座,中飾小藍寶石一,上銜青金石。補服前後繡虎,餘皆視武四品,惟衣裾四啟。帶用金黃色,凡宗室皆如之,覺羅用紅色。順治九年,定奉恩將軍冠頂上銜藍寶石,中節嵌小藍寶石一。帶用起花金鑲銀圓版。補服繡虎,餘與鎮國公同。

固倫額駙吉服冠,頂用紅寶石,戴三眼孔雀翎。吉服帶用金黃色。餘與貝子同。崇德元年,定固倫額駙冠服與貝子同。順治八年,定冠頂嵌東珠六。舍林嵌東珠二。金花嵌東珠一。帶用金鑲玉圓版四片,每片嵌東珠一。

和碩額駙吉服冠,頂用珊瑚,戴雙眼孔雀翎。朝帶色用石青或藍,金銜玉圓版四。餘與鎮國公同。崇德元年,定和碩額駙冠服與超品公同,如封爵在公以上者,仍照本階服用。順治八年,定冠頂嵌東珠四,舍林嵌東珠一。金花嵌綠松石一。帶用金鑲玉圓版四片,每片嵌貓睛石一。

郡主額駙朝帶,用鏤金圓版四,每具飾綠松石一。餘視武一品。崇德元年,定郡主額駙冠頂上銜紅寶石,嵌東珠一。帶用金圓版四片,嵌綠松石四。順治八年,定冠、帶與侯、伯同。康熙元年,定用四爪蟒補服。

縣主額駙冠服,視武二品。崇德元年,定縣主額駙冠頂上銜紅寶石。帶用金圓版四片,每片嵌紅寶石四。

郡君額駙冠服,視武三品。崇德元年,定郡君額駙冠頂上嵌藍寶石。帶用金圓版四片。

縣君額駙朝帶,用■J4金方鐵版四。餘與武四品同。崇德元年,定縣君額駙冠頂上銜水晶石。帶用■J4金方鐵版四片。

鄉君額駙朝帶,用■J4金方鐵版四。餘與武五品同。崇德元年,定鄉君額駙冠用金頂。帶用■J4金圓鐵版四片。並按固倫額駙若爵在貝子以上、和碩額駙爵在鎮國公以上者,冠服各從其品。郡主額駙以下皆如之。

民公朝冠,冬用薰貂,十一月朔至上元用青狐。頂鏤花金座,中飾東珠四,上銜紅寶石,夏頂制同。吉服冠頂用珊瑚。

端罩,貂皮為之,藍緞里。補服,色用石青,前後繡四爪正蟒。

朝服,藍及石青諸色隨所用。披領及袖俱石青,片金緣,冬加海龍緣。兩肩前後正蟒各一,腰帷行蟒四,中有襞積。裳行蟒八。十一月朔至上元,披領及裳俱表以紫貂,袖端薰貂。兩肩前後正蟒各一,襞積行蟒四,皆四爪。曾賜五爪蟒緞者,亦得用之。蟒袍,藍及石青諸色隨所用,通繡九蟒。

朝珠,珊瑚青金綠松蜜珀隨所用,雜飾惟宜。絳用石青色,朝帶色用石青或藍,鏤金玉圓版四,每具飾貓睛石一。佩帉下廣而銳,吉服帶佩帉下直而齊,版飾惟宜。雨冠、雨衣、雨裳俱用紅色。崇德元年,定民公冠頂上銜紅寶石,中嵌東珠一。帶用金圓版四片,嵌綠松石四。順治二年,定冠用起花金頂,上銜紅寶石,中嵌東珠三。帶用金鑲圓玉版四片,各嵌綠松石一。八年,定冠頂嵌東珠四,帶片各嵌貓睛石一。

侯朝冠,頂鏤花金座,中飾東珠三,上銜紅寶石。朝帶鏤金銜玉圓版四,每具飾綠松石一。餘皆如公。

伯朝冠,頂鏤花金座,中飾東珠二,上銜紅寶石。朝帶鏤金銜玉圓版四,每具飾紅寶石一。餘皆如侯。

子朝冠,頂鏤花金座,中飾東珠一,上銜紅寶石,補服前後繡麒麟。餘皆視武一品。

男朝冠,頂鏤花金座,中飾小紅寶石,上銜鏤花珊瑚。補服前後繡獅。餘皆視武二品。順治二年,定侯、伯冠用起花金頂,上銜紅寶石,中嵌東珠一。帶用金鑲方玉版四片,每片嵌紅寶石一。六年,定冠頂嵌東珠二,帶改用圓玉版。八年,定侯冠頂東珠三。帶片各嵌綠松石一。

皇子福晉朝冠,頂鏤金三層,飾東珠十,上銜紅寶石。硃緯。上周綴金孔雀五,飾東珠七,小珍珠三十九。後金孔雀一,垂珠三行二就。中間金銜青金石結一,飾東珠各三,末綴珊瑚。冠後護領垂金黃絳二,末亦綴珊瑚。青緞為帶。吉服冠頂用紅寶石。

金約,鏤金云九,飾東珠各一,間以青金石,紅片金里。後系金銜青金石結,貫珠下垂,三行三就。中間金銜青金石結二、每具飾東珠珍珠各四,末綴珊瑚。耳飾左右各三,每具金雲銜珠各二。

朝褂,色用石青,片金緣。繡文前行龍四,後行龍三。領後垂金黃絳,雜飾惟宜。吉服褂色用石青,繡五爪正龍四團,前後兩肩各一。朝袍用香色,披領及袖皆石青,片金緣,冬加海龍緣。肩上下襲朝褂處亦加緣,繡文前後正龍各一,兩肩行龍各一,襟行龍四,披領行龍二,袖端正龍各一,袖相接處行龍各二。裾後開。領後垂金黃絳,雜飾惟宜。蟒袍用香色,通繡九龍。

領約,鏤金為之,飾東珠七,間以珊瑚。兩端垂金黃絳二,中貫珊瑚,末綴珊瑚各二。採帨月白色,不繡花文,結佩惟宜。絳皆金黃色。朝裙片金緣,冬加海龍緣,上用紅緞,下石青行龍妝緞,皆正幅,有襞積。夏以紗為之。

朝服朝珠三盤,珊瑚一,蜜珀二。吉服朝珠一盤。珍寶隨所御。絳皆金黃色。

親王福晉吉服褂,繡五爪金龍四團,前後正龍,兩肩行龍。餘皆與皇子福晉同。側福晉冠頂等各飾東珠九。服與嫡福晉同。並按崇德元年,定親王嫡妃冠頂嵌東珠八,側妃嵌東珠七。順治九年,定嫡妃冠頂增嵌東珠二。服用翟鳥四團龍補、五爪龍緞、妝緞、滿翠四補等緞。側妃冠頂增嵌東珠二。服與嫡妃同。

世子福晉朝冠,頂鏤金二層,飾東珠九,上銜紅寶石。硃緯。上周綴金孔雀五,飾東珠各六。後金孔雀一,垂珠三行二就。中間金銜青金石結一,飾東珠各三,末綴珊瑚。冠後護領垂金黃絳二,末亦綴珊瑚。青綴為帶。

金約,鏤金雲八,飾東珠各一,間以青金石。後系金銜青金石結,垂珠三行三就。中間金銜青金石結二,每具飾東珠珍珠各四,末綴珊瑚。餘皆與親王福晉同。順治九年,定世子嫡妃冠服如親王側妃。其側妃冠頂嵌東珠八。服與嫡妃同。

郡王福晉朝冠,頂鏤金二層,飾東珠八,上銜紅寶石。硃緯。上周綴金孔雀五,飾東珠各五。後金孔雀一,垂珠三行二就。中間金銜青金石結一,末綴珊瑚。冠後護領垂金黃絳二,末亦綴珊瑚。青緞為帶。吉服冠與世子福晉同。

金約,鏤金雲八,飾東珠各一,間以青金石。後系金銜青金石結,垂珠三行三就。中間金銜青金石結二,末綴珊瑚。

吉服褂,繡五爪行龍四團,前後兩肩各一。餘皆與世子福晉同。崇德元年,定郡王嫡妃冠頂嵌東珠七,側妃嵌東珠六。順治九年,定嫡妃冠服與世子側妃同。其側妃冠頂嵌東珠七。服用蟒緞、妝緞,各色花、表緞。

貝勒夫人朝冠,頂鏤金二層,飾東珠七,上銜紅寶石。硃緯。上周綴金孔雀五,飾東珠各三。後金孔雀一,垂珠三行二就。中間金銜青金石結一,末綴珊瑚。冠後護領垂石青絳二,末亦綴珊瑚。吉服冠與郡王福晉同。

金約,鏤金雲七。餘同郡王福晉。耳飾亦與郡王福晉同。

朝褂,繡四爪蟒,領後垂石青絳。吉服褂前後繡四爪正蟒各一。餘與郡王福晉同。

朝袍,藍及石青諸色隨所用,領、袖片金緣,冬用片金加海龍緣。繡四爪蟒,領後垂石青絳。蟒袍通繡九蟒。領約、朝珠、採帨絳用石青色。餘皆與郡王福晉同。崇德元年,定貝勒嫡夫人冠頂嵌東珠六。側夫人嵌東珠五。順治九年,定嫡夫人冠頂、服飾如郡王側妃,其側夫人冠頂嵌東珠六。服與嫡夫人同。

貝子夫人朝冠,頂鏤金二層,飾東珠六。金約鏤金雲六,吉服褂前後繡四爪行蟒各一。餘皆與貝勒夫人同,崇德元年,定貝子嫡夫人冠頂嵌東珠五。側夫人嵌東珠四。順治九年,定嫡夫人冠頂服飾如郡王側妃。其側夫人冠頂嵌東珠五。服與嫡夫人同。

鎮國公夫人朝冠,頂鏤金二層,飾東珠五。金約鏤金雲五。吉服褂繡花八團。餘皆與貝子夫人同。崇德元年,定鎮國公嫡夫人冠頂嵌東珠四。順治九年,定嵌東珠五。服如貝子夫人。其側夫人冠頂嵌東珠四。服與嫡夫人同。

輔國公夫人朝冠,頂鏤金二層,飾東珠四。金約鏤金雲四。餘皆與鎮國公夫人同。崇德元年,定輔國公夫人冠頂嵌東珠三。順治九年,定冠頂嵌東珠四。服如貝子夫人。其側夫人冠頂嵌東珠三。服與嫡夫人同。

鎮國將軍夫人冠、服均視一品命婦。

輔國將軍夫人冠、服均視二品命婦。

奉國將軍淑人冠、服均視三品命婦。

奉恩將軍恭人冠、服均視四品命婦。

固倫公主冠、服制如親王福晉。崇德元年,定固倫公主冠頂嵌東珠八。順治九年,定冠頂增嵌東珠二。服用翟鳥五爪四團龍補、五爪龍緞、妝緞、滿翠四補等緞。

和碩公主朝冠、金約,制如親王世子福晉。餘與固倫公主同。崇德元年,定和碩公主冠頂嵌東珠六。順治九年,定冠頂增嵌東珠二。服與固倫公主同。

郡主朝冠、金約,制如郡王福晉。餘與和碩公主同。崇德元年,定郡主冠頂嵌東珠六。順治九年,定冠頂增嵌東珠二。服與和碩公主同。

縣主朝冠、金約,制如貝勒夫人。吉服褂制如郡王福晉。餘與郡主同。崇德元年,定縣主冠頂嵌東珠五。順治九年,定冠頂增嵌東珠二。服用蟒緞、妝緞,各樣花、素緞。

郡君朝冠、金約,制如貝子夫人。朝褂、龍袍、領約、朝珠、採帨、吉服褂、蟒袍均如貝勒夫人。餘同縣主。崇德元年,定郡君冠頂嵌東珠四。順治九年,定冠服與縣主同。

縣君朝冠、金約,制如鎮國公夫人。吉服褂制如貝子夫人。餘皆與郡君同。崇德元年,定縣君冠頂嵌東珠三。順治九年,定冠頂增嵌東珠二。服與郡君同。

鎮國公女鄉君朝冠、金約,制如輔國公夫人。吉服褂制如鎮國公夫人。餘同縣君。

輔國公女鄉君朝冠,頂鏤金二層,飾東珠三。金約鏤金雲三。餘與鎮國公女鄉君同。崇德元年,定鄉君冠頂嵌東珠二。順治九年,定鎮國公女鄉君冠頂嵌東珠三。服與縣君同。

王、貝勒側室女,封授視嫡降二等。冠、服各視所降品級服用。貝子、鎮國公、輔國公側室女,雖降等食五品、六品俸,其冠服仍與鄉君同。

民公夫人朝冠,冬用薰貂,夏以青絨為之。頂鏤花金座,飾東珠四,上銜紅寶石。前綴金簪三,飾以珠寶。護領絳用石青色。吉服冠,薰貂為之,頂用珊瑚。金約青緞為之,紅片金里。中綴鏤金火焰,飾珍珠一,左右金龍鳳各一。後垂青緞帶二,亦紅片金里。耳飾左右各三,每具金雲銜珠各二。

朝褂,色用石青,片金緣。繡文前行蟒二,後行蟒一。領後垂石青★,雜佩惟宜。朝袍,藍及石青諸色隨所用。披領及袖皆石青,冬用片金加海龍緣。繡文前後正蟒各一,兩肩行蟒各一,襟行蟒四,中無襞積。披領行蟒二,袖端正蟒各一,袖相接處行蟒各二。後垂石青絳,雜佩惟宜。吉服褂色用石青,繡花八團。

蟒袍,藍及石青諸色隨所用,通四爪九蟒。領約鏤金為之,飾紅藍小寶石五。兩端垂石青絳二,中貫珊瑚。末綴珊瑚各二。

朝珠,朝服用三,吉服用一。珊瑚、青金、蜜珀、綠松隨所用,雜飾惟宜。絳用石青色。採帨,月白色,不繡花,雜飾惟宜。絳皆石青色。朝裙,夏片金緣,冬加海龍緣,上用紅緞,下石青行蟒、妝緞,皆正幅,有襞積。崇德元年,定未入八分公夫人冠頂服飾,惟正室視其夫品級服用。

侯夫人朝冠,頂鏤花金座,中飾東珠三,上銜紅寶石,餘皆如民公夫人。

伯夫人朝冠,頂鏤花金座,中飾東珠二,上銜紅寶石,餘皆如侯夫人。

子夫人朝冠,頂鏤花金座,中飾東珠一,上銜紅寶石,餘皆如伯夫人。

男夫人朝冠,頂鏤花金座,中飾紅寶石一,上銜鏤花紅珊瑚。吉服冠頂鏤花珊瑚。餘皆如子夫人。

文一品朝冠,頂鏤花金座,中飾東珠一,上銜紅寶石。補服前後繡鶴,惟都御史繡獬豸。朝帶鏤金銜玉方版四,每具飾紅寶石一。餘皆如公。

武一品補服,前後繡麒麟。餘皆如文一品。

文二品朝冠,冬用薰貂,十一月至上元用貂尾,頂鏤花金座,中飾小紅寶石一,上銜鏤花珊瑚。吉服冠頂亦用鏤花珊瑚。補服前後繡錦雞。朝帶鏤金圓版四,每具飾紅寶石一。餘皆如文一品。

武二品補服,前後繡獅。餘皆如文二品。

文三品朝冠,頂鏤花金座,中飾小紅寶石一,上銜藍寶石。吉服冠頂亦用藍寶石。補服前後繡孔雀,惟副都御史及按察使前後繡獬豸。朝帶鏤花金圓版。餘皆如文二品。

武三品朝冠,冬用薰貂,補服前後繡豹。餘皆如文三品。惟朝服無貂緣及無端罩。一等侍衛戴孔雀翎。端罩猞猁猻,間以貂皮,月白緞里。餘如武三品。

文四品朝冠,頂鏤花金座,中飾藍寶石一,上銜青金石。吉服冠頂亦用青金石。補服前後繡雁,惟道繡獬豸。蟒袍通繡四爪八蟒。朝帶銀銜鏤花金圓版四。餘皆如文三品。

武四品補服,前後繡虎。餘皆如文四品。二等侍衛戴孔雀翎。端罩紅豹皮為之,素紅緞里。朝服冬、夏均翦絨緣,色用石青,通身雲緞,前後方襴行蟒各一,腰帷行蟒四,中有襞積。領、袖俱石青妝緞,餘如武四品。

文五品朝冠,頂鏤花金座,中飾小藍寶石一,上銜水晶石。吉服冠頂亦用水晶。補服前後繡白鷴,惟給事中、御史繡獬豸。朝服色用石青,片金緣,通身雲緞,前後方襴行蟒各一,中有襞積。領、袖俱用石青妝緞。朝帶銀銜素金圓版四。餘皆如文四品。

武五品補服,前後繡熊。餘皆如文五品。惟無朝珠。三等侍衛戴孔雀翎。端罩黃狐皮為之,月白緞里。朝服冬、夏俱翦絨緣。餘如武五品,惟得用朝珠。

文六品朝冠,頂鏤花金座,中飾小藍寶石一,上銜硨磲。吉服冠頂亦用硨磲。補服前後繡鷺鷥,朝帶銀銜玳瑁圓版四。餘皆如文五品,惟無朝珠。五品官以下,惟京堂、翰詹、科道得用貂裘、朝珠。六品官以下,惟太常寺、鴻臚寺、光祿寺、國子監所屬官,壇廟執事、殿庭侍儀得用朝珠。

武六品補服,前後繡彪。餘皆如文六品。藍翎侍衛朝冠頂飾小藍寶石一,上銜硨磲,戴藍翎。端罩、朝服、朝珠均同三等侍衛。餘如武六品。

文七品朝冠,頂鏤花金座,中飾小水晶一,上銜素金。吉服冠頂亦用素金。補服前後繡鸂鶒,朝帶素圓版四。蟒袍通繡四爪五蟒。餘皆如文六品。

武七品補服,前後繡犀牛。餘皆如文七品。

文八品朝冠,鏤花陰文,金頂無飾。吉服冠同。補服前後繡鵪鶉。朝服色用石青雲緞,無蟒。領、袖冬、夏皆青倭緞,中有襞積。朝帶銀銜明羊角圓版四。餘皆如文七品。

武八品補服如武七品。餘皆如文八品。

文九品朝冠,鏤花陽文,金頂。吉服冠同。補服前後繡練雀。朝帶銀銜烏角圓版四。餘皆如文八品。

武九品補服,前後繡海馬。餘皆如文九品。

未入流冠服制如文九品。

凡雨冠,民公、侯、伯、子、男,一、二、三品文、武官,御前侍衛,乾清門侍衛,上書房、南書房翰林,批本處行走人員,皆用紅色。四、五、六品文、武官,雨冠中用紅色,青緣。七、八、九、品文、武官,雨冠中用青色,紅緣。雨衣、雨裳,民公、侯、伯、子,文、武一品官,御前侍衛,各省督、撫,皆用紅色。二品以下文、武官,皆用青色。其明黃色行褂,則領侍衛大臣、御前大臣、侍衛班長、護軍統領、健銳營翼領及凡諸臣之蒙賜者,皆得用之。

凡帶,親王以下、宗室以上,皆束金黃帶。覺羅紅帶。其金黃帶、紅帶,非上賜者,不得給予異姓。

凡朝珠,王公以下,文職五品、武職四品以上及翰詹、科道、侍衛,公主、福晉以下,五品官命婦以上均得用。以雜寶及諸香為之。禮部主事,太常寺博士、典簿、讀祝官、贊禮郎,鴻臚寺鳴贊,光祿寺署正、署丞、典簿,國子監監丞、博士、助教、學正、學錄,除在壇廟執事及殿廷侍儀準用,其平時燕處及在公署,仍不得用。

凡孔雀翎,翎端三眼者,貝子戴之。二眼者,鎮國公、輔國公、和碩額駙戴之。一眼者,內大臣,一、二、三、四等侍衛,前鋒、護軍各統領、參領,前鋒侍衛,諸王府長史,散騎郎,二等護衛,均得戴之。翎根並綴藍翎。貝勒府司儀長,親王以下二、三等護衛及前鋒、親軍、護軍校,均戴染藍翎。

凡坐褥,親王冬用貂,夏用龍文赤繒。世子、郡王冬用猞猁猻、緣貂,夏蟒文青繒。貝勒冬用猞猁猻,夏青繒施採。貝子冬用白豹,夏採繒緣青繒。均藉紅白氈。鎮國公冬用全赤豹皮,夏青花赤繒。輔國公冬用方赤豹皮,夏赤花皁繒。均藉紅氈。鎮國將軍視一品,輔國將軍視二品,奉國將軍視三品,奉恩將軍視四品。民公冬用全虎皮,夏皁繒。侯、伯冬均用方虎皮,夏侯用緣花皁繒。伯用青雲繒。均藉紅氈。子、男各從其品。固倫公主額駙視貝子。和碩公主額駙視鎮國公。郡主額駙冬用貛,夏皁褐緣紅褐。均藉紅氈。郡君額駙視三品。縣君額駙視四品。鄉君額駙視五品。文、武官一品冬用狼,夏紅褐。二品冬用貛,夏紅褐緣皁褐。三品冬用貉,夏皁褐緣紅褐。四品冬用青山羊,夏皁布。均藉紅氈。五品冬用青羊,夏青布。六品冬用黑羊,夏椶色布。七品冬用鹿,夏灰色布。八品冬用包,夏土布。九品冬用獺,夏與八品同。均藉白氈。

凡寒燠更用冠服,每歲春季用涼朝冠及夾朝衣,秋季用暖朝冠及緣皮朝衣。於三、九月內,或初五日,或十五日,或二十五日,酌擬一日。均前一月由禮部奏請,得旨,通行各衙門一體遵照。

凡文、武候補、候選官頂帶均與現任同。崇德元年,定都統、尚書冠頂上銜紅寶石。帶用金圓版四片,嵌紅寶石四。內大臣、大學士、副都統、護軍統領、前鋒統領、侍郎冠頂上銜藍寶石。帶用金圓版四片。一等侍衛、護衛參領、學士、滿啟心郎、郎中冠頂上銜水晶。帶用■J4金鐵版四片。二等、三等侍衛,護衛,佐領,漢啟心郎,員外郎冠用金頂。帶用■J4金圓鐵版四片。護軍校、主事冠用金頂。帶用■J4金圓鐵版二片。順治二年,定一品官冠用起花金頂,上銜紅寶石,中嵌東珠一。帶用金鑲方玉版四片,每片嵌紅寶石一。二品官冠用起花金頂,上銜紅寶石,中嵌小紅寶石。帶用起花金圓版四片,嵌紅寶石一。三品官冠用起花金頂,上銜紅寶石,中嵌小藍寶石。帶用起花金圓版四片。四品官冠用起花金頂,上銜藍寶石,中嵌小藍寶石。帶用起花金圓版四片,銀鑲邊。五品官冠用起花金頂,上銜水晶,中嵌小藍寶石。帶用素金圓版四片,銀鑲邊。六品官冠用起花金頂,上銜水晶。帶用玳瑁圓版四片,銀鑲邊。七品官冠用起花金頂,中嵌小藍寶石。帶用素銀圓版四片。八品官冠用起花金頂。帶用明羊角圓版四片,銀鑲邊。九品官冠用起花銀頂。帶用烏角圓版四片,銀鑲邊。順治九年,定武官補服一品、二品用獅,三品用虎,四品用豹。又雍正五年,定奉國將軍及三品官冠用起花珊瑚頂。六品官冠用水晶石頂。

一品命婦朝冠,頂鏤花金座,中飾東珠一,上銜紅寶石。餘皆如民公夫人。

二品命婦朝冠,頂鏤花金座,中飾紅寶石一,上銜鏤花珊瑚。吉服冠頂亦用鏤花珊瑚。餘皆如一品命婦。

三品命婦朝冠,頂鏤花金座,中飾紅寶石一,上銜藍寶石。吉服冠頂亦用藍寶石。餘皆如二品命婦。

四品命婦朝冠,頂鏤花金座,中飾小藍寶石一,上銜青金石。吉服冠頂亦用青金石,朝袍片金緣,繡文前後行蟒各二,中無襞積。後垂石青絳,雜飾惟宜。蟒袍通繡四爪八蟒。朝裙片金緣,上用綠緞,下石青行蟒妝緞,均正幅,有襞積。餘皆如三品命婦。

五品命婦朝冠,頂鏤花金座,中飾小藍寶石一,上銜水晶。吉服冠頂亦用水晶。餘皆如四品命婦。

六品命婦朝冠,頂鏤花金座,中飾小藍寶石一,上銜硨磲。吉服冠頂亦用硨磲。餘皆如五品命婦。

七品命婦朝冠,頂鏤花金座,中飾小水晶一,上銜素金。吉服冠頂亦用素金。蟒袍通繡五蟒。餘皆如六品命婦。崇德元年,定命婦冠、服各視其夫官階。皇后侍從婦女冠用金頂,上銜紅寶石。貴妃侍從婦女冠用金頂,上銜水晶石。親、郡王妃侍從婦女與妃侍從婦女同。貝勒夫人侍從婦女冠用金頂。貝子夫人侍從婦女冠不用頂。首飾嵌珍珠、寶石、綠松石。

會試中式貢士朝冠,頂鏤花金座,上銜金三枝九葉。吉服冠頂用素金。狀元金頂,上銜水晶。授職後,各視其品。舉人公服冠,頂鏤花銀座,上銜金雀。公服袍,青綢藍緣。披領如袍式。公服帶,制如文八品朝帶。吉服冠,頂銀座,上銜素金。貢生吉服冠,鏤花金頂。餘同舉人。監生吉服冠,素銀頂。餘同貢生。生員冠,頂鏤花銀座,上銜銀雀。公服袍,藍綢青緣。披領如袍式。公服帶,制如文九品朝帶。吉服冠,頂與監生同。外郎、耆老,冠頂以錫。從耕農官,袍以青絨為之。頂同八品。祭祀文舞生冬冠,騷鼠為之,頂鏤花銅座,中飾方銅,鏤葵花,上銜銅三角,如火珠形。袍以綢為之,其色南郊用石青,北郊用黑,各壇廟俱用紅,惟夕月壇用月白。前後方襴銷金葵花。帶用綠綢。武舞生冠頂上銜銅三棱,如古戟形。袍以綢為之,通銷金葵花。餘俱與文舞生同。樂部樂生,冠頂鏤花銅座,上植明黃翎。樂部袍紅緞為之,一,前後方襴繡黃鸝,中和韶樂部樂生執戲竹人服之;一,通織小團葵花,丹陛大樂諸部樂生服之。帶均用綠雲緞。鹵簿輿士冬冠,以豹皮及黑氈為之,頂鏤花銅座,上植明黃翎,袍如丹陛大樂諸部樂生。帶如祭祀文舞生。鹵薄護軍袍石青緞為之,通織金壽字,片金緣。領、袖俱織金葵花。鹵簿校尉冬冠,平簷,頂素銅,上植明黃翎。袍、帶俱同鹵簿輿士。順治三年,定庶民不得用緞繡等服。滿洲家下僕隸有用蟒緞、妝緞、錦繡服飾者,嚴禁之。九年,定涼帽、暖帽圓月,惟職官用紅片金,庶人則用紅緞。僧道服,袈裟、道服外,許用紬絹紡絲素紗各色,布袍用土黑、糸由黑二色。康熙元年,定軍民人等有用蟒緞、妝緞、金花緞、片金倭緞、貂皮、狐皮、猞猁猻為服飾者,禁之。三十九年,定八旗舉人、官生、貢生、生員、監生、護軍、領催許服平常緞紗。天馬、銀鼠不得服用。漢舉人、官生、貢生、監生、生員除狼皮外,例亦如之。軍民胥吏不得用狼狐等皮。有以貂皮為帽者,並禁之。又兵民人等鞍轡不得用繡緞、倭緞、搭線、鑲緣及鍍金為飾。雍正元年,以職官不按定例,懸帶數珠,馬項下懸紅纓,使人前馬。又有越分者,坐褥至以綢為之。令八旗大臣、統領衙門及都察院嚴行稽察,如大臣等徇情疏忽,同罪。至諸王間賞所屬人員數珠等物,並行文本旗記檔,歲應匯奏。二年,又申明加級官員頂帶、補服、坐褥越級僭用之禁。官員軍民服色有用黑狐皮、秋香色、米色、香色及鞍轡用米色、秋香色者,於定例外,加罪議處。該管官員不行舉發亦如之。


\end{pinyinscope}