\article{志五十八}

\begin{pinyinscope}
禮二(吉禮二)

郊社儀制郊社配饗祈穀雩祀天神太歲朝日夕月

社稷先農先蠶地祇岳鎮海瀆山川直省神祇

郊祀之制太祖御極,焚香告天,建元天命。天聰十年,設圜丘德盛門外,方澤內治門外,壇壝始備。會征服察哈爾,獲元玉璽,躬親告祭,遂祀天南郊。舊制,祭饗用生牢,頒百官胙肉。帝曰:「以天胙而享於家,是褻也。」諭改神前分享用熟薦。尋征朝鮮,祭告天地,並祀北郊。世祖入關宅帝位,於是冬至祀圜丘,奉日、月、星辰、雲、雨、風、雷配。夏至祀方澤,奉岳、鎮、海、瀆配。南北分饗。著為例。四年,定郊祀薦生牢如初,惟躬祀南郊進胙牛一。十四年,詔言:「人君事天如父,歲止一郊,心有未盡。惟營殿禁中,歲時致祀,配以太祖、太宗,庶昭誠敬。」禮臣乃援唐天寶四時孟月擇吉祭上帝故事,謂構上帝殿奉先殿東,元旦,萬壽,三節,夏冬二至,親詣致虔,儀物如郊祀。惟內祭初安神位時讀祝辭,不用胙,不進酒,不燎牛。從之。至是始有禁中祀天禮。十七年,敕廷臣議合祭儀,奏言仿明會典,前期一日,祭告各壇廟,定從祀十二壇。是歲四月,禁中大饗殿遂合祀天、地、日、月暨諸神。聖祖嗣位,詔罷之。

康熙二年,定郊祀躬親行禮,無故不攝。四十六年,冬至大祀,會天寒,群臣以代請,勿許。四十八年,帝違和,始令李光地攝行郊壇大禮。越二年,祀圜丘如初。嗣是帝年逾六十,兼病足,復令大臣攝之。明年冬至,齋戒,猶力疾升壇省俎豆,量力拜跽,退處幄次,俟攝事者禮訖始還宮。臣工固請停躬詣,猶勿許。六十一年,祀南郊,始遣世宗恭代,距賓天止五日也。雍正八年冬至,遇聖祖忌日,禮臣援舊例請代,下大學士九卿議。奏言周禮春官稱大祭祀王不親則攝行。唐、宋制,大祀與國忌同日,樂備不作。議者謂饗神不可無樂,未若攝祀之當乎禮也,遣代便。可其奏。乾隆七年,定議周禮祀天用玉輅,唐、宋參用大輦,今親祀南郊,前期詣齋宮,宜御玉輦。是日,帝乘禮輿,易鑾輅,自降輦至禮成,如儀。十四年,展拓兩郊壇宇,更新幄次。越四載蕆事,規制始大備。仁宗中葉,自制南北郊說,祀典如故。咸豐八年、九年,帝疾不能親,猶宮內致齋,屆日詣大高殿皇穹宇行禮。穆宗、德宗,沖齡踐阼,皆遣代。定親政日躬行。宣統纘緒,監國攝政王行之。

郊社之儀,天聰十年,禮部進儀注,迄順治間,始定郊祀前期齋戒閱祝版玉帛香,省牲,祀日遲明,禮部太常官詣皇穹宇行禮。奉神牌置壇所,司祝奉祝版,帝出宮乘輦,陪祀王公集午門金水橋從行,餘序立橋南迎送。駕至昭亨門降輦,前引大臣十人,次贊引官、對引官導入更衣幄次,更祭服出,訖盥,詣二成拜位前,分獻官各就位。典儀贊「迎神燔柴」,司樂官贊「舉迎神樂」,贊引奏「升壇」,帝升一成。上詣香案前,跪上炷香,又三上香,復位,行三跪九叩禮。典儀贊「奠玉帛」,司樂贊「舉樂」,帝詣神位前,跪搢玉帛奠案,復位。典儀贊「進俎」,司樂贊「舉樂」,詣神位前,跪受俎拱舉,復位。典儀贊「行初獻禮」,司樂贊「舉初獻樂」,樂作,舞干戚舞,帝詣神位前,跪奠爵,俯伏。讀祝官捧祝跪讀訖,行三叩禮。自上香至獻爵,配位前儀同。復位,易文舞。亞獻、終獻舞羽籥,儀如初獻,不用祝,分獻官、陪祀官隨行禮。三獻畢,飲福受胙,帝升壇至飲福位,跪,奉爵官酌福酒,奉胙官奉胙,跪進,受爵、胙,三叩,興,復位。率群臣行三跪九叩禮,徹饌送神,司樂、典儀贊訖,率群臣行禮如初。有司奉祝,次帛,次饌,次香,各詣燎所,唱「望燎」。帝詣望燎位,半燎,禮成,還大次,解嚴。太常官安設神牌,如請神儀。若遣代,則行禮三成階下,升降自西階,讀祝跪二成階下。罷飲福、受胙禮。送燎,退立西偏。餘如制。雍正元年,令陪祀官先蒞壇祗候。

方澤,前期但閱祝版。上香畢,奠玉帛,用瘞貍。餘與郊天同。

南郊,詣壇齋宿,自順治十一年著例,無常儀。乾隆七年定制,前一日,鑾儀衛嚴駕陳午門外太和門階下。巳刻,太常卿詣乾清門奏請詣齋宮,帝御禮輿出太和門,降輿乘輦,警蹕鳴鐘鼓,至昭亨門外降。寺卿導入門左,詣圜丘視壇位。分獻官分詣神庫、神廚視籩豆牲牢。帝出內外壝南左門,至神路西升輦,如齋宮。從祀官俟帝入,退歸齋所。翼日屆時,寺卿導入大次,更禮服出,復導駕詣壇行禮,畢,還宮。

三十五年,高宗六旬,命禮臣酌減升級次數及降輦步行遠近。議言郊前一日乘步輦如齋宮,自此易禮輿,至神路西降,步詣皇穹宇上香,遣親王視壇。祀日自齋宮至神路西階下降輦步入,禮成,即於降輦處乘輿還宮。行禮時,初升至二成拜位,即升壇上香,復位迎神,升階行奠玉帛禮,以次進俎,三獻暨飲福、受胙,並於此行之。還拜位,謝福胙,送神,乃卒事。方澤亦如之。允行。猶慮子孫玩視大典,復於三十九年諭誡,年未六旬,毋減小節,著為令。次年,祀南郊,命諸皇子旁侍觀禮。越四年,於是帝年七十矣,諭迎神獻爵暨祖宗配位前上香悉如舊,其獻帛爵諸禮,自本年南郊始,令諸皇子代陳。五十一年,帝以春秋高,步履或遜,敕壇上讀祝拜位增設小幄次,然備而未用也。五十九年,祀方澤,配位前獻帛爵,仍皇子代行。歷仁宗朝,郊祀各儀節,悉遵高宗舊制云。

嘉慶十八年,林清變起,計日敉平,會長至祀圜丘,諭先一日赴壇不升輦,自宮至皇穹宇入齋宮,並御肩輿,用答嘉貺。宣統嗣位,監國攝行郊祀,祀日詣壇,不齋宿,百官不迎送。出入升降,仍由右門,在右階行禮。拜位設第二成,視帝位少後。去黃幄。即於行禮處受胙,畢,進福酒、胙肉。餘同親祀儀。

郊祀配饗順治五年冬至,祀圜丘,奉太祖配。十四年諭曰:「太祖肇興帝業,太宗繼述皇猷,功德並隆,咸宜崇祀。」以後大祀天地,益奉太宗配饗。於是上辛祈穀,上帝位東奉太祖神位,卜吉奉太宗位於其西。夏至配方澤如初禮。十七年,行大饗殿合祀禮,尋罷。康熙六年冬至,祀南郊,用禮臣言,奉世祖配饗上帝,越九日,配饗皇地祇,詣方澤行禮。九年,祈穀亦如之。雍正二年,奉聖祖配大饗殿,次太宗。十三年冬,高宗嗣服,諭言:「皇考世宗,德侔造化,宜祀郊壇。」命議禮以聞。議者謂宜乾隆二年冬至配圜丘,三年孟春上辛配大饗殿,夏至配方澤。帝意以為祔廟後配饗,去夏至近、冬至遠。先配方澤,前後已歧。若俟南郊,時日又曠。考之舊典,世祖、太宗配饗天地,莫不先圜丘後方澤,時或翼日、或旬日,禮儀粲然。稽之經傳,成周郊祀后稷以配天,宗祀文王於明堂,即月令所謂「季秋,大饗帝」也。召誥「三日丁巳用牲於郊」。釋者謂非常祀而祭天,以告即位也。宋皇祐三年,以大慶殿為明堂,合祭天地,三聖並侑,古者因事而郊,不必定在二至。因諭來年世宗配天大禮,準此行事。逾歲,遂諏吉夏至前奉世宗配圜丘。餘如議。

先是部臣進升配儀,未議及祗見上帝。帝曰:「皇考祔廟,先見祖宗,然後升座,今行配饗,先見上帝,於義始允。」已,所司具儀上。於是祀南郊奉世宗神位祗見上帝,夏至祀方澤,祗見皇地祇,位並次世祖。嗣是升配皆先祗見,以為常。嘉慶四年,奉高宗配饗,道光元年,奉仁宗配饗,並如儀。

三十年,帝弗豫,遺命罷郊配,略謂:「禘郊祖宗,伊古所重,我朝首太祖訖仁宗。厚澤深仁,允宜配饗郊壇,禮隆報本。若世世率行無已,益滋後人疵議,此不能不示限制也。」文宗踐阼,遂敕王大臣集議,禮親王全齡等僉云:「大行皇帝功德懿爍,郊配斷不可易,請仍遵成憲。」禮部侍郎曾國籓疏言:「郊配之罷,不敢從者二,不敢違者三。大行皇帝仁愛之德,同符大造,粒我烝民,後稷所以配天也。御宇卅載,無一日暇逸,無斯須不敬,純亦不已,文王所以配上帝也。具合撰之實,辭升配之文,臣心何能自安?不敢從者一。大行皇帝德盛化神,即無例可援,猶應奏請,矧有成憲,曷敢稍逾!傳曰:『君行意,臣行制。』在上自懷謙德,為下宜守成規。不敢從者二。壇壝規模,尺寸有定,一磚一石,皆按九五陽數,不能增改。幄內止容豆籩,幄外幾無餘地。大行皇帝慮億萬年後,或議廣壇壝,或議狹幄制,故定為限制,以身作則。嚴諭集議,尚未裁決遵行,則後人孰肯冒大不韙?將來必至修改基址,輕變舊章。不敢違者一。唐垂拱間,郊祀奉高祖、太宗、高宗並配,開元十一年,從張說議,而罷太宗、高宗。宋景祐間,郊祀奉藝祖、太宗、真宗並配,嘉祐七年,從楊畋議,而罷太宗、真宗。我朝順治間,大饗殿合祀,後亦罷其禮。大行皇帝慮億萬年後,或援唐、宋舊例,妄行罷祀,因諭以非天子不議禮,增配尚所不許,罷祀何自而興?不敢違者二。我朝孝治天下,遺命尤重,聖祖不敢違孝莊文皇后遺命,未敢竟安地宮。仁宗不敢違高宗遺命,故雖豐功偉烈,廟號未獲祖稱。此而可違,家法何在!且反覆申明,處己卑屈,處祖崇高,大孝大讓,亙古盛德。不敢違者三。默計皇上仁孝深心,不升配歉在闕禮,遽升配歉在違命,且多將來之慮。他日郊祀時,上顧遺訓,下顧萬世,或悚然而難安,禮臣益無所辭咎。」帝頗韙其言。已復博諮廷議,手降敕諭,謂:「周人郊祀后稷,唐、宋及明,或三祖並侑,或數帝分配。我朝歷聖相承,靡不奉配。第配位遞增,壇制有定。皇考德澤,列祖同符,應如所請。俟祔禮成,仍奉升配,並體遺訓,昭示限制。自後郊祀配位,定為三祖五宗,永為恆式。」於是咸豐二年夏大祀圜丘、方澤,三年春上辛祈穀,並奉宣宗配,位次高宗。

十一年,帝崩,穆宗以郊配大典,遺命定三祖五宗,聖心不自安。乃集群臣議,並奉兩宮皇太后稽眾詢謀,禮親王世鐸等先後疏言:「禮貴制宜,孝當承志,兩朝遺訓,宜謹遵循。」帝勉從之。遂停文宗郊配。同治建元,雲南學政張錫嶸援孝經明堂嚴父配天義,謂宜以季秋祀上帝大饗殿,奉顯皇帝配。世鐸等益以欽定孝經衍義釋之,謂迭饗並侑,非禮所宜。議遂寢。

祈穀順治間,定歲正月上辛祭上帝大饗殿,為民祈穀。帝親詣行禮,與冬至同。惟不設從壇,不燔柴。十七年,詔饗帝大典,不宜有異,自後祈穀、燔柴以為常,並改大饗殿合祀上帝百神在圜丘舉行。康熙二十九年,聖祖親制祝文。四十八年,帝疾,不能親,遣官代。會江、浙、魯、豫水旱洊臻,仍自制祝文祈之。故事,上辛在正月五日前,改用次辛。雍正八年,上辛為正月二日,部臣因元旦宴,請展十日,不許。先期齋戒如故。十三年正月十日上辛,未立春,帝曰:「此非乘陽義也。」命禮臣集議。奏言:「禮月令,立春日,天子迎春東郊,乃祈穀上帝。此禮本在立春後,請循例用次辛,或立春後上辛。」從之。乾隆十六年,和親王等以大饗為季秋報祀,義殊祈穀,請更錫名。群臣亦言非明堂本制,襲稱大饗,名實未協。得旨,改曰「祈年」。

凡祈穀,駕如南郊,至西天門內神路西降輦,入祈年左門,詣皇乾宮上香。禮成,詣祈年壇視位,畢,仍出左門升輦至齋宮。三十七年,更定前一日輦入西天門,自齋宮東乘禮輿,訖西磚城左門止。步詣皇乾殿上香,畢,還齋宮,親王視壇位。祀日出齋宮,乘輦,至甬道正中,易禮輿,至神路西降。自磚城步就幄次,入左門,禮同圜丘。四十七年正月四日上辛,禮臣先期請改次辛便,帝曰:「上辛在正月三日前,為須隔年齋戒也;在四日前,為因聖母祝釐也。茲非昔比,奚改為?其仍用上辛,著為例。」又諭:「孟春祈穀,所以迓陽氣,兆農祥。考諸經傳,是立春後上辛,非元旦後上辛也。惟在月初,舊臘,即當齋戒。然太廟祫祭,大禮攸關,宮中拜神,國俗所在。若以齋期行此,似非專一致敬之道。」因下廷臣議。尋奏:「上辛以立春後所得為準,與其用十二月上辛,不如用正月上辛,以重歲首。如值三日前,則改次辛。或四日前,則應一日齋戒,是日未入齋宮,宮殿拜祭,各不相妨。毋庸改期。」允行。咸豐四年,祈穀,帝患宿疾,敕禮臣酌損儀文。侍郎宋晉請仍舊貫遣代行。帝曰:「是非輕改舊章也,應天以實不以文,此意宜共喻之。」

雩祀關外未嘗行。順治十四年夏旱,世祖始禱雨圜丘,前期齋三日,冠服淺色,禁屠宰,罷刑名。屆期,帝素服步入壇,不除道,不陳鹵簿,壇上設酒果、香鐙、祝帛暨熟牛脯醢,祭時不奏樂,不設配位,不奠玉,不飲福、受胙。餘如冬至祀儀。其方澤、社稷、神祇諸壇,則遣官蒞祭。既得雨,越三日,遣官報祀。定躬禱郊壇儀自此始。越三年又旱,卜吉致齋,步至南郊,躬親告祭。於時天無片雲,頃之乃大雨。報祀如初。康熙九年夏旱,詔百官修省,禮部祈雨。明年,帝親禱。自後躬祀以為常。二十六年,親制祝文祈告,雨立降。又嘗設壇宮禁,跽禱三晝夜,日惟淡食,越四日,步禱天壇,雨驟澍,步還宮,衣履霑濕雲。

乾隆七年,御史徐以升奏言:「春秋傳:『龍見而雩,為百穀祈膏雨也。』祭法:『雩宗,祭水旱也。』禮月令:『雩,帝用盛樂,命百縣雩祀,祀百闢卿士有益於民者,以祈穀實,是為常雩。』周禮:『稻人,旱又共雩斂。』春秋書雩二十有一,有一月再雩者,旱甚也。是又因旱而雩。考雩義為籲嗟求雨,其制,為壇南郊旁,故魯南門為雩門,西漢始廢,旱輒禱郊廟。晉永和立壇南郊,梁武帝始徙東,改燔燎從坎瘞。唐太宗復舊制。宋時孟夏雩祀上帝。明建壇泰元門東,制一成,旱則禱。我朝雩祭無壇,典制似闕,應度地建立,以符古義。」下禮臣議。議言:「孟夏龍見,擇日行常雩,祀圜丘,奉列祖配。四從壇,皆如禮。孟夏後旱,則仿唐制,祭神祇、社稷、宗廟。七日一祈,不足,仍分禱。旱甚,大雩。令甲,祈雨必望祭四海,至是罷之。又行大雩,用舞童十六人,衣玄衣,分八列,執羽翳,三獻,樂止,乃按舞。歌禦制云漢詩八章,畢,望燎。餘同常雩。至久雨祈晴,宜仿春秋傳鼓用牲,通考禜祭制,伐鼓祀少牢。禜祭國門,雨不止,則伐鼓用牲於社。罷分禱,停僧道官建壇諷經。其直省州、縣舊置耤田壇祀,仍依雍正四年例。孟夏行常雩,患旱,先祭境內山川,次社稷。患霪潦祈晴,如京師式。」十七年,增祈雨報祭樂章。

二十四年,常雩不雨,帝步禱社稷壇,仍用玉。六月大雩,親制祝文,定儀節。前一日,帝常服視祝版,詣壇齋宿,去鹵簿,停樂。出宮用騎,扈駕大臣常服導從。至南郊,步入壇,視位上香。祀日,帝雨冠素服步禱,從臣亦如之。不燔柴,不晉俎,不飲福、受胙。三獻畢,舞童舞羽、歌詩,退,皆如儀。帝率★臣三拜,徹饌,望燎。禮成,還宮。

三十七年,帝以年老,命酌損儀節視圜丘。

嘉慶十八年,以欽天監雩祀擇日,頻年恆在立夏節,殊乖古義,敕立夏後數日蠲吉行。著為例。

道光十二年六月大雩,親制祝文,省躬思過。是夕雨。報謝如常儀。御史陳焯請再申虔禱。帝曰:「祭法有祈有報。以報為祈,非禮也。其勿逾舊制。」

天神順治初,定雲、雨、風、雷。既配饗圜丘,並建天神壇位先農壇南,專祀之。雍正六年,諭建風神廟。禮臣言:「周禮燎祀飌師,鄭康成注風師為箕星,即虞書六宗之一。馬端臨謂,周制立春丑日,祭風師國城東北,蓋東北箕星之次,醜亦應箕位。漢劉歆等議立風伯廟於東郊。東漢縣邑,常以丙戌日祀之戌地。唐制就箕星位為壇,宋仍之。今卜地景山東,適當箕位,建廟為宜。歲以立春後丑日祭。」允行。規制仿時應宮,錫號「應時顯佑」,廟曰宣仁。前殿祀風伯,後殿祀八風神。明年,復以雲師、雷師尚闕專祀,諭言:「虞書六宗,漢儒釋為乾坤六子,震雷、巽風,並列禋祀。易言雷動風散,功實相等。記曰:『天降時雨,山川出雲。』周禮以雲物辨年歲,是雲與雷皆運行造化者也。並官建廟奉祀。」於是下所司議,尋奏:「唐天寶五載,增祀雷師,位雨師次,歲以立夏後申日致祭,宋、元因之。明集禮,次風師以雲師,郡、縣建雷雨、風雲二壇,秋分後三日合祭。今擬西方建雷師廟,祭以立夏後申日。東方建雲師廟,祭以秋分後三日。」從之。乃錫號雲師曰「順時普應」,廟曰凝和;雷師曰「資生發育」,廟曰昭顯;並以時應宮龍神為雨師,合祀之。

嘉慶二年旱,禱雨既應,仁宗蒞壇報祀,入壇中門降輿,至壝南門外,盥畢入,升壇。以次詣雲、雨、風、雷神位上香,二跪六拜。初獻即奠爵、帛,讀祝,不晉俎,不飲福胙。餘如故。

太歲殿位先農壇東北,正殿祀太歲,兩廡祀十二月將。順治初,遣官祭太歲,定孟春為迎,歲暮為祖。歲正月,書神牌曰「某干支太歲神」,如其年建。歲除祭畢,合祝版燎之。凡祭,樂六奏,承祭官立中階下,分獻官立甬道左右,行三跪九拜禮。初獻即奠帛,讀祝,錫福胙,用樂舞生承事,時猶無上香儀也。

乾隆十六年,禮臣言同屬天神,不宜有異,自是二祭及分獻皆上香。太歲、月將神牌,舊儲農壇神庫,至是亦以殿廡具備,移奉正屋。臨祭,龕前安神座。畢,復龕。舊制,祭太歲遣太常卿行禮,兩廡用員分獻。二十年,改遣親王、郡王承祭。次年,定太常卿為分獻官。

雍、乾以來,凡祈禱,天神、太歲暨地祇三壇並舉,遣官將事,陪祀者咸與焉。前期邸齋一日,承祭官拜位。天神壇在南階下,太歲與常祀同,俱三跪九拜。天神用燎,太歲兩廡不分獻,不飲福、受胙。

朝日、夕月,初以大明、夜明從祀圜丘,罷春秋分祀。順治八年,建朝日壇東郊,夕月壇西郊。

朝日用春分日卯刻,值甲、丙、戊、庚、壬年,帝親祭,餘遣官。樂六奏,舞八佾。凡親祭,入自壇北門,至甬道更衣大次,盥畢,升西階就位,行三跪九拜禮。奠獻遣有司行。遣代則行禮階下,惟讀祝時跽壇上。初日壇用露祭。雍正四年,始援社稷例,立龕壇下芘風雨。乾隆十一年,具服殿成,罷更衣大次。是歲春分翼日日食,高宗蒞祭,不乘輦,不奏樂,不陳鹵簿。三十九年躬祭,入霝星左門,如幄次行禮,以年高酌減禮文,非恆式也。

夕月用秋分日酉刻,奉星辰配,凡丑、辰、未、戌年,帝親祭,餘遣官。樂六奏,儀視日壇稍殺,親臨較少。升壇行禮,二跪六拜,初獻奠玉帛,讀祝,餘如朝日儀。遣官則拜壇下。乾隆三年戊午,例遣官,帝因初舉祀典,仍親祭如禮。五十五年,酌損節文,如日壇例。嘉慶五年庚申,效高宗故事,仍親祭,不遣官。十九年,定親祭儀,祀配位用親王、郡王上香。二十三年,世宗忌日值月壇齋期,諭陪祀執事官改常服,餘如故。

社稷之祀自京師以至直省府、州、縣皆有之,其在京師者,建壇端門右。世祖宅帝位,祭告如儀。定制,歲春、秋仲月上戊日,祭大社、大稷,奉后土句龍氏、后稷氏配。祭日,帝親蒞,壇上敷五色土,各如其方。樂七奏,舞八佾。帝出闕右門降輦,道北門出入,祭時出拜殿,至壝北門外就位,自北階升壇上香,詣正位奠獻。有司分祭配位。升北階,降西階,不晉俎,三跪九拜。餘儀如北郊舊例。

祭日逢國忌,不改期,易素服。康熙三年,遇太宗忌日,始改中戊。

雍正二年,平青海,告祭行獻俘禮。自是平定籓部,獻俘以為常。

乾隆十七年,改送燎為望瘞。明年,增望瘞樂章。

三十七年,以年老更儀節。幄次先設拜殿,帝御輦至壇外門,易禮輿,入右門,至拜殿東階下,乃降。升階行禮,禮成,升輿如初。故事,祭日遇風雨,拜位香案徙殿中,神位祭品露設如故。帝曰:「社稷之制,不立棟宇,以承天陽。今神牌藏神庫,是在棟宇內也。移奉殿中,復何嫌忌?」四十一年,定祭日遇風雨,神牌安奉殿內,祭器、樂虡移設拜殿,猝遇則用木龕覆神牌,其拜殿別設香案。嘉慶五年,仁宗詣壇祈雨,視春秋致祭儀,惟祭品用脯醢、果實,不飲福。前三日及祭日,王、公、百官皆齋戒,禁屠宰,不理刑名。餘悉如故。並諭親詣祈禱、報祀均步行,以隆典禮。

其在府、州、縣者,順治元年建,歲祭亦用上戊,府稱府社、府稷,州、縣則云某州、縣社、稷。

世宗纘業,制定祭品,羊一,豕一,帛一、籩、豆四,鉶、簠、簋各二。有司齋二日,屆期朝服祭於壇。乾隆八年,始頒祝文,各直省定例,為民祈報,會城布政使主之,督若撫陪祀。道官駐地,府、州、縣主之,道陪祀。十六年,以尊卑未協,詔互易之。督、撫、道官或出巡,仍令布政使暨府、州、縣官攝祭。武官自將軍以下,皆陪祀。社、稷以次諸祭,悉準此行。

先農天聰九年,禁濫役妨農。崇德元年,禁屯積米穀,令及時耕種,重農貴粟自此始。順治十一年,定歲仲春亥日行耕耤禮。先期,戶、禮二部尚書偕順天府尹進耒耜暨穜棱種。屆期,帝親饗祭獻如朝日儀。畢,詣耕耤所,南鄉立。從耤者就位。戶部尚書執耒耜,府尹執鞭,北面跪以進。帝秉耒三推,府丞奉青箱,戶部侍郎播種,耆老隨覆。畢,尚書受耒耜,府尹受鞭。帝御觀耕臺,南鄉坐,王以下序立。三王五推,九卿九推,府尹官屬執青箱播種,耆老隨覆。畢,帝如齋宮。府尹官屬、眾耆老行禮。農夫三十人執農器隨行。禮畢,從府、縣官出至耕耤所,帝賜王公坐,俟農夫終畝,鴻臚卿奏禮成,百官行慶賀禮。賜王公耆老宴,賞農夫布各一匹,作樂還宮。其秋,年穀登,所司上聞,擇日貯神倉,備供粢盛。尋定先農歲祭遣府尹行,大興、宛平縣官陪祀。

唐熙時,聖祖嘗臨豐澤園勸相。雍正二年,祭先農,行耕耤。三推畢,加一推。頒新制三十六禾詞。賞農夫布各四匹,罷筵宴。頒賜各省嘉禾圖。

乾隆三年,帝初行耕耤禮,先期六日,幸豐澤園演耕,屆日饗先農,行四推。二十三年諭曰:「吉亥耤畝,所重劭農。黛耜青箱,畚鎛蓑笠,咸寓知民疾苦至意。吾民雨犁日耘,襏襫維艱,炎濕遑避。設棚懸彩,義無所取。且片時所用,費中人數十戶產也,其除之。」三十七年,群臣慮帝春秋高,籥罷親耕,不許。命仍依古制三推。嘉慶以降,仍加一推如初。

直省祭先農,清初未舉行。雍正二年,耤田產嘉禾,一莖三四穗。越二年,乃至九穗。諭言:「國以民為本,民以食為天。禮,天子耤千畝,諸侯百畝。是耕耤可通臣下,守土者允宜遵行。俾知稼墻艱難,察地力肥磽,量天時晴雨。養民務本,道實由之。」於是定議:順天府尹,直省督撫及所屬府、州、縣、衛,各立農壇耤田。自五年始,歲仲春亥日,率屬祭先農行九推。十月朔,頒時憲書,豫定次年耕耤吉期,下所司循用。祭品禮數,如社稷儀。

先蠶清初未列祀典。康熙時,立蠶舍豐澤園,始興蠶績。雍正十三年,河東總督王士俊疏請祀先蠶,略言:「周禮鄭注上引房星,以馬神為蠶神。蠶、馬同出天駟,然天駟可云馬祖,實非蠶神。淮南子引蠶經,黃帝元妃西陵氏始蠶,其制衣裳自此始。漢祀菀窳婦人、寓氏公主,事本無稽。先蠶之名,禮經不載。隋始有壇,建宮北三里,高四尺。唐會要,遣有司饗先蠶如先農。宋景德三年,命官攝祀。有明釐正祀典,百神各依本號,如農始炎帝,止稱先農神,則蠶始黃帝,亦宜止稱先蠶神。按周制,蠶於北郊。今京師建壇,亦北郊為宜。」部議然之。侍郎圖理琛奏立先蠶祠安定門外,歲季春吉巳,遣太常卿祀以少牢。未及行。

乾隆七年,始敕議親蠶典禮,議者以郊外道遠,且水源不通,無浴蠶所。考唐、宋時后妃親蠶,多在宮苑中,明亦改建西苑。高宗鑒往制,允其議。命所司相度,遂建壇苑東北隅。三面樹桑柘。壇東為觀桑臺,前桑園,後親蠶門。其內親蠶殿,後浴蠶池,池北為後殿。宮左為蠶婦浴蠶河。南北木橋二,南橋東即先蠶神殿也。左曰蠶署,北橋東曰蠶所,皆符古制云。

是歲定皇后饗先蠶禮,立蠶室,豫奉先蠶西陵氏神位。屆日辰初刻,後禮服乘鳳輦出宮,至內壝左門降,入具服殿,妃、嬪從。盥訖,升中階,就南階上拜位,六肅,三跪,三拜。謝福胙禮三減一。不讀祝。爵三獻。凡拜跪,妃、嬪壇下皆行禮。餘如饗先農儀。禮成還宮。越日,行躬桑禮。先是築臺桑田北,置蠶母二人,蠶婦二十七人,蠶宮令、丞各一人承其事。後散齋一日,從採桑妃、嬪以下畢齋。是日昧爽,從桑侍班公主等祗候南門內。巳初刻,後出宮,妃、嬪從,詣西苑,入具服殿。傳贊分引妃、嬪、公主等就採桑位,典儀奏請後行禮。出詣桑畦北正中,相儀二人,𧾷忌進筐、鉤,後右持鉤,左提筐,東行畦外。內監揚採旗,鳴金鼓,歌採桑辭,後東西三採畢,歌止。相儀𧾷忌受筐、鉤。後御觀桑臺,以次妃、嬪、公主等五採,命婦九採。訖。蠶母北面跪,典儀舉筐授之,祗受退。切之,授蠶婦,灑于箔。後御繭館,傳贊引妃、嬪等行禮訖。還宮。蠶事畢,蠶母、蠶婦擇繭貯筐以獻。卜吉行治繭禮,後復詣壇臨織室,繅三盆,手遂布於蠶婦以終事。尋侍郎三德疏言:「親蠶典禮,為曠世鉅儀,請將壇址宮殿規制,興工告成日期,宣付史館。」詔從之。九年三月,始親蠶如儀。

尋定後不親蒞,遣妃代行。行禮階下,升降自東階。不飲福、受胙,不陪祀。十四年,禮部請遣妃代祀。時皇貴妃未正位中宮,帝諭曰:「妃所代,代後也。位未正,何代為?」因命內府大臣行禮。洎皇后冊立,始親饗。嗣後或躬親,或官攝,或妃代,並取旨行。

其行省所祭,惟乾隆五十九年,定浙江軒轅黃帝廟蠶神暨杭、嘉、湖屬蠶神祠,歲祭列入祀典,祭器視先農。

地祇順治初,定岳、鎮、海、瀆既配饗方澤,復建地祇壇,位天壇西,兼祀天下名山、大川。三年,定北鎮、北海合遣一人,東嶽、東鎮、東海一人,西嶽、西鎮、江瀆一人,中嶽、淮瀆、濟瀆一人,北嶽、中鎮、西海、河瀆一人,南鎮、南海一人,南嶽專遣一人,將行,先遣官致齋一日,二跪六拜,行三獻禮。

八年,封興京永陵山曰啟運,東京陵山曰積慶,福陵山曰天柱,昭陵山曰隆業,並列祀地壇。十六年,徙東京陵祔興京,罷積慶山祀。明年,用禮臣言,改祀北嶽於渾源。康熙二年,賜號鳳臺山曰昌瑞,並祀之。六年,遣祭如初制。惟南鎮、南海各分遣一人。十六年,詔封長白山神秩祀如五嶽。自是歲時望祭無闕。

二十四年,東巡祀泰嶽,祝版不書御名。先一日致齋。太常齎祝版、香、帛、爵,有司備祭品牲薦。屆日衣龍袞,出行宮。樂備不作。至廟內降輿。入中門,俟幄次,出盥畢,詣殿中拜位,二跪六拜。奠、獻如常儀。不飲福、受胙。明年,復改祀北嶽、混同江。逾二年,始望祭。

三十五年正月,為元元祈福,始遣大臣分行祭告,凡岳五:曰東嶽泰山、南岳衡山、中嶽嵩山、西嶽華山,北岳恆山。鎮五:曰東鎮沂山、南鎮會稽山、中鎮霍山、西鎮吳山、北鎮醫巫閭山。海四:曰東海、南海、西海、北海。瀆四:曰江瀆、淮瀆、濟瀆、河瀆。又兀喇長白山。翕河喬嶽自此始。明年,朔漠平,遣祭嶽、鎮、海、瀆如故。雍正二年,賜號江瀆曰涵和,河瀆曰潤毓,淮瀆曰通佑,濟瀆曰永惠。並賜東海為顯仁,南為昭明,西為正恆,北為崇禮。乾隆二年,封泰寧山曰永寧,附祀地壇如故事。

越十年,以來歲奉太后秩岱宗,敕群臣議禮。奏言:「古者因名山以升中,有燔柴禮。聖祖因儀文度數,書缺有閒,議封禪者多不經。定以祀五岳禮致祭,允宜遵行。」明年蒞泰安,前一日,詣嶽廟三上香,一跪三拜。翼日祭,如聖祖祀岳儀。又明年,巡省中州,祀中嶽,如初。十六年,巡江、浙,遣祭江、淮、河神。自是南巡凡六,皆躬祭。十九年,巡吉林,望祭北鎮,長白山亦如之。

二十六年,用禮臣議,改嶽、鎮、海、瀆遣官六人,長白山、北海、北鎮一人,西嶽、西鎮、江瀆一人,東嶽、東鎮、東海、南鎮一人,中南二岳、濟淮二瀆一人,北嶽、中鎮、西海、河瀆一人,南海一人。當是時,海神廟饗,所在多有,惟北海尚闕。四十三年,始建山海關北海神廟。凡祈禱地壇行禮,位北階下,三跪九拜,用瘞。光緒初元,加太白山神曰保民,醫巫閭山神曰靈應。二十七年,兩宮幸西安,遣官祭所過山川,並告祭華、嵩二岳,如禮。

其他山川之祀,自聖祖北征朔漠,駐蹕噶爾圖,命大學士祭山川,出卡倫,命官祭域外山川。自是浙江、大沽、大通海神皆建廟修祀。雍正間,建湘江神、武昌江神廟,並賜號廣東海陽山神曰安流襄績。高宗纘業,定星宿海、西域山川、伊犁阿布拉山諸神祀。又以松花江導源長白,依望祭北海制行。大軍西征,祭阿勒臺、珠爾庫、博克達、阿拉克四山。復賜太白山、洞庭山、庫倫汗山、金山諸神號。川、陜平,建終南山神廟。木蘭秋獮,議定興安大嶺山祀典,常祭用少牢,告祭太牢,歲仲春望祭行禮,如祀五鎮儀。帛、尊、羊、豕各一,簠、簋各二,爵三,籩、豆各十。秋獮,王大臣致祭,登一,鉶二,餘同春祭。別建廟以祀,錫號協義昭靈。又封江西廬嶽神曰溥福廣濟。自仁宗迄德宗,封江南、湖北、山東、臺灣、安東、江神、漢神、海神,黃陂木蘭山、西藏瓦合山、四川瓘眉山神,皆以時肇封或崇祀。綜稽一代祀典,河神別見河渠篇,其餘名山大川錫號尚多,不悉舉云。

直省神祇順治初,令各府、川、縣建壇,歲春秋仲月,有司致祭。雍正三年,定制,有司齋二日,朝服蒞事,儀視社稷壇。乾隆八年,頒各省祀神祇祝文。二十二年,定各府、州、縣祭境內山川,以春秋仲月戊日。其風、雷諸神,特錫封廟號以祀。自世宗至德宗末,代有增錫。凡列祀典者,有司隨時致虔,用羊一、豬一、果五盤、帛一、尊一、爵三,讀祝叩拜如故事。


\end{pinyinscope}