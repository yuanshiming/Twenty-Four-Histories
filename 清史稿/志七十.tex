\article{志七十}

\begin{pinyinscope}
○樂二

△十二律呂尺度

黃鍾古尺徑三分三釐八豪五絲一忽,長九寸,積八百一十分。今尺徑二分七釐四豪一絲九忽,長七寸二分九釐,積四百三十分四百六十七釐二百一十豪。容黍一千二百粒。

大呂古尺徑三分三釐八豪五絲一忽,長八寸四分二釐七豪二百四十三分豪之二百三十九,積七百五十八分五百一十八釐五百一十八豪奇。今尺徑二分七釐四豪一絲九忽,長六寸八分二釐六豪三分豪之二,積四百零三分一百零七釐八百四十豪。容黍一千二十四粒。

太簇古尺徑三分三釐八亳五絲一忽,長八寸,積七百二十分。今尺徑二分七釐四豪一絲九忽,長六寸四分八釐積三百八十二分六百三十七釐五百二十豪。容黍一千零六十七粒。

夾鍾古尺徑三分三釐八豪五絲一忽,長七寸四分九釐一豪二千一百八十七分豪之一千一百八十三,積六百七十四分二百三十八釐六百八十三豪奇。今尺徑二分七釐四豪一絲九忽,長六寸0分六釐八豪二十七分豪之四,積三百五十八分三百一十八釐零八十豪。容黍九百九十九粒。

姑洗古尺徑三分三釐八豪五絲一忽,長七寸一分一釐一豪九分豪之一,積六百四十分。今尺徑二分七釐四豪一絲九忽,長五寸七分六釐,積三百四十分一百二十二釐二百四十豪。容黍九百四十八粒。

仲呂古尺徑三分三釐八豪五絲一忽,長六寸六分五釐九豪一萬九千六百八十三分豪之二千九百零三,積五百九十九分三百二十三釐二百七十三豪奇。今尺徑二分七釐四豪一絲九忽,長五寸三分九釐三豪二百四十三分豪之二百二十一,積三百一十八分五百零四釐九百六十豪。容黍八百八十八粒。

蕤賓古尺徑三分三釐八豪五絲一忽,長六寸三分二釐0豪之八十一分豪之八十,積五百六十八分八百八十八釐八百八十八豪奇。今尺徑二分七釐四豪一絲九忽,長五寸一分二釐,積三百零二分三百三十釐八百八十豪。容黍八百四十三粒。

林鍾古尺徑三分三釐八豪五絲一忽,長六寸,積五百四十分。今尺徑二分七釐四豪一絲九忽,長四寸八分六釐,積二百八十六分九百七十八釐一百四十豪。容黍八百粒。

夷則古尺徑三分三釐八豪五絲一忽,長五寸六分一釐八豪七百二十九分豪之四百七十八,積五百零五分六百七十九釐零一十二豪奇。今尺徑二分七釐四豪一絲九忽,長四寸五分五釐一豪九分豪之一,積二百六十八分七百三十八釐五百六十豪。容黍七百四十九粒。

南呂古尺徑三分三釐八豪五絲一忽,長五寸三分三釐三豪三分豪之一,積四百八十分。今尺徑二分七釐四豪一絲九忽,長四寸三分二釐,積二百五十五分零九十一釐六百八十豪。容黍七百一十一粒。

無射古尺徑三分三釐八豪五絲一忽,長四寸九分九釐四豪六千五百六十一分豪之二千二百六十六,積四百四十九分四百九十二釐四百五十五豪奇。今尺徑二分七釐四豪一絲九忽,長四寸0分四釐五豪八十一分豪之三十五,積二百三十八分八百七十八釐七百二十豪。容黍六百六十六粒。

應鍾古尺徑三分三釐八豪五絲一忽,長四寸七分四釐0豪二十七分豪之二十,積四百二十六分六百六十六釐六百六十六豪奇。今尺徑二分七釐四豪一絲九忽,長三寸八分四釐,積二百二十六分七百四十八釐一百六十豪。容黍六百三十二粒。

七音清濁

倍蕤賓下徵乙字倍林鍾清下徵高乙字

倍夷則下羽上字倍南呂清下羽高上字

倍無射變宮尺字倍應鍾清變宮高尺字

黃鍾宮聲工字大呂清宮高工字

太簇商聲凡字夾鍾清商高凡字

姑洗角聲六字仲呂清角高六字

蕤賓變徵五字林鍾清變徵高五字

夷則徵聲乙字南呂清徵高乙字

無射羽聲上字應鍾清羽高上字

半黃鍾變宮尺字半大呂清變宮高尺字

半太簇少宮工字半夾鍾清少宮高工字

半姑洗少商凡字半仲呂清少商高凡字

黃鍾同形管聲同形管周徑積分表繁,詳正義,不列。

八倍黃鍾之管黃鍾宮聲工字正黃鍾之管

七倍黃鍾之管大呂清宮高工黃鍾八分之七之管

六倍黃鍾之管太簇商聲凡字黃鍾八分之六之管

五倍黃鍾之管夾鍾清商高凡黃鍾八分之五之管

四倍黃鍾之管姑洗角聲六字黃鍾八分之四即二分之一之管

三倍半黃鍾之管仲呂清角高六黃鍾八分之三分半之管

三倍黃鍾之管蕤賓變徵五字黃鍾八分之三之管

二倍半黃鍾之管林鍾清變徵高五黃鍾八分之二分半之管

二倍加四分之一黃鍾之管夷則徵聲乙字黃鍾八分之二又加此一分之四分之一之管

二倍黃鍾之管南呂清徵高乙黃鍾八分之二即四分之一之管

正加四分之三黃鍾之管無射羽聲上字黃鍾八分之一又加此一分之四分之三之管

正加半黃鍾之管應鍾清羽高上黃鍾八分之一又加此一分之四分之二之管

正加四分之一黃鍾之管半黃鍾變宮尺字黃鍾八分之一又加此一分之四分之一之管

正加八分之一黃鍾之管半大呂清變宮高尺黃鍾八分之一又加此一分之八分之一之管

正黃鍾之管半太簇宮聲工字黃鍾八分之一之管

樂之節奏,成於聲調,聲也者,五聲二變之七音;調也者,所以調七音而互相為用者也。聲調之原,本自旋宮,因管律弦度七音取分之不同而旋宮異。古旋宮之法,合竹與絲並著之。自隋以來,獨以弦音發明五聲之分,律呂旋宮,遂失其傳。夫旋宮者,十二律呂皆可為宮,立一均之主,各統七聲,而十二律呂皆可為五聲二變也。聲調者,聲自為聲,調自為調,而調又有主調、起調、轉調之異,故以轉調合旋宮言之,名為宮調。五聲二變旋於清濁二均之一十四聲,則成九十八聲,此全音也。然調雖以宮為主,而宮又自為宮,調又自為調。如宮立一均之主,而下羽之聲,又大於宮,故為調之首,古所謂宮逐羽音是也。羽主調,宮立宮,一均七聲之位定,則當二變者不起調,而與調首音不合者亦不起調。蓋以羽起調,徵在其前,變宮居其後。二音與羽相近,得聲淆雜,而變徵為第六音,亦與羽首音不合。此所以當二變之位,與五正聲當徵位者,俱不得起調也。至於止調,亦取本調相合,可以起調之聲終之。當二變與徵位者,亦不用焉。其立羽位調首之音,自本聲起者,即為本調。首音與五音為羽,與角次相合。首音與三音為羽,與宮又次相合,且均調相應。首音與四音為羽,與商轉相合可出入。故本調為一調,自宮位起者為一調,自角位起者為一調,自商位起者復為一調。自羽位、宮位、角位起者為正,自商位起者為假借,故曰可出入,如曲中所謂與某宮某調相出入者是也。轉相合者,下羽之調首至角為第五位,商之第三音至正羽第八音亦五位也。一均四調,七均二十八調,合清濁之一十四均,則為五十六調。樂工度曲,七調相轉之法,四字起四為正調,樂工轉調,皆用四字調為準,以四乙上尺工凡六七列位,視某字當四字位者,名為某調。一如五聲二變遞轉旋宮之法,以四字當羽位為起調處也。乙字起四為乙字調,上字起四為上字調,尺字起四為尺字調,工字起四為工字調,凡字起四為凡字調,合字起四為合字調。此指笛孔言。四字調乙、凡不用,乙字調上、六不用,上字調尺、五不用,尺字調工、乙不用,工字調凡、上不用,凡字調合、尺不用,合字調五、工不用,即如羽聲主調,當變宮、變徵聲者不用也。又四字調乙、凡不得起調,而六字亦不得起調,即如羽聲當羽位主調,二變不得起調,而徵聲亦不得起調也。此七調之七字相轉,即五聲二變之旋相為宮,宮調聲字,實為一體。析而言之,則有四科:一曰七聲定位,以五聲二變立一定之位,自下羽至正羽,共列為八,顯明隔八相生之理,視下羽位聲字律呂,知其為某宮之某調,視宮位聲字律呂,知其為某調之某宮,視二變位,知某聲字、某律呂之當避。二曰旋宮主調,以五聲二變旋於七聲定位之下,亦分八位,如羽聲立下羽之下,宮聲立宮位之下,則為宮聲立宮而羽聲主調。又如商聲立下羽之下,變徵立宮位之下,則為變徵立宮而商聲主調。三曰和聲起調,以十二律呂兼倍半以備用,按所生之音,各隨其均序於旋宮之下,仍以調主相和之聲所起各調注本律、本呂之下,以正各調之名。如黃鍾立宮,則夷則立下羽之位以主調,倍無射、正蕤賓當二變不起調,正夷則立徵位亦不起調,故用倍夷則起調者為正羽調,起黃鍾宮聲為正宮,起太簇商聲為正商,起姑洗角聲為正角,此正宮之四調也。大呂立宮,則倍南呂立下羽之位主調,用以起調者為清羽調,起大呂宮聲為清宮,起夾鍾商聲為清商,起仲呂角聲為清角,此清宮之四調也。其餘立宮主調,皆依此例。四曰樂音字色,以律呂簫笛所命字色,隨聲調而序其次,列於律呂之下。如黃鍾為工字,而簫應黃鍾者為工字,笛應黃鍾者為五字,皆注于黃鍾本律之下,大呂為高工字,而簫之高工、笛之高五亦皆注于大呂本律之下。其立羽位之字,即為主調,其立宮位之字,即為立宮,其當二變之位,則不用當徵位者亦不以起調。以此四科列為表,旋宮、轉聲、主調、起調之理犁然矣。

黃鍾宮聲立宮,倍夷則下羽主調為上字調。

七聲定位旋宮主調律管簫笛起調

下羽下羽倍夷則上凡正羽調

變宮變宮倍無射尺合不起調

宮宮黃鍾工四正宮

商商太簇凡乙正商

角角姑洗合上正角

變徵變徵蕤賓四尺不起調

徵徵夷則乙工不起調

羽羽無射上凡同調首

大呂清宮立宮,倍南呂清下羽主調,為高上調。

七聲定位旋宮主調呂管簫笛起調

下羽清下羽倍南呂上凡清羽調

變宮清變宮倍應鍾尺六不起調

宮清宮大呂工五清宮

商清商夾鍾凡乙清商

角清角仲呂六上清角

變徵清變徵林鍾五尺不起調

徵清徵南呂乙工不起調

羽清羽應鍾上凡同調首

太簇商聲立宮,倍無射變宮主調,為尺字調。

七聲定位旋宮主調律管簫笛起調

下羽變宮倍無射尺合變宮調

變宮宮黃鍾工四不起調

宮商太簇凡乙商宮

商角姑洗合上姑洗商

角變徵蕤賓四尺商角

變徵徵夷則乙工不起調

徵羽無射上凡不起調

羽變宮倍無射半黃鍾尺六同調首

夾鍾清商立宮,倍應鍾清變宮主調,為高尺調。

七聲定位旋宮主調呂管簫笛起調

下羽清變宮倍應鍾尺六清變宮調

變宮清宮大呂工五不起調

宮清商夾鍾凡乙清商宮

商清角仲呂六上仲呂商

角清變徵林鍾五尺清商角

變徵清徵南呂乙工不起調

徵清羽應鍾上凡不起調

羽清變宮倍應鍾半大呂尺六同調首

姑洗角聲立宮,黃鍾宮聲主調,為工字調。

七聲定位旋宮主調律管簫笛起調

下羽宮黃鍾工四宮調

變宮商太簇凡乙不起調

宮角姑洗合上角宮

商變徵蕤賓四尺角商

角徵夷則乙工夷則角

變徵羽無射上凡不起調

徵變宮倍無射半黃鍾尺六不起調

羽宮黃鍾工五同調首

仲呂清角立宮,大呂清宮主調,為高工調。

七聲定位旋宮主調呂管簫笛起調

下羽清宮大呂工五清宮調

變宮清商夾鍾凡乙不起調

宮清角仲呂六上清角宮

商清變徵林鍾五尺清角商

角清徵南呂乙工南呂角

變徵清羽應鍾上凡不起調

徵清變宮倍應鍾半大呂尺六不起調

羽清宮大呂工五同調首

蕤賓變徵立宮,太簇商聲主調,為凡字調。

七聲定位旋宮主調律管簫笛起調

下羽商太簇凡乙商調

變宮角姑洗合上不起調

宮變徵蕤賓四尺變徵宮

商徵夷則乙工變徵商

角羽無射上凡變徵角

變徵變宮倍無射半黃鍾尺六不起調

徵宮黃鍾工五不起調

羽商太簇凡乙同調首

林鍾清變徵立宮,夾鍾清商主調,為高凡調。

七聲定位旋宮主調呂管簫笛起調

下羽清商夾鍾凡乙清商調

變宮清角仲呂六上不起調

宮清變徵林鍾五尺清變徵宮

商清徵南呂乙工清變徵商

角清羽應鍾上凡清變徵角

變徵清變宮倍應鍾半大呂尺六不起調

徵清宮大呂工五不起調

羽清商夾鍾凡乙同調首

夷則徵聲立宮,姑洗角聲主調,為合字調。

七聲定位旋宮主調律管簫笛起調

下羽角姑洗合上角調

變宮變徵蕤賓四尺不起調

宮徵夷則乙工徵宮

商羽無射上凡徵商

角變宮倍無射半黃鍾尺六徵角

變徵宮黃鍾工五不起調

徵商太簇凡乙不起調

羽角姑洗六上同調首

南呂清徵立宮,仲呂清角主調,為高六調。

七聲定位旋宮主調呂管簫笛起調

下羽清角仲呂六上清角調

變宮清變徵林鍾五尺不起調

宮清徵南呂乙工清徵宮

商清羽應鍾上凡清徵商

角清變宮倍應鍾半大呂尺六清徵角

變徵清宮大呂工五不起調

徵清商夾鍾凡乙不起調

羽清角仲呂六上同調首

無射羽聲立宮,蕤賓變徵主調,為四字調。

七聲定位旋宮主調律管簫笛起調

下羽變徵蕤賓四尺變徵調

變宮徵夷則乙工不起調

宮羽無射上凡羽宮

商變宮倍無射半黃鍾尺六羽商

角宮黃鍾工五羽角

變徵商太簇凡乙不起調

徵角姑洗六上不起調

羽變徵蕤賓五尺同調首

應鍾清羽立宮,林鍾清變徵主調,為高五調。

七聲定位旋宮主調呂管簫笛起調

下羽清變徵林鍾五尺清變徵調

變宮清徵南呂乙工不起調

宮清羽應鍾上凡清羽宮

商清變宮倍應鍾半大呂尺六清羽商

角清宮大呂工五清羽角

變徵清商夾鍾凡乙不起調

徵清角仲呂六上不起調

羽清變徵林鍾五尺同調首

倍無射變宮立宮,夷則徵聲主調,為乙字調。

七聲定位旋宮主調律管簫笛起調

下羽徵夷則乙工徵調

變宮羽無射上凡不起調

宮變宮倍無射半黃鍾尺六變宮宮

商宮黃鍾工五變宮商

角商太簇凡乙變宮角

變徵角姑洗六上不起調

徵變徵蕤賓五尺不起調

羽徵夷則乙工同調首

倍應鍾清變宮立宮,南呂清徵主調,為高乙調。

七聲定位旋宮主調呂管簫笛起調

下羽清徵南呂乙工清徵調

變宮清羽應鍾上凡不起調

宮清變宮倍應鍾半大呂尺六清變宮宮

商清宮大呂工五清變宮商

角清商夾鍾凡乙清變宮角

變徵清角仲呂六上不起調

徵清變徵林鍾五尺不起調

羽清徵南呂乙工同調首

弦音合律呂立論者,始自淮南子,淮南本之管子,管子曰:「凡將起五音凡首,先主一而三之,四開以合九九,以是生黃鍾小素之首以成宮。三分而益之以一,為百有八為徵,不無有三分而去其乘適足,以是生商;有三分而復於其所,以是成羽;有三分去其乘適足,以是成角。」夫審弦音,無論某弦之全分,定為首音,因而半之,平分為二。其聲既與首音相合而為第八音矣,次以首音之全分,因而四之,去其一分而用其三分,其聲應於全分首音之第四音。此度乃全分首音與半分八音之間,又平分為二分之度。是即管子所謂「凡將起五音凡首,先主一而三之,四開以合九九」者也。先主一而三之者,以全分首音一分度為主,而以三因之,其數大於全分為三倍也。四開以合九九者,以三倍全分之數,四分而取其一,以合九九八十一之度,為宮聲之分也。小素雲者,素,白練,乃熟絲,即小弦之謂,言此度之聲立為宮位,其小於此弦之他弦,皆以是為主,故曰以是生黃鍾小素之首以成宮也。以八十一三分益一為百有八為徵,乃此弦首音全分之度,此宮弦上生下徵之數。於是以百有八,三分去一,為七十二,是為商。商七十二,三分益一,為九十六,是為羽。羽九十六,三分去一,為六十四,是為角。司馬氏律書:徵羽之數小於宮,而管子徵羽之數大於宮者,用徵羽之倍數,所謂下徵、下羽也。首弦起於下徵,即白虎通弦音尚徵之義。今由三分損益之法詳推其數,黃鍾正徵上生皆得七十二,為正商;正商上生得九十六,為下羽;下生得四十八,為正羽;下羽、正羽皆得六十四,為正角;正角上生得八十五,小餘三三。為下於宮音之變宮;下生得四十二,小餘六六。為高於羽音之變宮;下於宮音之變宮,高於羽音之變宮,皆得五十六,小餘八八。為變徵:是為濁均。變徵上生得七十五,小餘八五。為清宮;清宮上生得一百有一,小餘一三。為清下徵:下生得五十,小餘五六。為清徵;清下徵清徵皆得六十七,小餘四二。為清商;清商上生得八十九,小餘八九。為清下羽;下生得四十四,小餘九四。為清羽;清下羽、清羽皆得五十九,小餘九三。為清角;上生得七十九,小餘九一。為下於清宮之清變宮;下生得三十九,小餘三三。為高於清羽之清變宮;下於清宮之清變宮、高於清羽之清變宮皆得五十三,小餘二七。為清變徵:是為清均。凡宮至商,商至角,角至變徵,徵至羽,羽至變宮,皆得全分,而變徵至徵,變宮至宮,則祗半分。管子起音篇,司馬氏律書皆五聲之正,淮南子始載二變之數,但不當以十二律呂名之。尤足取者,則二變之度分,與二變之比於正音,一為和、一為謬之說也。所謂應鍾,即弦音之變宮度也,所謂蕤賓,即弦音之變徵度也。弦音變宮之在下徵第一弦為第三音,居第三位,即如宮弦之角聲第三位,音雖不同,而分則恰值正聲之度,故曰姑洗生應鍾,比於正音為和也。變徵之在下徵第一弦為第七音,居第七位,即如宮弦之變宮第七位,音亦不同,而分則皆為變聲之度,故曰應鍾生蕤賓,不比正音為繆也。五聲二變之清濁,定弦音各分之等差,今列於表:

首弦首音起於下徵,全度一百八分。二音下羽,得全度一百八分之九十六。三音變宮,得全度一百八分之八十五。小餘三三。四音正宮,得全度一百八分之八十一。五音正商,得全度一百八分之七十二。六音正角,得全度一百八分之六十四。七音變徵,得全度一百八分之五十六。小餘八八。八音正徵,得全度一百八分之半。為五十四。

首弦首音起清下徵,全度一百一分。小餘一三。二音清下羽,得全度一百一分之八十九。小餘八九。三音清變宮,得全度一百一分之七十九。小餘九一。四音清宮,得全度一百一分之七十五。小餘八五。五音清商,得全度一百一分之六十七。小餘四二。六音清角,得全度一百一分之五十九。小餘九三。七音清變徵,得全度一百一分之五十三。小餘二七。八音清徵,得全度一百一分之半。為五十,小餘五六。

二弦首音起於下羽,全度九十六分。二音變宮,得全度九十六分之八十五。小餘三三。三音正宮,得全度九十六分之八十一。四音正商,得全九十六分之七十二。五音正角,得全度九十六分之六十四。六音變徵,得全度九十六分之五十六。小餘八八。七音正徵,得全度九十六分之五十四。八音正羽,得全度九十六分之半。為四十八。

二弦首音起清下羽,全度八十九分。小餘八九。二音清變宮,得全度八十九分之七十九。小餘九一。三音清宮,得全度八十九分之七十五。小餘八五。四音清商,得全度八十九分之六十七。小餘四二。五音清角,得全度八十九分之五十九。小餘九三。六音清變徵,得全度八十九分之五十三。小餘二七。七音清徵,得全度八十九分之五十。小餘五六。八音清羽,得全度八十九分之半。為四十四,小餘九四。

三弦首音起於變宮,全度八十五分。小餘三三。二音正宮,得全度八十五分之八十一。三音正商,得全度八十五分之七十二。四音正角,得全度八十五分之六十四。五音變徵,得全度八十五分之五六。小餘八八。六音正徵,得全度八十五分之五十四。七音正羽,得全度八十五分之四十八。八音少變宮,得全度八十五分之半。為四十二,小餘六六。

三弦首音起清變宮,全度七十九分。小餘九一。二音清宮,得全度七十九分之七十五。小餘八五。三音清商,得全度七十九分之六十七。小餘四二。四音清角,得全度七十九分之五十九。小餘九三。五音清變徵,得全度七十九分之五十三。小餘二七。六音清徵,得全度七十九分之五十。小餘五六。七音清羽,得全度七十九分之四十四。小餘九四。八音清少變宮,得全度七十九分之半。為三十九,小餘九五。

四弦首音起於正宮,全度八十一分。二音正商,得全度八十一分之七十二。三音正角,得全度八十一分之六十四。四音變徵,得全度八十一分之五十六。小餘八八。五音正徵,得全度八十一分之五十四。六音正羽,得全度八十一分之四十八。七音少變宮,得全度八十一分之四十二。小餘六六。八音少宮,得全度八十一分之半。為四十,小餘五。

四弦首音起於清宮,全度七十五分。小餘八五。二音清商,得全度七十五分之六十七。小餘四二。三音清角,得全度七十五分之五十九。小餘九三。四音清變徵,得全度七十五分之五十三。小餘二七。五音清徵,得全度七十五分之五十。小餘五六。六音清羽,得全度七十五分之四十四。小餘九四。七音清少變宮,得全度七十五分之三十九。小餘九五。八音清少宮,得全度七十五分之半。為三十七,小餘九二。

五弦首音起於正商,全度七十二分。二音正角,得全度七十二分之六十四。三音變徵,得全度七十二分之五十六。小餘八八。四音正徵,得全度七十二分之五十四。五音正羽,得全度七十二分之四十八。六音少變宮,得全度七十二分之四十二。小餘六六。七音少宮,得全度七十二分之四十。小餘五。八音少商,得全度七十二分之半。為三十六。

五弦首音起於清商,全度六十七分。小餘四二。二音清角,得全度六十七分之五十九。小餘九三。三音清變徵,得全度六十七分之五十三。小餘二七。四音清徵,得全度六十七分之五十。小餘五六。五音清羽,得全度六十七分之四十四。小餘九四。六音清少變宮,得全度六十七分之三十九。小餘九五。七音清少宮,得全度六十七分之三十七。小餘九二。八音清少商,得全度六十七分之半。為三十三,小餘七一。

六弦首音起於正角,全度六十四分。二音變徵,得全度六十四分之五十六。小餘八八。三音正徵,得全度六十四分之五十四。四音正羽,得全度六十四分之四十八。五音少變宮,得全度六十四分之四十二。小餘六六。六音少宮,得全度六十四分之四十。小餘五。七音少商,得全度六十四分之三十六。八音少角,得全度六十四分之半。為三十二。

六弦首音起於清角,全度五十九分。小餘九三。二音清變徵,得全度五十九分之五十三。小餘二七。三音清徵,得全度五十九分之五十。小餘五六。四音清羽,得全度五十九分之四十四。小餘九四。五音清少變宮,得全度五十九分之三十九。小餘九五。六音清少宮,得全度五十九分之三十七。小餘九二。七音清少商,得全度五十九分之三十三。小餘七一。八音清少角,得全度五十九分之半。為二十九,小餘九六。

七弦首音起於變徵,全度五十六分。小餘八八。二音正徵,得全度五十六分之五十四。三音正羽,得全度五十六分之四十八。四音少變宮,得全度五十六分之四十二。小餘六六。五音少宮,得全度五十六分之四十。小餘五。六音少商,得全度五十六分之三十六。七音少角,得全度五十六分之三十二。八音少變徵,得全度五十六分之半。為二十八,小餘四四。

七弦首音起於清變徵,全度五十三分。小餘二七。二音清徵,得全度五十三分之五十。小餘五六。三音清羽,得全度五十三分之四十四。小餘九四。四音清少變宮,得全度五十三分之三十七。小餘九五。五音清少宮,得全度五十三分之三十七。小餘九二。六音清少商,得全度五十三分之三十三。小餘七一。七音清少角,得全度五十三分之二十九。小餘九六。八音清少變徵,得全度五十三分之半。為二十六,小餘六三。

弦音旋宮轉調,其要有四:一,定弦音應某律呂聲字,即得某弦度分。如倍無射之律變宮合字定弦,則得徵弦之分;黃鍾之律宮聲四字定弦,則得羽弦之分;太簇之律商聲乙字定弦,則得變宮弦之分;姑洗之律角聲上字定弦,則得宮弦之分;蕤賓之律變徵尺字定弦,則得商弦之分;夷則之律徵聲宮字定弦,則得羽弦之分;無射之律羽聲凡字定弦,則得變徵弦之分。此陽律一均七聲定弦之正分也。陰呂定弦七聲之分亦如之。

一,弦音轉調不能依次遞遷,必以宮調為準,故七聲因之而變。如琴之正調為宮調,其商調以七弦遞高一音,但六弦、七弦太急易,或變宮調以七弦遞下一音,則一弦、二弦又慢不成聲,故宮調七弦立準,轉調則七弦內有更者,有不更者,有宜緊者,有宜慢者,弦之轉移間,宮調旋焉。如一弦、三弦、六弦俱慢下管律一音,在弦度為半分,而餘弦不移,即轉為商調。蓋正宮調一弦、六弦定倍無射之律;變宮合字得徵弦分者,下為倍夷則之律;羽聲凡字轉角弦之分,三弦定姑洗之律;角聲上字得宮弦分者,下為太簇之律,商聲乙字轉羽弦之分,其二弦、四弦、五弦、七弦不移者,仍應本律。但二弦、七弦原得羽弦分者,轉為徵弦之分;四弦原得商弦分者,轉為宮弦之分;五弦原得角弦分者,轉為商弦之分。其倍無射之律,變宮合字為徵弦分者,轉變徵應於倍應鍾之呂,清變宮高六,應姑洗之律;角聲上字為宮弦分者,轉變宮應於仲呂之呂,清角高上,此二分當二變不用。因三弦定太簇之律,商聲乙字得羽弦之分以起調,四弦原得商弦之分者,轉為宮弦之分以立宮,故曰商調。如二弦、四弦、五弦、七弦俱緊上管律半音,在弦度亦為半分,而餘弦不移,即轉為角調。蓋正宮調二弦、七弦定黃鍾之律,宮聲四字得羽弦分者,上為大呂之呂,清宮高五轉徵弦之分。四弦定蕤賓之律,變徵尺字得商弦分者,上為林鍾之呂,清變徵高尺轉宮弦之分。五弦定夷則之律,徵聲工字得角弦分者,上為南呂之呂,清徵高工轉商弦之分。其一弦、三弦、六弦不移,仍應本律,但一弦、六弦轉為角分,三弦轉為羽分,而轉變徵、變宮者不用。因三弦應姑洗之律,角聲上字得羽弦之分以起調,四弦原得商弦之分者,上為角弦之分,轉宮弦之分以立宮,故曰角調。如獨緊五弦管律半音,在弦度亦為半分,即轉為變徵調。四弦應蕤賓之律,變徵尺字得羽弦之分以起調。五弦原得角弦之分者,上為變徵之分,轉為宮弦之分以立宮,故曰變徵調。如獨慢三弦管律一音,在弦度為半分,即轉為徵調,因五弦應夷則之律,徵聲工字得羽弦之分以起調,一弦、六弦原得徵弦之分者,轉為宮弦之分以立宮,故曰徵調。如以一弦、三弦、六弦慢下管律一音,四弦慢下管律半音,在弦度俱為半分,即轉為羽調,因一弦、六弦應倍夷則之律,羽聲凡字得羽弦之分以起調,二弦、七弦原得羽弦分者,上為變宮之分,轉宮弦之分以立宮,故曰變宮調也。

一,弦音諸調雖無二變,而定弦取音,必審二變之聲,必計二變之分。如陽律一均,即徵弦七聲之分言之,散聲為全分首音,其二音與羽弦應者為羽分,三音與變宮弦應者為變宮分,至七音與變徵弦應者為變徵分,八音仍與全弦應,故為旋於首音。其各分與各弦相應,亦自與各律相應。計其分,則首音徵至二音羽,三音羽至三音變宮,皆得全分。三音變宮至四音宮,祗得半分。四音宮至五音商,五音商至六音角,六音角至七音變徵,皆得全分。七音變徵至八音徵,亦得半分。以宮弦七聲之分言之,散聲為全半首音,其二音與商弦應者為商分,與角弦應者為角分,三至七音與變宮弦應者為變宮分,八音仍與全弦應。而四音變徵至五音徵,七音變宮至八音宮,皆祗半分。蓋太簇商聲乙字所應之弦分至姑洗角聲上字所應之弦分,與無射羽聲凡字所應之弦分至半黃鍾變宮合字所應之弦分,其間必為半分,故各弦七聲之分不移,而所應聲律有間雜之別。各分全半之間,宮調旋焉。以宮調七弦為準,據每調徵弦七聲言之,商調之徵,乃宮調之羽轉而為徵分者也。宮弦之羽,全弦首音為羽,其變宮變徵在二音、六音,是二音至三音,六音至七音,為半分也。今全弦轉為徵,則變宮、變徵在三音、七音,是三音至四音,七音至八音,為半分矣。故全弦定黃鍾之律宮聲四字者不移,二音即應太簇之律商聲乙字,其間得全分三音。若取姑洗之律角聲上字,則二音至三音為半分,仍與宮調之羽同,是以必取仲呂之呂清角高上,其間弦度始得全分,其四音仍應蕤賓之律變徵尺字。蓋太簇乙字至姑洗上字為半分,加仲呂高上之半分,得一全分,而仲呂高上至蕤賓尺字為半分,此所以二音至三音得全分,為羽至變宮,而三音至四音為半分,乃變宮至宮分也。五音仍應夷則之律徵聲工字,六音仍應無射之律羽聲凡字,此四音至五音,五音至六音,亦得全分。至七音若取半黃鍾之律變宮合字,則六音至七音為半分,亦與宮調之羽同,必取半大呂之呂清變宮高六,其間弦度始得全分,其八音仍與首音同應黃鍾之律宮聲四字。蓋無射凡字至半黃鍾合字為半分,加半大呂高六之半分,得一全分,而半大呂高六至黃鍾四字為半分,此所以六音至七音得全分,為角至變徵,而七音至八音為半分,乃變徵至徵分也。角調之徵,乃宮調之變宮與清宮調之羽相雜而為徵分者也。宮調之變宮全弦首音即變宮,而變徵在五音,是首音至二音,五音至六音,為半分也。清商調之羽全弦首音為清羽,其清變宮、清變徵在二音、六音,是又二音至三音,六音至七音,為半分也,今全弦轉為徵,則三音至四音,七音至八音,為半分矣。首音若仍定太簇之律商聲乙字,則首音徵至二音羽所得全分,必當取於仲呂之呂清角高上,其本調羽弦,則亦應仲呂之呂清角高上,是清角調非正角調矣。因取姑洗之律角聲上字為正角調,故起調於羽弦者,必取姑洗正角聲,而徵弦羽分亦當應姑洗之律。是以角調徵弦散聲首音,反比正宮調變宮弦之散聲首音下半音,取清宮調之羽弦散聲,大呂之呂清宮高五,其分始合,蓋因本調羽聲得正角之律故也。二音應姑洗之律角聲上字為羽分,三音應蕤賓之律變徵尺字為變宮分,四音應林鍾之呂清變徵高尺為宮分,五音應南呂之呂清徵高工為商分,六音應半黃鍾之律正變宮六字為角分,七音應黃鍾之律正宮四字為變徵分,八音仍應大呂之呂清宮高五,是則三音蕤賓至四音林鍾為半分,七音黃鍾至八音大呂為半分,正為本調徵弦之變宮至宮,變徵至徵之二半分也。變徵調之徵,乃宮調之宮轉而徵分者也。宮調之宮,變徵、變宮在四音、七音,是四音至五音,七音至八音,為半分也。今全弦轉為徵,則三音至四音,七音至八音,為半分,移宮調之宮四音至五音半分,為三音至四音半分,則四音取南呂之呂清徵高工,三音夷則至四音南呂為半分,七音太簇至八音姑洗為半分。徵調之徵,乃宮調之商轉而為徵者也。宮調之商,變徵、變宮在三音、六音,是三音至四音,六音至七音,為半分也。今全弦轉為徵,則三音至四音,七音至八音,為半分,移宮調之商六音至七音半分,為七音至八音半分,則七音取仲呂之呂清角高上,三音無射至四音半黃鍾為半分,七音仲呂至八音蕤賓為半分。羽調之徵,乃宮調之角轉而為徵分者也。宮調之角,變徵、變宮在六音、五音,是二音至三音,五音至六音,為半分也。今全弦轉為徵,則三音至四音,七音至八音,為半分也。移宮調之商二音至三音半分、五音至六音半分,為三音至四音半分、七音至八音半分,則三音取半大呂之呂清變宮高五,六音取仲呂之呂清角高上,七音取林鍾之呂清變徵高尺,三音半大呂至四音黃鍾為半分,七音林鍾至八音夷則為半分。變宮調之徵,乃宮調之變徵與清宮調之角相雜而為徵分者也。宮調之變徵、變宮在首音四音,是首音至二音,四音至五音,為半分。清宮調之角,變徵變宮在二音、五音,是又二音至三音,五音至六音,為半分也。今全弦轉為徵,則三音至四音,七音至八音,為半分。爰定南呂之呂清徵高工為散聲首音,三音黃鍾至四音大呂為半分,七音夷則至八音南呂為半分,此弦音定陽律七調旋相為用之法也。定陰呂七調立調之羽分,亦必以陰呂為主,其各弦各分陰陽間用亦如之。

一,弦音諸調,惟宮與商徵得與律呂相和為用,宮調各弦之七聲,皆應陽律一均。二變七聲之分亦然。清宮調各弦七聲及二變七聲之分,皆應陰呂一均,此弦音宮調所以得與律呂相和。商調各弦之五正聲,皆應陽律,惟二變聲轉陰呂,清商調亦惟二變雜入陽律,此商調五正聲所以得與律呂相和。徵調各弦之五正聲變宮聲皆應陽律,惟變徵一聲取陰呂,清徵調亦惟變徵一聲雜入陽律,此又徵調五正聲變宮聲得與律呂相和也。至角調五正聲內,徵弦、宮弦、商弦皆應陰呂,而二變反得陽律。且商聲乙字、羽聲凡字,各弦各分皆不得用,遺此二聲字與宮調同,清角聲五聲二變陰陽相雜亦然。是角調不可與律呂相和,變徵調五正聲內宮弦應陰呂,二變亦得陽律,羽聲凡字各弦各分皆不得用,清變徵調亦宮弦雜入陽律,是變徵調不可與律呂相和,然祗借一音,即與宮調聲字為同,較角調則為正也。羽調五正聲內角弦應陰呂,二變應陰呂,清羽調角弦二變應陽律,是雖不可與律呂相和,然據弦音猶為七聲俱備之一調。變宮調五正聲內徵弦宮弦皆應陰呂,而二變反得陽律。且商聲乙字、羽聲凡字,各弦各分皆不得用,遺此二聲字與角調同,清變宮五聲二變陰陽相雜亦然,是亦不可與律呂相和也。

宮調

徵一弦,定倍無射之律,變宮合字,得下徵之分。

羽二弦,定黃鍾之律,宮聲四字,得下羽之分。

應太簇之律,商聲乙字,為變宮之分。

宮三弦,定姑洗之律,角聲上字,得宮弦之分。

商四弦,定蕤賓之律,變徵尺字,得商弦之分。

角五弦,定夷則之律,徵聲工字,得角弦之分。

應無射之律,羽聲凡字,為變徵之分。

徵六弦,定半黃鍾之律,變宮六字,得徵弦之分。

羽七弦,定半太簇之律,宮聲五字,得羽弦之分。

清宮調

徵一弦,定倍應鍾之呂,清變宮高六,得下徵之分。

羽二弦,定大呂之呂,清宮高五,得下羽之分。

應夾鍾之呂,清商高乙,為變宮之分。

宮三弦,定仲呂之呂,清角高上,得宮弦之分。

商四弦,定林鍾之呂,清變徵高尺,得商弦之分。

角五弦,定南呂之呂,清徵高工,得角弦之分。

應應鍾之呂,清羽高凡,為變徵之分。

徵六弦,定半大呂之呂,清變宮高六,得徵弦之分。

羽七弦,定半夾鍾之呂,清宮高五,得羽弦之分。

商調

角慢一弦,定倍夷則之律,下羽凡字,得變徵之分,轉角弦之分。

應倍無射之律,得倍應鍾之呂,變宮合字,清變宮高六,為下徵之分,轉變徵之分。

徵二弦,定黃鍾之律,宮聲四字,得下羽之分,轉徵弦之分。

羽慢三弦,定太簇之律,商聲乙字,得變宮之分,轉羽弦之分。

應姑洗之律,得仲呂之呂,角聲上字,清角高上,為宮弦之分,轉變宮之分。

宮四弦,定蕤賓之律,變徵尺字,得商弦之分,轉宮弦之分。

商五弦,定夷則之律,徵聲工字,得角弦之分,轉商弦之分。

角慢六弦,定無射之律,羽聲凡字,得變徵之分,轉角弦之分。

應半黃鍾之律,得半大呂之呂,變宮六字,清變宮高六,為徵弦之分,轉變弦之分。

徵七弦,定半太簇之律,宮聲五字,得羽弦之分,轉徵弦之分。

清商調

角慢一弦,定倍南呂之呂,清下羽高凡,得變徵之分,轉角弦之分。

應倍應鍾之呂,得黃鍾之律,清變宮高六,宮聲四字,為下徵之分,轉變徵之分。

徵二弦,定大呂之呂,清宮高五,得下羽之分,轉徵弦之分。

羽慢三弦,定夾鍾之呂,清商高乙,得變宮之分,轉羽弦之分。

應仲呂之呂,得蕤賓之律,清角高上,變徵尺字,為宮弦之分,轉變宮之分。

宮四弦,定林鍾之呂,清變徵高尺,得商弦之分,轉宮弦之分。

商五弦,定南呂之呂,清徵高工,得角弦之分,轉商弦之分。

角慢六弦,定應鍾之呂,清羽高凡,得變徵之分,轉角弦之分。

應半大呂之呂,得半太簇之律,清變宮高六,宮聲五字,為下徵之分,轉變徵之分。

徵七弦,定半夾鍾之呂,清宮高五,得羽弦之分,轉徵弦之分。

角調

角一弦,定倍無射之律,變宮合字,得下徵之分,轉角弦之分。

應黃鍾之律,宮聲四字,為下羽之分,轉變徵之分。

徵緊二弦,應太簇之律,得大呂之呂,商聲乙字,清宮高五,得變宮之分,轉徵弦之分。

羽三弦,定姑洗之律,角聲上字,得宮弦之分,轉羽弦之分。

應蕤賓之律,變徵尺字,為商弦之分,轉變宮之分。

宮緊四弦,應夷則之律,得林鍾之呂,徵聲工字,清變徵高尺,得角弦之分,轉宮弦之分。

商緊五弦,應無射之律,得南呂之呂,羽聲凡字,清徵高工,得變徵之分,轉商弦之分。

角六弦,定半黃鍾之律,變宮六字,得徵弦之分,轉角弦之分。

應半太簇之律,宮聲五字,為羽弦之分,轉變徵之分。

徵緊七弦,應半姑洗之律,得半夾鍾之呂,商聲乙字,清宮高五,得變宮之分,轉徵弦之分。

清角調

角一弦,定倍應鍾之呂,清變宮高六,得下徵之分,轉角弦之分。

應大呂之呂,清宮高五,為下羽之分,轉變徵之分。

徵緊二弦,應夾鍾之呂,得太簇之律,清商高乙,商聲乙字,得變宮之分,轉徵弦之分。

羽三弦,定仲呂之呂,清角高上,得宮弦之分,轉羽弦之分。

應林鍾之呂,清變徵高尺,為商弦之分,轉變宮之分。

宮緊四弦,應南呂之呂,得夷則之律,清徵高工,徵聲工字,得角弦之分,轉宮弦之分。

商緊五弦,應應鍾之呂,得無射之律,清羽高凡,羽聲凡字,得變徵之分,轉商弦之分。

角六弦,定半大呂之呂,清變宮高六,得徵弦之分,轉角弦之分。

應半夾鍾之呂,清宮高五,為羽弦之分,轉變徵之分。

徵緊七弦,應半仲呂之呂,得半姑洗之律,清商高乙,商聲乙字,得變宮之分,轉徵弦之分。

變徵調

商一弦,定倍無射之律,變宮合字,得下徵之分,轉商弦之分。

角二弦,定黃鍾之律,宮聲四字,得下羽之分,轉角弦之分。

應太簇之律,商聲乙字,為變宮之分,轉變徵之分。

徵三弦,定姑洗之律,角聲上字,得宮弦之分,轉徵弦之分。

羽四弦,定蕤賓之律,變徵尺字,得商弦之分,轉羽弦之分。

應夷則之律,徵聲工字,為角弦之分,轉變宮之分。

宮緊五弦,應無射之律,得南呂之呂,羽聲凡字,清徵高工,得變徵之分,轉宮弦之分。

商六弦,定半黃鍾之律,變宮六字,得徵弦之分,轉商弦之分。

角七弦,定半太簇之律,宮聲五字,得羽弦之分,轉角弦之分。

清變徵調

商一弦,定倍應鍾之呂,清變宮高六,得下徵之分,轉商弦之分。

角二弦,定大呂之呂,清宮高五,得下羽之分,轉角弦之分。

應夾鍾之呂,清商高乙,為變宮之分,轉變徵之分。

徵三弦,定仲呂之呂,清角高上,得宮弦之分,轉徵弦之分。

羽四弦,定林鍾之呂,清變徵高尺,得商弦之分,轉羽弦之分。

應南呂之呂,清徵高工,為角弦之分,轉變宮之分。

宮緊五弦,應應鍾之呂,得無射之律,清羽高凡,羽聲凡字,得變徵之分,轉宮弦之分。

商六弦,定半大呂之呂,清變宮高六,得徵弦之分,轉商弦之分。

角七弦,定半夾鍾之呂,清宮高五,得羽弦之分,轉角弦之分。

徵調

宮一弦,定倍無射之律,變宮合字,得下徵之分,轉宮弦之分。

商二弦,定黃鍾之律,宮聲四字,得下羽之分,轉商弦之分。

角慢三弦,定太簇之律,商聲乙字,得變宮之分,轉角弦之分。

應姑洗之律,得仲呂之呂,角聲上字,清角高上,為宮弦之分,轉變徵之分。

徵四弦,定蕤賓之律,變徵尺字,得商弦之分,轉徵弦之分。

羽五弦,定夷則之律,徵聲工字,得角弦之分,轉羽弦之分。

應無射之律,羽聲凡字,為變徵之分,轉變宮之分。

宮六弦,定半黃鍾之律,變宮六字,得徵弦之分,轉宮弦之分。

商七弦,定半太簇之律,宮聲五字,得下羽之分,轉商弦之分。

清徵調

宮一弦,定倍應鍾之呂,清變宮高六,得下徵之分,轉宮弦之分。

商二弦,定大呂之呂,清宮高五,得下羽之分,轉商弦之分。

角慢三弦,定夾鍾之呂,清商高乙,得變宮之分,轉角弦之分。

應仲呂之呂,得蕤賓之律,清角高上,變徵尺字,為宮弦之分,轉變徵之分。

徵四弦,定林鍾之呂,清變徵高尺,得商弦之分,轉徵弦之分。

羽五弦,定南呂之呂,清徵高工,得角弦之分,轉羽弦之分。

應應鍾之呂,清羽高凡,為變徵之分,轉變宮之分。

宮六弦,定半大呂之呂,清變宮高六,得徵弦之分,轉宮弦之分。

商七弦,定半夾鍾之呂,清宮高五,得下羽之分,轉商弦之分。

羽調

羽慢一弦,定倍夷則之律,下羽凡字,得變徵之分,轉下羽之分。

應倍無射之律,得倍應鍾之呂,變宮合字,清變宮高六,為下徵之分,轉變宮之分。

宮二弦,定黃鍾之律,宮聲四字,得下羽之分,轉宮弦之分。

商慢三弦,定太簇之律,商聲乙字,得變宮之分,轉商弦之分。

角慢四弦,應姑洗之律,得仲呂之呂,角聲上字,清角高上,得宮弦之分,轉角弦之分。

應蕤賓之律,得林鍾之呂,變徵尺字,清變徵高尺,為商弦之分,轉變徵之分。

徵五弦,定夷則之呂,徵聲工字,得角弦之分,轉徵弦之分。

羽慢六弦,定無射之律,羽聲凡字,得變徵之分,轉羽弦之分。

應半黃鍾之律,得半大呂之呂,變宮六字,清變宮高六,為徵弦之分,轉變宮之分。

宮七弦,定半太簇之律,宮聲五字,得下羽之分,轉宮弦之分。

清羽調

羽慢一弦,定倍南呂之呂,清下羽高凡,得變徵之分,轉下羽之分。

應倍應鍾之呂,得黃鍾之律,清變宮高六,宮聲四字,為下徵之分,轉變宮之分。

宮二弦,定大呂之呂,清宮高五,得下羽之分,轉宮弦之分。

商慢三弦,定夾鍾之呂,清商高乙,得變弦之分,轉商弦之分。

角慢四弦,應仲呂之呂,得蕤賓之律,清角高上,變徵尺字,得宮弦之分,轉角弦之分。

應林鍾之呂,得夷則之律,清變徵高尺,徵聲工字,為商弦之分,轉變徵之分。

徵五弦,定南呂之呂,清徵高工,得角弦之分,轉徵弦之分。

羽慢六弦,定應鍾之呂,清羽高凡,得變徵之分,轉羽弦之分。

應半大呂之呂,得半太簇之律,清變宮高六,宮聲四字,為下徵之分,轉變宮之分。

宮七弦,定半夾鍾之呂,清宮高五,得下羽之分,轉宮弦之分。

變宮調

羽一弦,定倍無射之律,變宮合字,得下徵之分,轉下羽之分。

應黃鍾之律,宮聲四字,為下羽之分,轉變宮之分。

宮緊二弦,應太簇之律,得大呂之呂,商聲乙字,清宮高五,得變宮之分,轉宮弦之分。

商三弦,定姑洗之律,角聲上字,得宮弦之分,轉商弦之分。

角四弦,定蕤賓之律,變徵尺字,得商弦之分,轉角弦之分。

應夷則之律,徵聲工字,為角弦之分,轉變徵之分。

徵緊五弦,應無射之律,得南呂之呂,羽聲凡字,清徵高工,得變徵之分,轉徵弦之分。

羽六弦,定半黃鍾之律,變宮六字,得徵弦之分,轉羽弦之分。

應半太簇之律,宮聲五字,為羽弦之分,轉變宮之分。

宮緊七弦,應半姑洗之律,得半夾鍾之呂,商聲乙字,清宮高五,得變宮之分,轉宮弦之分。

清變宮調

羽一弦,定倍應鍾之呂,清變宮高六,得下徵之分,轉下羽之分。

應大呂之呂,清宮高五,為下羽之分,轉變宮之分。

宮緊二弦,應夾鍾之呂,得太簇之律,清商高乙,商聲乙字,得變宮之分,轉宮弦之分。

商三弦,定仲呂之呂,清商高上,得宮弦之分,轉商弦之分。

角四弦,定林鍾之呂,清變徵高尺,得商弦之分,轉角弦之分。

應南呂之呂,清徵高工,為角弦之分,轉變徵之分。

徵緊五弦,應應鍾之呂,得無射之律,清羽高凡,羽聲凡字,得變徵之分,轉徵弦之分。

羽六弦,定半大呂之呂,清變宮高六,得徵弦之分,轉羽弦之分。

應半夾鍾之呂,清宮高五,為羽弦之分,轉變宮之分。

宮緊七弦,應半仲呂之呂,得半姑洗之律,清商高乙,商聲乙字,得變宮之分,轉宮弦之分。

右弦音旋宮轉調,就琴弦立論,以羽弦起調為主,故旋宮首徵黃鍾定二弦羽位為宮調。律呂後編以七音立論,立宮為主,黃鍾為宮,則弦之宮分聲應黃鍾,商分應太簇,角分應姑洗,變徵分應蕤賓,徵分應夷則,羽分應無射,變宮分應半黃鍾。即倍無射。大呂為宮,七音之分應陰呂亦然。以分言,則宮分應黃鍾者即黃鍾之分,商分即太簇之分,角分即姑洗之分,變徵分即蕤賓之分。至徵分應夷則者,則非夷則之分,而為林鍾之分。羽分應無射者,亦非無射之分,而為南呂之分。變宮分應半黃鍾者,非半黃鍾之分,而為應鍾之分。大呂為宮,變徵分則為變林鍾之分,徵分則為夷則之分,羽分則為無射之分,變宮分則為變黃鍾之分。其陰陽各七均,均各七弦,有表詳樂問,不備載。


\end{pinyinscope}