\article{列傳一百二十五}

\begin{pinyinscope}
塞楞額周學健鄂昌鄂樂舜彭家屏李因培常安福崧

塞楞額,瓜爾佳氏,滿洲正白旗人。康熙四十八年進士,授內閣中書,擢翰林院侍講。四遷至侍郎,歷刑、兵、禮諸部。雍正二年,出署山東巡撫,入為戶部侍郎。如廣東按將軍李杕縱部兵毀米廠、閧巡撫署,事竟,仍署山東巡撫。疏請以東平州安山湖官地分畀窮民栽柳捕魚為業,上許之,並令發耗羨備用銀為建屋制船;又疏請浚柳長河,開引河二,疏積水。復入為工部侍郎,緣事奪官。乾隆元年,賜副都統銜,如索倫、巴爾虎練兵。尋授鑲藍旗漢軍副都統。出為陜西巡撫,移江西。疏請築豐城石堤,封廣信府銅塘山,均許之。再移山東。十一年,擢湖廣總督。

十三年,孝賢皇后崩,故事,遇國恤,諸臣當於百日後薙發。錦州知府金文醇違制被劾,逮下刑部,擬斬候。上以為不當,責尚書盛安沽譽,予重譴。江蘇巡撫安寧舉江南河南總督周學健薙發如文醇,上並命逮治。因詔諸直省察屬吏中有違制薙發者,不必治其罪,但令以名聞。是時塞楞額亦薙發,湖北巡撫彭樹葵、湖南巡撫楊錫紱及諸屬吏皆從之。得詔,塞楞額具疏自陳,上命還京師待罪。諭謂:「文醇已擬斬決,豈知督撫中有周學健,則無怪於文醇;豈知滿洲大臣中有塞楞額,又無怪於學健。」因釋文醇,寬學健,皆發直隸,以修城自贖。樹葵、錫紱誤從塞楞額,錫紱並勸塞楞額檢舉,皆貸罪;令樹葵分任修城,示薄罰。塞楞額至刑部,論斬決。上謂:「祖宗定制,君臣大義,而違蔑至此,萬無可恕!以尚為舊臣,令宣諭賜自盡。」

學健,江西新健人。雍正元年進士,改庶吉士,散館授編修。五遷至戶部侍郎。命如山東按事,兩詣上下江會督撫治災賑、水利,出署福建巡撫、浙閩總督。加太子少保,授江南河道總督,坐違制薙發,奪官,命江西巡撫開泰籍其家。開泰發其往來私書,中有丁憂兗沂曹道吳同仁行賕學健,乞舉以自代。上為罷陳舉自代例,詔曰:「朕令大臣舉可以自代之人,凡以拔茅茹、顯俊乂之意也。今同仁囑學健許以兩千,朕不解焉。問之錢陳群,始知為賕。夫考績黜陟,何可為苞苴之門,豈朕若渴之誠尚未喻於二三大臣耶?朕甚恧焉!其罷之。」別詔又謂:「學健卞急剛愎,不料其不勵名檢竟至於此!」下兩江總督策楞覆勘,具得學健營私受贓、縱戚屬奴僕骫法狀,刑部引塞楞額及前步軍統領鄂善例論斬決。上謂學健違制罪已貰,婪贓鬻破薦舉事視鄂善尤重,賜自盡。

鄂昌,西林覺羅氏,滿洲鑲藍旗人,大學士鄂爾泰從子也。雍正六年,以舉人授戶部主事。七年,超擢陜西寧夏道。十年,遷甘肅布政使。十一年,署陜西巡撫,旋授四川巡撫。酉陽州土司冉元齡老病,子廣烜襲,土民苦其貪暴,鄂昌奏請改土歸流。十三年,總督黃廷桂劾鄂昌貪縱,命奪職,以楊馝代之。遣刑部侍郎申珠渾會馝按治,得鄂昌枷斃罪人及受屬吏銀瓶諸狀,命逮下刑部,論杖徒,遇赦免。乾隆元年,令在批本處行走。二年,授直隸口北道,遷甘肅按察使。山西民梁玥等在高臺遇盜死,知縣伍升堂捕良民鍛煉論罪,鄂昌雪其冤,得真盜置之法。巡撫黃廷桂疏陳鄂昌平反狀,旨嘉獎。九年,遷廣西布政使。十一年,署廣西巡撫。疏請以鄂爾泰祀廣西名宦,上責其私,不許。十二年,疏自陳舉布政使李錫泰自代,上復責其朋比。因命督撫不得舉本省籓臬自代,著為例。迭移江蘇、四川、甘肅諸省,署甘肅提督、陜甘總督。復移江西巡撫。時傳播尚書孫嘉淦疏稿有誣謗語,命諸行省究所從來。鄂昌以坐廣饒九南道施廷翰子奕度逮下刑部,鞫無據,雪其枉,召鄂昌詣京師待命。獄定,誅千總盧魯生。責鄂昌誤讞,下刑部,論杖徒,命貸罪,發往軍臺效力。十九年閏四月,命以甘肅貯官茶發北路軍備用,命鄂昌董其事。旋授甘肅巡撫,理軍需。

內閣學士胡中藻著堅磨生集,文辭險怪,上指詩中語訕上,坐悖逆誅。中藻故鄂爾泰門人,鄂昌與唱和。上命奪職,逮至京師下獄。大學士九卿會鞫,籍其家,得所著塞上吟,語怨望;又聞鄂容安從軍,輒云「奈何奈何」,上責以失滿洲踴躍行師舊俗。又得與大學士史貽直書稿,知貽直為其子奕簪請託,上為罷貽直。諭:「鄂昌負恩黨逆,罪當肆市。但尚能知罪,又於貽直請託狀直承無諱,朕得以明正官常,從寬賜自盡。」

中藻,江西新建人。乾隆元年進士。上舉其詩有曰「又降一世」,曰「亦天之子」,曰「與一世爭在醜夷」,無慮數十事,語悖慢;又有「西林第一門」語,斥其攀援門戶,恬不知恥。因及鄂爾泰及張廷玉秉政,各有引援,朋分角立。謂:「如鄂爾泰猶在,當治其植黨之罪。」命罷賢良祠祀。

鄂樂舜亦鄂爾泰從子,初名鄂敏。雍正八年進士,改庶吉士,授編修。秋讞侍班,刑部侍郎王國棟放縱愆儀。上命之退,鄂敏未引去。因以責鄂敏,奪官。逾年,復編修。出為江西瑞州知府,累遷湖北布政使。命更名鄂樂舜。遷甘肅巡撫,疏請茶引備安西五衛積貯;移浙江,修海塘;皆議行。尋移安徽,又移山東。未行,浙江按察使富勒渾密劾鄂樂舜在浙江時,布政使同德為婪索鹽商銀八千,命侍郎劉綸、浙閩總督喀爾吉善按治。綸等言鄂樂舜實假公使銀。上又命兩江總督尹繼善會鞫,得婪索鹽商狀,如富勒渾言,但無與同德事,鄂樂舜論絞,富勒渾亦坐誣治罪。上以定擬失當,擢富勒渾布政使,逮鄂樂舜至京師,賜自盡。時後鄂昌死未一年也。

彭家屏,字樂君,河南夏邑人。康熙六十年進士,授刑部主事,累遷郎中。考選山西道御史,外授直隸清河道。三遷江西布政使。移雲南,再移江蘇。以病乞罷。乾隆二十二年春,高宗南巡,家屏迎謁。上諮歲事,家屏奏:「夏邑及鄰縣永城上年被水災獨重。」河南巡撫圖爾炳阿朝行在,上以家屏語詰之,猶言水未為災,上命偕家屏往勘;又以問河東河道總督張師載,師載奏如家屏言,上謂師載篤實,語當不誑,飭圖爾炳阿秉公勘奏,毋更回護。上幸徐州,見饑民困苦狀,念夏邑、永城壤相接,被災狀亦當同;密令步軍統領衙門員外郎觀音保微服往視。上北還,發徐州,夏邑民張欽遮道言縣吏諱災,上申命圖爾炳阿詳勘。次鄒縣,夏邑民劉元德復訴縣吏施賑不實,上不懌,詰主使,元德舉諸生段昌緒,命侍衛成林監元德還夏邑按其事;而觀音保還奏夏邑、永城、虞城、商丘四縣災甚重,積水久,田不可耕;災民鬻子女,人不過錢二三百,觀音保收災民子二,以其券呈上。上為動容,詔舉其事,謂:「為吾赤子,而使骨肉不相顧至此,事不忍言。」因奪圖爾炳阿職,戍烏里雅蘇臺,諸縣吏皆坐罪。

成林至夏邑,與知縣孫默召昌緒不至,捕諸家,於臥室得傳鈔吳三桂檄,以聞上。上遂怒,貸圖爾炳阿遣戍及諸縣吏罪,令直隸總督方觀承覆按。召家屏詣京師,問其家有無三桂傳鈔檄及他禁書。家屏言有明季野史數種,未嘗檢閱,上責其辭遁,命奪職下刑部,使侍衛三泰按驗。家屏子傳笏慮得罪,焚其書,命逮昌緒、傳笏下刑部,誅昌緒,家屏、傳笏亦坐斬,籍其家,分田予貧民。圖爾炳阿又以家屏族譜上,譜號大彭統記,御名皆直書不缺筆。上益怒,責家屏狂悖無君,即獄中賜自盡。秋讞,刑部入傳笏情實,上以子為父隱,貸其死。上既譴家屏等,召圖爾炳阿還京師,逮默下刑部,命觀音保以通判知夏邑。手詔戒敕,謂:「刁頑既除,良懦可憫。當善為撫綏,毋俾災民失所也。」

李因培,雲南晉寧人。乾隆十年進士,改庶吉士,散館授編修。十三年,特擢翰林院侍講學士,督山東學政。十四年,再擢內閣學士。十八年,署刑部侍郎,兼順天府尹。蝗起,因培劾通永道王楷等不力捕,皆奪職;又劾涿州知州李鍾俾虧倉穀,論罪如律。衡水知縣劉士玉,因培鄉人也,以賄敗,為直隸總督方觀承論劾。冀州知州言誇喀謁因培,因培稱士玉冤,言誇喀因為申布政、按察兩司。十九年,直隸布政使玉麟以其事聞,因培坐奪職。甫三月,起光祿寺卿。復督山東學政。二十一年,移江蘇。二十四年,遷內閣學士。學政任滿,移浙江。二十七年,任又滿,復移江蘇。上南巡,賦詩以賜。二十八年,授禮部侍郎,尋改倉場侍郎,皆留督學。

二十九年,授湖北巡撫。上諭湖廣總督吳達善曰:「因培能治事,學問亦優,但未免恃才,好居人上。今初任民事,汝當留意,治事有不當,善規之;不聽,即以聞。朕久未擢用,亦欲折鍊其氣質。今似勝於前,但恐志滿易盈,負朕造就耳。」旋移湖南。三十一年,又移福建,將行,常德被水。上令速予災民一月糧,詔未至,因培令秋後勘災如故事。上責因培「以將受代,五日京兆,不恤民瘼」,下部議,當降調。甫兩月,授四川按察使。

因培在湖南日,常德知府錫爾達發武陵知縣馮其柘虧庫帑二萬餘。時因培報通省倉穀無虧,慮以歧誤得罪;示意布政使赫升額,令桂陽知州張宏燧代其柘償萬餘,不足,仍疏劾。會宏燧讞縣民侯岳添被殺,誤指罪人,為按察使宮兆麟所糾。因培及繼任巡撫常鈞覆讞不能決,上命侍郎期成額即訊,因得宏燧營私虧帑,及承因培指代其柘償金諸狀,以聞。上命奪因培官,逮送湖北對簿,具服。諭曰:「諸直省倉庫虧缺,最為錮弊。昔皇考嚴加重戒,硃批諭旨,不啻三令五申,人亦不敢輕犯。朕御極三十餘年,有犯必懲,乃近年營私骫法,屢有發覺。豈因稽查稍疏,故態復作?朕自愧誠不能感人,若再不能執法,則朕亦非甚懦弱姑息之主也。」期成額奏至,因培下刑部論斬決,上命改監候。秋讞入情實,賜自盡。

常安,字履坦,納喇氏,滿洲鑲紅旗人。以諸生授筆帖式,自刑部改隸山西巡撫署。雍正初,擢太原理事通判。世宗時,庶僚皆得上章言事。常安疏請裁驛站館夫及諸官署鐙夫,省科派,從之。尋擢冀寧道。遷廣西按察使,移雲南。就遷布政使,移貴州。疏言:「苗疆多事,由於兵役擾累。嗣後有擾累事,罪該管文武官。」下雲貴廣西總督議行。遷江西巡撫。十三年,以母喪去官。

乾隆元年,還京師,舟經仲家淺,其僕迫閘官非時啟閘越渡,高宗聞之,諭謂:「皇考臨御時所未嘗有!徒以初政崇尚寬大,常安封疆大吏,乃為此市井跋扈之舉,目無功令。」下東河總督白鍾山按治,奪官,下刑部論罪,當枷號鞭責,命貸之,往北路軍營董糧餉。四年,授盛京兵部侍郎。內移刑部侍郎,外授漕運總督。內閣學士雅爾呼達請增遣滿洲兵駐防口外,直隸總督孫嘉淦疏請於獨石口、張家口外擇可耕地屯兵招墾。常安以為侵蒙古游牧地,疏請寢其事。

六年,移浙江巡撫,謝上,因言:「屬吏賢否視上司為表率,惟有身先砥礪,共勵清操。」上諭曰:「廉固人臣之本,然封疆大臣非僅廉所能勝任。為國家計安全,為生民謀衣食,其事正多。觀汝有終身誦廉之意則非矣。」上念浙江海塘為民保障,詔詢近時狀,並命閩浙總督那蘇圖、杭州將軍傅森會常安詳勘。常安等議:「海寧至仁和原有柴塘,塘外臨水,仿河工絡壩之法,用竹簍盛碎石,層層排築,外捍潮汐,內護塘基。水去沙停,漸有淤灘,再用左都御史劉統勛議,改建石塘。」別疏又言:「塘工可大可小,大則終年興工,亦難保其無虞;小則應興則興,應停則停,惟期免於沖決。是在因時損益,不宜惜費,亦不宜糜費。乾隆四五年間所修石塘,竭力督催,明歲可望全完。各塘不無闊狹高低,必須整齊堅固。臣諭督塘兵培補鑲墊,俾塘有堅工,兵無閒曠。海寧塘後舊有土塘以備泛溢,令民間栽柳,根株盤結塘身,枝幹藉資工用。」八年,石工乃成。

常安在浙江久,有惠政:嘗用保甲法編太湖漁舟,清盜源;釐兩浙鹺政諸弊,蘇商困;以溫、處二府貧瘠鮮★K6藏,招商轉江蘇米自海道至,佐民食。江蘇巡撫陳大受疏論常安輕開海禁,常安疏辨。謂:「蘇視溫、處彼此雖殊,兩地皆皇上赤子,大受不當過分畛域。」上諭曰:「汝等以此而矛盾,皆為民耳,出於不得已。以後豐年可不須,若需穀孔亟,當視此行耳。」常安巡視寧波沿海諸地,泛海至鎮海,又至定海,疏陳內外洋諸島嶼狀,謂內洋宜招民廣墾,外洋宜封禁。上嘉其沖冒風濤,勤於王事。嘉、湖二府奸民迷誘民間子女,常安督吏捕治,悉獲諸奸民。上令視採生折割例從重定擬,飭常安寬縱。尋上疏言:「州縣親民吏,必於轄境事無繁簡、地無遠近莫不深知,而後有實政以及於民。應飭於齋戒停刑暇日親歷鄉村,以次而遍。引其父老,詢以疾苦,於地方利弊了然胸中,且籍以周知戶口。如遇災賑,董理易為力。」上深然之。錢塘江入海處近蕭山為南大亹,近海寧為北大亹,蜀山南別有中小亹。舊為江海匯流處,漸淤塞,水趨南大亹,逼海寧。九年,尚書訥親蒞視,議復中小亹故道。常安令就沙嘴為溝四,引潮刷沙,歷數年,沙漸去。十一年,疏言:「春伏兩汛已過,南沙坍卸殆盡,蜀山已在水中。倘秋汛不復湧沙,大溜竟行中小亹矣。」上諭曰:「此言豈可輕出?亦俟三五年後如何耳。如能全行中小亹,果可喜事也。」

十二年,閩浙總督喀爾吉善劾常安多得屬吏金,婪索及於鹽政承差、海關胥吏,縱僕取市肆珍貴物不予值,凡十數事。上命解任,以顧琮代之,令大學士高斌會顧琮按治。常安亦疏劾布政使唐綏祖徇私狂悖,上為下高斌等並按。高斌等按常安婪贓納賄狀皆不實,惟縱僕得賕;常安劾綏祖事盡虛,疏請奪常安官。上命大學士訥親覆按,未至,高斌等又言常安歲易鹽政承差,有婪索狀;訥親至,又言常安嘗以公使錢自私,按律擬絞,下刑部,卒於獄。

常安少受業於尚書韓菼,工文辭,有所論著,多譏切時事。其坐譴多舉細故,遽從重比。時論疑其中蜚語以死,非其罪也。

福崧,烏雅氏,滿洲正黃旗人,湖廣總督碩色孫也。乾隆中,授內閣中書,遷侍讀。外授四川川北道,遷甘肅按察使。再遷福建布政使,未行,蘇四十三亂作,從總督勒爾謹討賊,即移甘肅。事定,賜花翎。勒爾謹坐冒賑得罪,命福崧從總督李侍堯察通省倉庫,虧銀八十八萬、糧七十四萬有奇,立例清償,無力者以責上官。福崧亦應分償,上特免之。

四十七年,遷浙江巡撫。上以王亶望、陳輝祖相繼撫浙江,皆貪吏,復命察通省倉庫,虧銀一百三十萬有奇,立例清償如甘肅。桐鄉縣徵漕不如律,民聚,福崧令捕治,因疏陳嚴除漕弊,條四事,下部議行。四十九年,上南巡,兩浙鹽商輸銀六十萬,以海寧範公塘改柴為石,福崧為請,上允之。五十一年,福崧以諸屬吏清償倉庫虧銀未能如期,疏請展限;並言於正歲集司道以下等官設誓,共砥廉隅。上以期已三四年,乃復請展限,非是,且設誓亦非政體,命尚書曹文埴,侍郎姜晟、伊齡阿如浙江按治。會福崧請籌柴塘修費,上疑新建石塘無益,勞民傷財,令文埴等並按,召福崧還京師待命。文埴等疏陳浙江倉庫實虧數,為定善後章程;別疏言柴塘坦水為石塘保障,宜有歲修。上允其請,察福崧無敗檢事,失但在柔懦,命署山西巡撫。

旋以浙江學政竇光鼐劾平陽知縣黃梅貪黷,論如律,責福崧未能發,左授二等侍衛,充和闐幫辦大臣。五十二年,移阿克蘇辦事大臣。五十四年,再移葉爾羌參贊大臣。五十五年,授江蘇巡撫,署兩江總督。還授浙江巡撫。五十七年,疏請補修海塘石工,與前巡撫瑯玕改築柴壩異議,上命江蘇巡撫長麟往按,請如福崧議。浙江鹽道柴槙遷兩淮鹽運使,虧帑,私移兩淮鹽課二十二萬補之。兩淮鹽政全德疏劾,上以福崧領兩浙鹽政,慮有染,奪官,以長麟代之。命尚書慶桂會鞫,謂福崧嘗索槙賕十一萬,又侵公使錢六萬有奇。獄具,論斬,逮致京師,尋命即途中行法。福崧飲酖卒。

福崧為巡撫,治事明決,御屬吏有法度,民頌其治行。其得罪死,頗謂其忤和珅,為所陷。尤慮至京師廷鞫,或發其陰私,故以蜚語激上怒,迫之死雲。

論曰:居喪不沐浴,百日薙發,亦其遺意也。塞楞額坐是中危法,學健雖以他事誅,然得罪仍在初獄。鄂昌以門戶生恩怨,家屏以搢紳言利病,皆足以掇禍。羅織文字,其借焉者也。因培起邊遠,受峻擢,屢躓屢起,乃以欺罔傅重比。常安、福崧死於賕,然封疆有政聲。論者以為冤,事或然歟?


\end{pinyinscope}