\article{列傳二百二十九}

\begin{pinyinscope}
徐樹銘薛允升宗室延煦子會章汪鳴鑾長麟

周家楣周德潤胡燏棻張廕桓

徐樹銘,字壽蘅,湖南長沙人。道光二十七年進士,選庶吉士,授編修。典四川鄉試。咸豐二年,遷中允,簡山東學政。累遷內閣學士,授兵部右侍郎。督學福建,按試興、泉。適莆田、同安呂、黃二氏械斗,勢洶洶,樹銘喻以大義,手書勸諭文付二氏,躬祭鬥死者而哀之,二氏愧悔;復為立型仁、講讓二塾,訓其子弟,二氏愈益和。秩滿,乞歸養。同治五年,起署禮部左侍郎。明年,督學浙江,以薦舉人才中列已罷編修俞樾,嚴旨付吏議,謫遷太常寺少卿。

光緒初,鴻臚寺卿,遭父憂,終喪,起授通政司副使。十年,晉太常寺卿。永定河決,詔樹銘往勘,既至,奏罷河工酌用民力及折價交土章程,民德之。法越事急,念海道梗阻,乃疏請漕糧改歸河運,敕直隸總督治南運全河。十二年,補左副都御史。時議廢當十錢,復制錢,民心惶惑。樹銘言於戶部尚書閻敬銘,請發倉廩,俾民以當十錢購粟,糶平而錢不廢,民乃安。十五年,授工部右侍郎。歷充順天、浙江鄉試正副考官,會試總裁。二十年,中東構釁,樹銘數上封事,皆不報。旋遷左都御史,充經筵講官。疏請行蠶政,敕督撫令有司營辦,以從違為舉劾,上嘉納,下其疏各省。二十五年,拜工部尚書。旋病卒,予優恤。

樹銘幼穎異,問學於何桂珍、曾國籓、倭仁、唐鑒諸人。生平無私蓄,惟嗜鐘鼎書畫,藏書數十萬卷,雖耄猶勤學不倦云。

薛允升,字雲階,陜西長安人。咸豐六年進士,授刑部主事。累遷郎中,出知江西饒州府。光緒三年,授四川成綿龍茂道,調署建昌。明年,遷山西按察使。值大祲,治賑,綜覈出入,民獲甦。又明年,晉山東布政使,權漕運總督。淮上患劇盜久未獲,允升詗得其巢,遣吏士往捕。歲除夕,盜方飲酒,未戒備,悉就執。六年,召為刑部侍郎,歷禮、兵、工三部,而佐兵部為久。念國家養兵勇糜餉糈,因條列練兵裁勇機宜,上嘉納。十九年,授刑部尚書。

初,允升觀政刑曹,以刑名關民命,窮年討測律例,遇滯義筆諸冊,久之有所得。或以律書求解,輒為開導,而其為用壹歸廉平。凡所定讞,案法隨科,人莫能增損一字。長官信仗之,有大獄輒以相囑。其鞫囚如與家人語,務使隱情畢達,枉則為之平反。始以治王宏罄獄顯名。蓋民有墮水死者,團防局勇已不勝榜掠,承矣;允升覆訊,事白。厥後江寧民周五殺硃彪,遁;參將胡金傳欲邀功,捕僧紹棕、曲學如論死。侍讀學士陳寶琛糾彈之,上命允升往按,廉得實,承審官皆懲辦如律。

二十二年,太監李萇材、張受山構眾擊殺捕者,嚴旨付部議。允升擬援光棍例治之,而總管太監李蓮英為乞恩,太后以例有「傷人致死、按律問擬」一語,敕再議。允升言:「李萇材等一案,既非謀故鬥殺,不得援此語為符合。且我朝家法嚴,宦寺倍治罪。此次從嚴懲治,不能仰體哀矜之意,已愧於心;倘復遷就定讞,並置初奉諭旨於不顧,則負疚益深。夫立法本以懲惡,而法外亦可施仁。皇上果欲肅清輦轂,裁仰閹宦,則仍依原奏辦理。若以為過嚴,或誅首而宥從,自在皇上權衡至當,非臣等所敢定擬也。」疏上,仍敕部議罪。其時蓮英遍囑要人求末減,允升不為動。復奏請處斬張受山,至李萇材傷人未死,量減為斬監候,從之。二十三年,其從子濟關說通賄,御史張仲炘、給事中蔣式芬先後論劾,允升坐不遠嫌,鐫三級,貶授宗人府府丞。次年,謝病歸。

二十六年,拳禍作,兩宮幸西安。允升赴行在,復起用刑部侍郎,尋授尚書。以老辭,不允。二十七年,回鑾,從駕至河南。病卒,恤如制。箸有漢律輯存六卷、漢律決事比四卷、唐明律合編四十卷、服制備考四卷、讀例存疑五十四卷。子浚,光緒六年進士,官禮部郎中。

宗室延煦,字樹南,隸正藍旗,直隸總督慶祺子。以任子官禮部主事。咸豐六年,成進士,選庶吉士,授編修。十三年,車駕北狩,錄城防功,擢四品京堂。明年,授贊善。累遷內閣學士,除盛京兵部侍郎。同治六年,調戶部,數勘辦展邊墾地。十一年,移督倉場。與漢侍郎畢道遠疏請漕糧起運本色濟兵食,議行。光緒二年,出為熱河都統,以圍場曠莽,易叢奸宄,請增置營汛資守御。有土寇王致岡者,構眾擾平泉、赤峰、建昌諸處,積為民患,官軍莫能捕,至是遣守備松恩剿平之。尋移疾去。

九年,授左都御史。念會典事例自嘉慶間續修,中更六十餘年,典章制度,視昔彌劇。及今不修,恐文獻無徵,難免舛漏。疏請敕廷臣集議開館,限年修明憲典,得旨報可。十年,晉理籓院尚書,調禮部。萬壽聖節,大學士左宗棠未隨班叩祝,延煦上疏論劾。略謂:「左宗棠職居首列,鴻臚引班時,竟步出乾清門,不勝駴詫!國家優禮大臣,宗棠被恩尤重。縱捐頂踵,未報萬一,乃躬履尊嚴之地,絕無誠敬之心。如曰遘疾,曷弗請假?而必故亂班聯,害禮負恩,莫或斯等!」疏上,下宗棠吏議,以延煦語過當,詔革職留任。

會山東民墊決口,言者劾巡撫陳士傑誤工狀,命延煦偕祁世長往按,白其誣而言其失計。又以遵旨巡察海防,具圖說以上,謂:「煙臺、旅順對峙,海面至此一束,兩岸同心扼守要隘,津、沽得有鎖鑰。防守之法,應如何測淺深,審沙線,備船砲,設水師;募諳海戰之人,必有制勝之策。」上韙其議,特宣示。還京,再移疾,不允。十二年,兩宮祇謁東陵,詣孝貞顯皇后陵寢,慈禧皇太后不欲行拜跪禮,延煦持不可,面諍數四。方是時,太后怒甚,禮部長官咸失色,延煦從容無少變。太后卒無以難,不得已跪拜如儀。延煦起家貴介,以文詞受主知,而立朝大節侃侃無所撓,士論偉之。明年,卒。

子會章,光緒二年進士,歷官理籓院侍郎。戊戌政變,漢京朝官罹法網者眾。會章獨奏論刑獄貴持其平,不當以滿、漢分畛域,言人所不敢言,論者謂其伉直有父風。

汪鳴鑾,字柳門,浙江錢塘人。少劬學。同治四年,成進士,選庶吉士,授編修。遷司業,益覃研經學,謂:「聖道垂諸六經,經學非訓詁不明,訓詁非文字不著。」治經當從許書入手,嘗疏請以許慎從祀文廟。歷督陜甘、江西、山東、廣東學政,典河南、江西、山東鄉試,顓重實學,號得士。光緒三年,父憂歸,服闋,起故官。歷遷內閣學士,晉工部侍郎,兼筦戶部三庫。十六年,赴吉林按事,與尚書敬信俱。

二十年,主禮部試。時日韓釁起,朝議紛呶。詔行走總理各國事務衙門,充五城團防大臣。調吏部右侍郎,兼貳刑部。逾年,和議成,日人堅索臺灣,鳴鑾力陳不可,稱上意。時上久親政,數召見朝臣,鳴鑾奏對尤切直。忌者達之太后,故抑揚其語,太后信之,上不自安。其冬,遂下詔曰:「朕侍奉皇太后,仰蒙慈訓,大而軍國機宜,小而起居服御,體恤朕躬,無微不至。乃有不學無術之徒,妄事揣摩,輒於召對時語氣抑揚,罔知輕重。如侍郎汪鳴鑾、長麟,上年屢次召見,信口妄言,跡近離間。本欲即行治罪,因軍務方棘,隱忍未發。今特曉諭諸臣,知所儆惕。汪鳴鑾、長麟並革職,永不敘用。嗣後內外大小臣工有敢巧言嘗試者,朕必治以重罪。」既罷歸,主講杭州詁經精舍、敷文書院。三十二年,卒。

長麟,滿洲鑲藍旗人。光緒六年繙譯進士,授編修。累至戶部右侍郎。

周家楣,字小棠,江蘇宜興人。咸豐九年進士,選庶吉士。散館,改禮部主事,充總理各國事務衙門章京。其時教禍棘,四川總督駱秉章夙持正,外人以將軍崇實易與,遇事輒就決之,數興大獄,至殺平民二百人,勿之問。家楣上書執政,極言其害,請教案歸總督裁決,卒如所言。各國相繼換約,交涉益劇,枋事者多依違。家楣苦心經畫,凡議覲禮、遣使臣、護僑民,皆委曲歷久而後定。洎日本闚臺灣,海防亟,乃為策先謀足以制日者。於是大學士文祥舉立海軍、造船艦、築砲臺、制槍械、採煤鐵、招僑商,及用人、籌餉諸端,折衷眾說,屬草議上之。累遷郎中,擢五品京堂。

光緒改元,除太僕寺少卿,典四川鄉試。越二載,遷順天府府尹,兼總理各國事務大臣,遭憂去。服闋,署左副都御史,直總署如故。八年,再授順天府府尹。時吏治日弛,家楣自初蒞即奏增經費,劾污吏,練捕盜營,親決獄訟,設通州、良鄉官車局、近畿教養義塾、善堂、留養局,增貢院號舍,擴金臺書院,制孔廟祭器、樂器。及再任,益有興革,郡中一切皆治辦。

九年,霪雨河溢,州邑籥菑,亟疏請帑,復募集銀百餘萬。會關東大熟,勸募雜糧,亦獲數萬石,恤饑困。明年春,大舉工賑,濬京南鳳河,京東北運河,武清、寶坻兩減河、宛平龐穀莊百二十村溝洫。通州、涿州、霸州、保定堤壩決口,分助直、魯工賑皆鉅萬。僉謂京畿救荒之政,為百年所未有云。

家楣方負時望,累兼署禮、戶、兵三部侍郎,上意駸鄉用。既而恭親王奕罷政,朝局一變。法越事起,朝士激昂多主戰。家楣以法彊盛,不可輕敵,乃自具疏,略謂:「法人肆擾海疆,臺灣亟於戰御,餉械阻絕。敵以兵船十數游弋海口,伺隙抵巇,各國且潛濟之。臺灣雖勝,與內地隔。越南得手,得一地留一師,亦恐分兵致弱。今調停之說,發之自彼,權之在我,不得不別具深謀,欲擒先縱。至中國實能自強,轉無戰之可言。此大局之樞紐也。」疏上,自知其言不協時,曰:「吾終不以附和誤國。」給事中孔憲劾張廕桓洩漏機密,語連家楣及吳廷芬等,乃罷直總署,轉通政使。十三年,卒。順天士民感其遺惠,請建通州專祠,詔允之。

周德潤,字生霖,廣西臨桂人。同治元年進士,選庶吉士,授編修。遷司業,歷侍讀學士,充日講起居注官。光緒八年,除少詹事。星變陳言,上修理政刑六事。再遷內閣學士。十年,大學士左宗棠稱疾請解職,德潤力言:「宗棠不宜去位,請旨責其引退之非,示以致身之義。」稱旨。當是時,言路發攄,德潤先後劾巡撫李文敏、倪文蔚不職狀,有直聲。

法越構兵,倡救越議,數請力保籓封,速定戰計,條列急務十端,可危者八,不可和者五,宜用兵者七。又以防務不可歲月計,復請亟籌強邊積穀,以老敵師、操勝算。疏凡十餘上,上數召見,嘉其諳邊情。命行走總理各國事務衙門,兩次請敕廷臣集議。未幾,和議起,法人勒退兵,益索償費。議者欲與之,德潤持不可,謂:「茍傷國體,即一介不可與。請定志毋退縮。」已,議敘事棘,德潤獨具疏,略言:「籓封可棄,猶謂非域中也。邊界可分,猶謂非腹地也。商可通,兵可撤,猶謂守約非背約也。五條外橫生枝節,若猶遷就,其何能國?請嚴拒之。」並陳和戰機宜甚悉。上以單銜入告,乖和衷誼,罷直總署。及明詔與法宣戰,德潤遵旨覆陳臺、越戰計,力駁德璀琳、盛宣懷所擬和約,條列救臺復越六策,力主先戰後和。復上安徽釐稅、梧州關稅積弊狀,先後命大臣廉得實,設法整飭之,歲課贏數十萬。

明年,和議將成,德潤臚舉八事進,曰:習勤苦;責疆吏;清內宄;募銳卒;杜中飽;會辦北洋大臣宜分駐奉天海口,南北宜聯一氣;滇、粵宜籌善後;雲南宜設機器局。上嘉納焉。時法使浦理燮等赴越,朝命德潤詣滇治界務。德潤率道員葉廷眷等出關,勘都霙南丹古林箐,緣南溪河至河口保勝蠻耗。十二年,與法使狄隆等論界線,以緣邊二千餘里,議分五段,執志乘與爭,更正沒入越地三十餘里,險要地四十里,復大賭咒河外苗塘子諸地數百里。逾歲還,除刑部侍郎,督順天學政。十八年,卒,予優恤。

胡燏棻,字蕓楣,安徽泗州人,本籍浙江蕭山。同治十三年進士,選庶吉士。散館,改知廣西靈川縣,未上,納貲為道員,銓直隸。總督李鴻章俾筦北洋軍糈,補天津道。光緒十四年,鴻章將出閱海軍,有巨猾覬為變,流言胥動。各國領事詰鴻章,鴻章以其事屬燏棻,越三月捕治之,民乃定。海舟應徭自奉天運米豆輸天津,充戶長者,歲出金三萬,往往破家。燏棻廉得狀,上鴻章奏罷之。十六年,大水,民數萬止城上。燏棻擴北倉、西沽粥廠徙居之。鴻章用其言,募集銀三百數十萬,復督塞南北運河諸溢流凡八十餘處,民猶及種麥。十七年,遷廣西按察使,賜頭品服。逾歲到官,多所平反。兩權布政使,建遜業堂教士,下臨桂知縣督諸囚習藝。

二十年,入覲,會中東事起,命治糈臺。師挫,鴻章東渡行成。諸軍西入關,燏棻疏請資遣之。蔣希夷軍幾潰,燏棻單騎宣諭,卒解遣,無敢譁者。朝廷恫喪師,知募兵不足恃,命燏棻主練兵,成十營,頓小站,號定武軍。小站練兵自此始。燏棻上疏言變法自強,條列十事:曰開鐵路,自漢口至京為幹路,其分支南自光山、固始出六安,自應城、京山、安陸出荊門、當陽;西自懷慶出軹關逕蒲、解達關隴;東自開封、歸德過宿、泗抵清江。曰造鈔幣、銀幣,毋使各國壟市利。曰制機器,國家用槍砲船械,令民廠自造,可塞漏卮。曰開礦產,築路需煤鐵,鑄幣需金銀銅,制機器需五金,擇良吏主其事。曰折南漕,官祿軍糈並易以銀,仍就津市米儲通州,備緩急。曰減兵額,汰老弱,簡精壯,化無用為有用。曰創郵政,取其貲佐度支,驛站、提塘皆可廢。曰練陸軍,將知學問,械求畫一,兵取良家,厚將領月糈,嚴戒侵蝕。曰整海軍,軍置帥,總領緣海七省,隸中樞,不受疆吏節度。曰設學堂,農、商、工、礦、醫有顓家,水師、陸軍、女子、盲啞有教法,朝廷為定制,甄而用之。又言停武科,練旗兵,器械、營制、餉章並從西式。次第皆採用。是歲定議造鐵路,自盧溝至津,命燏棻充督辦。尋授順天府府尹,疏請展京西支路,首盧溝訖門頭溝,便煤運。

已,充總理各國事務大臣,時董福祥軍駐南苑,斫傷鐵路西國工程師,各公使訴於朝,請罷董軍。燏棻力爭,始留駐近畿,然卒以此罷直總署。燏棻夙以談洋務著稱。次年,拳匪入京,指為通敵,欲殺之,逸而免。膺會辦關內外鐵路之命,路為聯軍占,歲餘始與英使訂約接收,復歸於我。遷刑部右侍郎,三十二年,轉禮部,尋轉郵傳部。卒,恤如制。予天津建祠。

張廕桓,字樵野,廣東南海人。性通侻。納貲為知縣,銓山東。巡撫閻敬銘、丁寶楨先後器異之,數薦至道員,光緒二年,權登萊青道。時英國請闢煙臺租界,議倡馬頭捐以斂厚貲,廕桓持不可。又義塚一區為人盜售,有司已鈐契矣;復與力爭,卒返其地。七年,授安徽徽寧池太廣道。抉蕪湖關痼弊,稅驟進。會久霪雨,江流衍溢,州邑籥菑,出俸錢賑之。明年,遷按察使。徵還,賞三品京堂,命直總理各國事務衙門。十年,除太常寺少卿。

廕桓精敏,號知外務。驟躋巍官,務攬權,為同列所忌。給事中孔憲摭其致蘇松太道邵友濂私函為洩朝旨,劾之,詔出總署。又以語連同官,並罷周家楣等,朝列益銜之。左遷直隸大順廣道。

十一年,命充出使美日秘三國大臣。逾歲赴美,舟抵金山,稅司黑假索觀國書,廕桓謂非關吏所得預,峻拒之。電詰美外部,黑假踧踖慚謝。至伊士頓,地近洛士丙冷,華民簞食相迎,初,華民之傭其地也,為美工燔殺,數至二百餘人。前使鄭藻如索償所毀財產,久不得直,至是皆待命廕桓。廕桓既達美都,即與其外部辨論,凡償墨西哥銀十四萬七千有奇。金山華民故好械斗,嘗為文諷諭之。未幾,美設苛例,欲禁遏華工。廕桓曰:「與其系命它族,毋寧靳勿與通也。」於是倡自禁華工議。繼乃徇眾請,不果行。其它烏盧公司槐花園、澳路非奴、姑力、阿路美、的欽巴新蕾諸案,亦多所斡旋。又與日廷爭論小呂宋設官事,卒如所議。是歲,除太常寺卿,轉通政司副使。十三年,奏設古巴學堂,並籌建金山學堂、醫院。後三年還國,仍直總署。歷遷戶部左侍郎。

二十年,中日議和,命偕友濂為全權大臣,東渡,日人弗納。次年,復命與日使林董賡議商約,廕桓力爭優待利益、徵收稅則二事,成通商行船二十九款,語具邦交志。二十三年,奉使賀英,上以其領度支熟知外情,命就彼國兼議加稅,堅拒免釐。廕桓歷英、美、法、德、俄而還,條具聞見,累疏以陳。大恉謂宜屏外援,籌固圉,為箴膏起廢策。二十四年,京師設礦務鐵路總局,被命主其事。數言修內政以戢民志,治團練以裕兵力,敕並依行。

先是變法議起,主事康有為與往還甚密。有為獲譴,遂褫廕桓職,謫戍新疆。越二年,拳亂作,用事者矯詔僇異己,廕桓論斬戍所。二十七年,復故官。

論曰:光緒朝部院大臣多負物望,其兼直總署者,時方重交涉,權比樞廷。樹銘、允升通經明律,家楣、德潤議約論戰,燏棻熟時務,廕桓諳外交,皆各有建白,一時理亂,實隱系之。鳴鑾以妄言罷斥,論者疑非其罪。延煦爭謁陵拜跪,劾朝賀亂班,侃侃尤無愧禮臣云。


\end{pinyinscope}