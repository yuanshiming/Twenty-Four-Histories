\article{列傳一百十}

\begin{pinyinscope}
黃廷桂鄂彌達楊廷璋莊有恭李侍堯弟奉堯

伍彌泰官保

黃廷桂,字丹崖,漢軍鑲紅旗人。父秉中,官福建巡撫。廷桂,初襲曾祖憲章拖沙喇哈番世職。康熙五十二年,授三等侍衛,遷參領。聖祖幸熱河,屢扈從。世宗在潛邸,知其才,雍正三年,授直隸宣化總兵。五年,擢四川提督。疏言:「四川三面環夷。軍械多敝缺,現飭修補。川馬本不高大,又日系槽,多羸斃。令在豐樂場後荒山督牧。士卒驕奢,飭服用毋僭官制。歲十月,番入內地傭工,名曰『下壩』,次年夏初始歸,以禁攜婦女,致成群肆惡,飭攜家屬方許就雇。成都屬德陽、仁壽二縣,南北距數百里,駐一把總;永寧協駐貴州永寧城,中隔河,東隸黔,西隸蜀,兵民歧視,應更定汛守。」命會總督岳鍾琪議行。又奏請嚴捕竊賊及博奕之具,上諭曰:「禁令弗行,咎在不公不明,不在不嚴。法猶藥也,取攻疾而已。過峻厲則傷元氣,徒猛不足貴也。」又奏嚴治建昌降番劫掠,又奏省城設防火堆棚,營置救火兵二十,上並嘉之。六年,請於提標及城守等營各設義塾,上諭曰:「文武不可偏重。少年聰穎,稍通文墨,勢必流為怯懦,不原為兵。則營伍所餘,皆魯鈍一流。是非興文,實乃廢武。邀虛名而無實益,將焉用之?」

烏蒙米貼苗婦陸氏為亂,發永寧、遵義兵援剿。四川雷波土司楊明義陰助陸氏,誘附近結覺、阿路、阿照、平底諸苗劫糧。陸氏既擒,請剿明義,令廷桂率總兵張耀祖率兵往。軍至拉密,擒明義,並獲造謀人卑租及結覺酋雙尺、阿路酋魯佩及阿不羅酋覺逼,斬馘近萬。上諭曰:「覽奏,斬馘何啻獵人弋獸!儻兵退仍復如故,豈有盡行殺戮之理?當詳思善於措置之道。」師復進攻確里密、阿都、阿驢諸苗,砲殪確里密酋利耶。阿都苗擒其酋阿必以獻,阿驢苗降。七年,奏軍事竟,上以效忠奮勇嘉之。尋疏陳苗疆地方諸事,上命籌善後。復奏湖北容美土司田雯如在四川界徵花絲銀,咨湖北察究。上諭曰:「楚、蜀諸土司容美最富強,越分僭禮。應曉以大義,漸令革除。」又奏籌剿瞻對土司,上諭曰:「瞻對雖微,亦不可輕視。凡事概以敬慎出之。」奏請開採黃螂等處銅鉛,以資鼓鑄。上諭曰:「黃螂、雷波與新撫涼山諸夷錯壤,第宜示以靜鎮,胡可興起利端?若聽民開採,流亡無藉之徒必群相趨赴,釀生事故。速會同巡撫憲德將金竹坪、白蠟山諸地銅鉛礦廠概行封禁。脫至紛紜,黃廷桂、憲德之身家性命不足贖其辜也!」廷桂奏引罪,復以詳慎申戒之。

尋奏捕得妖言罪人楊大銘等,言其渠楊七匿酉陽土司所,已檄令擒獻。上諭曰:「此事尤宜詳慎!朕料酉陽土司未必為此事。」八年,奏於楊隘嘴獲楊七,非酉陽境內。上諭曰:「朕非有過人技,但較汝等克誠克公耳。人有利害是非之心,遇事接物,非過即不及。惟公與誠為對證之藥。」十二月,奏惈亂,發兵攻克金鎖關、黑鐵關、黃草坪諸地,恢復永善。得旨獎許。上嘗諭憲德,令密陳廷桂為人,奏稱「多疑偏聽,好勝矜人,是其病痛」。上終以實心任事嘉之。

九年,師討噶爾丹策零,分設四川總督,即以命廷桂,仍兼領提督。奏請將四川常平倉捐穀改銀,上諭曰:「四川本產米地,積貯尚易。遽請開捐,誤矣。且欲改穀作銀,又將銀買穀,更轉展滋弊,當另議增貯。」十年六月,奏建昌鎮轄竹核,當涼山之中,為苗疆腹心要地,請於附近各險隘增兵設鎮,上命大學士鄂爾泰詳議。尋議兵力宜合不宜分,蠻巢宜遠不宜近,但使我勢聯絡,不必隨處設防。請於竹核設兵三千,分駐吽姑、格落、魚紅、大赤口、阿都、沙馬、普雄諸地。敕下廷桂行之。

八月,兒斯番為亂,奏遣總兵趙儒剿捕,上責廷桂從前未料理妥協。十月,廷桂奏言:「雍正五年兒斯番為亂,臣檄副將王剛按治。時臣甫到川,地利夷情尚未諳習。今兇鋒既肆,由臣撫馭無方,已遵旨密諭趙儒凜遵料理。」十二月,擒兒斯酋,並剿定河東各寨勾結諸番。復奏言:「王剛前所懲創,不過兒斯一堡。今仰蒙指示,趙儒督勵將士,一切險巢重地,深林石穴,悉行蕩平。」上深獎之。

十三年,奏:「貴州古州苗亂,四川建昌、永寧俱與連界,已飭將吏加意撫輯。」上諭以「不動聲色,靜鎮慎密」。乾隆元年,裁總督缺,廷桂仍為提督。十二月,召詣京師。二年,授鑾儀使。尋授天津總兵。五年,遷古北口提督。六年,上幸熱河,道古北口,閱兵,營伍整肅,賜廷桂馬,並上用緞。尋授甘肅巡撫。十二年,署陜甘總督。

十三年,授兩江總督。疏言:「江西俗悍,有司因循姑息,動輒喧閧,飭嚴捕究治。」又言:「南方晴少雨多,各營操練閒曠,令於陰雨時擇公所或寬敞寺宇操練。」上諭曰:「汝至江南,整飭振作,但不可欲速,要之以久可也。」十五年,太子少保。疏劾「江蘇巡撫雅爾哈善以奏銷錢糧,奉旨訓飭;知縣許惟枚等經徵未完,不及一分,例止罰俸。忽奏請奪官。人必以為出自上意,居心巧詐」。雅爾哈善下吏議。

十六年,調陜甘總督。時四川復分設總督,十八年,仍以命廷桂。奏四川歲豐穀賤,上命轉輸二十萬石賑淮、揚被水州縣,禦制詩紀其事。進吏部尚書,留總督任。四川濱江諸縣引江水溉田。餘多山田,每苦旱。廷桂奏飭通省勘修塘堰,新都、蘆山等十州縣及青神蓮花壩、樂山平江鄉、三臺南明鎮次第修舉,悉成腴壤。二十年,奏請增爐鑄錢,為通省修城。上諭曰:「有益地方之事,詳妥為之。」授武英殿大學士,仍領總督事。打箭爐徼外孔撒、麻書兩土司構釁,金川、綽斯甲布袒麻書,革布什咱、德爾格忒袒孔撒,互攻殺。廷桂偕提督岳鍾琪飭諭解散。

六月,復調陜甘總督。師討阿睦爾撒納,陜、甘當轉輸孔道。廷桂途次以軍中調取營馬,並令州縣採買馬駝,即飭各驛馬十調五六,得馬數千匹佐軍。尋奏軍中文報,責成沿邊提鎮料理,詔如所請。二十一年四月,命駐肅州督辦軍需。奏言:「各處調解軍馬,口外嚴寒,自安西至哈密,經戈壁十餘站,飼飲不時,每致疲斃。現派專官分站料理,將積貯草豆、經過匹數、住歇時刻、行走臕分,按日呈報。」又奏:「山西解駝,先留安西牧放。陜西解馬,亦先調甘肅飼養。陸續前運,以濟實用。」先後送軍前駝馬七萬餘。又言:「西北兩路軍營向通商販,後因撤兵禁止。巴里坤軍營應用牛羊諸物,專自肅州販往,路遠價昂,難資接濟,請照舊通商。」上命籌濟庫車、阿克蘇糧運。廷桂奏:「夾山一路,可自哈密直趨闢展、吐魯番,其間騾駝通行,水草饒裕,較繞行巴里坤為近。擬即運糧貯吐魯番,轉運軍營,往返更加迅速。」又發銀二十萬,解阿克蘇買回城米,運糧十萬儲巴里坤。凡所經畫,屢合上指。十二月,上諭曰:「廷桂於西陲用兵,雖未身歷行陣,而籌辦軍需,每有朕旨未到,旋即奏至,與所規畫不約而同。體國奉公,精詳妥協,而又毫不累民,內地若無兵事,其功最大。」積功自太子太保進少保,自騎都尉進三等忠勤伯,先後賜雙眼孔雀翎、紅寶石帽頂、四團龍補服、白金二萬。二十四年正月,駐涼州,以病劇聞。命額駙福隆安率御醫診視,甫行,廷桂卒。上即命福隆安奠醊,禦制詩輓之,賜祭葬,謚文襄。喪還,上復親臨奠醊。二十五年,凱宴成功將士,追念廷桂,復賦詩惜之。尋命圖形紫光閣,禦制懷舊詩,列廷桂五督臣首。

孫檢,官副都統。乾隆四十九年,以刻廷桂奏疏,載兩朝批答,被嚴旨申飭。曾孫文煜,自侍衛累擢副都統,調馬蘭鎮總兵。

鄂彌達,鄂濟氏,滿洲正白旗人。初授戶部筆帖式。雍正元年,授吏部主事。累遷郎中。五年,命同廣東巡撫楊文乾等如福建察倉庫。六年,擢貴州布政使。八年,遷廣東巡撫。疏言:「鳥槍例有禁,瓊州民恃槍御盜,請戶得藏一,多者罪之。」梧州民陳美倫等謀亂,捕治如法。十年,署廣東總督。疏言:「總督舊駐肇慶,所以控制兩粵。今專督廣東,應請移駐廣州。」饒平武舉餘猊等謀亂,捕治如法。尋實授總督。安南民鄧文武等遇風入銅鼓角海面,鄂彌達畀以資,送歸國,國王以伽南、沉香諸物為謝,卻之,疏聞,上獎其得體。先後疏請移設將吏。又疏請於三水西南鎮建倉貯穀,並以米貴,會城設局平糶。又請升程鄉縣為直隸州,名曰嘉應。皆報可。十三年,命兼轄廣西,仍駐肇慶。貴州臺拱苗亂,鄂彌達發兵令左江總兵王無黨率以赴援,復發兵駐黔、粵界,上諭獎之。

乾隆元年,高宗命近鹽場貧民販鹽毋禁。鄂彌達疏言:「廣東按察使白映棠未遵旨分別,老幼男婦發票,稱四十斤以下不許緝捕,致奸徒借口,成群販私。」上獎鄂彌達洞悉政體,解映棠任。尋奏:「廣東鹽由埸配運省河及潮州廣濟橋轉兌各埠,請令到埠先完餉銀,開倉後繳鹽價。」下部議行。御史薛馧條奏廣西團練鄉勇,並設瑤童義學,下鄂彌達議。二年,奏言:「團練鄉勇,不若訓練土司兵,於邊疆有益。瑤童義學,韶、連等屬已有成效,應如馧所奏。」尋又疏言:「惠、潮、嘉應三府州民多請州縣給票,移家入川。臣飭州縣不得濫給,並遣吏於界上察驗。」又疏言:「貴州新闢苗疆,總督張廣泗奏設屯軍墾田。臣以今苗畏威安貼,將來生齒漸繁,地少人多,必致生怨。又恐屯軍虐苗激變,請撤屯軍於附近防守,其田仍給苗民。」上諭曰:「所見甚正。廣泗首尾承辦此事,持之甚力,朕則以為終非長策也。」

四年,調川陜總督。疏言:「榆林邊民歲往鄂爾多斯種地,牛具、籽種、日用皆貸於鄂爾多斯。秋收餘糧,易牛羊皮入內地變價,重息還債。請於出口時視種地多寡,借以官銀,秋收以糧抵,俾免借貸折耗之苦,倉儲亦可漸充。」上從之。又請發司庫銀十萬買穀分貯沿邊,又請修寧夏渠道,並加築沿河長堤。又奏:「安西鎮遠兵駐防哈密,承種屯田,在城兵僅數百。年來商民日增,請視涼州柳林湖例,募流民及營兵子弟墾田,撤兵回城差操。」均如議行。

五年,兩廣總督馬爾泰劾知府袁安煜放債病民,並及鄂彌達縱僕占煤山事。上解鄂彌達任,召詣京師。尋授兵部侍郎。六年,授寧古塔將軍,調荊州。九年,授湖廣總督。疏言:「武、漢濱江城郭民田,賴有堤以障。請於武昌蕎麥灣增築大堤,安陸沙洋大堤增築月堤,襄陽老龍石堤加備歲修銀。」十一年,上以鄂彌達不稱封疆,召詣京師。十五年,授吏部侍郎。十六年,授鑲藍旗漢軍都統。二十年,授刑部尚書,署直隸總督。二十一年,兼管吏部尚書、協辦大學士。二十二年,加太子太保。二十六年,卒,予白金二千治喪,賜祭葬,謚文恭。

楊廷璋,字奉峨,漢軍鑲黃旗人。世襲佐領。雍正七年,自筆帖式授工部主事。再遷郎中。授廣西桂林知府。乾隆二年,擢左江道。十五年,擢按察使。二十年,遷湖南布政使。二十一年,授浙江巡撫。上南巡,諭曰:「西湖水民間藉以溉田。今聞沿湖多占墾,湖身漸壅,田畝虞涸竭。已開墾成熟者,免其清出,不許再侵占。」廷璋因奏:「此類田地多礙水道,請概令開濬歸湖。沿岸栽柳,俾根株盤結,亦可固堤。」又請帑疏濬湖州七十二漊,洩水入太湖,免田地被淹。又奏:「仁和、錢塘、蕭山三縣江塘視海塘例,以二十丈為準,按段編號立石。仁、錢二縣江塘民房,堤岸外餘二十餘里,視海塘例,每里設堡夫一,建堡分防。」均從之。又請開臺州黃巖場沿海地,近場歸灶,近縣歸民。戶以百畝為率,分限起科,得腴產十萬畝。奏入,嘉許。

二十四年,授閩浙總督。請改設螺洲、大頭崎、烏龍江諸地塘汛。又奏內地商舶出洋,覈給船照。又奏臺灣穀賤,內地歉收,民每偷渡就食。請酌寬米禁,往來臺、廈橫洋船準運米二百石,塘船六十石。自鹿耳門出至廈門入,皆給照察驗。臺灣與生番接壤,前總督楊應琚飭屬勘界,挑溝築土牛以杜私墾。至是,廷璋議彰化、淡水與生番接壤,依山傍溪,挑溝築土牛為界;並於沿邊設隘寮,分兵駐守。二十六年,同福建巡撫吳士功奏劾提督馬龍圖借用公使錢,並以龍圖已歸款,請用自首例減等。上責其錯謬,下吏議奪官,士功戍巴里坤,廷璋留任。二十八年,加太子太保。旋授體仁閣大學士,留總督任。二十九年,廷璋入覲。福建水師提督黃仕簡奏廈門商舶出入,官署受陋規。上命尚書舒赫德、待郎裘曰修往按。具得廷璋令歷任廈門同知代市人葠、珊瑚、珍珠未發價狀,命解任。下吏議奪官,上以廷璋平時尚能任事,授散秩大臣。未幾,授正紅旗漢軍都統、工部尚書。

三十年,命署兩廣總督。三十一年,安南捕盜,竄入小鎮安土司怕懷隘,官兵捕得。廷璋照會安南遣頭人視行誅。安南復報其國隘口盜發,請遣兵堵截。廷璋遣兵守隘。事上聞,具言防邊宜鎮靜。上戒以「邊地夷情,當審度事理,因時制宜。若專務持重,養癰貽害,弊不可勝言也。」夏,崖州安岐黎為亂,擾客民,廷璋檄鎮道捕治。並奏:「客民編保甲,禁放債。黎民市易設墟場,熟黎令薙發。民出入黎峒必譏,以杜後患。」上從之。又奏:「小鎮安改設通判。南界接安南,於那波、者賴、者欣三村,建卡設兵。怕懷隘為小鎮安門戶,設兵巡緝。打面梁與雲南接界,建卡防守。」下部議行。師征緬甸,雲貴總督楊應琚以疾聞,上令廷璋赴永昌佐應琚治軍。三十二年,疏報應琚病愈,仍回廣東任。尋召授刑部尚書。

三十三年,授直隸總督,加太子少保。秋,滹沱水盛漲。廷璋請於正定西南築堤,槁城西北築埽,並以護城。又奏勘任丘濱澱諸地,以楊各莊諸地最低,請改種稻田;文安窪修築堤墊,並於龍潭灣諸地開堤洩水,並從之。三十四年,請撥通倉米十二萬運各災區平糶。又奏:「乾隆二十四年滹沱南徙,舊河淤墊。上年大漲,河行故道。束鹿木丘、傾井諸村遂成巨浸。請裁灣取直,並修築護城堤墊。」報聞。三十六年,復召授刑部尚書。預香山九老會。十二月,卒,年八十四,贈太子太保,賜祭葬,謚勤愨。

莊有恭,字容可,廣東番禺人。乾隆四年一甲一名進士,授修撰,直上書房。後三年,弟有信成進士,引見,有恭以起居注侍直,上問及之,有信選庶吉士。兄弟同請告省親。有恭累遷侍講學士,擢光祿寺卿。以父憂歸,服除,擢內閣學士。遷戶部侍郎。督江蘇學政。充江南鄉試考官,復督江蘇學政。十六年,授江蘇巡撫。十七年,署兩江總督。疏言:「太倉、鎮洋沿海田廬,賴海塘保障。前巡撫高其倬議自寶山湖口港至昭文福山港築土塘三萬四千七百餘丈,僅築湖口港至劉河南岸土、石塘。今年秋令風潮,劉河南賴以無恙。其北頗致損傷,士民自請挑築。惟恐一時難集,工不速竟。應築土塘九千丈有奇,請借庫銀一萬六千兩,令自募夫役,於伏汛前畢工。按畝扣輸,二年清款。」如所請行。有恭督學政時,浙人丁文彬獻所著文武記、太公望傳等。有恭以為病狂,置不問。至是,文彬以書上衍聖公孔昭煥,昭煥告巡撫楊應琚以聞。有恭疏請罪,坐罰學政養廉銀十倍。

十九年,御史楊開鼎條奏江南收漕諸弊,敕有恭覆奏。尋疏言:「江南收漕諸弊,以蘇、常、松、鎮、太五屬為尤甚。已酌定條例,勒石漕倉,遇收漕,飭糧道以下官周巡察訪。開鼎言需索不遂,借詞米不如式,勒令曬晾篩颺。漕糧上供天庾,自應乾圓潔凈。儻不如式,不堪久貯,必致貽誤倉儲。糧戶良頑不等,每次青腰、白臍、潮嫩、雜碎諸米強交;如令更易,即造作浮言挾制。自應分別察究,不得但責官吏,取悅刁民。」上獎其言公正。

二十一年,丁母憂,命予假百日回籍治喪,於伏汛前至淮安,署江南河道總督。泰興縣有硃者,坐主使殺人罪至絞,乞贖罪,有恭許之,臨行疏聞。上責其專擅,令家居待罪。總督尹繼善又言有恭監臨鄉試,察出有賄謀聯號者,復有以斗蟋蟀致訟者,皆令罰鍰,未奏聞。上命奪有恭官,逮詣京師,下大學士九卿論罪,當絞。上以贓不入己,貰之,令護母喪回籍後赴軍臺效力。方詣謫所,命戴罪署湖北巡撫。

二十四年,調浙江。二十五年,劾杭州將軍伊領阿、副都統劉揚達違例乘轎。上奪伊領阿等官,獎有恭,命議敘。三月,疏言:「紹興南塘、嘉興乍浦塘並屬要工。臣赴山陰勘得宋家樓為三江、曹娥二水交會,又適當潮汐之沖,為南塘首險,已改建石塘鞏固。復至蕭山龕、長等山,越南大亹至海寧中小亹、登文堂、葛嶴諸山,勘海寧南門外,西過戴家石橋,東至陳文港,工長五千丈有奇,根址堅實,不須重建。其必當修築者千六百餘丈,內七百七十餘丈殘缺過甚,作為要工,餘次第興修。自陳文港東至尖山,下有韓家池柴塘四百丈有奇,亦應重築。復循海而北,自海鹽至平湖,遍歷乍浦塘。海鹽東臨大海,南有臺駐,北有乍浦諸山,山趾角張。縣城以一面當潮汐,城外石塘,最為險要,間有沖損,已令隨時修補。」六月,又疏言:「西塘、胡家兜至海寧南門外,潮退沙漲,長十八里。前請辦戴家石橋要工,既有新沙外護,應先就迤東工段趲辦。再審量沙勢,分別緩急。」九月,又疏言:「緩修各工,陳文港十丈,令用魚鱗式逐層整砌。圓通菴前十丈,仍如式堅築。廿里亭西二十五丈,修整坦面,加用排椿,令緊貼塘身。」二十六年十二月,又奏言:「海寧西塘、老鹽倉諸地,經霉、伏兩汛,老沙汕刷,宜先事預防,先後拆鑲二百丈。自霜降後,臣往來察勘,見柴、石兩塘交接處水已臨塘,自此迤西,老沙仍多坍卸。請將接連前工七十丈,從速鑲辦。」均從之。

二十七年,上南巡,臨視老鹽倉、尖山諸地,令修築柴塘,並設竹簍、坦水諸工。九月,疏報海寧塘工竟,上嘉有恭能盡心,命議敘。是秋多雨水漲,有恭以嘉、湖兩府水歸太湖,河道多淤,下流尤壅閼;因請浚烏程、長興境內七十二漊,並遣吏至江南按行三江故道。十月,調江蘇巡撫。上命浙江海塘工程仍責成有恭專司其事,並免學政任內應罰銀。二十九年,擢刑部尚書,留巡撫任。

有恭疏請大修三江水利,略言:「太湖北受荊溪百瀆,南受天目諸山之水,為吳中巨浸,而分疏之大幹,以三江為要。三江者,吳淞江、婁江、東江也。東江自宋已湮,明永樂間,別開黃浦,寬廣足當三江之一,今亦謂之東江。三江分流,經吳江、震澤、吳、元和、昆山、新陽、青浦、華亭、上海、太倉、鎮洋、嘉定十二州縣境,其間港浦縱橫,湖蕩參錯。大概觀之,無處不可分洩。然百節之通,不敵一節之塞。太湖出水口,不特寶帶橋一處,如吳江十八港、十七橋,吳縣占魚口、大缺口,為湖水穿運河入江要道,今不無淺阻。又如入吳淞之龐山湖、大斜港、九里湖、澱山湖、漵浦,向來寬深,近以小民貪利,遍植茭蘆,圈築魚蕩,亦多侵占。劉河,古婁江也。今河形大非昔比,舟楫來往,必艤舟待潮。昆山外濠為婁江正道,淺狹特甚。蘇州婁門外江面僅寬四五丈,偶遇秋霖,眾水匯集。江身淺窄,先為潦水所占,俟其稍退,然後湖水得出,為之傳送,而上游已漫淹矣。東南財賦重地,水利民生大計,若及早為之,事半功倍。今籌治法,當於運河西凡太湖出水之口,皆為清釐占塞,俾分流無阻。其運河東三江故道,惟黃浦現在深通,但於泖口挑去新漲蘆墩,足資宣洩。吳淞江自龐山湖以下,婁江自婁門以下,凡有淺狹阻滯之處,宜濬治寬深,令上流所洩之數,足相容納。其江身所有植蘆插籪及冒占之區,盡數剷除,嗣後仍嚴為之禁。則水之停蓄有所,傳送以時,並即以挑河之土加培圩岸。現在徬座去海太近,難於啟閉者,酌量改移,庶渾潮不入,清水盛強,而海口之淤,亦將不挑而自去。總計所需雖覺浩繁,然散在十二州縣,通力合作,實亦無多。民間聞有此舉,咸樂趨事,原以民力為之。但分段督修,仍須官董其成;且工費繁多,若待鳩財而後興工,稍稽時日。懇發帑興工,仍於各州縣分年按畝徵還,則民力既紓,工可速集。」奏入,報可。於是選紳耆,賦工役,先疏橋港,次及河身。茭蘆魚蕩之圈占者,除之;城市民居之不可毀者,別開月河以導之。工始於二十八年十二月,至二十九年三月告竟,用公帑二十二萬有奇。

三十年正月,命協辦大學士,仍暫留巡撫任。南巡,復賜詩褒勉。八月,召詣京師。有恭劾蘇州同知段成功縱役累民,奪官,讞未定。巡撫明德察成功實受賕,詐稱病;按察使硃奎揚、知府孔傳鶺皆知之,不以言。上命奪奎揚等官,逮訊。三十一年正月,罷有恭協辦大學士。又遣侍郎四達按治,得有恭授意奎揚等有意從寬狀,並奪有恭官,下刑部獄。軍機大臣會鞫,並追繳學政任內應罰銀。二月,軍機大臣等讞上,有恭罪應斬,諭改監候。八月,命原之。授福建巡撫。三十二年,卒。仍免追繳學政任內應罰銀。

李侍堯,字欽齋,漢軍鑲黃旗人,二等伯李永芳四世孫也。父元亮,官戶部尚書,謚勤恪。侍堯,乾隆初以廕生授印務章京,見知高宗。累遷至正藍旗漢軍副都統。十七年,調熱河副都統。二十年,擢工部侍郎,調戶部。署廣州將軍。劾前將軍錫特庫廢弛馬政,錫特庫下吏議。奏定廣州滿洲、漢軍駐防官制兵額。二十一年,署兩廣總督。奏:「廣東各屬買補倉穀,兼雜上、中、下三等,而報以上價。應碾米,用上穀;應借糶,用中、下穀。」上諭以所言洞悉情弊,諭各省督撫嚴飭州縣買補當碾試,務得上穀。又請禁廣東制錢攙和古錢,並吳三桂偽號錢事。上諭以「前代錢仍聽行用。吳三桂利用偽號錢,令民間檢出,官為收換,供鼓鑄之用」。又奏廣州駐防出旗漢軍官兵曠米,平糶便民,上從之。二十三年,守備張彬佐禁村民演劇被毆,奏請飭讞。上謂:「未得懲創惡習之意。應先治刁民,後議劣弁,庶刁悍之徒知畏懼。」

二十四年,實授總督。奏:「廣東各國商舶所集,請飭銷貨後依期回國,不得住冬;商館毋許私行交易;毋許貸與內地行商貲本;毋許雇內地廝役。」二十五年,又奏:「粵海關各國商舶出入,例於正稅船鈔外有各種規禮,應請刪除名色,並為歸公銀若干。各口僕役飯食、舟車諸費,於此覈銷。」並下部議行。廣西巡撫鄂寶以貴縣僮民韋志剛不法,知縣石崇光察報,避重就輕,請奪官。上以事由崇光察報,命毋奪官;侍堯奏先經面諭崇光體勘,始行察報,上令逮崇光按鞫。又奏志剛實無不法事,崇光猜疑妄報,仍奪崇光官。上以侍堯與鄂寶各懷意見,飭以「秉虛公,除習氣」。

二十六年,召授戶部尚書、正紅旗漢軍都統,襲勛舊佐領。二十八年,授湖廣總督。奏:「湖廣行銷淮鹽,抬價病民,請酌中定價。」命兩淮鹽政高恆赴湖廣會議,奏請按淮商成本,酌加餘息,明定限制,從之。加太子太保。

二十九年,調兩廣總督。右江鎮總兵李星垣坐婪賄得罪,命侍堯按鞫,擬絞。上以侍堯嘗薦星垣,今擬罪輕縱,責侍堯回護,坐降調。以憂還京師。署工部尚書。三十一年,調署刑部。三十二年,回兩廣總督任。襲二等昭信伯。三十四年,師征緬甸,命侍堯傳檄暹羅。時暹羅方為甘恩敕所據,侍堯以為不宜傳檄;以己意宣諭暹羅各夷目,密偵緬甸,茍入境,令擒以獻,上韙之。豐順民硃阿姜謀為亂,督吏捕治。

三十八年,授武英殿大學士,仍留總督任。安南內亂,令廣西鎮、道嚴防。入覲,賜黑狐端罩。四十年,兵部以廣東民糾黨結盟,不數月至五起,當追論武職弛縱罪。侍堯奏言:「武職既協緝,復追論弛縱罪,則規免處分,必致暗為消弭,兇徒轉得漏網,請寬之。」上從其請,並諭曰:「侍堯此奏,意在挽回積習。然亦惟侍堯向不姑息屬僚,朕所深信,始可為此言。若他人,未可輕為仿效也。」

四十二年,雲貴總督圖思德奏緬甸投誠,籥請納貢。上命大學士阿桂往蒞其事,並調侍堯雲貴總督。緬甸頭人孟幹謁侍堯,請緩貢。侍堯偕阿桂奏:「孟幹等語反覆,遵旨斷接濟,絕偵探,示以威德,不予遷就。」上召阿桂還。緬甸歸所留守備蘇爾相,侍堯遣詣京師。緬甸乞遣孟幹等還,侍堯諭令歸所留按察使銜楊重英,上嘉其合機宜。四十三年,奏獲緬甸遣騰越州民入關為諜,送京師。尋奏:「永昌、普洱界連緬甸,擬每歲派兵五千五百,在張鳳街、三臺山、九龍口諸地防守。」上諭以「揆度邊情,不值如此辦理」。侍堯復請於杉木隴設大汛,撥騰越兵五百;千崖設小汛,撥南甸兵二百,輪駐巡防;並分守虎踞、銅壁等關。從之。四十五年,雲南糧儲道海寧訴侍堯貪縱營私狀,命尚書和珅、侍郎喀寧阿按治。侍堯自承得道府以下餽賂,不諱,上震怒,諭曰:「侍堯身為大學士,歷任總督,負恩婪索,朕夢想所不到!」奪官,逮詣京師。和珅等奏擬斬監候,奪爵以授其弟奉堯。又下大學士九卿議,改斬決,上心欲寬之,復下各直省督撫議。各督撫多請照初議定罪,獨江蘇巡撫閔鶚元迎上意,奏:「侍堯歷任封疆,幹力有為。請用議勤議能之例,寬其一線。」上乃下詔,謂:「罪疑惟輕,朕不為已甚。」改斬監候。

四十六年,甘肅撒拉爾回蘇四十三為亂,上遣大學士阿桂視師。特旨予侍堯三品頂戴、孔雀翎,赴甘肅治軍事。甘肅冒賑事發,總督勒爾謹得罪,命侍堯領總督事,會阿桂按治。勒爾謹及前布政使王亶望、布政使王廷贊、蘭州知府蔣全迪皆坐斬。上命諸州縣侵冒二萬以上擬斬決,一萬以下斬候,於是皋蘭知縣程棟等二十人皆坐斬。四十七年,奏:「皋蘭等三十四、州、縣虧庫帑八十八萬有奇、倉糧七十四萬有奇,請於現任總督以下各官養廉扣抵歸補。」上命寬免。又請豁免節年民欠三十萬兩。旋命予現任品級頂帶,加太子太保。四十九年,廣東鹽商譚達元訴侍堯任兩廣時,總商沈冀州斂派公費餽送,上命尚書福康安按鞫,請罪侍堯。上責侍堯償繳公費,免其罪。

蘇四十三亂既定,上屢諭侍堯密察新教回民。至是,鹽茶回田五等復為亂,侍堯會固原提督剛塔捕田五。田五自戕,得其孥誅之。無何,田五之徒復攻靖遠。侍堯駐靖遠,令剛塔督兵往,亂久未定。上命大學士阿桂、尚書福康安視師。渭城陷,西安副都統明善戰死,賊據石峰堡。上責侍堯玩延怯懦,奪官,仍在軍效力督餉。侍堯旋督兵赴伏羌。福康安至軍,發侍堯玩愒貽誤諸罪狀。逮熱河行在,王大臣按鞫,擬斬決。上仍令從寬改監候。五十年,諭釋之。署正黃旗漢軍都統。署戶部尚書。

湖北江陵民訴知縣孔毓檀侵賑,命侍堯往按。奏言毓檀未侵賑,但治賑遲緩,坐奪官。命署湖廣總督。奏上年孝感被災饑民劉金立等掠穀,生員梅調元糾眾毆殺金立,並生瘞二十三人。上逮前總督特成額及知縣秦樸等治其罪。未幾,實授。

五十二年,入覲。臺灣民林爽文為亂,調侍堯閩浙總督,駐蚶江。時前總督常青督兵渡臺灣,侍堯以兵力不足,調廣東、浙江兵濟師。又慮賊據笨港劫糧械,撥繪船分防鹿耳門、鹿仔港。上獎以籌濟有方。亂久未定,上以常青非將才,命福康安為將軍督師;並寄諭常青全師以歸,待福康安至,再籌進取。侍堯恐常青宣露上旨,人心惶惑,節錄發寄,並具疏請罪。上大悅,獎以「深合機宜,得大臣體」。賜雙眼孔雀翎。福康安劾提督柴大紀,上責侍堯徇隱。五十三年,侍堯亦奏大紀貪劣諸狀,自請治罪,上寬之。臺灣平,命仍襲伯爵。建福康安等生祠於臺灣,命侍堯居福康安、海蘭察之次。復命圖形紫光閣,列前二十功臣。

侍堯短小精敏,過目成誦。見屬僚,數語即辨其才否。擁幾高坐,語所治肥瘠利害,或及其陰事,若親見。人皆悚懼。屢以貪黷坐法,上終憐其才,為之曲赦。十月,疾聞,命其子侍衛毓秀往省。旋卒,謚恭毅。

弟奉堯,自官學生襲勛舊佐領,授藍翎侍衛。累遷江南提督。四十五年,襲伯爵。四十六年,調福建陸路提督。以漳、泉累有械斗,左授馬蘭鎮總兵。五十二年,署直隸提督。山東學政劉權之移家,舟經靜海被盜,下吏議。上以署事未久,且隨扈熱河,寬之。五十三年,侍堯還襲伯爵,加奉堯提督銜。五十四年,卒,謚慎簡。子毓文,乾隆六十年,侍堯督云、貴與局員通同偷減錢法事發,奪毓秀伯爵,命毓文承襲。

伍彌泰,伍彌氏,蒙古正黃旗人,副將軍三等伯阿喇納子。伍彌泰以雍正二年襲爵。授公中佐領,擢散秩大臣,遷鑲白旗蒙古副都統。乾隆十五年,賜伯號曰誠毅。二十年,授涼州將軍。旋命以將軍銜駐西藏辦事,二十四年,代還,授正藍旗蒙古都統。出為江寧將軍。二十七年,上以伍彌泰不勝任,召還,仍為散秩大臣。命協辦伊犁事務。哈薩克越境游牧,師逐之出塞。上以伍彌泰不諳軍務,令隨行學習。二十八年,命往烏魯木齊辦事。築精河屯堡,上賜名曰綏來。三十一年,代還,署鑲黃蒙古、正白漢軍兩旗都統。授內大臣。三十五年,命往西寧辦事。郭羅克土番劫洞庫爾種人行李,伍彌泰遣兵逐捕,得行李以還。奏聞,上以未痛剿,責伍彌泰怠忽。三十八年,改駐西藏辦事。四十一年,代還,擢理籓院尚書,兼鑲白旗漢軍都統。出為綏遠城將軍,調西安。四十三年,伊犁將軍伊勒圖請以屯田無眷屬之兵次第撤回,下伍彌泰議。選陜、甘綠營兵三千攜眷屬以往。四十五年,班禪額爾德尼詣京師,命伍彌泰護行,仍還西安。

四十六年,撒拉爾回蘇四十三等為亂,陷河州。上命伍彌泰選兵千人備徵發。伍彌泰奏提督馬彪已率兵赴河州,擬選滿洲兵千繼往。上以所奏與諭旨合,深嘉之。上命大學士阿桂視師,督軍攻華林山梁,命伍彌泰駐龍尾山為聲援。回亂旋定,捕得阿渾五。有海潮宗者,嘗出降,彪遣往開諭,遂留從亂。上責伍彌泰等不先奏聞,下吏議奪官,上寬之。

四十八年,授吏部尚書、協辦大學士、鑲白旗蒙古都統,充上書房總諳達。四十九年,上巡江、浙,命留京辦事,授東閣大學士。上以伍彌泰年逾七十,命與大學士嵇璜、蔡新俱日出後入朝,風雪沍寒,免其入直。五十年,預千叟宴。五十一年,卒,贈太子太保,賜祭葬,謚文端。

伍彌泰治事知大體。班禪額爾德尼至京師,王大臣多和南稱弟子。伍彌泰護行,與抗禮。

官保,烏雅氏,滿洲正黃旗人。初授刑部筆帖式,擢堂主事。累遷郎中。乾隆七年,授江南江寧知府。十一年,總督尹繼善奏官保不宜外任,復授刑部員外郎。轉郎中,改御史。擢刑科給事中,巡視臺灣。二十二年,擢鑲黃旗漢軍副都統,往西藏辦事。二十六年,授刑部侍郎。三十年,調工部。三十二年,復往西藏辦事,察知糧務通判吳元澄以庫銀貿易。上以官保初至藏即察奏,嘉其急公,讞實,論斬。歷正紅旗蒙古、滿洲都統,理籓院,刑、禮、戶諸部尚書。三十四年,協辦大學士。上幸熱河,命留京辦事。三十八年,調吏部。四十一年,以年逾八十乞休,命致仕。卒,賜祭葬,謚文勤。

論曰:廷桂嘗言:「事英主有法。若先有市惠、好名、黨援諸病,上所知,便一事不可行。」其言深中高宗之隱,被眷遇宜矣。侍堯眷遇尤厚,屢坐贓敗,屢屈法貸之。蓋特憐其才,非以其工進獻也。阿彌達、廷璋皆以不謹聞,亦未竟其罪。有恭撫江、浙,治海塘,重水利,有惠於民。其被譴尚非有所私,視侍堯輩故當勝。伍彌泰雖未嘗領疆寄,久於邊徼,恩被延登,在當時亦勞臣也,因附著之。


\end{pinyinscope}