\article{本紀四}

\begin{pinyinscope}
世祖本紀一

世祖體天隆運定統建極英睿欽文顯武大德弘功至仁純孝章皇帝,諱福臨,太宗第九子。母孝莊文皇后方娠,紅光繞身,盤旋如龍形。誕之前夕,夢神人抱子納後懷曰:「此統一天下之主也。」寤,以語太宗。太宗喜甚,曰:「奇祥也,生子必建大業。」翌日上生,紅光燭宮中,香氣經日不散。上生有異稟,頂發聳起,龍章鳳姿,神智天授。

八年秋八月庚午,太宗崩,儲嗣未定。和碩禮親王代善會諸王、貝勒、貝子、文武群臣定議,奉上嗣大位,誓告天地,以和碩鄭親王濟爾哈朗、和碩睿親王多爾袞輔政。丙子,阿濟格尼堪等率師防錦州。丁丑,多羅郡王阿達禮、固山貝子碩託謀立和碩睿親王多爾袞。禮親王代善與多爾袞發其謀。阿達禮、碩託伏誅。乙酉,諸王、貝勒、貝子、群臣以上嗣位期祭告太宗。丙戌,以即位期祭告郊廟。丁亥,上即皇帝位於篤恭殿。詔以明年為順治元年,肆赦常所不原者。頒哀詔於朝鮮、蒙古。

九月辛丑,地震,自西北而南有聲。壬寅,濟爾哈朗、阿濟格征明,攻寧遠衛。丙午,頒即位詔於朝鮮、蒙古。以太宗遺詔減朝鮮歲貢。辛亥,昭陵成。乙卯,大軍攻明中後所,丁巳拔之。庚申,攻前屯衛。

冬十月辛酉朔,克之。阿濟格尼堪等率師至中前所,明總兵官黃色棄城遁。丁丑,濟爾哈朗、阿濟格師還。壬午,篇古、博和託、伊拜、杜雷代戍錦州。

十二月壬戌,明守備孫友白自寧遠來降。辛未,朝鮮來賀即位。乙亥,罷諸王、貝勒、貝子管部院事。鄂羅塞臣、巴都禮率師征黑龍江。壬午,譚泰、準塔代戍錦州。

是歲,朝鮮暨土默特部章京古祿格,庫爾喀部賴達庫及炎楮庫牙喇氏二十六戶,索倫部章京崇內,喀爾喀部土謝圖汗、馬哈撒嘛諦塞臣汗、查薩克圖汗,圖白忒部甸齊喇嘛俱來貢。

順治元年春正月庚寅朔,御殿受賀,命禮親王代善勿拜。甲午,沙爾虎達率師征庫爾喀。己亥,來達哈巴圖魯等代戍錦州。鄭親王濟爾哈朗諭部院各官,凡白事先啟睿親王,而自居其次。

二月辛巳,艾度禮戍錦州。戊子,祔葬太妃博爾濟錦氏於福陵,改葬妃富察氏於陵外。富察氏,太祖時以罪賜死者。

三月丙申,地震。戊戌,復震。甲寅,大學士希福等進刪譯遼、金、元史。是月,流賊李自成陷燕京,明帝自經。自成僭稱帝,國號大順,改元永昌。

夏四月戊午朔,固山額真何洛會等訐告肅親王豪格悖妄罪,廢豪格為庶人,其黨俄莫克圖等皆論死。己未,晉封多羅饒餘貝勒阿巴泰為多羅饒餘郡王。辛酉,大學士範文程啟睿親王入定中原。甲子,以大軍南伐祭告太祖、太宗。乙丑,上御篤恭殿,命和碩睿親王多爾袞為奉命大將軍,賜敕印便宜行事,並賜王及從征諸王、貝勒、貝子等服物有差。丙寅,師行。壬申,睿親王多爾袞師次翁後,明山海關守將吳三桂遣使致書,乞師討賊。丁丑,師次連山,三桂復致書告急,大軍疾馳赴之。戊寅,李自成率眾圍山海關,我軍逆擊之,敗賊將唐通於一片石。己卯,師至山海關,三桂開關出迎,大軍入關。自成率眾二十餘萬,自北山橫亙至海,嚴陣以待。是日,大風,塵沙蔽天。睿親王多爾袞命擊賊陣尾,以三桂居右翼,大呼薄之。風旋定,賊兵大潰,追奔四十餘里,自成遁還燕京。封三桂為平西王,以馬步軍一萬隸之,直趨燕京。誓諸將勿殺不辜,掠財物,焚廬舍,不如約者罪之。諭官民以取殘不殺之意,民大悅,竄匿山谷者爭還鄉里迎降。大軍所過州縣及沿邊將吏皆開門款附。乙酉,自成棄燕京西走,我軍疾追之。

五月戊子朔,以捷書宣示朝鮮、蒙古。己丑,大軍抵燕京,故明文武諸臣士庶郊迎五里外。睿親王多爾袞入居武英殿。令諸將士乘城,廝養人等毋入民家,百姓安堵如故。庚寅,令兵部傳檄直省郡縣,歸順者官吏進秩,軍民免遷徙,文武大吏籍戶口錢糧兵馬親賚至京,觀望者討之。故明諸王來歸者,不奪其爵。在京職官及避賊隱匿者,各以名聞錄用,卒伍欲歸農者聽之。辛卯,令官吏軍民為明帝發喪,三日後服除,禮部太常寺具帝禮以葬。壬辰,俄羅塞臣、巴都禮、沙爾虎達等征黑龍江師還。故明山海關總兵官高第來降。癸巳,令故明內閣、部院諸臣以原官同滿洲官一體辦理。乙未,阿濟格等追擊李自成於慶都,敗之。譚泰、準塔等追至真定,又破走之。燕京迤北各城及天津、真定諸郡縣皆降。辛丑,徵故明大學士馮銓至京。己酉,葬故明莊烈帝後周氏、妃袁氏,熹宗後張氏,神宗妃劉氏,並如制。

六月丁巳朔,令洪承疇仍以兵部尚書同內院官佐理機務。己未,以駱養性為天津總督。庚申,遣戶部右侍郎王鰲永招撫山東、河南。壬戌,故明大同總兵官姜瓖斬賊首柯天相等,以大同來降。丙寅,遣巴哈納、石廷柱率師定山東。免京城官用廬舍賦稅三年,與同居者一年,大軍所過州縣田畝稅之半,河北府州縣三之一。丁卯,睿親王多爾袞及諸王、貝勒、貝子、大臣定議建都燕京,遣輔國公屯齊喀、和託、固山額真何洛會奉迎車駕。庚午,遣固山額真葉臣率師定山西。甲戌,故明三邊總督李化熙降。壬午,上遣使勞軍。癸未,艾度禮有罪,伏誅。甲申,遷故明太祖神主於歷代帝王廟。乙酉,鑄各官印兼用國書。

秋七月丁亥,考定歷法,為時憲歷。戊子,巴哈納、石廷柱會葉臣軍定山西。壬辰,以吳孳昌為宣大山西總督,方大猷為山東巡撫。癸巳,以遷都祭告上帝、陵廟。丁酉,故明德王硃由縢降。時故明福王硃由崧即位江南,改元弘光,以史可法為大學士,駐揚州督師,總兵劉澤清、劉良佐、黃得功、高傑分守江北。己亥,山東巡按硃朗鑅啟新補官吏仍以紗帽圓領臨民蒞事。睿親王多爾袞諭:「軍事方殷,衣冠禮樂未遑制定。近簡各官,姑依明式。」庚子,設故明長陵以下十四陵官吏。辛丑,免盛京滿、漢額輸糧草布疋。壬寅,大赦,除正額外一切加派。癸卯,罷內監徵收涿州、寶坻皇莊稅糧。甲辰,以楊方興為河南總督,馬國柱為山西巡撫,陳錦為登萊巡撫。免山東稅,如河北例。壬子,睿親王以書致史可法,勸其主削號歸籓。可法答書不屈。以王文奎為保定巡撫,羅繡錦為河南巡撫。裁六部蒙古侍郎。癸丑,雨雹。是月,建乾清宮。

八月丙辰朔,日有食之。丁巳,以何洛會為盛京總管,尼堪、碩詹統左右翼,鎮守盛京。辛酉,大學士希福有罪,免。癸亥,行總甲法。戊辰,免景州、河間、阜城、青縣本年額賦。己巳,定在京文武官薪俸。乙亥,車駕發盛京。庚辰,次蘇爾濟,察哈爾固倫公主及蒙古王貝勒等朝行在。壬午,徵故明大學士謝升入內院辦事。癸未,次廣寧,給故明十三陵陵戶祭田,禁樵牧。

九月甲午,車駕入山海關。丁酉,次永平。始嚴稽察逃人之令。己亥,建堂子於燕京。庚子,賊將唐通殺李自成親族乞降。辛丑,遣和託、李率泰、額孟格等率師定山東、河南。癸卯,車駕至通州。睿親王多爾袞率諸王、貝勒、貝子、文武群臣朝上於行殿。甲辰,上自正陽門入宮。己酉,太白晝見。庚戌,初定郊廟樂章。睿親王多爾袞率諸王及滿、漢官上表勸進。故明福王遣其臣左懋第、馬紹愉、陳洪範齎白金十餘萬兩、黃金千兩、幣萬匹求成。壬子,奉安太祖武皇帝、孝慈武皇后、太宗文皇帝神主於太廟。

冬十月乙卯朔,上親詣南郊告祭天地,即皇帝位,遣官告祭太廟、社稷。初頒時憲歷。丙辰,以孔子六十五代孫允植襲封衍聖公,其五經博士等官襲封如故。丁巳,以睿親王多爾袞功最高,命禮部建碑紀績。辛酉,上太宗尊謚,告祭郊廟社稷。壬戌,流賊餘黨趙應元偽降,入青州,殺招撫侍郎王鰲永,和託等討斬之。甲子,上御皇極門,頒詔天下,大赦。詔曰:「我國家受天眷佑,肇造東土。列祖創興宏業,皇考式廓前猷,遂舉舊邦,誕膺新命。迨朕嗣服,越在沖齡,敬念紹庭,永綏厥位。頃緣賊氛洊熾,極禍中原,是用倚任親賢,救民塗炭。方馳金鼓,旋奏澄清,用解倒懸,非富天下。而王公列闢文武群臣暨軍民耆老合詞勸進,懇請再三。乃以今年十月乙卯朔,祗告天地宗廟社稷,定鼎燕京,仍建有天下之號曰大清,紀元順治。緬維峻命不易,創業尤艱。況當改革之初,爰沛維新之澤。親王佐命開國,濟世安民,有大勛勞,宜加殊禮。郡王子孫弟侄應得封爵,所司損益前典以聞。滿洲開國諸臣,運籌帷幄,決勝廟堂,汗馬著勛,開疆拓土,應加公、侯、伯世爵,錫以誥券。大軍入關以來,文武官紳,倡先慕義,殺賊歸降,亦予通行察敘。自順治元年五月朔昧爽以前,官吏軍民罪犯,非叛逆十惡死在不赦者,罪無大小,咸赦除之。官吏貪賄枉法,剝削小民,犯在五月朔以後,不在此例。地畝錢糧,悉照前明會計錄,自順治元年五月朔起,如額徵解。凡加派遼餉、新餉、練餉、召買等項,俱行蠲免。大軍經過地方,仍免正糧一半,歸順州縣非經過者,免本年三分之一。直省起存拖欠本折錢糧,如金花、夏稅、秋糧、馬草、人丁、鹽鈔、民屯、牧地、灶課、富戶、門攤、商稅、魚課、馬價、柴直、棗株、鈔貫、果品及內供顏料、蠟、茶、芝麻、棉花、絹、布、絲釂等項,念小民困苦已極,自順治元年五月朔以前,凡屬逋征,概予豁除。兵民散居京城,實不獲已,其東中西三城已遷徙者,準免租賦三年;南北二城雖未遷徙,亦免一年。丁銀原有定額,年來生齒凋耗,版籍日削,孤貧老弱,盡苦追呼,有司查覈,老幼廢疾,並與豁免。軍民年七十以上者,許一丁侍養,免其徭役;八十以上者,給與絹釂米肉;有德行著聞者,給與冠帶;鰥寡孤獨、廢疾不能自存者,官與給養。孝子順孫義夫節婦,有司諮訪以聞。故明建言罷謫諸臣及山林隱逸懷才抱德堪為世用者,撫按薦舉,來京擢用。文武制科,仍於辰戌丑未年舉行會試,子午卯酉年舉行鄉試。前明宗室首倡投誠者,仍予祿養。明國諸陵,春秋致祭,仍用守陵員戶。帝王陵寢及名臣賢士墳墓毀者修之,仍禁樵牧。京、外文武職官應得封誥廕敘,一體頒給。北直、河南、山東節裁銀,山西太原、平陽二府新裁銀,前明已經免解,其二府舊裁銀,與各府新舊節裁銀兩,又會同館馬站、驢站館夫及遞運所車站夫價等銀,又直省額解工部四司料銀、匠價銀、磚料銀、★麻銀、車價銀、葦夫銀、葦課銀、漁課銀、野味銀、翎毛銀、活鹿銀、大鹿銀、小鹿銀、羊皮銀、弓箭撒袋折銀、扣剩水腳銀、牛角牛筋銀、鵝翎銀、天鵝銀、民夫銀、椿草子粒銀、狀元袍服銀、衣糧銀、砍柴夫銀、搬運木柴銀、抬柴夫銀、蘆課等折色銀、盔甲、腰刀、弓箭、弦條、胖襖、褲、鞋、狐麂兔貍皮、山羊毛課、鐵、黃櫨、榔、桑、胭脂、花梨、南棗、紫榆、杉條等木、椴木、桐木、板枋、冰窖物料、蘆席、蒲草、榜紙、赩罈、槐花、烏梅、梔子、筆管、芒帚、竹掃帚、席草、粗細銅絲、鐵線、鍍白銅絲、鐵條、碌子、青花棉、松香、光葉書籍紙、嚴漆、罩漆、桐油、毛、筀、紫、水斑等竹、實心竹、棕毛、白圓藤、翠毛、石磨、川二硃、生漆、沙葉、廣膠、焰硝、螺殼等本色錢糧,自順治元年五月朔以前逋欠在民,盡予蠲免,以甦民困。後照現行事例,分別蠲除。京師行商車戶等役,每遇僉役,頓至流離,嗣後永行豁除。運司鹽法,遞年增加,有新餉、練餉雜項加派等銀,深為厲商,盡行豁免,本年仍免額引三分之一。關津抽稅,非欲困商,準免一年,明末所增,並行豁免。直省州縣零星稅目,概行嚴禁。曾經兵災地方應納錢糧,已經前明全免者,仍與全免,不在免半、免一之例。直省報解屯田司助工銀兩,亦出加派,準予豁除。直省領解錢糧被賊劫失,在順治元年五月朔以前,一並豁免。山、陜軍民被流寇要挾,悔過自新,概從赦宥,脅從自首者前罪勿論。巡按以訪拿為名,聽信衙蠹,誣罰良民,最為弊政,今後悉行禁革。勢家土豪,重利放債,致民傾家蕩產,深可痛恨,今後有司勿許追比。越訴誣告,敗俗傷財,大赦以後,戶婚小事,俱就有司歸結,如有訟師誘陷愚民入京越訴者,加等反坐。贖鍰之設,勸人自新,追比傷生,轉為民害,今後並行禁止,不能納者,速予免追。惟爾萬方,與朕一德。播告遐邇,咸使聞知。」加封和碩睿親王多爾袞為叔父攝政王。乙丑,以雷興為天津巡撫。丁卯,加封和碩鄭親王濟爾哈朗為信義輔政叔王,復封豪格為和碩肅親王,進封多羅武英郡王阿濟格為和碩英親王,多羅豫郡王多鐸為和碩豫親王,貝勒羅洛宏為多羅衍禧郡王,封碩塞為多羅承澤郡王。葉臣等克太原。故明副將劉大受自江南來降。辛未,封貝子尼堪、博洛為多羅貝勒,輔國公滿達海、吞齊、博和託、吞齊喀、和託、尚善為固山貝子。定諸王、貝勒、貝子歲俸。癸酉,以英親王阿濟格為靖遠大將軍,率師西討李自成。戊寅,定攝政王冠服宮室之制。己卯,以豫親王多鐸為定國大將軍,率師征江南。檄諭故明南方諸臣,數其不能滅賊復,擁眾擾民,自生反側,及無明帝遺詔擅立福王三罪。

十一月乙酉朔,設滿洲司業、助教,官員子孫有欲習國書、漢書者,並入國子監讀書。故明福王使臣陳洪範南還,中途密啟請留左懋第、馬紹愉,自欲率兵歸順,招徠南中諸將。許之。壬辰,石廷柱、巴哈納、席特庫等敗賊於平陽,山西悉平。庚子,封唐通為定西侯。甲辰,罷故明定陵守者,其十二陵仍設太監二名,量給歲時祭品。丁未,祀天於圜丘。庚戌,封勒克德渾為多羅貝勒。遣朝鮮質子李★歸國,並制減其歲貢。

十二月丁巳,出故明府庫財物,賞八旗將士及蒙古官員。葉臣等大軍平直隸、河南、山西府九、州二十七、縣一百四十一。丁卯,以太宗第六女固倫公主下嫁固山額真阿山子誇扎。戊辰,多鐸軍至孟津,賊將黃士欣等遁走,濱河十五寨堡望風納款,睢州賊將許定國來降。己巳,多鐸軍至陜州,敗賊將張有曾於靈寶。丁丑,諭戶部清查無主荒地給八旗軍士。己卯,遣何洛會等祭福陵,鞏阿岱等祭昭陵,告武成。辛巳,有劉姓者自稱明太子,內監楊玉引入故明嘉定侯周奎宅,奎以聞。故明宮人及東宮舊僚辨視皆不識。下法司勘問,楊玉及附會之內監常進節、指揮李時廕等十五人皆棄市。仍諭中外,有以故明太子來告者給賞,太子仍加恩養。

是歲,朝鮮暨虎什喀裡等八姓部,鄂爾多斯部濟農,索倫部章京敖爾拖木爾,歸化城土默特部古祿格,喀爾喀部塞臣綽爾濟、古倫地瓦胡土克圖、餘古折爾喇嘛、土謝圖汗,蘇尼特部騰機思阿喇海,烏硃穆秦部臺吉滿瞻俱來貢。

二年春正月戊子,圖賴等破李自成於潼關,賊倚山為陣,圖賴率騎兵百人掩擊,多所斬獲。至是,自成親率馬步兵迎戰,又數敗之,賊眾奔潰。己未,大軍圍潼關,賊築重壕,堅壁以守。穆成格、俄羅塞臣先登,諸軍繼進,復大敗之。自成遁走西安。丙申,阿濟格、尼堪等率師抵潼關,賊將馬世堯降,旋以反側斬之。丁酉,命多羅饒餘郡王阿巴泰為總統,固山額真準塔為左翼,梅勒章京譚泰為右翼,代豪格征山東。庚子,以太宗第七女固倫公主下嫁內大臣鄂齊爾桑子喇瑪思。河南孟縣河清二日。壬寅,多鐸師至西安,自成奔商州。癸卯,大學士謝升卒。乙巳,真定、大名、順德、廣平山賊悉平。丙午,命房山縣歲以太牢祭金太祖、世宗陵。丁未,免山西今年額賦之半。更國子監孔子神位為大成至聖文宣先師孔子。庚戌,禁包衣大等私收投充漢人,冒占田宅,違者論死。壬子,免濟源、武陟、孟、溫四縣今年額賦及磁、安陽等九州縣之半。癸丑,免修邊民壯八千餘人。

二月丙辰,阿巴泰敗賊於徐州。己未,修律例。以李鑒為宣大總督,馮聖兆為宣府巡撫。降將許定國襲殺明興平伯高傑於睢州。辛酉,諭豫親王多鐸移師定江南,英親王阿濟格討流寇餘黨。丙寅,禁管莊撥什庫毀民墳塋。己巳,以祁充格為內弘文院大學士。庚午,阿濟格剿陜西餘寇,克四城,降三十八城。丁丑,多鐸師至河南,賊將劉忠降。

三月甲申朔,始祀遼太祖、金太祖、世宗、元太祖、明太祖於歷代帝王廟,以其臣耶律曷魯、完顏粘沒罕、斡離不、木華黎、伯顏、徐達、劉基從祀。庚寅,多鐸師出虎牢關,分遣固山額真拜伊圖等出龍門關,兵部尚書韓岱、梅勒章京宜爾德、侍郎尼堪等由南陽合軍歸德,所過迎降,河南悉平。辛卯,免山東荒賦。庚子,故明大學士李建泰來降。乙巳,遣八旗官軍番戍濟寧。丙午,朝鮮國王次子李淏歸。己酉,免薊州元年額賦。壬子,太行諸賊悉平。

夏四月丙辰,遣漢軍八旗官各一員駐防盛京。辛酉,以王文奎為陜西總督,焦安民為寧夏巡撫,黃圖安為甘肅巡撫,故明尚書張忻為天津巡撫,郝晉為保定巡撫,雷興為陜西巡撫。甲子,葬故明殉難太監王承恩於明帝陵側,給祭田,建碑。己丑,多鐸師至泗州。阿山等取泗北淮河橋,明守將焚橋遁,我軍遂夜渡淮。丁卯,諭曰:「流賊李自成殺君虐民,神人共憤。朕誕膺天命,撫定中華,尚復竊據秦川,抗阻聲教。爰命和碩豫親王移南伐之眾,直搗崤、函,和碩英親王秉西征之師,濟自綏德,旬月之間,全秦底定。憫茲黎庶,咸與維新。其為賊所脅誤者,悉赦除之,並蠲一切逋賦。大軍所過,免今年額賦之半,餘免三之一。」庚午,豫親王多鐸師至揚州,諭故明閣部史可法、翰林學士衛胤文等降。不從。甲戌,以孟喬芳為陜西三邊總督。以太宗第八女固倫公主下嫁科爾沁土謝圖親王巴達禮子巴雅斯護朗。丁丑,拜尹圖、圖賴、阿山等克揚州,故明閣部史可法不屈,殺之。辛巳,初行武鄉試。

五月壬午朔,河道總督楊方興進瑞麥。上曰:「歲豐民樂,即是禎祥,不在瑞麥。當惠養元元,益加撫輯。」癸未,以旱諭刑部慮囚。命內三院大學士馮銓、洪承疇、李建泰、範文程、剛林、祁充格等纂修明史。丙戌,多鐸師至揚子江,故明鎮海伯鄭鴻逵等以舟師分守瓜洲、儀真,我軍在江北,拜尹圖、圖賴、阿山率舟師自運河潛濟,梅勒章京李率泰乘夜登岸,黎明,我軍以次畢渡,敵眾咸潰。丁亥,以王志正為延綏巡撫。免高密元年額賦。賜諸王以下及百官冰,著為令。己丑,宣府妖民劉伯泗謀亂伏誅。庚寅,以王文奎為淮揚總督,趙福星為鳳陽巡撫。丙申,多鐸師至南京,故明福王硃由崧及大學士馬士英遁走太平,忻城伯趙之龍、大學士王鐸、禮部尚書錢謙益等三十一人以城迎降。興平伯高傑子元照、廣昌伯劉良佐等二十三人率馬步兵二十三萬餘人先後來降。丁酉,以郝晉為保定巡撫。免平度、壽光等六州縣元年額賦。戊戌,命滿洲子弟就學,十日一赴監考課,春秋五日一演射。故明中書張朝聘輸木千章助建宮殿,自請議敘。諭以用官惟賢,無因輸納授官之理,令所司給直。庚子,免章丘、濟陽京班匠價,並令直省除匠籍為民。甲辰,定叔父攝政王儀注,凡文移皆曰皇叔父攝政王。乙巳,免皇后租,並崇文門米麥稅。庚戌,宣平定江南捷音。乾清宮成,復建太和殿、中和殿、位育宮。

六月癸丑,免興濟縣元年額賦。甲寅,免近畿圈地今年額賦三之二。乙卯,以丁文盛為山東巡撫。丙辰,諭南中文武軍民薙發,不從者治以軍法。是月,始諭直省限旬日薙發如律令。辛酉,豫親王多鐸遣軍追故明福王硃由崧於蕪湖。明靖國公黃得功逆戰,圖賴大敗之,得功中流矢死。總兵官田雄、馬得功執福王及其妃來獻,諸將皆降。免永寧等四縣元年荒賦。丙寅,申薙發之令。免深、衡水等七州縣元年荒賦。丁卯,陜西妖賊胡守龍倡亂,孟喬芳討平之。戊辰,皇太妃薨。辛未,何洛會率師駐防西安。命江南於十月行鄉試。己卯,詔曰:「本朝立國東陲,歷有年所,幅員既廣,無意並兼。昔之疆場用兵,本冀言歸和好。不幸寇兇極禍,明祚永終,用是整旅入關,代明雪憤。猶以賊渠未殄,不遑啟居,爰命二王,誓師西討。而南中乘釁立君,妄竊尊號,亟行亂政,重虐人民。朕夙夜祗懼,思拯窮黎,西賊既摧,乃事南伐。兵無血刃,循汴抵淮。甫克維揚,遂平江左。金陵士女,昭我天休。既俘福籓,南服略定,特弘大賚,嘉與維新。其河南、江北、江南官民絓誤,咸赦除之。所有橫徵逋賦,悉與蠲免。大軍所過,免今年額賦之半,餘免三之一。」

閏六月甲申,阿濟格敗李自成於鄧州,窮追至九江,凡十三戰,皆大敗之。自成竄九宮山,自縊死,賊黨悉平。故明寧南侯左良玉子夢庚、總督袁繼咸等率馬步兵十三萬、船四萬自東流來降。丙戌,定群臣公以下及生員耆老頂戴品式。己丑,河決王家園。庚寅,詔阿濟格等班師。辛卯,改江南民解漕、白二糧官兌官解。壬辰,諭曰:「明季臺諫諸臣,竊名貪利,樹黨相攻,眩惑主心,馴致喪亂。今天下初定,百事更始,諸臣宜公忠體國,各盡職業,毋蹈前轍,自貽顛越。」定滿洲文武官品級。癸巳,命大學士洪承疇招撫江南各省。甲午,定諸王、貝勒、貝子、宗室公頂戴式。乙未,除割腳筋刑。癸卯,命吳惟華招撫廣東,孫之獬招撫江西,黃熙允招撫福建,江禹緒招撫湖廣,丁之龍招撫雲、貴。多鐸遣貝勒博洛及拜尹圖、阿山率師趣杭州,故明潞王出降,淮王自紹興來降。嘉興、湖州、嚴州、寧波諸郡悉平。分遣總兵官吳勝兆克廬州、和州。乙巳,改南京為江南省,應天府為江寧府。命陜西於十月行鄉試。

秋七月庚戌朔,享太廟。壬子,命貝勒勒克德渾為平南大將軍,同固山額真葉臣等往江南代多鐸。設明太祖陵守陵太監四人,祀田二千畝。癸丑,故明東平侯劉澤清率所部降。乙卯,以劉應賓為安廬巡撫,土國寶為江寧巡撫。丙辰,命謝弘儀招撫廣西。戊午,禁中外軍民衣冠不遵國制。己未,以何鳴鑾為湖廣巡撫,高鬥光為偏沅巡撫,潘士良撫治鄖陽。甲子,上太祖武皇帝、孝慈武皇后、太宗文皇帝玉冊玉寶於太廟。乙丑,免西安、延安本年額賦之半,餘免三之一。戊辰,西平賊首劉洪起伏誅,汝寧州縣悉平。河決兗西新築月堤。己巳,詔自今內外章奏由通政司封進。丁丑,以陳錦提督操江,兼管巡撫。故明總漕田仰陷通州、如皋、海門,鳳陽巡撫趙福星、梅勒章京譚布等討平之。己卯,以楊聲遠為登萊巡撫。

八月辛巳,免霸、順義等八州縣災賦。乙酉,免彰德、衛輝、懷慶、河南各府荒賦。己丑,英親王阿濟格師還,賜從征外籓王、臺吉、將佐金帛有差。癸巳,免真定、順德、廣平、大名災額賦。丙午,降將金聲桓討故明益王,獲其從官王養正等誅之,並獲鍾祥王硃孳菪等九人。丁未,以英親王阿濟格出師有罪,降郡王,譚泰削公爵,降昂邦章京,鰲拜等議罰有差。

九月庚戌,故明魯王將方國安、王之仁犯杭州,張存仁擊走之。癸丑,命鎮國公傅勒赫、輔國公札喀納等率師協防江西。丁巳,故明懷安王來降。辛酉,故明新昌王據雲臺山,攻陷興化,準塔討斬之。甲子,以河間、灤州、遵化荒地給八旗耕種,故明勛戚內監餘地並分給之。庚午,田仰寇福山,土國寶擊敗之。丁丑,江西南昌十一府平。

冬十月癸未,以馬國柱為宣大總督。戊子,故明翰林金聲受唐王敕起兵於徽州,眾十餘萬。洪承疇遣提督張天祿連破之於績溪,獲金聲,不屈,殺之。是時,故明唐王硃聿釗據福建,魯王硃彞垓據浙江,馬士英等兵渡錢塘結營拒命。庚寅,免寶坻縣荒賦。壬辰,免太原等府州災賦。癸巳,豫親王多鐸師還,上幸南苑迎勞之。丙申,以苗胙土為南贛巡撫。乙巳,以太宗次女固倫公主下嫁察哈爾汗子阿布鼐。丙午,以申朝紀為山西巡撫,李翔鳳為江西巡撫,蕭起元為浙江巡撫。戊申,加封和碩豫親王多鐸為和碩德豫親王,賜從征王、貝勒、貝子、公及外籓臺吉、章京金幣有差。命孔有德、耿仲明還盛京。

十一月壬子,以張存仁為浙閩總督,羅繡錦為湖廣四川總督。癸丑,故明大學士王應熊、四川巡撫龍文光請降。甲寅,以吳景道為河南巡撫,命巴山、康喀賴為左右翼,同洪承疇駐防江寧,硃瑪喇駐防杭州,貝勒勒克德渾率鞏阿岱、葉臣討湖廣流賊二隻虎等。己未,朝鮮國王李倧請立次子淏為世子,許之。丁卯,硃瑪喇敗馬士英於餘杭,和託敗方國安於富陽。士英、國安復窺杭州,梅勒章京濟席哈等擊走之。戊辰,以何洛會為定西大將軍,遣巴顏、李國翰帥師會之,討四川流賊張獻忠。戊寅,以陳之龍為鳳陽巡撫。

十二月己卯朔,日有食之。乙酉,故明閣部黃道周寇徽州,洪承疇遣張天祿擊敗之。故明總兵高進忠率所部自崇明來降。癸巳,佟養和、金聲桓進討福建,分兵攻南贛,敗故明永寧王、羅川王、閣部黃道周等數十萬眾。丙午,更定朝儀,始罷內監朝參。丁未,硃瑪喇等敗方國安、馬士英於浙東。固原賊武大定作亂,總兵官何世元等死之。

是歲,朝鮮,歸化城土默特部章京古祿格,鄂爾多斯部喇嘛塔爾尼齊,烏硃穆秦部車臣親王,席北部額爾格訥,喀爾喀部土謝圖汗、古倫迪瓦胡土克圖喇嘛、石勒圖胡土克圖、嘛哈撒馬諦塞臣汗,厄魯特部顧實汗子多爾濟達賴巴圖魯臺吉及回回國,天方國俱來貢。朝鮮四至。

三年春正月戊午,貝勒勒克德渾遣將敗流賊於臨湘,進克岳州。辛酉,固山額真阿山、譚泰有罪,阿山免職,下譚泰於獄。流賊賀珍、孫守法、胡向化犯西安,何洛會等擊敗之。金聲桓遣將攻故明永寧王於撫州,獲之,並獲其子硃孳榮等,遂平建昌。丙寅,故明潞安王、瑞昌王率眾犯江寧,侍郎巴山等擊敗之。戊辰,以宋權為國史院大學士。己巳,以肅親王豪格為靖遠大將軍,暨多羅衍禧郡王羅洛宏、貝勒尼堪、貝子屯齊喀、滿達海等帥師征四川。故明唐王硃聿釗兵犯徽州,洪承疇遣張天祿等擊敗之,獲其閣部黃道周殺之,進克開化。

二月己卯,貝勒勒克德渾破流賊於荊州,奉國將軍巴布泰等追至襄陽,斬獲殆盡。大軍進次夷陵,李自成弟李孜等以其眾來降。辛巳,免密雲荒賦。甲申,罷江南舊設部院,差在京戶、兵、工三部滿、漢侍郎各一人駐江寧,分理部務。乙酉,明魯王將劉福援撫州,梅勒章京屯泰擊敗之。何洛會遣將破流賊劉文炳於蒲城,賊渠賀珍奔武功。戊子,以柳寅東為順天巡撫。命肅親王豪格分兵赴南陽,討流賊二隻虎、郝如海等。丙申,遣侍郎巴山、梅勒章京張大猷率師鎮守江寧,甲喇章京傅誇蟾、梅勒章京李思忠率師鎮守西安。潛山、太湖賊首石應璉擁故明樊山王硃常闒為亂,洪承疇遣將擊斬之。丙午,命貝勒博洛為征南大將軍,同圖賴率師征福建、浙江。

三月辛亥,譯洪武寶訓成,頒行中外。乙卯,免近京居民田宅圈給旗人別行撥補者租賦一年。丁巳,何洛會敗賊劉體純於山陽。己未,以王來用總督山、陜、四川糧餉,馬鳴佩總督江南諸省糧儲。乙丑,賜傅以漸等進士及第出身有差。己巳,何洛會擊賊二隻虎於商州,大敗之。昌平民王科等盜發明帝陵,伏誅。壬申,多羅饒餘郡王阿巴泰薨。癸酉,封烏硃穆秦部塞冷、蒿齊忒部薄羅特為貝勒,阿霸垓部多爾濟為貝子。豪格師抵西安,遣工部尚書興能敗賊於邠州,固山額真杜雷敗賊於慶陽。故明大學士張四知自江南來降。

夏四月己卯,詔貝勒勒克德渾班師,孔有德、耿仲明、尚可喜、沈志祥各統所部來京。甲申,免錢塘、仁和間架稅。乙酉,命今年八月再行鄉試,明年二月再行會試。丁亥,免睢州、祥符等四州縣災賦。戊子,除貫耳穿鼻之刑。癸巳,除明季加征太平府姑溪橋米稅、金柱山商稅、安慶府鹽稅。乙未,免靜海、興濟、青縣荒賦。丙申,江西浮梁、餘干賊合閩賊犯饒州,副將鄧雲龍等擊敗之。戊戌,攝政王多爾袞諭停諸王大臣啟本。己亥,以張尚為寧夏巡撫。罷織造太監。辛丑,諭曰:「比者蠲除明季橫徵苛稅,與民休息。而貪墨之吏,惡其害己而去其籍,是使朝廷德意不下究,而明季弊政不終釐也。茲命大臣嚴加察核,並飭所司詳定賦役全書,頒行天下。」諭汰府縣冗員。甲辰,修盛京孔子廟。

五月丁未,蘇尼特部騰機思、騰機特、吳班代、多爾濟思喀布、蟒悟思、額爾密克、石達等各率所部叛奔喀爾喀部碩雷。命德豫親王多鐸為揚威大將軍,同承澤郡王碩塞等率師會外籓蒙古兵討之。四子部溫卜、達爾漢卓禮克圖、多克新等追斬吳班代等五臺吉。庚戌,申隱匿逃人律。戊午,金聲桓克南贛,獲其帥劉廣胤。辛酉,豪格遣巴顏、李國翰敗賊於延安。壬戌,故明魯王、荊王、衡王世子等十一人謀亂,伏誅。癸亥,以葉克書為昂邦章京,鎮守盛京。豪格遣貝勒尼堪等敗賊賀珍於雞頭關,遂克漢中,珍走西鄉。乙丑,貝勒博洛遣圖賴等擊敗故明魯王將方國安於錢塘。魯王硃彞垓遁保臺州。庚午,官軍至漢陰,流賊二隻虎奔四川,孫守法奔岳科寨。巴顏、李國翰追延安賊至張果老崖敗之。辛未,免沛、蕭二縣元、二年荒賦之半。

六月戊寅,免懷柔縣荒賦。丙戌,禁白蓮、大成、混元、無為等教。壬辰,以高士俊為湖廣巡撫。乙未,張存仁遣將擒故明大學士馬士英及長興伯吳日生等斬之。

秋七月甲寅,貝勒勒克德渾師還。丁巳,多鐸破騰機思等於毆特克山,斬其臺吉毛害,渡土喇河擊斬騰機思子多爾濟等,盡獲其家口輜重。又敗喀爾喀部土謝圖汗二子於查濟布喇克上游。戊午,碩雷子陣查濟布喇克道口,貝子博和託等復大敗之。碩雷以餘眾走塞冷格。庚申,李國翰、圖賴等拔張果老崖。壬戌,江西巡撫李翔鳳進正一真人符四十幅。諭曰:「致福之道,在敬天勤民,安所事此,其置之。」戊辰,豪格遣貝子滿達海、輔國公哈爾楚渾、固山額真準塔趨徽州、階州分討流賊武大定、高如礪、蔣登雷、石國璽、王可臣等,破之。如礪遁,登雷、國璽、可臣俱降。

八月丙子,多羅衍禧郡王羅洛宏薨於軍。丁丑,豪格遣纛章京哈寧阿攻武大定於三臺山,拔之。丁亥,博洛克金華、衢州,殺故明蜀王硃盛濃、樂安王硃誼石及其將吳凱、項鳴斯等,其大學士謝三賓、閣部宋之普、兵部尚書阮大鋮、刑部尚書蘇壯等降。浙江平。戊子,以孔有德為平南大將軍,同耿仲明、沈志祥、金礪、佟代率師征湖廣、廣東、廣西。免太湖、潛山二年及今年荒賦。癸巳,命尚可喜率師從孔有德南討。

九月己酉,故明瑞昌王硃誼氻謀攻江寧,官兵討斬之。甲子,免夷陵、石首等十三州縣荒賦十之七,荊門、江陵等四州縣十之五,興國、廣濟等十六州縣十之三。丙寅,故明崇陽王攻歙縣,副將張成功等敗之。丁卯,故明督師何騰蛟等攻岳州,官軍擊敗之。

冬十月丙子,鄭四維等克夷陵、枝江、宜都,改湖廣承天府為安陸府。己卯,和碩德豫親王多鐸師還,上郊勞之。辛巳,金聲桓遣將擒故明王硃常洊及其黨了悟等,誅之。甲申,以胡全才為寧夏巡撫,章於天為江西巡撫。金聲桓遣將克贛州,獲故明閣部楊廷麟殺之。癸巳,以李棲鳳為安徽巡撫。丁酉,免懷寧等四縣災賦。己亥,免延綏、莊浪災賦。壬寅,太和宮、中和宮成。

十一月癸卯朔,貝勒博洛自浙江分軍進取福建,圖賴等敗故明閣部黃鳴駿於仙霞關,遂克浦城、建寧、延平。故明唐王硃聿釗走汀州,阿濟格尼堪等追斬之,遂定汀州、漳州、泉州、興化,進克福州,悉降其眾。福建平。癸丑,免河間、任丘及大同災賦。丁巳,祀天於圜丘。己巳,豪格師至南部,時張獻忠列寨西充,鰲拜等兼程進擊,大破之,斬獻忠於陣,復分兵擊餘賊,破一百三十餘營。四川平。

十二月癸酉朔,故明遂平王硃紹鯤及其黨楊權等擁兵太湖,結海寇為亂,副將詹世勛等討斬之。庚戌,山東賊謝遷攻陷高苑,總兵官海時行討平之。壬午,故明高安王硃常淇及其黨江於東等起兵婺源,張天祿討平之。丙戌,以於清廉為保定巡撫,劉武元為南贛巡撫,免薊、豐潤等五州縣災賦。甲午,位育宮成。庚子,明金華王硃由揾起兵饒州,官軍擊斬之。

是歲,朝鮮,蒙古及歸化城土默特部古祿格,厄魯特部多爾濟達來巴圖魯、顧實汗,喀爾喀部買達里胡土克圖、額爾德尼哈談巴圖魯、戴青哈談巴圖魯、青臺吉,科爾沁部多羅冰圖郡王塞冷,蒿齊忒部多羅貝勒額爾德尼,索倫部、使鹿部喇巴奇,鄂爾多斯部濟農臺吉查木蘇,庫爾喀部賴達庫及達賴喇嘛,吐魯番俱來貢。朝鮮、厄魯特顧實汗、達賴喇嘛皆再至。

四年春正月戊申,輔國公鞏阿岱、內大臣吳拜等徵宣府。壬子,命副都統董阿賴率師駐防杭州。興國州賊柯抱沖結故明總督何騰蛟攻陷興國。總兵官柯永盛遣將擒抱沖及其黨陳珩玉斬之。乙卯,以楊聲遠為淮揚總督,黃爾性為陜西巡撫。辛酉,以硃國柱為登萊巡撫。壬戌,陜西官軍擊延慶賊郭君鎮、終南賊孫守法,敗之。洪承疇遣將擊賊帥趙正,大破之。

二月癸酉,以張儒秀為山東巡撫。乙亥,佟養甲平梧州。丁丑,副將王平等擊賀珍、劉二虎賊黨於興安,敗之。癸未,詔曰:「朕平定中原,惟浙東、全閩尚阻聲教,百姓辛苦墊隘,無所控訴,爰命征南大將軍貝勒博洛振旅而前。既定浙東,遂取閩越。先聲所至,窮寇潛逋。大軍掩追,及於汀水。聿釗授首,列郡悉平。顧惟僭號阻兵,其民何罪,用昭大賚,嘉與維新。一切官民罪犯,咸赦除之。橫徵逋賦,概予豁免。山林隱逸,各以名聞錄用。民年七十以上,給絹米有差。」己丑,洪承疇擒故明瑞昌王硃議貴及湖賊趙正,斬之。乙未,硃聿釗弟聿★僭號紹武,據廣州,佟養甲、李成棟率師討之,斬聿鐭及周王肅讋、益王思弮、遼王術雅、鄧王器、鉅野王壽圠、通山王蘊越、高密王弘椅、仁化王慈魶、鄢陵王肅汭、南安王企壟等。廣州平。戊戌,以佟國鼏為福建巡撫。

三月戊午,賜呂宮等進士及第出身有差。己未,以耿焞為順天巡撫,周伯達為江寧巡撫,趙兆麟撫治鄖陽。庚申,諭京官三品以上及督、撫、提、鎮各送一子入朝侍衛,察才任使,無子者以弟及從子代之。壬戌,免崇明縣鹽課、馬役銀。乙丑,大清律成。丙寅,佟養甲克高、雷、廉三府。丁卯,命祀郊社太牢仍用腥。己巳,禁漢人投充滿洲。庚午,罷圈撥民間田宅,已圈者補給。

夏四月丁丑,田仰率所部降。己卯,高士俊克長沙,昂邦章京傅喀蟾討劉文炳、郭君鎮,殲之。乙酉,貝勒博洛班師。是役也,貝子和託、固山額真公圖賴皆卒於軍。甲午,陜西官軍斬孫守法。

五月壬寅,舟山海賊沈廷揚等犯崇明,官軍討擒之。己酉,故明在籍通政使侯峒曾遣諜致書魯王,偽許洪承疇、土國寶以公、侯,共定江南,為反間計,柘林游擊獲之以聞。上覺其詐,命江寧昂邦章京巴山等同承疇窮治其事。庚戌,免興國、江夏等十州縣上年災賦。癸丑,以佟養甲為兩廣總督,兼廣東巡撫。辛酉,投誠伯常應俊、總兵李際遇等坐通賊,伏誅。癸亥,上幸南苑。乙丑,班代、峨齊爾、胡巴津自蘇尼特來降。

六月壬申,免成安等七縣上年災賦。丙子,朝鮮國王李倧遣其子橑來朝。庚辰,故明趙王硃由棪來降。戊子,免綏德衛上年災賦。己丑,封貝勒博洛為多羅郡王。癸巳,陜西賊武大定陷紫陽,總兵官任珍擊敗之。湖廣官軍克衡州、常德及安化、新化等縣。甲午,蘇松提督吳勝兆謀叛,伏誅。丁酉,免山東上年荒賦。

秋七月辛丑,加封和碩德豫親王多鐸為輔政叔德豫親王。癸卯,建射殿於左翼門外。甲辰,免徐州上年荒賦。己酉,封敖漢部額駙班第子墨爾根巴圖魯為多羅郡王。癸丑,以申朝紀為宣大總督。丁巳,鄖陽賊王光代用永歷年號,聚眾作亂,命侍郎喀喀木等剿之。戊午,改馬國柱為江南江西河南總督。甲子,詔曰:「中原底定,聲教遐敷。惟粵東尚為唐籓所阻,嶺海怨咨,已非一日。用移南伐之師,席卷惠、潮,遂達省會。念爾官民,初非後至,一切罪犯,咸赦除之。逋賦橫徵,概與豁免。民年七十以上,加錫粟帛。所在節孝者旌,山林有才德者錄用。南海諸國能鄉化者,待之如朝鮮。」丙寅,以祝世昌為山西巡撫。丁卯,上幸邊外閱武。是日,駐沙河。

八月庚午,金聲桓擒故明宗室麟伯王、靄伯王於瀘溪山,誅之。甲戌,次西巴爾臺。丙子,次海流土河口。壬午,次察漢諾爾。乙酉,豪格遣貝勒尼堪等先後克遵義、夔州、茂州、內江、榮昌、富順等縣,斬故明王及其黨千餘人。四川平。丙戌,次胡蘇臺。辛卯,以張文衡為甘肅巡撫。丙申,上還宮。

九月辛丑,京師地震。辛亥,淮安賊張華山等用隆武年號,嘯聚廟灣。丁巳,以李猶龍為天津巡撫。辛酉,官軍討廟灣賊,破之。

冬十月庚午,以王懩為安徽巡撫。壬申,喀喇沁部卓爾弼等率所部來降。癸未,以吳惟華為淮揚總督,線縉為偏沅巡撫。戊子,定直省官三年大計。壬辰,以廣東採珠病民,罷之。

十一月庚戌,以陳泰為靖南將軍,同梅勒章京董阿賴征福建餘寇。辛亥,免山西代、靜樂等十四州縣,寧化等六所堡,山東德、歷城等十五州縣災賦。裁山東明季牙、雜二稅。戊午,五鳳樓成。癸亥,祀天於圜丘。

十二月戊辰,免保定、河間、真定、順德災賦。壬申,以陳錦為閩浙總督。己卯,以太宗十一女固倫公主下嫁喀爾嗎索納木。甲申,蘇尼特部臺吉吳巴什等來歸。丙戌,大軍自岳州收長沙,故明總督何騰蛟等先期遁。次湘潭,敗桂王將黃朝選眾十三萬於燕子窩,又敗之於衡州,斬之,遂克寶慶,斬魯王硃鼎兆等。進擊武岡,桂王由榔走,追至靖州,下其城。復克沅州,岷王硃埏峻以黎平降。湖南平。庚寅,故明將鄭彩犯福州,副將鄒必科等敗走之。

是歲,科爾沁、喀喇沁、烏硃穆秦、敖漢、翁牛特、蘇尼特、札魯特、郭爾羅斯、蒿齊忒、阿霸垓諸部來朝。朝鮮暨喀爾喀部札薩克圖汗、墨爾根綽爾濟、額爾德尼綽爾濟、邁達禮胡土克圖、額爾德尼顧錫、伊拉古克三胡土克圖、嘛哈撒馬諦塞臣汗、俄木布額爾德尼、塞勒胡土克圖、滿硃習禮胡土克圖,札薩克圖汗下俄木布額爾德尼、巴顏護衛、舍晉班第、邁達禮胡土克圖,諾門汗下丹津胡土克圖,土謝圖汗下澤卜尊丹巴胡土克圖,碩雷汗下伊赫額木齊格隆、額參德勒哈談巴圖魯,厄魯特部臺吉吳霸錫、顧實汗,羅布藏胡土克圖下巴漢格隆、盆蘇克扎穆蘇,阿布賚諾顏下訥門汗、巴圖魯諾顏、達雲綽爾濟、鄂濟爾圖臺吉,蘇尼特部臺吉魏正,札魯特部臺吉桑圖,鄂爾多斯部濟農,歸化城土默特部章京托博克、諾爾布,唐古忒部及喇布札木綽爾濟、喇嘛班第達等俱來貢。

五年春正月辛亥,故明宜春王硃議衍據汀州為亂,總兵官於永綬擒斬之。癸丑,免太原、平陽、潞安三府,澤、沁、遼三州災賦。癸亥,和碩肅親王豪格師還。衍禧郡王羅洛宏卒於軍,至是喪歸,輟朝二日。

二月甲戌,金聲桓及王得仁以南昌叛。辛巳,江南官軍復無為州,福建官軍復連城、順昌、將樂等縣。癸未,免濟南、兗州、青州、萊州上年災賦。辛卯,以固倫公主下嫁巴林部塞卜騰。壬辰,以呂逢春為山東巡撫,李鑒為寧夏巡撫。故明貴溪王硃常彪、恢武伯向登位寇沅州,纛章京線國安等討斬之。

三月己亥,貝子吞齊、尚善等訐告和碩鄭親王濟爾哈朗,罪連莽加、博博爾岱、鰲拜、索尼等,降濟爾哈朗為多羅郡王,莽加等降革有差。辛丑,和碩肅親王豪格有罪,論死。上不忍置之法,幽系之。庚戌,命譚泰為征南大將軍,同何洛會討金聲桓。辛酉,以耿焞為宣大山西總督。甲子,武大定犯寧羌,游擊張德俊等大破之。

四月丁卯,以楊興國為順天巡撫。戊辰,免渭原、金縣、蘭州衛災賦。壬申,官軍復建寧,斬故明鄖西王硃常湖等。己卯,封科爾沁杜爾伯特鎮國公色冷為貝子。庚辰,遣固山額真阿賴等駐防漢中。壬午,大軍克辰州,遂破永寧,至全州,故明督師何騰蛟遁,獲貴溪王硃長標、南威王硃寅衛、長沙王硃由櫛等。銅仁、興安、關陽諸苗、瑤來降。丙戌,命劉之源、佟圖賴為定南將軍,駐防寶慶,李國翰為定西將軍,駐防漢中。丁亥,吳三桂自錦州移鎮漢中。

閏四月戊戌,復濟爾哈朗爵為和碩鄭親王。癸卯,以李國英為四川巡撫。己未,以遲日益為湖廣巡撫。癸亥,命貝子吞齊為平西大將軍,同韓岱討陜西叛回。

五月己丑朔,日有食之。戊辰,官軍破叛回於鞏昌,復臨洮、蘭州。辛未,游擊張勇破叛回於馬家坪,獲故明延長王硃識駉,斬之。壬午,以趙福星為鳳陽巡撫。癸未,以硃延慶為江西巡撫。甲申,官軍破金聲桓,復九江、饒州。己丑,以劉弘遇為安徽巡撫。

六月甲午朔,免西安、延安、平涼、臨洮、慶陽、漢中上年災賦。癸卯,以周文業為甘肅巡撫。甲辰,額塞等大破叛回於蘭州,餘黨悉平。丙辰,京師地震有聲。癸亥,太廟成。

秋七月丁丑,初設六部漢尚書、都察院左都御史,以陳名夏、謝啟光、李若琳、劉餘祐、黨崇雅、金之俊為六部尚書,徐起元為左都御史。

八月癸巳朔,金聲桓、王得仁寇贛州,官軍擊走之。己亥,陳泰、李率泰等敗鄭彩於長樂,又敗之於連江,復興化。己巳,命和碩英親王阿濟格、多羅承澤郡王碩塞等討天津土賊。丁未,禁民間養馬及收藏軍器。己酉,以王一品為鳳陽巡撫。壬子,令滿、漢官民得相嫁娶。乙卯,以夏玉為天津巡撫,張學聖為福建巡撫。

九月壬戌朔,官軍獲故明巡撫吳江等於南康湖口,斬之。甲子,和碩英親王阿濟格討曹縣土賊,平之。己巳,封貝勒勒克德渾為多羅順承郡王,博洛為多羅端重郡王。壬申,和碩鄭親王濟爾哈朗為定遠大將軍,討湖廣賊李錦。丁丑,封貝勒尼堪為多羅敬謹郡王。

冬十月壬寅,和碩禮親王代善薨。甲辰,佟圖賴復寶慶。丙辰,降將劉澤清結曹縣賊叛,澤清及其黨李洪基等俱伏誅。

十一月甲子,廣東叛將李成棟據南雄,結★蠻犯贛州,巡撫劉武元等擊走之。丙寅,總兵官任珍擊賀珍,破之。戊辰,祀天於圜丘,以太祖武皇帝配。追尊太祖以上四世:高祖澤王為肇祖原皇帝,曾祖慶王為興祖直皇帝,祖昌王為景祖翼皇帝,考福王為顯祖宣皇帝;妣皆為皇后。上詣太廟上冊寶。辛未,以配天及上尊號禮成,御殿受賀,大赦。辛未,和碩英親王阿濟格、多羅端重郡王博洛、多羅承澤郡王碩塞等帥師駐大同,備喀爾喀。

十二月辛卯朔,命郡王瓦克達,貝子尚善、吞齊等詣阿濟格軍。調八旗游牧蒙古官軍之半,戍阿爾齊土蘇門哈達。癸巳,姜瓖以大同叛,總督耿焞走陽和。丙申,免平山、隆平、清豐災賦。戊戌,阿濟格圍大同。辛丑,復遣梅勒章京阿喇善、侍郎噶達渾詣阿濟格軍。癸卯,免大同災賦。壬子,楊捷等復都昌,獲故明兵部尚書餘應桂,斬之。丁巳,以佟養量為宣大總督。

是歲,蘇尼特、扎魯特等部來朝。朝鮮,喀爾喀部俄木布額爾德尼、戴青訥門汗喇嘛、塞爾濟額爾德尼魏正、碩雷汗、邁達理胡土克圖、扎薩克圖汗下額爾德尼哈談巴圖魯,厄魯特部顧實汗、錫勒圖綽爾濟、諾門汗,索倫部阿濟布,鄂爾多斯部單達,蘇尼特部騰機忒,科爾沁貝勒張繼倫,歸化城固倫第瓦胡土克圖、丹津喇嘛額爾德尼寨桑,土默特部古祿格,烏思藏闡化王王舒克,湯古特達賴喇嘛俱來貢。朝鮮、厄魯特顧實汗、湯古特達賴喇嘛再至。

六年春正月壬戌,官軍復羅源、永春、德化等縣。癸亥,命多羅敬謹郡王尼堪等征太原。戊辰,諭曰:「朕欲天下臣民共登衽席,日夕圖維,罔敢怠忽。往年流寇作亂,慘禍已極,入關討賊,士庶歸心。乃邇年不軌之徒,捏作洗民訛言。小民無知輕信,惶惑逃散,作亂者往往而有。朕聞不嗜殺人,能一天下。書云:『眾非元後何戴,後非眾罔與守邦。』君殘其民,理所蔑有。自元年來,今六年矣,寧有無故而屠戮民者。民茍思之,疑且冰釋。至於自甘為賊,樂就死地,必有所迫以致此。豈督、撫、鎮、按不得其人,有司朘削,民難自存歟?將蠲免賦稅,有名無實歟?內外各官其確議興利除弊之策,朕次第酌行之。」辛未,姜瓖黨姚舉等殺冀寧道王昌齡,陷忻州,固山額真阿賴破走之。乙亥,諭曰:「設關征稅,原以譏察奸宄,非與商賈較輜銖也。其各以原額起稅,毋得橫徵以充私橐,違者罪之。」諭山西大同軍民,無為姜瓖脅誘,來歸者悉予矜免。戊寅,行保舉連坐之法。庚辰,諭言官論事不實者,廷臣集議,毋輒下刑部。辛巳,以金廷獻為偏沅巡撫。壬午,譚泰、何洛會復南昌,金聲桓投水死,王得仁伏誅,九江、南康、瑞州、臨江、袁州悉平。癸未,山西賊黨劉遷寇代州,阿濟洛遣軍破走之。

二月癸卯,攝政王多爾袞征大同。免直隸省六年以前荒賦、四川商民鹽課。辛亥,故明宗室硃森釜等犯階州,吳三桂擊斬之。

三月癸亥,多爾袞拔渾源州。丙寅,漢羌總兵官張天福平賊渠覃一涵,獲故明山陰王等斬之。丁卯,土賊王永強陷延安、榆林等十九州縣,延綏巡撫王正志等死之。己巳,應州、山陰降,多爾袞旋師,留阿濟格於大同。辛未,進封多羅承澤郡王碩塞、多羅端重郡王博洛、多羅敬謹郡王尼堪為親王。王永強陷同官。壬申,廣信府知府楊國楨等復玉山縣。寧夏官軍克臨河等堡。乙亥,甘、涼逆回米喇印、丁國棟復作亂,甘肅巡撫張文衡等死之。丁丑,輔政和碩德豫親王多鐸薨,攝政王多爾袞師次居庸,還京臨喪。甲申,減隱匿逃人律。譚泰、何洛會破賊於南康,進克信豐,叛將李成棟走死,復撫州、建昌。江西平。丙戌,博洛遣鰲拜等大破姜瓖於大同北山。吳三桂擊敗王永強,復宜君、同官。

夏四月庚寅,遣羅碩、卦喇駐防太原。癸巳,阿濟格復左衛。乙未,命貝子吳達海等代征大同。丙申,吳三桂克蒲縣。癸卯,福建官軍復平和、詔安、漳平、寧洋。甲辰,賜劉子壯等進士及第出身有差。乙巳,皇太后崩。壬子,諭曰:「兵興以來,地荒民逃,流離無告。其令所在有司廣加招徠,給以荒田,永為口業,六年之後,方議徵租。各州縣以招民勸耕之多寡、道府以責成催督之勤惰為殿最。歲終,撫按考核以聞。」癸丑,以董宗聖為延綏巡撫。官軍克福寧,福建平。乙卯,賊黨陷汾州,命和碩端重親王博洛為定西大將軍,帥師討之。和碩敬謹親王尼堪移師大同。丁巳,封貝子滿達海為和碩親王。

五月辛酉,遣屠賴率師赴太原軍。丙子,以李棲鳳為廣東巡撫,郭肇基為廣西巡撫。免太原、平陽、汾州三府,遼、澤二州災賦。丁丑,改封孔有德為定南王,耿仲明為靖南王,尚可喜為平南王。命孔有德徵廣西,耿仲明、尚可喜徵廣東,各挈家駐防。裁直隸、江南、山東、浙江、陜西同知十,直隸、江南、河南、湖廣、江西、浙江通判二十一。免寶坻、順義五年災賦。辛巳,吳三桂、李國翰復延安。壬午,四川邊郡平。乙酉,和碩端重親王博洛復清源、交城、文水、徐溝、祁等縣。

六月庚子,朝鮮國王李倧薨。壬子,免滄州、清苑六年以前荒賦。癸丑,封張應京為正一嗣教大真人。乙卯,免江西四年、五年逋賦。

秋七月戊午朔,攝政王多爾袞復徵大同。乙丑,滿達海、瓦克達征朔州、寧武。丁卯,免開封等府災賦。辛未,多爾袞至阿魯席巴爾臺,校獵而還。遣纛章京索洪等益滿達海軍。癸酉,官軍平黃州賊三百餘砦,斬故明王硃蘊肏等。甲申,廣東餘寇犯南贛,官軍擊卻之。丙戌,吳三桂、李國翰復延綏鎮城。

八月癸巳,攝政王多爾袞還京。山西賊黨陷蒲州及臨晉、河津,孟喬芳討平之。甲午,免真定、順德、廣平、大名災賦。滿達海復朔州、馬邑。丁酉,端重親王博洛拔孝義。丙午,鄭親王濟爾哈朗等克湘潭,獲何騰蛟,不屈,殺之。辰州、寶慶、靖州、衡州悉平。進克全州。丁未,封朝鮮世子李淏為朝鮮國王。辛亥,以張孝仁為直隸山東河南總督。壬子,遣英親王阿濟格、貝子鞏阿岱等征大同。癸丑,梅勒章京根特等拔猗氏。乙卯,大同賊被圍久,饑死殆盡,偽總兵楊震威斬姜瓖及其弟琳來獻。丙辰,寧武關偽總兵劉偉等率眾降,靜樂、寧化山寨悉平。

九月戊午,封鄂穆布為多羅達爾漢卓禮克圖郡王,蘇尼特部噶爾麻為多羅貝勒。甲子,鄂爾多斯部額林臣、布達岱、顧祿、阿濟格札穆蘇等來降,封額林臣為多羅郡王,布達岱子伊廩臣、顧祿子色冷為固山貝子,阿濟格札穆蘇為鎮國公。丙寅,以夏玉為山東巡撫。癸酉,封固倫額駙祁他特為多羅郡王。甲戌,滿達海、博洛克汾州、平陽。

冬十月戊子,封多尼為和碩親王,傑書為多羅郡王。壬辰,京師地震。甲午,封勞親為親王。官軍復鄆城。戊戌,降將楊登州叛,陷山陰。己亥,免山東東平、長山等十八州縣五年災賦,江西六年以前明季遼餉。辛丑,攝政王多爾袞征喀爾喀部二楚虎爾。乙巳,陜西總兵官任珍擊故明將唐仲亨於屠油壩,斬之,並誅故明王硃常渶、硃由杠等。丙午,官軍復潞安。丁未,官軍克榆林。己酉,滿達海等拔沁、遼二州。庚戌,命滿達海還京,留瓦克達等定山西。

十一月丙寅,免直隸開、元城等縣徭賦,陜西岷州災賦。甲戌,多爾袞自喀吞布喇克旋師。免宣府災賦。壬午,耿仲明軍次吉安,畏罪自殺。

十二月乙酉朔,山西興、芮城、平陸三縣平。戊子,故明桂王將焦璉寇全州,勒克德渾等擊敗之,進克道州。努山等拔烏撒城。宜爾都齊等克黎平。己酉,官軍復鄰水、大竹二縣。庚戌,寧波、紹興、臺州土寇平。

是年,朝鮮、阿霸垓、烏硃穆秦、土默特諸部,厄魯特部阿巴賴諾顏、績克什虎巴圖魯臺吉、顧實汗子下達賴烏巴什溫布塔布囊,鄂爾多斯部郡王額林臣,喀爾喀部土謝圖汗、碩雷汗、戴青諾顏,歸化城土默特部古祿格等,伊喇古克三胡土克圖下戴青溫布達爾漢囊蘇及達賴喇嘛俱來貢。朝鮮、喀爾喀土謝圖汗再至。

七年春正月庚申,官軍復永寧、寧鄉。壬戌,官軍復南雄。癸酉,封鄂爾多斯部單達為貝勒,沙克查為貝子。甲戌,故明德化王硃慈業、石城王硃議陷大田,官軍討平之。丁丑,和碩鄭親王濟爾哈朗師還。

二月丁亥,上太后謚曰孝端正敬仁懿莊敏輔天協聖文皇后。甲午,以劉弘遇為山西巡撫,王一品為廣西巡撫。李建泰據太平叛,官軍圍之,出降,伏誅。平陽、潞安、澤州屬境俱平。

三月己未,日赤色如血。

夏四月甲午,孔有德擒故明將黃順、林國瑞於興寧,降其眾五萬。丙申,封科爾沁貝勒張繼倫為郡王。甲辰,多羅謙郡王瓦克達師還。

六月乙酉,保德州民崔耀等擒故明將牛化麟,斬之,以城降。癸卯,官軍復寧都、石城。

秋七月壬子朔,享太廟。乙卯,攝政王多爾袞議建邊城避暑,加派直隸、山西、浙江、山東、江南、河南、湖廣、江西、陜西九省錢糧二百五十萬兩有奇。辛酉,幸攝政王多爾袞第。多爾袞以貝子錫翰等擅請臨幸,下其罪,貝子錫翰降鎮國公,冷僧機、鰲拜等黜罰有差。壬戌,以馬之先為陜西巡撫。辛未,免西寧各堡寨五年災賦。

八月丁亥,降和碩端重親王博洛、和碩敬謹親王尼堪為多羅郡王。己丑,封巴林部塞卜騰、蒿齊忒部孛羅特為多羅郡王,科爾沁國顧穆、喀喇沁部古祿思喜布為多羅貝勒,改承澤親王碩塞、親王勞親為多羅郡王。

九月甲寅,故明將鄭成功寇潮州,總兵官王邦俊擊走之。丙子,免蘄、麻城等七州縣五、六兩年荒賦。

冬十月辛巳朔,日有食之。己亥,定陜西茶馬例。庚子,官軍克邵武,獲故明閣部揭重熙等,斬之。己酉,免桐城等六縣荒賦。

十一月甲寅,免甘肅去年災賦。乙卯,吳三桂復府谷,斬故明經略高友才等,餘眾降。壬戌,攝政王多爾袞有疾,獵於邊外。乙丑,尚可喜復廣州,餘眾降。戊寅,祀天於圜丘。

十二月戊子,攝政和碩睿親王多爾袞薨於喀喇城。壬辰,赴聞,上震悼,臣民為制服。丙申,喪至,上親奠於郊。己亥,詔曰:「太宗文皇帝升遐,諸王大臣籲戴攝政王。王固懷撝讓,扶立朕躬,平定中原,至德豐功,千古無二。不幸薨逝,朕心摧痛。中外喪儀,合依帝禮。」庚子,收故攝政王信符,貯內庫。甲辰,尊故攝政王為懋德修道廣業定功安民立政誠敬義皇帝,廟號成宗。乙巳,諭曰:「國家政務,悉以奏聞。朕年尚幼,闇於賢否,尚書缺員,其會推賢能以進。若諸細務,理政三王理之。」

是年,喀爾喀、厄魯特、烏斯藏諸部巴郎和羅齊、達爾汗囊素、盆挫堅挫等來朝。朝鮮,喀爾喀部碩雷汗、札薩克圖汗、土謝圖汗、綽克圖魏正諾顏、戴青諾顏、那穆齊魏正諾顏、察哈爾墨爾根臺吉、索那穆,厄魯特部巴圖魯貝勒、臺吉鄂齊爾圖、乾布胡土克圖、噶木布胡土克圖、舒虎兒戴青,烏斯藏部闡化王,索倫、使鹿諸部,歸化城土默特部古祿格俱來貢。朝鮮再至。


\end{pinyinscope}