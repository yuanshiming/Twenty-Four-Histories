\article{志一百三十二}

\begin{pinyinscope}
○邦交五

△德意志

德意志者,日耳曼列國總部名也,舊名邪馬尼,居歐洲中原,同盟三十六國,而中惟布路斯最強。

咸豐十一年,布路斯及德意志諸國請照英、法等國換約,江蘇巡撫薛煥不可。其使臣艾林波赴天津,呈三口通商大臣,請立條約。王大臣以聞,命總理各國事務、倉場總督崇綸充全權大臣,赴天津會崇厚酌辦。布使呈條約四十二款,附款一條,通商章程十款,另款一條,稅則一冊,其代呈德意志公會各國部名,均照布國條約辦理。既又稱,日耳曼通商諸國欲在臺灣之雞籠、浙江之溫州通商,並照各國駐京辦事。崇綸覆以日耳曼各國通商,均歸布路斯統轄約束,只辦通商,不得涉別事;並諭以京師非貿易之區,不能派員常駐;至雞籠、溫州二處,為英、法兩國條約所無,不能增益。時當四國換約,法使哥士耆言:「日耳曼各國,其最大者為布路斯,此外尚有邦晏等二十餘國,一切章程歸布國議定。」崇綸等以所言告總署,總署令哥士耆代阻之。忽有布國人入京,直入輔國將軍奕權宅強住。總理各國事務、戶部左侍郎文祥赴英館晤英使普魯斯,言:「布國既不以禮來,我國即不能以禮往。」並告以:「艾林波如或來京,亦當拒之,不得謂中國無禮也。」普魯斯請牒知艾林波,令迅速調回。未幾,布人相率回津,而艾林波牒總署,猶要求如故。遂定議以五年後許派秉權大臣一員駐京,兼辦各國事,餘與法國條約略同。是為德意志與中國立約之始。約既定,總署又恐五年後布國派員來京,仿照英、法國住居府第,復函屬崇綸等令其將不住府第一層載明約內。艾林波允遞牒聲明將來不住府第,由中國給一空閒地基,聽其自行修蓋,許之。艾林波隨來京詣總署謁見,未幾回津。

同治元年冬,布使列斐士牒辦理通商事務大臣薛煥、江蘇巡撫李鴻章,謂換約一事,德意志公會內,除本國外,尚有二十二國,曰拜晏,曰撤遜,曰漢諾威,曰威而顛白而額,曰巴敦,曰黑辛加習利,曰黑星達而未司大,曰布倫帥額,曰阿爾敦布爾額,曰魯生布而額,曰撤遜外抹艾生納,曰撤遜麥寧恩,曰撤遜阿里廷部而額,曰撤遜各部而額大,曰拏掃,曰宜得克比而孟地,曰安阿而得疊掃郭定,曰安阿而得比爾你布而額,曰立貝,曰實瓦字部而魯德司答,曰實瓦字部而孫德而士好遜,曰大支派之各洛以斯,曰小支派之各洛以斯,曰郎格缶而德,曰昂布而士,曰模令布而額水林,曰模令布而額錫特利子,曰律百克,曰伯磊門昂布爾。請將和約照錄二十二冊,鈐印分送各國,薛煥等不許。久之,始議會同互換和約,列舉德意志拜晏以下各國,不再分送。明年,列斐士復遣隨員韋根思敦來京,要求分送各國條約,鈐用江蘇籓司印,並請收各國國書,許之。

三年春三月,布國遣使臣李福斯來京,欲見總署王大臣呈遞國書。三口通商大臣崇厚以聞,並稱布國坐來兵船,在大沽攔江沙外扣留丹國商船三艘。總署以布使不應在中國洋面扣留敵船,詰之。李福斯接牒,即將丹船放回二艘,並遣譯官謝罪,總署始允會晤。

七年夏四月,布路斯君主維利恩復以李福斯為秉權大臣,來華呈遞國書。八年,咸伯國商人美利士私在臺灣大南澳境伐木墾荒,閩浙總督以聞。總署以美利士違約妄為,牒布使詰問,請其查辦。十年春,李福斯遞國書,言德意志各國共推戴布國君主為德意志國大皇帝,中國覆書致賀。是年李福斯回國,以領事安訥克為署使。十一年,安訥克以條約十年期滿,牒中國請換約,未果。李福斯復來,十二月,復遞國書。明年正月,穆宗親政,請覲見,許之。屆時李福斯因病回國,署使和立本特備文慶賀,因聲明將來本國使臣朝覲,應按此次所定節略辦理,許之。光緒元年九月,德國安訥船在福建洋面遭水賊殺斃船主、大夥,並毀其船,閩撫丁日昌當將犯拏獲斬梟,並追贓一萬三千餘元。德使責中國賠償,總署以德約三十三款明言不能賠償贓物,不許。

二年,德以巴蘭德為駐華公使。春三月,直隸總督李鴻章始遣游擊卞長勝等五弁,赴德武學院學習陸軍槍砲操法。巴蘭德牒總督,催請換約。十月,巴蘭德復牒總署索三事:一,洋商在租界內售賣洋貨,不再抽釐金;二,發給存票,不立期限,並準其以存票支取現銀;三,德商入內地採買土貨,準攜現銀。又請於年內開辦上海一口;又求在大孤山添開口岸,鄱陽湖拖帶輪船,吳淞口上下貨物三端。總署拒之,屢辯駮,不省。明年五月,遂偕繙譯官阿恩德出京。既抵天津,往晤李鴻章,鴻章曉以兩國意見即有不合,應往返商辦,力勸之,巴使乃回京。總署促與開議,忽言俟十月間再議。是年德使館定居東交民巷,仍納租價。四年,以光祿寺少卿劉錫鴻為出使德國大臣,並遞國書。劉錫鴻尋奏,聞德外務大臣促巴蘭德速立新約,而巴蘭德於吳淞起卸貨物、鄱陽拖帶輪船、內地租住店房三條仍力爭,至是竟回國。明年閏三月,巴使復來華議約,仍著重前三條。時德丕裏約夾板船至山東榮成縣所屬海面觸礁,巴使要求賠償,拒之。巴使又以天津紫竹林無德國租界,要求在法界以上另添租界,不許。是年閏五月,以候選道李鳳苞為出使德國大臣。

六年春二月,朝廷因德約議久未成,特派總理各國事務、協辦大學士、兵部尚書沈桂芬,戶部尚書景廉為全權大臣,復與巴使開議。久之,巴使始允將「大孤山、鄱陽湖及洋商入內地」刪去,並照英國新約辦法,彼此條款略相抵;惟江蘇吳淞口一處,允德船只暫停泊,上下客商貨物,章程仍由中國江海關道自訂。遂於二月二十一日畫押,並聲明二事:一,德國夾板在中國口岸停泊十四日以外者,則自第十五日起,即於應交正數船鈔減半,先行試辦;一,第六款內「德國允,德國人等」條內有「游歷」二字,德譯與華文不符,應將德文字意更正。遂約自畫押之日起,限一年內互換。已,巴使於六月三十日又來牒,稱德國國法,凡議立條約,必須先問國會,國會允許,方能批準;本國國會約在明年,所議光緒七年三月初二日互換約章一款,請將期限改為光緒七年十月初十日。七年秋七月,巴使請定期互換條約,政府命景廉與巴使在北京總署畫押互換,是為中德續約十款,並善後章程九條。

八年夏六月,德始與朝鮮議約,中國派員蒞盟,聲明為中國屬邦。九年冬十月,議結德魯麟洋行地畝案。初,廣東汕頭新開附地有海坪官地,中國欲填築作為商埠,忽有德魯麟洋行買辦華民郭繼宗謂系伊地,陰結德駐汕頭領事沙博哈,及德水師兵船,豎旗強占。中國聞之,牒向德使詰問,並命出使大臣李鳳苞與德外部辯論。時德相為畢士馬克,電致巴使,命速令師船退出,並撤領事任。已,德使歸咎中國地方官,屢請派員查辦,議久不決。至是,總署從李鴻章議,令赫德派洋員會同粵員議辦,遂辦結。

十年,贈德皇景泰窯器,答歷次派員監造鐵艦、撥借魚雷及兵船教習等事,修好也。十二年春二月,出使英國大臣曾紀澤將回華,德駐英公使伯爵哈子斐爾德遣參贊官伯爵美塔尼克來言,德皇暨德相畢斯馬克欲與晤談,邀臨其國,遂游各制造局廠。十四年秋七月,德皇薨,命出使大臣洪鈞吊唁,德命駐華公使巴蘭德致謝。

二十年夏四月,德人阿爾和欲在漢口建火油池。初,德商在上海創設火油池棧,許之。既又欲於漢口購地踵建,不許。德使爭辯,旋議將火油照巿價收買,及償造油制器各費,德使仍不從。明年,又請增開天津、漢口租界,許之。二十二年春正月,德外部馬沙爾求在中國借地泊船,出使大臣許景澄以告。時李鴻章使德將還,留稅務司德璀琳與德外部商辦加稅事,德廷謂須中國讓給兵船埠地始允加稅,德璀琳阻之,不省。

二十三年十月,山東曹州府鉅野縣有暴徒殺德教士二人,德以兵艦入膠州灣,逼守將章高元退出砲臺,占領之。德使海靖向總署要求六款:一,革巡撫李秉衡職,永不敘用;二,給天主堂建築費六萬六千兩,賠償盜竊物品銀三千兩;三,鉅野、菏澤、鄆城、單縣、曹縣、魚臺、武涉七處,各建教師住房,共給工費二萬四千兩;四,保以後永無此等事件;五,以兩國人資本設立德華公司,築造山東全省鐵道,並許開採鐵道附近之礦山;六,德國辦理此案費用,均由中國賠償。總署屢與折沖,始將第一款「永不敘用」四字刪去;二、三兩款全允;四、六兩款全削除;五款許以膠州灣至濟南府一段鐵道由德築造。議漸就緒,忽曹州有驅逐教師、殺害洋人之說,德使復要求租借膠州灣。二十四年二月,總署與德使海靖另訂專條三章。一章,膠州灣租界:一,灣內各島嶼及灣口與口外海面之群島,又灣東北岸自陰島東北角起劃一線東南行至勞山灣止,灣西南岸自齊伯山島對岸劃一線西南行至笛羅山島止,又灣內全水面以最高潮為標之地,皆為租借區域;二,租借區域,德國得行使主權、建築砲臺等事,但不得轉租與他國;中國軍艦商船來往,均照德國所定各國往來船舶章程一例待遇;三,租借期限以九十九年為期,如限內還中國,則德國在膠州灣所用款項由中國償還,另以相當地域讓與德國;四,自膠州灣水面潮平點起,周圍中里一百里之陸地為中立地,主權雖歸中國,然中國若備屯軍隊,須先得德國許可,但德國軍隊有自由通過之權。二章,鐵道礦務辦法:一,中國準德國在山東築造自膠州灣經濰縣、青州等處至濟南及山東界,又自膠州灣至沂州經萊蕪至濟南之二鐵道;二,鐵道附近左右各三十里中國里內之礦產,德商有開採之權。三章,山東全省開辦各項事務:一,以後山東省內開辦何項事務,或須外資,或須外料,或聘外人,德國有侭先承辦之權。是為中德膠澳租界條約。

二十四年,山東日照教案起,德人進兵據城,案結仍不退。又中國擬修天津至鎮江鐵路,德人阻之,並欲自修濟南至沂州一段,總署不許。又要求中國借德款,用德工程師。二十五年,山東高密民人阻德人修鐵路,山東巡撫袁世凱諭解之,因立鐵路章程,設華商德商膠濟鐵路公司,立交涉局,招股購地丈量建築。又立膠澳交涉章程十一款:一,兩國交涉案件,須兩國會辦;二,德人游歷,須發護照;三,兩國交涉事,統由交涉官商辦;四,青島租界內華洋案件,歸交涉官提訊審斷;五,租界內華人牽涉德人案件,須德官會同山東交涉官審問;六,德雇用華民之案,須由德官審訊;七,華人案件,仍由華審斷;八,租界外罪犯逃入青島華民及德人住處者,分別由華官、德官提拏解交;九,華、德人在租界內外行兇,華、德兵均可拿禁解交;十,華、德官商辦案件,須和衷;十一,重大案件,本省不能結者,由總署及駐京德使商辦。

又與德議立礦務章程,未定,二十六年五月,駐京德使克林德為拳匪所戕。七月,德與英、法、俄、美、日本、荷蘭、意、比、奧、瑞十一國聯軍入北京,推德將瓦德西為總司令。瓦德西入居禁城儀鑾殿。時命李鴻章為全權大臣,入京議和。各國提出條款:一,中國政府為被戕德公使克林德置立石碑;一,中國政府應派親王前往德國謝罪;一,將總理衙門撤去;一,嚴辦禍首;一,廢去大沽口及直隸各處砲臺;一,禁止軍裝砲火入口;一,各省有曾經殺戮西人,停止鄉試小考五年;一,有事直達中國皇上;一,駐華各使館永遠設兵保護;一,由京至海電報郵政設兵保護;一,國家公司以及私產均照賠。久之始定議,共十二款,而為克林德立碑京城,及遣醇王載灃入德謝罪,均如所請行。十月,獲戕德使克林德犯恩海,交德駐京提督誅之。明年,醇親王載灃至德,見德皇遞書,時帶廕昌一人,俱行鞠躬禮。

二十八年秋七月,德商在漢口華界偪近襄河口請設立躉船,駮之。時政府要求德及英、法、日本撤兵,德使聞他國有在揚子江獨享中國特予權利者,請定明長江上下游進兵要隘不得讓與他國,以定撤兵日期,拒之。三十年,與德會訂小清河岔路合同。初,膠濟鐵路章程原不許擅行另造枝路,今為商務便利計,特委膠濟鐵路公司代辦。是年,德水艦隊擬入長江及各內河游巡演砲,阻之。

三十一年,德撤退膠州、高密兩處兵隊。初,德人在山東修造膠濟鐵路,因高密民聚眾阻工,先後由青島派兵赴膠、高保護鐵路。山東巡撫袁世凱派員查辦議結,駐膠德兵旋即撤回青島。既,拳匪滋事,德人又派兵分駐膠州,並於城北車站旁價購民地十四畝,修造兵房。二十九年秋,又於附近沈家河續租民地七畝,安設水管,以便取汲。高密兵隊先駐城內,後又在城外古城地方議租民地九十餘畝,修造兵房,議定以六個月為限。尋又修築由古城至小王莊火車站馬路一道。時六個月限期已滿,東撫商令退兵,屢延展,至是始訂撤兵善後事宜五款,遂議結。

又議商約,朝廷派呂海寰、盛宣懷為商約大臣。德人提出十四款,袁世凱、張之洞往返電商,海寰等與德使穆默、總領事克納俱迭次會議,彼此堅持。至三十三年,始議定條約十三款,在北京互換。第一款,釐金:中國政府與諸國立約裁撤現有之釐金,加增進出口之關稅以抵裁釐。此約須立約各國派員議決,德國政府亦允派員議結此事,惟中國須當擔保釐金定必全行裁撤方可。第二款,住居:德國人民及德國保護之人民,準在中國已開及日後所開為外國人民通商各口岸或通商地方,往來居住,辦理商工各業制造等事,以及他項合例事業;且準租買房屋、地基、經商之地及他項實產,並可在租買之地內建造房屋。第三款,關棧:中國政府允準在通商口岸設法屯積洋貨及拆包改裝等事。中國政府一經由德領事請將某德商或德國保護人民之棧得享關棧之利益,則中國政府須準如所請,惟須遵照海關所訂之專章辦理,以保餉源。海關官員又須與各國領事議定關棧專章,以及規費若干,須按照該棧離關遠近,屯何貨物,並工作早晚,酌量核定。凡在通商地方所設之關棧,德國人民及德國保護人民均準用之。第四款,礦務:中國政府振興礦務,並招徠外洋資本興辦礦業,故允自簽押此約之日起,於一年內,仿照德國及他國現行礦務章程,頒發礦務新章,以期一面振興中國人民之利益,於中國主權毫無妨礙,一面於招致外洋資財無礙,且比較諸國通行章程,於礦商亦不致有虧。是以中國政府須準德國人民及德國保護人民在中國地方開辦礦務及礦務內所應辦之事。凡所辦礦業,不得因稅項之故致其財源有所虧損,除徵抽凈利之稅及礦產之地稅外,不得另抽他項之稅。第五款,貨稅:還稅之存票,須自商人稟請之日起,如查系應領者,限於二十一日內由海關發給。此等存票,可用在各處海關,按所載銀數,除子口稅一項外,以抵各項出入口貨稅。至洋貨入口後三年之內,轉運外洋,凡執持此等存票者,即準任便在發給之港向海關銀號按全數領取現銀。倘請發存票之人意圖走漏關稅,一經查出,則須罰銀,照其所圖騙之數不得逾五倍,或將其貨入官。第六款,保護商標:凡中國商標,一經呈出在中國各領事所給之據,證明此項商標已在中國認可,且實屬於稟請之人者,均可在德國享保護之利益,與德國之商標相同。華商之姓名牌號,必須在德國保護,以免仿冒。德國商標亦須在中國保護,以防假冒,惟須呈出德國官員並領事所給之據,證明該商標實已在德國註冊,德商之姓名商標以及中國行名均須保護。凡德商包裹貨物之特法,在中國之同業曾已認為某行用以區別某項貨物者,亦須一律保護。德國保護之人民亦能享以上所言之利益。商標註冊局一經成立,保護商標章程亦已刊布,則中、德兩國必須開議特約,以便彼此保護商標。至此約未議之前,以上之款必須施行。第七款,營業:中國人民購買他國營業及公司之股票,是否合例,尚未明定。又因華民如此購買,為數頗巨,故中國現將華民或已購買或將來購買他國公司股票,均認為合例。凡同一合資公司,原入股購票者,彼此一律,不得稍有歧異。遇有華民購買德公司股份者,應將該人民購買股份之舉,即作為已允遵守該公司訂定法律章程,並原按德國公堂解釋該法律章程辦法之據。倘不遵辦,致被公司控告,中國公堂應即飭令買股份之華民遵守該章程,當與德國公堂飭令買股份之德國人民相等無異,不得另有苛求。德國人民如購中國公司股票,其當守本分,與華民之有股份者相同。凡尋常合資股東,及一人或數人有無限之責任,與一人或數人有有限之責任,為合資股東,在德屬經商之有限合資公司註冊,合辦會社有限公司,及各項商業公司等,均須按照以上二節辦理。茲並訂明,本約告成之時,凡曾經呈控公堂而由公堂判定,及不予準理之案,均與是款無涉。第八款,開埠:凡各國代其本國人民船舶索開之口岸地方,德國商人與德國保護之人民,及德國船舶,均可共享此益。第九款,行船:中國本知宜昌至重慶一帶水道宜加整頓,以便輪船行駛,所以彼此訂定,未能整頓以前,應準輪船業主聽候海關核準,自行出資安設拖拉過灘利便之件。其所安設利便之件,無論民船、輪船,均須遵照海關與創辦利便之人商議後所定章程辦理。其標示記號之臺塔及指示水槽之標記,由海關酌度何地相宜備設。將來整頓水道,及利於行船而無害於地方百姓,且不費中國國家之款,中國不宜拒阻。第十款,內港行船章程:前已特準在通商口岸行駛貿易,因是年七月二十八號及九月先後所訂此項章程間有未便,是以彼此訂明,從新修改。第十一款,圜法:中國允原設法定為合例之國幣,將來德國商人及德國保護人民並中國人民,應遵照以完納各項稅課及付一切用款。第十二款,禁令:一千八百八十一年九月二號中德條約附載之通商章程第五款第三節內開,「凡米穀等糧,德商欲運往中國通商別口,照銅錢一律辦理」等因,茲彼此應允,若在某處,無論因何事故,如有饑荒之虞,中國政府先於二十一日前出示禁止米穀等糧由該處出口,各商自當遵辦。倘船只為專租載運穀米,若在奉禁期前,或甫屆禁期到埠尚未裝完已買定之米穀者,仍可準於禁期七日內一律裝完出口。惟米穀禁期之內,應於示內聲明漕米、軍米有無出口。如運出口者,應於海關冊簿詳細登記進出若干,其餘他項米穀,中國政府必須設法一概不準轉運出口。其禁止米穀以及禁期內應運之漕米、軍米數目,各告示均須由中國政府頒發,以期共見。二十一日之期限,必須自京報登刊之日起計。限滿弛禁之告示,亦須載於京報,使眾得聞。至米穀等糧,仍不準運出外國。第十三款,中、德兩國於本約以前所立各條約,除因立本約有所更改外,均仍舊施行。嗣後如有文詞辯論之處,應以德文作為正義。

是年與德訂互寄郵件暫行章程。訂後,德使穆默牒總稅務司聲明三事:一,高密所設之德國郵局,應俟德軍撤屯方能裁撤;二,山東一帶涉及德人之處,所有華局酌用德文人員;三,山東鐵路允中國郵政得有任藉此路運送郵袋之權。總稅務司得牒,均照允,惟酌用德文人員,謂須視有無人才,方能照辦。會德人收中國商報,電政大臣袁世凱請外務部嚴禁。既而德允停收商報,並允中國電報局設在山東鐵路車站。已,復又請由煙臺至上海線及北京至大沽行軍陸線求借用,拒之。又拒德商禮和洋行私購湖南礦產。

又德定濟南、漢口、江寧等處領事兼管各處交涉事宜,照會外務部,略謂「山東省除登州府仍歸煙臺本國領事辦理本國交涉事宜,並膠澳租地歸駐青島德國總督外,其餘所有東省本國交涉事,統歸駐濟南商辦事件委員經理。其煙臺本國領事官,僅有登州府本國交涉事歸其經理。又定明漢口本國領事應辦本國交涉事宜,系湖南、陜西、甘肅三省。湖北除歸宜昌領事辦理各府外,並江西省之袁州府等處,悉歸漢口本國領事經理。至駐江寧府領事應辦本國交涉事宜,系安徽、江西二省。除歸漢口領事之袁州府外,又江蘇省之江寧府等處」雲云。

是年德福親王來京覲見。德皇子婚禮,命出使德國大臣廕昌往賀,並派學生往柏林留學。三十二年二月,德人始在津關請領聯單,赴新疆採買土貨。三月,德使穆默牒中國,請派員往柏林商議無線電會約章,政府約二次開會再行核辦。閏四月,德交還天津馬隊營盤等處房地,並砲隊、機器槍隊、屠牲場、養病院各房屋。是月,德在營口改設正領事。德使穆默回國,署使葛爾士牒中國,復以通商口岸限制洋人置地辦法與條約不符,請除限制,並謂德人地產收回公用,可會商。六月,德人李卜克在北京設立學堂,德使請中國攤出經費,不許。三十三年四月,以孫寶琦為出使德國大臣,遞國書。是月,外務部咨改訂青島租界制成貨物徵稅新章。初,青島設關徵稅一事,已於光緒二十五年與德使海靖議定辦法,嗣於三十一年又與德使穆默修改,其大意即系德國允在海邊劃一地界,作為停泊船隻、起下貨物之定所,凡出口貨在未下船以前,即完出口稅,進口貨除軍用各物暨租地內所用機器並建修物料免稅外,其餘百貨,於起岸後未出新定之界以前,即完進口稅,關員在彼辦理,德國相助無阻。又由中國允每於結底,將本結所收進口稅提出二成,撥交青島德國官憲應用。既因續訂章程,德租界內制成貨物徵稅一條,語義未盡,因與德使葛爾士再訂徵稅新章。

初,中國欲修天津至鎮江鐵路,與德、英借款,已立合同。至是,直隸、江蘇、山東三省京官請攬歸自修,命張之洞、袁世凱商辦,議改合同,德、英執不允。乃又增派外務部右侍郎梁敦彥會同張之洞等籌議。初,津鎮鐵路借款之開議也,德使增索接造支路二道,一由德州至正定,一由兗州至開封,為原議所無,不允。德使乃始變計:一,允由膠澳至沂州府一段,仍作為津鎮支路,歸入官路;二,允由濟南府往山東界之一道,包入津鎮官路。中國亦允由德州至正定府及由兗州府或幹路中之他處過濟寧州至開封府兩支路,於十五年內由中國自行籌辦,並聲明儻用洋款,須向德華公司商借。至是遂由梁敦彥與德、英銀行等改訂借款合同二十四款,名為中國國家天津浦口鐵路五釐利息借款。既定議,即由外務部牒德使,聲明膠沂、濟東路線應作為津鎮支路,其由德州至正定、兗州至開封支路,均由中國自造。已,復與德議訂電政合同,即青、煙、滬水線交接辦法,並購回京沽軍線條款,及山東鐵路附設電線辦法章程共十四款。是年,德柏林賽衛生民學會及萬國玩耍排列館請中國派員入會,許之。

宣統元年,山東巡撫孫寶琦與德立山東收回五礦合同。先是光緒三十三年,山東巡撫楊士驤與德商採礦公司議定合同八條,所指之沂州、沂水、諸城、濰縣四處,已次第查勘,惟第五處礦界內寧海州屬之茅山金礦,查勘未竟。會山東士民倡立保礦會,德公司遂欲將茅山轉售,向中國索價二百二十五萬馬克,並聲言此外四處一並歸還。中國官紳亦以收回為然。籌議久之,始以庫平銀三十四萬兩,分四年清還作結。

三年,山東巡撫孫寶琦與德訂收回各路礦權合同。初,德商礦務公司照約在坊子、馬莊開礦,屢禁華人在附近開礦,爭執有年。迨津浦借款合同簽定,又要索膠沂、津浦路內礦權,並請封禁大汶口華礦,政府不許。於是德使照會始有劃清礦權之語。孫寶琦即派道員蕭應椿等與德公司總辦畢象賢、領事貝斯商議收回,而畢象賢等則以中國欲收回三路礦權,須以相當之利益互換,否則不允。初議淄、博礦界,公司第一次繪送礦界圖,系淄川全境,並毗連博山,蕭應椿等以淄、博窮黎向以採煤為衣食,若兩境全為公司所有,勢必至華民無以為生,因議博境全留,淄境各半,以天臺、昆侖兩山為界,山北歸公司,山南歸華人,公司未允。蕭應椿因親赴淄川會畢象賢查勘,並邀集紳董礦商,旋議定淄川東南境由大奎山起斜經龍口鎮西北至淄川東境為界,界南礦產歸華商辦理,博山亦全讓還,次議淄川華礦,次議濰縣礦界,次議金嶺鎮鐵礦,次議償給勘礦購地費。自是公司已成之膠濟鐵路,未成之津浦鐵路,甫勘之膠沂路,及曹州教案條約許與公司之三十里礦權,均允取消。


\end{pinyinscope}