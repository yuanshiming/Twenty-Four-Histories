\article{本紀二十三}

\begin{pinyinscope}
德宗本紀一

德宗同天崇運大中至正經文緯武仁孝睿智端儉寬勤景皇帝,諱載湉,文宗嗣子,穆宗從弟也。本生父醇賢親王奕枻,宣宗第七子。本生母葉赫那拉氏,孝欽皇后女弟。同治十年六月,誕於太平湖邸第。

十三年,食輔國公俸。十二月癸酉,穆宗崩,無嗣。慈安皇太后、慈禧皇太后召惇親王奕脤、恭親王奕、醇親王奕枻,孚郡王奕譓、惠郡王奕詳,貝勒載澂,鎮國公奕謨,暨御前大臣、軍機大臣、內務府大臣,弘德殿、南書房諸臣等定議,傳懿旨,以上繼文宗為子,入承大統,為嗣皇帝。俟嗣皇帝有子,即承繼大行皇帝。

乙亥,王大臣等以遺詔迎上於潛邸,謁大行皇帝幾筵。丙子,上奉慈安皇太后居鍾粹宮,慈禧皇太后居長春宮。從王大臣請,兩宮皇太后垂簾聽政。皇太后訓敕稱懿旨,皇帝稱諭旨。詔停三海工程。乙卯,停各省貢方物。壬午,頒大行皇帝遺詔。懿旨,醇親王奕枻以親王世襲罔替。翰林院侍講王慶祺有罪,褫職。定服制,縞素百日,仍素服二十七月。伯彥訥謨祜、景壽俱管理神機營。癸未,詔惇親王、恭親王、孚郡王諭旨章奏勿書名,召對宴賚免叩拜。甲辰,詔以明年為光緒元年。

丁亥,上大行皇帝尊謚曰繼天開運受中居正保大定功聖智誠孝信敏恭寬毅皇帝,廟號曰穆宗。戊子,懿旨封皇后為嘉順皇后,皇貴妃為敦宜皇貴妃。諭中外臣工,於用人行政,據實直陳。飭臣民去奢崇實。敕各督撫求民疾苦,慎選牧令,考覈屬吏,並修明武備。壬辰,頒遺詔於朝鮮。甲午,禁內務府官結納太監。乙未,內務府大臣貴寶、文錫褫職。丙申,諭左宗棠督剿河州叛回。丁酉,祫祭太廟。

是月,免浙江被災鹽場灶課。

光緒元年乙亥春正月己亥朔,免朝賀。命吏部尚書英桂、兵部尚書沈桂芬並協辦大學士。戊申,予明故籓硃成功建祠臺灣,追謚忠節。庚戌,敕沈葆楨勘辦瑯軿築城建邑,籌開山撫番事宜。辛亥,祈穀於上帝。清江設廠,收養徐、海被水饑民。內閣侍讀學士廣安疏請廷臣會議大行繼嗣頒鐵券,斥之。丙辰,越南匪黨竄滇邊,巡撫岑毓英剿平之,戊午,上御太和殿,即皇帝位,頒赦詔,開恩科。辛酉,申諭督撫進賢懲貪,除夤緣奔競。

二月丁丑,諭刑部清釐積案。戊寅,祭大社、大稷,豫親王本格攝行。由是大祀皆遣代,至十二年冬至圜丘祀天始親詣。壬午,英繙譯官馬嘉禮被戕於雲南。劉錦棠等復河州。甲申,臺灣生番亂,提督唐定奎剿之。丙戌,賜琉球國王緞匹文綺及貢使緞匹。戊子,嘉順皇后崩。

三月戊戌朔,日有食之。己亥,上大行皇帝尊謚廟號。壬子,山東賈莊河工合龍。丙辰,越南匪黨蘇亞鄧等伏誅。乙丑,召景廉回京,授左宗棠欽差大臣,督辦新疆軍務,以金順為烏魯木齊都統副之。是月,普免各省欠糧,免江西、山西同治六年以前逋賦。

夏四月丁卯朔,享太廟。庚午,命穆圖善調所部馬隊來京,隸神機營,駐南苑。己卯,唐定奎克臺灣南路番社。壬辰,以沈葆楨為兩江總督,兼通商大臣,督南洋海防,李鴻章督北洋海防。

五月戊戌,興直隸水利,防軍墾咸水沽稻田。庚子,大考翰、詹,擢吳寶恕、瞿鴻禨、鈕玉庚、孫詒經一等,餘升黜有差。甲辰,停浙江貢綠玉簪鐲,並停各織造傳制諸品。劉岳昭督攻越南,復同文土州等城。戊申,上嘉順皇后尊謚曰孝哲嘉順淑慎賢明憲天彰聖毅皇后。辛亥,工部神庫火。壬子,刑部科房火。命李瀚章往雲南查馬嘉禮案,薛煥繼往會按之。乙卯,夏至,祭地於方澤。

六月戊辰,吉林將軍奕榕褫職遣戍。庚午,奉天匪據大東溝作亂,崇實討平之。停甘肅例貢。甲午,免直隸同治十年以前民欠旗租並補徵稅。懿旨命醇親王與御前大臣舉各署諳綠營、勇營紀律,及侍衛可任統兵者。壬午,以穆宗帝后梓宮奉移山陵,預戒有司毋備御道,旋禁苛擾。

秋七月戊戌,免直隸同治六年以前逋賦並稅糧。庚子,永定河決。諭各省詳理京控諸獄。貸太原等縣倉穀濟民食。癸卯,賞劉典三品京堂,幫辦陜、甘軍務。免湖北米穀釐金。甲辰,祕魯換約成。諭總署會籌保護華工。丙午,慈安皇太后聖壽節,停筵宴。壬戌,命李鴻章、丁日昌與英使威妥瑪就商馬嘉禮案。候補侍郎郭嵩燾、候補道許鈐身充出使英國大臣。

八月戊寅,免陜西被兵額賦。庚辰,免長蘆、兩淮鹽政應進物品。庚寅,命丁日昌督福建船政。

九月丁酉,諭穆圖善整飭吉林吏治、旗營。甲辰,申定外人游歷內地條約。吳棠督剿敘永竄匪。辛亥,免梓宮經過大興等州縣額賦十之五,遵化十之七,賞平毀麥田籽種銀,並免蠲賸錢糧及旗租。甲寅,奉安梓宮於隆福寺。乙卯,上謁諸陵。閱普祥峪、菩陀峪工程。丙辰,閱惠陵工程。丁巳,奉兩宮皇太后還宮。庚申,至自隆福寺。辛酉,諭王凱泰區處臺灣生番。癸亥,劉長佑剿敗越南匪,匪首黃崇英、周建新伏誅。

冬十月甲子朔,享太廟。癸酉,慈禧皇太后聖壽,停筵宴。甲戌,允丁寶楨請,於煙臺、威海衛、登州府築砲臺,設機器局。己卯,弛浙江南田島禁,聽民耕作。庚辰,賞京師貧民棉衣銀,每歲皆如之。敘永匪李增元等為亂,提督李有恆剿平之。癸未,賞八旗各營一月錢糧,歲以為常。湖南新化、衡、永匪亂,總兵謝晉鈞、提督趙聯升剿平之。丁亥,委散秩大臣吉和、內閣學士烏拉喜崇阿使朝鮮,封李熙子拓為世子。

十一月戊戌,岑毓英克鎮雄大寨,匪首鞠占能伏誅。劉岳昭以玩洩褫職。丁未,予郎中陳蘭彬以京堂候補,充出使美日祕大臣。乙卯,奉天大通溝匪平。戊午,冬至,祀天於圜丘。己未,免朝賀。庚申,祔穆宗帝後神牌於奉先殿。

十二月丙寅,奉安神御於壽皇殿。丁卯,除盛京養息牧鹼地額賦。甲戌,懿旨:「皇帝典學,內閣學士翁同龢、侍郎夏同善授讀於毓慶宮,御前大臣教習國語滿、蒙語言文字及騎射。」大學士文祥請解機務,慰留之。戊寅,免浙江被災新舊賦課。甲申,祈雪壇廟。辛卯,祫祭太廟。

是歲,朝鮮、琉球、緬甸入貢。

二年丙子春正月癸巳朔,免朝賀。戊戌,諭各省宣講聖諭廣訓。癸卯,免仁和等場未墾灶蕩課糧。癸丑,黔匪陷下江,尋復之。丙辰,祈雨。自是頻祈雨。辛酉,四川蠻匪平。

二月乙丑,詔自本年孟夏始,未親政以前,太廟時享及祫祭大祀,俱前一日親詣行禮。己卯,免海沙、蘆瀝等場灶額課。壬午,鄧川匪首羅洪昌、項和伏誅。免浙江逋賦。庚寅,陽萬土州判岑潤清作亂,嚴樹森剿平之。壬辰,東鄉匪聚眾抗官。

三月丙申,以旱故,詔清庶獄。己亥,予吳贊誠三品京堂,督辦福建船政。甲寅,已革都司陳國瑞遣戍黑龍江。丙午,免陜西六十六州縣逋賦。丁未,詔以慈安皇太后四旬萬壽,停本年秋決。貴州四腳牛賊巢及六峒匪平。戊申,以雨澤愆期,諭內外臣工直言闕失。

夏四月乙亥,停陜西進方物,免淮、揚等屬同治六年以前逋賦。壬午,上始御毓慶宮讀書。丙戌,賜曹鴻勛等三百二十四人進士及第出身有差。戊子,蘇熱達熱畢噶爾瑪薩哈進哀表,頒敕答之,並賜緞匹。

五月乙未,文祥卒。乙巳,以近畿亢旱,直隸、山東暨河南、河北等府小民艱食,諭長官撫恤,並捕蝗蝻。丙辰,御史潘敦儼請更上孝哲皇后謚號。予嚴議,尋褫職。

閏五月辛酉朔,賑近畿旱災。庚午,賑福建水災。辛未,以旱敕修省。壬申,孝陵大碑樓災。自春正月不雨,至於是日雨。甲申,階州齋匪平。乙酉,諭劉坤一防海練兵,亟圖整頓。

六月庚寅朔,諭文焜等嚴懲傳習邪術。壬辰,騰越練軍踞城作亂,並陷順寧、雲州。丁酉,以李鴻章為全權大臣,赴煙臺與英使威妥瑪議結馬嘉禮案。庚子,安徽蝗。戊申,開雲南實官捐例。辛亥,以江、皖、魯、豫匪擾,諭沈葆楨等分兵搜剿,解散脅從。丁巳,總兵孔才進攻瑪納斯,斬匪首馬得明等。是月,賑南豐、南昌、福建水災。

是夏,免淮、揚等屬逋賦,盛京同治六年以前逋賦,長蘆各場同治十年以前灶課,直隸同治十年以來逋賦。

秋七月辛酉,上兩宮皇太后徽號。辛未,復淮鹽楚岸引地。甲戌,東鄉匪首袁廷蛟伏誅。辛巳,劉長佑、潘鼎新復騰越各城,匪首蘇開先伏誅。戊子,馬嘉禮案議結,免案內官所坐罪。

八月辛卯,劉錦棠、金順擊敗回酋白彥虎,復烏魯木齊、迪化城,尋復昌吉、呼圖壁、景化各城。辛丑,許鈐身改出使日本大臣。丁未,賑浙江水災。辛亥,賑江西水災。孔才等復瑪納斯北城。

九月戊午朔,予上元、江寧兩縣一門殉難三十五家百九十五人旌恤建坊。壬戌,順天增設粥廠。己巳,定出使各國章程。以四川州縣民、教訟鬩,諭魁玉等持平訊斷。壬申,諭文煜等嚴緝福建、江西、安徽等省邪教匪黨。

冬十月丙午,賑皖北旱災。命景廉、李鴻藻在總理各國事務衙門行走。甲寅,召榮全來京,以金順為伊犁將軍。丁巳,賑口北、山東、安徽、江北饑。

十一月丁卯,金順、錫綸克瑪納斯南城,匪首何碌、馬有財伏誅。壬午,以新疆北路平,發帑汰遣金順軍。甲申,截漕一萬石,並提倉穀濟蘇、常留養災民。

十二月戊子,命侍講何如璋充出使日本大臣。甲辰,命督撫嚴查州縣,毋匿災,各省民、教案持平審理。戊申,賑江北淮、海災。己酉,回匪竄擾科布多,參贊大臣保英派兵遲緩,切責之。乙卯,免杭、嘉、松各場未墾地灶課。

三年丁丑春正月丁巳朔,免朝賀。戊午,命以左都御史景廉為軍機大臣。庚申,命前藏濟嚨呼圖克圖於達賴未出世以前掌商上事務,給「達善」名號。癸亥,以英桂為體仁閣大學士,載齡以吏部尚書協辦大學士。丙寅,免洪澤湖灘欠租。

二月戊子,穆坪夷匪伏誅。己丑,申諭各省墾荒田,禁械斗,慎舉劾,整營規。賑直隸、山東、山西、河南、安徽、江西、福建還籍饑民。巳亥,免湖北逋賦。庚子,懿旨:「梓宮在殯,皇帝沖齡,除朝賀大典外,其頒慶賞宴外賓典禮暫緩舉行。」辛丑,復淮鹽引地。壬寅,刑部平反浙江民人葛品連獄,巡撫楊昌濬、侍郎胡瑞瀾褫職,知府以下論罪有差。申諭各省理刑,期情真罪當,毋輕率。

三月丁巳朔,上釋服。以山陵未安,仍禁官中宴會演劇。辛未,免華陰被水糧課三年。癸酉,以劉錫鴻充出使德國大臣。賑沭陽災民。辛巳,除臺灣府屬雜餉,賑內山饑番。

夏四月辛卯,常雩,祀天於圜丘。甲午,馬邊惈夷結野番、黑夷出擾,魁玉等剿之。乙未,免酃縣被水逋賦。戊戌,劉錦棠等克七克騰木、闢展,復吐魯番滿、漢兩城。尋攻克達阪及託克遜賊壘,安集延酋帕夏自殺。己亥,總兵張其光攻臺灣率芒番社,克之。庚子,貸義州旗戶籽種銀。辛丑,賑貴陽地震災。壬寅,昭通、廣南匪作亂,官軍討平之。癸卯,以災區緩徵,吏胥舞弊,諭各省整頓。旌安貧樂道高郵增生韋弼諧。甲辰,越南遣使進方物,賚其國王緞匹。庚戌,賜王仁堪等三百二十九人進士及第出身有差。是月,江蘇、安徽蝗。

五月戊辰,日本阻琉球入貢,遣來使歸國。癸酉,山西旱,留京餉二十萬賑之。甲戌,監利會匪王澕漳等作亂,伏誅。撥帑銀一百二十萬解西征糧臺。戊寅,賑福州水災。壬午,懿旨以皇上萬壽值齋戒期,更定六月二十六日行慶賀禮,著為令。山西大旱,巡撫曾國荃請頒扁額為禱。以非故事,不許。諭曰:「禱惟其誠,當勤求吏治,清理庶獄,以迓和甘。」

六月戊子,詔工噶仁青之子羅布藏塔布克甲木錯即作為達賴喇嘛之呼畢勒罕,毋庸掣瓶。辛卯,廣東北江堤決,連州大水,詔賑撫災民。戊戌,先是穆宗祔廟位次,懿旨命大臣會議,醇親王復請定久遠至計,少詹事文治,鴻臚寺卿徐樹銘、少卿文碩,內閣侍讀學士鍾佩賢、司業寶廷並有陳奏。至是,仍命王大臣等詳議以聞,並命李鴻章妥議。丙午,以災祲疊見,誡臣工修省。庚戌,上萬壽,禦乾清宮受賀。

秋七月丁巳,撥海防經費助山西賑。己未,惇親王等議上穆宗帝後神牌位次,請於太廟中殿東西各四楹,遵道光初增奉先殿後殿龕座,修葺改飾,並從醇親王請,自今以往,不援百世不祧之例。戊辰,免江寧、上元等縣被災額賦十之三。己巳,留京餉漕折銀賑河南饑。

八月丁亥,諭各省修農田水利。壬辰,撥天津練餉十萬濟山西賑。甲午,免臺灣同治十年供粟及糯米易穀。庚子,諭劉坤一等整頓廣東捕務。戊申,撥銀四十萬賑山西、河南災,並留江安漕糧輸山西、河南各四萬石備賑。

九月甲寅,羅田匪首陳子鰲伏誅。戊午,命前侍郎閻敬銘往山西查賑。己未,申禁山西種罌粟,改植桑、棉。辛酉,撥山東冬漕各八萬石續賑山西、河南災。甲子,予漢儒河間獻王劉德從祀文廟。乙丑,詔求直言。丁卯,命李鶴年往河南查賑。戊辰,減緩山西、河南應協西征軍餉。庚辰,加賑祥符等縣災民口糧。辛巳,賑興化府屬風災。

冬十月壬辰,賑三姓雹災。庚子,諭各省安撫轉徙饑民。甲辰,免三姓被災銀穀。加賑陽曲等縣災民口糧。乙巳,增設內城粥廠。庚戌,劉錦棠進復喀喇沙爾、庫車兩城,尋復阿克蘇及烏什城。

十一月癸丑,詔戒各部院玩愒因循。乙卯,開山東運漕新河。丁巳,諭督、撫、府尹講求吏治。

十二月辛卯,緩赫哲貢貂。庚子,豫免山西、河南被災州縣來歲糧。

是冬,連祈雪。撥來年江、鄂漕米凡十二萬石賑山西,發帑金賑陜西。

是歲,山、陜大旱,人相食。

四年戊寅春正月辛未,賑河南饑。命郭嵩燾兼出使法國大臣。西軍復葉爾羌、喀什噶爾,和闐回眾降。己卯,諭各省清理詞訟。

二月辛巳朔,修成都都江堰。壬午,諭興北方水利。乙酉,命署兵部左侍郎王文韶為軍機大臣。庚寅,諭舉州縣能實行荒政者。壬辰,新疆平,匪首白彥虎遁入俄羅斯。論功,進左宗棠二等侯,劉錦棠二等男,予提督餘虎恩等世職有差。甲午,諭清庶獄。丁酉,賑呼蘭災。己亥,下詔罪己。賑山西、河南饑。丙午,瘞災地遺骸。庚戌,免侯官被水丁糧。

三月甲寅,諭被災各省試行區田法。壬申,賑直隸饑,撥察哈爾牧群馬三千匹給貧民耕作。甲戌,諭內務府,減經費,除浮冒。戊寅,英桂致仕。是月,河南雨。

四月壬午,沈葆楨請罷武科,斥之。壬辰,賑廣東風災。

五月庚戌朔,諭直省廣植桑、茶。命載齡為體仁閣大學士,管工部事,全慶以刑部尚書協辦大學士。辛未,以崇厚為出使俄國大臣。

六月丙戌,免陜西逋賦。庚寅,嚴私鑄禁。甲午,賑臺灣風災。庚子,諭刑部嚴定州縣侵賑罪。

秋七月乙卯,雲南官軍復耿馬土城。辛未,命禮部右侍郎王文韶、順天府府尹周家楣在總理各國事務衙門行走。壬申,嚴命案延玩處分。甲戌,以曾紀澤為出使英法大臣。丁丑,免平陽、蒲、解、絳今歲秋賦。是月,賑金、衢、嚴等府,浮梁等縣水災。

八月己卯,永定河決。丙戌,沁河決。戊子,賑崇安、浦城水災。

九月丁巳,諭東南疆吏豫救水患,清釐保甲,防會匪煽惑災民。癸亥,賑山西旱,免陽曲等縣逋賦,及徐溝等縣秋糧。戊辰,賑藍田水災。丙子,修樊口江堤。

冬十月壬午,廣西在籍總兵李揚才叛,命馮子材剿之。免通、海各處,淮安四衛逋賦並雜課。丁亥,賑濮、範、壽陽水災。癸巳,沁河復決。賑奉天水災。乙未,北新倉火。戊戌,臺灣後山加禮宛等社就撫,縛獻番目,誅之。免貴州被兵新舊額賦。

十一月丙辰,修北運河堤。辛酉,白彥虎寇邊,劉錦棠擊敗之。癸亥,李揚才踞越南長慶,楊重雅剿之。己巳,詔督撫整躬率屬。責軍機大臣勿避嫌怨,院部大臣力戒因循。甲戌,冬至,祀天於圜丘。乙亥,停朝賀。

十二月己丑,詔永罷捐輸事例。

是歲,免仁和鹽場逋課者二。朝鮮、廓爾喀入貢。

五年己卯春正月乙巳朔,停筵宴。乙丑,申諭停籌餉捐例。修高淳堤。辛未,賑山西饑。

二月壬午,吉州知州段鼎耀以吞賑處斬。癸未,詔復河運。甲午,諭山西清理荒田,編審丁口,均差徭。己亥,梓宮奉安山陵,禁有司科派擾累。賑文安等州縣水災。

三月丙午,賊目鍾萬新與李揚才合犯宣光,馮子材會師越南擊之。壬子,免梓宮所過大興、通、三河、薊、遵化額賦。庚申,頒吉嚨呼圖克圖敕書,並賚哈達、蟒緞。布魯特回酋合安集延賊酋寇邊,劉錦棠敗之。乙丑,奉兩宮皇太后謁東陵。己巳,謁昭西、孝東諸陵。庚午,葬穆宗於惠陵,孝哲後祔。癸酋,至自東陵。

閏三月乙亥,穆宗神主祔太廟,頒詔天下。丁亥,李揚才踞者巖。己丑,修襄陽、沔陽、天門江堤。庚寅,吏部主事吳可讀於東陵仰藥自盡,遺疏請豫定大統。懿旨,王大臣等集議以聞。乙未,命三品卿銜李鳳苞為出使德國大臣。

夏四月戊申,修通州北運河。癸丑,予吳可讀恤典。懿旨,以可讀原疏及會議摺,徐桐、寶廷、張之洞等摺,並前後諭旨均錄存毓慶宮。免河南被災州縣漕銀及逋課。己巳,先是嶧縣知縣硃永康以謀殺委員高文保論戍,尋下廷議。至是,奏上,詔以罪浮於法,改論死。

五月丙子,夏至,祭地於方澤。己卯,免兩淮、泰、海各場逋課。壬午,河南蝗。己亥,官軍剿平者巖賊。是月,賑清河、安東風災。山西雨。階、文、西和地震歷十有三日。

六月壬子,刑部言東鄉獄事,誣叛妄殺,已革知縣孫定揚、提督李有恆論死。尋文格、丁寶楨並坐奪職。命發帑二十萬,撥丁釐銀三十萬,濟山西賑需。己未,諭言事諸臣,交部議奏之事,不得攙越陳奏,亦不得雷同附和,相率瀆陳。普免山西積年民欠倉穀。烏拉特、阿拉善等旗蝗。甲子,懿旨允醇親王奕枻家居養疾,解職務。賑邠、乾、漢、鳳地震災。

七月庚辰,賑直隸水災。戊子,以星變、地震求直言。諭各省積穀。免絳、蒲、陽城被災夏課鹽稅。庚寅,復海運。

八月戊申,祭大社、大稷。詔各省舉文武堪備任使者。壬子,致仕大學士單懋謙卒。癸丑,賑博山等州縣水災。乙卯,江、皖各屬蝗。乙丑,賑階、文、西和地震及水災。

九月甲戌,賑直隸水災。壬辰,加上文宗、穆宗尊謚。己亥,重慶等府縣地震,賑之。

冬十月辛丑朔,免曲沃等州縣歉收額賦。乙卯,免奉天旗民站丁地課抵例賑口糧。丁巳,諭水師並習陸戰。癸亥,賑秀山等處水災。己巳,英桂卒。免齊齊哈爾、黑龍江、墨爾根屯糧,並原貸籽種。

十一月乙亥,李揚才伏誅。己卯,冬至,祀天於圜丘。庚辰,停朝賀。壬午,沈葆楨卒。甲申,以劉坤一為兩江總督,兼南洋大臣。庚寅,詔責崇厚與俄人定伊犁約,擅自回京,所議條約,廷臣集議。壬辰,免山西災重州縣稅契銀。

十二月己酉,懿旨,廷議俄約覆奏,下王大臣等再議,醇親王並預議以聞。乙卯,褫崇厚職,下獄。辛酉,諭修社倉,興社學。己未,免永濟等州縣秋糧。丙寅,祫祭太廟。詔洗馬張之洞會商俄約。戊辰,修山東運河。

是歲,朝鮮、廓爾喀入貢。

六年庚辰春正月己巳朔,停筵宴。辛未,命曾紀澤為出使俄國大臣,改議條約。甲戌,諭查營伍虛額占役。乙亥,西林苗匪平。丙子,命前工部尚書李鴻藻仍為軍機大臣。壬午,尋甸匪亂,官軍討平之。己丑,詔中外舉人才,疆吏飭邊備海防。命河北道吳大澂幫辦吉林軍務,通政使劉錦棠幫辦新疆軍務。辛卯,定崇厚罪,論斬。癸巳,戶部奏籌饟十條,詔各省推行。是月,除山西各屬荒地丁銀,免仁和等場荒蕩夏稅。

二月乙巳,永免榆次貢瓜。壬戌,甘肅總兵蕭兆元侵蝕軍糧,論斬。

三月甲戌,賑順直水災。乙亥,左宗棠出屯哈密,金順扼精河,張曜、劉錦棠分進伊犁。己卯,免山西洪洞、忻州各屬荒賦三年或四年。

四月庚子,祀天於圜丘。復設科布多昌吉斯臺、霍呢邁拉扈等八卡倫官兵。丙午,三姓設廠造輪船。甲寅,階州番匪哈力等作亂,伏誅。壬戌,賜黃思永等三百三十三人進士及第出身有差。乙丑,調李長樂為直隸提督,統武毅四營,鮑超為湖南提督,召來京。

五月丙子,賑洛陽等縣雹災。乙酉,階州番匪古旦巴等伏誅。丙戌,以徇俄人請,貸崇厚死,仍系獄。

六月丁酉朔,賑福建水災。癸卯,畀李鴻章全權大臣,與巴西議約。甲辰,禁徵糧浮收勒折。丙辰,賑廣州等處水災。丁巳,免交城等縣荒地缺課。命曾國荃督辦山海關防務。

七月壬申,召左宗棠來京,督辦關外事宜。癸酉,出崇厚於獄。癸未,賑揚州風災。甲申,命前浙江提督黃少春辦理浙江防務。

八月己亥,巴西商約成。戊申,召劉銘傳來京。庚戌,南北洋初置電線。壬子,江蘇捕蝗。癸亥,朝鮮來告與日本交聘。

九月己巳,命浙江提督吳長慶幫辦山東防務,節制防軍。庚午,免永濟貢柿霜。辛未,允朝鮮派工匠來天津學造器械。壬申,賑蒲城等處災。壬午,給曾國荃病假,命岐元節制各軍。癸未,減涼、肅番族馬貢。己丑,賑資陽、清溪災。庚寅,印度進樂器並所撰樂記,賚以金寶星。癸巳,除拉林旗佃租賦。

冬十月丙午,察木多帕克巴拉胡圖克圖進貢物,以哈達、大緞賜之。己酉,東明河決。辛亥,命前吏部尚書毛昶熙在總理各國事務衙門行走。甲寅,賑圍場海龍城及渮澤水災。甲子,懿旨醇親王管理神機營事務。

十一月乙丑朔,命侍講許景澄為出使日本大臣。己巳,以全慶為體仁閣大學士,靈桂以吏部尚書協辦大學士。甲申,冬至,祀天於圜丘。丙戌,江華瑤匪平。癸巳,免永平等屬逋賦。

十二月丙午,命楊昌濬會辦新疆善後。丙辰,免文安被水額賦。庚申,懿旨神機營選弁兵赴天津學制外洋火器。辛酉,濬漕運河道。

是冬,數祈雪。

是歲,朝鮮、廓爾喀入貢。

七年辛巳春正月甲子朔,停筵宴。沈桂芬卒。癸酉,敕各省慎舉孝廉方正。乙亥,達賴喇嘛遣人進哈達、佛香,命獻惠陵,賚以哈達、緞匹。戊寅,免浙江仁和等場荒坍灶蕩,各府州縣衛荒地新墾地六年逋課及額糧。辛卯,越南請官兵助剿積匪,不許。免海陽六年逋賦。壬辰,命左宗棠為軍機大臣,管兵部,兼總理各國事務衙門行走。除貴築、興義、八寨水銀等廠逋課。

二月癸巳朔,命李鴻章籌山海關防務,節制諸軍。以曾國荃為陜甘總督。戊戌,日本使臣戶璣來議琉球條款,不協,敕海疆戒備。己酉,修襄陽老龍石堤。辛亥,修濟陽壩工。甲寅,通政司參議劉錫鴻以誣劾李鴻章職。

三月甲子,除錦州官田租賦。丁卯,改築焦山都天廟砲臺。己巳,命李鳳苞兼出使義和奧大臣,黎庶昌為出使日本大臣。辛未,慈安皇太后不豫,壬申,崩於鍾粹宮。癸未,上大行皇太后尊謚曰孝貞慈安裕慶和敬儀天祚聖顯皇后。

夏四月癸巳,雷波夷匪平。己亥,命吳大澂督辦吉林三姓、寧古塔、琿春防務兼屯衛。免陜西咸寧等六十二州縣逋賦。辛丑,頒孝貞顯皇后遺誥於朝鮮。己酉,曾紀澤與俄國改訂新約成。丙辰,永禁明陵私衛。己未,懿旨,恭親王、醇親王會同左宗棠、李鴻章議興畿輔水利。初置琿春副都統。庚申,賑臺北地震災。

五月壬戌朔,日有食之。官軍擊散越南積匪。丁卯,詔疆臣於命盜重獄按月冊報,遲逾者罪之。戊寅,罷烏里雅蘇臺屯田。己丑,賑鹽源水災。賞鄭藻如三品卿銜,為出使美日祕大臣。

六月己亥,彗星見,詔修省。丙辰,萬壽節,停朝賀。己未,命李鴻藻協辦大學士。

秋七月癸亥,賞學行純篤廣東在籍知縣硃次琦、舉人陳澧並五品卿銜。戊子,召劉坤一來京,以彭玉麟署兩江總督兼南洋大臣。賑階州等處地震災。

閏七月壬辰,諭各省統覈釐卡出入,酌定撤留。癸巳,賑兩淮、泰州各場灶災。甲午,免榆社等縣五年逋賦。己亥,命金順督辦交收伊犁事宜,錫綸為特派大臣,與俄人會商界務。尋命升泰並為特派大臣。甲辰,命鮑超復裁所部營伍。乙巳,初置呼倫貝爾副都統。庚戌,禁州縣諱飾重獄。是月,賑江蘇、福建、四川水災,陜西雹災。

八月甲子,頒帑金二萬給養霍碩特流民。辛巳,以皇太后疾愈,命刑部停秋決。其緩決屆三次與未屆三次,分別差減之。癸未,孝貞顯皇后奉安,免所過州縣租賦。命劉錦棠為欽差大臣,督辦新疆軍務,張曜副之。丙戌,除伯都訥磽地賦額。全慶致仕。

九月甲午,賑寧海等縣水災。乙未,允彭玉麟解職,仍巡閱長江。劉坤一罷,以左宗棠為兩江總督,兼南洋大臣。丙午,葬孝貞顯皇后於定東陵。丁未,汝寧、光州捻匪平。己酉,再減金壇漕額十分之一分四釐。賞附居青海番眾八族青稞歲八百餘石。辛未,孝貞顯皇后神牌祔太廟。丙辰,賑臺灣颶風災。是月,甘肅、臺灣地震。

冬十月己巳,皇太后聖壽節,停筵宴。庚午,昭通匪陸松山等作亂,官軍討斬之。癸酉,以靈桂為體仁閣大學士,以刑部尚書文煜協辦大學士。甲戌,法人踞越南北境,諭滇、粵合籌弭釁。甲申,詔舉行察典,勿有舉無劾。賑泰和等縣水災。丁亥,安徽已革提督李世忠擅縶貢生吳廷鑒等,裕祿上其狀,詔處斬。

十一月庚寅,免吉林被水官莊及伯都訥地租。丙申,施南會匪楊登峻伏誅。丁酉,濬吳淞淤沙。戊戌,廣西果化土州匪首趙蘇奇伏誅。賑貴縣等處水災。甲辰,賑臺灣、澎湖災。

十二月乙亥,賞恭親王子載潢不入八分公,醇親王子載洸奉恩輔國公。是月,免浙江各府州縣衛荒廢及新種賦課,仁和等場灶課。免安州、任縣、文安澇地額糧。除吉林荒地租賦。

是冬,頻祈雪。

是歲,朝鮮、越南入貢。

八年壬午春正月戊子朔,免朝賀。辛卯,修洞庭西湖堤。自去年十一月不雨至於是月。己亥,雪。庚戌,修滹沱新河及子牙河堤。

二月己未,江蘇文廟火。壬戌,以朝鮮占種吉林邊地開墾歷年,令其領照納租隸籍。癸丑,申嚴門禁,更定稽察守衛章程。壬午,申禁私伐明陵樹木。乙酉,先是江寧疑獄,命麟書、薛允升往勘之。至是訊明,委員胡金傳以酷刑論斬。諭疆吏詳覈重獄,勿冤溢。

三月乙未,命左副都御史陳蘭彬在總理各國事務衙門行走。庚戌,李鴻章母憂,連疏請終制,許之;命百日後駐天津練軍,仍權理通商事務。辛亥,法、越構兵,諭李鴻章、左宗棠、張樹聲、劉長佑籌邊備。乙卯,築浙江海口砲臺。是月,俄人歸我伊犁。

是春,免陽曲逋糧、大城額賦及累年逋賦。

夏四月丙辰朔,永免山西荒地稅糧。戊午,免陜西前歲逋賦。己巳,法人入越南東京。起曾國荃署兩廣總督。甲戌,全慶卒。甲申,朝鮮請遣使來駐京師,不許,惟予已開口岸貿易。

五月丙戌朔,諭金順經畫伊犁,西北邊界以長順勘分,西南以沙克都林札布勘分。戊子,賑汀州風災。壬辰,召劉長佑來京,以岑毓英署云貴總督。乙巳,初置吉林分巡道。庚戌,直隸蝗。

六月丁巳,翰林院侍讀溫紹棠奏稱時事多艱,請皇太后勵精勤政。詔以皇太后尚未康復,飭之。命整頓八旗官學。乙亥,清安言俄兵至哈巴河。諭長順詳慎勘界,以杜覬覦。戊寅,朝鮮匪亂,命張樹聲剿平之。尋提督丁汝昌往援,吳長慶率師東渡。癸未,朝鮮焚日本使館,日本以兵船至。命李鴻章赴天津部署水陸軍前往察辦。是月,賑安徽水災,浙江、江西水災。

秋七月乙酉朔,三巖野番就撫。乙巳,懿旨損秋節宮費,賑安徽、浙江、江西三省災。丁未,吳長慶軍入朝鮮,執其大院君李正應。初置新疆阿克蘇、喀什噶爾分巡道。癸丑,朝鮮亂平。

八月丙辰,諭:「科布多界務,崇厚貽誤於前,曾紀澤力爭於後。茲訂新約,應就原圖指辦,酌定新界。清安等當與俄官量議推展,期後來彼此相安。」丁巳,諭有司慎覈秋審。甲子,詔雲南布政使唐炯出關視邊防。乙丑,安置李正應於保定。尋朝鮮國王乞釋歸,不許。丁丑,彗星復見東南,詔內外臣工修省。

九月乙酉,河決山東惠民、商河、濱州。癸巳,鬱林匪亂,官軍剿平之。

是秋,賑四川、浙江、山東、陜西、福建、江西、貴州水災,資州火災,臺灣風災水災。

冬十月乙卯,諭京師嚴緝捕,毋諱飾擾累。壬戌,河決歷城。甲子,諭捕啯匪。丁丑,王文韶連疏乞罷。溫旨慰留。

十一月丁亥,王文韶仍以養親乞罷,許之。命翁同龢為軍機大臣。戊子,命潘祖廕為軍機大臣。臺州匪首王金滿日久逋誅,下所司嚴緝。乙未,允朝鮮互市。辛丑,開天津塌河澱南新河。壬寅,以地震詔臣工勤職察吏。庚戌,詔中外保薦人才。是月,開銅山縣煤鐵礦。

十二月辛酉,命游百川赴山東勘河工。壬戌,設滬、粵沿海電線。乙丑,詔中外清理積案。壬申,自上月連祈雪,至是雪。

是冬,賑直隸地震災,四川、陜西雹災。免齊齊哈爾、墨爾根歉地,浙江州縣衛新舊屯地,仁和等場灶蕩額賦。

是歲,朝鮮入貢。

九年癸未春正月癸未朔,停筵宴。丙申,劉錦棠言沙克都林札布與俄使勘分新疆南界,不符舊約,諭長順等按約詰之。尋諭曾紀澤力爭重勘。戊戌,命宗人府丞吳廷芬在總理各國事務衙門行走。庚子,諭蠲免錢糧,民已輸官者,得抵翌年正賦,勿重徵。乙巳,撥鄂漕三萬石備賑順直饑。是月,越南匪覃四娣等降。

二月甲寅,直、魯流民紛集京師,諭有司撫恤。戊午,山東河決歷城,齊河諸縣民墊壞,命游百川等賑撫災民。己未,先是馬蘭鎮總兵景瑞修繕營房,為營兵匿控,總兵桂昂請兵激變,遣伯彥訥謨祜、閻敬銘查辦。至是覆陳,褫景瑞職,桂昂尋並褫職。禁各省酷吏非刑。命廣西布政使徐延旭出關籌防。戊辰,福建按察使張夢元督辦福建船政。癸酉,高州都司莫毓林聚亂,伏誅。庚辰,刑部言河南胡體洝一獄,原讞舛誤,覆審回護。詔褫巡撫李鶴年、河東河道總督梅啟照職,原審官譴戍有差。

三月戊子,鎮國公溥泰收受禁墾澱地,坐削爵,圈禁一年。法人陷南定。乙未,命唐炯統防軍守雲南邊境。諭倪文蔚保北圻。

是春,免潛山等縣夏糧,陜西被旱丁糧米折。賑濟南、武定水災,臺灣地震災。

夏四月己未,俄撤伊犁駐兵。甲子,諭嚴緝畿輔盜賊。甲戌,劉長佑以病免,授岑毓英雲貴總督。乙亥,賜陳冕等三百八人進士及第出身有差。

五月辛巳,詔李鴻章回北洋署任,部署海防。壬午,命升泰與俄使勘分塔爾巴哈臺西南界。丁亥,湖南會匪方雪敖倡亂,擒斬之。辛卯,禁私鑄錢。庚子,諭岑毓英等選募邊民,與官軍扼守滇、越要隘。戊申,懿旨醇親王會籌法、越事宜。先是,御史陳啟泰奏太常寺卿周瑞清包攬雲南報銷,御史洪良品、給事中鄧承修以事涉樞臣景廉、王文韶,相繼論劾。先後命惇親王、閻敬銘、潘祖廕、張之萬、麟書、翁同龢、薛允升會同察辦。至是覆陳,瑞清等罪如律,戶部尚書景廉,前侍郎王文韶、奎潤,前尚書董恂,與前雲貴總督劉長佑俱金雋三級,餘處罰有差。

六月庚戌,山東河決,壞歷城、齊東、利津民墊,諭堵塞賑撫並行。越將劉永福及法兵戰於河內,敗之。乙卯,修沁河堤。戊午,法國遣使托利古來議和約。太監王永和盜御用衣物,詔刑部按律擬罪,勿株連。丁卯,濬山東小清河。庚午,山東以水災開辦賑捐事例。

是夏,免雲南土司地租,甘肅舊欠糧賦。又免懋功被災、銅仁被水額糧。留漕糧凡十萬石、京餉十六萬兩賑山東災。

秋七月己卯,留京餉二十萬給廣西軍。壬午,諭令吳全美、方耀分巡廉、瓊洋面及欽州邊境。戊子,詔開雲南礦。辛卯,臺州匪首王金滿率眾降,詔免死,與餘眾留營效力。

八月庚戌,法人破順化河岸砲臺,越人停戰議和。壬子,永定河決。乙卯,考察部院官。諭修築沿海堤塘各工,並撫恤災戶。丙寅,詔舉謀勇兼優堪備任使者。己巳,詔彭玉麟赴廣東,會同張樹聲布置防務。

九月辛巳,法、越議和,立新約。丙戌,命何如璋督辦福建船政,倪文蔚為廣東巡撫,徐延旭為廣西巡撫。己亥,撥廣西庫銀十萬濟劉永福軍。丁未,唐炯以率行回省褫職,仍留任。

是秋,撥京倉及漕米五萬餘石,庫帑凡十萬,賑順天直隸。留漕五萬石,賑山東。賑熱河、長陽、崞縣等處水災。賑江南災。

冬十月戊辰,詔南北洋及沿江沿海諸省嚴戒備。辛未,河決齊東、蒲臺、利津。丙子,詔李鴻章舉將才。命岑毓英出關駐山西,唐炯回滇籌餉。

十一月辛巳,命署左副都御史張佩綸在總理各國事務衙門行走。壬午,趣徐延旭出關策應。辛卯,嚴內外城門禁。壬辰,越南民變,殺嗣王阮福時,命張樹聲戡定之,尋改命岑毓英往平亂。庚子,懿旨,清江設廠收養災民,命戶部發帑一萬接濟,並給順直、山東各四萬,湖北三萬,安徽二萬。壬寅,法人陷山西,劉永福退走。癸卯,詔以尚書文煜被劾,回奏積俸至三十六萬,命捐銀十萬充公。林肇元坐庫儲空虛奪職。

十二月戊申,祈雪。庚戌,法人進攻北寧,圖犯瓊州。命彭玉麟檄湘楚軍會合吳全美師船嚴防,起楊岳斌往福建會辦海防。官軍大敗法人於諒山。己未,以山東、淮、徐災民聚集清江等處,命所司撫恤,並隨時資遣。庚申,諭江西籌餉二萬濟王德榜軍。丁丑,追復故總兵陳國瑞世職。

是冬,免順天直隸等州縣秋賦,浙江被災州縣衛所額賦。除山西鳳臺等州縣荒地租糧。

是歲,朝鮮、越南入貢。

十年甲申春正月庚寅,岑毓英出鎮南關赴興化,節制邊外諸軍。

二月丁未朔,法人攻興化,官軍擊卻之。岑毓英與徐延旭進圖山西。諭嚴約束,勿擾越境。留江、浙漕米各五萬石賑通州、天津水災。尋撥京倉粟米三萬石賑順天災。丁丑,法人陷北寧,官軍退守太原。戊辰,命湖南巡撫潘鼎新赴廣西籌防。乙亥,法人陷太原,徐延旭、唐炯褫職逮問。

三月丁亥,岑毓英請免節制楚、粵諸軍,不許。以太原陷,提督黃桂蘭、道員趙沃並褫職逮問。戊子,懿旨以因循貽誤罷軍機大臣恭親王奕家居養疾,大學士寶鋆原品休致,協辦大學士李鴻藻、景廉俱降二級,工部尚書翁同龢褫職仍留任。命禮親王世鐸,戶部尚書額勒和布、閻敬銘,刑部尚書張之萬並為軍機大臣。工部侍郎孫毓汶在軍機學習。己丑,懿旨軍機處遇重要事,會同醇親王商榷行之。壬辰,授潘鼎新廣西巡撫,張凱嵩雲南巡撫。總兵陳得貴失守砲臺,副將黨敏宣臨陣退縮,詔並斬於軍前。以怡親王載敦為閱兵大臣。命貝勒奕劻管總理各國事務衙門事,內閣學士周德潤在總理各國事務衙門行走。癸巳,左庶子盛昱、右庶子錫鈞、御史趙爾巽各疏陳醇親王不宜與聞機務,不報。命刑部侍郎許庚身在軍機學習。甲午,詔李鴻章、左宗棠、曾國荃、岑毓英舉部將中沈毅勇敢有謀略者。己亥,閻敬銘、許庚身並在總理各國事務衙門行走。命潘鼎新赴鎮南關接統徐延旭軍。庚子,法人進據興化。

是春,免仁和荒蕪灶蕩上年逋課,陜西咸寧等處逋賦及雜欠。免穆坪土司馬匹糧草十年。

夏四月丙午,勘分新疆南界事竣。以侍講許景澄充出使法德義和奧大臣。庚戌,先是,法、越戰事亟,法水師將福祿諾屬稅務司德璀琳獻議媾和息兵。李鴻章以聞,許之,敕其籌定。至是,覆陳「當審勢量力,持重待時」。詔集廷議。懿旨醇親王並與議。允吳長慶兵還。辛亥,利津等決口合龍。癸丑,罷開馬頰河,濬宣惠河,修德州運河堤。戊午,命通政使吳大澂會辦北洋事宜,內閣學士陳寶琛會辦南洋事宜,侍講學士張佩綸會辦福建海疆事宜,皆許專奏。尋加佩綸三品卿銜。福祿諾出私議五條,因李鴻章上聞。敕鴻章「力杜狡謀,常存戒懼」。詔戶部裁冗費。庚申,授李鴻章全權大臣,與法使議約。癸亥,免褒城瀕江地畝額賦。乙丑,祈雨。丙寅,再發倉米賑順天。戊辰,吳大澂辭北洋會辦。上責其飾詞,不許。壬申,張樹聲以疾請免本職,專治軍事,許之。

五月丙子,命李成謀總統江南兵輪。己卯,岑毓英辭節制粵、楚各軍,許之。丁亥,授文煜武英殿大學士。戊子,額勒和布、閻敬銘並以戶部尚書協辦大學士。己丑,京師久旱,諭有司平糶。賞徽寧太廣道張廕桓三品卿銜,在總理各國事務衙門學習行走。辛卯,詔中外保薦文武人才。甲午,詔皇太后五旬萬壽,停秋決。丁酉,詔中外大臣「率屬盡職,勿耽逸樂、尚浮華」。戊戌,詔左宗棠仍為軍機大臣,毋庸常川入直,並管理神機營。免武昌、黃州二衛額糧。壬寅,詔舉宗室及旗、漢世職人才。

閏五月乙巳,命工部尚書福錕、理籓院尚書昆岡、左都御史錫珍、工部侍郎徐用儀、內閣學士廖壽恆並在總理各國事務衙門行走。丁未,命前提督劉銘傳督辦臺灣事務,錫珍、廖壽恆、陳寶琛、吳大澂往天津會商法約。庚戌,命太常卿徐樹銘勘獻縣新開橫河。法人犯觀音橋,潘鼎新擊敗之。辛亥,山東河堤工成。甲寅,以法使言和,調潘鼎新諸軍回諒山,岑毓英軍仍駐保勝。乙卯,自四月不雨,至於是日始雨。頒定蠲緩錢糧章程。庚申,思恩匪首莫夢弼伏誅。丙寅、法艦犯閩海。丁卯,諭曰:「法使延不議約,孤拔要求無理,我軍當嚴陣以待。彼如犯我,並力擊之。敢退縮者,立置軍法。」庚午,授曾國荃為全權大臣,與法使於上海議約,命陳寶琛會辦。

六月癸酉朔,以鄖西餘瓊芳獄事讞不實,下總督卞寶第、巡撫彭祖賢部議,承審各官貶斥有差。甲戌,河決歷城等縣。以乞援守城,追予沈葆楨妻林氏附祀廣信葆楨專祠。丙子,建昌、多倫匪首楊長清伏誅。丁丑,吳長慶卒,旌其子主事保初孝行。己卯,諭直省考察州縣官。壬辰,法人陷基隆。詔集廷臣議和戰。乙未,劉銘傳復基隆。己亥,懿旨,神機營選馬步軍三千,巡捕五營選練軍二千,以都統善慶為總統,前鋒統領托倫布為幫統,分防畿東,並抽調直隸練軍協守。命曾國荃、陳寶琛回江寧布防。是月,賑順德、青浦風災,葉縣水災。

秋七月乙巳,命吳元炳勘山東河工、海防。授張之洞兩廣總督。丙午,法人襲馬尾砲臺及船廠,陸軍擊退之。戊申,醇親王奏延煦劾左宗棠,斥為蔑禮不臣,肆口妄陳,任情顛倒。懿旨坐延煦奪職留任,罰俸一年。詔與法人宣戰,楊昌濬赴福建督師。癸丑,法人毀長門砲臺。丁巳,諭穆圖善、張佩綸毋退駐省城。詔授左宗棠為欽差大臣,督辦福建軍務,福州將軍穆圖善、漕運總督楊昌濬副之,張佩綸以會辦大臣兼署船政大臣。授曾國荃兩江總督,兼南洋大臣。丙寅,論北寧失守罪,已革道員趙沃、提督陳朝綱並論斬。戊辰,以楊昌濬為閩浙總督。普賑歷城等縣災民。是月,賑浮梁及齊河、長安等處水災。

八月壬申,命鴻臚寺卿鄧承修在總理各國事務衙門行走。論馬尾戰事功罪,褫何璟職及張佩綸卿銜,下部議,提督黃超群等頒賞進秩有差。建、邵匪首張廷源等伏誅。甲戌,河決東明。賑南海等縣水災。丙子,授李鴻章直隸總督、北洋大臣。戊寅,懿旨賞醇親王子載灃不入八分輔國公。文煜以病免。命崇厚、崇禮、文錫、文金舌輸財助饟。庚辰,賑臺灣風災。丁亥,法人復陷基隆。戊子,命道員徐承祖充出使日本大臣。己丑,詔刑部本年情重各案及秋、朝審官犯,並停查辦。癸巳,蘇元春及法人戰於陸岸,敗之。命楊岳斌幫辦左宗棠軍務。賑星子水災。戊戌,法人犯滬尾,提督孫開華擊敗之。

九月癸卯,逮唐炯下獄廷訊。乙巳,出帑金五萬賚劉永福軍。辛亥,嚴諭南北洋輪船悉援臺灣。壬子,劉銘傳為福建巡撫,駐臺灣督防,蘇元春幫辦潘鼎新軍務,楊昌濬等分防澎湖,張兆棟、何如璋並褫職。詔免雲南田稅,暫荒緩三年,永荒蠲十年。甲寅,劉銘傳自請治罪,詔原之。戊午,留新漕十萬備山東冬賑。庚申,以滬尾戰勝,予總兵孫開華世職,發帑銀一萬犒軍。授額勒和布體仁閣大學士。乙丑,以刑部尚書恩承協辦大學士。丙寅,賑鳳凰城潦災。庚午,官軍及法人戰於陸岸,又敗之,予蘇元春世職。辛未,新疆改建行省,置巡撫、布政使各一,裁南北路都統、參贊、辦事、領隊諸職。

冬十月壬申朔,懿旨晉封奕劻慶郡王,奕謨固山貝子。癸酉,以劉錦棠為甘肅新疆巡撫。戊寅,賑江北等處水災雹災。辛巳,皇太后五旬聖壽,上率王以下文武大臣等詣慈寧宮慶賀。辛卯,鮑超屢誤師期,切責之。癸巳,以託疾規避,奪提督王洪順職。甲午,張樹聲卒。乙未,朝鮮復亂,吳大澂往察辦,續昌副之。文煜卒。庚子,劉永福及法人戰於宣光,敗績。

十一月丁未,命提督孫開華幫辦臺灣軍務。戊申,逮徐延旭下獄廷訊。壬子,李鴻章調軍發朝鮮。癸丑,普洱地震。丙辰,禁州縣捏報災荒。丁巳,東明決口合龍。戊午,李秉衡赴龍州部署防軍。己未,祈雪。雲南巴蠻降。戊辰,諭各省積穀。

十二月戊寅,官軍敗法人於紙作社。壬午,唐炯、徐延旭並論斬。乙酉,官軍復宣光、興化、山西三省,安平府暨二州五縣。壬辰,祿勸夷匪平。丙申,雨雪。張佩綸、何如璋並褫職遣戍。

是歲,免鎮西荒地逋賦,文安四州縣澇地額賦。朝鮮入貢。越南國王阮膺登自殺,法人立其弟為國王。

十一年乙酉春正月癸卯,命馮子材襄辦廣西關外軍務。乙巳,法人陷諒山。丙午,官軍圍宣光,復美良城。甲寅,法人犯鎮南關,總兵楊玉科死之。乙卯,賜英將戈登恤金。甲子,法艦去臺灣。左宗棠等兵援浙江。乙丑,命李鴻章為全權大臣,偕吳大澂與日使議朝鮮事。庚午,朝鮮亂平,使來表謝,賚之。

二月甲戌,浙江提督歐陽利見敗法人於鎮海口。戊寅,褫潘鼎新職,以李秉衡署廣西巡撫,蘇元春督辦廣西軍務。辛巳,秦州地震。癸未,馮子材、王孝祺大敗法人於鎮南關外,遂復諒山。予楊玉科等世職。辛卯,法人請和。允之。壬辰,詔停戰撤兵。緬匪平。戊戌,岑毓英奏官軍大捷於臨洮。

三月乙巳,命李鴻章為全權大臣,與法使議約,刑部尚書錫珍、鴻臚卿鄧承修往津會商。丙午,朝鮮訂約成。庚戌,岑毓英復緬旺與清水、清山諸寨,獲越南叛臣黃協等誅之。癸丑,命吳大澂、依克唐阿會勘吉林東界。丙辰,免永平、張家口、順天等十府州積年民欠租賦。癸亥,命馮子材督辦欽、廉防務。乙丑,免陜西咸寧等處前歲逋糧。

夏四月己卯,祈雨。丙戌,趣岑毓英撤軍,毋爽約開釁。辛卯,諭除江西丁漕積弊。壬辰,趣劉永福撤回保勝軍。天津會訂中法新約成。

五月丁未,懿旨勘修南北海工程。詔整海軍,大治水師,下南北洋大臣等籌議。基隆法兵退,命楊岳斌等部署全臺事宜。除福建光緒初年逋賦。辛亥,許乍丫隨察木多入貢。癸丑,予蘇元春、馮子材三等輕車都尉,王孝祺、岑毓英雲騎尉,復王德榜原官優敘。辛酉,復祈雨。壬戌,雨。丁卯,以張曜為廣西巡撫。是月,賑基隆兵災、桐城等縣及鎮筸水災。

六月己巳,詔停秋決。庚午,懿旨命文金舌、崇禮、崇厚、文錫修建三海工程。許景澄兼出使比利時大臣。辛未,定內附越南民籍。甲戌,曾紀澤訂煙臺約成。丁丑,諭岑毓英察雲南銅礦。通諭曾國荃等勘東南各礦。賑裕州水災。癸未,命工部侍郎孫毓汶、順天府尹沈秉成、湖南按察使續昌均在總理各國事務衙門行走。召曾紀澤來京,命江西布政使劉瑞芬充出使英俄大臣,張廕桓充出使美日秘大臣。法兵去澎湖。命左宗棠等選將吏調輪船策應。辛卯,越南新約成,宣諭中外。詔誡建言諸臣挾私攻訐。追論御史吳峋劾閻敬銘、編修梁鼎芬劾李鴻章俱誣謗大臣,予嚴議。尋各降五級。甲午,授孫毓汶軍機大臣。是月,賑河南、廣東、廣西、江南、安徽、江西水災。

秋七月丁酉朔,設廣西南寧電線達雲南。己亥,懿旨發帑銀六萬賑兩廣水災。庚子,左宗棠連乞病,許之。丙辰,命周德潤往雲南,鄧承修往廣西,會同岑毓英、張凱嵩勘中、越界。壬戌,河決山東長清。甲子,開川、滇銅鐵礦。是月,賑黔陽、湘潭、輝縣、清江、當塗、汾陽等處水災。

八月丁卯朔,賑奉天水災。己巳,截漕糧十萬石充順直賑需。賑皋蘭等處雹災水災。乙亥,賑長沙等處水災。丁丑,山東歷城、章丘等處水,發帑五萬賑之。以水災故,停三海工作。李鴻章與法使議滇、粵陸路通商。戊寅,釋李正應歸朝鮮。辛巳,命蘇元春存撫越南入關流民。賑襄城水災。乙酉,左宗棠卒,贈太傅。辛卯,賑福建風災。

九月庚子,懿旨,醇親王總理海軍事務,奕劻、李鴻章會辦,漢軍都統善慶、兵部侍郎曾紀澤幫辦。改福建巡撫為臺灣巡撫,歸福建巡撫事於閩浙總督。英使來議印度、西藏通商。諭丁寶楨、色楞額等開導藏番毋生事。壬寅,靈桂卒。甲辰,裁伊犁參贊大臣,改設副都統二。裁塔爾巴哈臺滿洲領隊大臣,仍留額魯特領隊大臣。甲寅,賑賓川、恩安等處雹災。

冬十月丙寅朔,朝鮮王李熙以伏莽未除,來請鎮撫。李鴻章遣軍防衛之。戊辰,賑朝陽災。庚辰,截來年京餉銀五萬充山東冬賑。辛巳,命奕劻、許庚身與法使互換條約,劉瑞芬於英京互換煙臺條約,並議洋藥專條。丁亥,授穆圖善為欽差大臣,會同東三省將軍辦理練兵,節制副都統以下。甲午,撥年節宮用銀五萬賑給山東災區。嚴紫禁城門禁。

十一月壬寅,祈雪。乙巳,雲南地震。庚申,裁新疆各城回官。癸亥,懿旨,八旗都統釐剔旗營諸弊。授恩承體仁閣大學士,閻敬銘東閣大學士,戶部尚書福錕、刑部尚書張之萬並協辦大學士。以英人滅緬甸,嚴四川邊備。

十二月丙寅,續設三姓、黑龍江陸路電線。丙子,詔內務府禁止浮冒虛糜。己卯,趙莊決口合龍。

是冬,賑潮州、萬縣水災,臺灣風災。免永寧被水丁銀,浙江各州縣衛荒廢並新種地課。減文安、天津窪地糧賦。除徐溝、汾陽被水銀稅。

十二年丙戌春正月乙未朔,停筵宴。庚子,免湖北逋賦。辛丑,山東溞溝決口合龍。免臺灣舊欠供粟。癸卯,免奇臺被旱額賦。丙辰,命特爾慶阿等隨同穆圖善練兵。甲子,詔以謁陵,本年會試改三月十日入場。

二月乙丑朔,山東黃河南岸決。甲戌,張曜往勘何王莊決口。己卯,除漵浦積年被水額賦。戊子,設黑龍江綏化。辛卯,上奉皇太后謁東陵,免經過州縣稅糧十分之四。

三月乙未,謁諸陵。上詣定東陵。庚子,至自東陵。癸丑,賑廣寧災。是月,留山東新漕十萬石賑何王莊暨章丘、濟陽、惠民被水災民。

夏四月戊子,賜趙以炯等三百三十九人進士及第出身有差。是月,丘北地震及廣西州火,賑之。

五月庚子,臺灣生番歸化四百餘社,七萬餘人。賑臨潼等縣風雹災。壬寅,裁陽江鎮水師總兵,置北海鎮水陸總兵。改高州鎮陸路總兵為水陸總兵。

六月壬申,懿旨,欽天監於明年正月擇皇帝親政日期。甲戌,修復海鹽石塘。丙子,醇親王暨王大臣等合詞疏請皇太后仍訓政,不許。皇帝親政定於明年正月十五日舉行,命樞臣集議,整齊圜法。庚辰,醇親王暨禮親王等復申訓政之請,尚書錫珍、御史貴賢並以為言,懿旨勉從之。命醇親王仍措理諸務。

七月甲午,木邦土司請內附,卻之。丁酉,金順卒。辛丑,留江蘇漕米五萬石備順、保賑需。乙巳,錢法議定奏上。允行。甲寅,賑太原等縣水災。

八月壬戌,以色楞額為伊犁將軍。賑熱河水災。乙丑,禮親王暨廷臣請加上皇太后徽號,懿旨不許。丁卯,再撥江北漕米五萬石賑順天通州水災,並發帑金二萬散給災民,免陜西咸寧等處荒田逋賦。戊辰,以北運河決口漫溢,撥庫帑十萬充永平各府急賑,再發內帑二萬濟之。丙子,增設廣西太平歸順道,移提督駐龍州。增設柳慶鎮總兵駐柳州。庚辰,築懷柔白河漫口。乙酉,御史硃一新奏遇災修省,豫防宦寺流弊,言醇親王巡閱北洋,總管太監李蓮英隨往,恐蹈唐代監軍覆轍。懿旨命回奏。尋奏入,以執謬降主事。

九月辛卯朔,賑奉天、浙江水災。癸巳,賑甘肅雹災水災,留壩、南鄭水災。丁酉,以順直水災減各府各旗莊田租及其他租額。庚子,鮑超卒。乙巳,賑光山雹災。丙午,劉銘傳剿蘇魯馬那邦叛番。甲寅,賑上饒等縣水災。

十一月庚寅朔,壽張決口合龍。乙巳,宥徐延旭、唐炯罪,延旭戍新疆,炯戍雲南。丁未,命曾紀澤在總理各國事務衙門行走。庚戌,再撥京倉粟米三萬石備順天春賑。丙辰,冬至,祀天於圜丘,始親詣。除隆科城額賦。

十二月甲子,減安州、河間、隆平澇地糧賦。丁卯,祈雪。庚辰,懿旨再敕曾國荃等詳議兩江河道治法。丁亥,祫祭太廟。

是歲,朝鮮入貢。

十三年丁亥春正月己丑朔,停筵宴。辛丑,以親政遣官告天地、宗廟、社、稷,祈穀於上帝。癸卯,上始親政,頒詔天下,覃恩有差。壬子,撥江蘇漕米十萬石賑順直災民。懿旨購置機器於天津鼓鑄,一文以一錢為率,京、外毋得參差。

二月壬戌,雨雪。辛酉,責恭鏜嚴剿馬賊,整頓見有練軍。川、滇接修電線成。戊辰,祭大社、大稷。辛巳,賞唐炯巡撫銜,督辦雲南礦務。是月,懿旨醇親王以親王世襲罔替,朝廷大政事,仍備顧問。

三月己丑朔,上初詣奉先殿行禮。乙未,上奉皇太后謁西陵,免經過州縣額賦十分之三。己亥,謁陵。甲辰,至自西陵。辛巳,祀先農,親耕耤田,三推畢,加一推,自是歲以為常。甲寅,劉錦棠請解職省親就醫,不許;給假三月,在任調理。撥直籓庫帑八萬賑所屬饑民。除文安等處無糧地租。

夏四月戊午朔,享太廟。丁卯,命內閣侍讀學士林維源督辦臺灣鐵路及商務。己巳,祈雨。丙子,常雩,祀天於圜丘。

閏四月己酉,免江蘇各州縣衛逋賦逋課。壬子,賑昆明等縣水災。

五月戊午,夏至,祭地於方澤。己未,命前內閣學士洪鈞充出使俄德奧和大臣,大理卿劉瑞芬充出使英法義比大臣。癸未,賑隴州等處水災。甲申,雨。

六月丁亥朔,賑富陽各屬水災。乙巳,賑懷寧等縣水災。丁未,開州大辛莊河溢,灌山東境,截留新漕五萬石賑濮州等處災民。庚戌,賑羅田、石首水災。壬子,賑溫宿、烏什水災。癸丑,賑凌雲風雹災。

秋七月丙辰朔,日有食之。庚申,永定河、潮白河先後並溢。甲子,增設福建澎湖鎮總兵。乙丑,賑南陽等處水災。丁卯,除甘肅積年民欠銀糧暨雜賦。賑洮州等屬雹災。乙亥,增設雲南臨安開廣道。丁丑,黎匪平。辛巳,命道員黎庶昌充出使日本大臣。

八月戊子,祭大社、大稷。甲辰,沁河決。賑平彞水災。丙午,沔陽等州縣被水,留冬漕三萬石賑之。鄭州河決,南入於淮,褫河督成孚職,留任。己酉,撥京倉漕米五萬石賑順天通州各屬。截留京餉漕折銀三十萬賑河南。癸丑,懿旨發內帑銀十萬賑濟水災。

九月乙卯朔,免陜西各府州縣前歲逋賦。辛酉,以鄭州河決,豫留明年江北、江蘇河運米糧並運費充賑。辛未,準哷徵胡圖克圖入貢。乙亥,命薛允升赴河南察鄭工。丁丑,李鴻藻往河南會察河工。是月,賑武陟、安縣、雲陽、皖北水災,漢口、龍州水災,建水、通海雹災。

冬十月甲申朔,賑融縣火災。丁亥,馮子材以疾辭職,命留粵辦欽、廉防務。乞休,不許。己丑,賑惠、高、廉、雷、瓊、赤溪、陽江風災。己亥,穆圖善卒。賑給鄭州等災區貧民口糧。壬寅,以善慶為福州將軍,襄辦海軍事,並管神機營。免順直被水各州縣秋賦。乙巳,賑鎮西雪災。戊申,上侍皇太后臨視醇親王疾,自是頻視疾,至於明年七月有瘳。庚戌,賑長安等屬水災。癸丑,賑綏來霜災。

十一月乙卯,授定安欽差大臣,會同東三省將軍辦理練兵,節制副都統以下。辛酉,冬至,祀天於圜丘。壬戌,諭文碩訪呼畢勒罕,依制掣定。壬申,祈雪。

十二月丁亥,命李鴻藻督辦鄭州河工。己丑,除恩隆、百色等處荒田額賦。賑桂林等處火災。壬辰,免陽城等縣災緩稅租。丁酉,雨雪。戊戌,懿旨復閻敬銘、福錕、翁同龢、嵩申、孫詒經、景善、孫家鼐處分。庚子,以皖北被災,撥安徽漕折、蕪湖關常稅共銀十萬,備來年春賑。辛丑,置新疆伊塔道、伊犁府、霍爾果斯、塔城,設道、府、撫民通判、同知等官。壬寅,石屏、建水地震。己酉,撥山東冬漕五萬石備河南來年冬賑。

是歲,朝鮮入貢。

十四年戊子春正月癸丑朔,上親詣堂子行禮。丙辰,雪。免安徽被淹太和等州縣夏糧。己未,開黑龍江漠河金礦。庚申,開廣東昌化石綠銅礦。辛酉,展接騰越至雲南省垣電線。乙亥,劉錦棠乞疾。慰留,再給假四月。壬午,諭官鑄當十大錢,每文重至二錢以上者,一律行用。是月,撥留京倉及海運漕米凡十三萬石賑順直災。

二月乙酉,賑梧州火災。丙戌,賞裴廕森三品京堂,督辦福建船政。庚寅,文碩以擅行密疏稿於都察院,褫職。辛亥,祀先農,親耕耤田。是月,詔修葺頤和園,備皇太后臨幸。

三月丙辰,免浙江光緒五年以前逋賦。丙寅,賚班禪額爾德尼轉世呼畢勒罕哈達、念珠、如意。

夏四月庚寅,永定河決口合龍。辛卯,上奉皇太后始幸西苑。甲午,展接廣東電線自九江至大庾嶺。丁酉,雨雹。辛亥,命張曜幫辦海軍事務。賑惠州等屬水災。

五月乙卯,京師、奉天、山東地震。癸亥,夏至,祀地於方澤。丁卯,祈雨。

六月癸巳,雨。己亥,懿旨,皇帝大婚典禮,明年正月舉行。甲辰,彭玉麟以疾免兵部尚書,巡閱長江水師如故。壬寅,懿旨,明年二月初三日歸政。

七月庚申,以河工貽誤,褫李鴻藻、倪文蔚職,仍留任,李鶴年、成孚並戍軍臺。甲子,永定河復決。丙寅,閻敬銘罷。丁丑,諭吳大澂察覈河工。是月,津沽鐵路成。

八月丁亥,賑奉天各州縣,安徽懷寧等縣水災。己丑,詔直省清庶獄。壬辰,賑蒼梧等處水災。丁酉,截留江北漕米備蘇、皖賑。乙巳,醇親王以歸政有日,請解職務。懿旨,海軍署、神機營依前管理,歸政後奏事勿列銜。

九月丙辰,除陜西去年逋賦。甲戌,永定河決口合龍。

冬十月己卯朔,享太廟。癸未,懿旨立葉赫那拉氏為皇后。癸巳,撥京漕二萬石備順天冬賑。甲午,免水城等處丁糧。賑丹徒旱災、南昌等縣水災。庚子,免朝鮮紅葠釐稅。

十一月壬戌,滇越邊界聯接中法電線成。初置北洋海軍提督,以丁汝昌任之。丙寅,冬至,祀天於圜丘。丁卯,免朝賀。戊辰,免靜海積水澱地租。

十二月壬午,賑阿迷、蒙自等處疫災。乙酉,詔光緒十五年舉行恩科鄉試,十六年恩科會試。辛卯,增設吉林水師營總管各官。癸巳,太和門災。甲午,詔修省,敕臣工勤職。乙未,免陜西前歲民欠錢糧。丁酉,懿旨,以水災停減頤和園工作。御史餘聯沅、屠仁守、洪良品各疏請罷鐵路,徐會灃等條奏,同下海軍署與軍機大臣議。旋翁同龢、奎潤、游百川、文治等並言鐵路不當修,亦並下議。命太僕少卿林維垣襄辦臺灣開墾撫番事。庚子,賑威遠水災。辛丑,道員徐承祖前使日本,坐浮冒,褫職聽勘,籍其家。丙午,鄭州決口合龍。授吳大澂河東河道總督,復李鴻藻、倪文蔚原官,並優敘,釋成孚、李鶴年還。

十五年己丑春正月丁未朔,停筵宴。庚申,靖遠、皋蘭地震。辛酉,以張之萬為東閣大學士,徐桐以吏部尚書協辦大學士。海軍署會同軍機議駮停鐵路諸疏,覆請詳議。懿旨:「慶裕、定安、曾國荃、張之洞、黃彭年等,按切時勢,各抒所見以聞。」乙丑,惇親王薨。上奉皇太后臨奠。丁卯,御史屠仁守上言:「歸政伊邇,時事孔殷,密摺封奏,請仍書皇太后聖鑒,披覽後施行。」懿旨斥其乖謬,罷御史,下部議,原摺擲還。戊辰,御史林紹年請禁督撫報效。懿旨斥之。癸酉,大婚禮成,

二月戊寅,吳大澂請敕議尊崇醇親王典禮,懿旨斥之,通諭中外臣民。己卯,皇太后歸政。上御太和殿受賀,頒詔天下。丙戌,免江、淮光緒初年災熟各項稅糧。己丑,以齊東等州縣水災,撥山東庫帑五萬備賑。壬辰,加上皇太后徽號,頒詔覃恩有差。甲午,朝鮮慶賀歸政,進方物,賚其國王及王妃緞匹。

三月丙午朔,命侍講崔國因充出使美日秘大臣。丁未,彭玉麟辭巡閱職。溫諭慰留。濮州河決。癸丑,以布魯克巴部長歸化,予封號印敕。甲寅,撥黑龍江庫帑二萬加賑呼蘭屬災民。丁巳,皇后祀先蠶。己未,再加上皇太后徽號。庚午,免雲南被匪村寨錢糧。戊辰,上奉皇太后幸頤和園,閱水陸操。允閻敬銘回籍養病。

夏四月戊寅,撥南漕十萬石備山東賑。己卯,賑奉天、吉林災民。辛卯,賞湖南按察使薛福成三品京堂,充出使英法義比大臣。懿旨發內帑銀十萬備山東賑。庚子,賜張建勛等三百三十一人進士及第出身有差。

五月癸丑,停秋決。庚申,賑瀘州火災。

六月丙子,岑毓英卒。丁丑,以王文韶為雲貴總督。己卯,重修太和門。丁亥,賑周家口火災。壬辰,永定河道缺,李鴻章舉堪任之員。上疑於魁柄下移,予申斥。

秋七月丁未,章丘河決。己酉,除貴州西良山額課。庚午,齊河決。辛未,沁河決。是月,賑莒州、沂水雹災,周家口水災,長安、西鄉、鄜州水災雹災。賑雲南昆陽、太和,安徽霍丘等州縣水災。

八月乙亥,命李鴻章、張之洞會同海軍署籌辦蘆漢鐵路。丁亥,留新漕十萬石備山東賑。壬辰,以四川水災,捐款五萬賑災民。丁酉,天壇祈年殿災。庚子,賑伊犁、綏定等處地震災。辛丑,免貴州被賊府州縣衛未徵並民欠稅糧。

九月壬子,重修祈年殿。賑溫州等處風災水災。癸丑,免陜西各屬前歲逋賦。賑咸寧等處水災雹災。乙卯,賑皋蘭等處水災。壬辰,長垣堤決,黃水浸入滑縣。丙寅,諭定安等除東三省練兵弊習。丁卯,定明年祈穀暫於圜丘舉行。

冬十月乙亥,賑陽曲等處雹災水災。戊寅,設西安至嘉峪關電線。賑杭、嘉、湖屬水災。丁亥,以江、浙雨水為災,各撥庫儲五萬,並發內帑五萬賑濟。以張之洞訂購機器,遽立契約,詔切責之,嗣後凡創設之事,未先奏明,毋輕舉。己丑,撥武昌庫儲十萬備湖北賑需。壬辰,詔各省兩司仍專摺奏事。臺灣社番亂,副將劉朝帶等陣沒,敕劉銘傳剿辦之。甲午,再撥浙江庫儲十五萬賑杭、嘉、湖災。己亥,山東大寨河工合龍。壬寅,撥蘇、皖賑捐餘款修運河。賑綏德等屬雹災水災。

十一月丙辰,允海軍署請,戶部歲撥二百萬開辦鐵路。丁巳,詔汰冗員,刪浮費。戊午,撥安徽漕折銀三萬備安慶、寧國、泗州賑需。丙寅,浙江發常平倉穀賑天臺、仙居等處難民。

十二月壬申朔,免杭、嘉、湖應徵漕白糧並地丁稅。甲戌,留山東漕米四萬石備賑。丁丑,再撥武昌庫儲五萬備湖北賑需。丁亥,山東西紙坊漫口合龍。癸巳,申禁辦理蠲緩積弊。免雲南匪擾村寨錢糧。丁酉,免鄭州、淮寧、尉氏等州縣稅糧。免仁和等縣,杭、嚴衛所糧課。

是歲,朝鮮入貢。

十六年庚寅春正月壬寅朔,停筵宴。辛酉,免直隸十三年以前灶課。丁卯,諭本年萬壽毋庸告祭,停升殿禮,免各省文武大員來京祝嘏。

二月乙酉,張曜言統核山東河工需費二百八十八萬有奇。命所司籌給。壬辰,臺灣內山番社酋有敏等伏誅。是月,免榆林等州縣十三年逋賦。除東川被水官田稅糧。免文安、靜海、霸州澱泊逋租及伯都訥地課。

閏二月壬寅,賑桂林各屬火災。己酉,命太僕寺卿張廕桓在總理各國事務衙門行走。乙卯,上奉皇太后謁東陵,免所過地額賦十分之三。庚申,上臨奠端慧皇太子園寢。癸亥,至自東陵。乙丑,曾紀澤卒,尋予特謚。諭李鴻章整頓北洋水陸軍,定安等訓練東三省兵。

三月辛未,懿旨,劉銘傳幫辦海軍事務。西寧地震,賑恤之。辛卯,以二旬萬壽,頒詔天下,覃恩有差。乙未,濬餘杭南湖。瞻對番目撒拉雍珠與巴宗喇嘛結野番作亂,官軍剿平之。

夏四月庚寅,彭玉麟卒。庚戌,諭整頓土藥稅釐。命剛毅詳察徐州土藥出產及徵稅實額,嚴定整理章程。丁卯,賜吳魯等三百三十六人進士及第出身有差。

五月己巳朔,日有食之。辛未,色楞額卒,以長庚為伊犁將軍。丙子,以升泰為駐藏大臣。己卯,上詣大高殿祈雨。乙酉,御畫舫齋閱侍衛步射,至壬辰皆如之。己丑,雨。築閿鄉沿河石壩。賑淮寧等縣風災。

六月己亥朔,徙齊東各州縣瀕河村民二千餘戶。丁未,開三姓金礦。戊申,以藏事平,頒給布魯克巴部長敕印。自癸卯至己酉連祈晴。辛亥,近畿霪雨成災,京師六門外增設粥廠,命撥京倉米萬五千石煮賑,並發內帑五萬充賑需。壬子,永定河決口。癸丑,永北屬土司章天錫謀逆,官軍討斬之。丁巳,撥奉天運京粟米,並留江北漕米,備天津災賑。甲子,萬壽節,禦乾清宮受賀。

秋七月乙亥,鎮康土族亂,剿平之。詔責李鴻章堵合永定河決口。己卯,發帑五萬兩,大錢五十萬貫,米十萬石,賑順天各屬災。壬午,諭嚴懲領放賑款侵冒剋扣。庚寅,分撥部庫及海關銀凡三十萬,濟永定河工。癸巳,命翰林院侍讀許景澄充出使俄德和奧大臣,道員李經方充出使日本大臣。賑湖北、廣西、陜西、雲南水災。

八月壬寅,再撥京倉米十萬石備順天賑需。乙巳,上詣醇親王邸視疾。己酉,劉錦棠乞歸。仍予假。壬子,以劉銘傳擅興商礦,章程紕繆,諭止之,予部議。丁巳,留漕米五萬石,撥庫帑十萬,備山東賑。壬戌,以順直水患,諭王公各府京旗莊田並減租。是月,免陜西、江西逋賦。賑陜西水災雹災,雲南水災,臺灣風災。

九月乙亥,戶部言祿米倉虧十五萬石,倉場侍郎興廉、游百川下部議,尋並奪職。丙子,賑琿春、寧古塔潦災。壬午,御史吳兆泰請停頤和園工程,予嚴議。永定河決口合龍。甲申,賑甘肅雹災。壬辰,石埭會匪亂,剿定之。癸巳,撥部帑及倉米於順天備賑。

冬十月丁未,以劉坤一為兩江總督兼南洋大臣。庚戌,曾國荃卒,贈太傅。辛亥,再撥京倉米五萬石備順天賑。免奉、直、魯、豫商販雜糧稅捐。

十一月乙亥,賑湖南被水州縣災。乙酉,上奉皇太后臨醇親王邸視疾。丁亥,醇親王薨,輟朝七日,上奉皇太后臨邸視殮,皇太后賜奠。命王子鎮國公載灃即日襲王爵。上成服,懿旨定稱號曰「皇帝本生考」,己丑,懿旨賜謚曰賢。皇帝持服一年。

十二月壬子,懿旨晉封輔國公載洵入八分鎮國公,鎮國將軍載濤不入八分輔國公。乙卯,醇賢親王金棺奉移園寓,上送至適園。壬戌,緩南苑工程。甲子,免浙江各州縣場光緒初年逋賦。

十七年辛卯春正月癸巳,四川雷波夷匪就撫。

二月癸卯,留海運漕米十六萬石備順直春撫。己巳,御史高燮曾請舉行日講。詔以有名無實,不納。辛亥,命李鴻章、張曜會閱北洋海軍。劉錦棠以憂去,以陶模為新疆巡撫。雲南匪亂,陷富民、祿勸縣城,討平之。是月,免湖北、山西十三年以前逋賦。

三月丁卯,諭資遣難民歸籍。己巳,皇后祀先蠶。壬申,修寶坻、通、薊諸州縣河工。丁丑,命李鴻章督修關東鐵路。庚寅,命沙克都林札布會額爾慶額勘察哈巴河。辛卯,劉銘傳以疾免。

夏四月丁酉,立醇賢親王廟。丙午,復建祠。辛酉,頤和園蕆工,上奉皇太后臨幸自此始。

五月丁卯至庚午連雨。辛未,皇后躬桑。壬午,賑清江等處風災。是月,京畿蝗。總署以各省教案迭出,請飭辦。諭曰:「各國傳教,載在條約,商民教士,各省當力衛其身家。乃者焚毀教堂,同時並起。顯有匪徒布謠生事,各督撫其緝治之,俾勿有所擾害。」

六月戊戌,諭嚴緝會匪。戊申,詔會匪自首與密報匪首因而緝獲者原免之。辛亥,王文韶奏誅附亂參將鮑虎。巧家披沙蠻酋祿汶芢伏誅,滇支夷二十一寨就撫。

秋七月癸未,以王文韶言雲南猛參、猛角、猛董土司劃界息爭。予孟定土知府罕忠邦宣撫使銜,土目罕榮高管理猛角、猛董,予土千總準世襲。乙酉,張曜卒。

八月壬辰朔,予樂亭耆儒史夢蘭四品卿銜。癸巳,命奕劻總理海軍事務,定安、劉坤一襄辦。己亥,世祖御制勸善要言譯漢書成,頒行直省學官,朔望與聖諭廣訓一體宣講。寶鋆卒。癸丑,諭疆吏飭營伍,除積習,嚴禁句結包庇。

九月癸未,免陜西前歲逋課。丙戌,初,與國來使,自同治十二年以來,皆見於紫光閣。是月,德使巴蘭德謂視如籓屬,屢以易地為言。至是,奧使畢格哩本來,遂於承光殿覲見。戊子,雲南北勝土州同改土歸流。

十月丁酉,免隰、榆次等處逋賦及旗租。癸丑,詔班禪額爾德尼呼畢勒罕明年正月坐床,升泰、蘇呼諾門罕往視,頒寄敕書珍物。甲寅,予宋儒游酢從祀文廟。戊午,熱河朝陽匪亂,提督葉志超、聶士成剿平之。

十一月丁卯,以熱河匪首擒戮,諭民間無論入會否,並許自新,其自拔來歸者宥之。乙亥,命戶部侍郎崇禮、兵部侍郎洪鈞並在總理各國事務衙門行走。己卯,海運倉火。甲申,以喀喇沁旗匪亂,撥庫帑三萬賑撫之。賑漢口火災。

十二月丙申,免河南光緒初年逋賦。乙巳,賑熱河被匪災區。戊申,申諭內務府撙節用費。

是冬,免浙江、陜西本年民欠稅糧。

十八年壬辰春正月丁亥,濬運河。辛卯,撥庫帑五萬於熱河,賑敖罕、奈曼兩旗蒙古。癸丑,英兵入坎巨提,回部頭目逃避色勒庫爾,賑撫之。

三月庚申,閻敬銘卒。

夏四月己酉,葬醇賢親王。是月,臺灣內山番社作亂,剿平之。

五月甲子,陽江匪亂,首逆譚運青伏誅。庚午,祈雨。辛未,賜劉福姚等三百一十七人進士及第出身有差。乙亥,合肥等州縣旱蝗,賑之。是月,上林、賓州匪首莫自閑等伏誅。

六月庚寅,祈雨。丙申,雨。壬寅,命編修汪鳳藻充出使日本大臣。

閏六月己未,永定河決。庚申,賑汾州及歸綏七旱災。甲子,留江蘇江北河漕各五萬石於順直備賑。丙寅,阿克達春以奏對失辭,罷山西巡撫。丁丑,以近畿水災,撥部帑十萬備賑。庚辰,恩承卒。是月,京畿蝗。

秋七月辛丑,發庫帑十萬備云南各屬賑。壬寅,河南蝗。癸丑,諭唐炯整頓銅運。

八月丙寅,命奎煥與英使保爾議印藏商約。甲寅,命福錕為體仁閣大學士,麟書協辦大學士。留山東新漕備賑。

九月庚寅,撥江北漕米五萬石備鎮江各屬賑。己亥,福建德化匪首陳拱伏誅。壬寅,免陜西前歲民欠錢糧。

十月乙卯朔,留江南漕米三萬石備江寧諸縣賑。庚申,醴陵匪首鄧海山伏誅。己巳,賑莎車水災。免直隸通州等處糧租雜課。

十一月乙酉朔,免直隸通州等處逋賦。辛卯,賑臺灣等處潦災。辛丑,諭李鴻章、孫家鼐等察賑,被災州縣有玩視民瘼者,嚴劾以聞。壬寅,免江蘇各州縣衛逋賦。庚戌,發庫帑十萬賑太原等屬水旱霜雹災。癸丑,發內帑二萬賑順直各屬災民。

十二月乙卯朔,詔王大臣承辦皇太后六旬慶典,會同戶、禮、工部,內務府博稽舊典,詳議以聞。丙寅,召劉錦棠來京。丁卯,再發京倉米四萬石,賑順天災民。乙巳,懿旨,辦理慶典,一切撙節,內外臣工例貢免進獻。特頒內帑賑濟順直災區,每歲準此,畀順天府、直隸總督永濟窮黎。每省各賞銀二萬,自明年甲午始,俱發內帑畀各省疆吏散給之。諭已故貝勒那爾蘇為僧格林沁孫,惓念前勞,追封親王,後不得援例。丙子,賞徽寧池太廣道楊儒四品京堂,充出使美日祕大臣。

是歲,朝鮮入貢。

十九年癸巳春正月乙酉朔,詔以明歲皇太后六旬聖壽,今年舉行恩科鄉試,翌年舉行甲午恩科會試。丙戌,免長洲等州縣冬漕米石。己亥,免長沙等州縣逋賦。甲辰,詔明年應來京祝嘏蒙古與內札薩克王、公、臺吉等,除有年班外,俱止來京。癸丑,以口外七及大同等府災,命直、晉免收運商糧稅,撥部帑十萬賑之。

二月戊午,留江蘇漕米五萬石備賑安州等處。戊辰,見德使巴蘭德於承光殿。癸酉,留京餉五萬賑陜西北山等處災民。

三月辛卯,命以兩湖漕米六萬餘石變價賑山西災。

夏四月丙子,祈晴。己卯,以阿拉善札薩克和碩親王多羅特色楞游牧連年荒旱,頒帑三萬賑之。

五月乙酉,北新倉火。乙未,以伊克昭盟長札薩克固山貝子札那吉爾第游牧連年荒旱,頒帑一萬賑之。

六月乙卯,命直省擇保精曉天文、醫理、卜筮、數學及嫻於堪輿者,上之內務府。戊午,撥部帑三萬備賑醴陵等處災。庚申,見德使紳珂於承光殿。癸亥,祈晴。丁卯,普安匪首劉燕飛等伏誅。癸酉,京師雨災,詔於六門外等六處各設粥廠,撥京倉米萬石充賑。乙亥,再撥奉天粟米、江南北漕米備順直賑需。永定河決,南北汛並溢。丙子,免安徽積年逋賦,暨潛山等縣衛前欠夏糧。

秋七月甲申,諭順天府平糶。甲辰,近畿積潦漸消,諭遣就食貧民歸籍。

八月辛亥,賜故總督曾國荃孫廣漢四五品京堂。除華僑海禁,自今商民在外洋,無問久暫,概許回國治生置業,其經商出洋亦聽之。丁卯,採購奉、豫、魯省雜糧分備順直賑。

九月癸未,山東截留新漕六萬石賑瀕河州縣災民。再撥江南北漕米十萬石改折,復留江蘇漕米八萬石充賑順直,分半給之。癸卯,發京倉米三萬石賑順天。是月,免陜西各屬逋賦及額賦。

冬十月己酉朔,修太倉四州縣海塘。壬子,賞四川布政使龔照瑗三品京堂,充出使英法義比大臣。己未,命戶部歲納內務府銀五十萬。乙丑,免通州等處糧賦。

十一月己丑,申私錢之禁,有銷毀改鑄或載運者,所司訪緝嚴治之。戊子,甘肅、新疆地震。辛卯,命許振禕與李鴻章會勘永定河。甲午,免大興等縣秋稅。

十二月辛亥,命吏部侍郎徐用儀在軍機大臣上學習行走。壬子,詔京察嚴考覈。戊午,除內地人民出海禁。辛酉,賑安仁疫災。壬戌,免歸化等七租賦。丁卯,免烏拉捕東珠。壬申,撥京東倉米五萬石備順天春賑。癸酉,刑部奏革員周福清於考官途次函通關節,擬杖流,改斬監候。

二十年甲午春正月己卯朔,懿旨,六旬慶辰,晉封妃嬪名號,增恭親王護衛,奕劻晉封親王,醇親王載灃等賞賚有差。自中外大臣、文武大員、蒙古王公等以次恩錫。丙申,許振禕會勘永定河工程,命與李鴻章會籌。允歲增修費四萬,並撥部帑三十萬充經費。己亥,庫車地震。免鎮、迪各屬逋賦。庚子,重申科場禁例。辛丑,免鄂倫春貢貂。壬寅,滇緬續約成。

二月辛亥,詔殿廷考試閱卷大臣公慎校取勿濫。濬通惠河,築閘壩。甲子,命李鴻章閱海軍。甲戌,禁州縣非時預徵及濫用非刑。允許振禕請,盧溝橋置河防局,仿裘曰修成法,設浚船百二十艘。

三月戊寅朔,日有食之。諭疆吏毋濫保屬官。戊子,詔停秋決。

是春,免新疆各屬逋賦,雲南各屬額賦雜課。

夏四月戊申,韶州南雄匪亂,剿平之。己酉,漵浦匪首諶北海伏誅。甲寅,大考翰、詹,擢文廷式等六人一等,餘升黜有差。辛酉,見義使巴爾迪等於承光殿。辛未,賜張謇等三百十一人進士及第出身有差。壬申,諭直省清理京控積案。

五月丁亥,以畿輔多盜,諭嚴捕務。戊子,詔駐藏辦事大臣、幫辦大臣三年任滿得請覲,著為令。丁酉,初,朝鮮以匪亂乞師,李鴻章檄提督葉志超、總兵聶士成統兵往。上慮兵力不足,因諭綏靖籓服,宜圖萬全,尚須增調續發,以期必勝。壬寅,除免江蘇海運漂沒漕糧。乙巳,召劉銘傳來京。裁鄂倫春總管,升布特哈總管為副都統。

六月己酉,詔停道、府捐。癸丑,京師霪雨,祈晴。乙卯,見日使小村壽太郎於承光殿。戊午,命翁同龢、李鴻藻與軍機、總署集議朝鮮事。壬戌,停海軍報效。乙丑,諭:「湖南京漕折價,備順天賑。向有濟荒經費,亦報解存儲。」皆自今歲始,歲以為常。丁卯,命南澳鎮總兵劉永福赴臺灣。戊辰,召劉錦棠來京。辛未,上二旬萬壽,御殿受賀筵宴。命徐用儀為軍機大臣。壬申,召免出使日本大臣汪鳳藻回國。

秋七月乙亥朔,日本侵朝鮮,下詔宣戰。戊寅,命李瀚章毀南海舉人康祖詒所著書。己卯,諭遣道員袁世凱往平壤撫輯。丙辰,命臺灣布政使唐景崧、南澳鎮總兵劉永福助邵友濂籌防。辛巳,諭李鴻章擴充海軍,慎選將才,精求訓練,通籌熟計以聞。乙酉,免賓川等州縣田租。丙戌,敕神機營兵防近畿,駐通州,旋移南苑。戊子,命端郡王載漪、敬信練旗兵,以滿洲火器營、健銳營、圓明園八旗槍營暨漢軍槍隊充選。載漪尋管神機營。諭停不急工程。允吳大澂請,統湘軍赴朝鮮督戰。丁酉,賑會同、會樂二縣災。己亥,命葉志超總統駐平壤諸軍。敬信、汪鳴鑾均在總理各國事務衙門行走。癸卯,重訂中外保護華工約。

八月丙午,吳大澂督軍出關,自請幫辦海軍,不許。丁未,始釋奠於先師。己酉,劉錦棠卒。戊午,上皇太后徽號,頒詔覃恩有差。壬戌,李鴻章以師久無功,褫三眼孔雀翎、黃馬褂。丙寅,懿旨發內帑三百萬備軍需。命四川提督宋慶幫辦北洋軍務。丁卯,命承恩公桂祥統率馬步各營往駐山海關。戊辰,奉天援軍統領高州鎮總兵左寶貴及日人戰於平壤,敗績,死之。己巳,命吳大澂軍駐樂亭。庚午,懿旨,六旬慶辰停止頤和園受賀。撥京倉米三萬石賑順天各屬水災。

九月甲戌朔,懿旨起恭親王奕直內廷,管總署、海軍署事,並會同措理軍務。乙亥,命宋慶節制直、奉諸軍。罷葉志超總統。丁丑,諭在籍提督曹克忠募津勇駐山海關。召王文韶來京。調黃少春為長江水師提督。庚辰,命兵部侍郎王文錦等辦理團練。辛巳,免陜西咸寧等處旱荒田賦。壬午,海軍副將鄧世昌及日人戰於大東溝,死之。癸未,召張之洞來京。丁亥,賑瑞昌等縣潦災。戊子,以臨敵潰散,罷葉志超、衛汝貴統領,以聶士成統兩軍。庚子,日兵渡鴨綠江。辛丑,陷九連城。壬寅,命長順率吉林軍往奉天助剿,豐紳統三省練軍防東邊。

冬十月甲辰朔,諭裕祿飭金州戰備。乙巳,命提督唐仁廉募勇二十營,會定安、裕祿防剿。丁未,詔山西各省入衛。戊申,詔恭親王督辦軍務,各路統帥聽節制。命王大臣等分辦巡防、團防,廣西按察使胡燏棻駐天津督糧饟,許專奏。召劉坤一來京,以張之洞署兩江總督兼南洋大臣。寧夏鎮總兵衛汝貴以臨敵退縮,褫職逮問。己酉,命翁同龢、李鴻藻、剛毅並為軍機大臣。壬子,日人陷金州,副都統連順棄城遁。徐邦道及日人戰,敗績。丙辰,賑東省瀕河貧民,並撥帑撫恤。丁酉,各國使臣呈遞國書,賀皇太后六旬萬壽,上見之於文華殿。壬戌,日人陷岫巖州,豐升阿、聶桂林皆棄城走。額勒和布、張之萬罷軍機。定安以臨敵無功,奪欽差大臣、漢軍都統,暫留辦東三省練兵。依克唐阿以督兵畏葸褫職,戴罪圖功。丁卯,日人襲旅順船塢,總辦龔照興遁煙臺,黃仕林、趙懷業、衛汝成繼之,徐邦道與張光前、姜桂題、程允和奔復州依宋慶。諭李秉衡嚴防威海。吳大澂請自任山海關防務,並俟各軍會合,規復朝鮮。諭曰:「臨事而懼,古有明訓。切勿掉以輕心,致他日言行不相顧。」以旅順失守,責李鴻章調度乖方,褫職留任。壬申,奪丁汝昌海軍提督,暫留任。宋慶自請治罪,特原之。詔各路將帥嚴約束,禁擾累民間,犯者立正軍法。褫葉志超職。

十一月癸酉朔,褫龔照興職,尋逮問。己卯,以金州陷,褫副都統連順職,程之偉並褫職,趙懷業逮京治罪。庚辰,懿旨恭親王奕復為軍機大臣。辛巳,免順直被水州縣額賦。丙戌,日本陷復州。戊子,日本兵集金、復二州。諭宋慶率諸軍決戰。豐升阿、聶桂林自岫巖奔析木城,聞敵至,師復潰,日人取析木城。以程文炳為陸路提督。己丑,宋慶及日人戰於海城,敗績,退保田莊臺。庚寅,依克唐阿及日人戰於鳳凰城,侍衛永山死之。命榮祿在總理各國事務衙門行走。壬辰,豐升阿、聶桂林逮問。癸巳,逮葉志超、丁汝昌治罪。戊戌,褫提督程允和、張光前、總兵姜桂題職,俱留營效力。

十二月癸卯,停是月紫光閣、保和殿筵宴。褫提督衛汝成職,逮問。甲辰,御史安維峻以論李鴻章,坐妄言褫職,戍軍臺。命劉坤一為欽差大臣,關內外各軍均歸節制。褫提督黃仕林職,逮問。壬子,命張廕桓、邵友濂以全權大臣往日本議和,尋召還。丙辰,撥江蘇漕米十二萬石備順直春賑。丁巳,章高元及日人戰於蓋平,敗績。奉軍復戰,提督楊壽山死之,城陷。辛酉,懿旨,劉坤一駐山海關籌進止。趣吳大澂率師出關,會宋慶進剿。以近畿米貴,運豫、魯雜糧平糶。癸亥,衛汝貴處斬。甲子,命宋慶、吳大澂襄辦劉坤一軍務。乙丑,再撥京倉米三萬石備順天春賑。己卯,日本陷榮成。庚午,命王文韶襄辦北洋軍務。

是歲,朝鮮入貢。


\end{pinyinscope}