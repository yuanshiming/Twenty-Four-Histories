\article{志一百九}

\begin{pinyinscope}
○兵五

△土兵

土兵惟川、甘、湖廣、雲、貴有之,調征西南,常得其用。康熙間,莽依圖戰馬寶於韶嶺,瑤兵為後援。傅弘烈平廣西,亦藉土兵義勇之力。乾隆徵廓爾喀,調金川土兵五千,討安南,以土兵隨征。傅恆征金川,疏言:「奮勇摧敵,固仗八旗。鄉導必用土兵,小金川土兵尤驍勇善戰。」岳鍾琪平西藏,咸、同間討黔、蜀發匪,其明效也。

古西南夷多槃瓠遺種,曰僚、曰伶、曰、曰僮、曰瑤、曰苗。其後蕃衍,有西番、僰人、擺夷、麼些、佫人鹿、咱哩、惈、俍、瑤等目。苗蠻種類尤多,如花苗、紅苗、花仡佬、紅仡佬、白惈儸、黑惈儸皆是。土兵多出其中,故驍強可用。土兵之制,甘肅、四川、兩廣、湖南、雲、貴或隸土司,或屬土弁,或歸營汛。甘肅土兵附番部。四川土兵附屯弁、屯蕃。湖南土兵附練兵、屯兵。別有番民七十九族,分隸西寧、西藏。茲並述於篇。

甘肅土兵:

狄道州臨洮衛指揮僉事轄十五族。河州指揮僉事轄四十八戶。韓家集指揮使轄二族。

岷州宕昌城指揮使轄十六族。攢都溝外委百戶轄四十一莊。麻霙里外委土官轄二族。閭井外委百戶轄十一莊。歸安里副千戶轄土民四十八族,番民四十三族。

洮州卓泥堡指揮使兼護國禪師轄五百二十族,馬兵五百,步兵千五百,土守備一,千總、把總四,外委七。資卜指揮使轄七十六族,土守備一,千總、把總、外委各二,馬兵五十,步兵百。著遜百戶轄七族,兵十人。西寧縣寄彥才溝指揮使轄八族,土千總一,把總二,馬兵五十,步兵百。陳家臺指揮使轄一百十二戶,土千總、把總各一,馬兵五,步兵二十。乞塔城指揮使轄四十八族,土千總、把總各一,馬步兵各五十。納家莊指揮僉事轄百二十戶,土千總、把總各一,兵二十五人。西川海子溝指揮僉事轄番民十八戶,土民三十戶,土千總、把總,兵額同上。迭溝指揮僉事轄九十戶,土千總、把總,兵額同上。循化土千戶轄西鄉上四工韓姓撒拉。保安堡土千戶轄東鄉下四工馬姓撒拉。撒拉不同番回,似羌而奉回教,舊十三工,今隸循化八工,餘隸巴燕戎格。藏土百戶轄五族。

大通縣大通川土千戶轄五族。

碾伯縣勝番溝指揮同知轄七百戶,土千總二,把總四,馬步兵百。上川口指揮同知轄四千戶,土千戶一,百戶二,土千總四,把總六,馬步兵三百。趙家灣指揮同知轄百二十戶,土千總、把總各一,馬兵五,步兵二。白崖子指揮同知轄百五十戶,兵二十五。美都川指揮僉事轄三百戶,土千總、把總各一,兵二十五。硃家堡指揮僉事轄六十二戶,兵二十五。米拉溝指揮僉事轄七十戶,土千總、把總各一,馬步兵二十五。九家巷百戶轄百餘戶,兵二十五。王家堡百戶轄百餘戶,兵二十。喇守莊指揮僉事轄七十二戶。

莊浪掌印土司指揮使轄指揮僉事、指揮使、指揮同知、正副千戶各一,百戶二,土民十旗,番民八旗,文職隸甘涼道,武職隸西寧鎮。紅山堡掌土司印指揮僉事兵五十。古城及大營灣指揮使、大通峽口指揮同知、古城正千戶、馬軍堡副千戶、西坪正千戶、西六渠百戶均率親丁效力,不轄土民。

永昌縣流水溝寺千戶轄番民五旗。

甘肅番部:

狄道州三族,河州十八族,皆康熙時舊族,雜處二十四關內。

洮州八族,大小九十餘處,亦曰南番。土司楊積慶屬番民五百二十族。

昝天錫屬番民七十六族。楊永隆屬番民七族。著洛寺僧綱楊溯洛旺秀轄番民二十三族。麻人爾寺僧綱馬昂旺丹主轄番民二十一族。圓成寺僧正侯洛扎旦主轄番民四族。

岷州熟番四十三族,舊屬土司,後為歸安里,惟白水江以南、南山內外,皆黑番所在,亦稱若瓦。南山以東馬土司轄,以西楊土司轄,凡番寺三十五所,轄番民霙古喇哈等二十四族。

文縣番族五百族,番地二十二處。馬百戶番地二十八處,雍正八年,改番歸流曰新民。

西寧縣番民十三族,番寺三十八族。

貴德熟番舊五十四族,存五族,生番舊十九族,存五族,野番十九族,俱插帳河濱,番寺大者六所。

循化口內熟番十二族,口外西番四十九寨,口外南番二十一寨。

丹噶爾南鄉熟番一族,河南西番八族。

武威縣峽溝番民三族,沙溝一族,上下大水寺五族,南山八族。

鎮番縣八力曼插漢番民一族。

永昌縣番民五族。

古浪縣東山圍場溝番民四族。黃羊川五族。柏林溝二族。

平番縣熟番三十六族,舊十餘萬丁,同治間存千餘人,番寺十四所。洛洛城十三堡番民八族,二千三百餘丁。

張掖縣唐烏忒黑番三族,康熙間給首領劄銜。撫彞通判轄西喇古兒黃番五族,唐烏忒黑番三族,八族設正副頭目,給守備、千總職銜,番民俱充兵伍。

高臺縣唐烏忒黑番一族,每壯丁一,納馬一匹入營。西喇古兒黃番二族,隸紅崖營。

四川土兵:

松潘中營所屬土司七寨,土百戶二,千戶五。左營所屬土司二寨,土千戶、百戶各一。右營所屬土司一寨,土百戶一。漳臘營所屬土司五十二寨,土千戶十四,百戶二十五,土目十三。平番營所屬土司二寨、一寺,土千戶三。南坪營所屬土司二寨,寨首二人。

茂州疊溪營所屬土司六寨,土千戶、百戶各一。

龍安府龍安營所屬土司隘口一,堡一,長官司一,土通判、知事各一。

雜穀維州協左營所屬土司宣慰司一,轄大小二十八寨。右營所屬土司宣慰司一,轄十九寨,長官司三,轄四十五寨。

茂州營所屬土司長官司一,副長官司一,安撫司、土巡檢各一。

懋功懋功協所屬土司,安撫司、宣撫司各一,轄大小四十六寨。

建昌鎮中營所屬土司,河東長官司一,土百戶三,土目十一,民戶皆惈儸部落。阿都正長官司一,轄土目四人,阿都副長官司一,轄土目十一,民戶皆苗夷。沙罵宣撫司所轄土目五十,民戶皆蠻夷。右所屬河西土千總一,土目四,民戶皆平夷。

越俊越俊營所屬土司,工⼙部宣撫司一,土目十一。寧越營所屬暖帶密土千戶一,轄鄉總七,土目一。暖帶田壩土千戶一。松林地土千戶一,轄土百戶五。以上民戶皆番夷。

鹽源縣會鹽營所屬土司,木里安撫司一。瓜別安撫司一。馬喇副長官司一。古柏樹土戶一,轄土目二。中所、左所轄土目一,右所土千戶各一。前所、後所土百戶各一。以上民戶皆番夷。

冕寧縣冕山營所屬土千戶、土百戶十三,土目四,村戶皆夷也。

會理州會川所屬營司土千戶三,土百戶四,民戶皆番也。永定營所屬土千戶一,村戶皆夷也。

打箭爐泰寧營所屬沈邊長官司一,冷邊長官司一,民戶皆番也。

天全州黎雅營所屬穆坪宣慰司一。

清溪縣黎雅營所屬土千戶一,土百戶二。

打箭爐阜和協所屬明正宣慰司一,土千戶一,土百戶四十八。革什咱布安撫司一。巴底宣慰司一。喇袞安撫司一。霍耳竹窩安撫司一,轄土千戶、百戶各一。章谷安撫司一,轄土百戶四。納林沖長官司一。瓦述色他長官司一。瓦述更平長官司一。瓦述保科安撫司一。以上戶皆土民,多少不等。

德耳格忒宣慰司一,轄土百戶六,民戶皆番。霍耳白利長官司一。霍耳咱安撫司一,轄土百戶二。霍耳東科長官司一。春科安撫司一,副土司一。上瞻對茹長官司一。峪納土千戶一。蒙葛結長官司一。林蔥安撫司一。上納奪安撫司一,轄土千戶一,百戶三。下瞻對安撫司一,轄土百戶二。上瞻對撤墩土千戶一。中瞻對茹色長官司一。以上戶皆土民。

上述土司,其中如春科等,有已納印者,清季設專官治之。三瞻曾畀西藏,為其轄境。其後邊釁屢生,宣統初收復。

里塘糧務所屬里塘宣撫司一,副土司一,轄長官司三,土百戶二,戶皆番民。

巴塘糧務所屬巴塘宣撫司一,副土司一,轄土百戶七,戶皆土民。

石砫夔州協所屬宣慰司一。乾隆間改土通判。

瀘州瀘州營所屬長官司一。

雷波普安營所屬土千總一,土舍二。安阜營所屬土舍一。屏山縣所屬長官司四。以上民戶皆番夷。

馬邊馬邊營所屬土千戶一,百戶九,土外委一。

峨邊歸化汛、冷磧汛所屬嶺夷十二地,夷人頭目十二。赤夷十三枝。膽巴家土千總、把總各一,轄頭目四。哈納家土千總、把總各一,轄頭目三。蜚瓜家千總一,把總二,轄頭目二。魁西家土千總、把總各一。以上民戶娃子為多。娃子者,漢人被掠入夷巢之名。

四川屯弁:

雜穀維州協所屬雜穀腦屯守備一,轄屯千總二,屯把總四,屯外委八。乾堡寨上孟董、下孟董、九子寨均屯守備一,轄千總、把總、外委十四。以上民戶皆番。

懋功懋功協所屬攢拉八角碉屯守備一,千總、把總、外委六。撫邊屯所屬屯把總一。攢拉漢牛屯守備一,千總、把總、外委六。撫邊屯所屬攢拉別思滿屯守備一,千總、把總、外委七,馬爾富屯外委一,曾頭溝千總一。章穀屯屬攢拉屯守備一,千總、把總、外委八,分轄宅壟屯把總一,外委四。崇化屯屬促浸河東屯守備一,千總、把總、外委十五。綏靖屯屬促浸河西屯守備一,千總、把總、外委二十四。以上戶皆屯番。

四川已廢土司:

建昌道所屬天全六番招討司、副招討司各一。大涼山阿都宣撫司一。建昌壩南路安撫司一。河西宣慰司一、土百戶四。審札等處土百戶三。北路甸沙關土千戶一。

冕山營所屬寧番安撫司一、土百戶二。皮羅木羅等處土百戶四,頭人三枝。靖遠營土百戶四,頭人四枝。涼山等處番夷頭人六枝。如昆等處頭人九枝。冕山營徵收土千戶及頭人二枝。

雅州府屬司徒一、大國師一。

打箭爐屬中瞻對長官司一。

川東道屬宣慰司、長官司各一。

松茂道屬雜穀土司一。

兩廣土兵:

廣東高州府茂名縣瑤兵六百六十四,俍兵六百六十六,轄瑤山四十四。電白縣僮兵百六十五,轄瑤山二十一。信宜縣瑤兵百七十七,俍兵五百九十五,轄瑤山四十一。化州瑤兵五百二十四,俍兵百九十四,轄瑤山五十一。石城縣瑤兵四百九十七,轄瑤山二。廉州府牛藤閘俍總一,兵四十六。馬頭閘俍目一,兵十五。水鳴閘俍目一,兵三十四。冷水閘俍目一,兵二十三。九叉閘俍目一,兵十四。沙尾閘俍目一,兵二十。藤柯閘俍目一,兵二十。丹竹閘俍目一,兵十九。樟木閘俍目一,兵三十。

廣西桂林府龍勝二堡,堡目各一。臨桂十三堡,堡目十三。靈川五堡一隘,堡目五,隘長一。永寧州二鎮,俍長二。永福十一堡,堡目十一。義寧五堡,堡目五。全州隘長六。以上各土兵,自二十四至二百九十二。灌陽俍兵最少,臨桂最多。

柳州府雒容土舍一,堡目三。羅城十五堡,堡目十五。柳城二十一堡,堡目二十一。融縣二堡,堡目二。以上土兵自十四至二百六十五。融縣最少,雒容最多。

慶遠府宜山堡目一。天河堡目一。河池州堡目一。思恩堡目一。東闌土州目一。永定土司一。永順正副土司各一。土兵自三十二人至百十人,惟那地土州兵二百八十,南丹土州兵五百十二。土州又各分兵五十屬德勝鎮。又忻城土縣兵三百,數為最多。

思恩府上林土舍、頭目、總練三十八,兵五百七十五為多。土田州兵四百,陽萬土州判兵三百次之。土上林縣兵三十,武緣堡兵五十為少。

平樂府恭城鳳皇堡隊長六。賀縣田總一,哨長三,隊長十四。荔浦堡目二。修仁堡目五。永安土舍二。以上土兵自六十五至三百十。荔浦最少,永安最多。

梧州府岑溪俍總俍目。懷集耕總、哨長、耕兵、撫兵。二縣兵皆逾三百。

潯州府桂平、平南、貴縣皆俍兵,武宣為土勇、土兵,自三、四十至三百十四不等。

南寧府宣化土勇,隆安隘兵。橫州俍兵,永淳俍兵、耕守兵。遷隆土侗兵。自三十至三百不等。

太平府龍州屬下龍土司、兩關、三卡、十四隘。明江屬上石西州兵。崇善兵,安平土州兵。萬承土州九甲兵,應調運糧,及六坊土兵。茗盈、全茗、龍英、佶倫、鎮遠、思陵等土州兵。土江州兵。土思州兵。下石西土州兵。上下凍土州兵。羅陽土縣兵。上龍土巡檢隘兵。以上兵四、五十至五百不等。餘如都結土州頭目三,兵十六為極少,土思州兵七百餘,太平土州兵千餘為最多。

鎮安府府額土兵。小鎮安土勇。天保兵。歸順州隘兵。湖潤寨隘目兵。都康、上映兩土州兵。下雷土州土勇。自三十至二百五十不等,惟向武土州土目二百二十,土兵額千二百,其最多者也。

鬱林州北流俍兵。陸川俍目、俍兵。興業俍兵。皆不過三、五十。

綜廣西土兵,蓋萬三千八百有奇。

湖南土兵:

湖南苗疆,鳳凰設中營、右營守備各官,苗兵二千,練兵千,屯兵四千。乾州設守備各官,苗兵八百,屯兵六百。永綏設守備各官,苗兵千八百,屯兵二千。永順縣設守備各官,苗兵、屯兵各百。保靖縣設守備各官,苗兵、屯兵各三百。嘉慶十年,設屯弁統屯丁,原有備戰練兵千人,準營制操習,著為例。

雲南土兵:

鎮遠,大雅口土都司各一。

麗江府,大山茨竹寨土守備各一。中甸迭巴土守備二。

鎮邊黃草嶺,杉木籠隘,六庫,阿敦子,猛遮,普寧縣普藤,維西奔子欄,元江州,雲龍州老窩,威遠猛戛,永北羊坪,保山縣登梗,魯掌,麗江府,新平縣斗門磨沙,大中甸神翁,小中甸神翁,中甸江邊神翁,中甸格沙神翁,中甸泥西神翁,鎮邊猛角猛董,圈糯千總各一。臨安府稿吾卡,漕澗,奔子欄,阿敦子,瀾滄江,臨城,其宗喇普,思茅倚邦,易武,猛獵,六順,猛籠,橄欖壩,猛旺,整董,他郎儒林裏,定南里,威遠猛戛,猛班,騰越大塘隘,明光隘,古勇隘,卯照,下猛引,賢官寨,募乃寨,東河,元江州永豐裏,茄革裏,喇博,他旦,老是達,巖旺,烏猛,烏得土把總各一。迭賓土把總五。中甸江邊,小中甸迭賓,中甸格咱,中甸泥西土把總各三。

鎮邊大山分防,猛弄掌寨,猛喇掌寨,水塘掌寨,五畝掌寨,五邦掌寨,者米掌寨,茨桶壩掌寨,馬龍掌寨,瓦遮、宗哈正掌寨,瓦遮副掌寨,宗哈副掌寨,鬥巖掌寨,阿土掌寨、土外委各一。賓川州赤谷裏,保山縣練地,武定州勒品甸土巡捕各一。

止那隘,猛豹隘,壩竹隘,黃草嶺隘撫夷各一。八關撫夷。銅壁關、萬仞關、神護關、巨石關、鐵壁關正副撫夷,各有努練土兵,自二十五、六戶至百五十餘戶。虎踞關、天馬關、漢龍關正副撫夷。

貴州土兵:

貴陽府屬中曹長官司,養龍長官司,白納長官司、副長官司,虎墜長官司。

定番州屬程番長官司,上馬橋長官司,小程番長官司,盧番長官司,方番長官司,韋番長官司,臥龍番長官司,小龍番長官司,金石番長官司,羅番長官司,大龍番長官司,木瓜長官司、副長官司,麻向長官司。

開州屬乖西長官司、副長官司。

龍里縣屬大谷龍長官司,小谷龍長官司,羊場長官司。

貴定縣屬平伐長官司,大平伐長官司,小平伐長官司,新添長官司。

郎岱屬西堡副長官司。

歸化屬康莊副長官司。

永寧州屬頂營長官司,沙營長官司。

鎮遠府屬偏橋長官司,工⼙水長官司。

黃平州屬巖門宣化長官司。

思南府屬蠻夷長官司,朗溪長官司、副長官司,沿河祐溪長官司、副長官司。

平越州屬楊義長官司。

思州府屬都坪長官司、副長官司,都素長官司、副長官司,黃道溪長官司、副長官司,施溪長官司。

黎平府屬潭溪長官司、副長官司,歐陽長官司、副長官司,龍里長官司,亮寨長官司,中林驗洞長官司,古州長官司,湖耳長官司、副長官司,八舟長官司,新化長官司,洪洲泊里長官司、副長官司。

都勻府屬都勻長官司、副長官司,邦水長官司。

麻哈州屬平定長官司,樂平長官司。

獨山州屬爛土長官司,豐寧上長官司、下長官司。

銅仁府屬省溪長官司、副長官司,提溪長官司、副長官司。

松桃屬烏羅長官司、副長官司,平頭著可長官司、副長官司。

西藏土兵:

雍正九年,新撫南稱、巴彥等處番民七十九族,地居四川、西藏、西寧間。十年夏,川、藏暨西寧分遣專官會同勘定,近西寧者歸西寧管轄,近西藏者暫隸西藏雲。

西寧管轄四十族:阿哩克族,蒙古爾津族,雍希葉布族,玉樹族,噶爾布族,蘇魯克族,尼雅木錯族,固察族,稱多族,洞巴族,多倫尼托克安圖族,阿薩克族,克列玉族,克阿永族,克葉爾濟族,克拉爾濟族,克典巴族,隆布族,上隆布族,札武族,上札武族,下札武族,札武班右族,上阿拉克碩族,上隆壩族,下隆壩族,蘇爾莽族,白利族,哈爾受族,登坡格爾吉族,下格爾吉族,格爾吉族,巴彥南稱族,南稱桑巴爾族,南稱隆冬族,南稱卓達爾族,吹冷多拉族,巴彥南稱界住牧喇嘛,拉布庫克住牧喇嘛。

西藏管轄三十九族:納書克貢巴族,畢魯族,琫盆族,達格魯族,拉克族,色爾札族,札嘛爾族,阿札克族,上阿札克族,下阿札克族,夥爾川木桑族,夥爾札麻蘇他爾族,夥爾札麻蘇他爾,只多族,瓦拉族,夥爾族,麻魯族,寧塔,尼札爾,參麻布瑪,尼牙木札族,利松麻巴族,勒達克族,多麻巴族,羊巴族,依戎夥爾族,夥爾族,彭他麻族,夥爾拉賽族,上剛噶魯族,下剛噶魯族,瓊布拉克魯族,噶魯族,色爾札族,上多爾樹族,下多爾樹族,三札族,三納拉巴族,樸族。

以上四十族,共八千四百四十三戶。三十九族,共四千八百八十九戶。雍正間,定族內人戶千戶以上設千戶一,百戶以上設百戶一,不及百戶者設百長一,每千、百戶下設散百長數人。至乾隆末,別定三十九族總百戶一,百戶十三,百長五十三,後增為百戶十六,百長六十一。


\end{pinyinscope}