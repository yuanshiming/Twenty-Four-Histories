\article{列傳一百五}

\begin{pinyinscope}
阿桂子阿迪斯阿必達

阿桂,字廣庭,章佳氏。初為滿洲正藍旗人,以阿桂平回部駐伊犁治事有勞,改隸正白旗。父大學士阿克敦,自有傳。

阿桂,乾隆三年舉人。初以父廕授大理寺丞,累遷吏部員外郎,充軍機處章京。十三年,從兵部尚書班第參金川軍事。訥親、張廣泗以無功被罪,岳鍾琪劾阿桂結張廣泗蔽訥親,逮問。十四年,上以阿克敦年老,無次子,治事勤勉;阿桂罪與貽誤軍事不同,特旨宥之。尋復官,擢江西按察使,召補內閣侍讀學士。二十年,擢內閣學士。時方征準噶爾,命阿桂赴烏里雅蘇臺督臺站。逾年,父喪還京。旋復遣赴軍,授參贊大臣,命駐科布多,授鑲紅旗蒙古副都統。二十二年秋,授工部侍郎。輝特頭人舍楞約降,唐喀祿以兵往會,為所襲,阿桂率兵策應,上嘉之,賜花翎。上命阿桂與策布登扎布合軍擊舍楞,毋使逃入俄羅斯。阿桂言:「得降賊,謂舍楞將逃土爾扈特;或不達,且復回準噶爾。邀之中路,可擒獻。」上責其觀望,召還京。是年準部平,復命赴西路,與副將軍富德追捕餘賊。

霍集占叛,二十四年,命赴霍斯庫魯克從富德進討。八月,逐賊至阿勒楚爾,又至伊西洱庫爾淖爾,回眾降。霍集占走拔達克山。是年回部平。上以阿克蘇新附,為回部要地,命阿桂駐軍綏撫。二十五年,移駐伊犁。阿桂上言伊犁屯田、阿克蘇調兵諸事。上嘉其勇往,命專司耕作營造,務使軍士、回民皆樂於從事。時西域初定,地方萬餘里,伏莽尚眾,與俄羅斯鄰。上詔統兵諸大臣議,咸謂沙漠遼遠,牲畜凋耗,難駐守。阿桂疏言:「守邊以駐兵為先,駐兵以軍食為要。伊犁河以南海努克等處,水土沃衍,宜屯田。請增遣回民嫺耕作者往屯;增派官兵駐防,協同耕種;次第建置城邑;預籌馬駝,置臺站;運沿邊米赴伊犁;簡各省流人嫺工藝者,發備任使。」又奏定山川、土穀諸祀典,上用其議。阿桂造農器,督諸屯耕穫,歲大豐。

二十六年,疏言:「伊犁牧群蕃息,請停內地購馬駝。增招葉爾羌、喀什噶爾、阿克蘇、烏什回民詣伊犁,廣屯田。」皆稱旨。迭授內大臣、工部尚書、鑲藍旗漢軍都統,仍駐伊犁。奏瑪納斯庫爾、喀喇烏蘇、晶河三地屯田,人授十五畝。二十七年,疏定約束章程,建綏定、安遠二城,兵居、民房次第立,一如內地,數千里行旅晏然,予騎都尉世職。召還,賜紫禁城騎馬,命軍機處行走。調正紅旗滿洲都統,加太子太保。二十九年,命署伊犁將軍。尋調署四川總督。時金川土司郎卡與綽斯甲布等九土司構釁,阿桂巡邊,盡得郎卡狡獪怙惡狀,並悉其山川形勢,入奏。是冬,召還京。三十年,上南巡,命留京治事。

烏什回賴黑木圖拉作亂,詔馳赴烏什與將軍明瑞攻之,賴黑木圖拉中矢死,眾伯克復推額色木圖拉抗我師,自三月至八月,攻城不下。明瑞軍其北,阿桂軍其南,作長圍困之,絕其水道。賊糧盡,內訌,沙布勒者擒額色木圖拉以獻,烏什平。上責其遲延,示怯損威,部議奪官,命留任,駐雅爾城。旋復奪尚書,命還伊犁助明瑞治事。阿桂疏請移雅爾城於楚呼楚,從之。三十二年,授伊犁將軍。請自楚呼楚至烏爾圖布拉克設三臺,以通雅爾,下部行。

緬甸擾邊,總督劉藻、楊應琚先後得罪去,上命明瑞率師討之,至猛育,糧盡,戰沒。大學士傅恆自請行,三十三年,以傅恆為經略,阿桂及阿里袞為副將軍,仍授阿桂兵部尚書、雲貴總督。三十四年,以明德為總督,令阿桂專治軍事。阿桂請由銅壁關抵蠻暮,伐木造舟,俟經略至軍,進攻老官屯,且言軍糧不給。上以為畏怯,罷副將軍,改授參贊大臣。九月,舟成,傅恆亦至,分三路進:傅恆出萬仞關,由大金沙江西經猛拱、暮魯至老官屯;阿里袞率舟師循江下;阿桂率蠻暮新舟出江會之,先伏兵甘立寨。緬人從猛戛來拒,寨兵出擊,沉三舟,舟師噪應之,緬人大潰,殲其渠,遂與西岸軍合。老官屯守御堅,軍士多病瘴,阿里袞卒於軍,復授阿桂副將軍。傅恆亦病,上命班師,而緬酋懵駁亦懲甘立寨之敗,遣使議受約束,乃召傅恆還。命阿桂留辦善後,授禮部尚書。

三十五年,兼鑲紅旗漢軍都統。命赴騰越待緬人入貢。遣都司蘇爾相賚檄至老官屯,緬人拘之,索還木邦等三土司。疏入,上命罷尚書、都統,以內大臣留辦副將軍事。三十六年,疏請大舉征緬,入覲陳機密。上手詔詰責,命奪官留軍效力。是時金川酋郎卡已死,其子索諾木及小金川酋澤旺子僧格桑擾邊,四川總督阿爾泰徵之無功,上命阿桂隨副將軍、尚書溫福進討。十二月,署四川提督,克巴朗拉、達木巴宗各寨。三十七年二月,克資哩山,進克阿喀木雅。松潘總兵宋元俊亦復革布什咱。兩金川勢日蹙,合謀抗我師。上命溫福等三路進討,阿桂出西路阿喀木雅攻喇卜楚克,克之,奪普爾瑪寨,進逼美美卡。澤旺為子謝罪,索諾木亦代僧格桑請還侵地,上不許。時侍郎桂林代阿爾泰為總督,並領其眾,至墨隴溝,失利,副將薛琮死之,阿爾泰劾罷桂林。上授阿桂參贊大臣,命赴南路接剿。僧格宗者,小金川門戶也。甲爾木山梁為僧格宗要徑。阿桂乘賊怠,潛赴墨隴溝,夜半大霧,襲據之,進逼僧格宗,突入毀其碉,殲賊無算。上授溫福定邊將軍,豐升額、阿桂俱授副將軍,分道取美諾。阿桂克美都喇嘛寺,俯瞰美諾。僧格桑遁布朗郭宗,而溫福亦克西路來會,進剿布朗郭宗。僧格桑送孥金川而遁底木達,求見父澤旺,澤旺不納,渡河走金川。澤旺降,械送京師,小金川平。於是議討金川,金川賊巢二:曰噶拉依,曰勒烏圍。溫福由功噶爾拉,阿桂由當噶爾拉,合攻噶拉依;豐升額由綽斯甲布徑攻勒烏圍。復授禮部尚書。

三十八年正月朔,冒大雪,進奪當功噶爾拉諸碉,而溫福至木果木,索諾木誘降番叛襲軍後,斷登春糧道,我師潰,溫福死之。小金川與美諾等相繼陷。阿桂悉收降番械,毀碉寨,分置其人章谷、打箭爐,斬其桀驁者,親殿軍退駐達河。事聞,上怒甚,命發健銳、火器兩營,黑龍江、吉林、伊犁額魯特兵五千,授阿桂定西將軍,明亮、豐升額副將軍,舒常參贊大臣,整師再出。十月,攻下資哩。用番人木塔爾策,分師由中、南兩路進,潛軍登北山巔,遂取美諾,明亮等亦克僧格宗來會,凡七日,小金川平。

三十九年正月朔,阿桂抵布朗郭宗,人裹十日糧,分三隊進,轉戰以前,克喇穆左右二山,贊巴拉克山、色依谷山。二月,克羅博瓦山,勒烏圍門戶也。賊退守喇穆山。部將海蘭察從間道破色漰普寨,繞出山後,賊退守薩甲山嶺。海蘭察奪其峭壁大碉,諸寨奪氣,同時下,乘勝臨遜克爾宗。僧格桑死於金川,金川酋獻其尸,而死守遜克爾宗。十月,阿桂用策先克默格爾山及凱立葉,於是日爾巴當噶諸碉反在我師後,遂悉平之。賊退守康薩爾山。時豐升額出北路,師至凱立葉,望見煙火,以師來會;而明亮出南路,阻於庚額山;阿桂令移軍,冒雨破宜喜,與明亮軍隔河相望。十一月,克格魯克古丫口,金川東北之賊殆盡。

四十年正月,克康薩爾山梁。二月,克沿河斯莫思達寨。四月,克木思工噶克丫口。五月,克下巴木通及勒吉爾博山梁,進據得式梯,復克噶爾丹寺、噶明噶等寨。進攻巴占,屢攻不下。分兵從舍圖枉卡繞擊,牽賊勢。七月,克昆色爾及果克多山,進克拉栝寺、菑則大海山梁,旋克章噶。八月,克隆斯得寨,遂克勒烏圍。捷聞,上遣阿桂子阿必達齎紅寶石頂賜之。九月,克當噶克底諸寨。十月,克達木噶。十一月,克西里山雅瑪朋寨。十二月,克薩爾歪諸寨,進據噶占。四十一年正月,克瑪爾古當噶碉寨五百餘,遂圍噶拉依。索諾木母先赴河西集餘眾,大兵合圍,與其子絕,遂降。阿桂令作書招索諾木,而其頭目降者相繼,索諾木乃率眾降。金川平,安置降番,設副將、同知分駐其地。詔封一等誠謀英勇公,進協辦大學士、吏部尚書、軍機處行走。四月,班師。上幸良鄉城南行郊勞禮,賜御用鞍馬。還京獻俘,御紫光閣,行飲至禮,賜紫韁、四開褉袍。

初,阿桂去雲南,緬甸遣使議入貢,械送京師下獄。至是誅索諾木母子頭人,上命釋緬使令觀,譯告以故,縱之歸,冀以威武風動之。四十二年,署云貴總督圖思德奏:「懵駁已死,子贅角牙立,輸誠納貢,原歸中國人。請開關通市。」上以事重,當有重臣相度受成,命阿桂往蒞。五月,授武英殿大學士,管理吏部,兼正紅旗滿洲都統。緬甸使不至,遣蘇爾相等歸,遂召阿桂還。未幾,緬甸內亂。又十餘年,國王孟隕具表祝上八旬聖壽,定十年一貢。南徼始安。

四十四年,河決儀封、蘭陽,奉命往按。阿桂令開郭家莊引河,築攔黃壩;又於下流王家莊,築順黃壩:蓄水勢,逼溜直入引河。四十五年三月,堤工蕆,還京。兼翰林院掌院學士。旋命勘浙江海塘,築魚鱗石塘、柴塘,及範公塘。四十六年,工成,命順道勘清江陶莊河道高堰石工。

甘肅撒拉爾新教蘇四十三與老教仇殺,戕官吏。總督勒爾謹捕教首馬明心下獄,同教回民二千餘夜濟洮河犯蘭州,噪索明心。布政使王廷贊誅明心,賊愈熾。上命阿桂視師,時阿桂猶在工。命和珅往督戰,失利。賊據龍虎、華林諸山,道險隘。阿桂至,設圍絕其水道,進攻之,賊大潰。殲蘇四十三,餘黨奔華林寺,焚之,無一降者。甘肅冒賑事發,命按治,盡得大小官吏舞弊分賕狀,讞定,疏請增設倉廒,廣儲糧石,以濟民食。

秋,河決河南青龍岡,命自甘肅赴河南會河道總督李奉翰督塞河。故事,河決,當決處兩端築壩,漸近漸合,謂之「合龍」。十二月,兩壩將合,副將李榮吉謂水勢盛,宜緩,阿桂督之急。既合,屬吏入賀,榮吉獨不至,召之,則對使者曰:「為榮吉謝相公,壩不可恃,不敢離也。」越二日,果復決,阿桂馳視。榮吉已墮水,懸千金賞救之起,解御賜黑狐端罩覆之。因上疏自劾,請別簡大臣董其役,上詔答,略曰:「近年諸臣中能勝治河任者,舍阿桂豈復有人?惟當安心靜鎮,別求善策。」四十七年,奏請於下游疏引河,上游築大堤,並於北岸建壩,迫溜南趨。四十八年,工始竟,詣熱河行在,復命仍赴工次,審定章程。

浙江布政使盛住疏論總督陳輝祖籍王亶望家有所私,命阿桂如浙江按治。還,又命勘江南鹽河水道,又命勘河南蘭陽十二堡堤工,並於戴村建閘。四十九年,甘肅鹽茶回民張阿渾據石峰堡以叛。上遣福康安、海蘭察等討之,復命阿桂視師。兩月餘,破堡,戮張阿渾等,加一等輕車都尉世職。又命督河南睢州堤工。五十年,舉千叟宴,阿桂領班。又命勘河南睢州河工,並察洪澤湖、清口形勢。五十一年,又命勘清口堤工,並如浙江按倉庫虧缺,勘海塘;又命勘江南桃源、安東河決。再如浙江按治平陽知縣黃梅重徵,論如律。

五十二年,又命督塞睢州十三堡河決。時臺灣民林爽文叛,上命福康安討之,諮阿桂軍事。阿桂疏論師當扼要害,分道並進,先通諸羅道,廓清後路,自大甲溪進兵。諭曰:「所見與朕略同,已諭福康安奉方略。」睢州工竟,又命勘江南臨湖磚石堤工。五十三年,又命按湖北荊州水災。請疏窖金洲以導水,修萬城堤以護城。五十四年,命再勘荊州堤工。嘉慶元年,高宗內禪,阿桂奉冊寶。再舉千叟宴,仍領班,於是阿桂年八十矣,疏辭領兵部。二年八月,卒,仁宗臨其喪。贈太保,祀賢良祠,謚文成。

阿桂屢將大軍,知人善任使。諸將有戰績,獎以數語,或賚酒食,其人輒感激效死終其身。臨敵,夜對酒,深念得策,輒持酒以起,旦必有所號令。方溫福敗,受命代將。一日日欲暮,率十數騎升高阜覘賊砦。賊望見,獷騎數百環阜上。阿桂令從騎皆下馬,解衣裂懸林木,乃令上馬徐下阜。賊迫阜,從落日中睹旂幟,疑我師眾,方遣騎出偵,阿桂已還軍矣。師薄噶拉依,索諾木約以明日降,城柵盡毀。日暮,諸將謁阿桂,謂:「今日必生致索諾木,不然,慮有他。」阿桂不答,入帳臥。明旦,索諾木自縛詣帳下。阿桂謂諸將曰:「諸君昨日語,蓋慮索諾木他竄,或且死。我已得險要,竄安之?且能死,豈至今日?故吾以為無慮。」諸將皆謝服。及執政,尤識大體。康熙中,諸行省提鎮以次即有空名坐糧,雍正八年著為例。乾隆四十七年詔補實額,別給養廉。阿桂疏言:「國家經費驟加不覺其多,歲支則難為繼。此新增之餉,歲近三百萬,二十餘年即需七千萬。請除邊省外,無庸概增。」上不從。是時帑藏盈溢,其後漸至虛匱。此其一端也。乾隆末,和珅勢漸張,阿桂遇之不稍假借。不與同直廬,朝夕入直,必離立數十武。和珅就與語,漫應之,終不移一步。阿桂內念位將相,受恩遇無與比,乃坐視其亂政,徒以高宗春秋高,不敢遽言,遂未竟其志。

高宗圖功臣於紫光閣,前後凡四舉,列於前者親為之贊。

定伊犁回部五十人:大學士傅恆,將軍兆惠、班第、納木札爾,副將軍策布登扎布、富德、薩拉爾,大學士總督黃廷桂,參贊大臣親王色布騰巴爾珠爾,貝子扎拉豐阿,郡王羅卜藏多爾濟、額敏和卓,尚書舒赫德、阿里袞,總督鄂容安,侍郎明瑞、阿桂、三泰、鄂實,領隊大臣內大臣博爾奔察,提督豆斌、高天喜,副都統端濟布,護軍統領愛隆阿,前鋒統領瑪巘,副都統巴圖濟爾噶爾,散秩大臣齊凌扎布、噶布舒,郡王霍集斯,貝子鄂對,內大臣鄂齊爾,散秩大臣阿玉錫、達什策凌,副都統鄂博什、溫布、由屯、三格,侍衛奇徹布、老格、達克、塔納、薩穆坦、璊綽爾圖、塔瑪鼐、富錫爾、海蘭察、富紹、扎奇圖、阿爾丹察、五十保。

定金川五十人:將軍阿桂,副將軍豐升額、明亮,大學士舒赫德、于敏中,尚書福隆安,參贊大臣親王色布騰巴爾珠爾,都統海蘭察,護軍統領額森特、舒常,領隊大臣都統奎林、和隆武、福康安,副都統普爾普,荊州將軍興兆,參贊大臣提督哈國興,領隊大臣提督馬彪、馬全、書麟,副都統三保、烏什哈達、瑚尼爾圖、珠爾格德、阿爾都、阿爾薩朗、舒亮、科瑪、伊蘭保、佛倫泰、富興、德赫布、莽喀察,總兵海祿、敖成、官達色、成德、欽保、曹順、保寧、特成額、烏爾納,總兵敦柱,侍衛額爾特、托爾托保、泰斐英阿、柏凌、達蘭泰、薩爾吉岱,佐領特爾惇澈,副將興奎。

定臺灣二十人:大學士阿桂、和珅、王傑,協辦大學士福康安,領侍衛內大臣海蘭察,尚書福長安、董誥,總督李侍堯、孫士毅,巡撫徐嗣曾,成都將軍鄂輝,護軍統領舒亮、普爾普,提督蔡攀龍、梁朝柱、許世亨,總兵穆克登阿、張芝元、普吉保,散秩大臣穆塔爾。

定廓爾喀十五人:大學士福康安、阿桂、和坤、王傑、孫士毅,領侍衛內大臣海蘭察,尚書福長安、董誥、慶桂、和琳,總督惠齡,護軍統領臺斐英阿、額勒登保,副都統阿滿泰、成德。

功稍次者列於後,儒臣為之贊,惟阿桂與海蘭察四次皆前列。阿桂定金川元功,定臺灣首輔,皆第一;定廓爾喀以爵復第一,讓於福康安。道光三年二月,宣宗命配饗太廟。子阿迪斯、阿必達。

阿迪斯,初以三等侍衛坐阿桂征緬甸無功,奪職,發遣廣西右江鎮。逾年赦復官。累遷兵部侍郎,襲一等公。復累遷成都將軍。以川西盜發,逮問,發遣伊犁。赦歸。卒。

阿必達,初名阿彌達,高宗命更名。阿桂得罪,奪藍翎侍衛,發遣廣東雷瓊鎮。赦歸,復官。擢二等侍衛,命赴西寧祭告河神,探黃河真源,上命輯入河源紀略。累遷工部侍郎。卒。阿必達子那彥寶,官至成都將軍;那彥成,自有傳。

論曰:將者國之輔,智信仁勇,合群策群力冶而用之,是之謂大將。由是道也,佐天子辨章國政,豈有二術哉?乾隆間,國軍屢出,熊羆之士,因事而有功;然開誠布公,謀定而後動,負士民司命之重,固無如阿桂者。還領樞密,決疑定計,瞻言百里,非同時諸大臣所能及,豈不偉歟?


\end{pinyinscope}