\article{志五十一}

\begin{pinyinscope}
地理二十三

△新疆

新疆:古雍州域外西戎之地。漢武帝設西域都護,天山以南,城郭三十六國皆屬焉。天山以北,東匈奴右部,西烏孫,未嘗服屬。後漢,山北如故,山南分五十餘國,於闐、龜茲最著。自建武迄延光,三絕三通,設都護及長史治之。三國及晉,北為烏孫及鮮卑西部,南為於闐、龜茲諸國。北魏,柔然、烏孫、悅般、高車盡有山北地;後周,突厥、鐵勒據之。其南以鄯善為強。唐於西州置北庭大都護府,統沙陀、突厥、回鶻、西突厥,北部諸都督府。於龜茲置安西大都護府,統龜茲、于闐、疏勒、碎葉四鎮,濛池、昆陵等都護。中葉後,為吐蕃所有。五代並於吐蕃、回鶻。宋時烏孫、回鶻居山北,於闐、龜茲諸國入於遼。元置三行尚書省,蔥嶺以東屬巴什伯里行尚書省。尋增天山南、北宣慰司,北則巴什伯里,南則哈喇和卓,後為都哩特穆爾地。明,四衛拉特居北部,曰綽羅斯、曰杜爾伯特、曰和碩特、曰輝特。其南部則巴什伯里、葉爾羌、吐魯番諸國,回部派噶木巴爾諸族居之。順治四年,哈密內屬,吐魯番亦入貢,惟四衛拉特仍據其地。準噶爾即綽羅斯部。數侵喀爾喀,聖祖三臨朔漠征之,噶爾丹走死。其兄子策妄阿拉布坦遁伊犁,傳子及孫,從孫達瓦齊奪其位。乾隆十九年,杜爾伯特、和碩特、輝特先後來歸。二十年,執達瓦齊,準噶爾平。二十二年,以阿睦爾撒納叛,霍集占附之,再出師。二十三年,克庫車、沙雅爾、阿克蘇、烏什諸城;明年,收和闐、喀什噶爾、葉爾羌諸城,二酋走死,回部亦平。二十七年,設伊犁總統將軍及都統、參贊、辦事、協辦、領隊諸大臣,分駐各城,並設阿奇木伯克理回務。秩三品至七品。光緒十年裁,改設頭目,以六品為限。同治三年,安集延酋阿古柏作亂,陜回白彥虎應之。光緒八年,全部蕩平。九年,建行省,置巡撫及布政使司,以分巡鎮迪道兼理按察使銜,改甘肅迪化州及鎮西、哈密、吐魯番三來隸。迪化尋升府,建省治。又改阿克蘇為溫宿直隸州,喀喇沙爾、庫車、烏什、英吉沙爾並為,置分巡阿克蘇道轄之;喀什噶爾為疏勒、葉爾羌為莎車直隸州,英吉沙爾、瑪喇巴什為,及和闐直隸州,置喀喇噶爾兵備道轄之;庫爾喀喇烏蘇為直隸,轄於鎮迪道;又改伊犁為府,精河、塔爾巴哈臺為,置分巡伊塔道轄之。二十四年,升喀喇沙爾為焉耆府。二十八年,改庫車為直隸州,疏勒、莎車、溫宿三直隸州並為府,又改瑪喇巴什為巴楚州,隸莎車府。凡領府六,直隸八,直隸州二,一,州一,縣二十一。宣統三年,編戶四十五萬三千四百七十七,口二百六萬九千一百六十五。東界外蒙古喀爾喀扎薩克圖汗部;西界俄羅斯;南界西藏;北界阿爾泰山;東南界甘肅、青海;西南界帕米爾;東北界科布多;西北界俄羅斯。廣七千四百里,袤三千七百里。東北距京師,由南路八千六百八十九里,由北路八千五百七十六里。北極高三十四度至四十九度有奇。京師偏西二十一度至四十三度。其名山:蔥嶺、昆侖、天山、博克達。其巨川:塔裏木、葉爾羌、和闐、伊犁諸河。其道路:天山南、北。電線:由迪化東南通蘭州,西北通伊犁,西南通喀什噶爾。

迪化府:,沖,繁,難。巡撫、布政使、提學使、鎮迪道兼提法司銜、副將同駐。漢,卑陸等十三國地,兼有匈奴屬地及烏孫東境。後漢初,鬱立師、單桓、烏貪訾離為車師所滅,後復立,時稱車師六國。三國時,東西且彌、單桓、卑陸、蒲類、烏貪,並屬車師後部。晉屬鐵勒,亦曰高車。初屬蠕蠕。北魏時,大破蠕蠕。後周屬突厥。隋大業中,西突厥始大,鐵勒諸部皆臣之。唐貞觀時內屬。及滅高昌,置庭州。又置瑤池都督府及馮洛州各都督府,統於安西大都護府。武后時,改隸北庭大都護府。開元初,置北庭節度使。貞元後,其地屬吐蕃,又屬西州回鶻。宋為高昌北庭,臣服於遼。南宋屬西遼。元太祖時,稱回鶻別失八里。元末,猛可鐵木兒據之,為瓦剌國。至明正統中為乜先。嘉靖間,分為四衛拉特,為瓦剌之轉音。居烏魯木齊者為和碩特部。後為準噶爾臺吉游牧地。乾隆二十年,平準噶爾,始內屬,改名烏魯木齊,築土城。二十五年,設同知。二十八年,築新城於其北,名迪化。三十六年,設參贊大臣、理事、通判。明年,於迪化西八里築滿城,名曰鞏寧。三十八年,改參贊為都統,設領隊大臣,駐鞏寧。三十八年,升直隸州,隸甘肅布政司。光緒九年,建行省,十二年,升府來隸。廣一千四百里,袤五百二十里。北極高四十三度二十七分。京師偏西二十七度五十六分。領縣六。光緒七年與俄立約,定為商埠。迪化沖,繁,難。倚。光緒十二年置。天山自西來,橫亙境南。西南:雅馬拉克山,綿延二百里。東北:達阪城嶺。東南:哈拉巴爾噶遜山。烏魯木齊河出南山,二源:東南曰庫爾齊勒河,西曰阿拉塔濟河,合北流,經城西,又北,名老龍河。頭屯河自昌吉入,東北流入境,瀦為八段、馬廠二湖,溢出,北流,與西支合,為兩縣交界處,三屯河自昌吉北來注之。復東北流,合老龍河,北經沙漠,入古爾班托羅海。廟兒溝、羊圈溝、大東溝、小東溝諸水,均出縣東南,分流,南入吐魯番。達阪城水,源出阜康博克達山天馬峰,入縣境,南流,合大銅溝、華樹林、方家溝、白家溝諸水,經達阪城卡倫入吐魯番。東南:鄂門泊、達布遜泊。北:大戈璧,廣五百里,長三百里。卡倫七。臺八。驛四:鞏寧、柴俄堡、達阪城、黑溝。有回莊六十七。阜康沖,繁,難。府東北一百三十里。漢,鬱立師、車師後國。魏,蠕蠕地。周、隋,突厥。唐,浮圖、沙缽、憑洛、耶勒、俱六諸地,貞觀中置金滿縣。元,別失八里地。明,敦剌城,改名特訥格爾。乾隆二十五年築堡,置巡司,尋改縣丞。二十八年建城,改州判,隸安西道。三十八年並入迪化州境。四十一年裁州判,置縣。博克達山綿亙南境,最高者曰福壽山。迤北,小黃山、大黃山。縣境諸水均發源博克達山。西:水磨河,西北流,分大西溝、小西溝二水。東有三工河,北流,疏為五渠。又東有四工河,北流,疏為四渠。又東為土墩子河,北流,疏為六渠。又東有柏楊河,北流,疏為四渠。又東曰東溝、西溝,北流入沙漠,合流而北,復分為東、中、西三渠。卡倫四。臺四。驛三:在城、康樂、柏楊。縣境分區二十七。孚遠沖,繁,難。府東北三百六十里。兩漢,車師後國,及其後城長國。魏,蠕蠕。周,突厥。唐,金滿縣。元,北庭都元帥府舊治,乾隆三十七年築愷安城,四十一年設濟木薩縣丞,治愷安,屬阜康。光緒二十年重修城,改名孚遠,二十九年升置。博克達山支脈蜿蜒起伏入境。西南:無量山。東南冰山、迤北千佛洞,皆博克達山之麓。城南小水均發源冰山。曰太平、公盛二渠,由柏楊河分支。曰長山、三盛二渠,出四道橋北。曰濟木薩河,分大有、興隆二渠。曰長勝渠。曰大東溝,北流入慶陽湖。曰經二工河,北流,經老三臺驛,瀦為麻菰湖。泉水三:東、西曰大泉,中曰上暖泉,均北流入沙磧。驛二:保會、三臺。卡倫二。有回莊二十五。奇臺沖,繁,難。府東北五百五十里。漢,蒲類、車師後城長國。魏屬蠕蠕。唐,蒲類,後置甘露州。同光初入遼。宋南渡後為別失八里東境。元入畏吾兒。明,衛拉特地。康熙中,準噶爾內附,乾隆二十四年建奇臺堡,設管糧通判一。東吉爾瑪泰,管糧巡檢一。四十年築靖遠城。四十一年裁通判,置縣,隸鎮西。咸豐三年改隸迪化州。光緒十五年自靖遠徙今治。天山支脈自西南更格爾入境,至穆家地溝東出境,綿亙四五百里,土人謂之南山。沙山自濟木薩至縣境舊城北,迄鎮西,袤延三四百里。北:拜達克山。東北:哈布塔克山。縣境諸水皆自南山出,曰奇臺水、木壘河、木楊河、新戶梁水、中葛根水、西葛根水、永豐渠水、吉爾庫水、達阪河、更格爾水。柳樹河自孚遠東流入境,經縣北,又東至三個莊子,入沙磧。驛十:古城子、平營、木壘河、阿克他斯、烏蘭烏蘇、色必口、頭水溝、北道橋、黃草湖、元湖。縣境分區三十六。卡倫十六。臺七。舊城,巡司駐。光緒七年,俄約定古城為商埠。昌吉沖,繁。府西九十里。漢,單桓、東西且彌、烏貪訾離地。晉屬高車。魏,蠕蠕。隋,西突厥、鐵勒地。唐屬北庭。元屬回鶻五城,名昌都剌。明屬衛拉特。乾隆二十五年置,設通判。二十七年築寧邊城,設管糧巡檢。三十八年改州同。四十二年置縣。天山支脈,綿亙縣境。南:騷呼達阪、格柵圖山、草達阪、石梯子山、塔拉盤山。頭屯河出天山北麓,分東西二支:東支入迪化;西支經縣治東,復北流,至縣境合流入迪化。大西河亦出天山北麓,至縣治西,分二支:東為三屯河,東北流,注頭屯河;西為大西河,即洛克倫河,自焉耆東北流入境,折西北流,經呼圖壁,緣綏來界。呼圖壁河源出塔拉盤山西,自焉耆府北流入,棗溝水南來注之。又北經草達阪,東分頭工渠,西分土古里渠,又北至呼圖壁城。復分二渠,東曰梁渠,西曰西河,北經牛圈子、三家梁,至雙岔子,合洛克倫河,西北入綏來,瀦於阿雅爾淖爾。呼圖壁原名呼圖拜克,乾隆二十二年設洛克倫巡司。二十八年移駐呼圖壁。二十九年築城名景化,為巡司治所。光緒二十九年升縣丞。驛二:寧邊、景化。卡倫五。呼圖壁卡倫一。臺四。有大回莊四。呼圖壁分區二十六。綏來沖,繁,難。副將駐。府西北三百四十里。漢,烏貪訾離及烏孫東境。三國,烏孫。魏,高車。周,突厥。隋,西突厥、鐵勒諸地。唐,西突厥處密部內屬,隸北庭都護府。宋、元,回鶻地。明,衛拉特。乾隆二十八年築綏來堡。三十三年設縣丞。四十三年於舊陽巴勒噶遜城西建二城:北康吉,南綏來,中靖遠關。四十四年置縣,治康吉城。光緒十二年,合兩城為一,移治南城。天山支脈,蜿蜒南境。西南:額林哈畢山、古爾班多博克達山、博羅托山。東南有甘溝山、古爾多拜山。南有大小衛和勒晶嶺。瑪納斯河自焉耆府北流入,亦名龍骨河,經城西,折西北,瀦為各林各土淖爾,東北流,注阿雅爾淖爾。洛克倫河自昌吉西北流入,逕沙漠,亦入阿雅爾淖爾。和爾果斯河及安集海大小二水,皆出額林哈畢山。烏蘭烏蘇河出古爾班多博克達山。金溝水出博羅托山。塔西河出古爾多拜山。驛十二:在城、靖遠、樂土、烏蘭烏蘇、安集海、撞田、沙門、新渠、小拐、三岔口、唐朝渠、黃羊。有大回莊十一。卡倫七。臺五。

鎮西直隸:沖,繁,難。隸鎮迪道。巴里坤總兵駐。漢,東蒲類國。後漢屬伊吾盧。北魏屬蠕蠕。隋屬突厥,後分屬西突厥。唐,沙陀部與處月雜居,沙陀叛附吐蕃,徙居北庭。宋屬伊州,後入於遼。元為別失八里東境,屬亦都護。明為和碩特部地。明末固始汗遷青海,後為準噶爾臺吉游牧地。康熙三十六年,平準噶爾,阿爾泰山以東地內屬。雍正七年,建城於巴爾庫勒,改名巴里坤。九年,設安西同知,隸甘肅布政司。乾隆三十七年,於東南築會寧城,設領隊大臣。三十八年,升鎮西府,領宜禾、奇臺二縣。咸豐五年,仍為,移鎮迪道駐之。裁宜禾。光緒十二年來隸。西南距省治一千三百三十里。廣千里,袤八百里。北極高四十三度三十九分。京師偏西二十三度三十六分。天山支脈迤邐南部者為祁連山。西北有妙雷努雷山、鍋底山、那梅州山。東北有薩混子山。東有松山、千里格山。巴爾庫勒淖爾即蒲類海,在西北,皇渠、水磨河、高家湖合諸小水均瀦入之。北有鹽池。東北:察哈泉。東南:柳條河與昭莫多河合。驛八:曲底、奎素、松樹塘、蘇吉、下肋巴泉、務塗水、芨芨臺、上肋巴泉。卡倫二。境分區二十四。

吐魯番直隸:沖,繁,難。隸鎮迪道。回部郡王、臺吉駐。漢,車師前王庭,後置戊己二校尉。晉治高昌,後入涼。北魏為高昌國,並於蠕蠕。後立闞伯周為高昌王,傳至鞠嘉,為唐所滅,置西州,升安西都護府。貞元中,陷吐蕃。五代為回鶻所據,稱西州回鶻。宋建隆二年入貢。元太祖平其地,號畏吾兒,設都護,封察哈臺於此。明初為火州地,嗣稱吐魯番。順治三年,吐魯番阿布勒阿哈默特入貢。六年,助河西逆回,絕其使,尋復通。康熙二十四年,回疆平。雍正五年內徙,安置瓜州,建城闢展。乾隆二十四年,設建六城於闢展,置辦事大臣、管糧同知,仍以吐魯番廣安城為回城。回城四:曰魯克沁,曰色更木,曰哈喇和卓,曰托克遜。合吐魯番為六城。設阿奇木伯克理回務。四十四年,移同知駐吐魯番,並設巡檢,隸甘肅布政使司。四十五年,裁辦事大臣,改設吐魯番領隊大臣,歸烏魯木齊都統節制。光緒十年,裁領隊大臣。十二年,置直隸來隸。西北距省治五百三十里。廣八百餘里,袤五百餘里。北極高四十三度四十分。京師偏西二十六度四十五分。領縣一。天山橫亙北境,為群山之總幹。東北:柯格達阪。北:度吉爾山、阿布都爾山。西:湖洛海、合同察海、卡卡蘇各達阪。南:哈拉可山、庫木什達阪、覺洛塔哈山。東南:克子里、阿習布拉克、勝金臺山。白楊河自迪化入,東南流,逕托克遜沙山,瀦為覺洛浣。西:烏斯水、作洛滿若水、布而水,均出合同察海達阪,入焉耆。驛十一:楊和、勝金口、巠硜子、三角泉、布干臺、托克遜、小草湖、蘇巴什、阿哈布拉、桑樹園、庫木什。卡倫一。有回莊二十。回城巡司一。光緒七年,俄約定為商埠。鄯善沖,繁,難。東二百五十里。漢,車師前國東境樓蘭。元魏後為高昌白棘城。唐,柳中縣,屬西州交河郡地。宋,六種,屬高昌,後入遼。元,魯克察魯地。明,柳城。康熙末內屬。乾隆三十六年設闢展巡檢。光緒二十九年改置。天山分支亙於北境,有東西柯柯雅山、茂萌山、高泉達阪。縣境諸水,出自井泉,伏流地中。西北:五個泉、夾皮泉。北:柳樹泉。西南:馬廠湖。南:戈壁。驛八:齊克滕木、土墩子、西鹽池、惠井子、梧桐窩、七個井、車箍轤、連木沁。卡倫一。有大回莊七。

哈密直隸:沖,繁。隸鎮迪道。副將駐。漢,伊吾盧地,為匈奴呼衍王庭,後置宜禾都尉。三國屬鮮卑西部。

晉屬敦煌郡。北魏屬蠕蠕。隋築新城,號新伊吾,後屬西突厥。唐貞觀四年,置西伊州,尋改伊州,置都督府。天寶初,改伊吾郡,尋復初。廣德後,陷吐蕃。五代時,號胡盧磧。宋雍熙後,屬回鶻,元屬畏吾兒,後為宗室納勿里封地。明永樂四年,建哈密衛。正德中,服屬吐魯番。順治四年,哈密衛輝和爾都督入貢。六年,以助逆絕貢,復通。康熙三十六年,俘獻色布騰巴勒珠爾,賜額貝都拉扎薩克印。三十七年,編列旗隊,設管旗章京。雍正五年,始建城。回城在城西三里,回子郡王所居,康熙五十六年築。設協辦旗務伯克。十三年,設駐防兵。乾隆二十二年,準部平,其酋伊薩克內附,移靖逆、瓜州、黃墩各營駐之,撤駐防兵。二十四年,設辦事大臣、協辦大臣、撫民通判、巡檢,隸甘肅布政司。光緒十年,升直隸。十二年來隸。西北距省治一千六百二十里。廣四百五十餘里,袤二百五十里。北極高四十二度五十三分。京師偏西二十二度三十四分。北:天山。其分支,西北:截達阪、沙克拉山、可雅爾達阪、合塔手可拉山。東北:阿克相木山、坤翌圖山、阿里鐵洛可山、空多洛托山。哈密河出西蘇巴什湖,南流瀦為小南湖,又南流,東為碩洛浣,西為阿里浣,折西南,瀦為大泉海子,為沙漠所滲。東:乾河子。東北:烏拉臺水、安吉水、黑具瑪水、達子湖。西:依他拉可水、八道溝等水。南:戈壁。驛十四:伊吾、南山口、黃蘆岡、長流水、格子煙墩、苦水、沙泉子、星星峽、頭堡、三堡、三道嶺、尞墩、橙槽溝、一碗泉。新城,巡司一。境分區三十五。有大回莊十四。光緒七年,俄約定為商埠。

庫爾喀喇烏蘇直隸:沖,繁,難。隸鎮迪道。辦事、領隊大臣駐。漢,匈奴西境。晉後為鐵勒部。北周屬突厥。隋屬西突厥,為處木昆部。唐永徽中破之,設郡縣,屬昆陵都護府。開元中,置瀚海軍。後唐時屬遼。元,回鶻地。明,綽羅斯部地,後屬準噶爾。乾隆二十二年,平準部。二十七年,設辦事大臣。二十八年,築慶綏城。三十七年,設領隊大臣、縣丞。四十六年,設同知。明年設游擊。四十八年,築新城,定今名,設糧員,裁縣丞。隸烏魯木齊都統。光緒十二年,裁糧員,置直隸,改隸。東距省治七百里。廣三百三十里,袤五百四十里。北極高四十四度三十分。京師偏西三十一度。天山支脈在境南者,額林哈畢爾噶、托羅滾、沙得格果沙吐克土諸山,額布圖、古爾班、恰克、額爾圖諸嶺。奎屯河出托羅滾山,合托羅滾水、熱水泉、沙格得水,北流,至城東,分二支渠,至二臺驛,折西流,與濟爾噶朗河會。濟爾噶朗河出古爾班嶺,合札哈水、東鬥水、哈峽圖水,北流,會奎屯河,又西流,入精河,與固爾圖河會。固爾圖河出額爾圖嶺,有五源,合北流,折西入精河,與金屯河會。西流,瀦於喀喇塔拉阿西柯淖爾。精河巡司一。驛九:西湖、奎屯、普爾塔齊墩、木達、固爾圖、頭臺、二臺、小草湖、鄂倫布拉克。境分區九。有舊土爾扈特部游牧地。卡倫一。

伊犁府:沖,繁,疲,難。隸伊塔道。漢至晉為烏孫、伊烈兩國地,後入鐵勒。北魏,悅般國,又車高地。周,突厥地。隋,西突厥及石國。唐,西突厥及回鶻地,又西境為突厥施烏質勒部,又西突厥及笯赤建國、石國地。大歷後,葛邏祿居之。宋為烏孫,後入遼。元名阿力麻裏,為諸王海都行營處。明,綽羅斯部,後屬準噶爾。乾隆時,準部平,改烏哈爾里克為伊犁。二十五年,設辦事大臣。二十七年,設將軍,節制南北兩路,以參贊大臣副之。初設二員,尋裁一。二十九年,設錫伯營、索倫、察哈爾領隊大臣各一。三十年,設額魯特領隊大臣。三十四年,設惠寧城領隊大臣。築河北九城。曰惠遠,將軍、參贊大臣、各營領隊大臣駐。總兵先駐綏定,尋移駐。理事同知、撫民同知、巡司各一。改巴顏岱曰惠寧,領隊大臣駐,糧員、巡司各一。改烏哈爾里克曰綏定,總兵駐,糧員、巡司各一。改烏克爾博羅素克曰廣仁,屯鎮左營游擊駐。改察罕烏蘇曰瞻德,都司、守備駐。改霍爾果斯曰拱宸,參將駐,巡司一。改哈拉布拉克曰熙春,屯鎮都司駐,曰塔勒奇,屯鎮守備駐。改固勒扎曰寧遠,以居回民。設阿奇木伯克、伊什罕伯克各一。糧員一。同治五年陷回。後又為俄占。光緒初,全疆底定。八年,收回伊犁。十四年,以綏定城置府。將軍,副都統,參贊大臣,領隊大臣,索倫、額魯特、察哈爾、錫伯各領隊大臣,及滿洲八旗軍標副將,理事同知,同駐惠遠城。參將、霍爾果斯通判駐拱宸城。游擊駐廣仁城。守備駐瞻德城。都司駐熙春城。東距省治一千五百四十五里。廣一千五百餘里,袤一千一百餘里。北極高四十三度五十六分。京師偏西三十四度二十分。領縣二。咸豐元年,俄約定為商埠。綏定沖,繁,疲,難。倚。乾隆二十六年設巡檢。光緒十四年置,移巡檢駐廣仁城。天山支脈綿亙北境。北:塔勒奇山。東北:新開達阪、庫森木什達阪。伊犁河自寧遠西北流入。通惠渠南北二渠,烏拉果克水、大西溝、察罕爾烏蘇水皆注之,又西至河源卡,會霍爾果斯河,西流入俄界。南:大小博羅莊水、霍洛海莊水、沙拉諾海水、洪海水,均北流入沙磧。又東西阿里瑪圖水,北大小東溝水,亦入沙磧。北:賽裏木淖爾。驛七:沙泉子、惠遠城、蘆草溝、塔爾奇阿滿鄂博、勒齊爾鄂勒著衣圖博木、胡素圖布拉克。臺站一,屬錫伯營,回莊十六。額魯特部上三旗、下五旗,及察哈爾部游牧地。卡倫十三,為中、俄交界,歸額魯特、錫伯營分轄。牌博自南而西至西北,均連俄界。自沙拉諾海小山立第十六牌博,至頭胡第二十五牌博,凡十。寧遠繁,難。府東南一百二十里。伊塔道治所。乾隆間,築寧遠城於固勒札。光緒八年設同知,十四年改置。東南:博羅布爾噶蘇山、哈什山、大蒙柯圖山、烏土達阪、木尼克得山、額林哈必爾山、克里克子達阪。南:索達阪、色格三達阪。西南:喀喇套山、格登山、汗騰格里山、沙拉套山、諾海托蓋山。特克斯河出俄屬木薩爾山,自胡素圖卡南、諾海托蓋山北入,折東流,納戛雄河、大小霍洛海諸水,又東與崆吉斯河會。崆吉斯河自焉耆西北流入,逕耶裡格莊南,合特克斯河,西流,至阿瓦克莊西,與哈什河會。哈什河源出大蒙柯圖山,西流,納十二圍場水、皇渠、錫伯營渠,合西北流,為伊犁河,入綏定。北:賽裏木淖爾。驛一:在城。臺站七,屬錫伯營。回莊三十七。額魯特游牧地。卡倫七,為中、俄交界,歸額魯特營轄。縣境由南而西均界俄。自納林哈勒噶立第一牌博,至阿哩千谷第十五牌博,凡十五。

塔爾巴哈臺直隸:繁,疲,難。隸伊塔道。塔城左翼副都統、參贊大臣、領隊大臣、副將駐。漢,匈奴右地。三國,鮮卑右部。北魏,高車、蠕蠕地。北周、隋,屬突厥。唐屬車鼻南境,為葛羅祿,後南徙,地屬黠戛斯。後周時,貢於遼。南宋為乃蠻國。元封諸王昔里吉。明為土爾扈特部地,後屬準噶爾。乾隆二十二年,準部平,始內屬。二十九年,築城雅爾,名曰肇豐。三十一年,改築城於楚呼楚,距雅爾二百里。名曰綏靖,易其地名為塔爾巴哈臺。設參贊大臣、協辦領隊大臣,專理游牧,領隊大臣各一,管糧理事、撫民同知,尋改通判,隸伊犁將軍。光緒十四年,置直隸,改隸。於治東南里許築新城,改參贊大臣為左翼副都統。東南距省治一千六百二十四里。廣一千二百里,袤八百里。北極高四十七度五分。京師偏西三十度三分。天山支脈,蜿蜒南部。東:齋爾山、蘇海圖山、巴戛阿拉戛凌圖山。東南:喀圖山。西南:巴爾魯克山。東北:阿爾泰山、賽里山、和博沙克里山、芍隴山。北:塔爾巴哈臺山,支峰為毛海柯凌山、烏什嶺、額依賓山。額爾齊斯河自科布多部西流入,納哈布干諸水,西入俄界,瀦於齋桑淖爾。額敉勒河出額依賓山西麓,西流,至庫爾噶蘇臺,合南源,西流,烏拉斯臺水、烏宗戛拉水自俄境南流注之,博爾里河合察罕河北流注之,又東入俄境。和博克河出額依賓山東麓,東南流,合和博沙爾克河、巴杏薩拉水,又東南入昌吉,滲於沙。蘇爾圖河出齋爾山,東流,會納木河,又東瀦為艾拉克淖爾。達爾達木河亦出齋爾山,東流,瀦為鹽池。說爾噶其河亦出齋爾山,東南流,入綏來,瀦於阿爾雅淖爾。驛十二:郅支、乾吉莫多、色特爾莫多、固爾圖、霍洛、托羅布拉克、雅瑪圖、昆都倫、烏土布拉克、沙爾札克、烏納木、庫克申倉。有回莊九。額魯特部、察哈爾部十牛錄、舊土爾扈特部十四牛錄游牧地。哈薩克四部游牧地:曰柯勒依,附以新舊兩烏瓦克小部;曰賽布拉特,附以阿克奈曼部;曰曼畢特;曰吐爾圖。卡倫六。境西北與俄界,自精河至迤南,立土斯賽第三十四牌博,至布爾罕布拉克第五十五牌博,凡牌博二十二。又東循哈巴爾烏蘇塔爾巴哈臺山梁,至穆斯島,折北行,曰依生克裡的,曰布羅呵卡,曰二支河等處,凡立牌博二十七。咸豐元年,俄約定為商埠。

精河直隸:沖,繁,難。隸伊塔道。漢、魏,烏孫。晉,鐵勒部。北魏為金山以南諸部。隋、唐,西突厥,後設嗢鹿州都督府。元,曲只兒地。明,準噶爾各鄂拓克臺吉游牧地。乾隆二十二年,準部平,始建安阜城於精河,設典史。四十八年,於城東二里建新城,仍舊名,設都司、糧員、巡司,裁典史,隸烏魯木齊都統。光緒十四年,置直隸,改隸。東距省治一千七十五里。廣六百五十餘里,袤四百五十餘里。北極高四十四度四十分。京師偏西三十二度四十分。天山支脈自東北來,袤延境內。北:喀拉達阪、索達阪。南:登努勒臺山、烏蘭達阪、布裏沁達阪。西:德木克沁喀三達阪,別珍島。博羅塔拉勒河出西,東流,布哈水南流,庫森木什水北流注之。精河出登努勒臺山,有五水南來注之。奎屯河自庫爾喀喇烏蘇西流入,合古爾圖河,與博羅塔拉河、精河均瀦於喀喇塔拉額西柯淖爾。驛五:安阜、托裡托、和木圖、沙泉、托多克。博羅塔拉巡司一。北山,舊土爾扈特、察哈爾部游牧地。南山,哈薩克部游牧地。卡倫十三。為中、俄交界,歸察哈爾營轄。

溫宿府:沖,繁,疲。阿克蘇道治所。阿克蘇總兵駐。舊阿克蘇回城。阿克譯言「白」,蘇謂「水」也。漢,姑墨國。三國至北魏屬龜茲。南宋時屬西遼。元,別失八里西境,封宗王阿只吉。明永樂間入回部。後並於準噶爾。乾隆二十二年始內屬,改名阿克蘇。二十四年,回部平。四十四年,移烏什領隊大臣來駐。嘉慶二年,改設辦事大臣,隸喀什噶爾。光緒十年裁,置直隸州。二十八年升府。光緒九年,築新城為府治。東北距省治二千七百八十里。廣一千二百餘里,袤八百餘里。北極高四十一度九分。京師偏西三十七度十五分。領縣二。西南:格達爾山、鐵克列克達山、穀故提山。瑚瑪喇克河、哈拉和旦河、托什罕河,皆自溫宿東南入,合流,納畢底爾河,至賽裏木為渾巴什河,逕乙思坤莊,葉爾羌河自巴楚州北流注之,又納和闐河,東南流為塔裏木河,入沙雅。驛四:渾巴什、薩伊里克、喬裡呼圖、齊蘭臺。柯坪巡司一。大回莊十二。布魯特諾依古特部游牧地。溫宿沖,繁,難。府北二十五里。道光十九年築城,回城西北曰舊城。光緒九年設巡司。二十八年置縣。汗騰格里山為天山最高之峰,由縣西北蜿蜒東北,為與伊犁府及俄羅斯界山。西北支山有薩雷雅斯山、楚克達爾山、薩瓦巴齊山、帖列達阪。東北支山有薩巴齊山、烏西拉克山、木素達阪、鐵廠山、意什哈子山。鐵梁河出縣東,哈拉和旦河溢出之水,至縣南合流,注渾巴什河。瑚瑪喇克提河、托什罕河,均出縣西,東南流,托幹什河、畢底爾河自烏什東流注之,為渾巴什河,入府。驛十:在城、雖雅克、札木臺、哈拉玉爾滾、阿爾巴特、和約伙羅、巴圖拉特湖、斯圖托海、塔木哈塔什、噶克察哈爾。大回莊九。卡倫一。拜城沖,疲。府東四百五十里。漢,姑墨國地。唐為阿悉言城,後並於龜茲。乾隆二十二年內屬,置阿奇木伯克理回務。光緒十年置縣。天山綿亙北境。東北:哈雷克套山、冰山、大木素爾達阪、明布拉克山。東南:截達阪,有滴水崖、溫巴什、托和奈旦、鞏伯、和色爾銅礦五。木札拉提河,發源冰山,西南流,納鬧水、鐵敏水,折東南,納特拉布覺克水,為銅廠河。又納哈拉蘇水、宿什勒克水,為渭乾河,東南入庫車。驛五:姑墨、鄂伊斯塘、察爾齊、賽裏木、河色爾。回莊二十一。卡倫三。

焉耆府:要,沖,難。隸阿克蘇道。舊喀喇沙爾回城。喀喇譯言「黑」,沙爾「城」也。漢,焉耆、危須、尉犁諸國地。後漢至隋為焉耆國。唐貞觀六年來朝,十八年置焉耆都督府,後立碎葉鎮於此。貞元後,沒於吐蕃。宋西州回鶻地,後屬西遼。元,別失八里東境。明初朝貢,後徙天山南,據其地,號伊勒巴拉。康熙中,準部噶爾丹占為牧場,小策凌敦多布、噶爾丹策零先後據之。乾隆二十二年,準部平,改名喀喇沙爾。二十三年,始建城。城毀於火。置府後,就安集延回城拓大之。二十四年,設辦事大臣。轄布古爾、庫爾勒二城,設游擊,以阿奇木伯克理回務。三十八年,土爾扈特及和碩特移牧珠勒都斯,歸辦事大臣兼轄。光緒八年裁,設喀喇沙爾直隸。二十四年升府,易今名。北距省一千九十里。廣一千餘里,袤二千五百里。北極高四十二度七分。京師偏西二十九度十七分。領縣三。西:達蘭達阪、江布達阪、達哈特嶺。西北:胡斯圖達阪、澤達阪、察罕薩拉達阪、和屯博克嶺。北:硃勒都斯山。東:博爾圖山、鐵里達阪、薩薩爾達阪。東南:庫爾泰山、大石山、乾洛可達阪。開都河,源出和屯博克嶺,南流,經硃勒都斯山,分二支,復合扣克訥克水,折東南,納賽仁木諸小水;南流逕城西,匯於博斯騰淖爾。復溢出,逕庫爾勒回城,又匯為布它海子,入輪臺。崆吉斯河,源出城西北舉爾達阪,西北流,入寧遠。瑪納斯河,源出胡斯圖達阪,北流入綏來。呼圖壁河,源出府北天格爾達阪,北流入昌吉。驛九:在城、清水河、烏沙克塔、新井子、榆樹溝、柴泥泉、庫爾勒、上戶地、庫爾楚。大回莊八。土爾扈特部兩札薩克、和碩特部兩札薩克游牧地。卡倫五。新平疲,難。府南三百六十餘里。漢,尉犁國地。三國後入焉耆。明,後什尼戛地。舊名羅布淖爾,屬魯克沁回王。光緒十一年,設局蒲昌城理屯防。二十四年置縣,治羅布淖爾,以游擊駐蒲昌城。北:大石山支脈自府境迤邐入縣。塔裏木河自沙雅東流入,分二支,南支匯為小羅布淖爾,溢出東行,與北支合,渭乾南河自沙雅東流注之,又東注為六泊。渭幹北河自輪臺東流入,瀦為沖庫海子。孔雀河承布它海子水自府入,溢出東流,與沖庫海子溢出水合,東南流,入婼羌。古斯拉克河由塔裏木河所瀦之第五泊溢出,東北流,斜貫渭幹北河,為罕溪河,東北匯為小海子,入婼羌。驛九:在城、克泥爾、英氣蓋河、楷拉、英格可立、烏魯可立、古斯拉克莊、哈什墩、都拉里。回莊二十。卡倫三。輪臺沖,疲,難。府西南六百十五里。舊至古巡司。漢,輪臺、烏壘、渠犁。晉,龜茲國地。元魏後入吐谷渾。唐屬安西都護,與於闐、疏勒、碎葉為四鎮,後陷吐蕃。元為別失八里東境。乾隆中內附。二十四年,設阿奇木伯克理回務。光緒八年裁,設巡司。二十八年,以布古爾置縣。北:珠勒土斯山,蜿蜒數百里。的納爾河發源庫車之哈拉草湖,南流入境,納縣北諸水,又南流,匯為斯爾里克黑洗湖。渭幹北河自沙雅東北流入,又東入新平。東南:大戈壁。驛四:布古爾、洋薩爾、策達雅爾、野雲溝。回莊九十一。婼羌要。府東南一千二百餘里。漢,婼羌國。乾隆二十四年,設阿奇木伯克理回務。光緒二十四年裁,設卡克里克縣丞,隸府。二十八年置縣。昆侖支脈亙於境內。南:烏蘭達布遜山、阿勒騰塔格嶺、阿里哈屯山、大中小屈莽山。東:阿思騰塔格山。孔雀河自新平東南入,分二支:一東流,瀦為孔雀海子;一合阿喇鐵裏木河,至托乎沙塔莊,注羅布淖爾。卡墻河自於闐東北流入,並注淖爾。淖爾廣袤三四百里,古蒲昌海,亦鹽澤、水幼澤,伏流東南千五百里,再出積石為黃河。其東北碩洛浣,南庫木浣,東阿不旦海,並入於沙。驛六:在城、羅布、破城、托和莽、阿拉罕、哈拉臺。回莊十一。額魯特部游牧地。

庫車直隸州:沖,繁。隸阿克蘇道。舊回城。漢,龜茲國。後漢建武中,滅於莎車,尋復立,屬匈奴。永元三年內屬。晉太康中,為焉耆所滅,尋復立。唐貞觀中,置龜茲都督府。顯慶三年,徙安西大都護治之。北宋時入貢。南宋屬西遼。元為別失八里西境。明永樂中,並入回部。順治、康熙間,準噶爾兼有其地。乾隆二十三年,討霍集占,伯克阿集以城降,改名庫車。庫譯言「此地」,車謂「眢井」也。二十四年,設辦事大臣。設都司,以阿奇木伯克理回務。光緒十年裁,置直隸州。北距省治二千三十里。廣六百十里,袤七百里。北極高四十一度三十七分。京師偏西三十三度三十二分。領縣一。汗騰格里山支脈綿亙北境。東北有迭拉爾達阪。西北:阿爾齊里克達阪、馬納克齊達阪、泰來買提達阪、阿拉阿奇達阪。西:托和拉旦達阪、千佛洞。北:蘇巴什銅廠。龍口河源出迭拉爾達阪,西南流,納塔裏克水、托克蘇拉水、卡拉淖水、朵托水,至隨魯莊,分為葉斯巴什河、烏恰薩伊河、密爾特彥河,合流而東,瀦為沙哈里克草湖。拉依蘇河出城北,分二支,均南流,一入輪臺,一入沙哈里克草湖。渭乾河自拜城東南流入,逕千佛洞,南流入沙雅。驛五:鳩茲、托和拉旦、托和奈、哈爾巴、阿爾巴特。大小回莊一百二十六。卡倫四。沙雅州南百八十里。唐,突厥施沙雁州。乾隆二十四年,設阿奇木伯克理回務。光緒十年裁。二十九年,以沙雅爾回城置縣。西北:哈電克套山。渭乾河自州南入,折東流,至薩牙巴克莊,為鄂根河,逕沙克理克,分支南流入塔裏木河。又東逕阿洽,分二支:一東南流,出境為渭乾南河,入新平;一東北流,出境為渭幹北河,入輪臺。塔裏木河自溫宿東南入,至可可覓,納渭乾河支流,至喀喇墩,東流入新平。西南:下和里海子。西:草湖浣。驛二:在城、亮噶爾。回莊六十四。卡倫三。

烏什直隸:要,沖,疲,難。隸阿克蘇道。副將駐。漢,溫宿國。後漢內附。北魏入龜茲。唐貞觀中平之,置溫肅州,隸安西都護府。南宋屬西遼。元為別失八里西境,封宗王阿只吉。明永樂中,其王西遷,地入回部。後並於準噶爾,名圖爾璊。乾隆二十年,阿奇木伯克霍集斯擒達瓦齊,以城內屬,改名烏什。以烏赤山得名。二十三年,設辦事大臣、參將。三十一年,築永寧城,移喀什噶爾參贊大臣、協辦大臣駐之,又設領隊大臣。四十四年,移領隊駐阿克蘇。五十二年,移參贊、協辦駐喀什噶爾,仍留辦事大臣。光緒九年裁,置直隸。東北距省治三千二十里。廣一千一百八十里,袤三百七十里。北極高四十一度六分。京師偏西三十八度二十七分。天山支脈綿亙境內。西南:烏魯山達阪。南:木其別什達阪、登魯古達阪、屯珠素山。東南:庫魯克達哈山。西北:上齊哈爾達阪。北:廓喀沙勒山、戈什山、哈克善山。東北:貢古魯達阪、珍旦達阪、英阿瓦達山。托什罕河二源,一自伽師東北入,納上齊哈爾達阪水,合東流,納希布勒孔蓋河、玉簪河,至烏什莊,別疊水南流注之。又東為畢底爾河、貢古魯及可可容二水合為柳樹泉,南流注之,東流入溫宿。驛二:烏赤、洋海。回莊二十八。布魯特二部游牧地:曰奇里克,曰胡什齊。卡倫十三。境北及西北均界俄。自喀依車奇哈達阪,至齊恰爾達阪,立牌博六。

疏勒府:沖,繁,疲。喀什噶爾道治所。烏魯木齊提督同駐。舊喀什噶爾道徠寧城。喀什譯言「各色」,噶爾為「磚房」。漢,疏勒國地。永平中,龜茲並之,尋復立。元魏太延二年內屬。隋末屬西突厥。唐置佉沙都督府。宋開寶二年,並於於闐。南宋屬西遼。元至元二十五年,置達魯花赤,屯田於此,隸阿母河省。明為哈實哈兒國。明末瑪木特玉布素自亞剌伯來。奉回教。乾隆間,準噶爾汗囚其曾孫瑪罕木於伊犁,並其二子波羅泥都、霍集占。二十年,平伊犁,瑪罕木已死,定北將軍班第釋波羅泥都囚,使歸喀什噶爾統其眾,留霍集占於軍。旋逃至葉爾羌,據城叛。二十四年,將軍富德克之,阿渾以喀什噶爾降,始內屬。設參贊大臣,總辦天山南路八城事務。以阿奇木伯克理回務。領隊大臣、協辦大臣各一。專理喀什噶爾、英吉沙爾事務。總兵一。二十七年,於沽漊巴海築徠寧城。舊回城西北二里。三十一年,移參贊、協辦駐烏什,改設辦事大臣。五十二年,復設參贊、協辦。道光七年,於哈喇哈依築新城,名曰恢武。光緒九年,裁參贊、協辦,置直隸州。二十九年升府,增伽師,又巴楚州同隸。尋割巴楚屬莎車。東北距省治四千五百里。廣一千六百餘里,袤七百餘里。北極高三十九度二十五分。京師偏西四十二度二十五分。領縣二。烏蘭烏蘇河自府東流入,逕城南,復東北流入伽師。罕愛里克河、雅璊雅爾河亦自府東流入,注岳普爾湖之東庫山河。下游別什幹渠,自英吉沙爾東北流入,分數小水入沙磧。驛三:系弦、雅璊雅爾、雅卜藏。漢屯八,大小回莊六。布魯特部游牧地。卡倫二。咸豐元年,俄約定為商埠。疏附沖,繁,疲。府西北二十四里。舊回莊。光緒九年,劃烏蘭烏蘇河上游十一莊置。西北:烏孜別里山,為怱嶺、天山之過脈,蔥嶺支峰。西南:喀喇特山、瑪爾幹山、喀卜喀山、額依爾阿特山。南:烏魯瓦特山、阿依阿奇山、勒泰烏巴什山,皆在烏蘭烏蘇河南,天山支峰。西北:薩瓦雅爾德山、西康山、克子圖山、庫斯渾山、東克依克山,皆在烏蘭烏蘇河北。烏蘭烏蘇河源出蔥嶺,東流,納業金水、瑪爾堪蘇河、阿依阿奇水、庫斯渾水,逕城西南,一支渠東流入府,一東流,圖蘇克塔什河合察克瑪克河北來注之,入伽師。雅璊雅爾河自蒲犁北流入,東北出,一支渠入府。驛一,在城。回莊九。布魯特五部游牧地:曰胡什齊,曰沖巴噶什,曰嶽瓦什,曰希布察克,曰奈曼。卡倫三十三。牌博自西南烏孜別里山豁,至東北帖列克山豁,凡二十二。伽師沖,繁,難。府東一百六十里。漢,疏勒國地。唐,佉沙城。乾隆二十四年,設阿奇木伯克理回務。光緒二十九年,以牌素巴特回莊置。天山支脈迤邐北境。北:郭克阿勒山、以格孜達阪、阿奇克山、依提約爾山。西北:依得朗山、倫郭斯山。東:蘇潭山。烏蘭烏蘇河自境及疏附分支東流入境,逕城北,喀什噶爾河南流注之。又東北為二支,瀦為草湖,溢出復合,流入巴楚。驛五:在城、英阿瓦特、龍口橋、雅素里克、玉代里克。大回莊五。卡倫八。牌博自西北黑皮恰克,至烏圖魯達阪,凡五。

莎車府:沖,繁,難。隸喀什噶爾道。副將駐。舊葉爾羌回城。葉爾譯言「地」,羌謂「寬廣」也。漢,莎車國地。後漢並於於闐,元和後內附。三國屬疏勒。北魏為渠沙國,後疏勒並之。隋、唐至宋皆屬於闐。南宋屬西遼。元曰雅兒看,以封宗王阿魯忽。明曰葉爾羌,國最強。順治十三年,哈密、吐魯番入貢,其表均葉爾羌阿布都剌汗署名。康熙三十五年,破準噶爾,其王來朝,尋為準噶爾所阻。乾隆二十年,始內屬。二十四年,平霍集占,舊伯克回民以城降。二十六年,設辦事大臣、協辦大臣、兼領隊事務各一,副將一。道光八年,改參贊大臣,尋復舊制。光緒八年平回亂。九年,裁辦事、領隊大臣。二十四年,築新城,設直隸州。二十八年,升府。東北距省治四千七十三里。廣一千三百里,袤一千二百里。北極高三十八度十九分。京師偏西四十度十分。領一,州一,縣二。昆侖山脈綿亙本境。西南:協坦耿山、鐵格山、海立雅山。澤勒普善河自蒲犁東北流入,納喇斯庫木河,東北入巴楚。雜布河自葉城東北流入,逕別什幹莊,東流入葉城。驛四:在城、科科熱瓦牙、合哎勒克、和色爾。回城巡司一。大回莊十七。布魯特部游牧地。卡倫六。蒲犁府西南八百里。舊色勒庫爾地。漢,蒲犁、西夜、烏秅、依耐諸國地。後漢,德若國。魏,滿犁、億若二國,並屬疏勒。北魏及唐,喝盤陀國。宋、元,於闐國。明屬葉爾羌。順治後,為布魯特西部。光緒二十八年置。蔥嶺北支綿亙北境。東北有鐵里達阪。西北:克則勒借克山。西南:烏魯克瓦提達阪。南:喀楚特山。賽里河出南,喀楚特河東北流注之,北流至申底南,折東流,名托布隆河,納湯吉塔爾河,又東流入府,注澤勒普善河。奇盤河自葉城西北流入,合喇斯庫木河,折東北入府,為澤勒普善河。雅璊雅爾河自俄國東流入,納木吉河,東北流,入疏附。西北:愛南湖、喀喇庫湖、白希庫湖、布倫庫爾湖、霍什干大庫湖。驛十一:在城、申底、奇哈爾、塔爾拜什、托魯布倫、七里拱、拜塔、布達克、巴海開子、阿普里克、托乎拉克。大回莊二十七。布魯特部及塔吉克族游牧地。卡倫十一。巴楚州沖,繁,疲。府東二百四十里。漢,尉頭國。三國及北魏屬龜茲。隋入疏勒。唐,蔚頭州。宋屬疏勒。元、明,別失八里地。乾隆中內屬,設阿奇木伯克理回務。道光十二年,築城,設糧員。光緒九年置瑪喇巴什直隸,設水利撫民通判。二十九年改置,治巴爾楚克,易今名。天山支脈蜿蜒北境。東:烏果洛可山、覺裡孔山。南:克拉甫山。西:沙格山。澤勒普善河自莎車東北流入,合老玉河,為葉爾羌河,折東北,入溫宿。烏蘭烏蘇河合喀什噶爾河,自伽師東流入,至古鷹州城,折東北,注葉爾羌河,入溫宿。聽雜布河自葉城東北流入,分數支渠。又有蘇沙湖、咸海、故海、小海子。驛八:七臺、察巴克、圖木舒克、車底庫勒、雅哈爾庫圖克、色瓦特、屈爾蓋、卡拉克沁。回莊八十六。卡倫二。葉城沖,疲,難。府東南二百里。舊哈哈里克。漢,莎車、子合國地。後魏,渠沙、悉居半、硃俱諸國地。唐,沮渠、硃俱波西地,入于闐。明,葉爾羌。乾隆中內附。光緒九年,以哈哈里克置。蔥嶺北支綿亙縣境。有奇盤山、密爾岱山。南:瑪爾胡魯克山。西南:八沙拉達阪。東:玉拔達阪。奇盤河源出八沙拉達阪,北流,福新河自皮山西北流注之,又西北入蒲犁。聽雜布河為福新河分支,北流逕城西,又東北流入巴楚。驛二:哈哈里克、上波斯坎。大回莊十一。卡倫七。皮山沖,疲,難。府東南四百十里。舊侂瑪回莊。漢,皮山國地。後漢入于闐,尋復立。三國,皮山。北魏,蒲山。北周、隋、唐屬於闐。乾隆間內屬。光緒二十八年,於蘇各莊置澤普縣,尋移治侂瑪,易今名。蔥嶺山脈綿亙境內。南:卡拉胡魯木山、素蓋提山、桑珠山。東南:普下山、陽阿里克山、杜瓦山。福新河出卡拉胡魯木山,西北流,逕達爾烏孜莊入葉城。哈拉哈什河自和闐西北流入,瀦為別里克奇草湖,復溢出,東北流入和闐。驛五:侂瑪、淖洛克、木吉、裝桂雅、怕爾漫。大回莊四十三。卡倫六。縣南卡拉胡魯木山與英分界,立牌博一。

和闐直隸州:疲,難。和闐譯言「黑臺」,回人謂漢人也。隸喀什噶爾道。舊額里齊回城。漢,於闐國。後漢建武時,並於莎車,尋復立。北魏至唐,皆通朝貢。貞觀中,置毗沙都督府。儀鳳中,陷吐蕃,尋自立。後晉、後漢及北宋,朝貢不絕。南宋後,屬西遼。遼亡,屬乃蠻。元太祖九年,曷思麥裏殺乃蠻主內附。十六年,術赤取玉龍傑赤等城,後並可失哈兒、雅兒看,即莎車為三城,以封阿魯忽。至元初,阿魯忽叛。十六年,以忽必來別速臺為都元帥,戍斡端城,二十六年罷。明永樂四年入貢。明末並於回部。康熙中,入準噶爾。乾隆二十年,準部平,始內屬。二十四年,設辦事大臣、協辦大臣各一。駐伊爾齊,轄回城六,隸於葉爾羌大臣。又設都司一。光緒九年裁,設直隸州。東北距省治四千九百六十三里。廣二千三百里,袤一千二百里。北極高三十七度。京師偏西三十五度五十二分。領縣二。南:札客安巴山。西南:哈喇科隴山、尼蟒依山、阿拉克達阪、庫布哈達阪。東南:察察嶺、乙根達阪。東:卡浪古達阪、烏魯達阪。玉瓏哈什河源出尼蟒依山,西北流,至而梗勒司莊,折東北,納泥沙諸水,至八柵為州境,與洛浦分界,又東北與哈拉哈什河會。哈拉哈什河亦出尼蟒依山,西流,納庫布哈達阪水,折西北入皮山,復東北流,逕州境入洛浦。驛二:托彌、雜瓦。回莊二十九。於闐繁,難。州東四百六十里。漢,捍彌、渠勒、精絕、戎盧、且末、小宛諸國地。後漢為拘彌。北魏附蠕蠕。隋屬突厥。唐初為毗沙都督府地。儀鳳中,陷吐蕃,長壽時,復立國,屬於闐。石晉置紺州。宋仍屬於闐。南宋屬西遼,後屬乃蠻。元,阿魯忽封地。明並於回部。康熙時,屬準噶爾。乾隆二十年內屬。二十四年,設阿奇木伯克理回務。光緒九年置。治哈拉哈什,尋徙治克里雅。昆侖綿亙縣境。東南:昆折克圖拉爾山。東:蘇拉瓦克山大金廠,卡巴山小金廠。東北:阿里屯塔格山。西南:克里雅山、喀喇布拉克山、皮介山、阿羌山。卡墻河出縣東,納烏蘇克蘇水,阿克塔克水、阿里雅拉克水,西流,又納覺可沙衣水,跳提勒水,北流,分數渠,東北入婼羌。伊爾里克淖爾在縣西南,納阿羌山、皮介山諸小水,北流入沙磧。驛二:罕蘭、渠勒。回莊五十九。洛浦繁,難。州東七十里。光緒二十八年,析和闐東境玉河以東、于闐西境一根闌幹以西置。東南:鐵蓋列克山。玉瓏哈什河自州東北流,至八柵入,北流至塔瓦克,合哈拉哈什河,名和闐河,入溫宿,注塔裏木河。驛一:白石。回莊四十一。

英吉沙爾直隸:沖,繁,難。隸喀什噶爾道。故回莊。漢,依耐國地。後漢並於莎車。魏至隋,疏勒國地。唐,硃俱波國地。宋並於於闐。元為可失哈兒地,以封宗王。明末瑪木特玉素來自亞剌伯,遂為回教阿渾所居。乾隆二十四年,平霍集占,始內屬,定今名,英吉譯言「新」,沙爾,「城」也。設總兵。三十一年,設領隊大臣,隸喀什噶爾。光緒九年裁,置直隸。東北距省治四千二百七十四里。廣二百六十里,袤一百五十五里。北極高三十八度四十九分。京師偏西四十一度五十分。蔥嶺支脈環繞境東、西、南三面。西南:齊齊克山、鐵里達阪、哈拉山、哈什克素山、黑甲克山。西:科可山。西北:清氣山、佳音山。東南:黑子爾山。東:阿依普山。罕依拉克水源出齊齊克山,東北流,逕鐵列山,為庫山河。分二支,一繞城東南,又分為特爾木齊克河,折東北入沙磧。一東北流,為圖木舒河,旁分二支,一經城西南,瀦為阿哈海,溢出東流入沙磧,一經城北黃壤沙地,注英乙泉水。其正支東北流,又分二支,一與英乙泉水合,入草地,瀦為小湖,一逕阿克托八柵,為別什幹渠,東北至疏勒入沙磧。塔思滾水發源哈什克素山,東北流,分三支,至阿依普山麓入沙磧。東南,黑子爾泉、且木倫泉,均東北流,合為鐵列克水,入沙磧。驛三:依耐、托和布拉臺、黑子爾。回莊六十八。布魯特沖巴噶什等十四部游牧地。


\end{pinyinscope}