\article{列傳一百五十八}

\begin{pinyinscope}
顏伯燾怡良祁黃恩彤劉韻珂牛鑒

顏伯燾,字魯輿,廣東連平人,巡撫希深孫,總督檢子。嘉慶十九年進士,選庶吉士,授編修。道光二年,出為陜西延榆綏道、督糧道。歷陜西按察使,甘肅、直隸布政使。大軍征回疆,以轉運勞,賜花翎。署陜西巡撫。十七年,授雲南巡撫,改建滇池石徬,農田賴之。兼署云貴總督。伯燾累世膺疆寄,嫺習吏治,所至有聲。

二十年,擢閩浙總督。時定海已陷,伯燾至,劾水師提督陳階平於英兵前次攻廈門告病規避,又論琦善主款僨事,及林則徐守粵功罪。二十二年,奏請餉銀二百萬,造船募新兵及水勇八千,以備出洋禦敵。復疏陳廣東兵事,略曰:「閩、粵互為脣齒,呼吸相通。自正月虎門不守,粵事幾不可問。四月內夷船駛泊省西泥城,防勇望風潰遁,兵船被焚,砲臺棄去。當事者以洋銀六百萬元令知府餘保純重啗敵人,始允罷戰,猶報勝仗,指為就撫,以欺朝廷。夫撫非不可,然必痛剿之後,始能帖伏。今逆勢方張,資之庫藏,何不以養士卒?如謂曲徇商民所請,何不於誓師之始,申效死之義,與之同守?粵民非不可用,前有蕭關、三元裏等鄉數千人圍困義律,乃餘保純出城彈壓,始漸散去。保純以議撫之後,不應妄生枝節,是謂六百萬之資可以求安也。奕山、隆文已遠避數十里,楊芳,齊慎亦退入城。奕山、隆文等閱歷未深,楊芳年老耳聾,皆不足當重任。斯時惟有特簡親信重臣,督造船砲,用本省之人,作本省之兵,懸以重賞,未有不堪一用者。臣移駐廈門,督修戰具,但使船砲稍備,即當奮力攻擊,不敢老師糜餉,以取咎戾。」又薦裕謙、林則徐可任粵事。

伯燾主戰甚力,欲一當敵。七月,英兵三十餘艘犯廈門,投書索為外埠,即駛入攻擊,接戰,毀敵輪船一、兵艇五,敵遂聚攻砲臺,總兵江繼蕓、游擊凌志、都司張然、守備王世俊皆死之。伯燾所募水勇,以節餉議遣,未有安置。當戰時,呼噪應敵,英兵登岸,以臺砲回擊,廈門官署街市並毀,伯燾退保同安。英人得廈門不之守,越數日,移船赴浙洋,惟留數艘泊鼓浪嶼。詔斥不能豫防,倉猝失事,以廈門收復,免其治罪,議革職,從寬降三品頂戴留任。尋命侍郎端華至閩察勘,坐未能進剿罷職,時論仍右之。咸豐三年,召來京,將起用,道梗不得至,尋病卒。子鍾驥,宣統初,官至浙江布政使。

怡良,瓜爾佳氏,滿洲正紅旗人。刑部筆帖式,洊升員外郎。道光八年,出為廣東高州知府,調廣西南寧。歷云南鹽法道,山東鹽運使,安徽、江蘇按察使,江西、江蘇布政使。

十八年,擢廣東巡撫。禁煙事起,林則徐、鄧廷楨主之,怡良偕預其事。二十年,兼署粵海關監督。及琦善至,撤防議撫,疏請暫示羈縻,怡良及將軍阿精阿皆不列銜。二十一年正月,沙角、大角砲臺既失,琦善私許通商,並給香港,義律行文大鵬協撤回營汛。怡良疏陳曰:「自琦善到粵以後,辦理洋務,未經知會。忽聞傳說義律已在香港出示,令民人歸順彼國。提臣移咨副將鈔呈偽示,臣不勝駭異。大西洋自前明寄居澳門,相沿已久,均歸中國同知、縣丞管轄,議者猶以為非計。今英人竟占踞全島,去虎門甚近,片帆可到。沿海之地,防不勝防,犯法之徒,必以為藏納之藪,地方因之不靖,法律有所不行。更恐洋情反覆,要求不遂之時,仍以非禮相向,雖欲追悔,其何可及!聖慮周詳,無遠不照,何待臣鰓鰓過計。但忽聞海疆要地,外人公然主掌,天朝百姓,稱為英國之民,臣實不勝憤恨。一切駕馭機宜,臣無從悉其顛末。惟上年十二月二十八日欽奉諭旨,調集兵丁,預備進剿,並令琦善同林則徐、鄧廷楨妥辦,均經宣示。臣等請添募兵勇,固守虎門,防堵要隘。今英人窺伺多端,實有措手莫及之勢。不敢緘默,謹以上聞。」於是詔斥琦善專擅之罪,褫職逮治,怡良兼署總督。英兵尋陷虎門,命怡良會同參贊大臣楊芳進剿,合疏請許英屬港腳商船貿易,詔斥怠慢軍心,奪職留任。

是年秋,授欽差大臣,會辦福建軍務,署閩浙總督,尋實授。時英兵已去廈門,其留泊鼓浪嶼者僅數艘。及和議成,福州、廈門皆開口岸,命偕巡撫劉鴻翱議善後事宜,籌辦通商,兼署福州將軍。先是臺灣鎮、道禦敵,迭有擒斬,英人追訴其妄殺冒功,命怡良渡臺灣查辦,總兵達洪阿、道員姚瑩逮京。當和議初定,怡良不能為之剖雪,為時論所譏。二十三年,乞病歸。

咸豐二年,起授福州將軍,偕協辦大學士杜受田治山東賑務。三年,授兩江總督。江寧、鎮江已陷,暫駐常州。粵匪方熾,兵事由欽差大臣琦善、向榮主之,分駐大江南北。上海逆匪劉麗川踞城,連陷川沙、青浦、南匯、嘉定、寶山。麗川,粵人,商於滬。初起,冒用洋行公司鈐記出示,眾論洶洶,疑有通洋情事。怡良疏請閩、浙、江西絲茶暫行停運,使洋商失自然之利,急望克復,自能嚴斷濟賊。巡撫吉爾杭阿率兵進剿,逾年乃平。時各國因在廣東爭入城,與總督葉名琛齟,每赴上海有所陳議,諭怡良隨時妥辦,勿徇要求。

五年,粵匪攻金壇,遣總兵傅振邦、虎嵩林會西安將軍福興、漳州鎮總兵張國樑進剿,連捷,解圍。國樑進克東壩,福興與之不洽,詔怡良密察以聞。奏言:「國樑勇戰,福興所不及,人皆重張輕福。因有芥蒂,請分調以免貽誤。」尋命福興赴江西剿賊。大軍圍江寧,久無功,賊勢益蔓。七年,以病請解,允之。同治六年,卒。

祁,字竹軒,山西高平人。嘉慶元年進士,授刑部主事,遷員外郎。督廣西學政,任滿補原官。以承審宗室敏學獄不實,褫職。尋予刑部七品小京官,累遷郎中。道光四年,出為河南糧鹽道。遷浙江按察使,覆檢德清徐倪氏獄,得官吏受賄蒙蔽狀,尚書王鼎覆訊,如議。遷貴州布政使。九年,召授刑部侍郎。尋出為廣西巡撫。十二年,湖南、廣東瑤匪並起,遣兵防富川、恭城、賀縣,搜捕竄匪,追擊於芳林渡,斬擒千餘。瑤平,加太子少保。疏陳善後策,扼要移駐文武,稽查化導,如所議行。十三年,調廣東巡撫。時盧坤為總督,和衷撫馭,籌修海防。十五年,代坤兼署總督。十八年,召為刑部尚書。宣宗知習練法律,故有此授。京察,被議敘。

二十一年,靖逆將軍奕山督師廣東,命往治餉。琦善既黜,授兩廣總督。時英兵踞虎門,省城遷避過半,示以鎮靜,稍稍安集。參贊大臣楊芳主持重勿浪戰,奕山為其下所慫恿,商之。以敵方恣哃喝,大軍新至,乘銳而用,冀挫其焰,未阻止,遂突攻英艦於省河,敵猝未備,義律夜遁。遲明,英兵大至,逼砲臺,守兵潰,英兵進踞城北耆定臺,高瞰城中。與巡撫怡良亟守西南兩門,城外市屋盡毀,客兵皆撤入城。商民知兵不足恃,環請為目前計,款議遂決,予洋銀六百萬元。英艦退出虎門,而耆定臺兵未去,船泊泥城,登岸侵擾,其兵目伯麥闖入三元里,民憤,磔之。義律馳救,受圍,遣廣州知府餘保純護之出,令率眾盡退虎門外。於是鄉團日盛,紳士黃培芳、餘廷槐等合南海、番禺諸鄉立七社,萬人一呼而集儲穀十餘萬石,不動官帑。用林則徐堵塞省河之法,以資守御。

是年夏,英人交還虎門砲臺,偕奕山疏陳:「現練水陸義勇三萬六千餘名,並各鄉丁壯,分成團練。前調各省官兵,遵旨陸續分撤。」詔促規復香港,責與奕山各抒所見。奏:「欲收復香港,必先修虎門砲臺,然非設險省河,虎門亦難興工。先於獅子洋、蚺蛇洞諸要隘築堡守戍。」疏上,報聞。是時粵師實無力進剿,英人既得賂而去,兵勢趨重江、浙,得以茍安。奕山屢被嚴詰,麾下招誘海盜,獻計襲攻敵艦,奕山又為所動,勸寢其議。

二十二年,和議成,英商開市益驕,民怨益深,焚其館,擲貨於衢,濮鼎查責言,撫慰之,得無事。二十三年,虎門砲臺工竣,疏言:「舊式砲臺僅可御海盜,今仿洋法,以三合土築人字形,砲墻量宜增移改建。」又請就海壖圍沙成田一百六十餘頃,可給屯丁二千人,且耕且守防要隘。並陳粵民義奮、團練可用狀,諭責事期經久,俾濟實用。以病乞休,累疏乃得請。二十四年,卒,優詔依尚書例賜恤,謚文恪。

黃恩彤,字石琴,山東寧陽人。道光六年進士,授刑部主事,治獄數有平反。充提牢,以疏防越獄降調,尋復之。充熱河理刑司員,卻翁牛特蒙古公賄,黜其爵。累遷郎中。二十年,出為江南鹽巡道,遷按察使,署江寧布政使。英兵犯江寧,耆英、伊里布令恩彤偕侍衛咸齡赴敵艦議款,隨同定約。事竣,復隨伊里布赴廣東,籌議通商。改番舶互市歸官辦,增減稅則,稽查偷漏,悉由恩彤與粵海關監督文豐商定。調廣東按察使,遷布政使。美利堅人顧盛請入京,恩彤赴澳門辯折,止其行,賜花翎。

二十五年,就擢巡撫。恩彤疏陳洋務,略曰:「欲靖外侮,先防內變。粵民性情剽悍,難與爭鋒,亦難與持久。未可因三元里一戰,遽信為民足禦侮也。該夷現雖釋怨就撫,而一切駕馭之方與防備之具,不可一日不講。但當示以恩信,妥為羈縻,一面慎固海防,簡練軍實。尤必撫柔我民,所欲與聚,所惡勿施,以固人心而維邦本。庶在我有隱然之威,因以折彼囂凌之氣。」疏入,上韙之。尋屆京察,與耆英並被議敘。籌備海防,裁虎門屯丁,以沙田租稅充戰船砲臺歲修之費。二十六年,英人爭入城,議久不決,粵民憤不可諭,恩彤前疏不為時論所與,被劾。會監臨文武鄉試,疏請年老武生給予武職虛銜,詔斥其違例,褫職,交耆英差遣。尋以同知銓選。

二十九年,告養歸。咸豐初,在籍治團練。天津議和,命隨耆英往,恩彤至,則款議已定,仍請終養,同治中,以御捻匪功,予三品封典。光緒七年,鄉舉重逢,加二品銜。尋卒。

劉韻珂,字玉坡,山東汶上人。由拔貢授刑部七品小京官,洊遷郎中。道光八年,出為安徽徽州知府,調安慶。歷云南鹽法道,浙江、廣西按察使,四川布政使。二十年,擢浙江巡撫。定海已陷,韻珂於寧波收撫難民。沿海設防,欽差大臣伊里布駐鎮海督師,琦善方議以香港易還定海,韻珂疏言:「定海為通洋適中之地,英人已築砲臺、開河道,經營一切。彼或餌漁,盜為羽翼,其患非小。浙江為財賦之區,寧波又為浙省菁華所在,宜預杜覬覦。」尋詔斥伊里布附和琦善,罷去,以裕謙代之,命韻珂偕提督餘步雲治鎮海防務。二十一年,英兵退出定海,仍游奕浙洋,裕謙督師赴剿。定海再陷,鎮海、寧波相繼失守,裕謙死之。韻珂檄在籍布政使鄭祖琛率師扼曹娥江,總兵李廷揚、按察使蔣文慶、道員鹿澤良駐防紹興,募勇二萬人守省城,庀守具,清內奸,撫沙匪十麻子投誠效用,人心以安。英艦窺錢塘江,尋退去。揚威將軍奕經援浙。

二十二年春,規復寧波,不克,擾及奉化、慈谿,戰數不利,命韻珂偕欽差大臣耆英籌辦防務。韻珂疏言:「浙事有十可慮,皆必然之患,無可解之憂,若不早為籌畫,國家大事豈容屢誤?現在奕經赴海寧查看海口,文蔚留駐紹興調置前路防守,究竟此後作何籌辦,奕經等亦無定見。臣若不直陳,後日倘省垣不守,粉身碎骨,難蓋前愆。伏乞俯念浙省危急,獨操乾斷,飭令將軍等隨機應變,俾浙省危而復安,天下胥受其福。」又力薦伊里布「不貪功、不好名,為洋人所感戴。其家人張喜亦可用。儻令來浙,或英兵不復內犯。」疏入,上頗採其言,命伊里布隨耆英赴浙,相機辦理。

四月,乍浦陷,伊里布往說英人退兵,於是改犯吳淞,入大江,乃於江寧定和議。韻珂貽書耆英、伊里布等曰:「撫局既定,後患頗多,有不能不鰓鰓過慮者。英船散處粵、閩、浙、蘇較多,其中有他國糾約前來者,粵東又有新到。倘退兵之後,或有他出效尤,或即英人託名復出,別肆要求,變幻莫測。此不可不慮者一也。洋人在粵,曾經就撫,迨給銀後,滋擾不休,反覆性成,前車可鑒。或復稱國主之言,謂馬、郭辦理不善,撤回本國,別生枝節。此不可不慮者二也。上所獲之郭逆義子陳祿,皆云雖給銀割地,決不肯不往天津,而現索馬頭不及天津,殊為可疑。能杜其北上之心,方免事後之悔。此不可不慮者三也。通商既定,自必明立章程,各省關口應輸稅課,萬一洋人仍向商船攔阻,勢不能聽其病商攘課,一經阻止,又啟釁端。此不可不慮者四也。民人與洋人獄訟,應聽有司訊斷,萬一抗不交犯,又如粵東林如美之案,何以戢外暴而定民心?此不可不慮者五也。罷兵之後,各處海口仍須設防,修造戰船砲臺,添設兵伍營卡,倘洋人猜疑阻擾,以致海防不能整頓。此不可不慮者六也。今日漢奸盡為彼用,一經通商,須治奸民。內地民人投往者,應令全數交出,聽候安插。否則介夫洋漢之間,勢必恃洋犯法,不逞之徒,又將投入,官法難施,必尋釁隙。此不可不慮者七也。既定馬頭,除通商地面不容泊岸,倘有任意闖入,取掠牲畜婦女,民人不平,糾合抗拒,彼必歸咎於官,而興問罪之師。此不可不慮者八也。名曰通商,本非割地,而定海拆毀城垣,建造洋樓,挈眷居住,倘各省均如此,恐非通商體制,腹內之地,舉以畀人,轉瞬即非我有。此不可不慮者九也。中國凋敝,由於漏銀出洋。今各省有洋船,漏銀更甚,大利之源,勢將立竭。會子、交子之弊政將行,國用、民用之生計已絕。此不可不慮者十也。至於議給之款,各省分撥。浙省自軍興以來,商民捐餉賑災,寧波菁華為洋人搜括,歲事歉收,責以賂敵之款,勢必不應。若如四川之議增糧賦,江、浙萬不能行。故剿敵之款可捐,賂敵之款不可捐,他省完善之地可捐,浙省殘破之餘不可捐。惟亮虓之!」所言並切利害。

韻珂機警多智,數見浙兵不可恃,以戰事委之裕謙、奕經,專固省防,浙人德之。及事急,再創調停之說,而慮成議於浙,為天下詬,移禍於江蘇。然世多譏其巧於趨避。二十三年,擢閩浙總督。疏言:「浙江舊未與外洋交易,與廣東情事不同。應於耆英等所議章程稍加變通,先申要約。」又籌海疆善後事宜二十四則,下議行。二十四年,疏報廈門開市,鼓浪嶼尚有英兵棲止,恐久假不歸,請諭禁,與領事面訂預杜偷漏稽查洋眾條款。又奏天主教流弊,請稽查傳教之地,不令藏奸;或有藉端滋事,據事懲辦,不牽及習教,俾無藉口。

二十五年,英人始至福州,請於南臺及城內烏石山建洋樓,韻珂難之。士紳見廣東爭議久不決,亦援以拒。英人訴諸耆英,謂不踐原約,則鼓浪嶼且不退還,往復辯論,卒不能阻,而閩人歸咎於韻珂。三十年,文宗即位,以病乞假,特旨罷職回籍。咸豐二年,坐泉州經歷何士邠犯贓逃逸,追論寬縱,褫職。同治初,召來京,以三品京堂候補。復乞病歸,卒於家。

牛鑒,字鏡堂,甘肅武威人。嘉慶十九年進士,選庶吉士,授編修。遷御史、給事中。道光十一年,出為雲南糧儲道。歷山東按察使、順天府尹、陜西布政使,與巡撫不合,乞病歸。十八年,起授江蘇布政使,署巡撫。

十九年,擢河南巡撫。整頓吏治,停分發,止攤捐;籌銀二十萬兩,津貼瘠累十五縣;築沁河堤,濬衛河:甚有政聲。二十一年六月,河決祥符,水圍省城。鑒率吏民葺城以守,規地勢洩水,賑撫災黎。時水分二流,一環城西南,一由東南行,均注歸德、陳州,入江南境。鑒以正河斷流,決口難遽塞,議急衛省城。水漲不已,西北隅尤當沖,城垣坍陷十餘處,拋磚石成壩,絙鉅舟以御之。奇險迭出,晝夜臨陴,民感其誠,同心守護,有不受雇值者。當事急,河督文沖奏省城卑濕不可復居,請擇地遷移。鑒疏言:「一月以來,困守危城,幸保無虞者,實由人心維系。若一聞遷徙。各自逃生,誰與防守?恐遷徙未及,水已灌城,變生俄頃,奸民乘機搶掠,法令不行,情狀不堪設想。節交白露,水將漸消,惟有殫竭血誠,堅忍守御,但得料物應手,自可化險為平。」命大學士王鼎、侍郎慧成往勘。鑒與合疏言省城可守不可遷,決口可堵不可漫,並劾文沖漠視延誤狀,於是褫文沖職。秸料大集,繕治堤壩,水亦漸退,守城凡六十餘日而卒完。命偕王鼎等興工塞決。

會英兵犯浙江,裕謙殉於寧波,命鑒代署兩江總督,尋實授。十月,至蘇州受事,閱海口,偕提督陳化成治防,繕臺增砲,沿海以土塘為蔽,駐四營居中策應。二十二年四月,英兵既陷乍浦,遂窺吳淞口。五月,敵艦七十餘艘來攻,鑒偕化成督戰,擊沉賊船三,西砲臺及戰艦皆被毀。敵以小舟載兵由小沙背登陸,徐州鎮總兵王志元兵先潰,化成死之。鑒退嘉定,而寶山、上海相繼陷。又退昆山,收集潰兵。壽春鎮總兵尤渤守松江,敵兩次來犯,皆擊卻之。英艦聚泊吳淞口外,揚言將北犯天津。六月,突入江,乘潮上駛,直越圌山關,鑒由京口退保江寧。提督齊慎、劉允孝迎戰京口,不利,退守新豐。鎮江陷,副都統海齡死之。敵艦分薄瓜洲,揚州震動,鹽運使但明倫聽商人江壽民計,賂以六十萬金,遂犯江寧,艦泊下關。

鑒初專防海口,倚陳化成,沿江鵝鼻嘴、圌山關諸要隘倉猝調兵,益無足恃。化成既死事,鑒知不能復戰,連疏請議撫。耆英、伊里布先後奉命至,英人索五處通商及償款,諸臣未敢遽允;敵兵遂登岸,置大砲臨城,乃悉許之。合疏以保全民命為請,略曰:「江寧危急,呼吸可虞,根本一摧,鄰近皖、贛、鄂、湘皆可航溯。彼所請雖無厭,而通市外無他圖。與其結兵禍而毒生靈,曷若捐鉅帑以全大局?廈門敵軍雖退,尚未收復。香港、鼓浪嶼、定海、招寶山仍未退還,使任其久踞逡巡,不如歸我土地。既原循例輸稅,即為悔禍鄉風。此後彼因自護租岸,我即以捍蔽海疆,未始非國家之福。所請平禮虛文,不妨假借。事定之後,亦應釋俘囚以堅和好,寬脅從以安反側。」並附詳條目以聞。八月,和議成,英兵悉退出海洋。

尋以貽誤封疆罪,褫職逮問,讞大闢,二十四年,釋之,命赴河南中牟河工效力。工竣,予七品頂戴,以六部主事用,回籍。咸豐三年,粵匪北擾,予五品頂戴,署河南按察使。四年,命卸任,勸捐募勇,赴陳州,偕徐廣縉剿捻匪,破潁州賊李士林於阜陽方家集,焚其巢,加按察使銜。五年,又破之於霍丘三河,士林尋於湖北就撫。鑒深得河南民心,前勸捐中牟大工,得錢二百萬緡,至是集軍餉復及百萬。敘功,加二品頂戴。以病乞歸。八年,卒。

論曰:顏伯燾懷抱忠憤,而無克敵致果之具。怡良不附和琦善,亦無建樹。祁依違和戰之間,茍全而已。劉韻珂以術馭人,陰主和議。牛鑒以循吏處危疆,身敗名裂。要之籌邊大計,朝廷無成算,則膺封圻之寄者為益難,況人事之未盡乎?嗚呼!論世者當觀其微也。


\end{pinyinscope}