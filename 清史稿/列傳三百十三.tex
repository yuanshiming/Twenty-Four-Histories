\article{列傳三百十三}

\begin{pinyinscope}
屬國一

○朝鮮琉球

有清龍興長白,撫有蒙古,列為籓封。當時用兵中原,而朝鮮服屬有明,近在肘腋,屢抗王師。崇德二年,再入其都,國王面縛納質,永為臣僕,自此東顧無憂,專力中夏。

順治紹明,威震殊方。三年,琉球聞聲,首先請封。九年,暹羅,十七年,安南,相繼歸附。雍正四年,蘇祿,七年,南掌,先後入貢。蓋其時武義璜璜,陸懾水慄,殊國絕域,交臂詘膝,慕義歸化,非以力征也。

高宗繼統,國益富饒,帝喜遠略,蕩平回疆,兵不血刃,而浩罕、布魯特、哈薩克、安集延、瑪爾噶朗、那木干、塔什干、巴達克山、博羅爾、阿富汗、坎巨提相率款塞,通譯四萬,舉踵來王。乾隆中葉,再徵緬甸,三十四年,緬懼乞貢。五十七年,復徵服廓爾喀,稽首稱籓。於是環列中土諸邦,悉為屬國,版圖式廓,邊備積完,芒芒聖德,蓋秦、漢以來未之有也。

咸、同之際,內亂頻仍,撻伐十餘年,巨憝雖平,而國力凋τ,未遑圖遠。日夷琉球,英滅緬甸,中國雖抗辭詰問,莫拯其亡。而越南、朝鮮政紛亂作,國家素守羈縻屬國之策,不干內政,興衰治亂,袖手膜視,以至越南亡於法,朝鮮並於日,浩罕之屬蠶食於俄,而屬國所廑存者,坎巨提一隅而已。越南、朝鮮之役,中國胥為出兵,而和戰無常,國威掃地,籓籬撤而堂室危,外敵逼而內訌起,籓屬之系於國也如此。傳曰:「天子守在四夷。」詎不信哉?作屬國傳。

朝鮮又稱韓國。清初王朝鮮者李琿,事明甚謹。太祖天命四年,琿遣其將姜弘立率師助明來侵,軍富察之野,戰而大敗,姜弘立以兵五千降。帝留弘立,遣其部將張應京等十餘人還國,遺琿書曰:「昔爾國遭倭難,明以兵救爾,故爾國亦以兵助明,勢不得已,非與我有怨也。今所擒將吏,以王之故,悉釋還國。去就之機,王其審所擇焉。」先是明萬歷中,日本豐臣秀吉大舉侵朝鮮,覆其八道,明為用兵七年。會秀吉死,兵罷,朝鮮乃復國,故書中及之。朝鮮不報謝。又出境拒征瓦爾喀之師。烏拉貝勒布占泰侵朝鮮,帝與布占泰有連,諭止其兵,朝鮮亦不謝。及帝崩,復不遣使吊問。而明總兵毛文龍招遼民數萬守皮島,與朝鮮犄角,屢出師襲沿海城寨。

會朝鮮叛人韓潤、鄭梅來歸,請為鄉導,構兵端。時太宗天聰元年,朝鮮國王李倧嗣位之三年也。正月,命貝勒阿敏等率師征朝鮮。渡鴨綠江,敗文龍兵於鐵山,遁還皮島。遂克義州、定州及漢山城,屠其軍民數萬,焚糧百餘萬。長驅而進,渡青泉江,克安州,進師平壤,城中官民番遁走。乃渡大同江,次中和。倧惶遽甚,遣使求成,阿敏責數其罪。二月,師次黃州,國中震恐,求成之使絡繹於道,遂逼王京。倧勢蹙,挈妻子遁江華島,來告曰:「敝邑無所逃罪,惟上國命是從。」乃許其和。江華島在開州南海中,遣使赴島諭倧,而駐軍平山以待。倧遣族弟原昌君李覺等獻馬百、虎豹皮百、綿綢苧布四百、布萬有五千,於是遣劉興祚、巴克什庫爾纏往江華島蒞盟。三月庚午,刑白馬烏牛,誓告天地。和議成,約為兄弟之國。

初,朝鮮之求成也,諸貝勒等議以明與蒙古兩敵環伺,兵不可久在外,且俘獲已多,宜許其成。而阿敏慕朝鮮國都城郭宮殿之壯,不肯旋師。貝勒濟爾哈朗及岳託、碩託密儀,令阿敏軍平山,而先與朝鮮盟,事成始告阿敏。阿敏謂己不預盟,縱兵四掠,乃復使李覺與阿敏盟於平壤城。帝馳諭阿敏:「毋復秋毫擾!」分兵三千戍義州,振旅而還,以李覺歸。九月,從倧請,召還義州之兵,並許贖俘虜,定議春秋輸歲幣、互市。

二年二月,開市中江。是年,明經略袁崇煥殺毛文龍於皮島,諸島兵無主。五年,謀乘虛徵諸島,徵兵船於朝鮮。使至其國,三日乃見。倧覽書曰:「明國猶吾父也。助人攻吾父之國,可乎?船殆不可藉也。」自是漸渝盟。六年,巴都禮、察哈喇等使朝鮮,頒定貢額。還言倧於所定貢額止供什一,金銀、牛角非國所出,不肯從。七年正月,賜倧書,責其減歲幣額,並竊葠畜、匿逃人之罪,欲罷遣使,專互市。二月,遣備御郎格等往會寧城互市,倧拒之。是夏,文龍部將孔有德、耿仲明等叛明,以舟師二萬人渡海來降,帝遣使徵糧朝鮮,並索會寧城瓦爾喀逃人及布占泰之人,倧屢書陳辯,復加築京畿、黃海、平安三道白馬等十二城。帝歷數倧負義州互市之約。八年春,帝欲價倧與明議和,倧以書告皮島守將,迄無成議。冬,倧使羅德憲來,拒索逃人及互市,詞甚厲,且欲坐滿洲使臣於朝鮮大臣之下。帝怒,卻其幣,留德憲不遣,仍以書諭倧。

九年,平察哈爾林丹汗,得元傳國璽,八和碩貝勒及外籓蒙古四十九貝勒表請上尊號。帝曰:「朝鮮兄弟之國,宜與共議之。」於是內外諸貝勒各修書遣使約朝鮮共推戴,朝鮮諸臣爭言不可,且以兵守使臣。使臣英俄爾岱率並奪馬突門,倧遣人追付報書,又以書諭其邊臣戒嚴,有「丁卯年誤與講和,今當決絕」之語,英俄爾岱眾奪之以獻。十年四月,改國號「清」。朝鮮使李廓等來朝賀,不拜。賜書令送質子,復不報。

十一月,帝以朝鮮敗盟,將統大軍親征。先遣其使臣李廓等歸國,遺書國王,並馳檄朝鮮官民。十二月辛未朔,命鄭親王濟爾哈朗居守,武英郡王阿濟格、多羅饒餘貝勒阿巴泰分屯遼河海口,備明海師援襲之路。睿親王多爾袞、貝勒豪格分統左翼滿洲、蒙古兵,從寬甸入長山口,遣戶部承政馬福塔等率兵三百人潛往圍朝鮮王京,豫親王率護軍千人繼之。貝勒岳託等以兵三千濟師。帝親率禮親王代善諸軍進發。庚辰,渡鎮江。壬午,次郭山城,降定州、安州。丁酉,次臨津江。江在國都北百餘里,與都南漢江夾拱王城者也。時江冰未合,車駕至,冰驟堅,六師畢濟。馬福塔等以是月甲申潛襲王京,敗其精兵數千,倧倉皇遣使迎勞城外款兵,而徙其妻子江華島,自率親兵逾江保南漢山城。大軍入都城,多鐸、岳託亦定平壤,抵王京,合軍渡江圍南漢山城,連敗其諸道援師。帝至,分兵搜剿都城,而親率大軍渡江,益軍圍南漢。二年正月壬寅,擊敗全羅道援兵,遣使齎敕往諭朝鮮大臣。甲辰,大軍北渡漢江,營王京東二十里江岸。丁未,擊敗全羅、忠清兩道之師。其多爾袞、豪格左翼軍由長山口克昌州城,敗安州、黃州兵五百,寧邊城兵千,截殺援兵一萬五千,至是來會師。貝勒杜度送大砲至臨津江,冰泮復合如前。

城圍益急。癸丑,倧請成,不許。己未,再請成。庚申,降。敕令出城親覲,並縛獻倡議敗盟諸臣。是日,倧始奏書稱臣,乞免出城。帝命多爾袞以輪挽小船由陸出海,砲沉其大艦三十。小船徑渡入島城,獲王妃、王子、宗室七十六人,群臣家口百六十有六,客諸別室。甲子,諭倧速遵前詔出城來見。倧乃獻出倡議敗盟之弘文館校理尹集、修撰吳達濟及臺諫官洪翼漢,詣軍前。帝敕令去明年號,納明所賜誥命冊印,質二子,奉大清國正朔;萬壽節及中宮皇子千秋、冬至、元旦及諸慶吊事,俱行貢獻禮;遣大臣內官奉表、與使臣相見及陪臣謁見、並迎送饋使之禮,毋違明國舊例;有征伐調兵扈從,並獻犒師禮物;毋擅築城垣;毋擅收逃人;每年進貢一次,其方物黃金百兩、白金千兩、水牛角二百對、貂皮百張、鹿皮百張、茶千包、水獺皮四百張、青黍皮三百張、胡椒十斗、腰刀二十六口、順刀二十口、蘇木二百斤、大紙千卷、小紙千五百卷、五爪龍席四領、花席四十領、白苧布二百疋、綿綢二千疋、細麻布四百疋、細布萬疋、布四千疋、米萬包。

倧以孤城窮蹙,妻子被俘,八道兵皆崩潰離散,宗社垂絕,乃頓首受命。庚午,從數十騎朝服出降。二月,築壇漢江東岸三田渡,設黃幄,帝陳儀衛渡江,登壇作樂,將士擐甲肅列。倧率其群臣離南漢山五里許步行,令英俄爾岱、馬福塔迎於一里外,引至儀仗下立。帝降坐,率倧及其諸子拜天。禮畢,帝還坐,倧率其屬伏地請罪,宣詔赦之,令坐壇下左側西向,位諸王上。賜宴畢,還其君臣家屬,盡召回諸道兵,振旅而西。詔以朝鮮新被兵,先免丁丑、戊寅兩年貢物,以己卯年秋為始,如力有不逮,臨時定奪。朝鮮臣民樹碑頌德於三田渡壇下。

四月,倧送質子澂、淏至。五月,以朝鮮兵船助攻皮島功,賜倧銀幣、馬匹。十月,遣英俄爾岱、馬福塔、達雲等齎敕印制詔往封倧為朝鮮國王。十一月,倧遣陪臣表賀萬壽,冬至貢方物。十二月,賀元旦。嗣凡萬壽聖節、元旦、冬至,皆專遣陪臣表賀,貢方物,歲以為常。是年,定貢道,由鳳凰城。其互市約:凡鳳凰城諸處官員人等往義州市易者,每年定限二次,春季二月,秋季八月;寧古塔人往會寧市易者,每年一次;庫爾喀人往慶源市易者,每二年一次;由部差朝鮮通事官二人,寧古塔官驍騎校、筆帖式各一人,前往監視,定限二十日即回。

三年,徵朝鮮兵從征明,誤軍期,降詔切責。四年六月,遣使往封倧繼室趙氏為朝鮮王妃。東方庫爾喀叛入東海中熊島,命朝鮮討之。倧遣將由慶興西水羅前浦進師。七月,執叛首加哈禪來獻,賜倧銀二百兩。五年十月,諭倧以誕辰,恩減歲貢內米九千包。六年正月,攻明錦州,調朝鮮舟五千運糧萬石。尋倧奏言軍船、糧船三十二艘漂沒無存,帝知其飾詞,詔切責,刻期督催。復運糧萬石,船百十有五艘,由大小凌河口進至三山島,途中遭風礁壞船五十餘,又為明水師截擊,僅存五十二艘。至蓋州,不能前,請從陸運。詔以朝鮮三艘漂入明境通信,及見明兵船不迎敵,又不由水路進,嚴斥之。朝鮮臣林慶業大懼,請冒險出水路,帝仍許其改從陸,止留精砲兵千,廝卒五百,餘兵悉遣還。既而運糧士馬久不至,遣使詰責。三月,始有朝鮮總兵柳琳、副將刁何良等率兵至錦州軍。六月,倧遣陪臣李浣等獻新羅瑞金,奏言咸陽郡新溪書院,新羅古寺遺基也,居民袁年掘地得瓦罈一,蓋刻「一千年」三字,中有黃金二十斤,內一斤鐫「宜春大吉」四字。優詔答之,而原金付還。七年,錦州大捷,明遣使議和,帝敕詢倧令陳所見,倧以「止殺安民,上符天意」對。已復偵有明兵船二至朝鮮界,帝大怒,並得其閣臣崔鳴吉、兵使林慶業潛通明國書往來諸狀,逮訊治罪。八年九月,朝鮮擒獲明天津偵探兵船一,解至,賜倧銀。

是月,世祖即位,頒詔其國,並齎敕往諭,減歲貢內紅綠綿綢各五十疋、白綿綢五百疋、紵絲二百疋、布二百疋、腰刀六口、龍席二領,花席二十領。十月,倧遣其子橑奉表進香,貢方物。十二月,倧遣陪臣奉表賀登極。順治元年正月,諭倧停解瓦爾喀人民。五月,以破流賊李自成,底定燕京,宣示朝鮮。七月,倧遣陪臣表賀,貢方物。十一月,遣世子澂歸國,敕減歲貢內蘇木二百斤、茶十包、綿綢千疋、各色細布五千疋、布四百疋、粗布二千疋、順刀十把、刀十把,其元旦、冬至、萬壽慶賀貢物,以道遠俱於朝正時附進,著為令。二年三月,遣倧次子淏歸國。十一月,世子澂卒,封倧次子淏為世子。三年十月,免貢米。六年正月,以朝鮮年覲,原定閣臣、尚書各一員,書狀官一員代之,此後或閣臣、尚書一員代覲,書狀官仍舊。

六月,李倧薨。八月,遣禮臣啟心郎渥赫等往諭祭,賜謚莊穆。又遣戶部啟心郎布丹、侍衛撤爾岱充正副使,齎誥敕往封世子淏為朝鮮國王,妻張氏為王妃。七年正月,淏奏言日本「近以密書示通事,情形可畏,請築城訓練為守禦計」。遣使往訊,慶尚道觀察使李、東萊府盧協並言朝鮮、日本素和好,前奏不實,詔切責淏,褫其用事臣李敬輿、李景奭、趙洞等職。九年正月,淏表賀昭聖慈壽皇太后加上徽號。五月,國人趙照元等謀逆伏誅,遣使奏聞。十年三月,以朝鮮國王印有清文無漢篆,命禮部改鑄兼清、漢字印賜之。十二月,封淏子■H7為世子。十五年二月,以羅剎犯邊,諭朝鮮簡發鳥槍手二百從征。

十六年五月,李淏薨。九月,遣工部尚書郭科等往諭祭,賜謚忠宣。又遣大學士蔣赫德、吏部侍郎覺羅博碩會充正副使,往封世子■H7為朝鮮國王,妻金氏為王妃。十八年,聖祖即位,■H7遣陪臣進香,賀登極。康熙元年,命朝鮮表賀冬至、萬壽節及進歲貢,與朝正之使偕行。屢年國有大典,俱遣使朝賀。

十三年十二月,李■H7薨,諭禮部:「李■H7克盡籓職,可從優給恤典,於常例外加祭一次。」賜謚莊恪。遣內大臣壽西特、侍衛桑厄恩克往諭祭,兼封嗣子李焞為朝鮮國王,妻金氏為王妃。十五年十一月,焞奏言:「前明十六朝紀一書中載本國癸亥年廢光海君李琿立莊穆王李倧事,誣以篡逆。今聞纂修明史,特陳奏始末,乞刪改以昭信史。」禮部議不準行。二十年正月,王妃金氏故,遣官致祭。二十一年五月,遣使封焞繼室閔氏為王妃。是年,帝謁祖陵,焞遣陪臣至盛京迎覲,貢方物。二十四年,焞奏言國內牛多疫死,民失耕種,請暫停互市。禮部議焞託言妄奏,帝以外籓宥之,仍令照常貿易。

二十五年,朝鮮民韓得完等二十八人越江採葠,槍傷繪畫輿圖官役。讞上,斬韓得完等為首六人,餘免死,減等發落。焞奉表謝罪,附貢方物。帝以朝鮮王因謝罪進貢,宜不收,準作年貢,嗣後謝罪貢物著停止。三十年七月,禮臣奏朝鮮國貢使違禁私買一統志書,內通官張燦應革職發邊界充軍,正使李沈、副使徐文重等失於覺察,應革職。帝命從寬,免革職。三十二年正月,免朝鮮歲貢內黃金百兩及藍青紅木棉。

三十六年七月,封焞子昀為世子。十一月,焞疏請於中江貿易米糧,允之。三十七年正月,遣侍郎陶岱運米三萬石往朝鮮,以一萬石賑濟,二萬石平糶,有禦制海運賑濟朝鮮記。三十九年,焞表謝發回漂入琉球船隻恩,附貢方物。帝諭軫恤漂人,卻貢物,嗣後有若此例者停其貢。四十年十二月,王妃閔氏故,遣官致祭。先是漁採船並貿易人至朝鮮,往往侵擾地方。至是諭王令查驗船票人數姓名籍貫,開明報部,轉行原籍地方官,從重治罪。並諭各撫嚴飭沿海地方官,有以海上漁採貿易為名,往來外國販買違禁貨物者,嚴行禁止。四十一年,遣員外郎鄧德監收中江稅,以四千兩為額。四十二年二月,遣使封焞繼室金氏為王妃。四十三年十二月,焞遣官資送被風漂失商船,降諭褒之。四十五年十月,諭大學士曰:「朝鮮國王奉事我朝,小心敬慎。其國聞有八道,北道接瓦爾喀地方土門江,東道接倭子國,西道接我鳳凰城,南道接海外,尚有數小島。太宗平定朝鮮,國人樹碑於駐軍之地,頌德至今。當明之末年,彼始終服事,未嘗叛離,實屬重禮義之邦,尤為可取。」四十九年五月,朝鮮商人高道弼等被風壞船,漂至海州獲救,江蘇巡撫張伯行以聞。諭令高道弼等由部給文,馳驛歸國。

五十年五月,帝諭大學士曰:「長白山之西,中國與朝鮮既以鴨綠江為界,而土門江自長白山東邊流出東南入海,土門江西南屬朝鮮,東北屬中國,亦以江為界。但鴨綠、土門二江之間地方,知之不悉。」乃派穆克登往查邊界。十月,帝諭免朝鮮國王例貢物內白金一千兩、紅豹皮一百四十二張,治朝鮮國使沿途館舍。是年,禮臣覆準朝鮮國與奉天府金州、復州、海州、蓋州相近地方,令盛京將軍、奉天府尹嚴飭沿海居民,不許往朝鮮近洋漁採,或別地漁採人到朝鮮,並皆捕送。五十一年五月,焞奏謝減例貢恩,附貢方物,帝命謝恩禮物準作冬至、元旦禮物。是年,穆克登至長白,會同朝鮮接伴使樸權、觀察使李善溥立碑小白山上。五十四年,禮臣奏:「琿春之庫爾喀齊等住處,與朝鮮止隔土門江,恐往來生事,將安都立、他木努房屋窩鋪悉行拆毀。嗣後沿邊近處,不得蓋屋種地,軍民違者重罪之。」五十七年三月,焞表謝賜空青恩,附貢方物,帝命留作下次正貢。自是凡朝鮮奏謝附貢方物均留作正貢,迄於光緒朝不改。

五十九年十月,李焞薨,遣散秩大臣查克亶、禮部右侍郎羅瞻往吊祭,賜謚僖順。兼封世子昀為朝鮮國王,繼妻魚氏為王妃。六十一年二月,昀疏言:「臣萎弱無嗣,請以弟李昑為世弟,以續宗祧。」帝俞其請。四月,遣使往封今為朝鮮國王世弟。十二月,山東漁戶楊三等十四人遭風漂入朝鮮,審無信票,送回內地。帝命嗣後漂風船只人口,驗有票文未滋事者,照舊送回。如無票文,復生事犯法者,令王於審擬後咨部具題,俟命下行文完結,仍報部存案。雍正元年七月,諭禮部減朝鮮貢物內布八百疋、獺皮百張、青黍皮三百張、紙二千卷。朝鮮於九月內進萬壽表文,仍照例於十二月與年貢並進。昀遣陪臣進香,賀登極。二年五月,昀遣陪臣上孝恭仁皇后尊謚。

十二月,李昀薨,遣散秩大臣覺羅舒魯、翰林院學士阿克敦往諭祭,賜謚莊恪。兼封世弟昑為朝鮮國王,妻徐氏為王妃。三年七月,昑疏請封副室所生子李緈為世子,部議與例不符,帝特如所請行。八月,遣官封昑子緈為世子。五年正月,昑疏請更正先世臣倧誣逆事。部議:「昑四代祖倧,故明天啟三年請封。明十六朝紀以篡奪書,實屬冤誣,應予更正。俟明史告成後,以朝鮮列傳頒示其國。」從之。商人胡嘉佩虧帑,以朝鮮國民所負銀六萬兩呈抵,令赴中江質明辦理。部議昑咨文支飾,請按數追償。帝命從寬免追。又諭昑追拏內地盜賊潛逃朝鮮者,倘漏網不獲,王將其國防汛之員參處,王亦一並議處。六年二月。減朝鮮歲貢稻米、江米各三十石,每年止貢江米四十石,以供祭祀,著為例。十月,昑請朝鮮盜賊潛入內地,諭兵部檄盛京、山東邊境官嚴拏究治。七年正月,世子緈卒,遣官諭祭。十月,諭禮臣:「朝鮮國距京三千餘里,貢使往來勞費,嗣後凡謝恩章疏,與聖壽、冬至、元旦三大節表同時齎奏,不必特遣使臣,著為令。」八月,昑為嫂妃魚氏告哀,遣使諭祭。

九年五月,奉天將軍那蘇圖疏言:「鳳凰城邊外陸路防汛之虎耳山諸處,有草河、靉河二水,發源邊內,至邊外之莽牛哨,匯流入中江。中江之中有洲,名江心沱,沱西屬鳳凰城,東為朝鮮國界,歲有匪徒乘船出入,請於莽牛哨設水師防汛。」帝以詢朝鮮王昑,請仍遵舊例,從之。十年三月,昑以先臣李倧被誣事,蒙令史臣改正,乞早頒發諭,先將明史朝鮮列傳抄錄頒示。十三年九月,高宗即位,頒詔朝鮮。諭禮臣曰:「大臣官員之差往朝鮮者,向有餽食儀物之例,其照舊例減半。著為令。」

乾隆元年二月,諭禮臣:「朝鮮國今年所進萬壽表貢,例於十二月偕年貢同進。」由是歲以為常。二年四月,昑奏請仍中江通市舊例,每歲二、八月間,八旗臺站官兵齎貨赴中江與朝鮮互市。帝以旗人有巡守責,且不諳貿易,改令內地商民往為市。及昑奏入,從之。十一月,昑請封其副室子愃為世子。時愃甫三歲,部議格於例,特旨允行。三年正月,遣使往封愃為世子。四年五月,昑表謝頒給朝鮮列傳。

四年十一月,盛京侍郎德福等疏言:「朝鮮漁船被風飄至海寧界,資送漁戶金鐵等由陸路歸國。」嗣後凡朝鮮民人被風漂入內地者,俱給貲護送歸國。迄至光緒朝,撫恤如例。八年九月,帝詣盛京,昑遣使表貢,特賜御書「式表東籓」扁額,令使臣與諸王大臣宴。十一年九月,減中江稅額。十三年五月,盛京刑部侍郎達爾黨阿奏言:「十二年十二月,朝鮮貢使過萬寶橋,奴人士還以馬逸失銀,詭稱迷路,夜入入家,誣執宋二等為盜,訊明,照所誣罪加三等,擬杖徒。」帝諭從寬免罪。又朝鮮國人李云吉誘脅女口,越疆轉賣,照例擬絞監候。仍照乾隆五年定例,入於秋審冊內,覈擬具奏。又朝鮮國王咨稱,訓戎鎮越江東邊有烏喇民人造屋墾田。禮臣議照康熙五十四年定例行,令寧古塔將軍確察禁止,毀其房屋,其違禁民人,及不行察禁之該管官,照例辦理。又奏:「朝鮮人入山海關,所帶貨物,如系彼國土產,與鳳凰城總管印文相符,及出關所帶貨物與本部劄付相符,免其輸稅。此外如別帶物件,及不系彼國所產者,即照數按則輸稅。儻有違買禁物,監督查出,報部治罪。」是年,朝鮮國王咨稱,日本關白新立,照例通使,禮臣奏復,允之。

十四年七月,奉天將軍阿蘭泰奏言:「向例朝鮮貢使到邊,鳳凰城城守尉帶領官兵偕主客迎送通事等官至關門,稽其人馬車輿輜重各數,沿途設館舍,嗣兵部侍郎德沛出使其國,奏言置館非適中之所,貢使人多,不敷居住,聽來使隨時賃住民居。臣以貢使人數眾多,若聽其賃住村莊,恐多滋擾。應請嗣後貢使到關驗入後,務令合隊行走,照舊例每站設官一員,兵役二十人護送。令地方官先期代備旅舍,以資棲息,晝則護行,夜則巡邏。或貢使人役需置食物,護行官檢其出入人數兵役隨往,如內地人民與朝鮮人役生事,兵役拿稟護行官,付地方官究治。至貢使人役,惟迎送官與之相習,應專責成。倘地方官預備不周,許護行迎送通事官揭報府尹,照違令律議處。迎送通事官沿途約束不嚴,致貢使人役滋事,許護行官揭報禮部,照約束不嚴例議處。護行官看守不嚴,及兵役不足,許迎送通事官揭報將軍,照縱軍歇役律議處。迎送通事官瞻徇容隱,致擾居民,或護行官縱容兵丁通同徇蔽,許地方旗民官各揭報上司衙門,照私結外籓例議處。」奏入,報可。十五年,禮臣覆準朝鮮貢使入邊,其行李及貿易貨物,報明查驗車馬數目,沿途按界委地方官催趲車輛,與貢使同按程行走,並於報單內註明經過日期。如朝鮮員役有託故落後者,責成迎送通事官,如催趲車輛不力,專責其管旗民地方官。

十九年九月,帝謁盛京祖陵,昑遣使表貢,賞賜如例。二十二年六月,今以其母金氏之喪來告。王妃徐氏旋卒,二十三年,遣官諭祭。四月,大學士傅恆奏言:「朝鮮久為屬國,禮節語言均已嫻熟,所設通事官請改為八員。」從之。二十五年正月,遣官封昑繼室金氏為王妃。二十八年,朝鮮世子李愃卒,遣官諭祭。七月,封故世子愃之子算為世孫。二十九年三月,朝鮮民人金鳳守、金世柱等殺死內地披甲常德。部議金鳳守造意,應斬,金世柱加功,應絞。至朝鮮奸民屢次越境生事,皆王約束不嚴所致,應交部議處。帝諭金鳳守等從寬,改為監候;王免議處。昑以失於鈐束,褫平安道觀察使鄭淳等職。三十年五月,昑以越江行竊人犯金順丁等俱入緩決,案內疏防各官擬罪從寬,遣使表謝。三十六年八月,昑奏硃璘明紀輯略、陳建之皇明通紀載其先世之事,因訛襲謬,誣妄含冤,請並行刊去。禮臣議,硃璘輯略,浙江巡撫楊廷璋業經銷毀,其陳建通紀,京城書肆亦無售者。若二書彼國或有流傳,應令自行查禁焚銷。

四十一年,李昑薨,王妃金氏請以世孫算為國王,妻金氏為王妃,並請追賜故世子緈爵謚,及世子婦趙氏誥命,諭如所請。遣散秩大臣覺羅萬福、內閣學士嵩貴往諭祭,賜昑謚曰莊順,緈謚曰恪愍,封算為朝鮮國王,妻金氏為王妃。四十三年,帝謁祖陵,以不舉筵宴,敕止朝鮮朝賀。算仍遣官齎表迎駕,御書「東籓繩美」扁額賜之。四十五年,算遣正使吏曹判書徐有慶、副使禮曹參判申大升奉表賀七旬萬壽,貢方物。四十八年,帝謁祖陵,算遣陪臣至盛京迎覲,所有朝貢宴賚一切典禮,特加優渥,並賜禦制詩章及古稀說。四十九年,算疏稱世子年三歲,請封為世子。特旨遣使往封,給與誥敕。五十年正月,舉行千叟宴,算遣正使安春君李烿、副使吏曹判書李致中入貢,預宴比於內臣。帝聞算好學能詩,賞仿宋板五經全部,並筆墨諸物。因諭朝鮮歷年留抵貢物,悉行收受,以免展轉積存;嗣後隨表貢物,概行停止。

五十一年七月,世子病故,遣官諭祭。五十五年,禮臣奏言:「朝鮮國王先因李病故,今副室生男,當即為奏請冊封,不能拜跪行禮,請待其稍長,以永方來之福。」特旨允其國王所請。七月,算遣正使黃仁點、副使徐浩修奉表賀八旬萬壽,貢方物。五十六年,有法蘭西教士由中國往朝鮮傳天主教。五十八年,算請換買錢貨回國通用,部議不許。嘉慶元年,算遣使賀太上皇帝歸政,貢方物。使臣在寧壽宮入千叟宴,賜聖制千叟宴詩。四年正月,遣副都統張承勛、禮部侍郎恆傑赴朝鮮,頒大行太上皇帝遺詔。算遺使表賀,上高宗純皇帝尊謚,貢方物,留抵正貢。

五年,遣使敕封李算子鍚為世子。適李算薨,即以冊封世子之正副使往封李鍚為朝鮮國王。六年,鍚以本國殄除金有山等潛傳洋教顛末,臚章入告,並稱餘孽未靖,恐其潛入邊門,請飭沿疆大吏嚴查究辦。帝諭已飭沿邊大吏一體嚴查,設經盤獲,即發交國王自行辦理。十年,帝詣盛京,遣官接駕,特賜「禮教綏籓」扁額。十二年十一月,朝鮮義州商人白大賢、李士楫潛運米至麞子島,與邊民硃、張兩姓私市。王將白大賢等監禁,地方官革究,並繳進錢文、銅鐵等物。帝以王恭順可嘉,頒賞大緞四疋、玻璃器四件、雕漆器四件、茶葉四瓶,以示恩獎。諭飭盛京將軍督飭沿邊官弁嚴緝硃、張二姓,查明內地疏防官員,嚴行懲處。十七年三月,朝鮮義州土賊起,派祿成督兵討之。遣使敕封李鍚之子炅為世子。二十三年九月,帝詣盛京,鍚遣使迎覲表賀,賜禦制詩及「福」字。

道光元年,鍚奏言伊曾祖李昀患痼疾,經議政金昌集、中樞李頤命、左議政李健命、判中樞趙泰採請以李昑為世弟,參決國政,而相臣趙泰耇等誣金昌集四臣謀逆,肆行誅戮,幸蒙聖祖準李昑襲封,趙泰耇等論罪伏誅,金昌集四臣咸獲昭雪。而皇朝文獻通考載「四臣謀逆,事覺伏誅」等語,乞更正。部議通考所載,系據李昀奏報,非纂修之譌。今既籥懇為祖雪冤,應請刪去此條,以昭信史,從之。二年,頒給文獻通考刊正一編。鍚遣使表賀仁宗睿皇帝升配升祔,暨上皇太后尊號徽號,貢方物;又因賞賜緞匹頒詔謝恩,進皇帝、皇太后前各貢物,前三分收受,餘九分留抵正貢。又例貢外,並賀冊謚孝穆皇后,又為賜祭謝恩,進皇帝、皇太后前各貢物,前二分收受,後三分留抵正貢。八年,鍚遣使表賀平定回疆。又為頒給敕書暨加賞緞疋謝恩,貢方物,俱留抵正貢。九年,朝鮮國副使呂東植在榆關病故,賜銀三百兩。十一年,鍚奏請封嫡孫李怳為世孫,帝俞所請,遣使齎敕封李怳為朝鮮國王世孫。十二年,鍚奏:「英吉利商船駛入朝鮮古代島,要求通市,嚴拒之,相持旬餘,英船始去。」帝獎其忠,賜緞匹。

十五年,李鍚薨,王妃金氏請以世孫李怳襲封,因為故世子具陳請追賜爵謚,及世子婦誥命。二月,遣使諭祭,賜鍚謚宣恪;贈故世子炅為國王,謚康穆,妻為王妃;敕封世孫怳為朝鮮國王。怳表賀冊立皇后暨上皇太后徽號,貢方物。十六年,怳表賀皇太后六旬萬壽加上徽號,貢方物。禮部議準朝鮮使臣來京,禁從人在館外貿易。十七年,遣使敕封怳正室為王妃。十九年,怳表進大行皇后前貢物三分,發還。二十二年,諭禁內地人民私越邊界構舍墾田。二十四年,朝鮮國王妃薨,遣使賜祭。二十五年,遣使敕封怳繼室為王妃。向例派往朝鮮使臣帶通官五六員,至是減至一員,永為定例。是年,禮部奏:「據朝鮮國王咨稱,英船屢泊其境,量山測水,並問答中有交易之詞。」帝命耆英詳詢英使,遵照成約,婉言開導,不得復任兵船游弋,致滋驚擾。

二十九年,李怳薨,諭祭如例。十月,命瑞常、和色本齎敕往封怳子為朝鮮國王。咸豐元年,以伊祖李裀於嘉慶辛酉年間羅入其國邪黨案內,為其戚臣金龜柱等誣陷以死,恐內府編載其事,懇辨其誣。禮部奏稱:「當日上諭暨會典所載,並無李裀之名。以先世被誣,備陳枉抑,實屬為人後者之至情,應如所請,許其昭雪。」從之。表賀上孝和睿皇后暨宣宗成皇帝尊謚,貢方物。二年,遣使敕封李妻為王妃,表賀孝德皇后冊立禮成,貢方物,均留抵正貢。帝飭盛京將軍並沿海督撫嚴禁內地民船至朝鮮漁採。三年,表賀宣宗成皇帝升祔升配,並頒給詔書謝恩,貢方物,命留抵正貢,而受其表賀冊立皇后禮成貢物。四年,朝鮮國人張添吉私來京,帝命送交其國查辦。五年,朝鮮國護送美國難民四名至京,帝命遞至江南,交兩江總督查訊,令附該國商船回國。六年,表賀上孝靜康慈皇后尊謚,貢方物,收受。七年,禮臣奏準朝鮮帶來紅銅四千餘斤,聽在會同四譯館交易。帝諭越界之朝鮮人金益壽解送盛京,禮部轉解鳳凰城,交其國查收訊辦。十一年二月,帝幸熱河,遣使奉表詣行在,恭申起居。帝諭使臣到京後無庸前赴行在,禮部仍照例筵宴,並賜如意、緞疋、瓷器、漆器。

同治元年,遣使表賀登極,呈進兩宮皇太后貢物二分,均收受。其賀登極貢物一分,又頒詔賜緞謝恩進皇帝貢物二分,兩宮皇太后貢物四分,均留抵正貢。二年,表賀上文宗顯皇帝尊謚,並上兩宮皇太后尊號徽號,暨頒詔賞緞謝恩各貢物五分,收受,其十一分留抵正貢。是年,奏稱先世被誣,懇將謬妄書籍刊正。帝諭:「朝鮮國王先系源流,與李仁任即李仁人者,族姓迥別。我朝纂修明史,於其國歷次辨雪之言無不備載。今因見康熙年間鄭元慶所撰廿一史約編,記載其國世系多誣,籥請刊正。約編所稱康獻王為李仁人之子,實屬舛誤。惟系在明史未修以前,村塾綴輯之士,見聞未確,不免仍沿明初之訛。今其國奉有特頒史傳,自當欽遵刊布,使其子孫臣庶知所信從。約編一書,在中國久已不行,亦無所用其改削。著各省學政通行各學,查明曉諭,凡朝鮮事實,應以欽定明史為正,不得援前項書籍為據,以歸畫一而昭信守。」三年,禮臣奏準朝鮮國慶源地方官議修兩國交易官房,越圖們江擇偏僻地採取材木。

十月,李薨,遣使齎敕往封李熙為朝鮮國王,倧九世孫也。五年,俄羅斯兵艦抵朝鮮元山等處,力請通商。九月,法蘭西水師提督魯月率兵艦入漢江,抵漢城,砲擊數船,毀一砲臺而去。十月,法艦再抵江華島,進陷其城,掠銀十九萬佛郎。朝鮮募獵虎手八百名襲之,乃遁。先是,國王李熙年幼,其生父大院君李昰應執國政,惡西教,下令嚴禁,虐待天主教徒。至是,法國聲其罪,無功而還。熙表賀文宗顯皇帝升祔太廟,貢方物,留抵正貢。遣使敕封熙正室閔氏為王妃。

七年二月,侍郎延煦等奏接見朝鮮委員,並查勘鳳凰、靉陽兩邊門外大概情形。帝諭恭親王會同大學士等公同商議。恭親王等奏稱:「查勘各處私墾地畝,已無大段閒荒,而朝鮮所慮全在民物溷雜。欲除溷雜之弊,在乎邊境之嚴。」復經親王等會同延煦、奕榕酌商展邊一切事宜,並請飭盛京將軍會同原勘之延煦等悉心查辦。帝即派延煦、奕榕馳驛前往奉天,會同都興阿出邊查辦。諭曰:「事當創始,必綱舉而目始張。且與外籓交涉,尤應禁令修明,方能垂諸久遠。前次延煦接見之朝鮮使臣,所設問答,均極明晰,足見國王深明大體。即著禮部傳知朝鮮國王,俟報勘定議後,務須嚴飭其國邊界官,一體遵守。」

九年九月,朝鮮國王稱其國慶源府農圃社民李東吉逃往琿春,蓋屋墾田,嘯聚無賴,籥懇查拿。帝諭敏福密飭琿春協領等購線跴緝,盡數拿獲,解交其國懲治。是歲,朝鮮大雨雹,國內荒饑,餓莩載道,民人冒犯重禁,渡圖們江至琿春諸處,乞食求生,是為朝鮮流民越懇之始。帝諭朝鮮國王,將民人悉數領回約束,並自行設法招徠,嚴申禁令,不可復蹈前轍。尋有美國商船駛至朝鮮大同江附近擱淺,朝鮮人見之,誤為法船,大肆劫掠。十一年,熙遣使表賀大婚,加上兩宮皇太后徽號,貢方物。是年,美國水師提督勞直耳司率二鐵甲兵艦抵朝鮮江華島,毀砲臺三座,以報劫掠商船之役。十二年,熙遣使表賀親政,加上兩宮皇太后徽號,貢方物。

光緒元年,朝鮮國撥舟濟渡凱撤官兵,賜熙緞匹,熙遣使進香賀登極,貢方物,俱留抵正貢。發還朝鮮進穆宗毅皇帝萬壽聖節、冬至、元旦、令節各貢物,照例留抵正貢。熙請封世子,貢方物。帝允所請,其進獻禮物,準留抵正貢。尋遣使齎敕往封李坧為朝鮮國王世子。又諭:「奉省押解朝鮮進香貢物之佐領恩俸、驍騎校塔隆阿於五月初三日接領,至六月初五日始行起行,擅改由水路行走,兩月之久,尚未到京,難保無藉端需索情事。恩俸、塔隆阿均先行革職,並著崇實等查明,從重參辦。」二年,熙遣使表賀上穆宗毅皇帝及孝哲毅皇后尊謚,又表賀加上兩宮皇太后徽號,貢方物,俱留抵正貢。

是年,朝鮮與日本立約通商。先是同治十一年,日本外務卿副島種臣來北京議約,乘間詰問總理各國事務衙門:「朝鮮是否屬國?當代主其通商事。」答以:「朝鮮雖籓屬,而內政外交聽其自主,我朝向不預聞。」元年,日本乃以兵力脅朝鮮,突遣軍艦入江華島,毀砲臺,燒永宗城,殺朝鮮兵,劫其軍械而去。別以軍艦駐釜山要盟,而遣開拓使長官黑田清隆為全權大臣,議官井上馨副之,赴朝鮮議約。至是,定約十二條,大要認朝鮮為獨立自主國,禮儀交際皆與日本平等,互派使臣,並開元山、仁川兩埠通商,及日艦得測量朝鮮海岸諸事。

三年,朝鮮以天主教事與法國有違言,介駐釜山日本領事調停,書稱中國為「上國」,有「上國禮部」並「聽上國指揮」等語。日本大詰責,以「交際平等,何獨尊中國?如朝鮮為中國屬,則大損日本國體」。朝鮮上其事,總理衙門致書日本辨論,略曰:「朝鮮久隸中國,而政令則歸其自理。其為中國所屬,天下皆知,即其為自主之國,亦天下皆知,日本豈得獨拒?」

五年七月,軍機大臣寄諭北洋大臣、直隸總督李鴻章,密勸朝鮮與泰西各國通商。諭曰:「總理各國事務衙門奏:『泰西各國欲與朝鮮通商,事關大局。』等語。日本、朝鮮,積不相能。將來日本恃其詐力,逞志朝鮮,西洋各國群起而謀其後,皆在意計之中。各國曾欲與朝鮮通商,儻藉此通好修約,庶幾可以息事,俾無意外之虞。惟其國政教禁令,亦難強以所不欲。據總理衙門奏,李鴻章與朝鮮使臣李裕元曾經通信,略及交鄰之意。自可乘機婉為開導,俾得未雨綢繆,潛弭外患。」六年九月,鴻章遵旨籌議朝鮮武備,許朝鮮派人來天津學習制造操練,命津海關道鄭藻如等與朝鮮齎奏官卞元奎擬具來學章程奏聞。

七年二月,鴻章奏言:「朝鮮國王委員李容肅隨今屆貢使來京,於正月二十日赴津稟謁,據稱專為武備學習事,並齎呈其國請示節略一本,內載有領議政李昰應奏章,頗悔去年六月堅拒美國來使為非計,末則歸重於『及今之務,莫如懷遠人而安社稷』等語。又索中國與各國修好立約通商章程稅則帶回援照。其國軍額極虛,餉力極絀,誠慮無以自立。而所據形勢,實為東三省屏蔽,關系甚重。現其君相雖幡然變計,有聯絡外邦之意,國人議論紛歧,尚難遽決,自應乘機開誠曉諭,冀可破其成見,固我籓籬。惟其國於外交情事生疏,即如與日本通商五年,尚未設關收稅,並不知稅額重輕。設再與西國結約,勢必被欺,無益有損。臣因令前在西洋學習交涉之道員馬建忠與鄭藻如等,參酌目今時勢及東西洋通例,代擬朝鮮與各國通商章程底稿,豫為取益防損之計,交李容肅齎回,俾其國遇事有所據依。至其節略所詢各例條內,惟答覆日本國書稱謂一節,儻稍涉含混,即於屬邦體例有礙。臣查西洋各國稱帝稱王,本非一律,要皆平等相交。朝鮮國王久受我冊封,其有報答日本及他國之書,應令仍用封號。國政雖由其自主,庶不失中國屬邦之名也。」禮部議準朝鮮學習制器練兵等事,發給空白憑票,徑由海道赴津,以期便捷;至貢使來京,仍遵定例辦理。

先是光緒初元,吉林鄂多哩開放荒田,朝鮮茂山對岸外六道溝諸處,間有朝鮮人冒禁私墾者,漸次蔓延。至是,吉林將軍銘安、督辦邊防吳大澂奏言:「據琿春招墾委員李金鏞稟稱,土門江北岸,由下嘎牙河至高麗鎮約二百里,有閒荒八處,前臨江水,後擁群山,向為人跡不到之區,與朝鮮一江之隔。其國邊民屢被水災,連年荒歉,無地耕種,陸續渡江開墾,已熟之地,不下二千晌,其國窮民數千人賴以餬口。有朝鮮咸鏡道刺史發給執照、分段注冊等語。臣等查吉林與朝鮮毗連之處,向以土門江為界。今朝鮮貧民所墾閒荒在江北岸,其為吉林轄境無疑。邊界曠土,豈容外籓任意侵占?惟朝鮮寄居之戶,墾種有年,並有數千餘眾。若照例嚴行驅逐出界,恐數千無告窮民同時失所。殊堪憐憫,擬請飭下禮部,咨明朝鮮國王,派員會同吉林委員查勘明確,劃清界址。所有其國民人,寄居戶口,已墾荒地,懇恩準其查照吉林向章,每晌繳押荒錢二千一百文,每年每晌完佃地租錢六百六十文,由臣銘安飭司給領執照,限令每年冬季應交租錢,就近交至琿春,由放荒委員照數收納。或其國鑄錢不能出境,議令以牛抵租,亦可備吉省墾荒之用。其咸鏡道刺史所給執照,飭令收回銷毀。」從之。

十二月,鴻章奏言:「本年正月,總理衙門因屢接出使日本大臣何如璋函,述朝鮮近日漸知變計,商與美國立約,請由中國代為主持。擬變通舊制,嗣後遇有朝鮮關系洋務要件,由北洋大臣及出使日本大臣與其國通遞文函,相機開導,奉旨知照。臣維朝鮮久隸外籓,實為東三省屏蔽,與琉球孤懸海外者形勢迥殊。今日本既滅琉球,法國又據越南,沿海六省,中國已有鞭長莫及之勢。我籓屬之最親切者,莫如朝鮮。日本脅令通商,復不允訂稅則,非先與美國訂一妥善之約,則朝鮮勢難孤立,各國要求終無已時。東方安危,大局所系。中朝即不必顯為主張,而休戚相關,亦不可不隨時維持,多方調護。」

八年三月,朝鮮始與美國議約,請蒞盟。鴻章奏派道員馬建忠、水師統領提督丁汝昌,率威遠、揚威、鎮海三艘,會美國全權大臣薛裴爾東渡。四月初六日,約成,美使薛裴爾,朝鮮議約官申、金宏集盟於濟物浦,汝昌、建忠監之。十四日,陪臣李應浚齎美朝約文並致美國照會呈禮部及北洋大臣代表。未幾,英使水師提督韋力士、法駐津領事狄隆、德使巴蘭德先後東來,建忠介之,皆如美例成約。是役也,日本亦令兵輪來詗約事,其駐朝公使花房義質屢詰約文,朝鮮不之告;乃叩建忠,建忠秘之,日人滋不悅。

六月,朝鮮大院君李昰應煽亂兵殺執政數人,入王宮,將殺王妃閔氏,脅王及世子不得與朝士通,並焚日本使館,在朝鮮練兵教師堀本禮造以下七人死焉。日使花房義質走回長崎。時建忠、汝昌俱回國,鴻章以憂去,張樹聲署北洋大臣,電令建忠會汝昌率威遠、超勇、揚威三艘東渡觀變。二十七日,抵仁川,泊月尾島,而日本海軍少將仁禮景範已乘金剛艦先至。朝鮮臣民惶懼,望中國援兵亟。建忠上書樹聲,請濟師:「速入王京執逆首,緩則亂深而日人得逞,損國威而失籓封。」汝昌亦內渡請師。

七月初三日,日兵艦先後來仁川,陸兵亦登岸,分駐仁川、濟物浦,花房義質且率師入王京。初七日,中國兵艦威遠、日新、泰西、鎮東、拱北至,繼以南洋二兵輪,凡七艘。蓋樹聲得朝鮮亂耗即以聞,遂命提督吳長慶所部三千人東援,便宜行事,以兵輪濟師,是日登岸。十二日,薄王京。十三日,長慶、汝昌、建忠入城往候李昰應,減騶從示坦率,昰應來報謁,遂執之,致之天津,而亂黨尚踞肘腋。十六日黎明,營官張光前、吳兆有、何乘鰲掩至城東枉尋裏,擒百五十餘人,長慶自至泰利里,捕二十餘人,亂黨平。

日使花房義質入王京,以焚館逐使為言,要挾過當,議不行。義質惡聲去,示決絕。朝鮮懼,介建忠留之仁川,以李裕元為全權大臣,金宏集副之,往仁川會議,卒許償金五十萬元,開楊華鎮市埠,推廣元山、釜山、仁川埠行程地,宿兵王京,凡八條,隱忍成約。自是長慶所部遂留鎮朝鮮。

方李昰應之執歸天津也,帝命俟李鴻章到津,會同張樹聲向昰應訊明變亂之由及著名亂黨具奏。至是,究明李昰應乃國王本生父,秉政十年。及王年長親政,王妃閔氏崇用親屬,分昰應權,昰應怨望。六月初間,閔謙鎬分給軍餉,米不滿斛,軍人與胥役詰斗,謙鎬囚軍卒五人,將置諸法,軍人奔訴於昰應,遂變。初九日,殺閔謙鎬、金輔弦、李最應等,昰應入闕曉諭諸軍,自稱「國太公」,總攬國權,亦不捕治亂黨。鴻章奏言:「此次變亂,雖由軍卒索餉,然亂軍赴昰應申訴,如果正言開導,何至遽興大難。朝鮮臣庶皆謂昰應激之使變。即謂此無左證,而亂軍圍擊宮禁,王妃與難,大臣被害,兇焰已不可鄉爾。李昰應既能定亂於事後,獨不能遏亂於方萌?況乘危竊柄,一月有餘。春秋之義,入不討賊,片言可折,百喙難逃。儻再釋回本國,奸黨構煽,怨毒相尋,重植亂萌,必為後患。伏查朝鮮史略,元代高麗王累世皆以父子構釁。延祐年間,高麗王謜既為上王,傳位於其子燾,交構讒隙,元帝流謜於土蕃,安置王父,俱有前事。又至元年間,燾子忠惠王名禎,亦經元帝流於揭陽縣,其時高麗國內晏然,徒以宵小浸潤,遠竄窮荒。今李昰應無蒙產垂統之尊,有幾危社稷之罪,較謜、禎等情節尤重。惟處人家國父子之間,不能不兼籌並顧。儻蒙加恩,敕下臣等將李昰應安置近京之保定省城,永遠不準復回本國,優給廩餼,譏其出入,嚴其防閑,仍準其國王派員省問,以慰其私。既以弭其國禍亂之端,亦即以維其國倫紀之變。」帝俞其言,乃幽昰應保定舊清河道署。

是年,鴻章奏定朝鮮通商章程八條:一,由北洋大臣札派商務委員前往駐扎,朝鮮亦派大員駐津照料商務;二,朝鮮商民在中國各口財產罪犯等案,悉由地方官審斷,遵會典舊例;三,朝鮮平安、黃海道,與山東、奉天等省濱海地方,聽兩國漁船往來捕魚,不得私以貨物貿易,違者船貨入官;四,準兩國商民入內地採辦土貨,照納沿途釐稅;五,訂鴨綠江對岸柵門與義州二處,又圖們江對岸琿春與會寧二處,聽邊民往來交易,設卡徵稅,罷除館宇餼廩芻糧等費;六,申明嚴禁之物,紅葠一項,照例準售,酌定稅則;七,派招商局輪船,每月定期往返一次,由朝鮮政府協商船費若干;八,豫計增損之處,隨時商辦。禮部奏準停止會寧、慶源地方監視交易,惟本年輪屆會寧交易之期,恐彼處商民無官約束,別滋事端,應由盛京將軍、奉天府尹、吉林將軍就近派員會同朝鮮官妥為經理。熙表賀孝貞顯皇后升祔,恭進慈禧皇太后貢物。九年,熙表賀崇上孝貞顯皇后尊謚,恭進慈禧皇太后貢物,其因亂黨滋事出兵東援並派兵衛護謝恩貢方物,留抵正貢。

十年,朝鮮維新黨亂作。初,朝鮮自立約通商後,國中新進輕躁喜事,號「維新黨」,目政府為「守舊黨」,相水火。維新黨首金玉均、洪英植、樸泳孝、徐光範、徐載弼謀殺執政代之。五人者常游日本,暱日人,至是倚為外援。十月十七日,延中國商務總辦及各國公使並朝鮮官飲於郵署,蓋英植時總郵政也。是日,駐朝日兵運槍砲彈藥入日使館。及暮,賓皆集,惟日使竹添進一郎不至。酒數行,火起,亂黨入,傷其國禁衛大將軍閔泳翊,殺朝官數人於座,外賓驚散。夜半,日本兵排門入景祐宮,金玉均、樸泳孝、徐光範直入寢殿,挾其王,謬言中國兵至,矯令速日本入衛。十八日天明,殺其輔國閔臺鎬、趙寧夏、總管海防閔泳穆、左營使李祖淵、前營使韓圭稷、後營使尹泰駿;而亂黨自署官,英植右參政,玉均戶曹參判,泳孝前後營使,光範左右營使,載弼前營正領官,遂議廢立。

議未決,而勤王兵起。十九日,朝鮮臣民籥長慶平亂。長慶責日使撤兵,及暮不答。其臣民固請長慶兵赴王宮。及闕,日兵集普通門發槍。長慶疑國王在正宮,恐傷王,未還擊,而日兵連發槍斃華兵甚夥,乃進戰於宮門外。王乘間避至後北關廟,華軍偵知之,遂以王歸於軍,斬洪英植及其徒七人以殉,泳孝、光範、載弼奔日本。日使自焚使署,走濟物浦,朝民仇日人益甚。長慶衛其官商妻孥出王京。

朝鮮具疏告變,帝命吳大澂為朝鮮辦事大臣,續昌副之,赴朝鮮籌善後。日本亦派全權大臣井上馨至朝鮮,有兵艦六艘,並載陸軍登濟物浦,以五事要朝鮮:一,修書謝罪;二,恤日本被害人十二萬圓;三,殺太尉林磯之兇手處以極刑;四,建日本新館,朝鮮出二萬元充費;五,日本增置王京戍兵,朝鮮任建兵房。朝鮮皆聽命,成約。

十一年正月,日本遣其宮內大臣伊藤博文、農商務大臣西鄉從道來天津,議朝鮮約。帝命李鴻章為全權大臣,副以吳大澂,與議。諭曰:「日本使臣到津,李鴻章熟悉中外交涉情形,必能妥籌因應。此次朝鮮亂黨滋事,提督吳兆有等所辦並無不合。前據徐承祖電稱,日人欲我懲在朝武弁,斷不能曲徇其請。其餘商議各節,務當斟酌機宜,與之辯論,隨時請旨遵行。」三月,約成,鴻章奏言:「日使伊藤博文於二月十八日詣行館會議,當邀同吳大澂、續昌與之接晤。其使臣要求三事:一,撤回華軍;二,議處統將;三,償恤難民。臣惟三事之中,惟撤兵一層,尚可酌允。我軍隔海遠役,本非久計,原擬俟朝亂略定,奏請撤回。而日兵駐扎漢城,名為護衛使館,今乘其來請,正可乘機令彼撤兵。但日本久認朝鮮為自主之國,不欲中國干涉,其所注意不在暫時之撤防,而在永遠之輟戍。若彼此永不派兵駐朝,無事時固可相安,萬一朝人或有內亂,強鄰或有侵奪,中國即不復能過問,此又不可不熟思審處者也。伊藤於二十七日自擬五條給臣閱看,第一條聲明嗣後兩國均不得在朝鮮國內派兵設營,其所注重實在於此。臣於其第二條內添註,若他國與朝鮮或有戰爭,或朝鮮有叛亂情事,不在前條之列。伊藤於叛亂一語,堅持不允,遂各不懌而散。旋奉三月初一日電旨:『撤兵可允,求不派兵不可允。萬不得已,或於第二條內添敘:「兩國遇有朝鮮重大事變,可各派兵,互相知照。」至教練兵事一節,亦須言定兩國均不派員為要。』臣復恪遵旨意,與伊藤再四磋商,始將前議五條改為三條。第一條,議定兩國撤兵日期;第二條,中、日均勿派員在朝教練;第三條,朝鮮變亂重大事件,兩國或一國要派兵,應先互行文知照,及其事定,仍即撤回,不再留防。字斟句酌,點易數四,乃始定議。夫朝廷睠念東籓,日人潛師襲朝,疾雷不及掩耳,故不惜糜餉勞師,越疆遠戍。今既有互相知照之約,若將來日本用兵,我得隨時為備。即西國侵奪朝鮮土地,我亦可會商派兵互相援助,此皆無礙中國字小之體,而有益於朝鮮大局者也。至議處統將、償恤難民二節,一非情理,一無證據,本可置之不理。惟伊藤謂此二節不定辦法,既無以復君命,更無以息眾忿,亦系實情。然我軍保護屬籓,名正言順,誠如聖諭謂『提督所辦並無不合,斷不能曲徇其請』。因念駐朝慶軍系臣部曲,姑由臣行文戒飭,以明出自己意,與國無干。譬如子弟與人爭鬥,其父兄出為調停,固是常情。至伊所呈各口供,謂有華兵殺掠日民情事,難保非彼藉詞。但既經其國取有口供,正可就此追查。如查明實有某營某兵上街滋事,確有見證,定照軍法嚴辦,以示無私,絕無賠償可議也。以上兩節,即由臣照會伊藤,俾得轉圜完案。遂於初四日申刻,彼此齊集公所,將訂立專條逐細校對,公同畫押蓋印,各執一本為據。謹將約本封送軍機處進呈御覽,恭候批準。臣等稟承朝謨,反覆辯折,幸免隕越。以後彼此照約撤兵,永息爭端,俾朝鮮整軍經武,徐為自固之謀,並無傷中、日兩國和好之誼,庶於全局有裨也。」由是中國戍朝鮮兵遂罷歸。是年,吉林設通商局於和龍峪,設分卡於光霽峪、西步江,專司吉林與朝鮮通商事。又設越墾局,劃圖們江北沿岸長約七百里、寬約四十五里,為越墾專區。

當光緒己卯間,俄人以伊犁故,將失和,遣兵艦駛遼海,英人亦遣兵艦踞朝鮮之巨文島,以尼俄人。既而伊犁約成,英人慮擾東方大局,冀中國始終保護朝鮮,屢為總署言之。十二年,出使英法德俄大臣劉瑞芬致書鴻章,言:「朝鮮毗連東三省,關系甚重。其國奸黨久懷二心,飲鴆自甘,已成難治之癥。中國能收其全土改行省,此上策也。其次則約同英、美、俄諸國共相保護,不準他人侵占寸土,朝鮮亦可幸存。」鴻章韙之。上之總署,不可,議遂寢。是年,釋李昰應歸國,熙奉表謝恩,貢方物,留抵正貢。

十三年,鴻章遵旨籌議朝鮮通使各國體制,奏言:「電飭駐扎朝鮮辦理交涉通商事宜升用道補用知府袁世凱,轉商伊國應派駐扎公使,不必用「全權」字樣。旋於九月二十三日接據袁世凱電稟:準朝鮮外署照稱:『奉國王傳教,前派各使久已束裝,如候由咨文往返籌商,恐須時日,請先電達北洋大臣籌覆。』並據其國王咨稱:『近年泰西各國屢請派使修聘,諸國幅員權力十倍朝鮮,不可不派大公使。惟派使之初,未諳體制,未先商請中朝,派定後即飭外署知照各國,以備接待。茲忽改派,深恐見疑。仍請準派全權公使前往,待報聘事竣調回,或以參贊等員代理,庶可節省經費;並飭使至西國後,與中國大臣仍恪遵舊制。』等語,辭意甚為遜順。臣復加籌度,更將有關體制者先為約定三端:一,韓使初至各國,應請由中國大臣挈赴外部;一,遇有宴會交際,應隨中國大臣之後;一,交涉大事關系緊要者,先密商中國大臣核示,並聲明此皆屬邦分內之體制,與各國無干,各國不得過問。當即電飭袁世凱轉達國王照辦。茲復準王咨稱:『於十月杪飭駐美公使樸定陽、駐英德俄意法公使趙臣熙先後前往,所定三端並飭遵行。』臣查朝鮮派使往駐泰西,其國原約有遣使互駐之條,遂未先商請中國,遽以全權公使報聞各國。此時慮以改派失信,自是實情。既稱遣使後與中朝使臣往來恪遵舊制,臣所定擬三端又經遵行,於屬邦事例並無違礙。」

是年,吉林有朝鮮勘界之案。十六年,總理衙門疏言:「吉林將軍奏稱:『朝鮮流民占墾吉林邊地,光緒七年經將軍銘安、督辦邊防吳大澂奏將流民查明戶籍,分歸琿春及敦化縣管轄。嗣因朝王懇請刷還流民,咨由禮部轉奏。經將軍覆準,予限一年,由伊國地方官設法收回。復因限滿而流民仍未刷還,反縱其過江侵占,經將軍希元咨由總理衙門奏準派員會勘。乃其國始誤以豆滿、圖們為兩江,繼誤指內地海蘭河為分界之江,終誤以松花江發源之通化松溝子有土堆如門,附會「土門」之義,執意強辯。續經希元派員覆勘石乙水為圖們正源,議於長水分界,繪具圖說,於十三年十一月奏奉諭旨咨照國王遵辦在案。乃國王不加詳考,遽信勘界使李重夏偏執之詞,堅請以紅土山水立界,齟難合,然未便以勘界之故,遂置越墾為緩圖。現在朝鮮茂山府對岸迤東之光霽峪、六道溝、十八崴子等地方,韓民越墾約有數千,地約數萬晌。此處既有圖們江天然界限,自可毋庸再勘。其國遷延至今,斷難將流民刷還,應亟飭令領照納租,歸我版籍,先行派員清丈,編甲升科,以期邊民相安』等語。臣等查吉林、朝鮮界務,前經兩次會勘,其未能即定者,特茂山以上直接三汲泡二百餘里之圖們江源耳。至茂山以下圖們江巨流,乃天然界限。江南岸為朝鮮咸鏡道屬之茂山、會寧、鍾城、慶源、慶興六府地方,江北岸為吉林之敦化縣及琿春地方,朝鮮勘界使亦無異說。韓民越墾多年,廬墓相望,一旦盡刷還,數千人失業無依,其情實屬可矜。若聽其以異籍之民日久占住,主客不分,殊非久計。且近年墾民疊以韓官邊界徵租,種種苛擾,赴吉林控訴,經北洋大臣李鴻章咨臣衙門有案。現在江源界址既難克日劃清,則無庸勘辦處所,似宜及時撫綏。擬請飭下將軍,遴派賢員清丈升科,領照納租,歸地方官管轄,一切章程奏明辦理。」於是將軍長順頒發執照,韓民原去者聽其自便,原留者薙發易服,與華人一律編籍為氓,墾地納租。

是年,熙母妃趙氏薨,遣使奉表來訃曰:「朝鮮國王臣李熙言:臣母趙氏於光緒十六年四月十七日薨逝,謹奉表訃告。臣李熙誠惶誠恐頓首稽首。伏以小邦無祿,肆切哀惶之忱,內艱是丁,恭申訃告之禮。臣無任望天仰聖激切屏營之至,謹奉表告訃以聞。」告訃正使洪鍾永等為懇恩事:「竊以小邦祇守籓服,世沐皇恩,壬午、甲申之交,綱常得以扶植,土宇賴以廓清,尤屬恩深再造。自經喪亂,洊遭饑饉,民物流離,六七年來,艱難日甚。近又不幸,康穆王妃薨逝,舉朝哀戚,無計摒擋。主上念王妃遭兵構憫,八域困窮,向例喪祭之需,出自閭閻者,不得不一概蠲免,以舒民力,故凡喪祭俱從儉約。惟念大皇帝欽差頒敕,自昔異數,時恐星使賁臨,禮節儻有未周,負罪滋甚。與其抱疚於將來,孰若陳情於先事?況天恩高厚,有原必償,久如赤子之仰慈父母矣。為特敬求部堂俯鑒實情,擎奏天陛。儻有溫諭頒發,俾職敬謹齎回,免煩星使之處,出自逾格恩施,不勝急切兢懼之至。」

禮臣奏聞,帝諭曰:「朝鮮告訃使臣具呈懇請免遣使賜奠一摺,所陳困苦情形,自非虛飾。惟國王世守東籓,備叨恩禮,吊祭專使,載在典常,循行勿替,此天朝撫恤屬籓之異數,體制攸關,豈容輕改?特念朝鮮近年國用窘乏異常,不得不於率循舊章之中,曲加矜恤。向來遣使其國,皆由東邊陸路,計入境後,尚有十餘站,沿途供億實繁。此次派往大員,著改由天津乘坐北洋輪船,徑至仁川登岸,禮成,仍由此路回京。如此變通,則道途甚近,支應無多,所有向來陸路供張繁費,悉行節省。至欽使到國以後,應行典禮,凡無關冗費者,均應恪遵舊章,不得稍事簡略。將此諭由禮部傳諭國王知之。」九月,遣戶部左侍郎續昌、戶部右侍郎崇禮往諭祭。

十九年,朝鮮償日本米商金。先是十五年秋,朝鮮饑,其咸鏡道觀察使趙秉式禁糶,及次年夏弛禁。日人謂其元山埠米商折本銀十四萬餘元,責償朝鮮,朝鮮為罷秉式官,許償六萬,日人至三易公使以爭,至是卒償十一萬,事乃解。

初,中國駐朝道員袁世凱以吳長慶軍營務處留朝,充商務總辦兼理交涉事宜。時朝鮮倚中國,其執政閔泳駿等共善世凱。泳駿,閔妃族也,素嫉日本,而國中新黨厚自結於日人。甲申朝鮮之難,金玉均、樸泳孝等挾貲逃日本,而李逸植、洪鍾宇分往刺之。鍾宇,英植子,痛其父死玉均手,欲得而甘心,佯交歡玉均。二十年二月,自日本偕乘西京丸商輪船游上海,同寓日本東和旅館。二十二日,鍾宇以手槍擊殺玉均,中國捕鍾宇系之以詰朝鮮。朝人謂玉均叛黨,鍾宇其官也,請歸其獄自讞,許之。朝鮮超賞鍾宇五品官,戮玉均尸而以鹽漬其首。日本大言華,乃為玉均發喪假葬,執紼者數百人。會逸植亦刺泳孝於日本,未中,日人處逸植極刑。日、朝交惡,且怒中國歸玉均尸。

四月,朝鮮東學黨變作。東學者,創始崔福成,刺取儒家、佛、老諸說,轉相衍授,起於慶尚道之慈仁縣,蔓延忠清、全羅諸道。當同治四年,朝鮮禁天主教,捕治教徒,並擒東學黨首喬姓殺之,其黨卒不衰。洎上年徑赴王宮訟喬冤,請湔雪,不許。旋擒治其渠數人,乃急而思逞。朝鮮賦重刑苛,民多怨上,黨人乘之,遂倡亂於全羅道之古阜縣。朝鮮王以其臣洪啟勛為招討使,假中國平遠兵艦、蒼龍運船,自仁川渡兵八百人至長山浦登岸,赴全州。初戰甚利,黨人逃入白山,朝兵躡之,中伏大敗,喪其軍大半。賊由全羅犯忠清兩道,兵皆潰,遂陷全州、會城,獲槍械藥彈無算。榜全州城以匡君救民為名,揚言即日進公州、洪州直搗王京。

朝鮮大震,急電北洋乞援師。鴻章奏派直隸提督葉志超、太原鎮總兵聶士成率蘆榆防兵東援,屯牙山縣屯山,值朝鮮王京西南一百五十里,仁川澳左腋沔江口也。五月,電諭駐日公使汪鳳藻,按光緒十一年條約,告日本外部以朝鮮請兵,中國顧念籓服,遣兵代平其亂。日本外務卿陸奧宗光復鳳藻文謂:「貴國雖以朝鮮為籓服,而朝鮮從未自稱為屬於貴國。」乃以兵北渡,命其駐京公使小村壽太郎照約告於中國總署。復文謂:「我朝撫綏籓服,因其請兵,故命將平其內亂,貴國不必特派重兵。且朝鮮並未向貴國請兵,貴國之兵亦不必入其內地。」日使覆文謂:「本國向未認朝鮮為中國籓屬。今照日朝濟物浦條約及中日兩國天津條約,派兵至朝鮮,兵入朝鮮內地,亦無定限。」朝鮮亂黨聞中國兵至,氣已懾。初九日為朝兵所敗,棄全州遁,朝兵收會城。

亂平,而日兵來不已。其公使大鳥圭介率兵四百人先入王京,後隊繼至,從仁川登岸約八千餘人,皆赴王京。朝鮮驚愕,止之不可。中國以朝亂既平,約日本撤兵,而日人要改朝鮮內政。其外部照會駐日使臣,約兩國各簡大臣至朝,代其更革。鳳藻復文謂:「整頓內治,任朝鮮自為之,即我中國不原干預。且貴國既認朝鮮為自主之國,豈能預其內政?至彼此撤兵,中東和約早已訂有專條,今可不必再議。」而日人持之甚堅。時日兵皆據王京要害,中國屯牙山兵甚單。世凱屢電請兵,鴻章始終欲據條約要日撤兵,恐增兵益為藉口。英、俄各國使臣居間調停,皆無成議。鴻章欲以賠款息兵,而日索銀三百萬兩,朝論大譁,於是和戰無定計,而日本已以兵劫朝鮮。

日使大鳥圭介首責朝鮮獨立。六月,圭介要以五事:一,舉能員;二,制國用;三,改法律;四,改兵制;五,興學校。朝鮮為設校正,示聽命。十四日,朝鮮照會日使,先撤兵,徐議改政,不許。復責其謝絕為中國籓屬。朝鮮以久事中國,不欲棄前盟,駐京日使照會總署文略謂:「朝鮮之亂,在內治不修。若中、日兩國合力同心,代為酌辦,事莫有善於此者。萬不料中國悉置不講,但日請我國退兵。兩國若啟爭端,實惟中國執其咎。」遂遍布水雷漢江口,以兵塞王京諸門。十七日,袁世凱赴仁川登輪回國。二十一日,大鳥圭介率兵入朝鮮王宮,殺衛兵,遂劫國王李熙,令大院君李昰應主國事。矯王令流閔泳駿等於惡島,凡朝臣不親附者逐之。事無鉅細,皆決於日人。

二十二日,鴻章電令牙山速備戰守,乃奏請以大同鎮總兵衛汝貴率盛軍十三營發天津,盛京副都統豐伸阿統盛京軍發奉天,提督馬玉昆統毅軍發旅順,高州鎮總兵左寶貴統奉軍發奉天。四大軍奉朝命出師,慮海道梗,乃議盡由陸路自遼東行,渡鴨綠江入朝鮮。時牙山兵孤懸,不得四大軍消息,而距牙山東北五十里成歡驛為自王京南來大道,且南通公州。士成請於志超,往扼守,遂率武毅副中營、老前營及練軍右營於二十四日移駐成歡。鴻章租英商高升輪載北塘防軍兩營,輔以操江運船,載械援牙山,兵輪三艘翼之而東。而師期預洩,遂為所截,三輪逃回威海,操江懸白旂任掠去。日艦吉野、浪速以魚雷擊高升,沉之,兩營殲焉。是日牙山軍聞之,知援絕,而日人大隊已逼。士成請援於志超,二十六日,志超馳至,迎戰失利。二十七日,日兵踞成歡,以砲擊我軍,勢不支,遂敗。志超已棄公州遁,士成追及之,合軍北走,繞王京之東,循清鎮州、忠州、槐山、興塘、涉漢江,經堤川、原州、橫川、狼川、金化、平康、伊川、遂安、祥源,渡大同江至平壤,與大軍合,匝月始達。

七月初一日,諭曰:「朝鮮為我大清籓屏二百餘年,歲修職貢,為中外共知。近十年其國時多內亂,朝廷字小為懷,疊次派兵前往勘定,並派員駐扎其國都城,隨時保護。本年四月間,朝鮮又有土匪變亂,國王請兵援剿,陳詞迫切,當即諭令李鴻章撥兵赴援,甫抵牙山,匪徒星散。乃日人無故添兵,突入漢城,嗣又增兵萬餘,迫令朝鮮更改國政。我朝撫綏籓服,其國內政事向令自理;日本與朝鮮立約,系屬與國,更無以重兵強令革政之理。各國公論,皆以日本師出無名,不合情理,勸令撤兵,和平商辦。乃竟悍然不顧,迄無成說,反更陸續添兵,朝鮮百姓及中國商民日加驚擾,是以添兵前往保護。詎行至中途,突有敵船多只,乘我不備,在牙山口外海面開砲轟擊,傷我運船,殊非意料所及。日本不遵條約,不守公法,釁開自彼,公論昭然。用特布告天下,俾曉然於朝廷辦理此事,實已仁至義盡,勢難再與姑容。著李鴻章嚴飭派出各軍,迅速進剿,厚集雄師,陸續進發,以拯韓民於塗炭。」蓋中國至是始宣戰也。

是時中國軍並屯平壤為固守計。八月初,日兵既逼,諸將分劃守界。城北面左寶貴所部奉軍、豐伸阿之盛軍、江自康之仁字兩營守之,城西面葉志超所部蘆防軍守之,城南面迤西南隅衛汝貴之盛軍守之,城東面大同江東岸馬玉昆之毅軍守之,復以左寶貴部分統聶桂林策應東南兩面,志超駐城中調度,寶貴駐城北山頂守玄武門,諸將各以守界方位駐城外。十六日,日兵分道來撲,巨砲逼攻,各壘相繼潰,城遂陷,寶貴力戰中砲死。志超率諸將北走,軍儲器械、公牘密電盡委之以去。聶士成以安州山川險峻,宜固守,志超不聽,奔五百餘里,渡鴨綠江入邊止焉。自是朝鮮境內無一華兵,朝事不可問矣。

二十一年三月,馬關條約成,其第一款中國確認朝鮮為完全無缺獨立自主之國,凡前此貢獻等典禮皆廢之。蓋自崇德二年李倧歸附,朝鮮為清屬國者凡二百五十有八年,至是遂為獨立自主國云。

琉球,在福建泉州府東海中。先是明季琉球國王尚賢遣使金應元請封,會道阻,留閩中。清順治三年,福建平,使者與通事謝必振等至江寧,投經略洪承疇,送至京,禮官言前朝敕印未繳,未便受封。四年,賜其使衣帽布帛遣歸。是年,尚賢卒,弟尚質自稱世子,遣使奉表歸誠。

十年,遣使來貢。明年,再遣貢使,兼繳前朝敕印,請封,允之。詔曰:「帝王祗德底治,協於上下,靈承於天,薄海通道,罔不率俾,為籓屏臣。朕懋纘鴻緒,奄有中夏,聲教所綏,無間遐邇,雖炎方荒略,不忍遺棄。爾琉球國粵在南徼,乃世子尚質達時識勢,祗奉明綸,即令王舅馬宗毅等獻方物,稟正朔,抒誠進表,繳上舊詔敕印。朕甚嘉之,故特遣正使兵科副理官張學禮、副使行人司行人王垓,齎捧詔印,往封為琉球國中山王。爾國官僚及爾氓庶,尚其輔乃王,飭乃侯度,協抒乃忠藎,慎乂厥職,以凝休祉,綿於奕世。故茲詔示,咸使聞知。賜王印一、緞幣三十匹,妃緞幣二十匹;並頒定貢期,二年一貢,進貢人數不得逾一百五十名,許正副使二員、從人十五名入京,餘俱留閩待命。」既而學禮等至閩,因海氛未靖,仍掣回。

康熙元年,敕曰:「琉球國世子尚質慕恩向化,遣使入貢,世祖章皇帝嘉乃抒誠,特頒恩賚,命使兵科副理官張學禮等齎捧敕印,封爾為琉球國王。乃海道未通,滯閩多年,致爾使人率多物故。朕念爾國傾心修貢,宜加優恤,乃使臣及地方官逗留遲誤,均未將前情奉明,殊失朕懷遠之意。今已將正副使、督撫等官分別處治,特頒恩賚,仍遣正使張學禮、副使王垓令其自贖前非,暫還原職,速送使人歸國。一應敕封事宜,仍照世祖章皇帝前旨奉行。朕恐爾國未悉朕意,故再降敕諭,俾爾聞知。」於是學禮等奉往至其國,成禮而還。

三年,質遣陪臣吳國用、金正春奉表謝封,貢方物。四年,再遣貢使並賀登極。其貢物至梅花港口遭風漂失,帝諭免其補進。五年,質仍遣貢使補進前失貢物。帝諭曰:「尚質恭順可嘉,補進貢物,俱令齎回。至所進瑪瑙、烏木、降香、木香、象牙、錫速香、丁香、檀香、黃熟香等,皆非土產,免其入貢。其琉璜留福建督撫收貯。餘所貢物,令督撫差解來京。」即給賞遣歸。六年,貢使仍齎表入覲。七年,重建柔遠館驛於福建,以待琉球使臣。是年,王尚質薨。

八年,世子尚貞遣陪臣英常春來貢。琉球國凡王嗣位,先請朝命,欽命正副使奉敕往封,賜以駝鈕鍍金銀印,乃稱王。未封以前稱世子,權國事。十年、十三年,世子貞均遣陪臣來貢。十八年,貞遣陪臣補進十七年正貢。舊例貢物有金銀罐、金銀粉匣、金缸酒海、泥金彩畫圍屏、泥金扇、泥銀扇、畫扇、蕉布、苧布、紅花、胡椒、蘇木、腰刀、火刀、槍、盔甲、馬、鞍、絲、綿、螺盤,加貢之物無定額。十九年,陪臣來貢,帝俱令免進。嗣後常貢,惟馬及熟硫磺、海螺殼、紅銅等物。

二十年,貞遣陪臣毛見龍等來貢。帝以貞當耿精忠叛亂之際,屢獻方物,恭順可嘉,賜敕褒諭,兼賜錦幣十五。又常貢內免其貢馬,著為例。貞疏言:「先臣尚質於康熙七年薨逝,貞嫡嗣,應襲爵,具通國臣民結狀請封。」禮臣議航海道遠,應令貢使領封。見龍等固請,禮臣執不可,帝特允之。

二十一年,命翰林院檢討汪楫、內閣中書舍人林麟魛為正副使,齎詔敕銀印往封琉球國世子尚貞為王,賜御書「中山世土」額。禮成,還京,奏言:「中山王尚貞原令陪臣子弟四人來京受學。部議前明洪武、永樂、宣德、成化間,琉球官生入監讀書。今尚貞傾心向學,應如所請。」從之。貞遣陪臣毛國珍、王明佐等謝封,奏言:「前代封使,奉命後每遲至三四年甚有十餘年而後臨臣國者。今使臣汪楫、林麟魛朝拜命而夕就道。且當海疆多故之時,沖風冒險,而臣國又僻在海東,封舟開駕,恃西南風以行,中道無可倚泊,常兼旬經月而後至,甚者水米俱盡,事不可言。今在五虎門開洋,僅三晝夜而達小國。臣遣官迎護,親見舟行之次,萬鳥繞篷而飛,兩魚夾舟而進,經過之處,浪靜波平,倏抵琉球內地,通國臣民以為僅見。仰惟皇上文德功烈,格天感神,且有御筆在船,故徵應若此也。乞宣付史館,以彰嘉瑞。」又疏請飭令使官收受所辭宴金,帝命收受。

二十五年,貞遣官生梁成楫、蔡文溥、阮維新、鄭秉鈞四人入太學,附貢使船,遭風桅折,傷秉鈞,飄至太平山修船,二十七年二月,始至京師。十月,貞遣陪臣來謝子弟入監讀書恩,並貢方物。帝令成楫等三人照都通事例,日廩甚優,四時給袍褂、衫褲、鞾帽、被褥咸備,從人皆有賜,又月給紙筆銀一兩五錢,特設教習一人,令博士一員督課。二十八年,貞疏言:「舊例,外國船定數三艘貨物得免收稅。今琉球進貢船止二艘,尚有接貢船一艘,未蒙免稅,請照例免收,以足三船之數。」又:「人數例帶一百五十人,萬里汪洋,駕舟人少,不能遠涉,乞準加增。」禮臣議免入貢船稅,人數不準加增,帝特令加增至二百人。三十二年,貞遣陪臣來貢,請入監讀書官生歸國。賜宴及文綺,乘傳厚給遣歸。自是二年一貢如常例。

四十八年,琉球國內多災,宮殿焚,颱颶頻作,人畜多死。是年王尚貞薨,世子尚純先卒。四十九年,尚純子尚益以嫡孫立。五十一年,卒,未及請封。五十二年,尚益世子尚敬立。比年遣使入貢,稱「世曾孫」。五十七年六月,命翰林院檢討海寶、編修徐葆光充正副使,往封琉球國世曾孫尚敬為王。

五十八年,琉球國建明倫堂於文廟南,謂之府學,擇久米大夫通事一人為講解師,月吉讀聖諭衍義;三六九日,紫金大夫詣講堂,理中國往來貢典,察諸生勤惰,籍其能者備保舉。八歲入學者,擇通事中一人為訓詁師教之。文廟在久米村泉崎橋北,創始於康熙十二年。廟中制度俎豆禮儀悉遵會典。琉球自入清代以來,受中國文化頗深,故慕效華風如此。五十九年,琉球國王尚敬疏請續送官生入監讀書,從之。

雍正二年,敬遣陪臣王舅翁國柱及曾信等奉表賀登極,貢方物,兼送官生鄭秉哲、鄭繩、蔡弘訓等入監讀書。帝召見國柱等,御書「輯瑞球陽」額賜王,並玉器、緞幣等物,交國柱齎回。官生蔡弘訓病卒,賜銀百兩,交禮官擇近京地葬之,並以二百兩贍恤其家。三年,敬遣使表謝方物,帝命準作二年一次正貢。四年,敬遣使入貢,並進謝表方物,命存留作六年正貢;其六年表文,俟八年正貢時並進。是年,貢使歸,附官生鄭秉哲等歸國。六年,敬仍遣使入貢,帝命作八年正貢;若八年貢使已經起程,即準作十年正貢。八年,敬遣使入貢,疏言請遵舊制二年一貢,不敢愆期。帝諭仍遵前旨行;若十年貢物已遣使起程,即準作十二年正貢,十一年不必遣使。

乾隆二年六月,琉球所屬之小琉球國有粟米、棉花二船遭風飄至浙江象山,浙閩總督嵇曾筠資給衣糧遣還。事聞,帝諭:「嗣後被風漂泊之船,令督撫等加意撫恤。動用存公銀兩,資給衣糧,修理舟楫,查還貨物,遣歸本國。著為令。」三年,敬遣陪臣奉表賀登極,並貢方物。帝命貢使齎回禦書「永祚瀛壖」額賜王,並諭不必專使謝恩,俟正貢之年一同奏謝。五年,敬遣使入貢,並進謝恩方物。六年,禮臣議琉球謝恩禮物照雍正四年例,準作二年一次正貢,從之。五月,浙江提督裴鉽奏言:「江南商民徐淮華等五十三人遭風飄入琉球之葉壁山,國王資遣都通事阮為標護送歸國。」帝命禮臣傳旨獎之。十五年,敬遣通事阮超群等送回十四年被風失舟之商民吳永盛等四船九十二人。其林士興等六船一百三十人,先已撥給桅木廩餼資送回閩。事聞,賜敬緞疋。十六年,福建巡撫潘恩矩奏言:「琉球貢使毛如苞等貢船遇颶,飄還本島,今修葺補進。又前有閩縣遵風船戶蔣長興等、常熟縣商民瞿長順等三十九人,留養兩年,今亦隨船回閩。」奉旨嘉獎。是年,王尚敬薨。

十九年,世子尚穆遣使入貢,兼請襲封。二十年,命翰林院侍讀全魁、編修周煌充正副使,往封琉球國世子尚穆為王。二十四年,穆遣使入貢,並遣官生梁文治等入監讀書。帝命所進方物準作二十五年正貢。是年,資送遭風商民金任之、照屋等五十三人回國。以後迄於光緒朝,凡琉球遭風難民,皆撫恤如例。二十九年,遣官生梁文治等歸國。四十九年,穆遣陪臣毛廷棟等入覲,行慶賀禮。御書「海邦濟美」額賜之,並賜玉、磁、緞匹諸物。五十五年,穆遣使入貢,並進謝恩方物,懇恩免抵正貢。帝命如所請行。五十八年,諭軍機大臣:「琉球貢船,現距年節兩月有餘,即飭伴送員按程從容行走,祗須封篆前到京,便與年班各外籓同與宴賚。」五十九年,穆遣使謝特賜「福」字、如意恩,貢方物。是年,王尚穆薨。世子尚哲先卒,世孫尚溫權署國事。

嘉慶三年,世孫尚溫遣使入貢,兼請襲封。是年,尚溫建國學於王府北,又建鄉學三,國中子弟由鄉學選入國學。四年,命翰林院修撰趙文楷、編修李鼎元充正副使,往封琉球國世孫尚溫為王,賜御書「海表恭籓」額。五年,尚溫遣陪臣子弟四人入監讀書。七年,琉球那霸官民集貲請於王,建鄉學四。八年,琉球二號貢船,至大武侖洋遭風漂至臺灣,沖礁擊碎,其正貢船亦同時漂沒,福州將軍玉德等以聞。帝諭救獲官伴、水梢人等,照常例加倍給賞,貢物無庸另備呈進。十二年,王尚溫薨,世子尚成署國事,未及受封,病卒。

七月,命翰林院編修齊鯤、工科給事中費賜章往封世孫尚灝為王。是年,琉球接貢船復遭風沉沒,帝命給銀千兩作雇船資用,另給銀五百兩恤淹斃六十三人家屬。道光二年,琉球貢船至閩頭外洋遭風擊碎,溺死貢使十名,帝命給銀千兩,雇商船回國,免另備貢物。

又琉球遭風難夷米喜阜等,每名日給鹽菜口糧,俟回國之日另給行糧一月。七年,琉球國王尚灝遣使入貢,並謝賜御書恩,貢方物,呈懇免抵正貢,允之。十七年,王尚灝薨,遣使往封世子尚育為王。

十九年,尚育遣使謝冊封及賞御書,貢方物。又疏請飭使臣受宴金,帝不允,令來使齎回。初,琉球舊例,間歲一貢,上年改為四年朝貢一次。二十年十一月,其國王籥請照舊,允之。其陪臣子弟四人,準隨同貢使北上入監讀書。

琉球國小而貧,逼近日本,惟恃中國為聲援。又貢舟許鬻販各貨,免徵關稅,舉國恃以為生,其貲本多貸諸日本。國中行使皆日本寬永錢;所販各貨,運日本者十常八九。其數數貢中國,非惟恭順,亦其國勢然也。

二十六年,琉球入監官生向克期回國,途中病故,恤銀三百兩。咸豐元年,琉球國王世子尚泰遣使賀登極,貢方物,懇免留抵,允之。帝諭軍機大臣曰:「琉球恪守籓封,前以英人伯德令住居伊國,久未撤回,頻來呼籥,當經飭令徐廣縉曉諭文安委婉開導,令其撤回。文安設詞推諉,該督仍當隨時體察情形,加意控馭。」三年,賜琉球御書「同文式化」額。四年,琉球世子遣使慶賀冊立大典,貢方物。時賊氛遍東南,郵傳多阻,諭令使臣無庸繞道來京,即由閩回國。使臣仍懇入都,帝命王懿德等俟來歲道路疏通,派員護送。八年,琉球入監官生毛啟祥途中病故,賜恤銀三百兩。九年,琉球貢使到閩,帝以貢使遠涉輸誠,命王懿德等察看情形,如閩省上游及江、浙諸省道路已通,即派員伴送來京。十年,琉球入監官生葛兆慶病故,營葬張家灣,賜恤金如例。

同治三年,琉球國世子遣使賀登極,貢方物。是年,英人與日本構釁,將襲取琉球,駐海軍,事尋解。五年,遣使齎敕印往封琉球世子尚泰為王。六年,尚泰遣陪臣子弟四人入監讀書。十年,有琉球船遭風漂至臺灣,為生番劫殺者五十四人。十一年,復劫殺日本小田縣難民四人,日本大譁。既,中、日立約天津,要求痛懲生番,恤琉球、日本死難諸人,且言琉球為日本版圖,藉口稱兵臺灣,語具邦交志。

光緒元年,琉球國貢使蔡呈祚回國病歿山東,賜葬費銀。五年,日本入琉球,滅之,夷為沖繩縣,虜其王及世子而還。總理衙門以滅我籓屬詰日本,日人拒焉。六年,帝命北洋大臣李鴻章統籌全局,鴻章奏言:「琉球原部三十六島,北部九島、中部十一島、南部十六島,而周回不及三百里。北部中有八島早屬日本,僅存一島。去年日本廢滅琉球,中國疊次理論,又有美前總統格蘭忒從中排解,始有割島分隸之說,此時尚未知南島之枯瘠也。本年日本人竹添進一來津謁見,稱其政府之意擬以北島、中島歸日本,南島歸中國。又議改前約。臣以琉球初廢之時,中國體統攸關,不能不亟與理論。今則俄事方殷,勢難兼顧。且日人要索多端,允之則大受其損,拒之則多樹一敵,惟有暫從緩議。因傳詢在京之琉球官尚德宏,始知中島物產較多,南島貧瘠僻隘,不能自立。而琉球王及其世子,日本又不肯釋還。適接出使大臣何如璋來書,復稱詢訪琉球國王,謂『如宮古、八重山小島另立三子,不止吾家不原,闔國臣民亦斷斷不服。南島地瘠產微,向隸中山,政令由土人自主。今欲舉以畀琉球,琉球人反不敢受,我之辦法亦窮』等語。臣思中國以存琉球宗社為重,本非利其土地。今得南島以封琉球,而琉球不原,勢不能不派員管理。既蹈義始利終之嫌,且以有用之兵餉,守甌脫不毛之地,勞費正自無窮。而道里遼遠,實有孤危之慮,若憚其勞費而棄之不守,適墜人狡謀。且恐西人踞之,經營墾闢,扼我太平洋咽喉,亦非中國之利。是不議改約,而僅分我以南島,猶恐進退兩難,致貽後悔。今之議改前約,儻能竟釋琉球國王,畀以中、南兩島,復為一國,其利害尚足相抵,或可勉強允許。不然,彼享其利,我受其害,且並失我內地之利,竊所不取也。臣愚以為日本議結琉球之案,暫宜緩允。」由是琉球遂亡。


\end{pinyinscope}