\article{列傳六十三}

\begin{pinyinscope}
石琳兄子文晟徐潮子杞貝和諾子馬喇陶岱

傅霽覺羅華顯蔣陳錫子漣泂劉廕樞

音泰鄂海衛既齊

石琳,漢軍正白旗人,石廷柱第四子。初授佐領,兼禮部郎中。康熙元年,出為山東按浙江鹽運使。十二年,轉湖廣下荊南道。十三年,襄陽總兵楊來喜、副將洪福以南漳叛應吳三桂,據房縣、保康、竹山。琳偕總兵劉成龍率師討之,撫定各峒寨。十五年,遷河南按察使。禁旅南征,牧馬開封,當麥秋,琳與統兵諸將帥約,令兵毋驛騷,坐帳中四十餘日。及去,民得穫麥。

二十年,遷浙江布政使。時耿精忠初平,衢州被兵尤甚,戶口逃亡,丁賦皆責之里甲。琳覈實,請免之。師行供億浩繁,民多逋負,琳悉為釐定,裁革陋規,禁加耗尤嚴。嘗曰:「革一分火耗,可增一分正供。」二十三年,擢湖北巡撫。工部以修建太和殿,檄各省採柟、杉諸木。琳言柟產萬山中,挽運甚艱,請寬其程限。部議不許。特詔允之。

二十五年,調雲南。疏言:「詳覈賦役全書,應更改者八事。雲南自明初置鎮設衛,以田養軍曰屯田。又有給指揮等官為俸,聽其招佃者曰官田。其租入較民賦十數倍,猶佃民之納租於田主。國初吳三桂留鎮,以租額為賦額,相沿至今,積逋愈多,官民交困。宜改依民賦上則起科。雲南鹽井有九,以各井行鹽之多寡為每歲徵課之重輕。瑯井鹽斤徵課六釐,白井八釐,至黑井則倍。明末加徵,較明初原額不啻數倍。今請減黑、白二井之課如瑯井例。開化民田畝科糧二斗六升三合,較未設府以前加至十倍。通省民糧,惟河陽最重,今當減半,與河陽一例。元江由土改流,三桂於額糧外別立名色:曰田地講銀,曰茶商稅銀,曰普洱無耗秋米,曰浪媽等六寨地租。加賦倍徵,民不堪命,應請各減其半。通海六寨地糧較民賦重幾三倍,當改依新定民賦科則。咢嘉每糧一石,徵條編銀四兩有奇,亦為偏重。今既歸南安州附徵,應與州賦一律,每糧一石,徵銀一兩四分。麗江界連土番,古稱荒服。三桂叛後,割金沙江以內喇普地與蒙番,地去而糧存,當刪除。建水自明時設參將,歲派村寨陋規銀三百有奇、糧八十餘石,三桂遂編入正額,當裁革。新平之銀場,易門之銅廠,礦斷山空,宜盡豁課稅。」疏入,下所司議,刊入全書頒行。

二十八年,擢兩廣總督。瓊州總兵吳啟爵奏瓊屬黎地,請設州縣,築城垣,增兵防守。命琳勘奏,力陳其非要,上從之。四十一年,連州瑤作亂,遣都統嵩祝等會剿,平之。琳規畫善後,定官吏管轄,撥兵移防,悉協機宜。未幾,卒官。

兄子文晟,初授薊州同知,歷云南開化、山西平陽知府。康熙三十三年,上嘉其居官有聲,超擢貴州布政使。是歲,即遷雲南巡撫。為政務舉大綱。雲南屯賦科重民田數倍,琳官巡撫時,奏減而未議行;文晟復疏請,特允減舊額十之六。安南國王黎維正疏告國內牛羊、蝴蝶、普園三地為鄰界土司侵占,乞敕諭歸還。會文晟入覲,上問文晟,奏言:「此地明時即內屬,非安南地。妄言擅奏,不宜允。」乃降詔切責之。四十三年,調廣東。四十四年,擢湖廣總督。坐劾容美土司田舜年僭妄淫虐非實,部議當降調,上命留任。文晟以疾乞退,上諭大學士曰:「文晟粗鄙,若為土司事而罷,似未得體。今既引疾,可允其請。」罷歸。五十九年,卒。

徐潮,字青來,浙江錢塘人。康熙十二年進士,選庶吉士,授檢討,累擢少詹事。潮學問淹通,在翰林,應奉文字,多出其手。聖祖嘗御門召講易、論語,敷陳明晰,為之傾聽。三遷至工部侍郎,督理錢局,清介不茍隨俗。局官冒濫事發,潮獨無所連染。三十三年,典會試。以母憂歸,服闋,起刑部侍郎。

三十九年,授河南巡撫,上諭之曰:「河南火耗最重,州縣多虧欠,爾當籌畫禁止。」潮上官,令火耗無過一分,州縣私派,悉皆禁革。南陽承解黑鉛,衛輝辦兌漕米,向皆假手胥吏,恣為侵漁。潮悉心區畫,宿弊悉除。開封五府饑,疏請漕糧暫徵改折,以平市直。歸德屬永城、虞城、夏邑三縣被災地畝至一萬七千餘頃,出糶常平、義、社倉穀,借給貧民牛種,全活甚眾。四十一年,上巡幸畿甸,問巡撫李光地鄰省督撫賢否,光地舉潮對。上褒美,以潮與光地、張鵬翮、彭鵬、郭琇並稱。四十二年,上南巡,潮迎駕泰安,賜冠服及御書榜額。其冬,西巡,復迎駕,賞賚有加。上念汾、渭皆入河,議於河南儲穀,遇山、陜歲歉,自水道移粟,便於陸運。命潮會陜西、山西督撫勘議。潮與川陜總督博霽會勘三門砥柱。語見博霽傳。又別疏言:「汴水通淮,一自中牟東經祥符至宿遷,湮塞已久;一自中牟東南經尉氏至太和,今名賈魯河,尚可通流:請量加疏濬。鄭州北別有支河,舊跡尚存,若於此建閘,使汴與洛通,尤為民便。」上從之。

四十三年,擢戶部尚書,充經筵講官,兼翰林院掌院學士,教習庶吉士。四十四年,扈從南巡,命赴河南按事。時上以高郵、寶應諸州縣頻年被水患,由洪澤湖無所宣洩,宜於高堰二壩築堤束水入河,又於下河築堤束水入海。會潮按事還,上詢河壖形勢,因指授方略,命往董其役。四十五年,監修高家堰滾水壩、高郵車邏中壩,並濬文華寺減河。四十六年,監修武家壩、天然壩、蔣家壩及諸堤閘,先後畢工。四十七年,調吏部。四十九年,以病乞休,許以原官致仕。五十四年,卒,賜祭葬。

潮居官平易,不事矯飾,所至民咸稱頌。乾隆初,追謚文敬。子本,自有傳。

杞,字集功。康熙五十一年進士,官編修。由甘肅布政使巡撫陜西,入為宗人府府丞。予休致,卒。

貝和諾,富察氏,滿洲正黃旗人,濟席哈孫。自工部筆帖式授戶部主事,歷郎中,兼佐領,累遷大理寺卿。康熙三十五年,命往山東經理閘河。漕運總督桑額奏漕船盡過濟寧,較往歲早一月。上以遣官經理,於漕運便,命以為常。遷左副都御史,擢戶部侍郎。三十七年,朝鮮歲祲,國王李焞乞開市義州中江貿穀。詔發三萬石與為市,令貝和諾及侍郎陶岱監視。事已,焞上表謝「八道生靈,賴以全活」。是年,授陜西巡撫。疏報:「陜西開事例,積貯米麥,應存一百七十七萬石有奇,今實存僅十七萬。」上命尚書傅臘塔、張鵬翮往按。尋疏言長安、永壽、華陰等糴補三十八萬有奇,餘皆欠自捐生,請令補完。

三十九年,調四川。疏言:「打箭爐、木鴉等處番、民一萬九千餘戶歸順,請增設安撫使五、副使五、土百戶四十五,以專管轄。邊民運茶赴爐貿易,給官引五千六百道,定額徵課。川省行鹽,潼川、中江山路崎嶇,難於陸運,額運壅滯。惟冰江小溪通水運,請增給水引,商民交便。」貝和諾治事精詳,尚書張鵬翮按事還,於上前亟稱之。四十二年,召授兵部侍郎。

四十四年,擢雲貴總督,捕治富民盜李天極、王枝葉等。天極廣通諸生,與臨安硃六非造為符讖,師宗州枝葉,人素無行,天極等誘之,詭託明桂王孫,糾黨謀不軌。僭稱文興三年,散播印劄,圖劫掠廣南、開化,自蒙自竄入會城。貝和諾標兵詗得狀,誅六人,流其餘黨。四十九年,召拜禮部尚書。以太原流匪陳四等六十餘人詭稱赴雲南墾地,貝和諾得布政使牒報不察究,坐降調,授盛京工部侍郎。五十七年,復召為禮部尚書,以老乞休,詔慰留。六十年,卒官。

子馬喇,襲管佐領,兼護軍參領,累擢正紅旗滿洲副都統。雍正五年,西藏阿爾布巴等與貝子康濟鼐不睦,命馬喇往駐西藏。既,阿爾布巴戕害康濟鼐,後藏頗羅鼐率兵報仇,執阿爾布巴等。遣尚書查郎阿等讞其罪,磔之。詔頗羅鼐總管前後藏事,移達賴喇嘛於里塘。七年,命馬喇駐里塘守護,賜帑金二千,總藏事。擢護軍統領。還京,遷工部尚書,坐免。十一年,復以副都統銜往西藏辦事。卒官。

陶岱,瓜爾佳氏,滿洲正藍旗人。由主事歷戶部郎中,累擢吏部侍郎。朝鮮告饑,乞開市貿穀,命陶岱與貝和諾運米給糴,禦制海運朝鮮記紀其事。康熙三十八年,署兩江總督。尋授倉場侍郎,以漕運遲誤,降五秩,隨旗行走。尋卒。

博霽,巴雅拉氏,滿洲鑲白旗人。自護衛授鑾儀使,擢鑲白旗都統。康熙二十四年,授江寧將軍,調西安。三十五年,撫遠大將軍費揚古率師西剿噶爾丹,命博霽率滿洲兵自寧夏會師,大敗噶爾丹於昭莫多。敘功,授世職拖沙喇哈番。聖祖嘗諭大學士等曰:「博霽自江寧赴西安,軍民攀留泣送,直至浦口。非有善政,何能如此?誠可謂將軍矣!」四十二年,上幸西安閱兵,諭曰:「西安官兵皆嫺禮節,重和睦,尚廉恥,且人才壯健,騎射精練。朕巡幸江南、浙江、盛京、烏喇等處閱兵,末有能及之者,深可嘉尚!」賚博霽御用櫜建、弓矢。

四十三年,授四川陜西總督。上以山、陜屢歲祲,欲於河南儲粟備賑,溯黃河挽運,慮三門砥柱水急,舟不得上,命博霽偕山、陜、河南巡撫會勘。尋合疏言:「三門灘多水激,挽運險阻,仍以陸運為便。」從之。四十七年,卒,賜祭葬。

覺羅華顯,滿洲正紅旗人。初授宗人府主事,遷戶部理事官。康熙三十七年,授翰林院侍講學士,累遷內閣學士。三十九年,授甘肅巡撫,未上官,調陜西。四十年,擢川陜總督。甘肅流民數千人就賑西安,華顯與巡撫鄂海出俸為有司倡,集資計口授糧,並撥荒地為業。上幸西安閱兵,與博霽、鄂海同受賜。陜民困重斂,華顯飭有司禁私徵,屏絕餽遺,軍民稱頌。四十二年,卒官,加太子太保,贈兵部尚書,謚文襄。祀陜西名宦。

蔣陳錫,字雨亭,江南常熟人。父伊,康熙十二年進士,選庶吉士,授御史。疏陳民間疾苦,繪十二圖以進。累官河南提學道副使,卒官。

陳錫,康熙二十四年進士,授陜西富平知縣。歲饑,米斛直數千,發倉賑濟,不給,斥家資佐之,全活甚眾。行取,擢禮部主事。監督海運倉,革糧艘篷席例銀。遷員外郎。河道總督張鵬翮薦佐兩淮河務。四十一年,授直隸天津道,遷河南按察使,讞決平恕。豫省有老瓜賊為害行旅,陳錫廉得其巢穴,悉擒治之。

四十七年,遷山東布政使。未幾,擢任巡撫。疏請緩徵二十三州、縣、衛被災逋賦,廣鄉試解額,增給買補營馬直,免累及所司。條陳海防三事,言戰船當更番修葺,水手當召募熟諳水道之人,沿海村莊當舉行團練,互相接應;並以御史陳汝咸條議海疆弭盜,疏請漁舟編甲,閩、粵鳥船不許攜砲械,得盜舟火藥軍器,必究所從來。部議悉從之。長蘆巡鹽御史希祿請增東省鹽引,臨清關請增設濟寧等五州縣口岸,陳錫皆言其不便,並得請。

五十五年,擢雲貴總督。祿勸州土酋常應運誘沿江土夷攻卓乾寨,陳錫檄師會剿,平之,撥兵弁駐守其地。石羊緒礦廠硐老山空,課額不足,疏請嗣後硐衰即止,勿制定額。鎮遠至省三十二驛,山路崎嶇,驛夫苦累,下令非有符合,毋濫應夫馬。都統武格、將軍噶爾弼率師入西藏,以雲南糧運艱難,欲自四川運糧濟給。四川總督年羹堯奏言滇、蜀俱進兵,蜀糧不足兼供。乃命陳錫與巡撫甘國璧速運。五十九年,詔責其籌濟不力誤軍機,與國璧並奪職,令自備資斧運米入藏。明年,卒於途。雍正元年,山東巡撫黃炳言陳錫在巡撫任,侵蝕捐穀羨餘銀二百餘萬,部議督追。弟廷錫入陳始末,詔減償其半。子漣、泂。

漣,字檀人。進士,官編修,終太僕寺卿。

泂,字愷思。進士,歷工部郎中,出為雲南提學道。西陲用兵,命從軍,授甘肅涼莊道。西檄多卜藏、瑪嘉諸部與謝勒蘇、額勒布兩部逃人倚石門寺為巢,往來劫掠。泂料簡精銳,會涼州鎮官兵,分五路進剿,轉戰棋子山,殲賊之半。時羅卜藏丹津進逼西寧,復檄兵捍禦,羅卜藏丹津遁走。大將軍年羹堯上其功,遷山西按察使,進布政使。上嘉泂實心供職,免其父追償。雍正十年,加侍郎銜,往肅州經理軍營屯田。在事二年,闢鎮番柳林湖田十三萬畝,得糧三萬石。築河堤,擴二大渠,分濬支渠,並建倉儲糧,公私饒裕。副都御史二格協理軍需,劾泂侵帑誤公,逮治論死,下獄追贓。總督查郎阿等交章雪其誣,泂已病卒。

劉廕樞,字喬南,陜西韓城人。康熙十五年進士,授河南蘭陽知縣,有政聲。行取,擢吏科給事中,以憂歸,服闋,除刑科給事中。疏言:「廉吏必節儉。邇來居官競尚侈靡,不特車馬、衣服、飲食、器用,僭制逾等;抑且交結、奔走、餽送、夤緣,棄如泥沙,用如流水。俸不給則貸於人,玷官箴,傷國體。請敕申斥,以厲廉戒貪。」又疏言:「京師放債,六七當十;半年不償,即行轉票,以子為母。數年之間,累萬盈千。是朝廷職官,竟為債主廝養。乞敕嚴立科條,照實貸銀數三分起息。」並下部議行。尋調戶科。三十六年,詔求直言,廕樞疏請肅紀綱,覈名實,開言路,報可。

三十七年,外轉江西贛南道。贛俗健訟,廕樞晝夜平決,懲妄訴者,訟漸稀。將吏私徵門稅,廕樞令革之。米市有牙課,牙人籍以婪索。廕樞以其錢置田,徵租代課,除民累。署按察使,忤總督阿山,以讞獄前後獄辭互異,劾罷。四十二年,聖祖西巡,廕樞迎駕潼關,上識之,召對稱旨,復授雲南按察使。四十五年,遷廣東布政使。總督貝和諾稱其清廉勤慎,士民愛戴,雲南布政使缺員,請以廕樞調補,上從之。廕樞督濬昆明湖,築六河岸徬。會夏旱,發粟平糶,禱於五華山,得雨,民大悅。

四十七年,擢貴州巡撫。貴州苗、仲雜處,號難治。廕樞至,絕餽遺,省徭役,務以安靜為治。疏請廣鄉試解額,設南籠學,以振人文。先後請改石阡、丹川、西堡、寧谷、平州、大華諸土司,設流官。開驛道,自雲南坡至蕉溪二千餘里。又疏言貴州錢糧課稅僅十餘萬,鄰省歲協餉二十餘萬,稍愆期,軍士懸額待餉。請豫撥二十萬儲布政使庫。部議持不可,疏三上,詔特允之。其後紅苗叛,餉賴以無絀。烏蒙、威寧兩土司相仇殺,四川巡撫年羹堯遣吏勘問,土酋匿不出,疏聞,命四川、雲、貴督撫按治。廕樞先至,遣使招諭,威寧土酋聽命,烏蒙土酋亦自縛出就質,咸原伏罪釋仇,苗以無事。

五十四年,準噶爾策妄阿喇布坦侵哈密,詔備兵進討。廕樞累疏請緩師,略云:「小醜不足煩大兵。原皇上息怒,重內治,輕遠略。」上責其妄奏,命馳驛赴軍前周閱詳議。廕樞抵巴里坤,上疏數千言,請屯兵哈密,以逸待勞。旋稱病還甘肅,疏乞休,嚴旨譙讓,仍令回巡撫任。廕樞疏報病愈,上斥廕樞:「令詣軍前即稱病,令回任病頓愈,情偽顯然。」命解任詣京師。部議阻撓軍務,坐絞,上宥之,遣赴喀爾喀種地。年已八十二,居戍三年,釋還,復故官。六十一年,與千叟宴。世宗御極,召見,賜金歸里。尋卒,年八十七。

音泰,瓜爾佳氏,滿洲鑲紅旗人。初為西安駐防兵。康熙十三年,副都統佛尼勒討吳三桂將譚弘、吳之茂、王屏籓等,音泰隸麾下。師自漢中進克陽平朝天關,駐守梅嶺關,賊夜劫營,音泰力禦,中槍折齒,得上賞。明年,佛尼埒攻王輔臣秦州,臨壕列圍,賊突騎出犯,音泰射殪三人,賊駭遁。復進攻西和,屢敗之茂等於鹽關岐山堡。十七年,進攻四川,克保寧、敘州。敘功,授驍騎校,遷防禦。

三十五年,署參領,從西安將軍博霽會大將軍費揚古征噶爾丹,出西路。五月,上親征,出中路,至克魯倫河。值積雨,運糧滯,賊預焚草地,我軍紆道秣馬。音泰言於博霽曰:「聖駕親征,宜倍道前進。」乃急趨昭莫多,大軍繼進,噶爾丹敗遁。敘功,予雲騎尉世職。四十一年,遷佐領。四十二年,上巡西安,令官兵校射,音泰蒙賚與賜宴,尋授協領。

四十三年,擢西安副都統。四十四年,授西寧總兵官。上知其貧,詔陜西督撫助練兵犒賚之資。四十六年諭:「音泰久居西陲,諳習兵事,外籓蒙古及內地軍民交口稱譽。」命擢甘肅提督。四十八年,授川陜總督。入覲,賞花翎及冠服、鞍馬,並御書「攬轡澄清」榜賜之。

四十九年,幹偉番蠻羅都等掠寧番衛,戕冕山營游擊周玉麟,命四川巡撫年羹堯偕提督岳升龍往剿。羹堯至,升龍已擒羅都等三人械送勘問。既定讞,遂先還。升龍偕建昌總兵郝弘勛至會鹽招降,番蠻諸酋原率眾十萬貢納糧馬。音泰請以降酋為土司,分領其眾。因劾羹堯違旨先還,詔奪羹堯職,留任效力。未幾,升龍以疾解任,羹堯知其曾假帑金,議率屬捐俸代償,音泰不從。羹堯遂入告,上允行,並諭音泰宜與巡撫和衷。尋褒其潔清不瞻徇,實心任事。會奉詔申禁游民越境,令嚴劾縱容官吏。邠州諸屬拘系者四十餘案,每案至數十人。音泰疏言諸人皆藉技營生,無不法狀,應遞解原籍編管;如縱出境,議處所司,上韙之。

以病疏乞休,上曰:「朕前幸西安,知音泰義勇,洊擢至總督。寬嚴並用,軍民無不感戴。朕甚愛惜之,可令在任調攝。」五十二年,復請,許解任還京師,給第宅田畝,以旌其廉。並諭群臣曰:「朕初用音泰,人不知其善,後乃稱朕有知人之明也。」五十三年,卒,賜祭葬,謚清端。初授雲騎尉世職,特命世襲罔替。

鄂海,溫都氏,滿洲鑲白旗人。自筆帖式授內閣中書,歷宗人府郎中,兼佐領。康熙三十二年,聖祖親征噶爾丹,命鄂海赴寧夏儲備牲畜。陜西按察使員缺,上以命鄂海,且諭之曰:「初任外僚,每言潔其身以圖報。及蒞任,輒背其言。朕於數十從臣中簡爾為按察使,爾當益勵素行也。」三十七年,遷布政使。四十年,擢巡撫。

四十九年,授湖廣總督。鎮筸邊外紅苗為亂,令總兵張穀貞等召苗目宣諭,毛都塘等五十二寨、盤塘等八十三寨,先後薙發歸化,上嘉之。五十二年,移督川、陜。疏報甘肅洮、岷邊外大山生番請歸化,上以洮、岷邊外無生番,或為蒙古屬部,命詳察。鄂海奏大山在洮州東南土司楊汝松界外,非蒙古屬部,宜令汝松兼轄;復疏報四川會川營界外涼山番目阿木哨請歸化,歲貢馬,請給番目職銜,令轄所屬番、民:並從之。甘肅靖遠、固原、會寧歲歉饑,民乏食,疏給口糧資本,撫輯流移。

五十七年,大將軍貝子允等率師討策妄阿喇布坦,駐兵西寧、甘州、莊浪諸處。鄂海請發西安庫帑四十萬,並撥平涼、鞏昌、寧夏倉穀十萬,充餉;以陜西葭州、甘肅寧夏等二十八處轉輸軍需,請豁丁糧,紓民力。五十八年,復請豁甘肅逋欠錢糧草束,俾民得盡力輸納本年糧草以佐軍,戶部格不行,特旨允之。六十年,詔解任專治糧餉,以四川巡撫年羹堯代之。未幾,命往吐魯番種地效力。雍正元年,予原品休致,效力如故。尋卒。

衛既齊,字伯嚴,山西猗氏人。父紹芳,字猶箴,順治三年進士,授河南尉氏知縣。兵後修復城郭、學校,勤勸課,廣積儲,禁暴戢奸,尉氏民頌焉。行取兵部主事,累遷貴州提學道僉事、浙江巡海道副使。

既齊,康熙三年進士,改庶吉士,散館授檢討。講學志當世之務,上疏言時事,語戇直。會遭祖母喪,假歸。居久之,詣京師補官。上命以對品調外,授直隸霸州州判。既齊召民之秀良者曹試而教誨之,俾各有所成就。民貸於旗丁,子錢過倍,橫索無已。既齊力禁戢之,無敢逞。迭署固安、永清、平谷知縣,所至輒有惠政。巡撫於成龍疏薦。會既齊以母憂去,繼復遭父喪。一日,上御門,舉既齊諮於九卿,僉曰賢,命復授檢討。二十七年,服闋,詣京師補官。上知既齊講學負清望,超擢山東布政使。既齊感激,益自奮勉為清廉,令府縣輸款封還平餘。門懸鉦,吏民白事得自通。建歷山書院,仿經義、治事之例,設奎、壁二齋課士。護巡撫印者再。清庶獄,結八十餘案,株累數百人盡釋去。在官三年,有聲績。三十年,授順天府尹,疏請按行所部,黜陟屬吏賢不肖。上以為無益,不許。尋擢副都御史,聞山、陜蝗見,平陽以南尤甚,疏請賑恤,上責其懸揣。

旋授貴州巡撫。紹芳為提學,士民祠焉。既齊至貴州,謁父祠受事。黎平知府張瀲、副將侯奇嵩報古州高洞苗金濤匿罪人殺吏,請發兵進剿,既齊疏聞,即遣兵捕治;瀲、奇嵩復報兵至斬苗一千一百一十八人,既齊復以聞。旋察知瀲、奇嵩妄報,疏實陳,請奪瀲、奇嵩官勘治。上責既齊輕率虛妄,遣尚書庫勒納、內閣學士溫保往按。旋命逮既齊至京師,上令九卿詰責。既齊引罪請死,九卿議當斬,上命貸之,遣戍黑龍江。明年,赦還。家居,立社課士,斥家資供膏火。三十八年,上命承修永定河工。三十九年,又命督培高家堰,卒工次。

論曰:康熙中葉後,天下乂安,封疆大吏多尚廉能,奉職循理。若石琳改賦役,徐潮革火耗,博霽、華顯、音泰整飭武備,安不忘危,皆能舉其職者。劉廕樞志在休民,未知應兵之不容已,蔣陳錫、鄂海又以督餉稽遲蒙譴,衛既齊遭際殊異,而不獲以功名終,其治行皆有可稱,膏澤及於民,無深淺遠近,要為不沫矣。


\end{pinyinscope}