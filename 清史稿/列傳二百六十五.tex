\article{列傳二百六十五}

\begin{pinyinscope}
循吏三

張吉安李毓昌龔景瀚蓋方泌史紹登李賡蕓

伊秉綬狄尚絅張敦仁鄭敦允李文耕劉體重子煦

張琦石家紹劉衡徐棟姚柬之吳均王肇謙曹瑾

桂超萬張作楠雲茂琦

張吉安,字迪民,江蘇吳縣人。乾隆四十二年舉人,六十年,大挑知縣,發浙江。時清治各縣虧空,責彌補。富陽令惲敬獨不奉上官意旨,檄吉安往摘印署事。至則士民群集,乞留敬。吉安見之,默然徒手返,白臺司曰:「惲敬賢吏,乞保全之。且州縣賦入有常經,前官不謹致虧,責彌補於後來者,恐開掊克之漸。方今楚、豫奸民蜂起,皆以有司貪殘為口實。宜用讀書人加意拊循,乃無形之彌補耳。」聞者迂其言。委攝縣丞及杭州府通判,吉安自以不諧於時,乞改教職,上官留之。

嘉慶二年,署淳安,尋調象山。海盜由閩擾浙,沿海窮民業漁鹽者,多以米及淡水火藥濟盜,且為鄉導。吉安革船埠商漁之稅,嚴禁水、米出洋,盜漸窮蹙,值颶風覆盜艇,泅至岸,悉為舟師所獲。提督李長庚嘆曰:「牧令盡如張象山,盜不足平也。」又建議縣境南田為海中大島,宜如明湯和策,封禁以斷盜翼。韭山當海盜之沖,石浦、昌國兵力皆薄,請增兵以資鎮懾。事雖見格,後卒如所議。

四年,署新城,漕倉設省城,民輸折色,縣官浮收,運丁需索,習以為常。吉安平其折價,不及舊時十之六七,民感之。

五年,署永康,蛟水猝發,田廬蕩析,為棚廠以棲災民,阻水者具舟餉之,溺者具棺厝之,不待申詳報可,所以賑恤者甚至。上官或斥其有違成例,巡撫阮元素重之,悉如所請。六年,調署麗水,竭誠禱雨,旱不為災。縣多山,民處險遠者,艱於赴愬。吉安輒巡行就山寺讞獄,咸樂其便。

八年,署浦江,值水災,奸民糾眾掠富室,伐墓樹,鄰邑咸煽動。吉安曰:「非法無以止奸民,非米無以安良民,良民安則奸民氣散。」請運兵米所餘以賑之,民心漸定,乃擒首惡治如律。補餘杭,九年春,雨傷禾,糶倉穀以平米價,又運川米千石濟之。十年,復被水,分鄉設廠,煮粥以賑,規畫詳密,竟事無擁擠之擾。邑多名區,次第修復之。懲訟師,勤聽斷,修志、葺學,文教丕振。在餘杭七年,引疾歸,遂不出。歿後,永康士民請祀名宦,建立專祠。

當時吏治積弊,有南漕北賑之說,南利在漕,相率諱災。督撫藉詞酌劑,置災民於不問。茍有切求民瘼者,轉不得安於位。吉安官浙前後幾二十年,所蒞多災區,皆能舉職。在新城減漕之三四,時論尤以為難。北賑之弊亦然。同時江蘇知縣李毓昌,以不扶同侵賑致禍,仁宗優恤之,重懲諸貪吏,蓋欲以力挽頹風雲。

毓昌,字皋言,山東即墨人。嘉慶十三年進士,以知縣發江蘇。十四年,總督鐵保使勘山陽縣賑事,親行鄉曲,鉤稽戶口,廉得山陽知縣王伸漢冒賑狀,具清冊,將上揭。伸漢患之,賂以重金,不為動,則謀竊其冊,使僕包祥與毓昌僕李祥、顧祥、馬連升謀,不可得,遂設計死之。毓昌飲於伸漢所,夜歸而渴,李祥以藥置湯中進。毓昌寢,苦腹痛而起,包祥從後持其頭,叱曰:「若何為?」李祥曰:「僕等不能事君矣。」馬連升解己所系帶縊之。伸漢以毓昌自縊聞。淮安知府王轂遣驗視之,報曰:「尸口有血。」轂怒,杖驗者,遂以自縊狀上。

其族叔李太清與沈某至山陽迎喪,檢視其籍,有殘稿半紙,曰:「山陽知縣冒賑,以利啗毓昌,毓昌不敢受,恐負天子。」蓋上總督書稿,諸僕所未及毀去者。喪歸,毓昌妻有噩夢,啟棺視,面如生。以銀針刺之,針黑。李太清走京師訴都察院,命逮王轂、王伸漢及諸僕,至刑部會訊。山東按察使硃錫爵驗毓昌尸,惟胸前骨如故,餘盡黑。蓋受毒未至死,乃以縊死也。仁宗震怒,斬包祥,置顧祥、馬連升極刑,剖李祥心祭毓昌墓。轂、伸漢各論如律,總督以下貶謫有差。贈毓昌知府銜,封其墓。禦制愍忠詩,命勒於墓上。毓昌無子,詔為立後,嗣子希佐賜舉人,太清亦賜武舉。

龔景瀚,字海峰,福建閩縣人。先世累葉為名宦。曾祖其裕,康熙初,以諸生從軍,授江西瑞州府通判。滇、閩變起,率鄉勇為大軍鄉導,擢吉安知府。時府城為逆將所據,大軍駐螺子山,其裕供餉無乏。城復,撫瘡痍,多惠政。後官河南懷慶知府,濬順利渠,引濟水入城便民,終於兩淮鹽運使。歿祀瑞州、吉安、懷慶名宦祠。祖嶸,初仕浙江餘杭知縣,治縣民殺僕疑獄,為時所稱。擢直隸趙州直隸州知州,濬河興水利。再擢江蘇松江知府,渡海賑崇明災黎,全活甚眾。官至江西廣饒九南道,單騎定萬年縣匪亂,歿祀饒州名宦祠。父一發,乾隆十五年舉人,官河南知縣,歷宜陽,密縣、林縣,虞城四縣,治獄明敏,能以德化。在虞城值水災,勤於賑恤。朝使疏治積水,釃為惠民、永便諸渠,一發與災民共勞苦,治稱最。以病去,復起補直隸高陽。擢雲南鎮南知州,歿祀虞城名宦祠。

景瀚承家學,幼即知名。大學士硃珪督閩學,激賞之。乾隆三十六年成進士,歸班銓選。四十九年,授甘肅靖遠知縣,未到官。總督福康安知其能,檄署中衛縣,判牘如流,見者不知為初仕也。七星渠久淤,常苦旱,景瀚築石壩,遏水入渠,始通流。又濬常樂、鎮靜諸渠,重修紅柳溝環洞及減水各徬,溉田共三十萬畝,民享其利。五十二年,調平涼,地磽瘠,缺米粟,景瀚請鄰邑無遏糶。又當西域孔道,車馬取給商賈。鹽引敕派於民,官吏強買煤炭,皆為民病,一切罷之。由是商賈輻輳,食貨流通。修柳湖書院,與諸生講學,文風漸振。

五十五年,署固原州,漢、回雜處,時構釁。景瀚密偵諸堡,誅積匪,境內以安。五十九年,遷陜西邠州知州,嘉慶元年,總督宜綿巡邊,調景瀚入軍幕,遂從剿教匪,以功擢慶陽知府。宜綿總轄三省,從入蜀,幕府文書皆屬景瀚。尋調蘭州,仍在軍充翼長。

景瀚從軍久,見勞師糜餉,流賊仍熾,因上議備陳調兵、增兵、募勇三害,剿賊四難,謂:「先安民然後能殺賊,民志固則賊勢衰,使之無所裹脅。多一民即少一賊,民居奠則賊食絕,使之無所擄掠。民有一日之糧,即賊少一日之食。用堅壁清野之法,令百姓自相保聚,賊未至則力農貿易,各安其生;賊既至則閉柵登陴,相與為守。民有恃無恐,自不至於逃亡。其要先慎簡良吏,次相度形勢,次選擇頭人,次清查保甲,次訓練壯丁,次積貯糧穀,次籌畫經費。如是行之有十利。」反復數千言,切中事理。嗣是被兵各省舉仿其法,民獲自保,賊無所逞,成效大著。論者謂三省教匪之平,以此為要領。

五年,始到蘭州任,七年,送部引見,卒於京師。其後續編皇清文穎,仁宗特出其堅壁清野議付館臣載入。祀蘭州名宦祠。自其裕至景瀚,四世皆祀名宦,海內稱之。

景瀚子豐穀,官湖北天門知縣,亦有治績,不隳家聲焉。

蓋方泌,字季源,山東蒲臺人。嘉慶初,以拔貢就職州判,發陜西,署漢陰通判、石泉知縣。三年,署商州州同。治州東百里曰龍駒寨,寨之東為河南,南出武關為湖北。路四通,多林莽山徑,易憑匿。時川、楚教匪屢由武關入陜西。方泌始至,民吏掃地赤立,賊酋張漢潮擁眾至,乃罝藥面中,誘賊劫食,多死,遂西走,大軍乘之,漢潮由是不振。方泌集眾謀曰:「賊雖去,必復來。若等逃亦死,守不得耕種亦死。我文官無兵,若能為吾兵,當全活爾。」眾曰:「惟命。」乃築堡聚糧,戶三丁抽一,得三千人,無丁者以財佐糧糗兵械,親教之戰,辰集午散,無廢農事。

四年,賊屯山陽、鎮安,將東走河南,迎擊敗之;又擊賊於鐵峪鋪,賊據山上,而伏其半於溝,乃分兵翦伏,奪據東山上,數乘懈擊之,賊宵遁。後賊由雒南東逸,方泌馳至分水嶺,間道走鐵洞溝出賊前伏待之,賊錯愕迎戰,遂敗,斬首數百,鄉兵名由是大振。自武關至竹林關,鄉兵皆請隸龍駒寨。

五年,知州困於賊,方泌馳百九十里至北灣,賊驚曰:「龍駒寨兵至矣!」時賊屯州西及雒南、山陽各萬餘人,欲東出。方泌勒鄉兵二萬,列三大營以待。會官軍至,夾攻,賊大敗,幾盡殲。是役枕戈而寢者五十日。游擊某誣以事,解職,大吏直之,得留任。賊遂相戒無過商州。

八年,授盩厔知縣,猶時時入山搜賊,又獲寧陜倡亂者四十餘人。境內甫定,捐俸賑饑,旌死節婦,河灘、馬廠、鹽法,皆區畫久遠。擢寧陜同知。仁宗召見,問商州事甚悉。擢四川順慶知府。渠縣民變,大吏屬以兵。方泌曰:「此賽會人眾,至各相驚疑,訛言橫興,非叛也。」捕十二人而變息。調成都,母憂歸。服闋,授福建延平。尋調臺灣,兩署臺灣道。屢讞大獄,皆聚眾洶洶,稍激則變。方泌一以理喻,蔽罪如法。道光十八年,卒。

史紹登,字倬云,江蘇溧陽人,大學士貽直之孫。以謄錄敘布政司經歷,發雲南。乾隆六十年,署文水知縣。時滇鹽歸官辦,民苦抑配,紹登弛其禁,釋逋課者數百人。閱三載,配鹽之五十七州縣悉改商辦,以文山為法。

貴州苗亂,距文山尚數郡,紹登策其必至,集胥役健者親教技擊以備之。嘉慶元年,苗竄鄰境之丘北,潛與文山儂、惈通。紹登謂不救丘北,文山儂、惈必不靖,親率三百人往,人授刀一、鐵鑣三十。既至,當者輒僕,丘北廓清。而總督勒保剿苗失利,被圍於貴州黃草坪,巡撫江蘭檄紹登往援。至則賊圍數重,內外不相聞,七戰皆捷,乃達黃草坪。會貴州援兵亦至。比紹登上謁,總督曰:「若文官,亦遠來問我耶?」紹登陳解圍狀,不信。紹登請視戰所賊尸,鑣傷者,文山民壯所擊;若刃傷,請伏冒功罪。總督初欲劾之,勘實乃已。巡撫聞紹登忤總督,大懼,令所用軍費不得入報銷,以是虧帑二萬。

尋兼署蒙自縣事,兩城相距三百里。交阯賊儂福結粵匪犯文山,紹登馳一晝夜入城,率民壯出剿,擒其渠,峒卡悉復。擢雲州知州,仍留文山任。

四年,初彭齡來為巡撫,性好察,開化總兵因蒙自變時怯懦為民所輕,銜紹登,譖之,遂以虧空劾。士民刊章臚紹登政績,設匭醵金至三萬。彭齡聞之悔,以完虧奏留任,餘金無可返,建開陽書院焉。

七年,署維西通判。民恆乍繃為亂,巢險不可攻。紹登廉得巢後巖壁陡絕,阻大溪,乃以篾為絙,募善泅者系絙巖樹,對岸急引,如笮橋,攀援以登,壯士三百人從之。賊大驚亂,擒馘凈盡。九年,卒。

李賡蕓,字鄦齋,江蘇嘉定人。少受學於同縣錢大昕,通六書,蒼、雅,三禮。乾隆五十五年進士,授浙江孝豐知縣。調德清,再調平湖。下車謁陸隴其祠,以隴其曾宰嘉定,而己以嘉定人宰平湖,奉隴其為法,盡心撫字,訓士除奸,邑中稱神明。嘉慶三年,九卿中有密薦之者,詔詢巡撫阮元,元奏:「賡蕓守潔才優,久協輿論,為浙中第一良吏。」引見,以同知升用。五年,金華、處州兩郡水災,金華苦無錢,處州苦無米。賡蕓奉檄,於恩賑外領銀二萬,便宜為之。以銀之半易錢,運金華加賑,人百錢而錢價平。又以銀之半運米至處州,減價糶,轆轤轉運,而米亦賤。升處州府同知,調嘉興海防同知,署臺州府。尋擢嘉興知府,正己率屬,無敢以苞苴進者。治漕,持官、民、軍三者之平,上官每用其言。十年,水災,減糶有實惠,賑民以粥,全活者眾。以繼母憂去官。

服闋,補福建汀州,調漳州。俗悍,多械斗,號難治。賡蕓召鄉約、里正問之曰:「何不告官而私斗為?」皆曰:「告官,或一二年獄不竟,竟亦是非不可知,先為身累。」賡蕓曰:「今吾在,獄至立剖。有不當,更言之,無所徇護。為我告鄉民,後更有鬥者,必擒其渠,焚其居,毋恃賄脫。」眾皆唯唯退。已而有鬥者,賡蕓立調兵捕治,悉如所言,民大懼。賡蕓日坐堂皇,重門洞開,愬者直入,命役與俱。召所當治者,限時日。不至,則杖役。至則立平之釋去。即案前書獄詞,無一錢費。民皆歡呼曰:「李公活我!」漳屬九龍嶺多盜,下所屬嚴捕,擒其魁十數,商旅坦行。故事,獲盜當甄敘,悉以歸屬吏。尋擢汀漳龍道。二十年,擢福建按察使,署布政使,逾年實授。

賡蕓守漳州時,龍溪縣有械斗,令懦不治。署和平令硃履中內狡而外樸,庚蕓誤信之,請以移龍溪。久之,事不辦,始稔其詐。洎署布政使,改履中教職。履中虧鹽課,恐獲罪。具揭於總督汪志伊、巡撫王紹蘭,謂虧帑由道府婪索。督撫密以聞,解賡蕓職質訊。賡蕓之去漳,監造戰船工未竣,留僕督率之,僕假履中洋銀三百圓,詭以墊用告。賡蕓如數給之,僕匿不以償。福州知府塗以輈鞫之,阿總督意,增其數為一千六百,逼令自承,辭色俱厲,賡蕓終不肯誣服。慮為獄吏所辱,遂自經。

事聞,命侍郎熙昌、副都御史王引之往按其獄,得白。上以賡蕓操守清廉,眾所共知。其死由汪志伊固執苛求,而成於塗以輈勒供凌逼,褫志伊職,永不敘用。以輈、履中俱譴戍黑龍江,紹蘭亦以附和革職。

賡蕓家不名一錢,歿無以殮。鹽法道孫爾準與之善,為經紀其喪。初,志伊亦重賡蕓,曾薦舉之。及擢布政,乘新輿上謁,志伊諷以戒奢,賡蕓曰:「不肖為大員,不欲效布被脫粟之欺罔。」志伊素矯廉,銜其語。又以遇事抗執,嫌益深。及獄起,履中忽自承妄訐,諉原揭為其僕竊印,志伊怒,必窮詰之。論者謂漳廠修船,例由龍溪縣墊款,籓司發款,至道乃償之,非贓私也。賡蕓狷急,負清名,慮涉嫌不承,而志伊峻待紳士,不理於眾。與賡蕓善者,或以飛語中之。

方治獄使者至閩,士民上書為賡蕓訟冤,感泣祭奠,踵接於門,為建遺愛祠。熙昌等據情奏請賜額表揚,仁宗以「大員緣事逮問,當靜俟國法,若此心皦然,橫遭冤枉,亦應據實控告,朝廷必為昭雪;乃效匹夫溝瀆之諒,殊為褊急,不應特予旌揚。士民追思惠政,捐貲立祠,斯則斯民直道之公,聽之」。

伊秉綬,字墨卿,福建寧化人。乾隆五十四年進士,授刑部主事,遷員外郎。嘉慶三年,出為廣東惠州知府,問民疾苦,裁汰陋規,行法不避豪右,故練刑名,大吏屢以重獄委之,多所矜恤。陸豐巨猾肆劫勒贖,秉綬設方略,縛其渠七人戮之。六年,歸善陳亞本將為亂,提督孫全謀不發兵,秉綬乃遣役七十餘人夜搗其巢,擒亞本,餘黨竄入羊矢坑。未幾,博羅陳爛屐起事,請兵,提督復沮之。秉綬爭曰:「發兵愈遲,民之傷殘愈甚。」提督不得已,予三百人。秉綬復曰:「偵虛實,則三四人足矣。如用兵,以寡敵眾,徒僨事耳。」提督不聽,令游擊鄭文照率三百人往,孑身跳歸,亂遂成。秉綬適以他事罣議去官,士民籥留軍營。時提督既擁兵不前,其標兵卓亞五、硃得貴均通賊縱掠,為偽渠帥。秉綬憤懣,請兵益力,逢總督吉慶之怒,復以失察教匪論戍。會新總督倭什布至惠州,士民數千人訴秉綬冤,上聞,特免其罪,捐復原官,發南河,授揚州知府。

時秉綬方奉檄勘高郵、寶應水災,刺一小舟,棲戶枉渚,必親閱手記。及蒞任,劬躬率屬,賑貸之事,錙銖必覈,吏無所容其奸。倡富商巨室捐設粥廠,費以萬計。誅北湖劇盜鐵庫子輩,杖詭道誑愚之聶道和,它奸猾擾民者,悉嚴治之。民雖饑困,安堵無惶惑。歷署河庫道、鹽運使,胥稱職。尋以父憂去,家居八年,嘉慶二十年,入都,道經揚州,卒。

秉綬承其父朝棟學,以宋儒為宗。在惠州,建豐湖書院,以小學、近思錄課諸生;在揚州,宏獎文學。歿後士民懷思不衰,以之配食宋歐陽修、蘇軾及清王士禎,稱四賢祠。

狄尚絅,字文伯,江蘇溧陽人,寄籍順天。乾隆四十六年進士。五十七年,授安徽黟縣知縣,父憂去。嘉慶四年,起復,發廣東,署化州知州。瀕海獷悍,尚絅解除煩苛,治以簡易。補花縣,以鄉兵助剿博羅亂匪有功,旋攝香山。十年,銓授江西南康知府。有武舉調族侄婦,羞忿自盡,以無告發,事寢有年矣。尚絅甫下車,武舉以他事涉訟,反覆詰問,忽露前情。窮究得實,置諸法,群驚為神。不期年,理滯獄百餘,盡得情實。饒州有兩姓爭田,世相仇殺,尚絅為判斷調和,爭端永息。南安會匪李詳誥傳徒聚眾,事發,大吏檄尚絅按之。戴奉飛實罪首,詳誥為從,當減死。承審同官以詳誥巨富,欲引嫌。尚絅曰:「無媿於中,何嫌可避?」大吏亦慮與原奏不符,尚絅曰:「不護前非,乃見至公。聖明在上,何慮焉?」卒從其議,株連者亦多省釋。嘗言:「獄不難於無枉縱,惟干證之牽累,吏胥之需求,受害者不可窮詰。生平思此,時用疚心。」又曰:「人知命、盜巨案之當慎,不知婚姻、財產細務,尤不可忽。蓋必原情度勢,使可相安於異日,不釀成別故,斯為善耳。」

南康治濱湖,風濤險惡,宋郡守孫喬年築石堤百餘丈,內浚二澳,可泊千艘。硃子知南康,增築之,名紫陽堤。迤東水齧,浸及城址,明知府田琯增築石堤百餘丈以衛之,久俱圮。尚絅增修兩堤,一準舊制,堅固經久。蓼花池周五十里,受廬山九十九灣之水,北入湖,水門淺隘,尚絅疏濬之,積潦消洩,歲增收穀萬石。在任先後二十四年,所設施多規久遠。歷署饒州、吉安、廣信三府,攝糧道。敝衣蔬食,不問生產。引疾去官,不能歸,卒於南康。

張敦仁,字古愚,山西陽城人。乾隆四十年進士,授江西高安知縣,調廬陵。精於吏事,有循聲。遷銅鼓營同知,署九江、撫州、南安、饒州諸府事。嘉慶初,改官江蘇,歷松江、蘇州、江寧知府。六年,調授江西吉安。沿贛江多盜,遴健吏專司巡緝,責盜族擒首惡,毋匿逋逃,萑苻以靖,民德之。再署南昌,尋實授。所屬武寧民婦與二人私,殺其夫,前守以夫死途中,非由婦奸報。敦仁覆鞫詞無異,而其幼子但哭不言,疑之。請留前守同讞,遂得謀殺移尸狀,獄乃定。龍泉天地會匪滋事,巡撫檄敦仁往按,未至,鎮道已發兵擒二百餘人,民惶懼。敦仁廉知匪黨與溫氏子有隙,非叛逆,法當末減,坐為首二人。又會匪素肆掠,富室為保家計,多佯附,實未身與。事發株連,囹圄為滿。訊察其冤,盡得釋。道光二年,擢雲南鹽法道,尋以病乞致仕。敦仁博學,精考訂,公暇即事著述,所刻書多稱善本。寄寓江寧,卒,年八十有二。著書遭亂多佚。

鄭敦允,字芝泉,湖南長沙人。嘉慶十九年進士,選庶吉士,散館授刑部主事,遷員外郎。道光八年,出為湖北襄陽知府。襄陽俗樸,訟事多出教唆。敦允長於聽斷,積牘為空。訪所屬衙蠹莠民最為民患苦者十餘人,論如律。地號盜藪,請帑籌充緝捕費,多設方略,獲盜百餘。巨盜梅杈者,勇悍多徒黨,捕者人少莫能近,眾至則逸。偵知所在,夜往擒之,其徒追者數百人。令曰:「欲奪犯者,殺而以尸與之。」眾不敢逼。訴者麕集,曰:「久不敢言,言輒火其居。」敦允曰:「苦吾民矣!」遂置之法。棗陽地瘠民貧,客商以重利稱貸,田產折入客籍者多。敦允許貸戶自陳,子浮於母則除之,積困頓蘇。

漢水齧樊城,壞民居,議甃石堤四百餘丈,二年而成。明年,漢水大漲,樊城賴以全。

襄陽岸高水下,遇旱,艱於引溉。頒筒車式,使民仿制,民便之。調署武昌,會大水,樊城石工掣損,敦允固請回任守修。襄人走迎三百里,日夜牽挽而至,議增築子墊護堤根。災民就食者數萬,為草舍居老疾稚弱,令壯者赴工自食。敦允昕夕巡視,工未竟,致疾,未幾卒,祀名宦。

李文耕,字心田,雲南昆陽人。家貧,事親孝,服膺宋儒之學。嘉慶七年進士,以知縣發山東,假歸養母。母喪,服闋,補鄒平。到官四閱月,不得行其志,引疾去。以官累,不得歸。十九年,教匪起,壽張令以文耕嫺武事,招助城守,訓練、防御皆有法,賊不敢窺境。大吏聞其幹略,起復補原官。

在鄒平五年,治尚教化。民婦陳訴其子忤逆,文耕引咎自責,其子叩頭流血,母感動請釋,卒改行。聽訟無株累,久之,訟者日稀。善捕盜,養捕役,使足自贍,無豢賊,數親巡,窮詰窩頓。嘗曰:「治盜必真心衛民,身雖不能及者,精神及之,聲名及之。」終任,盜風屏息。課諸生,親為指授,勉以為己之學,民呼李教官,又呼為李青天。調冠縣,遷膠州,濬云、墨二河。道光二年,擢濟寧直隸州,未之任。巡撫琦善特薦之,宣宗夙知其名,即擢泰安知府。

調沂州,立屬吏程課,謂:「官不勤則事廢,民受其害。教化本於身,能對百姓,後然可以教百姓。」屬吏皆化之。沂郡產檞樹,勸民興蠶,建義倉備荒,捕盜如為令時。尋擢兗沂曹道。司河事,修防必躬親。屬請濬淤沙,需銀五萬,往視之,曰:「無庸!春漲,即刷去矣。」果如其言。

五年,遷浙江鹽運使,未幾,調山東。時鹺業疲累,充商者多無藉游民。文耕知其弊,請分別徵緩,以紓商力。責富商領運,不得因引滯賤價私賣,課漸裕。七年,擢湖北按察使,復調山東。嚴治胥役,詐贓犯輒置重典。斷獄寬平,責屬吏清滯獄,數月,積牘一空。謂:「山東民氣粗而性直,易犯法,亦易為善,故教化不可不先。」

居三歲,調貴州。州縣瘠苦,希更調,不事事。適權布政使,請以殿最為調劑,俾久任專責成。鑿桐梓葫蘆口,以息水患。黔產紬,無綿布,設局教之紡織。貧民艱生計,重利而薄倫常,撰文勸導,曰家喻戶曉篇。十三年,休致歸。

文耕平生以崇正學、挽澆風為己任,在山東久,民感之尤深,歿祀名宦。

劉體重,山西趙城人。乾隆五十四年舉人。嘉慶初,以知縣發湖南,歷署石門、新化、衡陽、寧武、衡山、湘陰。晉秩同知,改江西。道光中,補袁州同知,擢廣信知府。調吉安,又調撫州,所至有聲。在撫州治績最著,巡歷屬縣,問民疾苦,集父老子弟勉以孝弟力田。屬吏不職,參劾無徇。胥吏攬訟,痛懲之。厚書院廩餼,課士以經,動繩以禮法。遇大水,盡心賑恤,災不為害。建義倉,積穀五萬石。十四年,擢河南彰衛懷道,筦河事,修防有法。終任,黃流安瀾。沁水堤由民築,多單薄,擇其要區加築子墊,籌歲修費垂永久。漳河無堤防,勤疏濬,水患並息。創建河朔書院,仿硃子白鹿洞規條,以課三郡之士。十九年,擢江西按察使,遷湖北布政使。二十二年,乞病歸,卒於家。

體重廉平不苛,尤長治獄。所居,吏畏民懷,訟獄日簡。河北士民尤感之,歿祀名宦祠。

子煦,由拔貢授直隸知縣,歷權繁劇。咸豐初,遷開州知州。河決,賑災,全活數萬。治團練有功,署大名知府。十一年春,直隸、山東匪迭起,守城四十日,乘間出奇擊賊,城獲安。既而東匪西竄,勢甚張,畿輔震動。煦督師破清豐賊壘,乘勝進攻濮州老巢。遇大雨,賊決河自衛,煦激勵兵團,堅持不懈,賊窮蹙乞降,遂復濮州。開、濮之間,積水多沮洳,土人謂之水套,匪輒憑匿。至冬,復豎旗起事。煦率鄉團八千人,追賊於冰天泥淖之中,三戰皆捷,水套底定。同治元年,擢大順廣道,命偕副都統摭克敦布辦理直、東交界防剿事宜,以勞卒於官。優詔賜恤,大名及原籍並建專祠。

張琦,初名翊,字翰風,江蘇陽湖人。嘉慶十八年舉人,以謄錄議敘知縣。道光三年,發山東,署鄒平縣。抵任,歲且盡。閱四百七十村,麥無種者。即申牒報災,親謁上官陳狀。破成例請緩徵,因鄒平得緩者十六州縣。民失物,誤訟鄰邑長山,歸獄於琦。琦曰:「汝失物地,大樹北抑大樹南?」曰:「樹北。」琦曰:「若是,則我界也。」民愕然,曰:「誠鄒平耶?即不欲以數匹布煩父母官。」持牒去。後權章丘,鄒平民時赴訴,琦曰:「此於法不當受。」慰遣之。章丘民好訟,院、司、道、府五府吏皆籍章丘,走書請託,掎摭短長。琦任歲餘,無一私書至。結案二千有奇,無翻控者。

五年,補館陶,會久旱風霾,麥苗皆死,饑民聚掠。琦禱雨既應,嚴捕倡掠者。廉得富家閉糶居奇狀,按治之,民大悅。乃請普賑兩月。館陶地褊小,賑數多鄰邑數倍,大吏呵之。尋有詔責問歲饑狀甚切,乃按臨災區,民迎訴賑弊,惟館陶得實。始劾罷他邑令,厚慰琦。士有訟者,閱其辭不直,則曰:「課汝文不至,訟乃至耶?」先試以文,不中程,責後乃決事,士訟遂稀。館陶地斥鹵,不宜穀,又衛水數敗田。琦精求古溝防及區田法試行之,未竟,病卒。

在館陶八年,民愛戴之,理訟不待兩造集,即決遣之。以其辭質後至者,莫敢狡飾。有疑獄,亦不過再訊。胥吏擾民,必嚴論如法。然籌其生計必周,故無怨者。

琦少工文學,與兄編修惠言齊名,輿地、醫學、詩詞皆深造。五十後始為吏,治績尤著。時江西同知石家紹亦儒者,為治有古風,殆相亞云。

家紹,字瑤辰,山西翼城人。以拔貢為壺關縣教諭。道光二年成進士,授江西龍門知縣。發奸摘伏,以神明稱。調上饒,再調南昌。首邑繁劇,而盡心民事,理訟嘗至夜不輟。連年水患,饑民聞省會散賑,麕聚郭外。家紹與新建令同主賑,始散米,令饑民自爨。來者益眾,賑所瀕河,幾莫能容。乃改散錢,令各返鄉里,候截留漕米濟之。時水災益棘,家紹請開倉平糶,復分廠煮粥以賑。主者循例備三千人食,而就食者五萬,洶洶不可止。家紹至,諭之曰:「食少人眾,咄嗟不能辦。汝等姑退,詰朝來,不使一饑民無粥敢也。」眾皆迎拜曰:「石爹爹不欺人,原聽處置。」爹爹者,江西民呼父也。歷署大庾、新城、新建三縣,擢銅鼓營同知,署饒州、贛州二府,所至皆得民心。

家紹口吶吶若不得辭,自大吏、僚友、縉紳、士民、卒隸無不稱為循吏,顧自視欿然。嘗曰:「吏而良,民父母也;不良,則民賊也。父母,吾不能;民賊也,則吾不敢,吾其為民傭乎!」十九年,卒。五縣皆祀名宦,南昌民尤德之,建祠於百花洲。

劉衡,字廉舫,江西南豐人。嘉慶五年副榜貢生,充官學教習。十八年,以知縣發廣東。奉檄巡河,日夜坐臥舟中,與兵役同勞苦,俾不得通盜,河盜斂戢。署四會縣,地瘠盜熾。衡團練壯丁,連村自保。詗捕會匪,焚其籍,以安反側。祗治渠魁,眾乃定。調署博羅,城中故設徵糧店數家,鄉又設十站,民以為累,衡至即除之。俗多自戕,里豪蠹役雜持之,害滋甚。衡釋誣濫,嚴懲主使,錮習一清。補新興,父憂去。服闋,道光三年,授四川墊江,俗輕生亦如博羅,衡先事勸諭,民化之。獲啯匪初犯者,曰:「饑寒迫爾。」給貲使自謀生,再犯不宥,匪輒感泣改行。

調署梁山,處萬山中,去水道遠,歲苦旱。衡相地修塘堰,以時蓄洩,為永久之計。捐田建屋,養孤貧,歲得穀數百石,上官下其法通省仿行。尋調巴縣,為重慶府附郭,號難治。白役七千餘人,倚食衙前。衡至,役皆無所得食,散為民,存百餘人,備使令而已。歲歉,衡謂濟荒之法,聚不如散,命各歸各保,以便賑恤,是年雖饑不害。

衡嘗謂律意忠厚,本之為治,求達愛民之心。然愛民必先去其病民者,故心互寓寬於嚴。官民之阻隔,皆緣丁胥表裏為奸。所至設長幾於堂左右,分六曹為六鬲。吏呈案,則各就左幾鬲庋之,擊磬以聞。衡自取,立與核辦,置之右幾。吏以次承領,壅蔽悉除。有訴訟,坐堂皇受牘,親書牒令原告交里正,轉攝所訟之人,到即訊結。非重獄,不遣隸勾攝;即遣,必注隸之姓名齒貌於簽。又令互相保結,設連坐法,蠹役無所施技。性素嚴,臨訟輒霽顏,俾得通其情,抶不過十,惟於豪猾則痛懲不稍貸。嘗訪延士紳,周知地方利害,次第舉革。待丞、尉、營弁必和衷,時周其乏,緩急可相倚。城鄉立義學,公餘親課之。為治大要,以恤貧保富、正人心、端士習為主。總督戴三錫巡川東,其旁邑民訴冤者皆乞付劉青天決之,語上聞。

七年,擢綿州直隸州知州,宣宗召對,嘉其公勤。八年,擢保寧知府,九年,調成都。每語人曰:「牧令親民,隨事可盡吾心。太守漸遠民,安靜率屬而已,不如州縣之得一意民事也。」然所在屬吏化之,無厲民者。後擢河南開歸陳許道,未幾,病。巡撫為陳情及治蜀狀,請優待之,以風有位。特詔給假調理。久之,病不愈,遂乞歸。數年始卒。博羅、墊江、梁山、巴縣皆請祀名宦祠。

同治初,四川學政楊秉璋疏陳衡循績,並上遺書。穆宗諭曰:「劉衡歷任廣東、四川守令,所至循聲卓著。去官四十餘年,至今民間稱道弗衰。所著庸吏、庸言、蜀僚問答、讀律心得等書,尤為洞悉閭閻休戚,於興利除弊之道,籌畫詳備,洵無媿循良之吏。將歷任政績宣付史館,編入循吏傳,以資觀感。」衡所著書,皆閱歷有得之言,當世論治者,與汪輝祖學治臆說諸書同奉為圭臬。其後有徐棟著牧令諸書,亦並稱焉。

棟,字致初,直隸安肅人。道光二年進士,授工部主事,累遷郎中。究心吏治,以為天下事莫不起於州縣,州縣理,則天下無不理。稱州縣之職,不外於更事久,讀書多。然更事在既事之後,讀書在未事之先,乃匯諸家之說為牧令書三十卷。又以保甲為庶政之綱,天下非一人所能理,於是有鄉、有保、有甲。自明王守仁立十家牌之法,後世踵行,為弭盜設,此未知其本也。亦集諸說,成保甲書四卷。二十一年,出為陜西興安知府,調漢中,又調西安,所至行保甲,皆有成效。興安臨漢江,棟補修惠春、石泉兩堤,加於舊五尺,民頗苦其役。十數年後,大水冒舊堤二尺,乃感念之,肖像以祀。舊禁運糧下游,棟以興安卑濕,積穀易霉變。既不能久儲,又不能出境,圖利者改種菸葉、藍靛,歉年每至乏食。乃弛運糧之禁,民便之。舉卓異,二十九年,以病歸。咸、同之間,在籍治團練,修省城,有詔錄用,以老病辭,尋卒。祀興安名宦祠。

姚柬之,字伯山,安徽桐城人。七世祖文燮,見本傳。柬之少負異才,從族祖鼐學,道光二年成進士,授河南臨漳知縣,屢決疑獄。縣民張鳴武控賊殺妻,稱賊攀二窗櫺入室。柬之勘窗櫺窄,且夫未遠出。詰之,果夫因逐賊,誤斫殺妻。又常姚氏被殺,罪人不得。柬之察其時為縣試招覆之前夜,所取第一名楊某不赴試,疑之。召至,神色惶惑,詢其居,與常鄰。乃夜至城隍廟,命婦人以血污面,與楊語,遂得圖奸不從強殺狀。每巡行鄉曲,勸民息訟,有訴曲直者即平之。漳水溢,齎糧赴災區,且勘且賑,全活者眾。兼攝內黃,民服其治,鬧漕之風頓革。境與直隸大名毗連,多賊巢,掘地為窟,積匪聚賭,排槍手為拒捕計。柬之約大名會捕,賭窟除而盜風息。母憂去。

十二年,服闋,補廣東揭陽。瀕海民悍,械斗擄掠,抗賦戕官,習以為常。柬之訓練壯勇,集神耆於西郊,諭以保護善良,與民更化。最頑梗之區曰下灘,盜賊、土豪相勾結,柬之會營往捕,拒者或死或擒。一盜積犯十八案,召被害者環觀,僇之,境內稱快。有兇盜居錢坑,其地四面皆山,不可攻。潮州故事,凡捕匪不得,則爇其廬,空其積聚。柬之戒勿焚燒,召耆老,諭交犯,不敢出。乃乘輿張蓋入村,從僅數人,見耆老一一慰勞,皆感泣,原更始。民在四山高望者,咸呼「好官」,次日遂交犯。自下灘示威,錢坑示德,恩信大著。收穫時,巡鄉為之保護,樹催科旗;值械斗,則樹止鬥旗。一日,塗遇持火槍者,結隊行,望見官至,悉沒水中,命以漁網取之。訊為助鬥者,按以法,自此械斗浸止。興復書院,厚待諸生,回鄉以新政告鄉人,有變則密以聞,官民無隔閡。逋賦者相率輸將,強梗漸化,縣大治。

遷連州綏瑤同知,民、瑤構訟,判決時必使相安,遂無事。普寧縣匪徒戕官肆劫,奉檄從鎮道往捕治。匪以塗祥為巢穴,磨盤山為聲援,地皆險。乃設方略,正軍攻塗祥,調揭陽壯勇自磨盤嶺突進破賊巢,獲六百餘人。事定,言官誤論劾。朝使查勘,其誣得白。

十七年,署肇慶府,端溪大漲,城不沒數版,柬之日夜立城下守御。預放兵糧,以平米價,民不知災。十九年,擢貴州大定知府,俗好訟,柬之速訊速結,不能售其欺,期年而訟稀。白蟒洞地僻產煤、鐵,有汪擺片者,據其地聚眾結會,為一方害,捕滅解散,地連川、滇,得弭鉅患焉。大定民、苗雜居,宜治以安靜。大吏下令,柬之必酌地方之宜,不使累民。見多不合,遂引疾歸。數年始卒。

吳均,字雲帆,浙江錢塘人。嘉慶二十四年舉人,道光十五年,大挑知縣,發廣東,授乳源,調潮陽。歷署揭陽、惠來、嘉應、海陽。在海陽捕雙刀會匪黃悟空,置之法。舉卓異,署鹽運司運同,擢佛岡同知,署潮州知府。咸豐二年,惠州土匪肆劫,均奉檄往,獲匪千餘,分輕重懲治,遂肅清。三年,實授。時東南各行省軍事亟,福建、湖南大吏聞均名,先後奏調往襄剿匪,廣東方倚為保障,堅留之。四年,江南大營散兵回粵,結匪為亂。賊首陳娘康擁眾圍潮陽,分黨陷惠來,攻普寧。援軍失利,均親督戰,敗賊。甫解潮陽圍,海陽彩陽鄉匪首吳中庶乘間糾黨陳阿拾煽眾,旬日至萬餘人。大掠海陽,偪攻郡城,澄海匪首王興順亦與合。均檄潮陽令汪政分兵援郡城,戰城下,殲賊數千,圍解。自移軍澄海,冒雨破賊巢,分路搜捕,清餘孽。旋克惠來,斬陳娘康等於陣。未幾,以積勞卒於官。

均性清介,治潮最久,誅盜尤嚴。每巡鄉,輒以二旗開導,大書曰;「但原百姓回心,免試一番辣手。」化莠為良,保全彌眾。從役有取民間絲粟者,立斬馬前,民益畏服。在潮陽以濱海地咸鹵,開渠以通溪水,築堤六千餘丈,淡水溉田,瘠土悉沃。在海陽濬三利溪,加築北堤,為郡城保障。及守潮州,修復州東廣濟大橋。附郭西湖山高出城上,登瞰全城如指掌,舊有高墉為犄角,久圮。均築展新城,跨壕而過,圍山於城內。至是匪亂圍攻,竟不能破,民咸頌之。歿後,追贈太僕寺卿。光緒間,潮州建專祠。

王肇謙,字琴航,直隸深澤人。道光十四年舉人,授福建海澄知縣。馬口鄉民構釁互掠,親諭利害,積嫌頓解。捕巨盜許蟳置諸法,群盜斂跡。富紳爭產累訟,男婦數十人環跪堂下,援引古義喻之,更反自責。眾赧然,謂今日始知禮義,訟以是止。邑民李順發負楊茄柱金,為楊所留,乃以劫財訴諸教堂。教主移牒請嚴究,眾洶洶。肇謙白上官:「茄柱無罪,不必治;教士驕心,不可長。」總督劉韻珂嘉其抗直。閩縣上筸村故盜藪,檄肇謙往捕。至則召其父老開陳大義,曰:「我來活若一鄉,若列銃拒官,大府欲屠之,尚不知耶?」眾大恐,肇謙曰:「某某皆大盜,速縛來!三日繕齊保甲冊,吾保若無事。」遂立以盜獻。廈門洋人因賃屋與民齟,奉檄往治,據理剖決,兩無所徇,洋人帖服。

咸豐二年,署上杭,時粵匪據江寧,福建賊林俊遙應之,陷漳州、永春、大田諸郡縣。肇謙建碉儲粟,制器械,簡丁壯,為堅壁清野計,賴以無虞。三年,淫雨為災,且賑且治軍,率團勇越境剿松源縣賊四千。擢永春直隸州知州,募鄉兵二萬,破林俊於城南山,擒土匪邱師、辜八等。

署漳州知府,漳浦古竹社蔡全等為亂,肇謙設方略,約內應,生擒全,詔嘉之,晉秩知府。漳俗獷悍難治,肇謙謂民不奉法,由吏不稱職。課所屬清案牘,勤催科,懲械斗,嚴緝捕,表義行,振文教,以能否為殿最,漳人以為保障。署延建邵道,調署興泉永道,未行,粵匪竄入境,肇謙誓以死守,督軍隨按察使趙印川十三戰皆捷,以勞卒。詔贈光祿寺卿,祀上杭名宦祠。

曹瑾,字懷樸,河南河內人。嘉慶十二年舉人。初官直隸知縣,歷署平山、饒陽、寧津,皆得民心。賑饑懲盜,多惠政。補威縣,調豐潤,以事落職。尋復官,發福建,署將樂。又以失察邪教被劾,引見,仍以原官用。

道光十三年,署閩縣,旗兵與民械斗,持平曉諭利害,皆帖服。值旱,迎胡神於鼓山禱雨,官吏奔走跪拜街衢間,瑾斥其不載祀典,獨屹立不拜。大吏奇之,以為可任艱鉅。時臺灣歲歉多盜,遂補鳳山。問疾苦,詰盜賊,剔除弊蠹,順民之欲。淡水溪在縣東南,由九曲塘穿池以引溪水,築埤導圳。凡掘圳四萬餘丈,灌田三萬畝,定啟閉蓄洩之法,設圳長經理之。

二十年,擢淡水同知,海盜剽劫商賈,漳、泉二郡人居其間,常相仇殺,又當海防告警,瑾至,行保甲,練鄉勇,清內匪而備外侮。英吉利兵艦犯雞籠口,瑾禁漁船勿出,絕其鄉導,懸賞購敵酋,民爭赴之。敵船觸石,擒百二十四人。屢至,屢卻之。明年,又犯淡水南口,設伏誘擊,俘漢奸五、敵兵四十九人。事聞,被優賚。未幾,和議成,英人有責言。總督怡良知瑾剛直,謂曰:「事將若何?」瑾曰:「但論國家事若何,某官無足重,罪所應任者,甘心當之。但百姓出死力殺賊,不宜有負。」怡良嘆曰:「真丈夫也!」卒以是奪級。後以捕盜功晉秩,以海疆知府用。瑾遂乞病歸,數年始卒。

桂超萬,字丹盟,安徽貴池人。道光十二年進士,以知縣發江蘇。署陽湖四十日,巡撫林則徐賢之,捕荊溪。未任,父憂去。十六年,服闋,授直隸欒城。捕盜不分畛域,每於鄰邑交界處破賊巢,盜風息。濬洨河、金水河及城河,通溝洫,平道路,水潦無患。限紳戶免役不得過三十畝,免累民。勸樹畜,修井糞田,種薯芋以備荒。復書院,設義塾,化導鄉民,習異教者多改行。調萬全,署豐潤。值英吉利犯天津,沿海戒嚴。超萬訓練鄉勇,募打鴨善槍法者以備戰。後粵匪犯畿輔,天津練勇效超萬法,頗收鴨槍狙擊之效。詔舉賢吏,總督訥爾經額薦超萬持躬廉謹,盡心民事,遷北運河務關同知。

二十三年,擢授江蘇揚州知府。揚俗浮靡,超萬勵勤儉,嚴禁令,凡衙蠹、營兵、地棍、訟師諸害民者,悉繩以法。訟於府者,一訊即結。逾兩年,調蘇州。時漕弊積重,大戶短欠,且得規包納運丁,需索日增,官民交困。超萬為減幫費、均賦戶之議。乃訪懲豪猾,示均收章程,依限完納,即赦既往。請大吏奏定通行,積困稍甦。屯佃求減租,聚眾毆業主,糧艘水手因行海運失業,勾結滋事,勢皆洶洶。超萬處以鎮靜,先事戒備,得弭亂萌。署糧儲道。二十九年,擢福建汀龍漳道。乞病歸。咸豐中,粵匪擾安徽,超萬在籍治鄉團。同治初,福建巡撫徐宋幹薦之,署福建糧儲道,尋擢按察使。年八十,卒於官。

張作楠,字丹村,浙江金華人。嘉慶十三年進士,銓授處州府教授。擢江蘇桃源知縣,調陽湖。治事廉平,人稱儒吏。道光元年,擢太倉直隸州知州,三年,大水、作楠冒雨履勘災鄉,問民疾苦,停徵請賑,借帑平糶。疏濬境內河道,以工代賑。水得速洩,涸出田畝,不誤春耕,人刊婁東荒政編紀其事。尋奉檄赴松江讞獄,鄉民訛傳去官,慮仍收漕,紛紛奔訴。會瀕海奸徒乘間蠢動,作楠聞變,馳回,中途檄主簿蕭赴茜涇捕首惡,脅從罔治,事遂定。作楠勤於治事,案無滯牘。暇則篝燈課讀,妻、女紡織,常至夜分。人笑其為校官久,未改故態。

五年,擢徐州知府,受代,以平糶虧帑二萬金,彌補未完。作楠自危,巡撫陶澍曰:「救災民如哺兒,失乳即死。吾方咎汝請糶時,顧慮折耗不兌稍稽。遺大投艱者,胡亦泥此?且紳民已代致萬金,不汝責也!」徐州亦被災,籌賑甚力,民賴以甦。

在任兩載,乞養歸。鄉居二十餘年,足跡不入城市。三子皆令務農、工,或問:「何不仍業儒?」曰:「世俗讀書為科名,及入仕,則心術壞,吾不欲其墮落也。」作楠精算學,貫通中西。在官以工匠自隨,制儀器,刊算書。所著書,匯刻曰翠微山房叢書,行於世,學者奉為圭臬焉。卒,祀鄉賢祠。

雲茂琦,廣東文昌人。道光六年進士,授江蘇沛縣知縣。詢民疾苦,懇懇如家人。勸以務本分、忍忿爭,訟頓稀。縣地卑,多積潦,開濬溝洫,歲獲屢豐。籌緝捕經費,獲盜多,給重賞,盜賊屏跡。課諸生,先德行,後文藝,語以身心性命之學。鄰邑聞風而來,書院齋舍至不能容。總督蔣攸銛稱其有儒者氣象。調六合,連年大水,災民得賑,無流亡。邑多淫祀,毀其像,改書院。衛田多典質,為清理復業,運戶得所津貼,漕累以紓。考最,入覲,改官兵部郎中,又改吏部。未幾,告養歸。家居十數年,置田贍族,鄉邑興革,無不盡力。主講課士有法。


\end{pinyinscope}