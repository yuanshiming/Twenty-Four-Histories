\article{列傳六十七}

\begin{pinyinscope}
郎坦朋春薩布素瑪拉

郎坦,瓜爾佳氏,滿洲正白旗人,內大臣吳拜子。年十四,授三等侍衛。順治六年,進二等。從端重親王博洛討叛將姜瓖,次渾源,圍城。賊渡濠來犯,郎坦射其酋,貫心,殪,遂敗賊。師還,進一等。八年,以吳拜附和內大臣洛什等獲罪,並奪郎坦官。尋復之。康熙二年,代吳拜管佐領,遷護軍參領。從定西將軍圖海討李自成餘黨李來亨等於茅麓山,深入賊巢,獲所置官十一。四年,襲一等精奇尼哈番。十二年,京師有陳三道者,設壇以邪教惑眾,命郎坦與諸侍衛捕治。十三年,命行邊,獲逋盜張飛腿等。擢正白旗蒙古副都統,調本旗滿洲。

順治中,俄羅斯東部人犯黑龍江邊境,時稱為羅剎。九年,駐防寧古塔章京海塞遣捕牲翼長希福率兵與戰,師敗績。世祖命誅海塞,鞭希福百,仍駐寧古塔。十一年,固山額真明安達里率師討之,敗敵黑龍江。羅剎未大創,復侵入精奇里江諸處。上命大理寺卿明愛等諭令撤回,遷延不即去,據雅克薩城,於其旁耕種漁獵;又過牛滿、恆滾,侵擾索倫、赫哲、飛牙喀、奇勒爾諸部。

二十一年秋,遣郎坦及副都統朋春等率兵往索倫。比行,諭曰:「羅剎犯我境,恃雅克薩城為巢穴,歷年已久,殺掠不已。爾等至達呼爾、索倫,遣人往諭以來捕鹿。因詳視陸路遠近,沿黑龍江行圍,逕薄雅克薩城,勘其形勢。度羅剎不敢出戰,如出戰,姑勿交鋒,但率剎引退。朕別有區畫。」賜御用裘服、弓矢以行。及冬,郎坦等還京師,疏言:「羅剎久踞雅克薩,恃有木城。若發兵三千,與紅衣砲二十,即可攻取。陸行自興安嶺以往,林木叢雜,冬雪堅冰,夏雨泥淖,惟輕裝可行。自雅克薩還至愛滹城,於黑龍江順流行船,僅須半月,逆流行船,約須三月,倍於陸行,期於運糧餉、軍器、輜重為便。現有大船四十、小船二十六,宜增造小船五十餘應用。」上諭曰:「郎坦等奏攻取羅剎甚易,朕亦以為然。第兵非善事,宜暫停攻取。調烏拉、寧古塔兵千五百人,並制造船艦,發紅衣砲、鳥槍教之演習。於愛琿、呼瑪爾二地建木城,與之對壘,相機舉行。所需軍糧,取諸科爾沁十旗及錫伯、烏拉官屯,約得一萬二千石,可支三年。愛琿城距索倫五宿可至,其間設一驛。俟我兵將至精奇里烏拉,令索倫供牛羊。如此,則羅剎不得納我逋逃,而彼之逋逃且絡繹來歸,自不能久存矣。」尋擢郎坦前鋒統領。

二十二年,命與黑龍江將軍薩布素會議,駐兵額蘇哩。事還,奏額蘇哩七月即經霜雪,宜乘春和,以寧古塔兵分為三班,更番戍守。上以更番戍守非久長策,不允。二十三年,甄別八旗管兵官,罷郎坦前鋒統領,以世職隨旗行走。二十四年,命都統朋春率師征羅剎,郎坦以副都統銜隨征。師薄雅克薩城,羅剎酋額里克舍請降,郎坦宣詔宥其罪,引眾徙去,毀木城。是冬羅剎復來,踞雅克薩築城。二十五年,命郎坦偕副都統班達爾沙攜紅衣砲,率藤牌兵百人,往會將軍薩布素進兵。上以郎坦諳悉地勢,即令參贊軍務。六月,薄其城,鑿壕築壘,賊出拒,擊敗之,斬額里克舍。尋,俄羅斯察罕汗上書請釋雅克薩圍,上許之,令郎坦撤軍,還駐寧古塔。尋擢正白旗蒙古都統。二十八年,上遣內大臣索額圖等與俄羅斯使人費耀多囉等會於尼布楚,立約定界,命郎坦與議,乃毀所築城徙去。

二十九年,古北口外盜起,命郎坦偕侍衛赫濟爾亨等督兵捕剿,盡殲之。三十一年,噶爾丹侵喀爾喀部,擾及邊境,授郎坦安北將軍,率師駐大同。疏請出邊駐喀喇穆倫偵寇,詔暫駐歸化城。尋擢領侍衛內大臣,兼火器營總管,列議政大臣。三十二年,授昭武將軍,率師駐甘州。三十三年,移駐寧夏,與甘肅提督孫思克分道偵寇。上聞噶爾丹將逼圖拉,命郎坦移兵禦剿,以圖拉無警,引還。仍任領侍衛內大臣,列議政如故。三十四年,往盛京巡閱邊隘,還入塞,疾劇,遣太醫馳驛往視。尋卒,賜祭葬。

朋春,棟鄂氏,滿洲正紅旗人,何和禮四世孫。何和禮子和碩圖,進爵三等公;子何爾本、哲爾本、蘇布遞襲,至袞布,以恩詔進一等。朋春,哲爾本子也,順治九年,襲封。康熙十五年,加太子太保,授正紅旗蒙古副都統,調本旗滿洲。

二十一年,偕郎坦率兵至黑龍江覘羅剎形勢,賜御用裘服、弓矢。與郎坦還奏,上命寧古塔將軍巴海、副都統薩布素,建木城於黑龍江、呼瑪爾,調取所部兵一千五百人往駐焉。又命尚書伊桑阿赴寧古塔督造戰船。尋擢朋春正紅旗滿洲副都統。二十四年,詔選八旗及安置山東、河南、山西三省福建投誠藤牌兵,付左都督何祐率赴盛京,命朋春統之,進剿羅剎,以副都統班達爾沙、副都統銜瑪拉、鑾儀使建義侯林興珠、護軍統領佟寶參贊軍務,祐、興珠皆鄭氏將來降者也。師既行,上遣侍衛關保至黑龍江傳諭曰:「兵兇戰危,朕以仁治天下,素不嗜殺。以我兵馬精強,器械堅利,羅剎勢不能敵,必獻地歸誠。爾時勿殺一人,俾還故土,宣朕柔遠至意。」五月,師薄雅克薩城,遣人諭降,不從。分水陸兵為兩路,列營夾攻,復移紅衣砲於前,積薪城下,示將焚焉。羅剎頭目額里克舍詣軍前乞降,乃宥其罪,釋還俘虜,額里克舍引六百餘人徙去,毀木城,以歸附巴什裡等四十五戶及被掠索倫、達呼爾百餘戶安插內地。

二十九年,厄魯特與喀爾喀構釁,命裕親王福全為撫遠大將軍,出邊剿噶爾丹,以朋春與都統蘇努參贊軍務。蘇努率左翼,朋春率右翼,至烏闌布通。噶爾丹依山列陣,朋春所部為泥淖所阻,蘇努督兵沖擊,大破之。噶爾丹偽乞和,夜自大磧山遁走。部議朋春坐奪官,上命寬之,降級留任。三十一年,命解職赴西路軍前管隊。三十五年,復授正紅旗蒙古都統。旋以費揚古為撫遠大將軍,朋春仍參贊軍務,出西路,破噶爾丹於昭莫多。師還,以本隊護軍驍騎十八人戰死未收其骸,下部議。以師有功,免罪,仍錄戰績,增注敕書。三十八年,因病解職。尋卒。子增壽,改襲三等公。

薩布素,富察氏,滿洲鑲黃旗人。四世祖充順巴本,以勇力聞,世為嶽克通鄂城長。太祖時,其後人哈木都率所部來歸,屯吉林,遂家焉。薩布素自領催授驍騎校,遷協領。康熙十六年,聖祖遣內大臣覺羅武默訥等瞻禮長白山,至吉林,欲得識路者導引。寧古塔將軍巴海令薩布素率兵二百,攜三月糧以從。水陸行,至長白山麓,成禮而還,事具武默訥傳。

十七年,授薩布素寧古塔副都統。羅剎據雅克薩,二十一年,詔率兵偕郎坦等勘視雅克薩城形勢,並往視自額蘇哩至黑龍江及通寧古塔水陸道。尋郎坦還奏羅剎可圖狀,命建木城於黑龍江、呼瑪爾兩地,以巴海與薩布素統寧古塔兵千五百人往駐,造船備砲。二十二年,疏言:「黑龍江、呼瑪爾距雅克薩尚遠,若駐兵兩處,則勢分道阻,且過雅克薩有尼布楚等城。羅剎倘水陸運糧,增兵救援,更難為計。宜乘其積貯未備,速行征剿。俟造船畢,度七月初旬能抵雅克薩,即統兵直薄城下。」疏下王大臣議,如所請,上不許。尋命巴海留守吉林,以薩布素偕寧古塔副都統瓦禮祜率兵駐額蘇哩。額蘇哩在黑龍江、呼瑪爾之間,為進攻雅克薩要地,有田隴舊跡。薩布素因移達呼爾防兵五百人赴其地耕種,並請調寧古塔兵三千更番戍守。上念兵丁更戍勞苦,命在黑龍江建城,備攻具,設斥堠,計程置驛,運糧積貯,設將軍、副都統領之。擢薩布素為黑龍江將軍,招撫羅剎降人,授以官職,更令轉相招撫。

上命都統瓦山、侍郎果丕與薩布素議師期,薩布素請以來年四月水陸並進,攻雅克薩城,不克,則刈其田禾。上謂攻羅剎當期必克,倘謀事草率,將益肆猖狂。二十四年,以朋春等統兵進攻,薩布素會師,克雅克薩城,乃命薩布素移駐墨爾根,建城防禦。二十五年,疏言羅剎復踞雅克薩,請督修戰艦,俟冰泮進剿。上遣郎中滿丕往詗得實,乃命薩布素暫停墨爾根兵丁遷移家口,速修戰監,率寧古塔兵二千人往攻。又命郎坦、班達爾沙會師,抵雅克薩城。城西瀕江,薩布素令於城三面掘壕築壘為長圍,對江駐水師,未冰時泊舟東西岸,截尼布楚援兵,冰時藏舟上流汊港內;馬有疲羸者,分發墨爾根、黑龍江飼秣,計持久。上因荷蘭貢使以書諭俄羅斯察罕汗,答書請遣使畫界,先釋雅克薩圍,上允之,命撤圍。二十八年,俄羅斯使臣費耀多囉等至尼布楚,命內大臣索額圖等往會,令發黑龍江兵千五百人為衛。尋議以大興安嶺及格爾必齊河為界,毀雅克薩城,徙其人去。二十九年,薩布素入覲,賜賚優渥,命坐內大臣班。尋命總管索倫等部貢物,疏陳各部生計土俗採捕之事,擬為則例以上,上悉允行。

三十一年,奏建齊齊哈爾及白都訥城,以科爾沁部獻進錫伯、卦爾察、達呼爾壯丁萬四千有奇分駐二城,編佐領,隸上三旗,並設防守尉、防禦等官。噶爾丹入犯,疏陳進兵事宜,略言:「興安嶺北形勝地,以索約爾濟山為最。已遣識路官兵自盛京、吉林、墨爾根審度至山遠近,分置驛站,其無水處,掘井以待。山之東北呼倫貝爾等處有警,與臣駐軍地近,即率墨爾根兵先進,吉林、盛京繼之;山之西烏勒輝等處有警,則盛京兵先進,臣率部下及吉林兵繼之:皆會於索約爾濟山。」上可其奏。三十五年,上親征噶爾丹,自獨石口出中路,大將軍費揚古自歸化城出西路,命薩布素扼其東路,督盛京、寧古塔、科爾沁兵,自索約爾濟山剋期進剿。四月,上次克魯倫河,噶爾丹西竄,為費揚古所敗。詔分薩布素所部兵五百人隸費揚古軍。三十六年,召至京師,尋命回任。

初,邊境有墨爾哲勒屯長,累世輸貢。康熙初,屯長扎努喀布克托請率眾內移,寧古塔將軍巴海安輯於墨爾根,編四十佐領,號新滿洲。薩布素奏於墨爾根兩翼立學,設助教,選新滿洲及錫伯、索倫、達呼爾每佐領下幼童一,教習書義。是為黑龍江建學之始。三十七年,上幸吉林,褒其勤勞,予一等阿達哈哈番世職,並御用冠服,於眾前宣諭賜之。尋疏言黑龍江屯堡因災荒積欠米石,請俟年豐交倉。上以薩布素曾奏革任總督蔡毓榮經理十二堡,著有成效;嗣因官堡荒棄,請停止屯種,將壯丁改歸驛站,存貯倉米,支放無餘,致駐防兵餉匱乏,責令回奏。薩布素具疏引罪,請以齊齊哈爾、墨爾根駐防兵每年輪派五百人往錫伯等處耕種官田,穫穀運齊齊哈爾交倉。詔侍郎滿丕等往按,以薩布素將荒廢地妄報成效,並浮支穀石,應斬,命罷任,奪世職,在佐領上行走。尋授散秩大臣。

三十九年,卒。乾隆間,敕修盛京通志,列名宦,且稱薩布素諳練明敏,得軍民心,其平羅剎及黑龍江興學,有文武幹濟才云。

瑪拉,那喇氏,滿洲鑲白旗人,尚書尼堪從子。尼堪卒,無子,瑪拉與叔阿穆爾圖、阿錫圖及弟兆資分襲尼堪世職,瑪拉襲三等阿達哈哈番。初任理籓院筆帖式。順治五年,英親王阿濟格征叛將姜瓖,圍大同,令瑪拉調蒙古兵以從。累遷理籓院副理事官。康熙十四年,察哈爾布爾尼叛,聖祖命信郡王鄂扎帥師討之。瑪拉自陳久任理籓院習知蒙古狀,原赴軍前效力,遂命與員外郎色棱赴科爾沁諸部調選兵馬協剿。師還,擢通政使,遷禮部侍郎。十六年,擢工部尚書。偕內大臣喀岱往科爾沁諸外籓宣諭禁令。瑪拉初受任,上誡以工部積弊,宜殫心釐剔。十九年,坐不能清積弊,議降五秩,詔從寬留任。復以饗殿器用修造疏忽,奪尚書,仍留世職。

二十二年,上以俄羅斯數犯邊,擾及索倫、飛牙喀諸部,命集兵黑龍江,將進討,遣瑪拉往索倫儲軍實。尋疏言:「索倫總管博克所獲俄羅斯人及軍前招降者,皆迫於軍威,不宜久留索倫,應移之內地。」詔允行。復言:「雅克薩、尼布楚二城久為羅剎所據,臣密詗雅克薩惟耕種自給,尼布楚歲捕貂與喀爾喀貿易資養贍。請飭喀爾喀車臣汗禁所部與尼布楚貿易,並飭黑龍江將軍水陸並進,示將攻取雅克薩,因刈其田禾,則俄羅斯將不戰自困。」上然之,即以瑪拉所奏檄示喀爾喀。二十四年,遣都統朋春等帥師往黑龍江議進兵,授瑪拉副都統銜,參贊軍務。遣蒙古兵三十詗雅克薩城,生擒羅剎七人,得城中設備及乞援各部狀。是年夏,朋春等攻羅剎克之,逐其人。瑪拉在事有功。二十五年,黑龍江佐領鄂色以耕牛多斃,農器損壞,奏請儲備,命瑪拉往黑龍江督理農務。諭曰:「農事關軍餉,令嚴督合力播種。」值歲豐,收穫甚稔。二十七年,授護軍統領。

二十九年,噶爾丹侵掠喀爾喀,命瑪拉偕都統額赫納、前鋒統領碩鼐等率兵往討之,賜內廝馬以行。未幾,噶爾丹掠烏珠穆沁,命裕親王福全等分統大軍出塞擊之,噶爾丹敗遁。師旋,三十年,復來犯,至阿爾哈賚,無所掠而遁。時土謝圖汗、車臣汗率所部來歸,上幸塞外撫輯,瑪拉扈從。旋命偕都統瓦岱等率兵赴圖拉偵噶爾丹,抵克魯倫河,聞其遠竄,乃還。授西安將軍。

三十二年,準噶爾和碩特部臺吉巴圖爾額爾克濟農來降,上以其人未可信,命瑪拉徙入內地,毋令復逸。瑪拉疏言:「巴圖爾額爾克濟農率所屬二千餘口,窮乏來歸,揆其情狀,當不復逸。」遂遣官護送,並其子臺吉雲木春來朝,優賚遣之。未幾,瑪拉卒於官,賜祭葬,謚敏恪。

論曰:俄羅斯之為羅剎,譯言緩急異耳,非必東部別有是名也。初遣兵詗敵,郎坦主其事;取雅克薩城,朋春、薩布素迭為將,而郎坦與瑪拉實佐之。尼布楚盟定,開市庫倫,是為我國與他國定約互市之始。用兵當期必克,我茍草率,彼益猖狂,聖祖諭薩布素數言,得馭夷之要矣。


\end{pinyinscope}