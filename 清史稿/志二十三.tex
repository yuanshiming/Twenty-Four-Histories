\article{志二十三}

\begin{pinyinscope}
時憲四

△康熙甲子元法中

日躔用數

康熙二十三年甲子天正冬至為法元。癸亥年十一月冬至。

周天三百六十度。平分之為半周,四分之為象限,十二分之為宮,每度六十分,秒微纖以下皆以六十遞析。周天入算,化作一百二十九萬六千秒。

周日一萬分。時則二十四,刻則九十六,刻下分則一千四百四十,秒則八萬六千四百。

周歲三百六十五日二四二一八七五。

紀法六十。

宿法二十八。

太陽每日平行三千五百四十八秒,小餘三三0五一六九。

最卑歲行六十一秒,小餘一六六六六。

最卑日行十分秒之一又六七四六九。

本天半徑一千萬。

本輪半徑二十六萬八千八百一十二。

均輪半徑八萬九千六百零四。

宿度見天文志。

歲差五十一秒。

各省及蒙古北極高度、東西偏度、見天文志。

黃赤大距,二十三度二十九分三十秒。

最卑應,七度十分十一秒十微。

氣應,七日六五六三七四九二六。

宿應,五日六五六三七四九二六。

日乾,甲、乙、丙、丁、戊、己、庚、辛、壬、癸。

支,子、丑、寅、卯、辰、巳、午、未、申、酉、戌、亥。

宿名,角、亢、氐、房、心、尾、箕、斗、牛、女、虛、危、室、壁、奎、婁、胃、昂、畢、參、觜、井、鬼、柳、星、張、翼、軫。

時名,從十二支各分初、正。起子正,盡夜子初。

推日躔法求天正冬至,置周歲,以距元年數減一得積年乘之,得中積分,加氣應得通積分,上考往古,則減氣應得通積分。其日滿紀法去之,餘為天正冬至日分。上考往古,則以所餘轉與紀法相減,餘為天正冬至日分。自初日起甲子,其小餘以刻下分通之,如法收為時刻。周日一萬分為一率,小餘為二率,刻下分為三率,求得四率為時分。滿六十分收為一時,十五分收為一刻。初時起子正,中積分加宿應,滿宿法去之,為天正冬至值宿日分,初日起角宿。

求平行,以周日為一率,太陽每日平行為二率,天正冬至小餘與周日相減餘為三率,求得四率為年根秒數。又置太陽每日平行,以本日距冬至次日數乘之,得數為秒。與年根相並,以宮度分收之,得平行。

求實行,置最卑歲行,以積年乘之。又置最卑日行,以距冬至次日數乘之。兩數相並,加最卑應,上考則減最卑應。以減平行為引數。用平三角形,以本輪半徑三分之二為對正角之邊,以引數為一角,求得對角之邊倍之。又求得對又一角之邊,與本天半徑相加減。引數三宮至八宮則相加,九宮至二宮則相減。復用平三角形,以加倍之數為小邊,加減本天半徑之數為大邊,正角在兩邊之中,求得對小邊之角為均數。置平行以均數加減之,引數初宮至五宮為加,六宮至十一宮為減。得實行。求宿度,以積年乘歲差,得數加甲子法元黃道宿度,為本年宿鈐,以減實行,餘為日躔宿度。若實行不及減宿鈐,退一宿減之。

求紀日值宿,置距冬至次日數,加冬至,日滿紀法去之。初日起甲子,加冬至值宿,日滿宿法去之。初日起角宿,得紀日值宿。

求節氣時刻,日躔初宮丑,星紀。初度為冬至,十五度為小寒。一宮子,元枵。初度為大寒,十五度為立春。二宮亥,娵訾。初度為雨水,十五度為驚蟄。三宮戌,降婁。初度為春分,十五度為清明。四宮酉,大梁。初度為穀雨,十五度為立夏。五宮申,實沈。初度為小滿,十五度為芒種。六宮未,鶉首。初度為夏至,十五度為小暑。七宮午,鶉火。初度為大暑,十五度為立秋。八宮巳,鶉尾。初度為處暑,十五度為白露。九宮辰,壽星。初度為秋分,十五度為寒露。十宮卯,大火。初度為霜降,十五度為立冬。十一宮寅,析木。初度為小雪,十五度為大雪。皆以子正日躔未交節氣宮度者,為交節氣本日;已過節氣宮度者,為交節氣次日。乃以本日實行與次日實行相減為一率,每日刻下分為二率,本日子正實行與節氣宮度相減為三率,求得四率為距子正後之分數,乃以時刻收之,即得節氣初正時刻。如實行適與節氣宮度相符而無餘分,即為子正初刻。求各省節氣時刻,皆以京師為主,視偏度加減之。每偏一度,加減時之四分。偏東則加,偏西則減。推節氣用時法,以交節氣本日均數變時為均數時差,反其加減。又以半徑為一率,黃赤大距餘弦為二率,本節氣黃道度正切為三率,求得四率為赤道正切。檢表得度,與黃道相減,餘變時為升度時差。二分後為加,二至後為減。皆加減節氣時刻,為節氣用時。求距緯度,以本天半徑為一率,黃赤大距度之正弦為二率,實行距春秋分前後度之正弦為三率,實行初宮初度至二宮末度,與三宮相減,餘為春分前;三宮初度至五宮末度,則減去三宮,為春分後。六宮初度至八宮末度,與九宮相減,餘為秋分前;九宮初度至十一宮末度,則減去九宮,為秋分後。求得四率為正弦,檢表得距緯度。實行三宮至八宮,其緯在赤道北;九宮至二宮,其緯在赤道南。

求日出入晝夜時刻,以本天半徑為一率,北極高度之正切為二率,本日距緯度之正切為三率,求得四率為正弦,檢表得日出入在卯酉前後赤道度。變時,一度變時之四分,凡言變時皆仿此。為距卯酉分。以加減卯酉時,即得日出入時刻。春分前、秋分後,以加卯正為日出,減酉正為日入。春分後、秋分前,以減卯正為日出,加酉正為日入。又倍距卯酉分,以加減半晝分,得晝夜時刻。春分後以加得晝刻,以減得夜刻,秋分後反是。

月離用數

太陰每日平行四萬七千四百三十五秒,小餘0二一一七七。

太陰每時四刻。平行一千九百七十六秒,小餘四五九二一五七。

月孛即最高,每日行四百0一秒,小餘0七七四七七。

正交每日平行一百九十秒,小餘六四。

本輪半徑五十八萬。

均輪半徑二十九萬。

負圈半徑七十九萬七千。

次輪半徑二十一萬七千。

次均輪半徑一十一萬七千五百。

朔、望黃白大距四度五十八分三十秒。

兩弦黃白大距五度一十七分三十秒。

黃白大距中數五度0八分。

黃白大距半較九分三十秒。

太陰平行應一宮0八度四十分五十七秒十六微。

月孛應三宮0四度四十九分五十四秒0九微。

正交應六宮二十七度十三分三十七秒四十八微。

推月離法求天正冬至,同日躔。

求太陰平行,置中積分,加氣應詳日躔。小餘,不用日,下同。減天正冬至小餘,得積日。上考則減氣應小餘,加天正冬至小餘。與太陰每日平行相乘,滿周天秒數去之,餘數收為宮度分。以加太陰平行應,得太陰年根。上考則減,又置太陰每日平行,以距天正冬至次日數乘之,得數為秒。以宮度分收之,與年根相並,滿十二宮去之。為太陰平行。

求月孛行,以積日見前條,下同。與月孛每日行相乘,滿周天秒數去之,餘數收為宮度分。以加月孛應,得月孛年根。上考則減。又置月孛每日行以距天正冬至次日數乘之,得數為秒,以宮度分收之,與年根相並,滿十二宮去之。為月孛行。

求正交平行,以積日與正交每日平行相乘,滿周天秒數去之,餘數收為宮度分,以減正交應,正交應不足減者,加十二宮減之。得正交年根。上考則加。又置正交每日平行,以距天正冬至次日數乘之,得數為秒,以宮度分收之,以減年根,年根不足減者,加十二宮減之。為正交平行。

求用時太陰平行,以本日太陽均數變時,詳日躔。得均數時差。均數加者,時差為減;均數減者,時差為加。又以本日太陽黃、赤經度詳日躔。相減餘數變時,得升度時差。二分後為加,二至後為減。乃以兩時差相加減,為時差總。兩時差加減同號者,則相加為總,加者仍為加,減者仍為減。加減異號者,則相減為總,加數大者為加,減數大者為減。化秒,與太陰每時平行相乘為實,以一度化秒為法除之,得數為秒,以度分收之,得時差行。以加減太陰平行,時差總為加者則減,減者則加。為用時太陰平行。

求初實行,置用時太陰平行,減去月孛行,得引數。用平三角形,以本輪半徑之半為對正角之邊,以引數為一角,求得對角之邊三因之。又求得對又一角之邊,與本天半徑相加減。引數九宮至二宮相加,三宮至八宮相減。復用平三角形,以三因數為小邊,加減本天半徑數為大邊,正角在兩邊之中,求得對小邊之角為初均數,★求得對正角之邊。即次輪最近點距地心之線。乃置用時太陰平行,以初均數加減之,引數初宮至五宮為減,六宮以後為加。為初實行。

求白道實行,置初實行,減本日太陽實行得次引。即距日度。用平三角形,以次輪最近點距地心線為一邊,倍次引之通弦本天半徑為一率,次引之正弦為二率,次輪半徑為三率,求得四率倍之即通弦。為一邊;以初均數與引數減半周之度引數不及半周,則與半周相減,如過半周,則減去半周。相加,又以次引距象限度次引不及象限,則與象限相減;如過象限及過三象限,則減去象限及三象限,用其餘;如過二象限,則減去二象限,餘數仍與象限相減,為次引距象限度。加減之,初均數減者,次引過象限或過三象限則相加,不過象限或過二象限則相減。初均數加者反是。為所夾之角,若相加過半周,則與全周相減,用其餘為所夾之角。若相加適足半周或相減無餘,則無二均數。若次引為初度,或適足半周,亦無二均數。求得對通弦之角為二均數,如無初均數,以次輪心距地心為一邊,次輪半徑為一邊;次引倍數為所夾之角,次引過半周者,與全周相減,用其餘;在最高為所夾之內角,在最卑為所夾之外角,求得對次輪半徑之角為二均數。隨定其加減號。以初均數與均輪心距最卑之度相加,為加減泛限。泛限適足九十度,則二均加減與初均同。如泛限不足九十度,則與九十度相減,餘數倍之,為加減定限。初均減者,以次引倍度;初均加者,以次引倍度減全周之餘數,皆與定限較。如泛限過九十度者,減去九十度,餘數倍之,為加減定限。初均加者,以次引倍度;初均減者,以次引倍度減全周之餘數,皆與定限較。並以大於定限,則二均之加減與初均同;小於定限者反是。★求得對角之邊,為次均輪心距地心線。又以此線及次引,用平三角形,以次均輪心距地為一邊,次均輪半徑為一邊,次引倍度為所夾之角,次引過半周者,與全周相減,用其餘。求得對次均輪半徑之角為三均數,隨定其加減號。次引倍度不及半周為加,過半周為減。乃以二均數與三均數相加減,為二三均數。兩均數同號則相加,異號則相減。以加減初實行,兩均數同為加者仍為加,同為減者仍為減。一為加一為減者,加數大為加,減數大為減。為白道實行。

求黃道實行,用弧三角形,以黃白大距中數為一邊,大距半較為一邊,次引倍度為所夾之角,次引過半周與全周相減,用其餘。求得對角之邊為黃白大距,並求得對半較之角為交均。以交均加減正交平行,次引倍度不及半周為減,過半周為加。得正交實行。又加減六宮為中交實行,置白道實行,減正交實行,得距交實行。以本天半徑為一率,黃白大距之餘弦為二率,距交實行之正切為三率,求得四率為黃道之正切。檢表得度分,與距交實行相減,餘為升度差,以加減白道實行,距交實行不過象限,或過二象限為減,過象限及過三象限為加。為黃道實行。

求黃道緯度,以本天半徑為一率,黃白大距之正弦為二率,距交實行之正弦為三率,求得四率為正弦。檢表得黃道緯度,距交實行初宮至五宮為黃道北,六宮至十一宮為黃道南。

求四種宿度,依日躔求宿度法,求得本年黃道宿鈐。以黃道實行、月孛行及正交、中交實行各度分視其足減宿鈐內某宿則減之,餘為四種宿度。

求紀日值宿,同日躔。

求交宮時刻,以太陰本日實行與次日實行相減未過宮為本日,已過宮為次日。餘為一率,刻下分為二率,太陰本日實行不用宮。與三十度相減餘為三率,求得四率為距子正分數。如法收之,得交宮時刻。

求太陰出入時刻,以本日太陽黃道經度求其相當赤道經度。又用弧三角形,以太陰距黃極為一邊,黃極距北極為一邊,即黃赤大距。太陰距冬至黃道經度為所夾之外角,過半周者與全周相減,用其餘。求得對邊為太陰距北極度。與九十度相減,得赤道緯度。不及九十度者,與九十度相減,餘為北緯。過九十度者,減去九十度,餘為南緯。又求得近北極之角,為太陰距冬至赤道經度。乃以本天半徑為一率,北極高度之正切為二率,太陰赤道緯度之正切為三率,求得四率為正弦。檢表得太陰出入在卯酉前後赤道度,太陰在赤道北,出在卯正前,入在酉正後;太陰在赤道南,出在卯正後,入在酉正前。以加減前減後加。太陰距太陽赤道度,太陰赤道經度內減去太陽赤道經度即得。得數變時。自卯正酉正後計之,出地自卯正後,入地自酉正後。得何時刻,再加本時太陰行度之時刻,約一小時行三十分,變為時之二分。即得太陰出入時刻。

求合朔弦望,太陰實行與太陽實行同宮同度為合朔限,距三宮為上弦限,距六宮為望限,距九宮為下弦限,皆以太陰未及限度為本日,已過限度為次日。乃以太陰、太陽本日實行與次日實行各相減,兩減餘數相較為一率,刻下分為二率,本日太陽實行加限度上弦加三宮,望加六宮,下弦加九宮。減本日太陰實行,餘為三率,求得四率為距子正之分。如法收之,得合朔弦望時刻。

求正升斜升橫升,合朔日,太陰實行自子宮十五度至酉宮十五度為正升,自酉宮十五度至未宮初度為斜升,自未宮初度至寅宮十五度為橫升,自寅宮十五度至子宮十五度為斜升。

求月大小,以前朔後朔相較,日乾同者前月大,不同者前月小。

求閏月,以前後兩年有冬至之月為準。中積十三月者,以無中氣之月,從前月置閏。一歲中兩無中氣者,置在前無中氣之月為閏。

土星用數

每日平行一百二十秒,小餘六0二二五五一。

最高日行十分秒之二又一九五八0三。

正交日行十分秒之一又一四六七二八。

本輪半徑八十六萬五千五百八十七。

均輪半徑二十九萬六千四百一十三。

次輪半徑一百零四萬二千六百。

本道與黃道交角二度三十一分。

土星平行應七宮二十三度十九分四十四秒五十五微。

最高應十一宮二十八度二十六分六秒五微。

正交應六宮二十一度二十分五十七秒二十四微。

木星用數

每日平行二百九十九秒,小餘二八五二九六八。

最高日行十分秒之一又五八四三三。

正交日行百分秒之三又七二三五五七。

本輪半徑七十萬五千三百二十。

均輪半徑二十四萬七千九百八十。

次輪半徑一百九十二萬九千四百八十。

本道與黃道交角一度十九分四十秒。

木星平行應八宮九度十三分十三秒十一微。

最高應九宮九度五十一分五十九秒二十七微。

正交應六宮七度二十一分四十九秒三十五微。

火星用數

每日平行一千八百八十六秒,小餘六七00三五八。

最高日行十分秒之一又八三四三九九。

正交日行十分秒之一又四四九七二三。

本輪半徑一百四十八萬四千。

均輪半徑三十七萬一千。

最小次輪半徑六百三十萬二千七百五十。

本天高卑大差二十五萬八千五百。

太陽高卑大差二十三萬五千。

本道與黃道交角一度五十分。

火星平行應二宮十三度三十九分五十二秒十五微。

最高應八宮初度三十三分十一秒五十四微。

正交應四宮十七度五十一分五十四秒七微,餘見日躔。

推土、木、火星法

求天正冬至,同日躔。

求三星平行,以積日詳月離。與本星每日平行相乘,滿周天秒數去之,餘收為宮度分,為積日平行。以加本星平行應,得本星年根。上考則減。又置本星每日平行,以所求距天正冬至次日數乘之,得數與年根相並,得本星平行。

求三星最高行,以積日與本星最高日行相乘,得數以加本星最高應,得最高年根。上考則減。又置本星最高日行,以所求距天正冬至次日數乘之,得數與年根相並,得本星最高行。

求三星正交行,以積日與本星正交日行相乘,得數以加本星正交應,得正交年根。上考則減。又置本星正交日行,以所求距天正冬至次日數乘之,得數與年根相並,得本星正交行。

求三星初實行,置本星平行,減最高行,得引數。用平三角形,以均輪半徑減本輪半徑為對正角之邊,以引數為一角,求得對引數角之邊及對又一角之邊。又用平三角形,以對引數角之邊與均輪通弦相加求通弦法,詳月離。為小邊,以對又一角之邊與本天半徑相加減引數三宮至八宮相減,九宮至二宮相加。為大邊,正角在兩邊之中,求得對小邊之角為初均數。並求得對正角之邊為次輪心距地心線,以初均數加減本星平行,引數初宮至五宮減,六宮至十一宮加。得本星初實行。

求三星本道實行,置本日太陽實行減本星初實行,得次引。即距日度。用平三角形,以次輪心距地心線為一邊,次輪半徑為一邊,惟火星次輪半徑時時不同,求法詳後。次引為所夾之外角,過半周者與全周相減,用其餘。求得對次輪半徑之角為次均數,並求得對次引角之邊為星距地心線。乃以次均數加減初實行,加減與初均相反。得本星本道實行。求火星次輪實半徑,以火星本輪全徑命為二千萬為一率,本天高卑大差為二率,均輪心距最卑之正矢為三率,引數與半周相減,即均輪心距最卑度。求得四率為本天高卑差。又以太陽本輪全徑命為二千萬為一率,太陽高卑大差為二率,本日太陽引數之正矢為三率,引數過半周者與全周相減,用其餘。求得四率為太陽高卑差。乃置火星最小次輪半徑,以兩高卑差加之,得火星次輪實半徑。

求三星黃道實行,置本星初實行,減本星正交行,得距交實行。次輪心距正交。乃以本天半徑為一率,本道與黃道交角之餘弦為二率,距交實行之正切為三率,求得四率為正切。檢表得黃道度,與距交實行相減,得升度差,以加減本道實行,距交實行不過象限及過二象限為減,過象限及過三象限為加。得本星黃道實行。

求三星視緯,以本天半徑為一率,本道與黃道交角之正弦為二率,距交實行之正弦為三率,求得四率為正弦,檢表得初緯。又以本天半徑為一率,初緯之正弦為二率,次輪心距地心線為三率,求得四率為星距黃道線。乃以星距地心線為一率,星距黃道線為二率,本天半徑為三率,求得四率為正弦。檢表得本星視緯,隨定其南北。距交實行初宮至五宮為黃道北,六宮至十一宮為黃道南。

求黃道宿度及紀日,同日躔。

求交宮時刻,同月離。

求三星晨夕伏見定限度,視本星黃道實行與太陽實行同宮同度為合伏。合伏後距太陽漸遠,為晨見東方順行。順行漸遲,遲極而退為留退。初退行距太陽半周為退沖,退沖之次日為夕見。退行漸遲,遲極而順為留順。初順行漸疾復近太陽,以至合伏,為夕不見。其伏見限度,土星十一度,木星十度,火星十一度半。合伏前後某日,太陽實行與本星實行相距近此限度,即以本星本日黃道實行,用弧三角形,以赤道地平交角為所知一角,夕,春分後用內角,秋分後用外角;晨反是。實行距春秋分度為對邊,黃赤大距為所知又一角,求得不知之對邊。乃用所知兩邊對所知兩角,求得不知之又一角,夕,秋分後用內角,春分後用外角;晨反是。為限距地高。乃用弧三角形,有正角,有黃道地平交角,即限距地高。有本星伏見限度,為對交角之弧,求得對正角之弧,為距日黃道度。若星當黃道無距緯,即為定限度。又用弧三角形,有正角,有黃道地平交角,以本星距緯為對交角之弧,求得兩角間之弧,為加減差。以加減距日黃道度,緯南加,緯北減。得伏見定限度。視本星距太陽度與定限度相近,如在合伏前某日,即為某日夕不見;在合伏后某日,即為某日晨見。

求三星合伏時刻,視太陽實行將及本星實行,為合伏本日;已過本星實行,為合伏次日。求時刻,於太陽一日之實行即本日次日兩實行之較。內減本星一日之實行為一率,餘同月離求朔、望。

求三星退沖時刻,視本星黃道實行與太陽實行相距將半周,為退沖本日;已過半周,為退沖次日。求時刻之法,以太陽一日之實行與本星一日之實行相加為一率,餘同前。

求同度時刻,以兩星一日之實行相加減兩星同行則減。一順一逆則加。為一率,刻下分為二率,兩星相距為三率,求得四率為距子正之分數,以時刻收之即得。五星並同。

金星用數

每日平行三千五百四十八秒,小餘三三0五一六九。

最高日行十分秒之二又二七一0九五。

伏見每日平行二千二百十九秒,小餘四三一一八八六。

本輪半徑二十三萬一千九百六十二。

均輪半徑八萬八千八百五十二。

次輪半徑七百二十二萬四千八百五十。

次輪面與黃道交角三度二十九分。

金星平行應初宮初度二十分十九秒十八微。

最高應六宮一度三十三分三十一秒四微。

伏見應初宮十八度三十八分十三秒六微。

水星用數

每日平行與金星同。

最高日行十分秒之二又八八一一九三。

伏見每日平行一萬一千一百八十四秒,小餘一一六五二四八。

本輪半徑五十六萬七千五百二十三。

均輪半徑一十一萬四千六百三十二。

次輪半徑三百八十五萬。

次輪心在大距,與黃道交角五度四十分。

次輪心在正交,與黃道交角北五度五分十秒,其交角較三十四分五十秒。與大距交角相較,後仿此。南六度三十一分二秒,其交角較五十一分二秒。

次輪心在中交,與黃道交角北六度十六分五十秒,其交角較三十六分五十秒。南四度五十五分三十二秒,其交角較四十四分二十八秒。

水星平行應與金星同。

最高應十一宮三度三分五十四秒五十四微。

伏見應十宮一度十三分十一秒十七微,餘見日躔。

推金、水星法

求天正冬至,同日躔。

求金、水本星平行,同土、木、火星。

求金、水最高行,同土、木、火星。

求金、水伏見平行,同本星平行。

求金、水正交行,置本星最高平行,金星減十六度,水星加減六宮,即得。

求金星初實行,用本星引數求初均數,以加減本星平行,為本星初實行。及求次輪心距地心線,並同土、木、火星。

求水星初實行,用平三角形,以本輪半徑為一邊,均輪半徑為一邊,以引數三倍之為所夾之外角,過半周者與全周相減,用其餘。求其對角之邊,並對均輪半徑之角。又用平三角形,以本天半徑為大邊,以對角之邊為小邊,以對均輪半徑之角與均輪心距最卑度相加減,引數不及半周者,與半周相減;過半周者,減去半周,即均輪心距最卑度。加減之法,視三倍引數不過半周則加,過半周則減。為所夾之角,求得對小邊之角為初均數,並求得對角之邊為次輪心距地心線。以初均數加減水星平行,引數初宮至五宮為減,六宮至十一宮為加。得水星初實行。

求金、水伏見實行,置本星伏見平行,加減本星初均數,引數初宮至五宮為加,六宮至十一宮為減。即得。

求金、水黃道實行,用平三角形,以本星次輪心距地心線為一邊,本星次輪半徑為一邊,本星伏見實行為所夾之外角,過半周者與全周相減,用其餘。求得對次輪半徑之角為次均數,並求得對角之邊為本星距地心線。以次均數加減初實行,伏見實行初宮至五宮為加,六宮至十一宮為減。得本星黃道實行。

求金、水距次交實行,置本星初實行,減本星正交行,為距交實行。與本星伏見實行相加,得本星距次交實行。

求金、水視緯,以本天半徑為一率,本星次輪與黃道交角之正弦為二率,金星交角惟一,水星交角則時時不同,須求實交角用之,法詳後。本星距次交實行之正弦為三率,求得四率為正弦,檢表得本星次緯。又以本天半徑為一率,本星次緯之正弦為二率,本星次輪半徑為三率,求得四率為本星距黃道線。乃以本星距地心線為一率,本星距黃道線為二率,本天半徑為三率,求得四率為正弦,檢表得本星視緯,隨定其南北。初宮至五宮為黃道北,六宮至十一宮為黃道南。

求水星實交角,以半徑一千萬為一率,交角較化秒為二率,距交實行九宮至二宮用正交交角較,三宮至八宮用中交交角較,仍視其南北用之。距交實行之正弦為三率,求得四率為交角差。置交角,用交角之法與用交角較同。以交角差加減之,距交實行九宮至二宮,星在黃道北則加,南則減;三宮至八宮反是。得實交角。

求黃道宿度及紀日,同日躔。

求交宮時刻,同月離。

求金、水晨夕伏見定限度,本星實行與太陽實行同宮同度為合伏,合伏後距太陽漸遠;夕見西方順行,順行漸遲,遲極而退為留退。初退行漸近太陽,則夕不見,復與太陽同度為合退伏。自是又漸遠太陽,晨見東方。仍退行漸遲,遲極而順為留順。初順行漸疾,復近太陽,以至合伏,為晨不見。其伏見限度,金星為五度,水星為十度。其求定限度之法,與土、木、火星同,視本星距太陽度與定限相近。如在合伏前某日,即為某日晨不見;合伏后某日,即為某日夕見;合退伏前某日,即為某日夕不見;合退伏后某日,即為某日晨見。

求金、水合伏時刻,視本星實行將及太陽實行為合伏本日,已過太陽實行為合伏次日。求時刻之法,與月離求朔、望時刻之法同。

求金、水合退伏時刻,視太陽實行將及本星實行為合退伏本日,已過本星實行為合退伏次日。求時刻之法,與土、木、火星求退沖時刻之法同。

恆星用數

見日躔。

推恆星法求黃道經度,以距康熙壬子年數減一,得積年歲差,乘之。收為度分,與康熙壬子年恆星表經度相加,得各恆星本年經度。求赤道經緯度,用弧三角形,以星距黃極為一邊,黃赤大距為一邊,本年星距夏至前後為所夾之角,求得對星距黃極邊之角。夏至前用本度,夏至後與周天相減用其餘度。自星紀宮初度起算,為各恆星赤道經度。又求得對原角之邊,與象限相減,餘為赤道緯度。減象限為北,減去象限為南。

求中星,以刻下分為一率,本日太陽實行與次日太陽實行相減餘為二率,以所設時刻化分為三率,求得四率,與本日太陽實行相加,得本時太陽黃道經度。用弧三角形,推得太陽赤道經度,以所設時刻變赤道度一時變為十五度,一分變為十五分,一秒變為十五秒。加減半周,不及半周則加半周,過半周則減半周。得本時太陽距午後度。與太陽赤道經度相加,得本時正午赤道經度。視本年恆星赤道經度同者,即為中星。


\end{pinyinscope}