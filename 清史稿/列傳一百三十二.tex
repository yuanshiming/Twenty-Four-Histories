\article{列傳一百三十二}

\begin{pinyinscope}
永保惠齡宜綿子瑚素通阿

英善福寧景安秦承恩

永保,費莫氏,滿洲鑲紅旗人,勒保之弟也。以官學生考授內閣中書,充軍機章京,遷侍讀。乾隆三十七年,父溫福征金川,永保齎送定邊將軍印,遂隨軍。明年,溫福戰歿木果木,永保冒矢石奪回父尸,襲輕車都尉,遷吏部郎中。洎金川平,追論木果木之敗,咎在溫福,奪世職,仍留永保原官。出為直隸口北道,歷霸昌、清河兩道。遷布政使,調江蘇。四十九年,擢貴州巡撫,歷江西、陜西。五十一年,署陜甘總督。尋授塔爾巴哈臺參贊大臣。五十六年,哈薩克汗斡裡素勒坦遣子入覲,詔嘉永保撫綏有方,授內大臣,賞雙眼花翎。五十八年,調喀什噶爾參贊大臣,授戶部侍郎,留駐新疆。六十年,調烏魯木齊都統。

嘉慶元年春,湖北教匪起,永保奉詔入京,行抵西安,命偕將軍恆瑞率駐防兵二千,調陜西、廣西、山東兵五千會剿。三月,至湖北,總督畢沅疏陳各路剿殺不下數萬,而賊起益熾。詔分專責成:永保、恆瑞任竹山、保康一路;畢沅、舒亮任當陽、遠安、東湖一路;惠齡、富志那任枝江一路;鄂輝任襄陽、穀城、均州、光化一路;孫士毅任酉陽、來鳳一路。永保偕恆瑞復竹山,進房縣,擒賊首祁中耀;餘賊遁保康白雲寺山,復敗之,擒賊目曾世興等。永保疏言:「襄陽賊數萬,最猖獗,賊首姚之富、齊王氏、劉之協皆在其中,為四方諸賊領袖,破之則流賊自瓦解。宜俟諸軍大集,合力分攻。」帝韙之。五月,永保等馳赴襄陽,自樊城進取鄧桃湖,會軍呂堰。賊退屯雙溝,分軍五路夾擊,殲賊二千餘,賊分竄孝感,距漢陽百餘里,幸為潦阻,武昌戒嚴。時畢沅圍當陽數月不下,惠齡剿枝江賊亦無功,詔命永保總統湖北諸軍,先靖襄陽,而後分攻孝感、當陽兩路。參將傅成明等擊孝感賊,遇伏敗歿;永保令明亮馳救,復請調苗疆防兵助剿。六月,永保渡滾河,破梁家岡、張家垱賊營二十餘座,賊竄棗陽,潛踞隨州之梓山、青潭,連破之。復偕恆瑞、慶成破賊於紅土山,擒賊渠黃玉貴。於是襄陽、呂堰迤東百數十里,及棗陽、隨州、宜城無賊氛。孝感之賊,亦為明亮所殲。詔嘉永保調度協宜,加太子太保。

先是命署湖廣總督,及畢沅復當陽,永保請寢前命,允之。八月,移剿鍾祥,明亮以師來會。賊自溫浹口至千弓垱,依山結營,亙數十里。永保率大軍由西北進擊,繪圖陳奏。帝方以東南空虛,慮賊逃竄,適明亮疏言:「鍾祥為賊巢穴,宜四面夾攻,以防漏網。今永保以九千餘兵由西北追壓,而東南要截之兵僅三千餘,地闊兵單,難杜竄逸。」帝以永保擁眾自衛,切責之。明亮敗賊土門沖,永保不能夾擊,賊轉而北,永保偕明亮追至襄陽雙溝。賊分兩路竄河南:東由棗陽趨唐縣,西由呂堰趨鄧州。官軍躡西路,敗諸呂堰,獲姚之富母、媳及孫,而東路賊已入唐縣滹沱鎮。疏言:「追賊經月,兵力疲憊,難以痛殲,請增兵助剿。」詔斥其無能,調山東、直隸兵四千,復簡健銳、火器營各軍赴之。十一月,新兵既至,攻破唐縣賊屯十一。姚之富已遁,犯棗陽,復渡滾河而西,蹂呂堰,向光化、穀城。圍景安於鄧州魏家集,越二日,援兵始至。帝怒永保擁勁旅萬餘,徒尾追不迎擊,致賊東西橫躪無忌,褫職逮京,下獄,籍其家,並褫其子侍衛寧志、寧怡職,發往熱河。

三年,以兄勒保擒川賊王三槐功,推恩宥釋。勒保請將永保發軍營效力,不許。四年,勒保為經略大臣,予永保藍翎侍衛,齎經略印赴軍。尋擢頭等侍衛,署陜西巡撫。與明亮會剿張漢潮於終南華林山中,遇伏敗績;復與明亮不協,互攻訐。詔逮問,並坐前在湖北動用軍需受饋遺事,論大闢,詔原之,免罪,予八品領催,自備資斧赴烏里雅蘇臺辦事。六年,充參贊大臣。

七年,授雲南巡撫。八年,威遠、思茅惈匪擾邊,永保赴普洱,偕提督烏大經進討。肇亂土弁刁永和聞風遁,威遠惈匪亦退,擒思茅惈酋扎安波賽悶,餘匪奔逸。南興土司張輔國屢與孟連土司爭界構釁,至是勘定之。永保疏陳善後事:「內地雜居夷人不法,按律懲治;土司夷境滋事,但遣兵防範,不使內竄。」詔嘉得大體,弭邊釁,賞花翎。

十三年,兼署貴州巡撫,調廣東。尋擢兩廣總督,未至,卒於途。贈內大臣,詔念前勞,曾籍沒,家無餘貲,賜銀千兩治喪,謚恪敏。孫文慶,咸豐中官大學士,自有傳。

惠齡,字椿亭,薩爾圖克氏,蒙古正白旗人。父納延泰,乾隆中,官理籓院尚書、軍機大臣,加太子少保。因喀爾喀臺吉沁多爾濟規避軍事,不劾奏,罷職。復起用,終於理籓院侍郎。

惠齡由繙譯官補戶部筆帖式,充軍機章京。累遷員外郎,緣事奪職。起戶部主事,仍直軍機。乾隆四十年,予副都統銜,充西寧辦事大臣,調伊犁領隊大臣。擢工部侍郎,調吏部。充塔爾巴哈臺參贊大臣。五十年,回京,署正黃旗滿洲副都統。授湖北巡撫,調山東。五十六年,擢四川總督。征廓爾喀,命為參贊,赴西藏會剿,督治糧運。事平,圖形紫光閣,列前十五功臣中。五十八年,授山東巡撫,調湖北,再調安徽。六十年,授戶部侍郎。苗疆用兵,留署湖北巡撫,治糧餉。

嘉慶元年正月,教匪聶傑人、張正謨等倡亂於枝江、宜都,率師往剿,總兵富志那擒首逆聶傑人,而襄、鄖、宜、施諸郡賊並起。命惠齡專剿枝江、宜都一路,自春徂夏無功,以大雨為解,嚴詔切責。八月,克灌腦灣賊寨,擒張正謨等,加太子少保,署工部尚書,予二等輕車都尉世職。進攻涼山,搗其巢,擒首逆覃士潮,宜都、枝江悉平,移軍長陽黃柏山會剿。十一月,襄陽賊姚之富自黃龍垱偷渡滾河,竄河南,黜總統永保,以惠齡代之,馳赴襄陽。疏言:「襄、鄧平衍,無險可扼。賊習地勢,必不自趨絕地。惟有嚴防漢江潛渡,並堰唐河、白河,移難民於河西,守岸團練以蹙賊。」會之富折回湖北境,惠齡迎擊,遏其西軼,敗之茅茨畈,分兵五路兜剿。二年二月,敗賊於鮑家畈,擒賊首劉起榮;復敗賊於曾家店,鏖戰於鄭家河,殲獲甚眾,賞雙眼花翎,擢理籓院尚書,兼鑲白旗蒙古都統。惠齡偕恆瑞、慶成剿襄陽賊,屢破之,餘眾僅數千,勢甚蹙,分路竄河南境,官軍疲於尾追,不易得一戰,先後並入陜西,遂復猖獗。五月,李全、王廷詔、姚之富合為一路,由紫陽白馬石竄渡漢江,後五日,惠齡始至,奪宮銜、世職、花翎,易宜綿總統軍務,降惠齡為領隊,聽節制。

賊既分竄入川,十月,王廷詔、高均德復北犯,窺渡漢江,惠齡邀擊敗之,斬賊二千。詔嘉其僅兵二千當賊二萬,以少擊眾,復雙眼花翎。十一月,齊王氏、張漢潮、姚之富、高均德合入漢中南山,自黃官嶺至新集,連營二十里,欲渡漢。惠齡軍北岸,蹙其半濟,賊走寧羌,追敗之,折竄漢中。因移兵扼漢南,賊不得北竄,復分道入川,惠齡繞由西鄉、太平赴大寧、夔州兜剿。時川匪王三槐、徐天德竄梁山,羅其清、冉文儔分屯營山、儀隴。三年,陜、襄諸賊在川境者俱會於文儔,而三槐、天德自太平走與合,勢張甚。詔總統勒保會諸將,分路進剿,惠齡與德楞泰為一路,夾攻羅、冉二賊。五月,擊文儔於儀隴,其清及阮正通先後來援,皆敗之。賊屯大神山,連營數十里,六月,與德楞泰合攻,破之,斬賊甚眾。文儔走箕山龍鳳坪,與其清相犄角,阮正通等又與合。帝以首逆稽誅,屢詔嚴責,於是德楞泰破賊箕山,其清奔天鵬寨,惠齡分路進攻,十二月,其清就擒,檻送京師。四年正月,文儔就擒,予一等輕車都尉世職。丁母憂,會其清讞詞稱惠齡一軍較弱,帝斥其為賊所輕,命回京守制,降兵部侍郎。尋授山東巡撫。六年,擢陜甘總督,專剿南山餘匪。復以剿賊遲緩,降二品頂戴。七年,教匪平,復頭品頂戴、花翎。九年,卒,贈太子少保,封二等男,謚勤襄。子桂斌,官和闐幫辦大臣。

宜綿,初名尚安,鄂濟氏,漢洲正白旗人。由兵部筆帖式充軍機章京,累遷員外郎。從征金川,進郎中。乾隆四十三年,出為直隸口北道,擢陜西布政使。四十七年,擢廣東巡撫,以鹽商沈翼川獄瞻徇,褫職,戍新疆。尋予四品銜,充吐魯番領隊大臣。石峰堡回亂,駐守平涼。歷庫車、喀什噶爾辦事大臣,烏魯木齊都統。五十九年,入覲,道經固關,值水災,飭官吏賑撫,高宗嘉之,命改名宜綿。六十年,授陜甘總督。

嘉慶元年,教匪起,湖北、陜西戒嚴。宜綿駐軍商州,令副將百祥剿鄖陽、鄖西賊,克孤山大寨,賊首王全禮伏誅,漢江以北安堵,加太子太保,賞雙眼花翎。甘肅歲祲,命宜綿回蘭洲賑撫。是年冬,四川教匪起,由太平入陜境,擾安康、平利、紫陽諸縣,宜綿督軍馳剿,賊逼興安,分踞城南安嶺、城北將軍山,進攻克之,擒其渠王可秀、馮得士等。復殲漢江北岸大小米溪賊。偕提督柯籓、總兵索費英阿移攻漢南洞河、汝河諸賊,賊並五雲寨,乘雪夜火其寨,殲馘甚眾,詔宜綿進剿達州。二年春,攻太平賊於通天觀、高家寨、南津關,連敗之。川匪最悍者,達州徐天德,東鄉王三槐、冷天祿,巴州羅其清,通江冉文儔。天德、三槐等合陷東鄉,踞張家觀;其清踞方山坪,文儔竄王家寨,圖據周家河,梗運道,且乘間與張家觀合。宜綿遣兵攻王家寨,分襲張家觀,自率隊夜焚曾家山賊柵,天德分援兩路,遂乘虛下張家觀,復東鄉;餘賊奔清溪場、金峨寺,據險抗拒,四月,官軍分五路進克之。天德等竄重石子、香爐坪,將與巴州賊合。宜綿潛攻王家寨,賊走方山坪,天德來援,敗之。知縣劉清素得民心,令招諭諸賊,三槐率眾詭降,陰圖襲營,宜綿覺其詐,設伏擊退。五月,達州賊傾巢出犯,有備不得逞。宜綿駐軍大成寨,遣將襲三槐於毛坪,三槐中槍跳免。

時襄賊由漢江北渡入陜,署總督陸有仁以罪逮,乃調英善督陜甘,黜惠齡總統,命宜綿代之,兼攝四川總督。於是令明亮攻重石子,德楞泰與鄉勇羅思舉夾擊敗之,分二路竄,追殲孫士鳳於磨子壩。士鳳為四川教首,三槐等皆其徒也,至是為德楞泰所誅。餘賊西走徐家山,乘霧夜遁。其方山坪賊為百祥所截,舒亮圍賊林亮工於巴州白崖山,觀成、劉君輔破大寧賊,圍之於老木園,川賊漸蹙;而襄陽賊李全、王廷詔、姚之富等由陜分道入川,與之響應,勢復熾。雲陽賊伏陳家山,與襄賊約犯官軍,為羅思舉所殲。李全等踞開縣南天洞、火焰壩,旋奔雲安場,開、萬諸匪應之,謀犯夔州,附近賊蜂起,詔責宜綿專剿。七月,駐軍竇山關,開縣、東鄉交界地也。

川賊分立名號:羅其清稱白號,冉文儔稱藍號,踞方山坪;王三槐稱白號,徐天德稱青號,踞尖山坪。劉清率鄉勇與百祥、硃射鬥會剿方山坪,賊潰圍竄通江、巴州,與天德合。既而天德等竄青杠渡,圍巴州,其清、文儔欲從儀隴、南部分犯保寧,奪官軍餉道,百祥扼其前,退走黃渡河,旁掠儀隴;宜綿扼之官渡口,三槐等竄渠縣,其清、文儔走巴州。三槐復分攻鄰水,陷長壽,東趨重慶。時齊王氏、姚之富已竄湖北,李全、高均德先後分竄陜西。宜綿疏言:「惠齡、恆瑞、明亮、德楞泰皆入陜,惟臣一人在川。諸賊齊擾川東北運道,嘉陵江防孔亟,欲親赴保寧,則川東千里無人調度。請別簡總督治理地方,而己親督師專一辦賊。」帝亦以宜綿年老,十月,命勒保總統軍務,宜綿以總督兼理軍需。又疏言:「軍興以來,四川調兵一萬九千有奇,陜、甘合調二萬有奇,兩湖更無餘兵可調。各省募補者難備攻剿;州縣團勇,各衛村莊,尤難責其長驅赴敵。目前賊勢,明亮、德楞泰至襄陽,則鄖賊竄興安,宜昌賊回夔、巫;況雲陽、奉節伏莽尚多,兵力日分日薄。請敕添練備戰之兵,四川、陜甘、湖北各五千。至隨營鄉勇,費與兵等,賞過則驕,威過則散,究非紀律之師。不若選充營伍,賊平即補營額,費不虛糜,而驍悍有所約束。」詔行之。

三年春,調勒保四川總督,宜綿回任陜甘,駐陜境辦賊。未幾,高均德、齊王氏竄漢陰,褫明亮職,命宜綿赴軍督剿;而齊王氏、姚之富已為德楞泰、明亮所殲,阮正通、張漢潮先後犯陜境,川賊劉成棟走與合。宜綿自鎮安分路截剿,漢潮折向通江、巴州,正通竄城固,李全與高均德合屯五郎、鎮安、山陽間。宜綿偕明亮要之雒南,鏖戰兩河口,均德竄秦嶺,正通折入川。五月,賊分股北出鳳縣,掠兩當,闌入甘境,詔斥宜綿疏防。既而明亮敗賊於略陽,成棟、漢潮復由竹谿竄平利。命宜綿與額勒登保為一路,專剿平利之賊,尋敗之於孟石嶺,賊遁入川,責宜綿嚴遏回竄。八月,徐天德、冉文儔、高均德由儀隴竄廣元,漢潮北入南江,欲還湖北,官軍蹙之上游不得渡。宜綿檄兵扼寧羌、沔縣,漢潮竄太平。於是川、楚匪多流入陜境,其魁樊人傑、龍紹周、李澍、阮正漋各擁眾數千,迭擾安康、平利、紫陽諸縣。

四年,漢潮竄五郎,詔斥宜綿畏葸避賊,命解任來京,在散秩大臣上行走。既至,復斥其辨飾,降三等侍衛,赴烏里雅蘇臺辦事。五年,追論軍需冒濫,褫職,遣戍伊犁,罰銀二萬兩助餉,逾兩年釋回。及三省教匪平,以員外郎用。後帝閱方略,宜綿曾論鄉勇,切中時弊,追念前勞,擢大理寺卿。病免。十七年,卒。

子瑚素通阿,初名瑚圖靈阿。乾隆五十二年進士。由刑部員外郎改翰林院侍講,累遷左副都御史。嘉慶初,疏陳關稅、鹽課積弊;又請卻貢獻,停捐納。居官有聲,擢盛京刑部侍郎。宜綿遣戍,瑚素通阿以父老請代行,未允。在盛京,劾將軍琳寧寬縱番役及私葠、官吏分肥事,侍郎寶源查辦不實,寶源、琳寧並黜罷。內調刑部侍郎,赴河南讞獄,漏洩密封,降筆帖式。後起用,終刑部侍郎。

英善,薩哈爾察氏,滿洲鑲黃旗人。由親軍補侍衛處筆帖式,累遷刑部郎中。改御史,除甘肅蘭州道,以親老留京職。乾隆五十年,出為直隸按察使,遷湖南布政使,調江蘇,丁母憂歸。命署廣西布政使,調補四川,五十六年,護理總督。尋擢貴州巡撫,調湖北,以治西藏軍需,未之任。嘉慶元年,調廣東。旋召授刑部侍郎,而四川教匪起,仍留攝總督。

初,四川自金川木果木之敗,逃兵與失業夫役、無賴游民散匿剽掠,號為啯匪。官捕急,則入白蓮教為逋逃藪。及湖北襄陽敗匪竄入川,一旦揭竿,戰鬥如素習。至是,達州奸民徐天德等激於胥役之虐,與太平、東鄉賊王三槐、冷天祿等並起。英善率兵五百馳剿,復調成都駐防兵,副都統勒禮善、佛住率以往,連破賊巢,擒賊目何三元等。賊竄橫山子,據險負嵎,遣總兵袁國璜、何元卿分路進攻,戰三日,國璜、元卿並歿於陣。尋克馬鞍山賊寨,擒賊首徐天富;而王三槐、徐天德等合陷東鄉,佛住戰死,賊熾兵單,詔責英善固守毋輕進,命宜綿赴達州督師。二年二月,宜綿至,英善連破賊於貫子山、羅江口,通周家河運路;偕宜綿克張家觀,復東鄉。五月,命赴甘肅攝總督。王三槐等由通江、巴州分犯保寧,英善赴廣元迎剿,偕總兵富爾賽、硃射鬥擊之於儀隴、閬中,多所斬獲。賊逼蒼溪,設伏敗之,遂遁。

三年,命與福寧赴達州治四川糧運。四年,調兵部侍郎,充駐藏大臣,調吏部,駐藏如故。五年,帝以教匪久未平,追論始事諸臣玩寇罪,褫職,以四品頂戴仍留駐藏。七年,召授頭等侍衛。擢刑部侍郎,遷左都御史,兼正黃旗漢軍都統。十一年,以駐藏時於福寧私挪庫款,徇隱未舉,降太常寺卿。十四年,卒。

福寧,伊爾根覺羅氏。初隸貝子永固包衣。由兵部筆帖式洊擢工部郎中。乾隆三十三年,出為甘肅平慶道,累遷陜西布政使。五十五年,擢湖北巡撫,抬入鑲藍旗滿洲。調山東,治衛河運務,稱旨。五十九年,漳、衛二河溢,疏消積水,撫恤災黎。曹、單漫水,下流為豐、碭壩堰所阻,馳往會勘,酌開壩堰以水曳水,並協機宜。調河南,尋擢湖廣總督,駐襄陽,捕治教匪,獲首逆宋之清等寘諸法。

六十年,調兩江。會黔苗石柳鄧勾結楚苗石三保焚掠辰州,命留湖北會剿,福寧至鎮筸防後路。嘉慶元年,湖北教匪攻來鳳甚急,福寧馳抵龍山,擊敗之。賊屯旗鼓寨,偕四川總督孫士毅合剿,士毅卒於軍,福寧代之。偕將軍觀成、總兵諸神保進攻,破其寨,擒賊首胡正中,餘眾窮促乞降,誘入龍山城,駢誅二千餘人,以臨陣殲戮奏,加太子少保。移軍剿林之華、覃加耀於長陽、巴東,賊竄黃柏山;偕觀成、惠齡會剿未下,惠齡赴襄陽,觀成入川。二年,命額勒登保移師黃柏山,福寧以兵隸之。地形天險,圍攻數月,賊竄鶴峰芭葉山,繼竄大拏口,又竄建始、宣恩;十一月,始殲之華於長陽,加耀遁歸州,以剿賊不力,奪宮銜。三年,擒加耀於終報寨,帝猶斥諸將遷延貽誤,福寧有地方之責,咎尤重,褫職,罰銀四萬兩充餉;予副都統銜,偕英善駐達州,治四川軍需。

四年,英善調駐西藏,福寧遂專任其事。時軍營支用冒濫,統兵大員奢糜無度,兵勇口糧反多遲延,幾致枵腹,四川餉數更多於湖北數倍,屢詔訓戒,福寧不能綜覈,以奏報浮泛被詰。又奏賊數有增無減,勒保疏辨;命魁倫赴達州察視,覆陳賊數實減,而大股分為小股,賊名反多,得福寧理餉含混狀,詔褫副都統銜,留達州候命。尋以旗鼓寨殺降事覺,帝方以剿撫責諸路,而川賊高均德被擒,言賊黨恐投降仍遭誅戮,故多觀望。詔斥福寧此舉失人心而傷天理,逮治論罪,遣戍新疆,尋原之,命赴額勒登保軍前效力。會賊竄渡嘉陵江,由於福寧裁撤鄉勇所致,仍戍伊犁。五年,予三等侍衛,赴西藏辦事。九年,召還,授正白旗蒙古都統。十一年,以三品銜休致。十九年,追論在西藏擅借庫帑,及湖廣任內濫用軍需,久不完繳,下獄。尋卒。

景安,鈕祜祿氏,滿洲鑲紅旗人,和珅族孫也。由官學生授內閣中書,水存擢戶部郎中。出為山西河東道,累遷甘肅、河南按察使,河南、山西、甘肅布政使。乾隆五十六年,徵廓爾喀,命治西寧至藏臺站,留藏督餉運。事平,以親老歸。未幾,擢工部侍郎,歷倉場、戶部。六十年,授河南巡撫。

嘉慶元年,湖北教匪北犯,景安駐軍南陽,以籌濟恆瑞軍餉,加太子少保。十二月,姚之富犯鄧州,圍景安於魏家集,恆瑞援至始解。二年,淅川教匪王佐臣謀應賊,布政使完顏岱捕斬之。景安欲攘功,躡兵戮難民,以捷聞,賞雙眼花翎,封三等伯。時襄陽賊屢為惠齡、慶成等所破,窺北面可乘,遂分三路犯河南:王廷詔出北路,竄葉縣,焚保安驛,圍官軍於裕州,總兵王文雄兵至,乃引去,景安尾追至南召,聞桐柏有警,馳回防禦;李全出西路,竄信陽、確山,羅山、淅川,趨盧氏,出武關,慶成追之;姚之富、齊王氏出中路,竄南陽,掠嵩縣、山陽,惠齡追之。賊入河南後,虜脅日眾,不迎戰,不走平原,忽合忽分,以牽兵勢,先後並入陜西復合。景安頓兵內鄉,賊入陜後二十餘日,始追至盧氏,賊尤輕之,號為「迎送伯」。三年春,擢湖廣總督。四月,率師次荊門州,劉成棟來犯,與布政使高杞分路擊走之。六月,賊由竹谿竄入陜,詔切責。四年,張漢潮擾陜西五郎、洋縣,景安屯鄖陽,遣總兵王凱扼鄖西。漢潮已分路自安康折竄鎮安,景安疏稱赴鄖西迎剿,詔斥其不實。時仁宗初親政,以景安堵剿不力,撫治失當,解職,命治四川軍需。尋奪伯爵,戍伊犁。

是年冬,帝召見惠齡,論其恇怯縱寇及淅川冒功事,逮京讞,擬大闢,緩刑,禁錮。七年,教匪平,得釋,發熱河充披甲。逾年,宥還,以六部筆帖式用,效力河南河工。衡家樓工竣,晉秩員外郎,授直隸承德知府。擢山西按察使、陜西布政使。十一年,授江西巡撫,調湖南。召為內閣學士,累遷戶部尚書,加太子少保。二十五年,授領侍衛內大臣,守護昌陵。道光二年,休致。尋卒。

景安初附和珅,懵於軍事,然居官廉。當其逮京,值硃珪入見,帝曰:「景安至矣!軍事久不定,欲去一人以警眾,如何?」珪曰:「臣聞景安不要錢。」帝曰:「若乃知操守耶?」竟以是獲免。後復用之。

秦承恩,字芝軒,江蘇江寧人。乾隆二十六年進士,選庶吉士,授編修,擢侍講。出為江西廣饒九南道,累遷直隸布政使。五十四年,擢陜西巡撫。

嘉慶元年,教匪起荊、襄,承恩率師赴興安籌防。至冬,四川達州教匪自太平入陜犯興安,承恩偕總督宜綿迭擊敗之。十二月,會剿洞河、汝河諸賊。二年正月,擊安康賊於光頭山,首逆王劉氏伏誅,陜境略平。宜綿進剿川匪,承恩專任陜防。三月,襄匪由河南盧氏竄商南,勾結陜匪,紛起應之。承恩移軍商州,偕恆瑞殲山陽西牛槽賊。雒南石板溝奸民起,總兵富爾賽捕斬之。姚之富由商州犯孝義,窺西安,承恩扼之於秦嶺。惠齡等追擊,賊走鎮安,與李全、王廷詔合掠洵陽、安康。時陜西兵力僅有鄉勇萬餘人,提督柯籓守興安府城,兵止二百,無力攻剿。惠齡、恆瑞合擊賊於黃龍鋪,賊分竄復合,六月,由漢陰至紫陽渡漢江。詔斥承恩疏防,奪翎頂。賊走漢南,與川匪合,八月,復入陜,竄白河石槽溝。承恩率鄉勇扼安康要隘,賊分路來犯,御之於平利金堂寺。既而賊逼興安,偕惠齡擊走之,以功復翎頂。

三年春,丁母憂,軍事方亟,奪情視事。二月,高均德、齊王氏合竄漢陰觀音河,糾李全,王廷詔分道由城固、南鄭北出寶雞,合攻郿縣,掠盩厔,將犯西安,承恩恇懼,率師回防。總兵王文雄力戰,敗賊於焦家鎮、圪子村,大創之,賊復分竄。三月,文雄復破李全餘眾於翔峪、澧峪。四月,李全糾阮正通折回鎮安,西擾漢陰、石泉,高均德逾秦嶺走老林,承恩與文雄扼子午峪。既而均德、全與張天倫合為一路,正通由石泉、洋縣西竄,均德等尋竄入川。承恩進兵漢中。八月,川匪徐天德、冉文儔、樊人傑,襄匪張漢潮先後並入陜境。

承恩師久無功,四年,命解職回籍守制。會剿張漢潮於鳳翔,承恩遣游擊蘇維龍扼東路,戰失利,漢潮突圍遁;褫承恩職,逮京論大闢。詔以承恩書生,未嫺軍事,宥歸。尋遣戍伊犁,七年,釋還。起主事,纂修會典。出為直隸通永道,擢江西巡撫,遷左都御史,仍署巡撫事。十一年,召授工部尚書,調刑部,署直隸總督。十三年,以治宗室敏學獄瞻徇,降編修,效力文穎館。遷司經局洗馬,晉秩三品卿。十四年,卒。

論曰:方教匪之初起也,苗疆軍事未蕆,楚、蜀空虛,草澤麼,燎原莫制。永保、惠齡號曰總統,局於襄陽一隅。景安,秦承恩不諳軍旅,賊遂蹈瑕,蔓延豫、陜。宜綿受事,僅顧蜀疆,及勁兵移陜,束手求退矣。英善、福寧並皆庸材,三年之中,防剿無要領,如治絲而益紛。仁宗親政,赫然震怒,諸臣相繼罷譴,士氣一新,事機乃轉。廟堂戰勝,固有其本哉!


\end{pinyinscope}