\article{志一百三十一}

\begin{pinyinscope}
○邦交四

△美利堅

美利堅在亞美利加洲。初來華,貨船常至粵東。道光二十一年,英因鴉片之役,詔停貿易,美為英人請準貨船入口,不許。二十二年,與英和,許寧波互巿。美商船由定海駛至寧波,請報稅通商,浙撫劉韻珂以聞。朝旨以美通商向在粵東,不許。已,復請增商埠,將軍伊里布以聞,許之,命與英並議稅則。明年三月,美商船駛至上海求通商,拒以稅則未定。既聞英通商章程已議定,復請援英例開巿;又稱進口洋參、鉛斤二項稅則繁重,請減輕,以百斤取五為率。江督耆英等以洋參、鉛斤歲來無多,允酌改。美人福士又請入覲,不許。冬十月,福士忽稱有使臣顧盛來粵,仍求覲見,並遞國書,欲與中國商議定約,並稱沒蘭的彎兵船欲赴天津。諭令折回,不省。二十四年四月,美兵船進黃浦,阻之,答以進口專為約束商民,防範海盜,無他意。又責中國款待,要求甚堅者十款。耆英等屢與駮詰。於是酌定條款:如商船納鈔已畢,因貨未全銷,改往別口轉售,免重徵;又商船進口,並未開艙即欲他往,限二日出口,不徵稅鈔;又商船進口,納清稅餉,欲將已卸之貨運往別口售賣,免重納稅鈔;此外又許其於貿易港口租地建禮拜堂及殯葬處所;又許延請中國士人教習方言、佐理筆墨,及採買中國各項書籍。又增入商人擅赴五口外私行交易、及走私漏稅、攜帶鴉片及違禁貨物,聽中國官自行辦理治罪一款。遂定議。尋進國書,耆英請賜詔書褒美,許之。

二十六年,諭通商、傳教祗許在五口,不得羈留別地。緣美人在定海傳教非條約所許故也。十一月,美使義華業來粵呈遞國書,初欲入覲面呈,耆英等以條約折之,乃已。咸豐三年七月,美酋馬沙利來粵接辦本國公使事務,齎有國書,仍欲進京投遞。中國持定約不許。時賊氛未靖,美兵船忽至滬,揚言往鎮江等處察看賊情,並整頓海口商務,如督撫不與會晤,當繕奏齎往天津投遞。蘇撫許乃釗以聞。命赴粵聽欽差大臣察辦。同時美兵船又入琉球,琉球王世子咨閩浙總督王懿德,懿德以聞。命粵督葉名琛曉諭,使撤回兵船。四年六月,美人麥蓮至上海,要求赴揚子江一帶貿易,請代奏。江督怡良諭令回粵,候葉名琛察辦。麥蓮返粵,名琛不予接見,乃復回上海,與英、法人往見蘇撫吉爾杭阿,要求赴天津變通成約。吉爾杭阿拒之,不聽。既而船至天津,命長蘆鹽政文謙等復阻之。仍以進京求覲為詞,遞清摺要求十一款,駮之。惟華洋訴訟、豁免積欠及廣東茶稅每擔加抽二錢,允與商辦。麥蓮等遂去。

六年,美人伯駕來粵請換約。時英人包令、法人顧思同至,亦請換約,與伯駕同赴天津。朝命葉名琛阻之。旋駛至福建遞國書,要求公使駐京、中國遣大臣駐美京華盛頓。朝命閩浙總督王懿德約回廣東,嚴詞駮之,伯駕不省。八月,偕本國水師提督奄師大郎乘火輪至上海,云奉國主命,必須入京覲見,屢諭不從。是年減免美在滬未繳關稅,因粵賊滋擾,美商受損失故也。七年十月,美遣新公使列衛廉來粵代伯駕,會英人虜葉名琛,省城被據,美人來滬投遞牒大學士裕誠文,原勸和。裕誠覆以已命黃宗漢赴廣辦理外國事務,可速赴廣東會晤。八年二月,美隨英、法調兵船來津,命直隸總督譚廷襄等接晤。美使與俄使普提雅廷同見廷襄,欲變通舊約,未允。五月,命大學士桂良、吏部尚書花沙納為欽差大臣,與美使列衛廉定約。初,美條款要求添商埠、保教民、立塔表、鑄銀元、賠損失、防凌害、船隻駛揚子江及粵東珠江並各支流、文移達內閣、使臣駐北京、丈量船身計噸納鈔法、以各用法律治本國人民、特援最惠國利益均霑之例載入約中,迄未行。至是,復請。

冬十月,定通商稅則,桂良致書美與英、法使臣議通商善後事,極陳領事之弊。美列衛廉覆書,略謂:「美國商民進內地,按天津條約,利益均霑,是則美進內地所有請執照等情,應同英、法一例。俟國主及國會議允批準和約後,必明立律例交領事,禁止不請執照或強請執照等事,致免國民違犯中國憲典。又整理有約、無約各國之法,本大臣向知此事應變通,今請將中國所能行者略為陳列。按泰西各國公使,凡此國領事奉遣至別國者,若不得所往之國準信延接,即不得赴任。今凡有稱領事,而中華國家或省憲地方官不肯明作準信延接者,彼即無權辦事,是則中國於此等兼攝領事即可推辭不接,已延接者亦可聲明不與交往。設有美國人兼攝無約領事,藉作護身符以圖己益者,地方官可卻不與延款,遇有事故,令彼投明美國領事,自應隨時辦理。間或美國人兼攝領事,而代無約商民討求地方官協助申理,地方礙情代為辦理者,亦可對彼說明,並非職守當然,祗由於情面而已。又若此等自稱領事,有與海關辦理船隻餉項事宜者,地方官可卻以必須按照條約遵行。倘彼固執己見干犯則例者,中國地方官應用強禁阻。前在天津時,本大臣照會桂中堂、花塚宰,以中國必須購造外國戰艦火輪船者,特為此故,足徵所言非謬也。又領事不得干預貿易,現美國定制,凡干涉賣買者,不得派作領事官。又領事與地方官爭論,前此動多牴牾,本大臣深為恨憤,業經設法將一切事宜妥為辨正。嗣後果有仍前事款,請照知本大臣,定當修正。若領事官不合之處,地方官按理據實,直斥其非,不與共事,此最善之法也。總領事之設,美國奉使駐扎中華者,從無此制,領事官亦無發給旗號之事。本大臣復嚴諭領事,嗣後不得有此。以上據問直達。猶有管見須照知者,中國宜立國家旗號,俾中國公私船盡行升用。蓋美國制度,凡本國人必用本國旗號,泰西各國莫不皆然。今中華貿易之盛,而無旗號以保護,何不亦仿他國之法,使商船與盜賊有所區別,而免商民之借用與假冒外國旗號哉?」桂良據奏。厥後中國造輪船、購戰艦、用龍旗,多採其議。

九年夏五月,美使華若翰遵滬約,改道北塘呈遞國書,諭旨嘉獎。七月換約,還所擄前附和英人之蔣什坡。美使回滬,請照新章完納船鈔,及在潮州、臺灣先行開巿。欽差大臣兩江總督何桂清以前大學士桂良等給與照會,言明各口通商,俟英、法條約議定,再照新章辦理,不服。乃允先開潮州、臺灣兩口巿,及照新章納船鈔,餘仍從緩。十年,美船隨英法聯軍北駛。是年美國書及原本條約、稅則遺失,特命蘇撫薛煥先與說明,照俄國一律,以通行刊本為憑,美人許諾。

十一年四月,始至漢口通商。旋立九江巿埠。先是三月,美水師總領施碟烈倫以火輪船至九江,尋去。至是,美商擇地,勘定九江城西琵琶亭空地三十畝,以地勢低窪,興工建築,居民以未給價,阻之。領事別列子始赴道署,許照英國價例給發。九江關監督以此地在大街繁盛之區,與龍開河偏僻有水者不同,駮詰之,別列子去。監督因牒駐漢口總領事,始許依民間賣買,又增索至五十畝。是為美立九江巿埠之始。秋七月,美設領事於漢陽,並代理俄國漢口通商事務。又為美人在漢設領事之始。

同治元年,粵賊陷蘇、太各城,上海為各國通商之地,蘇松太道吳煦招募壯勇,雇洋人領隊。有美人華爾者,煦令管帶印度兵。既印度兵遣撤,煦令華爾管帶常勝軍,協守松江,屢出討賊有功,奏給翎頂。又白齊文者,亦美人,因華爾進,命並在松江教習兵勇,協同官軍剿賊,屢立功。華爾旋攻慈谿陣亡。秋七月,美伯理璽天德林肯亞伯剌罕遣使蒲玲堪安臣致皇帝書。二年,白齊文不遵調遣,毆傷道員楊坊,並劫餉四萬餘元。事聞,褫白齊文職,命蘇撫李鴻章拿辦。白齊文匿英兵艦,美使蒲安臣以白齊文為美國人,覆牒為代辨無罪。總署以白齊文受中國官職,應照中國法律懲辦。辨駮久之,美使始代白齊文認罪。白齊文尋投賊被獲,牒美使衛廉士述其罪狀,請照前議亟予正法。美使覆以請示本國,白齊文尋溺死。

六年十月,以美卸任使臣蒲安臣權充辦理中外交涉事務使臣。時外洋諸國公使、領事等先後來華,於是特派蒲安臣,以英人柏卓安、法人德善為左右,協理志剛、孫家穀二員同往會辦。緣蒲安臣充美公使最久,中外交涉,總署深相倚任,故特派往。特與議定條款,凡事須咨總署覈定,準駮試辦,以一年為期。又以中外儀節不同,呈遞國書,須存國體。又慮各國因蒲安臣系西人,以西例優待,當告以中國體制,使各國了解,不致疑中國將來無報施之禮。迭咨蒲安臣,蒲安臣遂西。

是年,美羅妹商船至臺灣之瑯軿洋面,遭風船破,被生番戕害。又前有美商船羅發遭風飄至臺灣極南海島,亦被害。至是,美住廈門領事李讓禮欲坐兵船赴臺住泊。八月到瑯軿,會臺灣鎮總兵劉明燈究詰此案,而龜仔角生番糾集十七番社謀抗拒,劉明燈招番目卓杞篤往諭,始知五十年前,龜仔角一社之番,悉被洋人殺害,僅存樵者二人,以致世世挾仇圖報。因諭番人解散,勸李讓禮無深究,免再結仇。李讓禮許諾,遂議結。既而李讓禮請在象鼻山設立砲臺,未允。

七年春二月,美使來言,前年九月有本國商船兩只在高麗擱淺被害,尚餘四人,請轉知高麗,設法救護。政府請高麗自行查明酌覈。六月,美人派兵船入高麗,國王李熙奏聞。中國查明並無羈留美人情事,函致美使代為解釋。美使乃無言,其兵船亦啟椗去。

是月,蒲安臣等至美遞國書,並增定條約,其要目有八:一,美國與他國失和,不得在中國洋面奪貨劫人;二,除原定貿易章程外,與美商另開貿易之路,皆由中國作主;三,中國派領事駐美通商各口;四,中、美奉教各異,兩國不得稍有屈抑;五,兩國人民互相往來游歷,不得用法勉強招致;六,兩國人民互相居住,照相待最優之國利益均霑;七,兩國人民往來游學,照最優之國優待,並指定外國所居之地,互設學堂;八,美國聲明並無干預中國內治之權。其時曾國籓等鑒於道、咸間條約失利,特建議遣使往訂此約,於領海申明公法,於租界爭管理權,於出洋華工謀保護,且預防干涉內治雲。九月,美使勞文羅斯來華遞國書,並呈書籍及五穀各種,請換中國書籍、穀種,許之。

九年三月,美遣鏤斐迪充出使中國大臣,遞國書,前使勞文羅斯回國。四月,中國出使大臣蒲安臣在俄病卒,特予一品銜,給恤銀萬兩。

十年正月,美致朝鮮函,請中國代達,謂將以兵船前往商辦事務。中政府以權宜許為轉達。旋接朝鮮咨,謂美使所投封函,專為曩年美商船來韓,一遭風遇救,一人沒貨無,以為一救一害,相懸太甚,欲請究治。朝鮮以己國無殘害美船之事,不允所請,並請中國降旨開諭美使。美使以降旨開諭,是以屬國相待,不受。乃以兵船抵朝鮮脅之。朝鮮人不服,與力爭,並報中國牒美使解之。十二月,美請援例開瓊州商埠。

十一年春二月,許美國領事官代辦瑞士國商務。瑞士國一名蘇益薩,又稱綏沙蘭,其商船至中國,向以無約小國不設領事官,至是請美領事官代辦商務。美使牒稱遂次蘭國,總署覆美使,以瑞士事務祗可照料,不能兼攝,至通商納稅等事,仍照向來無約各國祗許在海口通商,其內地口岸及內地游歷設局招工等事,均不得一律均霑。美使照覆更正遂次蘭為瑞士。美領事雖得照料瑞士國商務,不得稱瑞士國領事官。十二年春,穆宗親政,美隨英、法、俄、德請覲見。十三年,美使鏤斐迪回國,以艾忭敏為駐華全權大臣,覲見面遞國書。

光緒二年十一月,美旗昌公司歸並中國招商局,南洋大臣沈葆楨奏請給價銀二百二十萬兩,報可。四年,出使大臣陳蘭彬等蒞美呈遞國書,旋請設領事,言華人僑美各邦約二十餘萬,不設領事,無以保護華民。奏入,許之。五年,美前統領格蘭忒來華。值日本滅琉球,政府因格蘭忒將游日本,託其轉圜。格蘭忒至日本,函勸中國與日本各設領事,保護琉球中部,其南部近臺灣,為中國屬地,割隸中國,北部近薩摩島,為日本屬地,割隸日本。兩國均不允。又請派員會議,卒不得要領。

六年七月,美遣使臣安吉立及修約使臣帥腓德、笛銳克來華,請與中國大臣議事,總署以聞。並言:「同治七年中國與美續增條約,其第五款內有『兩國人民任便往來得以自由』等語。近來金山土人深嫉華人奪其工作,不能相容,上年美議院曾有限制華人之議,經其總統據約批駮。去年彼國開議,又欲苛待華人,經副使臣容閎牒外部,言與約不符,始將此例停止。是華人在彼得有保護者,惟恃續增條約之力居多。今遣使來華,恐有刪改續增條約之意,請派員商議。」奏入,命總署大臣寶鋆、李鴻藻為全權大臣,與美使議約。初,美續約第五款祗言兩國人民往來及游歷貿易久居等人,無「華工」字樣。至是,美使安吉立等遞修約節略,內稱華工分住各口不下十萬人,於本國平安有損,請整理限制禁止。總署以禁止一層與舊約不符,惟限制一層尚可酌擬章程。安吉立等以章程須由本國議院酌定,此次來華,祗求中國一言,許其自行定限。總署遂入奏,與安吉立等議定四款:凡傳教、學習、貿易、游歷人等仍往來自由,其已在美華工亦仍舊保護,惟續往承工之人,定人數年數限制,不得凌虐。遂畫押蓋印,期一年兩國御筆批準互換。既而美金山於中國招商局和眾輪船進口有額外加徵船鈔貨稅之事。出使美國大臣陳蘭彬等請乘美派人來華議約之際與交涉。時美使安吉立亦牒總署,詢中國徵收美國各船稅鈔與徵收中國及別國船稅鈔是否相同,又中國在常關納稅鈔之船是否均與新關納稅鈔之船相同各等語。又欲將兩國商民貿易有益之事,及兩國商民爭訟申明觀審辦法,加入約款。總署以商民貿易一款,原可隨時商辦,觀審一款,本煙臺條約所載,此次申明與原議亦無出入。因與定議,仍候兩國御筆批準互換。明年六月鈐印。

八年三月,美欲與朝鮮結約通商,遣總兵蕭孚爾為全權大臣,乘兵船往議約。朝鮮遣余允植赴保定謁見李鴻章,請代為主持,與美使商議。美使旋出所擬約稿,其約稿未提明朝鮮為中國屬邦。鴻章請刪改,蕭孚爾執不允。會美署使何天爵在京,與總署議,允增「屬邦」字樣,而內治外交仍許朝鮮自主。

九年,出使美國大臣鄭藻如請於美紐約設領事官,略言:「美國西通太平洋,以金山埠為首站,東通大西洋,以紐約埠為首站,兩埠為往來必經之路。金山業設領事。近紐約華民往者日見增多,土人不無嫉忌。兼以古巴一島與紐約水路相通,華民由古巴回籍者必假道紐約,實為通行要路。請仿金山例設領事以資保護。」報可。是年美與朝鮮換約,遣使駐朝鮮漢城,朝鮮遣使報之,仍咨中國,禮部僅報聞而已。十年,中、法因越南啟釁,招商局輪船商人籌照西國通例,暫售與美國旗昌洋商保管,旋事定,仍收回。

十二年春,美舊金山華民被美西人虐害,中國索賠,總統卻之。粵人聞之,大憤,爭欲起抗。粵督張之洞恐其滋事,一面曉諭粵民,一面致總署及駐美使臣與美交涉,請其賠償懲辦,因疏言:「出洋粵民所訴焚劫殺逐,種種遭害,臚列各案內,如光緒十年十二月,夭李架埠一案,焚鋪逐商,劫財七萬餘元;十年七月二十五日,洛巿丙冷埠一案,慘殺廖臣頌等二十八命,傷十五人,焚毀鋪屋財物值十四萬餘元;七月二十八日,舍路埠一案,慘殺莫月英等三命,焚燒煤廠,約值數萬,旋將華人盡逐;八月十一日,倒路粉坑一案,枉殺李駒南等五命;九月二十八日,喊罷埠一案,焚逐失財數萬;十二月初四日,尾矢近地一案,慘殺伍厚德等二命:皆為無辜被害。其餘密謀殺害,不可勝紀。以致卓忌埠、禮靜埠則有被逐之事,興當埠、拓市埠、喜路卜埠、鈴近埠、匿架巿埠、灑巿埠、缽倫埠、雲乃埠、坎下埠、古魯姐埠、粒卜綠埠亦皆有定期議逐之事。其金山大埠,華民住房則有十苦之訴,洗衣裳館則有六不近情之訴,統大小各埠工商人等則有七難之訴。所謂十苦者:金山大埠住房,每人限地八尺,不足八尺者查拏監禁,謂之祼房。祼房之苦,計地少絀,同居概捉。一也。監後寓財,盡竊無追。二也。回華有期,暫寓被禁。三也。到埠資乏,借寓亦拏。四也。畏捉夜行,臥街被打。五也。工藝出監,無處傭食。六也。監房地狹,疾癘益增。七也,入監勒銀,始任贖出。八也。監鬱鬢亂,被翦違制。九也。昏夜巡查,破窗越屋。十也。所謂六不近情者:洗衣館八九百間,木樓木屋,歷數十年,乃借防火私擅,勒令改建磚樓鐵門,既非美廷所命,別處又不一律。一也。拆改不獨勞費,工眾無處容身。二也。磚鐵本重租貴,主客兩受其害。三也。曬棚謬謂惹火,別處樓棚更多。四也。任意拏人罰銀,被擾至數百間。五也。洋館木樓曬棚,何以不用此律?六也。所謂七難者:一為欲守業之難,二為欲拒匪之難,三為求保護之難,四為居散埠之難,五為居大埠之難,六為業工者之難,七為業商者之難,等語。又言金山各埠,始則利華民之工勤價省,多方招徠開礦修路諸工,美商藉華工以獲利者,不知其幾千億萬。乃因埃利士黨人嫉石把持,合謀驅逐,殘毒焚掠,以奪其資財,勒逼行主辭用華工,以斷其生路。華工既無生計,華商亦遂賠折窮蹙,留不能留,歸不能歸,保護亦無從保護,情形實為危慘。假如將此十餘萬華民盡行驅歸中國,沿海各省何處容之?既屬可憫,亦多隱憂。此外南洋諸埠,設皆踵事效尤,何堪設想?美與中國雖無嫌隙,但此事系由美境土人專利而起,其視華工究不免稍分畛域。且美國官員,近亦多有埃利士黨人在內,多設苛政,實有此情。應請敕催美國嚴懲速辦。」初,沙面燒洋房十四間,償款至鉅。至是,出使美國大臣鄭藻如電張之洞,請查案援例。之洞以金山殺掠重情,過之十倍,應照本案華民所失之數賠足,並須財命兩究,電覆令與交涉。先是美使田貝允電本國速辦。時新任張廕桓為美使,仍留鄭藻如會同經理。既而美調兵緝匪,斃匪一名,傷數名,美總統及議院亦漸議護禁,久之始允賠。

尋議寓美華工約,定約六款:首言中國以華工在美受虐,申明續約禁止華工赴美;次言華工在美有眷屬財產者,仍準往來;三言華工以外,諸華人不在限禁之例,並準假道美境;四言華人在美,除不入美籍外,美國仍照約盡力保護;五言華工人被害各案,美國一律清償;六言此約定期二十年互換。議定畫押,復命張廕桓再與籌議。廕桓以三端要美:一,請酌減年限;二,請訂約以前回華之工,如有眷產,亦可稟報中國領事,補給憑批回美;三,回華工人在美財產不及千元者,作何辦法,亦應商及。議久不決。

十四年四月,廣西桂平縣美教士富利淳醫館被毀,領事索賠五千餘元,拒之。時粵民憤華工見拒,群起抵制,且歸咎張廕桓。會命翰林院侍講崔國因代為美日祕國出使大臣。十六年,國因到美,美戶部忽訂新例,於假道華民入境,索質銀二百元,出境發還。下議院又議立限清查寓美華民戶口給照。國因力與辯,例旋廢。初,金山新例,拘執華人令徙遷者限地界,以華工居處不潔釀疾為言,至是始廢例銷案。時換約期將屆,適楊儒出使,總署又以商改新例事委之。儒涖美,值美迫行華工註冊新例,當援條約駮詰。美外部始商允議院展限半年,被拘工人釋放,而於註冊之例堅不改移。華工以例專分別新舊工人,舊工固有安居樂業之便,而新工因限禁,不能到美,屢倩律師控訴察院,欲除此例。美外部以例經議院議定,不能廢,仍限華人註冊。而總署電儒,以先修約、後註冊為關鍵。儒當牒外部,並就十四年約稿刪去賠償一款,易為互交罪犯;原約二十年之期改為十年。旋又接總署電,言美必欲先行註冊,擬令寓華美民亦註冊以相抵制,屢議不決。既美外部謂交犯一款,與限禁華工保護華民不相涉,應另訂專約,不列款內;十年之期,可以允從,寓華美工,亦聽中國註冊。楊儒力爭寓華之美國教士亦須註冊。遂擬除工人外,寓華別項美民,自換約日起,美政府允每年造冊一次,報知中國政府。乃定議,並於第五款中寓華別項美民下,註包括教士在內。二十年二月,畫押蓋印,是為重訂限禁華工保護華民約款。又立互交罪犯約。

約既成,楊儒復籌寓美華民善後事宜,因上言:「華工在美,始自咸豐年間。光緒六年,始有限制工人之約。華人寓美,洋人指為風俗之害者,約有三端:一曰鴉片,一曰賭博,一曰械斗。今惟有將此諸弊力圖革除。一在申明律例,治以各項應得之罪,中國不為袒庇;一在詳示教條,使知目前限制之故,皆與煙賭械斗各弊有涉。俾各愧奮改圖,庶不至為人厭薄,此治本之法也。至於治標之法,一在嚴禁冒商,俾真商不至受累;一在疏通工路,使新來之工得以謀生海外。如此,不獨華民生計可紓,即中外邦交,從此愈固矣。」是年,中、日啟釁,美代中國保護在日本華商。明年,四川、福建教案相繼起,而古田案尤劇。美與英、法均請中國償款辦犯,議久不決。既而美使田貝函總署,稱有各國耶穌教人公舉在華辦理教務教士李提摩太惠志,繕冊摺擬呈查閱,請謁見,允之。

二十三年,美人在上海侵占租界外地。初,美所租同治初年止九百餘畝,后美領事西華自畫界,圈入未租民地萬餘畝。光緒十九年十月,兩江總督劉坤一飭將界線內東北未租地收回二千六百畝,而於西北界外所占之地未及清釐。至是,美領事在蘇州河邊自立界石,而河內地起建樓房。署兩江總督張之洞請與英、法界外侵占同嚴禁,疏入,交議。

二十四年,出使大臣伍廷芳見德與中國因膠州失和,請聯美,略謂:「美合眾為國,其保邦制治,國律以兼並他洲土地為戒。溯自海上用兵以來,美兵船皆由英軍牽率而至。道光二十一年,粵東議款,美實居間排解,遂得定盟。咸豐九年,英、法闌入大沽,毀我防具,美守前約,船由北塘駛入,呈遞國書,情詞謙遜,先換約而歸。是通商以來,美視諸國最為恭順。此次守約惟謹,不肯附和。雖因古巴議自主,檀島議兼隸,近在同洲,大局未定,不遑遠略,亦因與我交誼素篤,故不從合從之謀。若能聯絡邦交,深相結納,似與大局不無裨益。」又因檀香山歸並於美,請設領事,保護華民,略謂:「檀香山居太平洋之沖,前本君主,後改民主。近因弱小,求庇美邦,設為行省,美議院業經議行。此島華民不下三萬人,向由商董立中華會館,排難解紛。光緒七年,曾令商董陳國萬為領事。後美禁華工抵埠,華民出洋,皆趨檀島,請設領事。」報可。

是年中國議修盧漢、粵漢、寧滬、寧漢四路,借款各國,美國原貸四百萬鎊於粵漢路,旋聘美工師勘路。二十六年,拳匪作亂,各國聯軍入京,既各國會議條款,美惟增教案、被議人員不準復用之條,餘未與附和。會俄與中國訂退還東三省約,中國復請美政府排解。明年,和議成,議償款四百五十兆,美所分得償金三十二兆九十三萬有奇,合美金二十四兆四十四萬餘元。除給商人損失及海陸軍費外,尚有溢出數十二兆七十餘萬元。美總統羅斯福向議院提議,溢出金仍還中國,助中國教育,即以此款為格致學生留美之用。議行牒中國,中國特遣專使唐紹儀赴美申謝。既而各國賠款欲改銀為金,以金價算。美為商勸各國,並謂眾議合索四百五十兆兩,由各國自行均派,中國不管其易作何項金錢,是此項賠款,照約載金價核算,即四百五十兆海關銀數,照約銀數付還,亦即與用金付給無異。美旋允照約還銀。

二十八年春三月,議各國商約,美使不原加稅至十五,免釐與否,聽中國自便。是年,命呂海寰、盛宣懷議美約,與美使迭次磋商,張之洞、劉坤一通電參酌,始定議。因上言釐定約款十七條,大致與英約相同,而其中得失損益,稍有區別。第一款曰駐使體制。美使原送約文,聲明駐使可以行文各省將軍、督撫、駐扎大臣;駮以美國向由外部轉行,中國亦系由外務部咨轉,不能兩歧,駮令刪去,改為中國駐使為美國優待,是以美使駐京,中國亦一律優待,以昭平允。第二款曰領事權限。報施一如駐使,而聲明美國領事按例妥派,外務部按照公例認許,如所派不妥,或與公例不合,我即可不認,冀以挽回主權。第三款曰口岸利益。此系查照日本舊約,不能不許,因即比照日約核改妥協。第四款曰加稅免釐。此為全約主腦,美使初祗允加至值百抽十,並請我裁內地常關,又不提明銷場出廠等稅,以為中國主權所系,不欲有所干礙,屢費磋商,動至決裂。臣等往復電酌,彼始允加至十二五,其所裁內地常關之稅,任我改抽出產稅以為抵補。竊思內地常關不過十餘處,各省土貨未必悉所經由。按照英約載明進出口貨加稅後,均得全免重徵,則內地常關亦祗能徵土貨運出第一道之二五半稅。若非第一常關,則並無稅可收。至土貨未經第一常關徵過二五半稅者,出口時仍須徵足七五之數,是常關雖裁,亦無大礙。今既任我改抽出產稅,則從源頭處抽收,較無遺漏,似更合算。當時尚以與英約兩歧為慮,美使自認將來勸英照辦,祗得允裁。至於銷場稅、出廠稅及議增之出產稅,美使雖不原詳載名目,而於專條中聲敘本款所載各節,毫無干礙中國主權徵抽他等稅項之意,以渾括銷場等稅,保我主權。第五款曰稅則附表。彼請美國人在中國輸納稅項,較最優待之國,不得加重另徵。臣等索其增入中國人民在美國納稅亦如之一節。第六款曰準設關棧。系照英約酌辦。第七款曰振興礦務。前半悉照英約,彼請準美國人遵章開辦礦務。此本路礦衙門定章所許,因訂明美國人民辦理礦務居住之事,應彼此會定章程,以資鈐束。第八款曰存票抵稅。第九款曰保護商標。均與英約意義相等,而於存票款中聲明除去船鈔一項,以補英約所未及。第十款曰創制專照。此款深慮有礙中國工藝仿造,駮論再三,改為俟中國設立專管衙門,定有創制專律後,再予保護,其權仍自我操。第十一款曰保護版權。即中國書籍翻刻必究之意。與之訂明,若系美文由中國自繙華文,可聽刊印售賣;並中、美人民所著書籍報紙等件,有礙中國治安者,應各按律例懲辦,為杜漸防微之計。第十二款曰內港行輪。前兩節照英約大意,聲明嗣後無論何時修改,應由我查看酌辦;末節如奉天府安東縣開埠事,扼定自開,而辦法略有變通。第十三款曰改定國幣。將英約所附照會納稅仍照關平一節,增入款末。第十四款曰輯睦民教。教民犯法,不得因入教免究,並應遵納例定捐稅;教士不得干預中國官員治理華民之權,詳晰列明,冀資補救。第十五款曰治外法權。第十六款曰禁止嗎啡鴉片。皆我索其增添,與英約一律。第十七款曰修約換約期限。系照立約通例。復於約款之外,另行訂附件三端:一為內地徵抽鴉片、鹽斤稅捐之事,及保全稅捐防範走漏之法,均任由中國政府自行辦理;二為所留通商口岸之常關,設立分關,保持稅餉;三為申明第五款所載稅則附表,即前定切實值百抽五之稅則,至內地常關雖裁,並不藉此以裁北京崇文門並各城門及左右翼等處之稅,由美使備一照會存案。又第四款不礙徵抽他等稅項一語,尚涉籠統,由我備一照會,聲明他等稅項,即系包括銷場、出廠及改抽之出產各稅,應仍聽中國自行辦理。彼亦復一照會,言明彼此意見相同,分別簽押蓋印。是為中美商約,一名通商行船條約。

三十年春,美公司背約私售粵漢股票於比利時,允比在湘造湘陰過常德至辰州一路。張之洞致湖南巡撫趙爾巽請力阻,並援合同第十七條專認美公司,不得轉與他國人為主旨。湘人議自承辦,稟請廢約,趙爾巽力主之。時張之洞已奉廷寄廢約,遂以三省紳民力持廢約電致盛宣懷。宣懷旋電出使大臣梁誠牒美外部,略謂:「美公司顯背合同,必應作廢。續約十七款不得轉售他國。現查底股,比、法居多,事權他屬。正約四十款禁別人侵壞合同,現派非美公司之錫度來華干預。全路工程逾限,廣州一節,逾估甚鉅,請牒外務部註銷正續合同。」美政府覆牒允註銷合同,仍不允廢約。既而美公司舉前兵部路提等代議路事,中國亦延美前外部大臣福士達、鐵路律師良信等與之辯,始允再集股東議售股本購價,及合同特權等費,必須付現,又索賠給工程司執事人等合同未滿撤退,及註銷訂購物料合同之用二十五萬。久不決。至三十一年夏,始簽字。久之,始以美金六百七十五萬元還美,再加利息,定議簽押。時粵民因美禁華工,並苛待留美商民,私議抵拒美貨,不果。三十二年,遣學生赴美留學。三十三年,美教士在河南信陽州所屬雞公山購地造房,豫撫張人駿執條約公法教規與爭,始允撤房退地停工,卒延未撤銷。

三十四年八月,與美訂立公斷專約。初,美使康格曾奉其總統命,向中國提議,與英、法一律訂立公斷專約。嗣以美總統與議院意見不合,英、法約作廢,因罷議。至是第二次和會和解紛爭之約,又已畫押,各國多互訂公斷專約,美亦與英、法、日本訂約,中國即電致出使美國大臣伍廷芳,向美廷提議,遂訂條約四款,凡關於法律意義或條約解釋,為外交法不能議結者,皆屬之。換約以五年為限。是年美約請各國在滬會議禁鴉片事宜,中國命南洋大臣端方等蒞會。

宣統元年春正月,美使牒外務部,請免收東三省新開各埠一切雜稅。旋由外務部咨東三省,覆稱不能免收。因覆美使,謂:「現所收各稅,於各埠試辦章程並無妨礙。若必欲使洋貨於抽釐一事毫無轇轕,自非實行加稅免釐不可,中國固甚原各國贊成斯舉也。」五月,定留學生赴美名額,因美退還庚子賠款,為中國學生赴美游學費,議自退還之年起,初四年每年遣一百名,以後每年至少須遣五十名,遂訂辦法大綱。是年美工商部新頒華人入美保護例凡十條,大旨仍重在禁止限制華工影射赴美,而於商賈、教習、學生等游歷則從寬。牒外部立案,並同時通咨南北洋施行。二年九月,度支部大臣載澤與美使喀爾霍商定借款一千萬鎊,利息五釐,美招英、法、德、日結為借款團體,是為四國借款。


\end{pinyinscope}