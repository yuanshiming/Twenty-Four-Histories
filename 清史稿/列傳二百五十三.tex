\article{列傳二百五十三}

\begin{pinyinscope}
徐用儀許景澄袁昶立山聯元

徐用儀,字筱云,浙江海鹽人。由副貢生入貲為主事,官刑部。咸豐九年,舉順天鄉試。同治初,充軍機章京,兼直總理各國事務衙門。累遷鴻臚寺少卿,以憂歸。光緒三年,起太僕寺少卿,遷大理寺卿,直軍機如故。擢工部侍郎,始罷直。旋充總理衙門大臣,歷兵部、吏部侍郎,授軍機大臣。二十年,加太子少保。日朝構釁,舉朝爭議和戰,樞臣孫毓汶被劾罷,翁同龢繼入,主戰益力。用儀論事與同龢忤,遂出樞廷,並解總署事。二十四年,皇太后再訓政,復直總署,乃密薦太常寺卿袁昶。會許景澄奉使還,被命同入署。

二十六年,拳禍起。先是上以行新政為中外所推,而儲嗣久虛。載漪既用事,陰謀廢立,慮外人為梗,聞拳民有神勇,仇西教,欲倚以集事,召入京,遂縱恣不可制。用儀請嚴禁遏,不聽。俄戕德使克林德,用儀駭曰:「禍始此矣!」言於慶親王奕劻,厚斂之。各國兵艦至津沽,詔廷臣集議和戰。用儀、景澄、昶及尚書立山、內閣學士聯元並言:「奸民不可縱,外釁不可啟。」而載漪等主戰甚力,在廷大臣率依違不決。用儀以太后命詣使館議緩兵,當事者益目為奸邪。

景澄、昶先被害,用儀知不免,意氣自如。七月既望,遽發拳匪捕之於家,擁至莊王邸。用儀不置辯,第曰:「天降奇禍,死固分耳!」遂與立山、聯元同棄市。越三日,聯軍入京,而兩宮西狩。十二月,詔湔雪,復故官。宣統元年,追謚忠愍。浙人祠之西湖,與景澄、昶並稱「三忠」。

許景澄,字竹筼,嘉興人。同治七年進士,選庶吉士,授編修。明習時事,大學士文祥以使才薦。光緒六年,詔使日本,遭父憂,未行。服闋,補侍講。法越之役,條上籌備事宜,上褒納。十年,出使法德意和奧五國大臣,兼攝比國使務。時海軍初創,從德國購造鐵艦,未就。景澄躬歷船廠,鉤稽輯上外國師船表。又言海軍宜定屯埠膠州灣,設鐵甲砲船大沽口。轉侍讀,母憂歸。

十六年,充出使俄德奧和四國大臣,累遷至內閣學士。先是俄兵游獵,常越界,侵及帕米爾地,景澄爭之,俄援舊議定界起烏什別里山,自此而南屬中國,其西南屬俄。俄人則欲以薩雷闊勒為界。相持三載,俄始允改議,其帕界未定以前,各不進兵,以保和好。因著帕米爾圖說、西北邊界地名考證,為他日界約備。擢工部侍郎。是時俄、德迫日人還遼東,景澄曰:「俄謀自便,德圖償報,事故從此多矣!」疏請分遣兩使,從之。

二十三年,調充德國使臣。會俄建西比利亞鐵道,謀自黑龍江達海參崴,朝議拒之,乃更名商辦,許中國投貲五百萬,所謂東清鐵路公司也。詔景澄綜其事,力阻路線南溢,稽察運船毋漏稅。已而俄人索租旅順,充頭等公使,會駐俄使臣楊儒定議俄都。事竣,移疾歸,召授總理各國事務大臣兼禮部侍郎。調吏部,充大學堂總教習、管學大臣。意大利索我三門灣,景澄抗言爭之,事乃寢。

未幾,拳禍作,景澄召見時,歷陳兵釁不可啟,春秋之義,不殺行人,圍攻使館,實背公法。太后聞之動容,而載漪等斥為邪說。聯軍偪近畿,景澄等遂坐主和棄市。宣統元年,追謚文肅。

袁昶,字爽秋,桐廬人。從劉熙載讀,博通掌故。光緒二年進士,授戶部主事,充總理各國事務衙門章京。十八年,以員外郎出任徽寧池太廣道。誡僚屬,抑胥吏,多所興革;擴中江書院齋舍,課以實學;建尊經閣,購書數萬卷;汰常關耗費歲萬八千金,悉還諸公;定專條,納新關穀米出口稅,歲羨數十萬;督修蕪湖西南濱江圩堤,自大關亭至魯港,延袤十二里;更穿築新縷堤三百七十丈,自是蓄洩有資,田廬完固,民歌誦之。

膠州事起,下詔求言,昶條列時政二萬餘言,以:「德突據膠灣,其禍急而小;俄自西北至東北,與我壤地相錯,蒙喀四十八部將折入異域,其禍紆而大。宜及今預練勁旅,痛革吉、奉華靡風習。自頃兵力不能議戰,要不可不議守。我朝八旗初制,文武不分途,京外不分途,人皆兵,官皆將,故人才盛,國勢強。承平日久,文法繁密,諸臣救過之不暇,於是相率為鄉願,而舉國之人才靡矣!金田洪、楊之亂,其始一小民耳,猶窮全國之力僅而克之,況諸國互肆蠶食之心,有不乘吾敝而攻吾之短者哉?夫敵國外患,為殷憂啟聖之資。茍得其人,毋拘以文法,則理財、練兵、防海、交鄰之策,可次第就理。」上親書其綱要於冊,下中外大臣議行。二十四年,遷陜西按察使,未到官,擢江寧布政使,調直隸。未幾,內召,以三品京堂在總理衙門行走,授光祿寺卿,轉太常寺卿。時財用匱,議整釐稅。昶極言釐金名病商,實病民,不可議增。

義和團起山東,屠戮外國教士。昶與許景澄相善,廷詢時,陳奏皆忼慨,上執景澄手而泣。昶連上二疏,力言奸民不可縱,使臣不宜殺,皆不報。復與景澄合上第三疏,嚴劾釀亂大臣,未及奏,已被禍,疏稿為世稱誦。追謚忠節,江南人祠之蕪湖。

昶嘗慨士鮮實學,輯農桑、兵、醫、輿地、治術、掌故諸書,為漸西村叢刻。

立山,字豫甫,土默特氏,蒙古正黃旗人。光緒五年,以員外郎出監蘇州織造,歷四任乃得代。論修南苑工,賜二品服。累遷奉宸苑卿、總管內務府大臣、正白旗漢軍副都統、戶部侍郎。二十年,加太子少保。盜竊寧壽宮物,坐失察,鐫職留任。二十六年,擢戶部尚書。立山久典內廷,同列嫉其寵眷。會拳禍起,聯軍至天津,廷臣集議御前。載漪盛推拳民可用,立山適在側,太后謂:「汝言如何?」立山曰:「拳民雖無他,然其術多不驗。」載漪怒曰:「用其心耳,奚問術?立山必與外人通,請以立山退外兵!」立山曰:「首言戰者載漪也!臣主和,又不諳外事,不足任。」載漪益仇之,因其宅鄰教堂,乃中以蜚語,謂藏匿外人,竟論死。宣統元年,追謚忠貞。

聯元,字仙蘅,崔佳氏,滿洲鑲紅旗人。同治七年進士,選庶吉士,授檢討,累遷侍講。大考,左遷中允,再陟侍講。以京察,出知安徽太平府,調安慶。兩薦卓異,署滁和道,遷廣東惠潮嘉道。汕頭者,通商要衢也,奸人倚英領事為民暴,聯元裁以法,良善獲安。二十四年,擢安徽按察使,入覲,改三品京堂,在總理衙門行走。又明年,補內閣學士。拳民仇西教,載漪、剛毅助之,勢益橫,日夜圍攻使館,不能下。大臣負清望者徐桐、崇綺,皆謂:「民氣可用。」聯元與崇綺爭論帝前,謂:「民氣可用,匪氣不可用。」聯軍既陷大沽,載漪等猶壹意主戰。聯元謂:「甲午之役,一日本且不能勝,況八強國乎?儻戰而敗,如宗廟何?」載漪斥其言不祥,七月十七日,斬西市。昭雪後,予謚文直。順天府奏請立山、聯元合祠宣武門外,而聯元祖居寶坻,更於其地建專祠焉。

論曰:清代優禮廷臣,罕有誅罰。拳禍既起,忠諫大臣駢首就戮,豈獨非帝意哉?觀用儀諸人所論事勢利害,昭昭如此,乃終不能回當軸之聽,何其昧焉?世傳大節,並號「五忠」,不數日而遂昭雪,允哉!


\end{pinyinscope}