\article{列傳五十一}

\begin{pinyinscope}
郝維訥任克溥劉鴻儒劉楗硃裴張廷樞

郝維訥,字敏公,直隸霸州人。父傑,明崇禎進士。順治初,授行人,遷戶部給事中。迭疏請開經筵,祀闕里,廢斥諸臣才堪錄用者量予自新,朝賀大典內監不得入班行禮,俱下部議行。累遷戶部侍郎。卒。

維訥,順治四年進士,授刑部主事,再遷郎中。七年,出為福建督糧道僉事。師下漳南,糧運多阻,維訥督米二萬石浮海達泉州以濟軍。巨盜張自盛犯延、卲,徙維訥權延建卲道,設方略,用間散其黨,自盛就擒。尋署按察使,謝苞苴,絕羨耗。舉卓異,復用孫承澤、成克鞏薦,十一年,召授通政司右參議。累遷大理寺卿。十三年,擢戶部侍郎,調吏部。十六年,丁父憂。服闋,起戶部侍郎,復調吏部。

康熙三年,典會試,尋擢左都御史。維訥以開國二十餘年,南徼初定,民困未蘇,疏言:「天下大弊在民窮財盡,連年川、湖、閩、廣、雲、貴無不增兵增餉,本省不支,他省協濟。臣觀川、湖等省尚多曠土,若選綠旗及降兵精銳者隸之營伍,給以牛種,所在屯田,則供應減而協濟可以永除,閭里無追呼之困。」又疏言:「巡按既裁,地方巡視責歸督撫。督撫任重事繁,出巡動逾旬月,恐誤公務,況騎從眾多,經過滋擾。至屬官貪廉,閭閻疾苦,咨訪耳目,仍寄司道。請嗣後事關重大者,仍親身巡察,餘概停止。」又疏言:「山西、山東等省偏旱,發帑賑濟,聖恩至為優渥,特窮鄉僻壤恐難遍及,惟蠲免錢糧,率土均霑實惠。但田有田賦,丁有丁差,前者被災地方,例多免糧不免丁;其有丁無田者,反不得與有田之戶同霑恩澤。請丁銀均如田糧分數蠲免。」又疏言:「貪吏罪至死者,遇赦免死,並免交吏部議處。此曹饕餮狼藉,未可令其復玷名器,貽害地方。雖新例赴部另補,貪殘所至,播虐惟均。請敕部定議,凡贓款審實者,遇赦免罪,仍當奪官。庶官箴可肅,民害可除。」皆下部議行。

五年,遷工部尚書,調刑、禮二部。八年,調戶部。疏請停督撫勘災,申禁圈取民地,並得旨允行。十一年,調吏部。時兵興開捐納,正途日壅,維訥為斟酌資格,按缺分選,銓法稱平。十八年,給事中姚締虞請寬免科道風聞言事之禁,下廷臣議,維訥謂:「言官奏事,原不禁其風聞。但風聞奏參審問全虛者,例有處分,否則慮有藉風聞挾私報怨者,請仍照定例行。」從之。

維訥領吏、戶二部最久,法制多經裁定。凡事持大體,遇會議、會推、朝審,委曲斟酌,期於至當。敷奏條暢,所見與眾偶有同異,開陳端緒,不留隱情,上深重之,往往從其言。十九年,遭母憂。服闋,詣京師,未補官,卒,謚恭定。

任克溥,字海眉,山東聊城人。順治四年進士,授南陽府推官。卓異行取,十三年,授吏科給事中。疏言:「上勵精圖治,知親民之官莫過守令,特擇各府繁劇難治者,許三品以上各舉一人,破格任用。使保舉得當,一人賢則一郡安,人人賢則各省安,太平何難立致。乃為時未久,以貪庸劾罷者已有數人,前此保舉不能秉公慎選可知。乞敕部察處。」

十四年,轉刑科,疏言:「抗糧弊有三:宦戶、儒戶、衙蠹。宜分三項,各另造冊,申報總督、巡撫、巡按,宦欠者題參,衿欠者褫革,役欠者逮治。」復疏論順天鄉試給事中陸貽吉與同考官李振鄴、張我樸交通行賄鬻舉人,下吏部、都察院嚴鞫,貽吉、振鄴、我樸與居間博士蔡元禧,進士項紹芳,行賄舉人田耜、鄔作霖皆坐斬。命禮部覆試不及程者,褫奪流徙又二十五人,考官庶子曹本榮、中允宋之繩並坐降調。

十五年,充會試同考官,出闈,疏言:「伏讀上諭,令各衙門條奏興利除弊。時近兩月,僅見宗人府一疏,各衙門遲疑觀望。竊謂其病有二:一則因循既久,發論方新,恐無以贖往日曠官之咎;一則瞻望多端,指陳無隱,恐無以留後來遷就之門。臣子報國,止有樸忠,遇事直陳;稍一轉念,便持兩端,勢必摭拾瑣屑,剿說雷同,不能慷慨論列,又安望設誠致行?乞嚴飭不得浮泛塞責,並鑒別當否,示以勸懲。」又疏言:「近以各衙門胥役作奸犯科,詔令諸臣計議指摘。臣以為懲於弊後,不若杜於弊先,如吏部文選司推升原有定序,應先懸榜部門,序列姓名、資俸、薦紀、參罰,使共見共聞;考功司議處條例,亦畫一頒發,使不得輕重增減。至各官開缺,以科鈔為憑,向以發鈔後先轉移舞弊。如當逮問,先下刑部,與事止奪官、逕下吏部者遲速有異。應令即日鈔發,使不容操縱。」上以所奏切中時弊,下部詳議行。

轉禮科都給事中,疏言:「士為四民首,宜端習尚。請敕學臣,凡有請託私書,許揭送部科,差滿定為上考。並令舉優當訪學行著聞之士,懲劣則以抗糧為最重。」又疏言:「錢糧逋欠,非盡在民。臣前奏三款,部議分冊申報,得旨允行;而造冊奏報者,惟山西一省耳。諸省玩洩從事,不肯實心清理,徒以開荒增課,一時博優敘之榮,仍聽其逋欠而不之問,請飭部察覈;又紳衿抗糧,定有新條,蠹役尤應加嚴,並請敕部定例行。」十七年,遷太常寺少卿。十八年,遭父喪。

康熙三年,起補原官。六年,疏言:「朝廷欲薄賦,有司反加賦;朝廷欲省刑,有司反濫刑:皆由督撫不得其人。今方有詔令部院糾察,部院肯糾極貪大惡之督撫一人,天下為督撫者警;督撫肯糾極貪大惡之司道一人,天下為司道者警。督撫、司道廉潔,則有司不苦誅求,輕徭薄賦,政簡刑清,自寬然有餘地矣。」八年,應詔陳民生疾苦,言:「小民莫疾於加派,莫苦於火耗,已敕嚴禁矣。此外疾苦尚有數端:有司派殷戶催糧,糧單中多列逃亡絕戶,無可徵糧;且有糧冊無名,按時追比,致傾家以償者。郵傳供應,原有錢糧,或侵入私囊,僉民養馬應夫或充里長。使客往來,舟車飲食,責令設備。河漕附近,籍民應役,衣敝履決,力盡筋疲,而工食或至中飽。淺夫閘夫,賣富差貧,一名更至數十名,衙役捕系恫哧,民被累無窮。請敕督撫清釐懲禁。」上納其言,並特諭河工毋得累民。

尋遷右、左通政。十一年,疏言:「嘉魚知縣李世錫告湖廣巡撫林天擎索賄,以此知餽遺不絕,苞苴尚行,較世祖朝有司不敢餽遺督撫、不敢輕至省會風氣迥殊。督撫初受命,群餽裘馬、弓矢,而為督撫者亦飾觀瞻、趨奢侈,一時費累萬。上官後,為酬報取償地,遂苛索屬吏,貽累於民。請敕督撫赴官之先,屏絕餽送,勿鋪張行色,以儉養廉。督撫參罰科條甚密,部院亦當知督撫艱難繁重,依例處分,毋過為吹索,俾得專心吏治民生,無旁顧之憂。」先後諸疏並下部議行。

十二年,擢刑部侍郎。十八年,京察,以才力不及擬降調,命再議,改註不謹,遂奪官。三十八年,迎蹕臨清,復原銜。四十二年,南巡還蹕東昌,幸其所居園,賜松桂堂榜。以克溥年將九十,賜刑部尚書銜。是歲卒,賜祭葬。乾隆四十七年,高宗覽克溥條奏諸疏,善之,諭:「克溥逮事兩朝,抒誠建白,無愧直言謇諤之臣。」並命錄諸疏宣示。

劉鴻儒,字魯一,直隸遷安人。順治三年進士,授兵科給事中。疏言:「開國之始,首重安民,宜輕賦徭,革積弊。伏讀恩詔,賦制悉依萬歷初年,及觀順治二年徵數,並不減少,且復增重,請敕有司核實。州縣六房書吏,初房各二人,今則增至七、八十人,並請敕有司核簡。」上命指實,鴻儒復言:「臣籍遷安,明季丁銀,下下二錢,下中四錢,上地一畝七分有奇。民苦輸將,猶多逋賦。今蒙恩詔蠲免,而二年徵數,二錢者增至三錢六分,四錢者增至七錢二分,上地每畝增至八分有奇。一邑如此,他邑可知。乞敕清查蠲免。」下部確察。四年,調戶部。五年,坐糾鉅鹿知縣勞有學失實,左遷上林苑蕃育署署丞。十年,命復故官。十三年,補兵科,疏言:「畿輔近地,劫掠時聞。請嚴責成,謹防捕。」下部如所請。

轉戶科,十五年,疏言:「開國以來,度支屢見不敷。汱冗員,增榷務,廣輸納,督積逋,講求開節,已無不盡。今南服削平,萬方底定,宜總計財賦之數,準其出入,定為經久不易之規。請通計一歲內畝賦、丁徭、鹽徵、津稅,各省輕齎、重運及贖鍰事例等項,汰其猥瑣無藝者,所存金粟若干數;然後計一歲內上方供應、官吏俸祿、兵馬糧料、朝祭禮儀、修築工役,以至師生廩餼、胥役代食,罷其不經無益者,所需金粟若干數:務使出入相合,定為會計之準。用財大端惟兵,生財本計惟土。欲紓國計,莫善於屯田,朝廷下民屯之令。設官置役,多糜廩祿,得不償失,不旋踵而請罷。稽古屯制,不在民而在兵,請敕各省駐兵處所,無論邊腹地方,察有荒土,令兵充種。正疆界,信賞罰,則趨事自力;豐種具,寬程效,則收穫自充。此唐初府兵之制也。自頃四川、貴州已入版圖,所得之地,必需駐守;若令處處興屯,則根本自固,戰守咸資。此又趙充國之於先零,杜預之於宛、葉,確然可循之遺策也。順天左右郡縣,拱翊王畿,根本要地,自令舊人圈住,深得居重馭輕之意。但畿輔之民,多失恆業,撥補他地,皆有系屬,豈能據為己有?今喜峰、冷口諸關外,大寧以南,彌望千里,咸稱膏壤,請令民原出關開墾者,許承為己業。沃土新闢,獲利必饒,先事有獲,趨者自眾。數年以後,漸次起科,成聚成邑,堪資保障。二者皆軍國大計,若設誠致行,久之兵食充足,國基賴以不拔矣。」下部議,以滇、黔未靖,兵餉無數,難以預定會計;設置兵屯,及畿輔民出邊墾種,敕所司詳勘。

十七年,遷順天府府丞,再遷左通政。十八年,遷太常寺卿。康熙三年,遷通政使。六年,擢兵部侍郎。十年,調戶部。十二年,遷左都御史。

官戶部時,甘肅巡撫華善因擅發倉粟賑災,戶部循例題參,並議罰償,鴻儒無異議;及官都御史,又疏論華善不應參處,嗣後封疆大臣有利民之政,不宜拘以文法。給事成性疏劾,下部議,坐鴻儒先未異議,後又指摘沽名,降二級調用。尋卒於家。

劉楗,字玉罍,直隸大城人。順治二年進士。是歲選新進士十人授給事中,楗除戶科。疏論山東巡撫楊聲遠劾青州道韓昭宣受賄釋叛賊十四人,僅令住俸剿賊,罰不蔽辜,昭宣坐奪官。四年,轉兵科右給事中。疏論江南巡按宋調元薦舉泰州游擊潘延吉,寇至棄城走,調元濫舉失當,亦坐奪官。是歲大計,楗用拾遺例,揭山東聊城知縣張守廉贓款,下所司勘議,守廉以失察吏役得贓,罰俸;楗誣糾,坐奪官。十年,吏科都給事中魏象樞請行大計拾遺,因論楗枉,得旨,吏役詐贓,知縣僅罰俸,言官反坐奪官,明有冤抑,令吏部察奏,命以原官起用。授兵科左給事中。

十一年,疏言:「近畿被水地,水落地可耕。方春農事急,請敕巡撫檄州縣發存留銀,借災民籽種,俟秋成責償。仍飭巡行鄉村勘覈,不使吏胥得緣以為利。」

十二年,疏言:「鄭成功蹂躪漳、泉,窺伺省會。臣昔充福建考官,詢悉地勢。福清鎮東衛,明時駐兵防倭。倘復舊制,可以保障長樂,籓衛會城。宋、元設州海壇,明以倭患棄之。若設將鎮守,可與鎮東互為犄角。仙霞嶺為入福建門戶,與江西、浙江接壤,宜設官控制,招民以實其地,俾無隙可乘。成功數犯京口,泊舟平洋沙為巢穴。宜乘其未至,移兵駐鎮,使退無可據,必不敢深入內地。」疏入,敕鎮海將軍石廷柱等分別駐守。

十三年,授山西河東道副使。十五年,轉河南鹽驛道參議。十六年,授湖廣按察使,就遷右布政。十八年,總督張長庚、巡撫楊茂勛疏薦楗廉幹,協濟滇、黔兵餉至八百餘萬,清逋賦墾地,除鼓鑄積弊。楗以母憂歸。康熙二年,起江西布政使。

吳三桂亂作,措餉供兵,事辦而民不擾。十四年,授太常寺卿。十六年,遷大理寺卿。十七年,擢副都御史,疏言:「自吳三桂為亂,軍需旁午,大計暫停。今師所至,漸次蕩平。伏思兵後殘壞已極,正賴賢有司招徠安輯。若使不肖用事,何以澄吏治、奠民生、息盜賊?請令督撫速行舉劾,凡經薦舉者,改行易操,一體嚴察,不得偏徇。」下部如所請行。又疏言:「江西當亂後,民逃田墟,錢糧缺額不急予蠲免,逃者不歸,歸者復逃;荒者未墾,墾者仍荒。」上為特旨悉行蠲免。

旋以病乞休,諭慰留,遣太醫視疾。擢吏部侍郎。未幾,復擢刑部尚書。十八年,病劇,始得請還里。至家,卒,賜祭葬,謚端敏。

硃裴,字小晉,山西聞喜人。亦順治三年進士。知直隸易州,移河南禹州。裴治尚嚴,到官即捕殺盜渠。縣有諸生聘婦為盜掠,既復自歸。盜以奪婦訟生,婦以生貧且別娶,反為盜證。前政論生死,裴廉得實,為榜殺婦而出生於獄。擢刑部員外郎,遷廣東道御史,再遷禮科給事中。滿洲俗尚殉葬,裴疏請申禁,略言:「泥信幽明,未有如此之甚者。夫以主命責問奴僕,或畏威而不敢不從,或懷德而不忍不從,二者俱不可為訓。好生惡死,人之常情。捐軀輕生,非盛世所宜有。」疏入,報可。累遷工部侍郎。以疾乞休,歸。地震,傷於足,臥家九年,卒。

張廷樞,字景峰,陜西韓城人。父顧行,康熙六年進士,官江安督糧道。廷樞,二十一年進士,選庶吉士,授編修。三十八年,以侍讀主江南鄉試。四十一年,以內閣學士督江南學政。四十四年,聖祖南巡,賜御書、冠服。四十五年,遷吏部侍郎,充經筵講官。

湖廣容美土司田舜年揭其子昺如貪庸暴戾,昺如匿桑植土司向長庚所,不赴鞫。總督石文晟以聞,並劾舜年僭妄。命左都御史梅鋗、內閣學士二格會文晟按治。舜年詣武昌,文晟執之,病卒。鋗與文晟各具議疏陳,二格疏言佐證未集,未可即定議。詔廷樞偕大學士席哈納、侍郎蕭永藻覆勘,舜年各款俱虛,梅鋗以草率具奏,下部議奪官;文晟及湖北巡撫劉殿衡、偏沅巡撫趙申喬、提督俞益謨各降罰有差。

四十八年,進刑部尚書。民張三等盜倉米,步軍統領託合齊逮送刑部,滿尚書齊世武擬斬監候,廷樞持不可,擬充軍。下九卿議,廷樞改擬不當,當罰俸。上責廷樞偏執好勝,奪官。俄,託合齊得罪,五十一年,起廷樞工部尚書。江南總督噶禮、江蘇巡撫張伯行互訐,命尚書張鵬翮、總督赫壽按治,議奪伯行官。上復命廷樞與尚書穆和倫覆勘,如鵬翮等議。疏下九卿,上特命奪噶禮官,伯行復任。

五十二年,調刑部。五十六年,河南宜陽知縣張育徽加徵火耗虐民,盜渠亢珽結澠池盜李一臨據神垕寨為亂,並劫永寧知縣高式青入寨;閿鄉盜王更一亦藉知縣白澄豫徵錢糧,嘯聚圍縣城;巡撫張聖佐、總兵馮君侁不能平,又匿不以起釁所由入告。命廷樞與內閣學士勒什布按治,珽自縊;更一、一臨就擒,置之法;澄、育徽擬絞監候;聖佐、君侁奪官;並追咎原任巡撫李錫令屬吏加徵激變,論斬。蘭陽白蓮教首袁進等謀不軌,命廷樞並按,論罪如律。五十八年,南陽鎮兵為亂,辱知府沈淵,命廷樞偕內閣學士高其倬按治;浙江巡鹽御史哈爾金受商人賕,被劾,命廷樞偕內閣學士德音按治。並論如法。

廷樞還京師,疏言:「河南漕米自康熙十四年每石改折銀八錢解部,嗣因米賤,部議以一錢五分解部,餘交巡撫購米起運。巡撫分委州縣,州縣復派民買輸,甚為閭閻累。請交糧道購運,毋得派累民間。」下部議行。

世宗在籓邸,優徐採嗾傭者箠殺人,部議以傭抵。廷樞獨議罪在採,坐徙邊。世宗即位,褒廷樞抗直,復逮採論罪。雍正元年,以原任編修陳夢雷侍誠郡王得罪,命發黑龍江,廷樞循故事,方冬停遣,又出其子使治裝。尚書隆科多劾廷樞徇縱,命鐫五級,逐回籍。

子縉,進士,官中允,亦以告病家居。六年,陜西巡撫西琳劾廷樞受河督趙世顯贓六千,抗追不納,縉居鄉不法。詔奪廷樞及縉官,令所司嚴訊。廷樞被逮,道卒。總督岳鍾琪議縉當斬,籍其家,詔特寬免,令縉在川、陜沿邊修城贖罪。乾隆時,復廷樞官,追謚文端。子綖,亦進士,官戶部主事。

論曰:維訥論貪吏遇赦,不得遽復官;克溥言民生疾苦,戒加賦濫刑;鴻儒請定歲會之制;楗議兵後當復行計典;裴請禁殉葬:益於國,澤於民,言各有所當也。廷樞使車四出,惟張伯行事出上裁,他皆稱指。律嚴科場罪,所以重取士,乃草野私議輒以為過甚。克溥興丁酉順天之獄,卒以不謹罷,殆怨家所中歟?廷樞得罪,似亦有齮之者,詘而後申,足為謇直者勸矣。


\end{pinyinscope}