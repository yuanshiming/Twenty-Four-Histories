\article{志六十八}

\begin{pinyinscope}
禮十二(兇禮二)

皇太子皇子等喪儀親王以下及公主以下喪儀

醇賢親王及福晉喪儀忌辰賜祭葬賜謚

外籓賜恤品官喪禮士庶人喪禮服制

皇太子皇子及皇子福晉喪儀皇太子喪儀,有清家法,不立儲貳。至乾隆三年,皇次子永璉薨。高宗諭曰:「永璉為朕嫡子,雖未冊立,已定建儲大計,其典禮應視皇太子行。」禮臣奏言:「皇太子喪禮,會典未載。舊制,沖齡薨,不成服。今議,皇帝素服,輟朝七日。若親臨奠醊,冠摘纓。典喪大臣、奏遣之王公暨皇太子侍從官咸成服,內務府佐領、內管領下護軍、驍騎校等成服,以六百人為率,並初祭日除。直省官奉文日,咸摘冠纓素服三日,停嫁娶、輟音樂,京師四十日,外省半之。幼殤例無引幡,今請依雍正時懷親王喪儀,引幡仍用。外籓額駙、王、公、臺吉、公主、福晉、郡主服內來京,男摘冠纓,女去首飾。朝鮮使臣素服七日。金棺用桐木。」啟奠帝親祭酒,奉移親視送。禮部長官祭轝。初祭內外會集,帝至殯殿奠酒三爵,每奠眾一拜,是日除服薙發。將冊謚,先期遣告太廟後殿、奉先殿,謚曰端慧。禮成。禮部頒行各省,並牒朝鮮國王,文到率百官素服,軍民罷嫁娶、音樂各三日。八年,葬硃華山園寢。

皇子喪儀,順治十五年,榮親王薨,治喪視親王加厚,葬黃花山園寢。

康熙中,定制,凡皇子殤,備小式硃棺,祔葬黃花山,唯開墓穴平葬,不封不樹。

雍正六年,皇八子福惠卒,帝輟朝,大內素服各三日,不祭神,詔用親王禮葬。十三年,追封親王,謚曰懷。

乾隆十三年,皇子永琮甫二周薨,帝言:「建儲之意,朕雖默定,然未若端慧太子旨已封貯,喪儀應視皇子為優。」大祭親臨奠醊,謚悼敏,後追封哲親王。

越二年,皇長子永璜薨,金棺用杉木,其福晉及皇孫綿德等翦發去首飾,成服百日而除,素服二十七月。成服王公大祭日除。禮部以第三日移殯,請輟朝三日,詔改五日,追封定親王,謚曰安。初祭、大祭並親臨奠醊。

二十五年,皇三子永璋薨,詔用郡王例治喪,輟朝二日。大內、宗室素服咸五日,不祭神。追封循郡王。

四十一年,皇十二子永璂薨,詔用宗室公例治喪。嘉慶四年,追封貝勒。

道光十一年,皇長子奕緯薨,命依皇子例治喪。罷公主、福晉、命婦會集,園寢不建碑,追封貝勒,謚曰隱志。三十年,晉封郡王。

皇子福晉喪,定制,親王世子、多羅郡王下及奉恩將軍、固倫公主、和碩福晉下及固山格格、奉恩將軍妻咸會集。朝供饌筵,午果筵。初祭引幡一,楮幣十二萬,饌筵二十五,羊十五,酒七尊,讀文致祭。繹則陳楮幣三千,饌筵十二,羊、酒各七。百日、周年、四時致奠禮同。

嘉慶十三年,宣宗時為皇次子,其福晉鈕祜祿氏薨,帝命即日成服,初祭後除。未分府皇子福晉依親王福晉例,金棺、座罩皆紅色,以無儀仗,特賞金黃色座罩,儀仗仍視親王福晉例用,旗色用鑲白,著為令。

道光七年,皇長子奕緯福晉瓜爾佳氏薨,罷內外齊集及豫往暫安處接迎。十一年,追封貝勒夫人,諏吉遣官奉紙冊往殯所,讀文致祭。

親王暨福晉等喪儀順治九年,定親王喪聞,輟朝三日。世子、郡王二日。後改貝勒以下罷輟朝。斂具,親王至貝勒採棺,藉五層。貝子至輔國公棺同,藉三層。鎮國將軍以下硃棺,藉一層。初薨陳儀衛,鞍馬、散馬親王十五,世子、郡王各十四,貝勒各十三,貝子各十二,鎮國公各十,輔國公各八;鎮國將軍鞭馬七,輔國將軍五,奉國將軍四,奉恩將軍三。府屬內外咸成服,大祭日除。內外去冠飾、素服會集,各如其例。鎮國將軍以下不會喪。公主、福晉、命婦會喪,臨時請旨行。凡親王至輔國公,御祭二,遣官至墳讀文致祭。宗人府請賜謚,撰給碑文。工部樹碑建亭,貝勒以下碑自建,給葬費有差。鎮國將軍至奉國將軍賜祭二,文一。立碑、予謚,臨時請旨。奉恩將軍賜祭無文,不立碑,不予謚。

王至公婚娶之子卒,許陳鞍馬,祭品各如其父母例,唯不遣官致祭。未婚娶幼子不造墳。

凡葬期,親王期年,郡王七月,貝子以下五月。

又定親王福晉以下喪,內外會集如制,陳儀衛從其封爵,親王福晉、側福晉、世子福晉禦祭一。

十二年,定下嫁外籓公主喪,御祭一,遣官至塋所讀文致祭。

康熙四年,定貝勒至入八分公予謚請旨行。

九年,定親王至輔國公喪,本府官屬具喪服,其禮親王、肅親王、承澤親王、敬謹親王、饒餘親王、鄭親王、克勤郡王、恪僖貝勒、靖寧貝勒、顧爾馬洪貝子、福勒黑公十二支,凡為本支所分者,本身暨府屬官、命婦咸具喪服,非本支會喪者摘冠纓,從官如其主,尊屬無服。

五十二年,定貝勒生母薨,治喪如嫡夫人,遣官讀文致祭。五十四年,定固倫公主有子孫者,獲請建碑予謚。

雍正四年,遵旨議定嗣後皇帝子孫依五等服制,遇期服伯叔兄弟喪,依例具奏臨喪。其諸王以下,不論爵次,遇小功以上喪,會喪成服,期六十日、大功一月、小功七日除。乾隆三年,更定期服大祭日、大功初祭日、小功送殯日除。

二十一年,諭諸王側福晉予謚請旨行,予祭不逾一次,罷給祭文。三十六年,定貝勒、貝子、公兼一品職獲請予謚,鎮國暨輔國將軍兼一品職獲請賜恤。四十年,定凡側福晉為王等生母,獲請賜祭,降嫡福晉一等。五十六年,鎮國公晉昌夫人卒,詔罷賜祭,後仿此。

嘉慶十七年,貝勒綿懃子奕綬卒,命封為未入八分輔國公,嗣後宗室如追封公,俱作為未入八分,著為令。

公主以下喪儀,順治九年,定固倫公主喪視親王福晉,和碩公主視世子福晉,郡主視郡王福晉,縣主視貝勒夫人,郡君視貝子夫人,縣君視鎮國公夫人。十二年,定下嫁外籓公主至縣主並給諭祭文,遣官赴墳讀奠。郡君以下,致祭無文。道光二十四年,定公主薨,內務府請旨,如命官為治喪,一切典禮,即會禮部具奏。得旨,再牒各署治辦,額駙自行治喪,禮部應將會集處奏聞。公主以下喪,會集臨時請旨,如獲請,牒宗人府、五旗傳行。未釐降受封者,內務府治喪,不會集。

醇賢親王及福晉喪儀光緒十六年,醇親王奕枻薨,定稱號曰「皇帝本生考」,帝持服期年,縞素十有一日,輟朝如之。期年內御便殿仍素服。元旦謁堂子,詣慈寧宮,太和殿受朝,並禮服。唯升殿不宣表,樂設不作,罷宗親、廷臣筵宴。祭文、碑文書皇帝名。初祭、大祭暨奉移園寢並御青袍褂,冠摘纓,親詣行禮。又定廟制及祭葬,廟中殿宇及正門瓦色,中用黃琉璃,殿脊及正門四圍用綠琉璃。祀以天子禮。歲時饗,四仲月朔舉行,忌辰躬親致祭。葬以親王禮,帝親制碑文,謚曰賢。三十二年,其福晉葉赫那拉氏薨,稱「皇帝本生妣」,喪儀如醇賢親王例。

忌辰順治十年,定盛京、興京三陵忌辰,遣守陵官行禮,獻酒果,不讀祝,不奠帛。十二年,改定忌辰遣官,禮部具題請旨。康熙三年,復定三陵忌辰在隆恩殿神牌前揭幔致祭。雍正四年,帝以聖祖喪滿,哀慕無窮,思依三年內祭禮舉行,下禮臣議。尋議上,依周年祭祀例,遣在京或陵寢王公大臣一人承祭,在京王公百官遣三之一陪祭。著為令。十三年,高宗嗣服,議定聖祖忌辰,依陵寢四時大祭,用太牢,獻帛爵,讀祝文,遣官承祭,陵寢官悉陪祀,罷遣京官往。嗣後列聖、列後忌辰,永如例行。

定制,帝、後忌辰,內外俱素服,停宴會,輟音樂,不理刑名,帝詣奉先殿後殿上香行禮。如祀南郊,帝閱祝版,遇忌辰,御龍袍、龍褂、掛數珠,執事官蟒服、補褂、掛數珠。閱北郊、廟社暨各中祀祝版,則俱御龍褂、掛數珠,執事官咸補服、掛數珠。大祀齋期內,御常服、掛數珠,陪祀執事官亦如之。凡祭日遇忌辰,行禮時祭服作樂,禮畢仍素服。

賜祭葬世祖初入關,沿崇德間例,超品公,一、二、三等公卒,遣官祭三次;子、副都統二次;參領、佐領一次。陣亡與有勛勞者,遣官治喪,出自上裁。

順治三年,定制民公、侯、伯、子兼任內大臣、都統、大學士、尚書、鎮守將軍卒,候旨立碑,致祭一次。襲公、侯、伯、子在任不逾三年,止給祭品,無祭文,不立碑。二、三品官卒,給祭品。滿任三年給祭文。有戰功者,獲請立碑。

十三年,定佐領、員外郎、主事任滿三年,給祭品、祭文,未滿者無祭文。致仕同。

十五年,定部、院長官加秩至一、二品,致祭、立碑。三品滿三年者如之。未滿,但致祭而已。護軍統領、副都統、前鋒統領、步軍總尉考滿視三品。如為男爵,得致祭、立碑。參領、前鋒參領滿三年,致祭,不立碑。四品卿、少卿考滿者同,否則不給祭文。陣亡不論品級,獲請恩恤。內大臣、都統、大學士、尚書、護軍統領、副都統、前鋒統領、侍郎、學士、步軍總尉原品休致者,致祭、立碑同。現任輕車都尉、佐領、騎都尉、郎中、員外郎、主事,致祭、無碑文。承襲公、侯、伯有職任者,依職任予恤,否則止給祭品。

十七年,定本身所得民公、侯、伯、子及都統有職任內大臣、鎮守將軍給全葬。大學士,尚書,左、右都御史加級及宮保者,視一品給全葬,無加銜、加級視二品給全葬。侍郎無兼銜、加級而考滿者,視三品給全葬,未滿者半之。四品卿、少卿或兼少卿銜,視四品,止給祭品。護軍統領、前鋒統領、副都統、步軍總尉任滿給全葬,未滿者半之,並致祭一次。武職自參領、文職自郎中以下,俱不給祭品。陣亡者如故。

十八年,定本身所得民公、侯、伯造葬,致祭一次,加祭出特恩。都統、內大臣、大學士、尚書、右都御史、子、鎮守將軍及加銜、加級至一二品官,俱依品級造葬,致祭一次。三品侍郎、學士、通政使、大理寺卿考滿者給全葬,未滿者半之,俱致祭一次。參領、協領、郎中、佐領及三等侍衛、護衛官陣亡者,致祭一次。漢文職一、二品或三品考滿,俱致祭、造葬,未滿者半之,致祭一次。在外布、按以上,依京秩例行。武職加銜副將以上,造葬,致祭一次,無兼銜而考滿者同,未滿者半之,致祭一次。知縣、守備以上陣亡者,各依加贈品級造葬,致祭一次。凡滿、漢文武原官致仕者,恤典同現任。

康熙九年,定本身所得及承襲公、侯、伯給全葬,遣官讀文,致祭一次。內大臣、都統、子品級散秩大臣、大學士、尚書、左都御史、子、世襲子、鎮守將軍、提督,各依品級給全葬,遣官讀文,致祭一次。男品級散秩大臣、護軍統領、前鋒統領、副都統、侍郎、本身所得男、學士、副都御史、總督、總兵官、加級至二品巡撫,各依所加品級給全葬,遣官讀文,致祭一次。三品侍郎、學士、副都御史、巡撫、通政使、大理寺卿,任滿給全葬,未滿者半之,俱遣官讀文,致祭一次。布政使給全葬,致祭一次。雲騎尉、三等侍衛以上,文職知縣、武職守備以上陣亡者,各依加贈品級給全葬,致祭一次。

道光二十四年,定賜祭王、公以下儀,祭日,堂中陳儀衛,靈座前置供案,陳賜祭物品,左右分陳自備祭品。案前設遣官奠位,東設祝案,北鄉,南設燎位,具楮帛。遣官至,喪主率宗親及屬官跪迎大門外,禮部官奉祭文入自中門,陳東案,遣官隨入,就位立,喪主以下皆就位跪。讀祝官讀文訖,遣官跪奠三爵,每奠一叩。鎮國將軍以下立奠,喪主率眾隨行禮。畢,興,舉哀,燎祭文。喪主率眾望闕謝恩,三跪九叩。遣官出,跪送大門外。

賜謚親王例用一字,貝勒以下及文武大臣二字。郡王謚號,尚沿明制用二字,間有用一字者。聖祖時,追謚郡王,滿、漢文俱用一字,遂為定制。

順治九年,定親王以下喪聞,宗人府請謚,內院撰擬碑文。康熙四年,定諸王賜謚,封號上加一字,貝勒以下、入八分公以上,予否請旨行。乾隆三十六年,遵旨議定貝勒至輔國公兼一品職者予謚,仍請旨。其兼二品以下職暨不兼職者罷予謚。

定制,一品官以上予否請上裁,二品官以下不獲請。其得謚者,率出自特旨,或以勤勞,或以節義,或以文學,或以武功。破格崇褒,用示激勸。嘉、道以前,謚典從嚴,往往有階至一品例可得而未得者。世宗朝,一等公福善,大學士魏裔介,將軍佛尼勒、莽依圖,都統馮國相,尚書湯斌、徐潮、瑪爾罕輩,望實素高,入祀賢良。逮至高宗初元,始獲追謚。易名盛典,殊不易得。

令甲,得謚者禮部取旨,行知內閣典籍撰擬。至穆宗朝,大學士卓秉恬改歸漢票簽,唯侍讀司之。大學士及翰林授職者,始得謚「文」,亦有出自特恩而獲謚文者。侍讀擬八字,大學士選四字,餘則擬十六字,大學士選八字,並請上裁定。武臣有謚文者,如領侍衛內大臣索尼獲謚文忠,異數也。唯「文正」則不敢擬,出自特恩。文職內自三品卿、外自布政使以下,例不予謚。唯御史陸隴其謚清獻,侍講學士秦承業謚文愨,太常卿唐鑒謚恪慎,則以崇尚儒臣,篤念師傅,不為恆式。

咸豐三年,禮臣奏定文職二品官殉難,視一品予謚。如按察使優恤,禮部亦得援例以請。軍興而後,道、府、州、縣等官死綏不少,疆臣疏請,不拘常格矣。其武職死事,參將以下,視副將議恤;協領以下,視副都統議恤:皆得援新章奏請。唯武功未成者,不得擬用「襄」字。至十二年,諭:「嗣後文武各官,其官階例不予謚者,不得率行奏請。」至是限制稍嚴。

光緒四年,貴州巡撫黎培敬為已革總督賀長齡請謚。詔以易名之典,不容冒濫,嚴切申儆,且下培敬吏議。亦有得謚而被奪者,若沈德潛、卞三元,或追論其生平,或敗露於身後,削秩僕碑,以示誡也。

至朝鮮國王謚號,曏亦內閣撰擬,嗣以所擬之字有觸其國王先代名諱,則改由其國自擬八字以進,請帝裁定云。

外籓賜恤順治十三年,定蒙籓親王等喪,遣官★祭文至塋所宣讀致祭,喪主率屬跪迎。禮畢,望闕謝恩,行三跪九拜禮。自王以下,致祭如前儀,唯牲醴物品,則依其爵為隆殺。著有勛勞者,建碑優恤,特遣大臣、侍衛,出自恩旨。親王、郡王福晉喪,遣祭如儀。貝勒至公夫人,並遣祭,無祭文。

其朝鮮國王母妃、王妃、世子喪訃至,禮臣請賜恤,遣正、副使賚祭品、香鈔諭祭。乾隆五十一年,國王世子李喪,禮部奏聞。詔以朝鮮世守籓封,最稱恭順,命倍給祭品,示優恤。嘉慶十年,國王李鍚曾祖母莊順王妃訃至,賜祭一次。

琉球、越南國王卒,告哀,遣使諭祭,並給銀絹。母、妃、世子喪,俱不告哀,不賜恤。使臣來京病歿,則題請恤典,賜棺及祭,歸葬者聽。

品官喪禮定制,有疾遷正寢,疾革書遺言,三品以上官具遺疏,既終乃哭。立喪主、主婦。護喪諸執事人治棺,民公採板,侯、伯、一品官以下硃棺。訃告。設尸床、帷堂,陳沐具。乃含。三品以上用小珠玉,七品以上用金木屑五。襲衣,常服一稱,朝衣冠帶各以其等。明日小斂,陳斂床堂東,加斂衣,三品以上五稱,衣復三、襌二;五品以上三稱,衣復二、襌一;六品以下二稱,衣復、襌各一:皆以繒。衣復衾一。又明日大斂蓋棺,設靈床柩東,柩前設靈座,陳奠幾,喪主及諸子居苫次,族人各服其服。

朝夕奠肴饌,午餅餌。遇朔望,則朝奠具殷奠,肴核加盛。初祭,陳饌筵羊酒,具楮幣。公筵十五席,羊七,楮四萬;侯筵十二,楮三萬六千;伯筵十二,楮三萬二千:羊俱六。一品官筵十,羊五,楮二萬八千;二品筵八,羊四,楮二萬四千;三品筵六,楮二萬;四品筵五,楮萬六千:羊俱三。五品筵四,楮萬二千;六、七品筵三,楮萬:羊俱二。

族人齊集,喪主以下再拜,哭奠如禮。卒奠,大功者易素服,大祭同。初祭,期服者易素服,百日致奠薙發,三月而葬。

一品塋地九十步,封丈有六尺,遞殺至二十步封二尺止。繚以垣。公、侯、伯周四十丈,守塋四戶;二品以上周三十五丈,二戶;五品以上周三十丈,一戶;六品以下周十二丈,止二人守之。公至二品,用石人、望柱暨虎、羊、馬各二,三品無石人,四品無石羊,五品無石虎。其墓門勒碑,公、侯、伯螭首高三尺二寸,碑身高九尺,廣三尺六寸,龜趺高三尺八寸。一品螭首,二品麒麟首,三品天祿闢邪首。四至七品圓首方趺,首視公、侯、伯遞殺二尺至尺八寸止,碑身遞殺五寸至五尺五寸止,廣遞殺二寸至二尺二寸止,趺遞殺二寸至二尺四寸止。刻壙志用石二片,一為蓋,書某官之墓,一為底,書姓名、鄉里、三代、生年、卒葬月日及子孫葬地。婦人則隨夫與子孫封贈。二石相鄉,鐵束,埋墓中。

制柩轝,上用竹格,結以採,旁施帷幔,四角垂流蘇,繒荒、繒幃並青藍色。公、侯、伯織五採,一、二品用銷金,五品以上畫雲氣,六、七品素繒無飾。承以槓,五品以上魨硃,六、七品飾紅堊,障柩畫翣,五品以上四,六、七品二。引布二,功布一,靈車一,明器則從俗。

諏日發引,前夕祖奠,翌日遣奠,會葬者畢集。公鞍馬八,遞殺至二數。儀從前導,引以丹旐、銘旌,滿用丹旐,漢用銘旌。至墓所,乃窆。祀后土,題主,奉安,升車,反哭,乃虞。羊、酒、楮帛各視其等。祭畢,柔日再虞,剛日三虞。百日卒哭,次日祔家廟。期年小祥,再期大祥,遷主入廟。祝讀告辭,主人俯伏五拜。訖,改題神主,詣廟設東室,奉祧主藏夾室。乃徹靈座。後一月禫。喪至此計二十有七月。喪主詣廟祗薦禫事。

其在外聞喪者,訃至,易服,哭,奔喪。至家憑殯哭,翌日成服。喪期自聞訃日始。餘同。期以下聞喪,易服為位而哭,奔喪,則至家成服。官在職,非本生父母喪,雖期,猶從政,不奔喪。聞訃,易服為位而哭,私居持服,入公門治事仍常服。期喪者,期年不與朝、祭。服滿,則於私居為位哭,除之。

順治九年,定百官親喪祭禮以其子品級,子視父母,命婦視夫同。

康熙二十六年,禁居喪演戲飲博。凡官卒任所,或父母與妻喪,許入城治事。

乾隆間,諭京旗文武官遇親喪,百日後即入署治事,持服如故。罷與祭祀、朝會。

道光二十四年,定民公以下、軍民以上居喪二十七月,不宴會、作樂,不娶妻、納妾,門戶不換舊符。

宣統元年,禮部議畫一滿、漢喪制,自是滿官親喪去職,與漢官一例矣。

士庶人喪禮順治初年,定制,士、庶卒,用硃棺,櫬一層,鞍馬一。初祭用引幡,金銀楮幣各一千,祭筵三,羊一。大祭同。百日、期年祭,視初祭半之。一月殯,三月葬。墓祭紙幣、酒肴有定數。通禮,士斂衣衣復襌各一,衣復衾一,襲常服一稱,含用金銀屑三,用銘旌。庶人衣復衾一,含銀屑三,立魂帛。士塋地圍二十步,封高六尺。墓門石碣,圓首方趺。壙志二,如官儀。柩轝上竹格垂流蘇,槓飾紅堊,無翣。引布二,功布一。靈車一。明器從俗。庶人塋地九步,封四尺。有志無碣。轝以布衾覆棺,不施幃蓋。槓兩端飾黑,中飾紅堊。餘略仿品官,制從殺。

雍正初元,定軍、民故者,前後斂衣五襲,鞍馬一。初祭,祭筵二,羊一,大祭同,常祭減半。棺罩生、監用青絹,軍、民春布。

十三年,詔曰:「朕聞外省百姓喪葬侈靡,甚至招集親鄰,開筵劇飲,名曰鬧喪,且於喪所殯時雜陳百戲。匪唯背理,抑亦忍情。」敕督撫嚴禁陋習,違者治罪。又諭:「吉兇異道,不得相干。故娶在三年外而聘在三年內者,春秋猶以為非。三年之喪,創深痛鉅。乃愚民不知禮教,慮服喪後不獲嫁娶,遂乘父母疾篤或殯斂未終而貿然為之者,朕甚憫焉。自今伊始,齒朝之士,下逮生監,毋違此制。其皁隸編氓,窮而無告,父母臥疾,賴子婦治饔飧者,任其迎娶盥饋,俟疾瘳或服竟再成婚禮。」古者禮不下庶人,其斯之謂歟?曾子問:「親迎在途而婿之父母死,女改服布深衣、縞總以趨喪。」亦此義也。

服制順治三年,定喪服制,列圖於律,頒行中外。道光四年,增輯大清通禮,所載冠、服、絰、屨,多沿前代舊制。制服五:曰斬衰服,生麻布,旁及下際不緝。麻冠、絰,菅屨,竹杖。婦人麻屨,不杖。曰齊衰服,熟麻布,旁及下際緝,麻冠、絰,草屨,桐杖。婦人仍麻屨。曰大功服,粗白布,冠、絰如之,繭布緣屨。曰小功服,稍細白布,冠、屨如前。曰緦麻服,細白布,絰帶同,素屨無飾。

敘服八:曰斬衰三年,子為父、母;為繼母、慈母、養母、嫡母、生母;為人後者為所後父、母;子之妻同。女在室為父、母及已嫁被出而反者同;嫡孫為祖父、母或高、曾祖父、母承重;妻為夫,妾為家長同。

曰齊衰杖期,嫡子、眾子為庶母;子之妻同;子為嫁母、出母;夫為妻;嫡孫祖在為祖母承重。

曰齊衰不杖期,為伯、叔父、母;為親兄、弟;為親兄、弟之子及女在室者;為同居繼父兩無大功以上親者;祖為嫡孫;父、母為鏑長子及眾子;為嫡長子妻;為女在室者,為子之為人後者;繼母為長子、眾子;孫為祖父、母;孫女在室、出嫁同;女出嫁為父、母;為人後者為其本生父、母;女在室或出嫁而無夫與子者為其兄、弟、姊、妹及侄與侄女在室者;女適人為兄、弟之為父後者;婦為夫兄、弟之子及女在室者;妾為家長之父、母與妻及長子、眾子與其所生子。

曰齊衰五月,為曾祖父、母,女雖適人不降。

曰齊衰三月,為高祖父、母,女雖適人不降;為繼父昔同居者;為同居繼父兩有大功以上親者。

曰大功九月,祖為孫及孫女在室者;祖母為諸孫,父、母為諸子婦及女已嫁者;伯、叔父、母為侄婦及侄女已嫁者;為人後者為其兄、弟及姑、姊、妹在室者;既為人後,於本生親屬皆降一等;為人後者之妻為夫本生父、母;為己之同堂兄、弟及同堂姊、妹在室者;為姑、姊、妹已嫁者;為兄、弟之子為人後者;女出嫁為本宗伯、叔父、母;為本宗兄、弟及其子;為本宗姑、姊、妹及兄、弟之女在室者;妻為夫之祖父、母及伯、叔父、母。

曰小功五月,為伯、叔祖父、母;為同堂伯、叔父、母及同堂姊、妹已嫁者;為再從兄、弟及再從姊、妹在室者;為同堂兄、弟之子及女在室者;為從祖姑及堂姑在室者;祖為嫡孫婦;為兄、弟之孫及孫女在室者;為外祖父、母;為母之兄、弟、姊、妹;及姊、妹之子;為人後者為其姑、姊、妹已嫁者;婦為夫兄、弟之孫及孫女在室者;為夫之姑、姊、妹、兄、弟及夫兄、弟之妻;為夫同堂兄、弟之子及女在室者;女出嫁為本宗堂兄、弟及姊、妹在室者。

曰緦麻三月,祖為眾孫婦;祖母為嫡孫、眾孫婦;高、曾祖父、母為曾、玄孫,為乳母;為族曾祖父、母,族伯、叔父、母;為族兄、弟及族姊、妹在室者;為族曾祖姑及族祖姑、族姑在室者;為兄、弟之曾孫及曾孫女在室者;為再從兄、弟之子及女在室者;為祖姑、堂姑及再從姊、妹出嫁者;為姑之子、舅之子;為兩姨兄、弟;為妻之父、母;為婿;為外孫及外孫女;為兄、弟孫之妻;為同堂兄、弟之妻;為同堂兄、弟子之妻;婦為夫高、曾祖父、母;為夫伯、叔祖父、母及夫祖姑在室者;為夫堂伯、叔父、母及堂姑在室者;為夫同堂兄、弟及同堂兄、弟之妻;為夫同堂姊、妹;為夫再從兄、弟之子及女在室者;為夫同堂兄、弟之女已嫁者;為夫同堂兄、弟子之妻與孫及孫女在宦者;為夫兄、弟孫之妻及兄、弟之孫女已嫁者;為夫兄、弟之曾孫及曾孫女在室者;女已嫁為本宗伯、叔祖父、母及祖姑在室者;為本宗從伯、叔父、母及堂姑在室者;為本宗堂兄、弟之子及女在室者。

乾隆四十年,高宗特旨允以獨子兼祧,於是始定兼祧例。兼祧者從權以濟經,足補古禮之闕。會典服制別大宗、小宗,以大宗為重。大宗依服制本條持服,兼祧依降服持服。

道光九年,禮臣增議兩祧服制,以獨子之子分承兩房宗祧者,各為父、母服斬衰三年,為祖父、母服齊衰不杖期。父故,嫡孫承重,俱服斬衰三年。其本身為本生親屬俱從正服降一等,子孫為本生親屬祗論所後宗支親屬服制。

同治十年,允禮臣請,兼祧庶母服制,依定制為兼祧父、母服期,為兼祧庶母服小功。其以大宗子兼祧小宗與以小宗兼祧大宗者,以大宗為重。為大宗庶母服期年,小宗庶母服小功。其以小宗兼祧小宗者,以所生為重,為本生庶母服期年,為兼祧庶母服小功。至出嗣而非兼祧者,以所後為重,為所後庶母服期年,為本生庶母服小功。既降期而服小功,其兼祧庶母為兼祧子持服亦如之。


\end{pinyinscope}