\article{列傳三十四}

\begin{pinyinscope}
彭而述陸振芬姚延著畢振姬

方國棟於朋舉王天鑒趙廷標

彭而述,字子籛,河南鄧州人。明崇禎進士,官陽曲知縣,母憂歸。順治初,英親王徇湖廣,薦為提學僉事,遷永州道參議。孔有德定湖南,薦而述授貴州巡撫,予兵三千以行。次靖州,降將陳友龍叛,圍州城,而述夜開西門出,營山下,選勁騎乘霧沖陣,賊潰且走,副將賀進才戰死。城兵大噪,欲與友龍合,而述拔眾退守寶慶,告有德益師,與賊相持紫陽河上。永州陷,劾免官。

吳三桂征水西土司安坤,而述謀曰:「烏蒙、烏撒、鎮雄、東川四府與水西為脣齒,土司隴安籓又與安氏婚媾。今四府雖名內附,狼子野心,勢必顧惜其種類。以水西之強,而安籓與四府附之,安坤未易制也。莫如先定四府,馘安籓,然後西南可無患。」三桂用其策,誅安坤。遷廣西右布政使。三桂薦為雲南左布政使,而述乞歸,三桂留之,會有詔召,遂行,出會城三十里,一夕無疾卒。

陸振芬,字令遠,江南華亭人。順治六年進士。時兩粵未平,廷議破格用人,即新進士中遴才除道府。振芬授廣東惠潮道副使,從師南征。是冬,克南雄。七年春,度大庾嶺,次韶州。韶州以南望風降,進規會城,既下,振芬與總兵郭虎率師赴惠州,剿撫歸善、海豐諸寨。將至,諸寨窺兵寡,出拒。振芬選精銳數百人繞出其旁擊之,獲一隊,諸寨皆懼。於是諭以禍福,降者踵至。至海豐,守者抗不下。振芬與虎駐五坡驛,他將自羊氾嶺會師合攻之,遂克其城。碣石衛亦降。

八年,抵潮州,上官,聯結諸鎮,檢制土官,招集流亡,簡省徭役,民始有更生之樂。亂甫定,用法嚴,郡縣輒濫禁無辜。振芬與屬吏約,期五十日清庶獄,囹圄為空。九年,會師復平遠,總兵郝尚久故降將,陰持兩端,聞將改授水師副總兵,結山海諸寇僭立帥府。振芬牒大吏策弭變,不應。十年春,尚久自署新泰侯,舉兵圍道署。振芬諭以大義,不從,使告變。秋,固山兵至,振芬約為內應,引外兵入,誅尚久。事平,引疾歸里。家居四十年乃卒。

姚延著,字象懸,浙江烏程人。順治六年進士,除廣西慶遠知府。從師南征,調柳州,有守御功,又調平樂。遷廣東嶺南道副使,撫僮寨,擢江南按察使。

十六年,鄭成功內犯,陷鎮江,入攻江寧。延著佐總督郎廷佐繕守備,安輯危城,閭閻不擾。民間時有羊尾黨,事發,株連數百人。延著謂廷佐曰:「寇在門,不可興大獄、搖人心。」獄乃解。當事急,人多疑貳,民間有宿怨,輒誣以通敵。延著嚴治反坐,多所全活。城民有升高而望者,邏者執之,總管喀喀木以為敵諜,延著力爭,得不死。喀喀木部兵擾城市,延著捕得械斃之。吏卒私掠被難婦女,延著親駐江幹,召其家,遣還者一千七百人,以此忤喀喀木。事定敘功,擢河南左布政使。旋以憂歸,而金壇獄起。

鎮江之陷也,屬縣戒嚴。金壇知縣任體坤集縣中士大夫王重、袁大受等謀遣諸生十輩詣鎮江乞緩兵。丹徒亂民王再興兵起,復令書吏、耆民數十人送款,盡竊庫帑以遁。喀喀木等擊敗成功,體坤乃復至縣,賂重、大受謁大吏,謂士民送款,冀掩棄城罪。重、大受居鄉多不法,為諸生所撓。至是欲以叛坐諸生,洩私怨,列姓名以上。巡按馬勝聲疏聞,下廷佐令延著鞫其獄。延著縶縣吏李鍾秀,訊得實,欲但坐體坤,餘皆減罪。大受騰書京師為蜚語,欲並陷延著,御史馮班發其狀。時侍郎尼滿奉詔勘提督馬逢知獄,命即訊,乃坐重、大受及諸士大夫集議者。諸生及書吏、耆民送款者皆斬,體坤以被逼迫減為絞。巡按何可化又疏劾延著讞從叛罪人史記青、管得勝傅輕比,又有王天福、韓王錫並縱不擬罪,與金壇獄並論,亦坐絞。時喀喀木主軍事,新破敵,尤威重,素不慊於延著。民間謂延著之死,喀喀木實主之。就刑日,江寧為罷市,士民哭踴。喪歸,數百里祭奠不絕,建祠雞鳴山下私祀焉。

子淳燾,康熙六年進士,授內閣中書舍人。伏闕上書為延著訟冤。累擢湖廣提學道僉事,坐事罷,未行,值叛卒夏逢龍之亂,誓死不為屈。事聞,復官,授岳常澧道副使。卒。

畢振姬,字亮四,山西高平人。順治三年進士,授平陽教授。入為國子監助教,累遷刑部員外郎。曹事暇,獨坐陋室,布被瓦盆,讀書不稍倦。

十年,出為山東濟南道參議。歲旱,流民踞山谷為盜,振姬晝夜馳三百里往諭之,悉就撫,全活者七千餘人。泰山香稅,歲羨餘七千金,例充公使錢,振姬悉以佐餉。調廣東驛傳道僉事。時三籓使命往來絡繹,胥吏乘以私派折價,民苦之,振姬一繩以法,閱數月,減船數百,減費七萬有奇。調浙江金衢嚴道參政,擢廣西按察使。所至以廉能聞。遷湖廣布政使,乞病歸。

康熙中,詔舉博學鴻儒,左都御史魏裔介、副都御史劉楗疏薦之。十八年,命廷臣舉清廉吏,裔介復疏言:「振姬清操絕世,才略過人。請告十餘年,躬耕百畝,讀書不輟。」楗亦言:「振姬居官不染一塵。歸日一僕一馬,了無長物,真學行兼優之人。」下部議,以振姬老,置勿用。尋卒。

方國棟,字幹霄,順天宛平人。順治三年舉人,授蠡縣教諭。入為國子監助教,累擢至刑部郎中。

十六年,出為廣東海北道僉事。海寇鄧耀居島中,時出剽掠。國棟以三千人分五道進剿,檄鄰道出兵扼要隘,擒耀,解散餘黨。事平,雷、廉兩部諸富人為賊所誣,械系者眾,國棟察其冤,為辨雪。諸富人裒千金為報,國棟曰:「吾憫若無辜,柰何污我?」卻之。

遷山西寧武道參議。康熙六年裁缺,改江南蘇松常道參議。太湖堤岸傾圮,率吏民修葺,修沿海墩臺及吳淞、劉河兩徬,工費不擾民。師下閩、粵,徵調旁午,國棟一意與民休息,每遇急徵,從容部署。芻茭糧糗,預儲以待,軍興無乏,閭左晏然。戒屬吏無朘民,郡縣稍稍知斂戢,不敢事剝削。

連歲用兵,度支不給,詔各省籌裕餉之策。國棟言:「古今生財之說,開與節二者而已。議開於今日,已無可加,當議節,自朝廷始。舊制,江南歲市布五萬匹供宮府賚予,宜可罷,歲省帑金三萬。」議上,報可,滿洲兵駐防蘇州,議築營舍於王府基,當城中。國棟以兵民雜居難久安,持不可,乃改營南城隙地,民便之。宜興善權山中寺僧與豪族爭地,聚眾焚寺殺僧,知縣告亂,大吏將發兵。國棟單騎馳往,得首禍寘法,餘無所問。吳俗健訟,喜投缿告密,國棟輒不問,即有所案,亦從寬。馭吏嚴,而拊循士民具有恩意。十六年,卒。吳民思之,建祠虎丘山麓以祀。

於朋舉,字襄子,江南金壇人。順治六年進士,改庶吉士,散館授檢討。十二年,出為河南睢陳道副使,政不擾民。郾城盜殺縣官而逸,士民洶洶,謂城將受屠。朋舉馳至,撫諭毋恐。營將以兵至,拒不使入城。大吏召朋舉詰責,對曰:「郾城令,朋舉婦翁也。豈不欲甘心是盜?獨柰何苦良民!」大吏悟,止兵,亦得盜正其罪。

遷福建福寧道參政。興化瀕海,鎮將所部皆群盜受撫者。有材官辱張氏僕,張氏以告。鎮將撻材官,部卒大譁,毀張氏之室,欲劫鎮將為亂。鎮將避去,則縊被撻者寘張氏,謂其僕殺之。朋舉甫到官,廉得首惡,猝縛至,集文武吏會鞫,健兒帶刀環立瞋視。朋舉從容曰:「若曹乾軍法,罪重。念若曹約束無素,但用殺人律,罪有專屬。」眾乃泥首,言殺人者為張氏僕。朋舉曰:「若曹氣焰何等,彼能於千百健兒中奪一人縊之耶?」召訊證者,俱吐實,誅三人而事定。泉州提督剿海盜,盜逸入興化界,鎮將獲數百人。朋舉視其嘗薙發者,曰:「此良民被陷,當宥。」有年少者,曰:「童穉何知,又當宥。」全活甚眾。

鄭成功屯廈門,與漳州隔海相望。固山額真駐會城,遣兵戍漳州,番代歲四易,民苦供役。朋舉請駐防無屢更,不許;固請展其期,歲再易,民稍蘇息。擢四川按察使、山東右布政使。父憂歸。

起授湖南布政使。上官,見胥吏至數百,曰:「兵初罷,民方重困。此曹鮮衣美食,縱橫市井間,何所取諸?」汰其十九,擇謹願者,取足供文書而已。數為大吏言地方利病,有司賢不肖積與之忤,被劾鐫級,未行,而大吏以貪敗。士民惜之。尋卒。

王天鑒,字近微,直隸萬全人。順治三年進士,授山東恩縣知縣。縣接直隸界,自明季為盜藪,嘗一歲七被寇。天鑒上官,諭父老曰:「往歲寇至,縣輒不守,由人無固志。自今勿復逃,視知縣所向。」俄而寇大至,天鑒坐城上,從容指揮,寇疑有伏,逡巡去。於是葺樓櫓,治城隍,嚴候望,時巡徼,守具大備。按行鄉鄙,舉團練,立砦十有九,枹鼓相聞,久之得步卒萬八千、騎士三百。巡按御史疏聞,令天鑒自治兵。廉得境內賊渠數輩,夜突至其鄉呼之出,賊錯愕不能遁,皆誅之。寇據曹縣,巡撫檄天鑒與諸道兵會剿,率所部為前鋒,冒矢石深入,諸軍踵之,復其城。嘗以輕騎逐賊,日暮被圍,短兵相接,手格殺數賊,潰圍出,不失一騎。在恩四年,屢與寇戰,俘馘無算,降者安撫之。寇遠遁,招徠屯種,流亡復歸,墾荒千八百頃。建書院,弦誦不輟。政聲為山東最,上考,內遷禮部主事。十一年,始行耤田親耕禮,天鑒參酌古今,悉合禮宜。累遷郎中。主山東鄉試。十二年,出為陜西河西道參議。與屬吏約,毋獵民枉法。

天鑒固長治兵,按籍討軍實,誡將弁毋以軍糈肥私橐。性剛介負氣,數忤上官。歲餘,謝病歸。絕跡公府,門下士或有餽遺,不受,曰:「飭簠簋,惜名節,足以報舉主矣!」康熙初,大臣薦,不出。尋卒。

趙廷標,浙江錢塘人。順治三年,以拔貢生授福建永定知縣。廣東大埔逸寇江龍以萬餘人犯縣城,廷標城守。寇穴地入,瀦池水以待,地砲不得發;樹雲梯乘城,於城上懸柵墮之。持三月,食垂盡。值立春,廷標張鼓樂,開城門,迎春東郊。寇疑有伏,引去。密遣兵間道往伏兩山間,出不意夾擊,敗之。進至龍磜寨,捕斬略盡。

擢湖廣衡州同知,署府事。蠲賦墾荒,流亡復業。歲大饑,賑恤有實惠。經略大學士洪承疇薦廷標,十七年,擢雲南迤東道副使。安普諸番為土官所誘,競作不靖。廷標設方略、行間,解散之,遂復維摩舊地。移檄諭寧州彌勒、巴盤、八甸,罷捕逐之令,令諸持田器者皆為良民,持兵者乃為賊。巡行安撫,諸路悉平。治迤東十八年。康熙中,調廣東廣肇南韶道副使。安普民、蠻聞其去,塹道塞城留之。慰諭再三,乃得行。

兩粵八排諸山寇聞廷標來,望風解散。連州亂,至,立就撫。逾年以憂去。起湖南驛鹽道副使。捕治劇寇,誅其渠,餘悉縱歸農。湖南方用兵,芻茭械仗,儲峙供給,不誤晷刻,民不困役。兼攝糧道。會湘東民變,巡撫韓世琦令廷標往撫之。單騎馳諭,皆悔泣聽命,散遣之。事稍定,修岳麓書院,置田稟諸生。嘗行部至衡州,父老羅拜車下,號以「慈母」。俄遷陜西糧儲道參議。已病,值武昌兵變,軍書至,猶強起視事。病篤乞歸,至家卒。

論曰:自置督撫,而兩司權輕,況於各道;然以賢者處之,奉職循理,視民之所急,弭亂解嬈,亦足以為治。而述、振芬、振姬、天鑒皆有才略,根本尤在廉勤。延著、國棟、廷標當治亂用重之日,濟之以寬仁,雖以是罷貶,甚或中危法,而一不自恤,是皆能舉其職者。澤及於斯民,亦已多矣。


\end{pinyinscope}