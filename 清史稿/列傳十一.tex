\article{列傳十一}

\begin{pinyinscope}
張煌言張名振王翊等鄭成功子錦錦子克塽

李定國

張煌言,字玄箸,浙江鄞縣人。明崇禎十五年舉人。時以兵事急,令兼試射,煌言三發皆中。慷慨好論兵事。順治二年,師定江寧,煌言與里人錢肅樂、沈宸荃、馮元颺等合謀奉魯王以海。煌言迎於天臺,授行人。至紹興,稱「監國」,授翰林院修撰。入典制誥,出領軍旅。三年,師潰。歸與父母妻子決,從王次石浦,與黃斌卿軍相犄角,加右僉都御史。

魯王諸將,張名振最強。四年,江南提督吳勝兆請降,煌言勸名振援勝兆,遂監其軍以行。至崇明,颶作,舟覆,煌言被執。七日,有導之出者,走間道復還入海。經黃巖,追者圍而射之,以數騎突出,自是益習騎射。集義旅屯上虞、平岡。諸山寨多出劫掠,獨煌言與?王翊履畝勸輸,戢所部毋擾民。六年,覲王於健跳。七年,名振奉王居舟山,召煌言入。乃以平岡兵授劉翼明、陳天樞,率親軍赴之,加兵部侍郎。八年,聞父訃,浙江提督田雄書招降,卻之。師攻滃洲,名振奉王侵吳淞,冀相牽制。俄,師破舟山,乃奉王入金門,依鄭成功。成功用唐王隆武號,事魯王但月上豚、米,修寓公之敬。煌言嘗謂成功曰:「招討始終為唐,真純臣也!」成功亦曰:「侍郎始終為魯,與吾豈異趨哉?」故與成功所事不同,而其交能固,王亦賴以安居。九年,監名振軍,經舟山至崇明,進次金山。十年,復至崇明,師與戰,敗績。十一年,又自吳淞入江,逼鎮江,登金山,望祭明太祖陵。烽火達江寧,俄,退次崇明。再入江,略瓜洲、儀真,薄燕子磯,尋還屯臨門,皆與名振俱。十二年,成功遣其將陳六禦與名振取舟山,臺州守將馬信約降,煌言以沙船五百迎之。名振中毒卒,遺言以所部屬煌言。

十三年,師再破舟山,煌言移軍秦川,王去「監國」號,通表桂王。十四年,桂王使至,授煌言兵部侍郎、翰林院學士。兩江總督郎廷佐書招煌言,煌言以書報,略曰:「來書揣摩利鈍,指畫興衰,庸夫聽之,或為變色,貞士則不然。所爭者天經地義,所圖者國恤家仇,所期待者豪傑事功。聖賢學問,故每氈雪自甘,膽薪深厲,而卒以成事。僕於將略原非所長,祗以讀書知大義。左袒一呼,甲盾山立,濟則賴君靈,不濟則全臣節。憑陵風濤,縱橫鋒鏑,今逾一紀矣,豈復以浮詞曲說動其心哉?來書溫慎,故報數行。若斬使焚書,適足見吾意之不廣,亦所不為也。」

十五年,與成功會師將入江,次羊山,遇颶,引還。十六年,成功復大舉,煌言與俱,次崇明。煌言曰:「崇明,江、海門戶。宜先定營於此,庶進退有所據。」成功不從。師防江,金、焦兩山間橫鐵索,隔江置大砲,煌言以十七舟翦江而渡。成功破瓜洲,欲取鎮江,慮江寧援至,煌言曰:「舟師先搗觀音門,南京自不暇出援。」成功以屬煌言,煌言所將人不及萬,舟不滿百,即率以西。降儀真,進次六合,聞成功拔鎮江,煌言致書,言當先撫定夾江郡縣,以陸師趨南京,成功復不從。煌言進薄觀音門,遣別將以輕舟數十直上攻蕪湖,分兵掠江浦。成功水師至,會蕪湖已降,趣煌言往撫,部勒諸軍,分道略地,移檄諸郡縣。於是太平、寧國、池州、徽州、廣德及諸屬縣皆請降,得府四、州三、縣二十四。煌言所過,秋毫無犯,經郡縣,入謁孔子廟,坐明倫堂,進長吏,考察黜陟,略如巡按行部故事,遠近響應。

方如徽州受降,聞成功敗,還蕪湖收兵,冀聯合瓜洲、鎮江軍為守計,既,聞成功並棄瓜洲、鎮江入海,煌言兵遂潰。南江總督郎廷佐發舟師斷煌言東下道,書招煌言。煌言拒不應,率餘兵道繁昌,謀入鄱陽湖。次銅陵,師自湖廣至,煌言與戰而敗,撫殘兵僅數百,退次無為,焚舟登陸。自銅城道霍山、英山,度東溪嶺,追騎至,從者盡散。煌言突圍出,變服夜行,至高滸埠,有父老識之,匿於家數日,導使出間道,渡江走建德、祁門亂山間,痁作,力疾行,至休寧,得舟下嚴州。復山行,經東陽、義烏至天臺達海,收集舊部,成功分兵益之,屯長亭鄉,築塘捍潮,闢田以贍軍。使桂王告敗,桂王敕慰問,加兵部尚書。十七年,移軍臨門。十八年,廷議徙海上居民絕接濟,煌言無所得餉,開屯南田自給。

成功攻臺灣,煌言移書阻之,不聽。師下雲南,取桂王。煌言遣其客羅綸入臺灣,趣成功出兵,成功以臺灣方定,不能行;遣使入鄖陽山中,說十三家兵,使之擾湖廣,以緩雲南之師。十三家者,郝永忠、劉體純輩,故李自成部將,竄據茅麓山,衰疲不敢出。康熙元年,煌言復移軍沙堤。成功自攻江寧敗還,取臺灣謀建國。魯王在金門,禮數日薄,煌言歲時供億,又慮成功疑,十年不敢入謁。及聞桂王敗亡,上啟魯王,將奉以號召。俄成功卒,煌言還軍臨門,又有議奉魯王監國者,煌言使勸錦,以李亞子錦囊三矢相勖。

籠島,煌言不可?浙江總督趙廷臣復招煌言,煌言書謝之。煌言孤軍勢日促,或議入。二年,魯王殂,煌言慟曰:「孤臣棲棲海上,與部曲相依不去者,以吾主尚存也。今更何望?」三年,遂散遣其軍,居懸澳。懸澳在海中,荒瘠無人?,南汊港通舟,北倚山,人不能上,煌言結茅而處,從者綸及部曲數人,一侍者、一舟子而已。廷臣與提督張傑謀致煌言,得煌言故部曲,使為僧普陀,伺煌言,知蹤跡,夜半,引兵攀嶺入,執煌言及綸,與部曲葉金、王發,侍者湯冠玉。煌言至杭州,廷臣賓禮之。九月乙未,死於弼教坊,舉目望吳山,嘆曰:「好山色!」賦絕命詞,坐而受刃,綸等並死。煌言妻董、子萬祺先被執,羈管杭州,先煌言死。

綸字子木,丹徒諸生。方成功敗還,綸入謁,勸以回帆復取南都,成功不能用,乃從煌言。又有山陰葉振名,字介韜,嘗謁煌言論兵事,煌言薦授翰林院修撰、兵科給事中。既,復上策,欲擒斬成功,奪其兵,圖興復。煌言死,登越王嶺遙祭,為文六千五百餘言。與綸稱「張司馬二客」。

乾隆四十一年,高宗命錄勝朝殉節諸臣,得專謚者二十六;通謚忠烈百十三,煌言與焉;忠節百八;烈愍五百七十六;節愍八百四十三。祀忠義祠:職官四百九十五,士民千七百二十八。諸與煌言並起者,錢肅樂、沈宸荃、馮元颺,明史並有傅。

張名振,字侯服,應天江寧人。崇禎末,為石浦游擊。魯王次長垣,率舟師赴之,封定西侯。以所部屯舟山,移南田,迎王居健跳所,與阮進、王朝先共擊黃斌卿。斌卿,莆田人,崇禎末為舟山參將,唐王時封伯。名振奉魯王如舟山,不納。既,以王命進侯。斌卿法嚴急,配民為兵,籍大戶田為官田,先後戕荊本澈、賀君堯。王次健跳,令進告糴,又不應。至是,名振破舟山,沈斌卿於海,迎王居焉。使日本乞師,不應。成功襲破鄭彩,名振因聲彩殺熊汝霖、鄭遵謙罪,擊破其餘兵。俄,又襲殺朝先。師攻舟山,名振與煌言奉王南依成功。成功居王金門,名振屯?頭。成功初見名振不為禮,名振袒背示之,?「赤心報國」四字,深入膚,乃與二萬人,共謀復南京,攻崇明,破鎮江,題詩金山而還。復與成功偕出,師次羊山,颶作,舟多損,惟名振部獨完。再攻崇明,復入鎮江,觀兵儀真,侵吳淞,戰屢勝。順治十二年十二月,卒於軍。或云成功酖之。

王翊,字完勛,浙江餘姚人。順治四年,起兵下管,奉魯王破上虞。是時蕭山、會稽、臺州、奉化民兵並起結山寨,無所得餉,則不免剽掠。翊與煌言皆履畝科稅贍兵。陳天樞者,會稽山寨將也,薦劉翼明佐翊,武勇善戰。東徇奉化,師與遇,引卻。魯王授翊官,累進至兵部尚書。復陷新昌,越餘姚,拔滸山。固山額真金礪、浙江提督田雄合兵攻大嵐山。八年七月,翊走還山,團練執以獻,死定海。天樞與翼明攻陷新昌,視火藥驟焚,急投水,月餘死。翼明善大刀,治兵戒毋犯民,翊敗,死於家。

肅樂、宸荃謚忠節,翊謚烈愍,斌卿謚節愍。名振不與,而其弟名揚死舟山,謚烈愍。

鄭成功,初名森,字大木,福建南安人。父芝龍,明季入海,從顏思齊為盜,思齊死。崇禎初,因巡撫熊文燦請降,授游擊將軍。以捕海盜劉香、李魁奇,攻紅毛功?,代領其,累擢總兵。

芝龍有弟三:芝虎、鴻逵、芝豹。芝虎與劉香搏戰死。鴻逵初以武舉從軍,用芝龍功舉甲科進三秩,授都指揮使。累?掌印千戶。崇禎十四年,成武進士。明制,勛?,授錦衣遷亦至總兵。福王立南京,皆封伯,命鴻逵守瓜洲。順治二年,師下江南,鴻逵兵敗,奉唐王聿鍵入福建,與芝龍共擁立之,皆進侯,封芝豹伯。未幾,又進芝龍平國公、鴻逵定國公。

芝龍嘗娶日本婦,是生森,入南安學為諸生。芝龍引謁唐王,唐王寵異之,賜姓硃,為更名。尋封忠孝伯。唐王倚芝龍兄弟擁重兵。芝龍族人彩亦封伯,築壇拜彩、鴻逵為將,分道出師,遷延不即行。招撫大學士洪承疇與芝龍同縣,通書問,?鄉里,芝龍挾二心。三年,貝勒博洛師自浙江下福建,芝龍撤仙霞關守兵不為備,唐王坐是敗。博洛師次泉州,書招芝龍,芝龍率所部降,成功諫不聽。芝龍欲以成功見博洛,鴻逵陰縱之入海。四年,博洛師還,以芝龍歸京師,隸漢軍正黃旗,授三等精奇尼哈番。

成功謀舉兵,兵寡,如南澳募兵,得數千人。會將吏盟,仍用唐王隆武號,自稱「招討大將軍」。以洪政、陳輝、楊才、張正、餘寬、郭新分將所部兵,移軍鼓浪嶼。成功年少,有文武略,拔出諸父兄中,近遠皆屬目,而彩奉魯王以海自中左所改次長垣,進建國公,屯?門。彩弟聯,魯王封為侯,據浯嶼,相與為犄角。成功與彩合兵攻海澄,師赴援,洪政戰死。成功又與鴻逵合兵圍泉州,師赴援,圍解。鴻逵入揭陽,成功頒明年隆武四年大統歷。五年,成功陷同安,進犯泉州。總督陳錦師至,克同安,成功引兵退。六年,成功遣其將施瑯等陷漳浦,下雲霄鎮,進次詔安。明桂王稱帝,號肇慶,至是已三年。成功遣所署光祿卿陳士京朝桂王,始改用永歷號,桂王使封成功延平公。魯王次舟山,彩與魯王貳,殺魯王大學士熊汝霖及其將鄭遵謙。七年,成功攻潮州,總兵王邦俊御戰,成功敗走。攻碣石寨,不克,施瑯出降。成功襲?門,擊殺聯,奪其軍,彩出駐沙埕。魯王將張名振討殺汝霖、遵謙罪,擊彩,彩引餘兵走南海,居數年,成功招之還,居?門。卒。

八年,桂王詔成功援廣州,引師南次平海,使其族叔芝筦守?門。福建巡撫張學聖遣泉州總兵馬得功乘虛入焉,盡攫其家貲以去。成功還,斬芝筦,引兵入漳州。提督楊名高赴援,戰於小盈嶺,名高敗績,進陷漳浦。總督陳錦克舟山,名振進奉魯王南奔,成功使迎居金門。九年,陷海澄,錦赴援,戰於江東橋,錦敗績。左次泉州,成功復取詔安、南靖、平和,遂圍漳州。錦師次鳳凰山,為其奴所殺,以其首奔成功。漳州圍八閱月,固山額真金礪等自浙江來援,與名高兵合,自長泰間道至漳州,擊破成功。成功入海澄城守,金礪等師薄城,成功將王秀奇、郝文興督兵力禦,不能克。

上命芝龍書諭成功及鴻逵降,許赦罪授官,成功陽諾,詔金礪等率師還浙江。十年,封芝龍同安侯,而使齎敕封成功海澄公、鴻逵奉化伯,授芝豹左都督。芝龍慮成功不受命,別為書使鴻逵諭意,使至,成功不受命,為書報芝龍。芝豹奉其母詣京師。成功復出掠福建興化諸屬縣。十一年,上再遣使諭成功,授靖海將軍,命率所部分屯漳、潮、惠、泉四府。

成功初無意受撫,乃改中左所為思明州,設六官理事,分所部為七十二鎮;遙奉桂王,承制封拜,月上魯王豚、米,並厚廩瀘、溪、寧、靖諸王,禮待諸遺臣王忠孝、沈佺期、郭貞一、盧若騰、華若薦、徐孚遠等,置儲賢館以養士。名振進率所部攻崇明,謀深入,成功嫉之,以方有和議,召使還。名振俄遇毒死。成功託科餉,四出劫掠,蔓及上游。福建巡撫佟國器疏聞,上密敕為備。李定國攻廣東急,使成功趣會師。成功遣其將林察、周瑞率師赴之,遷延不即進。定國敗走,成功又攻漳州,千總劉國軒以城獻,再進,復陷同安。其將甘輝陷仙游,穴城入,殺掠殆盡。至是和議絕。

上命鄭親王世子濟度為定遠大將軍,率師討成功。十二年,左都御史龔鼎孳請誅芝龍,國器亦發芝龍與成功私書,乃奪芝龍爵,下獄。成功遣其將洪旭、陳六禦攻陷舟山,進取溫、臺,聞濟度師且至,隳安平鎮及漳州、惠安、南安、同安諸城,撤兵聚思明。濟度次泉州,檄招降,不納;易為書,成功依違答之。上又令芝龍自獄中以書招成功,謂不降且族誅,成功終不應。十三年,濟度以水師攻?門,成功遣其將林順、陳澤拒戰,颶起,師引還。

成功以軍儲置海澄,使王秀奇與黃梧、蘇明同守。梧先與明兄茂攻揭陽未克,成功殺茂,並責梧。梧、明並怨成功,俟秀奇出,以海澄降濟度。詔封梧海澄公,駐漳州,盡發鄭氏墓,斬成功所置官。大將軍伊爾德克舟山,擊殺六禦。成功攻陷閩安城牛心塔,使陳斌戍焉。十四年,鴻逵卒。師克閩安,斌降而殺之。成功陷臺州。

十五年,謀大舉深入,與其將甘輝、餘新等率水師號十萬,陷樂清,遂破溫州,張煌言來會。將入江,次羊山,遇颶,舟敗,退泊舟山。桂王使進封為王,成功辭,仍稱招討大將軍。十六年五月,成功率輝、新等整軍復出,次崇明,煌言來會,取瓜洲,攻鎮江,使煌言前驅,溯江上。提督管效忠師赴援,戰未合,成功將周全斌以所部陷陣,大雨,騎陷淖,成功兵徒跣擊刺,往來剽疾,效忠師敗績。成功入鎮江,將以違令斬全斌,繼而釋之,使守焉;進攻江寧,煌言次蕪湖,廬、鳳、寧、徽、池、太諸府縣多與通款,騰書成功,謂宜收旁郡縣,以陸師急攻南京。成功狃屢勝,方謁明太祖陵,會將吏置酒,輝諫不聽。崇明總兵梁化鳳赴援,江寧總管喀喀木等合滿、漢兵出戰,襲破新軍,諸軍皆奔潰,遂大敗,生得輝猶數萬,棄瓜洲、鎮江,出海,欲取崇明。江蘇巡撫蔣國柱遣兵赴援,化?殺之。成功收餘鳳亦還師御之,成功戰復敗,引還。煌言自間道走免。

上遣將軍達素、閩浙總督李率泰分兵出漳州、同安,規取?門。成功使陳鵬守高崎,族兄泰出浯嶼,而與周全斌、陳輝、黃庭次海門。師自漳州薄海門戰,成功將周瑞、陳堯策死之,迫取輝舟,輝焚舟。戰方急,風起,成功督巨艦沖入,泰亦自浯嶼引舟合擊,師大敗,有滿洲兵二百降,夜沈之海。師自同安鄉高崎,鵬約降。其部將陳蟒奮戰,師以鵬已降,不備,亦敗,成功收鵬殺之,引還。十七年,命靖南王耿繼茂移鎮福建,又以羅託為安南將軍,討成功。十八年,用黃梧議,徙濱海居民入內地,增兵守邊。

成功自江南敗還,知進取不易;桂王入緬甸,聲援絕,勢日蹙,乃規取臺灣。臺灣,福建海中島,荷蘭紅毛人居之。芝龍與顏思齊為盜時,嘗屯於此。荷蘭築城二:曰赤嵌、曰王城,其海口曰鹿耳門。荷蘭人恃鹿耳門水淺不可渡,不為備。成功師至,水驟長丈餘,舟大小銜尾徑進,紅毛人棄赤嵌走保王城。成功使謂之曰:「土地我故有,當還我;珍寶恣爾載歸。」圍七閱月,紅毛存者僅百數十,城下,皆遣歸國。成功乃號臺灣為東都,示將迎桂王狩焉。以陳永華為謀主,制法律,定職官,興學校。臺灣周千里,土地饒沃,招漳、泉、惠、潮四府民,闢草萊,興屯聚,令諸將移家實之。水土惡,皆憚行,又以令嚴不敢請,銅山守將郭義、蔡祿入漳州降。是歲,聖祖即位,戮芝龍及諸子世恩、世廕、世默。

成功既得臺灣,其將陳豹駐南澳,而令子錦居守思明。康熙元年,成功聽周全斌讒,遣擊豹,豹舉軍入廣州降。惡錦與乳媼通,生子,遣泰就殺錦及其母董。會有訛言成功將盡殺諸將留?門者,值全斌自南澳還,執而囚之,擁錦,用芝龍初封,稱平國公,舉兵拒命。成功方病,聞之,狂怒咬指,五月朔,尚據胡?受諸將謁,數日遽卒,年三十九。

成功子十,錦其長也,一名經。成功既卒,臺灣諸將奉其幼弟世襲為招討大將軍,使,引兵至臺灣。諸將有欲拒錦立?於錦告喪。錦出全斌使為將,以永華為咨議,馮錫範為侍世襲者,全斌力戰破之,錦乃入,嗣為延平王。世襲走泉州降。二年,錦還思明。泰嘗與臺灣諸將通書,錦得之,遂殺泰。泰弟鳴駿、賡,子糸贊緒亦走泉州降。詔封鳴駿遵義侯、纘緒慕恩伯,世襲、賡皆授左都督。諸將蔡鳴雷、陳輝、楊富、何義先後舉軍降。錦漸弱。

耿繼茂、李率泰大發兵規取金、?,出同安;馬得功將降卒,並徵紅毛兵,出泉州;黃梧、施瑯出海澄。錦令全斌當得功,遇於金門外烏沙,得功舟三百,紅毛夾板船十四,全斌以二十舟入陣沖擊,紅毛?皆不中,諸舟披靡,得功戰死;而同安、海澄二道兵大勝,直破?門。瑯復進克金門、浯嶼,錦退保銅山。三年,錦將杜輝以南澳降。銅山糧垂盡,全斌亦出降,封承恩伯。錦與其將黃廷堅守。繼茂等復以水師出八尺門,廷與諸將翁求多等以三,載其孥盡入臺灣。改東?萬人降,遂拔銅山,焚之,得仗艦無算。錦與永華及洪旭引餘都為東寧國,置天興、萬年二州,仍以永華綜國政。

詔授施瑯靖海將軍,周全斌、楊富為副,督水師攻臺灣,阻颶,不得進。四年,廷議罷兵。李率泰請遣知府慕天顏諭降,假卿銜,齎敕往。錦請稱臣入貢如朝鮮,上未之許。六年,徵瑯入京師。撤降兵分屯諸省,嚴戍守界,不復以臺灣為意。錦兵亦不出。相安者數年,濱海居民漸復業。

十二年,耿精忠將以福建叛應吳三桂,使約錦為援。十三年,精忠遂反,錦仍稱永歷年號。以永華輔長子克居守,與諸將馮錫範等督諸軍渡海而西,入思明,取同安。錦以族人省英知思明,省英,芝筦子也。集舟航,整部伍,方引軍復出,而精忠與爭泉州。泉州兵內亂,精忠所遣守將潰圍走,迎錦師入,復攻下漳州。精忠遣兵圍潮州,潮州總兵劉進忠降於錦,錦遣其將趙得勝入潮州,擊破精忠兵。

錦更定軍制,以錫範及參軍陳繩武贊畫諸政,諸將劉國軒、薛進思、何祐、許輝、施福、艾禎祥分領各軍。省英為宣慰使,督各郡錢糧,令人月輸銀五分,曰「毛丁」;船計丈尺輸稅,曰「樑頭」。鹽司分筦鹽場,鹽石值二錢,徵餉四錢;餉司科雜稅給軍。復開互市,英圭黎、暹羅、安南諸國市舶並至,思明井裡?火幾如承平時。

十四年,精忠使賀年,錦亦報禮,自是復相結。永春民呂花,保所居村曰「馬跳」,不應徵索,使進忠圍之,三月不下,誘花降而殺之。續順公沈瑞屯饒平,進忠攻之,何祐擊破援兵,遂執瑞及其孥歸於臺灣。海澄公黃梧卒,子芳度保漳州。錦自海澄移軍萬松關,祐亦自潮州攻平和,降守將賴升。芳度孤守漳州,圍合,總兵吳淑以城降,芳度死之,其孥皆殉。

十五年,康親王傑書下福建,精忠降,克泉州,國軒復圍之,兩月不下。李光地迎師自間道赴援,總兵林賢、黃鎬、樸子威以舟師會,國軒退次長泰,隳同安,稍進屯漳州溪西。師進擊國軒,國軒敗,棄長泰走。錦將許輝以二萬人攻福州,壁烏龍江。康親王遣副都統喇哈達等渡江奮擊,破其壘,逐北四十里。興、泉、汀、漳諸郡盡復,惟海澄未下。十六年,師克海澄,錦復破之,遂圍泉州。錦下教?國軒、淑、祐等功。副都統穆赫林等克泰寧、建寧、寧化、長汀、清流、歸化、連城、上杭、武平、永定,凡十縣。喇哈達等解泉州圍,錦撤兵還思明。十七年,康親王遣知府張仲舉招錦,不納。

國軒自長泰退據三汊河、玉洲、水頭、鎮門諸寨,屢遣兵攻石瑪、江東橋。錦又遣其將林耀、林英犯泉州,提督段應舉擊破之,獲耀。吳淑又自石瑪登陸,海澄公黃芳世、都統孟安擊破之,沈其舟。上令復徙濱海民如順治十八年例,遷界守邊。穆赫林、黃芳世會師灣腰樹,攻國軒,師敗績。國軒陷平和、漳平,遂復破海澄,段應舉、穆赫林及總兵黃藍死之。藍,梧族,芳度所遣詣京師奏事者也。國軒進圍泉州。詔趣諸軍合擊,將軍喇哈達、賴塔,總督姚啟聖,巡撫吳興祚,提督楊捷,分道並進,賢、鎬、子威以舟師會,克平和、漳平、惠安,復解泉州圍。啟聖與賴塔等逐國軒至長泰,及於蜈蚣山,大破之,斬四千餘級,進克同安,斬錦將林欽。賴塔又破錦兵萬松關,啟聖、捷及副都統吉勒塔布等,與國軒戰於江東橋、于潮溝,國軒屢敗。副都統瑚圖又擊吳淑於石街,盡焚其舟。錦斂兵退保思明。

詔厚集舟師,規取金、?。十九年,興祚出同安,與啟聖、捷會師,自陸路鄉?門。提督萬正色以水師攻海壇,分兵為六隊前進,自統巨艦繼;又以輕舟繞出左右,發砲毀錦師船十六,兵三千餘入水死,錦將硃天貴引退。正色督兵追擊,斬錦將吳內、林勛。湄洲、南日、平海、崇武諸澳皆下。天貴出降。副都統沃申擊破錦將林英、張志,水陸並進,趨玉洲,國軒走還思明。錦將蘇堪以海澄降。啟聖分遣總兵趙得壽、黃大來從賴塔擊破陳洲、馬洲、灣腰山、觀音山、黃旗諸寨。興祚復與喇哈達等逐錦兵至潯尾,遂克?門、金門,錦還臺灣。二十年,錦卒。

子克,自錦出師時為居守,永華請於錦,號「監國」。年未冠,明察能治事,顧乳媼子錫範等意不屬,先構罷永華兵,永華鬱鬱死;及錦卒,遂共縊殺克,奉錦次子克塽嗣為延平王。

克塽幼弱,事皆決於錫範。行人傅為霖謀合諸將從中起,事洩,錫範執而殺之,並及續順公沈瑞。詔用施瑯為水師提督,與啟聖規取臺灣。二十二年,國軒投書啟聖,復請稱臣入貢視琉球。上趣瑯進兵。時國軒以二萬人守澎湖。六月,瑯師乘南風發銅山,入八罩嶼,攻澎湖,擊沈錦師船二百,斬將吏三百七十有奇、兵萬餘。國軒以小舟自吼門走臺灣。七月吳啟爵持榜入臺灣諭軍民薙發?,克塽使請降,瑯疏聞。上降敕宣撫,克塽上降表,瑯遣侍。八月,瑯督兵至鹿耳門,水淺不得入,泊十有二日,潮驟長高丈餘,?舟平入。臺灣人咸驚,謂無異成功初至時也。克塽及國軒、錫範率諸將吏出降,詣京師,上授克塽公爵,隸漢軍正紅旗,國軒、錫範皆伯爵。諸明宗人依鄭氏者,寧靖王術桂自殺,魯王子及他宗室皆徙河南。上以國軒為天津總兵,召對慰勉。眷屬至,賜第京師。克塽請為成功子聰、錦子克舉等?官,上特許之。光緒初,德宗允船政大臣沈葆楨疏請,為成功立祠臺灣。

李定國,字鴻遠,陜西延安人。初從張獻忠為亂,與孫可望、劉文秀、艾能奇並為獻忠養子。獻忠入四川,遣諸將分道屠殺,定國為撫南將軍。順治三年,肅親王豪格率師入四自重慶而南,四年,破遵義,入貴?川,獻忠死西充。可望與定國等及白文選、馮雙禮率殘州。可望令定國襲破臨安,屠其城,盡下迤東諸郡縣,定國等皆自號為王。居年餘,可望用任僎議,自號為國主。

時能奇已前卒,定國、文秀故儕輩,不相下,而定國尤崛強。六年春,可望密與文秀謀,藉演武聲定國罪,縛而杖之百。已,復相抱哭,令取沙定洲自贖。定國憾可望,念兄事久,未可遽發難,乃率所部攻定洲,定洲降,械以歸,剝皮死。定國兵漸強。可望知不可制,乃通使桂王,思得封爵,彈壓諸將。桂王封可望公,尋進為王。定國與文秀亦自侯進公。八年,可望遣使迎桂王。九年,劫遷安隆所。會定南王孔有德師出河池向貴州,可望令定國與馮雙禮將八萬人自黎平出靖州,別遣馬進忠自鎮遠出沅州,兩軍會武岡,圖桂林。文秀亦出兵規取成都。可望言於桂王,進定國西寧王、文秀南康王。

定國自靖州進陷沅州,再進,陷寶慶,遂破武岡,與雙禮兵合。有德引師還桂林。定國使張勝、郭有銘為前鋒,趨嚴關,而令雙禮與高文貴、靳統武繼其後。有德遣兵逆戰驛湖,敗績,陷全州。定國與王之邦、劉之講、吳子聖、廖魚、卜寧率所部自西延大埠疾馳鄉桂林,勝、有銘已破嚴關。有德率師出戰,定國軍中象陣略退,斬馭象者以徇,所部戰甚力,驅象突陣,有德敗績,退保桂林。定國晝夜環攻,城陷,有德自殺。定國分兵徇廣西諸郡縣,梧州、柳州皆下,又遣白文選攻陷辰州。大將軍敬謹親王尼堪率師南征,次湘潭。馬進忠引退,師從之,次衡州。定國赴援,兩軍同時至,戰衡州城下,定國敗走。敬謹親王自率精騎追之,遇伏,沒於陣。定國收兵屯武岡。

定國轉戰廣西、湖廣,下數十城,兵屢勝,可望益嫉之,次沅州,召定國計事,將以衡州敗為定國罪而殺之。定國察其意,辭不赴。十年,率進忠等犯永州。大將軍、貝勒屯齊率師自衡州赴之,未至,定國度龍虎關復入廣西,次柳州。可望會雙禮追定國,自靖州進次寶慶。貝勒屯齊遣兵自永州要擊,可望敗走,還貴陽。定國自柳州道懷集,攻肇慶。師自廣州赴援,戰四會河口,定國兵敗,移軍破長樂,行略高、雷、廉三府,悉屬於定國。

桂王在安隆,馬吉翔為政,遙奉可望指。可望謀自帝甚急,王懼,與大學士吳貞毓謀,定國感泣,議奉迎,青陽密使報王。王復遣周官鑄「屏翰親?,密遣林青陽敕定國統兵入臣」金印賜之,定國拜受命。十一年,事為吉翔聞,啟可望,可望怒,遣其將鄭國按治,殺貞毓、青陽及諸與謀者凡十八人,獨官走免。定國發兵陷高明,進圍新城。平南王尚可喜、靖南王耿繼茂赴援,次三水,將軍珠瑪喇以師會,戰於珊洲,定國兵敗,退保新會。師進擊之,定國敗走。十二年,師進次興業,再進次潢州江上。定國戰屢敗,乃道賓州走南寧。可喜等撫定高、雷、廉三府及廣西橫州。十三年,師進攻南寧,定國戰復敗,將道安隆入雲南。可望詗知之,遣白文選移桂王貴陽。文選心不直可望,因密告王曰:「姑遲行,候西府。入雲南。文秀自四川還軍,可望令與諸將?」西府謂定國也。定國至,文選與共奉王自安南王尚禮、王自奇守雲南,亦不直可望,遂與沐天波迓王入居可望廨,進定國晉王,並封文秀、文選皆王,尚禮等公。令文選還貴陽喻意,可望奪文選兵,置之軍中。定國令靳統武收吉翔,將殺之,吉翔哀統武為言於定國,召入謁,叩頭,諂定國,定國薦於王,使入閣,復用事。

十四年,可望舉兵反攻定國,起文選為將,留雙禮守貴陽。定國與文秀率師御之,遇於三岔河。兩軍夾河而陣,文選棄其軍奔定國,可望遣張勝、馬寶自尋甸間道襲雲南,而自將當定國,戰方合,其將馬維興先奔,兵盡潰,可望走還貴陽。定國遣文秀追可望,引軍還雲南,遇勝於渾水塘,獲而殺之,寶降定國。可望至貴陽,雙禮言追兵且至,可望乃詣經略洪承疇降。雙禮盡取其子女玉帛,從文秀歸雲南,桂王進雙禮王、維興等公。

十五年,大將軍羅託自湖南,吳三桂自四川,將軍卓布泰自廣西,三道入貴州。文秀病卒。定國使劉正國、楊武守三坡、紅關諸隘,御三桂,馬進忠守貴州。會王自奇、關有才貳於定國,據永昌舉兵,定國自將擊之。羅託師自鎮遠入,定國不及援,卓布泰亦盡下南丹、那地、獨山諸州,兩軍會貴陽,進忠遁去。三桂師後入,至三坡,正國拒戰,大敗,自水西奔還雲南。師次開州,武迎戰倒流水,亦敗,遂取遵義。王拜定國招討大元帥,賜黃鉞,?謀禦敵。三桂亦入貴陽,大將軍信郡王多尼至軍,會師平越,戒期入雲南。定國與雙禮扼公背,圖復貴州,文選守七星關。三桂師自遵義趨天生橋,出水西,克烏撒,文選棄關走霑益。卓布泰兵次盤江,自下流宵濟,遂入安隆,定國將吳子聖拒戰,敗走。定國以全軍據雙河口,卓布泰師進破象陣,迭戰羅炎、涼水井,定國兵潰,妻子俱散失,諸將竄走不相顧。定國收兵還雲南,奉桂王走永昌。

十六年春,師自普安入雲南會城。定國使靳統武護桂王走騰越,文選自霑益追及定國,定國使斷後,屯玉龍關。師從之,文選戰而敗,自右甸走木邦,師遂克永昌,渡潞江,陟磨盤山。定國使其將竇民望、高文貴、王璽為三伏以待。師半度,以?發其伏,伏起力戰,自卯至午短兵接,死者如堵墻。民望彈穿脅,猶持刀潰圍出,乃死。璽亦死於陣。定國坐山巔督戰,飛?墮其前,土坌起撲面,遂奔,退走騰越。未至,馬吉翔以桂王走南甸。統武還從定國,雙禮渡金沙江走建昌,其部將執以出降。

王,與定國意?桂王入緬甸,定國次孟艮,如木邦,從文選謀,分屯邊境。文選將入異。定國乃移駐猛緬,收殘部,勢稍振。未幾,復移駐孟連。賀九儀招文秀將張國用、趙得勝歸定國。孟艮酋懼定國兼?,攻定國,定國擊破之,遂據其地。號召諸土司起兵,元江土司那嵩應定國,三桂討焉,嵩自焚死。三桂使招九儀,定國執而殺之。國用、得勝皆鞅鞅不為用,定國坐是終不競。十七年,文選自木邦攻阿瓦,求出桂王,不克,引兵會定國孟艮。十八年,合兵復攻阿瓦,定國上三十餘疏迎桂王,為吉翔所阻,不得達。文選使密啟王,得報書。與緬人戰,定國軍稍?,文選引兵橫擊之,緬人大敗,退城守,然終不肯出桂王。復議以舟師攻之,造船,為緬人所焚,乃移兵次洞鄔,國用、得勝挾文選北走,定國還孟艮。文選至耿馬,遇定國將吳三省,方得定國妻子,將歸諸定國,乃合軍駐錫箔,憑江為險。三桂與將軍愛星阿會木邦,倍道深入,文選降。師薄阿瓦,緬人執王歸於我師。

定國自景線走猛臘,遣將入車里、暹羅諸國乞師,皆不應;伺邊上求王消息。康熙元年,聞王兇問,號慟祈死。六月壬子,其生日也,病作,誡其子及靳統武曰:「任死荒徼,毋降!」乙丑,定國卒。統武尋亦卒。嗣興乃與文秀子震率所部出降。

論曰:當鼎革之際,勝國遣臣舉兵圖興復,時勢既去,不可為而為,蓋鮮有濟者。徒以忠義鬱結,深入於人心,陵穀可得更,精誠不可得沫。煌言勢窮兵散,終不肯為逭死之計。成功大舉不克,退求自保,存先代正朔。定國以降將受命敗軍後,崎嶇險阻,百折而不撓,比之擴廓帖木兒、陳友定輩,何多讓焉。即用明史例,次於開國?雄之列。既表先代遺忠,並以見其倔強山海間,遠至三十餘年,近亦十餘年。開創艱難,卒能定於一,非偶然也。


\end{pinyinscope}