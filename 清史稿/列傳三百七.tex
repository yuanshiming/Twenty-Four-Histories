\article{列傳三百七}

\begin{pinyinscope}
籓部三

○四子部落茂明安喀爾喀右翼烏喇特

鄂爾多斯阿拉善額濟訥

四子部落,在張家口外,至京師九百六十里。東西距二百三十五里,南北距二百四十里。東及北蘇尼特,西歸化城土默特,南察哈爾鑲紅旗牧廠。

元太祖弟哈布圖哈薩爾十五世孫諾延泰與其兄昆都倫岱青游牧呼倫貝爾,均稱阿嚕蒙古。昆都倫岱青裔詳阿嚕科爾沁部傳。諾延泰子四:長僧格,號墨爾根和碩齊;次索諾木,號達爾漢臺吉;次鄂木布,號布庫臺吉;次伊爾札木,號墨爾根臺吉。四子分牧而處,後遂為其部稱。

天聰四年,阿嚕諸部長內附,伊爾扎木來獻駝馬貂皮,賜宴,命坐大貝勒代善右以優異之。五年,僧格從征明大凌河,敗錦州援兵,獻俘百餘。賜酒勞飲,給陣獲甲仗。六年,僧格從征察哈爾。七年,索諾木、鄂木布、伊爾扎木相繼獻駝馬,賚甲胄、雕鞍、鞓帶及幣。八年,鄂木布、伊爾扎木復獻駝馬,命諸貝勒以次宴之。尋遣大臣赴碩翁科爾定諸籓牧,以都木達都騰格里克、鄂多爾臺為其部牧界。九年夏,伊爾扎木隨大軍收察哈爾汗子額哲,盡降其眾。冬,獻駝馬、貂皮。崇德元年,宣諭朝鮮,其部伊爾遜德齎書從,遇明皮島兵,擊斬二人,還,得優賚。是年,授鄂木布扎薩克,俾統四子部落。三年,伊爾扎木從征明山東。四年,從征松山。師旋,以前遣兵不及額,又弗朝正,議奪所屬人戶。詔從寬罰牲畜。五年,來朝,賚甲胄、弓矢、採幣。六年,上親征明,圍松山,其部將都爾拜隨大軍設伏高橋及桑阿爾齋堡,追杏山逃卒,獲之。

順治元年,從入山海關,擊流賊李自成。六年四月,追敘所屬昂安導鄂木布等來歸功,予世職。康熙十年,所部歉收,詔以宣府及歸化城儲粟賑之。十三年,調兵協剿陜西叛賊王輔臣,諭嘉其聞命即赴。十四年,由寧夏進剿,尋分防太原、大同。十五年,調赴河南,聽江西大軍檄剿逆籓吳三桂。十七年,以厄魯特額爾德尼和碩齊等掠烏喇特牧,諭嚴防汛。二十一年,詔發大同、宣府儲粟賑所屬貧戶,復以察哈爾牧產贍之。二十九年,選兵赴圖拉河偵噶爾丹。會噶爾丹由喀爾喀河追襲昆都倫博碩克圖袞布,詔移兵駐歸化城,尋撤還。二十四年,諭備兵聽西路軍調。三十五年,隨大將軍費揚古敗噶爾丹於昭莫多,復簡兵百與茂明安兵百防喀爾喀親王善巴汛。三十六年,朔漠平,賜從征及坐塘監牧諸弁兵銀。

雍正九年,從剿噶爾丹策凌。乾隆十一年,賑是部災。十八年,議剿達瓦齊,詔購駝馬送軍。

所部一旗,駐烏蘭額爾濟坡。其爵為扎薩克多羅達爾漢卓哩克圖郡王。同治中,以回匪東竄,命副都統杜嘎爾軍擇駐其地,以當漠南北之沖。徵駝馬備防戍襄臺差,皆較他部為亟。光緒十一年,察哈爾都統紹祺以勘土默特、達拉特爭界事經其部,奏:「四子王旗幫臺駝馬,自同治年間藉詞西北軍興,差役繁重,潛自回旗,至今十餘年之久,屢催罔應。所屬部落,聞私墾者十已七八。請下理籓院嚴催。」詔從之。二十六年,拳、教相仇,是部釀禍頗鉅。事定,議給教堂賠款銀十一萬兩。二十九年,置山西武川同知,以是部及茂明安、喀爾喀右翼寄居人民村落隸之。自回匪平,山西大同鎮練軍駐其地,設防卡。其後綏遠城將軍督辦墾務,貽穀屢奏請飭認墾。三十一年,是部呈因債作抵之忽濟爾圖地一段,請由官局放墾。三十二年,呈所部之察罕依嚕格勒圖地段認墾。有佐領二十。是部與茂明安、

喀爾喀右翼、烏喇特同盟於烏蘭察布。綏遠城將軍節制烏蘭察布、伊克昭二盟,故重大事件皆由將軍專奏焉。

茂明安部,在張家口外,至京師千二百四十里。東西距百里,南北距百九十里。東喀爾喀右翼,西烏喇特,南歸化城土默特,北瀚海。

元太祖弟哈布圖哈薩爾十三世孫鄂爾圖鼐布延圖子錫喇奇塔特,號土謝圖汗。有子三:長多爾濟,次固穆巴圖魯,次桑阿爾濟洪果爾,游牧呼倫貝爾,均稱阿嚕蒙古。多爾濟號布顏圖汗。子車根,嗣為茂明安部長。天聰七年,偕固倫巴圖魯暨臺吉達爾瑪岱袞、烏巴什等攜戶千餘來歸,獻駝馬。八年,臺吉揚固海杜凌、烏巴海、達爾漢巴圖魯、瑚棱、都喇勒、巴特瑪、額爾忻岱青、阿布泰繼至,均賜宴,賚甲胄、雕鞍、銀幣。九年,烏巴海、達爾漢巴圖魯、都喇勒叛逃喀爾喀,遣兵由鄂諾河往剿,至阿古庫克特勒,斬叛屬千餘;追至喀木尼哈,盡俘以還。崇德三年,巴特瑪、瑚棱等從征喀爾喀扎薩克圖汗,偵遁,乃還。嗣徵明山東,及蘇尼特、喀爾喀,皆以兵從。

康熙三年,授車根長子僧格扎薩克,俾統其眾。十三年,調兵剿陜西叛鎮王輔臣。十四年,駐防大同。十五年,調赴河南,聽江西大軍檄剿逆籓吳三桂。十九年,以厄魯特羅卜

藏丹臺吉等掠其部牧產,遣官諭厄魯特察歸所掠。二十七年,噶爾丹侵喀爾喀,諭嚴防汛。二十九年,噶爾丹襲喀爾喀昆都倫博碩克圖袞布,逾烏勒扎河,詔選兵駐歸化城。三十五年,從西路大軍擊噶爾丹。三十六年,朔漠平,賜從征弁兵銀。五十四年,所部歉收,以呼坦和朔儲粟賑之。雍正九年,從剿噶爾丹策凌,分兵赴固爾班賽堪駐防。十年,移駐伯格爾。十三年,撤還。

所部一旗,駐牧徹特塞哩,隸烏蘭察布盟。爵二:扎薩克一等臺吉一,附多羅貝勒一。道光十二年,與土默特爭界,命松筠往勘。八月,覆奏茂明安及達爾漢貝勒等所爭土默特游牧,有乾隆年間原案、原圖,並所設封堆鄂博,向該臺吉等逐加指示,心俱輸服。令按舊定界址各守游牧,毋相侵越。同治中,回匪東竄,是部被擾。九年十二月,綏遠城將軍定安奏獲茂明安等旗肆掠馬賊巴噶安爾等,誅之。十年,茂明安扎薩克綽克巴達爾琥等,以違砲臺站議處。是年,肅州回匪東竄烏拉特境,定安遣侍衛成山統吉林馬隊駐是部。光緒末,綏遠城將軍貽穀督墾,勸諭報地。三十三年,呈交水壕、帳房塔兩處地段認墾。實則是部租給商民墾地頗多,境內漢民村落亦眾。有佐領四。

喀爾喀右翼部,在張家口外,至京師千一百三十里。東西距百二十里,南北距百三十

里。東四子部落,西茂明安,南歸化城土默特,北瀚海。

元太祖十六世孫格哷森扎扎賚爾琿臺吉,有子七,號喀爾喀七旗,分東、西、中三路,以三汗掌之。其第三子諾諾和偉徵諾顏,有子二:長阿巴泰,號鄂齊賴賽因汗,為中路土謝圖汗祖;次阿布琥,號墨爾根諾顏。子三:長昂噶海,襲父墨爾根號;次喇琥里,號達賴諾顏,生本塔爾、巴什希、色爾濟、扎木素、額璘沁;次圖豪肯,號昆都倫諾楞,子車顏都朗,生袞布,皆為喀爾喀中路臺吉,隸土謝圖汗。

順治十年二月,本塔爾以與土謝圖汗袞布隙,偕弟巴什希、扎木素、額璘沁及袞布,率戶千餘來歸。色爾濟獨留喀爾喀,其孫禮塔爾後來歸,授扎薩克臺吉。見土謝圖汗部傳。三月,詔封本塔爾為扎薩克和碩達爾漢親王,統其眾,賜牧塔嚕渾河,與內扎薩克諸部列,是為喀爾喀右翼。其稱左翼者,為貝勒袞布伊勒登,亦自喀爾喀來歸,受封在本塔爾後,互見其傳。

康熙二十五年,喀爾喀扎薩克圖汗沙喇與土謝圖汗察琿多爾濟構釁,遣大臣蒞盟於庫倫伯勒齊爾,由歸化城齎糧往,詔所部扎薩克選駝助運。二十七年,選兵駐邊偵噶爾丹。二十九年,調赴圖拉河,酌留兵之半駐歸化城。三十一年,詔發殺虎口倉粟賑其屬貧戶。三十五年五月,從大將軍費揚古由西路敗噶爾丹於昭莫多,凱旋,詔留軍營餘米給部眾。十月,發軍前馬瘠者留其地飼牧。三十六年,費揚古檄所部兵會大軍於喀爾喀郡王善巴界。師旋,賚從征兵銀。五十四年三月,因久雪傷牧產,詔發呼坦和朔儲粟賑之。雍正九年,大軍剿噶爾丹策凌,詔簡兵駐歸化城。十年,復隨鄂爾多斯郡王扎木揚駐烏喇特西界。十三年,撤還。乾隆四年,遣大臣察閱備調兵,頒賞有差。

所部一旗,駐牧塔嚕渾河。爵四:扎薩克多羅達爾漢貝勒一,由親王降襲;附固山卓哩克圖貝子一,由郡王降襲;固山貝子一;鎮國公一。道光十二年,與土默特爭界,松筠往勘,仍如舊界定之。同治十一年,肅州回匪東竄烏喇特,杜嘎爾遣侍衛永德率兵進駐是部之和林果爾一帶堵截之。四月,杜嘎爾進軍剿竄賽盟阿爾必特公等旗之匪,飭是部與四子部落委員雇覓民駝趣應軍需。光緒末,議興西盟墾務。是部報卓克蘇拉塔一帶地段認墾。有佐領四。

烏喇特部,在歸化城西,至京師千五百二十里。東西距二百十五里,南北距三百里。東茂明安及歸化城土默特,西及南鄂爾多斯,北喀爾喀右翼。

元太祖弟哈布圖哈薩爾十五世孫布爾海,游牧呼倫貝爾,號所部曰烏喇特。子五:長賴噶,次布揚武,次阿爾薩瑚,次布嚕圖,次巴爾賽。後分烏喇特為三,賴噶孫鄂木布,巴爾賽次子哈尼斯青臺吉之孫色棱,及第五子哈尼泰冰圖臺吉之子圖巴,分領其眾,統號阿嚕蒙古。

天聰七年,率屬來歸,貢駝馬。八年,從大軍征明,由喀喇鄂博入得勝堡,略大同,克堡三、臺一。師旋,以柰曼、翁牛特部違令罪各罰駝馬,詔分給所部。嗣征朝鮮、喀爾喀及明錦州、松山、薊州,皆以兵從。順治五年,敘功,時鄂木布、色棱已卒,以圖巴掌中旗,鄂木布子諤班掌前旗,色棱子巴克巴海掌後旗,各授扎薩克,封鎮國公、輔國公爵有差。

康熙二十六年,上閱兵盧溝橋,命其部來朝人從觀。二十七年,噶爾丹侵喀爾喀,諭嚴防汛。二十九年,噶爾丹襲喀爾喀昆都倫博碩克圖袞布,逾烏勒扎河,命選兵駐歸化城。三十年,以自厄魯特來歸之巴圖爾額爾克濟農和囉理叛逃,詔備兵五百偵剿。三十一年,和囉理降,撤所備兵歸。三十五年,從西路大軍敗噶爾丹於昭莫多。三十六年,朔漠平,上由寧夏凱旋。四等臺吉南春迎覲賀捷,稱旨,晉授一等臺吉,並優賚從征及坐塘、監牧、鑿井諸弁兵。三十八年,以其屬有貧為盜者,諭諸扎薩克教養之。五十四年,所部歉收,以呼坦和朔儲粟賑之。雍正九年,大軍剿噶爾丹策凌,諭選兵防游牧。乾隆十九年,議剿達瓦齊,詔購駝馬送軍。

所部三旗,駐牧哈達瑪爾。爵三:扎薩克鎮國公二,輔國公一。是部■事最先。乾隆三十年,即將沿河牧地私租民人耕種。五十七年,以積欠商人二萬兩,允佃種五年之限。道光十二年,扎薩克鎮國公巴圖鄂齊爾充烏蘭察布盟盟長,以茂明安等旗爭地不報歸化城副都統,輒向理籓院越訴,奪盟長。咸豐三年,綏遠城將軍盛塤奏:「烏拉特三公旗生齒日繁,漸形窮苦。賒欠民人債物,及備辦軍臺差使借貸銀錢,無力償還,陸續私租地畝數十處,每處寬長百十里或數十里。酌擬變通,分別應禁應開。」下所司議行。

同治七年,回匪東竄,擾後套,山西大同鎮總兵馬升督兵往昆都侖、溝臺梁一帶防剿。九年,將軍定安奏:「烏拉特河北後套夙稱產糧之區,而糧所由產,皆出於內地民人私種蒙古游牧之地。現金順、張曜、老湘、卓勝各營軍糧無不購買於此。擬請將三公旗游牧墾出地畝,無論應開應禁,均暫準種耕,責令按畝收租,留備各項差使之用。所產糧石供各路軍糈。」時回匪陷磴口,擾及是部後套一帶。二月,諭定安遣宋慶一軍赴舍太一帶剿除北路竄匪。尋鄂爾多斯貝子烏爾那遜督隊擊退。六月,諭定安等勸烏拉特居民趕興耕作,以裕足食之源。十二月,諭金順防範烏拉特三旗地方游弋回匪。十年三月,回匪復自賽音諾顏之阿米爾畢特公旗擾是部中公旗洪庫勒塔拉地方。六月,匪又擾中公旗之什巴克臺。杜嘎爾奏:「吉額、洪額等軍大敗之於布特地方,金運昌遣提督王鳳鳴剿前竄洪庫勒塔拉之匪於奔巴廟、察洪噶爾廟,皆殄之。其後肅州回匪平,烏拉特始息警。自徵回軍興,西路文報及軍需駝馬,皆由是部設臺分段接替,至阿拉善而止。西陲肅清,始復舊制。」

二十三年,山西巡撫胡聘之請開烏拉特三湖灣地方屯墾。既得俞旨,理籓院以蒙盟呈有礙游牧,格其議。二十九年,護山西巡撫趙爾巽、吳廷斌先後奏置五原同知,以是暨鄂爾多斯之達拉特、杭錦兩旗寄居民人村落隸之。時兵部侍郎貽穀督墾,派員勸報地。三十三年,奏烏拉特前旗以達拉特旗東之什拉胡魯素、紅門兔等地段,後旗以黃河西岸之紅洞灣地段,中旗以黃河西岸熟地莫多、噶魯泰兩段報墾,並修壩工,擴渠道,防沖突,暢引灌。仍以民多官少,防範難周,蒙人時有爭渠阻墾情事入告。是部中旗有佐領十六,前旗十二,後旗六。

鄂爾多斯部,在河套內,至京師千一百里。東歸化城土默特,西阿拉善,南陜西長城,北烏喇特。東西北三面皆距河,袤延二千餘里。

元太祖十六世孫巴爾蘇博羅特始居河套,為鄂爾多斯濟農。子袞弼哩克圖墨爾根繼之。有子九,分牧而處,今鄂爾多斯七扎薩克皆其裔。長諾顏達喇襲濟農號,為扎薩克郡王額璘臣一旗祖;次巴雅斯呼朗諾顏,為扎薩克貝勒善丹一旗祖;次偉達爾瑪諾顏,為扎薩克貝子沙克扎、鎮國公小扎木素二旗祖;次諾捫塔喇尼華臺吉,為扎薩克貝子額琳沁一旗祖;次玻揚呼哩都噶爾岱青,為扎薩克臺吉定咱喇什一旗祖;次巴雅喇偉徵諾顏,為扎薩克貝子色棱一旗祖;次巴特瑪薩木巴斡;次納穆達喇達爾漢諾顏;次翁拉罕伊勒登臺吉:皆為濟農,屬察哈爾。

林丹汗虐,其部濟農額琳臣與喀喇沁、阿巴噶諸部長敗察哈爾兵四萬於土默特之趙城。天聰九年,大軍收林丹汗子額哲於黃河西托裏圖地,未至,額璘臣私要額哲盟,分其眾以行。我軍追及之,索所獲,額璘臣懼,獻察哈爾戶千餘。自是所部內附,頒授條約。

順治元年,選兵隨英親王阿濟格赴陜西剿流賊李自成。二年,師旋,得優賚。六年,臺吉大扎木素及多爾濟叛劫我使圖嚕錫。敕曰:「聞爾等背叛,即欲加兵。但念受朕恩有年,且生靈堪惜,故不忍遽用干戈。爾能悔過來朝,即宥罪恩養。儻恃險不即歸順,當發兵窮爾蹤跡,必不容爾偷生。」時額璘臣偕同族固嚕岱青善丹、小扎木素、沙克扎、額琳沁、色棱等,攜自額濟內阿喇克鄂拉徙牧博羅陀海。上嘉其不助逆,詔封郡王、貝勒、貝子、鎮國公有差,各授扎薩克,凡六旗。七年,大扎木素降,詔宥其罪。諭多爾濟降,不從。九年,遣兵擒斬多爾濟於阿拉善。

康熙十三年冬,調所部兵三千五百會剿陜西叛鎮王輔臣。十四年,復神木、定邊、花馬池各城堡,敘功,晉扎薩克等爵,臺吉各加一級。二十七年,噶爾丹侵喀爾喀,奉詔簡兵二千防汛。三十五年,上親征噶爾丹,至所部界,扎薩克等率屬渡河朝御營,獻馬。上手諭皇太子曰:「朕至鄂爾多斯地方,見其人皆有禮貌,不失舊時蒙古規模。各旗俱和睦如一體,無盜賊,駝馬牛羊不必防守。生計周全,牲畜蕃盛,較他蒙古殷富。圍獵嫺熟,雉兔復多。所獻馬皆極馴,取馬不用套竿,隨手執之。水土食物皆甚相宜。」三十六年,允扎薩克等請設站阿都海,軍奏及糧運俱由其地行。時扎薩克等率兵扈蹕,頒賚白金。是年冬,理籓院劾運米遲誤罪,詔寬免。五十一年,諭曰:「鄂爾多斯饑饉洊臻,戶口流散,可速遣官察覈,務令各遂生業。」五十二年,詔定其部牧界。先是郡王松喇布請暫牧察罕托輝,尚書穆和倫等往勘,議於柳苾、剛柳苾、房苾、西苾四臺外,暫令駐牧。至是寧夏總兵範時捷奏:「察罕托輝系版圖內地,蒙古游牧與民樵採混雜,不便。請令仍以黃河為界。」遣官勘,議從時捷所請。五十四年,詔簡兵二千從大軍防御策妄阿喇布坦。五十五年,所部歉收,遣官往賑,凡七千九百餘戶,三萬一千餘丁。雍正元年,復命賑恤。十年,以調赴固爾班賽堪兵三千,不堪用者五百,又中途逃歸四百餘,為將軍達爾濟所劾,論王、貝勒、貝子等罪,各降爵。尋以次予復。

乾隆元年,詔增設一旗,以一等臺吉定咱喇什領之,授扎薩克。是年,允陜西榆林、神木等處民邊種鄂爾多斯餘閒套地完租。四十九年,陜甘總督福康安奏:「黃河改向西流,原在河西民人反在河東。鄂爾多斯蒙古貪利,濫以現行黃河為界,謂民人占據所部游牧地方。」命侍郎賽音博爾克圖往勘,仍如前黃河舊流之地為界,釘椿立碑。

所部七旗,自為一盟,曰伊克昭。與哲哩木、卓索圖、昭烏達、錫林郭勒、烏蘭察布五盟同列內扎薩克。左翼前旗,一名準噶爾旗,駐札勒穀。左翼中旗,一名郡王旗,駐敖西喜峰。左翼後旗,一名達拉特旗,駐巴爾哈遜湖。右翼前旗,一名烏審旗,駐巴哈池。右翼中旗,一名鄂拓克旗,駐西喇布哩都池。右翼後旗,一名杭錦旗,駐鄂爾吉虎泊。後增一旗,曰左翼前末旗,一名扎薩克旗。爵八:扎薩克多羅郡王一;附輔國公一;扎薩克多羅貝勒一;扎薩克固山貝子四,一由鎮國公晉襲;扎薩克一等臺吉一。

是部墾事最早。乾隆以後,是部招墾民人近陜西者,分隸陜西神木、定邊兩理事同知,及神木、府谷、懷遠、靖邊、定邊等縣。近山西者,分隸薩拉齊、托克托城、清水河三,偏關、河曲等縣。而因地滋爭之案亦時有。道光八年,達拉特旗之才吉、波羅塔拉地方,以抵還債項,奏準租給商種五年。十四年,綏遠城將軍彥德奏:「達拉特旗臺吉人等招民私墾驛站草地,致越界侵種,其旗游牧地方貝子親往驅逐。民人恃眾,砍傷二等臺吉薩音吉雅等。」詔山西巡撫鄂順安派員捕治之。其後相沿奉部文而承種者有之,由臺吉私放者有之,由各廟喇嘛公放者有之。開墾頗多,產糧亦盛。

同治初元,回匪役興,辦團練,購糧儲,皆取濟於此。是年,調鄂爾多斯兵赴甘協剿。六年,回匪屢入境,皆為貝子扎那格爾第兵所敗。七年正月,陜西寧條梁之陷,匪遂大入游牧,南自依克沙巴爾、北至固爾根柴達木,焚掠殆遍。要地如古城、答拉寨、十里長灘諸處皆不守。蒙兵不能戰,屢請撤退。四月,綏遠城將軍德勒克多爾濟奏飭扎那格爾第簡精壯蒙兵五百,合準噶爾旗壯丁及察哈爾馬隊各五百,均歸統帶,擇駐神木要隘,相機迎剿。別以達拉特旗兵五百駐適中草地。朝旨飭寧夏副都統金順一軍援之。六月,金順深入蒙地,遇匪於野狐井、門家梁、王家溝,皆捷。嵩武軍統領提督張曜一軍亦赴援,屢挫之,古城、十里長灘之匪皆遁。張曜又敗匪於達拉特旗,進駐古城。而竄杭錦、烏審、郡王等旗之匪,亦為綏遠城將軍所遣達爾濟一軍所敗。是為鄂部七旗初次肅清。綏遠城將軍定安遂奏撤伊克昭盟兵一千九百回本游牧防守,仍留前挑兵五百,令扎那格爾第統帶探賊進剿。十二月,阿拉善之磴口不守,回匪又大入,昭鹽海子、纏金一帶皆被擾。時匪自磴口水路進撲,副都統杜嘎爾派參領成山等合烏爾圖那遜兵分往纏金及阿拉善旗烏蘭木頭地方剿之,匪皆敗遁。六月,張曜自古城進剿,屢敗匪於察罕諾爾、沙金托海,追至賀蘭山,達爾濟、扎那格爾第兩軍擊殄杭錦、達拉特、郡王諸旗之匪。朝旨又增遣宋慶一軍西援。八月,敗擾郡王旗之匪於東嶺,擊退擾烏審、鄂拓克等旗之匪,進至哈拉寨。金順軍磴口,張曜軍寧夏,沿途自舍太至三道河、石嘴山皆駐官軍。宋慶是冬追剿逆於準噶爾、昭鹽海子諸處,悉殄之。九年,金積回匪以官軍攻急,自石嘴北犯,冀梗我運道。於是沙金托海以西匪騎出沒,而準噶爾、杭錦、鄂拓克諸旗復擾。宋慶、達爾濟諸軍復進剿,迭捷。七月,烏審旗管帶官赤樓多爾濟以剿匪陣亡於霍裏木廟,然各旗亦屢挫來擾之匪。梅楞章京扎棟巴等以剿挫陜西懷遠邊外之匪,予優獎。是部再告奠定。至金積蕩平,而警報始息。歷次陣亡蒙旗官兵及出力者,均時予恤獎。其纏金諸地,則山西仍置防戍。

光緒二年,邊外馬賊肆擾,是部達拉特、杭錦等旗地戶商人蹂躪特重,渠廢田蕪,迄不可復。十年,伊克昭盟長貝子扎那濟爾迪呈:「準噶爾旗以頻年荒歉,請開墾空閒牧場一段,東西八十里,南北十五里,收租散賑,接濟窮蒙。」下理籓院議行。以招種民人分隸山西河曲、陜西府穀。時歸化城土默特與達拉特旗以黃河改道爭界,署山西巡撫奎斌、大理寺少卿郭勒敏布以綏遠城將軍斷分之案偏袒土默特,奏劾。命察哈爾都統紹祺往勘,援乾隆五十一年黃河舊漕為斷之諭,以南之地四成歸達拉特,以北之地六成歸土默特。尋經勘定,北自烏拉特界,南至準噶爾界,達拉特應分地周六百四十八里。十二年,伊犁領隊大臣長庚奏纏金等處宜開屯田。山西巡撫剛毅覆奏:「纏金即才吉地,在河北外套伊克昭盟之達拉特、杭錦兩旗牧界。河自改行南道,蒙古始招商租種分佃,修成渠道。西則纏金,計共五渠,東則後套,計共三渠,紆回約二百里,中間支渠曲折蜿蜒,不可枚數。後遭馬賊之擾,不特纏金、牛壩商號不過數家,即後套左右亦只二百餘家。達拉特旗昔歲收租銀十萬,近所收租錢不及三千串。閱伍至薩拉齊之包頭,面與伊克昭盟長貝子扎那吉爾迪籌商,謂當明示各旗,斷不使該旗牧界日久歸於民人。」因上議屯三端:曰分段,曰修渠,曰設官。下所司議,格。二十六年拳匪之案,鄂爾多斯七旗,如達拉特、鄂拓克、烏審、準噶爾各旗,釀禍均重。事定,議有賠款。達拉特一旗至三十七萬兩。教堂欲得銀,蒙旗欲抵地,久未結。

二十八年,命兵部侍郎貽穀辦晉邊墾務,咨調烏、伊兩盟長詣歸化商訂,迄未至,而呈理籓院請免開辦。廷旨下院嚴飭盟長迅與貽穀等會商,不得推諉。於是貽穀等先以贖還達拉特旗教案熟地二千頃給銀十七萬兩者,為墾務入手之策。二十九年,達拉特、杭錦兩旗始派員就議報墾,郡王、鄂拓克、烏審、準噶爾、扎薩克五旗亦相繼報地,而杭錦旗貝子阿爾賓巴雅爾時充盟長,仍請緩辦,堅拒出具交地印文。三十年,貽穀以抗不遵辦,掣動全局劾之,以副盟長烏審旗貝子察克都爾色楞代署。三月,套匪滋事,山西練軍平之。九月,察克都爾色楞等以烏審、扎薩克兩旗公中之地,北起阿拜素、南至巴蓋補拉克一段,歸官報墾,祝皇太后七旬萬壽。予察克都爾色楞郡王銜,沙克都爾扎布鎮國公銜。三十一年二月,阿爾賓巴雅爾復呈悔過情形,報出杭錦旗中巴噶地一段。貽穀奏烏、伊兩盟地皆封建,與察哈爾之比於郡縣者不同,定押荒歲租皆一半歸官,一半歸蒙,別提修渠費。旨下所司知之。七月,貽穀奏:「杭錦、達拉特兩旗地戶將原有各渠報墾歸公,因改長勝渠名長濟,纏金渠名永濟,挑濬深通,老郭等渠以次及之,計可溉田萬頃。後套地必附渠,渠日加多,即地日廣。就現在應收之款,悉歸工作,回環挹注,務竟其功。請各旗押荒地租各款應歸公者,均暫緩提撥,備渠工大修之費。」九月,準噶爾旗協理臺吉丹丕爾不悅於墾,糾眾抗阻,攻劫局所,貽穀遣兵捕治之。三十二年,貽穀奏定郡王等五旗旱地押荒歲租。陜西巡撫恩壽會奏以郡王、扎薩克兩旗墾地置東勝,隸山西歸綏道。三十三年,貽穀蒙譴,復阿爾賓巴雅爾盟長。信勤、瑞良等相繼為墾務大臣。

是部墾事進行未廢。佐領即左翼中旗十七,右翼中旗八十四,左右翼前旗各四十二,左翼後旗四十,右翼後旗三十六,左翼前末旗十三。

阿拉善厄魯特部,至京師五千里。東鄂爾多斯,西額濟訥,南寧夏、涼州、甘州,北逾瀚海接賽音諾顏、扎薩克圖盟。袤延七百餘里,即賀蘭山地駐牧蒙古。

系出元太祖弟哈布圖哈薩爾,與和碩特同族。和碩特舊為四額魯特之一,故稱額魯特部。哈布圖哈薩爾十九傳至圖魯拜琥,號顧實汗。有子巴延阿布該阿玉什,兄拜巴噶斯初育以為子。後自生子二:長鄂齊爾圖,次阿巴賴。游牧河西套,稱西套厄魯特。巴延阿布該阿玉什號達賴烏巴什。子十六,居西套者,曰和囉理,曰墨爾根,曰額爾克,曰都喇勒,曰哈什哈,曰陀音,曰土謝圖羅卜藏,曰博第,曰多爾濟扎布,曰諾爾布扎木素,曰愛博果特,曰鄂木布。和囉理號巴圖爾額爾克濟農,以來歸授扎薩克,賜牧阿拉善,諸昆弟子姓隸之。其居青海者,曰扎布,曰阿南達,曰伊特格勒,曰巴特巴。扎布授扎薩克,領其族。見青海厄魯特部傳。鄂齊爾圖號車臣汗,子三:長額爾德尼,子噶勒丹多爾濟;次噶爾第巴,子羅卜藏袞布阿拉喇布坦;次伊拉古克三班第達呼圖克圖。後皆絕嗣。阿巴賴裔為準噶爾所掠,故不著。

順治四年,鄂齊爾圖遣使貢駝馬。六年,阿巴賴繼至。七年,鄂齊爾圖使至,以喀爾喀煽蘇尼特部長騰機思叛,奏稱:「力能鋤逆,當相機為之。否則亦必修貢如初,不敢稍萌異志。」諭絕喀爾喀,勿私通好。嗣因額爾德尼、噶爾第巴、伊拉古克三班第達呼圖克圖及所部臺吉、宰桑等朝貢,至者相接。

準噶爾臺吉噶爾丹游牧阿爾臺,號博碩克圖汗,覬為厄魯特長。鄂齊爾圖妻以孫女阿努,尋與隙。康熙十六年,噶爾丹以兵襲西套,戕鄂齊爾圖,破其部。鄂齊爾圖妻曰多爾濟喇布坦,與喀爾喀墨爾根汗額列克妻,皆土爾扈特汗阿玉奇女兄也。額列克孫察琿多爾濟號土謝圖濟汗,偵噶爾丹侵鄂齊爾圖兵援之不及,多爾濟喇布坦奔土爾扈特。噶爾丹遣使獻俘,諭曰:「鄂爾齊圖汗與噶爾丹向俱納貢。今噶爾丹侵殺鄂齊爾圖,獻所獲弓矢等物,朕不忍納也。其卻之!」西套厄魯特既潰,或奔依達賴喇嘛,或被噶爾丹掠去。和囉理率族屬避居大草灘,廬幙萬餘,守汛者遣之去,仍逐水草,徒戀處邊外。

有楚琥爾烏巴什者,噶爾丹叔父也。子五:長巴哈班第,次阿南達,次羅卜藏呼圖克圖,次犖章,次羅卜藏額琳沁。噶爾丹以私憾襲殺巴哈班第,執楚琥爾烏巴什及羅卜藏額琳沁等禁之。巴哈班第子罕都為和囉理甥,時年十有三。其屬額爾德尼和碩齊攜之逃,以兵四百掠烏喇特戶畜,竄就和囉理,居額濟訥河。喀爾喀臺吉畢瑪里吉哩諦偵以告。會青海墨爾根臺吉等察獻額爾德尼和碩齊所掠,遣使詰知為準噶爾屬,諭噶爾丹捕額爾德尼和碩齊治罪,並收和囉理歸牧,或非所屬當以告。二十二年,噶爾丹奏和囉理等歸,達賴喇嘛已遣使召請,以醜年四月為限。是年蓋歲在亥。二十三年,罕都偕額爾德尼和碩齊遣使貢,請宥掠烏喇特罪,而和囉理戚屬嘗掠茂明安諸部牧產,前以服罪故宥之。至是諭曰:「和囉理既免罪,額爾德尼和碩齊等著一體赦。所貢準上納。」

先是羅卜藏袞布阿喇布坦避噶爾丹,走唐古特。以達賴喇嘛言,表請賜居龍頭山,轄西套遺眾。命兵部督捕理事官拉都琥往勘。奏言:「龍頭山,蒙古謂之阿拉克鄂拉,乃甘州城北東大山,山脈綿延邊境。山口即邊關,建夏口城,距水蚩川堡五里;山盡為寧遠堡,距龍頭山裏許,有昌寧湖界之。內地兵民耕牧已久,不宜令新附蒙古居。」上可其奏。

羅卜藏袞布阿喇布坦徙牧布隆吉爾,土謝圖汗琿多爾濟以女妻之。事聞,諭廷臣曰:「前鄂齊爾圖汗為噶爾丹所戕,其孫羅卜藏袞布阿喇布坦往求達賴喇嘛指授所居之地,達賴喇嘛令駐牧阿拉克鄂拉,因以為請。鄂齊爾圖汗從子和囉理前沿邊駐牧罾曾,檄噶爾丹收取之,令羅卜藏袞布阿喇布坦與喀爾喀互為犄角。噶爾丹欲以兵向和囉理等,則恐喀爾喀躡之;欲以兵向喀爾喀,則恐和囉理等襲之。此必非噶爾丹所能收取也。」二十四年,和囉理請賜敕印鈐部眾。廷臣以游牧未定,議不允。諭曰:「和囉理等以避亂,故離其舊牧,來至邊境,劫掠茂明安、烏喇特諸部,本應即行殄滅。朕俯念鄂齊爾圖汗世奉職貢,恪恭奔走,兼之彼亦迫於饑困,是以宥其罪戾。又羅卜藏袞布阿喇布坦系鄂爾齊圖汗孫,為和囉理從子,應令聚合一處。其遣官往諭朕旨,度可居地歸並安置,封授名號,給賜金印璽書,以示朕興滅繼絕至意。」理籓院尚書阿喇尼遵旨往諭。和囉理奏:「皇上令臣等聚處,乃殊恩。達賴喇嘛亦謂羅卜藏袞布阿喇布坦居布隆吉爾,地隘草惡,不若與臣同處。臣等欲環居阿喇克山陰,遏寇盜,靖邊疆。令部眾從此地而北,當喀爾喀臺吉畢瑪里吉哩諦牧地,由噶爾拜瀚海、額濟訥河、姑喇柰河、雅布賴山、巴顏努魯、喀爾占、布爾古特、洪果爾鄂隆以內,東倚喀爾喀丹津喇嘛牧,西極高河居之。」

奏至,遣使諭達賴喇嘛曰:「噶爾丹滅鄂齊爾圖汗時,和囉理及羅卜藏袞布阿喇布坦等紛紜離散,來至邊境,又以生計窘迫,妄行劫掠。朕宥其罪,不即發兵剿滅。和囉理等亦戴朕恩,屢請敕印,依朕為命。朕前諭噶爾丹收取,彼約以醜年四月為期,今逾期已數月矣。伊等骨肉分離,散處失所,朕心殊為惻然!鄂齊爾圖汗於爾喇嘛為護法久矣,何忍漠視其子孫宗族至於窮困?今朕欲將伊等歸並安置,爾喇嘛其遣使與朕使偕往定議!」

二十五年,達賴喇嘛奏已遣使,上遣拉都琥往會勘。拉都琥偕達賴喇嘛使約和囉理至東大山北,語之曰:「爾所謂噶爾拜瀚海地,聽爾游牧。外自寧夏所屬玉泉營西羅薩喀喇山嘴,後至賀蘭山陰一帶布爾哈蘇臺口,又自寧夏所屬倭波嶺塞口北努渾努魯山後甘州所屬鎮番塞口,北沿陶蘭泰、薩喇、椿濟、雷琿、希理等地,西南至額濟訥河,俱以距邊六十里為界,畫地識之。」定議:蒙古殺邊民論死;盜牲畜、奪食物者鞭之;私入邊游牧者,臺吉、宰桑各罰牲畜有差;所屬犯科一次,罰濟農牲畜以五九。時罕都及額爾德尼和碩齊請與和囉理同牧。羅卜藏袞布阿喇布坦偵其女兄阿努攜兵千赴藏,道嘉峪關外,懼襲己,備之,以故未即徙。拉都琥奏至,詔以所定地域及罰例檄甘肅守臣知之。蓋自是和囉理屬始定牧阿

拉善。

二十七年,噶爾丹侵喀爾喀,和囉理欲往援,察琿多爾濟乞師於朝。時諭噶爾丹罷兵。使已就道,詔不允和囉理請。而羅卜藏袞布阿喇布坦自率兵援喀爾喀,遇我使於道,宣諭之,亦撤歸布隆吉爾。察琿多爾濟尋為噶爾丹所敗,上復遣使諭噶爾丹,將行,命之曰:「噶爾丹若問和囉理事,爾等宜述醜年之約,並言達賴喇嘛向雖遣使定議,令和囉理與羅卜藏袞布阿喇布坦歸並安置,迄今尚未同居。和囉理雖居游牧邊地,亦未編設旗隊。前喀爾喀與額魯特交惡,和囉理曾請兵討爾。朕仍諭遣之曰:『朕欲使爾等安處游牧而已,豈肯給爾兵耶?』其以是告之,令罷兵。」噶爾丹不從。

二十八年,以羅卜藏袞布阿喇布坦卒,賜祭。其妻及宰桑等請召噶爾丹多爾濟轄部眾,允之。時噶爾丹多爾濟游牧準噶爾界,諭曰:「羅卜藏袞布阿喇布坦屬內附,所遺部眾恐致流亡。噶爾丹多爾濟尚幼,召之恐未即至。著和囉理前往布隆吉爾,暫為約束人民。俟噶爾丹多爾濟至,仍歸本地。務期共相扶掖,勿侵據所部。」噶爾丹多爾濟以所部饑,告不克即徙。詔授諾顏號,遣侍讀學士達琥諭恤所部貧民。其母扎木蘇攜噶爾丹多爾濟至,詔轄羅卜藏袞布阿喇布坦眾,附阿拉善牧。

有拜達者,罕都屬也,偕額爾德尼和碩齊誘其主棄和囉理,私以厄魯特兵千掠邊番。守汛者責之,為所戕,且抗官軍。甘肅提督孫思克以兵屯邊,將剿之。罕都懼,乃降詔宥罪,仍駐牧阿拉善。其叔父羅卜藏額琳沁尋自準噶爾至,奏為噶爾丹所禁十餘年,以準噶爾與喀爾喀戰,乘間脫,挈孥屬千餘至,乞與兄子罕都同居,允之。

三十年,和囉理以不遵旨徙牧歸化城,懼大兵討,叛遁。噶爾丹多爾濟、羅卜藏額琳沁、罕都等從之,分道竄。將軍尼雅漢等招降噶爾丹多爾濟屬納木喀班爾等五十餘戶、和囉理女弟之夫克奇及從者二十一人以聞,詔安置歸化城。時和囉理弟博第游牧中衛邊外,距阿拉善三百餘里,聞其兄叛遁,欲往會偵。副將軍陳祚昌等屯昌寧湖,遣子索諾木至軍,詭稱假道詢南山,否則請牧馬昌寧湖。祚昌知為緩軍計,令挈屬至歸化城。不從,擊之,斬五百餘級,博第僅以身免,走伊巴賴,遇和囉理屬臺吉齊奇克假糧馬,竄額濟訥河。三十一年,和羅理悔罪,降,命仍牧阿拉善。羅卜藏額琳沁、罕都、齊奇克等從和囉理降。尋復叛走。提督孫思克以兵追至庫勒圖,斬四十餘級。齊奇克就擒,詔宥死,附和囉理牧。羅卜藏額琳沁、罕都逸,遇自青海來歸之喀爾喀臺吉阿海岱青班第,掠其貲,復竄哈密。羅卜藏袞布阿喇布坦有女弟曰阿海,始與策妄阿喇布坦議婚,噶爾丹奪之。策妄阿喇布坦怒,噶爾丹徙額琳哈畢爾噶。上聞之,遣員外郎馬迪齎敕諭令絕噶爾丹。道哈密,羅卜藏額琳沁、罕都等偕噶爾丹屬圖克齊哈什哈、哈爾海達顏額爾克以兵劫之,由大草灘毀邊垣遁,為青海臺吉額爾德尼納木扎勒所擊,走死。三十三年,和囉理弟博第率屬百餘降,乞仍與兄同牧,許之,命輯所屬潰散者。未幾,齊奇克復叛遁。和囉理遣所部莽奈哈什哈等以兵追諸耨爾格山,諭之降,不從,擊斬之。

三十五年,所部兵隨西路大軍敗噶爾丹於昭莫多,副都統阿南達奉命設哨,以和囉理屬布爾噶齊達爾漢宰桑瑪賴額爾克哈什哈、齊勞墨爾根薩裡呼納沁齊倫琿塔漢占哈什哈、布達哩杜喇勒和碩齊等,分屯額布格特、阿木格特、昆都倫、額濟訥及布隆吉爾之博羅椿濟敖齊、喀喇莽奈諸地。時噶爾丹多爾濟竄徙嘉峪關外。有哨卒拜格者,其屬也。阿南達召至,遣歸說噶爾丹多爾濟曰:「上待汝恩甚厚,將撫育之,顧叛逃可乎?和囉理棄牧時,汝不能輯屬,故從往。上灼知汝情,念汝祖鄂齊爾圖汗,將玉成汝,汝其思之!」噶爾丹多爾濟遣告曰:「上念臣祖兄,令臣與和囉理接壤居。臣無知,從和囉理叛遁,今悔罪欲死。臣幼,臣母一婦人,未能達。乞以情代奏。」阿南達欲堅內附志,遣使歸,約如期會肅州,諭設哨援哈密,復檄哈密伯克額貝都拉曰:「噶爾丹至汝地,汝即召噶爾丹多爾濟援,勿復疑。」噶爾丹多爾濟遣宰桑阿約等齎降表,表至肅州。會上視師寧夏,阿南達馳疏至,詔優恤所部眾。未幾,唐古特部第巴煽青海諸臺吉盟察罕托羅海,繕軍械助之。檄噶爾丹多爾濟以兵往,辭不赴,遣使俄濟通問策妄阿喇布坦,自攜兵百會阿南達於布隆吉爾。阿南達偵噶爾丹死,其從子丹濟拉竄瀚海,遣噶爾丹多爾濟屬輝特臺吉羅卜藏等駝赴噶斯,而自偕噶爾丹多爾濟以兵繼之。至色爾滕,值俄濟歸,以丹濟拉將自郭蠻喇嘛所往附策妄阿喇布坦告。因撤噶斯兵,遣噶爾丹多爾濟仍赴布隆設哨,其屬阿勒達爾哈什哈、恭格等煽之叛,至西欣驛劫駝馬,奉母札木蘇由吉爾喀喇烏蘇遁。阿南達遣兵四百追之,不及,招降其屬茂海、烏納恩巴圖爾、阿喇木札木巴、阿喇木把及輝特臺吉羅卜藏等,遣歸阿拉善。羅卜藏後徙牧喀爾喀,即附扎薩克圖汗部之厄魯特扎薩克也。是年,和囉理以所部數叛,請視四十九旗例編佐領。廷臣議徙烏喇特界,諭曰:「若將和囉理移牧近地,則沿邊別部蒙古甚多,豈可盡徙?且治蒙古貴得其道,不系地之遠近。著停徙,仍游牧阿拉善地。」詔和囉理為多羅貝勒,給扎薩克印。復以噶爾丹多爾濟竄赴準噶爾,敕策妄阿喇布坦曰:「噶爾丹多爾濟率屬來降,安置耕種。今忽留其屬人,棄眾私遁,其中必有不得已之情,務即察明具奏。朕於噶爾丹多爾濟略無責備之意,且降旨收集其遺眾。儻往汝地,汝可善為撫恤。如欲內徙,即行遣歸。」時噶爾丹多爾濟陽附策妄阿喇布坦,陰貳之。策妄阿喇布坦將侵哈薩克,噶爾丹多爾濟詭以兵從,中道遁庫車,為回眾所殺。母札木蘇攜屬九百餘奔青海部,青海諸臺吉以獻。詔安置什巴爾臺,隸察哈爾。

四十三年,和囉理子阿寶尚郡王,授和碩額駙,賜第京師。四十八年,襲貝勒。五十四年,以參贊往會西安將軍廣柱等,駐巴里坤,襲擊準噶爾於伊勒布爾和碩、阿克塔斯、烏魯木齊諸地,皆捷。五十九年,參贊平逆將軍延信軍敗準噶爾,有克河、齊諾郭勒、綽瑪喇諸捷,護達賴喇嘛入藏。年羹堯奉諭遣歸游牧。未幾,來朝,上憫其勞,詔封多羅郡王。

雍正二年,大軍定青海,王大臣等議阿拉善為寧夏邊外要地,青海顧實汗諸子裔舊皆游牧山後,今或徙山前,請敕阿拉善扎薩克郡王阿寶飭青海眾歸牧山後,允之。阿寶奏:「臣祖顧實汗歸誠內附,百年於茲,受天朝恩甚厚。前青海昆弟阻兵構亂,上干天討,臣當束身受誅。重荷恩宥,令安游牧,感激莫報。乞賜青海曠地,令臣鈐諸部,不復萌異志。」詔以青海貝子丹忠所遺博囉充克克牧地給之,並諭撫遠大將軍年羹堯遣員齎餉助徙牧。博囉充克克者,即漢書地理志所稱潢水也。七年,阿寶以博囉充克克牧地隘,擅請徙烏蘭穆倫及額濟訥河界,議罪削爵。尋命復之。詔仍歸阿拉善牧,不復居青海。阿寶子袞布,八年,以所部兵赴巴里坤防準噶爾援樊廷,賊遁。九年,錄其勞,封輔國公。十年,晉貝子。

乾隆六年,降襲爵之索諾木多爾濟為鎮國公。二十一年,二等臺吉達瓦車凌從大軍剿危魯特竄黨,遇伏於博囉齊,奮擊之,陣歿。詔議恤,入祀昭忠祠。先是阿寶屬達瑪琳從靖邊大將軍傅爾丹擊準噶爾於和通呼爾哈諾爾,為所掠。至是攜孥及屬布庫勒等四十戶詣都統雅爾哈善軍,請歸阿拉善舊牧。詔如所請,徒眾仍置伊犁。

所部一旗。爵三:曰扎薩克和碩親王,由貝勒晉襲;附鎮國公二,一由貝子降襲,一由輔國公晉襲。阿寶次子羅卜藏多爾濟初襲貝勒。乾隆二十一年,詔以兵赴北路。二十二年,以俘逆賊巴雅爾功,晉郡王,授參贊大臣。二十三年,以剿俘已叛宰桑恩克圖功,予雙眼花翎。二十四年,以臺吉達瓦、佐領布岱等剿瑪哈沁及逆回布拉呢敦功,優賚之。三十年,晉羅卜藏多爾濟親王。三十七年十一月,以甘肅民人私挖阿拉善旗哈布塔哈拉山金沙,命勒爾謹捕治之。四十六年,大軍剿薩拉爾逆回於華林寺,四十九年,又剿逆回於石峰堡邸店。是部皆以兵從,均有功。五十一年,允阿拉善鹽由水路運至山西臨縣磧口。五十六年,是部鹽入銀八千兩。羅卜藏多爾濟子旺沁班巴爾襲親王。後嘗一為寧夏將軍,以袒庇屬人爭勘地界,罷之。

嘉慶四年,陜甘總督長麟奏徵是部征教匪兵歸其部。五年,甘肅按察使姜開陽疏言:「中衛邊外有大小鹽池,今為阿拉善王所轄,其鹽潔白堅好,內地之民皆喜食之。大約甘肅全省食阿拉善鹽者十分之六,陜西一省亦居其三。聞阿拉善王但於兩池置官收稅,不論蒙古、漢人,聽其轉運,故於民甚便。私販甚多,駱駝牛騾什佰成群,持梃格鬥,吏役不敢呵止。今擬令沿邊各州縣於各隘口鹽所從入之處,設局收稅,亦計所馱多少為稅之輕重。彼所收者池稅,我所收者過稅,既無礙於阿拉善王,又易私販為官販,兩便之道。」十一年,阿拉善王因回民私販麗法,獻其池歸官辦,置運判於磴口。每年予阿拉善王銀八千兩,池屬寧夏道專管。十七年,改歸商辦,酌定口岸,示以限制,改磴口大使為皇甫川大使,專司稽察。吉鹽水販止準運至皇甫川,以鹽池敕還阿拉善王,停其償歲,而以吉鹽八萬七千餘引配於潞引,由潞商包納吉課。咸豐四年七月,親王呈捐輸開採哈勒津庫察地方銀礦。定甘肅收阿拉善鹽商稅濟軍餉。同治初年,回匪滋事,屢徵是部兵協剿。三年,阿拉善親王貢桑珠爾默特以匪擾寧夏,呈理籓院乞援。時西路多警,是部設臺遞送,南自甘、涼,西自額濟訥土爾扈特,軍報至烏拉特以達歸化。四年四月,都興阿軍大破回匪於平羅、寶豐,是旗協理臺吉阿布哩亦敗撲入磨石口之匪。諭嘉獎貢桑珠爾默特,仍飭嚴防各口,兼辦駝運。七年,貢桑珠爾默特採買米麥濟穆圖善中鋪之軍,解耕牛一百餘只酌借貧民,俾時耕種。四月奏入,上復嘉獎之。十二月,回匪由平羅竄是部,大肆劫掠,至磴口踞之,攻圍王府,殺傷官兵。貢桑珠爾默特復咨穆圖善乞援。八年,定安派蒙員烏爾罔那遜往是部烏蘭木頭地方剿陸路回匪。四月,屢敗回匪於下永和姜、上永和姜。磴口踞匪還竄陜境。是月董馬原回匪竄是部境,圍定遠營城,毀塚塋、府第、寺廟。鄂爾多斯與額濟訥河土爾扈特文報路斷,貢桑珠爾默特督蒙古官兵嬰城固守。七月九日,提督張曜遣部將楊春祥等率兵解定遠城之圍,匪退廣宗寺,又敗之,越山遁。次日,楊春祥等進軍賀蘭山。八月,金順進軍磴口,遂次平羅。九月,張曜抵寧夏,沿途之沙金托海、三道河、磴口、石嘴山等處皆駐官軍。九年十一月,回匪復竄阿拉善南界之紅井一帶,貢桑珠爾默特派副佐領鄂肯會官軍副將郝永剛等敗之。匪竄永磴口,掠阿拉善,復設臺站十一處。十年五月,金順奏:「寧夏山後阿拉善旗有西來竄賊劫掠。現籌於南北要沖磴口、橫城等處派隊扼扎。」十一年,賽盟阿爾米畢特旗竄來回匪至沙爾雜一帶,張曜以阿拉善王請兵剿辦,令孫金彪分扎柳林湖一帶,兼顧蒙地。是年八月,陜甘總督左宗棠奏準蒙鹽仍祗從一條山、五卡寺至皋蘭、靖遠、條城,經安定、會寧、隴西、秦州,轉運漢南一帶銷售,每百斤收稅銀、釐銀各八分。十三年四月,袁保恆奏:「寧夏採運,須取道阿拉善額濟訥蒙古草地,以達巴里坤。而額濟訥牧地近年被匪蹂躪最深,無可藉資,必以阿拉善駝只為主。當飭阿拉善協理臺吉派員來寧商辦。臣與管旗章京瑪呢阿爾得那籌擬,按程設立三十四臺,專司帶領道路。另雇蒙駝一千五百,民駝五百,各以五百任運一段,班轉遞運,每次可運官斛八百石,限四十月運至巴里坤,間二十日由寧夏發運一次。」諭左宗棠酌度情形,派員赴寧夏接辦。光緒四年七月,以關內外肅清,裁阿拉善所設臺站。

二十六年,拳匪滋事,阿拉善亦出教案。二十七年三月,予各省官員上年保教不力懲處,阿拉善親王貢桑珠爾默特傳旨申飭。其後是部三道河一帶教堂租種地畝益多,引河為渠,開田萬頃,日以富饒。宣統二年,督辦鹽政大臣載澤奏:「山西行銷蒙鹽,西路以阿拉善為主,以鄂爾多斯輔之。有礦,有林木,幅員廣闊。其北毗連賽盟南境各旗,南鄰甘肅鎮番等九縣,為漠南蒙古大部落。自為一部,不設盟,受寧夏將軍節制。」有佐領八。

額濟訥,舊土爾扈特部,在阿拉善旗之西。東古爾鼐,南甘肅毛目縣丞地,北阿濟山,東南合黎山,南與東北、西北皆大戈壁,當甘肅省甘州府及肅州邊外。

系出翁罕六世孫,曰瑪哈齊蒙古。有子二:長曰貝果鄂爾勒克,有曾孫曰書庫爾岱青。第四子曰納木第凌,生納扎爾瑪穆特,為土爾扈特阿玉奇汗族弟。阿玉奇汗游牧額勒濟河。康熙四年,詔封納扎爾瑪木特之子阿喇布珠爾為固山貝子,賜牧色爾騰。先是阿喇布珠爾嘗假道準噶爾謁達賴喇嘛,既而阿玉奇與準噶爾策妄阿喇布坦修怨,阿喇布珠爾自唐古特還,以準噶爾道梗,留嘉峪關外,遣使至京師。上憫其無歸,故有是命。五十五年,阿喇布珠爾奏請從軍效力,詔率兵五百駐噶斯。旋卒,子丹衷襲。

雍正七年,來朝,晉貝勒。九年,以色爾騰牧通噶斯之察罕齊老圖,懼準噶爾掠,乞內徙。陜甘總督查郎阿令攜戚屬游牧阿拉克山、阿勒坦特卜什等處,尋定牧額濟訥河。乾隆四十八年,予世襲罔替。

同治中,回匪滋事,陷肅州。是部與連境,蹂躪特重。時西路文報梗,是部設臺站,遞至阿拉善以達歸化。九年以後,肅州回匪累出擾是部境以北,竄賽、扎兩盟,犯烏里雅蘇臺、科布多。福濟、定安、張廷岳先後奏:「賊匪皆來自土爾扈特貝勒游牧,請飭左宗棠撥軍防剿。」十二年,是部貝勒達什車凌以防堵回匪陣亡。光緒五年,大學士陜甘總督左宗棠為請恤。十二月,贈郡王銜,予恤銀一千一百兩。三十年,延祉等迎護達賴喇嘛往西寧,經是部。地雜戈壁,較諸部為瘠苦,北接扎盟南境。各旗有佐領一,不設盟長,受陜甘總督節制。


\end{pinyinscope}