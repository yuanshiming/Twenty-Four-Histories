\article{清史稿校刻記}

\begin{pinyinscope}
甲寅年始設清史館,以趙公爾巽為館長。修史者有總閱、總纂、纂修、協修及徵訪等職,先後延聘百數十人,別有名譽職約三百人。館中執事,有提調、收掌、科長及校勘等職,亦逾二百人,可謂盛矣。

開館之初,首商義例。館內外同人,如於君式枚、梁君啟超、吳君士鑒、吳君廷燮、姚君永樸、繆君荃孫、陶君葆廉、金君兆蕃、硃君希祖、袁君勵準、王君桐齡等,皆多建議。參酌眾見,後乃議定用明史體裁,略加通變。先排史目,凡本紀十二:曰太祖、太宗、世祖、聖祖、世宗、高宗、仁宗、宣宗、文宗、穆宗、德宗,而宣統紀初擬為今上本紀,後改定。志十六:曰天文、災異、時憲、地理、禮、樂、輿服附鹵簿、選舉、職官、食貨、河渠、兵、交通、刑法、藝文、邦交,初擬有國語、氏族、外教三志,皆刪。表十:曰皇子、公主、外戚、諸臣封爵、籓部、大學士、軍機大臣、部院大臣、疆臣、交聘,初以大學士與軍機合稱宰輔,後改。列傳十五:曰后妃、諸王、諸臣、循吏、儒林、文苑、疇人、忠義、孝義、遺逸、藝術、列女、土司、籓部、屬國,初擬

有明遺臣、卓行、貨殖、客卿、叛臣諸目,皆刪並。其取材則以實錄為主,兼採國史舊志及本傳,而參以各種記載,與夫徵訪所得,務求傳信,不尚文飾焉。

庚申,初稿略備,始排比復輯。丙寅秋,重加修正。自開館至是,已歲紀一周,其難其慎,蓋猶未敢為定稿也。丁卯夏,袁君金鎧創刊稿待正之議,趙公韙之,即請袁君總理發刊事宜,而以梁任校刻,期一年竣事。梁擬總閱全稿,先畫一而後付刊。乃稿實未齊,且待修正,祇可隨修隨刻,不復有整理之暇矣。是時留館者僅十餘人,於是公推以柯君劭忞總紀稿,王君樹棻總志稿,吳君廷燮總表稿。夏君孫桐、金君兆蕃分總傳稿,而由袁君與梁校閱付刊。本紀自太祖至世宗五朝為鄧君邦述、金君兆蕃原稿,高宗至穆宗五朝為吳君廷燮原稿,德宗及宣統二朝為瑞君洵原稿,而太祖、聖祖、世宗、仁宗、文宗、與宣統六紀為奭君良復輯,穆、德二紀為李君哲明復輯,柯君皆多刪正。志則天文、時憲、災異為柯君稿;地理為秦君樹聲原稿,王君樹棻復輯;禮為張君書云、王君大鈞、萬君本端等分稿;職官為金君兆豐、駱君成昌、李君景濂、徐君鴻寶等分稿,皆金君復輯,樂為張君採田稿;輿服為何君葆麟稿,選舉為張君啟後、硃君希祖、袁君勵準等分稿,張君書云復輯;食貨為姚君永樸、李君岳瑞、李君哲明、吳君懷清分稿,河渠為何君葆麟等原稿,交通為羅君惇曰融等分稿,皆吳君復輯,兵為俞君陛雲、秦君望瀾、田君應璜、袁君克文等分稿,俞君復輯;刑法為王君式通等分輯,後用許君受衡稿;藝文為章君鈺、吳君士鑒原稿,硃君師轍復輯;邦交為李君家駒、吳君廣霈、劉君樹屏等分稿,戴君錫章復輯。表則諸王、公主、外戚為吳君士鑒原稿,諸臣封爵為劉君師培原稿,軍機大臣為唐君邦治原稿,餘皆吳君廷燮稿。列傳則後妃、諸王為鄧君奭君及金君兆蕃原稿,皆金君復輯;諸臣原稿,凡在館諸君多有分纂,自開國至乾隆為金君兆蕃復輯,嘉道咸同為夏君孫桐復輯,光宣為馬君其昶、金君兆豐復輯,而梁又重補輯之;循吏及藝術皆夏君復輯,儒林為繆君荃孫稿,文苑為馬君稿,梁皆補之;疇人為陳君年原稿,柯君復輯,忠義為章君復輯;孝義及列女為金君兆蕃復輯,遺逸為王君樹棻及繆君原稿,梁復輯之;土司為繆君稿,籓部蒙古為吳君廷燮稿,西藏為吳君燕紹稿,屬國為韓君樸存稿。凡諸稿梁皆校閱,並有參訂。惜倉卒付刊,不及從容討論耳。昔萬季野參修明史,總閱全書,事必覈之實錄,誤者正之,漏者補之,此修史公例,不敢忽也。是秋趙公去世,柯君兼代館長,一仍舊貫。歲暮校印過半,乃先發行,至今夏全書告成,幸未逾預定之期。袁君創議於先,經營籌畫,力任其難,庶幾無負趙公之託。其間數經艱亂,皆幸無阻,良非初料所及。一代國史,所關甚大,其成否亦系乎天焉。初有議宣統紀從闕者,梁以春秋不諱定哀,力爭存之;又議斷代為史,凡歿於辛亥以後者皆不入傳,梁以明末遺臣,史皆並著,且清史實為舊史結束,後將別創新史,體例各異,諸人與清室相終始,豈容泯沒,故所補獨多。

校刻既竣,略記始末,以備參考。史稿本非定本,望海內通人不吝指教。當別撰校勘記,為將來修正之資,幸甚幸甚。戊辰端節金梁


\end{pinyinscope}