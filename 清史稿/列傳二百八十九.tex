\article{列傳二百八十九}

\begin{pinyinscope}
藝術一

吳有性戴天章餘霖劉奎喻昌徐彬張璐高斗魁周學海

張志聰高世栻張錫駒陳念祖黃元禦柯琴尤怡

葉桂薛雪吳瑭章楠王士雄徐大椿王維德吳謙

綽爾濟伊桑阿張朝魁陸懋修王丙呂震鄒澍費伯雄

蔣平階章攀桂劉祿張永祚戴尚文

自司馬遷傳扁鵲、倉公及日者、龜策,史家因之,或曰方技,或曰藝術。大抵所收多醫、卜、陰陽、術數之流,間及工巧。夫藝之所賅,博矣眾矣,古以禮、樂、射、御、書、數為六藝,士所常肄,而百工所執,皆藝事也。近代方志,於書畫、技擊、工巧並入此類,實有合於古義。

聖祖天縱神明,多能藝事,貫通中、西歷算之學,一時鴻碩,蔚成專家,國史躋之儒林之列。測繪地圖,鑄造槍砲,始仿西法。凡有一技之能者,往往召直蒙養齋。其文學侍從之臣,每以書畫供奉內廷。又設如意館,制仿前代畫院,兼及百工之事。故其時供御器物,雕、組、陶埴,靡不精美,傳播寰瀛,稱為極盛。

沿及高宗之世,風不替焉。欽定醫宗金鑒,薈萃古今學說,宗旨純正。於陰陽術數家言,亦有協紀辨方一書,頒行沿用,從俗從宜,隱示崇實黜虛之意,斯徵微尚矣。

中葉後,海禁大開,泰西藝學諸書,灌輸中國,議者以工業為強國根本,於是研格致,營制造者,乘時而起。或由舊學以擴新知,或抒心得以濟實用,世乃愈以藝事為重。採其可傳者著於篇,各以類為先後。卓然成家者,具述授受源流;兼有政績、文學列入他傳者,附存梗概;凡涉荒誕俳諧之說,屏勿載。後之覽者,庶為論世之資云。

吳有性,字又可,江南吳縣人。生於明季,居太湖中洞庭山。當崇禎辛巳歲,南北直隸、山東、浙江大疫,醫以傷寒法治之,不效。有性推究病源,就所歷驗,著瘟疫論,謂:「傷寒自毫竅入,中於脈絡,從表入里,故其傳經有六。自陽至陰。以次而深。瘟疫自口鼻入,伏於膜原,其邪在不表不里之間。其傳變有九,或表或裏,各自為病。有但表而不里者,有表而再表者,有但里而不表者,有里而再里者,有表裏分傳者,有表裏分傳而再分傳者,有表勝於里者,有先表後里者,有先里後表者。」其間有與傷寒相反十一事,又有變證、兼證,種種不同。並著論制方,一一辨別。古無瘟疫專書,自有性書出,始有發明。

其後有戴天章、餘霖、劉奎,皆以治瘟疫名。

天章,字麟郊,江蘇上元人。諸生。好學強記,尤精於醫。所著傷寒、雜病諸書,及咳論注、瘧論注、廣瘟疫論,凡十餘種。其論瘟疫,一宗有性之說。謂瘟疫之異於傷寒,尤慎辨於見證之始。辨氣、辨色、辨舌、辨神、辨脈,益加詳焉。為人療病,不受謝。子瀚,成雍正元年一甲第二名進士。

霖,字師愚,安徽桐城人。乾隆中,桐城疫,霖謂病由熱淫,投以石膏,輒愈。後數年,至京師,大暑,疫作,醫以張介賓法者多死,以有性法亦不盡驗。鴻臚卿馮應榴姬人呼吸將絕,霖與大劑石膏,應手而痊,踵其法者,活人無算。霖所著曰疫疹一得,其論與有性有異同,取其辨證,而以用達原飲及三消、承氣諸方,猶有附會表里之意云。

奎,字文甫,山東諸城人。乾隆末,著瘟疫論類編及松峰說疫二書,松峰者,奎以自號也。多為窮鄉僻壤艱覓醫藥者說法。有性論瘟疫,已有大頭瘟、疙瘩瘟疫、絞腸瘟、軟腳瘟之稱,奎復舉北方俗諺所謂諸疫證名狀,一一剖析之。又以貧寒病家無力購藥,取鄉僻恆有之物可療病者,發明其功用,補本草所未備,多有心得。同時昌邑黃元御治疫,以浮萍代麻黃,即本奎說。所著書流傳日本,醫家著述,亦有取焉。

喻昌,字嘉言,江西新建人。幼能文,不羈,與陳際泰游。明崇禎中,以副榜貢生入都上書言事,尋詔徵,不就,往來靖安間。披為僧,復蓄發游江南。順治中,僑居常熟,以醫名,治療多奇中。才辯縱橫,不可一世。著傷寒尚論篇,謂林億、成無已過於尊信王叔和,惟方有執作條辨,削去叔和序例,得尊經之旨;而猶有未達者,重為編訂,其淵源雖出方氏,要多自抒所見。惟溫證論中,以溫藥治溫病,後尤怡、陸鋰修並著論非之。

又著醫門法律,取風、寒、暑、濕、燥、火六氣及諸雜證,分門著論。次法,次律。法者,治療之術,運用之機;律者,明著醫之所以失,而判定其罪,如折獄然。昌此書,專為庸醫誤人而作,分別疑似,使臨診者不敢輕嘗,有功醫術。

後附寓意草,皆其所治醫案。凡診病,先議病,後用藥,又與門人定議病之式,至詳審。所載治驗,反覆推論,務闡審證用藥之所以然,異於諸家醫案但泛言某病用某藥愈者,並為世所取法。

昌通禪理,其醫往往出於妙悟。尚論後篇及醫門法律,年七十後始成。昌既久居江南,從學者甚多。

徐彬,字忠可,浙江嘉興人。昌之弟子。著傷寒一百十三方發明及金匱要略論注,其說皆本於昌。四庫著錄金匱要略,即用彬論注本。凡疏釋正義,見於注;或賸義及總括諸證不可專屬者,見於論。彬謂:「他方書出於湊集,就採一條,時亦獲驗。若金匱之妙,統觀一卷,全體方具。不獨察其所用,並須察其所不用。」世以為篤論。

張璐,字路玉,自號石頑老人,江南長洲人。少穎悟,博貫儒業,專心醫藥之書。自軒、岐迄近代方法,無不搜覽。遭明季之亂,隱於洞庭山中十餘年,著書自娛,至老不倦。仿明王肯堂證治準繩,匯集古人方論、近代名言,薈萃折衷之,每門附以治驗醫案,為醫歸一書,後易名醫通。

璐謂仲景書衍釋日多,仲景之意轉晦。後見尚論、條辨諸編,又廣搜秘本,反覆詳玩,始覺向之所謂多歧者,漸歸一貫,著傷寒纘論、緒論。纘者,祖仲景之文;緒者,理諸家之紛紜而清出之,以翼仲景之法。

其注本草,疏本經之大義,並系諸家治法,曰本經逢原;論脈法大義,曰診宗三昧:皆有心得。又謂唐孫思邈治病多有奇異,逐方研求藥性,詳為疏證,曰千金方釋義,並行於世。

璐著書主博通,持論平實,不立新異。其治病,則取法薛已、張介賓為多。年八十餘卒。聖祖南巡,璐子以柔進呈遺書,溫旨留覽焉。子登、倬,皆世其業。

登,字誕先,著傷寒舌鑒;

倬,字飛疇,著傷寒兼證析義:並著錄四庫。

高斗魁,字旦中,又號鼓峰,浙江鄞縣人。諸生。兄斗樞,明季死國難。斗魁任俠,於遺民罹難者,破產營救。妻因事連及,勒自裁。素精醫,游杭,見舁棺者血瀝地,曰:「是未死!」啟棺,與藥而甦。江湖間傳其事,求治病者無寧晷。著醫學心法;又吹毛編,則自記醫案也。其論醫宗旨,亦近於張介賓。

周學海,字澂之,安徽建德人,總督馥子。光緒十八年進士,授內閣中書,官至浙江候補道。潛心醫學,論脈尤詳,著脈義簡摩、脈簡補義、診家直訣、辨脈平脈章句。引申舊說,參以實驗,多心得之言。博覽群籍,實事求是,不取依託附會。慕宋人之善悟,故於史堪、張元素、劉完素、滑壽及近世葉桂諸家書,皆有評注。自言於清一代名醫,服膺張璐、葉桂兩家。證治每取璐說,蓋其學頗與相近。宦游江、淮間,時為人療治,常病不異人,遇疑難,輒有奇效。刻古醫書十二種,所據多宋、元舊槧藏家秘笈,校勘精審,世稱善本云。

張志聰,字隱庵,浙江錢塘人。明末,杭州盧之頤、繇父子著書,講明醫學,志聰繼之。構侶山堂,招同志講論其中,參考經論,辨其是非。自順治中至康熙之初,四十年間,談軒、岐之學者咸歸之。注素問、靈樞二經,集諸家之說,隨文衍義,勝明馬元臺本。

又注傷寒論、金匱要略,於傷寒論致力尤深,歷二十年,再易稿始成。用王叔和原本,略改其編次。首列六經病,次列霍亂易復並濕、暍汗、吐下,後列辨脈、平脈,而刪叔和序例,以其與本論矛盾,故去之以息辨。駁辨成無已舊注,謂:「風傷衛,寒傷營,脈緩為中風,脈緊為傷寒。傷寒,惡寒無汗,宜麻黃湯;中風,惡風有汗,宜桂枝湯:諸說未盡當。而風、寒兩感,營、衛俱傷,宜大青龍湯為尤謬。其注,分章以明大旨,節解句釋,兼晰陰陽血氣之生始出入,經脈藏府之貫通循行,使讀論者取之有本,用之無窮,不徒求之糟粕,庶免終身由之而不知其道也。」

又注本草,詮釋本經,闡明藥性,本五運六氣之理。後人不經臆說,概置勿錄。

其自著曰侶山堂類辨、針灸秘傳。志聰之學,以素、靈、金匱為歸,生平著書,必守經法,遺書並行於世,惟針灸祕傳佚。

高世栻,字士宗。與志聰同里。少家貧,讀時醫通俗諸書,年二十三即出療病,頗有稱。後自病,時醫治之,益劇;久之,不藥,幸愈。翻然悔曰:「我治人,殆亦如是,是草菅人命也。」乃從志聰講論軒、岐、仲景之學,歷十年,悉窺精奧。遇病必究其本末,處方不同流俗。志聰著本草崇原,未竟,世栻繼成之。又注傷寒論。晚著醫學真傳,示門弟子。自述曰:「醫理如剝蕉,剝至無可剝,方為至理。以之論病,大中至正,一定不移。世行分門別類之方書,皆醫門糟粕,如薛已、趙獻可輩,雖有穎悟變通,非軒、岐、仲景一脈相傳之大道。古人云:『不知十二經絡,開口舉手便錯;不明五運六氣,讀盡方書無濟。病有標有本,求其標,只取本,治千人,無一損。』故示正道,以斥旁門,使學者知所慎。」

後有張錫駒,字令韶,亦錢塘人。著傷寒論直解、胃氣論,其學本於志聰。

陳念祖,字修園,福建長樂人。乾隆五十七年舉人。著傷寒金匱淺注,本志聰、錫駒之說,多有發明,世稱善本。嘉慶中,官直隸威縣知縣,有賢聲。值水災,大疫,親施方藥,活人無算。晚歸田,以醫學教授,門弟子甚眾,著書凡十餘種,並行世。

黃元禦,字坤載,山東昌邑人。諸生。因庸醫誤藥損目,發憤學醫,於素問、靈樞、難經、傷寒論、金匱玉函經皆有注釋,凡數十萬言。自命甚高,喜更改古書,以伸己說。其論治病,主於扶陽以抑陰。

柯琴,字韻伯,浙江慈谿人。博學多聞,能詩、古文辭。棄舉子業,矢志醫學。家貧,游吳,棲息於虞山,不以醫自鳴,當世亦鮮知者。著內經合璧,多所校正,書佚不傳。

注傷寒論,名曰來蘇集。以方有執、喻昌等各以己意更定,有背仲景之旨,乃據論中有太陽證、桂枝證、柴胡證諸辭以證名篇,匯集六經諸論,各以類從。自序略曰:「傷寒論經王叔和編次,已非仲景之舊,讀者必細勘何者為仲景言,何者為叔和筆。其間脫落、倒句、訛字、衍文,一一指破,頓見真面。且筆法詳略不同,或互文見意,或比類相形,因此悟彼,見微知著,得於語言文字之外,始可羽翼仲景。自來注家,不將全書始終理會,先後合參,隨文敷衍,彼此矛盾,黑白不分。三百九十七法,不見於仲景序文,又不見於叔和序例,林氏倡於前,成氏和於後,其不足取信,王安道已辨之矣。繼起者,猶瑣瑣於數目,亦何補於古人?何功於後學哉?大青龍湯,仲景為傷寒中風無汗而兼煩燥者設,即加味麻黃湯耳。而謂其傷寒見風、傷風見寒,因以麻黃湯主寒傷營、桂枝湯主風傷衛、大青龍湯主風寒兩傷營衛,曲成三綱鼎立之說,此鄭聲之亂雅樂也。且以十存二三之文,而謂之全篇,手足厥冷之厥,或混於兩陰交盡之厥,其間差謬,何可殫舉?此愚所以執卷長籲,不能已也!」

又著傷寒論翼,自序略曰:「仲景著傷寒、雜病論,合十六卷,法大備。其常中之變,變中之常,靡不曲盡。使全書俱在,盡可見論知源。自叔和編次傷寒、雜病,分為兩書,然本論中雜病留而未去者尚多,雖有傷寒論之專名,終不失雜病合論之根蒂也。名不副實,並相淆混,而旁門歧路,莫知所從,豈非叔和之謬以禍之歟?夫仲景之言六經為百病之法,不專為傷寒一科,傷寒、雜病,治無二理,咸歸六經之節制。治傷寒者,但拘傷寒,不究其中有雜病之理;治雜病者,復以傷寒論無關於雜病,而置之不問。將參贊化育之書,悉歸狐疑之域,愚甚為斯道憂之。」論者謂琴二書,大有功於仲景。

尤怡,字在涇,江蘇吳縣人。父有田千畝,至怡中落。貧甚,鬻字於佛寺。業醫,人未之異也。好為詩,與同里顧嗣立、沈德潛游。晚年,學益深造,治病多奇中,名始著。性淡榮利,隱於花溪,自號飼鶴山人,著書自得。其注傷寒論,名曰貫珠集。謂後人因王叔和編次錯亂,辨駁改訂,各成一家言,言愈多而理愈晦。乃就六經,各提其綱,於正治法之外,太陽有權變法,斡旋法,救逆法,類病法;陽明有明辨法,雜治法;少陽有權變法;太陰有藏病、經病法,經、藏俱病法;少陰、厥陰有溫法、凊法。凡病機進退微權,各有法以為辨,使讀者先得其法,乃能用其方。分證甚晰,於少陰、厥陰、溫凊兩法,尤足破世人之惑。注金匱要略,名曰心典。別撰集諸家方書、雜病治要,足以羽翼仲景者,論其精蘊,曰金匱翼。又著醫學讀書記,於軒、岐以下諸家,多有折衷,徐大椿稱為得古人意。怡著述並篤雅,世以貫珠集與柯琴來蘇集並重焉。

葉桂,字天士,江蘇吳縣人。先世自歙遷吳,祖時、父朝採,皆精醫。桂年十四喪父,從學於父之門人,聞言即解,見出師上,遂有聞於時。切脈望色,如見五藏。治方不出成見,嘗曰:「劑之寒溫視乎病,前人或偏寒涼,或偏溫養,習者茫無定識。假兼備以幸中,借和平以藏拙。朝用一方。晚易一劑,詎有當哉?病有見證,有變證,必胸有成竹,乃可施之以方。」

其治病多奇中,於疑難證,或就其平日嗜好而得救法;或他醫之方,略與變通服法;或竟不與藥,而使居處飲食消息之;或於無病時預知其病;或預斷數十年後:皆驗。當時名滿天下,傳聞附會,往往涉於荒誕,不具錄。卒,年八十。臨歿,戒其子曰:「醫可為而不可為。必天資敏悟,讀萬卷書,而後可以濟世。不然,鮮有不殺人者,是以藥餌為刀刃也。吾死,子孫慎勿輕言醫!」

桂神悟絕人,貫徹古今醫術,而鮮著述。世傳所注本草,多心得。又許叔微本事方釋義、景嶽發揮。歿後,門人集醫案為臨證指南,非其自著。附幼科心法一卷,傳為桂手定,徐大椿謂獨精卓,後章楠改題曰三時伏氣外感篇;又附溫證證治一卷,傳為口授門人顧景文者,楠改題曰外感溫證篇。二書最為學者所奉習。

同里薛雪,名亞於桂,而大江南、北,言醫輒以桂為宗,百餘年來,私淑者眾。最著者,吳瑭、章楠、王士雄。

雪,字生白,自號一瓢。少學詩於同郡葉燮。乾隆初,舉鴻博,未遇。工畫蘭,善拳勇,博學多通,於醫時有獨見。斷人生死不爽,療治多異跡。生平與桂不相能,自名所居曰掃葉莊,然每見桂處方而善,未嘗不擊節也。著醫經原旨,於靈、素奧旨,具有發揮。世傳濕溫篇,為學者所宗,或曰非雪作。其醫案與桂及繆遵義合刻。

遵義,亦吳人。乾隆二年進士,官知縣。因母病,通方書,棄官為醫,用藥每出創意,吳中稱三家焉。

瑭,字鞠通,江蘇淮陰人。乾、嘉之間游京師,有名。學本於桂,以桂立論甚簡,但有醫案散見於雜證之中,人多忽之。著溫病條辨,以暢其義,其書盛行。

同時歸安吳貞,著傷寒指掌,亦發明桂醫案之旨,與瑭相同。

楠,字虛谷,浙江會稽人。著醫門棒喝。謂桂、雪最得仲景遺意,而他家不與。

士雄,字孟英,浙江海寧人。居於杭,世為醫。士雄讀書礪行,家貧,仍以醫自給。咸豐中,杭州陷,轉徙上海。時吳、越避寇者麕集,疫癘大作,士雄療治,多全活。舊著霍亂論,致慎於溫補,至是重訂刊行,醫者奉為圭臬。又著溫熱經緯,以軒、岐、仲景之文為經,葉、薛諸家之辨為緯,大意同章楠注釋。兼採昔賢諸說,擇善而從,勝楠書。所著凡數種,以二者為精詳。

同時浙西論醫者,平湖陸以湉、嘉善汪震、烏程汪曰楨,宗旨略同。

陽湖張琦、曜孫,父子皆通儒,以醫鳴,取黃元禦扶陽之說,偏於溫。曜孫至上海,或勸士雄往就正,士雄謝之。號葉氏學者,要以士雄為巨擘,惟喜用辛涼,論者謂亦稍偏雲。

徐大椿,原名大業,字靈胎,晚號洄溪,江蘇吳江人,翰林檢討釚孫。生有異稟,長身廣顙,聰強過人。為諸生,勿屑,去而窮經,探研易理,好讀黃老與陰符家言。凡星經、地志、九宮、音律、技擊、句卒、嬴越之法,靡不通究,尤邃於醫,世多傳其異跡。然大椿自編醫案,惟剖析虛實寒溫,發明治療之法,歸於平實,於神異者僅載一二。其書世多有,不具錄。

乾隆二十四年,大學士蔣溥病,高宗命徵海內名醫,以薦召入都。大椿奏溥病不可治,上嘉其樸誠,命入太醫院供奉,尋乞歸。後二十年復詔徵,年已七十九,遂卒於京師,賜金治喪。

大椿學博而通,注神農本草經百種,以舊注但言其當然,不言其所以然,採掇常用之品,備列經文,推闡主治之義,於諸家中最有啟發之功。

注難經曰經釋,辨其與靈樞、素問說有異同。注傷寒曰類方,謂:「醫家刊定傷寒論,如治尚書者之爭洪範、武成,注大學者之爭古本、今本,終無定論。不知仲景本論,乃救誤之書,當時隨證立方,本無定序。」於是削除陰陽六經門目,但使方以類從,證隨方定,使人可案證以求方,而不必循經以求證。一切葛藤,盡芟去之。所著蘭臺軌範,凡錄病論,惟取靈樞、素問、難經、金匱要略、傷寒論、隋巢元方病源、唐孫思邈千金方、王燾外臺秘要而止。錄方亦多取諸書,宋以後方,則採其義可推尋、試多獲效者,去取最為謹嚴。於疑似出入之間,辨別尤悉。

其論醫之書曰醫學源流論,分目九十有三。謂:「病之名有萬,而脈之象不過數十,是必以望、聞、問三者參之。如病同人異之辨,兼證兼病之別,亡陰亡陽之分。病有不愈不死,有雖愈必死,又有藥誤不即死。藥性有古今變遷,內經司天運氣之說不可泥。針灸之法失傳。」諸說並可取。

又慎疾芻言,為溺於邪說俗見者痛下針砭,多驚心動魄之語。醫貫砭,專斥趙獻可溫補之弊。諸書並行世。

大椿與葉桂同以醫名吳中,而宗旨異。評桂醫案,多所糾正。兼精瘍科,而未著專書,謂世傳外科正宗一書,輕用刀針及毒藥,往往害人,詳為批評,世並奉為善本。

同郡吳縣王維德,字洪緒,自號林屋山人。曾祖字若谷,精瘍醫,維德傳其學,著外科全生集。謂:「癰疽無死證,癰乃陽實,氣血熱而毒滯;疽乃陰虛,氣血寒而毒凝。皆以開腠理為要,治者但當論陰陽虛實。初起色紅為癰,色白為疽,截然兩途。世人以癰疽連呼並治,誤矣。」其論為前人所未發。凡治初起以消為貴,以托為畏,尤戒刀針毒藥,與大椿說略同,醫者宗之。維德兼通陰陽家言,著永寧通書、卜筮正宗。

吳謙,字六吉,安徽歙縣人。官太醫院判,供奉內廷,屢被恩賚。乾隆中,敕編醫書,太醫院使錢鬥保請發內府藏書,並徵集天下家藏秘籍,及世傳經驗良方,分門聚類,刪其駁雜,採其精粹,發其餘蘊,補其未備,為書二部。小而約者,以為初學誦讀;大而博者,以為學成參考。既而徵書之令中止,議專編一書,期速成,命謙及同官劉裕鐸為總修官。

謙以古醫書有法無方,惟傷寒論、金匱要略、雜病論始有法有方。靈、素而後,二書實一脈相承。義理淵深,方法微奧,領會不易,遂多譌錯。舊注隨文附會,難以傳信。謙自為刪定,書成八九,及是,請就謙未成之書,更加增減。於二書譌錯者,悉為訂正,逐條注釋,復集諸家舊注實足闡發微義者,以資參考,為全書之首,標示正軌。次刪補名醫方論,次四診要訣,次諸病心法要訣,次正骨心法要旨。書成,賜名醫宗金鑒,雖出眾手編輯,而訂正傷寒、金匱,本於謙所自撰。

其採引清代乾隆以前醫說凡二十餘家,張璐、喻昌、徐彬、張志聰、高世式、張錫駒、柯琴、尤怡,事具本傳。

其次者:林瀾,著傷寒折衷、靈素合鈔,兼通星象、堪輿之學;汪琥,著傷寒論辨注;魏荔彤,著傷寒金匱本義;沈明宗,著傷寒金匱編注;程應旄,著傷寒後條辨;鄭重光,著傷寒論條辨續注;周揚俊,著傷寒三注、金匱二注;程林,著金匱直解、聖濟總錄纂要;閔芝慶,著傷寒闡要編。而遺書湮沒無考者,尚六七家云。

綽爾濟,墨爾根氏,蒙古人。天命中,率先歸附。善醫傷。時白旗先鋒鄂碩與敵戰,中矢垂斃,綽爾濟為拔鏃,傅良藥,傷尋愈。都統武拜身被三十餘矢,昏絕,綽爾濟令剖白駝腹,置武拜其中,遂甦。有患臂屈不伸者,令先以熱鑊熏蒸,然後斧椎其骨,揉之有聲,即愈。

覺羅伊桑阿,乾隆中,以正骨起家,至鉅富。其授徒法,削筆管為數段,包以紙,摩挲之,使其節節皆接合,如未斷者然,乃如法接骨,皆奏效。故事,選上三旗士卒之明骨法者,每旗十人,隸上駟院,名蒙古醫士。凡禁庭執事人有跌損者,命醫治,限日報痊,逾期則懲治之。侍郎齊召南墜馬,傷首,腦出。蒙古醫士以牛脬蒙其首,其創立愈。時有秘方,能立奏效,伊桑阿名最著。當時湖南有張朝魁者,亦以治傷科聞。

朝魁,辰谿人,又名毛矮子。年二十餘,遇遠來乞者,朝魁厚待之,乞者授以異術,治癰疽、瘰癆及跌打、損傷、危急之證,能以刀剖皮肉,去淤血於臟腑。又能續筋正骨,時有劉某患腹痛,僕地瀕死,朝魁往視曰:「病在大小腸。」剖其腹二寸許,伸指入腹理之,數日愈。辰州知府某乘輿越銀壺山,忽墮巖下,折髃骨,朝魁以刀刺之,撥正,傅以藥,運動如常。

陸懋修,字九芝,江蘇元和人。先世以儒顯,皆通醫。懋修為諸生,世其學。咸豐中,粵匪擾江南,轉徙上海,遂以醫名。研精素問,著內經運氣病釋。後益博通漢以後書,恪守仲景家法,於有清一代醫家,悉舉其得失。所取法在柯琴、尤怡兩家,謂得仲景意較多。吳中葉桂名最盛,傳最廣,懋修謂桂醫案出門弟子,不盡可信。所傳溫病證治,亦門人筆述。開卷揭「溫邪上受、首先犯肺、逆傳心包」一語,不應經法,誤以胃熱為肺熱,由於不識陽明病,故著陽明病釋一篇,以闡明之。又據難經「傷寒有五」之文,謂:「仲景撰用難經,溫病即在傷寒中,治溫病法不出傷寒論外。」又謂:「瘟疫有溫、有寒,與溫病不同,醫者多混稱。吳有性、戴天章為治疫專家,且不免此誤。」著論辨之,並精確,有功學者。

懋修既棄舉業,不求仕進,及子潤庠登第,就養京邸,著述至老不倦。光緒中,卒。潤庠亦通醫,官至大學士,自有傳。

王丙,字樸莊,吳縣人,懋修之外曾祖也。著傷寒論注,以唐孫思邈千金方僅採王叔和傷寒論序例,全書載翼方中,序次最古,據為定本。謂:「方中行、喻昌等刪駁序例,乃欲申己見,非定論。」著回瀾說,爭之甚力。又著古今權量考,古一兩準今六分七釐,一升準今七勺七秒,承學者奉以為法。

呂震,字茶村,浙江錢塘人。道光五年舉人,官湖北荊門州判。晚寓吳,酷嗜醫,診療輒有奇效。其言曰:「傷寒論使學者有切實下手工夫,不止為傷寒立法。能從六經辨證,雖繁劇如傷寒,不為多歧所誤,雜證一以貫之。」著內經要論、傷寒尋源。懋修持論多本丙、震雲。

鄒澍,字潤安,江蘇武進人。有孝行,家貧績學,隱於醫。道光初,詔舉山林隱逸,鄉人議以澍名上,固辭。澍通知天文推步、地理形勢沿革,詩古文亦卓然成家,不自表襮。所著書,醫家言為多。傷寒通解、傷寒金匱方解、醫理摘要、醫經書目,並不傳。所刊行者,本經疏證、續疏證、本經序疏要。謂明潛江劉氏本草述,貫串金、元諸家說,反多牽掣,故所注悉本傷寒、金匱,疏通證明,而以千金、外臺副之。深究仲景制方精意,成一家之言。

費伯雄,字晉卿。與澍同邑,居孟河,濱江。咸、同間以醫名遠近,詣診者踵相接,所居遂成繁盛之區。持脈知病,不待問。論醫,戒偏戒雜。謂古醫以「和緩」命名,可通其意。著書曰醫醇,毀於寇。撮其要,成醫醇賸義,附方論。大旨謂常病多,奇病少,醫者執簡,始能馭繁,不可尚異。享盛名數十年,家以致富,子孫皆世其業。伯雄所著,詳於雜病,略於傷寒,與懋修、澍宗旨並不同。清末江南諸醫,以伯雄為最著,用附載焉。

清代醫學,多重考古,當道光中,始譯泰西醫書,王清任著醫林改錯。以中國無解剖之學,宋、元後相傳臟腑諸圖,疑不盡合,於刑人時,考驗有得,參證獸畜。未見西書,而其說與合。光緒中,唐宗海推廣其義,證以內經異同,經脈奇經各穴,及營衛經氣,為西醫所未及。著中西匯通醫經精義,欲通其郵而補其缺。兩人之開悟,皆足以啟後者。

蔣平階,字大鴻,江南華亭人。少孤,其祖命習形家之學,十年,始得其傳。遍證之大江南、北古今名墓,又十年,始得其旨;又十年,始窮其變。自謂視天下山川土壤,雖大荒內外如一也。遂著地理辨正,取當世相傳之書,訂其紕繆,析其是非,惟尊唐楊筠松一人,曾文辿僅因筠松以傳。其於廖瑀、賴文俊、何溥以下,視之蔑如。以世所惑溺者,莫甚於平砂玉尺一書,斥其偽尤力。自言事貴心授,非可言罄,古書充棟,半屬偽造。其昌言救世,惟在地理辨正一書。後復自抒所得,作天元五歌,謂此皆糟粕,其精微亦不在此,他無秘本。三吳兩浙,有自稱得平階真傳及偽撰成書指為平階秘本者,皆假託也。

從之學者,丹陽張仲馨,丹徒駱士鵬,山陰呂相烈,會稽姜堯,武陵胡泰徵,淄川畢世持,他無所傳授。姜堯注青囊奧語及平砂玉尺辨偽,總括歌,即附地理辨正中。

平階生於明末,兼以詩鳴。清初諸老,多與唱和。地學為一代大宗,所造羅經,後人多用之,稱為「蔣盤」云。

章攀桂,字淮樹,安徽桐城人。乾隆中,官甘肅知縣,累擢江蘇松太兵備道。有吏才,多術藝,尤精形家言。謂近世形家諸書,理當辭顯者,莫如明張宗道地理全書,為之作注,稍辨正其誤失。大旨本元人山陽指迷之說,專主形勢。攀桂既仕顯,不以方技為業,自喜其術,每為親族交友擇地,貧者助之財以葬。妻吳故農家,自恨門第微,攀桂為購佳壤葬其親,擇子弟秀異者撫教之,遂登進士第,為望族。

高宗數南巡,自鎮江至江寧,江行險,每由陸。詔改通水道,議鑿句容故破岡瀆,攀桂相其地勢,謂茅山石巨勢高,縱成瀆,非設閘不可成,儲水多勞費。請從上元東北攝山下,鑿金烏珠刀槍河故道,以達丹徒,工省修易。遂監其役,瀆成,謂之新河,百年來賴其利便,攀桂亦因獲優擢。

大學士於敏中於金壇里第築園,攀桂為之相度營建,敏中歿後,事覺,高宗惡之,褫職居江寧。晚耽禪理,歿時預知期日。兼通日者術,括協紀辨方精要為一書,曰選擇正宗,行於世。

劉祿,河南人。善風角。聖祖召直蒙養齋,欲授以官,屢辭。從上北征,會糧餉乏濟,命卜之,曰:「不出三日必至。」果如其言。後從幸熱河,一日,踉蹌至宮門,請上速徙高阜以避水厄。時方晴霽,夜山水漲發,果沖沒行宮。又善相人,謂張廷玉、史貽直皆異日太平宰相。六十一年冬,乞假歸,至十一月望日,忽命家人制縗服,北向哭,未幾,哀詔至,正聖祖崩之後二日也。後卒於家。

張永祚,字景韶,浙江錢塘人。幼即喜仰觀五緯,長通曉星學,究悉天象。年近三十,督學王蘭生稔其學,錄為諸生。閩浙總督嵇曾筠求通知星象者,試永祚策,立成數千言。薦於朝,授欽天監博士。屢引見,占候悉驗。詔刊二十二史,永祚校勘天文、律歷兩志。及書成,告歸。晚著書,曰天象原委。卒後,有女傳其學。壻沈度,亦善推步,守其書。

戴尚文,湖南漵浦人。諸生。從鴻臚卿羅典學,凡天官星卜諸書,無不究覽。嘗曰:「吾治經,師羅先生。吾術數,未知孰可吾師者?」聞江南某僧精六壬、奇門,往師焉,盡得其秘。歸,應鄉試長沙,同舍生失金,尚文為占曰:「君金若干,盜者青衣,手魚肉,前行,後一白衣隨之,肩荷重物。以某時,候驛步門外,可獲也。」如其言往,果驗。嘗侍母夜坐,心動,知偷兒入宅。取井泥塗灶門,書符封之,偷不得去。

嘉慶初,福康安征苗,招致才異,羅典薦漵浦兩生,一嚴如煜,一即尚文。謂曰:「嚴生負經濟才,應祿仕;汝疏散,為幕客,慎勿官職自羈也。」

尚文見福康安,長揖不拜,福康安欲試其術,握絲帶問曰:「君神算,知吾握中何物?」乃請一字析其數,以五行推之,曰:「絲縷耳。」大驚異,禮遇之,凡事必諮。時苗猖獗,恆夜撲營,尚文輒預卜知之。當五月,進攻旗鼓寨,占:「有大雹,賊伏林莽,師出不利。」勿聽。日午,將抵寨,忽大風,雷雨雹交下,如卵如拳,擊傷士卒,伏苗乘之,果敗。軍中呼曰「神仙」。又大軍在乾州,營龍頭,為苗所圍,斷水,軍不得食。尚文設壇鑿池,以法禳之,劇地,清泉滃出。四年,駐天心寨,尚文夜觀天象,知有咎,作書置幕府,辭歸。數日,福康安遽卒。尚文歸未幾,亦病,自知死日。卒後,其母傷之,焚所傳書。


\end{pinyinscope}