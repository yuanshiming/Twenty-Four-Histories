\article{志一百三十}

\begin{pinyinscope}
○邦交三

△法蘭西

法蘭西一名佛郎機,在歐羅巴之西。清順治四年來廣東互巿,廣東總督佟養甲疏言:「佛郎機國人寓居濠境澳門,與粵商互巿,仍禁深入省會。」法人素崇天主教,康熙以來,屢禁漢人入教。

道光二十五年,法商赴粵,詣總督署,請弛漢人習教之禁。總督耆英據以入告,許之開堂傳教,仍限於海口,禁入內地。咸豐三年十二月,有法輪船一駛入長江,未幾解纜去。而法與英、美又欲變通成約,廣東總督葉名琛以換約未屆期,拒之。遂偕英、美逕赴天津,要求如英、美,並請釋陜西傳教人,長蘆鹽政崇綸等以聞。上以定例五口通商外,不許外人擅入內地,何以陜西盩厔縣有法人傳教?飭令詳查,並嚴詞拒之,乃去。時粵賊踞上海,築砲堤防禦,吉爾杭阿因向法提督辣厄爾告以「賊築砲堤,爾國領事署首當其沖,應速遷以免受傷。」辣厄爾立毀其堤,並砲擊賊。事聞,獎之。六年六月,英、美各國求換約,法公使顧思照會兩廣總督葉名琛,援約與英、美一體,力阻不從。七年十二月二十一日,英人結法公使噶歷為援,襲入廣東省城,擄名琛以去。先是法人謂有人殺其說書老人,向名琛索犯,限三日交出,並要求五事:一,入城;二,索河南地;三,求改章程;四,索補兵費;五,求通商。限日答覆。名琛回牒許通商,餘皆不許,而又不設備,遂至被擄。英、法連檣赴天津,美、俄亦相繼至,各有所求。法人又欲推廣商埠,任意傳教,遣公使駐京,入內地買絲茶,並請查辦廣西西林縣殺馬神父案,皆不許。八年三月,法與英人攻踞海口砲臺,進逼天津。於是命大學士桂良、吏部尚書花沙納往議,徇所請。遂於五月定約,法得通商、傳教及兵費,幾與英等。

九年五月,法公使布爾布隆以進京換約為名,隨英公使普魯斯赴天津,拒不納,致傷敗數百人,折回上海,聲言調兵復仇。未幾,法人復北駛,分擾登、青等處。十年六月,隨英來攻,連陷新河、唐兒沽北岸砲臺,遂入天津。先是遣西寧辦事大臣文俊、武備院卿恆祺往議,不報。至是,又遣桂良、恆福為欽差大臣,往津會議,冀緩師,而法與英益恣要求。初,津約原許補法軍費二百萬,英四百萬。至是,英索倍加,法欲照英數,復要求天津通商、京師長駐。朝旨不許。乃隨英督兵北上,進逼通州,京師戒嚴。怡親王載垣等再議和,不就。進薄京師。八月,恭親王奕留守,再議和。九月,和議成,所得通商、軍費、權利與英等,而傳教、建堂初無限制。十月,始定傳教之人須薙須服中國衣冠,其入內地,預領中、法合同護照,向所過地方官鈐印,以為信據。法人以江南為新許商埠,欲早通商,請助剿粵賊,不許。十一年二月,法公使布爾布隆偕英使普魯斯由津如京,此為各國公使駐京之始。先是條約有還清軍費始行退出廣東省城之議。至是,法人哥士耆來言,原先撤兵退出粵城,並求廣東籓署賃作領事署,又索還京城及各省天主堂舊基,均許之。九月,交還廣東省城。

同治元年正月,粵賊陷蘇、松、常、太等郡,朝議募洋將助剿,法人與焉。是年,貴州提督田興恕殺教民,毀天主堂,法使哥士耆以為言,朝廷命崇實、駱秉章、勞崇光及張亮基入黔查辦,久不決。會哥士耆回國,新公使柏爾德密至,始允照中律擬結。同治四年,法請開江寧商埠。五年,議招工章程。七年冬,四川酉陽州有殺傷教士案,又有貴州遵義民教仇殺事。法使羅淑亞上書稱遠臣,歸咎於中國官吏,且言當離京往津,候本國水師提督到後偕行,以為要挾。命湖廣總督李鴻章查辦,久之始結。十二月,始遣欽使總理各國事務衙門章京志剛、孫家穀偕美前使蒲安臣至法遞國書,見其國主那波侖第三,復見其後,各致頌詞,成禮而退。

九年夏五月,天津民擊殺法領事豐大業。初,天津喧傳天主教堂迷拐幼孩,抉眼割心為藥料,人情洶洶。三口通商大臣崇厚等詣法領事豐大業赴堂同訊,觀者麕集。偶與教堂人違言,磚石相拋擊,豐大業怒,徑至崇厚署忿詈,至擬以洋槍。出遇劉傑,復以槍擊傷某僕,遂群起毆斃豐大業,鳴鑼集眾,焚毀教堂、洋房數處,教民及洋人死者數十人。事聞,命大學士直隸總督曾國籓赴津查辦。國籓至津,示諭士民,宣布懷柔外國、息事安民之意。法公使羅淑亞來見,以四事相要:曰賠修教堂;曰埋葬豐大業;曰查辦地方官;曰懲究兇手。尋牒請將府、縣官及提督陳國瑞抵罪,國籓拒之。與崇厚會奏,稱:「仁慈堂查出男女,訊無被拐情事,懇降諭各省,俾士民咸知謠傳多系虛誣,請將道、府、縣三員均撤任查辦。」奏入,報可。遂於八月擬結,辦為首十數人,天津府、縣減戍黑龍江。

十一年,法遣全權大臣熱福裏如京換約,並進書籍。十二年,穆宗親政,各國請覲見,法與焉。是年法人侵越南,入河內省城。光緒四年,始遣兵部左侍郎郭嵩燾以英使兼法使。明年,代以太常寺少卿曾紀澤。

越南向隸籓屬,自法據西貢,脅越人訂約,許於紅江通舟。曾紀澤與法外部言:「法、越私立之約,中國不能認。」不省。八年二月,法兵船由西貢駛至海防進口。三月,陷河內省。朝議始遣提督黃桂蘭等軍出關。既而法公使寶海向北洋大臣李鴻章要求中國退兵,及通商保勝,驅逐盜賊,畫紅江南北為界。朝廷下各督撫議。法人見不允所求,遂欲增軍撤使以相恫喝。

九年三月,戰事起。法據南定,旋為劉永福所敗。會越王薨,法以兵脅嗣王立新約二十七條,盡攘其兵權、利權、政權,並申明越境全歸保護,中國不得干預。中國聞之,乃命唐炯、徐延旭出關,彭玉麟辦粵防,張佩綸會辦軍務。會山西、北寧連陷,官軍退守太原,法乘勢擾浙、閩,陷基隆、澎湖,至是始宣戰。十年二月,諒山大捷,法忽請和,帝命吳大澂、陳寶琛、張佩綸會辦海防,以議和全權任李鴻章。先是福祿諾所擬五條,僅允不索兵費,不入滇境,而要挾中國不再與聞越事。議久不決。五月,法兵以巡防為名,忽攻諒山,敗走。藉口中國不能如約退師,責賠費,不允。法使巴德諾出京。六月,攻臺北基隆,為劉銘傳所敗。秋七月,法水師提督孤拔等率兵船入閩,泊馬尾等處,迫交船廠,欲據為質。時張佩綸以會辦海防兼船政大臣,漫不設備,法遂開砲毀船廠。復分兵擾東京、臺灣,陷基隆,窺諒山。十一年春正月,犯鎮南關,楊玉科戰沒。旋收復,大創之,並砲斃孤拔於南洋。法人乃請和,原照天津原約,不索償款。李鴻章與議新約十條:一,法自行弭亂,華不派兵赴北圻;二,法與越自立約,或已定或續立,中、越往來,不礙中國威望體面,亦不違此次約;三,六個月會勘界,北圻界處或稍改正,以期兩益;四,法保護人民欲過界入中國,邊員給照,華人入越,請法給照;五,保勝以西、諒山以北通商,華設關,法設領事,北圻亦可駐華領事;六,三個月內會定商款,法運越貨稅照他處較減;七,法在北圻造鐵路,中國若造鐵路,雇法工;八,此約十年再修;九,法即退基隆,二月內臺灣、澎湖全退,中、法前約照舊等語。旋法派戈可當代為駐華公使,欲改前約,出所擬二十四條。鴻章以與原約不符,不許。戈使又欲辦滇、粵礦務,及制造土貨,運越南食鹽,復拒之。又欲於雲南省城及廣西內地設領事。時正遣鄧承修、周德潤與法勘界,鴻章謂宜俟邊界勘明,方能指定通商碼頭。戈使又要求稅則減半,鴻章祗允五分減一。又另擬通商章程十八款,並將互交逃犯、洋藥進出口各條亦擬在內。法使復援咸豐八年約內第七款有「工作」二字,仍要求增入在口制造,許之。

時雲南界務,周德潤會商岑毓英後,出關與法使狄隆晤商,擬先勘保勝上游一二段,並同擬全局辦法八條:一,中、法兩國勘界大臣等說明所應勘之界,俱是現在之界;一,勘現界後,或有改正之處,兩國勘界大臣公同商酌,如彼此意見不合,各請旨商辦;一,續開勘云、越交界,中國大臣等意欲一律勘完,所以照會法國請旨;一,各大臣等商議先由老街勘到龍膊河,及龍膊河鄰近地方,復回老街,再勘老街鄰近地方;一,勘老街至龍膊河之界,中、法繪圖各官從紅江南岸歸,一路同走,中國繪圖官歸法國保護,自老街起至龍膊河止,兩國勘界大臣等各走雲、越邊界;一,紅河自北河岸之老鏊至南岸之龍膊,以河中為界;一,雲、越之界,遇有以河為界,均以河中為界,如有全河現在歸中國界者,仍歸中國,現在歸越南界者,仍歸越南;一,勘界時隨處開節略圖說,均由兩國大臣等畫押。以上節略,彼此畫押遵守。德潤與狄隆各按地圖校改,互有爭執,而於大小賭咒河、猛援、猛賴兩段,爭執尤力。會法勘路弁兵在者蘭被越游勇所戕,法指為雲南提督散勇,中國不承,狄隆欲緩勘,但就圖定界。粵東、粵西界務,鄧承修與張之洞、李秉衡等會商,其與法使浦理燮在關門文淵會議。承修執約內「北圻邊界必要更正,以期兩國有益」之語,欲以諒山迤西自艽葑、高平省至保樂州,東自祿平、那陽、先妥州至海寧府劃歸中界。浦使以據約不過於兩邊界址略為更改,不能及諒山及東西地。旋允請示本國,卒不行。十二年復議界,會浦理燮病,僅由鎮南起勘至平關而止,東西不過三百餘里,餘未履勘。浦理燮旋回國,法改派狄隆由滇赴粵,與鄧承修等議界。

先是鴻章欲先議界,後議商約,法使不從,乃復議商約。至是議成十九款:一,保勝以上某處、諒山以北某處,中國設關通商,許法設立領事;二,中國可在河內、海防二處設立領事,並可商酌在北圻他處設領事,惟須後日;三,兩國領事駐扎及商民通商,均須優待;四,中國人在越置地建屋,及官商往來公文、書信、電報,法允保護遞送;五,兩國游歷人過界,各發給護照;六、七,出口貨照稅則三分減一,進口貨照稅則五分減一,估價之貨為稅則所未載者,進出口仍照值百抽五徵收,至洋土各貨赴內地買賣,應完子口稅,不在減徵之列;八、九,載明洋、土各貨在邊關已完稅,復轉運通商各海關者,均照海關稅則另收正稅,不以邊關單作抵,其在邊關所領存票,亦只準在邊關抵稅,概不發還現銀;十至十二,嚴防詐偽偷漏之法;十三,定洋人自用雜物免稅之法;十四,定洋、土各藥不準販運買賣;十五,米穀等糧不準販運出中國邊關,進關準免稅,違禁物各禁;十六,中國商民僑居越南,所有命案、賦稅、詞訟等件,法國應優待;十七,中國人犯罪,照中律,法領事宜拘送,不得庇匿;十八、十九,定條約續修期限及互換遵守各事。是為滇粵邊界通商約。

商約既定,鄧承修即赴欽州之東興與狄隆議勘東界。狄隆以中國所屬江平、黃竹、白龍尾為越境。鄧承修以數地皆內地,有圖可據,不許。辯論不洽。狄隆又約履勘,承修欲照雲南分途履勘辦法,並請先撤江平法兵。越日,復議請旨立約三條:一,大段相合;二,較圖不合,作為未定,各請示本國;三,勒其去江平之兵及辦事官員。又令以後未定界內,不得再派兵及官員前往。狄隆不允,轉要承修不得於未定界內駐兵。時張之洞所派道員王之春、李興銳亦與會議。議界將及一年,中國屢請撤兵,法兵分屯江平、黃竹、石角、句冬、白龍尾等處如故。會總署允承修所定三條,承修命王之春往議,狄隆執不允,而法人突以兵踞白龍尾,驅害汛兵。華民築營壘,承修詰令撤退,狄隆諉之。時桂界已校竣,欽界南自嘉隆河、北抵北侖十萬山分茅嶺、西至峒中墟北,亦允歸中國,而白龍、江平,狄隆謂須以商務抵換。又以九頭山未議,及之春與議,亦無效。狄隆又欲議海界,以津約所無,未奉旨議海界,卻之。法又欲以白龍、江平抵換龍州通商。初恭思當來華也,即有求改商約之請,總署以界務方殷,且商約既經畫押,何能議改?拒之。至是復以為請,並以商務茍可通融,界務亦可稍讓。稱已奉本國訓,準令在京商辦。總署以狄隆與鄧承修議界久不決,允與商辦。恭思當始允中國廣東邊界除現在勘界大臣劃定外,所有白龍尾及江平、黃竹一帶地方,並云南邊界前歸另議之南丹山以北、西至狗頭寨、東至清水河一帶地方,均歸中國管轄。又議減稅,總署以俄國通商章程辦有成案,滇、桂邊界皆為陸路,不得不酌議減稅,以歸平允。於是議進口稅減十分之三,出口稅減十分之四,滇土藥每百斤定稅釐各二十兩,必完釐者,方準法商完稅接買,並不準法、越商人往入內地販運,高平、諒山往來之船只免徵稅,仍納船錢,惟運販食鹽、接辦鐵路及越南與滇、粵通商進出口稅則,均請減半,運中國土貨往中國各海口,稅則減三分之一各節,均拒絕刪節。計訂商務續約十條,界務續約四條。又照會緩設領事,及法在龍、蒙等處之領事等官,不得設立租界二端。是為與法勘界通商續約。

十四年,法領事藉口華船常到海防,向廉州請示諭船戶須向領事領照,無照即將船扣留。張之洞以條約向章所無,海防各國船只均可往,何獨華船不許?嗣聞法領事張貼告白,收取船規,每船輸銀自數元至數十元不等,云系法使所定。之洞致總署請其停止收規。是年,法人請接中國兩粵電線,許之。又芒街法兵越界焚劫那沙,之洞致總署,請向法使責賠償。十五年,法船駛進瓊州所屬崖州東百里之榆林港測探水道,上岸釘椿插標,阻之。法領事又在北海徵收漁船照費,政府以有侵中國主權,不許。十月,定界委員李受彤與法官勘東興一帶河界,定議此後河中淤有沙洲,近華者歸華,近越者歸越,河道即有更改,無論河在何境,兩國均許行船。是年,法使以華兵駐越南之板邦為言。又稱那沙墟不在中國界內,實在北圻橫模社對面先安河北岸,與板邦相近。又稱去冬官兵迎收被剿敗匪,系指離芒街八里之寧陽大廟對面大河北岸而言。並命查復。嗣李受彤復電,謂:「州西分界,自八莊歷板興、板山、冷峒止,前有溝離越南峒中三里,即以此溝為界,冷峒系醜艮寅向,峒中系未坤申向,那沙在西北,戌乾亥向,峒中墟居中,兩旁有溝,水向西合流入先安河。以方向論,溝西南概為越地,溝西北概為華地。以社論,那沙與板峒為建延社地,與峒中為橫模社地無涉。以交界論,那沙北歷那懷,約二十五里即北巖,系廣西上思州地。以欽差所定界圖論,那懷屬我,那沙即附連那懷,相離僅三里,前並無墟。去年正月,峒中墟華民始由峒中遷此。去年十一月以前,法未逾溝到此,十二月始有焚殺那沙墟事,擄去婦女,隨即給銀放回。其法官自向婦女言系逾界誤拏。再查界圖,西北有板邦隘,系廣西地。又土人言橫模西南離六十里有板邦,屬越地。峒中之東並無板邦,只有板奔,離峒中約九里,系內地。去年秋,萃軍防營駐此,因疫退駐板興,今板奔並無防勇。又查寧陽離芒街十餘里,在東興西南,中隔河,必船乃渡,即有勇亦難迎庇,且並無勇。」等語。又馮子材電亦云然。張之洞以兩說歧異,由於華民以溝水為界,法以先安河北岸為界。溝即河也,原圖均未指明。那沙系去年正月新立之墟,距界甚近,故致彼此爭執。既悉板邦隘另是一地,實屬廣西。

十六年九月,歸逃人魏名高等十八人。十七年八月,法使林椿改擬新咖雷多尼招工合同第十四條。緣第十四條中國原擬派員作「理事官」,林使不允,改作為「華工統領」,所得權利僅止赴訴公堂及請狀師理論。李鴻章以所改仍與工頭無異,焉得有權保護?不許。時湖南民攻詆洋教,法領事欲赴長沙開馬頭、設教堂,阻之。十九年四月,請東興、芒街接修電線。粵督以前辦界案,尚有數十里至今未定,遽與接線,界未劃定之處歸何人保護?必致多生轇轕。仍促先速定界。二十年,法使日海遞國書。又議寓越華人減身稅事,並論暹羅邊界。李鴻章據英與法議暹羅交界有甌脫地,應歸中國,日海不允。三月,與法會勘欽、越界。初,法派巴拉第、法蘭亭均以約內載明屬我之板興、嶺懷等處爭為己有,政府不允。至是法改派柯麻暨其總辦籞釐籥接辦。粵督李瀚章派李受彤與會勘,始知巴拉第、法蘭亭所爭險要,與越南皆隔深溝峻嶺,而溝尤多。因與約定,按界線有水處以水為界,有山處以山為界,計長四百里。陸界僅五十里,皆峻嶺,餘悉溝界,惟披勞縱橫約三里,各分一半。餘如原勘圖約所載,分茅嶺、板興、板典、嶺懷等處,及峒中十里,均歸中國。時滇、越亦議界。滇督王文韶不允爭已定界,祗就黃樹皮、箐門及猛岡各處向駐有華兵處,緩撤兵以待法防之至。界約遂定。二十一年,中、日約成,法求換商約、界約,遂許開龍州、蒙自等埠,並與越界線內猛烏、烏得二地。初,中國認此二地為寧洱縣屬車裏土司之地,法使謂舊屬越,遂歸法有。

二十三年,法要求瓊州不割讓租借於他國,許之。二十四年,法乘廣東雷州人殺其士民二人,以兵艦據廣州灣,來商租借,言為停船屯煤之所,無損中國主權,而所租借跨高、雷二府之間,由海岸以入內地,所得東海、匈洲各島,及赤坎、志滿、新墟等處,均歸入租界。又得吳川之半島及通明港。是年,又以兵強占上海、寧波四明公所義地,寧人罷巿,幾激變。久之始定。時廣西永安有殺斃法教民之事,方議辦犯、劾官、賠償、建堂四條,適值北海鐵路造至南寧,援龍州鐵路案,中、法合辦,法使遂要求將鐵路歸並教案。議久始允就案議結,不及他事。又施南、宜昌、長沙均因教堂、教民啟釁未結。二十六年春,拳匪亂,法人調兵與德、英、俄、美、日本聯軍入京,復督兵西進至廣昌,屢阻之。二十七年,展漢口租界。是年法遣鮑渥為駐華公使。二十八年,外務部與法隆興公司總辦彌樂石訂雲南礦務章程。先是彌樂石到滇,與礦務大臣唐炯議欲設中西礦務公司,唐炯入告,奉旨交雲貴總督魏光燾等與彌樂石議,歷七閱月始竣。乃入奏,略謂:一,初議限制中國公司延聘礦師,貸用洋款,後亦不入別國洋股,專用英、法礦師,定議;一,運礦自修鐵路,接通滇越幹路,訂明俟幹路成時再議,並禁售票搭載客貨,預存限制;一,公司收買山地,按民間租價,公平租賃,地由滇官指交,價由公司照給,逾限三年不辦,原地歸還業主;一,完納礦稅,議定按出井出爐礦質,每百抽五,抵納稅課,並派員分礦監收。適彌樂石由滇入京,向外務部催訂合同,外務部告以礦地未定,未便先議章程,並不準攬辦全省。彌樂石允指澂江、臨安、開化、雲南、楚雄等府及元江州、永北凡七處,載入章程第一款內,將原議「嗣後別國公司概不準來滇辦礦」,改為「嗣後別國公司概不準在公司所指之地勘採」,以清界限。彌樂石以原議包辦全省礦利,故原歲給京銅一百五十萬斤,並津貼員弁兵勇護廠銀二萬兩。今既改為七處,應請減議定繳京銅一百萬斤。護廠費由公司給發,不拘定數。招募土勇,改為稟請地方官招募,遴選武官一員管帶。遂定議。惟第一款內載有「公司尋出之金、銀、煤、鐵、五金、白銅、錫及火油、寶石、硃沙礦,允給公司承辦」等語,滇督魏光燾以礦類白金、白銅、錫三項為原章所無,因咨外務部,請照滇中前定原章,照會英、法公使,轉令彌樂石仍將三項刪除。

二十九年,總理外務部慶親王奕劻與法使呂班訂滇越鐵路條約三十四條:一,鐵路自河口抵蒙自,或由蒙自附近至雲南省城,日後擬改,須彼此商準;二至四,勘路繪圖及交地購地各事;五,各項廠棧同時開工;六,鐵軌寬一邁當;七,鐵路經過地方,不得損壞城垣公署;八、九,購料及挖取沙石、採伐林木各事;十,運路及暫時興工各地,用竣後即交還;十一,幹路造成,商接支路;十二,各執事凡須專門學者,可用外國人;十三、四,工匠之招募管理及賞恤傷亡、懲辦犯罪各辦法;十五,巡丁可募土民,不得請派西兵;十六,洋員請給護照事;十八,租賃房屋事;十九,不得損及民人產業,有則賠償;二十,火藥炸藥之運制及防險;二十一、二,運貨納稅、免稅各例;二十三,收費、減費、免費各例;二十四,鐵路不準載運交鹽及西國兵械,如中國有戰事,悉聽調度;二十八,設專門學堂;二十九,設電線、電話;三十一,滇省派員襄助公司;三十二,定公司補償中國查看費,各員來往照料費;三十四,此路十八年期滿,中國可與法國商議收回。是年,法人因吉林教案索賠償。三十年秋七月,法使館交還欽天監觀象臺儀器二十八件。三十一年春,法商欲自上海至紹興行輪,阻之。是年與各國定值百抽五稅則,法有違言,久之始允。三十二年春正月二十九日,南昌縣知縣江召棠被殺於天主堂。先是召棠辦教案頗持正。法教士王安之因上年荏港教案,有二教民鄧貴和、葛洪泰在南昌縣監禁,強請釋放,召棠向索縱囚,其一匿法教堂中,王安之不交,函約召棠會飲,被殺。民情大憤,集眾毀法教堂,傷斃王安之及教習等數名,並波及英教堂,久之始定。法人欲坐召棠自刎,及派兵船來贛責償。命鄂督張之洞查辦,屢執仵傷單及醫憑單與爭,終徇其請,賠以法銀二十餘萬。三十三年,法遣領事入滇商辦事。六月,蒙自法郵局設代收遞人役,詰之。九月,索還法人所占塘沽碼頭。宣統三年,與四國銀行定粵漢川漢鐵路借款合同。原借五百五十萬金鎊,五釐行息,專為築造粵漢、川漢兩路,法與英、德、美均與焉。


\end{pinyinscope}