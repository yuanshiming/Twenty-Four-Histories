\article{列傳一百三十六}

\begin{pinyinscope}
王文雄硃射鬥子樹穆克登布富成穆維

施縉李紹祖宋延清袁國璜何元卿諸神保達三泰

德齡保興凝德多爾濟扎布王凱王懋賞

惠倫安祿佛住西津泰豐伸布阿爾薩朗烏什哈達和興額

王文雄,字叔師,貴州玉屏人。由行伍從征緬甸、金川,擢至游擊,洊升直隸通州協副將。嘉慶元年,調剿襄陽教匪,從慶成戰劉家集、梁家岡、張家垱,賜號法佛禮巴圖魯。秋,賊圍鍾祥,進擊破之,擢南陽鎮總兵。冬,賊分竄河南,命率兵二千回境防禦。二年春,敗賊禹山,又敗之鄭家河;追剿至裕州四里店,值他軍與賊戰,夾擊敗之。夏,息縣奸民張雲路倡亂,馳剿即平。秋,仍赴襄陽。時姚之富等逼南漳,文雄駐軍五盤山,扼其沖,擊賊於白虎頭、峽口。聞賊竄陸坪,分兵擊之,追至羊角山,斬其渠。賊乃以數百人綴官軍,潛趨南漳城,文雄伏兵百步梯,火之,賊多墜崖死,遂赴陜西、河南界禦賊,且防興安江岸。

三年春,高均德自寧羌渡漢,齊王氏、姚之富乘官軍往剿,偕李全自西鄉、洋縣分道踵渡,掠郿縣、盩厔,西安戒嚴。文雄馳援,敗賊焦家鎮,追至屹子村,猝遇賊萬餘。文雄兵不滿二千,張兩翼待。賊亦分左右來犯,為火器擊退;復分四路至,又敗之,遂悉馬步圍官軍數重,文雄為圓陣外向,賊以千餘騎猛撲,令藤牌兵大呼躍出,賊馬驚,返奔,追殺數千人,斃其黨王士奇。自是賊不敢北犯,省城獲安。詔以文雄當數倍之賊,五戰,所殺過當,深嘉之,立擢固原提督。追敗賊於尹家沖,其分竄翔峪、澧峪者盡殲焉。夏,敗高均德於盩厔,又敗阮正通於南鄭。秋,張漢潮由南鄭東南竄,文雄冒雨疾馳兩晝夜,追及於廉水戺,賊踞山,以砲仰擊之,乃分馬步隊潛來鈔截,三路迎擊,斃賊千餘。正通竄西鄉西流河,而設伏於銅廠山梁,文雄分兵破其伏,自攻中堅,擒張金等。

四年,命與恆瑞分領總督宜綿所部兵,專剿陜境竄匪。秋,敗茍文明於倒水洞,連敗冉天元於沙田壩、景山坪、皮貨鋪,川賊龍紹周竄池壩,欲應天元,扼之貫子山。別賊冒齊家營者來犯,悉殲之。黃號伍義蘭、藍號曾六兒踞老鷹崖,分兵進擊,擒其黨李智花等,餘賊遁入川。冬,樊人傑、唐大信竄西鄉。文雄積勞嘔血,力疾督戰,溫詔慰勞。尋,黃號諸賊復自川入陜,令游擊梁煥擊之,遇伏幾殆,馳救,解其圍。疾復作,而賊之匿老林者,潛出犯南鄭、沔縣、略陽,欲渡嘉陵江,詔斥疏防,當治罪,以病原之。

五年夏,敗楊開甲於土門關。唐大信踞西鄉節草壩、大祥壩,夜襲克之。龍紹周與大信合,敗之魏家寨,又連敗之黑山萬曲灣、火石埡、山王廟,賊遁入川。未幾,高天德、馬學禮及戴家營賊竄西鄉堰口,窺縣城,迎擊敗之。偵賊眾潛屯法寶山,夜偕副將鮑貴等分三路進,賊擲石以拒,文雄督眾仰攻,突有騎賊從溝中出,截其後,山上賊出間道撲鮑貴隊,急趨救,賊乘勢悉眾下山,鏖戰至午,圍益急,文雄被創十餘,猶力鬥,左臂斷,墜馬,伏地北向呼曰:「不能仰報君恩矣!」遂卒。仁宗震悼,封三等子爵,祀昭忠祠,謚壯節,諭慰其母,賜銀千兩。逾年,獲戕文雄之賊馬應祥,命傳首就其家致祭。子開雲,襲子爵,官至山東鹽運使。

硃射鬥,字文光,貴州貴築人。幼讀書。入伍,從征緬甸、金川,功多,累擢至都司。果毅善戰,為將軍阿桂所激賞,洊升貴州平遠協副將。乾隆五十年,擢湖南鎮筸鎮總兵,調雲南普洱鎮,民、苗雜處,綏撫得宜,邊氓翕服。從征廓爾喀,歷福建福寧鎮、四川川北鎮。苗疆事起,率本鎮兵赴剿,迭克險隘。平隴之戰,潛師襲後山貫魚坡,賊乃潰。偕額勒登保攻石隆寨,伏溝下驀入,斷其要路,賊來爭,奮擊盡殲,遂斬賊魁石柳鄧,賜號幹勇巴圖魯。

嘉慶二年春,凱撤回川北,王三槐踞金瓘寺,合攻克之,連破王家寨、茨茹梁、富成寨,要擊於黃家山,三槐中槍,墜馬跳免。合攻重石子、香爐坪賊巢,擊秋波梁竄匪,殲之。偕總兵百祥攻羅其清、冉文儔於方山坪,敗走巴州。射鬥駐保寧,詔以本鎮轄地,責嚴守。三槐撲天華山營,力戰卻之。又合徐天德分撲風門鋪、角山、茶店,馳擊,賊遁走。三年,其清竄儀隴雙路場,偕穆克登布追剿,斬七百餘級。其清等踞大鵬寨,諸軍合攻,射鬥與恆瑞當其北,賊冒雨突營,出其後夾擊,賊竄伏深溝,悉擒之。及賊由青觀山逸出,追至方山坪,奮擊大潰,其清旋就擒。

四年春,從德楞泰破賊麻壩寨,獲文儔。既而蕭占國、張長庚竄營山,額勒登保迎擊黃土坪,令射鬥扼其西,占國、長庚就殲。夏,包正洪潛匿鄰水,連敗之唐家坪、趙家場,追至開縣九龍山,痛殲之;窮追及毛坪,賊踞山,以火槍仰擊,斃正洪,予騎都尉世職。秋,破卜三聘於八石坪,又截擊高天德、馬學禮,擒其黨潘受榮。

五年春,張世龍竄南江,迭敗之竹壩、草廟。會額勒登保、德楞泰先後赴陜,冉天元糾群賊乘虛入川。總督魁倫初任軍事,諸將中惟射鬥忠勇可恃,所部兵僅二千,至達州,賊已渡嘉陵江,乃自順慶渡河,迎擊於西充文井場,殲賊後隊;乘勝至蓬溪高院場,賊踞山下撲,眾數倍官軍,遂被圍。魁倫初約自率兵繼進而不至,射鬥力戰,隊伍沖斷,手刃十餘人,遇坎墜馬,歿於陣。仁宗悼惜,晉二等輕車都尉世職,依提督例賜恤,謚勇烈,入祀昭忠祠。後獲賊李自剛戕射鬥者,詔磔之,設射鬥靈致祭,復傳首祭墓。

射鬥從軍三十四年,受高宗知,仁宗尤以宿將重之。額勒登保入川數大捷,皆倚射鬥及楊遇春如左右手,賊畏之,號曰「硃虎」。在軍得士心,尤恤難民,前後拯濟不下萬人。歿後兵民胥流涕。賊既退,收遺骸,遺左足,川民於戰處得之,瘞於潼川鳳皇山仙人掌,建祠以祀。

子樹,襲世職,授戶部主事。道光中,累官漕運總督,休致歸。咸豐中,命治本籍團練捐輸事宜。同治初,卒。

穆克登布,鈕祜祿氏,滿洲正紅旗人,將軍成德子。乾隆中,成德駐西藏,入覲,高宗詢知穆克登布曾從征金川,授藍翎侍衛。累擢直隸提標游擊。嘉慶元年,從剿湖北教匪,以功賞花翎。遷山東參將,遂轉戰川、陜。四年春,從惠齡克麻壩寨,加總兵銜,擢貴州清江協副將。從額勒登保殲閬中賊蕭占國、張長庚,乘勝進剿冷天祿於岳池。令穆克登布先據人頭堰,與楊遇春夾擊,大破賊眾,殲天祿,賜號濟特庫勒特依巴圖魯。於是額勒登保軍威大振,遂任經略,穆克登布與楊遇春為左右翼長,常為軍鋒。冬,與七十五夾擊樊人傑於通江,敗之,擢山西太原鎮總兵。

時川賊徐天德、王登廷、冉天元合撓官軍,阻餉道。額勒登保以賊皆勁悍,集師合擊於蒼溪貓兒埡,議與穆克登布、楊遇春分三路進攻。穆克登布恃勇,先期往,為賊所乘,腹背受敵,傷亡副將以下二十四人、士卒數百。及遇春至,據險與賊相持,經略中軍亦被攻,血戰竟夜,黎明賊始卻,登廷旋就擒。偕遇春追天元至開縣,與德楞泰會師夾擊,賊勢乃蹙。

五年,從經略入陜。夏,與楊遇春合擊伍金柱於手扳崖、銅錢窖,追殲楊開甲於茅坪。秋,要擊張天倫於兩當剪子巖,追殺數十里。賊折奔階州,遇於佛堂寺,擊敗之,斬其渠曾印。六年春,冉學勝將入陜,雪夜率勁騎沖之,賊潰,又敗伍懷志於五郎江口,擢乾清門侍衛。夏,伍懷志糾黨由漢北東竄,分兵晝夜窮追,及之於秦嶺,擒懷志,餘黨盡殲,予雲騎尉世職。七年,調湖南永州鎮,擢甘肅提督。馳剿川東、湖北竄匪,破王國賢於平利,追入川,迭敗賊於青岡坪、太平坡,擒景英。是年,軍事將蕆,錄諸將功,擢御前侍衛,晉騎都尉世職。

八年春,搜捕餘匪,由巴峪關深入,擒宋應伏,又擒姚馨佐等於南江。應伏最悍,馨佐乃之富子,皆賊之著名者。應伏黨尚存馮天保、餘佐斌、熊老八,並百戰猾賊。熊老八年二十餘,死黨百餘,皆壯悍矯捷,所用矛長數丈,出沒老林,傷將士甚眾。至是,誘官軍入林,設伏狙伺。穆克登布卞急輕敵,勁卒又為他將分調,倉猝中矛,歿於陣,加予輕車都尉世職,並為二等男爵,謚剛烈。嚴詔捕熊老八,期必獲。武弁陳弼賄降俘取賊尸,偽冒以獻,立擢弼參將,傳首祭穆克登布墓。逾年,羅思舉始捕得老八,磔之,軍中不敢上聞。

子頤齡,襲爵,二等侍衛,孝全成皇后之父也。道光十四年,冊立皇后禮成,追封一等承恩侯,抬入鑲黃旗,謚榮僖,以孫瑚圖哩兼襲兩爵。三十年,文宗即位,晉封三等承恩公,以長子文壽襲,次子文瑞襲男爵。

富成,石莫勒氏,滿洲鑲黃旗人。起健銳營前鋒,從征烏什、大小金川,積勞至參領,歷火器營營總。出為廣西、直隸副將,擢山西太原鎮總兵。坐失察盜馬賊入邊,降京營游擊。復擢山東兗州鎮總兵。嘉慶元年,教匪起,率本鎮兵赴河南協剿。先清鄧州賊巢,進剿呂堰驛、隨州紅土山,黃玉書就擒,敘功,以提督升用。又連敗賊於鍾祥鄧家岡、香花園、南線畈。命兼領直隸、吉林新調兵。

二年,進攻梁家集,總統惠齡與賊戰槐樹岡,富成聞砲聲,馳往夾擊,大敗之。偕慶成合擊劉起榮,又敗賊於溫峽口。襄匪由河南竄入陜境,總督宜綿疏調富成赴西安,率甘肅兵二千、回兵二千助剿。夏,分兵五路圍賊於大涼山下,殲賊千餘,擒其渠李天德等,又連敗賊於雙河口、青莊坪、放牛坡、大石川,擢江南提督。赴漢中寧羌,扼川賊入竄之路,循漢南而西,與明亮夾攻,賊距江近,佯引兵入山,圖潛渡,富成繞出賊後兜擊之,斬獲甚眾,被獎賚。

三年春,赴達州擊退犯城賊,通新寧運道,又連敗賊於竇山關、木竹坪、白山寺,擢成都將軍。命剿徐天德,屢詔責戰甚急。冬,戰清涼寺,殲賊數百。四年,張映祥竄廣元、寧羌,擊之毛家山,又與恆瑞夾擊於略陽、階州。經略勒保疏言其兵力不足,未能制賊,褫職逮問。會富成連敗賊於黃家坪、大水溝、黨家坪、蔣家坪,詔免治罪,以披甲留營效力,駐鎮安防剿。五年夏,總督長麟追剿冉學勝、伍金柱等,而高天德、馬學禮亦來犯,富成馳援徽縣。賊襲長麟營,官軍敗績於架子山,富成力戰被重創,遂歿於陣。上初以剿張映祥久無功,故加重譴,至是惜之,命入祀昭忠祠,予雲騎尉世職,子三等侍衛普亮襲。

時軍事久不定,兵多,或事剽掠,鄉勇尤甚,人目為「紅蓮教」。富成與總兵穆維馭下較嚴,為時所稱云。

維,直隸清苑人。隸督標。乾隆中,山東王倫倡亂,以陣斬賊渠楊壘功,擢千總。賊聞京兵南下,掠糧艘造浮橋,圖西竄,維直搏獲賊砲二,焚其橋,賜號奮勇巴圖魯。累擢膠州協副將。嘉慶元年,偕富成赴襄陽。恆瑞攻劉家集,維率騎兵橫貫賊營,大軍躡其後,獲大捷。師次滾河,賊屯對岸董家岡、梁家坳,維偕王文雄選精兵夜潛渡,破賊營。二年,擢登州鎮總兵。冬,高均德、王廷詔分擾班鳩關,窺渡漢江,偕副都統六十七連敗之雙河塘、土門埡,被優獎。三年春,赴四川,從勒保敗王三槐、徐天德於石壩山,偕富成要賊竹峪關、洪口諸隘,又敗冉文儔於黑馬山。夏,賊出李家山西逸,要之大完山,以砲俯擊,賊退,他將乘勢追擊。維直搗李華寺,破賊巢,勞甚致疾,卒於軍,詔視陣亡例賜恤。

施縉,陜西定邊人。由行伍從征緬甸,累擢雲貴督標都司。苗疆事起,應調隨征,屢有功,賜號毅勇巴圖魯。累擢湖南參將。嘉慶二年,從總督勒保剿貴州仲苗。三月,連克關嶺、巴隴諸要隘,進逼永寧,克下山塘賊寨,解新城圍。五月,與總兵張玉龍分兩翼,進克望城坡、碧峰山賊寨,攻羊腸山,追賊至新店,擒其渠梁阿站等,擢副將。六月,從勒保攻克水煙坪,偕按察使常明設伏八角洞坡,進攻阿捧,毀寨十一。大軍進卡子河,縉分克納賴坡、雞灣寨,攻普坪,渡河解南籠圍;進攻九頭山,擒賊渠陸寶貴,焚其巢,克馬鞭田山寨。七月,破韋七綹須於普磨,擒其孥,圍阿召山梁李景寨,設伏破援賊,擢臨元鎮總兵。偕常明攻安有大寨,率勇士攀藤上,克之,擒賊渠賀阿豆、吳阿降。九月,從勒保克洞灑賊巢,擒首逆韋七綹須。十一月,搜剿上下羅障,直達關嶺,前後克寨二十。調貴州安義鎮。十二月,偕總兵七格等搜剿各路,乘勝擊壩鬱、邅峒諸寨。自捧鮓至黃草壩,賊皆凈盡。松林、紅巖、石門坎、香爐箐諸苗,尚負固抗拒,要擊破之,焚寨十九,特詔嘉獎,予優敘。三年,復從總督鄂輝進剿兩薛巖、師趙屯諸苗,克寨五十,苗境遂平。

五年春,四川教匪復熾,起用勒保,會貴州巡撫常明薦縉率貴州兵往協剿,仁宗知縉剿仲苗奮勇冠軍,為勒保舊部,兵將相習,命所領自為一軍;又慮地利賊情未悉,聽德楞泰節制。三月,至潼川,連破賊於大雙墩、潼河岸。四月,高天德、馬學禮由甘肅竄農安,從勒保迎擊盤龍驛、漩河口,敗之,偕阿哈保迎擊於黃連埡。白號、藍號眾賊竄合江口,奪渡嘉陵江,偕阿哈保分四路進擊,大敗之。詔以嘉陵江西肅清,貴州兵新到屢捷,特予褒敘。時高、馬二賊欲與藍、白諸號合屯竹子山,勒保以龍安西北兩面俱通番地,議分三路兜剿,自率一軍出東北,一軍出西北,而以縉軍由南進。甫抵山南,賊乘高下壓,縉揮軍迎擊,奮力急戰;賊來益眾,猝受矛傷,殞於陣。縉最為勒保所倚,至川以不習地勢致敗,優詔依提督例賜恤,稱為驍將,予騎都尉兼雲騎尉世職。子登科,襲騎都尉;占科,襲雲騎尉。

李紹祖,順天大興人。以武進士授三等侍衛。出為山東武定營游擊,累遷臨清協副將。嘉慶元年,赴襄陽,數擊賊有功,賞花翎。二年,從恆瑞赴四川,迭敗賊於田家壩、大寧山梁、金子梁。三年,擢甘肅巴里坤總兵。秋,合攻打石坡、插旗山、古戰坪,皆捷。冬,從惠齡克馬鞍山賊巢。四年夏,從德楞泰擊賊於王家壩、川埡子。秋,偕七十五破樊人傑於開縣,又敗之臨江市。五年春,冉天元等渡嘉陵江,總督魁倫調七十五往援,會其病,以兵付紹祖,率赴川西,進擊鹽亭、南部。德楞泰擊賊於江油白家壩,檄紹祖馳赴,賊踞箐林口,宵犯紹祖營,擊卻之。賊諜詭稱難民,詣營獻計,誘官軍往,德楞泰知其詐,率紹祖掩擊之,大捷,追敗之於包家溝,進戰火石埡。以功被優敘。詔以川西略定,命紹祖率貴州兵赴陜,額勒登保疏請仍留川,遂從德楞泰擊張子聰於中江黃鹿山、硃家坪,擒斬甚眾。調四川松潘鎮,旋調廣東高廉鎮,仍留軍。夏,敗張子聰、庹向瑤於達州土主河,又擊劉朝選於七孔溪山,大破之。追餘匪至大竹,遇茍文明屢夜來撲營,擊卻之。八月,徐萬富竄房縣,追敗之兩河口。賊竄木瓜鋪,偪近遠安縣城,紹祖扼之牛鹿坡。賊分二隊,一犯縣城,一薄紹祖營。紹祖力拒,賊佯敗走,匹馬追之,遇伏被害。依提督例賜恤,謚果壯,予騎都尉兼雲騎尉世職,子霖襲。

宋延清,山東招遠人。乾隆四十六年武進士,授藍翎侍衛。出為貴州都司,遷游擊。從征苗疆,迭克峒寨。從額勒登保攻鴨保山,率健卒奪賊卡,夜大風,攀崖縱火,克之,賜號蹻勇巴圖魯,擢參將。仲苗之役,勒保調回貴州,率兵為左翼,克關嶺、碧峰山諸隘,破洞灑、當丈賊巢。論功居最,擢大定協副將。嘉慶三年,從勒保赴四川,擊賊董谿口、大元山,皆力戰,斬馘多。乘勝追賊至楊家壩,中槍,歿於陣。延清驍勇出眾,勒保常置左右。剿仲苗時,每戰歸,持刀負首級累累,衣盡赤,勒保輒手酌酒慰勞。至川未逾月即戰歿,深惜之,加等賜恤,予騎都尉世職。

袁國璜,四川成都人。由行伍從征金川,屢克堅碉,擢守備。復革布什咱全境及達爾圖,功皆最,洊升游擊。金川平,擢江南狼山鎮總兵。乾隆五十三年,從征臺灣,克大埔尾、斗六門、水沙連、大里杙,賜號博濟巴圖魯。及林爽文竄匿東勢角,山徑深隘,徒步搜捕,生擒於老■K8崎,被優敘。病歸,起署四川建昌鎮,尋授重慶鎮總兵。從征廓爾喀,克象巴宗山、甲爾古拉卡。臺灣、廓爾喀兩次論功,再圖像紫光閣。六十年,從總督孫士毅由川境進剿苗疆,數有功,被褒賞。

嘉慶元年,四川教匪蜂起,蔓延數縣。川兵多赴苗疆,署總督英善倉猝偕副都統勒禮善、佛住馳往,兵僅數千,檄國璜及總兵何元卿進剿達州。賊屯天星橋,國璜奮擊,斬戮千餘。賊竄橫山子,偕元卿焚其卡,奪據山梁。賊自東鄉糾黨數千來犯,砲擊之退,次日復聚,迎擊,斃賊數百,而來者愈眾。國璜苦戰三日,力竭陣亡,依提督例賜恤,予騎都尉兼雲騎尉世職,子起襲。

何元卿,四川華陽人。從征金川、廓爾喀、苗疆,積勞擢副將。嘉慶元年,從福寧克旗鼓寨,擢陜西興漢鎮總兵。達州橫山子之戰,與國璜同遇害,予騎都尉兼雲騎尉世職。孫勝先襲,官至湖南沅州協副將。

諸神保,馬佳氏,滿洲正紅旗人。起護軍校,出為四川游擊,駐西藏,累擢重慶鎮總兵。廓爾喀之役,守絨轄要隘,賞花翎。調建昌鎮,從征苗疆。嘉慶元年,赴湖北剿教匪,從福寧破賊來鳳,克旗鼓寨,賜號喀勒春巴圖魯。二年,從額勒登保圍攻芭葉山,賊夜突營,由諸神保汛地逸出,坐褫職,留營自贖。尋擊賊紅土溪,被創墜馬陣亡,依參將例賜恤,予雲騎尉世職。

達三泰,原名達音泰,呢瑪奇氏,滿洲鑲黃旗人。由鳥槍藍翎長累遷副護軍參領。從征石峰堡,授陜西循化營參將。歷甘肅永固協副將,署西寧鎮。從征廓爾喀有功,賜號常勇巴圖魯,授四川松潘鎮總兵。乾隆六十年,湖南苗犯酉陽,率屯土兵擊之,克砲木山、石花諸寨。偕提督花連布進解永綏圍,又偕阿哈保、塞靈額攻納共山,攀縋而上,斬獲甚眾。克貴道嶺、馬鞍山,追賊黃土坡,被創力戰,大捷,特賜蟒服。又破貫魚坡,苗疆平。嘉慶二年,移軍湖北剿教匪,遂赴四川。齊王氏、姚之富趨達州,欲與王三槐等合,達三泰先據白帝城,連戰卻之,進援巫山、巴東,要擊之小河口,又追敗之均州、竹溪。賊復由陜入川,與明亮合擊於黃壩驛。三年,從大軍逼賊三岔河,齊、王二賊就殲,被優賚。尋擊高均德於山陽,合圍大神山,設伏誘賊,敗之靜邊寺,擒斬甚眾。會諸軍克箕山,擢甘肅提督。勒保調赴川東助剿冷天祿,攻手把巖,奪魚鱗口賊卡,遇伏被害。優恤,謚壯節,予騎都尉兼雲騎尉世職,子呢瑪善襲。

呢瑪善從父軍中,以戰功授藍翎侍衛。父歿,轉戰三省,累擢頭等侍衛,授河北鎮總兵,歷鄖、衢州、南陽諸鎮。道光初,擢成都將軍,平果洛克番匪。卒,謚勤襄。

德齡,納喇氏,滿洲鑲白旗人。由拜唐阿累擢鑾輿衛冠軍使。出為直隸副將,擢山西太原鎮總兵。調赴襄陽剿教匪,從慶成等轉戰,以功賜花翎。嘉慶二年,駐防夔州。三年,偕觀成合攻老木園。賊既殲,剿鐵瓦寺餘匪。四年秋,擊張金魁於岳池場、安家山,敗之。追至萬縣陳家坡,後隊為賊所襲,馳馬回戰,歿於陣,予騎都尉世職。

保興,承吉氏,滿洲鑲白旗人。鳥槍護軍隊長。從征緬甸、金川,累遷參領。出為陜西神木協副將,丁憂回旗。甘肅撒拉爾回叛,起署河州協。兵事初定,撫綏有法,軍民安之。調督標中軍,擢直隸宣化鎮總兵,歷陜西興漢鎮、甘肅河州鎮。嘉慶二年,赴川、陜剿教匪。偕硃射鬥擊賊營山,又敗之小埡口。王三槐擾大竹、廣安,要擊之。鄰水被圍,知縣楊為龍堅守,馳援,賊始退,被優賚,偕硃射鬥破賊天華山,乘勝連奪要隘。三年,攻彈子壩,殲賊渠。時王三槐犯開縣,羅其清、冉文儔合踞東鄉後河,將窺陜。保興繞出賊前,與楊秀夾擊,敗賊於固軍壩,賞花翎。賊自陜回擾達州,保興要擊於龍鳳埡。又戰石梯坎,徑路紛歧,會大風雨,賊壓而陣,遂遇害。予騎都尉世職,河州民為立祠。

凝德,烏雅氏,滿洲正黃旗人,尚書官保子。授藍翎侍衛,歷鑾輿衛治儀正、冠軍使。出為直隸獨石口副將,謂督標中軍。嘉慶元年,赴湖北軍,從破黃玉貴於紅土山,賞花翎。二年,赴孤山沖防剿,尋入川。王三槐擾渠縣,扼守紅春壩。四年,擢甘肅巴里坤總兵。從恆瑞剿賊甘肅,駐守三曹河。賊北走,追敗之老柏樹、牟家壩、兩河口。五年,辛聰餘黨竄秦安,訛言伏羌被圍,凝德率兵四百赴援,未至四十里遇賊,眾寡不敵,拒戰被害。予騎都尉世職。

多爾濟扎布,巴魯特氏,蒙古鑲黃旗人。由藍翎侍衛累擢湖北鄖陽參將。從剿鎮筸苗,遷副將。嘉慶元年,檄防竹山、竹谿。三年,署宜昌鎮總兵。從擊張漢潮於山中,躡蹤窮追,被嘉獎。五年,授廣東碣石鎮總兵。二月,剿陜匪於洵陽三岔山,乘勝深入,賊分隊繞襲後路,四面受敵,揮軍殺賊百餘,日暮力盡,被害。予騎都尉世職。

王凱,貴州貴築人。從征金川,積勞至游擊,累擢浙江定海鎮總兵。嘉慶二年,以不諳水師降副將,命赴貴州從勒保剿仲苗,補都勻協。三年,授宜昌鎮總兵,駐守鄖縣,敗賊於黃龍灘。率兵二千,分守鄖西、巴州,防張漢潮。四年,賊竄房縣,擊走之。五年,復來犯,大敗其眾,又破賊於東湖。夏,徐天德窺襄、鄖兵單,犯當陽、遠安,踞馬鞍山,合諸軍環攻,凱傍左麓進,賊走馬家營。師分三路入,賊張左右翼拒戰,別遣步隊鈔截後路,凱奮擊,賊稍卻,兵進遇伏,賊自林中出,猝被害。優恤,謚勇壯,予騎都尉世職。

王懋賞,山東福山人。乾隆四十一年一甲一名武進士,授頭等侍衛。出為雲南景蒙營游擊,累遷廣西潯州協副將。從征苗疆,克結石岡,破尖雲山,復乾州,皆有功。嘉慶二年,以剿西隆匪,回廣西。五年,調赴湖北軍。六年,敗賊佘家河、茅倫山,賞花翎。攻鵝坪坡、秦家坪,擢湖南永州鎮總兵,駐守興州、房縣、大竹,防川、陜竄賊。七年,曾家秀等竄保康,倍道窮追,賊踞馬鬃嶺拒戰,懋賞先登,中矛,歿於陣。予騎都尉世職。

惠倫,富察氏,滿洲鑲黃旗人,一等承恩公奎林子。出嗣伯父一等誠嘉毅勇公明瑞,襲爵,擢頭等侍衛、尚茶正、鑲藍旗護軍統領,授奉宸苑卿。嘉慶二年,命偕副都統阿哈保率東三省兵赴湖北剿教匪,時賊氛方熾,詔惠倫迅往襄陽,如明亮、德楞泰猶在賊後,即會同王文雄攻剿,聽景安調度。惠倫至襄陽,擊賊小河口,偕阿哈保追殺二十餘里。大兵適自荊州至,乘機夾擊,賊大敗,竄入南漳山中,優詔獎賚。又偕德楞泰擊賊耗子溝,賊眾猛撲,達三泰連射賊,惠倫揮軍突進,沖入賊陣,會明亮自楓樹埡夾攻,斬獲甚多。賊竄花石嶺,總兵長春誘之下山,達三泰設伏山半,惠倫以勁騎橫擊。賊敗竄黃龍灘,欲分走鄖陽斗河,無船可渡。追及草甸,賊五路迎拒,官軍亦分五隊,明亮等據山梁,賊上撲,擊敗之。別賊突出援,惠倫等又敗之。乃奔陳家山梁,乘霧圖遁。惠倫渡澗追擊,見一賊執旗指揮,知為渠魁,追至長坪,射之,應弦倒;餘賊競集,連射斃數賊,猝中槍,歿於陣。仁宗震悼,詔惠倫父子效命疆場,實為可憫,從優議恤,賜內帑三千兩治喪,以子博啟圖襲公爵,在御前侍衛行走。博啟圖自有傳。

安祿,多拉爾氏,滿洲鑲黃旗人,一等超勇公海蘭察子。以海蘭察平石峰堡功,推恩授二等侍衛、乾清門行走,並予騎都尉世職。從征廓爾喀,賜號哈什巴巴圖魯。乾隆五十八年,承襲公爵,擢頭等侍衛。嘉慶四年,命解餉赴四川,遂從額勒登保軍。時徐天德敗竄雞公梁,額勒登保乘夜追之,黎明,賊復拒戰,安祿偕格布舍以左翼沖賊陣,賊竄城隍廟,右翼楊遇春伏起,前後夾擊,殲戮無算。又敗王登廷,追至西鄉魚渡壩。王登明與齊家營股匪合踞青岡嶺,安祿等三路競進,賊大潰,鮮大川、茍文明窺開縣,偕硃射鬥敗之於枯草坡,乘霧奪汪家山,餘賊數千奔下山溝,安祿率五六騎大呼馳擊,賊眾披靡,突林中數矛攢刺,遂歿於陣。事聞,優恤,賜內帑一千兩,謚壯毅,加予騎都尉世職。仁宗深惜之,詔以惠倫、安祿皆名將子,膺五等之封,為莠民所戕,國威大損,戒統兵大臣以滿洲、東三省兵自為一隊,及鋒而用,勿致疏虞。子恩特賀莫札拉芬,襲公爵,兼騎都尉。尋議又加騎都尉,並為三等輕車都尉,以安祿弟安成襲。

佛住,瓜爾佳氏,滿洲正白旗人,侍郎三泰子。三泰殉難葉爾羌,封三等伯,佛住襲爵,為散秩大臣、世管佐領,充阿克蘇領隊大臣,授成都副都統。嘉慶元年,充哈密辦事大臣,行抵西安,聞達州教匪起,自請偕英善往剿,允之。時賊撲東山廟,與豐城賊合,佛住與副都統勒禮善分路進攻,冒雪由山路破賊卡,扼東山隘口。賊自大東林潛渡河,率協領塔克慎、知縣劉清隔岸砲擊之。又偕英善、勒禮善擒徐天富,被優賚。二年正月,豐城賊傾巢出,游擊範楙、守備楊成陣亡,賊遂逼東鄉,別賊復自張家觀來犯,佛住率眾力戰,歿於陣。詔:「佛住已調哈密,自請回川剿賊。今在東鄉捐軀,其父三泰亦系陣亡,尤為可憫,從優議恤。應給世職,並為一等子爵,加一雲騎尉。」子瑞齡襲。

西津泰,和色里氏,滿洲鑲黃旗人。前鋒侍衛。從征臺灣,累戰皆捷,賜號法爾沙臺巴圖魯,圖像紫光閣,擢護軍參領。從征苗疆,克榔木陀山、大坪山、雷公灘、大烏草河,圍高多寨,復連破賊於大坡腦、得勝山,克垂藤、董羅諸寨,焚大小天星寨,進克馬鞍山,擢頭等侍衛,加副都統銜。從額勒登保克石隆賊巢,石柳鄧就殲,予優敘。嘉慶二年,赴四川,破王三槐於冉家埡、金瓘寺,從宜綿擊賊於花潭子,又克香爐坪賊巢,迭被優賚。進擊安子坪,賊退精忠寺,圍之,傾巢出犯,西津泰沖入賊陣,手刃十餘賊,身受重創,陣亡。予騎都尉兼雲騎尉世職。

豐伸布,唐古忒氏,蒙古鑲紅旗人,福州駐防。由馬甲累擢協領。從征臺灣,擢西安右翼副都統。嘉慶元年,率軍駐興安,防湖北教匪。二年,移防商、雒要隘。賊犯雙樹卡,又間道攻縣城,連卻之,賞花翎。進駐竹谿,遏賊入陜。賊掠近地,屢擊走。高天升大股踞石槽溝,率兵千自竹山進剿。關廟河,要隘也,冒雨進扼之,賊來爭,豐伸布先據山梁,賊分兩路猛撲,殺傷相當,而賊益坌集,短兵相接,豐伸布受創甚,至暮大雨,息軍山巔,以傷殞。優恤,謚壯勇,予騎都尉兼雲騎尉世職。六年,高天升就擒,傳首祭墓。無子,以侄阿克當阿襲職。

阿爾薩朗,賴奇忒氏,蒙古鑲白旗人。以副前鋒參領從征金川,迭克山寨堅碉,破扎古功尤著。戰達撒穀受創,特詔慰問。累擢正紅旗蒙古副都統,賜號阿爾杭阿巴圖魯。金川平,圖像紫光閣。歷喀什噶爾、伊犁領隊大臣,召回京,會甘肅石峰堡回叛,自請從剿,連破賊於雲霧山、田家山,進圍石峰堡,攻其西北,以火攻克之,斬虜特多,授護軍統領,調正藍旗滿洲副都統。嘉慶元年,率健銳、火器營從永保剿教匪,轉戰河南、湖北,屢破賊。二年五月,駐兵王家坪,營壘未定,賊自山溝出襲,阿爾薩朗力戰,猝中槍,歿於陣。賜恤,予騎都尉世職。及高天升傳首京師,命祭其墓。

烏什哈達,伊爾根覺羅氏,滿洲正黃旗人。以前鋒從征緬甸有功,賜號法福哩巴圖魯,授三等侍衛。從征金川,屢克堅碉,擢二等侍衛、正白旗蒙古副都統,予騎都尉加一雲騎尉世職。充和闐領隊大臣,坐與辦事大臣德鳳互訐,褫職,效力烏什邊卡。尋復起授頭等侍衛、虎槍營營長、健銳營翼長。從征臺灣,率水師擒賊渠莊大田於瑯嶠,復勇號、世職。授吉林副都統,調鑲紅旗蒙古副都統。從征廓爾喀,烏什哈達臨陣勇敢,論功輒最,三次圖像紫光閣。召對,自伐戰績,高宗惡之,褫職戍伊犁。嘉慶元年,赦歸,請赴湖北軍剿匪自效,偕副都統鄂輝敗賊襄陽,進戰鍾祥。二年,駐守宜城西岸,賊窺古河口,擊走之。移防四川石砫,攻白巖山,克賊卡。三年,王三槐由梁山、墊江竄渠口,與白巖山賊潛結,引之渡江。烏什哈達兵少不敵,力戰遇害。予輕車都尉世職,子圖爾弼善襲。

和興額,葛濟勒氏,滿洲鑲白旗人。以鳥槍護軍從征緬甸、金川、撒拉爾、石峰堡,賜號佛爾欽巴圖魯,累擢廣州右翼副都統。坐事降調,授頭等侍衛,充巴里坤領隊大臣,復授廣州左翼副都統。嘉慶二年,仲苗擾及廣西西隆,從總督吉慶赴剿,敗賊於戛雄。苗屯亞稿,設伏山徑,由深箐繞出夾擊,殲之。進攻那地,西隆肅清。圍巖場寨,連敗之紅水江、板蜯、板階,解冊亨圍。仲苗平,調甘肅涼州副都統。五年,赴陜西防剿。冉學勝等由辛峪竄出,和興額不能御,奪勇號、花翎,降為防禦,隨營效力。尋破賊沔縣乾溝河,授佐領。六年,樊人傑由黑河西竄,和興額扼之於五丁關,擒斬甚眾,擢協領。冉學勝屯大壩,偕總兵楊奎猷擊之,和興額先進,遇伏,歿於陣,依副都統例賜恤,予騎都尉兼雲騎尉世職,子福格襲。

論曰:教匪之役,首尾十年,國史忠義傳所載副參以下戰歿至四百餘員,其專閫提鎮及羽林宿衛階列一二品者,且二十餘人。王文雄、硃射鬥,一時名將;穆克登布、施縉,亦號驍勇;惠倫、安祿,並貴胄俊才。倉猝摧僕,三軍氣熸。當寧為之震惻,旌恤特示優異;餘雖功過相參,要皆竭忠行間,殞身不顧。嗚呼,烈已!當日巖疆悍寇,軍事艱難,蓋可見雲。


\end{pinyinscope}