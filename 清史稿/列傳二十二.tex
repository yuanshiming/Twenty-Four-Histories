\article{列傳二十二}

\begin{pinyinscope}
圖賴準塔伊爾德努山阿濟格尼堪佟圖賴

圖賴,費英東第七子也。初隸鑲黃旗,後與兄納蓋、弟蘇完顏改屬正黃旗。天聰元年,太宗伐明,略寧遠,二年,伐察哈爾,圖賴皆從。三年,復伐明,薄明都,明大同總兵滿桂入援,屯德勝門,圖賴與戰,所殺傷甚眾。師還,授世職備御。四年,從貝勒阿敏守永平,明兵救灤州,阿敏遣巴都禮赴援,圖賴及梅勒額真阿山皆在行。及阿敏棄永平出邊,明將率步卒百人追擊,圖賴以十六人殿,還戰,盡殲之,進世職游擊。

五年,上伐明,圍大凌河城,命巴牙喇纛章京楊善、鞏阿岱等駐軍壕外,待敵度壕即與戰,而令圖賴與南褚、哈克薩哈當兩旗間,衛樵採;城兵出挑戰,圖賴銳入陣,達爾哈以所部繼,貝勒多爾袞亦督兵進,我師薄壕,舍騎步戰,敵阻壕與城上兵爭發砲矢。師退,副將穆克譚、屯布祿,備禦多貝、戈裏等皆戰死,圖賴亦被創。上怒曰:「圖賴輕進,諸軍從之入,朕弟亦沖鋒而進,有不測,將磔爾等食之!敵如狐處穴,更將焉往?朕兵天所授,皇考所遺,欲善用之,勿使勞苦。穆克譚我舊臣,死非其地,豈不可惜?」因誡諸臣毋視圖賴創,揚古利、鞏阿岱偕往存問,上復切責之。明監軍道張春等以四萬人來援,次長山,上率諸貝勒御之,圖賴當右翼,躍馬突陣,敵潰走,遂覆其師。

七年,從攻旅順口。八年,從伐明,徇大同,攻朔州,拔靈丘,進世職二等。旋追論攻朔州時越界出略,又不赴期會地,奪俘獲入官。九年,授巴牙喇纛章京,從貝勒多鐸等伐明。多鐸既入廣寧,令圖賴與固山額真阿山等以四百人為前鋒向錦州,擊殺明將劉應選,破其軍。師還,以功得優賚。崇德二年,授議政大臣。三年,上命睿親王多爾袞、貝勒岳託率師分道伐明。圖賴從岳託為前驅,逾墻子嶺入邊,克十一臺,遂南略山東。明將以八千人拒戰,蒙古阿藍泰旁卻,圖賴方督所部馳擊,敵百騎突至,圖賴搏戰陷堅,敵敗去。明大學士劉宇亮綴我師而北至通州,圖賴與固山額真譚泰擊破之,拔四城,進三等梅勒章京。六年,從鄭親王濟爾哈朗等伐明,圍錦州。明總兵祖大壽為明守,蒙古吳巴什、諾木齊等謀內應,事洩,大壽以兵攻吳巴什等,圖賴入其郛,力戰,援諾木齊出。先後破杏山、松山援兵,遂督烏真超哈拔塔山、杏山二城,進一等梅勒章京。師還,追論攻錦州時巴牙喇兵有怯退者,圖賴當罰鍰,上命寬之。八年,從伐明,拔中後所、前屯衛,進三等昂邦章京。

順治元年,從睿親王多爾袞帥師伐明,明將吳三桂迎師。四月戊寅,師距山海關十里,李自成遣其將唐通率數百騎出關,是夕遇於一片石,圖賴督巴牙喇兵與戰,通敗走。己卯,入關,從大軍擊破自成。自成還京師西遁,圖賴復從諸軍追擊,敗之於慶都。二年,敘功,超授三等公。時圖賴方從定國大將軍豫親王多鐸西討自成,豫親王師自懷慶而南,圖賴至孟津,率精兵渡河,明守將黃士欣等皆走,降瀕河寨堡十五。

明總兵許定國等以所部來附,進薄潼關。自成將劉宗閔據山為陣拒我師。噶布什賢章京努山、鄂碩等率兵向敵,敵迎戰,圖賴率百四十騎直前掩殺,一以當百,俘馘過半。是歲正月,自成將劉方亮以千餘人出關覘我師,圖賴與阿濟格尼堪等令正黃、正紅、鑲白、鑲紅、鑲藍等五旗各牛錄出巴牙喇兵,率以擊敵,大敗之。自成聞敗,親率馬步兵拒戰,又徵鑲黃、正藍、正白三旗兵相助,賊連夕攻我壘,皆敗走,遂破潼關。

陜西既定,豫親王移師下江南。四月,至揚州,令圖賴與拜音圖、阿山等攻之,克其城,執明大學士史可法殺之。進攻明南京,復令圖賴與拜音圖、阿山率舟師列江西岸助攻。南京既下,從貝勒尼堪等逐明福王至蕪湖。福王登舟,將渡江,圖賴扼江斷渡,明將田雄、馬得功以福王降。師還,圖賴上書攝政睿親王,略言:「圖賴昔年事太宗,王之所知也。今圖賴事上,亦猶昔事太宗時。不避諸王貝勒嫌怨,見有異心,不為容默;大臣以下、牛錄章京以上,亦不為隱惡。圖賴誓於天,必盡忠事上。圖賴有過失,王若不言,恐不免於罪戾。王幸毋姑息,不我教誡也!」

初,圖賴在軍,固山額真譚泰方從英親王阿濟格西征,遣使告圖賴曰:「我軍道迂險,故後至。請留南京畀我軍取之。」圖賴以其語告豫親王,別作書遣塞爾特報索尼,將使索尼啟攝政王。塞爾特以書示牛錄希思翰,希思翰慮書達,譚泰且得罪,令沉諸河。圖賴至京師,系塞爾特索前書,塞爾特詭言已達索尼。事聞於攝政王。三年正月,下諸大臣審勘,將罪索尼。攝政王親鞫塞爾特,始自承沉書狀。攝政王坐午門議譚泰罪,三日猶未決。圖賴詰王,語甚厲,攝政王怒曰:「爾亦過妄矣!曩逐流賊至慶都,議分道進兵。因諸將爭先,爾誚讓肅、豫、英諸親王,不顧而唾。今又以語凌我。似此怒色疾聲,將逞威於誰乎?予與諸王非先帝子弟乎!」語畢,遂還邸。諸王因執圖賴將罪之,王復返曰:「圖賴雖聲色過厲,然非退有後言者。且為我矢勤效忠,無他咎也。」命解其縛。獄既定,侍衛阿里馬私誚圖賴庇索尼,圖賴以告攝政王,王令捕阿里馬及其二弟索泥岱、鎖寧。阿里馬故驍勇,與索泥岱拔刀力拒。皆殺之,而釋鎖寧。尋授本旗固山額真。

二月,以貝勒博洛為征南大將軍,圖賴副之,帥師徇浙江、福建。五月,論破流賊及定河南、江南功,進圖賴一等公。是月,師至杭州,明魯王駐紹興,其將方國安等屯錢塘江東岸,綿亙二百里,艤舟拒我軍。我軍舟未具,會潮落沙漲,圖賴率諸將士策馬自上流逕渡,江廣十餘里,人馬無溺者。國安望見,驚,棄戰艦走還紹興,將劫魯王以降,魯王走臺州,圖賴師從之,獲其將武景科等。進克金華,殺明督師大學士硃大典。七月,復進克衢州,殺明蜀王盛濃及明將吳凱、項鳴斯等。浙江平。八月,博洛令諸軍分道入福建,圖賴自衢州出仙霞關,擊破明大學士黃鳴駿等。師度嶺,克浦城,分遣署巴牙喇纛章京杜爾德、噶布什賢章京拜尹岱等攻克建寧、延平諸府。明唐王自延平走汀州,復遣巴牙喇纛章京阿濟格尼堪、杜爾德等帥師追擊,克其城,執唐王及其宗室諸王送福州。明將姜正希以二萬人夜襲汀州,已登陴,我軍出御,擊殺過半;別軍自廣信出分水關,克崇安。共撫定興化、漳州、泉州諸府。福建平。師還,至金華,圖賴卒於軍。子輝塞,襲爵。貝子屯齊等訐鄭親王濟爾哈朗,因及圖賴嘗謀立肅親王豪格,及上即位,復附和鄭親王,輝塞坐奪爵。八年,上親政,念圖賴舊功,命配享太廟,謚昭勛,立碑紀績,輝塞復襲爵。雍正九年三月,定封一等雄勇公。

準塔,滿洲正白旗人,扈爾漢第四子也。天聰間,授世職牛錄章京,官甲喇額真。嘗與鰲拜共率師略明錦州,復與勞薩共率師迎護察哈爾來降諸宰桑。崇德二年四月,從武英郡王阿濟格攻明皮島,敵守堅。阿濟格集諸將問策,準塔與鰲拜對曰:「我二人誓必克之!不克,不復見王。」遂先眾連舟渡海,舉火招諸軍,敵倚堡為陣以拒,與鰲拜犯矢石力戰,卒取其島。論功,進世職三等梅勒章京,襲十二次,賜號「巴圖魯」,敕增紀其績。

三年八月,授蒙古固山額真。九月,從揚武大將軍貝勒岳託等伐明,攻密雲墻子嶺,準塔先據嶺,導諸軍毀邊墻以入,擊敗明太監馮永盛、總兵侯世祿等;又與武賴敗三屯營援兵,復進戰於董家口,破敵,行略地,克城二。師還,進世職二等梅勒章京。六年二月,從睿親王多爾袞攻錦州,以阿王指,遣士卒歸,又離城遠駐,議罪,當奪官籍沒,上命罰鍰以贖。八月,上自將攻錦州,九月,還盛京,命準塔從貝勒杜度等為長圍困之。七年三月,錦州既下,上命貝勒阿巴泰率師留戍。旋令準塔與固山額真葉臣等番代。

先是圍錦州時,城兵出犯鑲黃旗汛地,巴牙喇兵退入壕內,王貝勒等袒不舉,準塔坐阿附,議罪當罰鍰,上命貸之。十月,從阿巴泰、圖爾格帥師伐明,略山東,與葉克書等分兵攻孟家臺,不克,士卒有死者,準塔又妄稱嘗陷陣。師還,議罪,奪巴圖魯號,降世職一等甲喇章京,仍罰鍰。十二月,復命鎮錦州。

順治元年,從睿親王多爾袞入關擊李自成,遂至慶都,大破之;又與譚泰等率噶布什賢兵逐至真定,又破之。自成焚輜重,倉皇西走,於是京師以北、居庸關內外諸城堡,及畿南諸州縣悉定。論功,復三等梅勒章京。

二年正月,以饒餘郡王阿巴泰為帥,準塔將左翼,譚布將右翼,帥師徇山東。二月,聞明福王遣兵渡河,阿巴泰令準塔等迎戰。明兵方攻沛縣李家樓,馬步二千餘屯徐州,距城十五里,準塔師破其壘,斬其將六,明兵赴河死者無算,遂克徐州。五月,復自徐州南下,明總兵劉澤清遣其將高祐以舟師攻宿遷,擊破之,進次清河縣。黃河自西來,至縣境,淮水及清河皆入焉。澤清遣其將馬化豹、張思義等將兵四萬、舟千餘,據三水交匯處,連營十里。準塔遣梅勒章京康喀賴,游擊範炳、吉天相等率兵渡清河,結營相拒,發砲擊敵舟;復遣都司楚進功將步兵六百人屯黃河北岸,鳴砲相應;又分其兵為二:一出清河上游,一隔水,擊破明馬步軍;兵復合,逐入淮安界,斬其將三。師次清江浦,澤清引去,明將吏栢永馥、範鳴珂出降,遂克淮安。

明新昌王入海據雲臺山,糾眾陷興化,準塔遣將擊斬之,通州、如皋、泰興諸城皆下;鳳陽、廬州亦降。凡降明將吏二百十三,得舟五百餘、馬九百餘、橐駝二十五、砲一百二十。捷聞,進準塔三等昂邦章京,復巴圖魯號,命以固山額真鎮守廬、鳳、淮陽諸處。準塔帥師巡行諸州縣,安撫居民,設置官吏。江、淮間悉定。澤清尋亦以所部降。

三年正月,從肅親王豪格帥師徇陜西。時叛將賀珍據漢中,武大定、石國璽等分屯徽、階諸州,遙與相應。豪格師自西安向漢中,珍走西鄉。七月,令準塔與貝子滿達海等攻大定、國璽等,大定、國璽等以其眾七百人降。十一月,豪格擊張獻忠於西充,準塔指揮諸軍合戰,俘馘甚眾。四年八月,復與貝勒尼堪、貝子滿達海等分兵下遵義、夔州、茂州、榮昌、富順、內江、資陽諸郡縣。四川平,師還。尋卒。論功,進世職一等精奇尼哈番。十二年,追謚襄毅,立碑紀績。

準塔無子,弟阿拉密襲。遇恩詔,進三等伯。康熙中,準塔兄子舒書降襲一等精奇尼哈番。乾隆初,定封一等子。

伊爾德,滿洲正黃旗人,揚古利族侄也。天聰三年,從揚古利率師入明邊,略錦州、寧遠。既,復從攻北京。師還,敗山海關援兵於灤州,出塞為前驅,斬明兵守隘者。五年,從上圍大凌河城,城兵突出,伊爾德沖鋒殺敵,逐敵迫壕,乃引還。敵騎挾弓矢將犯御營,伊爾德馳斬之。秋,復略前屯衛,將十五人,捕敵軍邏卒。值別將噶思哈為敵困,奮擊,援之出。積功,授世職備御。尋擢巴牙喇纛章京。

崇德二年,從貝勒阿巴泰築都爾弼城,將巴牙喇兵四百人護工役。五年,從圍錦州,敵出戰,伊爾德領纛追擊,敗之。督屯田錦州、松山間,明人縱牧於野,伊爾德設伏烏欣河,驅其牲畜以歸。敵襲我軍後,伊爾德還擊,斬獲無算。超進世職三等梅勒章京。屢坐事當削世職,命罰鍰以贖。七年,復從圍錦州,明兵來奪砲,擊敗之,進一等。

順治元年,命駐防錦州。二年,加半個前程。世祖召伊爾德,命從豫親王多鐸南征,與尚書宗室韓岱等將蒙古兵自南陽下歸德,招撫甚眾。至揚州,獲戰監百餘,渡江先驅,破南京。明福王由崧走蕪湖,與固山額真阿哈尼堪等追擊,敗明將黃得功。三年,進世職一等昂邦章京。六年,偕大將軍譚泰討叛將金聲桓,下南昌,誅聲桓。師進,叛將李成棟陷信豐,攻克之,成棟夜遁,馬蹶,溺水死。分兵定撫州、建昌,破其將楊奇盛。江西悉平。師還,復移剿保定土寇。論功,進世職一等精奇尼哈番。八年,巴牙喇纛章京鼇拜訐伊爾德值上幸內苑擅令門直員役更番,私減守門護軍額數,又嫉忌鰲拜等,鞫實,論死,上貸之,命降世職一級,罰鍰以贖。尋授本旗固山額真。九年,三遇恩詔,累進一等伯兼拖沙喇哈番。從敬謹親王尼堪征湖南,師敗績,王沒於陣。十一年,師還,論罪,奪職籍沒。

初,明魯王以海與其將阮進等據舟山,以海走入海。至是,其將陳六禦、阮思等復據舟山為寇。十二年,上授伊爾德寧海大將軍,率師討之。六禦等遣所置總兵王長樹、毛光祚、沈爾序等登陸掠大嵐山。伊爾德遣巴牙喇纛額真車爾布、梅勒額真碩祿古、總兵張承恩引兵趨夏關,抵斗門,連擊敗之,斬長樹等;而自率師攻寧波,乘舟趨定海,分三道並進。六禦等列舟望江口山下以待,伊爾德揮眾進擊,敗之;追至衡水洋,斬六禦等,遂取舟山。十四年,師還,上命貝勒杜蘭等郊勞,復世職,論功,進一等伯。

十五年,從信郡王多尼南征,自貴陽至盤江,擊斬明將,進克雲南。十八年,卒於軍,謚襄敏。

孫巴琿岱,襲。自散秩大臣遷正黃旗滿洲都統。夏逢龍之亂,出為荊州將軍。聖祖征噶爾丹,參贊大將軍馬斯喀軍務。卒,謚恪恭。子馬哈達,降二等伯,世襲。乾隆中,加封號宣義。

努山,扎庫塔氏,世居鄂里。父塔克都,歸太祖,太祖命籍其眾為牛錄,以其長子瑚什屯為牛錄額真。旗制定,隸滿洲正黃旗。積功,授世職游擊。卒,無子,以努山子渾岱為後,襲職,而努山代為牛錄額真。從征伐,輒先驅覘敵。有功,授噶布什賢章京。太宗嘉其能,以瑚什屯世職改命努山,諭曰:「弟之子不若弟親也。」時為天聰八年五月。

尋從伐明,攻大同,努山與甲喇額真席特庫、納海執邏卒以獻。崇德元年,率甲士行邊,至冷口,遇明邏卒十四,斬三人,俘一人,獲馬十餘。三年,從貝勒岳託伐明,將入邊,遇明兵,斬四十人,俘三人。發明兵所置火藥。度墻子嶺,明總兵吳阿衡將六千人迎戰,擊之敗。與噶布什賢噶喇依昂邦勞薩逐明兵,獲馬數十及攻具。薄明都,破明兵為伏者,而自設伏道側,挑明太監高起潛戰,伏起夾擊,多所俘馘。即夕,起潛襲噶布什賢兵,努山與席特庫及甲喇額真鄂克合兵戰,起潛兵敗走,逐北,迫會通河,明兵多入水死,遂次涿州;分道從睿親王多爾袞徇山東,克濟南。師還,出塞,復與勞薩共敗明兵。

六年七月,與侍衛穆章等詗敵董家口、喜峰口,遇明兵,斬百餘人,俘四人。從圍錦州,是時上自將駐軍松山、杏山道中,明兵擊噶布什賢兵,努山力戰,斬五十二人,獲馬三十。明總督洪承疇出戰,努山與勞薩等陣而前,戰良久,王貝勒等各以所部合戰,大破明兵。十月,擢噶布什賢噶喇依昂邦。是時武英郡王阿濟格駐軍杏山河岸,上命努山濟師。敵騎千自寧遠至,猝與努山值,驚潰,努山逐之,至連山,斬三十人,獲馬三十有二。七年三月,與噶布什賢噶喇依昂邦吳拜共略寧遠,敵騎五十自中後所至,率噶布什賢兵縱擊,明兵四百人來援,並擊敗之;薄寧遠,守者背城陣,努山等與戰,俘二十三人:進世職二等甲喇章京。八年,上以貝勒阿巴泰等略山東未還,命努山率甲喇額真四、侍衛四、兵九十至界嶺口,詗師行距邊遠近,遇明兵,斬守備一、兵三百餘,俘數十人,獲馬騾二百餘。八月,與巴牙喇纛章京阿濟格尼堪帥師戍錦州。

順治元年,世祖既定鼎,命努山將左翼噶布什賢兵從豫親王多鐸西討李自成,自成兵出潼關拒戰,努山自間道劘其壘,斬殺過半,自成兵潰走。二年,移師定河南,下揚州,克明南京。明福王由崧走蕪湖,努山與諸將以師從之,得福王以歸。三年,從貝勒博洛徇浙江,明總兵方國安屯錢塘江東岸,以舟師出戰。努山從固山額真圖賴自上游渡,擊國安,敗之,盡得其舟;進略福建,擊斬明巡撫楊廷清、李暄。時巴牙喇纛章京都爾德等攻下建寧、延平諸府,明唐王聿鍵走汀州,努山馳七晝夜追及之,唐王入城守,令銳卒以巨木撞其門,後軍繼至,遂克之。

五年,從鄭親王濟爾哈朗定湖廣。明桂王由榔據廣西,其總督何騰蛟,總兵王進才、馬進忠、袁宗第等,分屯湖南諸郡邑。六年正月,努山至長沙,時席特庫亦遷噶布什賢噶喇依昂邦,將右翼噶布什賢兵,共簡精銳攻湘潭,與固山額真阿濟格尼堪等破北門入,騰蛟死之。四月,兵部尚書阿哈尼堪等徇寶慶,未至七十里,進才、進忠合軍出御,努山令所部舍騎步戰,明兵敗,薄城東門,進才等棄城走,逐之至武岡,殲進忠所將步兵三千,破進才及宗第等寨十餘,分克沅州、靖州;再進克全州,斬明閣部楊鰲及副將以下四十餘。累進二等阿思哈尼哈番。十三年,擢內大臣。十五年,卒。

阿濟格尼堪,滿洲正白旗人,達音布子。達音布戰死,長子阿哈尼堪襲三等甲喇章京,旋卒。阿濟格尼堪繼襲,授甲喇額真。從太宗伐察哈爾,自大同入明境,與雅賴共擊敗明兵於崞縣。崇德元年,從太宗伐朝鮮,擊敗明寧遠守邊兵。三年,從貝勒岳託伐明,擊破總兵侯世祿,得其印及騎。四年,擢巴牙喇纛章京。從肅親王豪格攻錦州,設伏於連山,俘五人,獲馬七。

六年,從鄭親王濟爾哈朗攻錦州,以七十人為伏,敗敵;進攻杏山,領纛直入敵壘,敵大潰。時錦州有蒙古諾木齊等原降,明總兵祖大壽發其謀,以兵圍之,不得出。阿濟格尼堪詗知之,乘夜薄城,力戰先登,入其郛,援諾木齊等皆出。進攻松山,戰屢捷。上以阿濟格尼堪少年能殺敵,進一等參將,賚白金四百。是年八月,明總督洪承疇集諸鎮兵救錦州,上自將屯松山、杏山道中,絕餉道。明總兵吳三桂、唐通等皆潛引去。上召阿濟格尼堪親授策,與鰲拜等追擊,大敗之。八年八月,命戍錦州。九月,鄭親王取中後所、前屯衛,阿濟格尼堪率所部及蒙古兵攻中前所,拔其城,俘明潰兵,無得脫者,加半個前程。

順治元年四月,從睿親王多爾袞入關,破李自成,追至慶都,進一等梅勒章京。十月,從豫親王多鐸帥師西討自成,渡孟津,薄潼關。賊鑿重壕為固,自成將劉方亮率千餘人出拒,阿濟格尼堪與圖賴、阿爾津等奮戰,方亮敗退。至夜,復來犯,阿濟格尼堪力戰卻之,連破賊二壘,遂麾兵逾壕,冒矢石先登,賊驚潰降竄,師入關。二年正月,克西安,自成自商州入湖廣。

豫親王移師下江南,四月,至淮安,遣阿濟格尼堪率所部趨揚州,屯城北,與親軍合攻,城遂下,獲戰艦二百餘;渡江克明南都,追擊明福王由崧於蕪湖,敗其舟師:進三等昂邦章京。三年,從端重親王博洛定浙江,徇金華、衢州,破仙霞關,略建寧、延平。明唐王聿鍵走汀州,阿濟格尼堪與都爾德進擊至城下,率精銳先登,遂克汀州。其總兵姜正希以二萬人赴援,阿濟格尼堪出御,所殺傷過半。進一等精奇尼哈番,賜敕世襲。五年,授正白旗滿洲都統。

六年,鄭親王濟爾哈朗征湖廣,以阿濟格尼堪參贊軍事。是時明總督何騰蛟,總兵王進才、馬進忠等,守湖南:騰蛟軍湘潭;進才、進忠軍寶慶。阿濟格尼堪至長沙,與兵部尚書阿哈尼堪為前鋒,攻湘潭,破北門入,執騰蛟。逐明潰兵至湘鄉,盡殲之,遂趨寶慶。未至七十里,進才、進忠合軍拒戰,阿濟格尼堪令步騎番進,薄寶慶東郭,進才等敗遁。遂下沅、靖,進克全州。七年正月,師還,進三等伯,賚白金五百,授議政大臣。四月,卒,謚勇敏。乾隆間,加封號襄寧。子宜理布,自有傳。

佟圖賴,漢軍鑲黃旗人。父養真。太祖克撫順,養真以從弟養性已降,挈其族來歸。從攻遼陽,以功授世職游擊。命駐鎮江,守將陳良策以城叛,養真及長子豐年皆死。

佟圖賴初名盛年,其次子也,襲世職,事太宗。天聰五年,從攻大凌河,破明監軍道張春兵,進世職二等參將。崇德三年,授兵部右參政。五年,從攻錦州,取白官兒屯臺。六年,復從攻錦州,取金塔口三臺。七年,從攻松山,明師以騎兵突陣,將奪我師砲,佟圖賴擊卻之;又敗其步兵,取塔山、杏山諸臺,遂克其城二:以功進世職一等。是歲始分漢軍為八旗,授正藍旗固山額真。師出略明邊,佟圖賴與固山額真李國翰等奏請直取燕京,上以「未取關外四城,何能即克山海」,優旨開諭之。八年,從鄭親王濟爾哈朗收前屯衛、中後所二城,加半個前程。

順治元年,從入關,調鑲白旗,與固山額真巴哈納、石廷柱等招降山東府四、州七、縣三十二。復移師下太原,招降山西府九、州二十七、縣一百四十一。師還,賜白金四百。尋從豫親王多鐸西討李自成,定河南。二年,移師徇江南,先後克揚州、嘉興,皆在行,進世職二等梅勒章京,賜蟒服、黃金三十、白金千五百。五年,授定南將軍,與固山額真劉之源率左翼漢軍駐寶慶。時馬進忠等寇衡、湘、辰、永間,陷寶慶。佟圖賴師至,克之。

六年,鄭親王濟爾哈朗徇湖廣,佟圖賴與固山額真碩詹等分兵趨衡州,陣斬明將陶養用,拔其城。時明將胡一清猶屯城南為七營,乘勝疾擊破之;逐一清,戰於望公嶺山峪口,又破之;一清走入廣西境,距全州三十里,立六營自保,與努山、阿濟格尼堪合軍奮擊,破之,遂下全州。師還,駐衡州。明兵犯常寧,遣牛錄額真陳天謨等馳援,破明兵石鼓洞,斬其渠。八年,師還,宴勞。授禮部侍郎。復調正藍旗固山額真。世職累進至三等精奇尼哈番。十三年,以疾乞休,世祖命加太子太保致仕。十五年,卒,賜祭葬,贈少保,仍兼太子太保,謚勤襄。

康熙間,以孝康章皇后推恩所生,贈一等公,並命改隸滿洲。世宗即位,追封佟養正一等公,謚忠烈,與佟圖賴並加太師。養真改曰養正,避世宗嫌名也。

論曰:圖賴忠鯁類父,督師南征,破福、唐二王,三江、閩、浙,以次底定,仍世侑饗,允哉!準塔綏徠畿輔,戡定江、淮;伊爾德橫海殺敵,破魯王餘眾,功與相並。努山、阿濟格尼堪、佟圖賴佐定江表,又合軍徇湘南。戮力佐創業,績亦偉矣!


\end{pinyinscope}