\article{志四十三}

\begin{pinyinscope}
地理十五

△湖南

湖南:禹貢荊州之域。明屬湖廣布政使司,置偏沅巡撫。清初因之。康熙三年,析置湖南布政使司,為湖南省,移偏沅巡撫駐長沙。雍正二年,改偏沅巡撫為湖南巡撫,並歸湖廣總督兼轄。七年,置永順府,升岳州之澧州。十年,升衡州之桂陽州。乾隆元年,升辰州之沅州為府。嘉慶二年,升辰州之乾州、鳳凰、永綏三。二十二年,置晃州。光緒十八年,置南州。領府九,直隸五,直隸州四,屬州三,縣六十四。東至江西義寧;三百五十里。西至貴州銅仁;一千七百三十五里。南至廣東連州;八百二十里。北至湖北監利。三百五十里。廣一千四百二十里,袤一千一百五十里。北極高二十四度四十九分至二十九度三十七分。京師偏西二度十四分至七度四十三分。宣統三年,編戶四百二十八萬八千一百六十四,口二千二百五萬二千一百五十九。其名山:衡岳、九疑、都龐、騎田、萌渚、幕阜。其巨川:湘、沅、資、澧。其澤:洞庭。驛道:自長沙北達湖北蒲圻;東南出插嶺關達江西萍鄉;南達廣西全州;西達貴州玉屏。鐵路幹:粵漢中段。支:萍株。航路:自長沙南達湘潭,北達漢口。電線:自長沙北達漢口,南通桂林,西通洪江,東通江西萍鄉、安源。

長沙府:沖,繁,難。巡撫治;布政、提學、提法三司,巡警、勸業、鹽法、長寶四道同治。府隸之。明隸湖廣布政使司。康熙中,偏沅巡撫自沅州徙駐,為省治。雍正二年改湖南巡撫。東北至京師三千五百八十五里。廣一千里,袤五百九十里。北極高二十八度十三分。京師偏西三度四十分。領州一,縣十一。長沙沖,繁,難。倚。東:天井。西:谷山。北:羅洋、石寶、麻潭、智度、銅山。巨川則大江,洞庭湖匯湘、沅、資、澧入焉。湘江自湘潭、善化入,納潦滸河及白沙河。又西北,右合下泥港,左桐樹港,納八曲河,逕銅官山,至靖港,為古新康江口。又西北,會喬口河,入湘陰。瀏陽河在縣南,源出大圍山,西北流,經縣界入湘。陶關在縣西南。有喬頭鎮巡司。喬頭、長沙二驛。長株鐵路。善化沖,繁,難。倚。南:昭山。西:嶽麓。西北:金盤。東南:錫山,湘水在西,自湘潭入,西北流,左納觀音港,至瓦官口,靳江水從西南來注之。北過水陸洲,入於長沙。東:瀏渭水自瀏陽入,北合金塘港,至長沙入湘水。又西,卯江水,一名滿官江,源出寧鄉嵇架山,東北與螺陂河水合,入長沙,是為八曲河。南有暮雲市廢巡司。有驛。長株鐵路。湘潭沖,繁,難。府西南一百里。西:烏臺。東:石潭。南:曉霞。東北:昭山,其下有昭潭。西北:韶山。西南:隱山。東南:鳳凰山。湘水自衡山入,東南流,過晚洲,屈而北,硃亭港水注之。又東北,過淦田市,東與醴陵縣界。又北,過空冷峽,又東北至於鑿石浦。屈而西,涓水自西南來會。水逕縣西易俗鄉,又名曰易俗河。又北至湘河口,左合漣水。又東北過縣治南,又西北為峨洲,入於善化。其西靳江水自寧鄉入,迤東至善化入湘水。二鎮:硃亭,縣丞駐;下灄,舊有巡司,廢。又永寧巡司亦廢。黃茅巡司,乾隆二十六年置,後遷縣東株洲市,更名。有南岸驛。有商埠,光緒三十一年奏開。有長株、株萍鐵路。湘陰沖,繁,難。府北一百二十里。北:黃陵。東:神鼎。東南:玉池。東北:汨羅山、玉笥山。西北:錫山。湘水在西,自長沙入,北合門涇江,又北流,西別出為濠河水,西北與資水分流,其合處曰臨資口。其正渠又北至縣治西南,白水江注之。又北過蘆林潭,錫江水合濠河水自西來會。又北合汨水,西與湄水合。又西北會羅水,為汨羅江,西北流,歧為二,至屈潭復合。西北過屈羅戍南,分流注湘水。湘水西北至磊石山,入於洞庭湖。鎮三:營田,蕭婆、大荊。縣丞治林子口。西北有營田巡司,後廢。新市、大荊鎮二巡司。湘陰、歸義二驛。寧鄉沖。府西北一百里。南:石鼓。北:香林。東:天馬。東南:嵇山。西:大溈山,溈水出,東南流,右納黃絹水,左瑕溪,至雙江口,流沙河水自西南來注之。又東北,左合玉堂江水,右烏江水,又東北至縣治南,屈而東,會平江水,又東北入於長沙,為新康江。又有靳江水在縣南,源出湘鄉,迤東至湘潭入湘。有唐市鎮。有驛。瀏陽繁,疲,難。府東一百五十里。西:洞陽山。北:道吾。東北:大光山。又大圍山,瀏水出,西南至雙江口,小溪水自東來會。又過縣治西南,瀏渭水北流入焉。又西與小河水合,西北入於長沙。北:石柱峰,潦滸河出,西南流,屈西北至長沙為澇塘水。又南川水即澄潭江,自江西萬載入,西南過江口入于醴陵,其下流是為淥水也。永興、居仁二鎮。梅子園一巡司,澄潭江,後遷縣西永安市。醴陵簡。府東南一百八十里。北:小溈山。東:王喬。東南:大屏。西南:君子山。湘水瀆自湘潭入。又南淥江水,有二源:北源曰南川水,自瀏陽入,西南至雙江口,萍水自南來會;水出江西萍鄉縣,是為南源。又西過縣治南,右納姜灣水,又西與鐵江水合。水一名北江,自攸縣入,北合清水江,又北流為泗汾河。水又北入淥水,至淥口入于湘。有插嶺關。淥口鎮巡司及驛,與醴陵驛為二。株萍鐵路。益陽沖,難。府西北二百里。北:五溪山。南:小廬。西:修山。西北:紫雲。西南:浮丘山。益陽江在南,一名茱萸江,即資水,自安化入,東合泥溪、沾溪、桃花江、志溪諸水,過縣治南,別出為蘭溪水。合喬江,東北流,北別出為甘溪,入沅江。東有喬江水,首受資水,自沅江入。西:西林港,歧為二,一東北入湘陰,一東南入長沙,皆合湘水。北有益水,出五溪山,東與甘溪水合,至沅江入資。有瓦湖鎮。有驛。湘鄉沖,疲,難。府西南二百一十里。北:仙女山。東:東臺。西:石佛。西北:靈羊。西南有大禹山。漣水一名湘鄉河,自邵陽入,北合金竹水,又北與藍田水會。東北流,左納西陽水,屈東南至大江口,側水合崖源水自西南來注。又東北過石魚山東,青陂水南流合焉,東北至湘潭入湘水。虞塘、定勝二鎮。縣丞治永豐市。婁底巡司。明置武障,乾隆三年徙改。攸繁,疲,難。府東南二百八十里。東:司空。東北:羅浮。西北:明月山。攸水在東,源出江西萍鄉縣西,合陽升江水,西南至縣治東南入洣水。水自茶陵入,亦名曰茶陵江也。洣水又西與陰山江合,入衡山。東北有鳳嶺巡司,雍正十一年置。安化簡。府西二百六十里。東:移風。南:浮青。北:大峰。西北:小辰。西南:大熊山。山與新化接界。資水在西,一名邵河,自新化入。西北合渠水,屈東北流,過縣治北,屈而東,敷溪水自南來注。又東納善溪水,入於益陽。東南有藍田水,亦自新化入,東北至湘鄉入漣水。又歸溪水,源出縣西司徒嶺,西南流,與湄江合。屈東南,至湘鄉合藍田水。茶陵州繁,難。府東南四百八十里。西:雲陽。東:皇雩。東北:景陽山,即茶山。洣水自酃入,亦曰茶陵江,西北流,右納洮水,北過州治東,茶水自東北來注,又西北入攸。有視渡口巡司,治州南視渡關,後遷高岡南關。

寶慶府:難。隸長寶道。舊隸湖廣布政使司,康熙三年來屬。東北距省治五百里。廣六百六十里,袤六百三十里。北極高二十七度四分。京師偏西五度六分。領州一,縣四。邵陽繁,難。倚。南:四望。東:大雲。東北有龍山。西北:首望。資水自武岡入,東納辰溪水,東北過府治北。邵水出龍山,南合桐江、檀江,屈西北流注之,北與漁溪合。西北會高平水,入新化。又漣水亦出龍山,東北入湘鄉。又烝水源出邪姜山,合大雲水,至衡陽入湘。又西洋江出西北隆回鄉,南至武岡,流合洞口水。有隆回巡司。其黑田鋪巡司,乾隆二十五年置,後廢。通判駐桃花坪。新化繁,難。府西北一百八十里。北:大熊山。東北有黃柏界山,皆與安化接界。南:梅山、長龍。西南:文仙。西北:清虛,一名大西山。資水在東,自邵陽入,西北過縣治北,雲溪水合洋溪自西南來注之。又北與油溪水合,入安化。西有渠江水,源出冷溪山,北至安化入資水。東有藍田水,上源曰墨溪,出邵陽,亦入安化。高平水出西南首望山,東南流,入邵陽,注資水。西北有蘇溪鎮巡司,乾隆四十年廢。城步難。府西南四百二十里。乾隆三年改隸靖州,七年復。東:羅漢山。東南:金紫山,與廣西全州接界。西南:金童,又有藍山。西北:楓門山。東北:青角山,即古路山,資水所出,一名都梁水,又名濟水。北流屈東,左會款溪水,入武岡。又有巫水,源出東北巫山,南屈而西為漁渡江,縣東南諸水皆入焉。至縣治西南,左納界背水,西北與清溪水合,入綏寧。西南有長平水,又曰藍山水,亦入綏寧,為臨川水。又有長灘水,出縣南,南至廣西龍勝,曰貝子溪,其下流是為潯江。同官水亦南入龍勝,為太平溪,流合貝子溪。有橫嶺峒巡司,本寨頭司,乾隆元年置,後遷橫嶺更名。江頭汛巡司治莫宜峒,乾隆六十年置長安營,同知駐。轄瑤峒五:曰蓬峒、牛欄、莫宜、扶城、橫嶺。為寨四十有八。武岡州繁,疲,難。府西南二百八十里。西北有武岡山,州以是名。又西北,天尊山,山與綏寧接界。南:雲山。東南:寶方,又名資勝山。資水在南,自城步入,東合威溪。又東過州治南,左合渠水,右納石門水,又東北流,蓼溪水自西來會。蓼溪一曰高沙市水,出綏寧。又北合洞口水,水上源曰平溪,出黔陽。東南流,右納岳溪水,東合西洋江,至平溪口入資水。資水又北,屈而東,逕紫陽山,曰紫陽河,龍江水北流入焉。又東南與夫夷水會,入邵陽。西北:𦰡溪自綏寧入,至黔陽入沅水。州同駐高沙市。峽口、石門司二巡司。紫陽、蓼溪二廢司。新寧繁,難。府西南三百里。西:花溪。南:金城。西南:峎山。東南:大雲。東北:高桂。又有紫雲山,山與武岡、東安接界。夫夷水在南,一名羅江水,其上源曰西延水,自廣西全州入,東北流,左納深沖水,又北至縣治西南,新寨水自西來注之。屈東過筆架山,合水頭水,又東北納涷江水,合小溪水,入武岡,為資水別源。東有靖位鎮巡司,康熙二十三年廢。

岳州府:沖,繁,疲,難。隸岳常澧道。舊隸湖廣布政使司。康熙三年來屬。初沿明制,領州一,縣七。雍正七年,澧升直隸州,石門、安鄉、慈利割隸。西南距省治三百里。廣三百八十里,袤三百四十里。北極高二十九度二十四分。京師偏西三度三十四分。領縣四。同知一。道光元年移治君山,後廢。有岳州商埠,光緒二十四年奏開。巴陵沖,繁,疲,難。倚。城內巴丘山。東:大雲、銅鼓,皆與臨湘縣接界。東南:靈屋、五龍。大江在西北,洞庭湖在西南。君山、扁山、石城山皆在湖中。湖周八百餘里,南連青草,西接赤沙,謂之三湖,湘、沅、資、澧諸水咸匯焉。東北至三江口,合大江,古謂之五渚。大江又東北入臨湘。有城陵磯,天險也。南有新墻河,即微水,自臨湘入,西南流,左納沙港,迤西至灌口,入洞庭湖。水出東南清水嶺,西南至湘陰合汨水。水邕湖在東南,一名翁湖,又東為角子湖。楊林街,縣丞。鹿角鎮,主簿。東:岳陽驛。舊有青岡驛,順治十六年置,有丞,乾隆十六年裁。臨湘沖,繁。府東北九十里。東:黃皋。西南,微落。東南:大雲。又龍窖山,跨湖北通城、蒲圻諸縣,微水所出,迤西逕土城,左納馬港,西南入巴陵。大江在縣西,自巴陵入,東北過彭城山,松陽湖水自東南來注之。又東北與白泥湖水合,過鴨欄磯,入湖北嘉魚。黃蓋湖在東北,縣東諸水皆匯焉,北注清江口,入大江。東南有桃林、長安巡司,城陵磯,乾隆二十六年徙長安鎮,更名,尋復故。雲溪、長安二驛。鴨欄磯、長安二鎮。華容疲,難。府西北一百八十里。北:黃湖。東:石門、墨山。東北:東山。東南:鼓樓山。大江右瀆自湖北監利入,東屈而南,入巴陵。北華容河,西湧水,皆首受大江水,自湖北石首入,東南流入洞庭湖。澧水在縣南,自安鄉入,合赤沙湖,亦注洞庭。東北大荊湖及團湖,合流入大江。黃家穴司巡司。黃家、鼓樓二鎮。平江疲,難。府東南二百四十里。北:永寧。西:湖源。東:道巖。東南:連雲。東北:幕阜山,一名天岳山,下有天岳關。又有汨水,自江西義寧入,西南流,右納紅橋水,左納白鉛諸水,又西南至白湖口,屈而北,鍾洞水南流西屈注之。又西合盧水,又西北與暹江水合,屈西南過縣治南,左納晉坑水。又西北至將軍山,昌山自東北來會,迤西入于湘陰。東有長壽巡司。

常德府:沖,繁,難。隸岳常澧道。舊隸湖廣布政使司,康熙三年來屬。東南距省治四百十五里。廣四百二十里,袤六百二十里。北極高二十九度一分。京師偏西五度十分。領縣四。商埠,光緒三十二年奏開。武陵沖,繁,疲,難。倚。西有平山,即武陵山,亦名河洑。北有陽山。東北:藥山。東南:善德山。沅水自桃源入,南逕河洑洲,屈而東,至府治東南,歧為馬家吉河水。又東南流,枉水自西南來注之。又東過牛鼻灘,北別出為小河水,東北匯連山湖,合漸水。沅水又東南入龍陽。漸水在北,一名澹水,源出安福,南流屈東,右納馬家吉河水,又東北至馬家洲,歧為二,一東與小河水合,一東北合麻河水,至沙夾入沅水。東北有沖天湖、直山湖、官塘湖,皆合漸水。縣丞一,治牛鼻灘。北有大龍巡司,乾隆四十一年置,後廢。有麻河驛。桃源沖,繁,難。府西南八十里。北:纛旗。南:綠蘿。西南:桃源。沅水在南,自沅陵入,東過高都鎮,左納大洑溪,又東北與小洑溪合。屈而東,夷望溪東流北屈注之。又東合水溪,過縣治東南,屈而北,延溪水自西來入。又東北與白洋河合。河出慈利,曰龍潭河,東南入縣境,合蘭溪、湯溪,又南流為漆家河,入沅水,又東南入武陵。新店、鄭家店二巡司。又高都、鄭家店二廢巡司。新店、鄭家店、桃源三驛。蘇溪、麻溪、高都三鎮。龍陽沖。府東南八十里。南:橫山,一名龍陽山,縣以此名。北:寶臺。東南:軍山。東北:洞庭湖。沅水自武陵入,東過小河口,屈而南,滄浪水自西南來注之。東過縣治東北,南別出為支港,東通後江湖,至沅江縣合資水。其正渠又東北至鼎港口,小河水分流來注之。又東北流為西河,漸水合小河水自西來會,入於洞庭。其入湖處謂之西河口也。東南有龍潭橋巡司。龍陽驛。小江、鼎口二鎮。沅江簡。府東南二百七十一里。西南:煙波山,西北:赤山,東北:明山,並濱洞庭湖。湖水西自龍陽受資水。資水自益陽入,迤東至毛角子口,南別出為喬江水,至湘陰會湘水。其正渠北屈而西,又西北過縣治東,白泥湖水首受益水,自西南來注之。又北至小河口,歧為二,一東北流,至益陽江口入洞庭,一西北與沅水合,匯於洞庭湖。

澧州直隸州:沖,繁,難。岳常澧道駐。舊為嶽屬州,雍正七年升,割石門、安鄉、慈利來隸,並置安福。十三年增永定。東南距省治六百有五里。廣四百三十五里,袤二百有五里。北極高二十九度三十七分。京師偏西四度四十四分。領縣五。西北:天供、大清。東南:關山、彭山、銅山、大浮。澧水在南,自安福入,東北流,東別出為內河,屈東南至道口,道水自西南來注之。又東至六塚口合澹水。水出石門東,過州治北,屈而南,又東至伍公嘴,涔水自西北來會。又南合澧水,至匯口入於安鄉。東有虎渡水,一名後小江,首受大江水,自湖北公安入,南流為一箭河水,其左岸則安鄉縣界也。又南至匯口入澧水。州判駐津市鎮。清化、順林二巡司。蘭松水、馬二驛。匯口、三汊河、津市、嘉山四鎮。順林司。後廢。石門難。州西南九十里。雍正七年自岳州來屬。西:石門。北:燕子。西北:層步,一名層山。又西北有盧黃山。澧水在南,又曰零陽河,自慈利入,北屈而東,與渫水合。水出西北龍門洞,東南流,右納黃水,左納溫水,又東南至渫口入澧水。澧水又東過縣治南,雙溪水自北來注之。又東北合朝陽溪水,入安福。又道水自慈利入,亦東北流入安福。西北有水南渡巡司。安鄉簡。州東南一百二十里。雍正七年自岳州來屬。北:黃山。東:石家。西:石龜。西北:古田。澧水在西,自州入,南至匯口,西別出為羌口河。又西南流為麻河,至武陵入漸水。其正渠東南匯於大鯨湖。又東過縣治南,長河水首受大江,北自公安來注之。又東南入南洲。又東,後江水,亦受大江,自湖北石首入,南流為景港水,至南洲入澧水。大溶湖北受澧水,注于沅。康熙十八年置焦圻、南平二驛,後廢。有羌口鎮。慈利簡。州西南一百六十里。雍正七年自岳州來屬。北:道人。東北:星子。西南:零陽。又有雲朝山。澧水在西,自永定入,東至褚溪口,右合九渡水,又東北與九谿河水合。水出湖北鶴峰州,即古漊水也。又東過縣治北,右納零溪水。入石門。又道水亦東北入石門。又龍潭河出西南,南至桃源入沅水。澧水在境為洲渚者八,為潭者二,為灘瀨者百三十二。有麻寮所、九谿衛城巡司。安福難。州西南六十里。雍正七年以慈利縣九谿衛地置,析澧州地益之,治裴家河,來屬。北:大銅。東:營駐。西南:大浮山,山跨石門、桃源、武陵諸縣。澧水在北。自石門入,迤東流,左納合溪,右納惡蛇溪,又東入澧州。又道水在縣南,亦自石門入,東北至澧州入澧水。有添平所、新安市巡司,乾隆三十二年廢。永定疲,難。州西南三百四十里。雍正十三年以慈利永定衛置,析安福縣地益之,治舊衛城,來屬。南:天門。西南:崇山。西北:馬耳。東北:香爐。澧水在南,自桑植入,南屈而東,武溪水自南來注之,又東與大庸溪合。又東流,左納無事溪,右納仙人溪,又過縣治東南,西溪水北流入焉,又東合社溪入慈利。又九渡水出縣南,東北至慈利入澧水。大庸所城在縣西。

南州直隸:繁,疲,難。隸岳常澧道。本華容縣地,咸豐四年,湖北石首縣藕池口決,江水溢入洞庭,淤為洲。光緒十七年置,治九都市,並析華容、巴陵、安鄉、武陵、龍陽、沅江諸縣地益之。東南距省治五百四十里。廣一百一十里,袤九十里。北極高二十九度二十一分。京師偏西四度一十三分。北:太陽山。東:明山。西南:清介。東南:洞庭湖。寄山、團山皆在湖中。西有澧水自安鄉入,東南逕白板口,歧為二,一西南至天心湖合沅水,一東與後江水合,又西南至冷飯洲,匯於洞庭。又有游橋水,首受後江,南至麻濠口入洞庭。又湧水自華容入,東南流,至治東北,南別出為神童港,與游橋水會。迤東過明山,其北岸則華容縣界也,又東至鋸子口入洞庭湖。

衡州府:沖,繁,難。隸衡永郴桂道。舊隸湖廣布政使司。康熙三年來屬。乾隆中增置清泉。東北距省治三百八十里。廣四百六十里,袤二百九十五里。北極高二十六度五十六分。京師偏西四度五分。領縣七。衡陽沖,繁,疲,難。倚。城內金鼇山。北:岣嶁山。西北:九峰、黃龍。西南:大雲山。東南:湘水左瀆自清泉入,北過府治東,北受烝水。水出邵陽東,合等江水,至陡江口,岳山水南流入焉。右納演陂,南流,武水自西南來會,納清化河,其右岸則清泉縣界也。東北逕石鼓山入湘水。又北,東入於衡山。有寒溪鎮。縣丞治查江市。有衡陽驛。清泉疲,難。倚。乾隆二十一年析衡陽縣東南鄉置,來屬。東:清泉山,縣以此名。南:回雁峰,衡岳之首峰也。南:雨母。西南:七寶、探山。湘水自祁陽入,迤東流,右界常寧縣,慄江水自西北來注之。又東過茭河口,西北過府治東。合耒水。北屈而東入衡山。西南:柿江水、清化水,皆東北至衡陽入烝水。東南有新城市巡司。廖田驛。衡山沖,繁,難。府東北一百里。西北:衡山,是為南嶽。東:靈山。東北:鳳凰。東南:楊山,又名武陽山。湘水自衡陽入,東北合龍隱港水,至茶陵江口,洣水合永樂江自東南來注之。北過縣治東,為觀湘洲。右納石灣港,左納樊田港,又北,東入湘潭。又涓水源出湘鄉,東合興樂江,東北至湘潭入湘水。有草市、永壽二巡司。雷家鎮有驛。耒陽沖,繁,難。府東南一百五十里。西:石臼。東:侯計山,跨安仁、永興二縣。東南:天門。東北:明月。耒水自永興入,東北流,右納肥江,西北至城南,屈東北,潯江水自東來注之,西北入清泉。其東馬水從之,亦至清泉入耒水。羅渡鎮有廢巡司。有驛。常寧難。府西南一百二十里。北:憩山。西南:塔山、液麻山。東南有逍遙。東北:盟山。西北:湘水右瀆自祁陽入,合吳水。又東北,左與清泉分岸。又東與宜水合。水出縣南西江山,北逕縣治西,左有藍江,右有潭水,皆流合焉,又東北至江口市入湘水。湘水又東北流,右納鹽湖,至茭河口,舂水北流西屈來會。水自桂陽州入,一名茭源河,其東岸則耒陽縣界也。湘水又北入清泉。有杉樹堡。西南壤接瑤峒。安仁簡。府東南一百五十里。南:大湖山。西:金紫。北:軍山。東北:排山。東南:大松山。西北:永樂江自永興入,北與浦陽港合。又北流,左納油陂港,右納蓮花港,北至安平市,大坪港西流合焉。又過城西北,宜陽港水自南來注之,西北至衡山入洣水。有潭湖鎮、安平鎮廢巡司。酃簡。府東南三百里。北:青臺。南:泰和。東南:萬陽。西南:屏水山。山與桂東接界,洣水出焉。迤北至雙江口,漠渡水北流西屈注之。又西合春江,即雲秋水,東北合洣水入茶陵,是為茶陵江。其東沔渡水,北為洮水,下流合於洣水。

永州府:沖,繁。隸辰沅永靖道。總兵駐。舊屬湖廣布政使司。康熙三年來屬。北距省治六百七十里。廣三百四十里。袤五百九十里。北極高二十六度九分。京師偏西四度五十三分。領州一,縣七。江藍同知一,嘉慶十九年移治江華縣濤墟市,後又遷於錦田所城。通判一,道光十二年移治新田縣楊家鋪。零陵沖,繁,難。倚。城內萬石山。西:西山。北:萬石。東北:巋山。東南:陽明山。西南:石城山、永山。湘水自東安合西南石頭江入,至府治西北,東南瀟水自道州合麻江水入,北與永水逕袁家渡至城南,合愚溪及鈷鉧潭來會,是為瀟湘。湘水北與蘆洪江水合,又北,東入祁陽。黃溪水出東南,馬子江出西南,並流合湘水。縣丞駐泠水灘。北有黃楊堡巡司,後廢。有驛。祁陽沖,繁。府東北一百里。北有祁山,縣以是名。南:白水。東南:樂山。東北:七星,即大雲山。西北:四望山。湘水自零陵入,東納浯溪,過縣城南,合祁水。水一名小東江,古曰水毛口,源出西北騰雲嶺,東南流,煙江水自北來會,入湘水。湘水迤東過白水鎮,白江水合黃溪水自西南來注之。屈東北,與清江水合。水出縣北鎮潭山,即古餘溪水也。東有歸陽市巡司,乾隆二十一年移治排山驛,尋復故。文明市有永隆廢巡司。有驛。有白水、樂山、文明、沙鎮、大營五鎮。東安簡。府西九十里。北:東山。西北:舜峰。東北:高霞。東南:伏虎。湘水自廣西全州流入,北屈而東,清溪江合宥江水自西北來注之,東與石期江水合,又東北入零陵。蘆洪江源出東北八十四渡山,東南流,左會龍合江,東南至零陵入湘水。有蘆洪市巡司,石期市廢司。淥埠、石期、荊塘三鎮道州難。府南一百五十里。城內元山。北:宜山。西北:瀟山、營道山。西南:營山。又都龐嶺界接永明,蓋五嶺之第三嶺也。瀟水在東。即古營水,又曰泥江,自寧遠縣入,西北至青口,與南源沱水合。水自江華入,北屈而西,合掩水,東北至州治南,營道水自西南來注之,今謂之小營水。又東北,左納宜江,會瀟水。其會流處曰三江口。瀟水又北納麻江水,入零陵。有白灘營。永安關界廣西灌陽。瑤山在東南。寧遠簡。府東南一百八十里。南:九疑山,跨道州、江華、藍山諸縣。北:陽明山、黃溪山。東北:舂陵,一名仰山。瀟水在南,源出九疑三分石,西北至江口會瀑水。水出東南舜源峰,即古泠水也,北流合漭水。又西北過縣治,都溪水自東北來注之,入瀟水。東北白江水,北入祁陽。其西大竹源水,一名楊柳溪,亦東北至祁陽。有梅岡鎮。九疑魯觀巡司。永明難。府西南二百二十里。北:永明嶺。即都龐嶺。東南:馬山。西南:荊峽鎮山,其下有鎮峽關,界接廣西恭城縣。掩水源出西北大掩峰,北過縣治西,右合古澤水,屈而東,角馬河水自東南來注之,東北至道州會沱水。西南沐水,南合遨水,西至桃川所城,右納皋澤,左納扶靈,西南入於恭城,其下流是為平樂水也。西南周棠寨巡司。又有白面墟司巡檢,後遷東南枇杷所城,更名,尋廢。桃川廢司。白象鎮。瑤山在縣西。江華繁。府南二百二十里。東:豸山。南:吳望。西南:蒼梧嶺,即臨賀嶺,又名萌渚嶺,跨廣西富川、賀縣,蓋五嶺之第四嶺也。沱水在東,上源曰中河,自藍山入,南屈而東,前河、後河皆流合焉,又西南逕錦田所城。宜遷水出廣東連山,西北流注之,西與靈江合。又西北合馮水,今謂之練江水也,至縣城東曰東河。西河曰萌渚水,自西南來會,又西北入道州。西南有錦岡巡司、錦田廢司。瑤山在縣東。新田簡。府東南二百八十里。南:七賢、藍山。西北:舂陵山,與桂陽州寧遠縣接界。舂水出焉,俗曰烏江水,東南逕夫人山,又南至縣城西南為西河水,東河水自東北來注之,又東屈而北,入於桂陽。東南白面墟廢司。東有瑤山。

桂陽直隸州:繁,疲,難。隸衡永郴桂道。舊桂陽州隸衡州府。雍正十年升為直隸府,仍所領。東北距省治六百三十里。廣二百二十七里,袤二百五十里。北極高二十五度四十九分。京師偏西四度零五分。領縣三。東:鹿峰。西:大湊,一名寶山。西北:壇山。西南:石門。東南:神渡。舂水自新田入,北過象鼻嘴,漼水合鼠峽水自西北來注之,即桂水也。屈東會鍾水,納泮溪,又北與楓江水合,至常寧入湘水。東南:仰天湖,屯湖水出,西北流,左合麻淪江,又北與泉田水合。屈東北,蓮蓬溪水北流來會,又東北入郴州。南牛橋鎮、北泗州寨二廢巡司。臨武簡。州西南一百四十里。北:八源,一名東山。西:舜峰。西南:華陰。又有西山,古名桐柏山,溱水出焉,東北流,左納貝溪,與秀溪水合。屈而北,武溪水合石江水自西來會。又東,赤土溪水南流合焉,東南入宜章。有赤土鎮。瑤山在縣南。藍山簡。州西南一百五十里。北:藍山。西:九疑。南:南風坳,界接廣東連州,鍾水出焉,西流屈北會巋水。水出九疑山,曰九疑水,亦謂之舜水,東北逕縣治南,左納濛溪,屈而東,毛俊水合華荊津水自東南來注之,又北與藍溪水合,東北入嘉禾。西南中河,入江華為沱水,下流合於瀟水。有毛俊鎮。大橋鎮巡司,後遷臨武營,更名。瑤山在縣南。嘉禾簡。州西南一百一十里。西:晉嶺,即藍山。北:石門。西北:石燕山。鍾水在南,自藍山入,東北流,至縣城東北,含溪水自西來注之,北至桂陽州入舂水。東南泮溪水,源出臨武,北與芹溪水合,亦至桂陽州入舂水。有兩路口廢巡司。

郴州直隸州:沖,繁,難。隸衡永郴桂道。舊隸湖廣布政使司。康熙三年來屬。北距省治六百八十里。廣三百四十里,袤二百三十里。北極高二十五度四十八分。京師偏西三度四十九分三十秒。領縣五。東:馬嶺山。東南:五蓋。西南:靈壽山。又黃岑山即騎田嶺,又名臘嶺,蓋五嶺之第二嶺也。耒水左瀆自興寧入,西北流,梓塘江自東南來注之,東北合郴水。出黃岑山,一名黃水,東北與沙江合。又北受千秋水,過縣治東,北合騾溪,又北至郴江口入耒水。耒水東北入永興,西有屯湖水,北逕棲鳳山,曰棲鳳水,又東北至永興入耒水。南有良田市巡司。有驛。永興沖,繁。州北八十里。城內三臺山。西:高亭。南:土富。北:金鵝。東北:桃源。西南:白豹山。耒水自州入,北合注江水,屈西過縣城西南,左納靈江,西北至森口,屯湖水合高亭水自西南來會,東北入耒陽。東大步江,源出興寧縣,合潦溪水,東北至安仁為永樂江。北安福、西南高亭二巡司,後廢。有驛。宜章沖,繁,難。州南九十里。北:黃岑。東北:漏天。南:西山。西南:莽山。溱水在南,亦曰武水,自臨武入,東南流,岑水合浯水自西北來注之,東南入廣東樂昌。縣北章水,南至樂昌為羅渡水,入武水。縣南長樂水,東北流,屈西,又東北至廣東乳源,為武陽溪,亦入武水。東赤石、西南白沙二巡司。有驛。有瑤山在縣南。興寧疲,難。州東北八十里。東:石牛。西:九峰。北:七寶。南:浦溪山。東南耒水自桂陽入,迤西至豐溪口,漚江自東北來注之,西北與資興江水合。水出縣東,即古清溪,亦謂之乙陂江,又西北合雷溪水,入郴州。縣北程江,西南至永興入耒水。東北小江水,一名大步江,亦西北入永興。又東春江,至酃合洣水。有滌口巡司,州門鎮廢巡司。桂陽簡。州東南一百六十里。南:屋嶺。東:洞靈。西:義通。西南:大官。東南:東嶺。耒水在南,出耒山,西北合淥水,秀溪水自西南來會。又西北與壽江水合,入興寧。北漚江自桂東入,右納淇江,為北水河,西北至興寧入耒水。縣南屋嶺水,南與藍田合,又南入仁化為恩溪。又益將河出東嶺,左合孤山水,東北至崇義為積龍水,下流合於章水。有益將、文明市二巡司。山口鎮、濠村鎮有二廢巡司。瑤山在縣南。桂東簡。州東北二百七十里。西:紫臺山。南:烏春。東:胸膛。東北:都寮山。又有屏水山,漚江出焉,一名澄江,南與螺川水合。西過縣城南,桂水自西北來會,又南為嚴溪,左東溪、右白竹皆流合焉,西納雙坑水,與大江水合,南流入桂陽。東南:泥湖山,大坪水出,入江西龍泉,為遂江水,入贛水。左溪水亦至龍泉合遂江水。西南有高分鎮廢巡司。

辰州府:沖,繁,難。隸辰沅永靖道。舊隸湖廣布政使司。康熙三年來屬。初沿明制,領州一,縣六。乾隆元年,沅州升府,黔陽、麻陽割隸。東距省治八百五里。廣三百五十里,袤六百五十里。北極高二十八度二十三分。京師偏西六度二十二分。領縣四。沅陵沖,繁,難。南:南山,一名客山。西北:小酉。東北:壺頭、明月。東南:聖人山。沅水在南,自瀘溪入,東北合藍溪,至府治西南,酉水合明溪、小酉溪自西北來注之。東北合深溪,北屈而東,左納硃洪溪、洞庭溪,右納怡溪,迤東入桃源。又冷溪出東南,北與三渡水合,又東北至桃源為夷望溪,入沅水。通判駐浦市。縣丞駐荔溪市。有馬底鎮、船溪二巡司。池蓬、明溪、會溪三廢巡司。辰陽、馬底二驛。瀘溪簡。府西南七十里。明,盧溪,清初改。東:權山。西:天橋,一名羊喬。北:虎頭。西南:踏湖山。沅水在東,自辰谿合浦溪入,北至縣城南,武水合沱江水自西來注之。水出乾州,曰武溪水,又名盧水也。沅水又東北入沅陵。西北潭溪水,西南大能水,皆流合武水。又太平溪出西南,東南至麻陽入沅水。南有溪洞廢巡司。辰谿沖。府西南一百一十里。南:五峴。西:大酉。北:熊頭。西南:房連、龍陽山。東南:沅水自黔陽入,北過茶龍山,合麻溪水,北入漵浦。又西北復入縣東,右納柿溪,迤西過縣城南,辰水自西來會,東北入瀘溪。縣南龍門溪,北流合辰水。有黃溪口巡司。山塘驛。有渡口、普市二鎮。漵浦繁,疲,難。府東南二百七十里。東:紅旗。東南:頓家。西北:盧峰。西南:大漵山。沅水在西,右會漵水,一名雙龍江,源出縣南金字山,逕龍潭溪,進馬江自東南來注之。屈而北,左納貓兒江,右納柿溪江,又北與龍灣江水合。又西北流,宣陽江東北自聖人山來會,西至縣治東南,大潭水南流合焉,又西合沅水,東北入辰谿。南有龍潭巡司。瑤山在縣南。

沅州府:沖。隸辰沅永靖道。本明沅州,隸辰州府。乾隆元年升為府。東北距省治一千一百三十五里。廣二百八十里,袤二百五十五里。北極高二十七度二十三分。京師偏西七度零三分三十秒。領縣三。芷江沖,繁,難。倚。乾隆元年,以州地置。北:明山。東北:武陽。東:花山。東南:高明。西南:羅山。西北:米公山。潕水即無水,自晃州入,東北流,左納柳林溪、粟米溪,屈東南,過府治南,楊溪東流屈北注之,與五郎溪合。東屈而南,豐溪水自東北來入,東南入黔陽。西南:中和溪,出晃州東南,至黔陽入沅水。縣丞治榆樹灣,懷化、便水二巡司。晃州、便水、羅舊、懷化四驛。黔陽沖。府東南九十里。本隸辰州府。乾隆元年來屬。南:赤寶。北:紫霄。東:龍標。東北:鉤崖。東南:羅公山。沅水在西,自會同入,東至托口寨,左合中和溪,右合渠水,屈東北至縣城西,與潕水會。其會流處曰清江口,即古無口也。又東南流,錯入會同,迤東北復入縣東,供溪水北流西屈注之。水出綏寧,其上流為𦰡溪水,東北入辰谿。東:石橋、安江二巡司,道光十二年廢。瑤山在東南。麻陽難。府西北一百二十里。本隸辰州府。乾隆元年來屬。北:紗帽。南:西晃。東:苞茅。東南有齊天。東北:雄山,其下有雄關。辰水在南,一名麻陽江,自貴州銅仁入,東與密粟溪水合。左納銅信溪,右納石橋溪,過縣治東南,屈而北,樂濠溪自西北來注之,又東合太平溪,至辰谿入沅水。縣丞治巖門寨,有高村巡司。巖門驛。

永順府:難。隸辰沅永靖道。明為永順等處軍民宣慰使司。領土州三:南渭、施溶、上溪;長官司六:臘惹峒、麥著黃峒、驢遲峒、施溶峒、白崖峒、田家峒。隸湖廣都司。雍正四年改流官置,隸辰州府。七年升為府。東南距省治一千八十里。廣五百里,袤五百五十里。北極高二十九度二分,京師偏西六度四十分。領縣四。永順難。倚。本永順宣慰司地。雍正七年置,治猛峒。東南距舊司治三十里。東:飛霞山、賀虎。東北:蟠龍。東南:羊峰。西北:萬笏。酉水中源自保靖合入逝溪,東與喇集溪合。溪出龍山,曰汝池河,東南過府治西南,小溪水自北來注之,南與牛路河合,入酉水。酉水又南屈而東,左納施溶溪,入沅陵。東南:明溪,亦南至沅陵入酉水。東北:上洞河,出縣北,過十萬坪入桑植,是為澧水南源。府經歷駐劉家寨。王村巡司。田家峒廢司。驛三:王村、毛坪、高望界。龍山簡。府西北二百二十里。雍正七年析永順宣慰司地置,治麂皮壩。乾隆元年又省大喇土司地入焉。南:洛塔。東南:鐵爐。西南:八面山。酉水在南,即北河,又名更始水。三源,其北源曰白水河,自湖北宣恩緣界流入,南逕縣治西北,中界湖北來鳳縣,又南流,果利河自東北來注之。又南與皮渡河合,為卯洞河,西南錯入酉陽州。其中源曰邑梅河,出秀山,北流東屈來會,又東復入縣西南境。其右岸則保靖縣界也。東與洗車河合,入保靖。東南:汝池河,至永順入酉水。有隆頭巡司。保靖難。府西南一百四十里。本保靖宣慰司地。領五寨、筸子坪二長官司。雍正四年改流官置,隸辰州府。七年改為縣來屬,治茅坪,西南距舊司治半里。西:煙霞、洛浦。北:雲臺。南:呂洞山。酉水自四川酉陽州入,迤東流,左界龍山,又東屈而南。其南源牛角河,出貴州松桃,東流屈北來會。又東過縣治北,左納蒙沖溪。又東與白溪水合,入永順。張家寨巡司。保靖、白棲關二站。桑植簡。府東北一百二十里。本桑植安撫司地。領美坪等苗峒凡十有八。雍正四年,改流官置,隸岳州府。七年改為縣,析慈利縣安福所地益之。治安福所城。乾隆元年,復省上峒、下峒、茅岡三土司地入焉。北:天星。東:陽岐。東南:簸箕山。澧水三源:西北源曰夾石河,出慄山坡,東南為綠水河,又東至兩河口;南源上峒河,自永順北流來會,又東與涼水口河合;河出西北七眼泉,是為澧水北源。東屈而南,至縣治西北,長酉水自東北來注之。又南入永定。又有繩子溪,出東北紅花嶺,東南至慈利入漊水。有下峒廢巡司。

靖州直隸州:繁,難。隸辰沅永靖道。本隸湖廣布政使司。康熙三年來屬。雍正五年,割天柱隸貴州黎平。東北距省治一千六十里。廣三百七十里。袤三百六十里。北極高二十度三十五分。京師偏西七度。領縣三。南:侍郎。東:鴻陵。西:飛山。西北:艮山。西南:青蘿。渠水在東,古謂之敘水,自通道入,北至縣治東南,右納老鴉溪,左納潩溪,西北入會同。西南有四鄉河,源出貴州開泰,東北至通道入渠水。有零溪巡司。會同難。州北九十里。北:巖屋。西北:八仙。東北:金龍。沅水在西北,自貴州天柱縣入,東北錯入黔陽。又東逕縣東北,巫水合若水溪自東南來注之,入於黔陽。西:渠水自靖州入,北逕縣治西北,右納平川,與吉朗溪合。水出貴州開泰,又名郎江水,西北至黔陽入沅水。西南堡子巡司。洪江司,廢。通道難。州南九十里。東:玉柱。東南:福湖。又佛子山,渠水出焉,西北過犁嘴山,播陽河自西南來會。河出開泰,曰六沖江,又名洪州江也。北與四鄉河水合,北至縣治西南,臨川河入焉,又東北入靖州。西南有播陽廢巡司。綏寧繁,難。州東南一百二十里。北:寶鼎。東北:藍溪。又有楓門山,巫水在西,即洪江,古謂之運水,又曰雄溪,自城步入,西北至界溪口,蒔竹水自南來注之。又北流為竹舟江,西北至會同入沅水。又蓼溪水,源出東北雞籠山,東為武陽水,又東北入武岡州,是為高沙市水也。南:長平水自城步入,西流,右納駕馬溪,又西與雙江水合,西北至通道合渠水。有青陂、雙江二巡司。

乾州直隸:繁,難。隸辰沅永靖道。明為鎮溪軍民千戶所,隸辰州府瀘溪縣。康熙三十九年改為乾州。四十七年置,治鎮溪所城,仍隸辰州府。嘉慶元年升直隸。轄苗寨一百一十有五。東北距省治九百六十五里。廣一百二十里,袤九十里。北極高二十八度十二分。京師偏西六度五十九分。東:鎮溪。西:武山,武水出焉,一名武溪,又名盧溪,迤東過治西,屈而南,萬溶江自鳳凰北流東屈注之。又東與鎮溪水合,東南入瀘溪。有河溪、乾州二廢巡司。鎮溪、喜鵲二營,皆嘉慶二年置。

鳳凰直隸:繁,難。鎮筸總兵、辰沅永靖道駐。明為五寨、筸子坪二長官司,隸保靖宣慰使司。康熙四十三年,改流官置通判,辰沅靖道僉事徙駐。雍正四年改鳳凰營。乾隆五十二年改,升通判為同知。嘉慶元年升直隸。轄紅苗寨一百有五。東北距省治一千五十里。廣一百八十四里,袤一百二十里。北極高二十七度五十三分。京師偏西七度三分。南:南華山。西:鳳凰山,上有鳳凰營,又有鳳凰營司巡檢,後廢。東南:觀景。南:二華。西南:都督。沱江自貴州銅仁入,迤東北流,烏巢江自北來注之。東過治北,又東北入於瀘溪,是為武水最南源也。又,萬溶江源出西北天星砦山,東屈而北,左納龍爪溪,西北至乾州合武水。西南:樂濠溪,東南至麻陽入辰水。祐營,知事駐。得勝營、五寨站有巡司。

永綏直隸:繁,難。隸辰沅永靖道。綏靖總兵駐。明,鎮溪千戶所、崇山衛地,隸辰州府瀘溪縣。雍正元年置吉多營,仍隸辰州府。嘉慶元年升直隸。七年移治花園堡。轄紅苗寨二百二十有八。東北距省治一千一百五十九里。廣九十里,袤一百五十五里。北極高二十八度四十三分。京師偏西七度。南:大排吾山。西:苞茅。西南:蠟爾。牛角河即酉水南源,自貴州松桃緣界流入,北至茶洞城,其左岸則四川酉陽州界也。屈而東北,界保靖縣。東過治北,臘爾堡河自西南來注之,東北入保靖。西南:高巖河,源出犀牛潭,入乾州為鎮溪,入武水。茶洞,廢知事,隆團、排補二砦廢司。有花園砦。

晃州直隸:沖。隸辰沅永靖道。本芷江晃州堡地,屬沅州府。嘉慶二十二年析置直隸,移涼傘通判治焉。東北距省治一千二百四十五里。廣五十二里,袤一百四十五里。北極高二十七度二分。京師偏西七度二十二分。西:龍溪。西南:尖坡。東南:寶駿山。潕水在南,一名無水,又名水舞水。上流曰鎮陽江,自貴州玉屏入,東北與龍溪合。過治南,左納木多溪,東流會平溪,東北入芷江。東南:中和溪,一名羅巖江,亦東北流入芷江。晃州、涼傘二巡司。有驛。


\end{pinyinscope}