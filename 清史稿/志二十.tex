\article{志二十}

\begin{pinyinscope}
時憲一

明之大統術,本於元之授時。成化以後,交食往往不驗。萬歷末,徐光啟、李之藻等譯西人之書為新法,推交食、凌犯皆密合,然未及施用。世祖定鼎以後,始絀明之舊歷,依新法推算,即承用二百六十餘年之時憲術也。光啟等齗齗辨論,當時格而不行,乃為新朝改憲之資,豈非天意哉!聖祖邃於歷學,定用均輪法以齊七政,以康熙甲子為元。雍正中,從監臣之請,推步改橢圓法,以雍正癸卯為元。道光中,監臣以交食分秒不合,據實測之數損益原用數,以道光甲午為元。自康熙至於道光,推步之術凡三改,而道光甲午元歷僅有恆星表。至於推日月交食、步五星,均未及成書云。西人湯若望,與徐光啟共譯新法者也,以四十二事證西人之密、中術之疏,疇人子弟翕然信之。宣城人梅文鼎研精天算,由授時以溯三統、四分以來諸家之術,又博考九執、回回術,而折衷於新法,皆洞其原本,究其異同,卒以績學受知聖祖,於是為推步之學者,始知中、西之學之一貫,不至眩晃於新法矣。與湯若望同時入中國者為穆尼閣,傳其學於淄川薛鳳祚,而吳江人王錫闡自創新法,用以推日月食,不爽秒忽,兩家之學,皆不列於臺官,然其精密,或為臺臣所不及焉。今為時憲志,詳考其推步、七政、四餘、根理、法數著於篇,諸家論說有裨數理者,亦撮其大要載之。明大統術、回回術,康熙初用之,以詳於明史,不具論。

推步因革

順治元年六月,湯若望言:「臣於明崇禎二年來京,曾依西洋新法釐訂舊歷,今將新法所推本年八月初一日日食,京師及各省所見食限分秒,並起復方位,圖象進呈,乞■期遣官測驗。」從之。七月,若望又推天象進呈。是月禮部言:「欽天監改用新法,推注已成,請易新名,頒行天下。」睿親王言:「宜名『時憲』,以稱朝廷憲天乂民至意。」從之。八月丙辰朔午時,日食二分四十八秒,大學士馮銓,同若望赴觀象臺測驗覆奏,惟新法一一菂合,大統、回回二歷俱差時刻,敕:「舊法歲久自差,非官生推算之誤,新法既密合天行,監局宜學習勿怠玩。」十月,頒順治二年時憲書。若望又言:「敬授人時,全以節氣交宮與太陽出入晝刻為重。今節氣之日時刻分與太陽出入晝夜時刻,俱照道里遠近推算,請刊入時憲書。」從之。十一月,以若望掌欽天監事。若望等言:「臣等按新法推算月食時刻分秒,復定每年進呈書目,刪其衣復重,以免混淆。」二年六月,若望等言:「舊法推算本年十二月己卯朔辰時日食三分強,回回科算見食一分弱。依新法推之,止應食半分強,且在日出之前,地平上不見,請臨期遣官測驗。」從之。至期天陰雨,推驗事遂輟。十一月,若望以明大學士徐光啟所譯崇禎歷書改名新法歷書進呈,上命發監局官生肄習,仍宣付史館,加若望太常寺卿銜。十年,賜若望通玄教師,以★其勤勞。

若望之法,以天聰戊辰為元。分周天為三百六十度。太陽一日平行五十九分八秒十九微四十九纖三十六芒,最高一年行四十五秒,戊辰年平行距冬至五十三分三十五秒三十九微,最高距冬至五度五十九分五十九秒。太陰一日平行一十三度一十分三十五秒一微,自行一十三度三分五十三秒五十六微,正交行三分一十秒,月孛行六分四十一秒,戊辰年平行距冬至六宮一度五十分五十四秒四十六微,自行距冬至六宮二十五度三十二分一十五秒三十四微,正交行距冬至一宮一十四秒,月孛行距冬至十一宮六度一十九分。土星諸行應平行距冬至為十一宮十八度五十一分五十一秒,本年最高行距冬至為九宮八度五十七分五十九秒,平行距最高即引數,為二宮九度五十三分五十二秒,正交行距冬至為六宮七度九分八秒。一平年平行為十二度十三分三十一秒,最高行一分二十秒十二微,以最高行減平行,得十二度十二分十五秒,乃一年之引數也。一閏年平行為十二度十五分三十五秒,引數為十二度十四分十五秒。正交行一年為四十二秒。木星諸行應平行距冬至為八宮二十八度八分三十一秒,本天最高行為十一宮二十七度十一分十五秒,平行距最高即引數,為九宮初度五十七分十六秒,正交行為六宮二十四度四十一分五十二秒。一平年距冬至平行為一宮零二十分三十二秒,最高行為五十七秒五十二微,兩數相減,得一宮零十九分三十四秒,乃一平年之引數也。一閏年距冬至平行為一宮零二十五分三十一秒,引數為一宮零二十四分三十三秒。正交行一年為一十四秒。火星諸行應平行距冬至為五宮四度五十四分三十秒,本天最高在七宮二十九度三十分四十秒,平行距最高即引數,為九宮五度二十三分五十秒,正交行為三宮十七度二分二十九秒。一平年距冬至平行為六宮十一度十七分一十秒,最高行一分十四秒,兩數相減,得六宮十一度十五分五十五秒。一閏年距冬至平行為六宮十一度四十八分三十六秒,引數為六宮十一度四十七分二十一秒。正交行一年為五十三秒。金星諸行應平行距冬至與太陽同度,為初宮初度五十三分三十五秒三十九微,平行距最高即引數,為六宮零五十六分五十五秒,伏見行從極遠處起,為初宮九度十一分七秒,最高行在六宮零十六分六秒。一平年距冬至為十一宮二十九度四十五分四十秒三十八微,自行引數為十一宮二十九度四十四分十七秒,伏見行為七宮十五度一分五十秒,最高行為一分二十一秒。一閏年距冬至及自行加五十九分八秒,伏見行加三度六分二十四秒,乃一日之行也。金星正交在最高前十六度,即五宮十四度十六分,其行極微,故未定其率,然於最高行無大差。水星諸行應平行距冬至與太陽同度,平行距最高即引數,為二十九度二十分二秒,伏見行從極遠處起,為三宮二十九度五十四分一十六秒,最高在十一宮零五十二分四十二秒。一平年距冬至亦與太陽同度,自行引數為十一宮二十九度四十三分五十一秒,伏見行滿三周外有一宮二十三度五十七分二十六秒。一閏年引數為十二宮零四十二分五十九秒,伏見行全周外為一宮二十七度三分五十二秒,正交行或曰與最高同度難測,故不敢定云。

若望論新法大要凡四十二事:曰天地經緯,天有經緯,地亦有之,以二百五十里當天之一度,經緯皆然。曰諸曜異天,諸曜高卑相距遠甚,駁舊歷認為同心之誤。曰圓心不同,太陽本圈與地不同心,二心相距,古今不等。曰蒙氣差,地有蒙氣,非先定蒙氣差不能密合。曰測算異古法,測天以弧三角形,算以割圜八線表。曰測算皆以黃道,測天用赤道儀,所得經度不合,新法就黃道經度,通以黃赤通率表,乃與天行密合。曰改定諸應,從天聰二年戊辰前冬至後己卯日子正為始。曰求真節氣,舊法平節氣,非真節氣,今改定。曰盈縮真限,用授時消分為平歲,更以最高最卑差加減之,為定歲。曰表測二分,舊法以圭表測冬至,非法之善者,今用春秋二分,較二至為密。曰太陽出入及晨昏限,從京師起算,各處有加減。曰晝夜不等,其差較一刻有奇,一緣黃道夏遲冬疾,一緣黃赤二道廣狹不同距,則率度不同分。曰改定時刻,定晝夜為九十六刻。曰置閏不同,舊法用平節氣置閏,非也,改用太陽所躔天度以定節氣。曰太陰加減,朔望止一加減,餘日另有二三,均數多寡不等。曰月行高卑遲疾,月行轉周之最高極遲,最卑極疾,五星準此。曰朔後月見遲疾,一因自行度遲疾,一因黃道升降斜正,一因白道在緯南緯北。曰交行加減,月在交上,以平求之必不合,因設一加減為交行均數。曰月緯距度,舊法黃白二道相距五度,不知朔望外尚有損益,其至大之距,五度三分之一。曰交食有無,月食以距黃道緯度較月與地景兩半徑★,日食則以距度較日月兩半徑★,距度為小則食,大則不食。曰日月食限不同,月食則太陰與地景兩周相切,以其兩視半徑較白道距黃道度,又以距度推交周度定食限,日食必加入視差而後得距度。曰日月食分異同,距度在月食為太陰心實距地景之心,在日食為日月兩心之距,但日食不據實距而據視距。曰實食中食,以地心之直線上至黃道者為主,日月五星兩居此線之上,則實食也;月與五星各居本輪之周,地心直線上至黃道,而兩本輪之心俱當線上,則中食也。曰視食,日食有天上之實食,有人所見之視食,視食依人目與地面為準。曰黃道九十度為東西差之中限,論天頂則高卑差為正下,南北差為斜下,而東西差獨中限之一線為正下,以外皆斜下。論其道則南北差為股,東西差恆為勾,高卑差恆為弦。至中限則股弦為一線,無勾矣。曰三視差,以地半徑為一邊,以太陽太陰各距地之遠為一邊,以二曜高度為一邊,成三角形,用以得高卑差,又偏南而變緯度得南北差;以黃道九十度限偏左偏右而變緯度,得東西差。曰外三差,東西、南北、高卑之差,皆生於地徑,外三差不生於地徑而生於氣。一,清蒙氣差;二,清蒙徑差;三,本輪徑差。曰虧復不一,非二時折半之說,新法以視行推變時刻,則了然於虧復時刻不一之故。曰交食異算,諸方各以地徑推算交食時刻及日食分。曰日食變差,據法應食而實不見食,必此日此地之南北差變為東西差,此千百年偶遇一二次,非無有者。曰推前驗後,新法諸表,上溯下沿,開卷了然,不費功力。曰五星準日,舊法於合伏日數,時多時寡,徒以段目定之,不免有差,今改正。曰伏見密合,舊法五星伏見惟用黃道距度,非也,今改正。曰五星緯度,太陰本道斜交黃道,因生距度與陰、陽二歷,五星亦然,新法一一詳求,舊未能也。曰金水伏見,金星或合太陽而不伏,水星離太陽而不見,用渾儀一測便知,非舊法所能。曰五星測法,測五星須用恆星為準。曰恆星東移,恆星以黃道極為極,各宿距星時近赤極,亦或時遠赤極,由黃赤二道各極不同,非距星有異行或易位。曰定恆星大小有六等之別,前此未聞。曰天漢釋疑,新法測以遠鏡,天漢乃無算小星攢聚而成。曰四餘刪改,羅★即白道之正交,計都即中交,月孛乃月行極高之點。至紫★一餘,無數可定,明史附會,今俱改刪。曰測器,新法增置者,曰象限儀、百游儀、地平儀、弩儀、天環天球紀限儀、渾蓋簡平儀、黃赤全儀,而新制之遠鏡,尤為測星要器。曰日晷,為地平晷、三晷、百游晷、通光晷。此外更有星晷、月晷,以備夜測之用。若望所言,大抵據新法以詆舊術之疏,然新法之精蘊,亦盡於此矣。

十四年四月,前回回科秋官正吳明炫言:「臣祖默河亦裡等十八姓,本西域人,自隋開皇己未抱其學重譯來朝,授為日官。一千五十九年,專司星宿行度吉兇,每年推算太陰五星凌犯,天象占驗,日月交食,即以臣科白本進呈為定例。順治三年,本監掌印湯若望令臣科凡日月交食及太陰五星凌犯、天象占驗俱不必奏進。臣查若望所推七政,水星二、八月皆伏不見,今水星二月二十九日仍見東方,又八月二十四日夕見,關系象占,不敢不據實上聞。乞復立回回科,以存絕學。」奏下所司。時新安★官生楊光先叩閽進摘謬論,糾湯若望新法之謬,且言:「時憲書有『依西洋新法』五字尤不合。」又進選擇議,糾若望選榮親王葬期用洪範五行,山向、日月俱犯忌殺。

康熙三年十二月,禮部議「時憲書面『依西洋新法』五字擬改『奏準』二字」,從之。四年,議政王等言:每日百刻,新法改為九十六刻;二十八宿次序,湯若望將觜、參二宿改易前後;又將四餘刪去紫★,俱不合。其選擇不用正五行,用洪範五行,以致山向、日月俱犯忌殺,事犯重大,將湯若望及科官等分別擬凌遲斬決。敕湯若望從寬免死,時憲科李祖白等五人俱處斬。於是復用大統舊術,以楊光先掌監務,光先抗疏屢辭,不允。光先於推步之學本不深,唐熙七年,謂明年當閏十二月,尋知其誤,自行檢舉,而時憲書已頒行,乃諭天下停止閏月雲。是年監副吳明烜言:「古法差謬,五官正暨回回科所進各不同,立加校正。」下禮部議。禮部覆奏:「五官正戈繼文等所算七政金水二星差錯太甚,主簿陳聿新所推七政未經測驗,亦有差錯,監副吳明烜所推七政與天象相近,理應頒行,仍令監臣同四科官,每日晝測晷景以定節氣,夜測月五星以定行度。」從之。

十一月,西人南懷仁言所頒各法不合天象,乃召南懷仁、利類思、安文思及監官馬祐、楊光先、吳明烜等至東華門,大學士李霨傳諭:「授時乃國家要政,爾等勿挾宿仇,以己為是,以彼為非。是者當遵用,非者當更改,務期歸於至善。」十二月,南懷仁劾吳明烜所造康熙八年七政時憲書糾謬,下王大臣、九卿、科道會議,議政王等言:「乞派大臣同南懷仁等測驗。」乃遣圖海、李霨等二十人赴觀象臺測驗。八年二月,議政王等議覆:「圖海等赴觀象臺測驗,南懷仁所言皆合,吳明烜所言皆謬,問監正馬祐等,亦言南懷仁所算實與天象合。竊思百刻雖行之已久,但南懷仁九十六刻之法既合天象,自應頒用。又南懷仁言羅★、計都、月孛系推算所用,故載於七政之後,其紫★星無用處,不應造入。應自康熙九年為始,用九十六刻之歷」,時明烜言「臣祗知天文,不知歷法」,光先言「臣不知歷法,惟知歷理」。光先語尤不遜,褫職。三月,授南懷仁欽天監監副。先是監官依古法推算,康熙八年十二月應置閏,南懷仁言雨水為正月中氣,是月二十九日值雨水,即為康熙九年正月,不應置閏,置閏當在明年二月。監官多直懷仁,從其言,改閏九年二月,於是大統、回回兩法俱廢,專用西洋法,如順治之初。八月,南懷仁劾楊光先誣陷湯若望叛逆,議政王等議「湯若望應復通微教師,照原品賜恤,楊光先應反坐」。敕「免議」。

十三年二月,新造儀象志告成,南懷仁加太常寺卿銜。十四年二月,諭監副安泰從何君錫學古歷法。十五年二月,欽天監奏五月朔日食,監副安泰依古法算,應食五分六十秒,南懷仁新法只應食二十微三分秒之一。至期登臺測驗,酉正食甚,將及一分,戌初刻復圓,古法所推分數失之甚遠,而新法亦不甚合。南懷仁曰:「此清蒙氣之所為,蒙氣能映小為大故也。」

十七年七月,欽天監進呈康熙永年表三十二卷。二十二年十月,監臣推算盛京九十度表告成。初,南懷仁奏:「各省北極高度不同,其交合之時刻食分俱不等,全憑各省之九十度表推算。向來不知盛京北極高度,即用京師之九十度表,今測得盛京北極比京師高二度,請依其高度推算九十度表。」從之。至是,以盛京九十度表進呈,諭「永遠遵守」云。

四十一年十月,大學士李光地以宣城貢生梅文鼎歷學疑問三卷進呈,上曰:「朕留心歷算多年,此事朕能決其是非。」乃親加批點還之,事具梅文鼎傳。文鼎論中、西二法之同異曰:「今之用新歷也,乃兼用其長,以補舊法之未備,非盡廢古法而從新法術也。夫西歷之同乎中法者,不止一端。其言五星之最高加減也,即中法之盈縮歷也,在太陰,則遲疾歷也。其言五星之歲輪也,即中法之段目也。其言恆星東行也,即中法之歲差也。其言節氣之以日躔過宮也,即中法之定氣也。其言各省真節氣不同也,即中法之裏差也。但中法言盈縮遲疾,而西說以最高最卑明其故;中法言段目,而西說以歲輪明其故;中法言歲差,而西說以恆星東行明其故。是則中歷所言者當然之運,而西歷所推者其所以然之理,此其可取者也。若夫定氣裏差,中歷原有其法,但不以法歷耳,非古無而今始有也。西歷始有者,則五星之緯度是也。中歷之緯度,惟太陽、太陰有之,而五星則未有及之者。今西歷之五星有交點、有緯行,亦如太陽太陰之詳明,是則中歷缺陷之大端,得西法以補其未備矣。夫於中法之同,亦既有以明其所以然之故,而於中法之未備者,又有以補其缺,於是吾之積候者,得彼說而益信,而彼說之若難信者,亦因吾之積候而有以知其不誣,雖聖人復起,亦在所兼收而並取矣。」

五十年十月,上諭大學士等:「天文歷法,朕素留心,西法大端不誤,但分刻度數之間,積久不能無差。今年夏至,欽天監奏午正三刻,朕細測日景,是午初三刻九分。此時稍有舛錯,恐數十年後所差愈甚。猶之錢糧,微塵秒忽,雖屬無幾,而總計之,便積少成多,此事實有證驗,非比書生論說可以虛詞塞責也。」又諭禮部考取效力算法人員,臨軒親試,取顧琮等四十二人。五十一年五月,駕幸避暑山莊,徵梅文鼎之孫梅★成詣行在。先是命蘇州府教授陳厚耀,欽天監五官正何君錫之子何國柱、國宗,官學生明安圖,原任欽天監監副成德,皆扈從侍直,上親臨提命,許其問難如師弟子。及徵★成至,奏對稱旨,遂與厚耀等同直內廷。五十二年五月,修律呂、算法諸書,以誠親王允祉、皇十五子允烜、皇十六子允祿充承旨纂修,何國宗、梅★成充匯編,陳厚耀、魏廷珍、王蘭生、方苞等充分校。所纂之書,每日進呈,上親加改正焉。

五十三年四月,諭誠親王允祉等:「古歷規模甚好,但其數目歲久不合,今修書宜依古歷規模,用今之數目算之。」十月,又諭:「北極高度、黃赤距度最為緊要,著於澹寧居後逐日測量。」乃制象限儀,儀徑五尺,範銅為之,晝測日度,夜測勾陳帝星。又制中表、正表、倒表各二,俱高四尺,中表測日中心,正表、倒表測日上下邊之景。惟六表所得日景尾數多參差不合。海★成言:「表高景澹,尾數難真,自古患之。昔郭守敬為銅表,端挾二龍,舉橫梁至四十尺,因其景虛澹,創為景符以取實影。其制以銅葉博二寸,長加博之二,中穿一竅若針芥然,以方木為趺,一端設機軸,令可開闔。稽其一端,使其針斜倚,北高南下,往來遷就於虛影之中。竅達日光,僅如黍米,隱然見橫梁於其中。」乃仿元史郭守敬制造景符六,如法用之,影尾數始毫末不爽。測得申昜春園北極高三十九度五十九分三十秒,比京師觀象臺高四分三十秒,黃赤大距二十三度二十九分,比舊測減二分雲。十一月,誠親王允祉等言:「郭守敬造授時術,遣人二十七處分測,故能密合。今除申昜春園及觀象臺逐日測驗外,如福建、廣東、雲南、四川、陜西、河南、江南、浙江八省,於里差尤為較著,請遣人逐日測量,得其真數,庶幾東西南北裏差及日天半徑,皆有實據。」從之。

五十八年二月,以推算人不敷用,敕禮部錄送蒙養齋考試,取傅明安等二十八人,命在修書處行走。六十年,禦制算法書成,賜名數理精蘊。諭:「此書賜梅文鼎一部,命悉心校對。」遣其孫梅★成齎書賜之。六十一年六月,歷書稿成,並律呂、算法,共為律歷淵源一百卷:一曰歷象考成上、下編,一曰律呂精義上、下編,續編,一曰數理精蘊上、下編。雍正元年,頒歷象考成於欽天監,是為康熙甲子元法。自雍正四年為始,造時憲書一遵歷象考成之法。又議準其禦制之書,無庸欽天監治理,其治歷法之西洋人授為監正。八年六月,監正明安圖言:「日月行度,積久漸差,法須旋改,始能密合。臣等遵禦制歷象考成推算時憲,據監正戴進賢、監副徐懋德推測,覺有微差。於本月初一日日食,臣等公同測驗,實測與推算分數不合,乞敕下戴進賢、徐懋德詳加校定修理。」從之。十年四月,修日躔、月離表成。

乾隆二年四月,協辦吏部尚書事顧琮言:「世宗皇帝允監臣言,請纂修日躔、月離二表,以推日月交合,★交宮過度,晦朔弦望,晝夜永短,以及凌犯,共三十九頁,續於歷象考成諸表之末。查造此表者,監正西洋人戴進賢;能用此表者,監副西洋人徐懋德與五官正明安圖。擬令戴進賢為總裁,徐懋德、明安圖為副總裁,盡心考驗,增補圖說。歷象考成內倘有酌改之處,亦令其悉心改正。」敕:「即著顧琮專管。」五月,琮復言:「乞命梅★成為總裁,何國宗協同總裁。」從之。十一月,命莊親王允祿為總理。

三年四月,莊親王允祿等言:「歷象考成一書,其數惟黃赤大距減少二分,餘皆仍新法算書西人第穀之舊。康熙中西人有噶西尼、法蘭德等,發第穀未盡之義,其大端有三:其一謂太陽地半徑差,舊定為三分,今測祗有十秒;其一謂清蒙氣差,舊定地平上為三十四分,高四十五度,祗有五秒,今測地平上止三十二分,高四十五度,尚有五十九秒;其一謂日月五星之本天非平圓,皆為橢圓,兩端徑長,兩腰徑短。以是三者,經緯度俱有微差。戴進賢等習知其說,因未經徵驗,不敢遽以為是。雍正八年六月朔日食,舊法推得九分二十二秒,今法推得八分十秒,驗諸實測,今法為近。故奏準重修日躔、月離新表二差,以續於歷象考成之後。臣等奉命增修表解圖說,以日躔新表推算,春分比前遲十三刻許,秋分比前早九刻許,冬夏至皆遲二刻許。然以測午正日高,惟冬至比前高二分餘,夏至秋分僅差二三十秒。蓋測量在地面,而推算則以地心,今所定地半徑差與蒙氣差皆與前不同,故推算每差數刻,而測量終不甚相遠也。至其立法以本天為隋圓,雖推算較繁,而損益舊數以合天行,頗為新巧。臣等闡明理數,著日躔九篇並表數,乞親加裁定,附歷象考成之後,顏曰禦制後編。凡前書已發明者,不復贅述。」報聞。七年,莊親王允祿等奏進日躔、月離、交宮共書十卷,是為雍正癸卯元法。

九年十月,監正戴進賢等言:「靈臺儀象志原載星辰約七十年差一度,為時已久,宜改定。康熙十三年修志之時,黃赤大距與今測不同,所列諸表,當逐一增修。三垣二十八宿以及諸星,今昔多寡不同,亦應釐訂。」敕莊親王、鄂爾泰、張照議奏。十一月,議準仍以三人兼管。是年更定羅★、計都名目,又增入紫★為四餘。十七年,莊親王允祿等言儀象志所載之星,多不順序,今依次改正,共成書三十卷,賜名儀象考成。是月莊親王等復奏改正恆星經緯度表,並更定二十八宿值日觜參之前後。敕大學士會同九卿議奏。十二月,大學士傅恆等言:「請以乾隆十九年為始,時憲書之值宿,改觜前參後。」從之。既而欽天監又以推算土星有差減平行三十分,自乾隆以後至道光初,交食分秒漸與原推不合。

道光十八年八月,管理欽天監事務工部尚書敬徵言:「自道光四年臣管理監務,查觀象臺儀器,康熙十三年所制黃赤大距,皆為二十三度三十二分。至乾隆九年重制璣衡撫辰儀,所測黃赤大距,則為二十三度二十九分,是原設諸儀已與天行不合,今又將百年,即撫辰儀亦有差失。臣將撫辰儀更換軸心,諸儀亦量為安置。另制小象限儀一,令官生晝測日行,夜測月星,每逢節氣交食,所測實數有與推算不合者,詳加考驗。知由太陽緯度不合之數,測得黃赤大距較前稍小,其數僅二十三度二十七分。由交節時刻之早晚,考知太陽行度有進退不齊之分。夫太陽行度為推測之本,諸曜宗之。而推日行,又以歲實、氣應兩心差曰本天最卑行度為據。擬自道光十四年甲午為年根,按實測之數,將原用數稍為損益,推得日行交節時刻,似與實測之數較近。至太陰行度,以交食為考驗之大端。近年測過之月食,較原推早者多,遲者少。故於月之平行、自行、交行內量為損益,按現擬之平行,仍用諸均之舊數,推得道光十四年後月食三次。除十七年三月祗見初虧,九月天陰未測,僅測得道光十六年九月十五日月食,與新數所推相近,然僅食一次,尚未可憑,仍須隨時考驗。現■本年八月十五日月食,謹將新擬用數推算得時刻食分方位,比較原推早見分秒,另繕清單進呈。至期臣等逐時測驗,再行據實具奏。」報聞。

二十二年六月,敬徵等又言:「每■日月交食,按新擬用數推算,俱與實測相近。至本年六月朔日食,新推較之實測,僅差數秒。是新擬之數,於日行已無疑義,月行亦屬近合。今擬先測恆星,以符運度,繼考日躔、月離,務合天行。請以道光十四年甲午為元,按新數日行黃赤大距,修恆星、黃赤道經緯度表,即於測算時詳考五緯月行,俾恆星、五緯、日月交食等書,得以次第竣事。」從之。是年七月,以敬徵為修歷總裁,監正周餘慶、左監副高煜為副總裁。

二十五年七月,進呈黃道經緯度表、赤道經緯度表各十三卷,月五星相距表一卷,天漢界度表四卷,經星匯考、星首步天歌、恆星總紀各一卷,為儀象考成續編。至日月交食、五星行度俱闕而未備云。時冬官正司廷棟撰凌犯視差新法,用弧三角布算,以限距地高及星距黃極以求黃經高弧三角,較舊法為簡捷。乾隆以後,歷官能損益舊法,廷棟一人而已。其不為歷官而知歷者,梅文鼎、薛鳳祚、王錫闡以下,江永、戴震、錢大昕、李善蘭為尤著。其闡明中、西歷理,實遠出徐光啟、李之藻等之上焉。


\end{pinyinscope}