\article{列傳二十一}

\begin{pinyinscope}
孔有德全節耿仲明子繼茂繼茂子昭忠聚忠尚可喜子之孝

沈志祥兄子永忠永忠子瑞祖大壽子澤潤澤溥澤洪澤洪子良璧

大壽養子可法從子澤遠

孔有德,遼東人。太祖克遼東,與鄉人耿仲明奔皮島,明總兵毛文龍錄置部下,善遇之。袁崇煥殺文龍,分其兵屬副將陳繼盛等。有德與仲明走依登州巡撫孫元化為步兵左營參將。

天聰五年,太宗伐明,圍大凌河城。元化遣有德以八百騎赴援,次吳橋,大雨雪,眾無所得食,則出行掠。李九成者,亦文龍步將,與有德同歸元化,元化使齎銀市馬塞上,銀盡耗,懼罪。其子應元在有德軍,九成還就應元,咻有德謀為變。所部陳繼功、李尚友、曹得功等五十餘人,糾眾數千,掠臨邑,凌商河,殘齊東,圍德平,破新城,恣焚殺甚酷。元化及山東巡撫余大成皆力主撫,檄所過郡縣毋邀擊,有德因偽請降。明年正月,率眾逕至登州,仲明與都司陳光福及杜承功、曹德純、吳進興等十五人為內應,夕舉火,導有德入自東門,城遂陷。元化自剄不殊,有德等以元化故有恩,縱使航海去。旅順副將陳有時、廣祿島副將毛承祿亦叛應有德,勢益張。有德自號都元帥,鑄印置官屬,九成為副元帥,仲明、有時、承祿、光福為總兵官,應元為副將,四出攻掠。明以徐從治為山東巡撫,謝璉為登萊巡撫,並駐萊州。有德等進陷黃縣、平度,遂攻萊州,從治中砲死城上。有德復偽請降,誘璉出,殺之。莊烈帝命侍郎硃大典督師討有德,援平度,斬有時,至昌邑,有德逆戰,大敗,復黃縣。有德等退保登州。

登州城東西南皆距山,北臨海,城北復有水城通海舶。大典督諸軍築長圍困之,九成出戰死,明師攻益急;有德乃謀來降,以子女玉帛出海,仲明單舸殿,經旅順,明總兵黃龍以水師邀擊,擒承祿、光福,殲應元,斬級千餘。有德等退屯雙島龍安塘,食盡,遣所置游擊張文煥、都司楊謹、千總李政明以男婦百人泛海至蓋州。蓋州戍將石國柱、雅什塔護使謁上,具言有德等舉兵始末,且請降。上諭範文程、羅什、剛林預策安置。有德等復遣所置副將曹紹中、劉承祖等奉疏,言將自鎮江登陸,上命貝勒濟爾哈朗、阿濟格、杜度帥師迓之。朝鮮發兵助明師,要有德等鴨綠江口。濟爾哈朗等兵至江岸,嚴陣相對,敵師不敢逼。有德等舟數百,載將士、槍砲、輜重及其孥畢登,三貝勒為設宴,上使副將金玉和傳諭慰勞。

七年六月,有德、仲明入謁,上率諸貝勒出德盛門十里至渾河岸,為設宴,親舉金卮酌酒飲之,賜蟒袍、貂裘、撒袋、鞍馬,有德、仲明亦上金銀及金玉諸器、採段、衣服。越二日,復召入宮賜宴,授有德都元師、仲明總兵官,賜敕印,即從所署置也。命率所部駐東京,號令、鼓吹、儀衛皆如舊,惟刑人、出兵當以聞。有德等怨黃龍,必欲報之。會聞龍發水師逐賊鴨綠江,旅順無備,上命貝勒岳託、德格類帥師襲之,以有德率為導。龍數戰皆敗,遂自殺,克其城。有德等兵入占官吏富民廨宅,多收俘獲。岳託、德格類聞於上,上置不問。有德墜馬傷手,與仲明留遼陽,詔慰之曰:「都元帥遠道從戎,良亦勞苦。行間諸事,實獲朕心。招撫山民,尤大有裨益。不謂勞頓之身,又遭銜橛之失。佇聞痊可,用慰朕懷。」別敕令旗纛用皁色,並誡軍士以時演習槍砲、弓矢;馬以牌,甲胄以帶,皆書滿洲字為識別。有德、仲明旋入朝,上誡毋餽遺貝勒大臣。八年,朝元日,命有德、仲明與八和碩貝勒同列第一班,遣官為營第,疏辭不允。

有德、仲明軍駐遼陽,官吏經其地,必躬迎款宴。上復誡諭之,謂:「爾等即朕子弟,款接諸臣理有未當。今後非貝勒,毋更迎宴。爾等偕至者如有困窮,當加愛養。爾等或遣使詣盛京,當令使者告禮部,禮部與館餼。不然,爾等新附,親知尚少,使來無居無食,不重困乎?」及尚可喜來降,上遇之亞有德、仲明。命更定旗制,以白鑲皁,號有德、仲明軍為天祐兵,可喜軍為天助兵。國語謂漢軍「烏真超哈」,有德等自將所部不相屬。八月,從上伐明,自大同入邊,有德遣所部黑成功、佟延以八十人擊敗明兵代州城東,獲馬二十。九年,有德等為部將請敕,上命自給札。鮑承先疏論當令吏部給劄付,上不允。有德等仍錄所部副將以下請敕,上曰:「爾等初來歸,朕許爾等黜陟部將。今復給敕,是背前言。朕非謂爾等無功不當畀敕書也,慮失信耳。」因賜有德、仲明、可喜人緞一、貂皮六十,副將以下白金有差。有德以新附者日眾,偕仲明輸糧佐餉,上卻之。

崇德元年夏四月,上受寬溫仁聖皇帝尊號,有德從諸貝勒奉寶以進,封恭順王。十二月,上自將伐朝鮮,命有德等從貝勒杜度護輜重繼後。二年二月,既下江華島,命有德等從貝子碩託以水師取皮島。師還,有言其部眾違法妄行者,上命申嚴約束,毋蹈故轍。三年,從攻錦州,有德等以砲攻下戚家堡、石家堡及錦州城西臺,降大福堡;又以砲攻下大臺一,俘男婦三百七十九,盡戮其男子;又以砲攻五里河臺,臺隅圮,明守將李計友、李惟觀乃率其眾出降,皆籍為民,勿殺。四年,從攻松山,以砲擊城東隅臺,臺上藥發,自燔,殲其餘眾,又降道旁臺二。上至松山,使有德等以砲攻其南郛。有德當郭門,仲明居右,馬光遠佐之;可喜居左,石廷柱佐之。自夜漏下至翌日晡,城堞盡毀。明守將金國鳳即夜繕治,守甚固,有德議穴地攻之,不克。六年,率兵更番圍錦州,破明師杏山。七年,松山、錦州相繼下。時析烏真超哈為八旗,有德等請以所部隸焉,乃分屬正紅旗。八年,從取中後所、前屯衛。

順治元年,從睿親王多爾袞入關,追擊李自成至慶都。九月,上至京師,賜有德等貂蟒朝衣。十月,上御皇極門大宴,復賜鞍馬。旋命有德從定國大將軍豫親王多鐸西討李自成。二年,陜西既定,移師下江南,克揚州,取明南京,攻江陰,有德皆有勞。八月,師還,賜繡朝衣一襲、馬二、黃金百、白金萬。命還鎮遼陽,簡士馬待徵發。

三年五月,諭兵部召有德等率所部會京師。八月,授有德平南大將軍,率仲明、可喜及續順公沈志祥、右翼固山額真金礪、左翼梅勒額真屯泰率師南征,策自湖廣下江西贛南入廣東,諭諸將悉受有德節制。是時明桂王稱號,湖廣總督何騰蛟駐湘陰,諸將李赤心、黃朝宣、劉承胤、袁宗第、王進才、馬進忠等分屯湖南北,號「十三鎮」,大抵自成餘黨及左良玉舊部。

四年春,有德師至,進才自長沙走湖北,騰蛟亦棄湘陰單騎奔衡州。有德遣梅勒額真卓羅、藍拜等躡進才,與所部水師遇,擊敗之。有德進次湘潭,朝宣以十三萬人屯燕子窩。有德率藍拜等將水師,可喜及卓羅等將陸師,分道並入,破明將徐松節。朝宣走衡州,有德以師從之,獲朝宣。有德令仲明、金礪、卓羅等將水師還詣長沙,明將楊國棟以二千人屯天津湖,巴牙喇甲喇章京張國柱、札薩藍等與戰,國棟自牛皮灘遁去。有德令金礪留駐衡州,復與仲明及卓羅等率師越熊飛嶺克祁陽,遂破寶慶,擊殺明魯王世子乾生,總兵黃晉、李茂功、吳興等。時明桂王居武岡,倚承胤為守。有德夜發寶慶,前隊梅勒章京黑成功等敗敵,焚木城,奪門入,明桂王走靖州,轉徙入桂林,承胤出降。

有德始自長沙下祁陽也,聞郝搖旗圍桂陽,令可喜及藍拜等別將兵赴援;郝搖旗部卒千四百人屯翔鳳鋪,令巴牙喇纛章京線國安、固山大蘇朗等擊破之,搖旗引去。至是國安等遂趨靖州,追明桂王。明總兵蕭曠、姚有性以萬二千人守靖州,國安師薄城,奪門入,獲曠、有性等,又破明侍郎蓋光英軍。藍拜略黔陽,進攻沅州,明將張宣弼以三萬人出戰,我兵奮擊,遂克其城。自出師至此,凡獲明宗室桂王子爾珠等二十七人,降明將自承胤以下四十七人,偏裨二千餘人、馬步兵六萬八千有奇。捷聞,賜有德黃金二百五十兩,仲明、可喜各二百,志祥百,將士賚白金有差。

五年春,復進克辰州,湖南諸郡縣悉定。又旁取貴州黎平府、廣西全州,招降銅仁、全州、興安、灌陽苗峒二百九十有奇,復獲明宗室榮王子松等四十餘人,及所置總兵以下諸將吏甚眾。上命有德班師,至京師,宴勞,賜黑狐、紫貂、冠服、採帛、鞍馬、黃金二百、白金二千。

六年五月,改封有德定南王,授金冊金印,令將舊兵三千一百、新增兵萬六千九百,合為二萬人,徵廣西,設隨征總兵官一、左右翼總兵官各一,以授馬蛟麟、線國安、曹得先。同時仲明、可喜各將萬人徵廣東,但設左右翼,制閷於有德。自有德師還,湖南諸郡縣復為赤心、進才、宗第等侵據,上命鄭親王濟爾哈朗為定遠大將軍,帥師討之,克長沙、寶慶、衡州諸府,獲騰蛟;而進忠猶據武岡,與曹志建、鄭思愛、劉祿、胡光榮、林國瑞、黃順祖、向文明等為寇靖、永、郴諸州,窺寶慶。

十月,有德師至衡州,遣副將董英、何進勝擊思愛,戰於燕子窩,擒斬之;進克永州,擊走明將胡一青。七年春,復進破龍虎關,殲志建,遂攻武岡,陣獲祿、光榮等。進忠負創走,克其城,並下靖州。復進戰興寧,獲順祖、國瑞,招文明等以五萬人降。師入廣西境,克全州。十二月,遂拔桂林,明桂王走南寧,留守大學士瞿式耜死之,斬靖江王以下四百七十三人,降將吏一百四十七人。桂林、平樂諸屬縣皆下。

八年春正月,有德奏移籓屬駐桂林,遣蛟麟、國安取梧州、柳州,略旁近諸州縣。有德進駐賓陽,復遣國安等分三道進取,定思恩、慶遠,明將陳邦傅以潯州來降。明桂王走廣南,南寧亦下。

九年四月,有德疏言:「臣荷先帝節錄微勞,錫以王爵。恭遇聖主當陽,兩粵八閩未入版圖,臣謬辱廷推,駐防閩海。同時有固辭粵西之役者,蓋以其地最荒僻,民少山多,百蠻雜處,諸孽環集,底定難預期也。臣自念受恩至渥,必遠闢巖疆,始敢伸首丘夙原,故毅然以粵西為請。受命以來,道過湖南,伏莽蔓延,六郡拮據,一載咸與掃除。乃進徵粵西,仰藉威靈,所向克捷。賊黨或竄或降,雖土司瑤、伶、俍、僮,古稱叛順靡常者,亦漸次招徠,受我戎索,粵西底定。臣生長北方,與南荒煙瘴不習。解衣自視,刀箭瘢痕,宛如刻劃。風雨之夕,骨痛痰湧,一昏幾絕。臣年邁子幼,乞恩敕能臣受代,俾臣得早覲天顏,優游終老。」疏入,得旨:「覽王奏,悉知功苦。但南疆未盡寧謐,還須少留,以俟大康。」

五月,有德率輕兵出河池,向貴州,留師柳州為後援。是時張獻忠將孫可望降於明,窺伺楚、粵,有德請敕剿撫。將軍續順公沈永忠駐沅州,扼門戶。時國安擢廣西提督,馬雄為左翼總兵,全節為右翼總兵,分守南寧、慶遠、梧州。未幾,明將李定國、馮雙禮自黎平出靖州,馬進忠自鎮遠出沅州,會於武岡。永忠使乞援,有德遣兵赴之,至全州。永忠已棄寶慶,退保湘潭,有德因還桂林。七月,定國自西延大埠取間道疾驅擊破全州軍,薄桂林,驅象攻城。城兵寡,定國晝夜環攻,有德躬守陴,矢中額,仍指揮擊敵。敵奪城北山俯攻,有德令其孥以火殉,遂自經,妻白氏、李氏皆死於火。事聞,謚有德武壯。十一年六月,有德女四貞以其喪還京師,上命親王以下、阿思哈尼哈番以上,漢官尚書以下、三品官以上,郊迎,賜白金四千,官為營葬,立碑紀績。尋復命建祠,祀春秋,以白氏、李氏配。

有德子廷訓,為定國所掠,越六年,乃殺之。及我師克桂林,隨征總兵李茹春求得遺骼,以其死事狀上聞,命予恤。四貞至京師,賜白金萬,視和碩格格食俸,旋嫁有德部將孫龍子延齡,延齡叛應吳三桂,自有傳。國安、雄皆附延齡為亂,附見延齡傳。

節,廣寧人。在明官參將。從有德降,授甲喇額真。有德既克桂林,以節為右翼總兵。克慶遠,使節戍焉。降宜山、河池、思恩、荔浦諸縣。順治九年七月,有德遣兵援寶慶,令節移屯梧州;聞定國兵且至,復令節移屯平樂。節方至柳州,定國已破桂林,柳州副將鄭元勛等叛降定國,謀襲節。節間道走梧州,與國安、雄合軍。定國來攻,我師戰而敗,節負重傷潰圍出,乘舟至肇慶。可喜遣水師助節,乃還定梧州、藤縣、潯州。十年正月,復平樂。馬雄守梧州,而與國安共擊破明將陳經猷、王應龍,遂克桂林。明將胡一青、龍韜、楊振威以數萬人屯陽朔、永福間,節屢戰破之。敘功,加都督,進三等精奇尼哈番。移屯武宣,平象州,獲明將韋文有、羅天舜。

十二年,移屯荔浦。時明宗室盛濃、盛添,明將李茂先、龔瑞屯富川,糾土寇王心、蔣乾相等及瑤、僮為亂,跨湖南、貴州界,依山結寨,為可望、定國聲援。節與國安等迭遣兵擊之,獲盛濃、盛添,諸瑤、僮百九十二寨皆下。十五年,上命國安徵貴州,奏請令節移屯柳州。十六年,復督兵討茂先、瑞,戰融縣,獲茂先;戰懷遠,瑞降。康熙元年,改右翼總兵為左江鎮總兵,即以命節。七年七月,卒,贈太子少保。

方定國破桂林也,節妻溫氏率妾婢自經,子成忠,年十一,被掠去。及洪承疇定貴州,得之降將趙三才所。至是,襲三等精奇尼哈番。

從有德降者,又有李尚友、徐元勛、胡璉、曹紹中、孟應春,皆受世職梅勒章京,分隸正黃、鑲黃二旗。

耿仲明,字雲臺,遼東人。初事明總兵毛文龍,文龍死,走登州依巡撫孫元化,皆與孔有德俱,元化以仲明為中軍參將。時總兵黃龍鎮皮島,所部有李梅者,仲明黨也,通洋。事覺,龍系之獄。仲明弟都司仲裕在龍軍,率部卒假索餉名,圍龍廨,擁之至演武場,折股去耳鼻,將殺之,諸將為救免。龍乃執殺仲裕,疏請罪仲明。元化劾龍蝕餉致兵譁。明莊烈帝命充為事官,而覈仲明主使狀。會有德已叛,還攻登州,仲明遂糾諸將同籍遼東者為內應。城陷,推有德為帥,受署置,稱總兵官。天津裨將孫應龍自言誇與仲明兄弟善,能令縛有德以降。巡撫鄭宗周使將二千人自海道往。仲明偽為有德首,紿之開水城,延使入,猝斬之,殲其眾,得巨艦,以為舟師。明師攻登州急,天聰七年五月,從有德來降,上禮遇優異,授以總兵官,號其兵曰天祐兵。語並詳有德傳。

仲明侵漁所部,所部愬於有德。有德因劾仲明,仲明引咎,請以所部赴愬者移屬有德。上敕獎有德,令善撫之;亦諭仲明善撫部下,毋念舊惡。即日並召入宮賜宴。越數日,又使賜羊酒,且諭之曰:「朕聞諸漢官從爾等教場角射,設筵饗爾等,意爾等必欲相酬報。爾等去家遠,可即以此羊酒藉教場為答宴也。」旋命與有德同駐遼陽。崇德元年,封仲明懷順王。上屢出師伐明,討朝鮮,仲明皆從。七年八月,命隸正黃旗。九月,所部甲喇額真石明雄訐仲明匿所獲松山、杏山人戶;有逃人被法,仲明為收葬設祭;復妄殺無辜:鞫實,罰仲明白金千兩。八年十一月,甲喇額真宋國輔、潘孝及明雄謀殺仲明,仲明以聞,鞫實,斬國輔等,籍其家畀仲明。順治初,從睿親王多爾袞入關,復從豫親王多鐸西討李自成,移師定江南。三年,有德為平南大將軍,帥師南征,仲明等以所部從。與明將楊國棟戰於牛皮灘,大破之;克衡州、祁陽、武岡諸郡縣;獲明將郭肇基。皆仲明功也。六年,改封靖南王,賜金冊金印。

仲明自降後,屢出征伐,恆與有德俱,未嘗獨將。是歲始與有德分道出師,有德徵廣西,仲明與尚可喜徵廣東。仲明將舊兵二千五百、新增兵七千五百,合為萬人,以徐得功為左翼總兵,連得成為右翼總兵。師既行,刑部奏論仲明部下梅勒章京陳紹宗等縱部卒匿逃人,罪當死。上因諭仲明,察隨征將士攜逃人以往者,械歸毋隱。仲明察得三百餘人械歸,上疏請罪,吏議當奪爵,上命寬之,紹宗等亦貸死。仲明未聞命,十一月次吉安,自經死。

子繼茂,順治初授世職昂邦章京。仲明死,繼茂在軍中,代領其眾,請襲爵,睿親王方攝政,持不可。繼茂從可喜俱南,定廣東諸郡縣。語見可喜傳。八年,世祖親政,繼茂嗣為王。九年,李定國陷桂林,孔有德死事。上聞報,命定遠大將軍敬謹親王尼堪自湖南移師赴之,敕可喜、繼茂俟尼堪至,合軍進攻,而繼茂先已與可喜遣兵赴援,復梧州及旁近諸郡。十年,潮州總兵郝尚久據城叛,繼茂與靖南將軍喀喀木、總兵吳六奇合軍討之,圍城逾月,城將王立功為內應,樹雲梯以登,尚久入井死,餘賊盡殲。潮州及饒平、揭陽、澄海、普寧諸縣悉平。十一年二月,命內翰林秘書院學士郎廷佐齎敕慰勞,賜白金三千,分賚將士。是歲李定國徇高、雷、廉三府,進犯新會。繼茂、可喜與靖南將軍珠瑪喇合軍進擊,再戰皆捷。定國還據南寧,復出攻橫州,繼茂自梧州帥師赴之,解橫州圍。進攻南寧,定國走安隆,獲明將李先芳,斬裨將杜紀等。十三年,賜敕紀功,增籓俸歲千金。

初,繼茂與可喜攻下廣州,怒其民力守,盡殲其丁壯。即城中駐兵牧馬。營靖南、平南二籓府,東西相望,繼茂尤汰侈,廣徵材木,採石高要七星巖,工役無藝;復創設市井私稅:民咸苦之。廣東左布政使胡章自山東赴官,途中上疏,言:「臣聞靖南王耿繼茂、平南王尚可喜所部將士,掠辱士紳婦女,占居布政使官廨,並擅署置官吏。臣思古封建之制,天子使吏治其國而納其貢稅焉,不得暴彼民也。二王以功受封,宜仰體聖明憂民至意,以安百姓,乃所為如是,臣安敢畏威緘默?乞敕二王還官廨,釋俘虜。」繼茂奏辯,可喜亦有疏自白,章坐誣論絞,上命貸死奪官。逾年,高要知縣楊雍建內遷給事中,疏陳廣東濫役、私稅諸大害,謂:「一省不堪兩籓,請量移他省。」朝議令繼茂移鎮桂林,未行。十六年三月,上命移四川。十七年七月,改命移福建。

時明將鄭成功據金門,窺伺閩、浙,繼茂既移鎮,與總督李率泰協謀征剿。康熙元年,成功死,子錦代領其軍。上命繼茂相機剿撫,繼茂疏報:「自順治十八年訖元年,招降將吏二百九十、兵四千三百三十四、家口四百六十七。」其後成功弟世襲、兄子纘緒及所置都督鄭賡先後出降,復得將吏七百有奇、兵七千六百有奇。二年十月,繼茂與率泰督兵渡海克廈門,水師提督施瑯以荷蘭夾板船來會,乘勝取浯嶼、金門二島。錦與其將周全斌等走銅山,復入犯雲霄、陸鼇諸衛,總兵王進功與戰,大破之。三年三月,繼茂復與率泰及海澄公黃梧合軍,自八尺門出海克銅山,錦以數十舟走臺灣。捷聞,上嘉其功,復增歲俸千金。十年正月,疏陳疾劇,乞以長子精忠代治籓政,上允其請。五月,卒,謚忠敏。精忠嗣為王,別有傳。

昭忠,繼茂次子;聚忠,繼茂第三子。順治間先後入侍世祖,授昭忠一等精奇尼哈番,以貝子蘇布圖女妻焉。昭忠例得多羅額駙,進秩視和碩額駙;聚忠尚柔嘉公主,為和碩額駙:同加太子少保,旋又同進太子太保。康熙十三年,精忠叛,昭忠、聚忠率子姓請死,系於家待命,逾年貰其罪,復秩如故。十四年,命聚忠齎敕招精忠,精忠拒不納。十五年,精忠降,授昭忠鎮平將軍,駐福州,代精忠治籓政。籓下參領徐鴻弼等訐精忠降後尚蓄逆謀,昭忠具以聞,並劾助逆曾養性等十餘人。上以精忠在軍,未即發。十七年,命昭忠以其祖父之喪還葬蓋平。十九年,召精忠詣京師,昭忠、聚忠疏劾精忠背恩為亂,違母周氏訓,蹙迫以死,誣祖仲明與吳三桂在山海關時先有成約,請予顯戮。尋命聚忠詣福州,議徙籓兵。聚忠疏陳籓兵當盡徙,稱旨,命以精忠家屬還京師。精忠既誅,昭忠、聚忠疏陳家屬眾多,艱於養贍,請如漢軍例,披甲食糧。下部議,編五佐領,隸漢軍正黃旗。二十五年,昭忠卒,謚勤僖。二十六年,聚忠卒,謚愨敏。

尚可喜,遼東人。父學禮,明東江游擊,戰歿樓子山。明莊烈帝崇禎三年,擢副總兵黃龍為東江總兵官,駐皮島,可喜隸部下。皮島兵亂,龍不能制,可喜率兵斬亂者,事乃定。後二年,孔有德等叛明,陷登州,旅順副將陳有時、廣鹿島副將毛承祿皆往從之。龍遣可喜及金聲桓等撫定諸島。有德黨高成友者據旅順,斷關、寧、天津援師,龍令游擊李維鸞偕可喜等擊走之,即移軍駐其地。旋以可喜為廣鹿島副將。明年秋七月,有德等從我師攻旅順,龍兵敗,自殺,部將尚可義戰死,蓋可喜兄弟行也。明以沈世奎代龍為總兵,部校王庭瑞、袁安邦等構可喜,誣以罪。世奎檄可喜詣皮島,可喜詗得其情,遂還據廣鹿島。

天聰七年十月,遣部校盧可用、金玉奎謁上乞降。上報使,賜以貂皮,並令車爾格等偵可喜蹤跡。八年正月,可喜舉兵略定長山、石城二島,行且至,上命諸貝勒集滿、漢、蒙古諸臣諭曰:「廣鹿島尚副將攜民來歸,非以我國衣食有餘也,承天眷佑,彼自來附。八家貝勒已出粟四千石,凡積粟之家,當量出佐餉,仍予以值。」二月,命貝勒多爾袞、薩哈廉往迓。三月,可喜至海州,上降敕慰勞。攻旅順時,獲可喜戚屬二十七人,至是,命歸諸可喜。四月,可喜入朝,上迎十里外,拜天畢,禦黃幄,可喜遙行五拜禮,進至上前再拜,抱上膝以見,所部將士以次羅拜,可喜跪進贐。上與宴,賜蟒衣、鞓帶、帽鞾、玄狐裘、雕鞍、馬、駝、羊,命諸貝勒以次設宴。旋授可喜總兵官,賜敕印,可用、玉奎皆為甲喇章京,號其軍曰天助兵,命駐海州。

尋從伐明,自宣化入邊,略代州。崇德元年四月,封智順王。十二月,從伐朝鮮。二年,朝鮮降。從貝勒碩託帥師克皮島,斬世奎,師還,賚蟒服、黃白金。可喜家僮訐可喜私得人戶、金帛、牲畜,法司以奏。上曰:「此豈王自得,必散於眾兵耳。其勿問。」三年,從伐明,攻錦州,屢攻下臺堡,更番駐牧,敵至輒擊敗之。七年,錦州下,賜所俘及降戶。可喜與有德等疏請以所部屬烏真超哈,分隸鑲藍旗。八年,從伐明,取中後所、前屯衛諸城。

順治元年,從入關,擊李自成,追至慶都,斬自成將穀可成等。十月,命從英親王阿濟格西討自成,出邊自榆林趨綏德,二年二月,師次米脂。自成兄子錦猶據延安,用可喜議,令諸軍分道進,錦走,克其城。會豫親王多鐸已破潼關,定西安,上命可喜從英親王追擊自成,分兵克鄖陽、荊州、襄陽諸郡,降自成將王光恩、苗時化等。復與英親王合軍下九江,聞自成竄死九宮山,乃班師,賜可喜繡朝衣一襲、馬二,還鎮海州。

三年八月,授有德為平南大將軍,征湖廣,命可喜率所部兵偕行。師次湘潭,明將黃朝宣以十三萬人屯燕子窩,可喜與梅勒章京卓羅等自陸路進擊,敗明將徐松節,遂逐斬朝宣。既,聞郝搖旗攻桂陽急,可喜與梅勒章京藍拜帥師赴援。郝搖旗以千四百人屯翔鳳鋪,巴牙喇纛章京線國安等與戰,郝搖旗敗走,桂陽圍解。湖南既定,師還,與有德等同賜冠服、金幣、鞍馬。

六年五月,改封平南王,賜金冊金印。旋命率舊兵二千三百、新增兵七千七百,合萬人,與耿仲明同征廣東,以許爾顯為左翼總兵官,班志富為右翼總兵官。仲明所部匿逃人,事發,因諭有德等檢校軍中得逃人悉送京師,仲明懼罪自裁。吏議可喜亦坐奪爵,上命納白金四千以贖。時明桂王駐肇慶,兩廣尚為明守。是歲除夕,可喜潛兵襲南雄,城兵三千出西門迎戰,擊敗之,立雲梯以登。明守將江起龍棄城走,斬其部將楊傑、董洪信、鄭國林等三十餘人、兵六千有奇。

七年正月,進克韶州。明守將羅成耀聞南雄破,已先遁,明桂王走梧州。復進下英德、清遠、從化諸縣,明將吳六奇等迎降。二月,師薄廣州。廣州城三面臨水,李成棟之叛,於城西築兩翼,令附城外為砲臺,水環其下。成棟死信豐,子元胤、建捷代將,元胤留肇慶,建捷守廣州。可喜令攻城,阻水不能進,乃鑿深壕,築堅壘,為長圍困之。建捷拒戰甚力,暑雨鬱蒸,我師弓矢皆解膠,久相持不下。元胤與明將陳邦傅等分道援廣州,邦傅與杜永和等以萬餘人自清遠赴戰,可喜擊敗之,獲裨將魏廷相等,明水師總兵梁標相來降,得戰船百五十助攻;復招潮州守將郝尚久、惠州守將黃應傑,皆以其城降,遣將士戍焉。圍合十閱月,永和部將範承恩助守廣州,約內應,決砲臺下水,可喜令諸軍皆舍騎藉薪行淖中以濟,遂得砲臺;據城西樓堞發砲擊城西北隅,城圮,師畢登,克廣州,俘承恩等,斬六千餘級,逐餘眾迫海濱,溺死者甚眾。明將宋裕昆自肇慶率所部來降。八年春,可喜遣爾顯等收肇慶,並下羅定,部將徐成功克高州。梁標相叛,遣兵討平之。

九年春正月,可喜與耿繼茂帥師南下,降明將蔡奎,遂入廉州,遣部將呂應學等攻克欽州,戰於靈山,獲元胤及明將袁勝、周朝,陣斬明益陽王、明將上官星拱。師將下雷、瓊,永和及明西平王縛明將李明忠以降。於是高、雷、廉、瓊四府皆定。七月,李定國陷桂林,有德死之。梧州、南寧、平樂、潯州、橫州皆復為明,東略化州、吳川。可喜遣兵與有德部將提督線國安,總兵馬雄、全節,合軍以進,廣西諸郡縣以次收復。十年八月,可喜別遣兵克化州、吳川。

十一年冬,定國以萬餘人侵廣東,擾高、雷、廉三府境,深入陷高明,分兵攻肇慶,圍新會,可喜與繼茂疏請發禁旅為援。上已先命珠瑪喇為靖南將軍,帥師援廣東。可喜等師次三水,遣兵援肇慶,破定國兵於四會河口,待珠瑪喇師至合軍擊定國,戰於珊洲,斬定國裨將一,俘十餘人,馘百五十餘,進薄新會。定國與其將吳子聖阻山而軍,馬步兵分屯嶺隘,可喜麾兵急擊,奪徑以登,斬獲甚眾。定國走,新會圍解。可喜與繼茂督軍攻高明,定國遣兵禦戰,獲其將武君禧等三十餘人,斬三百餘級,得馬騾、軍械無算。可喜遣梅勒章京畢力克圖等逐定國,戰於興業,定國敗走;復及於橫州江,殲馬步兵甚眾,獲象二。定國渡江焚橋引去,廣東高、雷、廉三府,廣西橫州諸州縣悉平。十三年四月,又克揭陽、晉寧、澂海三縣。閏五月,賜敕紀功,增歲俸千兩,並賚貂裘、鞍馬。自是明桂王徙雲南,定國等不復侵廣東,數歲無兵事。可喜與繼茂並開府廣州,所部頗放恣為民害,自左布政胡章以論可喜等得罪,無復言者。

十七年,移繼茂福建,可喜專鎮廣東。廣東初定,又以令徙瀕海居民,民失業去為盜。有鄧耀者據龍門,入掠雷陽;又有蕭國隆,與其徒洪彪、周祥、方泰、陳期新等分據恩平、開平、陽江、陽春諸山寨,掠廣州諸屬縣,並及肇慶。可喜先後遣兵討之,耀走死,斬彪、祥、泰、期新及其徒千五百人,國隆投水死。又有周玉,故戶,自號恢粵將軍,繒船數百,三帆八棹,沖浪若飛,習水戰。鄭成功兵至,輒助剽掠。康熙二年,可喜遣兵討之,獲玉,焚其舟。四年,碣石總兵蘇利叛,可喜遣潮州總兵許龍以舟師進擊,利出降。玉餘黨譚琳高竄據東湧海島,戶黃明初等濟以米糧。可喜遣部將佟養謨擊琳高,舒雲護等捕明初,皆就誅。

初,可喜遣長子之信入侍。十年十一月,疏言有疾,請令還廣東暫領軍事,上允其請。十二年二月,遣侍衛古德、米哈納使廣東勞軍,齎御用貂帽,團龍天馬裘、藍蟒狐腋袍各一襲,束帶一圍,賜可喜。三月,可喜疏乞歸老海城,諭曰:「王自航海歸誠效力,累朝鎮守粵東,宣勞歲久。覽奏,年已七十,欲歸老遼東,恭謹能知大體,朕深嘉悅。」下議政王大臣及戶、兵二部集議,議盡撤所部移駐海城。於是吳三桂、耿精忠相繼上章乞撤籓,上皆允其請,分遣朝臣料量籓兵移徙,具舟役芻糗,戶部尚書梁清標如廣東。十一月,三桂反,命罷撤平南、靖南二籓,召清標還。

十三年,精忠及定南王孔有德壻孫延齡反應三桂。三月,可喜疏言:「延齡檄並舉三籓,精忠復叛,臣與精忠為婚姻,不能不踧踖於中。臣叨忝王爵,年已七十餘,雖至愚豈肯向逆賊求富貴乎?惟知矢志捐軀,保固嶺南,以表臣始終之誠。」上溫旨嘉獎,並命與總督金光祖同心合力籌戰守。四月,潮州總兵劉進忠叛應三桂,可喜遣次子都統之孝帥師討之。疏言:「諸子中惟之孝端慎寬厚,可繼臣職。」上即命之孝襲王爵,之孝辭。可喜復疏言:「三桂遣兵二萬人屯黃沙河,若與延齡兵合,勢益猖獗,請遣將合軍進討。」上授副都統根特平寇將軍,自江西帥師赴廣東,與可喜合軍進討,並命兵部以各道進兵狀移告可喜。五月,上敕獎可喜忠貞,並諭與光祖等策討延齡。十月,可喜討平廣州土寇李三、官七。上命廣東督、撫、提、鎮俱聽可喜節制,遴補將吏,調遣兵馬,均得便宜從事。根特自長沙下廣西,卒於軍,上復授安親王岳樂為定遠平寇大將軍,率禁旅赴廣東。三桂、精忠方連兵寇江西,安親王師至,轉戰不能遽達。十二月,復命鎮南將軍尼雅翰率所部協守廣東。

十四年正月,進封可喜平南親王,以之孝襲爵,並授平南大將軍。廣東當寇沖,盜賊並起,博羅、河源、長寧、增城、從化諸縣先後告警,可喜輒分兵剿定。總兵張星耀等戰樂昌,俘斬千餘;副將李印香等戰碣石、白沙湖諸處,毀敵舟百餘:皆下部敘功。鄭錦自臺灣以兵攻海澄,進圍漳州,可喜疏聞,復請發重兵策應。尼雅翰亦言:「可喜年衰,臣才短,設有警,慮不支。」上命前鋒統領覺羅舒恕自江西帥師援廣東,旋代尼雅翰為鎮南將軍。

先是,之孝討進忠,復程鄉、大埔諸縣,遂克潮州。鄭錦遣其將劉國軒以萬人赴之,勢復張。之孝退保惠州,叛將祖澤清引延齡將馬雄、三桂將王宏勛等入高州,並陷雷、廉二郡。可喜疏言:「廣東十郡已失其四,將軍舒恕、總督金光祖退保肇慶,事勢危急,請敕安親王赴廣東辦賊。」上方責安親王定江西即下長沙取三桂,揚威大將軍簡親王喇布自江寧移師屯南昌,遂命簡親王發兵應可喜。師未至,十五年正月,錦攻陷漳州,三桂兵逼肇慶。可喜初請以長子之信襲爵,繼惡之信酗酒嗜殺,請更授次子之孝。之信陰通三桂,三桂兵日迫。之孝與進忠相持,上敕還廣州,不時至,二月,之信發兵圍可喜第,叛。可喜臥疾不能制,憤甚,自經,左右救之甦,疾益甚,十月卒。可喜疾亟,猶服太宗所賜朝衣,遺令葬海城。十六年,之信降,上敕部恤可喜,謚曰敬。及之信既誅,二十年五月,之孝乞迎可喜喪歸葬。九月,喪至,遣大臣覺羅塔達、學士庫勒納、侍衛敦柱至丁字沽親奠,諭曰:「王素矢忠貞,若人人盡能如王,天下安得有事?每念王懷誠事主,至老彌篤,朕甚悼焉!」可喜諸子,之信自有傳。

之孝初授可喜籓下都統,襲平南親王。授平南大將軍,帥師討劉進忠。上敕還廣州,未聞命,之信叛,脅之孝罷惠州軍,之孝還廣州侍可喜疾,及可喜卒,從之信居廣州。之信降,遣之孝還京師,上命以內大臣入直,秩視一品,食正一品俸。之孝請自效,授宣義將軍,駐南昌,募兵詣簡親王軍聽調遣,擊吳三桂軍吉安、贛州間,降其將林興隆、王國贊等;進次汀州,復擊破其將楊一豹、江機。江西定,召還京師,留所募兵編入綠旗營。之信誅,上貸之孝毋連坐,以內大臣入直如故。二十二年,奏乞守陵,議政大臣等劾削職。三十五年正月,卒。

之隆,可喜第七子。官至領侍衛內大臣。聖祖既誅之信,命有司還可喜海城田宅,置佐領二,以其一為可喜守墓,從之隆請也。

沈志祥,遼東人。毛文龍所部有沈世奎者,本市儈,倚女為文龍妾,橫行島中。累遷副總兵。及黃龍敗沒,明以世奎代龍為總兵官,鎮東江。時旅順已破,尚可喜又以廣鹿島降,世奎勢孤甚。後三年,太宗伐朝鮮,因移師克皮島,世奎戰敗,率舟師走,我師從之,副總兵金日觀戰死。登萊總兵陳洪範來援,不敢進,世奎亦戰死,志祥其從子也,時官副將,收潰兵保石城島,欲得世奎敕印,監軍者靳弗予,遂自稱總兵,明發兵討之。

崇德二年九月,太宗遣使齎書招志祥。三年二月,志祥遣所部將吳朝佐、金光裕詣盛京上疏請降,時上方出獵奎屯布喇克,留守諸王與宴,使貝勒杜度等轉粟迓志祥。志祥自黃石島至安山城,杜度等令駐沙河堡待命。從志祥降者,副將九、參將八、游擊十八、都司三十一、守備三十、千總四十、諸生二、軍民二千五百有奇。上獵還,命學士胡球、承政馬福塔等勞志祥,且令於鐵嶺、撫順自擇屯軍所。志祥言原駐撫順,畀以車騎,令率所攜軍民往。至,復為具屋宇,庀服物,俾得安處。七月,上聞志祥所攜軍民有亡去者,遣學士羅碩等諭其眾曰:「爾曹航海來歸,以朕能育爾曹也。朕不能育爾曹,任爾曹亡去未晚。爾曹初至,朕適出獵,故未及加恩,爾曹何去之速也!朕蒙天眷,朝鮮已平,蒙古、瓦爾喀諸部皆附,惟明僅存。倘天復垂佑,以明畀我,爾曹將安之?爾曹雖逃,為諸邊邏卒所得,不免於殺戮,朕心實所不忍。今後毋更逃,有貧不能自給者,朕為撫育之。」志祥入謁上,上御崇政殿受朝,授志祥總兵官,賚蟒衣,涼帽,玲瓏鞓帶,貂、猞猁猻、狐、豹裘各一襲,撒袋、弓、矢、雕鞍、甲、胄、駝、馬。初宴禮部,再宴宮中,命諸貝勒各與宴;及還鎮,遣官送五里外,復賜宴。四年正月,封續順公。九月,授志祥兄子永忠及所部許天寵等二十八人世職。

六年十月,命率所部助圍錦州。七年,師還,分賜俘獲。旋與孔有德等合疏請以所部屬烏真超哈,志祥隸正白旗。順治元年,從入關,逐李自成,至慶都。上至京師,賜志祥等貂蟒朝衣。十月,上御皇極門宴凱旋諸王大臣,志祥與焉,復賜鞍馬。三年,授孔有德平南大將軍,征湖廣,志祥率所部從。五年,湖南定,賜志祥黃金百、白金二千。尋卒,無子。

永忠,其兄子也,襲爵。五月,有德及耿仲明、尚可喜復分道出師征兩廣,亦命永忠率部將總兵官許天寵、郝效忠等徇湖南。六年,效忠遣參將馬如松將兵禦孫可望,戰於托口,俘其將李應元等。八年,天寵及阿達哈哈番張彥宏、護軍統領宋文科等擊敗明師,獲明將席世賢等一百七人,降牛萬才等二百五十六人、兵一萬八千有奇。可望等攻陷沅州,效忠遣守備吳進功等分屯要隘為備,復親將兵攻下黎平,屯四鄉所。可望詗我兵寡,驟以兵至,效忠力戰,馬蹶被執,不屈,死。效忠,遼東人。明副將,屬左良玉軍。良玉死,從其子夢庚來降,隸漢軍正白旗,授三等阿達哈哈番。至是,永忠以死事狀聞,上命予恤。

永忠退保湘潭,敕令激勵將士,相度險要,以同心並力,堅守疆土,毋輕戰,毋退縮。旋聞桂林陷,孔有德戰死,復敕令留屯寶慶,與總兵柯永盛合軍固守。十年二月,授永忠剿撫湖南將軍,鎮湖南。十一年,孫可望兵入湖南,沅、靖、武岡諸州皆陷,進攻辰、永。永忠還軍長沙。給事中魏裔介劾:「永忠手握重兵,望風宵遁,乞亟賜罷斥,毋俾誤及封疆。」十二年,議政王大臣議永忠喪師失地罪,當斬,來降有功,免死奪爵,上從其議。十七年,復命永忠為掛印將軍,鎮廣東。康熙初,命駐潮州。旋卒。

瑞,永忠子。方永忠之黜也,以從弟永興襲爵。永興卒,以瑞襲爵。時瑞方八歲,所部副都統鄧廣明駐潮州如故。十三年,潮州總兵官劉進忠叛應三桂,瑞部兵與巷戰三日,進忠引鄭錦兵入城,執瑞、廣明,驅將卒家屬二千餘人徙福建,置諸漳浦。十六年,復執瑞送臺灣。康親王傑書師定福建,疏言:「瑞所部及其孥無所統屬,應令有地得以總集。」上命副都統張夢吉、宋文科統其眾駐潮州,同將軍賴塔等協守,當給俸餉,令督餉侍郎達都視舊例從厚。夢吉等尋疏請送孥留京師,傑書又請以所部分隸督、撫、提、鎮,而處其孥於山西諸省。聖祖諭謂:「瑞及所部官兵素懷忠義,特以眾寡不敵,為賊所脅。」令駐潮州如故。

錦得瑞,爵以侯。瑞不原附錦,謀待我師至為內應。二十年十一月,錦將硃友以瑞謀告錦,錦遂幽瑞,瑞及妻鄭皆自殺,錦盡殺其拏。臺灣平,聖祖聞瑞死事狀,下廷臣議,求其族,以瑞從侄沈熊昭襲爵。

祖大壽,字復宇,遼東人。仕明為靖東營游擊。經略熊廷弼奏獎忠勤諸將,大壽與焉。天啟初,廣寧巡撫王化貞以為中軍游擊。廣寧破,大壽走覺華島。大學士孫承宗出督師,以大壽佐參將金冠守島。承宗用參政道袁崇煥議,城寧遠,令為高廣,大壽董其役。方竟,太祖師至,穴地而攻,大壽佐城守,發巨砲傷數百人。太祖攻不下,偏師略覺華島,斬冠,殪士卒萬餘。太宗即位,伐明,略寧遠,崇煥令大壽將精兵四千人繞出我師後,總兵滿桂、尤世威等以兵來赴,戰寧遠城下。會溽暑,我師移攻錦州,不克,遂引還。明人謂之寧錦大捷。

明莊烈帝立,用崇煥督師,擢大壽前鋒總兵,掛征遼前鋒將軍印,駐錦州。太宗嘗與大壽書,議遣使吊明熹宗之喪,且賀新君,大壽答書拒之。越二年,太宗伐明,薄明都。崇煥率大壽入衛,莊烈帝召見平臺,慰勞,令列營城東南拒戰。崇煥中太宗間,朝臣復論其「引敵脅和」,莊烈帝意移,復召入詰責,縛下獄。大壽在側股慄,懼並誅,出,又聞滿桂為武經略,統寧遠將卒,不肯受節制,遂帥所部東走,毀山海關出,遠近大震。莊烈帝取崇煥獄中書招之,孫承宗亦使撫慰,密令上章自列,請立功贖崇煥罪。大壽如其言,莊烈帝優旨答之。明年春,我師克永平等四城,太宗聞大壽族人居永平三十里村,命往收之,得大壽兄子一、子二及其戚屬,授宅居之,以兵監焉。師出塞,貝勒阿敏等護諸將戍四城。承宗令大壽與山西總兵馬世龍、山東總兵楊紹基會師率副將祖大樂、祖可法、張弘謨、劉天祿、曹恭誠、孟等攻灤州,灤州下,遂逼永平,阿敏等棄四城引兵還。大壽復駐錦州。

又明年七月,大壽督兵城大凌河。太宗策及其工未竟攻之,自將渡遼河,出廣寧大道,貝勒德格類等率偏師出義州。八月,師至城下,上曰:「攻城慮多傷士卒,不若為長圍困之。城兵出,我則與戰;援師至,我則迎擊。」乃分命諸貝勒諸將環城而軍:冷格裏當城北迤西,達爾哈當城北迤東,阿巴泰在其後;覺羅色勒當城正南,莽古爾泰、德格類在其後;篇古當城南迤西,濟爾哈朗在其後;武納格當城南迤東,喀克篤禮當城東迤北,多鐸在其後;伊爾登當城東迤南,多爾袞在其後;和碩圖當城西迤北,代善在其後;鄂本兌當城正西,葉臣當城西迤南,岳託在其後。諸蒙古貝勒各率所部彌其隙。佟養性率烏真超哈載砲跨錦州大道而營,諸將各就分地,周城為壕,深廣各丈許。壕外為墻,高丈許,施睥睨;距墻內五丈又為壕,廣五尺,深七尺五寸。營外又各為壕,深廣皆五尺。上陟城南岡,顧謂降將麻登雲、黑雲龍曰:「明善射精兵盡在此城。關內兵強弱,朕所素悉。」登雲對曰:「此城之兵,猶槍之有鋒,鋒挫柄存,亦復何濟?」上命射書城中,招蒙古兵出降。諸將攻撫城外諸臺堡,以次悉下;城兵出樵採,輒為我軍擒馘。圍合十餘日,上以書諭大壽,言原與明媾和,大壽置不報。

明援師自松山至,阿山、勞薩、圖魯什擊敗之;自錦州至,貝勒阿濟格等擊敗之。九月,遼東巡撫邱禾嘉,總兵官吳襄、鍾緯,合軍七千人赴援,上親率貝勒多鐸及圖魯什等以巴牙喇兵二百渡小凌河,乘銳擊破之。圍合已月餘,上度大壽必期援師至,出城兵夾攻,乃令廝卒去城十里所,發砲樹幟,驟馬揚塵,若為援兵自錦州至者,而親率巴牙喇兵入山為伏。大壽果以城兵出攻城西南隅臺,篇古、葉臣及蒙古諸貝勒督所部御戰,上親率巴牙喇兵自山上騰躍下。大壽知墜計,急收兵入城,死傷百餘人。自是閉城不復出。越數日,明監軍道張春及襄、緯等合馬步兵四萬來援,渡小凌河,為嚴陣徐進,上與貝勒代善等以二萬人御之。上率兩翼騎兵直入敵營,發矢射明軍。明軍發槍砲,上督騎兵縱橫馳突,矢雨集,明軍遂敗。襄先奔,佟養性屯敵營東發砲。黑雲起天際,風從西來,明軍縱火,勢甚熾,將逼我陣,忽驟雨,反風向明軍,明軍益亂。右翼兵入春營,逐北三十餘里,獲春及副將張弘謨、楊華徵、薛大湖,參將姜新等三十三人,斬副將張吉甫、滿庫、王之敬,襄等皆遁走。

十月,上復使招大壽,並命弘謨等各以己意為書勸降,大壽率將吏見使者城外,曰:「我寧死於此,不能降也!」上復與大壽書諭降,許以不殺。旋有王世龍者,越城出降,言城中糧竭,商賈諸雜役多死,存者人相食,馬斃殆盡。參將王景又以於子章臺降。我師克傍城諸堡,收糗糧,葺壕壘。大壽欲突圍,不得出。上復遣姜新招大壽,大壽見新於城外,遣游擊韓棟與新偕還,棟怵我師嚴整,歸以白大壽,大壽始決降。遂令其子可法出質,要石廷柱往議,上遣庫爾纏、龍什、寧完我與廷柱偕。廷柱度壕見大壽,大壽曰:「人安得不死?今不能忠於國,亦欲全身保妻子耳。我妻子在錦州,上將以何策俾我得與妻子相見耶?」上復令廷柱與達海往諭,即令大壽為計。大壽遣其中軍副將施大勇來,言降後欲率從者詐逃入錦州,伺隙以城獻。是時大凌河諸將皆原降,獨副將何可剛不從,大壽乃令掖以出城殺之。大壽使以誓書至,上率諸貝勒誓曰:「明朝總兵官祖大壽,副將劉天祿、張存仁、祖澤洪、祖澤潤、祖可法、曹恭誠、韓大勛、孫定遼、裴國珍、陳邦選、李雲、鄧長春、劉毓英、竇承武,參將游擊吳良輔、高光輝、劉士英、盛忠、祖澤遠、胡弘先、祖克勇、祖邦武、施大勇、夏得勝、李一忠、劉良臣、張可範、蕭永祚、韓棟、段學孔、張廉、吳泰成、方一元、塗應乾、陳變武、方獻可、劉武元、楊名世等,今以大凌河城降。凡此將吏兵民罔或誅夷,將吏兵民亦罔或詐虞。有違此盟,天必譴之!」誓畢,上使龍什告大壽,大壽即日出謁,上與語良久,定取錦州策,以御服黑狐帽、貂裘、金玲瓏鞓帶、緞鞾、雕鞍、白馬賜之。

次日,命貝勒阿巴泰等將四千人為漢裝,從大壽取錦州,會大霧,不果行。又次日為十一月朔,大壽以從子澤遠及從者二十六人入錦州,石廷柱、庫爾纏送之,夜渡小凌河,徒步去。上令大凌河將吏兵民薙發,斂軍中餘粟分賚之。方大凌河築城時,軍士、工役、商賈都三萬餘人,至是僅存萬一千六百八十二人,馬三十有二。後數日,大壽自錦州傳語諸裨將:「前日行倉猝,從者少。撫按防禦嚴,客軍眾,未得即舉事。」又遣使以告上,上報以書,誡毋忘前約。命隳大凌河城,引師還,至沈陽,命達海傳諭慰諸降將,大壽諸子孫賜宅以居,厚撫之。用貝勒岳託議,將以雪遼東、永平多殺謗也。

大壽初入錦州,詭言突圍出,遼東巡撫邱禾嘉知其納款狀,密聞於朝。莊烈帝欲羈縻之,因為用,置勿問;惟以蒙古將桑噶爾塞等赴援,戰不力,敗又先奔,令大壽誅之。桑噶爾塞等將執大壽來降,大壽與之盟乃定。莊烈帝召大壽入朝,使三至,辭不往。上自大凌河師還,略宣府,克旅順。居二年,遣阿山、譚泰、圖爾格先後徇錦州。又明年,上使貝勒多鐸帥師攻錦州,多鐸令阿山、石廷柱、圖賴、吳拜、郎球、察哈喇等以四百人前驅。大壽令副將劉應選、穆祿、吳三桂,參將桑噶爾塞、張國忠、王命世、支明顯將二千七百人出御,松山城守副將劉成功、趙國志率八百人來會。阿山等與遇大凌河西,多鐸引後軍自山下,塵起蔽天,應選等軍潰,殲五百人,獲游擊曹得功等,得馬二百餘、甲胄無算。多鐸旋引軍還。

又明年,改元崇德,行封賞,授澤潤三等昂邦章京,澤洪、可法一等梅勒章京,予世襲敕書。設都察院、六部,滿、漢、蒙古各置承政。漢承政皆授諸降將:可法、張存仁都察院,澤洪吏部,韓大勛戶部,姜新禮部,澤潤兵部,李云刑部,裴國珍工部。二年,更定部院官制,但置滿承政。諸降將改授左右參政,並以鄧長春代大勛,陳邦選代新。是時上北撫喀爾喀,南定朝鮮,敕大壽使密陳進兵策,大壽不報。

三年十月,上自將伐明,率鄭親王濟爾哈朗、豫親王多鐸出寧遠、錦州大道;睿親王多爾袞為左翼,自青山關入;貝勒岳託為右翼,自墻子嶺入。大壽方屯中後所,以兵襲多鐸,土默特之眾先奔,多鐸師敗績。次日,與濟爾哈朗合兵出,大壽斂兵不復戰。上親率師至中後所,使諭大壽曰:「自大凌河別後,今已數載。朕不憚辛苦而來,冀與將軍相見。至於去留,終不相強。曩則釋之,今乃誘而留之,何以取信於天下乎?將軍雖屢與我兵相角,為將固應爾,朕絕不以此介意。將軍勿自疑!」次日,又縱俘齎敕往曰:「曩大凌河釋汝,朕之諸臣每謂朕昧於知人。今將軍宜出城相見,若懷疑懼,朕與將軍可各將親信一二人於中途面語。朕欲相見者,蓋為朕解嘲,亦使將軍子侄及大凌河諸將吏謂將軍能踐言也。」大壽終不敢出。石廷柱、馬光遠、孔有德等攻克旁近諸臺堡,上乃命還師。左右翼深入,師大捷。

四年二月,上復自將伐明,以武英郡王阿濟格為前鋒,親督軍圍松山,分兵攻連山、塔山、杏山。明莊烈帝方召大壽入援,大壽甫行,我師至,乃還守寧遠。時澤遠守杏山,大壽遣部將三、兵九百自水道赴援,半入城。我噶布什賢兵躡其後,縱擊,得舟一,殺五十人。上遣使至錦州諭大壽妻,令以利害導大壽來降。大壽選蒙、漢兵各三百,授祖克勇及副將楊震、徐昌永等取道邊外趨錦州,至烏欣河口;阿爾薩蘭以滿、蒙兵一百六十戍焉,與戰,獲震,斬級八十四,得馬百五十。克勇等依山為寨,上親督巴牙喇兵破其寨,斬昌永,獲克勇,斬級三百十一,得馬四百十一。我兵攻松山,不克,會左右翼師還,上命罷攻還盛京。大壽復入錦州。是歲屢出師略錦州、寧遠、松山、杏山,皆未竟攻,得俘獲即引退。

五年三月,命鄭親王濟爾哈朗、貝勒多鐸率師屯田義州。五月,上幸義州視師,蒙古蘇班岱等牧杏山城西,使請降,上命濟爾哈朗等率巴牙喇兵千五百人往迓。大壽偵我師寡,令游擊戴明與松山總兵吳三桂、杏山總兵劉周智合兵七千人邀擊,濟爾哈朗引退以致敵,還擊,大敗之。上親閱錦州城,攻城東五里臺、城北晾馬臺,皆下,刈其禾而還。上命多爾袞、濟爾哈朗等將兵更番攻錦州。六年三月,濟爾哈朗令諸軍環城而營,大壽令蒙古守陴。邏卒至城下,蒙古兵自城上呼與語曰:「我城中積粟可支二三年,爾曹為長圍,豈遂足困我乎?」邏卒曰:「我師圍不解,自二三年至四五年,爾曹復何取食?」蒙古兵聞之皆懼。貝勒諾木齊等遂遣使約降,啟郭東門納我師。及期,大壽聞變,以兵出子城,蒙古兵與戰,我師逼城外,蒙古兵垂繩,援以登,吹角夾攻,大壽退保子城。我師入其郛,得裨將十餘及蒙、漢民男婦五千三百六十七人。明援兵自杏山至,濟爾哈朗為二伏,敗明兵,斬級一百七十,俘四千三百七十四人,得馬百十六、甲七十六。

五月,洪承疇督軍來援。六月,多爾袞番代。上遣學士羅碩以澤潤等書招大壽。七月,上自將破明師,降承疇。語見承疇傳。大壽弟總兵大樂,游擊大名、大成從承疇軍,被獲,上命釋大成,縱之入錦州。大壽使詣軍,言得見大樂,當降;既令相見,大壽再使請盟。濟爾哈朗怒曰:「城旦夕可下,安用盟為?」趣攻之。大壽乃遣澤遠及其中軍葛勛詣我師引罪。翌日,大壽率將吏出降,即日諸固山額真率兵入城,實崇德七年三月初八日也。上聞捷,使濟爾哈朗、多爾袞慰諭大壽,並令招杏山、塔山二城降,濟爾哈朗、多爾袞帥師駐焉。

阿濟格、阿達禮等以大壽等還,上御崇政殿召見,大壽謝死罪,上曰:「爾背我為爾主,為爾妻子宗族耳。朕嘗語內院諸臣,謂祖大壽必不能死,後且復降,然朕決不加誅。往事已畢,自後能竭力事朕則善矣。」又諭澤遠曰:「爾不復來歸,視大壽耳。曩朕蒞視杏山,爾明知為朕,而特舉砲,豈非背恩,爾舉砲能傷幾人耶?朕見人過,即為明言,不復省念。大壽且無責,爾復何誅?爾年方少壯,努力戰陣可已。」澤遠感激泣下。六月,烏真超哈分設八旗,以澤潤為正黃旗固山額真,可法、澤洪、國珍、澤遠為正黃、正紅、鑲藍、鑲白諸旗梅勒額真。大凌河諸降將初但領部院,至是始以典軍。大壽隸正黃旗,命仍為總兵,上遇之厚,賜賚優渥。存仁上言:「大壽悔盟負約,勢窮來歸。即欲生之,待以不殺足矣,勿宜復任使。」降將顧用極且謂其反覆,慮蹈大凌河故轍。上方欲寵大壽諷明諸邊將,使大壽書招明寧遠總兵吳三桂,三桂,大壽甥也,答書不從。大壽因疏請發兵取中後所,收三桂家族。

八年十月,濟爾哈朗帥師伐明,克中前所,並取前屯衛、中後所。明年,世祖定鼎京師,大壽從入關。子澤溥在明官左都督,至是亦降。十三年,大壽卒。

大壽初未有子,撫從子澤潤為後。其後舉三子,澤溥、澤洪、澤清。澤清叛應吳三桂,語見三桂傳。

澤潤初授三等昂邦章京。順治中,以從征叛將姜瓖,並遇恩詔,進一等精奇尼哈番又一拖沙喇哈番。從阿爾津帥師鎮湖南,卒於軍。乾隆初,定封二等子兼一雲騎尉。

澤溥初降,授一等侍衛。累遷福建總督。乞休,卒。

澤洪分隸鑲黃旗。順治元年,改參政為侍郎,澤洪仍任吏部。入關追擊李自成,斬其將陳永福;克太原,復擊敗叛將賀珍、姜瓖。敘功,並遇恩詔,累進一等精奇尼哈番,兼授內弘文院學士。以疾解任,卒。

子良璧,襲爵,授參領,兼佐領。從裕親王福全徵噶爾丹,擢西安副都統;復從撫遠大將軍費揚古出西路討噶爾丹,駐翁吉督餉。噶爾丹從子丹濟拉襲翁吉,良璧擊之,敗走。遷福州將軍,署福州總督。卒。乾隆初,定封一等男兼一雲騎尉。

可法,大壽養子。初質於我師。及降,授副將,隸正黃旗。順治初,從入關,擊走李自成,命以右都督充河南衛輝總兵。自成兵掠濟源、懷慶,總兵金玉和戰死,可法赴援力戰,自成兵乃引去。進都督,充鎮守湖廣總兵,駐武昌。以疾解任,卒,謚順僖。

澤遠,順治間,積功,並遇恩詔,授世職一等阿達哈哈番。累遷湖廣總督,加太子太保。京察左遷。尋卒。

論曰:有德、仲明,毛文龍部曲;可喜,東江偏將;志祥又文龍部曲之餘也。文龍不死,諸人者非明邊將之良歟?大壽大凌河既敗,錦州復守,相持至十年。明兵能力援,殘疆可盡守也。太宗撫有德等,恩紀周至,終收績效。其於大壽,不惟不加罪,並謂其「能久守者,讀書明理之效」。推誠以得人,節善以勵眾,其諸為興王之度也歟!


\end{pinyinscope}