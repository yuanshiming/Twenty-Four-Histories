\article{本紀十九}

\begin{pinyinscope}
宣宗本紀三

二十一年春正月己丑,英人寇廣東虎門,副將陳連陛及其子舉鵬死之。庚寅,以奕山為御前大臣。辛卯,琦善以虎門陷,下部嚴議,褫提督關天培頂戴。命奕山為靖逆將軍,隆文、楊芳為參贊大臣,督辦廣東海防。命賽尚阿在軍機大臣上行走。庚子,命訥爾經額駐天津,督辦海防。命哈哴阿赴山海關,督辦海防。命耆英等勤哨探。己巳,命伊里布回兩江總督任,以裕謙為欽差大臣,辦浙江軍務。辛亥,琦善褫大學士,仍下部嚴議。是月,賑奉天白旗堡水災旗戶。給江蘇江都、丹徒二縣水災倉穀,奉天小黑山站丁,江蘇廟灣場灶丁,安徽東流、繁昌二縣水旱災口糧。貸湖北沔陽等八州縣衛、湖南武陵縣、甘肅金州等五州縣水災籽種,江蘇上元等十一縣、甘肅皋蘭縣水災口糧,山西河曲縣雹災倉穀。

二月庚申,以伊里布遷延不進,下部嚴議。辛酉,琦善逮問,仍籍其家。以祁為兩廣總督,怡良兼署,李振祜署刑部尚書,授訥爾經額直隸總督,恩特亨額陜甘總督。丙寅,越南國王阮福皎卒,詔停貢方物。戊辰,英人去定海,以伊里布庸懦,褫協辦大學士,留兩江總督任。命寶興為大學士,仍留四川總督。以奕經協辦大學士。戊寅,命齊慎為參贊大臣,赴廣東會剿。壬午,英人陷廣東虎門砲臺及烏湧卡座,廣東水師提督關天培、署湖南提督祥福等死之。是月,展賑江蘇江寧、通州二府州災民。

三月丙戌朔,釋周天爵赴廣東軍營。甲午,上謁西陵,免經過地方額賦十分之三。乙未,致仕大學士文孚卒。丙申,英人兵船入廣東內港,楊芳等擊走之。戊戌,上謁泰陵、泰東陵、昌陵,至龍泉峪孝穆皇后、孝慎皇后、孝全皇后陵寢奠酒。己亥,上再謁昌陵,行敷土禮。詣隆恩殿行大饗禮。壬寅,上還京師。丙午,上臨故大學士文孚第賜奠。戊申,準米裏堅等國通商。庚戌,以裕謙奏,命沿海通商口岸照舊準商民貿易。壬子,楊芳等請仍準英國商船在廣東貿易。不許,命將楊芳、怡良嚴議。

閏三月乙卯朔,褫楊芳、怡良職,仍留任。丙寅,湯金釗降調,調卓秉恬為吏部尚書、協辦大學士,祁俊藻為戶部尚書,以許乃普為兵部尚書。丁卯,召伊里布來京,以裕謙為兩江總督,命定海防務交劉韻珂辦理。調梁章鉅為江蘇巡撫,以周之琦為廣西巡撫。乙亥,諭奕山等撫恤各國洋商。是月,貸山西吉州等十州縣暨和林格爾上年歉收倉穀。蠲緩江蘇宿遷縣被水灘租。

夏四月己丑,命裕謙仍為欽差大臣,督辦浙江海防。英人陷廣東城外砲臺。甲辰,禮部尚書奎照病免,以色克精額為禮部尚書。賜龍啟瑞等二百二人進士及第出身有差。辛亥,命睿親王等、大學士、軍機大臣、各部尚書會同刑部訊伊里布。癸丑,以廣東省城圍急,準奕山等奏,令英人通商。是月,緩徵山西朔州等六州縣逋賦。

五月丙辰,英船入浙洋,命裕謙申嚴各海口兵備。癸亥,鄧廷楨、林則徐遣戍伊犁。癸酉,英船去廣東虎門。穆彰阿免管理籓院,命賽尚阿代之。參贊大臣、戶部尚書隆文卒於軍。庚辰,調敬徵為戶部尚書,賽尚阿為工部尚書,恩桂為理籓院尚書。壬午,調吳文鎔為江西巡撫,錢寶琛為湖南巡撫。以奕興為綏遠城將軍。

六月,準祁等奏定商船赴天津等處章程。庚寅,褫伊里布職,發軍臺效力贖罪。準奕山等奏,分期撤兵。戊戌,琦善論斬。癸卯,河南下南河決。辛卯,褫文沖職,仍留河東河道總督任,牛鑒下部嚴議。

七月丙辰,命王鼎等赴東河督工。壬戌,以李振祜為刑部尚書。丁卯,以達賴喇嘛於四月坐床,頒敕書。戊辰,命前寧夏將軍特依順為參贊大臣,赴廣東。辛未,以河水汎濫,命牛鑒移民賑恤。己卯,南掌入貢。庚辰,英人陷福建廈門,總兵江繼蕓等死之。以故越南國王阮福皎子阮福暶為越南國王,命廣西按察使寶清往冊封。

八月癸未,以桂輪為熱河都統。丁亥,英人寇浙江。庚寅,以硃襄為河東河道總督。辛卯,萬壽節,上詣皇太后宮行禮。御正大光明殿,皇子及王以下文武大臣,蒙古使臣、外籓王公行慶賀禮。褫文沖職,枷號河干。以王鼎署河東河道總督。英人去廈門。丁酉,英人寇浙江雙澳、石浦等處,裕謙督兵擊走之。命怡良赴福建查辦軍務。以梁寶常署廣東巡撫。庚子,以趙炳言為湖北巡撫。辛丑,英人復大舉寇浙江。戊申,英人再陷定海,總兵王錫朋、鄭國鴻、葛云飛等死之。裕謙、餘步雲下部嚴議。是月,免陜西華州、大荔二州縣,河南睢州等八州縣水災額賦。

九月乙卯,英人陷鎮海,欽差大臣裕謙死之,提督餘步雲遁。命奕經為揚威將軍,哈哴阿、胡超為參贊大臣,督辦浙江海防。命怡良為欽差大臣,會同顏伯燾、劉鴻翱督辦浙江海防。以牛鑒署兩江總督,鄂順安署河南巡撫。丁巳,命文蔚為參贊大臣,赴浙江,胡超仍駐天津。命特依順為參贊大臣,赴浙江,哈哴阿仍駐山海關。命祁俊藻在軍機大臣上行走。授牛鑒兩江總督。辛酉,英人陷浙江寧波府。己巳,上閱火器營兵。是月,賑奉天遼陽等六州縣水災。

冬十月戊子,命僧格林沁等巡視天津海口。辛卯,英船入臺灣海口,達洪阿等擊退之。命王得祿赴臺灣協剿。是月,賑湖南華容縣、岳州衛,江西德化等十縣水災。加賑湖北沔陽等九州縣、山西薩拉齊災民、江蘇上元等十五縣衛災民,並免額賦。給安徽無為等十二州縣水災口糧屋費,並免額賦。

十一月庚午,以程矞採署江蘇巡撫。以青海玉樹番族雪災,免應徵銀二年。戊寅,英人陷浙江餘姚縣,復入慈谿。是月,賑江蘇上元、江寧二縣災民。

十二月戊子,褫顏伯燾職,以楊國楨為閩浙總督。己丑,以梁萼涵為山西巡撫。癸巳,英人陷浙江奉化縣。壬寅,湖北崇陽縣匪鍾人傑作亂,攻陷縣城,命裕泰等督兵討之。以程矞採為江蘇巡撫。丙午,英船寇浙江乍浦。戊申,英船寇臺灣淡水、雞籠,達洪阿等擊退之。是月,賑江蘇新陽縣災民。展賑河南祥符等六縣、江蘇上元等十縣災民。貸河南睢州、柘城縣貧民籽種口糧,並平糶淮寧縣倉穀。緩徵江西南昌等二十二縣逋賦,浙江橫浦、浦東二場灶課。

是歲,朝鮮、琉球、南掌入貢。

二十二年春正月丙辰,楊國楨病免,以怡良為閩浙總督,梁寶常為廣東巡撫。甲子,盛京將軍耆英改廣州將軍,以禧恩署之。己巳,湖北崇陽賊匪陷通山,裕泰遣兵擊敗之。丁丑,克復湖北崇陽縣,獲匪首鍾人傑。是月,賑安徽無為等十二州縣衛,奉天遼陽等六處、新民等四縣災民。給安徽泗州等二十二州縣衛、浙江海寧等七州縣水災口糧。貸江西德化等七縣、湖南武陵縣、湖北嘉魚等九縣衛、陜西葭州等五州縣水災籽種口糧,山西薩拉齊歉收倉穀,江蘇災區京右等營兵餉。蠲緩浙江海寧等九州縣衛水災新舊額賦。

二月丙戌,命林則徐仍戍伊犁。丙申,奕經等進攻寧波失利。釋伊里布赴浙江軍營。命耆英署杭州將軍。王鼎乞假。命齊慎仍為參贊大臣,辦理浙江軍務。丙午,命耆英為欽差大臣,會同特依順守浙江省城,並命劉韻珂會辦防務,責成奕經等守沿海各口岸。是月,賑盛京遼陽等處、江蘇上元等八縣災民。

三月壬子,上幸南苑。癸丑,上行圍,翼日如之。丁巳,上還圓明園。恩特亨額卒,以富呢揚阿為陜甘總督,壁昌為陜西巡撫。以慶昌為伊犁參贊大臣。是月,蠲緩河南鄭州積澇地畝逋賦。

夏四月癸未,英人復寇臺灣,達洪阿等擊走之。加達洪阿太子太保。己丑,英人去寧波府。甲午,上詣黑龍潭神祠祈雨。乙未,英人陷浙江乍浦,同知韋逢甲死之。庚子,褫餘步雲職逮問。丙午,鍾人傑伏誅。是月,貸湖南鳳凰等五縣屯丁苗佃籽種口糧,山西吉州等十四州縣倉穀。緩徵山西陽曲縣、薩拉齊逋穀。

五月己酉,大學士王鼎暴卒。丙辰,降湯金釗為光祿寺卿。丁巳,湯金釗乞休,允之。戊午,奕山以查奏不實,褫左都御史,並祁、梁寶常褫職留任。己未,禮部尚書色克精額卒,以恩桂代之。以吉倫泰為理籓院尚書。以奎照為左都御史。壬戌,英人陷江蘇寶山縣,提督陳化成死之。命耆英、伊里布赴江蘇,會同牛鑒防剿。丁卯,英人陷上海縣,典史楊慶恩死之。命賽尚阿為欽差大臣,會同訥爾經額防剿。是月,貸江蘇山陽縣及淮安等二縣衛歉收籽種。

六月戊寅朔,日食。蠲緩湖北被匪滋擾之崇陽等五縣衛新舊額賦。辛卯,以文慶為庫倫辦事大臣。壬辰,蠲緩浙江被擾之定海等十二縣新舊額賦。癸巳,英船寇京口。丙申,英船寇鎮江,齊慎等遁。丁酉,英人陷鎮江,副都統海齡死之。

秋七月甲寅,英船寇江寧省城。命伊里布等議款。命奕經進駐常州。己未,耆英奏與英兵官瑪禮遜等議罷兵。諭「朕以民命為重」,令妥行定議。癸亥,耆英等請與英兵官定約,鈐御寶。諭「朕因億萬生靈所系」,允所請。庚午,江南桃北河決。是月,賑巴里坤地震災。

八月戊寅,耆英奏廣州、福州、廈門、寧波、上海各海口,與英國定議通商。戊子,麟慶以貽誤河防,褫職留任。命敬徵、廖鴻荃赴江南查勘河工。是月,貸巴里坤地震災修屋費。

九月丁未,沈岐乞終養,允之。以李宗昉為左都御史。己酉,授禧恩盛京將軍。戊午,硃樹乞終養,允之。命周天爵以二品頂戴署漕運總督。己未,兩江總督牛鑒褫職逮問,命耆英代之。召奕山來京。以伊里布為欽差大臣、廣州將軍,辦理善後事宜。辛酉,河東河道總督硃襄卒,以慧成署之。癸亥,召奕經、文蔚來京。命齊慎回四川提督。甲戌,命伊里布議通商稅課事宜。乙亥,壁昌遷福州將軍,以李星沅為陜西巡撫。

冬十月庚辰,上閱圓明園八旗槍兵。丙戌,奕山、奕經、文蔚交刑部治罪,特依順、齊慎下部嚴議。庚寅,減免江蘇濱海被兵太倉等四十州縣衛新舊額賦有差。甲午,奕山、奕經、文蔚均奪職論斬,特依順、齊慎褫職留任。乙未,命戶部尚書敬徵協辦大學士,調恩桂為吏部尚書,以麟魁署禮部尚書。戊戌,慶郡王奕糸採緣事奪爵,不入八分輔國公釂性奪爵發盛京。是月,賑江蘇桃源、沭陽二縣水災。給湖北江陵等四縣、山西保德等三州縣災民口糧。貸奉天牛莊等處災民口糧。蠲緩江蘇海州等五州縣、湖南澧州等八州縣衛新舊額賦。

十一月丁未,召科布多參贊大臣固慶來京,以果勒明阿代之。召烏里雅蘇臺參贊大臣盛貴來京,以樂斌代之。召駐藏大臣孟保來京,以海樸代之。以潘錫恩為江南河道總督。授慧成河東河道總督。丙辰,允周天爵回籍守制,以廖鴻荃署漕運總督。甲子,命怡良查辦達洪阿等妄殺被難洋人。丁卯,牛鑒論斬。甲戌,給江蘇濱江被兵等丹徒六縣貧民口糧屋費,並免通州等十三州縣額賦有差。是月,給江蘇蕭縣、徐州衛水災口糧。蠲緩浙江淳安等三縣新舊額賦。

十二月辛巳,召廖鴻荃來京,以李湘棻署漕運總督。己丑,設通永鎮總兵,駐蘆臺,以向榮為通永鎮總兵。庚寅,召程矞採來京,以孫善寶為江蘇巡撫。乙未,托渾布病免,以程矞採為山東巡撫。戊戌,申命大學士、九卿、科道議餘步雲罪,處斬。己亥,調梁寶常為山東巡撫,程矞採為廣東巡撫。是月,給福建峰市等三縣水災口糧屋費。

是歲,廓爾喀、朝鮮、琉球來貢。

二十三年春正月辛亥,命李僡、成剛赴南河,會同潘錫恩督工。壬子,英兵官樸鼎查回香港,留馬禮遜等候議約。命伊里布等籌辦通商事宜。命李湘棻會同耆英籌辦江北善後事宜。是月,賑江蘇蕭縣、桃源縣災,並給沭陽等六縣衛口糧。貸湖北江陵等三縣衛,湖南澧州、洞庭二營水災籽種口糧。

二月乙未,欽差大臣、廣州將軍伊里布卒,命祁接辦通商稅則。丁酉,烏里雅蘇臺將軍奕湘改廣州將軍,以祿普代之。辛丑,調奕興為烏里雅蘇臺將軍,祿普為綏遠城將軍。是月,貸湖北荊州被水駐防倉穀。

三月庚戌,命耆英為欽差大臣,辦理江浙通商事宜。壁昌署兩江總督。丁巳,御試翰林、詹事等官,擢萬青藜五員為一等,餘升黜有差。乙丑,祿普遷鑲紅旗蒙古都統,調奕興為綏遠城將軍,以桂輪為烏里雅蘇臺將軍,起琦善為熱河都統。丙寅,起文蔚為古城領隊大臣。起奕經為葉爾羌幫辦大臣。丁卯,怡良奏達洪阿、姚瑩並無戰功,命褫職逮問。尋免達洪阿、姚瑩治罪。是月,貸山西絳州等六州縣、湖北荊州駐防被災倉穀,江寧駐防暨督協各營災歉兵丁銀,湖南鳳凰等五縣苗佃屯丁籽種口糧。

夏四月甲戌朔,以惟勤為烏魯木齊都統。丙子,授麟魁禮部尚書。丁丑,以御史陳慶鏞劾,仍奪琦善、文蔚、奕經職。奎照病免,以特登額為左都御史,薩迎阿為熱河都統。庚子,命耆英與英人會議通商。戊辰,怡良病免,以劉韻珂為閩浙總督,調吳其濬為浙江巡撫,以陸費瑔為湖南巡撫。

六月乙亥,湖南武岡州賊匪曾如炷作亂,戕知州徐光弼,命吳其濬討捕之。甲午,曾如炷伏誅。

秋七月乙巳,河決東河中牟九堡,慧成下部嚴議。允耆英奏,定通商稅則,先在廣州市易。改命敬徵、何汝霖赴東河查勘。丙午,命鄂順安賑沿河被水災民。

閏七月戊寅,直隸永定河決。乙酉,中牟決口未塞,命枷慧成河干。以鍾祥為河東河道總督。丙戌,召法豐阿來京,以德興為西寧辦事大臣。丁亥,命廖鴻荃往河南,會同督辦河工。己丑,起麟慶赴東河督辦河工。庚寅,命敬徵等議制紙鈔。甲午,調吳其濬為雲南巡撫,以管矞群為浙江巡撫。

八月乙巳,申諭程懋採撫恤安徽被水各州縣災民。是月,賑陜西沔縣等三縣水災雹災。

九月甲午,命李湘棻以三品頂戴署漕運總督。是月,賑山東福山縣水災。蠲緩直隸景州等二十七州縣、山東福山縣水災雹災正雜額賦。

冬十月己酉,耆英奏通商事竣,命回兩江總督任,辦善後及上海通商事宜,祁等辦粵省未盡事宜。庚戌,起琦善為駐藏辦事大臣。甲子,起達洪阿為哈密辦事大臣。是月,賑安徽太和等三縣、山西岢嵐州水災雹災。貸安徽太和等四縣、齊齊哈爾等四處歉收口糧。蠲緩奉天遼陽等六州縣、沈陽等三處,齊齊哈爾等四處,山東臨清等二十七州縣衛,安徽泗州等三十七州縣衛,山西岢嵐等七州縣,湖南澧州等六州縣衛被災新舊正雜額賦。

十一月己巳朔,日食。己卯,以王植為浙江巡撫。壬午,調程懋採為浙江巡撫,以王植為安徽巡撫。丁酉,上詣大高殿祈雪。是月,賑江蘇沭陽縣、大河衛災民。貸江西南昌等十五縣、陜西綏德等九州縣籽種口糧倉穀。蠲緩直隸新河等四縣、江蘇高郵等六十八州縣衛水旱災新舊額賦。

十二月辛丑,議定意大利亞國通商章程。甲辰,調梁寶常為浙江巡撫,以崇恩為山東巡撫。丙午,雪。丁巳,命劉韻珂辦寧波通商事宜。禮部尚書龔守正病免,以陳官俊代之。是月,蠲緩河南睢州等十六州縣被水新舊正雜額賦。

是歲,朝鮮、緬甸、暹羅入貢。

二十四年春正月辛卯,貸陜西葭州等四州縣、山西大同等三縣水災雹災籽種。

二月戊戌朔,祁病免,調耆英為兩廣總督,以壁昌署兩江總督。庚子,以謁東陵命肅親王敬敏等留京辦事。庚戌,以中牟壩工復蟄,褫麟魁、廖鴻荃職,給七品頂戴,仍留河工,鍾祥褫職,留東河總督任,鄂順安降三品頂戴。以特登額為禮部尚書,文慶為左都御史,調陳官俊為工部尚書,以李宗昉為禮部尚書,杜受田為左都御史。甲寅,命穆彰阿留京辦事。以程矞採奏米利堅使欲來天津朝覲,並議通商章程,命耆英赴廣東,會同程矞採妥辦米利堅等國通商事宜。丁卯,免經過地方田賦十分之三。是月,給江蘇海州等三州縣衛民屯口糧。

三月壬申,命耆英為欽差大臣,辦理通商善後事宜,仍令程矞採諭止米裏堅使來京。丙戌,鍾祥等奏河工善後事宜,諭:「一夫失所,罪在朕躬。卿等善為之。」是月,貸山西平定等十一州縣歉收倉穀。

夏四月己酉,修廣東虎門各內洋砲臺。壬子,臺灣匪平。辛酉,賜孫毓水桂等二百有九人進士及第出身有差。是月,加給河南睢州等十五州縣水災三月口糧。

六月丁酉,直隸永定河決。壬寅,湖南耒陽縣匪楊大鵬等作亂,命陸費瑔等討捕之。己酉,定米利堅通商條約。是月,緩徵山東臨清等二十二州縣並德州東昌二衛被災新舊額賦有差。

秋七月辛巳,富呢揚阿及提督周悅勝下部嚴議。甲申,湖南耒陽縣匪平,匪首楊大鵬伏誅。戊子,湖北荊州萬成堤決。辛卯,召奕興來京,以鐵麟署綏遠城將軍,阿彥泰署察哈爾都統。是月,加給河南中牟等九縣水災三月口糧。貸陜西葭州雹災籽種。

八月,賑山西汾陽縣水災雹災,並蠲緩汾陽等三縣額賦。

九月,給河南淮寧等三縣三月水災口糧。

冬十月甲午朔,準布魯特阿希木襲四品翎頂。己酉,葉爾羌參贊大臣奕經改伊犁領隊大臣,以麟魁代之。壬戌,伊犁參贊大臣達洪阿病免。命林則徐赴阿克蘇、烏什、庫車、和闐等處勘議開墾事宜。癸亥,以舒興阿為伊犁參贊大臣。是月,賑直隸霸州、永清二州縣旗民。給奉天錦州等八州縣水災口糧。蠲緩直隸霸州等三十七州縣、奉天金州等八州縣、湖北沔陽等二十九州縣衛水旱災雹災新舊額賦。

十一月乙丑,允桂良來覲,以吳其濬兼署云貴總督。前刑部侍郎黃爵滋以員外郎等官用。甲申,上詣大高殿祈雪。是月,貸盛京金州水師營歉收口糧。

十二月癸巳朔,上再詣大高殿祈雪。庚子,申命林則徐赴喀什噶爾查勘開荒。辛丑,上詣大高殿祈雪。命卓秉恬為大學士,以陳官俊為禮部尚書、協辦大學士,杜受田為工部尚書,祝慶蕃為左都御史。是月,加給河南睢州等十五州縣被災口糧,並貸籽種倉穀。貸江寧駐防兵丁、江蘇各營兵匠銀米。

是歲,朝鮮、暹羅入貢。

二十五年春正月乙丑,河南中牟河工合龍。庚午,調李星沅為江蘇巡撫,惠吉為陜西巡撫,以程矞採為漕運總督,黃恩彤為廣東巡撫。戊子,召容照來京,以麟慶為庫倫辦事大臣。是月,給直隸霸州、永清二州縣災民口糧。貸江西德化等五縣,湖北江陵等六縣衛,湖南沅江、安鄉二縣軍民籽種。庚戌,以福濟為總管內務府大臣。癸丑,睿親王仁壽坐濫保海樸,褫宗人府左宗正、領侍衛內大臣、內廷行走。敬徵坐濫保孟保,褫協辦大學士、戶部尚書。命兩廣總督耆英協辦大學士。調賽尚阿為戶部尚書,裕誠為工部尚書。以文慶為兵部尚書,成剛為左都御史。調僧格林沁為鑲黃旗領侍衛內大臣。以車登巴咱爾為正黃旗領侍衛內大臣。甲寅,調惠吉為福建巡撫,以鄧廷楨為陜西巡撫。乙丑,頒發五口通商章程。己巳,上閱圓明園八旗槍兵。癸未,麟慶病免,以成凱為庫倫辦事大臣。是月,貸山西忻州等十七州縣歉收倉穀。

夏四月癸卯,桂良留京,以賀長齡為雲貴總督。甲辰,調吳其濬為福建巡撫。惠吉為雲南巡撫,以喬用遷為貴州巡撫。丙午,上詣黑龍潭祈雨。壬子,富呢揚阿卒,以惠吉為陜西巡撫,鄧廷楨署之,以鄭祖琛為雲南巡撫。乙卯,賜蕭錦忠等二百十七人進士及第出身有差。丙辰,裕誠、許乃普降調,以敬徵為工部尚書,何汝霖為兵部尚書。

五月丙戌,雨。丁亥,上再詣黑龍潭祈雨。是月,給山東樂安等六縣水災口糧。

六月甲午,允比利時國通商。詔停本年秋決。丙申,命崇恩剿捕濮州、鄆城等處捻匪。辛丑,賑臺灣彰化縣地震災民。癸丑,阿克蘇辦事大臣輯瑞以墾荒未奏率即興工,褫職。己未,江蘇中河桃源汛河決。甘肅西寧鎮總兵慶和遇番賊於金羊嶺,死之。命惠吉剿捕番賊。是月,緩徵山東濱州等四十二州縣衛被災逋賦。

秋七月辛未,允丹麻爾國通商。命大學士卓秉恬管兵部。丙戌,命達洪阿赴甘肅查辦番賊。

八月壬辰,詔皇太后七旬萬壽,免道光二十年以前實欠正雜田賦。辛丑,調鄭祖琛為福建巡撫,梁萼涵為雲南巡撫,吳其濬為山西巡撫。敬徵病免,調特登額為工部尚書,以保昌為禮部尚書。丙戌,召林則徐回京,以四五品京堂候補。禧恩病免,調奕湘為盛京將軍。

冬十月甲午,加上皇太后徽號曰恭慈康裕安成莊惠壽禧崇祺皇太后。上進冊寶,率皇子及王、公、大臣等行慶賀禮。戊戌,皇太后七旬聖壽,上率皇子、王、公、大臣行慶賀禮。辛丑,李宗昉病免,以祝慶蕃為禮部尚書,魏元烺為左都御史。癸卯,以上皇太后徽號禮成,頒詔覃恩有差。丙午,免直隸道光二十年以前民欠各項旗租。是月,賑直隸寶坻等四縣災民。

十一月辛酉,陜甘總督惠吉卒,以布彥泰為陜甘總督,林則徐署之,薩迎阿為伊犁將軍,桂良為熱河都統。癸亥,御史陳慶鏞降調。是月,貸熱河圍場歉收兵丁銀。

十二月辛卯,上詣大高殿祈雪。戊戌,免臺灣道光二十年以前民欠租穀糧米。癸卯,上再詣大高殿祈雪。癸丑,上復詣大高殿祈雪。

是歲,朝鮮、越南入貢。

二十六年春正月庚辰,命賽尚阿、周祖培查勘江防。辛巳,弛天主教禁。以陸建瀛為雲南巡撫。是月,給奉天鳳凰城、岫巖旗民,直隸寶坻等四縣口糧。貸甘肅靜寧等十三縣災民籽種。

二月己丑,雲南永昌回匪作亂,命提督張必祿剿之。乙卯,以謁陵命定郡堊載銓等留京辦事。

三月癸亥,上謁西陵,免經過地方額賦十分之三。丁卯,上謁泰陵、泰東陵、昌陵、至孝穆皇后、孝慎皇后、孝全皇后陵奠酒。庚午,上幸南苑行圍。辛未,上行圍,翌日如之。乙亥,上還京師。興平倉火。乙酉,上詣黑龍潭祈雨。以林則徐為陜西巡撫。是月,貸山西平定等九州縣歉收倉穀。

夏四月辛丑,以雲南永昌回民藉端尋釁,命賀長齡查辦,丙午,上詣黑龍潭祈雨。庚戌,以瑞元為科布多參贊大臣。

五月壬戌,上詣黑龍潭祈雨。乙丑,張必祿敗回匪於永昌。以上年殺永昌內應回民過多,賀長齡下部議處。丁卯,上復詣黑龍潭祈雨。英人退出舟山。

閏五月乙酉朔,青海黑錯四溝番作亂,命布彥泰剿之。癸巳,永昌回匪遁入猛庭,賀長齡督兵剿之。戊申,以麟魁為烏里雅蘇臺參贊大臣。

六月戊午,命祁俊藻、文慶查辦天津鹽務。壬午,以予告大學士阮元重逢鄉舉,晉太傅,食全俸。癸未,達洪阿剿竄匿果岔番賊,敗之。

秋七月辛卯,禧恩以失察奸民,褫公爵,降鎮國將軍。壬寅,上閱吉林、黑龍江官兵馬步射。癸卯,以雲南漢、回積嫌未釋,命賀長齡持平辦理,勿分畛域。辛亥,申嚴門禁。是月,賑三姓及寧古塔等處水災。

八月壬申,命盛京、直隸、江南、浙江、福建、山東、廣東七省將軍、督、撫籌辦練兵儲餉。癸酉,上閱火器營兵。乙亥,賀長齡以防剿無功,降河南布政使。命李星沅為雲貴總督,調陸建瀛為江蘇巡撫,以張日晸為雲南巡撫。丙子,布魯特匪入喀什噶爾卡倫,命賽什雅勒泰剿之。

九月己亥,湖南新田縣匪王棕獻等作亂,捕誅之。戊申,以楊殿邦署漕運總督。辛亥,江蘇昭文縣匪金得順等作亂,捕誅之。是月,賑山東東平、萊蕪二州縣災民。賑三姓、琿春水災旗民。給山東汶上等四縣災民口糧。蠲緩奉天遼陽等十三州縣、直隸霸州等三十五州縣、山東東平等四州縣災歉新舊額賦。

十月丁巳,免黑錯四溝番民額賦。丙寅,以徐繼畬為廣西巡撫。是月,給河南汲縣等八縣,陜西府谷、神木二縣災民口糧。蠲緩湖南澧州等五州縣暨岳州衛被災額賦。

十一月乙酉,桂輪改荊州將軍,以特依順為烏里雅蘇臺將軍。乙未,上詣大高殿祈雪。丙午,命壁昌等籌議江蘇漕糧酌分海運。己酉,黃恩彤以奏請賜應試年老武生職銜,下部嚴議。辛亥,命山東嚴緝虜人勒贖匪。是月,賑山西垣曲縣災民。蠲緩山西保德等六州縣暨歸化城等三處、浙江餘杭等四十四縣衛、直隸安州等六州縣被災新舊額賦。

十二月癸丑,黃恩彤褫職,調徐廣縉為廣東巡撫,以程矞採為雲南巡撫,楊殿邦為漕運總督。癸亥,雲南猛統回匪竄入緬寧,命陸建瀛查辦。甲子,西寧辦事大臣達洪阿病免,以哈勒吉那代之。戊辰,以王兆琛為山西巡撫。庚午,命清釐刑部及直隸、山東、山西、河南、陜西、甘肅各省庶獄。命寶興留京管刑部。賞琦善二品頂戴,為四州總督。丙子,調鄭祖琛為廣西巡撫,徐繼畬為福建巡撫。是月,給浙江縉雲、宣平二縣水災口糧。

是歲,朝鮮、琉球入貢。

二十七年春正月癸未,調成凱為塔爾巴哈臺參贊大臣。乙酉,鐵麟遷荊州將軍,以裕誠為察哈爾都統。是月,給浙江富陽等六縣衛、安徽五河等三縣、江蘇桃源等五縣衛上年災歉口糧,河南河內等十三縣水旱災口糧籽種,並貸輝縣等八縣倉穀。貸陜西葭州等三州縣、直隸霸州等三十九州縣災歉籽種口糧倉穀。

二月己未,雲南雲州回匪作亂,命李星沅剿之。癸亥,以謁陵,命載銓等留京辦事。丙子,以福建海盜劫殺洋商,命劉韻珂等搜捕。戊寅,上謁東陵,免經過地方額賦十分之三。是月,給河南汲縣等五縣被災口糧。乙未,壁昌遷內大臣,調李星沅為兩江總督,以林則徐為雲貴總督,楊以增為陜西巡撫。戊戌,英船退出虎門。乙巳,以魏元烺為禮部尚書,賈楨為左都御史。

夏四月戊午,布魯特匪復攻色埒庫勒,伯克巴什等擊走之。賽什雅勒泰等奏英人據音底、努普爾,各部咸附之。丙寅,免熱河豐寧縣逋賦及旗租銀。癸酉,賜張之萬等二百三十一人進士及第出身有差。是月,貸江西上高、新昌二縣,湖南鳳凰等五縣屯丁苗佃籽種口糧。

五月丙戌,御試翰林、詹事等官,擢王慶雲四員為一等,餘升黜有差。何汝霖憂免,調魏元烺為兵部尚書,以賈楨為禮部尚書,孫瑞珍為左都御史。丁亥,命文慶、陳孚恩在軍機大臣上行走。辛卯,以廣東民情與洋人易啟釁端,命擇紳士襄辦交涉事宜。丁未,擢曾國籓為內閣學士。

六月,理籓院奏俄羅斯達喇嘛請在塔爾巴哈臺、伊犁、喀什噶爾通商,不許。

秋七月己卯,命林則徐讞雲南回民控訴香匪殺無辜一萬餘人之獄。乙未,命林則徐讞雲南回民杜文秀控訴被誣從逆之獄。癸卯,以河南旱災,發庫銀十萬兩,並撥鄰省銀二十萬兩賑之。

八月己酉,安集延匪犯喀什噶爾,吉明等擊走之。賽什雅勒泰自殺,調奕山為葉爾羌參贊大臣。癸亥,以布彥泰赴肅州調度,命楊以增署陜甘總督,恆春署陜西巡撫。甲子,以喀什噶爾卡外布魯特、安集延匪作亂,命布彥泰為定西將軍,奕山為參贊大臣,討之。以善燾為烏里雅蘇臺參贊大臣。以吉明署葉爾羌參贊大臣。戊辰,奕湘改杭州將軍,調奕興為盛京將軍,以英隆為綏遠城將軍。以河南災廣,再撥內帑銀三十萬兩,並命戶部撥銀三十萬兩賑之。丙子,安集延匪圍英吉沙爾城,命布彥泰駐肅州,遣兵討之。是月,賑甘肅西寧縣水災。緩徵山東樂安等六縣被水額賦,並永利等四場灶課。

九月丁丑朔,日食。戊寅,命文慶、張澧中赴河南查賑。辛巳,吉明等遣兵援喀什噶爾,擊安集延匪,大敗之。乙巳,以法蘭西兵船入朝鮮,命耆英言於法使,令其退兵。是月,給河南禹州等四十一州縣旱災口糧。蠲緩直隸安州等三十六州縣水旱災雹災新舊額賦。

冬十月辛酉,湖南新寧縣瑤人雷再浩等作亂,陸費瑔等捕討之。乙丑,上閱健銳營兵。戊辰,奕山等剿安集延匪於葉爾羌之科科熱依瓦特,大敗之。庚午,又敗之於英吉沙爾。壬申,安集延匪遁走。喀什噶爾辦事領隊大臣開明阿等褫職逮問。是月,蠲緩安徽泗州等三十九州縣水旱災新舊額賦。

十一月甲申,調英隆為黑龍江將軍,成玉為綏遠城將軍。壬辰,以張澧中為山東巡撫。乙未,湖南新寧賊平。庚子,湖南道州匪竄廣西灌陽縣,命鄭祖琛剿捕之。是月,給山西絳州等十一州縣口糧。蠲緩直隸安州等三州縣、山西絳州等十一州縣、河南禹州等六十四州縣被災新舊正雜額賦。

十二月戊午,湖南乾州苗匪作亂,命裕泰等剿捕之。甲戌,召耆英還,以徐廣縉署兩廣總督及欽差大臣,辦理通商。是月,給河南祥符等十七縣水災口糧,並貸鄭州等倉穀。

是歲,朝鮮、琉球來貢。

二十八年春正月丁丑,加潘世恩太傅,寶興太保,保昌、阿勒清阿、李振祜、成剛太子太保。甲申,湖南乾州苗匪降,命裕泰分別懲辦,仍搜餘匪。辛卯,命廓爾喀使附朝鮮、暹羅使筵宴。戊戌,越南國王阮福巿卒,停本年例貢。免喀什噶爾民、回各戶正雜逋賦。是月,展賑直隸鹽山等五縣災民。給安徽鳳陽等三縣水旱災口糧。貸湖南安鄉縣、山西寧遠等四縣、甘肅皋蘭等七縣災民口糧籽種。

二月壬子,吏部尚書恩桂卒,文慶罷軍機大臣,調為吏部尚書。以麟魁為禮部尚書,桂良改正白旗漢軍都統。以惠豐代為熱河都統,以保昌為兵部尚書。壬戌,江西長寧、崇義兩縣匪作亂,命吳文鎔剿捕之。甲子,以謁陵命睿親王仁壽等留京辦事。

三月戊寅,雲南趙州匪作亂,命林則徐剿捕之。以奕山為伊犁參贊大臣,吉明為葉爾羌參贊大臣。壬午,上謁西陵,免經過地方額賦十分之三。丙戌,上謁泰陵、泰東陵、昌陵,詣孝穆皇后、孝慎皇后、孝全皇后陵寢奠酒。庚寅,上還京師。癸卯,裕誠遷荊州將軍,以雙德為察哈爾都統。是月,貸山西吉州等七州縣歉收倉穀。

夏四月戊辰,雲南保山匪平。辛未,廣西灌陽、平樂、陽朔等縣匪平。

六月癸卯朔,以徐澤醇為山東巡撫。丙午,命耆英留京管禮部,授徐廣縉兩廣總督、欽差大臣,辦理通商。以葉名琛為廣東巡撫。癸丑,調耆英管兵部。甲寅,上詣黑龍潭祈雨。戊辰,以傅繩勛為江西巡撫。庚午,調吳文鎔為浙江巡撫。

秋七月庚寅,加林則徐太子太保,賞花翎。

八月丁巳,河南巡撫鄂順安褫職,以潘鐸代之。辛酉,俄羅斯商船請在上海貿易,卻之。

九月甲戌,潘錫恩免,以楊以增為江南河道總督,陳士枚為陜西巡撫。召成玉來京,以盛熏署綏遠城將軍。賑江寧等三府水災。乙酉,賑湖北水災。癸巳,召喬用遷來京,以羅繞典署貴州巡撫。是月,給湖南武陵等四縣水災口糧屋費。

冬十月甲寅,文華殿大學士寶興卒。丁卯,修巴爾楚克城。是月,賑直隸通州等七州縣、安徽無為等十六州縣水災。給安徽和州等十四州縣,湖南華容縣、岳州衛災民口糧。貸湖南安鄉縣、澧州災民籽種。蠲緩直隸通州等五十二州縣、湖北沔陽等三十九州縣衛、湖南澧州等九州縣、安徽泗州等二十四州縣被災新舊額賦。

十一月乙亥,封故越南國王阮福巿子福時為越南國王。己卯,命耆英為大學士,管兵部。以琦善為協辦大學士,仍留四川總督任。召瑞元來京,以慧成為科布多參贊大臣。御史張鴻升請鑄大錢,下部議。辛巳,命定郡王載銓、侍郎季芝昌查辦直隸鹽務,大學士耆英、侍郎硃鳳標查辦山東鹽務。丁亥,授耆英文淵閣大學士。丁酉,以托明阿為綏遠城將軍。是月,給江西德化等二十縣水災口糧。貸湖南提標及常德等協營災區兵餉。蠲緩江蘇泰州等七十七州縣衛、兩淮呂泗等二十場、江西德化等二十二縣、直隸安州等六州縣水災新舊額賦。

十二月丙午,上詣大高殿祈雪。甲寅,上詣大高殿祈雪。辛酉,上詣天壇祈雪。壬戌,以侍郎陳孚恩前署山東巡撫不收公費,賞一品頂戴,並御書扁額。乙丑,以倭什訥為吉林將軍,成剛為禮部尚書,柏葰為左都御史。丙寅,以張祥河為陜西巡撫。是月,賑直隸通州等十四州縣災民。

是歲,朝鮮、琉球、暹羅、越南入貢。

二十九年春正月癸未,以奕格為烏里雅蘇臺將軍。辛卯,命耆英、季芝昌查閱浙江營伍及倉庫。是月,加賑安徽無為等十四州縣衛水災。給湖南澧州等六州縣、安徽和州等十三州縣水災口糧。貸江西南昌等十二縣、湖南澧州等六州縣水災籽種。

二月庚子朔,日食。辛丑,命劉韻珂撫恤臺灣北路水災震災。丙午,諭李星沅辦江蘇賑務。辛亥,穆彰阿、潘世恩、陳官俊免上書房總師傅。命祈俊藻、杜受田為上書房總師傅,受田仍授皇四子讀。丙辰,四川中瞻對番工布朗結作亂,命琦善剿之。以裕誠兼署四川總督。是月,貸江蘇災區京左等八營一季兵餉。

三月庚寅,徐廣縉等奏,兵民互相保衛,內河外海,現飭嚴防,英人進省城一事,萬不可行。諭嘉納之。

夏四月壬寅,李星沅病免,以陸建瀛為兩江總督,調傅繩勛為江蘇巡撫,以費開綬為江西巡撫。丙午,陸建瀛等奏南漕毋庸改折,從之。丁未,徐廣縉奏英人罷議進城。封徐廣縉子爵、葉名琛男爵,均一等世襲。諭嘉獎粵人深明大義。

閏四月辛未,以顏以燠署河東河道總督。癸酉,調趙炳言為湖南巡撫,以羅繞典為湖北巡撫。辛巳,琦善剿中瞻對番,敗之。壬午,以德齡為葉爾羌參贊大臣。

五月乙巳,移廣東澳門稅口於黃埔。己酉,雲南騰越野夷作亂,林則徐討平之。己未,山西巡撫王兆琛以受賕褫職逮問,以季芝昌為山西巡撫。是月,貸山東滕縣雹災倉穀。

六月丙子,廣東陽山、英德等縣匪平。己丑,禮部尚書成剛卒。庚寅,調毓書為烏魯木齊都統,以惟勤為熱河都統。

秋七月丙申朔,福建閩縣匪林仕等作亂,捕誅之。戊戌,協辦大學士、吏部尚書陳官俊卒,調賈楨為吏部尚書。以孫瑞珍為禮部尚書,王廣廕為左都御史。以馮德馨為湖南巡撫。己亥,命祁俊藻協辦大學士。辛亥,命湖南布政使萬貢珍賑武陵等縣被水災民。丙辰,王兆琛遣戍新疆。己未,林則徐病免,以程矞採為雲貴總督,張日晸為雲南巡撫。降侍郎戴熙三品頂戴休致。是月,給江西德化等五縣、湖南澧州等九州縣衛水災口糧。蠲緩江蘇川沙等二十二縣新舊額賦。

八月丁丑,陸建瀛奏辦賑及水退情形。諭:「臣民之福,即朕之福。」丙戌,召季芝昌來京,以龔裕署山西巡撫。是月,給奉天錦州旗民、江西鄱陽等九縣、湖南澧州等十州縣衛水災口糧。

九月甲辰,布彥泰病免,以琦善署陜甘總督,裕誠署四川總督。丙午,授顏以燠河東河道總督。戊申,命署吏部右侍郎季芝昌在軍機大臣上行走。己酉,授琦善陜甘總督。以徐澤醇為四川總督,陳慶偕為山東巡撫。癸丑,雲南保山界外小宇江等處野夷作亂,程矞採剿平之。戊午,命服闋尚書何汝霖仍在軍機大臣上行走。是月,給貴州桐梓縣水災口糧,並蠲緩額賦。

冬十月庚午,以故朝鮮國王李怳子昪襲爵,命瑞常、和色本往冊封。甲申,大學士潘世恩請開缺,命免軍機大臣。庚寅,以賡福署熱河都統。是月,給湖南澧州等七州縣、山西徐溝縣被災口糧。蠲緩直隸薊州等三十七州縣、浙江富陽等二十一縣、山西薩拉齊等三縣被災新舊額賦。

十一月甲午朔,湖南新寧縣匪李沅發作亂,命馮德馨剿之。丙申,太傅、予告大學士阮元卒。甲辰,調龔裕為湖北巡撫,以兆那蘇圖為山西巡撫。乙巳,阿哥所火。庚戌,臺灣嘉義縣匪徒吳吮等作亂,捕誅之。是月,賑江西德化等十四縣水災。給齊齊哈爾等六城旗民、浙江仁和等八場灶丁口糧。蠲緩江蘇泰州等七十三州縣衛、江西德化等二十一縣被災新舊額賦,浙江海沙等十四場灶課。

十二月庚午,湖南道州匪黃三等作亂,命裕泰剿之。以扎拉芬泰為塔爾巴哈臺參贊大臣。辛未,皇太后不豫,上詣慈寧宮問安,自是每日如之。甲戌,皇太后崩。乙亥,奉安大行皇太后梓宮於慈寧宮。上居倚廬,席地寢苫。諸王大臣請還宮,不允。甲申,移皇太后梓宮於綺春園迎暉殿。自是上居慎德堂苫次。乙酉,李振祜病免,以陳孚恩為刑部尚書。丁亥,湖南新寧賊分竄廣西,鄭祖琛遣兵防剿。

是歲,朝鮮、琉球、越南入貢。

三十年春正月甲午朔,日食。丙申,以祁俊藻等查覆陜甘總督布彥泰清查關防不密,下部嚴議。丁酉,以王大臣再請停止親送大行皇太后梓宮,諭從之。戊戌,上大行皇太后尊謚曰孝和恭慈康豫安成熙聖睿皇后。庚子,上詣迎暉殿孝和睿皇后梓宮前行大祭禮。甲辰,上詣梓宮前行周月祭禮。乙巳,尊孝和睿皇后陵曰昌西陵。

丙午,上不豫。丁未,上疾大漸。召宗人府宗令載銓,御前大臣載垣、端華、僧格林沁,軍機大臣穆彰阿、賽尚阿、何汝霖、陳孚恩、季芝昌,總管內務府大臣文慶公啟鐍匣,宣示御書「皇四子立為皇太子」。是日,上崩於圓明園慎德堂苫次。硃諭「封皇六子奕為親王」。

四月甲戌,上尊謚曰效天符運立中體正至文聖武智勇仁慈儉勤孝敏成皇帝,廟號宣宗。咸豐二年二月壬子,葬慕陵。

論曰:宣宗恭儉之德,寬仁之量,守成之令闢也。遠人貿易,構釁興戎。其視前代戎狄之患,蓋不侔矣。當事大臣先之以操切,繼之以畏葸,遂遺宵旰之憂。所謂有君而無臣,能將順而不能匡救。國步之瀕,肇端於此。嗚呼,悕矣!


\end{pinyinscope}