\article{列傳三百}

\begin{pinyinscope}
土司二

○四川

四川邊境寥廓,歷代多設土司以相控制。明末,張獻忠屠蜀,石砫、酉陽、松潘、建昌等土司距險禦賊,其地獨全。清初,戡定川境,各土司次第效順。川之南有金川者,本明金川寺演化禪師哈伊拉木之後,分為大小金川。順治七年,小金川卜兒吉細歸誠,授原職。吳三桂亂後,康熙五年,其酋嘉納巴復來歸,給演化禪師印。其庶孫莎羅奔,以土舍將兵從將軍岳鍾琪征西藏羊峒番,雍正元年,奏授安撫司,居大金川;而舊土司澤旺居小金川,莎羅奔以女阿扣妻澤旺。澤旺懦,為妻所制。乾隆十一年,莎羅奔劫澤旺去,奪其印。十二年,又攻革布什札及明正兩土司。

朝廷調張廣泗總督四川,進駐澤旺所居美諾官寨,而以其弟良爾吉從征。時莎羅奔居勒烏圍,其兄子郎卡居噶爾厓,地在大金川河東,而河西亦有地數百里。廣泗調兵三萬,一路出川西攻河東,一路出川南攻河西。而河東一路又分為四,以兩路攻勒烏圍,以兩路攻噶爾厓,河西亦分兩路,攻庚特額諸山,刻期蕆事。阻險不前,上命大學士公訥親往視師,起岳鍾琪於廢籍。鍾琪與廣泗議定,自任由黨壩取勒烏圍,而廣泗由昔嶺取噶爾厓。會訥親至,下令限三日克噶爾厓,總兵任舉、參將買國良戰死。廣泗輕訥親不知兵,而惡其凌己,故飾推讓,實以困之,軍中解體。良爾吉夙與阿扣通,莎羅奔因使成配,倚作間諜,官軍動息輒為所備。師久無功,上怒甚,會訥親劾廣泗,於是逮廣泗入京,而命大學士傅恆為經略,代訥親。冬,殺廣泗,賜訥親死。十二月,傅恆至軍,斬良爾吉、王秋、阿扣以斷內應。

十四年春正月,奏言:「金川之事,臣到軍以來,始知本末。當紀山進討之始,惟馬良柱轉戰直前,逾沃日,收小金川,直抵丹噶,其鋒甚銳。其時張廣泗若速濟師策應,乘賊守備未周,殄滅尚易;乃坐失機會,宋宗璋逗留於雜穀,許應虎失機於的郊,致賊得盡據險要,增碉備御,七路、十路之兵無一路得進。及訥親至軍,嚴切催戰,任舉敗歿,銳挫氣索,晏起偷安,一以軍務委張廣泗。廣泗又聽奸人所愚,惟恃以卡逼卡、以碉逼碉之法,槍砲惟及堅壁,於賊無傷,而賊不過數人,從暗擊明,槍不虛發,是我惟攻石,而賊實攻人。且於碉外開壕,兵不能越,而賊得伏其中自下擊上。又戰碉銳立,高於中土之塔,建造甚巧,數日可成,隨缺隨補,頃刻立就。且人心堅固,至死不移,碉盡碎而不去,砲方過而人起,主客勞佚,形勢迥殊,攻一碉難於克一城。即臣所駐卡撒山頂,已有三百餘碉,計半月旬日得一碉,非數年不能盡。且得一碉輒傷數十百人,較唐人之攻石峰堡,尤為得不償失。惟有使賊失其所恃,而我兵乃得展其所長。臣擬俟大兵齊集,別選銳師,旁探間道,裹糧直入,逾碉勿攻,繞出其後,即以圍碉之兵作為護餉之兵。番眾無多,外備既密,內守必虛。我兵即從捷徑搗入,則守碉之番各懷內顧,人無固志,均可不攻自潰。至於奮勇固仗滿兵,而鄉導必用土兵,土兵中小金川尤驍勇。今良爾吉之奸諜已誅,澤旺與賊仇甚切,驅策用之,自可得力。至沃日、瓦寺兵強而少,雜棱、綽斯甲等兵眾而懦。明正、木坪忠順有餘,強悍不足。革什乍兵銳,可當一路。是各土司環攻分地之說雖不可恃,而未嘗不可資其兵力。臣決計深入,不與爭碉,惟俟四面布置,出其不意,直搗巢穴,取其渠魁,定於四月間報捷。」上屢奉皇太后息武寧邊之諭,命傅恆班師。時傅恆及鍾琪兩路連克碉卡,軍聲大振,莎羅奔乞降於鍾琪,鍾琪輕騎徑赴其巢,賊大感動,頂佛經立誓聽約束。次日,鍾琪率莎羅奔父子坐皮船出洞詣大軍,莎羅奔等叩顙,誓遵六事,歸各土司侵地,獻兇酋,納軍械,歸兵民,供徭役。乃宣詔赦其死。諸番焚香作樂,獻金佛謝。二月,捷聞,詔賞傅恆、鍾琪等。

既而莎羅奔兄子郎卡主土司事,漸桀驁。二十三年,逐澤旺及革布什札土司。三十一年,詔四川總督阿爾泰檄九土司環攻之。九土司者,巴旺、丹壩、沃日、瓦寺、綽斯甲布、明正、木坪、革布什乍及小金川也。巴旺、丹壩皆彈丸,非金川敵。明正、瓦寺形勢阻隔,其力足制金川。而地相逼者,莫如綽斯甲布與小金川。阿爾泰不知離其黨與,反聽兩金川釋仇締約,自是狼狽為奸,諸小土司咸不敢抗。時澤旺老病不問事,郎卡亦旋死,其子索諾木與僧格桑侵鄂克什土司地。

三十六年,索諾木誘殺革布什札土官,而僧格桑再攻鄂克什及明正土司,與官軍戰。上以前此出師,本以救小金川。今小金川反悖逆,罪不赦。賜阿爾泰死,命大學士溫福自雲南赴四川,以尚書桂林為四川總督,共討賊。溫福由汶川出西路,桂林由打箭爐出南路。僧格桑求援於索諾木,索諾木潛兵助之。三十七年春,桂林克復革布什札土司故地,溫福克資里及阿喀。朝廷以阿桂為參贊大臣,代桂林赴南路。十一月,阿桂以皮船宵濟,連奪險隘,直搗賊巢。十二月,軍抵美諾,進至底木達,俘澤旺,檄索諾木縛獻僧格桑,不應。

上命溫福為定邊將軍,阿桂、豐伸額為副將軍。溫福、阿桂奏六路進兵之策。溫福由功噶入,阿桂由當噶入,豐伸額由綽斯甲布入。三十八年春,溫福以賊扼險不得進,別取道攻昔嶺,駐營木果木,令提督董天分屯底木達,守小金川之地。溫福為人剛愎,不廣諮方略,惟襲廣泗故智,以碉卡逼碉卡,建築千計。初索諾木欲並小金川地,故留僧格桑挾以號召。六月,陰遣小金川頭目等由美諾溝出煽故降番使復叛。諸番見大軍久頓,蜂起應之,攻陷天營,遂劫糧臺,潛兵襲木果木,奪砲局,斷汲道,賊四面蹂入大營,溫福死之,將士隨員死者數十人,各卡兵望風潰。海蘭察聞警赴援,殿眾由間道退出,收集潰卒,尚萬數千人,其戰歿者三千餘,小金川地復陷。惟阿桂一軍屹然不動,乃整隊出屯翁古爾壟。

上在熱河聞報,召大學士劉統勛詣行在咨之。統勛前言金川不必勞師,至是亦主用兵。乃授阿桂定西將軍,豐伸額、明亮為副將軍。十月,阿桂改赴西路,明亮赴南路。豐伸額仍由綽斯甲布進取宜喜,阿桂入自鄂克什,轉戰五晝夜,直抵美諾,克之;明亮入自瑪爾里,所向皆捷,遂盡復小金川地。

惟大金川自十二三年以來,全力抗守,增壘設險,嚴密十倍小金川。七月,令諸軍分攻各碉寨,數十道並進。海蘭察率死士六百削壁猱引而上,趾頂相接,比明及其碉,一湧入,盡殲守賊。數十里賊寨聞之皆奪氣,悉破之,乘勝臨遜克宗壘。索諾木酖殺僧格桑而獻其尸,及其妻妾頭目,至軍乞赦己罪。阿桂檻送京師。四十年四月,阿桂先使福康安、海蘭察赴河西助明亮攻宜喜,遂分兵六路,盡滅河西二十里內之賊。五月,阿桂河東之軍破朗噶寨,距勒烏圍僅數里,環營進逼其巢。七月,抵勒烏圍。八月十五夜,進搗賊巢,四面砲轟官寨,破之。黎明,克轉經樓,逸賊皆溺水死。莎羅奔兄弟及各頭目已先期遁往噶爾崖。十一月,官軍攻克科布曲山。十二月,遂據瑪爾古山,噶爾崖即在其下。索諾木之母姑姊妹亦降。官軍三路合圍噶爾崖,斷其水道。索諾木使其兄詣營乞哀,不允。圍攻益急,索諾木從莎羅奔及其妻子挈番眾二千餘出寨,奉印獻軍門降,金川平。四十一年正月,獻俘廟社,封賞阿桂等,勒碑太學,並及兩金川。旋於大金川設阿爾古,小金川設美諾。四十四年,並阿爾古入美諾,改為懋功。

同治二年,粵匪石達開竄寧遠,假道工⼙部土司。土司先受官軍約束,引賊至紫打地。四面阻絕,達開糧罄路窮,射書千戶王應元買路,復使人說土司嶺承恩求緩兵,皆不應,日殺馬煮桑葉為食。四月,承恩、應元等偵賊力竭,率夷眾蹙攻,擒達開並賊官五人付官軍,檻送成都,四川總督駱秉章誅之。奏加承恩、應元二品銜,賊軍錙重悉為兩土司所得。

初,瞻對土司恃強不法,雍正八年,四川提督黃廷桂剿降之。乾隆十年,四川提督李質粹等率兵五千,取道東俄落,至里塘進兵,連破番寨,獲賊首噶籠丹坪。十一年,質粹會欽差大臣班第,統兵進克泥日寨,燒斃番酋姜錯太,撫定丫魯、下密等處番夷。嘉慶十九年,中瞻對土司洛布七力劫掠鄰番,抗捕傷兵。二十年,四川總督常明、提督多隆武領兵剿之,恃險死拒。重慶鎮總兵羅思舉力戰破其巢,洛布七力焚死,分其地入上下瞻對。

洎咸豐中,土司工布朗結為人沉鷙,兼並上下瞻對之地,欲擁康部全境以抗川拒藏,鄰近各土司割地求免,貢賦唯命。至是藏人怒,求四川出兵,秉章派道員史致康率師會藏進討。致康怯,頓打箭爐久,藏番需茶急,馳兵克之,殺工布朗結父子,致康始逡巡至。藏人索兵費銀十六萬兩,秉章未允,藏人因據其地,設官兵駐守。

光緒初,丁寶楨為四川總督,以瞻對藏官虐民,往往激變,每歲派員帶兵出關彈壓。劉秉璋繼之,稍寬縱,藏官益驕橫。各土司多被威脅,唯明正土司地大,不之服,頻年爭鬥。十五年,瞻對內訌,逐藏官,乞內附,秉璋不許,唯治番官及亂民數人罪,由藏易官,且添駐堪布一人,兵八百名助守。二十年,硃窩、章穀土司爭襲滋事,瞻對番官率兵越境干預,開槍傷我官兵。四川總督鹿傳霖奏瞻對為蜀門戶,宜設法收回內屬,派提督周萬順、知縣張繼率兵出關,擊敗番兵,不三月,克瞻對並德爾格忒即疊蓋。舊名保蓋。

全境,擒德格土司夫婦,解至成都,議敘改設流官。成都將軍恭壽憤傳霖不先會商,結駐藏大臣文海,密奏劾傳霖,翻原案,復德格土司職,仍以瞻對屬藏。

三十一年春,駐藏大臣鳳全被戕於巴塘,四川總督錫良奏請以四川提督馬維騏、建昌道趙爾豐進討。維騏率師先發。先是泰凝寺產沙金,錫良準商人採辦,★派兵彈壓。寺中喇嘛反抗,殺都司盧鳴颺,瞻對潛助其亂,維騏出關討平之。六月,攻克巴塘,擒正土司羅進寶、副土司郭宗隆保,誅之,移其妻子於成都安置。八月,爾豐至,殺堪布喇嘛及首惡數人祭鳳全。維騏班師回,爾豐接辦善後,派兵剿倡亂之七村溝,並搜捕餘匪,因移師討鄉城。次年閏四月,克之,並攻克稻壩、貢噶嶺,一律肅清。於是爾豐建籌邊議,錫良以聞。朝廷特設督辦川滇邊務大臣,授爾豐。邊地在川、滇、甘、藏、青海間,縱橫各四五千里,土司居十之五,餘地歸呼圖克圖者十之一,清代賞藏者十之一,流為野番者十之三。爾豐改巴塘、里塘地設治,以所部防軍五營分駐之。回川會商,錫良派道員趙淵出關坐鎮。

三十三年,爾豐護理四川總督,奏準部撥開邊費銀一百萬兩。三十四年,授爾豐駐藏辦事大臣,仍兼邊務大臣,募西軍三營,率之出關。時德格土司爭襲,構亂久,爾豐奏請往辦,經泰凝、道塢、章谷、倬倭、麻書、孔撒、白利、糸戎壩、擦玉龍、濯拉、擴洛垛以至更慶。十二月,攻逆酋昂翁降白仁青等於贈科,匪竄雜渠卡。宣統元年四月,攻雜渠卡。五月,戰於麻木。六月,追匪十日程至卡納,一戰肅清,改流其地,並改春科、高日兩土司地及靈蔥土司之郎吉嶺村歸流。十月,四川兵入藏,藏番扼察木多以西地阻之,劫糧擄官。爾豐率邊軍渡金沙江,逾雪山,抵察木多,送川兵行,於是三十九族、波密、八宿均請附邊轄。三十九族者:曰夥爾,曰圖嘛魯,曰吉寧塔克,曰尼牙木查,曰松嘛巴,曰勒達克,曰多嘛巴,曰達爾羊巴,曰他瑪,曰夥兒,曰拉寒,他瑪、夥兒、拉寒三族共一土司。

曰夥耳,曰瓊布噶,曰瓊布色爾查,曰瓊布納克魯,曰扎瑪爾,曰上阿扎,曰下阿扎,曰上奪爾樹,曰下奪爾樹,曰上剛噶爾,曰下剛噶爾,曰他瑪爾,曰提瑪爾,曰枳多,曰哇拉,枳多、哇拉二族共一土司。

曰麻弄,曰布川目桑,曰書達格魯克,曰奔盆,曰策令畢魯,曰色爾查,曰納布貢巴,曰結拉克汁,曰拉巴,曰三渣,曰樸樸,皆自為部落。設土總百戶或土百戶、土百長等以治之,歸駐藏大臣管轄。爾豐以其族素恭順,悉加慰遣;因派兵剿類伍齊、碩搬多、洛隆宗、邊壩等阻路之番人,又分兵取江卡、貢覺、桑昂、雜瑜,咸收服之。

二年,邊軍直抵江達,爾豐奏請以江達為邊藏分界。五月,邊軍返察木多。六月,爾豐率兵略乍丫地。八月,巡阿足返,設乍丫委員。聞定鄉兵變,派統領鳳山追剿。九月,三巖野番投書索戰,爾豐率兵赴貢覺。十月,派傅嵩矞攻三巖,一旬而克。十一月,設三巖委員。十二月,設貢覺委員。爾豐旋返巴塘。三年二月,爾豐以巴塘所屬之得榮浪藏寺數年不服,派兵攻克之,設得榮委員,並收服冷卡石。三月,爾豐調任四川總督,四川布政使王人文繼之為邊務大臣。爾豐奏請人文未到任前,以嵩矞代理。四月,同發巴塘,至孔撒、麻書,設甘孜委員,檄靈蔥、白利、倬倭、東科、單東、魚科、明正、魚通各土司繳印,改土歸流。色達及上羅科野番來歸。適駐藏大臣聯豫電請邊軍攻波密,因奏派副都統鳳山率兵二千往應。六月,爾豐至瞻對,藏官逃,收其地,設瞻對委員。旋經道塢、打箭爐,檄魚通、卓斯各土司繳印改流。爾豐入川,沿途收咱里、冷邊、沈邊三土司印,嵩矞復出關改流泰凝,而魚科土司結下羅科抗命。嵩矞令上羅科扼其險,擊平之,斃魚科土司,於是嵩矞奏請設西康省,而沃日、崇喜、納奪、革伯咱、巴底、巴旺、靈蔥、上納奪各土司,暨乍丫、察木多兩呼圖克圖,相繼繳印。惟毛丫、曲登乞緩,許之。

涼山夷惈儸者,居寧遠、越巂、峨邊、雷波、馬邊間,淺山部落頭目屬於土司。深入則涼山,數百里皆夷地。生夷黑骨頭為貴種,白骨頭者曰熟夷,執賤役。夷族分數百支,不相統屬。叛則出掠,擄漢民作奴,遇兵散匿。清興,雍正五年、七年,嘉慶十三年、十六年,迭經川吏剿撫,加以部勒。

同治末,越巂夷叛,成都將軍崇實兼攝四川總督,奏調貴州提督周達武率軍由陜回剿,前鋒羅應旒出清溪,撫大樹堡、左右王嶺各土司,進駐保安,攻降洽馬里、阿波落、跑馬坪、燕麥廠,遂克普雄石城,夷地四百里間咸受約束。官軍至靖遠,刷茲、林加、布約、尼錢、交腳等支亦降,更設靖遠新老兩營土千百戶,出漢奴數萬。迨爾豐經營關外,朝廷以其兄爾巽督川,爾巽欲悉平涼山夷以利邊務,光緒三十四年八月,派建昌鎮總兵鳳山、建昌道馬汝驥等,率兵暨民團剿寧遠吉狄、馬加、拉斯等支惈夷。進至裹足山梁,旋值國喪,罷兵。

宣統元年正月,令建昌鎮總兵田鎮邦、寧遠府知府陳廷緒再舉,征服淺山白母子、嗎噠拉施、三合等支,並收撫爭咱雞租、五支、別牛、租租等支,於是加拉及吉狄、馬加等支先後降。官軍進駐交腳,收撫八切、阿什並阿落、馬家、上三支、下三支,野夷悉請內附,不隸土司。先是馬邊夷阿侯蘇噶支戕英教士,拒捕,與馬邊協副將楊景昌軍相持。爾巽調總兵董南斌往剿,與寧遠軍夾擊,阿侯蘇噶降。兩軍於十月二十五日貫通涼山夷巢,會於吽吽壩。於是爾巽議禁黑夷蓄奴。先就交腳設縣治,餘地擇要屯守;而西南由美姑河至雷波,闢雷寧通道四百餘里,駐兵守護,以通商旅。是役也,得地幾及千里,夷眾凡十餘萬人。二年,振邦、廷緒等師還討會理土司,披砂、會理村、苦竹、者保、通安舟等悉改流,至是川境土司多非舊觀矣。今採傳世較永者著於篇。其國初歸附未久旋即絕滅者,尚不勝記云。

成綿龍茂道松潘鎮轄:

拈佐阿革寨土百戶,系西番種類。其先個個柘,康熙四十二年,歸附,授職。

熱霧寨土百戶,系西番種類。其先甲槓他,康熙四十二年,歸附,授職。

瓘眉喜寨土千戶,系惈夷種類。其先官布笑,雍正四年,歸附,授職。

毛革阿按寨土千戶,系惈夷種類。其先王乍,雍正四年,歸附,授職。

包子寺寨土千戶,系西番種類。其先噶竹,康熙四十二年,歸附,授職。以上松潘中營屬。

阿思峒寨土千戶,系西番種類。其先立架,順治十五年,歸附,授職。

羊峒寨土百戶,系西番種類。其先甲利,雍正二年,歸附,授職,由四川總督給以土百戶委牌一張。以上松潘左營屬。

下泥巴寨土百戶,系西番種類。其先林青,康熙四十二年,歸附,授職,由四川總督給以土百戶委牌一張。松潘右營屬。

寒盻寨土千戶,系西番種類。其先占巴笑,康熙四十二年,歸附,授職。

商巴寨土千戶,系西番種類。其先剛讓笑,康熙四十二年,歸附,授職。

祈命寨土千戶,系西番種類。其先龍盻架,康熙四十二年,歸附,授職。

羊峒踏藏寨土目,系西番種類。其先甲六笑,康熙四十二年,歸附,授土目。

阿按寨土目,系西番種類。其先六笑他,康熙四十二年,歸附,授土目。

挖藥寨土目,系西番種類。其先旦折笑,康熙四十二年,歸附,授土目。

押頓寨土目,系西番種類。其先拈爭笑,康熙四十二年,歸附,授土目。

中岔寨土目,系西番種類。其先捏盻目,康熙四十二年,歸附,授土目。

郎寨土目,系西番種類。其先郎那亞,康熙四十二年,歸附,授土目。

竹自寨土目,系西番種類。其先札布吉,康熙四十二年,歸附,授土目。

臧咱寨土目,系西番種類。其先出亞,康熙四十二年,歸附,授土目。

東拜王亞寨土目,系西番種類。其先點進笑,康熙四十二年,歸附,授土目。

達弄惡壩寨土目,系西番種類。其先達喇笑,康熙四十二年,歸附,授土目。

香咱寨土目,系西番種類。其先轄六,康熙四十二年,歸附,授土目。

咨馬寨土目,系西番種類。其先由仲笑,康熙四十二年,歸附,授土目。

八頓寨土目,系西番種類。其先革甲,康熙四十二年,歸附,授土目。

上包坐餘灣寨土千戶,系西番種類。其先札卜盻,康熙四十二年,歸附,授職。

下包坐竹當寨土千戶,系西番種類。其先本布笑,康熙四十二年,歸附,授職。

川柘寨土千戶,系西番種類。其先桑仲,康熙四十二年,歸附,授職。

穀爾壩那浪寨土千戶,系西番種類。其先郎借,康熙四十二年,歸附,授職。

雙則紅凹寨土千戶,系西番種類。其先郎那笑,康熙四十二年,歸附,授職。

以上各土司,皆頒有號紙。

上撒路木路惡寨土百戶,系西番種類。其先學賴,雍正二年,歸附,授職。

中撒路木路惡寨土百戶,系西番種類。其先隆笑,雍正二年,歸附,授職。

下撒路竹弄寨土百戶,系西番種類。其先迫帶,雍正二年,歸附,授職。

崇路穀謨寨土百戶,系西番種類。其先札務革柱,雍正二年,歸附,授職。

作路生納寨土百戶,系西番種類。其先郎刀,雍正二年,歸附,授職。

上勒凹貢按寨土百戶,系西番種類。其先借勒,雍正二年,歸附,授職。

下勒凹卜頓寨土百戶,系西番種類。其先林革秀,雍正二年,歸附,授職。

以上各土司,皆頒有印信號紙。

班佑寨土千戶,系西番種類。其先獨足笑,雍正元年,歸附,授職。

巴細蛇住壩寨土百戶,系西番種類。其先連柱笑,雍正元年,歸附,授職。

阿細柘弄寨土百戶,系西番種類。其先哈惰,雍正元年,歸附,授職。

上作爾革寨土百戶,系西番種類。其先轄頓,雍正元年,歸附,授職。

合壩奪雜寨土百戶,系西番種類。其先穀六笑,雍正元年,歸附,授職。

轄漫寨土百戶,系西番種類。其先額旺,雍正元年,歸附,授職。

下作革寨土百戶,系西番種類。其先郎納他,雍正元年,歸附,授職。

物藏寨土百戶,系西番種類。其先郎加蚌,雍正元年,歸附,授職。

熱當寨土百戶,系西番種類。其先拆戎架,雍正元年,歸附,授職。

磨下寨土百戶,系西番種類。其先的那,雍正元年,歸附,授職。

甲凹寨土百戶,系西番種類。其先革柯,雍正元年,歸附,授職。

阿革寨土百戶,系西番種類。其先甲亞,雍正元年,歸附,授職。

鵲個寨土百戶,系西番種類。其先羅六,雍正元年,歸附,授職。

郎惰寨土百戶,系西番種類。其先阿出,雍正元年,歸附,授職。

上阿壩甲多寨土千戶,系西番種類。其先拆達架,雍正元年,歸附,授職。

中阿壩墨倉寨土千戶,系西番種類。其先革杜亞,雍正元年,歸附,授職。

下阿壩阿強寨土千戶,系西番種類。其先頓壩,雍正元年,歸附,授職。

上郭羅克車木塘寨土百戶,系西番種類。其先噶頓,康熙六十年,歸附,授職。

中郭羅克插落寨土千戶,系西番種類。其先丹增,康熙六十年,歸附,授職。

下郭羅克納卡寨土百戶,系西番種類。其先彭錯,康熙六十年,歸附,授職。

上阿樹銀達寨土百戶,系西番種類。其先卜架亞,康熙六十年,歸附,授職。

中阿樹宗個寨土千戶,系西番種類。其先卜他,康熙六十年,歸附,授職。

下阿樹郎達寨土百戶,系西番種類。其先郎加劄舍,康熙六十年,歸附,授職。

小阿樹寨土百戶,系西番種類。其先達爾吉,康熙六十年,歸附,授職。以上松潘漳臘營屬。

丟骨寨土千戶,系西番種類。其先沙乍謨,康熙四十二年,歸附,授職。

雲昌寺寨土千戶,系西番種類。其先革都判,康熙四十二年,歸附,授職。

呷竹寺土千戶,系惈夷種類。其先七谷,康熙四十二年,歸附,授職。以上松潘平番營屬。

以上各土司,皆頒有號紙。

中羊峒隆康寨首,系西番種類。其先林柱,雍正二年,歸附,委以寨首。咸豐十一年,歐利娃作亂,陷南坪營,同治四年,周達武率武字、果毅各軍討平之。

下羊峒黑角郎寨首,系西番種類。其先六孝,雍正二年,歸附,委以寨首。

以上各土司,皆無印信號紙。以上松潘南坪營屬。

大姓寨土百戶,其先鬱氏,於唐時頒給左都督職銜印信,管束番眾。順治六年,鬱孟賢將唐時印信呈繳。

小姓寨土百戶,其先鬱從文,於明末歸附,授長官司職銜印信,管束番★。順治年間,將明時印信號紙呈繳。

大定沙壩土千戶,其先蘇忠,於明萬歷年間歸附,授土千戶職銜印信,管束番眾。順治年間,將明時印信號紙呈繳。

以上各土司,皆頒號紙。

大黑水寨土百戶,其先鬱孟賢,於明末歸附,授土百戶職銜,管束各番。順治年間,將明時號紙呈繳。

小黑水寨土百戶,其先於唐時歸附,授土百戶職銜印信,管束各番。順治年間,鬱從學將唐時印信呈繳。

以上各土司,皆給委牌。

松坪寨土百戶,其先韓騰,於明末歸附,授土百戶職銜印信,管束番眾。順治年間,將明末印信號紙呈繳,仍頒給號紙。以上茂州疊溪營屬。

靜州長官司,其先董正伯,自唐時歸附,授職。順治年間,賊屠茂州,土司董懷德率土兵捍禦,地方寧謐。九年,董應詔歸附。

隴木長官司,其先何文貴,於宋時剿羅打鼓生番有功,授職與印。順治九年,歸附。

岳希長官司,其先坤蒲,自唐時有功授職。康熙九年,歸附。

沙壩安撫司,其先蟒答兒,自明時剿黑水三溪生番有功授職。順治九年,歸附。

水草坪巡檢土司,其先蟒答兒次子住水草坪,授巡檢職。順治九年,歸附。

竹木坎副巡檢土司,其先坤兒布,自明時授職。順治九年,歸附。

牟托巡檢土司,其先燦沙,自唐時授職。順治九年,歸附。

以上各土司,皆頒印信號紙。

實大關副長官司,其先官之保,自明時授職。康熙十年,歸附,頒給號紙。以上茂州茂州營屬。

陽地隘口土長官司,始祖王行儉,由宋寧宗朝授龍州判官,世襲。傳三世,改守御千戶。元至正間,授宣御副使。明洪武七年,開龍州,改長官司。順治六年,王⼎各歸附,仍授原職,頒給印信號紙。

土通判,明洪武七年授王思恭為長官司,以王思民襲判官,旋授宣撫僉事。嘉靖間,改土通判。順治六年,王啟睿歸附,仍授原職,頒給號紙,無印信。

龍溪堡土知事,宋景定間,授薛嚴龍州知州,世襲。明隆慶間,改土知事。順治六年,薛兆選歸附,仍授原職,頒給號紙。以上龍安府龍安營屬。

瓦寺宣慰司,先世雍中羅洛思,與兄桑郎納思壩,前明納貢土物。正統六年,威茂、孟董、九子、黑虎等寨諸番跳梁,雍中羅洛思、桑郎納思壩奉調出藏,帶兵出力,即留住汶川縣塗禹山,給宣慰司印信號紙。順治九年,土司曲翊伸歸附,授安撫司。康熙五十九年,征西藏,土司桑郎溫愷隨征有功,加宣慰司銜。乾隆二年,加指揮使職銜。乾隆十七年及三十六年,徵剿雜穀土司蒼旺並金川等處,土司桑郎雍中隨征出力,賞戴花翎。嘉慶元年,隨征達州教匪,經四川總督勒保奏升宣慰司,換給印信號紙。

以上理番維州協左營屬。

梭磨宣慰使司,始祖囊素沙甲布,原系雜穀土目,自唐時歸附。雍正元年,徵剿郭克賊番有功,頒給副長官司印信號紙。乾隆十五年,換給安撫司印。三十六年,進剿大小金川,土司隨征,經將軍阿桂奏賞宣慰司職銜並花翎,換給印信號紙。

卓克基長官司,其祖良爾吉,系雜穀土舍。乾隆十三年,隨征大金川有功。十五年,頒給長官司印信號紙,尋以通匪伏誅。

松岡長官司,其祖系雜穀土目,自唐時安設。康熙二十二年,頒給安撫司印信號紙。乾隆十七年,土司蒼旺不法,伏誅。

黨壩長官司,其曾祖阿丕,系雜穀土舍。乾隆十三年,土舍澤旺隨征大金川有功,頒給長官司印信號紙。嘉慶元年,土司更噶斯丹增姜初隨征苗匪,賞花翎。

以上理番維州協右營屬。

成綿龍茂道提標轄:

沃日安撫司,始祖巴碧太。順治十五年,歸附,頒發沃日貫頂凈慈妙智國師印信號紙。乾隆二十年,頒給土司色達拉安撫司印信號紙,隨將舊印呈繳。二十九年,隨征金川有功,賞二品頂戴花翎。沃日地名更為鄂克什,原系維州協所轄。乾隆五十年,改隸懋功協管轄。宣統三年,改流。

綽斯甲布宣撫司,綽斯甲布印文曰「卓斯甲布」。卓斯,地名。甲者,家之誤。番人稱謂如德格則曰「德格家」,孔撒則曰「孔撒家」。布者,番人男子之稱。印以「綽斯甲布」為名,誤矣。

始祖資立,康熙三十九年,歸附。四十一年,頒給安撫司印信號紙。乾隆三十七年,出師金川,賞二品頂戴花翎。四十一年,頒給宣撫司印信號紙,隨將舊印呈繳。原系阜和協所轄。乾隆五十一年,改隸懋功協管轄。宣統三年,改流。

以上懋功懋功協屬。

建昌道建昌鎮轄:

河東長官司,其先自元迄明,世襲建昌宣慰司。順治十六年,安泰寧歸附,呈繳明印。雍正六年,改授長官司。管有大石頭、長村、繼事田三土百戶,利扼、上芍果、阿史、紐姑、上沈渣、下芍果、上熱水、小涼山、慕西、又利呃、阿史、者加十二土目。

阿都正長官司,其先結固,順治六年,歸附,授職。康熙四十九年,土司慕枝為招撫案內,授阿都宣撫司,頒給印信號紙。雍正六年,改土歸流。是年,涼山野夷不法,土司聚姑擒獻兇首,復授阿都正長官司。管有歪歪溪、咱古、喬山南、大河西四土目。

副長官司,雍正六年,剿撫涼山夷眾,歸附有功,授阿都副長官司。管有小涼山馬希、大梁山拖覺、阿乃、又阿史、結呃、派乃、者膩、那科、那俄、哈乃過、又阿驢十一土目。

沙罵宣撫司,其先安韋威,康熙四十九年,歸附,授職。管有那多、扼烏、咱烈山、撒凹溝、結覺五土目。以上西昌縣中營屬。

昌州長官司,其先盧尼古,明洪武九年,調守德昌、昌州,康熙四十九年,歸附,承襲。

普濟州長官司,其先吉三嘉,明洪武七年,授普濟州土知州。康熙四十九年,歸附,承襲,改長官司。

威龍州長官司,其先張起朝,明洪武七年,授職。順治十六年,歸附,世襲。以上西昌縣左營屬。

河西宣慰司,其先安吉茂,康熙五十一年,歸附。五十七年,吉茂歿,無子,嶺氏撫伊兄越巂土司嶺安泰之子為子,更名安祥茂。雍正六年,改土歸流,換給土千總職銜,世襲。管有囉慕、芍果、咱堡、沙溝四土目。

以上西昌縣右營屬。

以上各土司,皆頒印信號紙。

工部宣撫司,其先嶺安盤,康熙四十三年,歸附,授職。同治二年,土司嶺承恩助官軍擒石達開有功,賞二品銜。管有膩乃、阿谷、蘇呷、咱戶、慕虐、阿蘇、濫田壩、普雄、黑保、大疏山十土目。

以上越巂越巂營屬。

暖帶密土千戶,其先嶺安泰,康熙四十九年,歸附,授職。管有上官、六革、瓜惈、糾米、布布、阿多六磨、磨卡為呷、西糾七鄉總。

暖帶田壩土千戶,其先部則,唐熙四十四年,歸附,授職。

松林地土千戶,其先王德洽,康熙四十九年,歸附,授職。管有老鴉漩、白石村、六翁、野豬塘、前後山、料林坪六土百戶。以上越巂寧越營屬。

以上各土司,皆頒印信號紙。

木里安撫司,其先六藏塗都,雍正七年,歸附。

瓜別安撫司,系麼夷人。其先玉珠迫,康熙四十九年,歸附。

馬喇副長官司,系夷人。其先阿世忠,康熙十九年,歸附,頒給號紙。

古柏樹土千戶,系麼夷人。其先郎俊位,康熙四十九年,歸附。管有阿撒、祿馬六槽兩土目。

中所土千戶,系麼夷人。其先喇瑞麟,康熙四十九年,歸附。

左所土千戶,系麼夷人。其先喇世英,康熙四十九年,歸附。管有蓽苴蘆土目。

右所土千戶,系麼夷人。其先八璽,康熙四十九年,歸附。

前所土百戶,系麼夷人。其先阿成福,康熙四十九年,歸附。

後所土百戶,系麼夷人。其先白馬塔,康熙四十九年,歸附。以上鹽源縣會鹽營屬。

以上各土司,皆頒印信號紙。

酥州土千戶,其先姜喳。康熙四十九年,歸附,授職。

架州土百戶,其先里五,康熙四十九年,歸附,授職。

苗出土百戶,其先熱即巴,康熙四十九年,歸附,授職。

大村土百戶,其先也四噶,康熙四十九年,歸附,授職。

糯白瓦土百戶,其先紐吽,康熙四十九年,歸附,授職。

大鹽井土百戶,其先前布汪喳,康熙四十九年,歸附,授職。

熱即哇土百戶,其先牙卓撇,康熙四十九年,歸附,授職。

中村土百戶,其先歪即噶,康熙四十九年,歸附,授職。

三大枝土百戶,其先甲噶,康熙四十九年,歸附,授職。

河西土百戶,其先那姑,康熙四十九年,歸附,授職。以上冕寧縣冕山營屬。

窩卜土百戶,其先藍布甲噶,康熙四十九年,歸附,授職。

虛郎土百戶,其先濟布,康熙四十九年,歸附,授職。

白路土百戶,其先倪姑,康熙四十九年,歸附,授職。

阿得轎土百戶,其先募庚,康熙四十九年,歸附,授職。

瓦都土目,其先安承裔,康熙四十九年,歸附,授職。

木術凹土目,其先那咱,康熙四十九年,歸附,授職。

瓦尾土目,其先瀘沽,康熙四十九年,歸附,授職。

瓦都木、術凹、瓦尾三土司,皆於雍正五年,因徵三渡水刾俊違誤運糧參革,其部落戶口仍設土目管束。以上冕寧縣靖遠營屬。

七兒堡土目,原設土司,康熙四十九年,歸附,授職。雍正五年,降土目,管有耳挖溝土目。冕寧縣瀘寧營屬。

以上各土司,皆頒印信號紙。

黎溪舟土千戶,其先自必仁,康熙四十九年,歸附,授職。

迷易土千戶,其先安文,康熙四十九年,歸附,授職。

以上各土司,皆頒有印信號紙。

會理村土千戶,其先祿沙克,康熙三十二年,歸附,授職,頒給號紙。

者保土百戶,其先祿阿格,康熙四年,歸附,無印信號紙。

普隆土百戶,其先汪玉,康熙四十九年,歸附,承襲。

紅卜苴土百戶,其先刁氏,康熙四十九年,歸附,承襲。

以上各土司,皆頒有印信號紙。

苦竹壩土百戶,其先祿姐,康熙三十七年,歸附,承襲,頒給印信號紙。其通安舟土百戶另給鈐記。以上會理州會川營屬。

披砂土千戶,其先祿應麟,康熙四十九年,歸附,頒給號紙。會理州永定營屬。

祿氏五土司,傳二百餘年。宣統初,祿紹武死,無後,妻自氏據其遺產,祿、自兩姓群起爭襲,作亂。二年,趙爾巽派兵剿捕,先後擒逆首祿禎祥、嚴如松等,因移師討爐鐵梁子侯夷,悉平之。披砂、會理村、苦竹、者保、通安舟五土司地一律收回,改流設治。

建昌道提標轄:

天全六番招討司高躋泰,順治九年,歸附。副司楊先柱同。均於雍正六年追繳印信號紙,以其地為天全州。

穆坪董卜韓瑚宣慰使司,其先於明世襲土職。至康熙元年,堅參喃喀歸附,仍授原職,請領宣慰司印信。乾隆十年,頒給號紙。天全州黎雅營屬。

黎州土百戶,漢馬岱後。其先馬芍德,於明洪武八年世襲安撫司。萬歷十九年,馬祥無子,妻瞿氏掌司事,與祥侄構釁,降千戶。順治九年,馬高歸附,仍授原職。乾隆十七年,改百戶。

大田副土百戶,乾隆十七年,因防曲曲鳥,奏請添設副土百戶一員,世襲。

松坪土千戶,其先馬比必,康熙四十三年,歸附,授職。以上清溪縣黎雅營屬。

以上各土司,皆頒有印信號紙。

沈邊長官司,原籍江西吉水縣。其先余錫伯,前明從征來川,授土千戶。順治九年,餘期拔歸附,改名永忠。宣統三年,改流。

冷邊長官司,西番瓦布人。其先阿撒,順治元年,歸附。傳至周至德,於康熙六十年授職。宣統三年,改流。以上打箭爐泰寧營屬。

明正宣慰使司,其先系木坪分支。明洪武初,始祖阿克旺嘉爾參隨征明玉珍有功。永樂五年,授四川長河西寧魚通宣慰使。康熙五年,丹怎札克巴歸附。乾隆三十六年,甲木參德侵隨征金川有功,賞賜「佳穆伯屯巴」名號,並二品頂戴、花翎。五十六年,甲木參諾爾布隨征廓爾喀,賞花翎。嘉慶十四年,甲木參沙加領班進京恭祝萬壽,賞花翎,世襲,住牧打箭爐城。光緒三十四年七月,趙爾豐奏改打箭爐為康定府,設河口縣。宣統三年,土司甲木參瓊珀繳印,其地悉歸流。原管有咱哩木千戶,木噶、瓦七、俄洛、白桑、惡熱、下八義、少誤石、作蘇策、八哩籠、上渡噶喇住索、中渡啞出卡、他咳、索窩籠、惡拉、樂壤、扒桑、木轤、格窪卡、呷那工弄、吉增卡桑阿籠、沙卡、上八義、拉里、八烏籠、姆硃、上渣壩惡疊、上渣壩卓泥、中渣壩熱錯、中渣壩沱、下渣壩業窪石、下渣壩莫藏石、魯密東谷、魯密普工碟、魯密郭宗、魯密結藏、魯密祖卜柏哈、魯密初把、魯密昌拉、魯密堅正、魯密達媽、魯密格桑、魯密本滾、長結杵尖、長結松歸、魯密白隅、魯密梭布、魯密達則、魯密卓籠四十八土百戶。

革伯咱安撫司,其先魏珠布策凌,康熙三十九年,歸附,授職,頒給印信號紙。宣統三年,改流。

巴底宣慰司,其先綽布木凌,康熙四十一年,歸附,授巴底安撫司。宣統三年,改流。

巴旺宣慰司,與巴底土司同世系,分駐巴旺,共管地方土民。宣統三年,改流。

喇安撫司,其先阿倭塔爾,康熙四十年,歸附,授職。

霍耳竹窩安撫司,即倬倭。其先索諾木袞卜,雍正六年,歸附,授職。宣統三年,改流。原管有瓦述寫達、瓦述更平東撒兩土百戶。

霍耳章谷安撫司,其先羅卜策旺,雍正六年,歸附,授職。光緒二十年,瞻對欲奪其地,鹿傳霖派兵滅瞻對,同倬倭一並改流。後發還,而章谷無人承領,改為爐霍屯。宣統三年,改流。

納林沖長官司,其先諾爾布,雍正六年,歸附,授職,與章穀土司一家。

瓦述色他長官司,雍正六年,歸附,授職。

瓦述更平長官司,雍正六年,歸附,授職。

瓦述餘科長官司,其先沙克嘉諾爾布,雍正六年,歸附,授職。

霍耳孔撒安撫司,其先麻蘇爾特親,雍正六年,歸附,授職。宣統三年,改流。管有科則、圖根滿碟兩土百戶。

霍耳甘孜麻書安撫司,其先那木卡索諾木,雍正六年,歸附,授職。宣統三年,改流。原管有革賚、朿暑、又朿暑三土百戶。

德爾格忒宣慰司,其先丹巴策凌。雍正六年,歸附,授德爾格忒安撫司。十一年,改宣慰司。諸土司部落,以德格為最大,東連瞻對,西連察木多,南連巴塘,北連西寧。番人以其地大,有「天德格,地德格」之稱。鹿傳霖派兵攻瞻對時,訪得德格土司羅追彭錯妻玉米者登仁甲生子名多吉僧格,又與頭人通,生子名降白仁青,以是與夫反目。玉米者登仁甲本藏女,於瞻對藏官有姻誼,藏官助之抗其夫,故各攜其子分居焉。光緒二十年,官軍計誘羅追彭錯,言為之逐其婦及降白仁青,因入德格。洎傳霖被劾,罷改流議。土司夫婦旋病故,傳霖奏遣其二子回籍,多吉僧格暫管地方。降白仁青已為僧,繼而招致多人爭職,多吉僧格奔藏。德格頭人百姓以降白仁青非土司子,且殘暴,迎多吉僧格歸。降白仁青避位數年,頭人正巴阿登等嗾其再起爭職,並誘占多吉僧格之妾。多吉僧格夫婦復奔藏,控於駐藏大臣有泰、張廕棠。既而德格百姓復迎之歸,錮降白仁青。降白仁青脫出,聚黨作亂,人民多被殺戮,多吉僧格遣人至打箭爐告急。宣統元年四月,趙爾豐率兵討之,降白仁青敗逃入藏。多吉僧格夫婦請改流,爾豐不欲利其危亂,許以復職。多吉僧格泣曰:「德格地廣人稀,窺伺者眾,終恐不自保,原招漢人開墾,使地闢民聚,乃可圖存。」意極堅決。爾豐奏分其地為五區:中區德化州,南區白玉州,北區登科府,極北一區即石渠歸,西區則同普縣,而邊北道駐登科焉。多吉僧格納其財產於官,徙家巴塘,復以奏給養贍銀及其妻姒郎錯莫首飾捐助巴塘學費。爾豐奏賞頭品頂戴,並予其妻建坊。原管有四上革賚、雜竹嗎竹卡、籠壩,六土百戶。

霍耳白利長官司,其先隆溥特查什,雍正六年,歸附,授職。宣統三年,改流。

霍耳咱安撫司,其先阿克旺錯爾恥木,雍正六年,歸附,授職。管有兩下革賚土百戶。

霍耳東科長官司,其先達罕格努,雍正六年,歸附,授職。宣統三年,改流。

春科安撫司,其先袞卜旺札爾,雍正六年,歸附,授職。副土司與安撫司一家,同時歸附授職。宣統元年,改流。

高日長官司,其先自印布,雍正六年,歸附,授職。宣統元年,改流。

蒙葛結長官司,其先達木袞布,雍正六年,歸附,授職。

林蔥安撫司,其先袞卜林親,雍正六年,歸附,授職。宣統三年,改流。

上納奪安撫司,其先索諾木旺札爾,雍正六年,歸附,授職。宣統三年,改流。原管有上納奪土千戶,上納奪黎窩、上納奪、納奪黎窩三土百戶。

瞻對有上、中、下三名。上瞻對茹長官司、下瞻對安撫司,均雍正六年歸附授職;中瞻對長官司,乾隆十年授職。距打箭爐七日程。東連明正,南接里塘,西北與德格土司毗連。縱橫數百里,為鴉龍江之上游。同治初,川、藏會攻瞻對,川軍未至,藏兵先克瞻對,派民官一、僧官一,率兵駐守,由達賴喇嘛及商上選任咨請駐藏大臣奏明,每三年替換。藏官恣行暴政,誅求無厭,瞻對民不堪命,屢起抗官,疆吏率加壓服,仍令屬藏。光緒二十年,鹿傳霖討平瞻對,議改流,卒為恭壽、文海劾罷。三十四年,趙爾豐由川赴關外,德格土司百姓沿途控告瞻對藏官侵奪土地,四出虐民,並歷訴中朝兩次將瞻對歸藏時,藏官追究內附者一一孥戮之慘。藏官不自安,陰欲添兵攻爾豐,爾豐令傅嵩矞率兵赴昌泰扼之。宣統元年春,爾豐建議收瞻對,樞府令駐藏大臣聯豫、溫宗堯與藏人議贖未成,樞臣恐牽動外交,持不斷。於是爾豐與嵩矞議,決以計取之。三年夏,爾豐調任入川,偕嵩矞整兵經瞻對。藏官憚軍勢之盛,潛遁去,瞻人歡舞出迎。因收回設治。尋爾豐至川奏聞。

以上打箭爐阜和協屬。

以上各土司,均頒有印信號紙。

里塘宣撫司,其先番目江擺,康熙五十七年,歸附,授職。傳至索諾木根登,因不能約束帳下頭人云甸等,致滋事端,革去土職,以土都司布洛工布拔補。

里塘、巴塘兩土司例於頭人內揀補,與他土司不同。嘉慶十二年,希洛工布為竹馬策登等所害,以頭人阿策拔補,頒給印信號紙。

副土司,其先番目康卻江錯,與正土司同時歸附。雍正七年,授職。嘉慶八年,土司羅藏策登為正土司頭人云甸等戕害,以頭人阿彩登舟拔補,頒給印信號紙。向設守備一、把總一。光緒三十一年,川軍討巴塘亂,里塘頭人不支烏拉,糧餉不能轉連,趙爾豐誅頭人,正土司逃往稻壩貢噶嶺,嘯聚土人為亂。爾豐移師攻鄉城,分兵先剿稻壩。正土司敗逃入藏,稻壩平。先是鄉城喇嘛普中札娃強悍知兵,誘殺里塘守備李朝富父子。鹿傳霖派游擊施文明討之,為所擒,剝皮實草,懸以為號。三十二年正月,爾豐率兵督攻,大小數十戰,匪退喇嘛寺死守。爾豐圍之數月,斷其水道,普中札娃自縊,諸番皆降,改里塘為順化縣。三十四年秋,復改里化同知,以鄉城為定鄉縣,稻壩為稻城縣,貢噶嶺設縣丞。

瓦述毛丫長官司,其先番目索郎羅布,康熙六十一年,歸附。雍正七年,授職。

崇喜長官司,其先番目杜納臺吉,康熙六十一年,歸附。雍正七年,授職。

瓦述曲登長官司,其先番目康珠,康熙六十一年,歸附。雍正七年,授職。

瓦述啯隴長官司,嘉慶十二年,歸附,授職。

以上各土司,皆頒有印信號紙。

瓦述茂丫土百戶,其先番目側冷工,康熙六十一年,歸附。雍正七年,授職。瓦述麻裏土百戶,嘉慶十二年,歸附,授職。

以上各土司,皆頒有號紙。以上里塘糧務屬。

巴塘宣撫司,其先羅布阿旺,康熙五十八年,歸附,授職,頒給印信號紙。副土司同。由四川設糧員一、都司一、千總一,三年更替。其喇嘛寺設堪布一、鐵棒一,為僧官,亦三年另換。堪布掌管教務經典,鐵棒管理僧人條規。番人犯罪,土司治之。番人之喇嘛犯罪,鐵棒治之。土餉以賦相抵,不足由官補給,年約銀千餘兩。光緒三十年,駐藏幫辦大臣鳳全赴任,道經巴塘,見地土膏腴,即招漢人往墾,築墾場於茨梨隴,委巴塘糧員吳錫珍、都司吳以忠兼理。番人驚沮,土司堪布勸鳳全速入藏,不聽。三十一年春,七村溝番民聚眾劫殺墾夫,吳以忠陣亡,鳳全避入正土司寨,與亂民議和。亂民迫鳳全回川,許息事,鳳全信之。東行數里,至鸚哥嘴,被殺。夏,馬維騏、趙爾豐往討,六月十八日,克巴塘,誅兩土司並堪布喇嘛及首惡數人。爾豐搜剿餘匪,因移師定鄉城。三十二年秋,爾豐會錫良暨雲貴總督丁振鐸具奏改流,設巴安縣。三十四年,改巴安府,分設鹽井縣三壩通判,並設康安道,駐巴塘。原管有上臨卡石、下臨卡石、岡、桑隆、上阿蘇、下阿蘇、郭布等七土百戶。

巴塘糧務屬。

嶺夷十二地夷人頭目,嘉慶十三年,歸附,給有頭目牌。十六年,改流,更姓住牧。豹嶺岡姓高,趕山坪姓澤,阿葉坪姓惠,牛跌蠻姓周,芭蕉溝姓華,龍竹山姓夏,雪都都姓萬,小板屋姓年,牛心山姓海,月落山姓宇,鹽井溪姓成,桃子溝姓平。

赤夷十三支,嘉慶十三年,歸附,選拔土弁,給有委牌住牧。膽巴家土千總一、土把總一,管有雞疏、卑溪疏、哈疏、白魁四家。哈納家土千總一、土把總一,管有胃扭、雅札、哈什三三家。蜚瓜家土千總一、土把總二,管有媽、呆得二家。魁西家土千總一、土把總一。

凡各地支所部惈夷稱曰「娃子」。以上瓘邊瓘邊營冷磧汛屬。

川東道重慶鎮轄:

石砫宣慰使,其先馬定虎,漢馬援後。南宋時,封安撫使。其後克用,明洪武初加封宣撫使。崇禎時,土司千乘及婦秦良玉,以功加太子太保,封忠貞侯。子祥麟,亦加封宣慰使。順治十六年,祥麟子萬年歸附,仍授宣慰使職。乾隆二十一年,以夔州府分駐雲安廠同知移駐石砫。二十五年,設石砫直隸,改土宣慰使為土通判世職,不理民事。

夔州府夔州協屬。

酉陽宣慰使司,其先受明封。傳至奇鑣,於順治十五年歸附,仍授原職,頒給印信號紙。雍正十二年,土司元齡因事革職,以其地改設酉陽直隸州。原管有邑梅峒、平茶峒、石耶峒、地壩四長官司,均於乾隆元年改流。

重慶府綏寧營屬。

永寧道提標轄:

九姓土長官司,其先任福,江南溧陽人。明洪武初,從傅友德入蜀,招撫拗羿蠻,受封。傳至孟麒,以功擢安撫使。天啟元年,土司任世籓夫婦死難,子祈祿復以功授瀘衛守備。傳至長春,順治四年,歸附,更給知府劄副。吳三桂叛,長春來奔。十六年,復永寧,長春子功臣復率土民歸附,頒發劄付。康熙二年,江安縣賊吳天成等作亂,功臣以擒賊功議敘。子宗頊襲職,隨頒土長官司印信號紙,以武職屬瀘州州判及瀘州營管轄。嘉慶元年,移駐瀘衛。光緒三十四年,趙爾豐奏改瀘衛為古宋縣,存土司名。

瀘州瀘州營屬。

千萬貫土千總,其先自元時受封。明洪武四年,賜姓楊。康熙四十三年,土司楊喇哇歸附,頒給印信號紙。其後楊明義,於雍正六年因云南米貼夷滋事案參革。子明忠立功贖罪,賞土千總職銜,未經請頒印信號紙。管有頭目六十五名。

千萬貫土千戶,其先楊繼武,為土千總楊成胞叔。嘉慶七年,夷人滋事,繼武同成出力,賞給土千戶執照。

千萬貫土巡檢,其先安濟,明時授馬湖土知府。其後失職,復授土巡檢。雍正六年,土舍安保歸附,無印信號紙。管有頭目二十四名。以上雷波普安營屬。

黃螂土舍,其先為明時酋長。雍正五年,土舍國保歸附,無印信號紙。

凡千萬貫、黃螂四土司,所管黑、白骨頭二種惈夷,椎髻衣氈,耕種打牲為業。以上雷波安阜營屬。

平彞長官司,其先王元壽,原籍江南人,於明時受封。順治九年,土司王長才歸附。

蠻彞司長官司,其先文的保,原籍湖廣人,於明時受封。順治九年,土司文鳳鳴歸附。

泥溪長官司,其先王麒,自明時世襲。順治九年,土司王嗣傳歸附。

沐川長官司,其先於明時受封,賜姓悅。順治九年,土司悅嶢瞻歸附。

以上各土司,皆頒有印信號紙。以上屏山縣屏山汛屬。

明州樂土百戶,其先盔甲,涼山生夷。其後駱哥,康熙四十二年,歸附,授職。

油石洞土百戶,其先普祚,涼山生夷。子咀姑,康熙四十二年,歸附,授職。

旁阿姑土百戶,其先腳謨伯,涼山生夷。子駱束,康熙四十二年,歸附,授職。

大羊腸土百戶,其先六盔,涼山生夷。子紐車,康熙四十二年,歸附,授職。

膩乃巢土百戶,其先必祚,涼山生夷。子腳骨,康熙四十二年,歸附,授職。以上馬邊馬邊營煙峰汛屬。

挖黑土百戶,其先亦赤,涼山生夷。子三兒,康熙四十二年,歸附,授職。

阿招土百戶,其先阿直,涼山生夷。子秧哥,康熙四十二年,歸附,授職。

幹田壩土百戶,其先賒的,涼山生夷。子路引,康熙四十二年,歸附,授職。

麻柳壩土百戶,其先鄂車,涼山生夷。子六貴,康熙四十二年,歸附,授職。

以上各土司,皆領有號紙。

慄坪土千戶,其先卜佐,涼山生夷。其後阿二,嘉慶十三年,歸附,賞給職銜,領有委牌。

冷紀土外委,其先普祚,涼山生夷。子未鐵,雍正元年,歸附,授職。以上馬邊馬邊營三河口汛屬。

以上各土司外,有理番之雜穀腦屯、乾堡寨屯、上孟董屯、下董孟屯、九子寨屯,懋功之懋功屯、崇化屯、撫邊屯、章穀屯、綏靖屯等土弁,各設屯守備,暨所屬屯千總、屯把總、屯外委,均世及接頂,與地志、兵志互見。


\end{pinyinscope}