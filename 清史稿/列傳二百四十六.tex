\article{列傳二百四十六}

\begin{pinyinscope}
馮子材王孝祺陳嘉蔣宗漢蘇元春馬盛治

王德榜張春發蕭得龍馬維騏覃修綱吳永安

孫開華硃煥明蘇得勝章高元歐陽利見

馮子材,字翠亭,廣東欽州人。初從向榮討粵寇,補千總。平博白,賜號色爾固楞巴圖魯。改隸張國樑麾下,從克鎮江、丹陽,嘗一日夷寇壘七十餘。國樑拊其背曰:「子勇,餘愧弗如!」積勛至副將。國樑歿,代領其眾。取溧水,擢總兵。

同治初,將三千人守鎮江。時江北諸將多自置卡榷釐稅,子材曰:「此何與武人事?」請曾國籓遣官司之。所部可二萬,餉恆詘,無怨言。蒞鎮六載,待士有紀綱,士亦樂為所用。寇攻百餘次,卒堅不可拔。事寧,擢廣西提督,賞黃馬褂,予世職。赴粵平羅肅,移師討黔苗,克全茗、感墟。九年,出鎮南關,攻克安邊、河陽,凱旋,再予世職。光緒改元,赴貴州提督任。七年,還廣西。明年,稱疾歸。

越二年,法越事作,張樹聲蘄其治團練,遣使往趣駕。比至,子材方短衣赤足、攜童叱犢歸,啟來意,卻之。已,聞樹聲賢,詣廣州。適張之洞至,禮事之,請總前敵師干衛粵、桂。逾歲,朝命佐廣西邊外軍事。其時蘇元春為督辦,子材以其新進出己右,恆悒悒。聞諒山警,亟赴鎮南關,而法軍已焚關退。龍州危棘,子材以關前隘跨東西兩嶺,備險奧,乃令築長墻,萃所部扼守,遣王孝祺勤軍軍其後為犄角。敵聲某日攻關,子材逆料其先期至,乃決先發制敵。潘鼎新止之,群議亦不欲戰。子材力爭,親率勤軍襲文淵,於是三至關外矣。宵薄敵壘,斬虜多。

法悉眾分三路入,子材語將士曰:「法軍再入關,何顏見粵民?必死拒之!」士氣皆奮。法軍攻長墻亟,次黑兵,次教匪,砲聲震山谷,槍彈積陣前厚寸許。與諸軍痛擊,敵稍卻。越日復湧至,子材居中,元春為承,孝祺將右,陳嘉、蔣宗漢將左。子材指麾諸將使屹立,遇退後者刃之。自開壁持矛大呼,率二子相榮、相華躍出搏戰。諸軍以子材年七十,奮身陷陣,皆感奮,殊死鬥。關外游勇客民亦助戰,斬法將數十人,追至關外二十里而還。越二日,克文淵,被賞賚。連復諒城、長慶,擒斬三畫、五畫兵總各一,乘勝規拉木,悉返侵地。

越民苦法虐久,聞馮軍至,皆來迎,爭相犒問,子材招慰安集之,定剿蕩北圻策。越人爭立團,樹馮軍幟,願供糧運作鄉導。北寧、河內、海陽、太原競響,子材亦毅然自任。於是率全軍攻郎甲,分兵襲北寧,而罷戰詔下,子材憤,請戰,不報,乃挈軍還。去之日,越人啼泣遮道,子材亦揮涕不能已。入關至龍州,軍民拜迎者三十里。命督辦欽、廉防務,會辦廣西軍務,晉太子少保,改三等輕車都尉。

十三年,討平瓊州黎匪,降敕褒嘉。調雲南提督,稱疾暫留。二十年,加尚書銜。值中日失和,命募舊部至江南待調發。和議成,還防。二十二年,赴本官。二十六年,入省籌防,會拳亂作,請募勁旅入衛,上嘉其忠勇,止之。逾歲,調貴州。二十八年,病免。明年,廣西土寇蜂起,岑春煊請其出治團防。方募練成軍,率二子以進,而遘疾困篤。未幾,卒,年八十六,謚勇毅,予建祠。

子材軀幹不逾中人,而硃顏鶴發,健捷雖少壯弗如。生平不解作欺人語,發餉躬自監視,偶稍短,即罪司軍糈者。治軍四十餘年,寒素如故。言及國樑,輒涔涔淚下,人皆稱為良將雲。

王孝祺,本名得勝,安徽合肥人。初入淮軍,以敢戰名。從李鴻章規三吳,積勛至守備。又從張樹聲克常、昭諸城,釋平湖圍,歷遷副將。論克宜、荊、溧、嘉、常功,擢總兵,賜號壯勇巴圖魯。從援浙,連下湖州、長興。是時,樹聲弟樹珊攻湖北德安陣亡,坐失主將,貶秩。戰敗東捻,復故官。西捻平,晉提督,更勇號為博奇。旋赴山西防河,大搜馬賊。值晉饑,斥家財以濟,民德之,賊所竄匿,輒先詗以告。事寧,賜頭品秩。光緒六年,聲督督兩廣,奏自隨。歷署潮州、碣石總兵。九年,徙右江鎮,主欽、廉防務。

明年,潘鼎新來乞師,領勤軍赴龍州,而鼎新已遁,乃從子材詣鎮南關截潰勇。宵襲文淵,入街心,馬踣,亟易騎,率死士繞山後,攀崖上,破二壘。俄而法軍分路入,直攻關前隘,復自後路仰擊,敵稍卻。李秉衡集諸將舉前敵主帥,孝祺曰:「今無論湘、粵、淮軍,宜並受馮公節度。」秉衡稱善。右路者西嶺也,其部將潘瀛袒臂裸體,沖入敵陣,故傷亡獨多。至日暮,孝祺擊敗之,奪三壘而還。攻諒城,瀛執幟先登,並力克之,城復。取太原,予世職。明年,授北海鎮總兵。二十年,賞雙眼花翎。逾歲,謝病歸。越四年卒,恤如制。

陳嘉,字慶餘,籍廣西荔浦。從蘇元春征黔苗,累勛至副將,賜號訥思欽巴圖魯。平六硐,擢總兵,調赴湖南守寶慶。鼎新撫廣東,嘉引兵從。抵思恩,值土寇嘯亂,計擒其魁莫思弼,誅之。

法越之役,率鎮南軍出關扼穀松。敵至,砲甚猛,退頓堅老,已而戰船頭、陸岸,皆捷。法軍據紙作社,師設伏誘之,嘉出挑戰,敵悉眾迎拒,戰方酣,元春隱兵起,斬法將四人、兵二百八十餘。捷入,賞黃馬褂,授貴州安義鎮總兵。未幾,法軍大舉寇堅老,鏖戰數晝夜,被重創僕地,左右掖之去,既覺,麾刀叱退,仍奮擊敗之。逾歲,法軍薄長墻。左路即東嶺,嘉爭其三壘,宗漢繼之,七上七下,嘉被創者四,氣不少沮。孝祺自西來援,合擊之,遂奪還。以次復文淵、諒山,進規穀松,力疾赴前敵,詔嘉之,賜頭品秩,予世職。創發,卒於軍,年未五十,謚勇烈。

蔣宗漢,籍云南鶴麗。同治初,回寇入境,方居憂,其酋馬金寶逼令受先鋒印,佯以終制辭。潛歸里,至江幹,無舟可濟,追騎將及,仰天祝曰:「茍得留身報國,當建此橋!」果得浮槎以免。既貴,成金龍橋,亙數百丈,行旅至今賴之。初隸楊玉科麾下,每戰輒為先鋒。從攻豬供箐,其下有吳家屯,為寇儲糧地,備奧阻。宗漢間道得大溜口,率死士百,縋幽鑿險,忍饑抵壁下,置藥桶,設伏線,潛出約師,火發,大敗之,繇是知名。又從玉科迭下各郡邑,積勛至副將,賜號著勇巴圖魯。戰小圍埂,勒馬挺矛,當者輒靡。岑毓英見而嘆曰:「真虎將也!」大理平,擢提督,賞黃馬褂。攻錫臘、順寧,皆先據要險,設伏敗敵。人皆謂其善謀云。事寧,更勇號圖桑阿。克雲州,署騰越鎮總兵。攻克烏索,授順云協副將。

光緒改元,英繙譯官馬嘉理入滇邊,抵戶宋河遇害,坐疏防,鐫秩付鞫。明年,復騰越,起副將。五年,靖遠平,復故官。法越之役,率廣武軍出關,功與嘉埒。和議成,賜頭品秩,除貴州遵義鎮總兵。二十年,賞雙眼花翎。二十六年,署提督,調雲南。越二年,還貴州,予實授。明年,卒,予建祠。

蘇元春,字子熙,廣西永安人。父德保,以廩生治鄉團,御寇被害,州人建祠祀之。元春誓復仇,從湘軍。同治初,隨席寶田援贛、皖、粵,累功至參將,假歸。六年,領中軍征黔苗,破荊竹園,賜號健勇巴圖魯。連克要隘,更號銳勇。八年,統右路軍,值思州苗犯鎮遠,復擊卻之,進復清江,擢總兵。黃飄之役,黃潤昌戰死,元春馳救,亦敗退,幹吏議。克施秉,復故官。九年,攻施洞,拔九股河,又改法什尚阿勇號。薄臺拱,苗遁走,晉提督。明年,復丹江、凱里,軍威益振,賞黃馬褂。以次下黃飄、白堡,驛道始通。逾歲,循清水而南,所至輒靡,惟烏鴉坡猶負固。復自東南破張秀眉砦。殘苗將北走,黔軍遏之河干。元春麾軍馳之,截寇為二,斬數千級,降三萬餘人,苗砦悉平。元春留頓其地,撫降眾。論功,予雲騎尉。全黔底定,賜頭品秩。光緒初,平六硐及江華瑤,被賞賚。

十年,和議中變,法人大舉攻桂軍。潘鼎新薦其才,詔署提督。遂率毅新軍駐穀松,取陸岸,鏖戰五晝夜。上嘉其勇,命佐鼎新軍,再予騎都尉。規紙作社,敵緣江築壘,夜將半,師設伏誘之,其左樹木幽深,元春隱兵其中,敵至,於是夾擊,大破虜。既而法人犯穀松,師連戰失利。敵毀鎮南關,元春出隴窯御之,不克,退幕府。當是時,自南寧至桂林,居民大震。鼎新罷免,遂命主廣西軍事。十一年,法人寇西路,元春趨艽封截之,乃引去。俄攻關前隘,失三壘,元春亟馳救。詰朝,助子材扼中路,大捷,語具子材傳。長驅文淵,元春踵至,詗知敵據驅驢墟,乘其未整列逐之,敵奪門走,進扼觀音橋,而停戰詔下,諸軍分頓關內,元春駐憑祥,居中調度。和議成,授提督,晉三等輕車都尉,又改額爾德蒙額勇號。

還龍州,其南曰連城,號天險,建行臺其上,暇輒取健兒練校之,授以兵法。西四十里即關,崇山相崟,一道中達。元春相形勝,築砲臺百三十所,囑統將馬盛治鎮之。鑿險徑,闢市場,民、僮懽忭。復自關外達龍州,創建鐵路百餘里,增兵勇,設制造局,屹然為西南重鎮。加太子少保,晉二等輕車都尉。二十五年,入覲,命赴廣州灣劃界。

前後鎮邊凡十九年,閱時久,師律漸弛,兵與盜合而為一,蔓滋廣。朝命岑春煊督兩粵治之,御史周樹模劾元春演餉縱寇,敕春煊按覆。春煊謂不斬元春無以嚴戎備,詔奪職逮訊。初,湘軍舊制,軍餉月資衣食外,餘存主將所備緩急,歲餘乃給之,名曰「存餉」。元春蒞邊,凡所設施,不足,移十二萬濟之。刑部擬以斬監候,獄急,元春請以應領公款十六萬備抵償。於是部再疏其狀,謂其父死難,例得減,詔戍新疆。

元春軀幹雄碩,不治生產,然輕財好士,能得人死力。嘗與法人接,獨持大體。金龍峒者,安平土州地,為中、越要隘,法將據之,與爭不決。而游勇萬人恆出沒為法患,法莫能制。其總督入關來求助,元春悉召至資遣之,金龍七隘卒歸隸。法商李約德為寇所掠,總署慮啟釁,以屬元春。元春簡騶從詣山下,寇聞,送之出。時元春已積逋二十萬,或勸其請諸朝,元春嘆曰:「吾任邊事,致外人蹈絕險,尚敢欺朝廷要重利乎?」卒不可。法感其義,贈寶星。既入獄,年已六十矣,無子,幕士董左右之。法總統聞其狀,急電公使端貴等謀緩頰。喜,具以告,元春曰:「法,吾仇也。死則死耳,藉仇以乞生,是重辱也!君為我謝之。」居戍四年,御史李灼華疏其冤,事下張人駿,廉得實,請釋歸,而已卒於迪化。貧無斂,新疆布政使王樹棻為治其喪。宣統改元,復官,子承賜,戍所生。

馬盛治,字仲平,籍廣西永安。以孝著。初隨席寶田征黔苗,積功至游擊,賜號壯勇巴圖魯。苗疆平,更勇號哈豐阿,遷副將。從克六硐,擢總兵。越事急,遂率師出關。時宣光、太原、牧馬潰勇索餉譁變,盛治輕騎往撫,汰弱留強,軍紀以肅。逾歲,法人悉銳至,腹背受敵。盛治具餱糧,間道繞敵前,與元春諸軍夾擊之,遂復南關。克文淵、諒山、長慶,頻有功,賞黃馬褂。光緒十二年,除柳慶鎮總兵,仍統邊軍佐元春,築砲臺,設廛市,賞雙眼花翎。二十一年,會辦中越界務。連破西林、鬱林諸匪,晉提督。二十八年,移署左江鎮。南寧各屬故盜藪,至即麾軍搜剿,寇聞風遁。遂檄所屬練團築卡,堅壁清野,寇大困。其酋黃和順等猶負嵎,官軍攻隴賴,遇伏,槍彈雨坌,盛治被重創,眾掖之出,旋卒。

盛治居邊十七年,元春倚如左右手。元春尚寬,而盛治濟以嚴,邊境賴以寧謐。卒,年五十八,謚武烈,予思恩、南寧建祠。

王德榜,字朗青,湖南江華人。咸豐初,粵寇擾境,與兄吉昌毀家起鄉兵,戰數利。五年,援江西,攻奉新,吉昌戰死,德榜領其眾,誓復仇。七年,論克瑞州功,敘經歷、州同。明年,從將軍福興援浙,復衢、處各城,擢知州。又明年,從援安徽,克婺源,遷直隸州知州,援例加道員。其夏,殲賊浮梁景德鎮。十年,平廣信,寇遁入浙。徙防玉山,歸左宗棠節度。十一年,李世賢、李秀成先後來犯,並擊卻之,賜號銳勇巴圖魯。

同治改元,所部譁變,又不稟宗棠命,私越境駐廣豐,褫職留軍。尋還浙。世賢犯遂安,出常山、華埠截之。會宗棠耀兵龍游,令扼全旺。世賢遣驍賊分道馳救,德榜自右路夾擊,皆愕走。城寇猶未下,逾歲,偪城南,築三壘,寇夜遁,復官。移師浮梁,連下崇光、陽溪諸渡。三年,釋廣信圍。其秋,復東鄉,長驅江山、玉山、廣豐、鉛山,所至皆下,擢按察使。

是時,世賢合汪海洋出入江、廣邊,連陷龍巖、南靖、漳州。德榜將二千五百人馳援,合劉典軍為西路軍,攻莒溪,克之。四年春,授福建按察使。復古田,攻南陽,師少卻。俄而海洋率黃、白號悍黨可二萬列田壟,典先入,德榜為承,奮擊之,寇返西岸。德榜追至下車,海洋下馬痛哭,其黨挾之走。黃、白號衣者,海洋所蓄死士,號無敵,至是喪失過半矣。四月,邀擊世賢於安溪,進攻烏頭門,復漳城,馳大埔,郭揚維率四千人降,乘勝克南靖。易勇號曰達沖阿,遷布政使。十月,援嘉應,頓塔子墺,與諸軍環偪之。追寇,寇返奔。時宗棠軍大埔,麾下止八百人,勢岌岌,亟召德榜扼三河壩。地當潮州要沖,皆山道絕澗。德榜至,察地勢,度寇必不往,且主帥軍孤懸,寇直犯必不支,乃請當中路,卒與典軍出寇前遏之。十二月,復嘉應,誅海洋。捷入,賞黃馬褂。六年,遭父憂歸。

十年八月,宗棠征河州回,德榜詣軍所綜營務。時黑山壘林立,勢張甚。德榜率二千人自狄道渡洮,以石鼓墩左拂黑山,右扼邊家灣,形便控駕,乃築二壘其上,與諸軍痛擊,寇壘悉平。進駐迤南三甲集,率騎越山南下,大破之。剿東鄉,抵陰窪泉,遇伏,下馬督戰,寇潰。迭克要害,寇並入謝家坪。十一年,傅先宗戰歿新路坡,德榜接統其軍,申明紀律,誅將弁先潰者六人,士氣復新。羌地曠,夙患狼,往往百十成群,夜入幕帳噬人。德榜令將士習獵搜捕,狼患減。甘南既平,撫降回十餘萬。濬狄道河渠,獲沃壤百餘萬畝。降敕褒嘉,賜頭品秩。光緒元年,母憂,解職。六年,再赴新疆,以舊部駐張家口。七年,入京,教練火器、健銳諸營,兼興畿輔水利。

十年,越南事亟,率師赴難。抵龍州,募新軍八營,號定邊軍,單騎詣諒山,謁徐延旭陳方略。令提督張春發分兵駐朝陽山、半隴山左右,何秀清等駐驅驢墟,通運道,而自領兵赴鎮南關。北寧陷,權廣西提督。戰豐穀,敗,蘇元春不往援,德榜銜之;以故元春敗於穀松,亦不往救。德榜自負湘中宿將,與督師不洽。潘鼎新責其戰不力,劾罷之,以所部屬元春。九月,復被命赴那陽,進偪船頭,戰數捷。

明年,軍油隘,法軍犯長墻,出師夾擊,據文淵對山,鏖戰數日,殺傷略相當。越日,陳嘉爭東嶺三壘,德榜擊其背,克之。是日晨,出甫谷,敵援至,沖截為二,部將蕭得龍及春發戰最勇,殲法軍百餘人,獲糧械無算。敵被截,大潰。已,復合諸軍攻諒城,法軍扼驅驢墟,地故有德榜舊壘,堅且緻。平明,德榜殲其六畫兵總一,諸軍繼之,城復。穀松敵勢仍悍,又殲其三畫兵總一,於是法人大潰,悉返侵地。復故官,被賞賚。尋移疾去。十五年,授貴州布政使。十九年,卒,恤如制。

張春發,字蘭陔,江西新喻人。初隸劉松山麾下,充探騎,頻有功。累遷至副將,賜號傑勇巴圖魯。從征陜回,規寧靈,戰常陷堅,擢總兵。金積堡寇決渠淹我師,春發開溝築堤,引流反灌,破壘二百餘,更勇號曰哲爾精阿。復巴燕戎格及河州,晉提督。光緒二年,從劉錦棠取迪化,連克瑪納斯、達阪、托克遜,賞黃馬褂。進復西四城,予世職。五年,安集延布魯特入寇,春發度幕趨博斯塘特勒克,搗其巢,逐北至俄境。

法越肇釁,從德榜奪東嶺。法援大集,彈入右額,貫左頰,裹創力戰,大捷。除廣西右江鎮總兵,署廣東陸路提督,賞雙眼花翎。二十一年,平永安、長樂匪,予實授。二十六年,調湖北,逾歲,徙雲南。魏光燾劾其營務廢弛,論戍。三十二年,張之洞白其誣,復官,綜兩江營務。宣統三年,病免,旋卒。

春發治軍嚴,嘗云兵佚則驕惰,以故朝夕躬訓練,暇輒使濬河流,平道路。然木訥寡文,疏酬應,同官先施者恆不答禮,且往往氣凌其上,卒以此叢忌。

蕭得龍,籍湖南藍山。咸豐初,從援贛、浙,積勛至提督。調赴閩,克南陽、漳州。攻嘉應,寇遁,追扼北溪,大敗之,賜號博奇巴圖魯。光緒初,移師甘肅,克東鄉太子寺。越事危棘,與法人戰南關,殺傷略相當,奪東嶺三壘。功最,賞黃馬褂。事寧,署莊浪協副將。創發,卒於官,予優恤。

馬維騏,字介堂,雲南阿迷人。少從岑毓英軍征回寇,積功至都司,捕盜尤有名。越南事亟,又從毓英出關,以偏裨當一路。法越之戰,滇軍多有功,而以維騏及覃修綱、吳永安為著。師攻宣光,垂克,法援大集,圍劉永福軍,維騏銳身馳救,鏖戰二晝夜,擊卻之。從攻臨洮,功最,遷副將,賜號博多歡巴圖魯。

光緒十三年,襲攻惈黑,間道濟瀾滄江。賊驚潰,斬其酋張登發,闢地千里,晉總兵。頻年越匪亂,騷擾各州邑,設方略治之,邊境以安。二十四年,除廣東潮州鎮。越四年,擢四川提督。仁壽、彭山土寇起,焚教堂,殺教民,勢洶洶。岑春煊諗其嫺武略,軍事一以屬之,用兵數月,以次戡定。三十一年,打箭爐關外泰凝寺喇嘛謀叛,率師討平之。會巴塘蠢動,殺駐藏大臣鳳全,川邊大震。維騏剿撫兼施,克要害,擒渠率,賜頭品秩、黃馬褂。趙爾巽督川,改編巡防軍,奏充翼長,訓練士卒,創設將弁學堂,軍民綏戢。宣統二年,卒,恤如制。

覃修綱,籍廣西西林。隸毓英麾下,與維騏齊名。徵回有功,累遷至參將,賜號勤勇巴圖魯。從克雲州,晉副將,更勇號曰隆武。宣光之役,修綱獨扼夏和、清波,分兵取嘉喻關,復招越民九千,分頓要隘,綴法軍。緬旺前接山西、興化,後達十州、三猛,為敵所據,出不意襲克之。次年,永福戰失利,軍潰退,修綱仍堅持不動。戰臨洮,斬其二將,夜半時,率死士短兵搏擊,法人大敗。乘勝復各郡縣,北圻諸省皆響應。修綱出奇兵直搗越南中部,而奉命罷戍。

事寧,賞黃馬褂,署川北鎮總兵,仍留滇。歷權普洱、開化諸鎮,坐事免。光緒二十五年,起甘肅西寧鎮,留滇如故。三十一年,卒,予建祠,並毓英祠附祀。

修綱性忠勇,官開化久,有惠政,士民感頌,因寄籍文山雲。

吳永安,籍云南廣西州。毓英部將中稱驍果。以徵回功,累遷至副將,賜號尚勇巴圖魯。從克澂江,擢總兵。平館驛,晉提督,更勇號曰額特和,賞黃馬褂。毓英撫福建,奏署臺灣鎮,未之官,憂歸。起治雲南邊防。法人浮小舟渡沱江,永安乘其半濟,擊敗之。趨宣光,留三營扼守,而自間道還興化合岑軍。既而諸軍攻宣光,與修綱分扼要隘,取嘉喻關,攻臨洮,戰益利,予優敘。和議成,署昭通鎮。討平武定夷匪,補鶴麗鎮。光緒十九年,卒,附祀毓英祠。

孫開華,字賡堂,湖南慈利人。少從軍,從鮑超援江西,戰九江小池口,傷右臂。援湖北,再被創。池驛之役,夾擊敗敵,積勛至守備。同治初,轉戰皖、贛間,遷副將。克句容、金壇,賜號擢勇巴圖魯。以次攻金溪、南豐、新城、寧都、瑞金,並下之,晉總兵。廣東,嘉應亂,敗賊黃沙嶂,降者十餘萬,擢提督。五年,除漳州鎮總兵,仍北行追捻入楚。其秋,赴本官。總督文煜累疏薦其才。十三年,總督李宗羲治江防,設霆慶、霆匯諸營。廈門與臺、澎對峙,勢險要,開華以超舊將,被命治廈門海防。募勇成捷勝軍,赴臺北、蘇澳營辦開山,詔署陸路提督。

光緒二年,率師東渡,頓基隆,顧北路。其時後山阿綿、納納社番畔服靡心互,開華領所部抵成廣澳,量地勢,察番情,進駐水母丁。悍番分路迎拒,開華麾軍鏖戰,陣斬數人,餘敗潰。師入高崁,直搗其巢。潰番並入阿綿,其地水湍急,聳巘巉崗,砲臺錯列,備奧阻。開華轟擊之,縱以火箭,復繞道攻其後,番駭走,遂克之,擒其魁馬腰兵等梟於市。九日三捷,論功,賞黃馬褂。四年,霆慶軍統將宋國永卒,開華接統其眾。會加禮宛、巾老耶畔,據鵲子城,師攻不克。總督何璟以軍事棘,令開華進新城,許便宜行事。開華浮戰艦入自花黎,襲攻後山背。四日悉夷諸社,斬二百數十級。番乞款,縛姑乳斗玩以獻,寘之法。臺北平,被賞賚。明年,內渡,再署提督,秋,復渡臺。九年,回任。已,復出辦臺北防務。

十年,法人來犯,時劉銘傳主軍事。銘傳故淮軍宿將,知開華幹略,檄守水扈尾。初,法艦八艘至,開華度其必登岸,令諸將分伏砲臺後,露宿以待。部署甫定,而敵彈雨坌,煙焰翳天,偪臺而前。開華見勢猛,分路截擊,自夜至午,卻而復前者數四。臺既毀,短兵接戰。開華銳身入,手刃執旗卒,奪其旗以歸。諸軍士見之,氣益奮,斬馘二千餘級,法人遁走。歐洲諸國以失國旗為至辱。捷入,予世職,拜幫辦軍務之命。和議成,還本官,旋予實授。十九年,卒,謚壯武。子道仁,亦官福建提督。同時守滬尾者,硃煥明為最著。

煥明,籍安徽合肥。初從銘軍征粵寇,積功至游擊。平東捻,遷副將。西捻犯畿疆,躡之滄州、德平,戰數利,晉總兵。光緒元年,臺灣生番騷動,從唐定奎往討,連破竹坑山、內外獅頭,擢提督。法越之役,法軍分道犯滬尾。煥明當北路,被重創,戰益力,開華直入擊退之。旋移師臺北,平番社,軍嘉義鹿港。土寇數千薄城,煥明率三百人與戰,殞於陣。事聞,附祀定奎祠。

蘇得勝,亦籍合肥。從銘軍討捻,積功至游擊。戰常陷堅,賜號勵勇巴圖魯,屢遷提督。法艦寇臺灣,從銘傳守臺北。戰基隆,大捷,記名海疆總兵,更勇號曰西林。滬尾告警,銘軍回援,於是基隆再失。逮滬尾既復,得勝還駐六堵。規基隆,全軍會月眉山,曹志忠將左,劉朝祐將右,得勝居中。敵至左路,擊卻之。逾歲,法益兵攻志忠營,得勝領數百人往援,戰失利,提督梁善明陣亡,右師亦潰,月眉復不守。而得勝已先營六堵,築城十餘里,諸軍獲安。相持月餘,和議成,始開港。旋補建寧鎮,仍留防滬尾。數剿生番,感瘴成疾。光緒十六年,卒於軍。妻徐氏,絕食殉焉。恤如制,妻獲旌。

章高元,亦合肥人。初入淮軍,累至副將。銘傳檄為騎旅先鋒,轉戰魯、皖。安丘之役,以功擢總兵,賜號奇車巴圖魯。征臺灣,晉提督。法越事作,署澎湖鎮總兵,銘傳檄援滬尾。滬尾、基隆既復,論功,更勇號年昌阿,除登萊青鎮。中日失和,詔赴前敵,駐蓋平。日軍來攻,戰挫遂退。德軍艦襲膠澳,被幽,旋脫歸,稱疾罷。拳亂作,起署天津鎮,徙重慶,以病免。卒,年七十一。

歐陽利見,字賡堂,湖南祁陽人。咸豐初,入長沙水師,轉戰贛、皖間,積功至游擊。同治改元,偽護王陳坤書據太平,以兵艦銜尾西上,環泊花洋上駟渡,期水陸並進。利見領一軍為前鋒,兼程赴難。坤書陽令陸路悍黨擊我師船,而陰結筏自下游竊渡。利見詗知之,率所部長驅,乘風浪沖其筏為二。寇大困,倚河築壘,矢堅守。我師水陸分道進,利見駛入花山,擊其背。遲明,戰良久,寇陣不少動,援軍至,始退。利見進次馬音街,會水師將李朝斌偪花津而陣,步騎助之,寇潰。翼日,復戰,陸師將周萬倬遇伏被創,利見銳身馳救,苦戰竟日,焚象山寇舍。而寇艎聚泊小丹陽,歸護新市鎮。利見進石臼湖轟擊之,獲其船十二艘,遷參將。

二年,攻巢縣,偪城東門而軍,適彭毓橘軍至,燔其筏,毀浮橋。寇入城,利見先登克之,遂與霆軍復含山。四日連下三城,功最,賜號強勇巴圖魯,調補狼山鎮游擊。克嘉定,遷副將。下太倉、昆山、新陽,晉總兵。是時花涇港寇壘林立,與吳江、震澤寇相犄角。利見率師破之,毀其船二十艘,城寇援絕乞降。於是蘇、浙路梗,蘇寇無固志。李鴻章督師合圍,利見引兵從,迭克要害,寇宵遁。城復,晉提督。三年,攻嘉興,利見率謝世彩等與陸師夾擊,麾眾先登,自城上發巨砲轟之。城寇駴亂,城遂拔。以次下長興。坤書據常州,鴻章舉兵西,使利見造浮橋渡壕,四面環攻,坤書就擒。中吳大定,除淮揚鎮總兵。

四年,捻至曲阜,東南走滕、嶧,渡運,東北走蘭山,南走郯,趨贛榆、青口,圖南下。朝廷憂裏下河,詔備淮揚防。於是利見率砲艦四十艘泊清江,兼治糈臺。七年,黃河暴漲,利見乘流至德州,運防乃固。捻雖屢挫,然渡運之謀未已,盤旋河東上下。利見復下駛援應,與諸軍環擊,捻益不支。事寧,賞黃馬褂,更勇號曰奇車伯。光緒六年,調福山鎮。明年,擢浙江提督。

十年,法艦寇福建,浙江戒嚴。鎮海為浙東門戶,利見以三千五百人頓金雞山防南岸,提督楊岐珍以二千五百人頓招寶山防北岸,總兵錢玉興以三千五百人為游擊師。威遠、靖遠、鎮遠三砲臺,守備吳傑領之,而元凱、超武二兵艦泊海口備策應。諸將皆受利見節度。利見實以兵備道薛福成為謀主,乃量形勢,設防禦,蒐軍實,清間諜,杜鄉導,申紀律,勵客將,布利器,部署甫定,而敵氛已偪。法人狃馬江之役,頗輕浙防。利見督臺艦兵縱砲擊之,法主將坐船被傷,數以魚雷突入,皆被擊退。法艦並力猛進,又沉其一。敵計窮,相持月餘,終不得逞。事後知主將孤拔於是役殞焉。上嘉其功,賜頭品秩。

十五年,病免。二十一年,劉坤一被命援奉天,奏調利見赴軍。力疾北行,卒於道,年七十一。

論曰:法越之役,克鎮南,復諒山,實為中西戰爭第一大捷。摧強敵,揚國光,子材等之功也。開華等復滬尾,利見等守鎮海,與維騏等偕劉永福之拔宣光,並傳榮譽。當時挾戰勝之威,保臺復越,亦尚有可為。獨怪當事者為臺灣難保之說以自餒其氣,致使關外雖利,而越南終非我有。罷戰詔下,軍民解體,至今聞者猶有恨焉。


\end{pinyinscope}