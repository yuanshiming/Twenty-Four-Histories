\article{本紀六}

\begin{pinyinscope}
聖祖本紀一

聖祖合天弘運文武睿哲恭儉寬裕孝敬誠信功德大成仁皇帝,諱玄燁,世祖第三子也。母孝康章皇后佟佳氏,順治十一年三月戊申誕上於景仁宮。天表英俊,岳立聲洪。六齡,偕兄弟問安。世祖問所欲。皇二子福全言:「原為賢王。」帝言:「原效法父皇。」世祖異焉。

順治十八年正月丙辰,世祖崩,帝即位,年八歲,改元康熙。遺詔索尼、蘇克薩哈、遏必隆、鰲拜四大臣輔政。

二月癸未,上釋服。乙未,誅有罪內監吳良輔,罷內官。丙申,以嗣簡親王濟度子德塞襲爵。

三月丙寅,詔曰:「國家法度,代有不同。太祖、太宗創制定法,垂裕後昆。今或滿、漢參差,或前後更易。其詳考成憲,勒為典章,集議以聞。」

四月,予殉葬侍衛傅達理祭葬。甲申,命湖廣總督駐荊州。乙酉,命將軍線國安統定南部軍鎮廣西。丙戌,以拉哈達為工部尚書。癸卯,安南國王黎維祺遣使入貢。丙午,大學士洪承疇乞休,允之,予三等輕車都尉世職。戊申,賜馬世俊等三百八十三人進士及第出身有差。

五月,罷各省巡按官。己巳,以高景為工部尚書,劉良佐為江安提督。乙亥,安南叛臣莫敬耀來歸,封歸化將軍。

六月己卯,江蘇巡撫硃國治疏言蘇省逋賦紳衿一萬三千五百十七人,下部斥黜有差。辛巳,黑龍江飛牙喀部十屯來歸。庚寅,以嗣信郡王鐸尼子鄂扎襲爵。癸巳,大學士傅以漸乞休,允之。丁酉,罷內閣,復內三院。戊戌,吳三桂進馴象五,卻之。詔停直省進獻。

閏七月庚辰,以車克為吏部尚書,阿思哈為戶部尚書。甲午,以傅維鱗為工部尚書。壬寅,予蘇松提督梁化鳳男爵。

八月甲寅,達賴喇麻請通市,許之。

九月丁未,以卞三元為雲南總督,李棲鳳為廣東總督,郎廷佐為江南總督,梁化鳳為江南提督。

十月己酉,以林起龍為漕運總督。誅降將鄭芝龍及其子世恩、世廕。辛酉,裁順天巡撫。山東民於七作亂,逮問巡撫許文秀,總兵李永盛、範承宗,命靖東將軍濟世哈討平之。

十一月丙子朔,上親祀天於圜丘。己亥,世祖章皇帝升祔太廟。甲辰,湖南巡按御史仵劭昕坐贓棄市。

十二月丙午,平西王吳三桂、定西將軍愛星阿會報大軍入緬,緬人執明永歷帝硃由榔以獻。明將白文選降。班師。丁卯,宗人府進玉牒。

是歲免直隸、江南、河南、浙江、湖廣、陜西各州縣被災額賦有差。朝鮮遣使進香入貢。

康熙元年壬寅春正月乙亥朔。乙酉,享太廟。庚寅,錄大學士範文程等佐命功,官其子承謨等俱內院學士。

二月壬子,太皇太后萬壽節,上率群臣朝賀。

三月,以滇南平,告廟祭陵,赦天下。辛卯,萬壽節。己亥,遣官安輯浙江、福建、廣東新附官民。

夏四月丙辰,上太祖、太宗尊謚。

五月戊寅,夏至,上親祭地於方澤。

六月丁未,命禮部考定貴賤等威。

秋七月壬申朔,以車克為大學士,寧古禮為戶部尚書,張傑為浙江提督,施瑯為福建提督。

八月辛丑朔,大學士金之俊罷。

九月,裁延綏巡撫。

冬十月壬寅,以成克鞏為大學士。癸卯,尊皇太后為太皇太后。尊皇后為仁憲皇太后,母後為慈和皇太后。

十一月辛巳,冬至,祀天於圜丘,免朝賀。

十二月辛酉,命吳三桂總管雲南、貴州兩省。

是歲,天下戶丁一千九百一十三萬七千六百五十二,徵銀二千五百七十二萬四千一百二十四兩零。鹽課銀二百七十二萬一千二百一十二兩零。鑄錢二萬九千萬有奇。免直隸、江南各州縣災賦有差。朝鮮入貢。

二年癸卯春正月己亥,廣東總督盧崇峻請封民船濟師,斥之。

二月庚戌,慈和皇太后佟佳氏崩。

三月,荷蘭國遣使入貢,請助師討臺灣,優賚之。

五月丙子,以孫廷銓為大學士。乙酉,雲南開局鑄錢。丙戌,詔天下錢糧統歸戶部,部寺應用,俱向戶部關領,著為令。戊子,以魏裔介為吏部尚書。甲午,恭上大行慈和皇太后尊謚曰孝康慈和莊懿恭惠崇天育聖皇后。

六月,葬世祖章皇帝於孝陵,孝康皇后、端敬皇后祔焉。戊申,以龔鼎孳為左都御史。乙卯,故明將李定國子嗣興來降。乙丑,以哈爾庫為浙江提督。

八月癸卯,詔鄉、會試停制義,改用策論,復八旗繙譯鄉試。甲寅,命穆里瑪為靖西將軍,圖海為定西將軍,率禁旅會四川、湖廣、陜西總督討鄖陽逋賊李來亨、郝搖旗等。

冬十月壬寅,耿繼茂、施瑯會荷蘭師船剿海寇,克廈門,取浯嶼、金門二島,鄭錦遁於臺灣。

十一月,詔免諸國貢使土物稅。乙酉,冬至,祀天於圜丘。

十二月壬戌,祫祭太廟。

是歲,免直隸、江南、江西、河南、陜西、浙江、湖廣、四川、雲南、貴州等省二百七十餘州縣災賦。朝鮮入貢進香。

三年甲辰春正月,賜朝正外籓銀幣鞍馬。

二月壬寅,巡鹽御史張吉午請增長蘆鹽引。斥之。

三月丙子,耿繼茂等拔銅山。丙戌,賜嚴我斯等一百九十九人進士及第出身有差。

夏四月己亥,輔臣等誣奏內大臣飛揚古子侍衛倭赫擅騎禦馬,飛揚古怨望,並棄市,籍其家,鰲拜以予其弟穆里瑪。遣尚書喀蘭圖赴科爾沁四十七旗蒞盟。戊申,裁鄖陽撫治。

五月甲子,詔州縣私派累民,上官容隱者並罪之。

六月庚申,詔免順治十五年以前逋賦。

閏六月乙酉,以王弘祚為刑部尚書。丙戌,以漢軍京官歸入漢缺升轉。

秋七月丁未,以施瑯為靖海將軍,征臺灣。

八月甲戌,浙江總督趙廷臣疏報擒獲明臣張煌言。己卯,穆里瑪、圖海疏報進剿鄖陽茅麓山李來亨、郝搖旗,俱自焚,賊平。

九月癸丑,發倉粟賑給八旗莊田。乙卯,以查克旦為領侍衛內大臣。

十一月壬辰,冬至,祀天於圜丘。丁未,以魏裔介為大學士,杜立德為吏部尚書,王弘祚為戶部尚書,龔鼎孳為刑部尚書。

十二月戊午朔,日有食之。丙戌,祫祭太廟。是月,彗星見張宿、井宿、胃宿、奎宿,金星見,給事中楊雍建請修省。

是歲,免直隸、江南、江西、山東、陜西、浙江、福建、湖廣、貴州等省一百二十一州縣被災額賦有差。朝鮮入貢。

四年乙巳春正月壬辰,以郝惟訥為左都御史。己亥,停榷關溢額獎敘。辛丑,封承澤親王碩色子博翁果諾為惠郡王。致仕大學士洪承疇卒,予祭葬,謚文襄。

二月乙丑,太皇太后聖壽,免朝賀。己巳,吳三桂疏報剿平水西烏撒土司,擒其酋安坤、安重聖。丙戌,以星變詔臣工上言闕失。御史董文驥疏言大臣更易先皇帝制度,非是,宜一切復舊。

三月戊子,京師地震有聲。辛卯,金星晝見。以星變地震肆赦,免逋賦。山西旱,有司不以聞,下吏部議罪,免其積逋及本年額賦。壬辰,詔禁州縣預徵隔年稅糧。丙申,詔曰:「郡縣災荒,有司奏請蠲賦,而小民先期已完,是澤不下逮也。自今被災者,預緩徵額賦十之三。」甲辰,萬壽節,免朝賀。丙午,修歷代帝王廟。太常寺少卿錢綎請簡老成耆德博通經史者數人,出入侍從,以備顧問。

夏四月丙寅,詔凡災傷免賦者並免丁徭。戊辰,詔卿貳督撫員缺,仍廷推。

五月丁未,置直隸總督,兼轄山東、河南。裁貴州總督歸雲南,廣西總督歸廣東,江西總督歸江南,山西總督歸陜西,鳳陽、寧夏、南贛巡撫悉裁之。

六月乙丑,詔父子兄弟同役,給復一年。

秋七月己酉,吏部以山西徵糧如額,請議敘。詔曰:「曩以太原諸處旱災饑饉,督撫不以聞,議罪。會赦得原。豈可仍以催科報最。惟未被災之地方官,仍予紀錄。」

八月庚午,詔贓官遇赦免罪者,不許復職。

九月辛卯,冊赫舍里氏為皇后,輔臣索尼之孫女也。上太皇太后、皇太后尊號,加恩中外。

冬十月癸亥,上幸南苑校射行圍。甲戌,還宮。

十一月丁酉,祀天於圜丘。

十二月庚辰,祫祭太廟。

是歲,免直隸、江南、江西、山東、河南、浙江、廣東、貴州等省一百二十一州縣衛災賦有差。朝鮮、琉球、暹羅入貢。索倫、飛牙喀人來歸。

五年丙午春正月庚寅,以廣東旱,發倉穀七萬石賑之。以承澤親王碩色子恩克布嗣爵。

二月壬子朔,置平遠、大定、黔西三府。丁巳,以十二月中氣不應,詔求明歷法者。乙丑,詔自今漢軍官丁憂,準解任持三年喪。

三月,以胡拜為直隸總督。

五月丙午,以孫延齡為廣西將軍,接統定南部軍駐桂林。

六月庚戌朔,日有食之。癸酉,傅維麟病免,以郝惟訥為工部尚書。辛未,詔崇文門凡貨物出京者弛其稅。

秋七月庚辰朔,以硃之弼為左都御史。辛巳,琉球來貢,並補進漂失前貢。上嘉其恭順,命還之,自今非其國產勿以貢。

八月己酉,給事中張維赤疏請親政。

九月丁亥,上行圍南苑。癸卯,還宮。禮部尚書沙澄免。以梁清標為禮部尚書,龔鼎孳為兵部尚書,郝惟訥為刑部尚書,硃之弼為工部尚書。

冬十月,詔起範承謨為秘書院學士。

十一月丙申,輔臣鰲拜以改撥圈地,誣奏大學士管戶部尚書蘇納海、直隸總督硃昌祚、巡撫王登聯等罪,逮下獄。四大臣之輔政也,皆以勛舊。索尼年老,遏必隆闇弱,蘇克薩哈望淺,心非鰲拜所為而不能爭。鰲拜橫暴,又宿將多戰功,敘名在末,而遇事專橫,屢興大獄,雖同列亦側目焉。

十二月丙寅,鰲拜矯旨殺蘇納海、硃昌祚、王登聯。甲戌,祫祭太廟。

是歲,免直隸、江南、江西、河南、陜西、浙江、湖廣等省八十六州縣災賦有差。朝鮮、琉球入貢。

六年丁未春正月己丑,封世祖第二子福全為裕親王。丁酉,上幸南苑行圍。以明安達禮為禮部尚書。

二月癸亥,晉封故親王尼堪子貝勒蘭布為郡王。丁卯,以宗室公班布爾善為大學士。起圖海復為大學士。錫故總督李率泰一等男爵。

三月己亥,賜繆彤等一百五十人進士及第出身有差。

夏四月甲戌,加索尼一等公。甲子,江南民人沈天甫撰逆詩誣告人,誅之。被誣者皆不論。御史田六善言奸民告訐,於南人不曰「通海」,則曰「逆書」,北人不曰「於七黨」,則曰「逃人」,請鞫誣反坐。從之。

五月辛酉,吳三桂疏辭總理雲南、貴州兩省事。從之。

六月己亥,禁採辦楠木官役生事累民。

秋七月己酉,上親政,御太和殿受賀,加恩中外,罪非殊死,咸赦除之。是日,始御乾清門聽政。甲寅,命武職官一體引見。己未,輔臣鰲拜擅殺輔臣蘇克薩哈及其子姓。癸亥,賜輔臣遏必隆、鰲拜加一等公。

九月丙午,命修世祖實錄。

冬十月己卯,盛京地震有聲。

十一月丁未,冬至,祀天於圜丘。奉世祖章皇帝配饗。丁巳,加上太皇太后、皇太后徽號。

十二月丙戌,以塞白理為廣東水師提督。戊子,以馬爾賽為戶部增設尚書。戊戌,祫祭太廟。

是歲,免直隸、江南、江西、山東、山西、陜西、甘肅、浙江、福建、湖廣等省一百六十州縣災賦有差。朝鮮、荷蘭入貢。

七年戊申春正月戊申,以莫洛為山西陜西總督,劉兆麒為四川總督。戊午,加鰲拜、遏必隆太師。

二月辛卯,上幸南苑。

三月丁未,詔部院官才能卓越,升轉毋拘常調。

夏四月庚辰,浙江嘉善民鬱之章有罪遣戍,其子褒、廣叩閽請代。上並宥之。

五月壬子,以星變地震,下詔修省,諭戒臣工。

六月癸酉,金星晝見。丁亥,平南王尚可喜遣子之信入侍。

秋七月戊午,前漕運總督吳維華請徵市鎮間架錢,洲田招民出錢佃種。上惡其言利,下刑部議罪。庚申,以誇岱為滿洲都統。

八月壬申,戶部尚書王弘祚坐失察書吏偽印盜帑免。

九月庚子,以吳瑪護為奉天將軍,額楚為江寧將軍,瓦爾喀為西安將軍。壬寅,上將巡邊,侍讀學士熊賜履、給事中趙之符疏諫。上為止行,仍令遇事直陳。

冬十月,定八旗武職人員居喪百日,釋縞任事,仍持服三年。庚午,上幸南苑。

十一月癸丑,冬至,祀天於圜丘。

十二月癸酉,以麻勒吉為江南總督,甘文焜為雲南貴州總督,範承謨為浙江巡撫。癸巳,祫祭太廟。

是歲,免奉天、直隸、江南、山東、河南、浙江、陜西、甘肅等省二百十六州縣災賦有差。朝鮮、安南、暹羅入貢。

八年己酉春正月戊申,修乾清宮,上移御武英殿。

二月庚午,命行南懷仁推算歷法。庚午,上巡近畿。

三月辛丑,以直隸廢籓田地予民。

夏四月癸酉,衛周祚免,以杜立德為大學士。丁丑,上幸太學,釋奠先師孔子,講周易、尚書。丁巳,給事中劉如漢請舉行經筵。上嘉納之。

五月乙未,以黃機為吏部尚書,郝惟訥為戶部尚書,龔鼎孳為禮部尚書,起王弘祚為兵部尚書。戊申,詔逮輔臣鰲拜交廷鞫。上久悉鰲拜專橫亂政,特慮其多力難制,乃選侍衛、拜唐阿年少有力者為撲擊之戲。是日,鰲拜入見,即令侍衛等掊而縶之。於是有善撲營之制,以近臣領之。庚申,王大臣議鰲拜獄上,列陳大罪三十,請族誅。詔曰:「鰲拜愚悖無知,誠合夷族。特念效力年久,迭立戰功,貸其死,籍沒拘禁。」其弟穆里瑪、塞本得,從子訥莫,其黨大學士班布爾善,尚書阿思哈、噶褚哈、濟世,侍郎泰璧圖,學士吳格塞皆誅死。餘坐譴黜。其弟巴哈宿衛淳謹,卓布泰有軍功,免從坐。嗣敬謹親王蘭布降鎮國公。褫遏必隆太師、一等公。

六月丁卯,詔曰:「朕夙夜求治,念切民依。邇年水旱頻仍,盜賊未息,兼以貪吏朘削,民力益占,朕甚憫焉。部院科道諸臣,其以民間疾苦,作何裨益,各抒所見以聞。」戊辰,敕改造觀象臺儀器。壬申,詔復輔臣蘇克薩哈官及世職,其從子白爾圖立功邊徼,被枉尤酷,復其世職,均令其子承襲。戊寅,詔滿兵有規占民間房地者,永行禁止,仍還諸民。以米思翰為戶部尚書。戊子,詔宗人有罪,遽絕屬籍,心有不忍。自順治十八年以來,宗人削籍者,宗人府詳察以聞。

秋七月壬辰朔,裁直隸山東河南總督。壬寅,詔復大學士蘇納海、總督硃昌祚、巡撫王登聯原官,並予謚。

八月甲申,以索額圖為大學士,明珠為左都御史。

九月甲午,京師地震有聲。丁未,以勒貝為滿洲都統,塞白理為浙江提督,畢力克圖為蒙古都統。

冬十月甲子,上幸南苑,詔行在勿得借用民物。盧溝橋成,上為文勒之石。

十一月己亥,先是山西陜西總督莫洛、陜西巡撫白清額均坐鰲拜黨罷。至是,西安百姓叩閽稱其清廉,乞還任。詔特許之。壬子,太和殿、乾清宮成,上御太和殿受賀,入居乾清宮。

十二月己卯,顯親王福壽薨。丁亥,祫祭太廟。

是歲,免直隸、江南、河南、山西、陜西、湖廣等省四十五州縣災賦有差。朝鮮、琉球入貢。

九年庚戌春正月丙申,予宋儒程顥、程頤後裔五經博士。丁酉,饗太廟。辛丑,祈穀於上帝,奉太祖高皇帝、太宗文皇帝、世祖章皇帝配饗。起遏必隆公爵,宿衛內廷。己酉,詔明籓田賦視民田輸納。壬子,上幸南苑。

二月癸酉,以金光祖為廣東廣西總督,馬雄鎮為廣西巡撫。癸未,詔尚陽堡、寧古塔流徙人犯,值十月至正月俱停發。

三月辛酉,賜蔡啟僔等二百九十二人進士及第出身有差。

夏四月己丑,以蔡毓榮為四川湖廣總督。己亥,上幸南苑。

五月丙辰朔,加上孝康章皇后尊謚,升祔太廟,頒發恩詔,訪隱逸,賜高年,赦殊死以下。丙子,纂修會典。

六月丙戌朔,以席卜臣為蒙古都統。丁酉,以故顯親王福壽子丹臻襲爵。己酉,命大學士會刑部錄囚。

秋七月丁巳,以王輔臣為陜西提督。丁巳,奉祀孝康章皇后於奉先殿。

八月戊子,祭社稷壇。詔都察院糾察陪祀王大臣班行不肅者。乙未,復內閣,復翰林院。丁酉,上奉太皇太后、皇太后有事於孝陵。壬子,車駕還宮。

九月庚申,以簡親王濟度子喇布襲爵。

冬十月庚巳,頒聖諭十六條。甲午,改內三院,復中和殿、保和殿、文華殿大學士。丁酉,諭禮部舉經筵。

十一月癸酉,以艾元徵為左都御史。壬午,以中和殿大學士魏裔介兼禮部尚書。

十二月癸卯,以莫洛為刑部尚書。辛亥,祫祭太廟。

是歲,免河南、湖廣、江南、福建、廣東、雲南等省二百五十三州縣衛災賦有差。朝鮮入貢。

十年辛亥春正月丁卯,蒙古蘇尼特部、四子部大雪饑寒,遣官賑之。癸酉,封世祖第五子常寧為恭親王。庚辰,大學士魏裔介罷。以曹申吉為貴州巡撫。

二月丁酉,以馮溥為大學士,以梁清標為刑部尚書。乙巳,召宗人覺羅年七十以上趙班等四人入見,賜朝服銀幣。戊申,命編纂孝經衍義。庚戌,以尼雅翰為滿洲都統。

三月壬子朔,誥誡年幼諸王讀書習騎射,勿恃貴縱恣。癸丑,置日講官。庚午,以無雨風霾,下詔修省。

夏四月乙酉,命纂修太祖、太宗聖訓。詔宗人閒散及幼孤者,量予養贍,著為令。丙戌,詔清理庶獄,減矜疑一等。辛卯,始開日講。壬辰,上詣天壇禱雨。甲午,雨。

五月庚申,理籓院尚書喀蘭圖乞休,加太子太保,以內大臣奉朝請。癸酉,上幸南苑。

六月丁亥,以靳輔為安徽巡撫。甲午,金星晝見。是月,靖南王耿繼茂卒,子精忠襲封,仍鎮福建。

八月己卯朔,日有食之。丁未,上御經筵。戊申,以王之鼎為江南提督。

九月庚戌,上以寰宇一統,告成於二陵。辛亥,上奉太皇太后、皇太后啟鑾。蒙古科爾沁、喀喇沁、土默特、敖漢諸部王、貝勒、公朝行在。丁卯,謁福陵、昭陵。戊辰,祭福陵,行告成禮。庚午,祭昭陵,行告成禮。辛未,上幸盛京,御清寧宮,賜百官宴,八十以上召前賜酒。大賚奉天、寧古塔甲士及於傷廢老病者白金,民間高年亦如之。曲赦死罪減一等,軍流以下釋之。山海關外蹕路所經,勿出今年明年租賦。遣官祭諸王諸大臣墓。壬申,上自盛京東巡。

冬十月辛巳,駐蹕愛新。召寧古塔將軍巴海,諭以新附瓦爾喀、虎爾哈宜善撫之。己丑,上回蹕盛京,再賜老人金。辛卯,謁福陵、昭陵。命文武官較射。命來朝外籓較射。壬辰,上奉太皇太后、皇太后回鑾。

十一月庚戌,還京。壬申,以明珠為兵部尚書。

十二月丙午,祫祭太廟。

是歲,免直隸、江南、江西、浙江、山東、河南、陜西、湖廣等省三百二州縣衛災賦逋賦有差。朝鮮、琉球入貢。

十一年壬子春正月辛未,上奉太皇太后幸赤城湯泉,過八達嶺,親扶慈輦,步行下山。

二月戊寅,奉太皇太后至湯泉。辛卯,上回京。丙申,親耕耤田。丁酉,朝日於東郊。戊戌,上詣赤城。

三月戊辰,上奉太皇太后還宮。

夏四月乙巳,命侍衛吳丹、學士郭廷祚巡視河工。

五月乙丑,世祖實錄成。丙寅,上出德勝門觀麥。

六月庚寅,命更定賦役全書。

秋七月己酉,論征緬甸、雲南、貴州功,予何建忠等一百二十七人世職。丙辰,上觀禾。御史孟雄飛疏言孫可望窮蹙來歸,濫膺王封。及伊身死,已襲二次。今孫徵淳死,宜令降襲。詔降襲慕義公。

閏七月,復封尚善為貝勒。丁亥,詔治獄勿用嚴刑輕斃人命,違者罪之。

八月壬子,上幸南苑行圍。癸丑,詔曰:「帝王致治,在維持風化,辨別等威。比來官員服用奢僭,競相效尤。其議禁之。」庚申,上御經筵。壬戌,上奉太皇太后幸遵化湯泉。甲子,閱薊州官兵較射。丁卯,上謁孝陵。

九月丁丑,閱遵化兵、三屯營兵。

冬十月甲辰,上奉太皇太后還宮。壬子,命範承謨為福建總督。

十一月辛丑,上幸南苑,建行宮。

十二月丁未,裕親王福全、莊親王博果鐸、惠郡王博翁果諾、溫郡王孟峨疏辭議政。允之。戊午,上召講官諭曰:「有人請令言官風聞言事。朕思切中事理之言,患其不多。若借端生事,傾陷擾亂,深足害政。與民休息,道在不擾。虛耗元氣,則民生蹙矣。」己未,康親王傑書、安親王岳樂疏辭議政。不許。庚午,祫祭太廟。

是歲,免直隸、江南、浙江、山東、山西、河南、湖廣等省一百四十一州縣衛災賦有差。朝鮮入貢。

十二年癸丑春正月庚寅,上幸南苑,大閱。

二月辛亥,以吳正治為左都御史。壬子,上御經筵,命講官日直。戊辰,賜八旗官學繙譯大學衍義。

三月丁丑,上視麥。壬午,平南王尚可喜請老,許之;請以其子之信嗣封鎮粵,不許,令其撤籓還駐遼東。癸巳,賜韓菼等一百六十六人進士及第出身有差。

夏四月丁巳,遣官封暹羅國王。

五月壬申,學士傅達禮等請以夏至輟講。上曰:「學問之道,宜無間斷。其勿輟。」

六月壬寅,起張朝珍為湖廣巡撫,李之芳為浙江總督。丁未,上御瀛臺,召群臣觀荷賜宴。乙卯,禁八旗以奴僕殉葬。

秋七月庚午,平西王吳三桂疏請撤籓。許之。丙子,嗣靖南王耿精忠疏請撤籓。許之。壬午,命重修太宗實錄。

八月丁未,試漢科道官於保和殿,不稱職者罷。壬子,遣侍郎折爾肯、學士傅達禮往雲南,尚書梁清標往廣東,侍郎陳一炳往福建,經理撤籓。丁巳,諭禮部:「祭祀大典,必儀文詳備,乃可昭格。其稽古典禮酌議以聞。」

九月戊辰,禮部尚書龔鼎孳乞休。允之。乙亥,京師地震,詔修省。

冬十月壬寅,以王之鼎為京口將軍。己酉,上幸南苑行圍。

十一月丁卯,故明宗室硃議潀以蓄發論死。得旨免死入旗,給與妻室房地。庚午,詔民間墾荒田畝,以十年起科。

十二月壬子,以姚文然為左都御史。吳三桂反,殺雲南巡撫硃國治,貴州提督李本深、巡撫曹申吉俱降賊,總督甘文焜死之。丙辰,反問至,命前鋒統領碩岱率禁旅守荊州。丁巳,召梁清標、陳一炳還,停撤二籓。命加孫延齡撫蠻將軍,線國安為都統,鎮廣西。命西安將軍瓦爾喀進守四川。京師民楊起隆偽稱硃三太子,圖起事。事發覺,起隆逸去。捕誅其黨。詔奸民作亂已平,勿株連,民勿驚避。己未,命順承郡王勒爾錦為寧南靖寇大將軍,討吳三桂。執三桂子額駙吳應熊下之獄。庚申,命副都統馬哈達帥師駐兗州,擴爾坤駐太原,備調遣。辛酉,命直省巡撫仍管軍務。壬戌,詔削吳三桂爵,宣示中外。命都統赫業為安西將軍,會瓦爾喀守漢中。以倭內為奉天將軍。吳三桂陷辰州。甲子,祫祭太廟。

是歲,免直隸、山東、安徽、浙江、湖廣等省二十六州縣衛災賦有差。朝鮮、安南入貢。

十三年甲寅春正月乙亥,勒爾錦師行。庚辰,吳三桂陷沅州。丁亥,偏沅巡撫盧震棄長沙遁。己丑,以提督佟國瑤守鄖陽。總兵吳之茂以四川叛,巡撫羅森、提督鄭蛟麟降之。命總兵徐治都還守夷陵。庚寅,封世祖第七子隆禧為純親王。以席卜臣為鎮西將軍,守西安。

二月乙未朔,太皇太后頒內帑犒軍。丁酉,欽天監新造儀象成。壬寅,賊犯澧州,守卒以城叛,提督桑峨退荊州,陷常德。命鎮南將軍尼雅翰率師守武昌。癸丑,上御經筵。以趙賴為貴州提督。甲寅,吳三桂陷長沙,副將黃正卿叛應之,旁陷衡州。命都統覺羅硃滿守岳州,未至,岳州失。辛酉,命刑部尚書莫洛加大學士銜,經略陜西。孫延齡以廣西叛,殺都統王永年,執巡撫馬雄鎮幽之。

三月乙丑,命整飭驛站,每四百里置一筆帖式,接遞軍報,探發塘報。命左都御史多諾等軍前督餉。戊辰,吳三桂將犯夷陵,勒爾錦遣兵擊敗之。庚午,以額駙華善為安南將軍,鎮京口。庚辰,耿精忠反,執福建總督範承謨幽之,巡撫劉秉政降賊。癸未,鄖陽副將洪福叛,提督佟國瑤擊敗之。壬辰,襄陽總兵楊來嘉以穀城叛。命希爾根為定南將軍,尚書哈爾哈齊副之。命舒恕、桑遏、根特、席布率師赴江西。甲午,西安將軍瓦爾喀克陽平關。

夏四月癸卯,調西安副都統德業立守襄陽。丁未,吳三桂子應熊、孫世霖伏誅。初,三桂倉卒起兵,而名義不揚,中悔。至澧州,頗前卻。至是,方食聞報,驚曰:「上少年乃能是耶?事決矣!」推食而起。詔削孫延齡職。以阿密達為揚威將軍,駐江寧,賴塔為平南將軍,赴杭州。甲寅,潮州總兵劉進忠以城叛。戊午,以根特為平寇將軍,赴廣西討孫延齡。河北總兵蔡祿謀叛,命阿密達襲誅之。辛酉,詔削耿精忠爵。癸亥,詔以分調禁旅遣將分防情形寄示平南王尚可喜。

五月丙寅,皇子胤礽生,皇后赫舍里氏崩。戊寅,安西將軍赫業等敗吳之茂於劄閣堡,復朝天關。壬午,浙江平陽兵變,執總兵蔡朝佐,應耿精忠將曾養性,圍瑞安。命賴塔進兵討之。壬辰,副都統德業立敗洪福於武當。

六月丙午,命貝勒尚善為安遠靖寇大將軍,率師赴岳州,貝子準達赴荊州。庚戌,總兵祖弘勛以溫州叛。金華副將牟大寅敗耿精忠將於常山。壬子,命將軍喇哈達守杭州。乙卯,命康親王傑書為奉命大將軍赴浙江,貝勒洞鄂為定西大將軍赴四川。浙江溫州、黃巖、太平諸營相繼叛。命喇哈達守臺、寧。

七月辛未,以郎廷佐為福建總督,段應舉為提督。癸酉,賴塔敗耿精忠將於金華。是時精忠遣其大將馬九玉、曾養性犯浙江,白顯忠犯江西,所至土匪蜂應,江西尤甚。南瑞總兵楊富應賊,董衛國誅之。丁亥,貝勒察尼大戰賊將吳應麒於岳州七里山,敗之。

八月壬寅,平寇將軍根特卒於軍,以哈爾哈齊代之。海澄公黃梧卒,子芳度襲爵,守漳州。乙巳,金光祖報孫延齡陷梧州,督兵復之。丙午,上幸南苑。

九月壬戌,上御經筵,命每日進講如常。耿精忠將以土寇陷清谿、徽州,江寧將軍額楚、統領巴爾堪擊走之,連戰入江西,復樂平等縣。命碩塔等駐安慶。辛未,麻城土寇鄒君升等作亂,知府於成龍討平之。命簡親王喇布為揚威大將軍,率師赴江西,侍衛坤為振武將軍副之。廣西提督馬雄叛,命安親王岳樂為定遠平寇大將軍,率師赴廣東,宗室瓦山、覺羅畫特副之。

冬十月壬辰,喇布師行。丙申,岳樂師行。壬寅,上奉太皇太后幸南苑。辛亥,還宮。

十一月庚申朔,莫洛報吳之茂兵入朝天關,饟路中阻,洞鄂退守西安。命移西安軍守漢中,河南軍守西安。

十二月庚寅朔,傑書大敗曾養性於衢州,又敗之於臺州。王輔臣叛,經略莫洛死之。上議親征。王大臣以京師根本重地,太皇太后年高,力諫乃止。徵盛京兵、蒙古兵分詣軍前。丁未,命尚可喜節制廣東軍事。戊午,祫祭太廟。

是歲,免直隸、江南、山東、河南、陜西等省七十八州縣災賦有差。朝鮮、琉球入貢。

十四年乙卯春正月辛酉,尚可喜報賊犯連州,官兵擊敗之。戊辰,晉封尚可喜平南親王,命其子之孝佩大將軍印討賊。

二月癸巳,下詔切責貝勒洞鄂退縮失機,飭令速定平涼、秦州以通棧道。乙巳,康親王傑書遣兵復處州,進復仙居。王輔臣陷蘭州。西寧總兵王進寶大戰於新城,圍蘭州。洞鄂復隴州關山關。

三月己未朔,叛將楊來嘉犯南漳,總兵劉成龍擊走之。戊辰,饒州賊犯祁門,巡檢張行健被執不屈,死之。丁丑,命張勇為靖逆將軍,會總兵孫思克等討王輔臣。賊陷定邊城,命提督陳福駐寧夏討賊。丁亥,蒙古布爾尼反,命信郡王鄂扎為撫遠大將軍,大學士圖海為副將軍,討平之。戊子,以熊賜履為大學士。

夏四月己丑,以勒德洪為戶部尚書。署護軍統領郎肅等剿耿寇於五桂寨,斬級二萬,復餘干。乙未,封張勇靖逆侯,王進寶一等男。戊戌,以左都督許貞鎮撫州、建昌、廣信。戊申,王輔臣遣兵援秦州,官兵迎擊敗之。辛亥,上諭:「侍臣進講,朕乃覆講,互相討論,庶有發明。」癸丑,王進寶復臨洮,孫思克復靖遠。戊午,紹興知府許弘勛招撫降眾五萬人。

五月庚午,察哈爾左翼四旗來歸。庚辰,命畢力克圖援榆林。王輔臣兵陷延安、綏德。甲申,張勇復洮、河二州。

閏五月癸巳,上幸玉泉山觀禾。楊來嘉、洪福陷穀城。斬守城不力之副將馬郎阿以徇,削總兵金世需職,隨軍效力。壬子,額楚復廣信。樂平土寇復陷饒州,將軍希爾根擊之,復饒州。

六月,畢力克圖復吳堡,復綏德。丁丑,命將軍舒恕援廣東。己卯,命振武將軍佛尼勒開棧道援漢中。庚辰,上幸南苑行圍。壬午,張勇攻鞏昌。江西官軍攻石峽,失利,副都統雅賴戰死。甲申,克蘭州。畢力克圖復延安。以軍興停陜西、湖廣鄉試。

七月乙巳,陳福剿定邊,斬賊將硃龍。庚戌,江西官兵復浮梁、樂平、宜黃、崇仁、樂安諸縣。

八月戊午,上幸南苑行圍。洞鄂、畢力克圖、阿密達會攻王輔臣,斬賊將郝天祥。傅喇塔復黃巖。壬申,上奉太皇太后幸湯泉。甲申,上還京,御經筵。

九月,上次昌平,詣明陵,致奠長陵,遣官分奠諸陵。丙申,上奉太皇太后還宮。辛丑,詔每歲正月停刑,著為令。

冬十月癸亥,康親王兵復太平、樂清諸縣。丙寅,上謁孝陵。戊辰,祭孝陵。乙亥,還宮。陳福及王輔臣戰於固原,不利,副將太必圖戰沒。論平布爾尼功,封賞有差,及助順蒙古王貝勒沙津以次各晉爵,罰助逆奈曼等部。

十一月癸巳,貝勒察尼復興山。丁酉,復設詹事府官。壬寅,叛將馬雄糾吳三桂兵犯高州,連陷廉州。命簡親王喇布自江西援廣東。是月,鄭錦攻陷漳州,海澄公黃芳度死之,戕其家。

十二月丙寅,立皇子胤礽為皇太子,頒詔中外,加恩肆赦。乙亥,以勒爾錦師久無功,奪其參贊巴爾布以下職。寧夏兵變,提督陳福死之。壬午,祫祭太廟。

是歲,免湖廣、河南七府五州縣災賦有差。朝鮮入貢。

十五年丙辰春正月丁亥,以王進寶為陜西提督,駐秦州。甲午,以建儲恭上太皇太后、皇太后徽號。乙未,升寧夏總兵官為提督,以趙良棟為之。辛丑,上幸南苑行圍。

二月丁巳,詔軍中克城禁殺掠。壬戌,命大學士圖海為撫遠大將軍,統轄全秦,自貝勒洞鄂以下咸受節制。癸酉,上如鞏華城,諭扈從勿踐春田。乙亥,吳三桂將高大傑陷吉安。戊寅,安親王岳樂擊三桂將於萍鄉,敗之,復萍鄉。辛巳,上御經筵。贈死事副將張國彥太子太保,予世職。

三月癸未,贈海澄公黃芳度郡王。丙戌,王進寶、佛尼勒大敗吳之茂於北山。庚寅,傅喇塔圍溫州,曾養性、祖弘勛悉眾來犯,副都統紀爾他布擊走之。辛卯,岳州水師克君山。庚子,勒爾錦渡江與三桂之眾戰,迭敗之。乙巳,賜彭定求等二百九人進士及第出身有差。己酉,勒爾錦與三桂之眾戰於太平街,不利,退守荊州。壬子,移趙賴提督江西。

夏四月辛丑,馬雄、祖澤清糾滇賊犯廣東。尚可喜老病不能軍,屢疏告急,援兵不時至。至是,賊逼廣州,尚之信劫其父以降賊。總督金光祖,巡撫佟養鉅、陳洪明,提督嚴自明俱從降。福建巡撫楊熙、總兵拜音達奪門出。舒恕、莽依圖退至江西。上聞廣東變作,命移兵益江西。

五月壬午朔,日有食之。乙酉,復設鄖陽撫治,以楊茂勛任之。丙戌,鄂羅斯察漢汗使人來貢。己亥,撫遠大將軍圖海敗王輔臣於平涼。

六月壬子朔,王輔臣降,圖海以聞。詔復其官,授靖寇將軍,立功自效,諸將弁皆原之。己卯,耿繼善棄建昌遁。上諭傑書曰:「耿精忠自撤其兵,顯為海寇所逼。其乘機速進。」

七月辛巳朔,賜鄂羅斯使臣鞍馬服物。大學士熊賜履免。以慕天顏為江蘇巡撫。庚子,以姚文然為刑部尚書,郎廷相為福建總督。振武將軍佛尼勒會張勇、王進寶擊吳之茂於秦州,大敗之,賊眾宵遁。

八月甲寅,穆占復禮縣。壬戌,上奉太皇太后幸湯泉。乙亥,賴塔擊馬九玉於衢州,復江山,九玉棄軍遁。

九月庚辰朔,賴塔進擊馬九玉,破之,復常山。進攻仙霞關,賊將金應虎迎降,復浦城,連下建寧。癸未,張勇復階州。乙未,耿精忠戕前總督範承謨。山西巡撫達爾布有罪免。丙午,命穆占為征南將軍,移軍湖廣。

冬十月辛酉,上奉太皇太后還宮。乙丑,康親王傑書師次延平,賊將耿繼美以城降。耿精忠遣子顯祚獻偽印乞降,傑書入福州,疏聞。上命復其爵,從征海寇自效。其將曾養性、叛將祖弘勛俱降。浙江官兵復溫、處二府。撤兗州屯兵。癸酉,命講官進講通鑒。

十一月丙戌,海寇犯福州,都統喇哈達擊敗之。丙申,官兵圍長沙。寧海將軍貝子傅拉塔卒於軍。

十二月壬子,遣耿昭忠為鎮平將軍,駐福州,分統靖南籓軍。叛將嚴自明犯南康,舒恕擊走之。丁巳,尚之信使人詣簡親王軍前乞降,且乞師,疏聞。許之。吳三桂將吳世琮殺孫延齡,踞桂林。庚申,海澄公黃芳世自賊中脫歸。上嘉之,加太子太保,與其弟黃藍並赴康親王大軍討賊。建威將軍吳丹復山陽。辛未,頒賞諸軍軍士金帛。丙子,祫祭太廟。耿繼善棄邵武,海寇據之。副都統穆赫林擊之,賊將彭世勛以城降。

是歲,免直隸、江南、江西、陜西各省三十四州縣災賦有差。朝鮮入貢。

十六年丁巳春正月丙申,將軍額楚攻吉安失利,命侍郎班迪馳勘軍狀。

二月己未,上幸南苑行圍。甲子,大閱於南苑。免福建今年租賦,招集流亡。丙寅,以鄂內為討逆將軍,赴岳州。丁卯,康親王傑書敗鄭錦於興、泉,賊棄漳州遁,復海澄。遣郎中色度勞軍岳州,察軍狀。辛未,以靳輔為河道總督。癸酉,論花馬池剿寇功,蒙古鄂爾多斯貝勒索諾木等晉爵有差。乙亥,上御經筵。是月江西官軍復瑞金、鉛山。

三月甲申,以莽依圖為鎮南將軍,督兵廣東。己丑,諭禮部:「帝王克謹天戒,凡有垂象,皆關治理。設立專官,謹司占候。今星辰凌犯,霜露非時,欽天監不以實告,有辜職掌。其察議以聞。」庚寅,命翰林長於詞賦書法者,以所業進呈。乙未,原任總兵劉進忠、苗之秀詣康親王軍降,命隨大軍剿賊。癸未,詔:「軍興以來,文武官身殉封疆,克全忠節,其有旅親不能歸,妻子不得養者,深堪軫惻。所在疆吏察明,妥為資送,以昭褒忠至意。」甲辰,含譽星見,慶雲見。乙巳,吳三桂聚兵守長沙。命勒爾錦進臨江,圖海守漢中,喇布鎮吉安,莽依圖進韶州,額楚駐袁州,舒恕防贛州。

夏四月己未,康親王傑書疏言處州府慶元縣民人吳臣任等不肯從賊,結寨自固,守義殺賊,實為可嘉。已交浙江督撫,效力者錄用,歸農者獎賞,其陣亡札委守備吳受南等並請恩恤。從之。辛酉,上幸霸州行圍。以伊桑阿為工部尚書,宋德宜為左都御史。丁卯,提督趙賴敗土寇於泰和,擒賊目蕭元。戊辰,予死事溫處道陳丹赤等官廕。辛未,上制大德景福頌,書屏,上太皇太后。乙亥,莽依圖師至南安,嚴自明以城降,遂克南雄,入韶州。

五月己卯,尚之信降,命復其爵,隨大軍討賊。特擢謫戍知府傅弘烈為廣西巡撫。先是,弘烈以首吳三桂反狀謫梧州。及兵起,弘烈上書陳方略,故有是命。旋加授撫蠻滅寇將軍,與莽依圖規取廣西。甲午,額魯特噶爾丹攻敗喀爾喀車臣汗,來獻軍實,卻之。

六月丁巳,祖澤清以高州降。

秋七月庚子,鄭錦將劉國軒自惠州犯東莞,尚之信大敗之,賊將陳璉以惠州降。甲辰,上御便殿,召大學士等賜坐,論經史,因及前代朋黨之弊,諭加警戒。以明珠、覺羅勒德洪為大學士。

八月丁未,明宗人硃統琛起兵陷貴溪、瀘溪。己未,上御經筵。丙寅,冊立貴妃鈕祜祿氏為皇后,佟佳氏為貴妃。戊辰,傅弘烈等復梧州。

九月丙子,命宗室公溫齊、提督周卜世赴湖廣協剿。癸未,命額駙華善率師益簡親王軍,科爾科代接駐江寧。丁亥,上發京師,謁孝陵,巡近邊。丙申,次喀拉河屯。庚子,次達希喀布秦昂阿,近邊蒙古敖漢部札穆蘇等朝行在,敵駝馬,賜金幣。吳三桂將胡國柱、馬寶寇韶州,將軍莽依圖、額楚夾擊破之,賊遁,追之過樂昌,復仁化。

冬十月甲辰,上次湯泉。癸丑,還宮。傅弘烈敗吳世琮於昭平,復潯州。福建按察使吳興祚敗硃統琛於光澤,其黨執統琛降。癸亥,始設南書房,命侍講學士張英、中書高士奇入直。

十一月己卯,吳三桂將韓大任陷萬安,護軍統領哈克山擊敗之。庚子,封長白山神,遣官望祭。是月,官兵復茶陵、攸縣。

十二月乙巳,海寇犯泉州,提督段應舉等御之。辛亥,海寇犯欽州,游擊劉士貴擊敗之。命參贊勒貝、將軍額楚進取郴、永。己巳,以馮甦為刑部侍郎。辛酉,金星晝見。辛未,祫祭太廟。

是歲,免直隸、江南、江西、陜西、湖廣等省七十州縣災賦有差。朝鮮入貢。

十七年戊午春正月己丑,副都統哈當、總兵許貞擊韓大任於寧都,大任遁之汀州,詣康親王軍前降,命執送京師。壬辰,以郭四海為左都御史。乙未,詔曰:「一代之興,必有博學鴻儒振起文運,闡發經史,以備顧問。朕萬幾餘暇,思得博通之士,用資典學。其有學行兼優、文詞卓越之士,勿論已仕未仕,中外臣工各舉所知,朕將親試焉。」於是大學士李霨等薦曹溶等七十一人,命赴京齊集請旨。

二月甲辰,傅弘烈疏言吳三桂兵犯廣西,詔額楚、勒貝守梧州。己未,上御經筵,制四書講疏義序。丁卯,皇后鈕祜祿氏崩,謚曰孝昭皇后。辛未,莽依圖及吳世琮戰於平樂,失利,退守梧州。命尚之信及都統馬九玉會師守梧州。

三月丙子,湖廣官兵擊楊來嘉、洪福,敗之,復房縣。丁丑,海寇犯石門,黃芳世擊敗之。癸巳,祖澤清復叛應吳三桂。

閏三月癸卯,上巡近畿。乙丑,命內大臣喀代、尚書馬喇往科爾沁四十九旗蒞盟。丁卯,吳三桂將林興珠詣安親王軍前降,詔封建義侯,隨軍剿賊。逮問副都統甘度海、阿進泰,以在江西剿賊失機也。

夏四月庚午,海寇蔡寅陷平和,進逼潮州。甲戌,祖澤清犯電白,尚之信、額楚擊之,澤清遁。庚寅,慶陽土賊袁本秀作亂,官兵擊斬之。

五月庚子朔,海澄公黃芳世卒於軍,命其弟芳泰襲爵。戊申,福建總督郎廷相、巡撫楊熙、提督段應舉俱免,以姚啟聖為福建總督,吳興祚為福建巡撫,楊捷為福建水陸提督。甲寅,上幸西郊觀禾。額魯特部濟農為噶爾丹所逼,入邊,張勇逐出之。

六月壬申,尚善遣林興珠敗三桂舟師於君山。丁亥,上以盛夏亢旱,步禱於天壇。是日,大雨。壬辰,吳三桂將犯永興,都統伯宜理布、統領哈克山與戰,敗歿。海寇犯廉州,總兵班紹明等擊走之。吳三桂兵犯郴州,副都統碩岱與戰,不利,奔永興。丁酉,詔曰:「軍興以來,將士披堅執銳,盛暑祁寒,備極勞苦,朕甚憫焉。其令兵部察軍中有負債責者,官為償之,戰歿及被創者恤其家。」

秋七月,鄭錦陷海澄,前鋒統領希佛、副都統穆赫林、提督段應舉死之。甲辰,鄭錦犯泉州。甲寅,以安珠護為奉天將軍。壬戌,以魏象樞為左都御史。丙寅,召翰林院學士陳廷敬、侍讀學士葉方藹入直南書房。是月,吳三桂僭號於衡州。

八月己卯,安遠靖寇大將軍、貝勒尚善卒於軍,命貝勒察尼代之。庚午,西洋國王阿豐肅使臣入貢。癸未,上御經筵,以禦制詩集賜陳廷敬等。乙未,吳三桂死,永興圍解。頒行康熙永年歷。丙申,詔曰:「逆賊倡亂,仰服天誅。絓誤之徒,宜從寬典。其有悔悟來歸者,咸與勿治。」

九月,上奉太皇太后幸湯泉,晉謁孝陵。姚啟聖、拉哈達大敗海寇於蜈蚣山,劉國軒遁,泉州圍解。

冬十月癸未,上巡近邊,次灤河,閱三屯營兵。己丑,將軍鄂內敗吳應麒於石口。丁酉,皇四子胤禛生,是為世宗,母曰吳雅氏。

十一月己亥,拉哈達疏言海賊斷江東橋,兵援泉州難進。在籍侍讀學士李光地為大軍鄉導,修通險路,接濟軍需,請議敘。得旨:「李光地前當變亂之初,密疏機宜。茲又迎接大兵,備辦糧米,深為可嘉。即升授學士。」辛酉,上奉太皇太后還宮。癸亥,命福建陸路提督楊捷加昭武將軍,王之鼎為福建水師提督。

十二月丁亥,額楚、傅弘烈及吳世琮戰於藤縣,不利,退守梧州。乙未,祫祭太廟。

是歲,免直隸、江南、江西、湖廣等省七十州縣災賦有差。朝鮮、西洋入貢。

十八年己未春正月戊申,遣官分賑山東、河南。甲寅,貝勒察尼督水師圍岳州,賊將吳應麒遁,復岳州。上御午門宣捷。設隨征總兵官以處降將,旋裁之。壬戌,劉國軒犯長樂,總督姚啟聖偕紀爾他布、吳興祚擊敗之。甲子,岳樂復長沙。

二月丙寅,傅弘烈戰吳世琮於梧州,賊遁。己巳,詔數江西奸民從逆之罪,仍免其逋賦。甲戌,順承郡王勒爾錦督兵過江,分復松滋、枝江、宜都、澧州,叛將洪福以舟師降。戊寅,簡親王喇布遣前鋒統領希佛復衡州,賊將吳國貴、夏國相遁。庚辰,詔軍前王大臣議進取雲、貴事宜。以周有德為雲貴總督,桑峨為雲南提督,趙賴為貴州提督,並隨王師進討。以楊雍建為貴州巡撫。癸未,以誇扎為蒙古都統。

三月丙申朔,御試博學鴻詞於保和殿,授彭孫遹等五十人侍讀、侍講、編修、檢討等官。修明史,以學士徐元文、葉方藹、庶子張玉書為總裁。丁酉,上幸保定縣行圍。甲辰,以徐治都為湖廣提督。將軍穆占擊吳國貴於永州,敗之,復永州、道州、永明。己酉,上還宮。戊午,賜歸允肅等百五十一人進士及第出身有差。庚申,岳州陣歿諸將喪至,遣侍衛迎奠。福建陣沒將士喪至亦如之。

夏四月丙寅,以楊茂勛為四川總督,駐鄖陽。戊辰,以萬正色為福建水師提督。己卯,旱甚,上步禱於天壇。是日,大雨。莽依圖擊吳世琮於潯州,敗走之。壬寅,上出阜成門觀禾。

五月庚戌,劉國軒犯江東橋,賴塔大戰敗之。

六月辛未,詔曰:「盛治之世,餘一餘三。蓋倉廩足而禮教興,水旱乃可無虞。比聞小民不知積蓄,一逢歉歲,率致流移。夫興儉化民,食時用禮,惟良有司是賴。督撫等其選吏教民,用副朕意。」己卯,以希佛為蒙古都統。

秋七月甲午,靳輔疏報淮揚壩工成,涸出田地,招民種之。丁未,上視純親王隆禧疾。隆禧薨。乙卯,額楚敗吳世琮於南寧,世琮遁。庚申,京師地震,詔發內帑十萬賑恤,被震廬舍官修之。壬戌,召廷臣諭曰:「朕躬不德,政治未協,致茲地震示警。悚息靡寧,勤求致災之由。豈牧民之官苛取以行媚歟?大臣或朋黨比周引用私人歟?領兵官焚掠勿禁歟?蠲租給復不以實歟?問刑官聽訟或枉平民歟?王公大臣未能束其下致侵小民歟?有一於此,皆足致災。惟在大法而小廉,政平而訟理,庶幾仰格穹蒼,弭消沴戾。用是昭布朕心,原與中外大小臣工共勉之。」

八月癸亥朔,將軍穆占復新寧。甲子,傅弘烈復柳城、融縣。庚辰,提督趙國祚、將軍林興珠大破吳國貴於武岡,國貴死,復武岡州。

九月庚戌,以地震禱於天壇。辛亥,命簡親王喇布守桂林。甲寅,金光祖執叛鎮祖澤清送京,及其子良楩磔誅之。

冬十月辛未,詔將軍張勇、王進寶,提督趙良棟、孫思克取四川。王進寶、趙良棟行。癸未,王進寶克武關,復鳳縣。趙良棟復兩當。

十一月戊戌,王進寶擊叛將王屏籓,遁之廣元,復漢中。庚子,趙良棟復略陽,進克陽平關。丁酉,以許貞為江西提督。

十二月壬戌,以蔡毓榮為綏遠將軍,進定雲、貴。將軍佛尼勒、吳丹克梁河關,賊將韓晉卿遁,復興安、平利、紫陽、石泉、漢陰、洵陽、白河及鄖陽之竹山、竹溪。丁卯,上幸南苑。辛未,詔安親王岳樂率林興珠班師。壬午,授趙良棟勇略將軍。乙丑,祫祭太廟。

是歲,免順天、江南、山東、山西、河南、浙江、湖廣等省二百六十一州縣災賦有差。朝鮮、琉球、安南入貢。

十九年庚申春正月甲午,趙良棟復龍安府,進至綿竹,偽巡撫張文等迎降,遂入成都。詔以良棟為雲貴總督。王進寶克朝天關,復廣元,王屏籓縊死,生擒吳之茂。壬子,上幸鞏華城,遣內大臣賜奠昭勛公圖賴墓。

二月辛酉朔,詔吳丹會趙良棟進取雲南,王進寶鎮四川,勒爾錦取重慶,徐治都守荊州。乙丑,佛尼勒收順慶府,潼川、中江、南部、蓬縣、廣安、西充諸縣悉下。丁卯,詔莽依圖督馬九玉、金光祖、高承廕進兵雲南。己巳,上幸南苑。丙子,大閱。以於成龍為直隸巡撫。徐治都大敗叛將楊來嘉,復巫山,進取夔州。楊茂勛復大昌、大寧。癸未,萬正色敗海寇於海壇。

三月辛卯,吳丹復重慶,達州、奉鄉諸州縣悉定。楊來嘉降,送京。乙未,以伊闢為雲南巡撫。丁酉,安親王岳樂師旋,上勞於蘆溝橋。辛丑,馬承廕誘執傅弘烈。先是,馬雄踞柳州,死,其子承廕以柳州降。至是,復叛,執弘烈送貴陽,不屈,死之。平南將軍賴塔復銅山,命守潮州備承廕。萬正色擊海寇於平海嶴,克之,進克湄州、南日、崇武諸嶴。硃天貴降。拉哈達擊劉國軒,敗之,遁廈門。偽將蘇堪迎降,進平玉洲、石馬、海澄、馬州等十九寨,復偕吳興祚取金門。己酉,察尼下辰龍關,蔡毓榮復銅仁。

夏四月庚申朔,以賴塔為滿洲都統。癸亥,穆占、董衛國敗吳應麒,復沅州、靖州,進復黎平。丁卯,上以學士張英等供奉內廷,日備顧問,下部優敘,高士奇、杜訥均授翰林官。己巳,命南書房翰林每日晚講通鑒。丙子,上祈雨天壇,翌日,雨。己卯,頒行尚書講義。王進寶以病回固原,以其子總兵用予統軍駐保寧。庚辰,宗人府進玉牒。

五月壬辰,命甘肅巡撫治蘭州。乙巳,莽依圖會軍討馬承廕,復降,命執送京師。己酉,山海關設關收稅。

六月甲子,蔡毓榮復思南。丁丑,命五城粥廠再展三月,遣太醫官三十員分治饑民疾疫。壬午,副都統馬爾哈齊、營總馬順德以縱兵殺人論罪。

秋七月甲午,停捐納官考選科道。褒恤福建總督範承謨、廣西巡撫馬雄鎮,贈官予謚廕。乙巳,以折爾肯為左都御史。己酉,解順承郡王勒爾錦大將軍,撤還京。

八月戊辰,上御經筵。己巳,命賴塔移駐廣州,以博濟軍益之。戊寅,大學士索額圖免。壬午,將軍莽依圖卒於軍,以勒貝代之。甲申,尚之信以屬人王國光訐告其罪,擅殺之,詔賜之信死。其弟之節,其黨李天植,皆伏誅,家口護還京師。

閏八月乙未,命各將帥善撫綠旗軍士。壬子,以王永譽為廣東將軍。

九月癸亥,吳世璠使其將夏國柱、馬寶潛寇四川,譚弘復叛應之,連陷瀘州、永寧,夔州土匪應之。命將軍吳丹、噶爾漢,提督範達理、徐治都分道討之。乙丑,以賴塔為平南大將軍,率師進雲南。戊寅,吳丹復瀘州。

冬十月,仁懷失守,罷吳丹,以鄂克濟哈領其軍。戊戌,以阿密達為蒙古都統。噶爾漢復巫山。壬寅,大將軍康親王傑書師旋,上郊勞之。戊申,彰泰、穆占敗吳世璠於鎮遠。噶爾漢擊譚弘於鐵開峽,敗之。是月,王大臣議上師行玩誤之王貝勒大臣罪。得旨,勒爾錦革去王爵,籍沒B4禁。尚善、察尼均革去貝勒。蘭布革去鎮國公。硃滿革去都統,立絞。餘各褫官、奪世職、鞭責、籍沒有差。

十一月丙辰朔,冬至,祀天於圜丘。彗星見,詔求直言。甲子,貝子彰泰進復平越,遂入貴陽。逆渠吳世璠及吳應麒等夜遁。安順、石阡、都勻三府皆下。庚午,以達哈里為蒙古都統。丙子,川北總兵高孟敗彭時亨於南溪橋,復營山,進圍靈鷲寨,斬偽將魏卿武。甲申,提督周卜世復思南。

十二月壬辰,以徐元文為左都御史。甲午,高孟復渠縣。乙未,提督桑峨大敗吳世璠於永寧,追至鐵索橋,賊焚橋遁。土官龍天祐、沙起龍造盤江浮橋濟大軍。壬寅,高孟復廣安州。庚戌,以郝浴為廣西巡撫。癸丑,祫祭太廟。

是歲,免直隸、江南、山東、山西、陜西、江西、福建、湖廣等省一百八十六州縣災賦有差。朝鮮、琉球入貢。

二十年辛酉春正月壬申,叛將李本深降,械送京師。癸酉,總兵高孟復達州。甲戌,將軍噶爾漢復雲陽,譚弘死,進復忠州、萬縣、開縣。乙亥,命侍郎溫代治通州運河。丙子,將軍穆占、提督趙賴擊夏國相等,走之,復平遠。辛巳,增置講官。詔法司慎刑。是月,鄭錦死,其子克塽繼領所部。

二月己丑,貝子彰泰師至安南衛,擊賊將線緎於江西坡。賊列象陣拒戰。官兵分三隊奮擊,大破之。賊遁,公圖、達漢泰追擊,復敗之,復普安州、新興所。壬辰,副都統莽奕祿敗賊張足法等於三山。甲午,詔凡三籓往事為民害者悉除之。蠲奉天鹽引。大將軍賴塔師至廣西,大破賊於黃草壩,復安籠,入曲靖。高孟復東鄉,敗彭時亨於月城寨。戊戌,增欽天監滿監副一員。都統希福、馬緝、碩塔復馬龍州、楊林城,入嵩明州,賊遁。穆占復黔西、大定,斬其偽將張維堅。乙巳,貝子彰泰、大將軍賴塔、將軍蔡毓榮先後入滇。賊將胡國柄、劉起龍迎拒,官軍分擊敗之,斬國柄、起龍。辛亥,謁孝陵。

三月甲辰,宣威將軍鄂克濟哈以失援建昌自劾。詔以覺羅紀哈裡代之。辛酉,葬仁孝皇后、孝昭皇后於昌瑞山陵。詔行在批閱章奏,令大學士審校。壬戌,胡國柱犯建昌,將軍佛尼勒擊走之,復馬湖。癸亥,馬寶棄遵義,犯瀘、敘。詔佛尼勒、趙良棟急擊滇賊,勿令回援。丙寅,贈恤福建死事運使高天爵、知府張瑞午等官廕。戊辰,土官陸道清以永寧降。癸酉,上奉太皇太后幸遵化湯泉。

夏四月甲辰朔,王用予復納谿、江安、仁懷、合江。己酉,貝子彰泰遣使招撫諸路,武定、大理、臨安、永順、姚安皆降。壬子,上奉太皇太后還宮。

五月癸丑朔,提督周卜世取遵義,降偽官金仕俊等,復真安州、仁懷、桐梓、綏陽等縣。己未,遣官察閱蒙古蘇尼特等旗被旱災狀。乙丑,詔行取州縣曾陷賊中者勿選科道。辛巳,大將軍貝子彰泰報抵雲南省城,偽將李發美以鶴慶、麗江二府降。

六月戊子,除山西、陜西房號銀。

秋七月丁巳,以禮部尚書郭四海兼管刑部。庚申,詔四川民田為弁兵所占者察還之。辛酉,都統希福、提督桑峨擊馬寶於烏木山,大敗之。馬寶降,械送京師誅之。乙丑,趙良棟遣總兵李芳述擊敗胡國柱,復建昌,入雲南。戊辰,詔圖海率王輔臣還京。壬申,賜宴瀛臺,員外郎以上皆與焉,賜糸採幣。己卯,以施瑯為福建水師提督,規取臺灣,改萬正色陸路提督。

八月辛巳朔,日有食之。乙巳,上御經筵。

九月辛亥,上巡幸畿甸。故平南王尚可喜喪至通州,賜銀八千兩,遣官奠茶果。戊午,上次雄縣,召見知州吳鑒,問渾河水決居民被災狀。丙寅,上還京。詔停本年秋決。壬申,復運丁工銀。

冬十月癸未,偏沅巡撫韓世琦敗賊將黃明於古州。甲申,額魯特噶爾丹入貢。乙酉,大學士圖海師旋,上嘉勞之。壬辰,詔撤平南、靖南兩籓弁兵還京。癸卯,詔免吐魯番貢犬馬。

十一月辛亥,詔從賊諸人,除顯抗王師外,餘俱削官放還。以諾邁為漢軍都統。癸亥,定遠平寇大將軍貝子彰泰、平南大將軍都統賴塔、勇略將軍總督趙良棟、綏遠將軍總督蔡毓榮疏報王師於十月二十八日入雲南城,吳世璠自殺,傳首,吳三桂析骸,示中外,誅偽相方光琛,餘黨降,雲南平。是日,以昭告孝陵,車駕次薊州。丁卯,祭孝陵。辛未,召貝子彰泰、將軍趙良棟還京。乙亥,上獵於南山,發矢殪三虎。己卯,回鑾。

十二月戊子,設滿洲將軍駐荊州,漢軍將軍駐漢中。癸巳,群臣請上尊號。敕曰:「自逆賊倡亂,莠民響應,師旅疲於徵調,閭閻敝於轉輸。加以水旱頻仍,災異疊見。此皆朕躬不德所致。賴宗社之靈,削平庶孽。方當登進賢良,與民休息,而乃侈然自足,為無謂之潤色,能勿恧乎!其勿行。」補廣西鄉試。戊戌,大學士圖海卒。己亥,上御太和門受賀,宣捷中外。癸卯,加上太皇太后、皇太后徽號,頒發恩詔,賜宗室,賚外籓,予封贈,廣解額,舉隱逸,旌節孝,恤孤獨,罪非常赦不原者悉赦除之。以於成龍為江南江西總督,吳興祚為廣東廣西總督。丁未,祫祭太廟。

是歲,免直隸、江南、江西、山東、山西、浙江、福建等省七十五州縣災賦有差。丁戶一千七百二十三萬,徵銀二千二百一十八萬三千七百六十兩有奇。鹽、茶課銀二百三十九萬九千四百六十八兩。鑄錢二萬三千一百三十九萬。朝鮮、厄魯特入貢。


\end{pinyinscope}