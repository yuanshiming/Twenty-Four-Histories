\article{列傳二百五十五}

\begin{pinyinscope}
崇綺子葆初等志鈞延茂弟延芝色普徵額王懿榮熙元

宗室寶豐宗室壽富弟壽蕃等宋承庠王鐵珊

崇綺,字文山,阿魯特氏,蒙古正藍旗人,大學士賽尚阿子。以穆后父貴,升隸滿洲鑲黃旗。初為工部主事,坐其父出師無功,褫職。咸豐四年,粵寇謀犯畿輔,充督練旗兵處文案,事寧,敘兵部七品筆帖式。英吉利兵艦窺天津,錄守內城功,擢主事。嗣遷員外郎。同治三年,將軍都興阿以崇綺諳兵事,奏自隨,兵部疏留。是歲成一甲一名進士,立國二百數十年,滿、蒙人試漢文獲授修撰者,止崇綺一人,士論榮之。九年,遷侍講,出典河南鄉試,充日講起居注官。十一年,詔冊其女為皇后,錫三等承恩公。歷遷內閣學士,戶部、吏部侍郎。

光緒二年,充會試副考官,補鑲黃旗漢軍副都統。會河南旱,大吏匿不報,為言官所劾。上命偕侍郎邵亨豫按問,廉得實,巡撫李慶翔以下皆獲罪。四年,吉林駐防侍衛倭興額被盜誣控,詔與侍郎馮譽驥往讞,尋命崇綺署將軍專治之。倭興額控如故,事下侍郎志和覈覆,得誣告狀,崇綺自劾,被宥。五年,出為熱河都統。御史孔憲疏稱其忠直,宜留輔,不許。七年,調盛京將軍。

九年,謝病歸。旋授戶部尚書,再調戶部,復乞休。初,穆宗崩,孝哲皇后以身殉,崇綺不自安,故再引疾。二十六年,立溥俊為「大阿哥」,嗣穆宗。乃起崇綺於家,俾署翰林院掌院學士,傅溥俊。於是崇綺再出,與徐桐比而言廢立,甚得太后寵,恩眷與桐埒。義和團起,朝貴崇奉者十之七八,而崇綺亦信仰之。事敗,隨榮祿走保定,居蓮池書院,自縊死。榮祿以聞,賜奠醊,入祀昭忠祠,謚文節。

崇綺妻,瓜爾佳氏,先於京師陷時,預掘深坑,率子散秩大臣葆初及孫員外郎廉定,筆帖式廉容、廉密,監生廉宏,分別男女入坑生瘞,闔門死難,各獎恤有差。二十七年,命以曾孫法亮嗣廉定,襲爵。

志鈞,亦三等承恩公,滿洲鑲黃旗人。充散秩大臣。聞警,設醴祭先,率妻子皆衣冠對縊於中堂。恤如例,謚貞愍。

延茂,杜氏,內務府漢軍正白旗人。同治二年進士,銓禮部主事。光緒八年,歷遷至鴻臚寺少卿。上言八旗官學廢弛,宜變通章程。再遷內閣侍讀學士。

中法構釁,疏言:「我國士夫多懵外勢,請自今慎選使才,令其考察彼國政治利弊,圖其山川夷險,隨時奏聞。」又言:「名將必知地利而後可行師,廟堂必知地利而後可馭將。今宜北起盛京,南逾嶺廣,合臺、瓊為一氣。復自滇、粵邊外訖越南全境,分繪兩圖,更令諸疆臣各繪所轄地圖,上測緯度,下準方斜,俾知相距里數,為軍事之用。」上韙其議。

十三年,除奉天府府丞。越四年,入為大理寺少卿。二十四年,由駐藏辦事大臣擢吉林將軍,以倉廩災,上章自劾。明年,徵還,再授黑龍江將軍,未行而拳禍作。聯軍入都,偕弟延芝守安定門,城陷,闔室自焚死。贈太子少保,謚忠恪。妻並諸娣姒女子子皆獲旌。

色普徵額,舒穆魯氏,滿洲正白旗人。咸豐十年,賊竄畿疆,以健銳營前鋒校,從大學士瑞麟往討,裹創力戰。旋從僧格林沁剿捻,斬馘甚眾。同治初,又從都統穆騰阿軍畿南。光緒三年,遷參領。八年,軍政課最,授鑲紅旗漢軍副都統,充神機營專操大臣。二十四年,徙駐南苑。二十六年,擢寧夏將軍,未行,拳亂起,命守正陽門,晝夜徼循不少休。聯軍攻城,中砲死。贈太子少保,謚壯恪,予騎都尉兼雲騎尉世職。

王懿榮,字正孺,山東福山人。祖兆琛,山西巡撫。父祖源,四川成綿龍茂道。懿榮少劬學,不屑治經生藝,以議敘銓戶部主事。光緒六年成進士,選庶吉士,授編修,益詳練經世之務,數上書言事。十二年,父憂,解職。服闋,出典河南鄉試。二十年,大考一等,遷侍讀。明年,入直南書房,署國子監祭酒。會中東戰事起,日軍據威海,分陷榮城,登州大震,懿榮請歸練鄉團。和議成,還都,特旨補祭酒。越二年,遭母憂,終喪,起故官。蓋至是三為祭酒矣,前後凡七年,諸生翕服。

二十六年,聯軍入寇,與侍郎李端遇同拜命充團練大臣。懿榮面陳:「拳民不可恃,當聯商民備守御。」然事已不可為。七月,聯軍攻東便門,猶率勇拒之。俄眾潰不復成軍,乃歸語家人曰:「吾義不可茍生!」家人環跽泣勸,厲斥之。仰藥未即死,題絕命詞壁上曰:「主憂臣辱,主辱臣死。於止知其所止,此為近之。」擲筆赴井死。先是懿榮命浚井,或問之,笑曰:「此吾之止水也!」至是果與妻謝氏、寡媳張氏同殉焉。諸生王杜松等醵金瘞之。事聞,贈侍郎,謚文敏。懿榮泛涉書史,嗜金石,翁同龢、潘祖廕並稱其博學。

熙元,直隸總督裕祿子。光緒十五年進士,由編修累遷至祭酒。聯軍入,方家居守制,聞變,偕嫂富察氏、妻費莫氏仰藥以殉。贈太常寺卿,謚文貞。越三年,杜松等以兩祭酒大節昭著,籥請隆報饗,得旨,附祀監署唐韓愈祠。

宗室寶豐,字龢年,隸正藍旗。好讀書,有清尚。光緒十五年進士,選庶吉士,授編修,歷遷至侍講。二十五年,立溥俊為「大阿哥」,命直弘德殿,並賞高賡恩四品京堂,同授大阿哥讀。明年,兩宮西幸,寶豐以隨扈不果,憤甚,誓死職。自題絕命詞曰:「忠孝節廉,本乎天性。見利思義,見危授命。嗚呼寶豐,不失其正。」飲金死。贈太常寺卿。

宗室壽富,字伯茀,隸正藍旗,侍讀寶廷子。泛覽群籍,尤諳周官、禮、太史公書,旁逮外國史,通算術,工古文詩詞。光緒十四年,成進士,選庶吉士。嘗憤國勢不張,八旗人才日衰,箸勸八旗官士文,立知恥會,大旨警頑傲,勵以自強。浙江巡撫廖壽豐疏薦壽富才學堪大用,命赴日本考政治。既還,箸日本風土志四卷獻上,召見,痛陳中國積弊及所宜興宜革者,漏三下始退,上器之。政變作,遂杜門。

壽富性故矜貴,不通刺朝列。及拳亂起,乃上書榮祿,言董福祥軍宜託故令離畿甸,然後解散拳民,謂「董為禍根,拳其枝葉耳」。榮祿不省。妻翁內閣學士聯元既以論拳匪誅,家屬匿其宅,眾以壽富重新學,亦指為袒外,恚甚,或勸之他往,曰:「吾宗親也,寧有去理耶?」城陷,壽富自題絕命詞,並貽書同官曰:「國破家亡,萬無生理。乞赴行在,力為表明。侍已死於此地,雖講西學,未嘗降敵。」遂與弟右翼宗室副管壽蕃及一妹一婢並投繯死。贈侍講學士。

壽富刻苦孤峭。寶廷罷官早,家貧甚,性癖泉石。壽富事父能委曲以適其意旨。著有搏虎集。

宋承庠,字養初,江蘇華亭人。由拔貢考取小京官,銓工部。光緒四年,舉於鄉,遷主事。八年,充總理衙門章京,遷員外郎,轉御史。二十六年,巡視京城,聯軍入,遙望城內火光燭天,自言:「主辱臣死,義無可逃。」疾書一紙遺家人曰:「宗廟宮寢,已付一炬,敵人殘忍,不共戴天。讀聖賢書,惟有捐軀報國而已。我得死所,妻子勿以我為念。」時已仰藥,口不能言,越一日卒。贈四品卿銜。

王鐵珊,字伯唐,安徽英山人。光緒十五年進士,銓兵部主事。居久之,母年老,欲歸省。會拳亂作,知都城必危,遂不去。悉舉貲斧寄母,獨留百金,復分其半助邑館貧不能歸者。其人謂:「盍不偕南?」曰:「時勢至此,不能出力抗敵,已負朝廷;若更引身遠避,何以為人?且在京為大清官,在籍踐大清土,國茍不保,家將焉屬?」其人知其隱蓄死志,強之行,不可。兩宮既西狩,遂伏案作書寄弟,略云:「身非武職,恨不能執干戈衛社稷;官非臺諫,又不獲效忠言維國是。如都城不保,義不偷生。所恨居官以來,未能事母,長負此不孝之罪耳。」書畢,肅衣冠拜,默坐室中。聞內城陷,自縊死。遺書友人治後事,謂:「某非死節,不忍見國事敗壞耳。」事聞,贈員外郎,又追贈道員。廕一子入監讀書,以知縣用。

論曰:國都既陷,主辱臣死,此大義也。崇綺久著清節,終以一死自明。延茂等見危授命,義不茍生。色普徵額等執干戈衛社稷,死猶不瞑,至今皆凜凜有生氣焉。


\end{pinyinscope}