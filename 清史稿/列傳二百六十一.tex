\article{列傳二百六十一}

\begin{pinyinscope}
吳三桂耿精忠尚之信孫延齡

吳三桂,字長伯,江南高郵人,籍遼東。父襄,明崇禎初官錦州總兵。三桂以武舉承父廕,初授都督指揮。襄坐失機下獄,擢三桂總兵,守寧遠。洪承疇出督師,合諸鎮兵,三桂其一也。師攻松山,三桂戰敗,夜引兵去。松山破,承疇降,三桂坐鐫三秩,收兵仍守寧遠。三桂,祖大壽甥也,大壽既降,太宗令張存仁書招三桂,不報。

順治元年,李自成自西安東犯,太原、寧武、大同皆陷,又分兵破真定。莊烈帝封三桂平西伯,並起襄提督京營,徵三桂入衛。寧遠兵號五十萬,三桂簡閱步騎遣入關,而留精銳自將為殿。三月甲辰,入關,戊申,次豐潤。而自成已以乙巳破明都,遣降將唐通、白廣恩將兵東攻灤州。三桂擊破之,降其兵八千,引兵還保山海關。自成脅襄以書招之,令通以銀四萬犒師,遣別將率二萬人代三桂守關。三桂引兵西,至灤州,聞其妾陳為自成將劉宗敏掠去,怒,還擊破自成所遣守關將;遣副將楊珅、游擊郭云龍上書睿親王乞師。王方西征,次翁後,三桂使至,明日,進次西拉塔拉,報三桂書,許之。

自成聞三桂兵起,自將二十萬人以東,執襄置軍中;復遣所置兵政部尚書王則堯招三桂,三桂留不遣。越四日,王進次連山,三桂又遣雲龍齎書趣進兵。師夜發,逾寧遠,次沙河,明日,距山海關十里。三桂遣邏卒報自成將唐通出邊立營,王遣兵攻之,戰於一片石,通敗走。又明日,師至關,三桂出迎。王命設儀仗,吹螺,偕三桂拜天畢,三桂率部將謁王,王令其兵以白布系肩為識,前驅入關。自成兵橫亙山海間,列陣以待。王令諸軍向自成兵而陣,三桂兵列右翼之末。陣定,三桂先與自成兵戰,力鬥數十合。及午,大風塵起,咫尺莫能辨,師噪風止。武英郡王阿濟格、豫郡王多鐸以二萬騎自三桂陣右突入,騰躍摧陷。自成方立馬高岡觀戰,詫曰:「此滿洲兵也!」策馬下岡走,自成兵奪氣,奔潰。逐北四十里,即日王承制進三桂爵平西王,分馬步兵各萬隸焉,令前驅逐自成。三桂執則堯送王所,命斬之。自成至永平,殺襄,走還明都,屠襄家,棄明都西走。命三桂從阿濟格逐自成至慶都,屢戰皆勝。自成走山西,乃還師。

世祖定京師,授三桂平西王冊印,賜銀萬、馬三。明福王由崧稱帝南京,使封三桂薊國公,又遣沈廷揚自海道運米十萬、銀五萬犒師,三桂不受;尋遣其侍郎左懋第、都督陳洪範等使於我,復齎銀幣勞三桂,三桂仍辭不受。尋命英親王阿濟格為大將軍,西討自成,三桂率所部從,自邊外趨綏德,二年,克延安、鄜州,進攻西安。自成以數十萬人迎戰,三桂督兵奮擊,斬數萬級。自成出武關南走,師從之,自襄陽下武昌,自成走死。師復東徇九江。八月,師還,賜繡朝衣一襲、馬二,命進稱親王,出鎮錦州,所部分屯寧、錦、中右、中後、中前、前屯諸地。三桂疏言丁給地五晌,各所房屋灰燼,地土磽薄,請增給;並為珅、雲龍及諸將吳國貴、高得捷等請世職,屬吏童達行等乞優擢;又以父襄、母祖氏、弟三輔並為自成所殺,疏乞賜恤:並如所請。三桂辭親王,下部議,許之。三年,入覲,賜銀二萬。

五年,命與定西將軍墨爾根侍衛李國翰同鎮漢中。六年,明宗室硃森滏攻階州,三桂與國翰督兵擊斬之。有王永強者為亂,破延安、榆林等十九州縣,延綏巡撫王正志、靖遠道夏時芳死之;復陷同官、定邊、花馬池。三桂督兵克宜君、同官,擊斬七千餘級。進克蒲城、宜川、安塞、清澗諸縣,誅永強所置吏。定邊、榆林、府穀皆下。八年,入覲,賜金冊印。時明桂王由榔稱帝居南寧,張獻忠將孫可望、李定國等皆降於明,率兵擾川北諸郡縣。命三桂偕國翰率師討之。九年七月,三桂與國翰遣兵西撫漳臘、松潘,東拔重慶;進攻成都,明將劉文秀棄城走;復進克嘉定,駐軍綿州。文秀及王復臣復自貴州向四川,招惈儸為助,陷重慶,進破敘州。三桂屢戰不利。文秀、復臣圍巡按御史郝浴於保寧。浴趣三桂等赴援,擊斬復臣,文秀引兵走。浴疏劾三桂擁兵觀望狀,三桂摘疏中「親冒矢石」語劾浴冒功,浴坐謫徙。三桂敘功,歲增俸千。子應熊尚主,為和碩額駙,授三等精奇尼哈番,加少保兼太子太保。

十四年,可望反明,攻由榔,定國御之,可望敗走長沙,來降。詔授三桂平西大將軍,與國翰率師徇貴州;時大將軍羅託、經略洪承疇等出湖南,將軍卓布泰等出廣西:三道並進。三桂等發漢中,道保寧、順慶,次合州,破明兵,收江中戰艦。定國遣其將劉正國、楊武守三坡、紅關諸隘,石壺關者尤險峻,明兵阻關。三桂令騎兵循山麓,步兵陟其巔,以砲發其伏,明兵驚潰,遂下遵義,克開州。會羅託等已克貴陽,卓布泰亦自都勻、安遠入,信郡王多尼將禁旅至。國翰還師遵義,尋卒。三桂馳與羅託等會於平越楊老堡,議分道進兵。三桂自遵義出天生橋,聞白文選據七星關,遂繞出烏撒土司境,次霑益。多尼師進曲靖,敗文選。卓布泰師進羅平,敗定國。

十六年正月,由榔奔永昌。二月,三桂與尚善、卓布泰合軍克雲南會城,破文選玉龍關,取永昌,由榔走緬甸。師渡潞江,定國設伏磨盤山,詗知之,分八隊迎擊,斬殺過半。取騰越,追至南甸,乃振旅自永昌、大理、姚安還。明將馬寶、李如碧、高啟隆、劉之復、塔新策、王會、劉偁、馬惟興、楊武、楊威、高應鳳、狄三品等,及景東、蒙化、麗江、東川、鎮雄諸土司,先後來降。多尼、卓布泰等師還,留固山額真伊爾德、卓羅等分軍駐守,而詔三桂鎮雲南,命總管軍民事。諭吏、兵二部,雲南將吏聽三桂黜陟。定國求出由榔緬甸,軍孟艮。元江土司那嵩與降將高應鳳舉兵應定國。三桂督兵自石屏進圍元江,逾月,擊斬應鳳,嵩自焚死,收其地為元江府。

十七年,戶部疏言雲南俸餉歲九百餘萬,議檄滿洲兵還京,裁綠旗兵五之二。三桂謂邊疆不寧,不宜減兵力。是時三桂已陰有異志,其籓下副都統楊珅說以先除由榔絕人望。三桂乃疏言:「前者密陳進兵緬甸,奉諭:『若勢有不可,慎勿強,務詳審斟酌而行。』臣籌畫再三,竊謂渠魁不滅,有三患二難:李定國、白文選等分住三宣六慰,以擁戴為名,引潰眾肆擾,其患在門戶;土司反覆,惟利是趨,一被煽惑,遍地蜂起,其患在肘腋;投誠將士,尚未革心,萬一邊關有警,若輩乘隙而起,其患在腠理。且兵糧取之民間,無論各省餉運愆期,即到滇召買,民方懸磬,米價日增,公私交困,措糧之難如此;年年召買,歲歲輸將,民力既盡,勢必逃亡,培養之難又如此。惟及時進兵,早收全局,乃救時之計。」下議政王大臣會戶、兵二部議,令學士麻勒吉、侍郎石圖如雲南諮三桂機宜,乃決策進兵。命內大臣愛星阿為定西將軍,率禁旅南征。

三桂所部五丁出一甲,甲二百置佐領,積數十佐領,以吳應麒、吳國貴為左、右都統分統之。七月,三桂疏請部勒降兵,分置十營,營千二百人,以降將為總兵:馬寶、李如碧、高啟隆、劉之復、塔新策將忠勇五營;王會、劉偁、馬惟興、楊威、吳子聖將義勇五營。十月,又疏請置援剿四鎮,以馬寧、沈應時、王輔臣、楊武為總兵。皆允之。三桂請拊循南甸、隴川、千崖、盞達、車里諸土司,頒敕印;復檄緬甸,令執由榔以獻。定國、文選屢攻緬甸求出由榔,緬甸頻年被兵,患苦之,使告師破定國等,請以由榔獻。十八年,三桂遣使緬甸刻師期,令於猛卯迎師;遣副都統何進忠及應時、寧等率師出騰越,道隴川,三月,至猛卯。緬甸又與定國戰,道阻。既,緬甸使至迎師,會瘴發,進忠等引還。

三桂以馬乃土司龍吉兆稱兵應定國,遣寶、啟隆及游擊趙良棟等討之,攻七十餘日,破其寨,斬吉兆,以其地為普安縣。九月,瘴息。三桂與愛星阿及前鋒統領白爾赫圖,都統果爾欽、遜塔等督兵攻大理,復出騰越,道南甸、隴川至猛卯,分兵二萬,遣寧、輔臣別取道姚關、鎮康、孟定;又慮蠻暮、猛密二土司助定國阻我師後,留總兵張國柱將三千人屯南甸為備。十一月,會師木邦。文選毀錫箔江橋走茶山,定國走景線。三桂令寧等以偏師逐文選,而與愛景阿趨緬甸,復檄令執送由榔。十二月,師進次舊晚坡,距緬甸都六十里。緬甸使告請遣兵進次蘭鳩江濱捍衛,乃遣白爾赫圖將百人以往。緬甸遂執由榔及其母、妻等送軍前。寧等逐文選及於猛卯,文選以數千人降,師還。

康熙元年,捷聞,詔進三桂親王,並命兼轄貴州。召愛星阿率師還。四月,三桂執由榔及其子,以弓弦絞殺之,送其母、妻詣京師,道自殺。定國尚往來邊上伺由榔消息,三桂令提督張勇將萬餘人戍普洱、元江為備。未幾,定國走死猛臘。三桂招其子嗣興,以千餘人降,明亡。二年,遣會等攻隴納山蠻,破巢,斬渠。三年,遣之復及總兵李世耀率兵出大方、烏蒙,攻水西土司安坤、烏撒土司安重聖,並擊斬之,以其地設府:隴納曰平遠,大方曰大定,水西曰黔西,烏撒曰威寧。四年,奏裁雲南綠旗兵五千有奇。五年,復遣兵攻土司祿昌賢於隴箐,取其寨數十。迤東悉定,設府曰開化,州曰永定。

三桂初以開關迎師,位望出諸降將孔有德、耿仲明、尚可喜輩右。有德專征定湖廣,徇廣西,李定國破桂林,殉焉;可喜與仲明子繼茂分兵定廣東、福建;而三桂功最高。雲、貴初定,洪承疇疏用明黔國公沐英故事,請以三桂世鎮雲南。三桂復請敕云南督撫受節制,移總督駐貴陽,提督駐大理。據由榔所居五華山故宮為籓府,增華崇麗。籍沐天波莊田七百頃為籓莊。假濬渠築城為名,重榷關市,壟鹽井、金銅礦山之利,厚自封殖。通使達賴喇嘛,互市北勝州。遼東參,四川黃連、附子,就其地採運,官為之鬻,收其值。貨財充溢,貸諸富賈,謂之「籓本」。權子母,斥其羨以餌士大夫之無藉者。擇諸將子弟,四方賓客,與肄武備,謂以儲將帥之選。部兵多李自成、張獻忠百戰之餘,勇健善鬥,以時訓練。所轄文武將吏,選用自擅。各省員缺,時亦承制除授,謂之「西選」。又屢引京朝官、各省將吏用以自佐。御史楊素蘊疏論劾,三桂摘疏中「防微杜漸」語,請旨詰素蘊。素蘊覆奏,言「防微杜漸,古今通義。」事遂寢。

六年,三桂疏言兩目昏瞀,精力日減,辭總管雲、貴兩省事。下部議,如各省例,歸督撫管理,文吏由吏部題授。雲貴總督卞三元、雲南提督張國柱、貴州提督李本深交章陳三桂勞績,請敕仍總管。得旨:「王以精力日減奏辭,若召仍令總管,恐其過勞。如邊疆遇有軍事,王自應經理。」尋進應熊少傅兼太子太傅,命赴雲南視疾,仍還京師。三桂益欲攬事權,構釁苗、蠻,藉事用兵,私割中甸畀諸番屯牧,通商互市。迨三元乞歸養,甘文焜代為總督,不附三桂。三桂詐稱邊寇,檄赴剿;比至,又稱寇退,檄使還。籓屬將吏士卒糜俸餉鉅萬,各省輸稅不足,徵諸江南,歲二千餘萬,絀則連章入告,贏不復請稽核。是時可喜鎮廣東,繼茂子精忠鎮福建,與三桂並稱「三籓」,而三桂驕恣尤甚。

十二年二月,上遣侍衛吳丹、塞扈立勞三桂,賜御用貂帽、團龍裘、青蟒狐腋袍、束帶,亦遣使賚可喜。可喜旋疏引疾乞歸老,下部議,請並移所部。七月,三桂亦疏請移籓,並言:「所部繁眾,昔自漢中移雲南,閱三歲始畢。今生齒彌增,乞賜土地,視世祖時分畀錦州、寧遠諸區倍廣,庶安輯得所。」聖祖察三籓分鎮擅兵為國患,得三桂疏,下議政王大臣會戶、兵二部議奏。諸王大臣度三桂疏非由衷,遽議遷徙,必致紛紜,議移籓不便;獨尚書米思翰、明珠謂苗、蠻既平,三桂不宜久鎮,議移籓便。乃為二議以上:一議移三桂山海關外,別遣滿洲兵戍雲南;一議留三桂鎮雲南如故。上曰:「三桂蓄異志久,撤亦反,不撤亦反。不若及今先發,猶可制也。」遂命允三桂請移籓,並諭如當用滿洲兵,仍俟三桂奏請遣發。即令侍郎折爾肯、學士傅達禮齎詔諭三桂。

三桂初上疏,度廷議未即許,冀慰留久鎮。九月,詔使至,三桂大失望。與所部都統吳應麒、吳國貴,副都統高大節及其壻夏國相、胡國柱謀為亂,部署腹心扼關隘,聽入不聽出,與使者期以十一月己丑發雲南。先三日丙戌,邀巡撫硃國治脅之叛,不從,榜殺之。遂召諸總兵寶、啟隆、之復、足法、會、屏籓等舉兵反,自號周王天下都招討兵馬大元帥。蓄發,易衣冠,幟色白,步騎皆以白氈為帽。執折爾肯、傅達禮,按察使李興元,知府高顯辰,同知劉昆,不為三桂屈,具楚毒,徙置瘴地。國柱及總兵杜輝、柯鐸,布政使崔之瑛等皆降。三桂傳檄遠近,並致書平南、靖南二籓,及貴州、四川、湖廣、陜西諸將吏與相識者,要約響應。遣馬寶將兵前驅向貴陽,李本深謀應之。文焜馳書告川湖總督蔡毓榮,並趣折爾肯、傅達禮從官郎中黨務禮、員外郎薩穆哈、主事辛珠、筆帖式薩爾圖速還京師告變。三桂遣騎追之,辛珠、薩爾圖為所殺。文焜率數騎趨鎮遠,鎮遠副將江義已得三桂檄,以兵圍文焜,文焜死之。寶兵至,巡撫曹申吉、總兵王永清皆降。

十二月,黨務禮、薩穆哈至京師,三桂反問聞。上以荊州咽喉地,即日遣前鋒統領碩岱率禁旅馳赴鎮守。尋命順承郡王勒爾錦為寧南靖寇大將軍,率師討三桂,分遣將軍赫業入四川,副都統馬哈達、擴爾坤駐軍兗州、太原備調遣,並停撤平南、靖南二籓。王大臣等請逮應熊治罪,命暫行拘禁。三桂兵陷清浪衛;毓榮遣總兵崔世祿防沅州,三桂兵至,以城降;復進陷辰州。

十三年正月,三桂僭稱周王元年,部署諸將:楊寶廕陷常德,夏國相陷澧州,張國柱陷衡州,吳應麒陷岳州。偏沅巡撫盧震棄長沙走,副將黃正卿、參將陳武衡以城降。襄陽總兵楊來嘉舉兵叛,鄖陽副將洪福舉兵攻提督佟國瑤,擊破之;走保山寨,皆應三桂,受署置。三桂自雲南至常德,具疏付折爾肯、傅達禮還奏,語不遜。上命誅應熊及其子世霖,諸幼子貸死入官。六月,命貝勒尚善為安遠靖寇大將軍,與勒爾錦分道進兵。是時雲南、貴州、湖南地皆入三桂,通番市,以茶易馬,結惈儸助戰,伐木造巨艦,治舟師,採銅鑄錢,文曰「利用。」所至掠庫金、倉粟,資軍用。

勒爾錦師次荊州,三桂遣劉之復、王會、陶繼智等屢以舟師攻彞陵,勒爾錦遣將屢擊敗之,未即渡江。尚善師次武昌,以書諭三桂降,置不答。三桂傳檄所至,反者四起:提督鄭蛟麟,總兵譚弘、吳之茂反四川,巡撫羅森、降將軍孫延齡以有德舊部反廣西,精忠反福建,河北總兵蔡祿反彰德,三桂勢益張;又遣使與達賴喇嘛通好。達賴喇嘛為上書乞罷兵,上弗許。先後遣經略大學士莫洛、大將軍康親王傑書、貝勒董額等四出征撫,將軍阿密達擒祿誅之。上趣尚善攻岳州,三桂使吳應麒、廖進忠、馬寶、張國柱、柯鐸、高啟隆等分道拒戰,又遣兵窺江西,循江達南康,陷都昌;復自長沙入袁州,陷萍鄉、安福、上高、新昌諸縣。上命安親王岳樂為定遠平寇大將軍,徇江西;簡親王喇布為揚威大將軍,鎮江南。時王輔臣已為陜甘提督,復以寧羌叛應三桂,莫洛死之。三桂遣其將王屏籓入四川,與吳之茂合軍助輔臣。上復趣尚善速攻岳州,尚善疏請益兵,未即進。

十四年正月,上命岳樂自袁州取長沙,岳樂遣兵先後克上高、新昌、東鄉、萬年、安仁、新城諸縣,復進克廣信、饒州。夏國相堅守萍鄉,攻之不下。上以岳樂師向湖南,命喇布移鎮南昌。三桂遣將率兵七萬、惈儸三千防醴陵,築木城以守;又於岳州城外掘壕三重,環竹木為穽;於洞庭湖峽口植業木為椿,阻舟師;陸軍築壘皆設鹿角重疊,阻騎兵;乃自常德赴松滋,駐舟師虎渡口,截勒爾錦、尚善兩軍使不相應;揚言將渡江攻荊州,決堤以灌城,分嶽州守兵據彞陵東北鎮荊山,令王會、楊來嘉、洪福等合兵陷穀城,執提督馬胡拜,攻鄖陽、均州、南漳。勒爾錦遣貝勒察尼守彞陵,與都統宜理布等力禦之,疏請益兵。上責勒爾錦逗遛,不許。是歲,察哈爾布爾尼叛,上遣大將軍信親王鄂札、副將軍大學士圖海擊破之。

十五年,三桂遣兵侵廣東,授之信招討大將軍。時可喜已病篤,之信遂降。三桂別遣其將韓大任、高大節將數萬人陷吉安。上令喇布固守饒州,岳樂攻萍鄉,力戰破十二壘,斬萬餘級,國相引兵走,乃克之。師進復醴陵、瀏陽,復進攻長沙。三桂遣胡國柱益兵以守,馬寶、高啟隆自岳州以兵會。三桂自松滋移屯嶽麓山,為長沙聲援;又令大任、大節自吉安分兵犯新淦,屯泰和,復陷萍鄉、醴陵,斷岳樂軍後。上嚴趣喇布援岳樂,乃自饒州進復餘干、金谿,攻吉安,大節將四千人來拒,戰於大覺寺,以百騎陷陣,師左次螺子山。大節復以少兵力戰,喇布及副將軍希爾根倉卒棄營走,師敗績。會大任與大節不相能,大節怏怏死。喇布遣兵復圍吉安,大任不敢出戰。勒爾錦以三桂去松滋,率兵渡江取石首,遣貝勒察尼攻太平街三桂兵壘,師敗續,退保荊州。是歲大將軍、大學士圖海代董額征陜西,輔臣降。上令將軍穆占將陜西兵赴荊州,康親王傑書自浙江下福建,精忠降。之信亦遣使詣喇布降。延齡聞,亦原降,三桂使從孫世琮襲桂林,執而殺之,掠柳州、橫州、平樂、南寧。

十六年,尚善分兵送馬三千益岳樂軍,三桂邀奪於七里臺,復遣兵援吉安,與喇布軍相持。穆占自岳州進,與岳樂夾攻長沙,克之。三桂所遣援吉安諸軍皆引去,大任棄城走。吉安乃下。三桂自嶽麓徙衡州,分兵犯南安、韶州,並益世琮兵掠廣西。十七年,岳樂復平江、湘陰,三桂將林興珠率所將水師降。穆占攻永興,拔之,並下茶陵、攸、酃、安仁、興寧、郴、宜章、臨武、藍山、嘉禾、桂陽、桂東十二城。喇布亦與江西總督董衛國率師逐大任,及於寧都,大任敗走福建,詣傑書降。三桂遣馬寶、胡國柱等攻永興,都統宜理布、護軍統領哈克山出戰,死。穆占與碩岱等力守。

是歲,三桂年六十有七,兵興六年,地日蹙,援日寡,思竊號自娛。其下爭勸進,遂以三月朔稱帝,改元昭武,以衡州為定天府。置百官,大封諸將,首國公,次郡公,亞以侯、伯。造新歷。舉雲、貴、川、湖鄉試。號所居舍曰殿,瓦不及易黃,以漆髹之。構廬舍萬間為朝房。築壇衡山,行郊天即位禮,將吏入賀。是日大風雨,草草成禮而罷。俄病噎,八月,又病下痢,噤不能語。召其孫世璠於雲南,未至,乙酉,三桂死。寶、國柱攻永興方急,聞喪,自焚其壘,引軍還衡州。世璠,應熊庶子,留雲南,奔三桂之喪,至貴陽,其下擁稱帝,改號洪化,倚方光琛、郭壯圖為腹心。光琛,三桂所署大學士;壯圖,封國公。

三桂初起兵,其下或言宜疾行渡江,全師北向;或言直下金陵,扼長江,絕南北運道;或言宜出巴蜀,據關中,塞殽、函自固。三桂皆不能用,屯松滋,與勒爾錦夾江而軍,相持,皆不敢渡江決戰。既,還援長沙。晚乃欲通閩、粵道,糾精忠、之信復叛,攻永興未下而死。吳國貴復議舍湖南,北向爭天下,陸軍出荊、襄趨河南,水軍下武昌,掠舟順流撼江左。諸將俱重棄滇、黔,馬寶首梗議,乃罷。

上以勒爾錦頓兵荊州不進,時尚善卒,貝勒察尼代為安遠靖寇大將軍,攻岳州,吳應麒守堅,久未下。下詔將親征,聞三桂死,乃罷。趣諸軍分道並進,並敕招撫陷賊官民。察尼屯君山,不能斷湖道,至是造鳥船百、沙船四百餘,配以兵三萬,水師始成軍,以貝勒鄂鼐統之。用林興珠策,以其半泊君山,斷常德道;以其半分泊扁山、香爐峽、布袋口諸地;陸軍屯九貴山,斷岳州、衡州道。水陸綿亙百里,岳州餉竭援窮,應麒與諸將江義、巴養元、杜輝駕巨艦二百,乘風犯柳林嘴。察尼令水師棹輕舟,越敵艦,發砲擊之,毀過半,兵皆入水死。應麒復將五千人犯陸石,將軍鄂訥、前鋒統領杭奇率師擊之,應麒敗走。杜輝有子在師中,通使約降,事洩,應麒殺輝。諸將日構隙,陳華、李超、王度沖等以舟師降。應麒收殘卒,挾輜重,潰圍奔長沙,胡國柱亦棄城與俱走。察尼率師自岳州進克華容、安鄉、湘潭、衡山諸縣。

勒爾錦聞三桂死,率師自荊州渡江,三桂所部勒水師泊虎渡上游、陸師屯鎮荊山,皆潰走。分兵定松滋、枝江、宜都、石門、慈利、澧州,進克常德。喇布率師入衡州,進取祁陽、耒陽,復進克寶慶。是時吳國貴自衡州退屯武岡,與馬寶俱。吳應麒自岳州退屯辰州,胡國柱自長沙退屯辰龍關,相犄角力守。穆占師進克永明、江華、東安、道州,復進取永州。岳樂師自衡州復常寧,攻武岡,國貴以二萬人據楓木嶺拒戰。岳樂令林興珠與提督趙國祚督兵奮擊,國貴死,兵潰。貝子彰泰等逐至木瓜橋,大破之,武岡下。上召嶽樂還京師,彰泰代為定遠平寇大將軍,令與穆占議進取。是歲,將軍莽依圖等師徇廣西,世琮走死。

十九年春,將軍趙良棟自略陽破陽平關,克成都。王進寶自鳳縣破武關,取漢中。王屏籓走保寧,師從之,戰於錦屏山,薄城,屏籓自殺。保寧下,進克順慶。將軍吳丹、提督徐治都自巫山克夔州、重慶,楊來嘉、譚弘先後降。察尼攻辰龍關,出間道襲破之,克辰州。楊寶廕、崔世祿皆降。彰泰師克沅州,吳應麒、胡國柱走貴陽。上召勒爾錦、察尼還京師,趣彰泰與穆占、蔡毓榮等自沅州,喇布自南寧,吳丹、趙良棟自遵義,三道並進。世璠令應麒與王會、高啟隆、夏國相合兵入四川,掠瀘州、敘州,進陷永寧。譚弘復叛,陷夔州。上復趣彰泰速下貴陽,命賚塔為平南大將軍,盡護廣西諸軍。吳丹坐不援永寧,罷,命趙良棟盡護四川諸軍,仍三道入雲南。世璠召會、啟隆、國相自四川還援貴陽,令馬寶、胡國柱等掠四川。

十月,彰泰師克鎮遠,世璠將張足法等敗走。復進取平越,克新添、龍里二衛,薄貴陽。世璠與應麒等奔還雲南。貴陽與安順、石阡、都勻諸府並下。世璠所署侍郎郭昌、邱元,總兵臧世遠、齊聘金、文臺等,率將吏百數十人、兵一千三百有奇,詣彰泰軍降。師復進,世璠所署總兵蔡國昌、平遠知府鄭開樞等以平遠降。戰於永寧,至雞公背,世璠兵焚盤江鐵索橋走。普安土司龍天祜,永寧土司沙起龍、禮廷試造浮橋濟師。

二十年春,世璠以高啟隆為大將軍,與夏國相、王會、王永清、張足法等將二萬人拒彰泰,復陷平遠,屯城西南山上,穆占與提督趙賴進擊破之。啟隆等走,會降,復取平遠。彰泰師進次安南衛,世璠將線緎、巴養元、鄭旺、李繼業以萬餘人屯盤江西坡,為象陣,師初戰,為緎等所敗。越二日,彰泰令總兵白成功等進擊,戰於沙子哨,力鬥,自午至酉,師分隊奮進,緎等夜走。遣都統龔圖等逐之,至臘茄坡,再戰,緎等退保交水城。克新興所、普安州,黔西、大定諸府皆下,斬世璠所署巡撫張維堅。賚塔師自田州進次西隆州,世璠將何繼祖以萬人屯石門坎守隘。賚塔督兵分隊進攻,奪隘,復安籠所。繼祖退至新城所,復與世璠將詹養、王有功等,合兵二萬人屯黃草壩,為象陣堅守。賚塔督兵進,力戰,奪壘二十二,獲養、有功,俘兵千餘。復進破曲靖,取交水城。緎等復走,遂克馬龍州易龍所、揚林城,彰泰師亦至,兩軍會於嵩明。

二月,進攻雲南會城,屯歸化寺,世璠遣將胡國柄等將萬人為象陣拒戰。彰泰、賚塔督兵進擊,大破之,斬國柄及裨將九,俘六百餘,追之,薄城。世璠將張國柱、李發美等先後降。臨安、姚安、大理、鶴慶、麗江諸府悉下。世璠召馬寶、胡國柱、夏國相等還救雲南。上諭趙良棟等分兵邀擊寶等。寶自尋甸至楚雄,屯烏木,兵潰,與巴養元、趙國祚、鄭旺、李繼業、郎應璧等詣姚安降。國柱自麗江、鶴慶入雲龍州,窮蹙自縊死。夏國相自平越敗後走廣西,總兵李國樑遣兵圍之,亦與王永清、江義等出降,世璠援絕。趙良棟師自夾江克雅州,復建昌,渡金沙江,次武定,復進次綿竹。九月,進與彰泰、賚塔諸軍合。時圍城已數月未下,良棟議斷昆明湖水道,主速攻,督兵薄城,圍之數重。線緎等謀執世璠及郭壯圖以降,世璠與壯圖皆自殺。十月戊申,緎等以城降。穆占與都統馬齊先入城,籍賊黨,執方光琛及其子學潛、從子學範,磔於軍前。戮世璠尸,傳首京師。世璠所署將吏一千五百餘、兵五千有奇,皆降。雲南、貴州、四川、湖廣諸省悉平。上令宣捷詔,赦天下。二十一年春,從議政王大臣請,析三桂骸,傳示天下。懸世璠首於市。磔馬寶、夏國相、李本深、王永清、江義,親屬坐斬。斬高啟隆、張國柱、巴養元、鄭旺、李繼業,財產妻女入官。

三桂諸將,馬寶、王屏籓最驍勇善戰。寶初為流賊,降明桂王由榔為將。桂王奔南甸,寶降於三桂,為忠勇中營總兵。三桂反,率兵前驅,盡陷貴州至湖廣南境諸郡縣,封國公。再入廣西,一入四川,敗走姚安,詣希福軍降,至是死。屏籓亦三桂所倚任,代高啟隆為忠勇左營總兵。三桂反,令入四川為王輔臣聲援。自秦州退守保寧,敗我師蟠龍山。十九年,師克保寧,自殺。

諸專閫大將叛降三桂助亂者:雲南提督張國柱,貴州提督李本深,總兵王永清,副將江義,四川總兵譚弘、吳之茂,湖廣總兵楊來嘉,廣東總兵祖澤清,而陜西提督王輔臣兵最強,亂尤劇。

國柱,明副將,來降。從續順公沈永忠下湖南,又從可喜定廣東,累遷至提督。三桂反,授以大將軍,封國公。陷衡州,圍長沙,戰岳州,皆國柱力。師圍世璠,乃自大理出降。

本深,明總兵高傑甥,傑死,以提督代將。降於豫親王多鐸,授三等精奇尼哈番,累遷至提督。三桂反,授以將軍。彰泰師克貴陽,出降。

永清以黔西鎮,義以鎮遠協,戕文焜,先後附三桂。至是同死。

弘,初以明將降,累遷至總兵。三桂反,與四川提督鄭蛟麟、總兵吳之茂合謀叛。蛟麟,明都司,自松山降。三桂使犯漢中,戰敗,復出降。弘獨力戰,屢攻鄖陽。三桂授以將軍,封國公。弘死,子天秘走萬縣,久之始出降,送京師。是年五月,磔死。

之茂,與屏籓合軍援輔臣,攻秦州,力戰,敗走松潘。還與屏籓守漢中,城下,就擒,送京師誅之。

來嘉,初以鄭錦將降,授總兵。三桂反,與副將洪福同叛,三桂授以將軍。來嘉屢攻南漳,福屢攻均州。勒爾錦師渡江,福先降。來嘉敗走巫山,復走重慶。城下,出降,送京師,未至,死。

澤清,大壽子。以高州叛降三桂。尚之信降,澤清亦降。俄復叛,命之信討之,克高州,獲澤清及其子良楩,送京師磔死。

輔臣初為盜,號馬鷂子。從姜瓖為亂,降於英親王阿濟格。尋以侍衛從洪承疇南征,事承疇謹,除總兵。三桂留授援剿右鎮,從入緬甸,破桂王,遷提督。三桂反,招使叛,輔臣以聞,授三等精奇尼哈番,官其子繼貞。經略大學士莫洛自陜西入四川,以輔臣從。次寧羌,脅眾擊殺莫洛,反,三桂授以大將軍,固原、定邊、臨洮、蘭州、同州諸將吏悉附,大將軍貝勒董額討焉。輔臣保平涼,久不下。大學士圖海代將,督兵力攻,乃出降。詔復官爵,加太子太保,授靖寇將軍,從圖海駐漢中。輔臣內不自安,與其妻妾縊,獨不死;圖海師還,偕至西安,一夕死。上不深罪,但命停世襲,罷繼貞官。

耿精忠,靖南王繼茂子。順治中,繼茂遣入侍,世祖授以一等精奇尼哈番,尚肅親王豪格女,封和碩額駙。康熙十年,繼茂卒,襲爵。十二年,疏請撤籓,許之,遣侍郎陳一炳如福建料理。三桂反,命仍留鎮,召一炳還。三桂以書招精忠,精忠與籓下都統馬九玉,總兵曾養性、江元勛,參領白顯忠、徐文耀、王世瑜、王振邦、蔣得鋐等謀應三桂,獨九玉以為不可,養性等皆贊之。

十三年三月,發兵反,脅總督範承謨,不屈,執而幽之,並及其賓從眷屬。巡撫劉秉政降。精忠自稱總統兵馬大將軍。蓄發,易衣冠。鑄錢曰「裕民通寶」。以養性、顯忠、元勛為將軍,分陷延平、邵武、福寧、建寧、汀州諸府。約三桂合兵入江西。嗾潮州總兵劉進忠擾廣東。又招鄭錦發兵取沿海郡縣為聲援。浙江總督李之芳聞亂,出駐衢州,遣副將王廷梅等四出御戰。上命將軍賚塔出浙江,將軍希爾根出江西,削精忠爵,聲討。仍遣郎中周襄緒偕精忠護衛陳嘉猷齎敕招撫,精忠留之軍中。養性與林沖、徐尚朝、馮公輔、沙有祥等將萬餘人出仙霞關,陷江山、平陽,游擊司定猷縛總兵蔡朝佐,以城降。渡飛雲江,攻瑞安不下。移師攻溫州,總兵祖弘勛以城降。巡道陳丹赤、永嘉知縣馬閟死之。精忠授弘勛將軍,眾至十萬,陷樂清、天臺、仙居、嵊縣,而寧海、象山、新昌、餘姚諸縣土寇競起。養性請於精忠官其渠,使屯大嵐山,擾紹興、寧波。破黃巖,總兵阿爾泰降。分兵犯金華,精忠與其將周列、王飛石、桑明等陷廣信、建昌、饒州。復合玉山、永豐土寇東犯常山,陷開化、壽昌、淳安、遂安諸縣。別遣兵攻徽州、婺源、祁門。

上令康親王傑書為奉命大將軍,貝子傅喇塔為寧海將軍,率師下浙江。又以岳樂、喇布兩軍為聲援。將軍賚塔師次衢州,養性自常山來犯,賚塔同之芳遣兵擊卻之。副將牟大寅戰常山,斬精忠將張宏。洪起元戰紹興,復嵊縣。鮑虎戰淳安,擒飛石、明,復壽昌、遂安。養性、尚朝等以五萬人攻金華,副都統瑪哈達、總兵陳世凱等與戰於木道山,斬二萬餘級,養性走天臺。十四年,養性復以步騎數萬攻金華,遣其將硃飛熊率舟師水陸並進,傅喇塔擊斬飛熊,養性退屯茂平嶺。巡道許弘勛擊破大嵐山土寇,斬其渠。傅喇塔督兵自間道出茂平嶺背,養性兵潰,復黃巖、樂清。養性走保溫州,傅喇塔督兵合圍。瑪哈達擊敗尚朝、公輔、有祥等,復處州。穆赫林擊敗林沖,復仙居。喇布以兵助將軍額楚定徽州、婺源、祁門。將軍希爾根亦復建昌、饒州。

岳樂師次南昌,諭精忠,精忠以謾書答之。上復遣其弟聚忠齎敕諭降,至衢州,精忠拒不納。精忠母周氏,阻精忠毋叛,精忠不聽,周氏憤死。精忠攻衢州,戰屢敗,以馬九玉為將軍,率兵屯江山,而鄭錦兵至,據泉、漳諸地,與精忠手冓釁。尚可喜請援,上令喇布自江西下廣東。精忠遣其將邵連登等擾建昌,進攻撫州、贛州。三桂兵陷袁州、吉安,相犄角,阻師行。尚之信亦叛。

十五年春,傅喇塔自黃巖進攻溫州,力戰,屢破敵壘。養性憑江拒戰,累月未能薄城。上趣傑書自金華至衢州,下福建。八月,傑書與賚塔、之芳督兵擊九玉,戰於大溪灘,九玉敗走,克江山。招仙霞關守將金應虎降,遂入關,拔浦城。鄭錦兵侵興化,將及福州。精忠勢漸蹙,謀出降,先使人戕承謨及其客嵇永仁等。傑書師進次建陽,書諭降精忠,答書請宣詔赦罪。師復進,克建寧,次延平。精忠遣其子顯祚及襄緒、嘉猷出迎師。傑書使齎敕宣示,精忠乃出降,請從軍討錦自效。傑書以聞,詔復爵,以其弟昭忠為鎮平將軍,駐福州,命精忠從軍討錦。錦敗,還臺灣。乃移師趨潮州,進忠出降,令精忠駐焉。養性在溫州,屢出戰,傅喇塔督兵擊破之。養性墮水,復入城困守,精忠降,亦降,仍為籓下總兵。

十六年,遣顯祚入侍,授散秩大臣。籓下參領徐鴻弼等使赴兵部具狀,訐精忠降後尚蓄逆謀,昭忠亦以鴻弼等狀聞,上留中未發。十七年,上令精忠還福州,以其祖及父之喪還葬。是秋,三桂死,傑書疏請誅精忠,上諭曰:「今廣西、湖南、四川俱定,賊黨引領冀歸正者不止千百。驟誅精忠,或致寒心。宜令自請來京,庶事皆寧貼。」十九年,精忠請入覲,上以九玉為總兵,轄籓下兵。昭忠、聚忠又疏劾精忠,上乃下鴻弼等狀,令法司按治,系精忠於獄。遣聚忠赴福州宣撫所部。是歲,之信以悖逆誅。二十年,雲南平。二十一年,法司具獄上,上諭廷臣欲寬之。大學士明珠奏精忠負恩謀反,罪浮於之信。乃與養性、顯忠、元勛、進忠、文耀、世瑜、振邦、得鋐並磔於市。顯祚、弘勛等皆斬。秉政逮詣京師,道死。

尚之信,平南王可喜子。順治中,可喜遣入侍,世祖以可喜功多,令之信秩視公爵。康熙十年,聖祖允可喜請,令之信佐軍事。之信酗酒嗜殺,可喜老病,營別宅以居,號令自擅。十二年,可喜用其客金光策,上疏請以二佐領歸老海城,而以之信襲爵留鎮。

光,浙江義烏人,佐可喜久,以捕佛山亂民江鵬翥功,授鴻臚寺卿銜。屢以之信暴戾狀告可喜,為可喜謀,冀得見上自陳。上以可喜疏下部議,令並移所部,遣尚書梁清標如廣東料理。三桂反,命可喜仍留鎮,召清標還。總兵劉進忠以潮州叛,可喜遣次子之孝率兵討之。上授之孝平南大將軍,而命之信以討寇將軍銜協謀征剿。鄭錦遣兵助進忠。總兵祖澤清復以高州叛,孫延齡將馬雄引三桂將董重民、李廷棟、王弘勛等陷雷、廉二郡。之孝退保惠州。十五年春,可喜病益劇,之信代治事。三桂招可喜籓下水師副將趙天元、總兵孫楷宗相繼叛,之信遂降三桂,遣兵守可喜籓府,戒毋白事,殺光以徇。罷之孝兵,使侍可喜,可喜以憂憤卒。

三桂授之信招討大將軍、輔德公,旋進號輔德親王,而以重民為兩廣總督,駐肇慶。謝厥扶者,故戶,以繒船數百附馬雄。天元之叛,厥扶實誘之。三桂亦授以將軍,使與重民水陸相援應,屢檄之信出兵。之信賂以庫金十萬,乃不復相促迫。

之信旋遣使詣喇布軍,具疏請立功贖罪,上敕慰諭之。十六年,之信復疏請敕趣喇布軍入廣東。之信密嗾重民所部兵噪索餉,乘間擒重民,擊敗厥扶,走入海。乃遣副都統尚之瑛迎師,疏言闔屬歸正,並請敘籓下總兵王國棟、長史李天植等襄贊功。時方多故,而可喜有大勛,上優容之,命之信襲平南親王,國棟等復舊職。之信使入貢,上諭曰:「昔爾先人在時,屢獻方物。比年事變,信使弗通。每念爾先人忠貞不二,為國忘家,朕甚愍焉!王克承先志,遣使遠來,朕見物輒念爾先人。王其安輯粵東,以繼爾先人未盡之志。貢獻細務,勞人費事,今當暫止。」

是年秋,三桂遣其從孫世琮據廣西,巡撫傅弘烈率師討之,復梧州、潯州,規取桂林,之信令總兵尚從志以三千人從。上令之信自韶州進取宜章、郴州、永州,之信不赴。將軍莽依圖攻韶州,擊敗三桂將馬寶、胡國柱等。上命之信移師梧州,又不赴。十七年春,上以莽依圖深入廣西,命之信策應。之信仍以高、雷、廉三郡初定,疏請留鎮省城。上乃命發兵應莽依圖,之信遣國棟率兵赴宜章。及三桂死,之信乃請自進廣西,命為奮武大將軍,從師並進。十八年,天元出降,之信疏請誅之。師進次橫州,自言病作,遽還。上命以所部從莽依圖並進,之信令籓下總兵時應運率以往。及莽依圖將攻桂林,留應運守南寧。三桂兵據武宣,之信又疏言海寇宜防,將召應運還。上復諭趣之,十九年春,之信乃自將攻武宣。

之信與之孝不相能,以之孝嘗典兵,不欲其居廣州,疏請遣還京師。之信殘暴猜忌,醉輒怒,執佩刀擊刺,又屢以鳴鏑射人。楷宗叛復降,上貸其罪,之信杖殺之。護衛張永祥為之信齎疏詣京師,上召見,授總兵。之信故阻抑,復屢辱以鞭箠。怒護衛張士選語忤,射之,殘其足,諸護衛皆不平。國棟與副都統尚之璋、總兵甯天祚密謀圖之信。巡撫金俊疏言:「之信兇殘暴虐,猶存異志。臣察其左右俱義憤不平,因密約都統王國棟等共酌機宜,之信旦夕就擒。乞敕議行誅,以為人臣懷二心者戒。」國棟亦上疏自述與俊、之璋、天祚合謀圖之信,又代之信母舒氏、胡氏疏言:「之信怙惡不悛,有不臣之心。恐禍延宗祀,乞上行誅。」上諭趣之信出師。

之信既赴武宣,永祥、士選詣京師告變。上遣侍郎宜昌阿以巡視海疆至潮州,諭將軍賚塔移師,並令總督金光祖、提督折爾肯、副都統金榜選、總兵班際盛傳詔逮之信。之信與光祖、榜選、際盛等攻克武宣,之信入城。光祖等屯城外,得國棟檄,合兵圍城,傳詔逮之信。之信就逮,還廣州,上疏自辨。上令削爵,逮詣京師。籓下兵駐廣西,訛言師至雲南,即分置城守,眾情恟懼。上命宜昌阿、賚塔宣敕慰諭。七月,宜昌阿將以之信赴京師。天植怒國棟發難,白之信母,與之信弟之節、之璜、之瑛召國棟議事,伏兵殺之。賚塔率兵捕治,天植自服造謀,之信不與聞。護衛田世雄言之信實使天植殺國棟。獄上,上命賜之信死,之節、之璜、之瑛、天植皆斬。舒氏、胡氏貸其罪,並毋籍沒。世雄以不先發,坐杖流。上復諭宜昌阿曰:「之信雖有罪,其妻子不可凌辱,當護還京師。」又令察罷之信諸虐政。所部十五佐領改隸漢軍,駐防廣州。

之信初叛,提督嚴自明附之。自明,明參將,降,從總督孟喬芳征撫陜、甘,又擊張獻忠,破桂王,有功,授三等阿思哈尼哈番。之信遣攻南康,敗走南安,先之信降,授鑾儀使。病死。

孫延齡,漢軍正紅旗人。父龍,從孔有德來歸,授二等阿思哈尼哈番,從有德廣西。有德以女四貞字延齡。及有德死事,龍亦戰死,加拖沙喇哈番,以延齡襲。四貞尚幼,還京師,孝莊皇后育之宮中,賜白金萬,歲俸視郡主。長,命仍適延齡。

有德所部諸將,線國安功最高。國安與有德同起事,偕來降。從入關,西破李自成,南破桂王,累擢廣西提督,駐南寧。李定國陷桂林,盡殺其孥。國安與總兵馬雄、全節力戰復桂林,走定國。累加太子太保、征蠻將軍,封三等伯,統有德舊部駐桂林。康熙五年,以老乞休。

上以延齡有德壻,四貞生長軍中,習騎射,通武事,乃授延齡鎮守廣西將軍,代國安統有德舊部。予四貞郡主儀仗,偕赴鎮。延齡漸驕縱,十一年,御史馬大士劾延齡擅除武職,兵部既駁奏,延齡復疏請,恣肆不臣,上命申禁。十二年,所部都統王永年,副都統孟一茂,參領胡同春、李一第等列延齡縱兵殃民狀,牒總督金光祖,光祖以聞。上遣侍郎勒德洪按治,得實,請逮延齡治罪,特命寬之。三桂反,上授延齡撫蠻將軍,起國安都統。時節已前卒,雄代國安為提督,命與巡撫馬雄鎮合謀剿御。

十三年二月,延齡舉兵反,殺永年、一茂、同春、一第,幽雄鎮及其眷屬。詔奪官爵,聲討。延齡乃上疏言光祖、雄誘永年等謀害,上審其誣,諭尚可喜與光祖籌策進攻。延齡自稱安遠大將軍,移牒平樂、梧州諸郡。雄與總兵江義亦以柳州叛應三桂。國安病死。延齡招致萬羊山土寇,與所部合設五鎮,鎮兵二千。俄又自稱安遠王。慶陽知府傅弘烈當三桂未反,疏發諸不軌事,謫戍蒼梧,延齡既叛,授以將軍。弘烈說延齡迎師,四貞尤力勸之。十六年,延齡遣弘烈迎師江西。三桂詗知之,使從孫世琮率兵逼桂林,執殺延齡,四貞督兵禦戰。世琮乃留其將李廷棟戍桂林,出掠平樂、潯州、橫州、南寧。弘烈還至平樂,延齡將劉彥明、徐洪鎮、徐上遠等擒斬廷棟,與國安子成仁並出降。四貞還京師。

雄亦從有德南征有功,授二等阿思哈尼哈番。既與郭義叛,義偕嚴自明攻南康,敗走。雄旋病死雒容。子承廕出降,進伯爵,授左江總兵。十九年二月,復叛,紿弘烈登舟,襲破其營,殺之。六月,復降,逮詣京師,論死。義奪官,放還原籍。

論曰:聖祖初親政,舉大事書殿柱,即首「三籓」。可喜乞歸老,曷嘗言撤籓?撤籓自廷議,實上指也。三桂反,精忠等響應,東南六七行省皆陷寇。上先發兵守荊州,阻寇毋使遽北。分遣禁旅屯太原、兗州、江寧、南昌,首尾相顧,次第漸進,千里赴斗而師不勞。三桂白首舉事,意上方少,諸王諸將帥佐開國者皆物故,變起且恇擾。及聞上從容指揮,軍報迅速,閫外用命,始嘆非所料。制勝於廟堂,豈不然歟?上不欲歸咎建議撤籓諸臣,三桂等奉詔罷鎮,亦必曲意保全之。惜乎三桂等未能喻也!


\end{pinyinscope}