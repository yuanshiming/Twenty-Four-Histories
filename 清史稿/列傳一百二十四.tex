\article{列傳一百二十四}

\begin{pinyinscope}
盧焯圖爾炳阿阿思哈宮兆麟楊景素閔鶚元

盧焯,字光植,漢軍鑲黃旗人。入貲授直隸武邑知縣。縣舊有均徭錢供差費,遇差仍按里派夫,焯革除之,又歸火耗於公,捕盜尤力。雍正六年,解餉詣京師,世宗特召對。遷江南亳州知州,禁械斗。再遷山東東昌知府,總督田文鏡遣官弁四出訪事,東昌民逮下獄甚眾,焯至,悉判遣之。會有水災,焯疏運河,築護城長堤,動帑賑恤。上遣大臣閱視,獨東昌得完。九年,遷督糧道,移河南南汝道。十年,授按察使。十一年,遷布政使。

十二年,擢福建巡撫,賜孔雀翎。十三年,高宗即位,焯疏言被水州縣不成災,上諭曰:「被水雖不成災,仍須加意賑恤,毋使小民失所。」乾隆元年,請查丈建陽民田,上諭曰:「小民畏查丈如水火。汝初為加賦起見,今又以豁除掩非,一存觀望之心,所謂無一而可也。」尋奏減邵武永安所、霞浦福寧衛屯田徵米科則,豁閩、侯官諸縣額缺田地。又以平和、永安、清流諸縣田少丁多,請減免攤餘丁銀。又奏教民蠶績,疏濬省會城河。

三年,調浙江巡撫,兼鹽政。奏請停仁和、海寧二縣草塘歲修銀,減嘉興屬七縣銀米十之二。又奏陳鹽政諸事:請禁商人短秤;飭州縣捕私鹽毋擾民;毋捕肩挑小販;鹽場徵課不得刑比。上諭曰:「所奏各條皆是。汝先過刻,茲乃事事以寬沽名。過猶不及,汝其識之!」尋請裁鹽場協辦鹽大使,改海寧草塘為石塘。既,又請濬備塘河運石。五年,上諭曰:「盧焯至浙江,沽名邀譽,舉鄉賢名宦,絡繹不絕。海塘外已漲沙數十里,焯既請停草塘歲修,又請改建石塘。心無定見,惟事揣摩,已彰明較著矣。」六年,左都御史劉吳龍劾焯營私受賄,上解焯任,命總督德沛、副都統旺扎爾按治,事皆實,請奪官刑訊。事連嘉湖道呂守曾、嘉興知府楊景震。守曾已擢山西布政使,逮至浙江,自殺。杭州民數百為焯訟冤,毀副都統前鼓亭。德沛等以聞,上諭責辦理不妥。七年,讞上,焯、景震皆坐不枉法贓,擬絞。八年,焯以完贓減等,戍軍臺。十六年,上南巡,閱海塘,念焯勞,召還。

二十年,授鴻臚寺少卿,署陜西西安巡撫。二十一年,調署湖北,以陳宏謀代焯。宏謀未至,上命發歸化城米運金川饋軍,急驛諭宏謀。焯發視,奏言:「歸化城雖產米,路遠費重;西安有貯米,先發以饋軍。仍請擅行罪。」上嘉焯知大體,合機宜,實授湖北巡撫。二十二年,西安布政使劉藻入覲,言焯在西安入貢方物,但量給薄值;及調任湖北,欲借庫帑,未應付。上責焯負恩,奪官,戍巴里坤。二十六年,召還。三十二年,卒。

圖爾炳阿,佟佳氏,滿洲正白旗人。初授吏部筆帖式,累遷郎中。乾隆三年,授陜西甘肅道。累遷雲南布政使。十二年,擢巡撫。十五年,永嘉知縣楊茂虧銀米,圖爾炳阿令後政彌補結案。總督碩色論劾,上責圖爾炳阿欺隱徇庇,奪官,逮京師,下刑部治罪,坐監守自盜,擬斬監候。十七年,上以圖爾炳阿贓未入己,釋出獄。授吏部員外郎。未幾,授河南布政使,調山東,又復還河南。

二十年,擢巡撫。二十二年,上南巡,江蘇布政使夏邑彭家屏以病告家居,覲徐州行在,入對,言鄉縣被水。上諮圖爾炳阿,圖爾炳阿奏收成至九分,上責圖爾炳阿文過。圖爾炳阿又奏「去歲被水尚未成災」,上斥為怙惡不悛。遣員外郎觀音保密察災狀得實,上奪圖爾炳阿官,發烏里雅蘇臺效力。上發徐州,夏邑民張欽、劉元德詣行在訴知縣孫默諱災及治賑不實,上親鞫,元德言諸生段昌緒指使。上復遣侍衛成林會圖爾炳阿至夏邑按治,於昌緒家得傳鈔吳三桂檄。上諭曰:「圖爾炳阿察出逆檄,緝邪之功大,諱災之罪小。且以如此梗不知化之民,而治其司牧者以罪,是不益長澆風乎?免圖爾炳阿罪,仍留巡撫任治賑。圖爾炳阿若因有前此罪斥之旨,心存成見,或不釋然於災民,則是自取罪戾,亦不能逃朕洞鑒。」尋家屏亦以藏禁書罪至死,圖爾炳阿仍以匿災下吏議,奪官,命留任。逾數月,召詣京師,命往烏里雅蘇臺治餉。

二十八年,授貴州巡撫,二十九年,調湖南。三十年,病作,遣醫往視。卒。

阿思哈,薩克達氏,滿洲正黃旗人。自官學生考授內閣中書,累遷刑部郎中,充軍機處章京。乾隆十年,擢甘肅布政使。十四年,擢江西巡撫。疏言:「各營操演槍砲,須實子彈。營馬應令騎兵自飼。技藝以純熟得用為要,步法、架勢不必朝更夕改。」上嘉其言得要。旋調山西。十六年,平陽旱,未親往撫恤,詔責之。十七年,蒲、解等處復災,請以平陽富民捐款解河東道加賑。上諭之曰:「賑濟蠲緩,重者數百萬,少亦數十萬,悉動正帑,從無顧惜。富戶所捐幾何,貯庫助賑,殊非體制。此端一開,則偏災之地,貧民既苦艱食,富戶又令出貲。國家撫恤災黎,何忍出此?」責阿思哈卑鄙錯謬,不勝巡撫任,召還,奪官。尋授吏部員外郎。二十年,命以布政使銜往準噶爾軍前經理糧運。擢內閣學士。

二十二年,命署江西巡撫,蒞任,清理屯田,尋真除。學政謝溶生劾阿思哈婪賄派累,命尚書劉統勛、侍郎常鈞等按鞫,得實,擬絞。二十六年,詔免罪,以三品頂戴發烏魯木齊效力。二十八年,命往伊犁協同辦事。

二十九年,授廣東巡撫,調河南。三十年,疏言:「衛河運道淺阻,濬縣三官廟、老鸛嘴諸地砂礓挺據河心,重載尤艱浮送。向於上、下游淺處建築草壩以束水勢。詳考河形,夏秋水盛,無須草壩;冬令源澀,草壩亦屬無益。不如於上游先期蓄水,臨時開放。飭府縣督河員於九月望後起,至漕船出境止,暫閉外河以上民渠,使水歸官渠,重運自可疏通。鑿去砂礓,並集夫疏濬浮沙,以利漕運。」又請借司庫閒款,委員分購河工料物,以除沿河州縣按畝派累,均報聞。

三十四年,擢雲貴總督。師征緬甸,阿思哈出銅壁關至蠻暮軍中,奏軍中糧馬不敷。上責其畏難,解任,以副都統銜在領隊大臣上行走。旋召為吏部侍郎,入對失上指,奪官,戍伊犁。三十九年,釋回,仍充軍機章京。擢左都御史。大學士舒赫德師討王倫,命阿思哈偕額駙拉旺多爾濟率健銳、火器兩營以往。事定,拉旺多爾濟言城北搜剿王倫餘黨,阿思哈未同往,下吏議,奪官,命留任。四十一年,署吏部尚書,旋授漕運總督。卒,賜祭葬,謚莊恪。

阿思哈初撫江西,上眷之獨厚。廣西巡撫衛哲治入覲,上問各省督撫孰為最劣,哲治引罪,上謂:「姑置汝!」哲治舉阿思哈對,時以為難能。

宮兆麟,字伯厚,江南懷遠人。自貢生授湖北安陸通判,累遷至山東糧道。乾隆三十一年,授湖南按察使。桂陽州民侯七郎毆殺從兄岳添,賄其兄學添自承。知州張宏燧讞上,巡撫李因培疑之,令兆麟詳鞫得實。因培調福建去,巡撫常鈞庇宏燧,以七郎呼冤劾兆麟,兆麟亦入奏。上遣侍郎期成額會總督定長按治,如兆麟讞;兆麟又發宏燧買金行賄狀,期成額等奏聞,逮訊,買金非行賄,乃迎合因培及湖北布政使赫升額意指,代武陵知縣馮其柘補虧空。因培、赫升額、常鈞、宏燧皆坐譴。

三十二年,兆麟調雲南按察使。三十三年,遷布政使,擢廣西巡撫。雲南軍營需硝,敕兆麟籌畫,兆麟以廣西舊存硝七萬七千餘斤運剝隘,復撥通省營貯火藥二十萬斤繼運,得旨嘉許。調湖南。

三十五年,又調貴州。桐梓縣民為亂,命速赴任,會湖廣總督吳達善捕治。亂定,古州黨堆寨苗香要等為亂,復偕吳達善督兵捕誅之。兆麟奏黨堆寨苗老呴以阻香要亂被殺,令即寨立廟以祀良苗,並將死義被旌及香要叛逆伏誅狀,譯苗語榜廟門,俾令警戒;並請移駐將吏,建下江營土城,駐兵鎮撫。是夏,兆麟奏請於鄰省湖南、四川、廣西買米運貴州糶濟。至秋,豐收,復奏請停運。上斥其冒昧,勖令詳慎。兆麟復奏請簡發知府三員赴貴州,上以「此端一開,各省效尤,妨吏部選法;且開幸進之門」,下旨嚴飭。會貴州布政使觀音保入覲,訐兆麟粗率喜自言誇,口給便捷,人號為「鐵嘴」。上曰:「觀音保人已粗率,今尚以兆麟為粗率,則粗率更甚可知。」諭兆麟猛省痛改。尋詔詣京師,降補甘肅按察使。三十六年,坐貴州任內失察廠員虧欠鉛斤,奪官。四十一年,東巡,兆麟迎駕,詔與三品銜。四十六年,卒。

楊景素,字樸園,江南甘泉人,提督捷孫。父鑄,古北口總兵。景素孱弱,不好章句,貧不能自給。入貲授縣丞,發直隸河工效力。乾隆三年,補蠡縣縣丞,累遷保定知府。十八年,授福建汀漳龍道。漳浦民蔡榮祖欲為亂,景素率營卒擒斬之。調臺灣道。釐定漢民墾種地,並生熟番界址。革游民為通譯而不法者,代以熟番。又禁入山採木,借修造戰船材料為名,累諸番。三十三年,授河南按察使。三十五年,擢甘肅布政使,調直隸。命從尚書裘曰修勘察堤墊各工。坐失察雄縣知縣胡錫瑛侵蝕災賑,下吏議,奪官,命留任,俟八年無過,方準開復。

三十九年,壽張民王倫為亂,大學士舒赫德督兵討之。上命景素具車馬濟師,令分守河西。賊以糧艘結浮橋欲渡,景素與總兵萬朝興、副將瑪爾當阿等督兵御之,董勸回民助師。夜焚橋,賊不得渡。事旋定,擢山東巡撫。疏請編查保甲。四十年,疏請選京師健銳、火器營裨佐發山東,司營伍教演。四十一年,上東巡,臨視臨清毀橋斷道及亂民竄據所在,景素述當時戰狀,上嘉其勞,賜黃馬褂。汶上宋家窪舊渠淤墊,瀦水淹民田。四十二年,景素奏請濬舊渠,並開支河二,令仍趨南陽、昭陽二湖,下部議行。

擢兩廣總督,四十三年,調閩浙。疏言:「浙西歉收,總督楊廷璋請撥臺灣倉穀十萬接濟。北風盛發,未能即到。請於福州、福寧、興化、泉州四府屬撥倉穀十萬,聽商運赴嘉、湖出糶;仍飭臺灣運歸四府補倉。」得旨嘉獎。四十四年,調直隸。薦於易簡為布政使,上以易簡為大學士敏中弟,責景素。十二月,卒,贈太子太保,賜恤如例。

四十五年,兩廣總督巴延三奏景素操守不謹,並發官兵得贓縱盜狀。兩江總督薩載勘有河堤城垣工程,罰景素家屬承修。福康安又奏景素在兩廣婪索商捐六萬餘,責景素子炤限年繳還。五十四年,以福建吏治廢弛,追咎景素,戍召伊犁。五十九年,釋回。

閔鶚元,字少儀,浙江歸安人。乾隆十年進士,授刑部主事。再遷郎中,督山東學政。二十七年,自學政授山東按察使,調安徽。遷湖北布政使,調廣西、江寧。四十一年,遷安徽巡撫。四十四年,雲貴總督李侍堯以贓敗,罪至斬,下大學士、九卿議,請從重立決;復下各省督撫議,咸請如大學士九卿議。鶚元窺上指欲寬侍堯,獨奏言:「侍堯歷任封疆,勤幹有為,中外推服。請用議勤、議能例,稍寬一線。」上從之,侍堯得復起。

四十五年,調江蘇。四十六年,甘肅布政使王亶望坐偽災冒賑得罪,事連鶚元弟同知鵷元。上責鶚元隱忍瞻徇,知其事而不舉,降三品頂戴,停廉俸。四十八年,還原品頂戴,支廉俸如故。五十年,江南旱。五月,鶚元奏淮、徐、海三府如得雨二三寸,猶可種雜糧。上諭曰:「得雨二三寸未為霑足,焉能種雜糧?地方雨水,民瘼攸關。鶚元何得含混入告?」尋奏請截漕十萬石,淮、徐、海三府州被災較重,碾米治賑,如所議行。

五十五年,高郵巡檢陳倚道察知書吏偽印重徵,知州吳瑍置不問;牒上,鶚元亦置不問,揭報戶部。上諮鶚元,鶚元猶庇瑍不以實陳,乃遣尚書慶桂、侍郎王昶按治;責鶚元欺罔,奪官,逮鶚元等下刑部治罪。巡撫福崧劾鶚元得句容知縣王光陛牒發糧書侵挪錢糧,但令江寧府察覈。上責鶚元玩視民瘼,徇情骫法,命置重典。獄具,擬斬立決,命改監候。五十六年,釋還里。嘉慶二年,卒。

論曰:法者所以持天下之平。人君馭群臣,既知其不肖,乃以一日之愛憎喜怒,屈法以從之,此非細故也。焯、阿思哈、景素坐貪皆勘實,猶尚復起;圖爾炳阿匿災至面謾,反誅告者;兆麟口給,鶚元迎上指,至不勝疆政而始去之。高宗常謂:「朕非甚懦弱姑息之主,不能執法。」執法固難,自克其愛憎喜怒,尤不易言也。


\end{pinyinscope}