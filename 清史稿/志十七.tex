\article{志十七}

\begin{pinyinscope}
災異三

洪範曰:「木曰曲直。」木不曲直,則為咎徵。凡恆雨、狂人、服妖、雞禍、鼠妖、木冰、木怪、青眚、青祥,皆屬之於木。

順治二年二月,河源霪雨。三年二月,當塗恆雨。四月,南雄霪雨。四年四月,章丘霪雨四十餘日。六月,高郵大雨數晝夜;丘縣霪雨,平地水深二尺;蕭縣暴雨三閱月;永安州、安邑大雨。秋,壽元霪雨四十餘日,即墨暴雨連綿,水與城齊,民舍傾頹無算。五年春,新城霪雨六十餘日,水沒城及半;莒州大雨兩月;武城霪雨一百日;東平大雨渰禾。五月,衡水霪雨數旬;咸陽大雨四十餘日。八月,句容大雨,屋舍傾圮無算;陵川霪雨害稼;沁水霪雨兩月餘。六年五月,鳳陽霪雨八晝夜;阜陽、淮河霪雨晝夜不息。秋,沁水霪雨兩月餘,民舍傾倒。七年二月,全椒大雨。四月,射洪大雨三晝夜,城內水深丈許,人畜淹沒殆盡。五月,平陽霪雨四十餘日。六月,桐鄉霪雨。七月,解州、萊陽、萬泉霪雨;安邑大雨二十餘日,傾圮民舍。八年春,嘉興、海鹽、桐鄉霪雨。五月,潞安霪雨八十餘日,傷禾稼,房舍傾倒甚多。六月,江陰霪雨六晝夜,禾苗爛死;吳平大雨傾盆,一晝夜方息;當塗大雨。秋,沁水大雨,東阿霪雨,青浦大雨彌日。九年五月,陽信、霑化霪雨四十餘日,平地水深二尺;合浦大雨,城淹四尺。六月,壽陽霪雨四十餘日;襄陵霪雨兩匝月,民舍漂沒甚多;稷山霪雨;博興大雨傾盆四十七晝夜。七月,濟寧、東平霪雨害稼。九月,遵化州霪雨彌月。

十年五月,文安、大城、保定大雨如注十晝夜,平地水深二丈。六月,文登大雨三日,昌平霪雨,蓬萊霪雨四十餘日。十一年二月,蘭州大雨二十餘日。六月,亳州霪雨,壞民廬舍。七月,澄邁大雨,三日方止。十二年八月,鶴★州霪雨不止,田中水深三四尺。十三年五月,常山大雨。十五年二月,濟寧州霪雨傷麥,萬泉霪雨傷麥。秋,垣曲霪雨,儋州霪雨七晝夜,田禾多沒,城垣傾圮,興安、白河、洵陽霪雨四十餘日;平湖大雨數晝夜,平地水深二尺許。十六年正月,震澤、嘉定霪雨六十日方霽。二月,儀徵大雨彌月,平地水深丈餘。三月,蕭縣霪雨二十餘晝夜。秋,銅山霪雨三月餘,禾盡爛死;宿州大雨二十餘日,田廬漂沒殆盡;虹縣霪雨六十餘日,平地水深丈餘,淹沒田廬;梧州霪雨四十餘日;成都霪雨城圮。十七年五月,崇明大雨一晝夜;和平大雨,平地水深丈餘,漂沒田廬無算。十八年六月,貴陽大雨,武寧霪雨二月未止。閏七月,孝感霪雨三日,殺麥。

康熙元年八月,朝城霪雨七晝夜;曲沃霪雨二十日,壞城垣廬舍無算;成安霪雨五晝夜;永年霪雨匝月;吉州大雨,壞城垣廬舍;蕭州大雨彌月,城垣傾圮;解州大雨四十日;猗氏大雨二十餘日,民舍傾圮。四年春,蠡縣霪雨二十餘日。六月,府穀大雨。閏六月,延安霪雨彌月,壞廬舍。七月初七日,大城霪雨五晝夜,城垣倒壞十之六七,民房坍塌不下數萬間;東陽大雨,壞民舍。五年六月,福山霪雨兩月,禾稼盡傷。十一月,襄垣、武鄉大雨。六年六月,惠來大雨,平地水深三尺;房縣霪雨傷禾。七月,溫州大風雨,壞城垣廬舍;瑞安大風雨,壞城垣廬舍。七年四月,太平大雨如注。五月,太平積雨旬餘。六月,龍門大雨七日;武強霪雨;井陘大雨如注。七月,靈壽霪雨兩晝夜不止;元氏大雨七晝夜,城外水高數丈;真定府、懷來大雨七晝夜;內丘霪雨,淹沒民舍;房縣霪雨傷禾。八年六月,嘉興霪雨晝夜不息。九年五月,湖州霪雨連旬;德清霪雨連旬,田疇盡沒。六月,東陽大雨如注。

十年八月,嘉興大雨。十一年秋,寧波霪雨。十二年正月,海寧霪雨,至四月止。六月,高要霪雨四日,平地水深數尺,民舍傾圮;宿州大雨連綿兩日;陽江大雨。十三年正月,桐廬霪雨,至二月方止。四月,海豐霪雨十六日,平地水深數尺。六月,開平霪雨陷民居;高明霪雨,傷損禾稼。十五年五月,海寧霪雨匝月,傷禾;大冶霪雨。十六年七月,高密霪雨二十餘日,田禾淹沒。十七年正月,永年霪雨匝月。四月,平湖霪雨匝月。五月,金華霪雨傷稼。七月,太平霪雨,民舍傾圮;萊州、膠州大雨傷稼;萬載霪雨數晝夜。十八年八月,曲沃霪雨二十五日,城垣廬舍傾倒無算;太平霪雨;臨晉雨二十餘日,民舍盡圮;猗氏霪雨彌月不止;解州、安邑霪雨連旬;夏縣霪雨月餘,城垣傾倒,民居損壞,田禾淹沒;廣靈霪雨匝月不止;漢中霪雨四十日,如傾盆者一晝夜,淹沒民居;定遠霪雨四十日;甘泉霪雨彌月;興安大雨,田禾盡淹。十九年二月,襄垣大雨四十餘日。六月,高郵霪雨連旬,壞民舍無算。七月,龍門大雨,平地水深尺許;鎮洋霪雨累月;長子大雨四十日不止,城垣傾圮;蒲縣霪雨四旬,傷禾。八月,上海驟雨,城內水高五尺;咸陽大雨四十餘日。十一月,震澤霪雨三日。

二十年三月,處州大雨,至五月始止。四月,寧波霪雨一月。七月,階州大雨月餘,傾倒民房千餘間。二十一年三月,平遠州霪雨;紹興霪雨九旬,禾苗盡淹。五月,金華大雨五十餘日。二十二年春,蘇州霪雨十二日,殺麥;青浦霪雨傷麥;陽湖恆雨殺麥;海寧大雨,至四月始止;桐鄉恆雨,至四月始止;平湖自二月至四月大雨不止;湖州恆雨;石門恆雨傷麥;天臺霪雨至四月不止,二麥無收;太平霪雨,麥無收;浦江霪雨;衢州恆雨至四月,無麥;嚴州自春徂夏,陰雨連綿,二麥無收。五月,靈川大雨;通州霪雨,臺州霪雨,麥無收。六月,兗州大雨,平地水深三尺,田廬苗稼盡淹。二十三年春,恩縣霪雨;剡城霪雨,兩月不止。夏,昌樂霪雨害稼。七月十三日,臨縣大雨,至八月初八日止,平地水溢;太平霪雨四十餘日。八月,遂安霪雨兩月;隰州霪雨五十餘日,壞民舍甚多。二十四年四月,湖州大雨。六月,靈壽霪雨害稼;固安大雨,壞民舍。十月,福州大雨數晝夜。十二月,歙縣霪雨四十餘日,和順大雨連月。二十五年四月,宣平大雨五日,漂沒田廬,溺者無算;麗水大雨四晝夜,漂沒廬舍無算。閏四月,處州大雨,水高於城丈餘;松陽大雨四晝夜;景寧大雨三晝夜。六月,青州霪雨傷稼;壽光大雨兼旬。十一月,瓊州大雨連日如注,民舍多圮。二十六年六月,新城霪雨害稼。七月,章丘霪雨四十日,民舍傾圮千餘間。二十七年五月,玉屏大雨,壞城垣。二十八年四月,惠來大雨,廬舍淹沒無算。二十九年二月,開平大雨,至五月乃止。五月,湖州大雨一月,田廬俱損。七月,紹興大雨彌月,平地水深丈許,漂沒田廬人畜無算。

三十年六月,湖州霪雨害稼。閏七月,介休霪雨,東城圮數十丈。三十一年三月,武定大雨,平地水深丈許。秋,鎮安霪雨害稼。三十二年四月,丘縣霪雨四十餘日。八月,咸陽霪雨,墻垣倒者甚多。三十三年正月,海豐霪雨;咸陽大雨,水深二尺。十月,鄒平霪雨害稼。三十四年四月,盧龍大雨,壞城垣百餘丈。五月,房縣霪雨傷麥。六月,蘇州、青浦霪雨傷稼;固安大雨,平地水深丈餘。三十五年春,長山霪雨害稼。六月,昌邑霪雨害稼;樂平大雨彌月,沁州霪雨,三月方止。八月,饒陽大雨,七日方止;定州大雨八晝夜,傷稼;靜樂大雨兩晝夜;銅山霪雨,壞民居。九月,武定大雨七晝夜。冬,即墨霪雨六十日。三十六年正月,香山霪雨匝月。二月,遵化州大雨如注。三十七年八月,房縣霪雨傷稼。三十八年六月,南樂大風雨,拔樹。七月,杭州大雨,平地水高丈餘。八月,桐鄉、石門霪雨傷稼。三十九年正月,夏縣大雨壞城。

四十年九月,高密霪雨傷稼。四十一年四月,陽江霪雨,壞民居甚多。六月,寧陽、青州霪雨。八月初八日,香山大風雨,拔樹倒墻;寶雞霪雨。四十二年五月,慶雲霪雨,三旬不止。六月,東明、定州霪雨三旬不止;霑化霪雨連日,漂沒民舍無算;高苑霪雨六十日;昌邑、掖縣霪雨害稼;高密霪雨彌月,禾稼盡沒。八月,鄒平大雨害稼;齊河霪雨四十餘晝夜,民舍傾圮無算;濰縣、平度霪雨害稼。四十三年六月,沂州大雨;興安大雨,漂沒田廬。四十四年五月,萊州霪雨害稼;高郵霪雨閱月;鹽城霪雨越三月不止,平地水深數尺。十一月,江夏霪雨害稼。四十五年六月,東莞暴雨,平地水深五六尺,民居多圮。秋,宿州霪雨連月不止,傷稼。四十六年九月,吳川大雨四晝夜,傾圮民房無數。四十七年四月,石阡府霪雨。五月,嘉興大雨三日,田禾盡沒;海豐大雨三月,田廬悉被淹沒。六月,桐鄉恆雨,傷禾。七月,崇明霪雨百日;杭州暴風雨,田禾盡淹;江山大雨,壞民舍。四十八年三月,沛縣大雨六十日,湖州大雨連旬,銅山霪雨凡五月,咸陽大雨至五日始止。四月,石門霪雨傷麥。六月,宿州大雨如注,田禾盡沒;東平大雨,淹沒田禾;汶上大雨三晝夜,田禾淹沒;茌平霪雨兩月,民舍傾倒無算。秋,萊陽、榮成、文登霪雨害稼。四十九年秋,青浦霪雨十八日,桐鄉霪雨傷稼,東流大雨,淹沒田禾。

五十年十二月除夕,平樂驟雨達旦。五十一年七月二十二日,靈川大雨七晝夜。九月,鶴慶、龍川霪雨。五十二年四月,靈川大雨,平地水深數尺。五月,石城霪雨三月。七月,奉議州大雨,二旬始止,官署民房悉被淹沒。五十三年五月,遂安大雨連日,淹沒田禾。五十四年三月,震澤霪雨二十餘日。五十五年四月,武寧霪雨匝月。五月,湖州暴雨,平地水高六七尺;桐鄉霪雨,淹沒田禾。秋,桐廬大雨,平地水高尺許。五十六年七月,掖縣大雨,平地水深三尺;香山大風雨,壞屋舍;雞澤霪雨四日。五十七年三月,海陽霪雨,至五月始止。五十八年六月,雞澤霪雨四晝夜;萊州霪雨,壞民舍無算。七月,昌樂、諸城、即墨、掖縣霪雨害稼,壞民舍;萊陽、文登大雨水,房舍田禾盡沒。八月十九日,海陽大雨,損房舍無算。五十九年五月,龍南大雨閱月。六十年七月,高苑大雨,田禾盡淹。六十一年六月,霑化霪雨匝月。十二月,欽州大風雨,壞城垣二十餘丈。

雍正元年五月十九日,香山大雨,市可行舟;湖州恆雨,自秋及冬不絕。二年三月,麻城霪雨傷麥。夏,獻縣大雨六十餘日。三年五月,上海霪雨害稼;海豐大雨,至七月方止;東光大雨四十餘日。七月,青城霪雨兩月。八月,平原霪雨凡百日。九月,順德大雨三月。四年五月,震澤霪雨為災;當塗、無為大雨彌月,田禾盡淹;南陵霪雨,至秋不絕。六月,濰縣大風雨,壞民廬舍;慶陽大雨,平地水深四五尺。七月,陽信霪雨連旬。八月,杭州、嘉興、湖州大雨;青浦、蘇州、昆山霪雨十餘日,害稼。五年二月,吳興霪雨,鍾祥雨至四月不絕。五月,鎮海霪雨彌月。六月,揭陽、饒平霪雨一月。七月,惠來大雨害稼。六安州、霍山霪雨四十餘晝夜;陽信霪雨七晝夜,民舍傾圮甚多。六年五月,平利大雨,沖塌城垣六十餘丈。七年三月,陽春大雨,壞民居。八年五月,日照霪雨四十餘日。六月,東阿、泰安、肥城大雨七晝夜,壞民田廬殆盡;昌樂、諸城、掖縣、膠州、濰縣、日照、萊州霪雨兩月,壞廬舍無算。七月,丘縣大雨傷禾。八月,嘉興大雨,水害稼;鄒平、銅陵霪雨害稼。冬,齊河大風雨,傷禾稼。九年二月,連州大風雨,拔樹倒屋。六月,蒲臺霪雨害稼。秋,普安州霪雨,至次年春乃霽。十年六月,寧津大雨,平地行舟。十一年三月,沔陽霪雨。六月二十八日,景寧大雨。橋梁道路沖塌甚多。十二年春,五河霪雨。十三年五月,廣陽霪雨四十餘日。

乾隆二年八月,平陽大風雨七晝夜,田禾盡沒;祁州霪雨害稼;蔚州大雨三晝夜。九月,長子大雨,禾盡沒。三年秋,祁州大雨。四年五月,高要霪雨,壞民房。六月,瓊州霪雨閱月;東明大雨,平地水深三尺。五年七月,絳縣大雨害稼。六年五月,寧都霪雨。七年春,商南霪雨一百餘日。五月,山陽大雨,鹽城霪雨害禾稼。秋,泰州霪雨,阜陽霪雨一百二十餘日。八年四月,慶陽霪雨水夾旬。九年六月,資陽、仁壽、射洪暴雨如注,壞民房。七月,遂安霪雨六晝夜。

十年四月十六日,安遠驟雨,平地水高一丈餘,沖倒民房七百餘間。十一年五月,平度大雨,漂沒田禾;膠州霪雨害稼。六月,文登大雨傷禾;壽光、諸城霪雨閱月,田禾盡沒。十一月,高密霪雨兩月。十二年六月,福山、棲霞、文登霪雨匝月。七月,海豐大風雨,壞城垣數十丈;平陰、榮成大風雨,晚禾盡沒。十二年四月初五日,清河大風雨,民舍傾圮無數。五月,泰州、通州大風雨,拔木壞屋。十四年秋,清河霪雨兩月。十五年五月,高密霪雨害稼。六月,麻城大雨連旬,沖塌民房。十六年秋,平度州大雨兩月,福山、棲霞、榮成霪雨害稼。十七年八月,海豐大雨,淹沒田禾。十八年,高平自七月至十月霪雨;諸城大風雨,損禾。九月,解州陰雨連旬。十九年八月,石門大雨淹禾稼;桐鄉大雨數晝夜,淹禾稼;嘉興大風雨一晝夜,傷稼;日照霪雨。

二十年二月至四月,蘇州霪雨,麥苗腐。三月,蘄州大風雨,壞民居三百餘家;荊門州霪雨兩月不絕。五月,澄海狂風驟雨,沖倒城垣五十七丈,民房三百餘間。六月,蘇州大雨傷稼,高郵霪雨四十餘日。七月,贛榆大風雨害稼,石門、桐鄉霪雨害稼。八月,東明大風雨拔木,田禾盡淹;沂州恆雨。十月,潮州霪雨損麥。二十一年五月,介休霪雨,淹田禾六十餘頃。七月,曲沃霪雨數十日,廬舍多壞;芮城霪雨四旬,房舍多圮;和順霪雨二十餘日,害稼。八月,慶陽霪雨。二十二年夏,惠來霪雨連綿。七月,介休霪雨,淹田禾八十餘頃,廬舍沖塌大半。二十三年六月,介休大雨三日,淹沒田禾;陵川霪雨連月不止,房舍多圮。秋,長子大雨傷禾。二十四年四月,潮陽霪雨。六月二十九日,即墨大風雨一晝夜,大木盡拔,田禾淹沒。七月,潞安大雨兩月。二十五年五月,泰州連雨四十日。二十六年六月,雞澤霪雨。秋,垣曲霪雨四晝夜不止,城垣盡圮。二十七年四月,永年霪雨匝月始霽。七月,蘇州大風雨,積水經月,田禾盡沒;海鹽大雨壞民居;嘉善大雨,風拔木壞屋;桐鄉暴雨十餘日。二十八年七月,來鳳霪雨三晝夜,懷集多雨。二十九年八月,通渭雨經旬。

三十一年六月,即墨大雨三日,西南城垣頹。七月,臨邑霪雨三晝夜,平地水深數尺,壞民舍無算;黃巖大雨如注,平地水深丈餘,溺死無算。三十二年,南豐自正月雨至七月不絕。三十三年八月,永昌霪雨五十餘日。三十四年夏,湖州霪雨連旬。七月,仁和、海寧大風雨,淹沒田禾。三十五年八月,壽光大風雨害稼。三十六年五月,曲阜大雨,沂水霪雨。七月,長子大雨傷禾。三十七年八月,嘉興、石門、桐鄉大雨,自辰至午,水高丈餘。三十八年七月二十九日,薊州大風雨,拔木,熟禾盡損。三十九年六月,雲和大雨,二晝夜不息。七月,桐鄉大風雨,壞廬舍無算。

四十二年四月,山陽大風雨,拔木;代州大雨六日,水深數尺。四十四年春,江陵霪雨彌月。四十五年六月,常山大雨,民房多圮。四十六年正月,文登大風雨,傷稼。六月,濟南雨,水害稼;臨邑霪雨連月。四十七年八月,東昌、文登大雨,水壞民廬舍。四十八年秋,綏德州霪雨。

五十二年三月,山陽大雨傾盆,水高丈餘,漂沒人畜無算。五十三年秋,文登、榮成霪雨害稼。五十四年七月,潼關霪雨連旬,民居傾圮。五十五年四月,通州大雨,麥盡損。五月,莘縣霪雨,兩月始止。七月,濟南、臨邑、東昌大雨,平地水深數尺,禾盡淹。五十六年五月,保康大雨,水沖沒田廬,溺人無算;嘉興霪雨兩月。五十七年六月,房縣霪雨,至九月始止。五十八年八月,文登大雨。五十九年七月,青浦大雨十晝夜;嘉興大風雨,壞民舍;昌黎、新樂霪雨害稼。六十年五月二十一日,江山大雨一晝夜,壞廬舍,淹斃人畜。六月,石門霪雨。

嘉慶元年六月,滕縣大雨如注七晝夜。二年六月,武進大風雨,拔木壞屋。七月,寧都霪雨,壞民居。四年二月,監利大雨如注,平地水深尺許。七月,文登大風雨,傷稼。五年六月,金華大雨三日,傷稼。六年六月,邢臺、懷來、寧津大雨數晝夜,壞廬舍;清苑、新樂霪雨四十餘日。七年四月,義烏霪雨,禾盡淹沒。九年三月,桐鄉恆雨,傷麥。五月,嘉興、蘇州霪雨,傷稼。十年三月,嘉興、石門恆雨,傷麥。六月,黃巖大風雨,損稼。十一年夏,樂亭霪雨四十餘日。十三年五月,嘉興、石門大風雨,害稼。閏五月,新城大雨水,湖州霪雨。秋,漢陽霪雨彌月。十五年夏,臨邑霪雨四十餘日。十六年三月,永嘉霪雨匝月。七月,棲霞霪雨四十餘日。九月,榮成霪雨害稼。十七年春,嘉興、石門、桐鄉霪雨傷麥。十八年秋,東阿、曹縣霪雨四十餘日,田禾盡傷。十九年秋,漢陽霪雨傷稼。二十一年夏,滕縣大雨,平地水深數尺。二十三年五月二十日夜,濟南大雨水,壞城垣廬舍,民多溺死。六月,文登大雨,平地水深數尺,民多溺死。八月十三日,永嘉大雨如注十晝夜,平地水深數尺。二十四年六月,文登大風雨,害稼。二十五年七月,新城大雨九日,平地水深丈餘;宣平霪雨,壞田禾。

道光元年七月,涇州霪雨,沖沒橋梁田廬人畜。八月,臨邑霪雨連旬。二年五月,莘縣霪雨傷稼。八月,章丘、東阿霪雨四十餘日,壞田廬禾稼。三年三月,湖州霪雨,至五月不止;昌平霪雨傷麥;內丘大雨,三旬始止。四月,嵊縣霪雨,至九月始止。五月,金華、永嘉霪雨害稼;禮縣暴雨,漂沒民舍。七月,青浦霪雨兩月;泰州大雨,平地水深數尺,禾稼盡淹。四年二月,德州霪雨。五年八月,貴陽大雨,二十日始止。六年六月,宜昌大雨連綿,十日不止,損田禾。七年夏,恩施霪雨傷稼。八年七月,武城霪雨。十年五月,通山,崇陽霪雨連旬,漂沒田廬甚多。六月,恩施霪雨傷稼。八月,宣平大雨如注,民舍盡漂沒。十一年五月,永嘉大雨水,歉收;江夏霪雨彌月。六月,宜城、穀城霪雨二十餘日,傷稼。七月,菏澤、滕縣霪雨百餘日,平地水深數尺;曹縣大雨,水深二尺。十二年,光化霪雨,自六月至八月,禾苗盡傷;宜城大雨,晝夜不絕;定遠、保康霪雨兩月。七月,鄖陽大雨七晝夜,壞官署民房大半。冬,房縣霪雨害稼。十三年夏,湖州霪雨害稼。十四年四月,咸寧大風雨,拔木壞房。七月,麗水大風雨,平地水深數尺。十五年夏,即墨霪雨傷稼,文登、榮成大雨六十餘日。八月,宜城霪雨傷稼。十七年五月,崇陽、宜城霪雨害稼。十八年六月,益都、臨淄大雨水。十九年春,棲霞霪雨,南樂大雨。四月,招遠大雨十餘日;榮成大雨,至七月不止。九月,武進恆雨傷稼。二十年五月,邢臺大雨,平地水深三尺。六月,平谷霪雨匝月不止。二十一年二月,武進恆雨傷麥。二十二年七月,麗水大雨,漂沒田廬。冬至夜,滕縣大雨如注。二十三年五月,平度霪雨傷田禾。二十四年七月初九日,嵊縣大風雨,溺死男婦七十餘人。冬,松陽大雨連旬,壞田舍無數。二十五年春,棗陽霪雨八十餘日。六月,滕縣大雨,平地水深數尺,人多溺死。二十六年五月,東平大雨害稼。六月,樂平霪雨。二十八年,潛江自二月至七月雨不止。六月,光化大雨,平地水深數尺,三月始退,溺斃人無算;保康霪雨兩月,壞田廬無算。七月十四日,永嘉大風雨,壞孔子廟及縣署。十九日,景寧大風雨三晝夜,壞田廬無算。二十九年六月,樂亭大雨傷禾稼。七月,青浦霪雨五十日,湖州霪雨傷禾。三十年五月二十五日,兩當暴雨,漂沒人畜。

咸豐元年六月,禮縣霪雨四十餘日,傷禾。二年,青縣大雨傷禾。三年四月,靜海霪雨害稼。六月,永嘉、青田、景寧霪雨十晝夜;保康大雨十六日,漂沒田舍甚多;房縣霪雨七晝夜不止,壞田舍無算。七月,宜城大雨匝月,壞城垣一百五丈;遠州霪雨害稼。四年夏,湖州霪雨。五年七月初十日,景寧大雨如注,田廬盡壞。六年六月,昌平大雨傷稼。七年春,崇陽霪雨。八年四月,海囗縣大雨損禾苗。九年五月,蘇州大雨傷禾。十年二月,蘇州霪雨閱月。六月,寧津、東光大雨傷稼。十一年十一月,羅田大雨傷禾。

同治元年七月,蓬萊、黃縣、福山、招遠、萊陽、寧海大雨連綿,禾稼盡淹。二年春,應城霪雨傷麥。五月,青縣大雨傷禾。三年六月初十日,定海暴風疾雨,壞各埠船,溺死兵民無數。四年六月至七月,萊陽大雨,平地水深七八尺,禾稼淹沒,房舍傾圮無算。五年秋,魚臺霪雨,水深數尺,傷禾稼。六年八月,鄖陽霪雨三晝夜,壞官署民房甚多。七年五月,皋蘭、金縣大雨,至七月乃止。秋,景寧大雨,傾沒田廬無算。八年春,江夏霪雨損麥。四月,嵊縣大雨,壞田廬。九年六月,潛江霪雨傷稼。十年七月,東光、新樂、曲陽霪雨十餘日。十一年五月,東平霪雨害稼。十一月,青縣大雨害稼。十二年七月,太平大風雨,壞城垣數十丈,民房數百間。八月,化平霪雨不止,壞民舍。

光緒元年六月,日照大風雨,平地水深數尺。二年六月初八日,黃巖大風雨,拔木壞屋,田禾淹沒殆盡。三年六月,高陵大雨如注,平地水深三尺,田禾盡沒。四年九月,東平大雨傷禾稼。五年五月,登州各屬大雨四十餘日。六月二十一日,永嘉大風雨,壞官民居。八月,莘縣霪雨十日方止。六年三月,福山大雨。七年秋,灤州霪雨連旬。八年秋,宜城霪雨傷禾稼。冬,均州霪雨彌月。九年六月,化平大雨,水深四五尺,傷禾稼。十年八月,太平大雨,沖沒廬舍。十二年七月十四日,太平大風雨,二十日始止。十三年閏四月,德安大雨三日,水高五六尺。十五年七月二十六日夜,德安大雨如注,城崩百四十餘丈,淹斃男婦七十餘人。十六年六月,山丹驟雨壞城郭。二十二年春,寧津大雨壞民居。二十五年七月,秦安大雨連旬。二十七年七月,山丹大雨,平地水深數尺。

雍正三年七月,靈川五都廖家塘有村民同★入山砍竹不歸,一百四十餘日始抵家,所言多不經。

道光十七年,崇陽鄉民好服尖頭帽鞋,站步不穩,識者謂之服妖。

順治二年十二月,上海小南門姜姓家雞翼下各生一爪。三年八月,揭陽牝雞鳴數日乃已。四年四月,淄川民間訛言雞兩翅生骨,食之殺人,驗之果然。五月,忠州民家殺雞,腹內有一嬰兒;漢陽雞翅生爪。五年,崇明民家雞翼中生爪;巫山民間雞翅端皆生一爪如距;杭州民家雞生四足;湖州民家雞生四翼,能飛。十一年,合肥鄭家莊產一雞,三嘴、三眼、三翼、三足,色黃,比三日死。十六年,崇明民家雄雞生二卵。十八年,鎮澤民家雄雞生卵。

康熙十一年,廣平民家抱一雛雞,四足四翼。十二年,平湖民家雞生四足四翼。二十二年,迎春鄉民間雌雞化為雄。二十三年,麻城民田姓家雞生一卵,膜內皆有紋,其色硃;後七日又生一卵,有圖;又數日,毛成五色,飛去。

雍正二年,麻城雞翅遍生人指。五年,通州雌雞化為雄。

乾隆三十九年冬,慶元雄雞自斷其尾。六十年,貴陽民家雄雞生二卵,色赤甚鮮。

嘉慶十一年,樂清民家雞生四足。十七年,宜昌民間雞生四足,後二足微短,行不著地;又有三足者,其一生於尾下,如鼎足然。

道光元年秋,青浦民家雞翼兩旁生爪;湖州民家雞兩翅皆生五爪,飛去;永嘉雞翅生爪,食之殺人。十二年,永嘉民家雞四足,不能啼。二十二年,良鄉民家牝雞化為雄,能鳴,無距。三十年六月,蘄水縣民家雌雞化為雄,冠距儼然,唯啼聲微弱。

咸豐五年,隨州民家雄雞生卵。

同治元年六月,定遠民家雞生三足。六年,鍾祥民家雛雞生三翅。

光緒九年,興山民家雌雞化為雄。三十年,寧州民家雞生三足,後一足微短,行不著地。

康熙二十年五月,巴東鼠食麥,色赤,尾大;江陵鼠災,食禾殆盡。二十一年,西寧鼠食禾。二十二年夏,崇陽田鼠結巢於禾麻之上。二十八年,黃岡鼠食禾,及秋,化為魚。二十九年,孝感鼠食稼。四十二年,西鄉、定遠遍地生五色鼠。四十七年,黃濟鼠食禾。四十八年七月,崖州有鼠千萬卸尾渡江。五十二年五月,高淳、丹陽有鼠無數,食禾殆盡。六十一年夏,延安田鼠食稼。秋,安定黑鼠為災,食禾殆盡,有鄉民掘地得一鼠,身後半蝦蟆形,疑其所化也;清澗黃鼠食苗殆盡;葭州田鼠食苗。

雍正五年十一月,銅陵★鼠銜尾渡江。

乾隆元年,文縣鼠害稼。四年四月,什邡縣白鼠晝見羅寺經堂中,異香滿室。秋,彭澤★鼠銜尾渡江,食禾。十四年二月,中★田鼠食麥。十八年,池州田鼠叢生,忽入水化為魚。二十五年五月,池州田鼠叢生,有赤鷹來食之,遂滅。

道光四年,高淳鼠食麥。二十八年五月,沔陽常平倉忽有鼠數千頭在梁上,移時方散。

咸豐元年六月,德化★鼠銜尾渡江。四年,襄陽★鼠食禾。

同治七年,山丹田鼠食苗。九年二月,皋蘭土塊化為鼠。

光緒五年五月,三原鼠食禾殆盡。二十一年,西寧★鼠食苗。二十四年,皋蘭田鼠食麥。

順治六年十二月,咸寧木冰。十年十月,當塗雨木冰。十一月,江陰木冰,潛山木冰,宿州雨木冰。十二月,海寧木冰。

康熙元年十二月,嘉定木介。二十年正月朔,儀徵木冰。三十年正月朔,江浦雨木冰。三十一年正月朔,儀徵木冰。

雍正二年十二月,掖縣木介。

乾隆十一年正月,湖州雨木冰。二十年十二月,東流雨木冰。二十三年冬,諸城雨木冰。二十五年正月,曲阜雨木冰。五十五年十二月,黃巖木介,宣平木介。五十七年十二月辛卯,南陵雨木冰,五十八年正月,金華木冰。六十年冬,湖州木冰。

嘉慶三年十一月,崇陽木冰。

道光二十五年十二月,黃縣雨木冰。二十九年正月,登州木介。

咸豐三年冬,湖州木冰。四年十一月,黃岡雨木冰。十二月,武昌雨木冰。

同治二年正月,黃縣雨木冰。四年正月,武昌雨木冰。

光緒七年十二月,黃岡雨木冰。

順治元年,南陵上北鄉郭氏墓域有黃檀一株,腹內突產修竹數竿,外並無竹,觀者詫為異。二年七月,石門資福院僧鋸木,中有「太平」二字,墨痕宛然。三年,錢塘李樹生桃實;太蒼街銀杏樹孔中吐火,而木本無傷。四年五月,昆山西門外民家李樹生黃瓜。六年二月,封川李樹生桃。十一年七月初二日,婺源西寧村有楓樹自僕,居民薪其枝殆盡,十九夜有聲,樹忽自起。十二年三月,盧龍城東南角樓壁中出火,焚樓柱。十三年五月,曲陽文廟東古楊樹一株忽自焚,火數十丈,竟日不絕。十八年五月,石門李樹生黃瓜,長二寸,有子。

康熙三年六月,盧龍灤河溢,湧出材木無算,時修清節祠,適所用,有如夙構,人咸驚異。十三年春,含山、嘉定李樹生黃瓜。十六年,桐鄉李樹生黃瓜。十九年,封川李樹結桃實。二十二年四月,東陽、義烏李樹生桃,■木開梨花。二十三年,海鹽鄉民鋸樹,中有「王大宜」三字,清晰如寫。二十八年,黃岡李樹生黃瓜。四十五年四月,寧州通邊鎮白楊開花,狀如紅蓮。四十八年,秦州槐樹生蓮花。五十一年十一月,宿州樹頭生火。

雍正五年,津縣西鎮門內有唐開元所植荔支,是歲忽枯,至九年復活,枝葉茂盛,不遜於前。

乾隆元年,高淳李樹生黃瓜。五年,掖縣縣署古桐自焚。十五年九月,應城水陸寺楓樹夜放光,伐之乃滅。四十八年六月,桐鄉李樹生黃瓜。六十年夏,竹城大雨溪漲,有巨木數百,順流而下,時修學宮無材,適符其數;永嘉七聖廟大樟樹自焚,中藏竹箸無數。

嘉慶元年秋,鄖陽漢川水中湧出巨木無算。二年,枝江城東古樹作息哮聲。

道光二年,曹縣李樹生瓜。三年,隨州李樹生瓜。

咸豐六年六月,麗水大樹無故自倒。八年,黃安有大椿樹,每至午,樹中有笑聲。九年,武進李樹生瓜。

同治三年,京山李樹結桃實。五年,分宜玉虛觀古梓杪產素心蘭。

光緒三年,黃岡楓生梨實。二十二年,皋蘭民家杏樹開牡丹二★。

順治七年正月二十七日夜,望江西方有青氣★天。

康熙十七年六月十二日,平湖青眚見。


\end{pinyinscope}