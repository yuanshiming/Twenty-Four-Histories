\article{列傳一百六十八}

\begin{pinyinscope}
帥承瀛孫遠燡弟承瀚左輔姚祖同程含章康紹鏞

硃桂楨陳鑾吳其濬張澧中張日晸

帥承瀛,字仙舟,湖北黃梅人。嘉慶元年一甲三名進士,授編修,累遷國子監祭酒。先後督廣西、山東學政,歷太僕寺卿、通政使、副都御史,署倉場侍郎。授禮部侍郎,調工部、吏部。丁母憂,服闋,補原官,調刑部。論劾郎中寶齡婪賄狀,仁宗以承瀛到官浹月,釐剔宿弊,予議敘。奉命按山西雁平道福海、陜甘總督先福,罷之。又按山東徐文誥冤獄,得平反,劾承審官吏,降黜有差。

十五年,授浙江巡撫。浙鹽疲敝,議裁浙江鹽政,歸巡撫兼理,詔責承瀛整頓,疏言:「浙江運庫尚無虧挪,惟多移墊。擬以報存餘價追補,須足額後撥解。至收支數目,務劃清綱款,即有急務,不再以內款墊支。每年加價,應許停輸。向例灑帶鹽引,豫占年額,愈積愈多,請並停止,以紓商力。」又酌改章程十事:定鹽務官制,裁鹽政養廉,革掣規供應,灶課由場徵解,銷引先正後餘,引目通融行銷,收支力杜弊混,梟私商私並禁,掣驗改復兩季,甲商酌裁節費,下部議行。浙鹺自此漸有起色。寧波、溫、臺諸府濱海,土盜出沒,令兵船巡緝以遏其外,嚴詗口岸以防其內,洋面漸安。

兩江總督孫玉庭上八折收漕之議,廷臣多言其不可,下疆臣覆議。承瀛疏言:「漕弊始由州縣浮收,以致幫丁需索,而幫丁沿途用費亦因以漸增。迨幫丁用費愈大,需索愈多,州縣迫於幫費,有難循舊例徵收之勢,其究耗費歸之小民。由此包戶侵漁,刁衿挾制,積弊至不可回。八折之議,原以去其太甚,補救目前。無如因弊立法,而弊即因法以生。誠有如廷臣所議,惟嚴禁官役需索,沿途之規費除,即幫丁之用費省,而州縣浮收勒折之弊,亦力絕其萌,庶愛民恤丁兩有裨益。」疏上,前議遂寢。清釐倉庫虧缺,奏請先就現任各官次第彌補;又以浙西頻遭水患,應與江蘇合力疏濬,察勘形勢,偕孫玉庭等疏陳兩省水道原委,實共一流,請專任大員綜攬全局:詔韙之。尋去官。後陶澍至江蘇,乃先治吳淞江焉。

承瀛治浙數年,以廉勤著。陸名揚者,歸安鄉民,以抗浮收得一鄉心,久為官吏所嫉,請兵掩捕,鄉民集眾抗拒,而名揚逸。巡撫陳若霖遽以入告,遣兵往治,久之名揚始就獲。承瀛初至浙,誅名揚,後乃知由於官吏之釀變,深悔之。道光四年,丁父艱,服闋,至京,以目疾久不愈,乃乞歸。二十一年,卒於家。優詔軫惜,依總督例加恤,賜其孫遠燡舉人,尋祀浙江名宦祠。

遠燡,成道光二十七年進士,官編修。咸豐初,上書言軍事。納貲為道員,奏留江西勸辦捐輸。七年,總兵李定為粵匪困於東鄉,遠燡募勇往援。戰歿,予騎都尉世職,建專祠,謚文毅。

承瀛弟承瀚,嘉慶十年進士,由翰林院檢討歷官至副都御史,方正負時譽,名亞於承瀛。歿,祀鄉賢。

左輔,字仲甫,江蘇陽湖人。乾隆五十八年進士,授安徽南陵知縣,調霍丘。勤政愛民,坐催科不力免官,嘉慶四年,復之,補合肥,復以緝私役為鹽販毆斃獄坐奪職。尋初彭齡為安徽巡撫,薦輔人才可用,仁宗亦素知輔循名,能得民心,送部引見,復職,仍發安徽,補懷寧,遷泗州直隸州知州。河決,州境被災,輔躬親賑撫,民無失所。總督百齡疏保潔己奉公,政聲為一時最,以應升升用,擢潁州知府。十八年,盱眙民孫國柱誣周永泰謀逆,疆吏以聞。詔那彥成俟滑縣匪平,移師會剿,檄輔先率兵往。輔力言泗州屬縣無邪教,單騎往按之,得國柱誣告狀,大獄以息。尋捕誅阜陽教匪李珠、王三保等,予議敘。擢廣東雷瓊道,遷浙江按察使、湖南布政使。二十五年,就擢巡撫。

苗疆稅重,又苦官役苛擾,侍郎張映漢陳其弊,命輔偕總督陳若霖察治。奏減租穀二萬餘石,籌款買補倉儲六萬餘石,免民、苗積逋租穀七萬餘石。復挑補兵勇,裁撤委員,禁差役不得入苗寨,聽苗食川鹽,民、苗便之。長沙妙高★有宋儒張栻城南書院舊址,康熙中移建城內,已圮,規復重建,課通省士子,疏請御書扁額,以示嘉惠士林,詔嘉許焉。

輔官安徽最久,時稱循吏,晚被拔擢,數年中至封圻,年已老。道光三年,召來京,原品休致。十三年,卒於家。

姚祖同,字亮甫,浙江錢塘人。乾隆四十九年,南巡,召試,賜舉人,授內閣中書,充軍機章京,累遷兵部郎中。以纂輯剿平教匪方略,擢四五品京堂,補鴻臚寺少卿。歷通政司參議、內閣侍讀學士、鴻臚寺卿。二十年,出為河南布政使。請限制河工提款,清釐州縣交代,庫儲頓充。

二十一年,調山西,又調直隸。嚴查虧空,令州縣自報虧數,凡新任不得私受前任舊虧,其新虧者,勒停升補。倉穀自經饑祲,兼軍需支領,蕩然無餘。祖同飭各屬糴補數十萬石。雄縣、安州、高陽諸縣水道淤阻,連年漫溢,並遴員治理,相機疏濬。二十二年,畿輔旱災,重者二十有九州縣。先令停徵,截漕備賑;★歷災區,劾屬吏辦賑不實者;發米賈囤積數十萬石,責令平糶,民賴以濟。二十三年,仁宗東巡,灤河漲溢,祖同督造橋工成,賜花翎。面諭曰:「是非為橋工,因汝能實心辦事耳。」

二十四年,擢安徽巡撫。會河南大水,灌入渦河,下游諸縣被災,祖同乘小舟巡視賑恤。二十五年,調河南。時儀封大工未竣,黃、沁並漲,漫及馬營工壩尾,祖同相機堵御。疏陳政務雖多,河工為重;學習河務,以履勘為先。宣宗初即位,命祖同每屆旬以大工進占丈尺奏聞。及冬,口門漸狹,而大河冰堅,祖同親乘小舟督工鑿冰,歲杪大工始告蕆。道光元年,祖同疏陳河南情形,略曰:「河工之敝壞顯而易見,民生之凋瘵隱而難治。河工加價,自常賦三百六十餘萬外,逾額攤徵。衡工未已,睢工繼之;睢工未已,馬工、儀工又相繼接徵。此外復有各處堤工隨時攤徵之款,民力其何以堪?請概停緩三年,以紓積困。」從之。開封護城大堤,河溢時半圮,請繕完以資保障。

二年,河督嚴烺奏請馬營壩工拋護碎石,已奉俞允,復命祖同籌度。祖同言時當大堤放淤,遏其奔沖,既非順水之性,伏秋盛漲,壩西水勢加高,上游堤墊愈險,則河北可虞,且慮攔沁轉致攔黃,於實事為未便。乃下烺覆議,卒如祖同言。初,儀工經費,自祖同嚴覈弊竇,省帑金甚鉅。迨工員報銷,截長補短,蘄合成例,言官以浮冒入奏。是年,命左都御史玉麟、王鼎按之,事得白,而以八子錢五萬六千餘緡責祖同償補。八子錢者,工員以雜用不敷,議以銀易錢,銀一兩加扣八十文,祖同置弗問,卒以罣議,降補太常寺少卿。

五年,授陜西按察使。請建流芳祠以祀關中士女之死節義者。六年,詔來京另候簡用。七年,授廣東按察使。尋偕尚書陳若霖赴湖北察勘京山王家營堤工。未幾,召授通政司副使,累遷左副都御史。十八年,以年老重聽,原品休致。二十二年,卒。

程含章,雲南景東人。其先佐官吏捕殺土寇,懼禍,改姓羅。乾隆五十七年舉人。嘉慶初,大挑知縣,分廣東,署封川。坐回護前令諱盜,革職,投效海疆,屢殲獲劇盜,擢知州,署雷州府同知,率鄉勇破海盜烏石大,遷南雄直隸州;又坐失察屬縣虧空,革職,尋復官。以勘丈南雄州屬田畝,總督蔣攸銛疏薦,擢知府,補惠州。歷山東兗沂曹道、按察使、河南布政使。道光二年,疏言:「欲治河南,必以治河為先務。正本清源之道,在河員大法小廉,實心修築,加意堤防,自能久安長治。」宣宗韙其言,命每屆汛期,赴工稽查工料及工員才否。擢廣東巡撫,入覲,面奏請復姓,許之。調山東,又調江西。修築德化諸縣被水圩堤,設義倉,行平糶。

四年,召署工部侍郎,治直隸水利,上疏略曰:「雍正、乾隆間四次興大工,皆歷數年蕆事,費帑數百萬,自此畿內無水患者數十年。迨嘉慶六年後,河道漸淤。道光二三兩年淫雨,被水者多至百餘州縣。治水如治病,必先明病之源流,急則治標,緩則治本。循古人經驗之良方,參今時變遷之證候,然後疾可得而治也。天津為眾水出海孔道,諸減河皆所以洩水入海。東澱★環數百里,大清、子牙、永定、南運、北運五大川流貫其中。西澱容納順天、保定、河間三府二十餘河之水,南北兩泊容納正定、順德、廣平三十餘河之水,各有河道為傳送之區。今則消洩之尾閭無不阻塞,停蓄之腹部無不淺溢,流貫之腸無不壅滯,收納之脾胃無不平淺,傳送之機軸無不淤積,吐納之咽喉無不填閼,疏通之血脈無不凝滯,加以堤墊、閘壩、橋梁無不殘缺,霪潦一至,輒虞泛溢。此畿輔水道受病之情形也。伏思直隸河渠澱泊,前代不聞大患。自康熙三十九年以後,乃恆苦水潦,則永定、子牙二濁河築堤之所致耳。孫嘉淦有言,永定、子牙向皆無是,泥塗得流行田間,而水不淤澱。自永定築堤束水,而勝芳、三角澱皆淤;自子牙築堤束水,而臺頭等澱亦淤。澱口既淤,河身日高,則田水入河之道阻,於是澱病而全局皆病。即永定一河,亦已不勝其弊,總因濁水入澱,溜散泥沉,以致斯疾。此又畿輔水道致病之根原也。永定河自築堤以來,於今百有餘年。河身高出平地一丈有餘,既不能挑之使平,又不能廢堤不用,明知痼疾所在,無術可治。亦惟見病治病,多開閘壩以分其勢,高築堤墊以御其沖,使不致潰決為害而已。至通省全局工段繁多,自不能同時並舉。惟有用治標之法,先將各河澱挑挖寬深,取出之土即以築堤,使窪水悉得下注,然後廓清中部。俟大端就理,乃用治本之策,諸州縣支港溝渠,逐一疏通,俾民間灌溉有資,旱潦有備,三五年後,元氣漸復。此又辦理之先後次第也。造端宏大,倍於乾隆時,與其緩辦費多,不如速辦費少,計非一二百萬所能成事。請飭部寬籌經費,庶不致有始無終。」又疏陳應修各工,略謂:「治水在一『導』字。欲治上游,先治下游;欲治旁流,先治中流。挑賈家口以洩永定、子牙、北運、大清四河之水。挑西堤頭引河以洩塌水澱之水,挑邢家坨以洩七里海之水。另開北岸一河以分罾口之勢,修復減河以宣白、榆之源;挑濬三河頭水道,添建草壩,為東澱之扼要;挑濬馬道河、趙北口水道,為西澱之扼要。十二連橋橫亙澱中,亟應興修以利往來。修復增河,分白溝上游之勢,修復窯河,分白溝下游之勢,則水得就下之性,支派旁流,乃可次第導引。」疏上,並被嘉納。實授工部侍郎。尋調倉場侍郎。

五年,授浙江巡撫。六年,以病辭職,上以含章精力未衰,不許,調山東。七年,因浙江巡撫劉彬士治鹽操切,密疏劾其不職,命總督孫爾準按治不實,詔斥含章聽不根之言,無端入告,解職嚴議。彬士亦劾含章提用商綱銀,額外濫支,漏追餘款等事。含章疏辨,命總督琦善、學政硃士彥按之。詔以提用綱銀,歸還捐墊,僅屬見小,而先發妄奏之咎重,念其居官尚好,降補刑部員外郎。八年,授福建布政使,以病乞歸。十二年,卒。

康紹鏞,字蘭皋,山西興縣人,江西廣信知府基淵子。嘉慶四年進士,授兵部主事,充軍機章京。累遷郎中,擢鴻臚寺少卿。十八年,滑縣教匪起,紹鏞隨扈,以畿輔、山東、河南地形險易,將帥賢否,各鎮兵籍,列冊進御,受仁宗知。會有大名民人司敬武等十餘人傭工熱河、錦州,聞畿南寇起,馳歸,過山海關,關吏執之,誣其預聞逆謀,命紹鏞偕內閣學士文孚往鞫,白其誣,釋之。劾副都統以下,論如律。歷通政司參議、大理寺少卿。

十九年,出為安徽布政使。值大水,被災者四十餘州縣,倉穀缺乏,庫儲不給,勸紳商輸貲各恤其鄉,與官賑並舉,災民賴之。二十三年,就擢巡撫。宿州、靈壁以睢河堤堰崩圮,比年患水,紹鏞親往相視,奏請修復;又築無為州黃絲灘臨江堤千二百餘丈。先後捕獲鳳、潁等府土匪五十餘人,置諸法。二十四年,調廣東巡撫。

道光元年,詔各直省清查陋規雜稅,紹鏞疏陳,略曰:「廣東州縣所資辦公,專在兵米折價。因產穀少,民間皆原折納,相沿已久。在馴謹良民,向依舊規完納,而刁生劣監,不能無抗欠。有於正數之內絲毫無餘者,更有於正賦之內收不足數者,州縣往往以贏補絀,自行償補。今若定為折收額數,則所浮之價,悉為應輸之額,其掛欠代償,恐較前益甚。況貪官污吏,視所加者為分內應得之數,以所未加者為設法巧取之數。雍正時將地丁火耗酌給養廉,議者謂正賦之外又加正賦,將來恐耗羨之外又加耗羨。八九十年以來,錢糧火耗,視昔有加,不出前人所慮。兵米折價,與之事實相近。即能明察暗訪,堅持於數年之間,斷難遠慮周防,遙制於數十年之後。至雜稅及舟車、行戶、鹽當、規禮等款,名目不一,或此有而彼無,或此多而彼寡,願者減其數以求悅,黠者浮其數以取贏。究之浮者即浮,數已定而難改;減者非減,事甫過而仍加。此時毫發未盡之遺,即將來積重難返之漸。其中更有強狡之徒,向不完納平餘,致饋規禮。今以案經奏定,在有司視為當然,在小民視為非舊,兩相脅制,互為告訐,既不能指為官吏分外婪索,予以糾彈;又不能因民間不繳陋規,懲以官法:寬嚴兩窮。是雜稅諸項之難於清釐,較兵米折價尤甚。且各項所入,既名陋規,逐款臚列,上瀆聖聽,於國家體制,亦殊未協。事有窒礙,不敢不據實密陳。」疏入,與兩江總督孫玉庭所議同,其事遂寢。

二年,召署禮部侍郎。丁母憂歸,服闋,授廣西巡撫。禁土司科派擾累,懲土民刁訟者,緝治逸匪,邊境稍安。五年,調湖南,編查洞庭湖漁船,以軍法部伍之,盜無所容。澧州諸湖,上承涔水,下洩洞庭,兩岸悉垸田,地低下,水曳水不暢,檄道府率屬履勘疏濬,得可耕田萬四千餘畝,奏蠲淤田賦萬一千餘畝,從之。九年,入覲,面陳苗疆設立苗弁額數過多,倚勢虐使苗人,易激事端,請酌其可並省者,缺出不補,總督意不合,格不行。十年,召授光祿寺卿。尋值京察,以在湖南任內廢弛,降四品頂戴,休致。十四年,卒。

硃桂楨,字幹臣,江蘇上元人。嘉慶四年進士,授吏部主事。累擢郎中,遷御史。二十一年,出為貴州鎮遠知府。鎮遠民、苗雜居,無紡績之利,募工教織,於是始有苗布。大旱,民饑,急發庫藏平糶施粥,郡無殍人。事畢,自請擅動庫帑之罪,民感其惠。次年,感稔,爭醵金還庫。黃平州有盜,或告變,單騎臨之,呼眾縛為首者出,不戮一人,戍五人而已。興義苗閧,大吏已勒兵,桂楨曰:「此苗忿民欺,保不為變。」使人開諭,果服。在任三年,治行稱最,擢陜西潼商道。歷浙江按察使,甘肅、山東布政使。

道光三年,擢山西巡撫。丁父憂,服闋,署禮部侍郎。授倉場侍郎,嚴治花戶侵漁。初行海運,奏定漕糧到天津起卸撥運收貯章程,清覈於到壩之先,慎重於入倉之後,著為令。九年,遷漕運總督。疏言:「漕政之艱困,由於旗丁疲累,而水手多系無業游民,性成強悍,無以恤其力而服其心,寬猛皆無當,欲其不滋事甚難。惟密詗於未然,而重繩其既往。請責成督運官弁,遇有滋事者,立時拿辦者免議;日久無獲者重處。」時漕弊已深,桂楨力加整頓,必究弊源,不為苛刻,群情翕服。

十一年,調廣東巡撫,卻洋行陋規,遇事執法,外商獨嚴憚之。每月勾捕,不動聲色,臨事集官弁,曰往某所,閭里不擾,莠民斂跡。以儉素率屬,一日微服勘災歸,至西關,見千總輿從甚盛,叱止之,千總叩頭請罪乃已。惠、潮兩郡多械斗,數興大獄,痛繩以法,稍戢。創議諸郡山場荒地,援雷、瓊例,給照聽民墾種。設鄉約義塾,教養兼施,以弭匪僻。誡僚屬慎刑獄,治民以無冤濫始,每屆秋讞,多所平反。十三年,以病乞歸,宣宗時時詢其病狀,冀其出。二十年,卒,詔嘉「居官清正,勤政愛民」,依總督例優恤,賜其子鎮舉人,謚莊恪,祀鎮遠名宦祠。

陳鑾,字芝楣,湖北江夏人。嘉慶二十五年一甲三名進士,授編修。道光五年,出為江蘇松江知府。創行海運,鑾駐上海,多所贊助。署江寧,值下河諸縣水災,流民劫掠,預設防禁。設賑廠郊外,議宜散不宜聚,分各縣留養,大縣二千人,小縣千人,賑畢資遣,竟事無譁。調蘇州,歷蘇松太道、江西糧道、蘇松糧道、廣東鹽運使、浙江按察使,署布政使。水災治賑,親勘災湖州,諏訪土人,知湖高於田,漊港宣洩不暢,規建堤防,修築垸岸,以保田疇。十二年,遷江西布政使,調江蘇,護理巡撫。

鑾自為諸生時,兩江總督百齡闢佐幕,歷官江蘇最久,周知利病。會陶澍、林則徐先後為督撫,百廢俱舉,凡治漕,治運,濬吳淞江、劉河、白茆河,修寶山、華亭海塘,鑾並在事,澍、則徐皆倚如左右手。十六年,擢江西巡撫。明年,復調江蘇。十九年,陶澍以病解職,代署兩江總督。方嚴煙禁,籌海防,甚被倚畀。疏言:「自嘉慶以來,鄉曲細民多受邪教誘脅,為風俗人心之害,由於正教不明。請敕儒臣闡明聖諭廣訓,黜異端之旨,撰為韻言,布之鄉塾,俾士民童年誦習,以收潛移默化之效。」特詔允之。是年冬,卒於官,贈太子少保,依尚書例優恤。賜其子慶涵舉人,慶滋,光緒中官至江西按察使。

吳其濬,字瀹齋,河南固始人。父烜,兄其彥,並由翰林官至侍郎,屢司文柄。其濬初以舉人納貲為內閣中書。嘉慶二十二年,成一甲一名進士,授修撰。二十四年,典試廣東,其彥亦督順天學政,詞林稱盛事。道光初,直南書房,督湖北學政,歷洗馬、鴻臚寺卿、通政司副使,超遷內閣學士。十八年,擢兵部侍郎,督江西學政,調戶部。二十年,偕侍郎麟魁赴湖北按事,總督周天爵嫉惡嚴,用候補知縣楚鏞充督署讞員,制非刑逼供,囚多死,為言官論劾,大冶知縣孔廣義列狀訐之,訊鞫皆實,復得楚鏞榷鹽稅貪酷,及天爵子光岳援引外委韓云邦為巡捕事,天爵論褫職戍伊犁,革光嶽舉人,鏞荷校,期滿發烏魯木齊充苦役,巡撫伍長華以下降黜有差。命其濬署湖廣總督,尋授湖南巡撫。

二十二年,崇陽逆匪鍾人傑作亂,進窺巴陵,其濬偕署提督臺湧赴岳州防剿,檄鎮筸兵分布臨湘、平江諸隘,其濬移駐湘陰,賊襲平江,擊卻之。及人傑就擒,餘黨竄湖南者以次捕誅,被優敘。部議裁冗兵,其濬疏言:「湖南地逼苗疆,人情易擾。裁者無多,徒生驕卒之疑,而啟苗、瑤之伺。」總督裕泰尋定議苗疆近地並仍舊額。二十三年,調浙江,未行,武岡匪徒聚眾阻米出境,戕知州,捕治如律。奏請於洪崖洞設巡卡,編保甲,以靖禍萌。尋調雲南巡撫,署云貴總督。二十五年,調福建,又調山西,兼管鹽政。奏裁公費一萬兩,嚴捕煙販,時稱其清勤。二十六年,乞病歸。尋卒,贈太子太保,照例賜恤。尋復以其濬在山西裁革鹽規,潔己奉公。特加恩子孫以彰清節:子元禧主簿,崇恩知縣,榮禧通判,皆即選;又賜其子承恩、洪恩及孫樽讓舉人。

張澧中,字蘭沚,陜西潼關人。嘉慶二十二年進士,授刑部主事,充提牢,累遷郎中。執法明允,數從大臣讞獄黑龍江、奉天、江南、山東。道光十二年,出為直隸大順廣道。奸民倡無生教惑眾,澧中率兵役探其巢穴,得圖卷及名冊,悉焚之,歸正者概不株連。署按察使,遷福建按察使。署布政使,授直隸布政使,未之任,調山西,署巡撫。二十年,擢雲南巡撫,於刑獄尤矜慎。二十三年,召署刑部侍郎,尋實授。

二十七年,河南洊饑,頒庫帑百萬,命澧中偕尚書文慶治賑務。至,即飭查造丁口,按冊抽查戶口;調取籓庫戥抽查賑銀;令州縣按旬具報錢價,以備考覈;劾冒賑之考城令及造報舛錯各員。

尋授山東巡撫。清查交代,定追賠章程,考察鎮道等官失察盜案多寡,分別劾議。嚴責捕盜,先後獲匪盜七百餘名,治如律。疏言:「山東地廣民稠,一遇歉歲,曹州之捻匪,沂州之掖匪、幅匪,武定、臨清屬之梟匪,聚眾每至百餘人,隨地裹脅,蔓延不已。群匪多起於曹、沂,而兗、濟受害為尤甚。地方官展轉稽延,不能即正典刑,匪徒遂無顧忌。惟官不以盜為事,民始敢與盜通聲氣。殲厥渠魁,脅從自散。即牧令中亦非無長於緝捕勇敢任事之員,惟大法則小廉,人存則政舉。兇匪之橫行,咎在牧令;牧令之不職,責在上司。」詔嘉勉之。尋卒,依侍郎例賜恤。

張日晸,貴州貴築人。嘉慶二十二年進士,選庶吉士,授編修。道光九年,出為四川敘州知府,調成都。日晸勤於吏職,刻樹桑百益書以勸民蠶,創「勵節堂」以贍節婦貞女之無依者。政暇,招諸生於署,講析經義、語錄。郡屬馬邊、屏山等縣,毗連惈夷,令附近居民建修碉堡,編聯保甲,民賴以安。擢建昌道。十九年,越巂、瓘邊夷匪滋事,偕總兵包相卿督兵平之。招復逃亡,編集練勇,修築碉堡,於要隘建城,以資保障。遷浙江鹽運使,再遷湖北按察使,調四川。治獄平恕,不以平反矜能,遇有疑竇,飭另緝改辦,告戒屬吏以哀矜為重。遷河南布政使。河決中牟,值祥符工甫竣,兩次災區二十五州縣,附省災尤重。每馳詣賑所監視,於郊外隙地捐俸構屋,安戢災黎,遂成村聚。二十六年,擢雲南巡撫,未之任,丁母憂。服闋,仍授雲南巡撫。勤於察吏,免銅廠民欠工本銀六千餘兩。在任一年卒,祀四川、雲南名宦祠及鄉賢祠。

論曰:宣宗以恭儉為治,一時疆臣多清勤之選。帥承瀛等或由卿寺受知,或以守令拔擢,雖間有旋倔旋起、晚置閒散者,其猷為要並可觀焉。硃桂楨實心實政,治績稱最,獨膺易名之典,蓋非幸雲。


\end{pinyinscope}