\article{列傳四十八}

\begin{pinyinscope}
楊捷石調聲萬正色吳英藍理黃梧子芳度從子芳世芳泰

穆赫林段應舉

楊捷,字元凱,義州人,先世居寶應,明初,以軍功授後屯衛指揮使,世襲,遂家焉。捷初為明裨將,順治元年來降,授山西撫標中軍游擊。嵐縣土寇高九英等聚眾剽掠,巡撫馬國柱檄捷捕治,斬九英,毀其巢。國柱遷總督,以捷為督標中軍參將,旋擢副將。

四年,師定廣東,命捷率宣化、大同兵三千往鎮撫。五年,行次池州,金聲桓、李成棟叛。大將軍譚泰請以捷駐防九江會剿,即授九江總兵,率兵復都昌,獲聲桓所置吏餘應柱等,斬之。江西平,敘功,予世職拖沙喇哈番。十年,從靖南將軍喀喀木討廣東叛鎮郝尚久,復潮州。調陜西興安,經略大學士洪承疇請留原鎮,加右都督。調福建隨征右路總兵,十二年,敘復潮州功,進左都督。鄭成功侵掠福建,與戰雲霄、銅山諸處,屢捷。十六年,擢江南提督。會成功陷鎮江,窺江寧,加太子少保,充江南隨征左路總兵,駐揚州,防江北要汛。十八年,命署廬鳳提督,尋調山東。土寇於七敗竄入海,捷捕治其黨五十餘人,誅之。

康熙十二年,調江南。十七年,鄭錦攻漳州,陷海澄。調捷福建,轄水陸各軍,進少保兼太子太保。疏言:「臣前剿賊雲霄、銅山間,深知閩兵不力戰。自任江南提督,召募材健,訓諫有年。擬選三千人隨征福建。」詔允之。捷至福州,聞錦犯泉州,即督兵趨惠安。錦將劉國軒斷洛陽橋,以三千人據陳山壩阻我師,捷遣游擊李璉等襲破之。總兵黃大來與副都統禪布等會師洛陽橋南夾擊,國軒遁,泉州平。錦將王一鵬復窺惠安,捷令總兵張韜御之,捕斬略盡。其別將葉明、紀朝佐等出沒德化、永春間,蕭武等以舟師泊湄州,窺興化。捷遣將防守策應,移師至漳州。偕副都統吉爾塔布等敗國軒於江東橋,又分兵屯守柯坑山、鳳山、萬松關諸要隘,遣別將扼守榴山寨。

捷初上官,疏請別設水師提督,得以專御陸路。上授捷昭武將軍,領福建陸路提督事。十八年,國軒率眾劫榴山寨,欲奪江東橋。捷會平南將軍賚塔等分兩翼夾擊,大敗之於下坑山及歐溪頭,斬級千餘,獲甲仗無算。國軒屯獅子山,聯絡遠近各寨為聲援。十九年,捷親率健卒剿平烏嶼諸寨,與總督姚啟聖、總兵姚大來等分下玉洲、三汊、石碼,連破十九寨,進取海澄。錦將蘇侃以城降,遂乘勝與浙江提督石調聲復廈門,國軒自銅山竄歸臺灣。

是年,以老病乞罷,命還任江南提督。敘復海澄功,進世職三等阿達哈哈番。三十九年,卒,年七十四,贈少傅兼太子太傅,謚敏壯。孫鑄,襲職,請改籍揚州衛。

石調聲,漢軍鑲黃旗人。以佐領從征廣東,敘功,予世職拖沙喇哈番。遷參領,駐防福建。擢杭州副都統。耿精忠犯浙江,調聲迎擊,屢卻賊。擢浙江提督。康熙十七年,鄭錦遣劉國軒等犯海澄,詔趣調聲赴援,未至而海澄陷,康親王檄守惠安。賊陷同安,遂圍泉州,惠安亦陷。調聲退軍興化,與參贊大臣禪布攻復惠安,逐北至洛陽橋。泉州圍解。復偕副都統沃申破賊江東橋。頃之,國軒等復奪橋,斷餉道,將軍賚塔檄調聲迎擊,敗之。十九年,復廈門、金門,國軒遁。調聲還浙江任。初賊陷江山、惠安,戰士暴骨多未瘞,議者以咎調聲。二十一年,追論奪官及世職。尋卒。

萬正色,字惟高,福建晉江人。少入伍。以招降海寇陳燦等,敘功,授陜西興安游擊。康熙十二年,吳三桂反,正色從西安將軍瓦爾喀征四川。叛將譚弘等據陽平關拒戰,敗之於野狐嶺,乘勝復廣元、昭化。累擢岳州水師總兵。時三桂據岳州,扼守洞庭湖套,植木為椿阻我師。十七年,正色上官,率舟師夜入亂葦中,拔椿盡,擊賊,屢敗之。三桂將江義、巴養元、杜輝等率舟二百攻柳林嘴,正色與游擊唐善等擊之,毀其舟。是歲三桂死於衡州,其子應麒與輝、義等守岳州。正色遣千總魏士曾齎書十四分致應麒部將,士曾為所殺,應麒亦殺部將之受書者,遂內訌。其將陳華、李超、王度沖出降,應麒棄城遁,遂復岳州。正色為士曾請恤,贈守備。十八年,追敘克陽平關功,加左都督。

大將軍康親王傑書徵福建,耿精忠降,而鄭錦猶踞金門、廈門,陷海澄。正色自以閩人習海上事狀,因陳水陸戰守機宜,言:「福建負山枕海,賊蹤出沒靡常。宜擇官兵習於陸者分布要害,使賊不得登岸;水軍自萬安鎮順流直下金門,塞海澄以斷其歸路。賊自廈門來援,則從金門掩擊。更請蠲除沿海邊地雜派,設法招撫,善為安置,則賊黨自散。」疏入,詔加太子少保,調福建水師總兵,擢提督。時議檄調荷蘭國船進取廈門,正色疏言:「荷蘭船遲速莫必,延至三四月,風信轉南,即難前進。今新舊鳥船俱集,臣與撫臣吳興祚決計進討,臣率水師直攻海壇,興祚率陸兵為聲援。」

十九年,正色征海壇,分前鋒為六隊,親統巨艦繼之,又以輕舟繞出左右,並力夾攻,發砲擊沈敵艦,溺死三千餘人,遂取海壇。其將硃天貴遁,正色追躡至平海澳,天貴走崇武,正色掩擊,大敗之。與將軍拉哈達、總督姚啟聖、巡撫吳興祚、提督楊捷會師取廈門,天貴降。

錦竄歸臺灣。疏請分兵鎮守濱海要地,上遣兵部侍郎溫岱蒞視。尋議銅山、廈門諸處量設總兵以下官,留水師二萬人分鎮之。初,海壇既克,下兵部敘功。啟聖語溫岱:「正色先與天貴約乃進兵,未嘗與賊戰。」兵部疏聞,上命仍議敘,予世職拜他喇布勒哈番。上諭正色規取臺灣,正色請緩師。二十年,改陸路提督。

二十五年,調雲南。未幾,與鶴慶總兵王珍互訐,命與珍詣京師質問。總督範承勛劾正色納賄侵蝕,上遣侍郎多奇、傅拉塔按治,下刑部論死,上以正色功多,特宥之,奪官,仍留世職。三十年,卒。

吳英,字為高,福建莆田人。幼為海賊掠置島中,更姓王。康熙二年,赴泉州降,授守備劄。從提督王進功攻鄭錦,拔銅山城,加都司僉書銜。尋授浙江提標都司。

十三年,耿精忠反,其將曾養性侵浙,總兵祖弘勛以溫州叛應之,分犯寧波、紹興。英從提督塞白理擊敗之,降其將李榮春等,遷左營游擊。十四年,養性、弘勛率眾十餘萬犯臺州。英言於塞白理,陽修毛坪山徑,潛引兵間道自仙居襲賊後,賊踞黃巖半山嶺拒戰。英偕游擊曾承等冒矢石前進,斬其將劉邦仁等,遂復黃巖,遷中軍參將。

十五年,貝子傅拉塔規復溫州,養性、弘勛率三萬人乘夜劫營。英分兵五百伏賊後,自率精銳據大羊山,阻其要道,遇賊,殊死戰,身中數槍。師繼進,伏盡起,賊大潰,斬獲無算。尋從提督石調聲援象山,賊屯石門、西溪二嶺。英偕游擊侯奇等分兵三道抵慈谿,擊沈賊船,殲其眾,遂復象山。九月,康親王傑書進徵福建,精忠降,養性、弘勛引退。其將馮公輔猶踞松陽,英入山,招之降。其黨林惟仁等屯處州,英剿撫兼用,斬賊五百餘,降惟仁及兵千餘。

十七年,錦犯泉州,康親王檄調聲赴援,英率師從。錦將劉國軒據洛陽橋,英自上游陳山壩渡江,以奇兵出賊後,造浮橋濟師,前後夾攻,斬級六百有奇。遷福建督標中軍副將。率師援漳州,連克十九寨,轉戰至江口,發砲擊沈敵船,遂復海澄。十八年,國軒復擁眾數萬屯郭塘、歐溪頭,謀奪江東橋,英擊走之,擢同安總兵。

十九年,偕寧海將軍拉哈達、巡撫吳興祚自同安港口分兵,進克廈門,錦遁歸臺灣。是年英奏請復姓。二十二年,移興化,會施瑯進攻澎湖。英偕總兵硃天貴、林賢等自八罩嶼乘風進擊,游擊藍理陷圍,英沖入敵陣,拔之出。翌日,進取虎井嶼,英右耳中槍,益力戰,躍入敵艦,手刃其將鄭仁,餘悉駭竄。國軒與鄭克塽乞降,事具瑯傳。

二十四年,入覲,奏言:「臺灣地勢絕險,土番止求衣食,素無他原。自來小寇竊發,皆由內地奸民作崇,陸師搜捕易盡。前議設水師趕繒雙篷船百,請減十之八,留二十船分撥臺灣、澎湖二處,傳遞文書。臺灣、澎湖經制官兵一萬員名,前議以鹿皮、白糖通洋助餉,不能如期給發。臣見臺灣民田之外,別有水田,俱屬鄭氏親黨及其部將,耕牛甚多。請分四千屯田,每兵給田三十畝、牛一,課耕種。農隙操練,則兵有恆產,餉可省半。」疏入,命議行。尋移鎮浙江舟山。擢四川提督。

英先以軍功加左都督,授世職拖沙喇哈番。敘平臺、澎功,進世職三等阿達哈哈番。三十六年,調福建陸路提督,改水師。上南巡,英朝行在,賜御書榜額。召見,問:「福建今有無海寇?」英對曰:「海寇斷不至蔓延,若蔓延,任臣等何用?惟海中與城郭不同,一水汪洋,乘一小舟,隨處可藏匿。商賈失利,不得已走而為盜,往往有之,不可遽謂之海寇也。」上降詔獎英篤實而明達,尋授威略將軍,仍領水師提督事,復御制詩賜之,勖以黽勉防微。五十一年,卒,年七十六,贈太子少保。

藍理,字義山,福建漳浦人。少桀驁,膂力絕人。集族人勇健者擊殺海寇盧質,詣吏,欲因以為功,吏疑亦盜也,系之獄。康熙十三年,耿精忠反,悉縱系者,令赴籓下授職。理間道走仙霞關詣康親王軍降,為鄉導,破叛將曾養性於溫州。十五年,從師入閩,授建寧游擊。十七年,從都統賚塔敗海寇於蜈蚣統賚塔敗海寇於蜈蚣山,復長泰。十八年,遷灌口營參將。十九年,總督姚啟聖駐師漳浦,令理分兵守高浦,辭不赴,劾理虛兵冒餉,坐奪官。下部議罪,擬杖徒,理請剿海寇自贖,上允之,發軍前效力。

二十一年,提督施瑯征臺灣,知理英勇,奏署右營游擊領舟師,部議格之,特旨允行。瑯令理當前鋒,諸弟瑤、瑗、珠皆從。鄭克塽遣其將劉國軒守澎湖,令曾遂等率眾數萬迎敵,戰艦蔽海。理督兵與戰,自辰至午,戰益力。遂發砲,彈掠理而過,理僕,遂遙呼曰:「藍理死矣!」瑤扶理起立,理亦呼曰:「藍理在,曾遂死矣!」呼刀,族子法以授理,見理腹破腸流出,為掬而納諸腹,瑗傅以衣,珠持匹練縛其創。理呼殺賊,麾兵進,擊沈敵艦二,敵大潰。瑯過理舟慰勞之,令治創復戰。瑯舟膠淺沙,敵艦環圍之,理聞,赴援。理舟書姓名篷上,敵憚理,戰為稍卻,追擊,大敗之。得敵艦,請瑯易舟,出,逐敵至西嶼,殺傷殆盡,遂克澎湖。臺灣平,敘功,仍授參將,加左都督。

未幾,丁父憂。二十六年,服闋,詣京師,迎駕趙北口,召至御前,問澎湖戰狀,命解衣視其創,慰勞甚至,超授陜西神木營副將。尋擢宣化鎮總兵,掛鎮朔將軍印。二十九年,移定海。四十二年,復移天津。賜花翎、冠服,並御書榜曰「所向無敵」賚焉。四十三年,以舊傷疾作,乞解任,溫旨慰留,遣御醫診視。理以畿輔地多荒窪,請於天津開墾水田百五十頃,歲收稻穀,民號曰「藍田」。

四十五年,擢福建陸路提督。四十六年,上南巡,理迎駕揚州,賞賚有加,復御書榜曰「勇壯簡易」。四十七年,丁母憂,命在任守制。五十年,巨盜陳五顯等糾二千人擾泉州永春、德化諸縣。事聞逾數月,理始疏陳,並言村落安集如故,上斥其誑,命奪職,總督梁鼐、巡撫滿保先後劾理貪婪酷虐諸狀,遣侍郎和託、廖騰煃會督撫按治得實,論斬,詔從寬免死,入京旗。五十四年,師北征,剿策妄阿喇布坦,理請赴軍前效力,賜總兵銜,從都統穆爾賽協理北路軍務。以病回京,尋卒。詔免所追銀兩,遣其妻子回籍歸葬。

理虓勇善戰。性率直。官福建提督,政行於鄉里。捕治盜賊,遂及諸豪家。修橋梁,平道路,率富民錢,益積怨。泉州民繪虎為榜,列理諸累民狀,以是得罪。上念其舊功,終矜全之。弟瑤,未仕;瑗,官至金門總兵;珠,累官參將。

黃梧,字君宣,福建平和人。初為鄭成功總兵,守海澄。順治十三年,梧斬成功將華棟等,以海澄降。大將軍鄭親王世子濟度以聞,封海澄公。十四年,總督李率泰疏請益梧兵,合四千人,駐漳州。梧與李率泰及提督馬得功、都統郎賽水陸分道進,破七城,克閩安鎮。敘功,賜甲胄、貂裘,加太子太保。梧牒李率泰,薦委署都督施瑯智勇忠誠,熟諳沿海事狀,假以事權,必能剪除海孽;又言成功全藉內地接濟,木植、絲綿、油麻、釘鐵、柴米,土宄陰為轉輸,齎糧養寇,請嚴禁;並條列滅賊五策,復請速誅成功父芝龍。率泰先後上聞,瑯得擢用,芝龍亦誅。尋命嚴海禁,絕接濟,移兵分駐海濱,阻成功兵登岸,增戰艦,習水戰,皆用梧議也。

及成功病卒,其將萬義、萬祿、楊學皋、陳莽、陳輝、顏立勛、黃昌、黃義、餘期英等詣梧降。康熙二年,師攻廈門,靖南王耿繼茂出潯尾,梧偕李率泰出蒿嶼,督水陸將卒夾擊,斬獲無算,遂克廈門、金門、浯嶼三島。鄭錦遁據銅山。繼茂令梧統兵駐雲霄防剿。三年,梧招錦將周全斌、陳升、黃廷、何政、許貞、李思忠等來降。遂偕繼茂、李率泰及提督王進功乘夜渡海,拔銅山。錦走還臺灣。

梧疏言:「自海上歸誠,十二年中,先後招撫文武吏二百餘、兵數萬人,有蒙賜封侯伯且世襲者。臣公爵未定何等及承襲次數,乞敕部覈議。」尋命定封一等公,世襲十二次。七年,兵部議裁汰諸行省兵額,梧標下額定官三十員、兵一千二百人,餘移駐河南。十三年,耿精忠反,傳檄至漳州。梧方病疽,聞變驚恚,遂卒。

子芳度,字壽巖。梧既卒,陽以梧命答精忠,而陰募兵自守,凡二月餘,得壯士六千人,遂斬精忠所置都督劉豹等,誓師登陴,以蠟丸函疏,遣黃藍間道馳奏。上嘉梧忠藎,降詔優恤,以芳度襲爵;並諭師自浙江、江西、廣東三路入福建。芳度詗何路兵先到,迎會合剿。尋疏言:「漳州介耿、鄭二逆間,自八月以來,堅與耿拒,偽與鄭和。因得陰行招募,練成勁旅萬人,分布漳城及龍溪等五縣。無何,耿逆來犯,臣率眾迎擊,擒斬無算。二逆構怨已深,勢必俱敗。誠得粵省大兵乘勝進攻,臣當率師迎會,迅奏掃除之功。」十四年,復言:「臣拒耿餌鄭,固守一載有餘。近二逆通好,臣謀已洩。鄭逆遂撤回各鎮,蜂聚海澄,備糧繕器。臣知其狡謀,遣總兵楊壯猷等扼守平和,並令臣從兄芳泰突圍赴廣東,接引大兵。鄭逆率眾圍城,晝夜攻擊。臣連次出兵,斬其將黃鼎新、盧英等。但孤城缺餉,百計難支。計粵路援師,旦夕可至。乞密敕浙江、江西兩路兵迅速進發,俾二逆不能相顧,臣可會合奏功。」

漳州自五月被圍至七月,敵來益眾,豎雲梯攻城,砲毀城堞三十餘丈。芳度率將士拒戰,殲賊無算。敵環攻不退,芳度連疏告急。詔趣統兵諸將迅速赴援,並撥餉接濟。十月,城中糧盡,叛將吳淑引賊陷城。芳度率兵巷戰,力竭,赴開元寺井死,年二十有五。賊戕其尸,母趙、妻李自經。從父樞、從兄芳名、弟芳聲、芳祐並死。期功男女從死者三十餘人。賊又斫梧棺,毀其尸。副將蔡隆,游擊硃武,外委張瓊、戴鄰、陳謙俱罵賊死。事聞,優詔褒恤,贈芳度王爵,謚忠勇,如多羅郡王例,遣大臣致祭。隆、武、瓊、鄰、謙俱贈官有差。

梧兄子芳世,字周士。先於康熙元年齎梧疏入覲,留京師,授一等侍衛。及芳度遣藍齎疏告急,芳世自陳乞從大軍自廣東進援,上許之,以為福建隨征總兵官,降敕褒勉。芳世至廣東,會弟芳泰自漳州突圍出,芳世督兵赴援,距漳州僅二日,聞城陷,退屯惠州。芳度殉難,詔以芳世襲爵。十五年,叛將馬雄等誘芳世兄弟附三桂,不從,乘間脫走,至江西信豐,遣藍齎疏陳陷賊始末。上嘉之,加太子太保,命仍鎮守漳州。藍自參將擢海澄總兵,令馳赴康親王軍,俟漳、泉恢復,收集海澄公部下散失官兵,鎮守汛地。

十六年,芳世疏言:「臣叔梧遺骸遭賊殘毀,請與芳度一體議恤。臣叔樞罵賊而死,臣弟芳名、芳聲奮力守城,同日遇害,並乞賜恤。」詔贈梧太保,謚忠恪,樞贈按察使僉事,芳名、芳聲贈太常寺卿,各予廕;賜芳世蟒袍、弓矢、鞍馬,褒嘉甚至。

十七年,錦將劉國軒、吳淑犯海澄,芳世與總督郎廷相、副都統孟安等迭敗之觀音山、秬山頭、石瑪村等處。國軒退犯漳州,芳世率兵堵剿,殲賊甚眾。山寇蔡寅詐稱硃三太子,糾眾數萬,與錦通,犯漳州。芳世擊敗之於天寶山,斬其渠楊寧等。芳世疏言:「漳州亂後,臣叔梧、弟芳度舊部離散,臣漸次收集,得四千八百人,選補本標五營六百人,餘無額可補,乞汰留三千人,別立三營,視經制給餉。」部議從之。未幾,病卒,遺疏言:「閩省久困兵禍,漳州尤甚。原大師底定後,嚴飭有司輕徭薄賦,甦此殘黎。」並區畫海疆數事,復以子溥年才九歲,請以弟芳泰襲爵,詔贈少保,謚忠襄。

芳泰,字和士。少為諸生。佐芳度守漳州,突圍出乞援。城陷後,父母妻子皆遇害。至廣東,值尚之信叛,芳泰與芳世從巡撫楊熙力戰得出。尋授江南京口總兵。芳世卒,襲爵。屢出剿賊,復平和、漳平諸縣。總督姚啟聖疏言芳泰年少,不能轄標兵。下部議,令芳泰詣京師。芳泰疏請暫駐汀州,為兄芳度營葬。啟聖復言海澄公標下舊兵,聞芳泰在汀州,皆走依之,偽將吳淑兄弟以曾害芳度,不敢來降,請敕芳泰速離福建。十八年,芳泰至京師,上言:「臣久經行陣,不為幼弱。離漳已十月,不聞吳淑投誠。督臣無計辦賊,以臣藉口。臣當壯年,乞仍駐閩疆督剿,以報主恩。」上慰諭之。二十二年,許其回籍營葬。二十九年,卒,以子應纘為芳度後,襲爵。四十九年,應纘為芳泰請恤,贈太子少保。乾隆初,追謚襄愍。三十二年,高宗特詔以公爵世襲罔替。

應纘卒,謚溫簡。無子,以從子仕簡為後,襲爵。乾隆初,朝京師。高宗以其幼,令還里待命。十九年,授衢州總兵。二十四年,遷湖廣提督,歷廣東、福建陸路水師。疏發廈門商船陋規,上嘉之,諭:「汝知恩,朕亦知人。」漳、泉民流入臺灣,屢出劫掠,仕簡親渡海督兵捕治。再入覲,賜黃馬褂、雙眼花翎、黑狐端罩。病後偶躓,賜人參、高麗清心丸。淡水生番戕同知楊凱,復渡海督兵捕治,加太子太保。林爽文亂起,督兵討之,師久無功。總督常青、李侍堯先後劾仕簡貽誤,奪官,逮下刑部論斬,特宥之。尋赦歸,卒。

仕簡子秉淳前卒,以其孫嘉謨襲爵。秉淳初授藍翎侍衛,累遷至狼山總兵。嘉謨初授頭等侍衛,累遷至溫州總兵。

穆赫林,博爾濟吉特氏,滿洲正藍旗人。祖瑣諾木,為兀魯特貝勒。太祖時,從明安來歸。積戰閥,授二等總兵官。卒,順治間,追謚順良。再傳,子僧格襲世職,遇恩詔,累進三等伯。卒,穆赫林襲職。康熙五年,授正藍旗滿洲副都統,列議政大臣。

吳三桂反,十三年,偕都統拉哈達率兵駐防兗州,旋命移駐江寧。時耿精忠叛應三桂,大將軍康親王傑書、將軍貝子傅喇塔討之。穆赫林率所部喀喇沁、土默特兵赴浙江,與傅喇塔師會。十四年,從攻臺州,精忠將林沖糾眾萬餘,列十三寨拒戰。穆赫林督兵攻拔其寨,斬獲無算,復仙居。

師自黃巖進,精忠將曾養性偕叛將祖弘勛據溫州分水陸迎戰,穆赫林擊敗之上塘嶺,得戰艦三十餘。精忠將彭國明率眾五千瀕甌江列寨,穆赫林率兵至寶帶橋奮擊,斬級千餘,盡獲其槍械旗幟,遂薄江而陣,賊來犯,輒戰卻之。溫州繞城為壕屬甌江,為閘以蓄水,師爭閘,賊護甚力,久未能薄城。時康親王傑書駐金華,檄傅喇塔與穆赫林速攻城。穆赫林言必得大砲乃可克。十五年,上責王貝子等遷延,師無功。王因劾穆赫林與副都統吉勒塔布、提督段應舉等違令瞻顧狀,命事平議罪。八月,康親王自衢州攻克仙霞關,精忠降,徙養性、弘勛等至福州,檄穆赫林移師福建,駐守延平。

鄭錦使其將吳淑、吳潛自邵武來攻,穆赫林擊之浦塘隘口,陣斬其將楊大任等,乘勝復邵武、汀州二府及所屬縣。錦屢犯泉州,復侵潮州,穆赫林與副都統沃申、總兵馬三奇等分兵赴之,屢捷。十七年,錦犯海澄,穆赫林與海澄公黃芳世率兵迎擊於灣腰樹,戰失利,退保海澄。錦復糾眾環偪,據高阜瞰城中,砲石交下,穆赫林與應舉協力固守,糧盡,身負重創,未幾城陷,乃與應舉自經死。事平,吏議穆赫林征溫州師無功,守海澄聞援且至,不能突圍出,當奪官及世職,籍其家,上以穆赫林有戰功,貰籍沒,命其從子赫達色襲爵。世宗時,詔與應舉並入祀昭忠祠。

段應舉,漢軍鑲藍旗人。父思信,明廣寧千總。太祖取廣寧,來降,予世職備御。卒,應舉襲。從端重親王博洛討叛將姜瓖,攻汾州及太谷,克之。復從貝勒屯齊征湖南,屢有功。累擢鑲藍旗漢軍梅勒額真,進世職二等阿達哈哈番。偕鎮國將軍王國光赴廣東,駐防潮州。康熙三年,剿叛將蘇利於南塘鋪,賊敗遁,復碣石衛。敘功,進世職一等。尋署山東提督。十三年,率兵赴杭州剿御耿精忠,授福建提督。擊賊仙居、黃巖、太平、樂清,進圍溫州,皆捷。十五年,從康親王征福建,精忠降。時鄭錦據漳、泉、興化,與將軍拉哈達合兵進剿,復興化、泉州二城。復分兵定漳州及海澄等縣,應舉進駐海澄。十七年,劉國軒、吳淑等陷平和,穆赫林戰失利,詔責應舉不能平賊,調江寧提督楊捷代之,應舉仍以副都統從征。尋城陷,死之。

論曰:鄭氏為海疆患三十餘年,捷、正色捍衛艱難,內定泉、漳,外收金、廈;英、理遂佐施瑯越海恢疆,而理尤忠奮,稱虎將。方鄭氏亂時,有自海上降者,輒優以封爵,林興珠為建義侯,鄭鴻逵為奉化伯,周全斌為承恩伯,鄭纘緒為慕恩伯,梧最先降,授成功舊封。子芳世殉漳州,以忠延世。穆赫林等死海澄,孤城抗節,亦自有足稱者。悍寇死戰,御之艱,克之尤偉矣!


\end{pinyinscope}