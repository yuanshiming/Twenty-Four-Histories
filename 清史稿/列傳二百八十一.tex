\article{列傳二百八十一}

\begin{pinyinscope}
忠義八

姚懷祥全福舒恭受等韋逢甲長喜等麥廷章劉大忠等

韋印福錢金玉等龍汝元樂善魁霖等文豐

殷明恆高騰雲等高善繼駱佩德等林永升陳金揆等

李大本於光炘等黃祖蓮

姚懷祥,福建侯官人。嘉慶二十三年舉人。道光十五年,挑知縣,發浙江,權象山、龍游等縣。二十年,英吉利以欽差大臣林則徐在兩廣堅持鴉片之禁,耀兵寧波洋面,破定海,旋退出。二十一年二月,攻虎門,廣東水師提督關天培、湖南提督祥福;七月,攻廈門,總兵江繼蕓,游擊凌志;八月,復攻定海,總兵王錫朋、鄭國鴻、葛云飛;九月,攻鎮海,兩江總督裕謙,狼山鎮總兵謝朝恩;二十二年三月,攻慈溪,副將硃貴與子昭南;五月,攻吳淞,江南提督陳化成:均先後殉難,自有傳。懷祥於二十年適署定海篆,分募鄉勇,為死守計,總兵張朝發撤之。城陷南門,懷祥負傷,立城上呼兵,無應者,憤甚,投成仁塘死。

典史全福使酒仗氣,敵至,衣冠坐獄門。囚跳,嘆曰:「失城當死。況失囚耶?」敵入署,大呼殺賊,斃黑酋者一,叢刺死。翌年,再犯寧波、定海,則石浦同知舒恭受,游擊張玉衡、外委武英太同死難。都司李躍淵則隨總兵鄭國鴻戰曉峰嶺六晝夜,與把總胡大純、洪武琮,外委金釗同歿於陣。

是役也,慈溪大寶山死者,為即用知縣顏履敬,參將黃泰,守備田錫、陳芝蘭、徐宦、哈克里,千總阿本穰、魏啟明,把總林懷玉、盧炳、邸法德,外委張化鵬、馬龍圖、何海、毛玉貴、王保元、楊福增;死鎮海城者,為縣丞李向陽;戰金雞山死者,為都司孫汝鵬,守備李雲龍、王萬龍,千總陳慶三、陳守澍、周萬治,把總馬金龍、汪宗斌、解天培、金,外委林賡、吳定江;死招寶山者,為外委蔡步高。而山陰練勇袁樂忠以從間道導硃貴軍至長碕迎戰,為砲火所逼,從煙焰中躍起,投海死。

韋逢甲,山東齊河人。道光十六年進士,用知縣,發浙江,累權宣平、餘杭、浦江等縣。英吉利既再擾寧波洋面,將寇吳淞,先以弋船三十艘進攻乍浦。時逢甲以督鑄大砲,由鎮海赴乍浦設防,就權同知。四月,敵遽由東光山上陸,屯兵皆潰。逢甲帶鄉團御於西行汛,死之。

同死者,為駐防副都統長喜,前鋒協領英登布,佐領隆福,防禦貴順、額特赫,前鋒校佛印,驍騎校伊勒哈畚、根順、該杭阿及調浙助防之守備張淮泗,千總李廷貴,把總王榮、馬致榮、孫登霄,外委馬成功、硃朝貴。而伊勒哈畚尤慘,伏觀山射夷,殪甚眾,被執,磔死。子仁厚,襲職,殉粵寇。

麥廷章,廣東鶴山人。道光十二年,以外委隨剿連州瑤匪功,屢遷至游擊。林則徐查辦英吉利躉船鴉片,檄廷章率舟師駐九龍山巡防。英酋遞書辯論,開導不服,遽開砲,廷章以大砲應之,毀雙桅敵船。又潛約土密兵船助攻,復擊卻之。英人既陷浙定海,遂溯大洋至天津乞和,朝命直隸總督琦善馳粵與議,海防遽懈。二十年十二月,敵乘不備,突進占大角、沙角,廷章時佐提督關天培防守靖遠砲臺。明年二月,敵船擁入三門口,斷防禦椿練。南風作,復以大隊圍橫檔、永安,截我軍援道,進犯虎門。廷章奮勇御之,力竭死。

時同死者,為香山協副將劉大忠,游擊沈占鼇,守備洪達科等。參將周枋則以拒敵烏湧戰歿。三月,英人復由粵擾閩,攻廈門,犯內港,守備王世俊、蔣錫恩,千總張然迎擊之,均以力戰陣亡。

韋印福,江蘇上元人。由行伍隨剿滑縣匪,有膽略,嘗曰:「武官臨陣,死生度外事,畏死不作武官矣。」累擢千總,為兩江總督陶澍所賞,擢署金山營游擊。英吉利之窺吳淞也,提督陳化成守西砲臺,誓死戰,以印福忠勇,隸左右。二十二年五月,敵艦叢擊之,化成被傷,印福救護不及,歿於陣。

化成之歿,從殉者八十人,其尤烈者:千總錢金玉,臨危或勸避去,答曰:「金玉年十六即食國餉,今焉避?」遂及難;外委徐太華,善用砲,轉移如志,擊皆命中,被擊死;把總許攀桂,擁護化成,謂:「主將與某等同甘苦,公報國在今日,某等報公亦在今日!」眾心益固,卒飲劍死;把總龔增齡,迎戰,刃數人,敵人圍而擒之,釘手足於板,擲諸海;外委周林,率帳下巷戰,中槍,先化成死。

時督師兩江總督牛鑒,以砲毀演武,亟退去,之蘇州,又之江寧,敵遂由寶山徇上海,道以下官皆遁,典史劉慶恩投浦江死。內河不能深入艦隊,乃由福山口犯鎮江京口,副都統海齡戰不勝,自縊死,尋謚昭節。赴援游擊羅必魁、把總趙連璧,均死之。

駐防員弁同與難者,為馬甲長松,與子驍騎校祥雲;佐領景星、愛星布、心互明,防禦恆山、尚德、恆福、吉成,驍騎校伊克濟訥,文舉人噶喇,武舉人哈達海,筆帖式哈豐阿、恩喜,前鋒校松寶、文魁、阿勒金圖、喜興等。迫江寧欽差大臣耆英等奏定款局,而五口通商之約成。

龍汝元,順天宛平人。由行伍隨剿廣西會匪,以功累擢游擊,隸河南巡撫英桂軍營。咸豐八年,英吉利糾合法郎西、米利堅兩國,藉口換約,俄羅斯復陰助之,堅請在京師開議。議未定,艦隊集天津海口,朝命科爾沁親王僧格林沁辦理海防。汝元奉檄至,擢大沽協副將。九年五月,英、法兵船駛入內河,汝元手燃巨砲沉其船,旋中砲歿於陣,謚武愍。提督史榮椿同死,自有傳。

是役也,諸國受創甚。十年夏,艦隊復集天津大沽口,提督樂善奉命駐兵大沽,至則以關防交僧格林沁,令所部原留者聽,得千餘人,誓死守。六月,敵兵自北塘登岸,七月一日,自石縫砲臺擊敗之。相持一日,無後援。火藥局火起,兵多傷死。樂善知不可守,遂投河死。從死者副將、守備各一,失其名。樂善謚威毅。

時副將魁霖在通州巡防,檄至天津助戰,亡於陣,謚威肅。委翼長阿克東阿、侍衛扎精阿同死之。八月,敵遂北犯通州,圖占西倉,監督覺羅貴倫與同官玉潤衣冠對縊殉節。焚澱園,文豐外,員外郎泰清、苑丞泰衷全家自焚死。時文宗駐蹕熱河,命恭親王奕再議款局,而難始定。

文豐,董氏,內務府漢軍正黃旗人。內務府筆帖式,歷堂主事、員外郎、造辦處郎中,充杭州織造,授驍騎參領。道光二十一年,充粵海監督。二十三年,偕兩廣總督耆英等遵議英吉利五口通商章程十五條,下部議行。二十六年三月,授熱河副總管,充蘇州織造。差還,授堂郎中。咸豐四年,賞總管內務府大臣銜,歷正藍旗漢軍副都統、正藍旗護軍統領。七年二月,授總管內務府大臣,尋署正黃旗護軍統領。八年五月,管理圓明園事務,調正紅旗滿洲副都統,充崇文門副監督。又調正白旗滿洲副都統,署御藥房、太醫院事務。十年八月,命在圓明園照料一切事宜,是月英人闖入圓明園,文豐投水殉難。賜恤如例,贈太子少保銜,入祀京師昭忠祠。同治元年,追念忠節諸臣,以「文豐從容赴難,不愧完人」褒之,加恩予謚忠毅。

殷明恆,江西南昌人。由武童投效水師營,擢把總。光緒四年,赴閩,隸平海中營師船司砲。時佛郎西既並越南,將窺滇省,其酋領軍艦十四艘先犯福州,圖覆船政局。十年七月,在馬江發難,明恆陣亡。時毀兵船七,商船二,及艇哨各船俱燼,死者不可計。見奏報者,以參將高騰雲死最慘,五品軍功陳英戰最烈。船廠學生帶揚武艦葉琛,帶建勝艦林森林,均登了臺發砲,受彈,猶屹立指揮;充福星輪三副王漣受砲傷,猶槍斃敵兵多名,均以傷重陣亡。

是役也,戰鎮南關外,隸記名提督劉永福部下者,為武監生楊萼恩、哨弁何承文等;隸署提督蘇元春部下者,為總兵孫得勝,副將黃政德、邱福初、陳義新、劉德勝、張大壽、劉玉貴,參將胡延慶、王紹斌、蕭有明、黃世昌、石啟官、張興寬,游擊蕭寶臣、李純五、吳少懷,都司黃均、任有錫、李逢楨、吳述元、周同芳,守備黃效忠、楊承祿,千總蘇全璧、蔣全昌、李得勝,把總王有興、李明德、楊春林、徐國慶、葉亞吉、梁玉輝,外委曹正亮,六品軍功勞國豐,從九品黃汝霖等。

隸廣西巡撫潘鼎新部下,紙作社之役,為副將蘇玉標,都司陳福隆,把總張元鴻、顧玉芳;諒山之役,為提督劉思河,都司劉映谷、黃正寅、鄧晏林、杜光湔,守備羅雲高,千總俞諫臣、蔡得勝、孫其易,把總謝世和,六品軍功萬國發等。

隸福建布政使王德榜部下,戰豐穀等處,為總兵黃喜光,副將胡陽春、武鴻來,參將左廷秀、譚家璐、王得永、蔡玉堂、黃祖富、左占元,游擊陶得玉、聶章壽、王得才、柳臣玖,都司王天喜、陳永發、趙步雲、譚連勝、胡克勝、田玉貴,守備邱正亮、鄧青雲,千總謝廷蘭、張玉魁、楊大德、胡士英,把總蕭恩清、王成吉,外委劉云漢、謝薛昌,六品軍功黎占元、唐復興、譚以明等。

隸福建巡撫劉銘傳部下者,為總兵曾照禮,副將劉義高,千總殷有升,把總尤運農、祁文等。均分別上聞,贈恤有差,高州鎮總兵楊玉科,則以宿將有功,戰歿諒山,自有傳。

高善繼,字次浦,江西彭澤人。由附生舉同治元年孝廉方正,朝考用教職,署弋陽縣訓導。舉優行,皆寒畯士,積弊為清。尋調贛州府學教授,又調南安。光緒十四年,舉鄉薦,會試不第,謁李鴻章於天津,鴻章,其父執也,語不合,投通永鎮總兵吳育仁幕下。二十年,日本侵朝鮮,廷議主戰。六月,善繼佐營官駱佩德乘英國高升輪船運送軍實。駛至牙山口外,日本舉旗招撫,善繼不肯屈。管駕英人先逸去,善繼忿極,令懸紅旗示戰備,且進薄之。方與佩德指揮禦敵,忽船中魚雷,逾時,水勢注射益洶湧,眾強善繼及佩德亟下,善繼奮然曰:「吾輩自請殺敵,而臨難即避,縱歸,何面目見人?且吾世受國恩,今日之事,一死而已!」佩德曰:「如此,吾豈忍獨生?」高升船遂沉,善繼溺死,佩德從之。

時護行者為濟遠艦,亦為敵船在豐島襲擊,大副都司沈壽昌堅守砲位,竭力還攻。及中砲陣亡,則守備柯建章繼之;復陣亡,則黃承勛繼之。與軍功王錫三、管旗劉鵾同與於難,爭趨死地,奮不顧身,尤為當時所稱。廣乙快船管輪把總何汝賓,亦於是役中彈陣亡。

林永升,福建侯官人。入船政學堂肄業駕駛,派兵輪練習,周歷南北洋險要,以千總留閩,充船政學堂教習。復出洋留學,歸,晉守備,調直隸。從平朝鮮之亂,擢都司。赴德國接收代造經遠快船,保升游擊。光緒十五年,北洋海軍新設左翼左營副將,以永升署理。辦海軍出力,升用總兵。二十年八月,朝命海軍護送陸軍赴大東溝登岸援朝鮮,日本海軍來襲,我鐵艦十,當敵艦十有二。副將鄧世昌管帶致遠,都司陳金揆副之;參將黃建勛管帶超勇;參將林履中管帶揚威;經遠,則永升主之。永升夙與世昌等以忠義相激勵,既合諸艦,沖鋒轟擊,沉日艦三,卒以敵軍船快砲快為所勝,世昌戰歿。提督丁汝昌坐定遠督船,畏葸不知所為,又被傷,總兵劉步蟾代之。船陣失列,有跳而免者,永升仍指揮艦勇,冒死與戰,驟中敵彈,腦裂死。是役也,血戰逾三時,為各國海戰所僅見。

永升而外,金揆、建勛、履中及守備楊建洛、徐希顏,千總池兆濱、蔡馥,把總孫景仁、史壽箴、王宗墀、張炳福、易文經、王蘭芬,外委郭耀忠,五品軍功張金盛,六品軍功王錫山,均死之。世昌自有傳。

李大本,安徽六安州人。咸豐間投效江西軍營,以功累擢游擊,復投效直隸,充哨長,晉副將。光緒二十年,日本犯朝鮮,葉志超統軍往援,扼守公州,聶士成率五營駐成歡驛。敵軍來襲,大本與游擊王天培、王國祐同亡於陣。時武備學生於光炘、周憲章、李國華、辛得林並赳健士,伏要隘,狙擊敵前鋒,以接應不至,皆死焉。士成旋繞渡大同江至平壤與諸軍合,軍無鬥志,潰退相繼。獨左寶貴扼險惡戰,死最烈,自有傳。自是朝鮮無我駐軍,敵遂內犯。

黃祖蓮,安徽懷遠人。少有志節,嘗思立功異域。光緒初,入上海廣方言館,列優等,送美國游學。調天津水師駕駛學堂,旋派赴威遠兵輪練習。敘千總,署海軍中軍左營守備,充濟遠駕駛二副。海軍出力,以都司升用。中日釁啟,說丁汝昌以「嚴兵扼守海口,而以兵艦往搗之,攻其不備,否則載勁旅抵朝鮮東偏釜山鎮等處,深溝高壘,絕其歸路,分兵徇朝鮮諸郡邑,彼進則迎擊,彼退則尾追,又出偏師撓之。彼糧盡援竭,人無鬥志,必土崩瓦解,此俄羅斯破法蘭西之計也。」汝昌不從。及大東溝將戰,又說以「海戰宜乘上風,兵法貴爭先著。今西北風利,宜乘其兵輪未集,急擊不可失」。汝昌復不決,遂失利。

十二月,日人棄西路,南擾山東,祖蓮佐總兵劉步蟾等守威海。時官軍集關外,東路兵單,日軍由落風港登陸,攻陷榮成,全力萃威海。祖蓮揮將士開砲擊敵,敵少卻,既復大集,諸軍皆潰。二十一年正月,道員戴宗騫以力盡援絕投海,越數日,祖蓮與劉步蟾及總兵張文宣、楊用霖等俱死之。時汝昌書降於敵,且要敵軍不得殘餘軍,仰藥死。後以死綏上聞,旨不予恤。或謂汝昌實為所部脅降,憤而自盡,降書則死後出洋弁手也。

時旅順先陷,海軍掃地,黃海諸要隘皆失守,將士多死事,以奏報有缺,不得書。其見奏報者,三等侍衛永山,在鳳凰城戰歿;游擊李世鴻、副將李仁黨與提督楊壽山分守蓋平,禦敵大將乃木軍,戰最烈,同時以力盡陣亡。步蟾、宗騫自有傳。


\end{pinyinscope}