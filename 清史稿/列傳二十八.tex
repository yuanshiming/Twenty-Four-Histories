\article{列傳二十八}

\begin{pinyinscope}
科爾昆覺善甘都譚拜法譚席特庫藍拜鄂碩

伊拜弟庫爾闡阿哈尼堪星訥褚庫

科爾昆,阿顏覺羅氏,滿洲正藍旗人,世居瓦瑚木。祖翰,太祖時來歸。父碩色,官牛錄額真。

科爾昆初為貝勒阿巴泰護衛。事太宗,未冠,從伐察哈爾、朝鮮皆有功,令隸噶布什賢。崇德五年,從伐明,圍錦州。明兵數萬屯松山,科爾昆與牛錄額真索渾、巴牙喇甲喇章京瑚里布挑戰,敗之。明總督洪承疇、總兵祖大壽合兵十餘萬迎戰,科爾昆與索渾等陷陣,殪驍騎數十。六年,從英親王阿濟格伐明,駐杏山。明兵數千自寧遠至,科爾昆先眾馳擊,逐敵至連山,馬中流矢僕,科爾昆躍起殪敵騎,奪馬,乘以還。從英親王視壕,敵猝至,索渾陷圍中,科爾昆單騎翼以出。明兵數千自沙河所至,侵牧地,率噶布什賢兵擊破之。七年,從貝勒阿巴泰伐明,次豐潤,破明軍。次河西務,與巴牙喇甲喇章京鄂碩將數十騎偵敵,敵將將射,科爾昆先發,貫其臂,逐之,從馬上相搏,同墮水,敵將頎有力,握科爾昆胄,抑使入水,科爾昆捶其脛而踣,縶以歸。八年,授牛錄額真,兼兵部理事官。

順治元年,入關,擊破李自成,逐之至慶都。從固山額真葉臣攻太原,設伏殲敵。又從英親王阿濟格討自成湖廣,屢劘敵壘。敘功,授世職牛錄章京。三年,從肅親王豪格西討張獻忠,次漢中,擊破叛將賀珍。進擊獻忠,戰西充鳳凰山,大破之。獻忠既殪,復與輔國公嶽樂、尚書巴哈納等殲其餘黨。師還,累進二等阿達哈哈番。

六年,授噶布什賢章京。從鄭親王濟爾哈朗征湖廣,破湘潭,下寶慶、武岡,分兵趨沅州。與巴牙喇甲喇章京白爾赫圖以數十騎先驅,白爾赫圖陷陣失其馬,科爾昆奪敵馬掖之上,並馬突圍出。復縱騎奮擊破敵,進沅州,自道州出龍虎關。進世職一等,兼拖沙喇哈番。

九年,從敬謹親王尼堪徇衡州,明將李定國列象陣迎戰。科爾昆語巴牙喇甲喇章京西伯臣曰:「象不畏矢石,惟鼻脆,吾為君射之。」矢再發,貫象鼻,象奔,師從之,追奔數十里。敬謹親王聞勝,輕騎疾進,遇伏戰沒,科爾昆三入圍,求得王遺骸。師進次寶慶,明將孫可望以數萬人屯山巔,科爾昆督兵奮擊,可望潰走。貝勒屯齊遣學士碩岱與科爾昆還奏軍事,疏不言王戰沒。事聞,下議政王、貝勒、大臣會勘,科爾昆言不知疏云何,鄭親王呵之,科爾昆大言曰:「臣自髫齔侍太祖,弱冠事太宗,轉戰二十餘年。今奏事不明,死其分。奈何輕相侮?」上察其無罪,命寬之,但奪世職。十三年,擢巴牙喇纛章京。

十四年,從大將軍羅託下貴州。既定貴陽,令科爾昆以五千人取黃平,梅勒額真瑪爾賽副之。明將白文選據七星關,科爾昆令瑪爾賽將二千人出萬奇嶺大道,誘文選出戰,偽敗數十里,文選躡其後。科爾昆將三千人自間道疾趨出文選軍後,瑪爾賽還戰,文選敗走,克黃平。師還。

康熙元年,出定義州土寇。二年,從將軍穆里瑪、圖海下湖廣,討李自成餘黨李來亨等。圖海出歸州,穆里瑪出宜昌,科爾昆與噶布什賢噶喇依昂邦賴塔將五千人先驅,迭戰皆勝。次茅麓山,郝永忠以數萬人與來亨合,拒戰,科爾昆升山覘之,俟隙縱擊,破之。夜設伏,來亨以萬餘人襲我軍,伏發,敗走。明日復戰,來亨兵以大刀、藤牌護陣,我師張兩翼,科爾昆搗其中堅,陣潰。來亨倚譚家砦屯糧,計持久。科爾昆分兵破石坪,進圍砦。其將李嗣名出戰,中流矢死,科爾昆斷其後道,十餘日,其將高必玉等出降。科爾昆還與穆里瑪合軍,圖海亦至,令滿洲兵守隘,綠旗兵為長圍困之,來亨自經死,餘黨悉降。自成餘黨至是乃盡殄。師還,授世職拖沙喇哈番。

科爾昆從征伐,常為軍鋒。廉介,嫉惡遠勢。鰲拜專政,科爾昆獨不附。八年,卒。子巢可託,官至盛京刑部侍郎。

覺善,李佳氏,滿洲正紅旗人,世居薩爾滸。父通果,歸太祖,授牛錄額真。卒,覺善嗣。滅葉赫,克沈陽、遼陽,皆在行間,授世職備御,擢甲喇額真。

天聰三年,從太宗伐明,下永平四城,佐固山額真納穆泰等守灤州。明兵來攻,圍合,覺善勒兵出戰,奮逾塹,與甲喇額真阿爾津、牛錄額真庫爾纏趨擊,明兵潰奔,俄復集迫城下,覺善擊卻之。明兵發石壞城堞,覺善力禦,明兵不能登,凡五敗明兵。阿敏棄永平出關,納穆泰等亦突圍走,明兵阻道,力擊敗之。師還,與諸將待罪,上以覺善力守城,既出猶殺敵,釋其縛,進世職游擊。五年,上自將圍大凌河,明兵自錦州驟至,屯小凌河岸。上遣偏師渡河迎擊,兵不盈二百,覺善奮入陣,陷重圍,力戰得出。我兵別隊與明兵戰,有軍校為明兵所得,援之歸。明監軍道張春、總兵吳襄將步騎四萬距大凌河十五里駐軍,覺善從貝勒碩託以右翼兵直躪春壘,明兵敗挫,進世職二等甲喇章京。

崇德五年,授正紅旗梅勒額真,駐防義州。六年,從攻錦州,坐攻圍不力,罰鍰。上攻錦州,自將軍松山、杏山間,明兵薄我軍,謀奪砲,覺善以所部御之,明兵敗走。師圍松山,掘塹立營,明兵夜來侵,復戰卻之。八年,與梅勒額真譚布等駐錦州。又從鄭親王濟爾哈朗伐明,攻寧遠,明總兵吳三桂邀戰,擊卻之。進攻前屯衛,明兵出戰,蒙古兵稍卻,覺善督右翼兵奮擊,大破之,遂克其城。

順治元年,從入關,擊李自成,覺善創於砲,仍奮戰。二年,進世職一等。從順承郡王勒克德渾南征,次江寧。自成餘黨一隻虎等寇湖北,命移師討之。三年,師次石首,令與固山額真葉臣等率精銳徇荊州,破敵,分剿遠安、南漳、宜昌,悉定。師還,賜黃金十兩、白金三百兩。山東土寇擾恩、齊河、平陰諸縣,命覺善率兵討之,斬其渠掃地王,其眾萬餘殲焉。

五年,從大將軍譚泰討叛將金聲桓,七月,師薄南昌,至六年正月,克之。移師討叛將李成棟,攻信豐,覺善督所部樹雲梯先登,拔其城。師還,次贛州,復分兵戡定新喻、安福諸縣。敘功,並遇恩詔,世職累進二等阿思哈尼哈番,賜號「巴圖魯」。七年,從睿親王畋於中後所,坐私出射獵,降一等阿達哈哈番兼拖沙喇哈番。八年,上親政,復世職,擢都察院左都御史。尋命仍專領梅勒事,進世職三等精奇尼哈番。十五年,以老病乞罷。康熙三年,卒,謚敏勇。乾隆初,定封三等男。子吉勒塔布,自有傳。

甘都,先世自葉赫徙居巴林,因氏巴林。太祖時,率子弟來歸,授牛錄額真。旗制定,隸蒙古鑲藍旗。天聰元年,從伐明,次寧遠。明兵屯城北山岡,甘都手大纛直前,擊破之。三年,復從伐明,克大安口,復敗明兵於玉田。上自將取永平四城,克遵化,甘都與焉,即命佐察哈喇等駐守。四年,師棄遵化出邊,甘都殿,擊敗追兵。八年,予世職三等甲喇章京,授兵部參政。

崇德三年,考滿,進二等甲喇章京。尋更定部院官制,改兵部理事官。冬,從貝勒岳託等伐明,擊敗明太監高起潛,越明都,徇山東,克濟南。四年春,師還,道蠡縣,復克其城。以功進一等甲喇章京。五年,從索海等伐索倫部,索倫兵五百,據掛喇爾屯拒戰。甘都及理事官喀喀木督兵破柵入,斬級二百,俘二百三十人以歸。六年,從伐明,圍錦州,明總督洪承疇屯松山,屢以步騎出戰,甘都輒擊敗之。恭順王長史徐勝芳為敵困,甘都突入陣,援之出。七年,錦州下,以功加半個前程。

順治元年,從入關,破李自成。復從豫親王多鐸徇陜西,克潼關,取西安。二年五月,移師定江南,復與固山額真恩格圖、瑪喇布等下宜興、昆山諸縣,進三等梅勒章京。三年,從端重親王博洛略浙江,逐明將方國安至黃巖,國安入城守,圍合。甘都察國安勢蹙,撤圍縱使出,擊之,國安兵大潰,城遂拔。師入福建,甘都先眾克分水關,逐明唐王聿鍵至汀州,降漳州及漳平縣。五年,命署巴牙喇纛章京。從征南大將軍譚泰徇江西,討叛將金聲桓。七年三月,進二等阿思哈尼哈番。尋卒於軍。

譚拜,他塔喇氏,滿洲正白旗人。父阿敦,事太祖。天命元年正月朔旦,太祖始建號,諸貝勒大臣上表,阿敦與額爾德尼侍左右,受表,額爾德尼跪展讀如禮。阿敦尋領固山額真。太祖初徵明撫順,李永芳出降,阿敦引謁太祖。厥後事不著。

譚拜事太宗,天聰五年,以牛錄額真從伐明,圍大凌河城。祖大壽城守,遣百餘騎突圍出,譚拜與巴牙喇甲喇章京布顏圖追斬三十餘人,獲馬二十有四。八年,授世職牛錄章京,遷甲喇額真。九年,從伐察哈爾,收降人,遂伐明代州。譚拜與噶布什賢章京蘇爾德、安達立將四十人伏忻口,明邏卒三百經所伏地,斬馘過半。

崇德元年,從伐明,薄明都,北趨盧溝橋,再敗明兵。二年,與甲喇額真丹岱、薩蘇喀等將四十人略明邊,次清河,明兵七百拒守,擊之潰,搴纛二,並獲其馬。三年,從貝勒岳託伐明,入墻子嶺,攻豐潤,擊明兵,多墜壕死,復攻破明太監馮永盛諸軍。四年,從略錦州,率巴牙喇兵破明兵於城南,以功加半個前程。五年,授兵部參政。六年,兼任正白旗蒙古梅勒額真。七年冬,從伐明山東,克利津。八年春,出邊,以所部擊敗明總督趙光抃、範志完,總兵吳三桂、白廣恩諸軍。師還,賚白金,以功進三等甲喇章京。順治初,從入關。三年,擢兵部尚書。尋從肅親王豪格西討張獻忠,道陜西,與固山額真瑪喇希等擊敗叛將賀珍。下四川,屢破獻忠兵,復與固山額真李國翰渡涪江,敗獻忠將袁韜。四年,調吏部尚書。旋殲獻忠。入關後,世職四進至二等阿思哈尼哈番。七年三月,卒。子瑪爾賽,附鰲拜,語見鰲拜傳。孫多奇輝,降襲三等。乾隆初,定封三等男。

法譚,亦他塔喇氏,滿洲正紅旗人,世居瓦爾喀。初以巴牙喇壯達從滅葉赫,取遼陽,授牛錄額真。天聰三年,從攻寧遠,敗明兵於城北山岡。七年,取旅順。崇德三年,從伐明,敗密雲步卒,趣山東,克郯城。四年,從伐虎爾哈部,克雅屯薩城。六年,從圍錦州,御明總督洪承疇兵於松山,逐敵至塔山,擊之,多赴海死。八年,從攻寧遠,克前屯衛、中後所。順治元年,擢甲喇額真,兼工部理事官。從入關,破李自成。從順承郡王勒克德渾逐自成湖廣,其兄子錦犯荊州,法譚以精騎蹂之,斬獲甚眾,降自成弟孜及其將田見秀等。世職累進一等阿達哈哈番兼拖沙喇哈番。五年,授右翼步軍總尉。康熙元年,以病致仕。卒。

席特庫,佟佳氏,滿洲鑲藍旗人。父努顏,率族屬歸太祖,授牛錄額真。卒,席特庫嗣。事太宗,擢噶布什賢章京,率兵出錦州,得明諜,明兵自耀州至,席特庫赴援卻敵。從圍大凌河,裨將多貝陣沒,席特庫入陣,以其尸還。明兵自寧遠來援,與戰,一卒墜馬,席特庫領纛入陣援以出。

六年,與巴牙喇甲喇章京鰲拜等略明邊。八年,與噶布什賢章京圖魯什詗敵錦州、松山,皆有俘馘。察哈爾部人有散入席爾哈、席伯圖者,上命席特庫與蒙古布哈塔布囊等逐捕,斬七十餘級,得其戶口、牲畜。尋與卦爾察尼堪以二十騎往濟豐城偵明兵,至西拉木輪河,遇降明蒙古百人,席特庫設伏盡殲之。二人逸而奔,席特庫射殪其一,一為我國諜者所獲。上嘉席特庫以少勝多,賜甲胄旌之。

復從大貝勒代善略大同,敗明兵。自陽和轉戰,趣天城、左衛,徇宣府,與噶布什賢章京吳拜設伏破敵,進世職三等甲喇章京。九年,從貝勒多爾袞略山西,自平魯衛入寧武關,擊敗明兵。復與甲喇額真布顏等詗明兵錦州,與噶布什賢噶喇依昂邦勞薩等躡明兵冷口。

崇德三年,從貝勒岳託伐明,入墻子嶺。明兵自密雲突出,與勞薩分兵擊敗之,得巨砲二十。復擊敗明總督吳阿衡,攻真定,破太監高起潛兵,追至運糧河。敵夜犯本旗營,偕牛錄額真俄兌等力戰卻敵。六年,從鄭親王濟爾喀朗圍錦州,明兵自杏山赴援,鄭親王設伏,令席特庫以噶布什賢兵誘敵,伏發還擊,大破之。

明總督洪承疇出松山拒戰,席特庫與勞薩力戰破敵。師復圍錦州,承疇以十三萬人赴援,席特庫與噶布什賢八章京迎戰,擊敗其將王樸等。承疇退塔山,我師躡擊屢勝;復退杏山,席特庫縱橫馳突,追至筆架山,斬四百餘級,得馬二百四十有奇,獲纛六。明兵自松山、杏山二城潛遁,席特庫與噶布什賢章京布爾遜追擊,斬數百人,得其駝馬。七年,克松山,從豫郡王多鐸伐明,明兵自寧遠至,擊卻之。以功進世職二等甲喇章京。旋率兵自界嶺口毀邊墻入,敗山海關明兵。將攻薊州,明總兵白騰蛟、白廣恩合軍赴援,席特庫與噶布什賢章京瑚里布督兵奮擊,破陣斬將,得馬六百有奇。

順治元年,從入關,破李自成將唐通於一片石。固山額真葉臣徇山西,上命席特庫益其軍,至絳州,渡河,下汾州、平陽,降自成將康元勛,進攻黑龍關,降明將及其兵三千人。二年,移師略湖廣,逐自成至安陸,斬四百餘級,奪其戰艦,進世職一等。

三年,從豫親王討蘇尼特部騰機思,次土喇河,土謝圖等部以兵遮道,席特庫督兵追擊,斬獲無算,迭進一等阿思哈尼哈番。康熙五年,卒。

藍拜,亦佟佳氏,滿洲鑲藍旗人。父噶哈,太祖時來歸,授牛錄額真。藍拜事太宗,天聰八年,授巴牙喇甲喇章京。從固山額真阿山略錦州,又從噶布什賢噶喇依昂邦勞薩率兵迎察哈爾部眾之來歸者。尋擢梅勒額真。崇德四年,以不稱職解任。尋命偕承政薩穆什喀、索海征索倫部,仍領梅勒事,道虎爾哈部攻克雅克薩城,索倫部長博穆博果爾迎戰,與索海設伏夾擊,大破之,以功授世職牛錄章京,賜貂皮及所獲人戶。六年,從鄭親王濟爾哈朗圍錦州,明兵來奪砲,擊卻之,擢兵部參政。明總督洪承疇援錦州,藍拜與諸將進擊,破三營。敵乘雨侵右翼,藍拜及甲喇額真遜塔等與戰,敵敗走。尋調禮部。

順治元年,從入關,進世職三等甲喇章京。三年,復授梅勒額真。從大將軍孔有德征湖南,明桂王由榔據武岡,其總督何騰蛟遣其將王進才、黃朝宣、張先璧等拒戰。有德至長沙,擊走進才,令藍拜與梅勒額真卓羅追擊,殪其眾過半。下湘潭,朝宣屯燕子窩,藍拜與梅勒額真佟岱乘艦至瀘口,督兵破其營,尋從尚可喜援桂陽,還師攻道州。又與可喜合軍攻沅州,先璧自黔陽出,扼隘為五營。藍拜率先與戰,斬七千餘級,遂薄城,先璧又以三萬人拒戰,敗潰,遂克之,賜黃白金,進世職二等。六年,兼任禮部侍郎。八年,擢固山額真,兼工部尚書。九年,調刑部。尋命罷尚書,專領固山事。累進世職二等阿思哈尼哈番。

十年,命率兵鎮湖南。明將孫可望等出峽窺湖北,藍拜督兵防禦,敵不能犯。十三年,召還。上親勞以酒,諭曰:「爾等為朕宣力年久矣。今見爾等形貌★瘠,朕心惻然!」尋以老病乞罷,加太子太保。康熙四年,卒。

鄂碩,棟鄂氏,滿洲正白旗人。祖棆布,太祖時率四百人來歸,賜名魯克素,子錫罕,授世職備御。天聰初,從伐朝鮮,先驅戰沒。

鄂碩,錫罕子也。太宗以錫罕死事,進世職游擊,以鄂碩襲。八年,從貝勒多鐸伐明,攻前屯衛,斬邏卒。又從噶布什賢噶喇依昂邦勞薩率將士迎察哈爾部來歸者,授牛錄額真。九年,招察哈爾部伐明,自朔州至崞縣,斬邏卒。自平魯衛出邊,明兵邀戰,鄂碩與固山額真圖爾格擊卻之。進世職二等甲喇章京,擢巴牙喇甲喇章京。

崇德元年,與勞薩將百人偵明邊,至冷口,斬邏卒,得馬十五。二年,護甲喇額真丹岱等與土默特互市,赴歸化城,斬明邏卒。三年,從睿親王多爾袞伐明,自青山口入邊,擊敗明太監高起潛兵。四年,與噶布什賢章京沙爾虎達將土默特兵三百略寧遠,挑戰,明兵堅壁不出,得其樵採者以還。

五年,從圍錦州,以噶布什賢兵敗敵騎。明總督洪承疇赴援,上營松山、杏山間,命吳拜等以偏師營高橋東。鄂碩詗明兵自杏山潰出,告吳拜,吳拜未進擊,明兵復入城。上以鄂碩不親擊責之。六年,復圍錦州,分兵略寧遠,遇明兵六百騎,擊破之,得纛二、馬六十餘。七年,從伐明,自界嶺口入邊,敗明總督範志完軍於豐潤。明兵自密雲出劫我輜重,奮擊卻之,遂越明都趨山東。師出邊,明總兵吳三桂邀戰,復擊之潰,追斬數十級,得纛三、邏卒二十九、馬二百餘。

順治初,從入關,逐李自成至慶都,從豫親王多鐸討之。自成據潼關,倚山為寨,鄂碩與噶布什賢噶喇依昂邦努山攻拔之。二年,移師南征,鄂碩將噶布什賢兵先驅,至睢寧,敗明兵。從端重親王博洛下蘇州,擊明巡撫楊文驄舟師,得戰艦二十五。趨杭州,敗明魯王以海兵,獲總兵一。復與巴牙喇纛章京哈寧阿克湖州。世職累進二等阿思哈尼哈番。六年,擢鑲白旗滿洲梅勒額真。從鄭親王濟爾哈朗征湖廣。師還,賚白金三百。八年,授巴牙喇纛章京。十三年,擢內大臣。世職累進一等精奇尼哈番。十四年,以其女冊封皇貴妃,進三等伯。十四年,卒,贈三等侯,謚剛毅。子費揚古,自有傳。

羅碩,鄂碩兄也。初授刑部理事官。從入關,擢甲喇額真。順治六年,姜襄叛,命梅勒額真卦喇駐軍太原。瓖遣兵陷清源,與卦喇分道擊之,瓖兵棄城走,斬五千餘級。瓖遣兵犯太原,從端重親王博洛破賊壘,斬萬餘級。其徒圍絳州,擾浮山,迭戰勝之。八年,擢工部侍郎。進世職三等阿思哈尼哈番。九年,從征湖南,失利,奪官,降世職。尋授大理寺卿。十七年,以從女追冊端敬皇后,授一等阿思哈尼哈番。康熙四年,卒。

鄂爾多,羅碩孫。初授侍衛,累遷至侍郎,歷戶、刑二部。授內務府總管,擢尚書,歷兵、戶、吏三部。卒,謚敏恪。

伊拜,赫舍里氏,世居齋穀。父拜思哈,歸太祖,授牛錄額真。旗制定,隸滿洲正藍旗。卒,伊拜與其兄宜巴里、弟庫爾闡分轄所屬,為牛錄額真。太宗即位,察哈爾部貝勒圖爾濟來歸,命伊拜迎犒。天聰八年,上自將伐明,命伊拜徵科爾沁部兵,予世職半個前程。九年,遷正白旗蒙古固山額真。

崇德元年,從伐明,入長城,攻克昌平等州縣,俘獲甚眾。三年九月,從伐明,入青山口,薄明都,徇山東。五年,從伐明,圍錦州。明兵自杏山、松山赴援,城兵出戰,伊拜屢擊敗之。六年,復圍錦州,破明兵,進世職牛錄章京。洪承疇赴援,上自將擊之,命諸將分屯要隘,要明兵,伊拜與梅勒額真譚拜等依杏山而營。明兵敗走,伊拜逐擊至塔山,明兵多赴水死。七年,遂破承疇,下錦州,命伊拜戍杏山。八年,復命與輔國公篇古戍錦州。是時軍紀嚴,將士有過,輒論罰,伊拜屢坐罰鍰、罰馬。

順治元年,調正藍旗蒙古固山額真。從入關,擊李自成。尋與固山額真葉臣等徇山西,克太原,撫定旁近州縣。師還,賚白金三百。二年,從英親王阿濟格徇陜西,逐自成至武昌,屢擊破賊壘。三年,進三等阿達哈哈番。五年,從鄭親王濟爾哈朗徇湖南,時衡州、寶慶諸府尚為明守。六年,師克湘潭,伊拜與固山額真佟圖賴等分兵向衡州,未至三十里,明兵千餘人據橋立寨,伊拜與侍郎碩詹擊之潰。薄城,戰屢勝,斬明將陶養用,遂克衡州。別軍略寶慶及辰、沅、靖、武岡諸州,皆定。師還,賚白金三百。尋請老,授議政大臣。累進一等阿思哈尼哈番。十五年,卒,贈太子太保,謚勤直。第三子費揚武,襲世職。

庫爾闡,天聰間,以牛錄額真從伐黑龍江,有功,予世職半個前程。崇德三年,授都察院理事官,兼甲喇額真。五年,從伐索倫部,與其部長博穆博果爾力戰,卻之。從睿親王多爾袞圍錦州,攻松山,戰有功。六年,擢都察院參政。復從圍錦州,明兵自松山來,將奪軍中砲,庫爾闡擊卻之。率師依山為寨,明兵復來攻,勢甚猛,工部承政薩穆什喀欲遣兵助戰,庫爾闡辭焉,獨以所部迎戰,斬四十一級,得雲梯、槍砲、甲楯、旗幟,進世職牛錄章京。八年,遷正藍旗蒙古梅勒額真。

順治初,從入關,逐李自成至慶都,加半個前程。旋從豫親王多鐸破自成潼關,累進二等甲喇章京。四年,命帥師駐防濟南。淄川寇發,庫爾闡遣兵討之。部議責庫爾闡不親赴,當罰鍰,盡削其官職,上但命倍其罰。五年,遷都察院承政,尋仍改參政。六年,從譚泰討金聲桓江西,卒於軍,進一等阿達哈哈番。

阿哈尼堪,富察氏,滿洲鑲黃旗人,世居葉赫。天命時,曾祖椿布倫,偕兄楚隆阿、弟昂古裡來歸。阿哈尼堪初授牛錄額真。天聰九年,同蒙古兩黃旗將領布哈、阿濟拜略明寧遠,明兵千人追至,還擊,敗之。崇德二年,從征朝鮮,取江華島。五年,從承政薩穆什喀、索海伐虎爾哈部,克雅克薩城。博穆博果爾以兩烏喇兵六千來襲正藍旗後隊,索海設伏擊之,阿哈尼堪與焉。又攻掛喇爾,先入屯,授世職牛錄章京。擢禮部參政。六年,從伐明,圍錦州,擊敗松山援兵。又與固山額真宗室拜音圖敗明總督洪承疇兵。松山守將夜襲我軍,又遣步兵犯正黃旗蒙古汛地,阿哈尼堪擊卻之。擢鑲黃旗梅勒額真。

順治元年,從入關,擊李自成。世祖將遷都燕京,命內大臣何洛會鎮盛京,阿哈尼堪與梅勒額真碩詹將左右翼為之佐。尋命偕固山額真阿山等率兵之蒲州,助剿流寇。二年,進世職三等甲喇章京。大將軍豫親王多鐸南征,命阿哈尼堪會師,自河南下江南攻揚州,明兵來援,率甲喇額真班代等連戰皆捷,與固山額真瑪喇希克常熟。三年,從豫親王北討蒙古蘇尼特部,騰機思遁走,追擊,斬百餘級,俘獲無算,進世職一等。四年,擢兵部尚書。

六年,鄭親王濟爾哈朗師略湖廣,阿哈尼堪與固山額真劉之源別將兵趨寶慶,明將王進才、馬進忠城守。師夜薄城,平旦,明兵出戰,急擊殲之,遂克寶慶。明將馬有志等九營屯南山,阿哈尼堪乘勝奮進,陣斬有志等。師徇洪江,又破袁宗第十營,克沅州。師復進,留阿哈尼堪駐守。明將王強等來攻,阿哈尼堪遣署巴牙喇纛章京都爾德等迎擊,戰沅水上,大破之,斬裨將三、兵七百餘。七年,師還,賜白金三百。調禮部尚書,加世職拖沙喇哈番。

睿親王遣迎朝鮮王弟,阿哈尼堪啟巽親王滿達海等,以甲喇額真恩德代行。事覺,下王大臣會勘,論死,得旨,奪世職,罰鍰以贖。尋復世職,累進一等阿思哈尼哈番。八年,卒。

星訥,覺爾察氏,滿洲正白旗人。初事太祖,授二等侍衛,兼牛錄額真。從伐明,次塔山北,遇蒙古兵四百,射殺其渠。事太宗,伐察哈爾,以二十人偵敵張家口,遇明兵,御之四晝夜,俟貝勒阿濟格軍至,益兵二百擊破之。察哈爾部多爾濟蘇爾海倚山立寨,列火器拒守,星訥率巴牙喇兵先登破敵。天聰八年,復從上伐察哈爾,星訥佐額駙布顏代率蒙古兵進哈麻爾嶺,招其部俄爾塞圖等來降。移師伐明,與席特庫等略大同。論功,予世職半個前程,授刑部參政。

崇德三年,與承政葉克舒伐黑龍江,師有功,其兄辛泰、弟西爾圖戰沒,當得世職,合為三等甲喇章京。尋坐事降理事官。四年,授巴牙喇甲喇章京,兼議政大臣。尋遷梅勒額真。六年,授工部參政。八年,擢承政。

順治元年,從入關,改尚書,進世職一等。三年,從討張獻忠,師還,加太子少保。六年,從討姜瓖,攻大同。瓖以精銳出戰,填塹毀垣,星訥督將士持短兵,力戰卻之。瓖背城為陣,星訥督將士直壓其壘,師乘之,殲其精銳略盡,進世職二等阿思哈尼哈番。

八年,英親王阿濟格得罪,星訥故為王屬,坐奪官,削世職,籍家產之半。尋復授工部尚書、議政大臣。十年,以老致仕。十四年,星訥自訟軍功,復世職一等阿達哈哈番兼拖沙喇哈番。康熙十三年,卒,謚敏襄。

褚庫,薩爾圖氏,滿洲鑲黃旗人,先世居札魯特。祖柏德,遷居葉赫,來歸。天聰四年,師圍大凌河,褚庫年十七,從軍。明軍中蒙古將徹濟格突陣,褚庫迎擊,生獲以歸。復伐明,攻萬全左衛,褚庫先登,頸被創,猶力戰破其城。論功,授世職備御,賜號「巴圖魯」。授牛錄額真,兼甲喇額真。崇德三年,授吏部理事官。

順治元年,入關,從英親王阿濟格討李自成,略湖廣,自成將吳伯益以三千人拒戰,褚庫擊之,敗走。三年,從肅親王豪格討張獻忠,略陜西,與尚書星訥擊獻忠將高汝礪等,遂下四川,屢敗獻忠兵。六年,從討姜瓖,圍大同,敗襄將楊振威。師還,坐值宿失印鑰,解理事官。九年,從固山額真噶達渾征鄂爾多斯部,與其部長多爾濟戰賀蘭山,俘獲甚眾。世職累進二等阿達哈哈番。

十三年,鄭成功攻福州,時鄭親王世子濟度率師次漳州,遣梅勒額真阿克善與褚庫別將兵赴援。成功以戰艦二百自烏龍江來犯,褚庫督兵迎戰,逐至大江口,得舟十二。成功又以千餘人屯江岸,褚庫督兵奮擊,斬二百餘級。康熙二年,擢正紅旗蒙古副都統,進世職一等。七年,以老乞休。十四年,卒,謚襄壯。

論曰:科爾昆、覺善、甘都逮事太祖,譚拜以下諸將,則太宗所驅策,入關後四征不庭,成一統之業,皆與有功焉。科爾昆尤忠直,與席特庫、褚庫並以驍武搴旗陷陣。干城腹心,由此其選矣。


\end{pinyinscope}