\article{列傳四十三}

\begin{pinyinscope}
蔡毓榮哈占杭愛鄂善華善董衛國佟國正周有德張德地

伊闢王繼文

蔡毓榮,字仁庵,漢軍正白旗人。父士英,初籍錦州。從祖大壽來降,授世職牛錄章京。從轉戰有功。順治間,累遷至右副都御使。出為江西巡撫,疏陳兵後荒蕪,請除荒田賦額十萬八千五百四十頃有奇;又以瑞、袁二府科糧偏重,疏請蠲瑞屬浮糧九萬九千餘石,定袁屬賦額自一斗六升七八合減至九升三合:皆得請。又疏論銅塘封禁山不宜開採,咸為民所頌。尋改漕運總督,加兵部尚書,以疾告歸。十三年,卒,謚襄敏。

毓榮,其次子也。初授佐領,兼刑部郎中。尋授御史,兼參領,遷秘書院學士。康熙初,授侍郎,歷刑、吏二部。九年,授四川湖廣總督,駐荊州。累疏言:「四川民少田荒,請廣招開墾。招民三百戶,予議敘,墾田五年,起科」;「四川沖要營員用沿邊例題補」;「移駐官兵子弟得入籍應試」。並下部議行。

十二年,吳三桂反,毓榮遣沅州總兵崔世祿率兵入貴州,夷陵總兵徐治都、永州總兵李芝蘭繼進,上命速遣提督桑額守沅州。尋授順承郡王勒爾錦為大將軍,率八旗兵討三桂,駐荊州,諭毓榮督餉。十三年,分設四川總督,命毓榮專督湖廣,以招民墾荒功,加兵部尚書。三桂破沅州,世祿降。常德、澧州、長沙、岳州相繼陷。部議毓榮當奪官,命留任。尋居父喪,命在任守制,督綠旗兵進剿。毓榮令副將胡士英等分防江口。叛將楊來嘉據南漳,屢出掠,令襄陽總兵劉成龍御之,戰屢勝。廣西提督馬雄降三桂,騰書兩廣總督金光祖,言毓榮將率綠營兵赴岳州降三桂。光祖密使告毓榮,毓榮以聞,請解任,命殫心供職,毋以反間引嫌。

十四年,勒爾錦請增綠旗兵援、剿二營,領以兩副將,命毓榮統轄。十七年,毓榮督造戰艦成,率綠旗兵五千,從大將軍貝勒尚善進攻岳州,與討逆將軍鄂納等以舟八百餘入洞庭湖,擊三桂兵,大敗之,發砲沈其舟,殲寇甚眾。遣將艤君山,載土伐木塞諸港。分兵屯三眼橋、七里山,絕寇轉糧道。寇犯我糧艘,夾擊,復大敗之,斬級千餘。會三桂死,其孫世璠以喪還。師克岳州,進定長沙、衡州。十八年,疏言:「湖南境惟辰州尚為三桂守。楓木嶺、神龍岡兩道皆險隘。我師疲頓,當小休。俟糧草克繼,會師進攻。」上命給事中摩羅、郎中伊爾格圖傳諭曰:「賊敗遁負險,宜用綠旗步兵。毓榮所屬官兵強壯,不難攻取險隘,剿除餘寇。其具方略以聞。」毓榮疏請專責一人,總統諸路綠旗兵水陸並進,上即授毓榮綏遠將軍,賜敕,總統綠旗兵,總督董衛國、周有德、提督趙賴等並受節制。十九年,督兵分道出楓木嶺、辰龍關,水師並進,克辰州,再進克沅州,並復瀘溪、漵浦、麻陽諸縣。

大將軍貝子彰泰與會師,自沅州入貴州境。彰泰疏言綠旗兵已與滿洲兵會,若各自調遣,慮未能合力奏功。上命毓榮軍機關白大將軍。尋與衛國督兵克鎮遠、思南。世璠將夏國相等以二萬人屯平遠西南山,分兵據江西坡,坡天險,國相為象陣。我師迫險攻象陣,不能克,毓榮以紅旗督戰,眾奔不可止,師敗績。越二日復戰,鼓眾奮進,國相棄險走,遂克貴陽。二十年,從彰泰下雲南,次曲靖。會師進薄會城,屯歸化寺,奪重關及太平橋。世璠將餘從龍等出降,詗知其虛實。趙良棟師至,趣進攻,毓榮軍大東門。世璠自殺,城下。雲南平。毓榮還任湖廣總督。

二十一年,調雲貴總督。累疏區畫善後諸事:「一曰蠲荒賦。雲南陷寇八載,按畝加糧。驅之鋒鏑,地曠丁稀,無徵地丁。額賦應予蠲除,招徠開墾。二曰制土夷。前此土目世職,不過宣慰,三桂濫加至將軍、總兵。初投誠,權用偽銜給劄,今當改給土職。舊為三桂奪職者,察明予襲。三曰靖逋逃。三桂舊部奉裁,徵兵散失。八旗僕從,兔脫鼠竄。宜厚自首賞,重懲窩隱。所獲逃人,量從末減,庶聞風自歸。四曰理財源。雲南賦稅不足供兵食。地產五金,令民開採,官總其稅。省會及祿豐、蒙自、大理設爐鑄錢。故明沐氏莊田及入官叛產,均令變價,以裕錢本。田仍如例納賦,兵弁餘丁,墾荒起科,編入里甲,俾賦有餘而餉可節。五曰酌安插。逆屬嘗隨伍,當遣發極邊。若僅受偽銜,並未助逆,宜免遷徙。六曰收軍仗。私造軍器,應坐謀叛論罪。土司藏刀槍,民以鉛硝、硫黃貿易,皆嚴禁。七曰勸捐輸。雲南民鮮蓋藏,偶有災祲,無從告糴。請暫開捐監事例。八曰弭野盜。魯魁在萬山中,初為新習阿蒙土人所據,嘯聚為盜。內通新平、開化、元江、易門,外接車里、孟艮、鎮元、猛緬。三桂授以偽職,今雖改授土司。仍宜厚集土練,分駐隘口,防侵軼為患。九曰敦實政。兵後整理撫綏,其要在墾荒蕪,廣樹蓄,裕積貯,興教化,嚴保甲,通商賈,崇節儉,蠲雜派,恤無告,止濫差。州縣吏即以此十事為殿最。十曰舉廢墜。各府州縣學宮,自三桂煽亂,悉皆頹壞。今宜倡率修復。通省稅糧,既有成額,宜均本折定,留運驛站,酌加工食,俾民間永無派累。」疏入,廷臣議行。別疏言:「督標舊額兵四千,請增千為五營。吳三桂設十鎮,今改為六。在迤西者:曰鶴麗、曰永順、曰楚姚蒙景,在迤東者:曰開化、曰臨元澂江、曰曲尋武霑。」「中甸舊轄麗江土府,三桂割畀蒙、番互市。今互市已停,蒙、番所設喇嘛營官未撤,宜令土知府木堯仍歸其地。」

初,師自貴州下雲南,毓榮劾董衛國不聽調度,上命俟事平再議。二十二年,部議衛國未嘗違誤,且有復鎮遠功,請免議,上責毓榮妒功誣奏,下部議,削五級。二十五年,授總督倉場侍郎,改兵部。領侍衛內大臣佟國維等疏言侍衛納爾泰自陳前使雲南,毓榮令其子琳餽以銀九百;內務府又發毓榮入雲南以三桂女孫為妾,並徇縱逆黨狀:下刑部,鞫實,擬斬,籍沒,命免死,與琳並戍黑龍江。赦還。三十八年,卒。

哈占,伊爾根覺羅氏,滿洲正藍旗人。自官學生授鴻臚寺贊禮郎,累遷兵部督捕理事官。康熙八年,授秘書院學士。十一年,擢兵部侍郎。

十二年,授陜西總督。甫到官,吳三桂反,四川提督鄭蛟麟、總兵吳之茂等叛應之,與三桂將王屏籓謀寇陜西。上授都統赫業安西將軍,會西安將軍瓦爾喀討之,命哈占與巡撫杭愛督餉,並敕與提督張勇、王輔臣修邊備,輯軍民。十三年,復命尚書莫洛經略陜西,敕凡事諮哈占乃行。哈占以漢中、廣元山逕險峻,疏請造船略陽速糧運。尋又命貝勒董額為定西大將軍,護諸將出秦州,徇四川。寇劫略陽糧艘,上命四川總督周有德督川境轉餉。哈占疏請令山西協助,上以山西道遠多勞費,發帑十五萬,使在西安採運;並諭宜稍增其直,俾民樂輸送。會輔臣叛,莫洛遇害。董額以餉不繼,自漢中引還西安。

十四年,詔哈占分兵防蘭州,哈占疏言西安兵少不宜分遣。上命云貴總督鄂善率師駐興安、漢中,既又命守延安,哈占迭奏請留西安不遣。時輔臣據平涼,同州游擊李師膺叛,戕韓城知縣翟世琪,脅神道嶺營卒,合蒲城土寇陷延安。固原道陳彭、定邊副將硃龍皆以城叛。輔臣分兵四出,陷旁近諸州縣,遂破蘭州,巡撫華善走涼州。遣將逐賊邠州、淳化、三水、長武、漢陰、石泉、甘泉、寶雞諸處,戰輒勝。董額師克秦州,總兵王進寶亦復蘭州。定邊、延安皆下。上趣董額督兵合攻平涼。哈占聞興安游擊王可成叛,移潼關綠旗兵守商州,移西安滿洲兵守潼關。俄聞興安叛兵已破商州舊縣關,逼西安,疏請敕董額分兵赴援。上責哈占曰:「輔臣初叛,朕以蘭州近邊要地,令哈占發兵鎮守。哈占以西安兵少不遣,蘭州遂陷。又以延安居要沖,命鄂善屯守,哈占留之西安,延安復陷。哈占但知有西安,重兵自衛,貽誤非小!」別敕董額急攻平涼,仍遣將軍吳丹率師自太原移駐潼關,員外郎拉篤祜率榆林蒙古兵益西安。十五年,大學士圖海代董額為大將軍,圍平涼,輔臣降。哈占疏請安輯降眾,設置官吏。事皆下部議行。

十九年,將軍趙良棟克成都,王進寶克保寧,郡縣以次底定。哈占疏言軍餉自西安運保寧,應令四川接運。上以四川初定,未能任轉餉,命自略陽水道運敘州。尋敕哈占率師赴保寧,規復云南。哈占復疏請命四川督餉,戶部侍郎趙璟、金鼐疏言陜西轉餉入四川,四川吏不之恤,道遠民滋困。

尚書宋德宜言陜西、四川宜以一總督董理,庶兩省民勞逸得平,乃改設川陜總督,以命哈占。哈占師次保寧,時叛將譚弘、彭時亨四出劫掠為民害,上命速剿定,進攻雲南。哈占遣總兵高孟擊時亨,敗賊南溪羅石橋,復營山、渠二縣。二十年,鎮南將軍噶爾漢收忠、萬、開、建始、雲陽、梁山諸州縣。弘走死。孟逐時亨,亦復廣安、達、大竹、東鄉諸州縣。時亨勢蹙,降。敕哈占率師赴敘州,會建昌、永寧兩路兵進徵。哈占師發永寧,追擊三桂將馬進寶,入貴州。次畢節,進寶降。復進次威寧。大將軍貝子彰泰疏言雲南已合圍,師足用,兵多糧少,宜令哈占還四川。哈占復進次曲靖,聞命引還。尋以破時亨功,加兵部尚書銜。弘將牟一乾、一舉詣遵義降,分駐巴縣、涪州。哈占疏請移陜西,懦者歸農,強者入伍,上從之。二十二年,授兵部尚書。二十四年,調禮部。以疾乞休,上疏自述在軍時積勞成病。上以哈占未嘗立功,斥其妄,命仍殫力供職自贖。二十五年,卒。

杭愛,章佳氏,滿洲鑲白旗人。父古爾嘉琿,順治初為國子監祭酒。杭愛初授筆帖式,累遷吏部郎中。康熙十一年,超授山西布政使。諭曰:「朕知汝才能,外省事重,籓司職掌最要。其克盡忠誠,毋負簡任!」十二年,擢陜西巡撫。軍興,命督餉。十九年,調四川。叛將譚弘據萬縣為亂,命杭愛慰撫夔州諸路。二十年,建昌土司安泰寧謀亂,敕與將軍王進寶招之來降。哈占師進次永寧,命杭愛督趣輸運。自三桂亂,四川悉陷,民多流亡,兵占耕民田不納賦。杭愛疏請清釐,又乞蠲羅森妄報墾荒升科田四百餘畝,上特允之。二十二年,卒,謚勤襄。

鄂善,納喇氏,滿洲鑲黃旗人。初自侍衛授秘書院學士,遷副都御史。康熙九年,授陜西巡撫。十一年,擢山西陜西總督,尋改專督陜西。十二年,調雲南,以哈占代。三桂反,詔鄂善留湖廣。十三年,改兼督云、貴,命從師進徵。三桂陷湖南郡縣,吏議鐫五級,命留任。王輔臣叛,命與副都統穆舒渾率師自襄陽移守興安、漢中。十四年,次西安,哈占疏留助守。上復命移守榆林、延安,哈占再疏留不遣。及畢力克圖擊輔臣,復延安,鄂善乃遵上指移駐,招撫流民,分守棧道,寇來犯,擊之退。授甘肅巡撫。十七年,坐失察布政使伊圖蝕帑、清水知縣佟國佐苛斂,部議當奪官,命留任。十八年,以計典罷。尋卒。

華善,亦伊爾根覺羅氏,滿洲鑲黃旗人。初授筆帖式,累遷刑部郎中。順治十三年,從大將軍伊爾德克舟山,累進世職拜他喇布勒哈番兼拖沙喇哈番。康熙初,累遷弘文院學士。九年,授甘肅巡撫,疏請免逃荒額賦。西和、禮縣大疫,華善發帑治賑,並以春耕期迫,令市耕牛、具籽種,事竟乃疏聞,部議以違例當責償,上命寬之。輔臣反,攻蘭州,游擊董正己叛應之,布政使成額降寇,華善與按察使伊圖走永昌,疏請假提督張勇便宜討輔臣,與勇及王進寶、陳福、孫思克分道進兵,規復蘭州。華善與勇督兵赴臨洮,遣將收河、洮二州,復督兵攻鞏昌,克之,會進寶亦克蘭州,諭嘉勞。十五年,疏請免臨洮、鞏昌二府逋賦。尋卒於官。

董衛國,漢軍正白旗人。初授佐領,累官秘書院學士。順治十八年,擢山西巡撫。康熙四年,加工部尚書銜。十三年,改兵部尚書銜。

吳三桂反,陷長沙,衛國疏請發兵備袁州、吉安,上命副都統根特自兗州移兵赴援。耿精忠亦反,侵寧都、廣昌、南豐諸府縣,饒州參將程鳳、廣信副將柯升叛應之,構土寇破都昌,窺南康。衛國密疏聞,上命定南將軍希爾根會衛國剿御。精忠兵逼袁州,山民棚居與相結,謂之「棚寇」。衛國請設袁臨總兵,薦副將趙應奎有膽略堪任,上從之。南瑞總兵楊富謀叛,衛國廉得實,寘之法,並殲其黨,上嘉之。尋改設江西總督,以命衛國。精忠兵及棚寇分犯新昌、上高,衛國遣諸將佟國棟、趙登舉、張射光赴援,大破賊,斬其渠左宗榜。十四年,與希爾根等招降泰和、龍泉、永新、廬陵諸縣。參贊桑額自上高克新昌,被檄引去;寇抵隙復入,城並陷,遣其徒遏廣信糧道。衛國請督兵進剿,大將軍簡親王喇布駐師南昌,疏留之。十五年,遣諸將吳友明逐寇瑞州,復上高、新昌。復遣援靖安,諸將許盛、楊以松克泰和、定南。十六年,以土寇楊玉泰竊據宜黃、樂安、崇仁山谷中,發兵討之。崇仁寇蔡仕伯、宜黃寇沈鳳祥等出降。破賊於大嶺,克樂安,玉泰亦降。

湖南平江及銅鼓營寇起,衛國留提督趙賴守樂安,移兵入湖南,簡親王檄發衛國標下兵悉赴樂安。衛國疏聞,且言省城駐滿洲兵不過二百,慮不足守御,乞賜罷斥,上嚴旨詰簡親王,並諭此後徵發當諮衛國。衛國遣兵徇建昌,定瀘溪,自將出芳塘,別遣諸將出黃岡口,遂克銅鼓營。平江乃定。

未幾,精忠將韓大任侵寧都,時簡親王出駐吉安,衛國請與會師合剿,上命綠營兵聽便宜調遣。十七年,巡撫佟國正遣將破大任。精忠將郭應輔等分屯萬安、泰和諸縣,衛國督兵進擊,斬四萬餘,降者亦四萬六千有奇。

吳三桂犯永興,薄吉安,上命衛國守銅鼓營。三桂既死,其將據岳州、長沙,師圍之未下。衛國請自銅鼓營督兵援剿,上嘉許,並授以方略。未幾,岳州、長沙皆下。十八年,命會大將軍安親王岳樂謀進取,遂合軍出衡州、寶慶,破賊紫陽河、雙井鋪,克武岡。給事中李宗孔劾衛國為總督不治事,失民心,廷議奪官,上寬之。十九年,破鴨婆、黃茅諸隘,攻靖州。與都統穆占會師逐吳世璠將吳應麟等,克沅州。進薄鎮遠,力戰奪石港口,抵大巖門。世璠將張足法悉眾迎戰,衛國親督兵奮擊,大破之。足法夜遁,逐之至油閘關而還,遂克鎮遠。貴州既定,大將軍貝子彰泰下雲南,留衛國守貴陽。二十年,雲南平,命還任。

二十一年,調湖廣總督。衛國初自湖南入貴州,蔡毓榮以不聽調度論劾。事平,下廷議,上右衛國譴毓榮。御史蔣伊又論衛國縱兵俘掠,江西總督於成龍為疏辯。衛國朝京師,瀕行,諭曰:「爾在外二十餘年,民情宜悉知。前此方用兵,不免擾民。今天下承平,當思休養,興革利病,務在實行。朕知爾有勞,毋畏人言,勉圖後效。」月餘,卒,賜祭葬。

佟國正,佟佳氏,漢軍正黃旗人。自拔貢生授江南無為知州,累遷安徽按察使。康熙十三年,遷江西布政使。衛國改總督,白色純代為巡撫。十四年,色純卒,大將軍安親王岳樂奏國正得民心,擢巡撫。十五年,命出駐贛州。叛將嚴自明等偪南康,國正遣許盛等赴援,破賊庫鎮鋪,破其壘十七,逐北七十餘里。自明等走南安,又遣別將黃士標、王割耳等犯信豐,國正遣楊以松及諸將周球等分三道擊之,士標等走南雄。盛進克上猶,球進克龍泉。國正聞師定漳州,遣球及諸將劉體君等出間道援剿。十六年,破賊五里排,會里、瑞金、崇義以次下。韓大任自寧都敗竄萬安,國正遣兵四出斷道,並絕糧運;令以松等追擊,戰鸕鶿寨,戰老虎洞,屢敗之。大任走汀州,降。江西平。敘功,累進兵部尚書銜。十八年,左副都御史楊雍建疏論國正蒞任數載,治績無聞。京察循例自陳,降二級調用。四十七年,卒於家。

周有德,字彞初,漢軍鑲紅旗人。順治二年,自貢生授弘文院編修。五年,從英親王阿濟格討叛將姜瓖,還,遷侍讀。康熙元年,遷國史院侍讀學士,尋擢弘文院學士。

二年,授山東巡撫。三年,以獲逃人加工部侍郎銜。迭疏請寬登、萊、青三府海禁,俾居民得捕魚資生;請以歷城明季籓府地視民田科賦;請復孤貧口糧;請以德州駐防兵舊給民地五百餘頃仍還之民,駐防兵視陜西、浙江例支月糧;請蠲逋賦六十餘萬,暨察出逃亡荒蕪虛增田額戶口凡四十萬有奇,悉予免除。四年,濟南、兗州、東昌、青州四府旱災,請加賑恤;登州、萊州二府歉收,請免本年額賦;皆下部議行。

六年,擢兩廣總督。七年,上遣都統特錦等會勘廣東沿海邊界,設兵防汛,俾民復業。有德疏言:「界外民苦失業,聞許仍歸舊地,踴躍歡呼。第海濱遼闊,使待勘界既明,始議安插,尚需時日,窮民迫不及待。請令州縣官按遷戶版籍給還故業。」得旨允行。是冬,遭父喪,平南王尚可喜疏言沿海兵民,方賴經營安輯,請命在任守制。凡三年而事定。九年,疏請還京師治喪,許之。

十年,旱,求言,編修陳志紀疏言:「上憂勤惕厲,而嘗為督撫諸大臣方營第宅,蓄倡優,近在輦轂下,不守法度,何以責遠方大吏廉節?」上命指實,覆疏舉郎廷佐、張長庚、苗澄,祖澤溥、張朝璘、許世昌並及有德,下部嚴察,有德坐居喪營造,又於志紀覆疏未入時,囑託毋及其名,奪官,追繳誥命。

吳三桂反,十三年,起授四川總督。三桂將吳之茂、彭時亨等犯廣元,有德與副都統科爾寬分道擊敗之,陣斬裨將徐應昌等。上命經略尚書莫洛自陜西入四川,敕有德與巡撫張德地固守廣元諸路,並督軍餉。三桂將何德成等自昭化攻二郎關,謀奪我師儲峙,有德遣兵擊德成,走還昭化,復犯廣元;有德與科爾寬等復擊敗之,逐北三十餘里。時亨屯七盤、朝天諸關,劫略陽糧艘,廣元餉不給。寇窺陽平,將軍席卜臣屯蟠龍山為所劫,斷我師餉道,上命有德固守陽平諸路。

王輔臣叛,十四年,上命大將軍貝勒董額討之,以有德參贊軍務,命督諸軍協擊。董額克秦州,有德乞還誥命,吏部持非例,上特許之。十五年,從大將軍大學士圖海攻平涼,輔臣降。圖海疏令有德還駐西安。之茂等尚駐秦嶺,十七年,與副都統覺和託督兵擊之,降其裨將王世祜等。

十八年,調雲貴總督。師克漢中,上諭責「有德、德地等前駐廣元督餉遲誤,致數年來逆賊逋誅,兵民苦累。今大兵前進,督撫諸臣有誤餉運,以軍法從事。」王大臣議師自湖廣進征雲、貴,綠旗兵當有統帥,以湖廣總督蔡毓榮及有德名上,上以命毓榮,令有德受節制。有德尋疾作,留駐常德。十九年,卒。

張德地,初名劉格,漢軍鑲藍旗人。初以通曉國書,在戶部學習。順治九年,授宗人府主事,累遷戶部督捕理事官。康熙元年,擢順天府尹。二年,授四川巡撫。疏言:「四川自張獻忠亂後,地曠人稀,請招民承墾。文武吏招民百戶、墾田十頃以上,予遷轉。」下部議行。累加工部尚書銜。十年,武生劉琯等訐德地主武鄉試得賄鬻武舉,遣副都御史阿範等按治,德地坐斬,命免死奪官。德地叩閽稱枉,下部覆議,以事無據,復官。十三年,復授四川巡撫。時亨犯廣元,德地與有德督兵御之。十四年,王輔臣叛,命協守西安,尋又命出駐延安。廣元之役,有德劾德地棄城走,奪官。二十二年,卒。

伊闢,字盧源,山東新城人。順治五年,舉鄉試第一。十二年,成進士,改庶吉士。十三年,授御史。十四年,巡按山西,捕長治亂民勒化龍,窮治其黨與。十六年,還,掌京畿道,擢通政司參議。累遷大理寺卿。

康熙十九年,授雲南巡撫。時吳世璠未平,師自廣西、貴州、四川分道入,闢督餉。圍會城未下,同知劉昆不屈於三桂,為所縶,至是始脫出。闢從諮策,昆曰:「公用人寬,降人予原職。今安寧、晉寧、昆陽、呈貢諸縣令悉降人,昆池舟楫往來無禁。豈有父兄被圍而子弟不為轉輸者?」闢為罷諸降人,寇餉漸斷。師久次,慮餉不繼。闢疏請貴州、廣西二路協濟銀米,上以二路道險山多,轉運不便,遣戶部郎中明額禮、薩木哈詣軍酌議採買。軍中或議取食民間,布政使王繼文持不可,曰:「現糧支三日,昆陽、宜良寇遺糧,方具資庀役運詣軍前。兩廣隨軍餉銀十萬在曲靖,當請於總督金光祖,乞相假。過三日餉不繼,請正繼文軍法。」闢言於大將軍貝子彰泰,用其議。不三日,銀粟皆至,民以得安,餉亦無闕。闢疏言:「雲南地處天末,當得重臣彈壓。元鎮以親王,明則黔國公任留守。王師計日蕩平,臣自鎮遠至雲南,途次聞士民語,僉謂大將軍貝子彰泰、內大臣額駙華善所過不擾,請特簡一人鎮守。」章下所司。闢旋病作,遺疏薦繼文自代。卒,賜祭葬。

繼文,字在燕,漢軍鑲黃旗人。自官學生授弘文院編修,遷兵部督捕副理事官。順治十二年,考選御史,巡按陜西。初受事,即疏劾布政使黃紀、興屯道白士麟貪污不法,奪官逮治。十四年,還京師,都察院列上繼文在官劾文武吏四十餘,督開荒田七千頃有奇,招徠流移民五千八百餘,察出虛冒錢糧七千七百有奇,實心任事,允為稱職。遷戶部郎中。十八年,授江西饒九南道。康熙三年,調浙江寧紹臺道。六年,缺裁。

十三年,師討吳三桂,命以候補道從左都御史多諾等如荊州督餉,用繼文策度地建倉,分餽東西二路軍及水師。旋授雲南布政使,從師進徵。二十年,代闢為巡撫,佐將軍趙良棟攻克會城,雲南遂定。二十一年,與總督蔡毓榮疏言:「會城東南舊有金汁河,引盤龍江水入昆明池,舊存壩閘涵洞,積水溉田。世璠毀為壕塹,令官吏捐資修治。」下部議,捐銀百,紀錄一次。二十五年,以憂歸。二十八年,復授巡撫,疏言:「黑井鹽課,三桂月增課銀二千兩,請豁除。屯田科賦十倍於民田,重為民累,請分別改視民田起科。」三十年,疏言:「土司奏銷遲誤,例無處分,請比照流官計俸罰米,移貯附近常平倉備荒賑。」皆議行。

三十三年,擢雲貴總督。三十七年,討平魯魁山寇,釐定汛界,駐兵防守。又疏議收水西宣慰使地,改屬大定、平遠、黔西三州流官管轄,均如所請。是歲冬,朝京師,以老病乞致仕。尋命修理子牙河工。賜御書榜曰「煙霞耆舊」。四十年,加兵部尚書銜。四十二年,卒,賜祭葬。子用霖,官山東布政使。

論曰:毓榮統綠旗兵下雲南,廉清不逮趙良棟,戰績與相亞。哈占鎮陜西,衛國定江西,有德略四川,督餉治軍,其於戡亂皆與有功。雲南既下,撫綏安集之績,毓榮開之,繼文成之,自是西南遂底於平矣。


\end{pinyinscope}