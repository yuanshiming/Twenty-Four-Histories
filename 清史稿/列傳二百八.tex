\article{列傳二百八}

\begin{pinyinscope}
沈兆霖曹毓瑛許乃普子彭壽趙光硃嶟

李菡張祥河羅惇衍鄭敦謹龐鍾璐

沈兆霖,字朗亭,浙江錢塘人。道光十六年進士,選庶吉士,授編修。十九年,大考二等。二十五年,遷司業。二十六年,遷侍講,入直上書房,授惇郡王讀。二十九年,遷侍講學士,直南書房。歷詹事、內閣學士。咸豐二年,擢吏部侍郎,督江西學政。

三年,粵匪自武昌下九江,兆霖請速援南昌。上諮以軍事,兆霖奏言:「江西會城雖暫可無慮,賊擾外府,省兵不能兼顧。外府各有團練,如肯齊心協力,何藉分兵?即如撫州鄉團不下數萬,皆留保本村,官兵祗三百,已調赴會城。如團練不能合力,賊至何以御之?其故皆因堅壁清野,舊議祗守本村,並不出戰,不知事與嘉慶間川、楚教匪不同。川、楚教匪劫掠村莊,自以堅守堡寨為是,今賊專攻省會、郡縣城池,城既破,鄉勇亦相與解散矣。撫州如此,各省各府亦必皆然。乞飭直省當於練勇中精選十之二三,聯為鄉兵,統以練達有位望之人。遇本縣有警,互相救援。其外府、外縣仍不得調往,以免擾累。」得旨允行。尋以病乞罷。

五年,病痊,署吏部侍郎,仍直南書房。兆霖疏言:「安徽各郡,江北安、廬、和,江南池、太,皆為賊踞。巡撫駐廬州,東北徽、寧、廣三屬,幾為巡撫號令所不及。事急則向浙江請餉,事平則洩沓如前,不加整飭,旋收旋失,糜餉殃民。臣察徽、寧二府,山川險固,地皆可守,民亦健奮,歙、休寧二縣,尤多富民。宜於皖南設大員,專轄四府、一州,庶以飭吏治,固民心。度險設防,皖撫得專心於江北,浙撫亦不至牽制於皖南。」疏下廷議,改池太道為皖南道,得專摺奏事,如福建臺灣道例,從之。尋兼署工、兵二部。

六年,授吏部侍郎,調工部,復調戶部。八年,命往通州察覈通濟庫,奏請如戶部三庫例,以倉場侍郎兼管,佩印鑰,著為令。九年,擢左都御史。十年,署戶部尚書。七月,英吉利、法蘭西兵內犯,兆霖疏請專講守御,勿汲汲言撫。九月,授兵部尚書。撫議既定,上猶駐熱河,兆霖與諸大臣奏請回鑾,上命待明年。兆霖復奏請明年春融,即啟蹕還京。尋調戶部。

十一年,穆宗回鑾即位,命充軍機大臣。甘肅西寧撒回為亂,總督樂斌遣提督成瑞率兵討之,逗撓不進。樂斌用西寧辦事大臣多慧議招撫,亂久未定。上命兆霖偕尚書麟魁往按,盡發樂斌等瞻徇貽誤狀,樂斌戍新疆,成瑞、多慧逮京治罪。同治元年,命兆霖署陜甘總督,親督兵自碾伯進擊撒回,屢敗之,撒回乞降。七月,師還,次平番二道嶺溝,雨雹,山水驟發,兆霖及從行兵役並沒。水退,得兆霖尸,猶端坐輿中。布政使恩麟以聞,上深惜之,賜恤,贈太子太保,謚文忠。

曹毓瑛,字琢如,江蘇江陰人。道光十七年拔貢,授兵部七品小京官,遷主事,充軍機章京。二十三年,舉順天鄉試,再遷郎中。咸豐十年,擢鴻臚寺少卿。時江南大營潰,總督何桂清棄常州,蘇、常相繼陷。毓瑛疏陳軍事,略曰:「拯溺救焚,其事宜急而不宜緩。搗虛批亢,其事宜合而不宜分。臣前讀都興阿奏,擬自英山由豫境繞赴徐、宿,以達江北,而曾國籓通籌方略,擬分三路進剿,俟八月大舉。竊謂都興阿由豫境以達江北,程途紆遠,非兩月不能到。浙江自蕭翰慶陣亡,江長貴自平望退守,銳氣盡消。以屢潰之孱兵,御剽悍之勍賊,待至八月,松、太、杭、嘉、湖諸郡勢將瓦解,蔓延愈廣,規復愈難。為今計者,都興阿宜自英、霍取道臨、鳳以抵江北,不過旬日,即由通、泰渡江,直抵江陰,進攻常州、無錫為一路,而以周沐潤所募沙勇副之;鎮江現有兵萬餘,巴棟阿、馮子材、向奎進規丹陽為一路;薛煥在上海增募勇丁萬人,由嘉定、太倉、昆山進攻蘇州為一路,而命張玉良出嘉興、平望以副之;曾國籓率楚師由寧國取道廣德,進抵嘉、湖為一路,策應諸軍,而令米興朝攻宜興、溧陽,周天受攻高淳、東壩,曾秉忠督長龍船入太湖以副之。攻賊之所必救,據賊之所必爭。俟曾國籓新募勇至,然後分路進剿,庶於事有濟。」

英、法兩國合兵犯京師,上幸熱河,軍書旁午,樞臣未全從,上命擇章京資深才優者佐諸大臣辦事。毓瑛在直久,諸大臣欲舉以應,固辭,遂越次用焦祐瀛。十一年,穆宗即位,諸大臣皆譴罷,乃命毓瑛在軍機大臣上學習行走,遷順天府丞。同治元年,遷大理寺卿,授軍機大臣。二年,擢工部侍郎,調兵部。三年,江南平,加頭品頂戴,賜花翎,署兵部尚書。四年,擢左都御史,尋授兵部尚書。五年,卒,贈太子少保,謚恭愨。

方端華、肅順擅政,毓瑛獨不附。及佐樞政,廉慎勿懈,每謂:「軍旅大事,患在信任不專,事權不一。古來良將,率以掣肘不能成功。」時以為名言云。

許乃普,字滇生,浙江錢塘人。拔貢,考授七品小京官,充軍機章京。嘉慶二十五年,成一甲二名進士,授編修。道光三年,直南書房。四年,大考二等,擢洗馬。五年,督貴州學政,任滿回京,仍直南書房,累遷侍讀。十三年,復以大考二等擢侍講學士,督江西學政,三遷內閣學士。十八年,擢刑部侍郎,罷直南書房,專治部事。調吏部,又調戶部。二十一年,擢兵部尚書。二十五年,坐事鐫五級,補太常寺少卿,遷光祿寺卿。

三十年,文宗御極,命仍直南書房。詔求言,乃普疏言:「方今先務,莫急於正君心,培聖德。請敕館臣合列朝聖訓,依類分門,排日進呈,庶政奉以為宗。恩詔各省保舉孝廉方正,請敕下各直省學政考覈學官,學官得人,所舉庶幾可恃。刑部於致死胞伯叔及胞兄之案,以事關服制,往往夾簽聲明,並非有心干犯,巧為開脫。請敕下刑部斟情酌理,俾無枉縱。各省綠營弁兵平時宜加意訓練,武職到京,兵部驗看時,當令兼演火器。」疏上,得旨:「下所司議奏。」復申諭刑部及各督撫,服制案罪名務得實情。咸豐二年,授內閣學士。乃普疏論軍營奏報欺飾,得旨,令各路統兵大臣及各督撫力除積習,嚴為稽察,其朦混掩飾者,據實嚴參。擢兵部侍郎。三年,粵匪陷九江,擾皖北,覬覦北鄉,而廬、鳳守御單弱,乃普疏請調黑龍江兵,道山東、江南,徑赴安徽,遠可張蘇、浙之聲援,近可固廬、鳳之門戶。調刑部,尋擢工部尚書,調刑部。

國子監司業崇福奏請豫徵山西咸豐四年錢糧,軍機大臣等會議,推及陜西、四川兩省,乃普偕侍郎何彤雲奏言:「各省情形不一,應由各督撫體察情形。山西被賊各州縣及陜西之延安、榆林、綏德、興安,四川之寧遠各府,地瘠民貧,均請免其借徵。至畸零小戶,有田數畝或數十畝,僅足餬口,仍令照常例完納,庶民力不至重困。」又奏言:「時值嚴寒,用兵尤宜撫恤。聞通永鎮兵四百名,去賊最近,而強半尚衣秋衣;重以行營所在,百物昂貴,無錢者往往須取於民,以致負販裹足,兵士轉不免於饑寒。請飭統兵大臣悉心籌度。」從之。又言:「江南大營老師糜餉,皆由琦善等意見不和,舒興阿自陜赴皖,所在稽留,沿途需索。今命與江忠源會剿,不獨難以和衷,且恐因之掣肘。又方今餉需艱難,軍務一日未蕆,即度支一日不敷,惟在大師刻日奏功,以紓天下之困。請皇上嚴加督責,信賞必罰,以振暮氣。」疏上,嘉納之。

四年,刑部主事王式言坐承審命案,聽授請託,失入絞罪。事聞,上命裕誠等按治,乃普以式言本門生,奏請回避,弗許。既而裕誠等讞式言僕受賕,上責乃普回護,降補內閣學士,罷直南書房。尋遷禮部侍郎,擢左都御史。六年,遷工部尚書。八年,命督五城團防。九年,調吏部。十年,文宗三旬慶辰,加太子太保。九月,以病乞罷。同治五年,卒,謚文恪。

子彭壽,字仁山。道光二十七年進士,選庶吉士,授編修,累遷少詹事。咸豐十一年,文宗崩,命議郊配禮,彭壽偕大理寺少卿潘祖廕奏言:「臣讀大行皇帝聖制甲寅孟夏齋宮即事詩,末句『以後無須再變更』,注云:『天壇配享,三祖、五宗為定,永不增配位。恐後代無知故違,則儀文太繁。』臣等仰瞻聖藻,躬懸齋宮,言法行則,非博謙讓虛名。弓劍未寒,不忍頓生異議。」禮遂定。

時肅順等獲罪,彭壽請察治黨援,旨令指實。奏言侍郎成琦,太僕寺卿德克津泰,候補京堂富績,侍郎劉昆、黃宗漢。得旨:「糾彈諸事,朕早有聞,特懲一儆百,力挽頹靡。此後不咨既往,諸臣亦毋以黨援陳奏,致啟訐陷。」於是陳孚恩等譴黜有差。彭壽又以載垣等隨事刻深,戶部五宇官錢案請再清釐,從之。同治初,再遷內閣學士,署禮部左侍郎。五年,卒。

趙光,字蓉舫,雲南昆明人。嘉慶二十五年進士,選庶吉士,授編修。遷御史、給事中,轉光祿寺少卿,五遷內閣學士。擢兵部侍郎,調戶部。

文宗即位,奏陳時務,略言:「安民先察吏,州縣為親民之官,秩卑責重。捐例屢開,仕塗益雜。幕友招搖,書役播弄,賄囑情託,靡所不至。正供則挪移侵虧,訟案則株連擱壓,偶或參劾,輒籌抵制。大吏慮其噬臍,曲予寬容,同僚相率效尤,成為習慣。應請飭令督、撫、司、道,嚴行舉錯,以肅官方。國家糜餉養兵,冀收實用,近日營伍將弁,虛文操演,廝役士卒,養尊處優。空名漁利,器械不修,槍砲無準,而水師尤為窳敝。往往居岸自適,風沙水線,都未研習,洋面不靖,盜劫頻聞。前者海疆有事,船遠距而彈施,敵近前而藥罄,束手無策,慄體先逃。凡諸軍備,轉為寇齎。甚至軌律盡隳,沿途坐索,長官乞哀,乃始進行。軍威不肅,一至於此。夫練兵必先練將,材藝邁眾,忠勇無前,如昔時楊遇春輩,渺不可得,緩急何恃?應請飭令將軍、督、撫、提、鎮,整齊營伍,鼓勵人才,以修武備。詰奸除暴,莫如保甲,近來直隸、山東盜賊日眾,至河南之捻匪,四川之啯匪,廣東之土匪,貴州之苗匪,雲南之回匪,肆意強橫,目無法紀,邪教充斥,名目紛繁。煽誘既眾,蹂躪彌多。地方文武,恐滋事端,惟務姑息。胥差既豢賊縱容,兵弁復得規徇隱。幹吏嚴拘,則聲息潛通,奪犯戕官,釀成巨患。其愚懦者,但期文過,諱盜為竊,避重就輕,以至匪徒益無忌憚,禍不勝言。應請飭令各直省督撫,認真整頓,奉行保甲,緝捕勤能,據實獎勵;疲玩者撤參重處,以戢盜風。直省倉庫錢糧,各有定額,州縣官如果侭數徵解,交代清晰,何至虧空盈千累萬?其致此之由,厥有數端:或紈褲而登仕版,習尚奢華;或庸瞶而暱親隨,開銷浮濫;或負累已深,官項償其私債;或交游太廣,正款供其應酬。寅支卯糧,東挪西掩,有漕者藉口於幫丁之需索,解庫者歸咎於糧價之增昂。道府察知,往往礙於情面,曲意彌縫,後任慮招重怨而不敢發,上司恐興大獄而不敢參,即使查抄,終歸無著。是以州縣交代,有歷數任而未算結者,有合數十州縣而未盤查者。前者欽差大臣會同各督撫清查整理,嚴定章程,虧短各案,業已分別攤賠。第恐舊虧未完,新虧已續,應請敕令各直省督撫督同司道各官詳細查覈,交代未清者,停其委署升補,虧那者嚴參,以清積弊。」疏入,優詔嘉納。

三年,擢工部尚書,調刑部。八年,命偕尚書周祖培等督五城團防事宜,歷兼署工部、兵部、戶部、吏部尚書。四年,卒,謚文恪。

硃嶟,字致堂,雲南通海人。嘉慶二十四年進士,選庶吉士,授檢討,遷御史。道光十二年,畿輔災,廣東副貢生潘仕成捐貲助賑,賜舉人。有援案以請者,嶟疏言:「仕成本副貢,去舉人一間,賜以舉人,於破格之中,仍寓量才之意。厥後葉元堃、黃立誠次第援請,若因此遂成定例,生富人徼幸,阻寒士進修,於事不便。應請旨飭各督撫,水旱偏災,捐輸應獎,不得援引前案。」上嘉納之。五遷至內閣學士。十七年,擢兵部侍郎,迭兼署吏、戶二部,坐事鐫五秩。二十六年,補內閣侍讀學士。

御史劉良駒條奏銀錢畫一,上命各省督撫議奏。嶟疏言:「泉布之寶,國專其利,故定賦以粟,而平貨以錢。物賤由乎錢少,少則重,重則加鑄而散之使輕;物貴由乎錢多,多則輕,輕則作法而斂之使重。一輕一重,張弛在官,而權操於上。今出納以銀,錢幾置諸無用。雖國寶流通,然流於下而不轉於上。於是富商市儈,得乘其乏、操其贏,而任意以為輕重。若使官為定價,且必格而不行。要在因其便使人易從,通其變使人不怨,行其權使人不疑。方今鹽務疲敝,皆以銀貴錢賤為詞,以鹽賣錢而不賣銀也。賣錢即解錢,人必樂從,長蘆鹽價可解京充餉。請於東西城建庫藏錢,以戶、工左右侍郎掌之,按時價搭放各旗,就近赴庫請領,以免其轉運,並嚴禁克扣、短陌、攙雜諸弊。兩淮鹽價,解備河工歲修。淮上全工,水路皆通,輓運較易,工次雇夫購料,俱系用錢,此兩便之道也。農民以錢輸賦,天下十居七八。地方官收錢解銀,每致賠累。江西撫臣吳文鎔前奏:『本省坐支之項,收錢放錢;解部候撥之款,徵銀解銀;兵餉役食,請照時價改折。』其言不為無見。惟全行收錢,往返搬運,倍增勞費。通省絕無銀幣,亦未免偏枯。擬請州縣徵收,向來徵銀解銀者置無論,但照現在收錢者,量錢糧多少,視附近地方兵役眾寡,酌減應解銀數,以紓其困。除易銀解司之外,即以錢抵銀,每銀一兩,折錢若干,酌定數目,按照時價,支放兵餉役食。應有耗羨平餘,仍行提出解司,而本管同城之官俸,本州縣之書工、役食、祭祀、驛站,本地方分汛之兵餉,俱準坐支。餘則視道路之遠近,解存道、府、籓各庫,以放兵餉。時價則視省垣為準,以開徵前十日為定,由籓司通飭遵照,半年一更。餉銀每兩折錢多不過千七百,少不過千二百,取為定則,不得再減。至文武官廉俸無可坐支者,兵丁屯駐之區,附近州縣無收錢者,皆發銀如故。官局錢搭放向有成例者亦如故。如是,則雖變而實因,不至糾紛窒礙。至如百姓出粟米麻絲易錢輸賦,久已習為故常,向收若干,今折若干,凡自封投櫃者,不遽改折,是於民無擾也。兵丁領銀,仍須易錢然後適用。每至兵領餉時,不準鋪戶抑價,今照定價放給滿錢,此於兵無虧也。先時銀多,則官以收錢漁利;今時錢賤,則官以易錢賠累。多用錢則少解銀,即累亦因而減,迨銀價平時,又復可獲羨餘,此於官有益也。或謂錢收於上,則廛市一空,恐致錢荒。不知兵役領錢,仍行於市,地方官除存庫外,尚有大半必須易銀解司,則其錢亦行於市。且今日之弊,不在錢荒而在錢濫,欲救其弊,莫利於收錢,尤莫利於停鑄。當此錢賤之時,暫停鼓鑄,以工本之銀,發出易錢,實收上庫。薄小者汰之,則私鑄難行,而官錢日多,錢價可平,而制錢一千準銀一兩之例,可得而行矣。是知停鑄者用錢之轉關,平價者絕私之微權也。將欲平價,非使銀錢相埒不可,為平價而暫停鑄,迨價平而復開爐,所謂欲贏先縮,一張一弛之道也。夫損上必期益下,今錢值日賤,物價日貴,泉府費兩錢而成一錢,官兵領一錢則僅當半錢。無益於民,有損於國,孰得孰失,必有能辨之者。總之可用錢則用錢,必須用銀則仍用銀。附近則用錢,致遠則用銀。子母相權,贏縮有制,補偏救弊,無逾於此。惟各省情形不一,因地制宜,隨時變通。當責各督撫體察酌議盡善。」疏入,上命軍機大臣會同戶部議行。

歷通政副使、內閣學士。二十九年,授倉場侍郎。咸豐四年,病,乞罷。五年,病痊,復授戶部侍郎。六年,擢左都御史。迭署兵、禮二部尚書。十一年,又以病乞罷。同治元年,卒,謚文端。

李菡,字豐垣,順天寶坻人。道光二年進士,選庶吉士,授編修。再遷侍講,大考二等,擢侍講學士。二十一年,遷少詹事,督安徽學政,累遷通政使。二十五年,擢左副都御史。

咸豐元年,署禮部侍郎,應詔上疏:「請戒飭諸臣:一曰振因循。積習相仍,中外一轍。用兵無可退之理,乃引疾歸田,抽身保位,則因循在軍旅矣。治水為難緩之功,乃自冬徂夏,漫口未合,則因循在河防矣。雍沙番案,琦善以總督大員,猶復語多狡飾,以至往返鞫訊,則因循在刑法矣。順天武清縣逃犯,竟敢窩藏匪徒,浙江奉化縣刁民並敢迫脅官長,則因循又在郡縣矣。伏原皇上乾綱獨振,力挽頹風,聞嘉謨則立見施行,睹弊政則悉除支蔓。惰者責之,勇者獎之,勤者進之,昏者黜之,庶奮庸熙載,百廢俱修矣。一曰除欺飾。粵西逆匪,萌蘗在十數年之前,使撫臣早為奏聞,何難根株立絕?乃養癰成患,諱莫如深。比及有人指陳,勢已不可撲滅。年來勞師糜餉,迄無成功,禍首罪魁,實由欺始。夫獻可替否,宰相之責也;拾遺補闕,諫官之職也。伏望皇上開誠布公,虛懷善納,導之使言,言之使盡,執兩用中,歸於至當。至科道職司言責,尤朝廷耳目之官,風聞偶誤,小過可容,庶贛直得效其愚,萋菲莫行其罔,而宸聰四達矣。一曰屏偏私。人之氣質,不能無偏,意見少有參差,議論遂多齟。相持不下,教令紛更,屬員既無所適從,宵小遂從而讒構。嫌隙日深,乖氣致戾。刑部越獄一事,非其明驗乎?夫師克在和不在眾,兩粵會剿,湖南防堵,將帥不應有諉罪爭功之見,督撫不可存此疆爾界之私,同德同心,群策效力。茍無隙之可乘,定膚功之克奏。河、漕本屬一體,未有河不治而漕治者。從前督臣、漕臣,曾因參劾員,各執己見,現在漫口不能合龍,漕船何由利濟?億萬姓饑民待賑,數百萬帑項虛糜,正大臣憂患與共之時。此即屏除嫌怨,共秉公忠,猶恐難以濟時屯而紓民患;倘仍芥蒂未化,籌畫分歧,不和政龐,咎將誰執?伏讀仁宗御制和同論,諄諄以臣下偏私為戒。原皇上一德交孚,與百僚共襄上理焉。一曰防玩法。現今軍務、河工,貽誤諸臣,厥咎匪細。仰蒙寬典,僅予薄懲,恕其既往之愆,責其將來之報。而且失伍之將弁,準其帶罪立功,潰防之河員,許其留工效力,恢宏大度,格外矜全,天下皆曉然於聖人不得已之苦心,與夫通變權宜之計,該大臣等久蒙倚任,渥荷優容,自無不激厲圖功,竭忠矢志。第恐奔走禦侮,難得賢員,幸澤恃恩,復萌故智。始猶懼罪之不可逭,一旦獲宥,遂謂罪有可原矣;初猶慮法之不能逃,幸而茍免,遂謂法止於是矣。伏原皇上奮天錫之勇,播神武之風,寬大之詔,能發而即能收,希冀之恩,可一而不可再。則德威惟畏,玩縱之萌,不戢自止矣。以上四條,皆臣道之防,實切時之弊,而其本由於得人。進英銳,則因循者退矣;取誠篤,則欺飾者鮮矣。惟在皇上任賢勿疑,用材器使,俾朝無幸位,莫不圖易思艱,庶可挽天災民變之窮,而上副引咎納言之至意。」疏入,上嘉納之。

三年,授兵部侍郎,署倉場侍郎。廉得奸人把持倉務,置於法。十年,調工部,復調吏部。同治元年,擢工部尚書。二年,卒,謚文恪。

張祥河,字詩舲,江蘇婁縣人。嘉慶二十五年進士,授內閣中書,充軍機章京。遷戶部主事,累轉郎中。道光十一年,出為山東督糧道。十七年,擢河南按察使,以父憂去官。服除,仍授河南按察使,署布政使。二十二年,祥符決口合龍,賜花翎,詔以河南迭被水災,始終克勤其事,予優敘。二十四年,遷廣西布政使,擢陜西巡撫。西安、同州有刀匪擾害閭閻,祥河飭嚴捕百餘人置諸法,詔嘉之。三十年,文宗即位,應詔陳言,請述祖德,守成法,勵官方,蠲民欠。疏入,報聞。祥河優於文事,治尚安靜,不擾民,言者劾其性耽詩酒。

咸豐二年,東南軍事日棘,祥河奏言:「陜西興安等地毗連楚境,應舉行團練,擇要防堵。惟鄉勇良莠不齊,易聚難散,不如力行保甲,為緝奸良法。」三年,召還京。四年,授內閣學士,尋遷吏部侍郎,督順天學政。六年,以病罷。病痊,仍授吏部侍郎。八年,擢左都御史,遷工部尚書。十年,加太子太保。十一年,以病乞罷。同治元年,卒,謚溫和。

羅惇衍,字椒生,廣東順德人。道光十五年進士,選庶吉士,授編修。十七年,督四川學政,召對,上以惇衍年少,語多土音,留不遣。二十三年,大考一等,擢侍講。累遷侍讀學士,轉通政副使、太僕寺卿。二十六年,督安徽學政,遷通政使。

三十年,文宗即位,應詔陳言,略言:「古帝王治天下,根源祗在一心,要在覽載籍,勤省察,居敬窮理,以檢攝此心。聖祖仁皇帝御纂性理精義,於存養省察、致知力行,以及人倫性命,皆有程途階級,其論君道,尤極詳備。惟在皇上講習討論,身體力行。世宗憲皇帝硃批諭旨,於臣工奏摺,指示得失,明見萬里。皇上幾暇,日閱一二事,凡督撫陳奏,如能深謀遠慮,措置得宜,即予以褎答;若有飾詐懷私,亦為之指示,庶大吏皆知警戒。他若御纂資政要覽、庭訓格言諸書,皆本心出治,一以貫之。伏原皇上法祖以修己,推而知人安民,皆得其道。」又請諭部院大臣各舉所知,備京卿及講讀之任;敕直省督撫、提鎮、學政皆得犯顏直諫,指陳利病,無所忌諱,籓臬亦許密封由督撫代為呈奏。疏入,上嘉納之。咸豐元年,疏陳風俗侈靡,民生日困,請崇儉禁奢,以蓄物力。二年,署吏部侍郎,授左副都御史。

三年,擢刑部侍郎,仍兼權吏部。時軍需孔亟,戶部令京師商民以賃舍金一月納公家,惇衍以為非政體,疏乞明定限制。又疏薦廣東在籍給事中蘇廷魁等任籌軍餉。江寧既陷,寇氛復溯江上犯,惇衍疏請敕曾國籓練楚勇,自湖南移駐武昌,杜賊窺伺荊襄;蘇廷魁募粵勇援江西;袁甲三回河南防捻匪,並會同已革兩廣總督徐廣縉募新兵堵御鳳、潁,遏賊北竄諸路;多被採納。命隨同惠親王巡防京師,調戶部。五年,以父憂歸。

七年,英吉利兵攻陷廣州,八年正月,命惇衍及在籍太常寺卿龍元僖、給事中蘇廷魁為團練大臣。十年,款議定。十一年,召來京,擢左都御史。

同治元年,兩廣總督勞崇光被劾任用非人,調度乖方,命惇衍偕廣州將軍穆克德訥按治,崇光坐罷。遷戶部尚書,疏言:「吏治日壞,當獎廉懲貪。四川總督駱秉章、湖北巡撫嚴樹森、山西布政使鄭敦謹、山東按察使吳廷棟,清操較著,請獎之,以勵其餘。」又疏言:「皇上求賢若渴,應詔者寥寥,即有登諸薦牘者,或由他省督撫保舉,必待本省給咨,始能赴部,非所以示虛懷延攬之道。且但令封疆大吏保舉,而未及京卿,恐馴致外重內輕,不可不防其漸。內閣、六部、九卿等朝廷重臣,素所親信,必俾其各舉所知,眾正盈廷,然後可反危為安,轉亂為治。請不必限以時日,拘以人數,但有操守廉潔,才猷卓越者,即許隨時疏薦。倘所舉之人,將來或犯貪污,罪其舉主。」二年,兼署左都御史。

四年,兼管三庫,署翰林院掌院學士。伊犁參贊大臣聯捷、御史陳廷經先後論劾「陜西布政使林壽圖沉湎於酒,巡撫劉蓉未諳公事,舉劾悉聽壽圖」,及「蓉疏奏失體,漏洩密保」。命偕協辦大學士瑞常赴陜西按治。惇衍等為疏辨,僅以微過議處,吏議壽圖遷調,蓉革職留任。尋蓉復以他事罷,陜民為蓉、壽圖訟冤,總督楊岳斌以聞。惇衍等已回京復命覆奏,遂合疏言:「劉蓉秉性樸直,辦理甘肅潰勇,不動聲色,悉臻妥善。甘肅亂回竄擾,遣兵分布要隘,陜民以安。林壽圖身任勞怨,勤奮有為,惟參劾屬員,間有輕重失當,致謗毀紛興,而其廉潔之操,究不能稍加訾議。」詔蓉仍署巡撫,壽圖來京候簡用。六年,兼署工部。八年,以母憂歸。十三年,卒,謚文恪。

惇衍學宗宋儒,立朝正色,抗論時事,章凡數十上,無所顧避。著有集義編、百法百戒、庸言、孔子集語等書。

鄭敦謹,字小山,湖南長沙人。道光十五年進士,選庶吉士,散館授刑部主事。再遷郎中,出為山東登州知府,擢河南南汝光道。咸豐元年,泌陽土匪喬建德踞角子山,敦謹與南陽鎮總兵圖塔布督兵捕獲之,被議敘,署布政使。二年,授廣東布政使,仍留署任。

粵匪入湖北,命赴信陽,會南陽鎮總兵柏山扼要設防。三年,命河南巡撫陸應穀統兵駐南陽,會城及信陽有事,許敦謹專摺馳奏。欽差大臣琦善督師援安徽,檄敦謹總理信陽糧臺。及師屯江北,糧臺移設徐州,仍令敦謹往任其事。尋調授河南布政使,留筦糧臺如故。四年,光州、陳州捻匪起,巡撫英桂出駐汝陽,詔敦謹赴本任。省城戒嚴,敦謹督率官紳倡捐經費,興團練。皖捻犯永城、夏邑,增調兵勇防黃河各渡口,斷寇北竄。尋命暫署巡撫。

五年,坐欠解甘肅兩年協餉,降調。召還京,以四品京堂候補,授太常寺少卿。八年,督山東學政,累遷大理寺卿。同治元年,署戶部侍郎,復出為山西布政使,調署陜西布政使,調授直隸布政使,擢河東河道總督。四年,授湖北巡撫,尋召授戶部侍郎。五年,調刑部。

六年,擢左都御史。捻匪渡河入山西境,巡撫趙長齡、按察使陳湜疏防被劾,詔敦謹往按,長齡、湜並坐罷,即命敦謹署山西巡撫。七年,出省治防,移軍駐澤州欄車鎮,為各路策應。授工部尚書,仍留署巡撫。回匪入河套,近邊震動。敦謹移駐寧武督防,別遣兵守榆林、保德下游各隘。增募砲勇,補葺河曲邊墻。回匪窺包頭鎮,沿河堵御,會綏遠城將軍定安遣隊迎剿,總兵張曜自河曲截擊,破走之。八年,調兵部尚書,回京。

九年,調刑部。兩江總督馬新貽被刺,獲兇犯張汶祥,江寧將軍魁玉、漕運總督張之萬會讞,言汶祥為洪秀全餘黨,其戕新貽,別無主謀者。命敦謹往會鞫,仍以初讞上,論極刑。十年春,敦謹還京,至清江浦,上疏以病乞罷。光緒十一年,卒,謚恪慎。

龐鍾璐,字寶生,江蘇常熟人。道光二十七年一甲三名進士,授編修。咸豐二年,大考一等,擢庶子,遷侍講學士,署祭酒。明年,授光祿寺卿。八年,擢內閣學士,署工部侍郎,以父憂歸。十年,江南大營潰,蘇、常淪陷,督團勇防禦。上命鍾璐陳奏軍事,鍾璐疏言:「常、昭三面皆賊,惟恃民團抵御。器械不精,紀律不明,若大兵不速至,恐裹脅愈多,愈難措手。請飭督臣曾國籓迅由祁門統師南下,常、昭庫款無存,惟賴捐輸充餉,軍需浩穰,捐戶搜括無遺。並請飭督臣於就近完善之區,籌貲接濟。」又奏:「江北惟通州最完善,與常、昭有脣齒之依。在籍布政使徐宗幹廉能素著,請飭令督辦通、泰一路捐輸,並會籌常、昭防剿。」從之。

尋命督辦江南團練。賊由江陰東竄,偪常熟,鍾璐率團勇數戰,亡其精銳,奏請江北諸軍速援。上以水陸各軍勢難兼顧,溫詔慰勉。八月,賊陷常熟,鍾璐奏自劾,並請飭荊州將軍都興阿統楚師兼程進駐通州防北竄,上責令規復。鍾璐自崇明赴上海,設局勸捐,集團守御。薦上海知縣劉郇膏循聲卓著,為江南州縣之冠,報聞。又以軍需餉急,奏請令失守地方官罰鍰免治罪,諭有「捐輸巨款、募勇殺賊、隨官兵克復城池者,得據實聲明請旨」。尋奏言:「賊所脅之眾數百萬人,何一非皇上赤子?若非設法解散,窮無所歸,必鋌而走險。請明降諭旨,予以自新,釋兵歸降者勿殺,薙發投順者勿殺。又陷賊州縣,多設立偽官,迫索錢米,以減輕田賦,搖動人心。歷來被兵州縣,錢糧均奉恩旨蠲免。此次蘇省被賊,戶口散亡,收復之後,無從徵收,不如施恩於未復之先,使愚民不為所惑。」詔如所請。

十一年春,賊自平湖、乍浦窺金山,鍾璐督團勇進擊,斬馘甚眾。新埭賊擾大泖港,楓涇賊窺角鉤灣,復會官兵破之。是年冬,以蘇、常淪陷,吳民待援,有逾饑渴,復疏請敕曾國籓分兵急取蘇、常。與江蘇諸士紳貽書國籓,言:「上海餉源重地,請以奇兵萬人,一勇將統之,倍道而來,可當十萬之用。」國籓乃遣李鴻章率師浮江而東。俄、法兩國請助兵討寇,鍾璐奏言:「中國平內亂,原無待藉手外人,而值賊勢蔓延,兵力單薄,不能不為從權之計。惟外人助攻,為通商而起,必先自有把握,方裨大局。」諭江蘇巡撫薛煥妥籌酌行。

尋裁各省團練大臣,召還京,再授內閣學士。同治元年,遷禮部侍郎,迭署工、吏諸部,督順天學政。四年,呈所纂文廟祀典考。六年夏,畿輔亢旱,疏陳荒政十事,下部議行。命偕大學士賈楨等督五城團防,歷戶、兵、吏諸部。九年,擢左都御史,署工部尚書。十年,授刑部尚書。丁母憂,歸。光緒二年,卒,謚文恪。

子鴻文,光緒二年進士;鴻書,光緒六年進士:同官翰林院編修。鴻文至通政司副使,鴻書至貴州巡撫。

論曰:同治初政,沈兆霖、曹毓瑛入贊樞府,兆霖暫領陜督,督師定西寧,以死勤事;毓瑛慎密練達,克副簡拔。許乃普等皆以清謹負時望,鄭敦謹尤易又歷有名績。江寧之獄,論者多謂未盡得其情,敦謹未覆命,遽解官以去,其亦有所未慊於衷歟?


\end{pinyinscope}