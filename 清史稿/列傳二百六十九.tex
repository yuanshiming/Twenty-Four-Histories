\article{列傳二百六十九}

\begin{pinyinscope}
儒林三

馬宗梿子瑞辰孫三俊張惠言子成孫江承之郝懿行陳壽祺子喬■謝震何治運孫經世柯蘅許宗彥呂飛鵬沈夢蘭宋世犖嚴可均嚴元照焦循子廷琥顧鳳毛鍾懷李鍾泗李富孫兄超孫弟遇孫胡承珙胡秉虔硃珔凌曙薛傳均劉逢祿宋翔鳳戴望雷學淇王萱齡崔述胡培翬楊大堉劉文淇子毓崧孫壽曾方申丁晏王筠曾釗林伯桐李黼平柳興恩弟榮宗許桂林鍾文烝梅毓陳澧侯康侯度桂文燦鄭珍鄒漢勛王崧劉寶楠子恭冕龍啟瑞苗夔龐大★陳立陳奐金鶚黃式三子以周俞樾張文虎王闓運王先謙孫詒讓鄭杲宋書升法偉堂

馬宗梿,字器之,桐城人。由舉人官東流縣教諭。嘉慶六年成進士,又一年卒。少從舅氏姚鼐學詩、古文詞,所作多沉博絕麗,既而精通古訓及地理之學。鄉舉時,以解論語過位、升堂合於古制,大興硃珪亟拔之。後從邵晉涵、任大椿、王念孫游,其學益進。嘗以解經必先通訓詁,而載籍極博,未有匯成一編者,乃偕同志孫星衍、阮元、硃錫庚分韻編錄,適南旋中輟。其後元視學江、浙,萃諸名宿為經籍篡詁,其凡例猶宗梿所手訂也。生平敦實,寡嗜好,惟以著述為樂。嘗撰左氏補注三卷,博徵漢、魏諸儒之說,不茍同立異。所著別有毛鄭詩詁訓考證、周禮鄭注疏證、穀梁傳疏證、說文字義廣證、戰國策地理考、南海鬱林合浦蒼梧四郡沿革考、嶺南詩鈔,共數十卷,校經堂詩鈔二卷。

子瑞辰,字元伯。嘉慶十五年進士,選翰林院庶吉士。散館,改工部營繕司主事。擢郎中,因事罣誤,發盛京效力。旋賞主事,奏留工部,補員外郎。復坐事發往黑龍江,未幾釋歸。歷主江西白鹿洞、山東嶧山、安徽廬陽書院講席。發逆陷桐城,眾驚走,賊脅之降,瑞辰大言曰:「吾前翰林院庶吉士、工部都水司員外郎馬瑞辰也!吾命二子團練鄉兵,今仲子死,少子從軍,吾豈降賊者耶?」賊執其發爇其背而擁之行。行數里,罵愈厲,遂死,年七十九。事聞,恤廕如例,敕建專祠。

瑞辰勤學著書,耄而不倦。嘗謂:「詩自齊、魯、韓三家既亡,說詩者以毛詩為最古。據鄭志答張逸云:『注詩宗毛為主,毛義隱略,則更表明。』是鄭君大旨,本以述毛,其箋詩改讀,非盡易傳。而正義或誤以為毛、鄭異義。鄭君先從張恭祖受韓,凡箋訓異毛者,多本韓說。其答張逸亦云:『如有不同,即下己意。』而正義又或誤合傳、箋為一。毛詩用古文,其經字多假借,類皆本於雙聲、疊韻,而正義或有未達。」於是乃撰毛詩傳箋通釋三十二卷,以三家辨其異同,以全經明其義例,以古音、古義證其譌互,以雙聲、疊韻別其通借。篤守家法,義據通深。同時長洲陳奐著毛詩傳疏,亦為專門之學。由是治毛詩者多推此兩家之書。

子三俊,字命之。優貢生。舉孝廉方正,學宗程、硃。以國難家仇,憤欲殺賊。咸豐四年六月,率練勇追賊至周瑜城,力戰死,年三十五。著有馬徵君遺集。

張惠言,字皋聞,武進人。少受易經,即通大義。年十四為童子師,修學立行,敦禮自守,人皆稱敬。嘉慶四年進士,時大學士硃珪為吏部尚書,以惠言學行特奏改庶吉士,充實錄館纂修官。六年,散館,改部屬,珪復特奏授翰林院編修。七年,卒,年四十有二。

惠言鄉、會兩試皆出硃珪門,未嘗以所能自異,默然隨群弟子進退而已。珪潛察得之,則大喜,故屢進達之,而惠言亦齗齗相諍不敢隱。珪言天子當以寬大得民,惠言言:「國家承平百年餘,至仁涵育,遠出漢、唐、宋之上,吏民習於寬大,故奸孽萌芽其間,宜大伸罰以肅內外之政。」珪言天子當優有過大臣,惠言言:「庸猥之輩,幸致通顯,復壞朝廷法度,惜全之當何所用?」珪喜進淹雅之士,惠言言「當進內治官府、外治疆埸者」,與同縣洪亮吉於廣坐諍之。

惠言少為詞賦,擬司馬相如、揚雄之文。及壯,又學韓愈、歐陽修。篆書初學李陽冰,後學漢碑額及石鼓文。嘗奉命詣盛京篆列聖加尊號玉寶,惠言言於當事,謂舊藏寶不得磨治;又謂翰林奉命篆列聖寶,宜奏請馳驛,以格於例不果行。

生平精思絕人,嘗從歙金榜問故,其學要歸六經,而尤深易、禮。著有周易虞氏義、虞氏消息,序曰:「自漢成帝時,劉向校書,考易說,以為諸易家皆祖田何、楊叔、丁將軍,大義略同,惟京氏為異。而孟喜受易家陰陽,其說易本於氣,而後以人事明之。八卦六十四象,四正七十二候,變通消息,諸儒祖述之,莫能具。當漢之季年,扶風馬融作易傳,授鄭康成作易注。而荊州牧劉表、會稽太守王朗、潁川荀爽、南陽宋忠皆以易名家,各有所述。唯翻傳孟氏學,既作易注,奏上之獻帝。翻之言易,以陰陽消息六爻,發揮旁通,升降上下,歸於乾元用九而天下治。依物取類,貫穿比附,始若瑣碎,及其沉深解剝,離根散葉,暢茂條理,遂於大道,後儒罕能通之。自魏王弼以虛空之言解易,唐立之學官,而漢世諸儒之說微,獨資州李鼎祚作周易集解,頗採古易家言,而翻注為多。其後古書盡亡,而宋道士陳摶以意造為龍圖,其徒劉牧以為易之河圖、洛書也,河南邵雍又為先天、後天之圖,宋之說易者翕然宗之,以至於今,牢不可拔,而易陰陽之大義,蓋盡晦矣。大清有天下,元和徵士惠棟,始考古義孟、京、荀、鄭、虞氏,作易漢學,又自為解釋,曰周易述。然掇拾於亡廢之後,左右採獲,十無二三。其所述大氐宗禰虞氏,而未能盡通,則旁徵他說以合之。蓋從唐、五代、宋、元、明朽壞散亂千有餘年,區區修補收拾,欲一旦而其道復明,斯固難也。翻之學既邃,又具見馬、鄭、荀、宋氏書,考其是否,故其義為精。又古書亡,而漢、魏師說可見者十餘家,然唯鄭、荀、虞三家略有梗概可指說,而虞尤較備。然則求七十子之微言,田何、楊叔、丁將軍之所傳者,舍虞氏之注,其何所自焉?故求其條貫,明其統例,釋其疑滯,信其亡闕,為虞氏義九卷;又表其大旨,為消息二卷。」又著有虞氏易禮二卷,虞氏易候一卷,虞氏易言二卷。

初,惠棟作周易述,大旨遵虞翻,補以鄭、荀諸儒,學者以未能專一少之。儀徵阮元謂漢人之易,孟、費諸家,各有師承,勢不能合。惠言傳虞氏易,即傳漢孟氏易矣,孤經絕學也。惠言又著周易鄭氏義三卷,周易荀氏九家義一卷,周易鄭荀義三卷,易義別錄十四卷,易緯略義三卷,易圖條辨二卷。其易義別錄序,謂不盡見其辭而欲論其是非,猶以偏言決獄也。故其所著,皆羽儀虞氏易者。於禮有儀禮詞一卷,讀儀禮記二卷,皆特精審。又有茗柯文五卷,詞一卷。

子成孫,字彥惟。少時,惠言課以說文,令分六書譜之,成象形二卷。惠言著說文諧聲譜,未竟而卒,成孫後從莊述祖游,得其大要,乃續成之。卷第篇例多所增易,凡五十卷。其書分中、僮、薨、林、巖、筐、榮、蓁、詵、千、萋、肄、揖、支、皮、絲、鳩、芼、蔞、岨二十部,此乃於毛詩中拈其最先出之字為建首,加以易韻、屈韻,而又以說文之聲分從之,犁然不紊,有各家所未及者。嘗以示儀徵阮元,元嘆其超卓精細。成孫兼精天學,同裏董祐誠歿,為校刊其遺書。又著有端虛勉一居文集。

江承之,字安甫,歙縣人。學於惠言。時弟子從惠言受易、禮者十數,其甥董士錫受易,通陰陽五行家言;承之兼受易、禮,著有周易爻義、虞氏易變表、儀禮名物、鄭氏詩譜,年僅十有八。

郝懿行,字恂九,棲霞人。嘉慶四年進士,授戶部主事。二十五年,補江南司主事。道光三年,卒,年六十九。

懿行為人謙退,訥若不出口,然自守廉介,不輕與人晉接。遇非素知者,相對竟日無一語,迨談論經義,則喋喋忘倦。所居四壁蕭然,庭院蓬蒿常滿,僮僕不備,懿行處之晏如。浮沉郎署,視官之榮悴,若無與於己者,而一肆力於著述,漏下四鼓者四十年。所著有爾雅義疏十九卷,春秋說略十二卷,春秋比一卷,山海經箋疏十八卷,易說十二卷,書說二卷。

懿行嘗曰:「邵晉涵爾雅正義蒐輯較廣,然聲音訓詁之原,尚多壅閡,故鮮發明。今餘作義疏,於字借聲轉處,詞繁不殺,殆欲明其所以然。」又曰:「餘田居多載,遇草木蟲魚有弗知者,必詢其名,詳察其形,考之古書,以徵其然否。今茲疏中其異於舊說者,皆經目驗,非憑胸肊,此餘書所以別乎邵氏也。」懿行之於爾雅,用力最久,槁凡數易,垂歿而後成。於古訓同異,名物疑似,必詳加辨論,疏通證明,故所造較晉涵為深。高郵王念孫為之點閱,寄儀徵阮元刊行。元總裁會試時,從經義中識拔懿行者也。

其箋疏山海經,援引各籍,正名辨物,事刊疏謬,辭取雅馴。阮元謂吳氏廣注徵引雖博,失之蕪雜;畢沅校本,訂正文字尚多疏略;惟懿行精而不鑿,博而不濫。

懿行妻王照圓,字瑞玉。博涉經史,當時著書家,有「高郵王父子,棲霞郝夫婦」之目。著有詩說一卷,列女傳補注八卷,附女錄一卷,女校一卷。又與懿行以詩答問,懿行錄之為詩問七卷,其爾雅義疏亦間取照圓說;他著有詩經拾遺一卷,汲塚周書輯要一卷,竹書紀年校正十四卷,荀子補注一卷,晉宋書故一卷,補晉書刑法志一卷,食貨志一卷,文集十二卷。照圓又有列仙傳校正二卷。

陳壽祺,字恭甫,閩縣人。少能文。年十八,臺灣平,撰上福康安百韻詩並序,沉博絕麗,傳誦一時。嘉慶四年成進士,選翰林院庶吉士,散館授編修。尋告歸,性至孝,不忍言仕,家貧無食,父命之入都。九年,充廣東鄉試副考官。十二年,充河南鄉試副考官。十四年,充會試同考官,京察一等,記名御史。壽祺以不得迎養二親,常愀然不樂。將告歸矣,俄聞父歿,慟幾絕,奔歸。服除,乞養母,母歿,終喪。年五十三,有密薦於朝者,卒不出。

壽祺會試出硃珪、阮元門,乃專為漢儒之學,又及見錢大昕、段玉裁、王念孫、程瑤田諸人,故學益精博。解經得兩漢大義,每舉一義,輒有折衷。

兩漢經師莫先於伏生,莫備於許氏、鄭氏,壽祺闡明遺書,著尚書大傳箋三卷、序錄一卷、訂誤一卷,附漢書五行志,綴以他書所引劉氏五行傳論三卷。序曰:「伏生大傳,條撰大義,因經屬恉,其文辭爾雅深厚,最近大小戴記七十子之徒所說,非漢諸儒傳訓之所能及也。康成百世儒宗,獨注大傳,其釋三禮,每援引之。及注古文尚書,洪範五事,康誥孟侯,文王伐崇、愬耆之歲,周公克殷、踐奄之年,咸據大傳以明事,豈非閎識博通信舊聞者哉?且夫伏生之學,尤善於禮,其言巡狩、朝覲、郊尸、迎日、廟祭、族燕、門塾、學校、養老、擇射、貢士、考績、郊遂、採地、房堂、路寢之制,後夫人入御,太子迎問諸侯之法,三正之統,五服之色,七始之素,八伯之樂,皆唐、虞、三代遺文,往往六經所不備,諸子百家所不詳。今其書散逸,十無四五,尤可寶重。宋硃子與勉齋黃氏纂儀禮經傳通解,攟摭大傳獨詳,蓋有裨禮學不虛也。五行傳者,自夏侯始昌,至劉氏父子傳之,皆善推福著天人之應。漢儒治經,莫不明象數陰陽,以窮極性命。故易有孟、京卦氣之候,詩有翼奉五際之要,春秋有公羊災異之條,書有夏侯、劉氏、許商、李尋洪範之論。班固本大傳,攬仲舒,別向、歆,以傳春秋,告往知來,王事之表,不可廢也。是以錄漢書五行志附於後,以備一家之學云。」

又著五經異義疏證三卷,左海經辨二卷,左海文集十卷,左海駢體文二卷,絳趺堂詩集六卷,東越儒林文苑後傳二卷,東觀存槁一卷。

壽祺歸後,阮元延課詁經精舍生徒。元纂群經古義為經郛,壽祺為撰條例,明所以原本訓辭、會通典禮、存家法而析異同之意。後主泉州清源書院十年,主鼇峰書院十一年,與諸生言修身勵學,教以經術,作義利辨、知恥說、科舉論以示學者。規約整肅,士初苦之,久乃悅服。家居與諸當事書,於桑梓利弊,蒿目痗心,雖觸忌諱無所隱。明儒黃道周孤忠絕學,壽祺搜輯遺文,為之刊行。又具呈大吏,乞疏請從祀孔廟,議上,如所請。道光十四年,卒,年六十四。

子喬樅,字樸園。道光五年舉人,二十四年,以大挑知縣分發江西。歷官分宜、弋陽、德化、南城諸縣,署袁州、臨江、撫州知府。以經術飾吏治,居官有聲。同治七年,卒於官,年六十一。初,壽祺以鄭注禮記多改讀,又嘗鉤考齊、魯、韓三家詩佚文、佚義與毛氏異同者,輯而未就。病革,謂喬樅曰:「爾好漢學,治經知師法,他日能成吾志,九原無憾矣!」喬樅乃紬繹舊聞,勒為定本,成禮記鄭讀考六卷,三家詩遺說考十五卷。又著齊詩翼氏學疏證二卷,詩緯集證四卷。謂齊詩之學,宗旨有三:曰四始,曰五際,曰六情。皆以明天地陰陽終始之理,考人事盛衰得失之原,言王道治亂安危之故。齊先亡,最為寡證,獨翼奉存其百一,且其說多出詩緯,察躔象,推歷數,徵休咎,蓋齊學所本也。詩緯亡而齊詩遂為絕學矣。又著今文尚書經說考三十四卷,歐陽夏侯經說考一卷。謂:「二十九篇今文具存,十六篇既無今文可考,遂莫能盡通其義。凡古文易、書、詩、禮、論語、孝經所以傳,悉由今文為之先驅,今文所無輒廢。向微伏生,則萬古長夜矣。歐陽、大小夏侯各守師法,茍能得其單辭片義,以尋千百年不傳之緒,則今文之維持聖經於不墜者,豈淺尟哉!」又有詩經四家異文考五卷,毛詩鄭箋改字說四卷,禮堂經說二卷,最後為尚書說。時宿學漸蕪,考據家為世訾謷,獨湘鄉曾國籓見其書以為可傳。自元和惠氏、高郵王氏外,惟喬樅能修世業,張大其家法。

壽祺同里治古學者,有謝震、何治運。

震,原名在震,字甸男,侯官人。乾隆五十四年舉人,官順昌學教諭。震嘗與閩縣林一桂、甌寧萬世美俱精三禮,震尤篤學嗜古。然齗齗持漢學,好排擊宋儒鑿空逃虛之說。壽祺與震同舉鄉試,少震六歲,視為畏友。震重氣誼,有志用世,而不遇於時,年四十卒。弟子輯其遺著,有禮案二卷,精覈勝敖氏。又有四書小箋一卷,四聖年譜一卷。工詩,有櫻桃軒詩集二卷。

治運,字支阜海,閩縣人。嘉慶十二年舉人。洽聞彊識,篤志漢學。粵督阮元嘗聘纂廣東通志。後游浙中,巡撫陳若霖為鋟其經解及論辨文字四卷,名何氏學。道光元年,卒,年四十七。治運與壽祺友,及卒,壽祺以謂無與為質,不獲以輔成其學也。

孫經世,字濟侯,惠安人。壽祺弟子。壽祺課士不一格,游其門者,若仙游王捷南之詩、禮、春秋、諸史,晉江杜彥士之小學,惠安陳金城之漢易,將樂梁文之性理,建安丁汝恭、德化賴其煐、建陽張際亮之詩、古文辭,皆足名家。而經世學成蚤世,世以儒林推之。經世少喜讀近思錄,後沉研經義,謂不通經學,無以為理學;不明訓詁,無以通經;不知聲音文字之原,無以明訓詁。著說文會通十六卷,爾雅音疏六卷,釋文辨證十四卷,韻學溯源四卷,十三經正讀定本八十卷,經傳釋辭續編八卷。又著春秋例辨八卷,孝經說二卷,夏小正說一卷,詩韻訂二卷,惕齋經說六卷,讀經校語四卷。

柯蘅,膠州人。從壽祺受許、鄭之學,嘗以史、漢諸表為紀、傳之綱領,而譌誤舛奪,最為難治,乃條而理之,著漢書七表校補二十卷。為例十:一曰辨事誤,二曰辨文字誤,三曰辨注誤,四曰辨諸家考證之誤,五曰以本書證本書之誤,六曰史、漢互證而知其誤,七曰漢書、荀紀互證而知其誤,八曰漢書、水經注互證而知其誤,九曰據紀、傳以補表之闕,十曰據今地以證表之誤。鉤稽隱賾,凡前人之說,皆取而辨其是非,至前人未及者,又得二三十事,亦專門之學也。尤長於詩,著有聲詩闡微二卷,舊雨草堂詩集四卷,其說經、說史之作,門人集為舊雨草堂札記。

許宗彥,字積卿,德清人。九歲能讀經、史。善屬文,侍郎王昶愛其才,作積卿字說以贈。嘉慶四年進士,授兵部主事,就官兩月,以親老遽引疾歸。親歿,卒不出。居杭州,杜門以讀書為事。其學無所不通,探賾索隱,識力卓然,發千年儒者所未發。考周五廟二祧,以為周制五廟之外,別有二祧,為遷廟之殺,以厚親親之仁。宗廟之外,別立祖宗,與禘、郊同為重祭,以大尊尊之義。諸經無文、武二廟不毀之說,誤始於韋玄成,而劉歆因之,鄭康成亦因之。祧者遷廟,乃謂為不遷之廟,名實乖矣。又考文、武二世室,以為周文、武皆配於明堂太室,故有「文、武世室」之號。孔穎達誤謂伯禽稱「文世室」,周公稱「武世室」。以公羊傳周公稱「太廟」、魯公稱「世室」、群公稱「宮」證之,舛甚。

又考禹貢三江,以為漢志言「分江水首受江,東至餘姚入海」。夫曰「分江水」,曰「首受江」,則非南江之正流可知;曰「東至餘姚入海」,則非在吳入海可知,與禹貢三江無與。又考太歲、太陰,以為太歲者,歲星與日同次鬥杓所建之辰也。太陰始寅終丑,太歲始子終亥。漢律志曰:「太初元年,歲前十一月朔旦冬至,歲在星紀婺女六度,歲名困敦。」此太歲始子之碻證。武帝詔曰:「年名焉逢、攝提格。」此太陰始寅之碻證。漢書天文志始誤以甘、石之言太陰者系之太歲,而與太初之太歲遂差兩辰,乃以為星有贏縮,非矣。

又說六書轉注,以為從偏旁轉相注。說文曰:「轉注者,建類一首,同意相受,考老是也。」後序曰「其建首也,立一為耑」,即建類一首之謂也。如示為部首,從示之偏旁注為神祇等字,從神祇注為祠祀祭祝等字,展轉相注,皆同意為一類。戴震指爾雅詁訓為轉注,而不知詁訓出於後來,非制字時所豫有也。段玉裁引戴說,又言爾雅字多假借,而不知假借者本無其字,今如初、哉、首、基之訓,非本無首字,而假初、哉諸字以當之也。其他所著學說,能持漢、宋儒者之平。禮論、治論諸篇,皆稽古證今,通達政體。

尤精天文,得泰西推步秘法,自制渾金球,別具神解。嘗援緯書四游以疏本天高卑,而知不同心非渾圓之理。考周髀北極璿璣,以推古人測驗之法。七政皆統於天,而知東漢以前用赤道不用黃道,為得諸行之本。論日左右旋一理,以王錫闡解黃道右旋、赤道平行,戴震分黃、極為二行,其說頗不分明,為剖析之,洞徹微妙,皆言天家所未及。

性孝友,慎於交游,體羸而神理澂淡,見者皆肅然敬之。儀徵阮元,會試舉主也,重其學術行誼,以子女為★I2家。

呂飛鵬,字雲裏,旌德人。從寧國凌廷堪治禮,廷堪器之,以為能傳其學。山陽汪廷珍視學安徽,喜士通古經義,補飛鵬縣學附生。

飛鵬少讀周禮,長而癖嗜,廷堪嘗著周官九拜九祭解、鄉射五物考,援據禮經,疏通證明,足發前人所未發。飛鵬師其意而變通之,成周禮補注六卷。其大旨以鄭氏為宗,自序曰:「漢、魏之治周禮者,如賈逵、張衡、孫炎、薛綜、陳劭、崔靈恩之注,遺文軼事,散見群籍。或與鄭義符合,或與鄭義乖違,同者可得其會通,異者可博其旨趣。是用廣搜眾說,補所未備,條系於經文之下,或旁採他經舊注,或兼取近儒經說,要於申明古義而已。」又著周禮古今文義證六卷,嘗考康成本治小戴禮,後以古經校之,取其於義長且順者為鄭氏學。又注小戴所傳禮記四十九篇,又嘗作毛詩箋:「今取鄭氏之學證鄭氏之注,則辭易了然,即彼此互歧、前後錯出,亦不煩辭費而得失已明,故於三者刺取為多。至許氏說文解字,徵引周禮,彼此互異,取以推廣鄭義,不嫌牴牾。其他史冊流傳,事系本朝,禮遵周典,亦備採擇,用俟辯章。猶是鄭氏況以漢法之意也。」

平居書齋閣自銘誡,粹然出於儒先道學。鄉饑,籌粟倡賑,人多德之。有爭辯,一言立釋。嘗戒其子賢基曰:「成名易,成人難。」又曰:「言官不易為,毋陳利而昧大體,毋挾私而務高名。」其本行如此。賢基卒以忠節著。道光二十九年,卒,年七十三。子賢基,工部右侍郎,謚文節,自有傳。

有清為周禮之學者,有惠士奇、沈彤、莊存與、沈夢蘭、段玉裁、徐養原、宋世犖。

夢蘭,字古春,烏程人。乾隆四十八年舉人,官湖北宜都縣知縣。夢蘭博通諸經,實事求是,尤邃於周官,成周禮學一書。分溝洫、畿封、邦國、都鄙、城郭、宮室、職官、祿田、貢賦、軍旅、車乘、禮射、律度量衡十三門,取司馬法、逸周書、管子、呂覽、伏傳、戴記諸古書參互考證,合之書、詩、禮記、三傳、孟子,先儒所病其牴牾者,無不得其會通。為圖若干,並取經、傳文之與周官相發明者釋於篇。他著有易、書、詩、孟子學,五省溝洫圖說。其易學自序云:「自輯周禮學,於易象得井、比、師、訟、同人、大有若乾卦,錯綜參伍,知易之為道,先王一切之治法於是乎在。」而孟子學,則又以疏證周官之故,匯其餘說以成帙者。其溝洫圖說,卷不盈寸,凡南北形勢、河道原委、歷代沿革、眾說異同,與夫溝遂經畛之體,廣深尋尺之數,以及蓄水、止水、蕩水、均水、舍水、瀉水之事皆備。復證之周官,考究詳覈。官湖北時,奉檄襄築荊州堤工,上江堤埽工議及荊江論。沔陽水災,復奉檄會勘,作水利說以諭沔民。原本經術,有裨實用,皆此類也。

世犖,字卣勛,臨海人。乾隆五十三年舉人,以教習官陜西扶風知縣。地當川、藏孔道,夫馬悉斂之民。計畝率錢,名曰「公局」。世犖多所裁革,無妄取。時教匪初定,州縣多以獲盜遷擢。扶風民有持齋為怨家所訐者,大府飛檄至,捕而鞫之,皆良民,釋弗顧。罷歸,揅求經訓,熟於諧聲、假借之例,著周禮故書疏證六卷,儀禮古今文疏證二卷。

嚴可均,字景文,烏程人。嘉慶五年舉人,官建德縣教諭,引疾歸。可均博聞強識,精考據之學,與姚文田同治說文,為說文長編,亦謂之類考。有天文、算術、地理類,草木、鳥獸、蟲魚類,聲類,說文引群書、群書引說文類,積四十五冊。又輯鐘鼎拓本為說文翼十五篇,將校定說文,撰為疏義。孫星衍促其成,乃撮舉大略。就毛氏汲古閣初印本別為校議三十篇,專正徐鉉之失。

又與丁溶同治唐石經,著校文十卷,自序云:「餘弱冠治經,稍見宋槧本。既又念若漢、若魏、若唐、若孟蜀、若宋嘉祐、紹興各立石經,今僅嘉祐四石,紹興八十七石,皆殘本。而唐大和石壁二百二十八石,巋然獨存,此天地間經本之最完最舊者也。夫唐代四部之富,埒於梁、隋,而鄭覃、唐元度輩皆通儒,頗見古本。茍能栞正積非,歸於真是,即方駕熹平不難,而僅止於是。今也古本皆亡,欲復舊觀,已難為力,可嘅也!然而後唐彫版,實依石經句度鈔寫,歷宋、元、明轉刻轉誤,而石本幸存,縱不足與復古,以匡今繆有餘也。獨怪數百年來,學士大夫鮮或過問者,間有一二好古之士,亦與塚碣、寺碑同類而並道之。康熙初,顧炎武始略校焉,觀其所作九經誤字、金石文字記,刺取寥寥,是非寡當,又誤信王堯惠之補字以誣石經。顧氏且然,況其他乎?烏乎!石經者,古本之終,今本之祖。治經不及見古本,而並荒石經,匪直荒之,又交口誣之,豈經之幸哉?餘不自揆,欲為今版本正其誤,為唐石經釋其非,為顧氏等袪其惑。隨讀隨校,凡石經之磨改者、旁增者與今本互異者皆錄出,輒據注疏、釋文,旁稽史、傳及漢、唐人所徵引者,為之左證,而石臺孝經附其後焉。」

嘉慶十三年,詔開全唐文館,可均以越在草茅,無能為役,慨然曰:「唐之文,盛矣哉!唐以前要當有總集。斯事體大,是餘之責也。」乃輯上古三代秦漢三國六朝文,使與全唐文相接,多至三千餘家,人各系以小傳,足以考證史文,皆從蒐羅殘賸得之,覆檢群書,一字一句,稍有異同,無不校訂。一手寫定,不假效力。唐以前文,咸萃於此焉。又校輯諸經逸注及佚子書等數十種,合經、史、子、集為四錄堂類集千二百餘卷。

嚴元照,字九能,歸安人。十歲能為四體書,補諸生。儀徵阮元、大興硃珪深賞之。熟於爾雅,作匡名八卷,旁羅異文軼訓,鉤稽而疏證之。著有悔葊文鈔、詩鈔、詞鈔,娛親雅言等書。

焦循,字裏堂,甘泉人。嘉慶六年舉人,曾祖源、祖、父蔥,世傳易學。循少穎異,八歲在阮賡堯家與賓客辨壁上「馮夷」字,曰:「此當如楚辭讀皮冰切,不當讀如縫。」阮奇之,妻以女。既壯,雅尚經術,與阮元齊名。元督學山東、浙江,俱招循往游。性至孝,丁父及嫡母謝艱,哀毀如禮。一應禮部試,後以生母殷病愈而神未健,不復北行。殷歿,循毀如初。服除,遂託足疾不入城市者十餘年。葺其老屋,曰半九書塾,復構一樓,曰雕菰樓,有湖光山色之勝,讀書著述其中。嘗嘆曰:「家雖貧,幸蔬菜不乏。天之疾我,福我也。吾老於此矣!」嘉慶二十五年,卒,年五十八。

循博聞強記,識力精卓。每遇一書,無論隱奧平衍,必究其源,以故經史、歷算、聲音、訓詁無所不精。幼好易,父問小畜「密雲」二語何以復見於小過,循反復其故不可得。既學洞淵九容之術,乃以數之比例,求易之比例,漸能理解,著易通釋二十卷。自謂所悟得者,一曰哦曰旁通,二曰相錯,三曰時行。又以古之精通易理,深得羲、文、周、孔之恉者,莫如孟子。生孟子後,能深知其學者,莫如趙氏。偽疏踳駮,未能發明,著孟子正義三十卷。謂為孟子作疏,其難有十,然近代通儒,已得八九。因博採諸家之說,而下以己意,合孔、孟相傳之正恉,又著六經補疏二十卷。以說漢易者每屏王弼,然弼解箕子用趙賓說,讀彭為旁,借雍為甕,通孚為浮,解斯為廝,蓋以六書通借。其解經之法,未遠於馬、鄭諸儒,為周易王注補疏二卷。以尚書偽孔傳說之善者,如金縢「我之不闢」,訓闢為法,居東即東征,罪人即管、蔡,大誥周公不自稱王,而稱成王之命,皆非馬、鄭所能及,為尚書孔氏傳補疏二卷。以詩毛、鄭義有異同,正義往往雜鄭於毛,比毛於鄭,為毛詩鄭氏箋補疏五卷。以左氏傳「稱君君無道,稱臣臣之罪」,杜預揚其詞而暢衍之,預為司馬懿女壻,目見成濟之事,將以為司馬飾,即用以為己飾。萬斯大、惠士奇、顧棟高等未能摘奸而發覆,為春秋傳杜氏集解補疏五卷。以禮以時為大,訓詁名物,亦所宜究,為禮記鄭氏注補疏三卷。以論語一書,發明羲、文、周公之恉,參伍錯綜,引申觸類,亦與易例同,為論語何氏集解補疏三卷。合之為二十卷。又錄當世通儒說尚書者四十一家,書五十七部,仿衛湜禮記之例,以時之先後為序,得四十卷,曰書義叢鈔。又著禹貢鄭注釋一卷,毛詩地理釋四卷,毛詩鳥獸草木蟲魚釋十一卷,陸璣疏考證一卷,群經宮室圖二卷,論語通釋一卷。又著有雕菰樓文集二十四卷,詞三卷,詩話一卷。

循壯年即名重海內,錢大昕、王鳴盛、程瑤田等皆推敬之。始入都,謁座主英和,和曰:「吾知子之字曰里堂,江南老名士,屈久矣!」歿後,阮元作傳,稱其學「精深博大,名曰通儒」,世謂不愧雲。

子廷琥,字虎玉。優廩生。性醇篤,善承家學,阮元稱為端士。循嘗與廷琥纂孟子長編三十卷,後撰正義,其廷琥有所見,亦本範氏穀梁之例,為之錄存。循又以測圓海鏡、益古演段二書,不詳開方之法,以常法推之不合。既得秦道古數學九章,有正圓開方法,為開方通釋,乃謂廷琥曰:「汝可列益古演段六十四問,用正員開方法推之。」廷琥布策下算,一一符合,著益古演段開方補一卷。陽湖孫星衍不信西人地圓之說,以楊光先之斥地圓,比孟子之距楊、墨。廷琥謂古之言天者三家,曰宣夜,曰周髀,曰渾天。宣夜無師承,渾蓋之說,皆謂地圓。泰州陳氏、宣城梅氏悉以東西測景有時差,南北測星有地差,與圓形合為說。且大戴有曾子之言,內經有岐伯之言,宋有邵子、程子之言,其說非西人所自創。因博搜古籍,著地圓說二卷。他著有密梅花館詩文鈔。

顧鳳毛,字超宗,江蘇興化人。乾隆四十九年,南巡召試列二等,五十三年,副榜貢生。父九苞,字文子,長於詩、禮。九苞母任氏,大椿祖姑,通經達史。九苞之學,母所教也。乾隆四十六年進士,歸時卒於路,著述不傳。鳳毛亦受經於祖母,年十一,通五經。及長,與焦循同學,循就鳳毛問難,始用力於經。鳳毛又學音韻律呂於嘉定錢塘,撰楚辭韻考、入聲韻考、毛詩韻考,皆得塘旨。又撰毛詩集解,董子求雨考,三代田制考,未成而卒,年二十七。卒後,循理其喪,作招亡友賦哭之。

鍾懷、李鍾泗皆有名,均甘泉人。鍾懷,字保岐。優貢生。與阮元、焦循相善。共為經學,旦夕討論,務求其是。居恆禮法自守,不與世爭名,交游中稱為君子。嘉慶十年,卒,年四十五。著有■M2厓考古錄四卷。其漢儒考,較陸德明所載增多十餘人。

鍾泗,字濱石。嘉慶六年舉人,治經精左氏春秋,撰規規過一書,抑劉伸杜,焦循服其精博。

李富孫,字既汸,嘉興人。嘉慶六年拔貢生。良年來孫,良年自有傳。從祖集。字敬堂,乾隆二十八年進士,官鄖縣知縣。精研經學,以漢、唐為宗,嘗為學規論以課窮經、課經濟,著有原學齋文鈔。

富孫學有原本,與伯兄超孫、從弟遇孫有「後三李」之目。長游四方,就正於盧文弨、錢大昕、王昶、孫星衍,飫聞緒論。阮元撫浙,肄業詁經精舍,遂湛深經術,尤好讀易,著易解賸義。謂易學三派,有漢儒之學,鄭、虞、荀、陸諸家精矣;有晉、唐之學,王弼、孔穎達諸家,即北宋胡瑗、石介、東坡、伊川猶是支流餘裔;至宋陳、邵之學出,本道學之術,創為圖說,舉羲、文、周、孔之所未及,漢以後諸儒之所未言者,以自神其附會之說。理其理而非易之所謂理,數其數而非易之所謂數,而前聖之易道晦矣。唐李鼎祚所輯易解,精微廣大,聖賢遺旨,略見於此。然其於三十六家之說,尚多未採,其遺文賸義,間見他書,猶可蒐輯。爰綴而錄之,成書三卷,又成校異二卷。

又著七經異文釋,就經、史、傳、注、諸子百氏所引,以及漢、唐、宋石經,宋、元槧本,校其異同。或字有古今,或音近通假,或沿襲乖舛,悉據古誼而疏證之;而前儒之論說,並為蒐輯,使正其譌謬,辨其得失,折衷以求一是。凡易六卷,尚書八卷,毛詩十六卷,春秋三傳十二卷,禮記八卷。同裏馮登府稱其詳核奧博,為詁異義者集其大成。又謂說文一書,保氏六書之旨,賴以僅存。自篆變為隸,隸變為真,文字日繁,譌偽錯出。或有形聲意義大相區別,亦有近似而其實異,後人多混而同之。或有一篆之形,從某為古、籀,為或體,後人竟析而二之。經典文字,往往昧於音訓,擅為改易,甚與本義相迕,亦字學之大變。夫假借通用,說文自有本字,有得通借者,有不容通借而並為俗誤者。援據經典以相證契,俾世之踵謬沿譌焯然可辨,為說文辨字正俗八卷。同裏錢泰吉謂其書大旨折衷段注,而亦有段所未及者,讀說文之津梁也。

他著有漢魏六朝墓銘纂例四卷,鶴徵錄八卷、後錄十二卷,曝書亭詞注七卷,梅里志十六卷,校經廎文槁十八卷。

超孫,字引樹。嘉慶六年舉人,官會稽縣教諭。剖析經義,尤深於詩。嘗以毛詩草木蟲魚則有疏,名物則有解,地理則有考,而詩中所稱之人則未有纂輯成書者,因取詩人之氏族名字,博考經、史、諸子及近儒所著述,並列國之世次,洎其人之行事,搜羅薈集,為詩氏族考六卷。官會稽時,課諸生依寧化雷金宏學規條約,士習日上。又著拙守齋集。

遇孫,字金瀾,集孫。優貢生,處州府訓導。幼傳祖訓,淹貫經史,著有尚書隸古定釋文八卷。漢孔安國以科斗文難知,取伏生今文次第之,為隸古定,宋薛宣因之成古文訓。遇孫又以隸古文難知,引說文諸書疏通之,譌者是正,疑者則闕。性嗜金石,有芝省齋碑錄八卷,金石學錄四卷。官處州時,以處州地僻山遠,阮元兩浙金石志未免脫漏,乃搜輯數百餘種為括蒼金石志八卷。他著有日知錄補正一卷、校正一卷,古文苑拾遺十卷,天香錄八卷,隨筆六卷,詩文集十八卷。

胡承珙,字墨莊,涇縣人。嘉慶十年進士,選翰林院庶吉士,散館授編修。十五年,充廣東鄉試副考官,尋遷御史,轉給事中。自以身居言路,當周知天下利弊,陳之於上,方不負職。數年中陳奏甚多,多見施行。而其最切中時病者,則有條陳虧空弊端各條:「一曰冒濫宜禁。司庫支發錢糧,向有扣除二三成之弊,故籓司書吏將不應借支之款,冒支濫借。此在領者便於急需,不敢望其足數;而在放者利於多扣,不復問其合宜:則雖應放而仍與浮冒無異。一曰抑勒宜禁。州、縣交代,例限綦嚴,均不準充抵。近日仍多以議單欠票虛開實抵者,總由上司多方抑勒,逼令新任擔承。一曰糜費宜省。各省攤捐津貼名目,豈盡必不可省。聞州縣所解各上司衙門飯食季規等銀,逐歲增加。如邸報一事,安徽省每年通派各屬萬金。一省如此,他省可知;一事如此,他事可知。一曰升調宜慎。部選人員,多系初任,或尚能不敢輕易接受。惟佐雜題升,及調補繁缺二者,每多久歷仕途,習成狡滑。在題升者急於得缺,明知此地之多累,不復顧後而瞻前;在調補者遷就一時,轉因原任之有虧,希圖挪彼以掩此。究之擔承彌補,皆屬空名,不過剜肉補瘡,甚且變本加厲。」其言深切著明。二十四年,授福建分巡延建邵道,編查保甲,設立緝捕章程八條,匪徒斂跡。調署臺灣兵備道,緝獲洋盜張充等置於法。旋乞假回籍。臺灣素稱難治,承珙力行清莊弭盜之法,民、番安肅、自承珙去後,彰化、淡水即以械鬥起釁矣。道光十二年,卒,年五十七。

承珙究心經學,尤專意於毛詩傳,歸裏後鍵戶著書,與長洲陳奐往復討論不絕,著毛詩後箋三十卷。其書主於申述毛義,自注疏而外,於唐、宋、元諸儒之說,及近人為詩學者,無不廣徵博引,而於名物訓詁及毛與三家詩文有異同,類皆剖析精微,折衷至當。而其最精者,能於毛傳本文前後會出指歸,又能於西漢以前古書中反覆尋考,貫通詩義,證明毛旨。凡三四易,手自寫定。至魯頌泮水章而疾作,遺言囑陳奐校補,奐乃為續成之。又以鄭君注儀禮參用古、今文二本,撮其大例,有必用其正字者,有即用其借字者,有務以存古者,有兼以通今者,有因彼以決此者,有互見而並存者。閎意妙旨,有關於經實夥。遂取注中疊出之字,並「讀如」、「讀為」、「當為」各條,排比梳櫛,考其訓詁,明其假借,參稽旁採,疏通而證明之,作儀禮古今文疏義十七卷。又謂惠氏棟九經古義未及爾雅,遂補撰數十條,成二卷。小爾雅原本不傳,今存孔叢子中,世多謂為偽書,作小爾雅義證十三卷,斷以為真。復著有求是堂詩文集三十四卷。

胡秉虔,字伯敬,績溪人。嘉慶四年進士,官刑部主事,改甘肅靈臺縣知縣,升丹噶爾同知,卒於官。秉虔自幼嗜學,博通經史。嘗入都肄業成均,夜讀必盡燭二條。尤精於聲音訓話,著古韻論三卷,辨江、戴、段、孔諸家之說,細入毫芒,塙不可易。說文管見三卷,發明古音古義,多獨得之見。末論二徐書,有灼見語,蓋其所致力也。他著有周易、尚書、論語小識各八卷,卦本圖考一卷,尚書序錄一卷,漢西京博士考二卷。甘州明季成仁錄四卷,河州景忠錄三卷。

硃珔,字蘭坡,涇縣人。珔生三年而孤,祖命為季父後,嗣母汪未婚守志,珔孝事之與生母同,昆弟均相友愛。嘉慶七年成進士,選翰林院庶吉士,與幸翰林院栢梁體聯句宴。散館授編修,擢至侍讀。與修明鑒,坐承纂官累,降編修。道光元年,直上書房,屢蒙嘉獎,有「品學兼優」之褒。升右春坊右贊善,告養歸。植品敦俗,獎誘後進。歷主鍾山、正誼、紫陽書院,卒,年八十有二。

珔愛書如命,學有本原。主講席幾三十年,教士以通經學古為先。與桐城姚鼐、陽湖李兆洛並負儒林宿望,蓋鼎足而三云。著有說文假借義證二十八卷,經文廣異十二卷,文選集釋二十四卷,小萬卷齋詩文集七十卷。輯有國朝古文匯鈔二百七十二卷,又有詁經文鈔六十二卷,匯有清諸名家說經之文,依次標題,篇幅完善,尤足為後學津逮雲。

凌曙,字曉樓,江都人。國子監生。曙好學根性,家貧,讀四子書未畢,即去鄉,雜作傭保,而績學不倦。年二十為童子師,問所當治業於涇包世臣,世臣曰:「治經必守家法,專法一家,以立其基,則諸家漸通。」乃示以武進張惠言所輯四子書漢說數十事。曙乃稽典禮、考古訓,為四書典故覈六卷,歙洪梧甚稱之。既,治鄭氏學,得要領;又從吳沈欽韓問疑義,益貫穿精審。後聞武進劉逢祿論何氏公羊春秋而好之。及入都,為儀徵阮元校輯經郛,盡見魏、晉以來諸家春秋說。深念春秋之義,存於公羊,而公羊之學,傳自董子。董子春秋繁露,識禮義之宗,達經權之用。行仁為本,正名為先。測陰陽五行之變,明制禮作樂之原。體大思精,推見至隱,可謂善發微言大義者。然旨奧詞賾,未易得其會通,淺嘗之夫,橫生訾議,經心聖符,不絕如線。乃博稽旁討,承意儀志,梳其章,櫛其句,為注十七卷。又病宋、元以來學者空言無補,惟實事求是,庶幾近之,而事之切實無過於禮,著公羊禮疏十一卷,公羊禮說一卷,公羊問答二卷。家居讀禮,以喪服為人倫大經,後儒舛議,是非頗謬,作禮論百篇,引申鄭義。阮元延曙入粵課諸子,曙書與元商榷,乃刪合三十九篇為一卷。道光九年,卒,年五十五。

曙有甥儀徵劉文淇,貧而穎悟,愛而課之,遂知名,其學實自曙出雲。

薛傳均,字子韻,甘泉人。諸生。博覽群籍,強記精識。就福建學政陳用光聘,用光見所著書,恨相見晚。旋以疾卒於汀州試院,年四十一。傳均於十三經注疏功力最深,大端尤在小學,於許君原書,鉤稽貫串,洞其義而熟其辭,嘉定錢大昕文集內有說文答問一卷,深明通轉假借之義,傳均博引經史以證之,成說文答問疏證六卷。又以文選中多古字,條舉件系,疏通證明,為文選古字通十二卷。

劉逢祿,字申受,武進人。祖綸,大學士,謚文定,自有傳。外王父莊存與、舅莊述祖,並以經術名世,逢祿盡傳其學。嘉慶十九年進士,選翰林院庶吉士,散館改禮部主事。二十五年,仁宗大事,逢祿搜集大禮,創為長編,自始事至奉安山陵,典章具備。道光三年,通政司參議盧浙請以尚書湯斌從祀文廟,議者以斌康熙中在上書房獲譴,乾隆中嘗奉駮難之。逢祿攬筆書曰:「後夔典樂,猶有硃、均;呂望陳書,難匡管、蔡。」尚書汪廷珍善而用之,遂奉俞旨。四年,補儀制司主事。越南貢使陳請為其國王母乞人葠,得旨賞給。而諭中有「外夷貢道」之語,其使臣欲請改為「外籓」,部中以詔書難更易。逢祿草牒復之曰:「周官職方王畿之外分九服。夷服去王國七千里,籓服九千里,是籓遠而夷近。說文羌、狄、蠻、貊字皆從物旁,惟夷從大、從弓。考東方大人之國夷,俗仁,仁者壽,有東方不死之國,故孔子欲居之。乾隆間奉上諭申飭四庫館不得改書籍中『夷』字作『彞』,舜東夷之人,文王西夷之人,我朝六合一家,盡去漢、唐以來拘忌嫌疑之陋,使者無得以此為疑。」越南使者遂無辭而退。逢祿在禮部十二年,恆以經義決疑事,為眾所欽服類如此。

其為學務通大義,不專章句。由董生春秋闚六藝家法,由六藝求觀聖人之志。嘗謂:「世之言經者,於先漢則古詩毛氏,後漢則今易虞氏,文詞稍為完具。然毛公詳古訓而略微言,虞翻精象變而罕大義,求其知類通達、微顯闡幽者,則公羊在先漢有董生、後漢有何劭公氏、子夏喪服傳有鄭康成氏而已。先漢之學,務乎大體,故董生所傳非章句訓詁之學也。後漢條理精密,要以何劭公、鄭康成氏為宗,然喪服於五禮特其一端。春秋文成數萬,其旨數千,天道浹,人事備,以之貫群經,無往不得其原;以之斷史,可以決天下之疑;以之持身治世,則先王之道可復也。」於是尋其餘貫,正其統紀,為公羊春秋何氏釋例三十篇,又析其疑滯,強其守衛,為箋一卷,答難二卷。又推原穀梁氏、左氏之得失,為申何難鄭四卷。又博徵諸史刑、禮之不中者為儀禮決獄四卷。又推其意為論語述何、夏時經傳箋、中庸崇禮論、漢紀述例各一卷。別有緯略二卷,春秋賞罰格一卷。愍時學者說春秋皆襲宋儒「直書其事、不煩褒貶」之辭,獨孔廣森為公羊通義能抉其蔽,然尚不能信三科、九旨為微言大義所在,乃著春秋論上、下篇以張聖權。又成左氏春秋考證二卷,知者謂與閻、惠之辯古文尚書等。

逢祿於易主虞氏,於書匡馬、鄭、於詩初尚毛學,後好三家。有易虞氏變動表、六爻發揮旁通表、卦象陰陽大義、虞氏易言補各一卷。又為易象賦、卦氣頌,提其指要。尚書今古文集解三十卷,書序述聞一卷,詩聲衍二十七卷。所為詩、賦、連珠、論、序、碑、記之文約五十篇。道光九年,卒,年五十有六。弟子潘準、莊繽樹、趙振祈皆從學公羊及禮有名。

宋翔鳳,字於庭,長洲人。嘉慶五年舉人,官湖南新寧縣知縣,亦莊述祖之甥。述祖有「劉甥可師、宋甥可友」之語,劉謂逢祿,宋謂翔鳳也。翔鳳通訓詁名物,志在西漢家法,微言大義,得莊氏之真傳。著論語說義十卷,序曰:「論語說曰,子夏六十四人共撰仲尼微言,以當素王。微言者,性與天道之言也。此二十篇,尋其條理,求其恉趣,而太平之治、素王之業備焉。自漢以來,諸家之說,時合時離,不能畫一。嘗綜覈古今,有纂言之作。其文繁多,因別錄私說,題為說義。」又有論語鄭注十卷,大學古義說二卷,孟子趙注補正六卷,孟子劉熙注一卷,四書釋地辨證二卷,卦氣解一卷,尚書說一卷,尚書譜一卷,爾雅釋服一卷,小爾雅訓纂六卷,五經要義一卷,五經通義一卷,過庭錄十六卷。咸豐九年,重賦鹿鳴。逾年,卒,年八十二。

戴望,字子高,德清人。諸生。始好詞章,繼讀博野顏元書,為顏氏學。最後謁長洲陳奐,通聲音訓詁。復從翔鳳授公羊春秋,遂通公羊之學。著論語注二十卷,用公羊家法演逢祿論語述何之微言。他著有管子校注二十四卷,顏氏學記十卷,謫麟堂遺集四卷。

雷學淇,字瞻叔,順天通州人。父鐏,字宗彞,乾隆二十七年舉人,選江西崇仁縣知縣。道光初元,詔天下臣民嚴冠服之辨,鐏著古今服緯以申古義,抑奢侈。至九年書成,年九十矣。

學淇,嘉慶十九年進士,任山西和順縣知縣,改貴州永從縣知縣。生平好討論之學,每得一解,必求其會通,務於諸經之文無所牴牾。以父鐏著古今服緯,為之注釋,附以釋問一篇、異同表二篇。又以夏小正一書備三統之義,究心參考二十餘年。以堯典中星、諸經歷數,採虞史伯夷之說,據周公垂統之文,檢校異同,訂其譌誤,網羅放失,尋厥指歸,著夏小正經傳考二卷。又考定經、傳之文,為之疏證,成夏小正本義四卷。

每慨竹書紀年自五代以來頗多殘闕,爰博考李唐以前諸書所稱引者,積以九年之蒐輯,頗復舊觀。嘗謂:「孟子先至梁後至齊,此經之明文,即無他左驗,亦當從之為說。況竹書紀年曰『梁惠成王後元十五年齊威王薨』,『十七年惠成王卒』,然則惠王後元十六年齊宣王始即位,孟子至梁,當即在後元十六年王卒之前一歲也。史記誤謂惠王立三十六年即卒,故云三十五年孟子至梁,而以惠王改元之後十六年為襄王之世。今據竹書稱梁惠會諸侯於徐州,改元稱王,故孟子呼之曰王。史謂孟子至梁之二年惠王卒,襄王立,以本經考之,其言可信。但卒於改元後之十七年,非三十六年也。襄王既立,孟子見其不似人君,乃東至齊,據竹書即齊宣即位之二年也。梁至齊千數百里,故曰:『千里而見王』。若孟子先見齊宣王,由鄒之齊六百餘里,不得雲千里矣。齊人取燕,孟子明謂宣王時事,史記於齊失載悼子、侯剡二代,將威、宣之立,皆移前二十二年。於齊人伐燕事,不知折衷孟子,而年表謂在湣王十年,司馬溫公終求其說而不得,乃將宣之即位下移十年,以遷就孟子。自後說者疑信各半,實皆未有定論。今據紀年,則伐燕在宣王七年,實周赧王之元年。凡孟子書所記古人年歲,以史記、漢書之說推之皆不合者,以紀年推之無不合。」且以竹書長歷推驗列宿之歲差,歷代之日蝕,自唐、虞以來,無有差貸。嘗自云:「傳、箋、注、疏取舍多殊,非敢訾議前賢,期於事理之合云爾。」他著有校輯世本二卷,古今天象考十二卷,附圖說二卷,亦囂囂齋經義考及文集三十二卷。

王萱齡,字北堂,昌平人。道光元年副貢,旋舉孝廉方正,官新安、柏鄉兩縣教諭。嗜漢學,精訓詁,受業於高郵王引之,經義述聞中時引其說。著有周秦名字解詁補一卷,即補引之所闕疑者。

崔述,字武承,大名人。乾隆二十七年舉人,選福建羅源縣知縣。武弁多藉海盜邀功,誣商船為盜,述平反之。未幾,投效歸。著書三十餘種,而考信錄一書,尤生平心力所專注。凡考古提要二卷,上古考信錄二卷,唐虞考信錄四卷,夏商考信錄四卷,豐鎬考信錄八卷,豐鎬別錄三卷,洙泗考信錄四卷,洙泗餘錄三卷,孟子事實錄二卷,考古續說二卷,附錄二卷。又有王政三大典考三卷,讀風偶識四卷,尚書辨偽二卷,論語餘說一卷,讀經餘論二卷,名考古異錄。

其著書大旨,謂不以傳注雜於經,不以諸子百家雜於傳注。以經為主,傳注之合於經者著之,不合者辨之,異說不經之言,則闢其謬而削之。如謂易傳僅溯至伏羲,春秋傳僅溯至黃帝,不應後人所知反多於古人。凡緯書所言十紀,史所云天皇、地皇、人皇,皆妄也。謂戰國楊、墨橫議,常非堯、舜,薄湯、武,以快其私。毀堯則託諸許由,毀禹則託諸子高,毀孔子則託諸老聃,毀武王則託諸伯夷。太史公尊黃、老,故好採異端雜說,學者但當信論、孟,不當信史記。謂夏、商、周未有號為某公者,公亶父相連成文,猶所謂公劉也。「古公亶父」,猶言「昔公亶父」也。謂匡為宋邑,似畏匡、過宋本一事,「匡人其如予何」、「桓魋其如予何」,似一時一事之言,記者小異耳。其說皆為有見。

述之為學,考據詳明如漢儒,而未嘗墨守舊說而不求其心之安;辨析精微如宋儒,而未嘗空談虛理而不核乎事之實。然勇於自信,任意軒輊者亦多。他著有易卦圖說一卷,五服異同匯考三卷,大名水道考一卷,聞見雜記四卷,知味錄二卷,知非集三卷,無聞集五卷,小草集五卷。嘉慶二十一年,卒。年七十七。

胡培翬,字載平,績溪人。祖匡衷,字樸蘇,歲貢生。於經義多所發明,不茍與先儒同異。著有三禮劄記、周禮井田圖考、井田出賦考、儀禮釋官等書。其於井田多申鄭義,而授田一事,以遂人所言是鄉遂制,大司徒所言是都鄙制,鄭注自相違戾。作畿內授田考實一篇,積算特精密。其釋官則以周禮、禮記、左傳、國語與儀禮相參證,論據精確,足補注疏所未及。又著有周易傳義疑參十二卷,左傳翼服、論語古本證異、論語補箋、莊子集評、離騷集注、樸齋文集。年七十四,卒。

培翬,嘉慶二十四年進士,官內閣中書、戶部廣東司主事。居官勤而處事密,時人稱其治官如治經,一字不肯放過。絕不受財賄,而抉隱指弊,胥吏咸憚之。假照案發,司員失察者數十人,惟培翬及蔡紹江無所污,然猶以隨同畫諾鐫級歸里。後主講鍾山、雲間,於涇川一再至,並引翼後進為己任。去涇川日,門人設飲餞者相望於道。篤友誼,郝懿行、胡承珙遺書,皆賴培翬次第付梓。道光二十九年,卒,年六十八。

績溪胡氏,自明諸生東峰以來,世傳經學。培翬涵濡先澤,又學於歙凌廷堪,邃精三禮。初著燕寢考三卷,王引之見而喜之。既為儀禮正義,上推周公、孔子、子夏垂教之旨,發明鄭君、賈氏得失,旁逮鴻儒、經生之所議。張皇幽渺,闡揚聖緒,二千餘歲絕學也。其旨見與順德羅惇衍書曰:「培翬撰正義,約有四例:一曰疏經以補注,二曰通疏以申注,三曰匯各家之說以附注,四曰採他說以訂注,書凡四十卷,至賈氏公彥之疏,或解經而違經旨,或申注而失注意,不可無辨。別為儀禮賈疏訂疑一書。宮室制度,今以朝制、廟制、寢制為綱,以天子、諸侯、大夫、士為目。學制則分別庠、序館制則分別公、私,皆先將宮室考定,而以十七篇所行之禮,條系於後,名宮室提綱。陸氏經典釋文於儀禮頗略,擬取各經音義及集釋文以後各家音切,挨次補錄,名曰儀禮釋文校補。」培翬覃精是書凡四十餘年,晚歲患風痺,猶力疾從事。尚有士昏禮、鄉飲酒禮、鄉射禮、燕禮、大射儀五篇未卒業而歿。門人江寧楊大堉從學禮,為補成之。他著有禘祫問答,研六室文鈔。

大堉,字雅輪。諸生。篤學寡交,研窮經訓。初從元和顧廣圻、吳縣鈕樹玉游,備聞蒼、雅閫奧。著說文重文考六卷,純以聲音求叚借,以偏旁繁省求古、籀異同之變。又作五廟考,專駮王肅之失。江督陶澍以防海議試諸生,大堉洋洋千言,大略謂:「中國官恃客氣,居上臨下,視洋人若小負販。顧彼雖好利,而越數萬里海洋至此,此必非無所挾持者。鹵莽行之,必生邊隙。」時承平久,人習附和之談,獨大堉卓識正論,侃然無忌諱。若豫卜有義律、璞鼎查之事,讀者色變。他著論語正義、毛詩補注、三禮義疏辨正,皆佚。

劉文淇,字孟瞻,儀徵人,嘉慶二十四年優貢生。父錫瑜,以醫名世。文淇稍長,即研精古籍,貫串群經。於毛、鄭、賈、孔之書及宋、元以來通經解誼,博覽冥搜,折衷一是。尤肆力春秋左氏傳,嘗謂左氏之義,為杜注剝蝕巳久,其稍可觀覽者,皆系襲取舊說。爰輯左傳舊注疏證一書,先取賈、服、鄭三君之注,疏通證明。凡杜氏所排擊者糾正之,所剿襲者表明之。其沿用韋氏國語注者,亦一一疏記。他如五經異義所載左氏說,皆本左氏先師;說文所引左傳,亦是古文家說;漢書五行志所載劉子駿說。實左氏一家之學;經疏、史注、御覽等書所引左傳注不載姓名而與杜注異者,皆賈、服舊說。凡若此者,皆稱為舊注,而加以疏證。其顧、惠補注及近人專釋左氏之書,說有可採,咸與登列。末始下以己意,定其從違。上稽先秦諸子,下考唐以前史書,旁及雜家筆記、文集,皆取為證佐。期於實事求是,俾左氏之大義炳然著明。草創四十年,長編已具,然後依次排比成書,為左氏舊注疏證。又謂:「左傳義疏多襲劉光伯述議,隋經籍志及孝經疏,云述議者,述其義,疏議之。然則光伯本載舊疏,議其得失,其引舊疏,必當錄其姓名。孔穎達左傳疏序祗云據以為本,初非故襲其說。至永徽中諸臣詳定,乃將舊注姓氏削去,襲為己語。」因細加剖析,成左傳舊疏考正八卷。

又據史記秦楚之際月表,知項羽曾都江都。核其時勢,推見割據之述,成楚漢諸侯疆域志三卷。據左傳、吳越春秋、水經注等書,謂唐、宋以前揚州地勢南高北下,且東西兩岸未設堤防,與今運河形勢迥不相同,成揚州水道記四卷。又讀書隨筆二十卷,文集十卷,詩一卷。

文淇事親純孝,父年篤老,目眚,侍起居,朝夕扶掖,寒夜足凍,侍親以溫其足。舅氏凌曙極貧,遺孤毓瑞,文淇收育之。延同里方申為其師,並補諸生。申通虞氏易,皆其教也。卒,年六十有六。

子毓崧,字伯山。道光二十年舉優貢生。從父受經,長益致力於學。以文淇故,治左氏纘述先業,成春秋左氏傳大義二卷。以文淇考證左傳舊疏,因承其義例,著周易、尚書、毛詩、禮記舊疏考正各一卷。又謂六藝未興之先,學各有官,惟史官之立為最古。不獨史家各體各類並支裔之小說家出於史官,即經、子、集三部及後世之幕客書吏,淵源所仿,亦出於史官。班氏之志藝文,論述史官,尚未發斯旨。其敘九流,以明諸子所出之官,必有所授,而其中仍有分省失當者。既析九流中小說家流歸入史官,又辨道家非專出於史官,改為出於醫官。又增益者凡三家:曰名家,出於司士之官;兵家,出於司馬之官;藝術家,出於考工之官:統為十一家。博稽載籍,窮極根要,成史乘、諸子通義各四卷。又經傳通義十卷,王船山年譜二卷,彭城獻徵錄十卷,舊德錄一卷,通義堂筆記十六卷,文集十六卷,詩集一卷。卒,年五十。

孫壽曾,字恭甫。同治三年、光緒二年兩中副榜。毓崧主金陵書局,為曾國籓所重。毓崧卒後,招壽曾入局中,所刊群籍,多為校定。初,文淇治左氏春秋長編,晚年編輯成疏,甫得一卷,而文淇沒。毓崧思卒其業,未果。壽曾乃發憤以繼志述事為任,嚴立課程,至襄公四年而卒,年四十五。又讀左劄記,春秋五十凡例表,皆治左疏時旁推交通發明古誼者。他著昏禮別論對駮義,南史校義集評,傳雅堂集,芝云雜記,各若干卷。

方申,字端齋。少孤,受學於文淇,通易,著諸家易象別錄、虞氏易象匯編、周易卦象集證、周易互體詳述、周易卦變舉要。

丁晏,字柘堂,江蘇山陽人。阮元為漕督,以漢易十五家發策,晏條對萬餘言,精奧為當世冠。道光元年舉人。晏以顧炎武云梅賾偽古文雅密非賾所能為,考之家語後序及釋文、正義,而斷為王肅偽作。蓋肅雅才博學,好作偽以難鄭君。鄭君之學昌明於漢,肅為古文孔傳以駕其上,後儒誤信之。近世惠棟、王鳴盛頗疑肅作而未能暢其旨,特著論申辨之,撰尚書餘論二卷。又以胡渭禹貢錐指能知偽古文,而不能信好古學,踵謬沿譌,自逞臆見。後之學者,何所取正?既為正誤以匡其失,復採獲古文,甄錄舊說,砭俗訂譌,斷以己意。期於發揮經文,無取泥古。引用前人說,各系姓氏於下,輯禹貢集釋三卷。

生平篤好鄭學,於詩箋、禮注研討尤深。以毛公之學,得聖賢之正傳,其所稱道,與周、秦諸子相出入。康成申暢毛義,修敬作箋。孔疏不能尋繹,誤謂破字改毛。援引疏漏,多失鄭旨。因博稽互考,證之故書雅記,義若合符,撰毛鄭詩釋四卷。康成詩譜,宋歐陽氏補亡,今通志堂刊本譌脫踳駮。爰據正義排比重編,撰鄭氏詩譜考正一卷。以康成兼採三家詩,王應麟有三家詩考,附刊玉海之後,舛謬錯出,世無善本。乃蒐採原書,校讎是正,撰詩考補注二卷,補遺一卷。

鄭氏注禮至精,去古未遠,不為憑虛臆說。迄今可考見者,如儀禮喪服注,多依馬融師說。士虞記中月而禫,注二十七月,依戴禮喪服變除。周禮大司樂鼓,注依許叔重說,與先鄭不同。小胥縣鐘磬,注二八十六枚在一虡,依劉向五經要義。小宗伯注五精帝,依劉向五經通義。射人注稱今儒家,依賈侍中注。考工記山以章,注作麞,依馬季長注。禮記檀弓瓦不成味,注當作沫,依班固白虎通。王制大綏小綏,注當作緌,依劉子政說苑。玉藻元端朝日,鄭讀為冕,依大戴禮朝事義。祭法幽宗雩祭,鄭讀為禜,依許氏說文。鄭君信而好古,原本先儒,確有依據。凡此釋義,補孔之遺闕,皆前人未發之秘。疏通證明,若爟火。撰三禮釋注共八卷,又輯鄭康成年譜,署其堂曰「六藝」,取康成六藝論,以深仰止之思。然晏治經學不掊擊宋儒,嘗謂漢學、宋學之分,門戶之見也。漢儒正其詁,詁正而義以顯;宋儒析其理,理明而詁以精:二者不可偏廢。其於易,述程子之傳,撰周易述傳二卷;於孝經,集唐玄宗、宋司馬光、範祖禹之注,撰孝經述注一卷。

尤熟於通鑒,故經世優裕。嘗與人論鈔弊,謂輕錢行鈔,必有利而無害。論禁洋煙,謂不禁則民日以弱,中國必疲,禁則利在所爭,外夷必畔。且禁煙當以民命為重,不當計利。立法當以中國為先,不當擾夷。後悉如其言。在籍時辦堤工,司賑務,修府城,浚市河,開通文渠中支,均有功於鄉里。

咸豐三年,粵匪蔓延大江南、北,督撫檄行府縣,練勇積穀為守禦計。淮安以晏主其事,旋以事為人所劾,奉旨遣戌黑龍江,繳費免行。十年,捻匪擾淮安北關,晏號召團練,分布要隘,城以獲全。十一年,以團練大臣晏端書薦,敘前守城績,由侍讀銜內閣中書加三品銜。

晏少多疾病,迨長讀書養氣,日益強固。治一書畢,方治他書,手校書籍極多,必徹終始。光緒元年,卒,年八十有二。所著書四十七種,凡一百三十六卷,其已刊者為頤志齋叢書。

王筠,字貫山,安丘人。道光元年舉人,後官山西鄉寧縣知縣。鄉寧在萬山中,民樸事簡,訟至立判。暇則抱一編不去手。權徐溝,再權曲沃,地號繁劇,二縣皆治,然亦未嘗廢學。

筠少喜篆籀,及長,博涉經史,尤長於說文。說文之學,世推桂、段兩家,嘗謂:「桂氏專臚古籍,取足達許說而止,不下己意。惟是引據失於限斷,且泛及藻繢之詞。段氏體大思精,所謂通例,又前人所未知。惟是武斷支離,時或不免。」又謂:「文字之奧,無過形、音、義三端。古人之造字也,正名百物,以義為本,而音從之,於是乎有形。後人之識字也,由形以求其音,由音以考其義,而文字之說備。六書以指事、象形為首,而文字之樞機即在乎此。其字之為事,而作者即據事以審字,勿由字以生事。其字之為物,而作者即據物以察字,勿泥字以造物。且勿假他事以成此事之意,勿假他物以為此物之形,而後可與蒼頡、籀、斯相質於一堂也。今說文之詞,足從口,木從屮,鳥、鹿足相似從匕,茍非後人所竄亂,則許君之意荒矣。」乃標舉分別,疏通證明,著說文釋例二十卷。釋例云者,即許書而釋其條例,猶杜元凱之於春秋也。又以二徐書多涉草略,加以李燾亂其次第,致分別部居之脈絡不可推尋。段玉裁既創為通例,而體裁所拘,未能詳備。乃採桂、段諸家之說,著說文句讀三十卷。句讀雲者,用張爾岐儀禮鄭注句讀之名,謂漢人經說率名章句,此書疏解許說,無章可言,故曰句讀也。

筠治說文之學垂三十年,其書獨闢門徑,折衷一是,不依傍於人。論者以為許氏之功臣,桂、段之勁敵。又有說文系傳校錄三十卷,文字蒙求四卷。他著有毛詩重言一卷,附毛詩雙聲疊韻說一卷,夏小正正義四卷,弟子職正音一卷,正字略二卷,蛾術編、禹貢正字、讀儀禮鄭注句讀刊誤、四書說略。咸豐四年,卒,年七十一

曾釗,字敏修,南海人。道光五年拔貢生,官合浦縣教諭,調欽州學正。釗篤學好古,讀一書必校勘譌字脫文。遇秘本或雇人影寫,或懷餅就鈔,積七八年,得數萬卷。自是研求經義,文字則考之說文、玉篇,訓詁則稽之方言、爾雅,雖奧晦難通,而因文得義,因義得音,類能以經解經,確有依據。入都時,見武進劉逢祿,逢祿曰:「篤學若冕士,吾道東矣!」冕士。釗號也。

儀徵阮元督粵,震澤任兆麟見釗所校字林,以告元,元驚異,延請課子。後開學海堂,以古學造士,特命釗為學長,獎勸後進。嘗因元說日月為易為合朔之辨在朔易,更發明孟喜卦氣,引系辭懸象莫大乎日月,死魄會於壬癸,日上月下,象未濟為晦時。元以為足發古義,宜再暢言之,以明孟氏之學,因著周易虞氏義箋七卷。他著有周禮注疏小箋四卷,又詩說二卷,又詩毛鄭異同辨一卷,毛詩經文定本小序一卷、考異一卷、音讀一卷,虞書命羲和章解一卷,論語述解一卷,讀書雜志五卷,面城樓集十卷。

釗好講經濟之學,二十一年,英人焚掠海疆,以祁還督兩粵,番禺舉人陸殿邦獻議,填大石、獵德、瀝河道以阻火船。舉以問釗,釗言:「易稱設險者,不恃天塹,不藉地利,在人相時設之而已。入省河道三,獵德、瀝皆淺,由大石至大黃,水深數丈。三四月夷船從此入,當先事防之,以固省城。城固,然後由內達外。」甚韙之,委釗相度堵塞形勢,釗以大石為第一要區,糾南海、番禺二縣團勇三萬六千晝夜演練,防務遂密。二十三年,謀修復虎門砲臺,釗進砲臺形勢議十條,已而廉洋賊起,以釗習知廉州情形,委釗與軍事。海賊投首。咸豐四年,卒於家。

林伯桐,字桐君,番禺人。嘉慶六年舉人。生平好為考據之學,宗主漢儒,而踐履則服膺硃子,無門戶之見。事親孝,道光六年,試禮部歸,父已卒,悲慟不欲生。居喪悉遵古禮,蔬食、不入內者三年。自是不復上公車,一意奉母。與兩弟友愛,教授生徒百餘人,咸敦內行,勉實學。粵督阮元、鄧廷楨皆敬禮之。元延為學海堂學長,廷楨聘課其二子。二十四年,以選授德慶州學正,閱三年卒於官,年七十。

伯桐於諸經無不通,尤深於毛詩。謂傳箋不同者,大抵毛義為長,孔疏多以王肅語為毛意,又往往混鄭於毛。為毛詩學者,當分別觀之,庶幾不失家法。因考鄭箋異義,為毛詩通考三十卷,又著毛詩傳例二卷,又綴其碎義瑣辭,著毛詩識小三十卷,皆極精覈。他著有易象釋例十二卷,易象雅訓十二卷,三禮注疏考異二十卷,冠昏喪祭儀考十二卷,左傳風俗二十卷,古音勸學三十卷,史學蠡測三十卷,供冀小言二卷,古諺箋十一卷,兩粵水經注四卷,粵風四卷,修本堂槁四卷,詩文集二十四卷。

李黼平,字繡子,嘉應州人。幼穎異。年十四,精通樂譜。及長,治漢學,工考證。嘉慶十年進士,選翰林院庶吉士,散館改昭文縣知縣。事一以寬和慈惠為宗,不忍用鞭撲,獄隨至隨結。公餘即手一編,民間因有「李十五書生」之目。以虧挪落職系獄,數年乃得歸。會粵督阮元開學海堂,聘閱課藝,遂留授諸子經。所著毛詩紬義二十四卷。道光十二年,卒,年六十三。他著有易刊誤二卷,文選異義二卷,讀杜韓筆記二卷。

柳興恩,原名興宗,字賓叔,丹徒人。道光十二年舉人。受業於儀徵阮元。初治毛詩,以毛公師荀卿,荀卿師穀梁,穀梁春秋千古絕學,元刻皇清經解,公羊、左氏俱有專家,而穀梁缺焉。乃發憤沉思,成穀梁春秋大義述三十卷,以鄭六藝論云「穀梁子善於經」,遂專從善經入手,而善經則以屬辭比事為據,事與辭則以春秋日月等名例定之。其書凡例,謂聖經既以春秋定名,而無事猶必舉四時之首月。後儒謂日月非經之大例,未為通論。穀梁日月之例,泥則難通,比則易見。與其議傳而轉謂經誤,不若信經而並存傳說。述日月例第一。謂春秋治亂於已然,禮乃防亂於未然。穀梁親受子夏,其中典禮猶與論語夏時周冕相表裏。述禮例弟二。謂穀梁之經與左氏、公羊異者以百數,漢書儒林傳云:「穀梁魯學,公羊乃齊學也。」此或由齊、魯異讀,音轉而字亦分。述異文弟三。謂穀梁親受子夏,故傳中用孔子、孟子說,其他暗合者更多。述古訓弟四。謂自漢以來,穀梁師授鮮有專家,要不得擯諸師說之外。述師說弟五。謂漢儒師說之可見者,惟尹更始、劉向二家,然搜獲寥寥。其說已亡,而名僅存者,自漢以後並治三傳者亦收錄焉。述經師弟六。謂穀梁久屬孤經,茲於所見載籍之涉穀梁者,循次摘錄,附以論斷,並著本經廢興源流。述長編弟七。番禺陳澧嘗為穀梁箋及條例,未成,後見興恩書,嘆其精博,遂出其說備採,不復作。

他著有周易卦氣輔四卷,虞氏逸象考二卷,尚書篇目考二卷,毛詩注疏糾補三十卷,續王應麟詩地考二卷,群經異義四卷,劉向年譜二卷,儀禮釋宮考辨二卷,史記、漢書、南齊書校勘記,說文解字校勘記,宿壹齋詩文集。光緒六年,卒,年八十有六。

弟榮宗,字翼南。著有說文引經考異十六卷。同時為穀梁之學者,有南海侯康、海州許桂林、嘉善鍾文烝、江都梅毓。侯康自有傳。

許桂林、字同叔,海州人。嘉慶二十一年舉人。少孤,孝於母及生母,無間言。家貧,不以厚幣易遠游,日以詁經為事。道光元年,丁內艱,以毀卒,年四十三。桂林於諸經皆有發明,尤篤信穀梁之學,著春秋穀梁穀傳時日月書法釋例四卷。其書有引公羊而互證者,有駮公羊而專主者。陽湖孫星衍嘗以條理精密、論辨明允許之。又著易確二十卷,大旨以乾為主,謂全易皆乾所生,博觀約取,於易義實有發明,別有毛詩後箋八卷,春秋三傳地名考證六卷,漢世別本禮記長義四卷,大學中庸講義二卷,四書因論二卷。嘗以其餘力治六書、九數,著許氏說音十二卷,以配說文。又著說文後解十卷。又以岐伯言「地,大氣舉之」。氣外無殼,其氣將散;氣外有殼,此殼何依?思得一說以補所未及。蓋天實一氣,而其根在北,北極是也。北極不當為天樞,而當為氣母。因採集宣夜遺文,以西法通之,著宣西通三卷。又以算家以簡為貴,乃取欽定數理精蘊,撮其切於日用者,著算牖四卷。生平所著書四十餘種,凡百數十卷。甘泉羅士琳從之游,後以西算名世。

鍾文烝,字子勤,嘉善人。道光二十六年舉人,候選知縣。於學無所不通,而其全力尤在春秋。因沉潛反覆三十餘年,成穀梁經傳補注二十四卷。其書網羅眾家,折衷一是。其未經人道者,自比於梅鷟之辨偽書、陳第之談古韻,略引其緒,以待後賢。文烝兼究宋、元諸儒書,書中若釋禘祫、祖禰謚法以及心志不通、仁不勝道、以道受命等,皆能提要挈綱,實事求是。又著論語序詳正一卷。卒,年六十。

梅毓,字延祖,江都人。同治九年舉人,候選教諭。著有穀梁正義長編一卷。

陳澧,字蘭甫,番禺人。道光十二年舉人,河源縣訓導。澧九歲能文,復問詩學於張維屏,問經學於侯康。凡天文、地理、樂律、算術、篆隸無不研究。中年讀諸經注疏、子、史及硃子書,日有課程。初著聲律通考十卷,謂:「周禮六律、六同皆文之以五聲,禮記五聲、六律、十二管還相為宮,今之俗樂有七聲而無十二律,有七調而無十二宮,有工尺字譜而不知宮、商、角、徵、羽。懼古樂之遂絕,乃考古今聲律為一書。」又切韻考六卷、外篇三卷,謂:「孫叔然、陸法言之學存於廣韻,宜明其法,而不惑於沙門之說。」又漢志水道圖說七卷,謂地理之學,當自水道始,知漢水道則可考漢郡縣。

其於漢學、宋學能會其通,謂:「漢儒言義理,無異於宋儒,宋儒輕蔑漢儒者非也。近儒尊漢儒而不講義理,亦非也。」著漢儒通義七卷。晚年尋求大義及經學源流正變得失所在而論贊之,外及九流諸子、兩漢以後學術,為東塾讀書記二十一卷。

其教人不自立說,嘗取顧炎武論學之語而申之,謂:「博學於文,當先習一藝。韓詩外傳曰『好一則博』,多好則雜也,非博也。讀經、史、子、集四部書,皆學也,而當以經為主,尤當以行己有恥為主。」為學海堂學長數十年。至老,主講菊坡精舍,與諸生講論文藝,勉以篤行立品,成就甚眾。光緒七年,粵督張樹聲、巡撫裕寬以南海硃次琦與澧皆耆年碩德,奏請褒異,給五品卿銜。八年,卒,年七十三。

他著有說文聲表十七卷,水經注提綱四十卷,水經注西南諸水考三卷,三統術詳說三卷,弧三角平視法一卷,琴律譜一卷,申範一卷,摹印述一卷,東塾集六卷。

侯康,字君謨,亦番禺人。道光十五年舉人。少孤,事母孝。家貧,欲買書,母稱貸得錢。買十七史,讀之,卷帙皆敝,遂通史學。及長,精研注疏,湛深經術,與同里陳灃交最久。嘗謂:「漢志載春秋古經十二篇者左經也,經十一卷者公、穀經也。今以三傳參校之,大要古經為優。穀梁出最先,其誤尚寡。公羊出最晚,其誤滋甚。」乃取其義意可尋者疏通證明之,著春秋古經說二卷。又治穀梁以證三禮,以公羊雜出眾師,時多偏駮,排詆獨多。著穀梁禮證,未完帙,僅成二卷。又仿裴松之注三國志例注史,嘗曰:「注古史與近史異,注近史者,群書大備;注古史者,遺籍罕存。當日為唾棄之餘,今日皆見聞之助,宜過而存之。」因為後漢書補注續一卷,三國志補注一卷,後漢稱續者,以有惠棟注;三國志杭世駿注未完善,故不稱續也。又補後漢、三國藝文志,各成經、史、子四卷,餘未成。又考漢、魏、六朝禮儀,貫串三禮,著書數十篇,澧嘗嘆以為精深浩博。十七年,卒,年四十。

弟度,字子琴。與康同榜舉人,以大挑知縣分發廣西,署河池州知州。廣西賊起,度伐木為柵,因山勢聯絡,堅固可守。賊退,以病告歸,至家遂卒,年五十七。度洽熟經傳,尤長禮學,時稱「二侯」。嘉興錢儀吉嘗稱其研覈傳注,剖析異同,如辨懿伯、惠伯之為父子,三老、五更之為一人。證明鄭義,皆有據依。所著書為夷寇所焚,其說經文,刻學海堂集中。

桂文燦,字子白,文燿之弟。道光二十九年舉人。同治二年正月,應詔陳言:曰嚴甄別以清仕途,曰設幕職以重考成,曰分三途以勵科甲,曰裁孱弱以節糜費,曰鑄銀錢以資利用。若津貼京員,制造輪船,海運滇銅,先後允行。光緒九年,選湖北鄖縣知縣,善治獄,以積勞卒於任。文燦守阮元遺言,謂:「周公尚文,範之以禮;尼山論道,教之以孝。茍博文而不能約禮,明辨而不能篤行,非聖人之學也。鄭君、硃子皆大儒,其行同,其學亦同。」因著硃子述鄭錄二卷。他著四書集注箋四卷,毛詩釋地六卷,周禮通釋六卷,經學博採錄十二卷。

鄭珍,字子尹,遵義人。道光五年拔貢生。十七年舉人,以大挑二等選荔波縣訓導。咸豐五年,叛苗犯荔波,知縣蔣嘉穀病,珍率兵拒戰,卒完其城。苗退,告歸。同治二年,大學士祁俊藻薦於朝,特旨以知縣分發江蘇補用,卒不出。三年,卒,年五十九。

珍初受知於歙縣程恩澤,乃益進求諸聲音文字之原,與古宮室冠服之制。方是時,海內之士。崇尚考據,珍師承其說,實事求是,不立異,不茍同。復從莫與儔游,益得與聞國朝六七鉅儒宗旨。於經最深三禮,謂:「小學有三:曰形,曰聲,曰義。形則三代文體之正,具在說文。若歷代鐘鼎款識及汗簡、古文四聲韻所收奇字,既不盡可識,亦多偽造,不合六書,不可以為常也。聲則昆山顧氏音學五書,推證古音,信而有徵,昭若發蒙,誠百世不祧之祖。義則凡字書、韻書、訓詁之書,浩如煙海,而欲通經訓,莫詳於段玉裁說文注,邵晉涵、郝懿行爾雅疏及王念孫廣雅疏證。貫串博衍,超越前古,是皆小學全體大用。」

其讀禮經,恆苦乾、嘉以還積漸生弊,號宗高密,又多出新義,未見有勝,說愈繁而事愈蕪。故言三禮,墨守司農,不敢茍有出入。至於諸經,率依古注為多。又以餘力旁通子史,類能提要鉤玄。儀禮十七篇皆有發明,半未脫稿,所成儀禮私箋,僅有士昏、公食、大夫喪服、士喪四篇,凡八卷;而喪服一篇,反覆尋繹,用力尤深。又以周禮考工記輪輿,鄭注精微,自賈疏以來,不得正解,說者日益支蔓,成輪輿私箋三卷。尤長說文之學,所著說文逸字二卷、附錄一卷,說文新附考六卷,皆見稱於時。他著有鳧氏圖說、深衣考、汗簡箋正、說隸等書。又有巢經巢經說、詩鈔、文鈔,明鹿忠節公無欲齋詩注。

鄒漢勛,字叔績,新化人。父文蘇,歲貢生,以古學教授鄉里,闢學舍曰古經堂,與諸生肄士禮其中。其考據典物,力尊漢學,而談心性則宗硃子。漢勛通左氏義,佐伯兄漢紀撰左氏地圖說,又佐仲兄漢潢撰群經百物譜。年十八九,撰六國春秋,於天文推步、方輿沿革、六書九數,靡不研究。同縣鄧顯鶴深異之,與修寶慶府志。又至黔中修貴陽、大定、興義、安順諸郡志。咸豐元年,舉於鄉。訪魏源於高郵,同撰堯典釋天一卷。

會粵賊陷江寧,漢勛以援、堵、守三策上書曾國籓,謂不援江西、堵廣西,湖南亦不能守。國籓用其言,命偕江忠淑率楚勇千人援南昌,圍解,敘勞以知縣用。既,從江忠源於廬州,守大西門,賊為隧道三攻之,城坍數丈,賊將登陴,漢勛擊卻之。堅守三十七日,地雷復發,城陷。漢勛坐城樓上,命酒自酌,持劍大呼殺賊。賊至,與格鬥,手刃數人,力竭死之,年四十九,贈道銜。

所著讀書偶識三十六卷,自言破前人之訓故,必求唐以前之訓故方敢用;違箋傳之事證,必求漢以前之事證方敢從。以漢人去古未遠,諸經注皆有師承,故推闡漢學,不遺餘力。尤深音均之學,初著廣韻表十卷,晚為五均論,說尤精粹,時以江、戴目之。生平於易、詩、禮、春秋、論語、說文、水經皆有撰述,凡二十餘種,合二百餘卷。同治二年,土匪焚其居,熸焉。今存者讀書偶識僅八卷,五均論二卷,顓頊歷考二卷,斅藝齋文三卷、詩一卷,紅崖石刻釋文一卷,南高平物產記二卷。

王崧,字樂山,浪穹人。嘉慶四年進士,授山西武鄉縣知縣。崧學問淹通,儀徵阮元總督云、貴,延崧主修通志,著有說緯六卷。

劉寶楠,字楚楨,寶應人。父履恂,字迪九,乾隆五十一年舉人,國子監典簿,著有秋槎札記。

寶楠生五歲而孤,母氏喬教育以成。始寶楠從父臺拱漢學精深,寶楠請業於臺拱,以學行聞鄉里。為諸生時,與儀徵劉文淇齊名,人稱揚州二劉。道光二十年成進士,授直隸文安縣知縣。文安地稱窪下,堤堰不修,遇伏,秋水盛漲,輒為民害。寶楠周履堤防,詢知疾苦,爰檢舊冊,依例督旗屯及民同修,而旗屯恆怙勢相觀望,寶楠執法不阿,功遂濟。再補元氏,會歲旱,縣西北境蝗,袤延二十餘里。寶楠禱東郊蠟祠,蝗爭投阬井,或抱禾死,歲則大熟。咸豐元年,調三河,值東省兵過境。故事,兵車皆出里下。寶楠謂兵多差重,非民所堪,雇車應差,給以民價,民得不擾。

寶楠在官十六年,衣冠樸素如諸生時。勤於聽訟,官文安日,審結積案千四百餘事,雞初鳴,坐堂皇,兩造具備,當時研鞫。事無鉅細,均如其意結案,悖者照例治罪。凡涉親故族屬訟者,諭以睦★L5,概令解釋。訟獄既簡,吏多去籍歸耕,遠近翕然,著循良稱。咸豐五年,卒,年六十五。

寶楠於經,初治毛氏詩、鄭氏禮,後與劉文淇及江都梅植之、涇包慎言、丹徒柳興恩、句容陳立約各治一經。寶楠發策得論語,病皇、邢疏蕪陋,乃蒐輯漢儒舊說,益以宋人長義,及近世諸家,仿焦循孟子正義例,先為長編,次乃薈萃而折衷之,著論語正義二十四卷。因官事繁,未卒業,命子恭冕續成之。他著有釋穀四卷,於豆、麥、麻三種多補正程氏九穀考之說。漢石例六卷,於碑志體例考證詳博。寶應圖經六卷,勝朝殉揚錄三卷,文安堤工錄六卷。

恭冕,字叔俛。光緒五年舉人。守家學,通經訓,入安徽學政硃蘭幕,為校李貽德春秋賈服注輯述,移補百數十事。後主講湖北經心書院,敦品飭行,崇尚樸學。幼習毛詩,晚年治公羊春秋,發明「新周」之義,闢何劭公之謬說,同時通儒皆韙之。卒,年六十。著有論語正義補,何休論語注訓述,廣經室文鈔。

龍啟瑞,字翰臣,臨桂人。道光二十一年一甲一名進士,授翰林院修撰。二十四年,充廣東鄉試副考官。二十七年,大考翰詹二等七名,以侍講升用。七月,簡湖北學政,著經籍舉要一書,以示學者。又以學政之職有三要:一曰防弊,二曰勵實學,三曰正人心風俗。三十年,丁父憂回籍。咸豐元年六月,廣西巡撫鄒鳴鶴奏辦廣西團練,以啟瑞總其事。二年七月,省城圍解,以守城出力,以侍講學士升用。六年四月,授通政司副使。十一月,簡江西學政。七年三月,遷江西布政使。八年九月,卒於官。

啟瑞切劘經義,尤講求音韻之學,貫穿於顧、江、段、王、孔、張、劉、江諸家之書,而著古韻通說二十卷。以為論古韻者,自顧氏以前失之疏,自段氏以後過於密,江氏酌中,亦未為盡善。陽湖張氏分二十一部,言:「凡言古韻者,分之不嫌密,合之不嫌廣。惟分之密,其合之也脈絡分明,不至因一字而疑各韻可通,亦不至因各韻而疑一字之不可通。」啟瑞服膺是言,故其集古韻也,意主於嚴,而其為通說也,則較之顧氏而尚覺其寬。不拘成說,不執私見,參之古書,以求其是而已。其論本音、論通韻、論轉音,皆確有據依,而以論通說總之,故以名其全書焉。他著有爾雅經注集證三卷,經德堂集十二卷。

苗夔,字仙簏,肅寧人。幼即嗜六書形聲之學,讀許氏說文,若有夙悟。已,又得顧炎武音學五書,慕之彌篤。曰:「吾守此終身矣!」舉道光十一年優貢生,高郵王念孫父子禮先於夔,由是譽望日隆。夔以為許叔重遺書多有為後人妄刪或附益者,乃訂正說文八百餘字,為說文聲訂二卷。顧氏音學所立古音表十部,宏綱已具,然猶病其太密,而戈、麻既雜西音,不應別立一部。於是並耕、清、青、蒸、登於東、冬,並戈、麻於支、齊,定以七部,隱括群經之韻。字以聲從,韻以部分,為說文聲讀表七卷。詩自毛傳、鄭箋而後,主義理者多,主聲均者少,雖有陸元朗詩經音義,亦不能專主古音,然古音時有未盡改者。夔治毛詩,尤精於諧聲之學,嘗以齊、魯、韓三家證毛,而又以許洨長之聲讀參錯其間,採太平戚氏之漢學諧聲、詩經正讀,無錫安氏之均徵,為毛詩均訂十卷。咸豐丁巳五月,卒,年七十有五。

龐大堃,字子方,常熟人。嘉慶二十四年舉人,究心音韻之學,嘗謂顧、江、戴、段、孔、王諸家分部互有出入者,以入聲配隸無準耳。入聲有正紐、反紐,今韻多從正紐,古韻多從反紐,陽奇陰偶,兩兩相配,一從陸氏法言所定為正紐,一從顧、江、戴、王所定為反紐。其轉音之法有五:一正轉,同部者是也,一遞轉,同音者是也;一旁轉,相比及相生者是也;一雙聲,同母者是也。又謂欲明古音,必先究唐韻,乃可定其分合,為唐韻輯略五卷、備考一卷,形聲輯略一卷、備考一卷,古音輯略二卷、備考一卷,等韻輯略三卷。他著有易例輯略五卷。

陳立,字卓人,句容人。道光二十一年進士,二十四年,補應殿試。選翰林院庶吉士。散館改刑部主事,升郎中,授雲南曲靖府知府。請訓時,文宗有「為人清慎」之褒,時以道梗不克之任。少客揚州,師江都梅植之,受詩、古文辭;師江都凌曙、儀徵劉文淇,受公羊春秋、許氏說文、鄭氏禮,而於公羊致力尤深。

文淇嘗謂漢儒之學,經唐人作疏,其義益晦。徐彥之疏公羊,空言無當。近人如曲阜孔氏、武進劉氏,謹守何氏之說,詳義例而略典禮、訓詁。立乃博稽載籍,凡唐以前公羊古義及國朝諸儒說公羊者,左右採獲,擇精語詳。草創三十年,長編甫具。南歸後,乃整齊排比,融會貫通,成公羊義疏七十六卷。

初治公羊也,因及漢儒說經師法,謂莫備於白虎通。先為疏證,以條舉舊聞、暢隱扶微為主,而不事辨駁,成白虎通疏證十二卷。幼受爾雅,因取唐人五經正義中所引犍為舍人、樊光、劉歆、李巡、孫炎五家悉甄錄之。謂郭注中精言妙諦,大率胎此。附以郭音義及顧、沈、施、謝諸家切釋,成爾雅舊注二卷。

又以古韻之學敝蝕已久,而聲音之原,起於文字,說文諧聲,即韻母也。因推廣歸安姚氏說文聲系之例,刺取許書中諧聲之文,部分而綴敘之。以象形、指事、會意為母,以諧聲為子,其子之所諧,又即各綴於子下。其分部則兼取顧、江、戴、孔、王、段、劉、許諸家,精研而審核之,訂為二十部,成說文諧聲孳生述三卷。其文淵雅典碩,大抵考訂服制典禮及聲音訓詁為多,成句溪雜著六卷,卒,年六十一。

陳奐,字碩甫,長洲人。諸生。咸豐元年,舉孝廉方正。奐始從吳江沅治古學,金壇段玉裁寓吳,與沅祖聲善。嘗曰:「我作六書音韻表,惟江氏祖孫知之,餘尟有知者。」奐盡一晝夜探其梗概。沅嘗假玉裁經韻樓集,奐竊視之,加硃墨。後玉裁見之,稱其學識出孔、賈上,由是奐受學玉裁。高郵王念孫暨子引之、棲霞郝懿行、績溪胡培翬、涇胡承珙、臨海金鶚,咸與締交。

奐嘗言大毛公詁訓傳言簡意賅,遂殫精竭慮,專攻毛傳。以毛傳一切禮數名物,自漢以來無人稱引,韜晦不彰,乃博徵古書,發明其義。大抵用西漢以前舊說,而與東漢人說詩者不茍同。又以毛氏之學,源出荀子,而善承毛氏者,惟鄭仲師、許叔重兩家,故於周禮注、說文解字多所取說,著詩毛氏傳疏三十卷。又以疏中稱引,博廣難明,更舉條例,立表示圖,為毛詩說一卷。準以古音,依四始為毛詩音四卷。仿爾雅例,編毛傳為義類十九篇一卷。以鄭多本三家詩,與毛異,為鄭氏箋考徵一卷。又有詩語助義三十卷,公羊逸禮考徵一卷,師友淵源記一卷,禘郊或問、宋本集韻校勘記,各若干卷。

其論尚書大傳與毛傳同條共貫,論春秋之學,從公羊以知例,治穀梁以明禮。穀梁文句極簡,必得治禮數十年而後可明其要義。論釋名與毛傳、說文多不合,然可以討漢、宋說經家之源流。其論丁度集韻云:「集韻總字,具見類篇,先以類篇校集韻,再參之釋文、說文、玉篇、廣韻、博雅,則校讎之功過半矣。」又云:「陸氏釋文宋本,當於集韻求之。今尚書釋文,經開寶中陳諤等刪改之本,集韻則未經刪改者也。」於子書中尤好管子,嘗令其弟子元和丁士涵為管子案四卷。

家居授徒,從游者數十人。同郡管慶祺、丁士涵、馬釗、費鍔,德清戴望,其尤著也。同治二年,卒,年七十有八。

金鶚,字誠齋,臨海人。優貢生。博聞強識,邃精三禮之學。受知於山陽汪廷珍,與析難辨論,成禮說二卷。嘉慶二十四年,卒於京邸。所著求古錄一書,取宮室、衣服、郊祀、井田之類,貫串漢、唐諸儒之說,條考而詳辨之。鶚又嘗輯論語鄉黨注,釐正舊說,頗得意解。卒後稿全佚,陳奐求得之,釐為求古錄禮說十五卷,鄉黨正義一卷。

黃式三,字薇香,定海人。歲貢生。事親孝,嘗赴鄉試,母裘暴疾卒於家,馳歸慟絕。父老且病,臥床笫數年,衣食靧洗,必躬親之。比歿,持喪以禮,誓不再應鄉試。於學不立門戶,博綜群經,治易治春秋,而尤長三禮。論禘郊宗廟,謹守鄭學。論封域、井田、兵賦、學校、明堂、宗法諸制,有大疑義,必釐正之。有復禮說、崇禮說、約禮說。嘗著論語後案二十卷,自為之序。他著有書啟幪四卷,詩叢說一卷,詩序說通二卷,詩傳箋考二卷,春秋釋二卷,周季編略九卷,儆居集經說四卷,史說四卷。同治元年,卒,年七十四。子以周,從子以恭,俱能傳其學。

以周,本名元同,後改今名,以元同為字。同治九年優貢。旋舉於鄉,大挑以教職用,補分水縣訓導。以學臣奏加中書銜,以教授升用,旋選處州府教授,而年已七十,遂不就。以周篤守家學,以為三代下之經學,漢鄭君、宋硃子為最。而漢學、宋學之流弊,乖離聖經,尚不合於鄭、硃,何論孔、孟?有清講學之風,倡自顧亭林。顧氏嘗云:「經學即是理學。」乃體顧氏之訓,上追孔、孟之遺言,於易、詩、春秋皆有著述,而三禮尤為宗主。所著禮書通故百卷,列五十目,古先王禮制備焉。又以孟子學孔子,由博反約,而未嘗親炙孔聖。其間有子思子,綜七十子之前聞,承孔聖以啟孟子,乃著子思子輯解七卷。而舉子思所述夫子之教,必始於詩、書,而終於禮、樂,及所明仁義為利之說,謂其傳授之大恉,是深信博文約禮之經學,為行義之正軌,而求孟子學孔聖之師承,以子思為樞軸。暮年多疾,因曰:「加我數年,子思子輯解成,斯無憾!」既,書成而疾瘥,更號哉生。江蘇學政黃體芳建南菁講舍於江陰,延之主講。以周教以博文約禮、實事求是,道高而不立門戶。宗源瀚建辨志精舍於寧波,請以周定其名義規制,而專課經學,著錄弟子千餘人。卒,年七十有二。

以恭,字質庭。光緒元年舉人。著有尚書啟幪疏二十八卷,讀詩管見十二卷。

俞樾,字廕甫,德清人。道光三十年進士,改庶吉士。咸豐二年,散館授編修。五年,簡放河南學政,奏請以鄭公孫僑從祀文廟,聖兄孟皮配享崇德祠,並邀俞允。七年,以御史曹登庸劾試題割裂罷職。樾歸後,僑居蘇州,主講蘇州紫陽、上海求志各書院,而主杭州詁經精舍三十餘年,最久。課士一依阮元成法,游其門者,若戴望、黃以周、硃一新、施補華、王詒壽、馮一梅、吳慶坻、吳承志、袁昶等,咸有聲於時。東南遭赭寇之亂,典籍蕩然,樾總辦浙江書局,建議江、浙、揚、鄂四書局分刻二十四史,又於浙局精刻子書二十二種,海內稱為善本。

生平專意著述,先後著書,卷帙繁富,而群經平議、諸子平議、古書疑義舉例三書,尤能確守家法,有功經籍。其治經以高郵王念孫、引之父子為宗。謂治經之道,大要在正句讀,審字義,通古文假借,三者之中,通假借為尤要。王氏父子所著經義述聞,用漢儒「讀為」、「讀曰」之例者居半,發明故訓,是正文字,至為精審。因著群經平議,以附述聞之後。其諸子平議,則仿王氏讀書雜志而作,校誤文,明古義,所得視群經為多。又取九經、諸子舉例八十有八,每一條各舉數事以見例,使讀者習知其例,有所據依,為讀古書之一助。

樾於諸經皆有纂述,而易學為深,所著易貫,專發明聖人觀象系辭之義。玩易五篇,則自出新意,不拘泥先儒之說。復作艮宦易說,卦氣值日考、續考,邵易補原,易窮通變化論,互體方位說,皆足證一家之學。晚年所著茶香室經說,義多精確。古文不拘宗派,淵然有經籍之光。所作詩,溫和典雅,近白居易。工篆、隸。同時如大學士曾國籓、李鴻章,尚書彭玉麟、徐樹銘、潘祖廕,咸傾心納交。日本文士有來執業門下者。

樾湛深經學,律己尤嚴,篤天性,尚廉直,布衣蔬食,海內翕然稱曲園先生。光緒二十八年,以鄉舉重逢,詔復原官,重赴鹿鳴筵宴。三十二年,卒,年八十有六。著有群經平議三十五卷,諸子平議三十五卷及第一樓叢書,曲園雜纂,俞樓雜纂,賓萌集,春在堂雜文、詩編、詞錄、隨筆,右臺仙館筆記,茶香室叢鈔、經說,其餘雜著,稱春在堂全書。

同時以耆年篤學主講席者,則有南匯張文虎。文虎,字嘯山。諸生。嘗讀元和惠氏、歙江氏、休寧戴氏、嘉定錢氏諸家書,慨然嘆為學自有本,則取漢、唐、宋注疏、經說,由形聲以通其字,由訓詁以會其義,由度數名物以辨其制作,由語言事跡以窺古聖賢精義,旁及子史,莫不考其源流同異。精天算,尤長校勘。同治五年,兩江書局開,文虎為校史記三注,成札記五卷,最稱精善。卒,年七十有一。著有舒藝室遺書。

王闓運,字壬秋,湘潭人。咸豐三年舉人。幼好學,質魯,日誦不能及百言。發憤自責,勉強而行之。昕所習者,不成誦不食;夕所誦者,不得解不寢。於是年十有五明訓詁,二十而通章句,二十四而言禮。考三代之制度,詳品物之所用。二十八而達春秋微言,張公羊,申何學,遂通諸經。潛心著述,尤肆力於文。溯莊、列,探賈、董,其駢儷則揖顏、庾,詩歌則抗阮、左。記事之體,一取裁於龍門。

闓運刻苦勵學,寒暑無間。經、史、百家,靡不誦習。箋、注、抄、校,日有定課。遇有心得,隨筆記述。闡明奧義,中多前賢未發之覆。嘗曰:「治經:於易,必先知「易」字有數義,不當虛衍卦名;於書,必先斷句讀;於詩,必先知男女贈答之辭不足以頒學官、傳後世。一洗三陋,乃可言禮。禮明,然後治春秋。」又曰:「說經以識字為貴,而非識說文解字之字為貴。」又曰:「文不取裁於古則亡法,文而畢摹乎古則亡意。」又嘗慨然自嘆曰:「我非文人,乃學人也!」

學成出游。初館山東巡撫崇恩。入都。就尚書肅順聘。肅順奉之若師保。軍事多諮而後行。左宗棠之獄。闓運實解之。已而參曾國籓幕。胡林翼、彭玉麟等皆加敬禮。闓運自負奇才,所如多不合。乃退息無復用世之志。唯出所學以教後進。四川總督丁寶楨聘主尊經書院,待以賓師之禮,成材甚眾。歸為長沙思賢講舍、衡州船山書院山長。江西巡撫夏★J9延為高等學堂總教。光緒三十四年,湖南巡撫岑春蓂上其學行,特授檢討。鄉試重逢,加侍讀。闓運晚睹世變,與人無忤,以唯阿自容。入民國,嘗一領史館,遂歸。丙辰年,卒,年八十有五。

所著書以經學為多,其已刊者有周易說十一卷,尚書義三十卷,尚書大傳七卷,詩經補箋二十卷,禮記箋四十六卷,春秋公羊傳箋十一卷,穀梁傳箋十卷,周官箋六卷,論語注二卷,爾雅集解十六卷,又墨子、莊子、鶡冠子義解十一卷,湘軍志十六卷,湘綺樓詩文集及日記等。子女並能通經,傳其家學。次子代豐,早世,著有公羊例表。

王先謙,字益吾,長沙人。同治四年進士,選庶吉士,授編修。光緒元年,大考二等,擢中允,充日講起居注官。歷上疏言言路防弊,請籌東三省防務,並劾雲南巡撫徐之銘。六年,晉國子監祭酒。八年,丁憂歸,服闋,仍故官。疏請三海停工。出為江蘇學政。十四年,以太監李蓮英招搖,疏請懲戒。略言:「宦寺之患,自古為昭,本朝法制森嚴,從無太監攬權害事。皇太后垂簾聽政,一稟前謨,毫不寬假,此天下臣民所共知共見者。乃有總管太監李蓮英,秉性奸回,肆無忌憚。其平日穢聲劣跡,不敢形諸奏牘。惟思太監等給使宮禁,得以日近天顏;或因奔走微長,偶邀宸顧,度亦事理所有。何獨該太監言誇張恩遇,大肆招搖,致太監篦小李之名,傾動中外,驚駭物聽,此即其不安本分之明證。易曰『履霜堅冰』,漸也。皇太后、皇上於制治保邦之道,靡不勤求夙夜,遇事防維。今宵小橫行,已有端兆。若不嚴加懲辦,無以振綱紀而肅群情。」疏上不報。

先謙歷典云南、江西、浙江鄉試,搜羅人才,不遺餘力。既蒞江蘇,先奏設書局,仿阮元皇清經解例,刊刻續經解一千四百三十卷。南菁書院創於黃體芳,先謙廣籌經費,每邑拔取才士入院,而督教之,誘掖獎勸,成就人材甚多。開缺還家,歷主思賢講舍,嶽麓、城南兩書院,其培植人才,與前無異。三十三年,總督陳夔龍、巡撫岑春蓂奏以所著書進呈,賞內閣學士銜。宣統二年,長沙饑民圍撫署,衛兵開槍擊斃數人,民情愈憤,匪徒乘之放火燒署。省城紳士電請易巡撫,以先謙名首列,先謙不知也。總督瑞澂奏參,部議降五級。同鄉京官胡祖廕等以冤抑呈遞都察院,亦不報。國變後,改名遯,遷居鄉間,越六年卒。著有尚書孔傳參正三十六卷,三家詩集義疏二十八卷,漢書補注一百卷,荀子集解二十卷,日本源流考二十二卷,外國通鑒三十卷,虛受堂詩文集三十六卷等。

孫詒讓,字仲容,瑞安人。父衣言,自有傳。詒讓,同治六年舉人,官刑部主事。初讀漢學師承記及皇清經解,漸窺通儒治經、史、小學家法。謂古子、群經,有三代文字之通假,有秦、漢篆隸之變遷,有魏、晉正草之混淆,有六朝、唐人俗書之流失,有宋、元、明校讎之羼改。匡違捃佚,必有誼據,先成札迻十二卷。

又著周禮正義八十六卷,以為:「有清經術昌明,於諸經均有新疏,周禮以周公致太平之書,而秦、漢以來諸儒不能融會貫通。蓋通經皆實事、實字,天地、山川之大,城郭、宮室、衣服制度之精,酒漿、醯醢之細,鄭注簡奧,賈疏疏略。讀者難於深究,而通之於治,尤多謬盭。劉歆、蘇綽之於新、周,王安石之於宋,膠柱鍥舟,一潰不振,遂為此經詬病。詒讓乃以爾雅、說文正其訓詁,以禮經、大小戴記證其制度。研覃廿載,槁草屢易,遂博採漢、唐以來迄乾、嘉諸經儒舊說,參互繹證,以發鄭注之淵奧,裨賈疏之遺闕。其於古制,疏通證明,較之舊疏,實為淹貫。而注有違牾,輒為匡糾。凡所發正數十百事,匪敢壞『疏不破注』家法,於康成不曲從杜、鄭之意,實亦無誖。而以國家之富強,從政教入,則無論新舊學均可折衷於是書。」識者韙之。

光緒癸卯,以經濟特科徵,不應。宣統元年,禮制館徵,亦不就。未幾卒,年六十二。所著又有墨子閒詁十五卷,目錄、附錄二卷,後語二卷。精深閎博,一時推為絕詣。古籀拾遺三卷,逸周書斠補四卷,九旗古義述一卷。

鄭杲,字東甫,遷安人。父鳴岡,為即墨令,卒於官。貧不能歸,因家焉。杲事母孝。光緒五年,舉山東鄉試第一,明年成進士,授刑部主事。肆力於學,以讀經為正課,旁及朝章國故,矻矻終日,視仕進泊如也。嘗謂:「治經在信古傳,經者淵海,傳其航也。漢代諸儒,主乎此者不能通乎彼;唐、宋而降,能觀其通矣,乃舉古說而悉排之,惟斷以己意。若是者,皆非善治經者也。」杲以母憂歸,主講濼源書院。服闋,遷員外。時朝政維新,兩宮已積疑釁,杲獨惓惓言天子當竭誠以盡孝道。具疏草,莫敢為言者。二十六年夏,熒惑入南斗,復上書請修省,不報。未幾,卒。

杲之學深於春秋,其言曰:「左氏明魯史舊章,二傳則孔、孟推廣新意,口授傳指。公羊明魯道者也,穀梁明王道者也,左氏則備載當時行用之道。當時行用之道,霸道也。所以必明魯道者,為人子孫,道在法其祖也。穀梁則損益四代之趣咸在焉。惟聖人蹶起在帝位者,乃能用之也。」其為說兼綜三傳,而尤致嚴於事天、事君、事親之辨。謂:「春秋首致謹於元年正月,正月者,正即位也。正月謹始也,必能為父之子,然後能為天之子矣。春秋之有三正,由其有天、君、父之三命也。春者天也,王者君也,正月者父也,將以備責三正,而單舉正月,何也?事天、事君,皆以事親為始也。」凡杲所論著如此。

與杲同時者,有宋書升,字晉之,濰縣人。光緒十八年進士,改庶吉士。里居十年,殫心經術。易、書、詩均有撰述,尤精推步之學。法偉堂,字小山,膠州人。光緒十五年進士,官青州府教授,精研音韻之學,考訂陸德明經典釋文,多前人所未發。


\end{pinyinscope}