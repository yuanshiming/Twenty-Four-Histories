\article{列傳一百}

\begin{pinyinscope}
兆惠阿里袞子豐升額布彥達賚舒赫德子舒常

兆惠,字和甫,吳雅氏,滿洲正黃旗人,孝恭仁皇后族孫。父佛標,官至都統。兆惠,以筆帖式直軍機處。七遷至刑部侍郎、正黃旗滿洲副都統、鑲紅旗護軍統領。乾隆十三年,命兼領戶部侍郎。赴金川督糧運,疏論糧運事,並言諸將惟烏爾登,哈攀龍勇往,並及諸行省遣兵多不實。上命告經略傅恆覈實。師還,命覈軍需。調戶部侍郎。赴山東按傳鈔尚書孫嘉淦偽疏稿,暫署巡撫。十八年,命赴西藏防準噶爾。十九年,議用兵,命協理北路軍務,並督糧運。二十年,命駐烏里雅蘇臺。準噶爾臺吉噶勒藏多爾濟降,命兆惠畀以牲畜。是歲阿睦爾撒納叛,陷伊犁。命兆惠移駐巴里坤,兼督額林哈畢爾噶臺站。二十一年,師收復伊犁。上以定西將軍策楞不勝任,召兆惠還京授方略,未行,命逮策楞,並解扎拉豐阿定邊右副將軍以授兆惠。

時阿睦爾撒納北遁哈薩克,定西將軍達爾黨阿逐捕未得,上命還師。厄魯特諸宰桑從軍者謀為亂,綽囉斯汗噶勒藏多爾濟告兆惠,巴雅爾入掠其牧地。兆惠令寧夏將軍和起將百人徵厄魯特兵往御,而噶勒藏多爾濟從子扎那噶爾布及宰桑呢嗎、哈薩克錫喇、達什策零等陰通巴雅爾,中途變作,和起死之。

兆惠自伊犁將五百人逐捕,經濟爾哈朗至鄂壘扎拉圖,與達什策零戰,大敗之。逐賊戰於庫圖齊,再戰於達勒奇,殺賊數千。二十二年正月,至烏魯木齊。噶勒藏多爾濟、扎那噶爾布等諸賊皆會,日數十戰,馬且盡。師步行冰雪中,至特訥格爾,遂被圍。巴里坤辦事大臣雅爾哈善先遣侍衛圖倫楚將兵八百益兆惠軍。會兆惠遣軍校云多克德楞徹自圍中出,詣雅爾哈善言轉戰狀,事聞,上嘉兆惠奮勇,封一等武毅伯,授戶部尚書、鑲白旗漢軍都統、領侍衛內大臣。

及圖倫楚兵至,圍解,兆惠得新兵,復逐捕巴雅爾至穆壘河源。巴雅爾已徙牧他處,乃還師巴里坤。上以兆惠遠道旋師,逐賊不怠,賚御用玉韘、荷包、鼻煙壺,命同定邊將軍成袞扎布分路剪除厄魯特。兆惠旋偕參贊大臣鄂實等自額林哈畢爾噶進剿。時扎那噶爾布已殺噶勒藏多爾濟。會阿睦爾撒納自哈薩克盜馬竄還伊犁,掠扎那噶爾布牧地。

兆惠察回部頭人布拉呢敦、霍集占叛有跡,令參贊大臣富德逐捕阿睦爾撒納,而駐師濟爾哈朗以待。上責兆惠與成袞扎布急回部、緩阿睦爾撒納,失輕重。兆惠乃率師繼富德以北,遣使宣諭左右哈薩克,師復進次額密勒西岸。富德師至塔爾巴哈臺,獲逃渠巴雅爾及其孥,檻送京師,語詳富德傳。哈薩克汗阿布賚使獻馬,並具表請入覲,上降敕宣諭。阿布賚使言阿睦爾撒納以二十騎來投,約詰朝相見,令先收其馬並及牛羊。阿睦爾撒納驚走,獲其從子達什車凌、宰桑齊巴罕,縛送兆惠,兆惠以聞,命檻車致京師。兆惠分遣諸將圖倫楚、三達保、愛隆阿擊敗阿睦爾撒納屬眾,降其渠納木奇父子,送京師。兆惠復進,與富德軍合,詗阿睦爾撒納已入俄羅斯。上命還師。

旋授兆惠定邊將軍,討布拉呢敦、霍集占。兆惠奏請屯田烏魯木齊,以來春進討,倘不能即入回部,則且積穀市馬為持重,上責其怯懦。二十三年正月,兆惠以厄魯特人在沙喇伯勒尚萬戶,當先剿除,乃專力回部。上授雅爾哈善靖逆將軍,趣進師;命兆惠剿厄魯特事竟,別道合攻。並諭兆惠:「厄魯特性反覆,往往自殘殺。毋以其烏合稍眾,過疑慮。」兆惠與副將軍車布登扎布等分四道進剿:兆惠趨博羅布爾噶蘇,車布登扎布趨博羅塔拉,副都統瑚爾起等趨尼勒喀,侍衛達禮善等趨齊格特,皆會於伊犁。厄魯特眾紛紜潰竄,遂盡殲焉。

上以賊渠哈薩克錫喇、鄂哲特等十餘人皆未獲,命兆惠等加意奮勉。四月,兆惠獲鄂哲特送京師,疏言:「準噶爾事將蕆,請自伊犁移師合攻回部。」上仍責兆惠俘哈薩克錫喇等。既又令赴庫車察軍事,還京師,詔未至而兆惠師已發,會雅爾哈善圍庫車,霍集占突圍走。上逮雅爾哈善,以兆惠代將。兆惠中途疏言:「將八百人赴庫車,當與雅爾哈善協力剿賊,不原靦顏遽還。」上獎其肫誠勇往,賜雙眼孔雀翎。

既至軍,詗霍集占自庫車出入葉爾羌城守,乃帥師往捕。道阿克蘇,頭人頗拉特降。和闐頭人霍集斯故擒達瓦齊有功,至是亦來附,並招烏什頭人俱降,遂薄葉爾羌。兆惠兵止四百,自烏什至此千五百里,馬行乏,擇要隘屯兵。霍集占出戰,三敗,保城不復出。兆惠遣副都統愛隆阿以八百人扼喀什噶爾來路阻賊援,而率師臨蔥嶺南河為陣。蔥嶺南河者即喀喇烏蘇,譯言黑水,故時謂兆惠軍為黑水營。

兆惠念兵寡而城大,不任攻,諜言賊牧群在城南英峨奇盤山,乃帥輕騎躪其牧地,且致賊為野戰。渡黑水才四百騎而橋圮,霍集占挾數千騎出,師且戰且涉水,士卒殊死戰,五晝夜殺賊數千人。諸將高天喜、鄂實、三格、特通額皆戰死。兆惠馬再踣,面及脛皆傷,乃收兵築壘掘濠以為衛,賊亦築壘與我師相持。布拉呢敦自喀什噶爾至,助霍集占困我師。靖逆將軍納穆札爾等帥師赴援,中途遇回兵,力戰,皆死之。上先事發索倫、察哈爾、健銳營及陜、甘綠旗兵濟兆惠師;聞兆惠被圍,促富德赴援,又命阿里袞選戰馬三千送軍前。兆惠發阿克蘇,令舒赫德駐守。至是遣使令以被圍狀入奏,上獎兆惠統軍深入,忠誠勇敢,進封武毅謀勇一等公,並賜紅寶石帽頂、四團龍補服。

霍集占既逼我師為長圍,相持數月。賊自上游引水謀灌我師壘,我師於下游溝而洩之。我師壘迫深林,賊發槍彈著林木中,我師伐為薪,得彈,用以擊賊,常不匱。水不給,賊引水,反得飲,又掘井恆得泉。發地得藏粟一百六十窖,掠野得馬駝千餘。迫歲暮,圍合已三月,軍中糧漸盡,士卒煮鞍革,甚或掠回民以食。布拉呢敦、霍集占以圍久不下,會布魯特掠英吉沙爾,而兆惠即以是日率師焚賊壘,所殺傷過當,疑兆惠與布魯特相約,因遣使入我師請和。兆惠因其使射書諭以納款當入覲,二酋亦射書請撤圍相見。兆惠置不更答,而二酋自此攻稍緩。

二十四年正月,富德帥師至呼爾璊,遇回兵,轉戰五晝夜。阿里袞送馬至,合軍復戰。布拉呢敦出戰,中彈傷,還喀什噶爾。師至葉爾羌河岸,阿里袞與愛隆阿合軍為右翼,富德及舒赫德為左翼,逐賊,以次徐進。兆惠自圍中望見火光十餘里,馬駝群囂塵上,知援集,乃率餘軍破壘出,與諸軍相合,引還阿克蘇。上為賦黑水行紀其事。兆惠疏辭進封及章服,諭毋辭,並以其母老,時遣人存問。

霍集占之黨攻和闐,上以兆惠、富德既合軍急引還,謂富德不得以援兆惠為畢事,兆惠為帥被圍待援,尤不當遽引師退。諭趣富德援和闐,兆惠當就現在兵力加意奮勉,以竟全功。兆惠督諸將分道進攻,布拉呢敦棄喀什噶爾,霍集占亦棄葉爾羌同遁。兆惠師至喀什噶爾,撫定餘眾,富德亦收葉爾羌,為畫疆界,定貢賦,鑄泉幣,並分屯滿、漢兵駐守。富德師復進,追及霍集占,戰於阿勒楚爾,再戰於伊西洱庫爾淖爾。布拉呢敦、霍集占竄入巴達克山,師從之。巴達克山汗素勒坦沙初言霍集占中彈死,生獲布拉呢敦;復言兩酋已皆死,獻霍集占首。上加兆惠宗室公品級、鞍轡,並授其子侍衛。兆惠復撫定霍罕額爾德尼伯克所屬四城,並齊哩克布魯特、額德格納布魯特、阿濟畢部眾,請留兵分駐葉爾羌、喀什噶爾諸城。復定各城伯克更番入覲例。二十五年二月,師還,上幸良鄉,於城南行郊勞禮。兆惠入謁,賜朝珠及馬,從上還京。飲至,賚銀幣。圖形紫光閣。

二十六年七月,命協辦大學士,兼領刑部。旋令偕大學士劉統勛按楊橋河決。二十七年,復偕統勛勘江南運河。二十八年,直隸水災,命勘海口,疏天津、靜海諸縣水道。復命偕兩江總督尹繼善籌濬荊山橋河道。二十九年十一月,卒。上臨其喪,贈太保,謚文襄。嘉慶元年十一月,命配享太廟。

子扎蘭泰,尚高宗女和碩和恪公主,襲爵,授額駙。

阿里袞,字松崖,鈕祜祿氏,滿洲正白旗人,尹德第四子,而訥親弟也。乾隆初,自二等侍衛授總管內務府大臣。遷侍郎,歷兵、戶二部。五年,命與僉都御史硃必堦如山東勘巡撫碩色報歉收失實狀。疏言:「蘭山、郯城被水最甚,請緩徵新、舊賦,而以官帑市穀補社倉。」復命與江南河道總督高斌如江西勘巡撫岳濬等徇情納賄狀,鞫實,濬坐黜。

六年,侍郎梁詩正奏八旗兵丁當分置邊屯,復命與大學士查郎阿如奉天相度地勢。上言:「地宜耕者,吉林烏拉東北拉林、阿爾楚克,阿爾楚哈東飛克圖,齊齊哈爾東南呼蘭,西南黑爾蘇站、刷煙站,白都訥東八家子至登額爾者庫,皆沃壤;呼蘭東佛忒喜素素富林木,惟地高下各異。墨爾根寒暑早,齊齊哈爾砂磧,吉林烏拉無餘地,寧古塔山深,烏蘇裡產葠,皆不宜耕。」議政王大臣用其議,移屯自拉林、阿爾楚哈始。

八年,命如湖南勘巡撫許容劾糧道謝濟世狂縱狀,白濟世枉。命即署巡撫,歷河南、山西、山東諸省。十四年,訥親誅,令分任訥親償帑。旋以兄弟不相及,命免之。上將巡五臺,阿里袞疏請於臺懷建行宮,太原就巡撫署增建群室,上不許。阿里袞別疏薦參將傅謙,大學士傅恆弟也,上責其不當,詔切責。十五年,授湖廣總督。湖北巡撫唐綏祖為前總督永興劾罷,阿里袞白綏祖無受賕狀,永興坐黜。十六年,移兩廣總督。東莞民莫信豐謀為亂,討平之。尋居母憂,還京師。授戶部侍郎,擢尚書,歷刑、工、戶三部,兼鑲白旗漢軍都統。

二十一年四月,命軍機處行走。時上方責諸將逐捕阿睦爾撒納,定西將軍達爾黨阿出西路。五月,命阿里袞佐達爾黨阿,在領隊大臣上行走。九月,師至雅爾拉,遇賊再勝。十月,命與達爾黨阿還京師。二十二年正月,上以成袞扎布為定邊左副將軍,會師巴里坤,阿里袞仍在領隊大臣上行走。二月,達爾黨阿以失阿睦爾撒納削爵,阿里袞亦坐降戶部侍郎,旋兼正白旗蒙古副都統。

時回部大和卓木布拉呢敦、小和卓木霍集占分據葉爾羌、喀什噶爾為亂,於是沙拉斯、瑪呼斯諸部游牧與相應。九月,阿里袞與都統滿福自阿斯罕布拉克、和什特哷克取道至哈喇沙爾,搜山殺敵。復進至塔本順和爾、納木噶,俘男婦二百餘。十二月,滿福為郭多克哈什哈誘戕,沙拉斯、瑪呼斯遁庫車諸處。阿里袞復進次哈喇沙爾西南庫爾勒。二十三年正月,復進逐敵至呼爾塔克山,獲瑪呼斯得木齊額默根等。四月,阿里袞自魯克察還師,駐巴里坤。上先得伯克素賚瑪奏,阿里袞方搜捕瑪哈沁將還師,與阿里袞疏言師向呼爾塔克山不相應,上因責阿里袞中途遷延,罷侍郎,以副都統革職留任。

六月,靖逆將軍雅爾哈善攻庫車,霍集占赴援,入城守,已,復走還葉爾羌。上為罷雅爾哈善,而督定邊將軍兆惠攻阿克蘇,遂進逼葉爾羌。十一月,命阿里袞選馬三千、駝七百益兆惠軍。兆惠攻葉爾羌不克,瀕黑水結寨,霍集占為長圍困之。上聞,授富德定邊右將軍、阿里袞參贊大臣,援兆惠。是月命襲封二等公。十二月,授兵部尚書、正紅旗蒙古都統。二十四年正月,富德師至呼爾璊,霍集占出戰,五日四夜未決。阿里袞以駝馬至,乘夜分師為兩翼斫陣,斬千餘級。布拉呢敦中創,與霍集占並敗走。援兆惠全師以還。上以阿里袞送馬濟軍,如期集事,且殺賊多,加雲騎尉世職,例進一等公。七月,霍集占走巴達克山部,阿里袞與富德等帥師從之,降其眾萬二千有奇。阿里袞以五百人駐伊西勒庫爾淖爾西截隘,復分兵出其南,遇敵,奪其家屬輜重,降二千有奇。復將選兵二百逾嶺逐敵。巴達克山部旋納款,以霍集占首獻。行賞,賜阿里袞雙眼孔雀翎。

二十五年,召還京師。六月,自喀什噶爾行次葉爾羌,會雅木扎爾回酋邁喇木煽訛謂阿睦爾撒納復入阿克蘇,群起為亂。乃復還喀什噶爾,率八百人以出,至伯什克勒木,邁喇木等以千餘人拒戰,阿里袞督所部擊破之。賊入城堅守,麾兵合圍,夜四鼓,城人呼號乞降,邁喇木遁去。上獎阿里袞應機立辦,授其子拜唐阿豐升額藍翎侍衛。阿里袞旋捕邁喇木等送京師,復進豐升額三等侍衛,授其次子倭興額藍翎侍衛。十月,阿里袞還京師,授領侍衛內大臣,圖形紫光閣。二十八年,加太子太保。二十九年,授戶部尚書、協辦大學士。

時緬甸亂,南徼兵連數歲。三十一年春,將軍明瑞深入,上授阿里袞參贊大臣,馳傳至軍。二月,明瑞戰死猛臘,大學士傅恆出為經略,授阿里袞及阿桂為副將軍,並令暫領雲貴總督,率師駐永昌。朝議:「明年進兵。今歲秋夏瘴退,先收普洱、思茅邊外諸小部落。」阿里袞疏言:「邊外十三板納皆內屬不為亂,惟召散、整貝、猛勇三部附緬甸。」當用兵時,刑部尚書舒赫德在軍,與雲南巡撫鄂寧密疏議撫。六月,緬甸使頭人請款,阿里袞拒之,以聞。上命置毋答,並譴舒赫德等。七月,阿里袞疏請絕緬甸貿易,並治雲南省城至永昌道,撫慰沿邊諸土司,借帑俾市籽種牛具,皆得俞旨。十二月,阿桂兵至,共發兵出邊,未深入而還。

三十四年二月,上摘雲貴總督明德疏語,以軍中馬羸責阿里袞等,下部議奪職,命寬之。三月,傅恆至軍,與阿里袞等議進兵渡戛鳩江,西攻猛拱、猛養兩土司,向阿瓦。阿瓦,緬甸都也。偏師至猛密,夾江而下,造舟蠻暮通往來。七月,師行。初,阿里袞病瘍,上遣醫就視良愈,至是復大作。傅恆令留永昌治疾,阿里袞堅請行。師進,緬甸兵不出。十月,傅恆還師蠻暮,復進攻老官屯,駐戛鳩江口。緬甸兵水陸並至,傅恆、阿桂軍江東,阿里袞軍江西,迎戰。敵結寨自固,阿里袞率兵七百攻之,敵百餘棄寨走。把總姚卓殺敵,奪其旗,師銳進,敵四百餘亦遁。復戰,會日暮,敵不能堅守,皆引去。凡破寨三,殺敵五百餘。傅恆亦遘疾,諸將議毋更進兵,阿里袞曰:「老官屯賊寨,前歲額爾登額攻未克。距此僅一舍,不破之何以報命?」策馬行,傅恆以下皆從之,寨堅,攻不克。阿里袞疾甚,猶強起督攻,視槍砲最多處輒身當之。傅恆慮其傷,令將舟師,毋更與攻寨。十二月,卒於軍,謚襄壯,祀賢良祠。以豐升額金川功,追加封號為果毅繼勇公。子豐升額、倭興額、色克精額、布彥達賚。

豐升額,自三等侍衛襲封一等公,擢領侍衛內大臣,署兵部尚書、鑲藍旗蒙古都統。三十五年八月,命在軍機處行走。金川再用兵,定邊左副將軍溫福為帥,劾參贊大臣伍岱乖謬。上命豐升額往勘,因授豐升額參贊大臣。五月,豐升額攻東瑪寨,偽退以致敵,令章京佛倫泰、富爾賽突起偪寨,侍衛伸達蘇發鉅砲,敵驚卻,多墜崖死,遂克東瑪。六月,攻固卜濟山梁。師至色爾渠,令烏什哈達、巴三泰等左右進攻。豐升額出中路,發砲墮碉。烏什哈達等引兵出巖下,豐升額自山徑策應鏖戰,敵大奔。七月,復克色爾渠大碉及卡房百餘。卡房,敵所置堠也。旋與溫福大軍合,十月,克路頂宗、喀木色爾諸寨。復進克兜烏山梁及附近諸寨。十一月,克博爾根山,奪瑪覺烏大寨。再進克明郭宗,下碉卡九十餘。克嘉巴山,焚經樓。語詳溫福傳。十二月,授豐升額副將軍。

三十八年正月,與將軍溫福、副將軍阿桂議分道並進,溫福自功噶爾拉進攻噶爾薩爾,阿桂自僧格宗經納圍納扎木,至當噶爾拉,待溫福軍至,與合攻噶拉依。豐升額自章谷、吉地經綽斯甲布,溫福分遣參贊大臣舒常駐軍於此,與合攻勒烏圍。豐升額駐軍宜喜,於其地設糧臺,規進取。四月,考績,加太子少保。溫福師銳進,六月,次木果木。阿桂亦克當噶爾拉。上令豐升額攻大板昭,命未至,木果木師潰,溫福死之。上聞敗,命豐升額引兵自黨壩、三雜穀至巴朗拉為阿桂聲援。既聞阿桂自當噶爾拉全師而出,屯翁古爾壟,諭豐升額仍駐宜喜為犄角。

豐升額初未移軍,分兵駐智固山,防後路。阿桂以定西將軍為帥,十一月,收小金川全境。豐升額自宜喜攻克沙壩山梁碉卡,分敵勢。十二月,阿桂定策自取穀噶,而令豐升額攻凱立葉,進兵。上命豐升額以五千人往攻,三十九年正月,師次薩爾赤鄂羅山,占其南雪山,又分兵屯孟拜拉山梁。阿桂遣納木扎等將二千人與合軍。二月晦夜半,豐升額帥師自達爾扎克北山澖越石蹋雪以進。次日黎明,至凱立葉山麓。山絕險,凡大峰各置碉,見我師至且近,槍石並發。豐升額督師直前沖擊,與侍衛彰靄、明仁取第二峰,瑪爾占、伊達裡取第三峰,令領隊大臣五岱營第三峰下。捷聞,上以碉據峰巔,仰攻不易克,命留五岱於此,而移軍穀噶,與阿桂合軍攻勒烏圍。

阿桂遣諜告豐升額:「達爾扎克面當莫爾敏山,山旁地曰迪噶拉穆扎。師得此,繞出凱立葉後,夾攻易為力。」豐升額即遣兵占莫爾敏山,敵力爭,絕我師前後不相屬,卒敗敵,取迪噶拉穆扎。豐升額尋從上命移軍穀噶。六月,克色繃普,破碉十一。七月,克該布達什諾大碉。十月,自間道克墨格爾陟曰爾巴當噶西峰,破碉寨二百餘,得凱立葉山梁之半。命議敘,賚玄狐帽、貂馬褂。十一月,攻格魯古丫口,通黨壩,遂進逼勒烏圍。四十年正月,克甲爾納堪布卓沿河諸碉寨。四月,破噶爾丹寺及噶朗噶木柵十七。五月,克丫口石碉八、木城四。再進,盡隳遜克爾宗諸碉寨。敕獎其奮勉,命封號加「繼勇」字。七月,師至章噶,碉甚堅,碉外為壕三重,壕外立木柵。海蘭察攻其中,豐升額督官達色、仁和等攻其左右,毀柵覆壕以度師,緣碉側直上,自其巔俯攻,遂克之,並得其旁木城。八月,與阿桂合克勒烏圍。九月,復進向噶拉依。十二月,克格隆古科布曲山梁。四十一年正月,克瑪爾古當噶山梁。金川全部悉定。師圍噶拉依,上命加豐升額一等子,以其弟布彥達賚襲爵。尋移戶部尚書,賜雙眼孔雀翎。二月,金川酋索諾木出降,致京師。

四月,師還,賚御廝馬具鞍轡,圖形紫光閣。四十二年十月,卒,贈太子太保,謚誠武。

布彥達賚,自三等侍衛累遷武備院卿。嘉慶間,授戶部尚書、正白旗滿洲都統、步軍左翼總兵署統領。五年,卒,贈太子太保,謚恭勤。布彥達賚女為宣宗元妃,道光元年,冊謚孝穆皇后,禮成,追封三等公。

舒赫德,字伯容,舒穆魯氏,滿洲正白旗人,徐元夢孫也。舒赫德,自筆帖式授內閣中書,累遷御史,充軍機處章京。乾隆二年,疏言:「八旗生齒日繁。盛京、黑龍江、寧古塔三省土沃可耕。請將閒散移屯。並條議設公庫,以各省稅務專屬旗員,贖旗地典於民者,以官地畀無地旗丁。以十年為期,次第施行。」上以稅務專屬旗員為非是,諭曰:「舒赫德此議,但知旗人生計艱難,不知國家設關,欲稽察奸宄,非為收稅之員身家計也。朕日以砥礪廉隅勉臣工,尚恐其不能遵奉,而可以謀利導之乎?況各省稅務本未分滿、漢,旗員有廉潔者,何嘗不可派委。大抵為上者施逮下之仁,惟有勵以忠勤,示以節儉;為下者皆當早作夜思,宣力供職,以永受國家惠養。方可謂之計長久。蓋厚其生計,不可不思,而長貪以為惠下,則未見其利,而且貽害,非所以教旗員,亦非所以愛旗員也。」初,雍正間,京師設官米局,收旗丁餉米存儲平糶。舒赫德疏請復設,從之。五遷至兵部尚書,移戶部尚書。

十三年,命從經略大學士傅恆征金川,授參贊,加太子太保。十四年,師還,留辦軍需奏銷。命往雲南、湖廣、河南查閱營伍,並勘云南金沙江運銅水道。舒赫德疏言金沙江下游銅運無阻,上游四十餘灘多峻險,仍當陸運。總督張允隨言上下游皆疏通,語不實。古州總兵哈尚德因古州被水,請移城,上令舒赫德相度。舒赫德請城內外疏積水,無待移建。十月,復移兵部尚書。十五年,疏言:「定例額兵百人缺二,謂之『名糧』,為軍中公使錢。惟繕治軍器、巡防路費,每不給於用。馬兵不宜於東南,其在西北,十居其八,亦可量減。藤牌兵全無實用。擬於馬兵、藤牌兵內加增名糧,以備公用。」廷議允行。十二月,命如浙江勘海塘。十六年,命勘永定河工。又命如浙江按杭州將軍覺羅額爾登受賕狀。

十七年,命偕侍郎玉保赴北路軍防準噶爾。十八年,以準噶爾內亂,撤防,召還。命如江南塞銅山張家馬路河決。時準噶爾達瓦齊復為臺吉,所部杜爾伯特臺吉車凌等來降。準噶爾宰桑瑪木特,烏梁海得木齊扎木參、瑚圖克等追車凌,先後闌入北路卡倫。上命舒赫德如鄂爾坤治軍事,而令侍郎玉保、前鋒統領努三、散秩大臣薩喇爾佐定邊左副將軍成袞扎布。十九年春,舒赫德至軍,參贊大臣達清阿誘致瑪木特,將檻送京師,疏聞,上以瑪木特聞召即至,命釋使還。既,薩喇爾、努三帥師出邊,獲扎木參、瑚圖克,舒赫德等復疏請檻送京師。上以瑪木特誘致,扎木參等乃逐捕所得,事不同,責舒赫德謬誤,命以扎木參等囚置軍中。軍中方傳達瓦齊遣其將扎努噶爾布以五千人犯邊。成袞扎布等致書達瓦齊,言瑪木特、扎木參等以入邊被捕本末。上以為太懦,諭舒赫德等。上方以準噶爾內訌,將乘時收烏梁海,以薩喇爾本蒙古頭人,習邊事,將倚以招致。舒赫德等疏言達瓦齊復為臺吉,烏梁海等未易招致,令薩喇爾駐軍卓克索待後舉。上責舒赫德畏怯,使薩喇爾掣肘。蒙古貝勒額琳沁、公格勒克巴木丕勒以赴軍遷延得罪,舒赫德等疏言其至軍後奮勉,請贖罪。上下詔責其舛謬,並及行文達瓦齊事,下部議奪官,得旨寬免。上幸熱河,召舒赫德詣行在示方略。旋解成袞扎布將軍以授策楞。

七月,輝特臺吉阿睦爾撒納來降。舒赫德與策楞議留阿睦爾撒納及諸頭人軍中待命,以其孥移置蘇尼特。阿睦爾撒納有兄為瑪木特所獲,乞資以行糧俾赴援,舒赫德不許。是時上方欲倚阿睦爾撒納擒達瓦齊,事聞,上盛怒,詔罪狀策楞、舒赫德,略謂:「阿睦爾撒納初來降,乃以其眷屬移置戈壁南,相距數千里,使其父母妻子分析離居,失遠人歸附心。準噶爾內亂,所部叩關內附,正可示以懷柔,永綏邊境。策楞、舒赫德顛倒舛謬,至於此極!」皆奪職,以閒散在參贊大臣上效力贖罪,並籍其家,罪及諸子。二十年正月,上命阿睦爾撒納佐班第帥師討達瓦齊。阿睦爾撒納請移游牧於烏里雅蘇臺,上許之。命領隊大臣兆惠駐軍於此,予舒赫德章京銜佐兆惠。六月,師已定伊犁,諭曰:「策楞、舒赫德軍前效力,今大功已成,本欲施恩,開其自效。策楞已予都統銜,駐軍巴里坤。檢舒赫德筆札,雖無怨望語,乃效漢人習,日必記事作詩。嗣宜痛自改悔,令仍以章京留烏里雅蘇臺。」上分準噶爾故地,本眾建諸侯意,四衛拉特各為汗。阿睦爾撒納求為總統,上不許,遂叛。其妻子在烏里雅蘇臺,舒赫德偕兆惠收送京師。二十一年,喀爾喀臺吉青滾雜卜叛,驛道中梗。會察哈爾兵數百送羊至,舒赫德留之,分布諸臺站,軍報乃通。行邊至努兌木倫,護厄魯特人。掠馬者烏梁海人入邊,竄匿俄羅斯,馳檄往索。上嘉其治事尚協機宜,召還,授正黃旗漢軍副都統。

二十二年正月,上命成袞扎布為定邊將軍,逐捕阿睦爾撒納,授舒赫德參贊大臣。尋擢兵部尚書,兼鑲黃旗漢軍都統。三月,以舒赫德在軍獨具疏奏事,責其放縱,罷尚書。七月,疏請防範沙喇斯游牧內移,上斥其藉作歸計,嚴諭申戒。十二月,上以成袞扎布師久無功,詔罪狀舒赫德,略言:「舒赫德起自廢籍,初赴軍授方略,令傳諭成袞扎布,並戒其毋更恇怯。乃至軍後,諸事皆失機宜。即如招服克哷特、烏魯特等游牧,當收其馬以佐軍;乃任令屯駐山中,致兵過復叛。及朕有旨詰責,始東遮西露,往來道途,疲馬力於無用之地。舉此一端,可見諸事皆無成算。此實舒赫德未將朕旨宣示成袞扎布之所致也。舒赫德罪不勝誅,朕念成袞扎布去年擒青滾雜卜之功,貰舒赫德以不死。今奪職為兵,從軍贖罪。」

二十三年,予頭等侍衛銜,駐阿克蘇。十月,將軍兆惠逐捕霍集占,深入被圍。命定邊右副將軍富德往援,授舒赫德參贊大臣,會於巴爾楚克。舒赫德以阿克蘇通葉爾羌、喀什噶爾要隘,當設卡倫。上嘉之,擢吏部侍郎,遷工部尚書、鑲紅旗滿洲都統,賜孔雀翎。十二月,簡阿克蘇銳卒、諸路兵先至者馳援兆惠。二十四年正月,與富德合軍解兆惠圍,予雲騎尉世職。七月,命移駐葉爾羌,旋命仍駐阿克蘇。先後奏定回城賦稅,臺站酌設伯克,阿克蘇鑄騰格,以四存公,六畀回人。阿克蘇、庫車、哈喇沙爾、烏什、和闐置文武吏。皆得旨議行。尋以回部平,圖形紫光閣。二十八年,加太子太保。

二十九年,命如福建按提督黃仕簡劾廈門洋行陋規,總督楊廷璋以下皆得罪,語詳廷璋傳。三十一年,署陜甘總督,旋署戶部尚書。三十二年,如湖南北讞獄。三十三年,將軍明瑞徵緬甸,敗績,死之。上命大學士傅恆為經略,授舒赫德參贊大臣,先赴雲南籌畫進軍。舒赫德密疏議巡,忤上旨。下部議奪官,並削雲騎尉世職,命以都統銜參贊大臣,出駐烏什。

三十六年,土爾扈特汗渥巴錫等自俄羅斯來歸,眾疑其偽降,舒赫德力白無他志,命如伊犁宣撫,尋授伊犁將軍。十一月,授戶部尚書。三十八年,加太子太保,授武英殿大學士。九月,命如江南監黃河老壩口堤工。壽張民王倫叛,破臨清,命督師進剿,克之,倫自燔死。賜雙眼孔雀翎,復予雲騎尉世職,賚貂冠、黑狐褂。四十一年,金川平,圖形紫光閣。初,舒赫德為伊犁將軍,子舒寧在京杖斃二奴,得罪,上命發伊犁交舒赫德約束。及是,又以爭煤礦為山東民所訟,舒赫德縛舒寧送刑部,疏請罪。下部議奪官,命寬之。四十二年四月,卒,贈太保,謚文襄,祀賢良祠。

子舒常,始為侍衛。舒赫德議移置阿睦爾撒納妻子得罪,舒常亦奪官,發黑龍江披甲。及舒赫德召還為副都統,授舒常三等侍衛。舒赫德以佐成袞扎布無功再得罪,舒常復發黑龍江。乾隆二十三年二月,命釋還。累遷至鑲藍旗護軍統領。三十七年,將軍溫福征金川,授參贊大臣。金川平,圖形紫光閣,與舒赫德父子並列前五十功臣。舒赫德卒,令還京治喪,授工部侍郎。出為貴州巡撫,遷湖廣、兩廣總督。入為工部尚書。復出署江西巡撫,復為湖廣總督。荊州漢水決,奪官,授一等侍衛。擢都察院左都御史,改鑲黃旗蒙古都統。嘉慶初,署刑、兵二部尚書。卒,謚恪靖。

論曰:兆惠再就圍中受爵,得援師克竟其功;而為之援者,前則雅爾哈善,後則富德,顧坐法不克有終。訥親之誅也,高宗謂策楞、達爾黨阿皆愧奮,阿里袞獨內疑,遇事畏葸。然策楞、達爾黨阿先後僨事奪封,阿里袞以戰閥承世祚,豐升額繼之,慶延於後嗣。舒赫德初為御史有直聲,後出視軍,高宗屢言其懦,再被譴謫,終致臺司。功名始終之際,蓋亦有天焉。然其要必歸於忠謹,茲非彰彰可睹歟?


\end{pinyinscope}