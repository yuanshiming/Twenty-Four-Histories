\article{本紀一}

\begin{pinyinscope}
太祖本紀

太祖承天廣運聖德神功肇紀立極仁孝睿武端毅欽安弘文定業高皇帝,姓愛新覺羅氏,諱努爾哈齊。其先蓋金遺部。始祖布庫里雍順,母曰佛庫倫,相傳感硃果而孕。稍長,定三姓之亂,眾奉為貝勒,居長白山東俄漠惠之野俄朵裡城,號其部族曰滿洲。滿洲自此始。元於其地置軍民萬戶府,明初置建州衛。

越數世,布庫里雍順之族不善撫其眾,眾叛,族被戕,幼子範察走免。又數世,至都督孟特穆,是為肇祖原皇帝,有智略,謀恢復,殲其仇,且責地焉。於是肇祖移居蘇克蘇滸河赫圖阿喇。有子二:長充善,次褚宴。充善子三:長妥羅,次妥義謨,次錫寶齊篇古。

錫寶齊篇古子一:都督福滿,是為興祖直皇帝。興祖有子六:長德世庫,次劉闡,次索長阿,次覺昌安,是為景祖翼皇帝,次包朗阿,次寶實。

景祖承祖業,居赫圖阿喇。諸兄弟各築城,近者五里,遠者二十里,環衛而居,通稱寧古塔貝勒,是為六祖。景祖有子五:長禮敦,次額爾袞,次界堪,次塔克世,是為顯祖宣皇帝,次塔察篇古。時有碩色納、加虎二族為暴於諸部,景祖率禮敦及諸貝勒攻破之,盡收五嶺東蘇克蘇滸河西二百里諸部,由此遂盛。

顯祖有子五,太祖其長也。母喜塔喇氏,是為宣皇后。孕十三月而生。是歲己未,明嘉靖三十八年也。

太祖儀表雄偉,志意闊大,沈幾內蘊,發聲若鐘,睹記不忘,延攬大度。鄰部古勒城主阿太為明總兵李成梁所攻,阿太,王杲之子,禮敦之女夫也。景祖挈子若孫往視。有尼堪外蘭者,誘阿太開城,明兵入殲之,二祖皆及於難。太祖及弟舒爾哈齊沒於兵間,成梁妻奇其貌,陰縱之歸。途遇額亦都,以其徒九人從。

太祖既歸,有甲十三。五城族人龍敦等忌之,以畏明為辭,屢謀侵害,遣人中夜狙擊,侍衛帕海死焉。額亦都、安費揚古備禦甚謹,嘗夜獲一人,太祖曰:「縱之,毋植怨也。」使人愬於明曰:「我先人何罪而殲於兵?」明人歸其喪。又曰:「尼堪外蘭,吾仇也,原得而執之。」明人不許。會薩爾虎城主諾米納、嘉木瑚城主噶哈善哈思虎、沾河城主常書率其屬來歸,太祖與之盟,並妻以女,於是有用兵之志焉。是歲癸未,明萬歷十一年也,太祖年二十五。

癸未夏五月,太祖起兵討尼堪外蘭,諾米納兵不至,尼堪外蘭遁之甲版。太祖兵克圖倫城,尼堪外蘭遁之河口臺。兵逐之,近明邊,明兵出,尼堪外蘭遁之鵝爾渾。兵出無功,由於諾米納之背約,且洩師期也。殺諾米納及其弟奈喀達。五城族人康嘉、李岱等糾哈達兵來劫瑚濟寨,太祖使安費揚古、巴遜率十二人追之,盡奪所掠而返。

甲申春正月,攻兆佳城,報瑚濟寨之役也。途遇大雪,眾請還。太祖曰:「城主李岱,我同姓兄弟,乃為哈達導,豈可恕耶!」進之,卒下其城。先是龍敦唆諾米納背約,又使人殺噶哈善哈思虎,太祖收其骨歸葬。六月,討薩木占,為噶哈善哈思虎復仇也。又攻其黨訥申於馬兒墩寨,攻四日殲之。九月,伐董鄂部,大雪,師還,城中師出,以十二騎敗之。王甲部乞師攻翁克洛城,中道赴之,焚其外郭。太祖乘屋而射,敵兵鄂爾果尼射太祖,貫胄中首,拔箭反射,殪其一人。羅科射太祖,穿甲中項,拔箭鏃卷,血肉迸落,拄弓徐下,飲水數斗,創甚,馳歸。既愈,復往攻,克之。求得鄂爾果尼、羅科。太祖曰:「壯士也。」授之佐領,戶三百。

乙酉春二月,太祖略界凡,將還,界凡、薩爾滸、東佳、把爾達四城合兵四百人來追,至太蘭岡,城主訥申、巴穆尼策馬並進,垂及,太祖返騎迎敵,訥申刃斷太祖鞭,太祖揮刀斫其背墜馬,回射巴穆尼,皆殪之。敵不敢逼,徐行而去。夏四月,徵哲陳部,大水,令諸軍還,以八十騎前進。至渾河,遙見敵軍八百憑河而陣。包朗阿之孫扎親桑古里懼,解甲與人。太祖斥之曰:「爾平日雄族黨間,今乃畏葸如是耶!」去之。獨與弟穆爾哈齊、近侍顏布祿、武陵噶直前沖擊,殺二十餘人,敵爭遁,追至吉林岡而還。太祖曰:「今日之戰,以四人敗八百,乃天祐也。」秋九月,攻安土瓜爾佳城,克之,斬其城主諾一莫渾。

丙戌夏五月,徵渾河部播一混寨,下之。秋七月,征服哲陳部托漠河城。聞尼堪外蘭在鵝爾渾,疾進兵,攻下其城,求之弗獲。登城遙望,一人氈笠青棉甲,以為尼堪外蘭也,單騎逐之,為土人所圍,被創力戰,射殺八人,斬一人,乃出。既知尼堪外蘭入明邊,使人向邊吏求之,使齋薩就斬之。以罪人斯得,始與明通貢焉。明歲犒銀幣有差。

丁亥春正月,城虎闌哈達南岡,始建宮室,布教令於部中,禁暴亂,戢盜竊,立法制。六月,攻哲陳部,克山寨,殺寨主阿爾太。命額亦都帥師取把爾達城。太祖攻洞城,城主扎海降。

戊子夏四月,哈達貝勒扈爾干以女來歸,蘇完部索爾果率其子費英東等、雅爾古寨扈拉虎率子扈爾漢、董鄂部何和禮俱率所部來歸,皆厚撫之。秋九月,取完顏部王甲城。葉赫貝勒納林布祿以女弟那拉氏來歸,宴饗成禮,是為孝慈高皇后。

己丑春正月,取兆佳城,斬其城主寧古親。冬十月,明以太祖為建州衛都督僉事。

辛卯春正月,遣師略長白山諸路,盡收其眾。葉赫求地,弗與。葉赫以兵劫我東界洞寨。

壬辰冬十月二十五日,第八子皇太極生,高皇后出也,是為太宗。

癸巳夏六月,葉赫、哈達、輝發、烏拉四部合兵侵戶布察,遣兵擊敗之。秋九月,葉赫以不得志於我也,乃糾約扈倫三部烏拉、哈達、輝發,蒙古三部科爾沁、錫伯、卦爾察,長白二部訥殷、硃舍裏,凡九部之兵三萬來犯。太祖使武里堪偵敵,至渾河,將以夜渡河,逾嶺馳以告。太祖曰:「葉赫兵果至耶?其語諸將以旦日戰。」及旦,引兵出,諭於眾曰:「解爾蔽手,去爾護項,毋自拘縶,不便於奮擊。」又申令曰:「烏合之眾,其志不一,敗其前軍,軍必反走,我師乘之,靡弗勝矣。」眾皆奮。太祖令額亦都以百人挑戰。葉赫貝勒布齋策馬拒戰,馬觸木而踣,我兵吳談斬之。科爾沁貝勒明安馬陷淖中,易驏馬而遁。敵大潰,我軍逐北,俘獲無算,擒烏拉貝勒之弟布占泰以歸。冬十月,遣兵征硃舍裏路,執其路長舒楞格,遣額亦都等攻訥殷路,斬其路長搜穩塞克什,以二路之助敵也。

甲午春正月,蒙古科爾沁貝勒明安、喀爾喀貝勒老薩遣使來通好,自是蒙古通使不絕。

乙未夏六月,徵輝發,取多壁城,斬其城主。

丙申春二月,明使至,從朝鮮官二人,待之如禮。秋七月,遣布占泰歸烏拉,會其貝勒為部人所殺,遂立布占泰為貝勒。

丁酉春正月,葉赫四部請修好,許之,與盟。九月,使弟舒爾哈齊貢於明。

戊戌春正月,命弟巴雅拉、長子褚英率師伐安褚拉庫,以其貳於葉赫也。冬十月,太祖入貢於明。十一月,布占泰來會,以弟之女妻之。

己亥春正月,東海渥集部虎爾哈路路長王格、張格來歸,獻貂狐皮,歲貢以為常。二月,始制國書。三月,開礦,採金銀,置鐵冶。哈達與葉赫構兵,送質乞援,遣費英東、噶蓋戍之。哈達又私於葉赫,戍將以告。秋九月,太祖伐哈達,攻城克之,以其貝勒孟格布祿歸。孟格布祿有逆謀,噶蓋未以告,並誅之。

辛丑春正月,明以滅哈達來責,乃遣孟格布祿之子吳爾古岱歸主哈達。哈達為葉赫及蒙古所侵,使訴於明,明不應;又使哈達以饑告於明,亦不應。太祖乃以吳爾古岱歸,收其部眾,哈達亡。十二月,太祖復入貢於明。是歲定兵制,令民間養蠶。

癸卯春正月,遷於赫圖阿喇,肇祖以來舊所居也。九月,妃那拉氏卒,即孝慈高皇后也。始妃有病,求見其母,其兄葉赫貝勒不許來,遂卒。

甲辰春正月,太祖伐葉赫,克二城,取其寨七。明授我龍虎將軍。

乙巳,築外城。蒙古喀爾喀巴約忒部恩格德爾來歸。

丙午冬十二月,恩格德爾會蒙古五部使來朝貢,尊太祖為神武皇帝。是歲,限民田。

丁未春正月,瓦爾喀斐悠城長穆特黑來,以烏拉侵暴,求內附。命舒爾哈齊、褚英、代善及費英東、揚古利率兵徙其戶五百。烏拉發兵一萬遮擊,擊敗之,斬首三千,獲馬五千匹。師還,優賚褚英等。夏五月,命弟巴雅拉、額亦都、費英東、扈爾漢征渥集部,取二千人還。秋九月,太祖以輝發屢負約,親征,克之,遂滅輝發。

戊申春三月,命褚英、阿敏等伐烏拉,克宜罕阿林城。布占泰懼,復通好,執葉赫五十人以來,並請婚。許之。是歲,與明將盟,各守境,立石於界。

己酉春二月,遺明書,謂:「鄰朝鮮而居瓦爾喀者乃吾屬也,其諭令予我。」明使朝鮮歸千餘戶。冬十月,命扈爾漢征渥集呼野路,盡取之。

庚戌冬十一月,命額亦都率師招渥集部那木都魯諸路路長來歸。還擊雅攬路,為其不附,又劫我屬人也,取之。

辛亥春二月,賜國中無妻者二千人給配,與金有差。秋七月,命子阿巴泰及費英東、安費揚古取渥集部烏爾古宸、木倫二路。八月,弟舒爾哈齊卒。冬十月,命額亦都、何和里、扈爾漢率師征渥集部虎爾哈,俘二千人,並招旁近各路,得五百戶。

壬子秋九月,太祖親征烏拉,為其屢背盟約,又以鳴鏑射帝女也。布占泰御於河。駐師河東,克六城,焚積聚。布占泰親出乞和。太祖切責之,許其納質行成,而戍以師。師還。

癸丑春正月,布占泰復貳於葉赫,率師往征。布占泰以兵三萬來迎。太祖躬先陷陣,諸將奮擊,大敗之,遂入其城。布占泰至城,不得入,代善追擊之,單騎奔葉赫,遂滅烏拉。使人索布占泰,葉赫不與。秋九月,起兵攻葉赫,使告明,降兀蘇城,焚其十九城寨。葉赫告急於明,明遣使為解。師還,經撫順,明游擊李永芳來迎。與之書曰:「與明無嫌也。」

甲寅夏四月,帝八子皇太極娶於蒙古,科爾沁部莽古思之女也,行親迎禮。明使來,稱都督。上語之曰:「吾識爾,爾遼陽無賴蕭子玉也。吾非不能殺爾,恐貽大國羞。語爾巡撫,勿復相詐。」冬十一月,遣兵征渥集部雅攬、西臨二路,得千人。

乙卯夏四月,明總兵張承胤使人來求地,拒之。令各佐領屯田積穀。秋閏八月,帝長子褚英卒。先是太祖將授政於褚英,褚英暴伉,眾心不附,遂止。褚英怨望,焚表告天,為人所告,自縊死。冬十月,遣將征渥集部東格里庫路,得萬人。是歲,釐定兵制,初以黃、紅、白、黑四旗統兵,至是增四鑲旗,易黑為藍。置理政聽訟大臣五,以扎爾固齊十人副之。於是歸徠日眾,疆域益廣,諸貝勒大臣乃再三勸進焉。

天命元年丙辰春正月壬申朔,上即位,建元天命,定國號曰金。諸貝勒大臣上尊號曰覆育列國英明皇帝。命次子代善為大貝勒,弟子阿敏為二貝勒,五子莽古爾泰為三貝勒,八子皇太極為四貝勒。命額亦都、費英東、何和里、扈爾漢、安費揚古為五大臣,同聽國政。諭以秉志公誠,勵精圖治。扈爾漢巡邊,執殺盜葠者五十餘人。明巡撫李維翰止我使者綱古里、方吉訥。乃取獄俘十人戮於境上,綱古裡等得歸。

秋七月,禁五大臣私家聽訟。命扈爾漢、安費揚古伐東海薩哈連部,取三十六寨。

八月,渡黑龍江,江冰已合,取十一寨,徇使犬路、諾洛路、石拉忻路,並取其人以歸。

二年丁巳春正月,蒙古科爾沁貝勒明安來朝,待之有加禮。

是歲,遣兵取東海散居諸部負險諸島,各取其人以歸。

三年戊午二月,詔將士簡軍實,頒兵法。壬寅,上伐明,以七大恨告天,祭堂子而行。分兵左四旗趨東州、馬根單二城,下之。上帥右四旗兵趨撫順。明撫順游擊李永芳降,以為總兵官,轄輯降人,毀其城。明總兵張承胤等來追,回軍擊斬承胤等,班師。

五月,復伐明,克撫安等五堡,毀城,以其粟歸。

七月,入雅鶻關,明將鄒儲賢等戰死。

冬十月,東海虎爾哈部部長納哈哈來歸,賜賚有差。使犬各部路長四十人來歸,賜宴賞賚,並授以官。

四年己未春正月,伐葉赫,取二十餘寨。聞有明師,乃還。明經略楊鎬遣使來議罷兵,覆書拒之。楊鎬督師二十萬來伐,並徵葉赫、朝鮮之兵,分四路進。杜松軍由東路渡渾河出撫順、薩爾滸,劉綎軍由南路入董鄂。偵者以告。上曰:「明兵由南來者,誘我南也。其北必有重兵,宜先破之。」命諸貝勒先行。

三月甲申朔,清旦,師行。大貝勒代善議師行所向。四貝勒皇太極言:「宜趨界凡,我有築城萬五千人,役夫多而兵少,慮為所乘。」額亦都曰:「四貝勒之言是也。」遂趨界凡。向午,至太蘭岡,望見明兵,分千人援界凡。界凡之騎兵已乘明師半渡谷口,擊其尾,回守吉林崖。杜松留師壁薩爾滸,而自攻吉林崖。我軍至,役夫亦下擊,薄明軍。是時,上至太蘭察兵勢,命大軍攻薩爾滸,垂暮墮其壘,入夜夾攻松軍。松不支,及其副王宣、趙夢麟等皆死。追北至勺琴山,西路軍破。是日,馬林軍由東北清河、三岔至尚間崖。乙酉,代善聞報,以三百騎赴之。馬林斂軍入壕,外列火器,護以騎兵,別將潘宗顏屯飛芬山相犄角。上率四貝勒逐杜松後隊,殲其軍,聞馬林軍馳至。上趨登山下擊,代善陷陣,阿敏、莽古爾泰麾兵繼進,上下交擊,馬林遁,副將麻巖戰死,全軍奔潰。移攻飛芬,上率騎突入,斬宗顏,西北路軍破,葉赫兵遁。是時劉綎南路之軍由寬甸間道敗我戍將五百人,乘勢深入。上命扈爾漢將千兵往援,戍將托保以餘兵會之。丙戌,復命阿敏將二千人繼往。上至界凡,刲八牛祭纛。丁亥,命大貝勒代善、四貝勒皇太極南御,遇綎精騎萬餘前進。四貝勒以突騎三十奪阿布達里岡,代善冒杜松衣幟入其軍,軍亂,四貝勒馳下會戰,斬綎,又敗其後軍。乘勝至富察,綎監軍道康應乾以火器迎戰,大風起,煙焰返射,復大破之,應乾遁,朝鮮兵降。凡四日而破三路明兵。其北路李如柏之軍,為楊鎬急檄引還,至虎欄,遇我游騎二十人,登山鳴螺,呼噪逐之,如柏軍奔迸,踐斃又千餘人。甲辰,釋朝鮮降將姜弘立歸,以書諭其國主。

四月,遂築界凡。遣兵徇鐵嶺,略千人。

五月,朝鮮使來報謝。

六月,先是遣穆哈連收撫虎爾哈部遺民,至是得千戶,上出城撫之,賜以田廬牛馬。上率兵攻開原,克之,斬馬林等,殲其軍,還駐界凡。

秋七月,明千總王一屏等五人來降,暨前降守備阿布圖,各予之官。上攻鐵嶺,克之。是夕,蒙古喀爾喀部來援葉赫,敗之,追至遼河,擒其貝勒介賽。

八月己巳,徵葉赫。葉赫有二城,貝勒金臺什守東城,其弟布揚古、布爾杭古守西城。分軍圍之,隳其郛,穴城,城摧,我軍入城。命四貝勒領金臺什之子德爾格勒諭降再四,金臺什終不從,乃執而縊之。布爾杭古降。布揚古不遜,殺之。葉赫亡。師還駐界凡。

冬十月,蒙古察哈爾林丹汗使來,書辭多嫚,執其使。喀爾喀五部來使約伐明,上使大臣希福等五人蒞盟。旋有五部下屬人來歸,上卻之。

是歲,明以熊廷弼為經略。

五年庚申春正月,上報書林丹汗,斥其嫚。執我使臣。上亦殺其使。

二月,賜介賽子克什克圖、色特希爾裘馬,令其更代為質。

三月,論功,更定武爵。丙戌,左翼都統總兵官、一等大臣費英東卒,上臨哭之。

夏六月,諭樹二木於門,欲訴者懸其辭於木,民情盡達。

秋八月,上伐明,略沈陽,明兵不戰而退,乃還。

九月甲申,皇弟穆爾哈齊卒,車駕臨奠,因過費英東墓賜奠。

冬十月,自界凡遷於薩爾滸。

是歲,明神宗崩,光宗立,復崩,熹宗立,罷經略熊廷弼,以袁應泰代之。

六年辛酉春二月,上伐明,略奉集堡,至武靖營。

三月壬子,上大舉攻明沈陽,以舟載攻具,自渾河下。沈陽守禦甚備,環濠植簽,我軍拔簽猛進,明軍殊死戰,陣斬總兵賀世賢以下。乙卯,入沈陽。復敗其援軍總兵陳策等於渾河,敗總兵李秉誠於白塔鋪,援軍盡走。庚申,乘勝趨遼陽。袁應泰引水注濠,環城列砲,督軍出戰,不支而退,守城樓。壬戌,我右翼軍毀閘,左翼軍毀橋,右翼傅西城升陴,左翼聞之,畢登。明軍猶列炬巷戰,達旦皆潰,袁應泰自焚死,御史張銓被執,不屈死。癸亥,入遼陽。遼人具乘輿鼓樂迎上,夾道呼萬歲。命皇子德格類徇遼以南,所至迎降,兵宿城上,不入民舍。

六月,左翼總兵官、一等大臣額亦都卒,上臨奠,哭之慟。

秋七月壬寅,宴有功將士,酌酒賜衣。鎮江城人殺守將佟養真,降於明將毛文龍。

十一月乙卯,命阿敏擊毛文龍,敗之。喀爾喀部臺吉古爾布什來降。明復以熊廷弼為經略。

七年壬戌春正月甲寅,上伐明,攻廣寧。丙辰,克西平堡。明軍三萬來御,擊敗之,斬其總兵劉渠、祁秉忠,巡撫王化貞遁,游擊孫得功以城降。庚申,上入廣寧,降其城堡四十,進兵山海關,熊廷弼盡焚沿途村堡而走。乃移軍北攻義州,克之,還駐廣寧。蒙古厄魯特部十七貝勒來附,上宴勞之,授職有差。喀爾喀五部同來歸。

二月癸未,上還遼陽。遼陽城圮,遷於太子河濱。

秋七月乙未朔,一等大臣安費揚古卒。

八年癸亥春正月壬辰朔,蒙古扎魯特部巴克來朝,遣與質子俱還。

夏四月癸酉,遣皇子阿巴泰、德格類、皇孫岳託率師討扎魯特貝勒昂安,以其殺我使人也。昂安手巂孥遁。達穆布逐之,中槍卒。我軍憤,進殺昂安父子,並以別部桑土妻子歸。

六月,戒諸女已嫁毋凌其夫,違者必以罪。

冬十月丁丑,一等大臣扈爾漢卒,上臨哭之。

九年甲子春正月,喀爾喀貝勒恩克格爾來朝,求內遷,許之,以兵遷其民。

二月庚子,皇弟貝勒巴雅拉卒。上遣庫爾纏等與科爾沁臺吉奧巴盟,勿與察哈爾通。

四月,營山陵於東京城東北陽魯山,奉景祖、顯祖遷葬焉,是曰永陵。

五月,毛文龍寇輝發,戍將楞格禮、蘇爾東安追擊殲之。

秋八月壬辰,總兵官、一等大臣何和里卒,上聞之慟,曰:「天何不遺一人送朕老耶!」毛文龍之眾屯田于鴨綠島,使楞格禮襲其眾,殲之。

十年乙丑春正月癸亥,命皇子莽古爾泰率師至旅順,擊明戍兵,隳其城。

二月,科爾沁貝勒寨桑以女來歸四貝勒皇太極為妃,大宴成禮。

三月庚午,遷都沈陽,凡五遷乃定都焉,是曰盛京。遣喀爾達等征瓦爾喀,歸,降其眾三百。

夏四月己卯,宗室王善、副將達硃戶、車爾格征瓦爾喀,凱旋,宴勞備至。

六月癸卯,毛文龍兵襲耀州,戍將揚古利擊敗之。

秋八月,遣土穆布城耀州,明師來攻,擊走之,獲馬七百。命博爾晉征虎爾哈,降其戶五百,雅護征卦爾察部,獲其眾二千。毛文龍襲海州張屯寨,戍將戒沙擊走之。上著酒戒頒於國中。

十年己卯,皇子阿拜、塔拜、巴布泰徵虎爾哈,以千五百人歸。

十一月庚戌,科爾沁奧巴告有察哈爾之師,遣四貝勒皇太極及阿巴泰以精騎五千赴之,林丹汗遁。

是年,明使高第為經略,驅錦西人民入山海關。寧前道袁崇煥誓守不去。

十一年丙寅春正月戊午,上起兵伐明寧遠。至右屯,守將遁,收其積穀。至錦州,戍將俱先遁。丁卯,至寧遠。寧前道袁崇煥偕總兵滿桂、副將祖大壽嬰城固守。天寒土凍,鑿城不隳,城上放西洋砲,頗傷士卒,乃罷攻。遣武訥格將蒙古兵攻覺華島,奪舟二千,盡焚其軍儲,班師。

二月壬午,上還沈陽,語諸貝勒曰:「朕用兵以來,未有抗顏行者。袁崇煥何人,乃能爾耶!」

夏四月丙子,徵喀爾喀五部,為其背盟也,殺其貝勒囊奴克,進略西拉木輪,獲其牲畜。

五月,毛文龍兵襲鞍山驛及薩爾滸,戍將巴布泰、巴篤禮敗之,擒其將李良美。丁巳,科爾沁貝勒奧巴來朝,謝援師也。上優禮之,封為土謝圖汗。

六月,上書訓辭與諸貝勒。

秋七月,上不豫,幸清河湯泉。

八月丙午,上大漸,乘舟回。庚戌,至愛雞堡,上崩,入宮發喪。在位十一年,年六十有八。天聰三年葬福陵。初謚武皇帝,廟號太祖,改謚高皇帝,累謚承天廣運聖德神功肇紀立極仁孝睿武端毅欽安弘文定業高皇帝。

論曰:太祖天錫智勇,神武絕倫。蒙難艱貞,明夷用晦。迨歸附日眾,阻貳潛消。自摧九部之師,境宇日拓。用兵三十餘年,建國踐祚。薩爾滸一役,翦商業定。遷都沈陽,規模遠矣。比於岐、豐,無多讓焉。


\end{pinyinscope}