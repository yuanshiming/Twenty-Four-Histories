\article{列傳二百八十七}

\begin{pinyinscope}
遺逸一

李清李模梁以樟王世德閻爾梅萬壽祺鄭與僑

曹元方莊元辰王玉藻李長祥王正中董守諭

陸宇弟宇巇江漢方以智子中德等錢澄之

惲日初郭金臺硃之瑜沈光文陳士京吳祖錫

太史公伯夷列傳,憂憤悲嘆,百世下猶想見其人。伯夷、叔齊扣馬而諫,既不能行其志,不得已乃遁西山,歌採薇,痛心疾首,豈果自甘餓死哉?清初,代明平賊,順天應人,得天下之正,古未有也。天命既定,遺臣逸士猶不惜九死一生以圖再造,及事不成,雖浮海入山,而回天之志終不少衰。迄於國亡已數十年,呼號奔走,逐墜日以終其身,至老死不變,何其壯歟!今為遺逸傳,凡明末遺臣如李清等,逸士如李孔昭等,分著於篇,雖寥寥數十人,皆大節凜然,足風後世者也。至黃宗羲等已見儒林傳,魏禧等已見文苑傳,餘或分見於孝友及藝術諸傳,則當比而觀之,以見其全焉。

李清,字心水,號映碧,興化人。天啟辛酉舉人,崇禎辛未進士,授寧波府推官。考最,擢刑科給事中,同日上兩疏:一言御外敵當戰守兼治,不當輕言款;御內寇當剿撫並用,不當專言撫。一言治獄不宜置失入,而獨罪失出,因論尚書劉之鳳不職狀。尋以天旱,復疏言此用刑鍛鍊刻深所致,語侵尚書甄淑,淑遂劾清把持,詔鐫級,調浙江布政司照磨。無何,淑敗,即家起吏科給事中。疾朝臣日競門戶,疏言:「國家門戶有二:北門之鎖鑰,以三協為門戶;陪京之扃鍵,以兩淮為門戶。置此不問,而閧堂斫穴,長此安底?」疏入,不報。

京師陷,福王建號南京,遷工科都給事中。見朝政日壞,官方大亂,乃疏言:「大仇未雪,凡乘國難以拜官者,義將慚慟入地,宜急更前轍,以圖光復。」又憤時議以偏安自足,抗疏曰:「昔宋高之南渡也,說者謂其病於意足,若陛下於今日,其何足之有?以河、洛為豐、沛,則恭皇之舊封也,為恭皇所已有而不有,則不足;以金陵為長安,則高帝之始基也,為高帝所全有而不有,則不足。臣深望陛下無忘痛恥,以此志為中外倡也。儻陛下弛於上,則諸臣必逸於下,先帝之深仇,將安得而復哉?且宋之南渡,猶走李成,擒楊麼,以靖內制外。今則獻、瑤交熾,兩川危於累卵,汀、潮、南贛,並以警聞。北有既毀之室,南無可怡之堂,臣竊為陛下危之!」疏上,報聞而已。

有司始謚莊烈帝為思宗,清言廟號同於漢後主禪,請易之。又請補謚太子、二王及開國、靖難並累朝死諫諸臣,或以為迂,嘆曰:「士大夫廉恥喪盡矣!不於此時顯微闡幽,激發忠義之氣,更復何望耶?」清事兩朝,凡三居諫職,章奏後先數十上,並寢閣不行。

尋遷大理寺左寺丞,遣祀南鎮,行甫及杭,而南都失守矣。乃由間道趨隱松江,又渡江寓高郵,久乃歸故園,杜門不與人事。當道屢薦不起,凡三十有八年而歿。清忠義蓋出天性,莊烈帝之變,適在揚州,聞之,號慟幾絕。自是每遇三月十九日,必設位以哭。嘗曰:「吾家世受國恩,吾以外吏,蒙先帝簡擢,涓埃未報。」國亡後,守其硜硜,有死無二,蓋以此也。

晚著書自娛,尤潛心史學,為史論若干卷,又刪注南、北二史,編次南渡錄等書,藏於家。

李模,字子木,吳縣人。天啟乙丑進士,授東莞知縣。考最,入為御史。因劾論中官,謫南京國子監典籍。福王立,封四鎮為侯、伯,模上言:「擁立時,陛下不以得位為利,諸臣何敢以定策為功?甚至侯、伯之封,輕加鎮將。夫諸將事先帝未收桑榆之效,事陛下未彰汗馬之績,方應戴罪,何有勛勞?使諸將果忠義者,必先慰先帝殉國之靈,而後可膺陛下延世之賞。」報聞。尋改為河南道御史。馬、阮亂政,嘆曰:「事無可為矣!」即請告,不復出。杜門裏居,三十年如一日。幼與徐汧為總角交,汧死國事,為恤其家而存其孤,不渝舊好。年八十,卒於家。

梁以樟,字公狄,清苑人。與兄以棻、弟以桂,並知名,時號「三梁」。以樟負異才,八歲讀書家塾中,值壁裂,作壁裂歌云:「壁猛裂,龍驚出。」見者大奇之。十六歲補弟子員,受知左光鬥。崇禎己卯舉鄉試第一,明年成進士。命試騎射,進士皆書生,夙不習,以樟獨躍馬彎弓,矢三發,的皆應弦破,觀者嘆異。即授河南太康知縣。

中原盜起十餘年,所在荼毒,督撫莫能辦,率倡撫議,茍且幸無事,盜且服且叛。而河南比年大旱蝗,人相食,民益蜂起為盜。人為以樟危,僉都御史史可法以其有經世略,獨勸之行。抵任,探知境內賊凡三十六窟,於是練鄉勇,修城堡,嚴保甲;募死士,入賊巢。伺賊出入。嘗夜半馳風雪中,帥健兒密搗賊壘,賊驚佚,擒其渠,毀巢而歸。居半載,境內賊悉平。調商丘,時李自成犯開封,不能破,乃東攻歸德。以樟嬰城血戰三日夜,城陷,妻張率家人三十口自焚死,事具明史。

以樟被重創,僕亂尸中,死復甦,商民救之出,奔淮上,被逮讞請室。賊入潼關,復渡河東犯,京師震動。以樟乃從獄中上疏:「請皇太子撫軍南京,輔以重臣,假便宜從事,系人心。倡召豪傑義旅,大起勤王兵。擇宗室賢才,分建要地,而重督撫權,行方鎮遺意,合力拒。」疏上,執政尼之。

迨出獄,而都城陷。福王立,以樟自德州、臨清南下,與各郡邑建義文武吏及諸豪士歃血盟,人皆感憤流涕,受約束待命。渡淮見可法,因建議:「山東、河北為江南籓蔽,若無山東、河北,是無中原、江北,無中原、江北,區區江南,豈能自守耶?今宜於河南北、山東,設三大鎮,仿唐節度使、宋經制招討使之制,以大臣文武兼資者為之。寬其文法,使自為戰守,而閣部大治兵,居中馭之。」又言:「北方人心向順,宜及時撫為我用,否則忠者不能支,黠者反戈相向矣。」前後奏記百數十。而馬士英專政,貨鬻官爵,用逆黨阮大鋮為兵部尚書,競立門戶,斥忠讜之士,君臣日夜酣樂。左良玉、高傑、劉澤清等各擁兵跋扈,莫能制。以樟知事不可為,憤鬱成疾,辭去。可法仍舉以樟為兵部職方司主事,經理開、歸。

未幾,揚州破,可法死,南都相繼潰。以樟遂與以棻遯跡寶應之葭湖,買田數十畝,躬耕自給。清初,召用勝國諸臣,以樟年才三十七,朝貴致書勸駕,不應。自築忍冬軒,日與張★M9、孫爾靜講學其中,四方之士,若閻爾梅、王猷定、劉純學、崔干城、僧松隱暨其鄉人王世德父子,時時過以樟劇飲,慷慨激昂,繼以涕泣。晚年偕喬出塵、陳鈺、硃克生、劉中柱結文字社。康熙四年七月十五日,端坐作論學數百言,擲筆而卒,年五十八。世德之子潔、源,集其理學、經濟諸書及詩、古文合為一編,曰梁鷦林先生全書,今傳世者,惟卬否詩集而已。

世德,字克承,自號霜皋,北平人。少襲錦衣衛指揮僉事。北部陷,拔刀將引決,為僕所奪,妻魏已率婦女赴井死,遂易僧服,與以樟偕隱。嘗憤野史誣罔,不可傳信後世,欷歔扼腕,作崇禎遺錄一卷,自序之,康熙間修明史,有司錄其副本上史館。三十二年,卒,年八十有一。子源,以手槁殉葬。

閻爾梅,字用卿,號古古,沛縣人。崇禎庚午舉人。李自成陷北京,爾梅上書請兵北伐,並盡散家財,結死士,為前驅。自成黨武愫至沛,屢使招爾梅,以碎牒大罵下獄,愫敗,乃免。赴史可法之聘,參軍事,首勸渡河復山東,不聽。時高傑為許定國所殺,河南大亂,爾梅又說可法西行鎮撫之。傑部將約束待命,可法為設提督統其眾,而自退保揚州。爾梅力阻之,請開幕府徐州,號召河南北義勇,得以一成一旅規畫中原。又請空名告身數百紙,乘時布發,視忠義為鼓勵,俾逋寇叛帥不得以逾時渙散,少有睥睨。策皆不行,遂貽以書而去。

及可法殉節,爾梅走淮安,就劉澤清、田仰,畫戰守策,復不聽。師入淮,爾梅率河北壯士伏城外,眾懼阻,羽士陶萬明特庇之。巡撫趙福星以書招,爾梅痛哭謝之。乃散其眾,遁海上,祝發,稱蹈東和尚。復走山東,聯絡四方魁傑,謀再舉。又至河南,至京師,以山東事發被捕,下濟南獄,脫走還沛。名捕急,弟爾羹、侄御九皆就逮,妻、妾同自縊。

爾梅乃託死夜遁,變名翁深,字藏若,歷游楚、蜀、秦、晉九省。過關中,與王弘撰等往還。北至榆林,從寧夏入蘭州。凡十年,獄解,始還。未幾,為仇家所攀,復出亡,龔鼎孳救之,得免。北謁思陵,又東出榆關。還京,會顧炎武,復游塞外。至太原,訪傅山,結歲寒之盟。爾梅久奔走,歷艱險,不少阻。後見大勢已去,知不可為,乃還沛。寄於酒,醉則罵座。常慨然曰:「吾先世未有仕者,國亡,破家為報仇,天下震動。事雖終不成,疾風勁草,布衣之雄足矣!」遂高歌起舞。泣數行下。居數歲卒。年七十有七。

爾梅博學善詩,有白耷山人集。

萬壽祺,字介若,世稱年少先生,徐州人。與爾梅同郡,又同歲生,同舉鄉試,志節皆

同,既同舉事。南都破,江以南義師雲起。沈自炳、戴之俊、錢邦芑起陳湖,黃家瑞、陳子龍起泖,吳易起笠澤,皆與會師,謀恢復。兵潰,壽祺被執,不屈,將及難,有陰救之者,囚系月餘,得脫。乃渡江歸隱,築室浦西,妻徐、子睿,灌園以自給。髡首被僧衣,自稱明志道人、沙門慧壽,而飲酒食肉如故。時渡江而南,訪知舊,吊故壘。遺民故老過淮陰者,亦輒造草堂,流連歌哭,或淹留旬月。雖隱居,固未嘗一日忘世也。順治九年,卒。

壽祺善詩、文、書、畫,旁及琴、劍、棋、曲、雕刻、刺繡,亦靡弗工妙。爾梅論有明一代書,推為第一。著有隰西草堂集。

初,爾梅、壽祺同謀舉事,一起江北,一起江南,先後相呼應。及事敗,爾梅出走,思得一當。壽祺留江、淮觀世變,不幸先死。爾梅獨奔走三十餘年,亦終無所就。後世稱「徐州二遺民」,常為之太息云。

鄭與僑,字惠人,號確菴,濟寧人。五歲父歿,母張以祖遺田讓之仲,獨取遺書一篋授僑,曰:「兒讀此,可飽也!」與僑發奮力學,崇禎丙子舉於鄉。時流寇充斥山左,與僑以濟寧為漕艘咽喉地,倡義與城守張世臣、舉人孟瑄並力殺賊,城賴以完。有賊郭升者,將至濟寧州,吏議迎款,囑與僑草表,力拒乃止。及賊至,與僑率鄉人殲之,遂徙家淮陽。

史可法方開府淮上,聞與僑名,奏為儀真令,而吏部以其前守濟寧功,改除揚州府推官。揚州為興平伯高傑列籓地,其將卒多驕橫,稍不當意,抽刀剚人,與僑悉裁之以法。巡按御史何綸薦以推官監江、海軍,駐通州。

江南失守,與僑奉母之武林,總督張存仁、經略洪承疇奇其才,欲官之,皆謝不起。後歸濟上,立社教授生徒,絕口不談時事。嘗遍游秦、晉、川、蜀、荊、楚、吳、越諸勝,著有確菴稿、丹照集、爭光集、濟寧遺事、秦邊記要等書。卒,年八十有四。自為壙志。

曹元方,字介皇,海鹽人。父履泰,明兵部侍郎,以忠直著。元方,崇禎癸未進士,南京建號,授常熟知縣。時大學士馬士英擅國政,有薦元方署職方司事者,士英亦藉元方名,冀往謁附己,元方訖不往。上疏言原遵定制補外吏,語侵士英,士英怒,卒與令常熟。常熟為吳中煩劇邑最,當金陵草創,所在兵與民交狃無寧晷。元方措兵餉,惜民力,俱帖然,邑稱治。

金陵敗,棄官歸,履泰先獲譴謫戍,亦適歸。父子相謂,於義不可晏然以居。元方先變姓名,間道入閩,至建寧,謁唐王。即授吏部文選司主事,晉驗封司郎中。頃之,履泰亦由海道至,即授太常卿,晉兵部右侍郎。父子俱以忠義激發,間關來,一時咸偉之。

當是時,鄭芝龍久以桀寇內附,崇其秩號,姑息為養驕,至是益甚,志叵測。元方抗疏,自請出視江上師,閱封守,欲從外為重內計。得召對,加御史銜,賜白金,揮涕以行。至浦城,則江上潰兵接踵狼狽下,元方倉卒走,計後圖。履泰從唐王趨贛州,遇兵,投身崖石下,絕復甦。舁至僧舍,展轉至浦城,父子得相見。

履泰疾甚,先歸,旋卒於家。元方聞,乃亟歸,微服挈母及妻子行,寄食旅舍中。久之,事稍定,卜居硤石村,築草堂,自號耘庵。以老卒,年八十有二。

莊元辰,字起貞,晚字頑菴,鄞人,學者稱漢曉先生。賦性嚴凝,不隨人唯阿。崇禎丁丑進士,授南京太常博士。甲申之變,一日七至中樞史可法之門,促以勤王,福王立,議推科臣,總憲劉宗周、掌科章正宸皆舉元辰為首,而馬士英密遣私人致意曰:「博士曷不持門下刺上謁相公?掌科必無他屬。」峻拒之。中旨僅授刑部主事。已而阮大鋮欲興同文之獄,元辰曰:「禍將烈矣!」遽行,未幾而留都亡。

錢肅樂之起事也,元辰破家輸餉,時降臣謝三賓為王之仁所脅,以餉自贖。及肅樂與之仁赴江上,三賓潛招兵,眾疑之。明經王家勤謂肅樂曰:「浙東沿海皆可以舟師達鹽官,倘彼乘風而渡,列城且立潰矣,非分兵留守不可。」肅樂曰:「是無以易吾莊公者。」於是共推元辰任城守事,分兵千人屬之,以四明驛為幕府,家勤及林時躍參其事。元辰日耀兵巡諸堞里,人呼為「城門軍」,三賓不敢動。乃迎魯王於天臺,鄞始解嚴。

晉吏科都給事中,遷太常卿。上疏言:「殿下大仇未雪,舉兵以來,將士宣勞於外,編氓殫藏於內,臥薪嘗之不遑,而數月來,頗安逸樂。釜魚幕燕,撫事增憂,則晏安何可懷也?敵在門庭,朝不及夕,有深宮養優之心,安得有前席借箸之事,則蒙蔽何可滋也?天下安危,託命將相,今左右之人,頗能內承色笑,則事權何可移也?五等崇封,有如探囊,有為昔時佐命元臣所不能得者,則恩膏何可濫也?陛下試念兩都黍離麥秀之悲,則居處必不安;試念孝陵、長陵銅駝荊棘之慘,則對越必不安;試念青宮二王之辱,則撫王子何以為情;試念江干將士列邦生民之困,則衣食可以俱廢。」疏入,報聞。已又言中旨用人之非,累有封駁,王不能用。

時三賓夤緣居要,而馬士英又至,元辰言:「士英不斬,國事必不可為!」貽書同官黃宗羲、林時對云:「蕞爾氣象,似惟恐其不速盡者,區區憂憤,無事不痛心疾首,以致咳嗽纏綿,形容骨立。原得以微罪,成其山野。」遂乞休。

未幾,大兵東下,乃狂走深山中,朝夕野哭。元辰故美須眉,顧盼落落,至是失其面目,巾服似頭陀,一日數徙,莫知所止,山中人亦不復識。忽有老婦呼其小字曰:「子非念四郎邪?」因嘆曰:「吾晦跡未深,奈何?」順治四年,疽發背,戒勿藥,曰:「吾死已晚,然及今死猶可。」遂卒。

王玉藻,字質夫,江都人。崇禎癸未進士,授慈溪知縣。少詹項煜以從逆亡命,玉藻及慈民馮元飆均出其門,遂匿於馮氏。慈人斃煜於水,玉藻置不問。有明士習重闈誼,或以為過,玉藻曰:「吾豈能為向雄之待鍾會哉!夫君臣之與師友,果孰重?」聞者悚然。

金陵破,魯王監國,玉藻乃與沈宸荃起兵,晉御史,仍行縣。復募義勇,請赴江上自劾,略謂:「今恃以自保者,惟錢唐一江,待北兵渡江而後御,曷若御之於未渡之先?臣原以身先之!」乃解縣事,以兵科都給事往軍前。時駐兵江上者,有方國安、王之仁、孫嘉績、熊汝霖、章正宸、鄭道謙、錢肅樂、沈光文、陳潛夫、黃宗羲,咸各自為軍,兵餉交訌,莫敢先進。既不予玉藻以餉,復陳劃地分餉,又不聽,玉藻乃力請還朝。

既入諫垣,上封事十餘,略謂:「北兵之可畏者在勇,而我軍之可慮者在怯,怯由於驕,兵驕由於將驕。今統兵之將,無汗馬之勞,輒博五等之封,安得不啟以驕心?驕則畏戰,非稍加裁抑,恐無以戢其囂陵之氣。」又謂:「宜用海師窺吳淞,以分杭州北兵之勢。又劉宗周、祁彪佳諸臣,宜加褒忠之典。」以是不為諸臣所喜,乃力求罷職。時元辰為太常,固乞留之,謂:「古人折檻旌直,今令直臣去國,豈國家之福!」玉藻感其言,供職如初。

浙東再破,玉藻追魯王蹕,弗及,自投於池,水涸,不得死,乃以黃冠遯於剡溪。資糧俱盡,採野葛為食。妻李,遼東巡撫植女,知書明大義,在浙右時,屢脫簪珥佐軍興;偕入剡溪,命二子方岐、方嶷拾墮樵,不以窮厄易操。適四明山寨競起義軍,以書致玉藻,玉藻思乘間入舟山,為偵騎所遏,不果往。每臨流讀所作詩,輒激勵慷慨,仰天起舞,或朝夕悲歌,與門人熊亦方相和答。繼亦方以癲死,玉藻歸隱北湖,誓不易衣去發,作絕詞以逝。遺命不冠而斂。

李長祥,字研齋,達州人。崇禎癸未進士。初以諸生練鄉勇助城守,後選庶吉士,吏部薦備將帥之選。或曰:「天子果用公,計安出?」嘆曰:「不見孫白穀往事乎?今惟有請便宜行事,雖有金牌,亦不受進止。平賊後,囚首闕下受斧鉞耳!」聞者咋舌。賊日逼,上疏請急令大臣輔太子出鎮津門,以提調勤王兵。不果行,而京師潰,為賊所掠,乘間南奔。

福王立,改監察御史,巡浙鹽。魯王監國,加右僉都御史,督師西行,而江上師又潰。魯王航海去,長祥以餘眾結寨上虞之東山。時浙江諸寨林立,四出募餉,居民苦之。獨長祥與張煌言、王翊三營,且屯且耕,井邑不擾。監軍鄞人華夏者,為之聯絡布置,請引舟山之兵,連大蘭諸寨,以定鄞、慈五縣,因下姚江,會師曹娥,合偁山諸寨以下西陵。僉議奉長祥為盟主,刻期將集,而為降紳謝三賓所發,引兵來攻。前軍張有功被執,死。中軍與百夫長十二人,期以次日縛長祥為獻。晨起,十二人忽自相語:「奈何殺忠臣?」折矢扣刃,偕誓而遯。

長祥匿丐人舟中,入紹興城。居數日,事益急,復遯至奉化,依平西伯朝先。朝先亦蜀人,得其助,復合眾於夏蓋山,晉兵部左侍郎。請合朝先之眾,聯絡沿海,以為舟山衛。張名振忌之,襲殺朝先,長祥僅免。舟山破,亡命江、淮間,總督陳錦捕得之,安置江寧。未幾,乘守者之怠,逸去。由吳門渡秦郵,奔河北,遍歷宣府、大同,復南下百粵。晚歲,始還居毗陵,築讀易堂以老。

王正中,字仲手為,保定人。崇禎丁丑進士。魯王監國,以兵部職方司主事攝餘姚縣事。時義軍猝起,市魁、里正得一劄付,輒入民舍括金帛,郡縣不敢誰何。正中既視事,令各營取餉必經縣,否則以盜論。

總兵陳梧渡海掠餘姚,正中遣民兵擊殺之,諸營大譁,責正中擅殺大將。黃宗羲言於監國曰:「梧借喪亂以濟其私,致犯眾怒,是賊也。正中守土,當為國保民,何罪之有?」議乃息。張國柱、田仰、荊本徹各率所部過姚江,舳艫蔽空而下,以正中嚴備,不敢犯,皆帖帖趣行。國柱後從定海入,縱兵焚掠,正中單騎入其軍,呵止之,國柱迄不得逞。尋擢監察御史,諸軍從浙西來會,一聽約束,眾倚之若嚴城焉。

尋以株連系獄,論死。獄中有閩人柯仲蜅者,精星象,正中欲從受業,援黃霸從夏侯勝授經事為說,數年講習不怠,洞悉天官、律呂、度數諸書,復從黃宗羲學壬遁、孤虛之術。宗羲嘆曰:「傳吾絕學者,仲手為一人耳!」遂造監國魯元年丙戌大統歷以進。浙東亡,避竄山中,貧不能自存,傍鑒湖佃田五畝,佐以醫卜自給。康熙六年,卒,年六十九。著有周易註、律書詳註。

董守諭,字次公,鄞縣人。舉人。魯王監國,召為戶部貴州司主事。時熊汝霖、孫嘉績首事起兵,然皆書生,不知調度。乃迎方國安、王之仁,授之軍政,凡原設營兵、衛軍俱隸之。孫、熊所統,惟召募數百人。

方、王兵既盛,反惡當國者有所參決,因而分餉分地之議起。分餉者,正兵食正餉,田賊之出也,方、王主之;義兵食義餉,勸捐無名之徵也,熊、孫諸軍主之。分地者,某正兵,支某邑正餉;某義兵,支某邑義餉也。魯王令廷臣集議,方、王司餉者,皆至殿陛譁爭,守諭曰:「諸君起義旅,咫尺天威,不守朝廷法乎?」乃稍退。守諭又進曰:「義餉有名無實,以之饋義兵,必不繼。即使能繼,誰為管庫?今請以一切稅供悉歸戶部,計兵而後授餉,覈地之遠近,酌給之後先,則兵不絀於食,而餉可以時給也。」方、王雖不從,然所議正,無以難也。

之仁請收漁船稅,守諭曰:「今日所恃者人心耳,漁戶巳辦漁丁稅矣,若再苛求,民不堪命,人心一搖,國何以立?」久之,又請行稅人法,請塞金錢湖為田,官賣大戶祀田贍軍,三疏皆下部議,兵士露刃以待覆,守諭力持不可。之仁大怒,謂:「行朝大臣不敢裁量幕府,戶曹小臣敢爾阻大事邪?」檄召守諭,將殺之,魯王不能禁,令且避。守諭慷慨對曰:「司餉守正,臣分也。生殺出主上,武寧雖悍將,何為者?臣任死王前,聽武寧以臣血濺丹墀可耳!」於是舉朝忿怒,曰:「之仁反邪,何敢無王命而害餉臣!」之仁乃止。

明年,莊烈帝大祥,守諭請謁朝堂哭,三軍縞素一日,遷經筵日講官,兼理餉事。魯王航海,守諭不及從,遂遯跡荒郊,旋卒。著有攬蘭集。

陸宇,字周明,鄞縣人。諸生。慷慨尚氣節。時有弟子訟其師,師不得直。宇詣文廟,慟哭伐鼓,卒直其師而後止。明亡,嘗與黃宗羲謀舉事,其所與計畫者,皆四方知名士。其城西田舍,衣復壁柳車,雜賓死友。計敗,喜事乃益甚。江湖間多傳其姓名,以為異人。

南都破,甬東師起,宇毀家紓餉。翁洲又破,宇捐金與諜者,令訪死事消息。張肯堂之孫以俘至,亟治橐饘入獄視之,語其弟宇巇使為脫系。董志寧之喪在海上,宇致而葬之。旋為降卒所誣,捕入省獄,獄具,宇無所詿誤,脫械出門,未至館而卒。

宇以好事盡其家產,室中所有,惟草薦敗絮及故書數百卷。訃聞,家人整理其室,得布囊於亂書之下,發而視之,則赫然人頭也。宇巇識其面目,捧之而泣曰:「此故少司馬篤庵王公頭也!」初,司馬兵敗,梟城闕,宇思收葬之,每徘徊其下。一日,見暗中有叩首而去者,跡之,走入破室。宇曰:「子何人?」其人曰:「餘毛明山,曾以卒伍事司馬,今不勝故主之感耳!」宇相與流涕,而詣江子雲計所以收其頭者。子雲名漢,錢肅樂部將也。失勢家居,會端陽競渡,游人雜沓,子雲紅笠握刀,從十餘人登城遨戲。至梟頭所,問守卒曰:「孰戴此頭也者?」卒以司馬對。子雲佯怒曰:「嘻!吾怨家也,亦有是日乎?」拔刀擊之,繩斷墮地,宇、明山已豫立城下。方是時,龍舟噪甚,人無回面易視者,宇以身蔽,明山拾頭雜儔人而去。宇祀之書室,蓋十二年矣,而家人無知者。至是宇巇始瘞之。

宇巇,宇第五弟,字春明。負才自喜,俯視一切。宇風格棱棱不可犯,而宇巇稍濟之以和,故世人親之如夏日冬日之分。然其刻意勵行,雖嚬笑皆歸名節,則一也。丙戌後,棄諸生與諸遺民游,荒亭木末,時聞野哭。

同裏秀才杜懋俊,仗義死難,藏其遺孤。桐城方授,避地來鄞,宇巇館之湖樓中。授卒,宇巇經紀其喪,收拾遺文以致其家。性嗜異書,晚年,家既貧,不能具寫官,乃手鈔之,瀕病不倦。從子官山左,令其訪東萊趙士喆遺書,垂歿,尚以其書未至為恨。自棄諸生,即練衣蔬食,叢林以為佞佛,爭勸之披緇,宇巇笑不答。及遺命不作佛事,眾始瞿然。卒,年六十六。著觀日堂集八卷。

漢,錢塘人。為肅樂所倚恃,授以都督僉事總兵官。師至閩,幾下福州,漢功為多。侍郎馮景第之乞師日本也,請與偕行。及歸,漢曰:「東師必不出也!」已而果然。肅樂既卒,漢侍母居鄞,種蔬自給,四壁無長物,惟餘肅樂所贈寶刀一而已。每語及肅樂,則淚淋淋下,抑鬱終。

方以智,字密之,桐城人。父孔炤,明湖廣巡撫,為楊嗣昌劾下獄,以智懷血疏訟冤,得釋,事具明史。以智,崇禎庚辰進士,授檢討。會李自成破潼關,範景文疏薦以智,召對德政殿,語中機要,上撫幾稱善。以忤執政意,不果用。京師陷,以智哭臨殯宮,至東華門,被執,加刑毒,兩髁骨見,不屈。

賊敗,南奔,值馬、阮亂政,修怨欲殺之,遂流離嶺表。自作序篇,上述祖德,下表隱志。變姓名,賣藥市中。桂王稱號肇慶,以與推戴功,擢右中允。扈王幸梧州,擢侍講學士,拜禮部侍郎、東閣大學士,旋罷相。固稱疾,屢詔不起。嘗曰:「吾歸則負君,出則負親,吾其緇乎?」

行至平樂,被縶。其帥欲降之,左置官服,右白刃,惟所擇,以智趨右,帥更加禮敬,始聽為僧。更名弘智,字無可,別號藥地。康熙十年,赴吉安,拜文信國墓,道卒,其閉關高座時也。友人錢澄之,亦客金陵,遇故中官為僧者,問以智,澄之曰:「君豈曾識耶?」曰:「非也。昔侍先皇,一日朝罷,上忽嘆曰:『求忠臣必於孝子!』如是者再。某跪請故,上曰:『早御經筵,有講官父巡撫河南,坐失機問大闢,某薰衣,飾容止如常時。不孝若此,能為忠乎?聞新進士方以智,父亦系獄,日號泣,持疏求救,此亦人子也。』言訖復嘆,俄釋孔炤,而闢河南巡撫,外廷亦知其故乎?」澄之述其語告以智,以智伏地哭失聲。

以智生有異稟,年十五,群經、子、史,略能背誦。博涉多通,自天文、輿地、禮樂、律數、聲音、文字、書畫、醫藥、技勇之屬,皆能考其源流,析其旨趣。著書數十萬言,惟通雅、物理小識二書盛行於世。

子中德,字田伯,著古事比。以智構馬、阮之難,中德年十三,撾登聞鼓,訟父冤。父出亡,偕諸弟徒步追從。中通,字位伯,精算術,著數度衍,見疇人傳。中屨,字素伯,幼隨父於方外,備嘗險阻,著古今釋疑。

錢澄之,字飲光,原名秉鐙,桐城人。少以名節自勵。有御史巡按至皖,盛儀從,謁孔子廟,諸生迎迓門外。澄之忽前扳車,御史大駭,止車,因抗聲數其穢行。御史故閹黨,方自幸脫「逆案」,內懼不敢究其事。澄之以此名聞。是時復社、幾社始興,比郡中主壇坫者,宣城沈壽民,池陽吳應箕,桐城則澄之及方以智,而澄之又與陳子龍、夏允彞輩聯云龍社,以接武東林。澄之體貌偉然,好飲酒,縱談經世之略。嘗思冒危難,立功名。

阮大鋮既柄用,刊章捕治黨人,澄之先避吳中,妻方赴水死,事具明史。於是亡命走浙、閩,入粵,崎嶇險絕,猶數從鋒鏑間支持名義不少屈。黃道周薦諸唐王,授吉安府推官,改延平府。桂王時,擢禮部主事,特試,授翰林院庶吉士,兼誥敕撰文。指陳皆切時弊,忌者眾,乃乞假,間道歸里。結廬先人墓旁,環廬皆田也,自號曰田間,著田間詩學、易學。

澄之嘗問易道周,依京房、邵雍說,究極數學,後乃兼求義理。其治詩,遵用小序首句,於名物、訓詁、山川、地理尤詳。自謂著易、詩成,思所以翊二經者,而得莊周、屈原,乃復著莊屈合詁。蓋澄之生值末季,離憂抑鬱無所洩,一寓之於言,故以莊繼易,以屈繼詩也。又有藏山閣詩文集。卒,年八十二。

惲日初,字仲升,號遜菴,武進人。崇禎癸酉副榜。久留京師,應詔上備邊五策,不報。知時事不可為,乃歸隱天臺山。兩京亡,唐王立福州,魯王亦監國紹興,吏部侍郎姜垓薦日初知兵,魯王遣使聘之,固辭不起。大兵下浙,避走福州;福州破,走廣州;廣州復破,乃祝發為浮圖,復至建陽。

是時唐王被執死,魯王亦敗走海外,湖廣何騰蛟、江西楊廷麟等皆前後覆滅,而明遺臣尚擁殘旅,遙奉永歷。金壇人王祈聚眾入建寧,屬縣多響應。日初曰:「建寧,入閩門戶,能守,則諸郡安,然不扼仙霞關,建寧終不守也。欲取仙霞,宜先取蒲城。」乃遣長子楨隨副將謝南雲先趨蒲城,失利,皆死。而御史徐云兵連入數州縣,銳甚,日初說令夜入蒲城,自督兵繼進。會大雷雨,人馬沖泥淖,行不能速,軍遂潰。建寧被圍,王使兵部尚書揭重熙赴援。日初上書,請逕取蒲城,斷仙霞嶺餉道,徐與圍中諸將夾擊之。重熙巡至邵武,不能進,建寧遂破,王祈力戰死。日初收殘卒走廣信,尋入封禁山中,數日糧盡,喟然曰:「天下事壞散已數十年,不可救正。然莊烈帝殉社稷,薄海茹痛,小臣愚妄,謂即此可延天命。今乃至此,徒毒百姓,何益?」遂散眾,獨行歸常州。久之,張煌言與鄭成功軍薄江寧,敗走。訛傳張弟鳳翼乃日初門人,從師匿,縣官將收捕,日初色如常,曰:「吾當死久矣。」既而事解。卒,年七十有八。

少與楊廷樞等交,於百氏無所不窺,尤喜宋儒書。及從劉宗周游,學益進,嘗上書申

救,義聲震天下。丙戌後,累至山陰哭祭,為之行狀,近十萬言。晚服浮圖服,而言學者多宗之。無錫高世泰重葺東林書院,日初與同志習禮其間。知常州府駱鍾泰屢求見,不納。去官後,與一見,言中庸要領,喜而去,曰:「不圖今日得聆大儒緒論也!」

次子桓,在建寧被掠,不知所終;少子格,字壽平,見藝術傳。

郭金臺,字幼隗,湘潭人,本姓陳氏,名湜。年十五,遭家難,賴中表郭氏卵翼得脫,遂為繼。弱冠有聲黌序間,萬歷間,兩中副車。崇禎朝,屢以名薦,不起;例授官,亦不拜。既南渡,隆武鄉試登賢書,督師何騰蛟論薦,授職方郎中。再起監軍僉事,有司敦迫,皆以母老病辭不就。避跡山中,然於時事多所論列。一二枕戈泣血之士,崎嶇嶺海,經營措置,不遺餘力。當是時,潰卒猖獗,積尸盈野,百里無人煙。金臺請於督師,命偏裨主團練,力率鄉勇,鍛矛戟,峙芻糗,鄉人全活者以數萬計。

清初,當局特疏薦於朝,力請得免。晚授徒衡山,深衣幅巾,足不履戶外,絕口不談世事。惟論列當時殉難諸人,輒欷歔流涕。康熙十五年,以疾卒於家,年六十有七。自題其墓曰「遺民郭某之墓」。著有石村詩文集,五經駢語,博物匯編。

硃之瑜,字魯興,號舜水,餘姚人,寄籍松江。少有志概,九歲喪父,哀毀逾禮。及長,精研六經,特通毛詩。崇禎末,以諸生兩奉徵闢,不就。福王建號江南,召授江西按察司副使,兼兵部職方司郎中,監方國安軍,之瑜力辭。臺省劾偃蹇不奉詔,將逮捕,乃走避舟山,與經略王翊相締結,密謀恢復。渡海至日本,思乞師。魯王監國,累徵闢,皆不就。又赴安南,見國王,強令拜,不為屈,轉敬禮之。

復至日本,時舟山既失,之瑜師友擁兵者,如硃永祐、吳鍾巒等皆已死節,乃決蹈海全節之志,遂留寓長崎。日人安東守約等師事之,束脩敬養,始終不衰。日本水戶侯源光國厚禮延聘,待以賓師,之瑜慨然赴焉。每引見談論,依經守義,曲盡忠告善道之意。教授學者,循循不倦。

日人重之瑜,禮養備至,特於壽日設養老之禮,奉幾杖以祝。又為制明室衣冠使服之,並欲為起居第,之瑜再辭曰:「吾藉上公眷顧,孤蹤海外,得養志守節,而保明室衣冠,感莫大焉!吾祖宗墳墓,久為發掘,每念及此,五內慘烈。若豐屋而安居,豈我志乎?」乃止。

之瑜為日人作學宮圖說,商榷古今,剖微索隱,使梓人依其圖而以木模焉,棟梁枅椽,莫不悉備。而殿堂結構之法,梓人所不能通曉者,親指授之。度量分寸,湊離機巧,教喻縝密,經歲而畢。文廟、啟聖宮。明倫堂、尊經閣、學舍、進賢樓,廊廡射圃,門戶墻垣,皆極精巧。又造古祭器,先作古升、古尺,揣其稱勝,作簠、簋、籩、豆、登、鉶之屬。如周廟欹器,唐、宋以來,圖雖存而制莫傳,乃依圖考古,研覈其法,巧思默契,指畫精到。授之工師,或未洞達。復為揣輕重,定尺寸,關機運動,教之經年,不厭煩數,卒成之。於是率儒學生,習釋奠禮,改定儀注,詳明禮節,學者皆通其梗概。日人文教,為之彬彬焉。之瑜居日本二十餘年,年八十三卒,葬於日本長崎瑞龍山麓。日人謚曰文恭先生,立祠祀之,並護其墓,至今不衰。

之瑜嚴毅剛直,動必以禮。平居不茍言笑,唯言及國難,常切齒流涕。魯王敕書,奉持隨身,未嘗示人,歿後始出,人皆服其深密謹厚云。著有文集二十五卷,釋奠儀注一卷,陽九述略一卷,安南供役紀事一卷。

沈光文,字文開,一字斯菴,鄞人。少以明經貢太學,福王授太常博士,浮海至長垣,晉工部郎。閩師潰而北,扈從不及。聞粵中建號,乃走肇慶,累遷太僕卿。由潮陽航海至金門,閩督李率泰方招徠故國遺臣,密遣使以書幣招之,光文焚書返幣。知粵事不可支,卜居於泉州海口,浮家泛宅。忽颶風大作,舟人失維,飄泊至臺灣。時鄭成功尚未至,而臺灣為荷蘭所據,光文受一廛以居,與中土音耗隔絕。成功克臺灣,知光文在,大喜,以賓禮見。時海上諸遺老,多依成功入臺,光文與握手相勞苦。成功致廩餼,且以田宅贍之。

成功卒,子錦嗣,改父之臣與政,軍亦日削。光文作賦諷之,幾不測。乃變服為浮屠,逃入臺北鄙,結茅羅漢門山中以居,山旁有伽溜灣者,番社也。光文教授生徒自給,不足,則濟以醫。嘆曰:「吾二十載飄零絕島,棄墳墓不顧者,不過欲完發以見先皇帝於地下耳,而卒不克,命也夫!」已而錦卒,諸鄭復禮之如故。

康熙癸丑年,王師下臺灣,閩督姚啟聖招之,光文辭。啟聖貽書問訊曰:「管寧無恙?」且許遣人送歸鄞,會啟聖卒,不果。而諸羅令李麟光,賢者也,為粟肉之繼,旬日一候門下。時耆宿已盡,而寓公漸集,乃與宛陵韓文琦,關中趙行可,無錫華袞、鄭廷桂,榕城林奕丹,山陽宗城,螺陽王際慧等結詩社,所稱福臺新詠者也。尋卒於諸羅。

陳士京,字佛莊,先世本奉化硃氏,遷鄞,改姓陳。熊汝霖薦授職方司郎中,監三衢總兵陳謙軍。謙使閩,偕行,而唐、魯方爭頒詔事,謙死,遂遯之海上。鄭芝龍聞名,令與其子成功游,芝龍以閩降,成功不肯從,異軍特起,士京實贊之。已而汝霖奉魯王室,復以公義說成功,始致寓公之敬。會魯王上表粵中,成功亦欲啟事於粵,使士京往,加都御史,歸。

魯王入浙,特留閩,與成功相結,以為後圖。成功盛以恢復自任,賓禮遺臣,其最致敬者,尚書盧若騰,侍郎王忠孝,都御史章朝薦,及徐孚遠、沈光文,與士京數人而已。久之,見海師無功,粵事亦日壞,乃築鹿石山房於鼓浪嶼中,感物賦詩以自遣。尋卒。

吳祖錫,字佩遠,吳江人。崇禎壬午副貢。時中原大亂,料京師必危,預謀勤王。欲身任浙西,以浙東屬之許都,約未定而變作,故鎮臣陳洪範隨王師下江南,與有舊,自言其降出於不得已,而以奇策告祖錫,立出遺產四萬金畀之。已而薙發令下,遽委之去,改名鉏,字稽田。從陳子龍、徐孚遠謀恢復。偵事杭州,為仇家縛送江寧,羈系獄中,復髡而縱之。魯王授職方郎中,桂王亦官之如魯,仍往來吳、越間。

副將馮源淮駐軍嘉興,乃與結納,冀有所為。其部屬董某司詗察,馮耳目也,亦故與厚善。比孚遠歸自海外,有所謀,密館之。事稍聞於馮,馮遣董詣問,祖錫遽前握其手曰:「徐公在此,若欲見之乎?」董驚曰:「徐公果在此,顧肯令我見耶?」即引見,董叩頭泣下,道其鄉慕,矢不相負。因以譌言報馮,而陰遣弋船衛孚遠浮海去。

海師入江,祖錫實導之,且連歲在金陵,隱為之助。乃復遭刊章,事解,志不稍挫。將詣滇南,而先之鄖陽。時鄖陽十三營,尚保殘寨,乃勸出師撓楚以救滇。顧十三營已疲敝,不能用其策也。

桂王既入緬甸,思追從,道阻,不得達。復返吳。游中州,更由秦入楚,卒無所遇。康熙己未,客膠州大竹山,鬱鬱靡所騁。會懷宗忌日,慟哭嘔血死,遺命槁葬山中,年六十有二。距明亡已三十有五年矣。

凡明末三王遺臣逸士,其初或起義,或言事,各有所謀,其後或蹈海,或居夷,志不少沮,皆先後云亡。及祖錫死,徐枋為之傳曰:「自吳子歿,而天下絕援溺之望。」亦可悲矣!故以附於明末遺臣之末。


\end{pinyinscope}