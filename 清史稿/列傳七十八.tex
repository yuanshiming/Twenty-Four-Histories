\article{列傳七十八}

\begin{pinyinscope}
海望三和莽鵠立杭奕祿傅鼐陳儀劉師恕焦祈年李徽

王國棟許容蔡仕舢

海望,烏雅氏,滿洲正黃旗人。初授護軍校。雍正元年,擢內務府主事。累遷郎中,充崇文門監督。八年,擢總管內務府大臣,兼管戶部三庫,賜二品頂戴。九年,遷戶部侍郎,仍兼管內務府,授內大臣。十一年,命偕直隸總督李衛勘浙江海塘,與衛議奏在海寧尖、塔兩山間建石壩,使海潮外趨,並在仁和、海寧兩縣境改建大石塘。上命浙江總督程元章相度遵行。又奏請設專官總轄,令駐防將軍、副都統協同監修,及議敘在工人員工價以銀米兼發,並從之。十三年,振武將軍傅爾丹虐兵婪索事發,命海望赴北路軍營逮治。尋命辦理軍機事務。

世宗疾大漸,召同受顧命。是時辦理軍機事務鄂爾泰、張廷玉、訥親、班第、索柱、豐盛額、莽鵠立、納延泰及海望凡九人。高宗即位,命尚書徐本入直。旋設總理事務處,命鄂爾泰、廷玉總理,本、訥親及海望協辦,班第、納延泰、索柱差委辦事。尋命海望署戶部尚書。海望還自軍前,奏言:「鄂爾坤發遣罪人種地無實效,且恐生事,當改發他處。」世宗謂:「鄂爾坤方駐兵,當可彈壓,海望奏非是。」高宗以海望奏下總理事務處議,議上,上諭曰:「海望奏,前奉皇考申飭。朕推皇考之意,蓋以發遣罪人,皆身獲重罪,今令軍前種地,乃所以保全之。其中若有冤抑,自應聲明具奏寬釋。如但以不善開墾,遂爾改發內地,此曹既獲重罪,又不肯急公趨事,轉得遂其僥幸之心;且如以兵代之,兵若以不能力田為辭,則將移內地之民耕邊塞之地乎?此事之斷不可行者。海望心地純良,但識見平常,所奏豈可盡以為是?議覆觀望游移,後當以此為戒。」

乾隆二年,泰陵工成,授拖沙喇哈番世職。尋罷總理事務處,復設辦理軍機處,海望仍為辦理軍機大臣。敘勞,復加拖沙喇哈番世職。四年,加太子少保。初,上命停捐例,廷臣議但留收穀捐監,俾各省積穀備荒。六年,御史趙青藜請並停之,復下廷臣議,請仍其舊。海望奏:「外省收捐繁難,原議各省捐貯穀數三千餘萬石,今報部者僅二百五十餘萬石,不足十之一。不若停各省捐穀,令在部交銀,轉撥各省買穀,俟倉貯充盈,請旨停止。」上命在部交銀,在外交穀,聽士民之便。諭謂:「地方積穀不厭其多,賑恤加恩,亦所時有,正未易言倉貯充盈也。」

海望久充崇文門監督,御史胡定奏言:「崇文門徵稅,有掛錘、頂秤諸名,百斤作百四五十斤。稅額雖未增,實已加數倍。雜物自各門入,恣意需索,更數倍於稅額。外省各關,如杭州北新關,自南而北十餘里,稽察乃有七處,留難苛索,百倍於物價。蓋由官吏務欲稅課浮於舊額,吏胥藉得恣睢無忌,請敕嚴禁。」上曰:「海望領崇文門稅務,侭收侭解,盡行入官,因而見其獨多。如定所奏,種種苛索,朕信其必無。外省關課,應令督撫嚴察。」海望旋調禮部尚書。

十年,上以海望精力漸衰,罷辦理軍機。十四年,復調戶部尚書。十七年,以建築兩郊壇宇發帑過多,與侍郎三和等自行奏請嚴議,當奪官,上寬之。二十年,卒,遣散秩大臣博爾木查奠茶酒,賜祭葬,謚勤恪。

三和,納喇氏,滿洲鑲白旗人。初授護軍校,累遷一等侍衛。乾隆六年,授總管內務府大臣,遷戶部侍郎,調工部,復調還戶部。十四年,擢工部尚書。尋降授侍郎,調戶部,復調還工部。三十二年,授內大臣。三十八年,卒,賜祭葬,謚誠毅。

莽鵠立,字樹本,伊爾根覺羅氏,滿洲鑲黃旗人。曾祖富拉塔,居葉赫,天聰時來歸,隸蒙古正藍旗。祖莽吉圖,從睿親王伐明,徇山東,圍錦州,擊敗洪承疇援兵;入關逐李自成至慶都;又從下雲南。累擢正藍旗滿洲梅勒額真,授世職三等阿達哈哈番。

莽鵠立,事聖祖,初授理籓院筆帖式。累遷員外郎,迭充右翼監督、滸墅關監督。世宗即位,命協辦理籓院侍郎,旋擢御史。莽鵠立精繪事,令恭繪聖祖御容。雍正元年,改入滿洲,以本族別編佐領,俾莽鵠立世管。

出巡長蘆鹽政,疏言:「長蘆諸商行鹽地,有額引不能銷者,有額外多銷者。請通融運銷,量增引目。」從之。二年,疏請元年積引寬限分銷,部議不允,再疏請,特許之。又疏言:「山東加增引目,州縣多寡不均。請減多增寡,以甦商困。」又疏言:「增復引目,視現辦商人按名均分。」上允之。三年,疏言:「山東灶丁,遵康熙五十二年恩詔,審丁不加賦。」下部議行。又疏請清察灶地,敕直隸、山東督撫遣員清丈。遷大理寺卿,再遷兵部侍郎,領鹽政如故。天津改衛為州,初議隸河間府。莽鵠立請改為直隸州,以武清、靜海、青縣屬焉。並丁入地議起,莽鵠立以山東灶丁丁多地少,請以其半入地,其半仍按丁徵賦。下部議,從之。四年,以御史顧琮巡視鹽政,仍命莽鵠立監理。尋調禮部,令與顧琮監造天津水師營房,工久未竣,上以責莽鵠立,調刑部,召還京。五年,復調禮部,仍署長蘆鹽政。

授甘肅巡撫。六年,師入西藏,諭莽鵠立赴西寧料理。西寧道劉之珍等誤軍興,總督岳鍾琪疏劾,上以責莽鵠立,解巡撫,召還京。署正藍旗滿洲副都統,兼管理籓院侍郎。七年,擢正藍旗蒙古都統。八年,命協同辦理直隸水利營田。十年,調鑲白旗滿洲都統。十三年,與都統襲英誠公豐盛額並命辦理軍機事務。高宗即位,改設總理事務處,莽鵠立與豐盛額罷直回本任。尋署工部尚書,又調正藍旗滿洲都統。乾隆元年,卒,賜祭葬,謚勤敏。

杭奕祿,完顏氏,滿洲鑲紅旗人。初授中書。雍正元年,授額外員外郎。未幾,補御史,即遷光祿寺少卿。三年,遷光祿寺卿。上蠲蘇州、松江田賦四十五萬,杭奕祿疏言:「此為未有殊恩。有田納賦,既邀蠲免;無田而佃種人田者,納租業主,亦宜酌減,俾貧富均霑實惠。」上謂此奏甚公,下廷臣議,定業戶免額一錢,佃戶免租穀三升。上命如議速行。擢左副都御史,仍兼管光祿寺。

五年,命與內閣學士任蘭枝使安南宣諭。初,雲南總督高其倬奏安南國界有百二十里舊屬內地,應以賭咒河為界,安南國王黎維祹奏辯。上命云貴總督鄂爾泰覆覈,予地八十里,以鉛廠山下小河內四十里為界,維祹復奏辯。上敕維祹毋以侵占內地為嫌,疑懼申辯。至是,復命杭奕祿等往諭意,未至,維祹上表謝罪。六年,命鄂爾泰以鉛廠山下地四十里予安南,別頒敕命杭奕祿等齎往宣諭。杭奕祿至鎮南關,維祹使出關迎。進次貂瑤營,維祹復使迎勞,請儀注,議行其國禮,五拜三叩。杭奕祿等持不可,乃請聽命。渡富良江至長安門,維祹跪迎。杭奕祿等捧敕入自中門,維祹率將吏等聽宣敕,敕曰:「朕前令守土各官清理疆界,未及於安南也。總督高其倬職任封疆,考志乘,訪輿論,知開化府與安南分界當在逢春里之賭咒河,乃奏聞設汛。王疏陳,復命總督鄂爾泰秉公辦理。鄂爾泰體朕懷遠之心,定界於鉛廠山下小河,縮地八十里。誠為仁至義盡,此皆地方大臣職分所當為。朕統馭寰區,凡屬臣服之邦,皆隸版籍。安南既列籓封,尺地莫非吾土,何必較論此區區四十里之地?若王以至情懇求,朕何難開恩賜與?祗以兩督臣定界時,王激切奏請,過於觖望,失事上之禮,朕亦無從施惠。頃鄂爾泰以王本章呈奏,詞意虔恭。王既知盡禮,朕自可加恩,將此地仍賜王世守,並遣大臣前往宣諭。王其知朕意!」宣畢,維祹行三跪九叩禮。杭奕祿等復宣諭恩德,維祹誓世世子孫永矢臣節。杭奕祿等使還,維祹送至長安門,餽贐杭奕祿等,不受。至鎮南關,維祹使齎謝表請轉奏。杭奕祿等還京師,疏聞,請宣付史館,允之。授刑部侍郎,署吏部尚書。

六年,湖南靖州諸生曾靜遣其徒張熙變姓名投書川陜總督岳鍾琪,略言清為金裔,鍾琪乃鄂王後,勸令復金、宋之仇,同謀舉事。鍾琪大駭,鞫熙,熙不肯言其實;乃置熙密室,陽與誓,將迎其師與謀,始得熙及靜姓名,奏聞。上命杭奕祿及副都統覺羅海蘭如湖南,會巡撫王國棟捕靜嚴鞫。靜言因讀呂留良評選時文論夷、夏語激烈,遣熙求得留良遺書,與留良子毅中,及其弟子嚴鴻逵,鴻逵弟子沈在寬等往還,沈溺其說,妄生異心。留良,浙江石門諸生,康熙初講學負盛名,時已前死。上命逮靜、熙、毅中、鴻逵、在寬等至京師。靜至,廷鞫,自承迂妄,為留良所誤,手書供辭,盛稱上恩德。上命編次為大義覺迷錄,令杭奕祿以靜至江寧、杭州、蘇州宣講。事畢,命並熙釋勿誅,戮留良尸,誅毅中並鴻逵、在寬等,戍留良諸子孫。高宗即位,乃命誅靜、熙。

七年,授杭奕祿鑲紅旗滿洲副都統。八年,命解部事,尋復補禮部侍郎,署前鋒統領。上命杭奕祿偕侍郎眾佛保宣諭準噶爾。九年,師征準噶爾,上慮陜、甘民或以用兵為累,命杭奕祿與左都御史史貽直、署內務府總管鄭渾寶,率庶吉士、六部學習主事、國子監肄業拔貢生等宣諭化導。尋命杭奕祿協辦軍需。十年,命署西安將軍,授欽差大臣,察閱甘、涼及山西近邊營伍。十一年,諭責杭奕祿驕奢放縱,擾累兵民,奪官,在肅州荷校。

乾隆元年,召至京師,授額外內閣學士,補工部侍郎,充世宗實錄副總裁。遣駐西藏辦事。四年,奏言:「西藏西南三千里外巴爾布國有三汗:一曰庫庫木,一曰顏布,一曰葉楞,雍正十一年嘗通貢。近三汗交惡,貝勒頗羅鼐宣諭罷兵,三汗聽命,使呈進部落戶口數,並貢金銀、絲緞、珊瑚、念珠諸物。」報聞。尋召還,調刑部。五年,擢左都御史,列議政大臣。十年,以老乞休,諭留之。十一年,上察其老憊,命致仕。十三年,卒。

傅鼐,字閣峰,富察氏,滿洲鑲白旗人。初授侍衛。雍正二年,授鑲黃旗漢軍副都統、兵部侍郎。三年,調盛京戶部侍郎。世宗在潛邸,夙知傅鼐好事,既即位,令隆科多察其為人。隆科多稱傅鼐安靜。傅鼐在上前嘗言隆科多子嶽興阿甚怨其父,謂「我家受恩深,當將生平行事據實奏聞,若稍有隱飾,罪更不可逭」。及隆科多被譴追贓,嶽興阿隱其父財產。上以與傅鼐言不符,疑傅鼐與隆科多交結,慮且敗,預為嶽興阿地。會傅鼐任侍衛時,浙江糧道江國英被劾,為關說,得銀萬餘。事發,上命奪官,械系逮詣京師,下刑部按治。讞上,免死,發遣黑龍江。

九年,召還,赴大將軍馬爾賽軍營效力。尋予侍郎銜,授參贊大臣。十年,準噶爾臺吉噶爾丹策零入寇,額駙策凌御之額爾德尼昭,噶爾丹策零大敗,自推河竄走。時馬爾賽駐拜里城,有兵萬三千。策凌檄速發兵斷噶爾丹策零歸路,馬爾賽不能用。傅鼐進曰:「賊敗亡之餘,可唾手取也!請發輕騎數千,俾率以戰,事成,功歸大將軍;事敗,原獨受其罪。」馬爾賽默然,再三言不應,至長跪以請,終不許。傅鼐憤甚,將所部出城逐敵。噶爾丹策零已遁走,得輜重、牛羊萬計。事聞,上誅馬爾賽,賚傅鼐花翎。

平郡王福彭代為大將軍,傅鼐參贊如故。噶爾丹策零既大創,不敢深入,師亦未能遠征。上召策凌及大將軍查郎阿詣京師廷議,莊親王允祿及策凌等主進討,大學士張廷玉等言不若先撫之,不順則進討。兩議上,上問傅鼐,傅鼐贊撫議。降旨罷兵,遣傅鼐偕內閣學士阿克敦、副都統羅密諭噶爾丹策零。噶爾丹策零欲得阿爾泰山故地,傅鼐力折之。十三年,使還,予都統銜,食俸。

高宗即位,命署兵部尚書,尋授刑部尚書,仍兼理兵部。乾隆元年,疏言:「刑罰世輕世重。我朝律例,頒布於順治三年,酌議於康熙十八年,重刊於雍正三年。臣伏讀世宗遺詔曰:『凡諸條例,或前本嚴而朕改從寬,此乃昔時部臣定議未協,朕與廷臣悉心斟酌而後更定,應照更定之例行;若前本寬而朕改從嚴,此以整飭人心風俗,暫行一時,此後遇事斟酌,若應照舊例者,仍照舊例行。』臣思聖心惓惓於此,蓋必有所軫念而未及更正者也。皇上以世宗之心為心,每遇奏讞,斟酌詳慎。臣見大清律集解附例一書,現今不行之例猶載其中,恐刑官援引舛錯,吏胥因緣為奸。請簡熟悉律例大臣,詳加覈議。律文律注,當仍其舊。所載條例,有今已斟酌改定者,應從改定;有應斟酌而未逮者,悉照舊章:務歸於平允,逐條繕摺,恭請欽定纂輯頒布。」得旨允行。又疏言:「斷獄引用律例,宜審全文。若摘引律語,入人重罪,是為深文周內。律載:『官吏懷挾私仇,故勘平人致死者,斬監候。』又載:『若因公事干連在官,事須問鞫,依法拷訊,邂逅致死者,勿論。』律意本極平允。數年來,各督撫遇屬員誤將在官人犯拷訊致死,輒摘引『故勘平人』一語,擬斬監候。尚書張照又奏準:『如將笞杖人犯故意夾拷致死二命以上,及徒流人犯四命以上,俱以故勘平人論。』不思既非懷挾私仇,於故勘之義何居?若謂在官之人本屬無罪,則必有誣告之人,應照律抵罪;若謂輕罪不應夾訊,命盜等案,當首從未分,安能預定為笞杖為徒流?若謂拷訊不依法,自有『決罰不如法』律在,致死二人、四人以上,當議以加等。請敕法司酌改平允。」下部議行。

是秋,以勒借商銀,回奏不實,奪官。尋命暫署兵部尚書。二年,授正藍旗滿洲都統。三年,坐違例發俸,發往軍臺效力。尋卒。

陳儀,字子翽,順天文安人。康熙五十四年進士,改庶吉士,散館授編修。為古文辭,治經世學,大學士硃軾器之。雍正三年,直隸大水,諸河泛濫,壞田廬。世宗命怡親王允祥偕軾相度濬治。王求諳習畿輔水利者,軾以儀對。延見,諮治河所宜先,儀曰:「硃子言治河先低處。天津為古渤海逆河之會,百川之尾閭。今南北二運河、東西兩澱盛漲,爭趨三岔口,而強潮復來拒之,牴牾洄漩而不時下,下隘則上溢,其勢宜然。故欲治河,莫如先擴達海之口。欲擴海口,莫如先減入口之水。入口之水減,則達海之口寬。北永定,南子牙,中七十二沽,皆得沛然入三岔口而東注矣。」四年春,從王行視水利,教令章奏皆出儀手。軾以憂歸,王薦於朝,命以侍講署天津同知。轉侍讀,擢庶子,仍署同知如故。

五年,王奏設水利營田四局,儀領天津局,兼督文安、大城堤工。二縣地卑下,積潦不消。是秋復大水,堤內外皆巨浸。儀購秫稭十餘萬束,立表下楗以御水。堤本民工,儀言於王,請發帑興修,招民就工代賑,堤得完固。南運河長屯堤地隸靜海,吏舞法,歲調發霸州、文安、大城民協修,百里裹糧,咸以為苦,儀為除其籍。畿輔大小諸河七十餘,疏故濬新,儀所勘定殆十六七云。

八年,擢侍講學士。時議設營田觀察使二員,分轄京東西,以督率州縣。命儀以僉都御史充京東營田觀察使,營田於天津。仿明汪應蛟遺制,築十字圍,三面開渠,與海河通。潮來渠滿,閉渠蓄水以供灌溉,白塘、葛沽間斥鹵盡變膏腴。豐潤、玉田地多沮洳,儀教之開渠築圩,皆成良田。十一年,大雨,山水暴發,沒田廬。儀疏聞,諭籌賑,即命儀董其事,凡賑三十四萬餘口。十二年,轉侍讀學士。尋罷觀察使,還京師。

儀篤於內行,先世遺田數百畝,悉推以讓兄。既仕,分祿畀昆弟,周諸故舊。有故人子貧甚,囑門生為謀生業,事為人所訐,吏議當降調。乾隆二年,授鴻臚寺少卿。儀以老乞歸。七年,卒,年七十三。子玉友,雍正八年進士,官臺灣知府。勤其官,有惠政。

劉師恕,字艾堂,江南寶應人。父國黻,康熙二十一年進士,改庶吉士,授戶科給事中,歷督捕理事官。在戶科,建言民田畝有大小,地有上中下,請具載簡明賦役全書,明示天下。在督捕,詳考則例刊布之。往時以逃人為根,以一累百十,以逃案為市。取所歷州縣官職名待劾,弊不勝詰,皆剔除之,乃裁並兵部。改授鴻臚寺卿。

師恕,三十九年進士,選庶吉士,授檢討。累遷國子監祭酒。雍正元年,授貴州布政使。四年,遷通政使,轉左副都御史,擢工部侍郎。上以宜兆熊署直隸總督,調師恕禮部,協理總督事。五年,奏獲交河妖民孫守禮,嚴鞫治罪。上獎其遇事直達,不稍隱諱。師恕與兆熊議裁學政陋規,學政孫嘉淦言:「學政舊規,日得五十五兩,今減半即足用。」師恕言:「減至一兩亦不可行,當另奏撥解公費。」師恕與兆熊奏已與嘉淦會商裁革,嘉淦以實奏。上諭曰:「孫嘉淦非騷擾貪饕者比,爾等何不量至此?可仍循舊例而行。嘉淦,端士也,宜作成之。」初夏,保定諸府少雨,上以為憂。師恕等言:「今歲遇閏,此後得雨不遲。」上責其怠忽。尋奏裁驛站夫馬工料羨餘銀,上諭曰:「陋規自應裁,第當量情酌理為之,毋過刻,令後來地方諸事難於措辦也。」調吏部,仍留協理。大名諸生竇相可訴知府曾逢聖貪劣,布政使張適杖殺之,以獄斃報,兆熊、師恕匿不以聞。上命尚書福敏等按治得實,兆熊坐降調,上寬師恕,諭責其徇隱,命何世璂署直隸總督,仍令師恕協理。

七年,命師恕以內閣學士充福建觀風整俗使。八年,疏言:「海澄公舊以轄兵給印,後兵裁而印未繳。今海澄公黃應纘濫行印文,非所宜,當令繳銷。」並言外省世襲武職,年及二十,當令咨部引見,分京外學習。部議從之。十一年,師恕以病告,省觀風整俗使不復設。乾隆七年,寶應災,治賑,非貧民例不給。師恕族人諸生洞嗾不得賑者,閧堂罷市。上責師恕不能約束,奪官。南巡迎謁,賜侍讀學士銜。二十一年,卒。

是時廣東、湖南皆置觀風整俗使。焦祈年,字穀貽,山東章丘人。雍正元年進士,改庶吉士,授編修。考選云南道御史,擢順天府丞,權府尹,遷右通政。八年,命充廣東觀風整俗使,修建十府、二州書院,延通人為之師。濱海多盜,設策鉤捕,得劇盜百餘置諸法,盜差熄。奸民以符劄惑眾,擒治之,赦其株連者。西洋人置天主堂,使徙歸澳門。簡閱營伍,軍政以肅。擢光祿寺卿,召為順天府尹,旋調奉天。行次山海關,疾作,乞歸,卒於里。

李徽,字元綸,山西崞縣人。康熙五十二年,鄉試舉第一。雍正元年進士,改庶吉士,散館刑部主事。尋復授檢討。考選浙江道御史。是時遣御史巡察順天直隸諸府,順天、永平、宣化為一員,保定、正定、河間為一員,順德、廣平、大名為一員,徽巡察順德、廣平、大名三府。曾靜、張熙事起,上慮湖南士民為所惑,議遣使循行訓迪。以大學士硃軾薦,遣徽勸諭化導。尋授僉都御史,充湖南觀風整俗使。徽在官四年,察吏安民,能稱其職。坐事,降授倉監督。高宗即位,命復官,遽卒。

廣西學政衛昌績請設觀風整俗使,御史陳宏謀繼請。上諭宏謀等曰:「廣西通籍者本少,乃已有狂悖如謝濟世、陸生柟者,風俗薄劣可見。爾等不能端本澂源,躬先表率,而望秉鐸司教之官,家喻戶曉,易俗移風,所謂逐末而忘其本也。」議寢未行。

王國棟,字左吾,漢軍鑲紅旗人。康熙五十二年進士,改庶吉士,授檢討。累遷光祿寺卿。雍正初,查嗣庭、汪景祺坐文字謗訕見法。上謂浙江士習澆漓,四年,設浙江觀風整俗使,以授國棟。國棟至官,巡行宣諭,清逋賦,懲唆訟,飭營伍,嚴保甲,次第疏聞,上溫諭獎之。遷宗人府府丞。五年,上以浙江被水,米貴,命國棟同巡撫李衛發庫帑四萬,於杭州、嘉興、湖州三府修城、濬河、築堤,俾饑民就傭食力。國棟奏:「杭州至海寧塘河淤,當濬治。太湖堤閘及嘉興石塘多傾圮,當修理。冬春雨雪,工作多費,請俟九、十月水落興工。」上韙之。

尋擢湖南巡撫,以許容代為浙江觀風整俗使。上諭國棟曰:「初欲令爾在浙整飭數年,俾收成效。但湖南廢弛久,今以命爾,爾其勉之!」上命湖廣總督邁柱修兩省堤工。國棟疏言:「湘陰、巴陵、華容、安鄉、澧、武陵、龍陽、沅江、益陽九州縣環繞洞庭,居民築堤堵水而耕。地勢卑下,江漲反灌入湖,是岸沖決,現有四百餘處。正飭刻期完築,務加高培厚,工程堅固。」僉都御史申大成奏貴州屯田,民間賤價頂種,易啟紛爭。請仿民田買賣,畝納稅五錢,給照為業,並推行各省。國棟疏言:「湖南屯田瘠薄,應分別差等,微價頂種,令完稅五錢,給照如時價平買。未過戶者,視屯糧石稅五錢,已過戶者二錢。龍陽、武陵、長靖諸屯賦重,按券值兩稅三分。」均下部議行。

曾靜、張熙事起,上令侍郎杭奕祿至湖南會鞫。國棟聽靜自列,未窮究黨羽,允禩、允禟門下太監以罪徙廣西,流言於路,直隸、河南督撫俱疏上聞。國棟奏言:「湖南監送兵役未聞一語。」又茶陵民陳蒂西傳播流言,敕國棟按鞫,亦不得實證,坐是失上指,奪官,召還京。八年,命治刑部侍郎事,署山東巡撫。九年,河南祥符、封丘等縣水災,命往治賑。迭署江蘇、浙江巡撫。十年,仍還刑部。十二年,以議福建民藍厚正殺兄獄失當,吏議降調。十三年,復命署刑部侍郎。卒。

許容,河南虞城人。康熙五十年舉人,授陜西府穀知縣。內遷工部員外郎,考選廣西道御史。雍正元年,改會考府郎中,仍兼御史。出為直隸口北道,遷陜西按察使。劾河東巡鹽御史馬喀以積鹽變價入己,上奪馬喀官,命兼管河東巡鹽御史,按治。尋聞容刑逼商人,解容任,令總督岳鍾琪覆按。鍾琪言容無刑逼商人事,上擢浙江布政使。五年,代國棟為浙江觀風整俗使。尋偕廣東巡撫楊文乾清察福建倉庫。六年,遭母喪,給假治喪畢,命仍還浙江。旋擢甘肅巡撫,以蔡仕舢代為浙江觀風整俗使。容疏議更正律例,出贓過付人宜視完贓減二等,得贓者完贓減一等,倍完方減二等,連斃二命宜加等。上皆謂不當,責容愚妄。

八年,師征噶爾丹,上以容治軍需多推諉,命尚書查弼納赴陜西為之董理。及事竟,上諭容曰:「此次軍需,朕為挽將覆之轍,回已頹之波,救汝身家性命。較自御史五年內擢至巡撫之恩大矣!汝當知之。」上聞容追逋賦抵兵餉,限一年全完,民以大擾。諭曰:「朕念甘肅自軍興以來,輓運轉輸,資於民力,特將雍正八年額徵錢糧蠲免。容何得於蠲免之年行催徵之舉?令即停止。」九年,復以容查核錢糧過刻,諭毋累民。十二年,疏劾丁憂知府李綺虧空軍需,綺,衛兄也。上知容與衛有怨,戒容毋遷怒報復。容旋奏檄綺赴蘭州,虧空七千有奇,限半年回籍措繳。上諭曰:「所虧既有田房可抵,但當速遣回籍折變完補,何須勒限逼迫?」

乾隆元年,固原、環二縣歉收,容請借給貧民三月口糧,大口日三合,小口日二合。高宗諭曰:「政莫先於愛民。甘肅用兵以來,百姓急公踴躍。今值歉收,當加恩賑恤。汝治事實心,而理財過刻。國家救濟貧民,非較量錙銖時也。」尋,專筦軍儲大臣劉於義奏請加賑兩月,上責容褊隘卑庸,命解任。於義及陜西總督查郎阿劾容匿災殃民,奪官逮詣京師論罪,赦免。二年,署山西布政使。三年,調江蘇,署巡撫。四年,遭父喪,去官。

五年,命署湖南巡撫。請終喪,不許。服闋,真除。八年,以劾糧道謝濟世狂縱營私失實,奪官,發順義城工效力。事互詳濟世傳。九年,復命署湖北巡撫。御史陳大玠等疏諍,謂容既以欺罔得罪,不當復用,上命罷之。十五年,上巡中嶽,迎謁,復原銜。尋授內閣學士。以病乞歸,卒。

蔡仕舢,福建南安人。康熙三十二年舉人。五十八年,自刑部主事考選御史,出為浙江糧道。雍正六年,授僉都御史,充浙江觀風整俗使。七年,署巡撫。八年,坐事降調。上諭曰:「浙江風俗已漸改移,又有總督李衛善於訓導,不必再遣觀風整俗使。」仕舢旋卒。

論曰:海望、莽鵠立皆逮事聖祖,雍正、乾隆間參與政事。海望聞世宗末命,在軍機處較久,雖建樹未宏,要為當時親信大臣。杭奕祿使安南,傅鼐諭噶爾丹策零,皆不辱君命,傅鼐尤知兵。儀領屯田,有惠於鄉州。師恕、國棟等使車問俗,與民為安靜。以皆世宗特置之官,特謹而書之。杭奕祿又與史貽直宣諭陜西,非專官,貽直相高宗,故不著於斯篇。


\end{pinyinscope}