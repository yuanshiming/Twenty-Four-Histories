\article{列傳二百八十}

\begin{pinyinscope}
忠義七

張繼庚從弟張繼辛李翼棠等趙振祚趙起

馬善陳克家馬釗臧紓青竇元灝馬三俊張勛吳文謨

吳廷香孫家泰江圖悃程葆彭壽頤陳介眉亓祈年

唐守忠吳山俞焜戴煦張洵鍾世耀孫義

汪士驤錢松毛雝魏謙升金鼎燮巴達蘭布等

包立身王玉文孫文德李貴元等羅正仁陳起書

陳景滄何霖蹇諤趙國澍宋華嵩伯錫爾

張繼庚,字炳垣,江蘇江寧人。父介福,道光六年進士,湖南保靖縣知縣。繼庚少有志節,補諸生,幕游湖南。咸豐三年,從布政使潘鐸守長沙。圍既解,料賊必東竄,辭歸省母。江寧布政使祁宿藻方籌守御,稔其才,招與謀。繼庚慮兵不足,增募壯勇,舉諸生李翼棠等統之。明年,賊至,請仿古火城法,於城內開壕積薪,城上築兩墻,為孔以出火器。城下兩旁設牛皮柵,伏精兵以堵賊。時宿藻已卒,總督不能用。二月,城陷,繼庚率眾巷戰,從弟繼辛及李翼棠、侯敦詩等皆死。繼庚赴水不沉,旋陷賊中,為書算。自念死志已決,欲將有所為,乃以母託戚友,變姓名為葉芝發,陽與賊暱,盡得其虛實。會欽差大臣向榮軍至,因與諸生周葆濂、夏家銑及錢塘人金樹本謀結賊為內應,而使金和、李鈞祥、何師孟出報大營。有張沛澤者,悍賊也,同謀而中悔,首其事,家銑死之,繼庚以偽名免。

九月,復遣人上書向榮,言:「水西門賊所不備,有船可用,太平門近紫金山,越城亦易為力,緣城賊壘皆受約束。」既得報,益結死士張士義、劉隆舒、呂萬興、硃碩齡等,以待大軍。書七上,屢約屢爽。城中人情洶洶,事垂洩,繼庚泣謂其友曰:「事急矣!」夜縋入營,痛哭自請師期,諸將皆感動。張國樑欲留之,繼庚不可,歸而大軍復以雨雪不果至。他日繼庚出,遇沛澤於途,唶曰:「此葉芝發也!」執赴賊所,施嚴刑,不為動,徐曰:「我張炳垣,書生耳,焉預他事?沛澤食鴉片,懼我發之,乃誣我耶?」賊搜之,信,遂殺沛澤,繼庚被縶不得出。

明年二月,金和等引官兵易賈人服入城,與諸生賈鍾麟等伏神策門,殺巡更賊,以斧斷木柵,毀其半,賊驚走。亟舉砲,六品軍功田玉梅及敢死士張鴉頭先眾上城,斬守賊十餘人,援賊麕至,玉梅跳免。大索城中,鴉頭被獲,窮詰不得主名,乃益搒掠繼庚,楚毒備至。時廬州知府胡元煒降賊在坐,繼庚躍起謂曰:「若官江南,寧不知江南人孱弱,非老兄弟合謀,誰敢為內應者?」老兄弟,賊中呼楚、粵人之悍勇者也。賊信其言,繼庚索賊官冊一一指,賊輒殺之,橫尸東門者三十四人。賊旋悟,曰:「中汝計矣!」令速殺之。繼庚臨死,色不變,呼天者三,成絕命詞,有云:「拔不去眼中鐵,嘔不盡心頭血,籲嗟窮途窮,空抱烈士烈。殺賊苦無權,罵賊猶有舌。」遂車裂以死。事聞,贈國子監典籍,建專祠,予世職。

張士義,乳名鴉頭,江寧人。故無賴而有肝膽,能急人之急。在賊中與所素狎者醉歌,若無事然。繼庚遣劉隆舒招之,袖短刀二授之,曰:「汝能殺賊,當以功名顯。」士義慨然曰:「我何人,張先生義士乃下交,誓必殺賊,富貴非所望也!」繼庚獄急,趣士義速圖。眾請於大營,遣田玉梅等八人入城助之。咸豐四年二月二十二日夜,士義與劉隆舒、呂長興、硃碩齡等凡五十七人,乘晦登城。遇一賊手紅燈,騰身斫之,擲首城外以為信。復殺賊十餘人,而官軍終不進。乃下斬關,柵堅不可啟,擲火燒之,不燃。柵內賊起,抽矛刺之,環城賊皆起,角嗚嗚然,眾知事不濟,遂遁。明日,賊閉門大索,有沈獸醫者首之,士義等被執,窮其主使。士義叱曰:「欲殺則殺,主使不可得也!天下人皆欲殺汝,獨我哉?」遂與隆舒、長興、碩齡俱死。

是時繼庚以諸生舉義,鄉里士慷慨相從者:夏家銑,字季質,江寧人。工詩文。城陷,賊挾充書記,作詩罵賊,賊搜得之。時繼庚內應事洩,賊疑家銑知其謀,拷掠無所承,不知家銑實與聞也。賊誘之曰:「汝有父母妻子,以為質,則釋汝。」家銑時昏憊,遽以母妻對。賊至家,其妻蔡匿母,罵家銑曰:「汝母死且十年,何處得汝母耶?」遂與妻俱被戮死。

同預翻城之舉而未死者:金和,字亞匏,上元人。性兀傲,工詩賦,好聲色。縱酒,飲輒數斗。江寧失守,陷於賊,衣短後衣,與賊兵轟飲相爾汝,因廉得賊情。繼庚為其妻弟,與和通謀。和與賊稔,出入無所問,孑身叩向榮軍門,請以身質,家在賊中不顧也。事敗,和以質得脫。有秋蟪吟館集。

孫文川,字澂之,上元人。敏悟,工詩賦。洪秀全據金陵,以計奉母間道出,復入,與繼庚謀翻城應外兵。終日芒鞋手一筐如丐,奔走近賊地,不避風雪。得賊中曲折,具以報官軍,因是屢捷,而翻城事卒無成。嗣習互市案牘,知外人情偽。英人李泰國購輪舶助李鴻章戰,既乃要挾索費,不受中國進止。鴻章聞文川才,薦入都,盡發泰國陰謀,朝廷褫泰國總稅務司職,遣船回國,事得解。以功洊擢知府。著讀雪齋集。

周葆濂,字還之,江寧人。詩才清麗。陷城中,與內應事,謀洩,脫歸。選寶應訓導。著且巢詩存。

汪汝桂,字燕山,上元人。幼負奇力,或勸入伍及應武童試,皆不可。初陷賊中,追者至,手批殺一賊,擲過壕而免。田玉梅入城,汝桂與俱往還。習繪事,畫仕女尤工。

吳復成,字蔚堂,上元人。性慷慨,賈於粵久,咸豐初始歸。賊陷金陵,與人語多不辨,惟復成解之,以是為賊所信。因說賊設機杼,織緞匹,用匠十萬人,文弱陷賊者得以免;又說賊造船運柴薪,賊稱其能。婦嬰緣是遁者又六七千人。既,與繼庚謀內應,事洩,奔向榮軍,不知重也。及曾國籓欲諜賊虛實,或以復成薦,因蓄發入賊中,得曲折以告。曾國荃圍金陵,李秀成自蘇州來援,賊掘地道出攻,復成偵得之,報國荃為備,遂大破賊。以功敘縣丞,不就,卒以賈終。

胡恩燮,字煦齋,江寧人。與繼庚謀內應,出入賊中者三十六次。破衣草履,溷跡如丐。往往伏壕內,或潛立橋下堅冰中,屢瀕於危。母陷賊中,以奇計脫之。後以功敘知府。

田玉梅,字鼎臣,四川酉陽州人。入應事起,求敢內成者,吳復成以書抵玉梅,玉梅裹紅巾挺身從復成行。數日,出言賊情如繪,向榮乃信任之。奪門既無成,明日賊殺張士義等百餘人,不得主名,則令領石達開憑帖,無者逮訊。復成領數百紙貽諸同志。玉梅手一紙立通衢,發短言異,見賊往來反詰之,賊竟無知者,乃偕八人者俱出。後以功敘河南同知直隸州,補太康縣知縣。十年,英、法國聯軍犯天津,京師戒嚴,請濟師勤王,大府不許。自帥所部至汝陽沙官橋,為捻匪所阻,憤極死戰,被戕。恤贈太僕寺卿。諸人於是役皆冒死為之,例得附書,以竟事之本末。

趙振祚,字伯厚,江蘇武進人,順天宛平籍。道光十五年進士,改庶吉士,授編修。兩遇大考皆前列,二十二年,遷詹事府贊善。咸豐三年,寇陷金陵,蘇、常震動,振祚上書當事,原歸本籍辦團練。奏請,報可,遂歸。集貲置保衛局,募兵購械,仿行保甲,人心以安。常州北門瀕江,焦湖船屢出剽掠,積為民患。振祚乃擇要隘口岸立稽查卡房,並設水師戰艦,嚴備以待,境獲寧輯。六年,賊艎蔽江下,鎮江幾不守,避難者絡繹。振祚固結人心,訓練士卒,率眾詣丹陽,會督師者赴援,圍乃解。賞花翎,加翰林院侍讀銜。

時總督何桂清駐常州,郡人編修趙曾向出其門,振祚素輕之,以是常訐其短於桂清,遇事齮齕。嗣曾向被命佐常州團練,益掣其肘,不得已,請解事,保衛局遂廢。十年,和春軍潰丹陽,常州大警,桂清宵遁,曾向亦舉室渡江而北,於是紳民復請振祚出督團勇。是時兵單糧絕,寇氛日迫,事已不可為,復毅然誓眾固守,並率所練五十人出城招集潰勇。會北鄉石堰土盜蜂起,遂領眾往捕,以眾寡不敵,戰失利,力竭,死之。常州亦旋陷。事聞,贈太僕寺卿,予世職。

同治三年,李鴻章疏稱:「其六世祖尚書趙申喬為康熙時名臣,子姓分居蘇、常。江、浙淪陷,男女死者四十三人,其弟浙江經歷振禋亦死於難。」得旨,予振祚常州建祠,餘附祀。

振祚忼爽重節介,口素吃,遇不平事,憤懣謾罵,期期不避人,故多遭怨。然好獎借人才,人亦以此多之。善詩、古文詞,精漢學,著有明堂考一卷,文、詩集若干卷。

同族起,道光舉人。同時城守。城陷,命合室婦女自沉園池,遂整衣冠坐事。賊至,有識起者,勸令自全,大聲叱賊,引刀自剄。子諸生曾寅以身衛父,刃賊數人,被害。兄子浙江候補知縣祿保,罵賊,死尤烈。

馬善,字遇皋,長洲人,世居蘇垣北鄉。有智略,膂力過人。咸豐十年閏三月,金陵大營潰,總督遁,賊席卷而南。夏四月丁丑,蘇州陷。善先受檄主黃土橋團練,集七圖義勇三千人,朝夕訓練,庀水陸戰守具。聞變,嚴陣以待。明日,賊果至,迎擊金巷橋。又明日,賊大掠八字橋,又趣援之。越四日,賊分兩路至,一出齊門至宜橋,一出閶門至禪定橋。善率勇千人自當宜橋,遣子安瀾率勇數百當禪定橋,先後均有斬獲。賊將竄常熟,夜遣安瀾率千人潛至八字橋,盡括岸側灰窯遺棄磚瓦塞遠近橋下,居民已空,無知者。越數日,賊船至,不得過。城賊約滸關至青黛湖,合宜橋、禪定橋三路並進,善分兵拒之,而自擊青黛湖,失利,賊旋退。已而賊大至,善設伏青黛湖畔,遣弟增及安瀾誘賊入湖,伏發勝之,獲賊船十,俘賊首攀大福,梟其首,賊為奪氣。

偽忠王李秀成憤不得逞,大舉來攻,善盡銳御之,自辰至午,殺傷相當。賊退,團勇歸局午食,賊遽掩至,善率親兵迎戰,手刃騎馬賊三人,傷於胸,猶疾呼殺賊,飛鏃中頭角而踣。安瀾方赴常熟請軍火,馳歸斂之,面色如生。當是時詔舉團練,吳人脆弱,賊至則靡,獨善以能殺賊聞。恤贈知州銜,給世職。安瀾後從巡撫李鴻章軍,鄉導得力,卒復蘇州。

陳克家,字子剛,元和人。道光二十四年舉人。少英異,為桐城姚瑩所器重。抗心希古,落落寡合。文章自許北宋,儷體宗六朝,詩學黃庭堅。咸豐三年,挑教職。時金陵為賊據,欽差大臣向榮駐師城外,翼長福興阿聘克家入幕。福遷去,江南提督張國樑復聘之。十年閏三月,國樑檄克家主健勇營事,十五日,賊大至,督弁勇迎戰,兵敗死之。克家之死也,營中大亂,求尸不得。克家祖鶴,熟精明代事,為明紀一書,用通鑒義法,崇禎三年後猶闕,克家續成之,合為十六卷。

馬釗,字遠林,長洲人。與克家同歲舉人。治經學有名。咸豐三年,前江蘇巡撫許乃釗副向榮統兵金陵,釗入許營時,有川、楚兵所帶餘丁,率驍勇,而蘇垣空虛,釗建議募為一軍,得千餘人,號曰撫勇。粵匪劉麗川反嘉定,土匪周立春繼之,連陷青浦等六縣。向榮檄釗率撫勇卷甲赴之,至青浦,夜半,銜枚薄城,克之,獎內閣中書銜。事定,重赴金陵。十年春,浙江告急,偕總兵熊天喜赴援,復四安鎮、廣德州。奉調馳回,遇賊丹陽,戰白塔灣,中槍死。二人以文士從軍,卒死於陣,吳人稱之。

臧紓青,字牧菴,江蘇宿遷人。道光十一年舉人。自少倜儻好談兵,所交多不羈之士。當英吉利入寇,紓青見武備廢弛,人不知兵,寇至多被殘害,因團練鄉兵,凡萬人。嗣入靖逆將軍幕府,將軍主和,紓青獨主戰,後以和議成,奏獎同知銜,不受,曰:「以和受賞,不亦恥乎?」嘗以邳州知州勒捐案被牽涉,查辦大臣周天爵雪之。

時粵逆陷安慶,據江寧,淮南北捻匪乘釁為亂,聚黨多者至數千人,與粵逆互為聲勢。天爵因疏請紓青練勇剿匪,且聽自成一隊。匪素懾紓青威名,稱之曰「老虎兵」。所至撲滅解散,多原歸附效死者。天爵卒,副都御史袁甲三繼任,亦深倚之。累擢通判,賞四品銜。

先是桐城以三年十月失陷,士民先後乞援於圍廬州提督和春、圍舒城提督秦定三,幾一年,皆不應。甲三時駐兵臨淮,念桐人請救之殷,又欲取安慶以截江路,自請進剿。文宗以臨淮扼南北之要,不許。甲三於是疏請檄紓青剿辦,允之。

時侍郎曾國籓已克復武昌,破田家鎮,順流東下,使提督塔齊布、道員羅澤南進攻廣濟、黃梅。朝廷既允甲三請,復以國籓兵屢捷,於是命紓青速進兵潛、太接應。時和春、秦定三軍皆久無功,詔旨切責,令速破賊以圖會剿。紓青又得國籓書相期會,於是疾馳至桐,兩敗賊於大關、呂亭驛,追至城下,時四年十一月六日也。

紓青以舒、廬圍師率離城十餘里,不斷賊出入餉道,以故久無功。桐之南門通安慶,賊來援則當其沖,遂自率兵勇圍之,而令參將劉玉豹、同知李安中圍東門。時攻城之器未具,城堅不可猝拔。賊既敗於湖北,又懼桐城或破,則與湘軍成夾擊勢,悉力來援。紓青先後迎擊於王林莊、掛車河,皆勝之,追至陶沖驛,擒斬既多,獲械無算。卒以舒、廬軍不予接應,又不急攻城以分賊勢,賊用是得專事援桐。玉豹、安中又性懦,無能當賊,紓青至以「諸君不能戰,不能攻,又不能守,事事須我一人」誚之,弗恤也。

十七日,賊援大至,玉豹、安中卻走,城賊復突出西門焚營。紓青與諸生張勛殊死戰,殺三百餘賊,以後無繼者,賊伏遽起,紓青胸面間中二十餘創,死焉。紓青既死,賊復得志,武昌再陷。

紓青治兵有紀律,初,賊以土匪目官兵以惑民聽,至是一洗此恥。桐城破後,凡先以助餉團練,賊皆甘心焉。民以紓青來,秋毫無犯,雖被禍,無不感泣思之。事聞,賜三品卿銜,予騎都尉世職。後有竇元灝。

元灝,邳州人,咸豐元年舉人,援例為員外郎,分刑部。八年,捻賊大熾,竄徐州,邳當其沖,元灝集鄉團,先後偕知州畢培貞、周力城,都司濮楓等堵剿,擊斬甚多。十年,州城被圍,守御四晝夜,城賴以全。賊結幅匪大舉,由蘭、郯渡河,元灝與參將於殿甲合剿,被圍,力竭死。贈太僕寺卿銜,賞世職。

馬三俊,字命之,桐城人。祖宗璉,父瑞辰,皆進士,以經學顯名。三俊能世其家,顧屢困鄉舉。咸豐元年,以優行第一貢太學,又舉孝廉方正制科。三年,安慶失守,桐人恟懼,知縣遁去,奸民蜂起。官兵往來境上,亦乘亂為患。獨縣學生張勛誓死不避,三俊亦急起而坐鎮之,擒斬為首者十數人。又偕勛立法,勸富家給散貧者,亂稍定。

賊既陷安慶,盡趨江寧,諸統帥皆遠避,置安慶、蕪湖不堵截。三俊知賊之必回竄也,日夜在明倫堂訓練鄉兵,又時與勛往四鄉聯合團眾。於是桐城練勇,名聞江南北。賊犯太湖,與勛揚兵堵境上,賊莫測虛實,莫敢逼。已而賊攻江西,不克,回據安慶,修守備,桐人大恐。巡撫李嘉瑞駐廬州,前按察使張熙宇駐集賢關,皆畏安慶不敢至。

三俊上書巡撫,其略謂:「制寇之道,必能攻而後可守;守禦之策,必先據要害而後可保城池。全州不守,禍及湖南;岳州不守,禍及武昌;小孤不守,禍及安慶;安慶不守,然後禍及江寧、鎮江、揚州、大江南北。此明驗也。自粵西起事以來,賊之所破,多不守而破,非因守而破也;賊之所敗,多不戰而敗,非力戰而敗也。觀桂林、長沙、南昌、開封四省城,茍能死守,賊未有陷之者。六合小邑,殺賊數千,而賊不敢至。江浦、含山、許州皆以守而得全,不大可見乎?今江北全勢完固,虛實未為賊覺,而安慶之賊,又皆江西殘敗之餘,且未齊集。望於此時迅速進攻,而分兵守桐,以為接應。如安慶不利,當可退守桐城,以為舒、廬之障。此機一失,賊或竄桐、舒,以入廬州,則與北匪勾結,河南北東西、畿輔之地危矣。」巡撫韙其言,遣總兵恆興與熙宇合軍堵剿,實不前進。

十月,賊大至,熙宇、恆興兵皆卻走,三俊獨與勛率鄉團數百人拒之,不利。賊遂道桐城以入舒城,陷廬州,渡河而北,蔓延千里,皆如三俊所料。城陷時,三俊父被執不屈死。三俊以不孝不忠自責,誓復仇報國。

四年夏,與前任桐城知縣成福、六安參將慶麟,招集義勇於霍山,請助官軍殺賊,且言「事成不邀功賞,事敗則以身死之」。於是上三路進兵策,而自任桐城一路。先頓兵中梅河以俟,而提督秦定三軍之圖舒城者,延期不進。三俊既孤軍深入,恥不肯退。至周瑜城,援絕餉匱,奸民構賊夜襲營,力戰死之。

勛,字小嵩,與三俊同縣人。家貧,好倡舉義行。嘗搜羅桐城節孝貞烈婦女二千餘人無力上聞者,匯請旌表,著總旌錄四卷。桐城既破,三俊起義兵霍山,與之定計,即往見秦定三,以急擊舒城,與襲桐之師相應說之,定三不應,事遂敗。嗣聞紓青統兵至桐,往六安迎之,謂紓青曰:「桐近日賊勢與前不類,兵單援寡,難操勝算。不如先助攻舒,舒破,與秦軍合,事乃有濟。」又數以書勸定三,卒不應,紓青亦不肯往。十一月十七日,遂隨紓青督戰,死之。隨死者有吳文謨。

文謨,字翼甫,亦同縣人。年少負氣,與三俊子復震為友。三俊死,文謨不告其子,獨冒險往獲其尸。勛重其人,遂隨勛奔走,請兵不倦,殉節時年二十有一。

吳廷香,字奉璋,廬江人。敏博沉毅,與桐城戴鈞衡、馬三俊友,以文章風節相砥礪。以優貢生舉咸豐初元孝廉方正。上書論時事,有國士之目。三年,粵賊東下,陷安慶,廬江土寇應之,駸駸迫城下。邑團練鄉兵推廷香為督,擊寇,擒其渠,斬之,盡破其黨。

尋,粵賊棄安慶去,長驅薄金陵,踞其城。是年夏,復遣悍酋沿江西犯,再陷安慶,皖北震動。廷香復倡義募鄉勇六百人,自率之守梅山黃姑閘,遏江路。時賊張甚,官吏兵民所在迸散。賊自桐城北擾,舒、巢、無為相繼淪沒,獨廬江賴廷香固扼得全。十一月,廬州陷,巡撫江忠源死之,官軍、團練望風逃潰。十二月,廬江亦不守。廷香時在防次,扼腕慷慨,誓必得當以報國。

四年二月,提督和春敗賊於廬州;七月,提督秦定三大捷於舒城;舟師復自海道入扼東西梁山,斷賊歸路。賊悉眾北趨,諸州縣守賊少,曾國籓復率大軍下武昌。廷香聞則奮然起,言於眾,謂:「誠以此時出賊不意復邑城,益與江上、下諸路軍相應,合謀以圖皖中,賊可殲也。」乃召募三千人,與外委熊允升將之趨縣門,兼密約舊時勇目居城中者為內應。八月,大破守賊,賊渠任大剛走,追斬之,遂復廬江。大江東、西以鄉兵敗賊克城,蓋自廷香始。

既,賊知廬江無援,合安慶、桐城諸路來攻,廷香出擊,屢有斬獲,而賊聚益眾,江中賊亦逼城下。廷香豫乞救廬、舒大營,久未報。及賊大至,何桂珍檄蔡萼、沈承貽以六百人自六安赴援,至邑,則縱兵大掠,遇賊反走,賊益焚四野,火光燭天。廷香夜登陴,望救不至,拊膺泣曰:「吾志清逆亂,不克,而重禍鄉里。勢窮援絕,來者非人。吾死此,分耳,亂將若之何?」數日,糧竭,萼、承貽引遁,城遂再陷。廷香率死士巷戰,自午夜至黎明,從者僅三人,力盡死之,允升同及於難。

初,廷香將倡義,或危其事,尼之。廷香從容曰:「如若言,亂將誰拯耶?」其人悚然退。及事急,將自裁,或奪刀掖之行,廷香抗聲曰:「復城守城,雖非吾責,吾義也。城危而走,義何居焉?出郭一步,非死所也!」比戰歿,邑人求得其尸,槁葬之。詔建專祠廬江,予世職。子長慶繼其志,累官至提督,以功顯。

孫家泰,字引恬,壽州人。大父有善行,仁宗嘗書「盛世醇良」四字顏其門。家泰生有殊稟,嬉戲異群兒。每出語,長老驚若成人。未冠,補諸生。道光二十九年,入貲為員外郎,分刑部廣西司,治牘明決,為上官所器。咸豐三年,粵寇竄擾江、皖,工部侍郎呂賢基奉命回籍督辦團練,請以家泰從。時皖南北郡邑相繼不守,官吏望風避走,群盜蜂起。定遠陸遐齡倡亂據城,道路梗塞。朝命再起周天爵為安徽巡撫,天爵就詢策略,家泰密為擘畫,數旬之間,遐齡父子就擒,脅從解散,餘盜斂跡,壽春兵以驍勇聞。

軍興,徵調四出,留鎮者少,又乏食,巡撫檄家泰勸捐募兵為固圉計。壽故繁庶,富家大賈務厚藏,鮮遠識,無應者。家泰則盡貨其貲產以濟用,所募皆敢死士。明賞罰,嚴簡練,一軍肅然。廬、鳳、潁、六安諸寇憚其強,不敢窺。尋,天爵卒於潁州,舒城再陷,呂賢基死之。家泰失所隸,勢遂孤。尋為人所構,吏議落職。家泰語人曰:「時事糜爛,守土之吏,畏賊如虎狼,而視民如魚肉,是驅良入於暴也,吾無死所矣!」自是杜門家居,口不言兵事。既毀家佐軍,貧甚,菽水養親,晏如也。

既,賊氛益熾,諸州縣團練,多陰附賊,而鳳臺苗沛霖所部尤橫桀不可制。初,沛霖為諸生,請於知州金光箸欲練鄉團,而自為練總,光箸不之許。沛霖遂聚群不逞為亂,鄰邑豪猾多歸之。官軍畏其眾,遣人招撫,授以官,為羈縻計。數年累薦至川北道,加布政使銜。沛霖不奉命,南據正陽關,北扼下蔡,繼襲懷遠,陷之,號稱苗練,駸駸逼壽州。壽人恟懼,謀聚保,眾議非家泰莫屬,辭不獲,強起。號召部曲,上書軍帥,力主剿,未報。

沛霖遣諜入壽州,家泰殺之,沛霖益怒,盡發其黨來攻,守者恐不敵。忌家泰者,乃倡言獻家泰與其副蒙時中於賊,以紓壽禍。有司迫行,眾大譁,將以力抗。家泰夷然曰:「吾昔募健兒刺苗逆悍將,今又戮其諜,欲甘心者我也。守土非其人,順逆不明至此,事之不濟,天也。吾身許國矣,吾死而城安,其又奚恤?」遂仰藥死。既歿,按察使張學醇復縛時中付賊寨,並遇害。是年九月,沛霖卒陷壽州,家泰家屬被執,不屈,皆死之。

同治二年,科爾沁親王僧格林沁督師至,沛霖敗死,壽州平。聞家泰一門死事狀於朝,詔贈四品卿,照陣亡例賜恤,建祠壽州。父贈祖,弟家彥、家德,子傳洙,咸恤贈有差。

江圖悃,字汝華,旌德人。富膽略。經商,寓舒城。侍郎呂賢基辦團剿賊,過舒城,與圖悃一見相契,特命帶鄉團,扼守舒城沖要,賊不敢過。三年十月,桐城被陷,乘勝至舒城,賢基戰不利,死之。圖悃猶力戰,狂呼殺賊。久之,賊至益眾,援兵不至,歿於陣。圖悃前以助餉贈知府,至是歿,舒人義之,相與私謚曰仁惠。宣統初,補謚莊潔。

程葆,歙縣人。道光十三年進士,以主事分工部。咸豐二年六月,外授廣東肇慶府知府。時粵匪麕集皖境,謀犯浙江,葆赴任,道經杭州,巡撫何桂清奏令回籍治鄉團助剿。五年,賊陷休寧,葆率民團出境援,與官軍會擊於東、南二門,斃賊目,賊驚退入城。諸軍連夜進攻,賊由西門遁,遂拔休寧,乘勝克復石埭。自是葆益激勵鄉團,屢助官軍剿賊,徽郡肅清。旋檄赴杭助守,城陷,死之。

彭壽頤,字子文,江西萬載人。道光二十九年舉人。咸豐四年,粵匪連破江西郡縣,知縣李峼棄城遁,壽頤率團練御賊,追剿上高、新昌,皆捷。以籌餉忤李峼,峼袒奸民,壽頤揭前棄城事。巡撫陳啟邁夙諱賊,恐上聞,以蜚語誣捕壽頤,欲致死滅口。欽差大臣曾國籓奏言:「數年以來,諭旨諄諄,飭行團練,多無實效。惟湖南平江縣、江西義寧州以本地捐款練本地壯丁,屢殲悍黨,為賊深畏。四年,義寧之捷,巡撫陳啟邁冒功濫保,遍私親暱,人心解體,團練遂散。賊再攻州,抵拒經月,省兵竟無援救,城陷,屠民數萬。向使練丁尚存,何致慘禍如此?五年,饒州、廣信之失,鄱陽、興安之失,陳啟邁通融入奏,寬減處分。萬載之失,知縣李峼有避賊重咎,舉人彭壽頤有剿賊殊功,奸民彭三才有餽賊實據,陳啟邁竟袒庇屬僚,架誣團練義士。餽賊不斥,避賊不劾,獨於剿賊者,目為豺狼,指為逆黨。臬司惲光宸,逢迎喜怒,褫革逮拘,酷暑重刑,百端凌虐。臣以壽頤才識卓越深沉,疊商留營效用,陳啟邁堅僻不悟,釀成冤獄。義寧之團,以保舉不公毀於前;萬載之團,又以訟獄顛倒毀於繼。人心何由固結?大局恐致貽誤。」奉諭:「陳啟邁革職,惲光宸交新任巡撫文俊查辦。」壽頤早以刑斃矣。南昌梅啟照嘗云:「國籓雅度無怒容,惟於壽頤逮獄,深為憤痛。」七年,劉長佑敗,新喻、袁州三縣民率丁壯助軍,軍復振,世益以此思壽頤。

陳介眉,山東濰縣人。道光十八年拔貢生,朝考用知縣,發江蘇,署宿遷、鹽城等縣,擢通州知州。屢獲海洋巨盜,擢知府,授河南歸德府知府。咸豐三年,捻匪竄虞城之楊家集,介眉督兵追殲三百餘,生擒二百餘。粵賊陷歸德,褫職回籍。十一年,捻匪竄山東,抵濰縣,介眉迎剿,與候選訓導陳威鳳、武舉譚占元等,均力竭陣亡。復原官,恤贈太僕寺卿銜,賞世職,建專祠,並祀威鳳、占元及同日陣亡之武生千總銜陳執蒲等。

同縣人亓祈年,道光五年舉人,截取知縣。捻匪熾,祈年治西鄉團練,匪竄縣境,祈年登圩固守。圩破,率眾巷戰,力竭被縛,罵賊不屈死,侄文豐等同時陣亡。恤贈道銜,賞世職,建專祠,文豐等附。唐守忠,鉅野人。咸豐初,為平陽屯々官。四年,粵賊陷鉅野,土匪竊發,守忠聞警馳歸,遭匪劫,僅以身免。與鄉人生員張桂梯、職員姚鴻傑等議舉團練,為守衛計。旬日集義勇五千餘人,分三隊,捕斬土匪數十名,賊遂遁,嘉祥、鉅野間悉平。土匪懼,以所劫物展轉還守忠,並乞隨團剿賊,誓不為亂,守忠察其誠,納之。時年饑人乏食,守忠使子錫齡偕張桂梯各村勸捐助賑,富出貲,貧出丁,括計餘糧,計月分給,謂之均糧,而團練之勢愈固。曹州、濟寧兩屬鄉團來附,賊不得逞,去。

五年,河決銅瓦廂,鄆城、鉅野、嘉祥等縣當其沖,守忠聞豐工黃水下游淤涸成灘,官出示招墾,因率災民數萬人南下認種。仿屯田法,以教諭王孚、千總唐振海等分領之,名曰湖團,亙二百餘里,濬溝築圩,編保甲,嚴守望。徐州、蕭、碭、豐、沛等縣人聞賊警,則相率投避,得免於難者數年。

八年,捻匪來犯,守忠率團遮擊,擒賊樊三、丁豹等斬之,敘功給五品頂戴。十年,欽差大臣僧格林沁令守忠隨官軍助剿,敗賊大劉莊。同治元年,捐助軍餉,又捐已墾熟田為魚臺書院經費。二年,白蓮池教匪由滕縣偷渡湖西,守忠截擊,生擒賊目陳周等多名,餘匪悉遁。

四年九月,捻匪張總愚、任柱等悉眾來攻,守忠集丁堵御,一再請援兵不至,力戰六日,眾寡不敵,死之。方守忠被圍,賊數使招降,守忠誓死拒之。及戰敗,與族叔千總振海、子生員錫彤同被執。賊舁至銅山袁家廟,多方脅之降,守忠罵不絕口,遂並見害。江督曾國籓疏請優恤,建祠立傳,從之,贈道銜。子錫彤,照四品以下陣亡例議恤,給世職。尋在沛縣捐建專祠。

吳山,字巖青,河南光山人。生三日喪父,母周守節撫孤,家極貧,紡績供山讀書。道光二十五年,舉於鄉,會試不第,留京三載,與袁保恆、裴季芳相切磋,聲譽日起。時光山有匪患,山以寡母在堂,二子尚幼,又無期功強近之親,就揀選知縣職,倉卒歸。

先是,邑民郭三,兇黠。兄弟七人,郭五、郭六尤悍。郭三充縣皁役,滿布黨與。知縣水安瀾恇懦,為郭三等挾制,無所不至。彼時有「郭滿城」之謠。郭三充臥龍臺鄉保,倡首為匪,向四楞子、曾傳佐等,皆領桿頭目,肆行劫掠,並至各鄉按畝加糧供食,並勾通亳、壽各州各匪,謀殺官起事。山有鄉望,眾舉為團首,倡辦團練,地方恃以安,而郭三忌之。

咸豐四年四月,郭三糾眾突至小向店派糧,山拒而不納,尋,集鄉團與之抗。匪巢臥龍臺,距小向家集僅十二里,郭三揚言非殺山不可。或有勸山走避者,山曰:「我所以觸匪怒者,原以抗匪派糧,若臨難而逃,任匪所為,則初志謂何?今日之事,有死而已。」遂挺身督鄉團與戰,眾寡不敵,被擒,山罵不絕口,匪怒戕之。後俞御史劉毓楠奏建專祠。

俞焜,字昆上,浙江錢塘人。嘉慶二十五年進士,改庶吉士,授編修。道光十三年,遷御史,奏請申明律義,以正倫紀,略言:「律載『弟妹毆同胞兄姊死者皆斬』。注云:『毆死期親尊長,若分首從,則倫常斁矣。』此古今定律,所以維名教也。其聽從尊長,毆死以次期親尊長之犯,向律擬斬,定案時夾簽聲請,疊經改為斬監候,歸入服制情實。自道光三年御史萬方雍奏,將聽從尊長,毆死以次期親尊長,下手傷輕之卑幼,均科傷罪。刑部定為條例,至今沿之。因思例從律出,例因時變通,律一成不易。致死尊長,豈得仍論傷之重輕?今以勉從尊長,下手傷輕,止科傷罪,則與『死者皆斬』之律未符。此例既百無一抵,何以肅典刑而正人心?請仍遵不分首從本律,夾簽聲請,以昭平允。」下部議行。

十七年,授河南彰德府知府,以東河大工勞最,用道員,擢永定河道。調衡永郴桂道,緣事降調。咸豐九年,督辦團練,操防勤奮,復道銜。十年,粵賊亂熾,焜商遣駐防軍守獨松關,李秀成犯杭,焜與侍郎戴熙登陴拒守二十餘日。城陷,巡撫羅遵殿殉之。焜憑柵堵御,與滿城犄角,復相持五日。彈盡,柵毀,賊眾,焜猶手刃數賊,矛洞胸,歿於陣。明日,張玉良援師入,將軍瑞昌會擊,賊卻而焜已死。論者謂滿城之存,焜有力焉。賜謚文節,建專祠。同殉之繼室陳氏,女蘊祺、蘊璿附祀。

同縣戴煦,字鄂士。增貢生。候選訓導。精算術。西人艾約瑟見煦所著求表捷術,心折之。又工畫,神似倪迂,評者謂出乃兄熙上。熙既投水殉節,聞之嘆曰:「吾兄得死所矣!」亦投井死。著有莊子順文,陶靖節集註,四元玉鑒細草,對數簡法諸書。熙自有傳。

張洵,字肖眉,錢塘人。咸豐二年進士,改庶吉士,授編修,命在上書房行走,文淵閣校理。十年,粵匪由安徽竄浙江,杭州省城被圍,巡撫羅遵殿奏入,洵請假省親。上召見,垂詢浙省軍情。洵抵浙江,杭州失而旋復。先是洵母談氏,因賊逼杭城,率洵妻施氏,洵子惇典、從典、敘典、念典,女喜姑闔門赴水,被救得不死。施氏即命惇典、念典等護其姑出城。賊至,施氏遣喜姑先投井死,自率敘典躍池中殉焉。杭州將軍瑞昌以聞,上嘉施氏孝義兼全,下部旌恤。

尋,洵母自以老需人侍奉,為洵繼娶勞氏。未幾,丁母憂,洵省城無房產,僦居於仁和縣之永泰鎮。十一年,賊大股復犯浙江,餘杭、蕭山相繼失陷,省城被圍。洵念受恩至重,不忍坐視,乃自永泰鎮挈眷赴省,與官紳籌守御;並謀諸巡撫王有齡,會合駐防兵,力通江路。顧賊勢張甚,圍城兩月餘,城陷,洵與勞氏、惇典、從典、念典皆死之。洵兄濂之妻李氏及女九姑,亦先後殉焉。

方城之垂陷也,洵聞警,即索衣冠北向叩頭畢,賦詩三絕,有「白雲堆裏吾將去,前輩風流有戴公」之句。書竟,授僕張升,遂投井死。同治元年,太常寺卿許彭壽以聞,以「一門六口,同時殉難,實屬深明大義,忠烈可嘉」褒之。八年,國子監司業孫詒經復請加恩予謚建祠,允之,謚文節。

鍾世耀,字嘯溪,仁和人。道光二十一年進士,改庶吉士,散館,授兵部主事。移疾歸,負鄉望,城再陷,賊將授以偽官,絕粒殉節。

孫義,字樸堂,錢塘人。道光九年進士,官福建仙游縣知縣,有循聲。告歸後,課徒自給,同時殉難。

汪士驤,字鐵樵,錢塘人。襲世職,授杭州營千總。擅詩名,工篆隸,晚年作小楷尤精。咸豐十一年,賊再至,先以年老休致,居危城中,神色自若,日以忠義訓家人。賦詩有「我死家人生,辱家即辱我」等語。城破,全家皆躍水死。

錢松,字叔蓋,錢塘人。嗜金石篆刻,有文譽。賊初陷杭垣,先期具藥汁,誓死。家傍清波門,賊從此入,遂與家人同仰藥,麾侍者還其室,曰:「今日得死所,而男女顛僕一室可乎?」語定而絕。

毛雝,字西堂,錢塘人。諸生。事親孝。年十三,能作大字。工書,得潤筆盡給貧乏。督辦東北隅團練,城再陷,自縊死。

魏謙升,字滋伯,仁和人。九歲能文,弱冠後雄長壇坫。尤工書。以廩貢生選仙居縣訓導,不就。家居西馬塍,以著述自娛,垂五十年,有書三味齋稿。賊自湖州逼省城,家當其沖,或諷宜移居避之,不應。賊火其廬,乃挈妻子走靈隱山中。賊退,僑寓城中,嘯歌不輟,自號無無居士。城再陷,謙升方老病,驅至萬安橋下死,妻周氏同時殉節。周能書,世以鷗波夫婦擬之。

金鼎燮,字承高,秀水人。諸生。咸豐季年,署臨安訓導兼教諭。以事詣省城,寇至,圍久,糧絕,至煮篋上革以食。城破,雜難民中出,至臨安,率鄉團禦寇,死之。

庚辛之役,省城再陷,杭人殉難者至眾,而旗營死事尤烈。其著者:協領巴圖蘭布等守花市營門,佐領德克登額、佛爾國納、德勒蘇等守錢塘門,呼松額、格勒蘇、印福等督隊出湧金門,皆迎戰,奮刀殺賊,先後陣亡。又協領賽沙畚、連生等,佐領薩音納、伊勒哈春等,防禦貴祥、明阿納等,驍騎校志善、佛爾奇納等,文職如知府伊麗亨等,武職千總安忻保等,皆陣亡。合營縱火自焚,男婦死者八千餘人。

包立身,諸暨人。家五十八都之包村,世業農。性樸魯,里黨莫之重。咸豐十年,忽能言休咎,多奇中。節食茹素,夜則結跏趺坐。時賊氛漸偪,人懷憂懼,爭奔詢,立身惟以行善為勖。人疑信參半,不知其嫺武略也。

十一年九月,賊陷紹興府,他賊復自金華來,諸暨亦陷。於是首倡義旗,從者響應。村踞山,三面皆水田,惟一路由塍埒達村。賊焚掠至其地,立身以靜待動,入者輒為所斃。避賊者麕投之,棲止無隙地。乃益選壯勇成勁旅,賊來攻,數不勝。立身不出村剿賊,賊至則戰,戰則身先,當其鋒立踣。眾見賊易擊,雖文弱者亦揮戈從事,間諜入村者,罔弗獲。無事則焚香默坐,有所指揮,從之必勝,遠近驚以為神。賊憚甚,使素稔立身者招降,立斬之。乃悉糾數郡悍黨,更番進攻,而往者輒衄。群賊聞調攻包村,如就死地。相持八九月,大小數十戰,斃賊十餘萬,精銳強半盡。

賊目有周姓者,眇而通形家言,乃周覽村外,悉其川源山脈。會旱,溪流弱,賊壅其上流,遂無涓滴。村外井水,賊舉腐尸填之,出汲,則先以火器越井而陣,後人出尸乃得汲,腥穢不可飲,然且難得。人眾食寡,賊又四面斷糧道,不得達,賊遂索戰無已時。每合陣,所損相當,勢不能久持,終無一人言降者。賊遂陰穿隧道而以金鼓聲亂之,立身不之省。

七月朔日,賊穴隧道自村社廟出,即縱火焚廟,眾出不意,大亂。賊遇人即殺,未遇賊者亦倉皇圖盡。立身見事敗,與其妹鳳英率親軍數千人死戰,潰圍出,至馬面山。賊躡之,圍數匝,鏖戰不得脫,中砲死。鳳英亦力竭自刎死,全家皆遇害,從者亦無一得脫。合村死者,蓋六十餘萬人。

王玉文,字緯堂,金華人。性強毅,好談經濟。道光二年舉人。咸豐四年,授於潛教諭。會粵寇據金陵,數上書當道論兵事,指陳兩浙形勢甚悉。既而浙壤告警,奉檄領兵守天目山,又令塹於潛、臨安山谷,防賊闌入。既至,躬自履視,得某關廢阯,實為要隘,因建言修之。初偕昌化教諭高文祿行團練,於潛令素與玉文忤,多方撓之。及是議築關,益譁然以為多事,而玉文銳於自任,不之顧。

十年,賊陷杭州,玉文將百人扼關,欲乘賊歸擊其惰,文祿力贊之。於是昌化、臨安、新城及本境山氓,咸持梃原受節度,官紳交阻之,事遂寢。玉文恚甚,乞病歸,甫束裝,聞寇至,嘆曰:「臨難而去,非夫也!」乃輟行。適援軍至,玉文戒以守關毋出,不聽,戰五晝夜,眾寡不敵,棄關走。賊入城,官皆遁,有門下士偕二輿夫、一擔者來迎,玉文堅不去,迎者旋散。乃朝服挾刃坐,一賊當先入,格殺之,即舉火自焚。遺書付其子曰:「天熱,吾清白之體,不可俾鬱蒸,有鹽硝,舉以自化,汝曹毋過悲痛也!」寇退,得其尸池水中,朝服爇去,跣一足,眾哭殮之。以其先有告病牒,大吏不以殉難聞,士民咸以為冤。

孫文德,嘉善人。咸豐十年,年八歲,賊陷嘉善,家人攜出城,遇賊,相失,獨至村舍。薄暮,十餘賊入舍就炊,將休矣,文德潛乞砒毒於賣藥人何桂生,密啟釜置之。飯熟,賊方饑,食之,斃九人。二人未食,大駭,考掠文德,奮身大罵,賊殺之。

李貴元,字祥枝,永康人。事母孝,以強有力聞。賊至,年已八十,乃出其大鐵鐧擊賊。賊懼不敢動,貴元從容登樓。及群賊擁至,貴元遂遇害。越日,其子求遺骸以出,賊亦不之罪也。時錢塘汪玉璋、義烏金士玉、長興副貢生王泰東,均年逾八十,先後預於難。

富陽瞽者陳小福,避山中,從賊者識其神卜也,囚之。官兵攻急,賊勢蹙,乃命之卜。小福曰:「若輩必盡死,無遺類,何卜為?」賊怒,剜其目,磔之。

皮匠某,逸其姓名。十一年,圍急,閩兵絕糧,不欲戰,巡撫王有齡登陴泣。匠忽手百金至,叩首曰:「小人勸苦,蓄得百五十金,今留五十金自贍,餘請助餉。」有齡為榜示轅門示勸。城陷,匠自經死。

羅正仁,湖南郴州人。諸生。咸豐三年,土匪蜂起,三月十四日夜半,突有賊數百人攻入城,戕知州胡禮箴。正仁急起,倡辦團練,獲賊二十餘人,殺之。由是各處效法,不數日,諸匪咸撲滅,餘黨恨正仁刺骨。會粵匪陷州城,土寇與合,正仁復率團要擊之。賊懸賞購正仁,正仁走避。久之,聞母病,歸,賊偵知。一日昧爽,突有賊三人至其家,正仁猝無所備,乃率二子春官等御之。俄賊眾奔至,眾寡莫敵,遇害。二子亦受重創,佯死得免。後春官痛父,更集團,日以剿匪為事。五年,城再陷,率團復攻之,每戰奮不顧身,多斬馘,為鄉里所倚庇。

同州人陳起書,字通甫,幼從兄起詩講求經世學,由附貢生候選訓導。道光十三年,逆瑤趙金龍叛,起書條陳禦瑤策,知州姚華佐多採用之,州城得無患。金田賊起,起書謂西粵一隅地,賊不能久居,必竄楚。竄楚,則大軍必扼衡州,郴、桂將首受禍。遂畫守禦之計,州牧不能用。乃糾同志自集團丁於觀音寨、大頭隴,並築堡、修墻為堅壁計。無何,賊果至,聞州境有備,遂引去。時土寇邱倡道煽亂,擾及閭里,上官檄官軍剿之,不獲。起書命次子善墀、戚張樹榮依計誘擒之,並獲賊渠黃中鳳,事平。咸豐五年四月,廣東賊何祿寇宜章,五月,州城陷。起書率團練扼北鄉,賊不敢犯。有東鄉戚黨招起書為畫守御策,祿適湘鄉王珍率師由衡州來援,乃命善墀迎師,自往東鄉,行抵塘溪,擬聯絡瑤嶺鄉民以拒之。而土匪咸通賊,偵其往,中道要劫之,遂被執。群賊久耳起書名,擁於坐,宛轉誘降,起書罵不絕口,抵死不降,賊遂計議俟何祿至,乃縛其手,日夜環守之。起書於八月七日絕粒,死之。

陳景滄,字少海,龍陽人。父永皓,直隸長垣知縣,有聲於時。景滄幼凝重,守道義,留心經世之務。以咸豐元年舉人官內閣中書。粵亂作,湖北巡撫胡林翼治楚軍備賊,徵闢賢俊,以景滄佐軍事。積功保知府,命籌餉岳、灃。景滄剔除宿弊,事集而民不擾,嘗曰:「籌餉病民,已非善政;若更貪其利,是官民交病,吾不為也!」不數年,以親老辭歸里,閉戶山中,侍養之餘,以讀書為樂。同治六年,閩浙總督左宗棠調往福建,湖南巡撫劉琨亦強起景滄,景滄咸謝不赴。八年,丁父憂,哀毀廬墓,益遠人事。

初,軍興,募民為勇,越境擊賊,湖南尤盛。暨賊平,勇散歸,不事生業,相率入哥老會。哥老會者,起四川,異姓相約為昆弟,同禍福,結盟立會,千里相應。其盟長之大者,輒擁眾數千人,橫行郡邑,吏莫敢詰,良懦憚之,則入會求庇。入者既眾,勢乃益厚,流行湘、楚間。初但為奸盜,均其財,繼焚掠村市,抗官兵,窺城邑。長沙、衡州諸屬,屢撲屢熾。十年,益陽何春臺,龍陽劉鳳儀、劉繼漢等,率會眾為亂,聚縣西安化山中,距景滄家十餘里。景滄聞變急,密告巡撫,巡撫檄益陽、龍陽兩縣往捕,會眾方傳檄諸州縣黨人,約同時發難。未至期,捕者適至,遂先舉事,犯益陽。途中值景滄,執之,景滄責以大義,數其罪。被數刃,罵不絕口,賊群斫之。長子克枟、次子克權從行,以身障景滄,並及於難。

景滄長身玉立,恂恂孝友。與人交,訥然若不出口,至論古今忠孝及國政得失,輒慷慨流涕,義形於色。事聞,贈道員,給世職。

何霖,字雨人,廣西興安人。少讀書,以諸生食廩餼。抗志高尚,不屑屑治章句。性沉毅,有膽略。咸豐三年,興安盜王茍滿、趙廷蘭等作亂,陷縣城,囚官吏。霖聞變,匿老弱,自與族弟進賢急詣省求援,中道遇賊,為所劫。霖詭辭脫進賢,入見賊酋,賊素重霖,以上賓禮之。霖謬為甘言,飲啖自若。酒酣,因說賊酋曰:「君等舉大事,宜先收人望,蔣方第諸人,邑之豪俊,原假良馬利劍為君輩致之,非常之業可圖也!」賊喜,如約。霖遂以方第等六人至,留賊中,賊信不疑。霖陰謀方第間賊黨,將乘間舉事,會官軍擊賊靈川,屢勝,賊分兵攻金州亦敗,眾稍稍引散。霖遁歸,偕方第一夜集鄉兵,盡縛北鄉諸賊。分守要害,號召鄰鄉團眾,分三路攻城。賊不為備,遂復興安,擒茍滿等。官軍至,獻捷,主兵者攘其功,賞不及霖。益與方第倡言興團練,立規約,厚佽給,人樂為用。賊黨謀再舉,憚霖不敢發。

四年,恭城賊陷灌陽,霖率興安團屯邊隘,賊不得逞。相持數月,樂平賊自別道來援,霖與方第議增丁壯,移營前進,遏其鋒。十一月,次茗田,賊以大隊從大風坳出犯霖壘。所部祗五百人,續調者未至,霖麾眾迎擊,奮斗竟日,力竭戰死。方第暨其兄子二人並歿於陣。賊再入興安,焚霖廬舍,盡殺其家人,霖父挈孫走臨桂,得免。事平,興安民思其功,建祠祀之。

蹇諤,字一士,貴州遵義人。道光二十六年舉人。咸豐三年,大挑得教職。明年,諤還自京,適桐梓教匪楊龍喜作亂,長驅出婁山關,逼遵義。知府硃右曾要擊,敗還。賊遂以八月十六日圍城,營郭外雷臺山。是時黔中治平久,民老死不見兵革,初遭寇亂,眾洶洶欲潰。獨諤力言賊可擊,於是人心稍定。久之,官兵漸集,而賊亦日附。諤謁提督趙萬春、布政使炳綱於螺瑯堰,陳利害,請由石盤扼賊糧,拊其背,自領兵練四百人營馬家河,復募二百人益之,屢戰皆捷。

賊酋李七王者尤獷悍,以千餘人入貴陽大道,踞龍坪水口寺,諤率所部圍攻,盡殲之。七王自焚死,賊氣奪。十二月,官兵破東路櫻桃丫,賊憑險拒戰,不即克,諤以兵從中坪繚其後,大破之,乘勝進克羊耳丫。賊退屯金錢山,引渠灌田,計死守。諤令健卒負草涉凍薄而焚之,於是官兵攻雷臺益急,蜀兵亦進復桐梓,龍喜蹙,遂焚巢夜遁。

五年冬,龍喜餘黨鄒長保再叛,圍桐梓七晝夜,並據婁山關以遏遵援。諤復集兵練千餘名,攻奪婁山,解其圍。期必滅賊,屢深入至寺岡。寺岡,賊巢所在,危峰攢刺,往往雲霧,不見天日。諤勒兵直上,以身先之,猝遇伏,前鋒為所敗,諤親率卒二十人搏戰。賊眾麕至,矢石交下,諤力竭死之。隨行之王世洪、曾名標亦奮鬥死,時咸豐四年十一月十日也。恤道銜,給世職,立專祠。

趙國澍,貴州貴陽人。咸豐三年,黔中土匪起,國澍方為諸生,居青巖。其地扼定番、廣順之沖,為貴陽屏蔽。乃散家財,倡團練,城青巖自守,隨官軍四出剿賊。十年,粵酋石達開竄貴州,陷廣順,圍定番,眾號二十萬,貴州大震。國澍倡勇敢、救定番,民壯從者數百人。力戰城下竟日,賊斷其歸路,死亡略盡。國澍匹馬突圍還青巖,登陴堅守,賊亦卻退。會賊以廣順之眾益定番之圍,道出青巖,脅降,不可。圍三日,引去。

七月,定番陷,並力攻青巖,國澍隨機應御,賊攻六月,終不能下。伺賊稍懈,乞援提督田興恕,興恕遣侄麒麟來,大為民擾,國澍斬麒麟以徇。興恕親赴之,前鋒失利,責戰益急。國澍策賊食將盡,請斂軍堅壘障省城,檄清鎮、安平、大定清野以待。貸土匪陳文禮等死,密遣入賊縱火,內外夾擊,毀賊營二,賊每夜自驚。國澍以計間其悍酋,使相屠,遂大閧,因與興恕合兵乘之,賊崩潰。追奔至安平,復大破之,定、廣諸城皆復。先是國澍剿平定、廣土匪葛老巖、楊龍喜及平伐、擺金、平越、甕安諸賊,收復修文等城,累擢至候選同知、直隸州知州,賞花翎。至是興恕上其功,言:「國澍毀家、築城、練團、當巨寇,受攻半載,卒創賊,全省會,非優獎不足以勸士民。」命以道員即選,並總辦貴州團練事務。

十一年九月,安順仲苗匪警,國澍率黔勇七百,會總兵羅孝連剿之。十月,至安順,仲苗蔓延鎮、永二州,負險累年。其老巢曰養馬塞、烏束隴、蜜蜂屯、猛董山,孝連直搗烏束隴,國澍調團練分塞要隘,斷賊援。養馬塞賊懼,縛酋獻地降,國澍乘勝攻蜜蜂屯。十一月,破水西莊阿打洞屯,賊詐乞撫,國澍佯納之,使兵冒賊衣裝,夜入蜜蜂屯,遂克其巢。群賊蟻居猛董,復會諸軍圍而殲之。

同治元年正月,石頭寨等隘以次蕩平,安順肅清,加按察使銜。會楊巖保兵潰,上大坪苗夷槓匪踵敗兵渡清水江,國澍聞警馳赴郎岱,擊苗匪破之。連戰皆捷,進剿水城。賊散踞洞塞,地皆險奧,國澍分兵雕剿,自夏經冬,破洞塞百餘。

賊走渡江,遂沿江設守,乃還省,請增兵協餉,以備深入苗疆。而御史華祝三、湖南巡撫毛鴻賓劾田興恕苛斂,並及國澍殘刻狀。貴撫韓超為覆奏,辨甚晰,事乃寢。會開州知州戴鹿芝殺天主教士,法使愬於朝,復連國澍。蓋興恕嘗欲逐教民,而國澍左右遂背國澍有毀教堂、殺教民事。兩廣總督勞崇光與法使議,令國澍償金厚葬,事已平矣;開州案起,並發前事,法使愬不已,朝廷命將軍崇實等視其獄。二年三月,褫國澍職,遂撤團練局,苗事益急。

四月,大吏檄國澍督練勇渡泡江河。時沿江諸軍饑潰,賊再內犯,竄光沙,勢張甚。國澍兼程進,次百宜,賊眾兵寡,遂被圍。食盡援絕,力戰,死傷過半。親軍數十人,擁國澍潰圍出走,至徐家堰,賊大至,奮鬥死之。巡撫張亮基以聞,贈太常寺卿,賞騎都尉世職。子四,次以炯,光緒十二年進士第一,翰林院修撰。

宋華嵩,四川工⼙州人。咸豐九年,滇匪竄四川,華嵩自備軍糈,以武監生倡辦團練,保衛鄉里。十年,川匪藍大順圍州城,華嵩率團勇大破於五道碑,圍解。嗣防堵夾門關、青草坡、大進埠等處,凡自賊營逃出難民,資遣無算。藍逆撲蒲江,華嵩督團迎擊,屢勝之,賊竄去。十一年,藍逆由新津回竄,華嵩御之文華山南河岸,賊不得還,折入眉州。既而藍逆別股復竄蒲江,踞青水溪,華嵩率團進剿,多斬獲。卒因眾寡不敵,歿於陣。

華嵩團練數年,捐銀米數甚鉅,輕財好義,能得人心,故所部練勇如王德明、王富舉、王富義、楊鎮川等,咸效死不顧。同治元年,總督駱秉章上華嵩死狀,恤如例,於本籍建專祠。

伯錫爾,哈密回王也,其受封始祖曰額貝都拉,畏兀兒種人。康熙中,獻玉門、瓜州地,立為一等扎薩克。再傳曰額敏,晉封貝子。傳玉素卜,晉封貝勒,加郡王銜。三傳至伯錫爾,於道光十二年進封郡王。同治三年,以助開渠功,加親王銜,署理哈密幫辦大臣。會南、北路各城叛回煽變,八月二十九日,哈密漢裝回匪馬兆強、馬環等焚掠附城村莊,伯錫爾及辦事大臣文祺率回丁出戰,斬兆強、環,餘黨潰,敘功賞用黃韁。

九月初二日,圖古里克回匪馬添才戕稅局吏役及漢民七十餘家,南攻沁城,伯錫爾令章京巴海、守備趙英傑追捕,至北山板房溝,斬添才。四年二月,患隴右道梗,奏稱由肅州東歷蒙古漠南地,至山西歸化城,往還可百日,請由此轉餉。然臺站舊在漠北蒙古,力疲不能增設,時哈密協標兵僅五百餘人,安西協援兵二百人,不足分守。纏回及漢民雖眾,未習戰陣,吐魯番叛回頻來誘,人情煽動。五月,回匪黑老哇、纏匪蘇布格等反,辦事大臣札克當阿中彈死,賊毀漢城,入回城,幽伯錫爾。

五年六月,巴里坤總兵何琯令游擊凌祥趨救,攻拔回城,賊遁吐魯番,伯錫爾奏留凌祥為副將。旋以叛黨蔓延,奏由烏里雅蘇臺將軍檄召明安郡王蒙兵,合巴里坤、哈密諸兵,共攻吐魯番。又數遣使至肅州,請提督成祿出塞,皆不果。

十一月,蘇布格率南北各城叛回五千人復來侵,凌祥以民勇三千、伯克夏斯勒以纏回五千人出御,覆沒於柳樹泉,凌祥遁。或謂伯錫爾:「盍行乎?」伯錫爾嘆曰:「吾世受天子恩,備籓於此,臨難何可茍免?」收殘卒二千,復戰於頭堡,又大敗,被執。明年正月,罵賊死,詔贈親王。


\end{pinyinscope}