\article{本紀十五}

\begin{pinyinscope}
高宗本紀六

五十一年春正月丙午朔,日食,免朝賀。戊申,命戶部撥銀一百萬兩解往安徽備賑。辛酉,禮部尚書姚成烈卒,以彭元瑞代之。丙寅,以普福為駐藏大臣。庚午,江西巡撫何裕城奏糧價日昂,由江、楚販運過多所致。上以意存遏糴,切責之。命範建中往哈密辦事。

二月庚辰,上御經筵賜宴,命工歌新譜抑戒詩,歲為例。加福建水師提督黃仕簡太子太保。乙酉,上幸南苑行圍。辛卯,命尚書曹文埴,侍郎姜晟、伊齡阿往浙省盤查倉庫。壬辰,上詣西陵,巡幸五臺山,免經過地方額賦十分之三。丙申,上謁泰陵、泰東陵。丁酉,免直隸順德、廣平、大名三府屬上年災欠銀米。己亥,以圖薩布為湖北巡撫。癸卯,免山西忻州等六州縣逋賦。

三月丙午,上駐蹕五臺山。丙辰,兩江總督薩載卒,調李世傑代之。以保寧為四川總督,鄂輝為成都將軍。己未,上閱滹沱河,閱正定鎮兵。壬戌,上祭帝堯廟。癸亥,命李侍堯署戶部尚書。甲子,賑陜西朝邑等三縣災民。庚午,上還京師。辛未,以伊齡阿為浙江巡撫。

夏四月己卯,命大學士阿桂往江南籌辦河工。乙酉,浙江學政竇光鼐奏嘉興、海鹽、平陽三縣虧空各逾十萬,郡縣採買倉儲,俱折收銀兩,以便挪移。命曹文埴等嚴查覆奏。賑山西代州等六州縣水災。己丑,命竇光鼐會同曹文埴等查辦浙江虧空。

五月丙午,命阿桂赴浙,會同曹文埴等查辦虧空,並勘海塘。丙辰,富勒渾褫職,交阿桂等審訊。丁巳,以孫士毅為兩廣總督,調圖薩布為廣東巡撫,以李封為湖北巡撫。己未,以李侍堯署湖廣總督。辛未,上秋獮木蘭。賑四川打箭爐等地震災。是月,免江蘇上元等五十六州縣衛上年旱災額賦。

六月丁丑,上駐蹕避暑山莊。乙酉,以福崧署山西巡撫。丁亥,湖南常德府沅江溢。辛丑,調富綱為閩浙總督,以特成額為雲貴總督。以畢沅為湖廣總督,江蘭為河南巡撫。

秋七月戊申,免河南商丘等十二州縣上年旱災額賦。壬子,江蘇清河李家莊河溢。丁巳,命阿桂由浙江赴清口,會同李世傑等辦理堵築事宜。己巳,曹錫寶劾和珅家人劉全,不能指實,加恩革職留任。

閏七月庚辰,大學士、伯伍彌泰卒。召劉秉恬來京,以譚尚忠為雲南巡撫。己丑,浙江學政、吏部右侍郎竇光鼐褫職。庚寅,富勒渾論斬。乙未,命和珅為文華殿大學士,管理戶部事。福康安為吏部尚書、協辦大學士,仍留陜甘總督任。福長安為戶部尚書,綽克托署兵部尚書。戊戌,賑湖南武陵、龍陽水災。

八月丙辰,上幸木蘭行圍。庚申,調嵩椿為綏遠城將軍,積福為寧夏將軍。

九月戊寅,上駐蹕避暑山莊。丁亥,以勒保為山西巡撫。戊子,以永保為塔爾巴哈臺參贊大臣。以巴延三為陜西巡撫。壬辰,上還京師。甲午,調福長安署兵部尚書,以綽克托署戶部尚書。乙未,以瑯玕為浙江巡撫。己亥,皇長孫貝勒綿德卒。賑安徽五河等十七州縣並鳳陽等五衛水災。

冬十月辛丑朔,調富綱為雲貴總督,以常青為閩浙總督。丁未,降畢沅仍為河南巡撫,江蘭仍為河南布政使,授李侍堯湖廣總督。丁巳,免直隸安州等四州縣被災額賦有差。

十一月,賑安徽合肥等十七州縣水災。

十二月辛丑,福建南靖縣匪徒陳薦等作亂,捕治之。壬子,大學士梁國治卒。命兵部尚書王傑在軍機處行走。戊午,封鄭華為暹羅國王。丙寅,福建彰化縣賊匪林爽文作亂,陷縣城,知縣俞峻死之。命常青、徐嗣曾等剿辦。

是歲,朝鮮、琉球、暹羅來貢。

五十二年春正月辛未,林爽文陷諸羅竹塹。癸酉,命鄂輝署四川總督。乙亥,宥富勒渾罪。丁丑,調李侍堯為閩浙總督,常青為湖廣總督,仍留福建督辦軍務,命舒常署之。癸未,林爽文陷鳳山,知縣湯大全死之。甲申,常青以守備陳邦光督義民守鹿仔港,收復彰化奏聞。丁亥,命王傑為東閣大學士,管禮部事。調彭元瑞為兵部尚書,以紀昀為禮部尚書。庚寅,允戶部尚書曹文埴終養,以董誥代之。辛卯,命松筠往庫倫辦事。丁酉,命常青渡臺剿匪。

二月壬寅,林爽文復陷鳳山,犯臺灣府,柴大紀督兵民御之。癸卯,以李綬為左都御史。乙巳,以長麟為山東巡撫。壬子,免臺灣府屬本年額賦。丙辰,復諸羅。甲子,上詣東陵。丁卯,上謁昭西陵、孝陵、孝東陵、景陵。

三月癸酉,上回蹕。丙子,以重修明陵成,上臨閱,申禁樵採。辛巳,復鳳山。辛卯,以姜晟為湖北巡撫。黃仕簡以貽誤軍機褫職,令其長孫嘉謨襲公爵。乙未,逮黃仕簡下獄。

夏四月辛丑,以常青為將軍,恆瑞、藍元枚為參贊。調藍元枚為福建水師提督,柴大紀署陸路提督。戊午,上詣黑龍潭祈雨。壬戌,賜史致光等一百三十七人進士及第出身有差。甲子,上閱火器營兵。

五月丁卯朔,烏里雅蘇臺參贊大臣貢楚克扎布病免,以三丕勒多爾濟代之。戊辰,授蘭第錫河東河道總督。甲戌,上秋獮木蘭。庚辰,上駐蹕避暑山莊。湖南鳳凰苗作亂,總兵尹德禧討平之。

六月庚戌,免浙江仁和場潮沖蕩地額課。壬子,授柴大紀福建陸路提督,兼管臺灣總兵事。丙辰,召福康安赴行在,以勒保署陜甘總督。

秋七月壬辰,以海蘭察為參贊大臣,舒亮、普爾普為領隊大臣,率侍衛、章京等赴臺灣剿賊。癸巳,賑安徽懷遠、鳳陽等州縣水災。賑山西豐鎮等九州縣旱災。

八月,常青免,命福康安為將軍,赴臺灣督辦軍務。辛亥,上幸木蘭行圍。

九月壬申,上回駐避暑山莊。庚辰,上回蹕。壬午,調柴大紀為福建水師提督,以蔡攀龍為福建陸路提督,並授參贊。辛卯,以諸羅仍未解圍,催福康安徑剿大里杙賊,並分兵進大甲溪。

冬十月丁未,命福長安署工部尚書。戊申,修福陵。丁未,睢州下汛決口合龍。丙辰,命阿桂赴江南勘高堰等處堤工。戊午,免江蘇清河等二十三州縣及淮安等五衛本年水災漕項漕米有差。辛酉,以福州將軍恆瑞剿賊怯懦,召來京,調鄂輝代之。賑直隸保安等七州縣旱災。壬戌,命江蘇、浙江撥濟福建軍需錢各五萬貫。

十一月甲子朔,加李侍堯、孫士毅太子太保,柴大紀太子少保。賜臺灣廣東莊、泉州莊義民御書扁額。壬申,以柴大紀固守嘉義,封一等義勇伯,世襲。免臺灣嘉義縣五十四年額賦。以巴延三奏達賴喇嘛遣使稱「夷使」,申飭之。乙酉,奎林以婪贓,褫職逮問,以保寧為伊犁將軍。調李世傑為四川總督,以書麟為兩江總督,陳用敷為安徽巡撫。

十二月丁未,福康安等敗賊於侖仔頂莊等處,解嘉義圍,晉封福康安、海蘭察公爵,各賞紅寶石頂、四團龍補褂。己酉,遷常青福州將軍。以舒常為湖廣總督,福長安為工部尚書。以福康安劾柴大紀、蔡攀龍戰守之功多不確實,諭:「柴大紀堅持定見,竭力固守。蔡攀龍奮勇殺賊,竟抵縣城。或在福康安前禮節不謹,致為所憎。豈可轉沒其功,遽加無名之罪?」以孫士毅調兵運械,不分畛域,賞雙眼花翎。戊午,以德成奏稱柴大紀貪縱廢弛,命福康安、李侍堯據實參奏,並以喀什噶爾辦事大臣雅德在福建時徇隱,逮之。庚申,伍拉納護福建巡撫。以永鐸為盛京將軍,尚安為烏魯木齊都統。

五十三年春正月丁卯,免兵差經過之福建晉江等二十縣本年額賦有差。辛未,明興奏山西永寧等處河清。丙戌,柴大紀褫職逮問。福州將軍常青以徇隱柴大紀褫職。

二月甲午朔,獲林爽文,賞福康安、海蘭察御用佩囊,議敘將弁有差。晉封大學士和珅三等伯爵。大學士阿桂、王傑,尚書福長安、董誥議敘。予孫士毅輕車都尉世職。乙未,釋黃仕簡、任承恩。壬寅,伊犁參贊大臣海祿以劾奎林失實褫職,與奎林俱罰在拜唐阿上效力。乙巳,立先賢有子後裔五經博士。辛亥,上巡幸天津。庚申,獲臺灣賊首莊大田,議敘提督許世亨等有差。辛酉,免天津府屬逋賦。壬戌,上御閱武樓閱兵。

三月戊辰,命侍郎穆精阿赴湖北,會同舒常查案。壬申,林爽文伏誅。癸未,再賞福康安、海蘭察紫韁、金黃辮珊瑚朝珠及福康安金黃腰帶。

夏四月辛丑,以旱命刑部減徒以下罪。丙午,上閱健銳營兵。庚戌,免江蘇清河等十八州縣、淮安等五衛上年水災額賦有差。己未,富勒渾、雅德以失察柴大紀論絞。

五月丁卯,蠲河南商丘等六州縣上年水災額賦有差。癸酉,蠲直隸保安等七州縣上年水災民田旗地額賦。庚辰,上秋獮木蘭。癸未,宥常青罪。庚寅,賑臺灣難民。

六月丙申,富綱奏緬甸孟隕差頭目業渺瑞洞等齎金葉表文進貢,諭護送迅來行在。戊戌,賑湖南漵浦縣水災。免安徽鳳陽等四府州衛上年水災額賦有差。辛丑,賑湖北長陽縣水災。丁未,免陜西華州等三州縣五十一年水災額賦。戊申,安南人阮惠等叛逐其國王黎維祁,維祁來求援。命孫士毅赴廣西撫諭之。免山西大同等九州縣上年旱災額賦。

秋七月辛酉朔,以安南牧馬官阮輝宿奉黎維祁之母及子來奔,諭孫士毅等撫恤之。壬戌,賑山東膠州、壽光水災。湖北荊州江溢,府城及滿城均浸沒,諭舒常等查勘撫恤。丁丑,賞還閩浙總督李侍堯伯爵,予現襲之李奉堯提督銜。戊寅,湖北武昌、漢陽江溢。以畢沅為湖廣總督,伍拉納為河南巡撫,明興為烏什辦事大臣。賑安徽懷寧等州縣水災。柴大紀處斬。召姜晟來京,以惠齡為湖北巡撫。戊子,廓爾喀據後藏濟嚨、聶拉木,命成德與穆克登阿剿之。

八月甲辰,賑湖北監利、石首水災。丙午,上幸木蘭。庚戌,以木蘭大水,停行圍。癸丑,廓爾喀復陷宗喀,以鄂輝為將軍、成德為參贊大臣剿之。丙辰,安南阮岳等遁,命孫士毅督許世亨進剿,命富綱統兵進駐蒙自。戊午,上回駐避暑山莊。

九月壬戌,緬甸番目細哈覺控等入覲,諭暹羅、緬甸現均內附,二國應修好,不得仍前構兵。戊辰,賑湖北沔陽、黃岡水災。癸酉,免安徽宿州等二十一州縣衛上年水災額賦。

冬十月庚寅,廓爾喀侵後藏薩喀。命孫士毅出關督剿。甲午,賑湖北潛江水災。丙申,賑湖北江夏等三十六州縣水災。己亥,以黎維祁闇弱,諭孫士毅選擇黎裔入京朝貢。庚子,命雲南提督烏大經統兵出關,檄諭阮惠等來歸。癸卯,調舒濂為駐藏大臣,以恆瑞為伊犁參贊大臣。調都爾嘉為盛京將軍,恆秀為吉林將軍。改嵩椿為西安將軍,以興兆代之。琳寧為黑龍江將軍。乙卯,李侍堯病,命福康安署閩浙總督。

十一月辛酉,免安徽望江等二十六州縣衛本年被水額賦有差。癸亥,李侍堯卒,以福康安代之。以勒保為陜甘總督,海寧為山西巡撫。丙子,修湖北江陵、公安各堤。免湖北江陵等三十六州縣本年水災額賦有差。

十二月己丑,釋富勒渾、雅德。孫士毅奏敗賊於壽昌江。癸巳,又敗賊於市球江。丙申,收復黎城,復封黎維祁安南國王,封孫士毅為一等謀勇公,許世亨為一等子。戊申,命孫士毅班師。

五十四年春正月己未,以元旦受賀,朝班不肅,褫糾儀御史等職,尚書德保摘翎頂,都察院、鴻臚寺堂官均下部嚴議。庚申,成德以收復宗喀、濟嚨,克聶拉木奏聞。癸酉,禮部尚書德保卒,以常青代之。甲戌,以緬甸孟隕悔罪投誠,諭令睦鄰修好,並賜暹羅國王鄭華糸採幣,令其解仇消釁。免福建淡水等六縣災欠額賦。癸未,阮惠復陷黎城,廣西提督許世亨等死之。召孫士毅來京,削公爵。調福康安為兩廣總督。以伍拉納為閩浙總督,梁肯堂為河南巡撫。以海祿為廣西提督。甲申,安南國王黎維祁復來奔,命安插廣西。丙戌,褫孫士毅職,命仍以總督頂戴在鎮南關辦事。

二月庚寅,以京察屆期,予大學士阿桂等議敘,內閣學士謝墉等下部議處,理籓院侍郎福祿原品休致,予總督福康安等議敘。丁酉,勒保陛見,以巴延三署陜甘總督。和闐領隊大臣格繃額以婪索鞫實,處斬。甲寅,調蘭第錫為江南河道總督,李奉翰為河東河道總督。乙卯,以安南瘴癘炎荒,不值用兵,詳諭福康安。

三月甲子,免甘肅積年逋賦及未完籽種口糧。免陜西延安等三府州未完倉穀。諭福康安檄阮惠縛獻戕害提鎮之匪。乙丑,劉墉以上書房師傅曠職,降侍郎銜。以彭元瑞為吏部尚書,孫士毅為兵部尚書。丁卯,上幸盤山。

夏四月戊子,免奉天廣寧、鳳凰二城屬上年水災額賦,仍賑恤有差。丙申,晉贈許世亨伯爵,令其子承謨襲。召孫士毅回京。庚子,以恆瑞為烏里雅蘇臺將軍,福長安署兵部尚書。諭福康安安插安南黎氏宗族舊臣。予從軍出力之諒山都督潘啟德以都司用。壬寅,命阿桂覆勘荊州堤工。丁未,宣諭:「安南水土惡劣,決計不復用兵。阮惠已三次乞降,果赴闕求恩,可量加封號。朕撫馭外夷,無不體上天好生之德,從未敢窮兵黷武。」辛亥,賜胡長齡等九十八人進士及第出身有差。調都爾嘉為黑龍江將軍,嵩椿為盛京將軍,恆秀為綏遠城將軍,琳寧為吉林將軍。癸丑,以阮惠不親來籲懇,遣阮光顯入關進貢,諭福康安卻之。丙辰,豁直隸宣化等四縣上年旱災額賦。

五月己未,免官兵經過之廣西柳州等五府屬本年額賦。福康安等奏安南阮惠遣其侄阮光顯賚表貢乞降,並籲懇入覲。許之,卻其貢。乙酉,增伊犁惠遠城、惠寧城官。

閏五月庚寅,上秋獮木蘭。辛卯,免奉天廣寧等七城上年水災額賦。甲午,賑雲南通海等五州縣地震災民。

六月,免安徽安慶等七府州五十三年水災額賦。甲子,以管幹貞為漕運總督。戊辰,賑直隸蠡縣水災。庚午,命兵部尚書孫士毅軍機處行走。壬申,以郭世勛為廣東巡撫。癸酉,以陳步瀛為貴州巡撫。丙子,福康安奏,阮惠即阮光平,因赦其前罪,準令降附,具表謝恩進貢,並求於明年到京祝釐。上以其情詞肫切,冊封為安南國王,並賜敕諭。免湖北江夏等二十四州縣上年水災額賦。

秋七月乙酉朔,以決河下注泗州一帶,諭賑恤災民。丁酉,賑直隸安州等八州縣水災。庚子,戶部尚書綽克托卒。丙午,以巴延三為戶部尚書,秦承恩為陜西巡撫。戊申,安南貢使阮光顯等入覲。

八月乙丑,賑河南永城、臨漳等縣水災。戊辰,賑安徽宿州水災。己巳,上幸木蘭行圍。甲戌,賑直隸清苑等三十四州縣水災。

九月己丑,廓爾喀貢使入覲,封拉特納巴都爾王爵,巴都爾薩野公爵。庚寅,上回駐避暑山莊。辛卯,賑江蘇銅山等十一州縣水災。丙申,賑吉林屬琿春水災,豁應交義倉糧石及上年借給倉穀。丁酉,上回蹕。丙午,安南黎維祁自保樂襲牧馬,為阮光平所敗。諭福康安,如黎維祁來奔,收納之。辛亥,左都御史阿揚阿卒,以舒常代之。

冬十月癸丑,察哈爾都統烏爾圖納遜罷,以保泰代之。命伍爾伍遜為科布多參贊大臣。乙卯,以佛住為烏里雅蘇臺參贊大臣。賑吉林打牲烏拉等處水災。己未,睢寧決口合龍。辛酉,賑湖南華容等縣水災。

十一月乙酉,安南國王阮光平以受封進謝恩貢物,允之。丙戌,免安徽宿州等十四州縣衛逋賦。庚寅,命福康安將黎維祁及其屬人送京師,隸漢軍旗籍,以黎維祁為世管佐領。癸巳,四川總督李世傑病,命侍衛慶成帶醫診視,以孫士毅署之,彭元瑞署兵部尚書。戊戌,免盛京等五城借倉穀。

十二月庚申,追奪故大學士馮銓等謚。辛未,上以來年八旬萬壽,命金雋八徵耄念之寶。

五十五年春正月壬午朔,以八旬萬壽,頒詔覃恩有差。普免各直省錢糧。己丑,頒恩詔於朝鮮、安南、琉球、暹羅等國。壬辰,賞大學士和珅黃帶、四開褉袍。賜安南國王阮光平金黃鞓帶。乙巳,朝鮮國王李算表賀萬壽,貢方物。己酉,琉球國王尚穆進表謝恩,貢方物。

二月壬子朔,以河南考城城工錯繆,降江蘭道員,畢沅等褫職,仍留任。癸丑,免直隸永清、武清五十四年水災額賦。己未,上詣東陵、西陵,巡幸山東,免經過直隸州縣錢糧十分之三。壬戌,上謁昭西陵、孝陵、孝東陵。庚午,上謁泰陵、泰東陵。辛未,免直隸各屬節年因災緩徵錢糧。壬申,命福康安帶同阮光平入覲,郭世勛兼署兩廣總督。乙亥,免雲南通海等五州縣五十四年分地震災田額賦,並除傍海震沒田賦。免經過山東錢糧十分之三。降直隸總督劉瓘侍郎,以梁肯堂為直隸總督,調穆和藺為河南巡撫。戊寅,免山東各屬因災緩徵銀兩。以福崧為安徽巡撫。

三月乙酉,上登岱。甲午,上謁少昊陵。至曲阜謁先師廟。乙未,釋奠。賜衍聖公孔憲培及孔氏族人等章服銀幣有差。丙申,上謁孔林。庚子,免烏魯木齊各州縣額徵地糧十分之一。乙巳,緬甸國長孟隕遣使表賀萬壽,貢馴象,請封號。命封為緬甸國王。免直隸昌平等七州縣水災旗地租銀。南掌國王召溫猛表賀萬壽,貢馴象。己酉,免直隸長蘆等五場上年水災灶課。

夏四月丁巳,上幸天津府。諭伍拉納查浙江浮收漕糧情弊。己未,大學士嵇璜重與恩榮宴,禦制詩章賜之。辛酉,命吉慶會同嵩椿勘明英額邊至靉陽邊。乙丑,免安徽宿州、靈壁等八州縣衛上年水災額賦。上還京師。丙寅,上詣黑龍潭祈雨。閔鶚元罷,調福崧為江蘇巡撫,何裕城為安徽巡撫。庚午,以書麟覆奏欺飾,下部嚴議,仍留任。閔鶚元褫職逮問。壬申,免河南永城五十四年水災額賦。癸酉,以孫士毅為四川總督,李世傑為兵部尚書。乙亥,賜石韞玉等九十七人進士及第出身有差。己卯,免山西太原、遼州等十六府州並歸化城等處額賦十分之三。

五月庚寅,上幸避暑山莊。庚子,賞黎維祁三品職銜。壬寅,免西藏所屬三十九部落錢糧。己酉,書麟褫職逮問,福崧兼署兩江總督。韓鑅赴江南幫辦河工。

六月壬子,調孫士毅為兩江總督,保寧署四川總督,永保署伊犁將軍。乙卯,以陳用敷為廣西巡撫。閔鶚元論斬。丁巳,免直隸霸州等五十四州縣並各屬旗地上年水災額賦。戊午,除湖南乾州等五縣苗民雜糧。

秋七月己丑,安南國王阮光平入覲。庚寅,以硃珪為安徽巡撫。甲午,賑直隸朝陽、天津水災。丙申,賑奉天錦州九關臺,山東平原、禹城等縣水災。丁酉,兵部尚書李世傑以失察書吏休致。己亥,起劉瓘為兵部尚書。戊申,上還京師。賑江蘇碭山等縣,安徽宿州,河南永城、夏邑水災。江蘇碭山王平莊河決。命福崧赴宿州辦河工。丁未,賑山東臨清水災。

八月庚戌,暹羅國王鄭華表賀萬壽,貢方物。瑯玕以失察漕糧自劾,罷之。調海寧為浙江巡撫,書麟為山西巡撫。辛酉,上八旬萬壽節,御太和殿,王、貝勒、貝子、公、文武大臣,蒙古汗、王、貝勒、貝子、公、額駙、臺吉,回部王、公、臺吉、伯克,哈薩克、安南國王、朝鮮、緬甸、南掌貢使,各省土司,臺灣生番等行慶賀禮。禮成,寧壽宮、乾清宮賜宴如儀。己巳,刑部尚書喀寧阿卒,以明亮代之,命舒常兼署。

九月戊寅,賑安徽泗州水災。癸未,命安南國王阮光平歸黎維祁親屬及舊臣之在其國者。己丑,上閱健銳營兵。甲午,賑山東平原等二十七州縣水災。庚子,長麟以讞獄不實褫職,調惠齡為山東巡撫,以福寧為湖北巡撫,畢沅兼署之。

冬十月丙辰,賑山東平原等二十七州縣水災。甲子,命保寧回伊犁將軍,以鄂輝為四川總督。壬申,以福崧為浙江巡撫,起長麟署江蘇巡撫。賑甘肅皋蘭等三縣霜災。

十一月丁丑朔,以浦霖為福建巡撫,馮光熊為湖南巡撫。丙戌,加大學士王傑太子太保,尚書彭元瑞、董誥、胡季堂、福長安、將軍保寧太子少保。乙未,釋富勒渾、雅德。戊戌,命慶成同尹壯圖往山西盤查倉庫。壬戌,賑奉天錦縣等三州縣水災。戊辰,命吏部尚書彭元瑞協辦大學士。

五十六年春正月丁丑,賑江蘇蕭縣等三縣、安徽宿州等三州縣上年水災。己卯,賑直隸文安等三十州縣、山東平原等二十七州縣水災。乙酉,以尹壯圖覆奏欺罔,褫職治罪。戊戍,袁鳳鳴處斬。朝鮮、暹羅、緬甸均遣使謝恩,貢方物。賞賚筵宴如例。己亥,以保寧為御前大臣。甲辰,調劉墉為禮部尚書,紀昀為左都御史。

二月己酉,諭:「朕孜孜求治,兢惕為懷。尹壯圖逞臆妄言,亦不妨以謗為規。加恩免尹壯圖治罪,以內閣侍讀用。」戊午,御試翰林詹事等官,擢阮元等二員為一等,餘升黜有差。

三月乙亥,賑奉天錦州等處上年水災旗地人戶,並蠲租有差。戍寅,上幸盤山。甲申,免甘肅皋蘭等三縣上年霜災額賦。丁酉,以永保為內大臣。

夏四月丁卯,免山東臨清等三十州縣衛上年水災額賦。辛未,彭元瑞以瞻徇降侍郎,命孫士毅為吏部尚書。以書麟為兩江總督,長麟暫署。調馮光熊為山西巡撫。以姜晟為湖南巡撫。

五月庚寅,以長麟為江蘇巡撫。乙未,上秋獮木蘭。辛丑,上駐蹕避暑山莊。

六月甲辰朔,免直隸霸州等六十九州縣上年水災額賦。

秋七月庚辰,免江蘇江寧等五府州屬因災積逋半賦。甲申,以緬甸國王孟隕資送羈留內地人民,嘉賚之。己亥,蠲安徽宿州等十九州縣衛上年水災額賦。辛丑,蠲陜西朝邑等二縣逋賦。

八月丁未,命喇特納錫第為喀喇沁札薩克一等塔布囊。戊午,上幸木蘭行圍。甲子,上行圍。廓爾喀以逋欠誘圍喇嘛、噶布倫,擾西藏。命四川總督鄂輝、將軍成德剿之。命孫士毅署四川總督。己巳,命福康安來京祝其母生辰,郭世勛署兩廣總督。廓爾喀陷西藏定日各寨,據濟嚨。

九月丙子,上回駐避暑山莊。庚辰,召嵩椿回京,以琳寧為盛京將軍,調恆秀為吉林將軍。丙戌,上回蹕。戊子,唐古忒兵與達木蒙古兵禦廓爾喀失利,唐古忒公札什納木札勒及達木協領澤巴傑等死之。命乾清門侍衛額勒登保等赴西藏軍營。壬辰,以保泰懦怯褫職,命奎林赴藏辦事,賞舒濂副都統銜,協同辦理。以達賴喇嘛等堅守布達拉,嘉獎之。命劉墉署吏部尚書。甲午,以廓爾喀圍扎什倫布,諭鄂輝等進剿。辛丑,豁奉天廣寧縣逋賦。

冬十月乙巳,宥閔鶚元罪。丁未,廓爾喀入扎什倫布,尋遁去。癸丑,戶部尚書巴延三以浮估城工褫職,調福長安代之。以金簡、彭元瑞為滿、漢工部尚書。丙辰,以安南開關通巿,改廣西龍州通判同知。乙丑,諭王大臣不必兼議政虛銜。

十一月癸酉,授福康安為將軍,海蘭察、奎林為參贊,徵廓爾喀。辛巳,鄂輝、成德褫職,以惠齡為四川總督,奎林為成都將軍,吉慶為山東巡撫。癸未,以陳淮為貴州巡撫。

十二月辛亥,命海蘭察等及索倫、達呼爾兵由西寧進藏。丁卯,召都爾嘉回京。以明亮為黑龍江將軍,明興為喀什噶爾參贊大臣。

五十七年春正月壬申,賞七代一堂致仕上駟院卿李質穎御書扁額。免奉天、直隸、安徽、湖南、廣東逋賦。乙亥,以達賴喇嘛復遣丹津班珠爾等私與廓爾喀議和,諭止之。丙子,追論巴忠與廓爾喀議和擅許歲銀罪。甲午,以蘇凌阿為刑部尚書。

二月壬寅,成德奏敗賊於拍甲嶺。癸卯,予大學士阿桂等、尚書福長安等、侍郎德明等、總督福康安等、巡撫長麟等敘。裁河東鹽政、鹽運使等官。移山西河東道駐運城。丁未,命皇十五子嘉親王祭先師孔子。免奉天錦州府屬上年旱災額賦。己巳,命侍郎和琳管理藏務。鄂輝等奏收復聶拉木,諭以遲延斥之。

三月丁丑,上詣西陵,巡幸五臺山,免經過地方本年錢糧十分之三。戊寅,允濟嚨呼圖克圖「慧通禪師」法號。以帕克哩營官番眾收復哲孟雄、宗木,賚之。辛巳,上謁泰陵、泰東陵。壬午,免直隸大興等八州縣積欠米穀。甲申,加福康安大將軍。庚寅,免五臺本年錢糧十分之五,大同、朔平二府屬未完逋賦。辛卯,上駐蹕五臺山。

夏四月己亥朔,以和闐辦事大臣李侍政失察邁瑪特尼雜爾,下部嚴議。甲辰,上閱滹沱河。以貢楚克扎布為烏里雅蘇臺參贊大臣。丁未,上祭帝堯廟。甲寅,上還京師。乙卯,上詣黑龍潭祈雨。命刑部清理庶獄,減徒以下罪。

閏四月甲申,以久旱,諭臺灣及沿海各省詳鞫命盜各案,毋有意從嚴。蠲河南湯陰等五縣上年旱災額賦。丙申,以久旱,下詔求言。丁酉,雨。以失陷札什倫布,治仲巴呼圖克圖及孜仲喇嘛等罪。命和琳、鄂輝宣諭達賴喇嘛等。

五月辛丑,定安南國兩年一貢,六年遣使一朝。丁未,上幸避暑山莊,免經過地方錢糧十分之五。戊申,調長麟為山西巡撫,以奇豐額為江蘇巡撫。辛亥,允霍罕額爾德尼伯克那爾巴圖遣使入貢。癸丑,上駐蹕避暑山莊。

六月甲戌,福康安奏克廓爾喀所踞擦木耍隘。丁丑,賑江西南豐、廣昌水災。福康安奏殄瑪噶爾轄爾甲山梁之賊。己卯,福康安等奏克濟嚨。辛巳,調陳淮為江西巡撫,馮光熊為貴州巡撫。丙戌,福康安等奏攻克熱索橋。丁酉,福康安等奏攻克協布魯寨。

秋七月甲辰,賑直隸河間等處旱災,順直宛平、玉田等州縣蝗。己酉,福康安等克廓爾喀東覺山梁,並雅爾賽拉等處營卡,成德等克扎木、鐵索橋等處。

八月辛未,成德克多洛卡、隴岡等處。命孫士毅駐前藏督糧運。癸酉,命福康安為武英殿大學士,孫士毅為文淵閣大學士。調金簡、劉墉為吏部尚書,和琳為工部尚書,紀昀為禮部尚書,竇光鼐為左都御史。庚辰,以博興為庫倫辦事大臣。丙戌,福康安等奏克噶勒拉、堆補木城卡,阿滿泰、墨爾根保陣亡。成德等克利底、大山賊卡。戊子,福康安奏廓爾喀酋拉特納巴都爾等乞降。上以其悔罪乞降,許之,命班師。丙申,賑陜西咸寧等六州縣旱災。

九月丁酉,上還京師。己亥,論征廓爾喀功,賞福康安一等輕車都尉,晉海蘭察二等公為一等,議敘孫士毅等各有差。丙午,上命福康安、孫士毅等會商西藏善後事宜。命御前侍衛惠倫等齎金奔巴瓶往藏,貯呼畢勒罕名姓,由達賴喇嘛等對眾拈定。壬子,復廓爾喀王公封爵,定五年一貢。

冬十月戊辰,廓爾喀貢使入覲。己巳,賑河南安陽等十六縣災民,蠲緩新舊額賦有差。己卯,免嵇璜、阿桂翰林院掌院學士,以和珅、彭元瑞代之。壬午,賑直隸河間、任丘五州縣旱災,並免順天等十三府州屬被災旗民額賦。乙酉,郭世燾奏英吉利遣使,請由天津進貢,允之。丁亥,以鄂輝隱匿廓爾喀謝恩表貢褫職,交福康安等嚴鞫之。賑陜西咸陽等十四州縣旱災。癸巳,調圖桑阿為綏遠城將軍。

十一月丙午,賑山東德州等二十州縣旱災。

十二月庚午,定唐古忒番兵訓練事宜。鑄銀為錢,文曰「乾隆寶藏」。甲戌,免長蘆興國等五場並滄州等七州縣被災灶地額賦。丙子,以長麟為浙江巡撫,蔣兆奎為山西巡撫。以伊犁回民地畝雪災,免本年額穀。癸未,賑河南安陽等二十五縣旱災。辛卯,命永遠枷號鄂輝等於西藏。

五十八年春正月丙申,賑河南林縣等五縣、陜西咸寧等三州縣旱災。己亥,賑直隸保定等二十一州縣旱災。庚子,改杭州織造為鹽政兼管織造事,改鹽道為運司,南北兩關稅務歸巡撫管理。以全德為兩浙鹽政。恆秀回吉林將軍。乙巳,敕諭安南國王阮光平睦鄰修好,慎守封疆,賜以糸採幣。丙辰,安南國王阮光平卒,以世子阮光纘嗣。乙亥,免河南安陽等二十五縣上年旱災額賦。壬午,命喀什噶爾阿奇木伯克作為喀什噶爾協辦大臣。

三月丁酉,上幸盤山。庚子,上駐蹕盤山。甲辰,禮部尚書常青卒,以德明代之。戊申,諭於雍和宮設金奔巴瓶,飭理籓院堂官、掌印札薩克喇嘛等,公同掣蒙古所出之呼畢勒罕。丁未,上回蹕。乙卯,調馮光熊為雲南巡撫,以英善為貴州巡撫。戊午,領侍衛內大臣海蘭察卒。

夏四月壬申,命松筠為內務府總管大臣,在御前侍衛上行走。辛巳,通諭設金奔巴瓶於前藏大昭及雍和宮,公同掣報出呼畢勒罕,以除王公子弟私作呼畢勒罕陋習。乙酉,刪除大學士兼尚書銜、翰林院掌院學士兼禮部侍郎銜、順天府府丞兼提督學政銜。丁亥,賜潘世恩等八十一人進士及第出身有差。戊子,命於乾隆五十九年秋特開鄉試恩科,六十年春為會試恩科。庚寅,廓爾喀歸西藏底瑪爾宗地方。以西藏卡外之拉結、撒黨兩處歸廓爾喀。

五月乙未,命廣西按察使成林赴安南升隆城,賜奠冊封。丁未,上幸避暑山莊。己酉,以明興未奏遣回人赴霍罕等處辦理外籓事件,罷喀什噶爾參贊大臣,調永保代之。以伍彌伍遜為塔爾巴哈臺參贊大臣,貢楚克札布為科布多參贊大臣。以特成額為烏里雅蘇臺參贊大臣。辛酉,加封福康安為一等忠銳嘉勇公。癸丑,上駐蹕避暑山莊。

六月己卯,賑四川泰寧地震災。乙酉,英吉利貢船至天津。戊子,於通州起陸。命在天津筵宴之。

秋七月癸巳,命和琳稽覈藏商出入。壬寅,命英吉利貢使等住宏雅園,金簡、伊齡阿於圓明園分別安設貢件。己酉,以旱命刑部清理庶獄,減徒以下罪。庚午,上御萬樹園大幄,英吉利國正使馬戛爾尼、副使斯當東等入覲。辛未,調福康安為四川總督,以惠齡暫代,長麟為兩廣總督,調吉慶為浙江巡撫,惠齡為山東巡撫。壬午,免長蘆官臺等二場潮災灶地額賦。丙戌,上還京師。戊子,以慶桂為兵部尚書。庚寅,諭英吉利貢使由內河水路赴廣東澳門附船回國。

九月丁酉,加長麟太子少保。命松筠護送英吉利使臣等至浙江定海。甲辰,調福寧為山東巡撫,惠齡為湖北巡撫。丙午,以安徽無為等三州縣水災,賞口糧有差。

冬十月癸亥,安南國王阮光纘表進謝恩,貢物二分,納其一。戊子,以長麟奏英吉利使稱再進表章貢物,呈總督轉奏,諭:「系援例而行,並無他意,國王可安心,再來表貢,亦不拘定年限。」

十一月甲午,命和寧赴藏幫同和琳辦事。戊午,以上年各省奏報民數三萬七百四十六萬有奇,較康熙四十九年增十五倍,諭:「生之者寡,食之者眾,勢必益形拮據。各省督撫及有牧民之責者,務當勸諭化導,俾皆儉樸成風,服勤稼穡,惜物力而盡地利,共享升平之福。」己未,以安南等國進象已多,諭云貴、兩廣督撫檄卻象貢。

十二月癸未,伍拉納陛見,命吉慶署閩浙總督。

五十九年春正月庚寅,免直隸、山東、河南逋賦十分之三。庚戌,管幹貞病免,命書麟兼署漕運總督。乙卯,恆秀以侵帑褫職,調寶琳為吉林將軍,松筠署之。戊午,安置安南內附人黎維治於江南。

二月庚申,以明年元旦上元值日月食,諭修省,毋舉行慶典。癸亥,廓爾喀遣使進表貢。丁亥,增造廣東水師戰船。

三月己丑,恆秀論絞。庚子,上巡幸天津,免經過地方及天津府屬額賦十分之三,免天津府屬逋賦,免大興等十三州縣逋賦十分之四。壬子,上駐蹕天津府。

夏四月壬戌,常雩,命皇八子儀郡王永璇代行禮。癸亥,上還京師。丁丑,上詣黑龍潭祈雨。

五月丙申,京師雨。甲辰,郭世勛病免,調硃珪為廣東巡撫,陳用敷為安徽巡撫。丙午,以直隸保定等八十三州縣旱,命賞給一月口糧。減奉天商販豆麥等項經過直隸、山東關津稅。辛亥,上幸避暑山莊,免經過地方錢糧有差。

六月丙辰朔,以山東歷城山莊等五十一州縣旱,給貧民一月口糧,除山東臨清州水沖地畝田賦。丁巳,上駐蹕避暑山莊。庚午,設唐古忒西南外番布魯克巴、哲孟雄、作木朗、洛敏湯、廓爾喀各交界鄂博。

秋七月戊子,永定河決。庚寅,河南丹、沁二河決。辛卯,賑山西平定等處水災。己亥,賑山東臨清等州縣水災。辛丑,賑直隸天津等處水災。癸卯,河南豐北曲家莊河決。甲辰,書麟以徇隱鹽政巴寧阿交結商人褫職,調富綱為兩江總督,命蘇凌阿署之。調福康安為雲貴總督。以和琳為四川總督,孫士毅署之。以駐藏辦事松筠為工部尚書。乙巳,命馮光熊署云貴總督。大學士嵇璜卒,召孫士毅入閣辦事。癸丑,停本年及明年木蘭行圍。免直隸保定等府屬、河南★輝等府屬、山東臨清等五州縣、山西代州等三州縣被水額賦。

八月丁巳,以直隸天津、河間二府水災重,免因災緩徵額賦。戊午,永定河南工決口合龍。己巳,以明歲御宇屆六十年,普免各省漕糧一次。甲戌,上回蹕。調福寧為河南巡撫,穆和藺為山東巡撫,江蘭護之。福康安奏四川大寧教匪謝添秀等傳習邪教,蔓延陜西、湖北、河南,諭嚴為捕治。丁丑,免直隸通州等二十三州縣逋賦。甲申,畢沅降山東巡撫,罰繳湖廣總督養廉五年。以福寧為湖廣總督,穆和藺留為河南巡撫。

九月己丑,賑湖北沔陽等州縣水災。丙申,以秀林為吉林將軍。己亥,賑福建漳、泉二府水災。減直隸遵化內務府官地租。命福寧駐襄陽,督緝邪教案犯。辛丑,以校正石經,加彭元瑞太子少保銜。癸卯,賑廣東高要等縣水災。以湖北來鳳縣教匪段漢榮等糾眾拒捕,諭責畢沅廢弛。戊申,免齊齊哈爾等三城水災逋賦。

冬十月丙辰,免河南汲縣等九縣、山東臨清等十州縣逋賦。壬戌,勒保奏獲邪教首犯劉松。命安徽嚴緝其徒劉之協。癸亥,荷蘭入貢。乙丑,免福建漳州府屬四州縣本年水災額賦。戊辰,命將科布多威豁爾等七卡移駐原處北界,餘地賞杜爾伯特汗瑪克素爾札布等游牧。己卯,調陳用敷為湖北巡撫,惠齡為安徽巡撫。辛巳,釋恆秀罪。

十一月丙戌,以河南扶溝縣知縣劉清鼐疏防劉之協潛逃,革逮,穆和藺下部嚴議。壬辰,免山東臨清等州縣本年漕賦。壬寅,命富綱署刑部尚書。甲辰,穆和藺褫職,發烏魯木齊效力。以阿精阿為河南巡撫。

十二月丙辰,普免各省積年逋賦。丙子,吏部尚書金簡卒,以保寧代之。以明亮為伊犁將軍。戊寅,命舒亮為黑龍江將軍。改綏遠城將軍圖桑阿為西安將軍,以永琨代之。

六十年春正月甲申朔,日食,免朝賀。乙酉,賑直隸天津等二十州縣、河南汲縣等十四縣、山東臨清等十州縣上年被水貧民有差。丙戌,召蘇凌阿來京,調福寧為兩江總督,復以畢沅為湖廣總督,玉德為山東巡撫。戊子,調陳用敷為貴州巡撫,英善為湖北巡撫,畢沅兼署。乙未,以固倫額駙豐紳殷德為內務府大臣。辛丑,免山東積年逋賦。庚戌,免江蘇積年逋賦。免江西應緩徵銀穀。

二月癸丑朔,免廣東積年逋賦。陳用敷以查拏要犯劉之協辦理錯謬,褫職逮問。調姚棻為貴州巡撫,以成林為廣西巡撫。丙辰,免陜西積年逋賦。貴州松桃苗匪石柳鄧等、湖南永綏苗匪石三保等作亂。戊午,湖南苗匪陷乾州,同知宋如椿等死之。命福康安往剿,畢沅駐常德籌辦糧餉。庚申,以大學士阿桂等書上諭不能稱旨,停甄敘,侍郎成策等下部議處。予總督福康安等議敘。辛酉,貴州苗匪圍鎮遠鎮總兵珠隆阿於正大營。免奉天廣寧、錦州旗地逋賦。免甘肅皋蘭等四十五州縣積年逋賦。丙寅,命四川總督和琳赴酉陽州備苗,孫士毅仍留四川辦理報銷。丁卯,免浙江積年民地灶地逋賦。己巳,苗匪陷永綏鴉酉寨,鎮筸鎮總兵明安圖等死之。辛未,湖南永順苗匪張廷仲等作亂,擾保靖、瀘溪。丙子,免安徽積年逋賦。壬午,貴州苗匪擾思南、印江一帶,竄入四川秀山。福康安赴銅仁督剿。命德楞泰領巴圖魯侍衛等赴貴州軍營。

閏二月乙酉,福康安奏解正大營之圍。壬辰,馮光熊留為貴州巡撫,調姚棻為雲南巡撫。以苗匪亂,免貴州銅仁府屬松桃、正大等處額賦。乙未,上詣東陵,免經過地方錢糧十分之三。戊戌,上謁昭西陵、孝陵、孝東陵、景陵。己亥,福康安奏解嗅腦圍。乙巳,福康安奏攻克石城,剿除巖洞苗匪。丁未,上謁泰陵、泰東陵,尊孝賢皇后陵。免兩淮場灶積欠。戊申,福康安奏解松桃之圍。

三月乙卯,和琳奏肅清秀山後路,命往松桃與福康安會剿。以孫士毅署四川總督。己未,福康安奏殄除長沖、卡落苗匪,進兵楚境。命額勒登保迅赴福康安軍營。己卯,福康安奏解湖南永綏匪圍。

夏四月辛卯,臺灣彰化匪徒陳周全等作亂,陷縣城,尋復之。癸巳,竇光鼐以會試衡文失當,降調。以硃珪為左都御史,仍留廣東巡撫任。己亥,以魁倫劾洋盜肆行,命浦霖來京候旨,調姚棻為福建巡撫,以魁倫署之,江蘭為雲南巡撫。庚子,賜王以銜等一百十一人進士及第出身有差。癸卯,賞會試薦卷文理較優之舉人徐炘、傅淦、李端內閣中書。戊申,上詣廣潤祠祈雨。是夜,雨。丁未,免貴州官兵經過地方本年額賦有差。福康安等奏克黃瓜寨。己酉,以福寧、惠齡經理湖南軍務未竣,命蘇凌阿仍署兩江總督,費淳為安徽巡撫。庚戌,免福建龍溪等四縣上年水災額賦有差。匪首陳周全等伏誅。

五月丙辰,上幸避暑山莊。伍拉納、浦霖以辦理災賑不善,褫職鞫治。命魁倫兼署閩浙總督。免經過地方本年錢糧十分之三。丁巳,調費淳為江蘇巡撫,仍留惠齡為安徽巡撫。福康安等奏克構皮寨及蘇皮寨等處。調福康安為閩浙總督,勒保為四川總督。以宜綿為陜甘總督。壬戌,上駐蹕避暑山莊。甲子,以福建倉庫虧缺查實,申飭科道無人奏及,並命嗣後陳奏地方重大事件,毋忝言責。召阿精阿來京,以景安為河南巡撫。丁卯,召惠齡來京,以汪新為安徽巡撫。戊辰,命蘇凌阿駐清江浦,兼署江蘇巡撫。辛未,以於敏中營私玷職,褫輕車都尉世職。

六月壬午,以湖南苗匪擾鎮筸後路,諭責福寧怯懦,劉君輔株守。命惠齡仍署湖北巡撫。戊子,以旱命刑部清理庶獄,減徒以下罪,承德府如之。庚寅,福康安等奏克沙兜、多喜等處苗寨。乙未,賑廣東南海等縣水災。戊申,姚棻以質訊解任,命魁倫兼署福建巡撫,長麟署閩浙總督。

秋七月庚申,德明以罣累滋陽縣知縣陳照自縊,論絞。乙丑,免湖北江陵等十二州縣衛上年水災額賦。丙寅,以福康安等奏連克苗寨,渡大烏草河,賚珍物。壬申,哲布尊丹巴呼圖克圖等入覲,召見賜茶。

八月壬午,調永琨為烏里雅蘇臺將軍,恆瑞為綏遠城將軍。癸未,賜南掌國王召溫猛、緬甸國王孟隕敕諭,均賚文綺。丙申,允兵部尚書劉瓘乞休,以硃珪代之,仍留廣東巡撫任。以金士松為左都御史。丁未,免直隸通州等五十二州縣積欠旗租。福康安等進駐楊柳坪。

九月辛亥,上御勤政殿,召皇子、皇孫、王、公、大臣等入見,宣示立皇十五子嘉親王為皇太子,明年為嗣皇帝嘉慶元年。撫恤江蘇海州等七州縣水災。壬子,皇太子及王、公、內外文武大臣,蒙古王、公等各奏籲請俟壽躋期頤,再舉行歸政典禮,不允。丙辰,富勒渾、雅德以前總督婪贓,均褫職,分別發熱河、伊犁效力。己未,上閱健銳營兵。晉封福康安忠銳嘉勇貝子,和琳一等宣勇伯。庚申,上命皇太子謁東陵、西陵。乙丑,黑龍江將軍舒亮以婪索,褫職鞫治,調永琨代之。命圖桑阿為烏里雅蘇臺將軍。改恆瑞為西安將軍,以烏爾圖納遜代之。命博興為察哈爾都統。調特克慎為庫倫辦事大臣,策巴克為西寧辦事大臣。丙寅,明亮以任黑龍江將軍時侵漁貂皮褫職,命保寧為伊犁將軍。己巳,舒亮論絞。明亮留烏魯木齊效力。癸酉,以奉天、山西、四川、湖南、貴州、廣西賦無逋欠,免明年正賦十分之二。乙亥,免福建龍溪等六縣,華封、羅溪二縣上年被水額賦。

冬十月戊寅朔,頒嘉慶元年時憲書。庚辰,福康安等奏擒匪首吳半生。賞福康安之子德麟副都統銜,和琳黃帶,餘議敘賞賚有差。甲申,以伍拉納等貪黷敗檢,戍其子於伊犁。長麟以徇庇伍拉納、浦霖褫職,命來京。以魁倫署閩浙總督,姚棻署福建巡撫。乙酉,普免天下嘉慶元年地丁錢糧。丙戌,伍拉納、浦霖處斬。壬辰,以額勒登保、德楞泰剿捕苗匪奮勇,授內大臣。乙未,命定丙辰年傳位典禮。癸卯,命明年正月初吉,重舉千叟宴。

十一月丁巳,福康安等奏克天星寨等處。加和琳太子少保銜,賞福康安、和琳上用黃裏玄狐端罩各一。庚申,賑奉天金州、熊岳、錦州三城,寧海等三州縣旱災旗民,免額賦有差。乙丑,上命皇太子居毓慶宮。

十二月戊寅朔,諭曰:「朕於明年歸政後,凡有繕奏事件,俱書太上皇帝。其奏對稱太上皇。」戊子,賑貴州銅仁被擾難民。福康安等奏克天星等苗寨。壬寅,允硃珪收英吉利國王表貢,賜敕嘉賚,交英商波郎齎回,並以其表言勸廓爾喀投順,於賜敕內以無須英國兵力告之。甲辰,賜琉球國王尚溫敕諭。丁未,以來歲元旦,傳位皇太子為嗣皇帝,前期遣官告祭天地宗社。

是歲,緬甸、南掌、暹羅、安南、英吉利、琉球、廓爾喀來貢。

嘉慶元年正月戊申朔,舉行授受大典,立皇太子為皇帝。尊上為太上皇帝,軍國重務仍奏聞,秉訓裁決,大事降旨敕。宮中時憲書用乾隆年號。

三年冬,上不豫。四年正月壬戌崩,壽八十有九。是年,四月乙未,上尊謚曰法天隆運至誠先覺體元立極敷文奮武孝慈神聖純皇帝,廟號高宗。九月庚午,葬裕陵。

論曰:高宗運際郅隆,勵精圖治,開疆拓宇,四征不庭,揆文奮武,於斯為盛。享祚之久,同符聖祖,而壽考則逾之。自三代以後,未嘗有也。惟耄期倦勤,蔽於權倖,上累日月之明,為之嘆息焉。


\end{pinyinscope}