\article{志一百十六}

\begin{pinyinscope}
○兵十二

△馬政

清初沿明制,設御馬監,康熙間,改為上駟院,掌御馬,以備上乘。畜以備御者,曰內馬;供儀仗者,曰仗馬。御馬選入,以印烙之。設蒙古馬醫官療馬病。上巡幸及行圍,扈從官弁,各給官馬。以副都統或侍衛為放馬大臣,主其事。上謁祖陵,需馬二萬三千餘匹,東西陵需馬四千三百餘匹,悉取察哈爾牧廠馬應之。迨乾隆時,每扈從用馬匹輒二萬餘。嘉慶中,物力漸耗,停木蘭秋獮。十二年,減額馬之半。道光九年,如盛京謁陵,額馬視乾隆時,約略相等,計取給廠馬暨各盟長所進,蓋二萬六千餘匹雲。

順治十五年定軍馬,親王出征,馬四百匹,郡王三百,貝勒二百,貝子百五十,鎮國公百匹,輔國公八十,不入八分鎮國公七十,輔國公六十五,將軍八十,副將軍七十,護軍統領、前鋒統領、副都統皆六十,其下各有差,最少者護軍、領催各六匹。康熙三十五年,敕出征兵一人馬四匹,四人為伍,一伍主從騎八匹,馱器糧用具亦八匹。是歲,徵噶爾丹,以兵丁馬瘦,褫兵部尚書索諾和職。五十一年,覈定軍中職官馬數,大學士、尚書、左都御史十六匹,侍郎以下遞減,經略、大將軍各二十五匹,副將軍以下遞減。乾隆十六年,八旗牧官馬二萬七千七百餘匹,以萬匹於都城外牧養,熱河千匹,各莊頭二千匹,餘者分畀直隸標營。圈馬之設,始乾隆二十八年,從都統舒赫德請也。滿洲八旗,旗養馬二百匹。蒙古八旗,旗百匹。洎五十九年撤圈,分給各兵拴養。嘉慶十二年,諭成親王永星議復圈馬,大學士戴衢亨等會議,立章程十條,圈馬仍舊。道光末,軍興遂廢,後亦不復籌矣。同治元年諭曰:「馬政廢弛,積弊已深,以致軍馬罷瘠。牧廠大臣等應妥實整頓,差功罪以挽頹風,著為令。」溯自世祖入關,迄於康、乾之際,盛京、吉林、黑龍江、直隸、江南、浙江、廣東、福建、湖北、四川、陜、甘、山東、山西諸省設駐防滿洲營,馬凡十萬六千四百餘匹,惟福建水師駐防僅數十匹。乾隆季年,定西藏兵制,前藏供差營馬六十匹,後藏二十匹,舊塘四十三,共塘馬二百二十匹,新設番塘二十四,共番馬九十八匹。黑龍江兵向無額馬,道光十六年,從哈豐阿請,始設置之。

天聰時,征服察哈爾,其地宜牧,馬蕃息。順治初,大庫口外設種馬廠,隸兵部。康熙九年,改牧廠屬太僕寺,分左翼右翼二廠,均在口外。是時,大凌河設牧廠一,邊墻設廠二,曰商都達布遜諾爾,曰達里岡愛,隸上駟院。尋分設牧廠五,曰大凌河牧群馬營,曰養息牧哈達牧群馬營,曰養息所邊外蘇魯克牧牛羊群,及黑牛群牧營,曰養息牧邊外牧群牛營,並在盛京境。凡馬牡曰兒,牝曰果,不及三歲曰駒,及壯擇割其牡曰騸。別其騍騸以為群,率騍馬五配兒馬一,群無過四百匹。騍馬及羊三年一平群,牛六年,騍馬群三歲以息補耗,三馬而取一駒,騸馬群歲耗其十一。置牧長、牧副、牧丁任其事,轄以協領、翼長、總管,官兵皆察哈爾、蒙古人充之。飼秣所需木槽鏇钁串杓,每群各二,五年一給之。總管三年番代。二十四年,定牧群牲畜歲終匯報增減數目,視其贏絀,以第賞罰。二十六年,令八旗豢馬,春夏驅赴察哈爾牧放,曰出青,秋冬回圈,曰回青。四十四年,將軍楊福請市馬給兵丁,上不許,諭曰:「朝廷屢以太僕寺廠馬並茶馬給各兵丁,故無賠馬之苦。歷觀宋、明議馬政,皆無善策。牧馬惟口外最善,水草肥美,不糜餉而孳生甚多。如驅入內地牧之,即日費萬金不足矣。」雍正三年,定在廠馬以四萬匹為率。至乾隆五年,足額外,溢七千餘匹。兩翼牧廠,共騍馬百六十群,騸馬十六群,令分在兩翼廠牧放。八年,敕牧界毋許侵越。先是甘、涼、肅三州及西寧各設馬廠,分五群,群儲牝馬二百匹,牡四十。尋改甘州廠屬巴里坤。二十五年,伊犁設孳生馬駝廠,畀錫伯、察哈爾、索倫、厄魯特四營牧之。三十二年,定牧廠官屬所需馬,視內地驛傳例,按官品給之,不得逾額。嘉慶中,從都統慶溥言,撤回厄魯特人牧廠。初,富俊建言,撤大凌河牧廠,分歸東三省,仁宗嚴諭斥之。迨道光七年,上經杏山東閱馬廠,見河岸馬群壯整。因諭是間牧廠寬闊,水草蕃滋,馬恃以生息,若輕議裁,則散之甚易,聚之甚難。再有率為此請者,以違制論。咸豐四年,科爾沁親王僧格林沁剿捻,檄取察哈爾戰馬六百匹,不堪乘用,奏聞。上大怒,嚴諭都統慶昀整頓,蓋馬政漸衰弛矣。光緒九年,太僕寺言兩翼騍馬騸馬一百十四群,並孳生馬五群,駝亦五群,較乾隆時群數大減。嗣是穆圖善練兵,至黑龍江求馬無良,愀然曰:「地氣其盡乎!」迨於末葉,厲行新法,舊時牧政益廢不講,豈非時勢使然歟?

順治初,陜西設洮岷、河州、西寧、莊浪、甘州茶馬司,及開成、安定、廣寧、黑水、清平、萬安、武安七監,歲遣御史一人專理之。七年,喀爾喀、額魯特來市馬,諭令自章京監察之販客及賈人,與不系披甲者,概不許購,違者鞭一百,馬入官。蒙古攜馬來京,不許商販私買,胥役私購者罪之。康熙七年,裁茶馬御史,以馬政歸甘肅巡撫。三十四年,諭遣師中等往蒙古諸旗購馬,歸化城、科爾沁各二千匹,餘定額有差。乾隆十二年,禁朝鮮買馬。二十五年,敕烏魯木齊市易哈薩克馬百三十餘匹歸巴里坤。旋以五吉等言,選哈薩克所易馬撥往巴里坤,遂停購買。阿桂言伊犁易來哈薩克馬漸成大群,敕書嘉予。二十八年,定江寧、浙江、福建駐防馬匹出口採買例。三十二年,以伊犁易哈薩克馬累積至多,擇巴里坤善地牧放。尋烏里雅蘇臺馬缺,亦以哈薩克馬換易之。陜、甘營馬,例調自伊犁轉補,道遠耗時。咸豐四年,用賡福請,由伊犁、塔爾巴哈臺隨地變價,令各營自購。七年,並敕山東缺額馬,亦就近買補云。

貢馬昉於國初,歸化城土默特二旗,每歲四時貢馬百匹。順治十三年,吐魯番貢三百二十四匹,嗣減令貢西馬四匹,蒙古馬十匹。康熙八年,以邊外蒙古貢馬,沿途抑買,諭嚴禁之。三十年,諭土謝圖、車臣俱留汗號,貢白駝一、白馬八如初,自餘毋以九白進。三十五年,喀爾喀蒙古獻駝馬,多不可計,感聖祖破噶爾丹,得歸原牧地也。四川各土司例貢及折徵馬,各營少者一、二匹,最多十二匹。甘肅唐古特七族西喇古兒例貢馬匹,各營最多者八十二匹,少者遞減至二、三匹。乾隆元年,諭四川土司折價馬每匹納銀十二兩,通省營馬改從驛馬例,納銀八兩,永著為令。三十年,哈薩克沁德穆爾等獻馬。敕其餘馬赴伊犁,毋於喀什噶爾諸地貿易。尋令沙拉伯爾游牧之哈薩克,與沙拉伯爾一體貢馬。嘉慶元年,停葉爾羌進馬。十六年,諭烏里雅蘇臺將軍等貢馬及備用馬選取之。又諭伊犁進馬,材具佶閑,足供御用,令正備貢各五匹,有私帶者,以違制論。道光二年,從那彥成奏,青海屬玉樹番族歲納貢馬,據丁口數,依二十壯丁貢馬一匹例,按數遞裁。涼州屬番族歲仍納馬一匹。初內外蒙部多貴戚,每征伐,爭先輸馬、駝,漢、唐以來所未有也。康熙初,察哈爾親王、郡王、貝勒等,聞三籓叛,各獻馬匹佐軍。道光九年,章佳胡圖克圖捐馬百匹,收其半。二十三年,察哈爾蒙旗捐馬千九百七十匹。咸豐初,哲布尊丹巴等捐馬千匹,喀爾喀土謝圖等二千匹,錫林果勒盟長等三千匹,帝以其多,卻之。嗣聞已在途中,令擇善地牧以待用。自是三音諾顏部等,以軍事輸馬、駝,旋捐馬二千一百,錫林果勒盟等千二百,或留或否。七年,各部落蒙古王等捐馬六千四百匹,詔納之。時粵、捻擾畿東,利於用騎也。同治間,黑龍江將軍德英於呼倫貝爾各城勸捐軍馬。光緒初,豐紳托克湍辦海防,時昭烏達盟郡王捐馬六百匹,因請踵行推廣勸諭,以助軍實云。

驛置肇自前漢,歷代因之。清沿明制,設驛馬,為額四萬三千三百有奇。各省驛制,定於康熙二年,凡齎奏官驛馬之數,各籓馬五匹,公、將軍、提督、督、撫三匹,總兵、巡鹽御史二匹,從兵部侍郎石麟請也。邊外之驛,定於九年,凡明詔特遣,及理籓院飭赴蒙古諸部宣諭公務,得乘邊外驛馬。三十五年,徵噶爾丹,設邊外五處驛站,用便車糧運輸。又從理籓院言,自張家口外設蒙古驛。其大略也。驛傳在僻地者,僅供本州縣所需,亦曰遞馬,額不過數匹。沖繁州縣,置驛或二或三,額馬至六七十匹。驛差大者,皇華使臣,朝貢蕃客,餘如大臣入覲、蒞官、視鹺、監稅皆是。若齎奏員役,呈奉表冊,其小者也。要者,如星馳飛遞,刻期立赴之屬。若閔勞恤死,允給郵傳,其散者也。驛政弊壞,張汧嘗極言之。越數誅求,橫索滋擾,蠹國病民,勢所必至。已定例諸驛額馬,每年十踣其三,循例買補。咸豐中,粵氛孔熾,湖、湘境為賊據,劫失驛騎,焚毀號舍,往往有之。各州縣或買馬填補,或賃馬應差,其有失驛未設,即雇夫代馬。甘肅舊設馬額六千餘,亦以軍興廢弛。光緒九年,軍務既平,驛遞漸簡,所留馬視前減三分二,而驛政亦無所妨。十一年,新疆南路設驛。是時,綜通國驛站歲費,約三百萬餘金。二十九年,劉坤一、張之洞條陳新法,謂驛站耗財,不如仿外人之郵政。郵政遞信速,驛政文報遲。弊由有驛州縣馬缺額,又復疲瘦,驛丁或倚為利藪,因致稽延。請設驛政局,推行郵政,俾驛鋪經費專取給郵資,即三百萬歲耗可以省出矣。時韙其言。已而驛馬漸裁,嗣是驛遂廢不用。

順治初,建常盈庫,凡車駕司朋椿站銀,武庫司馬值,太僕寺馬價皆儲之。康熙初,改常盈庫儲歸戶部。乾隆十六年,敕云南營馬除十踣其三按例應賠外,其逾額踣斃者免賠椿銀。二十七年,定給留圈馬乾,每匹視綠營稍優異。三十八年,又令雲南買補馬價,每匹減銀三兩。初馬乾歲費約四十四萬有奇。道光中,從載銓等言,裁八旗官拴馬半額,以節出之費補兵餉焉。

清初定現任官得養馬,餘悉禁之。尋許武進士、武舉、兵丁、捕役養馬。康熙元年,禁民人養馬。有私販馬匹,為人首告者,馬給首告之人。其主有官職,予重罰。平民荷校鞭責。十年,令民人仍得養馬。二十六年,定出廠馬、駝,或踐食田禾,或縱逸侵擾,兵鞭責,官罰俸有差。其兵丁強人代牧,乃勒索擾累者,兵發刑部,官降調。凡牧馬斃,則驗其皮,踣斃例須賠抵,有一九、一七之罰。應取駒千匹者,以百匹為一分,百匹者以十匹為一分。雍正十三年,定馬、駝出廠時,毛齒皆有冊,回日覈驗,如疲瘠十不及三,免議,否則兵鞭責,官罰俸有差。乾隆初,禁牧丁等盜馬私售,及與人乘,峻其科罰。十六年,嚴牧馬減克料草之罪。二十八年,官馬出青,每百匹準倒十匹,逾額勒其買補。嘉慶十一年,行圍木蘭,查獲私販馬匹諸犯,重懲之。因諭:「我朝講武時巡,扈從均給官馬。大臣祿入較優,給馬較少。官員兵丁,視差務之繁簡,定馬數之多寡,少者一、二匹,多至五匹,事竣原馬還官。如踣斃,呈驗耳尾,仍按價折交。收放時,命王大臣督察。乃官兵等竟私鬻官單,察哈爾官兵收馬利,其折銀易於買補。積弊日深,大妨馬政。自後設有賣單及折收者,一體科罰。私買之馬販,從嚴問擬。大臣等其妥議定章以聞。」凡營馬或走脫竊失,責令賠補,謂之賠椿,年遞減十之一,至十年悉免之。應敵傷損者免賠。騎至三年踣斃者亦免。其餘一年或二三年內踣斃,賠額視其省而異,以十金為最多。同治二年,定古北口盤獲私馬逾三十匹者送京,不及三十匹賞與兵丁,著為令。


\end{pinyinscope}