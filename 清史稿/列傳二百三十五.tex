\article{列傳二百三十五}

\begin{pinyinscope}
丁日昌卞寶第塗宗瀛黎培敬崧駿崧蕃邊寶泉於廕霖饒應祺惲祖翼

丁日昌,字禹生,廣東豐順人。以廩貢生治鄉團,數卻潮州寇。選瓊州府學訓導。錄功敘知縣,補江西萬安,善折獄。坐吉安不守,罷免。參曾國籓戎幕,復官。李鴻章治軍上海,檄主機器局,積勛至知府。江寧既下,除蘇松太道。鴻章倚以辦外交,事有鉤棘,徐起應付,率皆就範。調兩淮鹽運使,淮鹽故弊藪,至則禁私販,糾貪吏,鬯運道,歲入驟增。同治六年,擢布政使,授巡撫。江南戎燼後,庶政不緝,日昌集流亡,除豪猾,設月報詞訟冊,定錢漕科則,下其法各省;又以州縣為親民官,疏請設局編刻牧令諸書。八年,奉敕訓勉臣工,日昌條上六事,曰:舉賢才,汰虛冗,益廉俸,選書吏,輸漕粟,變武科,言合旨要。遭憂歸。

光緒元年,起授福建巡撫,兼督船政,辭,不允。既蒞事,會霪雨,城內水逾丈,躬散賑,口煦手拊,卵翼備至,全濟災民數十萬。眾感泣,僉曰:「活我者,丁中丞也!」時臺灣生番未靖,遂力疾渡臺,自北而南,所至扶服蟻伏。惟鳳山轄境,悉芒社及獅頭、龜紋諸社素梗化,遣兵討平之,為立善後章程,皆遵約束。中路水埔六社不諳樹藝,雇漢民代耕,謂之「租坰」。復令有司計口給銀米,教之耕作;廣設義學,教之識字。又罷臺屬漁戶稅。擬築鐵路,開礦產,移關稅釐榷造船械,臺民漸喁喁望治矣。還閩,移疾去,吏民啼泣遮道。

四年,疾稍間,被命赴福州,理烏石山教案。先是道光間,英人就山築室傳教,疆吏不能爭,以山在城外,飾詞入告。厥後占地愈廣,閩人忿,幾釀變。日昌撫閩,與力爭,議易以城外電局空地。未及行,遽解職,英人占如故。閩人不能忍,聚眾毀教堂,英使責難亟,至是命日昌往按。鉤稽舊案,獲教士侵地左證,與英領事往復詰辯,卒徙教堂城外,閩人鑱石刊績焉。逾歲,還里。明年,詔加總督銜,令駐南洋會辦海防,水師統歸節度。復命充兼理各國事務大臣,以疾辭,不許。八年,卒,恤如制。

日昌性孝友,撫吳日,母黃年九十矣,迎養署中,孺慕如兒時。兄寢疾,藥饍躬侍,兄止之,則引李勣焚須事為喻。好藏書,成持靜齋書目五卷,世比之範氏天一閣、黃氏百宋一廛云。子五人,惠康最著,好學,多泛覽,有丁徵君遺集。

卞寶第,字頌臣,江蘇儀徵人。咸豐元年舉人。入貲為刑部主事,累遷郎中、浙江道監察御史。軍興,官吏多避罪冒功,奏請檢視各省兵糧數目、攻守要害,及失陷收復時日功罪,以資稽覈;其有獲罪之員,藉事開復保升,宜嚴定限制。又言:「苗沛霖、王來鳳乍服乍叛,宜專意主剿。」上皆韙之。同治元年,遷禮科給事中,劾江北水師總統黃彬侵釐通賊,督辦軍務侍郎勝保貪蹇,提督成明擁兵同州畏葸無戰志,一時推為敢言。擢順天府府丞,遷府尹,捕巨盜王景漋等。五年,乞開缺養親,不允。出為河南布政使,擢福建巡撫。時粵寇初平,游勇土匪肆掠,疏請就地正法,報可。九年,再乞終養,許之。

光緒八年,起湖南巡撫。平江方雪璈,龍陽曹小湖,安鄉周萬益、張景來,皆盜魁也,陰結徒黨,號「哥老會」。寶第悉置之法。署湖廣總督。法人侵越南,詔偕巡撫彭祖賢治江防,築砲臺田家鎮南北岸各三座,繪具圖說上之。時議建樊口石閘。寶第以謂:「樊口內有梁子諸湖,袤延八百里,水皆無源,江入其中,瀦為巨浸。以民情論,重在堵江水之入,不在洩內水之出。以地勢論,江水驟失此渟瀦八百里地,則下游堤防必致沖決。請緩建石閘,而漸除樊口內窪田額賦。」得旨允行。

十一年,還湖南巡撫任。法人款成,寶第上言:「各國通商,因利乘便,須具臥薪嘗膽之志,為苞桑陰雨之謀。」因條上求才、裕餉、船政、器械四事。又言:「國家財用,歲出大宗,莫如兵勇並設。直省旗綠各營兵額七十七萬,每年薪糧銀一千數百萬兩。養兵既多,費餉尤巨。兵多則力弱,餉巨則國貧。粵逆初起金田,僅二千人。廣西額兵二萬三千,土兵一萬四千。乃以三萬七千之兵,不能擊二千之賊,廣西兵不可用,他省可推。其後發、捻、田、苗等匪,悉賴湘、淮營勇勘定,綠營戰績無聞。大亂甫夷,伏莽未盡,兵不得力,勇難驟撤,於是歲支勇糧一千餘萬。賦入有常,豈能堪此耗費?查綠營馬兵每月一兩九錢,戰兵一兩四錢,守兵九錢零。月餼無多,必謀別業,遂弛專操,軍情瞬變,調發遷延。臣擬請裁額並糧,以兩額挑養一兵。如額兵一萬,半為駐守,半赴巡防,互相邏戍,共習辛勤,常則計日操演,變則隨時援應。副參任營官,都守充哨弁,室家無累,而後紀律可嚴。此宜變通營制者一也。兵擬減額,原設將弁亦應核減。綠營將弁歲領廉俸雜項,職大者可抵百兵數十兵,小者亦抵十餘兵。自來積弊,隱匿空糧,攤扣月餉,左右役使,無非額兵。裁汰之議,自非將弁所樂。擬請先裁將弁以並營,營兵必多,乃漸裁兵,老弱事故缺出停補,俟空千名,即補精壯五百,綠營不足,簡撥營勇,作為練軍。不啟兵眾之疑,自無阻撓之慮。此宜逐漸辦理者又一也。目前兵尚未練,勇已議裁,若欲節餉,則裁勇不足資緩急,裁兵為有備而無患。」下部議行。十四年,擢閩浙總督,兼管福建船政。十八年,以疾解職,卒於家。

寶第有威重,不為小謹,騶從甚盛,所至誅鋤奸猾,扶槙良願,民尤感之。子緒昌,戶部七品小京官。

塗宗瀛,號朗軒,安徽六安人。以舉人銓江蘇知縣。曾國籓督兩江,檄主軍糈,累保授江寧知府。同治九年,擢蘇松太道。明年,遷湖南按察使。湘民故健訟,都察院歲所下獄輒逾百數。宗瀛為立條教,允首悔,懲誣告,並嚴定審理功過章程,弊乃稍革。晉布政使,仿硃子社倉法,建立長沙府倉。光緒三年,拜廣西巡撫。苗、瑤、惈儸獷悍梗化,檄所屬廣建學塾,刊孝經、小學諸書,使之誦習;又自撰歌詞以勸戒之。時晉、豫大旱,移撫河南,割取俸餘萬二千金助賑,招流亡,給籽種,老穉無依者,設廠收養,強有力者任工作。世與曾國荃賑晉並稱云。

七年,調湖南巡撫。撫標兵譁變,懲四人而事定。及擢總督,又有武漢教匪之亂,捕誅數十人,亦遂安堵。言官先後糾彈,事下彭玉麟,坐才力竭蹶,糸圭吏議。無何,御史陳啟泰劾宗瀛務封殖,仍下玉麟按覆,玉麟後白其誣。時左宗棠督江南,欲規復淮鹽、減川引,宗瀛以減川增淮,關川省數十萬鹽丁運夫生計,因抗疏力爭,言:「按年減運,則未運者將盡化為私。縱使湖北置兵徼循,而巫峽流急,鹽船下駛,瞬息百里,兵少力不能制,多恐滋生事端。且鄂餉無著,下拂輿情,上虧國帑。」辭愷切。未幾,稱疾乞休歸。

初,宗瀛從廷棟講學,為刊遺集,以理學稱。家居十餘載,以徐延旭獲譴,追坐舉主,下部察議。二十年,卒,年八十三。

黎培敬,字簡堂,湖南湘潭人。咸豐十年進士,選庶吉士,授編修。同治三年,出督貴州學政。阻寇弗能進,乃從劉岳昭借軍數十,竟達貴陽。時總督勞崇光、巡撫張亮基不相協,軍事益壞。培敬上書言狀,朝廷始獲聞邊事。黔苗俶擾,謳誦寂寥。培敬曰:「士氣不伸,人心所繇不靖也。」於是出入寇氛,按試州縣,雖危棘不緩期,貴州士民始復知文教。道黔西,晤道員岑毓英,與語,知其諳戎事,遂請以滇中軍屬之。培敬秩滿,以太常寺卿石贊清薦,命權布政使。其時寇患方亟,賊酋潘名桀守龍里,久不下。培敬曰:「今附郭百里,倉廩猶實。不因以為資,若轉藉寇,吾屬必為所虜矣!」因說提督出城取龍里,逾歲,克之。旋復貴定,名桀遁去,黔軍克捷自此始。詔嘉之,予實授。繇是東定都勻,北靖開、修,南平陳喬生,西除林自清,蒞黔數載,境內悉平。

光緒改元,擢巡撫。繼曾璧光後,益嚴吏治。以上疏請釋前總督賀長齡處分並予謚建祠,鐫秩罷歸。五年,起四川按察使。時丁寶楨督蜀,課吏嚴。培敬至,寶楨出郊迎,曰:「此吾貴州賢使君也!」培敬以巡撫降官,絕無慍意,孜孜治事。寶楨數薦其賢。六年,擢漕運總督。漕督雖閒職,然膴仕,培敬誓不以自污,公費所餘,以之修驛館,建兵房,增書院餐錢,興釋奠禮器,官煤、利濟諸局亦賡續告成,人無敢干以私。七年,授江蘇巡撫。未上,疾作,遂告歸。明年,卒,優詔賜恤,謚文肅,予貴陽、清江浦建祠。

崧駿,字鎮青,瓜爾佳氏,滿洲鑲藍旗人。咸豐八年舉人,由兵部筆帖式累遷郎中。同治六年,出知廣東高州府,以憂解。服除,起授山東沂州府,歷廣西按察使、直隸布政使、漕運總督。光緒十二年,巡撫江蘇,調浙江,所至興利除弊。以南糧改折色,吏民交困,並減旗營民糧、織造匠糧,令州縣糶價以供漕,弊乃革。十五年,浙患水祲,奏請免漕,發帑賑之,而於京、協諸餉仍從容籌解,復集貲購米實倉儲。杭、嘉、湖三府暨蘇、松、常、太諸水源出於潛天目山附近,苕溪南北二湖為分洩地,歲久淤塞,用工賑法,招集流民疏濬之。其杭、嘉、湖、紹諸塘岸堰徬,靡不次第修治,民賴其利。十七年,卒於官。

崧駿以清廉自矢,於國計民生服念不忘。撫江、浙績尤著,民請祠之,得旨俞允。子昆敬,戶部郎中。

崧蕃,字錫侯,崧駿弟也。咸豐五年舉人,初入貲為吏部郎中。光緒五年,京察一等,簡四川鹽茶道,屢署按察使,保薦卓異。十一年,授湖南按察使,遷四川布政使。十七年,擢貴州巡撫。廣西寇陸亞漋煽亂西林,與貴州接壤,崧蕃遣將扼冊亨要隘,邊患遂平。調雲南巡撫,擢雲貴總督。檢視防營缺額積弊,劾副將雷家春,並自請議處,革職留任。

二十六年,奏請陛見,值拳匪肇亂,命留京會辦城防事。旋扈駕至太原,飭還本任。行次,調陜甘總督。於城南建立大學堂,分兩齋,東齋考文,西齋講武。而修濬寧夏七星渠,尤為民所利賴。寧郡堤工,創自乾隆時,魚鹽之利甲通省,後漸湮廢。中衛縣令王樹棻素講求水利,崧蕃檄令勘工,自七星渠上接白馬通灘,流濬通深百八十餘里,灌田六萬餘畝,磽確變為沃壤,逃亡復業。又以渠水分自黃河,勢洶湧,春夏山水驟發,與黃流渾合,泥沙雜下,旋濬旋塞。乃仿古人暗洞激水法,凡傍山之渠,架油松成洞,覆以石板,山水流石上,而渠水潛行洞中。又度地勢築高堤,導山水使入黃河,並於渠口築進水、退水兩壩,使黃流曲折入渠,不致沖漫。工竣,數經暴水,卒不圮。設農務局,招墾荒地,如平羅、渭源諸縣,先後報墾數百千畝。舊有機器局,漸次擴張。凡興作實事求是,不惟其名。三十一年,調閩浙總督,未上,以疾卒,追贈太子少保。子外務部主事豫敬,以員外郎補用。

邊寶泉,字潤民,漢軍鑲紅旗人。同治二年進士,授編修。十一年,補浙江道監察御史。大學士李鴻章總督直隸,奏清苑麥秀兩歧。寶泉疏論之曰:「祥瑞之說,盛世不言。臣來自田間,麥有兩歧,常所親見。地氣偏厚,偶然致此,何足為異?漢章之時,以嘉穀芝草,改元章和,何敞猶據經義面折宋由、袁安。至馬端臨文獻通考,乃舉歷代祥瑞,統曰『物異』。夫祥且為異,今以無異之物而謂之祥,可乎?上年畿輔水災甚鉅,迄今沒水田廬猶未盡出;永定河甫經蕆工,北岸又潰;邊軍未撤,民困未蘇。鴻章身膺重寄,威望素隆,當效何敞之公忠,懲宋由、袁安之導媚。皇上御極之初,庶吉士嚴辰散館考試,曲意頌揚,奉旨嚴飭。今鴻章為督撫大吏,非草茅新進可比,乃亦務為粉飾,於治道人心關系尤鉅。應請降旨訓飭。」是時鴻章又以永定河合龍,奏獎工員勞勩,奏上而河復決,寶泉又疏請撤銷保案。鴻章新建大功,寶泉再疏彈之,鴻章亦不以為忤,天下兩賢之。遷戶科給事中。

先是都御史胡家玉疏陳丁漕積弊,語侵巡撫劉坤一,坤一覆奏家玉逋賦未完,且私書囑託公事。寶泉復劾:「坤一藉詞箝制地方長吏,此端一開,啟天下輕視朝廷之漸。」疏入,坤一下部議處。

光緒三年,出為陜西督糧道,再遷布政使。九年,擢陜西巡撫。尚書閻敬銘議陜西收放糧米改徵折色,寶泉持不可,以謂:「穀數有定,今改折色,所收必有減於昔而民始樂從,所放必加多於前而兵乃足用。入不敷出,一時強為彌補,後將何所取償?昔歲大饑,終賴道倉儲粟,多所全活。今並此而去之,恐饑饉水存臻,益無可恃。」上韙其議。十二年,調河南巡撫,移疾歸。

二十年,即家起閩浙總督。閩鹽逋課積八十餘萬,前任奏報,率皆飛灑他項為挹注。寶泉至,盡發其覆,乃有停釐補課之奏。船政舊設大臣,後以總督兼之。寶泉特疏請復故制,且條上造船、購料、延教師、籌經費四事,而不私其權,人嘉其廉讓。二十四年,卒於官,贈太子少保。

於廕霖,字次棠,吉林伯都訥人。咸豐九年進士,改庶吉士,授編修。從大學士倭仁問學。光緒初,俄羅斯議還伊犁,廕霖疏劾崇厚擅許天山界地數百里。及崇厚被逮,有為之游說者,復嚴疏劾之,且劾樞臣畏葸欺罔。六年,授贊善,累遷中允。八年,出為湖北荊宜施道。是秋淫雨,漢水溢,檄所屬開倉賑濟。又濬紫貝淵上游,改閘為壩,疏支流,洩積潦,水患始息。新荊州書院,設經義、治事兩齋,生徒雲集,講舍至不能容。擒斬盜魁李人奴等,餘黨屏息。宜昌民教構訟,法領事袒教民,挾兵艦至,廕霖不為動,後卒無事。英商漏宜昌關稅,既覺,乃納賕請免,不許;請補稅,許之。英商嘆其廉。

十一年,擢廣東按察使。廣東素多盜,至白晝劫掠衢市。廕霖言於總督張之洞,奏請就地正法,報可。順德廩生簡明亮有學行,緣事系獄,察其枉,立出之。十二年,遷雲南布政使,丁母憂。服闋,改授臺灣布政使,未行,會弟編修鍾霖以前在籍與廕霖同辦賑務,為奸商湯連魁誣控獲譴,廕霖具疏辨。詔遣大臣即訊,頗得連魁行賄狀,然廕霖猶坐是落職,廢居京師。

二十年,日本戰事起,命往奉天襄依克唐阿軍。請募兵二萬自效,詔許募萬人,分四軍,與民團相應援。明年,和議成,總督張之洞、山東巡撫李秉衡交章論薦,詔賞三品頂戴。署安徽布政使,至則清釐田賦,杜絕欺隱,增墾田萬八千餘畝,撙節庫儲至二百萬金。二十三年,德人索膠州灣,又脅朝廷罷李秉衡,廕霖奮然曰:「是尚可為國乎!」上疏極論王大臣不職,因附陳修省五事,不報。二十四年,擢湖北巡撫。之洞為總督,頗主泰西新法,廕霖齗齗爭議,以為:「救時之計,在正人心、辨學術,若用夷變夏,恐異日之憂愈大。」之洞意迂之,然仗其清正,使治吏事。湖北財賦倚釐金,廕霖精心綜核,以舉劾為激揚,歲入驟增數十萬。

二十七年,調撫河南。時兩宮西狩,德、法兵日謀南下,而河北莠民往往仇殺教民,廕霖檄彰衛懷道馮光元捕誅首惡數人。德、法兵至順德,聞教案已結,乃還。二十八年,調湖北。會詔裁缺,改廣西。廷議廕霖不善外交,復降旨開缺,假居南陽。三十年,卒。

廕霖晚歲益潛心儒先性理書,雖已貴,服食不改儒素,硃子書不離案側,時皆稱之。

饒應祺,字子維,湖北恩施人。幼穎悟好學,試作渾天儀,旋轉合度。年十二,入邑庠,益究心經世學。咸豐九年,粵寇石達開自湘、鄂犯蜀,道恩施,應祺率鄉團助城守。由候選訓導議敘國子監學正。同治元年,舉於鄉,揀選知縣,援例為主事,分刑部。父卒,廬墓側。服闋,陜甘總督左宗棠檄參軍幕。以克金積堡、巴燕戎格諸處功,擢知府。光緒三年,署同州知府。時秦、晉亢旱,赤地千里,饑民洶洶,遮道不得前。應祺諭之曰:「此來賑汝饑耳!譁變者殺無赦。」乃捐俸錢為官紳倡,弛重糶禁,旬日得糧七十餘萬石,又截留他省糧運以助不繼。復為招流亡,定墾章,給牛種,蠲雜稅。歲稍轉,教民興水利,勤樹植,設義倉,行保甲。又規復豐登書院,創修府志,文化蔚興,士民為立生祠。

左宗棠疏薦應祺守絕一塵,才堪肆應,請以道府簡補。十年,授甘州知府。陜西自軍興,兵差旁午,設里局董之,凡四十一州縣大困。上命巡撫邊寶泉赴陜查辦,疏留應祺理其事。應祺量道路沖僻定收支之數,分別兵流,掃浮汰冗,歲省數十萬兩。是年冬,抵甘州任,賑饑勸學,設織紡局、孤嫠所,革徵草之弊,復七斤一束舊章。十一年,遷蘭州道。瀕行,士民攀轅留行,多泣下者。旋署按察使。嚴搶嫠為婚之禁,擒督署差弁及鄉人楊營弁置之法。手訂清理庶獄章程,以詔群吏,視其功過而黜陟之。

十五年,調新疆喀什噶爾道,改鎮迪道,兼按察使銜。十七年,署新疆布政使;十九年,實授。新疆兵燹後,民物凋弊,地多荒棄。伊犁故腴壤,回屯舊八千戶,四不存一。應祺建議伊犁將軍給新裁錫伯、索倫兵牛糧,使之屯種;給新裁察哈爾、厄魯特兵羊馬,使牧放;並招致關內災民,按丁授地,實行寓兵於農之法。羅布淖爾者,舊史所稱星宿海也,漢為且末、尉犁、婼羌諸國地,東西廣千六百餘里,南北袤千里或數百里,自陽關道梗,其地遂成甌脫。應祺建議巡撫築蒲昌城,設英格可力善後局、卡克里克屯防局,招徠漢回客纏,通道置驛,建堡濬渠,教以耕織。又請改防軍為標營,定額徵糧石每年折色之法,畫一錢法。

俄領事原議駐吐魯番,後求移駐省垣,將軍、巡撫難之。應祺謂:「此不必爭。我所應爭者,洋商稅則須與華商一律,同時議定。新省毗連英、俄,陸路進口地不一,北道伊犁,南道喀什,應設關,各以本道為監督;塔城、烏什、葉爾羌應設分卡,歸各道兼轄。」均如議行。南路初設領署,應祺貽書伊塔、喀什兩道曰:「交鄰之道,莫先於自治。我之用人行政,使彼族聞而敬服,則遇事不至以非禮相要,此為折沖禦侮第一要義。飲食往還,平時貴以情誼相聯。至華洋訴訟,必先得華民是非曲直實情而後與之爭,庶可關其口而奪之氣。一詞稍偽,彼將執以相例,而全案皆虛矣。情以籥之,理以盾之,又其次也。」新疆向受協餉,每苦款絀,應祺開源節流,數年庫儲逾百萬。

二十一年,河、湟回煽亂,蔓延甘、涼諸郡,其別股萬餘謀西竄。上命應祺署新疆巡撫,應祺檄提督牛允誠防安西、玉門諸處,拒寇境外。回酋劉四伏果竄玉門之昌馬,遇允誠軍,戰數不利,盡棄輜重,逾雪山西逸。應祺遣參將李金良要之紅柳峽,生擒劉四伏,降其眾八千,安置於羅布淖爾,設軍鎮撫。同時庫車回謀起事,寧遠回亦以爭新教相仇殺,洶洶思變。應祺皆先期撲滅,故四伏無內應,卒就殲。上嘉其功,實授巡撫。

應祺以新疆僻處國西北隅,密邇強俄,士卒眾而器械窳,生齒繁而司牧少,不足以固吾圉,乃購快槍萬枝於德國,而設機器廠制造子彈,奏設左右翼馬隊為游擊師。又開辦於闐、塔城金礦,墾荒田,開渠井,廣興實業,凡有利於民生者,皆次第舉。自是地利盡闢,兵備有資,較初建行省時迥異矣。

拳匪亂起,俄兵自薩馬進逼邊卡,應祺會總督魏光燾、伊犁將軍長庚仿東南各省,與各領事結互相保護之約,俄兵乃退。議成,應祺應詔陳言,略謂:「古今中外治法務在求實。舊章非無可守,守之不以實,成法亦具文;新法非不可行,行之不以實,良法亦虛飾。心之實不實,宜於行事之實不實驗之。」逾年,詔設武備學堂,編立常備、續備、巡警各軍。應祺主操練用新法,器械用新式;人惟求舊,必樸實勤奮久於戰陣者,方可入選。上疏極論之,並謂:「中國習洋操三十年,一敗於日本,再敗於聯軍,為務虛名而貽實禍之證。」所言皆切中時弊。

而尤齗齗於界約,不少遷就。帕米爾高原,國境也,有高宗御制平寇碑,立於蘇滿。英、俄交覷其地,而俄人先竊據之。應祺官布政使時,商之巡撫,以理退俄兵,遣軍戍焉。俄人悔失計,日聒於總署,要我撤兵。應祺持不可,謂:「我自守門戶,其理直。我退則英必至,英來則俄又必爭,是息事而益多事也。」後竟如應祺言。坎人求租種莎車屬喇斯庫穆荒地,應祺謂:「坎本我屬,宜示懷柔。其在玉河卡倫外者,可允其租墾,納賦比於華人;其在玉河東北屬邊內者,宜卻之,防後患。」總署與英使議界約,以坎部讓與印度,而塔墩巴什帕米爾及喇斯庫穆全境皆讓與中國。應祺抗言:「喇本我地,不得謂之讓。」而俄人轉謂中國以喇地讓與英人,利益宜均,以兵威相脅。應祺飭屬嚴備邊,而以議租原委及議約界限詳諭之,俄人始無辭。

應祺官西疆久,闢地安民,屢請建官設治以資鎮撫。二十八年,復疏言:「新疆自光緒四年改建行省,土地日闢,戶口日繁,原設州縣,轄境遼遠,非增設府,不足治理。西四城喀什噶爾道:疏勒州為極邊重要,請升為府;距府百八十里之排素巴特地屬唐伽師城,改為伽師縣;莎車地廣而腴,英商麕集,請升為府:府南為澤勒普善河,增設澤普縣;府西南色勒庫爾為古蒲犁國,實坎巨提出入要路,又與英、俄接壤,請設蒲犁分防通判;距於闐縣四百里之洛浦莊,增設洛浦縣;嗎喇巴什為古巴爾楚地,改為巴楚州。東四城阿克蘇道:溫宿州為南疆要沖,請升為府;舊城巡檢升為溫宿縣;距縣四百八十里之柯爾坪,增設柯坪縣丞;焉耆府南六百三十里布古爾分防巡檢為古之輪臺,請分設輪臺縣;卡克里克縣丞,其地為古婼羌國,改設婼羌縣;庫車土地廣沃,請改為州;州南沙爾雅增設沙雅縣。北路阜康縣之濟木薩縣丞,富庶逾於縣,舊驛名孚遠,升為孚遠縣;距吐魯番二百四十里之闢展巡檢地為古鄯善國,升為鄯善縣;昌吉縣所屬之呼圖壁巡檢向收錢糧,請改為縣丞。計升設府三,改直隸州二,增通判一、縣九、縣丞二。」又奏增設鄉試中額二名,會試中額一名,暨各府學官學額,先後皆議行。是年,調安徽巡撫,行抵哈密,病卒,賜恤如例。

惲祖翼,字叔謀,江蘇湖陽人。同治三年舉人。以知縣累至道員,再攝武昌道。教匪王覺一約期起事,祖翼時筦營務,乘夜率親兵掩捕之。總督塗宗瀛疏保祖翼有濟變才,光緒十五年,授督糧道。調漢黃德道,兼江漢關監督。以襄河漲發易壞舟,創設襄樊報水電,樹牌鳴鉦,各船備御,水至遂無患。晉按察使,擢浙江布政使。祖翼以州縣徵糧照舊折價,近年錢貴銀賤,民力不支,乃重定銀價,設櫃徵收,不得假手書役,人稱其惠。尤盡心水利,於嘉興開泖河,疏港建閘,以資蓄洩。於杭州浚上塘河,臨平、喬司等處農田三十餘萬畝皆獲灌溉之利。上虞南塘舊以土築,水至輒決。採眾議,改建石塘千一百丈,始免水患。

二十六年,北京拳亂報至,祖翼獨起抑阻。匪陷江山、常山,衢民復毀教戕官,英國欲以兵艦赴浙。祖翼亟遴員馳往鎮撫,獲真犯抵償,潛消兵釁。會兩江、湖廣總督與各國訂約保護南疆,電詢浙省。巡撫劉樹棠方臥病,祖翼即逕電以浙省附約,人心以安。

旋擢巡撫。以浙省防練各營積弊,疏請整飭,略言:「浙省水陸防練各營數逾制兵,陸續添募,餉實不敷。而統領各營哨,不顧操練緝捕為何事,汲汲焉唯浮冒剋減,食弊自肥。術愈出而愈奇,勇日雜而日弱,盜日防而日多。今將蕩滌宿垢而作新之。立法自上,責在督撫。臣任事即通飭各營,與之更始。以後如有貪劣將弁,仍敢浮冒剋減,決不姑容。擬先勵其廉恥,而兼課其材武。一面飭州縣查保甲,辦團練,以輔制兵之不逮;一面遴委廉幹道府,酌帶哨勇,分往浙東西,抽點名糧,認真校閱。遇有大股盜匪,督率營縣搜拿,務絕根株。總期合散為聚,化惰為勤,堪備一日之緩急。雖然,營衛小疾,疏解足矣,受病既深,斷非猛劑不治。天下之病,無一不根於利。統領營哨,聞見已慣,謂夫督撫所能操以繩其下者,撤之而已,參之而已。撤之則又顧而之他,參之則已飽颺而去;且未幾而又夤緣開復矣,未幾而以將才調用矣。惟督以峻法,務去泰甚,庶有以振暮氣而戢貪風。或震於各國一時之強,幾謂全恃火器,不知其本原仍在臨財廉,與士卒同甘苦。否則未戰先潰,火器徒以齎寇,直自伐耳。可否請旨飭下兵、刑各部,採臣治亂用重之議,嗣遇將弁贓證確鑿者,分別輕重,嚴定參革、追繳、倍罰、斬絞之例,庶軍心一振,於時局或有裨益。」疏入,詔飭各省著為令。未幾,丁母憂歸。卒,恤如例。浙人請立祠祀之。

論曰:疆吏當承平時,民生吏治,要在因地制宜而已。日昌、寶第皆以尚嚴著績效。宗瀛、廕霖飾之以儒術,亦後先稱治。培敬有為有守,崧駿兄弟所至盡職,寶泉勵清操,祖翼能濟變,並有可稱。至應祺官關隴、新疆垂四十年,邊地初闢,治績爛然,實心實政,其勞亦不可沒雲。


\end{pinyinscope}