\article{列傳三百十六}

\begin{pinyinscope}
屬國四

廓爾喀浩罕布魯特哈薩克安集延瑪爾噶朗那木干塔什乾巴達克山博羅爾阿富汗坎巨提

廓爾喀,在衛藏西南,與巴勒布各部相鄰。巴勒布三汗:曰陽布,曰葉楞,曰廓庫木;後皆為廓爾喀酋博納喇赤並吞,及小部二十三。其國境東西二千里,南北約五百里。東與哲孟雄、宗木、布魯克巴接壤,西與作木朗接壤,南距南甲噶爾,北連後藏邊境。傳至孫拉特納巴都爾,年幼嗣位,其叔巴都爾薩野用事,操國大權。

乾隆五十三年,廓爾喀人至藏貿易,以爭新鑄銀錢,與唐古忒開釁構兵,進侵藏界。帝命四川總督鄂輝、將軍成德往查,以巴忠熟悉藏情,令為會辦。巴忠遷就議和,稱內附,帝錫封廓爾喀王爵。廓爾喀私責後藏班禪喇嘛賠償銀兩,巴忠不以聞,既而後藏不能償,班禪復與弟紅帽喇嘛沙瑪爾巴不協,沙瑪爾巴因導廓爾喀內侵。五十六年,廓爾喀遂以唐古忒兵欠款、班禪負約為辭,遣兵圍聶拉木,唐古忒兵聞風潰,進至達木,番兵亦敗退。八月,廓爾喀圍札什倫布,將軍成德赴藏援剿,帝復命四川總督鄂輝督後隊赴援,鄂輝復調金川兵二千、雲南兵二千助討。九月,廓爾喀六七百人攻宗喀,陳謨、潘占魁等率唐古忒兵固守,擊卻之,斬首四十六,賊退濟嚨。帝始議大舉往征。

十月,召兩廣總督福康安入京,授以方略,命為將軍,督參贊海蘭察等由青海赴藏,總領大軍討廓爾喀。十二月,成德次聶拉木四十里,戰拍甲嶺,敗之。明年正月,攻克聶拉木東官寨,斬其酋呢瑪叭噶嘶及踏巴等。二月,以地雷破西北碉寨,獲咱瑪達阿爾曾薩野,巨酋瑪木薩野之侄也。聶拉木既平,進軍濟嚨。

三月,福康安抵後藏,詔晉為大將軍,各軍咸受節度。廓爾喀築寨據險死守。四月,福康安偕海蘭察由絨轄、聶拉木進,決議先剿擦木、濟嚨。擦木地最險,兩山夾峙,中亙山梁。五月六日,乘夜雨,分五隊,海蘭察等居中,哲森保等由東西山趨賊寨,墨爾根保等繞出賊背。黎明,攻擦木山梁兩石碉,克之,擒斬二百餘人。進至瑪噶爾轄爾甲,濟嚨援賊三百據山力拒,海蘭察趨進,馬中槍,揮軍奮擊,盡殲之。濟嚨賊聞官軍將至,建大寨山岡外,扼險築三大碉相犄角。福康安檄巴彥泰、巴彥寨、薩寧阿、長春攻西北臨河大碉,桑吉斯塔爾、克色保、籌保、巴哈、張占魁攻東北石上大碉,哲森保、墨爾根保攻東南山梁上大碉,蒙興保、綽爾渾等攻山下喇嘛寺,阿滿泰、額爾登保等攻大寨,以惠齡為策應之軍,海蘭察率騎兵張兩翼截擊逸賊。六月初六日,哲森保等攻克山梁大碉,蒙興保等克喇嘛寺,復會攻臨河及石上兩大碉,皆克之。設砲石上,戰一晝夜,破其東北隅,遂拔濟嚨,斬級六百餘,擒二百,獲賊目七。

當福康安之攻濟嚨也,先遣成德、岱森保率兵三千出聶拉木南行,牽綴賊勢,壁上木薩橋。賊築三卡於德親鼎山下,建木柵於下木薩橋,以拒官軍,岱森保悉攻破之。於是自擦木至濟嚨邊界盡復。濟嚨西南皆高山峻嶺,路險惡。距濟嚨八十里有熱索橋,其大河自東來注,渡橋即廓爾喀界也。賊屯北岸三四里外索喇拉山,設石卡一,南岸臨河,設石卡二。官軍進破索喇拉山卡,追至熱索橋。逸賊甫上橋,南岸守橋賊見追兵至,倉卒撤橋,逸賊皆落河死。官軍隔河施槍,河闊不能及,乃退還。密遣阿滿泰、哲森保、墨爾根保、翁果爾海等率土兵東出瓘綠大山,繞至上游,伐木編筏以濟。時賊與官軍隔河相持,不虞間道軍驟至,倉皇抵抗,不能敵,潰而奔,遂夷二石卡。

六月十七日,福康安、海蘭察、惠齡等渡熱索橋,進密里頂大山,山重疊無路徑,乃令烏什哈達、張芝元開路以進。明日,抵旺噶爾,山勢險峻,瑪爾臧大河傍山南注。我軍循河東,路逼仄,不能駐足,士卒皆露宿巖下,深入賊境百七十里,不見一賊。尋偵知旺噶爾西南協布魯克瑪賊樹木城,外環石壁,城西里許夾河築卡,城東三十里環克堆築寨,以相犄角。二十日,官軍由旺堆伐樹建橋,城賊居高施彈,橋不能成,我軍以砲轟其城,賊隨缺隨補,終不得渡。二十二日,福康安、海蘭察由間道越伯爾噶臧興三大山,攻克堆,賊阻河以拒。日暮大雨,我兵佯退伏叢林中,夜深偷渡,毀賊壘五,斬級三百餘,徑趨協布魯克瑪,與惠齡等前後夾擊,賊驚潰,木城石卡俱下。

協布魯克瑪既克,福康安分道而前。一由噶多趨東覺為正道,一由噶多東越山趨雅爾賽拉、博爾東拉為間道。海蘭察督桑吉斯塔爾、阿滿泰、珠爾杭阿等出間道,福康安出正道。命臺斐英阿等與賊相持於作木古拉巴載山梁,躬率額爾登保等潛趨噶多普。七月初六日晨,渡河破其碉卡,進毀寨十一、木城五,殛賊目蘇必達柰新及巴撒喀爾,斬徑四百。海蘭察亦破賊博爾東拉前山,毀木城三、石卡七,追至瑪拉,遇伏,擊破之。東覺餘賊俱盡,兩軍復合。進至雍鴉,賊據噶勒拉山梁,道路崎嶇,士卒履皆穿,跣足行石子上,多刺傷,又為螞蝗嚙,兩足腫爛。其地多陰雨,惟辰巳二時稍見日,屆午則雲霧四合,大雨如注,山顛氣寒凜,夜則成冰雪,於是頓兵休息。當是時,成德軍亦克札木,過鐵索橋,進至多洛卡,破賊隴岡,覆利底寨。

八月,福康安分軍為三,過雍鴉趨噶勒拉。廓爾喀境皆山,東西對峙,中貫大河。自過雍鴉,山勢皆南北向,噶勒拉、堆補木、甲爾古拉、集木集諸大山層層環抱,橫河阻之,我軍須渡河仰攻。初二日,破石卡,逼噶勒拉山顛木城。侍衛墨爾根保、圖爾岱,參將張占魁攀堞以登,中槍而殞,士益奮,拋火彈入焚其帳房,自辰至未,克木城石卡各二,殲賊三百餘,斃其目五,落崖死者無算。乘勝追數十里,抵堆補木山口之象巴宗,賊蜂擁出拒,袁國璜等陷入陣,斃賊百餘。復檄珠爾杭阿等攻集木集,阿滿泰、額爾登保等渡河撲甲爾古拉。賊扼險列木柵長數里阻官軍,阿滿泰與賊爭橋,中槍落水,額爾登保等奮呼而進,遂渡河,斬賊目三,斃賊百餘。大軍競進集木集,賊眾分三道來援,殊死鬥。福康安躬督戰,英貴殞於陣。臺斐英阿、張芝元、德楞泰往來奮擊,射死紅衣賊目二,賊始敗走。

是役也,連戰兩日一夜,克大山二、大木城四、石卡十一,斬賊目十三,進抵帕朗古,深入賊境七百餘里,斃六百餘人,廓爾喀酋震懼乞降。初,福康安破東覺,賊酋乞降,福康安不許,檄令拉特納巴都爾、巴都爾薩野躬親至軍,並獻禍首及所掠財物,賊不應。至是拉特納巴都爾、巴都爾薩野遣大頭人稟請交送札什倫布什物,繳出西藏所立條約,並獻禍首沙瑪爾巴之骨。

福康安、海蘭察、惠齡合疏入告曰:「竊臣等秉承廟算,統率勁兵,自察木進剿以來,連戰克捷,邊界肅清,遂奪熱索橋,深入賊境。協布魯、東覺、博爾東拉、噶勒拉、堆補木、帕朗古諸處皆系峭壁懸崖,大河急溜,我兵繞山涉水,間道出奇,賊匪碉卡木城悉行攻克,所向無前,賊匪敗衄奔逃。大兵進至雍鴉,送出上年被裹兵丁王剛諸人,具稟乞降。旋遣賊目噶布黨普都爾幫哩等迎赴軍前,悉將上年被裹之噶布倫丹津班珠爾及兵丁盧獻麟等全行送出,稟陳沙瑪爾巴唆使情形,悔罪哀祈。臣等嚴加駁飭,復進兵至帕朗古,移營進逼,賊匪益加震恐。即將沙瑪爾巴眷屬、徒弟、什物等項,及搶掠札什倫布銀兩物件,皆已遵檄呈交,並繳出私立合同二張,不敢復提西藏給銀之事。再三懇求聖主,逾格施恩,赦其已往,以全闔部番民之命。茲於八月初八日,遣辦事大頭目噶箕第烏達特塔巴、蘇巴巴爾底曼喇納甲、察布拉咱音達薩野、喀爾達爾巴拉巴達爾四名,恭齎表文進京,並虔備樂工、馴象、番馬、孔雀、甲噶爾所制番轎、珠佩、珊瑚串、金銀絲緞、金花緞、氈呢、象牙、犀角、孔雀尾、槍刀、藥材共二十九種,隨表呈進。另稟懇臣代奏,當即譯閱表文,詞意極為恭順懇至。並據第烏達特塔巴等伏地哀懇,叩頭乞命,至於泣下。跪稱:『廓爾喀部長拉特納巴都爾、部長之叔巴都爾薩野,本系邊外小番,曾歸王化,渥受大皇帝天恩,特加封爵,錫賚多珍,高厚恩慈,至今頂感。乃拉特納巴都爾年幼無知,巴都爾薩野罔識天朝法度,因沙瑪爾巴從中簸弄,唆使廓爾喀與唐古忒藉端滋事。拉特納巴都爾等輕聽其言,侵犯後藏,仰煩大皇帝天兵遠討,誅戮頭目人眾三四千人,攻據地方七八百里,天威震疊,廓爾喀膽落心驚。拉特納巴都爾及巴都爾薩野自知罪在不赦,惶懼尤甚。從前侵犯藏界之事,雖系被人煽惑,而孽實自作,萬不敢絲毫置辯,諉咎於人。惟有仰懇轉奏大皇帝大沛恩施,開一線之路。如蒙允準,免其誅滅,廓爾喀闔部地土、人民皆出大皇帝所賜,銜感宏施,曷其有極!前立合同混行開寫各條,萬不敢復提一字。廓爾喀永為天朝屬下,每屆五年朝貢之期,即差辦事噶箕一名,仰覲天顏,子子孫孫,恪遵約束。懇求大將軍據情轉奏』等語。臣等隨諭:『拉特納巴都爾、巴都爾薩野自速誅鋤,侵擾藏地,天兵至此,本應滅爾部落,焦類無遺。今拉特納巴都爾等敬凜大皇帝天威,萬分悔懼,屢懇投降,情詞恭順,本大將軍不敢壅於上聞,當即據實具奏。大皇帝如天好生,或可仰蒙鑒察,宥罪施恩。倘荷聖慈允準,從此爾部落惟當遵奉天朝法度,不得復滋事端,方可永受大皇帝天恩,保守境土。此次天兵威力,爾已深知,若稍抗違,即是自取滅亡,後悔無及。』其頭目跪聆之下,戰慄叩頭,感懼之誠,形於辭色。臣等伏思廓爾喀恃其險遠,構釁稱兵。上年藏事,遷就議和,兵威未加,罔所祇懼,是以投誠甫及兩年,復行反覆。此次興師問罪,仰承聖主指授機宜,士卒爭先用命,越險摧堅,兵到之處,屢戰屢勝,大半殲擒。廓爾喀在西番各部素稱強悍,今見天朝兵力精強,所向無敵,全部震讋,屢遣大頭人來營乞降,察看情辭,實出誠悃。伏查前承明旨,諭令臣等『酌量情形,倘軍臨賊境,賊匪心懷心習伏,悔罪乞哀,或可申明約束,俯允所請,納款班師』。仰見我皇上廟算精詳,幾先指示,義正仁育,威德覃敷,臣等實深欽服。今廓爾喀業已悔罪投誠,遣大頭人恭進表文,請於象馬方物之外,虔備樂工,使隸於太常,附各國樂舞之末,並懇定立貢期,遣使五年朝貢一次。詳察賊情,實屬傾心向化,不敢再滋事端,衛藏全境似可永底敉寧,相安無事矣。」

疏入,帝允受降,諭福康安等籌善後撤兵,仍以所獲熱索橋以西協布魯、雍鴉、東覺、堆補木、帕朗古各地還廓爾喀;熱索橋以內濟嚨、聶拉木、宗喀前屬藏地,為廓爾喀所據者,仍歸後藏。沿邊設立鄂博,如有偷越,即行正法。遇有遣使表貢,先行稟明,邊吏允許,始準進口。八月,廓爾喀酋遣蘇必達巴依喇巴忻喀瓦斯並親信瑪泌達拉喀瓦斯至營,呈水牛、豬、羊各百頭、米二百石、果品糖食百筐、酒百簍犒師。福康安諭留牛羊各十頭、米十石,以答其誠敬之意,餘皆發還。復賞錦緞各四疋,廓爾喀益感服,受約束。二十一日,班師。十月初三日,福康安還後藏。

五十八年正月,廓爾喀貢使噶箕第烏達特塔巴等齎貢物至京師,帝賜宴,命與朝鮮、暹羅各使同預朝賀,封拉特納巴都爾為廓爾喀王。自是五年一貢,聽命惟謹。

其後英吉利據印度,時時被侵略,迫訂西古利條約,廓爾喀始將西界克美心互山地及開利川河流域割於英。廓爾喀既為英逼,勤修國政,力保自主之權,英雖覬覦之,無如何也。光緒末,猶入貢中國云。

浩罕,古大宛國地,一名敖罕,又曰霍罕,蔥嶺以西回國也。東與東布魯特接,南與西布魯特接,西與布哈爾國接。有四城,俱當平陸。一曰安集延,東南至喀什噶爾五百里。其人長於心計,好賈,遠游新疆南北各城,處處有之,故西域盛稱安集延,遂為浩罕種人之名。從安集延西百有八十里為瑪爾噶朗城,又西八十里為那木干城,又西八十里為浩罕城。四城皆濱近納林河,惟那木干在河北。南北山泉支流會合,襟帶諸城之間,土膏沃饒,人民殷庶。其人奉回教,習帕爾西語,亦布魯特種也。其頭目冠高頂皮帽,衣錦衣。民人戴白氈帽,黃褐。諸城皆有伯克,而浩罕城伯克額爾德尼為之長,眾聽命焉。

乾隆二十四年,將軍兆惠追捕霍集占兄弟,遣侍衛達克塔納等撫布魯特諸部。至其境,額爾德尼迎之入城,日饋羊酒瓜果,詢中國疆域形勢,畏慕,奉表請內附。並上將軍書,稱為「至威至勇如達賚札木西特之將軍」。旋遣頭目托克托瑪哈穆等貢馬京師。二十五年,

遣侍衛索諾穆策凌齎敕往諭,額爾德尼率諸伯克郊迎成禮。是為浩罕屬中國之始。浩罕風俗與天山南路諸回部略同,而鷙勇過之,有「百回兵不如一安集延」之語。初,大軍追霍集占急,霍集占遣使欲投浩罕,不報。尋,霍集占兄弟為巴達克山所殲,波羅尼都次子薩木薩克逃入浩罕,浩罕藉其和卓木之名,居為奇貨。和卓木譯言「聖裔」也,回教徒尊之,所至景從。

嘉慶二十五年,薩木薩克次子張格爾,由浩罕糾布魯特寇邊。道光六年,張格爾復求助浩罕入寇,約破西四城,子女玉帛共之,且割喀什噶爾酬其勞。浩罕酋自將萬人至,則張格爾已探喀城無援,背前約。浩酋怒,自督所部攻喀城,不下,率兵宵遁。張格爾使人追誘其眾,歸投者二三千人,張格爾置為親兵。及西四城破,浩罕兵盡得府庫官私之財,並搜括回戶殆遍。楊芳追張格爾至阿賴嶺,遇浩罕伏兵二千,軍幾殆,鏖戰一晝夜始出險。八年,張格爾既伏誅,其妻子留浩罕。欽差那彥成檄令縛獻,不從。詔命絕其互市困之。那彥成並奏驅留商內地之夷,且沒入其貲產。諸夷商憤怒,乃奉張格爾之兄玉素普為和卓木,糾結布魯特、安集延數千入寇,圍喀什噶爾、英吉沙爾,犯葉爾羌,璧昌、哈豐阿等拒而破之。賊悉掠喀、英二城,遁出邊。十一年,浩罕聞大軍且至,由伊犁、烏什、喀城三路出師,築邊墻拒守。又乞俄援,俄弗許。浩罕念無外援,遂遣頭目至喀城謁欽差長齡呈訴,並請通商。長齡遣還二使,留其一使,令縛獻賊目,釋回被虜兵民。浩罕報言,被虜兵民可釋還,惟縛獻夷目事,回經所無。且通商求免稅,並給還鈔沒貲產。

長齡疏言:「安邊之策,振威為上,羈縻次之。浩罕與布哈爾、達爾瓦斯、喀拉提錦諸部落犬牙相錯,所屬塔什干、安集延等七處均無城池,其臨戰皆恃騎賊,然在馬上不能施槍砲。倘以鳥槍連環擊之,則騎賊必先奔。其卡外布魯特、哈薩克向受其欺凌、爭求內徙,而卡內回眾亦恨其虜掠無人理。果欲聲罪致討,但選精銳三四萬人整軍而出,並於伊犁、烏什邊境聲稱三路並進,先期檄諭布哈爾等部同時進攻,則不待直搗巢穴,而其附近諸仇部已乘釁並起,可一舉而平之矣。惟是大軍出塞,主客殊形。自喀浪圭卡倫至浩罕千六百餘里,中有鐵列克嶺,為浩罕、布魯特界山。兩山夾河,僅容單騎,兩日方能出山。此路奇險,勞師遠涉,勝負未可盡知。今擬遣還前所留來使一人,令伯克霍爾敦寄信開導,為相機羈縻之計。蓋浩罕四城外有三小城:曰窩什,在東南;曰霍占,在西南;曰科拉普,在西北。塔什干別為一部,屬右哈薩克,亦附浩罕,稱浩罕八城,故云所屬七處也。」奏入,詔一切皆如所請。浩罕大喜過望,遣使來抱經盟誓,通商納貢焉。

是時,浩罕酋謨哈馬阿里勢頗張,既與中國和,北結俄羅斯,南通印度。其人有才略,而性淫暴。徵民女,納父妾。布哈爾酋遣使責之,謨哈馬阿里怒,髡其使。布哈爾遂率眾攻浩罕,擒斬謨哈馬阿里及其父妾,並俘獲姬妾四十車,凱旋。以伊布拉興留守,遣使至中國卡倫告捷。時道光二十二年也。會伊布拉興虐浩罕民,浩罕叛,立西爾阿里。布哈爾遣兵二萬來伐。有謨蘇滿沽者,浩罕人,謂布酋曰:「此可說而下也!請先行。」布酋許之。至浩罕,乃力勸拒守。布哈爾兵至,攻四十日,不克,解圍去。於是謨蘇滿沽預國政。西爾阿里死,次子古德亞嗣立。謨蘇滿沽妻以女,防之甚嚴,不使接賓客。會塔什干人犯境,謨蘇滿沽挾以出征,兵交而古德亞逃入敵軍。後塔什干平,謨蘇滿沽獲之,復載回國。六月,黨人沙特殺謨蘇滿沽及其黨萬餘人。古德亞走布哈爾,眾立古德亞之弟馬拉。又二年,黨人基布查怨望,謀逆,殺馬拉。立古德亞從弟沙漠拉。古德亞之在外也,為人傭工,以塔什乾之力得復國。後阿林沽作亂,又出奔,商於外,復以布哈爾之力復國。

時俄兵日南,古德亞不能禦敵,請和。古德亞有子曰那西亞丁,頗得民心,種人謀立之,誅其貪吝者,於是國內亂,古德亞奔俄。那西亞丁立,率黨人叛俄,以俄非回教國也。

光緒二十九年,俄人滅其國,置費爾干省。

布魯特分東、西二部。東布魯特在伊犁西南一千四百里,天山特穆爾圖淖爾左右,古為烏孫西鄙塞種地。其部有五,每部各一鄂拓克。最著者三:曰薩雅克鄂拓克,曰薩拉巴噶什鄂拓克;曰塔拉斯鄂拓克。其酋長戴氈帽,似僧家毗盧,頂甚銳,卷末為簷。衣錦衣,長領曲袷,紅絲絳,紅革鞮。民人冠無皮飾,衣褐。

先是東布魯特為準噶爾侵偪,西遷安集延。乾隆二十年,準部平,得復故地。二十三年六月,將軍兆惠等追捕準部餘黨哈薩克沙喇至東布魯特界,遣侍衛烏爾金、托倫泰往撫,抵其游牧珠穆翰地。薩雅克、薩拉巴噶什兩鄂拓克不自主,別推一年長者瑪木克呼里主之。年九十餘,體碩,趺坐腹垂至地,不能遠行。遣使獻牛羊百頭,將軍等宴而示之講武,咸詫服天朝騎射之利,乞內附。於時兼撫定霍索楚、啟臺兩鄂拓克。七月,參贊大臣富德復遣侍衛伊達木札布往諭,薩婁鄂拓克阿克拜亦率眾五千戶來歸,同遣使入朝。其貢道由回部以達京師。

西布魯特與東布魯特相接,在回疆喀什噶爾城西北三百里。西接布哈爾國。道由鄂什山口逾蔥嶺至其地,蓋古之休循、捐毒也。凡十有五部,最著者四:曰額德格納鄂拓克,曰蒙科爾多爾鄂拓克,曰齊里克鄂拓克,曰巴斯子鄂拓克。衣冠風俗皆同東部。

乾隆二十四年,將軍兆惠既定山南,追捕逸回道其地。其渠長遮道奉將軍書曰:「額德格納布魯特部小臣阿濟比恭呈如天普覆廣大無外、如愛養眾生素賚滿佛之鴻仁、如古伊斯干達里之神威、如魯斯坦天下無敵之大勇、富有四海乾隆大皇帝欽命將軍之前。謹率所部,自布哈爾以東二十萬人眾盡為臣僕。頭目等以未出痘,不敢入中國,謹遣使入朝京師。」兆惠以聞,詔受之。於是東、西兩部皆內附。凡布魯特大首領稱為「比」,猶回部阿奇木伯克也。比以下有阿哈拉克齊大小頭目。喀什噶爾參贊大臣奏給翎頂二品至七品有差。歲遣人進馬,酌賚綢緞、羊只。商回以牲畜、皮張貿易至者,稅減內地商民三分之一。二十七年,阿濟比所屬鄂斯諸部地為浩罕所擾,新疆大臣諭還之。明年,別部長阿瓦勒比原以其地供內地游牧,帝喜,許之,賜四品服。

然布魯特人貧而悍,輕生重利,喜虜掠。乾隆以後,邊吏率庸材,撫馭失宜,往往生變。嘉慶十九年,孜牙憞之案,枉誅圖爾第邁莫特,其子阿仔霍逃塞外,憤煽種類圖報復。二十五年,叛回張格爾糾布魯特數百寇邊,有頭目蘇蘭奇入報,為章京綏善叱逐。蘇蘭奇憤走出塞,遂從賊。道光四年,張格爾屢糾布魯特擾邊。五年九月,領隊大臣色彥圖以兵二百,出塞四百里掩之,不遇,則縱殺游牧之布魯特妻子百餘而還。其酋汰列克恨甚,率所部二千人追覆官兵於山谷,賊遂猖獗。於是有八年重定回疆之役。

迨同治三年,布魯特叛酋田拉滿蘇拉滿與庫車土匪馬隆等句結為亂,逆回金相印等乘之,新疆淪陷十有餘年。光緒四年,欽差大臣左宗棠遣劉錦棠收復南八城,駐軍喀什噶爾,有布魯特頭目來謁錦棠,原仍歸中國。自言部落十四,蓋即向之西布魯特也。而東布魯特接伊犁邊者,又有五部:曰蘇勒圖,曰察哈爾,曰薩雅克,曰巴斯特斯,曰薩爾巴噶什,已投附俄羅斯矣。光緒初,俄人並吞浩罕後,西部亦大半為俄所脅收。其附近中國卡倫,喁喁內鄉,代為守邊,可紀者僅千餘家而已。

哈薩克部有三:曰東部,曰中部,曰西部。東哈薩克在舊準噶爾部之西北,東西千里,南北六百里。東界塔爾巴哈臺,西界右哈薩克部,南界伊犁,北界俄羅斯。漢康居國地也。哈薩克汗阿布賚之告順德納曰:「我哈薩克之有三玉茲,如準噶爾之有四衛拉特也。東部者,左部也,曰鄂圖玉茲,謂之伊克準。中部者,右部也,曰烏拉克玉茲,謂之多木達都準。西部最遠,曰奇齊克玉茲,謂之巴罕準。左部之汗曰阿布賚,右部之汗曰脫卜柯依,西部之頭人曰都爾遜。」

初,阿布賚乘準噶爾平,遣使往諭,阿布賚投誠。適阿睦爾撒納叛走哈薩克,阿布賚納之。我兵進,敗其眾。阿布賚大悔,密計擒阿逆以求臣於我。會阿逆遁歸準噶爾。二十二年,阿布賚以其兵三萬助攻阿逆,陳情謝罪,奉表請內附。後阿睦爾撒納奔俄而死,阿布賚乃擒其黨額布濟齊巴罕以獻。其別部和集博爾根及喀拉巴勒特並率其屬三萬戶來附。二十四年以後,屢遣使朝貢,並賜冠服,宴賚如例。

右哈薩克在左哈薩克之西二千里。東界左部,西界塔什干,南界布魯特、安集延諸部,北界俄羅斯,東南界伊犁。亦漢康居五小王地也。其汗曰阿布勒班畢特,即阿比里斯。其巴圖爾有三:曰吐裏拜,曰輝格爾德,曰薩薩克拜,而吐裏拜實專國政。乾隆二十二年,左部阿布賚既臣服,請招右部。會參贊大臣富德方以兵索逆賊哈薩克錫拉至右部,時吐裏拜方與塔什幹交兵,為平之,乃下。於是吐裏拜詣軍門,納款奉馬,進表請歸附。二十三年以後,屢遣使入朝,恩賜宴賚如例。其貢道均由伊犁以達京師。今則自中、俄定界後,哈薩克已分屬兩國矣。

安集延,亦大宛國地。喀什噶爾西北五百里,西至浩罕三百八十里。其貢道由回部以達京師。乾隆二十四年,將軍兆惠檄諭協擒逆回霍集占,其伯克以逆回未至彼境,即專使籥請入覲。二十五年,伯克托克托瑪哈墨第等來朝貢,賜宴賞賚如例。

瑪爾噶朗,在安集延西百八十里。乾隆二十四年,伯克伊拉斯呼裏拜率屬投誠。

那木干,在瑪爾噶朗西北八十里。其地東北與布魯特雜處,東境逾河即為塔什乾地。乾隆二十四年,與浩罕同時輸誠內附。

塔什干,在喀什噶爾西北一千三百里。漢為康居、大宛地,唐之石國也。居平原,有城郭。人民奉回教。與哈薩克同以三和卓分轄其眾:曰莫爾多薩木什,曰沙達,曰吐爾占。舊為準噶爾羈屬。莫爾多薩木什者,哈薩克所置和卓也。吐爾占逐之,與哈薩克構兵。乾隆二十三年,參贊大臣富德追捕哈薩克沙喇至其地,遣使往撫,軍於莽格特城外待之。時吐爾占方與哈薩克戰河上,因諭以睦鄰守土之義,皆感悟釋兵,和好如初。乃遣其屬默尼雅斯奉表求內屬,曰:「臣莫爾多薩木什恭奉諭音,若開瞽昧。蠢茲邊末,敢備外籓,罔或有二心。謹以準孽額什木札布獻之闕下。外臣草莽,冀瞻聖容,躬服彞訓,同歸怙冒,永永無極。」額什木札布者,阿睦爾撒納兄子也。帝宥而遣之。吐爾占亦貢馬稱臣,遣子入覲。塔什干至是自通於中國,列籓臣焉。嘉慶中,塔什乾附浩罕,為浩罕八城之一。同治三年,俄人以伐浩罕之師奪塔什干,開錫爾達利亞省,於是塔什干部遂亡。塔什乾居納林河流域之中樞,扼中亞細亞通道。納林河今又名錫爾河,西北流入咸海。由塔什乾西南行,逾錫爾河至薩馬爾罕,又逾阿母河,分入印度、波斯。北出痾倫不爾厄,越烏拉山脈達歐俄,而東行可至伊犁河以通中國,故俄人置土耳其斯坦總督駐之。塔什乾山泉暢流,其乞爾乞河、卡拉蘇河、安噶連河皆發源雪山,灌溉農田,地宜五穀,故人民常有餘糧。樹木叢雜,多果木。宜蠶桑,而棉花產額尤鉅云。

巴達克山,在葉爾羌西千餘里,居蔥嶺右偏。由伊西洱庫爾西稍南行,渡噴赤河至其國。有城郭,部落繁盛,戶十萬餘。其酋戴紅氈小帽,束以錦帕,衣錦氈衣,腰系白絲絳,黑革鞮。其民人帽頂制似葫蘆,邊飾以皮,衣黃褐,束白絲絳,黑革鞮,亦有用黃牛皮者。婦人不冠,被發雙垂,衣紫氈,餘與男子同。其國負山險,田地腴美,築室以居,耕而兼牧獵。

乾隆二十四年八月,回酋博羅尼都、霍集占兄弟敗奔巴達克山,富德率師至其地,以博羅尼都、霍集占逆狀諭示巴酋素爾坦沙,令擒獻。時二賊竄入巴達克山之錫克南村,詭稱假道往墨克祖國,大肆劫掠。素爾坦沙縛博羅尼都,而以兵攻霍集占於阿爾渾楚哈嶺。霍集占退保齊那爾河,不能支,傷背及乳,擒之,囚於柴札布。柴札布者,系囚處也。素爾坦沙遣使詣軍門投款,且報擒二賊。富德令獻俘,進軍瓦罕城以待。是時溫都斯坦方以兵臨巴達克山,謀劫霍集占兄弟。霍集占陰通巴達克山仇國塔爾巴斯。會謀洩,素爾坦沙遷霍集占兄弟密室,以二百人圍而殺之,刃其馘以獻,並率其部落十萬戶及鄰部博羅爾三萬戶以降。二十五年,遣額穆爾伯克朝京師,貢刀斧及八駿馬。二十七年,再遣使來朝。二十八年,貢馬、犬、鳥槍、腰刀。後其國為愛烏罕所並。巴達克山酋所居地曰維薩巴特,在喀克察河上。噴赤河自瓦罕帕米爾流入境,繞其東北,喀克察河西流入之,下流為阿母河。唐書言竭盤陀國治蔥嶺負徙多河,即巴達克山地也。

博羅爾,在巴達克山東,有城郭,戶三萬餘,四面皆山,西北則河水環之。乾隆二十四年,既與巴達克山同內附,遣其陪臣沙伯克等朝京師。二十七年十一月,博羅爾伯克沙呼沙默特貢劍斧諸物。二十九年,貢匕首。是時博羅爾與巴達克山屢構釁,沙呼沙默特乞援於葉爾羌,都統新柱遣諭巴達克山遵約束,還俘罷兵。至是,沙呼沙默特以所寶匕首進貢謝恩。三十四年,又進玉雙匕首。

博羅爾人別一種,築室而居,有村落,無文字,與諸回部言語不通,惟衣帽則與安集延相類。人皆深目高鼻,濃髭繞喙。男多女少,恆兄弟四五人共一妻,生子女次第分認,無兄弟者與戚里共之。土半沙鹵,故其人苦貧。地多桑,取葚曝乾為糧。飲山羊乳,以馬湩為酒。稱其酋曰「比」。以人口為賦稅,生子女納其半,賣於各回城為奴婢,值頗昂,每口值八九十金。後亦為阿富汗所並。

阿富汗,即愛烏罕。其國北界布哈爾,南界俾路支,東界印度,西界波斯,東西二千餘里。由巴達克山西南行約七百里,歷依色克米什、班因、察里克爾諸回部,越因都庫什山至喀布爾,其國都也。因都庫什山者,蔥嶺山脈右旋之支,迤邐而西,名伊蘭高原。其地波斯處其西,而阿富汗處其東。本罽賓故國。分七大部:首曰喀布爾部,內屬部七;曰岡大害部,內屬部四;曰射士當部;曰愛拉部,內屬部二;曰歐潑部,內屬部三;曰愛乍爾部;曰加非利士當部,內屬部七。西與波斯接壤。有沙磧,餘皆沃壤。其氣候。高地多寒,近低地則熱。物產,果木、棉花、甘蔗、煙草之屬。人皆土著,業農,無游牧。工織毛布,著名西域。戶口約五百餘萬,分二十四族,每族聚居一地,皆自治。其長之升降,則聽命於王焉。其人勇猛樸誠出天性,易撫循。

乾隆二十四年,大軍追討霍集占兄弟二賊,欲假道巴達克山赴阿,巴酋中道邀而殺之。其屬有奔阿者,告以情,阿酋愛哈摩特沙將興師,巴酋素爾坦沙懼,賂以御賜燈及中國文綺,阿遂罷兵,且遣使密爾漢偕巴使來納款,欲窺中國虛實也。二十七年,入貢良馬四,馬高七尺,長八尺。是為回疆最西之屬國。時阿富汗初離波斯獨立,自稱算端,勢張甚,六侵印度,北印度大半為所略。愛哈摩特沙死,國人爭立,紛擾者數十年。

道光六年,德司脫謨哈美德起兵喀布爾,統一阿富汗,愛哈摩特沙玄孫希耶速的逃印度,求庇於英。十九年,英印度總督奧克蘭德攻阿富汗,取乾陀羅、哥疾寧,遂陷喀布爾,立速的為阿富汗王。阿人厭速的,並起絕英軍歸路。英軍敗,德司脫謨哈美德仍復位。二十九年,始與英和。英之有事於阿富汗也,俄人滅布哈爾,次第南侵。英人以阿富汗為印度籓籬,抗之尤力。光緒間,帕米爾分疆之議起,英人復以保護阿富汗為名,出而干涉帕事矣。

帕米爾者,蔥嶺山中寬平之地,供回族游牧者也。帕地有八,其中皆小回部錯居。乾隆中,大部隸屬中國,羈縻之使弗絕。厥後迤北、迤西稍稍歸俄,迤南小部附於阿富汗,東路、中路則服屬於中國。於是帕米爾遂為中、俄、阿富汗三國平分之地。出帕米爾,南逾因都庫什山,即達印度,故俄人盡力經營之,而英人亦遂急起而隱為之備。英之為阿爭,即不啻為印度爭也。

初,乾隆二十四年,高宗平定回疆,窮追賊首至伊西洱庫爾,三戰三捷,遂蕆大功。高宗御製碑文勒銘淖爾,西域圖志所指為喀什噶爾西境外地者也。當日喀城邊卡西境之玉斯屯阿喇圖什卡,僅八十里;西南之鄂坡勒卡,僅一百二十里。道光間,欽定邊卡西至烏帕喇特卡,一百二十里;西北至喀浪圭卡,一百五十里。迨光緒間,克復新疆,劉錦棠始增設七卡於舊界之外。十五年,又設蘇滿一卡於伊西洱庫爾淖爾北十里,是卡距喀城千六百里,最為窵遠,僅以布魯特回人守之,未駐兵也。英使之初議分帕也,我國嚴拒之,未允其請。既而俄兵闌入帕地,我國責其稱兵越界,俄人即引咎退歸。光緒十七年,英兵入坎巨提,逐其頭目,其意在覷覦帕地也。新疆巡撫檄馬隊巡歷邊境,屯於蘇滿。十八年春,俄人來言帕地為中、俄兩屬,未經勘界,中國不應駐兵。總理衙門遂電疆撫退兵,而仍留蘇滿卡倫。俄復請盡撤新設諸卡,然後勘界。正相持間,而英人陰嗾阿兵突至蘇滿,脅擄布回而去,俄遂進兵與阿人戰於蘇滿,其東隊則游弋於郎庫里湖、阿克塔什,漸近喀邊。總理衙門疏言:「我國先駐蘇滿之兵不早撤回,則俄、阿戰事將自我啟之,轉難收束。阿雖占地而適致俄兵,蠻觸相爭,原可不必過問。但其東駸駸逼近邊境,頗為可慮耳。」蓋阿富汗自乾隆後朝貢不通,久置之度外矣,至是復一見焉。二十一年,帕米爾界議始定。

坎巨提,即乾竺特,在葉爾羌西南約一千五百里。自葉爾羌西行入蔥嶺,至塞勒庫勒之塔什庫爾干,即蒲犁也。由是西行,逾尼若塔什山口,又西南至塔克敦巴什帕米爾,為八帕之一。由是南逾瓦呼羅特、明塔戛兩山口,西為因都庫什山,東為穆斯塔格山。出山口順棍雜河南行,又順河折西抵棍雜,即坎巨提都城,城瀕棍雜河北岸。西域水道記言:「塞勒庫勒在葉爾羌之西八百里,為外蕃總會之區。自塞勒庫勒西五日程,曰黑斯圖濟;又西南三日程,曰乾竺特。」即坎巨提,譯文異耳。乾隆二十六年,其酋有黑斯婁者,始內附,即葉爾羌辦事大臣新柱奏稱「乾竺特伯克黑斯婁遣子貢金」者也。

其人皆奉瑪罕默德回教。其部落東西寬二十里,南北長六百里。兩山夾立,廣大峻削,中有大河,為入南疆要隘。坎部民住河西,河東則哪格爾所屬也。棍雜城大約三里。城北有大山曰溫吉爾,河曰崇帶雅。所轄村莊二十五,城中居民二千餘,其在各莊者約五千餘人,城鄉大小頭目一百四十。土產牛、羊、馬匹,無布帛,盡衣毛褐。五穀諸果俱備。敵國有犯境者,民即為兵,選精壯者出關御之。人皆業農,不納糧,不徵稅,惟歲與其酋耕斂而已。每歲貢中國砂金一兩五錢,派之民,農戶收麥十二斤,畜牧家則戶收羊羔一,以集此款,無他徭也。貢使至,朝廷賞大緞兩端。其貢至宣統間不絕。

道光間,克什米爾國王熱吉苦羅普散令其將布甫山率兵犯境,奪坎屬麻雲卡,坎酋夏孜牌爾敗之,追斬七千餘名。克什米爾遣使構和,年與坎酋洋銀一千五百元,元重二錢五分;坎酋以馬二匹、細狗二只報之。人謂入貢克什米爾者,妄也。同治四年,克什米爾國王令就貝爾薩再犯境,坎王艾贊木復戰敗之,蓋至是克什米爾已四犯坎屬矣。

光緒間,俄兵入帕米爾,英人聞之,率兵至哪格爾,並檄坎巨提修平道路,備兵進帕地。哪格爾首抗英,坎酋助之。十七年,英人敗哪格爾,直抵坎城,賽必德哎里罕戰敗,攜眷屬潛遁,英人遂據其地。先是賽酋私與俄通,上降書,押結約俄奪占帕米爾,修築堡壘於黑孜吉牙克、阿克素睦爾瓦、蘇滿三處,並建營於包子滾拜子,以扼要沖。俄人復書,報以金幣千元、金絲呢布諸貨六馱、快砲六桿。賽酋悖逆無信,不恤部眾,且狡而好利,屢挑釁英、俄以求賄,視其部為市販。其副目歪孜爾素執兵權,同惡相濟,部民皆深忌之。至是,率其眾五百餘人將奔俄,塔墩巴什頭目窩思滿集眾邀之。張鴻疇拘諸色勒庫爾,屢謀突城出,不得,後解省羈禁十有七年,嗣復安置庫車,其子米則拜爾及家屬男女五十二人,均編住莎車熱瓦奇莊,賽酋之外產也;脅從之眾悉送還部,並諭飭賽酋之弟買賣提哎孜木代理坎巨提頭目,以安民心。

出使英法義比大臣薛福成與英外部商定派員會立坎酋,其疏略云:「中國回疆之外,向有羈縻各回部,惟自咸豐、同治以來,中國內寇不靖,未遑遠略。俄國既以兵力吞並浩罕、布魯特、哈薩克、布哈爾諸回部,而巴達克山、魯善、什克南、瓦罕諸小部,則皆服屬於阿富汗。邇來阿富汗為英屬國,英之大勢駸駸由印度北鄉,有與俄國爭雄之意,而中國西邊之外,遂日以多事。坎巨提一部近喀什噶爾,南界在蔥嶺以南,厥地縱橫數百里,戶口約近萬人。近年屬回之入貢中國者祗此一部,蓋即新疆識略之乾竺特、一統輿圖及時憲書之喀楚特,同音而異譯也。英之印度總督歲貼坎巨提經費,以助彼整理防務為名,實隱收其內政之權。去年夏秋間,坎巨提已有赴喀什噶爾告急之舉,則以英人築一砲臺俯臨坎境也。本年正二月間,疊承總署電信,以英兵侵坎巨提,其頭目連戰不勝,率其眾逃詣卡外求援。臣以起釁情節詰英外部,詢知英兵修築一路直貫坎境,北抵興都哥士大山,意在扼此隘口,以杜俄眾南侵而保印度門戶。其頭目興師攔阻,為英兵擊敗,踞其所居之棍雜城。臣與英相兼外部尚書沙力斯伯里晤商,據稱並無滅坎之意,亦無阻坎入貢中國之意。祗以坎酋罪惡甚多,輕慢英官,不得不示以懲儆也。臣與總署電商,因坎酋聲名素劣,勢難必使復位。其部既系兩屬之國。與專屬中國者又稍不同,祗可酌就外部之辭與之理論。外部語言閃鑠,其初次存坎之說既甚游移,而必欲據坎之心則甚堅韌。幸而窺彼隱情,頗以俄焰方張,亟思聯絡中國,不欲斂怨樹敵,臣得就此設法磋磨。英廷近稱選得舊酋之弟買賣提哎孜木,可為坎巨提頭目,擬請中國派員會同英員行封立之禮,已由總署電告新疆巡撫選派妥員前往。臣與外部商訂儀節,華員、英員共為一班,克什米爾系英屬國,位次應稍居後。行禮之期,初訂在十八年閏六月二十三日,現展至七月二十五日,屆時彼此和衷妥辦,即可蕆事。」新疆巡撫陶模即委阜康縣知縣田鼎銘、都司張鴻疇前赴坎部,會同英員熱布生,更立買賣提哎孜木為坎巨提頭目。封立儀節,華員居右,英員次之,英屬克什米爾委員居左稍下,新酋又次之。張鴻疇宣布皇上德意,賞給大緞,諭令貢金照舊呈進,鎮撫部民,毋任剽掠。其酋悉俯首聽命云。

坎部國於山谷中,崇峰疊巘,道路險絕。中有喀喇闊魯穆大冰山,時至十一月,積雪甚厚,以長毛牛負囊橐而行。明塔戛山口高萬四千四百尺,路有巨石,蓋古時流冰所經地也。出山口裡許,有一流冰,過此即易行。再逾數澗,兩崖壁立,頂有積雪,至米斯戛。居人皆韃爾靼回教。不幕,有室廬,村各為堡,壘石為之。性強悍,以寇鈔為俗,然皆酋所使,所劫貨物大半歸酋,四出剽掠,或遠至庫車。雅爾山脈下垂如★M3,水流其間,土較腴美。近帕蘇又一流冰,其融處高八千尺。

光緒十五年,英人楊哈思班游至其部,坎酋言:「我受上帝命,親斷父母死罪而殺之,並殺其兄弟,投於山下,遂踐是位。」其悖逆如此。或謂其地立國最古,殆周時曹奴氏之所居。穆天子傳:「庚辰,濟於洋水;辛巳,入於曹奴,曹奴之人獻天子於洋水之上。」洋水即棍雜河。山海經言:「洋水西南流注于醜塗之水。」今棍雜河發源因都庫什山,西南流至幾勒幾特城,東南入印度河。醜塗為印度轉音,醜塗水即印度河也。


\end{pinyinscope}