\article{列傳五十五}

\begin{pinyinscope}
米思翰子李榮保顧八代瑪爾漢田六善杜臻薩穆哈

米思翰,富察氏,滿洲鑲黃旗人。先世居沙濟。曾祖旺吉努,當太祖時,率族來歸,授牛錄額真。父哈什屯,事太宗,以侍衛襲管牛錄。擢禮部參政,改副理事官。討瓦爾喀,招明總兵沈志祥。從攻錦州,明總兵曹變蛟夜襲御營,先眾捍禦,被創,力戰卻之。順治初,授內大臣、議政大臣,世職屢進一等阿達哈哈番兼拖沙喇哈番。睿親王多爾袞攝政,諸大臣鞏阿岱等並附之,哈什屯獨持正,忤睿親王,降世職拜他喇布勒哈番。肅親王豪格以非罪死,鞏阿岱等議殺其子富綬,哈什屯與巴哈力持,事乃已。世祖親政,累進世職一等阿思哈尼哈番加拖沙喇哈番。十二年,獎奉職恪勤諸大臣,加太子太保。康熙初卒,謚恪僖。

米思翰,其長子也,襲世職,兼管牛錄,授內務府總管。輔政大臣從假尚方器物,力拒之。聖祖親政,知其守正,授禮部侍郎。八年,擢戶部尚書,列議政大臣。是時各直省歲賦,聽布政使存留司庫,蠹弊相仍,米思翰疏請通飭各直省俸餉諸經費,所餘悉解部,由是勾稽出納權盡屬戶部。

十二年,尚可喜疏請撤籓,吳三桂、耿精忠疏繼入,下戶、兵二部議。米思翰與兵部尚書明珠議三籓並撤,有言吳三桂不可撤者,以兩議入奏。復集諸大臣廷議,米思翰堅持宜並撤,議乃定。既而吳三桂反,上命王貝勒等率八旗兵討之,議者謂軍需浩繁,宜就近調兵禦守。米思翰言:「賊勢猖獗,非綠旗兵所能制,宜以八旗勁旅會剿。軍需內外協濟,足支十年,可無他慮。」於是請以內府所儲分年發給,復綜覈各直省庫金、倉粟,以時撥運,悉稱旨。又疏言:「師行所至,屢奉明詔以正賦給軍需,恐有司尚多借端私派,請敕各督撫嚴察所屬,供應糧餉薪芻,一切動官帑,毋許苛派;其購自民間者,務視時價支給,勿纖毫累民。」上命如議速行。

米思翰尋卒,年甫四十三,上深惜之,予祭葬,謚敏果。時三桂勢方張,精忠及可喜子之信皆叛,議者追咎撤籓主議諸臣,上曰:「朕自少時,以三籓勢日熾,不可不撤。豈因其叛,諉過於人耶?」及事定,上追憶主議諸臣,猶稱米思翰不置。

米思翰子馬斯喀、馬齊、馬武,皆自有傳。

李榮保,襲世職,兼管牛錄,累遷至察哈爾總管,卒。乾隆二年,冊李榮保女為皇后,追封一等公。十三年,冊謚孝賢皇后,推恩先世,進封米思翰一等公。十四年,以李榮保子大學士傅恆經略金川功,敕建宗祠,祀哈什屯、米思翰、李榮保,並追謚李榮保曰莊愨。

顧八代,字文起,伊爾根覺羅氏,滿洲鑲黃旗人。父顧納禪,事太宗,從伐明,次大同,攻小石城,先登,賜號「巴圖魯」,予世職牛錄章京。旋授甲喇額真。順治初,從入關,定陜西、湖南、江南、浙江,皆在行間,進三等阿達哈哈番。子顧蘇,襲,進二等。

顧八代,其次子也。任俠重義,好讀書,善射。以廕生充護軍。順治十六年,從征雲南有功,授戶部筆帖式。旋以顧蘇及子佛岳相繼卒,無嗣,顧八代襲世職,遷吏部郎中。康熙十四年,聖祖試旗員第一,擢翰林院侍讀學士。

吳三桂陷湖南,遣其將掠兩廣。鎮南將軍莽依圖自江西下廣東,駐韶州。十六年,上命顧八代傳諭莽依圖規復廣西,即留軍,從征廣西。巡撫傅弘烈為三桂將吳世琮所敗,莽依圖引兵與相合。顧八代按行諸軍,謂結營散亂,敵至慮不相應。世琮兵至,師復敗,還駐梧州。世琮來追,擊卻之。顧八代策世琮且復至,益誡備。會除夕,世琮以三萬人奄至,又擊敗之。十七年,師進次盤江,與世琮軍遇,莽依圖病甚,以軍事屬顧八代;偕副都統勒貝等渡江,與世琮戰,分兵出敵後,破其左而合擊其右。世琮潰圍出,遣精騎追之,自殺。師進克南寧,叛將馬承廕與三桂軍合,可十萬,拒戰。諸將或難之,顧八代奮入陣,諸將皆力戰,遂破敵。

十八年,京察,掌院學士拉薩里、葉方藹以顧八代從征有績效,注上考;大學士索額圖改注「浮躁」,坐奪官。莽依圖疏言顧八代從征三載,竭誠奮勉,運籌決勝,請留軍委署副都統,參贊軍務,上命以原銜從征。十九年,莽依圖卒於軍,顧八代從平南大將軍賚塔下雲南,攻會城。顧八代議當先取銀錠山,俯瞰城內,攻得勢。及勇略將軍趙良棟師至,用顧八代策,先取銀錠山,克會城,雲南平。師還,授侍講學士。

二十三年,命直尚書房,累遷禮部侍郎。二十八年,授尚書。三十二年,坐事,上責其不稱職,奪官,留世職,仍直尚書房。三十七年,以病乞休。四十七年,卒。

顧八代直尚書房時,世宗從受學;及卒,貧無以斂,世宗親臨奠,為經紀其喪。雍正四年,詔復官,加太傅,予祭葬,謚文端,又以其貧,賜其家白金萬。八年,建賢良祠京師,諭滿洲大臣當入祀者五人,大學士圖海、都統賚塔,次即顧八代,及尚書瑪爾漢、齊蘇勒。

子顧儼,襲世職,自參領官至副都統。孫顧琮,自有傳。

瑪爾漢,兆佳氏,滿洲正白旗人。順治十一年,繙譯舉人,授工部七品筆帖式,累遷刑部員外郎。

康熙十三年,陜西提督王輔臣叛應吳三桂,上命揚威將軍阿密達自江寧移師討之,瑪爾漢以署驍騎參領從。十四年,與副都統鄂克濟哈、穆舒琿等自涇州進兵,屢破壘,斬級數百,克寧州。十五年,大將軍圖海督兵圍平涼,輔臣降,瑪爾漢還京師。圖海請調涼州、寧夏、固原諸鎮兵進攻興安、漢中,上命副都統吳丹及瑪爾漢赴諸鎮料理徵發,兼詢緩急機宜。甘肅提督張勇請緩師,上命圖海固守鳳翔、秦州諸要隘,分兵授征南將軍穆占下湖廣,命瑪爾漢從。十七年,授御史。

十九年,穆占師進貴州,二十年,師進雲南,瑪爾漢皆在行間,得功牌十二。雲南平,師還,追論征湖南不力援永興,致損將士,奪功牌九。二十一年,命巡視河東鹽政。御史許承宣、羅秉倫劾山西巡撫圖克善令平陽屬十三州縣增報鹽丁加課累民,下巡撫穆爾賽會瑪爾漢覈實,請免虛報一萬七千餘丁。二十五年,以按治歸化城都統固穆德不實,吏議左遷。二十六年,授理籓院司務。從大學士索額圖等使鄂羅斯定邊界,辭辨明析,鄂羅斯人折服。事聞,聖祖嘉其能。尋遷戶部郎中。三十三年,遷翰林院侍講學士,再遷兵部侍郎。三十五年,上親征噶爾丹,命瑪爾漢駐土木董理驛站,以送軍馬羸,吏議奪官,命寬之。

三十八年,遷左都御史。再遷兵部尚書,充經筵講官、議政大臣。四十三年,歲饑,流民就食京師。命與內大臣佟國維、明珠、阿密達等監賑。四十六年,調吏部。四十八年,以老病乞休。五十七年,卒,年八十五。上遣內大臣臨奠,賜祭葬。雍正八年,世宗諭獎瑪爾漢謹慎忠厚,事聖祖宣力多年,完名引退,贈太子太傅。賢良祠成,命入祀。乾隆元年,高宗命追謚恭勤。

田六善,字兼山,山西陽城人。順治三年進士,授河南太康知縣,時當兵後,勞來安集。九年,巡撫吳景道疏薦才守兼優,遷戶部主事,監臨清關,復監鳳陽倉兼臨淮關。罷濫徵,革奇羨,商民稱便。累遷郎中。十五年,授江南道御史。兵部議禁民乘馬,六善疏言其不便,下廷臣集議,弛禁。十六年,疏言:「欲安民在勸清吏,乞敕各督撫實行薦舉,吏部於各督撫蒞任一二年後,列奏薦舉何人,能否察吏安民,即可以是鑒別。議者或謂舉薦清吏,無以處乎不在清吏之列者,一難也;恐督撫依舊受賄徇私,二難也;徵糧緝逃處分罣礙,三難也。然臣謂清吏果得薦舉,則為清吏者見公道尚存,益堅其持守,一便也;群吏以不著清名為愧,力自濯磨,二便也;某省有清吏幾人,以驗政治修廢,三便也;天下曉然知有能必先有守,風俗丕變,四便也;向日督撫厭憎清吏無益於己,今必且卵翼而親愛之,五便也。不惑於三難,力致其五便,將循良興起,不讓前古矣。」下部議行。尋命巡視長蘆鹽政。十七年,還掌江南道事。

康熙元年,乞假歸。三年,補貴州道御史。四年,疏言:「兵部議裁山西、陜西、河南等處兵額,三營裁一營。遇裁之兵,挾久練之技,處坐困之時,窮無所歸,遂為賊盜。請諭總督、提督諸臣,察已裁之兵,如弓馬嫻熟、膂力精強,仍收入伍。自後老弱必斥,逃亡不補。所漸去者疲卒,不慮其為非;所招回者勁兵,可資其實用。」下部議,令各營汰去老弱,其年力精壯者仍留充伍。又疏言:「吏部於往日曾行之事,率皆援以為例,惟意所彼此,莫窮其弊。請敕部以上所裁定及有旨著為例者,匯為一冊,敬謹遵守,餘仍循舊章。」得旨,如所請。七年,命巡視京、通倉,還掌山東道事,得旨內升,回籍待缺。

十一年,授刑科給事中,秩視正四品。疏言:「臣里居讀上諭,以蘇克薩哈為鼇拜仇陷,殺其子孫,連坐族人白爾赫圖,恩予昭雪。臣思法律為天下共者也,以滿洲勞苦功高之人,因與執政諸臣意見相左,輒牽連興大獄,恐尤而效之,報復相尋,借端推刃。周禮有八議,罪大可減,罪小可赦。請特制昭示,滿洲犯罪非反叛有實跡者,一準於律,勿妄議株連。儲人才,固國本,於是乎在。」上韙其言,下王大臣議,從之。又疏言:「聖學宜先讀史。史者,古帝王得失之林也。其君寬仁明斷,崇儉納諫,則其民必安,其事必治,其世必興必平。若夫苛察因循,惡聞過,樂逞欲,其民必不安,其事必不治,其世必衰必亂。乞諭日講諸臣,以通鑒與經史並進。」得旨俞允。尋轉戶科掌印給事中。三遷至右僉都御史。

十三年,疏言:「吳三桂負恩叛逆,處必滅之勢。綠旗月餉,步兵一兩有奇,馬兵二兩有奇,甲胄不必堅強,弓刀不必精利,登山涉水,資以先驅。臣謂綠旗力雖弱,善用之則強;心雖渙,善收之則聚。供給宜足,勞逸宜均。至先登破陣,無分滿、漢,賞賚公平。斯忠勇自奮,克佐勁旅以奏膚功,今日所宜急計者也。」下部議鼓勵綠旗官兵敘給爵賞例。遷順天府尹。未幾,復遷左副都御史。十四年,疏言:「臣昔為河南知縣時,孫可望、李定國尚據雲、貴、四川,其勢不減於吳三桂。金聲桓叛江西,姜瓖叛大同,亦不異耿精忠、王輔臣。而當日民心未若今之驚惶疑懼者,由其時督撫有孟喬芳、張存仁、吳景道諸臣,敦行儉樸,慎守廉隅,吏治肅清,民生樂遂也。宜特頒嚴諭,令各督撫禁雜派,覈軍實。有司或剝民敗檢,立行糾劾,以省民力、安眾心。師行所至,更宜審酌剿御。近見江西、浙江報捷諸疏,屢言殺賊累萬。然必待殺盡而後入閩,恐愚頑之民無盡,草竊之賊亦無盡。臣謂先取精忠,則群賊自息。昔姜瓖乍叛,土寇群起,瓖滅,土寇亦盡,其明驗也。至三桂狡謀,覬以一隅之地困天下全力,我即以天下全力困此一隅。三桂授首,則四川、廣西不煩兵而自定。」又疏言:「臣籍山西,與陜西接壤。黃河自邊外折入內地,至蒲州一千餘里。蒲州上至禹門,為平陽府屬,河西為西安,有提督、總兵重兵駐守。自此以北,永寧州、臨縣為汾州府屬,渡口有孟門鎮、高家塔諸處;更北保德州為太原府屬,渡口有黑田溝、窮狼窩諸處。河西為延安,素稱荒野,河東為交城,路險山深,草竊潛匿。請敕巡撫、提督分兵駐防。」又疏言:「師已抵平涼,輔臣迫於必死,困獸猶鬥,殺賊百不償失兵一。宜駐軍城下,以逸待勞,急攻固原,絕其糧道。平涼地瘠,非比湖南地廣米多,可以持久。糧道不通,人心自散,必有斬輔臣獻軍門者。若賊東出則東應,賊西出則西應,疲我師徒,分我威力,固原圍解,賊氣貫通,此斷斷不可者也。」諸疏並下王大臣議行。

十六年,擢工部侍郎。十七年,以夏旱求言,疏言:「今日官至督撫,居莫敢誰何之勢,自非大賢,鮮不縱恣。道府歲納規禮,加之以搜括,則道府所轄官民,不啻鬻之道府矣。州縣歲納規禮,重之以勒索,則州縣所屬士民,不啻鬻之州縣矣。世祖朝,山東巡按程衡劾巡撫耿焞,江南巡按秦世禎劾土國寶,皆置重典,天下肅然。今巡按久停,雖欲議復,恐一時難得多人。惟有出自上意,欲清一省,則選一人遣往,不必一時俱發。出其不意,示以不測,使天下奸惡吏不敢恃督撫而肆志,即有不肖之督撫,亦莫敢庇貪而害民。」疏入,報聞。

調戶部。十八年,疏言:「國家有錢法以通有無、利民用,自秦、漢及唐、宋,公私皆悉用錢;至金、元,以銀與錢鈔並行;至明中葉,乃專資於銀。闖逆之亂,或沉江河,或埋山谷,又以貪吏厚藏,銀益少,民益困。今欲救天下之窮,惟有多鑄錢。鑄錢所資,銅六鉛四,而可採之山,所司每深諱之,蓋恐時有時無,貽累償稅。且上官聞其地開採,此挾彼制,誅求甚多也。臣謂宜令天下產銅鉛之地,任民採取,有則以十分二輸稅於官,無則聽之州縣自行稽察,毋使多官旁撓。報採多者予議敘,則官與民皆樂為,資以鼓鑄,錢不可勝用矣。」下九卿詳議,擬例以上,得旨:「採銅關系國計,其令各督撫率屬殫力奉行。」

六善以老病乞罷,上不許。二十年,命致仕。三十年,卒於家,年七十一。

杜臻,字肇餘,浙江秀水人。順治十五年進士,改庶吉士,散館,授編修。累遷內閣學士,擢吏部侍郎。

國初以海上多事,下令遷東南各省沿海居民於內地,畫界而設之禁。界外皆棄地,流民無所歸,去為盜。及師定金門、廈門,總督姚啟聖請以界外地按籍還民,弛海禁,收魚鹽之利給軍食,廷臣持不可。康熙二十二年,臺灣平,上命以界外地還民。會給事中傅感丁請以江、浙、閩、粵濱海界外地招徠開墾,乃命臻及內閣學士席柱赴福建、廣東察視展界,進臻工部尚書。臻與席柱如廣東,自欽州防城始,遵海以東而北,歷府七、州三、縣二十九、衛六、所十七、巡檢司十六、臺城堡砦二十一,還民地二萬八千一百九十二頃,復業丁口三萬一千三百。復如福建,自福寧州西分水關始,遵海以北,歷府四、州一、縣二十四、衛四、所五、巡檢司三、關城鎮砦五十五,還民地二萬一千一十八頃,復業丁口四萬八百。於是兩省濱海居民咸得復業。別遣使察視江南、浙江展界復業,同時畢事。臻以母喪還里,席柱復命,奏陳濱海居民還鄉安業。上曰:「民樂處海濱,以可出海經商捕魚,爾等知其故,前此何以不準議行?邊疆大臣當以國計民生為念,曩禁令雖嚴,私出海貿易初未嘗斷絕。凡議出海貿易不可行者,皆總督、巡撫自圖射利故也。」

臻喪終,起刑部尚書。舊制,方冬獄囚月給煤,獄吏率乾沒,囚多以寒疾死,臻力禁之。調兵部。時議裁各省駐防及督、撫、提、鎮標兵,臻謂:「兵冗可裁而不宜驟行,請自今老弱、物故、額缺概不補,數歲額自減。」從之。再調禮部。以疾告歸,尋卒於家。上南巡,書「眷懷舊德」額追賜之。

臻少貧力學,事祖母及父母孝,宏獎人才,詩文剴切中條理。

薩穆哈,吳雅氏,滿洲正黃旗人。順治十二年進士,授戶部主事,遷員外郎。

康熙十二年,聖祖允吳三桂疏請撤籓,遣薩穆哈偕郎中黨務禮、席蘭泰,主事辛珠,筆帖式薩爾圖如貴州,具舟及芻粟,諭以毋騷擾,毋遲悮。既至,三桂謀反,提督李本深與謀,書招貴州巡撫曹申吉,總督甘文焜得之,告薩穆哈等,趣詣京師告變,並請兵赴援。薩穆哈與黨務禮、席蘭泰行至鎮遠,三桂已舉兵,鎮遠將吏得三桂檄,不給驛馬。薩穆哈、黨務禮得馬二,馳至沅州。乃乘驛,十一晝夜至京師,詣兵部,下馬喘急,抱柱不能言,久之始蘇,上三桂反狀。席蘭泰自鎮遠乘小舟至常德,乃乘驛,後七日至。辛珠、薩爾圖不及行,死之。十三年,擢薩穆哈刑部郎中。十四年,敘告變功,薩穆哈、黨務禮、席蘭泰並應升光祿、太僕諸卿。

十五年,授太僕寺卿。十六年,再遷戶部侍郎。命監賑山東。十七年,還京師。疏言:「臣屢奉使命,所過州縣,間有藉差科派民財,深滋擾累。請嗣後有大事,特遣部院官,餘並責督撫料理。」上為下廷臣會議,定州縣科斂俱視貪吏治罪。調吏部。二十年,再遷工部尚書。二十一年,命察視石景山至盧溝橋石是,疏言:「堤內本官地,康熙初招民墾荒,致侵損堤根。請敕部免其賦,罷勿復耕。」從之。二十二年,命察視山西地震,疏請被災最重州縣發帑治賑。

二十四年,河道總督靳輔請於高郵、寶應諸州縣築堤,束黃河注海,按察使於成龍主濬海口,下廷臣議,用輔策。上詢日講官籍江南者,侍讀喬萊力請用成龍策。上曰:「鄉官議如此,未知民意如何?」令薩穆哈與學士穆成額,會漕運總督徐旭齡、巡撫湯斌,詳察民間利害。薩穆哈等行歷海口諸州縣,諸州縣民陳狀參差不一;檄諸州縣,令各擇通達事體者十人詢利害,皆言濬海口不便。二十五年,薩穆哈還奏,謂詳問居民,從成龍議;積水不能施工,從輔議;水中亦不能取土,請兩罷之。是時成龍召詣京師,上命廷臣及薩穆哈、成龍再議。成龍言濬海口當兼治串場河,費至百餘萬。廷臣以為費鉅,疏請停。未幾,斌入為尚書,奏言:「海口不急濬,再遇水,下游諸州縣悉付巨浸。」上召問薩穆哈,薩穆哈不堅執前奏。復下廷臣議,始定用成龍策。上責薩穆哈前覆奏不實,奪官。尋授步軍翼尉。

三十二年,仍授工部尚書。三十九年,上察知工部積弊,河工糜帑,受請託,發銀多侵蝕,詰責薩穆哈等。薩穆哈尋以老疾乞罷,上斥其偽詐,命奪官,仍留任,察工部積弊,一一自列。四十三年,以疏濬京師內外河道侵蝕帑銀,薩穆哈得賕,逮治擬絞。卒於獄。

論曰:米思翰贊撤籓之議,綢繆軍食,足以支十年,知定謀有由也。顧八代、瑪爾漢皆文臣,能克敵,復以廉勤建績。六善於軍事有建白,收綠旗之用,其效著於後矣。臻巡復海疆,兵後一大政也。薩穆哈以告變受賞,亦附著於斯篇。


\end{pinyinscope}