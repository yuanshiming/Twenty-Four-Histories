\article{列傳七十三}

\begin{pinyinscope}
王掞子奕清奕鴻勞之辨硃天保陶彞任坪範長發鄒圖云

陳嘉猷王允晉李允符範允高玢高怡趙成■孫紹曾邵璿

王掞,字藻儒,江南太倉人,明大學士錫爵孫。康熙九年進士,選庶吉士,授編修,為掌院學士熊賜履所器。遷左贊善,充日講起居注官。以病告八年,起右贊善。提督浙江學政,嚴剔積弊,所拔多宿學寒畯。龍泉知縣茅國璽以印揭薦武童,掞疏劾,國璽坐譴,別疏陳剔除積弊,報聞。累遷侍讀學士。三十年,超擢內閣學士。三十三年,遷戶部侍郎,直經筵。三十八年,調吏部,禁革臨選駁查、臨掣買簽諸弊,銓政以肅。偕尚書範承勛、王鴻緒督修高家堰河工。

四十三年,擢刑部尚書。刑部奏讞無漢字供狀,掞言:「本朝官制,兼設滿、漢,欲其彼此參詳。今獄詞不錄漢語,是非曲直,漢司官何由知之?若隨聲畫諾,幾成虛設。嗣後定讞,當滿、漢稿並具。」詔報可,著為令。累歷工、兵、禮諸部,務總紀綱,持大體。五十一年,授文淵閣大學士,兼禮部尚書,直經筵如故。五十二年,典會試。其冬,以疾疏辭閣務,溫旨慰留。越年春,疾愈,仍入直。孝惠章皇后祔太廟,議者欲祔於孝康章皇后之次,掞曰:「孝康章皇后雖母以子貴,然孝惠章皇后,章皇帝嫡配也,上聖孝格天。曩者太皇太后祔廟時,不以躋孝端文皇后之上,今肯以孝康章皇后躋孝惠章皇后上乎?」禮部不從,上果以為非,令改正。

時上春秋高,皇太子允礽既廢,儲位未定。掞年七十餘,自念受恩深,又以其祖錫爵在明神宗朝,以建儲事受惡名,欲幹其蠱。五十六年,密奏請建儲,疏入,留中。是年冬,御史陳嘉猷等八人復以為言,上不悅,遂並發掞疏,命內閣議處。忌掞者欲置重典,掞止宮門外不敢入。上顧左右,問:「王掞何在?」李光地奏掞待罪宮門。上曰:「王掞言甚是,但不宜令御史同奏,蹈明季惡習。汝等票擬處分太重,可速召其來。」掞聞命趨入,免冠謝。上招掞跪御榻前,語良久,秘,人不能知。

六十年春,群臣請賀萬壽,上勿許。掞復疏前事,請釋二阿哥,語加激切。既而御史陶彞等十二人連名入奏,上疑出掞意,大怒,召諸王大臣,降旨責掞植黨希榮,且謂:「錫爵在明神宗時,力奏建儲,泰昌在位,未及數月,天啟庸懦,天下大亂,至愍帝而不能守。明之亡,錫爵不能辭其罪。掞以朕為神宗乎?朕初無誅大臣之意,大臣自取其死,朕亦無如何。」令王大臣傳旨詰掞,令回奏。時舉朝失色,無敢與筆硯者。掞就宮門階石上裂紙,以唾濡墨,奏言:「臣伏見宋仁宗為一代賢君,而晚年立儲猶豫,其時名臣如範鎮、包拯等,皆交章切諫,須發為白。臣愚,信書太篤,妄思效法古人,實未嘗妄嗾臺臣共為此奏。」奏上,越五日,詔緩議罪,與諸御史俱赴西陲軍前效力。因掞年老,責其子奕清代往,為父贖罪。先是,掞嘗密奏請減蘇、松浮糧,言至剴切,疏久留中。至是忤旨,乃與建儲奏疏一並擲還。是年冬,上自熱河還京師。掞迎駕石槽,上望見,遣內侍慰問。六十一年元旦,諸大臣表賀,未列掞名,上發表命列名以進。翌日,賜宴太和殿,再召見西暖閣,賜坐,慰諭有加。尋起原官,視事如故。

雍正元年,以老乞休,世宗降旨褒嘉,以原官致仕,仍留京師備顧問。三年,上諭閣臣云:「王掞向人言,曾在聖祖前奏免蘇、松浮糧,未蒙允行。朕查閱宮中並無此奏。」因責掞藉事沽名,並涉其子奕清、奕鴻諂附年羹堯,目為奸巧,乃遣奕鴻與奕清同在軍前效力。六年,掞卒,年八十四。乾隆二年,奕清始請血阜於朝,賜祭葬如制。

奕清,字幼芬。康熙三十年進士,選庶吉士。歷官詹事。代父赴軍,歷駐忒斯、阿達拖羅海。奕清體羸善病,處之晏然。雍正四年,命赴阿爾泰坐臺。又十年,乾隆元年,召還,仍以詹事管少詹事。乞假葬父,尋卒。

奕鴻,字樹先。康熙四十八年進士,授戶部主事。歷湖南驛鹽、糧儲道。奕清赴軍,奕鴻盡斥其產與俱。後命赴烏里雅蘇臺效力。居邊十年,與奕清同釋還,官四川川東道。引疾歸,卒。

勞之辨,字書升,浙江石門人。康熙三年進士,選庶吉士,授戶部主事,遷禮部郎中。出為山東提學道僉事,報滿,左都御史魏象樞特疏薦之,遷貴州糧驛道參議。師方下雲南,羽書旁午,之辨安設驛馬以利塘報;復以軍米運自湖南,苦累夫役,白大府停運,就地採購,供億無匱。二十四年,擢通政使參議,遷兵部督捕理事官。連遭親喪。服闋,起故官。洊擢左副都御史,數有建白。

四十七年,皇太子允礽既廢,上日夕憂懣。既,有復儲意,王大臣合疏保奏,命留中。旋諭廷臣:「俟廢太子疾瘳,教養有成,朕自有旨,諸王大臣不得多瀆。」十二月,之辨密奏曰:「皇上之於皇太子,分則君臣,親則父子。皇太子初以疾獲戾,今疾已平復。孝友之本懷,固由至性;肅雍之儀表,久系群心。乞速渙新綸,收回成詔,敕部擇吉早正東宮,布告中外,俾天下曉然知聖人舉動,仁至義盡,大公無私。事莫有重於此者。今八荒清晏,一統車書,值星紀初,光華復旦,七廟將行大祫,萬國於以朝正。皇上以孝慈治天下,方且稱壽母萬年之觴,集麟趾繁昌之慶;而顧使前星虛位,震子未寧,聖心得無有遺憾乎?臣年已七十,報主之日無多,知無不言,統望乾斷速行。自此以往,皇上待皇太子與諸皇子,尤原均之以恩,範之以禮,則宜君宜王之美,不難上媲成周,遠超百代。至萬不得已而裁之以法,則非臣之所敢言也。」疏入,上不懌,斥為奸詭,命奪官,逮赴刑部笞四十,逐回原籍。

五十二年,赴京祝萬壽,復原秩。逾年,卒於家。

硃天保,字九如,滿洲鑲白旗人,兵部侍郎硃都訥子。康熙五十二年進士,選庶吉士,授檢討。五十六年,典山東鄉試。

五十七年正月,疏請復立二阿哥允礽為皇太子。時允礽廢已久,儲位未定,貝勒允禩覬得立,揆敘、王鴻緒等左右之,欲陰害允礽。硃天保憂之,具疏上,略曰:「二阿哥雖以疾廢,然其過失良由習於驕抗,左右小人誘導之故。若遣碩儒名臣為之羽翼,左右佞幸盡皆罷斥,則潛德日彰,猶可復問安侍膳之歡。儲位重大,未可移置如釭,恐有籓臣傍為覬覦,則天家骨肉之禍,有不可勝言者。」疏成,以父在,慮同禍,徘徊未即上。硃都訥察其情,趣之入告。時上方幸湯山,硃天保早出德勝門,群鴉阻馬前,硃天保揮之去。疏上,上欷歔久之。阿靈阿,允禩黨也,媒孽之曰:「硃天保為異日希寵地。」上怒,於行宮御門召問曰:「爾雲二阿哥仁孝,何由知之?」硃天保以聞父語對。上曰:「爾父在官時,二阿哥本無疾,學問弓馬皆可觀。後得瘋疾,舉動乖張,嘗立朕前辱罵徐元夢。於伯叔之子往往以不可道之言肆詈,爾知之乎?爾又云二阿哥聖而益聖,賢而益賢,爾從何而知?」硃天保亦以父聞之守者對。詰其姓名,不能答。上曰:「朕以爾陳奏此大事,遣人傳問,或將爾言遺漏,故親訊爾。爾無知稚子,數語即窮,必有同謀者。」硃天保對父與婿戴保同謀,遂逮硃都訥、戴保。

上復御門召問曰:「二阿哥因病拘禁,朕猶望其痊愈,故復釋放,父子相見。教訓不悛,始復拘禁。二阿哥以礬水作書與普奇,屬其保舉為大將軍,並謂齊世、札拉克圖皆當為將軍。朕遣內侍往詢,自承為親筆。此事爾知之否?」硃都訥自稱妄奏,應萬死。上曰:「爾奏引戾太子為比。戾太子父子間隔,朕於二阿哥常遣內監往視,賜食賜物。今二阿哥顏貌豐滿,其子七八人,朕常留養宮中,何得比戾太子?爾又稱二阿哥為費揚古陷害。費揚古乃功臣,病篤時,朕親臨視,沒後遣二阿哥往奠。爾何得妄言?爾希僥幸取大富貴,以朕有疾,必不親訊。今爾始知當死乎?」辭連硃都訥婿常賚及金寶、齊世、萃泰等,並逮訊議罪。硃天保、戴保皆坐斬。硃都訥與常賚、金寶皆免死荷校,齊世拘禁,萃泰奪官。

陶彞,順天大興人。康熙三十九年進士,授戶部主事。再遷郎中。考選廣西道御史,巡視兩浙鹽政。

六十年三月,彞與同官任坪、範長發、鄒圖雲、陳嘉猷、王允晉、李允符、範允、高玢、高怡、趙成麃、孫紹曾合疏奏曰:「皇上深恩厚德,浹洽人心。茲逢六十年,景運方新,普天率土,歡欣鼓舞,而建儲一事,尤為鉅典。懇獨斷宸衷,早定儲位。」疏入,下內閣。時大學士王掞正密疏請建儲。後數日,彞等疏又上,上震怒,斥掞植黨希榮。於是王大臣奏請奪掞及諸御史官,從重治罪。越日,諭廷臣曰:「王掞及御史陶彞等妄行陳奏,俱稱為國為君。今西陲用兵,為人臣者,正宜滅此朝食。可暫緩議罰,如八旗滿洲文官例,俱委署額外章京,遣往軍前效力贖罪。」雍正四年,世宗以諸御史不諳國體,心本無他,詔釋歸,以原職休致還籍。

坪,字坦公,山東高密人。康熙三十年進士。自刑部郎中考選山西道御史,轉掌陜西道。赴軍,駐忒斯河。大漠荒寒,盛夏冰雪,坪處之怡然。及歸,閉戶讀書,終老於家。

長發,字廷舒,浙江秀水人。康熙三十三年進士,授南城知縣。行取禮部主事,考選廣西道御史,轉掌浙江道。遣戍,予額外主事銜,隨都統圖臘赴征西將軍營。還,駐歸化城。後命赴察漢新臺。歸,以原職休致。

圖云,字偉南,江西南城人。康熙三十六年進士,授大竹知縣。行取禮部主事,考選河南道御史,轉掌山東道,巡視東城。

嘉猷,字訒叔,江南溧陽人。康熙三十九年進士。自吏部員外郎考選山西道御史。五十六年,王掞密請建儲。未幾,嘉猷與同官八人亦合疏陳請,上疑之,掞幾獲罪,事具掞傳。至是,嘉猷復與彞等申請,獲咎。

允晉,直隸清苑人。康熙四十五年進士。自戶部員外郎考選陜西道御史。

允符,字揆山,浙江嘉善人。康熙二十六年舉人,授什邡知縣。行取江西道御史。

允,字用賓,浙江錢塘人。康熙三十九年進士,授安平知縣。行取工部主事,考選山東道御史。

玢,字荊襄,河南柘城人。康熙二十七年進士。自禮部郎中考選廣東道御史,巡視東城。謫戍忒斯軍營,運糧西藏。居塞上六年,著出塞集,備言屯戍之苦。釋歸,終於家。

怡,字仲友,浙江武康人。康熙二十七年進士,授長洲知縣。善聽訟,吏胥憚之。尚書韓菼,怡師也,其姻黨系獄,以菼故請恕,怡怒杖之。選鄜州知州,行取工部主事。考選山東道御史。謫戍時,年逾六十。以原職釋歸。

成麃,字德培,江南吳縣人。康熙四十七年舉人,授內閣中書。累遷兵部郎中,考選福建道御史。

紹曾,字二乾,浙江山陰人。康熙二十五年舉人,授開縣知縣。行取戶部主事,授四川道御史。赴軍,駐歸化城,地當孔道。故事,徭役供張,取給於戍員。紹曾清介無餘資,困甚。迨釋還,卒於途。又有邵璿,亦以疏請建儲獲罪。

璿,字璣亭,江南無錫人。自拔貢生授芮城知縣。行取工部主事,授江南道御史,掌登聞院,巡視北城。六十年,遣戍軍前。時同謫者十三人,圖雲、允符、成麃、璿皆死於塞外,而給事中劉堂,御史柴謙、吳鎬、程鑣續以言事謫,同時釋還,仍為十三人,世稱「十三言官」。堂,彭澤人。謙,仁和人。鎬,漢陽人。鑣,錢塘人。

論曰:理密親王在儲位久,未聞顯有失德,而終遭廢黜,聖祖手詔,若有深痛鉅慝至不可言者。夫以聖祖之仁明,而不克全監撫之重,終父子之恩,讒人罔極,靡所不至,甚矣!掞力主復故,聖祖雖深罪之,固諒其無他心。勞之辨諫於初廢,大臣拜杖,已非故事;硃天保爭於再黜,遂以誅死,罪及其親。一則但責其沽名,一則深疑其受指,故譴有重輕歟?彞等但坐謫戍,已為寬典,拳拳效忠,固人臣之義也。


\end{pinyinscope}