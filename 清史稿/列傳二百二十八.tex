\article{列傳二百二十八}

\begin{pinyinscope}
潘祖廕李文田孫詒經夏同善張家驤

張英麟張仁黼張亨嘉

潘祖廕,字伯寅,江蘇吳縣人,大學士世恩孫。咸豐二年一甲三名進士,授編修。遷侍讀,入直南書房,充日講起居注官。累遷侍讀學士,除大理寺少卿。左宗棠被劾,召對簿,罪不測,祖廕上疏營救,且密薦其能,獄解,乃起獨領一軍。十一年,詔求直言,祖廕念車駕還都,首斥奸佞,綱紀一新,為上勤聖學、求人才、整軍務、裕倉儲四事。並請免賦以蘇民困,汰釐以紓民力,嚴軍律以拯民生,廣中額以收民心。纚纚數千言,稱旨。遷光祿寺卿。與修治平寶鑒,書成,被賞賚。先後糾彈官吏不職狀,書凡數上,文若欽差勝保、直隸總督文煜、陜西巡撫英棨、布政使毛震壽、甘肅布政使恩麟、道員田在田諸人;武若提督孔廣順、總兵閻丕敘、副將張維義諸人。繇是直聲震朝端。

同治三年,授左副都御史。坐會議何桂清罪未列銜,絓吏議。明年,恭親王奕獲譴,下群臣議。祖廕念重臣進退,關系安危,疏請持平用中,酌予轉圜,袪世人惑。補工部侍郎。七年,調戶部,充經筵講官。坐失部印,褫職留任。典順天鄉試,再坐中式舉人徐景春文理荒謬,鐫二級。十三年,特旨賞編修,仍入直。錄輸餉功,釋處分。

光緒改元,授大理寺卿,補禮部右侍郎。數遷工部尚書,加太子少保。五年,主事吳可讀以死請為穆宗立嗣,祖廕被命集議,與徐桐等請申不建儲,彞訓疏存毓慶宮。明年,偕惇親王奕脤等辦中俄交涉。約既成,籌善後,條列練兵、簡器、開礦、備餉四事進。命入直軍機,父憂歸。服闋,起權兵部尚書,調補工部,兼管順天府尹事。大婚禮成,晉太子太保。十六年,卒,贈太子太傅,謚文勤。寶坻士紳感其救災勤勞,籥建專祠,報可。

祖廕嗜學,通經史,好收藏,儲金石甚富。先後數掌文衡,典會試二、鄉試三,所得多真士。時與翁同龢並稱翁潘云。

李文田,字芍農,廣東順德人。咸豐九年一甲三名進士,授編修。入直南書房,充日講起居注官。同治五年,大考,晉中允。九年,督江西學政。累遷侍讀學士。秩滿,其母年已七十有七矣,將乞終養,會聞朝廷議修園籞,遂入都覆命。既至,謁軍機大臣寶鋆,告以東南事可危,李光昭奸猥無行,責其不能匡救。寶鋆曰:「居南齋亦可言,奚必責樞府?」文田日:「正為是來耳!」疏上,不報。逾歲,上停止園工封事,略言:「巴夏禮等焚毀圓明園,其人尚存。昔既焚之而不懼,安能禁其後之不復為?常人之家偶被盜劫,猶必固其門墻,慎其管鑰,未聞有揮金言誇富於盜前者。今彗星見,天象譴告,而猶忍而出此,此必內府諸臣及左右憸人導皇上以朘削窮民之舉。使朘削而果無他患,則唐至元、明將至今存,大清何以有天下乎?皇上亦思圓明園之所以興乎?其時高宗西北拓地數千里,東西諸國讋憚天威,府庫充盈,物力豐盛,園工取之內帑而民不知,故皆樂園之成。今皆反是,聖明在上,此不待思而決者矣。」疏入,上為動容。俄乞假歸。光緒八年,遭母憂。服竟,起故官,入直如故。數遷至禮部侍郎,充經筵講官,領閣事。二十年,疏請起用恭親王奕及前布政使游智開,依行。明年,卒,恤如制,謚文誠。

文田學識淹通,述作有體,尤諳究西北輿地。屢典試事,類能識拔績學,士皆稱之。

孫詒經,字子授,浙江錢塘人。咸豐十年進士,選庶吉士。聞杭州城陷,乞假歸,奉親闢居定海。參寧紹臺道張景渠軍,平浙東有功,還授檢討。以倭仁薦,入直南書房。同治四年,擢司業。上言:「弭災在恤刑,治獄先平法。本律盜案不分首從,聖祖、世宗加以區別。自頃盜風充斥,概用重典,行十餘年,案不減少。則知弭盜之術,不在用法之嚴。請敕刑部改成例,復祖制。」議行。會上將侍太后幸惇親王府,既,與夏同善諫罷。未幾,復將詣恭親王府祀神,詒經再上疏,言:「聖學方新,宸修宜懋。經帷屢曠,則神志難專;法駕時勤,則見聞易惑。一日行幸,一日已荒念典之功;今日行禮,異日或啟游觀之漸。」士論歸之。遭父憂去,服除,仍原官,入直如故。十年,遷侍講。五月朔,日食。詒經以天道感應,本諸人事,於是有遇災修省之請。十三年夏,彗星見,越數日,太白經天,人心惶駭。詒經復有廣開言路及罷圓明園工程之請。遷侍讀學士。德宗纘業,大考一等,擢詹事。召對,命直抒所見,連上澄吏治、慎海防機宜甚悉。

光緒六年,俄釁啟,東西海陸邊防亟。詒經言:「能戰然後能和,兵力專顧海口,北塘覆轍可鑒。」請調勁旅守東路,並津、永舉辦民團。再遷刑部侍郎,明年,調戶部。會左宗棠請修畿輔水利,乃疏薦張之洞、張佩綸資治理,並以山東河患,河員專治河堤,不講修導,建議購泰西機船及時修濬。十一年,入直毓慶宮。山東河工領部銀百萬,詒經廉得書吏史恩濤苛索狀,嚴責繳還,將懲治,章未上,而御史王賡榮等輒劾以輕縱。上令明白回奏,覆奏入,卒陷吏議,並罷直。有勸引退者,詒經曰:「吾被恩遇久,遑敢佚吾身邪?」於是專治部事,佐度支凡十年。時議設銀行,造鐵路,慮利權外溢,齗齗持異議。

詒經持躬清正,思以儒術救時敝。不阿權要,為同列所忌,卒不得行其志。先後數司文柄,深惡末學骫骳積習,擯之惟恐不遑,所得多知名士。生平論學不分漢、宋,謂經學即理學。又曰:「學所以厲行也,博學而薄行,學奚足尚?」一時為學者所宗。十六年,卒,優詔賜恤,謚文愨。

夏同善,字子松,浙江仁和人。咸豐六年進士,選庶吉士,授編修。累遷右庶子,充日講起居注官。十年,粵寇陷江南,諸軍無所統,請屬之曾國籓;又以北塘之役,僧格林沁軍退頓通州,桂良再就議款,同善建言敵情叵測,宜專任僧格林沁備戰守:敕並依行。父憂歸,服闋,起故官。同治六年,遷少詹事。其時傳言車駕將幸惇親王府,召集梨園,同善聞之,與孫詒經合疏諫止。略言:「皇上沖齡,敬天未至南郊,游幸先臨府第,未安者一。聖學端資養正,耳目玩好偶有所娛,恐疏而不密,未安者二。近頃軍事未寧,游觀之事傳播四方,曷以慰臣民望?未安者三。英、俄人士雜處京畿,稍示以懈,何能帖伏?未安者四。夫孝以禮為歸,禮以時為大,非時不舉,古有明箴。乞罷止以彰聖德。」出督江蘇學政,遭繼母喪去職。起詹事。十年,遷兵部右侍郎。秋,患霪雨,奉其狀以上,乞申虔禱,實行敦節儉、廣賑濟、開言路、清庶獄諸政,語至剴切。十三年,偕尚書廣壽詣四川按事,奏請撤永川等兵差局、綿竹等伕馬局。

光緒元年,命直毓慶宮授讀,固辭不獲,益屏家事勿問,退唯默坐觀書,思所以為獻納地。先後累言盜案刑例宜復舊制,分首從;畿輔旱,請鑿井灌田蘇之;晉、豫饑,請移海防關稅經費恤之。四年,復命視學江蘇,陛辭日,力陳捐納有礙民生,無裨國用,稱旨。明年,被命巡視山東黃河,條上治下游三事:曰濬海口,曰直河灣,曰通支河,請移機器局經費治之。其秋,閱緣江砲臺,又歷陳三不可恃,請合數省力助守江口,已築者毋廢,未築者毋增,上然其言。嘗割俸濬江陰城河,植松五萬餘於君山,民德之。六年,卒,德宗聞之遽泣,其忠誠荷主知如此。遺疏入,賜恤如例,謚文敬。子庚復,主事;敦復,御史。

張家驤,字子騰,浙江鄞縣人。同治元年進士,選庶吉士,授編修。督山東學政,調山西。遭父憂解職,服除,起故官。遷侍講,入直南書房。光緒元年,轉侍讀,充日講起居注官。五年,命直毓慶宮,遷侍講學士。明年,劉銘傳奉召入都,疏請籌造清江浦鐵路,下李鴻章等議。家驤念典學方新,講求上理,萬一言利之臣隨聲附和,一言僨事,關系匪輕,乃力陳三弊阻止之。疏入,仍令鴻章覈覆,鴻章力主銘傳策。然自是御史洪良品陳五害,侍講張楷陳九不利,並隨家驤而上諫書矣,事竟寢。數遷內閣學士,充經筵講官。九年,授工部右侍郎,調吏部。

家驤純謹好學,一謝時趨。蒞官端慎。授帝讀,朝夕納誨,頗能盡心所職。十年,卒,上悼惜,賜祭葬如制,謚文莊。

張英麟,字振卿,山東歷城人。同治四年進士,選庶吉士,授編修。十三年,命偕檢討王慶祺在弘德殿行走。英麟甫入直,即乞假歸省。未幾,穆宗崩,慶祺以有罪褫職。眾皆稱其志節。歷典福建、雲南鄉試,累遷祭酒,充經筵講官。光緒十七年,以詹事授奉天府丞,兼學政。奉省士民樸素,隨軺所至,力加獎勸,學風興起。晉內閣學士,簡順天學政,擢吏部侍郎。二十六年,通州試竣回京,兩宮西狩,官吏遷避,英麟獨守學政關防待交替。明年,召赴行在,應詔上疏,請力崇節儉。乘輿回鑾,議變法,英麟言祖宗法制,可整飭不可遽更張。二十九年,充會試副總裁,借闈河南,改試策論、經義。英麟嚴衡校,多取績學。會改官制,英麟以侍郎遷副都統,漢員授旗官自此始。旋晉都統。三十四年,授都御史。時議行憲政,許士民上書,英麟必詳審為代達。御史江春霖直劾親貴,斥回原衙門,英麟率全臺合疏留之。

宣統改元,攝政監國,復舉輪講之典。英麟撰資治通鑒講章以進,皆發明精義,比附近情,冀以誠意相感動,章上,但循故事留覽而已。三年,武昌變起,內閣改制,飭都察院及凡有言責者皆停奏事,英麟嘆息以為奇變。遜位詔下,遂乞罷歸。德宗永遠奉安,猶奔赴崇陵謁送。重宴瓊林,加太子太保。乙丑冬,卒,年八十有八。

張仁黼,字劭予,河南固始人。光緒二年進士,選庶吉士,授編修,入直上書房。出督湖北學政,以硃子小學、近思錄訓士。累遷洗馬,充日講起居注官,補侍講。二十年,日本釁起,樞臣被劾。乃與李文田等請起用恭親王奕,稱旨。遷鴻臚寺卿,典試四川。除奉天府府丞,父憂,未之官。

二十六年,拳亂作,奉命在籍治團練。服闋,赴行在。時財匱,議加丁口稅。仁黼謂:「今日國勢極危,而人心未去者,良由世祖除明季三餉;聖祖詔丁口以五十年為率,嗣後滋生永不加賦:深仁厚澤,民不能忘。今議加丁稅,違祖制,拂民情,必不可。」事遂寢。還京,擢順天府府尹。再遷兵部侍郎,典試江西,歷學部、法部。

三十三年,補大理院正卿,奏請敕部院大臣會訂法律,略言:「法律主要在乎組織立法機關,而所以成之者有三,曰:定法律宗旨,辨法律性質,編法律成典。中國數千年來,禮陶樂淑,人人皆知尊君親上。此乃國粹所在,必宜保存,用各國之法以補其不足。尤須造就法律人才,治法治人,相因為用,然後可收實效。」又言:「立法之要,規模不可不閎,推行必宜有漸。否則未當於人心而貿然以試,誠恐外國屬人主義勢力日益擴張,而吾國屬地主義處理愈形棼糾。有司奉行不善,反使外人得以藉口,為患甚大。」疏入,多議行。俄授吏部侍郎,充經筵講官。三十四年,丁母憂。未幾,卒。

仁黼內行修,不自標異。嘗被命治河,卻例饋節省金,同官懼,謂將興大獄。仁黼忽索取金,眾始安,然頗怪其失操。已而河南巡撫上言紳士助學校金,不受獎敘,數與之同。朝士益服其清不絕物云。

張亨嘉,字燮鈞,福建侯官人。光緒九年進士,選庶吉士,授編修。十四年,視學湖南,念儒官為士模範,不激濁揚清,曷以勵風教?疏薦文行交修者數人,士習為一變。二十三年,入直南書房。越二年,除司業,頻轉太常寺少卿。一歲五遷,殊數也。

二十六年夏,親貴大臣信拳民有神術能攘外,飾詞入告,上疑之,命亨嘉察視。亨嘉知其不可恃,條上弭釁機宜甚悉,疏甫入而亂作。西狩還,獨先賜用,徙大理寺卿。明年,出督浙江學政,頗採西國政教命題試士,多得通材。尚書張百熙、榮慶既為學務大臣,別置大學總監督,亨嘉遂被命任校事,仍不離內廷職。大學中更寇亂,肄業生不盈百,乃闢學舍,廣集高材生。類別學科,禮聘儒宿及東西邦學人專門教授。書籍儀器,粲然具備。兼攝進士館監督,進士習法政自此始。歷光祿寺卿、左副都御史、兵部侍郎。逾歲,疏辭校職,轉禮部侍郎,充經筵講官。

亨嘉為人敦實,嗜古精鑒賞。事母孝,母黃氏,壽百歲,同列奏庥瑞。中興後命婦享高耄者,與詹事袁葆恆祖母郭氏二人而已。上聞之嘆異,加恩賜予。三十四年,遭喪去,終服,仍入直。宣統二年,卒,賜祭葬,謚文厚。

論曰:同、光典學內直諸臣,每兼授讀,體制較隆;而文學侍從,亦多選績學,時備顧問,稱榮幸焉。祖廕好賢勤事,文田學識淹雅,同以通博稱。詒經重實學,同善崇聖德,家驤盡心誨納,英麟早勵風節,並無愧師儒。仁黼、亨嘉尤惓惓於明法修學,後先相望,其風採皆隱然可見焉。


\end{pinyinscope}