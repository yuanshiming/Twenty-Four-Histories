\article{志九十六}

\begin{pinyinscope}
○食貨二

△賦役倉庫

賦役一曰賦則。清初入關,首除明季加派三餉。時賦稅圖籍多為流寇所毀。順治三年,諭戶部稽覈錢糧原額,匯為賦役全書,悉復明萬歷間之舊。計天下財賦,惟江南、浙江、江西為重,三省中尤以蘇、松、嘉、湖諸府為最。六年,戶科右給事中董篤行請頒行易知由單。八年,世祖親政,分命御史巡行各省,察民間利病。蘇松巡按秦世楨條奏八事:曰,田地令業主自丈,明註印冊;曰,額定錢糧,俱填易知由單,設有增減,另給小單,以免奸胥藉口;曰,由單詳開總散數目,花戶姓名,以便磨對;曰,設立滾單,以次追比;曰,收糧聽裏戶自納簿櫃,加鈐司府印信;曰,解放先急後緩,勒限掣銷;曰,民差查田均派,與排門冊對驗;曰,備用銀兩,不得額外透支,徵解銀冊,布政司按季提取,年終報部。自後錢糧積弊,釐剔漸清。

十一年,命右侍郎王宏祚訂正賦役全書,先列地丁原額,次荒亡,次實徵,次起運存留。起運分別部寺倉口,存留詳列款項細數。其新墾地畝,招徠人丁,續入冊尾。每州縣發二本,一存有司,一存學宮。賦稅冊籍,有丈量冊,又稱魚鱗冊,詳載上中下田則。有黃冊,歲記戶口登耗,與賦役全書相表裏。有赤歷,令百姓自登納數,上之布政司,歲終磨對。有會計冊,備載州縣正項本折錢糧,註明解部年月。復採用明萬歷一條鞭法。一條鞭者,以府、州、縣一歲中夏稅秋糧存留起運之額,均徭里甲土貢雇募加銀之額,通為一條,總徵而均支之。至運輸給募,皆官為支撥,而民不與焉。頒易知由單於各花戶。由單之式,每州縣開列上中下則,正雜本折錢糧,末綴總數,於開徵一月前頒之。又佐以截票、印簿、循環簿及糧冊、奏銷冊。截票者,列地丁錢糧實數,分為十限,月完一分,完則截之,鈐印於票面,就印字中分,官民各執其半,即所謂串票也。印簿者,由布政司頒發,令州縣納戶親填入簿,季冬繳司報部。循環簿者,照賦役全書款項,以緩急判其先後,按月循環徵收。糧冊者,造各區納戶花名細數,與一甲總額相符。奏銷冊者,合通省錢糧完欠支解存留之款,匯造清冊,歲終報部核銷。定制可謂周且悉矣。

十五年,江西御史許之漸言:「財賦大害,莫如蠹役,官以參罰去,而此蠹役盤踞如故。請飭撫按清查,甚者處以極刑,庶積弊可冀廓清。」工科給事中史彪古請嚴禁正供外加派,並將申飭私派之旨刊入易知由單,俾民共曉。帝以所奏皆切中時弊,下所司詳議以聞。

聖祖即位,嚴申州縣官隱匿地畝、不納錢糧、捏報新墾之禁,更定州縣催徵議敘經徵督催各官處分。其州縣官挪用正款、捏稱民欠,及加派私徵者,罪之。帝以由單款項繁多,民不易曉,命將上中下等則地每畝應徵銀米實數列單內;由單報部,違限八月者,罪州縣衛所及轉報官。給事中姚文然上言:「災荒蠲免,有收完在前奉令在後者,以本年應蠲錢糧抵次年應納正賦,名曰流抵,自應載入由單,俾人沾實惠。但部題定額由單,於上年十一月頒發州縣,磨算編造,必在九十月間,而各省題報災傷,夏災以六月,秋災以九月,部中行查覆奏,咨行撫臣,飭知地方官吏,展轉需時,計已在頒發由單之後,其勢無由填入。應請於流抵之下年填入由單,以杜其弊。」下部議行。

直省徵收錢糧,夏稅於五六月,秋糧於九十月,其報部之數,責成各司於奏銷時詳加磨勘,按年送京畿道刷卷。自世祖定賦稅之制,正雜款繁多,咨題違錯,駁令查覆,印官即借部駁之名,擅行私派;其正賦錢糧本有定額,地方官吏遇有別項需用,輒令設法,實與加派無二。至是下令嚴禁,罷州縣欠糧、留任候代、完全開復之制。七年,以夏稅秋糧定限稍遲,恐誤協餉,仍復舊制,州縣開徵後,隨收隨解。凡各省地丁錢糧,巡撫於歲終奏銷,詳列通省錢糧起運存留、撥充兵餉、辦買顏料及餘賸之數,造冊具報。其黃冊、會計冊繁費無益,悉罷之。十五年,嚴定官民隱田罪例。官吏查出隱田,分別議敘。人民舉首隱地逾十頃者,即以其地與之。

十八年,令州縣每歲將日收錢糧流水簿解司磨對,罷赤歷。自順治間訂正賦役全書,至是二十餘年,戶口土田,視昔有加,按戶增徭,因地加賦,條目紛繁,易於淆混。二十四年,下令重修,止載起運存留漕項河工等切要款目,刪去絲秒以下尾數,名曰簡明賦役全書。二十六年書成。廷議以舊書遵行已久,歷年增減地丁銀米,俱有奏銷冊籍可稽,新書遂罷頒行。是歲諭各省悉免刊刻由單,以杜派費擾民之弊。

二十八年,令各省巡撫於每年奏銷時,盤查司庫錢糧。先是各州縣催徵用二聯串票,官民分執,不肖有司句結奸胥,以已完作未完,多徵作少徵,弊竇日滋。至是議行三聯串票,一存有司,一付役應比,一付民執照。其後更刊四聯串票,一送府,一存根,一給花戶,一於完糧時令花戶別投一櫃以銷欠。未幾,仍復三聯串票之制。各省紳衿本有優免丁銀之例,而豪強土著,往往詭寄濫免,更有紳衿包攬錢糧耗羨,盡入私橐,官民交累。有詔,詭寄地畝,悉退還業戶。三十年,以由單既停,令直省州縣衛所照賦役全書科則輸納數目,勒石署門外。復諭民間隱匿地畝,限兩年內自首,尋又展限兩年。諭福建清丈沿海地畝,釐定疆界,湖南幅員遼闊,先飭民人自行丈量,官府再事抽丈,隱漏者罪之。

時徵收錢糧,官吏往往私行科派,其名不一。闔邑通里共攤同出者,名曰軟抬,各里各甲輪流獨當者,名曰硬駝,於是設滾單以杜其弊。其法於每里之中,或五戶或十戶一單,於某名下註明田地若干、銀米若干、春秋應各完若干,分為十限,發與甲首,依次滾催,自封投櫃。一限既定,二限又依次滾催,其有停擱不完不繳者嚴懲,民以為便。浙江、湖北、山東諸省匠班銀,均歸入地丁徵收。四十五年,九江府丈出濱江蘆洲地畝三千餘頃,均按下則起科。

五十一年,四川巡撫年羹堯上言:「四川錢糧原額百六十一萬兩有奇,現僅徵及十分之一,宜立勸懲法,五年內增及原額之四五者準升,不及二分停升,不及一分降調,無增者褫其職。」御史段曦上疏駁之,略言:「川省自經明季兵燹,地廣人稀。我朝勘定之後,雖疊次清查,增報僅及原額十分之一。近日撫臣加意催查,增至二萬六千餘兩。今欲五年內增及原額十之二或十之四五,是增現糧三四倍也。賢能之吏,必罹不及分數之參處,不肖者抑勒首報,滋擾無窮。請川省隱漏錢糧,徹底清查,不必另立勸懲之法。」從之。五十九年,諭:「嗣後各州縣錢糧,隨徵隨解。若州縣批解後,而布政司抵充雜派,扣批不發,許州縣逕申督撫。」次年,又令各督撫將倉糧虧空,限三年補完。

聖祖在位六十年,政事務為寬大。不肖官吏,恆恃包荒,任意虧欠,上官亦曲相容隱,勒限追補,視為故事。世宗在儲宮時,即深悉其弊。即位後,諭戶部、工部,嗣後奏銷錢糧米石物價工料,必詳查覈實,造冊具奏。以少作多、以賤作貴、數目不符、覈估不實者,治罪。並令各督撫嚴行稽查所屬虧空錢糧,限三年補足,毋得藉端掩飾,苛派民間。限滿不完,從重治罪。瀕江沿海地,定例十年一清丈。雍正元年,諭令隨時清查,坍者豁免,漲者升科。

二年,以山西巡撫諾敏、布政使高成齡請提解火耗歸公,分給官吏養廉及其他公用。火耗者,加於錢糧正額之外。蓋因本色折銀,鎔銷不無折耗,而解送往返,在在需費,州縣徵收,不得不稍取盈以補折耗之數,重者數錢,輕者錢餘。行之既久,州縣重斂於民,上司苛索州縣,一遇公事,加派私徵,名色繁多,又不止於重耗而已。康熙季年,陜甘總督年羹堯請酌留秦省火耗充各官用度,餘者捐出彌補虧空,聖祖不許。至是諾敏等復以為言。詔從其請。諾敏又請限定分數。帝以「酌定分數,則將來竟成定例,必致有增無減。今耗羨與正項同解,州縣皆知重耗無利於己,孰肯加徵?若將應得之數扣存,勢必額外取盈,浮於應得之數」。於是定為官給養廉之制。河南巡撫石文焯請將捐穀耗羨充公,帝曰:「耗羨存庫,所以備地方公用也。國家經費,自有常額,豈可以耗羨牽入正項,致滋另取挪移諸弊乎!」又諭戶部曰:「州縣虧空錢糧,有闔屬百姓代償者,名曰樂捐,實無異強派,應飭禁止。」

蘇、松浮糧多於他省,詔蠲免蘇州額徵銀三十萬,松江十五萬,永著為例。江蘇巡撫張楷疏言:「江蘇每年額賦,除蠲免浮糧外,應實徵銀三百五十萬有奇。歷年積欠八百八十一萬有奇,計已達千二百餘萬。竭小民一歲所獲,勢難全完。現籌徵收之法,本年新糧,責令全完,舊欠勻作十分,自明年始,年徵其一,十年而畢,每歲奏銷時,另冊造報。嘉定一縣積欠至百四十餘萬,請勻作十五分分徵,上海、昆山、常熟、華亭、宜興、吳江、武進、婁、長洲九縣皆積至四十萬,應勻作十二分分徵,以紓民力。」帝深納之。

各省中賦稅繁重,蘇、松而外,以浙江嘉、湖二府為最。五年,詔減十之一,共銀八萬餘兩。又命浙省南、秋等米,每年額徵作十分覈算,別為一本題銷,如完解不全,罪承督各官。各省錢糧完欠細數,官吏多不宣示,胥吏因緣為奸,虧空拖欠,視為故常。詔各督、撫、布政飭州縣官每年將各鄉里完欠之數,呈送覆覈,張貼本里,俾民周知。如有中飽,許人民執串票具控。其分年帶徵之項,亦應將花戶每年應完之數,詳列榜示,俾不得額外溢徵。七年,蠲浙江額賦十之三,共十萬兩。其江蘇逋賦,自壬子年始,侵蝕包攬之項,分十年帶徵。實在民欠之項,分二十年帶徵。本年完納之項若干,次年即依其數蠲免額徵之糧。如額外多完,次年亦按多完之數蠲免。

十一年,安徽巡撫徐本條陳徵糧事宜:一,州縣徵收糧櫃,請逕用州縣封條;二,花戶完糧,宜仍用三聯串票;三,小民零星錢糧,一錢以下者,許其變通完納制錢。許之。十二年,修賦役全書。凡額徵地丁錢糧商牙課稅內,應支官役俸工驛站料價,以及應解本折絹布顏料銀硃銅錫茶蠟等項,分晰原額新徵總散之數,務為精覈。自後十年修輯一次。

江南、湖廣等省,蘆洲坍漲靡定,定制五年一清丈,不肖官吏,恆藉以納賄舞弊。乾隆元年,下詔清查。又禁各省虛報開墾。大學士硃軾請禁民間田地丈量首報。御史蔣炳奏州縣徵糧三弊:一,田畝科則不同,請每年照部頒定額,覈明刊示;一,州縣拆封如有短平,即於袋面註明數目,令花戶自行補交;一,州縣設立官匠,傾銷銀兩,勒索包完,侵漁重利,嗣後準花戶隨處傾銷,官匠永行禁革。皆從之。諭改減江南、浙江白糧十二萬石,免蘇、松浮糧額銀二十萬石。

自山西提解火耗後,各直省次第舉行。其後又酌定分數,各省文職養廉二百八十餘萬兩,及各項公費,悉取諸此。及帝即位,廷臣多言其不便。帝亦慮多取累民,臨軒試士,即以此發問,復令廷臣及督撫各抒所見。大學士鄂爾泰、刑部侍郎錢陳群、湖廣總督孫家淦皆言:「耗羨之制,行之已久,徵收有定,官吏不敢多取,計已定之數,與未定以前相較,尚不逮其半,是跡近加賦而實減徵也。且火耗歸公,一切陋習悉皆革除,上官無勒索之弊,州縣無科派之端,小民無重耗之累,法良意美,可以垂諸久遠。」御史趙青藜亦言:「耗羨歸公,裒多益寡,寬一分則受一分之賜。且既存耗羨之名,自不得求多於正額之外,請無庸輕議變更。」惟御史柴潮生以為耗羨乃今日大弊。詔從鄂爾泰諸臣議。先是各省解京餉銀,有隨平陋規。雍正初,曾有詔禁止。嗣因清查部庫虧空二百五十餘萬,怡親王議以京餉平餘彌補,每餉銀千兩,收平餘二十五兩,俱於耗羨內動支起解,較從前陋規減省已多。尋以彌補足額,減收其半。至是停止解部,存儲司庫,以充本省賑濟荒災及裨益民生之舉。自明以來,江南歲額錢糧地丁漕項蘆課雜稅之外,復有所謂雜辦者,款目甚多,匯入地丁分數奏銷。逮編賦役全書,止載應解之款,未列雜辦原委。至是乃妥定章程,以杜浮收,其實在缺額有累官民者豁免之,禁州縣徵糧浮收零尾。

十二年,大學士訥親等議江蘇錢糧拖欠至二百餘萬,不免吏役侵蝕,酌定自首減免之條。復諭黃廷桂等釐剔江蘇催徵諸弊。各省積欠錢糧,歲終奏報,然必待次歲五月奏銷,方能定完欠實數。諭:「嗣後各省每年完欠錢糧,隨奏銷時覈實具奏,毋庸循歲終奏聞之例。」二十二年,免江南乾隆十年以前積欠漕項銀米地價耗羨。江蘇巡撫陳宏謀奏:「江蘇錢糧積年未能歸款,由於州縣案卷,任書承攜貯私室,以致殘缺無由查考,應嚴飭各州縣將卷宗黏連蓋印,妥存署中。至江省用款繁多,州縣不免借墊,嗣後仍令隨時詳請抵兌。逾四月不詳報,數達五百兩以上者,參處;遲至一年,並府州題參。」均如所議行。

三十年,諭:「奏銷冊前列山地田蕩版荒新墾,次列三門九則額徵本折地丁起解存留,至為明晰。令嗣後刊刻賦役全書,以奏銷條款為式,止將十年內新坍新墾者添註,其瑣碎不經名目,概刪除之。」戶部議定各省徵收錢糧,及一切奏銷支放等事。凡銀悉以釐為斷,不及釐者,折衷歸減。米糧以勺為斷,奇零在五秒以上者作為一勺,不及五秒者刪除。搭放俸餉制錢以一文為止,而冊內有絲毫忽微虛數,一並刪除。至各州縣衛所應徵銀兩,統令於由總單數下將奇零歸減,其單內前列細數,仍存其舊,期與賦役全書、魚鱗冊數相符。三十三年,諭直省勛田,令民戶首報,一體輸納。

三十六年,以比歲蠲免天下錢糧,民力饒裕,令各督撫值輪免之年,將緩帶款項,務催徵完納,毋致次年有新舊同徵之累。四十七年,御史鄭澂請令督撫清查倉庫,如有虧缺,本員治罪償補,督撫從重議處,並加倍分賠。仍令各州縣將倉庫實貯之數,三月匯報,督撫隨時督覈。山東州縣恆多虧挪倉庫之弊,並有本無虧短,於離任時假捏虧數,私立欠約,移交後任,以為肥橐之計者。請飭下各督撫,查有前任虧缺、後任有欠約可憑者,除責成後任彌補外,仍令前任照數追繳入官,以杜短交濫接之弊。帝嘉納之。嘉慶初,復令各督撫於地方官交代,如限內未能交清,應將該員截留,俟款項交清,方準赴任回籍,並禁止私立議單。自是以後,禁網益密矣。御史彭希洛奏各省錢糧多有浮收之弊。諭嗣後各督撫務於開徵前,按時價覈實換銀上庫之數,榜示通衢,納銀折錢,聽民自便。

時各省地方官吏,於應徵錢糧,往往挪移新舊,以徵作欠,自三四年以來,積欠至兩千餘萬。有詔將各省歷年積欠,在民在官,一體清查,或留貯,或撥解,違者罪之。戶部奏:「近五年各省耗羨盈餘內借款,請責成督撫查明補歸原款,並將動支耗羨之款酌量刪減,其各項存貯閒款,並詳列以聞。」直隸清查各屬歷年虧短數達巨萬。安徽倉庫虧缺各項銀百八十餘萬。帝諭新虧各員,自本年始,限四年完繳舊虧。未完者,每年酌扣司道府州縣養廉九五成存庫歸款。部奏直隸等十五省,除緩徵帶徵,其未完地丁餘尚有八百七十餘萬,而十二年分又續增未完地丁銀二百九十餘萬。帝以上官於經徵之員,參限將滿,即設法調署,俾接署者另行起限,州縣藉是規避。令嗣後州縣調署,須先查任內果無應徵未完錢糧,咨部覈明,毋得於參限屆滿時,違例調署。給事中趙佩湘奏:「各省虧空,展轉清查,多致縣宕,請嚴行飭禁。」先是直隸因州縣虧欠倉庫,密令大吏清查,分別追賠。其後各省援例,請立局清查,挪新掩舊,弊竇潛滋,甚有借名彌補,暗肆朘削者,故佩湘以為言。帝諭直隸三次清查案內未完各款,分期勒令歸補,逾限不完者,即責成所管上司攤賠,自後永罷清查,有瀆請者罪之。

十七年,戶部綜計各省積欠錢糧及耗羨雜稅之數,安徽、山東各四百餘萬,江寧、江蘇各二百餘萬,福建、直隸、廣東、浙江、江西、甘肅、河南、陜西、湖南、湖北積欠百餘萬、數十萬、數萬不等。帝以大吏督徵不力,切責之,並令戶部於歲終將各省原欠已完未完各數,詳列以聞。各省逋賦,以江蘇為最多。巡撫硃理奏酌定追補之制,分年補完,杜絕新虧。然屬員掩視拖延如故。直隸自二年至十八年,積欠銀三百四十餘萬,米糧等項十四萬餘石。總督那彥成疏請酌予蠲免,詔嚴行申飭。山東州縣虧欠新舊六百餘萬兩,一縣有虧至六萬餘兩。乃嚴定科條,虧缺萬兩者斬監候,二萬以上者斬決。所虧之數,勒限監追,限內全完貸死,仍永不敘用,逾限不完斬無赦。

御史葉中萬請清釐籓庫借款,胡承珙請整頓直隸虧空諸弊。時各省籓庫,因州縣有急需,往往濫行借款,日久未歸,展轉挪抵,弊混叢生。而攤捐津貼,名目日增,州縣派累繁多,辦事竭蹶,虧欠正項勢所必然,雖嚴刑峻法不能禁也。當乾隆之季,天下承平,庶務充阜,部庫帑項,積至七千餘萬。嘉慶中,川楚用兵,黃河泛濫,大役頻興,費用不貲,而逋賦日增月積,倉庫所儲,亦漸耗矣。

道光二年,御史羅宸條陳直省解徵錢糧,請仿鹽引茶引法,防官吏侵蝕。帝以紛擾,不許。革州縣糧總、庫總,從御史餘文銓請也。乾隆初,州縣徵收錢糧,尚少浮收之弊。其後諸弊叢生,初猶不過就斛面浮收,未幾,遂有折扣之法,每石折耗數升,漸增至五折六折,餘米竟收至二斗五升,小民病之。廷議八折徵收,以為限制浮收之計。大學士湯金釗疏駁之。御史王家相亦言「八折之議,行之常、鎮、江、淮、揚、徐等府,或可嘗試,蘇、松糧重之地,窒礙孔多」。議遂寢。時東南財賦之區,半遭蹂躪。未被兵州縣,又苦貪吏浮收勒折,民怨沸騰,聚眾戕官之事屢起。州縣率以抗糧為詞,藉掩其浮勒之咎。江蘇蘇、松等屬,每遇蠲緩,書吏等輒向業戶索錢,名曰賣荒。納錢者,雖豐收仍得緩徵;不納者,縱荒歉不獲查辦。詔並禁之。湖北漕務積弊已久,巡撫胡林翼疏請折漕革除規費,民間減錢百四十餘萬千文,國帑增銀四十餘萬兩,節省提存銀三十餘萬兩。詔褒美之。

軍興以後,四川等省,辦理借徵,以充兵餉。裕瑞奏請勸諭紳民,按糧津貼,罷借徵。英桂奏:「交納錢糧半銀半錢之制,而官取民仍以銀,每錢二千作銀一兩,耗銀無出。請於應入撥之地丁,準搭官票,不入撥之耗羨,仍徵實銀。」部臣以辦法兩歧,請依原章,正雜錢糧,一體搭交官票。然地方官吏仍收實銀,而以賤值之票交納籓庫,帝令嚴禁。

同治元年,清查直省錢糧。二年,兩江總督曾國籓、江蘇巡撫李鴻章疏言:「蘇、松、太浮賦,上溯之,則比元多三倍,比宋多七倍;旁證之,則比毗連之常州多三倍,比同省之鎮江等府多四五倍,比他省多一二十倍不等。其弊由於沿襲前代官田租額,而賦額遂不平也。國初以來,承平日久,海內殷富,為曠古所罕有,故乾隆中年以後,辦全漕者數十年,無他,民富故也。至道光癸未大水,元氣頓耗,然猶勉強枝梧者十年。逮癸巳大水而後,無歲不荒,無縣不緩,以國家蠲減曠典,遂為年例。部臣職在守法,自宜堅持不減之名,疆臣職在安民,不得不為暗減之術。始行之者,前督臣陶澍、前撫臣林則徐也。又官墊民欠一款,不過移雜墊正,移緩墊急,移新墊舊,移銀墊米,以官中之錢完官中之糧,將來或豁免,或攤賠,同歸無著。故歷年糧冊,必除去墊欠虛數,方得徵收實數。蘇屬全漕百六十萬,厥後遂積漸減損。道光辛卯以後十年,連除官墊民欠,得正額之七八;辛醜以後十年,除墊欠,得正額之五六;咸豐辛亥十年,除墊欠,僅得正額之四成而已。自粵逆竄陷蘇、常,焚燒殺掠,慘不可言。臣親歷新復州縣,市鎮丘墟,人煙寥落。已復如此,未復可知。而欲責以數倍他處之重賦,向來暴徵之吏,亦無骨可敲、無髓可吸矣。細核歷年糧數,咸豐十年中,百萬以上者僅一年,八十萬以上者六年,皆以官墊民欠十餘萬在其中,是最多之年,民完實數不過九十萬也。成案如是,民力如是。惟籥請準減蘇、松、太三屬糧額,以咸豐中較多之七年為準,折衷定數,總期與舊額本經之常、鎮二屬通融覈計,著為定額。即以此後開徵之年為始,永遠遵行,不準再有墊完民欠名目。嗣後非水旱亦不準捏災,俾去無益之空籍,求有著之實徵。至蘇、松漕糧核減後,必以革除大小戶名為清釐浮收之原,以裁減陋規為禁止浮收之委。」制可。先是太常卿潘祖廕、御史丁壽昌交章言減賦事,皆下部議。覆奏準蘇、松減三之一,常、鎮減十之一。大抵蘇、松、太一畝之稅,最重者幾至二斗,輕者猶及一斗。列朝屢議覈減,率為部議所格。雍正間,從怡親王請,免蘇、松兩府額徵銀。乾隆間,又減江蘇省浮糧,皆減銀而不及米。至是詔下,百姓莫不稱慶。

三年,從閩浙總督左宗棠請,諭紹興屬八縣六場,正雜錢糧,統照銀數徵解,革除一切攤捐及陋規,計減浮收錢二十二萬有奇,米三百六十餘石。寧波屬一五縣六場,減浮收錢十萬四千有奇,米八百餘石。四年,浙江巡撫馬新貽請豁減金華浮收錢十五萬餘串,米五百餘石,衢州錢十萬餘串,米六十餘石,嚴州錢六萬餘串,米六千餘石,洋銀八十餘元,米百餘石,從之。是年宗棠克湖州,疏言南漕浮收過多,請痛加裁汰。事下部議。覆奏杭、嘉、湖漕糧,請仿江蘇例,減原額三十分之八,並確查賦則,按輕重量為覈減,所有浮收陋規,悉予裁汰。其南匠米石,無庸議減。計三府原額漕白、行月等米百萬餘石,按三十分之八,共減米二十六萬六千餘石。國籓請將蘇、松等屬地丁漕項一體酌減,不許。

自乾、嘉以來,州縣徵收錢糧,多私行折價,一石有折錢至二十千者。咸豐中,胡林翼始定核收漕糧,每石不得過六千錢。其後山東亦定每石收錢六千。江蘇定每石年內完者收四千五百,年外收五千。江西收錢三千四百。河南每石折銀三兩。安徽二兩二錢。漕糧浮收,其來已久。河運、海運,皆有津貼。嘉興一郡,徵漕一石,有津貼至七錢以上者。又徵收漕糧,例有漕餘,其數多寡不一,大抵視缺分肥瘠為準。歷來本折並收,而折色浮收,較本色更重。自正額減折價定,遂漸少浮收之弊。

直隸、奉天多無糧之地,名曰黑地,或旗產日久迷失,或山隅海涘新墾之田。咸豐季年,寶鋆等查出昌平黑地四百四十餘頃,試辦升科。詔直隸總督、盛京將軍、順天、奉天各府尹一體辦理。同治初,令黑地業戶各赴所管官署呈報升科,許永遠為業。御史陳俊奏:「直隸、奉天除昌平外,呈報升科者寥寥,蓋由地方官吏徵收入己,延不具報,甚有將報地人抑勒刑逼諸弊。」帝遣大臣分查。大學士倭仁疏陳黑地升科,州縣畏難茍安,請申明賞罰。尋定州縣查出隱地逾二十頃優敘,升科地多者獎之;有徇隱匿墾、吏胥詐賕,以溺職論;其無賴假稱委員,恐哧得贓,照例嚴懲。

德宗即位之初,復新疆,籌海防,國用日增。戶部條陳整頓錢糧之策,略云:「溯自發逆之平,垂二十年,正雜錢糧,期可漸復原額。乃考覈正雜賦稅額徵總數,歲計三千四百餘萬兩,實徵僅百四十五萬兩,賦稅虧額如此。財既不在國,又不在民,大率為貪官墨吏所侵蝕。約而言之,其弊有五:一曰報荒不實,二曰報災不確,三曰捏作完欠,四曰徵存不解,五曰交代宕延。覈計近年賦稅短徵,以安徽及江蘇之江寧為最,蘇州、江西次之,河南又次之。多者所收不及五分,少者亦虧一二分不等。請飭各督撫籓司認真釐剔,以裕度支。」詔從其請。然終清之世,諸弊卒未能盡革也。

二十年,中、日之戰,賠兵費二萬萬。二十六年,拳匪肇禍,復賠各國兵費四萬五千萬。其後練新軍,興教育,創巡警,需款尤多,大都就地自籌。四川因解賠款,而按糧津貼捐輸之外,又有賠款新捐。兩江、閩、浙、湖北、河南、陜西、新疆於丁漕例徵外,曰賠款捐,曰規復錢價,曰規復差徭,曰加收耗羨,名稱雖殊,實與加賦無大異也。

總計全國賦額,其可稽者:順治季年,歲徵銀二千一百五十餘萬兩,糧六百四十餘萬石;康熙中,歲徵銀二千四百四十餘萬兩,糧四百三十餘萬石;雍正初,歲徵銀二千六百三十餘萬兩,糧四百七十餘萬石;高宗末年,歲徵銀二千九百九十餘萬兩,糧八百三十餘萬石,為極盛云。

一曰役法。初沿明舊制,計丁授役,三年一編審,嗣改為五年。凡里百有十戶,推丁多者十人為長,餘百戶為十甲,甲十人。歲除裏長一,管攝一里事。城中曰坊,近城曰廂,鄉里曰里。里長十人,輪流應徵,催辦錢糧,句攝公事,十年一周,以丁數多寡為次,令催納各戶錢糧,不以差徭累之。編審之法,核實天下丁口,具載版籍。年六十以上開除,十六以上添註,丁增而賦隨之。有市民、鄉民、富民、佃民、客民之分。民丁外復有軍、匠、灶、屯、站、土丁名。

直省丁徭,有分三等九則者,有一條鞭徵者,有丁隨地派者,有丁隨丁派者。其後改隨地派,十居其七。都直省徭裡銀三百餘萬兩,間徵米豆。其科則最輕者每丁科一分五釐,重至一兩有餘。山西有至四兩餘,鞏昌有至八九兩者。因地制宜,不必盡同也。三等九則之法,沿自前明,一條鞭亦同。其法將均徭均費等銀,不分銀力二差,俱以一條鞭從事。凡十甲丁糧,總於一里,各里丁糧,總於一州縣,而府,而布政司。通計一省丁糧,均派一省徭役,里甲與兩稅為一。凡一州縣丁銀悉輸於官,官為僉募,以充一歲之役,民不擾而事易集。定內外各衙署額設吏役,以良民充之。吏典由各處僉撥,後改為考取,或由召募投充。役以五年為滿,不退者斥革。其府州縣額設祗候、禁子、弓兵,免雜派差役。又有快手、皁隸、門卒、庫子諸役,皆按額召募。額外濫充者謂之白役,白役有禁。然州縣事劇役繁,必藉其力,不能盡革也。又定州縣鋪司及弓兵之制,禁止私役。禁人民私充牙行、埠頭。

瀕河之地,例有夫役守護。順治四年,以御史佟鳳彩言,設直隸沿河堤夫。九年,河決封丘,起大名、東昌、兗州及河南丁夫數萬塞之。十二年,增給河夫工食。河工用民之例有二:曰僉派,曰召募。僉派皆按田起夫,召募則量給雇值。其後額設之夫,悉給工食,由僉派而召募,役民給值,較古制為善矣。十七年,禁州縣私派里甲之弊。

康熙元年,令江南蘇、松兩府行均田均役法。戶科給事中柯聳言:「任土作賦,因田起差,此古今不易常法。但人戶消長不同,田畝盈縮亦異,所以定十年編審之法,役隨田轉,冊因時更,富者無兔脫之弊,貧者無虻負之累。臣每見官役之侵漁,差徭之繁重,其源皆由於僉點不公,積弊未剔。查一縣田額若干,應審里長若干,每里十甲,每甲田若干,田多者獨充一名,田少者串充一名,其最零星者附於甲尾,名曰花戶,此定例也。各項差役,俱由里長挨甲充當,故力不勞而事易集。獨蘇、松兩府,名為僉報殷實,竟不稽查田畝,有田已賣盡而報裏役者,有田連阡陌全不應差者。年年小審,挪移脫換,叢弊多端。田歸不役之家,役累無田之戶,以致貧民竭骨難支,逃徙隔屬。今當大造之年,請飭撫臣通行兩府,按田起役,毋得憑空僉報,以滋賣富差貧之弊。其他花分子戶、詭寄優免、隔屬立戶、買充冊書諸弊,宜嚴加禁革。」下部議行。六年,嚴禁江西提甲累民。提甲之說,在明曰提編,現年追比已完,復提次甲,責成備辦。廣信諸府,有連提數甲者,實與加派無二。以御史戈英言,罷之。

七年,定驛遞給夫例。凡有驛處,設夫役以供奔走,其額視路之沖僻為衡,日給工食,皆入正賦編徵。此項人夫,大率募民充之,差役稍繁,莫不臨時添雇。水驛亦然。十二年,停河南僉派河夫,按畝徵銀,以抵雇值。十六年,河道總督靳輔上言:「河工興舉,向俱勒州縣派雇里民,用一費十。今兩河並舉,日需夫十餘萬,乃改僉派為雇募,多方鼓舞,數月而工成。」大工用雇募自輔始。是年禁有司派罰百姓修築城垛。二十九年,以山東巡撫佛倫言,令直省紳衿田地與人民一律差徭。

五十一年,諭曰:「海宇承平日久,戶口日增,地未加廣,應以現在丁冊定為常額,自後所生人丁,不徵收錢糧,編審時,止將實數查明造報。」廷議:「五十年以後,謂之盛世滋生人丁,永不加賦。仍五歲一編審。」戶部議:「缺額人丁,以本戶新添者抵補;不足,以親戚丁多者補之;又不足,以同甲糧多之丁補之。」

雍正初,令各省將丁口之賦,攤入地畝輸納徵解,統謂之「地丁」。先是康熙季年,四川、廣東諸省已有行之者。至是準直隸巡撫李維鈞請,將丁銀隨地起徵,每地賦一兩,攤入丁銀二錢二釐,嗣後直省一體仿行。於是地賦一兩,福建攤丁銀五分二釐七毫至三錢一分二釐不等;山東攤一錢一分五釐;河南攤一分一釐七毫至二錢七釐不等;甘肅,河東攤一錢五分九釐三毫,河西攤一分六毫;江西攤一錢五釐六毫;廣西攤一錢三分六釐;湖北攤一錢二分九釐六毫;江蘇、安徽畝攤一釐一毫至二分二釐九毫不等;湖南地糧一石,徵一毫至八錢六分一釐不等。自後丁徭與地賦合而為一,民納地丁之外,別無徭役矣。惟奉天、貴州以戶籍未定,仍丁地分徵。又山西陽曲等四十二州縣,亦另編丁銀。

二年,江西巡撫裴度奏裁里長。時廷臣有言大小衙署,遇有公事需用物件,恣行科派,總甲串通奸胥,從中漁利;凡工作匠役,皆設立總甲,派定當官,以次輪轉;又設貼差名目,不原赴官者,勒令出銀,大為民害。詔並禁止。然日久玩生,滋擾益甚。乾隆元年,復有詔申禁。又諭各處歲修工程,如直隸、山東運河,江南海塘,四川堤堰,河南沁河、孟縣小金堤等工,向皆於民田按畝派捐,經管裏甲,不無苛索,嗣後永行停止。凡有工作,悉動用帑金。十年,川陜總督慶復奏興修各屬城垣,請令州縣捐廉,共襄其事。帝曰:「各官養廉,未必有餘,名為幫修,實派之百姓,其弊更大。」不許。乃定各省城工千兩以下者,分年修補,土方小工,酌用民力,餘於公項下支修。二十二年,更定江西修堤力役之法。凡修築土堤,闔邑共攤,夫從糧徵,聽官按堤攤分,募夫修築。從巡撫胡寶瑔請也。二十五年,御史丁田樹言:「自丁糧歸於地畝,凡有差徭及軍需,必按程給價,無所謂力役之徵。近者州縣於上官迎送,同僚往來,輒封拏車船,奸役藉票勒派,所發官價,不及時價之半,而守候回空,概置不問,以致商旅裹足,物價騰踴。嗣後非承辨大差,及委運官物,毋得減發官價,出票封拏,違者從重參處。」得旨允行。三十二年,以用兵緬甸,經過各地,夫馬運送,頗資民力,特頒帑銀,每省十萬,分給人民。

田賦職役,本有經制,大率東南諸省,賦重而役輕,西北賦輕而役重。直隸力役之徵,有按牛驢派者,有按村莊派者,有按牌甲戶口科者,間亦有按地畝者。然富者地多可以隱匿,貧者分釐必科,雜亂無章,偏枯不公。其尤甚者,莫如紳民兩歧。有紳辦三而民辦七者,有紳不辦而民獨辦者,小民困苦流離,無可告訴。時有議仿攤丁於地之例,減差均徭,每畝一分,無論紳民,按地均攤。直隸總督顏檢力言其不可,並謂:「如議者所言,每地一畝,攤徵差銀一分,其意在藉賦以收減差之實效,不知適藉差而添加賦之虛名,累官病民,弊仍不免。」疏入,議遂寢。

咸豐時,粵西役起,徵調不時,不得不藉民力。糧銀一兩,派差銀數倍不等。事定,差徭繁重如故,且錢糧或有蠲緩,差銀則歉歲仍徵。

光緒四年,山西巡撫曾國荃疏陳晉省瘡痍難復,請均減差徭以舒民困,其略曰:「晉省右輔畿疆,西通秦、蜀,軍差、餉差、藏差,絡繹於道,州縣供億之煩,幾於日不暇給。車馬既資之民間,役夫亦責之里甲。而各屬辦理不同。有闔邑里甲通年攤認者,資眾力以應役,法尚公允。有分里分甲限年輪認者,初年攤之一甲一里,次年攤之二甲二里,各年差徭多寡不等,即里甲認派苦樂不均。豪猾者恃有甲倒累甲、戶倒累戶之弊,將其地重價出售,而以空言自認其糧。三五年後,乘間潛逃,於是本甲既代賠無主之糧,又代認無主之差,貽害無窮。計惟減差均徭,尚堪略為補救。除大差持傳單勘合,循例支應,其他概不得藉端苛派。如有擅索車馬者,治以應得之罪。」從之。五年,閻敬銘復條陳八事:一,裁減例差借差;二,由臬司發給車馬印票;三,喇嘛來往,須有定班;四,奉使辦事大臣,宜禁濫索;五,嚴除衙蠹地痞;六,令民間折交流差錢,由衙門自辦;七,嚴查驛馬足額備用;八,本省征防各兵,給予長車,由營自辦。下所司議行。八年,張之洞任山西巡撫,復言:「晉省虐民之政,不在賦斂而在差徭。向例每縣所派差錢,大縣制錢五六萬緡,小縣亦萬緡不等,按糧攤派,官吏朋分,沖途州縣,設立車櫃,追集四鄉牲畜,拘留過客車馬,或長年抽收,或臨時勒價,居者行者均受其患。現擬籌款生息,官設差局,嚴定應差章程,禁止差員濫支。」車櫃陋習遂革。

先是先代陵墓,皆設陵戶司巡查灑掃,例免差徭。又各先賢祠宇,凡有祭田,皆免其丁糧。軍民年七十以上者,許一子侍養,免其雜泛差役。

順治二年,免直省京班匠價,並除其匠籍。定紳衿優免例,內官一品免糧三十石、丁三十,二品免糧二十四石、丁二十四,其下以次遞減;外任官減其半。十四年,部議優免丁徭,本身為止。雍正四年,四川巡撫羅殷泰言,川省各屬,以糧載丁,請將紳衿貢監優免之例禁革。部議駁之。復下九卿議,定紳衿止免本身;其子孫族戶冒濫,及私立儒戶官戶者,罪之。乾隆元年,申舉貢生監免派雜差之令。三十七年,停編審造冊。時丁銀既攤入地糧,而續生人丁又不加賦,五年編審,不過沿襲虛文,無裨實政,至是因李瀚言,遂罷之。翌年,陳輝祖請將民屯新墾丁銀隨年攤徵。帝以所奏與小民較及錙銖,非惠下恤民之道,諭嗣後各省辦理丁糧,悉仍舊制,毋得輕議更張。

一曰蠲免賦稅。蠲免之制有二:曰恩蠲,曰災蠲。恩蠲者,遇國家慶典,或巡幸,或用兵,輒蠲其田賦。

世祖入關,首免都城居民被兵者賦役三年。順治二年,以山西初復,免本年田租之半。三年,收江南,免漕糧三之一。八年,世祖親政,給還九省加派額外錢糧,免山西荒地額糧一萬五千頃,及直隸、山東、河南、陜西荒殘額賦。恩蠲災蠲之詔,歲數四下。康熙十年東巡,免蹕路所經今年租。十三年,蠲免各省八九兩年本折錢糧積欠在民者。時海內大定,詔用兵以來積欠錢糧悉免之。二十七年南巡,免江南積欠地丁錢糧,及屯糧蘆課米麥豆雜稅。三十三年,蠲免廣西、四川、貴州、雲南四省應徵地丁銀米。四十五年,免直隸、山東本年積欠錢糧,其山西、陜西、甘肅、江蘇、浙江、安徽、江西、湖北、湖南、福建、廣東、廣西各省,自康熙四十三年以前,未完地丁銀二百十二萬有奇,糧十萬五千石有奇,悉行蠲免。

承平日久,戶口漸繁,地不加增,民生有不給之虞,詔直省自五十年始,分三年輪免錢糧一周。三年中計免天下地丁糧賦三千八百餘萬。五十六年,免直隸、安徽、江蘇、浙江、江西、湖廣、西安、甘肅帶徵地丁屯衛銀二百三十九萬餘兩,其安徽、江蘇所屬帶徵漕項銀四十九萬餘兩,米麥豆十四萬餘石,免徵各半。五十七年,以征策妄阿拉布坦,免陜、甘明年地丁百八十餘萬。聖祖嘗讀漢文帝蠲民田租詔,嘆曰:「蠲租乃古今第一仁政,窮穀荒陬,皆沾實惠。然非宮廷力崇節儉,不能行此。」故在位六十年中,屢頒恩詔,有一年蠲及數省者,一省連蠲數年者,前後蠲除之數,殆逾萬萬。

世宗即位,蠲免江蘇各屬歷年未完民屯地丁蘆課等銀千二百十餘萬。西藏、苗疆平,免甘肅、四川、廣西、雲、貴五省田租。又諭國家經費已敷,宜散富於民,乃次第免直省額賦各四十萬。乾隆元年,詔免天下田租,先後免雍正十三年以前各省逋賦、及江南錢糧之官侵吏蝕者。四年,免直隸本年錢糧九十萬,江蘇百萬,安徽六十萬,正耗一體蠲除。十年,普免天下錢糧二千八百二十四萬有奇,援康熙五十一年之例,將各省分為三年,以次豁免。三十一年,詔次第蠲各省漕米,五年而遍,其例徵折色者亦免之。三十五年,值帝六旬,明歲又際太后八旬,照十年之例,按各省額賦,分三年輪免一周。

四十二年,普免天下錢糧,自明年始,分三年輪免,計二千七百五十九萬有奇。各省漕糧,自四十五年普免一次。四十九年,豁免甘肅壓欠起運糧銀百六十餘萬,其存留項下民欠銀糧,起運項下民欠草束,悉免之。五十五年,高宗八旬,詔按各省額徵銀數,將所屬各府州縣次第搭配三次,按年輪免,三年而竣,一省之中,仍先侭上年災緩之區,首先蠲免。五十九年,普免各省應徵漕糧。六十年,普免各省積欠,及因災緩帶銀千五百五十餘萬兩、糧三百八十餘萬石,其奉天、山西、四川、湖南、廣西、貴州六省向無積欠,免下年正賦十之二。又以明年將歸政,免嘉慶元年各省應徵地丁錢糧,其省方時巡蹕路所經,輒減額賦十之三。

仁宗即位,以湖北、湖南教匪苗民蠢動,免次年兩省錢糧,並及川、陜被兵之區。四年,以郊祀升配禮成,普免各省積欠緩徵地丁耗羨,及民欠籽種口糧漕糧銀,並積欠緩徵民借米穀草束。十年,謁祖陵,免蹕路所經州縣錢糧之半。二十四年,以六旬萬壽,免天下正耗民欠,及緩帶銀穀,計銀二千一百二十九萬兩有奇、米穀四百餘萬石。四川、貴州兩省無民欠,免明年正賦十之二。

災蠲有免賦,有緩徵,有賑,有貸,有免一切逋欠。清初定制,凡遇災蠲,起運存留均減。存留不足,即減起運。順治初,定被災八分至十分,免十之三;五分至七分,免二;四分免一。康熙十七年,改為六分免十之一,七分以上免二,九分以上免三。雍正六年,又改十分者免其七,九分免六,八分免四,七分免二,六分免一。然災情重者,率全行蠲免。凡報災,夏災以六月,秋災以七月。既報,督撫親蒞災所,率屬發倉先賑,然後聞。康熙三年,戶部奏遇災之地,先將額賦停徵十之三,以待題免。四年,御史郝維訥請凡災地田賦免若干,丁亦如之。其後丁隨地起,凡有災荒,皆丁地並蠲。旨下之日,州縣不即出示,或蠲不及數、納不留抵者,科以侵欺之罪。乾隆元年,安徽布政使晏斯盛請「嗣後各省水旱應免錢糧之數,於具題請賑日始,限兩月造報,並請將丁銀統入地糧銀內覈算蠲免」。從之。聖祖、高宗兩朝,疊次普免天下錢糧,其因偏災而頒蠲免之詔,不能悉舉。仁宗之世,無普免而多災蠲,有一災而免數省者,有一災而免數年者。文宗以後,國用浩繁,度支不給,然遇疆臣奏報災荒,莫不立予蠲免。若災出非常,或連年饑饉,輒蠲賑兼施云。

倉庫京師及各直省皆有倉庫。倉,京師十有五。在戶部及內務府者,曰內倉,曰恩豐;此外曰祿米,曰南新,曰舊太,曰富新,曰興平,曰海運,曰北新,曰太平,曰本裕,曰萬安,曰儲積,曰裕豐,曰豐益。在通州者,曰西倉,曰中倉。各省漕運,分貯於此。直省則有水次倉七:曰德州,曰臨清,曰淮安,曰徐州,曰江寧,各一;惟鳳陽設二。為給發運軍月糧並駐防過往官兵糧餉之需。其由省會至府、州、縣,俱建常平倉,或兼設裕備倉。鄉村設社倉,市鎮設義倉,東三省設旗倉,近邊設營倉,瀕海設鹽義倉,或以便民,或以給軍。大抵京、通兩倉所放米,曰官俸,曰官糧,亦名甲米,二者去全漕十之六。其一,養工匠,名匠米。其一,定鼎時,宗臣封親王者六,封郡王者二,世宗之弟封親王者一,此九王子孫,自適裔外,並有封爵,以世降而隨之,統名恩米,二者去京倉百之一。是以雍正以前,太倉之粟常有餘。

乾隆二十八年,戶部侍郎英廉疏言:「邇年因賑恤屢截留漕運,間遇京師糧貴,復發內倉米石平糶,儲積漸減。請於湖廣、江西、江南、浙江產米之區,開捐貢監,均收本色,收足別貯。遇截漕之年,即於次年照數補運京倉。」下九卿議準,旋復停止。及嘉慶中,川楚盜起,水旱間作,工匠既倍於昔,而九王之後亦愈衍愈眾。咸豐後,復有粵寇之亂,運道不通,倉儲益匱,亂平稍復舊例。

向京師平糶,有五城米局,八旗米局。五城米局始於康熙。雍正四年,於內城添廠,並添五城、通州廠各一。乾隆二年,增五城為十廠,尋又添設八廠於四鄉。九年,於四路同知設四廠。八旗米局凡二十四,又通州左右翼兩局,皆設於雍正六年。乾隆元年,並為八局,旋仍舊。十五年,命二十四局分左右翼辦理,不拘旗分。十七年,以米價未平,且有勒買之弊,諭並通州兩局停止。

其直省常平、裕備等倉,順治十一年,命各道員專管,每年造冊報部。十七年,戶部議定常平倉穀,春夏出糶,秋冬糴還,平價生息,兇歲則按數給散貧戶。康熙六年,甘肅巡撫劉斗疏言:「積米年久恐浥爛,請變價糴新穀。」從之。七年,陜西巡撫賈漢復請將積穀變價生息。帝諭出陳入新,原以為民,若將利息報部,反為民累,著停止生息。十九年,諭常平倉留本州縣備賑,義倉、社倉留本村鎮備賑。三十年,戶部議令直隸所捐米石,大縣存五千石,中縣四千,小縣三千;嗣又令再加貯一倍。三十一年,議定州縣積穀,照正項錢糧交代,短少以虧空論。三十四年,議定江南積穀,每年以七分存倉,三分發糶,並著為通例。四十三年,議定州縣倉穀霉爛者,革職留任,限一年賠完復職;逾年不完,解任;三年外不完,定罪,著落家產追賠。

時各省州縣貯穀之數,山東、山西大州縣二萬石,中州縣萬六千石,小州縣萬二千石;江西大州縣一萬二千石;江蘇、四川率不過五六千石;而福建現在捐穀二十七萬石,常平又存五十六萬石;臺灣捐穀及常平為最多,共八十餘萬石。令酌留三年兵需,餘變價充餉。四十七年,議定州縣官於額貯外加買貯倉,準其議敘,若捐穀以少報多,或將現貯米捏作捐輸,後遇事發,除本管知府分賠外,原報督撫一並議處。至官將倉穀私借於民,計贓以監守自盜論,穀石照數追賠。五十四年,議定紳民捐穀,按數之多寡,由督撫道府州縣分別給扁,永免差役。

雍正三年,以南方潮濕,令改貯一米易二穀。四年,浙閩總督高其倬疏言:「閩省平糴有二大病:一,交盤之弊不清,各官授受,皆有價無穀,而價又不敷買補;一,平糶之價太賤,每石減價至一兩,且有不及一兩者,各屬雖欲買補,緣價短束手,而奸民乘此謀利,往往借價貴,煽惑窮民,竟欲平糶之期,一歲早於一歲,平糶之價,一年賤於一年。請嗣後視米之程高下,每石以一兩二錢或一兩三錢,穀則定以六錢五分或六錢,總以秋成後既平之價為準。」帝韙其言。尋定州縣倉廠敖不修,致米穀霉爛者,照侵蝕科斷,並將虧空各州縣解任。其穀令自行催還,限以一年,逾限者治罪。五年,定各省常平倉,每年底令本府州盤查。如春借逾十月不完,或捏造,俱行參處,照數追賠。又因福建常平倉各屬有銀穀兩空者,有無穀而僅存價者,查實,將虧空之州縣官更換。

十三年,內閣學士方苞上平糶倉穀三事:「一,倉穀每年存七糶三,設遇價昂,必待申詳定價,窮民一時不得邀惠。請令各州縣酌定官價,一面開糶,一面詳報。一,江淮以南地氣卑濕,若通行存七糶三,恐積至數年,必有數百萬霉爛之穀,有司懼罪,往往以既壞之穀抑派鄉戶。請飭南省各督撫,驗察存倉各穀色,因地分年,酌定存糶分數;河北五省倘遇歲歉,亦不拘三七之例。一,穀之存倉有鼠耗,盤糧有折減,移動有腳價,糶糴守局有人工食用,春糶之價即稍有贏餘,亦僅足充諸費。請飭監司郡守歲終稽查,但數不虧,不得借端要挾,倘逢秋糴價賤,除諸費外,果有贏餘,詳明上司別貯,以備歉歲之用。」下部議行。

乾隆三年,兩江總督那蘇圖疏言平糶之事,止須比市價酌減一二分。兩廣總督鄂彌達亦言:「平糶之價,不宜頓減。蓋小民較量錙銖,若平糶時官價與市價懸殊,則市儈必有藏以待價,而小民藉以舉火者,必皆仰資官穀。倉儲有限,商販反得居奇,是欲平糶而糶仍未平也。從來貨積價落,民間既有官穀可糴,不全賴鋪戶之米,鋪戶見官穀所減有限,亦必稍低其價以冀流通。請照市價止減十一,以次遞減,期年而止,則鋪戶無所操其權,而官穀不至虞其匱。」均報可。七年,諭:「從前張渠奏請減價糶穀,成熟之年,每石照巿價減五分,米貴之年減一錢。但思歉歲止減一錢,窮民得米仍艱。嗣後著督撫臨時酌量應減若干,奏明請旨。如有奸民賤糴貴糶,嚴拏究治。」

十三年,高宗諭大學士、戶部曰:「邇來常平倉額日增,有礙民食,嗣後應以雍正年間舊額為準。」尋議云南不近水次,陜、甘兼備軍務,向無定額,請以現額為準。雲南七十萬石,西安二百七十萬石,甘肅三百七十萬石,各有奇。又福建環山帶海,商運不通,廣東嶺海交錯,產穀無幾,貴州不通舟楫,積貯均宜充裕,以現額為準,福建二百五十餘萬石,廣東二百九十餘萬石,貴州五十萬石。其餘照雍正年間舊額:直隸二百一十萬石,奉天百二十萬石,山東二百九十萬石,山西百三十萬石,河南二百三十萬石,江蘇百五十萬石,安徽百八十萬石,江西百三十萬石,浙江二百八十萬石,湖北五十萬石,湖南七十萬石,四川百萬石,廣西二十萬石,各有奇,通計十九省貯穀三千三百七十餘萬石,較舊額四千四百餘萬石,應減貯千四百餘萬石。自是各省或額缺不補。二十三年,特諭採買還倉。三十一年,各省奏銷,報實存穀數,惟江西、河南、廣東與十三年定額相同。其視舊額增多者:湖南百四十三萬石,山西二百三十萬石,四川百八十五萬石,廣西百八十三萬石,雲南、貴州皆八十餘萬石。而浙江視舊額減少二百二十萬石,奉天減少百萬,甘肅減少百四十萬;其直隸、江蘇、安徽、福建、湖北、山東、陜西或減二十萬、或減五六十萬。蓋聚之難而耗之易如此。

嘉慶初,仁宗屢下買補之令。四年,諭曰:「國家設立常平倉,若不照額存儲,僅將穀價貯庫,猝遇需米之時,豈銀所能濟用?」命各省採買還倉。十七年,戶部浙江司所存常平倉穀數凡三千三百五十萬八千五百七十五石有奇,去乾隆中定額猶不遠。至道光十一年,副都御史劉重麟、御史卞士云先後疏言,各直省州縣於常平倉大率有價無穀,其價又不免侵用。帝命各督撫嚴覈究治。然據十五年戶部奏,查各省常平倉穀實數,仍止二千四百餘萬石,又非嘉慶時可比,況咸豐間天下崩亂之日乎。同治三年諭:「近來軍務繁興,寇盜蜂起,所至地方輒以糧盡被陷,其故由各州縣恣意侵挪,遇變無所依賴。嗣後各省常平倉,責成督撫認真整頓。」迨光緒初,直隸、河南、陜西、山西迭遭旱災,饑民死者日近萬人。四年,給事中崔穆之,八年,御史鄔純嘏,復先後請籌辦倉穀,於是各督撫始稍加意焉。

其社義各倉,起於康熙十八年。戶部題準鄉村立社倉,市鎮立義倉,公舉本鄉之人,出陳易新。春日借貸,秋收償還,每石取息一斗,歲底州縣將數目呈詳上司報部。六十年,奉差山西左都御史硃軾奏請山西建立社倉,諭曰:「從前李光地以社倉具奏,朕諭言易行難。行之數年,果無成效。張伯行亦奏稱社倉之益,朕令伊暫行永平地方,其有效與否,至今未奏。凡建設社倉,務須選擇地方敦實之人董率其事。此人並非官吏,借出之米,還補時遣何人催納?即豐收之年,尚難還補,何況歉歲?其初將眾人米穀扣出收貯,無人看守,及米石缺空,勢必令司其事者賠償,是空將眾人之米棄於無用,而司事者無故為人破產賠償也。社倉之法,僅可小邑鄉村,若由官吏施行,於民無益。今硃軾復以此為請,即令伊久住山西,鼓勵試行。」雍正二年,諭湖廣總督楊宗仁、湖北巡撫納齊喀、湖南巡撫魏廷珍等:「前命建社倉,本為民計。勸捐須俟年豐,如值歉歲,即予展限。一切條約,有司勿預,庶不使社倉頓成官倉。今乃令各州縣應輸正賦一兩者,加納社倉穀一石。聞楚省穀石現價四五錢不等,是何異於一兩正賦外加收四五錢火耗耶?」尋議定:凡州縣官止任稽查,其勸獎捐輸之法,自花紅遞加扁額以至八品冠帶。如正副社長管理十年無過,亦以八品冠帶給之。其收息之法,凡借本穀一石,冬間收息二斗。小歉減半,大歉全免,祗收本穀。至十年後,息倍於本,祗以加一行息。

三年,從江蘇巡撫何天培請,止頒行社倉五事:一,賑貸均預造排門冊存案;一,正副社長外,再舉一殷實者總司其事;一,州縣官不許干預出納;一,所需紙筆,必勸募樂輸,或官撥罰項充用;一,積穀既多,應於夏秋之交,減價平糶,秋收後照時價買補。

五年,因湖廣社倉虧空,諭:「邇年督撫辦社倉最力者,惟湖廣總督楊宗仁。今據福敏盤查,始知原報甚多,而現貯無幾。朕思舉行此法實難。我聖祖仁皇帝深知之,是以李光地奏請而未允,張伯行暫行而即罷。蓋在富民無藉乎倉,則輸納不前,而貧者又無餘粟可納。至於州縣官,實心者豈可多得?湖廣虧缺之數,倘系州縣私用,必嚴追賠補,或民間原未交倉,或交倉之數與原報多寡不符,若令照數完納,恐力未敷,須斟酌辦理。」六年,世宗諭曰:「前岳鍾琪請於通省加二火耗內應行裁減每兩五分之數,且暫徵收,發民買穀,分貯社倉,俟數足即行裁減,是以暫收耗羨之中,隱寓勸輸之法,實則應行斟酌之耗羨,即小民切己之貲財,而代民買貯之倉儲,即小民自捐之積貯。乃陜省官員以為收貯在官,即是官物,而胥吏司其出納者,遂有勒買勒借之弊。今特曉示,鐫石頒布,儻地方官有如前者,以撓擾國政、貽誤民生治罪。」

乾隆四年,戶部議準陜西巡撫張楷奏定社倉事例:一,社長三年更換;一,春借時酌留一半,以防秋歉;一,限每年清還;一,將借戶穀數姓名曉示;一,令地方官稽查交代分賠。五年,議定陜、甘社穀凡系民間者,聽自擇倉正、副管理。其系加二耗糧內留五分為社糧者,責成地方官經理,入於交代。自是之後,州縣官視同官物,凡遇出借,層遞具詳,雖屬青黃不接,而上司批行未到,小民無由借領。此後應請令州縣於每年封印後,酌定借期,一面通詳,一面出借,其期按耕種遲早以為先後。得旨允行。

十八年,直隸總督方觀承疏言:「義倉始於隋長孫平,至宋硃子而規畫詳備。雖以社為名,實與義同例。其要在地近其人,人習其事,官之為民計,不若民之自為計,故守以民而不守以官,城之專為備,不若鄉之多為備,故貯於鄉而不貯於城。今使諸有司於四鄉酌設,粟黍從便,並選擇倉正、副管理,不使胥吏干預。現據報捐穀數共二十八萬五千三百餘石,合百四十四州縣衛所,共村莊三萬五千二百一十,為倉千有五。」帝嘉之。三十七年,戶部議準,社倉仍令官經理出納。

嘉慶四年,又議準社義各倉出納,由正、副長經理,止呈官立案。道光五年,安徽巡撫陶澍疏言:「義倉茍欲鮮弊,惟有秋收後聽民間量力輸捐,自擇老成者管理,不減糶,不出易,不借貸,專意存貯,以待放賑。」如所議行。其後軍興,各省皆廢。同治六年,特諭興復。光緒中,惟陜西巡撫馮譽驥所籌建者千六百餘所為最多云。

其旗倉在東三省者,初皆貯米二千萬石。營倉自康熙二十二年始。時山海關各口建倉,達於黑龍江墨爾根。三十年,令江寧、京口等處各截留漕米十萬石存貯。三十六年,諭沿邊衛堡如榆林等處均貯穀。四十九年,以湖南鎮筸改協為鎮,撥借帑銀三千兩,買穀貯倉。五十四年,命貯米密雲、古北口。雍正三年,貯穀歸化城土拉庫。四十七年,先後命廣東提標各營暨諸鎮協均貯穀,其後復推行貴州、四川、浙江、福建、河南。十一年,命喜峰口貯穀。

乾隆元年,設河標營倉。十一年,又命山東河標設立。鹽義倉,自雍正四年始。時兩淮眾商捐銀二十四萬,為江南買穀建倉之用,巡鹽御史噶爾泰以聞,並繳公務銀八萬,共三十二萬。諭以三萬賞給噶爾泰,餘照所請,賜名「鹽義」。既而浙江眾商亦捐銀十萬,諭巡撫李衛於杭州建倉。乾隆九年,又準山東票商仿行。

庫之在京師屬內務府者,設御用監掌之。順治十六年改為廣儲司。十八年,分設緞庫、銀庫、皮庫、衣庫。康熙十八年,增設茶庫、磁庫,合之為六。其屬於戶部者,曰銀庫、曰緞庫、曰顏料庫,合之為三。此外盛京戶部銀庫,貯金銀、幣帛、顏料等物,以供二陵祭祀,及東三省官兵俸餉賞賚之用。各省將軍、副都統、城守尉庫,各貯官兵俸餉,及雜稅官莊糶買糧價。布政使司庫,貯各州縣歲徵田賦、雜賦銀。按察司庫,貯贓罰銀錢。糧道庫,貯漕賦銀、驛站馬夫工料。河道庫,貯河餉。兵備道庫,貯兵餉。鹽運使司鹽課各稅務由部差者,有監督庫。如道、府、、州、縣官兼理者,有兼理徘庫,均貯關鈔。地居沖要之分巡道庫、府庫、直隸州庫及分駐苗疆之同知、通判庫,均量地方大小,距省遠近,酌量撥司庫銀分貯。州、縣、衛所庫,貯本色正雜賦銀,存留者照數坐支,輸運者輸布政使司庫。

凡諸庫每歲出納之數,皆造冊送戶部察覈,惟贓罰例輸之刑部。河工兵餉又兼達兵、工兩部。戶部於直省庫儲,其別有五。曰封儲。如酌留各布政司銀兩,督撫公同封儲,有急需,題奏動支,擅用論斬是也。此制定於雍正五年。以直隸近京,獨無留貯。各省自三十萬至十萬,析為三等。其後直隸亦有之。惟盛京戶部銀庫,自乾隆四十二年由京撥給一千萬,永遠存貯。四十三年,復命將軍兼管。曰分儲。如各省道庫、府庫,封貯銀兩,遇州縣急需,請領即行發給,一面詳報籓司督撫,仍令各州縣將支銷銀兩,隨案具詳聽覈是也。其後各繁劇州縣,亦照京縣例撥貯,而未有定額。及雍正八年,乃定各省道、府、州、縣分貯之額,自三十萬至十萬,析為四等。曰留儲。如存留屬庫坐支銀兩,撥款給發,例免解司是也。曰解儲。如布政使司庫,儲府、州、縣、衛解送正雜賦銀;按察司庫,收贓罰銀;及將軍、副都統、城守尉庫,糧道庫,收各處移解官兵俸餉漕項等銀是也。曰撥儲。如各省兵備道庫,歲儲由布政司或鄰省撥解官兵銀,河道庫,歲儲本省及鄰省撥解官兵俸餉,並歲修搶修銀,及伊犁歲需俸餉銀,塔爾巴哈臺歲需新餉銀,西藏歲需臺費銀,雲南歲需銅本銀,貴州歲需鉛本銀,皆由各省撥解是也。戶部總稽之,俾慎其收發,令各省解部地丁,將足色紋銀傾鎔元寶,合部頒法馬,每枚五十兩,勿加滴珠。

凡起解餉銀,布政使親同解官兌封押字,令庫官鈐印,當堂裝鞘,給發兵牌。又州縣官錢糧交代,由接任官造具接收冊結,同監盤官印結,上司加結送司,詳請咨部,不得逾限。布政使升轉離任,將庫儲錢糧並無虧挪之處附奏,其新任接收,亦具摺奏聞,仍照例限詳題。按察使交代,由巡撫會同籓司查覈詳題,且時其盤查,令各督撫於布政使司庫錢糧奏銷交代時,親赴盤查,具結報題。督撫新任亦然。府、州、縣庫儲錢糧奏銷時,所管道、府親赴盤查結報,不得委查取結,及預示日期,縱令掩飾。

至戶部銀庫,康熙四十五年,以貯銀多,諭將每年新收銀別行收貯,至用銀時,將舊銀依次取用。乾隆四十一年,戶部奏準各直省解京銀兩,無論元寶、小錠,必鏨鑿州縣年月及銀匠姓名。嘉慶十九年,命各省銀解部,隨到隨交。道光十二年,又命官解官交。蓋向來京餉及捐項,皆由銀號交庫也,然其弊不易革。同治三年,戶部奏準凡由銀號交庫者,均收足色銀兩,錠面鏨明某號字樣,倘有弊端,即照原數加十倍罰賠。光緒四年,又奏準嗣後各省督撫並各路統兵大臣赴部領餉,須遵章遞印領,蓋所以重庫儲而杜流弊也。


\end{pinyinscope}