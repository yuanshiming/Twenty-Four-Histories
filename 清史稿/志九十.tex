\article{志九十}

\begin{pinyinscope}
○職官二

理籓院都察院五城兵馬司六科給事中通政使司大理寺翰林院文淵閣

國史館經筵講官起居注詹事府太常寺太僕寺光祿寺鴻臚寺國

子監衍聖公五經博士欽天監太醫院壇廟官陵寢官僧道錄司

理籓院管理院務大臣,滿洲一人。特簡大學士為之。尚書,左、右侍郎,俱各滿洲一人。間亦有蒙古人為之。額外侍郎一人。以蒙古貝勒、貝子之賢能者任之。其屬:堂主事,滿檔房滿洲二人、蒙古三人,漢檔房漢軍一人。領辦處,員外郎、主事,滿、蒙各一人。司務司務,滿、蒙各一人。筆帖式,滿洲三十有六人,蒙古五十有五人,漢軍六人。旗籍、王會、柔遠、典屬、理刑、徠遠六清吏司:郎中,宗室一人,柔遠司置。滿洲三人,旗籍、王會、典屬司各一人。蒙古八人。旗籍、王會、理刑司各二人。典屬、徠遠司各一人。員外郎,宗室一人,旗籍司置。滿洲十人,王會、柔遠、典屬、理刑司各二人。旗籍、徠遠司各一人。蒙古二十有四人。旗籍二人,王會三人,柔遠五人,典屬六人,理刑、徠遠司各一人。主事,滿洲二人,旗籍、典屬司各一人。蒙古七人。柔遠、典屬、理刑司各一人。王會、徠遠司各二人。筆帖式,滿洲三十有六人,蒙古五十有五人,漢軍六人。銀庫,司官二人,司官內奏委。司庫一人,正七品。庫使、筆帖式各二人。以上俱滿洲缺。

尚書掌內外籓蒙古、回部及諸番部,制爵祿,定朝會,正刑罰,控馭撫綏,以固邦翰。侍郎貳之。旗籍掌考內扎薩克疆里,大漠以南曰內蒙古,部二十有四:曰科爾沁,曰扎賚特,曰杜爾伯特,曰郭爾羅斯,曰敖漢,曰奈曼,曰巴林,曰扎魯特,曰阿魯科爾沁,曰翁牛特,曰克什克騰,曰喀爾喀左翼,曰喀喇沁,曰土默特,曰烏珠穆沁,曰浩齊特,曰蘇尼特,曰阿巴噶,曰阿巴哈納爾,曰四子部落,曰茂明安,曰烏喇特,曰喀爾喀右翼,曰鄂爾多斯,為旗四十有九。疇封爵,凡六等:一親王,二郡王,三貝勒,四貝子,五鎮國公,六輔國公。不入六等者,曰臺吉、塔布囊,亦分四等。辨譜系。凡官屬、扎薩克之輔曰協理臺吉。其屬曰管旗章京,曰副章京,曰參領,曰佐領,曰驍騎校。部眾會盟、盟地六:曰哲裏木,曰卓索圖,曰昭烏達,曰錫林郭勒,曰烏爾察布,曰伊克昭。置盟長、副盟長各一人,由扎薩克請簡。軍旅郵傳,並隸治之;兼稽游牧內屬者。凡歸化城土默特、黑龍江布特哈皆是。王會掌內扎薩克賓禮,典朝覲、貢獻儀式。凡饗賚、館餼,視等級以為差。典屬掌外扎薩克部旗封爵,大漠以北曰外蒙古,部四:曰土謝圖汗,曰賽音諾顏,曰車臣汗,曰扎薩克圖汗,為旗八十有六。又有杜爾伯特部,土爾扈特部,和碩特部,輝特部,綽羅斯部,額魯特部。別於蒙古者,曰和托輝特,曰哈柳沁,曰托斯,曰奢集努特,曰古羅格沁,並屬。以外扎薩克封爵有汗,以列王、貝勒、貝子、公之右。無塔布囊,有臺吉。治盟會。喀爾喀四盟:曰汗阿林,曰齊齊爾里克,曰克魯倫巴爾和屯,曰扎克畢拉色欽畢都爾諾爾。杜爾伯特二盟:曰賽因濟雅哈圖左翼,曰賽因濟雅哈圖右翼。土爾扈特五盟:曰南烏訥恩素珠克圖,曰北烏訥恩素珠克圖,曰東烏訥恩素珠克圖,曰西烏訥恩素珠克圖,曰青塞特奇勒圖。和碩特一盟:曰巴圖塞特奇勒。盟置盟長、副盟長各一人,於同盟扎薩克內簡用。惟青海之盟無長。置郵驛,頒屯田、互市政令;兼稽游牧內屬者。一曰察哈爾,二曰巴爾呼,三曰額魯特,四曰扎哈沁,五曰明阿特,六曰烏梁海,七曰達木,八曰哈薩克。柔遠掌治外扎薩克眾部,凡喇嘛、番僧祿廩、朝貢,並司其儀制。徠遠掌回部扎薩克、伯克歲貢年班,番子、土司亦如之;並典外裔職貢。附牧回城卡倫外,曰布魯特。內附者各給以銜,歲遣使輸馬。他哈薩克,若浩罕,若博羅爾,若巴達克山,若愛烏罕,並各效其職貢。理刑掌蒙古、番、回刑獄諍訟。領辦處掌綜領眾務。銀庫掌帑金出納。

其兼領者:蒙古繙譯房,員外郎、主事各一人,司官內奏委。校正漢文官二人,滿、蒙內閣侍讀學士、侍讀、翰林院侍讀、侍講學士、侍讀、侍講內奏派。主章奏文移。內、外館監督各一人,六部司員內充補。光緒三十三年省。主賓館繕完滌除。烏蘭哈達、三座塔、八溝司官各一人,分駐塔子溝筆帖式一人,嘉慶十五年撤回,並四處司員俱改為理事官,隸熱河都統,仍由本院司員內簡放。分主蒙古部落民人訟事。察哈爾游牧處理事員外郎十有六人,以在京蒙古各旗與察哈爾各旗官員內番選。由護軍、驍騎校選用者授員外郎。由中書、筆帖式選用者,先授主事,三年稱職,升員外郎。分主游牧察哈爾民人訟事。張家、喜峰、獨石、殺虎、古北諸口管理驛站員外郎、司員內奏委。筆帖式各一人,主蒙古郵驛政令。圍場總管一人,康熙四十五年置。乾隆十四年始來隸。嘉慶七年後,改隸熱河都統。左、右翼長各一人,章京八人,初制六品。乾隆十八年升五品。驍騎校八人,主守木蘭圍場,專司巡察。

初,崇德元年,設蒙古衙門,置承政、參政各官。三年,更名理籓院,定承政,左、右參政,各一人,副理事官八人,啟心郎一人。順治元年,改承政為尚書,參政為侍郎,滿、蒙參用。副理事官為員外郎,置二十有一人,康熙二十年增滿、蒙八人。乾隆四十二年省蒙古一人,四十九年改滿洲二人為蒙缺。後滿、蒙司官增減不一。啟心郎三人,滿洲一人,漢軍二人。十五年省。堂主事二人,康熙二十八年增漢文一人。司務二人,滿、蒙各一人。康熙三十八年省。雍正十年復故。漢副使一人。從八品。五年,增置漢院判、正六品。知事正八品。自副使以下,俱康熙三十八年省。各一人。四年,置唐古忒學教習一人。給六品俸。後改司業。其助教以他官兼。乾隆五年定為額缺,尋省。十六年,定以禮部尚書銜掌院事,侍郎銜協理院事。越二年,以隸禮部未合舊制,停兼銜,依六部例,令入議政,班居工部後。並設錄勛、賓客、柔遠、理刑四司,置滿、蒙郎中共十有一人,乾隆四十二年增蒙古一人。四十九年改滿洲二人為蒙缺。員外郎二十有一人,康熙二十年增滿、蒙八人。乾隆四十二年省蒙古一人。四十九年改滿洲六人為蒙缺。主事滿、漢各四人。康熙二十八年省漢缺。乾隆四十九年改滿洲二人為蒙缺。康熙二十年,增蒙古文主事二人。三十八年,析柔遠司為二,曰前司,曰後司。四十六年,設銀庫,初制,蒙古王、臺吉等入朝,由戶、工二部及光祿寺庀器用,具廩餼。至是始創設。置郎中、員外郎各一人,司員內奏派。司庫一人,庫使四人。雍正元年,始命王、公、大學士領院事,省庫使二人。乾隆二十二年,改錄勛司為典屬,賓客司為王會,柔遠後司為旗籍,前司仍曰柔遠。二十六年,合旗籍、柔遠為一,增設徠遠一司。明年,仍析旗籍、柔遠為二。二十九年,改典屬司為旗籍,舊旗籍為典屬。嘉慶四年,改滿洲郎中、員外郎各一人為宗室員缺。咸豐五年,定伊犁塔爾巴哈臺通商章程,始司外交職務。見第十七款。十年,定中俄續約,以軍機處及本院典外交文移。見第九款。後歸外部。光緒三十二年,更院為部,擬設殖產、邊衛二司。嗣先設編纂、調查二局,隸領辦處,以漢檔房、俸檔房、督催所改並。漢檔房主事缺未省。尋置員外郎、主事各一人。蒙古房改。俱蒙缺。宣統三年,改尚書為大臣,侍郎為副大臣,額外侍郎如故。

理籓一職,歷古未有專官,唯周官大行人差近之。秦、漢以降,略存規制。遐荒絕漠,統治王官,為有清創制。自譯署設,職權漸替已。

都察院左都御史,初制,滿員一品,漢員二品。順治十六年並改二品。康熙六年仍升滿員為一品,九年並定正二品。雍正八年升從一品。左副都御史,正三品。俱滿、漢二人。其屬:經歷司經歷,正六品。都事都事,正六品。俱滿、漢一人。筆帖式四十有二人。十五道掌印監察御史,初制,滿洲、漢軍三品,順治十六年改七品。康熙六年升四品,九年復為七品。雍正七年,改由編、檢、郎員授者正五品。由主事、中、行、評、博授者正六品。乾隆十七年並定從五品。滿、漢各一人。監察御史,京畿、江西、浙江、福建、湖廣、河南、山西、陜西八道,滿、漢各一人,江南道滿、漢各三人,山東道滿、漢各二人。

左都御史掌察覈官常,參維綱紀。率科道官矢言職,率京畿道糾失檢奸,並豫參朝廷大議。凡重闢,會刑部、大理寺定讞。祭祀、朝會、經筵、臨雍,執法糾不如儀者。左副都御史佐之。十五道掌彈舉官邪,敷陳治道,各覈本省刑名。京畿道分理院事,及直隸、盛京刑名,稽察內閣、順天府、大興、宛平兩縣。河南道照刷部院諸司卷宗,稽察吏部、詹事府、步軍統領、五城。江南道稽察戶部、寶泉局、左右翼監督、京倉、總督漕運,磨勘三庫奏銷。浙江道稽察禮部及本院。山西道稽察兵部、翰林院、六科、中書科、總督倉場、坐糧、大通橋監督、通州二倉。山東道稽察刑部、太醫院、總督河道,催比五城命盜案牘緝捕之事。陜西道稽察工部、寶源局,覆勘在京工程。湖廣道稽察通政使司、國子監。江西道稽察光祿寺。福建道稽察太常寺。四川道稽察鑾儀衛。廣東道稽察大理寺。廣西道稽察太僕寺。雲南道稽察理籓院、欽天監。貴州道稽察鴻臚寺。其祭祀、監禮、侍班糾儀,科道同之。經歷掌董察吏胥。都事掌繕寫章奏。其分攝者:巡視五城御史,滿、漢各一人,科道中簡用。一年更替。掌綏靖地方,釐剔奸弊。兵馬司指揮、正六品。副指揮、正七品。吏目,未入流。自正指揮以下俱漢員。五城各一人,掌巡緝盜賊,平治道路,稽檢囚徒,火禁區為十坊領之。

初沿明制,設都察院。天聰十年,諭曰:「凡有政事背謬,及貝勒、大臣驕肆慢上者,許直言無隱。」崇德元年,置承政、參政各官。明年定承政一人,左、右參政滿、蒙、漢理事官各二人。後省。順治元年,改左都御史掌院事,滿、漢各一人。左副都御史協理院事,各二人。漢左僉都御史一人。先用漢軍,後參用漢人。乾隆十三年省。外省督、撫,並以右系銜。右都御史、右副都御史、右僉都御史為督、撫坐銜。乾隆十三年停右都御史銜。司務,後改經歷。滿、漢各一人。都事,滿洲二人,乾隆十七年改滿、漢各一人。漢軍一人。康熙三十九年省。設十五道。河南道參治院事,置監察御史,滿洲六人,河南、江南、浙江、山東、山西、陜西掌印各一人。五年增十有七人。康熙二十八年增一人,後復省四人。乾隆十四年定江南、山東道各三人,京畿、河南、浙江、山西、陜西、湖廣、福建道各二人,四川、廣東、廣西、雲南、貴州道各一人。漢軍八人;協理河南道一人,餘隸江南等五道。康熙三十九年省入漢缺。漢員,江南道五人,內掌印一人。十八年省一人。康熙七年省二人。雍正四年增一人。乾隆十四年增一人。浙江道六人,內掌印一人。九年省一人,十八年省二人。康熙七年省一人。雍正四年增一人。乾隆十四年省一人。江西道六人,十六年省一人,十八年省三人。康熙七年省一人,雍正四年增一人。乾隆十四年省一人。福建道五人,十年省一人。康熙七年省二人。湖廣道六人,八年、九年、十五年俱省一人。康熙七年省一人。雍正四年增一人。乾隆十四年復省一人。河南道六人,內掌印一人。十年、十八年俱省一人。康熙七年省二人。乾隆六年增一人,十四年復省一人。山東道五人,內掌印一人。十八年省二人。康熙七年省一人。乾隆十四年增一人。山西道五人,內掌印一人。十年省一人,十八年省二人。乾隆六年增一人,十四年省一人。陜西道四人,內掌印一人。十八年省二人。雍正四年增一人。乾隆十四年省一人。四川道四人,十八年省二人。康熙七年省一人。雍正四年增一人。乾隆十四年省一人。廣東道五人,十八年省二人。康熙七年省二人。雍正四年增一人。乾隆十四年省一人。廣西道、雲南道各四人,十八年省二人。康熙七年各省二人。乾隆十四年各省二人。貴州道四人。十八年省二人。康熙七年省一人。雍正四年增一人。乾隆十四年省一人。京畿道無專員。乾隆十四年定滿、漢各一人。啟心郎,滿洲、漢軍各一人,十五年俱省。蒙古章京二人。康熙元年省。筆帖式,滿洲五十有一人,康熙三十八年省十有六人。漢軍七人。康熙三十八年省二人。雍正十二年置蒙古二人。光緒三十三年,滿、蒙、漢軍共酌留三十人。中、東、西、南、北五城兵馬司指揮各一人,副指揮各二人,康熙十一年省五城各一人。乾隆三十一年改東、西、南、北四城副指揮分駐朝陽、永定、阜成、德勝諸門外,鈐轄關廂,中城如故。吏目各一人。是歲定左都御史、左副都御史、監察御史許風聞言事。給事中同。二年,省京畿道。三年,定左副都御史滿、漢各一人。九年,復設京畿道,專司照刷各署卷宗。乾隆十四年改歸河南道。光緒三十二年停止刷卷。並置五城漢軍理事官,是為巡城之始。十年,定滿洲、漢軍、漢五城御史各一人。十八年各增滿員一人。雍正元年定滿、漢各一人。乾隆三十九年漢軍停開列。康熙二十九年,命左都御史馬齊同理籓院尚書阿喇尼列議政大臣。故事,二院長官俱不豫議政,豫議自此始。五十七年,增置蒙古監察御史二人。滿缺改。雍正二年,置內務府御史四人。十三年省。乾隆三年復置二人,本院御史內奏派。光緒三十二年停。五年,增置宗室御史二人。滿缺改。乾隆十四年復改二人,通舊為四人。七年,置五城鋪司巡檢各一人。乾隆初省。乾隆十四年,詔按道定額。先是設十五道,唯河南、江南、浙江、山東、山西、陜西六道授印信,掌印者曰掌道,餘曰協道,京畿道亦給印信,未設專官。湖廣等八道分隸之,曰坐道,不治事。掌河南道兼理福建道,掌江南道兼理江西、四川道,掌浙江道兼理雲南道,掌山東道兼理廣西道,掌山西道兼理廣東、貴州道,掌陜西道兼理湖廣道。至是各道並給印信,規制始稱。二十年,復命京畿道列河南道前,互易所掌,京畿道遂為要職。光緒三十二年,改定都御史一人、副都御史二人,按省分道。增設遼沈道,仿京畿道例,置掌道、協道各二人;析江南為江蘇、安徽二道,湖廣為湖北、湖南二道;並增甘肅、新疆二道,置滿、漢御史各一人。是為二十道。令訪求利病,專司糾察,後設之外務、農工商、民政諸部事件,多不關報。舊制,各部及各衙門分道稽察,至是停止。其制已灑然非舊云。

順治初,又有巡按御史,省各一人。十七年省。巡鹽御史,兩淮、兩浙、長蘆、河東各一人。十年停,十二年復故。康熙十一年停,尋復置。三十年復差福建、兩廣各一人。五十九年停兩廣鹽差。雍正元年停福建鹽差。明年停長蘆、河東鹽差。四年停兩浙鹽差。巡漕御史一人。十四年停。雍正七年定差淮安、通州各二人。乾隆二十年改差淮安、濟寧、天津、通州各一人。十七年增差通州四人。二十三年停差天津一人。二十六年復差天津一人。嘉慶十三年定科、道並差。道光二年俱停。巡視京、通各倉御史一人。七年停,八年復故。康熙七年又停。二十年定差滿、漢各一人,二十六年再停。雍正元年置巡察御史一人,總查倉弊。五年改京、通倉各差一人。乾隆十七年定科、道並差。四十三年增差內倉一人。五十九年改令科、道監放,停差查倉官。嘉慶四年復故。光緒二十八年又停。巡視江南上下兩江御史二人。六年省。巡視屯田御史一人。四年省。督理陜甘洮宣等處茶馬御史一人。康熙七年省,三十四年復故,四十二年又省。雍正間,置巡察各省御史,江寧、安徽一人,湖北、湖南一人,山東、河南一人。巡視吉林、黑龍江科道,滿洲二人。稽察奉天文武衙門御史一人。巡視山東、河南工務御史一人。直隸巡查御史:順天、永平、宣化二人,保定、正定、河間二人,順德、廣平、大名二人。巡農御史一人。先後俱省。

六科給事中,吏、戶、禮、兵、刑、工六科掌印給事中,滿、漢各一人。初制,滿員四品,漢員七品。康熙二年改滿員七品,六年復為四品。九年俱定七品。雍正七年升正五品。光緒三十二年升正四品。給事中,滿、漢各一人。初制七品。雍正七年升正五品。筆帖式八十人。吏、戶、兵、刑各十有五人,禮、工各十人。光緒三十二年酌留三十人。掌言職,傳達綸音,勘鞫官府公事,以註銷文卷;吏科分稽銓衡,註銷吏部、順天府文卷。戶科分稽財賦,言主銷戶部文卷。禮科分稽典禮,註銷禮部、宗人府、理籓院、太常寺、光祿寺、鴻臚寺、國子監、欽天監文卷。兵科分稽軍政,註銷兵部、鑾輿衛、太僕寺文卷。刑科分稽刑名,註銷刑部文卷。工科分稽工程,註銷工部文卷。有封駁即聞。

初沿明制,六科自為一署,給事中無員限,並置漢軍副理事官。順治十八年,定滿、漢都給事中,左、右給事中,各一人,都給事中由左給事中轉,左給事中由右給事中轉。漢給事中二人,省副理事官。康熙三年,六科止留滿、漢各一人。五年,改都給事中為掌印。雍正初,以六科內升外轉,始隸都察院。凡城、倉、漕、鹽與御史並差,自是臺省合而為一。光緒三十二年,省六科名,別鑄給事中印,額定二十人。

通政使司通政使,初制,滿員二品,漢員三品。順治十六年,並定為三品。康熙六年復故。九年仍改定正三品。副使,初制,滿員三品,漢員四品。順治十六年並定為四品,康熙六年復故,九年仍改定正四品。參議,初制,滿員四品,漢員五品。順治十六年並定正五品。俱滿、漢各一人。其屬:經歷司經歷、正七品。知事,初制四品,後改正七品。滿、漢各一人。筆帖式,滿洲六人,漢軍二人。

通政使掌受各省題本,校閱送閣,稽覈程限,違式劾之。洪疑大獄,偕部、院豫議。副使、參議佐之。經歷、知事,分掌出納文移。其兼領者:登聞鼓,以參議一人分直,知事帥役巡察。筆帖式,滿洲、漢軍各一人,掌敘雪冤滯,誣控越訴者論如法。

初,順治元年,詔:「自今內外章奏,俱由通政司封進。」置滿、漢通政使各一人,左通政使各一人。漢右通政使二人。乾隆十年省一人,十三年俱省。左參議,滿、漢各二人。康熙五十三年省漢一人。乾隆十三年各省一人。右參議,漢二人。康熙三十八年省一人。乾隆十三年俱省。滿、漢司務各一人。後改經歷。知事,滿洲二人、漢軍一人。乾隆十七年改滿、漢各一人。康熙六十一年,以登聞鼓筆帖式來屬。故事,通狀、通政司狀。鼓狀,登聞院狀。紛爭無已。自控訴者赴都察院,以給事中或御史一人主受訴訟,至是停科道差,改隸本司。乾隆十三年,改左通政使為副使,去左、右銜;參議亦如之。光緒二十四年,省入內閣,尋復故。二十八年,以改題為奏,職無專司,復省。

大理寺卿,初制,滿員二品,漢員三品。順冶十六年並定為三品。康熙六年復故,九年仍改定正三品。少卿,初制,滿員三品,漢員四品。順治十六年並定為四品。康熙六年復故,九年仍改定正四品。俱滿、漢一人。其屬:堂評事,初制四品。順治十六年改七品。康熙六年升五品,九年定正七品。滿洲一人。司務司務,滿、漢各一人。左、右寺丞,初制,滿員四品,漢員六品。順治十六年並定為六品。康熙六年升五品。九年仍改定正六品。滿洲、漢軍、漢俱各一人。左、右評事,漢各一人。筆帖式,滿洲四人,漢軍二人。

卿掌平反重闢,以貳邦刑。與刑部、都察院稱三法司。凡審錄,刑部定疑讞,都察院糾覈。獄成,歸寺平決。不協,許兩議,上奏取裁。並參豫朝廷大政事。少卿佐之。寺丞掌覈內外刑名,質成長官,參糾部讞。評事掌繕左、右兩寺章奏。

順治元年,定滿、漢卿各一人。少卿滿洲一人、漢二人。乾隆十三年省一人。滿寺丞一人。正五品。康熙三十八年省。漢司務二人。十五年定滿、漢各一人。左、右寺正,正六品。滿洲、漢軍、漢各一人;左、右寺副,從六品。漢各一人。康熙三十八年省。堂評事,滿、漢各一人;康熙三十八年省漢軍一人。左、右評事,漢各一人。十一年,差寺正、寺副各一人充各省恤刑官。刑部差郎中、員外郎十三人。尋省。乾隆十七年,改左、右寺正為寺丞。光緒二十四年,省入刑部,尋復故。三十二年,更寺為院。

翰林院掌院學士,初制正五品。順治元年升正三品。雍正八年升從二品。大學士、尚書內特簡。滿、漢各一人。侍讀學士、初制從四品。光緒二十九年升正四品。侍講學士,初制從四品。宣統元年升正四品。滿洲各二人,漢各三人。侍讀、初制正六品。雍正三年升從五品。光緒二十九年升正五品。宣統元年升從四品。侍講,初制正六品。雍正三年升從五品。宣統元年升從四品。滿洲各三人,漢各四人。修撰、初制從六品。編修、初制正七品。檢討、初制從七品。自修撰以下,宣統元年並改從五品。庶吉士,由新進士改用。試博學鴻詞入式,或奉特旨改館職者,間得除授。光緒末停科舉,改由外國留學畢業及本國大學畢業者,廷試後授之,食七品俸。或徑授編修、檢討,與舊制殊。俱無定員。其屬:主事,滿洲二人,漢軍一人。典簿典簿、從八品。孔目,滿員從九品,漢員未入流。俱滿、漢各一人。待詔待詔,從九品。滿、漢各二人。筆帖式,滿洲四十人,漢軍四人。

掌院掌國史筆翰,備左右顧問。侍讀學士以下掌撰著記載。祭告郊廟神祇,撰擬祝文。恭上徽號、冊立、冊封,撰擬冊文、寶文,及賜內外文武官祭文、碑文。南書房侍直,尚書房教習,咸與其選。修實錄、史、志,充提調、總纂、纂修、協修等官。庶吉士入館,分習清、漢書,吏部疏請簡用大臣二人領教習事。初以內院學士為之,侍讀等官亦間有與者。後令掌院兼其職。康熙六年,始以工部尚書陳元龍領之,自是尚書、侍郎、內閣學士並得充之。是為大教習。其小教習由掌院選派,始於康熙三十三年。雍正間停止,高宗復舊制。侍讀、侍講司訓課,派編、檢二人提調館餼。三年考試,分別散留。辦事翰林,滿、漢各二人,雍正元年,命俸淺編、檢主定稿說堂,此清秘堂辦事翰林之始。厥後人數稍增,有奏辦、協辦之目。侍讀、侍講間亦為之。掌帥官治事。主事、典簿、孔目,掌章奏文移,董帥吏役。待詔掌繕寫校勘。

初,翰林之職隸內三院。順治元年,設翰林院,定掌院學士為專官,置漢員一人,兼禮部侍郎銜。侍讀學士、侍講學士各二人。十五年各增二人。侍讀、侍講各二人。十五年各增一人。修撰、編修、檢討、庶吉士,無定員。典簿二人,十五年改一人為滿缺。孔目一人,十五年增滿洲一人。俱漢人為之。明年,省入內三院。十五年,復舊制,增滿洲掌院學士一人,兼銜如故。乾隆五十八年停。置待詔六人。滿員四人,滿員二人。十八年,復歸內三院。康熙九年,定滿、漢侍讀學士、侍講學士、侍讀、侍講,各三人;乾隆五十年省滿洲各一人。光緒二十九年增侍讀、侍講滿、漢各二人。典簿、孔目各一人,待詔各二人。康熙九年定滿、漢各一人。十六年,命侍講學士張英等入直南書房。先是詔冊詞命多由院擬,至是始為西清專職。後改歸軍機處。二十八年,以院務隳廢,命大學士徐元文兼掌院事,重臣兼領自此始。明年定尚書、侍郎、左都御史俱得兼攝。光緒二十九年,增置堂主事,滿洲二人、漢一人。是歲省詹事府,以詞臣敘進無階,增置滿、漢學士正三品。各一人,撰文正六品。宣統元年升正五品。各二人。三十三年,增置秘書郎,從六品。宣統元年升正五品。滿、漢各二人。並設講習館,令翰林官研習學科,備各部丞、參選。宣統元年,復崇侍講學士以下品秩,停止外班升用。初制、翰、詹出缺,編、檢不敷升轉,以部、院科甲出身司員升用,是為外班。初制,進士論甲第,修撰、編修、檢討不分升降。順治間,授編修程芳朝等為修撰,檢討李霨等為編修,姜元衡以編修降檢討,不為定制。又內三院編修等官不必盡由科目,靳輔、劉兆麟等並以官學生授編修,蓋亦創舉。庶吉士舊隸內弘文院,後設本院,始來屬。雍正十三年,建庶常館。故事,散館後始授職,然亦有未選庶常而遽授者,均異數也。

文淵閣領閣事三人,掌典綜冊府。大學士、協辦大學士、掌院學士兼充。直閣事六人,掌典守釐緝。內閣學士、少詹事、講讀學士兼充。校理十有六人,掌註冊點驗。庶子、講、讀、編、檢兼充。檢閱八人。內閣中書派充。內務府司員、筆帖式各四人。由提舉閣事大臣番選奏充。

國史館總裁,特簡,無定員。掌修國史。清文總校一人。滿洲侍郎內特簡。提調,滿洲、內閣侍讀學士或侍讀派充。蒙古、內閣蒙古堂或理籓院員司派充。漢翰林院侍讀學士以下官派充。各二人。總纂,滿洲四人,蒙古二人,漢六人。纂修、協修,無定員。蒙古由理籓院司官充。滿、漢由編、檢充。校對,滿、蒙、漢俱各八人。內閣中書充。光緒間,增置筆削員十人。

經筵講官,滿、漢各八人,掌進讀講章,敷陳訓典。歲仲春、仲秋兩舉之。滿員由大學士以下、都察院副都御史以上各官兼充。漢員由大學士、尚書、侍郎、副都御史、掌院學士、侍讀學士、侍講學士、詹事府詹事、少詹事、國子監祭酒等官,由翰林出身者兼充。講官滿、漢各二人。翰林院請旨簡派。

初制以大學士知經筵事。後定經筵講官滿、漢各六人,閣臣遂不進講。自徐元文、熊賜履輩相繼以尚書擢大學士,仍與兼充,嗣是以為常。宣統初,各部丞、參亦間有與者。

起居注館,日講起居注官,滿洲十人,漢十有二人。由翰、詹各官簡用。唯滿、漢掌院學士例各兼一缺。主事,滿洲二人,漢一人。以科甲出身者充之。筆帖式,滿洲十有四人,漢軍二人。日講官掌侍直起居,記言記動。經筵臨雍,御門聽政,祭祀耕耤,朝會燕饗,勾決重囚,並以二人侍班。凡謁陵、校獵、巡狩方岳,請旨、扈從、侍直,敬聆綸音,退而謹書之。月要歲會,貯置鐵,送內閣尊藏。主事掌出納文移,校勘典籍。

初,天聰二年,命儒臣分兩直,巴克什達海等譯漢字書,即日講所繇始,巴克什庫爾纏等記注政事,即起居注官所繇始。順治十二年,始置日講官。康熙九年,始設起居注館,在太和門西廡。置滿洲記注官四人,漢八人,以日講官兼攝。十二年增滿洲一人,漢二人。十六年復增滿洲一人。二十年增漢八人。三十年定漢員十有二人。時日講與起居注各自為職,並置滿洲主事二人,漢軍一人。五十七年省。雍正元年置滿洲二人。十二年增漢一人。二十五年停日講,其起居注官仍系銜「日講」二字。五十七年,省起居注館,改隸內閣,遇理事日,以翰林官五人侍班。雍正元年,復置日講起居注,滿洲六人,漢十有二人。乾隆元年,增滿員二人。嘉慶八年,復增滿員二人。於是日講、起居注合而為一。

詹事府詹事,正三品。少詹事,正四品。左春坊左庶子,正五品。左中允,正六品。左贊善,從六品。右春坊右庶子,右中允,右贊善,品秩俱同左。司經局洗馬,從五品。俱滿、漢各一人。其屬:主簿主簿,從七品。滿、漢各一人。筆帖式,滿洲六人。

詹事、少詹事掌文學侍從。經筵充日講官。編纂書籍,典試提學,如翰林。並豫秋錄大典。左、右春坊各官掌記注撰文。洗馬掌圖書經籍。主簿掌文移案牘。

順治元年,置少詹事一人,掌府事。其冬省入內三院。九年,復置詹事一人,少詹事二人,主簿一人,錄事、通事舍人各二人。並從九品。左、右春坊庶子、諭德各一人,中允、贊善各二人,司經局洗馬一人,正字二人,從九品。俱漢人為之,令內三院官兼攝。專置滿洲詹事一人,掌府印。十五年,省詹事府官。康熙十四年,復置滿、漢詹事各一人,漢員兼翰林院侍讀學士銜。少詹事各二人,漢員兼翰林院侍講學士銜。三十七年省滿員一人。乾隆十三年省漢員一人。主簿各一人,錄事各二人。三十七年省滿缺,留漢一人。五十二年俱省。左、右春坊置滿、漢左、右庶子各一人,滿員以四品冠帶食五品俸,左、右同。漢左庶子兼翰林院侍讀銜,右庶子兼翰林院侍講銜。左、右諭德各一人,漢員兼翰林院修撰銜。三十七年省滿右諭德一人。五十七年省漢右諭德一人。乾隆十三年俱省。左、右中允各二人,滿員以五品冠帶食六品俸。漢員兼翰林院編修銜。三十七年省滿員各一人。明年,省漢右中允一人。五十二年省漢左中允一人。左、右贊善各二人。漢員兼翰林院檢討銜。三十七年省滿員各一人。明年,省漢右贊善一人。五十二年省漢左贊善一人。司經局滿、漢洗馬各一人,漢員兼翰林院修撰銜。以上各兼銜,俱乾隆五十四年停。正字各二人。三十七年省滿員缺。明年,省漢一人。例以應選內閣中書者除授,遂為中書兼銜。乾隆三十六年俱省。二十五年,命詹事湯斌、少詹事耿介等為皇太子講官,尚沿宮僚舊制。三十一年,命徐元夢入直上書房,皇子在上書房讀書,選翰林官分侍講讀,簡大臣為總師傅。總師傅之稱,自乾隆二十二年以介福、觀保等為總師傅始,曩時俱稱入直。嗣是本府坊、局止備詞臣遷轉之階。嘉慶二年,以府事改隸翰林院。五年,復舊制。光緒二十四年,仍省入翰林院,尋復故。二十八年,再省入。

太常寺管理寺事大臣一人。滿洲禮部尚書兼。卿,正三品。少卿,正四品。俱滿、漢各一人。其屬:寺丞,正六品。滿、漢各二人。贊禮郎,宗室二人,滿、漢二十有八人;初制,滿員四品。順治十六年改九品。康熙四年升六品,六年升五品,九年仍改九品。尋定由護軍校、驍騎校選授者六品職銜,八品筆帖式、庫監生選授者八品職銜,無品筆帖式、庫使、前鋒護軍選授者九品職銜。乾隆元年改定以六品冠帶食七品俸。學習,宗室四人,滿洲五人,漢十有四人。正九品。讀祝官,宗室一人,滿洲十有一人;初制五品。康熙九年改正九品。尋定品秩如贊禮郎,視出身為差。乾隆元年改定以六品冠帶食七品俸。學習,宗室三人,滿洲五人。正九品。博士博士,滿洲、漢軍、漢各一人。典簿典簿,滿、漢各一人。滿洲司庫一人,博士以下並正七品。庫使二人。正九品。筆帖式,滿洲九人,漢軍一人。

卿掌典守壇壝廟社,以歲時序祭祀,詔禮節,供品物,辨器類。前期奉祝版,稽百官齋戒,祭日帥屬以供事。少卿佐之。寺丞掌祭祀品式,辨職事以詔有司,並遴補吏員,勾稽廩餼。贊禮郎、讀祝官分掌相儀序事,備物絜器,並習趨蹌讀祝,祭祀各充執事。博士考祝文禮節,著籍為式,壇廟陳序畢,引禮部侍郎省,並歲覈祀賦。典簿掌察祭品,陳牲牢,治吏役。庫使掌守庫藏。

順治元年,設太常寺,隸禮部。置卿,少卿,滿、漢各一人。滿洲寺丞一人,光緒十二年增一人。漢左、右丞各一人。典簿,博士,滿、漢各一人。讀祝官,滿洲四人。康熙十年,禮部改隸二人,尋增額外二人。雍正十一年改正額。嘉慶四年增一人。道光元年增一人。咸豐二年增一人。贊禮郎,滿、雍正十一年增八人。乾隆三十七年改二人隸鑾輿衛補鳴贊鞭官。嘉慶四年增二人。道光元年增二人。咸豐二年增二人。漢康熙三十八年省二人。雍正元年復故。乾隆二年增二人,九年省四人。各十有六人。犧牲所正千戶、五年更名所牧。副千戶,五年更名所副。漢各一人。從七品。乾隆二十四年改滿缺。二十六年改隸內府。滿洲司庫一人。乾隆十一年省。十六年,改歸本寺。康熙二年,復隸禮部。十年,仍歸本寺。十五年,敕諸官肄習雅樂。雍正元年,特簡大臣綜理寺事,並增庫使二人。乾隆十三年,改寺丞為屬官。先是沿明舊制,丞為正官,議者病贅餘,至是體制始協。明年,定禮部滿洲尚書兼管太常職銜。四十年,增學習贊禮郎、四十六年增三人。嘉慶十六年增三人。讀祝官,四十六年增三人。嘉慶十六年增三人。滿洲各二人。光緒二十四年,增宗室學習贊禮郎四人、讀祝官三人。尋省入禮部,旋復故。三十二年,仍省入。

光祿寺管理寺事大臣一人。特簡。卿,從三品。少卿,初制,滿員、漢軍四品,漢員五品。順治十六年並定正五品。俱滿、漢各一人。其屬:典簿典簿,從七品。大官、珍饈、良醖、掌醢四署署正,初制,滿員四品,順治十六年改六品。康熙六年升五品,九年定從六品。漢員同。亦如之。署丞,初制六品。康熙九年定從七品。滿洲各二人。銀庫司庫,滿洲二人。筆帖式,滿洲十有八人。

卿掌燕勞薦饗,辨品式,稽經費。凡祭祀,會太常卿省牲,禮畢,進胙天子,頒胙百執事。蕃使廩餼,具差等以供。少卿佐之。大官掌供豕物,備器用,稽市直,徵菜地賦額致諸庫。珍饈掌供禽兔魚物,大祭祀供龍壺、龍爵,辨燕饗等差。良醖掌供酒醴,別水泉,量曲蘗,並大內牛酪。掌醢掌供醢醬,筵燕廩餼皆供其物,徵果園賦額致諸庫。典簿掌章奏文移。司庫掌庫帑出納。別設督催所、當月處,俱派員分治其事。

順治元年,設光祿寺,置滿、漢卿各一人。少卿,滿洲一人,漢二人。康熙三十八年省一人。漢寺丞一人。康熙三十八年省。滿、漢典簿各一人。大官、珍饈、良醖、掌醢四署,滿、漢署正各一人;滿洲署丞各一人,康熙三十八年各增一人。漢署丞、十五年省。監事,十二年省。俱各一人。滿洲司庫二人。司牲司,漢大使一人。十五年省。凡事並由禮部具題,劄寺遵行。十年,定各省額解銀米徑送禮部,並司府、州、縣考成。十五年,仍歸本寺。十八年,復隸禮部。錢糧由寺奏銷,考成仍歸禮部。康熙三年,以禮部清釐無法,復改儲戶部。十年,仍以禮部精膳司所掌改歸本寺。乾隆十三年,始命大臣兼管寺事。光緒二十四年,省入禮部,尋復故。三十二年,仍省入。

鴻臚寺管理寺事大臣各一人。滿洲禮部尚書兼。卿,初制,滿員從三品,漢員正四品。順治十六年並定正四品。少卿,從五品。俱滿、漢各一人。其屬:鳴贊,從九品。滿洲十有四人,漢二人;學習,滿洲四人。序班,從九品。漢四人;學習,八人。主簿,從八品。滿、漢各一人。筆帖式,滿洲四人。

卿掌朝會、賓饗贊相禮儀,有違式,論劾如法。少卿佐之。鳴贊掌儐導贊唱。序班掌百官班次。主簿職掌同太僕寺。

順治元年,設鴻臚寺,置滿、漢卿各一人。滿洲少卿一人,漢左、右少卿各一人。十五年省一人。漢左、右寺丞各一人。正六品。十五年省一人。康熙五十二年省一人。滿、漢主簿各一人。鳴贊,滿洲十有六人,乾隆三十七年改隸鑾輿衛二人。漢八人。二年省一人,十二年省一人,十三年省二人。乾隆七年省二人。序班二十有二人,十五年省十人。康熙三十八年省六人。乾隆七年省二人。司賓序班二人,乾隆二年省。學習序班無恆額。雍正六年定以直隸、山東、山西、河南儒學生內考取。乾隆九年定為十二人。十七年定直隸六人,餘各二人。十七年省山東等省四人。凡事由禮部具題,十六年改歸本寺,十八年仍隸禮部。康熙十年復故,雍正四年復歸禮部統轄。乾隆十四年,始以滿洲尚書領寺事。五十九年,增置滿洲學習鳴贊四人。光緒二十四年,省入禮部,尋復故。三十二年,仍省入。

國子監管理監事大臣一人。滿、漢大學士、尚書、侍郎內特簡。祭酒,從四品。初制滿員三品。順治十六年俱定從四品。滿、漢各一人。司業,正六品。滿、蒙、漢各一人。其屬:繩愆監丞,初制,滿員五品,漢員八品。後並改正七品。博士博士,從七品。初制,漢員八品。乾隆元年改同滿員。典簿典簿,從八品。俱滿、漢各一人。典籍典籍,從九品。漢一人。率性、修道、誠心、正義、崇志、廣業六堂:助教,初制,從八品。乾隆元年升從七品。學正,學錄,率性、修道、誠心、正義四堂曰學正,崇志、廣業二堂曰學錄。初制,學正正九品,學錄從九品。乾隆元年並升正八品。各一人。八旗官學助教,俱滿洲二人,蒙古一人。教習,俱滿洲一人,蒙古二人,漢四人。恩、拔、副、優貢生內選充。筆帖式,滿洲四人,蒙古、漢軍各二人。

祭酒、司業掌成均之法。凡國子及俊選以時都授,課第優劣。歲仲春、秋上丁,釋奠,釋菜,綜典禮儀。天子臨雍,執經進講,率諸生圜橋觀聽。新進士釋褐,坐彞倫堂行拜謁簪花禮。監丞掌頒規制,稽勤惰,均廩餼,覈支銷,並書八旗教習功過。博士掌分經教授,考校程文,偕助教、學正、學錄經理南學事宜。典簿掌章奏文移。典籍掌書籍碑版。其兼領者:算法館,漢助教二人,特簡滿洲文臣一人管理。俄羅斯館,滿、漢助教各一人。琉球學,漢教習一人。肄業貢生選充。後俱省。又檔子房,錢糧處,俱派員司其事。

初,順治元年,定滿、漢祭酒各一人,兼太常寺少卿銜。滿洲司業二人,乾隆十三年省一人。蒙、漢各一人,兼太常寺寺丞銜。後停兼銜。滿、漢監丞、典簿俱各一人,漢博士三人。十年省一人。康熙五十二年省一人。建八旗官學,置滿洲助教十有六人,康熙五十七年省四人。雍正三年復故。蒙古八人。十八年省四人。雍正三年復故。分設六堂,置滿、漢助教,十五年省六人。康熙五十七年省四人。雍正三年復增四人。學正,康熙三十八年省一人。五十二年省二人。各十有二人;學錄六人,十五年省四人。典籍一人。隸禮部。十五年復故。十八年,置滿洲博士一人。康熙二年,復隸禮部。十年,仍歸本監。雍正元年,詔監丞等官停用捐納。明年,特簡大臣管監事。九年,建南學。在學肄業者為南學,在外肄業赴學考試者為北學。高宗涖治,鄉用儒術,以大學士趙國麟、尚書楊時、孫嘉淦領太學事,官獻瑤、莊亨陽輩綜領六堂,世號「四賢五君子」。乾隆四十八年,建闢雍於集賢門,國學規制斯為隆備。道光三年,以成均風勵中外,詔監臣無曠厥職。光緒三十三年,省入學部。嗣以文廟、闢雍典禮隆重,特置國子丞以次各官,分治其事。

初制,詔各省選諸生文行兼優者,與鄉試副榜貢生,入監肄業。聖祖初政,給事中晏楚瀾疏停鄉試副榜貢生,遂不復舉。康熙五年,徐元文為祭酒,始請學政間歲一舉優生,鄉試仍取副榜,自是為恆制。光緒間,並推廣舉人入監,時風稍振。未幾科舉廢,此制替已。

衍聖公孔氏世襲。正一品。順治元年,授孔子六十五世孫允植襲封。其屬:司樂,典籍,屯田管勾,俱由衍聖公保舉題授。管勾之屬,屯官八人,分掌鉅野、鄆城、平陽、東阿、獨山五屯。林廟守衛司百戶,秩視衛守備。以上為兵、農、禮、樂四司。知印,掌書,書寫,奏差,啟事,各一人。隨朝伴官六人。初制一人。乾隆十五年定為六人。自司樂以下,俱正七品,由衍聖公保舉題授或題補。聖廟執事官四十人。三品二人,四品四人,五品六人,七品八人,八品、九品各十人,由衍聖公會同山東學政揀選孔氏族人充補。翰林院世襲五經博士,正八品。孔氏北宗一人,順治元年,授孔子六十五世孫允鈺,奉子思廟祀。南宗一人。自明彥繩授職後,數世未襲。康熙四十一年,始授孔子六十六世孫興醽主衢州廟祀。東野氏、康熙二十三年,授元聖周公七十三世孫東野沛然。姬氏、乾隆四十三年,授周公七十七世孫肇勛,主咸陽廟祀。顏氏、順治元年,授復聖顏子淵六十八世孫紹緒。曾氏、順治元年,授宗聖曾子輿六十四世孫文達。孟氏、順治元年,授亞聖孟子子輿六十三世孫貞仁。仲氏、順治二年,授先賢仲子路六十一世孫於升。閔氏、康熙三十八年,授先賢閔子騫六十五世孫衍籀。冉氏、雍正二年,授先賢冉子伯牛六十五世孫士樸。冉氏、雍正二年,授先賢冉子仲弓六十七世孫天琳。端木氏、康熙三十八年,授先賢端木子貢七十世孫謙。卜氏、康熙五十九年,授先賢卜子夏六十四世孫尊賢。言氏、康熙五十一年,授先賢言子游七十三世孫德堅。顓孫氏、雍正二年,授先賢顓孫子張六十六世孫誠道,道光四年,改歸嫡長樹勛。有氏、乾隆五十三年,授先賢有子若七十二世孫守業。伏氏、嘉慶十年,授先儒伏子勝六十五世孫敬祖。韓氏、乾隆三年,授先儒韓子愈三十世孫法祖。張氏、康熙二十六年,授先儒張子載二十八世孫守先,主鳳翔廟祀。邵氏,康熙四十一年,授先儒邵子雍三十世孫延祀。俱各一人。硃氏二人。順治十二年,授先儒硃子熹徽派十五世孫煌,奉婺源廟祀。康熙二十九年,授閩派十八世孫溁,主建安廟祀。關氏三人。康熙五十八年,授關公羽五十七世孫霨,主洛陽廟祀。雍正四年,授五十二世孫居斌,奉解州廟祀。十三年,授五十二世孫朝泰,主當陽廟祀。其屬於孔氏者,又有太常寺世襲博士一人;正七品。順治九年,以孔允銘暫主聖澤書院祀。康熙二十六年,授六十七世孫毓琮。國子監學正一人;正八品。順治八年,授六十五世孫允齊,由衍聖公保舉。尼山書院學錄,正八品。順治元年,授六十二世孫聞然,由衍聖公咨送弟侄題補。洙泗書院學錄,順治元年,授六十四世孫尚澄。世襲六品官,由衍聖公揀選族人充補。各一人;孔、顏、曾、孟四氏教授,正七品。學錄歷俸六年升補。學錄,由衍聖公咨送孔氏生員題補。後改由移送撫臣驗看,送部具題。各一人。

衍聖公掌奉至聖闕里廟祀。聖賢後裔翰博各掌奉其先世祀事。聖裔太常博士掌奉聖澤書院祀。國子監學正掌奉儀封聖廟祀。學錄分掌尼山、洙泗兩書院祀。世襲六品官掌分獻崇聖祠。四氏教授、學錄掌訓課四氏生徒。執事官掌祭祀分獻,並司爵帛香祝。司樂掌樂章、樂器。典籍掌書籍及禮生。管勾掌祀田錢穀出入。百戶掌陵廟戶籍,典守樂器,祭祀則司滌濯。知印、掌書、書寫掌文書印信。奏差掌齎表箋章疏。隨朝伴官掌隨從朝覲辦事。

順治元年,復衍聖公及四氏翰博等爵封,命孔允植入覲,班列閣臣上。明年,改錫三臺銀印。十六年改滿、漢文三臺銀印。乾隆十四年,復改清、漢篆文三臺銀印。九年,世祖視學釋奠,召衍聖公孔興燮及四氏博士赴京陪祀觀禮,自後以為常。十三年,依例授光祿大夫。康熙六十一年,定錫廕視正一品,廕一子五品官,著為例。舊制,衍聖公錫廕依正二品。雍正八年,以崇奉祀典,廣置聖廟執事官,各按品級給予章服。乾隆二十一年,改世職知縣孔傳令為世襲六品官。先是曲阜知縣為世職,由衍聖公選族人題授。至是改為在外揀選調補。五十年,詔:「博士有枉法婪贓革職治罪者,停其承襲。」定例衍聖公歸長子襲,北宗博士次子襲,太常博士三子襲,餘並以嫡子襲。東野氏及聖門各賢裔,由衍聖公達部上名,餘各報部云。

欽天監管理監事王大臣一人。特簡。監正,初制,滿員四品。康熙六年升三品。九年,滿、漢並定正五品。左、右監副,初制,五品。康熙六年升四品,九年定正六品。俱滿、漢各一人。其屬:主簿主簿,正八品。滿、漢各一人。時憲科五官正,從六品。滿、蒙各二人,漢軍一人。春官正、夏官正、中官正、秋官正、冬官正,並從六品。漢各一人。司書,正九品。漢一人。博士,從九品。滿洲四人,蒙古二人,漢軍一人,漢十有六人。天文科五官靈臺郎,從七品。滿洲二人,蒙古、漢軍各一人,漢四人。監候,正九品。漢一人。博士,滿洲四人,漢二人。漏刻科挈壺正,從八品。滿、蒙各一人,漢二人。司晨,從九品。漢軍一人,漢七人。筆帖式,滿州十有一人,蒙古四人,漢軍二人。天文生,食九品俸。滿、蒙各十有六人,漢軍八人,漢二十有四人。食糧天文生,漢五十有六人。食糧陰陽生,漢十人。並給九品冠帶。助教助教一人,教習二人。

監正掌治術數,典歷象日月星辰,宿離不貸。歲終奏新歷,送禮部頒行。監副佐之。時憲科掌推天行之度,驗歲差以均節氣,制時憲書,以國書、蒙文譯布者,滿、蒙五官正司之。推算日月交食、七政相距、沖退留伏、交宮同度,漢五官正司之。頒之四方。天文科掌觀天象,書云物禨祥;率天文生登觀象臺,凡晴雨、風雷、雲霓、暈珥、流星、異星,匯錄冊簿,應奏者送監,密疏上聞。漏刻科掌調壺漏,測中星,審緯度;祭祀、朝會、營建,諏吉日,辨禁忌。主簿掌章奏文移,簿籍員數。天文生分隸三科,掌司觀候推算。陰陽生隸漏刻科,掌主譙樓直更,監官以時考其術業而進退之。助教掌分教算學諸生。

初,順治元年設欽天監,分天文、時憲、漏刻、回回四科,置監正、監副、五官正、保章正、挈壺正、靈臺郎、監候、司晨、司書、博士、主簿等官,並漢人為之,行文具題隸禮部。是歲仲秋朔日食,以西人湯若望推算密合,大統、回回兩法時刻俱差。令修時憲,領監務。十四年,省回回科,改其職隸秋官正,尋復舊制。十五年,定與禮部分析職掌。康熙二年,仍屬禮部。明年,增置天文科滿洲官五人,滿員入監自此始。又明年,定滿、漢監正各一人,左、右監副各二人,主簿各一人,滿、蒙五官正各二人。省回回科博士仍隸秋官正。置漢軍秋官正一人,春、夏、中、秋、冬五官正漢各一人。滿洲靈臺郎三人,乾隆四十七年改一人為蒙古員缺。漢軍一人,漢四人。滿洲挈壺正二人,乾隆四十七年改一人為蒙古員缺。漢二人。漢監候一人,保章正二人,正八品。十四年省。司書二人。十四年省一人。漢軍司晨一人,漢一人。十四年省。滿洲博士六人,乾隆四十七年改一人為蒙古員缺。漢軍二人,漢三十有六人。尋省十四人,五年復置二人,通舊二十有四人。並定監官升轉不離本署,積勞止加升銜,著為例。先是新安衛官生楊光先請誅邪教,鐫若望職。至是以光先為監副,尋升監正,仍用回回法。南懷仁具疏訟冤。八年,復罷光先,以南懷仁充漢監正,更名監修,用西法如初。雍正三年,實授西人戴進賢監正,去監修名。八年,增置西洋監副一人。乾隆四年,置漢算學助教一人,隸國子監。十年,定監副以滿、漢、西洋分用。十八年省滿、漢各一人,增西洋二人,分左、右。四十四年,更命親王領之。道光六年,仍定滿、漢監正各一人,左、右監副各二人。時西人高拱宸等或歸或沒,本監已諳西法,遂止外人入官。光緒三十一年,改國子監助教始來隸。

太醫院管理院事王大臣一人。特簡。院使,初制正五品。宣統元年升正四品。左、右院判,初制正六品。宣統元年升正五品。俱漢一人。其屬:御醫十有三人,內兼首領事二人。初制正八品。雍正七年升七品,給六品冠帶。宣統元年升正六品。吏目二十有六人,內兼首領事一人。初制八、九品各十有三人。宣統元年,改八品為七品,九品為八品。醫士二十人,內兼首領事一人,給從九品冠帶。醫生三十人。

院使、院判掌考九科之法,帥屬供醫事。御醫、吏目、醫士各專一科,曰大方脈、小方脈、傷寒科、婦人科、瘡瘍科、針灸科、眼科、咽喉科、正骨科,是為九科。初設十一科。後痘疹科歸小方脈,咽喉、口齒並為一科。掌分班侍直,給事宮中曰宮直,給事外廷曰六直。西苑壽樂房以本院官二人直宿。

順治元年,置院使,左、右院判各一人,吏目三十人,十八年省二十人,康熙九年復故。十四年省十人,雍正元年又復。豫授吏目十人,十八年省。康熙九年復故,三十一年又省。御醫十人,康熙五十三年省二人。雍正元年復故,七年增五人。道光二十三年省二人。醫士二十人。十八年省二十人。康熙九年復故,十四年省十人。雍正元年增二十人。凡藥材出入隸禮部。十六年,改歸本院。十八年,生藥庫復隸禮部。康熙三年,定直省歲解藥材,並折色錢糧,由戶部收儲付庫。雍正七年,定八品吏目十人,九品二十人。後定各十三人。乾隆五十八年,命內府大臣領院務。宣統元年,院使張仲元疏請變通舊制,特崇院使以次各官品秩。初制,入院肄業,考補恩糧,歷時甚久,軍營、刑獄醫士悉由院簡選。光緒末葉,民政部醫官,陸軍部軍醫司長,與院使、院判品秩相等。至是釐定,崇內廷體制也。又定制,院官遷轉不離本署。同治間,曾議吏目食俸六年,升用按察司經歷、州判。嗣以與素所治相剌,乃寢。

壇廟官天壇尉,地壇尉,各八人。五品一人,六品七人。太廟尉十人。四品二人,五品八人。社稷壇尉五人。五品一人,六品四人,並隸太常寺。堂子尉八人。七品二人,八品六人,隸禮部。俱滿員。掌管鑰,守衛直宿,朔望奉薌以行禮。天壇、地壇、朝日壇、夕月壇、先農壇,各祠祭署奉祀、從七品。祀丞,從八品。俱各一人。日、月二壇祀丞後省。帝王廟祠祭署無專員。以漢贊禮郎、司樂內一人委充,並隸樂部。俱漢員。掌典守神庫,以時巡視,督役氾埽;凡葺治墻宇、樹藝林木,並敬供厥事。四品尉以五品序升,其下以是為差。唯太廟尉以各壇六品尉及各部院休致郎員間次選授。六品等尉吏部牒八旗番送除授,奉祀以祀丞序升,祀丞以祝版生番選除授。

陵寢官三陵總理事務大臣,盛京將軍兼充。光緒三十年改歸東三省總督。承辦事務衙門大臣,光緒三十一年,改盛京守護大臣置。各一人。主事,委署主事,各一人。讀祝官八人。贊禮郎十有六人。四品、五品、七品官各一人,六品官四人,外郎九人。舊置戶部六品官二人。禮部六、七品官,工部四、五、六品官,各一人。又戶、禮、工三部外郎二十人。光緒三十一年,省外郎十有一人。自讀祝以下,並改隸三陵總理事務衙門。永陵:掌關防官,四品。副關防官兼內管領,正五品。副關防官兼尚膳正,五品。尚茶副,尚膳副,副內管領,並八品。各一人。筆帖式二人。福陵、昭陵:掌關防官各一人,副關防官各二人。五品。尚茶正,尚膳正,並五品。尚茶副,尚膳副,內管領,正五品。副內管領,俱各一人。筆帖式各二人。掌守衛三陵。凡班直、饗獻、氾埽,以時分司其事。

東陵:總管大臣一人。泰寧鎮總兵兼內務府大臣簡充。承辦事務衙門禮部主事,筆帖式,各二人。石門衙署工部郎中一人。員外郎,筆帖式,各四人。昭西陵:內務府掌關防郎中,嘉慶十五年調往景陵,仍管昭西陵事務。員外郎,主事,尚茶正,尚膳正,並四品。內管領,各一人。筆帖式二人。禮部郎中一人。員外郎,讀祝官,各二人。贊禮郎,筆帖式,各四人。工部郎中一人。孝陵:內務府掌關防郎中,員外郎,主事,尚茶正,尚膳正,內管領,副內管領,正六品。各一人。筆帖式二人。禮部郎中一人。員外郎,讀祝官,各二人。贊禮郎四人。筆帖式二人。工部員外郎一人。孝東陵:內務府掌關防郎中,員外郎,主事,尚茶正,尚膳正,尚茶副,尚膳副,並正七品。內管領,副內管領,各一人。筆帖式二人。禮部員外郎,讀祝官,各二人。贊禮郎,筆帖式,各四人。工部員外郎一人。景陵:內務府總管,從五品。員外郎,主事,尚茶正,內管領,副內管領,各一人。尚膳正,筆帖式,各二人。禮部郎中一人。員外郎,讀祝官,各二人。贊禮郎,筆帖式,各四人。工部員外郎一人。景陵皇貴妃園寢:內務府員外郎,尚膳正,各一人。禮部讀祝官二人。贊禮郎三人。景陵妃園寢:內務府尚茶副,尚膳副,副內管領,委署副內管領,七品銜。各一人。禮部讀祝官二人。贊禮郎三人。筆帖式二人。裕陵:內務府掌關防郎中,員外郎,主事,尚茶正,尚膳正,內管領,副內管領,各一人。筆帖式二人。禮部郎中一人。員外郎,讀祝官,各二人。贊禮郎,筆帖式,各四人。工部員外郎一人。裕陵皇貴妃園寢:內務府尚茶副,尚膳副,並七品。副內管領,委署副內管領,各一人。禮部讀祝官二人。贊禮郎二人。端慧皇太子園寢:內務府內管領,副內管領,尚茶副,尚膳副,各一人。禮部讀祝官二人。贊禮郎三人。定陵:內務府掌關防郎中,員外郎,主事,尚茶正,尚膳正,內管領,副內管領,各一人。筆帖式二人。禮部郎中,員外郎,讀祝官,各二人。贊禮郎四人。普祥峪定東陵:內務府掌關防郎中,員外郎,主事,尚茶正,尚膳正,內管領,各一人。筆帖式二人。禮部員外郎,讀祝官,各二人。贊禮郎四人。菩陀峪定東陵:內務府掌關防郎中,員外郎,主事,尚茶正,尚膳正,內管領,各一人。筆帖式二人。禮部員外郎,讀祝官,各二人。贊禮郎四人。定陵妃園寢:內務府副內管領,委署副內管領,尚茶副,尚膳副,各一人。禮部讀祝官二人。贊禮郎三人。惠陵:內務府掌關防郎中,員外郎,主事,尚茶正,尚膳正,內管領,各一人。筆帖式二人。禮部郎中一人。員外郎,讀祝官,各二人。贊禮郎四人。惠陵妃園寢:禮部讀祝官二人。贊禮郎三人。內務府不設官,暫置領催一人,閒散拜唐阿一人。

西陵:總管大臣,泰寧鎮總兵兼內務府大臣簡充。承辦事務衙門主事,委署主事,各一人。筆帖式四人。易州衙署工部郎中一人,員外郎三人,主事一人,筆帖式二人。泰陵:內務府總管員外郎,主事,尚茶正,尚膳正,尚茶副,九品。尚膳副,九品。內管領,副內管領,各一人。筆帖式二人。禮部郎中,員外郎,各一人。讀祝官二人。贊禮郎,筆帖式,各四人。工部郎中,主事,各一人。泰東陵:內務府掌關防郎中,員外郎,主事,尚茶正,尚膳正,尚茶副,尚膳副,內管領,各一人。筆帖式二人。禮部員外郎,讀祝官,各二人。贊禮郎,筆帖式,各四人。工部員外郎一人。泰陵皇貴妃園寢:內務府主事,副內管領,各一人。禮部主事一人。讀祝官二人。贊禮郎三人。工部主事一人。昌陵:內務府掌關防郎中,員外郎,主事,尚茶正,尚茶副,尚膳副,內管領,副內管領,各一人。尚膳正,筆帖式,各二人。禮部郎中一人。員外郎,讀祝官,各二人。贊禮郎,筆帖式,各四人。工部員外郎一人。昌西陵:內務府掌關防郎中,員外郎,主事,尚茶正,尚膳正,尚茶副,尚膳副,內管領,各一人。筆帖式二人。禮部員外郎,讀祝官,各二人。贊禮郎四人。工部員外郎一人。昌陵皇貴妃園寢:內務府主事,副內管領,各一人。禮部讀祝官二人。贊禮郎三人。慕陵:內務府掌關防郎中,員外郎,主事,尚茶正,尚茶副,尚膳副,內管領,副內管領,各一人。尚膳正,筆帖式,各二人。禮部郎中一人。員外郎,讀祝官,各二人。贊禮郎四人。工部員外郎一人。慕東陵:內務府掌關防郎中,員外郎,主事,尚茶正,尚膳正,內管領,各一人。尚茶副,尚膳副,委署副內管領,筆帖式,各二人。禮部員外郎,讀祝官,各二人。贊禮郎四人。工部主事一人。後省。慕東陵皇貴妃園寢:內務府尚茶副,尚膳副,委署副內管領,各一人。禮部讀祝官二人。贊禮郎三人。東陵宗室主事,昭西陵宗室員外郎,泰陵宗室員外郎、主事,各一人。餘並滿洲員缺。

總管大臣掌督帥官兵巡防游徼,以翊衛陵寢。內務府官掌奉祭祀奠享之禮,司掃除開闔。禮部官掌判署文案,監視禮儀,歲供品物,以序祀事。工部官掌修葺繕治,凡祭祀供厥楮幣。順治十三年,置福陵、昭陵掌關防等官。康熙二年,復置各陵寢內府、禮部、工部司官。光緒三十一年,改盛京戶、禮、工三部陵寢官隸總理三陵事務衙門。宣統三年,陵寢郎、員、主各缺並改歸內務府,帶禮部、工部銜如故。

僧錄司正印,副印,各一人。□品。左、右善世,正六品。闡教,從六品。講經,正八品。覺義,從八品。俱二人。道錄司一人。□品。左、右正一,正六品。演法,從六品。至靈,正八品。至義,從八品。俱二人。分設各城僧、道協理各一人。僧官兼善世等銜,道官兼正一等銜,給予部劄。協理給予司劄。龍虎山正一真人。正三品。提點,提舉,法籙局提舉,由太清宮法官充補。各一人。副理二人。贊教四人。知事十有八人。自提點以下,並由正一真人保舉,報部給劄。

初,天聰六年,定各廟僧、道以僧錄司、道錄司綜之。凡諳經義、守清規者,給予度牒。順治二年,停度牒納銀例。八年,授張應京正一嗣教大真人,掌道教。康熙十三年,定僧錄司、道錄司員缺,及以次遞補法。十六年,詔令僧錄司、道錄司稽察設教聚會,嚴定處分。雍正九年,嘉法官婁近垣忠誠,授四品提點,尋封妙正真人。十年,定提點以次員缺。乾隆元年,酌復度牒,並授正一真人光祿大夫,妙正真人通議大夫。五年,正一真人詣京祝萬壽,鴻臚寺卿梅成疏言:「道流卑賤,不宜濫廁朝班。」於是停朝覲筵宴例。十七年,改正一真人為正五品,不許援例請封。三十一年,以法官品秩較崇,復升正一真人正三品。三十九年,真人府監紀司張克誠留京,置協理提點二人。四十二年,授克誠提點,兼京畿道錄司,省協理。


\end{pinyinscope}