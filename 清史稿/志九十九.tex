\article{志九十九}

\begin{pinyinscope}
○食貨五

△錢法茶法礦政

錢法太祖初鑄「天命通寶」錢,別以滿、漢文為二品,滿文一品錢質較漢文一品為大。天聰因之。世祖定鼎燕京,大開鑄局,始定一品。於戶部置寶泉局,工部置寶源局。「順治通寶」錢,定制以紅銅七成、白銅三成搭配鼓鑄。錢千為萬,二千串為一卯,年鑄三十卯。每錢重一錢。二年,增重二分,定錢七枚準銀一分,舊錢倍之。民間頗病錢貴,已更定十枚準一分。各省、鎮遵式開鑄,先後開山西、陜西、密雲、薊、宣、大同、延綏、臨清、盛京、江西、河南、浙江、福建、山東、湖廣及荊州、常德、江寧三府鑄局。五年,停盛京、延綏二局。六年,移大同局於陽和。七年,開襄陽、鄖陽二府鑄局。八年,停各府、鎮鑄。十年,復開密雲、薊、宣、陽和、臨清鑄局。初戶部以新鑄錢足用,前代惟崇禎錢仍暫行,餘準廢銅輸官,償以直,並禁私鑄及小錢、偽錢,更申舊錢禁。嗣以輸官久不盡,通令天下,限三月期畢輸,逾限行使,罪之。

是年廷議疏通錢法,以八年增重一錢二分五釐為定式,幕左漢文「一釐」二字,右寶泉鑄一字曰「戶」,寶源曰「工」,各省、鎮並鑄開局地名一字,如太原增「原」字、宣府增「宣」字之類,錢千準銀一兩,定為畫一通行之制。禁私局,犯者以枉法贓論。時官錢壅滯,通以斂散法,酌定京、外局錢,配搭俸餉。錢糧舊制徵銀七錢三,皆著為令。而直省局錢不精,私鑄乘之,卒壅不行,悉罷鑄,專任寶泉、寶源,精造一錢四分重錢,幕用滿文,俾私鑄艱於作偽。現行錢限三月銷毀。更定私鑄律,為首及匠人罪斬決,財產沒官,為從及知情買使,總甲十家長知情不首,地方官知情,分別坐斬絞,告奸賞銀五十兩。

十七年,復直省鑄,令準重錢式,幕兼用滿、漢文。康熙元年,鑄紀元錢,後凡嗣位改元,皆鑄如例。高宗內禪,鑄乾隆錢十二,嘉慶錢十八,非常例也。自改鑄一錢四分錢,奸民輒私銷,乃定律罪之比私鑄。遂禁造銅器,為私銷也。十八年,申嚴其禁,軍器、樂器之屬,許造用五斤以下者。時重錢銷益少,直苦昂。二十三年,允錢法侍郎陳廷敬糾復一錢舊制。久之,錢貴如故,乃申定錢直禁,銀一兩易錢毋得不足一千,然錢直終不能平。季年銀一兩易錢八百八十至七百七十。乃發五城平糶錢易銀以平其價。

自舊錢申禁,而閩地僻遠,猶雜制錢行之。二十四年,巡撫金鋐以為言,學士徐乾學疏稱:「自古皆古今錢相兼行使,聽從民便。」因歷數歷代舊事,謂「自漢五銖以來,未嘗廢古而專用今。隋銷古錢,明天啟後盡括古錢充鑄,錢之變也。且錢法敝,可資古錢以澄汰,故易代仍聽流通。矧閩處嶺外,宜聽民行使」。上韙其言,盡寬舊錢廢錢之禁。是年定旗籍私鑄私銷罪如律。四十一年,以循舊制改輕錢,私鑄復起,廷臣請罷小制錢,仍鑄一錢四分重錢,新舊錢暫兼行,新錢千準銀一兩,舊錢準七錢。詔從之。然私鑄竟不能止。

四十五年,山東請鑄大錢。會獲得常山私鑄,上以私鑄不盡大錢,必多私銷,宜先收後禁,乃令錢糧銀一兩折收二千文,錢盡,折收銅器。戶部以新錢不敷,請展至五年後毀舊鑄。越二年,襄陽私鑄錢潛貯漕艘入京,大理卿塔進泰奉命會查,疏請嚴禁收毀,再犯私鑄私販罪如律,船戶運弁罪同私鑄,地方官知情,斬決,沒其家;失察,奪職。法益加嚴。

官局用銅,自四十四年兼採滇產。雍正元年,巡撫楊名時請歲運滇銅入京。廷議即山鑄錢為便,因開雲南大理、霑益四局,鑄運京錢,幕文曰「雲泉」。上以錢為國寶,更名「寶云」,並令直省局錢,幕首「寶」字,次省名,純滿文。其後運京錢時鑄時罷。

乾隆二年,以錢價久不平,飭大興、宛平置錢行官牙以平錢價。上念私銷害尤甚,益厲行銅器禁。官非三品以上不聽用,舊有銅器限三年內輸官,逾限以私藏禁物論,已禁仍造,罪比盜鑄為從。遂通令禁造銅器。尋益嚴限制,惟一品始聽用,餘悉禁之,藏匿私用,皆以違禁論。十二年,上以錢重則私銷,輕則私鑄,令復一錢二分舊制。十三年,定翦錢邊律罪為絞監候。先是尚書海望以銅禁病民,疏陳四弊,高宗然之,遂罷禁銅收銅令。

復以京師錢價昂,銀一兩僅易八百文,詔發工部節慎庫錢平價。御史陶正靖疏陳錢價不平,弊由經紀蠹害錢法,遽命革除之。浙江布政使張若震言錢貴弊在私毀。如使配合銅鉛,參入點錫,鑄成青錢,則銷者無利。試之驗,因採其議,鑄與黃錢兼行。定私鑄鉛錢禁,為首及匠人絞監候,為從及知情買使,減一等。申嚴販運及囤積制錢之禁,凡積錢至百千以上,以違例論。上諭廷臣曰:「今之言禁者,亦第補偏救弊,非能正本清源也。物之定直以銀不以錢,而官民乃皆便錢不便銀,趨利之徒,以使低昂為得計,何輕重之倒置也?嗣是宜重用銀,凡直省官修工程,民間總置貨物,皆以銀。」

二十二年,兩廣總督李侍堯請禁舊錢、偽錢。上以民間雜用吳三桂「利用」、「洪化」、「昭武」諸偽錢,第聽自檢出,官為易之以充鑄,舊錢仍聽行使。二十四年,回部平,頒式於葉爾羌,鑄「乾隆通寶」,枚重二錢,幕鑄葉爾羌名,左滿文,右回文,用紅銅,並毀舊普爾錢充鑄。越二年,阿克蘇請鑄,如葉爾羌例。復允西藏開鑄銀錢,重一錢與五分二種,文曰「乾隆寶藏」,幕用唐古忒字,邊郭識年分。以上二類錢,第行之回、藏,內地不用。二十九年,令回部鑄錢,永用乾隆年號。

時至中葉,錢直昂,直省皆增爐廣鑄,價暫趨於平。會銅運遲滯,市儈居奇增直,害錢法,通飭督撫毋得輕請停爐減卯。季年私鑄益多,四川、雲、貴為淵藪,流布及江、浙。雲、貴官錢亦以不善罷鑄。又自律嚴私鑄,常寬之以收毀,莠民恃以行詐,私錢日出不窮。五十七年,湖廣總督畢沅請收買毋立限。上謂湖北乃私鑄總匯,不圖禁絕而預思所以卸過,命嚴稽私販,仍予寬限二年。五十九年,以官私錢錯出,錢賤,乃暫罷直省鑄,私錢通限一年收繳,而吏胥緣為奸。嘉慶元年,復直省鑄。至十年,直省未盡復卯,錢復貴,通飭各督撫按卯鼓鑄。然嗣是局私私鑄相踵起,京局錢至輪郭肉好糢糊脆薄,「寶蘇」鑄中雜沙子,擲地即碎,而貴州、湖廣私鑄盛行,江蘇官局私局秘匿。至道光間,閩、廣雜行「光中」、「景中」、「景興」、「嘉隆」諸夷錢,奸民利之,輒從仿造。貴陽大定官局亦別鑄底大錢,錢法自是益壞。

時華洋互市,以貨易銀,番船冒禁,歲漏出以千萬計,御史黃中模、章沅咸以為言。而大髻、小髻、蓬頭、蝙蝠、雙柱、馬劍各種番銀,亦潛輸內地以規利,自閩、廣通行至黃河以南。而洋商復挾至各省海口,陽置貨而陰市銀,至洋銀日多,紋銀日少而貴。上患之,命粵督申嚴禁約,然所禁不及洋銀,仿鑄之廣板、福板、杭板、吳莊、行莊,耗華銀如故。御史黃爵滋請並禁使出洋,更立專條,議從重科。十七年,詔沿江沿海督撫、海關監督,飭屬嚴稽偷漏,定功過,行賞罰,而海內銀卒耗竭,每兩易錢常至二千。廷臣謀所以重錢以殺銀之勢,而議格不行。

先是道光中葉,銀外洩而貴,朝野皆欲行大錢以救之。廣西巡撫梁章鉅疏言其利。文宗即位,四川學政何紹基力請行大錢以復古救時。上意初不謂然,卒與官票、寶鈔行焉。鈔嘗行於順治八年,歲造十二萬八千有奇。十年而罷。嘉慶間,侍講學士蔡之定請行鈔。咸豐二年,福建巡撫王懿德亦以為請。廷議以窒礙難行,卻之。是時銀虧錢匱重,而軍需河餉糜帑二千數百萬,籌國計者,率以行官票請。次年,命戶部集議。惠親等請飭部制造錢鈔與銀票相輔並行。票鈔制以皮紙,額題「戶部官票」,左滿、右漢,皆雙行,中標二兩平足色銀若干兩,下曰「戶部奏行官票」。凡原將官票兌換銀錢者,與銀一律,並準按部定章程,搭交官項。偽造者依律治罪。邊文龍。鈔額題「大清寶鈔」,漢字平列,中標準足制錢若干文,旁八字為「天下通寶,平準出入」,下曰「此鈔即代制錢行用,並準按成交納地丁錢糧一切稅課捐項,京、外各庫一概收解」。邊文如票。大錢當千至當十,凡五等,重自二兩遞減至四錢四分。當千、當五百,凈銅鑄造,色紫;當百、當五十、當十,銅鉛配鑄,色黃。百以上文曰「咸豐元寶」,以下曰「重寶」,幕滿文局名。四年,以乏銅,兼鑄當五鐵錢及制錢。已而更鑄鉛制錢。乾隆間,京局用銅,滇、洋兼資,後專行滇運。時以道梗銅滯,故權宜出此。定議票銀一兩抵制錢二千,鈔二千抵銀一兩,票鈔亦準是互相抵,民間完納丁糧稅課及一切官款,亦準五成,京、外應放庫款如之。大錢上下通行如票鈔,抵銀如制錢之數,輸官以三成,鐵錢通用如大錢。阻撓罪以違制,偽造鈔票斬監候,私鑄加嚴。通飭京、外設置官錢局。尋以直省延不奉行,嗣後議於各府置鈔局,發大錢於行店,俾錢鈔通融互易以便民,丁糧搭收票鈔,零星小戶銀鈔尾零,搭交銅鐵大錢,皆先從直隸、山東實行。官吏折勒骫法,商民交易不平價,從嚴處治。七年,令順天直隸各屬錢糧,自本年上忙始,以實銀四成、寶鈔三成、當十銅鐵大錢三成搭交,一切用項,亦按成搭放。尋從戶部議,自本年下忙始,直隸照銀七票三徵收,大錢三成即納在鈔票三成內,交票交錢聽便。

然鈔法初行,始而軍餉,繼而河工,搭放皆稱不便,民情疑阻。直省搭收五成,以款多抵撥既艱,搭放遂不復肯搭收。民間得鈔,積為無用,京師持鈔入市,非故增直,即匿貨,持向官號商鋪,所得皆四項大錢,不便用,故鈔行而中外兵民病之。其後京師以官號七折錢發鈔,直益低落,至減發亦窮應付,鈔遂不能行矣。大錢當千、當五百,以折當過重最先廢,當百、當五十繼廢,鐵錢以私票梗之而亦廢,乃專行當十錢。盜鑄叢起,死罪日報而不為止。局錢亦漸惡,雜私鑄中不復辨,奸商因之折減挑剔,任意低昂。商販患得大錢,皆裹足,三成搭收,徒張文告,屢禁罔效。法弊而撓法者多,固未有濟也。當十錢行獨久,然一錢當制錢二,出國門即不通行。咸豐之季,銅苦乏,申禁銅、收銅令。同治初,鑄錢所資,惟商銅、廢銅,當十錢減從三錢二分。光緒九年,復減為二錢六分。

時孝欽顯皇后銳意欲復制,下廷臣議,以滇銅運不如額,姑市洋銅,交機器局試鑄。戶部奏稱機器局鑄錢並京局開爐之不便,懿旨罪其委卸,卒命直隸總督李鴻章於天津行之,重準一錢,遂賞唐炯巡撫銜,專督云南銅政。十四年,廣東試鑄機器錢,以重庫平七分識於幕。二十四年,命直省鑄八分錢。而京師以制錢少,行當十錢如故。三十二年,鑄銅幣當十錢,民不樂用,於是創鑄銀、銅圓,設置銀行,思劃一幣制,與東西洋各國相抗衡。

初,洋商麕集粵東,西班牙、英吉利銀錢大輸入,總督林則徐謀自鑄圖抵制,以不適用而罷。嗣是墨西哥、日本以國幣相灌輸。光緒十四年,張之洞督粵,始用機器如式試鑄,李鴻章繼任續成之,文曰「光緒元寶,庫平七錢二分,廣東省造」,幕絞龍。並鑄三錢六分、一錢四分四釐、七分二釐、三分六釐四種小銀圓。中國自行銀錢自此始。湖北、江西、直隸、浙江、安徽、奉天、吉林以次開鑄。尋以廣東、湖北、江西所鑄最稱便用,許以應解京餉撥充鑄本。直省未開鑄者,飭從附鑄。京、外收放庫款,準搭三成。因命劉坤一、張之洞、陶模籌議三局造鑄事宜。已復由戶部核定,七省所鑄規模成色苦參差,不利通行。會造幣總廠成,擬撤其三,而留江南、直隸、廣東為分廠。初鑄準重墨圓,議者頗非之。之洞始於湖北試行一兩銀幣。戶部亦以中國立算,夙準兩錢分釐,因定主幣為庫平一兩,而以五錢、一錢小銀幣暨銅圓、制錢輔助之,令總分廠如式造行。

銅元鑄始閩、廣,江蘇繼之。時京局停鑄,命各運數十萬入京,由戶部發行備用。沿江、沿海省分,並飭籌款附鑄。而直省陸續開鑄,造幣總廠反後成。總廠擬鑄之幣凡三品:曰金,曰銀,曰銅。最先鑄銅幣。自當制錢二十降至當二,自重四錢降而四分,凡四種,文視直省小異大同。直省曰「光緒元寶」,總廠初同直省,嗣定曰「大清銅幣」,皆識某所造,幕皆龍文,紫銅鑄,直省間亦用黃銅。凡私造銅幣、偽造紙幣,罪視制錢加等。初鑄銅元,為補制錢之不足,旋艷其餘利,新政餉需皆取給焉,競鑄爭售,乃至不能敷鑄本。兩江總督周馥首疏其弊,戶部為立法限制之。繼與政務處上補救八事。旋以開鑄者多至十七省,省至二三局,恐終難言畫一,乃令山東歸並直隸,湖北歸並湖南,江南、安徽歸並江寧,浙江歸並福建,廣西歸並廣東,合奉天、河南、四川、雲貴為九廠,由部派員會辦,遣大臣周歷察核,與戶部籌定會辦事宜。顧銅元以積賤,當十錢僅能及半數,民私局私頗叢奸弊。應準銀者,銅元折合,類致虧損,物價翔貴,民生日益凋敝。省與省復相軋,至不相流通。山東巡撫袁樹勛繼陳十害。時總廠初鑄銅幣,尚留寶泉鑄六分制錢。廣東請改鑄一文錢,由總廠頒式通行。三十四年,命各銅元廠加鑄一文新錢,如銅圓式,蓋存一文舊制,藉為銅圓補救也。

自大理少卿盛宣懷奏設通商銀行,議者以東西洋各國皆有國立銀行,能持國內外財政,二十九年,允戶部請,設置官銀行,以部專其名,糾合官商資本四百萬,通用國幣、發行紙幣、官款公債皆主之。尋為發行紙幣,並開紙、印刷二廠。會戶部改度支,更銀行名曰「大清」,設正副監督各一,造幣總廠亦如之。銀行內並附設儲蓄銀行。畫一幣制,載入各國新定商約。部議宜先審定銀幣,試行效,則積金鑄幣三品之制,可使同條共貫。第計元計兩,尚持兩端。德宗下其事於督撫。適有以實行商約速定幣制請者,下政務處核議,各督撫亦先後議上。主兩者至十一省,主圓者僅八省。度支部前亦頒布用兩,遂定一兩為主幣。復由部設幣制調查局,而審慎於鑄造推行、畫一成色分量之間。至宣統二年,仍前定名曰「圓」,銀幣一圓為主幣,五角、二角五、一角三種,鎳幣五分一種,銅幣二分、一分、五釐、一釐四種,為輔幣。銀幣重七錢二分,餘遞降。並撤直隸銀銅造幣廠,而留漢口、廣東、成都、雲南四廠。前所鑄大小銀元,暫照市價行使,將來由總廠銀行收換改鑄。

三品之制,首金,次銀。光緒中葉,英金磅歲騰長,每磅自華銀四兩一錢六分五釐增至八兩有奇。御史王鵬運、通政司參議楊宜治嘗建議積金仿鑄。三十年,戶部疏請備造幣之用,納官者皆準金。出使大臣汪大燮極言用金之利。孫寶琦則請對內用銀,對外必預計用金。廷臣之論國幣者,亦以不臻至用金,幣制不為完善,皆請速定用本位金,卒未能實行雲。

茶法我國產茶之地,惟江蘇、安徽、江西、浙江、福建、四川、兩湖、雲、貴為最。明時茶法有三:曰官茶,儲邊易馬;曰商茶,給引徵課;曰貢茶,則上用也。清因之。於陜、甘易番馬。他省則召商發引納課,間有商人赴部領銷者,亦有小販領於本籍州縣者。又有州縣承引,無商可給,發種茶園戶經紀者。戶部寶泉局鑄刷引由,備書例款,直省預期請領,年辦年銷。茶百斤為一引,不及百斤謂之畸零,另給護帖。行過殘引皆繳部。凡偽造茶引,或作假茶興販,及私與外國人買賣者,皆按律科罪。

司茶之官,初沿明制。陜西設巡視茶馬御史五:西寧司駐西寧,洮州司駐岷州,河州司駐河州,莊浪司駐平番,甘州司駐蘭州。尋改差部員,又令甘肅巡撫兼轄,後歸陜甘總督管理。四川設鹽茶道。江西設茶引批驗大使,隸江寧府。

歲徵之課,江蘇發引江寧批發所及荊溪縣屬張渚、湖汊兩巡檢司。安徽發引潛山、太湖、歙、休寧、黟、宣城、寧國、太平、貴池、青陽、銅陵、建德、蕪湖、六安、霍山、廣德、建平十七州縣。江西發引徽商及各州縣小販。此三省稅課,均於經過各關按則徵收。浙江由布政使委員給商,每引徵銀一錢,北新關徵稅銀二分九釐二毫八絲,匯入關稅報解。又每歲辦上用及陵寢內廷黃茶共百一十餘簍,由辦引委員於所收茶引買價內辦解。湖北由咸寧、嘉魚、蒲圻、崇陽、通城、興國、通山七州縣領引,發種茶園戶經紀坐銷。建始縣給商行銷。坐銷者每引徵銀一兩,行銷者徵稅二錢五分,課一錢二分五釐,共額徵稅課銀二百三十兩有奇。行茶到關,仍行報稅。湖南發善化、湘陰、瀏陽、湘潭、益陽、攸、安化、邵陽、新化、武岡、巴陵、平江、臨湘、武陵、桃源、龍陽、沅江十七州縣行戶,共徵稅銀二百四十兩。陜、甘發西寧、甘州、莊浪三茶司,而西安、鳳翔、漢中、同州、榆林、延安、寧夏七府及神木亦分銷焉。每引納官茶五十斤,餘五十斤由商運售作本。每百斤為十篦,每篦二封,共徵本色茶十三萬六千四百八十篦。改折之年,每封徵折銀三錢。其原不交茶者,則徵價銀共五千七百三十兩有奇。亦有不設引,止於本地行銷者,由各園戶納課,共徵銀五百三十兩有奇。四川有腹引、邊引、土引之分。腹引行內地,邊引行邊地,土引行土司。而邊引又分三道,其行銷打箭爐者,曰南路邊引;行銷松潘者,曰西路邊引;行銷邛州者,曰邛州邊引。皆納課稅,共課銀萬四千三百四十兩,稅銀四萬九千一百七十兩,各有奇。雲南徵稅銀九百六十兩。貴州課稅銀六十餘兩。凡請引於部,例收紙價,每道以三釐三毫為率。盛京、直隸、河南、山東、山西、福建、廣東、廣西均不頒引,故無課。惟茶商到境,由經過關口輸稅,或略收落地稅,附關稅造銷,或匯入雜稅報部。此嘉慶前行茶事例也。

厥後泰西諸國通商,茶務因之一變。其市場大者有三:曰漢口,曰上海,曰福州。漢口之茶,來自湖南、江西、安徽,合本省所產,溯漢水以運於河南、陜西、青海、新疆。其輸至俄羅斯者,皆磚茶也。上海之茶尤盛,自本省所產外,多有湖廣、江西、安徽、浙江、福建諸茶。江西、安徽紅綠茶多售於歐、美各國。浙江紹興茶輸至美利堅,寧波茶輸至日本。福州紅茶多輸至美洲及南洋群島。此三市場外,又有廣州、天津、芝罘三所,洋商亦麕集焉。蓋茶之性喜燠惡寒,喜濕惡燥,又必避慓烈之風,最適於中國。泰西商務雖盛,然非其土所宜,不能不仰給於我國,用此駸駸遍及全球矣。

其業此者,有總商,有散商。領引後,行銷各有定域。亦有兼行票法者,如四川自乾隆五十二年開辦堰工茶票後,名目甚繁,然第行於產多或銷暢之區,非遍及各州縣也。惟甘商舊分東、西二櫃,東櫃多籍隸山西、陜西,西櫃則回民充之。自咸豐中回匪滋事,繼以盜賊充斥,兩櫃均無人承課。總督左宗棠勘定全省,乃奏定章程,以票代引。遴選新商採運湖茶,是曰南櫃。時領票止八百餘張。嗣定為三年一案,領票準加不準減。計自光緒十三年至二十七年,逐案加增。三十年,又於湖票外更行銷伊、塔之晉票。迄於宣統二年,茶務日盛。

茶之與鹽,辦法略相似。惟鹽為歲入大宗,故掌國計者第附於鹽而總核之。其始但有課稅,除江、浙額引由各關徵收無定額外,他省每歲多者千餘兩,少祗數百兩或數十兩。即陜、甘、四川號為邊引,亦不滿十萬金。咸豐以來,各省次第行釐,光緒十二年,福建冊報至十九萬餘兩,他省款亦漸多,未幾收數復絀。宣統三年豫算表所載,茶稅特百三十餘萬而已。

順治初元,定茶馬事例。上馬給茶篦十二,中馬給九,下馬給七。二年,差御史轄五茶馬司。時商人多越境私販,番族利其值賤,趨之若鶩。兼番僧馳驛往來,夾帶私茶出關,吏不能詰。戶部奏言:「陜西以茶易馬,明有照給金牌勘合之例。今可勿用,但定價值。至番僧所至,如官吏縱容收買私茶,聽巡按御史參究。」茶馬御史廖攀龍又言:「茶馬舊額萬一千八十八匹,崇禎三年增解二千匹,請永行蠲免。」並從之。四年,命巡視茶馬滿、漢御史各一,直隸河寶營地當張家口之西,明時鄂爾多斯部落曾於此交易茶馬,旋封閉。至是,戶部差理事官履勘,以狀聞。諭仍準互市。七年,以甘肅舊例,大引篦茶,官商均分,小引納稅三分入官,七分給商。諭嗣後各引均由部發,照大引例,以為中馬之用。又舊例大引附六十篦,小引附六十七斤。定為每茶千斤,概準附百四十斤,聽商自賣。

十三年,以甘肅所中之馬既足,命陳茶變價充餉。十四年,復以廣寧、開成、黑水、安定、清安、萬安、武安七監馬蕃,命私馬私茶沒入變價。原留中馬支用者,悉改折充餉。十八年,從達賴喇嘛及根都臺吉請,於雲南北勝州以馬易茶。康熙四年,遂裁陜西苑馬各監,開茶馬市於北勝州。七年,裁茶馬御史,歸甘肅巡撫管理。十九年,以軍需急,加福建茶課銀三百五十九兩,至二十六年豁免,並除湖廣新增茶稅銀。時四川產茶多,其用漸廣,戶部議增引,迄康熙末,天全土司、雅州、邛、榮經、名山、新繁、大邑、灌縣並有所增。

二十四年,刑科給事中裘元佩言洮、岷諸處額茶三十餘萬篦,可中馬萬匹。陳茶每年帶銷,又可中數萬匹。請遣員專管。三十六年,遂差部員督理茶馬事務。四十年,以陜西私茶充斥,令嚴查往來民人,凡攜帶私茶十斤以下勿問,其馱載十斤以上無官引者論罪。四十四年,以奸商恃有前例,皆分帶零運,私販轉多,飭照舊緝捕,停差部員,仍歸甘肅巡撫兼理。自康熙三十二年,因西寧五司所存茶篦年久浥爛,經部議準變賣。後又以蘭州無馬可中,將甘州舊積之茶,在五鎮俸餉內,銀七茶三,按成搭放。尋又定西寧等處停止易馬,每新茶一篦折銀四錢,陳茶折六錢,充餉。至六十一年,復增西寧、莊浪、岷州、河州茶引,各處所存舊茶,悉令變賣。

雍正三年,遂議自康熙六十一年始,五年內全徵本色,五年後即將舊茶變賣。嗣是出陳易新,總以五年為率。四年,定陜西行茶,改令產茶地方官給發船票,照商人引目茶數開明,如於部引外搭行印票,及附茶不遵定額者,照私鹽律論,查驗失察故縱,均加處分。八年,命陜西商運官茶,於舊例每百斤準附帶十四斤外,再加耗茶十四斤。又諭:「四川茶稅皆論園論樹,夫樹有大小,園有寬狹,豈能一致?若據以為額,未得其平。應照斤兩收納,著該撫詳議。」尋議:「舊例每斤徵課二釐五毫,今但徵四絲九忽有奇,前後懸絕,應酌減其半,無論邊、土、腹引,俱納銀一釐二毫五絲。」時川茶行銷,引尚不敷,於是復增,各府、州、縣再行給發。九年,命西寧五司復行中馬法。十年,又命中馬應見發茶。時安徽亦增引,照四川例,以餘引暫存司庫,遇不敷時,配給行運。十三年,復停甘肅中馬。始定雲南茶法,以七斤為一筒,三十二筒為一引,照例收稅。

乾隆元年,令甘肅官茶改徵折色,每篦輸銀五錢。時西寧五司陳茶充牣,令每封減價二錢,刻期變賣。二年,以江西南昌等三十二州縣地不產茶,四川成都、彭、灌等縣滯銷,其引或停或減,並豁除課銀。七年,免甘肅地震處之課,乃命西寧五司徵本色。八年,免四川天全所欠乾隆七年前之羨餘截角,成都、彭、灌等縣之未完銀兩。十一年,甘肅巡撫黃廷桂奏言:「西寧、河州、莊浪三司,番、民錯處,惟茶是賴。邇年以糧易茶,計用茶六萬五千五百餘封,易雜糧三萬八千一百餘石,請著為例。」報可。十三年,定甘肅應徵茶封,每年收二成本色、八成折色,並申明水陸各路運商驗引截角法,推行安徽、浙江、四川、雲南、貴州。二十四年,從甘肅巡撫吳達善言,命西寧五司茶封,照康熙三十七年例,搭放各營俸餉。二十五年,吳達善又言:「甘省茶課向為中馬設。今其制已停,在甘、莊二司地處沖衢,西河二司附近青海,猶有銷路,惟洮司偏僻,商銷茶斤,歷年俱改別司售賣,而交官茶封,仍歸洮庫,往往積至數十萬封,始請疏銷。應將洮司額頒茶引,改歸甘、莊二司給商徵課,俟洮司庫貯搭餉完日,即行裁汰。」

二十七年,陜甘總督楊應琚復條上疏銷事宜四:「一,官茶應改徵折價也。查甘肅庫貯官茶,向例如存積過多,改徵折色。今五司庫內,自乾隆七年至二十四年,已存百五十餘萬封。經前撫臣吳達善奏準每封作價三錢,搭放兵餉,已搭放四十餘萬封。在市肆官茶日多,非十年之久,不能全數疏銷。且每年商人又增配二十四萬封,商茶既多,官茶益滯。莫若將商交二成官茶五萬四千餘封,照例每封徵折價三錢,俟陳茶銷售將完,再徵本色。一,商茶應準減配也。查甘肅茶法,商人每引交茶五十斤,無論本折,即系額課。外有充公銀三萬九千餘兩,亦系按年交納,無殊正供。至商人自賣茶封,每引止應配正茶五十斤,連附茶共配售三十餘萬封,商人即以配售之茶納課。經吳達善奏準增配以紓商力,並無課項。第茶封既增,又有搭放兵餉之官茶,勢致愈積愈多,難免停本虧折。今商人原每引止五封,內應減無課茶十五萬八千三百十六封,共止配茶四十萬九千四百四十封,二成本色茶封既議改徵折價,無庸配運。一,陳積茶封應召商減售也。查各司俱有陳茶,而洮司為多。現每封四錢發售,商民裹足。請仍照原議,每封定價三錢,召商變賣。一,內地、新疆應一體搭放也。查乾隆二十四年吳達善奏準滿、漢各營以茶封搭餉。至新疆茶斤,向資內地。今官茶以沿途站車輓運,無庸腳費,其自肅州運至各處,將腳價攤入茶本之內,較之買自商賈,尚多減省。」疏入,議行。

二十九年,裁甘肅巡撫,茶務歸陜甘總督兼理。三十四年,以甘省庫貯官茶漸少,復徵本色一成。三十六年,又以伊犁等處安插投誠土爾扈特等眾,賞給茶封,仍議照舊徵收二成。三十八年,四川總督劉秉恬奏準三雜穀等處土司買茶,以千斤為率,使僅敷自食,不能私行轉售。四川設邊引,商人納稅領運於松潘等處銷售,無論土司蠻商,俱準赴邊起票販運。嘉慶七年,以陜西神木官銷茶引久經撥歸甘省商銷,令豁除舊存羨餘名目。四川教匪滋擾,蠲除大寧、廣元、太平、通江、南江五州縣茶稅。十年,復免大寧、太平、通江、巫山四縣稅課。十七年,以甘肅庫茶充羨,定商納官茶,全徵折色。二十二年,諭:「閩、皖、浙商人販運武夷、松羅茶赴粵銷售,向由內河行走,近多由海道販運,夾帶違禁貨物私賣。飭令茶商仍由內河行走,永禁出洋販運,違者治罪、茶入官。」

道光三年,諭:「那彥成奏定新疆行茶章程,經戶部議覆,烏里雅蘇臺、科布多磚茶不得侵越新疆各城售賣。茲將軍果勒豐阿等奏,此項磚茶,由歸化城、張家口請領部票納稅而來,已六十餘年,未便遽行禁止。惟新疆既為官茶引地,商茶究有礙官引,令嗣後商民每年馱載磚茶一千餘箱,前赴古城,仍照例給票,無許往他處售賣。」六年,諭:「前因新疆各城運茶,前將軍等請給引招商納課。茲據慶祥等奏稱,各城無殷實之戶,若遽令承充官商,必致運課兩誤。著北路商民專運售雜茶,並在古城設局抽稅,即以所收銀抵蘭州茶商課。俟試行三年,再行定額。至附茶仍由甘商運銷。」八年,欽差大臣那彥成言:「甘肅官茶,年例應出關二十餘萬封。近來行銷至四五十萬封,皆以無引私茶影射,價復遞加,每附茶一封,售銀七八兩至十餘兩不等。請嗣後每封定價,阿克蘇不得過四兩,喀什噶爾不得過五兩,並於嘉峪關外及阿克蘇等處設局稽查。」詔如所請。九年,命甘肅茶務責成鎮迪道總司稽查,奇臺縣就近經管。

咸豐三年,閩浙總督王懿德奏請閩省商茶設關徵稅。五年,福建巡撫呂佺孫復言:「閩茶向不頒給執照,徵收課稅。自道光二十九年,直隸督臣訥爾經額以閩商販運,官私莫辨,議由產茶之崇安縣給照,經過關隘,驗稅放行。嗣因產茶不止一處,商人散赴各縣購買,繞道出販,復經撫臣王懿德奏請,自咸豐三年為始,凡出茶之沙、邵武、建安、甌寧、建陽、浦城、崇安等縣,一概就地徵收茶稅,由各縣給照販運,先後下部議準。前歲因粵匪竄擾,江、楚茶販不前,暫弛海禁,各路茶販,遂運茶至省,不從各關經過,不特本省減稅,即浙、粵、江西亦形短絀。臣履任後,遍詢茶商獲利,較前不啻倍蓰。商利益厚,正賦轉虧。現粵匪未平,軍需孔急,眾商身擁厚貲,什一取盈,初無所損。且徵諸販客,不致擾累貧民,完自華商,無慮糾纏洋稅,以天地自然之利,為國家維正之供,迥非加增田賦者比。但閩茶不止數縣,必在附省扼要處所設關增卡,給印照以憑查核。連界各省,亦應一體設立,俾免趨避。請自咸豐五年始,凡販運茶斤,概行徵稅,所收專款,留支本省兵餉。惟創行伊始,多寡未能預定,俟行一二年後,再行比較定額。」自此閩稅始密。然至十年,猶未報部,經部飭催,乃按期奏報。六年,允伊犁將軍扎拉芬泰請,伊犁產茶,設局徵稅,充伊犁兵餉之用。十一年,廣東巡撫覺羅耆齡奏請抽收落地茶稅。

同治元年,飭下湖南、湖北、江蘇、安徽、江西、浙江、福建各督撫,詳查本省產茶及設茶莊處所,妥議章程具奏。二年,兩江總督曾國籓疏,略言:「江西自咸豐九年,定章分別茶釐、茶捐。每百斤除境內抽釐銀二錢,出境又抽一錢五分有零外,向於產茶及設立茶莊處所勸辦茶捐,每百斤捐銀一兩四錢或一兩二錢不等,填給收單,準照籌餉事例匯齊請獎。臣仍照舊章辦理。本年據九江關署監督蔡錦青詳,請遵照戶部奏準,飭將鹽、茶、竹、木四項統徵關稅,已於三月起徵。江西茶葉運至九江,有華商、洋商之分。洋商既完子口半稅,固不抽釐,華商既納潯關正稅,亦未便再令完釐。臣即照部章,於義寧州開辦落地稅。惟原奏內大箱凈茶科則稍重,分別核減。參酌茶捐向章,每百斤,義寧州等處徵一兩四錢,河口鎮徵一兩二錢五分,概充臣營軍餉,由臣刊發稅單護票,委員經收。或業戶自行完納,或茶莊代為完稅領單,至發販時,統由茶莊繳銷稅單。華商換給護票,洋商即憑運照,販至各處銷售。除華商完納九江關稅、洋商完納子口半稅外,經過江西、安徽各釐卡,驗明放行。如此辦理,與戶部原奏、總理衙門條約,一一符合。稅單雖系茶莊經手,稅銀實為業戶所出。洋商不得藉口於子口半稅,而禁中國之業戶不完中國之地稅。華商既免逢卡抽釐,亦不至紛紛私買運照,冒充洋商。」得旨允行。

五年,戶部奏準甘省引滯課懸,暫於陜西省城設官茶總店,潼關、商州、漢中設分店。商販無引之茶,到陜呈報。上色茶百斤收課銀一兩,中色六錢,下色四錢。所收解甘彌補欠課。七年,議準歸化城商人販茶至恰克圖,假道俄邊,前赴西洋各國通商,請領部照,比照張家口減半,令交銀二十五兩,每票不得過萬二千斤。十一年,議準甘省積欠舊課,仍追舊商。召募之新商試新課。其雜課、養廉、充公、官禮四項緩徵。十三年,議準甘省仿淮鹽之例,以票代引,不分各省商販,均令先納正課,始準給票。其雜課歸並釐稅項下徵收。各項名色概予刪除。行銷內地者,照納正課三兩外,於行銷地各完釐稅,每引以一兩數錢為度,多不過二兩。出口之茶,則另於邊境局卡加完釐一次,以示區別。

光緒十年,戶部統籌財政,於茶法略言:「據總理衙門單開,光緒八、九等年出口茶數多至萬九千餘萬斤。查道光年間英國所收茶稅,約每百斤收銀五十兩,而我之出口稅僅納二兩五錢,不及十一。擬照甘肅茶封之例,每五十斤就園戶徵銀三錢。增課既多,洋人無所藉口。或照寧夏、延、榆、綏等處茶引每道徵銀三兩九錢之例,於產茶處所設局驗茶,發給部頒茶照,每照百斤,徵銀三兩九錢,經過內地關卡,另納釐稅,驗照蓋戳放行,不準重衣復影射。所有茶照,按年豫行赴督請領,原照一年後作廢。或於產茶處所驗茶發給部照,既完課三兩,再倍收銀三兩九錢,前後共徵七兩八錢,一切雜費均予豁除。惟於各海關及邊卡,凡應納洋稅,仍照向章完納。若在內地行銷販運,無論經過何省何處釐卡關榷,均免再徵。則改釐為課,改散為總,既便稽查,復免侵漁。惟園戶及販商若何防其走漏,應令各省參酌定章,覆奏辦理。」

十二年,以山西商人在理籓院領票,詭稱運銷蒙古地方,實私販湖茶,侵銷新疆南北兩路。一票數年,循環轉運,往往逃釐漏稅。經部奏準,嗣後領票,言主明「不準販運私茶」字樣。如欲辦官茶,即赴甘肅領票繳課完釐。倘復運銷私茶,查出沒官。

是時泰西諸國嗜茶者眾,日本、印度、意大利艷其利厚,雖天時地質遜於我國,然精心講求種植之法,所產遂多。蓋印度種茶,在道光十四年,至光緒三年乃大盛。錫蘭、意大利其繼起者也。法蘭西既得越南,亦令種茶,有東山、建吉、富華諸園。美利堅於咸豐八年購吾國茶秧萬株,發給農民,其後愈購愈多,歲發茶秧至十二萬株,足供其國之用。故我國光緒十年以前輸出之數甚鉅,未幾漸為所奪。印度茶往英國者,歲約七十三萬二千石,價約二千四萬兩。吾國茶往者八十九萬八千石,價約千八百六十八萬兩。印度茶少於華,而價反多。迨二十二年我國運往,乃止二十一萬九千四百餘石而已。日本之茶,多售於美國,亦有運至我國者。光緒十三年,我茶往日本者萬二千餘石,而彼茶進口萬六千餘石。其專尚華茶取用宏多者惟俄。蓋自哈薩克、浩罕諸部新屬於彼,地加廣,人加眾,需物加多,而茶尤為所賴。光緒七年定約,允以嘉峪關為通商口岸,而往來益盛。十年後我國運往之茶,居全數三之一。十三年,並雜貨計,出口價九百二萬兩有奇,而進口價僅十一萬八千餘兩,凡輸自我者八百九十萬兩。然十二年茶少價多,十三年茶多價少,華商已有受困之勢,厥後亦兼購於他國,用此華茶之利驟減。蓋我國自昔視茶為農家餘事,惟以隙地營之,又採摘不時,焙制無術,其為他人所傾,勢所必至。

三十三年,茶葉公會以狀陳於度支部,稅務司亦以茶稅減少為言,於是命籌整理之策。宣統初,農工商部遂有酌免稅釐之議。漢口、福州皆自外國購入制茶機器,且由印度聘熟練教師。江西巡撫又籌款貸與茶戶。自是銷入歐洲及北阿非利加洲者乃稍暢旺。

夫吾國茶質本勝諸國,往往澀味中含有香氣,能使舌本回甘,泰西人名曰「膽念」,他國所產鮮能及此。故日本雖有茶,必購於我,荷蘭使臣克羅伯亦言爪哇、印度、錫蘭茶皆不如華茶遠甚。然則獎勵保護,無使天然物產為彼族人力所奪,是不能不有望於今之言商務者。

礦政清初鑒於明代競言礦利,中使四出,暴斂病民,於是聽民採取,輸稅於官,皆有常率。若有礙禁山風水,民田廬墓,及聚眾擾民,或歲歉穀踴,輒用封禁。

世祖初開山東臨朐、招遠銀礦,順治八年罷之。十四年,開古北、喜峰等口鐵礦。康熙間,遣官監採山西應州、陜西臨潼、山東萊陽銀礦。二十二年,悉行停止。並諭開礦無益地方,嗣後有請開採者,均不準行。世宗即位,群臣多言礦利。粵督孔毓珣、粵撫楊文乾、湘撫布蘭泰、廣西提督田畯、廣東布政使王士俊、四川提督黃廷桂相繼疏請開礦,均不準行,或嚴旨切責。十三年,粵督鄂彌達請開惠、潮、韶、肇等府礦,下九卿議行。上以妨本務停止。蓋粵東山多田少,而礦產最繁,土民習於攻採。礦峒所在,千百為群,往往聚眾私掘,嘯聚剽掠。故其時礦東開礦,較他省尤為厲禁。

乾隆二年,諭凡產銅山場,實有裨鼓鑄,準報開採。其金銀礦悉行封閉。先是,五年允魯撫硃定元請,開章丘、淄川、泰安、新泰、萊蕪、肥城、寧陽、滕、嶧、泗水、蘭山、剡城、費、莒、蒙陰、益都、臨朐、博山、萊陽、海陽各州縣煤礦,而槁城知縣高崶請自備貲開嶧、滕、費、淄、沂、平陰、泰安銀銅鉛礦則禁之。然貴州思安之天慶寺、鎮遠之中峰嶺,陜西之哈布塔海哈拉山,甘肅之扎馬圖、敦煌、沙洲南北山,伊犁之皮裏沁山、古內、雙樹子,烏魯木齊之迪化、奎騰河、呼圖壁、瑪納斯、庫爾喀喇烏蘇、條金溝各金礦,貴州法都、平遠、達摩山,雲南三嘉、麗江之回龍、昭通之樂馬各銀礦,相繼開採。嘉慶四年,給事中明繩奏言民人潘世恩、蘇廷祿請開直隸邢臺銀礦。上謂:「國家經費自有正供,潘世恩、蘇廷祿覬覦礦利,敢藉納課為詞,實屬不安本分。」命押遞回籍,明繩下部議。六年,保寧以請開塔爾巴哈臺金礦,明安以請開平泉州銅礦,均奉旨申飭。

道光初年,封禁甘肅金廠、直隸銀廠。蓋其時歲入有常,不輕言利。惟雲南之南安、石羊、臨安、個舊銀廠,歲課銀五萬八千餘兩;其餘金礦歲至數十兩,銀礦歲至數千兩而止。又旋開旋停,興廢不常,賦入亦鮮。銅鉛利關鼓鑄,開採者多邀允準,間有蠲除課稅者。廣東自康熙五十四年封禁礦山,至乾隆初年,英德、陽春、歸善、永安、曲江、大埔、博羅等縣,廣州、肇慶兩府,銅鉛礦均行開採。百餘年來,雲、貴、兩湖、兩粵、四川、陜西、江西、直隸報開銅鉛礦以百數十計,而雲南銅礦尤甲各行省。蓋鼓鑄鉛銅並重,而銅尤重。秦、鄂、蜀、桂、黔、贛皆產銅,而滇最饒。

滇銅自康熙四十四年官為經理,嗣由官給工本。雍正初,歲出銅八九十萬,不數年,且二三百萬,歲供本路鼓鑄。及運湖廣、江西,僅百萬有奇。乾隆初,歲發銅本銀百萬兩,四五年間,歲出六七百萬或八九百萬,最多乃至千二三百萬。戶、工兩局,暨江南、江西、浙江、福建、陜西、湖北、廣東、廣西、貴州九路,歲需九百餘萬,悉取給焉。礦廠以湯丹、碌碌、大水、茂麓、獅子山、大功為最,寧臺、金釵、義都、發古山、九度、萬象次之。大廠礦丁六七萬,次亦萬餘。近則土民遠及黔、粵,仰食礦利者,奔走相屬。正廠峒老砂竭,輒開子廠以補其額。故滇省銅政,累葉程功,非他項礦產可比。

道光二十四年,詔雲南、貴州、四川、廣東等省,除現在開採外,如尚有他礦原開採者,準照現開各廠一律辦理。二十八年,復詔「四川、雲、貴、兩廣、江西各督撫,於所屬境內確切查勘,廣為曉諭。其餘各省督撫,亦著留心訪查,酌量開採,不準託詞觀望。至官辦、民辦、商辦,應如何統轄彈壓稽查之處,朝廷不為遙制」。一時礦禁大弛。咸豐二年,以寬籌軍餉,招商開採熱河、新疆及各省金銀諸礦。三年,詔曰:「開採礦產,以天地自然之利還之天地,較之一切權宜弊政,無傷體制,有裨民生。當此軍餉浩繁,左藏支絀,各督撫務當權衡緩急,於礦苗豐旺之區,奏明試辦。」時軍興餉乏,當時開採者,僅新疆噶爾,蒙古達拉圖、噶順、紅花溝之金礦,直隸珠窩山、遍山線、室溝、土槽子、錫蠟片、牛圈子溝,蒙古哈勒津、羅圈溝、庫察山、長杭溝之銀礦,新疆迪化、羅布淖爾、三個山之銅錫礦數處。同治七年,吉林請開火石嶺子等處煤礦,以伏莽未靖,格部議不果行。十三年,以滇礦經兵燹久廢,諭飭開辦,從滇督岑毓英請也。

是年海防議起,直隸總督李鴻章、船政大臣沈葆楨請開採煤鐵以濟軍需,上允其請,命於直隸磁州、福建臺灣試辦。光緒八年,兩江總督左宗棠亦言北洋籌辦防務,制造船砲,及各省機器輪船所需煤鐵,最為大宗,請開辦江蘇利國驛煤鐵。報聞。嗣是以次修築鐵路,煤鐵益為當務之急。於是煤礦則吉林大石頭頂子、亂泥溝、半拉窩、雞溝、二道河、陶家屯、石牌嶺,黑龍江太平山、察漢敖拉卡倫,直隸開平、唐山,內丘縣之上坪、永固、磁窯溝、南陽寨,臨城縣之岡頭、石固、膠泥溝、楊家溝、新莊、竹壁、牟村、焦村,宣化府之雞鳴、玉帶、八寶寺山,阜平縣炭灰鋪村,曲陽縣白石溝、野北村,張家口海拉坎山、馬連圪達,宛平縣青龍澗、碑碣子,承德府榆樹溝,奉天海龍府遠來、義和、進寶、玉盛、永順、永益、萬利、人和、同德、順發,錦州府大窯溝,錦西碭石溝,本溪縣王乾溝,興京蜜蜂溝,遼陽州窯子峪,江西萍鄉、永新、餘干,山東嶧縣,安徽貴池、廣德、繁昌、東流、涇縣,湖北荊門,河南禹州,山西平定、鳳臺,浙江桐廬、餘杭,江蘇上元、句容,湖南湘鄉、祁陽,廣西富川、賀縣、奉議、恩陽、南寧、那坡,陜西白水、澄城、同官、宜君、邠州、隴州、淳化。鐵礦則直隸遷安縣、灤州,湖北大冶,廣西永寧州,江西永新縣,雲南開、廣兩府,貴州青谿,皆先後開採,而秦、晉商民零星開採,尤難悉數。

二十二年,詔開辦各省金銀礦廠。自光緒初年,開直隸窯溝銀礦,甘肅西寧、甘、涼,黑龍江漠河觀音山、奇乾河各金礦外無聞焉。自明令頒行而後,金礦則直隸之平泉州屬轉山子,建昌縣屬金廠溝,撫寧縣屬雙山子,濼平縣屬寬溝,豐寧縣大營子、西碾子溝,翁牛特旗之紅花溝、水泉溝、拐棒溝,而遷安縣所產尤旺。奉天之鳳凰、安東、遼陽、通化、寬甸、懷仁、鐵嶺、開原、通化、海城、錦縣,蒙古之賀連溝、大小槽、碾溝、除虎溝、硃家溝、板橋子、珠爾琥珠、克勒司、布恭、特勒基、哈拉格囊圖、奎騰河、圖什業圖汗,四川之冕溝,湖南之平江,浙江之諸暨,黑龍江之黑河,新疆之和闐、焉耆。銀礦則四川之天全、盧山、大穴山頭,皆報明開採。

而銅、錫、鉛、銻、石油、硫磺、雄黃等礦,亦接踵而起。銅則雲南迤東湯丹、茂麓正廠六,子廠十一。迤西回龍、得寶正廠八,子廠九。楚雄永北及云武所屬萬寶、雙龍,又永安順寧、臨安、開化、曲靖各廠,均招商承採。而江西贛州,陜西鎮安,湖南綏寧,新疆拜城、庫車亦有銅廠。錫則廣東儋州,廣西南丹土州、富川、賀縣。鉛則湖南常寧、湘鄉、臨武,四川會理,浙江鎮海、奉化、象山、寧海、太平。銻則湖南益陽、邵陽、新化、沅陵、慈利、湘鄉、祁陽、新安、漵浦,貴州銅仁,四川秀山,廣東曲江、防城、乳源,廣西南太、泗鎮、陵陽都。石油則陜西延長,甘肅玉門,新疆庫爾喀喇烏蘇。硫磺則山西陽曲,奉天遼陽、錦州。雄黃則湖南慈利。或官辦,或商辦,或官商合辦。或用土法,或用西法。

九年,詔各省煤礦招商集股舉辦。自是雲南、四川均設招商及礦務局,貴州設礦務公商局,山西設礦務公司。粵東瓊州之銅礦,浙江寧波之鉛礦,皆率招商集股開辦。開辦歷數十年,惟開平、萍鄉之煤,大冶之鐵,規模宏遠。次則平江之金,益陽之銻,常寧之鉛,猶為民利。漠河金礦所產雖富,歲解部銀僅二十萬兩。滇銅自十三年命唐蜅督辦,歲運京銅不過百餘萬,各省鼓鑄,猶以重直購洋銅。鐵產為漢陽廠鍊鋼造軌,略供輪路之需。粵、桂、晉出鐵雖饒,以提鍊不精,國內制造,仍多購自英廠。

二十四年,詔設礦務鐵路總局於京師,以王文韶、張廕桓主之。奏定章程二十二,準華商辦礦,假貸洋款,及華洋合股,設立公司。自是江西萍鄉煤礦則借德款,湖北大冶鐵礦則借日本款,浙江寶昌公司則借義款,直隸臨城煤礦則借比款。當其議定合同,於抵押息金外,輒須延聘礦師,甚者涉及用人管理。至直隸井陘、安徽宣城煤礦,山西盂平、澤、潞、平陽,四川江北煤鐵礦,新疆塔城,直隸霍家地、廠子溝金礦,廣西上思,貴州正安鉛鐵,福建邵武、建寧、汀州,直隸八道河,奉天尾明山,及吉林新舊礦,均華洋合辦,一經訂約,時生轇轕。若福公司之於晉礦,其尤甚者也。二十四年,河南豫豐公司以其專辦懷慶左右黃河以北各礦之權,山西商務局以其專辦盂平、澤、潞、平陽煤鐵各礦之權,同時讓與辦理。一公司壟斷兩省礦務,更議修鐵道自晉訖汴,因礦及路,利權損失,爭持三年,始允合辦。汴既侵攘華官主權,晉復干涉人民開採。全晉紳民,堅持廢約。遲之又久,始以銀二百七十餘萬贖回。他如陜西延長,四川富順、巴、萬石油礦,湖南常寧龍王山,湖北興國龍角山礦,均因商民私相授受,釀成交涉。

自議訂膠濟、東清路約,附路十三里內華人無開礦權。而開平煤礦,漠河觀音山金礦,復因內亂為外人所侵占。開平煤礦,自光緒元年直隸總督李鴻章集官商之力,經營二十年,效力大著。二十六年,拳匪亂後,洋員德璀琳因督辦張翼委其保護,與礦師胡華私立賣約,而張翼亦即簽押移交,轉以加招洋股中外合辦奏聞。由是而唐山西山、半壁店、馬家溝、無水莊、趙各莊、林西各礦,秦皇島口岸地畝附屬之承平、建平、永平金銀礦,悉操於英公司。嚴詔責令收回,赴英控訴,卒未就緒。三十四年,籌辦灤州煤礦,英公司阻撓之。乃劫為營業聯合之法,合設開灤總局。觀音山金礦,亦因拳亂為俄人占據。三十二年,始以俄銀萬二千盧布贖回。

二十八年,外務部改定礦章,凡華洋商人得一體承辦礦務,惟必稟部批準,乃為允行之據。是年皖撫聶緝椝許英人凱約翰承辦歙、銅陵、大通、寧國、廣德、潛山礦產,嗣以專辦銅陵之銅官山,訂約定期百年,占地三十八萬四千餘畝。皖中紳民合力爭之,始以銀四十萬兩贖回自辦。法人彌樂石亦於是年以勘辦全滇礦務請於滇督及外務部,皆拒之,仍獲澂江、臨安、開化、雲南、楚雄、元江、永北等府、、州礦權以去。繼是英商立樂德以合辦東、昭兩府金銀礦不獲,遂援彌樂石例,索廣南、曲靖、麗江、大理、順寧、普洱、永昌七府礦,亦堅拒未允。一時舉國上下,咸以保全礦產為言。由是蜀設保富公司,華洋承辦川省礦務,購地轉租事宜屬之。閩設商政局,旋奏設礦務總公司,凡請辦各礦場,查核準駁之權屬之。山西保晉公司,安徽礦務總局,類能集合殷富,鳩貲開辦。湘、鄂則於所屬礦地勘明圈購,以杜私售。

二十五年,江南籌辦農工礦路各學堂,兩湖復籌設高等礦業學堂。三十一年,商部以洋商私占礦地礦山,疏請申明約章,以維權限。尋奏設各省礦政調查局,以勘明全國礦產、嚴禁私賣為先務。鄂督張之洞條上礦務正章七十四,附章七十三。蓋自二十四年以來,礦章屢易,每因礦務齟,洋商輒引為口實。二十九年,商約大臣呂海寰與各國議訂商約,許以開採礦產之利,但必須遵守中國礦章。而中國礦章,則比較各國通行者為之準則,特詔張之洞擬定。乃取英、美、德、法、比利時、西班牙礦章參互考證,區別地面地腹,釐定礦界礦稅,分晰地股銀股,暨華洋商,限制至周;尤注重於中國主權,華民生計,地方治理。閱數年乃成,下部議行,中國礦章始具云。


\end{pinyinscope}