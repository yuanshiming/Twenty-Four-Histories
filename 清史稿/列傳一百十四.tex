\article{列傳一百十四}

\begin{pinyinscope}
劉藻楊應琚子重英蘇爾相明瑞

劉藻,字素存,山東菏澤人。初名玉麟,以舉人授觀城教諭,乾隆元年,薦舉博學鴻詞,試一等,授檢討,更名。累遷左僉都御史。圓明園工興,疏言:「園工不過少加補葺,視前代飾臺榭之觀者度越何啻萬萬?臣愚以為奢靡之漸,不可稍開。乞皇上慎始慮終,為天地惜物力,為國家培元氣,來歲諸工酌量停減。」上嘉納。遷通政使。六年,擢內閣學士。督江蘇學政。尋以高郵諸生求賑而閧,左授宗人府府丞。藻居揚州候代,有吳之黼者,以文求教,藻行,饋糟魚,受之,中途發視,得白金四百,藻畀兩淮運使硃續晫還之黼。上聞,諭曰:「如此方不愧四知!」旋乞養歸。孝賢皇后及長皇子定安親王喪,藻詣闕入見。會大學士張廷玉乞歸失上指,因獎藻,謂其知君臣休戚相關大義,以媿廷玉,加藻內閣學士銜,賜人葠二斤,命歸養母。母喪終,二十一年,授陜西布政使。

二十二年,擢雲南巡撫。加太子少保,兼領貴州巡撫。二十九年,例行大計,巡撫圖爾炳阿未至,藻疏請先期舉行,上嘉之,旋授雲貴總督。三十年,疏言:「年來木梳野匪與緬甸所屬木邦構釁,又與耿馬土司毗連。自木邦至滾弄江,應設卡防守,請於各土司就近派撥。」詔如所請。

三十一年,移湖廣總督,未行,尋奏:「副將趙宏榜等赴孟連、耿馬剿逐莽匪,鎮臣烏爾登額赴滾弄江口。臣於普洱、思茅各隘調度。」又奏言:「由小猛侖進攻九龍江、橄欖壩諸寨,多斬獲。惟參將何瓊詔、游擊明浩派赴整控江防禦,冒昧渡江,遇賊敗沒。」尋奏瓊詔等未死,請治貪功輕進之罪,上以「瓊詔、明浩等遇賊敗逃,又復妄言敗沒。此法所難宥,藻反稱冒昧貪功輕進,何憒憒乃爾」?詔言:「藻本書生,軍行機宜,非其所習,朕不責以所不能。至調度賞罰,並可力為籌辦,乃舛謬若此,豈堪復勝總督之任?」因左授湖北巡撫,命楊應琚往代。復諭:「應琚未至,藻當實力經理。若自以為五日京兆,致誤事機,必重治其罪!」部議奪職,留雲南效力。藻聞上怒,惶迫自殺,巡撫常鈞疏報。上令應琚至普洱,為求醫治療,傷平,傳旨逮問。常鈞旋奏藻死,上復詔責其張皇畏葸,旅襯歸葬,不得聽其家立碑書歷官事實。

三十二年,巡撫鄂寧奏言:「緬甸本莽瑞體之後。乾隆十八年,木梳頭目甕籍牙逐其酋莽打喇而自立。夷人遂呼緬甸為木梳,或呼緬,或呼莽,非二種也。」

楊應琚,字佩之,漢軍正白旗人,廣東巡撫文乾子。應琚起家任子。乾隆初,自員外郎出為河東道,調西寧道。巡撫黃廷桂薦其才,高宗曰:「若能進於誠而擴充之,正未可量也。」累遷至兩廣總督。先後疏請練水師,籌軍食,修漓水、陡河堤壩,貯柳、桂、慶、梧餘鹽,皆如所請行。暹羅貢使毆傷通事,其國王鞫實,擬罰鍰,遣使牒禮部。應琚曰:「屬國陪臣無上交。」好語諭遣之,稱旨。二十二年,移閩浙總督。二十三年,加太子太保。

二十四年,移陜甘總督。疏言伊犁底定,宜先屯田,留兵五千墾特諾果爾、長吉、羅克倫。復以陜、甘非一督能治,請更西安總督為川陜總督,四川總督為巡撫,甘肅巡撫為總督,上遂命應琚督甘肅,陜西提鎮受節制,進太子太師。嘗募巴爾楚克回戶治多蘭溝渠,墾喀喇沙爾以西各臺,又增置兵備道、總兵,分駐阿克蘇、葉爾羌二城,逐為重鎮。應琚奏辦屯墾,遣兵購畜,部署紛煩;至是,疏自言其非,請因利乘便規久遠。帝嘉納,下其疏示中外。二十九年,移駐肅州,拜東閣大學士。

三十一年,緬甸大入邊,滇事棘。緬酋莽達拉自為木梳長所篡,擊敗貴家木邦,貴酋宮裹雁奔孟連。時應琚子重穀為永昌知府,誘殺之,木酋亦走。緬益橫,入犯思茅。上移應珺雲貴總督視師。應琚至楚雄,緬人漸退,師乘間收復。應琚往孟良、整賣正經界,集流亡,釐戶口,定賦稅,而令召丙、叭先俸分據之,請賞給三品指揮使。上以為能,賜珍物,官其孫茂齡藍翎侍衛。又使人誘致孟密、孟養、蠻暮令獻地,實則地懸緬境,內附特空言。諸將希應琚指,爭謂緬勢孤,易攻取。應琚初猶弗聽,曰:「吾官至一品,年逾七十,復何所求,而以貪功開邊釁乎?」副將趙宏榜慫恿之,遂下道、鎮、府、州合議,亦謂寇勢大,邊釁不可開,總兵烏爾登額阻尤力,應琚滋不懌。

永昌知府陳大呂懼更初議,應琚乃往永昌受降,並為文檄緬,侈言水陸軍五十萬陳境上,不降即進討。緬遂大發兵溯金沙江而上。其時宏榜頓新街,卻走。應琚聞警即遘疾,上命楊廷璋往代,遣侍衛福靈安攜御醫往診;並諭其子江蘇按察使重英、寶慶知府重穀省視。比廷璋至而疾已愈,乃令諸軍進擊,總兵硃侖出鐵壁關,攻楞木,不克,寇勢益張。提督李時升告急,應琚不報。緬陽議款,遂以楞木大捷入告,而緬已漸入戶臘撒。

時總兵劉德成擁兵干崖,飲酒高會,時升屢趣罔應。應琚遣緬寧通判富森持令箭督戰,德成始抵盞達。緬懼擊其後,潛引去,應琚仍以捷聞。緬甸復入猛卯,參將哈國興等引還,砲械多遺失,應琚又報捷;並傳令硃侖兼剿撫,陰示以和蕆事,緬果累乞和。逾歲,奏言:「緬甸酋弟卜坑率聶渺遮乞款附,懇予蠻暮、新街互市。」上察其偽,數訶責。嗣木邦告警,國興軍抵蠻暮,寇★K4退,應琚又以復新街奏。上視所進地圖,疑寇既屢敗,何以尚據內地土司境,降旨駁詰。會福靈安先被命廉軍事,具言宏榜諸人失守狀,應琚亦劾德成等遲留不進,於是俱逮問,而以楊寧為提督,且以應琚不勝任,召明瑞代統其軍。明瑞至,首發其欺罔罪,謂誤木緬別為一事尤妄誕,鄂寧亦糾其掩敗為勝。應琚恐,乃上言大舉征緬,調湖廣、川、滇軍五萬,五路並進,請敕暹羅夾攻,朝論皆斥之。未幾,詔逮問,賜死。重穀亦坐笞殺人,棄市。

重英初至雲南,隱以監軍自居,嗣為鄂寧所劾,命以知府從軍。明年,軍士患饑,緬嗛詐媾和,參贊珠魯訥遣重英往報,被執。上以重英且降緬,下其子長齡獄。已,緬歸俘卒,賚貝葉書,附重英書乞罷兵,拒弗納。四十一年,緬出都司蘇爾相議和,仍弗許。五十三年,緬聞暹羅受封,乃款關求貢,並還重英。重英陷緬後,獨居佛寺逾二十年,未改中國衣冠。上大悅,進道員,釋長齡出獄,比以蘇武之節,禦制蘇楊論旌之。俄,病卒。

蘇爾相,甘肅靈州人。自行伍從征緬甸、金川有勞,累遷雲南奇兵營都司。三十五年,雲貴總督彰寶以緬甸表貢久不至,遣爾相齎檄往諭,被留,迫使上書阿桂申表貢之議。上謂爾相且降緬,命甘肅疆吏執爾相妻孥致京師,子一、女二死於獄,妻死於道。四十一年,緬始送爾相還。上命阿桂傳諭,令其詣京師,引見,授游擊,賜詩亦比以蘇武。累遷騰越鎮總兵,兼署云南提督。卒。

明瑞,字筠亭,富察氏,滿洲鑲黃旗人,承恩公富文子。自官學生襲爵。乾隆二十一年,師征阿睦爾撒納,明瑞以副都統銜授領隊大臣,有功,擢戶部侍郎,授參贊大臣,於公爵加「毅勇」字,號承恩毅勇公。二十四年,師征霍集占,復有功,賜雙眼花翎,加雲騎尉世職。師還,圖形紫光閣,擢正白旗漢軍都統。二十七年,出為伊犁將軍,進加騎都尉世職。

三十年二月,烏什回為亂,駐烏什副都統素誠自戕,亂回推小伯克賴黑木圖拉為渠,拒守。明瑞遣副都統觀音保往討,而帥師繼其後。烏什回二千餘出御,明瑞與觀音保力戰破之,奪砲臺七。賊入城,師合圍。明瑞疏陳素誠狂縱激變,及參贊納世通虐回民,阻援師,副都統弁塔哈掩敗妄奏諸狀,上令尚書阿桂至軍,按誅納世通、弁塔哈。賊夜襲我軍,我軍詗知之,預為備,射賴黑木圖拉殪,賊擁其父額色木圖拉為渠。明瑞以兵六百餘夜攜雲梯薄其城,不克,則毀其堞,且斷汲道。賊待阿富汗援不至,乃縛獻額色木圖拉等四十二人降,明瑞悉斬之,其脅從及婦稚萬餘送伊犁。烏什平。上以明瑞得渠魁,未詳鞫為亂狀,亂回至圍急始縛獻首惡,不可輕宥,所措置皆不當,與阿桂同下部議,奪職,命留任。旋條上善後事,如所請。

是時緬甸為亂犯邊,總督劉藻戰屢敗,自殺。大學士楊應琚代為總督,師久無功,賜死。三十二年二月,命明瑞以雲貴總督兼兵部尚書,經略軍務。明瑞議大軍出永昌、騰越攻宛頂、木邦為正兵,遣參贊額爾登額出北路,自猛密攻老官屯,會於阿瓦。十一月,至宛頂,進攻木邦,賊遁,留參贊珠魯訥、按察使楊重英守之,率兵萬餘渡錫箔江攻蠻結。寇二萬,立十六寨,寨外浚溝,溝外又環以木柵,列象陣為伏兵。明瑞統兵居中,領隊大臣扎拉豐阿、李全據東山梁,觀音保、長青據西山梁。賊突陣西出,觀音保、長青力戰,明瑞督中軍進,殺賊二百餘,賊退保柵。明瑞令分兵為十二隊,身先陷陣,目傷,猶指揮不少挫。賊陣中群象反奔,我兵毀柵進,無不一當百。有貴州兵王連者,舞藤牌躍入陣,眾從之,縱橫擊殺,馘二十餘,俘三十有四,賊遁走。捷聞,上大悅,封一等誠嘉毅勇公,賜黃帶、寶石頂、四團龍補服,原襲承恩公畀其弟奎林。扎拉豐阿、觀音保勸明瑞乘勝罷兵,明瑞不可。

師復進,十二月,次革龍,地逼天生橋渡口,賊踞山巔立柵。明瑞令別軍出大道,若將奪渡口,而督軍從間道繞至天生橋上游,乘霧徑渡,進據山梁。賊驚潰,浮馘二千餘。復進至象孔,糧垂罄,欲退,慮額爾登額師已入,聞猛籠土司糧富,且地近猛密,冀通北路軍消息,乃移軍猛籠。賊尾我軍後,至章子壩,我軍且戰且行。明瑞及觀音保等殿,日行不三十里,至猛籠已歲除,土司避匿,發窖粟二萬餘石。駐三日,復引軍趨猛密,人持數升粟,焚其餘積。賊躡我軍行,至夕駐營,初相距十餘里。賊詗我軍饑疲,經蠻化,我軍屯山巔,賊即營山半。明瑞謂諸將曰:「賊輕我甚,不一死戰,無焦類矣!賊識我軍號。明旦我軍傳號,若將起行,則盡出營伏箐待。」明旦賊聞聲,蟻附上山。我軍突出發槍砲,賊反走,乘之,斬四千有奇。自此每夜遙屯二十里外,明瑞令休兵六日。賊柵於要道,我軍攻之不能拔,得波霙人引自桂家銀廠舊址出。上聞明瑞深入,命全師速出。詔未達,三十三年正月,賊攻木邦,副都統珠魯訥師潰自戕,執重英以去。額爾登額出猛密,阻於老官屯,月餘引還。繞從小隴川緩行,巡撫鄂寧檄援,不應,於是明瑞軍援絕,而賊自木邦、老官屯兩道並集。二月,至小猛育,賊麕聚五萬餘。我軍食罄,殺馬騾以食;火藥亦竭,槍砲不能發。明瑞令諸將達興阿、本進忠分隊潰圍出,而自為殿,血戰萬寇中。扎拉豐阿、觀音保皆死。明瑞負創行二十餘里,手截辮發授其僕歸報,而縊於樹下,其僕以木葉掩尸去。

事聞,上震悼,賜祭葬,謚果烈。建旌勇祠京師,諸將死事者扎拉豐阿、觀音保、李全、王廷玉,命並祀,珠魯訥以自戕不與。額爾登額及提督譚五格坐失機陷帥,逮詣京師,上廷鞫,用大逆律磔額爾登額,囚其父及女,並族屬戍新疆;譚五格亦棄市,而以其明日祭明瑞及扎拉豐阿、觀音保,上親臨奠。

明瑞無子,以奎林子惠倫為嗣,襲爵。自侍衛累遷奉宸院卿。嘉慶初,剿教匪湖北,自荊門、宜城逐賊入南漳山中,賜玉搬指、荷包;復逐賊至長坪,射賊渠,殪,餘賊兢集,中槍死,賜白金三千。

論曰:藻起詞科,以廉被主知,陟歷中外。應琚持節臨邊,著聲績。要皆不習軍旅,措注失條理,事敗身殉。明瑞深入,度敵不可勝,遣諸軍徐出,而躬自血戰,誓死不反顧,功雖不成,忠義凜烈,足以讋敵矣!


\end{pinyinscope}