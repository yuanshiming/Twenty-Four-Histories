\article{列傳三十}

\begin{pinyinscope}
沙爾虎達子巴海安珠瑚劉之源吳守進巴山張大猷喀喀木

梁化鳳子鼐劉芳名胡有升楊名高劉光弼劉仲錦

沙爾虎達,瓜爾佳氏,其先蘇完部人,居虎爾哈。太祖時,從其父桂勒赫來歸,授牛錄額真。天命初,從伐瓦爾喀部,有功,授世職備御。天聰元年,太宗自將伐明,攻大凌河,圍錦州,沙爾虎達以噶布什賢章京從,屢戰輒勝。三年,復從伐明,拔遵化,薄明都,沙爾虎達戰郭外,敗明兵,進世職游擊。自是數奉命與噶布什賢章京勞薩等率游騎入明邊,往來松山、杏山間,獲明邏卒十八及牙將為邏卒監者,並得牲畜、器械甚夥。大凌河城下,明將祖大壽降,既,復入錦州為明守。上遣諸將略錦州,使沙爾虎達懸書十三站山坡諭大壽。九年,與白奇超哈將領巴蘭奇等徇黑龍江,加半個前程。冬,復略錦州,還,獻俘,命分賚將士。

崇德元年,從伐朝鮮,破敵南漢山城。二年,列議政大臣。甲喇額真丹岱、阿爾津等如土默特互市,上盧明兵要諸途,命沙爾虎達帥師詣歸化城護行。三年,與噶布什賢噶喇依昂邦吳拜將八十人行邊,至紅山口,遇明兵,斬裨將二;擊走明騎兵自羅文峪至者,搴其纛,得馬四十;又破明步兵自密雲至者,斬百餘級。四年,上自將伐明,沙爾虎達將噶布什賢兵自義州向錦州,復將土默特兵二百人入寧遠北境,與甲喇額真蘇爾德、鄂碩、布丹為伏,以數騎致明師,明師堅壁不應,乃掠其採薪者以歸。五年,進世職二等甲喇章京。

六年三月,從睿親王多爾袞圍錦州,坐從王令離城遠駐,當奪職,籍家產之半,上命罰鍰。八月,遷噶布什賢噶喇依昂邦。上自將禦洪承疇,部分諸將擊敵,賜沙爾虎達馬,使將所部屯高橋東界,諭曰:「敵敗,當自杏山西臺截大道躡擊之,毋使入城。」且誡之曰:「汝平日行不逮言,今當自勉!」既戰,明師敗,沙爾虎達違節制,縱潰兵二百餘入城。上命系而問之,沙爾虎達稽首對曰:「殺臣祗一死,宥當效命。」上乃宥之,降授甲喇額真。七年,與珠瑪喇率師伐虎爾哈部,降喀爾喀木等十七人、戶千餘,得馬騾牲畜。師還,宴勞,賚布帛有差。

順治元年,伐庫爾喀,伐黑龍江,皆有功。復從擊李自成,破潼關。二年,從攻江寧,下杭州,進世職一等甲喇章京。四年,授梅勒額真。帥師屯東昌,討平土寇丁維嶽、張堯中,加半個前程。五年,從討江西叛將金聲桓。遷巴牙喇纛額真,復為議政大臣。六年,定河間土寇。七年,調鑲藍旗滿洲梅勒額真。累進一等阿思哈尼哈番。九年七月,命帥師駐防寧古塔。十年,擢固山額真,仍留鎮,賜冠服、鞍馬。十五年七月,俄羅斯寇邊,沙爾虎達擊之走,多所俘馘。十六年,卒,謚襄壯。以其子巴海襲。

巴海初以牛錄額真事世祖,累遷秘書院侍讀學士。既襲世職,上諭吏部曰:「寧古塔邊地,沙爾虎達駐防久,得人心。巴海勤慎,堪代其父。授寧古塔總管。」十七年,俄羅斯復寇邊,巴海與梅勒章京尼哈裡等帥師至黑龍江、松花江交匯處,詗敵在飛牙喀西境,即疾趨使犬部界,分部舟師,潛伏江隈。俄羅斯人以舟至,伏起合擊,我師有五舟戰不利。既,俄羅斯人敗,棄舟走,巴海逐戰,斬六十餘級。俄羅斯人入水死者甚眾,得其舟槍砲若他械,因降飛牙喀百二十餘戶。敘功,加拖沙喇哈番。明年,以巴海奏捷諱未言有五舟戰不利,盡削原襲及功加世職。

康熙元年,改設黑龍江將軍,仍以命巴海。十年,上東巡,詣盛京,巴海朝行在。上問寧古塔及瓦爾喀、虎爾哈諸部風俗,巴海具以對。諭曰:「朕初聞爾能,今侍左右,益知爾矣。飛牙喀、赫哲雖服我,然其性暴戾,當迪以教化。俄羅斯尤當慎防。訓練士馬,整備器杭州,進世職一等甲喇章京。四年,授梅勒額真。帥師屯東昌,討平土寇丁維嶽、張堯中,加半個前程。五年,從討江西叛將金聲桓。遷巴牙喇纛額真,復為議政大臣。六年,定河間土寇。七年,調鑲藍旗滿洲梅勒額真。累進一等阿思哈尼哈番。九年七月,命帥師駐防寧古塔。十年,擢固山額真,仍留鎮,賜冠服、鞍馬。十五年七月,俄羅斯寇邊,沙爾虎達擊之走,多所俘馘。十六年,卒,謚襄壯。以其子巴海襲。

巴海初以牛錄額真事世祖,累遷秘書院侍讀學士。既襲世職,上諭吏部曰:「寧古塔邊地,沙爾虎達駐防久,得人心。巴海勤慎,堪代其父。授寧古塔總管。」十七年,俄羅斯復寇邊,巴海與梅勒章京尼哈裡等帥師至黑龍江、松花江交匯處,詗敵在飛牙喀西境,即疾趨使犬部界,分部舟師,潛伏江隈。俄羅斯人以舟至,伏起合擊,我師有五舟戰不利。既,俄羅斯人敗,棄舟走,巴海逐戰,斬六十餘級。俄羅斯人入水死者甚眾,得其舟槍砲若他械,因降飛牙喀百二十餘戶。敘功,加拖沙喇哈番。明年,以巴海奏捷諱未言有五舟戰不利,盡削原襲及功加世職。

康熙元年,改設黑龍江將軍,仍以命巴海。十年,上東巡,詣盛京,巴海朝行在。上問寧古塔及瓦爾喀、虎爾哈諸部風俗,巴海具以對。諭曰:「朕初聞爾能,今侍左右,益知爾矣。飛牙喀、赫哲雖服我,然其性暴戾,當迪以教化。俄羅斯尤當慎防。訓練士馬,整備器械,毋墮其狡謀。爾膺邊方重任,當黽勉報知遇!」

邊外有墨爾哲之族,累世輸貢,巴海招之降。其長扎努喀布克托等請內徙,巴海請徙置寧古塔近地,置佐領四十,以授扎努喀布克托及其族屬,分領其眾,號為新滿洲。十三年冬,巴海率諸佐領入覲,上錫予有差,賜巴海黑狐裘、貂朝衣各一襲。十七年,敕獎巴海及副都統安珠瑚撫輯新滿洲有勞,予世職一等阿達哈哈番兼拖沙喇哈番。

二十一年,巴海疏言官兵捕採葠者,當視所得多寡行賞。上為下部議,並誡非採葠者毋妄捕。是歲,上復東巡,詣盛京,幸吉林,察官兵勞苦。既還京師,諭巴海罷採鷹、捕鱘鰉諸役。二十二年,以報田禾歉收不實,部議奪官,削世職,上猶念巴海撫輯新滿洲有勞,命罷將軍,降三等阿達哈哈番。二十三年,授鑲藍旗蒙古都統,列議政大臣。三十五年,卒。子四格,襲職。

安珠瑚,瓜爾佳氏,滿洲正黃旗人,先世居蘇完。父阿喇穆,任牛錄額真。順治元年,從入關,擊李自成,戰死,授世職半個前程。安珠瑚襲職,遇恩詔,累進三等阿達哈哈番。擢甲喇額真,兼刑部郎中。從大將軍伊爾德攻舟山,從將軍濟什哈討萊州土寇於七,皆有功。康熙六年,授寧古塔副都統。十五年,增設吉林烏喇副都統,以命安珠瑚,佐巴海撫新滿洲,進世職如巴海。十七年,擢盛京將軍。二十一年,上東巡,見邊界多戰骨暴露,諭-9587-安珠瑚遍察收瘞。二十二年,以疾乞休,上責安珠瑚失職,奪官,發吉林烏拉效力。二十四年,授索倫總管。二十五年,卒。

安珠瑚入對,嘗言所轄士兵皆藐視之,上知其庸懦,及卒,命削其世職。

劉之源,漢軍鑲黃旗人。天聰九年,授甲喇額真。崇德五年,從上伐明,攻錦州,距城東五里發砲隳其臺。復列砲城北擊晾馬臺,殪明兵。尋代馬光遠為正黃、鑲黃二旗漢軍固山額真。六年,從睿親王多爾袞圍松山,發砲隳臺四,獲明將王希賢、崔定國、楊重鎮等,又斬裨將三。七年,從鄭親王濟爾哈朗圍塔山,列砲城西,毀其垣二十餘丈,殲城兵,隳杏山城北臺,又擊毀其垣,城兵懼,乃出降,授世職二等甲喇章京。分設漢軍八旗,之源仍領鑲黃旗。八年,從鄭親王攻克中後所,斬明將吳良弼、王國安等;進攻前屯衛,發砲隳其城:進世職一等。

順治元年,從入關,命與固山額真李國翰剿定畿南餘寇。復從固山額真葉臣等西征,克太原。又與固山額真巴哈納自汾州逐寇至平陽,斬馘四千餘。山西寇始盡。師還,優賚。二年,從順承郡王勒克德渾下湖廣,討李自成,與國翰合師破應山。降將馬進忠復叛,與固山額真金礪擊敗之武昌,得舟六十餘,遂徇湖北。五年,授定南將軍,從鄭親王再下湖-9588-廣。六年,攻湘潭,明總督何騰蛟分三隊出戰,之源分兵應之,敗明兵,克其城,獲騰蛟。夜督兵逐進忠,平明劘其壘。復進克寶慶,並破南山坡九壘,斬明將馬有志、胡進玉等,進忠跳而免。又擊破明將袁宗第於洪江、王永強於便水驛。敘功,遇恩詔,世職累進一等阿思哈尼哈番兼拖沙喇哈番。

八年,與金礪駐防杭州。明大學士張肯堂與其將阮進、張名振擁魯王以海屯舟山,之源與總督陳錦、總兵田雄合師攻之,破明兵於橫水洋,獲進。逼螺頭門,肯堂城守十餘日,師以雲梯登,肯堂及魯王諸臣李向中、吳鍾巒、硃永佑等縱火自焚死。名振以魯王遁三盤島,之源遣總兵馬進寶等追擊破之,焚其積聚;復敗之於沙埕,收各奧戶口八千五百餘,悉令歸農。論功,進三等精奇尼哈番。

十六年八月,授鎮海大將軍,駐防鎮江。疏言:「京口百川匯流,江南財賦自此輓運北輸。近因鄭成功入犯,幾至橫截運道。宜先練習水師,以資防禦。防海策有三:出海會哨,勿使入江,上也;循塘拒敵,勿使登陸,中也;列陣備兵,勿使近城,斯下矣。顧練水師當先造船,火器、水手、舵工,百無一備,何以禦賊?」上敕兵部下總督郎廷佐制備。十七年,疏言:「京口水師造船二百,募水手、舵工八千餘,一時難以集事。沿海民有雙桅沙船,造作堅固,其人熟於洋面水道,請查驗船堪用者予收用,船戶給以糧餉。舊設戰船低小,不必修-9589-補。邊海砲臺、烽墩、橋路,請敕督撫下沿海州縣修葺高廣。」下兵部,並從之。尋得成功遣諜與提督馬逢知關通狀,疏聞,命侍郎尼滿會之源鞫實,逢知坐誅。

康熙三年,召還京,仍任都統。四年,以病乞休,加太子太保,致仕,以其子光代為都統。鰲拜得罪,之源、光坐黨附,奪官論死,上命寬之。之源尋卒。妻胡叩閽,訴之源功罪足相當,詔復官,並予三等精奇尼哈番,仍以光襲。三傳,降襲三等阿思哈尼哈番。乾隆初,定封三等男。

吳守進,漢軍正紅旗人,初籍遼陽。太祖時來歸,從征伐有勞,授世職游擊。天聰五年,授戶部承政。八年,考滿,進世職一等甲喇章京。時始設漢軍世管牛錄額真,命守進兼任。崇德三年,改左參政。四年,坐賕,論罪至死,命貸之,削世職,解參政,籍其家之半,仍攝正紅旗漢軍梅勒額真。旋真除。

六年,從睿親王多爾袞、武英郡王阿濟格攻錦州,守進發砲克塔山四臺,獲明將王希賢、崔定國等,多所斬馘。七年,擢本旗固山額真。率師攻松山、杏山,明兵屯呂洪山口,與固山額真金礪擊破之。明兵保杏山側二臺,復與固山額真劉之源擊破之,遂拔杏山。尋命與梅勒額真馬光輝等詣錦州督鑄砲。八年,從攻寧遠,取中後所、前屯衛。

順治元年,從入關,復授世職二等甲喇章京。從固山額真葉臣徇山西,克太原。復從豫親王破李自成,下江南,敗明師,克揚州、江陰,復進破嘉興。敘功,進一等。四年,授定西將軍,駐漢中。五年,卒。子國柄襲。從征湖廣,官梅勒額真,加世職拖沙喇哈番。

巴山,瓜爾佳氏,滿洲鑲黃旗人,世居哈達。祖巴岱,國初率眾來歸,授世管牛錄額真。再傳至巴山。天聰五年,從太宗伐明,圍大凌河。城兵出戰,梅勒額真屯布祿、牛錄額真郎格等戰沒,巴山馳入陣,以其尸還。六年,從伐察哈爾,其部人竄入大同,往取之。師還,巴山與承政圖爾格殿,明兵追襲,設伏邀擊,斬馘甚眾。八年,授世職牛錄章京。尋擢甲喇額真。

崇德元年,從上伐朝鮮,與甲喇額真屯泰等先眾破敵。三年,兼任工部理事官。從貝勒岳託伐明,自墻子嶺入邊,薄明都,擊敗明太監馮永盛兵;攻鉅鹿,率所部以雲梯先登,克之:加半個前程。五年,與承政薩穆什喀、索海等伐虎爾哈部,攻掛喇爾屯。七年,從奉國將軍巴布泰率師駐錦州。

順治元年,從入關,督所部步兵擊敗李自成,擢工部侍郎,進世職三等阿達哈哈番。二年,授梅勒額真,鎮守江寧。三年,命總管江寧駐防滿洲兵,特置總督糧儲兼理錢法,駐江寧,以協領鄂屯兼任,加戶部侍郎,以重其事。時江北諸山寨並起,號為明守。江寧民有謀為應者,巴山詗知之,捕斬三十人。未幾,明潞安王誼石以二萬人分三道攻江寧,巴山會招撫大學士洪承疇等督兵御之,誼石敗走。語詳承疇傳。明故左通政嘉定侯峒曾以二年死難,四年,其子元瀞通表魯王以海,取敕書及其將黃斌卿致承疇書以歸。柘林游擊陳可得之,有「內殺巴、張二將」語,指巴山及提督張大猷也。事聞,上以敵謀設間,詔慰承疇,而諭獎巴山及大猷「嚴察亂萌,公忠盡職」。

六年,江南總督馬國柱討六安山寇,巴山及大猷以師會,斬其渠張福寰,諸寨悉平,進三等阿思哈尼哈番。是歲,裁總督糧儲錢法,不復置。九年,將軍金礪討鄭成功,請益師,部議調江寧駐防兵二百,鄂屯與理事官額赫納、烏庫理率以行,攻海澄。成功兵劫我軍砲,鄂屯與額赫納擊卻之。成功兵十餘萬逆戰,鄂屯督兵縱擊,成功兵退,斷橋。鄂屯與烏庫理策馬逕渡,成功兵潰,破其壘數十,降數千人。尋召巴山還京,以喀喀木代。十一年,復錄江寧平寇功,進世職二等。康熙十二年,卒。

子舒恕,襲世職。從大學士圖海討王輔臣,次平涼城北虎山苾,擊敗輔臣兵。復從都統穆占討吳三桂,擊敗三桂兵於松滋,進圍雲南,屢敗吳世璠將胡國柄、劉龍、黃明等,又困其將馬寶、巴養元等於烏木山。論功,進世職一等。卒,子長清,改襲一等阿達哈哈番。

張大猷,漢軍鑲黃旗人,初籍遼陽。太祖克遼陽,大猷以千總自廣寧來降,授牛錄額真。天聰初,明邊將遣諜招我新附之眾,大猷發其事。太宗嘉之,予世職游擊。崇德三年,授刑部理事官。尋擢漢軍梅勒額真。四年,更定漢軍旗制,授鑲黃旗梅勒額真。五年,從睿親王多爾袞圍錦州,率本旗兵攻五里臺及晾馬山、馬家湖,皆下,又克金塔口臺。六年,從鄭親王濟爾哈朗圍錦州,明騎兵自松山至,謀奪砲,大猷擊卻之。復與固山額真劉之源等攻克塔山、杏山及附近諸臺。論功,進二等甲喇章京。七年,遷兵部參政。十月,從貝勒阿巴泰伐明,築橋渾河濟師,擊破明總督範阿衡軍。八年,從攻寧遠,取中後所、前屯衛,進世職一等。

順治元年,從固山額真葉臣徇山西,克太原,與固山額真李國翰撫定諸郡縣。二年,師定江南,與固山額真吳守進下浙江,次石門,明兵自杭州夜來襲,卻之。還,克嘉興。三年,命與巴山率兵鎮守江寧,總管漢軍及綠旗兵。旋授提督江南總兵官。論功,進世職三等梅勒章京。六年,同討張福寰。總督馬國柱奏大猷身先士卒,履險摧鋒,功第一,進世職三等精奇尼哈番。九年,卒。三傳,降襲三等阿思哈尼番。乾隆初,定封三等男。

喀喀木,薩哈爾察氏,滿洲鑲黃旗人,先世居烏喇部。父塘阿禮,當太祖時,率百人來歸,授牛錄額真。從伐遼東有功,予世職游擊。從伐瓦爾喀,射熊,為所傷,卒。

喀喀木嗣領牛錄。崇德三年,授吏部理事官。五年,從伐虎爾哈部,敵據柵拒戰,喀喀木督兵破柵,斬級二百,俘一百三十。七年,從伐明,攻松山,本旗將領失律未察舉,降世職牛錄章京。八年,擢吏部參政。順治元年,署梅勒額真。從入關,加半個前程。尋改侍郎。四年,復世職三等甲喇章京。鄖陽總兵王光恩坐事逮系,其弟光泰叛據鄖陽,提督孫定遼戰死,勢甚張,上命喀喀木率兵討之。師將薄鄖陽,光泰遁走,喀喀木與副將王平率師逐捕,戰房縣,斬級千餘。光泰走四川,喀喀木駐軍鄖陽。

五年,金聲桓自江西窺湖廣,總督羅繡錦疏請留喀喀木駐荊州。六年,召還。七年,授鑲黃旗梅勒額真,世職累進三等阿思哈尼哈番。八年,命與固山額真噶達渾等率兵討蒙古鄂爾多斯部長多爾濟。九年,師出寧夏,至賀蘭山,擊斬多爾濟,並殲其部眾,俘其餘以歸,得馬駝數百、牛千餘、羊萬餘。

尋命代巴山為鎮守江寧總管。十年,明將李定國兵犯廣東,潮州總兵郝尚久叛應之,授喀喀木靖南將軍,率師會靖南王耿繼茂討尚久。圍逾月,督兵以雲梯登,尚久入井死,潮州及旁近州縣皆定。還駐江寧。

十六年,鄭成功大舉入犯,破鎮江,復陷瓜洲,溯江上。喀喀木與總督郎廷佐、提督管效忠謀禦敵,檄總兵梁化鳳赴援。會梅勒額真噶哈、瑪爾賽自貴州旋師,循江東道江寧,入城同守。喀喀木曰:「賊勢盛,宜乘其未集先擊之。」簡精銳逆擊,成功前軍為少卻,得舟二十餘。俄成功兵大至,連營八十有三,舟蔽江,喀喀木晝夜防守。化鳳援兵至,乃議使綠旗兵先出戰。化鳳出儀鳳門,效忠出鍾阜門,夾擊,破成功兵,獲其將餘新等。明日,喀喀木與噶哈、瑪爾賽督兵出神策門,成功兵據白土山列陣,乃分兵左右仰攻,與化鳳率精銳搗其中堅,獲其將甘輝及裨佐數人,斬馘無算。成功兵潰,走出海。事聞,部議失鎮江、瓜洲當議罪,上以固守江寧功大,命免議。

康熙元年,改總管為將軍,仍以命喀喀木。七年,卒,授其子喇揚阿一等阿達哈哈番兼拖沙喇哈番。

梁化鳳,字翀天,陜西長安縣人。順治三年武進士。四年,授山西高山衛守備。五年,從英親王阿濟格討叛將姜瓖,克陽和城,擒瓖將郭二用。擢大同掌印都司。時大同、左衛、渾源、太原、汾、澤群盜競起應瓖。六年,化鳳攻大同,破北窯溝,寇據山巔,懸柴以火燔之,獲其渠李義、張豹。攻渾源,徇韓村、玉合堡、張家堡,破賈莊,獲其渠王平;乃克渾源,又獲其渠方三、唐虎誅之。攻左衛,降雲岡、高山二堡,遂合圍。化鳳中三矢,戰愈力,寇以城降。敘功,超加都督僉事,以副將推用。進攻太原,寇出戰,化鳳左臂中槍,矢集於髀,益奮斗,執所置巡撫姜建勛,乃克太原。進解平陽圍,攻汾州,敗其渠沈海。攻孝義,寇渠張爾德來援,與戰大破之,乃克汾州,獲爾德。海復以兵至,再戰擊敗之,走潞安。迭下曹家堡、記古寨、善信堡。介休、平遙、祁、徐溝諸縣悉降。進攻太谷,克之,獲其渠蘇升,乃克潞安,海走九仙臺。拔長子,進攻九仙臺,山峻,騎不得上;以火攻之,寇不支,海出降。進定澤州。是歲凡二十二戰皆捷。七年,復殲餘寇於牛鼻寨,獲其渠袁忠。山西悉定。

八年,借補江南蕪永營參將。討平石皿、鷺鷥二湖盜,獲其渠楊萬科。十二年,擢浙江寧波副將。明將張名振屯崇明平洋沙,總督馬國柱檄化鳳署蘇松總兵。名振攻高橋,化鳳馳赴戰,迭擊敗之,遂復平洋沙。十三年,真除蘇松總兵。化鳳以平洋沙懸隔海中,戍守不及。沿海築壩十餘里使內屬,並引水灌田,俾海濱斥鹵化為膏腴。

會鄭成功攻崇明,遣諜疑眾,化鳳擒斬之,督兵迎戰,獲其將侯丁秀、宮龍、陳義等。又遣諸將設伏,斬其將陳正,縛致曾進等十一人。成功引去,七月,復大舉入寇,連舟蔽江,號百萬,陷鎮江,直犯江寧,南北中梗。化鳳將所部兵三千人疾馳赴援,升高了敵,見成功軍不整,樵蘇四出,軍士浮後湖而嬉,乃率五百騎夜出神策門,破白土山敵壘。明日,督兵出儀鳳門,提督管效忠出鍾阜門,夾擊搏戰,拔巨纛,毀其木寨,簡驍勇乘屋,發火器,矢石雜下,成功兵奔潰,逐至龍江關,獲其將餘自新等。成功收餘眾,連營屯白土山,眾猶數十萬。又明日,復與總管喀喀木等出神策門,直攻白土山,督將士仰擊,寇迎拒,殊死戰。甘輝者,成功驍將也,化鳳入陣生獲之。成功兵奪氣,遂奔不可止,逐北斬馘。迫江上,化鳳先遣別將焚其舟,成功兵自蹂藉及入江死者無算。成功遁入海,化鳳策成功當還攻崇明,先遣別將為備。成功出海攻崇明,化鳳自江寧還援;成功度不能克,括民舟將渡白茅口,化鳳與相值,絕流迅擊,砲石蕩海波,成功復大敗,跳而免。敘功,授世職三等阿達哈哈番,賜金甲、貂裘。

十七年,擢蘇松提督,加太子太保、左都督。化鳳疏言:「蘇、松濱海,地袤八百餘里,標兵止二千餘。請酌調省兵三千八百,立六營,資捍禦。」下部議,從之。十八年,上復錄化鳳功,進世職三等阿思哈尼哈番。尋裁江安廬鳳提督,以化鳳為江南提督。時議者以臺灣未復,用廣東、福建例,蘇、松濱海立界,徙居民於內地。化鳳曰:「沿海設兵,賦擬棄之地以養之。國既足兵,民無廢業,遷界何為?」奏入,上從其言。康熙十年,卒,贈少保,謚敏壯。聖祖巡西安,遣官祭其墓。乾隆初,定封三等男。

鼐,其次子也。以廕授川陜督標左營游擊。吳三桂亂起,總督哈占令鼐率兵駐黑水峪,敗王輔臣之兵於觀音堂。調興安城守游擊。從征漢中,戰屢捷,克達州,加都督僉事。三遷至福建陸路提督。四十五年,擢福建浙江總督。上南巡,書「旂常世美」字賜之。初,金世榮為總督,謂出洋大船易藏盜,奏定漁船不得用雙桅,商船悉令改造,樑頭不得過丈有八尺。鼐力言無益海疆,徒累於商民,上命弛其禁。四十七年,疏言嘉、湖諸水皆洩入太湖,通津要道,發帑疏治;支河淤淺,勸民開濬。上諭支河勸民開濬,慮有司藉此私派,當並發帑疏治。四十九年,以母喪去官。五十三年,卒。

劉芳名,字孝五,漢軍正白旗人,初籍寧夏。仕明至柳溝總兵。順治元年,降,命仍原官。二年,調寧夏,賜白金、冠服。時陜西初定,多盜,悍卒復伺隙謀亂。芳名撫綏訓練,冀樹威望,銷亂萌,總督孟喬芳疏獎其才。武大定叛固原,賀珍叛漢中,師進討,芳名皆有功。三年,方赴鞏昌剿寇,寧夏兵遽變,戕巡撫焦安民。芳名馳還,察知裨將王元、馬德首亂,遣德署花馬池副將,分元勢;偵元將出城就寇渠洪大誥,芳名設伏,俟元至,伏發,元力拒,諸將樊朝臣、姜九成等衷擊之,元敗奔,副將馬寧等追擊,獲以歸。芳名別遣將搜斬大誥。德聞元誅而懼。四年春,芳名偕河東道馬之先出師惠安,德乘間糾黨劫軍資,遁入山,合寇渠賀宏器等自紅古城出口,襲破安定。螺山寇王一林戕參將張純以應之,橫行寧固、平慶間。芳名督所部兵進次亂麻川,破賊;復進次預望城,再破賊,斬一林,德以四騎走,追及之河兒坪,縛而磔之,亂乃定,授三等阿達哈哈番,擢四川提督、定西將軍。尋命以右都督留鎮寧夏。五年,討平香山寇李彩。

馬德之誅,副將劉登樓預有功。登樓居榆林寧塞,多力而狡。六年,以延安叛應姜瓖,易衣服,自署「大明招撫總督」,戕靖邊道夏時芳,騰書致芳名。芳名以見汙,怒,封其書示巡撫李鑒,鑒以聞。登樓西犯花馬池,下興武諸營堡,逼寧州。時定邊屯蒙古札穆素叛逃賀蘭山,芳名遣兵擊破登樓,登樓走定邊屯,結札穆素寇寧夏西境,犯河東,陷鐵柱、惠安、漢伯諸堡。將犯靈州,會固山額真李國翰師至,乃定策:鑒守寧夏,御札穆素;芳名引兵東渡河,趨榆林,與登樓戰於官團莊,大破之。登樓退據漢伯,師從之,絕其水道,遂合圍。芳名督兵逼壘東南,當矢石沖。諸將進曰:「當移數武避賊鋒。」芳名厲色叱之曰:「死則死耳,何懼為?且士卒多傷痍,而我避鋒鏑可乎?」士卒益奮,攻十二日,克之,斬登樓,餘眾悉降。

亂定,進世職二等。疏言:「寧夏孤懸河外,延袤千里。鎮兵屢徵發,兵單力薄。請自後徵發缺額,即令招補備守御。」又請以減等罪人僉發沿邊,資生聚。皆下部議行。

十六年,調隨征江南右路總兵,加左都督,率寧夏三營駐江寧。鄭成功攻崇明,芳名與提督梁化鳳共擊敗之。十七年,疏言:「臣奉命剿賊,不意水土未服,受病難瘳。所攜寧夏軍士,訓練有年,心膂相寄。今至南方,半為痢瘧傷損。及臣未填溝壑,敢乞定限更調。」上報以優旨。旋卒於軍,加太子太保,謚忠肅。命所部將士還寧夏本鎮。

胡有升,錦州人。崇德元年,睿親王多爾袞、豫親王多鐸率師攻錦州,有道人崔應時者,與州民張紹禎,門世文、世科,秦永福等謀以城降,使有升持書詣師,期內應。豫親王與書齎還。明將詗知之,執應時等下獄。有升與紹禎、世文、世科、永福脫走來歸,各賜冠服、鞍馬、妻室、奴僕。授世職,有升得三等梅勒章京,隸漢軍鑲黃旗。屢從征伐,進二等。

順治四年,授南贛總兵。五年,金聲桓、王得仁以南昌叛,犯贛州。副將高進庫出戰而敗,巡撫劉武元與巡道張鳳儀分守城東西,有升率健卒循城策應。得仁兵穴城,將置火具仰攻,有升以石窒其竇。圍三月,糧匱,有升出戰,得仁敗走。聲桓聞征南大將軍譚泰師至,引退,有升督兵迫擊,多所斬馘。未幾,李成棟復來攻,有升乘成棟兵方鑿壕,出戰大破之。語互見武元傳。初,柯永盛自南贛總兵遷湖廣總督,請以鎮兵二千自隨。有升疏言:「贛地江、湖關鍵,聲桓亂未平,鎮兵習水土,便徵剿,宜遣還鎮。」上從之。六年,聲桓誅,成棟走死。譚泰師還,土寇猶未靖,上猶劉飛、龍南葉芝、石城鄒華、雩都彭順慶、瑞金陳其綸,皆負固為亂,有升與武元分遣諸將次第討平之。敘功,加左都督,賜紫貂冠服、甲胄、佩刀、鞍馬,進世職三等精奇尼哈番。

十年,以尚可喜、耿繼茂疏論有升功,復加太子少保。十七年,以老解官。康熙三年,武元子瀇疏請加敘守贛州功,有升亦以請,進一等。九年,卒,子啟泰襲,改隸正白旗。再傳,降襲一等阿思哈尼哈番。乾隆初,定封一等男。

楊名高,漢軍鑲黃旗人,初籍遼東。太宗時,率其族百餘人來歸,授牛錄額真,兼任兵部理事官。崇德間,屢從征伐,克塔山、杏山,擊敗明總督範志完,取前屯衛、中後所,皆在行。順治元年,授世職牛錄章京。二年,遷甲喇額真。三年,擢都察院參政。

六年,授福建漳州提督。明新建王由模據大田,■H0延平高峰諸土寇為亂。七年,名高率師破石磯巔,由模走永安,副將王愛臣追獲之。高峰寨渠陳光等招德化土寇鄭文薦來援,名高令副將韓尚亮等率師截擊,圍寨。光奪圍走,名高督兵奮擊,寇多墮壕死。師進次大田,寇潰走,敗之龍門橋,擒其將郭奇,廖明正,諸寨悉降。

尋又率師徇邵武,寇走入江西新城,名高分兵三道進,與總兵王之綱殿,逐寇三十餘里,擒其將洪國玉、李安民、王恆美等,得牛馬、槍砲無算。敘功,進世職二等阿達哈哈番。九年,鄭成功自廈門陷長春、漳浦、海澄、南靖諸縣,以二十餘萬人寇漳州,屯鳳巢山。名高督兵擊破之,成功退屯海澄,所陷諸縣皆復。尋復出,陷漳州及所屬諸縣。給事中魏裔介劾名高怠玩,下總督佟岱按治,坐奪官。尋卒。

劉光弼,漢軍鑲藍旗人,初籍遼陽,冒曹氏。天聰五年,命守耀州。率兵從太宗伐明,圍大凌河,克城旁三臺。城兵出戰,光弼先眾馳擊,我兵有陷陣者,力援之出。明監軍道張春、總兵吳襄等自錦州赴援,光弼馳入陣,斬其裨將。崇德五年,授甲喇額真。從攻錦州,與墨爾根轄李國翰同克呂洪山諸臺。屢擊敗松山、杏山馬步兵。明兵屯山口阻我師,與國翰督兵奮戰,明兵引去。錦州既下,發砲攻克塔出、杏山兩城,及附近臺堡。敘功,予世職牛錄章京。七年,擢鑲藍旗漢軍梅勒額真。八年,偕固山額真劉之源詣錦州督鑄砲。尋從鄭親王濟爾哈朗攻寧遠,取前屯衛、中後所。

順治元年,從入關,擊李自成。旋從固山額真葉臣徇山西,克太原。三年,從端重親王博洛下浙江,拔金華,進定福建。五年,授禮部侍郎。從大將軍譚泰討金聲桓,克南昌,譚泰疏請以光弼署江西提督。六年,平廣昌土寇,旋命真除。土寇張自盛、洪國玉等據湖東為亂。光弼督參將陳升等討平之。其黨董明魁、郭承氏等皆降。遇恩詔,世職累進一等阿達哈哈番。十三年,賜鞍馬、弓矢。十六年,以老病致仕。康熙十二年,卒。

劉仲錦,漢軍正藍旗人,初籍遼陽東寧衛。崇德五年,以牛錄額真從睿親王多爾袞等伐明,圍錦州,騎兵千餘出迎戰,仲錦擊破之,追薄城下始還。復擊敗松山、杏山、呂洪山口敵兵。七年,從鄭親王濟爾哈朗等攻塔山,發砲擊城圮,仲錦率所部兵先登,克之。進攻杏山,復發砲擊城,毀其垣,城人遂降。敘功,予世職半個前程。八年,從巴牙喇纛章京阿爾津、哈寧阿等伐黑龍江虎爾哈部,克博和哩、諾爾噶勒、都裏三屯,降大小噶勒達蘇、綽庫禪、能吉勒四屯。賜貂皮、白金。復從攻寧遠,取中後所、前屯衛。進世職甲喇章京。

順治元年,從入關,授戶部理事官,兼甲喇額真。從固山額真葉臣等徇山西,克太原。又從英親王阿濟格西討李自成,自陜西下湖廣,敗其將馬進忠,得舟十一。五年,擢兵部侍郎。六年,從睿親王討姜瓖,攻渾源、左衛,進攻汾州,皆發紅衣砲克之。七年,授山東臨清總兵,加都督同知,世職累進一等阿達哈哈番加拖沙喇哈番。十年,改福建右路總兵,加左都督,駐泉州。十一年,以疾解任。旋卒。

論曰:滿洲兵初入關,分駐都會,其後乃久屯,置總管。沙爾虎達招徠新滿洲,劉之源、巴山、喀喀木鎮撫江南,喀喀木合群力摧大敵,厥功尤著。漢兵入關後來附者,不復入烏真超哈,循舊制分設提鎮。化鳳援江寧,與喀喀木同功。芳名偕馬之先守寧夏,有升佐劉武元守贛州,皆有殊績。名高等以卿貳出專閫,亦能稱其職者。若富喀禪鎮西安,烏庫理守盛京,皆見於他篇,故不復著。


\end{pinyinscope}