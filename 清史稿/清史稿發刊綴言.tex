\article{清史稿發刊綴言}

\begin{pinyinscope}
爾巽承修清史十四年矣。任事以來,慄慄危懼。蓋既非史學之專長,復值時局之多故,任大責重,辭謝不獲,蚊負貽譏,勉為擔荷。開館之初,經費尚充,自民國六年,政府以財政艱難,銳減額算。近年益復枯竭,支絀情狀,不堪縷述。將伯呼助,墊借俱窮,日暮途遠,幾無成書之一日。竊以清史關系一代典章文獻,失今不修,後來益難著手,則爾巽之罪戾滋重。瞻前顧後,寢饋不安。事本萬難,不敢諉卸。乃竭力呼籥,幸諸帥維持,並敦促修書同人黽勉從事,獲共諒苦衷,各盡義務,竭蹶之餘,大致就緒。本應詳審修正,以冀減少疵纇。奈以時事之艱虞,學說之厖雜,爾巽年齒之遲暮,再多慎重,恐不及待。於是於萬不獲已之時,乃有發刊清史稿之舉,委託袁君金鎧經辦,數月後當克竣事。誠以史事繁鉅,前史每有新編,互證得失。明史之修,值國家承平,時歷數十年而始成,亦不無可議之處,誠戛戛乎其難矣。今茲史稿之刊,未臻完整,夫何待言。然此急就之章,較諸元史之成,已多時日。所有疏略紕繆處,敬乞海內諸君子切實糾正,以匡不逮,用為後來修正之根據。蓋此稿乃大輅椎輪之先導,並非視為成書也。除查出疏漏另刊修正表外,其他均公諸海內,與天下人以共見,繩愆糾謬,世多通人。爾巽心力已竭,老病危篤,行與諸君子別矣,言盡於此。以上所述,即作為史稿披露後向海內諸君竭誠就正之語,幸共鑒之。

中華民國十六年丁卯八月二日趙爾巽時年八十四歲

清史館職名

△館長趙爾巽

△兼代館長總纂柯劭忞

△總閱於式枚

△總纂王樹枬郭曾炘李家駒繆荃孫吳士鑒吳廷燮馬其昶夏孫桐秦樹聲金兆蕃

△纂修鄧邦述章鈺王大鈞袁勵準萬本端陶葆廉王式通顧瑗楊鍾羲簡朝亮張採田何葆麟陳曾則姚永樸夏曾佑唐恩溥袁克文金兆豐

△協修俞陛雲羅惇曰融吳廣霈吳懷清張書云張啟後韓樸存李嶽瑞駱成昌胡嗣芬吳昌綬硃孔彰李景濂姚永概黃翼曾檀璣戴錫章陳曾矩李哲明呂鈺余嘉錫邵瑞彭奭良瑞洵陳田葉爾愷徐鴻寶王崇烈方履中商衍瀛陳能怡王以慜劉樹屏硃師轍史思培趙文蔚劉焜陳敬第藍鈺陳毅李葆恂張仲炘陳延韡宋伯魯李焜瀛喻長霖田應璜趙世駿楊晉齊忠甲硃希祖吳璆秦望瀾李汝謙羅裕樟傅增淯硃方飴

△提調李經畬陳漢第金還周肇祥邵章

△文牘科長伍元芝

△圖書科長尹良

△會計科長劉濟

△庶務科長錫廕

△收發處長張玉藻

△校勘孟昭墉諸以仁奎善劉景福趙伯屏

△收掌董清峻胡慶松秦化田史錫華惠澂

△總理史稿發刊事宜總閱袁金鎧

△辦理史稿校刻事宜總閱金梁

按:清史館職名,關外一次本與此相同,獨纂修唐恩溥之名誤列在協修史思培下誤出。關內本與此歧異頗多,附錄於後,以供參考。

清史館職名

館長趙爾巽

兼代館長總纂柯劭忞

總纂王樹棻

總纂吳廷燮

總纂夏孫桐

纂修金兆蕃

纂修章鈺

纂修金兆豐

協修俞陛云

協修吳懷清

協修張書云

協修李哲明

協修戴錫章

協修奭良

協修硃師轍

校勘兼協修孟昭墉

提調李經畬

提調陳漢第

提調金還

提調周肇祥

提調邵章

總纂繆荃孫馬其昶秦樹聲吳士鑒

纂修王大鈞鄧邦述姚永樸萬本端張爾田陳曾則唐恩溥袁勵準王式通何葆麟劉師培夏曾佑

協修張啟後李嶽瑞韓樸存硃孔彰姚永概黃翼曾陳敬第吳昌綬吳廣霈羅惇曰融駱成昌胡嗣芬李景濂陳田檀璣葉爾愷瑞洵王崇烈田應璜硃希祖徐鴻寶藍鈺劉樹屏楊晉陳能怡方履中商衍瀛趙世駿袁嘉穀秦望瀾吳璆史思培唐邦治張仲炘傅增淯邵瑞彭陳曾矩

校勘董清峻周仰公秦化田奎善劉景福趙伯屏史錫華曾恕傳

收掌尚希程王文著胡慶松

總理史稿發刊事宜袁金鎧

辦理史稿校刻金梁


\end{pinyinscope}