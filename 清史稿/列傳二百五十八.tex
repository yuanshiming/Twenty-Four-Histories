\article{列傳二百五十八}

\begin{pinyinscope}
盛宣懷瑞澂

盛宣懷,字杏蓀,江蘇武進人。以諸生納貲為主事,改官直隸州知州,累至道員。嘗贊置輪船招商局,開採湖北煤鐵礦,李鴻章頗信任之。英商擅築鐵軌,首滬逕寶山訖吳淞,上海道數阻,弗聽。宣懷與英官梅輝立折辯,償銀二十八萬有奇,始歸於我。光緒五年,署天津道。時鴻章督畿輔,方鄉新政,以鐵路、電報事專屬宣懷。宣懷以英、丹所設水陸線漸侵內地,乃集貲設津滬陸線,建電報學堂,並援萬國公例與爭,始克嚴定條款。會訂水線相接合同,於是與輪船招商同為商辦兩大局。八年,英、法、德、美議立萬國電報公司,增造自滬至香港水線,壟利權。宣懷復勸集華商自設緣海各口陸線,以絕覬覦。

十年,署天津海關道。會法越構釁,海防急。乃移金州礦貲治蘇、浙、閩、粵電線,便軍事,而部議指為含混,科以降級調用。左宗棠為言於上,事下南洋大臣曾國荃等,上其績狀,始改留任。十二年,授山東登萊青道。法領事林椿詣煙臺與訂越南北圻線約,朝旨既報可矣,而張之洞執言不可行。宣懷曰:「今琿春、海蘭泡欲接俄線,俄方有挾求。法既許接線,彼必易就範。且英、丹皆與約,奚拒法!」總署然之。果不數年而俄約成。十八年,除真。滬上織布局廠災,宣懷籌設華盛總廠,復任彌漢冶鐵廠虧耗。於是之洞賞其才,與王文韶交薦之,遂擢四品京堂,督辦鐵路總公司。入覲,奏言築路與練兵、理財、育才互為用,並請開銀行,設達成館,稱旨,補太常寺少卿。與比訂貸款草約。二十四年,詔趣造粵漢路。宣懷建議貸美款歸自辦,具改歸商辦本末以上;而言者盛毀其所為遲滯,被訶責。宣懷具報曲折,上乃慰而勉之。宣懷自請解職,仍留京會議洋貨稅則。已而徐桐劾兩局有中飽,適剛毅按事南下,銜命察覆。宣懷具以實對,奏上,被溫旨。

二十六年,拳禍作,各國兵艦紛集江海各口。宣懷倡互保議,電線、江、鄂、閩諸疆吏,獲同意,遂與各領事訂定辦法九條,世所稱東南保護約款是也。又電奏請下密詔平亂,發國電國書懲禍首,恤五忠,所言動關大計。事寧,加太子少保,除宗人府府丞。明年,充辦理商稅事務大臣。以和約既成,償費過鉅,乃奏豫籌四策,而注重加稅。復以債款稱息負累劇,請婉商各國,分攤免息。嗣與各國商加稅免釐,議垂成,英忽中悔。厥後宣懷數續議,仍無效。是歲奏設勘礦總公司。越二年,而有爭粵漢廢約事,滬寧、蘇杭甬踵之,眾大譁。詔禁宣懷干預,命唐紹儀代督兩局。宣懷遂奏罷鐵路總公司。後四年,浙路事益棘,上終以宣懷諳路政,復召見問籌策。宣懷言:「既借款,不應令商造;既商造,不應再借款。民情可用,不順用之恐激變。」上是之,拜郵傳部右侍郎。命甫下,而浙路總理湯壽潛因言宣懷短,請離路事。壽潛獲嚴譴,宣懷亦不復久居中,仍命詣滬辦商約。

宣統改元,奏言推廣中央銀行,先齊幣制,附陳辦法成式。逾歲,命充紅十字會會長。先是日俄戰爭,宣懷與呂海寰等謀加入瑞士總會,中國有紅十字會自此始。既拜命入都,時朝廷方整麗幣制,遂敕還郵部本官,參與度支部幣制事。晉尚書,數上封事,凡收回郵政,接筦驛站,規畫官建各路,展拓川藏電線,釐定全國軌制,稱新政畢舉,而以鐵路收為國有,致召大變,世皆責之。

先是給事中石長信疏論各省商民集股造路公司弊害,宜敕部臣將全國幹路定為國有,其餘枝路仍準各省紳商集股自修。諭交部議,宣懷復奏言:「中國幅員廣袤,邊疆遼遠,必有縱橫四境諸大幹路,方足以利行政而握中樞。從前規畫未善,致路政錯亂紛歧,不分枝幹,不量民力,一紙呈請,輒準商辦。乃數載以來,粵則收股及半,造路無多;川則倒帳甚鉅,參追無著;湘、鄂則開局多年,徒供坐耗。循是不已,恐曠日彌久,民累愈深,上下交受其害。應請定幹路均歸國有,枝路任民自為,曉諭人民,宣統三年以前各省分設公司集股商辦之幹路,應即由國家收回,亟圖修築,悉廢以前批準之案,川、湘兩省租股並停罷之。」於是有鐵路國有之詔,並起端方充督辦粵漢、川漢鐵路大臣。

宣懷復與英、德、法、美四國結借款之約,各省聞之,群情疑懼,湘省首起抗阻,川省繼之。湘撫楊文鼎、川督王人文先後以聞,詔切責之,諭:「嚴行禁止,儻有匪徒從中煽惑,意在作亂者,照懲治亂黨例,格殺勿論。」宣懷又會度支部奏收回辦法:「請收回粵、川、湘、鄂四省公司股票,由部特出國家鐵路股票換給,粵路發六成,湘、鄂路照本發還,川路宜昌實用工料之款四百餘萬,給國家保利股票。其現存七百餘萬兩,或仍入股,或興實業,悉聽其便。」詔飭行。四川紳民羅綸等二千四百餘人,以收路國有,盛宣懷、端方會度支部奏定辦法,對待川民,純用威力,未為持平,不敢從命。人文復以聞,再切責之。趙爾豐等復奏:「川民爭路激烈,請仍歸商辦。」不許,川亂遂成,而鄂變亦起,大勢不可問矣。資政院以宣懷侵權違法,罔上欺君,塗附政策,釀成禍亂,實為誤國首惡,請罪之,詔奪職,遂歸。後五年,卒。

宣懷有智略,尤善治賑。自咸豐季葉畿輔被水菑,嗣是而晉邊,而淮、徐、海,而浙,而鄂,而江、皖,皆起募款,籌賑撫。因討測受菑之故,益究心水利,其治小清河利尤溥。唯起家實業,善蓄藏,稱富,亦往往冒利,被口語云。

瑞澂,字莘儒,滿洲正黃旗人,大學士琦善孫,將軍恭鏜子。以貢生官刑部筆帖式,遷主事,調升戶部員外郎。出為九江道,有治聲,移上海道。滬地交涉繁,瑞澂應付縝密,頗負持正名。尤顓意警政,建總局,廓分區,設學堂,練馬巡,中外交誦其能。光緒三十三年,授江西按察使,遷江蘇布政使。時江、浙梟匪蠢動,出沒滬、杭孔道,釀成巨案。侍郎沈家本建議辦清鄉,朝命瑞澂主蘇、松、太、杭、嘉、湖捕務,六屬文武受節度。瑞澂添募水師,購置兵輪,仿各國海軍制,編成聯隊。擒獲巨魁夏竹、林聲為,匪徒斂跡。

宣統改元,稱疾,乞解職,溫旨慰留。總督端方密薦其才,遷巡撫。既蒞事,澄吏治,肅軍紀,嚴警政,條具整飭本末以上,上嘉納,命署湖廣總督。逾歲,到官,旋實授。劾罷巡警道馮啟鈞、勸業道鄒履和。湘民饑變,復糾彈前祭酒王先謙、主事葉德輝、道員孔憲穀阻撓新政狀,中旨分別懲革,繇是威望益著。其時朝廷籌備立憲,瑞澂希風指,凡置警、興學、設諮議局、立審檢,一切皆治辦。名流如張謇輩咸與交驩,而懿親載澤方用事,則又為其姻婭,聲勢駸駸出南北洋上。

三年七月,被命會辦川漢、粵漢鐵路。居無何,督辦端方上言鄂境鐵路收歸國有,詔嘉之。越月,武昌變起。先是黨人謀亂於武昌,瑞澂初聞報,憂懼失措,漫不為備,惟懸賞告密,得黨人名冊,多列軍人名,左右察知偽造,請銷毀以安眾心。瑞澂必欲按名捕之,獲三十二人,誅其三,輒以平亂聞。詔嘉其弭患初萌,定亂俄頃,命就擒獲諸人嚴鞫,並緝逃亡,於是軍心騷動,翌日遂變。瑞澂棄城走,詔革職,仍令權總督事,戴罪圖功,並令陸軍大臣廕昌督師往討,薩鎮冰率兵艦、程允和率水師援之,而瑞澂已乘兵艦由漢口而蕪湖而九江,且至上海矣。

黨軍推陸軍第二十一混成協統領官黎元洪稱都督,置軍政府。既占武昌,復取漢陽,據漢口,乃起袁世凱為湖廣總督,督辦剿撫,節制長江水陸各軍,副都統王士珍副之。召廕昌還,命軍諮使馮國璋總統第一軍,江北提督段祺瑞總統第二軍,俱受世凱節制。國璋與黨軍戰於灄口,水陸夾擊,復漢口,連克漢陽,指日下武昌,而世凱授總理內閣大臣,遽令停攻。復起魏光燾督湖廣,士珍暫權,段芝貴護,又命祺瑞攝之。時瑞澂已久遁上海,始以失守武昌,潛逃出省,偷生喪恥,詔逮京,下法部治罪,而瑞澂不顧也。瑞澂居上海四年,病卒。

論曰:辛亥革命,亂機久伏,特以鐵路國有為發端耳。宣懷實創斯議,遂為首惡。鄂變猝起,瑞澂遽棄城走,當國優柔,不能明正以法。各省督撫遂先後皆不顧,走者走,變者變,大勢乃不可問矣。嗚呼!如瑞澂者,謚以罪首,尚何辭哉?


\end{pinyinscope}