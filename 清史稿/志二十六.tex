\article{志二十六}

\begin{pinyinscope}
時憲七

△雍正癸卯元法下

月食用數

朔策二十九日五三0五九0五三。

望策一十四日七六五二九五二六五。

太陰交周朔策一十一萬零四百一十三秒,小餘九二四四一三三四。

太陰交周望策六宮一十五度二十分零六秒五十八微。

中距太陰地半徑差五十七分三十秒。

太陽最大地半徑差一十秒。

中距太陽距地心一千萬。

中距太陰距地心一千萬。

中距太陽視半徑一十六分六秒。

中距太陰視半徑一十五分四十秒三十微。

朔應一十五日一二六三三。

首朔太陰交周應六宮二十三度三十六分五十二秒四十九微。餘見日躔、月離。

推月食法

求天正冬至,

求紀日,

求首朔,

求太陰入食限,並同甲子元法。視某月太陰平交周入可食之限,即為有食之月。交周自五宮十四度五十一分至六宮十五度九分,自十一宮十四度五十一分至初宮十五度九分,皆可食之限。再於實時距正交詳之。

求平望,同甲子元法。

求實望實時,先求泛時,用兩日實行較,同甲子元求朔望法。次設前、後兩時,各求日、月黃道實行。復用兩時實行較,得實望實時。又以實時各求日、月黃道實行,視本時月距正交入限為有食。自五宮十七度四十三分至六宮十二度十七分,自十一宮十七度四十三分至初宮十二度十七分,皆有食之限。

求實望用時,用實時太陽均數及升度求法,同甲子元法。比視日出入亦同。

求食甚時刻,用平三角形,以一小時太陰白道實行化秒為一邊,本時次時二實行較。一小時太陽黃道實行化秒為一邊,實望黃白大距為所夾之角,求得對小邊之角為斜距交角差。以加實時黃白大距,為斜距黃道交角。又以斜距交角差之正弦為一率,一小時太陽實行為二率,實望黃白大距之正弦為三率,求得四率,為一小時兩經斜距。又以半徑千萬為一率,斜距黃道交角之餘弦、正弦各為二率,實望月離黃道實緯為三率,各求得四率,為食甚實緯南北與實望黃道實緯同。及距弧。又以一小時兩經斜距為一率,一小時化秒為二率,食甚距弧為三率,求得四率為食甚距時。以加減實望用時,月距正交初宮、六宮為減,五宮、十一宮為加。得食甚時刻。

求太陽太陰實引,置實望太陽引數,加減本時太陽均數,得太陽實引。又置實望太陰引數,加減本時太陰初均數,得太陰實引。

求太陽太陰距地,用平三角形,以日躔倍兩心差為對正角之邊,以太陽實引為又一角,三宮內用本度,過三宮與六宮相減,過九宮與全周相減,用其餘。求得對太陽實引之邊為勾。又求得對原不知角之邊為分股,與二千萬相加減,實引三宮內九宮外加,三宮外九宮內減。為股弦和與勾,求得股。與分股相加減,實引三宮內九宮外減,三宮外九宮內加。得太陽距地。又以實望月離倍兩心差如法求之,得太陰距地。

求實影半徑,以太陰距地為一率,中距太陰距地為二率,中距太陰最大地半徑差為三率,求得四率為本時太陰最大地半徑差。又以六十九除之,為影差。又以太陽距地為一率,中距太陽距地為二率,中距太陽視半徑為三率,求得四率為太陽視半徑,與本時太陰最大地半徑差相減。又加太陽最大地半徑差,為影半徑,又加影差,為實影半徑。

求太陰視半徑,以太陰距地為一率,中距太陰距地為二率,中距太陰視半徑為三率,求得四率,為太陰視半徑。

求食分,以太陰全徑為一率,十分化作六百秒為二率,並徑實影視太陰兩半徑並。內減食甚實緯,餘化秒為三率,求得四率為秒,以分收之,即食分。

求初虧、復圓時刻,以並徑與食甚實緯相加化秒為首率,相減化秒為末率,求得中率為秒,以分收之,為初虧、復圓距弧。又以一小時兩經斜距為一率,一小時化秒為二率,初虧、復圓距弧為三率,求得四率為初虧、復圓距時,以加減食甚時刻,得初虧、復圓時刻。減得初虧,加得復圓。

求食既、生光時刻,以兩徑較實影視太陰兩半徑相減之餘。與食甚實緯相加化秒為首率,相減化秒為末率,求得中率為秒,以分收之,為食既、生光距弧。求距時時刻,與初虧、復圓法同。食在十分以內,則無此二限。

求食限總時,同甲子元法。

求食甚太陰黃道經緯宿度,以一小時化秒為一率,一小時太陰白道實行為二率,食甚距時化秒為三率,求得四率,為距時月實行。以加減實望太陰白道實行,加減與食甚距時同。得食甚太陰白道經度。又置實望月距正交,加減距時月實行,得食甚月距正交。再求黃道經緯宿度,同月離。

求食甚太陰赤道經緯宿度,以半徑千萬為一率,食甚太陰距春、秋分黃道經度正弦為二率,食甚太陰黃道經度不及三宮者,與三宮相減;過三宮者,減三宮;過六宮者,與九宮相減;過九宮者,減九宮。食甚太陰黃道緯度餘切為三率,求得四率為餘切,檢表得太陰距二分弧與黃道交角,以加減黃赤大距,食甚太陰黃道經度九宮至三宮,緯南加,緯北減,皆在赤道南,反減則在北。三宮至九宮加減反是。為太陰距二分弧與赤道交角。又以太陰距二分弧與黃道交角之餘弦為一率,半徑千萬為二率,食甚太陰距春、秋分黃道經度之正切為三率,求得四率,為太陰距二分弧之正切。又以半徑千萬為一率,太陰距二分弧與赤道交角之餘弦為二率,太陰距二分弧正切為三率,求得四率為正切,檢表為距春、秋分赤道經度。加減三宮九宮,食甚太陰黃道經度不及三宮,與三宮相減,過三宮者加三宮。過六宮者,與九宮相減,過九宮者加九宮。得食甚太陰赤道經度。求緯度宿度,同甲子元法。

求初虧、復圓黃道高弧交角,以半徑千萬為一率,黃赤大距正弦為二率,影距春、秋分黃道經度正弦為三率,求得四率為正弦,檢表得影距赤道度。影距春、秋分度數與太陽同,太陽在赤道北,影在南,太陽在赤道南,影在北。又以影距春、秋分黃道經度餘弦為一率,黃赤大距餘切為二率,半徑千萬為三率,求得四率為正切,檢表為黃道赤經交角。乃用弧三角形,以北極距天頂為一邊,影距赤道與九十度相加減為一邊,北則減,南則加。初虧、復圓各子正時刻過十二時者,與二十四時相減。變赤道度,各為所夾之角,求得對北極距天頂之角。各為赤經高弧交角,以加減黃道赤經交角,太陰在夏至前六宮,食在子正後則減,為限西。食在子正前則加,加過九十度,與半周相減,為限東。不及九十度,則不與半周相減,變為限西。在夏至後六宮反是。各得黃道高弧交角。若食在子正,影在正午,無赤經高弧交角,則黃道赤經交角即黃道高弧交角。太陰在夏至前為限西,後為限東。

求初虧、復圓並徑高弧交角,以並徑為一率,食甚實緯為二率,半徑千萬為三率,求得四率為餘弦,檢表為並徑交實緯角。如無食甚實緯,即無此角,亦無並徑黃道交角。又置九十度,加減斜距黃道交角,得初虧、復圓黃道交實緯角。食甚月距正交初宮、六宮,初虧減,復圓加。五宮、十一宮,初虧加,復圓減。各與並徑交實緯角相減,為初虧、復圓並徑黃道交角。並徑初交實緯角小,距緯南北與食甚同。大則反是。以加減黃道高弧交角,虧限東,復圓限西,緯南加,緯北減。初虧限西,復圓限東,加減反是。各得並徑高弧交角。如無並徑黃道交角,則黃道高弧交角即並徑高弧交角。

求初虧、復圓方位,即以並徑高弧交角為定交角,求法同甲子元。但以並徑高弧交角初度初虧在限東為正下,限西為正上;復圓在限東為正上,限西為正下。據京師北極高度定,與甲子元法同。

求帶食分秒,用兩經斜距,不用月距日實行,餘與甲子元法同。

求帶食方位,用帶食兩心相距,不用並徑求諸交角,如初虧、復圓定方位。食甚前與初虧同,食甚後與復圓同。

求各省月食時刻方位,理同甲子元法。

繪月食圖,同甲子元法。

日食用數

太陽光分一十五秒,餘見日躔、月離、月食。

推日食法

求天正冬至,

求紀日,

求首朔,

求太陰入食限,並同月食,惟不用望策,即為逐月朔太陰交周。視某月入可食之限,即為有食之月。交周自五宮八度四十二分至六宮九度一十四分,又自十一宮二十度四十六分至初宮二十一度一十八分,皆可食之限。

求平朔,

求實朔實時,並同月食求望法,惟不加望策。視本時月距正交入食限為有食。自五宮十一度三十四分至六宮六度二十二分,又自十一宮二十三度三十八分至初宮十八度二十六分,為有食之限。

求實朔用時,與月食求實望用時同。比視日出入,同甲子元法。

求食甚用時,與月食求食甚時刻法同。

求太陽太陰實引,

求太陽太陰距地,並同月食。

求地平高下差,先求本日太陰最大地半徑差,法同月食。乃減太陽最大地半徑差,得地平高下差。

求太陽實半徑,先求太陽視半徑,法同月食。內減太陽光分,得太陽實半徑。

求太陰視半徑,法同月食。

求食甚太陽黃道經度宿度,求經度與月食求太陰白道法同;求宿度同日躔。

求食甚太陰赤道經緯宿度,用黃赤大距,法同月食求太陰黃道。

求黃赤及黃白、赤白二經交角,以食甚太陽距春、秋分黃道經度餘弦為一率,黃赤大距餘切為二率,半徑千萬為三率,求得四率為餘切,檢表得黃赤二經交角。冬至後黃經在赤經西,夏至後在赤經東,如太陽在二至,則無此角。又以前所得斜距黃道交角,即為黃白二經交角。實朔月距正交初宮、十一宮,白經在黃經西;五宮、六宮,在黃經東。二交角相加減,為赤白二經交角。二交角同為東同為西者相加,白經在赤經之東西仍之。一為東一為西者相減。東西從大角。如減盡,則無此角。如無黃赤二經交角,則黃白即赤白,東西並同。

求用時太陽距午赤道度,以食甚用時與十二時相減,餘數變赤道度,得用時太陽距午赤道度。

求用時赤經高弧交角,用弧三角形,以北極距天頂為一邊,太陽距北極為一邊,赤緯在南,加九十度;在北,與九十度相減。用時太陽距午赤道度為所夾之角,求得對北極距天頂之角,為用時赤經高弧交角。午前赤經在高弧東,午後赤經在高弧西。若太陽在正午,則無此角。

求用時太陽距天頂,以用時赤經高弧交角正弦為一率,北極距天頂之正弦為二率,用時太陽距午赤道度之正弦為三率,求得四率為正弦,檢表得太陽距天頂。

求用時高下差,以半徑千萬為一率,地平高下差化秒為二率,用時太陽距天頂之正弦為三率,求得四率為秒,以分收之,為用時高下差。

求用時白經高弧交角,以用時赤經高弧交角與赤白二經交角相加減,得用時白經高弧交角。東西同者相加,白經在高弧之東西仍之。一東一西者相減,東西從大角。如無赤白二經交角,或無赤經高弧交角,則即以所有一角命之,東西並同。如二角俱無,或同度減盡,則無此角。食甚用時即真時。用時高下差與食甚實緯,南加北減,即食甚兩心視相距。

求用時對兩心視相距角,月在黃道北,取用時白經高弧交角;月在黃道南,取用時白經高弧交角之外角,實距在高弧之東西,月在北則與白經同,在南則相反。皆為用時對兩心視相距角。若自經高弧交角過九十度,緯南如緯北,緯北如緯南。

求用時對兩心實相距角,用平三角形,以食甚用時兩心實相距為一邊,即食甚實緯。用時高下差為一邊,用時對兩心視相距角為所夾之角,即求得用時對兩心實相距角。

求用時兩心視相距,以用時對兩心實相距角之正弦為一率,用時兩心實相距為二率,用時對兩心視相距角之正弦為三率,求得四率,即用時兩心視相距。白經在高弧西,兩心視相距大於並徑者,或無食或未及等者,用時即初虧真時,在高弧東為已過及復圓真時。若小於並徑,高弧西為初虧食甚之間,東為復圓食甚之間。

求食甚設時,用時白經高弧交角東向前取,西向後取,角大遠取,角小近取,遠不過九刻,近或數分。量距用時前後若干分,為食甚設時。

求設時距分,以食甚設時與食甚用時相減,得設時距分。

求設時距弧,以一小時化秒為一率,一小時兩經斜距為二率,設時距分化秒為三率,求得四率,為設時距弧。

求設時對距弧角,以食甚實緯為一率,設時距弧為二率,半徑千萬為三率,求得四率為正切,檢表得設時對距弧角。

求設時兩心實相距,以設時對距弧角之正弦為一率,設時距弧為二率,半徑千萬為三率,求得四率,即設時兩心實相距。

求設時太陽距午赤道度,

求設時赤經高弧交角,

求設時太陽距天頂,

求設時高下差,

求設時白經高弧交角,以上五條,皆與用時同,但皆用設時度分立算。

求設時對兩心視相距角,月在黃道北,以設時白經高弧交角與設時對距弧角相減,月在黃道南則相加,又與半周相減,餘為設時對兩心視相距角。相減者,對距弧角小,實距在高弧之東西與白經同;對距弧角大則相反。相加又減半周者,實距在高弧之東西,恆與白經反。如兩角相等而減盡無餘,或相加適足一百八十度,則無交角,亦無對設時兩心實相距角,即以設時高下差與設時兩心實相距相減,餘為設時兩心視相距。若白經高弧交角過九十度,緯南如緯北,緯北如緯南。

求設時對兩心實相距角,

求設時兩心視相距,皆與用時同。

求設時白經高弧交角較,以設時白經高弧交角與用時白經高弧交角相減,即得。

求設時高弧交用時視距角,以設時白經高弧交角較與用時對兩心實相距角相加減,即得。緯北為減,緯南為加。若白經高弧交角過九十度,反是。

求對設時視行角,以設時高弧交用時視距角與設時對兩心實相距角相加減,即得。兩實距同在高弧東,或同在西,則減;一東一西者,則加;加過半周者,與全周相減,用其餘。如無設時對兩心實相距角,設時高下差大於設時兩心實相距,則設時高弧交用時視距角即對設時視行角;設時高下差小於設時兩心實相距,則以設時高弧交用時視距角與半周相減,餘為對設時視行角。

求對設時視距角,用平三角形,以用時兩心視相距為一邊,設時兩心視相距為一邊,對設時視行角為所夾之角,即求得對設時視距角。

求設時視行,以對設時視距角之正弦為一率,設時兩心視相距為二率,對設時視行角正弦為三率,求得四率,為設時視行。

求真時視行,以半徑千萬為一率,對設時視距角餘弦為二率,用時兩心視相距為三率,求得四率,為真時視行。

求真時兩心視相距。以半徑千萬為一率,對設時視距角正弦為二率,用時兩心視相距為三率,求得四率,為真時兩心視相距。

求食甚真時,以設時視行為一率,設時距分為二率,真時視行為三率,求得四率,為真時距分,以加減食甚用時,白經在高弧西則加,在高弧東則減。得食甚真時。

求真時距弧,

求真時對距弧角,

求真時兩心實相距,以上三條,法與設時同,但皆用真時度分立算。

求真時太陽距午赤道度,

求真時赤經高弧交角,

求真時太陽距天頂,

求真時高下差,

求真時白經高弧交角,

求真時對兩心視相距角,

求真時對兩心實相距角,

求考真時兩心視相距,以上八條,法與用時同,但皆用真時度分立算。

求真時白經高弧交角較,法同設時,但用真時度分立算。

求真時高弧交設時視距角,法同設時,加減有異。月在黃道北,設時真時兩實距在高弧東西同,惟白經異。設時白經高弧交角小則加,大則減。若白經亦同,反是。若兩實距一東一西,則皆相減。月在黃道南,設時交角小則加,大則減。如無設時對兩心實相距角,設時高下差大於設時兩心實相距,則真時白經高弧交角較,即真時高弧交設時視距角;設時高下差小於設時兩心實相距,則以真時白經高弧交角較與半周相減,餘為真時高弧交設時視距角。若白經高弧交角過九十度,緯南如緯北,緯北如緯南。

求對考真時視行角,法同設時。如設時實距與高弧合,無東西者,設時高下差大於設時兩心實相距,則相減,小則加。如真時白經高弧交角較與設時對兩心實相距角相等,而減盡無餘,則真時對兩心實相距角,即對考真時視行角。或相加適足半周,則真時對兩心實相距角與半周相減,即對考真時視行角。

求對考真時視距角,

求考真時視行,以上二條,法同設時,但用考真時度分立算。

求定真時視行,如定真時視行與考真時視行等,則食甚真時即為定真時。如或大或小,再用下法求之。

求定真時兩心視相距,以上二條,法同真時,用考真時度分立算。

求食甚定真時,以考真時視行為一率,設時距分與真時距分相減餘為二率,定真時視行為三率,求得四率,為定真時距分。以加減食甚設時,白經在高弧東,設時距分小測減,大則加。白經在高弧西,反是。得食甚定真時。

求食分,以太陽實半徑倍之為一率,十分為二率,並徑內減定真時兩心視相距餘為三率,求得四率,即食分。

求初虧、復圓前設時,白經在高弧西,食甚用時兩心視相距與並徑相去不遠,即以食甚用時為初虧前設時,小則向前取,大則向後取,量距食甚用時前後若干分,為初虧前設時。與食甚定真時相減,餘數與食甚定真時相加,為復圓前設時,白經在高弧東,先取復圓,後得初虧,理並同。

求初虧前設時距分,

求初虧前設時距弧,

求初虧前設時對距弧角,初虧前設時在食甚用時前為西,在食甚用時後為東。

求初虧前設時兩心實相距,以上四條,法同食甚設時,但用初虧前設時度分立算。

求初虧前設時太陽距午赤道度,

求初虧前設時赤經高弧交角,

求初虧前設時太陽距天頂,

求初虧前設時高下差,

求初虧前設時白經高弧交角,以上五條,法同食甚用時。

求初虧前設時對兩心視相距角,法同食甚用時,加減有異,月在黃道北,二角東西同,則相加;一東一西,相減。月在黃道南,反是。又與半周相減。若白經高弧交角過九十度,則緯南、緯北互異。餘同食甚設時。

求初虧前設時對兩心實相距角,

求初虧前設時兩心視相距,以上二條,法同食甚用時,但用初虧前設時度分立算。

求初虧後設時,視初虧前設時兩心視相距小於並徑,則向前取,大則向後取,察其較之多寡,量取前後若干分,為初虧後設時。以下逐條推算,皆與前設時同,但用後設時度分立算。

求初虧視距較,以前後設時兩心視相距相減,即得。

求初虧設時較,以前後設時距分相減,即得。

求初虧視距並徑較,以初虧後設時兩心視相距與並徑相減,即得。

求初虧定真時,以初虧視距較為一率,初虧設時較為二率,初虧視距並徑較為三率,求得四率,為初虧真時距分。以加減初虧後設時,後設時兩心視相距大於並徑為加,小為減。得初虧真時。乃以初虧真時依前法求其兩心視相距,果與並徑等,則初虧真時即初虧定真時。初虧真時對兩心實相距角即初虧方位角。如或大或小,則以初虧前後設時兩心視相距與並徑尤近者,與考真時兩心視相距相較,依法比例,得初虧定真時。

求復圓前設時諸條,法同初虧,但用復圓前設時度分立算。

求復圓後設時,視復圓前設時兩心視相距小於並徑,則向後取,大於並徑,則向前取,察其較之多寡,量取前後若干分,為復圓後設時。逐條推算,皆與前設時同,但用後設時度分立算。

求復圓視距較,

求復圓設時較,

求復圓視距並徑較,

求復圓定真時,以上四條,皆與初虧法同,但用復圓度分立算。

求食限總時,置初虧定真時,減復圓定真時,即得。

求初虧、復圓定交角,初虧白經在高弧之東,以初虧方位角與半周相減,在高弧之西,即用初虧方位角;復圓反是:皆為定交角。

求初虧、復圓方位,法與甲子元同,但以定交角初度初虧白經在高弧東為正上,在西為正下;復圓在東為正下,在西為正上。

求帶食用日出入分,同甲子元法。

求帶食距時,以日出入分與食甚用時相減,即得。

求帶食距弧,法同食甚設時,但用帶食距時立算。

求帶食赤經高弧交角,以黃赤距緯之餘弦為一率,北極高度之正弦為二率,半徑千萬為三率,求得四率為餘弦,檢表得帶食赤經高弧交角。

求帶食白經高弧交角,法與食甚用時同,但用帶食度分立算。

求帶食對距弧角,

求帶食兩心實相距,

求帶食對兩心視相距角,以上三條,法與食甚設時同,但用帶食度分立算。

求帶食對兩心實相距角,用地平高下差,餘法同食甚用時。

求帶食兩心視相距,法同食甚用時,但用帶食度分立算。

求帶食分秒,與求食分同,用帶食相距立算。

求帶食方位,在食甚前者,用初虧法;在食甚後者,用復圓法。

求各省日食時刻方位,理同甲子元法。

繪日食圖,同甲子元法。

繪日食坤輿圖,取見食極多之分,每分為一限。止於二十一限。又取見食時刻早晚,每刻為一限。止於九十六限。交錯相求,反推得見食各地北極高下度、東西偏度。乃按度聯為一圖。又按坤輿全圖所當高度偏度各地名,遂一填註。

相距用數,見月離及五星、恆星行。

推相距法,同甲子元推凌犯法。

推步用表

甲子元及癸卯元二法,除本法外,皆有用表推算之法,約其大旨著於篇。

甲子元法:

一曰年根表,以紀年、紀日、值宿為綱,由法元之年順推三百年,各得其年天正冬至次日子正太陽及最卑平行,列為太陽年根表;太陰及最高、正交平行,列為太陰年根表;五星及最高、正交、伏見諸平行,為各星年根表。

一曰周歲平行表,以日數為綱,由一日至三百六十六日,積累日、月、五星及最卑、最高、正交、伏見諸平行,各列為周歲平行表。

一曰周日平行表,以時分秒為綱,與度分秒對列三層,自一至六十,積累日、月、五星及最高、正交、伏見、月距日、太陰引數、交周諸平行,各列為周日平行表。

一曰均數表,以引數為綱,豫推得逐度逐分盈縮遲疾,備列於表。太陰別有二三均數表,以引數及月距日為綱,縱橫對列,推得二三均數,備列於表。土、木、金、水四星,則以初均及中分、次均及較分,同列為一表。火星則以初均及次輪心距地數、次輪半徑本數、太陽高卑差數,同列為一表。皆為均數表。

一曰距度表,以黃道宮度為綱,列所對赤道南北距緯,為黃赤距度表。以月距正交為綱,分黃白大距為六限,列所對黃道南北距緯,為黃白距度表。

一曰升度表,以黃道宮度為綱,列所對赤道度,為黃赤升度表。

一曰黃道赤經交角表,以黃道宮度為綱,取所對黃道赤經交角列於表。

一曰升度差表,以月、五星距交宮度為綱,各列所當黃道度之較,各為升度差表。

一曰時差表,以黃道為綱,取所當赤道度之較變時,列為升度時差表。又以引數為綱,取所當均數變時,列為均數時差表。

一曰地半徑差表,以實高度為綱,取所當太陽、太陰及火、金、水三星諸地半徑差,各列為表。

一曰清蒙氣差表,以實高度為綱,取所當清蒙氣差,列為表。

一曰實行表,以引數為綱,取所當太陽、太陰及月距日實行,各列為表。

一曰交均距限表,以月距日為綱,取所當之交均及距限,同列為一表。

一曰首朔諸根表,以紀年、紀日、值宿為綱,由法元之年順推三百年,取所當之首朔日時分秒及太陽平行,太陽、太陰引數,太陰交周,五者同列為一表。

一曰朔望策表,以月數為綱,自一至十三,取所當之朔、望策及太陽平行朔、望策,太陽、太陰引數朔、望策,太陰交周朔、望策,十事同列為一表。

一曰視半徑表,以引數為綱,取所當之日半徑、月半徑、月距地影半徑、影差,五者同列為一表。

一曰交食月行表,以食甚距緯分為綱,自初分至六十四分,與太陽、太陰、地影,凡兩半徑之和分,自二十五分至六十四分,縱橫對列,取所當之月行分秒列為表。其太陰、地影兩半徑之較分與和分同用。

一曰黃平象限表,以正午黃道宮度為綱,分北極高自十六度至四十六度為三十一限,取所當之春分距午、黃平象限、限距地高,三者同列為一表。

一曰黃道高弧交角表、以日距限為綱,自初度至九十度,分限距地高自二十度至八十九度為七十限,取所當之黃道高弧交角列為表。

一曰太陽高弧表,列法與黃道高弧交角表同。

一曰東西南北差表,以交角度為綱,自初度至九十度,與高下差一分至六十三分,縱橫對列,取所當之東西差及南北差,同列為表。

一曰緯差角表,以並徑為綱,自三十一分至六十四分,與距緯一分至六十四分,縱橫對列,取所當之緯差角列為表。

一曰星距黃道表,以距交宮度為綱,取所當星距黃道數各列為表,水星獨分交角自四度五十五分三十二秒至六度三十一分二秒為二十限。

一曰星距地表,以星距日宮度為綱,取所當之星距地列於表。

一曰水星距限表,以距交宮度為綱,取所當之距限列為表。

一曰五星伏見距日黃道度表,以星行黃道經表為綱,分晨夕上下列之,取各星所當距日黃道度,同列為一表。

一曰五星伏見距日加減差表,列法同黃道度表,但不分五星,別黃道南北自一度至八度。

癸卯元法所增:

一曰太陽距地心表,以太陽實引為綱,取所對之太陽距地心真數對數,並列於表。

一曰太陰一平均表,以太陽引數為綱,取所當之太陰一平均、最高平均、正交平均,並列於表。

一曰太陰二平均表,以日距月最高宮度為綱,取所當太陽在最高之二平均及高卑較秒,並列於表。

一曰太陰三平均表,以月距正交宮度為綱,取所當之三平均列為表。

一曰太陰最高均及本天心距地表,以日距月天最高宮度為綱,取所當最高均及本天心距地數,並列於表。

一曰太陰二均表,以月距日宮度為綱,取所當太陽在最高時二均及高卑較數,並列於表。

一曰太陰三均表,以相距總數為綱,取所對之三均列於表。

一曰太陰末均表,以實月距日宮度為綱,與日月最高相距,縱橫對列,取所當之末均列為表。

一曰太陰正交實均表,以日距正交宮度為綱,取所對之正交實均列為表。

一曰交角加分表,以日距正交宮度為綱,取所當之距交加分加差,並列於表。

一曰黃白距緯表,列法與升度差表同。

一曰太陰距地心表,以太陰實引為綱,取所當最大、最小兩心差各太陰距地心數及倍分,並列於表。其名同而實異者,太陰初均表分大、中、小三限,黃、白升度差表列最小交角及大、小較秒,太陰地半徑差表、太陰實行表俱分大、小二限。


\end{pinyinscope}