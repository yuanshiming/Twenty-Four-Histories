\article{志三}

\begin{pinyinscope}
天文三

日月五星恆星黃赤道十二次值宿昏旦中星

日月五星自古言天之精者,知日月五星為渾象而已。近代西人制大遠鏡,測得諸曜形體及附近小星暈氣各種,古今不同,就其著者錄焉。

日之面有小黑形,常運行二十八日滿一周。月之面以日光正照顯明景,偏照顯黑景。其面有凹凸,故雖全明之中,亦有淡黑雜景。

土星之體,徬彿卵形,舊測謂旁有兩耳,今測近於赤道星面相逼甚窄,於遠赤道所宕甚寬。旁有排定小星五點,最近第一星,約行二日弱;第二星行三日弱;第三星行四日半強;第四星略大,行十六日;第五星行八十日。俱旋行土星一周。

木星之面,常有平行暗景,外有小星四點。第一星行一日七十三刻;第二星行三日五十三刻;第三星略大,行七日十六刻;第四星行十六日七十二刻。俱旋行木星一周。

火星之面,內有無定黑景。

金、水星俱借日為光,合朔弦望如月。

恆星歷象考成云:「恆星之名,見於春秋,而四仲中星及斗、牽牛、織女、參、昴、箕、畢、大火、農祥、龍尾、鳥帑、元駟、元黿之屬,散見於尚書、易、詩、左傳、國語。至周禮春官馮相氏掌二十八星之位,而禮記月令、大戴禮夏小正稍具諸星見伏之節。蓋古者敬天勤民,因時出政,皆以星為紀。秦炬之後,羲和舊術,無復可稽,其傳者惟史記天官書,而所載簡略。後漢張衡云:『中外之官,常明者百有二十四,可名者三百二十,為星二千五百』,而其書不傳。至三國時,太史令陳卓始列巫咸、甘、石三家所著星圖,總二百八十三官,一千四百六十四星。隋丹元子作步天歌,敘三垣二十八宿,共一千四百六十七星,為觀象之津梁,然尚未有各星經緯度數。自唐、宋而後,諸家以儀象考測,始有各星入宿去極度分,視古加密。

「新法算書恆星圖表,共星一千二百六十六,分為六等:第一等星一十七,第二等星五十七,第三等星一百八十五,第四等星三百八十九,第五等星三百二十三,第六等星二百九十五,外無名不入等者四百五十九。康熙壬子年欽天監新修儀象志,恆星亦分六等,而其數微異。第一等星一十六,第二等星六十八,第三等星二百零八,第四等星五百一十二,第五等星三百四十二,第六等星七百三十二,共計一千八百七十八。蓋觀星者以目之所能辨,因其相近,聯綴成象而命之名。其微茫昏暗者,多不可考。故各家星官之數,多少不能畫一。然列宿及諸大星,則古今中西如一轍也。」

又云:「恆星行即古歲差也,古法俱謂恆星不動,而黃道西移;今謂黃道不動,而恆星東行。蓋使恆星不動而黃道西移,則恆星之黃道經緯度宜每歲不同,而赤道經緯度宜終古不變。今測恆星之黃道經度,每歲東行,而緯度不變。至於赤道經度,則逐歲不同,而緯度尤甚。自星紀至鶉首六宮之星,在赤道南者,緯度古多而今漸少,在赤道北者反是。自鶉首至星紀六宮之星,在赤道南者,緯度古少而今漸多,在赤道北者反是。凡距赤道二十三度半以內之星,在赤道北者,可以過赤道南,在赤道南者,亦可以過赤道北,則恆星循黃道東行,而非黃道之西移明矣。新法算書載西人第穀以前,或云恆星百年而東行一度,或云七十餘年而東行一度,或云六十餘年而東行一度,隨時修改,訖無定數,與古人屢改歲差相同。迨至第谷,方定恆星每歲東行五十一秒,約七十年有餘而行一度,而元郭守敬所定歲差之數亦為近之。至今一百四十餘年,驗之於天,雖無差忒,但星行微渺,必歷多年,其差乃見。然則第穀所定之數,亦未可泥為定率,惟隨時測驗,依天行以推其數可也。」

儀象考成云:「康熙十三年,監臣南懷仁修儀象志,星名與古同者,總二百五十九座,一千一百二十九星,比步天歌少二十四座,三百三十五星。又於有名常數之外,增五百九十七星。又多近南極星二十三座,一百五十星。近年以來,累加測驗,星官度數,儀象志尚多未合。又星之次第多不序順,亦宜釐正。於是逐星測量,推其度數,觀其形象,序其次第,著之於圖。計三垣二十八宿,星名與古同者,總二百七十七座,一千三百一十九星,比儀象志多十八座,一百九十星,與步天歌為近。其尤與古合者,二十八宿次舍,自古皆觜宿在前,參宿在後,其以何星作距,古無明文。唐書云:『古以參右肩為距』,失之太遠。文獻通考載宋兩朝天文志云:『觜三星,距西南星;參十星,距中星西一星。』西法,觜宿距中上星,參宿亦距中西一星。今按觜宿中上星在西南星前僅六分餘,而西南星小,中上星大,則以中上星作距可也。若參宿以中西一星作距星,則觜宿之黃道度已在參宿後一度餘,即赤道度亦在參宿後三十一分餘。今依次順序,以參宿中三星之東一星作距星,則觜宿黃道度恆在參前一度弱,與觜前參後之序合。其餘諸座之星,皆以次順序,無凌躐顛倒之弊。又於有名常數之外,增一千六百一十四星。近某座者即名某座增星,依次分註方位,以備稽考。其近南極星二十三座,一百五十星,中國所不見,仍依西測之舊。共計恆星三百座,三千八十三星。」

黃赤道十二次值宿古者分十二次即節氣,故冬至為醜中,春分為戌中,夏至為未中,秋分為辰中。後人則以中氣,而冬至在星紀之初。古不知列宿循黃道東行,且不見有歲差,即以所在星象名其次,故奎、婁為降婁,房、心、尾為大火,後人悉仍其名,而星象之更則不論。積數千年,將所謂蒼龍、玄武、白虎、硃雀之四象且易其方,然則十二次之名,存古意爾。今以康熙甲子年推定十二次初度所值宿,及乾隆甲子年改定十二次初度所值宿,並紀於左。

康熙甲子年黃道十二次初度值宿:

星紀箕三度一十分;

元枵牽牛初度二十三分;

娵訾危一度;

降婁營室一十度五十七分;

大梁婁初度二十七分;

實沈昴五度一十二分;

鶉首觜觿一十度三十八分;

鶉火東井二十九度零五分;

鶉尾七星七度零四分;

壽星翼一十度三十七分;

大火角一十度三十四分;

析木房一度三十九分。

康熙甲子年赤道十二次初度值宿:

星紀箕三度三十九分;

元枵南斗二十三度二十七分;

娵訾危二度三十四分;

降婁東壁初度四十二分;

大梁婁五度四十二分;

實沈昴八度四十分;

鶉首觜觿一十度二十九分;

鶉火東井二十九度;

鶉尾張五度五十七分;

壽星軫初度零二分;

大火亢一度;

析木房五度零三分。

乾隆甲子年黃道十二次初度值宿:

星紀箕二度一十九分一十三秒;

元枵南斗二十三度二十四分一十八秒;

娵訾危初度一十二分四十四秒;

降婁營室一十度五分四十七秒;

大梁奎一十一度八分五十二秒;

實沈昴四度九分三十九秒;

鶉首參八度五十五分一十五秒;

鶉火東井二十八度一十六分五十秒;

鶉尾七星六度一十七分一秒;

壽星翼九度四十八分一十七秒;

大火角九度四十三分三十九秒;

析木房初度三十七分三十五秒。

乾隆甲子年赤道十二次初度值宿:

星紀箕二度四十分一十四秒;

元枵南斗二十二度三十五分四十七秒;

娵訾危一度五十分二十七秒;

降婁營室一十七度零三十八秒;

大梁婁四度五十二分三十三秒;

實沈昴七度三十四分三秒;

鶉首參八度一分五十五秒;

鶉火井二十八度八分一十五秒;

鶉尾張五度一十二分一秒;

壽星翼一十八度八分三十一秒;

大火亢初度一十分三十秒;

析木房四度八分一十七秒。

昏旦中星自虞書紀四仲昏中之星,而月令並舉逐月昏旦。然虞書仲冬星昴,月令則昏中東壁,相去約二千年,中星相差四宿。雖由歲差之故,而古法疏略無度分,固難深論也。今以康熙壬子年所定恆星經緯度,推得雍正元年癸卯各節氣昏旦中星列於志。若求乾隆九年甲子以後各節氣昏旦中星,則當按乾隆甲子年改定恆星經緯度備推焉。

春分系交節初日,後同。昏北河二中偏西四度三十四分。旦尾中偏東一度七分。

因無當中之星,故用近中之星而紀其偏度。又星宿並用第一星,間有第一星距中太遠而用餘星者,則紀其數,如北河二及參四氐四之類。

清明昏七中星偏東五度十四分。旦帝座中偏東一度五十九分。

穀雨昏軒轅十四中偏西四度五十九分。旦箕中偏東四度十三分。

立夏昏五帝座中偏西三十二分。旦箕中偏西四度九分。

小滿昏角中偏東二度二十三分。旦南斗中偏西三度八分。

芒種昏氐中偏東三度二十九分。旦河鼓二中偏東二度二十一分。

夏至昏房中偏東二度八分。旦須女中偏東一度四十三分。

小暑昏尾中偏西四十分。旦尾中偏東三度二十五分。

大暑昏帝座中偏西三度二十五分。旦營室中偏西一度五十六分。

立秋昏箕中偏西二度三十七分。旦土司空中偏東一度四十分。

處暑昏南斗中偏西二十六分。旦婁中偏西一度四十六分。

白露昏南斗中偏西八度三十二分。旦天囷中偏西四度四十一分。

秋分昏河鼓二中偏東三十四分。旦畢中偏西三度七分

寒露昏牽牛中偏西五十三分。旦參四中偏西十三分。

霜降昏須女中偏西三度四十一分。旦天狼中偏西五度三十七分。

立冬昏虛中偏西三度二十分。旦輿鬼中偏東一度二十七分。

小雪昏北落師門中偏東五度四十一分。旦七星中偏西二度十六分。

大雪昏營室中偏西五度五十七分。旦翼中偏東二度五十五分。

冬至昏東壁中偏西四度二十六分。旦五帝座中偏西二度一分。

小寒昏婁中偏東三度三十三分。旦角中偏東六度二十四分。

大寒昏胃中偏西二度二十分。旦亢中偏東四度十八分。

立春昏昴中偏西五度三十四分。旦氐中偏東一度二十八分。

雨水昏參七中偏西四十五分。旦氐四中偏西二度三十二分。

驚蟄昏東井中偏西三度六分。旦房中偏西二度四分。


\end{pinyinscope}