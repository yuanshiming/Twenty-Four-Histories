\article{志二十二}

\begin{pinyinscope}
時憲三

康熙甲子元法上上卷述立法之原,中卷志七政恆星之順軌,下卷志諸曜相距之數。

日躔立法之原:

一,求南北真線以正面位。用方案極平,作圜數層,植表於圜心取日影。識表末影切圜上者,視左右兩點同在一圜聯為直線,即正東西;取東西線正中向圜心作垂線,即正南北。於京師以羅針較之,偏東四度餘。乾隆十七年改為二度三十分。

一,測北極高度以定天體。於冬至前後,用儀器測勾陳大星出地之度,酉時此星在北極之上,候其漸轉而高,至不復高而止。卯時此星在北極之下,候其漸轉而低,至不復低而止。以最高最低之度折中取之,為北極高度。恆星無地半徑差,勾陳距地又高,蒙氣差亦微,其數確準。以此測得申昜春園北極高三十九度五十九分三十秒。

一,求地半徑差以驗地心實高、地面視高之不同。康熙五十四年五月甲子午正,在申昜春園測得太陽高七十三度一十六分零二十三微,同時於廣東廣州府測得太陽高九十度零六分二十一秒四十八微。申昜春園赤道距天頂三十九度五十九分三十秒,廣州府赤道距天頂二十三度十分,偏西三度三十三分。時夏至後八日,日躔最高,用平三角形推得地半徑與太陽距地心比例,如一與一千一百六十二。又康熙五十五年三月丙申午正,在申昜春園測得太陽高五十三度零三分三十八秒一十微,同時於廣東廣州府測得太陽高六十九度五十四分零八秒三十六微。時春分後八日,日躔中距,推得地半徑與太陽距地心比例,如一與一千一百四十二。乃以太陽最高與本天半徑比例數一0一七九二0八與地半徑比例數一一六二之比,為太陽最卑與本天半徑比例數九八二0七九二與地半徑比例之比,得一千一百二十一。既得三限距地心之遠,用平三角形逐度皆推得地半徑差。

一,求黃赤距緯以正黃道。康熙五十三年,於申昜春園累測夏至午正太陽高度,得視高七十三度二十九分十餘秒。加地半徑差五十秒,得實高七十三度三十分。減去本地赤道高五十度零三十秒,餘二十三度二十九分三十秒,為黃赤大距。用弧三角形逐度皆推得距緯。

一,求清蒙氣差以驗地中游氣映小為大、升卑為高之數。明萬歷間,西人第穀於其國北極出地五十五度有奇,測得地平上最大差三十四分。自地平以上,其差漸少,至四十五度,其差五秒,更高無差。其測算之法,如太陽視高十度三十四分四十二秒,距正午八十三度,於時日躔降婁宮三度三十六分,距赤道北一度二十六分。北極距天頂五十度零三十秒,用距正午、距赤道北、北極距天頂三度,作弧三角形,求得太陽實高十度二十七分五十三秒。與視高相減,又加地半徑差二分五十七秒,得九分四十六秒,為地平上十度三十五分之蒙氣差。本法仍之。

一,測歲實以定平行。康熙五十四年二月癸未午正,於申昜春園測得太陽高五十度零三十二秒三十五微,加地半徑差一分五十六秒零五微,得實高五十度零二分二十八秒四十微。此所加地半徑差,仍新法算書舊數加之,其實地半徑與太陽距地心比例,高、卑、中距三限,次年始定,覆推無異,故不改也。至求地半徑差,取春分及夏至後八日,亦仍舊算。其實最高之限,累日測得,不在預定。夏至中距之限既未定,歲實亦轉由最卑而得其準。最高最卑之比例,則在交食也。其廣州府偏西度,蓋先測月食時刻得之。與赤道高五十度零三十秒相減,餘一分五十八秒四十微,為太陽在赤道北之緯度。知春分時在午正前,以此緯度及黃赤大距作弧三角形,推得黃道度四分五十七秒四十三微,為太陽過春分經度。次日午正,復測得緯度,推得太陽過春分一度零四分零六秒零三微,兩過春分度相減餘為一日之行五十九分零八秒二十微,比例得本日春分在巳初三刻十四分十秒四十八微。又康熙五十五年二月戊子午正,於申昜春園測得太陽高四十九度五十四分四十九秒五十一微,依法求之,得本日春分在申初三刻二分五十五秒四十八微。總計兩春分相距三百六十五日五時三刻三分四十五秒,為歲實;為法,除天周,得每日平行。

一,求兩心差及最高所在以考盈縮。康熙五十六年二至後,申昜春園逐日測午正太陽高度,求其經度,各用本日次日比測之實行。推得五月甲戌辰正一刻零四十秒四十五微交未宮七度,乙亥巳初一刻十四分五十七秒二十七微交未宮八度,十一月丁丑子正一刻一十二分五十七秒四十一微交丑宮七度,本日夜子初三刻十二分二十七秒四十七微交丑宮八度。用此兩數以立法,如圖甲為地心,即宗動天心,乙丙丁戊為黃道,與宗動天同心,乙為夏至,丙為秋分,丁為冬至,戊為春分。又設己點為心,作庚辛壬癸圈,為不同心天,庚為最高,當黃道子,壬為最卑,當黃道醜,寅卯為中距,過己甲兩心作庚丑線,則平分本天與黃道各為兩半周。夏至乙至冬至丁,引出乙丁線,割不同心天之左半大於半周歲。秋分丙至春分戊,引出丙戊線,割不同心天之下半小於半周歲。今測未宮七度至丑宮七度,歷一百八十二日一十六時一十二分一十六秒五十六微,大於半周歲一時一十七分五十四秒二十六微;未宮八度至丑宮八度,歷一百八十二日一十四時二十七分三十秒二十微,小於半周歲二十六分五十二秒一十微。即知未宮七度在最高前如辰,八度在最高後如巳,醜宮七度在最卑前如午,八度在最卑後如未。以大小兩數相並,與辰巳或午未一度之比,同於大於半周歲之數與辰子或午丑之比,得四十四分三十六秒四十八微,與乙辰或丁午之七度相加,為高卑過二至之度。以最高卑每歲有行分,今合高卑以立算,定為本年中距過秋分之度。又用比例法推得秋分後丙午日巳正一刻十三分四十九秒過中距,若在黃道,應從最高子行九十度至寅,為辰宮七度四十四分三十六秒四十八微。以實測求之,在申不及二度零三分零九秒四十微,檢其正切,得三五八四一六為設本天半徑一千萬之己甲兩心差。又本年申昜春園測得春分為二月癸巳亥初二刻六分四十七秒,立夏為三月己卯亥正二刻一分三十六秒,秋分為八月庚子申初二刻四分三秒,各計其相距之日,推得平行度以立算。如圖甲為地心,乙丙丁戊為黃道,戊為春分,巳為夏至,丙為秋分,庚為冬至,辛為立夏。子丑寅卯為不同心天,壬為天心,春分時太陽在子,立夏在癸,秋分在寅。丑為最高,卯為最卑,求壬甲兩心差,並求辛甲乙角,為最高距立夏。取甲辰子平三角形及壬己甲勾股形,求得壬甲為三五八九七七,比前數多一千萬分之五百六十一。又求得甲角五十三度三十八分二十五秒五十五微,為最高距立夏,內減夏至距立夏四十五度,得最高過夏至後八度三十八分二十五秒五十五微,皆與前數不合。於是定用於兩心差分設本輪、均輪之法。

一,求最高行及本輪、均輪半徑以定盈縮。康熙十七年,測得最高在夏至後七度零四分零四秒。五十六年,測得最高在夏至後七度四十三分四十九秒,約得每年東行一分一秒十微。又定本天半徑為一千萬,用兩心差四分之三為本輪半徑,其一為均輪半徑。如圖甲為地心,即本天心,乙丙丁戊為本天,注左右上下為本輪,最小圈為均輪,寅為太陽最高,辰為最卑。本輪心循本天周起冬至右旋為平行,均輪心循本輪周起最卑左旋為引數。二輪之行相較,即最卑行。太陽循均輪周右旋,均輪在最高最卑,則最近於本輪心,如寅、辰;均輪在中距,則最遠於本輪心,如卯、己。其行倍於均輪積點者,舊設不同心天,數與均輪不合。

一,立矇影刻分限以定晨昏,測得在太陽未出之先、已入之後,距地平一十八度內。

月離立法之原:

一,求平行度。依西人依巴穀法,定為一十二萬六千零七日四刻為兩月食各率齊同之距,會望轉終,皆復其始。計其中積,凡為會望者四千二百六十七,為轉終者四千五百七十三。置中積日刻為實,會望數除之,得會望策。乃以天周為實,會望策除之,為每日太陰平行距太陽之度。加太陽每日平行,為每日太陰平行白道經度。又置中積日刻為實,轉終數除之,得轉終分。置天周為實,轉終分除之,為每日太陰自行度。每日白道經度與自行度相減,為每日最高行。

一,推本輪半徑及最高以考遲疾。西人第穀測三月食,如第一食日躔鶉首宮七度三十五分四十七秒五十三微,月離星紀宮度分秒同,月行遲末限之初。第二食日躔壽星宮初度,月離降婁宮度同,月行遲初限將半。第三食日躔星紀宮二度五十四分零二秒四十九微,月離鶉首宮度分秒同,月行疾末限之初。第一食距第二食一千一百八十日二十二時一十四分零四秒,實行相距八十二度二十四分一十二秒零七微,平行相距八十度二十一分一十秒,自行相距三百零八度四十七分零七秒二十七微。第二食距第三食一千九百一十八日二十三時零五分五十七秒,實行相距九十二度五十四分零二秒四十九微,平行相距八十五度零二十五秒,自行相距二百三十一度一十二分五十二秒三十三微。用平三角形推得本輪半徑為本天半徑十萬分之八千七百,又推得最高行度,計至崇禎元年首朔月過最高三十七度三十四分三十四秒,然泛以三月食推之,本輪半徑之數不合,故設均輪。

一,立四輪之行以定遲疾。西人第穀徵諸實測,將本輪半徑三分之,存其二為本輪半徑,其一為均輪半徑。本法仍之。定本輪心起本天冬至右旋為平行度,增一負均輪之圈。其半徑為新本輪半徑,加一次輪半徑之數。其心同本輪之心。本輪負而行,不自行,移均輪心從最高左旋,行於此圈之周,為自行引數。第穀又將次輪設於地心,而增次均輪。本法易之,定次輪心行均輪周,從最近右旋為倍引數,其半徑為本天半徑千萬分之二十一萬七千。次均輪心行次輪周,起於朔望,從次輪最近地心點右旋,行太陰距太陽之倍度為倍離,其半徑為本天半徑千萬分之一十一萬七千五百。太陰行次均輪之周,從次均輪最下左旋,亦行倍離。如圖甲為地心,即本天心,乙丙丁為本天之一弧,丙甲為半徑,戊為半輪最高,癸為最卑,酉為負圈最高,醜為最卑,壬為均輪最遠,辛為最近,寅為次輪最遠,亥為最近,土為次均輪最上,木為最下,即均輪心在最高又當朔望之象。又圖太陰在戌,是均輪既左旋,又當朔望之象。其得次輪、次均輪半徑於上下弦,當自行三宮或九宮時累測之,得極大均數七度二十五分四十六秒。其切線一百三十萬四千,內減本輪均輪★半徑,餘半之,即次輪半徑。於兩弦及朔望之間,當自行三宮或九宮時累測之,均數常與推算不合,差至四十一分零二秒,依法求其半徑,得次均輪半徑。

圖形尚無資料

一,以兩月食定交周。順治十三年十一月庚申望子正後十八時四十四分十五秒,月食十五分四十七秒,在黃道南,日纏星紀宮十度三十九分,在最卑後三度四十九分,月自行為三宮二十七度四十六分。康熙十三年十二月丙午望子正後三時二十三分二十六秒,月食十五分五十秒,在黃道南,日纏星紀宮二十一度五十二分,在最卑後十四度二十一分,月自行為三宮二十五度二十四分。相距中積二百二十三月。用西人依巴穀朔策定數五千四百五十八為一率,交終定數五千九百二十三為二率,二百二十三月為三率,得四率二百四十一又五千四百五十八分之五千四百五十一,為兩次月食相距之交終數。又以兩次月食相距中積六千五百八十五日零八時三十九分十秒,與每日太陰平行經度相乘,以交終數除之,得一百二十九萬零八百一十二秒小餘八七九五九八,為每一交行度。與周天秒數相減,餘五千一百八十七秒小餘一二0四0二,為每一交退行度。又以交終數除兩次月食相距中積日分,得二十七日二一二二三三,為交周日分。乃以交周日分除每一交退行度,得三分十秒三十七微,為兩交每日退行度。與太陰每日平行相加,得十三度十三分四十五秒三十八微,為太陰每日距交行。因兩次月自行差二度半,食分差三秒,故比依巴穀所定距交行差一微,仍用依巴穀所定數。

一,求黃白大距度及交均以定交行。於月離黃道鶉首宮初度,又在黃道北距交適足九十度時,俟至子午線上測之,得地平高度,減去赤道高及黃赤距緯度。一在朔望時,得大距四度五十八分三十秒;一在上下弦時,得大距五度一十七分三十秒,以之立法。如圖甲為黃極,乙丙丁戊為黃道,用兩距度相加折半,為黃白大距之中數,為半徑如巳甲,作本輪如巳庚辛壬。又取兩距度相減折半為半徑如巳癸,作均輪如癸子丑寅。其心循本輪左旋,每日行三分十秒有餘。白道極循均輪,起最近,左旋,行倍離之度。行至癸,則大距為乙卯;行至丑,則大距為乙辰。行子丑寅之半交行疾,行寅癸子之半交行遲。

一,求地半徑差如太陽。申昜春園測得太陰高六十二度四十分五十一秒四十三微,同時於廣東廣州府測得太陰高七十九度四十七分二十六秒一十二微,於時月自行三宮初度,月距日一百八十度,以之立法,用平三角形推得地半徑與太陰在中距時距地心之比例,為一與五十六又百分之七十二。依此法於月自行初宮初度月距日九十度時測之,求得地半徑與太陰在最高時距地心之比例,為一與六十一又百分之九十八。又於月自行六宮初度月距日九十度時測之,求得地半徑與太陰在最卑時距地心之比例,為一與五十三又百分之七十一。復用平三角形逐度皆推得地半徑差。

一,考隱見遲疾以辨朓朒。一驗在春分前後各三宮,黃道斜升而正降,日入時月在地平上高,朔後疾見,在秋分前後各三宮,黃道正升而斜降,日入時月在地平上低,朔後遲見,晦前隱遲、隱早反是。一驗距黃道北,見早隱遲,距黃道南反是。一驗視行遲,隱見俱遲;視行早,隱見俱早。

交食立法之原:

一,求日月視徑以定食分淺深。用正表、倒表,各取日中之影,求其高度。兩高度之較以為太陽視徑。數年精測,得太陽最高之徑為二十九分五十九秒,最卑之徑為三十一分零五秒。用墻為表,以其西界當正午線,人在表北,依不動之處,候太陰之西周切於正午線,看時辰表時刻;俟太陰體過完,其東周才離正午線,復看時辰表時刻;與前相減,變度以為太陰視徑。數年精測,得太陰最高之徑為三十一分四十七秒,最卑之徑為三十三分四十二秒。

一,求地影半徑以定光分。地半徑與太陽太陰距地心既得比例,日月視徑又得真數,太陽、太陰自高至卑視徑地半徑與太陽、太陰實徑比例。日食,人在地面見與不見。月食,太陽照地背成黑影,太陽大而地小,故成錐形。太陽有高卑,故地影有長短廣狹;太陰有高卑,故入影有淺深;皆可預推而以立法。地影半徑常大於實測,康熙五十六年八月戊戌月食,其實引為二宮三度四十一分零三秒,距地心五十七地半徑零百分之四十一。測得緯度在黃道北三十六分十八秒,月半徑為十六分十秒,食分為二十三分三十秒,乃以黃緯求得白道緯為食甚,距緯與食分相加,內減月半徑,餘四十三分四十六秒,為地影半徑。若依推算,太陽在最高,太陰在中距,地影半徑應得四十八分三十四秒,以實測之數率之,應得四十四分四十三秒,所差三分五十一秒。因驗得太陽光芒溢於原體之外,能侵削地影。以實測比算,定太陽之光分為地半徑之六倍又百分之三十七。如圖甲為地心,戊己為地徑,乙丁為太陽所照影,末當至於庚。辛壬為溢出光分侵削影,末漸次狹小,至於醜而已盡。圖形尚無資料

五星行立法之原:

一,求土星平行度。古測定二萬一千五百五十一日又十分日之三,距恆星之度分等,距太陽之遠近又等。土星行次輪會日、沖日各五十七次。置中積日分為實,星行次輪周數五十七為法,除之得周率。乃以每周三百六十度為實,周率除之,為每日距太陽之行。與太陽每日平行相減,得土星每日平行。本法仍之。

一,用三次沖日求土星本輪、均輪半徑及最高以定盈縮。明萬歷間,西人第穀測土星三次沖日。如第一次日躔娵訾宮一度零三分二十七秒,土星在鶉尾宮度分秒同;第二次日躔娵訾宮二十一度四十七分三十九秒,土星在鶉尾宮度分秒同;第三次日躔降婁宮一十六度五十一分二十八秒,土星在壽星宮度分秒同。第一次距第二次一萬一千三百四十三日五時三十六分,其實行相距二十度四十四分十二秒,平行相距十九度五十九分五十四秒;第二次距第三次七百五十五日二十時三十一分,實行相距二十五度零三分四十九秒,平行相距二十五度十九分十六秒。用不同心圈取平三角形,推得兩心差,為本天半徑千萬分之一百一十六萬二千,析為本輪半徑八十六萬五千五百八十七,均輪半徑二十九萬六千四百一十三。又推得萬歷十八年最高在析木宮二十六度二十分二十七秒,每年最高行一分二十秒一十二微。本法仍之。

一,求土星次輪半徑以定順逆。西人第穀測得次輪半徑為本天半徑千萬分之一百零四萬二千六百。本法仍之。定本輪心從本天冬至右旋為平行度,均輪心從本輪最高左旋為自行引數,次輪心從均輪最近右旋為倍引數,星從次輪最遠右旋,行本輪心距太陽之度。本輪、均輪之面與本天平行,次輪之面與黃道平行。如圖甲為地心,即本天心,乙丙丁為本天之一弧,丙甲為半徑,戊為本輪最高,己為最卑,庚為均輪最遠,辛為最近,壬為次輪最遠,癸為最近。

一,求木星平行度。古測定二萬五千九百二十七日又千分日之六百一十七,木星行次輪會日沖日皆六十五次。置中積日分為實,星行次輪周數六十五為法,除之得周率。以每周三百六十度為實,周率除之,得每日木星距太陽之行。與每日太陽平行相減,為每日木星平行度。本法仍之。

圖形尚無資料

一,用三次沖日求木星本輪、均輪半徑及最高以定盈縮。明萬歷間,西人第穀測木星三次沖日,如第一次日躔鶉尾宮七度三十一分四十九秒,木星在娵訾宮度分秒同;第二次日躔大火宮二十度五十六分,木星在大梁宮度分同;第三次日躔析木宮二十五度五十二分二十七秒,木星在實沈宮度分秒同。第一次距第二次八百零四日一十五時三十五分,實行相距七十三度二十四分十一秒,平行相距六十六度五十三分二十秒;第二次距第三次三百九十九日一十四時四十四分,實行相距三十四度五十六分二十七秒,平行相距三十三度十三分零八秒。用不同心圈取平三角形,推得兩心差,為本天半徑千萬分之九十五萬三千三百,析為本輪半徑七十萬五千三百二十,均輪半徑二十四萬七千九百八十。又推得萬歷二十八年最高在壽星宮八度四十分,每年最高行五十七秒五十二微。本法仍之。

一,求木星次輪半徑以定順逆。西人第穀測得木星次輪半徑為本天半徑千萬分之一百九十二萬九千四百八十。本法仍之。定諸輪左右旋起數及輪面如土星。

一,求火星平行度。古測定二萬八千八百五十七日又千分日之八百八十三,火星行次輪會日沖日各三十七次。置中積日分為實,星行次輪周數三十七為法,除之得周率。以每周三百六十度為實,周率除之,得每日火星距太陽之行,與每日太陽平行相減,為每日火星平行度。本法仍之。

一,用三次沖日求火星本輪、均輪半徑及最高以定盈縮。明萬歷間西人第穀測火星三次沖日,如第一次日躔元枵宮一十八度五十八分三十八秒,火星在鶉火宮度分秒同;第二次日躔娵訾宮二十三度二十二分,火星在鶉尾宮度分同;第三次日躔大梁宮一度,火星在大火宮度同。第一次距第二次七百六十四日一十二時三十二分,實行相距三十四度二十三分二十二秒,平行相距四十度三十九分二十五秒;第二次距第三次七百六十八日一十八時,實行相距三十七度三十八分,平行相距四十二度五十二分三十五秒。用不同心圈取平三角形,推得兩心差,為本天半徑千萬分之一百八十五萬五千,析為本輪半徑一百四十八萬四千,均輪半徑三十七萬一千。又推得萬歷二十八年最高在鶉火宮二十八度五十九分二十四秒,每年最高行一分零七秒。本法仍之。

一,求火星次輪半徑以定順逆。西人第穀累年密測,於太陽、火星同在最卑時,測得次輪最小之半徑,為本天半徑千萬分之六百三十萬二千七百五十;又於太陽在最卑火星在最高時,測得次輪半徑六百五十六萬一千二百五十;與最小半徑相較,為本天高卑之大差。又於火星在最卑、太陽在最高時,測得次輪半徑六百五十三萬七千七百五十,與最小半徑相較,為太陽高卑之大差。乃用比例求得火星逐時次輪半徑。本法仍之。定諸輪左、右旋起數及輪面如土、木星。

一,求金星平行度。古測定二千九百一十九日又千分日之六百六十七,金星行次輪會日退合日各五次。置中積日分為實,星行次輪周數五為法,除之得周率。以每周三百六十度為實,周率除之,得每日金星在次輪周平行,一名伏見行。其本輪心平行,即太陽平行。本法仍之。

一,求金星最高及本輪均輪半徑以定盈縮。明萬歷十三年,西人第穀於晨夕時,逐日累測金星,得距太陽極遠度,晨夕相等,定兩平行距高卑、左右度亦等。以兩平行宮度相加折半,即最高或最卑線所當宮度。又擇晨夕時距太陽極遠度相較,定小度為近最高,大度為近最卑。測得最高在實沈宮二十九度一十六分三十九秒,每年最高行一分二十二秒五十七微。又用兩測擇平行度,一當最高,一當最卑。距太陽極遠者,用平三角形及轉比例,推得兩心差為本天半徑千萬分之三十二萬零八百一十四,析為本輪半徑二十三萬一千九百六十二,均輪半徑八萬八千八百五十二。本法仍之。如圖己為地心,辛己為兩心差,戊為最高,庚為最卑,午未為金星平行,即太陽平行,甲丙為金星實行。又圖戊庚為平行,亥角為實行。

圖形尚無資料

一,求金星次輪半徑以定順逆。西人第穀測得金星次輪半徑為本天半徑千萬分之七百二十二萬四千八百五十。本法仍之。定本輪心行即太陽平行,均輪心從本輪最高左旋,為自行引數;次輪心從均輪最近右旋,為倍引數。星從次輪平遠右旋行伏見度。取金星次輪徑線不與地心參直,與本輪高卑線平行,徑線遠地心之端為平遠,近地心之端為平近,與太陰次輪均輪徑線平行者同。本輪、均輪面與黃道平行,次輪面有交角。如圖甲為地心,乙為本天半周,丙為本輪,丁為均輪,戊為次輪,己為平遠,庚為平近。

一,求水星平行度。古測定一萬六千八百零二日又十分日之四,水星行次輪會日退合日一百四十五次。置中積日分為實,星行次輪周數一百四十五為法,除之得周率。以每周三百六十度為實,周率除之,得每日水星伏見行。其本輪心平行如金星。本法仍之。

一,求水星最高及本輪、均輪半徑以定盈縮。明萬歷十三年,西人第穀如測金星法,測得水星最高在析木宮初度一十分一十七秒,每年最高行一分四十五秒一十四微。定兩心差為本天半徑千萬分之六十八萬二千一百五十五,析為本輪半徑五十六萬七千五百二十三,均輪半徑一十一萬四千六百三十二。本法仍之。

一,求水星次輪半徑以定順逆。西人第穀測得水星次輪半徑為本天半徑千萬分之三百八十五萬。本法仍之。定本輪心平行即太陽平行,均輪心從本輪最高左旋,為自行引數;次輪心從均輪最遠右旋,為三倍引數。星從次輪平遠右旋行伏見度。諸輪之面,與金星同。

一,求五星與黃道交角及交行所在以定距緯。新法算書載崇禎元年天正冬至,次日子正,土星正交在鶉首宮二十度四十一分五十二秒,中交在星紀宮二十度四十一分五十二秒,每年交行四十一秒五十三微,本天與黃道交角二度三十一分。木星正交在鶉首宮七度零九分零八秒,中交在星紀宮七度零九分零八秒,每年交行一十三秒三十六微,本天與黃道交角一度一十九分四十秒。火星正交在大梁宮一十七度零二分二十九秒,中交在大火宮一十七度零二分二十九秒,每年交行五十二秒五十七微,本天與黃道交角一度五十分。金星正交恆距最高前十六度,在實沈宮一十四度一十六分零六秒,中交在析木宮一十四度一十六分零六秒,每年交行一分二十二秒五十七微,次輪面交黃道之角三度二十九分。水星正交恆與最卑同在實沈宮一度二十五分四十二秒,中交在析木宮一度二十五分四十二秒,每年交行一分四十五秒一十四微。次輪心在正交當黃道北之角五度零五分十秒,當黃道南之角六度三十一分零二秒;次輪心在中交當黃道北之角六度一十六分五十秒,當黃道南之角四度五十五分三十二秒;次輪心在兩交之中交角皆五度四十分。凡五星交行皆順行。本法仍之。

一,求伏見限。西人多錄某測得金星當地平,太陽在地平下五度;木星水星當地平,太陽在地平下十度;土星當地平,太陽在地平下十一度;火星當地平,太陽在地平下十一度三十分;為星見之限。本法仍之。

一,求平行所在。新法算書載崇禎元年天正冬至,次日子正,土星平行距冬至八宮二十八度零八分二十七秒,木星十一宮一十八度五十一分五十一秒,火星五宮零四度四十五分三十秒,金、水同太陽。本法仍之。

一,求地半徑差。測得地半徑與土星距地心之比例,為一與一萬零九百五十三。與木星距地心之比例,為一與五千九百一十八。與火星在最高距地心之比例,為一與三千一百二十三;在中距之比例,為一與一千七百四十四;在最卑之比例,為一與四百一十。與金星在最高距地心之比例,為一與一千九百八十三;在最卑之比例,為一與三百零一;中距與太陽同。與水星在最高距地心之比例,為一與一千六百三十三;在最卑之比例,為一與六百五十一;中距與太陽同。土、木二星極遠、高、卑細數不計。用平三角形各推得地半徑差。

恆星立法之原:

一,求各星見行所在。康熙十三年,測定恆星經緯度,以十一年壬子列表。

一,求東行度。明萬歷間,西人第穀占精推測,定恆星循黃道每年東行五十一秒。本法仍之。


\end{pinyinscope}