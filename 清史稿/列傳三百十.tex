\article{列傳三百十}

\begin{pinyinscope}
籓部六

○杜爾伯特舊土爾扈特新土爾扈特和碩特

杜爾伯特部,游牧金山之東烏蘭固木地。東薩拉陀羅海、納林蘇穆河,接唐努烏梁海;南哈喇諾爾、齊爾噶圖山,接科布多牧場及明阿特;西索果克河,接阿爾泰烏梁海;北阿斯哈圖河,接烏里雅蘇臺卡倫。本額魯特綽羅斯種,與內扎薩克之隸科爾沁右翼一旗同名異族。

厄魯特舊設四衛拉特,杜爾伯特其一也,輝特隸之,後並稱衛拉特。詳青海厄魯特部傳。準噶爾臺吉噶爾丹虐諸昆弟子姓,兄子策妄阿喇布坦棄之,徙博羅塔拉,杜爾伯特諸臺吉從往,分牧額爾齊斯。迄準噶爾族亂,杜爾伯特內附,設扎薩克十有四,附輝特扎薩克二,統稱賽因濟雅哈圖杜爾伯特部。

杜爾伯特祖曰博羅納哈勒,與準噶爾祖額斯墨特達爾漢諾顏為昆弟。博羅納哈勒子額什格泰什,三傳至達賴泰什。子七:長敏珠,裔不著;次垂因;次陀音,其裔皆隸察哈爾;次鄂木布岱青和碩齊,為扎薩克汗車凌、親王車凌烏巴什、貝勒剛多爾濟三旗祖;次袞布;次達延泰什;次塔爾琿泰什,其裔隸各扎薩克。達賴泰什弟曰保伊爾登,子四。長鄂爾羅斯,為扎薩克臺吉恭錫拉、達什敦多克二旗祖;次巴特瑪多爾濟,為扎薩克貝勒色布騰、貝子班珠爾,輔國公剛、巴圖蒙克、臺吉額布根五旗祖;次額璘沁巴圖爾,為扎薩克貝子根敦、瑪什巴圖,臺吉巴爾三旗祖;次伯布什,為扎薩克郡王車凌蒙克一旗祖。和碩特臺吉鄂齊爾圖,為衛拉特首汗,綽羅斯諸臺吉隸之。

順治十四年,杜爾伯特臺吉陀音遣使哈什哈等自鄂齊爾圖所,以貢馬至。十五年,鄂木布岱青和碩齊子伊斯扎布復遣使額爾克貢馬。

康熙十四年,臺吉額勒敦噶木布從鄂齊爾圖使入貢,自稱為阿勒達爾泰什族。阿勒達爾泰什者垂因子也,時蓋為所部長。十六年,噶爾丹戕鄂齊爾圖,遣使告,自稱博碩克圖汗,因脅諸衛拉特奉己令。諭給諸貢使符驗,不從,詭稱杜爾伯特及和碩特、土爾扈特雖隸準噶爾,以牧地遠,不及給。二十四年,定四衛拉特貢例,噶爾丹使入關額二百人,餘市張家口及歸化城,其綽羅斯自貢之噶爾瑪岱青和碩齊、杜爾伯特臺吉阿勒達爾泰什及和碩特、土爾扈特長如之。

三十三年,臺吉巴拜來歸。巴拜者陀音子也,噶爾丹以附牧,強取其戚屬。巴拜索之不獲,畏弗敢爭。嗣從噶爾丹侵喀爾喀,至烏蘭布通,欲棄之降,為伊拉古克三呼圖克圖所陰阻。至是偕從子齊克宗至。上以其習邊外,不便駐內地,詔隸喀喇沁牧。

三十六年,臺吉車凌復來歸。車凌為阿勒達爾泰什孫,其父烏爾袞從噶爾丹侵喀爾喀,為大軍所敗,攜屬三百餘竄圖拉河境。上聞之,諭遣護軍統領瑪喇曰:「爾等馳赴圖拉,遣人問故。伊等或欲內附,懼為喀爾喀阻;或力不能至而在彼,可收之至。如欲往阿勒臺則聽之。既不內附,又不前往,則當相機行事。」瑪喇至,偵不獲蹤。噶爾丹再侵喀爾喀,烏爾袞復從至,和托輝特臺吉根敦陣斬之。車凌從噶爾丹竄牧巴顏烏蘭,根敦以告。詔使諭車凌降,不至。噶爾丹尋敗遁,車凌將乞降,我師不知而擊之,乃逸。其屬綽克圖巴圖爾、宰桑莽奈哈什哈、都喇圖巴圖爾、班丹哈什哈、宰桑扎爾瑚齊什賁達爾漢、宰桑蘇穆齊扎爾瑚齊、阿哈雅扎爾瑚齊、畢哩克扎爾瑚齊等率眾百餘內附。時巴拜屬從至,詔置張家口外。巴拜遣宰桑博克請賜所屬,遣官察給之。巴拜尋來朝,請效力禁廷,諭曰:「爾先眾來降,朕自有加恩之處。其仍率所屬駐喀喇沁牧。」

車凌敗,知噶爾丹不足恃,遣使奏:「杜爾伯特部自始貢中國,至阿勒達爾泰什,往來朝請已五世。前蒙恩遣巴扎爾傳諭臣屬功格額爾克,令臣歸誠,許恩待。臣遵旨降,反為將軍所擊,臣復懼而逃,乞賜恩綸。」諭曰:「車凌來歸時,我綠營、蒙古兵不知而擊之。今復遣使奏請,理籓院其檄令速降,朕將優恤之。」會遣使招噶爾丹,詔以其使從。至則車凌他徙,其使齎檄往諭。車凌遣功格額爾克奉表降,自詣大將軍費揚古所告曰:「烏蘭布通戰後,臣父烏爾袞降志誠,不獲達。臣前為大軍擊,心甚懼,率殘卒十餘奔達瑪爾,遇噶爾丹,偕赴薩克薩克圖固哩克。未浹旬,棄之走額克阿喇勒。臣知噶爾丹罪,與彼伍,徒就死。聞上撫厄魯特降人咸得所,集臣屬二百五十餘戶內徙,道逾汗阿林翁吉,閱四月始至。乞以此情代奏。」費揚古馳疏聞,留其拏屬於張家口外,遣車凌覲行營。詔授散秩大臣,巴拜如之。

明年,詔以巴拜、車凌屬隸察哈爾正白旗,編佐領二:車凌屬六品官班丹畢哩克及壯丁百餘,以功格額爾克為驍騎校領之;巴拜屬五品官戴和碩齊、納木喀琳沁、額爾德尼達木巴,六品官達爾扎巴圖蒙克、色棱泰墨爾根伊什德克及壯丁百餘,以達木巴領之。後巴拜卒,無嗣。車凌卒,子策旺達爾濟嗣。

五十四年,詔招降臺吉丹津於阿勒臺。丹津者鄂木布岱青和碩齊孫也,與車凌為昆弟,游牧阿勒臺,戶千餘。和托輝特臺吉博貝請赴阿勒臺招丹津降,抗即以兵取之。諭車凌遣使齎書從。比至,丹津徙策旺阿喇布坦牧。

五十九年,靖逆將軍富寧安擒臺吉垂木伯爾於伊勒布爾和碩。蓋是時策妄阿喇布坦假兵力據四衛拉特,令諸臺吉環牧烏魯木齊、額爾齊斯為負嵎計。我大兵因屯巴里坤、阿勒臺兩路遏之,偵準噶爾襲唐古特,詔大軍往討罪,復以兵分擊準噶爾境。垂木伯爾者丹津族臺吉也,率屬駐烏魯木齊,設哨伊勒布爾和碩、阿克塔斯路。富寧安以兵至阿克塔斯設哨,賊遁,尾至伊勒布爾和碩擊之,擒垂木伯爾歸,烏魯木齊眾聞之咸竄。

乾隆十八年冬,臺吉三車凌來歸。三車凌者:曰車凌,曰車凌烏巴什,曰車凌蒙克,統稱杜爾伯特臺吉,巴約特其屬部也。杜爾伯特以車凌為長,車凌烏巴什次之。巴約特以車凌蒙克為長,聚族額爾齊斯。準噶爾臺吉舊有策凌敦多布二,大策凌敦多卜善謀,小策凌敦多卜以勇聞,策妄阿喇布坦及子噶爾丹策凌倚任之。大策凌敦多卜孫達瓦齊襲殺噶爾丹策凌嗣而自立。小策凌敦多卜孫訥默庫濟爾噶勒與構兵,各令杜爾伯特族助。車凌等欲拒之,不敵,欲事之,莫知所從,集族言曰:「依準噶爾,非計也,不如歸天朝為永聚計。」有喀爾喀卒額璘沁達什者,為準噶爾所掠,聞其謀,脫歸以告。詔定邊左副將軍喀爾喀親王成袞扎布俟車凌等至,察其誠可納之。既而三車凌棄額爾齊斯牧,由準噶爾東烏蘭嶺烏英齊而行,越旬有九日至博東齊,遣使巴顏克什克、都圖爾噶等馳赴巴顏珠爾克,以降故告,而留其眾於額克阿喇勒以待。成袞扎布遣守汛者視,慮詐,檄喀爾喀兵備之,以聞。諭曰:「車凌等降,非叵測也。達瓦齊與訥默庫濟爾噶勒構兵,車凌等助之,勝負難預定,幸而從者勝,卒為人役,不若歸降之為得計也。既遣使以情告,若仍令處汛外,恐遣兵至或有失,可即徙入內汛,暫給牧畜,徐議安置事宜。先以車凌、車凌烏巴什及從至者酌遣數人,令其瞻仰朕躬,朕自優加恩賚。」遣侍郎玉保齎賞物往諭。甫就道,上念所部習邊外,以未出痘者生身,若即令至內地,雖傷一僕從不忍,詔俟明歲受朝塞外,勿遽來京師,以負矜恤意。而三車凌懼準噶爾兵襲,請急徙入汛,且獻馬為贄。成袞扎布納之,令暫駐烏里雅蘇臺。達瓦齊遣宰桑桑禡木特以兵襲,不及乃逸。玉保至,三車凌忭迎十里外,宣諭之。詭奏:「噶勒丹策凌時,思內附,以眾志未變,且法嚴,故不獲間。今避亂來歸,思覲天顏,蒙恩軫念避痘,令緩入覲期,請先以宰桑等朝京師。」車凌使曰和通、巴顏克什克,車凌烏巴什使曰哈錫塔,車凌蒙克使曰巴圖。明年正月,使至,詔與朝正諸籓臣宴。上以所部間道至,駝馬疲甚,且乏畜產,不忍遽遠徙,詔視推河、扎克拜達里克、庫爾奇勒可耕地置之,穀種取諸歸化城。復賜車凌、車凌烏巴什羊各五千,車凌蒙克羊三千贍之。尋定牧扎克拜達里克。

車凌烏巴什屬巴啟、齊倫等叛逸。喀爾喀卒盜車凌屬伊爾都齊馬,索不給,且射殺之。詔喀爾喀扎薩克以鄂爾坤防秋兵百視牧,復檄諸扎薩克鄰汛者弋叛賊務獲。後巴啟等就擒論罪。四月,諭曰:「內扎薩克及喀爾喀咸設正副盟長,董理牧務。今新降臺吉車凌等攜至戶口,悉編旗分佐領,其設正副盟長如內扎薩克及喀爾喀例,賜賽因濟雅哈圖盟名。」五月,駕幸熱河,駐蹕避暑山莊。三車凌率諸臺吉至,賜宴萬樹園,命觀火戲。諭曰:「杜爾伯特臺吉等皆準噶爾渠酋,向慕仁化,率萬餘眾傾心來歸,宜敷渥澤,錫予封爵,以示懷柔至意。其各鈐所屬,令安分謀業,勿負朕恩。」時所部設扎薩克十有三,自三車凌外,曰色布騰,曰蒙克特穆爾,曰根敦,曰班珠爾,曰剛,曰巴圖蒙克,曰禡什巴圖,曰達什敦多克,曰恭錫喇,曰巴爾,封親、郡王、貝勒、貝子、公、一等臺吉有差。後蒙克特穆爾以從車凌蒙克子巴朗叛逃,別授其弟額布根為扎薩克,餘仍爵,詳列傳。秋七月,將軍策楞請徙三車凌牧於歸化城青山東。時議備兵征達瓦齊,諭曰:「巴朗等甫叛竄,若徙之,將滋新降疑懼,且非辦理準噶爾本意,其令安處舊牧,勿他徙。」

三車凌之至也,告族臺吉訥默庫留準噶爾戶千餘,剛多爾濟、額爾德尼、巴圖博羅特如之,將乘間內徙。至是果偕輝特臺吉阿睦爾撒納、和碩特臺吉班珠爾至,詔賜牧畜,置塔楚,鄰三車凌牧。十月,駕由盛京旋,駐蹕避暑山莊。訥默庫等入覲,復賜宴,錫之爵。曰訥默庫,封郡王;曰剛多爾濟,曰巴圖博羅特,封貝勒;曰布圖克森,曰額爾德尼,曰羅壘雲端,封貝子;曰布顏特古斯,曰蒙克博羅特,封輔國公;曰烏巴什,曰伯勒克,封一等臺吉。凡設扎薩克十,詔編旗分佐領,如三車凌例,分左、右翼,設正副盟長各一。訥默庫者,車凌烏巴什兄子。剛多爾濟、布圖克森、額爾德尼、羅壘雲端、烏巴什、伯勒克,皆車凌烏巴什曾祖察袞裔。布顏特古斯、巴圖博羅特、蒙克博羅特亦戚族也。後訥默庫晉親王,子喇嘛扎卜授貝勒,以叛除爵。布圖克森、羅壘雲端、烏巴什,皆無嗣停襲。伯勒克卒,子多第巴襲。多第巴卒,子尼爾瓦齊襲。尼爾瓦齊卒,無嗣,以多第巴弟布顏德勒格爾襲。布顏德勒格爾卒,無嗣停襲。布顏特古斯卒,子舍夌襲,以叛除爵。剛多爾濟無嗣,以從子達瓦丕勒襲。額爾德尼卒,無嗣停襲。巴圖博羅特、蒙克博羅特皆以叛除爵。故自剛多爾濟外,皆不立傳。

二十年,烏梁海降臣察達克招服包沁,察獲杜爾伯特屬以獻,詔給所部。尋從大軍征達瓦齊。三車凌既入覲歸,詔選兵二千,以車凌領其一,隸北路;車凌蒙克、色布騰從之,以車凌烏巴什領其一,隸西路:各授參贊大臣。訥默庫等繼至,請從軍,詔隸西路。以車凌烏巴什、訥默庫皆幼不更事,詔調車凌蒙克赴西路軍,從車凌烏巴什、訥默庫等行。而是時阿睦爾撒納為北路副將軍,訥默庫其妻弟也,固請隸北路軍,允之。以故偕三車凌至者隸西路副將軍薩拉勒隊,偕訥默庫至者隸北路副將軍阿睦爾撒納隊。賜車凌整裝銀二千,車凌烏巴什、訥默庫各減十分之二,給從軍者羊及餱有差。復詔使車凌及車凌蒙克遣宰桑以善耕卒百赴額爾齊斯,蓋杜爾伯特眾兼耕牧業,視喀爾喀專以牧為業者異。將遣綠旗及喀爾喀兵屯耕額爾齊斯,以所部識水泉道,且善耕,命簡卒往導,俟大功成,遣牧眾歸額爾齊斯。會北路軍奏至,以訥默庫參贊列名,詔西路軍奏如之,列三車凌及色布騰名,次參贊大臣鄂容安後。復諭定北將軍班第,俟伊犁定,遣車凌、車凌烏巴什等率新降諸臺吉入覲。

初,議徵達瓦齊,上以衛拉特諸臺吉後先附,凡數萬眾,錯處內牧,非得地眾建之不可。詔俟準噶爾定,將復設四衛拉特,以車凌為杜爾伯特汗,別以班珠爾為和碩特汗,以阿睦爾撒納為輝特汗,以噶爾丹策凌子姓為綽羅斯汗。車凌等赴軍時輒聞命。大兵抵伊犁,達瓦齊就擒。班第以車凌烏巴什、訥默庫及新降之綽羅斯臺吉噶勒藏多爾濟、和碩特臺吉沙克都爾曼濟、輝特臺吉巴雅爾等列入覲初班。駕幸木蘭,車凌等至,召覲行幄慰諭之。旋蹕避暑山莊,御淡泊敬誠殿受朝,詔以車凌為杜爾伯特汗,諸扎薩克隸之。扎薩克而下,設管旗章京、副管旗章京、參領、佐領、驍騎校等職。時阿睦爾撒納覬轄四衛拉特,知不可得,叛竄。班珠爾以附逆,械至。噶勒藏多爾濟、沙克都爾曼濟、巴雅爾仍各賜汗爵,統所部眾。諭曰:「準噶爾互相殘殺,群遭塗炭,不獲安生。朕統一寰區,不忍坐視,特發兩路大兵進討。諸臺吉、宰桑等畏威懷德,率屬來歸,從軍自效。今已平定伊犁,擒獲達瓦齊,是用廣沛仁恩,酬庸效績。準噶爾舊有四衛拉特汗,令即仍其部落,樹之君長,其各董率所屬,務勤養教,共圖生聚,受朕無疆之福。」其後綽羅斯汗噶勒藏多爾濟叛,從子扎納噶爾布戮之,所部就滅。輝特汗巴雅爾以叛為大軍所擒誅。和碩特汗沙克都爾曼濟懷貳志,副都統雅爾哈善殲其眾於巴里坤。惟杜爾伯特部恪守臣節,世受封爵罔替。

是年十二月,車凌等以乏牧產,請徙額克阿喇勒。諭曰:「前議平定伊犁後遣歸舊牧額爾齊斯,若額克阿喇勒,距額爾齊斯較扎克、拜達里克路更邇,且附內汛外,調所部兵亦易。俟擒獲阿逆後,仍當遣歸舊牧。所部生計既艱,其給籽種六百石,務令及時耕種,毋誤農期。至從軍所給駝馬,自應交納。但念往返道遠,牲畜不無疲瘠,可姑緩期二載。」

訥默庫之將從征達瓦齊也,請徙牧拜達里克北扎布堪河源博囉喀博齊爾至鄂爾海、喀喇烏蘇界,允之,諭努力成功,勿念游牧眾。至是以車凌等將徙牧,詔往會。而訥默庫隱有叛志,謀竄就阿睦爾撒納。剛多爾濟、巴圖博羅特、布顏特古斯等阻之,卒不戢,率眾復乘間劫驛騎,戕守汛弁,奪運糧商民駝物及貲。二十一年春,駐防烏里雅蘇臺辦事大臣阿蘭泰偕車凌、車凌烏巴什等以兵擒訥默庫及其孥,械至,論如律。詔不附逆諸扎薩克各安游牧,勿疑懼。復諭曰:「剛多爾濟等屬妄行劫掠,應交部議扎薩克罪。但念伊等新降,未諳內地禁例,姑從寬免。」夏,以所部鄰扎哈沁,盜不戢,諭曰:「伊等生計全賴牧畜,若復盜竊相仍,不獲蕃孳,生計焉能充裕?其各鈐束部眾,務期守分安生,副朕休養群生至意。」

有伯什阿噶什者,伊什扎布之曾孫也,祖扎勒,父車凌多爾濟。伯什阿噶什兄曰布達扎卜、曰達瓦克什克,弟曰達瓦濟特、曰格咱巴克,聚牧伊犁河西沙拉伯勒,境鄰哈薩克牧。達瓦齊虐其眾,伯什阿噶什將棄之,懼襲而寢。大軍征達瓦齊,抵伊犁,班第遣使招,因獻籍三千餘戶降。將遣從車凌等入覲,告哈薩克數掠所部,請歸視。比抵牧,偵哈薩克集兵,遣告,且請大軍援,諭嘉其恭順。

會阿睦爾撒納叛,逆黨擾伊犁,遣和碩特輔國公納噶察齎敕往諭曰:「準噶爾內亂頻仍,各部人眾咸失生業。朕為一統天下之君,懷保群生,無分中外,特發大軍往定伊犁。方欲施恩立制,永安反側,乃逆賊潛懷叛志,妄思並吞諸部,肆其荼虐,罪狀已著,畏誅潛遁。朕已命將窮追,務期弋獲。逆賊一日不獲,諸部一日不安。爾臺吉輸誠歸命,果能仰體朕旨,去逆效順,或以兵協剿阿逆,或俟至爾牧擒獻之,朕必大沛殊恩。爾其奮勉自效!」達瓦齊復奏伯什阿噶什及庫木諾顏、臺吉諾爾布必無異志,命遺之書,未達,而伯什阿噶什徙牧。初傳偕諾爾布內附,久之不至,或以居博囉塔拉告。詔將軍策楞等偵之,無其蹤。時阿睦爾撒納敗竄,諭參贊大臣侍郎玉保等偵阿逆赴伯什阿噶什牧,即諭擒獻,或故縱,以兵剿之。伯什阿噶什養子博東齊尋偕宰桑諾斯海挈眾至,以哈薩克侵牧告。宰桑賽音伯勒克,得木齊恩克、濟爾哈爾等踵至,告哈薩克追掠,間走乃免。詔博東齊以兵迎其父,暫置從眾於額爾齊斯,諾斯海護視之。賽音伯勒克或從博東齊往,或留牧額爾齊斯,惟其便。博東齊將行,伯什阿噶什攜戶八百餘抵額爾齊斯,請內附。烏巴什其族臺吉也,從至。詔封伯什阿噶什為扎薩克和碩親王,烏巴什為扎薩克固山貝子,賜諭曰:「爾誠心感戴,率眾投誠。前大軍抵伊犁,即謁將軍大臣,甫欲加恩封賞,旋遇阿逆背叛,未獲舉行。爾為哈薩克所掠,展轉遷徙,始克內附。爾眾甫至,不必簡兵往從大軍,亦無須徙內地,即游牧額爾齊斯所。爾族臺吉車凌等將歸舊牧,爾等聚族而處,實為允協,不必遠離故土,徒勞往返也。」命甫下,伯什阿噶什等攜眾抵哈達青吉勒,詔暫留,俟明歲歸額爾齊斯牧。

七月,車凌、車凌烏巴什、剛多爾濟等以徙牧額爾齊斯,請定入覲年班。諭嘉其誠悃,詔自來年始,定三班,前給從軍駝馬,姑緩期納,示恤。九月,伯什阿噶什來朝,弟達瓦濟特及兄子丹巴、都噶爾、布魯特扣肯以視牧故,各遣宰桑代至。賜宴,賚馬七百、牛百五十、羊三千,詔編旗分佐領,如三車凌及剛多爾濟等來歸例。別為一盟,以伯什阿噶什為盟長,烏巴什副之,丹巴都噶爾授協理臺吉。

伯什阿噶什甫歸牧,其妻卒,遣侍衛佛保往醊。伯什阿噶什尋卒,無子,詔副都統唐喀祿賻祭,宣諭以丹巴都噶爾為扎薩克固山貝子,以達瓦濟特為扎薩克公,轄伯什阿噶什眾,聽歸車凌牧及內徙。而丹巴都噶爾與佐領色布騰互攘畜產,佛保將至牧,駝馬為所掠。詔撤恩命還,復諭烏巴什勿驚懼,俟事定歸車凌牧。後烏巴什卒,停襲。

二十二年,車凌以哈薩克不擒獻阿逆,諸厄魯特叛擾邊,請由額爾齊斯徙牧烏蘭固木避之。時喀爾喀貝子車布登扎布遵旨遣兵剿掠佛保賊,收伯什阿噶什屬戶給喀爾喀,將遣博東齊歸車凌牧,族臺吉布圖庫、班珠爾、布林等挈屬至,稱與車凌等析處久,請異牧,允之。布圖庫等抵汛,聞佛保自哈達青吉勒歸,和碩特臺吉桑濟復掠諸道,遣從卒馳馬迎。上聞之,諭曰:「車凌等自歸誠以來,感激朕恩,約束屬眾,甚為寧謐。邇因叛賊紛起,亟請內徙游牧,其歸附之心益堅,可允所請,並給穀種,令為謀生資。博東齊雖與杜爾伯特同族,若往歸之,反仰賴車凌等養贍,著遣往烏里雅蘇臺,交車布登扎布,酌徙呼倫貝爾、通肯呼裕爾等處。布圖庫、班珠爾等迎接侍衛佛保,俟至烏里雅蘇臺軍所,各給幣賞之。」後博東齊及布圖庫等咸置呼倫貝爾。布圖庫、班珠爾以內附誠,各授二等臺吉。而貝勒巴圖博羅特、輔國公舍棱不從車凌等徙牧,叛應阿睦爾撒納,副都統瑚爾起以兵擒諸輝巴朗山,妻拏悉論誅。

先是杜爾伯特及烏梁海未內屬,錯牧額爾齊斯。後杜爾伯特諸臺吉至,游牧扎克拜達里克,初徙牧額克阿喇勒,再徙額爾齊斯。烏梁海就撫,以烏蘭固木地給之。車凌等復請由額爾齊斯往徙,遣都統納穆扎爾往勘杜爾伯特及烏梁海牧界。車凌復請以烏蘭固木為屯耕地,而游牧於科布多、額克阿喇勒,允之,詔嚴禁所屬勿攘竊。尋以錯牧不便,定烏蘭固木為杜爾伯特牧,別以科布多為烏梁海牧。

二十四年,烏梁海以科布多產貂不給捕,請徙就阿勒臺陽額爾齊斯。諭車凌烏巴什等曰:「額爾齊斯為爾舊牧,今爾移處烏蘭固木,烏梁海察達克請游牧額爾齊斯地,向曾降旨,爾等若原歸舊牧,聽爾便。今哈薩克已全部內附,伊犁厄魯特賊眾復殲無孑遺。若爾果原歸舊牧,可即徙往額爾齊斯,所遺烏蘭固木,自可給烏梁海處之。但哈薩克新附,非爾等久為內屬者比,務宜嚴飭所屬安靜無事。若爾部眾既遵鈐束,而哈薩克反來肆擾,可即擒誅之。爾等或安土重遷,則額爾齊斯地與其為哈薩克、俄羅斯所竊據,不若令烏梁海往徙之也。」車凌烏巴什等奏:「察達克所請地,系烏梁海舊牧,距臣等牧遠。且烏蘭固木地肥不磽,臣等游牧久,請勿徙,以額爾齊斯地給烏梁海。」詔如所請。是年十月,以大軍定回部蕆功,諭車凌烏巴什等知之。十二月,偵哈薩克襲烏梁海,以兵三百餘擊走,得旨獎賚。

二十五年四月,以所部有溫圖呼爾者,貧不給,聞其弟居察哈爾牧,告諸扎薩克往就之。諭曰:「杜爾伯特自歸誠以來,編設旗分佐領,原欲伊等各安生業。若不善恤之,漸至析處,殊為可憫。其各加意撫綏,令守分謀生,勿至流離失所,副朕恫一體之懷。」七月,車凌烏巴什等扈蹕行圍,奏所部蒙恩安置,牧產漸饒,嗣請自備駝馬。上嘉其誠悃,不忍驟勞之,詔仍官給駝馬。

二十七年,詔左、右翼各設副將軍一,右翼用正黃旗纛,左翼用正白旗纛,以敕印軍符給之。所部旗十有六,爵如之:扎薩克特古斯庫魯克達賴汗一;扎薩克和碩親王一;扎薩克多羅郡王一;扎薩克多羅貝勒二;扎薩克固山貝子二;扎薩克鎮國公一,由貝子降襲;扎薩克輔國公二;扎薩克一等臺吉四;輝特扎薩克一等臺吉二。四十五年,命烏里雅蘇臺將軍巴圖查辦喀爾喀侵占杜爾伯特、扎哈沁等部界址。

道光二年,修科布多眾安廟。三月,科布多參贊大臣那彥寶奏定蒙民、商民貿易章程。杜爾伯特、扎哈沁、明阿特、額魯特均準給票與商民貿易。六年,回疆軍興,杜爾伯特汗、王、公、扎薩克等獻駝馬助軍。十二月,以杜爾伯特汗齊旺巴勒楚克等復輸駝助軍,上嘉賚之。九年,杜爾伯特貝子奇默特多爾濟呈控科布多參贊大臣額勒錦需索馬匹,擾累各部。鞫實,罷之。十八年,是部以兵從烏里雅蘇臺參贊大臣車林多爾濟驅逐闌入烏梁海之哈薩克。十八年十二月,以烏里雅蘇臺參贊大臣車林多爾濟奏科布多參贊大臣管理烏梁海八部落,地方遼闊,多興訟端,允增置幫辦大臣。十九年,給是部官兵俸賞行裝銀。咸豐三年二月,是部汗、王、公等捐助軍需,溫旨卻之。

同治三年,烏魯木齊等城回匪滋事,調是部兵援之。尋以不得力,撤歸。八年,以杜爾伯特汗嗣絕,將軍麟興等奏:「左翼汗旗下舊管十佐領戶一千五百有奇,右翼親王旗下舊管十一佐領戶一千二百上下,右翼貝勒旗下舊管二佐領僅一百六十餘戶。以爵而論,貝勒較輕;以戶口而論,不過抵汗三十分之一。擬親王棍布扎布令折回承襲汗爵,以貝勒巴雜爾扎那承襲親王,貝勒一缺如無可承襲之人,俟汗王襲爵定後,即將貝勒暫行停襲。」下所司。九年,命以故汗密什多爾濟族弟噶勒章那木濟勒襲汗,棍布扎布等襲親王、貝勒如故。回匪東竄,陷烏里雅蘇臺。十一月,科布多參贊大臣奎昌等奏:「匪撲烏里雅蘇臺地方,各臺潰散,科城街市商民惶惑,調附近之杜爾伯特、扎哈沁、明阿特、額魯特盟長、總管等,即發兵來城聽候調遣。」尋奏杜爾伯特左翼兵四百名、右翼及明阿特、額魯特兵各二百名、扎哈沁公兵及總管兵各五十名,均到科城收伍,命撥科布多餉銀十萬兩。十一年十一月,予辦差無誤之杜爾伯特右翼盟長棍布扎布等獎。是月,科布多參贊大臣長順等奏:「十月十七、十八等日,匪徑撲本城,參將英華督弁兵登壁迎擊,匪始敗退,守備賀遐齡等陣亡。十九日,匪復攻撲南關,不得逞。二十日,由東南山路仍向扎哈沁部落奔竄。」自後回匪出沒於扎哈沁、土爾扈特諸部之地,是部警備益嚴。至西路肅清,始息警撤戍。

光緒七年,以改議俄約,增城科布多之戍,事定,撤之。二十六年,拳匪事起,北路戒嚴。科布多參贊大臣瑞洵議舉辦蒙古團練,令杜爾伯特每旗挑選兵丁二百名,一半馬隊,一半步隊,駐防本旗。十月,事定,裁撤。二十八年四月,瑞洵以杜爾伯特正副盟長等保全俄商遺棄貨物,毫無損失,請準獎敘,允之。七月,賑杜爾伯特右翼公多諾魯旗災,並給耔種大小麥一百石,引渠溉舊墾波什圖、那米拉、察罕哈克三處之地。二十九年閏五月,予杜爾伯特左翼正盟長副將軍特固斯庫魯克達賴汗噶勒章那木濟勒紫韁,副盟長貝勒納遜布彥、左翼扎薩克郡王圖柯莫勒、右翼正盟長副將軍扎薩克親王索特納木扎木柴三眼花翎,左翼扎薩克貝勒納遜布彥等雙眼花翎,餘給獎有差。是年,辦布倫托海屯田渠工,以杜爾伯特左、右翼助借駝只,均給幫價銀。其後參贊大臣連魁等議開烏蘭固木等屯田。

宣統二年四月,索特納木扎木柴為資政院欽選議員。三年,庫倫獨立,喀爾喀四部無梗抗者。是部汗噶勒章那木濟勒獨不附,聽參贊大臣溥節制如故。

其地雜耕牧,有礦,有鹽。共有佐領三十五。

杜爾伯特附近之部同隸科布多參贊大臣者,曰扎哈沁,東扎薩克圖汗部,南新疆鎮西,西阿爾泰烏梁海,北科布多屯田官廠。

初,禡木特,額魯特人,號庫克辛,為準噶爾之扎哈沁宰桑。扎哈沁者,譯言「汛卒」,以宰桑領之。禡木特守阿爾泰汛,游牧布拉罕察罕托輝。其東為喀爾喀,有烏梁海界之;其西為準噶爾,有包沁雜準及噶拉雜特、塔本集賽界之。包沁為回族,準噶爾呼砲曰「包」,以回人司砲,故名。噶拉雜特、塔本集賽,皆準噶爾鄂拓克。鄂拓克如各旗佐領。

乾隆十一年,準噶爾臺吉策妄多爾濟遣禡木特請赴藏熬茶。十八年,杜爾伯特臺吉車凌棄準噶爾來降,臺吉達瓦遣禡木特追之,由博爾濟河入喀爾喀汛,復逸出。諭責駐防烏里雅蘇臺達青阿罪。明年春,達青阿誘擒之,詔宥罪遣歸。有準噶爾宰桑,別號通禡木特,游牧諾海克卜特爾,近索勒畢嶺,為布拉罕察罕托輝下游。禡木特將掠通禡木特,為請降計,通禡木特覺,誘執之。內大臣薩喇勒諜得狀,由烏蘭山陰以兵驟至,通禡木特就擒,索得禡木特,責負恩罪。禡木特請徙牧內屬,遣扎哈沁得木齊招所部六百餘戶降。薩喇勒檻示馬木特至軍,詔仍釋之。入覲京師,上鑒歸附志誠,授內大臣,賜冠服。二十年,詔與朝正會宴。以通禡木特卒,諭禡木特善視其戚屬。時議徵達瓦齊,詔阿睦爾撒納為定邊左副將軍,以禡木特參贊軍務。禡木特密奏:「阿睦爾撒納,豺狼也,雖降,不可往,往必為殃。」上以「不逆詐」諭之,詔授禡木特總管號。

初,準噶爾定扎哈沁、包沁納賦例,比年獻脯,間年供牲贍喇嘛,遇軍事令助。詔如舊例,恤免期年賦。禡木特與阿睦爾撒納會軍於額德里克,尋抵伊犁。詔晉禡木特三等公爵,賜信勇號,賞雙眼孔雀翎、四團龍服,命常服之。先是諭班第俟伊犁定,偕禡木特議準噶爾善後事。至是班第以禡木特兼管扎哈沁、包沁牧,請仍至阿爾泰,增喀爾喀籓籬,允之。尋撤大軍還,扎哈沁兵三百遣歸牧,禡木特以疾留伊犁。聞阿睦爾撒納驟叛,將脫歸牧之兵衛,為逆黨哈丹等所遮,脅之降,不從,擒赴阿睦爾撒納所。阿睦爾撒納慰之,禡木特唾而詈之,為阿睦爾撒納縊殺。明年二月,定西將軍策楞諜阿睦爾撒納戕禡木特,以聞。諭曰:「禡木特年就邁,效力行間,甚為奮勉。今逆賊戕之,深為憫惻!其孫扎木禪,令仍襲公爵。」大軍定伊犁,械逆黨至,訊得禡木特就死狀,上制詩憫之。扎木禪乾隆二十一年襲三等信勇公。

三月,以阿睦爾撒納煽烏梁海梗赴哈薩克,詔從北路將軍哈達哈剿烏梁海叛賊。九月,賜牧哲爾格西喇呼烏蘇。諭曰:「扎哈沁既與喀爾喀鄰牧,即設哨附近卡倫,視喀爾喀例支領錢糧,以資養贍。」二十四年,從參贊大臣齊努渾追剿瑪哈沁,至阿爾齊圖。以兵先遇賊哈喇呼山,奮擊之,屢就擒,獎賚幣。二十五年,扎木禪子門圖什扈蹕行圍,乞喀爾喀親王成袞札布代請駝馬勿官給。上以扎哈沁甫定牧畜之生計,諭仍官給。二十六年,理籓院議禡木特歸誠後,扎哈沁屬相繼附,置佐領九,得二千餘口,雖補總管,未給印,請以總管扎哈沁一旗總管印給扎木禪轄其眾,允之。四十年,扎木禪卒。以扎哈沁原非禡木特之阿爾巴圖,撤出佐領,設一旗屬科布多參贊大臣。其扎木禪族丁及其阿爾巴圖三十餘戶,亦附近科布多之烏裕克齊、博多克齊游牧。至四十五年五月,諭將軍巴圖等不可令扎薩克圖汗部侵占扎哈沁之烏英濟等處隙地。

嘉慶五年,以扎木禪之孫托克托巴圖之屬已足百五十丁,復編一佐領,即以托克托巴圖為總管。十一年,以前科布多參贊大臣恆博招民人開採是部煤窯,議處。道光二年,定是部準給票與商民貿易。六年,回疆軍興,是部捐助駝馬。

同治三年,以烏魯木齊失陷,調杜爾伯特諸部兵援古城。旋仍令撤歸。四年,以古城陷,撤是部南境察罕通古等通古城三臺,歸沙扎蓋以北五臺支應西路各差。九年十月,回匪陷烏里雅蘇臺而復竄去,科布多告警,參贊大臣奎昌等調是部二旗兵各五十名赴城收伍。是部東南通扎薩克圖部,南接新疆,為用兵要沖。十一年十月,回匪由是部犯科布多,不得逞,仍竄是部,聚扎盟南境。十二年九月,匪擾察罕通古臺站,掠景廉軍營軍裝餉銀,竄新土爾扈特貝子游牧布拉噶河一帶,科城西南兩路臺站紛紛逃散。匪又由巴里坤紅柳峽一帶竄踞扎哈沁之博東齊。十月,科布多幫辦大臣保英率兵敗之於博東齊以西,匪竄扎盟阿育爾公旗。光緒二年四月,回匪由布倫托海竄沙扎蓋地方,額勒和布等派官兵剿之。金順以索倫各隊扼扎烏魯木湖,堵截分竄。其後烏魯木齊諸城克復,是部始息警。

十二年,甘肅新疆巡撫劉錦棠以古城屬漢三塘驛,來往商賈,時有劫案,咨科布多大臣飭屬緝匪沙克都林扎布。因奏:「漢三塘驛與科城所屬土爾扈特、扎哈沁等旗地界毗連,萬里沙漠,四通八達,更兼白塔山商賈由此經過,屢被劫掠,又北八站一帶搶臺劫站之案,亦層見迭出。請將扎哈沁旗內揀派駐察罕淖爾官兵移駐鄂隆布拉克臺,保安商民,搜捕盜匪。」允之。

二十六年,拳匪事起,邊戍戒嚴,參贊大臣瑞洵檄是部信勇公策林多爾濟、總管三保、額魯特總管喇嘛札布、明阿特總管達什哲克博舉辦團防,保護俄商貨物,用弭邊釁。二十九年閏五月,一再請獎。奏入,予策林多爾濟貝子銜,三保等均二品頂戴。三十一年五月,瑞洵奏:「科布多所轄扎哈沁應用之五臺,尤為大雪封壩。復赴阿爾泰必由之路,信使絡繹,地當其沖。扎哈沁共二旗,最為瘠苦,公一旗戶口甚稀。幫辦大臣英秀由哈巴河回科布多,臣赴新疆督辦收撫,信勇公策林多爾濟調集烏拉,奔走恐後,保其子臺吉棍布瓦齊爾,請賞二品頂戴。」允之。宣統三年,參贊大臣溥奏賑扎哈沁災,公旗貧民三百五十六丁口,總管旗貧民一千有一十一丁口,將賞銀五千兩分別重輕散放。下所司。

額魯特、明阿特亦與是部同隸科布多。額魯特本臺吉達木拜屬。達木拜有罪削爵,以其眾屬科布多,游牧在新和碩特之西。明阿特本出烏梁海,復為扎薩克圖汗部中左翼左左旗之屬,乾隆三十年撤出。設一旗屬科布多,游牧在阿爾泰烏梁海之西。乾隆五十七年,設額魯特、明阿特總管各一,參領以下有差。同治十年,以防守科城及供大兵西進勞,額魯特、明阿特總管與扎哈沁信勇公及總管均予獎。兩旗皆無扎薩克,論者謂此蒙部之同於郡縣者也。

舊土爾扈特,始祖元臣翁罕,姓不著。七傳至貝果鄂爾勒克,子四,長珠勒扎幹鄂爾勒克,生子一,曰和鄂爾勒克,居於雅爾之額什爾努拉地。初衛拉特諸酋以伊犁為會宗地,各統所部不相屬。準噶爾部酋巴圖爾琿臺吉者,游牧阿爾臺,恃其強,欲役屬諸衛拉特。和鄂爾勒克惡之,挈族走俄羅斯,牧額濟勒河,俄羅斯因稱為己屬。

順治十二、三、四年,和鄂爾勒克子書庫爾岱青、伊勒登諾顏、羅卜藏諾顏相繼遣使奉表貢。書庫爾岱青子朋蘇克,朋蘇克子阿玉奇,世為土爾扈特部長,至阿玉奇始自稱汗。康熙中,表貢不絕。五十一年,復遣使假道俄羅斯貢方物。上嘉其誠,且欲悉所部疆域,遣內閣侍讀圖理琛等賚敕往,歷三載乃還,附表奏謝。自是時因俄羅斯請於中朝,遣所部人赴藏熬茶。乾隆二十一年,所部使吹扎布等入覲,稱奉其汗惇羅布喇什令,假道俄羅斯,三載方至,請赴唐古忒謁達賴喇嘛,遣官護往。二十二年,自唐古忒還,頒惇羅布喇什幣物。

二十三年,伊犁平,有附牧伊犁之土爾扈特族臺吉舍棱等奔額濟勒河。既而惇羅布喇什卒,子渥巴錫嗣為汗。三十五年,舍棱誘渥巴錫攜所部之土爾扈特、和碩特、輝特、杜爾伯特等人眾於十月越俄羅斯之坑格圖喇納卡倫而南,俄羅斯遣兵追之不及。渥巴錫既入國境,由巴爾噶什淖爾而進,至克齊克玉子地方,與哈薩克臺吉額勒里納拉里之眾相持。伊犁將軍令哈薩克毋許土爾扈特越游牧而行,渥巴錫遂向沙喇伯可而進,布魯特群起劫之。渥巴錫走沙喇伯可之北戈壁,無水草,人皆取馬牛之血而飲,瘟疫大作,死者三十萬,牲畜十存三四。三十六年,至他木哈地方,近內地卡倫,布魯特始斂兵退。將軍伊勒因遣侍衛普濟問來意,渥巴錫與其臺吉、喇嘛計議數日始定,以投誠為詞,獻其祖所受明永樂八年漢篆敕封玉印及玉器、宣窯磁器等物。先是上聞渥巴錫之來,命烏什參贊大臣舒赫德往伊犁經紀其事。至是因受其降,存七萬餘眾,賑以米、麥、牛、羊、茶、布、棉裘之屬,用帑二十萬兩。三十六年九月,渥巴錫等入覲熱河,封渥巴錫舊土爾扈特卓里克圖汗,渥巴錫從子額墨根烏巴什固山巴雅爾圖貝子,拜濟瑚輔國公,從弟伯爾哈什哈一等臺吉,均授扎薩克,各編一旗。四十七年,均予世襲罔替。

初分所部為四路,南路凡四旗,曰扎薩克卓理克圖汗旗,曰中旗,曰右旗,曰左旗。三十七年,賜牧齋爾。三十八年,徙牧珠勒都斯,隸喀喇沙爾辦事大臣,與北路三旗、東路二旗、西路一旗統受節制於伊犁將軍。

嘉慶四年,高宗大行,舊土爾扈特汗霍紹齊之母請納俸諷經,不許。道光六年,回匪張格爾擾喀什噶爾等城,徵是路土爾扈特及和碩特蒙兵赴阿克蘇一帶助剿。十月,擊退犯渾巴什河之賊,賚貝子巴爾達拉什、臺吉烏圖那遜等及兵丁等緞疋、翎頂、銀兩有差。自是回疆有事,皆徵其兵。十年十一月,以貝子巴爾丹拉什率兵援喀、英等城,卒於軍,命其子蒙庫那遜晉襲貝勒。十八年六月,以是部南路盟長福晉喇什丕勒指修喀喇沙爾城垣,予獎。二十一年六月,又獻伊拉里克水源,卻之。二十七年,布魯特擾喀什噶爾等城,亦徵是路蒙兵防剿,事定撤回。

同治三年,回匪變亂,庫車失陷,徵是路兵剿之,不利,退守游牧。是年,喀喇沙爾等城均失陷,是路部落屢與回匪接戰,被蹂躪離散。六年十二月,盟長布雅庫勒哲依圖請赴京,允之,命烏里雅蘇臺將軍麟興等設法安插其部落游牧。七年三月,布雅庫勒哲依圖請率屬剿回逆,上嘉之,命赴布倫托海候李云麟酌辦,並飭戶部籌撥歷年俸銀俸緞,李云麟接濟所屬游牧人眾。六月,以舊土爾扈特蒙兵接仗失利,移至大小珠勒都斯,催布雅庫勒哲依圖赴布倫托海,命明瑤等接濟照料。十一月,麟興奏布雅庫勒哲依圖困苦情形,下所司議。八年三月,賚舊土爾扈特汗布雅庫勒哲依圖、貝勒固嚕扎布、輔國公曼吉多爾濟等旗銀二萬兩。六月,命烏里雅蘇臺將軍福濟安插舊土爾扈特汗布雅庫勒哲依圖及隨帶官兵。

光緒元年,布雅庫勒哲依圖卒,以福晉恩克巴圖署盟長。二年八月,撥部庫銀予恩克巴圖撫綏人眾,擇地安插。三年,劉錦棠等軍復喀喇沙爾。四年十二月,伊犁將軍金順奏土爾扈特南部落人眾,自逆回構亂以來,逃散伊犁空吉斯及西湖等處,署盟長派員前往收集,約計一萬餘人,現已移回珠勒都斯游牧。諭以其部人眾困苦,賞銀四萬恤之,由左宗棠發給。八年,是部難民由伊犁續歸三百三十餘丁口,舊有府第,兵燹之後,尚未修復,大小水渠,年久墊淤。欽差大臣劉錦棠奏:「恩克巴圖請賑恤,並籌借銀兩。權為籌撥銀一萬兩,作為渠工宅第經費。喀喇沙爾善後局員照章給賑,通融接濟牛種,待賑丁口糧,俾資耕作。請分別核銷及作正開銷。」允之。九年,設新疆喀喇沙爾直隸撫民同知兼理事銜,兼管土爾扈特游牧事宜。十三年,新疆巡撫劉錦棠奏:「土爾扈特等蒙眾向隸辦事領隊管轄者,應改歸地方官管轄。恐各蒙民未能戶曉,請飭理籓院申明新設定制,轉行各蒙部。」下所司知之。

二十二年三月,甘肅回匪西竄出關,伊犁將軍長庚電奏賊窺珠勒都斯,檄南部落署盟長福晉色裏特博勒噶丹等揀選有槍馬之蒙兵五百名,由貝勒恭噶那木扎勒統之,分派參領奔津等各帶官兵駐哈布齊沿山口及哈哈爾達巴罕、達蘭達巴罕等處,扼珠勒都斯之東,逼喀喇沙爾、庫爾勒要隘。八月,事定,撤歸。

新疆置省後,舊土爾扈特諸部仍隸伊犁將軍,俸銀俸緞均由伊犁發給。蒙古惟舊土爾扈特等部之在新疆者,汗、王、公、扎薩克等卒,襲子不及歲,以前皆由已歿汗、王、公等之妻或母署印。有鹽,有礦,地兼耕牧。佐領共五十四。

北路凡三旗,盟曰烏訥恩素珠克圖,在塔爾巴哈臺城東,當金山之西南霍博克薩里,東噶扎爾巴什諾爾,南戈壁,西察漢鄂博,北額爾齊斯河。渥巴錫族子策伯克多爾濟等,乾隆三十六年,從渥巴錫來歸,獻金削刀及色爾克斯馬。三十七年,入覲,封策伯克多爾濟扎薩克和碩布延圖親王,授其弟奇哩布扎薩克一等臺吉,轄右翼,賜牧霍博克薩里,為舊土爾扈特北路,以策伯克多爾濟領之,授盟長。四十年,授奇哩布弟阿克薩哈勒扎薩克一等臺吉,轄左翼。四十三年,策伯克多爾濟卒,奇哩布襲,銷右翼印。五十年,授策伯克多爾濟之子公品級一等臺吉恭格車棱扎薩克,詔轄其父屬眾,別鑄右翼扎薩克印賜之。五十七年,封輔國公。道光二年,卒。子多爾濟那木扎勒降襲公品級扎薩克一等臺吉。

同治四年,塔城回變,親王策林拉布坦以調兵遲延,為參贊大臣錫霖劾革其爵,以捐輸復之。九年,奎昌等立塔爾巴哈臺新界鄂博,奏飭親王策林拉布坦、圖普伸克什克、扎薩克喇扎爾巴達爾隨時留意偵察,舊界亦有割棄。十二年十月,回匪竄擾是部薩巴爾山地方,劫掠牲畜衣物,烏素圖等三臺逃散。十二月,參贊大臣英廉奏匪已遠竄,飭策林拉布坦等妥為安插被難蒙民,一面將原設七臺照舊安設。尋論設臺站之勞,予黃韁。

宣統元年,以阿爾泰烏梁海復在是部薩里山陰度冬,提每年租馬十成之一給是部三旗作水草之租。是部金礦頗著名,地雜耕牧。有佐領十四。

東路凡二旗,跨濟爾哈朗河。東奎屯河,接甘肅綏來,南南山,西庫爾喀喇烏蘇,北戈壁。渥巴錫族弟巴木巴爾等從渥巴錫來歸。乾隆三十七年,入覲熱河,封扎薩克多羅畢錫埒勒圖郡王,弟奇布騰固山依特格勒貝子,盟名亦曰烏訥恩素珠克圖。初隸庫爾喀喇烏蘇大臣,統侖,統受伊犁將軍節制。同治末,俄人以北路舊土爾扈特取所屬哈薩克馬駝,執是部貝子普爾普噶丹為質,尋釋之。光緒初,給撫恤銀一萬兩。十一年,設庫爾喀喇烏蘇同知兼理事銜,釐是部民、蒙交涉事件。清末,襲郡王者帕勒塔嘗請出洋,又入貴胄學堂,以本旗事為伊犁將軍廣福劾,議處。是部共有佐領七。

西路一旗,當天山之北精河東岸。東精河屯田,南哈什山陰,西托霍木圖臺,北喀喇塔拉額西柯淖爾。渥巴錫族叔父默們圖從渥巴錫來歸。乾隆三十七年,入覲熱河,封扎薩克濟爾噶朗貝勒,賜牧精河,受伊犁將軍節制。咸豐十年,貝勒鄂齊爾以捐餉予雙眼花翎。光緒初,以被擾,予撫恤銀一萬兩。十三年,設精河同知兼理事銜,釐是部民、蒙交涉事。有佐領四。

新土爾扈特,在科布多西南,當金山南烏隆古河之東。東新和碩特,南胡圖斯山,西與北均阿爾泰烏梁海,東南扎哈沁。

土爾扈特翁罕十四世孫舍棱率諸昆弟附牧伊犁,為準噶爾屬臺吉。大軍征準噶爾,獲達瓦齊,阿睦爾撒納等以叛相次誅滅,舍棱獨抗不降,竄匿庫庫烏蘇、喀喇塔拉境。乾隆二十三年,詔定邊將軍成袞扎布等剿之。舍棱奔俄羅斯,我軍追及之於勒布什河源,舍棱乃詭約降,計戕我副都統唐喀祿,馳逾喀喇瑪嶺,歸額濟勒土爾扈特游牧。三十六年,復誘其汗渥巴錫來踞伊犁,抵他木哈,知內備固,計無所出,不得已,隨渥巴錫歸順。詔宥舍棱罪。三十七年,與從子沙喇扣肯入覲熱河,封舍棱多羅弼里克圖郡王,沙喇扣肯烏察喇勒圖貝子,均授扎薩克。舍棱所部曰左翼旗,沙喇扣肯曰右翼旗,定盟名曰青色特啟勒圖,舍棱充盟長,沙喇扣肯副之。四十八年,詔世襲罔替,隸科布多參贊大臣。

道光六年,回疆軍興,是部輸馬駝助軍。咸豐三年,是部王、貝子等請捐助軍需,溫旨卻之。

同治三年,徵是部兵援古城等城,以散潰,撤之。六年,於是部之布倫托海地方設辦事大臣,以李云麟為之。七年五月,布倫托海兵民潰變,李云麟走青格里河。諭福濟、錫綸前往查辦明瑤、棍噶扎拉參,曉諭解散。七月,布倫托海變民竄烏龍古河。九月,以棍噶扎拉參挑噶爾為喇嘛成軍,諭福濟等督率進剿布倫托海變民,撥部庫銀十萬兩解科布多,為布倫托海剿匪及賑濟難民之用。調福濟為布倫托海辦事大臣。十月,以守科布多城出力,予是部郡王凌扎棟魯布親王銜。十二月,以是部仍屬科布多管轄。八年二月,以哈薩克圍殺布倫托海變民,命是部郡王凌扎棟魯布進剿。四月,福濟遷烏里雅蘇臺將軍,文碩代之。七月,布倫托海變民傷俄國卡兵,棍噶扎拉參營於克林河,諭福濟等疾籌進剿,飭知遵行。是月棍噶扎拉參剿變民於和博克托裏,勝之。八月,棍噶扎拉參復布倫托海,變民降,收撫之,賊首張匊等伏誅。諭福濟等籌給布倫托海難民口食。九月,命塔城額魯特暫安舊居,阿爾泰山俗眾居青格里河。十月,徙布倫托海人眾於阿爾泰山,予布倫托海在防之索倫及綠營官兵銀兩。十一月,裁新設布倫托海辦事大臣,撤回旗、綠官兵,命索倫、額魯特領隊大臣及棍噶扎拉參應辦事宜統歸科布多參贊大臣經理,改派奎昌辦布倫托海與俄分界事宜。

十二年九月,肅州回匪竄是部貝子旗布拉噶河一帶,科布多參贊大臣托倫布等調回駐察罕淖爾之黑龍江馬隊暨蒙古馬隊,分赴布拉噶河防剿。十一月,烏魯木齊領隊大臣錫綸奏:「七月十六日,率所募民勇自阿爾泰山南移營烏龍古河南岸,聞東路布爾根河一帶有警,科布多屬之扎哈沁及和碩特、土爾扈特邊界皆被擾,阿爾泰附近之烏梁海臺站逃散,匪由和碩特、土爾扈特等喇嘛營子西竄至青格里河。」十二月,錫綸奏:「回匪擾及烏梁海部落,臣帶民勇民團追匪至噶扎爾巴爾淖爾,匪已由薩勒布爾山南竄沙山子,即由山北取道布凌河,疾馳至霍博克河上游之庫克辛倉,探得匪在河下游之科科墨頓林木中扎營五座,於夜分潛師進薄賊壘,擊潰賊三營,又取後一營,匪眾敗遁,尋由阿雅爾淖爾竄綏來縣之大小拐,回瑪那斯。」科布多幫辦大臣保英奏:「十月十九日,親率馬隊由吉慶淖爾西行,二十七日抵土爾扈特之青格里河。賊竄布倫托海,經錫綸進剿,斬獲甚多。匪已西竄,臣將官兵駐青格里河,檄飭烏梁海、土爾扈特、和碩特、扎哈沁速將軍臺移回原處安設。」其後烏魯木齊、瑪納斯諸城克復,是部始息警。

光緒九年,劃科城中、俄界幫辦大臣額爾慶額安插歸中國之哈薩克,以奎峒山左右暨哈巴河源諸山為夏季游牧,以阿拉別克河東暨果里子克河、哈巴河、阿拉克臺為冬季游牧。實皆是部地。二十九年,瑞洵奏創修布倫托海渠工,開辦屯田,給土爾扈王旗、貝子旗借用駝只幫價銀,飭扎哈沁、土爾扈特、烏梁海左右翼擇水草較好地,從扎哈沁沙扎蓋臺起,至布倫托海止,安設十三臺。二十九年閏五月,錄科布多所屬各旗保護俄商遺棄貨物有裨大局之勞,予土爾扈特正盟長扎薩克郡王密錫克棟古魯布紫韁,副盟長扎薩克貝子瑪克蘇爾扎布雙眼花翎。三十二年十二月,劃科布多、阿爾泰分轄之界,以是部二旗及新和碩特一旗、阿爾泰烏梁海七旗均隸阿爾泰。

是部地兼耕牧,有金礦。布爾津河通輪船。共有佐領三。

近是部者,有哈弼察克新和碩特。乾隆三十六年,和碩特臺吉巴雅爾拉瑚族蒙袞率屬來歸,原附新土爾扈特貝子沙喇扣肯之旗。詔予一等臺吉,給半佐領,令其附居。五十七年,移杜爾伯特近處哈密察克游牧。嘉慶元年,科布多參贊大臣奏蒙袞妻察彥率子布彥克什克詣言生齒日繁,求給扎薩克印,不食俸。道光六年,回疆軍興,後至咸豐初,是部皆偕杜爾伯特諸部捐馬駝、捐餉助軍。同治末,回匪北竄,是部與新土爾扈特同被擾。署伊犁將軍榮全以商論伊犁事,自科布多西行,是部設臺供支。光緒二十九年,錄庚子舉辦防團保護俄貨之勞,予扎薩克臺吉布彥克什克鎮國公銜。三十三年正月,卒,以子達木鼎第得恩襲。初有出缺請旨之例,實亦世襲。牧地東扎哈沁,南與西皆新土爾扈特,北阿爾泰烏梁海。有佐領一。

和碩特部,在新疆焉耆府北。東烏沙克塔爾,南開都河,西小珠勒都斯,北察罕通格山。舊為四衛拉特之一,系出元太祖弟哈布圖哈薩爾。有博貝密爾咱者,始稱汗。子哈尼諾顏洪果爾嗣之,有子六,牧青海、西套、伊犁諸境。詳青海厄魯特部傳。其第三子昆都倫烏巴什,第四子圖魯拜琥,裔蕃衍。圖魯拜琥號顧實汗,其裔或稱青海厄魯特,設扎薩克二十有一;或稱阿拉善厄魯特,設扎薩克一;或隸察哈爾旗,設爵三,皆不著。和碩特部昆都倫烏巴什,號都爾格齊諾顏,子十六:長邁瑪達賴烏巴什,次烏巴什琿臺吉,次多爾濟,次額爾克岱青鄂克綽特布,次第巴卓哩克圖,次噶布楚諾顏,次蒙固,次青巴圖爾,次伊納克巴圖爾,次伊勒察克,次賽巴克,次哈喇庫濟,次羅卜藏達什,次塔爾巴,次色棱,次朋素克。今和碩特設扎薩克四,皆多爾濟及額爾克岱青鄂克綽特布裔。

崇德七年,昆都倫烏巴什遣索諾木從達賴喇嘛使貢駝馬,賜布幣及朝鮮貢物。順治八年,貢所產馬及黑狐皮。九年,復貢駝馬。嗣數遣使至。康熙十六年,邁瑪達賴烏巴什子丹津琿臺吉遣達爾漢宰桑入貢。二十一年,復遣杭勒岱等至,諸昆弟遣使從,凡百餘人。二十四年,定四衛拉特貢例,使入關以二百人為額,諭所部知之。詳杜爾伯特部傳。

時準噶爾稍強,和碩特族懼其威,咸奉令。後噶爾丹亂定,顧實汗諸子姓游牧青海者咸內附。噶爾丹從子策妄阿喇布坦偪和碩特族與同處,表請青海復舊業如噶爾丹時,將陰謀為己屬。上燭其奸,諭責之,令遣和碩特歸舊牧,勿私據,不從。有羅卜藏車凌者,多爾濟曾孫也,策妄阿喇布坦以女妻之。雍正八年,靖邊大將軍傅爾丹屯科布多,將擊準噶爾。或告曰:「噶爾丹策凌以兵萬授羅卜藏車凌,遣御哈薩克,設汛阿里馬圖沙拉伯勒境。羅卜藏車凌棄之,率戶三千餘由噶斯走青海,將內附。噶爾丹策凌遣宰桑烏喇特巴哈曼集等追之,為所敗。復遣喀喇沁宰桑都噶爾往襲,不之及也。」傅爾丹以聞,詔副都統達鼐:「偵防噶斯路。俟羅卜藏車凌降,遣入覲,以兵監從眾,置內汛,勿墮詭降計。」久之,羅卜藏車凌不至。

乾隆二十年,大軍征達瓦齊,抵伊犁。有善披嶺集賽之得木齊蘇克都爾格齊霍什哈及古裡特鄂拓克之得木齊和通喀喇博羅莽鼐、伊什克特咱瑪博勒等,告舊為羅卜藏車凌屬,獻籍六百餘戶。羅卜藏車凌子曰諾爾布敦多克,游牧額琳哈畢爾噶,遣長子鄂齊爾馳降。定北將軍班第遣招其族,臺吉三濟特聞之,獻籍三百戶。

丹津琿臺吉子曰阿喇布坦,有子二:長噶爾丹敦多布,生沙克都爾曼濟;次敦多布車凌,生明噶特。達瓦齊善沙克都爾曼濟,倚任之。小策凌敦多卜孫訥默庫濟爾噶爾與達瓦齊構兵,沙克都爾曼濟擊之,殲其孥。班第等至,達瓦齊竄格登,沙克都爾曼濟乃降。有班珠爾者,顧實汗裔也,與輝特阿睦爾撒納異父同母,陰比之。前避達瓦齊亂來歸,授多羅郡王。詔俟厄魯特定,將以為和碩特汗。時從大軍抵伊犁,私奪諾爾布敦多克、沙克都爾曼濟諸臺吉屬產。班第禁之,乃稍戢。尋定入覲次,以沙克都爾曼濟及班珠爾列初班,三濟特、鄂齊爾次之。阿睦爾撒納阻其行,詭稱沙克都爾曼濟將叛迎達瓦齊,請以班珠爾屯特穆爾圖諾爾護降眾,班第斥詞妄。班珠爾詭入覲,赴塔密爾牧,取阿睦爾撒納孥,謀偕遁,參贊大臣阿蘭泰擒之。沙克都爾曼濟入覲避暑山莊,上御澹泊敬誠殿受朝,詔封和碩特汗,授盟長,諭董所屬勤養教,圖生聚。三濟特、鄂齊爾繼至,詔授三濟特扎薩克一等臺吉,鄂齊爾閒散一等臺吉,遣歸牧。

定西將軍策楞將以大兵剿阿睦爾撒納,詔沙克都爾曼濟往會,甫就道,諜者以阿睦爾撒納據伊犁告。諭遣親信宰桑馳諭所部備兵,勿為逆煽,而以身從大軍擊賊。班珠爾械至禁獄所,請遺三濟特、鄂齊爾書,令和碩特眾分剿阿逆。三濟特既得書,言諾爾布敦多克、沙克都爾曼濟皆鄰牧,且族臺吉瑪尼巴圖、巴蘇泰、瑪賚烏巴什、弩庫特圖魯孟克、阿穆爾弩斯海、薩望等皆無異志,當以書遺之。鄂齊爾稱原歸告父共剿逆。而我副將軍薩拉爾集伊犁宰桑等定議,約諾爾布敦多克及沙克都爾曼濟子圖捫以兵至博囉塔拉、布爾哈蘇臺、闥勒奇嶺剿阿睦爾撒納。諾爾布敦多克、圖捫各遣使至巴里坤告故,諾爾布敦多克表曰:「臣父羅卜藏車凌,前噶爾丹策凌時謀內附,不獲間。大軍征達瓦齊,臣族班珠爾倚阿睦爾撒納奪臣屬,臣原奮志剿賊。」上嘉其誠,詔封公爵,以班珠爾所奪給之。班珠爾尋伏誅。

二十一年,諾爾布敦多克來歸。薩拉爾等既定謀,阿睦爾撒納偵知之,先備。諾爾布敦多克以兵擊諸伊犁之諾羅斯哈濟拜甡,不勝,偕薩拉爾間道行,由珠勒都斯至巴里坤。時沙克都爾曼濟抵策楞軍,詔令遺書其子圖捫,以兵護牧。書未達,明噶特附阿睦爾撒納叛,脅所部眾。圖捫不之從,挈戚屬抵珠勒都斯,請內徙,上憫之,詔封多羅貝勒,賜銀千兩,賞雙眼孔雀翎,諭由額琳哈畢爾噶往會沙克都爾曼濟。有圖什墨勒厄爾哲者,從大軍剿阿睦爾撒納,中道強取諾爾布敦多克屬,詔責之,察所取以歸。

諾爾布敦多克及子鄂齊爾尋相繼卒,詔以鄂齊爾弟博爾和津襲公爵,諭曰:「諾爾布敦多克舊牧與哈薩克接壤,恐或掠之。若欲徙歸額琳哈畢爾噶,惟其便。」沙克都爾曼濟攜子圖捫及博爾和津等由珠勒都斯至巴里坤,乞屯牧近地。副都統雅爾哈善以聞,諭曰:「沙克都爾曼濟以舊牧乏生計,跋涉遠至,殊堪憫惻。準噶爾頻年不靖,諸部生計維艱。然使臺吉等各收其屬,安處游牧,以耕畜為業,善自謀生,不數年間,可復舊業。今沙克都爾曼濟等雖暫處巴里坤,究非故土,難以久處。又喀爾喀附近之和碩特、杜爾伯特、輝特等,俱將遣歸舊牧,且諭令各安生業,嚴戢盜賊。沙克都爾曼濟等自宜仍歸舊牧,但甫從遠道至,遽令之歸,不免困頓,可令暫處巴里坤附近地,賞給糧米如戶口數。」復遣使諭沙克都爾曼濟及綽囉斯汗噶爾藏多爾濟、輝特汗巴雅爾曰:「爾等自入覲歸牧後,遵朕諭旨,約束所屬,守分安居,已逾一載,甚勞遠念。今特遣官存問,並令齎賜食物佩飾,以示優眷。逆賊阿睦爾撒納現竄匿哈薩克,茍延殘喘。朕遣官兵征剿經年,時屆寒冬,暫行撤還。第逆賊狡詐百出,儻遣人赴爾等游牧,詭計煽惑,爾等即行擒獻。至沙克都爾曼濟奏請游牧巴里坤附近地,已諭酌賜口糧,俟明春復賞給耔種,耕耨廋集額卜齊布拉克地,秋收後遣歸舊牧。爾等其善自謀生,永享升平之福。」沙克都爾曼濟尋獻所部盜馬者請論罪,諭曰:「厄魯特劫奪成風,不可不嚴加懲創。爾等擒獲竊賊,解送內地,甚屬恭順。嗣後可自治之。」復以博爾和津幼不更事,諭沙克都爾曼濟留心護視,並令其族摩羅及宰桑新登等暫理牧務。

既而諸衛拉特復不靖,巴雅爾詭稱沙克都爾曼濟掠所部牧,將以兵襲巴里坤。噶爾藏多爾濟及兄子扎納噶爾布叛擾邊境,有普爾普者,以其主沙克都爾曼濟私通扎納噶爾布告,詔雅爾哈善察之。時沙克都爾曼濟設汛哨內防禦,遣諜赴巴里坤偵大軍狀,子圖捫死,不以告。雅爾哈善召之,稱病不至,疑果叛,宵抵其營殲之,斬眾四千餘,察獲博爾和津。奏請安置地,詔徙京師,停襲公爵。沙克都爾曼濟弟桑濟竄徙額爾齊斯境,掠奉使杜爾伯特之侍衛佛保駝馬,佛保奮擊之,乃逸。杜爾伯特汗車凌遣親王車凌烏巴什等追剿,桑濟走死,和碩特叛黨始靖。

而其族多爾濟之裔恭格等,有偕土爾扈特部游牧俄羅斯之額濟勒河境者,三十六年,從土爾扈特汗渥巴錫自俄羅斯來歸。尋入覲,詔封恭格為土謝圖貝勒,族叔父雅蘭阿穆爾聆貴為貝子,授族弟諾海及巴雅爾拉瑚一等臺吉,均為扎薩克,各編一旗,賜盟名曰巴圖色特啟勒圖,餘悉如土爾扈特例。三十七年,賜牧珠勒都斯。四十年,設正副盟長各一。嘉慶二年,恭格從子博騰特克卒,無嗣。十一月,以所管佐領分給貝子鄂齊爾二,扎薩克臺吉齊業齊三,烏爾圖那遜一,除其爵。

道光六年,回疆軍興,徵是部兵協剿。敗回匪於阿克蘇之渾巴什河,予緞疋、銀兩及翎頂各有差。自是回疆有事,皆偕土爾扈特兵應徵調,統隸於伊犁將軍。

同治三年,回亂,是部被蹂躪,戶口散失大半,中路左旗扎薩克臺吉喇什德勒克率餘眾避居博爾圖山中,竭力保守。光緒三年,劉錦棠收復托克遜,喇什德勒克謁劉錦棠。八月,進兵,以後隨同官軍馳驅,於地勢險夷,賊情虛實,水道深淺,具陳實狀。師逾開都河,遂遷其部於河東。欽差大臣左宗棠請獎,疏入,予花翎。先是中旗貝子多爾那齊那木札勒、右旗扎薩克洞魯布旺扎勒皆避出,至是始歸所牧。是部佐領尚呈左宗棠,請以其兩旗人眾隸喇什德勒克。事尋寢。八年,設喀喇沙爾直隸同知兼理事銜,釐是部蒙、民交涉事。二十二年,甘肅回匪竄出關,伊犁將軍長庚檄是部貝子棍布扎普派扎薩克臺吉貢噶那木扎勒統兵駐都木達塔什哈地方,扼博斯騰淖爾通羅布淖爾之徑,事定,撤歸。

其地出產同舊土爾扈特南部落。佐領共十一。


\end{pinyinscope}