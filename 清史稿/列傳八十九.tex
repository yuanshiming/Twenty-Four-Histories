\article{列傳八十九}

\begin{pinyinscope}
徐本汪由敦子承霈來保劉綸子躍云劉統勛子墉孫鐶之

徐本,字立人,浙江錢塘人,尚書潮子。本,康熙五十七年進士,改庶吉士,授編修。雍正五年,提督貴州學政,授贊善,遷侍讀。七年,擢貴州按察使。八年,調江蘇,遷湖北布政使。十年,擢安慶巡撫。奏定比緝盜賊章程,竊案責府州,盜案責臬司。案多而未獲,巡撫親提。比立限,定勸懲。上嘉之。十一年,疏言:「雲、貴、廣西改流土司安置內地,例十人給官房五楹,地五十畝。安慶置二十一人,地遠在來安。請變價別購,俾耕以食。」又疏言:「州縣徵糧,例由府道封櫃,請改州縣自封。完糧十截串票改仍用三連由票,零戶銀以下以十錢當一分。」又疏言:「壽州濱淮,盜聚族而居,假捕魚為業,每出劫掠,已次第捕治,令漁船編甲。孫、平、焦、鄧諸姓設族正,有盜不時舉發。」皆下部議行。

召授左都御史。十二年,遷工部尚書、協辦大學士。浙江衢州民王益善邪教惑眾,命本會總督程元章按治,請改設衢州總兵、金衢嚴巡道以下官,並更定營制,下部議行。十三年五月,命同寶親王,果親王,大學士鄂爾泰、張廷玉等辦理苗疆事務。高宗即位,命在辦理軍機處行走,調刑部尚書。尋命協辦總理事務。

乾隆元年,授東閣大學士兼禮部尚書,充世宗實錄總裁。二年,直南書房。以協辦總理事務,予拖沙喇哈番世職。三年,授辦理軍機大臣。四年,加太子太保。七年,兼管戶部尚書。九年六月,以病乞休,加太子太傅致仕。遣御前侍衛永興齎賜御用衣冠、內府文綺貂皮,上親臨其第慰問賜詩。命其子侍講學士以烜送歸里,在籍食俸。明年,上念本歸將一載,復賜詩。十二年,本卒,加少傅,發白金千治喪。浙江巡撫顧琮往祭,謚文穆。上南巡,所經郡縣遣祭舊臣,禮部奏請未及本,上特命遣祭。祀京師賢良祠。

以烜,進士,官至禮部侍郎。

汪由敦,字師茗,浙江錢塘人,原籍安徽休寧。雍正二年進士,選庶吉士。遭父喪,以篡修明史,命在館守制。喪終,三遷內閣學士,直上書房。乾隆二年,廷臣妄傳除目,為言官執奏,語連由敦,未得旨,由敦具疏辨。上詰由敦何以先知,足見有為之耳目者,其人必不謹。左授侍讀學士。累遷工部尚書,調刑部,兼署左都御史。十一年,命在軍機處行走。十四年,金川平,加太子少師。是歲命協辦大學士。由敦出大學士張廷玉門,其直軍機處,廷玉薦也。時軍機處諸大臣,鄂爾泰已卒,廷玉為班首,而訥親被上眷,日入承旨,出令由敦屬草,慮不當上意,輒令易稿,至三四不已,傅恆為不平。及訥親誅,傅恆自金川還朝,引諸大臣共承旨以為常。廷玉致仕將歸,以世宗遺詔許配享太廟,乞上一言為券,謝恩未親至。傳旨詰責,傅恆與由敦承旨,由敦免冠叩首,言廷玉蒙恩體恤,乞終始矜全,若明旨詰責,則廷玉罪無可逭。次日,廷玉早入朝,上責由敦漏言,徇師生私恩,不顧公議。解協辦大學士,並罷尚書,仍在尚書任贖罪。十五年,命復任。

上閱永定河工,令由敦同大學士傅恆、總督方觀承會勘南岸建壩,請於張仙務、雙營葺舊壩二,馬家鋪及冰窖以東增新壩亦二,如所議。四川學政硃荃以匿喪黷賄得罪,由敦所薦舉,吏議奪職。上以由敦謹慎,長於學問,命降授兵部侍郎。俄,永定河堤決,復命赴固安監塞口。有請別開新河者,由敦主仍濬舊河,亦如所議。十六年,調戶部侍郎。命同大學士高斌勘天津等處河工,請濬永定河下流,疏王慶坨引河,增鳳河堤壩,培東岸堤障東澱。十七年,授工部尚書。十九年,加太子太傅,兼刑部尚書。二十年,準噶爾平,軍機大臣得議敘。二十一年,調工部尚書。二十二年,授吏部尚書。二十三年,卒,上親臨賜奠,贈太子太師,謚文端。

由敦篤內行,記誦尤淹博,文章典重有體。內直幾三十年,以恭謹受上知。乾隆間,大臣初入直軍機處,上以日所制詩用丹筆作草,或口授令移錄,謂之「詩片」。久無誤,乃使撰擬諭旨。由敦能彊識,當上意。上出謁陵及巡幸必從,入承旨,耳受心識,出即傳寫,不遺一字。其卒也,諭稱其「老誠端恪,敏慎安詳,學問淵深,文辭雅正」,並賦詩悼之。又以由敦善書,命館臣排次上石,曰時晴齋法帖。上賦懷舊詩,列五詞臣中,稱其書比張照雲。

子承沆、承霈、承澍。

承霈,字春農。由敦既卒,喪終,承霈以賜祭葬入謝。傅恆為言承霈書類由敦,授兵部主事,充軍機處章京。累遷郎中,除福建邵武知府。時母年八十,請軍機大臣為陳情,留京供職,復補戶部郎中。三十六年,師討小金川,上命戶部侍郎桂林出督餉,以承霈從。三十七年,阿爾泰、宋元俊劾桂林以金與土酋贖所掠軍士,辭連承霈,命逮治。俄,事白,仍以郎中充軍機處章京。累遷工部右侍郎。甘肅冒賑事發,部議凡在甘肅納捐監生,應禁革毋許應試,及自別途出身。承霈奏人數甚多,乞開自新之路,令納金如例,許考試及自別途出身,得旨俞允。四十年,上校射,承霈連發中的,賞花翎。調戶部右侍郎。五十四年,坐監臨順天鄉試失察,左遷通政使。累遷復至侍郎。嘉慶五年,授左都御史,遷兵部尚書,兼領順天府尹。六年,永定河水溢,上命治賑,得旨獎敘。七年,上將幸木蘭,承霈請罷停圍,不許。尋改左都御史,署兵部尚書。北城盜發,上責承霈不稱職,以二品冠服致仕。十年,卒,詔視尚書例議恤。

來保,字學圃,喜塔臘氏,滿洲正白旗人。初隸內務府。康熙中,自庫使授侍衛,再奪職。五十七年,復授三等侍衛。雍正初,擢內務府總管。坐內務府披甲裁額,眾閧廉親王允禩第,來保等奏不實,復奪職。起景陵掌關防郎中,再遷復為內務府總管,署工部尚書。疏言:「滿洲騎射較優,沿邊古北口諸處提鎮以下,請兼用滿洲,資控制。」從之。乾隆元年十二月,大學士管浙江總督嵇曾筠、江蘇巡撫邵基疏請停辦戊午銅運,下部議。來保奏:「積欠數盈六百萬,應停辦一年,以清舊款。但己未以後,仍招商採買,行之數年,積欠復多,又當停辦。請敕部並下各直省督撫曉諭,聽商具貲本出洋採買,不必先給價值,隨到即收,不拘多寡,但不得克扣抑勒,重滋商累。」總理王大臣議覆允行。

二年六月,上以運河水淺,糧船至臨清以北,尤多阻滯,由於衛河上游各渠口居民私洩過多。敕直隸、河南督撫等照前河臣靳輔題準定例,稽查嚴禁。來保奏言:「水淺運阻,查禁不得不嚴。但衛水發源河南,至臨清五百餘里。沿河居民不知幾千萬家,待溉之地不知幾千百頃。今秋成在望,已非灌溉之期,所慮者有司奉行過當。後雖運河未至淺阻,而一入五月,渠口盡行堵塞,坐使有用之利置之無用,恐不無廢時失業者,不稱仁育萬民之意。當使漕運不致淺阻,民田亦得灌溉,或暫禁於淺阻之年,而不禁於深通之歲。應令督撫、河道諸臣悉心調劑,以期兩便。」疏入,上命侍郎趙殿最、侍衛安寧會同督撫查勘,請於漕船將抵臨清,視運河水盈縮,定渠閘啟閉。十二月,授工部尚書,兼議政大臣。四年,病,請解任,上不許。十二月,授內大臣,賜紫禁城內騎馬。五年,調刑部尚書。

上以來保奉職勤,命改隸正白旗滿洲,所立佐領準世襲。六月,御史沈世楓奏來保誠愨有餘,習練不足,不勝刑部繁要之任。諭曰:「來保人實可信,然世楓所言,頗中其病。儻因此自知省惕,則心志虛公,而才識亦將日進。此聞過而喜,所以稱賢也。」九年,命如奉天按將軍額洛圖侵餉納賄狀,論如律。十年,調禮部尚書,加太子太保,授領侍衛內大臣。尋授吏部尚書,協辦大學士。十二月,授武英殿大學士。十三年九月,命為軍機大臣。十四年,金川凱旋,進太子太傅,兼管兵部、刑部事。十五年三月,來保年七十,上制詩賚之。十六年,兼管吏部事。二十五年,來保年八十,復賜禦制詩。二十六年,兼管禮部事。二十九年,卒,年八十四,贈太保,祀賢良祠,謚文端。四十四年,禦制懷舊詩,列五閣臣中。

來保能知人。舒赫德官烏里雅蘇臺將軍,疏請徙阿睦爾撒納眷屬於邊。上以其傷遠人心,震怒,遣使封刀斬之。來保爭甚力,以為才可大用。上亦悔,第曰:「已降旨!」來保曰:「即上有恩命,臣子成麟善騎,遣追前使還。」上允之。歸召成麟,使齎詔追前使還。成麟日夜馳三百餘里,先前使三日到,舒赫德賴以免。來保善相馬,上嘗為相馬歌賜之。

劉綸,字蜰涵,江蘇武進人。少俊穎,六歲,能綴文,長工為古文辭。乾隆元年,以廩生舉博學鴻詞,試第一,授編修。預修世宗實錄,遷侍講,進太常寺少卿。四遷,擢內閣學士。十二年,扈蹕木蘭,奏秋郊大獵、哨鹿二賦,稱旨。十四年,直南書房,授禮部侍郎,調工部。十五年,命軍機處行走。十六年,土默特貝子哈木噶巴雅斯朗圖不按原議年限驅種地流民,命綸偕侍讀學士麒麟保往勘。六月,疏言:「出口民價典旗地,應遵原議三年、五年限外撤還原主。其領地耕種為佃戶,受雇力作為傭工,皆浮寄謀生,初無占地意,應許力耕餬口。至領地墾荒,積累辛勤,始得成熟,不同價典,年滿先還原主。所需自種地有贏,仍給種以償前勞。木頭城、三座塔居人稠密,許照常居住。設三座塔巡檢一,資彈壓。」詔從其議。父憂歸。服闋,十八年,除戶部侍郎。

十九年,兼順天府尹。故事,順天府公牘,治中、通判不署名。綸請以錢穀屬治中,獄訟屬通判,先署牘呈尹可否之。大軍西征準噶爾,師行,役車供偫,壹切辦治無誤。二十年,準噶爾平,予獎敘。浙江按察使富勒渾劾巡撫鄂樂舜授意布政使同德勒派商銀,命綸如浙江偕兩江總督尹繼善等會訊。二十一年,覆奏鄂樂舜受銀屬實,擬絞候;同德未知情;富勒渾誣劾,擬杖流。上以富勒渾參款已實,不應議罪,責綸等失當。部議奪官,有旨從寬留任,罷直軍機處。二十二年,命仍入直。二十四年六月,奏薊州、寶坻等縣蝻子萌動,州縣官事繁,督捕未能周遍,飭千總、外委同佐雜分捕,參將偕監司巡察勤惰,報可。進左都御史。二十五年,偕侍郎伊祿順赴西安勘將軍嵩阿禮剋兵糧、勒餽送等款,得實,論如律。二十六年,進兵部尚書。二十八年,調戶部,協辦大學士,加太子太保。三十年,母憂歸。甫除喪,詔起吏部尚書,仍協辦大學士。三十六年,授文淵閣大學士,兼工部尚書。三十八年,卒,命皇子臨其喪,贈太子太傅,祀賢良祠,謚文定。

綸性至孝,親喪三年不御酒肉。直軍機處十年,與大學士劉統勛同輔政,有「南劉東劉」之稱。器度端凝,不見有喜慍色。出入殿門,進止有恆處。自工部侍郎歸,買宅數楹。後服官二十年,未嘗益一椽半甓。衣履垢敝不改作,朝必盛服,曰:「不敢褻朝章也!」侍郎王昶充軍機處章京,嘗嚴冬有急奏具草,夜半詣綸,綸起燃燭,操筆點定。寒甚,呼家人具酒脯,而廚傳已空,僅得白棗十數枚侑酒。其清儉類此。校士尤矜慎,嘗曰:「衡文始難在取,繼難在去。文佳劣相近,一去取間於我甚易,獨不為士子計乎?」較量分寸,輒至夜分不伴奏倦。文法六朝,根柢漢、魏;於詩喜明高啟,謂能入唐人門閾。

子躍雲,字服先。乾隆三十一年進士及第,授編修。累遷禮部侍郎。六十年,充會試副考官,以校閱失當下吏議,左遷奉天府府丞,罷歸。嘉慶四年,召為大理寺少卿,遷工部侍郎。上御門,躍雲誤班未至,左遷內閣學士。復授兵部侍郎。休致,卒。殿試例糊名,躍雲對策,高宗親置上第,喜曰:「此劉綸子,不意朕竟得之!」及視學江西,有清名。高宗意鄉用,以忤和珅,主會試,坐浮言,黜。仁宗召起,老矣,終不竟其用。子逢祿,見儒林傳。

劉統勛,字延清,山東諸城人。父棨,官四川布政使。統勛,雍正二年進士,選庶吉士,授編修。先後直南書房、上書房,四遷至詹事。乾隆元年,擢內閣學士。命從大學士嵇曾筠赴浙江學習海塘工程。二年,授刑部侍郎,留浙江。三年,還朝。四年,母憂歸。六年,授刑部侍郎。服闋,詣京師。

擢左都御史。疏言:「大學士張廷玉歷事三朝,遭逢極盛,然晚節當慎,責備恆多。竊聞輿論,動云『張、姚二姓占半部縉紳』,張氏登仕版者,有張廷璐等十九人,姚氏與張氏世婚,仕宦者姚孔鋹等十人。二姓本桐城巨族,其得官或自科目薦舉,或起襲廕議敘,日增月益。今未能遽議裁汰,惟稍抑其遷除之路,使之戒滿引嫌,即所以保全而造就之也。請自今三年內,非特旨擢用,概停升轉。」又言:「尚書公訥親年未強仕,綜理吏、戶兩部。典宿衛,贊中樞,兼以出納王言,時蒙召對。屬官奔走恐後,同僚亦爭避其鋒。部中議覆事件,或展轉駁詰,或過目不留,出一言而勢在必行,定一而限逾積日,殆非懷謙集益之道。請加訓示,俾知省改。其所司事,或量行裁減,免曠廢之虞。」兩疏入,上諭曰:「朕思張廷玉、訥親若果擅作威福,劉統勛必不敢為此奏。今既有此奏,則二臣並無聲勢能箝制僚寀可知,此國家之祥也。大臣任大責重,原不能免人指摘。聞過則喜,古人所尚。若有幾微芥蒂於胸臆間,則非大臣之度矣。大學士張廷玉親族甚眾,因而登仕籍者亦多。今一經察議,人知謹飭,轉於廷玉有益。訥親為尚書,固不當模棱推諉,但治事或有未協,朕時加教誨,誡令毋自滿足。今見此奏,益當自勉。至職掌太多,如有可減,侯朕裁定。」尋命以統勛疏宣示廷臣。

命勘海塘。十一年,署漕運總督。還京。十三年,命同大學士高斌按山東賑務,並勘河道。時運河盛漲,統勛請濬聊城引河,分運河水注海。德州哨馬營、東平戴村二壩,皆改令低,沂州江楓口二壩,俟秋後培高,俾水有所洩。遷工部尚書,兼翰林院掌院學士,改刑部尚書。十七年,命軍機處行走。十八年,以江南邵伯湖減水二閘及高郵車邏壩決,命偕署尚書策楞往按。合疏言河員虧帑誤工,詔奪河督高斌、協辦河務巡撫張師載職,窮治侵帑諸吏。九月,銅山小店汛河決,統勛疏論同知李焞、守備張賓呈報稽誤。上以焞、賓平日侵帑,聞且窮治,自知罪重,河漲任其沖決,立命誅之,並縶斌、師載令視行刑。統勛駐銅山督塞河,十二月,工成。統勛偕策楞疏陳稽察工料諸事,詔如所議行。大學士陳世倌疏言黃河入海,套櫃增多,致壅塞,命統勛往勘。統勛疏言:「海口舊在雲梯關,今海退河淤,增長百餘里,櫃套均在七曲港上,河流無所阻遏。」上又命清察江南河工未結諸案,統勛疏言未結款一百一十一萬有奇,請定限核報。又以河道總督顧琮請於祥符、滎澤諸縣建壩,並濬引河,命統勛往勘。統勛議擇地培堤壩,引河上無來源,中經沙地,易淤墊,當罷,上從之。

十九年,加太子太傅。五月,命協辦陜甘總督,賜孔雀翎。時方用兵準噶爾,統勛請自神木至巴里坤設站一百二十五,並裁度易馬、運糧諸事,命如所議速行。二十年,廷議駐兵巴里坤、哈密,命察勘。統勛至巴里坤,阿睦爾撒納叛,攻伊犁,伊犁將軍班第死事,未得報。定西將軍永常自木壘引師退,統勛疏請還守哈密。上責其附和永常,置班第於不問,命並永常奪職,逮治。其子墉亦奪職,與在京諸子皆下刑部獄,籍其家。旋上怒解,諭:「統勛所司者糧餉馬駝,軍行進止,將軍責也。設令模棱之人緘默不言,轉可不至獲罪。是其言雖謬,心尚可原。永常尚不知死綏,何怪於統勛?統勛在漢大臣中尚奮往任事,從寬免罪,發往軍營交班第等令治軍需贖罪。」釋其諸子。

二十一年六月,授刑部尚書。尋命勘銅山縣孫家集漫工,解總河富勒赫任,即命統勛暫攝。是冬,工竟。二十二年,命赴徐州督修近城石壩,加太子太保。二十三年,調吏部尚書。二十四年,命協辦大學士。二十六年,拜東閣大學士,兼管禮部、兵部。八月,偕協辦大學士兆惠查勘河南楊橋漫工。十二月,工竟。二十七年,上南巡,復命偕兆惠勘高、寶河湖入江路,疏請開引河,擇地築閘壩。上諭謂:「所議甚合朕意。」又以直隸景州被水,命勘德州運河,疏請移吏董理四女寺、哨馬營兩引河,毋使淤閼。二十八年,充上書房總師傅,兼管刑部,教習庶吉士。三十三年,命往江南酌定清口疏濬事宜。三十四年,復勘疏運河。

三十八年十一月,卒。是日夜漏盡,入朝,至東華門外,輿微側,啟帷則已瞑。上聞,遣尚書福隆安齎藥馳視,已無及。贈太傅,祀賢良祠,謚文正。上臨其喪,見其儉素,為之慟。回蹕至乾清門,流涕謂諸臣曰:「朕失一股肱!」既而曰:「如統勛乃不愧真宰相。」

統勛歲出按事,如廣東按糧驛道明福違禁折收,如雲南按總督恆文、巡撫郭一裕假上貢抑屬吏賤值巿金,如山西按布政使蔣洲抑屬吏補虧帑,如陜西按西安將軍都賚侵餉,如歸化城按將軍保德等侵帑,如蘇州按布政使蘇崇阿誤論書吏侵帑,如江西按巡撫阿思哈受賕,皆論如律。其視楊橋漫工也,河吏以芻茭不給為辭,月餘事未集。統勛微行,見大小車載芻茭凡數百輛,皆弛裝困臥。有泣者,問之,則主者索賄未遂,置而不收也。即令縛主者至,數其罪,將斬之。巡撫以下為固請,乃杖而荷校以徇,薪芻一夕收立盡。逾月工遂竟。方金川用兵,統勛屢議撤兵,及木果木軍覆,上方駐熱河,統勛留京治事,天暑甚,以兼上書房總師傅,檢視諸皇子日課。廷寄急召,比入對,上曰:「昨軍報至,木果木軍覆,溫福死綏。朕煩懣無計,用兵乎,抑撤兵乎?」統勛對曰:「日前兵可撤,今則斷不可撤。」復問誰可任者,統勛頓首曰:「臣料阿桂必能了此事。」上曰:「朕正欲專任阿桂,特召卿決之。卿意與合,事必濟矣。」即日令還京師。戶部疏論諸行省州縣倉庫多空缺,上欲盡罷州縣吏不職者,而以筆帖式等官代之。召統勛諭意,且曰:「朕思之三日矣,汝意云何?」統勛默不言。上詰責,統勛徐曰:「聖聰思至三日,臣昏耄,誠不敢遽對,容退而熟審之。」翌日入對,頓首言曰:「州縣治百姓者也,當使身為百姓者為之。」語未竟,上曰:「然。」事遂寢。上為懷舊詩,列五閣臣中,稱其「神敏剛勁,終身不失其正」云。子二:墉、堪。

墉,字崇如,乾隆十六年進士,自編修再遷侍講。二十年,統勛得罪,並奪墉官下獄,事解,賞編修,督安徽學政。疏請州縣約束貢監,責令察優劣。督江蘇學政,疏言府縣吏自瞻顧,畏刁民,畏生監,兼畏吏胥,闒冘怠玩。上嘉其知政體,飭兩江總督尹繼善等淬厲除舊習。授山西太原知府,擢冀寧道。以官知府時失察僚屬侵帑,發軍臺效力。逾年釋還,命在修書處行走。旋推統勛恩,命仍以知府用,授江蘇江寧知府,有清名。再遷陜西按察使。丁父憂,服闋,授內閣學士,直南書房。遷戶部、吏部侍郎。授湖南巡撫,遷左都御史,仍直南書房。命偕尚書和珅如山東按巡撫國泰貪縱狀,得實,授工部尚書,充上書房總師傅。署直隸總督,授協辦大學士。五十四年,以諸皇子師傅久不入書房,降為侍郎銜。尋授內閣學士,三遷吏部尚書。嘉慶二年,授體仁閣大學士。命偕尚書慶桂如山東讞獄,並按行河決,疏請寬濬下游。四年,加太子少保。疏陳漕政,僉丁不慎,途中盜米,致有鑿舟自沉,或鬻及檣舵,舟存而不可用,請飭各行省僉丁宜求殷實,皆如所議行。九年,卒,年八十五,贈太子太保,祀賢良祠,謚文清。墉工書,有名於時。

鐶之,統勛次子堪之子也。乾隆四十四年進士。自檢討累遷至戶部尚書,兼領順天府府尹。嘉慶二十二年,上自熱河還京師,鐶之入見。上以順天府奏事稀、捕教匪不時得詰,鐶之不能對,但言方旱災不敢急捕賊。上又問賑災當設粥廠幾所、需米若干,鐶之又不能對。上降旨責其玩愒,命以侍郎候補。復累遷吏部尚書,加太子少保。道光元年,卒,謚文恭。

論曰:明內閣主旨擬,承旨撰敕,其在唐、宋,特知制誥之職。以王命所出入,密勿獻替,遂號為宰相。軍機處制與相類。世謂大學士非兼軍機處,不得為真宰相。勝此任者,非以其慎密,則以其通敏。慎密則不洩,通敏則不滯,不滯不洩,樞機之責盡矣。本,世宗舊臣,由敦、來保、綸、統勛次第入直。由敦左遷而未罷直,統勛罷而復入,尤以決疑定計見契於高宗,許為有古大臣風,亮哉!


\end{pinyinscope}