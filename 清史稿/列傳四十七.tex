\article{列傳四十七}

\begin{pinyinscope}
姚啟聖子儀吳興祚施瑯硃天貴

姚啟聖,字熙止,浙江會稽人。少任俠自喜。明季為諸生。順治初,師定江南,游通州,為土豪所侮,乃詣軍前乞自效。檄署通州知州,執土豪杖殺之,棄官歸。郊行,遇二卒掠女子,故與好語,奪其刀殺之,還女子其家。去附族人,籍隸鑲紅旗漢軍。舉康熙二年八旗鄉試第一,授廣東香山知縣。前政負課數萬,系獄,啟聖牒大府,悉為代償。尋以擅開海禁,被劾奪官。

十三年,耿精忠反,兵入浙江境,陷溫州傍近及臺、處諸屬縣。聖祖命康親王傑書統師進討,啟聖與子儀募健兒數百詣軍,以策干王。檄署諸暨知縣,剿平紫瑯山土寇。十四年,以王薦,超擢溫處道僉事。從都統拉哈達克松陽、宣平二縣。十五年,偕副都統沃申、總兵陳世凱等剿賊石塘,焚其木城,斬獲甚眾,乘勝復云和。

先是,精忠以書招鄭錦,錦至復拒之,將士多為內應,錦遂取泉、漳二府,據廈門。精忠與戰,復屢敗。啟聖又使儀破精忠將曾養性於溫州。十月,師入仙霞關,趨福建,精忠降。擢啟聖福建布政使,率兵討錦。吳三桂將韓大任驍勇善戰,世稱小淮陰者也,自贛入汀,謀與錦合。啟聖說之降,簡其部卒,得死士三千人,以為親軍。十六年,從康親王復邵武、興化,盡取漳、泉地。錦遁歸廈門。總督郎廷佐奏啟聖與子儀屢著戰功,贍軍購馬,具甲胄弓矢,糜白金五萬,皆出私財,詔嘉獎。

十七年,錦遣其將劉國軒、吳淑、阿佑等復犯漳、泉,海澄公黃芳世、都統穆赫林、提督段應舉等與戰,敗績,遂陷海澄、長泰、同安、惠安、平和諸縣。詔擢啟聖福建總督,條上機宜,「請調福寧鎮兵助攻泉州,調衢州、贛州、潮州三鎮兵助攻漳州,復設漳浦、同安二總兵,增督標兵五千。通省經制兵萬八千,申明臨陣賞罰,禁廝役占兵額」。下議政王大臣議,衢、贛、潮三路皆重地,未便徵發,既增督標兵,毋庸復廣通省兵額,餘皆從其議。七月,偕海澄公黃芳度自永福進克平和、漳平。國軒等解泉州圍,進逼漳州,壁於蜈蚣嶺。啟聖率壯士鍾寶、張黑子等出戰,將軍賚塔、都統沃申等夾擊,連破賊寨,斬其將鄭英、劉正璽等十餘人,國軒遁海澄,乘勝復長泰。敘功,進正一品。九月,復遣儀率兵攻同安,敵棄城遁,斬其將林欽等。尋偕副都統吉勒塔布、提督楊捷等進攻海澄,敗國軒於江東橋,又敗之於潮溝。

十八年,國軒與淑、佑等踞郭塘、歐溪頭,欲斷江東橋以犯長泰。啟聖偕賚塔、捷及巡撫吳興祚等邀擊,大敗之,先後招降所置吏四百餘、兵一萬四千有奇。國軒等復率萬餘人謀奪榴山寨,啟聖偕賚塔及副都統石調聲擊敗之,至太平橋、潮溝,斬千餘級。十九年,會賚塔等攻海澄。時提督萬正色先克海壇,啟聖及總兵趙得壽、黃大來等分兵七路並進,破十九寨;別遣將渡海,拔金門、廈門,降錦將硃天貴等,錦退保澎湖,盡復所陷郡縣:進兵部尚書、太子太保。

二十年,左都御史徐元文劾「啟聖疏請借司庫銀十二萬,經營取息,侵占民利;題報軍前捐銀十五萬,皆剋軍餉朘民膏而得。閩民極困,啟聖不能存撫,拆毀民居,築園亭水閣,日役千人,舞女歌兒充牣房闥;又強取長泰戴氏女為妾。海壇進師,力為阻撓,及克廈門,又言當直取臺灣。始欲養寇,繼欲窮兵。吳興祚、萬正色平海奏功,啟聖慚妒,妄謂正色與錦將硃天貴有約,讓海壇而去。險詐欺誣,乞敕部嚴議」。上令啟聖覆奏,啟聖言:「臣於康熙十七年十月進兵至鳳凰山,因一時投誠者多,犒賞不繼,與撫臣吳興祚議外省貿易,頗有微利,前督臣李率泰、經略洪承疇嘗借帑為之,遂冒昧上疏,未蒙俞允。臣自入仕,京師未有產業,而軍前捐銀十五萬有奇者,香山罷官後,貿易七年,得積微貲,並臣浙江祖產變價,及親朋借貸,經年累月而後有此。臣於十七年七月至省,見總督官廨為耿精忠屯兵毀傷傾圮,因捐貲修整,日役不過數十,柵外員役私舍,令其自行撤除。至臣妾皆有子女,年已老大,並無歌兒舞女,強取戴氏女,尤無其事。十八年十一月,臣密陳進剿機宜,請水陸五道進兵,並未阻撓。至得廈門即攻臺灣,先於十八年九月預陳,亦非屆時發議。撫臣、提臣拜疏出師,平賊首功已定,臣何所容其慚妒。硃天貴應撫投誠,天貴言之而臣始知之。臣任三閩三職,雖無妒功之心,實有溺職之咎。乞敕部嚴議,別簡賢能。」疏入,報聞。二十一年,敘克海澄、金門、廈門功,授世職拜他喇布勒哈番兼拖沙喇哈番。

方鄭錦屢入寇,徙濱海居民入內地,俾絕接濟、避侵掠,下令越界者罪至死,民多蕩析。及禁旅班師,驅系良民子女北行,啟聖白王嚴禁。復捐貲贖歸難民二萬餘人,並請開海界、復民業,聽降卒墾荒,民困漸蘇。及錦死,子克塽仍其爵,稱延平王,凡事皆決之國軒等。啟聖令知府卞永譽、張仲舉專理海疆,多以金帛間其黨與。克塽乃遣使齎書,原稱臣入貢,不薙發登岸,如琉球、高麗例。啟聖以聞,上不許,趣水師提督施瑯進徵。

二十二年六月,瑯進攻臺灣,取澎湖。啟聖駐廈門督饋運,以大舟載金、糸曾、貨、米至軍,大賚降卒,遣之歸,臺民果攜貳。復設間使克塽與國軒互相猜,眾莫為用。瑯遂定臺灣,克塽、國軒等皆降。語具瑯傳。啟聖還福州,未幾,疽發背,卒。明年,部議以啟聖修繕船舶、軍械,浮冒帑金四萬七千有奇,應追繳,上念其勞,免之。

子儀,膂力絕人,雄偉與父埒。初以捐納知縣從征,累戰有功。康親王檄署游擊。議敘,內擢郎中。上以儀有才略,且自陳原以武職自效,改都督僉事,以總兵用。歷狼山、杭州、沅州、鶴慶諸鎮總兵,鑲紅旗漢軍副都統。卒,賜祭葬。

鍾寶,少業屠,流為盜。啟聖令香山,招之降。後啟聖征福建,寶偕同降者二十人隸麾下,每戰輒當前,所向有功。累進秩都督僉事。啟聖卒,遂歸。後數年,部議注官,授潼關參將,遷靖邊協副將。卒。寶撫兵民有恩,稱為鍾佛子。

韓大任,降後入覲,聖祖以其為三桂將,留為內務府包衣參領。二十九年,從佟國綱征噶爾丹,次烏闌布通,伏發,國綱歿於陣。大任驚曰:「吾聞臨陣失帥,兵家大罪。吾以叛逆之黨,蒙恩不死。今豈可坐必死之律,復對獄吏乎?」因馳入賊陣,手刃數十人,死之。

吳興祚,字伯成,漢軍正紅旗人,原籍浙江山陰。父執忠,客禮親王代善幕,授頭等護衛。興祚自貢生授江西萍鄉知縣。金聲桓叛,郡縣多被寇,萍鄉以有備獨完。坐事罷。旋以守御功復官,授山西大寧知縣,遷山東沂州知州。白蓮教嘯聚為患,興祚開諭散遣之。復坐事降補江南無錫知縣。縣吏虧庫帑,更數政未得償,官罷不能去。興祚至,為請豁除,其當償者出私財代輸。清丈通縣田,編號繪圖,因田徵賦。飛詭隱匿,皆不得行。縣徭役未均,最煩苦者為圖六。興祚以入官田徵租雇役,民害乃除。歲饑,為粥食餓者。八旗兵駐防蘇州,興祚請於領兵固山,單騎彈壓。兵或取民雞,立笞之,皆奉約束。塘溢,兵不得渡,立竹於塘旁,懸燈以為識,騎行如坦途。

康熙十三年,遷行人,仍留知縣事,用漕運總督帥顏保薦,超擢福建按察使。有硃統琛者,號明裔,耿精忠私署敉遠將軍,及精忠降,自稱宜春王,據貴溪為亂,與福建錯壤。興祚輕騎至光澤,撫其將陳龍等,遣降將陽自歸為內應,令龍導師入,其將馮珩等縛統琛,率兵三千以降。

十七年,擢巡撫。時鄭錦踞臺灣,遣其驍將劉國軒等陷漳、泉屬縣,復圍泉州。興祚率標兵自興化赴援,至仙游,錦將黃球等率二千人結土寇萬餘屯白鴿嶺。興祚分兵三道,自當中路,與戰,自辰至酉,相持不即退。興祚遣兵自間道奪白鴿嶺關口,斬級六百,墮岸溺水死者甚眾,寇乃潰走,追敗之於嶺頭灣,復永春、德化二縣。國軒自泉州走入海,以巨艦數百出沒赤嶼、黃崎諸處。興祚遣總兵林賢等統水師出海,分三路夾攻,焚敵艦六十餘,俘斬六千有奇。疏報捷,並言:「海逆逼犯漳、泉,大軍由陸路進發,跋涉疲難。臣前捐募水兵,一戰破賊,但兵力稍薄,未易輕取廈門。若得水師二萬,再添造戰船,可直搗巢穴,掃蕩鯨波。」詔允行。

十八年,國軒率兵二千至郭塘、歐溪頭,欲斷江東橋以犯長泰,興祚與都統吉勒塔布、總督姚啟聖會師擊走之。興祚遣驛傳道王國泰等招降錦將蔡沖琱、林忠等三百八十五人,兵萬二千五百,拔難民千二百,得舟六十七。敘前後功,進秩正一品。

十九年,疏言:「鄭錦盤踞廈門,沿海生靈受其荼毒。臣去冬新造戰船,水師提督萬正色分配將士,自閩安出大洋操練。俟舊存船艘修葺完整,江南砲手齊集,即相機進取廈門。」二月,正色師進海壇,興祚自泉州會寧海將軍拉哈達、總兵王英等赴同安,攻克汭洲、潯尾諸隘。渡海,拉哈達出中路,英右,興祚左,奮戰,敵大潰,遂克廈門。時正色已取海壇,降錦將硃天貴等,復遣兵取金門,餘眾悉竄臺灣。捷聞,詔嘉獎,下部優敘。興祚因請留澳民防守,蠲荒田租糧,減關課。正色亦請於海澄、廈門分兵駐守。上命侍郎溫岱赴福建會議。溫岱至,啟聖與言正色復海壇,與天貴先有約乃進兵,無殺賊攻克事。溫岱還京師,兵部據其言,議興祚冒功,上命仍議敘,予世職拜他喇布勒哈番兼拖沙喇哈番。

二十年,擢兩廣總督。興祚上官,疏言尚之信在廣東橫徵苛斂,民受其害數十年。因舉鹽埠、渡稅、稅總店、漁課諸害,悉奏罷之。自遷界令下,廣東沿海居民多失業,興祚疏請展界,恣民捕採耕種。上遣尚書杜臻、內閣學士石柱會興祚巡歷規畫,兵民皆得所。又言潮州海汛遼闊,商民往來貿易,恐宵小潛蹤,應令澄海協達濠營水汛官兵船只改歸南澳水師鎮統轄,與碣石鎮互相聯絡,巡防外海島嶼,詔並允行。二十四年,疏請於廣東、廣西二省設爐鼓鑄,給事中錢晉錫、御史王君詔疏劾興祚鼓鑄浮冒,下吏議,當鐫秩,命以副都統用。

三十一年,授歸化城右翼漢軍副都統,復坐事鐫秩。三十五年,上征噶爾丹,命自呼坦和碩至寧夏安十三塘,興祚原效力坐沙克舒爾塘,未幾,復原秩。三十六年,卒。

興祚為政持大體,除煩苛,卒後遠近戴之。歷官之地,並籥祀名宦。

施瑯,字琢公,福建晉江人。初為明總兵鄭芝龍部下左沖鋒。順治三年,師定福建,瑯從芝龍降。從征廣東,戡定順德、東莞、三水、新寧諸縣。芝龍歸京師,其子成功竄踞海島,招瑯,不從。成功執瑯,並縶其家屬。瑯以計得脫,父大宣、弟顯及子侄皆為成功所殺。十三年,從定遠大將軍世子濟度擊敗成功於福州,授同安副將。十六年,成功據臺灣,就擢瑯同安總兵。

康熙元年,遷水師提督。時成功已死,其子錦率眾欲犯海澄,瑯遣守備汪明等率舟師御之海門,斬其將林維,獲戰船、軍械。未幾,靖南王耿繼茂、總督李率泰等攻克廈門,敵驚潰,瑯募荷蘭國水兵,以夾板船要擊,斬級千餘,乘勝取浯嶼、金門二島。敘功,加右都督。三年,加靖海將軍。

七年,瑯密陳錦負嵎海上,宜急攻之。召詣京師,上詢方略,瑯言:「賊兵不滿數萬,戰船不過數百,錦智勇俱無。若先取澎湖以扼其吭,賊勢立絀;倘復負固,則重師泊臺灣港口,而別以奇兵分襲南路打狗港及北路文港海翁堀。賊分則力薄,合則勢蹙,臺灣計日可平。」事下部議,寢其奏。因裁水師提督,授瑯內大臣,隸鑲黃旗漢軍。

二十年,錦死,子克塽幼,諸將劉國軒、馮錫範用事。內閣學士李光地奏臺灣可取狀,因薦瑯習海上事,上復授瑯福建水師提督,加太子少保,諭相機進取。瑯至軍,疏言:「賊船久泊澎湖,悉力固守。冬春之際,颶風時發,我舟驟難過洋。臣今練習水師,又遣間諜通臣舊時部曲,使為內應。俟風便,可獲全勝。」二十一年,給事中孫蕙疏言宜緩征臺灣。七月,彗星見,戶部尚書梁清標復以為言,詔暫緩進剿。瑯疏言:「臣已簡水師精兵二萬、戰船三百,足破滅海賊。請趣督撫治糧餉,但遇風利,即可進行,並請調陸路官兵協剿。」詔從之。

二十二年六月,瑯自桐山攻克花嶼、貓嶼、草嶼,乘南風進泊八罩。國軒踞澎湖,緣岸築短墻,置腰銃,環二十餘里為壁壘。瑯遣游擊藍理以鳥船進攻,敵舟乘潮四合。瑯乘樓船突入賊陣,流矢傷目,血溢於帕,督戰不少卻,總兵吳英繼之,斬級三千,克虎井、桶盤二嶼。旋以百船分列東西,遣總兵陳蟒、魏明、董義、康玉率兵東指雞籠峪、四角山,西指牛心灣,分賊勢。瑯自督五十六船分八隊,以八十船繼後,揚帆直進。敵悉眾拒戰,總兵林賢、硃天貴先入陣,天貴戰死。將士奮勇衷擊,自辰至申,焚敵艦百餘,溺死無算,遂取澎湖,國軒遁歸臺灣。克塽大驚,遣使詣軍前乞降,瑯疏陳,上許之。八月,瑯統兵入鹿耳門,至臺灣。克塽率屬薙發,迎於水次,繳延平王金印。臺灣平,自海道報捷。疏至,正中秋,上賦詩旌瑯功,復授靖海將軍,封靖海侯,世襲罔替,賜御用袍及諸服物。瑯疏辭侯封,乞得如內大臣例賜花翎,部議謂非例,上命毋辭,並如其請賜花翎。

遣侍郎蘇拜至福建,與督撫及瑯議善後事。有言宜遷其人、棄其地者,瑯疏言:「明季設澎水標於金門,出汛至澎湖而止。臺灣原屬化外,土番雜處,未入版圖。然其時中國之民潛往生聚,已不下萬人。鄭芝龍為海寇,據為巢穴。及崇禎元年,芝龍就撫,借與紅毛為互市之所。紅毛聯結土番,招納內地民,漸作邊患。至順治十八年,鄭成功盤踞其地,糾集亡命,荼毒海疆。傳及其孫克塽,積數十年。一旦納土歸命,善後之計,尤宜周詳。若棄其地、遷其人,以有限之船,渡無限之民,非閱數年,難以報竣。倘渡載不盡,竄匿山谷,所謂藉寇兵而齎盜糧也。且此地原為紅毛所有,乘隙復踞,必竊窺內地,鼓惑人心。重以夾板船之精堅,海外無敵,沿海諸省,斷難安然無虞。至時復勤師遠征,恐未易見效。如僅守澎湖,則孤懸汪洋之中,土地單薄,遠隔金門、廈門,豈不受制於彼,而能一朝居哉?臣思海氛既靖,汰內地溢設之官兵,分防兩處:臺灣設總兵一、水師副將一、陸營參將二、兵八千;澎湖設水師副將一、兵二千。初無添兵增餉之費,已足固守。其總兵、副將、參、游等官,定以二三年轉升內地。其地正賦雜糧,暫行蠲豁。駐兵現給全餉,三年後開徵濟用,即不盡資內地轉輸。蓋籌天下形勢,必期萬全,臺灣雖在外島,關四省要害,斷不可棄。並繪圖以進。」疏入,下議政王大臣等議,仍未決。上召詢廷臣,大學士李霨奏應如瑯請。尋蘇拜等疏亦用瑯議,並設縣三、府一、巡道一,上命允行。

瑯又疏請克塽納土歸誠,應攜族屬與劉國軒、馮錫範及明裔硃桓等俱詣京師,詔授克塽公銜,國軒、錫範伯銜,俱隸上三旗,餘職官及桓等於近省安插墾荒。復疏請申嚴海禁,稽核貿易商船,命如所議。

二十七年,入覲,溫旨慰勞,賞賚優渥。上諭瑯曰:「爾前為內大臣十有三年,當時尚有輕爾者。惟朕深知爾,待爾甚厚。後三逆平定,惟海寇潛據臺灣為福建害,欲除此寇,非爾不可。朕特加擢用,爾能不負任使,舉六十年難靖之寇,殄滅無餘。或有言爾恃功驕傲,朕令爾來京。又有言當留勿遣者,朕思寇亂之際,尚用爾勿疑,況天下已平,反疑而勿遣耶?今命爾復任,宜益加敬慎,以保功名。」瑯奏謝,言:「臣年力已衰,懼勿勝封疆之重。」上曰:「將尚智不尚力。朕用爾亦智耳,豈在手足之力哉?」命還任。三十五年,卒於官,年七十六,贈太子少傅,賜祭葬,謚襄壯。

瑯治軍嚴整,通陣法。尤善水戰,諳海中風候。將出師,值光地請急歸,問瑯曰:「眾皆言南風不利,今乃刻六月出師,何也?」瑯曰:「北風日夜猛。今攻澎湖,未能一戰克。風起舟散,將何以戰?夏至前後二十餘日,風微,夜尤靜,可聚泊大洋。觀釁而動,不過七日,舉之必矣。即偶有颶風,此則天意,非人慮所及。鄭氏將劉國軒最驍,以他將守澎湖,雖敗,彼必再戰。今以國軒守,敗則膽落,臺灣可不戰而下。」及戰,雲起東南,國軒望見,謂颶作,喜甚。俄,雷聲殷殷,國軒推案起曰:「天命矣!今且敗。」人謂瑯必報父仇,將致毒於鄭氏。瑯曰:「絕島新附,一有誅戮,恐人情反側。吾所以銜恤茹痛者,為國事重,不敢顧私也。」子世綸、世驃,自有傳;世範,襲爵。

硃天貴,福建莆田人。初為鄭錦將。康熙十九年,師下海壇,以所部二萬人、舟三百來降,授平陽總兵。瑯攻澎湖,天貴以師會。國軒拒戰,天貴以十二舟薄敵壘,焚其舟,殺傷甚眾,戰益力,俄,中飛砲僕舟中,猶大呼殺賊,遂卒,贈太子少保,謚忠壯。

論曰:臺灣平,瑯專其功。然啟聖、興祚經營規畫,戡定諸郡縣。及金、廈既下,鄭氏僅有臺澎,遂聚而殲。先事之勞,何可泯也?及瑯出師,啟聖、興祚欲與同進,瑯遽疏言未奉督撫同進之命。上命啟聖同瑯進取,止興祚毋行。既克,啟聖告捷疏後瑯至,賞不及,鬱鬱發病卒。功名之際,有難言之矣。大敵在前,將帥內相競,審擇堅任,一戰而克。非聖祖善馭群材,曷能有此哉?


\end{pinyinscope}