\article{列傳三百一}

\begin{pinyinscope}
土司三

○雲南

雲南古滇國。自越巂蠻夷任貴自領太守,漢光武即授以印綬,不以內地官守例之。若爨、若蒙,皆以本土大姓,就官累世,為一方長。元封梁王於滇,與大理之段分治。明破梁王,滅大理,就土官而統馭之,分宣慰使、宣撫使、安撫使、正副長官司、土府、土州以治之。

清順治十七年,平西王吳三桂定雲南,明永明王走緬甸,以沐府舊地封三桂,永鎮雲南。康熙十四年,撤籓,三桂遂叛。三桂死,其孫世璠襲。二十一年,克之,世璠自殺,雲南大定。

雍正初,改土歸流之議起。四年夏,先革東川土目,即進圖烏蒙。時烏蒙土府祿萬鍾、鎮雄土府隴慶侯皆年少,兵權皆握於其叔祿鼎坤、隴聯星。鄂爾泰令總兵劉起元屯東川,招降祿鼎坤。惟祿萬鍾制於漢奸,約鎮雄兵三千攻鼎坤於魯甸,鄂爾泰遣游擊哈元生敗之;又檄其相仇之阿底土兵共搗烏蒙,連破關隘,賊遂敗走鎮雄。鄂爾泰復招降隴聯星,而鼎坤亦以兵三千攻鎮雄之脅,兩酋皆遁四川,於是兩土府旬日平。以烏蒙設府,鎮雄設州,又設鎮於烏蒙,控制三屬,由四川改隸雲南,以一事權。其東川法戛土目祿天祐、烏蒙米貼土目祿永孝,尚各據巢患邊。六年春,遣兵破擒法戛,又遣副將郭壽域以兵三百捕米貼賊,逃渡小金沙江,糾四川沙馬司及建昌、涼山各夷惈數千潛回,襲陷官兵。鄂爾泰遣總兵張耀祖、參將哈元生三路搜討。詔四川建昌、永寧官兵聽鄂爾泰節制。於是自小金沙江外,沙馬、雷波、吞都、黃螂諸土司地,直抵建昌,袤千餘里,皆置營汛,形聯勢控,並擒雷波土司楊明義;而哈元生回軍復敗阿盧土司之眾數千,屯田東川,歲收二萬餘石,課礦歲萬金,資兵餉。軍甫定,祿鼎坤以功擢河南參將,怏怏失望。其子祿萬福乞回魯甸治產,見總兵劉起元軍律不肅,陰會其舊部謀變。時烏蒙商民萬計,有險可扼,且賊止標弩,無大砲,而劉起元惟媮餒賄和,賊遂陷鎮城,盡戕兵民,遍煽東川、鎮雄及四川涼山蠻數萬叛。鄂爾泰奏言:「臣用人僨事,請別簡大臣總督兩省,暫假臣提督,將兵討賊雪憤。」世宗慰留之。鄂爾泰調官兵萬餘,土兵半之,三路進攻。先令總兵魏翥國率兵二千,七日馳抵東川,得不陷;而魏翥國旋為祿鼎明刺傷,乃以官祿代翥國。烏蒙委總兵哈元生、副將徐成貴,鎮雄委參將韓勛。勛以兵四百扼奎鄉,敗賊四千,連破四寨。哈元生以千餘兵討烏蒙,先至得勝坡,遇賊二萬。其黑寡、暮末二渠皆萬人敵。黑寡持長槍,直犯元生,元生左格槍,右拔矢,應手殪之;又射殪暮末,即竿揭二首以進,賊奪氣。再戰再捷,進至倚那岡。賊數萬,連營十餘里。我兵三千、土兵千,夜設伏賊營左右,而嚴陣以待。黎明,賊數路來犯,不動。將偪陣,砲起,大呼奮擊,山後伏兵左右夾攻,賊大潰,盡破其八十餘營,獲甲械輜重山積。即日抵烏蒙,賊見元生旗,即反走,克三關,祿萬福兄弟、祿鼎坤均伏誅。

六年,鄂爾泰總督三省,其土州安於蕃、鎮沅土府刁澣,及赭樂長官土司、威遠州、廣南府各土目,先後劾黜。惟刁氏之族舍土目煽糾威遠黑惈復反,戕知府劉洪度。於是盡徙已革土司土目他省安置,並搜剿黨逆之威遠、新平諸惈,冒瘴突入,擒斬千計,而我將士亦患瘴死二百餘。又進剿瀾滄江內孟養、茶山土夷,即明王驥兵十二萬,大舉再徵,諸蠻驚謂「自古漢兵所未至者」也。鄂爾泰先檄車裏土兵截諸江外,官兵各持斧鍬開路,焚柵湮溝,連破險隘,直抵孟養,據蠻坡通餉道;其六茶山巢穴四十餘寨,乃用降夷鄉導,以賊攻賊,於是深入數千里,無險不搜。惟江外歸車裏土司,江內地全改流。升普洱為府;移沅江協副將駐之。於思茅、橄欖壩各設官戍兵,以扼蒙緬、老撾門戶。於是廣南府土同知、富州土知州,各原增歲糧二三千石,並捐建府、州城垣。孟連土司獻銀廠,怒江野夷輸皮幣,而老撾、景邁二國皆來貢象,緬甸震焉。乾隆三十四年,遷孟拱土司於關外。緬甸事詳見緬甸傳。

雲南府:

羅次縣

煉象關土巡檢,居煉象關大街。清順治十六年,土巡檢李文秀歸附,仍授舊職。傳至李東祚,乾隆五十年,改為從九品土官,世襲。

祿豐縣

南平關土巡檢,居土官村。清順治十六年,土巡檢李楚南歸附,仍授舊職。傳至李東來,乾隆五十年,改為從九品土官,世襲。

大理府:

趙州

定西嶺土巡檢,居定西嶺。清順治十六年,土巡檢李齊鬥歸附,仍授舊職。

雲南縣土縣丞,在縣城。清順治十六年,土知縣楊玉蘊子岳歸附,仍授土知縣世職。康熙六年,雲南縣改設流官知縣,其知縣改縣丞,世襲。

雲南縣土主簿,居土官村,離城十里。清順治十六年,土主簿張維歸附,仍授世職。

鄧川州

青索鼻土巡檢,在青索鼻。清順治十六年,土巡檢楊應鵬歸附,仍授舊職。傳至楊榮昌,乾隆五十年,改為從九品土官,世襲。

浪穹縣

浪穹縣土典史,在縣城。清順治十六年,土典史王鳳州歸附,仍授世職。

蒲陀崆土巡檢,在蒲陀崆,距縣城十五里。清順治十六年,土巡檢楊爭先歸附,仍授世職。

鳳羽鄉土巡檢,在鳳羽鄉,距縣城三十里。清順治十六年,土巡檢尹德明歸附,仍授世職。

上江嘴土巡檢,在上江嘴,距縣一百二十里。清順治十六年,土巡檢楊康國歸附,仍授世職。

下江嘴土巡檢,在下江嘴,距縣九十里。清順治十六年,土巡檢何應福歸附,仍授世職。

雲龍州

箭桿場土巡檢,居箭桿場。清順治十六年,土巡檢字題鳳歸附,仍授世職。舊屬鄧川州,康熙二年,改隸雲龍州。

十二關長官司,在府東三百里。清順治十六年,長官司李恬森歸附,仍授世職。

老窩土千總,居老窩。清順治十六年,土知州段德壽歸附,後裁。乾隆十二年,德壽孫維精剿秤戛夷賊有功,十七年,授土千總世職。道光元年,永北軍務,段克勛帶練擒賊,給五品頂戴。

六庫土千總,居六庫。其先段復健,明土知州段保十七世孫。清乾隆十二年,徵秤戛夷賊有功,十七年,授土千總世職。道光元年,永北軍務,段履仁帶練擒賊,給五品頂戴。

漕澗土把總,居漕澗。清順治十八年,左文燦以堵御功授土官長官司,子停襲。乾隆十二年,文燦曾孫左世英隨征秤戛夷賊有功,授土把總,世襲。

鄧川州土知州,清順治十六年,土知州阿尚夔歸附,仍授世職。曾孫遠,因縱賊殃民,雍正四年改流,安插江西。

臨安府:

納樓茶甸長官司,在府西南一百八十里。清順治十六年,長官司普率歸附,仍授世職。康熙四年,率附王祿叛,官兵討之,乞降,赦之,以子向化襲。

納更山土巡檢,距府東南二百八十里。清順治十六年,土巡檢龍天正歸附,仍授世職。

虧容甸長官司,在府西南一百四十里。清順治十六年,長官司孫大昌歸附,仍授副長官世職。

思陀鄉土舍,在府西南二百五十里。清順治十六年,長官司李秉忠歸附,仍授長官司、副長官世職。後絕,改土舍。康熙二十年,以李世元繼襲。

溪處長官司副長官,在府西南三百一十五里。清順治十六年,長官司恩忠歸附,仍授副長官世職。康熙四年,附祿昌賢叛,伏誅,改土舍。

瓦渣鄉長官司,在府西南二百四十里。清順治十六年,錢覺耀歸附,仍授副長官世職。康熙四年,通王祿叛,官兵擒斬之,職除,改土舍。

左能寨長官司,在府西南二百三十里。清順治十六年,吳應科歸附,以非滇志所載,下臨安府查核,稽其譜系,蓋應科為明蚌頗十一世孫,因改土舍,準襲。

落恐甸長官司,在府西南二百里。清順治十六年,明授副長官司陳玉歸附,因號紙無存,給便委土舍,仍準世襲。

阿邦鄉土舍,在府東南二百一十里。明授土守備。清順治十六年,土守備陶順祖歸附,守職如故。旋議土司不宜加武職,改土舍。

慢車鄉土舍,在府西南一百四十里。清順治間,元江土夷亂,漫車土目刀岡隨官軍協剿,授土舍世職。

稿吾卡土把總,在府東南二百八十里。清雍正間,納更土目龍在渭隨征元普逆夷有功,給土把總職銜。嘉慶二十二年,江外夷匪滋事,龍定國父子陣亡,奏準世襲土把總。

十五猛,縱橫四百餘里。明初為沐氏勛莊。清順治十七年,吳三桂請並云南荒田給與籓下壯丁耕種。康熙七年,奏旨圈撥。叛後,變價歸建水徵收。猛各設一掌寨,督辦錢糧。管有猛喇、猛丁、猛梭、猛賴、猛蚌、茨桶壩、五畝、五邦、者米、猛弄、馬龍、瓦遮、斗巖、阿土、水塘十五寨。

教化三部長官副長官。清順治十六年,副長官龍升歸附,仍以張長壽為名,許之,授世職。康熙四年,附王祿叛,誅之,以其地為開化府,設流官。

王弄山長官司副長官。清順治十六年,副長官王朔歸附,授世職。康熙四年,朔與祿昌賢叛,官兵討之,朔自焚死,以其地屬開化府。

阿迷州土知州,舊有土目李阿側。清康熙四年,從討王朔有功,授土知州世職。傳至李純,濫派橫徵,為惈夷所控。雍正四年,籍其產,安置江西,改流。

寧州土知州,清順治十六年,祿昌賢歸附,仍授世職。十七年,降州同。明年,以舉首梅道人等謀逆,復原職。康熙四年,以叛伏誅。

寧州土州判。清康熙十九年,滇有李者祿歸附,準世襲州判。後絕,停襲。

習峨縣土知縣。清順治十六年,祿益歸附,仍授世職。康熙四年,與祿昌賢等叛改流。

習峨縣土主簿。清順治十六年,王揚祖歸附,仍授世職。康熙四年,與祿昌賢等叛,伏誅,職除。

蒙自縣土縣丞。土知縣陸氏被黜,其土舍寧州祿重據土官村,溺於酒色,不能馭下。其目把李輔舜等叛歸沙源,源以兵乘之,遂破有土官村。沙定洲踞會城,令李輔舜子日芳竊據蒙自。定洲敗,日芳遂家於蒙。清康熙四年,日芳弟日森子世籓、世屏附寧州祿昌賢叛,總兵閻鎮破之。世籓遁,追斬之;世屏出降,免死,充大理軍。後吳三桂反,給世屏偽總兵劄。大師復滇,世屏持劄歸附,授蒙自縣土縣丞職,不準世襲。

楚雄府:

楚雄縣土縣丞。清順治十六年,土縣丞楊春盛歸附,仍授舊職。乾隆五十年,改為正八品土官,世襲。

鎮南州土州同,居本城。清順治十六年,土州同段光贊歸附,仍授世職。

鎮南州土州判,居鎮南州城東北。清順治十六年,土州判陳昌虞歸附,仍授世職。

阿雄關土巡檢,居鎮南州屬。清順治十六年,土巡檢者光祖歸附,仍授世職。

鎮南關土巡檢。清順治十六年,土巡檢楊繼祖歸附,仍授舊職,傳至楊文輝,乾隆五十年,改為從九品土官,世襲。

姚州土州同,居姚州西界彌興官莊。清順治十六年,土州同高顯錫歸附,仍授舊職。傳至高配忝,乾隆五十年,改為從六品土官,世襲。

廣通縣

回磴關土巡檢,居回磴關。清順治十六年,土巡檢楊忠藎歸附,仍授舊職。傳至楊怡,乾隆五十年,改為從九品土官,世襲。

沙矣舊土巡檢。清順治十六年,土巡檢蘇鑒歸附,仍授舊職。傳至蘇敬,乾隆五十年,改為從九品土官,世襲。

定遠縣土主簿,居本城。清順治十六年,土主簿李世卿歸附,仍授舊職。傳至李毓英,

乾隆五十年,改為正九品土官,世襲。

姚安府土同知。清順治十六年,土同知高映歸附,仍授世職。傳至李厚德,雍正三年,以不法革職,安置江南。

澂江府:

新興州土州判,居州南研和邑。清康熙十九年,復滇,土人王鳳授偽游擊,迎至廣西路投誠;隨征石門坎、馬別河、黃草壩皆有功,授土州判世職。

河陽縣安插土官。清順治初,土官刀韜歸附,止給劄,仍準世襲。沿至刀廷俊,裁革。

新興州

鐵爐關土巡檢。清順治十六年,土巡檢王先榮歸附,授世職。康熙四年,同王耀祖叛,削除。

廣南府:

廣南府土同知。清順治十六年,儂鵬歸附,授同知世職。傳至儂毓榮,乾隆三十一年,從征普洱、緬甸。三十七年,頒給土同知關防。子世昌,嘉慶二年從徵貴州仲苗,加銜一等,賞戴花翎,世襲。

富州土知州,在府東二百六十五里。清順治十六年,土知州沈昆巘歸附,仍授世職。康熙九年,頒給州印。後以罪黜,傳至沈肇乾。雍正八年,肇乾復以罪黜。

順寧府:

雲州

大猛麻土巡檢。清順治十六年,土巡檢俸新命歸附,仍授世職。

緬寧

猛猛土巡檢,明末奔竄,失其印信號紙,未能請襲。傳子紫芝,清康熙五十四年歸附,貢象,仍授世職,頒給鈐記。乾隆二十九年,改屬順寧府。

孟連宣撫司,在順寧府邊外南境,舊隸於永昌府。清康熙四十八年,刁派鼎貢象,歸附,授宣撫司世職。派鼎死,子刁派春年幼,叔祖刁派烈撫孤。有刁派猷謀殺派烈,奪印爭職,安插省城,另給宣撫司鈐記便委。傳至刁派新,因地處極邊,界連外域,定為經制宣撫司,頒給印信號紙。乾隆二十九年,改隸順寧府。

猛緬長官司,清乾隆十一年,歸流,改其地為緬寧,設流官通判駐其地。

曲靖府:

平彞縣土縣丞,居平彞縣竹園村。清順治初,土縣丞龍闊歸附,仍舊世襲。


\end{pinyinscope}