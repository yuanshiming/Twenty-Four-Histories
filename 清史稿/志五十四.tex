\article{志五十四}

\begin{pinyinscope}
地理二十六

△青海

青海:禹貢西戎之域。袤延二千餘里。至京師五千七十里。東及北界甘肅,西界西藏,南界四川。三代屬西羌。漢為張掖、武威、金城、隴西四郡之西塞外,蜀郡之北徼外,屬先零、燒當等諸羌地。王莽時,置西海郡。歷後漢、魏、晉,皆諸羌所居。東晉後,又為吐谷渾所據。隋平吐谷渾,置西海、河源等郡。隋末,吐谷渾復據之。唐龍朔三年,吐蕃滅吐谷渾,盡有其地。宋亦為吐蕃地。元為貴德州及吐蕃朵甘思等處,屬吐蕃等處宣慰司。明為西番地。正德四年,始為蒙古部酋所據。清初,有元太祖弟哈布圖哈薩爾之裔,號顧實汗,自西北侵有其地,遣使通貢。自分部眾為左右二境。左境東自西寧邊外棟科爾廟,西至嘉峪關邊外洮賚河,南自西寧邊外博羅充克克河北岸,北至涼州邊外西喇塔拉。右境東自棟科爾廟,西至噶斯池,南自松潘邊外漳臘嶺,北至博羅充克克河南岸。康熙三十七年,悉眾內附。雍正元年,臺吉札什巴圖爾子羅卜藏丹津誘眾犯邊,大軍討平之,越歲而定。三年一貢,分三班,九年一周。置互市於西寧日月山。開拓新邊,增設安西鎮於布隆吉爾,闢地千餘里。三年,編其部落為四,旗二十九,後又增置土司四十。設西寧辦事大臣以統轄之。廣千餘里,袤千餘里。面積二百四十萬方里。人十五萬口。北極高三十一度四十五分至三十八度三十分。京師偏西十四度三十分至十七度。東:阿木尼末倫山。東南:阿木尼塞爾泰山。西南:阿木尼那凌通布山。西北:阿木尼巴延尊崔山、阿木尼洞舒山;阿木尼天沁察罕山,其峰甚峻,無雪而白,故名;阿木尼兀善通布山。西:阿木尼巴爾布安山,其峰高險,色黑,故名。西北二百餘里,有阿木尼厄枯山,東北近甘、涼二州之邊,有阿木尼岡噶爾山,又名龍壽山。涼州邊外有阿木尼巴延哈拉山,又名大荒山。又阿木尼扣肯古爾板山,在黃河東岸哈爾吉山東,山有二峰獨高,積雪不消;其一為阿木尼麻禪母孫山,即大雪山也。番語稱祖為「阿木尼」。西海十三山,番俗皆分祭之,而以大雪山為最。凡環繞青海之濱者,亦有十三山,土人皆名烏爾圖,謂之「十三角」云。又南曠野中,有漢陀羅海山、西索克圖山、西南索克圖山,地多瘴氣。西南:烏克陀羅海山,高峰壁立。黃河西岸、青海西南,有固爾班伊瑪圖山,三山相接,皆名伊瑪圖,繞獨羅池。有蘇羅巴顏喀喇山,在伊瑪圖山東北,石崖色黑,多冷瘴,故名。南:黃河北岸有巴爾陀羅海岡。近青海南岸有巴漢哈圖嶺。巴漢哈圖嶺東,伊克哈圖嶺;其西南,察察嶺。察察嶺東,納布楚爾嶺。南少西,蘇羅嶺,即蘇羅巴顏喀喇山之東支。黃河西岸、蘇羅嶺東,登諾爾臺嶺。拖孫池東南,忒伯呼圖嶺。有海努克嶺。東北,布呼圖嶺。西,烏蘇圖搜吉嶺。青海西南,殷德爾碧柳圖嶺,相近有好來嶺。青海西南二百餘里,烏蘭布拉克嶺。西寧邊外,納拉薩拉嶺。其西,齊布秦爾嶺;相近有哈拉嶺,即拉喇山也。洮州衛邊外,有達爾濟嶺,即託禮嶺也。洮河發源西傾山之脊,嶺最高大,其上平坦,草木茂盛。東南有和爾河,源出納拉薩拉嶺,西北流,入青海。北少東,哈爾濟河,源出青海北岸哈爾濟山,東南流,入青海。北:伊克烏蘭和碩河,源出巴顏山,南流入青海。其西,巴漢烏蘭和碩河,南流入青海。東南:巴顏池,周四十餘里。西南:多羅池,周一百五十餘里。洮賚河亦作滔來、陶賴、討來,在肅州南,下流合張掖河,即古呼蠶水也。漢書地理志祿福縣,「呼蠶水出南羌中,東北至會水,入羌谷」。寰宇記:「呼蠶水一名潛水,俗謂之祿福河,西南自吐谷渾界流入。」一統志:「按今討來河發源州西南五百餘里番界中,有三派,最西曰討來河,其西又有輝土巴爾呼河,北流,與討來河合。又東北百餘里,南有巴哈、額濟納二河,分流而合,又北與討來河會為一,又東北流入邊。繞州南,至州東北,合西來之水,又東北出邊,過金塔寺,稍折而北,又轉東與張掖河合,又北入居延海。」布隆吉爾河,在今安西州北,即南籍端水也。一統志:「按輿圖及新志,今有蘇賴河,亦名布隆吉爾河,發源靖逆衛南山,曰昌馬河。北流轉而西,逕舊柳溝衛北,會十道溝水為蘇賴河。又西逕安西衛北,又西逕沙州衛西北,黨河自南來注之。又西北流,瀦為合拉池。其流長七百餘里。池方數十里,即古南籍端水也。今三衛屯田,俱藉此水灌溉。」塞爾騰海,在舊沙州衛西南,水出雪山之陰,西北流,瀦為澤,為青海要道。西爾噶拉金河,即黨河,在沙州衛西,古氐置水也。漢書地理志龍勒縣,「有氐置水,出南羌中,東北入澤,溉民田」。一統志:「按輿圖,今有黨河在西,會南來一水,又折北流,繞沙州舊城之東、新城之西,入蘇賴河,溉田甚廣,當即古氐置水。」穆魯烏蘇河,又作胡胡烏蘇河,在黃河西大雪山北,源出索諾木達什嶺,北流四十餘里,折東北,合南來之密喇河、北來之薩爾哈卜齊海、阿爾昂諸水,東流入黃河。噶斯池,在黃河上流鄂靈海東北、固爾班蒙滾陀羅海山東南。有三池:一名鄂博圖噶斯池,周二十五里;一名多木達噶斯池,周十五里;一名察罕噶斯池,周十餘里。俱在黃河鄂博池之東,番名固爾班噶斯池。

青海和碩特部二十一旗:元太祖弟哈布圖哈薩爾七傳至阿克薩噶勒泰,生子二。長,阿魯克特魯爾,今內札薩克科爾沁、杜爾伯特、郭爾羅斯、札賚特、阿魯科爾沁、四子部落、茂明安、烏喇特八部之祖也。次,烏魯克特穆爾,十傳至哈尼諾顏洪果爾,生六子。其第四子圖魯拜琥,號顧實汗,後裔繁衍。游牧青海者十九旗。又有游牧西套之阿拉善旗,游牧察哈爾之和碩特旗。顧實汗長兄哈納克土謝圖,其裔為青海和碩特部所屬之西右翼中旗。顧實汗季弟色樓哈坦巴圖爾,其裔為青海和碩特部所屬之西右翼後旗。此二旗合顧實汗裔為二十一旗。顧實汗第三兄昆都倫烏巴什,其裔為游牧珠勒都斯之中路和碩特旗,游牧科布多之新和碩特旗。青海和碩特部在西寧邊外。北極高三十四度三十三分。京師偏西十五度十四分。西前旗顧實汗之子。康熙四十二年封多羅郡王。雍正三年授札薩克,世襲。佐領八。牧地在布喀河南岸。東至烏圖起爾沙陀羅海,南至西拉庫圖爾、果庫圖爾,西至察罕烏蘇呼魯恭納,北至布喀河濱納令布楞。班禪商上堪布喇嘛牧場,在旗境額勒池水南。前頭旗顧實汗之孫。康熙四十年封多羅貝勒。五十六年晉封郡王。雍正三年授札薩克,世襲。佐領十一。牧地南當黃河之曲,有小哈柳圖河,入於黃河。東至拉布楞希拉得布沙,南至和託果爾希里克,西至巴爾鄂博巴顏烏拉,北至額爾德尼布烏魯勒卜達巴。黃河重源,再顯於巴顏喀喇山之東麓,二泉流數里,合而東南流,曰阿爾坦河。阿爾坦,蒙古言「金」也,水色微黃似之。東北流三百餘里,至鄂屯塔拉,為古星宿海,元史所謂火惇腦兒也,直西寧邊外西南一千一百餘里。星宿海於群山環繞中,有地平曠,可三百里,有泉千百,隨地湧出,大小錯列,望若列星。阿爾坦河自西南來,皆匯入焉。東北流百餘里,又東南注札凌海。海周三百餘里,東西長,南北狹,河亙其中而流。番語謂白為「札」,長為「凌」,以其水色白也。又東南注鄂凌海。海在札凌東五十餘里,周亦三百餘里,形如匏瓜,西南廣,東北狹。番語謂青為「鄂」,言水色青也,即元史所謂匯二巨澤,名阿剌腦兒者也。由海東北流出,折東南,南抵巴顏渾嶺下,復正南流百五十里,水色始變綠而黃。又東南流,曲曲七百餘里,繞大雪山南,古積石山也。番名阿木尼麻禪母遜阿林。阿木尼謂「祖」,麻禪謂「險惡」,母遜謂「冰」,猶言「大冰山」也。山自巴顏喀喇東來,當黃河北岸,綿亙三百餘里,上有九峰,甚高,冬夏積雪。在西寧邊外西南五百三十餘里。元史謂之亦耳麻不莫剌。黃河依山南麓東流,折而東北,有三坤都倫河前後自東南來注之。三坤都倫者,一曰德特坤都倫,出賴楚山,西北流三百餘里入黃河,即元史納鄰哈喇河,自白狗嶺北流者。一曰都爾達都坤都倫,出納克多木精山,西北流,屈曲三百數十里入黃河,當河流自南轉東北處,即元史乞里馬出河,自威茂州西北岷山之北北流者。一曰多拉坤都倫,源出岡芧山,西北流六百數十里入黃河,正當大河於烏蘭莽鼐山麓折而西北流之處,即元史鵬★山西北七百里,過札塞塔失地與河合者。黃河既納此三水,勢甚盛,至烏蘭莽鼐山下,始折而西北流二百里,小哈柳圖河自東北來入之。小哈柳圖源出東北魯察布拉山,二源,西南流百里合,又西入河,當游牧西、土爾扈特南、前旗東。前左翼頭旗顧實汗之孫。康熙四十三年封多羅貝勒。雍正元年晉郡王,三年授札薩克,世襲。佐領九。牧地在大通河南岸。東至阿木達賴臺,南至固爾班塔拉之北沙克圖,西至齊擦擦呢布楚勒,北至巴顏布拉克。大通河源出青海西北阿木尼尼庫山南諾爾,東南流,曰烏蘭木倫河。又東,哈爾渾河自北來注之。又東北,曲曲流,南受一水,又東北,滿楚喀河自西北來注之。東逕甘州邊外番大山,東南流八百里,北受小水六,南受小水五,至西大通堡南,又東南會湟水,又東南入黃河,即古浩亹水也。河北為西右翼前旗游牧地。西後旗顧實汗之裔。康熙五十五年封多羅貝勒。雍正三年授札薩克,世襲。佐領九。牧地跨柴集河,其水北注鹽池。東至錫喇鹽海子、察罕託羅海,南至合約爾巴爾克,西至布隆吉爾河源,北至果庫圖爾、希拉庫圖爾。鹽池在青海西南,周百餘里,產青鹽。蒙古名達布遜淖爾。其水自錫喇庫特爾山之莫和爾河,與布拉克地之察罕烏蘇河,西來匯為此池。又自池東南流出,會西來之巴爾虎河。又七十餘里,柴集河自東南來入之,名曰鹽河。復東南流,淪於功額池。凡青海蒙古與西寧一郡軍民,並各種番、回,食鹽皆取給於此。北右翼旗顧實汗之孫。康熙四十四年封輔國公。雍正元年晉貝勒,後降固山貝子。三年,授札薩克,世襲。佐領六。牧地在青海北岸。東至沙拉哈吉爾,南至庫庫諾爾齊津,西至吹吉烏立圖阿拉爾,北至烏蘭和碩。有伊克烏蘭和碩、巴哈烏蘭和碩二河,在旗境西,西北自庫德里山南流百餘里,入庫庫諾爾。北左翼旗顧實汗之孫。康熙四十四年封輔國公。雍正元年晉固山貝子。三年,授札薩克,世襲。佐領三。牧地在布隆吉爾河南岸。東至哈喇諾爾,南至科爾魯克,西至窩果圖爾,北至伊克柴達木。烏蘭烏蘇河出東南沙磧中,西北行五百餘里,入達布遜諾爾。南左翼後旗顧實汗之裔。康熙五十年封輔國公。雍正三年授札薩克,世襲。佐領一。牧地在大通河南岸,青海正北。東至吉噶素臺鄂蘭布拉克,南至和洛海,西至布都克圖烏蘭和碩,北至青海。北前旗顧實汗之裔。康熙五十年封輔國公。雍正三年授札薩克,世襲。佐領二。牧地在青海西岸。東至科依特陀羅海,南至柴吉希巴立臺,西至車吉,北至哈達圖。南右翼後旗顧實汗之裔。康熙五十年封輔國公。雍正三年授札薩克,世襲。佐領四。牧地在青海東岸。東至賀爾,南至哈沙圖,西至哈拉素布魯漢,北至庫庫諾爾。坤都倫河自察罕鄂博圖山兩源合而南入西寧河。有世宗聖制碑,在旗界。西右翼中旗顧實汗伯兄之裔。雍正三年,領公中札薩克,授一等臺吉,世襲。佐領一。牧地跨柴達木河。東至諾木罕河,南至諾木罕木魯,西至滔賚,北至希勒沿。舒哈河自旗西無名海子流出,西北入於沙。柴達木河出河源北托遜淖爾,西流至西拉珠爾格塔拉,阿拉克淖爾水東來入之,合而西北流,格德爾古河、烏蘭烏蘇河、布隆吉爾河俱自其東注之,又西入於沙。西右翼前旗顧實汗之裔。雍正三年授札薩克一等臺吉,世襲。佐領二。牧地在大通河北岸。東至察罕阿爾吉永安,南至約呼賚口,西至柴達木察罕巴彥託羅海,北至希立永安。南右翼中旗顧實汗之裔。康熙五十九年封輔國公。雍正三年授札薩克。乾隆四十年,降一等臺吉,世襲。佐領五。牧地當魯察布拉山之西。東至庫克烏松,南至齊克特尼諾爾,西至僧克圖木齊,北至庫克烏松西山。魯察布拉山,舊作羅插普拉,即禹貢之西傾山也,一名西彊山,亦名嵹臺山,在洮州西南三百三十餘里。史記夏本紀「道九山」,索隱「九山古分三條,馬融以西傾為中條。鄭康成分四列,汧為陰列,西傾次陰列」。漢書地理志隴西郡臨洮,「禹貢西傾山在縣西」。北史吐谷渾傳:「阿豺升西彊山觀洮江源。」水經注:「山東即洮水源。嵹臺,西傾之異名也。」括地志:「西傾山今嵹臺山,在洮州臨潭縣西南三百六十六里。」元和志:「嵹臺山在臨潭縣西南三百里。」一統志:「西傾山,番名羅插普喇山,近黃河自東折而西北之東岸,綿亙千餘里。凡黃河以南諸山,無大於此者。洮河發源於此。」南左翼中旗顧實汗之裔。康熙五十年封輔國公,晉貝子、貝勒,後降襲札薩克一等臺吉,世襲。佐領四。牧地西濱黃河。有恰克圖河,東南來流入之。東至巴哈圖爾根,南至阿爾坦果爾,西至伊克圖爾根,北至巴哈圖爾根。恰克圖河在洮州西六百餘里黃河東岸,源出伊克圖爾根山,東北流,折而北,會巴哈圖爾根山之水,折而西北流百餘里,有伊西克山之水,自東北來會,又西北入黃河。又有碩爾渾河,舊作碩爾郭爾,在恰克圖河之北,源出古爾班圖爾哈山,會三小水,西北流入黃河。北左末旗顧實汗之裔。雍正三年授札薩克一臺吉,世襲。佐領四。牧地東至柴吉沁。南至鹽海,西至哈唐和碩,北至和特克。北右末旗顧實汗之裔。雍正三年授札薩克一等臺吉,世襲。佐領二。牧地在布喀河源沙爾諾爾之西。東至色爾柯克達巴,南至察罕陀羅海,西至薩爾魯克,北至庫爾魯克。布喀河在青海西,源出青海西北阿木尼厄枯山南,名喀喇錫納河,南流與英額池水會。池周一百五十餘里,其水東南流,會於喀喇錫納河。復東南流,至天沁察罕峰北,與沙爾諾爾水會,即所稱善池也。諾爾周六十餘里,其水東流,至天沁察罕峰前,亦入喀喇錫納河,又東流,受北來之羅子河、西爾哈河。又東,受北來之濟拉瑪爾臺河,乃名布喀河。又東流注青海。其河受六大水,岸闊流深,夏月人不可渡。青海左右諸水,無有大於此者。東上旗顧實汗之孫。雍正三年授札薩克一等臺吉,世襲。佐領一。牧地在青海東北岸。東至阿拉賴達巴木魯,南至柴吉,西至青海,北至烏爾肯希巴立臺。南左翼次旗顧實汗之裔。雍正三年授協理臺吉。九年,晉札薩克一等臺吉,世襲。與前左翼頭旗共佐領九。牧地有鹽池。東至沙拉圖,南至海達克,西至努克孫山鄂昔齊,北至烏蘭墨爾河。鹽池在青海西南,周百餘里,產青鹽。柴集河自東南來注之。南左翼末旗顧實汗之裔。康熙三十六年封貝勒,後削爵。雍正三年授札薩克一等臺吉,世襲。佐領二。牧地當博羅充克克河源。東至囊吉立圖巴爾布哈,南至圖祿根河,西至恰克圖北山木魯,北至恰克圖河。博羅充克克河,舊作波洛沖科克,即古湟水,一名洛都水者也。在西寧府西北邊外,當青海之東,源出噶爾藏嶺,元人所謂祁連山,明志之熱水山也。有三泉,一曰伊克烏拉古兒臺,一名土爾根烏拉古爾臺,一名察哈烏拉古爾臺,南流匯為一水,名博羅充克克河。其東有布虎圖嶺二泉,亦南來合,曰昆都倫河,東南流,與巴哈圖河合流入博羅充克克河。又東南流,至棟科爾廟南,有土爾根察罕河,自西南來會,水勢始盛。轉東流,入西寧邊鎮海營,是為西寧河,即湟水也。又東流三百餘里,南至莊浪衛降唐堡入大通河。漢書地理志金城郡臨羌,「西北至塞外,有西王母石室、仙海、鹽池,北則湟水所出,東至允吾入河」。水經注:「湟水出塞外,東逕西王母石室,東南流,逕龍夷城,故西零之地也。又東南,逕卑禾羌海北,有鹽池,世謂之青海。東流逕湟中城北,故小月氏之地也。又東,右控四水,導源四溪,東北流注于湟。又東逕赤城北而東入,逕戎峽口,右合羌水,又東逕臨羌縣故城北,又東,盧溪水注之。又東逕臨羌新縣故城南,又東,右合溜溪、伏溜、石杜、蠡四川,左會臨羌溪水。又東,龍駒川水注之。又東,長寧川水注之。又東,牛心川水注之。又逕西平城北,又東逕土樓南,右則五泉注之。又東,右合蔥谷水,又東逕東亭北,東出漆峽,東流,右則漆谷常溪注之,左則甘夷川水入焉。又東,安夷川水注之。又東逕安夷縣故城。又東,左合宜春水,又東,勒且溪水注之。又東,左則承流谷水南入,右會達扶東西二溪水,東流,期頓、雞穀二水北流注之。又東,吐那孤、長門兩川南流入之。又東逕樂都城南,東流,又合來穀、乞斤二水,左會陽非、流溪、細谷三水,東逕破羌縣故城南,六谷水自南、破羌川自北,左右翼注之。又東逕小晉興城北,又東與閤門河合,即浩亹河也。又東逕允吾縣北,為鄭伯津,與澗水合。又東逕允街縣故城南,又東逕枝陽縣,逆水注之。」後漢書注:「湟水一名洛都水,西自吐谷渾界入,在今湟水縣。」元和志:「湟水一名湟河,亦謂之洛都水,出青海東北亂山中,東南流,至蘭州西南入黃河。」唐書吐蕃傳:「湟水至濛谷,抵龍泉,與黃河合。」元史河源附錄:「湟水源自祁連山下,正東流一千餘里,注浩亹河,與黃河合。」冊說:「西川河源出西塞外海夷部落,東流,由石峽入境,至衛西北,受北川河,又東合南川河,而經城北,名西寧河。又至衛東北,受沙塘川水,又東南經碾白堡,名湟河。又東南接莊浪所界,合西大通河。又東合莊浪河,又東南至蘭州西南入黃河。北川河,番名阿爾坦河,源出西寧邊外,北至阿爾坦山,南流,會二小水,入北川河。又南流,入西寧北川邊內。又東南流,至西寧城南,入湟河。南川河番名西喇苦特河,源出西寧邊外西南西喇苦特山,東北流,至西寧城西北入湟河。又喀喇河在西寧邊外西北湟河之東,源出察罕鄂波圖嶺,合二小水,東南流,入西寧邊內,又流五十餘里,入湟河。南右翼末旗顧實汗之裔。康熙三十六年封輔國公,晉固山貝子。雍正元年削爵。三年,授札薩克一等臺吉,世襲。佐領一。牧地在黃河北岸,有錫尼諾爾。東至烏蘭布拉克,南至黃河舒爾古勒渡口,西至西拉珠爾格西山木魯,北至巴顏布拉克。錫尼諾爾在旗東界,其南岸與烏蘭河北入黃河之處相直。黃河自此北折,東逕貴德北,入西寧府界。西右翼後旗顧實汗之裔。雍正三年授札薩克一等臺吉,世襲。佐領一。牧地跨柴達木河。東至希昔,南至諾們罕木魯,西至烏拉斯臺,北至柴達木。西左翼後旗顧實汗弟之裔。雍正三年授札薩克一等臺吉,世襲。佐領一。牧地跨柴達木河。東至巴彥陀羅海,南至桑陀羅海,西至烏爾圖,北至瑪尼圖沙納圖。

青海綽羅斯部二旗:本準噶爾族。乾隆十九年,準噶爾平,其族遂微。附牧賽音諾顏部者曰額魯特。附牧青海者曰綽羅斯。轄旗二:南右翼頭旗,北中旗。北極高三十六度十八分。京師偏西十五度四十二分。南右翼頭旗準噶爾族。康熙四十二年封多羅貝勒。雍正元年晉郡王。三年授札薩克。乾隆三十年降貝勒,世襲。佐領四。牧地當青海東南岸。東至博爾巴齊他爾、察罕鄂博、哈拉烏素,南至固爾班他拉貢諾爾,西至窩爾登諾爾、伊克察罕哈達,北至青海。察罕陀羅海,南有巴顏淖爾,東北有蒙古圖布拉克,會東來二水,又東北有烏蘭布拉克。二水合流而西,會南來之巴顏淖爾水,為和爾必拉,北入青海。北中旗準噶爾族。康熙五十五年授公品級一等臺吉兼札薩克。雍正三年晉輔國公。乾隆十五年晉固山貝子,世襲。佐領二有半。牧地在青海西北岸。東至濟爾瑪爾臺,南至布喀沿。西至西爾哈落薩。北至濟爾瑪爾臺。西爾哈河,西北出槐滿阿林,東南流,又有羅色河,西北出庫得里阿林,西南流來合,南入布喀河。又西北,濟爾瑪爾臺河,屈曲南入布喀河,其南岸即和碩特北前旗也。

青海輝特部南一旗:姓伊克明安。有卓哩克圖和碩齊者,其子號青諾顏,游牧青海。雍正元年來降。三年,授札薩克一等臺吉。九年,晉輔國公,世襲。佐領一。牧地當巴彥諾爾之南。東至巴彥諾爾東山木魯,南至窩蘭布拉克、僧裏鄂博、哈立噶圖,西至博爾楚爾、哈立噶圖河,北至納蘭薩蘭。北極高三十六度十八分。京師偏西十五度四十二分。巴彥諾爾在青海東南,周四十餘里。水西北流出,屈曲三百數十里,入和爾必拉。

青海土爾扈特部四旗:元臣翁罕,數傳至博第蘇克,自稱青海土爾扈特臺吉。順治八年始通貢。雍正三年,編轄旗四。北極高三十五度十五分。京師偏西十七度十五分。南中旗翁罕之裔。雍正三年授札薩克一等臺吉,世襲。佐領四。牧地當登努爾特達巴罕之陽,東至果庫圖爾,南至果庫圖爾山木庫爾,西至庫克烏松,北至袞阿爾臺。西旗翁罕之裔。雍正三年授札薩克一等臺吉,世襲。佐領四。牧地在阿屯齊老圖,有阿勒淖爾泊。東至袞阿爾臺,南至黃河,西至哈爾古爾希立,北至庫克烏蘇唐素楞。南前旗翁罕之裔。雍正元年授札薩克一等臺吉,世襲。佐領一。牧地當大哈柳圖河之南,小哈柳圖河之北。東至古魯半博爾齊沙拉圖,南至黃河,西至宗科爾,北至恰克圖。大哈柳圖河,蒙古曰伊克哈柳圖,在洮州西六百餘里黃河北岸,源出納莫哈山烏蘭俄爾吉嶺,當布庫吉爾地。三源,東流百餘里,折而西南,合流,又西北流,入黃河。小哈柳圖河,源出魯察布拉山,二源,西南流百里合,又西入河。當旗境東。察漢諾們罕喇嘛游牧在旗境東北。南後旗翁罕之裔。雍正三年授札薩克一等臺吉,世襲。佐領三。牧地當碩羅巴顏哈拉山之陽,曰鄂博圖。東至莫古立源,南至袞阿爾臺,西至庫克烏松木魯,北至登納吉爾尼。薩爾哈布齊海河,自西來屈曲而南,有哈爾渾舍裡小河,舊以為阿爾坦河,自北來會,合而南入黃河。呼呼烏蘇河,在阿爾坦河之西北,源出蘇羅達巴罕,南入黃河。

青海喀爾喀部一旗:南右一旗。元太祖之裔。徙牧青海,隸和碩特族。雍正元年來歸。編旗一。乾隆三年,授公中札薩克一等臺吉。佐領一。牧地在青海南岸。東至察罕哈達,南至南山木魯,西至烏蘭布拉克,北至青海。北極高三十六度三十五分。京師偏西十六度三十二分。自東而西,有阿木尼塞爾沁阿林、伊克哈圖達巴罕、巴哈哈圖達巴罕、巴哈察罕哈達、伊克察罕哈達諸山。東有和爾河。西,札哈蘇太河。中,無名河六。俱北流入青海。達賴商上堪布喇嘛牧場在柴積河南。以上各部共二十九旗及察罕諾門為一盟,不設盟長,歸西寧辦事大臣統轄。

土司青海所屬凡四十:玉樹四司。一司、二司在木魯烏蘇河東。三司、四司在河西。阿拉克碩二司、白利、阿薩克、阿永,在河南。尼牙木錯、固察、拉布,在河北。札武三司在河東。隆布、吹冷多爾多,在布壘、布楚兩河間。上格爾吉在布楚河西。中格爾吉、下格爾吉、哈爾受、隆壩二司、隆東綽火爾、覺巴拉、蘇爾莽、葉爾吉、列旺、安圖、興巴、拉爾吉,俱在河北。桑色爾、巴顏囊謙,在河南。洞巴在河西。蘇魯克在索克河南。稱多在瑪楚河西。蒙古爾津、永河普,在黃河西。二阿里克,在齊普河東。西北境有阿克達木山、巴薩通拉木山,皆長數百里。極西北二十里,有錫津烏藍拖羅海山,託克託乃烏藍木倫河出其西南。為勒科爾烏藍達布遜山。喀齊烏藍木倫河,東流八百餘里,會烏蘇河,以下通稱木魯烏蘇河。有阿克達木河,出阿克達木山,屈曲流七八百里,北注之。折北流,會託克託乃烏藍木倫河。又轉東,受南來之布輝伯河。又東,受南來大水二,入玉樹司界。折西南,有那木齊圖烏蘭木倫河,出瀚海地,東流千餘里,折南注之。又南折東,逕拉瑪察喀山,北受齊齊爾納河。又東南為布壘河,入四川雅州所屬土司界,是為金沙江上源。阿克河出巴薩通拉木山,南流折東入喀喇烏蘇河。河自前藏東流入境,行三百餘里,折南流,有索克河出阿克達木山西,屈曲東流八百里而南注之,仍南入前藏界。布楚河上源曰格爾吉河,出上格爾司境,東南流,逕各司南,至洞巴司西,折南入前藏界,是為瀾滄江上源。瑪楚河出固察司東,東南流,入雅州土司界,是為雅龍江上源。黃河發源巴顏喀喇山東麓,名阿爾坦河,東北流,匯為大澤,名鄂端諾爾,即星宿海也。又東貫查靈海,南入鄂靈海,會西來烏蘭河,至永河普司東,又折東入額魯特界。齊普河上源有二,曰圖聲圖河,曰得爾多河,北流而合,環阿里克境,西北入黃河。以上納賦於西寧辦事大臣。


\end{pinyinscope}