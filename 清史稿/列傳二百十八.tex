\article{列傳二百十八}

\begin{pinyinscope}
郭松林李長樂楊鼎勛唐殿魁唐定奎

滕嗣武駱國忠

郭松林,字子美,湖南湘潭人。咸豐六年,隸曾國荃軍,從援江西,克安福,從剿永新、太和、萬安、蓮花、龍泉,敘獎把總。進圍吉安府城,七年,石達開率悍黨來援,邀擊於吉水三曲灘,松林首陷陣,多斬獲,收復新喻、峽江、吉水。八年,隨克吉安,擢守備。九年,克景德、浮梁,賜花翎。十年,圍安慶,會剿陳玉成於小池驛,進壁集賢關,每戰皆捷。十一年,克安慶,擢游擊,賜號奮勇巴圖魯。克廬江、無為、運漕鎮,下沿江要隘,擢參將。

同治元年,李鴻章率淮軍八千赴上海,松林從,與偽忠王李秀成偽慕王譚紹光大戰滬西,破賊眾十萬。會攻太倉,砲擊城隳,士卒爭進,浮橋斷,賊乘之,死數百人,松林力禦,始得收軍。二年,克太倉,松林敗賊茜涇、支塘,會克昆山、新陽,以總兵用。李秀成合水陸數十萬援江陰,犯常熟,劉銘傳謀乘賊未定擊之。賊北自北漍,南至張涇橋,東自陳市,西至長壽,縱橫六七十里,築壘憑河,勢大熾。銘傳進北漍攻其左,松林進南漍攻其右,周盛波等進麥市橋為中路,黃翼升以水師助之。松林敗賊陳市,越南漍趨張涇,揮刀蕩決,血染衣盡赤,賊大潰走。銘傳、盛波等同破賊,自顧山以西皆盡,以總兵記名。尋克江陰,以提督記名。又破賊緱山、梅村、麻塘橋,松林受矛傷,既而蘇州、無錫皆復,加頭品頂戴。

三年,克宜興、荊溪,敗賊張渚,毀賊壘,收溧陽,解常熟圍,授福山鎮總兵。大破三河口賊營,賊爭道,六浮橋盡斷,尸塞河,水為不流。克常州,進剿浙西,克長興,復湖州,功皆最。賊走廣德、徽州,合江寧、杭州賊自江西竄閩。四年,李鴻章檄松林率五千人航海赴援,克漳州、漳浦、雲霄、詔安,賊竄廣東嘉應,遂破滅。

五年,曾國荃調松林率新募湘軍剿捻匪於德安,克應城、雲夢,復敗之皁河、楊澤。追至臼口,中伏,松林傷足,臥地不能起。將卒不見松林,復闖入陣,負而出之。弟芳珍戰死。松林以創重假歸。六年,創愈,李鴻章令統萬人號武毅軍。時東捻任柱已斃,餘黨走壽光,松林要擊,破之杞城。賊沿海南走,阻瀰河,捻酋牛喜子麾白旗賊犯劉銘傳軍;賴文光麾藍旗賊犯松林軍。兩軍縱擊,賊大潰,壽光民圩皆出助殺,賊赴瀰河死,浮尸二萬餘,俘萬餘人,奪獲騾馬二萬匹。賊酋徐昌先、範汝增、任定皆伏誅。賴文光鳧水南奔,松林疾馳六百里,追至清江。文光死奔,至揚州瓦窯鋪,為吳毓蘭所擒。東捻平。

七年春,西捻犯畿輔,松林敗之安平,再破之茌平。自臨邑築長圍至馬頰河,松林偕潘鼎新、王心安守之,敗賊於海豐,追至德州,歷十六晝夜,斬捕過半。六月,松林會潘鼎新大破之沙河,俘斬四千。捻走黃、運、徒駭河間,松林與銘傳縱橫要擊,張總愚赴水死。西捻平,賜黃馬褂,予輕車都尉世職。授湖北提督,調直隸。光緒六年,卒於官,優恤,建專祠,謚武壯。

李長樂,字漢春,安徽盱眙人。同治元年,以外委從郭松林隸淮軍,充營官。克柘林、奉賢、南匯、川沙、金山,解松江圍,復青浦,擢千總。戰四江口,松林軍方泰鎮,長樂率所部深入,近賊壘。夜半,趣軍士起,曰:「今陷賊中,旦明賊覺,無得脫者。盍出奇計劫之!」遂投火賊幕,鼓角乘之,賊驚擾,長樂奮呼進,大破之。又設伏黃渡,擊之半濟,又敗之吳淞江南。四江口圍解,擢都司,賜花翎。

二年,進屯常熟王莊,援賊踞陳市,阻官軍進路。從松林自南漍攻賊右,連破賊營,直趨長涇。長樂陷陣傷脛,裹創力戰敗賊,擢參將,賜號侃勇巴圖魯。尋克江陰,規無錫,出新塘橋。賊憑壘鳴砲俯擊,長樂濡絮裹身越溝進,敗之;逐奔至亭子橋,刺賊酋黃子隆中肩,又設伏兵敗援賊。李秀成圍大橋角營,從松林往援,奪其舟,賊退走,盡平梅村諸壘。會諸軍圍攻無錫,率輕騎掩至,梯城入,黃子隆就擒,長樂獲其子德懋。尋坐失察部勇,褫職留軍。進規常州,援總兵唐殿魁於奔牛,解其圍。三年,敗賊上湖橋,克宜興,復官。移軍溧陽、金壇,戰皆捷。回援常熟,解其圍,連破賊於楊舍、華墅、周莊、三河口,會攻常州。四月,合圍,長樂先登,賊酋陳坤書、黃和錦就擒,復常州,擢副將,賜號尚勇巴圖魯。

從松林進克浙江長興,以總兵記名。進湖州,破呂山賊。攻賊酋黃文金於尹隆橋,官軍不利,長樂率三營別屯李家港,保糧道。賊傾巢來爭,長樂偕易用剛夾擊之,斬賊酋黃十四,破尹隆橋,遂復湖州。四年春,從松林援福建,戰於海澄赤嶺。松林分兵為八隊,長樂居中當賊首李世賢,破走之,竄漳州。長樂進屯古田,據山東形勝,賊悉銳力爭,擊卻之,復漳浦、雲霄。南趨詔安,破之梅村,復其城,加提督銜。福建平,旋師江蘇,屯鎮江。

曾國籓督師剿捻匪,松林已歸,長樂代將其眾以從,兼統忠樸三營,為游擊之師,轉戰河南、山東間。六年,李鴻章代國籓督師,松林復至軍,增松林軍至二十餘營,號武毅軍,長樂所部曰武毅軍前軍。破任柱於贛榆,要賴文光於濰縣,長樂等並力奮擊,賊鳧水東走,躡至餘家寨,賊受創甚鉅,復要之壽光南北洋河、巨瀰河間,擒斬三萬,文光竄揚州被擒,賜黃馬褂。

七年,從剿西捻,戰安平,馬軍失利,長樂等以步卒馳援,賊大潰;追至饒陽楊家村,又要之深州李家村,破其馬隊,斬獲無算。三月,敗賊大坯山。援提督陳振邦於大河村,解其圍,追挫之茌平、滄州,援副都統春壽於海豐郝家寨。六月,追至樂陵,擒總愚子正江、弟得華,戰商河,槍傷總愚。西捻平,以提督總兵遇缺題奏,晉博奇巴圖魯。十年,署湖北提督,尋實授。光緒五年,調湖南。六年,調直隸。近畿海防重要,奏令長樂駐蘆臺,扼大沽、北塘門戶。十五年,卒官,優恤,謚勤勇。

楊鼎勛,字少銘,四川華陽人。咸豐二年,應募從軍,初隸湖北按察使李孟群,克漢陽,擢把總。七年,隸提督鮑超軍。八年,戰湖口,擢千總。十年,鮑超與陳玉成大戰小池驛,鼎勛見玉成立陣中指揮,獨從壯士十數人突前擊之,玉成駭走,復太湖、潛山兩城。敘功,賜花翎。李秀成踞黟縣,鼎勛擊賊城下,奪門入,大軍繼之,復其城。十一年,復建德,擢都司。破安慶赤岡嶺賊壘,擢游擊。初,小池驛之戰,鮑超嘉其功,令將五百人,所向有功。諸將嫉之,譖於超。

同治元年,李鴻章督師上海,遂去超從淮軍。虹橋、四江口諸戰有功,累擢副將。募淮勇千人,號勛字軍,屯金山張堰,扼平湖乍浦要沖,習西洋槍隊,每戰輒為軍鋒。二年,破新昌賊壘,連克楓涇,斬賊四千,生擒五百;再戰西塘,裹創奮擊,大敗之,擢總兵,賜號鋒勇巴圖魯。從程學啟規蘇州,鼎勛攻下城外堅壘,蘇州復,加提督銜。三年,會克宜興、荊溪、溧陽,解常熟、無錫圍。攻常州,賊因蘇州之殺降,惟死守。鼎勛以蜀人將淮勇,懼諸將輕己,每戰輒先,晝夜環攻,盡毀城外賊壘,血戰三日。造浮橋,率死士先登城,砲彈洞胸達背,左右扶之,絕而復蘇,遂克常州,以提督記名簡放。創愈,進克浦口,復長興,招降湖州賊黨,會克其城。追賊至皖境,克廣德。四年,偕郭松林援福建,攻烏頭門賊壘,復漳州,授江蘇蘇松鎮總兵。

五年,調赴河南剿捻匪,敗賊硃仙鎮,躡擊至定陶、睢寧。六年,破賊於黃陂、孝感,擢浙江提督,調湖南。十月,破賊於山東濰縣,追至夏灣,賊酋陳懷忠乞降,分軍出周家寨襲賊,大破之。追擊於諸城、膠州。東捻平,論功,予騎都尉世職。七年,馳援畿輔,破捻匪於安平,追至楊家村,降賊酋張志清。偕郭松林擊賊濬縣大邳山,又敗之衛輝,陣斬賊酋王建瀛、熊八,擒悍賊何士喜、周久於龍王廟。賊竄山東,自德州趨天津,鼎勛守運河,修墻浚壕,賊來犯,輒擊走之。會舊傷發,遽卒,數日而西捻平。李鴻章疏聞,贈太子少保,謚忠勤,建專祠。

唐殿魁,字藎臣,安徽合肥人。咸豐十年,巡撫翁同書檄率鄉團援壽州,力解城圍。又從克合肥三河汛,解六安圍,敘千總。同治元年,李鴻章率淮軍援上海,殿魁從,隸劉銘傳,克南匯、川沙、奉賢、金山衛、柘林五城,積功累擢都司,賜花翎。二年,克江陰楊舍汛城,擢游擊,賜號振勇巴圖魯。復江陰縣城,擢參將。尋克無錫,以總兵記名。

從劉銘傳攻常州,銘傳受重傷,令殿魁偕副將黃桂蘭督兵進。甫至奔牛鎮,而常州、丹陽兩路賊麕至,圍之。殿魁據石營依河岸,壁壘悉為賊轟毀,堅守二十餘日。銘傳裹創往救,殿魁從內夾擊,苦戰數日,圍始解。三年,克常州,生擒賊首陳坤書,以提督記名。四年,增募所部至三千人。從劉銘傳渡淮剿捻匪,破張寨賊壘。五年,授浙江衢州鎮總兵。追賊至湖北,克黃陂。捻匪自山東回竄,銘傳督兵追至烏官屯,殿魁繼進,殺賊五百餘。六年,調廣西右江鎮。

捻首張總愚竄安陸。銘傳與鮑超約會戰於永漋河,銘傳欲先出,殿魁請少待,不從。超未至,銘軍先遇賊,部將田履安、李錫增戰沒。殿魁戰小挫,受重傷。及聞霆軍大捷,復裹創力戰,遂殞於陣。贈太子少保,予騎都尉兼雲騎尉世職,謚忠壯,建專祠。

唐定奎,字俊侯,殿魁弟。偕兄轉戰江蘇。從劉銘傳剿捻於山東、河南、安徽、湖北,積功累擢副將,賜花翎。同治六年,殿魁戰歿永漋河,定奎方省母回裡,奔赴軍,誓殺賊復仇,遂代領兄舊部,轉戰河南、山東。六年,殄任柱於贛榆,破賴文光於壽光,所部殺賊最多。東捻平,以提督記名。七年,從剿西捻於直隸、山東,賜號呼敦巴圖魯。銘軍凱旋,告歸終養。九年,丁母憂。劉銘傳赴陜西剿回匪,調定奎接統銘武軍,定奎請終制,命俟陜西軍事平,回籍終制。十年,定奎回防徐州。

十三年,日本擾臺灣,生番滋事。船政大臣沈葆楨奏請援師,李鴻章薦定奎率所部往。七月,至臺灣,駐鳳山,擇險分屯。龜紋番社引日兵與刺桐腳莊民尋仇相閧,定奎示以兵威,日人引去。時疫流行,士卒先後死千餘人,定奎拊循周至,兵氣不衰,賜黃馬褂。

楓港、獅頭諸社番屢出戕害良民,光緒元年,游擊王開俊進剿,中伏死。內外番社結黨劫殺,各社就撫,皆懷觀望。定奎分遣七營屯東港南勢湖,自率四營當其沖,葆楨檄諸軍並聽節制。定奎上書陳兵事,略曰:「逆番晝伏莽中,夜燎山頂,精於標槍,伺間輒發。專恃深林密箐,狙擊我師,我進彼隱,我退彼見。今欲掃其巢穴,必先翦其荊棘。宜增募土勇,導引兵丁,隨山刊木,務絕根株,然後分道進兵,草薙擒獮。其有奸民接濟鹽米火藥者,按軍法,庶幾一舉可以成功。」葆楨據以入告。於是開山進兵,攻克萃山、竹坑、本武諸社。獅頭社猶負險抗拒,定奎令諸將得險即守,自剿獅頭兩社,別遣師扼斷外援,遂攻下之。移營駐守,被脅十餘社皆歸命,給衣履酒食,譯示朝廷威德,咸受約束。設招撫局,示約七條,曰:遵薙發,編戶口,獻兇逆,禁仇殺,立總目,墾番地,設番塾。以龜紋番酋充諸社總目,赦其脅從。臺南大定,詔褒獎,命內渡休養士卒。授直隸正定鎮總兵。尋擢福建陸路提督。

沈葆楨調兩江總督,奏統所部駐防江陰。九年,傷發乞休,不允。法越用兵事起,海防戒嚴,詔促力疾赴防。十一年,和議定,病請開缺,允之。十三年,卒,優恤,謚果介。

滕嗣武,湖南麻陽人。咸豐初,從軍湖北。十年,小池驛之戰,功多,超擢都司。從攻安慶,嗣武率所部扼要築砲壘,壘未成,賊突出萬餘來爭,嗣武力擊破之。十一年,克安慶,敘功擢參將。同治元年,改隸淮軍,從李鴻章至上海,解松江圍,賜號偉勇巴圖魯。屯北簳山扼賊沖,賊犯寶山,與諸軍夾擊破之,進拔南匯,以總兵記名。

二年,偕程學啟規蘇州,敗賊於正義鎮。地當要沖,以嗣武守之,輔以水師,分軍伏橋口伺賊。昆山賊勢蹙,啟西門遁。伏起,水師以巨砲環擊,賊大潰,立復其城。移軍會攻江陰。賊自無錫來援,連營數十,柵壘棋布。軍分三路進,嗣武率八營當中路,攻麥市橋,以輕兵伏河堤,燃砲毀賊壘,賊潰走,追及之三巴橋,殲獲殆盡。進次無錫城下,賊首李世賢以全軍拒戰。嗣武身先士卒,怒馬突陣,敗之謝家橋,又敗之蕩口。賊退據硃王橋堅守,嗣武出奇兵襲擊,擒斬千餘,加提督銜。既而克無錫,以提督記名。

三年,會攻常州,破援賊於奔牛鎮,攻下宜興、荊溪,嗣武傷右股。四月,會圍常州,嗣武當南門,砲毀城垣,克之。七年,從李鴻章剿捻。畿輔事平。八年,授湖北鄖陽鎮總兵。十一年,卒,賜恤,謚武慎。

駱國忠,安徽鳳陽人。初陷於粵匪,授偽職,知賊必敗,陰圖反正。常熟久為賊踞,福山與狼山夾江對峙,賊設屯以扼後路,國忠任城守。同治元年,李鴻章蒞江蘇,兵威日振,國忠因水師游擊周興隆舉城薙發降。鴻章令興隆、國忠選驍健萬人,分守水陸要沖,以防蘇州竄賊。福山守賊胡經元、江勝海原約俱降,國忠遣人召之,比至福山,不得入。國忠夜率兵往,令其弟國孝攻其南,自與興隆攻其北,斷賊登舟之路,槍殪賊將侯得龍,賊舟師遁走。經元、勝海殺賊渠數人,率所部出,與國忠合。國孝越重壕毀賊壘;興隆等分兵盡拔許浦、白茅、徐涇諸壘,賊將錢壽仁亦自太倉率所部二千詣鴻章降。總兵鞠耀乾率師船泊徐涇,千總袁光政入城助守。

十二月,李秀成等以眾數萬自蘇州來攻,連營十餘里。國忠乞濟師,常勝軍五百人自海道往援,而賊由江陰再陷福山,聲援隔絕。鴻章令潘鼎新、劉銘傳、張樹珊以三千人趨福山,與黃翼升水師並進。福山城小而堅,攻之不下。常熟被圍愈亟,國忠斂兵入城,興隆屯城西虞山相犄角,為死守計。二年,賊以砲壞城東垣,國忠力拒不得入,樹雲梯緣城,亦擊卻之。賊增壘掘隧,數道並進,城危甚。會鼎新、銘傳諸軍急攻福山,賊分兵赴援,留者僅數千,國忠乃開城出戰,毀賊壘,擒其渠硃衣點。福山既克,諸軍來會,城圍始解。捷聞,優詔嘉獎,擢授國忠副將,加總兵銜,編降眾為忠字八營。會攻江陰,戰甚力,既克,賜號勁勇巴圖魯。署京口水師副將,留守江陰。三年,破丹陽援賊,以總兵記名。尋克常州,以積勞傷發,乞假歸。

五年,從劉銘傳剿捻匪,轉戰湖北、河南、山東,所鄉有功。六年,東捻平,以提督記名,賜黃馬褂。九年,銘傳督辦陜西軍務,調國忠從行。十二年,卒於乾州軍中,賜恤,謚勇肅。

論曰:郭松林、李長樂、楊鼎勛、滕嗣武皆由湘、楚舊部改隸淮軍,平吳、平捻,卓著聲績。唐殿魁淮軍驍將,惜未竟功。定奎席兄餘光,名位轉出其上,固有幸有不幸哉。駱國忠智勇堅毅,識時為傑,當時名滿江南,成績可紀也。


\end{pinyinscope}