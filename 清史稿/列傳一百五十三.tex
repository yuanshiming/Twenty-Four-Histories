\article{列傳一百五十三}

\begin{pinyinscope}
孫玉庭、蔣攸銛、李鴻賓

孫玉庭,字寄圃,山東濟寧人。乾隆四十年進士,選庶吉士,授檢討。五十一年,出為山西河東道,父憂去,服闋,補廣西鹽法道。嘉慶初,就遷按察使,歷湖南、安徽、湖北布政使,舉發道員胡齊侖侵冒軍需,詔嘉之。

七年,擢廣西巡撫,調廣東。安南國王阮光纘為農耐、阮福映所逼,叩關乞內避,命玉庭馳赴廣西察辦。福映已滅光纘,遣使納款,玉庭疏陳其恭順,請受之。尋福映請改國名曰南越,仁宗疑之。玉庭言:「不可以語言文字阻外夷鄉化之心。其先有古越裳地,繼並安南。若改號越南,亦與中國南粵舊名有別。」乃報可。廣東海盜日橫,玉庭議防急於剿,請增兵嚴守口岸,禁淡水米糧出海以制之。尋調廣西,十年,復調廣東。時總督那彥成專意招撫,玉庭意不合,疏陳其弊,謂:「盜非悔罪,特為貪利而來。官吏貪功,不惜重金為市。陽避盜名,陰攖盜實。廢法斂怨,莫此為尤。」上韙其言,那彥成由是獲罪。

十三年,英吉利兵船入澳門,總督吳熊光但停貿易,未遣兵驅逐,上斥畏葸,罷熊光,調玉庭貴州。尋百齡至粵,追論熊光,且劾玉庭不以實入告,坐罷歸。已而予官編修,在文穎館行走。十五年,授雲南巡撫,兼署云貴總督。調浙江。二十年,英吉利貢使不原行跪拜禮,廷議以其倔強,遣之。會玉庭入覲,面奏馭夷之道:「妄有干求,當折以天朝之法度;歸心恪順,不責以中國之儀文。」反覆開陳,上意乃解。

二十一年,擢湖廣總督。未幾,調兩江。漕、鹽、河為江南要政,日臻疲累。玉庭久任封圻,治尚安靜,整頓江西、湖北引岸緝私,籌款生息,津貼屯丁,減省漕委,隨事為補苴之計,稍稍相安。宣宗即位,特加太子少保銜。時用尚書英和言,清查直省陋規,立以限制,下疆臣議久遠之法。玉庭疏言:「自古有治人無治法。果督撫兩司皆得人,則大法小廉,自不虞所屬苛取病民;非然者,雖立限制,仍同虛設,弊且滋甚。各省陋規,本干例禁。語云:『作法於涼,其弊猶貪。』禁人之取猶不能不取;若許之取,勢必益無顧忌。迨發覺治罪,民已大受其累。府、、州、縣祿入無多,向來不能不藉陋規為辦公之需,然未聞準其加取於民垂為令甲者,誠以自古無此制祿之經也。伏乞停止查辦,天下幸甚。」疏入,詔褒其不媿大臣之言。

道光元年,授協辦大學士,仍留總督任。是年入覲,與玉瀾堂十五老臣宴。帝詢淮鹽疏銷之策,玉庭言:「漢口為淮南售鹽總岸,向來船到隨時交易,是以暢銷。自乾隆中立封輪法,挨次輪售,私鹽乘間侵越。」因臚陳六害,請復舊章,從之。又言漕糧浮收不能禁革,不如明與八折為便。御史王家相奏言事類加賦,侍郎姚文田、湯金釗亦論之,事遂寢。然州縣困於丁費,浮收仍難禁絕,胥吏上下其手,專累良懦,因玉庭議不行,疆臣不敢復請;至同治初,始定漕耗,卒如玉庭議。

四年,拜體仁閣大學士,留任如故。會高家堰決,河督張文浩遣戍,部議玉庭革職,詔念前勞,寬之,留任。尋復以借黃濟運無效,褫職,予編修休致。戶部復劾其不行海運,而河病運阻,責償滯漕剝運費十之七,命留濬運河。工竣,回籍。十四年,重宴鹿鳴,加四品頂戴。尋卒,年八十有三。

子善寶,以舉人廕生授刑部員外郎,官至江蘇巡撫;瑞珍,道光三年進士,由翰林官至戶部尚書,謚文定。孫毓溎,道光二十四年一甲一名進士,官至浙江按察使;毓汶亦以一甲二名進士,官至兵部尚書,自有傳。曾孫楫,咸豐二年進士,翰林院庶吉士,官至順天府尹。四世並歷清要,家門之盛,北方士族無與埒焉。

蔣攸銛,字礪堂,漢軍鑲紅旗人。先世由浙江遷遼東,從入關,居寶坻。乾隆四十九年,成進士,年甫十九,選庶吉士,授編修。嘉慶初,遷御史,敢言有聲,受仁宗知。五年,出為江西吉南贛道,署按察使。八年,廣昌齋匪廖幹用作亂,攸銛率兵平之。疆臣上其功,會丁母憂去。十年,特起署廣東惠潮嘉道,歷江西按察使、雲南布政使。十四年,調江蘇,就擢巡撫。調浙江,擢江南河道總督,以不諳河務辭,詔回原任。

十六年,擢兩廣總督。嚴於治盜,遴勤幹文武大員駐廣、肇、韶、連諸郡居中之地,分路搜截,飭州縣官赴鄉勸導耆老,使境內不得藏奸,舉劾嚴明,吏皆用命。歷擒匪盜七百餘名,自首者許自新,特詔褒獎。十八年,應詔陳言,略曰:「我朝累代功德在民,而亂民愍不畏法,變出意外,此皆由於吏治不修所致。臣觀近日道、府、州、縣,貪酷者少而委靡者多。夫闒冗之釀患,與貪酷等。竊以為方今急務,莫先於察吏,而欲振積習,必用破格之勸懲。凡貪酷者固應嚴參,平庸者亦隨時勒休改用,勿俟大計始行覈辦。其有勤能者,即請旨優獎。果道、府、州、縣得人,則禍亂之萌自息。」次年,又上疏曰:「道府由牧令起家者十之二三,由部員外擢者十之七八。聞近來司員少卓著之才,由於滿洲之廕生太易,漢員之捐班太多。請飭部臣隨時考覈,其不宜於部務者,以同知、通判分發各省,使練民事,部曹亦可疏通。今之人才沉於下位者多矣,請飭大臣薦達,擇其名實相副者擢用。抑臣更有請者,任事之與專擅,有義利之分,若任事而以專擅罪之,人皆推諉以自全矣。協恭之與黨援,有公私之別,如協恭而以黨援目之,人且立異以遠嫌矣。此近今之積習,為大臣者當力除之。至翰林儒臣,務在崇正學,黜浮華,養成明體達用之才,不必以文章課殿最。科道為耳目之官,敷陳能否得體,糾劾是否為公,詢事考言,難逃洞鑒。其有卓越清正者,當由京堂而擢卿貳,與翰詹參用。用人之道,因才因地因時,臣下無可市之恩,君上有特操之鑒。人無求備,政在集思,此之謂也。」疏入,上嘉納之。

英吉利兵船入內洋,攸銛飭停貿易,乃聽命引去。請禁民人為洋人服役,洋行不許建洋式房屋,鋪商不得用洋字店號,清查商欠,不準無身家者濫充洋商,及內地人私往洋館,並如議行。商人負暹羅國貨價,以官錢代償,既而貢使來繳還。攸銛以奉旨頒給,乃示懷柔,不得復收回,卻之,詔嘉其得體。

二十二年,調四川總督。四川兵故驕縱,一裁以法。民多帶刀劍,禁鄉村設爐制兵刃。城市編牌取結,有犯連坐。以義倉租息助灌縣都江堰歲修,禁派捐累民。重修文翁石室,興學造士。言官請禁非刑,飭屬銷毀違法刑具,而嚴戒縱匪,不得博寬厚虛名,貽閭閻實害。二十四年,率土司頭目入都祝嘏,賞賚有加。時因慶典,普免天下積欠錢糧,獨四川無欠可免,詔嘉其撫綏有方,予優敘。二十五年,仁宗崩,入謁梓宮,宣宗諭褒為守兼優,加太子少保。

道光二年,召授刑部尚書。尋授直隸總督。值水災,請截南漕四十萬石,賑款先後二百萬兩,逾年賑事竣。時方治畿輔水利,命侍郎張文浩蒞其事,尋以程含章代之,攸銛與合疏言東西兩澱,大清、永定、子牙、南北運五河,及天津海口、千里堤,不可緩之工,請部撥銀一百二十萬兩;又疏陳千里堤章程,規復兩澱垡船汊夫,移改管河員弁駐所,添建巡防堡房。並如議行。命協辦大學士,仍留總督任。五年,拜體仁閣大學士,充軍機大臣,管理刑部。以回疆平,加太子太保。

七年,授兩江總督。疏言總督於河務非專責,與河臣同治,徒掣其肘,請毋庸駐清江浦,從之。時清水不能敵黃,漕運屢阻。攸銛初在浙,不主海運,至是見河、漕交困,試行海運便利,遂請續行,並預儲銀六十萬兩,備河運盤壩之用。廷議方主倒塘濟運法,且疑其畏難便私,不許。攸銛疏辯,極言倒塘之不足恃,上終不以為然,姑許海運,而禁言盤壩。未幾,海運亦罷。以張格爾就擒,追論贊畫功,晉太子太傅。

黃玉林者,鹽梟巨魁,以儀徵老虎頸為窟穴,長江千里,呼吸皆通,詔責嚴捕,玉林投首,乞捕私自效。十年,攸銛病,乞假,假滿,召回京供職,而玉林復圖販私,攸銛疏請嚴治,發遣新疆,尋復慮其潛回滋事,密請處絞。詔誅玉林,切責攸銛茍且從事,嚴議褫職,加恩降兵部侍郎。未至京,卒於途,優詔軫惜,依尚書例賜恤。

攸銛精敏強識,與人一面一言,閱數十年記憶不爽。勇於任事,不唯阿。尤長於察吏,薦賢如不及,所舉後多以事功名節著。子霨遠,官至貴州巡撫,自有傳。

李鴻賓,字鹿蘋,江西德化人。嘉慶六年進士,選庶吉士,授檢討。遷御史、給事中。十八年,巡視東漕。會林清之變,數疏陳時政利弊;又以山東、河南、直隸毗連之地,頻年遭兵,條上善後事,始受仁宗知。命偕河督吳璥、巡撫同興按河督李亨特貪劣不職狀,得實以聞。

十九年,超授東河副總河。時微山湖蓄水盡涸,運河淤塞。鴻賓自巡漕時講求疏泉濟運之策,至是疏瀹上游,湖水通暢,瀦蓄充盈,漕運無阻,被褒獎,命赴睢工,會同吳璥塞河。二十年,擢河東河道總督。由諫官不三年而膺方面,為時所罕。尋丁母憂,賜金治喪,予諭祭,異數也。服闋,署禮部、兵部侍郎,命赴河南、山東讞獄,並察黃河、運河、湖水情形。二十三年,署廣東巡撫。二十四年,授漕運總督,復調河東河道總督。河決蘭陽、儀封,命偕尚書吳璥治之,鴻賓專駐儀封。會北岸馬營壩復決,合疏言馬營土質沙松,河溜尚勁,未能遽定壩基,被詰責,遂自陳不勝河督之任。詔斥其見吳璥辦工遲緩,慮同獲咎,預為地步,褫職,予郎中銜,留河南專司大工錢糧。二十五年,命營山東運河事務,兼署山東巡撫,專駐張秋,籌備趲運事。尋授安徽巡撫。道光元年,調漕運總督。

二年,擢湖廣總督。初,湖廣行銷淮鹽,用封輪法,大商壟斷,小商向隅,甫改開輪,又有跌價爭售之害。鴻賓請設公司,簽商經理,無論鹽船到岸先後,小商隨到隨售,大商按所到各家計引均銷。試行兩月後,販運踴躍,著為令。時議折漕以資治河,鴻賓疏言徵收折色,弊竇叢生,莫若令民間完交本色,由州縣賣米易銀,轉解河工,詔以易啟抑勒捏價、加收平色諸弊,未允行。

調兩廣總督。廣東通商久,號為利藪。自嘉慶以來,英吉利國勢日強,漸跋扈。故事,十三行洋商有缺,十二家聯保承充,虧帑則攤償。英領事顛地知洋行獲利厚,欲以洋廝容阿華充商,諸商不允,乃賄鴻賓得之。顛地曰:「吾以為總督若何嚴重,詎消數萬金便營私耶!」於是始輕中國官吏。容阿華尋以淫侈耗貲逃,勿獲,官帑無著,不能責諸商代償,乃以抽分法為彌補,眾商藉以漁利,夷情不服,日益多事。鴉片流行日廣,漏銀外洋,鴻賓屢疏陳查禁之法及禁種罌粟,並增築虎門大角砲臺,以資控御,而奉行具文,未有實效。十年,協辦大學士,仍留總督任。

十一年,崖州黎匪亂,鴻賓駐雷州,令提督劉榮慶、總兵孫得發剿平之。給事中劉光三奏廣東匪徒立會滋擾,鴻賓疏陳:「無三點會名目,惟搶劫打單,勒索民財,根株未絕。隨時訪拿,準自首免罪。請廣、潮、肇、嘉諸府州山場荒地,令無業游民報墾,永不升科,庶衣食有資,免流匪僻。」如議行。入覲,賜花翎。十二年春,湖南瑤趙金龍倡亂,廣東連州瑤聞風蠢動,遣兵防剿。五月,鴻賓赴連州,三路進兵,雖有斬獲,兵弁傷亡多,疏請俟湖南事竣進剿,詔斥任賊蔓延;提督劉榮慶衰庸,不早糾劾,嚴議革職,改留任。命尚書禧恩等由湖南移師赴粵剿辦,禧恩言:「粵兵多食鴉片,不耐山險,鴻賓陳奏不實。」褫職逮治,遣戍烏魯木齊。十四年,釋還,予編修。家居久之,二十年,卒。

論曰:宣宗初政,勵精求治。孫玉庭、蔣攸銛並以老成膺分陜之寄,大事多以諮決。其時鹽、河、漕皆積困,玉庭持重,晚稍模棱。攸銛直行己意,眷注遂衰,然其汲引人才,識量遠矣。李鴻賓初以建言驟起,後乃簠簋不飭,貽海疆隱患。三人皆不能以功名終,公私之殊,不可概論也。


\end{pinyinscope}