\article{列傳二百二十三}

\begin{pinyinscope}
沈桂芬李鴻藻翁同龢孫毓汶

沈桂芬,字經笙,順天宛平人,本籍江蘇吳江。道光二十七年進士,選庶吉士,授編修。咸豐二年,大考一等,擢庶子。累遷內閣學士。先後典浙江、廣東鄉試,督陜甘學政,充會試副總裁。八年,丁父憂。服闋,補原官。晉禮部左侍郎。同治二年,出署山西巡撫,明年,實授。連上移屯、練兵諸疏,並稱旨。桂芬以山西民食不敷,自洋藥弛禁,栽種罌粟,糧價踴增。於是刊發條約,飭屬嚴禁。疏陳現辦情形,上韙之,頒行各省,著為令。旋丁母憂。六年,起禮部右侍郎,充經筵講官,命為軍機大臣。歷戶部、吏部,擢都察院左都御史,兼總理各國事務大臣。遷兵部尚書,加太子少保。光緒元年,以本官協辦大學士。京畿旱,編修何金壽援漢代天災策免三公為言,請責斥樞臣,諭交部議。桂芬坐革職,特旨改為革職留任。旋復原官,充翰林院掌院學士,晉太子太保。

桂芬遇事持重,自文祥逝後,以諳究外情稱。日本之滅琉球也,廷論多主戰,桂芬獨言勞師海上,易損國威,力持不可。及與俄人議還伊犁,崇厚擅訂約,朝議紛然;桂芬委曲斡旋,易使往議,改約始定,而言者猶激論不已。桂芬久臥病,六年,卒,年六十有四,贈太子太傅,謚文定。

桂芬躬行謹飭,為軍機大臣十餘年,自奉若寒素,所處極湫隘,而未嘗以清節自矜,人以為難云。

李鴻藻,字蘭孫,直隸高陽人。咸豐二年進士,選庶吉士,授編修。典山西鄉試,督河南學政。十年,上擇儒臣為皇子師,大學士彭蘊章以鴻藻應。召來京,明年,特詔授大阿哥讀。穆宗登極,皇太后懿旨命直弘德殿。同治元年,擢侍講。累遷內閣學士。署戶部左侍郎。四年,命直軍機。五年,授禮部右侍郎。遭母憂,皇太后懿旨,援雍正、乾隆年大臣孫嘉淦等故事,命鴻藻開缺守孝,百日後仍授讀,兼參機務。並諭:「移孝作忠,勿以守禮固辭。」鴻藻懇終制,不允。倭仁等亦代為陳請,仍命恭親王傳諭慰勉。鴻藻連疏稱疾,遂得賜告,卒終制始出。

七年,捻擾畿疆,鴻藻方里居,以各路統兵大員事權不一,疏請特派親王為大將軍,坐鎮京師,以固北路;左宗棠、李鴻章為參贊大臣,分扎保定、河間東西兩路,各率所部兵勇相機剿辦;陳國瑞為幫辦軍務,專統一軍為游擊之師;直隸總督官文專顧省城,籌備諸軍餉需,以資接濟;丁寶楨駐扎直、東交界,防賊東竄;李鶴年駐扎直、豫交界,防賊南竄;直、晉交界,由左宗棠等分撥勁旅扼要駐扎;並請敕下各該大臣和衷商辦,迅奏膚功。奏入,上遂命各路統兵大臣均歸恭親王節制。旋起禮部左侍郎,仍直弘德殿及軍機如故。

十年,擢都察院左都御史,加太子少保。時有修葺圓明園之旨,朝臣同起力爭。鴻藻亦言:「粵、捻初平,回氛方熾,宜培養元氣,以固根本。不應虛糜帑糈,為此不急之務。」乃止。十三年,上有疾,命代批答章奏;旋崩,自劾輔導無狀,罷弘德殿行走。

光緒二年,命兼總理各國事務衙門。尋丁本生母憂,服闋,起故官,以兵部尚書協辦大學士,調吏部。時崇厚與俄擅定伊犁約,鴻藻堅持不可,爭於廷。卒治崇厚罪,議改約。及法越啟釁,言路愈奮發,劾罷樞臣。鴻藻謫遷內閣學士。後復累遷禮部尚書。

十三年,河決鄭州,上命鴻藻馳往督辦。先是河道總督李鶴年、河南巡撫倪文蔚議於西壩興工,鴻藻至,仍之。又續興東壩工。疊遇奇險,皆力為固守。會伏秋汛至,西壩失事,請暫停工。上以鴻藻督率無方,革職留任;並奪李鶴年河道總督,命鴻藻暫行署理。尋回京,復以禮部具奏典禮漏繕簽改日期,再議革職,上特寬免。大婚禮成,復原官。

二十年,日韓事棘,命鴻藻商辦軍務,再授軍機大臣。與翁同龢皆主戰,並爭和約,卒不能阻。旋以禮部尚書協辦大學士,調吏部。歷蒙頒賞書畫及諸上方珍物。充鄉試、會試、殿試等閱卷大臣。二十三年,以病乞假,疾篤,賞給藥餌,命御醫往視。卒,年七十有八。遺疏入,上震悼,予謚文正,贈太子太傅。子焜瀛、煜瀛,均賞給郎中。

鴻藻性至孝,為學守程硃,務實踐,持躬儉約。傅穆宗十餘年,盡心啟沃。一日,穆宗學書,故為戲筆。鴻藻立前捧上手曰:「皇上心不靜,請少息。」穆宗改容謝之。其在樞府,獨守正持大體。御史王鵬運諫止修頤和園,幾獲重譴,鴻藻力解之,得免。德宗間日一往頤和園侍起居,時留駐蹕。言官有言其不便者,太后大怒,欲黜之,鴻藻謂如此必失天下臣民之望,乃止。所薦引多端士。朝列有清望者,率倚以為重,然亦不免被劫持云。

翁同龢,字叔平,江蘇常熟人,大學士心存子。咸豐六年一甲一名進士,授修撰。八年,典試陜甘,旋授陜西學政,乞病回京。同治元年,擢贊善。典山西試。父憂歸,服闋,轉中允。命在弘德殿行走,五日一進講,於簾前說治平寶鑒,兩宮皇太后嘉之。累遷內閣學士。母憂服闋,起故官。同龢居講席,每以憂勤惕厲,啟沃聖心。當八年武英殿之災也,恭錄康熙、嘉慶兩次遇災修省聖訓進御,疏言:「變不虛生,遇災而懼。宜停不急之工,惜無名之費。開直臣忠諫之路,杜小人幸進之門。」上覽奏動容。又圓明園方興工,商人李光昭矇報木價,為李鴻章所劾論罪。廷臣多執此入諫,恭親王等尤力諍,上不懌。同龢面陳江南輿論,中外人心惶惑,請聖意先定,待時興修。乃議定停園工,並有停工程、罷浮費、求直言之諭。

光緒元年,署刑部右侍郎。明年四月,上典學毓慶宮,命授讀,再辭,不允。旋遷戶部,充經筵講官,晉都察院左都御史。遷刑部尚書,調工部。六年,廷臣爭俄約久不決,懿旨派惇親王、醇親王及同龢與潘祖廕每日在南書房看摺件電報,擬片進呈取進止,至俄約改定始止。八年,命充軍機大臣。十年,法越事起,同龢主一面進兵,一面與議,庶有所備。又言劉永福不足恃,非增重兵出關不可。旋與軍機王大臣同罷,仍直毓慶宮。前後充會試總裁、順天鄉試考官,兩蒙賜「壽」,加太子太保,賜雙眼花翎、紫韁。嘗請假修墓,傳旨海上風險,命馳驛回京,恩眷甚篤。

二十年,再授軍機大臣。懿旨命撤講,上請如故。同龢善伺上意,得遇事進言。上親政久,英爽非復常度,剖決精當。每事必問同龢,眷倚尤重。時日韓起釁,同龢與李鴻藻主戰,孫毓汶、徐用儀主和。會海陸軍皆敗,懿旨命赴天津傳諭李鴻章詰責之,同龢並言太后意決不即和。歸薦唐仁廉忠赤可用,請設巡防處籌辦團防。於是命恭親王督辦軍務,同龢、鴻藻等會商辦理。上嘗問諸臣:「時事至此,和戰皆無可恃!」言及宗社,聲淚並發。及和議起,同龢與鴻藻力爭改約稿,並陳:「寧增賠款,必不可割地。」上曰:「臺灣去,則人心皆去。朕何以為天下主?」毓汶以前敵屢敗對,上責以賞罰不嚴,故至於此。諸臣皆引咎。上以和約事徘徊不能決,天顏憔悴。同龢以俄、英、德三國謀阻割地,請展期換約,以待轉圜。與毓汶等執爭,終不可挽,和約遂定。明年,兼總理各國事務大臣。二十三年,以戶部尚書協辦大學士。

二十四年,上初召用主事康有為,議行新政。四月,硃諭:「協辦大學士翁同龢近來辦事多不允協,以致眾論不服,屢經有人參奏。且每於召對時諮詢事件,任意可否,喜怒見於詞色,漸露攬權狂悖情狀,斷難勝樞機之任。本應查明究辦,予以重懲;姑念其在毓慶宮行走有年,不忍遽加嚴譴。翁同龢著即開缺回籍,以示保全。」八月,政變作,太后復訓政。十月,又奉硃諭:「翁同龢授讀以來,輔導無方,往往巧藉事端,刺探朕意。至甲午年中東之役,信口侈陳,任意慫恿。辦理諸務,種種乖謬,以致不可收拾。今春力陳變法,濫保非人,罪無可逭。事後追維,深堪痛恨!前令其開缺回籍,實不足以蔽辜,翁同龢著革職,永不敘用,交地方官嚴加管束。」三十年,卒於家,年七十有五。宣統元年,詔復原官。後追謚文恭。

同龢久侍講幃,參機務,遇事專斷。與左右時有爭執,群責怙權。晚遭讒沮,幾獲不測,遂斥逐以終。著有瓶廬詩稿八卷、文稿二十卷。其書法自成一家,尤為世所宗云。

孫毓汶,字萊山,山東濟寧州人,尚書瑞珍子。咸豐六年,以一甲二名進士授編修。八年,丁父憂。十年,以在籍辦團抗捐被劾,革職遣戍。恭親王以毓汶世受國恩,首抗捐餉,深惡之。同治元年,以輸餉復原官。五年,大考一等一名,擢侍講學士。先後典四川鄉試,督福建學政。光緒元年,丁母憂。服闋,起故官。尋遷詹事,視學安徽。擢內閣學士,授工部左侍郎。十年,命赴江南等省按事。時法越事起,毓汶以習於醇親王,漸與聞機要。適奉硃諭盡罷軍機王大臣,毓汶還,遂命入直軍機,兼總理各國事務大臣。時當國益厭言路紛囂,出張佩綸等會辦南北洋、閩海軍務,餘亦因事先後去之,風氣為之一變。十五年,擢刑部尚書,尋調兵部,加太子少保。歷典會試、順天鄉試,賞黃馬褂、雙眼花翎、紫韁。二十年,中日媾和,李鴻章遣人齎約至。廷臣章奏凡百上,皆斥和非計。翁同龢、李鴻藻主緩,俄、法、德三國亦請毋遽換約。毓汶素與鴻章相結納,力言戰不可恃,亟請署,上為流涕書之,和約遂成。明年,稱疾乞休。二十五年,卒,予謚文恪。

毓汶權奇饒智略,直軍機逾十年。初,醇親王以尊親參機密,不常入直,疏牘日送邸閱,謂之「過府」。諭旨陳奏,皆毓汶為傳達。同列或不得預聞,故其權特重雲。

論曰:光緒初元,復逢訓政,勵精圖治,宰輔多賢,頗有振興之象。首輔文祥既逝,沈桂芬等承其遺風,以忠懇結主知,遇事能持之以正,雖無老成,尚有典型。及甲申法越、甲午日韓,外患內憂,國家多故。慈聖倦勤,經營園囿,稍事游幸,而政紀亦漸弛矣。鴻藻久參樞密,眷遇獨隆。桂芬以持重見賞,同龢以專斷致嫌。毓汶奔走其間,勤勞亦著,大體彌縫,賴以無事。然以政見異同,門戶之爭,牽及朝局,至數十年而未已。賢者之責,亦不能免焉。


\end{pinyinscope}