\article{列傳五十九}

\begin{pinyinscope}
湯若望楊光先南懷仁

湯若望,初名約翰亞當沙耳,姓方白耳氏,日耳曼國人。明萬歷間,利瑪竇挾天算之學入中國,徐光啟與游,盡其術。崇禎初,日食失驗,光啟上言:「臺官用郭守敬法,歷久必差,宜及時修正。」莊烈帝用其議,設局修改歷法,光啟為監督,湯若望被徵入局掌推算。光啟卒,以李天經代,奏進湯若望所著書及心互星屏障。迭與臺官測日食,候節氣,並考定置閏先後,湯若望術輒驗。莊烈帝知西法果密,欲據以改大統術,未行而明亡。

順治元年,睿親王多爾袞定京師,是歲六月,湯若望啟言:「臣於明崇禎二年來京,用西洋新法釐正舊歷,制測量日月星晷、定時考驗諸器。近遭賊毀,擬重制進呈。先將本年八月初一日日食,照新法推步。京師日食限分秒並起復方位,與各省所見不同諸數,開列呈覽。」王命湯若望修正歷法。七月,禮部啟請頒歷,王言:「治歷明時,帝王所重。今用新法正歷,以敬迓天休,宜名時憲歷,用稱朝廷憲天乂民之至意。自順治二年始,即用新歷頒行天下。」湯若望復啟言:「敬授人時,全以節氣交宮,與太陽出入、晝夜時刻為重。今節氣、日時、刻分與太陽出入、晝夜時刻,俱照道里遠近推算,增加歷首,以協民時,利民用。」王獎其精確。八月丙辰朔,日有食之。王令大學士馮銓與湯若望率欽天監官赴觀象臺測驗,惟新法吻合,大統、回回二法時刻俱不協。

世祖定鼎京師,十一月,以湯若望掌欽天監事。湯若望疏辭,上不許。又疏請別給敕印,而以監印繳部,謂治歷之責,學道之志,庶可並行不悖,上亦不許。並諭湯若望遵旨率屬精修歷法,整頓監規,如有怠玩侵紊,即行參奏。加太僕寺卿,尋改太常寺卿。十年三月,賜號通玄教師,敕曰:「國家肇造鴻業,以授時定歷為急務。羲和而後,如漢洛下閎、張衡,唐李淳風、僧一行,於歷法代有損益。元郭守敬號為精密,然經緯之度,尚不能符合天行,其後晷度遂以積差。爾湯若望來自西洋,精於象緯,閎通歷法。徐光啟特薦於朝,一時專家治歷如魏文魁等,實不及爾。但以遠人,多忌成功,終不見用。朕承天眷,定鼎之初,爾為朕修大清時憲歷,迄於有成。又能潔身持行,盡心乃事。今特錫爾嘉名,俾知天生賢人,佐佑定歷,補數千年之闕略,非偶然也。」旋復加通政使,進秩正一品。

欽天監舊設回回科,湯若望用新法,久之,罷回回科不置。十四年四月,革職回回科秋官正吳明炫疏言:「臣祖默沙亦黑等一十八姓,本西域人。自隋開皇己未,抱其歷學,重譯來朝,授職歷官,歷一千五十九載,專管星宿行度。順治三年,掌印湯若望諭臣科,凡日月交食及太陰五星陵犯、天象占驗,俱不必奏進。臣察湯若望推水星二八月皆伏不見,今於二月二十九日仍見東方,又八月二十四日夕見,皆關象占,不敢不據推上聞。乞上復存臣科,庶絕學獲傳。」並上十四年回回術推算太陰五星陵犯書,日月交食、天象占驗圖象。別疏又舉湯若望舛謬三事:一、遺漏紫炁,一、顛倒觜參,一、顛倒羅計。八月,上命內大臣愛星阿及各部院大臣登觀象臺測驗水星不見,議明炫罪,坐奏事詐不以實,律絞,援赦得免。

康熙五年,新安衛官生楊光先叩閽進所著摘謬論、選擇議,斥湯若望新法十謬,並指選擇榮親王葬期誤用洪範五行,下議政王等會同確議。議政王等議:「歷代舊法,每日十二時,分一百刻,新法改九十六刻。康熙三年立春候氣,先期起管,湯若望妄奏春氣已應參、觜二宿,改調次序,四餘刪去紫炁。天祐皇上,歷祚無疆,湯若望祗進二百年歷。選榮親王葬期不用正五行,反用洪範五行,山向年月俱犯忌殺,事犯重大。湯若望及刻漏科杜如預、五官挈壺正楊宏量、歷科李祖白、春官正宋可成、秋官正宋發、冬官正硃光顯、中官正劉有泰皆凌遲處死;故監官子劉必遠、賈文鬱、可成子哲、祖白子實、湯若望義子潘盡孝皆斬。」得旨,湯若望效力多年,又復衰老,杜如預、楊宏量勘定陵地有勞,皆免死,並令覆議。議政王等覆議,湯若望流徙,餘如前議。得旨,湯若望等並免流徙,祖白、可成、發、光顯、有泰皆斬。自是廢新法不用。

聖祖既親政,以南懷仁治理歷法,光先坐譴黜,復用新法。時湯若望已前卒,復通微教師封號,視原品賜恤,改「通玄」曰「通微」,避聖祖諱也。

楊光先,字長公,江南歙縣人。在明時為新安所千戶。崇禎十年,上疏劾大學士溫體仁、給事中陳啟新,舁棺自隨。廷杖,戍遼西。

國初,命湯若望治歷用新法,頒時憲歷書,面題「依西洋新法」五字。光先上書,謂非所宜用。既又論湯若望誤以順治十八年閏十月為閏七月,上所為摘謬、闢邪諸論,攻湯若望甚力,斥所奉天主教為妄言惑眾。聖祖即位,四輔臣執政,頗右光先,下禮、吏二部會鞫。康熙四年,議政王等定讞,盡用光先說,譴湯若望,其屬官至坐死。遂罷新法,復用大統術。除光先右監副,疏辭,不許;即授監正,疏辭,復不許。

光先編次其所為書,命曰不得已,持舊說繩湯若望。顧學術自審不逮遠甚,既屢辭不獲,乃引吳明烜為監副。明烜,明炫兄弟行,明炫議復回回科不得請,至是明烜副光先任推算。五年春,光先疏言:「今候氣法久失傳,十二月中氣不應。乞許臣延訪博學有心計之人,與之制器測候,並飭禮部採宜陽金門山竹管、上黨羊頭山秬黍、河內葭莩備用。」七年,光先復疏言:「律管尺寸,載在史記,而用法失傳。今訪求能候氣者,尚未能致。臣病風痺,未能董理。」下禮部,言光先職監正,不當自諉,仍令訪求能候氣者。

是時朝廷知光先學術不勝任,復用西洋人南懷仁治理歷法。南懷仁疏劾明烜造康熙八年七政民歷於是年十二月置閏,應在康熙九年正月,又一歲兩春分、兩秋分,種種舛誤,下議政王等會議。議政王等議,歷法精微,難以遽定,請命大臣督同測驗。八年,上遣大學士圖海等二十人會監正馬祜測驗立春、雨水兩節氣及太陰火、木二星躔度,南懷仁言悉應,明烜言悉不應。議政王等疏請以康熙九年歷日交南懷仁推算,上問:「光先前劾湯若望,議政王大臣會議,以光先何者為是,湯若望何者為非,及新法當日議停,今日議復,其故安在?」議政王等疏言:「前命大學士圖海等二十人赴觀象臺測驗,南懷仁所言悉應,吳明烜所言悉不應,問監正馬祜,監副宜塔喇、胡振鉞、李光顯,皆言南懷仁歷法上合天象。一日百刻,歷代成法,今南懷仁推算九十六刻,既合天象,自康熙九年始,應按九十六刻推行。南懷仁言羅睺、計都、月孛、推歷所用,故入歷;紫炁無象,推歷所不用,故不入歷。自康熙九年始,紫炁不必造入七政歷。」又言:「候氣為古法,推歷亦無所用,嗣後並應停止。請將光先奪官,交刑部議罪。」上命光先但奪官,免其罪。

南懷仁等復呈告光先依附鼇拜,將歷代所用洪範五行稱為滅蠻經,致李祖白等無辜被戮,援引吳明烜誣告湯若望謀叛。下議政王等議,坐光先斬,上以光先老,貸其死,遣回籍,道卒。刑部議明烜坐奏事不實,當杖流,上命笞四十釋之。

南懷仁,初名佛迪南特斯,姓阜泌斯脫氏,比利時國人。康熙初,入中國。時湯若望方黜,楊光先為監正,吳明烜為監副,以大統術治歷,節氣不應,金、水二星躔度舛錯。明烜奏水星當見,其言復不售。乃召南懷仁,命治理歷法。南懷仁劾光先、明烜而去之,遂授南懷仁監副。

時康熙八年三月,南懷仁言是歲按舊法以十一月置閏,以新法測驗,閏當在九年正月。既又言是月二十九日雨水,乃正月中氣,即為康熙九年之正月,閏當在是年二月。上命禮部詢欽天監官,多從南懷仁,乃罷八年十二月閏,移置九年二月;節氣占候,悉用南懷仁說。六月,南懷仁請改造觀象臺儀器,從之。十二月,儀器成,擢南懷仁監正。儀凡六:曰黃道經緯儀,曰赤道經緯儀,曰地平經儀,曰地平緯儀,曰紀限儀,曰天體儀;並繪圖立說,次為靈臺儀象志。十七年,進康熙永年表,表推七政交食,為湯若望未竟之書,南懷仁續成之。二十一年,命南懷仁至盛京測北極高度,較京師高二度,別為推算日月交食表上之。南懷仁官監正久,累加至工部侍郎。二十七年,卒,謚勤敏。

自是欽天監用西洋人,累進為監正、監副,相繼不絕。五十四年,命紀理安制地平經緯儀,合地平、象限二儀為一。乾隆中,戴進賢、徐懋德、劉松齡、傅作霖皆賜進士。道光間,高拱宸等或歸國,或病卒。時監官已深習西法,不必復用西洋人,奏奉宣宗諭,停西洋人入監。方聖祖用南懷仁,許奉天主教,仍其國俗,而禁各省立堂入教。是時各省天主堂已三十餘所。雍正間,禁令嚴,盡毀去,但留京師一所,俾西洋人入監者居之。入內地傳教,輒繩以法。迨停西洋人入監,未幾海禁弛,傳教入條約,新舊教堂遍內地矣。

論曰:歷算之術,愈入則愈深,愈進則愈密。湯若望、南懷仁所述作,與楊光先所攻訐,淺深疏密,今人人能言之。其在當日,嫉忌遠人,牽涉宗教,引繩批根,互為起僕,誠一時得失之林也。聖祖嘗言當歷法爭議未已,己所未學,不能定是非,乃發憤揅討,卒能深造密微,窮極其閫奧。為天下主,虛己勵學如是。嗚呼,聖矣!


\end{pinyinscope}