\article{列傳二百九十四}

\begin{pinyinscope}
疇人二

李潢汪萊陳傑丁兆慶張福僖時曰淳李銳黎應南

駱騰鳳項名達王大有丁取忠李錫蕃謝家禾吳嘉善

羅士琳易之瀚顧觀光韓應陛左潛曾紀鴻夏鸞翔

鄒伯奇李善蘭華衡芳弟世芳

李潢,字雲門,鍾祥人。乾隆三十六年進士,由翰林官至工部左侍郎。博綜群書,尤精算學,推步律呂,俱臻微妙。著九章算術細草圖說九卷,附海島算經一卷,共十卷。

其自序重差圖云:「圖九,望遠,海島舊有圖解,餘八圖今所補也。同式形兩兩相比,所作四率,二三率相乘,與一四率相乘同積。如欲作圖明之,第取一三率聯為一邊,又取二四率聯為一邊,作相乘長方圖之,自然分為四冪。又以斜弦界為同式句股形各二,則形勢驗矣。舊圖於形外別作同積二方,至兩形相去遼遠者,又必宛轉通之,皆可不必也。圖中以四邊形、五邊形立說,似與句股不類,然於本形外補作句股形,則亦句股也。四率比例法,在九章粟米謂之今有,一為所有率,二為所求率,三為所有數,四為所求數,在句股則統目之為率。劉氏注云:『句率股率,見句見股者是也。』今祗雲同式相比者,取省易耳,異乘同除則一也。」書甫寫定,潢即病。俟吳門沈欽裴算校,方可付梓。越八年,其甥程矞採家為之校刊,以成其志。

九章初經東原戴氏從永樂大典中錄出,一刻於曲阜孔氏,再刻於常熟屈氏,悉依戴氏原校本刊刻。其時古籍甫顯,校訂較難,不無間有捍格,自是天下之習九章者,莫不家★L3一編,奉為圭臬。而劉徽九章亦從此有善本矣。潢又嘗因古算經十書中,九章之外最著者,莫如王孝通之輯古。唐制開科取士,獨輯古四條限以三年,誠以是書隱奧難通。世所傳之長塘鮑氏、曲阜孔氏、羅江李氏各刻本,又悉依汲古閣毛影宋本,祗有原術文而未詳其法,且復傳寫脫誤。雖經陽城張氏以天元一術推演細草,但天元一術創自宋、元時人,究在王氏後,似非此書本旨。爰本九章古義,為之校正,凡其誤者糾之,闕者補之,著考注二卷。以明斜袤廣狹割截附帶分並虛實之原,務如其術乃止。稿未成,潢歿後,為南豐劉衡授其鄉人,以西士開方法增補算草,並附圖解,刻於江西省中,喧賓奪主,殊亂其真。矞採取江西刻本削去圖草,仍以原考注刊布。

武進李兆洛為之序,曰:「輯古何為而作也?蓋闡少廣、商功之蘊而加精焉者也。商功之法,廣袤相乘,又以高若深乘之為立積,今轉以積與差求廣袤高深,所求之數,最小數也。曷為以最小數為所求數?曰,求大數,則實方廉隅,正負雜糅。求小數,則實常為負,方廉隅常為正也。觀臺羨道,築堤穿河,方倉圓囤,芻甍輸粟,其形不一,概以從開立方除之何也?曰,一以貫之之理也。物生而後有象,象而後有滋,滋而後有數。斜解立方,得兩巉堵,一為陽馬,一為鱉臑。陽馬居二,鱉臑居一,不易之率也。今於平地之餘續狹斜之法,無論為巉堵、為陽馬、為鱉臑,皆作立積。觀其立積內不以所求數乘者為減積,以所求數一乘者為方法,再乘者為廉法,所求數再自乘為立方,即隅法也。從開立方除之,得所求數。若繪圖於紙,令廣袤相乘,以所求數從橫截之。剖平冪為若干段,又以截高與所求數乘之。分立積為若干段,若者為減積,若者為方,若者為廉,若者為隅,條段分明,歷歷可指。作者之意,不煩言而解矣。其云廉母自乘為方母,廉母乘方母為實母者之分,開方之要術也。先生於是書立法之根,如鋸解木,如錐劃地,又復補正脫誤,條理秩然,信王氏之功臣矣!爰述大旨,以告世之習是書者,無復苦其難讀云。」

汪萊,字孝嬰,號衡齋,歙縣人。年十五,補博士弟子。弱冠後,讀書於吳葑門外,慕其鄉江文學永、戴庶常震、金殿撰榜、程徵君易疇學,力通經史百家及推步歷算之術。嘉慶十二年,以優貢生入都,考取八旗官學教習,會御史徐國楠奏請續修天文、時憲二志,經大學士首舉萊與徐準宜、許澐入館纂修。十四年,書成。議敘,以本班教職用,選授石埭縣訓導。十八年,應省試,得疾歸,卒於官,年四十有六。先是十一年夏,黃河啟放王營減壩,正溜直注張家河,會六塘河歸海。兩江督臣奉上命,查量雲梯關外舊海口與六塘河新海口地勢高下,延萊測算,蓋其精算之名,久為官卿所知。曾制渾天、簡平、一方各儀器觀測。

與郡人巴樹穀最友善,客江、淮間,又與焦孝廉循、江上舍籓、李秀才銳,辯論宋秦九韶、元李冶立天元一及正、負開方諸法。天性敏絕,極能攻堅,不肯茍於著述。凡所言,皆人所未言,與夫人所不能言。

嘗以古書八線之制,終於三分取一,用益實歸除法求之,其一表之真數,僅得十之二。因悟得五分之一通弦與五分之三通弦交錯為三角形,比例立法,以取五分之一之通弦,而弦切之數益密。梅氏環中黍尺,有以量代算之術,惟求倚平儀外周之兩角,而縮於內半周之角未詳。其法較易,因立新術,量取不倚外周之角度,而三角之量法乃全。堆垛有求平三角、立三角、尖堆積法,不及三乘方以上,又復推而廣之,自三乘、四乘以上之尖堆,皆可由根知積。並及諸物遞兼之法,以補古九章所未備。

又糾正梅文穆公句股知積術,及指識天元一,正、負開方之可知、不可知。其糾正句股知積術也,文穆赤水遺珍稱:「有句股積及股弦和較求句股,向無其術,苦思力索,立法四條。」其門人丁維烈又造減縱翻積開三乘方法,文穆許之。萊謂:「句股形等積、等弦和,帶縱立方形等基、等高闊和,皆有兩形互易。如句二十,股二十一,弦二十九,句弦和四十九,句股積二百一十。若句十二,股三十五,弦三十七,句弦積亦四十九,句股積亦二百一十。設問者暗執一形,則對者交盲兩數。梅、丁諸公法成而不可用,蓋兩句弦較,與一句弦和,恆為連比例之三率。其兩句弦較,即首、末二率;兩較減一和之餘,即中率;而句弦和必為三率人並。遂創立有兩積相等、兩句弦和相等、求兩句股形之法。以四倍句股積自乘,句弦和除之,為帶縱長立方積。以句弦和為縱,開得數為兩句弦較之中率,自乘為帶縱平方積。又以中率與句弦和相減為長闊和,求得長闊兩根為兩句股較,用求兩句股形各數。又同積之邊,彼此可互,三次之乘,先後可通,故四倍句股積自乘,即兩形之倍句相乘為底,兩形之股相乘為高,即猶以中末乘首。中化為中率,再乘為立方三率,人並為帶縱。由是推得立方形兩高數恆為首末二率,高闊和恆為三率,人並數與等積、等弦和之兩弦較及弦和絲毫無異。如高九闊十,高闊和十九,立方積九百。若高四闊十五,高闊和亦十九,立方積亦九百,其數莫不由兩形相引而出。故其法即命積為帶縱長立方積,以高闊和為所帶之縱。用帶縱長立方法開得本方根,為兩形高數之中率。與高闊和相減,餘為帶縱之平方長闊和。中率自乘,為帶縱平方積。用帶縱平方長闊和法開之,得長闊一根,為兩形之兩高數。兩高與和相減,為兩闊數。」

其指識正、負開方也,「元李冶傳洞淵九容術,撰測圓海鏡、益古演段,以明天元如積相消,其究必用正、負開方,互詳於宋秦九韶數學九章。梅文穆公雖指天元一為西人借根方所由來,而正、負開方則未有闡明者。元和李秀才銳特為讎校,謂少廣一章,得此始貫於一。好古之士,翕然相從。萊獨推其有可知、有不可知。如測圓海鏡邊股第五問『圜田求徑二百四十步與五百七十六步共數』,而李仁卿專以二百四十為答。數學九章田域第二題『尖田求積二百四十步與八百四十步共數』,而秦道古專以八百四十為答。乃自二乘方以下,縷析推之,得九十五條。凡幾根數為帶縱長闊較則可知,為帶縱長闊和則不可知。又推得幾真數少,幾根數又多,幾平方與一立方積等多少雜糅,和較莫定。立法以審之,以幾平方數用幾立方數除之,得數乘幾根數,以較幾真數。若少於真數,則以幾平方為高闊較,是為可知。若多於真數,則或幾平方為通分法,三母總數、幾真數為三母維乘之共數,幾根數為通分之共子,如二、如六、如十二。設真數一百四十四,少二百八,根數多二十,平方積與一立方積相等,則三數皆同,是為不可知。」

蓋以一答為可知,不止一答為不可知。故李秀才銳跋其書,括為三例以證明之。謂:「隅實同名者不可知;隅實異名,而從廉正負不雜者可知;隅實異名,而從廉正負相雜,其從翻而與隅同名者可知,否則不可知。隅實異名,即帶縱之長闊較也,較僅一答;隅實同名,即帶縱之長闊和也,和則不止一答。」銳以隅實同名、異名,明一答與不止一答;萊以長闊、和較,明可知、不可知,其義一也。著有衡齋算學七冊,考定通藝錄磬氏倨句解一冊。

陳傑,字靜弇,烏程諸生。考取天文生,任欽天監博士,供職時憲科兼天文科,司測量。累官國子監算學助教。道光十九年,謝病歸,卒於家。生平邃於算學,尤神明於比例之用。初著輯古算經細草一卷,後十餘年,又為之指畫形象,成圖解三卷;又博採訓詁,考正其傳寫之舛譌,稽合各本之同異,別成音義一卷。

其自述比例言有曰:「比例之法,昉自九章,傳由西域,在古法曰異乘同除,在西法曰比例等。假如甲有錢四百,易米二斗,問乙有錢六百,易米幾何?答曰三斗。法以乙錢為實,甲米乘之,得數,甲錢除之,即得。錢與米異名相乘,與錢同名相除,故謂之異乘同除,此古法也。以甲錢比甲米,若乙錢與乙米。凡言以者一率,言比者二率,言若者三率,言與者四率。二三相乘,一率除之,得四率,此西法也。古法元、明時中土幾以失傳,不知何時流入西域。明神宗時,西人利瑪竇來中國,出其所著算書,中人矜為創獲,其實所用皆古法,但異其名色耳。茲以西人名色解王氏,固取其平近,亦以名中、西之合轍也。」

又有論曰:「二十一史律志無不用比例者,他如九章、緝古、十種算書,多用比例,無如古人總不言比例。如緝古第二問,求均給積尺,欲以本體求又一形之體,忽取兩面冪之數,一用以乘,一用以除,而得數。又第九問求員囤,第十問求員窖,忽以周徑乘除,即如方亭法求之,諸數悉得。走作圖解,審諦久之,而始知為比例,乃明言比例以揭之。嗣是而閱古算書者,罔弗比例矣。」

又自道光以來,嘗親在觀象臺督率值班天文生頻年實測黃、赤大距為二十三度二十七分,未經奏明,故當時未敢用。迨甲辰歲修儀象考成續編,監臣即取此數上之,而欽定頒行焉。

晚年所譔為算法大成,上編十卷,首加、減、乘、除,次開方、句股,次比例、八線,次對數,次平三角、弧三角。門分類別,皆先列舊法,而以新法附之,圖說理解,不憚反覆詳明,

專為引誘初學設也。下編十卷,則有目無書。其言曰:「算法之用多端,第一至要為治歷,故下編言在官之事,先治歷,次出師,次工程錢糧,次戶口鹽司,次堆積丈量;儒者則考據經傳,下及商賈庶民,則貲本營運,市廛交易,持家日用,凡事無鉅細,各設題為問答,以明算法之用,蓋如此之廣云。」下編似未成。其門人丁兆慶、張福僖均以算名。

兆慶,字寶書,歸安人。沉潛好學,為項學正兩邊夾角逕求對角新法圖說,謂其講解明晰,戛戛獨造。

福僖,字南坪,烏程諸生。精究小輪之理,著有慧星考略。

時曰淳,字清甫,嘉定人。精算術。發明古人術意,無不入微。咸豐末,與長沙丁取忠同客胡林翼幕府,每與商榷數理,見丁氏數學拾遺之百雞術,謂與二色方程暗合。因為廣衍,立二十八題,以「舊學商量加邃密、新知培養轉深沉」十四字識其上下,為十四耦。諸題皆借方程為本術,並述大衍求一術以博其趣,作百雞術衍二卷。

自序略曰:「張丘建算經雞翁雞母題問,甄、李兩註及劉孝孫草,皆未達術意,不可通。近焦理堂所釋尤誤。讀吾友丁君果臣數學拾遺,設術與二色方程暗合,乃通法也。駱氏藝游錄用大衍求一術,以大小較求中數,取徑頗巧,然遇較除共較實適盡者,則不可求。方程術則遇法除實得中數,不盡者以分母與減率相求而齊同之,無不可得。駱氏殆未知有方程本術耳。夫題祗本經一術,算理之微妙,不如孫子不知數一問,而術文各隱秘。彼則但舉用數,此亦僅著加減三率,於前半段取數之法皆闕如。豈古人不傳之秘,必待學者深思而自得乎?孫子求一術,至宋秦道古發之,獨是題襲謬傳訛,無借方程以問途者。曰淳蓄疑既久,今年春與果臣連榻鄂城,復一商榷,別後數月乃通之。怡然渙然,了無滯凝,亦窮愁中一快事也。因衍方程術為數學拾遺補,求負數法及加減率求答數法,附述求一術為藝游錄補。以中小較求大數法,及大中較、大小較互求得中數、小數法,引伸鉤索,溫故知新,庶足以大暢厥旨乎!易翁、母、雛為大、中、小,設數不必以百,而統以百雞命之者,識斯術所自昉也。」

李銳,字尚之,元和諸生。幼開敏,有過人之資。從書塾中檢得算法統宗,心通其義,遂為九章、八線之學。因受經於錢大昕,得中、西異同之奧,於古歷尤深。自三統以迄授時,悉能洞澈本原。

嘗謂:「三統,世經稱殷術,以元帝初元二年為紀首,是年歲在甲戌。推而上之,一千五百二十歲而歲值甲寅為元首,又上四千五百六十年而歲復甲寅為上元。以此積年,用四分上推,太初元年得至朔同日,而中餘四分日之三,朔餘九百四十分之七百五,故太初術虧四分日之三,去小餘七百五分也。《漢書》載三統而不著太初,其實一月之日,二十九日八十一分日之四十三,是日法、月法與三統同。賈逵稱太初術斗二十六度三百八十五分,是統法周天又與三統同。蓋四分無異於太初,而太初亦得謂之三統。鄭注召誥,周公居攝五年二月三月,當為一月二月,不云正月者,蓋待治定制禮,乃正言正月故也。江徵君聲、王光祿鳴盛以為據洛誥十二月戊辰逆推之,其說未核。今案鄭君精於步算,此破二月三月為一月二月,以緯候入蔀數,推知上推下驗,一一符合,不僅檢勘一二年間事也。」

因據詩大明疏,鄭注尚書文王受命,武王伐紂時日皆用殷歷甲寅元,遂從文王得赤雀受命年起,以乾鑿度所載之積年推算,是年入戊午蔀,二十九年歲在戊午,與劉歆所說殷歷周公六年始入戊午蔀不同。歆謂文王受命九年而崩,崩後四年武王克殷,後七年而崩,明年周公攝政元年,較鄭少一年。又載召誥、洛誥俱攝政七年事,其年二月乙亥朔,三月甲辰朔,十二月戊辰朔,並與鄭不合。乃以推算各年及一月二月,排比干支,分次上下,著召誥日名考,此融會古歷以發明經術者也。

當是時,大昕為當代通儒第一,生平未嘗親許人,獨於銳則以為勝己。大昕嘗以太乙統宗寶鑒求積年術日法一萬五百歲,實三百八十三萬五千四十八分二十五秒為疑。銳據宋同州王湜易學,謂每年於三百六十五日二千四百四十分之外,有終於五分者,有終於六分者,有終於五六分之間者。終於五分者,五代王樸欽天歷是也,以七千二百為日法。終於六分者,近年萬分歷是也,以一萬分為日法。終於五六分之間者,景祐歷法載於太乙遁甲中是也,以一萬五百分為日法,此暗用授時法也。試以日法為一率,歲實為二率,授時日法一萬為三率,推四率,得三百六十五萬二千四百二十五分,即授時之歲實也。探本窮源,一言破的。

近世歷算之學,首推吳江王氏錫闡、宣城梅氏文鼎,嗣則休寧戴氏震亦號名家。王氏謂土盤歷元在唐武德年間,非開皇己未;梅氏謂回回歷實用洪武甲子為元,而託之於開皇己未。其算宮分,雖以開皇己未為元,其查立成之根,則在己未元後二十四年,二說並同。

戴氏謂回回歷百二十八年閏三十一日,是每歲三百六十五日之外,又餘百二十八分日之三十一也。以萬萬乘三十一,滿百二十八而一,得二千四百二十一萬八千七百五十,地谷所定歲實三百六十五日二十三刻三分四十五秒,通分內子以萬萬乘之,滿日法而一,亦得二千四百二十一萬八千七百五十,與梅氏疑問所云合。是三家所論,未嘗不確知灼見,然均未得其詳。銳據明史歷志、回回本術,參以近年瞻禮單,精加考核,謂回回歷有太陽年,彼中謂為宮分;有太陰年,彼中謂為月分。宮分有宮分之元,則開皇己未是也;月分有月分之元,則唐武德壬午是也。自開皇己未至洪武甲子,積宮分年七百八十六,自武德壬午至洪武甲子,積月分年亦七百八十六,其惑人者即此兩積年相等耳,因著回回歷元考。有求宮分白羊一日入月分截元後積年月日法,以為不明乎此,雖有立成,不能入算也。稿佚未刊。

梅氏未見古九章,其所著方程論,率皆以臆創補,然又囿於西學,致悖直除之旨。銳尋究古義,探索本根,變通簡捷,以舊術列於前,別立新術附於後,著方程新術草,以期古法共明於世。古無天元一術,其始見於元李冶測圓海鏡、益古演段二書,元郭守敬用之,以造授時歷草,而明學士顧應祥不解其旨,妄刪細草,遂致是法失傳。自梅文穆悟其即西法之借根方,於是李書乃得鄭重於世。其有原術不通,別設新術數則,更於梅說外辨得天元之相消,有減無加,與借根方之兩邊加減法少有不同。

且不滿顧氏所著之句股、弧矢兩算術,謂:「弧矢肇於九章方田,北宋沈括以兩矢冪求弧背,元李冶用三乘方取矢度,引伸觸類,厥法綦詳。顧氏如積未明,開方徒衍,不亦傎乎?」爰取弧矢十三術,入以天元,著弧矢算術細草。並仿演段例,括句股和較六十餘術,著句股算術細草,以導習天元者之先路。

又從同里顧千里處得秦九韶數學九章,見其亦有天元一之名,而其術則置奇於右上,定於右下,立天元一於左上。先以右上除右下,所得商數與左上相生,入於左下。依次上下相生,至右上末後奇一而止,乃驗左上所得以為乘率。與李書立天元一於太極上,如積求之,得寄左數與同數相消之法不同。因知秦書乃大衍求一中之又一天元,秦與李雖同時,而宋元則南北隔絕,兩家之術,無緣流通,蓋各有所授也。

銳嘗謂:「四時成歲,首載虞書,五紀明歷,見於洪範。歷學誠致治之要,為政之本。乃通典、通考置而不錄,邢雲路雖撰古今律歷考,然徒援經史,以侈卷帙之多。梅氏祗有欲撰歷法通考之議,卒未成書。因更網羅諸史,由黃帝、顓頊、夏、殷、周、魯六歷,下逮元、明數十餘家,一一闡明義蘊,存者表而章之,缺者考而訂之,著為司天通志,俾讀史者啟其扃,治歷者益其智。」惜僅成四分、三統、乾象、奉天、占天五術注而已。餘與開方說皆屬稿未全。

開方說三卷,銳讀秦氏書,見其於超步、退商、正負、加減、借一為隅諸法,頗得古九章少廣之遺,較梅氏少廣拾遺之無方廉者,不可以道里計。蓋梅氏本於同文算指、西鏡錄二書,究出自西法,初不知立方以上無不帶從之方。銳因秦法推廣詳明,以著其說。甫及上、中二卷而卒,年四十有五。其下卷則弟子黎應南續成之。

應南,字見山,號斗一,廣東順德人。嘉慶戊寅順天經魁,以書館議敘,選浙江麗水縣知縣,調平陽縣知縣。海疆俸滿,加六品銜,卒於官。

駱騰鳳,字鳴岡,山陽人。嘉慶六年舉人,道光六年,大挑一等,用知縣。以母老不原仕,改授舒城縣訓導。未一年,告養歸,教授里中,學徒甚眾。二十二年八月,卒於家,年七十有二。性敏銳,好讀書,尤精疇人術。在都中從鍾祥李潢學,研精覃思,寒暑靡間。

著開方釋例四卷,自序略謂:「天元一術,見宋秦九韶大衍數中,不言創於何人。元李冶測圓海鏡、益古演段二書,亦用此例。冶稱其術出於洞淵九容,今不可詳所自矣。是書自平方以至多乘,悉用一術,即芻童、羨餘諸形,亦可握觚而得,洵算術之秘鑰也。西法借根方實原於此,乃以多少代正負,徒欲掩其襲取之跡。不知正負以別異同,多少以分盈朒,毫釐千里,必有能辨之者。」

又著游藝錄二卷,自識云:「餘於正、負開方之例,既為釋例以明其法矣。至於衰分方程、句股等法,以及九章所未載,與夫古今算術之未能該洽者,輒為溯其源,正其誤。不敢掠前哲之美以為名,亦不為黯黮之詞以欺世也。隨所見而識之,匯為一編。」遺稿凡十餘萬言,即今傳本也。

南匯張文虎嘗與青浦熊戶部其光書論之曰:「承示駱司訓算書二種,讀竟奉繳。李四香開方說,詳於超步、商除、翻積、益積諸例,而不言立法之根,令初學者茫不解其所謂。駱氏於諸乘方、方廉、和較、加減之理,皆質言之,而推求各元進退、定商諸術,尤足補李書所未備,誠學開方者之金鎖匙。汪孝嬰創設兩句股同積同句股和一問,以兩句弦較中率轉求兩句弦較,立術迂回。駱氏以正、負開方徑求得兩句,頗為簡易。衡齋亦當首肯也。」其為人所推服如此。

項名達,字梅侶,仁和人。嘉慶二十一年舉人,考授國子監學正。道光六年,成進士,改官知縣,不就,退而專攻算學。三十年,卒於家,年六十有二。著述甚富,今傳世者,但有下學庵句股六術及圖解,復附句股形邊角相求法三十二題,合為一卷。以句股和較相求諸題術稍繁難,爰取舊術稍為變通。分術為六,使題之相同者通為一術,釐然悉有以御之。第一、二、三術及第四術之前二題,悉本舊解,餘為更定新術,皆別注捷法,各為圖解,以明其意。第四、五、六術其原皆出於第三術,可釋之以比例。第三術以句弦較比股,若股與句弦和,以股弦較比句,若句與股弦和,是為三率連比例。凡有比例加減之,其和較亦可互相比例。故第四、五、六術諸題,皆可由第三術之題加減而得,即可因第三術之比例而另生比例。因比例以成同積,而諸術開方之所以然遂明。名達又創有弧三角總較術,求橢員弧線術,術定,未有詮釋,以義奧趣幽,難猝竟事,故六術獨先成云。

名達與烏程陳傑、錢塘戴煦契最深,晚年詣益精進,謂古法無用,不甚涉獵,而專意於平弧三角,與傑意不謀而合。與傑論平三角,名達曰:「平三角二邊夾一角,逕求斜角對邊,向無其法,竊嘗擬而得之,君聞之乎?」傑曰:「未也。」錄其法以歸。蓋以甲乙邊自乘與甲丙邊自乘相加,得數寄左;乃以半徑為一率,甲角餘弦為二率,甲乙、甲丙兩邊相乘倍之為三率,求得四率,與寄左數相減,鈍角則相加,平方開之,得數即乙丙邊。

又嘗謂泰西杜德美之割圜九術,理精法妙,其原本於三角堆,董方立定四術以明之,洵為卓見。惟求倍分弧,有奇無偶,徐有壬補之,庶幾詳備。名達嘗玩三角堆,嘆其數祗一遞加,而理法象數,包蘊無窮,夫方圜之率不相通,通方圜者必以尖,句股,尖象也;三角堆,尖數也。古法用半徑屢求句股得圜周,不勝其繁。杜氏則以三角堆御連比例諸率,而弧弦可以互通,割圜術蔑以加矣。然以此制八線全表,每求一數,必乘除兩次,所用弧線,位多而乘不便,董、徐二氏大、小弧相求法亦然。向思別立簡易法,因從三角堆整數中推出零數,但用半徑,即可任求幾度分秒之正餘弦,不煩取資於弧線及他弧弦矢。且每一乘除,便得一數,似可為制表之一助。

又著象數原始一書,未竟,疾革時,囑戴煦。後煦索稿於名達子錦標,校算增訂六閱月而稿始定,都為七卷。原書之四,僅六紙,並第七卷皆煦所補也。卷一曰整分起度弦矢率論,卷二曰半分起度弦矢率論,卷三、卷四曰零分起度弦矢率論,皆以兩等邊三角形明其象,遞加法定其數,末乃申論其算法。卷五曰諸術通詮,取新立弧弦矢求他弧弦矢二術、半徑求弦矢二術及杜、董諸術,按術詮釋之。卷六曰諸術明變,雜列所定弦矢求八線術,開諸乘方捷術,算律管新術,橢員求周術,以明皆從遞加數轉變而得。卷七曰橢員求周圖解,原術以袤為徑,求大員周及周較,相減而得周,補術則以廣為徑,求小員周,周較相加而得周,末系以圖解。徐有壬巡撫江蘇,郵書索煦寫定本梓行,刻甫就而有壬殉難,書與板皆毀焉。

有王大有者,字吉甫,仁和諸生。翰林院待詔。窮究天算,問業於處士戴煦。凡煦所著述,皆錄副本去,名達見之,因與煦訂交。大有嘗校割圜捷術合編。後殉於杭州。

丁取忠,字果臣,長沙人。研究象數,不求聞達,刻算書二十有一種,為白芙堂叢書。光緒初,卒於家,年逾七十。所自譔者為數學拾遺一卷,以所演算草較詳,可便初學,又意在拾遺,故未暇詳其義之出自何人。

又譔粟布演草二卷,自序曰:「道光壬辰,餘始習算,友人羅寅交學博洪賓以難題見詢,久無以應。同治初元,始獲交南豐吳君子登太史,馭以開屢乘方法,餘始通其術,然未悉其立法之根也。後吳君游嶺表,餘推之他題,及展轉相求,仍多窒礙。又函詢李君壬叔,蒙示以廉法表及求總率二術,而其理始顯。後吳君又示以指數表及開方式表,李君復為之圖解以闡其義。由是三事互求,理歸一貫。餘因取數題詳為演草,並捷法圖解,都為一卷。質之南海鄒君特夫,君復為增訂開屢乘方法,並另設題演草,補所未備。即算家至精之理,如圜內容各等邊形,皆可借發商生息以明之,誠快事也!」

後又譔演草補一篇,序云:「余前年與左君壬叟共輯粟布演草,原為商賈之習算者設,或一例而演數題,或一題而更數式。或用真數,或用代數。其式或橫列,或直下,雜然並陳,無非欲學者比類參觀,易於領悟也。乃初學習之,猶謂茫無入門處,蓋商賈所習算書,大都詳於文而略於式。況代數又古算術所無,宜其卒然覽之而不解也。茲更擬一題附後,特仿數理精蘊借根方體例,專詳於文,庶初學讀之,可因文知義。算理既明,則全書各式,可渙然冰釋,或兼可為習代數者之先導乎?」其鄉人李錫蕃,亦以演算名。

錫蕃,字晉夫。道光三十年早卒,著有借根方句股細草一卷,衍為二十有五術,取忠刊入叢書。

謝家禾,字和甫,錢塘舉人。與同學戴氏兄弟熙、煦相友善。少嗜西學,點線面體四部,靡不淹貫。已,復取元初諸家算書,幽探冥索,悉其秘奧。乃輯平時所得析通分加減,定方程正負,以標舉立元大耍,撰演元耍義一卷。其自序云:「元學至精且邃,而求其要領,無過通分加減,凡四元之分正負,及相消法,互隱通分法,大致原於方程。方程者,即通分之義。方程不明,由於正負無定例,加減無定行,以譌傳譌,如梅宣城精研數理,未暇深究,他書可知矣。九章算經正負術甚明,而釋者反以意度,古誼之不明,可勝道哉!唯以衍元之法正方程之義,由是方程明而元學亦明。著演元要義,綜通分方程而論列之,附以連枝同體之分等法。通乎此,則四元庶可窺其涯涘耳。」

又以劉徽、祖沖之之率求弧田,求其密於古率者,撰弧田問率一卷。同里戴煦為之序曰:「古率徑一周三,徽率劉徽所定,徑五十周一百五十七也。密率乃祖沖之簡率,徑七周二十二也。諸書弧田術皆用古率,郭太史以二至相距四十八度,求矢亦用古法。顧徽、密二率之周既盈於古,則積亦盈於古,試設同徑之圓,旁割四弧,其中兩弦相得之方三率皆同,知三率圓積之盈縮,正三率弧積之盈縮也。徽、密二率弧田古無其術,惟四元玉鑒一睹其名,而設問隱晦,莫可端倪。穀堂得其旨,因依李尚之孤矢算術細草設問立術,亦足發前人所未發也。」

又以直橫與句股弦和較展轉相求,撰直積回求一卷,其自序云:「始戴諤士著句股和較集成,予亦著直積與和較求句股弦之書,然二書為義尚淺,且直積與句弦和求三事,用立方三乘方等,得數不易,而又不足以為率,其書遂不存。近見四元玉鑒直積與和較回求之法,多立二元,嘗與諤士思其義蘊,有不必用二元者。蓋以句弦較與句弦和相乘為股冪,股弦和與股弦較相乘為句冪,而直積自乘,即句冪股冪相乘也。如以句弦較乘股弦較冪,除直積冪,即為句弦和乘股弦和冪矣。句弦和乘股弦和冪,即弦冪和冪共內少半個黃方冪也。蓋相乘冪內去一弦冪,所餘為句股相乘者一,句弦相乘者一,股弦相乘者一,此三冪合成和冪,則少一半黃方冪。半黃方冪,即句弦較股弦較相乘冪也。加一半黃方冪,即為弦冪和冪共矣。加二直積,即二和冪也。減六直積,即二較冪也。又句弦和乘股弦較冪,為句冪內少個句股較乘股弦較冪也。股弦和乘句弦較冪,為股冪內多個句股較乘句弦較冪也。減一句股較乘股弦較冪,尚餘一句股較冪矣。術中精意,皆出於此。其他之參用常法者,可不解而自明耳。草中既未暇論,恐習者不知其理,因揭其大旨於簡端,見演段之不可不精也。」

家禾歿後,戴熙搜遺稿,囑其弟煦校讎而授諸梓。煦精算,見忠義傳。著有補重差圖說,句股和較集成消法簡易圖解,對數簡法,外切密率,假數測圓,及船機圖說等。

吳嘉善,字子登,南豐人。咸豐十一年進士,改翰林院庶吉士,散館授編修。與徐有壬同治算學。同治改元,避粵匪亂游長沙,識丁取忠。逾年,客廣州,因鄒伯奇又識錢塘夏鸞翔。三人志同道合,相得益彰。光緒五年,奉使法蘭西,駐巴黎。後受代還,旋卒。

所譔算書,首述筆算。次九章翼,曰今有術,曰分法,曰開方,曰平方平員各術。推演方田者,曰立方立員術,推演商功者,曰句股,曰衰分術,曰盈不足術,曰方程術。於句股術後,次附平三角、弧三角測量高遠之術。又次則專述天元四元之書,為天元一術釋例,為名式釋例,為天元一草,為天元問答,為方程天元合釋,為四元名式釋例並草,為四元淺釋。自序曰:「算學至今日,可謂盛矣。古義既彰,新法日出,前此所未有也。餘與丁君果臣皆癖此,既忘其癖,更欲以癖導人。嘗苦近世津逮初學之書無善本,梅文穆公所刪之算法統宗,今亦不傳。因商榷述此,取其淺近易曉,以為升高行遠之助云。」

羅士琳,字茗香,甘泉人。以監生循例貢太學,嘗考取天文生。咸豐元年,恩詔徵舉孝廉方正之士,郡縣交薦,以老病辭。三年春,粵匪陷揚州,死之,年垂七十矣。少治經,從其舅江都秦太史恩復受舉業,已乃棄去,專力步算,博覽疇人書,日夕研求數年。

初精西法,自譔言歷法者曰憲法一隅。又思句股、少廣相表裏,而方田與商功無異,差分與均輸不殊。按類相從,摘九章中之切於日用者,悉以比例馭之,匯為十二種。以各定率冠首,以借根方繼後,以諸乘方開法附末,凡四卷,曰比例匯通,雖悔其少作,實便初學問途。

後見四元玉鑒,服膺嘆絕,遂壹意專精四元之術。士琳博文強識,兼綜百家,於古今算法尤具神解,以硃氏此書實集算學大成,思通行發明,乃殫精一紀,步為全草,並有原書於率不通及步算傳寫之譌,悉為標出,補漏正誤,反覆設例,申明疑義,推演訂證。就原書三卷二十有四門,廣為二十四卷,門各補草。

嘗為提要鉤元之論,謂:「是書通體弗出九章範圍,不獨商功修築、句股測望、方程正負已也。如端匹互隱、廩粟回求寓粟布,如意混和寓借衰,茭草形段、果垛疊藏,如像招數寓商功中之差分,直段求源、混積問元、明積演段、撥換截田、鎖套吞容寓方田、少廣諸法。他若分索隱之為約分命分,方員交錯、三率究員、箭積交參之為定率兼交互。至於或問歌彖、雜範類會,以其各自為法,不能比類。故一則寄諸歌詞,一則編成雜法,均似補遺。大旨皆以加、減、乘、除、開方、帶分六例為問,每門必備此例,略簡易而詳繁難,尤於自來算書所無者,必設二問以明之。如混積問元中既設種金田及句三股四八角田為問。撥換截田中復設半種金田,鎖套吞容中復設方五斜七八角田為問。又果垛疊藏兩設員錐垛,雜範類會既設徽率割員,又設密率割員是矣。更有一門專明一義者,如和分索隱之分開方,三率究員兩儀合轍之反覆互求是矣。是書但云如積求之,如積有用定率為同數相消者,有如問加減乘除得積為同數相消者。祖序謂:『平水劉汝諧撰如積釋鎖,惜今不傳。』意者其釋此例歟?」

道光中,得硃氏算學啟蒙於京師廠肆,士琳復加斠詮刊布之。此書總二十門,凡二百五十九問,其名術義例多與玉鑒相表裏。士琳為之互斠,始於天元,終於四元,義主精邃,所得甚深。考大德四年莫若序,計後此書四年。此書首列乘除布算諸例,始於超徑等接之術,終於天元如積開方,由淺近以至通變,循序漸進,其理易知。名曰啟蒙,實則為玉鑒立術之根,此一證也。玉鑒原本十行,行十九字,「今有」低一格,「術曰」又低二格,與此書同,此二證也。玉鑒斗斛之「斗」別作「」,此假借字,本漢書平帝紀及管子乘馬篇,尚雜見於唐以前之孫子、五曹、張丘建諸算經,鈞石之「石」,說文本作「柘」,玉鑒作「碩」,「碩」「石」古雖互通,然假「碩」為「石」,則僅見於毛詩甫田疏引漢書食貨志,而算書罕見,又玉鑒田之「」,雖見李籍九章音義,為字書所無,此書並同,此三證也。玉鑒雖亦三卷,而門則為二十四,問則二百八十八,較多此書四門二十九問,然以四字分類,其體裁同。且如商功、修築、方程、正負之屬,則又二書互見,此四證也。玉鑒如意混和第一問,據數知一秤為十五斤,適與此書之斤秤起率合,此五證也。玉鑒鎖套吞容第九問,方五斜七八角田左右逢元第六、第十三、第二十諸問,有小平小長,皆向無其術。此書卷首明乘除段,即載平除長為小長,長除平為小平之例。其田畝形段第十五問,復載方五斜七八角田求積通術,此六證也。他如玉鑒或問歌彖第四問,與此書盈不足術第七問,又玉鑒果垛疊藏第十四問,與此書堆積還原第十四問,又玉鑒方程正負第四問,與此書方程正負第五問,題皆約略相同,此七證也。知系硃氏原書佚而復出,並其算法一則,亦為附列,間有魚豕,悉仍其舊,但各標識於誤字旁,別記刊誤於卷末。

又嘗以乾隆間明氏捷法校得八線對數表,一度十三分二十秒正切第五字「0」誤「一」;又六度四十一分十秒正切第五字「0」誤「六」;又十二度五十分正弦第六字「七」誤「五」;又十六度三十二分十秒正切第七字「九」誤「0」;又四十二度三十二分四秒正切第九字「五」誤「四」。可見西人所能,中人亦能之。

又因會通四元玉鑒如像招數一門,更取明氏捷法,御以天元,知密率亦可招差,其弧與弦矢互求之法,與授時歷之垛積招差一一符合。且以祖氏綴術失傳,其法廑見於秦書,即大衍之連環求等遞減遞加,亦與明氏捷法相近。爰融會諸家法意,撰綴術輯補二卷。

又甄錄古今疇人,仍阮氏體例為列傳,採前傳所未收者,得補遺十二人,附見五人,續補二十人,附見七人,合共四十有四人,次於前傳四十六卷之後。

集所校著都為觀我生室匯十二種。如四元玉鑒細草二十四卷,釋例二卷,校正算學啟蒙三卷,校正割圜密率捷法四卷,續疇人傳六卷,皆別有單行本。

外已刻者尚得七種,曰句股容三事拾遺三卷,附例一卷,本繪亭監副博啟法補其遺,取內容方邊員徑垂線交互相求,一以天元馭之。曰三角和較算例一卷,取斜平三角形中兩邊夾一角術鎔入天元法,用和較推演成式。曰演元九式一卷,括玉鑒中進退消長諸例,借無數之數,以正負開方式入之。曰臺錐積演一卷,以玉鑒茭草、果垛二門可補少廣之闕,爰取臺錐形段引而伸之。曰周無專鼎銘考一卷,以四分周術佐以三統漢術,推得宣王十有六年九月既望甲戌,與銘辭正合。曰弧矢算術補一卷,以元和李四香原術未備,為增補二十七術,合成四十術。曰推算日食增廣新術一卷,推廣正升斜升橫升之算法,以求太陰隨地隨時之明魄方向分秒,復推其術,以求交食限內之方向,及所經歷之諸邊分。

餘若春秋朔閏異同考、綴術輯補交食圖說舉隅、句股截積和較算例、淮南天文訓存疑、博能叢話,凡若干卷,未有刻本。其同縣友有易之瀚者,亦以算名。

易之瀚,字浩川。知士琳有四元玉鑒補草,因從問難,為撰四元釋例一卷。凡開方例二十九則,天元例十一則,四元例十三則。

顧觀光,字尚之,金山人。太學生,三試不售,遂無志科舉,承世業為醫。鄉錢氏多藏書,恆假讀之。博通經、傳、史、子、百家,尤究極天文歷算,因端竟委,能抉其所以然,而摘其不盡然。時復蹈瑕抵隙,蒐補其未備。如據周髀「笠以寫天,青黃丹黑」之文及後文「凡為此圖」雲云,而悟篇中周徑里數皆為繪圖而設。天本渾員,以視法變為平員,則不得不以北極為心,而內外衡以次環之,皆為借象,而非真以平員測天也。

開元占經魯歷積年之算不合,因用演積術,推其上元庚子至開元二年歲積,知占經少三千六十年。又以占經顓頊歷歲積考之史記秦始皇本紀,知其術雖起立春,而以小雪距朔之日為斷。蓋秦以十月為歲首,閏在歲終,故小雪必在十月,昔人未及言也。李尚之用何承天調日法考古歷日法朔餘強弱不合者十六家,以為未能推算入微。爰別立術,以日法朔餘展轉相減,以得強弱之數。但使日法在百萬以上皆可求,惟朔餘過於強率者不可算耳。授時術以平定立三差求太陽盈縮,梅氏詳說未明其故。讀明志乃知即三色方程之法。謂凡兩數升降有差,彼此遞減,必得一齊同之數。引而伸之,即諸乘差,則八線、對數、小輪、橢員諸術,皆可共貫。讀占經所載瞿曇悉達九執術,知回回、太西歷法皆源於此。其所謂高月者即月孛,月藏者即月引數,日藏者即日引數,特稱名不同,亦猶回歷稱歲實為宮日數,朔策為月分日數也。

其論婺源江氏冬至權度,推劉宋大明五年十一月乙酉冬至前以壬戌丁未二日景求太陽實經度,而後求兩心差,乃專用壬戌。今用丁未求得兩心差,適與江氏古大今小之說相反。蓋偏取一端,其根誤在高沖行太疾也。西法用實朔距緯求食甚兩心實相距,術繁而得數未確。改以前後兩設時求食甚實引徑得兩心實相距,不必更資實朔,較本法為簡而密矣。

西人割圜,止知內容各等邊之半為正弦,而不知外切各等邊之半為正切。乃依六宗、三要、二簡諸術,別立求外切各等邊之正切法,以補其缺。杜德美求員周術,用員內容六邊形起算,巧而降位稍遲,謂內容十等邊之一邊,即理分中末線之大分,距周較近。且十邊形之邊與周同數,不過遞進一位,而大分與全分相減即得小分,則連比例各率,可以較數取之。入算尤簡易,可用弧度入算,不用弧背真數。然猶慮其難記,仍不能無藉於表,因又合兩法用之,則術愈簡,而弧線、直線相求之理始盡。錢塘項氏割圜捷術,止有弦矢求餘線術,以為可通之割、切二線,因補其術。西人求對數,以正數屢次開方,對數屢次折半,立術繁重。李氏探原以尖錐發其覆,捷矣,而布算術猶繁。且所得者皆前後兩數之較,可以造表而不可徑求。戴氏簡法及西人數學啟蒙,又有新術,而未窮其理。乃變通以求二至九之八對數,因任意設數,立六術以御之,得數皆合。復立還原四術,並推衍為和較相求八術,為自來言對數者所未有也。又謂對數之用,莫便於八線,而西人未言其立表之根,因冥思力索,仍用諸乘方差,迎刃而解,尤晚歲造微之詣也。其它凡近時新譯西術,如代數、微分、諸重學,皆有所糾正,類此。

所著曰算賸初、續編凡二卷。曰九數存古,依九章分為九卷,而以堆垛、大衍、四元、旁要、重差、夕桀、割圜、弧矢諸術附焉,皆採古書而分門隸之。曰九數外錄,則隱括四術為對數、割圜、八線、平三角、弧三角各等面體、員錐三曲線、靜重學、動重學、流質重學、天重學,凡記十篇。曰六歷通考,則據占經所紀黃帝、顓頊、夏、殷、周、魯積年而加以考證。曰九執歷解,曰回回歷解,皆就原法疏通證明之。曰推步簡法,曰新歷推步簡法,曰五星簡法,則就原術改度為百分,省迂回而歸簡易,蓋於學實事求是,無門戶異同之見,故析理甚精,而談算為最云。其友人韓應陛,亦以表章算書顯。

應陛,字對虞,婁縣人。道光二十四年舉人,官內閣中書舍人。少好讀周、秦諸子,為文古質簡奧,非時俗所尚。既而從同里姚椿游,得望溪、惜抱相傳古文義法。西人所創點、線、面、體之學,為幾何原本,凡十五卷,明萬歷間利譯止前六卷。咸豐初,英人偉烈亞力續譯後九卷,海寧李壬叔寫而傳之。應陛反覆審訂,授之剞劂,亞力以為泰西舊本弗及也。外若新譯重、氣、聲、光諸學,應陛推極其致,往往為西人所未及雲。

左潛,字壬叔,大學士宗棠從子。補縣學生。於詩、古文辭無不深造,尤明算理。長沙丁取忠引為忘年交。早卒,士林惜之。所學自大衍、天元及借根方、比例諸新法,無不貫通。且能自出己意,變其式,勘其誤,作為圖解,往往突過先民。嘗增訂徐有壬割圜綴術,既成,忽悟通分捷法析分母、分子為極小數,根同者去之,凡多項通分,頃刻立就。因演數草,為通分捷法一帙。

所譔綴術補草四卷,自序曰:「自泰西杜德美創立割圜九術,以屢乘屢除通方圜之率,我朝明氏、董氏各為之說,而杜書之義,推闡靡遺。顧八線互求,尚無通術,未足以盡一圜之變,非明氏、董氏之智力,不能因法立以盡其變也。其能窮杜氏之義也,資於借根方;其不能廣杜氏之法也,亦限於借根方。蓋借根方即天元一之變術,究不如元術之巧變莫測也。是書祖杜宗明,又旁參以董氏之法,八線相求,各立一式,因式立法,因法入算。鄉之不可立算者,今皆能馭之以法,即有不能立法布算者,而其式存,則能濟法之窮;而度圜諸線,一以貫之矣。推其立式之由,所謂比例術,即明氏定半徑為一率,所有為二率或三率之法也。所謂還原術,即明氏弧背求正矢,又以正矢求弧背之法也。所謂借徑術,即明氏借十分全弧通弦率數求百分全弧通弦率數,求千分全弧通弦率數諸法也。所謂商除法,又即還原術之變法。是故綴術胎於明氏,而又足以盡明氏之變。明氏之未立式者,以借根方取兩等數,其分母、分子雜糅繁重,既不可通,其多號、少號,展轉互變,又不可約。試取明氏書馭之以綴術,其遞降各率,頃刻可求。則是書也,其真能因法立法,別樹幟於明、董之後者歟?書為徐君青先生所作,吳君子登成之,顧詳於式而略於草。敬考其立法之原,不可遽得,學者難焉,潛因於暇日為補草四卷,因綴數語於簡端云。」

又譔綴術釋明二卷,湘鄉曾紀鴻為之序,略曰:「易系云:『極其數遂定天下之象。』則綜天下難定之象以歸有定,莫數若矣。在昔聖神,制器尚象,利物前民,於數理必有究極精微,範圍後世者,代久年湮,漸至失傳。近三百年,泰西猶能推闡古法,而中國才智之士,或反率其成轍。孔子曰:『天子失官,學在四夷。』正今日數學之謂也。中國舊有弧矢算術,而未標角度八線鈐表,則雖有用其理以入算者,而無表可檢。則每求一數,必百倍其功,而所得數仍非密率。明代譯出泰西八線表及八線對數表,覈其立法得數之原,甚屬繁難,而成表之後,一勞永逸。大至無外,細及極微,莫不以此表測之,則其用之廣大可想。然得表之後,雖無事於再求,而任舉一數,無從較其訛誤。若仍用舊術,則非★月經旬,不能得一數,此明靜菴、董方立推演杜德美弧矢捷術之所以可貴也。向來求八線者,例用六宗、三耍、二簡各法,若任言一弧,必不能考其弦矢諸數。至杜氏創立屢乘屢除之法,則但有弧徑,而八線均可求。董方立解杜術,先取其線之極微者,令與與弧線合,而後用連比例以推至極大。又考諸率數與尖錐理相合,故用尖錐以釋弧矢,而弧矢之數理以顯。明靜菴解杜術,先取四分弧與十分弧之通弦直線之極大者,用連比例以推至千分、萬分弧通弦之極微者,考其乘除之率數,與杜術乘除之原理合,故用綴術以釋弧矢,而弧矢之數理亦出。董、明二氏,均為弧矢不祧之宗,無庸軒輊。邇百年中繼起者,如戴、徐、李三氏所著書,雖自出心裁,要皆奉董、明為師資也。吾友左君壬叟,於數學尤孜孜不倦,遇有疑難,必窮力追索,務洞澈其奧窔。嘗謂方員之理,乃天地自然之數,吾之宗中宗西,不必分畛域,直以為自得新法也可。曾釋君青徐氏綴術,又釋戴鄂士求表捷術,茲又釋明靜菴弧矢捷術,而一貫以天元寄分之式,於員率一道三致意焉,可謂勤矣。孰意天厄良才,壬叟竟於甲戌秋不永年而逝,凡在同人,無不嘆惜!況餘與之為兩世神交,安能無愴切耶!」

曾紀鴻,字慄誠,大學士國籓少子。恩賞舉人。早卒。紀鴻少年好學,與兄紀澤並精算術,尤神明於西人代數術。銳思勇進,創立新法,同輩多心折焉。謂大衍求一術亦可以代數推求,依題演之,理正相通,撰對數詳解五卷,始明代數之理,為不知代數者開其先路。中言對數之理,末言對數之用,明作書之本意。其於常對、訥對,辨析分明。先求得各真數之訥對,復以對數根乘之,即為常對數。級數朗然,有條不紊,雖初學循序漸進,無不可相說以解焉。

夏鸞翔,字紫笙,錢塘人。以輸餉議敘,得詹事府主簿。為項梅侶入室弟子。講究曲線諸術,洞悉員出於方之理。匯通各法,推演以盡其變,譔洞方術圖解二卷,自序略曰:「自杜氏術出,而求弦矢得捷徑焉。顧猶煩乘除,演算終不易,思一可省乘除之法而迄未得。丁巳夏,客都門,細思連比例術者,尖堆底也。尖堆底之比例,與諸乘方之比例等。以之求連比例術,必合諸乘方積而並求之。設不得諸乘方積遞差之故,方積何能並求?且並求方積而欲以加減代乘除,又必得諸較自然之數而後可,誠極難矣。既而悟曰,方積之遞加,加以較也。較之遞生,生於三角堆也。較加較而成積,亦較加較而成較。且諸乘方積之數與諸乘尖堆之數,數異而理同。三角堆起於三角形,故屢次增乘,皆增以三角。方積起於正方形,故累次增乘,皆增以正方。三角之較數,增一根則增一較;方積之較數,增一乘則增一較,理正同也。累次相較,較必有盡,惟其有盡,乃可入算。相連諸弦矢所以愈相較而較愈均者,正此理矣。諸較之理,皆起於天元一,而生於根差。遞加根一,諸乘方根差皆一。一乘之數不變,故可省乘。若增其根差,非復單一,則乘不能省。弦矢弧背之差,或一秒,或十秒,即以一秒、十秒弧線當根差,按根遞求,即可盡得諸乘方之較。以較加較,即盡得所求弦矢各數矣,豈不捷哉!爰演為求弦矢術,俾求表者得以加減代乘除。並細繹立術之義,以俟精於術數者採擇。」

又譔致曲術一卷,曰平員,曰橢員,曰拋物線,曰雙曲線,曰擺線,曰對數曲線,曰螺線,凡七類。類皆自定新術,參差並列,法密理精。復著致曲圖解一卷,謂天為大員,天之賦物,莫不以員。顧員雖一名,形乃萬類。循員一匝,而曲線生焉。西人以線所生之次數分為諸類,一次式為直線;二次式有平員、橢員、拋物線、雙曲線四式;三次式有八十種;四次式有五千餘種;五次以上,殆難以數計矣。今但二次式四種,溯其本源,並附解諸乘方。拋物線形雖萬殊,理實一貫。諸曲線式備具於員錐體,員錐者,二次曲線之母也。橢員利用聚,拋物線利用遠,雙曲線利用散,其理皆出於平員。茍會其通,則制器尚象,仰觀俯察,為用無窮矣。今為一一解之,其目為諸曲線始於一點終於一點第一,諸式之心第二,準線第三,規線第四,橫直二徑第五,兌徑亦名相屬二徑第六,兩心差第七,法線切線第八,斜規線又名曲率徑第九,縱橫線式第十,諸式互為比例第十一,八線第十二。

又嘗立捷術以開各乘方,不論益積、翻積,通為一術,俱為坦途,可徑求平方根數十位,成少廣縋鑿一卷。

鸞翔同治三年卒。因方積之較而悟求求弦矢之術,駸駸乎駕西人而上之,然微分所棄之常數,猶方積之方與隅也。所求之變數,猶兩廉遞加之較也。其術施之曲線,無所不通,鸞翔猶待逐類立術,是則不能不讓西人以獨步。然西法開方,自三次式以上,皆枝枝節節為之,不及中法之一貫。鸞翔又於中法外獨創捷術,非西人所能望其項背雲。

鄒伯奇,字特夫,南海諸生。聰敏絕世,覃思聲音文字度數之源。尤精天文歷算,能薈萃中、西之說而貫通之,靜極生明,多具神解。嘗作春秋經傳日月考,謂:「昔人考春秋者多矣,類以經、傳日月求之,未能精確。今以時憲術上推二百四十二年之朔閏及食限,然後以經、傳所書,質其合否,乃知有經誤、傳誤及術誤之分。」又謂:「尚書克殷年月,鄭玄據乾鑿度,以入戊午蔀四十二年克殷,下至春秋,凡三百四十八年。劉歆三統術以為積四百年,近人錢塘李銳皆主其說。今以時憲術上推,且以歲星驗之,始知鄭是劉非。」其解孟子「由周而來,七百有餘歲」句,謂閻百詩孟子生卒年月考據大事記及通鑒綱目,以孟子致為臣而歸在周赧王元年丁未,逆數至武王有天下,歲在己卯,當得八百有九年。然周共和以上年數,史遷已不能紀,可考者魯世家耳,此為劉歆歷譜所據。然將歆譜與史記比對,歆於煬公、獻公等年分多所加,共計五十二。若減其所加,則歆所謂八百有九年者,實七百五十七年耳。

又謂向來注經者,於算學不盡精通,故解三禮制度多疏失,因作深衣考,以訂江永之謬。作戈戟考,以指程瑤田之疏。以文選景福殿賦「陽馬承阿」證古宮室阿棟之制。以體積論樐氏為量,以重心論懸磬之形,皆繪圖立說,援引詳明。

又嘗謂群經注疏引算術未能簡要,甄鸞五經算術既多疏略,王伯厚六經天文篇博引傳注,亦無辨證。因即經義中有關於天文、算術,為先儒所未發,或發而未闡明者,隨時錄出,成學計一得二卷。

天象著甲寅恆星表、赤道星圖、黃道星圖各一卷,自序略曰:「甲寅春,制渾球,以考證經史恆星出沒歷代歲差之故。然制器必先繪圖,繪圖必先立表,此恆星表之所由作也。史、漢、晉、隋諸志,於恆星但言部位,至唐、宋始略有去極度數,蓋舊傳新圖,大抵據步天歌意想為之,與天象不符。國朝康熙初,南懷仁作靈臺儀象志,然後黃、赤經、緯各列為表。乾隆九年,增修儀象考成,補正缺誤。道光甲辰,再加考測,為儀象考成續編,入表正座一千四百四十九星,外增一千七百九十一星,洵為明備。今逾十載,歲漸有差,故復據現時推測立表,庶繪圖制器密合天行也。」

又謂:「繪地難於算天,天文可坐而推,地理必須親歷。近人不知古法,故疏舛失實。因考求地理沿革,為歷代地圖,以補史書地志之缺。」

又手摹皇輿全圖,自序略曰:「地圖以天度畫方,至當不易。地球經緯相交皆正角,而世傳輿圖,至邊地竟成斜方形,殊失繪圖原理,其蔽在以緯度為直線也。昔嘗為小總圖,依渾蓋儀,用半度切線,以顯跡象。然州縣不備,且內密外疏,容與實數不符,故復為此圖。其格緯度無盈縮,而經度漸狹,相視皆為半徑與餘弦之比例。橫九幅,縱十一幅,合成地球滂沱四頹之形,欲使所繪之圖與地相肖也。

又變西人之舊,作地球正變兩面全圖,其序略曰:「地形渾員,上應天度,經緯皆為員線。作圖者繪渾於平,須用法調劑,方不失其形似。然視法有三,其一在員外視員,法用正弦,則經圈為橢員,緯圈為直線,其形中廣旁狹,作簡平儀用之。其一在員心視員,法用正切,則經圈為直線,緯圈為弧線,其形中曲旁殺,內密外疏,作日晷用之。斯二者,線無定式,量算繁難。且經緯相交,不成正角。其邊際或太促褊,或太展長,以畫地球,既昧方斜本形,復失修廣實數,所不取也。其一在員周視員,法用半切線,經緯圈皆為平員,雖亦內密外疏,而各能自相比例,西人以此作渾蓋儀,最為理精法密。今本之為地球圖,分正背兩面。正面以京師為中線,其背面之中,即為京師對沖之處,尊首都也。旁分二十四向,審中土與各國彼此之勢,定準望也。經緯俱以十度為一格,設分率也。」

因推演其法,著測量備要四卷,分備物致用、按度考數二題。備物致用其目四:一丈量器,曰插標、曰線架、曰指南尺、曰曲尺、曰丈竹、曰竹籌、曰皮活尺、曰蕃紙簿、曰鉛筆;二測望儀,曰指南分率尺、曰立望表、曰三腳架、曰矩尺、曰地平經儀、曰平水準、曰紀限儀、曰回光環、曰折照玻璃屋、曰千里鏡、曰象限儀、曰秒分時辰標、曰行海時辰標、曰析分大日晷、日風雨針、曰寒暑針;三檢覈書,曰志書、曰地圖、曰星表、曰星圖、曰度算版、曰對數尺、曰八線表、曰八線對數表、曰十進對數表,曰現年行海通書、曰清蒙氣差表、曰太陽緯度表、曰日晷時差表、曰句陳四游表、曰大星經緯表、曰對數較表、曰對數較差表;四畫圖具,曰大小幅紙、曰硯、曰墨、曰硃、曰顏色料、曰筆、曰五色鉛筆、曰筆殼、曰指南分率矩尺、曰長短界尺、曰平行尺、曰分微尺、曰機翦、曰交連比例規、曰玻璃片、曰橡皮。

按度考數其目四:一明數,曰尺度考、曰畝法、曰里法、曰方向法、曰經緯里數;二步量,曰量田計積、曰步地遠近、曰記方向曲折、曰認山形、曰準望所見;三測算,曰測量方向遠近法、曰測地緯度法、曰論平陽大海地平界角、曰測地經度法、曰經緯方向裏數互求法;四布圖,曰正紙幅、曰定分率、曰縮展、曰識別設色。

又因修改對數表之根求析小術,是開極多乘方法,可徑求自然對數,即訥對數,以十進對數根乘之即得十進對數,著乘方捷術三卷。

又創對數尺,蓋因西人對數表而變通其用,畫數於兩尺,相並而伸縮之,使原有兩數相對,而今有數即對所求數。一曰形制,二曰界畫,三曰致用,四曰諸善,五曰圖式,為記一卷。

又嘗撰格術補一卷,同郡陳澧序之,略曰:「格術補者,古算家有格術,久亡,而吾友鄒徵君特夫補之也。格術之名,見夢溪筆談,其說云:『陽燧照物,迫之則正,漸遠則無所見,過此則倒,中間有礙故也。如人搖艫,臬為之礙,本末相格,算家謂之格術。』又云:『陽燧面窪,向日照之,則光聚向內,離鏡一二寸,聚為一點,著物火發。』筆談之說,皆格術之根源也。宋以前蓋有推演為算書者,後世失傳,遂無有知此術者。徵君得筆談之說,觀日光之景,推求數理,窮極微眇,知西人制鏡之法皆出於此。乃為書一卷,以補古算家之術。蓋古所謂陽燧者,鑄金以為鏡也,西洋鐵鏡,即陽燧,玻璃為鏡,亦同此理。故推陽燧之理,可以貫而通之。有此書而古算家失傳之法復明,可知西人制器之法,實古算家所有,此今世之奇書也。至若古算失傳,如此者當復不少,吾又因此而感慨系之矣!」

同治三年,郭嵩燾特疏薦之,堅以疾辭。曾國籓督兩江日,欲以上海機器局旁設書院,延伯奇以數學教授生徒,亦未就。八年五月,卒,年五十有一。

李善蘭,字壬叔,海寧人。諸生。從陳奐受經,於算術好之獨深。十歲即通九章,後得測圓海鏡、句股割圜記,學益進。疑割圜法非自然,精思得其理。嘗謂道有一貫,藝亦然。測圓海鏡每題皆有法有草,法者,本題之法也;草者,用立天元一曲折以求本題之法,乃造法之法,法之源也。算術大至躔離交食,細至米鹽瑣碎,其法至繁,以立天元一演之,莫不能得其法。故立天元一者,算學中之一貫也。並時明算如錢塘戴煦,南匯張文虎,烏程徐有壬、汪曰楨,歸安張福僖,皆相友善。咸豐初,客上海,識英吉利偉烈亞力、艾約瑟、韋廉臣三人,偉烈亞力精天算,通華言。善蘭以歐幾裏幾何原本十三卷、續二卷,明時譯得六卷,因與偉烈亞力同譯後九卷,西士精通幾何者尟,其第十卷尤玄奧,未易解,譌奪甚多,善蘭筆受時,輒以意匡補。譯成,偉烈亞力嘆曰:「西士他日欲得善本,當求諸中國也!」

偉烈亞力又言美國天算名家羅密士嘗取代數、微分、積分合為一書,分款設題,較若列眉,復與善蘭同譯之,名曰代微積拾級十八卷。代數變天元、四元,別為新法,微分、積分二術,又借徑於代數,實中土未有之奇秘。善蘭隨體剖析自然,得力於海鏡為多。

粵匪陷吳、越,依曾國籓軍中。同治七年,用巡撫郭嵩燾薦,徵入同文館,充算學總教習、總理衙門章京,授戶部郎中、三品卿銜。課同文館生以海鏡,而以代數演之,合中、西為一法,成就甚眾。光緒十年,卒於官,年垂七十。

善蘭聰彊絕人,其於算,能執理之至簡,馭數至繁,故衍之無不可通之數,抉之即無不可窮之理。所著則古昔齋算學,詳藝文志。世謂梅文鼎悟借根之出天元,善蘭能變四元而為代數,蓋梅氏後一人云。

華衡芳,字若汀,金匱人。能文善算,著有行素軒算學行世。其筆談一書,猶為生平精力所聚。凡十二卷,第一卷論加、減、乘、除之理;第二卷論通分之理;第三卷論十分數;第四卷論開方之理;第五卷論看題、馭題之法,以明加、減、乘、除、通分、開方之用;第六卷論天元及天元開方;第七卷論方程之術,已寓四元之意,末乃專論四元;第八卷論代數釋號及等式;第九卷論代數中助變之數及虛代之法;第十卷論微分;第十一卷論積分,分十六款以明之;第十二卷一論各種算學不外乎加、減、乘、除,二論一切算稿宜筆之於書,三論算學中可以著書之事,四論學算與著書並非兩事,五論繙算學之書,六論疇人傳當再續。綜計自加、減、乘、除、通分以至微分、積分,由淺入深,術本繁難,而括之以簡易之旨;理本艱深,而寫之以淺顯之詞。

又於同治十三年,與英士傅蘭雅共譯代數術二十五卷,衡芳序之曰:「代數之術,其已知、未知之數,皆代之以字,而乘、除、加、減各有記號,以為區別,可如題之曲折以相赴。迨夫層累已明,階級已見,乃以所代之數入之,而所求之數出焉。故可以省算學之工,而心亦較逸,以其可不假思索而得也。雖然,代數之術誠簡便矣,試問工此術者,遂能不病其繁乎?則又不能也。夫人之用心,日進而不已,茍不至昏眊迷亂,必不肯終輟。故始則因繁而求簡,及其既簡也,必更進焉,而復遇其繁,雖迭代數十次,其能免哉?自是知代數之意,乃為數學中鉤深索隱之用,非為淺近之算法設也。若米鹽零雜之事,而概欲以代數施之,未有不為市儈所笑者也。至於代數、天元之異同優劣,讀此書者自能知之,無待餘言也。」

又與傅蘭雅共譯微積溯源八卷,序之曰:「吾以為古時之算法,惟有加、減而已。其乘與除乃因加減之不勝其繁,故更立二術以使之簡易也。開方之法,又所以濟除法之窮者也。蓋學算者自有加、減、乘、除、開方五法,而一切簡易淺近之數,無不可通矣。惟人之心思智慮日出不窮,往往以能人之所不能者為快,遇有窒礙難通之處,輒思立法以濟其窮,故有減其所不可減,而正負之名不得不立矣;除其所不受除,而寄母通分之法又不得不立矣。代數中種種記號之法,皆出於不得已而立者也。惟每立一法,必能使繁者為簡,難者為易,遲者為速,而算學之境界,藉此得更進一層。如是屢進不已,而所立之法,於是乎日多矣。微分、積分者,蓋又因乘、除、開方之不勝其繁,且有窒礙難通之處,故更立此二術以濟其窮,又使簡易而速者也。試觀圜徑求周、真數求對數之事,雖無微分、積分之時,亦未嘗不可求,惟須乘、除、開方數十百次,其難有不可言喻者。不如用微積之法,理明而數捷也。然則謂加、減、乘、除、代數之外,更有二術焉,一曰微分,一曰積分可也。其積分猶微分之還原,猶之開方為自乘之還原,除法為乘法之還原,減法為加法之還原也。然加與乘,其原無不可還,而微分之原,有可還有不可還者,是猶算式中有不可還原之方耳,又何怪焉!如必曰加減乘除開方已足供吾之用,何必更求其精?是舍舟車之便利,而必欲負重遠行也。其用力多而成功少,蓋不待智者而辨矣。又代數術中末卷之中,載求平員周率簡捷法式,為猶拉所設。未有此法之時,曾有算學士固靈用平員內容外切之多等邊形,費極大工夫,算得三十六位之數。設徑為一,周為三一四一五九二六五三五八九七九三二三八四六二六四三三八三二七九五零二八八。其臨死之時,囑其家以此數刻於墓碑,蓋平時得意之作,恐其磨滅,故欲傳之永久,亦猶亞基默得之墓,刻一球形與員柱形也。」

又與傅氏共譯三角數理,此書為英士海麻士所譔。海麻士專精三角、八線之學,著書十有二卷,皆言三角數理,即用為名。首明三角用比例之理;次論兩角或多角諸比例數;次論造八線比例表之法;次解平三角諸形;次論諸角比例乘約變化之理;紀彼國算士棣弗美創例也,附以專論對數術及諸三角形設題一百則,為書三卷,以引學者;次總說球上各圈及弧三角形之界;次解正弧斜弧三角形之法;次雜論求弧三角數種特設之表;終以弧三角形設題二十七則焉。然書中說解過於煩費,仍不能變外角和較與垂弧、次形、總較諸舊法,故自海氏書出,益覺徐有壬拾遺三術難能可貴,超越西人。

又與傅氏共譯代數難題解法十六卷。

其弟世芳,字若溪。亦通算術,著有近代疇人著述記。


\end{pinyinscope}