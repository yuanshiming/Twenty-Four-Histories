\article{列傳一百三十五}

\begin{pinyinscope}
賽沖阿溫春色爾滾蘇爾慎阿哈保綸布春格布舍

札克塔爾桑吉斯塔爾馬瑜蒲尚佐薛大烈羅聲皋薛升

賽沖阿,赫舍里氏,滿洲正黃旗人。襲雲騎尉世職,充十五善射,授健銳營參領。征臺灣力戰,賜號斐靈額巴圖魯,圖形紫光閣。歷吉林、三姓副都統。

嘉慶二年,率吉林兵赴四川,始終隸德楞泰麾下。張漢潮等竄平利,敗之澍河口,又敗之大寧黑虎廟。追齊王氏、姚之富入寧羌山中,要之羅村壩,以勁騎橫沖賊陣,往來擊射,大破之。三年春,破高均德於洋縣金水鋪,躡追至安子溝。賊夜突營,偕總兵達音泰躍壘而出,斬賊千餘。齊、姚二賊復與均德合擾安康。師次判官嶺,賊隱深林,遣數百人誘戰,賽沖阿鼓勇先入,敗之。賊走山陽,截擊於壩店,遂與明亮、德楞泰三路進逼,大破之於鄖西三岔河,齊、姚二賊投崖死。敘功,被珍賚。四月,分剿均德於華州,連敗之洋縣茅坪、關西溝。均德合諸賊奔渠縣大神山,會諸軍克之。自秋徂冬,迭克箕山、大鵬寨、青觀山,遂擒羅其清、冉文儔,功皆最。

四年夏,敗徐天德於開縣旗桿山,敗張天倫於太平修溪壩。秋,龔文玉踞夔州八石坪。從德楞泰進攻,破賊寨,追敗之竹谿大禾田,擒文玉。冬,擊高均德於大市川,遂破高家營,擒均德。進兵川北,殲張金魁於通江空水河,擒符曰明等於廣元野人村。復移軍川北,迭敗茍文明、鮮大川於貓兒梁、馬家營。

五年春,從德楞泰由陜回川西,擊冉天元於江油新店子,又大戰馬蹄岡,並深入遇伏,先挫後勝,天元就擒。詳德楞泰傳。乘勝破賊劍州李家坪、石門寨。俄而張子聰、雷世旺犯蓬溪,圍成穀、太和、仁和、仁義四寨。偕溫春往援,斬世旺。破冉天泗、王士虎於南江長池壩,破鮮大川、茍文明於岳池新場,擢固原提督。命赴陜專剿高天德,馬學禮諸賊,德楞泰素倚吉林馬隊,賽沖阿尤得眾心,士卒聞其將去,環跪乞留,累疏陳狀,請權緩急,暫留川,允之。秋,從德楞泰擊趙麻花、王珊於雲陽寒池壩、濫泥溝,並殲之。冬,敗楊開第、李彬、齊國謨於觀音河。

六年春,破高天升於鎮安野雞坪,殲之;又破唐明萬等於和岡溪,進至黃花廟,賊夜犯營,奮擊,潰走。夏,從德楞泰擊徐天德、樊人傑等於白河黃石坡,擒陳朝觀,偕溫春破天德於寧陜兩河口,蹙之於紫陽仁和、新灘,天德赴水死,授西安將軍。

諜報龍紹周由川入楚,率勁騎躡擊,先俘其妻子,復殲其兄紹華、弟紹海;至平利岳家坪,冒雨雪縱兵沖踏,陣斬紹周,並蕭四餘匪盡殲之,予騎都尉世職。冬,迭破劉朝選於東鄉土黃壩,奉節、大寧邊境。七年春,又大破之潘家槽,擒斬殆盡,朝選僅以千餘人逸;敗宋國品於梁山柏林槽,擒席尚文於東鄉袁家壩;與勒保部將夾擊陳自得於大竹、鄰水,大破之:調寧夏將軍。

夏,從德楞泰赴楚剿樊人傑,人傑與蒲天寶相犄角,迭敗人傑於雞公山、譚家廟,又克天寶於大埡口。人傑竄平口河腦,自黃茅埡進逼之,無去路,人傑投河死。額勒登保檄赴陜,駐太平河,截剿川、楚竄匪。是冬,大功戡定,詔論諸將戰績,以賽沖阿與楊遇春居最,予輕車都尉世職。九年,調西安將軍,命偕德楞泰檢捕南山殘匪,隨同奏事。尋以遲延降騎都尉。賊平,還舊職,調廣州將軍。

十一年春,海盜蔡牽犯臺灣,命副德楞泰往討,會牽為李長庚擊走,乃專任賽沖阿為欽差大臣,提督以下受節制。尋責專辦陸路,至則鳳山已復,南北兩路僅餘零匪,請停調兩粵、福州駐防兵,水師責成李長庚,陸路責成許文謨。詔嘉其曉事,調福州將軍。秋,牽復入鹿耳門,檄鎮將擊沉賊艦十一,獲船十,擒賊目林略等。十二年,蔡牽、硃濆皆窮蹙,乃赴本官。

十四年,調西安,尋調吉林。十六年,入覲,途見岫巖、復州流民,奏下副都統松箖安撫。會松箖疏請驅逐流民出境,詔斥其誤,命如賽沖阿所議行。十八年,調成都將軍。二十年,剿陜西南山匪,連破之木竹壩、太陽灘,進薄漢北,凡兩月肅清,封二等男爵,賜雙眼花翎。二十一年,廓爾喀與披楞構兵,互請援,命賽沖阿行邊防之而已。誤會上旨,馳檄諭詰,復請越境脅以兵威,詔斥貪功構釁,奪雙眼花翎,降二品頂戴。尋以兩國言和,復之。二十二年,召為正白旗漢軍都統、御前大臣、領侍衛內大臣。尋授盛京將軍。二十四年,復召為理籓院尚書,兼御前諸職如故。宣宗即位,加太子少保,賜紫韁,管理咸安宮蒙古、唐古忒,托忒諸學。

道光元年,出為西安將軍。三年,入祝萬壽,賜宴玉瀾堂,列十五老臣,繪像,禦制詩褒之。四年,召授內大臣、鑲藍旗蒙古都統,充總諳達。六年,以疾乞休。尋卒,贈太子太師,命皇子賜奠,謚襄勤。子額圖渾,三等侍衛。孫特克慎,襲男爵,坐事除名。曾孫清福,襲官四等侍衛。

溫春,默爾丹氏,滿洲正黃旗人。由拜唐阿累擢三等侍衛。從征廓爾喀。乾隆五十九年,高宗幸南苑行圍,以殺虎超擢頭等侍衛。明年,從征苗疆,連克蘇麻寨、大烏草河,賜號克酬巴圖魯。大戰尖雲山,與總兵達音泰分將左右軍,晝夜鏖斗,遂復乾州。苗平,從德楞泰赴四川。嘉慶二年,敗賊東鄉馬耳溝,又敗齊王氏、姚之富於夔州白帝城,加副都統銜,充領隊大臣,駐守竹谿、平利,賊來犯,並卻之。李潮、張世虎餘黨走渡漢,率索倫騎兵蹴之於中流,殲賊千。

三年,破高均德,殲齊王氏、姚之富,破羅其清、冉文儔。四年,破張天倫,擒龔文玉,擒高均德。諸役皆與賽沖阿同為軍鋒,名績相埒。方高家營之未破也,賊扼大市川,倚險抗拒,鼓勇先登,馬蹶,易騎而上,殺賊獨多,詔特嘉之。五年,授正紅旗蒙古副都統。江油新店子及馬蹄岡之戰,並分當一路,瀕危,克捷。冉天元餘黨與張子聰、庹向瑤等合竄潼河西岸,追及渡口,殲其後隊千餘,迭敗賊於蓬溪、中江。秋,偕賽沖阿擊鮮大川於新場,偕薛大烈擊湯思蛟於倒流水,從勒保擊庹向瑤於長壩,皆捷。六年,偕賽沖阿殲徐天德。其秋,擊龍紹周於湖北境,紹周合眾萬餘,已進和岡溪,後隊攻天平寨誘戰,而伏千賊截官軍後,賽沖阿擊攻寨者,溫春扼溪口以要伏賊,遂入峽攻其中堅,大敗之,追斬紹周於岳家坪,予雲騎尉世職。七年,偕賽沖阿敗劉朝選於土黃壩,分兵破庹文正於潘家槽,擒之;又偕賽沖阿破樊人傑於平河口腦,陣斬其弟人禮及二子,人傑走死。是年功蕆,被優賚。凱旋,授虎槍長、正紅旗護軍統領。

十一年,寧陜兵變,赴陜協剿。十五年,充烏里雅蘇臺參贊大臣,行抵烏蘭博木圖,病卒,帝憫之,命其子護喪歸,予祭葬。子烏凝襲世職,官至護軍參領。

色爾滾,莫爾丹氏,黑龍江正黃旗人。由打牲兵襲佐領。從征廓爾喀,以功賜號托默歡武巴圖魯,遷副總管。嘉慶二年,從德楞泰剿教匪。三年,殲齊王氏、姚之富於鄖西,受槍傷,擢協領。合攻箕山,破賊於順水寺、郭家廟、廖家碥,及賊由青觀山敗竄,要擊於濛子灘,擒羅其清,又敗冉文儔於麻壩寨。四年春,擢總管。從德楞泰入陜,破高均德於大市川,擒之,色爾滾戰功居最。五年,從戰馬蹄岡,冉天元負創逸,追至包家溝,天元就擒,又敗賊於石門寨、風如井、鐵山關,加副都統銜。夏,截擊劉朝選於東鄉茨竹林,躡擊張子聰等於九亭場,進搗通江長池壩冉天士賊巢,皆敗之。秋,剿鮮大川、茍文明於巴州元口鎮,沿江兜截,與大軍合擊,斬賊渠吳耀國、鮮文炳,擒茍文禮。又擊湯思蛟、趙麻花於茅坪、倒流水。冬,殲麻花於大禾田,被獎敘。

六年,從德楞泰入陜,擒龔如一、高天升;合擊龍紹周、徐天德,先後擒殲。冬,擊茍文明於槽子溝,陷陣被創。七年,從德楞泰追樊人傑入楚,馳三百里繞其前;又偕蒲尚佐破蒲天寶於鮑家山,徒步入賊巢,天寶走死。詔嘉其奮勇,命在乾清門侍衛行走。又殲戴仕傑於興山施家溝。八年,搜剿餘匪,肅清,被優敘。歷阿勒楚克副都統、伊犁領隊大臣。

十四年,叛兵蒲大芳等在戍所煽亂,將軍松筠令色爾滾往誅之,詔嘉所使得人,召來京,授鑲藍旗蒙古副都統。歷伯都訥、阿勒楚克副都統。十八年,命協剿滑縣教匪李文成,遁,設伏白土岡敗之。賊固守司寨,毀垣入,登樓殺賊,文成自焚死,加都統銜,予雲騎尉世職。歷黑龍江副都統、呼倫貝爾辦事大臣。道光七年,乞病,給全俸。十三年,卒,賜金治喪,謚壯勇。子明晉,孫濟克扎布,襲佐領兼雲騎尉。

蘇爾慎,蘇都里氏,滿洲正黃旗人,黑龍江馬甲。從征廓爾喀。嘉慶初,從德楞泰剿教匪,積功授三等侍衛,改隸京旗。五年,馬蹄岡之戰,初不利。德楞泰憩山上,賊至,馳下奮擊,蘇爾慎射冉天元馬,應弦倒,天元就擒,賊遂大潰。論功最,擢二等侍衛、乾清門行走。其冬,攻大埡口,陷陣被創,賜號西林巴圖魯。六年,戰紅花垛、鯽魚埡,追賊至陜境黃石阪,首先躍馬沖入賊陣,擒賊渠龐士應、方文魁,尋殲徐天德、樊人傑、茍朝獻,戰皆力。七年,破鳳皇山、雞公梁、桂林坪,先登奪隘,軍中號為勇敢。凱旋,擢頭等侍衛。

十八年,林清黨犯禁城,聞警入,首先殺賊,加副都統銜,命為領隊大臣,率巴圖魯侍衛赴山東剿教匪。詔稱其材武出眾,可當百人,愛惜之,戒勿步戰。破曹州、武定賊巢十一,復偕提督馬瑜破賊於滑縣潘章村,擒賊目郭明山。事定回京,授鑲紅旗蒙古副都統,充上書房諳達。二十四年,上幸熱河,乘馬蹶,蘇爾慎控止之,擢鑲藍旗蒙古都統。道光元年,隨扈昌陵,馬逸,突乘輿,降藍翎侍衛。逾年,以二等侍衛休致。未幾,卒,贈副都統銜。

阿哈保,鄂拉氏,滿洲正黃旗人。由司轡護軍授侍衛。從征臺灣,解諸羅圍,擒林爽文,賜號錫特洪阿巴圖魯,圖形紫光閣。繼從征廓爾喀,擢二等侍衛。苗疆事起,轉戰最力,論功居上等,迭擢頭等侍衛、正黃旗蒙古副都統。嘉慶二年,命率吉林兵赴襄陽,偕景安剿教匪,擊賊於獨樹塘、楓樹埡,擒斬甚眾。三年,追賊入川,合攻大神山,分克插旗山賊卡,盡殲之。四年,命擊徐天德於渠河,又破之於譚家壩,賊大潰。冬,設伏白水碉,殲賊千餘,被獎敘。

五年,冉天元等犯川西,御之場院,失利,責領新到貴州兵戴罪立功。從德楞泰擊天元,獨當火石埡一路,先敗後勝。冬,偕薛大烈擊楊開第於安仁溪山梁,追越大山數重,至兩臺山,所過賊寨皆下,開第伏誅,被優賚,擢御前侍衛。六年秋,復偕大烈擊青、黃、藍三號賊於巴州石鐍山,分路設伏,夜襲之,殲戮二千餘,授正紅旗護軍統領,並賜其子阿顏托克托為藍翎侍衛。搜剿老林,擒老教首鄧金祥,予雲騎尉世職。尋合擊高見奇等於大茅坪,因病赴達州醫療。七年,召回京。逾歲,以扈駕神武門,陳德突御輿,失於防護,褫職,予副都統銜,在乾清門行走。歷正白旗蒙古副都統、正紅旗護軍統領。十年,病,加都統銜,遂卒,依都統例賜恤。子阿顏托克托襲世職,兼三等侍衛。

綸布春,羅佳氏,滿洲鑲白旗人。以黑龍江學圍駐京,授司轡。從征廓爾喀、苗疆,賜號色默爾亨巴圖魯。累擢二等侍衛。嘉慶元年,裹創克騾馬岡險隘,加副都統銜。平隴賊寨尤固,綸布春從獅子坡入,囊土填壕,毀墻柵,出間道撫其背,大軍進薄石隆,遂擒石柳鄧。

二年,苗平,從額勒登保剿湖北教匪,破林之華於芭葉山,追賊紅土溪、鐵礦坡、羅鍋圈,迭敗之,授鑲藍旗蒙古副都統。三年,擒覃加耀於終報寨,移軍入川,敗高均德於野豬坪,擊李全等於紫泥嶺。賊走湖北,額勒登保自漢江下襄陽,令綸布春將騎兵由陸出平利。遇張漢潮於南漳,敗之於菩提河、孟石嶺,殲賊數千。尋,漢潮與詹世爵、李槐合,眾可二萬,偕明亮扼之清池子山口,漢潮先遁,世爵、槐於隘口抗拒,綸布春以勁騎截擊,木石並發,賊窘,多觸崖死,世爵、槐並殲焉。秋,從額勒登保擊高均德於吳家河口,賊自林中出,矛傷左脅,力戰敗之。進攻張公橋,擒漢潮子正漋及劉朝佐等。

四年春,械送諸賊至京,命偕侍衛十八人解餉回川,坐報侍衛等患病失實,降黜。未幾,敗漢潮於黃牛鋪,諸軍合擊之張家坪,漢潮就殲,綸布春獲其尸,擢乾清門侍衛。迭破餘賊於教場壩、藥壩、茨溝、板房子,那彥成疏陳戰績超眾,屢詔褒賚。

五年,隨那彥成出寶雞,遏白號賊北犯,破之於龍山鎮,授鑲黃旗蒙古副都統。黃號賊分屯,連營十餘里,綸布春潛師先破八里灣,回擊牛氾嶺,賊傍秀金山列隊以拒,徑沖入陣,手刃數賊,遽卻;進援卡狼寨,扼石峽口夾擊,大敗之。夏,偕穆克登布擊楊開甲於七盤溝,而高天德、馬學禮犯漢中,提督王文雄戰死,詔責綸布春專剿,敗之於白溪。俄,冉學勝渡漢北,將與伍懷志合,偕總兵汪啟邀擊於留壩,又會諸軍敗之於太吉河、魚洞河。

六年春,以追剿學勝久無功,被劾褫職,以馬甲留營效力,從穆克登布擊伍懷志於五郎鐵鎖橋,率三十人先驅沖敵,殺賊數十。賊據山拒鬥,躍登橫擊,賊眾披靡,追擊於紅水河,徒步奮戰,奪山梁。詔嘉其愧奮,授藍翎侍衛。復偕穆克登布躡賊,偵知潛匿老林一層窯,地險絕,督兵猱升而上,懷志與黨六七人惶急投崖下,為綸布春所獲,授二等侍衛,復巴圖魯。其冬,病卒於漢中,依頭等侍衛議恤。

格布舍,鈕祜祿氏,滿洲正白旗人。父薩克丹布,以吉林新滿洲留京為前鋒。乾隆中,從海蘭察征石峰堡、臺灣有功,累擢三等侍衛,賜號伯奇巴圖魯,圖形紫光閣。又從額勒登保征苗疆,擢二等侍衛。遂從剿教匪,破芭葉山,其大金坪、抱窩山兩戰尤力。以病解軍事,久之始卒。臨歿,仁宗念前勞,加副都統銜。

格布舍亦起前鋒,累遷三等侍衛。隨父赴苗疆,平隴之役,從額勒登保克巖人坡、大壩角諸寨,賜號庫奇特巴圖魯。及赴湖北黃柏山,戰頻有功,又殲逃賊於巫山。嘉慶四年,殲冷天祿。奏諸將功,格布舍第一。上夙知其將門子,善用鳥槍,特嘉經略所列公允,加副都統銜。五年,偕楊芳夾擊楊開甲於兩岔河,陷陣,被創墜馬,躍上再戰,追斬甚眾,予恩騎尉世職。又偕楊遇春殲伍金柱、宋國富,六年,擒王廷詔及高天德、馬學禮,功皆最,晉雲騎尉世職。其冬,擊辛鬥於黑龍洞。七年,從額勒登保追剿茍文明,冒雨深入老林,文明就殲。留川、陜邊界檢捕殘匪。凱旋,授正黃旗漢軍副都統、乾清門行走。十二年,出為伊犁領隊大臣,尋授寧夏副都統。召還,授鑲紅旗漢軍副都統。

十八年,命往河南剿教匪,將行,值匪犯禁城,急入捕賊,被優敘,命充領隊大臣,率火器營赴軍。迭敗賊於道口,進圍滑縣,敗援賊於城北,掘東門隧道,為賊覺,復踞西南隅,穴成火發,格布舍仍攻東門,以雲梯先登,獲賊目徐安國於地窖,擢御前侍衛,予騎都尉世職,遷正藍旗護軍統領。坐失察部下私攜俘獲子女,議褫職,帝曰:「格布舍出兵時,聞警,由德勝門奔赴大內,朕不忍負之。」改留任,予副都統銜、頭等侍衛,在大門行走。既而直乾清門,帝閱步射,中三矢,賞黃馬褂,擢寧夏將軍。道光初,回疆軍事起,命駐哈密為聲援,調烏里雅蘇臺將軍,移師守吐魯番。八年,召為正白旗蒙古都統,復出為寧夏將軍。十年,卒,謚昭武。子秀倫,襲騎都尉。

札克塔爾,張氏,滿洲正黃旗人,初金川土番也。父為索諾木所殺。年未二十,密獻入番路徑於將軍阿桂,隨征,洊擢守備。高宗憐之,命隸內務府旗籍,擢二等侍衛、乾清門行走,兼正白旗蒙古副都統。

嘉慶四年,從尚書那彥成赴陜軍,擊高天德、馬學禮於灰峪林,又擊川匪於龍草坪。五年,偕綸布春夾擊白號賊於秦安龍山鎮,擒賊渠餘禮等,賜號瑚爾察巴圖魯。又偕擊王廷詔、楊開甲於牛氾鎮,由山梁馳下,馬蹶,復起力戰,大破之,遷鑲白旗護軍統領。那彥成破張天倫於岷州林家鋪,轉戰鞏昌、文縣,賊據河岸,且擊且濟,逼賊郭家山,自中路仰攻,擒高天德子狗兒;又偕綸布春破伍金柱、楊開甲於分水嶺。

是年夏,召那彥成還京,札克塔爾留聽額勒登保節制。每戰猛銳無前,軍中號曰「苗張」。楊開甲等竄湖北,間道邀擊於鄖西黃鶯鋪,擒斬千餘,予恩騎尉世職。偕楊遇春破伍金柱於手扳崖、銅錢窖,殲楊開甲於茅坪。詔以是役得其分擊之力,優予賚敘。諸賊循渭東竄,札克塔爾邀擊於寬灘,乃折趨棧道。帝廑陜事急,趣其還軍,乃偕慶成駐褒城、西鄉,兼顧川、楚。竄匪高天德、馬學禮窺渡漢,從額勒登保鈔截,屢敗之。

六年元旦,破賊五郎坪,躡伍懷志餘黨於瓦子溝,擒教首彭九皋,遇賊南鄭狼渡磏,躍馬沖賊為二,擒其渠王凌高。夏,追冉學勝於棧東,夜襲黃安壩賊營,破之。偕楊遇春夾擊於天池山,突占山梁,擒其黨陳學文,追敗之竹谿、草鞋峽,賊竄陜。又偕遇春夾擊姚馨佐、曾芝秀於南唐嶺、劉家河口。諸賊尋與學勝合,又敗之孫家坡、渭子池,與遇春同被褒賚。

七年,從額勒登保追剿茍文明,賊匿太白山老林,了於山巔,軍至即遁。札克塔爾以圍捕非計,撤辛峪口兵誘之,果出,晝夜追奔,扼其三面,偕楊遇春夾擊於鎮安石門溝,賊復竄老林,屢出屢入,詔斥曠日持久,褫職留任。歷數月,獲文明妻子,始復之。

八年,凱旋,充奏事處領班。扈駕回宮,入神武門,有男子陳德突犯御輿,札克塔爾手擒之,封三等男爵。十一年,寧陜兵變,從德楞泰往剿,戰於方柴關,不利。既,叛兵就撫,德楞泰以震懾乞降奏。上召札克塔爾詢狀,斥其隱飾,褫職留男爵,回四川,以副將用。尋予副都統銜,充科布多參贊大臣。十三年,召還,授護軍統領,兼武備院卿。十七年,卒,賜金治喪。子常安,襲爵。

桑吉斯塔爾,滿洲正黃旗人,亦四川土番。應募征金川,歷石峰堡、廓爾喀之役,賜號察爾丹巴圖魯。累擢頭等侍衛,改隸內務府滿洲。嘉慶四年,與札克塔爾同赴陜軍,迭敗張漢潮於黃牛鋪、二郎壩、洵陽壩。迨漢潮就殲,加副都統銜,連擊教場壩、大壩、韭菜坪,並下之。五年,隴山鎮、林江鋪、郭家山諸戰,皆與札克塔爾俱,又殲劉允恭於陜境大中溪,敗伍金柱於鎮安手扳崖,被優敘。尋,金柱為楊遇春所殲,其餘黨西走,要其去路,躡追,自文縣、寧羌至龍安擊之,賊竄打箭爐寨,山徑險★C7,棄馬徒步,及於窄口子,痛殲之。分兵擊木蘭溝伏賊,僅存二百餘人,遁三岔河,與冉學勝合。詔斥遲留,額勒登保為疏辯,得白。六年,偕札克塔爾迭敗賊於狼渡磏、天池山、孫家坡。賊自孫家坡敗竄,桑吉斯塔爾設伏楊柏坡以待,擒斬幾盡,高見奇就誅,被獎敘。是冬,召回京。

八年,偕札克塔爾捕陳德,予騎都尉世職。十一年,率巴圖魯侍衛赴寧陜剿叛兵。及還,坐召對遲到,降頭等侍衛。尋授正藍旗漢軍副都統。十八年,率火器營赴滑縣剿賊,以火攻,克城先登,復在御前行走。坐軍中攜俘童當黜,原之;又坐事褫副都統,仍以頭等侍衛乾清門行走。二十三年,卒,賜金優恤。子策楞訥爾,三等侍衛,襲騎都尉,請葬父於近京,允之,賜葬貲焉。

馬瑜,甘肅張掖人。祖良柱,官四川松潘鎮總兵,遂寄籍華陽。瑜少以武生入伍,從征廓爾喀、苗疆,累遷游擊。嘉慶元年,赴達州剿教匪,戰大園堡、安子坪,數有功,賜號達春巴圖魯。三年,從德楞泰殲齊王氏、姚之富於鄖西,瑜間諜功居多,擢參將。擊高均德於雒南鐵釘埡,賊奔就冉文儔,合踞大神山,諸軍合擊,瑜攻其東,克之。及攻大鵬寨,瑜冒雨毀其南門。四年春,文儔就擒,授四川督標副將。從德楞泰入楚,擒高均德,尋赴援陜、甘。

五年春,復從德楞泰回川西,擊冉天元,戰江油新店子,進攻重華堰,深入火石埡,瑜分路助擊有功。追賊石門寨、開封廟,至嘉陵江岸,迭敗之。又設伏敗藍號匪於七孔溪,克長池壩賊巢,擢貴州安義鎮總兵,調重慶鎮。瑜祖故溫福部將,勒保與有舊,甚倚之,又久從德楞泰為翼長,軍事多所贊畫。八月,白號庹向瑤竄長壩,將渡河,瑜率步騎掩至,蹙之,向瑤赴水逸。

六年春,徐天德自洵陽北竄,留後隊於峪河口,前隊奪渡漢江,追及乾溝,擒斬千餘,賊奔鎮安,雪夜間道出野豬坪要之。時龍紹周分黨入太平老林,自率大隊赴楚,欲與天德合,蹙之竹山官渡河,夜聞追騎聲,爭赴水,漂溺泰半。夏,從德楞泰追天德,破之黃石阪,進逼毗河鋪,賊勢瓦解,天德竄死河灘。遂偕賽沖阿等追紹周入川,戰菜子埡、雲霧溪,皆捷,賊西趨陜。冬,殲紹周於平利岳家坪,於是黃號略盡。又敗賊於通江劉家壩,俘獲甚眾。

七年春,師次巫山十二峰,檢捕線號殘匪。夏,擊樊人傑等於東湖雞公山梁,又敗蒲景於大埡口,人傑走死。冬,追賊老山施家溝,山徑險★C7,徒步而入,擒其渠趙鑒,殲餘匪於中子洋。偵巴、巫界上有匿匪,月夜搗其巢,悉殲之,被獎賚。時賊勢窮蹙,瑜自巫山向北搜剿。八年,擒王三魁於馬家壩,三槐之弟也。會楚匪復逼入川,偕色爾滾破之鐙盞窩,餘匪殆盡。三省設防,瑜駐川界徐家壩,擊陜境逸匪,殲之。九年,擢江南提督,調雲南,皆未之任,留辦善後。殲湖北竄匪茍文華等,被優賚。尋坐添紫城疏防,奪巴圖魯、花翎。率兵二千入老林追賊,攻克鳳凰寨,擒斬數百。既而茍文潤就殲,復花翎、勇號。

十年,赴本官,歷江南、直隸提督。十八年,從車駕幸熱河,校射,中三矢,賜黃馬褂。其秋,滑縣賊起,命偕總督溫承惠進剿,破南湖、北湖賊,進擊道口。尋赴開州搜捕,毀潘章、李家莊、袁家莊諸賊巢。事平,優敘。十九年,調江南。坐事左遷徐州鎮總兵,調兗州鎮。二十四年,復任江南提督。未幾,卒,以前勞優恤,謚壯勤。

蒲尚佐,四川松潘人。由行伍拔補千總,從征苗疆,累擢游擊。嘉慶三年,從德楞泰殲齊王氏、姚之富於鄖西,賜號勁勇巴圖魯。克箕山有功,擢參將。五年,偕馬瑜合擊藍號賊於陡坎子山,大破之,擢四川維州協副將。圍趙麻花於石虎林,賊夜突圍者三,皆擊卻,次日盡殲焉,被獎敘。

六年,從德楞泰破高天升於洵陽江岸,追至二峪河,雪夜出山徑進攻,天升就誅,擢雲南鶴麗鎮總兵。敗龍紹周於茅壩,迭敗徐天德於廟坪、黃石阪,又追擊於川、陜境上。每戰輒殪數百,遂躡入楚,沿路搜剿,及紹周為賽沖阿等所殲,其餘黨竄竹山,圍剿殲戮無遺。

七年,從德楞泰轉戰川、楚,諜知樊人傑屯杉木嶺,蒲天寶屯代峰,別有賊屯雞公山為聲援,先破之。人傑走霧露河,尚佐迎擊,轉戰七晝夜,斬獲無算。天寶走當陽,偕色爾滾偵蹤追擊,賊收殘眾屯興山桂連坪,襲破之。賊走踞鮑家山,德楞泰沖其前,尚佐等攀危崖,繞出賊巢上,痛殲之,餘賊狂奔出山,僅數百人,竄入老林。天寶被追急,墜崖死,被優賚,兼乾清門侍衛。又偕副都統富僧德殲戴仕傑於興山,擒崔連樂、崔宗和於房縣,斬陳仕學於巴東。

八年,青號劉渣鬍子與黃號陳大貴踞老鴉寨,尚佐乘霧雨襲之,賊棄寨循當陽河走,遇富僧德伏兵,爭赴水死,擒大貴。駐巫山,搜捕餘匪,賊氛遂凈。十三年,擢湖南提督,調甘肅。二十年,以病解職,歸,卒。

薛大烈,甘肅皋蘭人。由行伍從征臺灣、廓爾喀,累遷都司。嘉慶二年,從總督宜綿剿教匪,由陜入川,數有功,擢游擊。三年,迭克賊於白沙河、蘭場。時王三槐踞東鄉安樂坪,勒保令劉清招降。清遣劉星渠偕二武員往,留為質。三槐偕至大營,星渠密請擒之。大烈爭曰:「舍守備、千總二員易一賊,褻國體,失軍心。」乃止。越數日,三槐復自來,遂羈留,而以陣擒上聞,勒保受上賞,大烈亦賜號健勇巴圖魯,擢參將。未幾,擢四川提標副將,充翼長。善伺勒保意,預諸將黜陟,軍中屬目焉。

五年,擢川北鎮總兵。勒保以罪逮,魁倫代之,諸將不用命,賊益猖,遂連渡嘉陵江、潼河,大烈偕阿哈保等御之。尋復起勒保督師,從剿賊於保寧。別賊自開封廟截大軍後路,大烈擊卻之。偕阿哈保扼嘉陵江,賊不得渡,被獎敘。夏,連敗白號賊於龍安鐵籠堡、竹子山,遂從勒保擊茍文明,解高寺寨圍。追賊循嘉陵江至石板坨,德楞泰躡其後,勒保繞其前,賊分遁。大烈掩擊餘匪於飛龍場,盡殲之。九月,敗賊下八廟,進扼倒流水。會賽沖阿、溫春兵至,夾擊,大破之,殲湯思舉。冬,偕阿哈保破楊開第於渠縣安仁溪,追奔百餘里,至巴州兩臺山,擒斬二千餘。開第逸入營山柏林場,亂矛斃之。

六年春,剿楊步青於大寧金竹坪,乘雪進擊,連敗之白馬廟、大蓋頂。樊人傑、徐萬富屯儀隴碑寺寨,偕阿哈保夜襲之,殲萬富,賊奔川東,追及之,人傑跳崖遁,散竄老林。大烈進剿楊開第、張漢潮餘黨,拔九杵寨,追擊於沙箕灣,擒賊目李尊賢。藍號曹世倫竄南江九嶺子,偕田朝貴合擊殲之。夏,青、藍兩號賊竄東鄉,犯仁和、永興二寨,師分三路入,大烈由右,蹙之華尖壩河濱,殲茍文通、鮮俸先,又擊賊巴州石鐍山,遣兵伏龍鳳埡,自與阿哈保奮擊,擒賊渠徐天壽、王登高等,詔獎賚,授其子千總。白號高見奇、魏學盛合竄棧道,大烈要之於大茅坪山半,偕阿哈保夾擊,勒保督諸將自山頂下壓,賊大潰。見奇竄空山壩,與冉學勝合,屯南江盧家灣,乘不備擊之,擒學勝,予雲騎尉世職。冬,敗白號賊於達州盧硐寺,又追敗之開縣,擒黎朝順,賊竄西鄉漁渡壩。大烈裹糧追躡,由陜入川,敗之於通江羅村,復偕羅聲皋等尾擊之。師次八臺山,別賊圍趙家坪寨峒,掩擊敗之。又殲黃號餘賊於太平邀仙崖,乘勝破八卦山,殪賊渠李顯林。

七年,搜剿老林,連敗茍文明於雙河口、圓嶺山,擒其黨姚青雲。額勒登保檄回剿川賊,大烈乞病,解職回籍。九年,病痊,命在乾清門行走。扈從墜馬,遣蒙古醫療治,給頭等侍衛歲俸。尋授天津鎮總兵,擢直隸提督,賞黃馬褂。十一年,從德楞泰赴寧陜剿撫叛兵,調固原提督。明年,偕楊遇春平瓦石坪之亂,予優敘。調江南,復調直隸。坐為子娶所屬守備女,降天津鎮總兵。尋授廣東提督。復坐動用馬乾銀,再降漢中鎮總兵,調河北鎮。二十年,以睢工出力,加提督銜。卒於官,錄前勞,依提督例賜恤,謚襄恪。

羅聲皋,四川雙流人。由行伍授把總。從孫士毅赴湖北剿匪,克旗鼓寨、芭葉山,擢守備。嘉慶三年,勒保調回四川。四年,從額勒登保破徐天德、冷天祿,累擢游擊。五年,授提標中軍參將。破冉天士於南江長池壩,賜花翎。六年,偕薛大烈殲曹世倫,追湯思蛟、劉朝選入楚,敗之於竹山柳林店。青、藍號賊擾東鄉,偕大烈敗之,又偕擊賊石鐍山,徐天壽就擒,賜號濟特庫勒特依巴圖魯。遂合擊高見奇,擒冉學勝。冬,偕張績擒蕭焜於太平。黃號餘賊屯茨竹溝,聲皋自花角園進攻,大軍繼之,擒葛士寬等。

七年,遷督標中軍副將,充翼長。張簡與湯思蛟合擾東鄉,敗之於老生園、楊家壩,偕田朝貴兵合擊,蹙之河濱,賊爭赴水,擒思蛟弟思武,追擒汪貴於太平梧桐坪。庹向瑤竄東鄉鳳皇山,偕達思呼勒岱合圍,殲其眾,擒向瑤。川匪漸清。楚匪被剿急,多竄川境。偕達思呼勒岱合擊,殲賴飛龍於雲陽閻王碥;又偕羅思舉追賊巴州,分兩路遁,思舉擒簡,聲皋獲思蛟於東鄉村店。八年,搜剿餘匪,擒青號張朝隴、李明學。軍事大定,赴達州辦理凱撤兵勇事宜。十三年,從勒保剿馬邊涼山彞匪,克曲曲烏彞寨,擢重慶鎮總兵,調松潘鎮。二十年,剿中瞻對叛番,克滄龍溝。番酋洛布七力守險,未大創,乞降,受之,以專擅褫職,戍伊犁。逾三年赦歸,卒於家。

薛升,貴州畢節人。以鄉勇剿仲苗,授把總。嘉慶三年,從勒保赴四川軍,偕羅思舉攻安樂坪,攀援絕壁入賊營,斬馘多,進攻祖師觀,夜伏手把巖下,拔柵而登,又從薛大烈設伏,破撲營賊,常為軍鋒,擢守備。四年,殲龔文玉、包正洪,升皆從戰有功,賜花翎。五年,兜剿川東竄匪,升率兵分駐黃草壩,尋擊賊八石坪,追至東鄉南壩場,敗之。軍駐蘆花嶺,賊夜撲營,先伏兵山洞伺擊,賊大潰,擢都司。偕桂涵破猴兒巖賊巢,擒唐大魁。六年,從薛大烈擊賊巴州石鐍山,分路要截,多有斬獲,擒徐天壽於王家坪,擢游擊。七年,從勒保殲張天倫,遂從田朝貴防川、陜邊界,擒徐天培於徐羅壩,殲楊呂清於白巖峒。八年,入山搜捕,擊走茍朝九股匪於八百谿,擢雲南新習營參將。軍事蕆,赴本官,歷東川、尋霑參將。十八年,調剿滑縣教匪,攻克南門,擢副將,尋回雲南。二十三年,從剿臨安夷匪,授永昌協副將。道光元年,剿大姚夷匪,擢鶴麗鎮總兵。歷陜西河州鎮、直隸大名鎮,擢直隸提督,調湖南。十六年,新寧瑤生藍正樽習教拒捕,犯武岡城,鎮筸兵滋事戕官,事皆旋定,吏議鐫級留任。升年已七十,總督林則徐疏論其老於軍事而無振作。未幾,以楊芳代之,調升廣西提督。二十二年,英吉利犯廣東,赴潯梧治防。因病乞假歸,尋休致,以舊勞予食全俸。咸豐元年,卒,謚勤勇。

論曰:額勒登保以楊遇春、穆克登布為翼長,德楞泰以賽沖阿、馬瑜為翼長,勒保以薛大烈、羅聲皋為翼長,觀偏裨之人材,其成功可知矣。是諸人者,其後多膺軍寄,二楊而外,亦無赫赫功,豈非材器有所限哉?勒保部將差弱,蓋賴羅思舉、桂涵等鄉勇之力為多焉。


\end{pinyinscope}