\article{列傳三百三}

\begin{pinyinscope}
土司五

○廣西

廣西為西南邊地。秦,桂林郡。漢,始安。唐,桂管。宋,靜江府。元,靜江路。明建廣西省。瑤、僮多於漢人十倍,盤萬山之中,踞三江之險。明時,因元之舊,多設土司,以資鎮壓。叛服不常,韓雍之定藤峽,王守仁之撫田州,沈希儀、俞大猷之戰功,殷正茂、翁萬達之成績,僅得勘定。清朝,廣西莠民四起,土司獨安靖無事。鄂爾泰經略三省,革泗城土府岑映宸職,割江北地隸貴州。雍正六年八月,首討思陵州之八達寨,扼其餉道,屯兵二三里外,量大砲所能及,漸轟進偪。賊窘急,斬土目顏光色兄弟以獻,尚閉寨不出,遂為官兵所毀。八年,復檄討思明土府所屬之鄧橫寨,三路進攻,一鼓而克。於是遠近土目爭繳軍器二萬餘。巡邊所至,迎扈千里,三省邊防皆定。

慶遠府:秦,象郡。漢,交阯、日南二郡界。唐置粵州,天寶初,改龍水郡;乾元中,更宜州。宋升慶遠軍節度,咸淳初,改慶遠府。元為慶遠路。明仍改慶遠府。清因之。

東蘭土州,在府西南。宋置蘭州,以韋氏世襲。元改東蘭州。明因之。傳至韋光祚,清順治初,歸附,予舊職,雍正七年,改設流官知州。

忻城土縣,在府南。宋慶歷間,隸宜州。元以莫保為八仙屯千戶。明洪武初,設流官。後仍任土官,以莫氏世襲。傳至莫猛,清順治九年,歸附,仍準襲職。

南丹土州,在府西北。宋開寶初,土官莫洪內附;元豐三年,置州,管轄諸蠻。明洪武初,莫金納土。金叛被誅,以金子祿襲。傳至莫自乾,清順治九年,歸附,仍準襲職。

那地土州,在府西。宋熙寧初,土人羅世念來降;崇寧間,遂置地、那二州,以羅氏世知地州。元仍之。明洪武元年,土官羅黃貌附,詔並那、地為一州,予印授,黃貌世襲,以流官吏目佐之。傳至羅德壽,清順治九年,歸附,仍準世襲。

永順正土司,在府西南。明設土司,弘治間,以鄧文茂為之。傳至鄧世廣,清順治九年,歸附,仍準世襲。

副土司。彭希聖,同。

永定土司,在府西南。明成化十二年,設土司,以韋萬秀為之。傳至韋盛春,清順治九年,歸附,仍準世襲。

思恩府:古百粵。漢屬交阯。唐天寶元年,改為橫山郡。元置田州路軍民總管府。明正統五年,升為思恩府。弘治末,改流官,清因之。

上林土縣,在府西南二百七里。宋置,隸橫山寨。元屬田州。明洪武二年,以黃嵩為土知縣,仍屬田州;嘉靖初,改隸思恩軍民府,佐以流官典史。傳至黃國安,清初,歸附,仍襲舊職。

白山土司,在府東北。宋皇祐間,隨狄青有功,世襲土舍。明嘉靖七年,以王受明為白山土巡檢。傳至王如綸,清初,歸附,仍襲舊職。

興隆土司,在府東北八十里。明嘉靖七年,以韋貴為土巡檢。傳至韋萬安,清順治十七年,歸附,仍準世襲。

那馬土司,在府西北九十里。明嘉靖七年,以黃理為土巡檢。傳至黃天倫,清初,歸附,仍準世襲。

定羅土司,在府西一百四十里。明嘉靖七年,以徐伍為土巡檢。傳至徐朝佐,清初,歸附,仍準世襲。

舊城土司,在府西北一百二十里。明嘉靖七年,以黃集為土巡檢。傳至黃世勛,清初,歸附,仍準世襲。

下旺土司,在府西二百十里。明嘉靖七年,以韋良保為土巡檢。傳至韋際弦,清初,歸附,仍準世襲。

安定土司,在府北。明嘉靖七年,以潘良為土巡檢。傳至潘應璧,清初,歸附,仍準世襲。

都陽土司,在府西北六百里。明嘉靖七年,以黃留為土巡檢。傳至黃宏會,清初,歸附,仍準世襲。

古零土司,在府東。明嘉靖七年,以覃益為土巡檢。子文顯,征大藤峽有功,加千總。傳至覃恩錫,清初,歸附,仍準世襲。

田州土州,在府西四百五十里。唐天寶元年,橫山郡。乾元元年,改為田州。宋屬橫山寨。元置田州路軍民總管府。明改田州府,尋復為州。嘉靖九年,以岑芝主田州。傳至岑漢貴,清順治初,歸附,仍準世襲。近改百色直隸,置流官。

歸順州,舊為峒。元隸鎮安路。明因之。弘治年間,升為州,以岑瑛為知州,世襲,改隸思恩府。傳至岑繼綱,清順治初,歸附,仍予舊職。雍正七年,改隸鎮安府。八年,巡撫金鉷以土司岑佐不法狀題參,革職改流。

泗城府:古百粵地。宋置泗城州。元屬田州路。明隸思恩府。洪武初,以岑善忠為知府,世襲。傳至岑繼祿,清順治十五年,歸附,隨征滇、黔有功,改為泗城軍民府。繼祿死,子齊岱襲。齊岱傳子映宸。雍正五年,映宸以罪參革,改設流官。

下雷州。元屬鎮安路。明初,降為峒。萬歷三十二年,許應珪以軍功復職。傳至許文明,清順治初,歸附,仍襲舊職。

向武州。宋置,隸橫山寨。元隸田州路。明初,以黃世威為知州。傳至黃嘉正,清順治初,歸附,仍襲舊職。

都康州。宋置,隸橫山寨。元屬田州路。明隸思恩府,以馮斌為知州。傳至馮太乙,清順治九年,歸附,仍襲舊職。

南寧府:唐邕州也。元,邕州路,泰定中,改南寧路。明置南寧衛,後改府。清因之。

果化土州。宋置。元屬田州路。明洪武二年,授土官趙榮為知州。弘治中,改隸南寧。傳至趙國鼎,清初,率眾歸附,仍襲舊職。

歸德土州,在府西。其先黃氏。宋征交阯有功,建歸德州。明洪武二年,以黃隍城為知州。傳至黃道,清初,歸附,仍襲世職。

忠州土州,在府西南一百九十里。宋置。明洪武二年,以黃威慶為土知州。傳至黃光聖,清順治初,歸附,仍予世職。

遷隆峒,在府西南二百四十里。明洪武元年,以黃威鋆為土官,以失印廢為峒,降巡檢。傳至黃元吉,清初,歸附,仍予世職。

太平府:漢屬交阯。唐為羈縻州。宋平嶺南,置五寨,一曰太平,領州縣。元置太平路。明洪武二年,改為太平府。清因之。

太平州,在府西北。明洪武二年,以李以忠為知州。傳至李開錦,清順治十六年,歸附,仍予世職。

鎮遠州,在府東北。舊名古隴。宋置州。元隸太平路。明亦屬太平路。明初,以趙昂升為知州。傳至趙秉義,清順治十六年,歸附,仍予世職。

茗盈州,在府北。宋置。元屬太平路。明初,以李鐵釘為知州。傳至李應芳,清順治十六年,歸附,仍予世職。

安平州,舊名安山,在府西北。唐置波州。宋設安平州。元隸太平路。明洪武初,以李郭祐為知州,使守交阯各隘。傳至李長亨,清順治十六年,歸附,仍準世襲。

萬承州,在府東北,舊名萬陽。唐置萬承、萬形二州。宋省萬形隸太平寨。元屬太平路。明洪武初,以許郭安為知州。傳至許嘉鎮,清順治十六年,歸附,仍予世職。

全茗州,在府北,舊名連岡。宋置,隸邕州。元屬太平路。明洪武初,以許添慶為知州,給印。傳至許家麟,清順治十六年,歸附,仍予世職。

結安州,在府東北,舊名營周。宋置結安峒。元改州,屬太平路。明洪武元年,以張仕榮為知州。傳至張邦興,清順治十六年,歸附,仍予世職。

龍英州,在府北,舊名英山。宋為峒。元改州,屬太平路。明洪武二十二年,以趙世賢為知州,給印。傳至趙廕昌,為族人繼祖所殺。清順治十六年,歸附,誅繼祖。廕昌無子,以邦顯子廷耀襲。

佶倫州,在府東北,舊名邦兜。宋置安峒,隸太平寨。元改州,屬太平路。明洪武二年,以馮萬傑為知州。傳至馮嘉猷,清順治十六年,歸附,仍予世襲。

都結州,在府東北。元屬太平路。明洪武三年,以農武高為知州。傳至農廷封,清順治十六年,歸附,仍予世襲。

上下凍州,在府西。宋置凍州。元分凍州為上凍、下凍二州。明隸太平府,洪武元年,以趙帖從為知州。傳至趙長亨,清順治十六年,歸附,仍予世襲。

恩城州,在府西北。唐置。宋分上下恩城二州。元屬太平路。至正間並為一。明洪武元年,以趙雄傑為知州。傳至趙貴炫,清順治十六年,歸附,仍予世襲。

羅陽土縣,在府東,舊名福利。宋置,隸遷隆寨。元屬太平路。明隸太平府,明初,以黃宣為知縣。傳至黃啟祥,清順治十六年,歸附,仍予世襲。

思陵州。宋置州,隸永平寨。元屬思明路。明初,省入思明府,後復建,仍隸太平府;洪武二十一年,以韋延壽為知州。傳至韋懋選,清順治十六年,歸附,仍予世襲。

思明州。唐置,屬邕州。宋隸太平寨。元改思明路。明為府,洪武元年,以黃忽都為知府。傳至黃戴乾,清順治十六年,歸附,仍予舊職。黃觀珠襲。以安馬、洞郎等五十村改流,隸南寧。明降府為州,移治伯江哨。雍正十年,五十村目怨觀珠,殺觀珠嬖人,欲因以謀不靖。太平知府屠嘉正、新太協副將崔善元安定之。觀珠以罪參革,改流。又思明州與思明府本兩地,土官亦黃姓,於康熙五十八年改流。

下石西州,在府西二百十里。宋閉鴻為知州。明初,仍給世襲。傳至閉承恩,清初,歸附,仍襲舊職。

上石西州。明崇禎間,並入本府。清雍正十二年,改隸明江同知。

上龍司。漢屬交阯。唐置龍州。宋隸邕州。元大德中,改為萬戶府。明初,屬太平。洪武八年,改直隸州,尋改隸太平。以土官趙帖堅襲知州,以流官吏目佐之。其後事具明史。傳至趙有涇,為庶兄有濤所殺。有涇子國梁愬父冤,有濤逃入交阯。清平廣西,更名趙祿奇,自交阯逃回歸附,仍予舊職。死,傳子廷楠。時國梁父冤既白,應襲,而廷楠拒之;國梁復出奔,適雲南煽動,遂率賊兵破州城,殺廷楠。未幾撲滅。而廷楠無子,乃以庶支趙元基孫國桓襲。傳子殿灴,雍正三年,以貪殘參革,析其地為上龍司、下龍司;改設兩巡檢,平通判兼攝。後改龍州。

憑祥州。宋為憑祥峒,屬永平寨。元隸思明路。明洪武初,李升內附,置憑祥鎮。永樂二年,置縣;成化八年,升州,以升孫李廣寧為知州。時又屬安南,仍歸明,屬太平府。傳至李維籓,清順治十六年,歸附,仍予世襲。

江州。宋置,屬古萬寨。元隸思明路。明因之,洪武初,以黃威慶為知州。傳至黃廷傑,清順治十六年,歸附,仍襲舊職。

鎮安府:在省西。宋時於鎮安峒建右江軍民宣撫司。元改鎮安路。明洪武元年,改府,授土官岑天保為知府。清順治間,土官故絕,沈文崇叛據其地;十八年,發兵撲滅之。康熙二年,改置流官通判。雍正十年,改知府。

都康州。宋置,隸橫山寨。元屬田州路。明洪武三十二年,復置州。永樂初,以馮斌為知州,隸思恩府。傳至馮太一,清順治九年,歸附,襲舊職。雍正七年,鎮安設府,改隸鎮安。

上映峒。宋置州。明初,廢為峒,以許尚爵襲。傳至許國泰,清順治初,歸附,仍予舊職。

湖潤寨。宋時置州。明初,廢州為寨,降巡檢司。傳至宗熙,清順治九年,歸附,仍給巡檢司印,世襲。


\end{pinyinscope}