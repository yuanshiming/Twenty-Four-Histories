\article{列傳四}

\begin{pinyinscope}
諸王三

太祖諸子二

鎮國勤敏公阿拜鎮國克潔將軍湯古代子鎮國公聶克塞

莽古爾泰輔國?厚將軍塔拜

饒餘敏郡王阿巴泰子安和親王岳樂溫良貝子博和託博洛

博和託子貝子彰泰阿巴泰孫悼愍貝子蘇布圖

鎮國恪僖公巴布泰德格類巴布海阿濟格

輔國介直公賴慕布

鎮國勤敏公阿拜,太祖第三子。天命十年,偕塔拜、巴布泰伐東海北路呼爾哈部,俘千五百戶,還,太祖出城迎勞,授牛錄章京。天聰八年,授梅勒額真。崇德三年,授吏部承政。四年,封三等鎮國將軍。六年,駐防錦州。八年,以老,罷承政。順治四年,進二等。五年二月,卒。十年,追封謚。

阿拜子有爵者三:鞏安,襲三等鎮國將軍,進輔國公;乾圖、灝善封輔國公,乾圖謚介直。鞏安、灝善之後,皆以奉恩將軍世襲。

鎮國克潔將軍湯古代,太祖第四子。事太宗,授固山額真。取永平四城,湯古代偕圖爾格、納穆泰守灤州。天聰四年,明兵攻灤州急,貝勒阿敏怯不敢援,遣巴都禮率數百人突圍進,夜三鼓,入灤州。既,明兵以?壞城,城樓火,湯古代等棄城奔永平。既還,太宗廷詰之,湯古代引罪請死。太宗曰:「汝不能全師而歸,殺汝何益?」下所司論罪,免死,罷固山額真,奪所屬人口,籍其家。八年,授三等梅勒章京。崇德四年,封三等鎮國將軍。五年,卒。

子二:穆爾察,初封三等奉國將軍,襲爵,進二等。卒,謚恪恭。聶克塞,襲穆爾察三等奉國將軍。從多鐸略寧遠,從多爾袞定京師,逐李自成至慶都,皆有功,累進鎮國公。坐事降三等鎮國將軍。康熙四年,卒。無子,爵除。

莽古爾泰,太祖第五子。歲壬子,從太祖伐烏喇,克六城,莽古爾泰請渡水擊之,太祖曰:「止!無僕何以為主?無民何以為君?我且削之。」遂毀六城,移軍富勒哈河。越日,於烏喇河建木城,留兵千守焉。天命元年,授和碩貝勒,以序稱三貝勒。

四年,明經略楊鎬遣總兵杜松以六萬人出撫順關,劉綎以四萬人出寬甸。莽古爾泰從太祖御松界凡,伏兵薩爾滸谷口,伺明兵過將半擊之,我軍據吉林崖,明兵營薩爾滸山,復偕貝勒代善等以千人益吉林崖,而合師攻薩爾滸,大破之,松戰死。又從太祖還軍擊斬綎。八月,從伐葉赫。五年,太祖伐明,略懿路、蒲城,令莽古爾泰以所部逐敵,率健銳百人追擊明兵,至渾河乃還。六年,鎮江守將陳良策叛投毛文龍,莽古爾泰偕代善遷金州民復州。十年,攻克旅順口。

察哈爾林丹汗侵科爾沁部,圍克勒珠爾根城,莽古爾泰赴援,至農安塔,林丹汗遁。十一年,太祖伐喀爾喀巴林部,先命諸貝勒略錫拉穆楞,皆以馬乏不能進;莽古爾泰獨領兵夜渡擊之,俘獲無算。

塔山糧運。三年,從太宗徵明,阿巴泰自龍井?,又以偏師?天聰元年,攻明右屯關入,攻漢兒莊。莽古爾泰偕多爾袞、多鐸為繼,降其城,旋諭降潘家口守將。上克洪山口,逼遵化。莽古爾泰自漢兒莊合軍擊敗明總兵趙率教,擒其副將臧調元。師進次通州,薄明都,明諸道兵入援。莽古爾泰遣巴牙喇兵前行,與多鐸殿,值明潰卒來犯,擊殲之。從上閱薊州,破山海關援兵。四年二月,克永平、遵化。還,與明兵遇,敗之。

五年,從圍大凌河,正藍旗圍其南,莽古爾泰與德格類率巴牙喇兵策應。明總兵吳襄、監軍道張春赴援,距城十五里而營。莽古爾泰從上擊之,獲春等。當圍大凌河時,莽古爾泰以所部兵被創,言於上。上偶詰之曰:「聞爾所部兵每有違誤。」莽古爾泰恚曰:「寧有是耶?」上曰:「若告者誣,當治告者;果實,爾所部兵豈得無罪?」言已,將起乘馬,莽古爾泰曰:「上何獨與我為難?我固承順,乃猶欲殺我耶?」撫佩刀,頻目之。貝勒德格類,其母弟也,斥其悖,拳毆之,莽古爾泰益怒,抽刃出鞘。左右揮之出,上憤曰:「是固嘗弒其母以邀寵者!」諸貝勒議莽古爾泰大不敬,奪和碩貝勒,降多羅貝勒,削五牛錄,罰銀萬及甲胄、雕鞍馬十、素鞍馬二。

六年,從伐察哈爾,林丹汗遁。移師伐明,略大同、宣府。十二月,卒,上臨喪,漏盡三鼓,始還;又於中門設幄以祭,哭之慟,乃入宮。

九年,莽古爾泰女弟莽古濟格格所屬冷僧機告莽古爾泰與德格類、莽古濟格格盟誓怨望,將危上,以莽古濟格格夫瑣諾木為證。搜得牌印十六,文曰「大金國皇帝之印」。追奪莽古爾泰爵。莽古濟格格及莽古爾泰子額必倫坐死,餘子並黜宗室。

輔國?厚公塔拜,太祖第六子。天命十年,伐東海北路呼爾哈部有功,授三等甲喇章京。天聰八年,進一等。尋封三等輔國將軍。崇德四年九月,卒。順治十年,追封謚。塔拜子八,有爵者三:額克親、班布爾善、巴都海。額克親,崇德元年,從阿濟格伐明,偪燕京。明兵自涿州來拒,親陷陣,破之。四年,封三等奉國將軍。尋襲爵。五年,從多爾袞攻錦州,復從多鐸追擊明兵於塔山。六年,上圍錦州,敗洪承疇兵十三萬。移軍近松山,掘壕困之。明總兵曹變蛟夜突上營,額克親偕內大臣錫翰力禦,卻之。?功,賜銀八十。順治元年,從多爾袞入山海關,破李自成,有功,累進鎮國公。七年,授正白旗滿洲固山額真,復進貝子。八年,坐附羅什博爾惠諂媚諸王造言構釁,削爵,黜宗室。九年,復入宗室,授內大臣。十二年,卒。班布爾善,累進封輔國公。以附鰲拜,譴死。附見鰲拜傳。巴都海,亦封輔國公,謚恪僖。

饒餘敏郡王阿巴泰,太祖第七子。初授臺吉。歲辛亥,與費英東、安費揚古率師伐東海窩集部烏爾固辰、穆棱二路,俘千餘人,還。天命八年,偕臺吉德格類等伐扎嚕特部,渡遼河,擊部長昂安。昂安攜妻子引牛車遁,師從之,昂邦章京達音布戰死。阿巴泰繼進,還,太祖郊勞,並賚從征將士。?斬昂安及其子,俘其

太宗即位,封貝勒。阿巴泰語額駙揚古利、達爾漢曰:「戰則擐甲胄,獵則佩弓矢,何不得為和碩貝勒?」語聞,上曰:「爾等宜勸之,告朕何為?」天聰元年,察哈爾昂坤杜棱來歸,與宴。阿巴泰不出,曰:「我與諸小貝勒同列。蒙古貝勒明安巴克乃位我上,我恥之!」上以語諸貝勒,貝勒代善與諸貝勒共責之曰:「德格類、濟爾哈朗、杜度、岳託、碩託早從五大臣議政,爾不預焉。阿濟格、多爾袞,多鐸,先帝時使領全旗,諸貝勒皆先爾入八分。爾今為貝勒,得六牛錄,已逾分矣!乃欲與和碩貝勒抗行,得和碩貝勒,不更將覬覦耶?」阿巴泰引罪,罰甲胄、雕鞍馬四、素鞍馬八。

二年,與岳託、碩託伐錦州,明師退守寧遠,克墩臺二十一,毀錦州、杏山、高橋三城,還。三年,從伐明,自喀喇沁波羅河屯行七日,偕阿濟格率左翼四旗及蒙古軍攻龍井關,夜半克之。明將易愛自漢兒莊赴援,擊斬之,取其城。會上克洪山口,逼遵化,敗明山海關援兵,克之。復趨通州,明總兵滿桂、侯世祿屯順義,阿巴泰偕岳託擊走之,獲馬千餘、駝百,順義亦下。

時袁崇煥、祖大壽以兵二萬屯廣渠門外,阿巴泰偕莽古爾泰等率師攻之。聞敵伏兵於右,諸貝勒相約入隘必趨右,若出中路,與避敵同。豪格趨右,敗伏兵,轉戰至城壕。阿巴泰出中路,亦破敵,與豪格師會。罷戰,諸貝勒議違約罪,阿巴泰當削爵。上曰:「阿巴泰非怯,以顧其二子,與豪格相失,朕奈何加罪於吾兄?」宥之。徇通州,焚其舟,略張家灣。四年,從上圍永平,與濟爾哈朗邀斬?。從上至薊州,明兵五千自山海關至,奮擊,殲其叛將劉興祚。尋命守永平。明兵攻灤州,偕薩哈璘赴援,明兵引退,代還。

五年,初設六部,掌工部事。從上圍大凌河,正黃旗圍北之西,鑲黃旗圍北之東,阿巴泰率巴牙喇兵為策應。大壽降,阿巴泰偕德格類、多爾袞、岳託以兵四千易漢裝,從大壽夜襲錦州,二更行,?發不絕聲。錦州人聞之,謂大凌河兵逸,爭出應之,師縱擊,斬馘甚。霧,兩軍皆失伍,乃引還。七年,築蘭磐城,賜御用蟒衣一、紫貂皮八、馬一。詔問征?明及朝鮮、察哈爾三者何先,阿巴泰請先伐明。八月,略山海關,俘數千人還。上迎勞,責其不深入。八年,從徵宣府,至應州,克靈丘及王家莊。九年,阿巴泰病手痛,上曰:「爾自謂手痛不耐勞苦。不知人身血脈,勞則無滯。惟家居佚樂,不涉郊原,手不持弓矢,忽爾勞動,疾痛易生。若日以騎射為事,寧復患此?凡有統帥之責者,非躬自教練,士卒奚由奮?爾毋媮安,斯克敵制勝,身不期強而自強矣。」

崇德元年,封饒餘貝勒。偕阿濟格等伐明,克雕鶚堡、長安嶺堡,薄延慶,分兵克定興、安肅、容城、安州、雄、東安、文安、寶坻、順義、昌平十城。五十六戰皆捷,俘十數萬。師還,上出城十里迎勞,酌以金?。上伐朝鮮,留防噶海城。三年,上伐喀爾喀,阿巴泰與代善留守,築遼陽都爾弼城,復治盛京至遼河道,道廣十丈,高三尺,濬壕夾之。副多爾袞率師伐明,毀邊墻入,越明都趨涿州,直抵山西。復東趨臨清,克濟南。略天津、遷安,出青山關,還。賜馬二、銀五千。四年,偕阿濟格略錦州、寧遠。

五年,偕多爾袞屯田義州,分兵克錦州城西九臺,刈其禾;又克小凌河西二臺。偕杜度伏兵寧遠,截明運道,奪米千石。移師敗明杏山、松山兵。時大軍更番圍錦州,阿巴泰屢往還其間。六年,坐從多爾袞去錦州三十里為營及遣士卒還家,論削爵,奪所屬戶口。詔寬之,罰銀二千。尋從上破洪承疇援兵十三萬。七年,錦州降,偕濟爾哈朗圍杏山,克之,還守錦州。?功,賜蟒緞七十。

十月,授奉命大將軍伐明,內大臣圖爾格副之。自黃崖口入邊,敗明將白騰蛟等於薊州,破河間、景州。趨兗州,擒斬明魯王以派等。分徇萊州、登州、青州、莒州、沂州,南至海州。還略滄州、天津、三河、密雲。凡克城八十八,降城六,俘三十六萬,得金萬二千、銀二百二十萬有奇。八年五月,師還,上遣濟爾哈朗、多爾袞等郊迎三十里,賜銀萬。順治元年四月,進郡王。二年,統左右兩翼兵鎮山東,剿滿家洞土寇,尋還。三年,薨。康熙十年,追謚。

阿巴泰子五,有爵者四:尚建、博和託、博洛、岳樂,而岳樂襲爵。

安和親王岳樂,阿巴泰第四子。初封鎮國公。順治三年,從豪格徇四川,擊斬張獻忠。六年,封貝勒。八年,襲爵,改號安郡王。九年,掌工部事,與議政。十年,命為宣威大將軍,駐歸化城,規討喀爾喀部土謝圖汗、車臣汗。尋行成,入貢,乃罷兵。十二年,掌宗人府事。十四年,進親王。

康熙十三年,吳三桂、耿精忠並反,犯江西。命為定遠平寇大將軍,率師討之,自江西規廣東,次南昌,遣兵復安福、都昌。十四年,復上高、新昌。戰撫州唐埠、七里岡、五桂寨、徐汊,屢破敵,復餘干、東鄉。詔移師湖南,疏言:「江西為廣東咽喉,當江南、湖三千,固??廣之沖,今三十餘城皆陷賊。三桂於醴陵造木城,增偽總兵十餘人,兵七萬、守萍鄉諸隘。若撤撫州、饒州、都昌諸路防兵盡赴湖南,則諸路復為賊有。否則,兵勢單弱,不能長驅。廣東諸路,恐亦多阻。臣欲先平江西,無?顧憂,然後移師。」疏聞,上令速定江西。岳樂督兵攻建昌,精忠將邵連登率數萬人迎戰長興鄉,擊走之,克建昌,並下萬年、安仁。師進克廣信,再進克饒州,破敵景德鎮,復克浮梁、樂平。分兵徇宜黃、崇仁、樂安,皆下。並諭降泰和、龍泉、永新、廬陵、永寧及湖廣茶陵諸縣。師再進,克靖安、貴溪。疏言:「三桂聞臣進取,必固守要害,非綠旗兵無以搜險,非紅衣?無以攻堅。請令提督趙國祚等率所部從臣進討,並敕發新造西洋?二十。」又疏言:「精忠將張存遣人稱有兵八千屯順昌,俟大軍入閩為應。」詔以簡親王喇布專主福建軍事,而趣岳樂赴長沙。

十五年,岳樂師克萍鄉,遂薄長沙。疏言:「敵船集長沙城下,我師無船,難以應敵。長沙附近林木頗盛,請先撥戰艦七十艘,仍令督撫委員伐木造船。」如所請。八月,詔曰:「朕聞王復萍鄉,直抵長沙,甚為嘉悅。王其善撫百姓,使困苦得紓;即脅從者皆朕赤子,當加意招徠。」十六年,遣兵破敵瀏陽,斬千餘級,克平江。十七年,破敵七家洞。三桂將林興珠等自湘潭來降。九月,三桂既死,詔趣岳樂進師。岳樂請赴岳州調度諸軍。上命大將軍察尼規取岳州,而令岳樂仍攻長沙。十八年正月,岳州降。長沙賊亦棄城遁,遂入長沙,遣兵復湘潭。尋會喇布軍克衡州、寶慶,分兵守焉。復與喇布合軍攻武岡,破敵寶慶巖溪,斬級數百,獲舟四十。師次紫陽河,敵於對岸結營,師逕渡,分兵出敵後夾擊之,敵潰走。三桂將吳國貴、胡國柱以二萬人守隘,發?殪國貴,奪隘。貝子彰泰逐敵至木瓜橋,遂克武岡及楓木嶺。詔召嶽樂還京師,以敕印付彰泰。十九年正月,下詔褒岳樂功。岳樂至京師,上於盧溝橋南二十里行郊勞禮。

順治初,故明外戚周奎家有自稱明太子者,使舊宮人及東宮官屬辦視非是。三桂反,京師又有硃慈璊者,自稱三太子,私改元廣德,糾黨舉火為亂,事敗,慈璊走免。鞫其黨,謂其真姓名為楊起隆。及岳樂駐師楓木嶺,於新化僧寺得硃慈燦,自言為莊烈帝長子,闖難奔南京,福王置諸獄,釋為民,從朽木和尚為僧,往來永州、寶慶間。以三桂悖逆反覆,將募兵聲討,三桂死,乃止。至是,岳樂攜慈燦來京,詔令慈璊黨相見,復不相識,乃斬之。

二十年,仍掌宗人府事。二十七年,偕簡親王雅布往蘇尼特防噶爾丹。二十八年二月,薨,予謚。二十九年,貝勒諾尼訐岳樂掌宗人府,聽讒,枉坐諾尼不孝罪,追降郡王,削謚。

岳樂子二十,有爵者三:蘊端、瑪爾渾、經希。蘊端封勤郡王,坐事降貝子;復坐事奪爵。經希封僖郡王。岳樂得罪,降鎮國公,卒,停襲。瑪爾渾,襲爵。瑪爾渾好學能文章,蘊端亦善詩詞。瑪爾渾又輯宗室王公詩為宸萼集,一時知名士多從之游。四十八年,薨,謚曰懿。子華?,襲。五十八年,薨,謚曰節。雍正元年十二月,詔曰:「曩安郡王岳樂諂附輔政大臣,每觸忤皇考,蒙恩始終寬宥,而其諸子全不知感,傾軋營求,妄冀封爵。瑪爾渾、華?相繼夭折,爵位久懸。岳樂諸子伍爾占、諸孫色亨圖等,怨望形於辭色。廉親王允禩又復逞其離間,肆為讒言。安郡王爵不準承襲。」乾隆四十三年,高宗以阿巴泰、岳樂屢著功績,封華?孫奇昆輔國公,世襲。

溫良貝子博和託,阿巴泰第二子。初封輔國公。崇德元年,從征朝鮮,圍南漢山城,。三年,從伐明,自董家口略明都西南六府,入山西界。移師?偕尼堪擊走其援兵,斬殪甚克濟南。師還,賜銀二千。七年,從阿巴泰伐明,自黃崖口入。及還,賜銀三千。順治元年,從入關,破李自成,進貝子。三年,從多鐸擊喀爾喀蘇尼特部騰機思、騰機特等。五年九月,卒,予謚。子六,彰泰,襲貝子。

彰泰襲爵,進封。康熙十三年春,吳三桂陷湖南,上命貝勒尚善為大將軍,率師下嶽州,以彰泰參贊軍務。十五年,詔責行師延緩。彰泰與尚善議水陸並進,遣額司泰等破敵洞庭湖,獲舟五十餘。敵立椿套湖峽口阻我師。十七年,督兵伐椿,棹輕舟破敵柳林嘴,發?毀其船。八月,尚善卒於軍,貝勒察尼代為大將軍,授彰泰撫遠將軍。九月,督兵出南津港。十月,破敵陸石口,屯白米灘,絕三桂兵運道。十八年,三桂將陳珀等以乏食出降,吳應麒走衡州。都統珠滿等克湘陰,彰泰克華容、石首。會安親王岳樂復長沙,簡親王喇布復衡州,詔彰泰與會師。自衡州進攻武岡,擊破三桂將吳國貴等。十一月,召嶽樂還京師,命彰泰代為定遠平寇大將軍。

十九年,復沅州,靖州,三桂所置綏寧諸將吏及附近土司俱降。疏言:「將軍蔡毓榮調遣漢兵,今進取貴州,若不相聞,恐礙事機。」詔毓榮軍事關白大將軍。十月,次鎮遠,關,截其隘,而與毓榮督兵躪敵壘。所遣兵亦奪十向口,破敵大巖門,逐之至?遣兵攻鎮遠,趨貴陽。三桂孫世璠及應麒等俱走還雲南。迭克安?,遂復鎮遠。進下平越及新添?偏橋順、石阡、都勻、思南諸府。十一月,復永寧,破敵安籠鋪,逐之至雞公背山鐵索橋,師駐貴陽。詔趣彰泰進規云南。

二十年正月,渡盤江,破敵沙子哨,進次臘茄坡,復新興所,逐北三十里,克普安、霑益。大將軍賚塔自廣西入曲靖,會於嵩明州,合圍雲南會城,距三十里。世璠將胡國柄、劉起龍等以萬餘人列象陣拒戰。賚塔軍其右,彰泰軍其左,自卯達午,殊死戰,破敵陣,斬、走馬街、雙塔寺、得勝橋、重關諸地?國柄、起龍等,俘獲無算。令諸軍分扼南壩、薩石,於是大理、臨安、永順、姚安、武定世璠所置將吏,相繼詣軍前降。

世璠將馬寶、胡國柱等自四川,夏國相自廣西,還救雲南,彰泰遣兵迎擊,寶次姚安,亦乞降。國柱走鶴慶、麗江,希福攻雲龍州,國柱自經死。國相走廣西,李國樑等圍之西板橋,國相亦降,與寶同檻送京師。將軍趙良棟師自四川至,彰泰偕賚塔及良棟等屢破敵南壩、得勝橋、太平橋、走馬街諸地。師薄城環攻,世璠自經死,其將何進忠等出降。彰泰戒將士毋殺掠,入城安撫,收倉庫,戮世璠尸,函首獻闕下。雲南平。授左宗正。二十一年十月,師還,上迎勞盧溝橋南二十里。

二十二年,議初下嶽州遷延罪,以功不坐。賜金二十、銀千。二十四年,坐濫舉宗人府屬官,罷左宗正。二十九年正月,卒。子屯珠,襲鎮國公。授左宗正、禮部尚書。五十七年,卒。贈貝子,謚恪敏。孫逢信,以輔國公世襲。

博洛,阿巴泰第三子。天聰九年,從伐明,有功。崇德元年,封貝子。二年,與議政。三年,授理籓院承政。從攻寧遠,趨中後所。明將祖大壽襲我軍後,巴牙喇纛章京哈寧阿等與相持,博洛突前奮擊,大壽引?。五年,從濟爾哈朗迎來歸蒙古蘇班岱,擊敗明兵,賜良馬。尋與諸王更番圍錦州。六年,洪承疇以十三萬人援錦州,博洛偕阿濟格擊之,至塔山,獲筆架山積粟;又偕羅洛渾等設伏阿爾齋堡,擊敗明將王樸、吳三桂。

順治元年,從入關,破李自成,進貝勒。從多鐸征河南。二年,破自成潼關。多鐸南征,下江寧,分師之半授博洛,下常州、蘇州,趨杭州,屢敗明兵。師臨錢塘江岸,明兵以為江潮方盛,營且沒,會潮連日不至,明潞王常淓以杭州降,淮王常清亦自紹興降。克嘉興,徇吳江,破明將吳易,攻江陰亦下。師還,賜金二百、銀萬五千、鞍馬一。

三年,命為征南大將軍,率師駐杭州。明魯王以海監國紹興,明將方國安營錢塘江東,亙二百里。師無舟,會江沙暴漲,固山額真圖賴等督兵徑涉,國安驚遁,以海走臺州。師入紹興,進克金華,擊殺明蜀王盛濃等,再進克衢州,浙江平。明唐王聿鍵據福建,博洛率師破仙霞關,克浦城、建寧、延平。聿鍵走汀州,遣阿濟格、尼堪、努山等率師從之,克汀州,擒聿鍵及曲陽王盛渡等。明將姜正希以二萬人夜來襲,擊之?,斬萬餘級。又破敵分水關,克崇安。梅勒額真卓布泰等克福州,斬所置巡撫楊廷清等,降其將鄭芝龍等二百九十餘人、馬步兵十一萬有奇。師復進,下興化、漳州、泉州諸府。十一月,遣昂邦章京佟養甲徇廣東,克潮州、惠州、廣州,擊殺明唐王聿及諸王世子十餘人,承制以養甲為兩廣總督。四年,師還,進封端重郡王。五年,以所獲金幣、人口賚焉。

偕阿濟格防喀爾喀,徇大同,討叛將姜瓖。六年正月,偕碩塞援代州,克其郛。三月,瓖將馬得勝以五千自北山逼我師,博洛率千餘騎應之,與巴牙喇纛章京鰲拜等奮擊,大破之,斬馘過半,瓖閉城不敢出。睿親王多爾袞自京師至軍議撫,承制進親王,命為定西大將軍。移師汾州,下清源、交城、文水、徐溝、祁諸縣,戰平陽、絳州;又遣軍克孝義,戰壽陽、平遙、遼州、榆次:屢捷。英親王阿濟格、敬謹親王尼堪圍大同,巽親王滿達海、謙郡王瓦克達定朔州、寧武。召博洛還京師,疏言:「太原、平陽、汾州所屬諸縣雖漸次收復,然未下者尚多,恐撤軍後,賊乘虛襲踞,請仍留守御。」上從之。瓖既誅,與滿達海合軍克汾州,復嵐、永寧二縣,戰絳州孟城驛、老君廟諸地,盡殲瓖餘黨,乃還師。七年,偕滿達海、尼堪同理六部事。再坐事,降郡王。世祖親政,復爵。尋命掌戶部。九年三月,薨,謚曰定。

子齊克新,襲。十六年,追論博洛分多爾袞遺財,又掌戶部時尚書譚泰逞私攬權,不力阻,奪爵、謚,齊克新降貝勒。十八年,卒,謚懷思。無子,爵除。博洛子塔爾納封郡王,卒,謚敏思。坐博洛罪,追奪爵。

悼愍貝子蘇布圖,阿巴泰孫。父尚建,追封貝子,謚賢?。蘇布圖初封輔國公。順治二年,從勒克德渾駐江寧,移師征湖廣。三年,從定荊州、襄陽有功,賜金五十、銀千,進貝子。五年,復從濟爾哈朗徇湖廣,卒於軍,謚悼愍。子顏齡,封鎮國公。卒。無子,爵除。蘇布圖弟強度,封貝子,謚介潔,亦不襲。

鎮國恪?公巴布泰,太祖第九子。天命十年,偕阿拜、塔拜伐東海北路呼爾哈部,有功。十一年,命理正黃旗事。天聰四年,從阿敏駐永平。明兵攻灤州,巴布泰不能御,坐罷。八年,授梅勒額真。從伐明,克保安州。巴布泰匿所獲不以聞,復坐罷。崇德六年,授三等奉國將軍。順治元年,從入關,逐李自成至慶都。二年,進一等。三年,從勒克德渾伐湖廣,戰安遠、南漳、西峰口、關王嶺、襄陽,屢破敵。四年,進輔國公。六年,偕務達海討姜瓖,進鎮國公。十二年正月,卒,予謚。子噶布喇,封輔國公;祜錫祿,襲三等鎮國將軍。其後並以奉恩將軍世襲。

德格類,太祖第十子。初授臺吉。天命六年,師略奉集堡,將還,有一卒指明兵所在,德格類偕岳託、碩託進擊之,擊敗明將李秉誠。復偕臺吉寨桑古閱三岔河橋,至海州,城中官民張樂舁輿迎德格類等,令軍士毋擾民,毋奪財物,毋宿城上,毋入民居。翌日,遣視三岔河者還報橋毀無舟楫,乃還。八年,偕阿巴泰伐喀爾喀扎嚕特部。十一年,復從代善伐扎嚕特部。天聰三年,偕濟爾哈朗略錦州,焚其積聚。?功,進和碩貝勒。

五年,初設六部,掌戶部事。從圍大凌河,德格類率師策應,擊破明監軍道張春。十。六年,偕濟爾哈朗等略歸化城。復?月,祖大壽降,偕阿巴泰等偽為明軍襲錦州,擊斬甚偕岳託略地,自耀州至蓋州迤南。七年,攻克旅順口。八年,從伐明,撫定蒙古來歸人戶。克獨石口。攻赤城,未拔。入保安州,會師應州,還。九年十月,卒。上臨其喪,痛悼之,漏盡三鼓乃還。設幄坐其中,撤饌三日。

逾月,莽古爾泰既卒,為冷僧機所訐,以大逆削籍,德格類坐同謀,追削貝勒。子鄧從豪格征張獻忠,戰死,世祖詔其子輝爾食一等阿?什庫,並坐,削宗籍;德克西克,以侍思哈尼哈番俸。子五,云柱,授一等阿達哈哈番。康熙五十二年,聖祖命復宗籍,賜紅帶。

巴布海,太祖第十一子。初授牛錄章京。天聰八年,授一等甲喇章京。嘗命偕鎮國將軍阿拜祭陵,巴布海不待阿拜,先往祭。牛未至,取民牛代,以祭牛償民,民以小不受,訟焉,罰銀三十償民,又不與,再訟。巴布海聞上,上責其愚黯,且謂其受制於妻,妻,揚古利女也。崇德四年,授梅勒額真,封鎮國將軍。七年,巴布海語固山額真譚泰曰:「原罷我梅勒額真。堪為梅勒額真者,多於草木!」譚泰語折之,誓曰:「若口與心違者,天日鑒之!」圖海奉命差擇牛錄貧富,巴布海曰:「我所領牛錄甚富。」語聞,巴布海曰:「我非太祖之子歟?譚泰等顧厚誣我。」廷鞫皆實,罪當死,上寬之,但奪爵。世祖即位,有為飛書訐譚泰者,投一等公塔瞻第。鞫其僕,謂得之巴布海家。內監逮訊,不承,巴布海及其妻並子阿喀喇皆坐死,籍其家予譚泰。順治九年,譚泰誅,乃以其孥及遺產畀巴布泰。

阿濟格,太祖第十二子。初授臺吉。天命十年,從貝勒莽古爾泰伐察哈爾,至農安塔。十一年,偕臺吉碩託伐喀爾喀巴林部,復從貝勒代善伐扎魯特,皆有功,授貝勒。天聰元塔山糧運。會師錦州,薄寧遠,?年,偕貝勒阿敏伐朝鮮,克五城。從上伐明,偕莽古爾泰明兵千餘人為車營,掘壕,前列火器,阿濟格擊殲之。總兵滿桂出城陣,上欲進擊,諸貝勒以距城近,諫不可,獨阿濟格請從。上督阿濟格馳擊明騎兵至城下,諸貝勒皆慚,奮不及胄,亦進擊其步軍,明兵死者大半。二年,以擅主弟多鐸婚,削爵,尋復之。

三年,偕濟爾哈朗略明錦州、寧遠,焚其積聚,俘三千。復從上伐明,克龍井關,下漢兒莊城,克洪山口。進次遵化,擊斬明總兵趙率教。薄明都,袁崇煥、祖大壽以兵二萬赴援,屯廣渠門外,師逐之,迫壕,阿濟格馬創,乃還。尋偕阿巴泰等略通州,至張家灣。尋從上閱薊州,遇明山海關援兵,阿濟格偕代善突入敵陣,大破之。

四年,復從伐明,趨廣寧,會師大凌河。夜圍錦州,明兵襲阿濟格營,霧不見人,阿濟格嚴陣待。青氣降,霧豁若門闢,急縱擊,獲明裨將一、甲械及馬二百餘。上酌金?親勞之,授圍城方略。尋聞明增兵,上命揚古利率八旗巴牙喇兵之半以益軍。大壽弟大弼逐我軍中偵騎近上前,上擐甲與戰,阿濟格馳至,明兵步騎鵷出,奮擊?之,斬明裨將一。上以所統兵付阿濟格,明監軍道張春援至,又戰於大凌河,截殺過半,逐北四十里。

六年,從伐察哈爾,林丹汗遁。上移師伐明,令阿濟格統左翼及蒙古兵略大同、宣府,盡得張家口所貯犒邊財物。七年,城通遠堡,迎降將孔有德,拒明及朝鮮兵。詔問攻明及朝鮮、察哈爾三者何先,阿濟格言當攻明。偕阿巴泰略山海關,詔責其不深入,阿濟格言;「臣欲息馬候糧,諸貝勒不從。」上曰:「汝果堅不還,諸貝勒將棄汝行乎?」八年,從伐明,克保安,拔靈丘。

崇德元年,進武英郡王。偕饒餘貝勒阿巴泰及揚古利伐明,自雕鶚堡入長安嶺,薄延慶。越保定至安州,克昌平、定興、安肅、寶坻、東安、雄、順義、容城、文安諸縣,五十六戰皆捷,俘人畜十餘萬。又遣固山額真譚泰等設伏,斬遵化三屯營守將,獲馬百四十餘。得優旨,賜鞍馬一。師還,上迎勞地載門外十里,見阿濟格勞瘠,為淚下,親酌金?勞之。上伐朝鮮,命守牛莊。二年,碩託攻皮島未下,阿濟格督所部水陸並進,克之。上遣使褒勞。

四年,從伐明,阿濟格揚言欲以紅衣?攻臺,守者懼,四里屯、張剛屯、寶林寺、旺民屯、於家屯、成化峪、道爾彰諸臺俱下。尋還守塔山、連山,俘人馬千計。復偕阿巴泰略錦州、寧遠。六年,偕濟爾哈朗圍錦州。守郛蒙古臺吉吳巴什等議舉城降,祖大壽覺之,擊蒙古兵,阿濟格夜登陴助戰,明兵敗,徙蒙古降者於義州。屢擊敗明兵,賜銀四千。

洪承疇率諸將王樸、吳三桂等援錦州,號十三萬。上親視師,營松山。明兵奔塔山,阿濟格追擊之,獲筆架山積粟,又偕多爾袞克敵臺四,擒明將王希賢等,樸、三桂僅以身免。明兵猶守錦州、松山、杏山、高橋諸地,上還盛京,命阿濟格偕杜度、多鐸等圍之。承疇二千皆降。七年,圍?環射之,明兵敗還,城閉不得入,其?夜出松山襲我軍,阿濟格等督杏山,遣軍略寧遠。三桂以四千人駐塔山、高橋,不戰而退,縱兵四擊,又迭敗之。八年,復,攻城西,斬馘四千餘?偕濟爾哈朗攻寧遠,軍城北,布雲梯發?,城?,克之;抵前屯,明總兵黃色棄城遁,復克之。

順治元年,從入關破李自成,進英親王,賜鞍馬二。命為靖遠大將軍,自邊外入陜西,斷自成歸路,八戰皆勝,克城四,降城三十八。時自成為多鐸所敗,棄西安走商州。詔多尚二十萬,規取南京。阿濟格以師從之?鐸趨淮、揚,而命阿濟格率師討自成。自成南走,及於鄧州,復南至承天、德安、武昌、富池口、桑家口、九江,屢破敵,自成走死,斬其將劉宗敏,俘宋獻策。宗敏,自成驍將;獻策,自成所倚任,號軍師者也。

明將左良玉子夢庚方駐軍九江,師至,執總督袁繼咸等,率馬步兵十萬、舟數萬,詣軍門降。是役凡十三戰,下郡縣:河南十二,湖廣三十九,江西、江南皆六。捷聞,上使赴軍慰勞,詔曰:「王及行間將士馳驅跋涉,懸崖峻嶺,深江大河,萬有餘里,勞苦功高。寇氛既靖,宜即班師。其招撫餘兵,或留或散,王與諸大臣商榷行之。」詔未至,阿濟格率師還京師。睿親王多爾袞責阿濟格不候詔班師,又自成未死時,先以死聞,遣人數其罪;又在午門張蓋坐,召而斥之。復議方出師時,脅宣府巡撫李鑒釋逮問赤城道硃壽?及擅取鄂爾多斯、土默特馬,降郡王。尋復之。五年,剿天津、曹縣土寇。十一月,率師駐大同,姜瓖叛,督兵討之。旋命為平西大將軍,率固山額真巴顏等討瓖。六年,瓖將劉遷犯代州,遣博洛赴援,圍乃解。

多爾袞至大同視師,時阿濟格兩福晉病卒,命歸視,阿濟格曰:「攝政王躬攝大政,為國不遑,吾敢以妻死廢國事?」阿濟格自以功多,告多爾袞曰:「輔政德豫親王征流寇至,追騰機思不取,功績未著,不當優異其子。鄭親?慶都,潛身僻地,破潼關、西安不殲其王乃叔父之子,不當稱『叔王』。予乃太祖之子,皇帝之叔,宜稱『叔王』。」多爾袞斥其妄,令勿預部務及交接漢官。尋復偕鞏阿岱攻大同,會降將楊振威斬瓖降,隳其城睥睨五尺,乃還。八年正月,多爾袞薨於喀喇城,阿濟格赴喪次,諸王夜臨,獨不至,召其子郡王勞親以兵脅多爾袞所屬使附己。喪還,上出迎,阿濟格不去佩刀。勞親兵至,阿濟格張纛與合軍。多爾袞左右訐阿濟格欲為亂,鄭親王濟爾哈朗等遣人於路監之。還京師,議削爵,幽禁。逾月,復議系別室,籍其家,諸子皆黜為庶人。十月,監守者告阿濟格將於系所舉火,賜死。

阿濟格子十一,有爵者三:和度、傅勒赫、勞親。和度,封貝子,先卒。勞親與阿濟格同賜死。

傅勒赫,初封鎮國公。坐奪爵,削宗籍。十八年,諭傅勒赫無罪,復宗籍。康熙元年,追封鎮國公。子構孳、綽克都,並封輔國公。綽克都,事聖祖。從董額討王輔臣,守漢中,攻秦州,師無功。授盛京將軍,又以不稱職,奪爵。上錄阿濟格功,以其子普照仍襲輔國公,坐事奪爵,以其弟經照仍襲輔國公。雍正間,普照亦以軍功復爵,卒。世宗諭曰:「普照軍前?力,且其兄女為年羹堯妻,故特予封爵。今羹堯負恩誅死,此爵不必承襲。」居數年,經照亦坐事,奪爵。普照、經照皆能詩。乾隆四十三年,命阿濟格之裔皆復宗籍。經照子孫遞降,以奉恩將軍世襲。

輔國介直公賴慕布,太祖第十三子。天聰八年,授牛錄章京。崇德四年,與議政。七年,從阿濟格伐明,敗寧遠兵。上御篤恭殿賚師,阿濟格不待賞先歸。賴慕布坐不勸阻,奪職,罷議政。順治二年,封奉恩將軍。三年,卒。十年五月,追封謚。子來祜,襲。累進輔國公。坐事,奪爵。高宗以其孫扎昆泰襲奉恩將軍,一傳,命停襲。


\end{pinyinscope}