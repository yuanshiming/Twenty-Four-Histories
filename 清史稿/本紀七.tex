\article{本紀七}

\begin{pinyinscope}
聖祖本紀二

二十一年壬戌春正月壬戌,上元節,賜廷臣宴,觀燈,用柏梁體賦詩。上首唱云:「麗日和風被萬方。」廷臣以次屬賦。上為制升平嘉宴詩序,刊石於翰林院。丙寅,調蔡毓榮為雲貴總督。戊辰,王大臣奏曰:「耿精忠累世王封,甘心叛逆,分擾浙、贛,及於皖、徽,設非師武臣力,蔓延曷極。李本深、劉進忠等多年提鎮,高官厚祿,不能革其鴞音,俯首從賊,抑有何益。均宜從嚴懲治,大為之防,以為世道人心之範。謹擬議請旨。」得旨:耿精忠、曾養性、白顯中、劉進忠、李本深均磔死梟首。耿精忠之子耿繼祚,李本深之孫李象乾、李象坤,其侄李濟祥、李濟民,暨祖弘勛等俱處斬。為賊絓誤之陳夢雷、李學詩、金境、田起蛟均減死一等。己巳,特封安親王岳樂子岳希為僖郡王。

二月庚辰,以達都為左都御史。癸未,以平滇遣官告祭岳瀆、古帝陵、先師闕里。甲申,上御經筵。丙戌,以佟國維為領侍衛內大臣。辛卯,上齋居景山,為太皇太后祝釐。癸巳,上東巡,啟鑾。皇太子胤礽從。蒙古王貝勒等請上尊號,不許。以穆占為蒙古都統。妖人硃方旦伏誅。戊戌,次山海關,遣大臣祭伯夷、叔齊廟。

三月壬子,上謁福陵、昭陵,駐蹕盛京。甲寅,告祭於福陵。丙辰,告祭於昭陵。大賚將軍以下,至守陵官、年老致仕官及甲兵廢閒者。曲赦盛京、寧古塔。蠲蹕路所過租稅。己未,上謁永陵,行告祭禮。上具啟太皇太后、皇太后進奉達魚、魚。庚申,上由山道幸烏拉行圍。辛酉,望祭長白山。乙亥,泛舟松花江。

夏四月辛巳,上回鑾。賜寧古塔將軍、副都統宴,賚致仕官及甲士。乙巳,次中後所。流人王廷試子德麟叩閽乞代父戍,部議不準。上諭:「王德麟所言情甚可憫。遇朕來此,亦難得之遭。其父子俱讀書人,可均釋回。」

五月辛亥,上還京。壬子,詔寧古塔地方苦寒,流人改發遼陽。己未,大學士杜立德乞休,溫旨允之。丙寅,免吉林貢鷹,減省徭役。戊辰,以王熙為大學士。

六月乙酉,以佟國瑤為福州將軍。庚寅,以公倭赫為蒙古都統。甲辰,大學士馮溥乞休,溫旨允之,差官護送,馳驛回籍。

秋七月庚戌,以杭艾為左都御史。甲寅,命刑部尚書魏象樞、吏部侍郎科爾坤巡察畿輔,豪強虐民者拘執以聞。乙卯,以三逆蕩平宣示蒙古。

八月丙子,詔內閣學士參知政事。癸卯,譚弘之子譚天祕、譚天倫伏誅。

九月戊申,賜蔡升元等一百七十六人進士及第出身有差。甲子,詔每日御朝聽政,春夏以辰初,秋冬以辰正。

冬十月甲申,定遠大將軍貝子彰泰、征南大將軍都統賴塔凱旋,上郊勞之。己丑,以黃機、吳正治為大學士。辛卯,詔重修太祖實錄,纂修三朝聖訓、平定三逆方略。

十一月甲寅,以李之芳為兵部尚書,希福為西安將軍,瓦岱為江寧將軍。戊午,詔廣西建雙忠祠,祀巡撫馬雄鎮、傅弘烈。庚申,以趙賴為漢軍都統。戊辰,以施維翰為浙江總督,以噶爾漢為滿洲都統。

十二月己卯,前廣西巡撫陳洪起從賊論死,命流寧古塔。癸未,以許貞為廣東提督。戊子,錄達海之孫陳布祿為刑部郎中。癸巳,論行軍失律罪,簡親王喇布奪爵,餘遣戍降黜有差。庚子,郎談使黑龍江還,上羅剎犯邊事狀。命寧古塔將軍巴海、副都統薩布素率師防之。建木城於黑龍江、呼馬爾,分軍屯田。

是歲,免直隸、江南、江西、山東、山西、浙江、湖廣等省七十八州縣衛被災額賦有差。朝鮮、安南入貢。

二十二年癸亥春正月乙卯,宴賚廷臣。己未,上閱官校較射。

二月癸酉,帥顏保罷,以介山為禮部尚書,喀爾圖為刑部尚書。甲申,上幸五臺山。

三月戊申,還京。戊午,以噶爾漢為荊州將軍,彭春為滿洲都統。

夏四月乙亥,命提鎮諸臣以次入覲。庚辰,命巴海回駐烏拉,薩布素、瓦禮祜帥師駐額蘇裡備邊。辛卯,以公坡爾盆為蒙古都統。

五月丙午,設漢軍火器營。甲子,命施瑯征臺灣。

六月丁丑,上閱內庫,頒賚廷臣幣器。戊寅,以伊桑阿為吏部尚書,杭艾為戶部尚書。癸未,上奉太皇太后避暑古北口。

閏六月戊午,施瑯克澎湖。庚申,諭飭刑官勘獄勿淹系。

秋七月,車駕次胡圖克圖,賜隨圍蒙古王公冠服,兵士銀幣。甲午,上奉太皇太后還宮。

八月庚子,命經筵大典,大學士以下侍班。戊申,以哈占為兵部尚書,科爾坤為左都御史。戊辰,施瑯疏報師入臺灣,鄭克塽率其屬劉國軒等迎降,臺灣平。詔錫克塽、國軒封爵,封施瑯靖海侯,將士擢賚有差。

九月癸酉,以丁思孔為偏沅巡撫。己卯,上奉太皇太后幸五臺山。壬辰,次長城嶺,太皇太后以道險回鑾。上如五臺山。限額魯特入貢人數。

冬十月,上至五郎河行宮,奉太皇太后還京。丁未,群臣以臺灣平,請上尊號,不許。癸亥,以薩布素為新設黑龍江將軍。乙丑,詔沿海遷民歸復田里。

十一月癸未,授羅剎降人宜番等官。戊子,上以海寇平,祭告孝陵。癸巳,上巡幸邊界。

十二月甲辰,上還京。丁未,從逆土司陸道清伏誅。壬子,以紀爾他布為蒙古都統。乙卯,易經日講成,上制序頒行。尚書硃之弼、左都御史徐元文以薦舉非人免。乙丑,祫祭太廟。

是歲,免山東、山西、甘肅、江西、湖廣、廣西等省二十州縣災賦有差。朝鮮、琉球入貢。

二十三年甲子春正月辛巳,上幸南苑行圍。丙戌,加封安親王岳樂子袁端為勤郡王。壬辰,命整肅朝會禮儀。羅剎踞雅克薩、尼布潮二城,飭斷其貿易,薩布素以兵臨之。

二月乙巳,上御經筵。癸丑,上巡幸畿甸。丙寅,還駐南苑。大學士黃機罷。乙丑,給事中王承祖疏請東巡,命查典禮以聞。

三月壬申,以劉國軒為天津總兵官,陛辭,賜白金二百、緞匹三十、內廄鞍馬一。丁亥,上制五臺山碑文,召示廷臣。諭之曰:「近人每一文出,不樂人點竄,此文之所以不工也。」

夏四月己酉,設臺灣府縣官,隸福建行省。壬子,刑部左侍郎宋文運乞休,命加太子少保致仕。庚申,諭凡一事經關兩部,俱會同具奏。乙丑,諭講官:「講章以精切明晰為尚,毋取繁衍。朕閱張居正尚書、四書直解,義俱精實,無泛設之詞,可為法也。」江南江西總督於成龍卒,予祭葬,謚清端。

五月丁卯,裁浙江總督。以公瓦山為滿洲都統。己巳,修大清會典。丙子,以孫思克為甘肅提督。辛巳,命廷臣察舉清廉官。九卿舉格爾古德、蘇赫、範承勛、趙侖、崔華、張鵬翮、陸隴其。癸未,起巴海為蒙古都統。甲申,上幸古北口,詔蹕路所經勿踐田禾。乙未,惠郡王博翁果諾坐陪祀不謹削爵。王大臣議奏侍郎宜昌阿、巡撫金僑查看尚之信家產,隱蝕銀八十九萬,並害殺商人沈上達,應斬。郎中宋俄託、員外郎卓爾圖及審讞不實之侍郎禪塔海應絞。從之。詔追銀勿入內務府,交戶部充饟。

六月丁未,琉球請遣子弟入國子監讀書。許之。甲寅,暹羅國王森列拍臘照古龍拍臘馬呼陸坤司由提呀菩挨遣陪臣言貢船到虎跳門,阻滯日久,每致損壞。乞諭粵省官吏準其放入河下,早得登岸,貿易採辦,勿被攔阻。從之。諭一等侍衛阿南達曰:「朕視外旗蒙古與八旗一體。今巡行之次,見其衣食困苦,深用惻然。爾即傳諭所過地方蒙古無告者,許其來見,詢其生計。」於是蒙古扶老手巂幼,叩首行宮門。上詳問年齒生計,給與銀兩布疋。乙卯,上閱牧群,賜從臣馬。刑部尚書魏象樞再疏乞休。允之。丁巳,以湯斌為江蘇巡撫。

七月乙亥,以宋德宜為大學士。辛巳,上駐蹕英尼湯泉。以佟佳為蒙古都統。

八月戊申,上還京。甲寅,大學士李霨卒,遣官奠茶酒,賜祭葬,謚文勤。甘肅提督靖逆侯張勇卒,予祭葬,謚襄壯。

九月甲子朔,停本年秋決。丙寅,以張士甄為刑部尚書,博濟為滿洲都統。以錢貴,更鑄錢,減四分之一。聽民採銅鉛,勿稅。丁卯,改梁清標為兵部尚書,餘國柱為戶部尚書。庚午,以蒙古都統阿拉尼兼理籓院尚書。癸酉,以陳廷敬為左都御史,莽奕祿為蒙古都統。丁亥,詔南巡車駕所過,賜復一年。辛卯,上啟鑾。

冬十月壬寅,上次泰安,登泰山,祀東嶽。辛亥,次桃源,閱河工,慰勞役夫,戒河吏勿侵漁。臨視天妃閘。與河臣靳輔論治河方略。壬子,上渡淮。甲寅,次高郵湖,登岸行十餘里,詢耆老疾苦。丙辰,上幸焦山、金山,渡揚子江,舟中顧侍臣曰:「此皆戰艦也。今以供巡幸,然艱難不可忘也。」丁巳,弛海禁。戊午,上駐蘇州。庚申,幸惠山。諭巡撫:百姓遠道來觀,其不能歸者資遣之。

十一月壬戌朔,上駐江寧。癸亥,詣明陵致奠。乙丑,回鑾。泊舟燕子磯,讀書至三鼓。侍臣高士奇請曰:「聖躬過勞,宜少節養。」上曰:「朕自五齡受書,誦讀恆至夜分,樂此不為疲也。」丁卯,命伊桑阿、薩穆哈視察海口。諭曰:「海口沙淤年久,遂至壅塞。必將水道疏通,始免昏墊。即多用經費,亦所不惜。」辛未,臨閱高家堰。次宿遷。過白洋河,賜老人白金。戊寅,上次曲阜。己卯,上詣先師廟,入大成門,行九叩禮。至詩禮堂,講易經。上大成殿,瞻先聖像,觀禮器。至聖跡殿,覽圖書。至杏壇,觀植檜。入承聖門,汲孔井水嘗之。顧問魯壁遺跡,博士孔毓圻占對甚詳,賜官助教。詣孔林墓前酹酒。書「萬世師表」額。留曲柄黃蓋。賜衍聖公孔毓埏以次日講諸經各一。免曲阜明年租賦。庚寅,上還京。以馬哈達為滿洲都統。

十二月壬辰朔,以石文炳為漢軍都統。癸卯,命公瓦山視師黑龍江,佟寶、佛可托副之,備羅剎。甲辰,賜公鄭克塽、伯劉國軒、馮錫範田宅,隸漢軍。丙午,命流人值冬令,過嚴寒時乃遣。丙辰,上謁陵,賜守陵官兵牛羊。己未,還宮。

是歲,免直隸、江南、江西、河南、湖廣等省二十六州縣災賦有差。朝鮮、暹羅入貢。

二十四年乙丑春正月癸酉,享太廟。諭曰:「贊禮郎讀祝,讀至朕名,聲輒不揚,失父前子名之義。自今俱令宣讀。」癸未,命公彭春赴黑龍江督察軍務。命侯林興珠率福建藤牌兵從之。以班達爾沙、佟寶、馬喇參軍事。乙丑,試翰詹官於保和殿,上親定甲乙,其不稱者改官。戊子,命蒙古科爾沁十旗所貢牛羊送黑龍江軍前。

二月庚子,命周公後裔東野氏為五經博士,予祀田。以額赫納為滿洲都統。癸卯,上御經筵。乙卯,上巡幸畿甸。庚申,還京。再賜劉國軒第宅。以範承勛為廣西巡撫。

三月壬戌,上撰孔子廟碑文成,親書立碑。重修賦役全書。辛巳,賜陸肯堂等一百二十一人進士及第出身有差。

夏四月辛卯,予宋儒周敦頤裔孫五經博士。丙申,授李之芳輕車都尉世職。戊戌,馬喇以所俘羅剎上獻,命軍前縱遣之。辛丑,詔以直隸連年旱災,逋賦六十餘萬盡免之,並免今年正賦三分之一。詔醫官博採醫林載籍,勒成一書。庚戌,設內務府官學。

五月癸未,詔厄魯特濟農違離本部,鄉化而來,宜加愛養,予之田宅。修政治典訓。甲申,以原廣西巡撫郝浴歷官廉潔,悉免應追帑金。彭春等攻雅克薩城,羅剎來援,林興珠率藤牌兵迎擊於江中,破之,沈其船,頭人額里克舍乞降。

六月庚寅朔,上巡幸塞外,啟鑾。戊戌,上還京。癸卯,詔曰:「鄂羅斯入我邊塞,侵擾鄂倫春、索倫、赫哲、飛牙喀等處人眾,盤踞雅克薩四十年。今克奏厥績,在事人員,咸與優敘。應於何地永駐官兵,即會議具奏。」上試漢軍筆帖式、監生,曳白八百人,均斥革,令其讀書再試。乙巳,上巡幸塞外。

秋七月壬申,設吉林、黑龍江驛路,凡十九驛。

八月丙午,上駐蹕拜巴哈昂阿,賜朝行在蒙古王貝勒冠服銀幣。

九月戊午朔,上聞太皇太后違豫,回鑾。己未,上馳回京,趨侍醫藥,旋即康復。辛巳,陜西提督王進寶卒,贈太子太保,予祭葬,謚忠勇。甲申,命副都統溫代、納秦駐防黑龍江,博定修築墨爾根城,增給夫役,兼令屯田。乙酉,以吳英為四川提督。

冬十月甲午,上幸南苑。戊戌,厄魯特使人伊特木坐殺人棄市。己亥,以瓦代為滿洲都統。庚子,定外籓王以下,歲貢羊一只、酒一瓶。丙午,慶雲見。己酉,靳輔請下河涸出田畝,佃民收價償工費。上曰:「如是則累民矣。其勿取。」甲寅,以博霽為江寧將軍。

十一月丁巳朔,日有食之。庚申,以莽奕祿為滿洲都統,塔爾岱為蒙古都統。甲戌,上大閱於盧溝橋。丙子,靳輔、於成龍遵召至京,會議治河方略。靳輔議開六河建長堤。於成龍請開濬海口故道。大學士以聞。上云:「二說俱有理,可詢高、寶七州縣京官,孰利民。」侍讀喬萊奏,從於成龍議,則工易成,而百姓有利。上令於成龍興工。旋以民情不便而止。己卯,上賜鄂內、坤巴圖魯散秩大臣,聽其家居,二人皆太宗朝舊臣也。乙酉,詔曰:「日蝕於月朔,越十六日月食。一月之中,薄蝕互見。天象示儆,宜亟修省。廷臣集議以聞。」

十二月庚寅,以察尼為奉天將軍。己亥,謁孝陵。癸卯,上還宮。甲寅,祫祭太廟。

是歲,免江南、江西、山東、山西、湖廣等省七十四州縣衛災賦有差。朝鮮、琉球、噶爾丹入貢。

二十五年丙寅春正月丙申,命馬喇督黑龍江屯田。鄂羅斯復據雅克薩,命薩布素率師逐之。

二月甲辰,重修太祖實錄成。丁未,詔曰:「國家削平逆孽,戡定遐荒,惟宜宣布德意,動其畏懷。近見雲、貴、川、廣大吏,不善撫綏,頗行苛虐,貪黷生事,假借邀功。朕思土司苗蠻,既歸王化,有何杌隉,格★靡寧。其務推示誠信,化導安輯,以副朕撫馭遐荒至意。」停四川採運木植。己酉,文華殿成。壬子,告祭至聖先師於傳心殿。癸丑,上御經筵。以津進為領侍衛內大臣。

三月戊午,命修棲流所。己未,命纂修一統志。甲戌,以湯斌為禮部尚書,兼管詹事府。

夏四月乙酉朔,命阿拉尼往喀爾喀七旗蒞盟。庚寅,詔曰:「趙良棟前當逆賊盤踞漢中,首先入川,功績懋著。復領兵直抵雲南,攻克省城之後,獨能恪守法紀,廉潔自持,深為可嘉。今已衰老解任,應復其勇略將軍、兵部尚書、總督以示眷注。」命郎談、班達爾沙、馬喇赴黑龍江參贊軍務。贈陜西死事平逆將軍畢力克圖、參贊阿爾瑚世職。甲午,詔求遺書。戊申,調萬正色雲南提督,以張雲翼為福建陸路提督。辛亥,始令順天等屬旗莊屯丁,編查保甲,與民戶同。

閏四月辛未,以範承勛為雲南貴州總督。

五月丁亥,詔毀天下淫祠。

六月乙亥,錄平南大將軍賴塔、都統趙賴以次功,各予世職有差。戊寅,以阿蘭泰為左都御史。

秋七月己酉,錫荷蘭國王耀漢連氏甘勃氏文綺白金,命其使臣賚書致鄂羅斯。吏部奏定侍讀、庶子以下各官學問不及者,以同知、運判外轉。從之。辛亥,上巡幸塞外。

八月辛未,上駐蹕烏爾格蘇臺。丙子,上還京。以索額圖為領侍衛內大臣。丁丑,詔薩布素圍雅克薩城,遏其援師,以博定參軍事。戊辰,詔天下學宮崇祀先儒。庚辰,詔增孔林地十一頃有奇,從衍聖公孔毓埏請也,除其賦。

九月己丑,以班達爾沙為蒙古都統。乙巳,以圖納為四川陜西總督。丁未,以陳廷敬為工部尚書,馬齊為山西巡撫。己酉,鄂羅斯察漢汗使來請解雅克薩之圍。許之。是月,內大臣拉篤祜奉詔與羅卜藏濟農及噶爾丹定地而還。

冬十月丙辰,調張士甄為禮部尚書,以胡升猷為刑部尚書。

十一月庚子,上謁孝陵。賞蒙古喀喇沁兵征浙江、福建有功者。

十二月癸丑,上還宮。丙辰,命侍郎薩海督察鳳凰城屯田。癸亥,諭:「糾儀御史糾察必以嚴,設朕躬不敬,亦當舉奏。」戊寅,祫祭太廟。

是歲,免直隸、江南、浙江、湖廣、甘肅等省二十七州縣被災額賦有差。朝鮮、安南、荷蘭、吐魯番入貢。

二十六年丁卯春正月戊子,遣醫官往治雅克薩軍士疾,羅剎原就醫者並醫之。丙申,蒙古土謝圖汗、車臣汗及濟農合疏請上尊號。不許。乙巳,大學士吳正治乞休。允之。

二月癸丑,上大閱於盧溝橋。原任湖廣總督蔡毓榮隱藏吳三桂孫女為妾,匿取逆財,減死鞭一百,枷號三月,籍沒,並其子發黑龍江。原讞尚書禧佛等坐隱庇,黜革有差。甲寅,以餘國柱為大學士。庚申,命八旗都統、副都統更番入值紫禁城。丁卯,以張玉書為刑部尚書。壬申,戶部奏滸墅關監督桑額溢徵銀二萬一千餘兩。得旨:「設立榷關,原為稽察奸宄。桑額多收額銀,乃私封便民橋,以致擾害商民。著嚴加議處。嗣後司榷官有額外橫徵者,該部其嚴飭之。」

三月己丑,以董訥為江南江西總督。癸巳,以王鴻緒為左都御史。癸卯,上御太和門視朝,諭大學士等詳議政務闕失,僉以無弊可陳對。上曰:「堯、舜之世,府修事和,然且兢兢業業,不敢謂已治已安。漢文帝亦古之賢主,賈誼猶指陳得失,直言切諫。今但云主聖臣賢,政治無闕,豈國家果無一事可言耶?大小臣工,各宜盡心職業,視國事如家事,有所見聞,入陳無隱。」以馬世濟為貴州巡撫。

夏四月己未,上諭大學士曰:「纂修明史諸臣,曾參看前明實錄否?若不參看實錄,虛實何由悉知。明史成日,應將實錄並存,令後世有所考證。」丙寅,以田雯為江蘇巡撫。癸酉,罷科道侍班。

五月己亥,宗人府奏平郡王納爾都打死無罪屬人,折傷手足,請革爵圈禁。得旨:「革爵,免圈禁。」庚辰,詔曰:「今茲仲夏,久旱多風,陰陽不調,災孰大焉。用是減膳撤樂,齋居默禱。雖降甘霖,尚未霑足。皆朕之涼德,不能上格天心。政令有不便於民者更之。罪非常赦不原者咸赦除之。」戊子,上召陳廷敬、湯斌十二人各試以文。諭曰:「朕閒與熊賜履講論經史,有疑必問。繼而張英、陳廷敬以次進講,大有裨益。德格勒每好評論時人學問,朕心以為不然,故茲召試,茲判然矣。」壬辰,上制周公、孔子、孟子廟碑文,御書勒石。

六月丁酉,上素服步行,祈雨於天壇。是夜,雨。辛丑,改祀北海於混同江。以楊素蘊為安徽巡撫。

秋七月戊子,鄂羅斯遣使議和,命薩布素退兵。丙午,戶部請裁京員公費。得旨勿裁。

八月己酉,上巡幸塞外。癸丑,次博洛和屯行圍。甲戌,賜外籓銀幣。

九月己卯,上還京。辛巳,於成龍進嘉禾。上曰:「今夏乾旱,幸而得雨,未足為瑞也。」壬午,以李之芳為大學士。乙未,調湯斌為工部尚書。起徐元文為左都御史。

冬十月癸丑,上巡幸畿甸。甲子,上還駐申昜春園。

十一月甲申,以李正宗為漢軍都統。丙申,太皇太后不豫。上詣慈寧宮侍疾。

十二月乙巳朔,上為太皇太后不豫,親制祝文,步行禱於天壇。癸亥,以王永譽為漢軍都統。乙丑,湖廣巡撫張為御史陳紫芝劾其貪婪,侍郎色楞額初按不實。至是,命於成龍、馬齊、開音布馳往提拏,究擬論死,陳紫芝內升。己巳,太皇太后崩。上哭踴視襝,割辮服衰,居慈寧宮廬次。甲戌除夕,群臣請上還宮。不允。

是歲,免直隸、山東、山西、江西等省四州縣災賦有差。朝鮮入貢。

二十七年戊辰春正月戊子,上居乾清門外左幕次。乙未釋服。丁酉聽政。

二月壬子,大學士勒德洪、明珠、餘國柱有罪免,李之芳罷御史,郭琇具疏論列也。尚書科爾昆、佛倫、熊一瀟俱罷。甲寅,以梁清標、伊桑阿為大學士,李天馥為工部尚書,張玉書為兵部尚書,徐乾學為刑部尚書。定宗室襲封年例。

三月乙亥,以馬齊為左都御史。辛巳,上召廷臣及董訥、靳輔、於成龍、佛倫、熊一瀟等議河務。次日亦如之。乙酉,色楞額以按張獄欺罔論死,總督徐國相以徇庇,侍郎王遵訓等以濫舉,俱免官。己丑,以王新命為河道總督。辛卯,裁湖廣總督。丁酉,論河工在事互訐諸臣,董訥、熊一瀟、靳輔、慕天顏、孫在豐俱削官,並趙吉士、陳潢罪之。己亥,增遣督捕理事官張鵬翮、兵科給事中陳世安,會內大臣索額圖與鄂羅斯議約定界。壬寅,賜沈廷文等一百四十六人進士及第出身有差。李光地坐妄舉德格勒議處。得旨:「李光地前於臺灣一役有功,仍以學士用。」

夏四月癸卯朔,日有食之。戊申,以傅拉塔為江南江西總督。己酉,上躬送太皇太后梓宮奉安暫安奉殿。其後起陵,是曰昭西陵。回蹕至薊州除發。甲寅,以厄魯特侵喀爾喀,使諭噶爾丹。戊辰,上還宮。庚午,命侍郎成其範、徐廷璽查閱河工。

五月己卯,吏部尚書陳廷敬、刑部尚書徐乾學以疾罷。甲午,以紀爾他布為兵部尚書。丙申,上謁祭暫安奉殿。

六月甲辰,湖廣督標裁兵夏逢龍作亂,踞武昌,巡撫柯永升投井死,署布政使糧道葉映榴罵賊遇害。命瓦岱佩振武將軍印討之。庚申,阿喇尼奏噶爾丹侵厄爾德尼招,哲卜尊丹巴、土謝圖汗遁。發兵防邊。戊辰,起熊賜履為禮部尚書,徐元文為左都御史。以翁叔元為工部尚書。

秋七月癸酉,以輔國公化善為蒙古都統。乙酉,湖廣提督徐治都大敗夏逢龍於應城,於鯉魚套焚賊舟,賊遁黃岡。丙戌,上巡幸塞外。戊子,南陽總兵史孔華復漢陽。庚寅,瓦岱復黃州,獲夏逢龍,磔誅之,賊平。壬午,雲南提督萬正色侵冒兵饟,按律論死。上念其前陷賊時抗志不屈,行間血戰勞績甚多,免死,革提督,仍留世職。壬辰,上駐喀爾必哈哈達,有峰舊名納哈里,高百數十丈,上發數矢皆過峰頂,賜今名。

八月癸卯,上駐巴顏溝行圍。葉映榴遺疏至,贈工部侍郎,下部優恤。乙卯,張玉書奏查閱河工,多用靳輔舊議。

九月壬申,遣彭春、諾敏率師駐歸化城防邊。是時喀爾喀為噶爾丹攻破,徙近邊內。遣阿喇尼往宣諭之,並運米賑撫。辛卯,上還京。癸巳,復設湖廣總督,以丁思孔為之。

冬十月癸卯,移楊素蘊為湖廣巡撫。庚戌,以輔國公綽克託為奉天將軍。乙卯,上大行太皇太后尊謚曰孝莊文皇后。辛酉,升祔太廟,頒詔中外。

十一月辛卯,荊州將軍噶爾漢等坐討賊逗遛奪職,鞭一百,官吏從賊受官者逮治,餘貸之。

十二月庚子,以希福為蒙古都統。甲辰,建福陵、昭陵聖德神功碑,禦制碑文。上謁孝莊山陵。乙巳,以尼雅翰為西安將軍。己酉,進張玉書為禮部尚書;徐元文刑部尚書,再進戶部尚書。丙寅,上還京。兵部、工部會疏福建前造砲船核減工料銀二萬餘兩,應著落故總督姚啟聖名下追賠。上以姚啟聖經營平臺甚有功績,毋庸著追。

是歲,免江南、江西、湖廣、雲南、貴州等省三十三州縣災賦有差。朝鮮、琉球入貢。

二十八年己巳春正月庚午,詔南巡臨閱河工。丙子啟鑾。詔所過勿令民治道。獻縣民獻嘉禾。壬午,詔免山東地丁額賦。甲申,上駐濟南。乙酉,望祀泰山。庚寅,次剡城,閱中河。壬辰,次清河。癸巳,詔免江南積欠二十餘萬。乙未,上駐揚州。詔曰:「朕觀風問俗,鹵薄不設,扈從僅三百人。頃駐揚州,民間結糸採盈衢,雖出自愛敬之誠,不無少損物力。其前途經過郡邑,宜悉停止。」

二月辛丑,上駐蘇州。丁未,駐杭州。詔廣學額,賚軍士,復因公降謫官,賜扈從王大臣以次銀幣,賜駐防耆民金。辛亥,渡錢塘江,至會稽山麓。壬子,祭禹陵,親制祭文,書名,行九叩禮,制頌刊石,書額曰「地平天成」。癸丑,上還駐杭州。閱騎射,賜將軍以及官兵大酺。丁巳,次蘇州。故湖廣糧道葉映榴之子敷迎鑾,為其父請謚。上書「忠節」二大字賜之。松江百姓建碑祈壽,獻進碑文。江南百姓■B6留停蹕,獻土物為御食,委積岸上。令取米一撮,果一枚,為留一日。浙江巡撫金鋐有罪,削職遣戍。以張鵬翮為浙江巡撫。增設武昌、荊州、常德、岳州水師。癸亥,上駐蹕江寧。甲子,祭明陵。賜江寧、京口駐防高年男婦白金。乙丑,上閱射,賜酺。上詣觀星臺,與學士李光地咨論星象,參宿在觜宿之先,恆星隨天而動,老人星合見江南,非隱見也。江寧士民■B6留聖駕。為留二日。

三月戊辰朔,發江寧。甲戌,閱高家堰,指授治河方略。丙戌,上還京。聞安親王岳樂之喪,先臨其第哭之,乃還宮。丁亥,命八旗科舉先試騎射。戊子,詔靳輔治河勞績昭然,可復原官。丁酉,增設八旗火器營,副都統領之。

閏三月壬子,予安親王岳樂祭葬立碑,謚曰和。己未,上謁陵。丙午,謁孝莊皇后山陵,謁孝陵。辛酉,上還京。

夏四月乙亥朔,上制孔子贊序及顏、曾、思、孟四贊,頒於學宮。壬辰,復命索額圖等赴尼布楚,與鄂羅斯定邊界。喀爾喀外蒙古內附告饑。命內大臣伯費揚古往賑撫之。命臺灣鑄錢。

五月乙巳,以阿蘭泰、徐元文為大學士,顧八代為禮部尚書,郭琇為左都御史。壬戌,頒行孝經衍義。癸亥,命歸化城屯兵備邊。

六月乙亥,以佟寶為寧古塔將軍。兩廣總督吳興祚以鼓鑄不實黜官。

秋七月,以石琳為兩廣總督。癸卯,冊立貴妃佟氏為皇后。甲辰,皇后崩,謚曰孝懿。

八月癸酉,上巡幸邊外。戊寅,駐博洛和屯,賜居民銀米。

九月癸卯,上還京。戊午,以倭赫為蒙古都統,額駙穆赫為漢軍都統。

冬十月丙寅,以郎談為滿洲都統。辛未,增設喀爾喀兩翼扎薩克,招集流亡,編置旗隊。癸酉,左都御史郭琇以致書本省巡撫請託降官。甲戌,葬孝懿皇后,上臨送。是月,岷州生番內附。

十一月丙申,上還宮。辛酉,孝懿皇后祔奉先殿。

十二月乙丑,詔免雲南二十一年至二十三年民欠。丙寅,上朝皇太后於慈寧新宮。戊辰,以張英為工部尚書。乙亥,內大臣索額圖疏報與鄂羅斯立約,定尼布楚為界,立碑界上,以五體文書碑。

是歲,免直隸、浙江、湖北等省十一州縣災賦有差。朝鮮入貢。

二十九年庚午春正月癸丑,上幸南苑。庚申,遣官賑蒙古喀爾喀。

二月甲子,以岳樂子馬爾渾嗣封安郡王。乙丑,遣大臣巡視直隸災區流民。五城粥廠寬期,倍發銀米,增置處所。己巳,上謁孝莊山陵,謁孝陵。庚午,大雨。癸酉,上還京。甲戌,上御經筵。戊子,起陳廷敬為左都御史。

三月壬辰朔,除長蘆新增鹽課。乙未,詔修三朝國史。癸卯,命都統額赫納、護軍統領馬賴、前鋒統領碩鼐率師征厄魯特。先是,噶爾丹兵侵喀爾喀,迭詔諭解不從,兵近邊塞。至是,命額赫納等蒞邊御之。辛亥,除云南黑井加增鹽課。以張思恭為京口將軍。

夏四月丁丑,以旱赦殊死以下系囚。甲申,建子思子廟於闕里。大清會典成。

五月辛卯朔,命九卿保舉行取州縣堪為科道者。

六月癸酉,大學士徐元文免。戊寅,噶爾丹追喀爾喀侵入邊。命內大臣蘇爾達赴科爾沁徵蒙古師備御。命康親王傑書、恪慎郡王岳希師駐歸化城。

秋七月庚寅朔,以張英為禮部尚書,以董元卿為京口將軍。辛卯,噶爾丹入犯烏珠穆秦。命裕親王福全為撫遠大將軍,皇子胤禔副之,出古北口。恭親王常寧為安遠大將軍,簡親王喇布、信郡王鄂扎副之,出喜峰口。內大臣佟國綱、索額圖、明珠、彭春等俱參軍事,阿密達、阿拉尼、阿南達俱會軍前。己亥,以陳廷敬為工部尚書,於成龍為左都御史。癸卯,上親征,發京師。己酉,上駐博洛和屯,有疾回鑾。

八月乙未朔,日有食之。撫遠大將軍裕親王福全大敗噶爾丹於烏闌布通,噶爾丹以喇嘛濟隆來請和,福全未即進師。上切責之。乙丑,上還京。丙子,噶爾丹以誓書來獻。上曰:「此虜未足信也。其整師待之。」

九月癸巳,先是,烏闌布通之戰,內大臣公佟國綱戰歿於陣。至是,喪還,命皇子率大臣迎之。凡陣亡官咸賜奠賜恤有差。戊申,停今年秋決。壬子,弛民間養馬之禁。

冬十月己未,上疾少愈,召大學士諸臣至乾清宮輪對。乙亥,以鄂倫岱為漢軍都統。辛巳,領翰林院學士張英失察編修楊瑄撰擬佟國綱祭文失當,削禮部尚書,楊瑄褫官戍邊入旗。

十一月己亥,以熊賜履為禮部尚書。甲辰,達賴喇嘛請上尊號。不許,並卻其貢。己酉,裕親王福全等至京聽勘。王大臣議上。上薄其罪,輕罰之。將士仍敘功。

十二月丁丑,上謁陵,行孝莊文皇后三年致祭禮。庚辰還京。

是歲,免直隸、江南、浙江、甘肅等省三十二州縣衛災賦有差。朝鮮入貢。

三十年辛未春正月戊申,封阿祿科爾沁貝勒楚依為郡王,以與厄魯特力戰受傷被執不屈而脫歸也。其十二旗陣亡臺吉俱贈一等臺吉,賜號達爾漢,子孫承襲。噶爾丹復掠喀爾喀。命瓦岱為定北將軍,駐張家口,郎談為安北將軍,駐大同,川陜總督會西安將軍駐兵寧夏備之。命在籍勇略將軍趙良棟參軍事。乙卯,以馬齊為兵部尚書。

二月丁巳朔,日有食之。乙丑,上御經筵。命步軍統領領巡捕三營,兼轄五城督捕。戊午,厄魯特策旺阿拉布坦使來,噶爾丹之侄也,厚賚其使,比旋,遣郎中桑額護其行。

三月戊子,繙譯通鑒綱目成,上制序文。己酉,賜戴有祺等一百四十八人進士及第出身有差。

夏四月戊午,左都御史徐乾學致私書於山東巡撫錢鈺,事發,並褫職。丁卯,上以喀爾喀內附,躬蒞邊外撫綏。是日,啟鑾。

五月丙戌,上駐多羅諾爾。喀爾喀來朝。先是,喀爾喀土謝圖汗聽哲卜尊丹巴唆,殺其同族扎薩克圖汗得克得黑墨爾根阿海,內亂迭興,為厄魯特所乘。至是,遣大臣按其事。土謝圖汗、哲卜尊丹巴具疏請罪。上赦之。以扎薩克圖汗,七旗之長,飭其弟策旺扎布襲汗號,封為親王。丁亥,上御行幄,土謝圖汗、哲卜尊丹巴入覲,俯伏請罪。大臣宣赦,泣涕謝恩。賜茶賜宴賜坐,大合樂,九叩首而退。戊子,復召土謝圖汗、哲卜尊丹巴、策旺扎布、車臣汗及喀爾喀諸部濟農、偉徵、諾顏、阿玉錫諸大臺吉三十五人賜宴。諭曰:「朕欲熟識爾等,故復饗宴。」賜之冠服。策旺扎布年幼,以皇子衣帽數珠賜之。以車臣汗之叔扎薩克濟農納穆扎爾前勸車臣汗領十萬眾歸順,身為之倡,請照四十九旗一例,殊為可嘉,許照舊扎薩克,去其濟農之號,封為郡王。餘各封爵有差。傳諭喀爾喀曰:「爾等困窮至極,互相偷奪,朕已拯救愛養。今與四十九旗一體編設各處扎薩克,管轄稽察,其各遵守。如再妄行,則國法治之矣。」己丑,上御甲胄乘馬,遍閱各部。下馬親射,十矢九中。次大閱滿洲兵、漢軍兵、古北口兵,列陣鳴角,鳥槍齊發,聲動山谷。眾喀爾喀環矚駭嘆曰:「真神威也!」科爾沁喀爾喀各蒙古王貝勒請上尊號。不許。庚寅,上按閱喀爾喀營寨,賚牛羊及其窮困者。辛卯,遣官往編喀爾喀佐領,予之游牧。烏珠穆秦臺吉車根等以降附厄魯特,按實罪之。壬辰,上回鑾。癸卯,還京。辛亥,分會試中卷南左、南右、北左、北右、中左、中右,從御史江蘩之言也。壬子,群臣請上尊號。不許。

六月乙卯,以李天馥為吏部尚書,陳廷敬為刑部尚書,高爾位為工部尚書。

秋七月甲申,西安將軍尼雅翰奉詔督兵遷巴圖爾額爾克濟農於察哈爾,濟農憚行遁去,尼雅翰追之不及,按問論死。命總督葛思泰追討之。朝鮮使人以買一統志發其國論罪。致仕大學士杜立德卒,予祭葬,謚文端。

閏七月丙辰,葛思泰疏報濟農之弟博濟在昌寧湖,經總兵柯彩派兵剿敗,生擒博濟及前禁之格隆等,均斬之。乙亥,上巡幸邊外。

九月辛酉,上回鑾,道遵化,謁孝莊山陵,謁孝陵。乙丑,還京。庚午,以公阿靈阿為蒙古都統。甲戌,命侍郎博濟、李光地、徐廷璽偕靳輔視河。

冬十月庚寅,謝爾素番盜殺參將硃震,西寧總兵官李芳述擒盜首華木爾加誅之。癸巳,以巴德渾為滿洲都統,杭奕祿為荊州將軍。丁未,甘肅提督孫思克討阿奇羅卜藏,斬之。先是,使於厄魯特之侍讀學士達虎還及嘉峪關,為阿奇羅卜藏所害,命思克討之。至是,捷聞。

十一月丁巳,以索諾和、李振裕為工部尚書,以伊勒慎為滿洲都統。己未,詔曰:「朕崇尚德教,蠲滌煩苛,大小諸臣,咸被恩禮。即因事罷退,仍令曲全鄉里。近來交爭私怨,糾結不已,頗有黨同伐異之習,豈欲釀明季門戶之禍耶?其各蠲私忿,共矢公忠。有怙終者,朕必窮治之。」是時徐元文、徐乾學、王鴻緒既罷,而傅臘塔等抉摘瑣隱,金句連興獄,故特詔儆飭焉。甲戌,詔曰:「欽天監奏來歲正月朔日食。天象示儆,朕甚懼焉。其罷元日筵宴諸禮。諸臣宜精白供職,助朕修省。」

十二月甲申,詔曰:「朕撫馭區宇,惟以愛養蒼生,俾臻安阜為念。比歲地丁額賦,迭經蠲免,而歲運漕米,尚在輸將,時切軫念。除河南已經蠲免外,其湖廣、江蘇、浙江、安徽、山東漕米,以次各免一年,用紓民力。」丁亥,移旗莊壯丁赴古北口外達爾河墾田。遣侍郎阿山、德珠等往陜西監賑。壬辰,諭督、撫、提、鎮保舉武職堪任用及曾立功者,在內八旗旗員,令都統等舉之。

是歲,免直隸、江南、江西、河南、山東、陜西、湖廣、雲南等省一百八十八州縣災賦有差。朝鮮、安南、琉球入貢。

三十一年壬申春正月辛亥朔,日有食之,免朝賀。甲寅,上禦乾清門,出示太極圖、五音八聲八風圖,因言:「律呂新書徑一圍三之法,用之不合。徑一尺圍當三尺一寸四分一釐,積至百丈,所差至十四丈外矣。寧可用邪?惟隔八相生之說,試之悉合。」又論河道閘口流水,晝夜多寡,可以數計。又出示測日晷表,畫示正午日影至處,驗之不差。諸臣皆服。庚午,上幸南苑行圍。

二月辛巳,以靳輔為河道總督。乙酉,以陜西旱災,發山西帑銀、襄陽米石賑之。丁亥,上巡幸畿甸。辛卯,陜西巡撫薩弼以賑災不實褫職。戊戌,上還京。己亥,上御經筵。乙巳,以馬齊為戶部尚書。

三月丙辰,遣內大臣阿爾迪、理籓院尚書班迪赴邊外設立蒙古驛站。乙丑,命府丞徐廷璽協理河工。加甘肅提督孫思克太子少保,予世職。致仕大學士馮溥卒,予祭葬,謚文敏。以阿席坦為滿洲都統。置雲南永北鎮。

夏四月庚辰朔,以希福為滿洲都統,護巴為蒙古都統。己丑,發帑銀百萬賑陜西,尚書王騭、沙穆哈往視加賑。戊戌,上幸瀛臺,召近臣觀稻田及種竹。河道總督靳輔請建新莊、仲家淺各一閘,下部議行。

五月庚寅,諭戶部,山西平陽豐收,可遣官購買備荒。命王維珍董其事。癸卯,定喀爾喀部為三路,土謝圖為北路,車臣為東路,扎薩克圖為西路,屬部各從其分地畫為左右翼。

六月庚辰,以宋犖為江寧巡撫。乙未,蒙古科爾沁進獻錫伯、卦爾察、打虎爾一萬餘戶,給銀酬之。

秋七月乙亥,上巡幸塞外。

八月己丑,以翁叔元為刑部尚書,以博濟為西安將軍,李林隆為固原提督,李芳述為貴州提督。

九月戊申,噶爾丹屬人執我使臣馬迪戕之。庚戌,上還次湯泉。己未,還京。丁卯,上御經筵。壬申,上大閱於玉泉山。

冬十月己卯,詔曰:「秦省比歲兇荒,加以疾疫,多方賑濟,未甦積困。所有明年地丁稅糧,悉予蠲免。從前逋欠,一概豁除。用稱朕子惠元元至意。」庚辰,以李天馥為大學士。壬午,上謁陵。曲赦陜西,非十惡及軍前獲譴者,皆免死減一等。以佛倫為川陜總督,宗室董額為滿洲都統。庚寅,上還京。癸巳,以熊賜履為吏部尚書,張英為禮部尚書。庚子,停直省進鮮茶暨賚送表箋。

十一月庚戌,以阿靈阿為滿洲都統。甲寅,命熊賜履勘察淮、揚濱河涸田。丙寅,加孫思克振武將軍。以覺羅席特庫為蒙古都統。

十二月壬午,河道總督靳輔卒,予祭葬,謚文襄。以於成龍為河道總督,董訥為左都御史。壬辰,以郎化麟為漢軍都統。辛丑,以西安饑,運襄陽米平糶。加希福建威將軍,移戍右衛。召科爾沁蒙古王沙津入京,面授機宜,使誘噶爾丹。

是歲,免陜西、江南、四川等省十三州縣災賦有差。朝鮮入貢。

三十二年癸酉春正月甲子,詔朝鮮歲貢黃金木棉永行停止。

二月乙亥朔,發帑金,招商販米西安平市價。丙子,遣內大臣坡爾盆等往督歸化城三路屯田。詔修南河周橋堤工,往年靳輔與陳潢所經度者,至是閱河大臣繪圖進呈,特詔修之。策旺阿拉布坦遣使入貢,報告使臣馬迪被害及噶爾丹密事,以糸採緞賚之。癸未,上御經筵。改宣府六十衛為一府八縣。戊子,命郎談為昭武將軍,偕阿南達、碩鼐帥師赴寧夏,將軍博濟、孫思克參軍事。庚寅,上巡幸畿甸,閱霸州苑家口堤工,諭巡撫郭世隆修之。庚子,上還京。貴州巡撫衛既齊疏報剿辦土司失實,奪職戍黑龍江。

三月丙午,遣皇子胤禔祭華山。丁未,移饒州府駐景德鎮。乙卯,置廣東運司、潮州運同。庚午,詔趙良棟系舊臣,可暫領寧夏總兵。

夏四月丙戌,喀爾喀臺吉車凌扎布自鄂羅斯來歸,賚之袍服,賜克魯倫游牧。癸巳,命檢直省解送物料共九十九項,減去四十項免解。丁酉,以心裕為蒙古都統。

五月庚戌,命內大臣伯費揚古為安北將軍,駐歸化城。

六月乙亥,廣八旗鄉、會中額。

八月甲戌,免廣西、四川、貴州、雲南四省明年地丁稅糧。癸未,上巡幸塞外行圍。蒙古科爾沁諸部朝行在,賜冠服銀幣。

九月丁未,修盛京城。丙寅,琉球來貢,遣其質子還國。丁卯,上還京。

冬十月壬申,詔曰:「給事中彭鵬奏劾順天考官,請朕親訊,是大臣皆不可信矣。治天下當崇大體,若朕事事躬親,則庶務何由畢理乎?」壬辰,上大閱於玉泉山。丁酉,鄂羅斯察漢汗來貢。上諭大學士曰:「外籓朝貢,固屬盛事,傳至後世,未必不因而生事。惟中國安寧,則外患不生,當培養元氣為根本耳。」

十一月辛丑,上奉皇太后謁孝莊山陵、孝陵。庚申,還宮。甲子,詔免順天、河間、保定、永平四府明年稅糧。

十二月辛未,以宗室公楊岱為蒙古都統。丁亥,上幸南苑行圍。諭:「滿洲官兵近來不及從前之精銳,故比年親加校閱,間以行圍。頃見諸士卒行列整齊,進退嫻熟,該軍校等賞給一個月錢糧,該管官賞給緞疋,以激戎行。」丁酉,祫祭太廟。

是歲,免直隸、江南、江西、浙江、山西、湖廣等省六十九州縣災賦有差。朝鮮、琉球入貢。

三十三年甲戌春正月乙卯,盛京歉收,命馬齊馳往,以倉穀支給兵丁,海運山東倉穀濟民食。丙辰,召見河道總督於成龍,問曰:「爾前言減水壩不宜開,靳輔糜費錢糧,今竟何如?」成龍曰:「臣前誠妄言。今所辦皆照靳輔而行。」上曰:「然則爾所言之非,靳輔所行之是,何以不明白陳奏,尚留待排陷耶?」因諭大學士曰:「於成龍前奏靳輔未曾種柳河堤,朕南巡時,指河干之柳問之,無辭以對。又奏靳輔放水淹民田,朕復至其地觀之,斷不至淹害麥田。而王騭、董訥等亦附和於成龍言之。」下部議,將於成龍革職枷責。上曰:「伊經手之工未完,應革職留任。」王騭休致,董訥革職。

二月辛未,上御經筵。癸酉,大學士請間三四日一御門聽政。上曰:「昨諭六十以上大臣間日奏事,乃優禮老臣耳。若朕躬豈敢暇逸,其每日聽政如常。」丁丑,以諾穆圖為漢軍都統。庚辰,上巡幸畿甸。敕修通州至西沽兩岸堤工。

三月辛丑,上還京。禮部尚書沙穆哈以議皇太子祀奉先殿儀注不敬免官。辛酉,賜胡任輿等一百六十八人進士及第出身有差。以範承勛為左都御史。

夏四月庚午,理籓院奏編審外籓蒙古四十九旗人丁二十二萬六千七百有奇。辛巳,以查木揚為杭州將軍。

五月戊寅,步軍統領凱音布奏天壇新修之路,勿令行人來往。上曰:「修路以為民也。若不許行,修之何益。後若毀壞,令步兵隨時葺治。」順天學政李光地丁母憂,令在京守制。甲辰,命翰林院、詹事府、國子監日輪四員入直南書房。辛亥,以紀爾他布為滿洲都統,噶爾瑪為蒙古都統。甲寅,詔修類函。丁巳,上巡幸畿甸,閱視河堤,諭扈從衛士魚貫而行,勿踐田禾。戊午,上閱龍潭口。己未,閱化家口、黃須口、八百戶口、王家甫口、筐兒港口、白駒場口,薄弱之處,咸令增修。庚申,閱桃花口、永安口、李家口、信艾口、柳灘口等處新堤。上曰:「觀新堤甚屬堅固,百姓可免數年水患矣。」壬戌,上還京。

閏五月庚午,上試翰林出身官於豐澤園。

六月辛丑,加湖廣提督徐治都鎮平將軍。丙辰,以範承勛為江南江西總督。

秋七月丁卯,以蔣弘道為左都御史,轉王士禛戶部左侍郎,王掞戶部右侍郎。巴圖爾額爾克濟農奏報降人祁齊克逃遁,遣兵追斬之。丁亥,上求文學之臣。大學士舉徐乾學、王鴻緒、高士奇及韓菼、唐孫華以對。上曰:「韓菼非謫降之人,當以原官召補。徐乾學、王鴻緒、高士奇可起用修書。並召徐秉義來。」他日試唐孫華詩佳,授禮部主事、翰林院行走。己丑,江南江西總督傅拉塔卒,贈太子太保,予祭葬,謚清端。庚寅,上巡幸塞外。

八月己未,上駐蹕拜巴哈昂阿。喀爾喀哲布尊丹巴來朝,賜之冠服。

九月己巳,廣八旗入學學額。己卯,上還京。壬午,以石文炳為漢軍都統,以王繼文為雲南貴州總督。

冬十月丙申,以吳赫為四川陜西總督。乙巳,以金世榮為福州將軍。

十一月丁卯,溫僖貴妃鈕祜祿氏薨。癸酉,以張旺為江南提督。戊寅,起陳廷敬為戶部尚書。

十二月庚戌,以覺羅席特庫為滿洲都統,杜思噶爾為蒙古都統。

是歲,免直隸、山東等省十二州縣災賦有差。朝鮮入貢。

三十四年乙亥春正月丁亥,以護巴為滿洲都統。

二月己亥,以郭世隆為浙江福建總督。丁巳,太和殿工成。休致大學士李之芳卒,予祭葬,謚文襄。

三月丙戌,以石文英為漢軍都統。

夏四月丁酉,平陽府地震。甲辰,遣使冊立班禪胡土克圖。己酉,追敘趙良棟平蜀、滇功,授一等子世職。其部將升賞有差。己未,以李輝祖為河南巡撫。

五月壬寅朔,遣尚書馬齊察賑地震災民。巡撫噶世圖以玩災免。辛未,命在京八旗分地各造屋二千間住兵。壬申,上巡幸畿甸,閱新堤及海口運道,建海神廟。戊子,還京。

六月丁酉,策封皇太子胤礽妃石氏。庚子,以久雨詔廷臣陳得失,禮部祈晴。庚申,漕運總督王樑奏參衛千總楊奉漕船裝帶貨物。諭曰:「商人裝帶貨物,於運何妨。王樑乃將貨物搜出棄置兩岸,所行甚暴,即解任。」

秋七月己丑,以覺羅舒恕為寧夏將軍,鄂羅順為江寧將軍。趙良棟告赴江南就醫,命給與南巡舊船。

八月壬辰,上巡幸塞外。辛丑,博濟奏報噶爾丹屬下回子五百人闌入三岔河汛界,肅州總兵官潘育龍盡俘之,拘於肅州。丙午,次克勒和洛。命宗室公蘇努、都統阿席坦、護巴領兵備噶爾丹。己酉,次克勒烏理雅蘇臺。調董安國為河道總督,桑額為漕運總督。

九月辛巳,上還京。癸未,詔順天、保定、河間、永平四府水潦傷稼,免明年地丁錢糧,仍運米四萬石前往平糶。

冬十月丁未,命內大臣索額圖、明珠視察噶爾丹。

十一月己未朔,日有食之。壬戌,命大軍分三路備噶爾丹,裹八十日糧,其駝馬米糧,令侍郎陳汝器、前左都御史於成龍分督之。丙寅,停今年秋決。庚午,命李天馥復為大學士。庚辰,上大閱於南苑。戊子,命安北將軍伯費揚古為撫遠大將軍。遣大臣如蒙古徵師,示師期。

十二月己亥,命將軍博濟、孫思克師出鎮彞。乙巳,平陽地震,命蠲本年糧額,並免山西、陜西、江南、浙江、江西、湖廣、廣東、福建等省逋賦,赦殊死以下,其政令有不便於民者,令督撫以聞。以齊世為滿洲都統。

是歲,免直隸、山西、江西、福建、廣東等省十二州縣災賦有差。朝鮮、琉球入貢。

三十五年丙子春正月甲午,下詔親征噶爾丹。賚隨征大臣軍校宴。甲申,命公彭春參贊西路軍務。

二月丁亥朔,上謁陵。辛卯,上還京。壬辰,以碩鼐為蒙古都統。癸丑,告祭郊廟社稷。甲寅,命皇太子胤礽留守。丙辰,上親統六師啟行。

三月戊辰,上出行宮觀射。辛未,次滾諾,大雨雪,上露立,俟軍士結營畢,乃入行幄。軍中畢炊,乃進膳。以行帳糧薪留待後至者。庚辰,予故巡撫王維珍祭葬,謚敏愨。

夏四月辛卯,上次格德爾庫。壬辰,上駐塔爾奇拉。諭:「茲已抵邊界,自明日始,均列環營。」前哨報噶爾丹在克魯倫,命蒙古兵先進據河。

五月丙辰朔,上駐蹕拖陵布拉克。辛酉,次枯庫車爾。壬戌,偵知噶爾丹所在,上率前鋒先發,諸軍張兩翼而進。至燕圖庫列圖駐營。其地素乏水,至是山泉湧出,上親臨視。癸亥,次克魯倫河。上顧大臣曰:「噶爾丹不知據河拒戰,是無能為矣。」前哨中書阿必達探報噶爾丹不信六師猝至,登孟納爾山,望見黃幄網城,大兵雲屯,漫無涯際,大驚曰:「何來之易耶!」棄其廬帳宵遁。驗其馬矢,似遁二日矣。上率輕騎追之。沿途什物、駝馬、婦孺委棄甚眾。上顧謂科爾沁王沙津曰:「虜何倉皇至是?」沙津曰:「為逃生耳。」喀爾喀王納木扎爾曰:「臣等當日逃難,即是如此。」上上書皇太后,備陳軍況,並約期回京。追至拖納阿林而還,令內大臣馬思喀追之。戊辰,上班師。是日晨,五色雲見。癸酉,次中拖陵。撫遠大將軍伯費揚古大敗噶爾丹於昭莫多,斬級三千,陣斬其妻阿奴。噶爾丹以數騎遁。癸未,次察罕諾爾。召見蒙古諸王,獎以修道鑿井監牧之勞,各賜其人白金。

六月癸巳,上還京。是役也,中路上自將,走噶爾丹,西路費揚古大敗噶爾丹,唯東路薩布素以道遠後期無功。甲午,論喀爾喀郡王善巴盡以馬匹借軍功,晉封親王,貝子盆楚克偵敵有勞,封為郡王。諸臣行慶賀禮。乙未,賜察哈爾護軍月饟加一金,喀爾喀人六金,限給三年。詔停本年秋審。壬子,以吳琠為左都御史,調張旺為福建水師提督,張雲翼為江南提督。

秋七月戊午,以平定朔漠勒石太學。以李輝祖為湖廣總督。癸亥,廣直省鄉試解額。戊辰,改吳英福建陸路提督,岳升龍為四川提督。

八月丁酉,索諾和以乏軍需免,以凱音布為兵部尚書。

九月甲寅朔,回回國王阿卜都裏什克奏:「臣仗天威,得以出降。遣臣回國葉爾欽,請敕策旺阿拉布坦勿加虐害。」乙卯,賜厄魯特降人官秩衣糧。壬申,上巡幸塞外。丙子,次沙城。詔:「年來宣化所屬牧養軍馬,供億甚繁,深勞民力,其悉蠲明年額賦。」丁丑,副都統祖良璧敗噶爾丹部人丹濟拉於翁金。

冬十月甲申朔,遣官齎賜西路軍士衣裘牛羊。丁亥,次昭哈。賜右衛、大同陣亡軍士白金。庚寅,大將軍費揚古獻俘至。賜銀贖出,令其完聚。戊申,上臨視右衛軍士,賜食。傳諭曰:「昭莫多之役,爾等乏糧步行而能禦敵,故特賜食。悉免所借庫銀。其傷病之人,另頒賜之。」眾叩首歡謝。庚戌,上駐蹕麗蘇。上皇太后書,謝賜裘服。

十一月戊寅,噶爾丹遣使乞降,其使格壘沽英至,蓋微探上旨也。上告之曰:「俟爾七十日,過此即進兵矣。」庚辰,回鑾。

十二月壬寅,上還京。以宗室費揚固為右衛,祁布為滿洲都統,雷繼尊為漢軍都統。庚戌,詔:「陜、甘沿邊州縣衛所,當師行孔道,供億繁多,閭閻勞苦,其明年地丁銀米悉行蠲免。」

是歲,免江南、江西等省三十二州縣災賦有差。朝鮮入貢。

三十六年丁丑春正月丙辰,上幸南苑行圍。戊辰,哈密回部擒噶爾丹之子塞卜騰巴爾珠爾來獻。己巳,遣官存問勇略將軍趙良棟,賜人葠鹿尾。甲戌,諭:「朕觀明史,一代並無女後預政,以臣陵君之事。我朝事例,因之者多。朕不似前人輒譏亡國也。現修明史,其以此諭增入敕書。」

二月丁亥,上親征噶爾丹,啟鑾。是日,次昌平。阿必達奏哈密擒獲厄魯特人土克齊哈什哈,系害使臣馬迪之首犯。命誅之,子女付馬迪之家為奴。戊戌,上駐大同。丁未,次李家溝。戊申,詔免師行所過岢嵐、保德、河曲等州縣今年額賦。是日,次輦鄢村,山泉下湧,人馬霑足。庚戌,遣官祭黃河之神。

三月丙辰,上駐蹕屈野河。厄魯特人多爾濟、達拉什等先後來降。賜哈密回王金幣冠服。丁巳,趙良棟卒,上聞之,嗟悼良久,語近臣曰:「趙良棟,偉男子也。」辛酉,次榆林。戊辰,次安邊城。寧夏總兵王化行請上獵於花馬池。上曰:「何如休養馬力以獵噶爾丹乎?」辛未,次花馬池。丙子,上自橫城渡河。遣皇長子胤禔賜奠趙良棟及前提督陳福。丁丑,上駐蹕寧夏。察恤昭莫多、翁金陣亡弁兵。己卯,祭賀蘭山。庚辰,上閱兵。命侍衛以御用食物均賜戰士。

閏三月辛巳朔,日有食之。庚寅,康親王傑書薨。寧夏百姓聞上將行,懇留數日。上曰:「邊地磽瘠,多留一日,即多一日之擾。爾等誠意,已知之矣。」

夏四月辛亥,上次狼居胥山。甲寅,回鑾。庚申,命直省選文行兼優之士為拔貢生,送國子監。甲子,費揚古疏報閏三月十三日噶爾丹仰藥死,其女鍾齊海率三百戶來降。上率百官行拜天禮。敕諸路班師。是日,大雨。厄魯特降人請慶賀。止之。先是,上將探視寧夏黃河,由橫城乘舟行,至湖灘河朔,登陸步行,率侍衛行獵,打魚射水鴨為糧,至包頭鎮會車騎。

五月乙未,上還京。丁酉,以傅拉塔為刑部尚書,席爾達左都御史,翁叔元罷,以吳琠為刑部尚書,張鵬翮左都御史。癸卯,禮部請上尊號。不許。

六月甲寅,禮部請於師行所過名山磨崖紀功。從之。予故勇略將軍一等子趙良棟祭葬,謚襄忠。

秋七月癸未,群臣請上皇太后徽號,三上,不允。乙未,以朔漠平定,遣官祭告郊廟、陵寢、先師。賜李蟠等一百五十人進士及第出身有差。晉封大將軍伯費揚古一等公,參贊以下各授世職。辛丑,免旗兵借帑。乙巳,遣官賚外籓四十九旗兵。丁未,上巡幸塞外。

八月乙亥,上駐巴圖舍裏,賜蒙古王、公、臺吉銀幣。

九月癸未,厄魯特丹濟拉來歸。上獨御氈幄召見之。丹濟拉出語人曰:「我罪人也,上乃不疑,真神人也。」甲午,上還京。庚子,以都統凱音布兼步軍統領。壬寅,上御經筵。乙巳,振平將軍、湖廣提督徐治都卒,贈太子少保,予祭葬,謚襄毅。賑黑龍江被水居民。以席爾達為兵部尚書,哈雅爾為左都御史。

冬十月己巳,始令宗室應鄉、會試。壬戌,詔曰:「比年師行出入,皆經山西地方,有行齎居送之勞。其免山西明年額賦。」敘從征鎮國公蘇努功,晉封貝子。庚午,上謁陵。甲戌,內監劉進朝以訛詐人論死。

十一月辛巳,上還京。丙戌,和碩恪靖公主下嫁喀爾喀郡王敦多布多爾濟。戊戌,朝鮮告糶,命運米三萬石往賑。甲辰,詔直省報災即察實以聞。

十二月丁卯,改宗室董額為滿洲都統。乙亥,祫祭太廟。

是歲,免直隸、江南、安徽、江西等省五十九州縣災賦有差。朝鮮、琉球、安南入貢。

三十七年戊寅春正月庚寅,策旺阿拉布坦奏陳第巴匿達賴喇嘛圓寂之事,斥班禪而自尊,懇請睿鑒。上答之曰:「朕曾敕責第巴具奏認罪,若怙終不悛,朕不輕恕也。」並遣侍讀學士伊道等齎敕往。癸卯,上巡幸五臺山。甲辰,次涿州。命皇長子胤禔、大學士伊桑阿祭金太祖、世宗陵。

二月辛亥,詔免山西三十六年逋賦。癸丑,上駐蹕菩薩頂。乙丑,遣官賑山東。戊辰,上還京。

三月丙子朔,上御經筵。丁丑,封皇長子胤禔為直郡王,皇三子胤祉為誠郡王,皇四子胤禛、皇五子胤祺、皇七子胤祐、皇八子胤禩俱為貝勒。戊子,禁造燒酒。辛卯,直隸巡撫於成龍奏偕西洋人安多等履勘渾河,幫修挑濬,繪圖呈進。得旨:「於六月內完工。」

夏四月癸亥,減廣東海關稅額。己巳,詔溫郡王延壽行止不端,降為貝勒,貝子袁端削爵。壬申,以貝子蘇努管盛京將軍。癸酉,上閱漕河。

五月甲戌,武清民請築外堤。上曰:「築外堤恐損民田。」民曰:「河決之害,更甚於損田。」上曰:「水潦將降,暫立木椿護堤,開小河洩水,俟明春雨水前為爾等成之。」癸未,上還京。壬寅,裁上林苑。以李林盛為陜西提督,張旺為廣西提督。是月,策旺阿拉布坦上言與哈薩克構兵,及將丹津鄂木布拘禁各緣由。命示議政大臣。

六月辛亥,移吳英為福建水師提督。丁巳,改四川梁萬營為化林營,設參將以下官。己未,雲南巡撫石文晟奏三籓屬人奉旨免緝者,準其墾田應試。從之。

秋七月癸酉朔,張玉書丁母憂,以吳琠為大學士,王士禛為左都御史。辛卯,命吏部月選同、通、州、縣官引見。癸巳,霸州新河成,賜名永定河,建河神廟。己亥,以盧崇耀為廣州將軍,殷化行為廣東提督。庚子,以蘇爾發為滿洲都統。辛丑,上奉皇太后東巡,取道塞外。

八月癸丑,上奉皇太后臨幸喀拉沁端靜公主第,賜金幣及其額駙噶爾臧。甲子,皇太后望祭父母於發庫山。己巳,賜端敏公主及其額駙達爾漢親王班第金幣。湖南山賊黃明犯靖州,陳丹書犯茶陵州,官兵討平之。

九月壬申,上次克爾蘇,臨科爾沁故親王滿珠習禮墓前酹酒,孝莊皇后之父也。癸巳,上駐扎星阿。賜黑龍江將軍薩布素等金幣冠服。庚子,停盛京、烏拉本年決囚。

冬十月癸卯,上行圍,射殪二虎,其一虎,隔澗射之,穿其★。丁未,上行圍,槍殪二熊。是日,駐蹕輝發。己酉,裁雲南永寧府,置永北府。癸巳,上駐蹕興京。甲寅,上謁永陵。遣官賜奠武功郡王禮敦墓。改貴州水西土司,置大定、平遠、黔西三流官。丁巳,上謁福陵、昭陵,臨奠武勛王揚古利、直義公費英東、弘毅公額亦都墓。免奉天今年米豆。壬戌,上奉皇太后回鑾。

十一月癸未,上奉皇太后還宮。丙戌,詔曰:「朕巡幸所經,敖漢、奈曼、阿祿科爾沁、扎魯特諸蒙部水草甚佳,而生計窘迫,蓋因牲畜被盜,不敢夜牧耳。朕即遣郎中李學聖等往為料理,盜竊衰止。其他處蒙古亦宜照此差遣。旗員有原往蒙古教導者,準其前往。命盜各案,同聽決之。」庚寅,以張鵬翮為江南江西總督。

十二月辛丑朔,命徐廷璽協理河務,命尚書馬齊,侍郎喻成龍、常綬察視河工。庚戌,諭宗人府:「閒散宗室,材力幹濟,精於騎射,及貧無生計者,各察實以聞。」詔官民妻女緣事牽連,勿拘訊,著為令。改四川東川土司為東川府,設知府以下官。戊午,詔八旗察訪孝子節婦。己未,以巴錫為雲南貴州總督,馬自德為京口將軍。己巳,祫祭太廟。

是歲,免直隸、江南、福建、浙江、湖廣等省三十五州縣災賦有差。朝鮮入貢。

三十八年己卯春正月辛卯,詔:「朕將南巡察閱河工,一切供億,由京備辦。預飭官吏,勿累閭閻。」

二月壬寅,詹事尹泰以不職解任。癸卯,上奉皇太后南巡啟鑾。戊申,以天津總兵潘育龍訓練有方,賜御服貂裘。

三月庚午,上次清口,奉皇太后渡河。辛未,上御小舟,臨閱高家堰、歸仁堤、爛泥淺等工。截漕糧十萬石,發高郵、寶應等十二州縣平糶。壬申,上閱黃河堤。丙子,車駕駐揚州。諭隨從兵士勿踐麥禾。壬午,詔免山東、河南逋賦,曲赦死罪以下。癸未,車駕次蘇州。辛卯,車駕駐杭州。丙申,上閱兵較射。戊戌,上奉皇太后回鑾。

夏四月庚子朔,回次蘇州。詔免鹽課、關稅加增銀兩,特廣江、浙二省學額。乙巳,以丹岱為杭州將軍。己酉,車駕次江寧。上閱兵。庚申,次揚州。辛酉,以彭鵬為廣西巡撫。丙寅,渡黃河,上乘小舟閱新埽。

五月辛未,次仲家閘,書「聖門之哲」額,懸先賢子路祠。乙酉,上奉皇太后還宮。丁亥,以馬爾漢為左都御史,王鴻緒為工部尚書。

六月戊戌朔,起郭琇為湖廣總督,以鎮國公英奇為蒙古都統。

秋七月甲申,河決淮、揚。

閏七月戊戌,敏妃張佳氏薨。誠郡王胤祉其所出也,不及百日薙發,降貝勒。癸丑,先是,苗賊黃明屢報獲報死,仍報犯事。至是,遣官按鞫,並其夥陳丹書、吳思先等三十餘人誅之。其奏報不實之督撫麻勒吉等降黜有差。上巡幸塞外。

九月丙午,上還京。丙辰,上御經筵。改揚岱為滿洲都統,魯伯赫、拖倫、崇古禮俱為蒙古都統。戊午,大學士阿蘭泰卒,上悼惜之,遣皇長子胤禔視疾,賜奠加祭,謚文清。

冬十月癸酉,上巡視永定河工。庚辰,上還宮。大學士李天馥卒,予祭葬,謚文定。

十一月乙巳,上謁陵。壬辰,以馬齊、佛倫、熊賜履、張英為大學士,陳廷敬為吏部尚書,李振裕為戶部尚書,杜臻為禮部尚書,馬爾漢、範承勛為兵部尚書,王士禛為刑部尚書。壬寅,命滿、漢給事中各四員侍班。丙午,令寶源局收買廢錢。

十二月戊辰,上還京。癸巳,祫祭太廟。

是歲,免直隸、江南、江西、浙江、福建、陜西、湖廣等省七十三州縣災賦有差。朝鮮、琉球入貢。

三十九年庚辰春正月己未,朝鮮國王李焞以遣回難民進方物,上還之。癸亥,上閱永定河工。

二月甲戌,上乘舟閱郎城、柳岔諸水道,水淺,易艇而前,指示修河方略。壬午,還京。己丑,命內大臣費揚古、伊桑阿考試宗室子弟騎射。

三月甲午,上御經筵。吏部奏安徽巡撫李★被參一案,請交將軍、提督查按。上曰:「將軍、提督不與民事,部議不合。」嚴飭之。尚書庫勒納旋罷。癸卯,改張鵬翮為河道總督。鵬翮請撤協理官及效力員,部臣寬文法,以責成功。從之。甲寅,以宗室特克新為蒙古都統。丙辰,賜汪繹等三百一人進士及第出身有差。四川巡撫於養志、提督岳升龍互訐,遣官按鞫,俱削職。

夏四月庚辰,上閱永定河。命八旗兵丁協助開河,以直郡王胤禔領之,僖郡王岳希等五人偕往。壬午,上閱子牙河。壬辰,還京。

五月丁未,以阿山為江南江西總督。甲寅,以阿靈阿為蒙古都統。

六月癸亥,張鵬翮報修浚海口工成,河流申昜遂,改攔黃壩為大通口,建海神廟。杜臻罷,以王澤弘為禮部尚書,李柟為左都御史。丁亥,停宗室科舉。

秋七月甲午,理籓院議覆喇嘛商南多爾濟所奏策旺阿拉布坦遣兵往青海一事,毋庸議。上曰:「此事目前甚小,將來關系甚大。該部擬以勿庸議,倘青海問商南多爾濟,何以答之?策旺阿拉布坦為人狡猾,素行奸惡,鄰近諸部,俱與仇讎。其稱往征第巴,道遠險多,或虛張聲勢以恫哧青海,未可知也。要使不敢構釁為是。」乙巳,定翰林官編、檢、庶吉士月給銀三兩例,學道缺出,較俸派出。壬子,故振武將軍孫思克卒,命皇長子胤禔奠酒,賜鞍馬二匹,銀一千兩,謚襄武。丁巳,上巡幸塞外。命李光地、張鵬翮、郭琇、彭鵬詳議科場事宜。

八月辛未,上次齊老圖。

九月癸巳,停今年秋決。詔張鵬翮專理河工,範成勛等九人撤回。給事中穆和倫請禁服用奢侈,閣臣票擬申飭。上曰:「言官耳目之職,若因言而罪之,誰復言者。惟其言奢侈在康熙十年後則非,乃在輔臣時耳,今少息矣。」

冬十月辛酉,皇太后六旬萬壽節,上制萬壽無疆賦,親書圍屏進獻。癸酉,上巡閱永定河。戊寅,上還京。己卯,命本年行取科道未補官者,作為額外御史,隨班議事。

十一月庚寅,命青海鄂爾布圖哈灘巴圖爾移駐寧夏。詔侍郎溫達查視陜、甘驛站。王澤弘免,以韓菼為禮部尚書。命大臣及清要官子弟應試者,編為官號,限額取中。辛亥,上巡幸邊外。命卓異官如行取例引見。戊午,四川打箭爐土蠻作亂,遣侍郎滿丕偕提督唐希順討之。

十二月己未朔,上駐蹕暖泉,賜外籓王以下至官兵白金。戊辰,上還京。癸酉,移蕭永藻為廣西巡撫,彭鵬為廣東巡撫。壬午,故安親王岳樂坐前審擬貝勒諾尼一案失入,追降郡王,子僖郡王岳希、貝子吳爾占俱降鎮國公。丁亥,祫祭太廟。

是歲,免直隸、江南、安徽、陜西、浙江等省五十七州縣災賦有差。朝鮮入貢。

四十年辛巳春正月辛亥,以河伯效靈,封金龍四大王。甲寅,以心裕為滿洲都統。

二月己未朔,上巡閱永定河。諭李光地曰:「河水涸必致淤塞,此甚難治,當徐議之。」乙丑,滿丕、唐希順討打箭爐土蠻平之,蠻民萬二千戶內附。庚辰,上還宮。

三月戊子,上御經筵。丁酉,張鵬翮請以治河方略纂集成書。上斥之曰:「朕於河務之書,罔不披閱,大約坐言則易,實行則難。河性無定,豈可執一法以繩之。編輯成書,非但後人難以仿行,即揆之己心,亦難自信。張鵬翮試編輯之!」給事中馬士芳劾湖北布政使任風厚年老。調來引見,年尚未衰。上因諭曰:「坐而辦事,必得老成練達者,方能得當,州縣官則不可耳。」

夏四月己未,調李林盛為甘肅提督,擢潘育龍為固原提督,移藍理為天津總兵官,以曹秉桓為漢軍都統。丙子,刑部尚書王士禛請假回籍。上諭大學士曰:「山東人性多偏執,好勝尋仇,惟王士禛無之。其詩甚佳,居家惟讀書。若令回籍,殊為可惜。給假五月,不必開缺。」丁丑,上閱永定河。諭李光地:「隆冬冰結,可照常開洩。清水流於冰下,為冰所逼,沖刷河底愈深。」閱大灣口,諭:「石堤尚未興工,可以南來杉木排椿,爾等勿忽。」閱子牙河。乙酉,上還京。

五月癸巳,黑龍江管水手官員缺,部臣擬補遣戍道員周昌。上曰:「周昌既遣戍矣,又補官烏拉,是終身不得歸也。可令八旗官原效力者為之。」戊申,御史張瑗請毀前明內監魏忠賢墓。從之。丙辰,上巡幸塞外。

六月庚辰,授宋儒邵雍後裔五經博士。

秋七月丁亥,領侍衛內大臣公費揚古隨扈患病,上為停鑾一日,親往視疾。隨以不起聞,賜鞍馬三匹,散馬四匹,銀五千兩,遣大臣護送還京,予祭葬,謚襄壯。

八月乙丑,上幸索嶽爾濟山。詔曰:「此山形勢崇隆,允稱名勝。嗣後此處禁斷行圍。」甲申,上次馬尼圖行圍,一矢穿兩黃羊,並斷拉哈裏木,蒙古皆驚。

九月辛丑,簡親王雅布隨扈薨,命大臣送還京,皇長子胤禔、皇三子胤祉出迎,遣官治喪,賜銀四千兩,皇子合助銀三千兩。發引時,皇子侍往送,予祭葬立碑,謚曰修。乙巳,上還京。庚戌,上御經筵。大學士王熙以衰疾乞休,溫旨衛慰諭,加少傅致仕。噶爾丹之女鍾齊海到京,命與其兄一等侍衛色卜騰巴爾珠爾同居,配二等侍衛蒙古沙克都爾。

冬十月戊午,以宗室特克新為滿洲都統,迓圖布爾塞為蒙古都統。己未,召大學士張玉書還朝。詔免甘肅來年額賦。庚申,以梁鼐為福建陸路提督。辛酉,免江蘇明年額賦。起岳升龍為四川提督。辛未,改普奇為滿洲都統,孫渣齊為蒙古都統,以華顯為四川陜西總督。癸酉,大學士張英乞休,溫旨慰諭令致仕。御史靳讓疏言為州縣者,須令家給人足,方為良吏。命改靳讓通州知州。詔總督郭琇、張鵬翮、桑額、華顯,巡撫李光地、彭鵬、徐潮各舉賢能。平悼郡王訥爾福薨,子訥爾素襲爵。

十一月甲午,詔:「朕詳閱秋審重案,字句多誤。廷臣竟未察出一二,刑部尤為不慎,其議罰之。」

十二月壬申,廣東連山瑤匪作亂,命都統嵩祝討之。辛巳,祫祭太廟。

是歲,免直隸、江南、河南、陜西、廣東等省四十二州縣災賦有差。朝鮮、琉球入貢。


\end{pinyinscope}