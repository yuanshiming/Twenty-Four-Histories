\article{列傳一百三十一}

\begin{pinyinscope}
勒保額勒登保胡時顯德楞泰

勒保,字宜軒,費莫氏,滿洲鑲紅旗人,大學士溫福子。由中書科筆帖式充軍機章京。乾隆三十四年,出為歸化城理事同知。坐事當褫職,高宗以溫福方征金川,特原之。授兵部主事,仍直軍機處。累遷郎中,出為江西贛南道,調安徽廬鳳道。以母憂去官,命為庫倫辦事章京。四十五年,充辦事大臣。累擢兵部侍郎,仍留庫倫。五十年,內召。未幾,授山西巡撫。五十二年,署陜甘總督,尋實授。五十六年,大軍征廓爾喀,治西路駝馬、裝糧、臺站,加太子太保。

初,安徽奸民劉松以習混元教戍甘肅,復倡白蓮教,與其黨湖北樊學明、齊林,陜西韓龍,四川謝添繡等謀不軌。五十九年,勒保捕劉松誅之,而松黨劉之協、宋之清傳教於河南、安徽。以鹿邑王氏子曰發生者,詭明裔硃姓,煽動愚民,事覺被捕。詔誅首惡,赦餘黨,發生以童幼免死,戍新疆。之協遠颺不獲,各省大索,官吏奉行不善,頗為民擾。武昌府同知常丹葵在荊州、宜昌株連數千人,川、楚民方以苗事困軍興,無賴者又因禁私鹽、私鑄失業,益仇官,亂機四伏矣。

六十年,勒保調雲貴總督。湖南、貴州苗疆不靖,福康安督師進討,勒保赴軍,安撫正大、銅仁、鎮遠降苗,並治軍需。雲南威遠惈匪擾邊,勒保將赴剿,會惈匪即平,福康安、和琳相繼卒於軍,命偕明亮、鄂輝接辦軍務,未至,而湖北教匪熾,蔓延川、陜。林之華、覃加耀踞長陽黃柏山,福寧攻之不克,勒保往會剿,嘉慶二年春,連戰敗之。方乘勝薄其巢,而貴州南籠仲苗王囊仙等叛,詔勒保督師討之。王囊仙者,洞灑寨苗婦,當丈寨韋七綹須,以囊仙有幻術,推為首。分遣其黨大王公、李阿六、王抱羊圍南籠府,及府屬之永豐、黃草壩、捧鮓、新城、冊亨,安順府屬之永寧、歸化諸城。冊亨陷,滇、黔道梗。三月,勒保至,令總兵德英額、札郎阿、袁敏分守東、西、北三路。其南際滇、粵,咨兩廣總督吉慶、雲南巡撫江蘭防之;自率按察使常明、副將施縉,進克關嶺。抵永寧,副將巴圖什裡已解其圍,都司周廷翰援歸化,圍亦解。會提督珠隆阿擊永豐,自率總兵張玉龍、七格,解新城圍,進至南籠,圍始解。詔嘉南籠固守,賜名興義。遣常明、施縉解黃草壩圍。賊悉眾圍捧鮓、永豐益急,分兵援之,先解捧鮓圍,自率常明、施縉攻洞灑、當丈賊巢。賊縱火自焚,都司王宏信、千總洪保玉冒烈焰入,擒王囊仙、韋七綹須,旋解永豐圍。吉慶亦自廣西至,復冊亨。六月,仲苗平,詔改永豐曰貞豐,錫封勒保一等侯爵,號曰威勤。

九月,調湖廣總督。時川、楚賊氛愈熾,立青、黃、藍、白、線等號,又設掌櫃、元帥、先鋒、總兵等偽稱。先命永保總統諸軍,易以惠齡,又易以宜綿,皆不辦;至是宜綿薦勒保以自代,允之。三年正月,至四川梁山,賊曾柳起石壩山,而白號王三槐、青號徐天德、藍號林亮工諸賊聚開縣。勒保先破石壩山,斬曾柳,詔嘉為入川第一功。調授四川總督。三槐走達州,與藍號冉文儔合,惟亮工仍在開縣之開州坪,勒保令副都統六十七、總兵富森布剿之;親追三槐,九戰皆捷。賊走巴州,掠閬中、蒼溪而西,追之急,復東入儀隴。勒保以賊蹤靡定,所至裹脅,乃畫堅壁清野策,令民依山險扎寨屯糧,團練鄉勇自衛。賊由儀隴趨孫家梁,欲與白號羅其清合。偕惠齡、恆瑞截剿,三槐南竄渠縣,文儔遁入其清寨。勒保留惠齡、恆瑞剿孫家梁,仍親躡三槐。五月,三槐犯大竹,分竄梁山、墊江、新寧,東奔開縣,亮工出為犄角,擊走之,斬其黨林定相。天德來援,敗之,擒其黨張洪鈞,天德奔新寧。三槐與冷天祿踞雲陽安樂坪,進圍之。七月,誘三槐降,擒之,械送京師,詔晉封公爵。

天祿盡有三槐之眾,負嵎抗拒,圍攻久不下;黃號龍紹周、龔建、樊人傑來援,擊卻之。十月,天祿糧盡,詭請降,夜突營,大為所挫,尋走新寧。四年正月,天德為額勒登保所敗,亦竄新寧仁市鋪,與黃號王光祖合。偕額勒登保夾擊,天德走墊江,天祿走忠州。勒保令額勒登保截擊天德,總兵百祥追天祿,自率大軍策應。仁宗以前此諸軍事權不一,特授勒保經略大臣,節制川、楚、陜、甘、豫五省軍務,明亮、額勒登保為參贊。勒保以賊勢重在四川,請暫駐梁山、大竹等處督師。尋破天德,天祿分竄鄰水、長壽,復敗之,天祿為額勒登保所殲。二月,移駐達州。疏言扎寨團練,行之四川有效,請通行於湖北、陜西、河南;又言安民即以散賊,請各省被賊之區,蠲免今歲應徵錢糧:並如議行。四月,追剿天德、紹周、建、人傑及張子聰等,賊遁開縣東鄉。旋分竄竹峪關、渡口場,意圖入陜。五月,子聰勾合藍號冉天元北竄,遣額勒登保兜擊,逼回川境。子聰竄通江,藍號包正洪竄雲陽,青號王登廷竄東鄉,天德、紹周、建、人傑及線號龔文玉,白號張天倫竄大寧老林,勒保檄調諸軍分剿。六月,總兵硃射鬥殲正洪於雲陽;七月,德楞泰擒文玉於大寧;八月,提督七十五擒建、人傑於開縣:賊勢浸衰矣。

會治餉大臣福寧劾勒保月餉十二萬兩,視他路為多,所辦賊有增無減;而天德復由大寧闌入湖北境,總督倭什布飛章告警。詔褫職,命尚書魁倫赴川勘問,以額勒登保代為經略。勒保能得軍心,而八旗兵素驕,稍裁抑之,遂騰蜚語,及就逮,所部將士為之訟冤。魁倫窺帝怒不測,未以上聞,稍為申辨糜餉縱賊罪,卒坐以明亮、恆瑞不聽調度;副都統訥音兵譁閧,不據實參奏;又賊犯楚境不即馳報,玩視軍務,論大闢。帝念前功,改為斬監候,解部監禁。

五年春,額勒登保等剿賊陜西,魁倫專任川事,而將士不用命。天元、子聰合黃號徐萬富、青號汪瀛、線號陳得俸,渡嘉陵江,魁倫退守潼河,事聞,起勒保赴川。三月至,賊已越潼河,赴中江截剿,連敗之,詔逮魁倫,授勒保四川提督,兼署總督。時德楞泰已大破賊於馬蹄岡,冉天元、陳得俸、雷世旺先後殄滅;合剿汪瀛於嘉陵江口,擒之。四月,擊敗高天升、馬學禮,賊遁甘肅番境,五月,復犯龍安,罷提督,專任總督。六月,賊北走甘肅,遣副都統阿哈保追之,自率兵剿川東、川北諸賊。七月,與德楞泰合擊白號茍文明、鮮大川於岳池新場,敗之,大川走死,實授總督。

八月,白號賊與青號趙麻花合,進擊,殲其黨湯思舉。麻花復合王珊向陜境,欲迎天德入川。勒保截之於江口,斃麻花,珊亦為德楞泰所誅。十二月,藍號李彬、白號楊開第、黃號齊國謨自巴州竄儀隴,德楞泰擊斃國謨,勒保亦斬開第,獨彬遁走。六年正月,移師川東,敗藍號楊步青於大寧,而樊人傑、徐萬富合藍號王士虎、冉天士擾廣元、蒼溪。遣阿哈保往援,賊偽向儀隴,陰沿嘉陵江南下,欲潛渡;馳至南部與阿哈保合擊,殲萬富。二月,藍號張士龍竄巴州,遣七十五擊斬之;自擊藍號陳朝觀、白號魏學盛,敗之巫山、雲陽間。賊北竄入陜、楚界,追至竹山。六月,賊回竄東鄉,擊敗之,擒青號何子魁,殲藍號茍文明、鮮俸先。七月,又擒徐天壽、王登高。八月,白號高見奇合魏學盛竄廣元,邀擊之,追至通江。適藍號冉學勝自老林至與合,乘夜攻之,擒學勝。詔封三等男。九月,見奇、學盛分竄南江及陜西西鄉。勒保抵南江,聞李彬方掠巴州、蒼溪,恐逾嘉陵江,亟往,賊已東竄通江;乃移兵大竹,剿湯思蛟、劉朝選,追至太平,擒其黨蕭焜。

是冬,偕額勒登保、德楞泰疏言:「剿匪大局已定,請酌撤官兵。」詔以「巨賊未盡除,遽思將就了事」,嚴斥之。七年正月,復疏言:「川省自築寨練團,賊勢十去其九。擬分段駐兵,率團協力搜捕餘匪;遣熟諳軍事之道、府,正、佐各員,分專責成。兵力所不到,民力助之;民力所不支,兵力助之:庶賊無所匿。」詔如議行。是月,擒青號何贊於忠州。二月,李彬竄南江,為建昌道劉清所擒。三月,張天倫、魏學盛擾川北,遣總兵田朝貴往剿,不利;親率羅思舉等繼進,大敗賊於巴州,天倫、學盛並就殲。五月,遣羅聲皋、達斯呼勒岱剿擒白號庹向瑤;總兵張績剿青號,擒徐天培;田朝貴剿藍號,殲楊步青。七月,劉朝選糾青、藍、黃號殘匪竄大寧,勒保遣將擊之,羅思舉擒朝選,達斯呼勒岱殲賴飛隴,詔晉一等男。十月,羅思舉擒張簡,而湯思蛟敗竄亦就獲。十一月,思舉擒黃號唐明萬。時川中著名逆首率就擒殲;餘匪竄老林,不復成股。在陜、楚者亦多為額勒登保、德楞泰所殲。十二月,合疏馳奏蕆功,晉封一等伯爵,仍以「威勤」為號。

八年,搜捕餘匪,擒白號茍文富、宋國品、張順,青號王青,招降黃號王國賢,偕額勒登保、德楞泰會奏肅清。未幾,陜西南山餘孽復起,至九年八月始平。十年,入覲,詔曰:「自嘉慶四年,勒保在川省令鄉民分結寨落,匪始無由焚劫,且助官軍擊賊。其後陜、楚仿行,賊勢乃促。今三省閭閻安堵,實得力此策為多。加太子太保、雙眼花翎,回鎮四川,與民休息。」時解散鄉勇,令入伍為兵。

十一年秋,陜西寧陜鎮新兵倡亂,遣總兵唐文淑往援剿,叛將蒲大芳縛首逆乞降,德楞泰受之。勒保奏劾:「叛兵罪重於逆匪,率以納降。不知畏威,安能悔罪?他兵從而生心,益驕難制。」帝韙其言,命赴陜西會治善後事宜。尋聞四川綏定新兵亦叛,桂涵捕擒首逆,磔之,餘黨並論如律。十三年,涼山夷匪擾馬邊,剿平之。十四年,拜武英殿大學士,仍留總督任。

十五年,召來京供職。坐在四川隱匿名揭帖未奏,降授工部尚書,調刑部。十六年,出為兩江總督。尋內召,復授武英殿大學士,管理吏部,改兵部,授領侍衛內大臣。十八年,充軍機大臣,兼管理籓院。十九年,以病乞休,食威勤伯全俸。二十四年,卒,詔贈一等侯,謚文襄。

勒保短小精悍,多智數。知其父金川之役以剛愎敗,一反所為,寄心膂於諸將帥,優禮寮屬,俾各盡其長,卒成大功。晚入閣,益斂鋒芒,結同朝之歡,而內分涇、渭。既罷相,帝眷注不衰,命皇四子瑞親王娶其女,以恩禮終。

子九,長英惠,科布多參贊大臣,襲三等威勤侯,卒;孫文厚,嗣爵。第四子英綬,工部侍郎;孫文俊,江西巡撫。

額勒登保,字珠軒,瓜爾佳氏,滿洲正黃旗人。世為吉林珠戶,隸打牲總管。乾隆中,以馬甲從征緬甸大小金川,累擢三等侍衛,賜號和隆阿巴圖魯,乾清門行走。四十九年,剿甘肅石峰堡回匪。五十二年,平臺灣。疊遷御前侍衛。五十六年,從福康安征廓爾喀,攝駐藏大臣。攻克擦木賊寨,七戰七勝,抵帕朗古河,班師殿後,加副都統銜。論臺灣、廓爾喀功,兩次圖形紫光閣。尋授副都統兼護軍統領,擢都統。

六十年,貴州松桃苗石柳鄧、湖南永綏苗石三保相繼叛,陷乾州。福康安視師,請額勒登保偕護軍統領德楞泰率巴圖魯侍衛赴軍。至則松桃圍已解,石柳鄧逸入石三保黃瓜寨中。額勒登保由松桃進攻,解永綏圍,克黃瓜寨。攻賊首吳半生於蘇麻寨,克西梁;半生遁高多寨,擒之:授內大臣。又獲乾州賊目吳八月,餘黨據平隴,進抵長吉山,敗之。嘉慶元年,福康安卒,和琳代。時石三保就擒,石柳鄧在平隴,乃進兵復乾州,賜花翎,署領侍衛內大臣。秋,和琳卒於軍,統兵者惟額勒登保、德楞泰及湖南巡撫姜晟三人。詔將軍明亮、提督鄂輝往會剿。十月,克平隴,石柳鄧遁踞養牛塘山梁,分兵克之。十二月,斬石柳鄧,苗縛吳八月子廷義以獻。軍事告竣,詔嘉其功最,錫封威勇侯,賜雙眼花翎。

二年,移師剿湖北教匪。時林之華、覃加耀踞長陽黃柏山,地險糧足,總督福寧攻之久不下。三月,額勒登保至,克四方臺。賊遁鶴峰芭葉山,其險隘曰大拏口,六月克之。賊竄宣恩、建始,分兵三路進,十月,斃之華於大茅田,而加耀遁施南山中,尋竄長樂硃里寨,三面懸崖,惟東南一徑。十二月,遣死士縋登,掘地窖火藥轟之,賊爭走,墜崖,坑穀皆滿。惟加耀偕賊二百遁,踞歸州終報寨。詔斥額勒登保縱賊,降三等伯爵。三年春,加耀始就擒,仍以蕆事緩,奪爵職、花翎,予副都統銜,命赴陜西協剿襄匪高均德、姚之富、齊王氏等。會李全自盩厔至藍田,欲與諸賊合,擊走之。姚之富、齊王氏失援,遂為明亮、德楞泰所殲。進剿均德於兩岔河,賊分竄商州、鎮安。四月,赴荊州會剿張漢潮,敗之竹山,躡追,由陜西入四川。九月,擊漢潮於廣元,擒其子正漋。與德楞泰等合剿川匪羅其清。其清踞營山之箕山,已為德楞泰所破,竄大鵬寨。額勒登保與德楞泰、惠齡、恆瑞四路進攻,十月合圍。其清突走青觀山,樹柵距險。額勒登保鑒於黃柏山、芭葉山頓兵之失,議主急攻,親逼柵前,席地坐,令楊遇春督兵囊土立營,且戰且築,諸軍繼之,攻擊七晝夜。賊不支,竄渡巴河,踞遂風寨廢堡。德楞泰同至,圍之數重,勢垂克,薄暮,忽傳令撤圍。賊傾巢夜潰,遲至黎明始馳追,賊四路逃竄,至方山坪已散盡,獲其清於石穴,逸匪數日內並為民兵擒獻。是役,賊趨絕地,無外援,開網縱之,饑疲就縛,士卒不損,竟全功焉,復花翎。十二月,追徐天德、冷天祿於合州。

四年春,詔以勒保為經略大臣,額勒登保與明亮同授副都統為參贊。三月,追冷天祿於大竹,聞蕭占國、張長庚由閬州竄營山,回軍迎擊。賊踞黃土坪,臨江負山,令總兵硃射鬥繞出雞猴寨,截其西;自率楊遇春由東襲攻城隍廟,賊西走,為射鬥所扼,夾擊,殲其半,越山竄走尚數千。乘夜圍擊於譚家山,隕崖死及生擒幾盡,斬占國、長庚。有冒難民逃出者,投冷天祿,述兵威,天祿曰:「我曾於安樂坪破經略兵數萬,何懼此乎?」時踞岳池,距大軍不遠,天祿遣大隊先行,自率悍黨八百殿後。額勒登保冒雨由間道進至廣安,令穆克登布據石頭堰以待,楊遇春潛出賊後;自將索倫勁騎沖之,賊死鬥,天祿斃於箭。次日,迫其大隊於石筍河,斬溺過半,先渡者追殲之。旬日間連殄三劇賊,疊詔嘉賚,先封二等男爵,晉一等。四月,追剿白號張子聰於雲陽,子聰糾合黃號樊人傑、線號蕭焜、卜三聘等,疊敗之寒水壩,賊稍散。五月,子聰復合冉天元窺陜境,扼御之。子聰竄通江,追敗之於茍家坪,又敗天元於木老壩。七月,天元竄鎮龍關,欲與王登廷合,登廷屯馬鞍寨,擊走之。窮追至大竹、東鄉,援賊麕至,分兵進擊,擒斬甚眾,仍躡登廷。

額勒登保戰績為諸軍最,湖北道員胡齊侖治餉餽送諸將,事發,獨無所受,詔嘉其「忠勇公清,為東三省人傑」。八月,勒保以罪逮,命代為經略,授領侍衛內大臣,補都統。疏陳軍事曰:「臣前數年止領一路偏師,今任經略,當籌全局。教匪本屬編氓,宜招撫以散其眾,然必能剿而後可撫,必能堵而後可剿。從前湖北教匪多,脅從少;四川教匪少,脅從多。今楚賊盡逼入川,其與川東巫山、大寧接壤者,有界嶺可扼,是湖北重在堵而不在剿。川、陜交界,自廣元至太平,千餘里隨處可通,陜攻急則入川,川攻急則入陜,是漢江南北剿堵並重。川東、川北有嘉陵江以限其西南,餘皆崇山峻嶺,居民近皆扼險築寨,團練守御;而川北形勢更便於川東,若能驅各路之賊偪川北,必可聚而殲旃:是四川重在剿而不在堵。但使所至堡寨羅布,兵隨其後,遇賊迎截夾擊,以堵為剿,事半功倍,此則三省所同。臣已行知陜、楚,曉諭修築,並定賞格,以期兵民同心蹙賊。至從征官兵,日行百十里,旬月尚可耐勞,若閱四五年之久,騾馬尚且踣斃,何況於人?續調新募者,不習勞苦,更不如舊兵。臣一軍尚能得力者,以兵士所到之處,亦臣所到之處;兵士不得食息,臣亦不得食息。自將弁以及士卒,無不一心一力,而各路不能盡然。近日不得已,將臣兵與各提鎮互相更調,以期人人精銳。」又言:「軍中出力人員,應隨時鼓勵,令各路領兵大員,自行保奏,以免咨送遲延。」帝並韙之。

時徐天德敗於湖北,折回川東,漸衰弱;而王登廷與冉天元、茍文明合阮正漋竄廣元,賊勢重在川北。九月,率楊遇春殲正漋於雲霧山。十一月,登廷、天德、天元及樊人傑會合抗拒,疊戰於巴州何家院、東君壩,擒賊目賈正舉、王國安,追至蒼溪貓兒埡。額勒登保以天元善戰,令楊遇春、穆克登布合左右翼力擊。穆克登布輕進,為天元所乘,傷亡甚眾;賊萃攻經略中營,血戰竟夜,賊始退,次日,登廷在南江為鄉團所擒。額勒登保以實聞,詔嘉其不諱敗,不攘功,不鬼大臣。天元竄開縣,額勒登保病留太平,遣楊遇春、穆克登布追之。將與德楞泰夾擊,而楊開甲、辛聰、王廷詔、高天升、馬學禮諸賊以川北守禦嚴,無所掠,乘間由老林竄陜西城固、南鄭,提督王文雄不能御,前路賊且入甘肅。額勒登保疏請以川事付魁倫、德楞泰,自力疾赴陜,而德楞泰先已西行赴援,不及回軍。

五年春,天元糾脅日眾,乘魁倫初受事,遂奪渡嘉陵江,硃射鬥戰死。未幾,潼河復失守,川中震動。詔逮魁倫,起勒保與德楞泰同辦川賊,責額勒登保與那彥成專剿陜賊。時那彥成破南山餘賊於隴山、伏羌,德楞泰追王廷詔、楊開甲於成縣。額勒登保亦至,乃令德楞泰回川西,自與那彥成分三路,遏賊入川及北竄之路。楊遇春、穆克登布破張天倫於岷州,慶成等破張世龍於洮河。廷詔、開甲合犯大營,擊走之,分兵追賊。大軍移剿高天升、馬學禮,迭敗之,賊逾渭北竄,尋要之於鞏昌,又要廷詔、開甲於岷州。諸賊並逼回渭南,而張世龍等走秦州,將趨北棧。留那彥成追高、馬二賊,自率楊遇春、岱森保回陜,令王文雄及總兵索費英阿等分扼南北棧。張漢潮已為明亮所殲,餘黨留陜者糾合復眾。張世龍、張天倫為大兵所驅,竄滇安,皆注漢北山中,東向商、雒,賊復蔓延。嚴詔詰責,召那彥成回京。閏四月,額勒登保率楊遇春連敗賊於商、雒、兩岔河,令遇春扼龍駒寨,使不得犯河南。賊乃回竄,留後隊綴官軍,連破之洵陽大、小、中溪,設伏溪口,擒斬三千餘,斃藍號劉允恭、劉開玉,於是漢潮餘黨略盡,晉封三等子。楊開甲、辛聰、張世龍、張天倫、伍金柱、戴仕傑等皆西竄。五月,令楊遇春等追擊金柱等於漢陰手扳崖,陣斃賊目龐洪勝等。進攻楊開甲等於洋縣茅坪,賊踞山巔,誘之出戰,伏兵繞賊後夾擊,陣斬開甲。六月,賊竄甘肅徽縣、兩當,藍號陳傑偷越棧道,擒之。八月,遇春斬伍金柱於成縣,斃宋麻子於兩當,賊復回竄陜境。疏陳軍事,略謂:「賊蹤飄忽,時分時合,隨殺隨增,東西回竄,官軍受其牽綴,稍不慎即墮術中,堵剿均無速效,自請治罪。」又言:「地廣兵單,請將防兵悉為剿兵,防堵責鄉勇,促築陜、楚寨堡以絕擄掠。」溫詔慰勞,以剿捕責諸將,防堵責疆吏,分專其任。會賊逼武關,截擊走之。

六年春,奏設寧陜鎮為南山屏障,如議行。二月,楊遇春擒王廷詔於川、陜交界鞍子溝,擒高天德、馬學禮於寧羌龍洞溪,三賊皆最悍。詔晉二等子,復雙眼花翎。時賊之著者,陜西冉學勝、伍懷志,湖北徐天德、茍文明,四川樊人傑、冉天泗、王士虎等,尚不下十餘股。四月,剿學勝於渭河南岸,又蹙之於漢南,賊遁平利。張天倫糾合五路屯洵陽高塘嶺、劉家河,令楊遇春擊走之。五月,穆克登布擒伍懷志於秦嶺。七月,遇春擒冉天泗、王士虎於通江報曉埡,徐天德、冉學勝並為他師所殲;而姚之富子馨佐及白號高見奇、辛斗等方擾寧羌,督諸將進剿,逼入川北。九月,總兵楊芳等擒辛鬥於通江。十月,豐伸、桑吉斯塔爾擒高見奇於達州。於是賊首李元受、老教首閻天明等各率眾降,賊勢窮蹙。條上搜捕事宜,詔嘉獎,晉封三等伯。十一月,茍文明合各路殘匪竄階州,裹脅復眾,回竄廣元、通江。十二月,敗之於瓦山溪,文明竄開縣大寧。七年正月,斬黃號辛聰於南江,文明由西鄉偷渡漢江。額勒登保自請罪,降一等男,詔以川匪責德楞泰、勒保等,額勒登保兼西安將軍,仍專辦陜賊。二月,文明竄入南山,與宋應伏、劉永受合,督師入山搜剿。六月,殲其眾於龔家灣,文明僅以身免,劉永受潛遁,為鄉民所殲。七月,殲文明於寧陜花石巖,晉一等伯。疏陳軍事將竣,請撤東三省及直隸、兩廣兵,遠地鄉勇分別遣留。遂窮搜南山餘匪,八月,擒茍文齊,斃張芳。赴平利與德楞泰會剿楚匪,五戰,擒斬過半。十月,斃青號熊方青於達州,盡殲竹溪股匪。十一月,令穆克登布追賊通江鐵鐙臺,擒景英、蒲添香、賴大祥,及湖北老教首崔連樂,晉三等侯。著名匪首率就殲,零匪散竄老林。十二月,疏告蕆功,詔嘉額勒登保:「運籌決策,悉中機宜,躬親行陣,與士卒同甘苦,厥功最偉。」晉封一等侯,世襲罔替,授御前大臣,加太子太保,賜用紫韁。餘論封行賞有差。

八年春,留陜搜捕,擒姚馨佐、陳文海、宋應伏等於紫陽。穆克登布遇伏戰歿。六月,移師入川,擒熊老八、趙金友於大寧,熊老八即戕穆克登布者。疏陳善後事宜:「各省酌留本省兵勇:四川一萬二千,湖北一萬,陜西一萬五千,分布要地。隨征鄉勇有業歸籍,無業補兵,分駐大員統率。」七月,馳奏肅清,命暫留四川經理善後。編閱陜、楚營卡事竣,振旅還京。十二月,至,行抱見禮於養心殿,獎賚有加,命謁裕陵。

九年春,因前遭母憂不獲守制,補持服。尋命赴四川偕德楞泰殲餘孽。十年,回京,總理行營,充方略館總裁。八月,上幸盛京,額勒登保以病不克從,謁陵禮成,特詔加恩晉三等公爵。是月,卒於京師,年五十八。上聞震悼,回鑾親奠,禦制述悲詩一章。於地安門外建專祠,曰褒忠,謚忠毅,命吉林將軍修其祖墓立碑焉。

額勒登保初隸海蘭察部下,海蘭察謂曰:「子將才,宜略知古兵法。」以清文三國演義授之,由是曉暢戰事。天性嚴毅,諸將白事,莫敢仰視。然有功必拊循,戰勝親餉酒肉,賞巨萬不吝,人樂為用。嘗謂諸將曰:「兵條條生路,惟舍命進戰是一死路;賊條條死路,惟舍命進戰是一生路。惟有出其不意、攻其不備之一法。追賊必窮所向,不使休息。師行整伍,倉卒遇賊,即擊。每宿,四路偵探;臨敵,矢石從眉耳過,勿動。」於同列不忌功,亦不伐己功,尤嚴操守。凱旋過盧溝橋,他將輜重累累,獨行李蕭然,數騎而已。歿時,子謨爾賡額生甫數月,帝臨奠,抱置膝上,命襲侯爵,尋殤,以侄哈郎阿嗣,承襲一等威勇侯,自有傳。

額勒登保不識漢文,軍中章奏文牘,悉倚胡時顯。

時顯,字行偕,江蘇武進人。少困科舉。乾隆中,侍郎劉秉恬治金川糧餉,從司文牘獨勤。薦授兵部主事,充軍機章京,累遷郎中。和珅用事,數與抗,出為廣東雷州知府,以親老乞留。尋從福康安征苗有功,賜花翎。洎額勒登保剿教匪,從贊軍務,剛直無所徇,額勒登保能容之。每日跨馬與諸將偕,或有逗留,輒叱之。遇賊務當其沖,諸將無敢卻者。回營後,凡戰地曲折夷險,糧運斷續,器仗敝壞,兵卒勞饑,及賊出沒情狀,諸將功過,一一言之。軍中敬畏時顯與經略等。陳奏戰事必以實,上嘉經略,並嘉時顯。貓兒埡之戰,及擒王登廷,章奏不欺,特賜三品卿銜。在軍凡五年,累擢內閣侍讀學士、鴻臚寺卿。以勞卒於興安軍次,贈光祿寺卿,賜祭葬。

德楞泰,字惇堂,伍彌特氏,正黃旗蒙古人。乾隆中,以前鋒、藍翎長從征金川、石峰堡、臺灣,皆有功,累遷參領,賜號繼勇巴圖魯。五十七年,從福康安征廓爾喀,冒雨涉險,攻克熱索橋賊寨。加副都統銜,圖形紫光閣。尋授副都統,遷護軍統領。

六十年,率巴圖魯侍衛從福康安征湖南苗,與額勒登保並為軍鋒。福康安既解松桃、永綏圍,高宗悅,將待以不次之賞,於是德楞泰建議深入苗地為犁庭埽穴計。苗酋吳半生踞大烏草河以抗,大兵連克沿河諸寨,渡河抵盛華哨。苗於山半立木城,堅甚,斷其汲路,火攻克之,又克古丈坪,進攻摩手寨,由間道出寨後,奪據石城,遂偕額勒登保擒半生,授內大臣。進攻鴨保寨,克木城、石卡三十餘,又克天星寨木城七,石卡五,擒賊目吳八月。

嘉慶元年,福康安、和琳相繼卒於軍,先克乾州,又從將軍明亮克平隴,擢御前侍衛,署領侍衛內大臣。克險隘養牛塘山梁,賊首石柳鄧就殲,苗疆略定,錫封二等子爵,賜雙眼花翎。二年,命偕明亮移軍四川剿教匪。時賊首徐天德、王三槐踞重石子、香爐坪,南曰分水嶺,北曰火石嶺,賊卡林立,進戰,奪嶺,三槐撲營受創逸。五月,破重石子,明亮亦破香爐坪,追殲教首孫士鳳。會襄陽賊齊王氏、姚之富、樊人傑等竄入四川,與徐、王二匪合屯開縣南天洞,擊破之,賊分走雲陽、萬縣。雲陽教首高名貴欲與天德合,以計擒之,盡殲其眾於陳家山。七月,齊王氏等由奉節、巫山東走湖北,與明亮繞出宜昌迎剿,賊南趨,留明亮屯宜昌;自赴荊州解遠安圍。八月,賊犯荊門、宜城,往援之,會總督景安以索倫勁騎至,合剿大捷,二城得全。賊欲北竄河南,扼要隘,斬賊目袁萬相等,截回湖北,賜紫韁。九月,殲賊於房縣、竹谿、竹山,賊走陜西平利,圖入川東,敗之樹河口。賊北走紫陽,又合白號高均德,西走漢中。十一月,賊窺渡漢江,令副都統烏爾圖納遜突擊於江濱,竄入川境。

三年正月,均德復擾陜西褒城,與明亮夾擊,連敗之於洋縣、城固、洵陽。齊王氏、姚之富方竄廣元寧羌山中,乘虛由石泉渡漢,與均德合,東走漢陰。詔斥明亮戰不力,褫其職;嘉德楞泰每戰在前,責速剿。三月,與明亮追齊、姚二匪,由山陽至鄖西,日行百七十里,連破之於石河、甘溝,鄉勇遏其前,賊無去路,踞三岔河左右,兩山盡銳,圍攻悉殲之。齊王氏、姚之富投崖死,傳首三省。均德由鎮安竄雒南,敗之兩岔河,餘賊與李全、張天倫合。五月,又敗之五郎廟,均德走寧羌、廣元,合龍紹周、冉文儔踞渠縣大神山,有眾二萬。詔斥縱賊,奪爵職,留副都統銜。七月,偕惠齡、恆瑞攻克大神山,賊竄營山,蹙之黃渡河。均德中槍,逸入箕山坪,與羅其清合。箕山圍徑百餘里,三面陡絕,惟東南有路可通。徐天德、王登廷、樊人傑踞鳳凰寺,阻糧道,與為犄角。八月,克鳳凰寺,賊奔箕山,負固不下。十月,分三路進攻,克之。其清退踞大鵬寨,額勒登保自閬中來會剿。十一月,賊被攻急,乘夜雨撲營。德楞泰偵知之,潛伏賊寨南門,梯而登,火其寨;額勒登保等亦襲破西門,殲其清父從國;合兵窮追,擒其清於巴州方山坪,復花翎。冉文儔竄踞東鄉麻壩,乘除夕大破之於通江。

四年元旦,生擒文儔,盡殲其眾,予一等輕車都尉。經略勒保疏陳諸將惟額勒登保、德楞泰尤知兵,得士心,詔德楞泰專剿徐天德。天德與冷天祿竄涪州,冒難民入鶴田寨,擊走之,又敗之於開縣。三月,天德自大寧北趨,追及於太平;又遇龍紹周、唐大信等,迭擊之,賊不得犯陜境。既而天德入大寧老林,與紹周、大信及樊人傑、龔建、卜三聘、張天倫、辛聰等合,牽綴大軍。天德、建竄太平山箐,令賽沖阿分兵擊之;自擊人傑、紹周、大信、天倫於安康、紫陽,連破之,驅入川東,遂犯湖北。七月,線號龔文玉亦自夔州至,分兵追剿,擒文玉、三聘於竹谿,加予騎都尉世職。八月,命額勒登保為經略,德楞泰為參贊,赴興山截擊天德,逼回川東;躡追天倫及聰等入陜。十月,高均德改名郝以智,率賊萬,踞高家營,欲由白河窺渡漢。紹周及冉天元竄放馬場,欲趨紫陽。率賽沖阿、溫春回援,先破放馬場,進攻高家營,擒均德,檻送京師,晉封二等男爵。十一月,進兵川北,殲白號張金魁於通江,擒其黨符曰明等於廣元。十二月,追鮮大川、茍文明至川東,賊瞷大兵俱在川境,遂先後竄陜、甘。

五年正月,偕額勒登保分路抵秦州,而冉天元糾合徐萬富、汪瀛、陳得俸、張子聰、雷世旺眾五萬,遽乘間渡嘉陵江,分擾南部、西充、魁倫不能制,詔促德楞泰回援。二月,天元踞江油新店子,乃由間道進剿。賊分四路迎戰,銳甚,賽沖阿、溫春深入被圍;自馳援,夾擊竟日,殺傷相當,擒得俸,斬冉天恆,皆悍賊也。轉戰連奪險隘。三月,天元屯馬蹄岡,伏萬人火石埡後。德楞泰令賽沖阿攻包家溝,阿哈保攻火石埡,溫春攻龍子觀,自率大隊趨馬蹄岡,過賊伏數重始覺。俄伏起,八路來攻,人持束竹、濕絮御箭銃,鏖斗三晝夜,賊更番迭進,數路皆挫敗。德楞泰率親兵數十,下馬據山巔,誓必死。天元督眾登山,直取德楞泰,德楞泰單騎沖賊中堅,將士隨之,大呼奮擊,天元馬中矢蹶,擒之,賊遂瓦解。鄉勇亦自山後至,逐北二十餘里,擒斬無算。天元雄黠冠川賊,專用伏以陷官軍,至是五日四戰,致死決勝負,血戰破之,群賊奪氣,詔晉三等子。是月,復大破賊於劍州,又破張子聰、雷世旺於蓬溪,斬世旺,晉二等子,授成都將軍。

魁倫以失守潼河逮問,起勒保代為總督,與德楞泰合兵剿賊。四月,賊分擾遂寧、安岳,逼中江,欲趨成都。與勒保夾擊,連破之,邀擊於嘉陵江口,俘斬溺斃者數千;餘賊渡江,為達州鄉勇所敗,擒汪瀛:潼河兩岸肅清。自此德楞泰威震川中,諸將往往假其旗幟,賊望見輒走。閏四月,追賊至達州、新寧,殲劉君聘、茍文富;而白號茍文明、鮮大川、樊人傑等復由陜入川。五月,移師川北,賊走營山、渠縣,六月,敗之恩陽河;又與勒保合擊,殲茍文禮於岳池。七月,大川為民寨誘斬,文明遁。八月,追剿白號賊於東鄉,殲湯思舉,餘賊與趙麻花、王珊合。九月,與勒保夾擊於雲陽,麻花、珊先後斃。十月,湖北黃、白、藍、線四號賊合犯夔、巫。龍紹周由太平、通江北竄,兵至賊去,兵去賊至;樊人傑、冉學勝、王士虎遂由川入陜;徐天德由陜入楚。詔斥德楞泰堵剿不力,降一等男。十二月,李彬、楊開第、齊國謨合窺嘉陵江。與勒保合擊,連敗之於渠縣安仁溪、儀隴觀音河,斃開第、國謨,晉三等子。

六年正月,白號高天升自洵陽偷渡漢江,圖竄河南,追及於山陽乾溝,破之,追殲之於野豬坪,復一等子。二月,擊龍紹周於興安,逼入川境,連敗之於大寧長壩、二郎壩。紹周竄湖北竹山、房縣,復敗之,走太平,復雙眼花翎。四月,徐天德、樊人傑合曾芝秀、陳朝觀竄陜西白河,分擾民寨。遣兵直攻其巢,擒朝觀。五月,大破賊於西鄉,天德竄紫陽。率賽沖阿、溫春蹙之仁和新灘。大雨水漲,天德溺斃。紹周乘虛闌入房縣、竹谿,截擊之,復回太平,擒其黨陳文明。八月,追至巫山、巴東,擒王鵬、李天棟。九月,紹周遁平利,令賽沖阿等追殲之,晉封二等繼勇伯,仍用巴圖魯舊號也。十二月,茍文明西擾寧羌,與額勒登保夾擊。賊竄川北,大敗之於通江,走開縣,遣兵追之。自率輕騎赴大寧,斷其入楚之路。

七年正月,文明復入陜北,竄老林,至秋,乃為陜軍所殲。川東零匪猶四擾,詔德楞泰仍專辦川賊。二月,破線號餘匪於奉節,又破白號張長青於雲陽。時樊人傑及崔宗和、胡明遠、戴仕傑、蒲天寶等麕聚湖北境。四月,率精兵間道抵東湖,繞出賊前,夾攻雞公山賊巢。天寶別屯當陽河,五月,冒雨進擊,天寶負創走,又敗之於穆家溝,分兵留剿;自移師東趨,直取人傑,冒雨入馬鹿坪山中,出賊不意,痛殲之。人傑竄竹山,投水死。人傑倡亂最久,諸賊聽指揮,與冉天元埒,至是伏誅,晉三等侯。七月,天寶乘間奪踞興山、房縣交界鮑家山,死守抗拒。以大軍綴其前,令總兵色爾袞、蒲尚佐率精兵出深箐攻賊巢,截其去路,擒斬殆盡。天寶遁,至竹谿墜崖死。

時巴東、興山尚有餘匪,皆百戰之餘,悉官軍號令及老林路逕,屢合圍,輒乘霧溜崖突竄。分軍遇之則不利,大隊趨之則兔脫,所餘無幾,而三省不能解嚴。與額勒登保、吳熊光會於竹谿議搜剿,額勒登保專任陜境,德楞泰專任楚境,先後殲戴仕傑、趙鑒、崔連洛、崔宗和、陳仕學、熊翠諸賊,迨十一月,捕斬略盡,優詔,晉封一等侯,加太子太保,命其子蘇沖阿★K7珍賚至軍宣慰。八年,駐巫山、大寧,捕逸匪曾芝秀、冉璠、張士虎、趙聰等,先後擒殲。至冬事竣,入覲熱河行在,帝大悅,禦制詩賜之,恩賚優渥。尋以陜西南山餘孽擾及川境,命回鎮成都。遣將招降,數為賊害,坐降二等侯。九年,偕額勒登保窮搜老林,斬首逆茍文潤,餘匪悉平,復一等侯。十年,召授領侍衛內大臣,充方略館總裁,總理行營事務,管理兵部。

十一年,寧陜鎮新兵陳達順、陳先倫等作亂,命馳往剿治。叛將蒲大芳等乞降,縛獻達順等,磔之。大芳等遣戍回疆。議以降眾歸伍,詔斥寬縱,奪職。尋授西安將軍。十三年,剿定瓦石坪叛匪。十四年,晉三等公。尋卒,柩至京師,帝親奠,禦制詩輓之,謚壯果。詔四川建立專祠,入祀京師昭忠祠。

德楞泰英勇超倫,戰必身先陷陣,名與額勒登保相亞。馬蹄岡之戰,轉敗為勝,時稱奇績。既卒,奉詔褒恤,特舉是役保障川西數十萬生靈,厥功最偉。在軍俘獲,必詳訊省釋,未嘗妄殺良民婦女,保全甚眾,蜀民尤感頌焉。

子蘇沖阿,一品廕生,授侍衛。每德楞泰戰勝,輒擢其官,累遷至盛京副都統,署黑龍江將軍,襲一等侯。孫倭什訥,杭州將軍;曾孫希元,吉林將軍:並嗣爵。次孫花沙納,官至吏部尚書,自有傳。

論曰:仁宗親政,以三省久未定,卜於宮中,繇曰:「三人同心,乃奏膚功。」後事平,敘勞:額勒登保第一,德楞泰次之,勒保又次之。論戰績,勒保未足與二人比,然當德楞泰偕明亮由楚入陜,見民苦虜掠,陳堅壁清野策,廷議以築堡重勞,未之許也;勒保至四川,始力行之,推之三省,賊竟由是破滅。三人者相得益彰,未容有所優劣:勒保寬能容眾,額勒登保忠廉忘私,德楞泰仁及俘虜,識量並有過人。為國方、召,延世侯封,豈偶然哉!


\end{pinyinscope}