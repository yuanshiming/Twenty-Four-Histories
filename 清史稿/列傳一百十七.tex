\article{列傳一百十七}

\begin{pinyinscope}
福康安孫士毅明亮

福康安,字瑤林,富察氏,滿洲鑲黃旗人,大學士傅恆子也。初以雲騎尉世職授三等侍衛。再遷頭等侍衛。擢戶部侍郎、鑲黃旗滿洲副都統。

師征金川,以溫福為定邊將軍,阿桂、豐升額為副將軍,高宗命福康安齎印往授之,即授領隊大臣。乾隆三十八年夏,至軍,阿桂方攻當噶爾拉山,留福康安自佐。木果木師敗,溫福死事,復命阿桂為定西將軍,分道再舉。攻喇穆喇穆,福康安督兵克其西各碉,與海蘭察合軍,克羅博瓦山;北攻,克得斯東寨。賊夜乘雪陟山,襲副將常祿保營,福康安聞槍聲,督兵赴援,擊之退。賊屯山麓,乘雨築兩碉,福康安夜率兵八百冒雨逾碉入,殺賊,毀其碉,上手詔嘉其勇。進克色淜普山,破堅碉數十,殲賊數百。又與額森特、海蘭察合軍,攻下色淜普山南賊碉,遂盡破喇穆喇穆諸碉卡,並取日則丫口。再進克嘉德古碉,攻遜克爾宗西北寨。賊潛襲我軍後,福康安擊之退。賊以距勒烏圍近,屢夜出擊我師,福康安與戰屢勝。

阿桂慮賊守隘不時下,改道自日爾巴當噶路入;檄福康安攻下達爾扎克山諸碉。再進,攻格魯克古,率兵裹糧,夜逾溝攀崖,自山隙入當噶海寨,克陡烏當噶大碉、桑噶斯瑪特木城石卡。再進,克勒吉爾博寨。阿桂令福康安將千人從海蘭察赴宜喜,自甲索進攻得楞山,焚薩克薩古大小寨數百,渡河取斯年木咱爾、斯聶斯羅市二寨。再進,次榮噶爾博山。擢內大臣,賜號嘉勇巴圖魯。再進,至章噶。福康安偕額森特攻巴木圖,登直古腦山,拔木城、碉寨五十,焚冷角寺,遂克勒烏圍。

阿桂令取道達烏圍進攻噶拉依,分其軍為七隊,福康安率第一隊,奪達沙布果碉、當噶克底、綽爾丹諸寨為木柵,斷科思果木走雅瑪朋道。進克達噶木碉二,阿穰曲前峰碉木城各二十。焚奔布魯木護起寨。取舍勒圖租魯傍碉一、寨二,格什格章寨一,薩爾歪碉寨三,阿結占寨二。陟科布曲山梁,盡得科布曲諸寨。四十一年春,再進,克舍齊、雍中二寺。自拉古爾河出噶拉依之右,移砲擊其寨。噶拉依既下,金川平。論功,封福康安三等嘉勇男。師還,郊勞,賜御用鞍轡馬一。飲至,賜緞十二端、白金五百。圖形紫光閣,賜雙眼花翎。授正白旗滿洲都統,出為吉林、盛京將軍。

授雲貴總督。南掌貢象,自陳為交趾所侵,乞以餘象易砲。福康安諭以國家法制有定,還其象,不予砲。疏入,上深韙之。移四川總督,兼署成都將軍。四川莠民為寇盜,號啯匪,命福康安捕治。逾年,福康安疏言盜已徐戢,陳善後諸事。擢御前大臣,加太子太保。召還京,署工部尚書。授兵部尚書、總管內務府大臣。

四十九年,甘肅回田五等立新教,糾眾為亂。授參贊大臣,從將軍阿桂討賊。旋授陜甘總督。師至隆德,田五之徒馬文熹出降。攻雙峴賊卡,賊拒戰,阿桂令海蘭察設伏,福康安往來督戰,殲賊數千,遂破石峰堡,擒其渠。以功,進封嘉勇侯。轉戶、吏二部尚書,協辦大學士。

五十二年,臺灣林爽文為亂,命福康安為將軍,而以海蘭察為參贊大臣,督師討之。時諸羅被圍久,福建水師提督柴大紀堅守。上褒大紀,改諸羅為嘉義,以旌其功。陸路提督蔡攀龍督兵赴援,圍未解。福康安師至,道新埤,援嘉義,與賊戰侖仔頂,克俾長等十餘莊。會日暮,雨大至,福康安令駐師土山巔,賊經山下,昏黑無所見,發銃仰擊。福康安戒諸軍士毋動。既曙,雨霽,海蘭察已自他道入,師與會,圍解。進一等嘉勇公,賜紅寶石帽頂、四團龍補服。

大紀以方在圍中,謁福康安未具櫜鞬禮,福康安銜之,疏論大紀骫法、牟利諸罪狀,並及攀龍陳戰狀不實。上以大紀困危城久,攀龍亦有勞,意右之,詔謂「二人或稍涉自滿,在福康安前禮節不謹,為所憎,遂直揭其短」,戒福康安宜存大臣體。然大紀卒以是坐死。時論冤大紀,亦深非福康安嫉能,不若傅恆遠也。福康安復劾攀龍,左遷;而福州將軍恆瑞師逗遛不進,福康安與有連,力庇之,詔亦斥其私。

福康安既解嘉義圍,令海蘭察督兵追捕爽文,檻致京師;復得副賊莊大田。臺灣平,賜黃腰帶、紫韁、金黃辮珊瑚朝珠。命臺灣、嘉義皆建生祠塑像,再圖形紫光閣。疏請募熟番補屯丁,並陳善後諸事,要在習戎事,除奸民,清吏治,肅郵政,上悉從之。旋授閩浙總督。

五十四年,安南阮惠攻黎城,孫士毅師退。上移福康安兩廣總督,詔未至,福康安疏請往蒞其事。上獎福康安忠,謂:「大臣視國如家,休戚相關,當若此也。」惠更名光平,乞輸款,福康安為疏陳,請罷兵,上允之。御史和琳劾湖北按察使李天培為福康安致木材,令湖廣糧船運京師,福康安疏請罪。上手詔謂阮光平方入朝,特寬之;命奪職留任,仍罰總督俸三年、公俸十年。五十五年,福康安率光平朝京師,以獲盜免罰總督俸。

五十六年,廓爾喀侵後藏,命福康安為將軍,仍以海蘭察為參贊大臣,督師討之,免罰公俸。五十七年三月,福康安師出青海,初春草未盛,馬瘠,糧不給,督諸軍速進。行四十日,至前藏,自第理浪古如絨轄、聶拉木,察地勢,疾行向宗喀,至轄布基。諸道兵未集,督所部分六隊,趨擦木,潛登山,奪賊前後二碉,殲賊渠三、賊二百餘,擒十餘。進次瑪噶爾轄爾甲山梁,賊渠手紅旗,擁眾登,令設伏誘賊進,至山半,伏起橫擊,搴旗賊盡殪。進攻濟隴,濟隴當賊要隘,大碉負險,旁列諸碉卡,相與為犄角;乃分兵先翦其旁諸碉卡,並力攻大碉,縛大木為梯,督兵附碉登,毀壘。戰自辰至亥,克其寨,斬六百,擒二百。捷聞,上為賦志喜詩書扇,並解御用佩囊以賜。

六月,自濟隴入廓爾喀境,進克索勒拉山。度熱索橋,東越峨綠山,自上游潛渡。越密里山,攻旺噶爾,克作木古拉巴載山梁。攻噶勒拉、堆補木諸山,破甲爾古拉、集木集兩要寨。轉戰深入七百餘里,六戰皆捷。上詔褒福康安勞,授武英殿大學士。福康安恃勝,軍稍怠,督兵冒雨進;賊為伏以待,臺斐英阿戰死。廓爾喀使請和,福康安允之。廓爾喀歸所掠後藏金瓦寶器,令大頭人噶木第馬達特塔巴等齎表進象、馬及樂工一部,上許受其降。師還,加賜福康安一等輕車都統畀其子德麟,授領侍衛內大臣,視王公親軍校例,置六品頂戴藍翎三缺,官其傔從。復圖形紫光閣,大學士阿桂讓福康安居首。

福康安初征金川,與海蘭察合軍討亂回,同為參贊;及征臺灣、定廓爾喀,皆專將,海蘭察為參贊,師有功,受殊賞。上手詔謂:「福康安能克陽布,俘拉特納巴都爾、巴都爾薩,當酬以王爵。今以受降班師,不克副初原。然福康安孝賢皇后侄,大學士傅恆子,進封為王,天下或議朕厚於後族,富察氏亦慮過盛無益。今如此蕆事,較蕩平廓爾喀倍為欣慰。」陽布,廓爾喀都城;拉特納巴都爾等,其渠名也。五十八年,疏陳西藏善後十八事,詔從之。

安南國王阮光平卒,上慮其國且亂,命福康安如廣西。福康安母卒於京師,令在任守制。福康安途中病,命御醫往視。福康安疏言:「安南無事,乞還京師,冀得廬墓數日。」詔許之,加封嘉勇忠銳公。移四川總督。旋又率金川土司入覲。恆秀時為吉林將軍,以採參虧庫帑累民,命福康安蒞讞,擬罪輕,上責福康安袒戚誼。復移雲貴總督。方寒,賜御服黑狐大腿褂。

六十年,貴州苗石柳鄧,湖南苗吳半生、石三保等為亂,命福康安討之。柳鄧圍正大營、嗅腦營、松桃三城,福康安師至,力戰,次第解三城圍,賜三眼花翎。福康安率貴州兵破老虎巖賊寨,詗得柳鄧蹤跡。和琳時為四川總督,將四川兵來會,攻滿華寨,焚賊寨四十。柳鄧入湖北,投三保,三保方圍永綏,福康安督兵赴援。師當渡,賊築卡拒守。分兵出上流,縛筏,縱民牧牛,設伏;待賊至掠牛,伏起,奪賊船,所縛筏亦順流至,師盡濟。攻石花寨,越得拉山戰,殺賊甚眾,令總兵花連布間道援永綏,師從之,戰三日,圍解。

進次竹子山,賊屯蘭草坪西北崖,以板為寨,樹旗東南山闕;乃設伏對山,仍督兵若將自山闕入。賊來戰,伏兵發砲,賊潰,退保瑯木陀山;再進,克之。山西為登高坡,與黃瓜山對,分兵出五道,冒風雨克黃瓜山,焚寨五十六;攻蒩麻寨,奪大小喇耳山,焚寨四十。半生、三保悉眾拒戰,分兵攻雷公山,阻其援兵,擊破西梁上中下三寨。再進至大烏草河,循河克沙兜寨、盤基坳山;戰於板登塞,再戰於雷公灘,賊屢敗。取右哨營,渡河,於群山中越險,進克馬蝗沖等大小寨五十。至狗腦坡,山益險,兵皆附葛藤,冒矢石,行陟其巔,破賊寨;再進,克蝦蟆峒、烏龍巖。攻茶它,降者七十餘寨。上移福康安閩浙總督,進封貝子。

再進,克巖碧山,焚巴溝等二十餘寨。再進攻麾手寨山,總兵花連布將廣西兵克苗寨四十,賜貂尾褂。圍高多寨,吳半生窮蹙出降。上官福康安子德麟副都統,在御前侍衛上行走。再進攻鴨保寨,鴨保右天星寨,為賊中奇險處,督兵自雪中求道,進取木城七、石卡五,克垂藤、董羅諸寨,賜御服黃裏玄狐端罩。旋克大小天星寨。進攻木營,乘風雪夜進,拔地良、八荊、桃花諸寨。自平隴復乾州,盡克擒頭坡、騾馬峒諸隘,焚其寨三百。嘉慶元年,再進,克吉吉寨、大隴峒等寨。戰於高吉陀,再戰於兩岔溪,屢敗賊。賊襲木營,攻擒頭坡,皆以有備敗走。克結石岡,焚牧牛坪等大小寨七十。進克官道溪,再進攻大麻營石城,至廖家沖,奪山巔石卡。夜間,道出連峰坳,奪山梁七。上褒福康安,命贈傅心互貝子。

福康安染瘴病作,猶督兵進,五月,卒於軍。仁宗制詩以誄,命加郡王銜,從傅恆配太廟,謚文襄。子德麟,襲貝勒,遞降至未入八分公,世襲罔替。

福康安受高宗殊寵,師有功。在軍中習奢侈,犒軍金幣輒巨萬,治餉吏承意指,糜濫滋甚。仁宗既親政,屢下詔戒諸將帥毋濫賞,必斥福康安。德麟迎喪歸,將吏具賻四萬有奇,責令輸八萬。德麟旋坐雩壇視牲誤班,降貝子。

孫士毅,字智冶,一字補山,浙江仁和人。少穎異,力學。乾隆二十六年進士,以知縣歸班待銓。二十七年,高宗南巡,召試,授內閣中書,充軍機章京。遷侍讀。大學士傅恆督師討緬甸,以士毅典章奏。敘勞,遷戶部郎中。擢大理寺少卿。出為廣西布政使。擢雲南巡撫。總督李侍堯以贓敗,士毅坐不先舉劾,奪職,遣戍伊犁,錄其家,不名一錢。上嘉其廉,命纂校四庫全書,授翰林院編修。書成,擢太常寺少卿。復出為山東布政使。擢廣西巡撫,移廣東。初上官,疏言:「廣東海洋交錯,奸宄易藏。惟有潔以持身,嚴以察吏,不敢因循諱飾。」上諭以勉效李湖,湖為廣東巡撫,以風厲有聲為上所深賞也。

尋署兩廣總督。陜甘總督福康安議練兵,詔下雲、貴、四川、兩廣、福建諸行省令仿行。士毅疏請廣東練水陸兵二萬八千五百三十二人,廣西練兵一萬一千二百九十六人,選人材精壯、技藝嫺習,責督、撫、提、鎮實心訓練;請嚴立科條,以懲積習。上諭曰:「此可徐徐為之,而必以實。」尋還巡撫任。廣東民悍,多逋賦,州縣吏當上計,或以私財應,冀課最,民益延抗為得計。士毅詳覈積逋,遣幹按治逋賦最多諸州縣,自乾隆四十年後,具冊督追。州縣吏以私財應計政者,察無他私弊,以督追所得償之。上獎其能,惟謂:「州縣吏職催科,乃以不能振作,民多逋賦。以私財應計政,不罪其誑已為寬典;若以督追所得償之,將何以示儆?令續徵逋賦當悉入官。」茭塘者,群盜所聚,拒捕傷官。士毅擒其渠,戮以徇。上復嘉其能,賜花翎。兩廣總督富勒渾縱其僕受賕,事聞,下士毅按治得實,富勒渾坐譴。上以士毅持正,即遷兩廣總督。富勒渾疏論廣東鹺政,請增運艘,按季徵餉價,復三十九埠運商清積逋。士毅受事,疏言:「增運艘,當去封押之擾,定經久之規,俾新舊船戶皆各樂從;按季徵餉價,當復舊例,歲終奏銷;三十九埠運商以逋課黜,中鉛山、南康、上猶、英德四埠當先復,清積逋當自三十九埠始。」皆下部議行。

五十二年,臺灣林爽文為亂,士毅詣潮州戒備。師行,遣兵助剿,芻茭、器械皆立辦,加太子太保,賜雙眼翎、一等輕車都尉世職。五十三年,臺灣平,圖形紫光閣。會安南國王黎維祁為其臣阮惠所逐,其母、妻叩關告變。士毅以聞,督兵詣龍州防鎮南關,帝嘉其識輕重、知大體,命自廣西入安南,別遣雲南提督烏大經自蒙自進。阮惠遣將拒於壽昌江,又分兵屯嘉觀。士毅師至,擊破惠所遣將,渡壽昌江,再進至市球江,惠守備甚設。士毅令陽於下游為浮橋,若將渡;密遣總兵張朝龍自上游渡,出賊後,賊恇擾。士毅勒兵乘筏渡,賊棄寨走;縱擊,賊自投江中死,尸蔽江。游擊張純等亦擊破惠屯嘉觀軍,副將慶成等設伏擒惠將。師再進至富良江,江南即黎城,惠令盡收戰艦泊南岸拒守。士毅縛筏載兵,令提督許世亨將二百人夜過江,掠小舟數十,更番渡兵。黎明,兵渡者二千餘。惠軍以舟遁,張純追及之,分焚其舟,盡殲之,遂復黎城,阮惠走富春。維祁至軍中,士毅承旨封為安南國王。捷聞,封一等謀勇公,賜紅寶石頂。士毅辭,不許。命班師,士毅猶豫未即行。

五十四年春正月,阮惠率其徒攻黎城,維祁亦挈其孥潛遁。士毅引兵退,渡市球江,駐江北。惠軍追至,總兵李化龍殿,度浮橋,墮水死;浮橋斷,提督許世亨等皆戰死。士毅還入鎮南關,維祁與母子偕至,置諸南寧。上以士毅不遵詔班師,有此挫折,罷封爵,並撤紅寶石頂、雙眼花翎,解總督任,以福康安代之。方惠追我師至富良江,士毅欲復渡江與決戰,世亨力諫,謂損大臣、傷國體,令千總薛忠挽其韁而退。至是具疏自劾,令駐鎮南關治事。惠尋遣使求內附,福康安至,與士毅嚴斥之。既,以黎氏瞀亂,不堪復立國,遂偕奏安南不必用兵狀,帝從其議。尋召士毅還京師,授兵部尚書,充軍機大臣,直南書房。是年冬,命署四川總督,逾歲真除。未幾,兩江總督書麟坐高郵書吏偽印冒徵被譴,以士毅代之,諭以江南吏治廢弛久,當黽勉整飭,毋徇隱。徐州王平莊河決,築毛城鋪堤堰,賑被水諸州縣,俱稱旨。五十六年,召授吏部尚書、協辦大學士。

廓爾喀用兵,命攝四川總督,督餉。士毅自打箭爐出駐察木多,師已入後藏,復馳詣前藏,饋運無匱。以勞,復賜雙眼花翎。五十七年,廓爾喀平,再圖形紫光閣。旋授文淵閣大學士,兼禮部尚書。偕福康安、和琳駐前藏謀善後。福康安率金川土司入覲,命士毅再權四川總督。福康安移雲貴總督,以和琳代之。上令士毅留四川董理討廓爾喀之役軍需奏銷,士毅乞留福康安、和琳會覈,上不許。

六十年春,湖南苗為亂,入四川秀山境,士毅督兵駐守擊賊。嘉慶元年,湖北教匪為亂,侵四川酉陽境。士毅移軍來鳳,戰屢勝,封三等男。賊屯茶園溪,大雨旬日,詗無備。夜擊賊,人持短兵坌湧入,千總張超執長矛先登,斬其魁,追奔四十餘里。賊退據旗鼓寨,士毅移軍從之。六月,卒於軍中,贈公爵,謚文靖。以其孫均襲伯爵。

士毅故善和珅,病篤,遺書請入旗,高宗特許之,命均入漢軍正白旗,授散秩大臣。尋以幼罷。十一年,自陳廢疾,請以同祖弟玉墀襲爵,仁宗諭曰:「士毅克黎城,皇考命班師。士毅意在貪功,遲延失事,兵潰入關。所奏多有虛飾。朕體皇考遺意,未予追求。今均既病廢,士毅原授伯爵當裁撤,並令均出旗歸原籍。」

明亮,富察氏,滿洲鑲黃旗人,都統廣成子,亦孝賢高皇后侄也。初以諸生尚履親王允祹女,為多羅額駙,授整儀尉。累遷鑾儀衛鑾儀使。乾隆三十年,授伊犁領隊大臣,從征烏什亂回。再移寧古塔副都統。從征緬甸,有功。

三十六年,兩金川為亂,命以護軍統領佐四川總督桂林出師。明年,桂林師出墨壟溝,敗績,明亮未以聞,上責其隱,奪職。旋授頭等侍衛銜,令從軍自效。時阿桂以參贊大臣代將,令明亮仍出墨壟溝,潛襲甲爾木,奪第一山梁。地高寒,不俟令引還,阿桂奏劾,降二等侍衛銜。復攻甲爾木,乘雪陟其中峰,克所築碉卡,授二等侍衛。尋攻真登梅列,斷賊糧道,遷頭等侍衛,加副都統銜。復自都恭進破噶察、丹嘉諸寨,與阿桂會於僧格宗。阿桂授副將軍,命明亮為領隊大臣。再進,自僧格宗渡河,東攻美諾,令侍衛德赫布等為前隊,明亮繼,逐賊至美都喇嘛寺,圍美諾,戰一晝夜,克之。小金川悉定。

進討大金川,溫福出西路,豐升額出北路,而阿桂出南路,明亮為參贊。三十八年正月,師次當噶爾拉山,亙二十餘里,賊築十四碉拒守。明亮攻克第五、第四兩碉。居數月,溫福師敗,僧格宗、美諾皆陷。從阿桂斂師退駐翁古爾壟,擢廣州將軍。十月,師再舉,阿桂出西路,授明亮定邊右副將軍,出南路,當一面。自思紐順河取得里、得木甲諸寨,襲破宅壟,復取僧格宗,與阿桂會美諾。小金川復定,賜御用黑狐冠。三十九年正月,與阿桂策定進軍道,明亮自巴旺、布拉克底土司進次馬奈。馬奈山峻險,河南有地曰斯第,為賊寨障。明亮夜攻馬奈,遣參贊大臣富德自駱駝溝出寨後夾攻,戰二日,克之。再進,次絨布寨。分兵授領隊大臣奎林,以皮船渡河,取斯第山梁木城二。再進攻卡卡角,其前地曰庾額特,山負河而立,危峰護其右,勢絕險,山腰徑隘,賊夾以巨碉。屢攻不能下,於其右築五碉衛餉道。攻穆穀諸寨,賊拒守益力,而奎林軍以乏水移駐深嘉卜。明亮詗得泉,使富德、奎林移軍就之。分道攻斯第,賊前後並至,斷我軍為數部,戰甚力,侍衛阿爾都陟險焚賊卡,乃破圍出。明亮策攻正地,深入不遇賊,慮阻險設伏,未即進。阿桂令改出北路,與參贊大臣舒常合軍攻宜喜,進克達爾圖山梁。賊築十八碉,迭戰克其十五,復自木克什進次帶石,東取穀爾提,西攻沙壩山,焚碉卡二百餘。賊據隘斷我軍道,別得道出。

四十年四月,阿桂令參贊大臣海蘭察助攻宜喜,分兵十餘道攻賊碉。明亮與海蘭察、舒常巡行督戰,克薩克薩谷山梁,達爾圖、得楞、沙壩山諸賊皆潰,並得日旁諸寨,授內大臣。再進克基木斯丹當噶山,海蘭察還佐阿桂。明亮軍進次扎烏古,攻碉未即下,令奎林以砲擊賊,破石真噶,北取瑯谷,移師駐其地。阿桂已克勒烏圍,進攻噶拉依,令明亮攻碾占。未即下,明亮疏請簡精銳佐阿桂並力出西路。上不謂然,詔切責,乃自瑯穀進攻納木迪。阿桂遣駐美諾兵千餘助明亮。明亮策賊守納木迪,扎烏古備必疏,遣奎林出間道襲破之。自日斯滿至阿爾古山梁,上下二十餘里,諸碉卡盡下,納木迪賊焚寨走。再進攻日斯滿先取得耳谷,斷賊後路;令和隆武等夾擊,大破賊,還攻碾占。碾占為乃當山巔,其北曰阿爾占,其南曰甲雜。明亮襲破阿爾占,夜督兵縋下峭壁,陟山梁,盡破諸碉寨,遂攻乃當,賊潰遁。圍甲雜,缺一面當水,賊走,師乘之,皆墮水死。阿桂軍臨噶拉依,明亮取獨松趨正地,降馬爾邦,令奎林等軍於巴布朗穀。督兵與阿桂軍會,偕阿桂疏報噶拉依圍合。四十一年春,命封一等襄勇伯,賜雙眼花翎。師克噶拉依,金川平。時議以成都將軍駐雅州總邊政,以授明亮。明亮以雅州地隘,請還駐成都,陳善後諸事,皆從之。夏,師還,上郊勞,賜銀幣、鞍馬。冬,復率諸土司入覲,命在軍機處行走。四十三年,改授四川提督。四十五年,復率諸土司入覲。

四十六年,甘肅撒拉爾回亂,攻蘭州。明亮將四川兵自鞏昌入甘肅,合軍討賊。上幸木蘭,覲行在,改授烏魯木齊都統。員外郎開泰罪譴,命永遠枷號;明亮徇協領富通請釋之,未以聞。四十八年,移伊犁將軍,而富通當引見,開泰懼失庇,投水死。事聞,上逮明亮詣京師,獄成,罪絞待決。四十九年,甘肅固原回復亂,大學士阿桂出視師,命釋明亮,賜藍翎侍衛從軍。亂定,授頭等侍衛。累遷鑲紅旗蒙古都統。五十五年,授刑部尚書。五十六年,出為黑龍江將軍。五十八年,移伊犁將軍。六十年,復入為正紅旗漢軍都統。坐在黑龍江令兵輸貂予賤值,奪職,留烏魯木齊自效。

貴州苗石柳鄧、湖北苗石三保等為亂,嘉慶元年,命明亮出佐湖南軍,授頭等侍衛,旋以副都統銜署廣州將軍。賊久據孝感,署湖廣總督永保討之未克,明亮將三千五百人以往,至潼川鋪,賊出戰,分兵伏黃金廟,攻賊壘,伏起,賊砲裂,斂入城。明亮令積柴城門外縱火,賊突出,皆墮壕,三日火始燼,城遂破,賜輕車都尉世職。攻鍾祥,得賊渠張家瑞等。戰於雙溝,屯呂堰,賊至,擊敗之。再進攻平隴,破養牛塘、剛息沖諸隘。圍石隆,奮戰,斬石柳鄧,獲其孥,封二等襄勇伯,賜雙眼花翎。

是時教匪起,延及四川、陜西、湖北三省,命明亮督兵赴四川,與總督宜綿合軍討賊。二年,明亮自永綏入四川,與宜綿軍合。轉戰,焚金峨寺,破重石子、香爐坪,克分水嶺、火石嶺諸卡。賊渠王三槐出戰,大破之,三槐中槍逸,賊死者萬餘人。復戰精忠寺,俘三槐母。襄陽賊渠姚之富、齊王氏等竄四川,與三槐及達州賊渠徐添德合,勢復張。之富等據開縣南天洞,明亮擊破之,逐賊,戰於大涼山。雲陽賊渠高名貴應賊,明亮與宜綿策擒名貴,殲其從。賊攻白帝城,明亮循江下宜昌,賊來犯,擊破之。逐賊至獨樹,會湖廣總督景安師至,合擊,逼賊入南漳山中。度賊且渡漢北入河南境,令總兵長春屯穀城為備;督兵出隆中,賊北走,擊之潰,賜紫韁。

賊屢敗,不能北渡,乃自房縣入陜西境。明亮逐賊,屢戰皆捷,先後殺六千餘人。賊走紫陽,明亮師次白沔峽,之富等與諸賊渠張漢潮、高均德分道竄走,明亮逐漢潮、均德入漢中。上責明亮不當置群盜而但逐漢潮、均德,奪爵及雙眼花翎、紫韁。之富等亦渡江與均德合走漢陰,其徒入城固、南鄭,乃奪職,逮詣京師。旋以軍事急,命留軍自效。督兵逐之富、齊王氏自山陽至鄖西,急擊之,之富、齊王氏皆投崖死,賜副都統銜、花翎。命捕治均德。

師進次西鄉,漢潮與諸賊渠詹世爵、李槐合萬餘人,自竹谿至平利、太平,明亮追及於池子山,戰,馘世爵、槐,而漢潮還走南鄉,復攻陷西鄉、石泉,命奪花翎。漢潮入河南境,攻盧氏,明亮赴援,漢潮復走陜西,攻五郎。四年,上授勒保經略大臣,授明亮副都統、參贊大臣,逐漢潮入漢中。勒保弟永保先以孝感、鍾祥剿賊無功坐譴,嫉明亮;至是起署陜西巡撫,與明亮不相能,漢潮往來奔竄,不以師應。上徵勒保還,命明亮代將,遷正紅旗漢軍都統。明亮劾永保軍久駐不進,永保言明亮有手札尼其移軍。上為奪明亮職,逮詣京師,明亮方追賊入子午谷,戰於張家坪,殲漢潮。師還,就逮,罪斬待決。

五年,上追錄前功,以領催詣湖北從陜甘總督松筠討賊,旋授藍翎侍衛、領隊大臣。敗賊石花街,遷二等侍衛。再敗賊斑竹園、遠安鎮,命以五品銜授宜昌鎮總兵。賊窺荊、襄,明亮與戰敗之。賊欲西走陜,明亮守七星關,賊復折而東,戰於硃家嘴,大破賊,進秩視三品。賊復入陜西境,明亮與巡撫倭什布合擊之,賊還南竄。上命赴四川討賊,明亮以陜西賊渠高二、馬五等將至竹谿,馳赴迎擊。上責明亮不即赴四川,復左授藍翎侍衛。明亮已擊破高二、馬五,復擢三等侍衛、領隊大臣。還師湖北,戰於壽陽坪,破賊渠徐添德,戰於獅子巖、佘家河,破賊渠茍文明,復授宜昌鎮總兵。時湖北賊漸定,上念明亮老,召還,授二等侍衛。

七年,自副都統外授烏魯木齊都統。三省教匪平,行賞,封一等男。九年,內授都統,遷兵部尚書。十年,進一等子。十四年,加太子少保,進三等伯。十五年,賜雙眼花翎,命協辦大學士。十六年,以輿夫聚博,上聞,不以實奏,左授副都統。十七年,出為西安將軍。十八年,內授都統、左都御史。十九年,復授兵部尚書、協辦大學士。二十二年,授武英殿大學士,進太子太保。二十四年,進三等侯。道光元年,致仕,食全俸。二年,卒,年八十七。宣宗親臨奠,賜陀羅經被。謚文襄,祀賢良祠。

論曰:福康安起戚里,然亦自知兵。征廓爾喀,賊守隘,命前軍更番與戰,而設伏隘側,前軍敗退,賊逐出隘,伏起,賊駭走,我軍蹙之入隘。福康安策騎督戰,諸軍悉度隘,遂夷賊屯。其才略多類此。士毅入安南,度重險,寀入其庭。是時諸將多驕侈,士毅獨廉,蓋亦有不可沒者。明亮知兵過福康安,廉侔士毅,師屢有功,輒有齮之者,未能竟其績。立朝既久,躬享上壽,進受封拜,非幸致也!


\end{pinyinscope}