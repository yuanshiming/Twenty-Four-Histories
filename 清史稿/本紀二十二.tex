\article{本紀二十二}

\begin{pinyinscope}
穆宗本紀二

六年丁卯春正月己未,任、賴諸匪竄孝感、德安,官軍失利,總兵張樹珊死之。壬戌,復靖遠。丙寅,革官文總督,召來京。以李鴻章為湖廣總督,調李瀚章為江蘇巡撫,以劉昆為湖南巡撫。己巳,張錫嶸剿捻匪於西安魚化鎮,死之。劉松山軍大捷。命喬松年專辦陜西軍務。辛未,命左宗棠為欽差大臣,督辦陜、甘軍務,賞劉典三品卿銜,幫辦軍務。乙亥,哈密回匪竄巴里坤,官軍擊退之。訥爾濟病免,以伊勒屯為巴里坤領隊大臣。丙子,命徐繼畬仍在總理各國事務衙門行走,管新設同文館事務。己卯,官軍復鎮雄。

二月乙酉朔,劉銘傳追剿任、賴於鍾祥,失利。鮑超進擊,大敗之。庚寅,命李鴻章督軍赴豫。壬辰,京師疫。甲午,擢劉松山為廣東陸路提督。丁酉,陜回馬生彥等降。減廣州屬徵收米折銀十九萬有奇,著為令。乙巳,桂軍復泗城。庚戌,以丁寶楨為山東巡撫。辛亥,洮州復陷。壬子,雲貴總督勞崇光卒,以張凱嵩代之。

三月丁巳,鄂軍剿賊於蘄水,失利,道員彭毓橘等死之。癸亥,總兵段步雲軍潰於鄜州。戊辰,鮑超累乞病,諭仍赴黃州。乙亥,命倭仁在總理各國事務衙門行走,辭,不允。丁丑,諭李云麟等安頓新疆難民。辛巳,曹克忠軍復洮州。壬午,回匪馬占鰲等犯西寧。是春,免浙江仁和等場被擾逋課、山西平定等處民欠倉穀。

夏四月丁亥,允琉球國子弟入監讀書。予鮑超病假。戊子,何琯軍復哈密。己丑,周祖培卒。癸巳,吉林馬賊平。丙申,日斯巴尼亞使來換約。壬寅,劉松山大破捻、回於同州。丙午,贈哈密殉難扎薩克郡王伯錫爾親王,建祠。德勒克多爾濟病免,命麟興為烏里雅蘇臺將軍,調榮全為參贊。丁未,瞻對番目大蓋折布伏誅。庚戌,貴德回匪叛,陷城。

五月甲寅,哈密回匪竄玉門,官軍擊退之。以旱,命恤難民、育嬰孩、揜暴露、贍陣亡者家屬。戊午,諭廣購書籍,並重刊御纂欽定經史,頒發各學。己未,郭寶昌、劉松山兩軍破張總愚於朝邑。免郭寶昌遣戍。辛酉,命曾國籓為大學士,駱秉章協辦大學士。丙寅,詔清理庶獄。丁卯,桂軍復荔波、義寧。戊辰,詔求直言,覈減宮廷用款。己巳,捻匪渡運河,予丁寶楨嚴議。庚午,賊竄長垣,官軍擊退之。癸酉,以剿賊無功,褫曾國荃頂戴,與李鶴年下部嚴議。諭李鴻章戴罪圖功。京師地震。庚辰,董福祥陷陜西甘泉。

六月甲申,總理各國事務衙門言俄人窺伺新疆,下大學士、尚書、左都御史會總理王大臣妥議。丙戌,申禁州縣浮收漕糧。甲午,倭仁乞病,罷職務,仍以大學士直弘德殿。乙未,官軍敗捻匪於即墨。庚子,順直久旱,饑,賑恤之。允鮑超回籍。辛丑,李鴻章檄劉銘傳、潘鼎新等軍防運河,扼膠、萊。命成祿節制黃祖淦、王仁和兩軍。以畿內亢旱,撥閩、廣、贛釐捐三十萬,浙、閩海關洋稅三十五萬備賑需。癸卯,甘回陷陜西華亭,旋復之。丁未,免昌平例貢果品。己酉,自三月不雨以來,上頻祈雨。至是日雨。是月,免陜西乾州等屬災擾額賦。

秋七月己未,雨。陜軍復甘泉。庚午,永定河決。己卯,以捻匪過膠萊河,諭各路扼守河、運兩防,奪丁寶楨職,仍留任。是月,免湖南晃州被擾逋賦。

八月丙戌,停奉天冬圍。戊子,湖北匪首劉漢忠伏誅。庚寅,命黎培敬會辦貴州剿撫及屯田事宜。壬辰,奉軍剿平孤山、法庫等處賊匪。辛卯,署貴州提督趙德光剿賊於安平,死之。丙申,穆隆阿等軍剿梟匪於文安,失利。濟陽土匪作亂,剿平之。丁酉,迤西回犯姚州。戊戌,貴州巡撫張亮基開缺嚴議,命曾璧光署之,布政使嚴樹森以逗遛褫職。壬寅,召陳國瑞來京。丙午,以淮、楚各軍所至騷擾,諭李鴻章嚴申軍律。己酉,裁熱河木稅。庚戌,創建福建船塢。

九月壬子,允左宗棠調曹克忠赴陜。丙辰,賴、任諸匪犯運河,牛師韓軍擊退之。丁巳,河、狄、西寧回眾投誠。庚申,停山東例貢。辛酉,安置額魯特游牧於額爾齊斯河。甲子,總理各國事務衙門言預籌修約事。諭曾國籓等各抒所見以聞。己巳,命丁日昌赴上海辦理義國換約。壬申,撫恤巫山被水災民。丁丑,命榮全與棍噶札拉參籌辦哈薩克剿撫機宜。己卯,命馮子材赴左江,專辦南、太軍務。賑襄陽等府災民。

冬十月癸未,諭各路統兵大臣及各督撫嚴申軍律。甲申,察哈爾都統色爾固善卒,以庫倫辦事大臣文盛代之。乙酉,以張廷岳為庫倫辦事大臣。丙戌,陜軍復寧條梁及宜君。飭席寶田軍赴沅州,統援黔軍務。壬辰,迤西回陷定遠、大姚。癸巳,汪元方卒。命沈桂芬在軍機大臣上學習行走。丙申,曾國荃病免,以郭柏廕為湖北巡撫,蘇鳳文為廣西巡撫。賑山東被水災民。乙巳,派美前使蒲安臣往有約各國辦理中外交涉。己酉,回匪陷寶雞、正寧,旋復之。

十一月庚戌朔,命道員志剛、郎中孫家穀往有約各國充辦理交涉事務大臣。壬子,劉銘傳等軍剿賊贛榆,大捷,任柱伏誅。癸丑,以梟匪蔓延,褫劉長佑職,仍責自效。命官文署直隸總督。丙辰,陜軍剿捻洛川,遇回匪,失利,提督李祥和死之。癸亥,張總愚陷延川、綏德。甲子,增設布倫托海辦事大臣,以李云麟為之,明瑤為幫辦,福濟為科布多幫辦。甲寅,劉銘傳軍剿賊於諸城,大捷。丁丑,陜軍復延川、綏德。

十二月壬午,張總愚竄吉州,左宗棠、趙長齡均褫職留任。成祿剿回匪於肅州,失利,總兵黃祖淦死之。癸未,賞陳國瑞頭等侍衛,隸左宗棠軍。劉銘傳等剿賊於壽光,大捷。迤西回陷祿豐、廣通、元謀。己丑,官軍復吉州。壬辰,直隸梟匪平。甲午,賞劉長佑三品頂戴,命率所部回籍。永定河堤工合龍。丙申,命蔣益澧以按察使候補,隸左宗棠軍,率楚勇回籍。丁酉,駱秉章卒。劉松山等敗張總愚於洪洞。調吳棠為四川總督,以馬新貽為閩浙總督,李瀚章調浙江巡撫,丁日昌為江蘇巡撫。戊戌,淮軍剿賊高郵大捷,獲賴文光等,誅之。辛丑,東捻平,加賚李鴻章、曾國籓世職,賞劉銘傳、英翰及郭松林、楊鼎勛、善慶世職有差,復曾國荃頂戴。壬寅,以左宗棠督師入晉,命庫克吉泰、喬松年、劉典督辦陜西軍務。甲辰,命楊占鰲署甘肅提督,接辦西路軍務。戊申,左宗棠檄喜昌、劉松山等赴磁州迎剿。諭張曜、劉銘傳等會剿。己酉,命鄭敦謹往山西查辦事件。是月,免浙江仁和等場未墾灶課、雲南嵩明等屬歉收額糧。

是歲,朝鮮、琉球入貢。

七年戊辰春正月庚戌朔,捻首李允等率眾降於盱眙,詔誅之,遣散餘眾。命硃鳳標協辦大學士。乙卯,回匪復陷正寧。丙辰,喜昌等擊張總愚於河內,大捷。西寧回陷北川。李云麟乞病。不許。以錫綸為布倫托海幫辦大臣。辛酉,張總愚北竄定州,保定戒嚴,官文、左宗棠均褫職留任。諭玉亮統神機營兵剿賊。壬戌,張總愚犯清苑,劉松山、郭寶昌等軍繞賊前剿之,予優敘。陳國瑞、宋慶、張曜均以軍至保定。達賴請宥里塘犯東登工布死罪,允之。命賈楨等設團防總局。癸亥,諭令天津洋槍、練軍各隊赴河間,與山東軍聯絡防剿。甲子,李鴻章遣周盛波等軍北援。趣左宗棠赴保定北方督剿。命恭親王會同神機營王大臣辦巡防。壬申,允英翰入衛畿疆,命統牛師韓軍駐黃河以南。飭程文炳軍赴河間會剿。癸酉,張總愚陷饒陽,旋復之。賈楨以病致仕。乙亥,命左宗棠總統各路官軍。

二月辛巳,官軍復渭源。癸未,命恭親王節制各路統兵大臣。戊子,回匪復陷寧條梁。己丑,回匪竄伊克沙巴爾,官軍擊退之。褫趙長齡、陳湜職,遣戍。壬辰,陜軍復寶雞。癸巳,滇軍解鎮雄圍。迤西回陷楚雄。乙未,豫、皖各軍敗張總愚於束鹿。庚子,左宗棠、李鴻章等軍剿賊,迭破之。回匪陷懷遠、神木。壬寅,白泥苗匪降。乙巳,以朝鮮請嚴邊禁,命延煦、奕榕赴奉天,會都興阿勘展邊事宜。

三月壬子,張凱嵩乞病,諭責其逗留規避,褫職。回匪陷鄜州,劉典駐三原督剿。癸丑,以劉岳昭為雲貴總督,岑毓英為雲南巡撫。乙卯,陜軍復鄜州。癸亥,諭庶吉士散館仍試詩賦。戊辰,張總愚竄延津、封丘,劉松山、郭寶昌擊敗之。辛未,命沈桂芬為軍機大臣。乙亥,命硃鳳標為大學士。丙子,迤西回陷易門。丁丑,張總愚竄滑縣,擊敗之。是月,免直隸安州等處澇地逋賦。

夏四月己卯朔,哈密回陷五堡,官軍擊退之。甲申,張總愚陷南皮。丁亥,諭左宗棠、李鴻章、丁寶楨等,督各軍於運河東西分路防剿。己丑,苗匪何正觀降。庚寅,陜軍剿回匪於邠州,失利,譚玉龍死之。己巳,永定河決。乙未,召都興阿來京。戊戌,黎平苗犯晃、沅各境,官軍擊退之。辛丑,寧條梁回擾鄂爾多斯游牧,貝子札那格爾濟擊退之。回匪犯哈密,伊勒屯等會擊退之。癸卯,賜洪鈞等二百七十人進士及第出身有差。是月,免四川各土司三年租賦。

閏四月戊申朔,迤西回竄陷昆陽、新興、晉寧、呈貢、嵩明。戊午,回匪復陷神木。癸亥,陜軍復延長。甲子,董福祥投誠,諭立功自贖。乙丑,回匪踞烏紳旗,分擾準噶爾旗,偪托克托城。丁卯,程文炳、陳國瑞、劉松山等軍擊張總愚於高唐、茌平、博平,大捷。賊竄東光。己巳,回匪再陷慶陽及寧州、合水,知縣楊炳華死之。辛未,命都興阿為欽差大臣,會同左宗棠、李鴻章剿捻、調遣春喜、陳國瑞、張曜、宋慶四軍,崇厚幫辦軍務。

五月戊寅,劉松山等軍剿張總愚於鹽山、海豐,大捷。己卯,創設長江水師,置岳州、漢陽、湖口、瓜州四鎮總兵官。癸未,陜軍擊退竄邠、鳳回匪。壬辰,北山土匪犯延安,官軍失利,副將劉文華等陣沒。庚子,滇軍復元謀、武定、祿勸、羅次。是月,免湖南晃州被擾逋賦。

六月己未,郭松林等剿捻於臨邑、濱州、陽信,大捷。諭水師嚴扼運防。辛酉,桂軍復歸順。癸亥,金匪竄寧古塔界,官軍剿平之。甲子,陜軍克宜川。丙寅,張總愚犯運河岸,官軍擊敗之,捻眾多降。戊辰,又擊之於商河,大捷。乙亥,李云麟褫職查辦。命明瑤為布倫托海辦事大臣。浙江海塘工竣。

秋七月丁丑,蠲直、魯、豫被擾各州縣田賦。己卯,春壽以欺飾褫職。壬午,撫恤滄州等處被擾難民。乙酉,張總愚赴水死,捻匪平。加李鴻章、左宗棠太子太保銜,鴻章以湖廣總督協辦大學士,丁寶楨、英翰、崇厚並加太子少保銜,復官文銜翎,晉劉銘傳一等男,郭松林一等輕車都尉,賞宋慶、善慶二等輕車都尉,劉松山黃馬褂、三等輕車都尉,郭寶昌、張曜、溫德勒克西騎都尉,黃翼升加一雲騎尉,復陳國瑞提督世職,餘升敘有差。命惇親王祭告定陵。允彭玉麟回籍終制。丙戌,召左宗棠、李鴻章入覲。丁亥,滎澤河決。辛卯,毛昶熙言軍務漸平,宜益思寅畏,旋御史張緒楷疏請保泰持盈,及時講學,並嘉納之。壬辰,允左宗棠請,資遣降眾回籍。癸巳,武陟沁河堤決。乙未,調曾國籓為直隸總督,馬新貽為兩江總督,以英桂為閩浙總督。命彭玉麟赴江、皖會籌長江水師事宜。戊戌,諭蘇、皖、豫、魯各屬修圩寨,飭鄉團。庚子,予宋儒袁燮從祀文廟。援黔川軍復龍里、貴定。川軍剿越巂夷匪,勝之,俘其酋勒烏立。授曾璧光貴州巡撫。辛丑,布倫托海變民竄烏隴古河。德勒克多爾濟卒。癸卯,撫恤滎、鄭災民。甘回擾白水、郃陽,陜軍擊退之。甲辰,援黔湘軍復甕安。

八月乙巳朔,褫御史德泰職,以奏請修理園庭也。庫守貴祥妄陳希利,發黑龍江為奴。永定河決。己酉,諭明瑤等規復布倫托海舊制。命馬新眙兼辦理通商事務大臣。壬子,延安土匪扈彰降。癸亥,諭左宗棠兼顧山西軍務。戊辰,諭吉林嚴定開墾圍荒界限。辛未,諭金順專辦援陜軍務。是月,免皖、蘇、魯、豫、鄂被擾積年逋賦。

九月壬午,官軍復慶陽。甲申,肅州回攻敦煌,官軍擊退之。諭伊勒屯等籌辦巴里坤屯田。乙酉,援黔川軍會復平越。辛卯,命延煦出關查辦奉天展邊事宜。癸巳,滇軍復晉寧、呈貢。是月,免浙江橫浦等場歉收灶課。

冬十月丁未,回匪犯涇州、靈臺,擊退之。乙卯,文麟抵哈密,諭興辦蔡巴什湖等處屯田。丙辰,穆圖善克河州。賑濟南、武定水災。丁巳,戍李云麟黑龍江。戊午,命李鴻藻仍直弘德殿及軍機。庚申,以守科布多功,加土爾扈特郡王凌札棟魯布親王銜。己巳,黔苗復陷興義,旋復之。

十一月甲戌,援黔川軍復麻哈。丁亥,涼州總兵周盛波以不戢所部,褫職。回匪擾鄂爾多斯等旗,竄榆林。諭定安等截剿。壬辰,諭除吏胥積弊。己亥,黔軍克都勻,賞張文德黃馬褂。庚子,臺灣英領事縱洋將掠船,踞營署,焚局庫,勒兵費。諭總署詰辦,飭英桂等遴員交涉。壬寅,熱河匪平。免吉林雙城堡被水屯田租賦。

十二月甲辰朔,川軍剿西昌夷匪,連捷,各夷部降。援黔湘軍復天柱。丙午,回匪犯包頭,蒙軍失利。丁未,熱河匪首彌勒僧格伏誅。甲寅,以曾國籓言川私病楚,諭籌止川鹽濟楚章程,撤局停稅。丁巳,滇軍復澂江。庚申,申諭各省禁種罌粟。壬戌,黔苗竄擾河池,官軍擊退之。乙丑,諭朝審緩決三次以上者並減等。永定河工竣。戊辰,麒慶罷,以慶春為熱河都統。庚午,劉松山剿賊大理川,大捷。壬申,截鄂餉二十一萬賑河南災。是月,免江蘇荒地糧賦,山東泰安、河南汝寧等屬被擾逋糧。

是歲,朝鮮入貢。

八年己巳春正月癸酉朔,停筵宴。丁丑,川、湘、黔、桂各軍會剿苗匪,黔軍復長寨。戊寅,滇軍克富民。己丑,劉松山等軍擊土、回各匪,敗之於清澗。成祿克肅州,與楊占鰲並賞黃馬褂。甲午,滎工合龍。丙申,劉松山軍敗賊於靖邊,董侍有等以鎮靜堡及靖邊降。迤西回犯昆明,岑毓英等擊退之。辛丑,雷正綰克涇州董家堡。

二月戊申,命袁保恆督辦西征糧饟。

三月癸酉朔,林自清戕興義知縣,提督陳希祥誘誅之,賞希祥黃馬褂。甲戌,援黔湘軍復鎮遠府、衛兩城。己卯,甘肅提督高連升部兵變,戕連升,部將周紹濂擊逆黨於同官,殄之。乙酉,諭督撫于克復州縣慎選牧令,拊循流亡。庚寅,回匪陷磴口。甲午,吐魯番回匪犯哈密,官軍迭敗之。乙未,桂軍克憑祥。己亥,懿旨,大婚典禮,力崇節儉。

是春,免江蘇山陽、直隸安州等屬災、擾額賦,兩淮富安等場逋欠灶課。

夏四月癸卯朔,迤西回陷楊林營,劉岳昭退守曲靖,嚴責之。乙巳,麟興以畏事褫職。以福濟為烏里雅蘇臺將軍,文碩為布倫托海辦事大臣。己酉,雷正綰、黃鼎軍復鎮原、慶陽。援黔川軍復甕安。己未,援黔湘軍會復清江。庚申,允劉銘傳乞病。辛酉,免陳湜遣戍。是月,免山東東昌等屬逋賦。

五月庚辰,援黔湘軍復施秉,進攻黃飄賊壘,失利,按察使黃潤昌、道員鄧子垣、提督劉長槐死之。壬午,回匪陷澂江。甲申,杜嘎爾等軍大破賊於杭錦旗。辛卯,命李鴻章赴四川察辦吳棠劾案。申誡岑毓英任用通賊練目,苛斂民捐。以馬如龍為雲南提督。丙申,官軍剿匪於保安,大捷,匪首袁大魁等伏誅。自春正月不雨至於是月,上頻禱祈。丁酉,雨。

六月辛亥,援軍會克尋甸。壬子,命董恂、崇厚辦理奧斯馬加換約。甲寅,永定河決。戊午,予黃飄死事提督榮惟善、總兵羅志宏等世職加等。辛酉,武英殿災。癸亥,倭仁、徐桐、翁同龢請勤修聖德,以弭災變,上嘉納之。丙寅,諭督撫考課農桑。庚午,回匪犯阿拉善定遠營,蒙兵失利。

秋七月辛未朔,日有食之。癸酉,張曜等軍敗回匪於察漢淖爾。命吳坤修赴沿江各屬撫恤災民。甲戌,滇軍復嵩明,克白鹽井。甲申,桂軍會越南軍克九葑、洛陽等隘。乙酉,諭錫綸賑恤額魯特人眾。丙戌,朝鮮請鴨綠江北禁游民建屋墾田。趣都興阿等妥辦。壬辰,何琯軍敗賊於木壘河等處。是月,免晃州被擾逋賦。

八月庚子朔,俄商船泊呼蘭河口,求吉、黑內地通商,諭總署按約止之,禁軍民私與貿易。癸卯,內監安得海出京,丁寶楨奏誅之。黔匪復陷都勻。丙午,桂軍會復越南高平。庚戌,申諭約束太監。壬子,官軍剿平杭錦旗屬竄回。癸丑,寧夏官軍剿賊失利,副將方大順陣亡。戊午,棍噶札拉參軍復布倫托海,賊首張匊等伏誅。己未,官軍剿達拉特旗竄匪,殄之。是月,賑浙江杭、湖各屬,湖南安鄉等縣水災。

九月庚午,高臺勇潰,褫成祿職,留任。壬申,撥京餉三十萬濟武、漢等屬工賑。甲戌,馬化龍復叛,襲陷靈州。官軍復威戎堡、水洛城。戊寅,滇軍復易門。壬午,免暹羅補歷年貢品。庚寅,烏魯木齊匪竄哈密,何琯等擊敗之。乙未,福建新造第一輪船成,命崇厚勘驗。戊戌,諭福濟等額魯特各安舊居,僧眾居阿爾泰山南,俗眾居青格里河。

冬十月庚子,劉松山敗回匪於吳忠堡等處。辛丑,金順又敗之於納家徬。命楊占鰲署甘肅提督,辦肅州善後事宜。法使羅淑亞與其水師提督以兵船赴贛、鄂、川省查教案,諭所在按約待之。乙巳,雷正綰、黃鼎敗回匪於固原、鹽茶。丁未,命毛昶熙、沈桂芬在總理各國事務衙門行走。辛丑,命文碩等會勘布倫托海分界事宜,董恂辦理美國換約。甲寅,滇軍復楚雄、南安、定遠。劉岳昭移軍昆明。己未,哈密官軍剿西路回匪,大捷。甲子,鳳凰城匪首王慶等伏誅。乙丑,劉松山軍復靈州。是月,賑雲南水災,直隸旱災。

十一月丙子,茌平教匪孫上汶等謀逆,捕誅之。丁丑,裁新設布倫托海辦事大臣。庚辰,賑江寧水災。癸未,免科布多屬貢貂。甲申,滇軍復昆陽。丙戌,甘軍復靖遠。庚寅,永定河口合龍。乙未,命文碩來京,改奎昌辦理分界。是月,免直隸東明被淹、被擾,安徽無為等州縣衛被水逋賦。

十二月庚子,援滇川軍克魯甸。乙巳,劉松山軍攻金積堡,總兵簡敬臨等死之。乙卯,披楞侵占哲孟雄各地,廓爾喀與唐古特構嫌,諭恩麟防維開導。布魯克巴內閧,並諭恩麟解釋撫綏。丁巳,越南匪平。諭蘇鳳文嚴申邊禁。癸亥,賑畿南災。

是歲,朝鮮、越南、琉球入貢。

九年庚午春正月丁卯朔,停筵宴。癸酉,滇軍復祿豐。甲戌,甘軍擊敗援賊於王家甿。己卯,回匪陷定邊。癸未,神武門木庫火,詔修省。庚寅,回匪陷安定。陜軍復定邊。甲午,馬德昭留辦潼關防務。

二月辛丑,劉松山督剿金積堡回匪,中砲卒。賞道員劉錦棠三品卿銜,接統其軍。以俄官往齊齊哈爾、吉林商界務,諭富明阿、德英據約待之,毋遷就。乙巳,回匪分竄安邊、清澗,陜軍擊走之。丙午,又分竄花馬池、榆林,宋慶軍剿之。戊申,官軍擊敗米脂竄匪。壬子,命李鴻章赴陜西督辦軍務。甲寅,回匪竄同官、宜君,陜軍剿敗之。丙辰,法使因教案藉兵要挾,諭各疆吏通商大臣迅結交涉事宜。辛酉,寧夏各堡降回復叛。

三月丁卯朔,回匪竄準噶爾旗,馬玉昆擊敗之。辛巳,雷正綰以疏防峽口,褫職留營。諭誡西征各軍貪功銳進。乙酉,滇軍復彌渡、賓川、麗川、緬寧。辛卯,回匪分擾岐、鳳,李輝武擊敗之。

夏四月甲辰,譚廷襄卒。

五月庚午,命崇實赴貴州,會同曾璧光查辦教案。癸酉,始允英國設置沿海各口電線。甲戌,援黔川軍克黃飄、白堡等苗寨。庚寅,天津人與天主教啟釁,焚毀教堂,毆斃法領事。命曾國籓與崇厚會商辦理。乙未,諭疆吏飭禁播謠惑眾,保護通商傳教各區。李鴻章督軍入關,請調郭寶昌軍,允之。命崇厚為出使法國大臣。以成林署三口通商大臣。是月,免直隸安州等屬逋賦。

六月戊戌,奎昌赴塔爾巴哈臺,與俄使勘辦立界。壬寅,賽音諾顏部蒙兵剿回匪失利。丁未,滇軍復威遠。己酉,命彭玉麟赴江南,會同沿江督撫整頓長江水師。庚戌,甘軍敗回匪於鞏昌。乙卯,永定河決。庚申,以疏防民教啟釁,褫天津知府張光藻、知縣劉傑職,下部治罪。辛酉,滇軍復姚州。癸亥,命毛昶熙會同曾國籓查辦教案。曾國籓言:「善全和局,為保民之道。備御不虞,為立國之基。」諭旨嘉勉。命丁日昌赴天津幫辦洋務。

秋七月戊辰,以琿春邊務事繁,加協領副都統銜,為定制。丙子,法使羅淑亞以曾國籓不允府、縣論抵,回京。諭國籓迅緝原兇,從速辦結。丁丑,召崇厚還。命毛昶熙署三口通商大臣。甲申,周盛傳等剿散北山餘匪。丙戌,諭曰:「海上水師,與江上水師截然不同。欲捍外侮圖自強,非二十年之久,未易收效。然因事端艱鉅,畏縮不為,則永無自強之日。近年內外臣工,值事急時,徒事張皇。禍患略平,又為茍安之計。即創立戰守章程,而奉行不力,使朝廷謀議均屬具文。積習因循,焦憂曷釋。茲閩、滬兩廠輪船告成,馬新貽、丁日昌、英桂、沈葆楨各擇統將出洋,窮年練習,以備不虞。廣東亦應籌備輪船,瑞麟、李福泰務切實辦理。將校有熟諳風濤沙線者,隨時擇保,即山野中或長於海戰,亦當隨時物色,量材超擢。各督撫其統籌全局,以副委任。」庚寅,南路甘軍復渭源、狄道。是月,免晃州被擾逋賦。

八月丁酉,汝陽人張汶祥刺殺馬新貽。命曾國籓為兩江總督,李鴻章調直隸總督,李瀚章為湖廣總督。戊戌,設黃河水師。庚子,北山匪首李凡覺伏誅。壬寅,命張之萬會同魁玉訊張汶祥。己酉,召毛昶熙還。命李鴻章會曾國籓查辦天津教案。癸丑,桂軍剿平安邊、河陽賊匪,梁添錫伏誅。允越南進方物及馴象。己未,命李成謀為新設輪船統領。

九月戊辰,滇軍復新興。庚午,諭崇實仍赴遵義辦教案。甲戌,治天津民教啟釁罪,張光藻、劉傑遣戍,誅逞兇殺害之犯十五人。

是秋,川東、荊州、熱河被水,賑撫之。

冬十月乙未,沈葆楨丁憂,命百日後仍經理船政。丙申,命劉銘傳督辦陜西軍務。諭嚴禁四川州縣苛派。撥款續賑北山難民。辛丑,以江北漕船阻淺,由陸路轉運臨清。甲辰,天津制造局成。庚戌,日本請立約通商,允總署遴員議約。辛亥,免科布多貢貂。壬子,裁三口通商大臣,命直隸總督經理,如南洋大臣例,給欽差大臣關防。嚴諭疆吏慎密交涉,有漏洩者立誅之。丙辰,以水旱疊見,詔修省。戊午,移周盛傳軍衛畿輔。陜回禹生彥等竄平番,官軍失利,提督張萬美等死之。庚申,設直隸津海關道。劉錦棠各軍克漢伯等堡,合圍金積堡。

閏十月乙丑,俄使倭良嘎哩來京。庚午,湘潭會匪平。乙亥,滇軍復永北、鶴慶、鎮南、楚雄。回匪陷烏里雅蘇臺。丙子,永定河合龍。諭曾國籓籌河運。戊寅,越南吳亞終等伏誅。

十一月癸巳,命鄭敦謹會鞫張汶祥獄。尋定讞,磔張汶祥於江寧。丁酉,回匪竄涼州,副將謝元興陣沒,王仁和擊退之。辛丑,援黔湘軍復臺拱。戊申,福濟、榮全以匪入烏里雅蘇臺,褫職留任。命曾國籓兼通商大臣。庚戌,甘肅總兵周東興侵賑,命斬於軍前。庚申,劉坤一以漏洩密諭,褫職留任。

十二月甲子,諭嚴禁河工偷減侵蝕諸弊。辛未,滇軍復鄧川、浪穹。回目馬源發戕提督丁賢發等,捕誅之。

是冬,免貴州興義等州縣衛、陜西綏德等州縣災擾逋賦。

是歲,朝鮮入貢。

十年辛未春正月辛卯朔,停筵宴。壬辰,官軍克河西王甿賊壘,賞金順黃馬褂,加張曜一雲騎尉。乙未,黔軍平貴定等處賊壘,克都勻,賞提督林從泰、總兵何雄輝黃馬褂。己亥,諭馮子材赴太平進剿牧馬、諒山匪。壬寅,官文卒。是月,免直隸安州等屬被水額賦。

二月壬戌,劉錦棠等軍克金積堡,匪首馬化龍等伏誅,加左宗棠一騎都尉,賞劉錦棠雲騎尉、黃馬褂,開復雷正綰處分,及陳湜原官,賞黃鼎、金運昌黃馬褂。置就撫陜回於華亭之化平川,設通判、都司以綏靖之。前知靈州彭慶章坐為賊主謀,處斬。壬午,獲叛將宋景詩,誅之。丁亥,調江蘇按察使應寶時赴津,籌辦日本通商事。命瑞常為大學士,文祥協辦大學士。

三月癸巳,金順等軍克寧夏,匪首馬萬選伏誅。己丑,滇軍復澂江,克江那土城,匪首馬和等伏誅。辛丑,普使李福斯致國書,以德意志各國及自主之三漢謝城共復一統,受尊稱為德意志皇帝,復書賀之。丁未,以倭仁為文華殿大學士,瑞常為文淵閣大學士。自春初至於是月,上連祈雨。庚戌,雨。

夏四月丙寅,援黔湘軍復新城、巖門司等城,克高坡等苗寨。己巳,寧夏納家徬回眾降。己卯,陜回竄擾平番、碾伯,官軍擊退之。辛巳,倭仁卒。甲申,賜梁耀樞等三百二十三人進士及第出身有差。築大沽、北塘砲臺。乙酉,福濟革職,以金順為烏里雅蘇臺將軍。丙戌,回匪復竄擾賽音諾顏部,焚掠固爾班賽汗等處。

五月庚寅朔,雨。乙未,左宗棠請禁絕回民新教,不許。戊戌,苗酋聞國興等降,八寨等城俱復。壬寅,回匪擾烏拉特,杜嘎爾、薩薩布軍合擊之。丙午,援黔湘軍復丹江、凱裡等城,賞蘇元春黃馬褂。己酉,以李世忠尋仇鬥很,陳國瑞演劇生事,褫世忠職,降國瑞都司,並勒回籍,畀有司管束。辛亥,鄭親王承志有罪,褫爵逮訊。命李鴻章辦日本商約,應寶時、陳欽為幫辦。乙卯,金順乞假守制葬親。不許。己未,滇軍復云龍。

六月壬戌,太白晝見。益陽等處會匪平。己巳,陜回白彥虎結西寧回眾擾河州。庚午,黔軍克永寧、鎮寧、歸化苗寨,破郎岱、水城各峒寨。乙亥,命瑞麟為大學士,仍留兩廣總督任。己卯,阜陽匪擾沈丘、汝陽,官軍捕誅之。辛巳,以廣東盜賊橫行,諭飭嚴緝。丁亥,德宗生於醇邸。戊子,賑天津災。

秋七月己丑朔,桂軍剿越南竄匪,克長慶,斬匪首趙雄才。壬辰,杜嘎爾軍剿賊於布拉特,勝之。甲午,永定河復決。丙申,穆圖善赴北山剿賊。金運昌軍剿烏拉特竄匪,勝之。丁未,河內沁河決。乙卯,昌圖賊匪竄擾,都興阿遣軍剿平之。

八月壬申,以副都統慶至襲封鄭親王。甲戌,桂軍克安世賊寨,追剿太原竄匪,蘇國漢赴廣東乞降。丁丑,詔各省設局收養流寓孤寡。

九月丙申,革高郵徵糧弊習。丁酉,甘軍克康家崖要隘。趣榮全赴伊犁。給劉銘傳假三月。壬寅,諭奉、吉整頓吏治,嚴緝盜賊。命恩錫往上海辦奧國換約。丁未,喬松年等會堵侯家林決口。

是秋,賑順直各屬及菏澤等州縣災,免濮州被水、晃州被擾逋賦。

冬十月戊午朔,達爾濟以撤營縱賊,褫職逮治。命曹克忠接統劉銘傳軍,赴肅州防剿。庚申,以湖南匪變,命李鴻章查辦。壬辰,命景廉為烏魯木齊都統。癸未,詔免伊犁被脅官吏軍民等罪。以參領貢果爾接統達爾濟軍。

十一月癸巳,甘軍克河州,禹得彥等降。丁未,西寧回匪竄烏拉特及中衛,張曜軍擊退之。乙卯,肅州回匪復犯敦煌,文麟援剿之。

十二月辛未,予先儒張履祥從祀文廟。丁丑,香山匪徒曾大鵝幅等作亂,捕誅之。

是歲,朝鮮、琉球、越南入貢。

十一年壬申春正月丙戌朔,停筵宴。己丑,以紀年開秩諭減刑。文碩以乞病褫職。辛卯,桂軍復越南從化,克鎮山。癸巳,甘軍連破甘坪、大貝坪等處賊壘,進攻太子寺。庚子,黔軍克清平、黃平、重安。辛丑,援黔湘軍克黃飄、白堡苗寨。辛亥,命侍郎崇厚、太常寺少卿夏家鎬在總理各國事務衙門行走。

二月庚申,允江蘇辦米試行河運,漕白二糧仍由海運。丙寅,曾國籓卒,贈太傅。戊辰,褫劉銘傳職,以前功仍留一等男爵。庚午,起彭玉麟巡閱長江水師。甲申,侯家林決口合龍。越南匪首蘇國漢等伏誅。是月,賑四川各屬災。

三月乙酉朔,黔軍復貞豐。丙戌,甘軍剿太子寺回匪失利,提督傅先榮、徐文秀死之。褫提督楊世俊黃馬褂,降參將。甲午,免達爾濟等罪,仍褫職效力。丁酉,以奉匪擾朝鮮境,嚴緝之。辛丑,瑞常卒。

是春,免湖北黃陂、直隸安州、甘肅河州等處被擾逋賦。

夏四月丙辰,回匪竄定邊、靖邊,陜軍擊退之。己未,西寧回目馬占鰲、陜回崔三、米拉溝回目治成林等,先後乞降。丙寅,停淮關傳辦活計。諭內務府力求撙節。己卯,通政司副使王維珍疏陳先意承志,孝思維則。予嚴議,尋褫職。是月,免貴州興義等屬被擾逋賦。

五月甲申朔,日有食之。免熱河騰圍旗民租課三年。乙酉,自三月初旬,慈禧太后弗豫,月餘不視朝。至是,御史李宏謨請勤召對。諭責其冒昧,嚴飭之。癸巳,徐占彪軍剿肅回屢捷。左宗棠劾成祿糜帑遷延,命穆圖善查辦。乙未,貴州苗匪平,賞席寶田騎都尉。丙申,陜回宋全德等降。予伊犁殉難已革尚書陳孚恩暨其眷屬旌恤加等。庚子,命李鴻章為大學士,仍留直隸總督任。乙巳,滇軍克永平及雲南。

六月甲午,硃鳳標致仕。命文祥為大學士,全慶協辦大學士。丁卯,諭停本年秋審、朝審句決。以單懋謙協辦大學士。

秋七月癸未朔,滇軍會克興義。己丑,免廓爾喀例貢。賑達木蒙古及三十九族被災兵民。戊戌,回匪竄擾寧夏西路及阿拉善旗,官軍擊退之。己亥,直隸呈進瑞麥,御史邊寶泉疏論之。諭李鴻章勤恤民隱,補救偏災,毋鋪張瑞應。庚子,永定河北下汛溢。是月,免湖南晃州被擾逋賦。

八月庚午,截江北漕米十萬餘石賑畿輔被水災民。癸酉,金順以遷延罷,常順署烏里雅蘇臺將軍。辛巳,以單懋謙為大學士。

九月癸未,滇軍克趙州、蒙化並大理上下關,賞楊玉科、李維述黃馬褂。左宗棠言地產瑞麥瑞穀,諭卻之。乙未,冊立皇后阿魯特氏,自王大臣以次推恩加賚,頒詔天下,覃恩有差。永定河工合龍。丙午,允彭玉麟乞病回籍,仍命每年巡閱長江水師。庚戌,榮全請令慶符招撫纏、民,英廉等馬隊駐庫爾喀喇烏蘇,酌募民勇,允之。

十月丁巳,甘肅潰勇首犯馮高山等伏誅。己未,加上兩宮皇太后徽號。戊辰,廣西隆安、岑溪土匪,西隆苗匪平。壬寅,諭統兵大臣約束委員,治騷擾逾限者罪。允恭親王請,復軍機處舊制。丙子,何璟憂免,以張樹聲署兩江總督。

十一月乙酉,朝鮮匪船越境侵擾,都興阿等水師緝剿之。回匪擾哈密東山,官軍剿勝之。禁殿廷、鄉、會考試請託冒替。己卯,瓊州土匪平,誅匪首何亞萬等。辛卯,滇軍剿館驛等踞匪,迤東、迤南肅清。乙未,肅回竄扎薩克汗各旗,官軍擊走之。黔軍會克新城。下江苗匪亂,張文德軍剿除之。全黔底定。丙申,捻匪竄擾太湖,水師剿平之。允軍民入哥老會者自首免罪。丁酉,申禁各省種罌粟。辛丑,劉錦棠等軍剿回匪,大捷。丁未,陜軍剿陜北二道河等處竄匪,殄之。李鴻章奏設招商局,試辦輪船分運江、浙漕糧。

十二月己未,駐藏幫辦德泰坐事褫職回旗。丙辰,諭吏部、兵部、理籓院,親政後,各署有請旨及軍務摺片,均用漢文。丁卯,釋田興恕回。丙子,左宗棠乞病,溫旨不許,己卯,祫祭太廟。

是歲,朝鮮入貢。

十二年癸酉春正月辛巳朔。癸未,官軍擊回匪於那瑪特吉幹昭,敗之。丙戌,以李宗羲為兩江總督,兼通商大臣。辛丑,成祿以苛捐誣叛,褫職逮問,趣金順接統其軍。甲辰,滇軍克大理,回酋杜文秀、楊榮、蔡廷棟等伏誅。賞岑毓英黃馬褂、騎都尉世職,開復劉岳昭處分,賞楊玉科騎都尉。乙巳,兩宮皇太后以親政屆期,懿旨勉上「祇承家法,講求用人行政,毋荒典學」。勖廷臣及中外臣工「公忠盡職,宏濟艱難」。丙午,上親政,詔「恪遵慈訓,敬天法祖,勤政愛民」。己酉,諭內務府核實撙節,於歲費六十萬外,不得借支。

二月庚戌朔,軍機大臣、六部九卿會議黃、運兩河辦法。諭李鴻章悉心籌辦奏聞。下詔修省,求直言。諭直省舉賢才,杜侵蠹。戊午,加上兩宮皇太后徽號,翌日頒詔覃恩有差。劉錦棠軍克大通向陽堡。庚午,以謁東陵,命惇親王等留京辦事。乙亥,金順軍抵肅州剿回匪,敗之。

三月癸未,上奉兩宮皇太后謁東陵。丁亥,回鑾。免蹕路經過本年額賦。己丑,大通、巴燕戎格及五工撒拉各回眾降。西寧匪首馬桂源等伏誅。庚寅,上奉兩宮皇太后還宮。丙申,回匪白彥虎等竄甘州。命議定各國公使覲見禮節。榮全乞病,不許。庚子,以英廉為塔爾巴哈臺參贊大臣。丁未,滇軍克順寧。

是春,免江蘇邳州、陜西鄜州等屬被擾逋賦。

夏四月乙卯,設廉州北海關。丙辰,日本換約成。乙丑,回匪竄阿拉善旗及阿畢爾米特,諭定安遣軍會防兵夾擊。己巳,官軍克肅州塔爾灣賊巢。

五月庚寅,滇軍克雲州。丁酉,允各國公使覲見。癸卯,成祿交刑部治罪。丙午,命成瑞署烏魯木齊提督。

六月壬子,上幸瀛臺,日使副島種臣、俄使倭良嘎哩、美使鏤斐迪、英使威妥瑪、法使熱福哩、荷使費果蓀覲見於紫光閣,呈遞國書。庚申,嚴趣金順出關。丁卯,甘軍復循化,匪目馬玉連等伏誅。

閏六月甲申,李鴻章覆陳黃、運兩河淮、徐故道難復,請仍海運。其舊河涸地,酌量升課。議行。丙戌,硃鳳標卒。滇軍克騰越,予岑毓英一等輕車都尉,賞劉岳昭黃馬褂,楊玉科一等輕車都尉。以雲南軍興十有八年,郡縣多為賊蹂躪,詔免十一年以前積欠糧賦,並永遠停徵濟軍釐穀。諭劉岳昭慎選牧令,察吏安民。甲午,京畿久雨,上祈晴。丙申,詔查各省畝捐、釐捐及丁漕違制者,次第豁除。庚子,甘軍剿白彥虎等於敦煌,失利,副將李天和等死之。永定河決。免阿爾泰烏梁海七旗貢貂。

秋七月辛亥,桂軍剿西林、西隆匪,平之。甲子,賑順天災。是月,免山東青城被水新舊額賦。

八月丁丑朔,都興阿乞病,慰留之。辛巳,直隸運河堤決。榮全復以病乞免,不許。富和有罪褫職。戊子,白彥虎等陷馬蓮井營堡。召劉岳昭入覲,以岑毓英兼署云貴總督。壬辰,白彥虎等圍哈密,犯巴里坤,官軍失利。乙未,諭景廉督軍赴援,調錫綸為烏魯木齊領隊大臣,以明春為哈密幫辦大臣。是月,賑直隸各屬、永順府屬暨公安水災。

九月丙寅,命軍機大臣會刑部審擬成祿罪。癸酉,永定河合龍。

冬十月丙子朔,御史沈淮疏請緩修圓明園。諭令內務府僅治安佑宮為駐蹕殿宇,餘免興修。己亥,官軍克肅州,匪逆馬文祿伏誅。上詣兩宮皇太后賀捷。庚子,論功,命左宗棠以陜甘總督協辦大學士,加一等輕車都尉;復金順職,賞還黃馬褂;予徐占彪、穆圖善雲騎尉。

十一月己未,越南王疏請會剿河陽、興化、山西、宣光邊地諸匪。諭劉長佑、馮子材議奏。辛酉,法、越構釁,法兵破河內省城,越匪擾北寧。越人求援。諭瑞麟飭軍由欽州出關,會桂軍援剿之。甲子,御史吳可讀請將成祿明正典刑。己巳,岑毓英奏整頓吏治營伍,並請撤勇停捐,自雲南始。詔嘉之。庚午,疏濬運河。壬申,成祿論斬。吳可讀坐刺聽朝政降調。

十二月甲申,回匪竄擾烏梁海等部,錫綸軍追剿,敗之。戊子,以磨勘順天舉人徐景春試卷荒謬,考官尚書全慶、都御史胡家玉等降黜有差。辛卯,命額勒和布赴烏里雅蘇臺查辦事件。丙申,賞故提督劉松山一等輕車都尉。命張曜、金順分軍西進。壬寅,以慈禧皇太后四旬慶典,推恩近支王公及中外大臣,賚敘有差。

是歲,朝鮮入貢。

十三年甲戌春正月乙巳朔,停筵宴。甲寅,湘軍剿古州苗匪,平之。丙辰,命編修張英麟、檢討王慶祺直弘德殿。辛酉,以劉坤一、胡家玉互參,降坤一三品頂帶,褫職留任,家玉金雋五級調用。癸亥,諭築東明長堤。己巳,官軍援沙山子擊回匪,勝之,賞福珠哩黃馬褂。

二月己卯,回匪擾巴里坤境,明春等會剿之。丙申,以法取越南地,越匪擾山西,偪滇疆,諭岑毓英部署邊防。禁京師私鑄。丁酉,上奉兩宮謁西陵。

三月甲辰,還宮。乙巳,賑奉天災民。丙午,命寶鋆協辦大學士。己酉,修海寧石塘。辛酉,論肅清貴州功,復陶茂林提督,賞提督何世華等世職。辛未,日本兵艦泊廈門,諭沈葆楨統兵輪往,相機籌辦。命李鴻章與秘魯公使會議華工事宜。

夏四月甲戌,詔撥帑十萬撫恤烏里雅蘇臺災擾部落。丁丑,上幸瀛臺。單懋謙因病乞休,允之。覲見俄使布策等於紫光閣。辛巳,上幸圓明園還宮。癸未,瑪那斯回匪犯奎屯等處,官軍進剿失利,景廉兵援之。丙戌,日本兵船抵臺灣登岸,與生番尋釁。命沈葆楨辦海防,兼理各國事務大臣,江、廣沿海各口輪船,以時調遣。辛卯,常順緣事褫職,命額勒和布為烏里雅蘇臺將軍,慶春為察哈爾都統,托倫布為科布多參贊大臣。丁酉,賜陸潤庠等三百三十七人進士及第出身有差。辛丑,景廉再乞病,不許。

五月壬寅朔,法、越和議定,諭邊將安輯內遷難民。壬子,上幸圓明園還宮。日本攻臺灣番社。丁巳,以慈禧太后聖節,予在京旗官六十以上者恩賞,停本年秋審、朝審人犯句決。己未,彗星見。乙丑,詔賑奉天災民。丙辰,允沈葆楨請,建臺灣海口砲臺,撫番社,撤疲兵。戊辰,日本師船游弋福建各海口。日使柳原前光與總署王大臣商臺灣兵事。

六月乙亥,諭飭總兵孫開華接辦廈門防務。己卯,召楊岳斌、曾國荃、閻敬銘、趙德轍、丁日昌、鮑超、蔣益澧、郭嵩燾來京。壬午,烏索寨降眾復叛,滇軍剿平之。癸未,允李鴻章請,以徐州唐定奎軍渡海赴臺。乙酉,諭戶部撙節不急之需,豫籌海防經費。諭沈葆楨部署南北路防守。丁酉,命翁同龢仍直弘德殿。

秋七月丁未,李鶴年請閩省陸路選立練軍,議行。庚戌,瑪納斯回匪犯西湖,官軍擊退之。壬子,命左宗棠為大學士,仍留陜甘總督任,景廉為欽差大臣,督辦新疆軍務,金順幫辦軍務。庚申,覲見比使謝恩施等於紫光閣。甲子,內務府大臣貴寶以任郎中時,於知府李光昭報效木植,欺罔奏陳,嚴議褫職。乙丑,馬賊陷寧古塔,旋復之。允福建軍餉借用洋款二百萬,由海關稅分年抵還。己巳,停修圓明園工程。庚午,諭責恭親王召對失儀,奪親王世襲,降郡王,仍為軍機大臣,並革載澂貝勒郡王銜。白彥虎等犯濟木薩,官軍擊敗之。

八月辛未朔,懿旨復恭親王世襲及載澂爵銜,訓勉之。諭修葺三海工程,力求撙節。丙戌,河南蝗。戊子,李光昭論斬。庚寅,諭各省整頓捕務。乙未,命左宗棠督辦西征糧臺轉運事宜,以內閣學士袁保恆為幫辦。詔各省酌裁釐局,禁種罌粟。丁酉,上幸南苑。戊戌,閱御前王大臣、乾清門侍衛射。己亥,上行圍。

九月庚子朔,上幸晾鷹臺,撒圍。辛丑,上幸晾鷹臺,閱神機營兵。壬寅,閱王大臣、侍衛等射。丁未,瑞麟卒,以英翰為兩廣總督。庚戌,日本續遣大久保利通來,與總署王大臣論臺灣番社兵事。丙辰,寧古塔匪首王文拴伏誅。辛酉,王大臣與日使成議,退兵回國,給日本難民恤金及臺灣軍費共五十萬。乙丑,賈楨卒。丙寅,諭李鴻章等於總署條奏海防、練兵、簡器、造船、籌餉、用人、持久諸事,詳議以聞。

十月辛未,以慈禧皇太后四旬萬壽,復劉銘傳提督。己卯,上慶賀禮成,賞廢員職銜,免王公、文武官處分,餘進敘有差。庚辰,恤廣東颶災。癸巳,命廣壽、夏同善赴陜西查事。己亥,上不豫,命李鴻藻代閱章奏。

十一月甲辰,命恭親王代繕批答清文摺件。丁未,賑徐、海水災。己酉,命內外奏牘呈兩宮披覽。以寶鋆為大學士。壬子,日本退兵。癸丑,冬至,祀天圜丘,遣醇親王代。頒部帑百五十萬築石莊戶堤工。甲寅,上以兩宮調護康吉,崇上徽號,詔刑部及各省罪犯分別減等。庚申,議行河南練軍。甲子,以石莊戶堤難就,允丁寶楨請,於賈莊一帶建壩築堤。

十二月辛未,詔蠲免雲南被擾荒地錢糧十年。甲戌,李宗羲病免,以劉坤一署兩江總督。上疾大漸,崩於養心殿,年十九。

慈安皇太后、慈禧皇太后召惇親王奕脤、恭親王奕、醇親王奕枻,孚郡王奕譓、惠郡王奕詳,貝勒載治、載澂,公奕謨,御前大臣伯彥訥謨祜、奕劻、景壽,軍機大臣寶鋆、沈桂芬、李鴻藻,內務府大臣英桂、崇綸、魁齡、榮祿、明善、貴寶、文錫,直弘德殿徐桐、翁同龢、王慶祺,南書房黃鈺、潘祖廕、孫詒經、徐郙、張家驤入奉懿旨,以醇親王之子承繼文宗為嗣皇帝。

光緒元年二月戊子,皇后阿魯特氏崩。三月己亥,上尊謚曰繼天開運受中居正保大定功聖智誠孝信敏恭寬毅皇帝,廟號穆宗。五年三月庚午,葬惠陵。

論曰:穆宗沖齡即阼,母後垂簾。國運中興,十年之間,盜賊劃平,中外乂安。非夫宮府一體,將相協和,何以臻茲?泊帝親裁大政,不自暇逸。遇變修省,至勤也。聞災蠲恤,至仁也。不言符瑞,至明也。藉使蘄至中壽,日新而光大之,庸詎不與前古媲隆。顧乃奄棄臣民,未竟所施,惜哉!


\end{pinyinscope}