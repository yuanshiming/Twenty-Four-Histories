\article{列傳一百九十六}

\begin{pinyinscope}
塔齊布畢金科多隆阿孫壽長鮑超宋國永婁云慶譚勝達唐仁廉劉松山

塔齊布,字智亭,陶佳氏,滿洲鑲黃旗人。由火器營鳥槍護軍擢三等侍衛。咸豐初,揀發湖南,以都司用,署撫標左營守備。以長沙守城功擢游擊,署中軍參將。侍郎曾國籓在籍治鄉兵,月調官兵會操。每校閱,塔齊布從侍,國籓與語,奇之,試所轄兵,特精整。為副將清德所忌,嗾提督鮑起豹將加摧辱。國籓劾罷清德,薦塔齊布「忠勇可大用,如將來出戰不力,甘與同罪」,加副將銜,兼領練軍。巡撫張亮基亦特薦之,以副將用。

三年,剿平茶陵、安化土匪,賜花翎。四年,率所部進剿粵匪,至湖北通城、崇陽,賊由岳州上犯,奉檄援寧鄉。未至,湘潭亦陷,賊勢甚張,遂改援湘潭。長驅至高嶺,猝遇賊,塔齊布手持大旗陷陣,麾軍縱擊,斬其酋數人,逐北數里,至城下。明日,賊大出,塔齊布伏兵山左右,賊近,砲殪百餘人,伏起夾擊,殭僕枕藉,燔城外賊柵皆盡。水師會戰,焚賊舟,浮尸蔽江。賊棄城走,六日而湘潭平。時曾國籓師挫於靖港,長沙震動,賴此一戰破賊,人心始定。捷聞,加總兵銜,賜號喀屯巴圖魯。詔斥鮑起豹畏葸不出戰,罷之,即超擢塔齊布署提督,尋實授。初,所部辰勇與標兵私鬥有釁,鮑起豹頻齮齕之;至是代其位,遍賞提標兵,示無修怨意,標兵大讙。眾見其由都司不三年立功驟膺專閫,莫不驚服,軍氣頓振。

賊自湘潭敗後,退走岳州,分黨陷常、澧。塔齊布馳抵新墻為援,進與羅澤南合軍,會水師攻岳州,七月,克其城。賊退泊城陵磯,勢猶盛,水陸夾擊,屢挫之。曾國籓親率新募水師至,戰失利。越日,賊由城陵磯舍舟登陸踞險,三路來撲,塔齊布分路迎擊,匹馬陷陣,士卒皆猛進,破其中路,賊復包鈔;愈戰愈奮,賊敗走,追至擂鼓臺,斬馘八百,落水者無數。迭偕羅澤南合力攻賊,旬日三捷。水師乘隙進剿,賊勢始衰,岳州危而不失。閏七月,偕羅澤南、李續賓進高橋,賊出二萬人抗拒。塔齊布首先沖入,諸軍繼之,會大雨,賊砲不燃;逾溝入壘,連破賊營十三座,殲斃及逃散者數千。水師亦分路剿殺,賊遁走,追擊二百餘里,破之於羊樓峒,又破之於崇陽,克其城,咸寧亦復。曾國籓師抵金口,令羅澤南攻花園,塔齊布趨洪山。八月,武昌賊遁走,塔齊布預設伏,賊至,要擊,左右夾湖無去路,殲戮溺斃八九千人,武、漢同時克復。進攻大冶,克之。

十月,與羅澤南會攻田家鎮,澤南攻半壁山,塔齊布屯富池口,中隔小河,作浮橋以通兩軍之路。賊以萬人來爭,澤南率李續賓奮戰,塔齊布隔港對擊,浮橋成。賊復由田家鎮渡江撲富池口營壘,迎擊敗之。遂與水師約大舉,楊岳斌、彭玉麟分隊毀其橫江鐵鎖,陸師從半壁山擁下,鏖戰一晝夜,鐵鎖盡毀,賊舟盡焚。賊棄壘而遁,克田家鎮,蘄州亦復,賜黃馬褂,予騎都尉世職。

偕羅澤南渡江至蓮花橋,遇伏,前隊少卻,塔齊布手刃賊目,追奔五十里,遂克廣濟。悍酋秦日綱、陳玉成、羅大綱效力守黃梅,以數萬賊布小池口、孔壟驛,而大河埔、龍頭寨皆立堅壘。軍抵雙城驛,賊突來襲,堅持不動,旋突起憑高下擊,斬其渠。賊奔大河埔,糾黨返斗,連擊敗之,殪三千餘,進攻黃梅,肉薄而登。塔齊布被石擊,流血被面,督戰益力,克其城。賊麕聚孔壟驛,三面築土城,塔齊布從西南進,累肩為梯,卓矛而躍,大破之。賊悉竄小池口,分黨奔湖口,與九江之賊相犄角。曾國籓率水師抵九江,塔齊布偕羅澤南渡江會攻。詔嘉諸將轉戰直前,同心效力,特頒珍賚。十二月,攻九江西南門不克,驍將童添元死之。會水師為賊所襲,喪失輜重。羅澤南攻小池口,塔齊布親率勇士二十人往督戰,眾寡不敵,且戰且退,匹馬沖突,為諸營捍蔽。有黃衣賊酋三來犯,塔齊布以套馬竿圈一酋斬之,奪其馬,餘賊皆靡,俟大隊沿江上,始單騎渡江回營,已除夕三鼓。

五年正月,城賊出犯,斬獲二百餘,又伏地雷誘賊來撲,斃之,戰屢捷而城不下。三月,總督楊霈師潰,武昌復陷,塔齊布分兵遣將回援。時水師半頓鄱陽湖,半回湖北,陸師留攻九江,力甚單,賊益堅拒。六月,與曾國籓會於青山議軍事,國籓謂宜移師東渡,剿湖口、東流、建德,塔齊布誓攻九江。七月,方傳令薄城,遽氣脫卒於軍,年三十有九。事聞,文宗震悼,詔依將軍例賜恤,湖南省城建專祠,謚忠武。同治三年,江南平,加三等輕車都尉世職,入祀昭忠祠。

塔齊布忠勇絕倫,自擢提督,涅「忠心報國」四字於左臂。每戰,匹馬當先,不使士卒出己前。他軍被圍輒馳救。背負槍,挾弓矢,二卒持長矛、套馬竿從,皆精絕,無虛發。每逼賊壘覘形勢,瀕危輒免,賊驚為神,而從容謙退,未嘗自伐其能。在嶽州,率四騎覘擂鼓臺,忽有悍酋獰髯睅目,持槊來犯。健卒黃明魁矛刺酋墜馬,塔齊布手刃殪之,獲其旗,知為偽丞相曾天養,驍桀稱最,群賊奪氣,尋皆引去。先是水師毀天養坐船,已報殲斃。塔齊布不欲爭功,終不上聞。軍中與下卒同甘苦,嘗共中夜絮語家事,念及老母,泣下。其卒也,軍民皆慟。湘潭、岳州兩捷,關系湘軍大局。曾國籓尤痛惜焉。

畢金科,字應侯,雲南臨沅人。以征開化苗功,敘外委。從王國才赴湖北,破賊荊州龍會橋、天門丁司橋,累擢都司。曾國籓奇其才,令從攻九江,改隸塔齊布部下。及塔齊布歿,石達開擾江西。金科每戰陷陣,驍勇為諸軍冠。五年冬,破賊樟樹鎮,而周鳳山軍敗,尋失之。六年,破賊章田渡,未幾,饒州陷,又失之。金科憤為他部所累,募死士攻取饒州。誓曰:「今日上岸不破賊,吾不復歸舟!」一鼓克其城,賜號呼爾察巴圖魯,補臨沅鎮都司,以游擊升用。名大振而忌者眾,軍食不繼,金科鬱鬱,思立奇功。江西大吏責其破景德鎮始給餉。七年正月,驟往攻之,入市不見一人,率十卒搜捕,賊蜂起,傷其七,亡其三,只身縱橫擊刺,踐血而出。賊以噴筒環攻於王家洲,殞焉。曾國籓為勒碑紀事,稱其勇與塔齊布相埒。洎江南平,疏請優恤,贈總兵銜,謚剛毅,立祠景德鎮。

多隆阿,字禮堂,呼爾拉特氏,滿洲正白旗人,黑龍江駐防。由前鋒補驍騎校。咸豐三年,從勝保剿粵匪,解懷慶圍。及賊擾畿輔,僧格林沁督師,徵兵黑龍江,多隆阿率二起馬隊從克連鎮、馮官屯,擢佐領。

五年,調援湖北,隸將軍都興阿部下。破賊黃州、新洲,從克廣濟。六年,克武昌、漢陽,加副都統銜,補協領,充行營翼長。進剿蘄州,敗賊於曹家河,復廣濟,次孔壟;賊復來犯廣濟,擊走之。時湘軍圍攻九江,賊於對岸小池口築土城,環以堅壘,附近數十里內,段窯、楓樹坳、獨山鎮等處賊壘凡數十。七年三月,都興阿與鮑超攻小池口,令多隆阿趨段窯,甫至,賊數千來拒,一戰破之,毀其壘。揚言攻獨山鎮,而暗襲楓樹坳,賊三路分拒,分擊之。別遣隊繞山南襲賊營,賊陣亂紛竄,進殪三千餘,乘勝疾趨獨山鎮。四鼓至,月明如晝,見賊壘浚深壕,木椿竹簽環之,不易攻。以輕騎誘賊出,散隊設伏,伺賊至,以勁騎沖突,又分隊潛越壕縱火,賊大奔,追殺至曉,斃賊五千,生擒數百。自是賊畏其軍,見旗輒走。

陳玉成率悍黨踞黃梅,連營百里,官軍屢挫。六月,多隆阿偕鮑超赴援,戰於黃梅十里鋪,分兵潛攻西路億生寺賊壘,賊出不意,駭奔,而十里鋪之賊亦大敗;水師進毀童司牌賊壘,湘軍自九江來援,合擊,大破賊於黃蠟山,平賊壘凡百餘,逐北至宿松城下,遂克黃梅,以副都統記名。賊尋棄宿松而去,多隆阿率馬隊駐守。鮑超以步隊屯二郎河。九月,賊陷太湖,分路來犯,偕鮑超合擊於涼亭河,破之;又合擊於楓香驛,賊死抗,鏖戰逾時,盡破其壘,乃遁太湖。八年春,賊由渡船口等處上犯,將綴官軍,以緩九江之攻。多隆阿伺其初至,急擊走之。

四月,九江克復,多隆阿從都興阿進規安慶,石牌為要沖,賊據山阻水為堅壘,水陸重兵守之。多隆阿攻上石牌,鮑超攻下石牌,同時並下。餘壘驚竄,馬步截殺及落水溺斃者六千餘人,其酋以數十騎逃入安慶;遂進軍逼安慶,破城外九壘,城賊屢出戰,皆擊敗之。會李續賓戰歿三河,桐、舒、潛、太諸縣皆不守。安慶圍師牽動,多隆阿退保宿松。次日,賊麕至,值大霧,多隆阿驅勁騎陷陣,敢死士隨之,斫殺無算。鮑超軍夾擊,呼聲震天,賊驚潰,自相踐踏,陳玉成精銳損失過半。自三河失利後,得此捷,軍聲復震。

是年冬,都興阿以病離營,奏多隆阿素當前敵,請所部悉令統帶。詔責成督率將士,就近聽胡林翼調度。九年春,進逼太湖。諸將猶謂賊銳,宜稍避,多隆阿曰:「不入虎穴,焉得虎子?」賊憑城出鬥,力戰挫之,營壘乃就。胡林翼遣唐訓方會攻,而石牌復為賊踞,攻太湖城連月不克。多隆阿謂必先取石牌而後太湖可下,乃選精銳,自茶婆嶺進兵,用火攻困之。賊由潛山、安慶兩路來援,分馬隊擊卻之。九月,復猛攻,焚其壘,殲賊酋霍天燕、石廷玉等,遂克石牌,令部將雷正綰駐守之。時湘軍圍安慶,陳玉成糾合捻匪縱十餘萬來援,太湖當其沖。胡林翼調集諸將為備,多隆阿已授福州副都統,戰略威望最著,遂令前敵諸軍並受節制。歲將盡,賊分三路至,鮑超屯小池驛,蔣凝學屯龍家涼亭,多隆阿自以馬步各隊駐新倉,硃品隆與唐訓方合軍仍圍太湖,初戰,中賊伏,頗有傷亡。賊勢專趨小池驛,鮑軍為所困。多隆阿慮牽動局勢,僅分隊為護餉道。會金國琛等軍出潛山高橫嶺、仰天庵,密約夾攻。

十年正月,賊移壘羅山沖、白沙畈,與城賊互應。多隆阿定計以大圍包裹援賊,以伏兵橫截城賊,令步隊誘敵,馬隊驟起圍擊。唐訓方鈔其後,硃品隆扼其右,鮑超遏其前,自率馬步沖突陷陣,賊大敗。次日,分軍三路,鮑超等東出小池驛,硃品隆等西趨羅山沖,多隆阿自居中路,見賊屯袤廣二十餘里,陳玉成踞羅山沖,尤為悍賊所聚,列隊進攻,為賊陣所壓,遂督中西兩路效力攻山,奮呼直上,賊始敗竄。鮑超亦由小池驛連破四路之賊,合隊追奔,同攻賊壘,乘風縱火,賊柵、賊館頃刻延燒,大小營壘百餘,一律平毀。金國琛等沿山兜擊,賊前後受敵,奪路狂竄,連夜追剿,擒斬無算。城賊聞敗,宵遁,伏兵四起,截殺未逸者,盡數殲之,即日克復太湖,乘勝追賊至潛山城下,亦克之。是役時稱奇捷,推多隆阿首功,詔加頭品頂戴。賊既敗,回踞桐城,增壘為固。七月,多隆阿率軍進逼城西,晝夜環攻,其西北山岡曰毛狗洞,賊壘最據形勢,攻下之。俯瞰全城,掘隧道轟之,未克。陳玉成復糾捻匪自舒城來援,十月,於掛車河隔河而陣,連戰敗之。復與李續宜約期合攻,裹賊於中,戰酣,以馬隊鈔擊,賊大敗,殲殪近萬,解散脅從萬餘,賊棄壘夜遁,賜黃馬褂。

陳玉成屢為多隆阿所挫,知不敵,乃謀犯湖北。是年冬,又糾眾繞英、霍,陷蘄水,掠黃州、德安。十一年春,折回趨安慶,經掛車河,耀兵而過。多隆阿曰:「此示假道,不欲戰也。」設伏山隘,令賊過呼噪勿擊,而以輕騎躡之,斬馘甚眾。玉成入安慶,築壘集賢關,多隆阿進駐高路埔。桐城、廬江諸賊二萬餘,將與玉成聯合。多隆阿分五路進擊,迭敗之於練潭、橫山堡、金神墩、新安渡,餘賊遁回桐城。未幾,悍賊黃文金糾眾二萬餘踞天林莊,擊走之。陳玉成留悍黨守集賢關,自率馬步五六千竄馬踏石,欲與桐城諸賊會合。多隆阿要擊於河岸,卻之。四月,玉成復率諸酋合粵、捻三萬餘人圖上犯,以解安慶之圍。多隆阿分路設伏,扼之於掛車河,左右往來沖擊,伏發,四面夾攻,殲斃八九千,追剿,五戰皆捷。賊仍退桐城,安慶之援遂絕。

官文、胡林翼疏陳多隆阿樸誠忠勇,智略冠軍,為眾所悅服,於是奉幫辦軍務之命。八月,安慶克復,急令穆圖善攻桐城,即日克之。數日中連克宿松、黃梅,而舒城賊亦棄城走廬州,予雲騎尉世職。擢正紅旗蒙古都統,又擢荊州將軍。進規廬州,同治元年春,連破賊,絕其運道,賊黨相率投誠,散遣千餘人。四月,大破援賊,陳玉成戰敗不敢入城,竄走,遂克廬州。令穆圖善、雷正綰追玉成,玉成奔壽州,為練總苗沛霖擒獻勝保營,誅之。捷聞,優詔褒嘉,加予騎都尉世職。

尋命督辦陜西軍務,率所部西征。時粵匪陳得才合捻匪姜臺凌、張洛行眾二十萬,三路窺陜。多隆阿令雷正綰、陶茂林率三千人前驅,大軍繼之,七月,抵商南。陳得才躡後路,圖截餉道,乃率穆圖善回軍掩擊,大破賊於荊子關。賊夜遁,令馬隊追賊,步隊休息,自攜數十人入商南,姜臺凌大隊突薄城下。調衛隊四營猶未至,陽示鎮靜,設伏城外,親率百餘人開城沖出,伏兵齊應,賊不知眾寡,倉皇退竄。次日,復出城誘戰,正與相持,總兵硃希廣率四營由間道來援,連日力戰,擒斬二千餘,賊乃西竄,檄溫德勒克西馬隊要截,王萬年步隊躡追。金順守荊子關,陶茂林遏武關,自率親軍於捉馬溝築壘,賊夜來襲,俟其近,排槍砲擊之,穆圖善自外夾攻,斃賊無算。至曉,見賊蟻聚,亙數十里,令降俘指認賊旗居中之紅邊白旗為姜臺凌,先集攻之。戰方酣,自率穆圖善從山側繞擊,賊敗如山倒,斫殺萬計,追至三角池,截其尾隊。姜臺凌僅以身免,張洛行聞風亦遁。詔嘉其旬日內剿除巨寇,頒賜黃馬褂及江綢刀,以示優異。

時勝保入陜督師,移多隆阿赴南陽防剿,連敗賊於樊城、唐縣。尋復命赴陜。十一月,入潼關。勝保以罪逮,詔授多隆阿欽差大臣,督辦軍務。

回匪方熾,遍擾東西北三路,陜南則為粵、捻、川匪所出沒。多隆阿令雷正綰任西路,自剿東路,克韓村、背坡諸賊營,同州解圍。二年春,督軍並攻王閣村、羌白鎮,破之。回匪自倡亂,至是始被痛創,遂進攻倉頭鎮。多隆阿積勞致病,將士亦多染疫,遣將分攻龐谷、雷化、喬千、孝義諸鎮,皆克,惟倉頭為老巢,負嵎未下。四月,移營進逼,揮軍縱擊,破其土城,賊大奔,追殺無算,東路肅清。令曹克忠一軍赴西安護運道,自率穆圖善等攻高陵,分路夾擊,八月,克之,掃蕩附近賊巢。

關輔略定,而漢南諸賊紛擾。川匪藍朝柱近踞盩厔,三年春,親督兵力攻,城小而固,多隆阿憤甚,臨高指揮督戰,城已垂破,忽中槍,傷頭目,將士攻城益力,旋克之。事聞,溫詔慰勞,賜上方藥,遣其子馳驛省視。尋命督辦陜、甘兩省軍務。四月,創甚,卒於軍。贈太子太保,予一等輕車都尉世職,入祀京師昭忠祠,立功地建專祠,謚忠勇。未幾,江寧復,加一雲騎尉,並為一等男爵。子雙全襲,官頭等侍衛。

孫壽長,光緒中,官正黃旗滿洲副都統,統奉天仁字軍,因事革職。二十六年,俄兵入邊,壽長力請戰,召回京,未行,為俄人所執,不屈死。

鮑超,字春霆,四川奉節人。咸豐初,以行伍從提督向榮廣西剿匪,尋入湖南協標。四年,曾國籓治水師,調充哨長。勇銳過人,每以單舸沖賊隊,當者闢易。從克岳州、武昌、漢陽,破賊田家鎮、武穴,積功擢守備,賜花翎。五年,武昌復陷,赴援,胡林翼拔充營官。擊賊於漢陽小河口、占魚套,屯沌口,破宗關賊壘,擢都司。會金口陸軍潰,賊聚攻胡林翼於高廟。超飛棹往救,力戰卻之。德安、應城之賊復由溳口來犯,火其舟,拔林翼於重圍。進搗賊營,右肋中砲,裹創而戰,復金口。論功最,擢游擊,賜號壯勇巴圖魯。

六年,林翼疏薦超勇敢冠軍,曉暢兵略,以水師總兵記名。夏,會攻漢陽,扼沙口,斷賊往來,江面肅清,擢參將。武昌既復,林翼令赴長沙募勇三千,創立霆字五營,改領陸軍。七年,補陜西宜君營參將。攻小池口,破賊於孔壟,援黃梅。時總兵王國才戰歿濯港,賊甚張。眾議水陸暫扼守,超不可,主速戰,多隆阿贊之,以騎兵助攻億生寺賊壘。戰一日夜,傷左膝右臂,不退,遂破黃蠟山賊巢,生擒賊渠,斬馘五千有奇。擢副將,加總兵銜。乘勝焚黃梅後山,進屯宿松二郎河,平涼亭、祝家塝賊壘。陳玉成擁眾數萬踞楓香驛,連破之,奪其十三壘。八年,援麻城,克黃安,偕多隆阿進規太湖。超攻北門,燒賊火藥庫,破雷公埠、石牌賊營,斬馘萬餘,授湖南綏靖鎮總兵。進攻安慶省城,而三河軍敗,陳玉成糾捻眾上犯,都興阿令超退守二郎河,遏賊沖。超偕多隆阿大破賊於宿松東北花涼亭,斬偽成天侯韋廣新以下渠目三百餘,殲賊八千,散脅從數萬。捷聞,優敘。

九年,會諸軍圍太湖,陳玉成糾眾十餘萬來援。多隆阿總統諸軍,撤圍分屯,備大戰。超壁小池驛,十二月,賊至,壓超軍而壘,凡百餘座。超破其十餘壘,賊悉銳更番環逼,晝夜力禦,棚帳皆為砲裂,士卒傷痍,糧道將斷,超志氣彌奮,相持二十餘日。十年正月,援軍自潛山天堂出,諸軍乃約期夾擊。超空壁而出,賊圍之數重,為方陣拒戰,四路賊皆破。合諸軍盡焚賊壘,斬馘無算,遂克太湖。官文等奏捷,謂:「非超勇鷙堅強,以二千人獨御前敵,血戰兼旬,則援應各師,必有緩不濟急之勢。」詔加提督銜。超與多隆阿不相下,為胡林翼故,勉屈聽節制。臨危,多隆阿復不力救,雖成功,頗觖望,林翼慰解之,遂乞假省親去軍。

曾國籓方規皖南,奏令超增募萬人以從,未至,悍賊黃文金由浙入贛,李秀成亦由蕪湖上犯,取包圍遠勢。詔促超赴軍,而寧國陷,褫勇號,責圖克復。賊已直犯祁門大營,國籓兵單,誓死守。超至休寧,聞警,日馳百餘里,連戰皆捷,驅賊出嶺,國籓亦不意超軍遽至也。詔嘉其神速,賜號博通額巴圖魯。進援江西景德鎮,與左宗棠會剿,因雨遲至。宗棠假霆軍旗幟,賊見之卻走。復回踞洋塘、謝家灘。十一年正月,超至,大戰破之。黃文金負創遁,追敗之黃麥鋪,復建德。曾國籓奏請以超軍為江、皖游擊之師。陳玉成與安慶城賊夾攻官軍,頗為所困。超渡江援之,大破賊於赤岡嶺,生擒悍黨劉瑲琳。既而李秀成犯江西,連陷二十餘城。超破之於樟樹鎮,斬馘萬餘,被珍賚。又進解撫州圍。調援江北,至南昌,聞安慶已克復,回軍戰於貴溪、雙港、湖坊河口,大破賊,遂克鉛山,解廣信圍,李秀成遁走。命遇提督缺出侭先題奏。規取青陽,敗援賊,盡毀城外賊壘。

同治元年,詔推恩諸將,嘉超屢著戰功,賜黃馬褂,授浙江提督。時賊聚皖南,東連蘇、浙,西瀕江,上自建德、東流,下至銅陵、蕪湖。超東西策應,解銅陵圍,克青陽、石埭、太平、涇縣,大破楊輔清於寧國,復其城,予雲騎尉世職。賊首洪容海、張遇春先後投誠,受降,編其眾為啟化營、春字營,從戰皆有功。是年冬,丁母憂,請終制,詔奪情留軍。二年,戰涇縣。賊設伏來誘,超亦潛伏山坳以伺,斷賊後路,夾擊,大破之,遂克西河,灣沚。黃文金竄鄱陽,方欲赴援,李秀成又陷江浦、浦口,超馳救,破賊青溪鎮,連克巢縣、含山,和州、江浦、浦口,北岸肅清;遂會水師克九洑洲,而青陽又被圍,馳至,賊遁,追破之於曹塘,進攻東壩賊巢,克之。賊酋先後率眾降者數萬,建平、溧水皆復。曾國籓奏以東壩為重隘,令超駐守,以備游擊。

三年春,克句容、金壇。時蘇、浙敗賊聚於江西,命超馳援,破賊於豐城。會江寧克復,論功,予一等輕車都尉世職。七月,破許灣賊巢,連克崇仁、宜黃、東鄉、奎谿、南豐。賊酋陳炳文以六萬人降,受之。追賊贛南,解寧都圍,殲賊萬計,賜雙眼花翎。賊酋汪海洋遣黨詐降,整軍以待,驟擊之,潰,入瑞金,城下尸積為阜,城賊亦遁,追至福建境。洪秀全幼子福瑱為贛軍所擒,詔錫封超一等子爵。

先是,超請回籍葬親,賜銀五百兩,命俟江、皖肅清後予假。是年冬,申前請,允之,復命假滿率舊部出關援新疆。所部多南人,畏遠征,疆臣多以為言,請留剿粵匪餘孽,曾國籓亦請先留甘肅內地。超已令部將宋國永率八千人先發,四年春,至湖北金口,軍潰。詔急起超於家,免其出關,改赴福建,命沿途招撫潰勇。潰勇多降眾,仍由江西趨粵與匪合,超由贛州進剿。時粵匪餘黨聚踞嘉應州,汪海洋已為閩軍所殲,賊中推譚體元為首。十二月,戰於平成鋪,賊踞嶺而陣。超合閩、粵諸軍大破之,追至城下,宵遁。預設伏於黃沙障及北溪、白沙壩,五路兜擊,譚體元中槍墜崖死,諸酋擒斬無漏網者,獲叛勇歐陽輝、黃矮子等磔之。粵匪蕩平,加一雲騎尉世職。五年,仍授浙江提督,命移師剿捻,追逐於湖北、河南、陜西界上,賊望風輒走。疆臣爭欲得其兵為助,以西安戒嚴,詔飭赴陜。

六年正月,抵樊城,聞捻匪至,與淮軍將劉銘傳約期於安陸永隆河夾擊。銘傳先至,為賊所敗,夷傷頗重。超至,擊賊背,大破之。任柱、賴文光遁走,俘其妻孥,奪回所失軍裝。超久為名將,銘傳後起與之埒。是役超自以轉敗為勝有功,而銘傳咎其後至,李鴻章右銘傳,超大憤,稱病。迭詔慰勉,曾國籓及鴻章馳書相繼。超終乞罷去軍,所部三十營,令部將宋國永、唐仁廉分領。詔婁云慶代將,皆慮其軍難制,遣散過半焉。

超既歸,屢敕問病狀。十三年,召來京,因病未復,仍續假。光緒六年,起授湖南提督,募軍駐樂亭防俄羅斯,事定回任。八年,復以病請解職。十一年,法越戰起,命率師駐雲南馬白關外。和議成,撤防回籍。十二年,卒,贈太子少保,賜銀三千兩治喪,立功地建專祠,謚忠壯。子祖齡襲爵,官浙江金衢嚴道。

超治軍信賞必罰,不事苛細,得士卒死力。進戰,疾如風雨,賊望而披靡,棄械跪馬前,即不殺,以此服其威信。所部多驍將,宋國永、婁云慶最為所倚。譚勝達、唐仁廉亦並至專閫。

國永,四川人。由軍功補千總。初從鮑超隸水師,以戰金口功,擢守備。破賊童司牌、黃蠟山,克麻城、黃安,累擢參將。霆軍初立,為營官。咸豐十年,曾國籓調霆軍赴皖南。鮑超方假歸,國永暫統其軍。及超至,從攻休寧,分兵復黟縣,連破賊於羊棧嶺、盧村、洋塘、黃麥鋪,功皆最,超擢以總兵記名。十一年,補廣西梧州協副將。從援江西,破賊樟樹鎮,加提督銜。克鉛山,解撫州、廣信圍,以提督記名。同治元年,克青陽、寧國,授直隸宣化鎮總兵。時楊輔清仍踞寧國附近圖返攻,國永屯老祖山,迭破來犯之賊。二年,進克西河、灣沚,賜黃馬褂。

三年,江南平,鮑超回籍,國永與婁云慶分領其軍,調赴福建,未行。四年,鮑超將赴新疆,國永率所部由江西先發,軍中索餉鼓噪,撫定之。道經湖北,復譁潰於金口。坐不能約束,褫職留營。從克嘉應州,復原官。從剿捻匪,自永隆河破賊後,鮑超乞病,軍中事一倚國永。及超去軍,國永先請散遣己所部眾,餘付婁云慶統之。八年,授雲南鶴麗鎮總兵。李鴻章疏陳國永戰績,稱為膽識兼優、不可多得之才。留於兩江委用,駐防鎮江。光緒初,調赴福建。四年,卒,詔念前功,允祀四川、湖北霆軍昭忠祠。

雲慶,湖南長沙人。初入水師,累功至都司,尋充霆軍營官。咸豐十年,小池驛之戰,功最,擢參將。從戰皖南,會鮑超赴援江西,留雲慶率四營扼漁亭。賊聞大軍遠出,突來犯,擊走之。追至巖勍,斃賊酋黃世瑚等,復擊敗上溪口賊。十一年,會克休寧。既而攻徽州,諸軍失利,云慶仍挫賊,全軍而退。尋會張運蘭戰盧村,遂克徽州,以總兵記名。從鮑超轉戰江西,數破賊,功最,授直隸正定鎮總兵。同治元年,從克青陽,乘勝攻石埭,云慶率士卒負板薄城,蟻附而登,克之。時霆軍威名益著,營隊日增。曾國籓令云慶與宋國永為其軍分統,克寧國,以提督記名,賜黃馬褂。三年,分兵克金壇。及江寧既下,調援江西。既而鮑超奉命西征,分兵令宋國永赴陜甘,云慶率萬人援福建。國永軍再譁潰,云慶軍不遠役,又得餉,未為搖動。尋從鮑超滅賊於嘉應,始赴正定鎮本任。六年,鮑超病歸,眾慮霆軍難制,曾國籓薦雲慶才能應變,詔飭接統。遂裁撤全軍,改募五千人,號曰霆峻營,駐防湖北。明年,捻平,云慶請歸養。光緒初,復起授正定鎮總兵。十七年,擢湖南提督。三十年,以老乞歸,卒於家。

勝達,湖南長沙人。咸豐中,投效霆軍,無役不從。石牌、羊棧嶺、洋塘、赤岡嶺諸戰,功皆最,累擢至副將。從戰雙港,克鉛山,賜號協勇巴圖魯。同治元年,赴援銅陵,戰橫塘,斬賊酋於陣。進攻城外賊壘,勝達偕唐仁廉冒砲煙逾壕,奪其一壘,餘壘皆下。賊夜遁,復銅陵,以總兵記名。又戰於寒亭,勝達橫沖賊隊截為四,不能成伍,大破之,復寧國,加提督銜。二年,分兵解涇縣圍,連奪西河、灣沚要隘,詔遇總兵缺先行簡放。三年,克句容,以提督記名。鮑超以東壩為重隘,令勝達守之。賊至,蔽山谷。勝達陷陣,刺殺其酋,賊大潰。踐尸追擊,殲斃數千。尋赴援江西,克新城,解寧都圍。四年,霆軍以索餉毆傷糧道段起,勝達坐褫職,尋復之。及嘉應殄滅粵匪,賜黃馬褂,授直隸正定鎮總兵。八年,始赴任,練軍捕盜,濬河修堤,頗著勞勩。光緒元年,卒於官,賜恤,謚勇愨。

唐仁廉,湖南東安人。初隸楊岳斌部下。粵匪韋志俊以池州降,仁廉從彭玉麟往受之。賊黨忽變,仁廉手刃其悍者數人,岳斌嘉其勇,令選降眾立仁字營。咸豐十年,改隸霆軍。從戰太平、石埭間,擢守備。克黟縣、建德,擢游擊,賜號壯勇巴圖魯。破安慶援賊於赤岡嶺,戰豐城,克鉛山,累擢副將。同治元年,克青陽,以總兵記名。三年,克金壇,以提督記名。四年,戰嘉應,粵匪蕩平,賜黃馬褂。五年,從剿捻匪,率馬隊逐賊於鄂、豫之交。六年,大破賊於永隆河,連敗之於鍾祥池河、棗陽平林店。鮑超解軍事,仁廉分統其眾,從李鴻章剿匪。東捻平,論功,遇提督侭先簡放。西捻張總愚犯畿輔,仁廉追賊於直隸、河南、山東三省之間,連敗之濬縣大伾山、海豐郝家寨、商河李家坊。又偕郭松林合擊於沙河,總愚中槍遁,再敗之於高唐盧寨。西捻平,以一等軍功議敘。九年,從李鴻章援陜西,平北山土匪。尋調防畿輔,駐青縣馬廠。十三年,授通永鎮總兵。光緒十年,擢廣東水師提督。二十年,皇太后萬壽恩,詔加尚書銜。日本犯遼東,時以唐仁廉為霆軍舊將,召至京。仁廉奮發陳方略,請募二十營當前敵,允之。及成軍出關,和議旋定,遂還。二十一年,卒,賜恤。

劉松山,字壽卿,湖南湘鄉人。初應募入湘營,隸王珍部下,從平永州、郴、株諸匪,以功擢千總。咸豐七年,克崇陽、通山,擢守備,始領一營。從援江西,克廣昌、樂安,擢都司。王珍卒,張運蘭分領其軍。松山從戰克建昌,擢游擊。賊由福建回竄江西,陷安仁。松山從破賊於青山鋪,進攻安仁,攀堞先登,克之,擢參將。會剿廣東連州踞賊,擒其酋,折回江西。

九年,轉戰至徽州,屯祁門。賊自盧村來犯,突擊敗之。會諸軍克景德鎮,追至浮梁,爭渡橋,賊返斗,城賊出助。松山據橋血戰,軍賴以全,遂克浮梁,擢副將。十年,追敘連州功,加總兵銜,賜號志勇巴圖魯。十一年,克建德、黟縣,進攻徽州,賊夜劫營,諸營皆潰,松山列隊月下不少動,賊不敢逼。遮諸將曰:「我第四旗劉松山也!」戒勿奔,眾始定。曾國籓自是待之以國士。賊再入黟縣,再克之。毀樟嶺、盧村賊壘,賊棄徽州遁,進克休寧,以總兵記名。楊輔清復糾黨圍徽州,松山四戰皆捷。援軍至,會擊於巖市,賊引去。同治元年,克旌德。張運蘭以病歸,松山與總兵易開俊分領其眾。守寧國,大疫,士卒多病,松山加意撫循,力疾戰守。二年,援涇縣,破賊於金村、李村,而賊乘虛襲寧國,松山馳還,設伏敬亭山,伺賊至,分三路鼓噪而進,伏起夾擊,伏尸塞途,蹙餘賊水濱多死。三年,大軍克江寧,松山收降潰賊四千人。皖境肅清,署皖南鎮總兵。

四年,授甘肅肅州鎮總兵,仍調皖南鎮。曾國籓督師剿捻匪,奏以松山獨統湘軍從征,屯臨淮。時湘將久役思歸,又不習北方水土,皆不原從。惟松山投袂而起,立率所部渡江。有譁餉者,誅數人而定。五年,敗捻首張總愚、牛洛紅於湖團,又敗之於徐州西,追剿入河南。張總愚踞西華,牛洛紅踞上蔡,設伏萬金寨,圖鈔襲官軍。松山與總兵李祥和擊破之,進攻雙廟,大破之,又敗之郾城、南陽、新野。總愚挾眾竄陜西,自此與任柱等分,不復合,號為西捻。

時議遣援剿之師,因陜境殘破,諸將皆觀望。惟松山毅然自任,率師西行,曾國籓尤重之。六年,擢廣東陸路提督。張總愚與回匪合,踞郿縣,進擊走之。轉戰扶風、岐山間,於涇陽要擊竄賊,殲斃數千。追至富平,破其壘,而陜軍戰灞橋失利,賊犯同州、朝邑。松山疾趨,及賊於晉成堡、姜彥村,張兩翼擊之,賊敗走。追至許家莊,復返斗,血戰四時,大破之,同、朝圍解,被珍賚。賊勢猶張,渡渭犯西安,松山會戰於城南,斬馘數千,解散萬人。六月,左宗棠蒞陜督師,張總愚復結回匪窺同州、朝邑,分黨踞流曲鎮、王寮鎮以阻師。松山連拔二鎮,繞北山趨朝邑,截賊前。賊走高陵,復渡涇而東,松山據涇,濬壕築墻而守。賊鋌走入北山,陷綏德州。十一月,松山偕郭寶昌擊敗之。

賊棄綏德城,踐冰渡河,入山西,陷吉州、鄉寧。松山偕郭寶昌追剿,克二城,解河津、稷山之圍,又追敗之洪洞。賊由垣曲入河南境,七年正月,逕犯畿輔。松山間道逾太行,冒雪日行百數十里,先諸軍抵保定,特詔嘉獎,優敘。敗賊於獻縣商家林,又敗之深州、博野。偕郭寶昌、張曜、宋慶合擊於深澤,大破之,賊渡滹沱南竄。畿輔解嚴,晉號達桑阿巴圖魯。迭追擊於河南延津、封丘,山東海豐,直隸長垣、慶雲、滄州、吳橋,大小數十戰,與淮、楚諸軍長圍困賊,六月,張總愚赴水死。捻匪平,賜黃馬褂,予三等輕車都尉世職。從左宗棠還陜剿回。

松山在軍十餘年,僅因募勇一歸里,聘婦二十年未娶;至是婦家待於洛陽,成禮旬日即行。冬,抵陜,議先平土匪,乃可專力剿回。次綏德,分軍攻懷遠大理川回巢。自督攻小理川、店子寺、周家嶮,悉拔之。破定邊回酋馬萬得、馬棘子眾數萬。八年,部卒合會匪叛,踞綏德,松山馳捕首逆百餘人而定,自請重處,革職留任。進剿西北路諸堡,收降董福祥等眾凡十七萬人,榆、延、綏、鄜四郡皆肅清。

秋,度隴規靈州,破李旺堡、黑城子回寨數百,克靈州,開復處分。敗匪乞撫,察其詐,擊之,平大小堡寨數十。進攻金積堡。堡酋馬化隆悍狡為諸回之最,黨眾糧足,負嵎已久,官軍屢為所挫。松山先籌糧運,敗其黨援,大舉穩進。西寧、河州、臨洮、靖遠諸回皆震其威,不敢來救,先平堡北諸莊寨。九年正月,賊在秦渠南,踞石家莊及馬五、馬七、馬八諸寨,負嵎抗拒。松山先破石家莊,督攻馬五寨,破其援賊,毀外卡,縱火焚寨門。垂克,砲中左乳,墜馬,諸將來視,叱令整隊速攻,毋亂行列,遂破馬五寨。松山創甚,顧諸將曰:「我受國恩未報,即死,毋遽歸我尸,當為厲鬼殺賊。」遂卒,年三十有八。

事聞,詔嘉其謀勇兼優,無愧名將,贈太子少保,加騎都尉兼一雲騎尉,入祀京師昭忠祠,立功地建專祠,謚忠壯。松山既歿,兄子錦棠代領其眾,留其柩未歸以系軍心。次年,克金積堡,特詔賜祭一壇。十二年,甘回悉平,追論前功,加一等輕車都尉,並世職為二等子爵。嗣子鼒襲,官至山西按察使。

論曰:曾國籓湘軍初起,賴塔齊布為助,及規江寧,清江、皖後路,則鮑超之力為多。胡林翼由鄂規皖,悉倚多隆阿、鮑超二人。塔齊布不幸早歿。多隆阿才略冠時,朝廷倚以剿回,中道而殞,未竟其用。鮑超攻戰無敵,動招眾忌,功成身退,亦以保全之。劉松山後起,忠誠獨著,左宗棠平捻、平回,胥資其力;使獲永年,其建樹未可量也。


\end{pinyinscope}