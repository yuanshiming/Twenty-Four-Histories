\article{列傳六十二}

\begin{pinyinscope}
格爾古德金世德趙士麟郭世隆傅臘塔馬如龍

格爾古德,字宜亭,鈕祜祿氏,滿洲鑲藍旗人。自筆帖式授內院副理事官。康熙三年,從定西將軍圖海平湖廣茅麓山李自成餘部。師還,遷弘文院侍讀,進翰林院侍讀學士,充日講起居注官。十三年,從安親王岳樂討吳三桂。三桂將林興珠降,上策請分水師,泊君山,斷常德道;泊香爐夾扁山,斷長沙、衡州道:則三桂將坐困。安親王令格爾古德馳奏,並以興珠語聞,上密諭駐岳州諸將議行。師還,擢詹事,遷內閣學士。

二十一年,授直隸巡撫。上諭之曰:「金世德、於成龍為巡撫有聲,爾承其後,得名甚難。若急於求名,或致僨事,爾其懍諸!直隸旗下莊頭與民雜處,倚恃聲勢,每為民害。爾其嚴察懲創,即皇莊亦毋寬宥。」八旗圈地屬於王公大臣者,輒置莊,設莊頭,主徵租,遂以病民,上深知之,故以諭格爾古德。格爾古德尋疏言:「自鬻投旗之人,或作奸犯科,冀逃法網;或游手好閒,規避差徭。本主聽其仍居本籍,放債牟利,則諱旗而稱民;窩逃構訟,遇官長訪聞,又舍民而稱旗。詐害良善,官不敢問。應責成本主,止留農戶在莊,餘俱收回服役。有徇縱者議處。」下所司飭禁,並諭戶部:「凡鬻身之人,先經犯罪,投旗冀幸免者,與知情之本主,並從重治罪。」時大學士明珠所屬佐領下人戶指圈民間塚地,民訴於戶部,事下巡撫,令宛平縣察勘。知縣王養濂言無礙民塚,格爾古德疏劾圈占塚地屬實,養濂罣吏議。詔嗣後有如此者,嚴懲不貸。自康熙初,鼇拜柄政,總督硃昌祚等以圈地獲罪,由是無敢訟言其失者。至奸民竄入旗下,尋仇傾陷,狡桀莫能制。獨格爾古德承上指,執法嚴懲,時稱為「鐵面」。

二十三年,上幸五臺山,格爾古德迎駕,詢地方賢吏,以靈壽知縣陸隴其對。尋疏薦井陘道李基和、盧龍知縣衛立鼎與隴其廉能,下部擢用。頃之,以疾乞休,優詔慰留。會詔廷臣公舉清廉官,首以格爾古德列奏。上念其羸疾,遣御醫診視。未幾,卒,賜恤加等,謚文清。

格爾古德清介,布衣蔬食,卻餽遺,纖毫不以自污。上嘗責漕運總督碩幹居官無狀,碩幹言:「臣為眾所忌,故未能致聲譽。」上曰:「格爾古德為巡撫,沒後人猶思慕稱頌。居官茍善,豈有不致聲譽者?」為上所重如此。祀直隸名宦。

金世德,字孟求,漢軍正黃旗人,兵部侍郎維城子。淹貫經史,精國書。以廕生授內院博士,累擢左副都御史。康熙七年,授直隸巡撫。是時尚循明制,直隸不置兩司,世德請設守道理錢穀,巡道理刑名,如外省布政、按察二司。由是始有專司。畿北諸郡,旗、民雜處,易於容奸,請立屯長以治之。唐縣等三十七州縣,田一千六百餘頃,河流沙擁,民不能耕。歲輸銀二萬有奇、穀豆三百九十石,歷年責原戶納賦為民害,世德為奏請除額。地震通州等九州縣,復請賑恤,並蠲免錢糧。皆如所請行。師南征,供億繁急。世德單騎行營中,躬料芻糗,軍無橫索,吏無侵漁,市肆晏然。十九年,卒,謚清惠。

趙士麟,字麟伯,雲南河陽人。康熙三年進士,授貴州平遠推官。改直隸容城知縣,緝盜衛民,創正學書院,與諸生講學。行取,授吏部主事。歷郎中,擢光祿寺少卿,三遷至左副都御史。疏請臺灣改郡縣比內地,設總兵鎮守,省沿海之戍卒,詔報可。

二十三年,授浙江巡撫。杭州民貸於駐防旗兵,名為「印子錢」,取息重,至鬻妻孥賣田舍;不償,則閧於官。營兵馬化龍毆官,成大獄。士麟移會將軍掣繳券約,捐資代償。將軍令減子歸母,母復減十之六。事遂解,民大稱頌。詔裁浙江總督,總督駐衢州,督標兵三千被汰,乏食譁掠,民罷市。士麟仍濟以餉,因奏設副將一,定額兵八百餘,留撥各營缺額。眾乃定。浙中豪右衙蠹,驕悍不法,為民害。士麟廉得其狀,悉置之法,強暴斂跡。省城河道久淤,督役疏濬,半載訖工,民以為便。復繕城隍,修學校,親蒞書院,與諸生講論經史及濂、洛、關、閩之學,士風大振。禁革規費,積弊一清。二十五年,移撫江蘇。浙人懷之,繪圖以志去思,並於西湖敬一書院肖像祀之。尋召為兵部督捕侍郎,調吏部,皆能舉其職。三十七年,卒。祀浙江名宦。

士麟潛心正學,以硃子為歸。躬行實踐,施於政事,士愨民恬,所至皆有聲績。

郭世隆,字昌伯,漢軍鑲紅旗人。父洪臣,原籍汾州。順治二年,英親王阿濟格下九江,洪臣隨明將左夢庚來降,入旗,授佐領,分轄降眾。累官湖廣道州總兵。康熙四年,世隆襲管佐領,授禮部員外郎,改御史。二十七年,盛京福陵守兵訴其兄冤死,命世隆往按,得誣良刑偪自縊狀,原審侍郎阿禮瑚等坐失實奪官。頃之,超擢內閣學士。聖祖謁孝陵,經通州,山西禮縣民訴知縣萬世緯及知府紀元婪索狀,命世隆會督撫按治。世緯坐貪婪、科派、杖斃無罪人,元坐受賕薦世緯卓異,皆論死。

二十九年,代於成龍為直隸巡撫。先是,罷任安溪知縣孫鏞告福建巡撫張仲舉、布政使張永茂侵蝕庫帑,遣郎中吳爾泰會總督勘訊,至即拘訊知府六人,連引州縣官數十人。上聞疑之,命世隆往按,發仲舉與前布政使張汧竄改賦冊、侵隱已徵額銀捏作民欠,又汧遷湖廣巡撫虧福建庫帑三十餘萬,仲舉前任湖南布政亦虧帑,相約互抵;嗣仲舉聞汧以贓敗,而福建庫尚未完,飭屬代為彌縫,左證悉合。仲舉、永茂俱論罪如律。

世隆之任,帝諭曰:「於成龍居官甚善,繼之不易,爾當勤慎任事。」順天、保定、真定、永平諸府旱,世隆奉命履勘,疏言:「被災者七十四州縣,請蠲本年及來年額賦。霸、文安等十四州縣災尤重,請治賑。」迭疏籌積貯,並以奉天歲豐,請飭山海關暫聽民間轉糶,仍限肩挑馱負,不得以大車裝載,皆如所請。又疏言:「真定地當沖要,所屬贊皇縣,西有大峪曰子午套,素為盜藪,請移紫荊關副將駐真定;調馬、步兵二千分防霸州。」子牙河決,淹沒田畝,請修築大城等縣堤岸,並濬王家口、黑龍港諸支流堙塞者,皆報可。

三十四年,擢閩浙總督。歲歉,率閉糴居奇。世隆疏請蠲賦,並發帑二十萬,乞糴江、浙,海運平糶,詔俞之。先是浙省奏請鼓鑄,官吏射利,請減其分數。由是私鑄者眾,每錢不及七八分,壅滯不行。三十八年,上南巡,世隆迎駕,至杭州,民擁輿赴訴。乃停官爐,發帑收毀私錢,錢得流布。上聞,為褒美。鄞縣沿海田,被水沖決一千七十餘畝,請永免額賦。

四十一年,調兩廣總督。廣東海疆二千餘里,守汛遼闊,盜賊出沒無常。世隆疏定營制,增設兵船巡哨,迭擊敗海盜,沉其舟四十五。疏報擒海陽巨盜蔡玉也等五人。上遣刑部侍郎常綬往勘,因議世隆平時禁賊不嚴,盜發,朦朧掩飾,坐奪官。

四十六年,起湖廣總督、疏陳防守紅苗,請沿邊安設塘汛,禁內地民與苗往來,並勿與為婚姻。未幾,召為刑部尚書。五十年,以山西流匪陳四等潛入湖廣,鳩黨劫掠,世隆前任總督坐失察,奪官。五十二年,萬壽,復原品。居三年,卒。直隸、福建、浙江、兩廣、湖廣皆祀名宦。

傅臘塔,伊爾根覺羅氏,滿洲鑲黃旗人。自筆帖式授內閣中書,遷侍讀。康熙十九年,授山東道御史,有聲臺中。二十五年,出為陜西布政使。二十六年,擢左副都御史,遷工部侍郎。二十七年,偕侍郎多奇往雲南按提督萬正色與總兵王珍互訐事。讞實,正色、珍俱論罪有差。調吏部,授兩江總督。陛辭,上諭曰:「爾當潔己奉公,督兩江無如於成龍者,爾效之可矣!」傅臘塔至官,清弊政,斥貪墨,讞獄尤明慎。贛縣民訴知縣劉瀚芳私徵銀米十餘萬,並蠹役不法。傅臘塔因劾布政使多弘安、按察使吳延貴、贛南道鍾有德於吏役婪贓不速勘,復從輕擬,曲為庇護,弘安、延貴、有德並坐罷。

二十八年,上南巡,閱運河,命傅臘塔會河道總督王新命勘儀真河閘。疏言:「閘外為北新洲,北新洲外又有漲沙平鋪江中。應疏北新洲支河,直通四閘。糧艘循漲沙尾入新河口,可以通行。」別疏言:「江寧廛稅累民,內輸房稅,外輸廊鈔,更外輸棚租,請予蠲免。」皆如所請。二十九年,淮、徐饑,發常平倉穀賑恤,災民賴焉。蘆洲丈量,例委佐貳,民苦需索。傅臘塔定五年一行,悉以印官理其事。歷年逋賦,量為帶徵,由是積困頓甦。是年,監臨江南鄉試,疏稱士子應試者萬有餘人,請廣科舉額,下部議,增廣額四十名。疏劾大學士徐元文、原任尚書徐乾學縱子弟招權罔利,巡撫洪之傑徇私袒庇。詔毋深究,予元文休致。沭陽民周廷鑒叩閽訟降調侍郎胡簡敬居鄉不法,並及之傑瞻徇狀,命傅臘塔按治,得實,簡敬及其子弟並治罪,之傑奪官。

三十二年,廣東巡撫江有良與巡鹽太常少卿沙拜互訐。傅臘塔往按,有良、沙拜並坐受賕,奪官。三十三年,疏言:「淮、揚所屬多版荒,巡撫宋犖曾請緩徵,格於部議。臣履畝詳勘,鹽城、高郵等州縣因遇水災,業戶逃亡者眾。今田有涸出之名,人無耕種之實,小民積困。熟田額糧尚多懸欠,何能代償盈萬之荒賦?請恩賜蠲除,庶逃戶懷歸,安居樂業。」疏入,下部議,不許,上特命免徵。旋卒於官。上聞,諭廷臣曰:「傅臘塔和而不流,不畏權勢,愛惜軍民。兩江總督居官善者,於成龍而後,惟傅臘塔。」遣太僕寺卿楊舒赴江寧致祭,贈太子太保,謚清端,予騎都尉世職。士民懷之,為建祠江寧。四十四年,上南巡,經雨花臺,賜祠額曰「兩江遺愛」。雍正中,入祀賢良祠。

馬如龍,字見五,陜西綏德州人。康熙十一年舉人。十四年,陜西提督王輔臣據寧羌叛,其黨硃龍寇綏德,陷之。如龍糾鄉勇倚山立寨,寇至,屢擊卻之。輔臣誘以偽劄,斬其使。會平逆將軍畢力克圖兵至,如龍渡河迎,呈偽劄,並陳賊虛實,因率所部為前鋒,克綏德。畢力克圖以聞,即便宜令攝州事。總督哈占亦疏言如龍倡義拒賊狀,請優敘。

十六年,授直隸灤州知州。州民猾而多盜,如龍鋤暴安良,豪右斂跡。州有民殺人而埋其尸,四十年矣;如龍宿逆旅,得白骨,問之,曰:「此屋十易主矣。」縶最初一人至,鉤其情得實,置諸法。昌平有殺人獄不得其主名,使如龍按之。閱狀,則民父子殺於僧寺,並及僧五,而民居旁二姓皆與民有連,問之,謝不知。使跡之,二人相與語曰:「孰謂馬公察,易欺耳。」執訊之,乃服。自是民頌如龍能折獄。十九年,以察出民間隱地,敘勞,入為戶部員外郎,歷刑部郎中,榷浙江北新關稅務。

二十四年,遷杭州知府。杭州民貸於旗營,息重不能償,質及子女。如龍請於將軍,覈子母,以公使錢代償。杭州民咸頌如龍。二十八年,上南巡,聞其治行,超擢按察使。平反庶獄,多所全活。海賊楊士玉竄跡島嶼,勾土賊胡茂等剽掠商船,如龍設策擒之,盡殲其首從,巡撫張鵬翮以聞。二十九年,遷布政使,屬吏有歲餽,悉禁絕之。是年,紹興大水,庫儲絀,無可救濟。如龍檄十一郡合輸米二萬餘石,按戶賑給,告屬吏曰:「是逾於歲餽多矣,」

三十一年,授江西巡撫。整飭常平倉,春以羨米出貸,秋收還倉。飭州縣廣積儲,備兇荒。仿白鹿洞遺法,建書院以教士。嚴溺女之禁。疏請罷追轉漕腳耗。三十八年,入覲,賜御書「老成清望」榜。時淮、揚薦饑,如龍以江西連歲豐稔,率僚屬捐米十萬賑之。以老病累疏乞休,詔輒慰留。四十年,卒,賜祭葬。

論曰:守成世為大臣者,以仁心行仁政,培養元氣,其先務也。兵革初息,瘡痍未復,格爾古德等任封疆之重,拊循安輯,與民休息,政績卓卓在耳目。廷褒老成,野留遺愛,有以哉!


\end{pinyinscope}