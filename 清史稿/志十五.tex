\article{志十五}

\begin{pinyinscope}
災異一

傳曰:天有三辰,地有五行,五行之沴,地氣為之也。水不潤下,火不炎上,木不曲直,金不從革,土爰稼■,稼■不熟,是之謂失其性。五行之性本乎地,人附於地,人之五事,又應於地之五行,其洪範最初之義乎?明史五行志著其祥異,而削事應之附會,其言誠韙矣。今準明史之例,並折衷古義,以補前史之闕焉。

洪範曰:「水曰潤下。」水不潤下,則為咎徵。凡恆寒、恆陰、雪霜、冰雹、魚孽、蝗蝻、豕禍、龍蛇之孽、馬異、人痾、疾疫、鼓妖、隕石、水潦、水變、黑眚、黑祥皆屬之於水。

順治九年冬,武清大雪,人民凍餒;遵化州大雪,人畜多凍死。十年冬,保安大雪★月,人有凍死者;西寧大雪四十餘日,人多凍死。十一年冬,灤河大雪,凍死人畜無算。十三年冬,武強大雪四十日,凍死者相繼於塗;昌黎、灤州大雪五十餘日,人有陷雪死者。

康熙三年三月,晉州驟寒,人有凍死者;萊陽雨奇寒,花木多凍死。十二月朔,玉田、邢臺大寒,人有凍死者;解州、芮城大寒,益都、壽光、昌樂、安丘、諸城大寒,人多凍死;大冶大雪四十餘日,民多凍餒;萊州奇寒,樹凍折殆盡;石埭大雪連釂,深積數尺,至次年正月方消;南陵大雪深數尺,民多凍餒;茌平大雪,株木凍折。十一年三月,文水大雪嚴寒,人多凍死。冬,昌化大雪,平地深三尺。十五年十一月,咸陽大雪深數尺,樹裂井凍。十六年九月,臨淄大雪深數尺,樹木凍死;武鄉大雨雪,禾稼凍死;沙河大雪,平地深三尺,凍折樹木無算。二十二年十一月,巫山大雪,樹多凍死;太湖大雪嚴寒,人有凍死者。二十四年四月,定陶烈風寒雨,人有凍死者。二十七年,郝昌大雪,寒異常,江水凍合。二十八年冬,衢州大雪,寒異常。二十九年十一月,高淳大雪,樹多凍死;武進大寒,木枝凍死。十二月,廬江大寒,竹木多凍死;當塗大雪,橘橙凍死;阜陽大雪,江河凍,舟楫不通,三月始消;宜都大雪囗樹,飛鳥墜地死;竹谿大雪,平地四五尺,河水凍;三水大雪,樹俱枯;海陽大寒,凍斃人畜;揭陽大雪殺樹;澄海大雨雪,牛馬凍斃。三十年冬,房縣酷寒,人多凍死。四十二年春,房縣雨雪大寒。五十五年冬,高淳大雪盈丈。五十七年七月,通州大雪盈丈。十二月,太湖、潛山大雪深數尺。五十八年正月,嘉定嚴寒,太湖、潛山大雪四十餘日,大寒。

雍正五年冬,屯留大雪嚴寒,井凍。

乾隆五年正月,嵊縣大雨雪,奇寒;福山大寒。九年正月,曲沃大寒,井中有冰。十三年十二月,上海大寒雨雪。十六年三月,武強大雪,平地深尺許,人畜多凍死。二十二年正月,豐順雨雪大寒,人畜凍斃。二十四年冬,永年大寒。二十六年冬,福山大寒,樹多凍死;文登、榮成大雪寒甚;婁縣大寒,河冰塞路;臨朐大寒,井水凍;餘姚大寒,江水皆冰。五十七年六月,房縣大寒如冬。五十九年七月,湖州寒如冬。

嘉慶元年正月,青浦雨雪大寒,傷果植;灤州大寒井凍,花木多萎;永嘉大風寒甚,冰凍不解;湖州大雪,苦寒殺麥;義烏奇寒如冬;桐鄉大風雪寒。十二月,金華大雪,麥幾凍死。三年五月初五日,青浦大寒,廚★皆冰。十年十二月,棗陽大雪,結寒冰厚五尺。十九年秋,招遠、黃縣大寒,海凍百餘里,兩月始解。二十四年十二月,南樂大雪,平地深數尺,人畜多凍死。

道光十一年冬,元氏、南樂大雪,井凍,冰深四五尺。十一年十二月朔,撫寧大雪,平地深三尺,飛鳥多凍死。二十一年正月,登州府各屬大雪深數尺,人畜多凍死。冬,高淳大雪深五尺,人畜多凍死;黃川大雪深數尺,經兩月始消,民多凍餒;羅田大雪深丈餘,民多凍餒。

咸豐八年七月,大通大雪厚二尺,壓折樹枝,穀皆凍,秕不收。九年六月,青浦夜雪大寒;黃巖奇寒如冬,有衣裘者。十一年十二月,臨江府及貴溪大寒,樹多凍折;蒲圻大雪,平地深五六尺,凍斃人畜甚多,河水皆冰。

同治元年六月,崇陽大寒。冬,咸寧冰凍奇寒。四年正月十四日,三原大風雪,人多凍死;棗陽雨雪連旬,樹多凍死。十六日,鍾祥、鄖陽大雪;漢水冰,樹木牲畜多凍死。十二年十一月,三原大雪六十餘日,樹多凍死。

光緒二年五月,遂昌奇寒,人皆重棉。

順治四年三月,獻縣、肅寧昏霧,四晝晦。十四年十月二十八日,東陽大霧,竟日不散。十五年正月朔,潛江大霧,晝晦。

康熙元年三月初八日,臨榆昏霧竟日。十六年,清河陰霧四十餘日。二十年三月,桐鄉恆陰。二十二年三月,蕭縣重霧傷麥。三十年正月,江浦大霧蔽天。四十三年八月,慶雲昏霧障天。六十一年六月,濰縣濃霧如煙。

雍正二年十二月十五日,掖縣大霧。

乾隆元年十一月二十一日,海陽大霧。二十六年八月初四日,獨山州宿霧冥濛,近午始霽。三十三年二月十六日,梧州大霧。

嘉慶元年三月二十六日,宜城昏霧,晝晦。四年十二月朔,蓬萊大霧竟日,氣如硫磺。十五年正月,榮成大霧。

道光六年五月,肅州大霧。二十九年正月,雲夢晝晦六閱月,天氣陰霾。三十年正月朔,登州陰霧。

咸豐元年十二月除夕,泰安、通渭大霧。二年正月二十四日,陵川大霧,晝晦。

同治元年正月庚寅,青浦大霧,著草如棉,日午始散。二年正月二十四日,陵縣大霧,晝晦。三月十六日,涇州大霧,通渭、泰安大霧,至四日乃止。六年二月,日照大霧。

光緒十一年,邢臺大霧。

順治元年四月,棲霞隕霜殺麥。二年八月,垣曲隕霜。六年四月,莊浪隕霜殺麥。七年四月,靈臺隕霜殺麥。十五年四月,東昌隕霜殺麥。六月,高唐隕霜。十六年三月,榮河隕霜殺麥。十七年春穀雨後,岳陽霜屢降;萬全隕霜殺麥。

康熙二年四月二十三日,高苑隕霜殺麥。六月,雒南、商南隕霜三次。三年四月二十一日夜,清河風霜★作。二十三日,新城、鄒平、陽信、長清、章丘、德平隕霜殺麥。二十四日,益都、博興、高苑、寧津、東昌、慶雲、雞澤隕霜殺麥。十一年四月,樂安隕霜殺麥。五月,通州隕霜殺麥。七月,岢嵐州、吉州隕霜殺禾。十二年正月四日,壽光隕霜殺麥。十四年四月,平度、掖縣、萊陽、昌樂、安丘、館陶、濱州、蒲臺隕霜殺麥。五月,冠縣隕霜殺麥。十五年四月,武強隕霜殺麥。十七年春,碭山、潁上、銅山隕霜殺麥。十八年三月,無極隕霜殺麥。十九年四月,榆社隕霜殺菽。六月,沂州隕霜。八月,高州大雪。二十一年三月,太平隕霜殺麥。二十二年七月,靜樂隕霜殺禾。二十三年四月,儀徵、靜寧州隕霜殺麥。二十六年七月,西寧隕霜。二十七年三月,臨潼隕霜殺麥。七月,岳陽隕霜殺禾。二十九年三月,商州隕霜。四月,長治隕霜。三十年五月,長治隕霜。八月,武進隕霜殺稼。五月朔,平遠雨雪。六月,龍川隕霜殺禾。三十三年八月,懷來隕霜殺稼。三十四年七月,盂縣隕霜殺禾。八月十五日,嵐縣、永寧州、中★、絳縣、垣曲隕霜殺禾。三十五年八月,靜寧、介休、沁州、沁源、臨縣、陵川、和順、延安各處隕霜殺稼。三十六年七月,樂平、保德州隕霜殺禾。八月,岳陽隕霜殺禾;沁、涿霜災。九月,龍門大雪;西鄉隕霜殺稼。三十七年七月,陽高隕霜殺禾。四十四年三月,碭山、湖州大雪。六月,桐鄉、湖州大雪;狄道州隕霜殺禾。四十七年二月,鶴慶隕霜殺麥。四十八年七月,德州隕霜殺禾。五十年正月,潮陽隕霜。五十六年二月,涇陽隕霜殺麥。七月,通州大雪盈丈。五十九年七月,安定隕霜殺禾。八月,德州隕霜殺禾。六十年五月,臨朐隕霜殺麥。

雍正元年八月,懷安隕霜殺禾。二年八月,江浦隕霜殺稼。六年七月,甘泉隕霜殺禾。八年八月,沁州隕霜殺禾稼。九年八月,沁州復隕霜殺禾稼。

乾隆四年四月,通州隕霜殺麥。八年七月,無為大雪。八月初一日,東光隕霜殺禾。十年七月,廣陵隕霜殺禾。十二年六月丙子,蘇州雨雪,己卯、庚辰又微雪。十三年四月,同官隕霜殺麥。十六年四月,同官隕霜殺麥。九月,龍川隕霜殺禾。二十年七月,正寧隕霜殺禾。八月,葭州隕霜殺禾。二十七年七月,會寧、正寧隕霜殺禾。二十八年五月,和順隕霜殺稼。三十一年三月,高邑隕霜。五十一年五月,通渭隕霜殺麥。五十二年七月,宣平隕霜殺菽。五十五年三月,平度、鄒平、臨邑隕霜殺麥。四月,範縣隕霜殺麥。五十六年三月,壽光、安丘、諸城隕霜殺麥;平陰隕霜殺麥,數日後復發新苗。

嘉慶十年三月,東平、濟寧、莘縣隕霜殺麥。十三年四月,靖遠、樂清隕霜殺禾。十四年立夏前三日,江山雨雪。十九年八月,望都隕霜殺稼。二十二年八月,涿州、望都隕霜殺稼。二十五年八月,貴陽隕霜殺稼。

道光十二年四月,諸城隕霜殺麥。七月,望都、寧津隕霜殺禾。十五年四月,黃縣隕霜殺麥。十八年八月,元氏隕霜殺禾。十九年八月,狄道州隕霜殺禾。二十年七月,臨朐隕霜殺禾。二十二年四月初八,秦州隕霜殺麥。

咸豐九年二月,沁源隕霜殺麥。

同治九年二月,沁源隕霜殺麥。

光緒二年八月初八日,寧津、東光、臨榆隕霜殺禾。十八年四月,化平川隕霜。二十八年八月,莊浪隕霜殺禾。

順治元年,沙河大雨雹。二年三月,平樂雨雹,大如鵝卵。五月二十四日,武鄉雨雹,大如鵝卵;南雄雹,拔木。四月,文安大雨雹,傷麥。四年五月,岑溪雨雹,大如碗。五年二月,丘縣大雨雹。三月,海豐雨雹,小者如雞卵,損麥。閏三月三日,昆山雨雹,大如斗,破屋殺畜。六年六月,臨淄大雨雹;壽光大雨雹,平地深數尺,木葉盡脫。九月,定遠雨雹,傷麥。十月十五日,咸寧大雨雹,所過赤地。七年五月,應山大雨雹;信陽雨雹,傷麥。六月,武鄉雨雹,其形如刀。八年二月十六日,順德雨雹,大如斗,擊斃牛馬。五月,丘縣大雨雹;汾西雨雹,大者如拳,小者如卵,牛畜皆傷,麥無遺莖。七月,黎城雨雹,大如鵝卵。九年四月二十三日,潞安雨雹,大如雞卵,屋瓦俱碎;長治雨雹,大如雞卵。五月十六日,臨縣雨雹,大如雞卵,積地尺許;嵐縣大雨雹,傷禾;膠州雨雹,大如雞卵。六月,臨縣雨雹,陽曲雨雹,大如鵝卵。十年四月四日,貴池雨雹,大如碗,屋瓦皆碎;武寧雨雹如石,殺鳥獸;崇陽雨雹,人畜樹木多傷。五月,海寧雨雹,大如雞卵,屋無存瓦,樹無存枝;涇陽雨雹,大如拳;永壽雨雹,大如拳,小如卵,積地五寸,二日始消,大傷禾稼。十月十日,袁州雨雹,大如栲栳者甚多,有一雹形如杵,長可一丈一尺有奇。十一年二月十五,蒼梧大雨雹。三月,松滋大雨雹。五月,長樂雨雹;漢陽雨雹,大如雞卵,平地深一尺。六月,雒南大雨雹,積地尺許,人不能行。十四年六月初三,猗氏大雨雹;霸、薊等七州,寶坻等二十一縣雨雹。十五年三月,寧波大雨雹,擊斃牛羊,桑葉盡折;鎮海大雨雹。閏三月朔,上虞、龍門大雨雹,★忽高尺許,或如拳,有巨如石臼,至不能舉,人畜多擊死。十六年閏三月初四,順德大雨雹。四月,蕭縣大雨雹,殺麥。六月,清澗雨雹,大如鵝卵。八月,膠州雨雹,傷稼。九月,新河雨雹,傷數十人,至三月始消。十七年四月壬寅,清河雨雹,大如斗。十一月,鶴山雨雹,大如雞卵。十八年正月二十七日,順德大雨雹,傷人畜;揭陽雨雹,大如拳,屋瓦皆碎。三月初六,萍鄉雨雹,其狀或方或圓,或如★,屋瓦皆碎。八月,懷安雨雹,大如雞卵,厚盈尺。冬,清澗雨雹,大如鵝卵,有徑尺者,積數尺。

康熙元年三月二十一,海寧大雨雹;河間雨雹,大如斗。五月,懷安大雨雹,人畜有傷;龍門大雨雹;榆社大雨雹,人畜多傷。二年正月二十八,望江雨雹。二月,安陸雨雹。三月朔,襄陽雨雹。四月十六日,鎮洋大雨雹。六年六月,香河雨雹,大如碗,平地深數尺,田禾盡傷,屋瓦皆碎,遠近數十里。八月,保安州大雨雹,傷人畜;宣化大雨雹,傷禾;懷來大雨雹,傷人畜。七年五月,新安雨雹,大如甑,屋舍禾稼盡傷。十二年三月,行唐大雨雹。七月,盧龍雨雹,大如斗。十七年三月,連山雨雹,大如拳,擊死牛畜。十八年正月,惠州雨雹,大如拳。十九年七月,陽曲雨雹,大如雞卵,有大如磑碾者,擊死人畜甚多。二十六年四月,平湖雨雹,大如升,小如拳。六月二十四日,文縣雨雹,大如雞卵,割之,內有小魚、松苔。三十三年二月,開平大雨雹。五月,汶上雨雹,大者徑尺,擊死數人。三十四年三月,江夏雨雹,大如豆,中有黑水。三十六年閏三月,黃安大雨雹。四月,湖州大雨雹,三十七年正月十六,靈川雨雹,大如雞卵;安南雨雹,大如拳,麥無收。三十九年七月,元氏大雨雹。四十年二月,鶴慶大雨雹。四十二年三月,桐鄉大雨雹,損菜菽;湖州大雨雹;龍門大雨雹,或如拳、如臂、如首,或長或短,或方或圓,積深二三尺,壞民居無算,虎豹雉兔斃者甚多;崖州大雨雹,如霜,著樹皆萎;蒲縣雨雹。四十三年六月,翁源大雨雹;蒲縣雨雹,傷禾。七月,定襄雨雹,傷禾。四十四年三月,桐鄉大雨雹;湖州大雨雹。八月,密雲雨雹,傷禾。四十六年二月,湖州雨雹。三月初四日,陵川雨雹;瓊州雨雹,大如拳。六月,東明大雨雹,麥盡傷。四十八年二月初六日,荔浦雨雹,大如鵝卵,積地尺許。夏,大埔雨雹,白如繭,積地數尺;江浦、來安雨雹。五月十六日,雞澤大雨雹,傷人百餘。秋,代州雨雹。五十一年五月,解州雨雹;沁源雨雹,大如雞卵。七月,黃岡雨雹,擊斃人畜。五十二年三月二十七日,全州大雨雹,屋瓦皆飛。五十三年五月,固安西雨雹。七月朔,平大街雨雹,傷禾。五十四年三月,江浦雨雹,大如升,傷麥。五十五年夏,新樂雨雹,大如碗;浮山雨雹,大如雞卵,田禾盡傷;崇陽雨雹。五十七年三月,龍川雨雹,大如斗,壞民舍,牛馬擊斃無算。五十九年六月,雞澤雨雹,大如雞卵,傷禾。六十年三月,連平雨雹,毀民舍;鎮安、慈谿、上虞、餘姚雨雹,小者如碗,大者如斗。七月,柏鄉大雨雹,如雞卵,傷禾。六十一年四月,平定、樂平冰雹。五月,安丘大雨雹。十一月初十日,香山雨雹。十二月,贛州雨雹,大如雞卵,傷牲畜。

雍正元年正月,鶴慶大雨雹。三月,融縣雨雹。二年五月,福山雨雹,大如雞卵。八月,秦州雨雹,擊斃牛馬鳥雀無算;東安雨雹,傷稼。三年正月,定州大雨雹。三月,昆山大雨雹。夏,長寧雨雹,大如雞卵,傷鳥獸甚多。四年正月,甘泉雨雹,大者如斗,小者如升,屋舍盡毀。三月,吳川雨雹。五月,舒城雨雹,大如斗。六年五月,商南大雨雹,損屋舍。七年四月,惠來大雨雹,如雞卵,傷禾。高平、岑溪雨雹,樹皆折。七月,靜寧州雨雹,大如碗。八年六月,安遠大雨雹,擊斃禽畜甚多。八月,海寧、沁州大雨雹,毀屋舍。十年二月,連州大雨雹,損麥。八月,白水雨雹。九月,湖州、桐鄉大雨雹。十一年三月,海寧雨雹;桐鄉雨雹,傷麥。八月,陽信雨雹,大如雞卵,深三尺餘,田禾盡損。冬,嘉興雨雹,傷麥。十二年三月,無為大雨雹;鶴慶大雨雹;蒲圻大雨雹。四月,湖州雨雹,損麥。

乾隆元年二月,廣州大雨雹。三月,榮經冰;方山大雨雹。五月十七日,青城雨雹,大如胡桃。六月,鄖西雨雹,鳥獸多擊死。七月二十五日,南和大雨雹;平鄉大雨雹,毀房廬,傷田禾;懷安雨雹,傷禾。九月,長子大雨雹,片片著禾如刈。十一月,京山雨雹。三年正月十四日,武寧雨雹,大者重四五斤。四月,白水大雨雹,傷麥。四年三月,北流雨雹;富平、臨清雨雹,傷禾。四月丙戌,蘇州大雨雹,損麥;昆山大雨雹,損麥。五年六月,絳縣雨雹,傷禾。六年秋,廣靈雨雹,傷稼。七年三月,畢節雨雹,大如雞卵。四月,涿州雨雹,大如雞卵。八年四月初五日,安州雨雹,大如雞卵,深三尺。初九日,昆山大雨雹,損麥。閏四月,高邑大雨雹。七月,高苑大雨雹,傷麥。十年五月,涿州雨雹。初八日,青城雨雹,大如酒杯。六月丁未,同官雨雹,大如彈;戊午,又雨雹,壞廬舍無算。八月,慶陽大雨雹,傷禾。十一年三月,禮縣大雨雹。四月,金鄉、魚臺、莒州雨雹,大如雞卵,傷麥。五月,曲沃雨雹,大如車輪。十二年六月十一日,高平、文鎮大雨雹,傷稼。七月二十五日,安化雨雹,傷禾。十三年正月初二日,鶴慶、信宜、象州、恩縣、遂安雨雹,大如斗,傷麥。四月初四日,上海雨雹,傷麥豆;昆山大雨冰雹,擊死人畜無算。五月十一日,泰州、通州大雨雹,壞屋。十三日,滕縣大雨雹,大如臼,民舍損壞無算。六月,樂平雨雹,傷稼。秋,懷來、懷安、西寧、蔚州、保安雨雹成災。十二月,忠州、西鄉大雨雹,傷禾。十四年二月初七日,忠州雨雹。四日,太平雨雹。六月朔,高邑大風雹。十月,樂平、稷山雨雹,傷禾。十一月,正定府屬雨雹,傷稼。十五年五月,彭澤大雨雹,重三十餘斤。五月初四日,宜昌大雨雹。六月十五日,膠州、濱州大雨雹,傷人畜禾稼。八月戊子,白水雨雹,傷稼。九月,鄖縣、房縣大雨雹,傷人畜。十二月,信豐大雨雹。十六年三月,榮成大雨雹。十八年四月二日,定番州大雨雹,壞民舍百餘間。二十年三月,黃岡雨雹,長三十餘里,大者徑尺。四月初三,玉屏大雨雹,壞屋。五月十七日,高平大雨雹,人有擊斃者。二十一年六月,潮陽大雨雹,周遭二十餘里,禾稼多傷。二十二年八月,即墨大雨雹,深尺許。二十三年三月,龍川大雨雹;東湖雨雹,大如卵,積盈尺者十餘里。四月二十九日,永平大雨雹,形如★,人有擊斃者。五月,中部雨雹,大如卵,厚尺許;莊浪、環縣大雨雹。六月十六,長子大雨雹,十一日方止。二十三年三月,宜昌雨雹,大如卵,積地盈尺。四月,陵川雨雹,大如雞卵,深盈尺。十一月,武寧大雨雹,重五六斤。二十八年十月,羅田雨雹。二十九年二月,慶元大雨雹。三十年三月,臨邑大雨雹,鳥獸死者相枕藉。六月二十四日,樂平雨雹,傷稼。三十一年五月,鄞縣冰雹。三十二年五月,邢臺大雨雹,深尺許。三十三年四月,莒州大雨雹。三十五年五月二十三,東平大雨雹。三十九年二月,樂平雨雹,傷麥。五月,黃縣大雨雹,厚積數寸。四十年三月十七日,屏山大雨雹。四十二年六月二日,壽陽雨雹,深者四尺,淺者二尺,月餘方消。四十三年五月,房縣雨雹,或方或圓,或如磚,傷人畜無算;合肥大雨雹。四十四年四月,平度大雨雹。五月,黃縣雨雹,傷麥。四十七年四月戊子,寶雞雨雹,傷麥。五月,文登大雨雹,傷禾。五十年二月二十三,瀘溪雨雹。三月,潛江雨雹。五十四年五月初四,洛川大雨雹。五十五年二月,荊州大風雹。四月初六,青浦雨雹,大如拳,擊死一牛。八月,江陵大雨雹。五十六年二月,永安州大雨雹。十月初八日,東光大雨雹。五十七年五月三日,泰州大雨雹;莒州雨雹,大如鵝卵,厚三尺,傷禾稼;禹城、陵縣、壽光大雨雹。七月,黃縣大雨雹,傷禾。五十八年三月,武寧雨雹,壞民舍。五十九年四月,黃州雨雹,大如★,人畜多擊斃。十二日,江山大雨雹。

嘉慶元年正月,平谷大雨雹,形如雞卵。四月,邢臺雨雹,大者如斗。二年六月,中部雨雹,大如卵,小如杏,傷人畜;枝江雨雹,大如雞卵,鳥獸擊傷。十二月,雲和冰雹,大如斗,屋瓦皆碎。三年四月,宜城雨雹,大如雞卵。四年四月,襄陽大雨雹。五年四月,白河縣雨雹,大如雞卵,深尺餘。五月,滕縣雨雹,大如碌碡。七月,延安大雨雹,屋瓦皆碎,秋禾無存。六年五月,博興大雨雹,壞官民舍。十年八月,中部雨雹,大如卵,積地五寸。十一年,灤州大雨雹,平地積尺許。十二年二月,貴陽雨雹,大如馬碲。四月,沂水大雨雹,如杯者盈尺,有大如碌碡者。八月,武強大雨雹,有如鵝卵者,屋瓦皆碎,禾葉盡脫;邢臺雨雹。十三年春,武強大雨雹。十四年四月,薊州雨雹,傷麥;襄陽大雨雹;荊門州大雨雹。六月,南樂雨雹,大如雞卵。十五年三月,宜都雨雹,傷麥。八月,章丘雨雹;東光雨雹。十六年三月,枝江大雨雹。十七年三月,宜都雨雹,禾盡傷。秋,博野雨雹,成災。二十一年四月,棲霞雨雹,傷麥;定遠大雨雹,鳥獸多斃。二十二年五月,滕縣雨雹,平地深半尺,禾黍盡傷。二十三年五月,蘇州大雨雹;湖州大雨雹。

道光四年五月,日照大雨雹,傷禾。十月,曲陽大雨雹盈尺。五年四月八日,羅田雨雹,損麥豆無數;蘇州雨雹。五月,皋蘭大雨雹。八月初九,復雨雹。六年四月十七,雲夢雨雹,大如拳。七年十二月,宣平雨雹,折樹碎瓦。十三年五月癸未,臨朐雨雹,大如馬首。秋,博野等十三州縣雨雹。十月二十四日,宜城雨雹。十四年四月三日,三原雨雹,傷禾。初六,諸城雨雹,傷麥。十六年二月十六日,湖州大雨雹。四月二日,光化大雨雹。十七年七月十三日戊子,平谷大雨雹,如雞卵,秋禾盡平,屋瓦皆碎。十八年五月,通渭大雨雹。七月十八日丁巳,灤州雨雹,大如卵,秋禾盡損。十九年三月,元氏雨雹,厚尺許。二十年四月,黃安大雨雹,傷稼。七月二十六日,隨州大雨雹,傷稼。二十一年八月,陵縣大雨雹。九月又雨雹。二十三年五月二十二日,孝義大雨雹,狀如磚,有重數十斤者,人畜觸之即斃。七月十二日,雲和雨雹,大如斗,屋瓦皆碎,損傷人畜甚多。二十五日,安定雨雹,大如雞卵,山巔有徑尺者,數日不化。二十六日,隨州大雨雹,禾稼多傷。二十五年正月,崇陽大雨雹。四月,安丘大雨雹,損麥,三日不消。二十七年春,龍川大雨雹。夏,黃巖大雨雹。二十九年四月,應山雨雹,大如拳,鳥雀多擊死。六月,興山大雨雹,傷稼。七月,西寧大雨雹。三十年三月,黃岡雨雹,大如瓜,小如彈丸,壞稼傷人。

咸豐元年三月甲子,大雨雹,傷人畜,壞屋宇;懷來大雨雹。五月丙午,東光大雨雹,屋瓦皆毀,傷人畜。三年三月,崇仁雨雹,大者如★,小者如拳,屋瓦盡毀。四年四月,黃安雨雹,重十餘斤,損麥。九年七月,黃岡雨雹,大如卵。十年七月,羅田大雨雹,傷禾無數;麻城雨雹,大如雞卵,擊斃牛馬;黃安大雨雹,樹俱折。十一年十一月,麻城、羅田、宜都雨雹,大如雞卵,傷禾稼,損屋舍。

同治元年六月,東湖大雨雹,擊斃牛馬無算。六月,狄道州雨雹,大如雞卵,禾蔬盡傷。二年五月,元氏雨雹,大如拳,禾稼盡傷,田廬俱損。六月,孝感雨雹,大如雞卵。四年正月十三日,日照大雨雹,傷禽獸;武昌、黃囗、宜都雨雹,大如雞卵。二月,青田大雨雹,損麥。四月,均州雨雹,大如雞卵,破屋折樹。五月,房縣大雨雹,數百里禾稼盡傷。五年正月,均州大雨雹,積地深數尺。四月,隨州、江陵大雨雹,損麥。五月,通渭、泰安大雨雹,傷牛馬。六年七月,懷來、青縣大雨雹,秋禾損。九月十六日,高淳雨雹,大如拳,損屋舍。七年三月十八日,黃安、江夏大雨雹,鳥獸多擊死。八年五月十一日,肥城雨雹,平地深尺許,大如鵝卵。八月十三日,灤州大雨雹,闊十五里。九月,泰安大雨雹。九年三月十四日,潛江雨雹,大如雞卵。五月二十三日,階州大雷電,雨雹如注。十年二月,青田大雨雹。四月,上饒大雨雹。五月二十二日,階州、白馬關大雨雹,平地水深數尺,淹斃二百餘人。十一年二月,新城大雨雹。三月十一日,嘉興大雨雹;柏鄉大雨雹,重者十七斤;湖州大雨雹;景寧雨雹,大如★;青田尤甚。十三年三月,黃岡雨雹,大如升,數十里麥盡損。四月乙未,青浦雨雹,有重至十餘斤者。

光緒元年四月二十二日,邢臺雨雹,大如核桃,積地二寸許。二年四月,惠民大雨雹,鳥雀多擊死。三年四月十五日,沔縣雨雹,大如雞卵。六年夏,均州雨雹,大如雞卵。八年四月十一日,均州雨雹,大如鵝卵,袤百餘里,廣十餘里;二十五日,復雨雹,災尤重。八月,皋蘭雨雹,大如雞卵。九年七月初四日,山丹雨雹,大如雞卵。九月初二日,孝義雨雹,大如雞卵。十年五月二十五日,興山大雨雹,傷稼。八月,灤州大雨雹。十二年五月二十日,莊浪大雨雹,無極大雨雹。十四年六月十三日,新樂大雨雹,三十村禾盡損。十五年五月,化平川雨雹如蛙形,傷禾稼。十九年五月,狄道州雨雹,大如碗。二十二年九月,南樂大冰雹。二十四年四月二十四日,涇州雨雹,大如雞卵。五月,河州大風雨雹,平地水深三尺。二十五年五月初五日,海城雨雹,大如雞卵,擊死牛羊一千有餘。二十六年八月,南樂大雷雨雹。三十年七月,山丹大雨雹,傷禾。三十一年七月二十四日,洮州雨雹,大如雞卵,傷禾。

順治二年正月初一日,上元大雪雷雨。三年五月初一日,齊河雷火焚孔子廟;夏,陽城大雷,人有震死者。十二月初二日,吳川雷鳴,岑溪大雷雨。五年十月,揭陽大雷雨霹靂。六年正月十一日,潞安雨雪雷電。五月初九日,石門大雷雨;安丘雷擊二人。十一月二十五日,鎮洋大雷電。七年正月二十七日,震澤大雷電。八月,河源雷震大成殿。冬至後二日,解州雷鳴。十二月除夕,上元大雪雷電。九月,涇陽雷震,十月朔,雷;江陰雷;蕭縣雷。十二月二十八日,膠州雷。除夕,昆山雷,臨邑雷震。九年正月朔,黃陂震雷大雪,蘄水震雷大雪,應山大雷電。二月二十六日,石門雷震死三人。十月十五日,杭州大雷電。十六日,揭陽雷大震。十一月十四日,上海大雷,凡震三次;青浦雷。十二年九月,震澤雷電大雨。十月二十五日,香山雷鳴;二十七、二十八日,復鳴。十二月十三日,遂安雷震柏山庵。十三年二月二十九日,鍾祥震雷。十月,安丘雷震大雨。十四年正月十四日,遼州雷電大震。十一月十一日,永安州大雷電。十八日,杭州大雷電,銅陵雷。三十日,咸寧大雪雷電。十五年十一月,咸陽大雪雷鳴。十六年十二月,高淳大雷。十七年十一月,鶴山大雨雷電。十八年正月十七日,陽信、海豐大雷。

康熙元年二月,鶴慶雷鳴。三年正月,通州迅雷達旦,望江雷擊南城樓。五年十二月,封州雷鳴。六年正月,南樂迅雷。十二月,開平大雷雨,鳳凰洲雷鳴,揭陽雷,澄海雷鳴,欽州雷電。七年八月,平遠州雷擊右營守備署。八年二月十四夜,思州雷火起大成殿北角。十一月,西充大雷電。十二月,黃巖大雷。九年正月,烏城大雷電。十二月,蘇州雷電。七月,東陽大風雨雷電。十二年正月初六日,富陽大雷電。十月十四日,貴州雷,東流大雨。十月,婺源雷震儒學櫺星。十二月,雷震孔子廟戟門。十三年正月,蘇州雷,青浦雷,嘉定震雷自四鼓達旦。十二月除夕,桐鄉雷電交作。十六年正月初一日,湖州雷震大雪。十七年正月,巢縣雷。十八年正月朔,蘇州震雷,沛縣雷。十九年正月朔,蘇州雷。二十年正月,宿州雷雨雹。二十一年正月,宿松雷電。二十二年正月,解州雷電,石門雷電。二十三年正月,丹陽雷電雨雪,含山大雷電,兗州雷震。二十四年正月十七日,巢縣雷。二十五年十一月,信宜雷鳴。二十八年正月,沛縣雷電。十一月初九日,義烏大雷電。十二月十六日,巢縣大雷。三十一年正月,武進雷電。三十三年正月初十日,巢縣雷。十二日,瓊州雷鳴。十四夜半,萊縣大雷電。十二月,青浦雷電大雨。三十四年正月初一日,瓊州雷。三十六年正月初一日,昆山雷,青浦雷。三十七年十二月除夕,開平雷鳴。三十八年十二月初三日,吳川雷大震,次日又震。三十九年正月,解州雷。四十二年十二月,湖州雷。四十三年正月二十日,蘇州雷鳴。十一月二十八日,欽州雷鳴,揭陽雷鳴。十二月,澄陽雷鳴,普宣雷鳴。四十五年正月,巢縣大雷。四十九年正月初七日,香山雷鳴。十一月二十一日,景寧雷鳴。五十年十一月,大埔雷。十二月,陽春雷鳴。除夕,平樂雷電霹靂,驟雨達旦。五十三年十二月,涇州大雷電,福山大雷雹,湖州雷鳴。五十四年正月朔,大埔雷鳴。十一月,阜陽雷鳴。五十五年十月,通州雷。十一月,銅陵雷震。五十六年正月,湖州雷。十月二十五日,香山雷鳴。五十九年七月,南籠大雷雨。九月,通渭縣暴雷,震死一人。六十年十一月,潮陽雷鳴,岑溪雷鳴。六十一年十一月,順德大雷,廣寧雷電。十二月,欽州雷電大作,風雨暴至,吹塌城垣二十餘丈;陽春大雷雨,揭陽雷鳴,澄海雷鳴。

雍正十二年二月初八日,蒲圻大雷電。十月,揭陽雷鳴。

乾隆元年三月,邢臺雷震府學奎星樓,海陽震雷霹靂。二年十一月,贛縣大雷電。十二月二十五日,普寧雷鳴。九年二月十一日,昆山雷擊馬鞍山浮圖末級。十月十五日,岐山雷電風雨。十年四月十五日,橫州雷擊大成殿西柱。十二年四月二十五日,順潭村狂風迅雷大作,樹木盡拔,倒屋二千餘間,壓斃三十餘人。十月朔,膠州雷。十三年十二月初八日,上海大雷。十四年十二月,信豐大雷電雨雹,畢節大雷電。十六年十月,平度、海鹽震雷。十七年五月十一日,長子大風雷。十一月二十七日,揭陽雷鳴。十八年十二月,宜都大風雨雷電。二十年正月,贛榆大雷電雨雪。二十二年除夕,龍川雷鳴。二十四年十一月十一日,荊門州大雷。二十八年十月初四日,武進大雨雷電。三十五年十二月,嘉善雷電。三十六年十二月戊寅,蘇州大雷電,湖州雷電。三十九年十月,陽湖大雷。四十年十一月初六日,房縣大雷電。四十六年十二月十二日,桐鄉雷電。五十五年十二月二十四日,黃巖大雷雨,蘇州大雷電。除夕,雲夢大雪大雷。五十七年十二月乙巳,南陵雷電交作。五十九年十二月,江山大雷電。

嘉慶六年正月,陽湖雷。七年正月十七日,東光雷電。十年十一月,滕縣大雪聞雷。十九年十二月,滕縣雨雪聞雷。二十年二月,湖州雷電大雪。二十三年二月,金華雷電。

道光三年正月十四日,湖州雷。二月十五日,監利大雷。五年冬至後一日,章丘雷。七年十二月,湖州大雷雪。十二月,崇陽大雨雷電。十二年十二月,麗水雷電大雪。十三年十二月,宜城雷電雨雹。十八年十月,太平大雷。十一月二十九日,雲夢大雷。十二月除夕,湖州大雷電;青浦大雷電;隨州大雷。十九年正月,湖州大雷,棗陽雷鳴,雲夢雷鳴。十二月,文登大雷。除夕,靖遠雷電大雨。二十年正月十六日,武定大雷震。二十一年正月三十日,定遠雨雪雷鳴。二十二年冬至夜,滕縣雷鳴。二十三年十月,應城雷電。二十四年十月,崇陽雷。十二月,鄱陽大雨雷電,麗水雷電,即墨雷電。二十五年正月,崇陽大雷電。七月,榆林雷震。二十六年五月二十五日,滕縣雷火焚城南樓,貴陽雷。二十七年冬,武昌雷震,黃岡大雷。二十九年六月,武昌雷震。

咸豐二年八月,崇陽雷鳴。四年正月十三日,平鄉雨雪雷鳴。十月初六日,應山雷。八年十一月二十三日,南安雷震。十一年十月初一日,東光雷電。十一月十二日夜,宜都大雷。

同治元年冬,方山雷震北峰塔。二年正月初十日,定遠雷鳴。三年正月,青浦雷。四年正月十三日,平鄉雨雪雷鳴,震教堂;東光大雨雪雷電;永嘉大雷電;太平雪中聞雷;武昌震雷;黃岡震雷;隨州雷電;麻城震雷;棗陽大雨迅雷;陵縣大雷電;日照大雷電;房縣雷;曹縣大雷電;菏澤大雷電。五年正月初八日,均州雷電,鄖縣大雷電,房縣雷。十四日,孝義大雪雷電。十一年十一月,臨榆大雨雷震。

光緒元年八月甲戌,青浦雷震南門塔。五年十一月十五日,京山大雷雨,安陸大雷電;夜,蘄水雷電四次。八年八月二十八日,玉田大雷,自二更徹夜。十一年十月二十日,東光聞雷。十二年十一月二十八九兩夜,德安大雷。十三年正月,德安大雷而雨。

順治八年二月,柴胡塞出大魚,長十丈餘,形似海★。

康熙元年正月朔,臺州見二巨魚★於江內,三日,其一死,肉重四百餘斤。三年三月,萊陽羊圈口潮上巨魚,長六丈餘,聲如雷,旋死。五年三月三日,綏德州天雨魚。十一年,海康鯨魚入港,長五丈,闊二丈,以千人拽之岸。十五年十二月,海鹽有大魚,長十丈餘,形如車輪,頭似馬首。二十一年六月,綦江縣雨魚。二十二年四月,海寧海濱有魚長二十餘丈,無鱗,有白毛,土人呼之為海象。二十六年四月,文縣雨魚。三十四年七月,嘉定有二巨魚★於海,聲如雷,其一死者虎首人身,長丈餘,腥聞數里。四十二年八月,青浦龍安橋下有二大魚上游,形如船,旁有小魚無數。四十七年二月初,臺州有巨魚至中津橋,向人朝拜,十二日隨潮而逝。

乾隆五年,黃縣海出大魚六丈,其骨專車四。十三年,涪州彈子溪巨魚見,長約丈餘,相傳歲歉則上,是歲果大荒。二十六年三月二十三日,平湖海濱來一大魚,其聲如牛,長六丈七尺,徑一丈四尺。

咸豐四年五月,黃巖有巨魚數十入內港,色黃如牛,大者重五六斤。六年六月,平湖金門山一魚死海濱,取得一齒,形如★,重十三斤。十一年,平湖鯉魚數十頭從空飛過。

順治三年七月,延安蝗;安定蝗;欒城蝗,蔽天而來;元氏蝗,初蝗未來時,先有大鳥類鶴,蔽空而來,各吐蝗數升;渾源州蝗。九月,洪洞蝗,宣鄉蝗。四年三月,元氏、無極、邢臺、內丘、保定蝗。六月,益都、定陶旱蝗,介休蝗,山陽、商州雹蝗。七月,太谷、祁縣、徐溝、岢嵐蝗;靜樂飛蝗蔽天,食禾殆盡;定襄蝗,墜地尺許;吉州、武鄉、陵州、遼州、大同蝗;廣靈、潞安蝗;長治飛蝗蔽天,集樹折枝;靈石飛蝗蔽天,殺稼殆盡。八月,寶雞蝗,延安蝗,榆林蝗,涇州、莊浪等處蝗。九月,交河蝗,落地積尺許。五年五月,衡水蝗。六年三月,陽曲蝗,盂縣蝗。五月,陽信蝗,害稼。六月,德州、堂邑、博興蝗。七年七月,太平、岢嵐蝗,介休、寧鄉蝗。十年十一月,文安、府穀蝗。十三年正月,徐海蝗。三月,玉田大旱蝗。五月,定陶大旱蝗。七月,昌平、密雲、新樂、臨榆蝗,灤河蝗,東平蝗。冬,昌黎大雨蝗。十五年三月,邢臺、交河、清河大旱蝗,害稼。

康熙四年四月,東平、真定、日照大旱蝗。五年五月,蕭縣蝗;任縣飛蝗自東來蔽日,傷禾;日照、江浦大旱蝗。六年六月,杭州大旱蝗;靈壽、高邑大旱,蝗,害稼。八月,東明、灤州、靈壽蝗。八年八月,海寧飛蝗蔽天而至,食稼殆盡。九年七月,陽囗大旱蝗,食稼殆盡。麗水、桐鄉、江山、常山大旱蝗。六月,寧海、天臺、仙居大旱蝗,定陶大旱蝗,虹縣、鳳陽、巢縣、合肥、溧水大旱蝗。七月,全椒、含山、六安州、吳山大旱蝗,濟南府屬旱蝗害稼,麗水蝗,桐鄉、海鹽、淳安大旱蝗,元城、龍門、武邑蝗。十一年二月,武定、陽信蝗害稼。三月,獻縣、交河蝗。五月,平度、益都飛蝗蔽天,行唐、南宮、冀州蝗。六月,長治、鄒縣、邢臺、東安、文安、廣平蝗。定州、東平、南樂蝗。七月,黎城、芮城蝗,昌邑蝗飛蔽天,莘縣、臨清、解州、冠縣、沂水、日照、定陶、菏澤蝗。十六年三月,來安蝗,三河、內丘蝗。十八年正月,蘇州飛蝗蔽天。夏,全椒蝗。七月,寧津、撫寧、五河、含山蝗。二十一年,信陽、莒州蝗。二十三年四月,東安蝗,永年蝗。二十五年春,章丘、德平蝗。六月,平定、無極、饒陽、井陘蝗。二十六年,東明、★城蝗。二十九年五月,臨邑、東昌、章丘蝗。七月,平陸、武清蝗。三十年五月,登州府屬蝗。六月,浮山、翼城、岳陽蝗,萬泉飛蝗蔽天,沁州、高平落地積五寸,乾州飛蝗蔽天,寧津、鄒平、蒲臺、莒州飛蝗蔽天。七月,昌邑、濰縣、真定、盧龍、平度、曲沃、臨汾、襄陽蝗,平陽、猗氏、安邑、河津、蒲縣、稷山、絳縣、垣曲、中部、寧鄉、撫寧等縣蝗。三十一年春,洪洞、臨汾、襄陵、河津,夏,浮山蝗。三十三年五月,高苑、樂囗、寧陽蝗。三十六年,文安、元氏蝗。三十八年,遵化州、晉州、盧龍、撫寧蝗。三十九年秋,祁州、盧龍、撫寧蝗。四十三年,武定、濱州蝗。四十四年九月,密雲、盧龍、新樂、保安州蝗。四十六年,邢臺、肅寧、平鄉蝗。四十八年秋,昌邑、盧龍、昌黎蝗。五十年夏,莘縣、鄒縣、廬州蝗。五十三年秋,沛縣、合肥、廬江、舒城、無為、巢縣蝗。五十七年二月,江浦、天鎮蝗。五十九年,膠州、掖縣蝗。

雍正元年四月,銅陵、無為蝗,樂安、臨朐大旱蝗,江浦、高淳旱蝗,棲霞、臨朐蝗。三年冬,海陽、普宣蝗。十三年九月,東光、獲鹿、蒲臺蝗。

乾隆三年六月,震澤、日照旱蝗。四年六月,東平、寧津蝗。五年八月,三河飛蝗來境,抱禾稼而斃,不為災。九年七月,阜陽、亳州、滕縣、滋陽、寧陽、魚臺蝗。獻縣、景州蝗。十三年夏,蘭州、郯城、費縣、沂水、蒙陰旱;諸城、福山、棲霞、文登、榮成蝗;高密、棲霞尤甚,平地湧出,道路皆滿。十五年夏,掖縣飛蝗蔽天。十六年六月,諸城、交河、祁州蝗;河間蝗,有鳥數千自西南來,盡食之。十七年四月,柏鄉、雞澤、元氏、東明、祁州蝗。七月,東阿、樂陵、惠民、商河、滋陽、範縣、定陶、東昌蝗。十八年秋,永年、臨榆、樂亭蝗。二十年六月,蘇州大雨蝗。二十三年夏,德平、泰安蝗,有★鳥食之,不為災。二十四年夏,高郵大旱,蝗集數寸。二十八年三月,臨邑、靜海、灤州、文安、霸州、蒲臺飛蝗七日不絕。二十九年夏,吳川大旱,蝗損禾;東昌、安丘蝗。三十年三月,黃安、寧陽、滋陽蝗。三十三年七月,武清、慶雲蝗。三十七年二月,景寧飛蝗蔽天,大可駢三尺;淄川、新城蝗;鳳陽旱蝗,三十九年二月,安丘、壽光、沂水蝗。八月,文登蝗。四十三年三月,黃安、南陵旱蝗。九月,武昌蝗;江夏縣、潛江大旱蝗。四十九年冬,濟南大旱蝗。五十年六月,日照縣大旱,飛蝗蔽天,食稼;蘇州、湖州、泰州大旱蝗。五十一年五月、七月,房縣、宜城、棗陽、陽春旱蝗;羅田、麻城大旱蝗。五十二年四月初二日,麻城蝗,積地寸許。七月,黃岡、宜都、麻城、羅田、荊門州蝗。五十三年六月,平度縣大旱,飛蝗蔽天,田禾俱盡。五十六年六月,寧津、東光大旱,飛蝗蔽天,田禾俱盡。五十七年五月,武城、黃縣、高唐旱蝗。五十八年春,歷城旱蝗,有蟲如蜂,附於蝗背,蝗立斃,不成災。七月,安丘、章丘、臨邑、德平蝗。

嘉慶七年,蓬萊、莘縣、高唐、鄒平、諸城、即墨、文登、招遠、黃縣蝗。十年春,博興、昌邑、諸城蝗;臨榆蝻生。夏,滕縣飛蝗蔽天,食草皆盡。秋,昌邑蝗,食稼;寧海蝗。十九年,菏澤、曹縣、博興蝗。

道光三年,莘縣、撫寧蝗。四年,東平、清苑、望都、定州蝗。五年七月,清苑、定州飛蝗蔽天,三日乃止;內丘、新樂、曲陽、長清、冠縣、博興旱蝗。六年二月,灤州、撫寧蝗。十四年五月,潛江、棗陽旱蝗,雲夢旱蝗。十五年春,黃安、黃岡、羅田、江陵、公安、石首、松滋大旱蝗。五月,均州、光化蝗。七月,濱州、觀城、鉅野、博興、穀城、應城蝗。八月,安陸、玉山、武昌、咸寧、崇陽蝗;黃陂、漢陽大旱蝗。十六年夏,定遠蝗,紫陽蝗,宜都、黃岡、隨州、鍾祥旱蝗。七月,穀城、鄖縣、鄖西蝗。十七年春,應城蝗蝻。五月,鄖縣旱蝗;秋,復旱蝗。十八年夏,鄖縣蝗,應山大旱蝗,博興旱蝗。八月,東光蝗,不為災。十九年九月,應山蝗。二十三年三月,鄖西旱蝗。二十五年七月,光化、麻城蝗。二十七年夏,應城蝻生,元氏旱,霑化蝗。十月,臨邑蝗。

咸豐四年六月,唐山、灤州、固安、武清蝗。五年四月,靜海、新樂蝗。六年三月,青縣、曲陽蝗。六月,靜海、光化、江陵旱蝗,宜昌飛蝗蔽天,松滋蝗。八月,昌平蝗,邢臺蝗,香河、順義、武邑、唐山蝗。七年春,昌平、唐山、望都、樂亭、平鄉蝗;平谷蝻生,春無麥;青縣蝻★生;撫寧、曲陽、元氏、清苑、無極大旱蝗;邢臺有小蝗,名曰蠕,食五穀莖俱盡;武昌飛蝗蔽天;棗陽、房縣、鄖西、枝江、松滋旱蝗;宜都有蝗長三寸餘。秋,咸寧、漢陽、宜昌、歸州、松滋、江陵、枝江、宜都、黃安、蘄水、黃岡、隨州蝗;應山蝗,落地厚尺許,未傷禾;鍾祥飛蝗蔽天,■數十里;潛江蝗。八年三月,撫寧、元氏蝗蝻生。六月,均州、宜城蝗害稼,應城飛蝗蔽天,房縣、保康、黃巖蝗害稼。秋,清苑、望都、蠡縣、歸州蝻★生。十月,黃陂、漢陽蝗。十一月,宜都、松滋蝗。十年六月,棗陽、房縣蝗。

光緒三年夏,昌平、武清、灤州、高淳、安化旱蝗。秋,海鹽、柏鄉蝗。四年九月,靈州蝗。七年六月,武清蝗。七月,臨朐蝗。八年春,玉田蝻生。九年夏,邢臺蝗。十七年三月,寧津旱蝗傷稼。三十三年五月,山丹蝗。

順治五年,杭州民家★生三耳八足;蒙陰縣民家★產象,旋卒。九年八月,香山寺前★生二人頭,只眼,頭上一角,人身★足,無毛。

康熙元年八月,天門民家★生一豕,一身、二首、八蹄、二尾。十二年九月,揭陽民家★產麒麟。十八年二月,棲霞民家★生異獸,旋斃。五十一年,深澤縣民家豕生子,大物,大倍別子,色白,無毛,二目駢生頂上。

雍正五年,博山民家★產象,長鼻,白色。

嘉慶十年,樂清民家豕生象。十八年,黃巖民家豕生象。

同治三年,新閏民家豕生象,未幾即斃。

光緒元年,豕生象,色灰白,無毛。十三年,皋蘭民家豕生一象。

順治六年十一月,儀徵有四龍見於西南。十一年,淶水縣興雲寺梁上有蛇,身具五彩,十日後變為白色;六月十五日,狂風驟雨,霹靂不絕,殿中若有龍★,及霽,蛇乃不見。

康熙元年七月二十九日,嘉興二龍起海中,赤龍在前,青龍在後,鱗甲發火,過紫家埭,倒屋百餘間,傷一人;九月初九夜半,火龍見。二年四月十六日,崇明龍見;三臺東南出一蛇,長數丈,腰圍約三尺,身有鱗甲,赤光。三年五月二十一日,京山龍見,鱗甲俱現。七月朔,鎮洋大風海溢,有龍下麋場,傷數人。八月初四日,天晴無雲,黃龍見於東南。七年七月,咸寧有龍游於縣署前,雨霽,不能升躍,市人系其頸以游於市。十二年六月,深澤馬鋪民家龍起,大風雨,破壁而去。十二月十八日,丹陽見兩龍懸空,移時始去。十三年夏,永嘉龍見;萬載大水,龍出。十七年六月,咸寧大墓山有龍突現頭角,三日,鱗甲晃如赤金,白晝飛騰,穿山為河,傷民畜。十八年十月十五日,鎮洋龍見於東南。二十一年十月,青浦、興化龍見。二十六年六月,黃縣龍晝見於硃家村,煙霧迷濛,火光飛起。三十六年三月,畢節龍見赤水河。四十年八月,獨山州南羊角村有龍見。四十一年六月初九日,鰲泉有白龍躍於平地,飛去。四十五年五月初六日,金山之巖有龍出,金光閃爍。四十七年,靈州井中有龍,時見其首尾,數日,忽大雨霹靂,騰空而去。六十年六月,金壇學宮前懸一龍,腥氣逆鼻,焚香禱之,騰空而去。七月十三日,南籠大雷雨,龍見於城西。

雍正二年七月,北流飛龍見。十二月,木門海子起煙霧,有蛟龍飛出之狀。五月,橫州有龍起。七年春,安定文葦塔見一龍騰空而去。九年四月,安南有龍見於東北。六月,青浦龍見於沙灘。

乾隆二年二月,潮陽白龍見。三年正月,枝江龍見於城西。九月,青浦龍★於泖,自西南至東北入海。五年五月,高郵大風,有白龍舞空中,鱗甲俱現。六年六月十三日,昆山東鄉設網村有白龍手卷去民房十七家。二十五日,席家潭有白龍手卷去周家莊大舟★二人,墜巴城鎮三里岸渚,復手卷去鎮民盛某,擲地,身無恙。九年六月十二日,浮山有龍飛入民間樓舍,須臾煙起,樓盡焚。七月壬辰,建□天頓黑,有白龍尾垂二丈餘。十二年八月,高州龍見於小華山。十四年七月初五日,高淳龍起於永豐圩下,首尾鱗甲俱現。十五年七月,正寧秦家店有龍破屋而升,俄大雷雨。十九年秋,濟南巨治河有龍★。二十年五月二十日,澄海狂風驟雨,有雙龍自東而來,由蓬州所東門經過,★倒城垣五十七丈,民房三百餘間,有壓斃者。二十一年六月,招收、龍井地方有龍自空冉冉而下。二十六年五月二十七日,葭州赤龍見於張體兩川圍中。六月初七日,高平火龍見於石末村。七月十四日,泰安蛟起夏輝村西河,高二丈,彩色灼爛,橫飛東南,風雲隨之。二十九年四月十三日,天門烏龍見,頭角爪甲俱現。四十三年三月,安丘龍見。四十六年八月十二日,莒州★龍見於吳山東北。五十五年五月,定海舟山龍起,漂沒田廬,淹斃人口;越三日,龍斬三段,尾不見,其鱗巨如葵扇。五十六年六月,莒州赤龍見於龍王峪,先大後小,長數丈,所過草木如焚。六十年春,青浦有白龍自東至金澤鎮南,去地祗三四尺,所過屋瓦皆飛。

嘉慶六年,東湖修孔子廟,見白龍乘風飛去。九年,曲陽濟瀆河水暴發,見龍車數乘涉水而沒,水退。十四年五月,有龍戲於瑞州城隍廟江均河,水立丈餘。二十年六月,黃岡柳子巷蛟起,傷一百四十餘人,沖沒田宅無算。二十一年六月,蛟見於嬰武水。

道光四年七月,麻城龍見於月望巖。五年七月甲辰,武進龍見於芙蓉湖。六年六月初五日,宜都蛟起,壞民居,溺人無算。七年五月初十日,房縣汪家河水溢,蛟起,壞民田無算。九年十一月二十二日,滕縣見青龍,長約數十丈,鱗甲俱現。十年六月,松滋城原寺出龍,過洋州上升。七月十二日,永嘉起蛟,裂山而出,漂沒田廬,淹斃人畜無算。十六年七月甲申,武進有龍陷地成潭。二十八年五月,監利龍見於洪湖。七月二十三日,太平五龍同見空中,是夜颶風大作。

咸豐二年五月十七日,枝江天無片雲,有白龍降於瓦★湖,蜿蜒行數里,忽騰去。三年七月初七日,西鄉白龍見,長數十丈。七月十五日,黃陂龍見於聶口,鱗甲宛然,擁船隻什物於空中。十一月,西寧西納川降★龍,臭聞數里。五年七月二十三日,石首風雷大作,頃之二龍接尾而上。六年五月,鄱陽縣兩頭蛇見。七年五月初八日,來鳳縣曾氏塘風雨驟至,有物長丈餘,乘風入塘,形似牛,身備五色,目灼灼有光,水噴起。八年六月十七日,雲夢有龍入城,壞廬舍無數,繞城東北去。十年三月,麻城龍見。五月,松滋天鵝塘出龍,行陸地,所過禾稼盡偃。十一年冬,平湖有二龍★於海。

同治三年,蘇州有龍★。四年正月,宜城龍見於芳草洲。六年五月初五日,高淳見三龍。十年三月二十二日,湖州有龍★,狂風驟雨,拔木覆舟。五月十二日,高淳龍見。七月底,城有蛟起於井中。

光緒十九年正月,靈臺龍見於井中。二十一年十月,大通龍見於惠廣寺。

順治十年四月,吳川有山馬二:一渡石塹,一自城東南角入。

康熙三年七月,畢節民家馬生駒,五足。五年六月,孝豐有馬見於魚池鄉之安市,毛鬃如凡馬,背有肉鞍,往來田間,月餘不知所終。十五年五月,南樂生員趙豪馬生雙駒,一牝一牡。二十六年,平遠州民家馬產雙駒。

雍正八年二月,江津縣民家馬產雙駒。

乾隆二十一年夏,豐順湯坑寨有白馬成隊,夜出食禾,驅之不見。

道光二十九年二月,定州中谷村民家白馬產二青騾。

同治元年,西寧鎮海營騾馬同胎而生。

順治八年,歙縣民吳全妻呂氏一產四男。

康熙元年,萊陽民徐維平妻生男四目、四手、四足。十一年,晉州民郭好剛妻馮氏一產四男。十三年,東陽民某姓兄弟,其婦俱孕,及產,一產★魚,魚頭蛇身;一產獼猴,手爪俱備。十六年,畢節民彭萬春女七歲出痘,及愈,變為男。十七年春,清河民家生子無首,兩目在乳,口在臍,殆形天類歟?二十五年五月,忠州民雷氏女化為男,後為僧。三十二年,德州民王邦彥妻一產四男。三十四年,長治民張自富妻王氏一產三男。三十七年,保安州民岳戍妻李氏一產三男。三十八年,潞城民常通妻一產三男。三十九年,湖州陸氏婦產一男,兩首四臂。四十一年,西寧縣賈文舉妻一產四男。四十四年夏,石首縣民張若芝妻一產三男。四十六年,吳縣民譚某家女子化為丈夫。五十三年,廣元民婦產二蛇,無恙;萊陽縣民高萬言妻一產四男。五十四年六月,東平州民孫子芳妻一產三男。五十六年,東流民檀上元妻洪氏一產三男;潁上民張某妻一產三男。五十七年,信陽州民邢序妻一產三男。五十八年,興化縣民趙自顯妻一產三男;趙城民賈則宜妻一產三男。五十九年,邯鄲民王某妻一產三男。六十一年,鉅野縣民史★妻一產三男。

雍正元年,郫縣民某妻一產三男,青州民李福奎妻一產二男一女,高密民劉巨卿妻一產三男,巢縣民馬少步妻龐氏一產三男。二年秋,南陵民毛起美妻一產三男,汾陽民賈三聘妻一產三男,簡州民王之佐妻一產三男。三年二月,齊河民甄養武妻一產三男,潞城民秦述賢妻郭氏一產三男,襄垣民郝世惠妻武氏一產三男,陽城民張國澤妻劉氏一產三男。四年,陶縣民徐來振妻一產三男,襄陵縣民慄星奇妻一產三男。五年,東河縣民劉虎妻一產三男,陽城民李珍妻一產三男。六年,定興縣民任萬通妻、榆次民劉志龍妻俱一產三男,東山村民家產婦生魚,亭山縣民田禹妻一產三男。七年,錢塘縣民邵學桂妻吳氏、天臺縣民褚伯賢妻劉氏、蕭山縣民高耀妻俱一產三男,新建縣民周義士妻夏氏一產三男,彭澤縣民羅翰聲妻宋氏一產三男,合肥民龔紹衣妻陳氏一產三男,安邑縣民馮維明妻薛氏一產三男,施縣民王進祿妻崔氏一產三男。八年,商縣民孫作聖妻一產三男,黃縣民高從義妻一產三男,錢塘縣民楊大成妻嚴氏、遂安縣民洪文錫妻毛氏俱一產三男,壺關民某妻李氏一產四男,崇陽縣民孫文林妻王氏一產三男,興安民龔章純妻一產三男,臨海縣民項如茂妻林氏、鎮海縣民陳道才妻應氏俱一產三男。十年,洪洞縣民許元生妻鄭氏一產三男,趙玉錫妻章氏一產三男,介休民燕居宇妻武氏一產三男,陵川民秦遇妻一產三男,什邡縣杜某婦一產三男,永嘉縣民李天錫妻林氏一產三男,浮梁縣民魏經武妻李氏一產三男,房縣民吳士貴妻一產三男。十一年,冀州民白起妻薛氏一產三男,遂安縣民姜自周妻胡氏一產三男。十二年春,齊河民劉鈖妻官氏、新城民趙允中妻俱一產三男。六月,潛山民汪祝三妻一產三男,開化縣民畢懋增妻一產三男。十三年四月,灤州民張德福妻一產三男。九月,灤州民胡在梁妻一產三男,臨海縣民榮宗棣妻奚氏一產三男,南昌縣民硃中祿妻曹氏一產三男。

乾隆元年,武強民楊守有妻蔡氏一產三男,定遠縣民羅旌友妻楊氏一產三男。二年二月,平湖監生徐士穀妻張氏一產三男。九月,景州民張自立妻王氏一產三男,靜海民婁蒙貴妻某氏一產三男。十一月,石城縣民董永瑁妻李氏一產三男。三年六月,甘泉民蔣國泰妻蘇氏一產三男。八月,秀水縣民葛漢文妻徐氏一產三男,營山縣民周銘妻文氏一產三男。十一月,灤州民李廷璽妻一產三男,景州民張世勛妻劉氏一產三男一女。四年四月,妻蘇氏一產三男。八月,秀水縣民葛漢文妻徐氏一產三男,營山縣民周銘妻文氏一產三岳池縣民荀稀聖妻李氏一產三男,婁縣民何效章妻陸氏一產三男。六月,稷山民張桂妻劉氏一產三男。五年三月,潛山縣民馮某妻一產三男。五月,無為生員魏海元子婦一產三男,潞澤營兵丁謝金成妻魏氏一產三男。八年五月,稷山民趙傑妻一產三男。九年五月,大埔民羅淑鄞之妻李氏一產三男。十年,銅山民劉瑞發妻韓氏一產三男,夏名魁妻劉氏一產三男,貴池民吳來盛妻葉氏一產三男。十一年,大埔縣民危肇彬妻詹氏一產三男,貴陽民劉允福妻喻氏一產三男。十四年,無極縣民袁文孝妻焦氏一胎產四子,兩男兩女,皆活。十五年,監利縣民何名周妻黃氏一產三男。十六年,南昌縣民徐仲先妻萬氏一產三男,盂縣民田世隆妻石氏一產三男。十七年四月,寧河民劉守秀妻趙氏一產三男。十八年,平利縣民張寧妻呂氏一產三男。十九年,濟陽民賈含福妻谷氏一產三男。二十年,深澤民蘇勇妻宋氏、劉邦林妻閻氏俱一產三男,灤州民高宗義妻一產三男。二十一年五月,定州民張照妻徐氏一產三男,濟寧州民王盡忠妻一產三男,營山縣民魏國平妻陳氏一產三男。二十二年,資陽縣生員宋如衡妻蘇氏一產三男,浦江縣民葛有聖妻徐氏一產三男。二十三年八月初五日,兵丁劉任妻黃氏產一男,越日產一女,午刻又產一男。二十五年,陵川諸生馬伯顧妻一產三男,南豐縣民鄧君奇妻硃氏、安義縣民熊壁湘妻彭氏俱一產三男,雲夢民冷少松妻許氏一產三男。二十六年七月,鳳臺縣民陳全妻一產三男,武昌縣民劉升妻一產三男。二十七年,鄒縣民田成妻一產三男。二十八年三月,武進縣民巢雲五妻一產三男。二十九年,武城縣民劉成妻高氏一產三男,即墨縣民高岱妻王氏一產三男,平陸縣民高懷妻一產三男,清水縣民喬國祥妻王氏一產三男。三十年,臨清州民楊維桐妻一產三男。三十一年,新城民硃振連妻一產三男,臨清州劉德員妻一產三男,樂至縣民羅景璋妻周氏一產三男。三十二年五月,府谷縣民王友妻一產三男。六月,臨縣民李映實妻一產三男。十月,德州民陳三妻一產三男。三十四年,寧河於邦朝妻蘇氏一產三男,黃縣民王偲妻高氏一產三男。三十五年,武昌縣徐定貴妻一產二男二女。三十八年,高平縣民張萬全妻李氏一產三男。三十九年二月,諸城縣民郭榮妻一產三男。四十年,南昌縣民李菁妻梁氏一產三男,樂平縣民王彩珍妻廖氏一產三男。四十一年六月,龍里縣民家生子,目中有臂三寸許,青陽民曹正送妻董氏一產三男。四十二年,貴池民孫全愷妻謝氏一產三男。四十三年九月,德州民趙楷妻崔氏一產三男,陵川民李鎯妻王氏一產三男,洵陽寄籍楚民張希賢妻雷氏一產三男。四十四年,昭化縣民王宰仕妻張氏一產三男,此婦四孕,每產必三,亦異婦也;光化縣民許文思妻柯氏一產三男。四十六年,高郵縣民於志學妻管氏一產三男,梓潼縣民羅全義妻楊氏一產三男。四十七年,寧州民彭國治妻葉氏一產三男。四十八年,茂州民文廷柱妻一產三男。四十九年十一月,新建縣民黎獻文妻熊氏一產三男。五十年,太平縣民傅學妻羅氏一產三男。五十二年二月,房縣民張大業妻一產三男,京山縣民李★來妻一產三男。五十三年,廣州民廖伯萬妻硃氏一產三男。五十四年三月,瑞昌縣民周全萬妻陳氏一產三男,昭化縣民張應輝妻劉氏一產三男,宜賓縣民萬方麟妻陳氏一產三男,簡州民藍學榮妻王氏一產三男,莒州民劉翰閣妻一產三男。五十七年五月,高郵縣民閔立禮妻李氏一產三男。五十八年,貴陽女子莫二陽化為男子,石首縣民譚盛治妻一產三男,陽信縣民王學皆妻張氏一產三男。五十九年,天全州民劉祥遠妻熊氏一產三男。六十年,沂水縣民趙有佐妻王氏一產三男。

嘉慶二年,莘陽縣民楊國玉妻簡氏一產三男,分宜縣民羅大成妻藍氏一產三男,鄒平民樊梅清妻一產三男,諸城縣民王立妻一產三男。四年七月,博興縣民張維慶妻一產三男,溪陽縣民吳正彩妻劉氏一產三男。十二月,定州民薛際昌妻趙氏一產三男。五年正月,隨州民聶中妻一產三男。六年,營山縣龍宣江妻郭氏、廣元縣民董在義妻俱一產三男,長樂縣民張茂榮妻劉氏一產三男,竹山男子李大鳳化為女,棲霞邱家村王氏婦化為男子。八年三月,臨淄縣民王氏婦一產四男,新城民岳景妻一產三男。十年三月,日照縣民張延妻徐氏一產三男。十二年二月,諸城縣民王授堯妻曲氏一產三男。三月,東阿縣民蔡光輝妻金氏一產三男,東鄉縣民黎鳳蘭妻趙氏一產三男。十三年五月,留壩縣民靡永鉗妻陳氏一產三男,應城縣民某妻一產三男。十五年正月,黃濟縣民金澤妻生子無耳目口鼻,兩頭一角,扣之有聲如銅。十八年,益都縣民梁氏子驟長一丈有奇。十九年,巴縣民劉天才妻一產三男,博興縣民李敬昌妻趙氏一產三男,靖遠縣民張守和妻王氏一產三男。二十一年六月,武城縣武庠生王靈妻刁氏一產三男。九月,湖口縣民吳紹榮妻時氏年四十五歲,初胎一產三男,應城縣魯姓婦遺腹一產三男。二十二年正月,彭澤縣民何奇★妻王氏一產三男。二十三年十一月,博興縣民孫在興妻白氏一產三男。二十四年四月,樂安游氏女春桃年十五歲化為男。二十五年,日照縣民宋★妻周氏一產三男,定海民陳宏球妻一產三男。

道光元年九月,日照縣民趙希常妻張氏一產三男。三年五月,中★縣民吳興妻一產三男。五年,樂平縣民甘德喜妻陳氏一產三男。六年七月,樂清縣民戴萬春妻林氏一產三男,麻城民甘學楷妻一產三男。七年,宜城縣民張金福妻一產三男,狄道州民潘永周妻一產三男。九年十月,樂陵縣民張志芳妻柳氏一產三男。十年七月,濱州民趙登坡妻張氏一產三男。十一年七月,萊陽縣民孫洪妻一產三男。十三年,崇陽縣民傅調鼎妻一產二男一女。十四年,日照縣民馬立太妻一產三男。十五年七月,利津縣民馬恭妻宋氏、商縣民張曲寅妻胡氏俱一產三男。十六年,日照縣民郭忠妻劉氏一產三男。十七年,樂陵民陳吉順妻宋氏一產三男,觀城縣民陳珩妻錢氏一產三男。二十年,貴定民王某妻一產三男。二十二年十一月,南鄉有女化為男。二十五年十二月,平度民蘭種玉妻一產三男。二十七年十月,公安縣民婦產一女,手足各四,三日而口有齒。二十八年四月,葛家坡盧氏女年十二化為男。二十九年,西寧縣民張侈倫妻一產三男,玉山縣民李前鄰妻周氏一產三男。三十年三月,應城縣民宋爽先妻張氏一產三男,黃陂縣民李允肙妻劉氏一產三男。

咸豐元年,崇陽縣民某婦一產五男。二年二月,黃縣民王經魁妻一產三男。五月,宜都杜氏女十三化為男。五年,平湖民黃某妻一產四女一男。六年,黃安縣民婦產一子,二首一身。十一年,興國縣民曾世紅女許字王氏子,幼,即收養夫家,及年十四,化為男,遣歸。

同治三年,即墨縣民家有男化女,孕生子。四年,秀水陳氏婦產四鼠。五年,東南鄉民有女化為男。八年九月,靈州民惠澤之妻孕三歲不產,忽小腹潰裂,子從孔出,如人形,頃之子死,腹復合無痕。十年冬,襄陽民徐氏子生而有佛像三,下作蓮花紋,在其左偏。

光緒三年四月,皋蘭庠生張文煥妻一產四男。十八年六月,寧州民馬壽隆妻生一子,三眼、三足,一眼在額上。三十三年,寧州民馮某家生一子,深目長喙,爪背有毛長寸,能左右顧,啼聲如猿。

順治元年,懷來大疫,龍門大疫,宣化大疫。九年,萬全大疫。十三年,西寧大疫。

康熙元年五月,欽州大疫,餘姚大疫。七年七月,內丘大疫。九年正月,靈川大疫。十二年夏,新城大疫。十六年五月,上海大疫。六月,青浦大疫。七月,商州大疫。十九年正月,蘇州大疫,溧水疫。八月,青浦大疫。二十年,晉寧疫,人牛多斃;曲陽大疫。二十一年五月,榆次疫。二十二年春,宜城大疫。三十一年三月,鄖陽大疫。五月,房縣大疫,廣宗大疫。六月,富平疫,同官大疫,陜西大疫,鳳陽大疫,靜寧疫。三十二年七月,德平大疫。三十三年夏,湖州大疫,桐鄉大疫。秋,瓊州大疫。三十六年夏,嘉定大疫,介休大疫,青浦疫,寧州疫。三十七年春,壽光、昌樂疫。夏,浮山疫,隰州疫。四十一年三月,連州疫。四十二年春,瓊州大疫,靈州大疫。五月,景州大疫,人死無算。六月,曲阜大疫,東昌疫,鉅野大疫。八月,文登大疫,民死幾半。四十三年春,南樂疫,河間大疫,獻縣大疫,人死無算。六月,菏澤疫。秋,章丘大疫;東昌大疫;青州大疫;福山瘟疫,人死無算;昌樂疫,羌州、寧海大疫;濰縣大疫。四十五年夏,房縣大疫,蒲圻大疫,崇陽疫。四十六年五月,平樂疫,永安州疫。七月,房縣大疫,公安大疫。八月,沔陽大疫。四十七年二月,公安大疫。三月,沁源大疫。五月,靈州大疫,武寧大疫,蒲圻大疫,涼州大疫。四十八年三月,湖州大疫。四月,桐鄉大疫,象山大疫,高淳大疫,溧水疫。五月,太湖大疫,青州疫。六月,潛山、南陵、銅山大疫,無為、東流、當塗、蕪湖大疫。十月,江南大疫。四十九年秋,湖州疫。五十二年冬,化州大疫,陽江大疫,廣寧大疫。五十三年夏,陽江大疫。五十六年正月,天臺疫。六十年春,富平疫,山陽疫。六十一年七月,桐鄉疫,嘉興疫。

雍正元年秋,平鄉大疫,死者無算。二年六月,陽信大疫。四年四月,上元疫,曲沃疫。五月,大埔疫,獻縣疫。五年夏,揭陽大疫,海陽大疫。秋,澄海大疫,死者無算。冬,漢陽疫,黃岡大疫,鍾祥、榆明疫。六年三月,武進大疫,鎮洋大疫,常山疫。四月,太原疫,井陘疫,沁源疫,甘泉疫,獲鹿疫,枝江疫,崇陽大疫,蒲圻大疫,荊門大疫。夏,巢縣疫,山海★大疫,鄖西大疫。十一年,鎮洋大疫,死者無算;昆山疫;上海大疫,寶山大疫。

乾隆七年六月,無為疫。十年十一月,棗陽大疫。十二年五月,蒙陰大疫。十三年春,泰山大疫,曲阜大疫。夏,膠州大疫,東昌大疫,福山大疫。秋,東平大疫。十四年五月,青浦大疫,武進大疫。七月,永豐、溧水疫。二十一年春,湖州大疫,蘇州大疫,婁縣大疫,崇明大疫,武進大疫,泰州大疫。夏,通州大疫。十一月,鳳陽大疫。二十二年四月,桐鄉大疫。七月,陵川大疫。二十五年春,平定大疫。六月,嘉善大疫。冬,靖遠大疫。三十二年八月,嘉善大疫。三十五年閏五月,蘭州大疫。四十年春,武強大疫。四十八年六月,瑞安大疫。五十年冬,青浦大疫。五十一年春,泰州大疫,通州大疫,合肥大疫,贛榆大疫,武進大疫,蘇州大疫。夏,日照大疫;範縣大疫;莘縣大疫;莒州大疫,死者不可計數;昌樂疫;東光大疫。五十五年三月,鎮番大疫。八月,雲夢大疫。五十七年九月,黃梅大疫。五十八年冬,嘉善大疫。六十年十二月,瑞安大疫。

嘉慶二年六月,寧波大疫。三年五月,臨邑大疫。五年五月,宣平大疫。十年二月,東光大疫。三月,永嘉大疫。十六年七月,永昌大疫。十九年閏二月,枝江大疫。二十年春,泰州疫。四月,東阿疫,東平疫。七月,宣州疫,武城大疫。二十一年,內丘大疫。二十三年十一月,諸城大疫。二十四年五月,恩施大疫。二十五年七月,桐鄉大疫,太平大疫,青浦大疫。八月,樂清大疫,永嘉大瘟疫流行。冬,嘉興大疫。

道光元年三月,任丘大疫。六月,冠縣大疫;武城大疫;範縣大疫;鉅野疫;登州府屬大疫,死者無算。七月,東光大疫,元氏大疫;新樂大疫;通州大疫;濟南大疫,死者無算;東阿、武定大疫;滕縣大疫;濟寧州大疫。八月,樂亭大疫;青縣時疫大作,至八月始止,死者不可勝計;清苑、定州瘟疫流行,病斃無數;灤州大疫;元氏、內丘、唐山、蠡縣大疫;望都大疫;臨榆疫;南宮、曲陽、武強大疫;平鄉大疫。九月,日照大疫,沂水大疫。二年夏,無極、南樂大疫,臨榆大疫,永嘉疫。七月,宜城大疫,安定大疫。三年春,泰州大疫。秋,臨榆大疫。四年,平谷、南樂、清苑大疫。六年冬,霑化疫。七年冬,武城疫。十一年秋,永嘉瘟。十二年三月,武昌大疫,咸寧大疫,潛江大疫。四月,蓬萊疫。五月,黃陂、漢陽大疫;宜都大疫;石首大疫,死者無算;崇陽大疫;監利疫;松滋大疫。八月,應城大疫,黃梅大疫,公安大疫。十三年春,諸城大疫。四月,乘縣大疫。五月,宜城大疫,永嘉大疫,日照大疫,定海大疫。十四年六月,宣平大疫,高淳大疫。十五年七月,範縣大疫。十六年夏,青州疫,海陽大疫,即墨大疫。十九年九月,雲夢大疫。二十二年正月,高淳大疫。夏,武昌大疫,蘄州大疫。二十三年七月,麻城大疫,定南大疫。八月,常山大疫。二十七年秋,永嘉大疫。二十八年春,永嘉大疫。二十九年五月,麗水大疫。

咸豐五年六月,清水大疫。六年五月,咸寧大疫。十一年春,即墨大疫。六月,黃縣大疫。

同治元年正月,常山大疫。四月,望都、蠡縣大疫。六月,江陵大疫,東平大疫,日照大疫,靜海大疫。秋,清苑大疫;灤州大疫;寧津大疫;曲陽、東光大疫;臨榆、撫寧大疫;莘縣大疫;臨朐大疫;登州府屬大疫,死者無算。二年六月,皋蘭大疫,江山大疫。八月,藍田大疫,三原大疫。三年夏,應山大疫,江山大疫,崇仁大疫。秋,公安大疫。五年五月,永昌大疫。六年二月,黃縣大疫。七月,曹縣大疫。九月,通州疫,泰安大疫。八年六月,寧遠、秦州大疫。七月,麻城大疫。九年秋,麻城大疫。冬,無極大疫。十年五月,孝義疫。六月,麻城大疫。十一年夏,新城大疫,武昌縣大疫。

順治六年六月,太平啟山縣山鳴如雷,移時乃止。十月,階州山鳴。九年九月,武強天鼓鳴。

康熙元年七月七日,夜聞有聲。二年四月二十三日,萊陽有聲如海嘯,自南起,至子時方息。四年正月初九日,西山鳴,永嘉山鳴,瑞安山鳴。七年八月,泰山鳴。九年夏,黃巖天鼓鳴。十七年十二月二十二日未時,棗強、同官中有聲如雷,起自西北。二十四年五月,樂昌有聲如雷,自西南之東北。四十七年七月,霑化無雲而雷。五十年十月十一日,玉屏南山有聲如鳴鼓。五十六年七月,合肥縣城墻夜哭三次。六十年十一月十九日午刻,遵化有聲自西南來,其聲如雷。

雍正七年九月,嘉平無雲而雷者三。十二年正月初三,武定有聲如雷,自東北至西南,移時乃止。

乾隆六年八月,宜昌★山有聲如牛鳴,聲聞數十里,數十晝夜不息,自止。十年五月,寧津無雲而雷。十一年四月,分水南慈山夜半石鳴,逾日乃止。十二年正月十三日,解州無雲而雷。十二月乙酉,肥城仁貴山有聲如雷,移時乃止。十七年二月,忻城夜中有聲如雷,移時乃止。十八年五月,池州東南山鳴。二十一年八月,秦州邽山鳴。二十三年三月,東萊清嶺鳴聲如殷雷。

嘉慶元年二月,榮成有聲如雷,自西北向東南。十年三月三日,袁州空中有聲。

道光二十年九月,星子五老★有聲如雷。二十六年八月,平湖四城鳴如鳥啾啾不已。

咸豐元年六月,浮梁城隍廟有哭號聲。八月二十八日,隨州有聲如雷。三年七月,元氏天鼓鳴,自東北至西南,數日始止。十年二月,臨朐逢山石鼓鳴。

同治元年正月初二日,三原東鄉夜聞兵馬聲。六月,狄道州鳳凰山鳴聲如雷,數日不息。四年四月,通渭、泰安有聲鳴如鼓。六年夏,江山江郎山鳴。二年六月十四日,漳縣有巨聲三作,聲聞數十里。

光緒二十二年四月戊子,南樂無雲而雷。二十三年五月,南樂無雲而雷。

順治五年六月,貴池隕石。十年四月,瀘州星隕化為石,大如斗。

康熙十三年五月,寧遠墜二星,化為紅石。十五年五月,青浦星隕,墜地有聲,居民掘之,見一黑石,按之尚熱,重九十斤,擊碎,刀摩之,火光四射。二十年正月二十日,海豐有星隕化為石,其形三角,重九斤。二十四年正月初六,饒平星隕黃岡五丈港,聲聞數十里,化為石,其大如斗,其色外青內白。

雍正八年八月,府穀星隕,入土四尺,掘之得一黑石。

乾隆三十五年三月,樂安空中有光如炬,掘地得一石,鐵色,大如斗,叩之有聲,欲舁之,不語則舉,語則雖大力不能舉。四十年八月,鉅□縣屬吳家集隕星一,化為黑石。四十七年八月,滕縣星隕忠三保楊氏院中,化為石,色青白,重約百斤,孔數百,大容拳,小容粟。五十八年四月,分宜隕石於田,巨聲如雷,黑色。

嘉慶二十三年十一月二十五日,長星落,有聲如雷,土人視其隕處成一坑,掘之,得一石,長二尺餘,闊尺餘,形方而角圓,★碎之,中分五色。

咸豐十一年七月三十日,光化隕星三,化為石。

同治十二年六月十四日,漳縣馬成龍川有巨聲三作,聞數十里,空中墜石三塊,高可四尺五寸。十月,羅田隕石,觸地而碎。

光緒二十年正月二十二日,皋蘭隕星如火球,土人識其處,掘之,得一鐵卵。

順治十一年二月初九日,香山河水如血,次日復故;崖州東荔枝塘水赤如血,旬日乃已。十二年,萬泉井水黑。十三年,江州泉水忽赤如血。十四年三月,畢節雙井出紅水,龍潭出黑水。十五年四月,潮陽江水變色黑而濁。十八年八月,通州河水黑如墨。

康熙十五年九月,渭水赤。三十二年,襄陵水赤,半月始復。

雍正二年七月,桐鄉海水入內河,味如滷。

乾隆三十三年六月,歙縣西鄉池塘、井泉之水沸起如立,移時乃平。

道光元年六月,曹縣城中坑水赤。

咸豐三年七月,應城堰水無故由南趨北,湧起如山,南北水二道中凹見底,移時始合;安陸水★。四年十一月,蘄水水湧,躍高數尺;青浦水無故自湧。五年六月,雲夢池水自溢,高尺許,頃復故。十一月初五,宜昌堰水,無風水自湧起尺許。

光緒四年五月十二日,孝感塘水忽沸起,高二尺許,逾時始定;黃岡水自湧;雲夢塘自溢,久之始定。

宣統元年六月,隴水赤三日。

順治元年八月,東陽大水,邢臺大水。二年四月,萬載大水,淹沒田禾;東安大水。七月,嵊縣大水,邢臺大水,棗強大水,真定滹沱河溢,雞澤大水,單陽大水。三年二月,阜陽大水,亳州大水。五月,兗州大水,漂沒廬舍,沂州、蒙滇大水。七月,高平大水,臨淄大水。四年四月,萬載大水。六月,平樂、蕭縣、銅山、望江、無為、阜陽、亳州大水。七月,瑞安、曲阜、沂水、樂安、汶上、昌樂、安丘大水。八月,高州、高郵大水,寧陽汶水溢。五年春,五河、平原、汶上大水。五月,平樂、永安、密雲、獻縣、新河、柏鄉、霸州大水;白河堤決。六月,武強、平鄉、南和、永年、棗強、密雲、晉州、宿松大水;建德蛟起大水。七月,潁上、亳州、太平、常山大水。六年四月,九江、漢陽、鍾祥大水。五月十八日,阜陽淮河漲,平地水深丈許,壞民舍無算。七月,鹽城、文安、真定、順德、廣平、大名、河間大水。七年正月,漢陽九真山蛟發水。五月,齊河河決,長清河決,荊隆口平地水深丈餘,村落漂沒殆盡,黃河決;剡城、日照大水。六月,蒼梧、遂昌、臺州、湖州、興安、安康大水。秋,東阿大水,淹沒六十七村;東明、荊隆口決,河溢,陸地行舟;茌平、昌邑濰水決,漂沒田禾;石城、膠州、恩縣、堂邑、武定府屬大水。十月,仙居大水,城北隅塌,壞田廬無數,民多溺死;撫寧、欒城大水。八年正月,石埭、蘇州大水;景州河決。四月初七日,潛山蛟出千百條,江暴漲,壞民居無算;望江大雷雨。五月,旌德大雨,蛟發水,平地水深丈餘,溺死人畜無算。八月,烏程、瑞安、高淳、鎮洋大水傷禾。十月,廣宗、南樂、玉田、邢臺、寧河、南和大水。九年二月,東流大潦,湖水出,江湧高丈餘。三月,齊東黃河決。五月,臨清、平定、樂平、壽陽、武定、商河、樂陵大水,村落多淹沒。六月,樂平、岳陽、平陽、榮河、壽光、昌樂、安丘、高苑大水。七月,蒙陰、秦州、隴西、烏程、鍾祥、開平大水害禾稼。八月,普寧、桐鄉大水。

十年四月,石首、枝江大水;松滋堤潰。五月,沁水、壽陽、興安大水;欽州海水溢。六月乙卯,蘇州大風雨,海溢,平地水深丈餘,人多溺死;安定、白河雷雨暴至,水高數丈,漂沒居民;陽穀大水,田禾淹沒,民舍多圮,陸地行舟;文登大雨三日,海嘯,河水逆行,漂沒廬舍,沖壓田地二百五十餘頃。七月,鎮洋、蕭縣、嘉興大水。八月,莘縣、臨清大水。十一年三月,武昌縣雷山寺蛟起,水平地深丈許;沔陽堤潰大水。五月,興寧、龍川大水。六月,茌平黃河決,村墟漂沒。十二年正月,鹽城海溢,人民溺死無算。四月,石埭、嘉興、鍾祥、潛江大水。六月,漳水溢,平地水深丈許,陸地行舟。十三年五月,武強、湖州大水,興寧大水,陸地行舟。六月,萬載、萍鄉、寧都大水。十月,平湖、烏程、天臺大水。十四年六月,太平、石埭、銅陵大水。秋,望都、高要、安丘大水。十五年三月,臺州、臨海大水。夏,歸州、峽江、宜昌、松滋、武昌、黃州、漢陽、安陸、公安、嵊縣大水;宜城漢水溢,浮沒民田;當陽水決城堤,浮沒田廬人畜無算;荊門州大水,漂沒禾稼房舍甚多。秋,蘇州、五河、石埭、舒城、婺源大水,城市行舟;鍾祥大水;天門漢堤決;潛江大水。十六年四月,湖州、信宜大水。五月,衢州、江山、常山、江陵大水。六月,江夏、漢川、沔陽大水。十一月,仙居、通州、延川大水。十二月,望都、獻縣大水。十八年五月,龍川、峽江、萬載大水。六月,河源、平樂、蒼梧、武強大水。八月,淳安、慶元、南昌各府大水。

康熙元年五月,廣州大水。六月,洵陽、白河、興安、榆林大水。七月,孝感、沔陽、廣陵、江陵、松滋、鉅鹿、興化、蕭縣、沛縣、寧州大水。八月,天門漢水溢,堤決,舟行城上,成安、鍾祥、潛江大水。九月,冀州、阜城大水。二年六月,漢中、漢江、交河大水。七月,永安州、平樂、貴州、咸寧、大冶、蘄州、江陵大水。八月,松滋堤決,大水浸公安,民溺無算;枝江大水,漂沒民居,浮尸旬日不絕;宜都、黃岡、鍾祥、麻城、鉅鹿大水。九月,浦江、當塗、望江大水。十二月,蒲圻、大冶、沔陽、天門大水;江陵郝穴堤潰,大水。三年三月,阜城、萬載大水。六月,偏關河水暴發,壞民舍甚多,城內水深丈餘;海寧海決,水入城壕,天門、大埔大水。閏六月,延安、昌黎大水。七月,交河、梧州大水。八月,餘姚、山陰大水害稼,仙居、桐鄉大水。十二月,汾州府屬大水。四年三月,阜陽、望都大水,鳳陽水入城。七月,平定嘉水溢,景州、肥鄉、湖州、麗水、萍鄉、望都、雞澤大水,天門水決入城。八月,高邑、仁化、平樂、梧州大水。六年八月,懷來、河間、蠡縣大水,萊陽大水高數丈。七年五月,麻城、玉田、大埔大水。六月,欒城、南宮、★城、磁州大水。七月,趙州、臨城、高邑、深澤、安平、永年、蠡縣、鉅鹿、黃巖、樂清、萍鄉大水。八月,交河、高平、蒼梧大水。八年六月,三水、茂名、化州大水;房縣大水,壞田廬;東莞潦潮大溢。九年,鍾祥、應城、蒲圻、崇陽、枝江、鳳陽大水,湖州太湖水陡漲丈餘,漂沒人畜廬舍無算;青浦、全椒、五河、鄞縣、上虞大水;博野等二十九州縣大水。

十年秋七月,松滋、宜都大水。八月,文安、安肅、濟寧州大水,沭陽、石首大水。十一年,巴縣、忠州大水入城,酆都、遂寧、平樂、永安州、任縣大水。六月,湖州、宜興大水,漂沒民房,英德、杭州、邢臺大水,宜都、潛江、松滋、太平、烏程大水。十二年六月,高要、蒼梧、虹縣、濟南府屬大水。十三年三月,蘇州大水,霸州等十一州縣水。五月,任縣、萬載大水;瓊州海水溢,民舍漂沒入海,人畜死者無算。十四年六月,五河、新城、蠡縣、肅寧大水。八月,梧州大水。十五年正月,潛江、穀城大水;宜城漢水溢,漂沒人畜禾稼房舍甚多。五月,白河、永安州、平樂、武昌、大冶、蒲圻、黃陂、孝感、沔陽、廣濟、宜城、天門、梧州大水。六月,黃岡、江陵、監利、蘇州、青浦大水;廣濟江決,大水;懷集、震澤、蕭縣大水。九月,銅山、南樂大水。十六年二月,高郵、銅山、蕭縣大水。四月,潛江、望江大水。七月,河間、安丘、任縣、雞澤、欽州、蒼梧、橫州、潯州大水。十七年四月,龍川、和平、湖州大水。六月,欽州、惠來、遂州、合江大水。七月,任縣、邢臺、蕭縣、銅山、延安、平樂大水。十八年七月,祁州、肅寧大水。八月,漢中大水,潛江堤決。十九年六月,廣濟、宜都、宜昌、宜興、武進、福山、沂水、蒙陰、滕縣大水。七月,峽江、宜昌、宜都大水。八月,太湖溢,湖州大水。

二十年四月,常山、封川大水。五月,昌化、湯溪、江陵、監利大水,死者無算;新建等十四州縣水。二十一年春,秀水大水。五月,封川、枝江、建德大水入城。十七日,嚴州府屬六邑大水,二十一日方退。六月初五,水復大至。七月,平樂、蒼梧、建德、震澤、太湖、宿松、鄒平大水。二十二年七月,永安州、蒼梧大水。十月,★城、單縣、寧□大水。二十三年正月,銅陵、東昌大水。四月,寧州、莘縣、樂安、★城大水。二十四年正月,饒陽、臨城、遷安、獻縣、河間、樂亭、東平大水。夏,江夏、通城、黃岡、蘄水、麻城、黃陂、黃梅、廣濟、羅田、鍾祥、沔陽、荊州、江陵、監利、孝感、蒲圻、公安、高苑、安平、武強大水。二十五年六月,常山、樂安、壽光、昌樂、蓬萊大水。七月,臺州、薊州大水。二十六年,高明、連州大水。秋,震澤、高苑大水。二十七年五月,澄海、澤州、定遠大水。二十八年夏,永安州、平樂大水;河源大水,陸地行舟。二十九年八月,餘姚大水,蛟蜃出者以千計,平地水深丈餘;諸暨、上虞皆被水,田禾盡淹;薊州、寶坻大水。

三十年,永寧河決,淹沒田二百餘頃。三十一年二月,新城、新安、鄒平大水。七月,嘉定、眉州、釂州、灌縣、新津、威遠河水漲,損民舍,傷稼。九月十二日,新市河中水忽湧立高丈餘,徑圍俱有丈餘。三十二年七月,陽高、高郵、保定、順天、武定、河間大水。三十三年十二月,銅山溢,陽湖、高郵、東明大水。三十四年五月,湖州、桐鄉、澄海、公安、三水、樂安、震澤大水。三十五年六月,新安、即墨、★城大水。七月,江夏江水決;崇陽溪、黃陂、蒲圻、江陵大水;黃潭堤決;枝江大水入城,五日方退,廬舍漂沒殆盡。八月,黃岡、饒陽、秦州、歙縣、沛縣、遷安大水。九月,深澤、榮成大水。三十六年七月,昆山、臨榆大水。三十七年五月,婺源、堂邑、鳳陽、東昌、五河、新安、建昌大水。三十八年六月,新城、泰順、建德、新安、無極大水。閏七月,杭州大水。八月,臺州大水,平地高丈餘;金華、湯溪、西安、江山、常山、贛縣、沔陽大水。三十九年七月,剡城、沂州、高郵大水。

四十年,平樂、鶴慶、廣平、連州、廣州大水。六月,大埔、黃岡、海陽大水。四十一年五月,英山、澄海、寧縣大水。四十二年五月,高唐、南樂、寧津、東阿、江陵、監利、湖州大水;平樂漓江漲,平地水深丈餘,民舍傾圮;青城、陽穀、沂州、平遙、南樂、廣平大水;恩縣大水,陸地行舟;★河決。七月,登州府屬大水。十一月,漢中府屬七州縣大水,濟南府屬大水。四十三年二月,景州、漢江、天門、沔陽、監利大水。五月,連州、山陽大水,平地深丈餘;蒼梧、湖州、漢陽、漢川、監利、邢臺大水。四十四年,新建、豐城、廬陵、吉水大水。秋,青浦、柏鄉、六合大水。十一月,隨州溳水溢,壞民居;江夏、嘉興、漢川、潛江、天門、沔陽、監利、當陽大水。四十五年,清江、新淦、瑞金、穀城、鍾祥、天門大水。秋,沛縣、銅陵、阜陽大水。四十六年五月,鶴慶、龍門、河源、蒼梧、鄒平大水。冬,霸州六州縣大水。四十七年五月,杭州、高淳、南匯、太平、銅陵、無為、廬江、巢縣、太湖、南陵、昆山大水。六月,太湖水溢。七月,西安、常山、江陵、上海、武進、丹陽、蘇州大水。冬,當塗、蕪湖、翼山大水。四十八年春,潁川、阜陽、臨安大水。五月,慶元、江陵、監利、應城、荊門州、漢陽、漢川、孝感、潛江、光化大水。六月,婺源大水,漂沒田廬;黃河溢;水欒河溢;東安、單縣、臺州大水。四十九年八月,銅陵、無為、舒城、巢縣、嵊縣大水。十一月,棗強、霸州、慶雲、崇陽大水。

五十年五月,沂水大水。十月,平陽大水,漂沒居民數百人。五十二年五月,海陽、興安、鶴慶大水,石城河決,浸入城,田舍漂沒殆盡;贛州山水陡發,沖圮城垣。八月,臺州、廬州大水。五十三年五月,石城、肅□大水。五十四年春,梧州、鎮安府、昆山大水,江夏七州縣大水。四月,全州大水,城內深四五尺。五月,澄海大水,堤決;丘縣、壽光、獲鹿、獻縣、武定、濱州、海豐、陽信大水;長山河溢,湧起數丈。六月,蘇州大水,城水深五六尺,廬舍田地沖沒殆盡;杭州、枝江大水。秋,東昌河決。十一月,德平大水。五十五年三月,黃梅、廣濟、江陵、監利大水。五月,昌化、常山、寧武、建昌、丘縣、樂安大水;漳水決,寧陽、濟寧、汶上均受其災;崇陽、黃陵、天門、銅陵、太湖大水。九月,濟南府屬大水;潛山江水泛溢,田廬盡淹。五十七年三月,萬全、光化大水。五月,大埔大水。六月,旌德大水,漂沒人民橋梁無算;海豐、普寧、嘉應州、黃定縣、崇陽大水。秋,黃陂大水。五十八年正月,清河大水。七月,福山、日照、濰縣大水;膠州大水,平地深丈餘,漂沒廬舍無算,城垣崩圮。五十九年五月,龍川、海陽、澄海、慶元、桐鄉、高郵大水。六月,石首大水,漂沒民居殆盡;蒲圻、漢陽、漢川、沔陽大水。七月,橫州、宣化、隆安、永淳、蒼梧大水。六十一年六月,東阿河決;沂水河決,山東曹、單、濮等州縣均受其災;海州海溢;齊河金龍口河決。

雍正元年夏,東流、房縣大水;海陽韓江漲,保康水溢。七月,上海、大埔大水。二年二月,饒陽、肅宣、新樂、三河、寧河大水。四月,饒平大水。五月,澄海大水,堤決四十餘丈;光化漢水溢,傷人畜禾稼;房縣大水入城,漂沒民居甚多;穀城大水,一月始退;潛江、天門大水入城;鍾祥大水,堤決;沔陽、江陵、慶元大水。六月,東阿河決,陸地行舟。七月,泰州海水泛溢,漂沒官民田八百餘頃;南匯大風雨,海潮溢,田廬鹽場人畜盡沒;海寧海潮溢,塘堤盡決;餘姚海溢,漂沒廬舍,溺死二千餘人;海鹽海水溢;太湖溢;定海大風海溢,漂沒廬舍;鎮海大風雨,海水溢;鄞縣、慈谿、奉化、象山、上虞、仁和、海寧、平湖、山陰、會稽、嵊縣、永嘉,於七月十八日同時大水。八月己丑,蘇州海溢;樂清大水;即墨大水,民舍多圮。十二月,漢水暴發入城。三年正月,寶坻大水。二月,濟南、齊河、濟陽、德州大水。四月,廣州西江水溢。五月,饒平大水。六月,沂州河決;武強滹沱河溢,平地水深數尺,田禾盡淹沒;普宣大水;澄海大水,堤決五百丈。八月十五夜,大埔大水,陸地行舟;曲陽、武強、雞澤、邢臺、棗強、薊州、清苑、遵化州大水;新安大水,南北堤同日決。四年,濟南府屬大水。六月,大埔、應城、黃梅、黃岡、江陵、監利大水;蘄州江水高起丈餘;天門大水,陸地行舟。七月,嘉應、信宜、慶陽、漢陽、漢川、黃陂、江夏、武強、祁州、唐州、黃安、平鄉、饒平、蒼梧、普宣、濟寧州、兗州、東昌大水;崇陽蛟起,水浸入城。八月,桐鄉、南昌、新建、豐城、進賢、清江、新淦、建昌、德化、高淳、鶴慶大水。十二月,曹縣、單縣、菏澤、兗州、東昌大水。五年,漢水溢,武昌、安陸、荊州三府堤決。五月,蒼梧、安南、荊門州、黃岡、蘄州、廣濟大水。六月,平魯山水暴發,漂沒民居;慶陽、蒼梧、石城大水。七月,臨安、孝豐兩縣蛟出,山水陡發,餘杭、新城、安吉、德清、武康俱被水;蘄州江水漲;羅田、石首、公安、廣濟、嵊縣、安肅、容城大水;霍山蛟發水,黃河高數丈,沿河居民漂沒甚★。十月初三日,昌邑海溢,人多溺死;高郵、銅陵、廬江、舒城大水。六年,崇陽、漢陽、潛江大水。七年五月,大庾、南康大水。八年五月,蘇州、震澤大水。八年六月,武定、濱州、海豐、利津、霑化、滕縣、寧陽、兗州大水;濟南小清河決,傷禾稼;萊州霪雨兩月,河水暴發,田禾漂沒,民多溺死;衡水、沙河、雞澤、大名、順德、廣平、永年、高苑、博興、樂安大水;慶雲北河溢,清澗、黃河、無定河溢,漂沒人畜。九年春,樂安、壽光、東昌、長寧、慶雲大水。四月,宜昌溪水暴溢,壞民田。六月,碭山、長山大水。十月,濟南、鄒平大水。十年四月,富川大水。五月,瓘眉大水,沖塌房七十九間,淹斃人口九十五口;榮經、雅安、南安、南昌、撫州、瑞州、吉安大水。六月,黃岡大水。七月,蘇州大風雨,海溢,平地水深丈餘,漂沒田廬人畜無算;鎮洋颶風,海潮大溢,傷人無算;昆山海水溢;寶山颶風兩晝夜,海潮溢,高丈餘,人多溺斃;嘉定海溢;崇明海溢,溺人無算;青浦大風海溢。八月,昆山海水復溢,溺人無算。十一年,武強、邢臺、饒陽、豐潤、薊州、肅寧、沙河、盧龍、昌黎、獻縣大水;三河、寧河溢;沙州山水驟發,沖塌民房五百七十餘間。八月,剡城、高淳大水。十二年三月,懷安大水入城。

乾隆元年,鍾祥漢水溢;漢川、江陵、沔陽、天門大水。七月,鄞縣海水溢,慶元大水。二年二月,樂清、永嘉、瑞安大水。五月,鳳臺、黃岡大水。七月,武強、饒陽、獲鹿、欒城、平山、景州、容城、獻縣、新樂、新河、高邑、順天、莘縣大水,東昌★河決。三年七月,黃岡、麻城、柏鄉、肅寧、滄州、武強、東安、新安、饒平、獻縣、遂寧、合江、邢臺大水;渾河溢,秋禾被災者一百九十村;深澤、無極、澬河水溢。四年四月,亳州河決,潁上、阜陽、五河大水。秋,陽穀、壽張大水,禾盡淹;潤德泉溢。六年四月,鍾祥、天門、沔陽大水。五月,龍川、潮陽、寧都大水。七月,永嘉海溢,瑞州海溢,寶山海溢,蒼梧、湖州大水。八月,鍾祥南郊大水。七年六月,光化、宜城、江陵、枝江、南安府屬、永寧大水;游水發,田廬盡沒。七月,鹽城河決,毀民居數萬間;銅山河決,漂沒廬舍;安丘水溢六七里,人有溺斃者;膠河溢;剡城、袁州、江夏、嘉魚、東流、漢陽、漢川、黃陂、孝感、鍾祥大水,潁上、五河、亳州大水。八年夏,黃岡、宜都、興國、高淳大水。九年,天津、河間、霸州、撫寧大水。五月,澄海大水;東林堤決六十餘丈,沖倒民房數百間;大埔洪水入城,漂沒民房一百九十餘間。六月,漢川、遂寧、簡州、崇慶、綿州、★州、成都、華陽、金堂、新都、郫縣、崇寧、溫江、新繁、彭水、什邡、羅江、彭山、青神、樂山、仁壽、資陽、射洪大水,溺死居民六百餘人。七月,當陽江水暴發,田禾盡淹;紹興、徽縣巖水發,海溢,田禾盡淹;常山大水,溺人無算;淳安江濤暴漲,城市淹沒;桐廬江水驟漲,市城水高二丈,凡浸五日方退;昌化、建德、嘉善大水。

十年四月,西桂、普安州大水,潛江、沔陽等九州縣大水。五月,泰州海溢;亳縣水災;七沃、滄河大水,淹沒人畜無算;渭水溢;秦州藉水溢;白沙北堤決,水入城,民居漂沒甚多;隴石、棗陽、江陵大水。十一月,濟南大水。十一年,棗陽、潛江、沔陽、袁州、高苑大水。六月,連州、臨武大水。七月,鳳陽、潁上、亳州大水。十月,江陵、萬城堤潰,潛江被水災甚重。十一月,即墨大水。十二年五月,游仙山水驟發。六月,應州、渾源、大同三州縣大水。七月,海寧潮溢;鎮海海潮大作,沖圮城垣;蘇州颶風海溢;常熟、昭文大水,淹沒田禾四千四百八十餘頃,壞廬舍二萬二千四百九十餘間,溺死男女五十餘人;昆山海溢,傷人無算;泰州大風潮溢,淹鹽城,傷人甚多;棗陽大水,淹沒田禾;濟陽、德平、平原、霑化、兗州、濟寧州、嘉祥、剡城、莒州、蒙陰、日照、蘭山大水;東□、赤城水災。十三年五月,日照海溢、金鄉、魚臺、濟寧州、寧陽、範縣、壽光、膠州、岐山、潤德、肥城、潛江、漢川、天門、沔陽、江陵、監利大水,太原汾水溢。九月,鄖西、房縣大水。十四年三月,壽光海溢,海豐、全州、太湖大水。八月,宜都漢水漲,沖沒民居百餘家;沔陽、潛江、天門、江陵、監利、漢川大水。十五年三月,平遠大水,連日洪水漲發,壞田屋,漂沒人畜無算。五月,樂亭海潮,運河上,田禾盡淹;英山大水,淹沒田廬;肅寧、阜平、武進、阜陽大水;淳安水驟發,田禾淹沒。六月,日照水溢;隨州溳水溢,壞民田舍;富平、容城、祁州大水。十六年三月,濰縣海水溢;掖縣大風雨,海水溢,漂沒人畜。四月,平度海溢;兗州府屬大水。七月,東昌、日照、利津、霑化、惠民、蒲臺、壽光、永樂大水,灤州河溢。十七年正月,鄖縣、鍾祥、京山等十六州縣大水。四月,洛川水。六月,雷州、文登、榮成、遵化、陵縣、臨邑大水。七月,仁和、海寧水驟至,田禾盡淹。八月,襄陽、棗陽、宜城、穀城、均州、龍川大水。冬,桐鄉南柵大水。十八年二月,峽江、潛江、沔陽、天門、吉安、蘄水大水。六月,饒平大水,漂沒民房五百六十餘間。八月,海豐、利津海溢,壽光海溢,濱州、霑化、蘭山、剡城、日照大水。九月,淮水溢,壞民舍;涑水漲,淹沒西王等村;太湖、鳳陽、五河大水;信宜大水,淹沒廬舍二百餘間,男婦五十餘口。十月,黃河溢,漂沒民舍甚多;慶雲大水。十二月,天門江溢。

二十年,金鄉、魚臺、潛江、沔陽、荊門、江陵、監利、光化大水。十二月,潛江團湖垸堤潰,光化、壽州、鳳陽、潮州大水。二十一年十二月,五河、德州、金鄉、魚臺、壽張大水,東昌★河決。二十三年,青浦、金鄉、魚臺、濟寧州大水,普寧大水入城。二十四年八月,泰州大風潮溢,淹沒禾稼;臨清★河決;太湖、潛山大水。二十五年五月,慶元、洵陽、柏鄉大水。秋,屏山縣百溪水暴漲。二十六年五月,潛江、沔陽等七州縣大水。六月,南宮河水溢;雲夢河水漲,高湧丈餘,田宅盡淹,死者無算;峽江大水溢;江陵、婁縣、固安、永清、寧河、文安、望都、容城、盧龍大水;樂陵、金鄉、魚臺、寧陽、汶上、壽張大水。八月,東昌★河決。二十七年四月,慶雲、棗強、安肅、望都大水。七月,丘縣漳水溢,淹沒田禾;海鹽潮溢塘圮,水入城,漂沒民居;仁和、錢塘、海寧、餘杭大風雨,山水驟發,★場、田禾盡淹;平湖、蒲臺、義烏、青浦、東昌、德平、黃縣大水。二十八年五月,瑞安潮溢,陸地行舟;資陽大水。二十九年二月,南昌、吉安、九江、漢陽、漢川、武昌、江夏大水。四月,黃安、黃州、黃岡、蘄水、廣濟、石首大水;洞庭湖漲,漂沒民居無算。五月,宣平、達州大水。

三十年三月,長清、惠民、諸城大水。七月,府穀河漲;薊州大水;北山蛟水陡發,漂沒房舍。三十一年秋,東昌★河決,濟南、禹城、惠民、商河、利津、金鄉、魚臺大水。三十二年,江夏、武昌、黃陂、漢陽、荊門州、黃岡、蘄水、羅田、廣濟、江陵、枝江大水。三十三年七月,太原、武清、慶雲、寧河、南樂、安肅、望都大水。三十四年五月,蒼梧、懷集、新樂、溧水大水。六月,太湖溢,武進、潛山、湖州、嘉善大水。十月,江夏、武昌、崇陽、黃陂、漢陽、黃岡、廣濟、江陵、枝江大水。三十五年春,鄞縣、慶元大水。夏,古北口山水暴發,滄州、寶坻、武清、喀喇河屯、望都、洵陽、白河、武寧大水,鄖西漢水溢。秋,濟南、東昌大水。壽光大風雨,海溢,傷民畜無算。三十六年正月,鳳陽大水。五月,寧陽、安丘、壽光、博興大水。秋,五河、鄒平、商河、惠民、東昌、德平大水。

四十年春,直隸省四十州縣大水。八月,河津汾水溢,近城高數尺,次日退。四十一年六月,海子山水驟發,浪高丈許,壞城垣廬舍,人多溺死。秋,代州秋峪口河決,田廬多沒。四十四年六月,臨清★河決;施南清江水溢;鍾祥漢水溢,入城,壞民廬舍;江陵大水,田禾盡淹;宜都、武昌大水。四十五年三月,慶元大水。五月,袁州、義烏大水入城,鍾祥、沔陽、潛江、荊州三★大水。六月,常山大雨,湖水暴發,民房多圮;武清、房山、滕縣大水。九月,慶元、金華大水。四十六年十二月,宜城、江陵、壽光、博興大水。四十七年六月十七日,妻阜、涪二江漲,頃刻水高丈餘,民田廬舍淹沒殆盡。中江、三臺、射洪、遂安、蓬溪、鹽亭同日大水,江夏、武昌、黃陂、漢陽、安陸、德安、瑞安大水。四十八年五月,宣平大水,漂沒田禾。六月,江夏、黃梅、武昌三★、黃岡、廣濟大水。

五十一年春,霑化、崇陽大水。八月,江陵大水。五十三年五月,宜昌大水,沖去民舍數十間;常山、慶元、南昌、新建、進賢、九江、臨榆大水。六月,荊州萬城堤決,城內水深丈餘,官署民房多傾圮,水經兩日始退。漳河溢;枝江大水入城,深丈餘,漂沒民居;羅田大水,城垣傾圮,人多溺死;江夏、漢陽九★、武昌、黃陂、襄陽、宜城、光化、應城、黃岡、蘄水、羅田、廣濟、黃梅、公安、石首、松滋、宜都大水。七月,江陵萬城堤潰,潛江被災甚重;漢陽大水。五十四年五月,瑞安、寧海、東湖大水。八月,安州、臨榆大水。五十五年七月,長清、濱州大水;運河決,水溢,禹城、平原等縣水深數尺。八月,灤州灤河溢;樂亭、武強、高唐大水。五十六年正月,湖州大水。十月,即墨沽河水溢。十一月,保康大水,田廬多沒。五十七年十一月,臨江、吉安、撫州、九江大水。五十八年春,青浦大水;貴定大水,壞民舍。四月,隨州、黃安、南昌大水。七月,海鹽潮溢,壞民舍。大名、元城大水。五十九年三月,★河溢,武城大水,襄陽、光化、宜城、黃安、清苑、蠡縣、撫寧大水,滹沱河溢。六十年五月,漢水溢,麗水、分宜、玉山、潛江、沔陽、松滋大水,硃家阜堤決。

嘉慶二年六月,武進、東平、良鄉、天津、靜海、青縣、滄州大水。七月,樂亭、永清大水;寧都霪雨,水驟發,毀民居瓦房一萬八千九百三十間,草房一千二百四十五間,淹斃男婦四千三百九十二名。三年夏,武昌、文安大水。四年二月,蠡縣大水。七月,長清大水。五年,霸州、河間、任丘、隆平、晉寧、定州大水。六年春,禹城運河決,水至城下;長清、觀城、任丘、靜海、黃縣、平鄉大水;滹沱河溢,田禾盡沒;鎮西堤決。六月,武清、昌平、涿州、薊州、平谷、武強、玉田、定州、南樂、望都、萬全、大興、宛平、香河、密雲、大城、永清、東安、撫寧、南宮、金華大水,灤河溢,永定河溢。七月,義烏大雨,江水入城;新城、縉雲大水。七年四月,義烏大水,禾盡沒。五月,定海大水,田禾盡沒。七月,新城大水,漂沒民房一萬七千餘間;漢川、沔陽、鍾祥、京山、潛江、天門、江陵、公安、監利、松滋等州縣連日大雨,江水驟發,城內水深丈餘,公安尤甚,衙署民房城垣倉★均有倒塌,而人畜無損。九月,鄖西大水,鍾祥大水,堤決。八年五月,隨州大水。冬,黃河溢,大水;東阿河決,壞民田廬舍;東昌河決;蒲臺、利津、濱州、霑化、雲夢、範縣、觀城大水。九年三月,南昌、撫州、贛州、九江大水。十年六月,文安、安州、新城、霸州大水。十一年七月,溫州、寧波、鍾祥、珙縣大水。十三年三月,武進、望都、清苑、定州大水。五月,新城、慶元大水。七月,慶元復大水。九月,南宮、袁州、九江大水。十四年四月,望都、房縣大水。六月,南宮大水。十五年四月,新林、宜城大水。六月,濟南大水。七月,永定河溢,南宮、平度、廣元、鹽源大水。十月,宜城大水。十六年四月,保定、文安、大城、固安、永清、東安、宛平、涿州、良鄉、雄縣、安州、新安、任丘大水。秋,肥城、即墨、平度、寧海大水。十七年春,南昌、臨江大水。五月,竹溪大水入城。六月,麗水、房縣大水。二十年四月,歷城、長清大水。二十二年七月,宜城、穀城、嬰武大水。二十四年二月,黃縣大水,沖塌民房,人多溺死。四月,唐山、灤州大水。二十五年,宣化、寧晉、寧河、寶坻、文安、東安、涿州、高陽、安州、靜海、滄州、埠山、大名、南樂、長垣、保安、萬全、懷安、西寧、懷來、新河、豐潤大水。六月,麗水大水。

道光元年三月,寧津大水。五月,保康、隨州、博興、即墨大水。秋,濟南、惠民、商河、霑化、潛江、任康大水。二年正月,鍾祥大水,堤決;潛江大水。五月,光化漢水溢;竹山、鄖縣大水。六月,武城河決;武強河水溢;清苑、唐山、蠡縣、任丘、曲陽大水。七月,定遠、應城大水。八月,霑化徒駭河溢;東昌★河決,壞民田;長清、日照、菏澤、觀城、鉅野大水。三年三月,石首、江陵大水;郝穴堤淤;平鄉、固安、武清、平谷、清苑、蠡縣、任丘、青縣、曲陽、玉田、霸州大水。六月,武城河決;江山、黃梅、鉅野、通州大水;東昌★河決。七月,太湖溢;鮑家壩決,下河禾稼盡淹,蘇州、高淳大水。四年二月,大興、宛平等九州縣大水。七年五月,房縣汪家河水溢,壞田廬無算;西河水溢入城;蘄州大水,漂沒田廬人畜;江陵大水。六月,枝江大水入城;日照大水。八月,崇陽山水陡發,城中水深數尺;潛江大水堤潰。九年秋,霑化、長清大水。

十年五月,通山水陡發高數丈,淹沒田廬人畜無算;崇陽大水。六月,枝江大水入城,漂沒田廬;宜都、興山大水。十一年,貴築、黃安、黃岡、麻城、蘄水、公安、宜都大水;石首堤潰。六月,雲夢堤決,漂沒田廬無算;房縣、安陸大水。七月,日照、清苑、惠民、商河、霑化、高淳、武進大水。八月,鍾祥大水漫堤,黃陂、漢陽大水。十一月,陸河水大漲,房縣、黃州、應山、武昌、南昌、南康、瑞州、袁州、饒州、撫州、文安、清苑大水。十二年夏,松滋堤決;江夏、應山、麻城、鄖縣大水,民房多壞;玉田大水。七月,鍾祥大水,堤決;漢江暴漲,城圮二百四十餘丈,溺人無算;堵水溢,壞官署民房過半;襄陽、宜城大水。八月,均州漢水溢入城,深七尺,民房坍塌無算;應城水溢,青田大水。九月,觀城、鉅野大水,武城河決。十三年春,平鄉大水。四月,貴溪、江山、咸寧、江夏、黃陂大水;武昌大水至城下。五月,公安、宜都、歸州大水。六月,漢江溢;黃岡、蘄州、黃梅大水;大興、宛平、望都、撫寧、石首、公安、松滋大水。五月,麗水、孝義大水。六月,榆林大水,淹沒田禾;縉雲大水。十五年五月,沔縣漢水溢,漂沒田廬;鍾祥大水。七月,霑化、蒲臺、邢臺大水。十六年春,寧海海溢,淹沒民田。七月,鍾祥大水堤潰。十八年六月,宜都水溢,南陽淹沒民居甚多。七月,恩施清江水溢。十九年正月,惠民、霑化、濟寧州大水。三月,枝江大水入城,公安、松滋、鄖西大水。四月,鍾祥大水堤潰。六月,武昌、臨江大水;文昌、天門、公安、枝江、宜都、松滋大水。六月,汶水溢;臨邑、陵縣大水;玉田大水,相繼五年,被災甚重。秋,靜海溢,禾稼盡沒;霑化大水;沔縣漢水溢。

二十一年夏,武昌、黃陂、漢陽、松滋、黃州、鍾祥大水。二十二年五月,江陵大水入城,松滋大水。二十四年七月,嵊縣堤潰,溺死七十餘人;江陵大水,城圮;松滋、枝江大水入城;南昌、袁州、饒州、南康、惠民、霑化、蒲臺大水。二十五年六月,東平大水。七月,青縣、縉雲、雲和、太平、公安大水,樂亭海溢。二十六年正月,灤河溢。五月,枝江大水入城;青浦大水,漂沒數千家。六月,汶水漲,堤決;青縣大水。二十七年,鹽山等二十六州縣大水。二十八年,松滋、安陸、隨州大水;黃州大水至清源門;保康大水,田廬多損。六月,南昌、袁州、饒州、南康、陵縣大水;雲夢山水陡漲,堤盡潰;咸寧、江夏、黃陂、漢陽、高淳、武清大水;蒲圻水漲,高數丈。十二月,隨州、應山、黃岡、江陵、公安大水。二十九年四月,應山大水,居民漂沒無算;黃岡大水入城;蘇州、嘉興大水;湖州大水,田禾盡淹。五月,興安、黃陂、漢陽大水,蠻水溢。六月,公安、羅田、麻城、蘄水、歸州、宜昌、蒲圻、咸寧、安陸大水,黃州大水入城,枝江大水入城。七月,三原河溢,漂沒田舍,溺人甚多;日照大水;武昌大水,陸地行舟。十二月,桐鄉大水,田禾盡淹。三十年六月,黃河漲,漂沒田廬無算;青田、東平大水。

咸豐元年正月東平,夏太平大水。秋,懷州大水。二年六月,平河、高陽大水。七月,鍾祥、穀城、襄陽、潛江、公安大水。三年三月,麗水大水。五月,孝義、嵊縣、太平大水。六月,左田、如德山水暴漲,平地深丈餘。七月,保定府屬大水;宜城漢水溢,堤潰,城垣圮一百五十丈;均州大水入城。四年五月,松陽大水;廣昌蛟出水,西南北三面城圮,淹斃男婦以萬計,官、民舍僅存十之一二。秋,保定府屬大水。五年七月,麗水、雲和大水;景寧山水暴發,田廬盡壞,黃陂、麻城、黃岡、蘄州、廣濟陂塘水溢。十二月,鍾祥水暴溢。六年五月,嵊縣、太平大水。七年夏,松滋、枝江大水。七月,縉雲、濱州大水。八年十二月,江陵、松滋、公安大水。十一年六月,鍾祥大水堤潰。七月,景寧大水。

同治元年五月,公安大水,日照大水。秋,臨江大水。二年春,湖州海水溢。六月,鍾祥大水;潛江高家拐堤決;保康大水,淹沒田舍;公安大水。秋,鄖西大水。三年夏,公安大水。秋,鄖西大水。四年四月,公安大水。五年夏,公安、德安、崇陽、咸寧大水。九月,臨江、江夏大水。六年三月,羅田大水。五月,江陵、興山大水。八月,宜城漢水溢,入城深丈餘,三日始退;襄陽、穀城、定遠、沔縣、鍾祥、德安大水;潛江硃家灣堤潰。九月,臨邑大水,黃河溢。九年六月,滹沱河溢;宜城漢水溢;公安、枝江大水入城,漂沒民舍殆盡;歸州江水暴溢;黃岡、黃州大水。秋,孝義、武昌、黃陂大水。十年夏,武清、平谷大水。秋,公安大水,泗河堤潰。十一年三月,公安、枝江大水。六月,滹沱河溢,漫入滋河;直隸諸郡大水;高淳、甘泉、臨朐大水。十二年六月,公安大水。秋,臨朐、高淳大水,灤河溢,青縣黑港河決。秋,潛江大水。十三年五月,公安大水。秋,甘泉、孝義大水;潛江大水深丈餘,宣平北門外洪水泛濫,水高丈許,沖塌民房八十餘間,男婦二十餘人。

光緒元年二月,青浦、魚臺河決,境內淹沒過半;潛江大水。二年五月,南昌、臨江、吉安、撫州、饒州、南康、九江、潛江大水。六月,青田、宣平大水。八月,邢臺白馬河溢。三年五月,宣平大水。四年夏,常山大水入城,南昌、臨江、吉安、撫州、南康、九江、饒平、廣信、武昌大水。五年五月,玉田、薊運河決;階州大水;文縣大水,城垣傾圮,淹沒一萬八百三十餘人。六月,文縣南河、階州西河先後水漲,淹沒人畜無算。八年三月,武昌、德安大水;常山大水,田禾盡淹,秋復大水。九年正月,玉田、孝義、皋蘭、順天大水。十一年五月,黃河溢,惠民徒駭河溢,霑化大水。十三年秋,灤州、洮州大水。十八年六月,南樂★河決,洮州大水。二十年七月,太平、松門溢,堤盡潰;南樂★河決。

宣統元年六月,蘭州黃河漲,泰安大水。

順治六年七月二十日,上海晚日中黑氣一道,直貫天頂,須臾,海中黑氣一道上升,與日中黑氣相接如橋,至暮乃滅。七年十月十四戌刻,湖州有黑氣一道,自西■東,長百餘丈。九年正月十五日,黃岡雨黑水如墨。十三年正月初一日,衡水有黑氣自西北來,如煙。十四年七月,昆山黑眚見。十一月,含山黑眚見。十五年夏,平湖黑眚見。

康熙二十年八月初四日,澄海見黑氣一條入東門,至北門東林村始滅。十月,宜昌夜間黑眚見。三十四年四月,襄陵黑眚見。

雍正六年三月初九日,鎮洋見黑氣如疋布,良久方散。

乾隆三十五年七月,東光黑氣迷漫,移時方滅。三十九年二月朔,高邑黑眚迷人。四十年四月初五日,高邑黑眚,咫尺不辨。

嘉慶元年秋,棗陽有黑氣自東向西,啐嚓有聲。十四年正月朔,東光有黑氣一道,自西北抵東南,長竟天。

道光二十八年六月,昌黎見黑氣沖。

咸豐三年三月十六日,中★有黑黃氣二道,直沖天際。五年七月初十,曹縣見黑氣寬二三丈,長■天。

同治二年六月,肅州日昃時有黑氣長竟天,半夜方滅。


\end{pinyinscope}