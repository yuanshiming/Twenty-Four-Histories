\article{志八十七}

\begin{pinyinscope}
○選舉七

△捐納

清制,入官重正途。自捐例開,官吏乃以資進。其始固以蒐羅異途人才,補科目所不及,中葉而後,名器不尊,登進乃濫,仕途因之殽雜矣。捐例不外拯荒、河工、軍需三者,曰暫行事例,期滿或事竣即停,而現行事例則否。捐途文職小京官至郎中,未入流至道員;武職千、把總至參將。而職官並得捐升,改捐,降捐,捐選補各項班次、分發指省、翎銜、封典、加級、紀錄。此外降革留任、離任,原銜、原資、原翎得捐復,坐補原缺。試俸、歷俸、實授、保舉、試用、離任引見、投供、驗看、回避得捐免。平民得捐貢監、封典、職銜。大抵貢監、銜封、加級、紀錄無關銓政者,屬現行事例,餘屬暫行事例。

歷代捐例,時有變更,惟捐納官不得分吏、禮部,道、府非由曾任實缺正印官,捐納僅授簡缺,則著為令。銓補則新捐班次視舊班為優,此通例也。捐事戶部捐納房主之,收捐或由外省,或由部庫,或省、部均得報捐。咸豐後,並由京銅局。

凡報捐者曰官生,部予以據,曰執照。貢監並給國子監照。俊秀納貢監或職銜,貢監納職銜,由原籍地方官查具身家清白冊,季報或歲報。納職官者,查明有無違礙,取具族鄰甘結,依限造報。逾限或查報不實,罪之。其大略也。

文官捐始康熙十三年,以用兵三籓,軍需孔亟,暫開事例。十六年,左都御史宋德宜言:「開例三載,知縣捐至五百餘人。始因缺多易得,踴躍爭趨。今見非數年不克選授,徘徊觀望。宜限期停止,俾輸捐恐後。既有濟軍需,亦慎重名器。」帝納其言。滇南收復,捐例停。嗣以西安、大同饑,又永定河工,復開事例。五十一年,增置通州倉廒,科臣有請開捐者,廷議如所請。侍郎王掞抗疏言:「鄉里童騃,一旦捐資,儼然民上。或分一縣之符,或擁一道之節,不惟濫傷名器,抑且為累地方。宜禁止,以塞僥幸之路,杜言利之門。」帝韙之,為飭九卿再議。青海用兵,饋餉不繼,內大臣議停各途守選及遷補,專用捐資助餉者。刑部尚書張廷樞言;「惟捐納所分員缺可用捐員,正途及遷補者宜仍舊。」從之。

雍正二年,開阿爾臺運米事例。五年,直隸水災,議興營田,從大學士硃軾請,開營田事例。雲貴總督鄂爾泰以滇、黔墾荒,經費無著,請開捐如營田例。帝曰:「墾田事例,於地方有裨益。向因各捐例人多。難於銓選,降旨停止。年來捐納應用之人,將次用完,越數年,必致無捐納之人,而專用科目矣。應酌添捐納事款。除道、府、同知不許捐納,其通判、知州、知縣及州同、縣丞等,酌議準捐。」下九卿議行。十二年,開豫籌糧運例。

先是俊秀淮貢得輸資為教職。已,慮異途人員不勝訓迪表率之責,康熙三十三年,令俊秀準貢捐學正、教諭者改縣丞,訓導改主簿。雍正元年,諭「捐納教職,多不通文理少年,以之為學問優長、年高齒長者之師可乎?」詔改用如前例。

高宗初元,詔停京、外捐例。乾隆七年,上下江水災,命刑部侍郎周學健、直督高斌往同督、撫辦理。尋合疏言賑務、水利需費浩繁,請仿樂善好施例,出資效力者,量多寡敘職官。詔以京官中、行、評、博以下,外官同知、通判以下,無礙正途,如所請行。嗣是上下江、直隸、山東、河南屢告災,輒徇臣工請,許開捐例。十三年,進剿大金川,四川巡撫紀山奏行運米事例,部議運米石抵捐銀二十五兩,納官以是為差。川陜總督張廣泗言:「軍前口糧領折色,石發銀五、六兩。事例既開,各員以存米納捐,計貢監納即用同知不過千餘金,即用小京官不過數百金,請令如數交銀,以杜弊端。」報可。三十九年,再徵金川,復開川運例。惟四庫館謄錄、議敘等職,多靳不令捐納,餘得一體報捐。貢監納道、府例,自雍正五年後,數十年無行者,至是復行。

五十八年,詔曰:「前因軍需、河工,支用浩繁,暫開事例,原屬一時權宜。迄今二十餘年,府庫充盈,並不因停捐稍形支絀。可見捐例竟當不必舉行。不特慎重名器,亦以嘉惠士林,我子孫當永以為法。倘有以開捐請者,即為言利之臣,當斥而勿用。」

嘉慶三年,從戶部侍郎蔣賜棨請,開川楚善後事例,帝慮正途因之壅滯,飭妥議條欸。尋議:「京官郎中、員外郎,外官道、府,有理事親民之責,未便濫予登進。進士,舉人,恩、拔、副、優、歲貢,始許捐納。非正途候補、候選正印人員,亦得遞捐。現任、應補、候選小京官、佐貳,止準以應升之項捐納。」從之。嗣以河屢決,續開衡工、豫東、武陟等例。十一年,定捐納道、府,系曾任知府、同知、直隸州知州並州、縣正印等官加捐,及現任京職,堪勝繁缺者,許以繁簡各缺選用。其貢監初捐,及現任京職僅堪簡缺,並外任佐雜等官遞捐者,專以簡缺選用。

宣宗、文宗御極之初,首停捐例,一時以為美談。自道光七年開酌增常例,而籌備經費,豫工遵捐,順天、兩廣及三省新捐,次第議行。其時捐例多沿舊制,惟於推廣捐例中準貢生捐中書,豫工例中準增、附捐教職而已。咸豐元年,以給事中汪元方言,罷增、附捐教職,其已選補者,不許濫膺保薦。是年特開籌餉事例;明年,續頒寬籌軍餉章程。九年,復推廣捐例。時軍興餉絀,捐例繁多,無復限制,仕途蕪雜日益甚。同治元年,御史裘德俊請令商賈不得納正印實官,以虛銜雜職為限。下部議行。尋部臣言捐生觀望,有礙餉需,詔仍舊制。四年,山東巡撫閻敬銘言:「各省捐輸減成,按之籌餉定例,不及十成之三。彼輩以官為貿易,略一侵吞錢糧,已逾原捐之數。明效輸將,暗虧帑項。請將道、府、州、縣照籌餉例減二成,專於京銅局報捐。」從之。時內則京捐局,外則甘捐、皖捐、黔捐,設局遍各行省。侵蝕、勒派、私行減折,諸弊並作。

光緒初,議者謂乾隆間常例,每歲貢監封典、雜職捐收,約三百萬。今捐例折減,歲入轉不及百五十萬。名器重,雖虛銜亦覺其榮,多費而有所不惜。名器輕,則實職不難驟獲,減數而未必樂輸。所得無幾,所傷實多。停捐為便。時復有言捐官宜考試,花翎及在任、候選等捐宜停者。輒下部議。五年,帝以捐例無補餉需,實傷吏道,明詔停止。未幾,海疆多故,十年,開海防捐,如籌餉例,減二成核收,常例捐數並核減。是時臺灣甫開實官捐。他如四川按糧津貼捐,順天直隸、河南、浙江、安徽、湖北各賑捐,戶部廣東軍火捐,福建洋藥、茶捐,雲南米捐,自海防例行,惟川捐如舊,餘或並或罷。十三年,河南武陟,鄭州沁、黃兩河漫決。御史周天霖、李士錕先後請開鄭工例,以濟要工。部議停海防捐,開鄭工捐。十五年,籌辦海軍,復罷鄭工,開海防新捐。新捐屢展限,行之十餘年。二十六、七年間,江寧籌餉,秦、晉實官捐,順直善後賑捐,次第舉辦。江寧順直捐視新海防例,秦、晉捐但獎五品以下實官。庚子變後,帝銳意圖治,言者多謂捐納非善政,詔即停止。然報效敘官,舊捐移獎,且繼續行之。但有停捐之名而已。

武職捐,雍正初惟納千、把總。乾隆九年,直賑捐有納衛守備者。三十九年,川運例,參、游、都、守始得遞捐。但武生、監生捐止都司。嘉慶三年,川楚善後例,武營捐納,略如川運。同治五年,閩浙總督左宗棠言:「閩省武營捐班太多,應嚴加區別,以肅軍政。」並請罷武職捐,從之。光緒二十一年,新海防例展限,議增武職捐。於揀發外別立一班,俾捐輸踴躍。三十一年,兵部奏:「開捐十年,入款僅十餘萬,無裨國帑,有兒營伍。請將實官、虛銜捐復翎銜、封典一切停罷。」報可。捐例初開,慮其弊也,嘗設為限制,往往不久而其法壞。康熙十八年,定捐納官到任三年稱職者,具題升轉,不稱職者題參。然疆吏罕有以不職上聞者。已,令道、府以下捐銀者免具題,照常升轉。左都御史徐元文言:「國家大體所關,惟賢不肖之辨。三年具題,所以使賢者勸,不肖者懼。輸銀免具題,是金多者與稱職同科。此曹以現任之官營輸入之計,何所不至?急宜停止。」

順治間,準貢、例監出身官不得升補正印。康熙六年,定為保舉之法,各途出身官,經該堂官及督、撫保舉稱職者,升京官及正印。無保舉者,升佐貳、雜職。三十年,大軍征噶爾丹,戶部奏行輸送草豆例,準異途人員捐免保舉。御史陸隴其言:「捐納一事,不得已而暫開,許捐免保舉,則與正途無異。且督、撫保舉之人,必清廉方為合例。保舉可捐免,是清廉可納資得也。」又言:「督、撫于捐納人員,有遲至數年不保舉亦不糾劾。乞敕部通稽捐納官到任三年無保舉者,開缺休致。」疏下九卿,議:「捐免保舉,無礙正途。若三年無保舉即休致,則營求保舉益甚,應毋庸議」。隴其持之益堅,廷議隴其不計緩急輕重,浮詞粉飾,致捐生觀望,遲誤軍機,擬奪職。帝特宥之。自是吏員例監出身者,欲升補或捐納京、外正印官、必先捐免保舉,惟準貢獨否。初,納歲貢者同正途,故捐免保舉例開,貢監雖同一捐納,而軒輊殊甚。乾隆二十六年,部議御史王啟緒奏豫工例內,捐貢納京、外正印官,捐免保舉,如例監例。先納官者,補行捐免。不原者,以佐貳改補。成例為一變矣。漢軍捐納官,非經考試,不得銓選,如漢官保舉例。康熙間,並準捐免。六十一年,帝以捐納部員補主事未久即升員郎,外官道、府亦然,飭議試俸之法。尋議郎中、道、府以下,小京官、佐雜以上,於現任內試俸三年,題咨實授,方許升轉,從之。乾隆間,試俸復得捐免。四十一年,戶部奏請保舉、考試、試俸、捐免例,列入常捐。限制之法,至是悉弛。

官吏緣事罷譴,降革留任,非數年無過,不得開復。康熙間,大同賑饑,部議京察、大計罷黜者,悉予捐復。徐元文力言不可。議遂寢。三十三年,河道總督於成龍以黃、運兩河,工費繁鉅,請仿陜西賑饑例開捐,革職、年老、患疾、休致人員得捐復。帝面諭捐納稱貸者多,非朘削無以償逋負,事不可行。尚書薩穆哈等議成龍懷私妄奏,擬褫職,得旨從寬留任。乾隆九年,直賑捐,部議捐復條款,京察、大計及犯私罪者,降調人員,無論是否因公,及比照六法條例,武職軍政糾參及貪婪者,不準捐復。因公罣誤無餘罪,悉得捐復。三十五年,帝念降革留任人員,因公處分,輒停升轉,詔許捐復。三十九年,川運例增進士、舉人捐復原資例。四十八年,定革職、降調官,分段承修南運河工程捐復例。嘉慶三年,川楚善後,推廣其例,凡常捐不準捐復人員,酌核情節,得酌加報捐。奉旨,降革除犯六法外,因公情節尚輕人員,得加倍捐復。大計劾參,有疾休致,調治就痊,及特旨降革留任限年開復人員,加十分之五捐復。十年,部臣疏請於常例捐復外,增文、武大員捐復革職留任例。帝曰:「大員身罣吏議應罷斥,經改革職留任,開復有一定年限。若甫罹重譴,即可捐復,此例一開,亳無畏忌。有資者脫然為無過之人,無資者日久不能開復。殊失政體。」不允行。咸豐二年,王、大臣等議寬籌軍餉。凡降革不準捐復人員,除實犯贓私外,餘準加倍半捐復。降革一、二品文、武官,向不在捐復之列者,許捐復原官頂帶,允行。但飭一、二品大員捐復原銜須請旨。嗣復推廣,文職京察、大計六法,武職軍政被劾,無奸贓情罪,亦許捐復原銜。終清世踵行,不復更也。

捐納官或非捐納官,於本班上輸資若干,俾班次較優,銓補加速,謂之花樣。康熙十三年,知縣得納先用、即用班,工部侍郎田六善極言其弊,謂宜停止。三十三年,戶部議行輸送草豆例,臺臣請增應升、先用捐。御史陸機言:「前此有納先用一例,正途為之壅滯。皇上灼見其弊,久經停止。納先用者,大都奔兢躁進。多一先用之人,即多一害民之人。不待辨而知其不可。」乾隆年事例屢開,惟雙月、單月,不論雙月選用及雙月先用,不論雙、單月即用等尋常班次。蓋是時正途銓補,未病雍滯,無庸加捐花樣,納資者亦至是而止。七年,部議鼓勵江省賑捐,各班選用特優。道光年,增插班間選、抽班間選、遇缺、遇缺前等名目。咸豐元年,省遇缺、遇缺前,而增分缺先、本班侭先。三年,復增分缺間、不積班。九年,先後奏設新班遇缺、新班侭先、分缺先前,分缺間前、本班侭先前、不論班侭遇缺選補等班。推廣捐例,又有保舉捐入候補班、候補捐本班先用例。花樣繁多,至斯已極。

自籌餉例開,既多立班次以廣捐輸,復減折捐例以期踴躍。時納捐率以餉票,成數或不及定額之半。同治三年,另訂加成新章。於是有銀捐新班、侭先、遇缺等項,輸銀不過六成有奇,而選用之優,他途莫及。八年,吏部以銀班遇缺占缺太多,擬改分班輪用,刪不積班,於新班遇缺上,別設十成實銀一班,曰新班遇缺先,是謂大八成花樣。維時分缺先前、分缺間前、本班侭先前、新班遇缺、新班遇缺先,統曰銀捐。而新班遇缺先最稱優異,新班遇缺次之。序補五缺一,先用新班遇缺先三人,然後新班遇缺及各項輪補班各得其一。光緒二年,江蘇巡撫吳元炳言:「新班遇缺先、新班遇缺等班,序補過速,有見缺指捐之弊。請停捐免試用例,以救其失。」格於部議。四年,實官及各項花樣一律停捐。七年,御史葉廕昉復言:「近年大八成各項銀捐班次,無論選、補,得缺最易,統壓正途、勞績各班。今捐例已停,請改訂章程,銀捐人員,祗列捐班之前。」疏下部議。然積重難返,進士即用知縣,非加捐花樣,則補缺綦難,他無論已。十年,臺灣海防相繼例開,三班分先、分間、侭先,復得一體報捐,而知縣並增海防新班。十三年,鄭工新例增遇缺先班捐例等,大八成班次亦相埒,海防新例因之。至二十七年,各項花樣隨實官捐並停。

初捐納官但歸部選,乾隆間,為疏通選途,許加捐分發。二十六年,豫工例,京職郎中以下,得捐分各部、院。外官道、府以下,得捐分各省。三十九年,川運例,知州、同知、通判捐分發如舊。知縣有兒正途補用,靳不與。四十年,兵部侍郎高樸言:「捐班知縣,不許分發,恐有兒舉班。查壬辰科會試後,揀選分發,已閱四年,湖北、福建均因差委乏人,奏請揀選,可見舉班漸已補完。請變通事例,川運捐不論雙單月即用者,許一體報捐分發。」部議如所奏行。惟大省分發不得逾十二人,中省不得逾十人,小省不得逾八人。雲、貴兩省需員解送銅鉛,雲南得分發二十人,貴州如大省額。從之。是年兵部奏請候補、候選衛守備、衛千總如文職例,加捐分發,隨漕學習。明年,浙江巡撫三寶奏請教職捐不論雙單月即用者,設加捐分發,到省委用。均報可。川運例停分發,歸入常例報捐,為永例。四十二年,以山東布政使陸燿言東省分發佐雜漸多,停布政司經歷、理問、州同以下佐雜官分發例。四十六年,候補布政司經歷鄭肇芳等、候選州同張衍齡等具呈戶部,以投供日久,部選無期,各省佐雜班已疏通,請準報捐分發,為奏行如舊例。嘉慶四年,給事中廣興請將俊秀附生報捐道、府、州、縣者,停銓實缺,準加捐分發。責成督、撫試看三年,酌量題補。帝以停選示人不信,令加捐分發,有礙政體,不允行。道、咸間,增加捐指省例。光緒四年,捐例停,而分發指省以常例得報捐如故。五年,御史孔憲以指省分發,流弊不可勝言,請罷之。格部議,不果行。八年,復申前請,部覆如議。未幾,海防例開,仍準報捐。時分發人員擁擠殊甚,疆吏輒奏停分發,期滿復請展限,各直省比比然也。

定例,捐納官分發各部、院學習三年,外省試用一年。期滿,各堂官、督、撫實行甄別奏留,乃得補官。嘉慶十六年,諭:「捐納員簽分部、院學習行走年滿,當詳加甄別。近來該堂官於行走報滿人員,無不保留。市恩邀譽,不顧登進之濫,可為寒心。」道光八年,諭:「酌增常例報捐,分發人員為數更多,著各督、撫、鹽政留心察看,不必拘定年限,認真甄覈。」然奉行日久,長官循例奏留,徒有甄別之名,不盡遵上指也。咸豐七年,從御史何兆瀛請,詔各部、院考試捐納司員,察其能否辦理案牘。尋兵部試以論題,御史硃文江以為言,詔切責之。命嗣後毋得以考試虛文,徒飾觀聽。外官分發到省,例由督、撫考試,分別等第,黜陟有差。光緒初,各省遵例考試,顧雲南有咨回降調者。五年,詔各省考試捐納人員,府、、州、縣試論一,佐雜試告示判語。八年,閩浙總督何璟言:「閩省應試府、、州、縣百五十四員,鹽大使五十五員,佐雜五百九十六員,知府、直隸州知州、鹽大使取留十之五,同、通、佐雜留十之四。」報聞。三十三年,憲政編查館議覆御史趙炳麟疏,捐納道、府、同、通、州、縣佐雜未到省者,入吏部學治館肄業半年。已到省,入法政學堂肄業,長期三年,速成一年有半。尋議上考驗外官章程,各省遵章考試,間亦罷黜數人,以應明詔,而於澄清吏治之道無補也。

貢監捐清初已行。監捐沿明納粟例。順治十二年,開廩生捐銀準貢例,從御史楊義請也。十七年,禮部以亢旱日久,請暫開準貢,令士民納銀賑濟。允之。貢監例得考職,康熙六年,御史李棠言:「進士、舉人遲至十年始得一官,今例監考補中書,三年後即升部屬,應停罷。」部覆如議。自是貢監考職,祗以州同、州判、縣丞、主簿、吏目用。初考職例行,各省監生或憚遠道跋涉,或因文理不通,多倩代頂冒者。世宗深知其弊,特遣大臣司考試。雍正五年,令與考者千一百餘人悉引見,時以頂冒避匿者九百餘人。帝於引見員中揀選七十餘人,授內、外官有差。乾隆元年,停考職。三年,令捐納貢監如歲貢例,分別等第,以主簿、吏目考取。捐監未滿三年者不與。道光後,考職例罷。

雍正間,帝以積貯宜裕,允廣東、江、浙、湖廣以本色納監。乾隆元年,罷一切捐例。廷議捐監為士子應試之階,請於戶部收捐,備各省賑濟,從之。三年,詔復行常平捐監例,各省得一體納本色。原定各省捐穀三千餘萬石,數年僅得二百五十餘萬石,復令戶部兼收折色。十年,湖廣總督鄂彌達言:「捐監事例,穀不如銀。銀有定數,穀無成價。易捐穀為捐銀,倘遇荒歉,亦可動支採買。」允行。大學士等復言:「各省納本色,有名無實,請停止,專由部收折色。」得旨:「各省收捐不必停,在部捐折色者聽。」三十一年,以陜、甘監捐積弊最甚,詔停罷。尋並罷安徽、直隸、山西、河南、湖南北,惟雲南、福建、廣東收本色如舊。三十九年,陜西巡撫畢沅、陜總督勒爾謹請如例收納監糧,允之。是年甘省奏報六個月內捐監萬九千十七名,監糧八十餘萬石。帝疑之。布政使王亶望主其事,私收折色,減成包辦,更虛報賑災,侵冒鉅款。繼任布政使王廷贊知其弊,不能革。事覺,置亶望、勒爾謹、廷贊於法,官吏緣是罷黜者數十人,報捐監生或加捐職官者,分別停科、罰俸、停選。其後監捐無復納粟遺意矣。貢捐屬常例,向於部庫報捐。嘉慶間,疆吏屢以為請,輒阻部議。十二年,部臣言庫帑充裕,請變通常例,各省一體收捐。報可。

此外尚有捐馬百匹予紀錄、運丁三年多交米三百石給頂帶之例。其樂善好施例內,凡捐資修葺文廟、城垣、書院、義學、考棚、義倉、橋梁、道路,或捐輸穀米銀兩,分別議敘、頂帶、職銜、加級、紀錄有差。餘如各省鹽商、士紳,捐輸鉅款,酌予獎敘。皆出自急公好義,與捐納相似,而實不同也。


\end{pinyinscope}