\article{列傳八十七}

\begin{pinyinscope}
沈起元何師儉唐繼祖馬維翰餘甸王葉滋劉而位

沈起元,字子大,江南太倉人。康熙六十年進士,選庶吉士,改吏部主事。擢員外郎,以知府發福建用。總督高其倬令權福州,調興化。時世宗聞福建倉穀虧空,遣廣東巡撫楊文乾等往按,被劾者半,受代者爭為煩苛,起元獨持平。莆田民因訟互斗,其倬恐釀亂,令捕治。起元責兩人而釋其餘,報曰:「罪在主者,餘不足問也。」尋攝海關,裁陋規萬餘金。巡撫常安有奴在關,以索費困商舶。起元聞,立督收稅如額,令商舶行,白常安斥奴。自是人皆奉法。其倬奏開南洋,報可。已,復令商出洋者,必戚里具狀,限期返,逾者連坐。起元曰:「人之生死,貨之利鈍,皆無常,戚里豈能預料?且始不聽出洋則已,今聽之,商造船集貨費不貲,奈何忽撓以結狀?若令商自具狀,過三年不歸,勿聽回籍,不猶愈乎?」其倬從之。

調臺灣。臺灣田一甲準十一畝有奇,賦三則:上則一甲穀八石,中則六石,下則四石,視內地數倍。然多隱占,民不甚困。時方清丈,占者不得匿。其倬欲使臺灣賦悉視內地下則,恐不及額致部詰。起元令著籍者仍舊額,丈出者視內地下則。俟隱占既清,更減舊額重者均於新額,賦不虧而民無累。起元在福州,以辨冤獄忤按察使潘體豐,體豐中以他事,鐫四級,遂告歸。

高宗即位,起江西驛鹽道副使。乾隆二年,擢河南按察使。會久雨,被災者四十餘縣,饑民四走,或議禁之。起元謂:「民饑且死,奈何止其他徙?」令安置未被水諸縣,給以糧,遂無出河南境者。巡撫雅爾圖檄府縣修書院,以起元總其事,乃教群士省身克己之學。立章善坊,書孝子、悌弟、義夫、貞婦名,採訪事實,為章善錄版行,一時風動。

七年,遷直隸布政使。大旱議賑,總督高斌欲十一月始行,起元力請先普賑一月,俟戶口查竣,再分別加賑。有倡言賑戶不賑口者,起元曰:「一戶數口,止賑一二,是且殺七八人矣!」檄各屬似此者罪之。戶部尚書海望奏清理直隸旗地,有司違限,旨飭責。斌將劾數州縣應命,起元不可,曰:「旗地非旦夕可清,州縣方賑災,何暇及此?獨劾起元可也!」乃止。九年,內轉光祿寺卿。十三年,移疾歸。

起元自少敦厲廉恥,晚歲杜門誦先儒書。臨沒,言:「平生學無真得。年來靜中自檢,仰不愧,俯不怍,或庶幾焉!」

何師儉,字桐叔,浙江山陰人。以納貲,於康熙六十年選授兵部員外郎。奉職勤懇,常數月不出署。雍正元年,遷廣西右江道僉事,部請留任一年,世宗命以升銜留任,賜人葠、貂皮。師儉以執法卻重賄,忤要人,因誣以避瘴故留部。侍郎李紱昌言曰:「今部曹不名一錢,才者尤勞瘁,茍得郡,爭趨之,況監司耶?」期滿,復請留,加按察司副使銜。司疏奏皆出其手,他司事難治者亦時委之。

三年,出為江南驛鹽道副使,上召對,勉以操守,復賜人葠、貂皮,許上疏言事。四年,調廣東糧驛道副使。歲大祲,師儉以存留米五萬石給餉,飭所屬緩徵。或疑專擅獲咎,師儉曰:「請而後行,民已苦箠楚矣!」總督孔毓珣與巡撫楊文乾不相能,以師儉署鹽法道,欲引以為助。文乾疑為毓珣黨,令買銅,將以賠累困之。明年,文乾入覲,上示以毓珣彈事,亦及師儉,乃知師儉非阿毓珣者。令署按察使,毓珣又疑師儉暱文乾。及文乾卒,劾師儉違禁開礦,侵蝕銅價。逾年,署巡撫傅泰會鞫,事始白。上知其無罪,命往陜西佐治軍需。

師儉在兵部,諳悉諸邊形勢扼塞、戰守機宜、芻餉緩急。至涼州,每集議,指畫如素習,總督查郎阿深重之。署涼莊道參政。師過涼州,檄至肅州支餉。兩路遙遠,師儉即以涼州所蓄給之,師行無乏。一日羽書數過,師儉策必調取生兵,峙餦以待。已而果然。肅州師將行,飛檄令截取公私騾馬,官民皇皇。師儉曰:「在道官商皆赴肅者,若官頓於途,貨棄於地,非軍前所宜。進剿未有定期,何如聽其至肅,釋所載而後供役?軍前得人與貨,亦省芻茭解送之煩,是獲兩利也。檄雖嚴,吾自當之。」於是官商皆安,軍事亦無誤。

尋調補西安鹽驛道副使。關中旱,詔以湖廣米十萬石自商州龍駒寨運陜西。師儉董其役,未半,大雨谿漲,騾馬少,不足供轉輸。商於山中無頓積所,水次隘,運艘不齊。師儉以秋穀將登,請止運,民亦不饑。軍中馬缺,檄取驛馬。師儉謂:「置郵傳命,如人身血脈,不能一日廢。」拒不可,事竟寢。

擢按察使,數平疑獄。吏有故入人罪者,必按如法,雖貴勢賢親不徇縱。十三年,以目疾乞休。高宗即位,赦詔至,時目已失明,令吏誦案牘,諦聽,得邀赦典者,立出之而後上陳。留兩月,畢其事始歸。後卒於家,陜西祀名宦。

唐繼祖,字序皇,江南江都人。康熙六十年進士,選庶吉士。雍正元年,散館,授編修,轉禮部員外郎。五年,考選浙江道御史。七年,授工科給事中。命察八旗虧帑,律侵挪皆不赦,犯者貧,羈獄二三十年不結。繼祖為核減開除,奏請豁免,積牘一清。命巡西城,回民聚居,頑獷不法,嚴治之,有犯必懲,悉斂戢。建倉東便門外,多發塚墓,毀祠宇,繼祖陳其不便,改地營建,塚墓祠宇並修復。南漕愆期,命赴淮安巡視。繼祖馳至,不更張成法,惟選幹吏催督,懲其疲惰。兩閱月,糧艘悉抵通州。條上漕務利病,下部議行。

七年,命往湖南讞獄,並巡察湖南、湖北兩省,裹糧出,有餽觴酒豆肉,皆卻之,令行禁止。與巡撫趙申喬同按永順苗變獄,群情帖服,苗疆以安。湖南捕役多通盜,奏請捕役為盜,加重治罪,報可,入新例。八年,擢通政司參議。九年,擢鴻臚寺卿。尋命以本銜署河南按察使,旋授湖北按察使。繼祖在兩湖久,熟知吏民情偽。楚俗刁健,黠吏與奸豪通,伺官喜怒,訟益難治。繼祖閉諸胥於一室,不令與外通,訟風衰減。雪監利女子冤獄,按鍾祥民變,皆為時所稱。世宗馭吏嚴,內外大僚凜凜,救過不暇。繼祖一意展舒,所陳奏無不允。上欲大用之,出巡察,賜以摺匣,許奏事,曰:「朕於督撫賢者始賜摺匣,汝宜好為之!」調江西,未之任,以疾乞歸。病愈將出,遽卒。

馬維翰,字墨麟,浙江海鹽人。康熙六十年進士。雍正元年,授吏部主事。甫視事,杖奸胥,銓政清肅。轉員外郎,考選陜西道御史,遷工科給事中,監督倉場,所至有聲。六年,命赴四川清丈田畝,時同奉使者四。維翰分赴建昌道屬,具有條理,糧浮於田者必請減,逾年事竣。御史吳濤在川東丈田不實,以維翰助之。至則發其弊,遂以維翰代任。巡撫憲德薦可大用。八年,留補建昌道副使,疏陳二事:四川俗好訟,州縣斷獄茍簡,案牘不具,奸民輒翻控,淆亂是非,請設幕職以襄治理;又民鮮土著,多結草屋,輕於遷徙,焚劫輒致災,請發官款造磚甓,勸民多建瓦屋。上斥其非政要,以其疏示憲德,謂:「汝薦可大用者乃若此!」然維翰勇於任事,相度要害,改黎州千戶所設清溪縣。烏蒙苗亂,出師會剿,維翰治軍需,供糗糧芻茭,鑿雪通道,與廝卒同甘苦。論剿撫悉中機宜,事乃定。涼山地震數百里,勘災散賑,民感之。礦廠擾蠻,起為亂,方進剿。維翰力陳營兵不戢及各廠病蠻狀,請罷廠撤兵,撫各番,止誅其魁。

在川七年,不阿上官。旋被構,維翰揭部請解職赴質。時親王總部事,特威重,捽使免冠。維翰以手按冠抗聲曰:「奉旨不免冠!」譙問故,則又抗聲曰:「旨解職,非革職也!」部乃疏請奪官。事旋白。乾隆二年,起授江南常鎮道參議。丁父憂,歸,卒於家。

餘甸,字田生,福建福清人。康熙四十五年進士。居鄉勵名節,巡撫張伯行重之,延主鼇峰書院。授四川江津知縣,民投牒者,片言立決遣,訟為之簡。日與諸生誦說文藝,疏解性理。所徵賦即儲庫,不入私室。時青海用兵,巡撫年羹堯督餉,多額外急徵,檄再三至,甸不應。乃使僕持檄告諭,自朝至晡,甸不出,使者譁。甸坐堂皇,命反接,將杖之,丞簿力為請,久之乃釋其縛。越日,使者索檄,甸曰:「汝還報,我閉門待劾,檄已達京師矣。」羹堯亦置之。行取吏部主事,時尚書張鵬翮、侍郎湯右曾皆以幹濟名,甸遇當爭辯者,侃侃無所撓。主選三年,權要富人請託多格不行。將告歸,條文書已駁議未奏者十餘事,曰:「此皆作奸巧法易為所蒙,必上聞,吾乃去。」父憂免喪,猶廬墓。

以河道總督陳鵬年疏薦,擢山東兗寧道。釐工剔弊,一袪積習,甚得士民心。鵬年卒,齊蘇勒為河督,以工事劾甸,行河至濟寧,士民群聚乞還甸。齊蘇勒疏陳,召入見。雍正二年四月,授山東按察使。攜二僕,買驢之官,務崇禮教,輕刑罰,政化大行。十一月,召詣京師。三年,擢順天府丞。

甸歷官盡革陋規,為按察使,愍囚不能自衣食,取鹽商歲饋三之一以資給之。兼完囹圄,修學宮、書院,委有司出入注籍。既去官,上命內閣學士繆沅清察山東鹽政諸弊,舉是劾甸,奪官,歸。甸用唐人詩語為人書楹帖,其人有怨家,訐於有司,以為怨望。有司以甸所書也,並下甸於獄。事白,遽卒。

王葉滋,字槐青,江南華亭人。弱冠,補諸生。浙江巡撫硃軾闢佐幕,器其才。雍正元年,重開明史館,軾薦之,引見稱旨,命入館纂修。舉順天鄉試。福敏督湖廣,世宗命葉滋往贊其幕。五年,應禮部試,甫畢,上召見,問湖廣吏治、民生利弊,奏對甚悉,趣馳傳還湖廣。榜發中式,未與殿試,賜二甲進士,即授常德知府。常德例,知府至,行戶更新照,規費四千金,葉滋革其例。境數被水災,請帑增築花貓新陂堤堰,豁被水荒田額糧,民德之。辰州關木稅為利藪,時議移關常德,葉滋恐累民,拒之,請仍舊制。行法不避豪貴,興學造士,薦舉優行諸生陳悌為武平知縣,貴金馬為上蔡知縣,劉樵為清平知縣,並為良吏。

署岳州、辰州二府,攝岳常道副使。久之,授辰沅靖道副使。時苗疆初闢,清林箐,增汛堠,規模肅然。所屬綏寧、城步與黔疆犬牙錯。嘗率數騎,持酒肉鹽菸,循行苗砦。群苗迎拜,謂「上官親我」。召諸頭人集校場,賜花紅銀牌,宣上德意,勸以禮義。因偕總兵閱兵耀軍容,群苗帖服。署按察使,調糧儲道,舊有漕費,悉歸公用。值貴州苗亂,師進剿,葉滋駐辰州治軍需,剋期辦。綏寧苗蠢動,為貴州苗應。葉滋條上剿撫事,悉中窾要。大吏令駐綏寧指揮,積勞疾作,卒於山中。

葉滋初以文學受知,及官於外,所至有聲績。卒時年僅五十五,世咸惜之。

劉而位,字爾爵,山西汾陽人。康熙五十二年舉人,授河南安陽知縣。有兄弟爭產構訟十餘年者,為據理剖解,至淚下,皆叩頭求罷,案牘遂稀。雍正中,遷福建泉州知府,再遷興泉道參議。鹽政窳敝,商居奇索高直,民苦淡食,不獲已,增價以市。既而鹽不足,民惡其壟斷,聚而毆之。海舶私梟動逾千百,往捕則持械拒,大獄迭興,羅織牽連,數歲不息。而位創議裁引革商,歲額課稅歸灶完納,如農完賦,任人轉運,聽其所之,則諸弊可革而國賦不乏。巡撫趙國麟心韙之,格於例不行。未幾,引疾歸。乾隆三年,起官四川鹽茶道副使。蜀鹽產於井,課由井納,民便之。雍正中有請設引招商增課者,四川鹽政自此壞。商無餘貲,運不足額,民持錢不得鹽,而井鹽滯積不售,因以致訌。而位欲事釐剔,大吏畏難不可,力爭,愈嫉之。改松茂道,調永寧道參議。居常鬱鬱,不得行其志,惟與諸生講學。尋卒於官。

而位生平服膺王守仁,曰:「尊所聞,行所知,須不流於弊。尊陽明而不知其流弊,非善學陽明;尊硃子而不知其流弊,亦非善學硃子。」蓋謂王氏高明,弊在躐等;硃子格物,弊恐拘而不化。著省克引、劉氏家訓,為學者所稱。

論曰:起元深於經術,當朝政尚嚴,能持以平恕。師儉以勤敏,繼祖以明肅,並見重於時。維翰有幹局,甸尤能澤以儒效。葉滋撫循苗疆,未竟其用。而位議變鹽法,亦不得申其志,而但以學術名。國家重視監司,所以擴循良之績,儲封疆之選,若諸人者,可謂無忝矣。


\end{pinyinscope}