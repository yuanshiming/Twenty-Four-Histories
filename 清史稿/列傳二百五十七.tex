\article{列傳二百五十七}

\begin{pinyinscope}
志銳劉從德春勛良弼宗室載穆萬選德霈同源

文瑞承燕克蒙額恆齡德霈等樸壽謝寶勝姚靄雲

黃忠浩楊讓梨等

志銳,字公穎,他塔拉氏,世居扎庫木,隸滿洲正紅旗,陜甘總督裕泰孫。父長敬,四川綏定府知府。志銳幼穎異,光緒六年成進士,選庶吉士,授編修。與黃體芳、盛昱輩相勵以風節,數上書言事。累遷詹事,擢禮部右侍郎。中東事起,上疏畫戰守策累萬言。慮陪都警,自請募勇設防,稱旨,命赴熱河練兵。未逾月,以其妹瑾、珍兩妃貶貴人,降授烏里雅蘇臺參贊大臣,釋兵柄。遂迂道出張家口,策馬逾天山西絕幕。所逕臺站,輒周咨山川、風俗、宗教,箸詩記事。居數年,將軍長庚令赴邊外釐中俄積案,凡六閱月,結千餘起。前後五上疏籌西北防務,發強鄰狡謀,中當軸忌,左遷索倫領隊大臣。領隊例不得專摺奏事,居則鉤稽地形戹塞,出則徼循鄂博、卡倫,冀得當以報。又數年,改授寧夏副都統,疏請發帑二十萬濬城外故渠,獲沃壤數千頃。頻上疏,多言人所不敢言。

宣統二年,遷杭州將軍。明年,調伊犁將軍,加尚書銜。入覲,條上弭邊患、御外侮機宜甚悉;又力陳新政多糜費,請省罷,壹意練兵救危局。並請邊地練兵費百萬,部議止予二十萬。抵新疆,聞武昌變,或勸少留,不可。逾月,到官,日討軍士而申儆之。已,蘭州軍譁變,寧夏繼之。伊犁協統楊纘緒以兵叛,夜據南北軍器庫,攻將軍署。群議舉志銳為都督,峻拒之;迫詣商會,亦弗從,起發槍擊之,遂遇害。其僕呂順奔走營棺斂,撫尸號慟,亦為叛軍所戕。

又武巡捕官劉從德,四川人;教練官春勛,京旗人:並及於難。事聞,贈志銳太子少保,謚文貞。

志銳夙負奇氣,守邊庭逾十稔,自號為窮塞主。工詩詞。熟察邊情,懼禍至無日。其赴伊犁也,以手書遍告戚友,言「以身許國,不作生入玉門想」。其致命遂志,蓋已定於拜疏出國門日雲。

良弼,字賚臣,紅帶子,隸鑲黃旗,大學士伊里布孫。少孤,事母孝。劬學,留學日本陸軍學校,畢業歸,入練兵處。歷陸軍部軍學司監督副使,補司長。時新設禁衛軍,任第一協統領兼鑲白旗都統,遷軍諮府軍諮使。平日以知兵名,改軍制,練新軍,立軍學,良弼皆主其謀。尤留意人才,自將帥以至軍士,莫不延納。思有所建樹,頗為時忌。

武昌亂起,各省響應,朝論紛呶,王公貴人皆氣餒,莫知所為。良弼獨與三數才傑朝夕規畫,外聯群帥,內安當國,思以立憲弭革命,圖救大局,上下皆恃以為重。時袁世凱來京,方議國體,人心不安甚矣。一日,良弼議事歸,及門,有人遽擲炸彈,三日而卒。事聞,震悼,優恤如例。其後官紳請立祠於北京祀之。

良弼剛果有骨氣,頗自負,雖參軍務,無可與謀,常以不得行其志為恨,日有憂色。及遇刺,醫初謂可療,忽有進以酒者,遂死。死未旬日,而遜位詔下,時皆悼之。

宗室載穆,字敬修,隸滿洲鑲藍旗,恂勤郡王允五世孫。祖綿翔,鎮國將軍。父奕云,一等侍衛,記名副都統。載穆年二十,除三等侍衛,累遷頭等,兼辦事章京。以忼直忤上官意,數歲不遷。光緒二十六年,拳亂起,兩宮西幸,痛哭自盡者再,遇救獲免。三十二年,授太原城守尉。明年,有詔遞裁駐防,分遣歸農,乃倡農桑,勸女工,興學校。比去晉,旗民男婦務耕作、嫺織紝者達二百人。省城門有八,舊閉其二。阜城門當汾水沖,河決土壅,不能通車馬,群議閉之。載穆曰:「此汾西數十村入城孔道也,請於舊門南闢新門。」民稱便。秩滿,將入覲,巡撫丁寶楨疏留之,報可。

宣統三年,簡京口副都統。鄂難作,緣江戒嚴。載穆繕城郭,犒軍士,設練兵處,定營防城守章條,晝夜徼循,旗、漢民雜居者皆安堵。已而新軍徙頓鐵道旁,運槍械者繦屬。載穆知有異,遣使如江寧告急,弗應。江蘇巡撫程德全號獨立,傳檄鎮江,防營乃潛通蘇軍,全城益怖懼。於是官紳集議,定滿、漢聯合策,約毋戰,且要旗營繳軍械。載穆知事不可為,罷會大慟,語左右曰:「吾上負朝廷,所欠止一死耳!」左右環跽,請系眾心,維危局。翼日,鎮紳楊邦彥詣軍門趣繳械,不許。會新軍入據漢城,旗營大譁,乃進旗眾而語之曰:「駐防兵單糧儲竭,吾戰死甘如飴。顧糜吾民肝腦膏鋒刃,吾奚忍?若曹其徇眾議,紓急禍。吾身為大臣,且天潢親也,宜效死。」是時驍騎校萬選力爭,請毋止戰,不見用,頓足大哭。印班德霈亦憤甚,曰:「大局休矣!吾寧死以報國。」載穆嘿不語,乃繕遺疏,手自緘印,遣佐領良才賚至京師。復草遺書致商會,猶殷殷以七千人生命相囑。隨行四僕皆遣歸。有李順者,去復返,朝夕侍其側,偶退休,詰朝入寢室,則已自經死矣。郡人哀之,殯斂如禮,且為置田安厝焉。將軍鐵良上聞,命覈覆死事。江寧失,鐵良走,宗人府亦無奏報,故褒贈之典弗及雲。

萬選、德霈並殉。先是驍騎校同源以旗人將失所,忍死爭旗產。至是乃語家人曰:「吾可以從殉國諸公後矣!」沐浴整衣冠,不食而死。萬選,字子昭,蒙古敖漢氏。著有易注、筆諫、金石賞心、火龍攻戰略諸書。德霈,字雨田;同源,字子清:並蒙古人。

文瑞,鈕祜祿氏,滿洲鑲紅旗人。世襲男爵,充頭等侍衛,出為馬蘭鎮總兵。中日之役,喜峰口迫近戰地,策守御,遏內匪,轄境以寧。坐陵樹蟲災免,頃之被宥,除歸化城副都統,兼署綏遠城將軍。拳匪亂,蔓延蒙旗,教案紛糾。文瑞至,與外人推誠商榷,償款獨輕,綏民德之。調青州,念旗民乏生計,為闢工廠,興學校,編制軍隊,滿城一切皆治辦。移成都,未之官,擢西安將軍。興學、勸工,為治復仿青州。

議辦移墾授田法,未及行而鄂變作,西安新軍應之,先據漢城,緣塗縱火,煙焰張天。疾趨南街,遇新軍,前騶戈什哈數人被擊死,紆道歸。與左翼副都統承燕、右翼副都統克蒙額籌應變策,遣軍士畫陴而守,兩軍合戰,自申及亥不少休。翼日昧爽,新軍分攻東、南門,旗兵多傷亡,文瑞督懾益力。未幾,新軍請停戰會議,遣協領葆鈞往,迄未得要領。復貽書新軍,反覆開喻,亦不答。而新軍又兩路夾攻,旗營火器竭,漸不支。日方午,東門破,進滿城,終夕巷戰,旗兵死者二千餘人,餘皆屠殺。麾下壯士從者十餘,及其子熙麟而已。於是環請引避圖恢復,文瑞愾然曰:「吾為統兵大員,有職守不能戡亂,重負君恩,惟有死耳!」乃口授遺疏,趣熙麟書之,命乘間達京師,而自從容整衣冠赴井死。幕僚秦鶴鳴斂之。

承燕同時投井死。

克蒙額,字哲臣,滿洲鑲藍旗人。先請巡撫發新式軍械,遲不應,激戰三晝夜,力竭陣亡。

恆齡,字錫九,舒穆魯氏,滿洲正藍旗人,湖北荊州駐防。恆齡少嗜學,嫺武幹,尤熟中外兵家言。以附生官筆帖式,遷驍騎校,累擢佐領。旗營久習窳惰,罕知兵事,乃創編新軍,設講武堂教之。拳匪亂作,湘人旅荊者被煽動,燔沙市躉船及稅關、領事署,外國僑民多逃避,勢岌岌。恆齡率二百人往鎮撫,誅首要,宥脅從,外人避難者護持之。事寧,軍政課最。將軍綽哈布疏綜營務,恆齡條上四事,曰:設警察,興學校,釐財政,練常備軍,並奏行。設八旗高等學堂、陸軍小學堂,俾任校事。顧其時風氣闇僿,款無所出,遂走謁總督張之洞,面陳規畫,獲助萬金,始成立;猶不足,省新軍陋規益之,歲以為常。於是訂章條,甄材穎,走書幣聘海內名儒,分科教授,校風肅然。學部曹司考察,稱荊州第一。旋領振威新軍,調督練處參議,總辦陸軍小學。將軍恩存、總督陳夔龍交章論薦。

宣統改元,調充熱河練軍統領。汰老弱,補缺額,申嚴紀律,凡兩閱月,獲匪首葛蘭亭等,推功將校。二年,授寧夏副都統,朝陽紳民籥留,夔龍上聞。廷議以西陲邊要,趣到官。既蒞事,首嚴煙禁;開渠屯田,久無效,設方略整飭之。

三年,遭父憂。令甲,旗員百日服除即視事。恆齡固請終制,解職去,奉父喪於萬縣,抵宜昌,鄂亂作,道塗阻絕,將軍連魁疏請參軍事,上命署荊州左翼副都統。恆齡援「墨絰從戎」義,愾然任城守,而援絕餉匱,兵人疲饉則言華變,乃斥家財餉之,涕泣誓眾,令毋擾沙市啟外釁。時方患癰劇,裹創策騎出,晝夜徼循,血痕猶濡縷然。無何,事益亟,外城失。恆齡晨起,公服端坐堂上,發手槍洞胸而歾。家人得其與弟恆廣、子裕文書,曰:「吾家世受國恩,宜竭力圖報。今城既失,義當死。所憾者老母在堂,忠孝不獲兩全。第吾母有子能盡忠,亦甚得。我死,汝曹能闔門殉節固善,否則善事吾母,以補吾不孝之罪,毋以吾死狀令老人知也。」恆齡死數日,連魁與右翼副都統松鶴開門納民軍,荊州遂失。事聞,上震悼,謚壯節。

參謀長德霈自經死。恩霈亦自經,家人救之,憤不欲生,後數日卒。

樸壽,字仁山,滿洲鑲黃旗人。光緒二十年舉人,授吏部主事,累遷郎中。拳亂起,聯軍入城,首與各國謀保商民。出為山西歸綏道,簡庫倫辦事大臣。三十二年,召授鑲藍旗滿洲副都統,遷正黃旗漢軍都統。明年,除福州將軍,整旗務,嚴煙禁,專志訓練,得精卒四千人。宣統三年,省城民軍起,率防軍與搏,火器猛利,民軍幾不支。然民軍雖被創,輒隨時募集,防軍以猛斗故,傷亡多,卒敗潰。樸壽被執,受挫辱,不屈,遂支解之,棄尸山下,其死狀為最烈云。事聞,贈太子太保,予二等輕車都尉世職,謚忠肅。

謝寶勝,字子蘭,安徽壽州人。初隸金順麾下,從征西陲。嗣隨宋慶、馬玉昆克肅州及關外諸城,積勛至都司。以事與玉昆左,棄冠服走博克達山為黃冠。光緒十五年,玉昆提督畿輔,鳩集舊部,獨偉視寶勝,招之出。敦促備至,寶勝愾然曰:「玉昆知我者,義不忍卻!」乃棄黃冠,詣軍所獻方略。二十一年,朝鮮告警,從出關,與日軍數十戰,勇敢躐倫等。玉昆弟陷重圍,銳身救之出。和議成,憤甚,復為道士裝,羈跡京師白雲觀,如是者數年。

拳亂作,柴洪山統武衛護軍,榮祿檄領前路後營,已留河南,更名精銳軍,領左營,尋筦豫北軍。忌者中以蜚語,巡撫吳重憙疏辨其冤,上卒優容之。駐軍河、陜、汝最久,將士積相畏服,軍麾所指,紀律肅然。累遷至副將。

宣統改元,授河北鎮總兵。明年,移南陽,河、陜、汝軍仍受節度。寶勝益感奮,尤嚴治盜,所蒞毌擾民。恆短衣執械先士卒,或宵行數十百里,偽為小商,詗虛實。村民通匪者憚其至,嘗置毒飲水處,寶勝則自攜水■L7,懷麥餅,食盡,忍饑渴以為常,以是寇鮮漏網。洛陽張黑子、嵩縣王天縱、汝州董萬川尤鷙悍,張、董並計擒之,天縱懼不敢出。豫西數十州縣皆安堵,而南陽王八老虎猶嵎負。寶勝至,移書期決鬥。會天大雪,前期五日,潛師薄其巢,賊不戒,據中庭轟拒。寶勝奮身入,眾繼之,火其廬,卒就縛,置之法。自是南陽無遺寇。寶勝短軀幹,目光炯炯能懾人。視盜如仇,待士卒若子弟。勞無吝賞,遇喪亡,賻恤尤厚。餉饋無所受,無兼衣餘食,統兵十餘年,而負債鉅萬。巡撫寶棻上聞,中旨敕司庫償九千餘金,異數也!

三年,移師嵩縣。值鄂亂作,亟還籌戰守。其時襄樊已應和,土寇處處飆起。豫南與陜、鄂壤地接,市言訛鷖日數至。檢勒部曲,日夕巡徼不少休。支振數十日,而襄樊軍闌入,士民與通款,將內訌。諸將意沮,咸莫能奮,惟都司姚靄雲慷慨原從戰。無何,新野陷,大吏飛檄戒毋妄動。寶勝憤激,赴校場,與眾誓死守,而府縣官已委印綬去。翌日元旦,獨朝服詣萬壽宮行禮,痛哭不能止。俄傳南軍入,煙焰翳天,各營亦以食盡而潰。不得已,退頓裕州,比至,城皆樹白幟矣,乃止舍。至夕而遜位詔至,召將卒勵以忠義,麾之去,夜半時,屏僕從,肅衣冠,嘔血數升,以槍自擊死。平旦,將卒趨視,皆哭失聲,以大纛裹尸,舁至獨頭鎮斂之。

靄雲,陜西人。舊為多隆阿部將,後從寶勝軍,隸營務處,亦為民軍所戕云。

黃忠浩,字澤生,湖南黔陽人。通經術,嗜讀儒先性理書。以優貢生入貲為內閣中書。主沅州講席,銳意地方利弊,建西路師範學堂,勸民植桑育蠶,尤顓志礦業。陳寶箴、趙爾巽先后撫湘,設礦局及公司,採平江金礦、常寧水口山鉛礦,至今稱厚利,皆其謀也。

光緒二十一年,以東事籌防,募鄉勇五百人入鄂,守田家鎮砲臺。總督張之洞一見重之,調領武靖營,駐洪山。二十三年,治軍長沙,統毅字軍,軍故征苗舊旅,日久窳敝,不可用。寶箴納其議,別募威字新軍,俾主之。二十六年,之洞檄募師勤王。二十八年,徙駐岳州,緝新堤土寇,平之。再入貲為道員。赴日本參觀大操,歸,益詳練戰術,知兵名大著。明年,爾巽檄綜湖南營務處,統忠字旗五營。其冬,母憂去職。

逾歲,廣西降匪陸亞發陷柳州,湘邊大震。起忠浩率所部援桂,直搗梅寨,用少擊眾,寇大創,降敕褒嘉。寇奔福祿村,村故瑤地,箐壑深岨,中有危塗垂線縷,容一人行。忠浩乃短衣芒蹻,徒步深入。會天酷暑,鬱為瘴癘,兵士死相繼,忠浩亦遘膨疾,然治軍勤如故,寇卒不敢近。捷上,授狼山鎮總兵,請終制,改署任道員授總戎,特例也。是時岑春煊駐桂林,檄與議軍事,奏署右江鎮。服闋,予實授。未幾,乞假去。再至湖北,爾巽留綜營務處,兼統全省防軍,荊襄水師受節度。

宣統二年,從爾巽入川,署提督,乞歸。三年,京師開全國教育會,忠浩與焉。爭鐵路國有為非計,議大濬洞庭湖,紓湘菑,議論侃侃無所撓。還長沙,值巡撫餘誠格新蒞官,黨人謀日亟。誠格慮新軍有異志,以中路巡防十營屬之,不就。誠格下席揖請至再,不獲已,始受事。甫三日,鄂亂起。九月朔,新軍變,將入城,協統蕭良臣遁,防軍為內應。忠浩方晨謁,隨誠格出,撫諭至再,勢洶洶不可遏,要誠格為都督。誠格從間道出,召水師,水師亦變。誠格投江,左右援之,不得死。忠浩猶留署,火起,護弁強之出,及門,遇亂兵,被執,脅降不從,劫之走,刃傷臂及股,至小吳門城樓,遂遇害。家人奉喪歸葬,緣塗設奠者數百里。繼忠浩死者有楊讓梨。

讓梨,字劭欽,籍湘鄉。少與王珍子詩正友善。詩正援臺灣,戰失利,嘗負之以免,軍中咸壯之。積勛至守備。轉戰新疆、河州、西寧,數有功,累擢參將,賜號鏗色巴圖魯。既,還長沙,隸忠浩麾下。宣統二年,補鎮筸鎮標中軍游擊。明年,武漢事起,忠浩電調援長沙。次辰州,聞省城亂,乃扼辰龍關,誓死守。筸兵故悍銳,為民軍所憚。時總兵周瑞龍持兩端,其子瓚齎金至,將以餌筸兵,哨弁李鳳鳴潛告讓梨,得為備。瑞龍稱疾,檄讓梨還,代以他將。讓梨乃上書責以大義滅親,辭激昂,且傳檄捕瓚,瓚遁。已而瑞龍降,道府官委印綬去。讓梨痛哭,犒遣軍士,獨棹小舟至清浪灘,踴身入水。舟子泅出之,讓梨恚甚,曰:「奚活我為?」瓚出代其軍,遣人追縶讓梨及其子傳孔,鎖送長沙。逕常德,遇龍璋巡按西路,勸之不屈,遂斬之。臨刑,肅衣冠北鄉拜,觀者萬餘人,皆泣下。傳孔釋還。

有陳萁者,讓梨從子壻也。當讓梨被縛時,萁即奪起擊縛者,僕一人,攢刃交下,傷其首,斷一足,並死之。

論曰:辛亥之變,各省新軍既先發難,防營不能獨支,而京外旗兵久無軍備,又多被殘困,死行陣者,自寥寥可數。志鈞等權輕勢孤,艱難搘柱,思以一隅挽全局;及事不可為,乃以死報,志節皎然,可敬亦可哀矣!


\end{pinyinscope}