\article{列傳一百七十九}

\begin{pinyinscope}
賽尚阿訥爾經額

賽尚阿,字鶴汀,阿魯特氏,蒙古正藍旗人。嘉慶二十一年繙譯舉人,授理籓院筆帖式,充軍機章京。宣宗命樞臣甄別所屬,賽尚阿列一等,予優敘。洊遷郎中。道光十一年,擢內閣侍讀學士,偕將軍富俊按吉林將軍福克精阿剋扣兵餉,得實,劾罷之。予頭等侍衛,充哈密辦事大臣,擢內閣學士。丁父憂回旗,留京,遷理籓院侍郎,兼副都統,調工部。迭赴盛京、廣東、察哈爾按事。十五年,命在軍機大臣上學習行走。調戶部,擢理籓院尚書,兼都統,調工部。

二十一年,海疆戒嚴,詔赴天津、山海關勘築砲臺,復偕御前大臣僧格林沁查閱海口。二十二年,命為欽差大臣,赴天津治防。和議成,撤防回京。初,京師添設槍隊,命賽尚阿偕左都御史恩桂司訓練。至是上閱武,槍隊獨整,嘉其督率有方,賜花翎。二十四年,命覆訊通州民婦康王氏勒斃親姑獄,白其冤,論坊官逼供罪如律。調戶部尚書,赴江南查閱江防善後事宜。三十年,兼步軍統領、協辦大學士。咸豐元年,拜文華殿大學士,管理戶部。

時廣西匪亂方熾,巡撫周天爵、提督向榮會剿,不能制賊,起用林則徐,未至,道卒。李星沅督師,諸將不用命,亦無功。文宗深憂之,以賽尚阿親信近臣,命為欽差大臣,赴湖南防堵,將以代星沅也,特賜遏必隆刀,給庫帑二百萬兩備軍餉。副都統巴清德、達洪阿率京軍隨行,姚瑩、嚴正基參軍事;又調湖南在籍知縣江忠源赴營。未幾,星沅卒於軍,趣賽尚阿馳往督師,授內大臣。六月,至廣西,疏陳汰兵勇,明紀律,購間諜,散脅從,斷接濟五事,詔嘉其能通籌全局。

周天爵與向榮不協,解其任,以鄒鳴鶴代之。又疏陳賊勢,略言:「粵西股匪繁多,馮雲山、洪秀全、凌十八等俱奉天主教,兇狠稱最,來往於金田、東鄉、廟旺、中坪,官兵壁上環觀,有無可如何之勢。宜先用全力攻剿大股,一經得手,則分兵剿辦,方免顧此失彼之虞。省垣兵少,暫居中調遣,分派巴清德、達洪阿進剿。」於是向榮連破賊於中坪及桂平新墟。烏蘭泰設伏,殲賊甚眾。賊竄踞紫荊山,以新墟、雙髻隘為門戶。達洪阿、烏蘭泰攻雙髻,毀其巢,賊自焚新墟而逸。官軍失利,遂陷永安州,賽尚阿坐失機,降四級留任。

詔責諸軍並力進攻,水竇為永安要隘,烏蘭泰攻拔之,乃合圍。向榮任北路,烏蘭泰任南路。永安城小而堅,環攻四閱月不能下,嚴詔趣戰。二年正月,賽尚阿親往督之,用向榮策,缺城北一隅不置兵,縱其出,因而擊之。烏蘭泰爭之不得,素與榮不協,至是益相水火。二月,賊果由此路突出,官軍不能御,僅獲洪大全,檻送京師,以收復永安上聞;而賊遽犯桂林,向榮走間道入城守御,烏蘭泰尾追至將軍橋,猝被砲傷,旋殞於軍,總兵長瑞、長壽、董光甲、邵鶴齡亦戰歿。賽尚阿自請治罪,詔責戴罪以圖補救,命兩廣總督徐廣縉率師赴援。

賊見桂林守具已完,援師漸集,解圍北竄,連陷興安、全州。賽尚阿始入駐省城,遣提督餘萬清、總兵劉長清進攻全州。江忠源破賊於蓑衣渡,斃悍賊馮雲山。賊遂入湖南,連陷道州、江華、永明、嘉禾、藍山、桂陽,賽尚阿尾之,抵衡陽。賊由郴州分竄醴陵、攸縣,尋犯長沙,勢益鴟張。湖南巡撫羅繞典以聞,文宗震怒,詔斥賽尚阿調度無方,號令不明,賞罰失當,以致勞師糜餉,日久無功,褫職逮京治罪。命大學士等會鞫,賽尚阿伏地流涕,自言不忍殺人辜負聖恩,論大闢,籍其家,三子並褫職。未幾,釋出獄,發往直隸,交訥爾經額差遣,調京隨辦巡防。五年,遣戍軍臺,尋釋之,命練察哈爾蒙古兵。十年,回京,總統左翼巡城事宜,予侍郎銜,授正紅旗蒙古副都統。以病免。光緒元年,卒。子崇綺,自有傳。

訥爾經額,字近堂,費莫氏,滿洲正白旗人。嘉慶八年繙譯進士,授妃園寢禮部主事,調工部,洊升郎中。道光元年,出為山東兗沂曹道,遷湖南按察使,丁憂去職。三年,起署山東按察使,尋實授。承鞫教匪馬進忠獄得實,賜花翎,就遷布政使。六年,擢漕運總督。九年,調山東巡撫。十二年,擢湖廣總督。十六年,湖南新寧瑤生藍正樽習教傳徒,聚眾數千,攻武岡州城,為官兵擊退。捕獲黨羽,而正樽逃逸,詔責訥爾經額嚴緝,久不獲,革職留任。十七年,京察考績,詔斥訥爾經額玩洩無能,降湖南巡撫,限一年捕正樽。尋以正樽已被鄉勇毆斃,奏下繼任總督林則徐確查虛實,則徐疏言鄉勇毆斃三賊,有正樽在內,以衣物為證,詔斥衣物出於事後呈驗,不足信,褫訥爾經額職,予三等侍衛,充駐藏辦事大臣。逾年,晉頭等侍衛,調西寧辦事大臣。二十年,擢熱河都統。俄授陜甘總督,未之任,命署直隸總督,尋實授。

二十一年,英吉利兵船游弋秦王島,命訥爾經額移駐天津籌防,加太子太保。時漸多事,財政支絀,疆臣猶因襲承平舊制,憚於興革。廷議興屯墾及畿輔水利,訥爾經額疏言:「屯田不能行於畿輔,先朝試行水利,屢興屢廢。良由南北異宜,民多未便。」寢其議。又言官請長蘆懸岸鹽額如河南、山東,改歸官辦。訥爾經額言:「懸岸由於私充引滯,但使梟販斂跡,民販亦可持久,諸商不招自至。不必務官辦之虛名,徒事更張,無裨實用。」咸豐二年,以直隸總督協辦大學士,尋拜文淵閣大學士,仍留總督任。

三年,粵匪既踞江寧,分黨由安徽入河南,歸德、睢州、寧陵、蘭封相繼陷,河南巡撫陸應穀敗績。賊窺開封,命訥爾經額防守大名,遏賊北竄。令總兵花裏雅遜布屯延津防河,雙祿守彰德為後繼,而賊酋林鳳祥、李開芳已自汜水渡河,陷溫縣,犯懷慶。訥爾經額檄總兵董占元赴援,自駐臨洺關,請增調盛京、吉林步騎。詔授訥爾經額為欽差大臣,節制河南、北諸軍。賊圍懷慶久,知府餘炳燾率紳民固守,賊周樹木柵為久困計。援軍四集,惟都統勝保、將軍托明阿軍戰最力,花裏雅遜布、董占元等隔丹水駐軍,畏賊不敢進。勝保屢以為言,詔促訥爾經額進師夾擊,並防賊竄入山西,乃進駐清化鎮。八月,諸軍五路合擊,破賊柵,賊大潰,圍乃解。文宗大悅,賜訥爾經額雙眼花翎、黃馬褂,賚擢諸將有差。

賊之敗竄也,諸軍以久戰疲罷,未能力追;山西兵多調援,設防不密。賊遂由濟源入太行山,連陷垣曲、陽城、曲沃,犯平陽府,擾及洪洞,並失守。追軍皆落後,惟勝保先進,戰於平陽,挫之。繞前扼賊北路,賊乃東趨。訥爾經額回駐臨洺關,素不知兵,束手無措。或告潞城、黎城間有孔道,循太行東出武安,密邇臨洺,然險隘可扼。訥爾經額以非直隸轄境,咨山西巡撫守御。既而賊果破黎、潞,猶謂賊不能遽至。忽有冒欽差大臣旗幟責州縣供張者,蓋賊之前驅已出山矣。俄而麕至,官軍出不意,驚潰,訥爾經額以數十人走保廣平府城,關防、令箭、軍書、資械委棄皆盡。事聞,褫職,留於直隸隨同辦理軍務。賊遂大熾,畿輔半被蹂躪,京師震動。命惠親王綿愉為大將軍,科爾沁郡王僧格林沁副之,勝保督師前敵追剿。於是逮訥爾經額下獄,論斬監候。逾年始殄賊,先後擒首逆林鳳祥、李開芳伏誅,畿輔肅清。赦訥爾經額出獄,遣戍軍臺。逾年釋回,予六品頂戴,命守慕陵。尋以四五品京堂候補。七年,卒。子蘊秀、衍秀,並官內閣學士。

論曰:清沿故事,有大軍事,輒以滿洲重臣督師。乾、嘉時,如阿桂、福康安、勒保、額勒登保等,皆胸有韜略,功在旂常。道光以來,惟長齡平定回疆,差堪繼武。其後禧恩之徵瑤,奕山、奕經之防海,或以驕侈召謗,或以輕率僨事。至粵匪初起,李星沅不勝任,易以賽尚阿,馭將無方,遂致寇不可制。訥爾經額庸懦同之,畿甸震驚,自是朝廷始知其弊。惟僧格林沁猶以勛望膺其任,不復輕以中樞閣部出任師干,即有時親籓遙領,亦居其名不行其實。蓋人材時會使然,固不可與國初入關時並論也。


\end{pinyinscope}