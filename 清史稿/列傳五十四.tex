\article{列傳五十四}

\begin{pinyinscope}
張玉書李天馥吳琠張英子廷瓚廷璐廷彖陳廷敬

溫達穆和倫蕭永藻嵩祝王頊齡

張玉書,字素存,江南丹徒人。父九徵,字湘曉。順治二年,舉鄉試第一。九年,成進士。博學礪行,精春秋三傳,尤邃於史。歷吏部文選郎中。出為河南提學僉事,考績最,當超擢,遽引疾歸。

玉書,順治十八年進士,選庶吉士,授編修。累遷左庶子,充日講起居注官。康熙十九年,以進講稱旨,加詹事銜。二十年,擢內閣學士,充經筵講官。尋遷禮部侍郎,兼翰林院掌院學士。三籓平,有請行封禪者,玉書建議駁之,事遂寢。二十三年,丁父憂,上遣內閣學士王鴻緒至邸賜奠。服闋,即家起刑部尚書,調兵部。

二十七年,河道總督靳輔奏中河工成。時學士開音布往勘稱善,監高郵石工,疏請閉塞支河口為中河蓄水。上以於成龍嘗奏輔開中河無益累民,今中河工成,乃命玉書偕尚書圖納等往勘,並遍察毛城鋪、高家堰及海口狀。瀕行,上謂玉書曰:「此行當秉公陳奏,毋效熊一瀟託故推諉為也。」玉書等還奏:「勘閱河形,黃河西岸出水高。年來水大,未溢出岸上,知河身並未淤塞。海口岸寬二三里,河流入海無所阻。中河工成,舟楫往來,免涉黃河一百八十里之險。但與黃河逼,河寬固不可,狹又不能容運河及駱馬湖之水。擬請於蕭家渡、楊家莊增建減水壩,相時宣洩。閉塞支河口,應如開音布議。」上悉從之。

浙江巡撫金鋐以民杜光遇陳訴駐防滿洲兵擾民,下布政使李之粹察訊。之粹咨杭州將軍郭丕請申禁,郭丕以聞。上遣尚書熊賜履往按,賜履丁憂去,改命玉書。尋調禮部。二十八年,上南巡,駐蹕蘇州,玉書還奏杜光遇無其人,所陳訴皆虛妄。金鋐、李之粹皆坐奪官,流徙。二十九年,拜文華殿大學士,兼戶部尚書。

三十一年,靳輔奏高家堰加築小堤,復命玉書偕圖納往勘。還言:「曩者黃漲,淮流被逼,故洪澤湖水視昔為高。今擬築堤,距高家堰甚近;若水漲,則高家堰大堤且不保,築小堤何益?因條列高家堰河工,自史家刮至周橋一萬四百餘丈,宜築堤三官廟。諸口宜改石工。今擬築小堤處,宜令河臣每歲親勘。」上深然之。

三十五年,上親征噶爾丹,玉書扈行,預參帷幄。師次克魯倫河,噶爾丹北竄,大將軍費揚古截擊,斬殺幾盡,噶爾丹僅以身免。玉書率百官上賀。三十六年,充平定朔漠方略總裁官。丁母憂,遣官賜祭,並賜御書松廕堂榜。三十八年,上南巡,玉書迎謁,賜賚有加。三十九年,服未闋,召至京,入閣視事。四十年,扈駕南巡,駐蹕江寧,召試士子,命為閱卷官。御舟次高資港,玉書奏言前去鎮江不遠,請幸江天寺,留駐數日,上為留一日。

四十六年,河道總督張鵬翮請開溜淮套河,上南巡,次清口勘視,見所樹標竿多在民塚,召鵬翮極斥其非。玉書奏曰:「向者老人白英議引汶水南北分流,不若別作壩引汶水通漕,其下流專以淮水敵黃。黃水趨海,此萬世利也。」上善其言,遂諭鵬翮罷開溜淮套,事具鵬翮傳。

四十九年,以疾乞休,溫旨慰留。五十年,從幸熱河,甫至疾作,遂卒,年七十,上深惜之,親制輓詩,賜白金千。命內務府監制棺槨衾絞,驛送其喪還京師,加贈太子太保,謚文貞。五十二年,上追念舊勞,擢其子編修逸少為侍讀學士。

玉書謹慎廉潔,居政地二十年,遠避權勢,門無雜賓,從容密勿,為聖祖所親任。自奉儉約,飲食服御,略如寒素。雍正中,入祀賢良祠。

李天馥,字湘北,河南永城人。先世在明初以軍功得世襲廬州衛指揮僉事,家合肥。有族子占永城衛籍,天馥以其籍舉鄉試。順治十五年,成進士,選庶吉士,授檢討。博聞約取,究心經世之學,名藉甚。累擢內閣學士,充經筵講官。每侍直,有所見,悉陳無隱,聖祖器之,康熙十九年夏,旱,命偕大學士明珠會三法司慮囚,有矜疑者,悉從末減。尋擢戶部侍郎,調吏部。杜絕苞苴,嚴峻一無所私,銓政稱平。二十七年,遷工部尚書。河道總督靳輔議築高家堰重堤,束水出清口,停濬海口;於成龍主疏濬下河。上召二人詣京師入對,仍各持一說,下廷臣詳議,天馥謂下河海口當濬,高家堰重堤宜停築,上然之。歷刑、兵、吏諸部。

三十一年,拜武英殿大學士。上曰:「機務重任,不可用喜事人。天馥老成清慎,學行俱優,朕知其必不生事。」三十二年,以母憂回籍,上賜「貞松」榜御書,勉以儒者之學;復謂:「天馥侍朕三十餘年,未嘗有失。三年易過,命懸缺以待。」三十四年,服將闋,起故官,入閣視事。上親征厄魯特,平定朔漠,兵革甫息,天馥務以清靜和平,與民休息。嘗謂:「變法不如守法。奉行成憲,不失尺寸,乃所以報也。」三十八年,卒,謚文定。

天馥在位,留意人才,嘗應詔舉彭鵬、陸隴其、邵嗣堯,卒為名臣。為學士時,冬月慮囚,有知縣李方廣坐當死,天馥言其有才,得緩決,尋以赦免。刑部囚多瘐斃,為庀屋材,多為之所,別罪之輕重以居,活者尤眾。事親孝,居喪廬墓,有雙白燕飛至,不去,人名其居為白燕廬。子孚青,進士,官編修。父喪歸,不復出。

吳琠,字伯美,山西沁州人。順治十六年進士,授河南確山知縣。縣遭明季流寇殘破,琠拊流亡,闢蕪廢,墾田歲增,捕獲盜魁誅之。師下雲南,縣當孔道,輿馬糧餉,先事籌辦而民不擾。康熙十三年,以卓異入為吏部主事,歷郎中。累遷通政司右參議。刑部尚書魏象樞亟稱其賢。二十年,特擢右通政,累遷左副都御史。疏請復督撫巡方,略言:「令甲,督撫於命下之日,即杜門屏客;蒞任,守令不得參謁。凡有舉劾,惟據道府揭報,愛憎毀譽,真偽相亂,督撫無由知。革火耗而火耗愈甚,禁私派而私派愈增。請敕督撫親歷各屬,以知守令賢否。或謂巡方恐勞擾百姓,夫督撫賢,則必能禁迎送、卻供應;如其不肖,雖端坐會城,而暮夜之餽踵至,豈獨巡方足以勞民哉?」又言:「巡撫及巡守道無一旅之衛,而提鎮各建高牙。前撫臣如馬雄鎮,道臣如陳啟泰,懷忠秉義,向使各有兵馬,奚至束手?宜及此時復舊制,使巡撫、巡守道仍各管兵馬。減提督,增總兵,以一鎮之兵酌分數鎮,聽督撫節制。」

二十八年,遷兵部侍郎,尋授湖廣巡撫。湖北自裁兵亂後,奸猾率指仇人為亂黨,株連不已,琠悉置不問,而懲其妄訐者,人心大定。陜西饑,流民入湖廣就食,令有司分賑,全活甚眾。三十一年,詔以荊州兵船運漕米十萬石至襄陽備賑,琠議:「兵船泊大江下至漢口受米,復西上抵襄陽,計程二千餘里。令原運漕船若乘夏水順道赴襄陽,僅七百餘里,即以便宜行事。」疏入,上嘉之。未幾,丁母憂,服未闋,即授湖廣總督,仍聽終制乃赴任。故事,土司見州縣吏不敢抗禮,其後大吏稍稍假借之。琠至,絕餽遺,飭謁見長吏悉循舊制,或犯約束,檄諭之,無敢肆者。

三十五年,召為左都御史。三十六年,典會試。上北征回鑾,顧迎駕諸臣,褒琠及河道總督張鵬翮居官之廉,即擢琠為刑部尚書,而以鵬翮為左都御史。三十七年,拜保和殿大學士,兼刑部。琠熟諳舊章,參決庶務,靡不允當。奏對皆竭忱悃,上每稱善。所薦引多賢能吏。

三十九年,復典會試,上手書「風度端凝」榜賜之。尋具疏乞休,不允。上嘗臨米芾書以賜琠,書其後曰:「吳琠寬厚和平,持己清廉。先任封疆,軍民受其實惠。朝中之事,面折廷諍,能得其正。朕甚重其能得大臣之體。」四十四年,卒,謚文端。翰林院撰祭文,上以為未能盡琠,敕改撰。吏部奏大學士缺員,上以琠喪未歸,懸缺未即別除,曰:「朕心不忍也。」

琠所至多惠政,兩湖及確山皆祠祀。初,沁州薦饑,琠糴米賑之,全活無算。有司議增沁糧一千三百石,琠力爭乃已。鄉人德之,立祠以祀。雍正中,祀賢良祠。

張英,字敦復,江南桐城人。康熙六年進士,選庶吉士。父憂歸,服闋,授編修,充日講起居注官。累遷侍讀學士。十六年,聖祖命擇詞臣諄謹有學者日侍左右,設南書房。命英入直,賜第西安門內。詞臣賜居禁城自此始。時方討三籓,軍書旁午,上日禦乾清門聽政後,即幸懋勤殿,與儒臣講論經義。英率辰入暮出,退或復宣召,輟食趨宮門,慎密恪勤,上益器之。幸南苑及巡行四方,必以英從。一時制誥,多出其手。

遷翰林院學士,兼禮部侍郎。二十年,以葬父乞假,優詔允之,賜白金五百、表裏緞二十,予其父秉彞恤典視英官。英歸,築室龍眠山中,居四年,起故官。遷兵部侍郎,調禮部,兼管詹事府。充經筵講官,奏進孝經衍義,命刊布。二十八年,擢工部尚書,兼翰林院掌院學士,仍管詹事府。調禮部,兼官如故。編修楊瑄撰都統、一等公佟國綱祭文失辭,坐奪官流徙;斥英不詳審,罷尚書,仍管翰林院、詹事府,教習庶吉士。尋復官,充國史、一統志、淵鑒類函、政治典訓、平定朔漠方略總裁官。三十六年,典會試。尋以疾乞休,不允。三十八年,拜文華殿大學士,兼禮部。

英性和易,不務表襮,有所薦舉,終不使其人知。所居無赫赫名。在講筵,民生利病,四方水旱,知無不言。聖祖嘗語執政:「張英始終敬慎,有古大臣風。」四十年,以衰病求罷,詔許致仕。瀕行,賜宴暢春園,敕部馳驛如制。四十四年,上南巡,英迎駕淮安,賜御書榜額、白金千。隨至江寧,上將旋蹕,以英懇奏,允留一日。時總督阿山欲加錢糧耗銀供南巡費,江寧知府陳鵬年持不可,阿山怒鵬年,欲因是罪之,供張故不辦;左右又中以蜚語,禍將不測。及英入見,上問江南廉吏,首舉鵬年,阿山意為沮,鵬年以是受知於上為名臣。四十六年,上復南巡,英迎駕清江浦,仍隨至江寧,賜賚有加。

英自壯歲即有田園之思,致政後,優游林下者七年。為聰訓齋語、恆產瑣言,以務本力田、隨分知足誥誡子弟。四十七年,卒,謚文端。世宗讀書乾清宮,英嘗侍講經書,及即位,追念舊學,贈太子太傅,賜御書榜額揭諸祠宇。雍正八年,入祀賢良祠。高宗立,加贈太傅。

子廷瓚,字卣臣。康熙十八年進士,自編修累官少詹事。先英卒。廷玉,自有傳。

廷璐,字寶臣。康熙五十七年,殿試一甲第二名進士,授編修,直南書房,遷侍講學士。雍正元年,督學河南,坐事奪職。尋起侍講,遷詹事。兩督江蘇學政。武進劉綸、長洲沈德潛皆出其門,並致通顯,有名於時。進禮部侍郎,予告歸,卒。

廷彖,字桓臣。雍正元年進士,自編修累官工部侍郎,充日講官。起居注初無條例,廷彖編載詳贍得體。既擢侍郎,兼職如故。終清世,已出翰林而仍職記注者惟廷彖。乾隆九年,改補內閣學士,兼禮部侍郎。典試江西,移疾歸。廷彖性誠篤,細微必慎。既歸,刻苦礪行,耿介不妄取。三十九年,卒,年八十四。上聞,顧左右曰:「張廷彖兄弟皆舊臣賢者,今盡矣!安可得也?」因嘆息久之。

廷璐子若需,進士,官侍講。若需子曾敞,進士,官少詹事。

自英後,以科第世其家,四世皆為講官。

陳廷敬,初名敬,字子端,山西澤州人。順治十五年進士,選庶吉士。是科館選,又有順天通州陳敬,上為加「廷」字以別之。十八年,充會試同考官,尋授秘書院檢討。康熙元年,假歸,四年,補原官。累遷翰林院侍講學士,充日講起居注官。十四年,擢內閣學士,兼禮部侍郎,充經筵講官,改翰林院掌院學士,教習庶吉士。與學士張英日直弘德殿,聖祖器之,與英及掌院學士喇沙裏同賜貂皮五十、表裏緞各二。十七年,命直南書房。丁母憂,遣官慰問,賜茶酒。服除,起故官。二十一年,典會試。滇南平,更定朝會燕饗樂章,命廷敬撰擬,下所司肄習。遷禮部侍郎。

二十三年,調吏部,兼管戶部錢法。疏言:「自古鑄錢時輕時重,未有數十年而不改者。向日銀一兩易錢千,今僅得九百,其故在毀錢鬻銅。順治十年因錢賤壅滯,改舊重一錢者為一錢二分五釐,十七年又增為一錢四分,所以杜私鑄也。今私鑄自如,應改重為輕,則毀錢不禁自絕。產銅之地,宜停收稅,聽民開採,則銅日多,錢價益平。」疏下部議行。

擢左都御史。疏言:「古者衣冠、輿馬、服飾、器用,賤不得逾貴,小不得加大。今等威未辨,奢侈未除,機絲所織,花草蟲魚,時新時異,轉相慕效。由是富者黷貨無已,貧者恥其不如,冒利觸禁,其始由於不儉,其繼至於不廉。請敕廷臣嚴申定制,以挽頹風。」又言:「方今要務,首在督撫得人。為督撫者,不以利欲動其心,然後能正身以董吏。吏不以曲事上官為心,然後能加意於民;民可徐得其養,養立而後教行。宜飭督撫凡保薦州縣吏,必具列無加派火耗、無黷貨詞訟、無朘削富民。每月吉集眾講解聖諭,使知功令之重在此。而皇上考察督撫,則以潔己教吏,吏得一心養民教民為稱職,庶幾大法而小廉。」又言:「水旱兇荒,堯、湯之世所不能盡無,惟備及於豫而周當其急,故民恃以無恐。山東去年題報水災,戶部初議行令履勘,繼又行令分晰地畝高下,今年四月始行覆準蠲免。如此其遲回者,所行之例則然耳。臣愚以為被災分數既有冊結可據,即宜具覆豁免,上宣聖主勤民之意,下慰小民望澤之心,中不使吏胥緣為弊竇。」疏並議行。

二十五年,遷工部尚書。與學士徐乾學奏進鑒古輯覽,上嘉其有裨治化,命留覽。時修輯三朝聖訓、政治典訓、方略、一統志、明史,廷敬並充總裁官。累調戶、吏二部。二十七年,法司逮問湖廣巡撫張汧,汧曾齎銀赴京行賄。獄急,語涉廷敬及尚書徐乾學、詹事高士奇,上置勿問。廷敬乃以父老,疏乞歸養,詔許解任,仍管修書事。

二十九年,起左都御史,遷工部尚書,調刑部。丁父憂,服闋,授戶部尚書,調吏部。四十二年,拜文淵閣大學士,兼吏部,仍直經筵。四十四年,扈從南巡,召試士子,命閱卷。四十九年,以疾乞休,允之。會大學士張玉書卒,李光地病在告,召廷敬仍入閣視事。五十一年,卒,上深惜之,親制輓詩一章,命皇三子允祉奠茶酒;又命部院大臣會其喪,賜白金千,謚文貞。

廷敬初以賜石榴子詩受知聖祖,後進所著詩集,上稱其清雅醇厚,賜詩題卷端。嘗召見問朝臣誰能詩者,以王士禎對,又舉汪琬應博學鴻儒,並以文學有名於時。上御門召九卿舉廉吏,諸臣各有所舉,語未竟,上特問廷敬,廷敬奏:「知縣陸隴其、邵嗣堯皆清官,雖治狀不同,其廉則一也。」乃皆擢御史。始廷敬嘗亟稱兩人,或謂曰:「兩人廉而剛,剛易折,且多怨,恐及公。」廷敬曰:「果賢歟,雖折且怨,庸何傷?」

溫達,費莫氏,滿洲鑲黃旗人。自筆帖式授都察院都事,遷戶部員外郎。康熙十九年,授陜西道御史。遷吏科給事中,兼管佐領。授兵部督捕理事官。三十五年,上親征噶爾丹,命溫達隨皇七子允祜、都統都爾瑪管鑲黃旗大營。三十六年,擢內閣學士。三十八年,遷戶部侍郎。四十年,命赴山西、陜西察驗驛馬,還,授議政大臣。雲貴總督巴錫劾游擊高鑒讞獄不當,並論提督李芳述徇隱,芳述亦劾巴錫,命溫達往按,鑒罪應徒,巴錫左遷,芳述罰俸。四十一年,調吏部,擢左都御史。四十二年,復命往貴州按威寧總兵孟大志侵餉,論罪如律。四十三年,遷工部尚書,充經筵講官。四十六年,授文華殿大學士,纂修國史、政治典訓、平定朔漠方略、大清一統志、明史,並充總裁。五十年,命八旗及部院舉孝義,因諭曰:「孝為百行首。如大學士溫達,尚書穆和倫、富寧安之孝,不特眾所知,朕亦深知之也。」御制詩以賜,復褒其孝友。五十三年,以老乞休,許致仕。尋諭溫達雖老,尚自康健,命仍任大學士。五十四年,卒,命皇子奠茶酒,賜祭葬,謚文簡。

穆和倫,喜塔臘氏,滿洲鑲藍旗人。自兵部筆帖式四遷為御史,又三遷為內閣學士。命往山東察賑,自泰安至郯城。康熙四十三年,遷工部侍郎。四十八年,授禮部尚書。四十九年,調戶部。上稱穆和倫孝,其母年已九十,御書「北堂眉壽」榜賜之。兩江總督噶禮與巡撫張伯行互劾,命穆和倫往按,右噶禮,上責其是非顛倒,終直伯行。尋以老病乞休,復起授戶部尚書。坐事當左遷。尋卒。

蕭永藻,漢軍鑲白旗人。父養元,管佐領。永藻自廕生補刑部筆帖式。康熙十六年,授內閣中書,遷禮部員外郎,襲佐領。遷郎中,監湖口稅務。授御史,再遷順天府尹。三十五年,擢廣東巡撫。疏言:「錢多價賤,每千市價三錢二三分,兵領一兩之餉,不及數錢之用。民亦因錢賤,貨物難行。請暫停鼓鑄。」又疏言:「開山發礦,多人群聚,良莠淆雜,臣通飭嚴禁。近有長寧匪徒集眾私採,知縣尤鵬翔請飭部議處。」鵬翔坐奪官。

三十九年,給事中湯右曾劾永藻與總督石琳於黎人爭鬥事,遲至一載始行具題;縱屬吏朘民,民困而為盜,海則電白、陽江,山則英德、翁源,橫行劫掠。上命與廣西巡撫彭鵬互調,入覲,上諭當命效鵬所行,並誡薦舉當擇清廉。四十五年,遷兵部侍郎。湖廣總督石文晟劾容美土司田舜年不法,命左都御史梅鋗、內閣學士二格往讞,與文晟異議;復命永藻與大學士席哈納、侍郎張廷樞覆讞,還奏舜年已死,無諸僭越狀。

四十六年,擢左都御史,遷兵部尚書。四十八年,湖南巡撫趙申喬與提督俞益謨交惡互劾,命永藻偕副都御史王度昭往按,得益謨違例缺兵額狀,申喬事事苛求,非大臣體,並擬奪官,上罷益謨,留申喬。四十九年,調吏部,旋授文華殿大學士。五十六年,列議政大臣。

六十一年,世宗即位,加太子太傅,命駐馬蘭峪守護景陵。雍正五年,宗人府奏護陵宗室廣善越分請安,永藻不先阻,當奪官,上責永藻自恃其有操守,驕矜偏執,惟知阿諛允,長其傲慢狂肆之罪,如議奪官,仍命護陵自效。七年,卒,年八十六。

嵩祝,赫舍里氏,滿洲鑲白旗人。父岱袞,事太宗,協管佐領。兄來袞,自侍衛累遷至內三院學士,授世職拜他喇布勒哈番。嵩祝襲職,康熙九年,管佐領。二十三年,遷護軍參領。三十三年,擢內閣學士。

三十四年,盛京旱,命與侍郎珠都納偕往,發海運米萬石散貧民,萬石平糶。還京,命復偕珠都納往開原等散米。上諭曰:「將軍等請散米,但言兵不言民。此皆朕赤子,當一並給與,月與米一斗五升,至來歲四月。」嵩祝等散米如上指,事畢還京師。

三十五年,上親征噶爾丹,嵩祝管正黃旗行營。師還,命統後隊緩行,待西路章奏。遷兵部侍郎,改護軍統領。三十六年,復扈上出塞駐寧夏,命昭武將軍喀斯喀等窮追噶爾丹,嵩祝參贊軍務。噶爾丹竄死,師至摩該圖,引還。

四十年,遷正黃旗漢軍都統。廣東官兵剿連州瑤失利,命嵩祝偕副都統達爾占、侍郎傅繼祖往會總督石琳,調廣西、湖南兵進剿,即授廣州將軍。瀕行,上諭以相機招撫。四十一年,師次連州,檄三省官兵分布要隘。瑤人薙發請降,執戕官兵者九人誅之。師引還,調正紅旗。

四十八年,署奉天將軍。海盜舟泊雙島,挾火器出掠,遣兵擊殺三十餘人,得其舟一。疏請山東水師兼巡奉天屬金州鐵山,又請選盛京滿洲兵千人習鳥槍,設火器營,皆從之。遷禮部尚書。

五十一年,授文華殿大學士。五十五年,上幸熱河,嵩祝從。久不雨,上憂旱甚,遣嵩祝還京師,察諸大臣祈雨不躬至者劾奏。六十一年,世宗即位,加太子太傅,修聖祖實錄及玉牒,並充總裁。雍正五年,奉天將軍噶爾弼奏貝子蘇努為將軍時,借放庫銀三萬餘,嵩祝坐徇隱,奪官。十三年,卒,年七十九。

王頊齡,字顓士,江南華亭人。父廣心,字農山。有文名。順治六年進士,官御史,巡視京、通二倉,釐剔漕弊,奸猾屏跡。

頊齡,康熙十五年進士,授太常寺博士。十八年,舉博學鴻儒,召試一等,授編修,纂修明史,充日講起居注官。二十一年上元節,聖祖御乾清宮賜廷臣宴,仿柏梁體賦詩,頊齡與焉。遷侍講,督四川學政。累遷侍講學士。二十八年,左都御史郭琇疏劾少詹事高士奇與頊齡弟鴻緒植黨營私,並詆頊齡與士奇結婚媾,交關為奸利。頊齡、士奇、鴻緒並休致,尋命頊齡留任如故。轉侍讀學士,以父憂歸,服闋,起故官。累擢禮部侍郎。四十三年,上南巡,幸頊齡所居秀甲園,賜御書榜。四十六年,上南巡閱河,再幸其第。尋調吏部,充經筵講官。擢工部尚書,典會試。五十五年,拜武英殿大學士。

雍正元年,詔開恩榜,頊齡重與鹿鳴宴,加太子太傅。以老,累疏乞休,上以頊齡先朝舊臣,勤勞歲久,諳習典章,輒與慰留。三年,痰作,命御醫治疾,賜參餌。尋卒,年八十四,上為輟朝一日,令朝臣出其門下者素服持喪、各部院漢官會祭,贈太傅,謚文恭。

弟九齡,字子武,進士,授編修,官至左都御史;鴻緒,自有傳。

論曰:玉書等遭際承平,致位宰相。或以文學進,或以功能著,或以節操用,皆循循乎矩度。即朝旨所褒許,於玉書則曰「小心」,於天馥則曰「勤慎」,英曰「忠純」,琠曰「寬厚」,廷敬曰「清勤」,溫達「孝」,永藻「廉」,嵩祝「老成」,頊齡「安靜」。諸臣之行詣顯,世運之敦龐亦可見矣。


\end{pinyinscope}