\article{志六十五}

\begin{pinyinscope}
禮九(軍禮)

親征凱旋命將出征奏凱受降獻俘受俘大閱會閱暨京師訓練附

秋獮日食救護

三曰軍禮。國之大事,在祀與戎。周官制六軍,司九伐,權屬司馬。而大軍旅、大田役,其禮則宗伯掌之。是因治兵、振旅、茇舍、大閱之教,而寓蒐、苗、獮、狩之儀,以為社、礿、祊、烝之祭。如是,則講武為有名,而殺獸為有禮。有清武功燀赫,凡師征、受成、講肄、行圍諸禮節,厥制綦備。爰■A8古誼,分錄事要,著之於篇。古者日食救護,太僕贊鼓,亦屬夏官,今亦類附云。

親征天命三年,太祖頒訓練兵法書,躬統步騎征明,謁堂子,書七恨告天,是親征所由始。

崇德初元,太宗伐朝鮮,前期誓天、告廟,頒行軍律令,分兵為左右翼。至日,駕出撫近門,陳鹵簿,吹螺奏樂。祗謁堂子,三跪九拜。外建八纛,致祭如初。禮畢啟行。

康熙三十五年,討噶爾丹,躬率六師出中道。前三日,祭告郊、廟、太歲,屆期遣祭道路、砲、火諸神。帝禦征衣佩刀,乘騎出宮,內大臣等翊衛。午門鳴鐘鼓,軍士鳴角螺,祭堂子、纛神如儀。導迎樂作,奏祐平章。駕出都門,詣陳兵所,聲砲二。旗軍繼發,王公百官𧾷忌送。軍士整伍,以次扈蹕。每舍周視地勢,御營建正中,各營環向,繚以幔城,南設旌門。遠斥堠,嚴刁斗。置巡警二十一所,內大臣等率親軍宿衛。外設網城,東、西、南三門。巡警八所,護軍統領率羽林軍徼循。禁語譁,稽出入。又外布幕為重營,設四門,重各置十人嚴守。其從征各官,列幕重營外。大軍分翼牧馬,禁越次。駕駐行營,諸軍皆止。從官奏事如常。夜漏初下,嚴更鼓,斷行人,內外禁旅番巡。五漏交,御營鳴鐘,前營角聲起。初嚴,外營蓐食治裝;再嚴,前軍拔營;三嚴,左右軍、後軍發輜重,從征官俟旌門外。辨色,舉砲警蹕。六師所過,守土官迎本境,大吏則出境以迎,外籓王公暨所部紳耆跪接,悉同時巡儀。軍行,隨時遣祭風、雨、山、川諸神,軍中堠望。聖祖躬巡,整軍伍,御旌門,簡閱將士,至西巴臺,使者奉敕諭噶爾丹。敵望見大軍,棄甲走,帝率前軍長驅拖諾,分遣將軍進躡,乃還。

噶爾丹未悛,是歲秋,駕巡北邊,聲出塞試鷹,減從。十月,抵白塔,駐南關,蒙古王以下貢獻駱驛。帝賜戰勝兵士食,引近御坐遍賚之。次日,益徹御膳犒軍。逾月,至呼坦和碩,渡河,降者踵至。噶爾丹就撫,乃班師。明年,帝三駕北征,啟行如初禮,至橫城止。令守土大臣臨河迎蹕。時哈密俘噶爾丹子送軍所,額魯特部多納款者,噶爾丹仰藥死,駕自黃河汎舟還。

凱旋崇德二年,太宗征服朝鮮。班師日,其君臣出城十里外送駕,三跪九拜如禮。歸則遣大臣二人送之。啟蹕,即軍前祭纛。守土官道迎,俟駕過,隨軍次承命,遙坐賜酒。將至盛京二十里,會鄭親王等齎奉賀表,遂先除道,張黃幄,俟駕至,伏迎道左。帝入幄坐,王等跪進表,大學士受之。宣讀畢,王等三跪九拜,乃大宴,宴罷啟行。至盛京,禮謁堂子,還宮。

康熙三十五年,聖祖征噶爾丹,破之,還蹕拖諾,捷入,焚香謝天。入行營,大學士等進賀表,王公百官畢賀。留牧蒙古王等迎駕行禮,喀爾喀札薩克等集營東門請瞻覲,皆稽首呼萬歲。賜茶及宴,賚銀物有差。沿途迎獻羅拜者,繦至輻湊。至清河,皇子、王公暨群臣跪迎郊外五里,八旗軍校、近畿士民亦焚香懸採,扶攜俯伏。命前驅毋警蹕,環集至數百萬人,歡聲雷動。帝謁堂子如儀。

明年,朔漠平,班師亦如之。還宮後,遣祭郊、社、宗廟,遍群神,謁陵寢,御殿受賀。直省官咸進表文,頒詔如制。帝自勒銘鑱石,並建碑太學云。

命將出征崇德初元,太宗命睿王多爾袞等出師征明,躬自臨送,祭堂子、纛神,如親征儀。遂至演武場,諭誡將士。順治元年,命英王阿濟格為靖遠大將軍,徵流寇,賜敕印。其儀,午門外具鹵簿,陛上張黃幄,設御座。陳敕印簷東案,王公百官會集。帝升座,大將軍率出征官詣拜位跪,內院大臣奉宣滿、蒙、漢三體敕書,授大將軍敕印,畢,啟行。

十三年,定出師前一日,午門前例頒衣馬弓刀,並傳集出征各官,面授方略。賜筵宴。行日,咸戎服俟午門外,頒敕印如初禮。

康熙十三年,命將分出湖廣、四川。禮畢,駕出長安右門送行。出征王率各官行至陳兵所,禮部設祖帳,光祿寺備茶酒,內大臣等奉引謝恩。首途,如故。或帝不親送,則令親王、內大臣往。噶爾丹之役,先自歸化驛召費揚古為撫遠大將軍,至日賞宴,聖祖御太和門,大臣隅坐,其出征運糧大臣分坐金水橋北左右。作樂陳百戲,命大將軍進御前,親賜卮酒。跪受叩飲訖,都統、副都統繼進,則令侍衛授酒。參領以下十人一列,跪飲階上而已。復命大臣等遍視眾軍飲宴畢,賜與宴者御用蟒幣,餘賜幣,兵賜布。同謝恩出,大學士始以敕印授大將軍。

雍正七年,定命將前一日告廟。行日告奉先殿,並遣官。若先出師疆埸,即軍前命為大將軍者,則命正、副使齎敕印往。大將軍率屬俟教場,事設黃案,陳敕印。大將軍跪,宣敕文正使授敕,宣印文副使授印,大將軍以次祗受,轉授左右從官,行三跪九叩禮。禮成,奉入大營。

乾隆十四年,定命將儀三:一曰授敕印,經略大將軍出師,皇帝臨軒頒給。二曰祓社,凡出師前期,告奉先殿,禮堂子,祭纛。三曰祖道,經略啟行,皇帝親餞賜酒,命大臣送郊外,具祖帳暨宴,儀並詳前。徂征儀二:一曰整旅,經略前隊列御賜軍械,次令箭,次敕印,次標旗,大隊軍旅殿。令箭、標旗數皆十二。二曰守土官相見,經略過境,將軍、督、撫蟒服出郭迎候,文自司道、武自總兵以下,跽道右及事。經略正坐,將軍、督、撫側坐,文司道、武提督以下,行庭參禮。啟行候送如前儀。若頒敕印不御殿,即除鹵簿、樂懸,百官無職事者不會集。

三十四年,命大學士傅恆經略云南軍務,高宗不升殿,不禮堂子,不祭纛,不親送。內閣學士奉敕印至太和殿,經略等先俟陛階,大學士二人立殿外。屆時經略升陛,印官從大閣學士入奉敕印出,經略跪受。禮畢,奉敕印官前,經略後,及階下,置敕印採亭內,前張黃蓋,列御仗,從征侍衛前引,餘俱後隨,至經略第止。敕印陳案上。屆日肅隊行。

奏凱天聰初元,朝鮮奏捷,班師。車駕出城,頓武靖營野次。設行幄御營一里外,率諸貝勒逾行幄數武,立馬以待凱旋。既至,遂依次排列,立纛、拜天,入覲,帝出位迎之。諸貝勒行跪拜禮,賜筵宴。崇德元年,徵明凱旋,太宗率群臣出城十里迎勞,王、貝勒等依次成列,建纛鳴螺,帝率同拜天,三跪九叩。畢,升座。王、貝勒進獻捷表,大學士接受,奉御前讀訖,跪叩如儀。頒旨行抱見禮。於是王、貝勒進御前一跪三叩,賜坐、設宴同。

順治二年,南京平,豫王班師還。世祖赴南苑迎勞,樹十餘大纛,如初禮。十三年,定制出征王大臣凱旋,遣王公一人偕大臣郊勞。

康熙元年,定凱旋次日,帝御殿。禮成。免將軍等行禮,筵宴免桌席,止宰牲。

二十一年,大將軍貝子章泰等自雲南奏凱,駕至盧溝橋迎勞駐蹕,有司治具,翼日駕蒞至,齊眾拜天,以為故事。乾隆十四年,定奏凱功成,祭告天地、廟社、陵寢,釋奠先師,勒碑太學,命儒臣輯平定方略垂奕。經略大將軍師旋,將入城,遣廷臣郊勞,帝臨軒,經略率有功諸臣謝恩,繳印敕,儀同受敕。宴禮既畢,兵部覈敘勛績,頒爵賞有差。

厥後定邊將軍兆惠等、定西將軍阿桂等奏凱,高宗均駐蹕黃新莊行宮,築臺郊勞,百官咸會。設黃幄正中,南鄉,兩翼青幕各八,東西鄉。臺在幄南,其上建左右纛,中設帝拜褥。東西下馬紅柱各一。帝御龍袞詣臺,鳴螺,奏鐃歌樂。將軍暨從征大臣、將士皆擐甲胄,跪紅柱外俟駕。帝就拜位立,將軍暨群臣班分東西,鴻臚官贊「跪」,則皆跪。贊「叩,興」。帝拜天,三跪九叩,將軍等如之。畢,帝御幄升座,王公百官立東班幕下。禮成,帝出幄乘騎,凱歌作,奏鬯皇威章,駕還行宮。餘依康熙間故事。

咸豐五年,科爾沁親王僧格林沁平高唐亂。還朝日,文宗御養心殿,行抱見禮,慰勞備至。先是出師頒參贊大臣關防,賜訥庫尼素光刀,至是同時獻納。

受降崇德二年春,朝鮮王服罪請降。乃築壇漢江東岸,設黃幄,駕出營,樂作。濟江登壇,鹵簿具。朝鮮王率陪臣步行來朝,遣官出迎一里外。引入,帝率同拜天,升座。國王等伏地請罪,贊「行三跪九拜禮」。賜坐,位列親王上,諸子列貝勒子。錫筵宴,還其俘,並賜王以下貂服。

六年,蒙古貝勒等投誠,朝見已,命較射,選力士角牴,賜宴俾盡歡,殊典也。所貢方物悉卻之。

乾隆十四年,議制凡軍前受降,飛章入告。報可。乃大書露布示中外,築壇大營左,南鄉。壇南百步外樹表,建大旗,書「奉詔納降」字。降者立其下,經略大將軍戎服出,鼓吹聲砲,參贊大臣等騎從。將至壇,降者北面匍伏,經略登壇正坐。參贊僉坐,諸將旁立,餘皆肅班行。降者膝行詣壇下,俯首乞命,經略宣上德意,量加賞賚。營門鼓吹殷然,降者泥首謝,興,退。

獻俘受俘清初太祖、太宗以武功徵服邊陲,俘虜甚眾,其時獻受猶無定制也。雍正二年,討平青海,俘至京,始定諏吉先獻廟、社。俘白組系頸,行及太廟街門外北鄉立,承祭官朝服至,俘伏,儀同時饗。至社稷街亦如之。承祭官入壇致祭,儀同春、秋祈報。監俘者以俘出。翼日,帝御午門樓受俘,正中設御座,簷下張黃蓋,鹵簿陳闕門南北,仗馬次之。輦輅陳金水橋南,馴象次之。王公百官咸集,解俘將校立金鼓外,俘後隨。班位既序,帝御龍袞,乘輿出宮,至太和門,大樂鐃吹,金鼓振作。登樓升座,贊「進俘」,丹陛大樂作,奏慶平章。鴻臚寺官引將校入,北面立,贊「行禮」,俘入匍伏。兵部官跪奏,平定某地所獲俘囚,謹獻闕下,請旨。制曰:「所獻俘交刑部」。刑部長官跪領旨訖,械系出。丹陛大樂作,王公百官行禮如常儀。若恩赦不誅,則宣旨釋縛,俘叩首,將校引出。是日賜將校宴兵部,次日賜冠履銀幣有差。凡平定疆宇,受俘儀並同。

乾隆時,版圖日廓。二十年,剿平準噶爾,獲達瓦齊暨青海羅卜藏丹津,先後檻入。一歲中兩行斯典。越五年,底定回疆,討平攢拉促浸,皆遞舉盛儀。先後六歲,凱歌四奏,時論稱極盛云。

大閱天聰七年,太宗率貝勒等督厲眾軍,練習行陣,是為大閱之始。

順治十三年,定三歲一舉,著為令。尋幸南苑,命內大臣等擐甲胄,閱騎射,並演圍獵示群臣。

康熙十二年,閱兵南苑,聖祖擐甲,登晾鷹臺,禦黃幄,內大臣、都統等各束部曲,王、貝勒等各率旗屬,並自西而東。既成列,槍鳴號發,自東結陣馳以西,按翼分植。閱畢,命樹侯臺上,親發五矢,皆中的,復騎而射,一發即中。釋甲賜宴,乃還。厥後行閱,或盧溝橋,或玉泉山,或多倫諾爾,地無一定,時亦不以三年限也。

三十四年,復幸南苑行閱,分八旗為三隊,帝率皇子擐甲,內大臣等扈從,後建龍纛三,上三旗侍衛隨行。遍閱驍騎、護軍、前鋒、火器諸營。立馬軍前,角螺鳴,伐鼓,行陣舁鹿角進。甲士麾紅旗,槍砲齊發。鳴金止,再伐鼓,發槍砲如初。如是者九。初進率五丈,再進亦如之。至十進,槍砲環發無間。開鹿角成八門,首隊出,二隊、三隊從。既成列,門闔,角鳴,呼譟進。兩翼隊皆雁綴進,鳴金收軍。立本陣,結隊徐旋,首隊殿。罷閱,還行宮,申敕明賞罰。未閱前,賜軍士食,既閱,賜酒。

雍正七年,世宗幸南苑,閱車騎營兵,諭曰:「此第訓練一端耳,遇敵決勝,在相機度勢,神而明之,存乎其人,豈區區陣伍間遂足以制敵耶?」是日操演,各依方位、旗色為陣式。後北征,屢以車戰勝。

乾隆二年,大閱,幸南苑,御帳殿。軍隊既齊,步軍整列進。以十丈為率,餘儀同。令甲,大閱日,行宮外陳鹵簿,駕出,作鐃歌大樂,奏壯軍容章。及還,作清樂,奏鬯皇威章。凡操時鳴砲三,駕出及還同。即日賜各旗饌筵、羊豕、薪炭。迄嘉慶間,皆如故事行。

會閱為康熙三十年創典,時喀爾喀新附,聖祖思訓以法度,特命會閱上都七溪,乃集其部眾,並四十九旗籓王、臺吉,豫屯百里外。駕出都,上三旗兵從,下五旗兵自獨石來會。布營設哨,三旗護軍為一營,居中。八旗前鋒為二營,五旗護軍為十營,火器營兵為四營,環御營而屯。前鋒為四哨,護軍為二十四哨,各設廬帳,繞營而居。蒙古、喀爾喀諸屯徙近五十里,禁入哨。釐賞九等,序坐七列。網城設宸幄,正中御床,左右行帳各二,儀仗、樂懸具。依次置宴。蒙古王等居左,喀爾喀居右,順序習舞,眾技畢陳。乃命喀爾喀汗、濟農、諾顏等進御前,賜卮酒,餘令侍衛分送。禮成。翼日各營就列,陳巨砲,帝擐甲,閱畢宣敕,去其汗號,以王、貝勒、貝子、公名爵分錫之。臺吉分四等,比四十九旗,依等賜賚,恩禮有加,餘如儀。

京營訓練,歲以春、秋季月合操四次,春貫甲,秋常服,營陣規制如大閱。仲春、孟秋則按旗登城習鳴螺。兵部遣官稽閱,歲為常制。護軍驍騎營一歲三校騎射,前鋒護軍營三歲一較騎射,內大臣、本旗都統等臨視之。至直省講武,則以督、撫、提、鎮為閫帥,歲季秋霜降日,校閱演武場。先期立軍幕,屆日黎明,將士擐甲列陣,中建大纛,閫帥率將士行禮。軍門鼓吹,節鉞前導,遍閱行陣,還登將臺。升帳,中軍上行陣圖式,請令合操。遂麾旗,聲砲三、鳴角、擊鼓。軍中聞鼓聲前進,鳴金則止。行陣發槍如京營制。閱畢,試材官將士騎射,申明賞罰,犒勞軍士。

漕河訓練同八旗。水師操防,出洋信候,各省不同。歲春、秋季月或夏季,遇潮平風正,則乘戰艦列陣,張颿馭風,鳴角聲砲,具如軍律。綠營水師同。

秋獮清自太祖奮跡東陲,率臣下講武校獵習兵,太宗踵行之。世祖統一區夏,數幸南苑,令禁旅行圍,始立大狩扈從例。

康熙初元,定車駕行圍駐所置護軍統領、營總各一人,率將校先往度地勢,武備院設行營,建帳殿。繚以黃髹木城,立旌門,覆以黃幕。其外為網城,宿衛屯置,不越其所。十年,罷木城,改黃幔。康熙二十年,幸塞外,獵南山。尋出山海關,次烏拉,皆御弓矢校獵。越二年六月,幸古北口外行圍,木蘭蒐獵始此。

木蘭在承德府北四百里,屬翁牛特。先是籓王進獻為蒐獵所,周千三百餘里,林木蔥鬱,水草茂,群獸聚以孳畜焉。至是舉行秋獮典,間有冬令再出者。三十三年,設虎槍營,分隸上三旗,置總統、總領。大狩行田,遇有猛獸,列槍以從。並命各省駐防兵歲番獵以為常。六十一年,復幸塞外行圍,賞蒙古王公等衣物,定為恆制。

雍正八年,令八旗人習步圍,旗各行圍二三次。

乾隆初元,置綜理行營王公大臣一人,凡啟行、校獵、駐蹕、守衛諸事皆屬之。六年,御史叢洞奏請暫停行圍。諭曰:「古者蒐苗獮狩,因田獵講武事。皇祖行圍,既裨戎伍,復舉政綱。至按歷蒙籓,曲加恩意,尤為懷遠宏略。且時方用兵,數有徵發,行圍偶輟,旋即興舉。況今承平日久,人習宴安,弓馬漸不如舊,豈可不加振厲?是秋木蘭行圍,所過州縣,寬免額賦十之三,永為例。」圍場凡六十餘所,每歲大獮,或十八九圍,或二十圍,逾年一易。設圍所在,必豫戒期,首某所,逕某所,訖某所收圍,並編定其處。屆日官兵赴場布列,祗俟禦蹕臨圍。自放圍處作重圍,令虎槍營士卒及諸部射生手專射自圍內逸出諸獸。

高宗每行獵,自舊籓四十九旗暨喀爾喀、青海諸部分班從圍,綏輯備至。洎平西域,遠籓如左右哈薩克,東西布魯特,安集延,布哈爾,朝謁踵集,唯恐後時。土爾扈特亦皆挈部眾越數萬里來庭。帝嘗御布固圖昌阿撫慰之,旋賜名曰「伊綿」,國語會極歸極也。

二十年,更定網城植連帳百七十五,設旌門三,分樹軍纛曰金龍。去網城連帳外十許丈為外城,植連帳二百五十四,設旌門四,分樹軍纛曰飛虎。去外連帳六十丈,周圍警蹕,立帳房四十,各建旗幟,八旗護軍專司之。其規制詳密如此。

凡秋獮,先期各駐防長官選材官赴京肄習。年例,蒙籓選千二百五十人為虞卒,謂之「圍墻」,以供合圍役。

屆期,帝戎服乘騎出宮,扈引如巡幸儀。既駐行營,禁兵士踐禾稼、擾吏民,訶止夜行,違者論如律。統圍大臣蒞場所,按旗整隊,中建黃纛為中軍,兩翼斜行建紅、白二纛為表,兩翼末國語曰烏圖哩,各建藍纛為表,皆受中軍節度。管圍大臣以王公大臣領之,蒙古王、公、臺吉為副。兩烏圖哩則各以巴圖魯侍衛三人率領馳行,蟬聯環匝,自遠而近。蓋圍制有二,馳入山林,圍而不合曰行圍,國語曰阿達密。合圍者,則於五鼓前,管圍大臣率從獵各士旅往視山川大小遠近,紆道出場外,或三五十里,或七八十里,齊至看城,是為合圍,國語曰烏圖哩阿察密。看城者,即黃幔城也。圍既合,烏圖哩處虞卒脫帽以鞭擎之,高聲傳呼「瑪爾噶」,蒙語謂帽也。聲傳遞至中軍,凡三次,中軍知圍合,乃擁纛徐行。

日出前,帝自行營乘騎先至看城少憩,俟藍纛至,駕出,御櫜鞬,入中軍周覽圍內形勢。凡疾徐進止,口敕指麾。獸突圍,發矢殪之。御前大臣、侍衛皆射其逸圍外者,從官追射。或遇猛獸,虎槍官兵從之。或值場內獸過多,則開一面使逸,仍禁圍外諸人逐射。獲獸已,比其類以獻。駕還行宮,謂之散圍。頒所獲於扈從者,大獮禮成,宴賚有差。

哨鹿者,凡鹿始鳴,恆在白露後,效其聲呼之,可引至。厥制與常日不同。侍衛等分隊為三,約出營十餘里,俟旨停第三隊。又四五里,停第二隊。又二三里,將至哨鹿所,則停第一隊。時扈從諸臣止十餘騎而已。帝命槍獲鹿,群引領俟旨,而三隊以次至御前,高宗蒐獵木蘭時,親御名駿,命侍衛等導入深山中。望見鹿群,命一侍衛舉假鹿頭作呦呦聲,引牝鹿至,亟發矢殪之,取其血以飲。不唯益壯,亦以習勞也。嘉慶時秋獮仿此。

日食救護順治元年,定制,遇日食,京朝文武百官俱赴禮部救護。康熙十四年,改由欽天監推算時刻分秒,禮部會同驗準,行知各省官司。

其儀,凡遇日食,八旗滿、蒙、漢軍都統、副都統率屬在所部警備,行救護禮。順天府則飭役赴部潔凈堂署,內外設香案,露臺上爐檠具,後布百官拜席。鑾儀衛官陳金鼓儀門兩旁,樂部署史奉鼓俟臺下,俱鄉日。欽天監官報日初虧,鳴贊贊「齊班」。百官素服,分五列,每班以禮部長官一人領之。贊「進」,贊「跪,叩,興」。樂作,俱三跪九叩,興。班首詣案前三上香,復位。贊「跪」,則皆跪。贊「伐鼓」,署史奉鼓進,跪左旁,班首擊鼓三聲,金鼓齊鳴,更番上香,祗跪候復圓。鼓止,百官易吉服,行禮如初。畢,俱退。是日禮部祠祭司官、欽天監博士各二人,赴觀象臺測驗。鄉日設香案,初虧復圓,行禮如儀。

若月食,則在中軍都督府救護,尋改太常寺,如救日儀。直省遇日、月食,各按欽天監推定時刻分秒,隨地救護。省會行之督、撫署,府、、州、縣行之各公署,並以教職糾儀,學弟子員贊引,陰陽官報時。至領班行禮,則以督撫及正官一人主之。上香、伐鼓、祗跪,與京師救護同。


\end{pinyinscope}