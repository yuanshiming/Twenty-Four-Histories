\article{列傳二百十}

\begin{pinyinscope}
宗稷辰尹耕雲王拯穆緝香阿游百川沈淮

宗稷辰,字滌甫,浙江會稽人。道光元年舉人,授內閣中書,充軍機章京。遷起居注主事,再遷戶部員外郎。咸豐元年,遷御史。疏請飭各省實行保甲,略言:「州縣宜久任,時日宜寬假,填寫門牌當詳細核對,董事胥役毋派費累民,酌用丞簿以為襄助,先編巨族以為聯屬,並可申明讀法之典,兼收團練、社倉之益。」詔下直省督撫,各就地體察參酌行之。又疏言通籌出入,宜崇實去偽,舉清查、報效、生息三端;又疏請酌改經徵處,分令州縣戴罪嚴催:並下戶部覈議。五年,聞上將謁陵,未有旨戒行,稷辰疏言:「畿南州縣被水,連歲用兵,民氣甫行休息,籥請展緩一年。」上諭曰:「每歲謁陵,事同典禮,如果畿輔民力未逮,亦必權衡時勢,暫緩舉行。今茲並未降旨何日謁陵,宗稷辰揣度陳奏,徒博敢諫之名而無其實。此風不可長!」下部議處。

尋又奏言:「自粵匪竄據長江,數年以來,文臣武將,能戰者稀。如烏蘭泰、塔齊布、江忠源皆難得之將,而多不盡其用,且以死殉。如勝保、張亮基、袁甲三皆勇於任事,而亦未盡其用,以罪罷去。近日支持兩湖,賴有一二書生,如胡林翼、羅澤南,能以練膽為士卒先。此二人者,實曾國籓有以開之。此時若開文武兼資一科,誠足濟當時之急,而臣工多不敢薦舉者,一恐其才疏而得過,一恐其遇蹇而罔功。處愁眉焚頂之時,守蹈常習故之轍,見有敗衄,動以餉匱為辭。餉固不可不籌,試思用兵乏人,雖斂金百萬,棄如土苴,終歸無用。臣聞見隘陋,非能盡識天下之才,所知湖南有左宗棠,通權達變,為疆吏所倚重,若使獨當一面,必不下於林翼、澤南。其屢經論薦,難進易退,肝膽經術,實可取材者,有若湖州之姚承輿。其策議深沉,才識過人者,有若常州之周騰虎、管晏,桂林之唐啟華,皆關心時務,今尚鬱鬱伏處田間。誠能破格招賢,連茹並進,則得一人可以平數州,得數人可以清一路。長江雖阻,當不難分道建功,剋日平定。伏乞皇上命內外臣工各舉所知,無論已仕未仕,果能文武兼資,皆許徵起,必可網羅而盡得之。」疏入,下各督撫,命以宗棠等加考送部引見。宗棠自此膺簡拔,論者謂其知人。

遷給事中。時京師行大錢,商民苦之。稷辰上疏請復用制錢,號曰「祖錢」,而大錢改純用鐵鑄,兼行並用。下部議,格不行。又以畿輔水患,疏請急賑,從之。尋授山東運河道,捻匪入境,於濟寧牛頭河濱築戰墻,北岸六千三百丈,南岸八千六百丈,賴以守御。以功加鹽運使銜。同治六年,引疾歸,尋卒。

稷辰父霈正,官湖南零陵知縣,廉無餘貲。稷辰事母孝。為學宗王守仁、劉宗周。罷官後,主餘姚龍山書院、山陰蕺山書院。官京朝,請祀總兵葛云飛本籍;官山東,請修方孝孺祠,並刻正學集:其振勵風教多類此。

尹耕雲,字杏農,江蘇桃源人。道光三十年進士,授禮部主事,再遷郎中。咸豐五年,粵匪犯畿輔,惠親王綿愉為大將軍,僧格林沁參贊軍務,闢耕雲佐幕府,上書論防務,為文宗所知。八年,授湖廣道監察御史,署戶科給事中。時方多事,封章月數上。直隸總督訥爾經額坐貽誤封疆罷,復起。耕雲疏言:「訥爾經額之罪,天下共聞共見,未喻其復行起用之故。方今江、淮、楚、豫軍務未靖,秉鉞之臣,星羅棋布,所以奮不顧身,必欲滅此朝食者,固由篤於忠義,亦以國家信賞必罰,有以畏服其心。萬一效尤解體,患何可言?昔宣宗起用琦善,以陳慶鏞之言而罷。伏原紹述心傳,收回成命。」

時粵匪復窺武漢,耕雲疏言:「武漢地踞上游,北窺關陜,南脅湖湘,東撼吳越,西制巴蜀,自古南北用兵,皆出死力爭之。今賊窺伺楚北,分擾廣濟、黃岡,逼近省城,撫臣胡林翼兵勇數千,眾寡懸絕,江路綿遠,首尾不能兼顧。侍郎曾國籓忠勇樸誠,應請授為欽差大臣,率其所部援湖北,較諸他臣事半功倍。」

粵匪陷定遠,耕雲疏言:「定遠失守,粵、捻新合,必謀北竄,恃山東為之屏蔽。撫臣崇恩幸其不戕官據城,於賊退後虛報勝仗,內則巧為彌縫,掩一人耳目;外則恣其朘削,竭萬姓脂膏。惟懇俯念籓籬重地,立予罷斥,簡大員往代。於洪湖多募水師,兼飭傅振邦全軍移駐固鎮、靈壁,冀收皖北,以固山東。」及廬州失守,又疏言:「昔人建省安慶,與九江、江寧為犄角,控扼長江。上年徙治廬州,已失形勝,茲並廬州亦不能守。胡林翼等自武漢進逼九江,而安徽之賊,或自英、霍走湖北,牽我上游,或自徽、歙擾浙西,窺我腹地。我軍分道救援,罷於奔命。賊有四達之路,我無三面之圍,雖日克一城,何益?撫臣福濟屢挫損威,候補京堂袁甲三素得民心,如以為巡撫,必奮身圖報。」

及國籓進師,疏言:「軍興以來,徵調半天下,糜餉數千萬,卒未能掃穴擒渠,則以屢後時而數失機也。今曾國籓蓄養精銳,所向克捷。陳玉成、張洛行率悍賊數十萬,齊向潛山、太湖抗拒,眾寡之數,十倍於我,一有疏虞,關系甚重。此時廬、鳳、六合賊勢必單,請飭袁甲三、張國樑刻期搗其巢穴,逼令反顧,或令間道為楚師聲援,亦足褫其狂魄。」別疏劾河道總督庚長,請以甲三兼攝;又論云南回匪不宜專意主撫;又陳京師本計,平糶、採買、周恤、蓄積諸事宜並舉;又言錢法積弊:諸疏多見採納。

英、法合軍犯天津,耕雲專疏者七,會疏者二,力主決戰,上命王大臣集議。與鄭親王端華等議不合,耕雲抗辯痛哭而罷。耕雲初在禮部,肅順頗重之,乃是為所憎。九年,科場獄起,以科道失糾下吏議,而耕雲以充內監試譴獨重,鐫二級調用。十年,京師戒嚴,上將幸熱河,耕云代團防大臣草疏諫阻,復自以書抵肅順,卒不聽。侍郎文祥提督九門,遇耕雲東城,相持哭,因為規畫留守諸事。

胡林翼疏薦耕雲胸有權略,請起用。會副都御史毛昶熙治河南團練,疏調從軍。同治元年,率部卒五千,從僧格林沁平金樓寨教匪,又偕提督張曜克張岡捻巢,以道員記名,賜花翎。三年,署河陜汝道。西征軍購糧陜州,市斛小,責屬縣償其不足,凡數百萬斤,耕雲悉請罷之。客軍有不法者,斬以徇。境多刀匪,請得節制河、陜兵,饋餉以時,兵咸用命。

四年,張總愚犯畿輔,耕雲從巡撫李鶴年進軍磁州,建策築長圍斷賊歸路。兩署糧儲鹽法道,佐治善後事,濬惠濟河,塞河決,敘勞加布政使銜。十三年,補河陜汝道。河、陜徭役重,亞於常賦,耕雲立定制,嚴稽覈,民困稍甦。光緒三年,大旱,條上救荒七事,未及行,卒於官。

耕雲在言路著直聲,出任監司,巡撫張之萬、李鶴年皆倚重之,軍事多所贊畫。卒後,巡撫李慶翱以災荒被劾,牽及冒領兵餉事,辭連耕雲,後終得白雲。

王拯,初名錫振,字定甫,廣西馬平人。道光二十一年進士,授戶部主事,充軍機章京。大學士賽尚阿視師廣西,以拯從,拯感時多難,慷慨思有所建白。咸豐間,自郎中累遷大理寺少卿。同治二年,降捻宋景詩由陜西還擾直隸、山東,拯奏言:「景詩岡屯磚圩,儼然嵎固,自陜逸回,其黨不過數百。崇厚等一再養癰,裹脅逾萬。近復於昌邑、莘、聊城、臨清四州縣,令村莊將所獲麥與佃戶平分,運送岡屯,是其名為降伏,心跡轉益兇悖。請密敕直隸督臣劉長佑計調來營,暴其罪而誅之。若抗違不至,直隸官軍猶能越境進剿。景詩既除,如楊蓬嶺、程順書等首惡,皆可駢誅,以除巨憝,以安畿輔。」疏入,未行。其後景詩卒以叛誅。

軍事未定,曾國籓議於廣東籌餉,勞崇光創辦釐金,諸弊叢起。拯疏言:「兩粵為肇亂之區,岑溪、容縣,數載皆為賊踞。信宜陳金缸尤為巨憝,群賊相為一氣,滋蔓難圖。勞崇光舉辦釐金,率令紳商包充墊繳,燃眉剜肉,事何可常?及崇光去任,徵收減少。近乃有釐務委員,或為眾所毆傷,或為民間枷號,雖民情頑獷,而官吏惡劣亦可概見。以積年久亂之地,有負嵎圜視之賊,當一切利孔、百方搜剔之時,臣竊恐利未十而害已百。萬一兩粵復糜爛,更不知何所措手足,豈惟釐金不能辦而已?」因薦廣東道員唐啟廕、兩淮運使郭嵩燾、浙江運使成孫詒。旋用嵩燾督廣東釐金,自拯疏發之也。

三年,遷太常寺卿,署左副都御史。疏論:「總理各國事務大臣侍郎崇綸、恆祺、董恂、薛煥委瑣齷齪,通國皆知,竊恐外邦輕侮,以為中朝卿貳之班,大都不過如若曹等,未免為中朝恥辱。就令人材難得,或於總理衙門位置為宜,上應量為裁抑,或處以散職,或畀以虛銜,庶外邦服我旌別之嚴。四方聞之,亦釋然於朝廷宥納群倫、羈縻彼族之意。」

尋遷通政使,仍署左副都御史。疏言:「近日蘇、杭迭克,直、東肅清。臣觀從來將興之業,垂成之功,未有不矢以小心,而始能底定者。金陵賊窟雖計於三四月間可拔,而丹陽與常州犄角,百戰悍賊如李秀成等,麇集死守。杭、嘉既克,餘黨歸並湖州。其自皖南竄越江西之賊,蔓延玉山、鉛山、金谿、建昌二三百里,眾號八九萬,並有闌入福建境者。又聞李世賢自率巨股由淳安、遂安接踵而至,曾國籓、左宗棠等用兵日久,前此屢陳不亟求功旦夕,同一老謀深計,獨於皖、浙毗境豫作防維之策,則國籓意在徽、寧各飭所部分防,宗棠以為不若並力取廣德扼賊竄路。兩議未及定,賊已由皖竄贛。賊又草竊已久,人數太眾,勢多不能聚殲而弗使一賊他遁。臣則以此賊人多勢劇,一意奔突,前股未痛剿,後股又踵接。萬一深入江西腹地,燼餘復熾,又至燎原。且由贛逾閩,可以直走汀、潮,為數年來竄匪熟路。黃文金由此而來,石達開由此而去,前事可為深警。疊蒙諭旨,曾國籓、左宗棠、李鴻章、沈葆楨及閩、粵各督撫諄諄戒備。當此大功將竟,惟當效力一心,互籌戰守,務將分竄諸賊,前截後追,必使所至創夷,日就衰殘零落,不得喙息,以成巨患。臣尤有請者,皖、浙諸軍與賊相持不為不久,所需餉項,國籓、宗棠等各於江、楚等省自為籌畫。國籓奏於江省設立總臺,以一省捐釐之數,為皖軍十萬養命之源。浙軍固不能分撥,即國籓所部月餉,傳聞亦祗放數成,不得已而籌及廣東釐捐,乃又不能遽辦。夫民之不能見遠而各為其私者,情也。廣東有之,江西豈獨不然?日前沈葆楨奏請將江西茶稅、牙釐等款歸本省任收,旋用部議允留其半,在國籓等斷不至觖望。惟軍前將卒,當枕戈喋血切望成功之時,忽聞軍餉來源將減,眾心或生疑懼,何以得飽騰而資鼓舞?擬請飭贛、皖、楚、粵各疆臣,值此事機至緊,無論如何變通為難,總當殫竭血誠,同心共濟。甘肅回氛未戢,中州餘捻尚存,汝南陳大喜等竄逸湖北,自隨、棗逼襄、樊;張總愚自南臺山中出竄內、淅,時虞合並;漢中之賊,全竄寧、陜、商州一路,聞將會齊襄、樊回援金陵,誠亦未可輕忽。目前陜省軍務,政出多門,李云麟追賊商於,忽卷旆而西,其在興安,未能遏賊竄逸,其在漢陰,遇賊避匿,縱勇淫掠,宜量加裁抑。劉蓉素嘗學問,懷負非常,漢中之賊,本所專辦,而竄擾四出,尤當誓志蕩除,方為不負。多隆阿聲望最優,眾口爭傳為第一名將,乃近日聲望漸損,宜申聖諭訓飭。雷正綰所向克捷,諒足當一面之寄,顧全甘官吏,未有一二正人支持其間。現聞蘭州與慶陽隔絕,恩麟權督印,不過使令便闢之材,識見陋劣;熙麟坐守慶陽、寧夏一區,又為慶昀種種紕繆所誤。臣愚以為亟宜遴簡公正有為之大臣,鎮撫整飭。今之天下,何易遽言率土奠安,而南北軍務漸定,西事再能就緒,亦即為大致之澄清。朝廷者天下之本,宮府清明嚴肅,與疆場奮迅振拔之氣,相感而自通。天下大勢日轉,而亦正多難鉅之事,或遽以為時局清明,事機暢遂,若已治已安者然。人情大抵喜新狃常,畏難而務獲,獨有當幾至誠君子,為能深察而切戒之。昔諸葛亮為三代下一人,史獨稱之以謹慎。硃子進戒宋孝宗曰:『使宴安酖毒之害,日滋而日長;將臥薪嘗膽之志,日遠而日忘。』臣不勝私憂過計,冒昧瀝陳。」疏入,報聞。尋告歸,卒。

穆緝香阿,字居南,滿洲鑲紅旗人。由工部主事再遷郎中。同治四年,授山東道監察御史。疏請慎擇宦寺,略言:「皇上沖齡御極,聖學日新,知識日開,左右侍從之輩,宜豫加慎選,勿使將來蠱惑聖聰。溯自漢末及前明,朝政之失,半由宦寺。蓋宦寺出身之始,每以小忠小信,便捷逢迎,無非售其固寵邀恩之計。及黨與已成,則驕肆專橫,而箝制其上,雖英明之主,竟有百計不能除之者。當時臣民,切齒痛恨,終歸無可如何。我朝列聖相承,遠邁前代,不但不準此輩干預政事,雖應對進退間亦不假以辭色,使無由讒諂面諛,浸潤膚受。是以二百餘年,從不為患。雖然如此嚴防,尚有防不勝防之慮。嘉慶癸酉之變,猶有通賊者,是此輩反覆已有明徵也。今皇太后垂簾聽政,洞悉其弊,杜漸防微,有鑒於前,不使宵小蒙蔽。所以知人善任,朝政肅清。即數年後皇上親政,亦斷不致寵任此輩,貽誤事機,何待臣下鰓鰓過慮?然獻曝之忱,有不能已者。當此之時,正聖學擴充之際,雖臣工皆能盡心輔佐,而宦寺尤宜加意斟酌。臣以為宦寺之設,無非效奔走、供指使而已,萬不可使年輕敏捷之人,常侍左右。請皇太後選忠正老成者為我皇上朝夕侍從,庶將來親政,必不致受其欺蒙蠱惑,而無疆之聖德,基於此矣!」

五年,疏論大學士曾國籓督師討捻,日久無功,請量加譴責。上以國籓迭疏引咎,特命回任專辦餉糈,雖未蕆全功,非貽誤軍情者可比,斥所奏過當,置不議。出為山西蒲州知府,尋卒。

穆緝香阿通知國故,家藏邸報,自國初以來幾備。

游百川,字匯東,山東濱州人。同治元年進士,選庶吉士,授編修。六年,遷御史,巡西城。宗室寬和等所行多不法,奏劾懲治,一時貴近斂跡。七年,捻匪自山東竄直隸,百川奏請飭統兵大臣迅速剿辦,又請嚴禁各省栽種罌粟,上皆採納。疏論內外官署胥吏積弊,詔通飭嚴禁。復言:「除吏弊在肅官方,尤在揚士氣。請飭部院堂官於每司中擇賢俊數員,付以事權,專其責任。察有胥吏舞弊,據實上陳,仍以勤惰定功過。賞罰既明,人才自奮。至外省地方官,本有懲治胥吏之權,嚴飭各督撫為地擇人,毋以人試地。舉賢劾不肖,再簡廉正大員,以時巡察,遇有貪官蠹吏,列狀奏聞。」

黃河北徙,山東郡邑屢被水。百川疏請賑恤,河督文彬、巡撫丁寶楨請仍挽復淮、徐故道,命廷臣集議。百川疏言:「黃水宜南宜北,必將折衷一是。如議挽復故道,論工程,論經費,引黃濟運,有未可遽定者三端:如即以大清河為黃水經流,舊道斷不能容,河面必須加寬,民間田廬如何移徙,如何安置,則度地宜審也;且即河面加寬,仍恐萬難容納,別開支河,勢不容已,徒駭、馬頰、鉤盤、鬲津猶可指名,可否開行,有無貽害,則分水宜權也;黃水北行,其事為創,萬一不善料理,人情騷動,物議沸騰,則相機宜慎也。請特派大臣履行上下游詳勘,然後定策。」

十二年,上親政,命葺治圓明園,奉皇太后駐蹕。御史沈淮疏請暫緩修理,上特諭宣示孝養兩宮之意,專修安佑宮供奉列聖御容,暨皇太后駐蹕之所,治事之地,量從節儉,不事華靡,此外均不必興修。百川繼疏申諫,上召入詰責,百川侃侃正言無所撓,上為動容,一時敢諫之名動朝野。尋以憂歸,服除補官,遷給事中。

光緒五年,出為湖南衡永郴桂道,遷四川按察使,擢順天府尹,遷倉場侍郎。九年,山東河決,被災者數十州縣,命百川往會巡撫陳士傑治工賑。百川輕騎周歷河南北岸、上下游,先散急賑。會奏請築兩岸遙堤,復於其內築縷堤,使黃水不致泛濫;又奏請濬小清河,分黃水入海:如議行。還京,以倉廒被火,罷歸。居數年,卒。

淮,字東川,浙江鄞縣人。道光二十九年舉人,授內閣中書,充軍機章京。咸豐十年,文宗狩熱河,淮不及從,慟哭欲投井,家人守之不得死。遷刑部主事,進員外郎,授陜西道監察御史。疏劾戶部主事楊鴻典攬權納賄,下刑部逮治,僅以小過議鐫級,及閻敬銘為尚書,始奏劾譴黜。園工興,淮疏首上,當時與百川齊名。光緒元年,充順天鄉試監試,力疾從事,出闈,旋卒。家固中人產,官京師,斥賣殆盡,人尤服其清節。

論曰:用兵之際,事機千變,京朝官以傳聞有所論列,往往不能切中。宗稷辰歸重得人,尹耕云論諸將帥罪,王拯請調和疆吏,一意辦賊,為能見其大。拯所言尤詳盡,蓋直樞廷,見軍報,較得諸傳聞者異矣。穆緝香阿請慎選宦寺,游百川等阻修圓明園,謇謇負直諫名,良不虛也。


\end{pinyinscope}