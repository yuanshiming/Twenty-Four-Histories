\article{列傳一百八十五}

\begin{pinyinscope}
宗室祥厚霍隆武福珠洪阿恩長陳勝元祁宿藻陳克讓劉同纓

瑞昌傑純錫齡阿

宗室祥厚,隸鑲紅旗,襲騎都尉世職,授鑾儀衛整儀尉。累擢鑲紅旗蒙古副都統,歷山海關、熊岳、金州副都統。道光二十八年,擢江寧將軍。

咸豐三年正月,粵匪既陷武昌,兩江總督陸建瀛赴上游督師,祥厚偕江蘇巡撫楊文定留守江寧。賊已蔽江而下,壽春鎮總兵恩長戰歿,建瀛遽退,文定亦不候旨逕赴鎮江。祥厚偕副都統霍隆武、提督福珠洪阿、布政使祁宿藻疏言:「督臣藉口江寧吃緊,趕回布置,沿途險要,並不屯扎,上駛師船,一概撤回,專守水路之東西梁山。蕪湖為江蘇門戶,亦不設防。十八日只身抵省,遂致闔城驚擾。臣等函勸速統舟師迎擊,乃督臣晏坐衙齋,三日不覆。撫臣執意移駐鎮江,挽留不顧,民情加倍驚惶。自今固結民心,尚恐緩急難恃;若任其紛紛遷徙,土匪因而竊發,奸細尤易勾結。是未御外侮,將成內變。現在督撫臣首鼠兩端,進退無據,以致省城震動。雖有旗兵志切同仇,無如兵力太單。賊船順流下竄,朝發夕至,守御萬分緊迫,督同道府等官及八旗協領,激勵官兵,安慰居民,竭盡血誠,認真辦理。請飭琦善、陳金綬迅速繞出賊前,協力堵剿,以固省城根本,維持南北大局。」疏入,詔逮建瀛治罪,命祥厚兼署總督,與霍隆武、福珠洪阿、祁宿藻悉心防禦,以在籍前廣西巡撫鄒鳴鶴熟悉賊情,命同籌辦。

江寧城周九十六里,合旗、漢兵僅五千,城外江寧鎮、龍江關、上河分駐鄉勇不及三千,臨時召募,皆不足恃。賊過蕪湖,福山鎮總兵陳勝元率舟師戰歿,遂無御者,長驅直抵城下,四面環攻。守逾旬,賊於儀鳳門穴地轟城,傾十餘丈,復由水西門、旱西門、南門緣梯而登,城遂陷。祥厚偕霍隆武斂兵守駐防城,婦女皆助戰,逾日亦陷。祥厚手刃數賊,身被數十創,死之。事聞,贈太子太保,予二等輕車都尉世職,謚忠勇。入祀京師昭忠祠,於江寧建專祠,死事者附祀焉。

霍隆武,鈕祜祿氏,滿洲鑲黃旗人,福州駐防。由武舉前鋒校歷官福建水師旗營協領。咸豐元年,擢江寧副都統。賊圍城,偕祥厚登陴固守,歷十餘晝夜,外城陷,同守內城,策馬督戰,受傷墮,力竭陣亡,贈都統,予騎都尉兼雲騎尉世職,謚果毅。

當時駐防旗兵戰最力。錫齡額者,事母孝,將軍本智異之,擢為參領。曰:「求忠臣必於孝子之門。」事急,戒其妻:「國家豢養,無所報;脫不利,當闔門死。」自守城,即不返家,舉室皆殉。炳元,官佐領,勇力冠軍。儀鳳門之陷,率死士奮斗,賊為之卻,忽有狙擊者,殞於陣。賊破內城,屠戮尤慘,男婦幾無孑遺。

福珠洪阿,蘇完瓜爾佳氏,滿洲正黃旗人,副都統佛安子。由鑾儀衛整儀尉累擢總兵,歷鎮筸、伊犁、西寧、天津諸鎮。道光末,授江南提督,調陜西。粵匪起,江南籌防,仍調回舊任,駐守省城,所部兵僅數百人。地雷發,迎擊於城缺,斬悍賊,而諸門先後破。賊四面至,往來巷戰,死之。贈太子少保,予二等輕車都尉世職,謚壯敏。

恩長,赫舍里氏,滿洲鑲紅旗人。由親軍、十五善射,累遷安徽寧國營副將。道光中治江防,被獎。累擢壽春鎮總兵。初率兵守安慶,陸建瀛赴九江上游,調充翼長,為軍鋒。與賊戰江中,毀賊船三十餘艘,眾寡不敵,死之。贈提督,予騎都尉兼雲騎尉世職,謚武壯。

陳勝元,福建同安人。由行伍歷官福建參將。捕洋盜有功,累擢江南福山鎮總兵。率水師防江,賊至太平四合山,迎擊,追至蕪湖,中砲落水,死之。贈提督,予騎都尉世職,謚剛勇。

祁宿藻,字幼章,山西壽陽人,大學士俊藻弟也。道光十八年進士,選庶吉士,授編修。以召對受宣宗知,特簡授湖北黃州知府,調武昌。連年大水,城幾沒,堵御獲全。治急賑,煮粥施錢及衣棺藥餌,全活災民甚眾,政聲最。超擢廣東鹽運使,遷按察使,又遷湖南布政使。會韶州數縣土匪起,詔留宿藻督兵往剿,七戰皆捷,匪首就擒。事平,賜花翎。調江寧布政使。咸豐元年,河決豐北,山東、江北皆被水。大學士杜受田奉命臨賑,疏請以宿藻督辦江北賑務,章程出其手定,奏頒兩省行之。

及粵匪將東下,宿藻馳返江寧,括庫儲治軍械,盡移兵糈及南門外商市囤米入城,號召義勇之士備戰守。見督撫倉皇失措,各存意見,勸諫不聽,乃偕祥厚等密疏上聞。建瀛既被罪失眾心,宿藻獨任事,賊至,力疾登陴指揮,歷三晝夜,城大兵單,援師不至,知事不可為,在城上嘔血數升,卒。文宗悼惜,加等優血⼙,贈右都御史,廕一子以知州用。同治初,江南平,兄俊藻遣尋其遺櫬,得之城北僻地。曾國籓以聞,請附祀祥厚專祠,追謚文節。當城陷時,署布政使鹽巡道塗文鈞、江安糧道陳克讓、江寧知府魏亨逵、同知承恩、通判程文榮、上元知縣劉同纓、江寧知縣張行澍同死之。

克讓,奉天承德人。道光三年進士,授吏部主事。累擢四川綏遠知府,調成都。咸豐元年,擢江安糧道,居官清正。賊將至,或勸以督運出。克讓曰:「江寧東南都會,失則大局危。去將焉往?」又請徙其孥,其妻泣曰:「去為民望,不如死!」宿藻死而不瞑,克讓撫之曰:「庫尚有儲金,當募死士以成君志。」克讓守清涼山,督兵戰,殞於陣。弟克誠,子松恩,同遇害。妻李,自經死。賜恤,予騎都尉世職,本籍請建專祠,追謚忠節。

同纓,江西石城人。拔貢。歷官鹽城、泰興、江浦、上元、六合、江寧諸縣,皆有聲。江寧治防,儲糧練團,胥賴其力。賊初至,假向榮書請入城,同纓察其詐,卻之。砲裂城,率死士御擊復完。及城陷,賦絕命詞,投水死,恤典加等,贈道銜,謚武烈。

瑞昌,字雲閣,鈕祜祿氏,滿洲鑲黃旗人。六世祖敖德,以軍功予騎都尉世職。瑞昌由拜唐阿授鑾儀衛整儀尉,累遷冠軍使。道光二十九年,擢正白旗漢軍副都統,歷金州、吉林副都統。

咸豐三年,擢杭州將軍,未之任,率盛京兵赴淮、徐,專辦山東防剿。尋從僧格林沁、勝保剿賊畿輔。四年,連戰靜海、河間、東光。五年,會攻連鎮,扼河西,毀賊巢木城。賊首林鳳祥就擒,被詔嘉獎,命赴本任。十年二月,粵匪由廣德入浙境,省城兵單,分防湖州、孝豐、餘杭。賊分股突犯杭州,瑞昌令副都統來存出武林門御之,自守錢塘門,偕巡撫羅遵殿布置甫定,賊已麕至,縱火撲城。越十日,地雷發,城陷。瑞昌率旗兵迎擊於湧金門,殺傷相當。退守駐防子城,賊屢攻,力拒卻之。相持六日,會張玉良率援兵至,夾擊,賊棄城走,遂復杭州,特詔嘉獎,賜黃馬褂,予二等輕車都尉世職。

既而江南大營潰,常、蘇兩郡陷。張玉良以罪黜,命瑞昌總統江南諸軍,江長貴副之,規復蘇州,而賊已陷長興、武康,復諭先顧杭城,再圖進取。嘉興尋為賊踞,命瑞昌督張玉良往攻,亦未果。十月,賊陷富陽、餘杭,復撲杭州,瑞昌親督副都統傑純、副將吳再升擊走之。十一年,賊勢益張,由嘉興進陷石門,湖州亦被圍,浙東諸郡相繼失守。自紹興為賊踞,杭州愈危,遂被圍,瑞昌偕巡撫王有齡嬰城固守逾兩月。張玉良戰城下,傷殞,軍心益渙。外援不至,糧道皆絕。瑞昌憂憤成疾,旗兵精壯多傷亡,乃集將校,誓死報國,家給火藥。及城陷,瑞昌先舉火自焚,闔營次第火起,同死者,杭州副都統關福及江蘇糧儲道赫特赫納以下男婦四千餘人。事聞,詔優恤,贈太子太保,晉一等輕車都尉世職,謚忠壯。入祀京師昭忠祠,杭州建專祠,死事者附祀焉。

同治三年,杭州復,左宗棠奏瑞昌妾吳,於城破時挈兩幼子緒成、緒恩出走失散。事定,尋得緒恩,護送回京。詔念瑞昌忠烈,命本旗傳交其長子內閣中書緒光收養,飭宗棠購訪緒成下落,迄未得。後以兩世職並為三等子爵。

傑純,布庫魯氏,蒙古正白旗人,杭州駐防。由驍騎校累遷協領。忠勇得士心,為瑞昌所倚。杭州初破,瑞昌欲自剄,傑純與副都統來存言賊以偏師疾至,未有後繼,猶可力保駐防城以待外援,瑞昌從之,乃登陴守御。傑純當武林門,日與賊戰,長子前鋒校納蘇鏗陣亡,不之顧,殮其尸,不哭,曰:「汝先得所歸矣!」及援兵至,怒馬突出,賊披靡,追擊出城十里外。以復城功,賜花翎。擢寧夏副都統,留浙協同團練大臣統率練勇,出省復富陽。是年冬,賊復犯杭州,迎剿於觀音橋,手刃數賊,率西湖水勇截擊,斬馘甚眾,又連破撲城之賊,追至留下,進克餘杭,賜號額騰伊巴圖魯。調授乍浦副都統,仍留防省城。

十一年,城再陷,傑純戰一晝夜,所部傷亡略盡,遣次子出避,以存宗祀,闔門自焚,獨策馬入賊陣,死之。詔嘉其一門忠烈,依都統例賜恤,予騎都尉兼雲騎尉世職,杭州、乍浦並建專祠,子婦孫僕皆附祀。後復加恩入祀京師昭忠祠,謚果毅。擢次子固魯鏗知府,改歸京旗。

錫齡阿,扎哈蘇氏,蒙古正白旗人,荊州駐防。以佐領率兵從戰沔陽、監利、潛江、應城、漢陽、宜昌。積功累擢福州副都統,調乍浦副都統。十一年,賊來犯,督兵出戰,城中內應起,折回巷戰,全軍皆沒,與兩子榮輝、榮耀同殞於陣。贈都統銜,予騎都尉兼雲騎尉世職,謚武烈,入祀京師昭忠祠。嗣以荊州紳民感念保境功,請建專祠。子榮輝、榮耀並予雲騎尉世職。

論曰:清制,行省要區置旗兵駐防,其尤重都會,兵額多者,以將軍領之。蓋監制疆臣,備不虞也。承平恬嬉,非復國初勁旅,小有變動,可資鎮懾;鉅寇燎原,力不足以御之。江南之失,誤於陸建瀛不預設防。祥厚倉猝專任,以孤城當方張之寇,寧有幸焉。杭州初陷,賊僅偏師,故瑞昌能守內城以待援;及蘇、常既失,輔車無依,終不能保,大勢然也。然二人者,皆能以忠義激勵,城亡與亡,婦嬰皆知效死,烈已!祁宿藻孤忠盡瘁,傑純智勇能軍,並一時傑出之才。炎岡同燼,世尤惜之。


\end{pinyinscope}