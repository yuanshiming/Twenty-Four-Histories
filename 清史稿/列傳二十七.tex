\article{列傳二十七}

\begin{pinyinscope}
李國英劉武元庫禮胡全才申朝紀馬之先劉弘遇

於時躍蘇弘祖吳景道李日芃劉清泰佟岱秦世禎陳錦

李國英,漢軍正紅旗人,初籍遼東。仕明隸左良玉部下,官至總兵。順治二年,與良玉子夢庚來降。三年,從肅親王豪格下四川,討張獻忠,授成都總兵。五年,擢四川巡撫。

獻忠既滅,其將孫可望、劉文秀等降於明,分遣所部王命臣等竄川南,譚弘、譚文、譚詣、楊展、劉惟明等竄川東,與李自成舊部郝搖旗、李來亨、袁宗第、劉二虎、邢十萬、馬超等遙為聲援。弘犯保寧,國英擊敗之。命臣據順慶,國英分兵三道,水陸並進,克其城,獲其將李先德、硃朝國等。邢十萬、馬超所據地近保寧,國英偕總兵惠應詔討之,獲其將胡敬,復潼川,逐之至綿州,獲所置吏呂濟民等。尋招惟明、展來降,遂下綿州。六年,進復安縣,克彰明,破曲山關,徇石泉。有謝光祖者,據寨抗,師行,遣兵破斬之。七年,遣副將曹純忠、劉漢臣徇川北諸郡縣,設伏擊斬寇渠老鐵匠、黃鷂子。九年,可望、文秀大舉寇保寧,橫列十五里,勢張甚。國英督兵搗其中堅,別遣兵出間道擊其後,大破之,授世職二等阿達哈哈番。

十一年,加兵部尚書。時可望等破成都,重慶、夔州、嘉定皆為明守。吳三桂、李國翰駐軍漢中,國英請敕進兵。十三年,加太子太保。十四年,擢陜西四川總督。三桂等自漢中下重慶,遂趨貴州。文、弘、詣、二虎等分屯忠州、萬縣,合軍攻重慶,總兵程廷俊、嚴自明御之,敗走。文又合十三家兵逼重慶,國英自保寧赴援,次合江,詣殺文以降。國英入城安撫,弘亦與其將郝承裔、陳達先後出降。文所部猶據涪、忠二州,國英遣總兵王明德擊破之。十七年,承裔據雅州復叛,國英督兵至嘉定,分三道進剿,破竹箐關入,承裔走黎州,追獲之。十八年,川、陜各設總督,命國英專轄四川。

康熙元年,明石泉王奉鋡攻敘州,國英討平之。時搖旗、來亨、二虎、宗第等據茅麓山,出掠四川、湖廣、陜西錯壤諸州縣。議三省合軍討之,國英疏言:「賊巢橫據險要,我師進攻,未能聯合。宜豫會師期,分道並入,使賊三路受敵,彼此不暇兼顧。一路既平,就近會師,賊可盡殲。」上命將軍穆里瑪、圖海將禁旅討之,國英與西安將軍富喀禪、副都統都敏會剿。明年,督兵進巫山,趨陳家坡,破二虎壘。二虎走死,搖旗、宗第夜遁。總兵梁加琦、佐領巴達世逐之至黃草坪,獲搖旗、宗第及所置吏洪育鰲等。又遣總兵李良楨破小尖寨,獲明東安王盛蒗,叛將賀珍子道寧以所部降。四年,疏言:「全川底定,裁留通省兵四萬五千名,以馬二、步一戰守各半定額。」從之。五年,卒,謚勤襄。七年,追敘國英功,授世職一等阿思哈尼哈番。

孫永升,襲職。雍正間,官南陽總兵。坐事戍軍臺。世宗念國英前勞,召還,洊擢至工部尚書。以永升從子時敏襲職。乾隆初,定封一等男。

劉武元,字鎮籓,漢軍鑲紅旗人,初籍遼東。仕明官游擊,佐祖大壽守大凌河,天聰五年,從大壽出降。崇德六年,授刑部參政。順治元年,改授甲喇額真,予世職三等甲喇章京。二年,授天津兵備道。三年,擢南贛巡撫。四年,遣副將劉伯祿、徐啟仁等剿捕瑞金、石城,興國、龍安,寧都、上猶諸州縣土寇,克魚骨、蓮花、丁田、鉤刀嘴諸寨,斬其渠葉南枝、劉志諭、劉飛等。

五年正月,金聲桓、王得仁以南昌叛,江西諸郡縣皆附,外連閩、粵,贛州介其間。武元召諸將歃血誓,得仁以二十萬人來攻,啟仁出降,圍合。武元城守三月,糧盡,斥家財佐軍,勵士卒奮戰,遂破得仁兵。得仁退屯東山,引武元空城出戰,將設伏邀擊。武元知其謀,天未明,兵數百持炬為前驅,得仁兵望見,伏盡出,力戰,得仁中創遁。聲桓聞我師至九江,謀退保南昌,武元出奇兵襲其後,敗之太湖港,斬獲無算。

十月,叛將李成棟復來攻,眾號百萬。武元先出兵數百撓之,夜縋城出死士劫破十餘壘,遂令諸將分兵東、西、南三門出戰,大破之,成棟以數騎走。敘功,加右都御史,兼兵部侍郎,賜紫貂冠服、甲胄、佩刀、鞍馬。六年,征南大將軍譚泰既克南昌,遣梅勒額真覺善等與武元會師,克信豐,成棟宵遁,墮水死。武元分遣副將先啟玉、參將鮑虎、游擊左雲龍等捕成棟餘黨,定瑞金、雩都、崇義諸縣。進攻梅嶺,破木城五,獲成棟將劉治國。

七年,平南王尚可喜徇廣東,師自南安入,武元遣副將慄養志以兵從,克南雄、韶州二府。又遣副將高進庫,游擊楊繼、洪起元等剿寧都土寇彭順慶,副將楊遇明、劉伯祿、賈熊、董大用等剿大庾土寇羅榮。順慶應聲桓為亂,自號軍門,窺伺郡邑;榮自明季倡亂楚、粵間,自號五軍都督,聚眾數萬,阻山結寨二十餘,四出劫掠:至是皆就戮。敘功,加太子太保、兵部尚書。遇恩詔,進世職一等阿達哈哈番又一拖沙喇哈番。十年,引疾還京。十一年,卒,贈少保,謚明靖。

瀇,武元長子,襲職。疏請追敘武元贛州全城功,進二等阿思哈尼哈番。官至副都統。

浩,武元次子。康熙間,官廣西潯州知府。孫延齡叛,城陷被戕,並及其子中樞、中梁、中柱、中楫。事聞,贈太僕卿。

庫禮,喜塔臘氏,滿洲正白旗人。太祖創業初,其四世祖昂果都理巴顏來歸。庫禮事太宗。

崇德初,徵朝鮮兵從征伐,命庫禮將其軍。五年,睿親王多爾袞等伐明,圍錦州。上遣戶部參政碩詹使朝鮮,發水師五千人、米萬斛詣大凌河,庫禮與梅勒額真洪尼哈將三十人導。六年,從鄭親王濟爾哈朗圍錦州,克其郛,斬八百餘級。復與噶布什賢噶喇依昂邦薩穆什喀攻松山北崖,庫禮以朝鮮兵二百餘先登。科爾沁部人或降於明,發砲中庫禮手,庫禮不為動,督戰益力,卒破明兵。攻松山,明兵擊正紅、鑲藍二旗分守地,庫禮與左翼將領勒卜忒擊之,明兵引卻。以功授世職牛錄章京,賚所獲牲畜。七年,擢戶部參政。

順治初,改戶部侍郎。論定都功,加半個前程。旋坐阿豫親王多鐸指,集視八旗女子,論罰鍰。二年,命如淮安總理漕儲。四年九月,鹽城土寇竊發,庫禮與漕運總督楊聲遠親往撫慰。未幾,其渠周文山等以八百人夜襲淮安,自夾城東門缺口入,攻庫禮官廨。庫禮率中軍張大治、旗鼓王國印將帳下卒數十人御之,其妻盡出廨儲矢,僕婢齎送助戰,眾皆一當百,自丑至辰,所殺傷過當。文山等潰走,逐斬百八十餘級,盡收其印劄、軍械,城賴以全。

有稱明益王者,奉唐王聿鍵隆武號,屯廟灣,有眾數千、舟百餘,將攻淮安,庫禮與聲遠等計,設伏以待。敵舟揚帆直上,至車家橋,伏發,水陸夾擊,敵死者過半,餘眾走還廟灣。固山額真張大猷、巡撫陳之龍以師從之,敵據劉莊場,為屯凡十,以次剿撫,旬日乃盡定。考滿,進三等阿達哈哈番。尋召還。

七年,致仕,復進一等阿達哈哈番加拖沙喇哈番。卒,謚僖恪。

胡全才,山西文水人。明崇禎進士,官兵部主事。順治元年,固山額真葉臣定山西,疏薦,起原官。二年,自郎中授陜西漢羌道,駐漢中。時叛將賀珍為亂,全才上官,撫綏彫瘵,安集流亡。招明將趙光遠部曲齊升、王明德、李世勛等來降,盡收其軍械,與知府楊可經等練士卒,聚芻糧為備。珍突至圍城,升等奮勇沖擊,世勛中流矢死。城守三十餘日,援師至,珍遁走,漢中得全。工部侍郎趙京仕疏言漢中重地,宜設巡撫,且薦全才才稱仕。

三年,擢寧夏巡撫。四年,疏請頒本朝律典及性理、通鑒諸書,令士子誦習。又疏言:「寧夏舊額兵三萬有奇,設總兵及中軍副將分統之。其後兵裁及半,罷中軍副將。往者總兵應徵發,叛將王元遂乘隙戕巡撫焦安民為亂。宜復舊制,廣兵額,設中軍,調徵興慶副將馬寧嘗擒斬王元,請仍補斯缺。」下部議,並如所請。元黨馬德既降復叛,全才與總兵劉芳名發兵討誅之。語詳芳名傳。是歲山、陜蝗見,全才為捕蝗法授州縣吏,蝗至,如法捕輒盡,不傷稼。因以其法上聞,命傳示諸直省。

初,全才任漢羌道時,令凡受賀珍劄付者,許自首,仍予劄付如其官。旋揭告漢羌總兵尤可望苛罰冒餉,藏匿偽官,可望即以擅給劄付訐全才,並坐罷。全才詣部自陳,部議以全才功大罪小,復除江西饒南道。

十年,經略洪承疇奏薦,令從征湖南。尋命撫治鄖陽,提督軍務。李自成將郝搖旗、劉體純等降於明,及明桂王走南徼,遂屯聚房、竹群山間為盜。全才分兵扼沖要,馳察穀城、南漳諸地形勢,檄諸將進討,戰屢勝。十三年,明桂王所置總兵李企晟入鄖陽,與搖旗等合,全才遣諸將硃光祚等密捕之,執企晟。旋擢湖廣總督,卒官,贈兵部尚書,謚勤毅。

申朝紀,漢軍鑲藍旗人,初籍遼東。天聰八年,授刑部啟心郎。文館硃延慶疏陳時事,薦朝紀溫雅正直,練達世務,處家儉,守身約,訥言敏行,足任鴻鉅。崇德元年,賜人戶、牲畜。

順治元年,授河南河北道,駐懷慶,李自成之黨二萬餘來犯,朝紀登陴守御,晝夜不少懈,有渠乘白馬薄壕,麾眾攻城,朝紀舉砲殪之,賊悉驚竄。二年,遷江南布政使,擢山西巡撫。三年,疏言:「驛遞累民,始自明季,計糧養馬,按畝役夫。臣禁革驛遞濫應、里甲私派。請飭勒石各驛,永遠遵守,俾毋蹈前轍。」又疏言:「各省驛站銀舊額十五萬有奇,明季裁充兵餉。驛費不足,輒私派於民。請敕部復原額。」又疏言:「賦役全書應裁、應留諸項,請覈實詳酌,俾有司不得私徵濫派。」疏並下部議行。四年,陽城民王希堯、賈國昌等以邪教倡亂,朝紀遣中軍都司白璧同冀南道武延祚率兵捕治,悉誅希堯、國昌等。汾州營卒李本清、任自興等據永寧銅柱寨為亂,朝紀赴汾州,遣冀寧道王昌齡等率兵捕治,獲本清等,焚其寨。寧鄉民楊春暢等復以左道據冷泉寨為亂,朝紀遣平陽副將範承宗等討平之,擢宣大山西總督。五年,卒。

延慶,漢軍鑲黃旗人。入關,官至江西巡撫。

順治間,治山、陜著績效者,又有馬之先、劉弘遇。

馬之先,漢軍鑲藍旗人,初籍金州衛。順治初,以諸生授昌平知州。四遷至湖廣布政使。七年,授江西巡撫。土寇王才據終南山肆掠,之先遣游擊陳明順等自子午鎮進剿,才竄走,敗之高關峪,又敗之化羊峪,獲才。又捕治諸盜何紫山、孫守金、唐珍玉等。十一年,自成餘黨劉二虎、郝搖旗等侵入陜西境,之先與漢興總兵趙光興發兵三道迎擊,破小廣峪寨,斬其將傅奇,遷宣大山西總督。十三年,調川陜總督,加兵部尚書,入覲,上諭之曰:「陜西天下咽喉,爾當視孟喬芳倍加勤慎,方克有濟。」十四年,卒,謚勤僖。

劉弘遇,漢軍正藍旗人,初籍遼東。與弟奇遇,並以諸生入祖大壽幕,佐軍諮。天命間,太祖伐明,次三岔河,弘遇與奇遇挈家來歸,籍明諸邊兵馬數目,並畫戰守事陳奏。上曰:「得廣寧,當官汝!」久之未用。崇德元年,上疏乞自效,命大學士範文程等試之,授弘文院副理事官。

順治元年,譯遼、金、元三史成,賜白金、鞍馬。尋授工部理事官,遷山西朔州道。二年,與副將侯大節等捕治蔣家峪、黑草嘴土寇,擢陜西布政使。五年,授安徽巡撫。金聲桓叛江西,皖北盜蜂起。弘遇如池州,分遣鎮將逐捕盜渠王貳甫等,移駐安慶,與總督馬國柱捕治英山、霍山、潛山諸盜,得其渠孔文燦等,餘盜悉平。六年,裁缺召還。

七年,授山西巡撫。時姜瓖亂初定,其黨竄匿保德、五臺、府穀諸縣山谷間。弘遇請免逋賦,甦驛困,矜恤諸死事家。又疏言:「兵後民田荒蕪殆盡,前此師討姜瓖,竭蹶供芻糧。今捕治餘寇,日需輸輓。值二麥未收,秋禾遇蝗災,農失耕時。」得旨,下所司蠲賑。又與總督佟養量、總兵剛阿泰剿五臺山寇劉永忠、高鼎,降陜西土寇楊茂。

弘遇撫山西四年,建忠烈祠祀守土諸臣死姜瓖亂者,並修太原、陽曲學宮,築汾河諸堤,山西民誦其惠。旋以捕治土寇未入奏即籍沒,給事中張璿論弘遇專擅,尋奉詔甄別督撫,弘遇左授福建督糧道。十八年,卒。

於時躍,漢軍正白旗人,初籍廣寧。順治二年,以諸生授安徽合肥知縣。尋遷河南懷慶知府。四年,擢河南道。靈寶、盧氏二縣寇發,時躍與副將寇徽音、游擊孔國養等入山捕治,破其寨,斬寇渠劉芳、張進澤、張三桂等,寇乃平。七年,遷山西按察使。時躍善聽訟,訟至即定讞,民稱之曰於不落。九年,遷山西布政使。坐在陜西薦舉屬吏失當,左遷。經略洪承疇薦其才,命赴軍前效用。尋復薦補湖廣驛鹽道。

十二年,超擢廣西巡撫。明宗人盛濃、盛添據富川,結土寇王心、蔣乾相等,■H0集瑤、僮,窺旁近郡縣。時躍會提督線國安、總兵全節討平之。十三年,明將龍韜屯柳州,時躍密約國安與定南王護衛李茹春、總兵溫如珍等,督兵攻之,陣斬韜,逐北三十餘里,餘眾悉遁。十四年,師下雲南,時躍疏請賓州設兵防守,並分屯柳州備策應,下所司議行。明桂王由榔號召諸降附土寇,假以公侯,分據郡縣:鬱林則李勝、李喬華,懷集則何奎豹、李盛功,富川、賀縣則馬寶、梁忠,南寧、太平則賀凡儀、曹友,並倚險為巢,四出侵掠。僮寇羅法達、廖仁倫等復擾臨桂、永福、荔浦、修仁諸縣。時躍親督兵捕治,所陷城邑次第克復,敘加都察院副都御史。十八年,擢廣西總督。明德陽王至濬走安南,時躍招使來降。敘功,加右都御史。康熙二年,卒。

蘇弘祖,漢軍正紅旗人,初籍遼陽。崇德三年,以舉人授戶部啟心郎,賜朝衣一襲,免丁四。八年,考滿,授世職牛錄章京。順治初,授河南河北道。累遷陜西布政使,世職累進三等阿達哈哈番。十年,坐計典失實,左授福建福寧道。十三年,遷左僉都御史。十五年,授南贛巡撫。十七年,雩都寇發,弘祖斥資造火器,遣兵搗其巢,擒其渠李玉廷。別有土寇謝上逵、羅一鑒、徐黃毛等,據廣東平遠五指石,界連閩、贛。弘祖發兵討之,上逵詐降,潛走匿紅畬。弘祖遣將李宗韜以計擒斬一鑒、黃毛等七人,夜進兵,逐賊至柑子窩中木溪,毀五指石寨,攻紅畬,賊縛上逵獻,斬之。十八年,遣游擊王把什捕治廣昌土寇,乘雨攻不備,破滴水、羊石二寨,斬千餘級,擒其渠幸連升、蕭來信。康熙元年,甄別督撫,弘祖解任。三年,卒。

吳景道,漢軍正黃旗人,初籍遼東廣寧衛。天聰間,授吏部啟心郎。崇德元年,改都察院理事官。疏劾刑部理事官郎位貪污不法狀,鞫實,黜郎位,追贓貸死。郎位銜景道甚,誘都察院筆帖式李民表與同居,訐景道,鞫虛,民表坐誅,籍郎位半產。景道以不察民表違禁移居他旗,罰如例。景道疏論睿親王多爾袞專擅,坐奪官。

順治二年,起授河南布政使,擢巡撫。時河北初定,河南五府餘寇未靖。寶豐宋養氣、新野陳蛟、商城黃景運等各聚數千人,侵掠城邑。景道檄總兵高第、副將沈朝華等分道捕治,誅養氣等。四年,鄖陽土寇王光泰率千餘人犯淅川,景道遣參將尤見等與總兵張應祥合兵擊卻之。五年,羅山土寇張其倫據雞籠山寨,出掠,景道遣都司硃國強、佟文煥等督兵討之,破寨,擒戮其倫,並其黨硃智明、趙虎山等。曹縣土寇範慎行等煽寧陵、商丘、考城、虞城、儀封、蘭陽、祥符、封丘諸縣土寇,並起為盜,屯黃河北岸。景道檄第督兵討之,寇退保長垣,第以師從之,寇走蘭陽。景道遣文煥督兵追擊,斬千餘級。薄曹縣,寇列柵拒守。景道檄總兵孔希貴自衛輝道肥城,斷寇東走路。游擊趙世泰、都司韓進等率精騎分道夾擊,戰於東明,殲寇數千,獲慎行誅之,餘眾悉潰散。敘功,加兵部侍郎。七年,進尚書。八年,商州土寇何紫山等掠盧氏,夜襲世泰營,第督兵扼擊,走商南。景道檄應祥督兵討之,寇盡殲。九年,以塞汴河決口,與河道總督楊方興同賜鞍馬、冠服。十年,以老疾乞休。十三年,卒,贈太子太保,謚愨僖。

李日芃,漢軍正藍旗人,初籍遼陽。太宗時,命以諸生入內院理事,賜五戶。順治元年,授永平知府。三年,遷霸州兵備道。授知州張儒策,諭降土寇李振宇等數百人,擢僉都御史。四年,加右副都御史,授操江巡撫。金聲桓以江西叛,日芃親督兵屯小孤山磨盤洲,令同知趙廷臣、參將汪義、游擊袁誠等迎擊。五年,戰於彭澤,得舟二十餘,寇中砲及溺死者無算。六年,裁安徽巡撫,命日芃攝其事。土寇餘尚鑒挾明宗室統錡■H0聲桓餘黨據險為二十餘寨,掠桐城、潛山、太湖諸縣。日芃遣副將梁大用等督兵討之,克皖澗寨,進圍飛旗寨,斷水道,分兵四路合擊,拔之。又破桃圍等寨,擒戮統錡、尚鑒,餘大小和山等十八寨皆降。九年,加兵部侍郎。十年,討平徽州赤嶺土寇張惟良。十一年,甄別直省督撫,加兵部尚書。明將張名振屢自海入江犯鎮江、瓜州,劫漕艘。日芃令於鎮江檀家洲測江水,淺則植椿,深則編筏,環以鐵索,阻來舟。兩岸置砲,南自鎮江至圌山,北自瓜洲至三江口,建新堤,設木橋,通巡兵往來。令圌山、瓜洲等四營守備更番督水師防禦。五里置一汛,譏察詳密。諸寇匿江為藪,俘斬略盡。十二年,加太子太保。旋卒,謚忠敏。

劉清泰,漢軍正紅旗人,初籍遼陽,名朝卿,以諸生歸太宗,賜今名。崇德六年,試一等,入內院辦事。順治二年,擢弘文院學士。九年,充會試副考官。授浙江福建總督。

時鄭成功據廈門,陷漳浦、海澄、南靖諸縣,上命其父芝龍作書,敕清泰諭降。十年二月,清泰疏劾巡撫張學聖、巡道黃澍、總兵馬得功前此偵成功赴粵,潛襲廈門,攫其家貲,致成功修怨,連陷城邑,學聖等並坐黜。三月,清泰得成功報芝龍書,略言就撫後,原得浙東、嶺南地駐兵。清泰疏上聞,並論成功語浮言誇,議撫當詳慎,上嘉其遠慮。五月,平南將軍金礪攻海澄,以餉不繼,還軍漳浦。會上敕封成功海澄公,畀以泉、漳、惠、潮四郡地,遂罷兵。清泰請駐軍浦城備不虞,從之。十一年,疏言:「成功雖降,不薙發,其黨逞掠如故,降無實意。宜發禁旅赴福建,駐要地,資策應。」下諸王大臣議。清泰旋以病乞假,還駐杭州。成功發兵攻陷漳、泉,上授鄭親王世子濟度為定遠大將軍,率師討之。左都御史龔鼎孳疏劾清泰當金礪攻海澄,不能同心合力,及招撫未定,又不控扼險要,致海疆被陷,坐奪官。

十八年,聖祖即位,起秘書院學士,授河南總督。康熙三年,以報墾荒地萬餘頃,加兵部尚書。四年,以疾致仕。卒。

佟岱,漢軍正藍旗人,先世居佟佳。父佟三,歸太祖,任梅勒額真。佟岱與兄養量同授牛錄額真。養量順治初官至宣大總督,駐陽和,有惠於民。佟岱崇德元年從伐朝鮮,以縱掠降民坐死,命奪官,罰鍰以贖。三年,授吏部副理事官,兼甲喇額真。六年,師圍錦州,七年,攻塔山、杏山,皆在行,擢正藍旗漢軍梅勒額真。八年,從克前屯衛、中後所,予世職牛錄章京。

順治元年,從克太原。二年,從討李自成,師自陜西徇湖廣,遂下江南。與總兵金聲桓駐守九江,定南康、南昌、瑞州、袁州諸府,以所俘獲奏聞。因疏言:「故明鍾祥王慈若等衰殘廢棄,或存其餘喘,彰我朝浩蕩之仁。」得旨:「故明諸王赴京朝見。」旋令攝湖廣總督。三年,還京,授兵部侍郎。復從征湖南,自岳州進長沙,戰衡州,克寶慶、武岡。六年,復從討姜瓖,拔渾源、左衛、朔州、汾州、太谷諸城。世職累進一等阿達哈哈番兼拖沙喇哈番,歷戶、吏諸部。

十一年,代清泰為浙江福建總督。疏請申海禁,斷接濟,片帆不得出海,違者罪至死。十二年,成功陷舟山,十三年,復陷臺州。佟岱與巡撫秦世禎不協,互劾。上為移世禎操江巡撫,召佟岱還京,以李率泰代。佟岱不即行,復疏自敘剿撫功,上責其冒功戀祿,下李率泰等按狀,奪官,留軍功三等阿達哈哈番。卒。

秦世禎,漢軍正藍旗人,初籍廣寧。順治二年,以貢生除直隸文安知縣。三年,行取授御史,疏請畫一各省裁免賦役,從之。四年,巡按浙江。八年,甄別臺員,列一等。尋命巡按江南。世禎察淮、揚各郡蠹役害民,嚴治其罪。徒黨聚盟,仇訴告者,世禎執為首者系之獄,疏上其事,並言懲蠹於事後,不若使不為蠹。請飭督撫以下至州縣,毋於經制外濫設胥役,並定年限,毋令久充,上從之。

時方大兵後,田畝淆亂,官為丈量,胥役因緣為奸。世禎令編列「魚鱗冊」,使民自丈量,贏縮胥復其舊,荒坍皆有別。州縣徵賦,民或逾額輸納,世禎限夏稅五月,秋糧九月,先給「易知單」,示以科則定數。又令每甲匯列賦額及輸戶為「滾單」,使里長按戶遞傳,輸賦則填注。先行之蘇州,民以為便,條列以聞,通行諸府。又以徵銀設櫃,有司奉行不實,請增司府印封,立日收簿,輸戶自封投櫃,驗數書之簿。又請革僉點糧長之例,改官收官兌。並下部,著為令。巡撫土國寶貪酷病民,以世禎劾,罷。

十年,還京,遷大理寺丞。十一年,擢浙江巡撫,疏請增造戰艦,精選水師;別疏言沿海漁舟,往往通寇,請按保甲法,以二十五舟為一隊,無事聽採捕,有事助守御:並議行。十二年,與佟岱互劾,調操江巡撫,解佟岱任,命暫管總督事。尋以李率泰等疏論成功陷舟山,世禎不能辭咎,與佟岱並奪官。卒。

陳錦,字天章,漢軍正藍旗人,初籍錦州。仕明官大凌河都司,崇德間來降,予世職牛錄章京,加半個前程。漢軍旗制定,授牛錄額真。

順治元年,自內院副理事官授登萊巡撫。青州土寇楊威、秦尚行結明將劉澤清為亂,錦遣兵討平之。二年,土寇張廣焚掠掖、濰諸縣,遣兵擊敗之。廣降於澤清,復寇平度,犯萊州,錦遣兵捕治,授策設伏徐家甿,射殺廣,盡殲其眾。擢操江總督,與招撫大學士洪承疇並駐江寧。三年,明瑞昌王誼石等密結城人為亂,錦與承疇詗知之,閉城捕治諸為亂者。誼石以兵至,擊破之。四年,疏言:「圌山為鎮江咽喉,江寧門戶,宜建立砲臺,置兵備。江北要口設臺亦如之。兩岸兵船接哨分防,沿江設烽墩,使聲勢相通。」章下部議行。

遷浙江福建總督。鄭成功為寇,據延平將軍寨,地高險,俯瞰諸縣,攻不能破。錦命壘土高與寨等,乘以登陴,遂克之。歲大饑,錦遣兵次第收復,撫輯流亡,民賴以安。五年,成功將鄭彩以舟師入據長樂、連江諸縣,錦與靖南將軍陳泰等分兵收復。師進次興化,斬成功將顧世臣等十一人。六年,遣總兵張應夢、馬得功等復羅源、永春、德化、福安諸城。江西山寇侵延平,陷大田、尤溪,錦遣兵收復,獲明新建王由模等。七年,疏請進攻舟山。八年,錦與固山額真金礪、劉之源,提督田雄等會師,以大艦隨潮出,敗明兵於橫洋,獲其將阮進;乘霧攻舟山,明魯王以海出走,遂克之,隳其城,置定關總兵,駐師守焉。九年,成功寇漳浦、平和,錦督兵赴援,戰江東橋,敗績,左次同安,賊夜入其帳,刺中要害,遂卒,贈兵部尚書。

論曰:國初民志未壹,諸依山海險岨而起者,往往自託於明遺,要之為民害,廓清摧陷,封疆之責也。國英定四川,合師討茅麓山,績最高。武元守贛州,庫禮守淮安,全才守漢中,御寇全城,亦其亞也。朝紀等捕治土寇,皆能勤其官者。若清泰策鄭成功,謂挾怨而叛,殊不中事理。錦屢勝而挫,遽為何人所賊,防衛亦稍疏矣。


\end{pinyinscope}