\article{列傳二百七十四}

\begin{pinyinscope}
忠義一

特音珠阿巴泰固山僧錫等納密達炳圖等書寧阿感濟泰等

穆護薩覺羅蘭泰等索爾和諾齋薩穆等席爾泰滿達理

卓納納海覺羅鄂博惠覺羅阿賚等同阿爾

董廷元弟廷儒廷柏常鼎白忠順等格布庫阿爾津等

濟三瑚密色等敦達裡安達里許友信成升等

清天命、天聰年間,明御史張銓,監軍道張春,均以被擒不屈,聽其自盡,載諸實錄,風厲天下。厥後以明臣來歸者,有功亦入貳臣傳;死軍事之尤烈者,於京師祀昭忠祠:褒貶嚴矣。文武一二品以上,既入大臣傳,以下則另編忠義傳,列翰林院職掌,凡自一二品以下,或死守土,或死臨陣,備載出身、官階、殉難時地,及予謚、建祠、贈官、廕後。二百數十年,綜八千餘人,略以類別。

入關之先,如降服烏喇、哈達、索倫、葉赫諸部落為特音珠等二十人是。征朝鮮則勞漢等十人是。其伐明也,自天命三年至崇德八年,始克撫順,屢偪近畿,分下山西、山東諸郡縣,尤以沈陽、大凌河、皮島、松山數役為大,為西佛萊百六十二人是。

順治元年,定鼎燕京,後追擊流賊、奠定各省者,為恩克伊等一千二百四十五人。

康熙朝,討平逆籓及殲滅附逆諸鎮將,為索諾穆等九百四十七人。親征噶爾丹之役,為富成額等百人。厄魯特之役,為諾里爾達等五十五人。羅剎、西藏諸役,為紐默淳等七人。平各省土賊及海寇、苗、瑤諸役,先後為郝爾德等二百八十五人。

雍正朝,承康熙征厄魯特之役,用兵準噶爾,為和溥等三百六十二人。其先青海之役,為姬登第等十四人。外則滇、黔、蜀、桂土司苗亂與夫臺灣土番等役,為劉洪度等二百十三人。

乾隆朝,始蕩平準部,旋戡定回疆,則為傅澤布等五百十二人。初年,湖南苗亂,為李如松等十五人。廣西土賊,為倪國正等十人。瞻對土司之亂,為陳文華等十三人。隨傅清、拉布敦同死西藏,為策塔爾等六人。金川用兵,其初定也,為楊先春等百又四人;其再定也,為占闢納等八百五人。緬甸用兵,為馬成龍等百六十七人。安南用兵,為英林等百六十人。廓爾喀用兵,為索多爾凱等七十六人。逆回蘇十三、田五之亂,為新柱等百又十人。山東王倫之亂,有音濟圖等十八人。臺灣林爽文及陳周全之亂,有耿世文等百五十九人。黔、楚等省苗亂及川、楚、陜三省教匪,均始乾隆末年,而定於嘉慶,苗亂有六達色等二百七十八人;教匪之亂,為楊治寧等七百四十二人。仲苗滋事,為胡慶遠等百十三人。閩、粵洋面蔡牽之亂,為陳名魁等六十七人。先後以巡洋遇風死者,為黃勇等十七人。滑縣李文成之亂,為強克捷等六十三人。追剿陜匪及瞻對永北夷匪等役,為馬魁等十四人。馬營壩搶險者盧順。

道光重定回疆一役,為劉發恆等二百六人。江華瑤滋事,為馬韜等五人。陜、甘番滋事,為胡文秀等十三人。雲南永昌回匪滋事,為硃日恭等九人。臺灣嘉義土匪,為方振聲等七人。山西曹順之亂,為楊延亮等。英吉利開釁,為硃貴等八十八人。發匪之亂,熾於咸豐而殄於同治,其先為廣西會匪,始道光季年,為王叔元等五十一人。已而竄陷各省,為褚汝航等五百七十九人。捻匪之亂,為龍汝元等七十八人。

咸豐、同治之交,滇匪滋事,為林廷禧等四十二人。

同治朝,甘肅回匪滋事,為訥勒和春等三十七人。

其自嘉慶迄光緒先後剿辦各省匪徒等役,為和致等三十八人。咸豐換約起釁,殉澱園者,為覺羅貴倫、玉潤等。辦匪而以勞卒者,為李文安等十三人。蓋原傳可數者如此。中以不從尚之信叛而死之金光,私家傳述,心跡殊異,則出以存疑。

將帥之死事者,既有專傳,凡上列諸人之義烈尤著者,與夫官書既漏而不能無紀載者,則別編為傳,觕見本末。若夫道光以後死於外釁,及光緒庚子拳亂,宣統辛亥革命,於義宜詳,並備列之,用資後鑒云。

特音珠,滿洲鑲藍旗人,姓完顏。清初,偕阿巴泰來歸。阿巴泰,姓覺爾察,屬滿洲正白旗。太祖始編佐領,以特音珠兼管六佐領事,設札爾固齊十人,阿巴泰預焉。乙未年,特音珠從額駙揚古利征輝發部,奪塔思哈橋,掌纛者中砲僕,佐領五岱代舉之。薄城,為飛石所中,與額駙托柏、佐領和羅俱歿於陣。特音珠先登,克其多璧城。己亥年,從征哈達,城上矢石如雨,佐領耶陳奮勇登,被戕,特音珠在事有功。庚戌年,阿巴泰從內大臣額亦都招撫東海窩集部之那木都祿、綏芬、寧古塔、尼瑪察四路,降其長康古哩等。復取雅蘭路,阿巴泰力戰陣亡。

辛亥年,特音珠從揚古利攻呼爾哈路札庫塔城,三等侍衛貝和,佐領貴三、松阿里戰歿,特音珠負創,戰益力。三等侍衛阿達海先登,克其城,阿達海,額亦都第五子也。癸丑年,烏拉貝勒布占泰負恩叛,大兵討之,布占泰率兵三萬由富哈城而東,特音珠、阿達海率護衛業中額等邀擊之。阿達海、業中額及閒散米拉渾均歿於陣,大兵敗布占泰,遂平烏拉,特音珠尋以創發卒。征烏拉之役,死事者有阿蘭珠、納蘭察,均自有傳。

固山,滿洲正黃旗人,姓哲爾德,世居界凡。初任佐領,天聰三年,徵明,固山偕驍騎校僧錫、閒散達蘭從揚古利為前哨,攻永平、遵化。達蘭先登,圍明都,固山步戰大紅門,上下高坡,騰躍如飛,明兵奪氣,涿州援兵至,敗之。崇德元年,復隨揚古利征明,攻順義,僧錫先登。十二月,太宗親征朝鮮,豫親王多鐸等先驅,圍其國都,固山等從,屢斬馘。朝鮮國王李倧遁南漢,追圍之。太宗至臨津江,冬暖冰泮,多鐸令僧錫等潛測江水,欲浮馬以濟。僧錫等夜至,大風,冰復堅,還報,大軍安驅而渡,抵南漢山城西。二年正月,全羅、忠清兩道巡撫、總兵來援,多鐸與揚古利迎戰,揚古利率僧錫冒霧馳擊,援兵敗走。復依山列陣,矢石如雨,僧錫與雲騎尉鄂海,參領特穆爾,佐領弼雅達、阿紐、都敏俱力戰,歿於陣。

進偪山頂敵營,敵兵棄馬遁。達蘭率二十人乘夜用雲梯襲南漢山城,先登,中槍卒。又命分兵攻江華島,將渡江,敵船百餘,分兩翼以拒,舟師從中沖入。固山手發紅衣砲,皆敗竄,既登岸,鳥槍手千人,復列岸以拒,固山力戰陣亡。大兵繼進,盡殲其岸兵,遂克江華島。李倧降,朝鮮以定。

納密達,滿洲鑲白旗人,姓索綽羅,世居吉林。天聰八年,從大兵征明,攻雄縣,梯城首登。崇德元年,親征朝鮮,明總兵沈世魁、副總兵金日觀駐皮島,為朝鮮援。納密達偕閒散扈習從攻南漢城,有功。二年正月,朝鮮降,世魁等不能救。先是明帥毛文龍據皮島,欲牽掣我師。既而文龍為巡撫袁崇煥所殺,世魁代領其眾,失士卒心,勢益弱,猶乘間擾邊。

三月,命武英郡王阿濟格、貝子碩託,率恭順王孔有德、智順王尚可喜等攻皮島,以納密達及護軍參領炳圖為前隊,佐領巴雅爾圖、武爾格以大臣子弟從征。巴雅爾圖,額駙揚古利之從子;武爾格,弘毅公額亦都之孫,內大臣圖爾格之子也。師攻鐵山,頭等侍衛拜音臺柱、佐領珠三先登,克之。世魁遁入石城。

四月,阿濟格令納密達等乘小舟攻皮島西北隅,日觀列兵堡上。沖入,將及岸,巴雅爾圖、武爾格躍登,明人闢易,納密達、炳圖並登,而後隊金玉和等不進。日觀見師少,復進戰,武爾格陣亡。納密達等往來沖突,拜音臺柱、珠三及護軍校彰吉泰急棹小舟登岸援之,明入空城出戰,納密達、巴雅爾圖、炳圖、拜音臺柱、珠三、彰吉泰並戰歿。有德等乘巨艦攻東北隅,日觀殊死鬥,有德等部將洪文魁等多戰死,阿濟格麾八旗騎兵蹴之,護軍參領瑚什、雲騎尉果科暨扈習奮勇先入,歿於陣,大兵繼之,陣斬日觀,追擊世魁,戮之。是役也,敗明兵一萬七千有奇,俘三千餘,自是明不復守皮島。

書寧阿,滿洲正黃旗人,姓札庫塔。崇德三年八月,命睿親王多爾袞統左翼,貝勒岳託統右翼,分道徵明。書寧阿以佐領偕騎都尉感濟泰、參領扈敏屬右翼。九月,攻墻子嶺,感濟泰力戰,歿於陣。師入青山口,攻豐順護軍校扈護、巴雅拉,攻靈壽閒散噶普碩,攻南皮騎都尉阿延圖,攻深州閒散巴林,均戰歿。岳託攻欒城,明督師盧象升來援,書寧阿乘其未至,麾眾薄其城,護衛袞布躍登城樓,火藥發,焚死。書寧阿復沖入,克其城。轉戰,下慶都,奮勇陷陣,被戕。

十二月,兩翼連營大戰鉅鹿之賈莊,象升戰死。於是分徇山東,四年正月,左翼克濟南,右翼分兵略地,破茌平護軍三晉、破臨清佐領花應春、破館陶佐領佟桂、破濟寧佐領祖大春、破鄒縣佐領尚安福、破滕縣騎都尉傅察,俱歿於陣。二月,大軍還,扈敏復攻破首陽及順德,負重傷,戰益力。還至永平,與佐領巴海、烏納海俱遇伏,死之。騎都尉阿爾休隨大軍同徇山東,克濟南,復從承政索海征索倫,陣亡。

穆護薩,滿洲正黃旗人,姓賴布,世居佛阿拉。崇德五年,以武備院卿從大兵征明,距錦州城五里列陣,以砲攻城北晾馬臺,克之。七月,睿親王多爾袞遣卒刈城西北禾稼,明兵突出,槍砲並施,穆護薩與護軍參領覺羅蘭泰、署護軍參領溫察力戰,明兵大潰,追至壕,掩殺之,克臺九,及小凌河西岸臺二。錦州外城蒙古貝勒諾木齊等見大兵困城,志必得,謀來降,遂持書縋城下,約內應。信洩,大兵至,明總兵祖大壽出拒戰,城內蒙古縋繩,前隊援之以登,吹角夾攻,穆護薩躍上,被創卒。覺羅蘭泰、參領宏科俱陣歿。鏖戰久,明師退守內城,大兵遂入外城。

明年五月,明總督洪承疇率六總兵兵六萬來援,屯松山北岡,擊斬其二千,敵勢猶勁,騎都尉旦岱、參領彰庫善、三等侍衛博朔岱陷陣死。八月,大軍駐松山、杏山間,立營截大路。承疇率馬步兵十三萬,營松山城北亂峰岡,旋犯汛地。閒散輝蘭同參領囊古擊卻之。參領阿福尼越眾沖突,負重傷,猶斬將奪幟,諸軍繼之,敵奔塔山,遂進兵松山城外。十二月,承疇以兵六千夜至,輝蘭奮殺,既出,復進擊,與溫察、啟心郎邁圖皆歿,復沿壕射擊,殺四百餘人,敵退入松山城。

圍既合,明總兵曹變蛟欲突圍出,至正黃旗汛地,佐領彰古力戰死,變蛟亦中創奔還。七年二月,克松山,擒承疇及明巡撫邱民仰,總兵王廷臣、變蛟等。時明總兵吳三桂猶駐塔山,鄭親王濟爾哈朗率兵至城下,列紅衣砲攻之,佐領崔應泰被創死,參領邁色力戰陣亡,城壞二十餘丈,諸軍悉登,遂克塔山。先是蒙古兵有降於明者,特穆德格執而戮之,及兩師酣戰,復有訥木奇突出鵕陣,乘馬沖入多爾袞營,將行刺,特穆德格只身奮救,相抱持急,卒遇害。

索爾和諾,滿洲鑲紅旗人,姓科奇理,世居瓦爾喀。少孤,兄瑚禮納撫之,瑚禮納為仇所害,嘗手刃仇二人祭兄墓,宗黨義之。崇德三年,來歸,授佐領,從征錦州、松山,皆有功。七年十月,命饒餘貝勒阿巴泰為奉命大將軍征明,索爾和諾率驍騎校佟噶爾為前隊,次黃崖口。阿巴泰使三等輕車都尉齋薩穆,佐領綽克托,護軍多羅岱、圖爾噶圖伏隘口舉火,明兵驚潰。遂入薊州,敗明總兵白廣恩軍。齋薩穆、綽克托及佐領額貝、參領五達納、護軍校渾達善皆歿於陣。分攻霸州,多羅岱先登,攻定州,圖爾噶圖先登,俱克之,並以傷重卒。

閏十月,次河間,明分守參議趙珽、知府顏允紹城守。既進攻,允紹發砲拒擊,參領署都統陳維道陣亡。砲裂,毀城堞,護軍薩爾納冒火躍上,明兵死鬥,被戕。允紹完堞拒守,馳檄四出請援,阿巴泰連營圍之。時明於山海關內外分設總督,復設昌平、保定二總督,又有寧遠、永平、順天、密雲、天津、保定六巡撫,寧遠、山海、中協、西協、昌平、通州、天津、保定八總兵,皆擁兵壁旁縣,懾不敢近。索爾和諾曰:「河間不下者,恃外援也。破其一營,皆瓦解矣。」阿巴泰從之,遣將襲明總兵薛敵忠營,敵忠遁,諸援師悉潰。使人諭速降,允紹等守益力,急攻之,索爾和諾梯登,師繼進,破其城。珽、允紹並死,索爾和諾亦戰歿。

十二月,大兵徇山東,諸州縣各設城守,攻臨清閒散瑚通格,攻泗水護軍校務珠克圖,攻新泰閒散特庫殷,攻冠縣閒散特穆慎,攻館陶閒散東阿,攻滕縣閒散赫圖、富義,攻郯縣閒散貴穆臣,攻費縣閒散索羅岱,攻兗州佟噶爾及驍騎尉屯岱,皆戰死。諸州縣皆下,乘勝至海州,八年五月,旋師。

席爾泰,姓棟鄂。父綸布,清初,率四百人來歸,賜名普克素,編佐領,使席爾泰統之。有功,授世職,在十六大臣之列。時明總兵毛文龍籠絡遼陽沿海居民,踞皮島為重鎮,時窺邊界。鎮江城中軍陳良策潛通文龍,令別堡之民詐稱文龍兵至,大譟,城中驚擾。良策乘亂城守,席爾泰偕同族佐領格朗擊卻之。後復偕格朗從攻沈陽,陣亡於渾河。其妻嘗違禁屠馬祭夫,例當死,削世職,原之。

時戰渾河者為滿達理。滿達理,正黃旗人,姓納蘭,世居布顏舒護魯。任佐領,隨揚古利軍沈陽。明兵二萬渡渾河來援,長矛大刀,鎧冒重棉,氣甚銳。參領西佛先歿於陣,滿達理繼進,敗之。明總兵李秉誠率三千人守奉集堡,效死者無算,卒大創之,遂克沈陽。滿達理以先登功最,隨攻遼陽,明經略袁應泰急注太子河於隍,閉西閘,環城列守,大兵軍其城東南,秉誠暨總兵侯世祿以兵五萬背城五里而陣;擊走世祿,奪橋,從小西門緣梯登城,遂拔之,旋歿於陣。

卓納,姓納喇氏,滿洲鑲藍旗人,哈達貝勒萬之孫。太祖時來歸,授佐領,賜姓覺羅。天聰五年,徵明,圍大凌河城。明監軍道張春,總兵吳襄、宋緯等率馬步兵四萬自錦州來援,副都統綽和諾冒砲矢力戰,殞於陣。備禦多貝先戰歿,卓納繼之。時襄兵先敗,逐北三十餘里。張春復收潰眾立營,風起,黑雲見,春大縱火,風順火熾,卓納益銳進,與管武備院事達穆布、二等輕車都尉硃三、佐領拜桑武、騎都尉尼馬禪、護軍校愛賽、雲騎尉瓦爾喀均戰死。天忽雨,反風,大軍乘之,緯敗走,生擒春。

信勇公費英東子納海亦於是役被創,齒落其三,復從舟師攻旅順。明總兵黃龍御甚力,納海與參領岳樂順、護軍校額德、千總程國輔、騎都尉塔納喀等奮勇登城,冒矢石而殞,遂克旅順口。

覺羅鄂博惠,興祖玄孫,隸鑲紅旗;阿賚,景祖曾孫,隸鑲黃旗:並為佐領,隨征有功。天聰三年,大兵征明,並從貝勒岳託克大安口。抵遵化,明巡撫王元雅嬰城守。命分旗環攻之,鑲紅旗西之東,鑲黃旗西之南,各分領前隊,與正藍、正黃、正白各旗兵並進,城上矢石如雨,乘護軍校阿海躍登,急攻之,克其城。大貝勒代善率護軍及火器營至薊,沖明山海關援兵,阿賚死之。趨永平沙庫山,鄂博惠中創歿。

雍貴,隸正白旗。崇德三年,從睿親王多爾袞徵明,下山東。四年,師旋,敗通州河岸兵。五年,從圍錦州,敗松山兵,破杏山援兵,皆有功。七年,復圍錦州,同覺羅薩哈連等直前沖陣,大敗其眾。明總督洪承疇以十三萬眾來援,薩哈連戰歿,雍貴同護軍統領伊爾德連敗之,乘雨偪松山,擊走其馬軍,復率本旗兵攻塔山。明總兵曹變蛟夜犯鑲黃旗汛地,復隨伊爾德擊走之。八年九月,隨鄭親王濟爾哈朗征寧遠,抵中後所,偕護軍參領額爾碧沖入敵陣,拔其城。十月,進攻前屯衛,以第五人登,中砲歿。大兵繼進,遂克之。

登西克,隸鑲黃旗。官散秩大臣。順治二年,隨揚威大將軍豫親王多鐸追流賊李自成至西安,激戰於天沙山,中槍陣亡。

阿克善,景祖兄索長阿三世孫,隸正黃旗。隨大兵征明於錦州、寧遠及入關擊李自成,皆有功。歷官至兵部侍郎。順治九年,同都統噶達渾征剿鄂爾多斯部叛逃蒙古多爾濟等,殲之賀蘭山,以失究興安總兵任珍家屬淫亂、擅殺多人事解兵部,管副都統事。十一年,隨征湖廣,敗賊兵於湘潭、常德、龍陽等處。十三年,鄭親王世子濟度徵海賊鄭成功,阿克善率兵從大軍至烏龍江,以水險難渡,乃潛取道山間,徑趨福州。未至,聞成功在高齊,即分兵令佐領褚庫等先往迎戰,擊走之。又分遣署護軍統領伊色克圖往侯官,徵剿水路賊,遂抵福州。又偵知賊船三百餘尚泊烏龍江,親督水路,約營總星鼐等在陸路合擊,追至三江口,斬偽都督總兵等,俘獲甚眾。以賊犯羅源,駐防兵被圍,率兵赴援,力戰陣亡。

薩克素,隸鑲藍旗。康熙十三年,以佐領從平南大將軍賚塔征耿精忠。賚塔駐衢州,遣防臺州黃巖縣。賊將曾養性率眾六萬來犯,堅守,攻不能下。參將武灝通賊獻城,薩克素力戰,死之。

星德,隸鑲紅旗。亦以耿精忠叛,從江寧將軍額楚討之於江西建昌,敗賊帥邵連登八萬餘眾,在事有功。後於十六年攻吉安,與賊將馬寶戰於陳岡山,陣歿。

果和里,隸鑲黃旗。以委署參領隨平遠定寇大將軍安親王岳樂征吳三桂,戰於湖廣瀏陽,陣歿。

努赫勒,隸鑲黃旗。以一等侍衛從征三桂。十七年六月,三桂遣其黨江義、巴養元、杜輝等率二萬餘賊,駕巨艦二百餘,乘風犯柳林嘴。努赫勒隨討逆將軍鄂訥率水師,棹輕舟,飛越賊艦,發砲擊之,溺死無算。賊退犯君山,又以舟師進擊,追至湘陰。十九年,隨固山貝子彰泰復遵義、安順、石阡、思南等府,追剿至鐵索橋。偽總統高起隆、夏國相等擁眾二萬餘屯平遠,與江西坡賊相犄角。大兵分道進剿,努赫勒從擊平遠西南山賊,力戰陣歿。

海蘭,隸正白旗。由侍衛擢副都統。雍正七年,授參贊大臣,從靖邊大將軍、公傅爾丹征準噶爾。九年六月,分三隊渡科卜多河,與蒙古副都統常祿皆列後隊。初戰庫列圖嶺,旋移營和通呼爾哈諾爾。海蘭與常祿據山梁之東,殺賊千餘。適大風,雨雹,師被圍,常祿陣亡。海蘭突圍出,殺賊五百餘,卒以察哈爾兵潰,海蘭死之。

同阿爾,蒙古鑲紅旗人,世居巴林,以地為氏。授驍騎尉。崇德三年,多羅貝勒岳託徵明,同阿爾與焉。當師之出邊也,副都統席喇命率護軍防守七晝夜,敗敵者再。六年五月,隨睿親王多爾袞圍錦州,明總督洪承疇率重兵來援,以步兵三營犯左翼三旗,護軍不能勝,奔壕塹。同阿爾偕同旗同族僧格,及蒙古鑲紅旗人阿桑布嚴守汛地,奮勇戰死。蒙古正紅旗拜渾岱、正黃旗阿布喇庫、鑲黃旗布齋,均先後歿於陣。

董廷元,正白旗漢軍。與弟廷儒、廷柏並以閒散從征。天命六年,兵攻沈陽,廷元先登陷陣,授寬甸守備。從攻大凌河、察哈爾、旅順口、江華島,皆有功。崇德二年,從恭順王孔有德征皮島,明總兵沈世魁陣海口。廷元以小舟從北沖入,明兵砲碎之,與家丁六人歿於海。

廷儒積功為大同守備。順治五年,大同總兵姜瓖謀叛,以廷儒勇略過人,為士卒愛憚,佯以宴射誘至署,諷以同叛。廷儒以嚴詞斥之,不聽,即拔佩刀與鬥,賊群執之,罵不絕口,剖其腹,支解之,並其子開國,男婦二十七人俱被害。

廷柏初任驍騎尉。崇德五年,從征明,同參領孫有光敗松山步兵、杏山騎兵、閭洪山守兵。明兵夜犯填塹,手發紅衣砲擊卻。隨攻塔山及前屯衛、中後所等城,均以紅衣砲克之,績稱最。順治二年,從豫親王多鐸南征,破流賊,定河南,克揚州、嘉興等處,俱在事有功。時明魯王硃以海據紹興,大兵營錢塘江上。明督師大學士張國維以兵九千人乘夜劫營,廷柏從都統吳守進敗之。後從鄭親王濟爾哈朗征湖廣,明總督何騰蛟招流賊,連營拒敵。從副都統金維城率兵至馬河,力戰,歿於陣。

常鼎,鑲紅旗漢軍。順治元年,以副將隨懷慶總兵金玉和討流寇。李自成之西竄也,英親王阿濟格由邊外趨延綏,斷其歸路。至望都,佐領劄圖被創卒。入陜至延安府,虛銜章京哈爾漢率甲士守南山,力戰死。侍衛察瑪海、騎都尉嘉龍阿、參領折爾特、護軍校朔瑪,俱以陣亡。餘黨二萬餘,散在河南。圍濟源,攻孟縣,蔓延鄧州、內鄉縣及清化鎮。鼎隨玉和援濟源,至則城已陷,戕典史李應選。鼎夜半遇賊,力戰,與玉和俱陣歿。玉和自有傳。

時懷慶鎮標同死者,守備則白忠順、佘國諫、陳應傑、石斗耀、康虎,千總則宋國俊、趙國相、李中、王國臣、楊虎、劉奉相、高友才,把總則張進仁、張光裕、陳廷機、張景泰、許養和、黨中直、廖得仁、薛貴等。賊旋圍孟縣,知縣王曰俞、參將陳國才嬰城守。賊攻七晝夜不能下,將引去,會大雨,城壞,賊入。曰俞、國才率兵巷戰,國才被戕,執曰俞,脅降,不屈死。賊又圍鄧州,道標中軍鄭國泰戰死。大兵救鄧州,賊解圍去。轉攻內鄉縣,執知縣胡養素,索金帛,不應,死之。賊分兵犯清化鎮,署同知史燦麟蒞任甫兩月,執法嚴,奸民憾之,引賊入,執燦麟,怒罵不屈,賊忿,磔其尸,妻高氏及婢僕同殉。

嗣後土賊創亂者二年,有輝縣寇,據北山大伍穀諸險,列三十一寨。官兵仰攻,賊以死據,不克登,久之,乞降。官兵防其他逸也,把總田貴、羅思明守寨口,賊乘夜斫寨門遁。貴與思明倉卒出鬥,皆遇害。五年,寇起武陟之寧郭驛,驛接太行山,為盜藪,舊設捕盜通判駐其地。賊偽稱獵者,馳入驛西郭門,騎百餘,披甲持刀仗,焚劫。入通判張可舉署,可舉力鬥,遇害。十四年,睢州賊婁三嘯聚沙窩,乘夜登郾城縣城,開北門,引眾入。知縣荊其惇督家丁眾役守庫印,力禦之,受刃傷,會典史樊世亨率牌甲奔救,賊乃遁,其惇創重死,庫印卒無失。

格布庫,滿洲正白旗人,姓伊爾根覺羅,世居雅爾虎。順治元年,以參領從睿親王多爾袞剿流賊李自成,追至慶都。復隨英親王阿濟格、貝勒尼堪敗之。三年,肅親王豪格征流賊張獻忠於蜀,格布庫及參領西特庫,隊長古朗阿、巴揚阿、烏巴什隨焉。獻忠遣賊黨環營抵抗,格布庫破賊第一營步兵。賊分兩翼,豪格復遣偕佐領蘇拜攻右翼,都統準塔巴圖魯攻左翼。賊自右翼下山來犯,格布庫率本旗兵沖擊之,旋從準塔翦其左翼。賊圍正藍旗兵,格布庫偕佐領阿爾津、噶達渾、西特庫、烏巴什往援,格布庫中箭殞,西特庫、烏巴什俱歿於陣,賊退。

時偽將高汝勵據三寨山,豪格遣古朗阿擊之,大破其眾。獻忠發大隊迎敵,古朗阿直沖其陣,賊奔潰,未幾復合,古朗阿偕瑚里布破之。賊率馬步兵分三路來犯,古朗阿奮勇進擊,與巴揚阿均陣亡。

濟三,滿洲正黃旗人,姓扎庫塔。自崇德六年,以佐領從大兵有功。順治元年,與騎都尉色勒布,雲騎尉祖應元,參領金應得,驍騎尉西來,閒散達魯哈、薩門、岱納,並從定國大將軍豫親王多鐸南征。二年四月,大兵渡淮,薄揚州城。應元、應得、岱納以紅衣砲攻城,城頹,岱納先登,與應元、應得同陣亡。克揚州,大兵渡江,令左翼舟師留泊北岸備敵。敵駕舟來犯,色勒布迎擊,中砲死。分兵江陰縣,薩門以雲梯先登,被戕。達魯哈繼進,亦陣歿。六月,多鐸定南京,分大兵之半,令多羅貝勒博洛等進徇蘇州,下之,擢濟三副都統,駐守。明福王總兵黃蜚潛納蘇州叛卒來襲。濟三聞變,率兵擒剿,敵合圍,濟三戰死。大兵至浙,攻嘉興,砲毀其城,西來率所部先登,克之。旋回兵取昆山縣城,被砲死。

瑚密色,滿洲鑲黃旗人,姓佟佳,世居加哈。崇德元年,以佐領銜從征明,屢有功。順治元年,從入關,敗流賊唐通於一片石,追至安肅、望都,殲賊無算。嗣隨多鐸軍渡江,屢破明兵句容。時明魯王硃以海踞紹興,博洛遣參領王先爵徇湖州,士兵蜂至,元爵戰歿。博洛次杭州,魯王遣其督師侍郎孫嘉績、熊汝霖渡錢塘江來犯。瑚密色偕騎都尉色赫等擊敗嘉績兵,擒其隊帥,追至江中,汝霖兵殊死戰,瑚密色中槍戰死。色赫從定浙江,旋下福建,還過平湖,遇土寇,亦以中槍陣亡。

敦達里,滿洲人。幼事太宗,後分隸肅親王豪格。崇德八年八月庚午,太宗崩,敦達里以幼蒙恩養,不忍永離,遂以身殉。諸王貝勒等義之,以敦達里志不忘君,忠忱足尚,贈甲喇章京,子孫永免徭役。

安達里,葉赫人。來歸時,太宗憐而養之,洊授官職,亦請殉,諸王貝勒等亦甚義之,予衣一襲,豫議恤典,加贈牛錄章京為梅勒章京,子孫世襲,如敦達里例。既定議,召安達里諭之。臨殉時,謂諸王貝勒等曰:「若先帝在天之靈,問及後事,將何以應?」諸王貝勒等對曰:「先帝肇興鴻業,我等翊戴幼主,嗣位承基,當實心輔理。儻邀呵護,是所原也。」

許友信,以軍弁隨明將左夢庚投誠,隸鑲白旗漢軍。隨大兵征閩、粵有功,定南大將軍貝勒博洛委署潮州副總兵。順治四年,明桂王由榔遣兵略境,友信單騎出戰,遇伏死。

是年,桂王兵部尚書張家玉陷東莞,署總兵成升、副將李義均陣亡。桂王兵科給事中陳邦彥同時犯廣州,游擊閻行龍、王士選、熊師文俱死之。桂王既由監國僭號,志在興復,其始略有兩廣、雲、貴、湖南、江西、四川各地。且鄭成功出沒閩、浙,奉其偽號,遙相應和,聲勢頗張。經大軍先後戡定,桂王已窮竄土司,肅清在邇,而孫可望、李定國等復群相擁戴,作螳臂之拒者有年。至定國與可望內訌,順治十四年十月,可望走湖南乞降,於是洪承疇、吳三桂乃奏請乘時大舉,逐漸進剿,軍行有利。十八年,三桂兵及緬甸,緬人執獻由榔軍前,事乃大定。

十餘年中,死事或被執不屈者:如四年,剿廣東假明封號土賊,有廣東巡按劉顯名等;六年,剿靈山土賊,有廣東都司僉書李昌等;七年,徵廣州,有輕車都尉尚可福等;八年,李定國分兵窺全州,有廣西巡按王荃可等;九年,犯辰州,有分巡辰常道劉升祚等;犯平樂,有府江道周永緒等;犯柳州,有分守右江道金漢蕙等;陷桂林,有右翼總兵曹成祖、提標游擊馬騰龍等;十年,犯羅定,有兵備道鄔象鼎等;犯靖州,有湖南副總兵楊國勛等;犯連州,有廣東運署都司僉書竇明運等;犯化州,有防守參將應太極等;十一年,犯電白,有從征八品官費揚古等;十四年,海賊乘亂竄雷州,有徐聞營游擊傅進忠等。孫可望之從亂也,六年,賊黨一隻虎犯永州,有新擢陜西布政使、右參議李懋祖等;九年,犯衡州,有隨定遠大將軍敬謹親王尼堪部下副都統武京等;犯成都,有敘州府知府周基昌等;十三年,犯臨藍,有委署參將殷壯猷等。

至為鄭成功而死者:三年,成功族人鄭彩據廈門,掠連江,有知縣宋人望等;六年,成功犯長泰,有知縣傅永吉等;犯漳浦,有總兵楊佐等;八年,犯海澄,有知縣甘體垣等;十二年,犯仙游,有知縣陳有虞等;十三年,犯海澄,有一等輕車都尉哈勒巴等;犯福州,有二等輕車都尉巴都等;十五年,犯臺州,有海門營水師游擊李宏德等;犯溫州,有盤石衛水師游擊熊應鳳等;十七年,犯江寧,有一等輕車都尉瑚伸布祿、二等輕車都尉猛格圖等;犯崇明,有知縣陳慎等;犯臺州之太平,有左營都司李柱國等;犯廈門,有護軍統領伊勒圖、前鋒參將佟濟、前鋒校鄂勒布等。蓋明籓自立,以兵力削除者,桂王為最棘。

同時附唐王硃聿鍵,而陸梁於江西郡邑者,則為金聲桓,參領布達理、布政使遲變龍、分守湖東道成大業、宜黃知縣馮穆等皆死之。魯王以監國踞浙,偪福建興化,則知府黎樹聲等;據舟山內擾,則紹興府推官劉方至死之。

其無所附麗而以叛聞者為姜瓖,五年,踞大同,催餉騎都尉鍾固、山西兵備道宋子玉等死之。六年,從英親王阿濟格等軍進討者,騎都尉索寧、雲騎尉洛多理等皆陣亡。分援河東、井坪、蒲州、神木等處,則鄭宏國、佟國仕、武韜、鄭世英等亦先後陣亡。

時天下初定,人心反側。各省土賊蜂起,或剿或守。在順治一朝,死者尤夥。獨著其關系大局者,見有清開國艱難之大概焉。


\end{pinyinscope}