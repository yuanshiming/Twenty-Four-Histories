\article{列傳一百六十五}

\begin{pinyinscope}
黃爵滋金應麟陳慶鏞蘇廷魁硃琦

黃爵滋,字樹齋,江西宜黃人。道光三年進士,選庶吉士,授編修,遷御史、給事中。以直諫負時望,遇事鋒發,無所回避,言屢被採納。十五年,特擢鴻臚寺卿。詔以爵滋及科道中馮贊勛、金應麟、曾望顏諸人均敢言,故特加擢任,風勵言官,開忠諫之路,勉其勿因驟得升階,即圖保位,並以誥誡臣工焉。尋疏陳察天道,廣言路,儲將才,制匪民,整飭京城營衛,申嚴外夷防禁六事,又陳漕、河積弊,均下議行。

時英吉利船艦屢至閩、浙、江南、山東洋面游奕,測繪山川地圖。爵滋疏言:「外國不可盡以恩撫,而沿海無備可危。」十八年,上禁煙議疏曰:「竊見近年銀價遞增,每銀一兩,易制錢一千六百有零,非耗銀於內地,實漏銀於外洋也。蓋自鴉片流入中國,道光三年以前,每歲漏銀數百萬兩,其初不過紈褲子弟習為浮靡。嗣後上自官府搢紳,下至工商優隸,以及婦女僧道,隨在吸食。粵省奸商勾通兵弁,用扒龍、快蟹等船,運銀出洋,運煙入口。故自道光三年至十一年,歲漏銀一千七八百萬兩;十一年至十四年,歲漏銀二千餘萬兩;十四年至今,漸漏至三千萬之多;福建、浙江、山東、天津各海口合之亦數千萬兩。以中土有用之財,填海外無窮之壑,易此害人之物,漸成病國之憂,年復一年,不知伊於胡底。各省州縣地丁錢糧,徵錢為多,及辦奏銷,以錢為銀,前此多有贏餘,今則無不賠貼。各省鹽商賣鹽得錢,交課用銀,昔之爭為利藪者,今則視為畏途。若再數年,銀價愈貴,奏銷如何能辦?積課如何能清?設有不測之用,又如何能支?今天下皆知漏卮在鴉片,而未知所以禁也。夫耗銀之多,由於販煙之盛;販煙之盛,由於食煙之眾。無吸食自無興販,無興販則外洋之煙自不來矣。宜先重治吸食,臣請皇上準給一年期限戒煙,雖至深之癮,未有不能斷絕者。至一年仍然服食,是不奉法之亂民,加之重刑不足恤。舊例吸煙罪止枷杖,其不指出興販者,罪止杖一百、徒三年,俱系活罪。斷癮之苦,甚於枷杖與徒,故不肯斷絕。若罪以死論,臨刑之慘急,苦於斷癮之茍延,臣知其原死於家而不原死於市。況我皇上雷霆之威,赫然震怒,雖愚頑沉溺之久,自足以發聾振瞶。皇上之旨嚴,則奉法之吏肅,犯法之人畏。一年之內,尚未用刑,十已戒其八九。已食者藉國法以保餘生,未食者因炯戒以全身命,止闢之大權,即好生之盛德也。伏請飭諭各督撫嚴行清查保甲,初先曉諭,定於一年後取具五家互結,準令舉發,給予優獎。倘有容隱,本犯照新例處死,互結之家照例治罪。通都大邑,往來客商,責成店鋪,如有容留食煙之人,照窩藏匪類治罪。文武大小各官,照常人加等,子孫不準考試。官親幕友家丁,除本犯治罪外,本管官嚴加議處。滿、漢官兵,照地方官保甲辦理;管轄失察之人,照地方官辦理。庶幾軍民一體,上下肅清,漏卮可塞,銀價不至再昂,然後講求理財之方,誠天下萬世臣民之福也。」疏上,上深韙之,下疆臣各抒所見,速議章程。

先是,太常寺少卿許乃濟疏言,煙禁雖嚴,閉關不可,徒法不行,請仍用舊制納稅,以貨易貨,不得用銀購買,吸食罪名,專重官員、士子、兵丁,時皆謂非政體。爵滋劾乃濟,罷其職,連擢爵滋大理寺少卿、通政使、禮部侍郎,調刑部。十九年,廷臣議定販煙、吸煙罪名新例,略如爵滋所請。

林則徐至粵,盡焚躉船存煙,議外國人販煙罪。英領事義律不就約束,兵釁遂開。二十年,命爵滋偕左都御史祁俊藻赴福建查辦禁煙,與總督鄧廷楨籌備海防。洎英兵來犯,廷楨屢挫敵於廈門,上疑之。爵滋與俊藻方至浙江按事,復命赴福建察奏。疏陳:「廷楨所奏不誣;定海不可不速復;水師有專門之技,宜破格用人。」具言戰守方略。又言浙江為閩、粵之心腹,與江蘇為脣齒,請飭伊里布不可偏聽琦善,信敵必退。及回京,復極言英人勞師襲遠不足慮,宜竟與絕市,募兵節餉,為持久計,以海防圖進。既而琦善在粵議撫不得要領,連歲命將出師,廣東、浙江皆不利。二十二年,英兵由海入江,乃定和議於江寧,煙禁自此弛矣。尋丁父憂去官。

爵滋為御史時,稽察戶部銀庫,嘗疏言庫丁輕收虧帑之弊。二十三年,銀庫虧空九百萬兩事發,追論管庫、查庫諸臣,罪皆褫職責賠,賠既足,次第予官。爵滋以員外郎候補,病足家居,上猶時問其何在。三十年,至京,會上崩,遂不出。逾三年,卒。

爵滋以詩名,喜交游,每夜閉閣草奏,日騎出,遍視諸故人名士,飲酒賦詩,意氣豪甚。及創議禁煙,始終主戰,一時以為清流眉目。所著奏議、詩文集行於世。

金應麟,字亞伯,浙江錢塘人。以舉人入貲為中書。道光六年,成進士,授刑部主事,總辦秋審,先後從大臣讞獄四川、湖北、山西。累擢郎中,改御史,遷給事中。疏請修改刑例,於鬥毆、報盜、劫囚、誣告、私鑄、服舍違式、斷罪引律、奴婢毆主、故禁故勘平人、應捕人追捕罪人、犯罪存留養親、官司出入人罪、徒流遷徙地方、外省駐防逃人,逐條論列,多被採取改定;又論銅船恣橫不法及驛站擾累諸弊,並下各省督撫禁革。先後封事數十上,劾疆臣琦善、河臣吳邦慶尤為時稱。宣宗嘉其敢言,擢太常寺少卿。遭憂歸,服闋,授鴻臚寺卿。疏論水師廢弛,漕政頹紊。十九年,出為直隸按察使,鞫護理長蘆鹽運使楊成業等得贓獄,論遣戍,前運使陳崇禮等並罣議。尋召為大理寺少卿。

二十二年,疏言:「海疆諸臣欺罔,其故由於爵祿之念重,而趨避之計工。欲破其欺,是在乾斷。資格不可拘,嫌疑不必避,舊過不妨宥,重賞不宜惜。近頃長江海口鎮兵足守,而敵船深入,逃潰時聞。竭億萬氓庶之脂膏。保一二庸臣之軀命。議者諉謂無人無兵無餉無械。竊以無人當求,無兵當練,無餉械亦當計度固有,多則持重,少則用謀,作三軍之氣,定邊疆之危,在皇上假以事權,與任事者運用一心而已。」復疏進預計度支圖、火器圖、籌海戰方略甚悉。二十三年,以親老乞歸省,不復出。著有廌華堂奏議及駢體文。

陳慶鏞,字頌南,福建晉江人。道光十二年進士,選庶吉士,散館授戶部主事,遷員外郎,授御史。二十三年,海疆僨事,獲罪諸臣浸復起用。

慶鏞上疏論刑賞失措,曰:「行政之要,莫大於刑賞。刑賞之權,操之於君,喻之於民,所以示天下之大公也。大學論平天下之道,在於絜矩。矩者何,民之好惡是已。海疆多事以來,自總督、將軍以至州縣丞倅,禽駭獸奔。皇上赫然震怒,失律之罪,法有莫逭。於是辱國之將軍奕山、奕經,參贊文蔚,總督牛鑒,提督餘步雲,先後就逮,步雲伏法。血氣之倫,罔不拊手稱快,謂國法前雖未伸於琦善,今猶伸於餘步雲。乃未幾起琦善為葉爾羌幫辦大臣。邸報既傳,人情震駭,猶解之曰:『古聖王之待罪人,有投四裔以禦魑魅者。』皇上之於琦善,殆其類是,而今且以三品頂戴用為熱河都統矣,且用奕經為葉爾羌幫辦大臣,文蔚為古城領隊大臣矣。琦善於戰事方始,首先示弱,以惰軍心,海內糜爛,至於此極。既罷斥終身不齒,猶恐不足饜民心而作士氣。奕經之罪,雖較琦善稍減,文蔚之罪,較奕經又減。然皇上命將出師,若何慎重。奕經頓兵半載,曾未身歷行間,騁其虛憍之氣,自詭一鼓而復三城;卒之機事不密,貽笑敵人,覆軍殺將,一敗不支。此不待別科騷擾供億、招權納賄之罪,而已不可勝誅。臣亦知奕經為高宗純皇帝之裔,皇上親親睦族,不忍遽加顯戮。然即幸邀寬典,亦當禁錮終身,無為天潢宗室羞,豈圖收禁未及三月,輒復棄瑕錄用?且此數人者,皇上特未知其見惡於民之深耳。倘俯採輿論,孰不切齒琦善為罪魁,誰不疾首於奕山、奕經、牛鑒、文蔚,而以為投畀之不容緩?此非臣一人之私言也。側聞琦善意侈體汰,跋扈如常,葉爾羌之行,本屬怏怏;今果未及出關,即蒙召還。熱河密邇神京,有識無識,莫不撫膺太息,以為皇上鄉用琦善之意,尚不止此。萬一有事,則熒惑聖聰者,必仍系斯人。履霜堅冰,深可懍懼。頃者御試翰詹,以『烹阿封即墨』命題,而今茲刑賞顧如此,臣未知皇上所謂阿者何人?即墨者何人?假如聖意高深,偶或差忒,而以即墨為阿,阿為即墨,將毋譽之毀之者有以淆亂是非耶?所望皇上立奮天威,收回成命,體大學絜矩之旨,鑒盈廷毀譽之真,國法稍伸,民心可慰。」疏上,宣宗嘉之,諭曰:「朕無知人之明,以致琦善、奕經、文蔚諸人喪師失律,惟有反躬自責,不欲諉罪臣工。今該御史請收回成命,朕非文過飾非之君,豈肯回護?」復革琦善等職,令閉門思過。於是直聲震海內。

二十五年,遷給事中,巡視東城,以事詿吏議,左遷光祿寺署正。二十六年,乞歸。文宗即位,以大學士硃鳳標薦,復授御史,蹶而再起,氣不少撓,疊上疏多關大計。自粵匪起,福建群盜蠢動,蔓延泉、漳、興、永諸郡。咸豐三年,慶鏞疏陳利害,命回籍治團練。惠安妖婦邱氏煽亂,偵獲置諸法,賜花翎。俄以病請開缺。七年,逆匪林俊糾莆陽、仙游、永春、南安群賊犯泉州,慶鏞激厲士民固守,賊攻圍數日而退。論功,以道員候選。八年,卒於泉州,贈光祿寺卿,賜祭葬,廕一子知縣,祀鄉賢祠。

慶鏞精研漢學,而制行則服膺宋儒,文辭樸茂,著有籀經堂文集、三家詩考、說文釋、古籀考等書。

蘇廷魁,字賡堂,廣東高要人。道光十五年進士,選庶吉士,授編修。二十二年,遷御史。海疆兵事方亟,迭上疏論列,請修築虎門砲臺及燕塘墟、大沙河、龜岡諸要隘,以防敵回擾粵,既而和議成。二十三年春,有白氣自天西南隅直掃參旗,因災異上疏數千言,極論時政乖迕,歸罪樞臣穆彰阿等,請立罷黜;並下罪己詔,開直諫之路:語多指斥。宣宗覽奏動容,嘉其切直,朝野傾望豐採。遭憂去官,服闋,遷給事中。

咸豐元年,上謹始疏,請求宏濟之道,執勞謙之義,防驕泰之萌,推誠任賢,慎始圖治,選擇翰詹為講官,嚴取孝廉方正備採用,文宗嘉納之。賽尚阿出督師,援引內閣侍讀穆廕擢五品京堂,在軍機大臣上學習行走。廷魁疏劾其壞舊制,用私親,超擢太驟,易啟幸進之門,請俟賽尚阿還,令回章京本任,詔斥擅預黜陟,猶以素行端方,不之罪。上先隱其名,出疏示賽尚阿,賽尚阿退,飲臺垣酒,問:「誰實彈我?」廷魁出席曰:「公負國,某不敢負公。」再以憂歸。四年,廣東紅巾匪起,將犯省城。或獻議借外兵,以鋪捐為餉糈,力爭,罷其議。

八年,英法聯軍踞廣州,廷魁與侍郎羅惇衍等倡設團防局,嚴清野,絕漢奸,招募東莞及三元里、佛山練勇得數萬人,聲言戒期攻城,敵師出,擊斬百餘級。敵始有戒心,稍戢,連艘北犯,既而天津議和,廣東敵兵未退,民益憤,廷魁等請留練局以防土寇。敵謂既媾和何復募勇,且以懸金購領事巴夏禮為責言。議和大臣桂良慮撓成議,奏請撤局。初,艇匪擾廣寧,圍四會、肇慶,兵疲糧罄,或勸之去,廷魁曰:「予團防大臣也,誓與城為存亡!」會提督昆壽克梧州,以兵來援,城得完。疆臣屢欲上其功,皆固辭。

同治初,以中外大臣薦,授河南開歸陳許道,歷布政使,擢東河總督。七年,河決滎澤,未奪溜,革職留任,閱三月工竣,復之。逾年,內召,去官,稱疾歸。光緒四年,卒。

硃琦,字伯韓,廣西臨桂人。父鳳森,嘉慶六年進士,官河南濬縣知縣,有政聲。滑縣教匪起,率團練御之,屢破賊,城守卒完。遷河南府通判。歿,祀名宦。

琦,舉鄉試第一。道光十五年,成進士,選庶吉士,授編修。慕同里陳宏謀之為人,以氣節自勵。遷御史,值海疆事定,禍機四伏,而上下復習委靡,言路多容默,深以為憂。著名實說,略曰:「天下有鄉曲之行,有大人之行。鄉曲、大人,其名也,考之其行,而察其有用與否,其實也。世之稱者,曰謹厚,曰廉靜,日退讓,三者名之至美也,而不知此鄉曲之行也,非所謂大人者也。大人之職,在於經國家、安社稷,有剛毅之大節,為人主畏憚;有深謀遠慮,為天下長計。合則留,不合以義去。身之便安,不暇計也;世之指摘,不敢逃也。今也不然。曰:吾為天下長計,則天下之釁必集於我;吾為人主畏憚,則不能久於其位;不如謹厚、廉靜、退讓,此三者可以安坐而無患,而名又至美也。夫無患而可久於其位,又有天下美名,士何憚而不爭趨於此?故近世所稱公卿之賢者,此三者為多矣。當其瓘冠襜裾,從容正步,趨於廊廟之間,上之人不疑,而非議不加,其沉深不可測也。一旦遇大利害,搶攘無措,鉗口撟舌而莫敢言,而所謂謹厚、廉靜、退讓,至此舉無可用,於是始思向之為人主畏憚而有深謀遠慮者,不可得矣。且謹厚、廉靜、退讓三者,非果無用也。古有負蓋世之功而思持其後,挾震主之威而唯恐不終,未嘗不斤斤於此,故又於鎮薄俗、保晚節。後世無其才而冒其位,安其樂而避其患,假於名之至美,僴然自以為足。是藏身之固,莫便此三者。孔子之所謂鄙夫也,其究鄉願也。是張禹、胡廣、趙戒之類也,甚矣其恥也!」於是數上疏切論時務,皆留中不報。時咸推其抗直,稱為名御史。

琦以言既不見用,二十六年,告歸。越數年,廣西群賊蜂起,其言皆驗。家居治團練,助守御。賊中梟傑張家祥者,悔罪投誠,當事猶疑之。琦知其忠勇可用,以全家保之,乃受降,改名國樑,卒為名將。琦以守城勞議敘,以道員候選。咸豐六年,再至京師。居兩歲,從欽差大臣桂良至江蘇,無所遇,王有齡獨重之,有齡撫浙,闢贊軍事。十一年,粵匪犯杭州,總理團練局。守清波門,城陷,死之。贈太常寺卿,予騎都尉世職,祀昭忠祠。

琦學宗程、硃,詩古文皆有法,著有怡志堂集、臺垣奏議。

論曰:禁煙之議,創自黃爵滋,行之操切,而邊釁遂開,繼之游移而國威愈墮,誠不可以此歸咎始議之人。然謀國萬全,決勝千里,非恃意氣為也。行固維艱,言亦豈易易哉?金應麟同被拔擢,亦始終主戰。陳慶鏞、蘇廷魁、硃琦時稱「三直」;合之應麟,又稱「四虎」。所言有用有不用,凜凜然有生氣,要足以砭頑振懦矣。


\end{pinyinscope}