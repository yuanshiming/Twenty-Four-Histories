\article{列傳三百五}

\begin{pinyinscope}
籓部一

○科爾沁扎賚特杜爾伯特郭爾羅斯喀喇沁土默特

清起東夏,始定內盟。康熙、乾隆兩戡準部。自松花、黑龍諸江,迤邐而西,絕大漠,亙金山,疆丁零、鮮卑之域,南盡昆侖、析支、渠搜,三危既宅,至於黑水,皆為籓部。撫馭賓貢,夐越漢、唐。屏翰之重,所以寵之;甥舅之聯,所以戚之;銳劉之衛,所以懷之;教政之修,所以宣之。世更十二,載越廿紀,虔奉約束,聿共盟會,奧矣昌矣。若夫元之戚垣,自為風氣;明之蕃衛,虛有名字,蓋未可以同年而語。帶礪之盛,具見世表。茲綜事實,列之為傳。揆文奮武,悅近來遠,疏附禦侮,可得大凡。末造顛頹,乃彰畔渙。盛衰得失,斯可鑒已。

科爾沁部,在喜峰口外,至京師千二百八十里。東西距八百七十里,南北距二千有百里。東扎賚特,西扎嚕特,南盛京邊墻,北黑龍江。

元太祖削平西北諸國,建王、駙馬等世守之,為今內外扎薩克蒙古所自出。

科爾沁始祖曰哈布圖哈薩爾,元太祖弟,今科爾沁六扎薩克,及扎賚特、杜爾伯特、郭爾羅斯、阿嚕科爾沁、四子部落、茂明安、烏喇特、阿拉善、青海和碩特,皆其裔。哈布圖哈薩爾十四傳至奎蒙克塔斯哈喇,有子二:長博第達喇,號卓爾郭勒諾顏;次諾捫達喇,號噶勒濟庫諾顏。

博第達喇子九:長齊齊克,號巴圖爾諾顏,為土謝圖汗奧巴、扎薩克圖郡王布達齊二旗祖;次納穆賽,號都喇勒諾顏,為達爾漢親王滿珠習禮、冰圖郡王洪果爾、貝勒棟果爾三旗祖;次烏巴什,號鄂特歡諾顏,見郭爾羅斯傳;次烏延岱科托果爾;次托多巴圖爾喀喇;次拜新;次額勒濟格卓哩克圖,裔不著;次愛納噶,號車臣諾顏,見杜爾伯特傳;次阿敏,號巴噶諾顏,見扎賚特傳。諾捫達喇子一,曰哲格爾德,為扎薩克鎮國公喇嘛什希一旗祖。

蒙古強部有三:曰察哈爾;曰喀爾喀;曰衛拉特,即厄魯特。明洪熙間,科爾沁為衛拉特所破,避居嫩江,以同族有阿嚕科爾沁,號嫩江科爾沁以自別。扎賚特、杜爾伯特、郭爾羅斯三部與同牧,服屬於察哈爾。

太祖癸巳年,科爾沁臺吉齊齊克子翁果岱,納穆賽子莽古斯、明安等,隨葉赫部臺吉布齋,糾哈達、烏拉、輝發、錫伯、卦爾察、珠舍裏、納殷諸部來侵,攻赫濟格城不下,陳兵古哷山。上親御之,至扎喀路,諭諸將曰:「彼雖眾,皆烏合。我以逸待勞,傷其一二臺吉,眾自潰。」命巴圖魯額亦都率百騎挑戰,葉赫諸部兵罷攻城來御,逆擊之。明安馬蹶,裸而遁,追至哈達部柴河寨南,俘獲甚眾。戊申,征烏拉部,圍宜罕阿林城,翁果岱復助烏拉臺吉布占泰,我師擊敗之。於是莽古斯、明安、翁果岱先後遣使乞好。

天命九年,翁果岱子奧巴率族來歸。尋為察哈爾所侵,我援之,解圍去。天聰二年,會大軍征察哈爾。三年,從征明,克遵化州,圍北京。五年,圍大凌河,降其將祖大壽。六年,從略大同、宣府邊。八年,復從征明。

十年春,大軍平察哈爾,獲元傳國玉璽。奧巴子土謝圖濟農巴達禮偕臺吉烏克善、滿珠習禮、布達齊、洪果爾、喇嘛什希、棟果爾,及扎賚特、杜爾伯特、郭爾羅斯、喀喇沁、土默特、敖漢、柰曼、巴林、扎嚕特、阿嚕科爾沁、翁牛特諸部長來賀捷。以上功德隆,宜正位號,遺朝鮮國王書,示推戴意。四月,合疏上尊號,改元崇德。禮成,敘功,詔科爾沁部設扎薩克五:曰巴達禮,曰滿珠習禮,曰布達齊,曰洪果爾,曰喇嘛什希,分領其眾,賜親王、郡王、鎮國公爵有差。十月,命大學士希福等赴其部,鞫罪犯,頒法律,禁奸盜,編佐領。二年,從征喀木尼堪部及朝鮮。三年,徵喀爾喀。四年春,徵索倫。秋,圍明杏山、高橋。八年,隨饒餘貝勒阿巴泰、護軍統領阿爾津徵明及黑龍江諸部。

順治元年,偕扎賚特、杜爾伯特、郭爾羅斯兵隨睿親王多爾袞入山海關,走流賊李自成,追至望都。二年,隨豫親王多鐸定江南。三年,復隨剿蘇尼特叛人騰機思,敗喀爾喀土謝圖汗、車臣汗援兵。七年,科爾沁復設扎薩克一,以棟果爾子彰吉倫領之,由貝勒晉郡王爵。十三年,上以科爾沁及扎賚特、杜爾伯特、郭爾羅斯、喀喇沁、土默特、敖漢、柰曼、巴林、扎嚕特、阿嚕科爾沁、翁牛特、烏珠穆沁、浩齊特、蘇尼特、阿巴噶、四子部落、烏喇特、喀爾喀左翼、鄂爾多斯諸扎薩克歸誠久,賜敕曰:「爾等秉資忠直,當太祖、太宗開創之初,誠心歸附,職效屏籓。太祖、太宗嘉爾勛勞,崇封爵號,賞賚有加。朝覲貢獻,時令陛見,飲食教誨,為數甚多。凡有懷欲吐,俱得陳奏,心意和諧,如同父子。朕荷祖宗鴻庥,統一寰宇,恐於懿行有違,成憲未洽,恆用憂惕。親政以來,六年於茲,未得與爾等一見,雖因萬幾少暇,而懷爾之忱,時切朕念。每思爾等效力有年,功績卓著,雖在寤寐,未之有斁。誠以爾等相見既疏,恐有壅蔽,不能上通,故特遣官齎敕賜幣,以諭朕意。嗣後有所欲請,隨時奏聞,朕無不體恤而行。朕方思致天下於太平,爾等心懷忠藎,毋忘兩朝恩寵。朕世世為天子,爾等亦世世為王,享富貴於無窮,垂芳名於不朽,不亦休乎!」

康熙十三年,徵所部兵討逆籓吳三桂。十四年,剿察哈爾叛人布爾尼。先是科爾沁內附,莽古斯以女歸太宗文皇帝,是為孝端文皇后。孫烏克善等復以女弟來歸,是為孝莊文皇后。曾孫綽爾濟復以女歸世祖章皇帝,是為孝惠章皇后。科爾沁以列朝外戚,荷國恩獨厚,列內扎薩克二十四部首。有大征伐,必以兵從,如親征噶爾丹,及剿策妄阿喇布坦、羅卜藏丹津、噶爾丹策凌、達瓦齊諸役,扎薩克等效力戎行,莫不懋著勤勞。土謝圖親王、達爾漢親王、卓哩克圖親王、扎薩克圖郡王四爵俸幣視他部獨增,非惟禮崇姻戚,抑以其功冠焉。所部六旗,分左右翼。土謝圖親王掌右翼,附扎賚特部一旗、杜爾伯特部一旗;達爾漢親王掌左翼,附郭爾羅斯部二旗,統盟於哲裏木。右翼中旗駐巴顏和翔,左翼中旗駐伊克唐噶哩克坡,右翼前旗駐席喇布爾哈蘇,右翼後旗駐額木圖坡,左翼前旗駐伊岳克里泊,左翼後旗駐雙和爾山。爵十有七:扎薩克和碩土謝圖親王一;附多羅貝勒一;扎薩克和碩達爾漢親王一;附卓哩克圖親王一;多羅郡王二,一由親王降襲;多羅貝勒一;固山貝子一;輔國公四,一由貝子降襲;扎薩克多羅扎薩克圖郡王一;扎薩克多羅冰圖郡王一;扎薩克多羅郡王一,由貝勒晉襲;附輔國公一,由貝子降襲;扎薩克鎮國公一。左翼中旗扎薩克達爾漢親王滿珠習禮之玄孫色布騰巴勒珠爾,乾隆十一年三月尚固倫和敬公主。二十年,準噶爾之平,以功加雙俸,尋以阿睦爾撒納叛事,奪爵。二十三年,復封和碩親王。三十七年,與征金川,又以附富德劾阿桂,奪爵。四十年,復之。

四傳至棍楚克林沁,襲鎮國公,官至御前大臣,卒。其後左翼中旗輔國公二,左翼後旗輔國公一,均停襲。左翼後旗扎薩克多羅郡王僧格林沁,以軍功晉博多勒噶臺和碩親王。同治二年,予世襲罔替。四年,以剿捻匪陣亡,自有傳。其旗增多羅貝勒一,輔國公二,皆以僧格林沁功。

僧格林沁子伯彥訥謨祜,初封輔國公。同治三年,晉貝勒。四年七月,襲博多勒噶臺親王,為御前大臣。十一月,命與左翼中旗扎薩克達爾漢親王索特那木朋蘇克等選馬隊剿奉天馬賊。五年二月,大破馬賊於鄭家屯。三月,命捕吉林餘匪。六月,條陳奉天善後事宜,詔如所請行。匪平,回京。光緒初,德宗典學,命在毓慶宮行走,授兼鑲黃旗領侍衛內大臣。十七年,卒。

自道光季年海防事起,洎咸豐三年粵逆北犯,八年海防又急,皆調東三盟兵協同防剿,科爾沁部為之冠,予爵職、給廕襲者,皆甲諸部。僧格林沁之亡,始撤哲裏木盟兵旋所部。

初,科爾沁諸旗以距奉天近,皆招佃內地民人開墾。乾隆四十九年,盛京將軍永瑋等奏:「賓圖王旗界內所留民人近鐵嶺者,達爾漢王旗所留民人近開原者,即交鐵嶺縣、開原縣治之。」嘉慶十一年十月,盛京將軍富俊等以左翼後旗昌圖額勒克地方招墾閒荒,經歷四載,人民四萬有奇,請增置理事通判治之。達爾漢王旗界內所留人民,亦交通判就近並治,時諸旗扎薩克、王、公等多招民人墾荒,積欠抗租,則又請驅逐。廷議非之,嚴定招墾之禁,已佃者不得逐,未墾者不得招。道光元年,左翼中旗扎薩克達爾漢親王布彥溫都爾瑚竟以墾事延不就鞫,奪扎薩克。然私放私墾者仍日有所增,流民游匪於焉麕集。同治中,以昌圖匪亂,通判秩輕,升為理事同知。光緒二年,署盛京將軍崇厚奏設官撫治,以清盜源。遂升昌圖同知為府,以原墾達爾漢王旗之梨樹城、八面城地置奉化、懷德二縣隸之。七年,又設康平縣於康家屯,隸之。二十八年,盛京將軍增祺奏設遼源州於蘇家屯,隸之。皆治左翼三旗墾民。

是年,右翼前旗扎薩克圖郡王烏泰以放荒事屢被劾,命禮部尚書裕德會增祺勘治。四月,覆奏言:「烏泰已放荒界南北長三百餘里,東西寬一百餘里,外來客民有一千二百六十餘戶。烏泰不諳放荒章程,以致嗜利之徒,任意墾占,轉相私售,實已暗增數千餘戶,新開荒地又增長三百餘里,寬一百餘里。梅楞齊莫特、色楞等復袒護荒戶,阻臺吉壯丁在新放荒地游牧。協理臺吉巴圖濟爾噶勒遂以斂財聚眾,不恤旗艱,控之理籓院。經傳集烏泰等親自宣導,均各悔悟,原湔洗前愆,驅除讒慝,和同辦理旗務。請將烏泰、巴圖濟爾噶勒暫革,仍準留任,勒限三年,限滿經理得宜,由闔旗呈請開復,否則永遠革任;齊莫特、色楞等均分別屏黜,不準干預旗務。並為定領荒招墾章程,荒價則一半報效國家,一半歸之蒙旗。升科則每晌以中錢二百四十為籌餉設官等經費,以四百二十作蒙古生計,自王府至臺吉、壯丁、喇嘛,各有得數。仍酌留餘荒,講求牧養。」均報可。十月,增祺又奏勘明是旗洮爾河南北已墾未墾之地,約有一千餘萬畝,派員設局丈放。三十年,以其地置洮南府,並置靖安、開通二縣隸之。三十一年,盛京將軍趙爾巽以右翼後鎮國公旗墾地置安廣縣,而法庫門舊為左翼中達爾漢王諸旗招墾地,亦置同知治之。三十四年,東三省總督徐世昌以右翼中旗和碩土謝圖親王墾地置醴泉等縣。於是科爾沁六旗墾地幾遍,郡縣亦最多,諸扎薩克王公等得租豐溢,而化沙礫為膏沃,地方亦日臻富庶。

諸扎薩克王公等世次皆見表,惟右翼和碩土謝圖親王色旺諾爾布桑寶以庚子之變,中外多故,殞於非命。裕德等勘奏,謂為屬員逼勒而死,因請治偪勒者如律。尋增祺奏以族子業喜海順承襲,傳爵如故。

凡蒙旗,扎薩克為一旗之長,制如一品,與都統等。其輔曰協理臺吉。屬曰管旗章京,副章京,參領,佐領。蒙語管旗章京曰梅楞,參領曰札蘭,佐領曰蘇木。蘇木實分治土地人民。其佐領之額,右翼中旗二十二,左翼中旗四十六,右翼前旗、後旗均十六,左翼前旗、後旗均三。凡哲裏木盟重大事件,科爾沁六旗以近奉天,故由盛京將軍專奏。郭爾羅斯前旗一旗以近吉林,郭爾羅斯後旗、扎賚特、杜爾伯特三旗以近黑龍江,故各由其省將軍專奏。

扎賚特部,元太祖弟哈布圖哈薩爾十五傳至博第達喇,有子九,阿敏其季也。與兄齊齊克、納穆賽等鄰牧,號所部曰扎賚特。天命九年,阿敏子蒙袞偕科爾沁臺吉奧巴遣使乞好,優詔答之,遂率屬來歸。順治五年,授蒙袞子色棱扎薩克,以與科爾沁同祖,附之,隸哲裏木盟。旗一,駐圖卜紳察罕坡。其爵為扎薩克多羅貝勒,由固山貝子晉襲。

光緒二十五年,黑龍江將軍恩澤等奏:「以戶部咨,黑龍江副都統壽山條奏,請放蒙古各旗荒地,派員赴扎賚特旗剴切勸商,原將屬界南接郭爾羅斯前旗,東濱嫩江之四家子、二龍梭口等處,指出開放,南北約長三百餘里,東西寬百餘里或三四十里,設局勘辦。並謂若大東以至大西,使沿邊各蒙旗均能招民墾荒,則強富可期,即可無北鄙之驚。」下所司議行。先是哲裏木盟諸旗皆以禁墾甲令過嚴,無敢明言招墾者,至是始接踵開放雲。三十一年,以墾地置大賚治之。是部有佐領十六。

杜爾伯特部,在喜峰口外,至京師二千五十里。東西距百七十里,南北距二百四十里。東及北皆黑龍江,西扎賚特,南郭爾羅斯,北界索倫籓部。蒙古稱杜爾伯特部者二,同名異族。一姓鮮囉斯,為衛拉特臺吉孛罕裔,旗十有四,駐牧烏蘭古木,稱外扎薩克,別有傳。一姓博爾濟吉特,為元太祖弟哈布圖哈薩爾裔,即今駐牧喜峰口外之內札薩克也。

哈布圖哈薩爾十六傳至愛納噶,始以名其部。天命九年,愛納噶子阿都齊偕科爾沁臺吉奧巴遣使乞好,優詔答之,遂率屬來歸。順治五年,授阿都齊子色夌扎薩克,以與科爾沁同祖,附之,隸哲裏木盟。旗一,駐多克多爾坡。其爵為扎薩克固山貝子。

同治二年,杜爾伯特貝子貢噶綽克坦咨黑龍江將軍,請將交界重立封堆。尋勘明:「巴勒該岡以北黑龍江界內,有杜爾伯特蒙人等居屯四處,牌莫多以南杜爾伯特界內,有黑龍江省屬人等居屯八處,舊界所占均系曠地,應準各就其所,以安生計。蒙古越占巴勒該岡地,應將南榆樹改為新界,省屬人等越占牌莫多地,應將四六山改為新界,共立界堆十七。」奏入,詔如議。四年,貢噶綽克坦復咨以所立界堆將蒙古田地草廠歸入省界,有兒蒙古生計。詔派副都統克蒙額與哲裏木盟長及杜爾伯特會勘,劃還塔爾歡屯以東第十、第十一封堆之西蒙古墳塋房基,平毀二十顆樹封堆之南蒙界旗屯房屋,又增立界堆十有九,並以牌莫多以南官屯舊占蒙屯較巴勒該岡以北蒙屯舊占省屯多地十三里,撥二十顆樹封堆之南省屬空閒地如數補之。七年六月奏結,請飭貝子貢噶綽克坦嚴約屬人照界永遠遵守,報可。十年,以是旗私招民人墾荒,嚴申禁令,革其協理臺吉。光緒二十五年,將軍恩澤以招墾蒙地,關邊圉富強大計,復奏派員商勸放墾。時東三省鐵路之約既成,是部當鐵路之沖,交涉煩多,商民萃集。三十二年,因以所部墾地置安達治之,隸黑龍江。是部一旗,有佐領二十五。

郭爾羅斯部,在喜峰口外,至京師千八百九十七里。東西距四百五十里,南北距六百六十里。南盛京邊墻,東吉林府,西及北科爾沁。

元太祖遣弟哈布圖哈薩爾征郭爾羅斯部,十六傳至烏巴什,即以為所部號。子莽果仍之。

天命九年,莽果子布木巴偕科爾沁臺吉奧巴遣使乞好,優詔答之,遂率屬來歸。會察哈爾林丹汗掠科爾沁,遣軍由郭爾羅斯境往援,至農安塔。林丹汗遁,不敢復犯科爾沁及郭爾羅斯諸部。嗣設扎薩克二:曰布木巴,爵鎮國公;曰固穆,為布木巴從弟,爵輔國公。以與科爾沁同祖,附之,隸哲裏木盟。旗二:前旗駐固爾班察罕,後旗駐榛子嶺。爵三:扎薩克輔國公一,扎薩克臺吉一,附鎮國公一。

是部布木巴一旗為前旗,近吉林。嘉慶五年,吉林將軍秀林奏以郭爾羅斯墾地置長春理事通判,並請分徵其租,上以非體斥之。十傳至喀爾瑪什迪,於光緒九年削扎薩克,公爵如故。以其族等臺吉巴雅斯呼朗代為扎薩克。光緒十三年,復升長春為府。於是旗界內遼黃龍府舊地置農安縣,隸之。三十四年,又以墾地增廣,分置長嶺縣。宣統二年,分長春府地置德惠縣。旋又定國家與蒙古分收民租例。是旗置郡縣凡四,皆隸吉林。

固穆一旗為後旗,近黑龍江,亦當東三省鐵路之沖。光緒三年,以墾地置肇州,隸黑龍江。後又分置肇東經歷。是部二旗,墾地分隸吉林、黑龍江二省。前旗有佐領二十三。後旗有佐領三十四。

喀喇沁部,在喜峰口外,至京師七百六十里。東西距五百里,南北距四百五十里。東土默特及敖漢,西察哈爾正藍旗牧廠,南盛京邊墻,北翁牛特。

元時有札爾楚泰者,生濟拉瑪,佐元太祖有功。七傳至和通,有眾六千戶,游牧額沁河,號所部曰喀喇沁。子格哷博羅特繼之。

生子二:長格哷勒泰宰桑,為扎薩克杜棱貝勒固嚕思奇布及扎薩克一等塔布囊格哷爾二旗祖;次圖嚕巴圖爾,為扎薩克鎮國公色棱一旗祖。格哷勒泰宰桑子四:長恩克,次準圖,次鄂穆克圖,均居喀喇沁。天聰二年二月,恩克曾孫蘇布地以察哈爾林丹汗虐其部,偕弟萬丹偉徵等乞內附,表奏:「察哈爾汗不道,喀喇沁被虐,因偕土默特、鄂爾多斯、阿巴噶、喀爾喀諸部兵,赴土默特之趙城,擊察哈爾兵四萬。還,值赴明請賞兵三千,復殪之。察哈爾根本動搖,事機可乘。皇帝儻興師進剿,喀喇沁當先諸部至。」諭遣使面議。七月,遣喇嘛偕五百三十八人來朝,命貝勒阿濟格、碩託迎宴,刑白馬烏牛誓。九月,上親征察哈爾,蘇布地等迎會於綽洛郭勒,賜賚甚厚。三年正月,敕所部遵國憲。六月,蘇布地及圖嚕巴圖爾孫色棱等率屬來歸,詔還舊牧。十月,上徵明,以塔布囊布爾哈圖為導,入遵化,駐兵羅文峪。四年,布爾哈圖為明兵所圍,擊敗之,擒副將丁啟明及游擊一、都司二。詔嘉其功,賜莊田僕從及金幣。六月,由都爾弼從征察哈爾,林丹汗遁,以所收察哈爾糧貯遼河守之。復分兵隨貝勒阿濟格略明大同、宣府邊。八年正月,偕巴林、阿嚕科爾沁、阿巴噶諸部兵收撫察哈爾流民。五月,從征明大同,至朔州。九年正月,詔編所部佐領,以蘇布地子固嚕思奇布掌右翼,色棱掌左翼。五月,選兵從征明,敗之於遼河源。

崇德元年,詔授布爾哈圖一等子,賜號岱達爾漢塔布囊。二年,遣大臣阿什達爾漢等赴其部理庶獄。三年九月,隨大軍自密雲入明邊,敗其兵六千。十月,從征前屯衛及寧遠。七年,從圍薊州,過北京,下山東。

順治元年,從入山海關,擊流賊李自成。六年,從征喀爾喀。康熙十三年,大軍剿逆籓耿精忠等,所部塔布囊霍濟格爾偕土默特塔布囊善達等,以兵赴兗州。十七年,上諭曰:「塔布囊霍濟格爾等前自兗州赴浙江,聽康親王傑書調度。各統所屬官兵征剿逆賊,深入閩省,同大兵平定逆籓耿精忠。行間效力,身先士卒,沖鋒陷陣,奮勇用命,深為可嘉。宜降恩綸,即行議敘,以勵後效。」二十年,上駐蹕和爾和,諭曰:「塔布囊霍濟格爾出征時最著勤勞,今已溘逝。朕至此地,遣散秩大臣鄂齊等攜茶酒往奠。」二十五年,敘平浙江、福建功,賜參領巴雅爾等十人世職。

二十九年,從征噶爾丹,敗之於烏蘭布通。四十四年,詔增設一旗,以塔布囊格哷爾領之。五十四年,徵所部兵千赴推河防御策妄阿喇布坦,尋命侍郎覺和托等攜帑萬兩賜之,雍正九年,從征噶爾丹策凌。所部初設二旗,右翼駐錫伯河北,左翼駐巴顏珠爾克;後增一旗,駐左右翼界內。爵六:親王品級扎薩克多羅杜棱郡王一,由貝勒晉襲;附鎮國公一,由貝子降襲;輔國公一;扎薩克多羅貝勒一,由貝子晉襲;扎薩克固山貝子一,由鎮國公晉襲;扎薩克公品級一等塔布囊一。

乾隆四十一年,以所部墾地設平泉州。嘉慶八年,降爵。貝子丹巴多爾濟以獲逆犯陳德功,予貝勒,官至領侍衛內大臣、御前大臣,卒。光緒二十三年,扎薩克一等臺吉塔布囊巴特瑪鄂特薩爾以事革,復以貝勒熙凌阿襲。存爵五。

是部招民墾地最在先。乾隆十四年,始定不許容留民人多墾地畝之禁。道光十九年,復定喀喇沁、土默特種地民人不得以所種地畝折算蒙古賒貸銀錢例。光緒十七年,敖漢部金丹道匪之變,是部同時被擾。事平,特頒帑賑恤之。二十九年,熱河都統錫良以左翼旗招華商承辦全旗五金各礦,中旗同道勝銀行立有合同,開八里罕等地金礦,與定章應聲明華、洋股本若干,及只準指定一處不準兼指數處者不符,請飭外務部妥議辦法。下所司議申定章約束之。

是部右翼旗有佐領四十四,中旗有佐領三十八,左翼旗有佐領四十,與土默特二旗統盟於卓索圖。嘉慶中,設熱河都統後,是盟與昭烏達盟重大事件,皆由都統專奏。道光末,籌直隸海防,咸豐初,剿粵匪,皆徵是盟之兵,與哲裏木、昭烏達號東三盟兵,頗著功績雲。

土默特部,在喜峰口外,至京師千里。東西距四百六十里,南北距三百有十里。東養息牧牧廠,西喀喇沁,南盛京邊墻,北喀爾喀左翼及敖漢。土默特分左右翼,異姓同牧。主左翼者為元臣濟拉瑪裔。自濟拉瑪十三傳至善巴,與喀喇沁為近族。主右翼者為元太祖裔。自元太祖十九傳至鄂木布楚琥爾,生子固穆,與歸化城土默特為近族。

天總三年,善巴、鄂木布楚琥爾各率屬來歸。八年六月,選兵從征明,頒示軍律。七月,由獨石口入明邊,會大軍於保安州,分兵隸都統武訥格,略察哈爾邊。九年,詔編所部佐領,設扎薩克三:曰善巴,曰賡格爾,曰鄂木布楚琥爾。賡格爾者,善巴族也。崇德二年,以罪削扎薩克,善巴領其眾。自是土默特分左右翼,命善巴及鄂木布楚琥爾掌之。是年遣大臣阿什達爾漢等赴其部理庶獄。六年,從圍明錦州,敗總督洪承疇援兵。八年,隨饒餘貝勒阿巴泰徵明。

順治元年,從入山海關,擊流賊李自成。三年,隨剿蘇尼特部叛人騰機思。康熙元年,喀爾喀臺吉巴爾布冰圖來歸,詔附土默特牧。十三年,大軍剿逆籓耿精忠等,詔所部塔布囊善達偕喀喇沁塔布囊霍濟格爾以兵赴兗州聽調。十七年,調赴浙江,隨康親王傑書進剿。閩地悉定,諭優敘。五十五年,詔選兵千隨公傅爾丹屯鄂爾坤。五十九年,以旱歉收,賜帑賑之。雍正三年,塔布囊沙津達賚隨大軍防禦準噶爾。七年,封鎮國公。九年,大將軍傅爾丹擊準噶爾於和通呼爾哈諾爾,沙津達賚陣逃,削爵;而土默特部將之隨參贊內大臣馬蘭泰者,敗賊西爾哈昭,斬獲甚眾,稍雪恥焉。

所部二旗,左翼駐海他哈山,右翼駐巴顏和朔,隸卓索圖盟。爵三:扎薩克多羅達爾漢貝勒一,由鎮國公晉襲;附喀爾喀貝勒一;扎薩克固山貝子一。

乾隆四十一年,以所部墾地置朝陽縣。同治九年,以右翼旗箭丁等屢控扎薩克貝子索特那木色登科派太重,於是管旗章京阿阿尚等以因公派錢不能體恤,均革。熱河都統庫克吉泰因奏變通土默特比丁章程,申明交納丁錢舊章,箭丁子女不許妄行役使及隨侍陪嫁,八枝箭丁仍歸土默特管束。光緒十七年,敖漢部金丹道匪之變,是部同時被擾。事平,賑恤之。左翼有佐領八十,右翼有佐領九十,於諸旗為特多焉。


\end{pinyinscope}