\article{列傳二百四十七}

\begin{pinyinscope}
左寶貴弟寶賢等永山鄧世昌劉步蟾林泰曾等戴宗騫

左寶貴,字冠廷,山東費人。咸豐初,隸江南軍。嘗令當前敵,陣既接,旗兵中砲,殪,寶貴持其幟沖鋒入,大捷,繇是知名。獲苗沛霖,克金陵,頻有功。後以游擊從僧格林沁討捻,積勛至副將。光緒初,尚書崇實巡視奉天、吉林,奏自隨。既至,斬高希珍於土門,誅宋三好於石砬子。邊外東北廟溝金宮四構黨圖大舉,復捕治之,餘燼悉平,賜號鏗色巴圖魯,晉記名提督。授高州鎮總兵,仍留奉天。平朝陽教匪,賞黃馬褂、雙眼花翎,駐沈陽。

二十年,朝鮮亂起,日本進兵。朝議既決戰,衛汝貴、馬玉昆、豐紳阿各率所部往御之,寶貴自奉天來會,是為四大軍。慮海道梗,乃繞道自遼東行,渡鴨綠江入平壤。是時葉志超虛飾戰勝狀,電李鴻章入告,遂拜總統諸軍命。於是汝貴、玉昆軍南門外大同江,志超部將江自康軍北門外小山,寶貴任城守。未止舍,日軍猝至,寶貴與豐紳阿擊卻之。敵退龍岡,分道來攻,又敗之。志超乃聚全軍為嬰城計。

時寶貴扼玄武門,日軍大隊至。志超將潰圍北歸,寶貴不從,以兵守志超勿令逸。寶貴狃於捕馬賊之功,頗輕敵。日軍轝砲散置山巔,諜者以告,若弗聞。登城指麾,中砲踣,猶能言,及城下,始殞。其部將負尸開城走,遇日軍,又棄之,於是諸軍皆潰。事聞,贈太子少保,謚忠壯,予騎都尉兼一雲騎尉世職,子國楫襲。

弟寶賢、寶清先後於直隸、奉天剿匪陣亡。

永山,袁氏,漢軍正白旗人,黑龍江駐防,吉林將軍富明阿子,黑龍江將軍壽山弟。以廕授侍衛,歸東三省練軍。中日戰起,從將軍依克唐阿軍,率黑龍江騎旅駐摩天嶺。永山臨敵輒深入,為士卒先。與日軍戰數有功,連殲其將。既克龍灣,乘勝渡草河,規鳳凰,依克唐阿策襲其城,檄永山為軍鋒,偕壽山分率馬步隊深入攻之。抵一面山,距城八里,張左右翼,各據一坡以待。永山為右翼,尤得地勢。敵作散隊,伍伍什什冒死前,復以大隊橫沖我左翼。左翼潰,右翼亦不支,乃相繼退。永山獨為殿,遇伏,連受槍傷,洞胸踣,復強起督戰,大呼殺賊而逝。事聞,謚壯愍,予建祠奉天。

鄧世昌,字正卿,廣東番禺人。少有幹略,嘗從西人習布算術。既長,入水師學堂,精測量、駕駛。光緒初,管海東雲艦,徼循海口。日本窺臺灣,扼澎湖、基隆諸隘,補千總,調管振威艦。以捕海盜,遷守備。李鴻章治海軍,高其能,調北洋。從丁汝昌赴英購鐵艦,益詳練海戰術。八年,朝鮮內亂,復從汝昌泊仁川,為吳長慶陸軍後距。事寧,遷游擊,賜號勃勇巴圖魯。管揚威快艦,往來天津、朝鮮;冬寒冰沍,巡視臺、廈海防。尋充經遠、致遠、靖遠、濟遠四船營務處,兼致遠管帶。

十四年,臺灣生番畔,以副將從汝昌往討。戰埤南,毀其碉寨,擢總兵。時定海軍經制,借補中軍副將,而以汝昌為提督,其左右翼總兵則閩人林泰曾、劉步蟾也。汝昌故不習海戰,威令不行。獨世昌以粵人任管駕,非時不登岸,閩人咸嫉之。

二十年夏,日侵朝,絕海道。鴻章令濟遠、廣乙兩船赴牙山,遇日艦,先擊,廣乙受殊傷;轟濟遠,都司沈壽昌,守備楊建章、黃承勛中砲死。濟遠逃,日艦追之,管帶方柏謙豎白幟,追益亟,有水手發砲擊之,折日艦了樓,柏謙虛張勝狀,退塞威海東西兩口。世昌憤欲進兵,汝昌尼其行,不果。已而日監集大連灣,窺金州,我國海軍乃大發,泊鴨綠江大東溝,以鐵艦十當敵艦十有二。汝昌乘定遠居中,列諸船左右張兩翼。日艦魚貫進,據上風,汝昌令轟擊,距遠不能中。日艦小,運棹靈,倏分倏合,彈雨坌集,定遠被震,大纛僕。世昌見帥旗沒,慮軍心搖,亟取致遠纛豎之。戰良久,定遠擊沉其西京丸,我之超勇毀焉。

世昌乘致遠,最猛鷙,與日艦吉野浪速相當,吉野,日艦之中堅也。戰既酣,致遠彈將罄,世昌誓死敵。將士知大勢敗,陣稍亂,世昌大呼曰:「今日有死而已!然雖死而海軍聲威弗替,是即所以報國也!」眾乃定。世昌遂鼓輪怒駛,欲猛觸吉野與同盡,中其魚雷,鍋船裂沉。世昌身環氣圈不沒,汝昌及他將見之,令馳救。拒弗上,縮臂出圈,死之。其副游擊陳金揆同殉,全船二百五十人無逃者。經遠管帶總兵林永升、超勇管帶參將黃建寅、揚威管帶參將林履中並殞於陣。

事聞,世昌謚壯節,餘皆優恤。世昌既死,諸船或沉或逃,遂不復成軍。世昌臨戰以忠義相激勵,死狀尤烈,世與左寶貴並稱雙忠云。永升等,忠義有傳。

劉步蟾,侯官人。幼穎異,肄業福建船政學堂,卒業試第一。隸建威船,徼循南北洋資實練。同治十一年,會考閩、廣駕駛生,復冠其曹。自是巡歷海岸河港,所蒞輒用西法測量。臺灣地勢、番部風土尤諳習,為圖說甚晰。光緒改元,赴歐學槍砲、水雷諸技,還留福建,敘守備。以丁寶楨、李鴻章論薦,擢游擊,會辦北洋操防。十一年,赴德國購定遠艦。維時海軍初立,借才異地,西人實為管帶,步蟾副之。已而西人去,頗能舉其職。十四年,以參將赴歐領四快船歸,遷副將,賜號強勇巴圖魯,擢右翼總兵。

二十年,中日戰起,海軍浮泊大東溝。日艦至,督攝諸藝士御之,鏖戰三時許,沉敵艦三艘,運送銘軍八營,得以乘間登岸。論功,晉記名提督,易其勇號曰格洪額。明年,戰威海,中彈死。步蟾通西學,海軍規制多出其手。顧喜引用鄉人,視統帥丁汝昌蔑如也,時論責其不能和衷,致僨事。然華人明海戰術,步蟾為最先,雖敗挫,殺敵甚眾。上嘉其忠烈,詔優恤。

其左翼總兵林泰曾,亦籍侯官,同為船政學堂卒業生。管鎮遠,戰大東溝,發砲敏捷,士卒用命,撲救火彈甚力,機營砲位無少損,賜號霍春助巴圖魯。駛還威海,艦觸礁受傷,憤恨蹈海死。副將左翼中營游擊楊用霖、廣東大鵬協右營守備黃祖蓮並殉焉。優恤各如制。祖蓮等,忠義有傳。

戴宗騫,字孝侯,安徽壽州人。少以廩生治鄉團,捻酋苗沛霖數陷州,宗騫潛結各圩寨以攜貳其黨。同治初,謁李鴻章,上平捻十策,深器之,遂留參戎幕,積勛至知縣。十一年,治南運河堤工。時畿輔興水利,計臣慮餉詘,議裁兵。宗騫上書,略謂:「津沽為九河故道,漳、衛交匯,水菑衍溢。宜闢減河洩其勢,澌枝河分其漲,俾淮、練軍治之,則兵農合一,事半而功倍。」鴻章以其議上聞,遂命董其役,成稻田六萬餘畝。箸海上屯田志紀其事。

光緒六年,中俄失和,吳大澂被命佐吉林邊務,奏宗騫自隨。大澂兼攝屯政,宗騫為治道路,築砲臺,設江防,徙直、東流民,假予產業,分部護之。塞外灌莽千里,馬賊為民患,宗騫曰:「此屯政蠹也!」率將士步馳八九百里,獲渠率王林等駢誅之。又以緣邊荒墾,戶籍殘耗,客民渙居不相顧,因令屯聚一處,略仿內地保甲,杜絕奸宄。復設制造局、採金廠,行之期年,商民輻湊。大澂上其績狀,遷知府。

八年,徙防洋、蒲河兩海口。遭母憂歸,鴻章疏留,宗騫請終制,弗許。時興海軍,練水師,闢軍港,檄防威海。十三年,詣軍所,壁金線頂山,分鞏軍駐南岸,綏軍駐北岸。明年,建兩岸海臺各三,南曰趙北嘴、鹿角嘴、龍廟嘴,北曰北山嘴、黃泥崖、祭祀臺。後路分築陸臺四,南岸口較闊,更建日島地阱砲臺,屹然為東防重鎮。十七年,校閱海軍禮成,論功晉道員。威海地瘠,士氣衰,更斥貲立義塾,延名師,至是始聞謳誦聲。

二十年夏,日艦來攻,率師御之,傷其艦四艘,再至再敗之。既而旅順、大連相繼淪沒,威海勢益孤,電請北洋、山東益師,久弗應。其冬,連失文登、寧海。時宗騫守北岸,分統劉超佩守南岸,宗騫與約,寇至互相應。歲除,大風雪,戰橋頭集,綏軍大困,銳身救之出。

逾歲,日軍至,輒敗去,折而南。宗騫往援,而超佩蹌踉遁,三臺拱手讓敵,反訴巡撫李秉衡,誣宗騫背約。宗騫抗辯,願復三臺贖罪。乃募敢死士奪還二臺,唯龍廟嘴未復。日軍倏大集,二臺仍不守,宗騫退歸,登祭祀臺。所部卒譁變,宗騫佯弗省,行數武,槍齊發,材官追斬一人,眾散走。宗騫既登,乃無一從者。夜宿藥庫,丁汝昌詣籌戰守策,宗騫曰:「綏、鞏軍已西去,孤臺危棘,恐資敵。」汝昌令毀臺,強掖之下。宗騫念南北各有地阱臺,此其勢尚可為,乃詣劉公島就副將張德山。德山無戰守志,宗騫飲金死,威海師遂熸。鴻章以死事聞,詔優恤。復以秉衡請,贈太常寺卿。

論曰:中東之戰,陸軍皆遁,寶貴獨死平壤;海軍皆降,世昌獨死東溝。中外傳其壯節,並稱「雙忠」。及日兵入奉,永山獨死鳳城,敵遂長驅進矣。旅、大既失,威海勢孤,步蟾、宗騫皆先後誓死。士氣如此,豈遂不可一戰?此主兵者之責。五人雖敗,猶有榮焉!


\end{pinyinscope}