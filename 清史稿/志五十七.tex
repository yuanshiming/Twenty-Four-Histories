\article{志五十七}

\begin{pinyinscope}
禮一(吉禮一)

自虞廷修五禮,兵休刑措。天秩雖簡,鴻儀實容。沿及漢、唐,訖乎有明,救敝興雅,咸依為的。煌煌乎,上下隆殺以節之,吉兇哀樂以文之,莊恭誠敬以贊之。縱其間淳澆世殊,要莫不弘亮天功,雕刻人理,隨時以樹之範。故群甿蒸蒸,必以得此而後足於憑依,洵品匯之璣衡也。斟之酌之,損之益之,修明而講貫之,安見不可與三代同風!

世祖入關,順命創制,規模閎遠。順治三年,詔禮臣參酌往制,勒成禮書,為民軌則。聖祖歲御經筵,纂成日講禮記解義,敷陳雖出群工,闡繹悉遵聖訓。高宗御定三禮義疏,網羅議禮家言,折衷至當,雅號鉅制。若皇朝三通、大清會典,其經緯禮律,尤見本原。

至於專書之最著者:一曰大清通禮,乾隆中撰成,道光年增修;一曰皇朝禮器圖式,曰祭器、曰儀器、曰冠服、曰樂器、曰鹵簿、曰武備;一曰滿洲祭神祭天典禮,其始關外啟蓽,崇祭天神暨群祀祖禰,意示從儉。凡所紀錄,悉用國語、國書。入關後,有舉莫廢。逮高宗時,依據清文,譯成四卷。祭期、祭品、儀注、祝辭。與夫口耳相傳,或小有異同者,並加釐訂,此國俗特殊之祀典也。德宗季葉,設禮學館,博選耆儒,將有所綴述。大例主用通禮,仿江永禮書例,增曲禮一目。又仿宋太常因革禮例,增廢禮、新禮二目,附後簡。未及編訂,而政變作矣。

其祀典之可稽者,初循明舊,稍稍褒益之。堂子之祭,雖於古無徵,然昭假天神,實近類祀。康熙間,以禁中祭上帝、大享殿合祀天地日月及群神、太廟階下合祭五祀非古制,詔除之。又罷禘祭,專行祫祭。高宗修雩祀,廢八蠟,建兩郊壇宇,定壇廟祭器,舉廢一惟其宜。宣宗遺命罷郊配祔廟,文宗限以五祖三宗,慮至深遠。穆宗登遐,禮臣援奉先殿增龕座例,主升祔。議者病簡略,然亦迫於勢之不容已耳。光緒間,依高宗濮說辨,稱醇親王為本生考,立廟別邸,祀以天子禮。恩義兼盡,度越唐、明遠矣。

若夫郊廟大祀,無故不攝,誠敬仁孝,永垂家法,尤舉世所推。今為考諸成憲,循五禮序,條附支引,凡因襲變創,所以因時而制宜者,悉臚其要於編。

壇壝之制神位祭器祭品玉帛牲牢之數祀期齋戒祝版祭服祭告習儀陪祀

五禮,一曰吉禮。凡國家諸祀,皆屬於太常、光祿、鴻臚三寺,而綜於禮部。惟堂子元日謁拜,立桿致祭,與內廷諸祀,並內務府司之。

清初定制,凡祭三等:圜丘、方澤、祈穀、太廟、社稷為大祀。天神、地祇、太歲、朝日、夕月、歷代帝王、先師、先農為中祀。先醫等廟,賢良、昭忠等祠為群祀。乾隆時,改常雩為大祀,先蠶為中祀。咸豐時,改關聖、文昌為中祀。光緒末,改先師孔子為大祀,殊典也。天子祭天地、宗廟、社稷。有故,遣官告祭。中祀,或親祭、或遣官。群祀,則皆遣官。

大祀十有三:正月上辛祈穀,孟夏常雩,冬至圜丘,皆祭昊天上帝;夏至方澤祭皇地祇;四孟享太廟,歲暮祫祭;春、秋二仲,上戊,祭社稷;上丁祭先師。中祀十有二:春分朝日,秋分夕月,孟春、歲除前一日祭太歲、月將,春仲祭先農,季祭先蠶,春、秋仲月祭歷代帝王、關聖、文昌。群祀五十有三:季夏祭火神,秋仲祭都城隍,季祭砲神。春冬仲月祭先醫,春、秋仲月祭黑龍、白龍二潭暨各龍神,玉泉山、昆明湖河神廟、惠濟祠,暨賢良、昭忠、雙忠、獎忠、褒忠、顯忠、表忠、旌勇、睿忠親王、定南武壯王、二恪僖、弘毅文襄勤襄諸公等祠。其北極佑聖真君、東岳都城隍,萬壽節祭之。亦有因時特舉者,視學釋奠先師,獻功釋奠太學,御經筵祗告傳心殿。其岳、鎮、海、瀆,帝王陵廟,先師闕里,元聖周公廟,巡幸所蒞,或親祭,或否。遇大慶典,遣官致祭而已。各省所祀,如社稷,先農,風雷,境內山川,城隍,厲壇,帝王陵寢,先師,關帝,文昌,名宦、賢良等祠,名臣、忠節專祠,以及為民御災捍患者,悉頒於有司,春秋歲薦。至親王以下家廟,祭始封祖並高、曾、祖、禰五世。品官逮士庶人祭高、曾、祖、禰四世。其餘或因事,或從俗,第無悖於祀典,亦在所不禁。此其概也。

若夫壇壝神位,祭獻品物,齋戒告虔,及一切度數節文,詳其異同,識其顛末,無遺無衣復,庶覽者可考而知已。

壇壝之制天聰十年,度地盛京,建圜丘、方澤壇,祭告天地,改元崇德。天壇制圓,三成,上成九重,周一丈八尺;二成七重,周三丈六尺;三成五重,周五丈四尺:俱高三尺。垣周百十有三丈。地壇制方,二成,上成方六丈,高二尺;下成方八丈,高二尺四寸。垣周百三十有三丈。制甚簡也。世祖奠鼎燕京,建圜丘正陽門外南郊,方澤安定門外北郊,規制始拓。圜丘南鄉,三成,上成廣五丈九尺,高九尺;二成廣九丈,高八尺一寸;三成廣十有二丈,高如二成。甃磚合一九七五陽數。陛四出,各九級。欄楯柱覆青琉璃。內壝圓,周九十七丈七尺五寸,高八尺一寸。四面門各三,門柱各二。燔柴爐、瘞坎各一。外壝方,周二百四丈八尺五寸,高九尺一寸。四門如內壝。北門後為皇穹宇,南鄉,制圓。八柱環轉,重簷金頂。基周十三丈七寸,高九尺。陛三出,級十有四。左右廡各五楹,陛一出,七級。殿廡覆瓦俱青琉璃。圍垣周五十六丈六尺八寸,高丈有八寸。南設三門。外壝門外北神庫、神廚各五楹,南鄉。井亭一。其東為祭器、樂器、椶薦諸庫。又東為井亭、宰牲亭。壇內垣北圓,餘皆方。門四:東泰元,南昭亨,西廣利,北成貞。成貞北為大享殿。壇圓,南鄉。內外柱各十有二,中龍井柱四。金頂,簷三重,覆青、黃、綠三色琉璃。基三成,南北陛三出,東西陛一出,上二成各九級,三成十級。東西廡二重,前各九楹,後各七楹。前為大享門,上覆綠琉璃,前後三出陛,各十有一級。東南燔柴爐、瘞坎,制如圜丘。內壝周百九十丈七尺二寸。門四,北門後為皇乾殿,南鄉,五楹,覆青琉璃。陛五出,各九級。東磚門外長廊七十二,聯簷通脊,北至神庫、井亭。又東北宰牲亭,薦俎時避雨雪處也。壝外圍垣東、西、北各有門,南接成貞。又西北曰齋宮,東鄉,正殿五楹,陛三出,中級十有三,左右各十五。左設齋戒銅人,右設時辰牌。後殿五楹,左右配殿各三楹。內宮墻方百三十三丈九尺四寸。中三門,左右各一。環以池,跨石梁三。東北鐘樓一,外宮墻方百九十八丈二尺二寸,池梁如內制。廣利門外西北為神樂觀,東鄉。中凝禧殿,五楹。後顯佑殿,七楹。西為犧牲所,南鄉。又西為鐘樓,其大享殿圍垣南接圜丘,東、西轉北為圓形。內垣高一丈一尺,址厚九尺,頂厚七尺,周千二百八十六丈一尺五寸。外垣高一丈一尺五寸,址厚八尺,頂厚六尺,周千九百八十七丈五尺。西鄉門二,南北並列焉。乾隆八年,修齋宮,改神樂觀為所。十二年,修內外垣,改築圜丘,規制益拓。上成徑九丈,二成十五丈,三成二十一丈,一九三五三七,皆天數也。通三成丈四十有五,符九五義。量度準古尺,當營造尺八寸一分,又與九九數合。壇面甃磚九重,上成中心圓面,外環九重,磚數一九累至九九。二三成以次遞加。上成每面各十有八,二成各二十七,三成各四十五,並積九為數,四乘之,綜三百有六十,以應周天之度。其高上成五尺七寸,二成五尺二寸,三成五尺。欄、柱、階級並準今尺。古今尺度嬴縮稍差,用九則一。復改壇面為艾葉青石,皇穹宇臺面墁青白石,大享殿外壇面墁金磚。壇內殿宇門垣俱青琉璃。十六年,更名大享殿曰祈年。覆簷門廡壇內外壝垣並改青琉璃,距壇遠者如故。尋增天壇外垣南門一,內垣鐘鼓樓一,嗣是祭天壇自新南門入,祭祈年殿仍自北門入。二十年,改神樂所為署。五十年,重建祈穀壇配殿。光緒十五年,祈年殿災,營度仍循往制云。

方澤北鄉,周四十九丈四尺四寸,深八尺六寸,寬六尺,祭日中貯水。二成,上成方六丈,二成方十丈六尺,合六八陰數。壇面甃黃琉璃,每成陛四出,俱八級。二成南列岳鎮五陵山石座,鏤山形;北列海瀆石座,鏤水形:俱東西鄉。內壝方二十七丈二尺,高六尺,厚二尺。正北門三,石柱六。東、西、南門各一,石柱二。北門外西北瘞坎一。外壝方四十二丈,高八尺,厚二尺四寸。門制視內壝。南門後皇祇室,五楹,北鄉。垣周四十四丈八尺,高一丈一尺。正門一,外壝西門外,神庫,神廚,祭器、樂器諸庫,井亭,宰牲亭在焉。西北曰齋宮,東鄉。正殿七楹,陛五出,中九級,左右俱七級;南北陛一出,各七級。左右配殿各七楹。宮墻周百有十丈二尺。門三,東鄉。東北鐘樓一,壇內垣周五百四十九丈四尺,北、西門各三,東、南門各一。外垣周七百六十五丈。西鄉門三。雍正八年,重建齋宮,制如舊。乾隆十四年,以皇祇室用綠瓦乖黃中制,諭北郊壇磚壝瓦改用黃。明年,改築方澤墁石,壇面制視圜丘。上成石循前用六六陰數,縱橫各六,為三十六。其外四正四隅,均以八八積成,縱橫各二十四。二成倍上成,八方八八之數,半徑各八,為六八陰數,與地耦義符。尋建東、西、南壝門外南、北瘞坎各二。又天、地二壇,立陪祀官拜石如其等。

闕右社稷壇,制方,北鄉。二成,高四尺。上成方五丈,二成方五丈三尺。陛四出,各四級。上成土五色,隨其方覆之。內壝方七十六丈四尺,高四尺,厚二尺,飾色如其方。門四,柱各二。壝西北瘞坎二。北拜殿,又北戟門,楹各五,陛三出。外列戟七十二,其西南神庫、神廚在焉。壇垣周百五十三丈四尺,覆黃琉璃。北三門,東、西、南各一門。西門外宰牲亭一、井一。西南為奉祀署。壇東北正門一,左右門各一,俱東鄉,直闕右門,乘輿躬祭所出入也。東南為社稷街。乾隆二十一年,徙瘞坎壇外西北隅。舊制壝垣用五色土,至是改四色琉璃磚瓦。及省社稷壇高二尺一寸,方廣二丈五尺,制殺京師十之五雲。

朝日壇在朝陽門外東郊,夕月壇在阜成門外西郊,俱順治八年建。制方,一成,陛四出。日壇各九級,方五丈,高五尺九寸。圓壝,周七十六丈五尺,高八尺一寸,厚二尺三寸。壇垣前方後圓,周二百九十丈五尺。月壇各六級,方四丈,高四尺六寸。方壝,周九十四丈七尺,高八尺,厚二尺二寸。壇垣周二百三十五丈九尺五寸。兩壇具服殿制同。燎爐,瘞坎,井亭,宰牲亭,神庫,神廚,祭器、樂器諸庫咸備。其牌坊曰禮神街。雍正初,更名日壇街曰景升,月壇街曰光恆。乾隆二十年,修建壇工,依天壇式。改內垣土墻甃以磚,其外垣增舊制三尺。光緒中,改日壇面紅琉璃,月壇面白琉璃,並覆金磚。

天神、地祇、先農三壇制方,一成,陛皆四出,在正陽門外。先農壇位西南,周四丈七尺,高四尺五寸。東南為觀耕臺,耕耤時設之。前耤田,後具服殿。東北神倉,中廩制圓。前收穀亭,後祭器庫。內垣南門外,神祇壇在焉。神壇位東,方五丈,高四尺五寸五分。北石龕四,鏤雲形,分祀云、雨、風、雷。祇壇位西,廣十丈,縱六丈,高四尺。南石龕五,鏤山水形。分祀岳、鎮、海、瀆。二壇方壝,俱周二十四丈,高五尺五寸。正門分南、北,餘如日、月壇。又內垣東門外北齋宮,五楹,後殿,配殿,茶、膳房具焉。乾隆時,更命齋宮曰慶成宮。壇外垣周千三百六十八丈。南、北門二,東鄉,南入先農壇,北入太歲殿。殿七楹,東、西廡各十有一。其前曰拜殿,燎爐一。

先蠶壇,乾隆九年,建西苑東北隅,制視先農。徑四丈,高四尺,陛四出。殿三楹,西鄉。東採桑臺,廣三丈二尺,高四尺,陛三出。前為桑園臺,中為具服殿、為繭館,後為織室。有配殿,環以宮墻。墻東浴蠶河,跨橋二。橋東蠶署三,蠶室二十七,俱西鄉。外垣周百六十丈,各省先農壇高廣視社稷,餘如制。

神位、祭器、祭品、玉、帛、牲牢之數神位,圜丘第一成,正位昊天上帝,南鄉。配位八,首太祖訖宣宗,東西鄉。凡位皆施幄。第二成從位,東大明,次星辰。西夜明,次雲、雨、風、雷。常雩如冬至、大祀、大雩,有從無配。祈穀位次視圜丘第一成,無幄。方澤第一成,正位皇地祇,北鄉,配列祖、列宗,東西鄉。第二成從位,東五岳,啟運、隆業、永寧三山,次四海。西五鎮,天柱、昌瑞二山,次四瀆。因事祗告天地,不設配從位。順治十七年,合祀大享殿,其正位左天帝,右地祇,南鄉。東太祖,西太宗,配之。從祀十二壇,大明位東,星辰、五岳、啟運、四海、太歲、名山大川次之。夜明位西,雲、雨、風、雷、五鎮、天柱、隆業、四瀆、帝王、天下神祇次之。社稷壇中植石主,別設神牌,正位。東大社,西大稷。北鄉。東配后土句龍氏,西後稷氏。無幄。壇下龕用木。日壇東大明,無幄。月壇正位夜明,配北斗二十八宿、周天星辰,共一幄。天神壇正中,左雲師,次風伯,右雨師,次雷師,南鄉。地祇壇正中五岳,右五鎮,次四海,左五陵,次四瀆,北鄉。右旁京師山川,左旁天下山川。無幄。各省府、州、縣神祇位次,正中雲、雨、風、雷,左山川,右城隍。其郊壇神位,皇穹宇、皇乾殿、皇祇室奉之。神祇、社稷、日月神位,神庫奉之,祭時並移壇所。太廟、奉先殿神牌置寢室龕位,祭時移前殿寶座。至傳心殿、歷代帝王、先師各廟龕位,或分或合,無恆制。

祭器,圜丘正位,爵三,登一,簠、簋二、籩、豆十,篚、俎、尊各一,配從同。惟大明、夜明■D9三十,夜明鉶皆二,雲、雨、風、雷視夜明。常雩如冬至、大祀、大雩,正、從位俱籩六、豆二,告祭正位同。方澤祈穀壇正、配位,暨方澤從位,並視圜丘。■D9、鉶視夜明。太廟時享,帝、後同案,俱爵三,簠、簋二,籩、豆十有二,登、鉶、篚、俎各一。尊前後殿同。祫祭如時享,東廡每案爵三,簠、簋二,籩、豆十,鉶、篚、俎各一,尊共八案,分二座,爵、鉶倍之。西廡同,惟簠、簋一,籩、豆四。告祭,中、後殿俱籩六,豆二。社稷壇大社、大稷,俱玉爵一,陶爵二,登、篚、俎、尊各一,鉶、簠、簋各二。配位同,惟爵皆用陶。祈告,籩六,豆二。直省祭社稷,爵六,鉶一,籩、豆四,簠、簋、篚、俎、尊各一,如大社稷。日壇、月壇、先農、先蠶壇,俱爵三,■D9三十,籩、豆十,鉶、簠、簋各二,登、篚、俎、尊各一。直省祭先農如祭社稷。天神壇四案,凡祈祀爵共十二,各用籩六、豆二、尊一、篚一。地祇壇如之,惟案七共爵二十七耳。報祀神祇,每案與日壇同,惟無■D9。直省祭神祇,爵三,籩、豆四,鉶、簠、簋各二,篚、俎、尊各一。時巡祭嶽鎮、海瀆同。報祀增鉶一,因事遣祭仍用二。餘同。有司致祭無登、■D9。太歲殿準先農,報祀亦如之。祈祀,籩六、豆二,不羞俎。先師正位視圜丘,惟用鉶二。四配視正位,惟用籩、豆八,無登。十二哲位,各爵三,鉶一,簠、簋一,籩、豆四,篚、俎、尊共用二。兩廡二位同案,位一爵,凡獻爵六,共篚二,尊、俎俱各六,簠、簋各一,籩、豆各四。視學、釋奠同。

乾隆三十三年,頒內府周鼎、尊、卣、罍、壺、簠、簋、觚、爵各一,陳列大成殿,用備禮器。崇聖祠正位五案,案設爵三,籩、豆八,鉶、簠、簋各二,篚、俎、尊各一。配位五案,設爵三,籩、豆四,鉶、篚、簠、簋各一,共俎二,尊二。兩廡三案,案各與配位同,惟共篚為二。

光緒三十二年,增先師正位籩、豆為十二,崇聖祠籩、豆為十,闕里、直省文廟暨崇聖祠祭器視太學。歷代帝王正位十六案,案設爵三,登一,鉶、簠、簋各二,籩、豆十,篚一,共俎七,尊七。兩廡配位二十案,案設爵十二,鉶二,籩、豆四,簠、簋、篚各一,共俎四,尊四。傳心殿正位九案,案設爵、尊各三,鉶、篚各一,籩、豆二。配位二案,案設爵三,籩、豆二,鉶、篚、尊各一。關帝、文昌帝君俱爵三,籩、豆十,鉶、簠、簋各二,登、篚、俎、尊各一,惟後殿籩、豆八。各省準京式。先醫三皇位,位設爵三,籩、豆十,簠、簋、篚、俎、尊各一。兩廡六案,案設簋、簠一,篚、尊各二,籩、豆四,共爵六。都城隍爵三,籩、豆十,鉶、簠、簋各二,篚、俎、尊各一。火神、東嶽廟,俱果盤五,篚、俎、尊各一。黑龍潭、玉泉山、昆明湖各龍神祠、惠濟祠、河神廟俱三案,案設爵三,簠、簋二,籩、豆十,篚、俎、尊各一。

初沿明舊,壇廟祭品遵古制,惟器用瓷。雍正時,改範銅。乾隆十三年,詔祭品宜法古,命廷臣集議,始定制籩編竹,絲絹里,魨漆。郊壇純漆,太廟採畫。其豆、登、簠、簋,郊壇用陶,太廟惟登用之,其他用木,魨漆,飾金玉。鉶範銅飾金。尊則郊壇用陶。太廟春犧尊、夏象尊、秋著尊、冬壺尊、祫祭山尊,均範銅。祀天地爵用匏,太廟玉,兩廡陶。社稷正位,玉一陶二。配位純陶。又豆、登、簠、簋、鉶、尊皆陶。日、月、先農、先蠶亦如之。帝王、先師、關帝、文昌及諸祠,則皆用銅。凡陶必辨色,圜丘、祈穀、常雩青,方澤、社稷、先農黃,日壇赤,月壇白。太廟陶登,黃質採飾,餘俱白。盛帛用竹篚,魨色如其器。載牲用木俎,魨以丹漆。毛血盤用陶,色亦如其器。嘉慶十九年,定太廟簠、簋、豆與凡祭祀竹籩,三歲一修。光緒三十二年,先師爵改用玉。

祭品,凡籩、豆之實各十二,籩用形鹽、薨魚、棗、慄、榛、菱、芡、鹿脯、白餅、黑餅、糗餌、粉餈,豆用韭菹、醓醢、菁菹、鹿醢、芹菹、兔醢、筍菹、魚醢、脾析、豚拍、酏食、糝食。用十者,籩減糗餌、粉餈,豆減酏食、糝食。用八者,籩減白、黑餅,豆減脾析、豚拍。用四者,籩止實形鹽、棗、慄、鹿脯,豆止實菁菹、鹿醢、芹菹、兔醢。籩六者,用鹿脯、棗、榛、葡萄、桃仁、蓮實。豆二者,止用鹿醢、兔醢。登一,太羹。鉶二,和羹。簠二,稻、粱。簋二,黍、稷。

玉、帛、牲牢:玉六等,上帝蒼璧,皇地祇黃琮,大社黃珪,大稷青珪,朝日赤璧,夕月白璧。舊制,社稷壇春秋常祀用玉,禱祀則否。乾隆三十四年,會天旱禱雨,諭曰:「玉以芘廕嘉穀,俾免水旱偏災,特敕所司用玉將事。」自此為恆式。帛七等:曰郊祀制帛,南北郊用之。上帝青十二,地祇黃一。曰禮神制帛,社稷以下用之。社稷黑四,大明赤一,夜明白一,日月同。星辰斗宿白七,青、赤、黃、黑各一。天神、雲、雨、風、雷,青、白、黃、黑各一,方澤從位,岳鎮各五,五色。五陵山白五。四海隨方為色。四瀆黑四。地祇黃二,青、赤各三,黑七、白十二。先農、先蠶俱青一,先師正、配位,十二哲,兩廡,崇聖祠正位,東、西廡,俱各一用白。帝王各位、關帝、文昌正位、後殿,太歲正位,北極佑聖真君、東岳都城隍亦如之。惟先醫正位三,崇聖配位四,太歲兩廡十二,火神赤一。曰告祀制幣,祈報祭告用之。祈穀、雩祀、告祀圜丘俱青一,祭告方澤黃一。曰奉先制幣,郊祀配位、太廟用之,圜丘、方澤配位各一,太廟帝後每位一。曰展親制幣,親王配饗用之,太廟東廡位各一。曰報功制幣,功臣配饗用之,太廟西廡位各一。三者俱白,昭忠等祠同,並織滿、漢文字。曰素帛,帝王廟兩廡位各一,先醫廟兩廡共四,餘祀亦尚素。牲牢四等:曰犢,曰特,曰太牢,曰少牢。色尚騂或黝。圜丘、方澤用犢,大明、夜明用特,天神、地祇、太歲、日、月、星辰、雲、雨、風、雷、社稷、岳鎮、海瀆、太廟、先農、先蠶、先師、帝王、關帝、文昌用太牢。太廟西廡,文廟配哲、崇聖祠、帝王廟兩廡,關帝、文昌後殿,用少牢。光緒三十二年,崇聖正位改太牢。直省神祇、社稷、先農、關帝、先醫配位暨群祀用少牢。火神、東嶽、先醫正位,都城隍,皆太牢。太牢:羊一、牛一、豕一,少牢:羊、豕各一。

大祀入滌九旬,中祀六旬,群祀三旬。大祀天地,前期五日親王視牲,二日禮部尚書省牲,一日子時宰牲。帝祭天壇,前二日酉時宰之,太廟、社稷、先師前三日,中祀前二日。禮部尚書率太常司省牲,前一日黎明宰牲。惟夕月屆日黎明宰之。令甲,察院、禮部、太常、光祿官監宰,群祀止太常司行。乾隆十七年,定大祀、中祀用光祿卿監宰。初,郊壇大祀,帝前期宿齋宮,視壇位、籩豆、牲牢。乾隆七年,更定前一日帝詣圜丘視壇位,分獻官詣神庫視籩豆,神廚視牲牢。尋定視壇位日,親詣皇穹宇、皇乾殿上香。故事,省視籩豆牲牢,或臨視,或否。三十五年,定遣官將事,自後以為常。

祀期郊廟祭祀,祭前二歲十月,欽天監豫卜吉期。前一歲正月,疏卜吉者及諸祀定有日者以聞。頒示中外。太常寺按祀期先期題請,實禮部主之。世祖纘業,詔祭祀各分等次,以時致祭。自是大祀、中祀、群祀先後規定祀期,著為例。嘉慶七年,復定大、中祀遇忌辰不改祀期。咸豐中,更定關帝、文昌春秋祀期不用忌辰。其祭祀時刻,順治十三年,詔祭天、地五鼓出宮,社稷、太廟並黎明。康熙十二年,依太宗舊制,壇廟用黎明,夕月用酉時。嘉慶八年,諭祭祀行禮,當在寅卯間,合禮經質明將事古義。凡親行大祀,所司定時刻,承祭官暨執事陪祭者祗候,率意遲早者,御史糾之。

齋戒順治三年,定郊祀齋戒儀。八年,定大祀三日、中祀二日公廨置齋戒木牌。祀前十日,錄齋戒人名冊致太常,屆日不讞刑獄,不宴會,不聽樂,不宿內,不飲酒、茹葷,不問疾、吊喪,不祭神、掃墓。有疾與服勿與。大祀、中祀,太常司進齋戒牌、銅人置乾清門黃案。大祀前三日,帝致齋大內,頒誓戒。辭曰:「惟爾群臣,其蠲乃心、齊乃志,各揚其職。敢或不共,國有常刑。欽哉勿怠!」前祀一日,徹牌及銅人送齋宮,帝詣壇齋宿。十四年祀圜丘,致齋大內二日,壇內齋宮一日。陪祭官齋於公署,圜丘齋於壇。

雍正五年,遣御史等赴壇檢視。九年,詔科道遇祀期齋戒。明年,仿明祀牌制制齋牌,敕陪祭官懸佩,防褻慢。乾隆四年,禮臣奏,郊壇大祀,太常卿先期四日具齋戒期,進牌及銅人置乾清門二日、齋宮一日。太廟、社稷,置乾清門三日。中祀,前三日奏進,置乾清門二日。並祭日徹還。後饗先蠶,奏進亦如之。惟由內侍置交泰殿三日。

七年,定郊祀致齋,帝宿大內二日,壇內齋宮一日。王公居府第,餘在公署,俱二日。赴壇外齋宿一日。若遣官代祭,王公不與。祭太廟、社稷,王公百官齋所如前儀,俱三日。祭日、月、帝王、先師、先農,王公齋二日,遣代則否。後饗先蠶,齋二日,公主、福晉、命婦陪祀者,前二日致齋。十二年,詔郊祀、祈穀、大雩,祭日宣誓戒,陪祀者集午門行禮,符古者百官受戒遺意。既有司具儀上,行之。尋罷。惟嚴敕大臣齋宿公所,領侍衛內大臣等齋宿紫禁城,違則治罪。

初,齋宮致齋鳴鼓角,十四年諭云:「齋者耳不聽樂,孔子曰:『三日齋,一日用之,猶恐不敬,二日伐鼓何居?』言不敢散其志也。吹角鼓鼙,以壯軍容,於義未協,不當用也。」遂寢。

十九年,敕群臣書制辭於版,前期三日,陳設公堂,俾有所警。嘉慶十三年,諭誡齋戒執事暨查齋監禮者,循舊章,肅祀典。宣統初,監國攝政王代行,帝宮內致齋,停進齋戒牌及銅人。

祝版以木為之,圜丘、方澤方一尺五寸,徑八寸四分,厚三分。祈穀壇方一尺一寸,徑一尺,厚如之。太廟後殿方一尺二寸,徑八寸四分。前殿方二尺,徑一尺一寸,厚並同徑。常雩,日、月壇,社稷壇與太廟後殿同。中祀、群祀方徑各有差。天壇青紙青緣硃書,地壇黃紙黃緣墨書,月壇、太廟、社稷白紙黃緣墨書,日壇硃紙硃書,群祀白紙墨書不加緣。太常司令祝版官先期褾飾,祀前二日昧爽送內閣,授中書書祝辭,大學士書御名,餘祀太常司自繕。

凡親祭,先二日太常卿奏請,前一日閱祝版。圜丘、祈穀、常雩御太和殿,方澤、太廟、社稷御中和殿。祝案居正中少西,案設羊角鐙二,視版日,案左楹東置香亭,右楹西置奉版亭、奉玉帛香亭。屆時太常卿詣乾清門啟奏,帝出宮詣案前。閱畢,行一跪三拜禮。贊禮郎徹褥,寺卿韜版,導帝至香亭前,拜跪如初禮。司祝奉版薦黃亭送祭所,庋神庫。大祀遣代,停止祝版具奏。中祀、群祀,寺官赴內閣徑請送祭所,不具奏。其視玉、帛、香如閱祝版儀。

祭服圜丘、祈穀、雩祀,先一日,帝御齋宮,龍袍袞服。屆期天青禮服。方澤禮服明黃色,餘祀亦如之。惟朝日大紅,夕月玉色。王公以下陪祀執事官咸朝服。嘉慶九年,定祀前閱祝版執事官服色制,南郊祈穀、常雩、歲暮祫祭、元旦、萬壽、告祭太廟,蟒袍補褂,罷朝服。社稷、時享太廟,服補服。十一年,諭郊壇大祀若遇國忌,仍御禮服,禮成還宮更素服。十九年,諭郊祀遇國忌,前一日閱祝版,帝服龍袍龍褂,執事官蟒袍補服。大祀、中祀,帝龍褂,執事官補服。著為令。二十三年,定制大祀齋期遇國忌,悉改常服。中祀則限於承祭官及陪祀、執事官,餘素服如故。二十五年,諭大祀親祭或遣官致祭遇國忌,齋期一依向例,中祀親祭同。其遣官致祭,與執事、陪祀官常服掛珠,否則仍素服。

祭告凡登極授受大典,上尊號、徽號,祔廟,郊祀,萬壽節,皇太后萬壽節,冊立皇太子,先期遣官祗告天地、太廟、社稷。致祭嶽鎮、海瀆、帝王陵寢、先師闕里、先師。改大祀亦如之。大婚冊立皇后,祗告天地、太廟。尊封太妃、冊封皇貴妃及貴妃,祗告太廟後殿奉先殿。追上尊謚廟號、葬陵,祗告天地、社稷、太廟後殿、奉先殿,並致祭陵寢、后土、陵山。親征命將,祗告天地,太廟,社稷,太歲,火砲、道路諸神。凱旋奏功,祗告奉先殿,致祭陵寢,釋奠先師,致祭嶽鎮、海瀆、帝王陵廟、先師闕里。謁陵、巡狩,並祗告奉先殿,回鑾亦如之。巡幸所蒞,親祭方岳。其所未蒞者,命疆臣選員遍祭嶽、鎮、海、瀆、所過名山大川。其祭文香帛,遣使自京齎送。帝王陵寢、聖賢忠烈暨名臣祠墓,凡在三十里內,遣官祭之。歲暮祫祭,功臣配饗,祗告太廟中殿、後殿。監國攝政,並遣官祭告太廟。耕耤田,祗告奉先殿。御經筵,祗告奉先殿、傳心殿,修建郊壇、太廟、奉先殿,祗告天地、太廟、社稷。興工、合龍,祭后土、司工諸神。迎吻,祭琉璃窯神暨各門神。歲旱祈雨,祗告天神、地祇、太歲。越七日,祭告社稷。三請不雨,始行大雩。凡告祀,不及配位從壇。至為元元祈福,則遣大臣分行祭告,頒冊文香帛,給御蓋一,龍纛御仗各二,蓋猶喬嶽翕河茂典云。

習儀凡大祀前四十日,中祀前三十日,每旬三、六、九日,太常卿帥讀祝官、贊禮郎暨執事、樂舞集神樂署,習儀凝禧殿。故事,祭祀先期,太常寺演禮壇廟中。雍正九年諭曰:「是雖義取嫻熟,實乖潔齊嚴肅本旨也。」乃停前一日壇廟演禮。其前二日凝禧殿如故。饗太廟,以王公一人監視宗室、覺羅官。祀先師,祭酒、司業監視國子師生,同日習樂殿庭,令樂部典樂監視亦如之。謁陵寢,讀祝官等亦遇三、六、九日習儀皇陵。又歲暮將祭享,選內大臣打莽式,例演習於禮曹。時議謂發揚蹈厲,為公庭萬舞變態雲。

陪祀順治時,詔陪祀官視加級四品以上。康熙二十五年,以喧語失儀,諭誡陪祀官毋慢易。尋議定論職不論級。郊壇陪祀,首公,訖阿達哈哈番,佐領。文官首尚書,訖員外郎,滿科道,漢掌印給事中。武訖游擊。祭太廟、社稷、日月、帝王廟,武至參領,文至郎中,餘如前例。御史、禮曹並糾其失儀者。既以浙江提督陳世凱請,文廟春秋致祭,允武官二品以上陪祀。三十九年,申定陪祀不到者處分。乾隆初元,定陪祀祗候例,祭太廟,俟午門鳴鼓;祭社稷,俟午門鳴鐘;祭各壇廟,俟齋宮鐘動:依次入,鵠立,禁先登階。並按官品制木牌,肅班序。七年,定郊廟、社稷赴壇陪祀制,遣官代行,王公內大臣等不陪祀,餘如故。明年,定郊祭前一日申、酉時及祭日五鼓,禮部、察院官赴壇外受職名,餘祀止當日收受。二十七年歲杪,諭通覈陪祀逾三次不到者,分別議懲。咸豐十年,諭朝日陪祀無故不到或臨時稱疾,並處罰。光緒九年,申定祗候例,大祀夜分、中祀雞初鳴,朝服蒞祭所。


\end{pinyinscope}