\article{列傳十四}

\begin{pinyinscope}
常書弟揚書常書子察哈喇孫葉璽曾孫辰布祿察哈喇子富喇克塔揚書子達爾漢達爾漢子鄂羅塞臣康果禮弟喀克都哩哈哈納哈哈納弟綽和諾綽和諾從子富喀禪葉克書葉克書子道喇博爾晉子特錦孫瑪沁曾孫康喀喇雅希禪子恭袞訥爾特拉篤渾舒賽舒賽子西蘭西蘭子席特庫景固勒岱景固勒岱從弟崇阿揚善弟伊遜訥都祜從弟武賴冷格里子穆成格冷格里弟納穆泰從弟譚布薩穆什喀弟雅賴洪尼雅喀子武拉禪洪尼雅喀弟薩蘇喀阿山

常書,郭絡羅氏。與其弟揚書,同為蘇克蘇滸河部沾河寨長。太祖起孤露,奮復祖父仇,歸罪尼堪外蘭,未遽訟言仇明也。明庇尼堪外蘭,宣言將築城甲班,使為滿洲主。於是旁近諸部及太祖族人,皆欲害太祖,附尼堪外蘭。蘇克蘇滸河部薩爾滸城長諾米納有兄曰瓜喇,忤尼堪外蘭,尼堪外蘭譖於明,見詰治。諾米納與常書、揚書及同部嘉木湖寨長噶哈善哈思虎相為謀曰:「與其倚此等人,何如附愛新覺羅寧古塔貝勒乎?」遂相率歸太祖。太祖椎牛祭天,將與盟,常書等言於太祖曰:「我等率先來歸,幸愛如手足,毋以編氓遇我!」乃盟。既而諾米納貳於尼堪外蘭,常書等請於太祖誘而殺之。

太祖以同母女弟妻揚書、噶哈善哈思虎,是歲癸未。明年正月,太祖從叔龍敦,構太祖異母弟薩木占,邀噶哈善哈思虎殺諸途。太祖聞,大怒,欲收其骨;族昆弟皆與龍敦謀,不肯往。太祖率近侍數人行,太祖族叔棱敦尼之曰:「同族皆仇汝,否則汝女弟之夫何至見殺?宜勿往。」太祖勿聽,環甲躍馬,登城南橫巘,引弓疾馳。向城大呼曰:「有害我者速出!」聞者憚太祖英武,不敢出,遂收其骨以歸,移置室中,解所御冠履衣服以斂,厚葬之。遂帥師討薩木占及其黨訥申、萬濟漢等,為噶哈善哈思虎復仇。

常書兄弟事太祖,分領其故部,為牛錄額真。旗制定,隸滿洲鑲黃旗,旋改隸鑲白旗。常書兄弟皆卒於太祖朝,揚書之喪,太祖親臨焉。常書子布哈圖、察哈喇,並為牛錄額真無所表見。?,改隸鑲白旗。布哈圖事

察哈喇事太宗。各旗設調遣大臣,察哈喇與焉,佐正紅旗。天聰三年,從上伐明,取遵化,薄明都。四年二月,師還。命署固山額真,與範文程率蒙古兵守遵化。四月,與武納格設謀,即樵採地設伏敗敵,獲馬二百餘。明將合馬步軍四千攻大安口,復與武納格整兵奮擊,盡殲之。五月,明兵復灤州。貝勒阿敏等謀棄諸城,引兵出邊,令察哈喇棄遵化。會明兵已逼,察哈喇與鄂本兌等突圍出,全師以還。五年五月,偕總兵官冷格里、喀克篤禮伐明南海島,師次海濱,掠敵舟以渡,舟未足,駐師待之。明兵渡海來擊,牛錄額真穆世屯戰死。察哈喇督兵力戰,別遣人沉其舟,敵還求舟不得,溺者大半。六年五月,上伐明,略歸化城,將渡河,與承政車爾格以兵五百為前鋒,具舟濟師。十一月,與承政巴篤理使朝鮮,定職貢歲額。九年,從貝勒多鐸攻明錦州,與固山額真阿山、甲喇額真吳拜等以兵四百為前鋒,渡大凌河,遇明兵三千,相向列陣。使告多鐸督諸軍繼至,明兵潰,察哈喇等分道迫擊,俘馘無算。逾年,卒。

布哈圖有子曰葉璽,事太宗。崇德三年,從睿親王多爾袞伐明,自青山口毀邊墻入,破薊遼總督吳阿衡軍。五年,從圍錦州。順治元年,從武英親王阿濟格西征。二年,破李自成兵於延安,移軍下江南,至安陸,獲敵艦四,復與護軍統領哈寧阿泛江擊賊,至富池口,水陸屢戰皆捷。三年四月,蘇尼特部騰機思等叛入喀爾喀,葉璽從多鐸等討之,追至布爾哈圖山,俘七人,降二十五戶。七月,師自圖拉河西行,至扎濟布拉克,遇喀爾喀土謝圖汗二子率兵二萬御,戰,沒於陣。葉璽時官巴牙喇甲喇章京,事聞,贈巴牙喇纛章京,予世職拜他喇布勒哈番。

布哈圖有孫曰辰布祿,初任牛錄額真,兼工部理事官。崇德三年,從多爾袞伐明,克陽信。順治三年,從定西大將軍和洛輝擊賀珍漢中,從肅親王豪格討張獻忠,皆有功,授拜他喇布勒哈番。十三年,從討鄭成功,敗其將陳六禦等於舟山,進三等阿達哈哈番。十七年,卒。

察哈喇有子曰富喇克塔,任牛錄額真、都察院理事官。崇德八年,遷工部參政。順治元年四月,授正藍旗滿洲梅勒額真,旋擢本旗蒙古固山額真。從睿親王多爾袞入關破李自成,追至慶都,授牛錄章京世職。從豫親王多鐸攻潼關,自成將劉宗敏據山為陣,富喇克塔與都統拜音圖發?擊之,潰。二年,從下江南,與馬喇希等為前鋒,克揚州。三年,從貝勒博洛定浙江,克處州。略福建,與都統漢岱克分水關,趨泉州,下撫州及所領縣三。加半個前程。五年二月,坐事,解固山額真。尋從征南大將軍譚泰討金聲桓,敗賊於九江,得戰艦百餘。與何洛會以偏師截餉道,得糧艘二百,遂攻南昌。聲桓及王得仁以兵七萬守隘,富喇克塔以舟二十為前鋒,薄城力戰。明年,賊平。師還,卒於軍,進一等阿達哈哈番。

揚書有子曰達爾漢,太祖甥也。改隸鑲藍旗。初為牛錄額真。太祖妻以女,為額駙。積戰功,授一等副將世職。太宗即位,列八大臣,領鑲黃旗。從大貝勒代善伐扎魯特部,單騎逐敵,獲其臺吉。復伐棟揆部,俘塔布囊古穆楚赫爾、杜喀爾、代青多爾濟三人及其子,進三等總兵官。天聰元年,從伐朝鮮,克義、定、安三州,斬其府尹李莞等。朝鮮國王李倧請行成,使與納穆泰等蒞盟。師還,上賜宴勞之。復從伐明,攻錦州,有功。貝勒阿巴泰以賜宴不得與大貝勒同坐,屬達爾漢代奏,上使勸諭之。復宴,阿巴泰又以為言,乃解達爾漢固山額真示意,旋命復任。三年,從伐明,圍遵化,率所部攻城西迤北,克之。四年,蒙古敖漢、柰曼諸部兵攻昌黎,不克,命達爾漢與喀克篤禮等以兵千人往會攻,城未下,焚近郭廬舍而還。五年七月,從伐明,圍大凌河城,率所部攻城北迤東,浚壕築壘,與冷格裡等環城固守。八月,城人以步騎五百出戰,達爾漢率八十人擊敗之。越日,敵復出挑戰,達爾漢督所部邀擊,明兵墮壕死者百餘人。

六年,從伐察哈爾,師次哈納崖。達爾漢從者盜馬,遁入察哈爾,告師至,林丹汗舉部西奔,驅歸化城富家渡黃河西遁。達爾漢坐降一等副將。七年,明將孔有德來降,達爾漢與篇古屯兵江岸守其舟。八年,復從伐察哈爾,遂略明邊,自上方堡毀邊墻入,經朔州,分。是歲,命免功臣徭役,達爾漢與焉,並增牛錄人戶。九年,上遣諸貝勒?兵至宣府右伐明,略山西,命達爾漢及阿山等出屯,牽制明寧、錦諸道兵,使不得西援。道遇敵,擊敗之,斬明將劉應選。崇德元年五月,從武英郡王阿濟格伐明,攻順義,以所部先登,進一等總兵官。尋以順義復失,論罰。六年,從鄭親王濟爾哈朗等伐明,攻錦州,達爾漢坐濟爾哈朗召議禦敵不時至,嗾其僚爭功,罷固山額真,奪世職。順治元年,卒。

達爾漢有子曰鄂羅塞臣,事太宗,官甲喇章京,領擺牙喇兵。天聰三年,從伐明,薄燕京,與哈寧阿共破明經略袁崇煥營。太宗嘉其善戰,授備御。四年,署固山額真。從貝勒至,迎擊,敗之。五年,從伐明,圍大凌河城,屢敗城兵。?阿巴泰等守永平,明兵自開平擊敵。累功,進二等阿?八年,從貝勒薩哈廉略山西,明兵自崞縣至,鄂羅塞臣從第三隊先達哈哈番。崇德元年,從伐朝鮮,與薩穆什喀等敗其援兵。二年,授議政大臣。三年十月,從豫親王多鐸伐明,侵寧遠、錦州。十一月,豫親王至中後所,將與鄭親王濟爾哈朗軍會。明總兵祖大壽兵來襲,甲喇額真翁克等及從征土默特部兵先奔,鄂羅塞臣及哈寧阿等且戰且退,士卒有死傷者。論罰,奪世職。

六年三月,從睿親王多爾袞伐明,圍錦州。六月,復從鄭親王濟爾哈朗伐明,圍錦州,祖大壽以步兵出戰,左翼三旗騎兵避敵勿敢擊,鄂羅塞臣與同官阿桑喜率擺牙喇兵直前奮擊,大壽乃引去。肅親王豪格庇三旗之未戰者,睿親王多爾袞和之,使誡鄂羅塞臣毋言戰勝皆出擺牙喇兵,亦毋言戰時未見騎兵,功罪置勿論。明年,事聞,上令多爾袞出白金五百,豪格出白金千,畀鄂羅塞臣,進二等阿達哈哈番,擢梅勒額真。八年,與參政巴都禮等定黑龍江。順治二年,從討李自成,克潼關,鄂羅塞臣先登。五年正月,命帥師駐滄州。十二月,從武英親王阿濟格討姜瓖。六年七月,擢正藍旗蒙古固山額真。尋兼任刑部侍郎。

鄂羅塞臣,公主子,世臣,從征伐有功。兩遇恩詔,累進二等精奇尼哈番。七年,坐讞獄徇情,罷侍郎。八年,授都察院左都御史。尋命專任固山。十六年,與安南將軍明安達禮帥師駐荊州。鄭成功犯江寧,明安達禮、鄂羅塞臣以舟師赴援,成功敗走。十七年,還京,仍任都統。康熙三年,卒,贈太子太保,謚敏果。子勒貝,自有傳。

康果禮,先世居那木都魯,以地為氏。歲庚戌,太祖命額亦都將千人,徇東海渥集部,降那木都魯、綏芬、寧古塔、尼馬察四路。康果禮時為綏芬路屯長,與其弟喀克都里及他屯長明安圖巴顏、泰松阿、伊勒占、蘇爾休,明安圖巴顏子哈哈納、綽和諾,泰松阿子葉克為六牛錄,以康果禮、?書等,凡十九輩,率丁壯千餘來歸。太祖為設宴,賚以金幣,分其喀克都里、伊勒占、蘇爾休、哈哈納、綽和諾世領牛錄額真。

旋授康果禮三等總兵官。以貝勒穆爾哈齊女妻之,號「和碩額駙」。旗制定,隸滿洲正白旗。天命三年,從上伐明,取撫順,克撫安、三岔兒等十一堡,入鴉鶻關,破清河。六年,復從伐明,下沈陽,樹雲梯先登,遂克其城。太宗即位,列十六大臣,佐正白旗。尋擢擺牙喇纛章京。天聰元年,從貝勒阿敏伐朝鮮。三年,從上伐明,入洪山口,克遵化,薄明都。上駐軍德勝門外,明督師袁崇煥入援,壁於城東南。上命康果禮從諸貝勒擊之,諸貝勒逐敵迫壕,康果禮與甲喇章京郎球、漢岱等不及壕而返,並坐削爵,罰鍰,奪俘獲。五年,卒。

子六,色虎德,繼為牛錄額真;邁色,為擺牙喇甲喇章京,從伐明,戰塔山,沒於陣;賴塔,自有傳。

喀克都里,與康果禮同隸滿洲正白旗。初授三等總兵官。太宗即位,列八大臣,領正白旗。天聰元年,從伐朝鮮,有功。三年,上伐明,圍遵化,八固山環城而攻,分隅列陣。繼之,城遂拔。上嘉喀克都裏造攻具如法,督兵?喀克都里所部兵薩木哈圖,樹雲梯先登,先諸軍登城,親酌金??勞,進二等總兵官,賜號噶思哈巴圖魯,言其勇敢善戰,疾如飛鳥也。

薩木哈圖亦賜「巴圖魯」號,授備御世職。四年正月,上復伐明,克永平,明兵潰走昌黎。上遣敖漢、柰曼、巴林、扎魯特諸部兵攻之,命喀克都里與固山額真達爾漢等將千人繼往為助,守堅不能下,焚附城廬舍,引還。上既錄遵化功,察薩木哈圖猛士,心愛惜之,戒喀克都里毋使更先登。及攻昌黎,薩木哈圖運木築柵,復樹雲梯欲登,聞上命罷攻,乃止。上以喀克都裏不恤戰士,深詰責之。

五年五月,與固山額真冷格里分率左右翼步、騎兵伐明,規取南海島,徵舟於朝鮮,不至,師次海濱,不能渡,引還。明兵邀戰,屢擊敗之,多所俘獲。八月,上復伐明,圍大凌河城,喀克都里率所部軍城東北,城人食盡,祖大壽以城降,引還。六年,從上伐察哈爾,與諸將分道並入,籍所俘人戶及帛、馬、牛、羊以獻,賜賚有差。七年,上詢諸貝勒大臣伐明及朝鮮、察哈爾宜何先,喀克都裏言:「宜先伐明,以承天佑、協人情,且利在神速,攻其不備。」上嘉納之。

八年,喀克都里家人訐喀克都裏將亡歸瓦爾喀,以財貨藏那木都魯故屯。上曰:「喀克都裏安有此?果欲負朕,天必鑒之!」以訐者付喀克都裏殺之。逾數月,喀克都裏卒。其兄康果禮妻,故貝勒舒爾哈齊女,言喀克都裏謀亡去事不誣,諸子坐此不得紹封。

哈哈納,亦那木都魯氏,明安圖巴顏子也。隸滿洲鑲紅旗。初與伊勒占、蘇爾休同授備御。太祖妻以宗女。尋從伐烏喇,被數創,力戰敗敵。上命將所部出駐賽明吉,未至,其戍兵叛亡,守將瑪爾圖追弗及。哈哈納聞之,兼程疾進,斬三百餘級,收男婦五百餘以還。上賜以所得叛渠及鞍馬、弓矢。天命四年三月,明經略楊鎬部諸將四路來攻,上督諸貝勒出御,破之,遂進克開原、鐵嶺。哈哈納皆在軍有功。六年,從攻遼陽,與博爾晉伺敵城下,敗其援兵;復分攻沙嶺城,破援兵自廣寧至者。太宗即位,設各旗調遣大臣,以哈哈納佐鑲紅旗。天聰八年,帥師略錦州,進攻寧遠,明兵驟至,哈哈納馬殕,徒步益奮擊,卒破明兵。城海州,明兵來爭,哈哈納以所部首當敵,敵潰走。復援耀州,解其圍,逐敵,獲馬三十。崇德元年,從武英郡王阿濟格伐明,入長城,克昌平、涿州。創發,病廢,致仕。尋卒。

子費揚古,事聖祖。以佐領從軍,討吳三桂。師次荊州,戰宜昌,戰永興,皆捷;攻常寧、耒陽,先驅。累遷鑲紅旗漢軍副都統。卒。

綽和諾,亦隸鑲紅旗。其初歸太祖,別率所部百人偕,太祖賚予甚厚。從太祖征伐,臨陣衷綿甲,奮起直前,所向披靡。歲辛亥,從何和禮伐呼爾哈部,克扎庫塔城。天命四年,擊明總兵馬林尚間崖。六年,取沈陽、遼陽,並有功,授游擊。帥師戍科木索、寧古塔。掠輜重亡去,綽和諾追及海濱,斬就善,並殲其黨,上命以所獲輜?有就善者,戕守吏,率重犒之。太宗即位,列十六大臣,佐鑲紅旗。天聰五年,從上伐明,圍大凌河城。明監軍道迎擊,力戰,沒於陣。上厚恤其家,進?張春、總兵吳襄等率兵萬餘自錦州來援,綽和諾先世職一等參將。無子,其兄翁格尼襲,以新附呼爾哈百人益所轄牛錄。旋以翁格尼才不勝,改授其子富喀禪。

富喀禪初以擺牙喇壯達事太宗。大凌河之役,深入敵陣,綽和諾戰死,富喀禪亦被創墮馬,裹創步戰,搴敵纛;擺牙喇壯達瑤奎亦墮馬,富喀禪復前援,與俱歸。八年,攻大同,復被創,仍奮進克敵寨。是歲代其父為牛錄額真,襲職。崇德元年,從伐朝鮮。三年,授工部理事官,兼甲喇章京。從豫親王多鐸伐明,攻寧遠,敗敵中後所城西。

順治初,從入關擊李自成,加半個前程。三年,授西安駐防總管。自成餘黨劉文炳、郭君鎮等掠延安、慶陽。四年三月,富喀禪帥師討之,逐賊三水,斬君鎮;別遣游擊胡來覲、守備徐國崇等逐文炳至宜君藍莊溝,獲之,俘斬其黨略盡。

五年,回民米喇印、丁國棟等陷河州為亂,富喀禪與總督孟喬芳遣兵攻討,諸回皆受撫,而喇印復叛,陷甘州。富喀禪帥師進攻,深溝高壘相持,賊出城來犯,戰輒勝,並殲其樵採者。城既下,馘喇印。國棟又與?回土倫泰等陷肅州,遣副將馬寧、張勇討平之。

六年,姜瓖以大同叛,旁近郡縣皆陷。富喀禪遣諸將根特、杜敏赴援,戰猗氏,獲瓖登方;戰合水,斬瓖將劉宏才。論功,遇恩詔,累進一等阿思哈尼哈番。聖祖?所署監軍道即位,改西安駐防總管為將軍,富喀禪任事如故。時自成餘黨李來亨、郝搖旗、袁宗第等屯歸州、興山間。康熙二年,上遣將往討,命富喀禪與總督李國英、副都統杜敏等會師,戰於陳家坡,賊潰遁,進至黃草坡,復大敗之,進三等精奇尼哈番。五年,卒。子穆成額,自有傳。

葉克書,輝和氏,尼瑪察部長泰松阿子也。歸太祖,授牛錄額真,隸滿洲正紅旗。天殺敵。累功授三等?命六年,從伐明,攻遼陽,敵背城而陣,葉克書沖鋒突擊;攻沙嶺,先副將。太宗即位,列十六大臣,佐正紅旗。天聰五年,授兵部承政。六年,授固山額真。八年八月,從貝勒代善伐明,入得勝堡;略大同,下諸城堡;西至黃河,合軍朔州。十一月,考滿,進二等副將世職。九年,貝勒多爾袞伐明,自大同入邊,分兵授葉克書,從貝勒多鐸屯寧遠、錦州間,綴明援師,斬明將劉應選,俘其偏裨。

崇德元年,從武英郡王阿濟格伐明,自延慶入邊,克十二城。師還,坐所部失伍及攘獲、擅殺諸罪,罷官,削世職,仍領牛錄。二年正月,太宗伐朝鮮,命從承政尼堪等帥師伐。三?瓦爾喀,師出會寧,擊敗朝鮮兵。十一月,從參政星訥伐卦爾察,至黑龍江,俘獲甚年,師還,上特遣大臣迎勞。尋授兵部右參政。四年七月,授梅勒額真。十一月,從承政索海等帥師伐索倫。五年四月,復任固山額真。七月,授牛錄章京世職。

復從睿親王多爾袞伐明,圍錦州,與固山額真圖爾格率所部三百人為伏城西南烏欣河,捕城人出牧者。敵兵千餘逆戰,葉克書馬中矢蹶,圖爾格馳救之,上馬復戰,殺敵。比還,敵潛躡其後,葉克書收兵還擊,敵潰。以功進三等甲喇章京。六年九月,從貝勒杜度伐明,圍錦州,與固山額真譚泰、阿山等鑿壕環守,擊明總督洪承疇於松山。十一月,從貝勒阿巴泰等伐明,師至黃崖口,葉克書與譚泰定策分兩道夾擊,入邊薄長城,麾軍先登;攻薊州,敗明總兵白騰蛟、白廣恩諸軍。尋遣兵攻孟家臺,陷敵,坐罷官,奪世職。

順治元年三月,世祖復命為梅勒額真,帥師駐寧遠。四月,率步兵從入關擊李自成,身被三十一創,毀一目,戰彌厲,大破賊軍。二年,從肅親王豪格略山東,賊渠十餘輩據滿家洞,憑險為巢,凡二百五十一窟,葉克書與尚書車爾格合兵搜剿,殲其渠,悉堙諸窟。以功累進二等阿達哈哈番。三年,授鎮守盛京總管,恩詔進三等阿思哈尼哈番。十四年,坐昭陵總管鍾柰有罪,失不劾,罷官,奪世職。十五年,卒。子道喇。

道喇以擺牙喇兵從征伐,積功至擺牙喇甲喇章京。崇德三年,從伐明。五年,圍錦州,戰松山、杏山,皆有功。順治元年,調噶布什賢甲喇章京。睿親王多爾袞與李自成戰於一片石,從噶喇昂邦鄂碩當自成將唐通,通大敗。入關逐賊,戰安肅、慶都,乘勝躡擊,斬馘。尋從固山額真葉臣略山西,至汾州,敗自成將白輝。授牛錄章京世職。三年,從順承?甚郡王勒克德渾攻荊州,擊走李錦。五年,從大將軍譚泰下江西,討金聲桓,五敗賊,獲所署總兵以下。九年,擢正紅旗梅勒額真。十年,從靖南將軍哈哈木復潮州,討郝尚久。旋帥師駐荊州。十四年,授本旗蒙古固山額真。十六年,從信郡王多尼平雲南,攻元江土司,克其城。累功,並遇恩詔,進一等阿達哈哈番。

康熙初,以老乞致仕,徙居盛京。十二年,聖祖加恩諸老臣,加太子少傅。二十一年,幸盛京,召見賜坐,侍茶酒,優賚。二十二年九月,卒,年八十一,謚勤襄。以弟之孫伊濟納襲職。葉克書次子夏穆善,第三子瑚葉,皆有戰功,授世職:夏穆善二等阿達哈哈番,瑚葉三等阿達哈哈番。

博爾晉,世居完顏,以地為氏。太祖初起兵,有挾丁口來歸者,籍為牛錄,即使為牛錄額真,領其眾。順治間,定官名皆漢語,謂之「世管佐領」。博爾晉領牛錄,隸滿洲鑲紅旗,尋授侍衛。歲癸巳,太祖侵哈達,略富爾佳齊寨,博爾晉與族弟西喇布從。西喇布被二矢死,博爾晉拔其矢還射,殪發矢者西忒庫,為西喇布報仇。

天命六年,授扎爾固齊。城薩爾滸,命博爾晉董其役。役竟,從伐明,攻沈陽,擊敗明總兵賀世賢、陳策。沈陽下,進攻遼陽,明總兵李懷信、侯世祿、蔡國柱、姜弼、董仲葵合軍五萬,屯城東南五里,左翼四旗與戰,大破之。城兵自西門出援,博爾晉方奉命詗敵,傍城行,遂合兩紅旗兵邀擊,明兵敗,入城爭門,相蹂踐死者枕籍。會左翼四旗兵已登陴,畢登,遼陽亦下。復分兵拔沙嶺,擊敗明廣寧援軍。八年,與達音布、雅希禪帥?博爾晉麾師伐扎魯特部,其貝勒昂安突走,達音布戰死,博爾晉與雅希禪奮進,斬昂安,俘其孥。師還,上優賚之。十年,擢梅勒額真。將兵二千伐東海虎爾哈部,收五百戶以歸,上郊迎宴勞。

如故。天聰元年正月,從伐朝鮮。五月,上?太宗即位,列八大臣,領鑲紅旗,兼侍自將圍錦州,屯城西二里。博爾晉自沈陽帥師至,敗明兵,追至寧遠城下盡殲之。?先後戰功,授一等副將。旋卒。以失敕書,子孫不得襲。康熙三年,其子特錦疏請立碑紀績,部議無左證,持不可,聖祖以博爾晉事太祖,勤勞夙著,特詔許之,並追謚忠直。特錦及博爾晉孫瑪沁、曾孫康喀喇,皆有戰績。

特錦,博爾晉第四子也。初任牛錄額真。天聰八年,授牛錄章京世職。崇德五年,從鄭親王濟爾哈朗帥師屯田義州。蒙古多羅特部蘇班代等降明,居杏山西五里臺,使通款,上命鄭親王移師迎護。明總兵祖大壽、吳三桂、劉周智屯杏山拒戰,特錦以偏師擊敗之。六年,從伐明,圍松山,攻寧遠,皆力戰敗敵。

順治初,從入關,逐李自成至慶都,與梅勒額真和託合軍大敗之,進三等甲喇章京,任兵部理事官。考滿,進二等甲喇章京。三年,從肅親王豪格下四川,討張獻忠,戰三水,敗其將胡敬德;復戰禮縣,敗其將高汝礪。獻忠死西充,餘賊負山,將斷我兵後,特錦擊之走;又戰馬湖,破其將楊正。六年,從討姜瓖,略壽陽,賊犯兩藍旗分地,徇汾州,賊七千夜擊兩紅旗軍壘,特錦連擊敗之。平遼、遼州、榆社以次悉平。

七年,擢兵部侍郎,兼鑲紅旗蒙古梅勒額真,進三等阿思哈尼哈番。十二年,擢本旗蒙古固山額真、議政大臣。十五年,從信郡王多尼徵貴州、雲南,進二等。十八年,轉本旗滿洲都統。康熙十一年,卒,謚襄壯。

瑪沁,博爾晉孫。父本託輝,博爾晉長子。官牛錄額真,兼都察院理事官。崇德三年,以擺牙喇甲喇章京從貝勒岳託伐明,自墻子嶺入邊,明薊遼總督吳阿衡以馬步兵六千來援,瑪沁與勞薩等率兵擊敗之,獲其馬及砲。六年,從圍錦州,敗敵於松山。順治初,從入關,破流賊,授牛錄章京世職。五年,擢鑲紅旗蒙古副都統。七年,恩詔加半個前程。尋從鄭親王濟爾哈朗征湖廣,至衡州,疾,卒。無子,以兄子康喀喇襲。

。順治四年,蘇尼特部騰機思與其弟騰機特叛,?康喀喇,博爾晉曾孫。初為二等侍康喀喇從豫親王多鐸帥師往討,大破之,陣斬騰機特。進二等阿達哈哈番。十五年,從寧南大將軍洛託徵貴州。康熙十年,遷護軍參領。十二年,吳三桂反,順承郡王勒爾錦帥師討之,康喀喇將護軍從。十三年,攻岳州,戰荊河口,戰城陵磯,破三桂將吳應麒。十六年,攻長沙,復茶陵,戰攸縣,破三桂將王輝。十七年,取耒陽,下常寧、新寧諸縣,又克郴州,康喀喇皆在行間。二十五年,授鑲紅旗滿洲副都統。二十九年,從裕親王福全徵噶爾丹。三十年,卒。

雅希禪,先世居馬佳,以地為氏。父尼瑪禪,當太祖兵初起,從其兄赫東額率五十餘戶來歸,任牛錄額真。雅希禪事太祖,積戰功,授備御,為扎爾固齊。天命四年,蒙古喀爾喀五部遣使請盟,太祖命額克星格、綽護爾、雅希禪、庫爾纏、希福往蒞。是歲,從上御明師,戰於界凡,雅希禪先眾克敵,復擊明總兵馬林於尚間崖,破其中堅,以功進二等參將。七年,從上克遼陽,進三等副將。及沙嶺之戰,為敵所創,戰敗,降一等參將。八年,從貝勒阿巴泰等伐扎魯特部,與達音布、博爾晉率兵逼貝勒昂安寨,昂安以其孥行,達音布戰死,雅希禪與博爾晉共擊殺昂安。尋卒。順治十二年,世祖追錄太祖、太宗諸將,賜謚勒碑,雅希禪謚敏果。子三:恭袞、訥爾特、拉篤渾。

恭袞襲職,坐事,析世職為二備御,與其弟訥爾特分襲。崇德三年,授刑部副理事官。四年,從伐索倫,陣沒。部議恭袞不從軍令,乃為敵所戕,當奪世職,籍家產三之一,上念其父雅希禪有功,特貰之。

訥爾特,初從太宗伐明,敗敵小凌河。復自大同入邊,選善射者使訥爾特將之,攻克小石城。既,襲備御。復從圍錦州,屢敗敵松山、杏山。崇德七年,授刑部參政,兼梅勒額真。師方攻松山,松山明兵夜遁,訥爾特與擺牙喇纛章京鰲拜,馳塔山南海濱,先敵至,蓐食以待。夜擊明兵,達旦,明兵據山巔,訥爾特率所部冒矢石仰攻,明兵敗走,乘勝逐之,明。八年,從伐明,初入邊,擊敗明守將。師渡渾河,方築梁,明兵千餘起撓?兵入水死者甚之,訥爾特擊之走。復敗明援兵於三河,進略山東,克武定。師還,將出邊,明將以步兵追躡,謀劫砲,訥爾特與固山額真準塔還擊,破之,賜白金五百。九月,復從鄭親王濟爾哈朗伐明,攻寧遠,明總兵吳三桂出拒,訥爾特力戰,陣沒,贈游擊。

拉篤渾從父兄在軍,戰比有功。恭袞戰死,襲備御。崇德六年,從伐明,圍錦州,陣沒,加半個前程。

舒賽,世居薩克達,以地為氏。歸太祖,隸滿洲鑲藍旗。天命四年,從太祖御明師,進二等參將。太宗即?與雅希禪等攻馬林於尚間崖,以功授備御。尋從伐瓦爾喀,俘獲甚位,列十六大臣,佐鑲藍旗。天聰元年,從伐朝鮮,師還,命與固山額真阿山等帥師戍義州。八年,上自將伐明,鄭親王濟爾哈朗居守,舒賽與梅勒額真蒙阿圖等副之。舒賽善戰,攻城輒被棉甲先登,太祖嘉其勇,又慮其輕進,溫諭誡止之。舒賽益感奮,先後克十六城。太宗特敕旌其功,進三等梅勒章京。崇德六年十月,卒。順治十二年,追謚壯敏。

子西蘭,初任牛錄章京,授備御世職。順治元年,以擺牙喇甲喇章京從豫親王多鐸討李自成,攻潼關,三戰皆勝。二年,從貝勒博洛定江南,下松江,徇福建,克平和。論功,遇恩詔,進三等阿達哈哈番。七年,卒。

西蘭子席特庫,崇德六年,襲大父舒賽世職三等梅勒章京。八年,授甲喇額真。從伐,以砲克城,斬明總兵李輔明。順治初,從入關,進略山西,佐固山額真葉臣?明,攻前屯等克太原。二年,從英親王阿濟格徇陜西,敗賊延安。自成走湖廣,躡擊至安陸,與鰲拜等屢破敵,進二等梅勒章京。四年,改二等阿思哈尼哈番。五年四月,卒。乾隆間,定封二等男。

景固勒岱,扎庫塔氏。初居呼爾哈部,烏喇招之,不往。太祖遣將伐東海渥集部,景。尋挈孥及諸兄弟率所屬三十戶來歸,隸滿洲?固勒岱徒步從軍,攻取烏爾固宸路,俘馘甚正白旗,任牛錄額真。天命三年,從上伐明,入鴉鶻關,攻克清河城,擢甲喇額真,仍兼領牛錄。上規取遼、沈,景固勒岱並在軍有功。天聰八年五月,授世職二等甲喇章京。十二月,命與甲喇額真吳巴海率兵四千伐瓦爾喀部,降其屯長芬達里及所屬五百餘戶,俘阿庫里尼滿部千餘人,獲貂、虎、狐、貉、猞猁猻、獺、青鼠諸毛毳之屬。九年六月,師還,上令禮部諸臣宴勞,以所獲分賚將士,進世職一等甲喇章京。崇德二年,從武英郡王阿濟格攻明皮島,克之,賚裘服、鞍、馬、銀、布、駝、牛諸物。順治初,恩詔,累進二等阿思哈尼哈番。十一年八月,卒,謚忠直。

從弟崇阿,任牛錄額真。天聰八年,從伐明,徇大同,略回雁堡。崇德元年,從伐朝鮮,敗敵桃山村。六年,從伐明,圍錦州,入其郛,巷戰。七年,從伐明,敗敵渾河之濱,入山東,至壽光。順治初,從入關。二年,從下浙江,拔湖州,進取福建,敗敵福寧。五年,從討金聲桓,敗王得仁於南昌。從討李成棟,破其軍,六年,戰南康,圍信豐,蹙成棟赴水死。累功,遇恩詔,進一等阿達哈哈番。十八年,卒。

揚善,瓜爾佳氏,費英東弟音達戶齊之子也。費英東諸弟:音達戶齊、吳爾漢、郎格齊,皆事太祖,隸鑲黃旗;而音達戶齊諸子:揚善、伊遜、鍾金、吉賽、納都祜、吉遜?、,改隸鑲白旗。

揚善亦逮事太祖,授備御。太宗即位,旗設調遣大臣二,揚善佐鑲黃旗,尋授巴牙喇纛章京。天聰三年,從伐明,受上方略,沖鋒攻堅,所至有功。五年,攻大凌河,與明監軍道張春戰,冒矢石陷陣,胸腕皆被創,進游擊,擢內大臣。六年,從伐察哈爾,林丹汗既遁有遁入明境沙河堡者,使揚善齎書索以歸。崇德二年,略大同,蒙古有被掠者,悉?,其部取以還,授議政大臣。

順治初,肅親王豪格得罪,都統何洛會誣告揚善及其子羅碩諂附豪格為亂。羅碩能通滿、漢、蒙古文字,太宗召直文館,授內國史院學士、噶布什賢章京,兼刑部理事官。至是,父子俱棄市。世祖親政,誅何洛會,復揚善世職,以其孫霍羅襲。

伊遜,音達戶齊第三子。太宗即位,列十六大臣,佐鑲黃旗。天聰三年,從伐明,攻遵化,伊遜先登,中砲傷臂,太宗親臨視,授游擊,尋遷兵部承政。七年,偕英俄爾岱使朝鮮,定互市約。崇德二年,坐事,罷。三年,復為兵部承政。四年,命與工部承政薩穆什喀等伐虎爾哈部,分兵循喇里闡,下兀庫爾城,設伏鐸陳城,敗敵,斬七十級。師還,坐為博穆博果爾所襲,亡輜重、士卒,論罰。八年,卒。順治十二年,追謚襄壯,建碑紀績。子噶達渾,孫沙爾布,相繼襲職。

納都祜,音達戶齊第八子。順治初,任護軍參領。從入關,破李自成,克潼關,定西安。移師下江南,追明福王至蕪湖。並有俘馘,授半個前程。三年,從討騰機思,土謝圖汗、碩羅汗拒戰,皆擊敗之。五年,從討金聲桓,有功。八年,擢正白旗梅勒額真,改副都御史,進拜他喇布勒哈番。又以伊遜無嗣,納都祜當並襲,復遇恩詔,覈改一等阿思哈尼哈番兼拖沙喇哈番。十四年,都察院請更定世職襲次,上疑其徇私,坐罷官。十七年,卒。無子,以鍾金孫貴欽、吉賽子盧柏赫分襲。

武賴,吳爾漢子也。隸滿洲鑲黃旗。天聰四年,與布爾堪等將精兵百人略明邊,渡大。八年,任甲喇額真。九年,擢固山額真,領正藍旗。崇德元年七月?凌河,馳斬俘獲甚來窺伺,盡殲之。師還,坐出邊不收後隊?,從武英郡王阿濟格伐明,明遵化三屯營守備率,誑言阿濟格逼脅,臨陣敗走,罰白金四百。十二月,上自將伐朝鮮,武賴從,與豫親王多鐸共擊敗諸道援兵。復與固山額真譚泰等率阿禮哈超哈兵攻漢城,樹雲梯以登,守陴者奔竄,盡收其輜重牲畜以歸。三年,從貝勒岳託伐明,至山東,擊敗明內官馮永盛、總兵侯永祿等,經董家口,敵兵千餘,依山為陣,武賴與戰屢捷,?其壘。明將復率兵要我軍輜重,武賴與準塔擊破之,遂乘勝行略地。以功,授牛錄章京。五年,從睿親王多爾袞伐明,刈禾錦州,明兵出拒,武賴追擊,迫使入城,遂略松山。八年,從貝勒阿巴泰伐明,至渾河,擊敗明兵。師還,經密雲,明兵以火器斷歸路,武賴與固山額真鰲拜奮勇馳突,明兵潰走;度塞,復敗敵,整軍出邊。以功加半個前程。順治初,入關破李自成,三詔,進至一等阿思哈尼哈番。以老乞休。尋卒,謚康毅,建碑紀績。

齊子鰲拜,郎格孫席卜臣,皆別有傳。

冷格里,舒穆祿氏,滿洲正黃旗人,揚古利弟也。少事太祖,從征伐。?功,自備禦累進一等副將。明將毛文龍分兵自朝鮮義州城西渡鴨綠江,入海島中,闢田以耕。天命九年秋八月,上命冷格裏將左翼兵、吳善將右翼兵襲擊之。道得諜,知明兵晝渡江穫於島,夜還屯江岸。冷格裏夜引兵自山蹊潛行,平旦,度明兵已渡江,即疾馳,揭支流以濟。入島,爭舟,多墮水死,焚島中積聚而還。?明將卒皆驚,奔潰,追斬五百餘級,餘

太宗即位,以其弟納穆泰為八大臣領本旗,而冷格里列十六大臣佐之。蒙古扎魯特部貳於明,大貝勒代善等帥師討之,冷格里及甲喇額真阿山將六百人為前鋒,略喀爾喀巴林部,逐守卒,縱火燎原,張軍勢,轉戰而前,獲扎魯特部貝勒巴克等十四人,俘二百七十一,掠駝、馬、牛、羊三千九百四十有二。師還,上率諸貝勒大臣迎勞,進三等總兵官。

天聰元年,從貝勒阿敏等伐朝鮮,夜引兵八十人襲明邊,一夕入六堠,盡俘其堠卒,遂襲義州,克之。論功,進一等總兵官。三年二月,明兵自海島移屯朝鮮鐵山,冷格里率精兵攻之,多所斬馘。九月,從揚古利率兵逐逃人雅爾古,遇毛文龍部卒以採葠至者,俘數十人以還。四年,納穆泰以棄灤州黜,擢冷格里為八大臣,領本旗。五年五月,與喀克篤禮分將左右翼兵伐南海島,有功。八月,太宗伐明,冷格里從,圍大凌河城,冷格里以所部軍於城西北。

上招明總兵祖大壽降,大壽未?,先使裨將韓棟出謁,出冷格里所守門。冷格里令軍士戎服執戟,立營門內外,示棟軍容。棟既謁上還,將入城,冷格里呵使止門外,問姓名,審形貌,然後令入。棟具以語大壽,大壽怵我軍嚴整,乃決降。

七年六月,從貝勒岳託等將右翼兵伐明,取旅順,師還,上迎勞如初。是年冬,冷格里有疾,十二月,上親至其第視疾。八年正月,卒。上臨其喪,哭之慟,駕還,設幄於丹墀,坐而嘆息,漏下二鼓始入宮。明年,上行幸,道經其墓,下馬酹而哭之。順治十二年,追謚武襄。

子穆成格。天聰四年,從伐明,克永平四城。薄明都,明侍郎劉之綸率兵出御,戰敗,所將兵盡殲,之綸匿石巖下,穆成格射殺之。八年,襲一等總兵官,尋改一等昂邦章京。官至刑部左參政。卒,子穆赫林,襲。順治初,改一等精奇尼哈番。恩詔,累進一等伯。康熙中,其孫吉當阿襲,復為一等精奇尼哈番。乾隆間,定封一等子。

納穆泰,揚古利幼弟,其母襁負來歸者也。少從太祖征伐。太宗即位,擢為八大臣,領本旗,以篤義貝勒巴雅喇子拜音圖及其兄冷格里為十六大臣佐之。天聰元年,從伐朝鮮。三年冬,從伐明,攻遵化,率所部軍其城西北。四年春,復克永平,降遷安,下灤州,是為永平四城。師還,命貝勒阿敏督諸將戍守,納穆泰與圖爾格、庫爾?、高鴻中率正黃、正紅、鑲白三旗分守灤州。

明經略孫承宗銳意復四城,四月,遣兵攻灤州,不能克而退。五月,監軍道張春、監紀官邱禾嘉,總兵祖大壽、馬世龍、楊紹基,副將祖大樂、祖可法、張弘謨、劉天祿、曹恭來攻,納穆泰與圖爾格分門而守,矢石競發,出精銳繞城搏戰,驅敵出壕外。?誠、孟弢悉敵復突至,攻納穆泰所守門,焚城樓,或執纛緣雲梯先登,我兵阿玉什斬之,奪其纛,敵稍卻,求援於阿敏。阿敏守永平,使巴篤禮以數百人往,夜突圍入城。敵以砲攻,我兵不能御,守四日夜,棄城奔永平就阿敏。阿敏旋引師還,永平四城復入於明。納穆泰坐論死,上命宥之,奪官,籍其家。

五年,將兵入明邊逐逋,斬六人,執九人以歸。明寧遠人張士粹來降,詭言明築大凌河城,使納穆泰與圖爾格將千人往詗之,還言士粹等言妄,悉誅之。尋擢兵部承政,授游擊世職。復與圖爾格略錦州、松山。八年,改官制,授固山額真、三等甲喇章京。秋,從上伐明,自上方堡入,八月,克靈丘縣王家莊,先登有功。九年二月,命貝勒多爾袞將萬人,收察哈爾林丹汗子額爾克孔果爾額哲,納穆泰將右翼,圖爾格將左翼。師還,入明境,自平魯略代州,至崞縣出邊,納穆泰、圖爾格以兵千人殿。明總兵祖大壽率馬步兵三千人追至,?,獲人畜七萬六?圖爾格奮擊破之;潰兵合馬步五百餘據臺為陣,納穆泰麾兵圍攻,盡殲其千二百。?功,加三等梅勒章京。十月,卒。上欲臨其喪,諸貝勒諫止,賜御服以斂。順治四年,改世職三等阿思哈尼哈番。三傳,降襲。揚古利從弟譚泰,自有傳。

譚泰弟譚布,天聰初,為巴牙喇甲喇章京。五年,從伐明,圍大凌河城,城人出樵採,率先邀擊,斬三人,俘二人,復與希福等擊敗明援兵自錦州至者。崇德三年,授議政大臣。四年十一月,與薩穆什喀、索海等伐索倫部,取道虎爾哈部,攻雅克薩城,得丁壯三百餘。索倫部長博穆博果爾迎戰,擊卻之,護所俘以歸。授牛錄章京,賜貂皮及人戶。五年,擢,明兵驟至,殘屯丁,論罰如例。六年,?十六大臣。時我兵屯田義州,譚布及覺善率兵為伐明,圍錦州。明總兵祖大壽以步卒出戰,譚布沖堅力戰,復敗其騎卒,斬材官一以徇。明總督洪承疇來援,譚布從其兄譚泰迎戰,敵騎至,譚布屢奮戰挫敵。以功,加半個前程。祖大壽既降,上命諸大臣與較射,賞諸中侯者,譚布賜駝一。八年正月,復與覺善戍錦州。九月,從鄭親王濟爾哈朗伐明,略寧遠。

順治元年,從入關,擊李自成,追至慶都,進二等甲喇章京。二年,從饒餘郡王阿巴泰鎮山東,與準塔徇徐州,擊敗明軍,得舟五百餘、砲五十有七。時豫親王多鐸下江南,自泗州渡河趨揚州,而明總兵劉澤清、總漕田仰猶保淮安,譚布與準塔師至清江浦,澤清、仰皆走,遂定淮安,下如皋、通州,撫輯附近諸州縣。進一等甲喇章京,加半個前程。三年,從肅親王豪格擊張獻忠。

出城?六年,從端重親王博洛討姜瓖,圍大同。瓖潛結援賊倚北山綴我軍,而自糾為;又分兵徇太原、?夾擊。譚布與鰲拜、車爾布等先破賊援,還擊瓖,迫使入城,斬殪甚平陽、汾州。論功,遇恩詔,累進一等阿思哈尼哈番。八年三月,授工部尚書。是年八月,譚泰誅,詔兄弟毋連坐。尋罷尚書,復為三等阿思哈尼哈番。康熙四年,卒。

薩穆什喀,佟佳氏,扈爾漢第三弟也。隸滿洲正白旗。少從太祖轉戰,積功授游擊。嘗以十二人逐敵山麓,斬百人,獲五十三人,馬、牛、羊千計。太宗即位,列十六大臣,佐鑲白旗。

天聰四年,從伐明,攻灤州。七年,復從貝勒岳託等伐明,規取旅順。時師自陸行,舍馬徒行。至水次,岳託勉薩穆什?皆乘馬,薩穆什喀曰:「師潛進,安用乘馬為?」乃率喀努力,薩穆什喀對曰:「如貝勒言。此城誓必下,不空歸也!」遂與白奇超哈章京巴奇蘭以舟先,身被百創,戰益厲,遂破旅順。師還,太宗郊勞,親酌金?以賜,進一等參將。八,取?年,授甲喇額真。從貝勒杜蘭戍海州。十二月,命副巴奇蘭伐黑龍江虎爾哈部,降其其地。九年四月,師還。加三等梅勒章京,授白奇超哈章京。

崇德元年,從武英郡王阿濟格等伐明,入長城,與額駙蘇納帥師攻容城,先登,克之。三年,授議政大臣。復從武英郡王阿濟格攻皮島,督擺牙喇兵渡江,先至岸,與固山額真阿山、葉臣等共攻克之,斬其守將沈世魁,進二等。七月,授工部承政。

四年,與刑部承政索海分將左右翼伐索倫部,部人達爾布尼、阿哈木都戶、白庫都、漢必爾代據厄庫爾城拒我師,薩穆什喀合左右翼攻克之。進攻鐸陳,未下,牛錄額真薩必圖等引兵助攻,鐸陳、阿撒津二城兵潛出邀戰,薩穆什喀設伏敗之,斬七十人。五年,師還,上郊勞賜宴。吏議薩穆什喀伐索倫,得三屯,復叛,其長博穆博果爾掠正藍旗輜重,坐視不救,當削職、籍沒,上命削職,貰籍沒。薩穆什喀陳辯:「博穆博果爾掠輜重,率兵追擊里許,乃與右翼索海等兵遇,索海等攘功。」上命王、貝勒、議政大臣勘覈,以薩穆什喀言妄,論死,上特宥之。復追論戍海州時備不嚴,屯丁為敵殺,論罰鍰。

七年,從伐明,攻錦州,敵犯塞,薩穆什喀力戰,敵三至三卻。錦州下,復授世職牛錄章京。八年,卒。子羅什,襲職。

雅賴,扈爾漢第七弟也。事太祖,從伐烏喇,略地朝鮮,數被創。從攻遼東,破蒙古兵。從伐察哈爾,先登殺敵。天聰三年五月,與甲喇額真羅璧等將千人略明新城路,遇毛文,殺九千?龍舊部採參者,斬六十人,毀其舟。九月,從揚古利逐逃人雅爾古,復遇文龍部六百餘人,獲千總三及從者十六。十一月,太宗伐明,薄明都,袁崇煥來援,攻擺牙喇兵,城兵出應,雅賴力戰?之。五年,從攻大凌河,屢勝。嘗單騎入敵陣,出戰死者尸。七年,取旅順口,與薩穆什喀同舟先濟,敵據岸列陣以拒。雅賴超躍登岸,大呼曰:「雅賴先登矣!」遂入敵陣。黎明,與敵戰,入城被創,戰益奮,我兵或少卻,輒手刃之。城下,授世職備御。崇德二年,授議政大臣。八年,加半個前程。順治初,從入關,擊李自成。二年,從破自成兵潼關,定河南、江南。論功,遇恩詔,進一等阿思哈尼哈番兼拖沙喇哈番。八年三月,擢戶部尚書。四月,坐駐防河間,牛錄額真碩爾對訐告發餉不均,罷,並削拖沙喇哈番。康熙三年,卒。乾隆初,定封三等男。

洪尼雅喀,吳扎庫氏,世居噶哈里。太祖初起時,扈倫諸部方強,烏喇尤橫肆,聞洪尼雅喀以材武豪於所部,劫其孥,迫使歸附。洪尼雅喀既偕往,念烏喇貝勒不足事,中途棄,隸滿州鑲紅旗。?走;與弟薩蘇喀、薩穆唐阿率其族四十人歸太祖。授牛錄額真,俾領其天命三年,從伐明有功,擢甲喇額真。天聰二年,太宗自將伐明,攻錦州。師薄城,洪尼雅死,洪?喀先登,毀其堞,墜傷足,敵迫之,將執而縶焉,季弟薩穆唐阿以壯達從軍,馳護尼雅喀乃免。八年五月,授世職三等甲喇章京。尋卒。子武拉禪。

武拉禪襲世職。順治元年,授擺牙喇甲喇額真。十月,從豫親王多鐸西討李自成。十二月,至潼關,甫立營,賊掩至,擊?之。二年,從端重親王博洛下浙江,趨平湖,敗敵,獲來攻,武拉禪與戰於赭山、於硃橋、於範村,屢勝?戰艦。進略杭州,馬士英、方國安擁。四年,授鑲紅旗蒙古梅勒額真。五年正月,增設滄州、大名駐防,命武拉禪以梅勒額真駐大名。金聲桓為亂,從征南大將軍譚泰攻南昌,五合五勝。聲桓以步騎七萬拒戰,率本旗兵合擊,大破之。聲桓既死,剿餘寇於袁州,擊敗明將硃翊鏦,定府一、縣二。

千人;別遣甲喇?六年七月,有趙鳳岡者,為亂於畿南,武拉禪討之,斬鳳岡,殲其額真哈其哈等擊賊寶山村,獲其渠田東樓、楊牌子。七年五月,授刑部侍郎。?功,遇恩詔,世職屢進,尋定為二等阿思哈尼哈番。十二年,從寧海大將軍宜爾德攻舟山,明將陳六禦等以三萬人拒戰,武拉禪督纛奮擊。以功,進一等阿思哈尼哈番。復以恩詔,加拖沙喇哈番阿拉那於市,武拉禪勘獄,反罪阿拉那,坐?內大臣額爾克戴青家奴毆侍?。十六年,領侍枉抑,削所加拖沙喇哈番。十七年,以病免。康熙六年十月,卒。

薩蘇喀,洪尼雅喀仲弟也。事太祖,授擺牙喇甲喇額真。天命七年,從太祖伐明,攻廣寧,戰於沙嶺。我師有都爾根者,馬蹶,敵騎三共取之,兩刃交下,薩蘇喀馳入敵陣,躍馬大呼,斬一人,排一人撲地,遂翼之出,無敢逼者。天聰三年,從太宗伐明,薄明都,薩蘇喀為前驅偵敵。五年,師圍大凌河城,城兵突出,薩蘇喀率兵追擊,及壕而返;城兵尋復,從噶布什賢噶喇昂邦勞薩擊敗寧遠兵,獲馬二十有?出,又擊敗之。八年二月,略明前屯二。,敵三百屯城外,奮擊,敵潰走,逐之至城下,斬獲甚?六月,師至大同,以三十人偵左。九年,從貝勒多爾袞招察哈爾林丹汗子額哲,進略明邊。固山額真圖爾格設伏敗敵,敵?潰走,薩蘇喀躡其後,斬級最,授半個前程。尋擢禮部參政。崇德二年,與甲喇額真丹岱等以八十人略明邊,次清河,敵七百屯守,與戰大勝,獲纛二、馬二十餘。五年,圍錦州,守,戰比有功。順治初,擢鑲紅旗?木魯河。六年,圍松山。八年,攻寧遠,取中後所、前屯滿洲梅勒額真。從入關,擊李自成,與梅勒額真和託共驅入敵營,中砲沒,贈三等甲喇章京。

阿山,伊爾根覺羅氏,世居穆溪。父阿爾塔什,率阿山及諸子阿達海、濟爾垓、噶賴,以七村附太祖。太祖妻以同族女兄弟,號「額駙」,而以阿山等屬貝勒代善。代善置閒散,觖望,與諸弟及其子塞赫等逃之明。上收其孥,貝勒阿敏以兵追之,射殪阿山二子,阿山亦被創,兄弟相失。穆克譚追射阿達海,阿達海斫穆克譚,墜馬幾死,遂奪其馬,與阿山等入明邊,尋復自歸。太祖問其故,對曰:「舉族相投,矢?命疆埸,豈直充?役乎?」乃置諸左右。旗制定,隸滿洲正藍旗。

天命六年,從伐遼陽,授二等參將。太宗即位,旗置大臣一為將,其次置大臣二為佐,又其次置大臣二備調遣。使阿山佐正白旗,阿達海與同旗備調遣。是歲,貝勒代善等帥師伐扎魯特部,上令阿山與冷格里以兵六百入喀爾喀巴林部逐邏卒,縱火張軍威。師還,進三等副將。

天聰元年,從伐朝鮮,克義州。阿達海坐匿太祖御用兜鍪,鞭五十。又違上命,為貝勒多鐸媒聘國舅阿布泰女,論死,上宥之,命奪官,籍其家之半。阿達海託言捕魚,以十騎逾赫圖阿喇城遁,克徹尼追之還。阿達海私語從人曰:「我欲亂箭射殺克徹尼,如爾輩何!」語聞,上命誅之。

三年秋,阿山復與弟噶賴子塞赫及阿達海子查塔、莫洛渾奔明寧遠,上收其孥,遣兵往追之,阿山等將入明境,遣從者先,明守塞兵執而殺之。阿山等懼,復還,請罪,上復宥之,還其孥,使復職。阿山乃訐雅蓀與同謀,雅蓀者起微賤,以葉赫攻兀扎魯城時,戰有功,太祖寵任之,雅蓀矢言殉太祖。太祖崩,不果殉,臨喪慢。至是,鞫得實,遂坐誅。

冬,從上伐明,克洪山口城,薄明都,軍於城東南。阿山與圖魯什周視敵營,請速進攻,上命即夜漏三下列陣,詰旦遂戰,大破明軍,陣斬武經略滿桂等。四年,攻永平,上命阿山及葉臣選部下猛士二十四人,乘夜挾雲梯以攻,諭曰:「登梯當令四人先分立梯端二旁,次令四人登,又次令十六人相繼上,又次則爾曹督其後,復令各旗出將一兵千人助攻。」次日,日加寅,薄城樹梯,犯矢石奮戰。俄,城上砲裂藥發,敵兵自驚擾,阿山督所部冒火銳上,諸軍繼進,遂克其城。

五年,攻大凌河,率銳騎邏錦州、松山,俘明兵,明守將出援,與勞薩、圖魯什以三二千,斬百餘級,獲纛三。上勞以金?,尋授固山額真。六年,上自將伐察哈爾?百人敗其,阿山與梅勒額真布爾吉方行邊,聞上至西拉木倫河,帥師來會,上命率精騎三百助圖魯什為前驅。察哈爾汗遁去,上引還,復命阿山等帥師防邊。七年,與布爾吉偵鹿島,多所俘獲。八年,與圖魯什略錦州,貝勒岳託謂圖魯什曰:「軍中調遣,當就阿山商榷,勿違其言。」既,復從伐察哈爾,斬蒙古逃人。追錄克永平功,進三等昂邦章京,免徭役;並分以虎爾哈俘百人,隸所領牛錄。

九年,師入明邊,略山西,明兵自山海關赴援。上命貝勒多鐸軍廣寧,阿山與固山額真石廷柱率噶布什賢兵四百前驅趨錦州,明副將劉應選等以兵三千五百人來御,遇於大凌河。將戰,多鐸後軍驟至,自山而下,士馬騰踔,軍容甚盛,明兵驚沮。阿山突起掩擊,我師從之,陣斬應選,殲其兵五百,克臺堡一。師還,賜良馬、鎧甲。

崇德元年,從武英郡王阿濟格伐明,下雕鶚、長安嶺二城,率本旗兵獨克東安縣。師還,明兵來追,阿山殿,擊斬略盡。二年,取皮島,與葉臣將左翼舟師攻其西北隅,先登,斬守將沈世魁,進一等昂邦章京世職。六年八月,復圍錦州,城兵突圍出攻我師,松山守將潛謀奪火器,阿山迭擊敗之。七年十月,復從貝勒阿巴泰伐明,入墻子嶺,轉戰至兗州。師還,賚銀幣。

順治元年,從入關,擊李自成,自成敗走,阿山偕左翼梅勒額真阿哈尼堪、右翼固山額真馬喇希,濟薄津擊破之,克平陽。以功,進三等公。二年,豫親王多鐸自陜西移師下江南,阿山及諸將從。與馬喇希等取淮河橋,渡淮拔揚州;率舟師溯江上,克江寧,獲明福王。江南既定,從貝勒博洛、固山額真拜音圖徇浙江,師次杭州,明潞王常淓降。嘉興、湖州、紹興、寧波、嚴州皆下。師還,賚金銀、鞍馬。

阿山自太宗時,屢坐事被論,輒貸之。三年,坐妄聽巫者言,罪所部,被訐,罷官,奪世職。旋復授一等昂邦章京。四年,改一等精奇尼哈番。旋卒。乾隆初,定封一等男。從弟阿爾津,自有傳。

,其子孫皆能以驍勇?論曰:太祖時,鄰近諸部族歸附,常書兄弟最先,康果禮等最自?。博爾晉、雅希禪殺敵致果,蓋勞薩、圖魯什之亞也。揚善、冷格里、薩穆什喀皆有戰績,非藉父兄顯者。洪尼雅喀尤以材武名。阿山屢去復歸,誅弟而用兄,駕馭梟桀,惟恩與法,握其要矣。


\end{pinyinscope}