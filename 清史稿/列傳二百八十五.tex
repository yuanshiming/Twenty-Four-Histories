\article{列傳二百八十五}

\begin{pinyinscope}
孝義二

盧必升李應麒李中德張文齡黎安理易良德

方立禮丁世忠汪良緒賈錫成王長祚劉國賓曹超

黎興岕夏汝英金國選張愫李志善弟志勃彭大士

錢孝則任遇亨族子裕德陸國安徐守質兄基

黃簡程原學鬱褒姚易修胡夢豸賀上林何士閥

陳嘉謨林長貴弟長廣戚弢言李敬躋

張大觀楊璞蔡應泰張士仁潘瑂劉希向

沈嗣綬謝君澤馮福基黃向堅顧廷琦李澄

劉獻煜錢美恭趙萬全劉龍光李芳巇唐肇虞

繆士毅子秉文陸承祺弟承祚汪龍方如珽張燾

硃壽命潘天成翁運槐弟運標楊士選徐大中

沈仁業魏樹德李汝恢鄭立本李學侗

董士元李復新黨國虎嚴廷瓚陸起鵾弟起鵬

虞爾忘弟爾雪黃洪元弟福元顏中和顏鼇

王恩榮楊獻恆任騎馬李巨勛任四

王國林藍忠

盧必升,字寀臣,浙江山陰人。九歲,父芳病,思得蟚蜞炙,必升挾筐求之沙上,潮至,幾死,不釋筐。明季遇寇,芳獨行入山,必升行求得之歸。必升為叔父茂後,順治初,寇縶茂舟中,必升繞岸哭,三晝夜,不絕聲。寇引使見茂,脅茂降,拔刃屢欲下,必升叩頭流血,乞貸死。久之,寇中有義其行者,脫茂使共還。茂有女忌必升,嗾母遣必升往松江,使盜擊諸途。盜察必升且死,曰:「爾死勿我仇,誰某實使我。」必升陽死,盜擲之水,復以救免。必升書告所後母,但自謝不謹被盜,所後母為感悟,為母子如初。

李應麒,雲南昆明人。遘亂,與其父相失,被略至迤東,乞食歸。喪母,勸父再娶,後母至,遇應麒虐,應麒賣卜以養。失後母意,輒笞楚,跪而受杖。後乃被逐,事父母愈謹。父生日,賣卜得雞米,持歸為壽。佃人田,方耕,聞後母病,輟耕走三十里求醫藥。後母生三子,友愛無間。後母久乃悟,卒善視焉。

李中德,漢軍旗人。康熙初,父從征福建,中德亦出參陜西軍事,奉母以行。事畢,還京師,父先自福建還,已娶妾生子矣。中德母至,父暱妾而出嫡,拒不相見。中德為請,叩頭流血,父終不聽。請得居別室,亦不聽,及營室東直門外奉母,早晚侍父側無幾微憾,善視諸庶弟。越六年,父病棘,乃告父迎母還,父深悔焉,旋卒,妾亦死。中德母撫妾生四子如己出,中德亦友愛如父在時。

張文齡,字可庭,河南西華人。父暱妾而憎其母,文齡事父撫庶弟甚篤,庶弟亦感之,而父終不悟,逐文齡。文齡號泣呼天自懲艾,謂不復比於人,未嘗一言揚親過。遠近慕其行,遣子弟從游,得束脩,因庶弟以獻其父,或不得通,循墻走,泣且望,見者皆泣下。雍正五年,成進士,父榮之,意稍改。八年,就吏部選,京師地震,死者眾,文齡亦與焉。鄒一桂與為友,歸其喪,父始悟其孝,為之慟。

黎安理,貴州遵義人。祖母卒,復娶而悍,父不容於後母,客授四川灌縣,遂卒,葬焉。母還母家,安理方十歲,留祖父母所。祖母遇之虐,晝則令刈薪,夜督舂,舂重不舉,繩絡碓,以足挽之。恆不使得飽。嘗取毒蠚納其口。誘之溪側,推墮水。皆瀕死,遇救蘇。既長,習舉子業,出客授佐家。祖父卒,為治喪葬。祖母病,侍疾不倦,卒,又為治喪葬,無缺禮。其事祖父母凡三十有四年。痛父客死,恆詣灌縣謁墓。母復歸,事之孝。兩弟不勝祖母虐,出走,安理往來黔、蜀,求得仲弟還。季弟客死,撫其孤。安理晚舉乾隆四十四年鄉試,授永清教諭,遷山東長山知縣,有治績。告歸,卒於家。

易良德,湖南黔陽人。出為世父志宰後。志宰性急,屢撫兄弟子,皆不相能,遣還本支。最後得良德,良德能先意承志,得其歡心。有疾,晝夜侍,寢食俱廢,里人無子者恆舉良德相慰藉。

方立禮,江蘇江都人。母歾,後母遇之虐,怒輒與大杖,立禮謹受無懟。一日,杖幾絕,及蘇,無變容。父歾,遂逐立禮。立禮時時候門外問起居,疾則憂懼不食,愈乃已。妻洪,亦孝謹,日受鞭撻,後母稍自悔,為少戢。後母勿,為之哀毀。後母二子皆早死,立禮育其子女如己出。

丁世忠,湖南黔陽人。母初未有子,父娶妾,母生世忠。妾亦有子女而悍,惡世忠,嘗酖之,不死。父懦,令別室居,世忠事兩母無懟。庶弟無禮於世忠,嫡母喪,不欲持服,世忠皆不與較。庶弟坐事破家,世忠亦中落,仍割田畀之。

汪良緒,江蘇吳江人。父嗜博,母諫,忤父,為父逐。良緒日夜號泣,求返其母。父怒,並逐之,乃奉母依其妻父居。父以博破家,亦來與共居,母出奩貲易田,盡為父所鬻,良緒客授以養。方暑,父撤床上帳償博進,屢易屢鬻,良緒亦不具帳。晨起,蚊跡遍其體。母多病,良緒必親視湯藥。出客授,母疾病,方冬,水凍舟阻,履冰而還。母既歾,哭泣無常,寢

不解絰,稍寐輒呼阿母,寤則大慟,未終喪而卒。卒後視其枕,麻布包土★M6也。

賈錫成,江蘇宜興人。父映乾,性嚴。錫成生而生母吳以小過逢映乾怒,遂去不返。錫成稍長,鄰兒嘲無母,問得其故,悲不勝。甫成童,屢出訪母。過無錫,夢至尼菴,嫗予食,甚慈愛。因遍訪諸尼庵,方雪,老尼問里居,曰:「宜興。」因曰:「吾徒亦宜興。」入見之,即其母也。相持哭,母終不肯歸。錫成數省視饋食。及母卒,以喪還葬,上塚哭必慟。映乾遘疫卒,錫成痛甚,伏柩側喃喃若共父語,夢中或歡笑,寤則大慟。疾作遽卒,距映乾卒才五日。

王長祚,字爾昌,湖南衡陽人。父喬年,以富名。明季張獻忠破衡陽,喬年出避,游騎縶長祚與次子璠求喬年所在,榜掠終不言。寇挽長祚發,加刃於頸,璠號泣求代。寇中有騎者言:「此父子皆孝,奈何殺之?」遂得釋。

劉國賓,芷江人。國初流寇入縣境,國賓負母出避,道遇寇,劫母衣,刃創國賓,血流至足。國賓忍痛跪乞還母衣,語迫至,寇愍其孝而還之。康熙中,吳三桂兵至,掠族弟國宥,其母嫠也,哭之喪明。國賓行求國宥,逾年以歸,其母目復明。貧不能自存,國賓分田百畝與之。

曹超,安徽和州人。順治中,鄭成功兵至,超奉父母出避,遇寇欲殺之,超號泣求代,並得免。居喪,負土為墳。家有紫薇,父手植也,久枯,每對之哀慟,非時復發花。

黎興岕,湖南湘陰人。張獻忠破長沙,略湘陰,興岕父嘉品為賊縶,將殺之。興岕八歲,請代父死,賊幼之,舉刀令申頸,泣曰:「此恐欺我,既殺我,復殺父,乞但殺我一人。」引頸就刀,賊兩釋之,里人稱之曰「孺孝」。

夏汝英,湖南安化人。順治初,游兵掠其家,汝英九歲,衛母不去左右,游兵掠汝英去。道中告以母孤苦,乞釋還,賊憐而許之。

金國選,湖南黔陽人,吳三桂之亂,賊掠其父母去。國選七歲,牽衣痛哭,求釋,不得。罵賊,賊哧以白刃,不舍。擊以杖,終不舍,乃釋其父母。

張愫,湖南湘陰人。年十歲,寇至,從其祖走避。寇執其祖,將殺之,愫哀號求代,身蔽祖,被數創,不顧。寇嗟嘆,舍之去。

李志善、志勃,湖南安化人。父步武。諸生。流寇破縣,縛步武,志善十六、志勃十四,

號泣求免。賊詰步武里中孰為富,步武罵賊,賊殺之。志善、志勃奪賊刀殺賊,皆為賊所殺。

彭大士,湖南湘陰人。順治初,李自成餘黨破縣,執大士母求金。大士紿賊:「金在井側。」請偕往,因赴井,母走免。大士年十八,妻仇歸大士僅二十日,亦入井死。

錢孝則,江南桐城人。方明福王時,父以黨人被逮急,變姓名,挈家人亡命至震澤。兵起,母及弟、妹皆赴水死,孝則與父匿稻田中得脫。兵過,收葬母及弟、妹,走福建。未幾,福建亂作,父子奔避相失。孝則走廣東,數年還福建,求父十三年,始得與父俱歸。父續娶於徐,徐有富名。父他往,盜夜至,毀牖,縛孝則迫令導入徐室,孝則不可。盜斫以斧,顱裂死。

任遇亨,江南昆山人。生有膂力。國初盜大起,遇亨負父逃,盜劫其父去。遇亨持刀突入,負父出,身被數創,腸出,遇醫得不死,扶父徙居嘉定以老。

族子裕德,有土豪積怨於其父,伺隙持刀欲殺之。裕德年十一,身蔽父,兩手奪刀,正言曉以禍福,土豪擲刀去。父病痢三年,裕德晝夜扶持,躬滌濯污穢。父卒,居喪哀毀。友於兄,幼即請代兄杖。兄老而無藉,養生送死皆任之甚具。

陸國安,浙江山陰人。父華宇,順治初,縣境寇作,縛華宇入砦,求金以贖。國安歸自海上,奮入寇砦,馘寇,救華宇歸,被重創,卒無恙。

徐守質,江南常熟人。順治初,守質與兄基奉母避亂,母老病,兵至,度不能去。守質謂基曰:「毋徒死,絕徐氏後。兄速行,守質當奉母。」基不可。兵迫,守質慍,促基行。守質有妹適袁氏,早寡,攜子與母俱。基乃棄妻、子,挾孤甥而遁。事定,基還,母與袁氏妹俱自沉井,守質被二創僕,死。

黃簡,字敬之,湖南祁陽人。父用忠,諸生。簡事親孝,順治十年二月,李定國兵略湖南,其將郝永忠屠祁陽,簡奉父母避兵竹山。母渴,命簡取飲,兵遽至,簡父竄山陽,簡妻張,奉姑竄山陰。簡取飲至,不見父母,升高望之,見亂兵縛一人置釜上將烹,則其父也。簡大呼,往乞代,亂兵釋簡父,執簡求賂,不得,遂烹之。村民哀簡,名其山湯鑊嶺。

程原學,字奐若,江南儀真人。順治十六年,鄭成功兵退,縣人坐連染死者二十餘,原學祖故睢州知州紹儒與焉。父免死徙塞外,原學以幼留。稍長,將出塞求父,慮死且無後,乃娶妻生子。妻死,挾子行道中,子病,還,計行待子長。居恆喪服,食但啜粥,不飯,不食果蔬,衣不帛不棉。僦居學舍旁,授經不出戶。訓導顧靄慕其賢,屢過皆不見。偕其弟子出不意往語原學:「何自苦?」原學對曰:「原學有隱痛,不可以為人,非以自苦也。」明日報謁,贄硯與畫,靄謝曰:「子無所受於人,今吾受子遺,亦原以報子。」原學乃持硯與畫去。他日復過之,已他徙矣。俄卒,靄求得其硯,銘曰「廉士硯」。

鬱褒,字子弁,浙江嘉善人。父之章,順治六年進士,以大理寺丞坐罪徙尚陽堡。京師修治官廨,許罪人出家財佐工贖罪,褒請任刑部官廨,之章得贖還。工未如程,例當復徙,褒叩閽,請棄官代行。褒弟諸生廣,叩閽,言身當代父徙,留褒侍父疾。部議子代父徙非舊例,仍用沖突儀仗例治罪。聖祖愍其孝友,並宥之。之章還鄉里,褒以貢生授江西永豐知縣。

姚易修,字象亭,江南元和人。父宗甲,康熙初客閩浙總督範承謨幕。耿精忠為亂,執承謨,盡縶其幕客,宗甲與焉。易修聞,詣精忠,齧指作血書原代父死,賊乃釋宗甲而系易修獄,脅使降,易修不為屈。康熙十五年,師至,乃得脫歸。易修母聞變,悲泣,兩目盲,易修晨起舐母目,母目復明。鄰家火,易修突火入,負父出;又入,負母出。發盡燎,兩足焦爛,而父母俱無恙。

胡夢豸,江南江都人。康熙中,從父至紹興省墓,道遇盜劫民財,斥其不義,盜怒,將刃之。夢豸從後至,奔赴,擊盜僕,民群起毆殺盜。盜大至,欲屠其里,夢豸曰:「不可以我故,危一鄉也。」入盜寨,獨承殺盜,遂被殺。

賀上林,江蘇丹陽人。父天敘,以事忤知縣,系獄,將殺之。上林年十八,謀脫父。聞巡撫將上官,涉江溯淮,迎舟呼,騶從呵之,不得前,乃發憤投水,發沒數寸,復躍起大呼。巡撫見,令救,已死,檢其衣,得白父冤系狀。巡撫按部黜知縣,釋天敘出獄,鄉人為立賀孝子祠。

何士閥,安徽南陵人。族人破其祖母塚以葬,士閥訟不得直,巡撫檄知縣詣勘,族人持之力,事未定。士閥慟,觸墓碑,腦裂,死。知縣乃責族人他葬,治其罪,葬士閥,碑曰「義士」。

陳嘉謨,江蘇興化人。順治初諸生。父弘道,為怨家所誣,系揚州府獄。獄卒絕其橐饘,嘉謨求見父不得,知怨家計必殺之,乃痛哭禱於神,自沉於水。明日,鹽運使得嘉謨訟冤血書,而嘉謨僕又訴失嘉謨。求其尸,七日得於鈔關水次,植立風濤中,發上指。遂出弘道獄,葬嘉謨,而抵誣告者罪。

林長貴、長廣,福建福清人。父宗正,業曬鹽。入城,至星橋,海潮暴至,溺死。長貴聞之,奔救不及,仰天長號,投橋下殉;長廣繼至,繞崖痛哭,亦自沉。時雍正九年七月。里人憫其孝,收三尸斂焉。

戚弢言,字魏亭,浙江德清人。父麟祥,官翰林院侍講學士。坐事戍寧古塔,弢言從,備艱苦。麟祥遣令歸就試,成雍正八年進士,除福建連江知縣,勤其官。乾隆初,赦流人,麟祥不得與,弢言深痛之。總督郝玉麟將入覲,弢言刺指血為書求赦父,詣玉麟乞代上,玉麟難之。弢言叩首持玉麟裾號泣,引佩刀欲自裁,玉麟乃許之。詣京師,以弢言書上,高宗憫之,赦麟祥。麟祥就弢言養連江,明年卒。弢言持喪還,哀甚,亦卒。

李敬躋,字翼茲,雲南馬龍州人。父盛唐,雍正八年進士,官四川松茂道,以所部有罪坐監臨官,戍卜魁。卜魁距雲南萬四千里,敬躋三往省。嘗遇暴水,喪其僕馬,徒步行,路人哀之,與之食,導使詣盛唐,盛唐輒令還侍祖母,迫使歸。敬躋成乾隆二十二年進士,授福建將樂知縣,計贖盛唐還。盛唐死戍所,敬躋遂發病,日嗚嗚而啼,未幾亦死。

卜魁有範傑者,與盛唐善,盛唐倚以居二十年,至是歸其喪。閩人吳阿玉嘗欲從敬躋之官,盛唐喪過京師,吳為送還雲南。

張大觀,河南偃師人。乾隆二十六年秋,伊、洛水溢。灌偃師,民避水奎星樓上,大觀奉母亦登焉。水撼樓,樓傾,柱壓大觀手,臂折,奮入水求母。望母髻露水中,得之,負出水,攀樹以上,泳而求食以食母。水退,負母歸其室,即夕創重死。

同時有楊璞,與其弟奉母居。水至,弟以筏載其妻逃山上,母呼不應。璞棄妻子背襁母,浮水至神堤灘,或援之,得登。頃之,有婦抱子從水下,母遙望,呼曰:「吾婦與孫也!」拯之,皆不死。而弟乘筏即至山下,樹折壓筏沉,夫婦俱死。

又有蔡應泰,居母喪,柩在堂。水至,以繩系母柩,跪而負之,入水中疾駛,亦至神堤灘。村民以長鉤引至岸,舁以上。日暮,其妻、子亦得救。

張士仁,江南昆山人。六歲,母有疾,泣禱請代,母良愈。十三從父寢,仇伏榻下,露刃出。士仁呼父未應,手捍之,指欲墮,涕泣語仇請代,仇為感動,呼其父醒,曰:「爾有此子,吾不忍殺爾。」父惶遽,良久始定,與矢天日,釋怨。母喪盡禮,後母虐士仁,士仁孝敬無稍渝,後母亦感悟。火作,負父出,復入火負後母,後母抱幼子,幾不勝,風反得無恙。居父及後母喪如喪母,里或忤父母,必泣勸之,悔乃已。

潘瑂,浙江錢塘人。父出遠游,家遇火,母出篋令瑂負以行,及門回視,不見母,委篋復入,家人自火出,止瑂毋入,瑂不可,入與母俱死。瑂女兄珠姑嫁範氏,歸寧,亦在火中,家人欲掖以出,珠姑揮之曰:「汝男子,何可掖我!我從我母死耳。」火熄,瑂與母、姊三尸相環結,時乾隆四十四年十二月望。瑂聘妻王,家江幹,聞喪來歸,事舅以孝聞。

劉希向,江南山陽人。火,其父入火中求先人木主遺像。希向自外歸,突火入,求其父不得,號而出;復入,火方盛,救者以為劉氏父子死矣。俄而墻圮,顧見庭樹下人影往來,乃爭入負其父出,左奉像,右握木主,希向牽父衣,額半焦矣。後數年,父病,希向為割股,良愈。希向年六十,病噎,其子亦割股,刀鈍,肉不決,剪之,乃下,然希向竟不瘳。

沈嗣綬,字森甫,江陰人。父燿鋆,湖北通判,咸豐二年死於寇。嗣綬奉母還,寇至,徙避江船,高不可攀,展被以其母登。至通州,轉徙山東、河南,結繩床舁母,步從之,千數百里,不去左右。未至蘭山,道遇寇。嗣綬涕泣乞免,寇感其孝,遣四騎護行。至蘭山,方閉城拒寇,嗣綬求入城,守者疑諜也,趣縛之,涕泣言其故,乃得釋。既,亦得官湖北,以母病不赴。侍養十六年,進湯藥,夜起,慮履聲驚母,雖嚴寒必跣。凡事婉曲稱母意,見者感嘆。

謝君澤,江蘇武進人。父祜曾,事母以孝聞。寇亂,為賊虜,君澤冒死依護。父齒豁,不能食,恆嚼以哺。賊欲戕之,則號泣乞代父死,賊首感動,並釋之。

馮福基,代州人。父焯,為安徽潛山天堂司巡檢。咸豐七年,寇至,福基年十四,匿母他所,藏利刃,計伺隙殺賊,不可得。日夜涕泣從至黃梅,市毒藥置飯中,斃賊十七,亦吞藥死。巡撫李續宜奏言:「福基以童穉之年,護母陷賊,計殺兇黨多人,從容就義。奇節至性,深可嘉愍!」被旨旌恤。

黃向堅,字端木,江南吳縣人。父孔昭,崇禎間,官云南大姚知縣,挈孥之官,向堅獨留。鼎革後,孔昭阻兵不得歸,向堅日夜哭,將入雲南,親朋、妻子頗危之,向堅決行。至白鹽井,得父母並弟向嚴皆無恙,留一年乃歸,時為順治十年。行二萬五千里有奇,向堅次山川道途所經,自為圖十二記之,吳人作樂府紀其事。

顧廷琦,江南長洲人。父繩詒,崇禎間,官四川仁壽知縣,死張獻忠之難。事定,廷琦徒步入四川,閱四年,乃至成都。展轉求得繩詒墓龍腦橋側,持喪歸,自撰入蜀記述其事。

李澄,字仲瀾,雲南昆陽人。明季,充選拔貢生。父兆旂,官廬江訓導,死寇難,幼子淳從死。澄奔赴,收父骨返葬,請於當事,得立祠,晨必詣祠拜且泣。寇至,奉母洪避山谷。洪病亟,言不原以山谷終,負母投佛寺,遽卒,負遺骸攢祖墓。順治初,山惈入州城,劫官舍,發藏粟。省吏以兵至,執澄將殺之,兵中有識澄者,乃免。澄因言:「山惈迫饑寒,無與百姓事。今固不宜累百姓,即山惈亦不宜輕言剿,否則且反戈。」乃坐其渠,州民以安。兄弟凡八,與仲弟俱,老,相友愛。

劉獻煜,字臺凝,陜西華陰人。父濯翼,明崇禎間官武昌,母與偕,遘亂絕消息。順治初,獻煜徒步求父母,亂初定,道阻,屢瀕險乃達。哭山徑中,遇叟識濯翼殯所,發得磚,硃書姓名裏貫皆具,猶濯翼所自記也。乃負骨歸葬。

錢美恭,浙江山陰人。父士驌,明官云南陽宗知縣,與妾之官,美恭留侍母。康熙元年,美恭得請於母,求父,至雲南,乃知士驌遷嵩明知州,卒葬通海。美恭至通海,得故僕導詣士驌墓,得庶母及幼弟。貧無貲,留五年,乃負骨歸葬。

趙萬全,浙江會稽人。父應麟,明季客授北游,萬全始二歲。既長,問母:「父安在?」母告以故。年十九,出求父。應麟初客京師,遇亂轉徙死馬邑。萬全遍訪江、淮間,亦至京師,心疑應麟死,見道有遺骸,刺血滲之,不得入,則號於路。又自京師西,亦至馬邑。馬邑人張文義,嘗招應麟主書者,死為之殯。一日遇萬全,問得其事,導至殯所,慟絕良久,乃裹應麟骨負以歸。既卒,吏為之祠,琢石表異孝。

劉龍光,字蓼蕭,湖南長沙人。父廷諤,仕明為益王長史。師下江西,克建昌,益王遁,廷諤逃山中。龍光以應試家居,聞亂疾作。居五年,乃行詣建昌,不得父母所在。禱於神,夢聞人語在石際,諮石際所在,有女僧示以路。行小徑萬山中,經藤峽至白石嶺。徑絕險,攀援顛頓,蒲伏上下。嶺盡至石際,於村民姚氏家遇其母,廷諤已前一年卒。居數月,輿櫬奉母歸。所居村曰見娘堡,相傳宋王龍山於此遇母,故得名云。

李芳巇,小字葵生,湖南湘鄉人。明季流寇至,湘鄉當孔道,三復三陷,芳巇父母皆被掠。兄弟死於兵者三,芳巇收葬之,棄家,求父母所在。行數年至貴陽,遇鄉人必為言父狀,或謂軍中某所頗有狀似所言者,詣求之,果得父。父脫軍中籍與歸。再出,又數年至寶慶,暮投山家宿,見二嫗操作,其一方理炊,乃似母。芳巇自陳尋母狀,嫗聞遽呼曰:「汝葵生耶?吾即汝母也!」蓋母避兵轉徙,方從此嫗為傭,遂奉母還。

唐肇虞,江南人,失其縣。父卒,肇虞尚幼,晝夜哭。母止之,曰:「母哭,能止兒勿哭耶?」順治初,江南寇大起,母被掠。肇虞遍求諸村落及旁郡縣,渡江北,復南行數千里,屢與寇遇,僅乃免,卒不得母。至江寧,眾問所自來,泣以情告。一嫗前問曰:「若母非戴姓

耶?」曰:「然。」嫗引至家,則其母在焉,相見大慟,遂侍母歸。

繆士毅,江南天長人。父廊賓,富。順治十七年,寇掠其家牛馬,怨家誣以助寇,廊賓見法,妻子徙奉天。士毅以後世父得免,依從母以長。既聞父死母徙狀,從母語之曰:「而母將行,抱汝乳,且言兒僅此一乳,乳當飽,生死與兒訣矣!」士毅聞,號泣,欲行求母,恐去不得還,先娶妻生子,康熙二十二年乃決行。至沈陽,遇族人同徙者,知母在烏喇為流人薛氏妻。乃行求得母,母不相識,士毅具言姓名及兩女兄適誰某,皆信,相抱哭,觀者多流涕。母於法不得還,乃辭歸。居數年,復往,母又徙愛琿。行未至,聞母死,求得母葬所,遂居其側僧廬,不復歸。

子秉文,長,躬至愛琿,泣請歸,士毅終不可。又數年,卒母葬所。秉文乃發祖母瘞,並持父骨還葬。

陸承祺,字又祉,浙江仁和人。父夢蘭,客死鬱林。方軍興,逾年乃得問。承祺與弟承祚號慟,走萬里,歷險阻,僅得達。睹叢箐中敗棺,刺血漉骨皆不入,兄弟哭愈哀。途中有知夢蘭者,告其棺在佛寺,兄弟從以往,撫棺慟,皆隕絕,觀者嗟嘆呼孝子。持水飲之,承祚

徐甦,承祺氣結不屬,竟死。承祚匱兩骸擔以歸。母王得承祚報,知得夢蘭骨及承祺死狀,悲慟不食,七日,未見承祚歸,遽卒。

汪龍,江南歙縣人。祖客死蘇州,父往迎喪,溺採石,龍時六歲。稍長,聞祖喪未歸,如蘇州求祖柩,無知者。久之,遇灌園叟與徙其祖柩,引詣殯舍,諸柩縱橫,匍匐諦審,柩有祖名,乃奉以歸。龍侍母孝,一夕,疽發背,委頓甚,自力勿使母聞,越數旬始瘥,母竟未知也。

方如珽,休寧人。國初,其曾祖避兵客死潛山。祖前卒,父不在側,道梗,喪未歸。如珽既長,問老婢,言有族姑嫁程氏。年七十餘,訪之,則嘗會其曾祖喪。偕往蹤跡,至黃石阪,於洞中得敗棺,得白金簪,族姑驗之,其曾祖斂時物也。乃負骨歸葬,距其曾祖卒時,已五十有六年矣。

張燾,福建連江人。父震公,家縣東岱堡,海寇破岱堡,張氏殲焉。震公適他往,獨免。燾方七歲,為所掠,轉徙傭於清漳。康熙十年,燾年二十餘矣,時時念父母。顧被掠時幼,不審鄉縣,以人謂其語音似連江,而追憶父似名天貞,乃走還連江,數日無所鄉。或問何為,以張天貞問。震公聞之,曰:「天貞,吾亡弟,彼焉識之?」走視問其詳,喜挾以歸,使見母。燾追憶母容貌,曰:「非吾母也。」震公曰:「汝母已死於賊,此汝後母耳。」燾大慟,為母補行喪服三年,而事後母如母。

硃壽命,江西餘干人。康熙十三年,遇寇,與母李相失,壽命日夜泣。既,聞母為禁旅所俘,屬正藍旗。壽命徒步走京師,乞於市,忍饑積錢將贖母。久之得母所在,而主者邀重購,拒壽命。壽命日跽其門外,膝為痺。侍讀學士邵遠平高其行,為捐金以贖,暫留遠平家。母卞,小不當意輒詬罵,或捽而批其頰,壽命益嬉笑。居數月,附舟還。壽命不知書,語質,每言:「在母腹日敢母血三合,那忍不報?」

潘天成,字錫疇,江南溧陽人。年十三,遇家難,父母挈子女出避仇。天成行後,幾為仇所斃。既得免,乃行求父母。經青陽白沙廟,宿廢廟,聞虎聲,為詩述悲。往來徽州、寧國所屬州縣,跡父母所在,至則又他徙。天成行經村聚,輒播兆鼓作鄉語大呼。至江西界,母金自巷出,就問之,始相識。乃得父及其弟、妹,皆無恙。時天成年十五,欲歸苦無貲,出行貸。又六年,使其弟從父歸,天成奉母挈妹以行。遇風雪,負母行數里,還抱妹,往復跣行,足流血,入雪盡殷。既歸,出行販以養,暇則讀書。荊溪湯之錡出高攀龍門,治性理之學,賢天成,天成從受業焉。同縣許國昌遇天成尤厚,使為童子師。鄰家兒詈母,天成召其鄉老人呼兒共懲之,兒悔謝乃巳。及父母卒,游學桐城,遂隸籍為安慶府學生。居二十餘年,移家江寧,天成學益進,狷潔不以乾當道。終窮餓,年七十四卒,葬惠應寺側。國昌子重炎,師天成,編刻其遺書為鐵廬集。

翁運槐,字楫山;運標,字晉公:浙江餘姚人。父瀛,往廣西,道湖南。一夕,泊舟祁陽新塘,失所在,舟人求不得,還報,歸其行篋,鎖在而鑰亡。時運槐、運標皆幼,運槐年十三,行求父不得,以病歸。運標,雍正元年成進士,與運槐復求父,遍湖南境,更二年不得。一夕,復泊新塘,遇土人鄭海還,言距今三十年,弟海生墮水,格敗葦不死。視葦間有尸,因瘞之白沙洲,身有鑰在囊,藏為識。乃遣力以囊鑰還,鑰與行篋鎖牝牡合,囊則運槐女兄昔年制以奉父者也。乃痛哭啟攢,以父喪還葬,而於瘞處留封樹焉,時雍正五年八月也。

運標謁選,得湖南武陵知縣。嘗有兄弟爭田訟,運標方詣勘,忽掩涕。訟者請其故,曰:「吾兄弟日相依,及官此,與吾兄別。今見汝兄弟,思吾兄,故悲耳。」訟者為感泣罷訟。縣東堤圮,水虐民,縣又無書院,運標為修築,民以運標姓名其堤與書院。擢道州知州,縣通

郴、桂,鑿山八十餘里為坦道。疫,親持方藥巡視,曰:「我民父母,子弟病,奈何不一顧耶?」年六十,卒官。

運標知武陵,建祠白沙洲,起鑰亭,買田,俾鄭氏世董之。知道州,拜祠下,哀感行路。

楊士選,字有貞,江南吳縣人。方六歲,入塾,塾師為說古人孝行,輒窮其本末,歸告父母:「兒他日亦當如是。」父商於河南。喪貲而病。士選年十六,往省,渡河風雨,士選泣禱得不覆,人稱「孝子舟」,奉其父還里。歲饑,士選與妻唐食糠籺,共營甘旨奉父母。居喪營葬,身穿負土,唐為姑吮疽。

徐大中,湖北潛山人。潛山俗重風水,大中喪母,厝棺居室傍未葬。乾隆四十七年,縣大水,齧前和,失其尸,大中大慟。水初退,求尸於沙中,得一足,衣蔑敗猶未盡,色餘黃,其母斂時裝也。大中抱足泣,路人見者語曰:「去此二里許,樹上懸尸,濕綿裹,缺一足。」奔視良是,但脫頤下骨,負歸改斂。忽有人若丐入其家,曰:「吾拾得頤下骨。」取與合,人傳為異。學官欲上其事,大中曰:「我久不葬母,乃遘此禍,我天地間一罪人耳。舉我孝,於及時葬親者謂何也?」堅卻之。

沈仁業,字振先,江蘇吳縣人。父賈於安南,娶婦生子女,仁業八歲從父歸,而母為外國女,例不得入中國,不能從。仁業長而思母,父卒,乃圖父像,渡海省母。安南有兵事,母挾幼子女竄山谷中,仁業行求得之,不食七日矣。居二年,有義其行者為具舟,舟入海,颶作,觸海中山。仁業抱母泣,風轉,挾母過山至瓊州。吏執例拒仁業母不得入,仁業涕泗請,莫應。久之,有老吏謂康熙間有故事,檢文書得之,仁業乃奉母及弟妹以歸。

魏樹德,陜西蒲城人。父季龍,出佐幕客游,樹德猶在娠。幼劬學,母力針黹以活。季龍久不歸,樹德以嘉慶十五年舉於鄉,乃行求父。初聞季龍自福建轉客廣東,先詣福建,求不得,乃詣廣東,遇知季龍者,為約略言葬處,遍求之,得志石荒塚中,乃持喪還。逾年,母卒,廬墓三年。除高陵訓導,求呂柟遺書,授諸生。久之,以老乞歸,卒。

李汝恢,江西廬陵人。父仲鴻,業醫,游無方。汝恢年十三,出求父。初至四川,又至廣東,皆未遇。乃節日用得百金,復出,遍涉江湖,遇仲鴻貴築。仲鴻有弟亦出游,既歸,日念弟。汝恢乃更出求其從父,得諸柳州。仲鴻乃樂甚,遽無疾而卒,汝恢喪葬盡禮。母痺,奉事尤謹。

鄭立本,江蘇蕭縣人。父相德,坐罪戍新疆,立本方四歲。年十八,辭母以求父,母哭而誡之曰:「汝父左手小指缺一節,中有橫紋,幸相值,以此為驗。」立本貧無貲,乞且行,至庫車。聞父戍綏來,綏來至庫車,三千餘里,張格爾亂未定,官道塞,乃里糧求路,獨行迷失道,還庫車。待亂定,乃行至綏來,則父歿已數年。相德在戍授同戍子弟讀,歿,弟子為治葬。立本哭墓而病,居二年,相德弟子力護視,故得不死。病起,啟父瘞,體久化,左手獨存小指,缺一節,有橫紋,如母言。立本駭慟,聞其事者皆嘆異,乃負骨歸葬,往還凡八年。同治中,大學士曾國籓駐軍徐州,聞立本事,招往見,立本舉孟子召役往,召見不往語,謝不往見。國籓高其義,檄知縣以時存問。

李學侗,山西介休人。諸生。生廷儀,道光中客死貴州荔波縣,有同行者斂而葬焉。學侗志欲歸父喪,貧,客授十餘年,積數百金,始克行。詣荔波,時方亂,貴州境亦騷動,屢遇險,乃達。廷儀葬社稷壇山下,或以為先農壇,語廷儀同行者音轉,又以為西龍塘。學侗至,求西龍塘,無其地。慟哭周行諸叢塚,乃於社稷壇得焉。學侗持喪還葬,族人有客死而旅殯者,並載以歸。既葬,日必往視,持盂飯以祭。晚治易,有所撰述。

董士元,直隸臨榆人。父行健,嘉慶中出關,去三月而士元生,行健遂不歸。士元幼思父,六歲,嘗失所在,翼日得之關外二里店。母問其故,涕泣言曰:「欲尋父也。」年十五,戚商於奉天,士元請於母,從之往,求父消息不能得。越十餘年,至阿什河,有言十年前在三姓南淘淇,嘗遇臨榆人,董姓,今不知存亡。士元乃往淘淇,地僻,行失道,久之始得達。舉父姓名里居問居人,有知者,曰:「是嘗漁於此,死數年矣。」士元大慟,得槁葬地,發塚審視,齧指血滴入骨,函以歸。至奉天,乃具棺還葬。居二十餘年,母歿,喪葬如禮。至光緒初卒。

李復新,湖北襄城人。崇禎末歲饑,復新出糴於郾。土寇賈成倫劫殺其父際春,復新歸,痛甚,誓復仇。時方亂,法不行,而成倫悍甚,復新乃謬懦示無復仇意,成倫易之。順治初,復新始告官,獄成,會赦,成倫得減死。吏監詣徒所,復新伏道旁,俟其至,舉大石擊之,死。詣縣請就刑,縣愍其孝,上府,請勿竟獄,且旌表其門。府駮議,謂成倫已遇赦減死,復新擅殺,當用殺人律坐罪。縣有老掾復具牘上府曰:「禮言父母之仇,不共戴天。又言報仇者,書於士殺之無罪。赦罪者一時之仁,復仇者千古之義。成倫之罪,可赦於朝廷,復新之仇,難寬於人子。成倫且欲原貸,復新不免極刑,平允之論,似不如是。復新父子何辜,並遭大戮?凡有人心,誰不哀矜!宜貰以無罪,仍旌其孝。」府乃用縣議,表其門日「孝烈」。

黨國虎,陜西富平人。明末,父兄為族子所殺,國虎方幼。順治初,國虎稍長,誘族子於野,撾殺之,並其子,詣縣自首入獄。知縣郭傳芳將貸之,國虎念父兄仇已雪,遂自經獄中。唐時縣人梁悅復親仇,傳芳立孝義祠,首悅而配以國虎。

嚴廷瓚,浙江烏程人。父時敏。族子暘,以姑為明大學士溫體仁妻,怙餘勢,時敏嘗斥其非。暘陽與出游,擠墮水死。廷瓚稍長,聞父死狀,訟暘論斬。暘賄上官反其獄,得脫,益肆。廷瓚奉母避長興,買斧誓復仇。歲還里省墓,遇暘,陽暱就之,暘以為畏己也。母卒,以喪歸。方村演劇,暘高坐以觀。廷瓚直前斧裂其首,斷項,詣縣自首。縣嘉其孝,欲生之,獄上,按察使將援韓愈復仇議為請,廷瓚遽死獄中,或曰暘家賄獄吏殺之。

陸起鵾、起鵬,貴州安順人。父希武。明末水西安邦彥叛,破安順,陸氏舉室自焚,希武與起鵬幸得脫。起鵾自火中跳而出,遇賊,為所掠。居數月,賊攻貴陽,自間道出求父及弟,未得。順治初,師下安順,起鵾乃歸。詗知起鵬所在,鬻產贖以歸。起鵬具言父為邦彥黨羅戎所殺,被掠鬻入土司中。時戎已就撫,起鵾兄弟訴父前為戎殺事,下巡道,巡道判戎罰鍰。起鵾始不肯受,既而曰:「不受金,是使戎知吾必報也。」乃受金,戎謂訟已決,不為備。起鵬故善騎射,結壯士七,日夜伺戎隙。一日,戎以事入安順,其徒皆從,起鵾、起鵬與七人者盟,挾弓弩伏城外,令所親醉戎。戎既醉而出,起鵬射戎中肩,即前斫之,七人者皆起,盡縛其徒,得與戎同殺父者四人,剖心以祭父。起鵾令起鵬走避,戎黨訴巡道,起鵾赴質,抗辯不稍屈,巡道釋不問。

虞爾忘、爾雪,江南無錫人。國初江南多盜,爾忘、爾雪父罕卿董鄉團,捕盜,盜惎焉。一日自縣還,聞門外呼,罕卿出,為盜縛去。爾忘、爾雪方田作,聞馳救,罕卿死橋下矣。爾忘、爾雪既葬父,仍董鄉團,乃更其初名,「忘」,警忘仇;「雪」,冀雪恨也。每獲盜,必詰執殺罕卿者,久之,知為盜杜息。息方謀入海,與所左右二人夜治行,爾忘、爾雪詗知之,將壯士奄至息家,縶息及二人者至罕卿死所。比明,爾忘抱罕卿木主至,爾雪於其旁爇釜,爾忘取息舌,爾雪探心肝,且祭且敢,爾忘乃斷息頭。將刃二人者,一讋死,一乞哀,沉諸河。爾忘、爾雪持息頭懸罕卿墓,時距罕卿死方逾月。

黃洪元,江南丹陽人。父國相,與同里虞庠不相能。方社,國相被酒夜行,庠遣惡少綁而沉諸河。洪元與弟福元皆幼,稍長,微聞父死狀,庠欲壻洪元以自解,洪元巽言謝之。母喪,既葬,洪元、福元同詗庠所在。又值社,洪元見庠在社所,還呼福元,各持斧往,洪元入迫庠,字庠曰:「逸群,我死汝!」庠起猶曰:「孺子醉耶?」洪元曰:「將醉汝血!」兩斧並舉,遂殺庠。詣縣自陳狀,有司義之,免福元,下洪元獄。明年,亦赦出,為浮屠以終。

顏中和,吳縣人。父弘仁。順治初,怨家周昌乘亂誘而殺之,棄其首。中和礪斧束槁如人形,書昌姓名以試斧。昌聞之,輕中和幼,不為備。中和懷斧出跡昌,值市中,尾之行。稍前,遽揮斧中昌,昌左右顧,又斧之。母遣其兄孟和走視弟,昌已死。乃相與詣縣,兄弟爭自承殺人,市人言殺昌者實中和,乃下中和獄。明年巡按御史錄囚,釋中和。中和,明義士佩韋從孫也。

同時又有顏鼇,父仲常,國初為其仇金瑞甫所殺。鼇淬刃挾以出入,一日,遇諸胥口,鼇刺瑞甫,入水,鼇從之。瑞甫脫去,誣鼇以盜。兵備道王紀、同知劉瑞訊得實,為誅瑞甫。

中和復仇時年十六,鼇年十八。

王恩榮,字仁庵,山東蓬萊人。縣有小吏寵於官,恩榮父永泰與有隙,被毆死。恩榮方九歲,祖母、母皆劉氏。祖母以告官,不得直,畀埋葬銀十兩,內自傷,遽縊。母泣血三年,病垂死,以官所畀銀授恩榮曰:「汝家以三喪易此,汝志之不可忘!」

恩榮依其舅以居,稍長,補諸生。志復仇,以斧自隨,其舅戒之曰:「汝志固宜爾,然殺人者死,汝父母其餒矣。」乃娶妻,生子,辭於舅,挾斧行。遇小吏,揮斧不中,投以石,僕,得救免;又遇於門,直前斫其首,帽厚,傷未殊。訴官,時去永泰死十九年,事無證。恩榮出母所授銀,其上有硃批,旁鈐以血書。知縣嘆曰:「孝子也!吾欲聽爾,違國家赦令;吾欲撓爾,傷人子至情。周官有調人,其各相避已耳。」於是恩榮哭,堂上下皆哭,小吏避之棲霞。

居八年,一日,方入城,過小巷,恩榮與遇,小吏無所逃,乞貸死。恩榮曰:「吾父遲爾久矣!」斧裂其腦,以足蹴其心,死。乃詣縣,小吏家言永泰故自縊,非毆死,當發棺以驗。恩榮曰:「民原抵罪死,不原暴父骸。」叩頭流血。知縣諮於眾,皆曰:「恩榮言是。」具狀上按察使,按察使議曰:「律不言復仇,然擅殺行兇人,罪止杖六十,即時殺死者不論,是未嘗不許人復仇也。恩榮父死時未成童,其後屢復仇不遂,非即時,猶即時矣。況其視死無畏,剛烈有足嘉者,當特予開釋,復其諸生。」有司將請旌,其舅為辭罷。

楊獻恆,山東益都人。父加官,與濟南楊開泰有隙,詈其門,開泰訟焉。加官率獻恆走求援,開泰遣其徒紿使出小徑,要而毆之,加官死焉。獻恆死復蘇,開泰以他事誣之,下濟南獄。山東初設總督,獻恆訟焉,下青州府勘問,直獻恆,開泰以賄免。獻恆走京師叩閽,下山東巡撫會鞫,罰開泰納埋葬銀四十兩,迫獻恆具領。獻恆藏銀典肆,再走京師叩閽,下山東巡撫,以獄已定罪,獻恆妄訴,笞四十。開泰計必欲殺獻恆,遣其子承恩至青州謀諸吏。獻恆潛知之,持鐵骨朵挾刃至所居。承恩方與吏耳語,伺其出,以鐵骨朵擊之,僕,急拔刀斷其喉,又抉其睛啖之,詣縣自陳,出所藏銀為證。縣具獄,得末減,遣戍。

任騎馬,直隸新城人。父為仇所戕,死以四月八日,方賽神,被二十八創。騎馬時方幼,至七歲,問母,得父死狀,慟憤,以爪刺胸,血出。悲至,輒如是,以為常。其仇姓馬,因自名騎馬。長,慮仇且疑,乃字伯超,詭自況馬超也。母欲與議婚,力拒。母死,治葬,且營祭田。年十九,四月八日復賽神,騎馬度仇必至,懷刃待於路。仇至,與漫語,指其笠問值,騎馬左手脫笠授仇,蔽其目,右手出刃急刺,洞仇胸,亦二十八創乃止。仇妻子至,怖甚,騎馬曰:「吾殺父仇,於汝母子何與?」乃詣縣自首。知縣欲生之,曰:「彼殺汝,汝奪刃殺之耶?」騎馬對曰:「民痛父十餘年,乃今得報之,若幸脫死,謂彼非吾仇,民不原也。」因袒,出爪痕殷然,見者皆流涕。獄具,得緩決。

在獄十餘年,知縣嘗使出祭墓,辭,怪而問之,曰:「仇亦有子,假使效我而斫我。我死,分也,奈何以累公?」新城人皆賢之,請於縣,築室獄傍,為娶妻生子。久之,赦出。知縣後至者欲見之,輒辭。聞其習形家言,以相宅召,又謝不往,曰:「官宅不同於民,若言不利,且興役,是以吾言擾民也。」既卒,總督曾國籓旌其廬曰「孝義剛烈」。

李巨勛,甘肅禮縣人。回亂,土豪羅五殺其父,巨勛欲赴死,母以弟幼沮之,命之娶,不可,乃訟五,五系獄,始娶生子。五以賄出獄,巨勛與弟恆挾刃伺五。光緒初,竟擊殺五,巨勛自首系獄,瘐死。母不食,亦卒。妻張,撫孤子成立。

任四,甘肅渭源人,農也。徙家狄道,父死於虎,四乃習鳥槍,誓殺百虎報父仇。遇虎,槍一發立殕。鄰縣有虎,輒迎四往捕,必得。四已老,計所殺虎九十有九,復入山伺虎,虎驟至,槍不及發,幾為所噬。俄雲起晝晦,虎自去,四歸祭父,戒子孫毋更仇虎,遂以無疾卒。卒時,猶寢虎皮也。

王國林,湖南長沙人。有膂力。虎咥其父,國林奮擊,折虎左牙。虎怒,爪其腹,腹破,腸出尺許,而父卒死。國林死復甦,家人納其腸,為縫腹,得愈。乃制火器獵虎,最後獲一虎,左牙折,知為咥父者,烹之,告父墓。

藍忠,福建漳浦人。家萬山中,父元章,與叔裕比屋居。有虎夜出,中伏弩,跳踉入所居村。裕夢中聞虎至,呼,虎撲門不得入,登屋毀杗桷直下,齧殺裕。元章聞裕為虎殺,復呼,虎循聲至,破屋撲元章,僕。忠持長刀直前,刺虎中喉,刃入腹三尺許。虎舍元章撲忠,忠拔刀柄脫,妻卓搤虎頸,連呼曰:「斧!」忠自門後取斧力斫之。天明:力且盡,視虎已殪。元章尚臥地,忠與妻扶就寢,越日,創甚竟死。


\end{pinyinscope}