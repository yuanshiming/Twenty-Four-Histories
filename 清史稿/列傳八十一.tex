\article{列傳八十一}

\begin{pinyinscope}
李衛田文鏡憲德諾岷陳時夏王士俊

李衛,字又玠,江南銅山人。入貲為員外郎,補兵部。康熙五十八年,遷戶部郎中。世宗即位,授直隸驛傳道,未赴,改雲南鹽驛道。雍正二年,就遷布政使,命仍管鹽務。三年,擢浙江巡撫。四年,命兼理兩浙鹽政。疏言:「浙江戶口繁多,米不敷食。請撥鹽政歸公銀十萬,委員赴四川採運減糶,款歸司庫;有餘,以修理城垣。」衛整理鹽政,疏言:「諸場有給丁灘蕩者,以丁入地,計畝徵收;無給丁灘蕩者,暫令各丁如舊輸納。」又言:「浙省私販出沒,以海寧長安鎮為適中孔道,請設兵巡隘。」又言:「江南蘇、松、常、鎮四府例食浙鹽,鎮江接壤,淮鹽偷渡。請敕常鎮道及京口將軍標副將、城守參將等督飭將吏水陸巡緝。五年,奏修海寧、海鹽、蕭山、錢塘、仁和諸縣境海塘。

尋授浙江總督,管巡撫事。六年,奏言:「江、浙界上盜賊藏匿,浙省究出從盜,咨江南震澤縣捕治,竟以替身起解。案中諸盜,江南督臣範時繹留以待讞。今察出有舉人金士吉等徇庇,當請褫奪,並提江南所留諸盜窮究黨羽,剪除巢穴。」得旨嘉獎。溫、臺接壤,瀕海有玉環山,港嶴平衍,土性肥饒。前總督滿保因地隔海汊,禁民開墾。衛遣吏按行其地,奏請設同知,置水陸營汛。招民墾田,於本年起科;設灶煎鹽,官為收賣;漁舟入海,給牌察驗;魚鹽徵稅,充諸項公用。衛經畫浙東諸縣水利:鄞縣大嵩港溉田數萬畝,歲久淤淺,衛令疏濬,築塘設閘,開支河溉田。鎮海靈巖、大丘二鄉有浦口通海,舊有閘已圮,衛令修築。定海多曠土,衛令察丈清理。上虞瀕海潮汐沒民田,衛為奏請除額;縣有夏蓋湖,積淤多已成田,衛令察丈,許民承業升科。

上以江南多盜,時繹及巡撫陳時夏非戢盜之才,命蘇、松等七府五州盜案,令衛兼領,將吏聽節制。時議增築松江海塘,並以舊塘改土為石,上復以時繹未能董理,令衛勘議。衛詣勘,奏言:「松江海塘已築二千四百餘丈,未築者當令仿效海鹽舊塘,石塘後附築土塘,宜一例高厚,歲派員修治。」上從之,仍令衛會時繹、時夏董理。上以衛留心營務,江南軍政舉劾,復命衛會同考核。尋遣侍郎彭維新等如江南清察諸州縣積欠錢糧,亦令衛與聞。七年,加兵部尚書。入覲,遭母喪,命回任守制。尋復加太子少傅。江寧有張云如者,以符咒惑民,衛遣詗察,得其黨甘鳳池、陸同庵、蔡思濟、範龍友等私相煽誘狀。八年,衛令游擊馬空北往捕,時繹故與雲如往還,與按察使馬世烆庇不遣,賄空北還稟衛。衛疏劾,上遣尚書李永升會鞫,時繹奪官,世烆、空北皆坐譴,雲如等論斬。九年,疏請改定蘇州府營制。

衛在浙江五年,蒞政開敏,令行禁止。上以查嗣庭、汪景祺之獄,停浙江人鄉會試,衛以文告嚴督。逾年,與觀風整俗使王國棟疏言兩浙士子感恩悔過,士風丕變,乃命照舊鄉會試。上督責各直省清釐倉庫虧空、錢糧逋欠,衛召屬吏喻意,簿書、期會、吏事皆中程,民間亦無擾。

十年,召署刑部尚書,授直隸總督,命提督以下並受節制。十一年,疏劾步軍統領鄂爾奇壞法營私,紊制擾民。上為奪鄂爾奇官,命果親王及侍郎莽鵠立、海望按治,得實,請罪鄂爾奇。上以鄂爾奇為鄂爾泰弟,曲宥之;獎衛,命議敘。乾隆元年,命兼管直隸總河,裁營田觀察使,敕衛覈議。衛請以營田交諸州縣收管,分轄通永、霸易、天津、清河、大名五道,統率經理。下部如所議。二年,疏發誠親王府護衛庫克與安州民爭淤池,赴州囑託。上命治庫克罪,嘉衛執法秉公,賜四團龍補服。三年,疏劾總河硃藻貪劣,藻弟蘅挾制地方官,干預賑事。上命尚書訥親、孫嘉淦按治,奪藻官,並罪蘅如律。

衛在直隸六年,蒞政如在浙江時。屢奏請審正府縣疆界,改定營汛,增置將吏。衛尤長於治盜。盜匿山澤間,詗得其蹤跡,遣將吏捕治,必盡得乃止。以是所部乃無盜。病作,乞解任,遣御醫診視。卒,賜祭葬,謚敏達。

世宗在籓邸,知衛才,眷遇至厚,然察衛尚氣,屢教誡之。其在雲南,或有餽於衛,衛又令制「欽用」牌入儀仗。上諭之曰:「聞汝恃能放縱,操守亦不純。川馬骨董,俱當檢點。又制『欽用』牌,是不可以已乎?爾其謹慎,毋忽!」衛奏言:「受恩重,當不避嫌怨。」上又諭之曰:「不避嫌怨,與使氣凌人、驕慢無禮,判然兩途。汝宜勤修涵養,勉為全人,方不負知遇。」及赴浙江,時河決硃家海,上命中途與河道總督齊蘇勒議施工。衛見齊蘇勒,決口已合龍,議頗不相協。衛錄問答語以聞。會衛族弟懷謹等居鄉放縱,衛令淮徐道捕送拘禁,族人騰謗。衛疏言:「臣開罪範時繹,又與齊蘇勒不無芥蒂,皆臣本籍大吏,恐因家事心跡難明。」上諭之曰:「時繹不足論,齊蘇勒與有芥蒂,或汝禮貌疏慢所致,咎不在齊蘇勒。凡審事辨公私最為不易,向日於鄰里鄉黨間先存嫌怨,則又當別論。朕每言公中私、私中公,樞機正在於此。」及在直隸,上復諭之曰:「近有人謂卿任性使氣,動輒肆詈。丈夫立身行己,此等小節不能操持,尚何進德修業之可期?當時自檢點,從容涵養。」

高宗南巡,見西湖花神廟衛自範像並及其妻妾,號「湖山神位」,諭曰:「衛仰借皇考恩眷,任性驕縱,初非公正純臣。託名立廟,甚為可異!」命撤像毀之。

田文鏡,漢軍正黃旗人。康熙二十二年,以監生授福建長樂縣丞,遷山西寧鄉知縣,再遷直隸易州知州。內擢吏部員外郎,歷郎中,授御史。五十五年,命巡視長蘆鹽政,疏言:「長蘆鹽引缺額五萬七千餘道,商人原先輸課,增復原引。自五十六年為始,在長清等縣運行。」得旨:「加引雖可增課,恐於商無益。」下九卿議行。山東巡撫覈定題覆如所議。尋擢內閣侍讀學士。雍正元年,命祭告華嶽。是歲山西災,年羹堯入覲,請賑。上諮巡撫德音,德音言無災。及文鏡還,入對,備言山西荒歉狀。上嘉其直言無隱,令往山西賑平定等諸州縣,即命署山西布政使。

文鏡故有吏才,清釐積牘,剔除宿弊,吏治為一新。自是遂受世宗眷遇。二年,調河南,旋命署巡撫。疏請以陳、許、禹、鄭、陜、光六州升直隸州。尋命真除。文鏡希上指,以嚴厲刻深為治,督諸州縣清逋賦,闢荒田,期會促迫。諸州縣稍不中程,譴謫立至。尤惡科目儒緩,小忤意,輒劾罷。疏劾知州黃振國,知縣汪諴、邵言綸、關陳等。上遣侍郎海壽、史貽直往按,譴黜如文鏡奏。四年,李紱自廣西巡撫召授直隸總督,道開封,文鏡出迓。紱責文鏡不當有意蹂躪讀書人,文鏡密以聞,並謂紱與振國為同歲生,將為振國報復。紱入對,言振國、諴、言綸被論皆冤抑,知縣張球居官最劣,文鏡反縱不糾。上先入文鏡言,置不問。球先以盜案下部議,文鏡引咎論劾。是冬,御史謝濟世劾文鏡營私負國、貪虐不法,凡十事,仍及枉振國、言綸、諴,庇球諸事,與紱言悉合。上謂濟世與紱為黨,有意傾文鏡,下詔嚴詰,奪濟世官,遣從軍,振國、諴論死,戍言綸、陳於邊。振國故蔡珽屬吏,既罷官,以珽薦復起。及珽得罪,上益責紱、珽、濟世勾結黨援,擾國政,誣大臣,命斬振國。

文鏡疏請以河南丁銀均入地糧,紳衿富戶,不分等則,一例輸將,以雍正五年始。部議從之。五年,疏言黃河盛漲,險工迭出。宜暫用民力,每歲夏至後,將距堤一二里內村莊按戶出夫,工急搶護,事竟則散。若非計日可竣者,按名給工食。下部議行。尋授河南總督,加兵部尚書。文鏡初隸正藍旗,命抬入正黃旗。六年,上褒文鏡公正廉明,授河南山東總督,諭謂此特因人設官,不為定例。文鏡疏言:「兩省交界地易藏匪類,捕役越界,奸徒奪犯,每因拒劫,致成人命,彼界有司仍復徇庇。請嗣後越界捕盜,有縱奪徇庇者,許本省督撫移咨會劾。」上從之。文鏡先以河南漕船在衛輝水次受兌,道經直隸大名屬濬、滑、內黃三縣,隔省呼應不靈。請以三縣改歸河南。既,又以河南徵漕舊例,河北三府起運本色,餘皆徵折,在三府採買,偏重累民。請以儀封、考城及新改歸河南濬、滑、內黃等五縣增運本色。距水次最遠靈寶、閿鄉二縣,減辦米數,歸五縣徵輸。南陽、汝寧諸府,光、汝諸州,永寧、嵩、盧氏諸縣,皆以路遠停運,分撥五縣協濟,按道路遠近,石加五分至二錢三分各有差。又疏言:「山東倉庫虧空,挪新掩舊。請如河南交代例,知府、直隸州離任,所轄州縣倉庫,令接任官稽察,如有虧空,責償其半,方得赴新任。道員離任,所轄府、直隸州倉庫亦視此例。」又疏言:「山東錢糧積虧二百餘萬,雍正六年錢糧應屆全完之限,完不及五分,由於火耗太重、私派太多。請敕山東巡撫、布政使協同臣清察,期以半年參追禁革,毋瞻徇,毋容隱。」上皆用其議。七年,請設青州滿洲駐防兵,屯府北東陽城址,下議政王大臣議行。尋加太子太保。疏請以高唐、濮、東平、莒四州升直隸州,改濟寧直隸州降隸兗州府。

旋命兼北河總督。是歲山東水災,河南亦被水,上命蠲免錢糧。文鏡奏今年河南被水州縣,收成雖不等,實未成災,士民踴躍輸將,特恩蠲免錢糧,請仍照額完兌。部議應如所請,上仍命文鏡確察歉收分數,照例蠲免,現兌正糧,作下年正供。九年,諭曰:「上年山東有水患,河南亦有數縣被水,朕以田文鏡自能料理,未別遣員治賑。近聞祥符、封丘等州縣民有鬻子女者。文鏡年老多病,為屬吏欺誑,不能撫綏安集,而但禁其鬻子女,是絕其生路也。豈為民父母者所忍言乎?」並令侍郎王國棟如河南治賑。文鏡以病乞休,命解任還京師。病痊,仍命回任。十年,復以病乞休,允之。旋卒,賜祭葬,謚端肅。命河南省城立專祠。又以河道總督王士俊疏請,祀河南賢良祠。

高宗即位,尚書史貽直奏言士俊督開墾,開捐輸,累民滋甚。上諭曰:「河南自田文鏡為督撫,苛刻搜求,屬吏競為剝削,河南民重受其困。即如前年匿災不報,百姓流離,蒙皇考嚴飭,遣官賑恤,始得安全,此中外所共知者。」並命解士俊任,語詳士俊傳。乾隆五年,河南巡撫雅爾圖奏河南民怨田文鏡,不當入河南賢良祠。上諭曰:「鄂爾泰、田文鏡、李衛皆皇考所最稱許者,其實文鏡不及衛,衛又不及鄂爾泰,而彼時三人素不相合。雅爾圖見朕以衛祀賢良,借文鏡之應撤,明衛之不應入。當日王士俊奏請,奉皇考允行,今若撤出,是翻前案矣!」寢雅爾圖奏不行。

憲德,西魯特氏,尚書明安達禮孫也。父善,官頭等侍衛。憲德初以廕生授理籓院主事,再遷刑部郎中。雍正四年,授湖北按察使。時布政使張聖弼坐虧空論罪,憲德上官,聖弼詣謁,憲德下諸獄。疏聞,上獎其能執法。尋就遷巡撫。

五年,調四川。張獻忠之亂,四川民幾盡。亂初定,吳三桂叛,其將吳之茂、王屏籓等入川,與我師久相持,民受其害,土曠人稀。康熙間,休養久,墾闢漸廣,經界未正,田糧多不實。巡撫馬會伯奏請清丈,以調湖北未行,上以諮憲德。憲德奏:「四川昔年人民稀少,田地荒蕪。及至底定,歸復祖業,從未經勘丈,故多所隱匿。歷年既久,人丁繁衍。奸猾之徒,以界畔無據,遂相爭訟。川省詞訟,為田土者十居七八,亦非勘丈無以判其曲直。」上復諮川陜總督岳鍾琪,奏與憲德略同,乃下九卿議行。遣給事中高維新、馬維翰,御史吳鳴虞、吳濤如四川,會同松茂、建昌、川東、永寧四道分往諸州縣丈量:維新永寧道,維翰建昌道,鳴虞松茂道,濤川東道。鳴虞先期示復明舊額,憲德阻止之。他道凡民間屋基、墳墓、界埂、水溝、園林皆不入勘丈,鳴虞獨不然,民驚擾,又需索丈費。憲德疏請罷鳴虞,維新事先竟,上令續勘松茂道。濤治事迂鈍,維翰事亦竟,憲德請以佐濤。萬縣民愬濤丈量不公,懸旗聚眾,墊江、忠州民亦以為言。維新松茂道事又竟,憲德又疏請罷濤,以維新、維翰分勘川東道。七年十一月,通省勘丈畢。舊冊載上、中、下田地都計二十三萬餘頃,丈得四十四萬餘頃,增出殆及半;而諸土司地納糧以石計,亦次第具報,視原額加增。戶部奏請視丈出田地照則徵糧,上諭曰:「從前隱瞞,科則止據實更定,毋追咎。至額糧稍重諸州縣,即比照就近適中科則核減,俾紓民力。」憲德奏:「各屬徵糧科則,輕重懸殊。原重通江諸縣,籥請減輕;原輕郫、灌、溫江三縣,亦據實呈請原增。臣等擬原重田地,令與接壤地方相等比照科算;原輕田地,亦應按則加增,不致小民偏枯委曲。」於是成都、華陽、新津、郫、溫江、長壽諸縣俱增上則,灌縣增中則,綿州、綏寧改分上、中、下三則,江油增下則,潼川、屏山、雅州、名山、榮經、蘆山、峨眉、夾江、通江賦偏重,均視鄰縣量減,巴縣賦最輕,上田不及一分,以地瘠不增,他州縣皆仍舊則。其有丈見田少糧多,經原戶聲請,皆予開除。上命招他省民入川開墾丈增田畝,憲德奏請以丈增地畝分科則編字號,計數均分,戶給水田三十畝,或旱地五十畝;有餘丁,增水田十五畝或旱地二十五畝。丁多不能養贍,臨時酌增。或有多餘三五畝,亦一並給墾;畸零不成丘段者,酌量安置,給以照票,並牛種口糧,分年升科。皆下部如所議行。

八年,墊江、忠州民楊成勛等群聚為亂,署川陜總督查郎阿遣兵捕治,成勛自經死。獲其徒陳文魁、楊成祿等,得所為怨白,言禍起戊申年奉旨清丈,科派需索累民。查郎阿疏聞,諭曰:「四川清丈之議,始於馬會伯,而成於憲德。朕慎選科臣前往科理,誡以剔除積弊,安插善良,並非為加增賦稅而起。勘丈造冊,各官供應,皆令動帑支給,不使幾微煩擾我民。今年事竟,憲德具本代川民謝恩,謂通省士民,咸稱清理疆界,使強無兼並,弱無屈抑;又將田不敷糧之戶,悉予開除。疆界既已分明,額賦尤為公溥,朕以為經理得宜矣,豈意奸民嘯聚,竟以清丈苛虐為言?怨白稱奉旨清丈,豈憲德等但以清丈稱為奉旨,於前者奏請未曉諭於眾耶?陳文魁訴狀,並稱頌川省上司,是必憲德等沽譽干名,何不將朕德意宣播,而乃蒙混含糊,使奸民得以藉口耶?憲德既稱通省士民歡呼感戴,何以尚有陳文魁等暗結邪黨、肆行誹謗?可見平日化導未周,董戒不力,令憲德將朕此旨刊布曉諭。」

憲德撫四川七年,屢請更定州縣疆界,有所省置,收天全土司改流設州,並升雅州為府隸焉。憲德議開紫古礦廠,會兒斯堡生番入邊殺掠商民,上令封閉。憲德以川省米貴,請暫停商販。逾年歲稔,上令弛禁毋遏糴。初上官,以四川驛、鹽、茶三政皆屬按察使兼領,未足司稽覈,請增設驛鹽道專司其事,從之。及清丈事將竟,奏言鹽、茶積弊,請令清查地畝科道諸員兼司搜查。上諭曰:「川省鹽、茶既特設道員,自有責成,如不能勝任,當予參劾,別擇賢能。鹽、茶積弊,相沿已久,應從容清理,安可如此嚴急?奏請搜查,更屬謬妄。汝諸事料理過於促迫,不肯實心任事,於此奏畢見,後當深戒。」十一年,憲德奏鹽道曹源邠混發引目累商,諭曰:「鹽課引務,汝有督率之責。曹源邠果不法,當列款糾參。若止改撥不當,何難商酌更正。今但請敕部察議,將鹺政視如無涉,誠不知汝何意?朕甚鄙汝玷督撫統轄訓飭之任也!」

尋召還京,授工部尚書。十二年,調刑部,仍兼工部,署正紅旗滿洲都統。乾隆元年,命赴泰陵督工。五年,卒。子夢麟,自有傳。

諾岷,納喇氏,滿洲正藍旗人。先世居輝發。祖恩國泰,習漢書,天聰八年舉人,直秘書院,授禮部理事官,洊擢尚書。父那敏,官鑲黃旗滿洲都統。

諾岷,自筆帖式授戶部主事,再遷郎中。雍正元年,擢內閣學士,授山西巡撫。各直省徵賦,正供外舊有耗羨,數多寡無定。州縣以此供上官,給地方公用而私其餘;上官亦往往藉公用,檄州縣提解因以自私。康熙間,有議歸公者,聖祖慮官俸薄,有司失耗羨,虐取於民,地方公用無從取辦,寢其議不行。諾岷至山西,值歲屢歉,倉庫多虧空。諾岷察諸州縣虧空尤甚者,疏劾奪官,離任勒追;餘州縣通行調任,互察倉庫;並慮州縣不得其人,請敕部選賢能官發山西補用。二年,諾岷疏請將通省一歲所得耗銀提存司庫,以二十萬兩留補無著虧空,餘分給各官養廉。各官俸外復有養廉自此起。

布政使高成齡奏言:「直省錢糧向有耗羨,百姓既以奉公,即屬朝廷財賦。臣愚以為州縣耗羨銀兩,自當提解司庫,憑大吏酌量分給,均得養廉。且通省遇有不得已例外之費,即以是支應。至留補虧空,撫臣諾岷先經奏明,臣請敕下各直省督撫,俱如諾岷所奏,將通省一歲所得耗銀約計數目先行奏明,歲終將給發養廉、支應公費、留補虧空各若干一一陳奏,則不肖上司不得借名提解,自便其私。」上命總理事務王大臣九卿集議,議略謂提解火耗,非經常可久之道,請先於山西試行。上諭曰:「州縣火耗原非應有之項,因通省公費、各官養廉不得不取給於此。朕非不原天下州縣絲毫不取於民,而勢有所不能。州縣徵收火耗分送上司,州縣藉口而肆貪婪,上司瞻徇而為容隱,此從來之積弊所當削除者也。與其州縣存火耗以養上司,何如上司撥火耗以養州縣。至請先於山西試行,此言尤非。天下事惟有可行不可行兩端。譬如治病,漫以藥試之,鮮有能愈者。今以山西為試,朕不忍也。提解火耗,原一時權宜之計;將來虧空清楚,府庫充裕,有司皆知自好,各省火耗自漸輕以至於盡革,此朕之深原。各省能行者聽,不行者亦不強也。」自後各直省督撫以次奏請視山西成例提解耗羨,上以諾岷首發議,諭獎其通權達變,於國計民生均有裨益。上屢飭各省督察有司,耗羨既歸公,不得巧立名目,復有所取於民。給養廉,資公用,尚有所餘,當留備地方公事。河南耗羨餘款最多,特免地丁錢糧四十萬,即以所餘抵補。上諭謂此項出自民間,若公用充裕,仍當加恩本地官民,不令歸入公帑也。三年,諾岷以病乞假,命回旗調理。

初,貝子允禟以罪徙西寧,道出平定,太監李大成毆諸生,諾岷按讞,以大成方病,置未深究。上責諾岷瞻徇,命繼任巡撫伊都立覆讞,罪大成,奪諾岷官。十二年,卒。

陳時夏,字建長,雲南元謀人。康熙四十五年進士,考授內閣中書。三遷工部郎中,考選廣西道御史。雍正元年,授河南開歸道,仍帶御史銜。尋奏河北連年歉收,請發帑治賑,蠲免錢糧,上嘉允之。二年,遷湖北按察使,以在開歸道任封丘生員罷考,坐不能彈壓,奪官。三年,授直隸正定知府。四年,遷長蘆鹽運使,加布政使銜,署江蘇巡撫。疏陳蘇、松水利,請發帑興工。命副都統李淑德、原任山東巡撫陳世倌會勘,議先濬婁江,常熟福山塘、昭文白茆河、太倉七浦河、上海嘉定吳淞江、武進孟瀆、德勝新河、丹陽九曲河次第疏治。時夏復疏言江南錢糧,請視直隸、河南正耗統解布政使,督撫以下各給養廉,地方公事用耗銀報銷,從之。上知時夏有老母,命雲南督撫贈資斧,護至蘇州,復賜人葠。

六年,江蘇布政使張坦麟調山東,時夏以坦麟任內錢糧未清,疏請停赴新任;坦麟亦奏時夏令新任布政使趙向奎勒掯交代。上責時夏褊淺,才識不足,不能勝巡撫,命改署山東布政使,即以坦麟署江蘇巡撫。是時江蘇巡撫所屬七府五州,自康熙五十一年至雍正四年,積虧地丁錢糧至八百十三萬有奇,巡撫張楷請分年帶徵。及時夏至江蘇,催追促迫,民艱於輸納,事久未竟,上命時夏留江蘇會辦虧空。時夏請以舊欠均派新糧,分年徵收,上諭曰:「舊欠自有本人,舍此不追而均派新糧,是刁民因積欠而得利,良民因先輸而倍徵。從此人人效尤,誰復輸供正賦?且舊欠派入新糧,必致舊欠未完,新糧又欠。時夏因朕留之在蘇,乃欲藉此草率完結。命暫停徵比,交新任巡撫尹繼善清察。」上又遣侍郎彭維新等佐尹繼善察出積欠實一千萬有奇,上命以其中侵蝕、包攬四百數十萬分十年帶徵,民欠五百數十萬分二十年帶徵,並令視直隸、河南諸省已行例,每歲帶徵若干,次年免正賦若干。諭謂「蠲逋賦使頑戶偏蒙其澤,不若免新徵使眾民普受其惠也」。

七年,尹繼善劾時夏所舉知縣蔡益仁貪黷不職,下部議,降調。八年,以母憂歸。十二年,詣京師,命以僉都御史銜授霸州營田觀察使。奏文安、大城兩縣界內修築橫堤,請於堤東南尚家村建閘,堤內濬河,引子牙河水溉田,仍於北岸多用涵洞,俾水得宣洩。乾隆二年,奏請用區田法,選屬吏租民地試行。皆從之。授內閣學士。三年,卒。

王士俊,字灼三,貴州平越人。康熙六十年進士,改庶吉士。雍正元年,上特命以知州發河南待缺,除許州。田文鏡為巡撫,惡以科第起家者,有意督過之,士俊懼將及。文鏡增鹼地稅,民不堪,士俊具牒爭,冀以是劾罷邀名。布政使楊文乾奇士俊,曲護之。三年,文乾遷廣東巡撫,奏以士俊從。四年,題授肇高廉羅道。五年,署巡撫阿克敦察士俊所轄黃江廠稅虧稅銀千餘,疏劾。上諭之曰:「王士俊尚有用,小過猶可諒。當嚴飭令悛改。」尋召士俊詣京師。士俊發黃江廠庫官為布政使官達索規禮,阿克敦即令官達按鞫。士俊請改員嚴訊,阿克敦令按察使方原瑛會鞫。士俊即以阿克敦、官達、方原瑛朋謀徇私,揭吏部奏聞。會文乾亦以他事劾阿克敦、官達,上命解官達、原瑛任,令總督孔毓珣及文乾會鞫,並令士俊署布政使。士俊行至曲江,聞命,還廣東上官。會文乾卒,上命傅泰署巡撫,復遣通政使留保等如廣東會鞫,阿克敦等皆坐譴。六年,實授廣東布政使。九年,擢湖北巡撫。

十年,文鏡解任還京師,擢士俊河東總督,兼河南巡撫。十一年,疏劾學政俞鴻圖納賄行私,命侍郎陳樹萱按鞫,得實,鴻圖坐斬。文鏡在河南督州縣開墾,士俊承其後,督促益加嚴,又令州縣勸民間捐輸。高宗即位,戶部尚書史貽直奏言:「河南地勢平衍,沃野千里,民性純樸,勤於稼穡,自來無土不耕,其不耕者大都斥鹵沙磧之區。臣聞河南各屬廣行開墾,一縣中有報開十頃、十數頃至數十頃者,積算無慮數千百頃,安得荒田如許之多?推求其故,不過督臣授意地方官多報開墾,屬吏迎合,指稱某處隙地若干、某處曠土若干,造冊申報。督臣據其冊籍,報多者超遷議敘,報少者嚴批申飭,或別尋事故,掛之彈章。地方官畏其權勢,冀得歡心,詎恤後日官民受累,以致報墾者紛紛。其實所報之地,非河灘沙礫之區,即山岡犖確之地;甚至墳墓之側,河堤所在,搜剔靡遺。目下行之,不過枉費民力,其害猶小;數年後按畝升科,指斥鹵為膏腴,勘石田以上稅,小民將有鬻兒賣女以應輸將者。又如勸捐,乃不得已之策,今則郡縣官長,驅車郭門,手持簿籍,不論鹽當紳民,慰以好言,令其登寫,旋索貲鏹。地方官一年數換,則籍簿一年數更,不惟大拂民心,亦且有損國體。請敕廉明公正大臣前往清察。」上諭曰:「田文鏡為總督,苛削嚴厲,河南民重受其困。士俊接任,不能加意惠養,借墾地之虛名,成累民之實害。河南民風淳樸,竭蹶以從,甚屬可嘉。然先後遭苛政,其情亦至可愍矣!河南仍如舊例,止設巡撫。」以傅德代士俊。士俊至京師,命署兵部侍郎。

乾隆元年,復命署四川巡撫。士俊在河南,上蔡知縣貴金馬奉檄開墾,迫縣民加報地畝錢糧,武生王作孚等詣縣辨訴。貴金馬以聚眾閧堂揭士俊,士俊諭定讞毋及開墾,妄坐作孚等勒減鹽價,擬斬。傅德疏劾,下部議,士俊當奪官,上命仍留任。

士俊密疏陳時政,略言:「近日條陳,惟在翻駁前案,甚有對眾揚言,只須將世宗時事翻案,即系好條陳。傳之天下,甚駭聽聞。」又言大學士不宜兼部,又言各部治事,私揣某省督撫正在褒嘉,其事宜準;某省督撫方被詰責,其事宜駁。不論事理當否,專以逢合為心。又言廷臣保舉,率多徇情,甚或藉以索賄。上覽奏,怒甚,發王大臣公閱。御史舒赫德因劾:「士俊奸頑刻薄,中外共知。其為河南總督,勒令州縣虛報墾荒,苦累小民。近日巡撫傅德論劾,外間傳說士俊已命逮治,皇上猶冀其改惡向善,曲賜矜全。乃士俊喪心病狂,妄發悖論,請明正其罪。」上召王、大臣、九卿等諭之曰:「從來為政之道,損益隨時,寬猛互濟。記曰:『張而不弛,文武弗能;弛而不張,文武弗為。』堯因四岳之言而用鯀,鯀治水九載,績用弗成;至舜而後殛鯀。當日用鯀者堯,誅鯀者舜,豈得謂舜翻堯案乎?皇考即位之初,承聖祖深仁厚澤,休養生息,物熾而豐;皇考加意振飭,使紀綱整齊,此因勢利導之方,正繼志述事之善。迨雍正九年以來,人心已知法度,吏治已漸澄清,又未嘗不敦崇寬簡,相安樂易。朕纘承丕緒,泣奉遺詔,向後政務應從寬者悉從寬。凡用人行政,兢兢焉以皇考諴民育物之心為心,以皇考執兩用中之政為政。蓋皇祖、皇考與朕之心初無絲毫間別。今王士俊訾為翻駁前案,是誠何心?朕躬有闕失,惟恐諸臣不肯盡言;至事關皇考,而妄指前猷,謂有意更張,實朕所不忍聞。至謂大學士不宜兼部,大學士兼部正皇考成憲,士俊欲朕改之,是又導朕以翻案也,彼不過為大學士鄂爾泰而發。士俊河南墾荒,市興利之善名,行剝民之虐政,使敗露於皇考時,豈能寬宥?彼欲掩飾從前之罪,且中傷與己不合之人,其機詐不可勝詰。至謂部件題駁,懷挾私心,保舉徇情,夤緣賄囑,諸臣有則痛自湔除,無則益加黽勉,毋為士俊所訕笑,以全朕委任簡用之體可也。」解士俊任,逮下刑部,王大臣等會鞫,請用大不敬律擬斬立決,命改監候。二年,釋為民,遣還里。

六年,以爭占甕安縣民羅氏墓地,縱僕毆民,民自經死,民子走京師叩閽。命副都御史仲永檀如貴州,會總督張廣泗鞫,得實,論罪如律。二十一年,卒。

論曰:世宗以綜覈名實督天下,肅吏治,嚴盜課,實倉庫,清逋賦,行勘丈,墾荒土,提耗羨,此其大端也。衛、文鏡受上眷最厚,衛以敏集事,文鏡以驕府怨;然當時謂衛、文鏡所部無盜賊,斯亦甚難能矣。勘丈激亂,四川為最著;耗羨歸公,山西為最先;田賦懸逋,江蘇為最鉅;開墾害民,河南為最劇。世宗親決庶政,不歸罪臣下,故諾岷蒙褒,而憲德不尸其咎;時夏才短,事未克竟,亦不深責也。士俊及高宗初政,絀而猶用,乃創翻案之說,欲以熒主聽,箝朝議。心險而術淺,其得譴宜哉。


\end{pinyinscope}