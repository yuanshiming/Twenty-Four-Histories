\article{志七十六}

\begin{pinyinscope}
○樂八

清代樂制,有中和韶樂、丹陛大樂、中和清樂、丹陛清樂、導迎樂、鐃歌樂、禾辭桑歌樂、慶神歡樂、宴樂、賜宴樂、鄉樂,器則隨所用而各異,悉依樂部次第,臚列而備舉之。所獲籓屬樂器,列於宴樂,古所未詳,尤不可略。然第志其名稱形制而已,若夫尺度聲律,則有司存。

中和韶樂,用於壇、廟者,鎛鐘一,特磬一,編鐘十六,編磬十六,建鼓一,篪六,排簫二,塤二,簫十,笛十,琴十,瑟四,笙十,搏拊二,柷一,敔一,麾一。先師廟,琴、簫、笛、笙各六,篪四,餘同。巡幸祭方岳,不用鎛鐘、特磬,琴、簫、笛、笙各四,瑟、篪各二,餘同。用於殿陛者,簫四,笛四,篪二,琴四,瑟二,笙八,餘同。

鎛鐘,範金為之,凡十二,應十二律。其制皆上徑小,下徑大,縱徑大,橫徑小。乳三十六。兩角下垂。十二鐘各虡,大小異制。黃鍾之鐘,兩欒高一尺八寸二分二釐,甬長一尺零八分,以次遞減至應鍾之鐘,兩欒高九寸六分,甬長五寸六分八釐。黃鍾之鐘,十一月用之;大呂之鐘,十二月用之;太簇之鐘,正月用之;夾鍾之鐘,二月用之;姑洗之鐘,三月用之;仲呂之鐘,四月用之;蕤賓之鐘,五月用之;林鍾之鐘,六月用之;夷則之鐘,七月用之;南呂之鐘,八月用之;無射之鐘,九月用之;應鍾之鐘,十月用之。鐘之簨虡凡四,皆塗金、上簨左右刻龍首,脊樹金鸞,咮銜五採流蘇,龍口亦如之,下垂至趺。中簨有業,鏤雲龍。附簨結黃絨紃以懸鐘。左右兩虡,承以五採伏獅。下為趺,趺上有垣,鏤山水形。黃鍾、大呂、太簇三虡尺度同,夾鍾、姑洗、仲呂三虡尺度同,蕤賓、林鍾、夷則三虡尺度同,南呂、無射、應鍾三虡尺度同,用時不並陳,如以黃鍾為宮,則祗懸黃鍾之鐘。餘月仿此。

特磬,以和闐玉為之,凡十二,應十二律。其制為鈍角矩形,長股謂之鼓,短股謂之股,皆兩面為雲龍形,穿孔系紃而懸之。十二磬各虡,大小異制。黃鍾之磬,股長一尺四寸五分八釐,鼓長二尺一寸八分七釐。以次遞減,至應鍾之磬,股長七寸六分八釐,鼓長一尺一寸五分二釐。愈小者質愈厚,黃鍾之磬,厚七分二釐九豪,遞增至應鍾之磬,厚一寸二分九釐六豪。黃鍾之磬,十一月用之;大呂之磬,十二月用之;太簇之磬,正月用之;夾鍾之磬,二月用之;姑洗之磬,三月用之;仲呂之磬,四月用之;蕤賓之磬,五月用之;林鍾之磬,六月用之;夷則之磬,七月用之;南呂之磬,八月用之;無射之磬,九月用之;應鍾之磬,十月用之;磬之簨虡亦四,惟上簨左右刻鳳首,趺飾臥鳧,白羽硃喙。十二磬不並陳,當月則懸其一,與鎛鐘同。

編鐘,範金為之,十六鐘同虡,應十二正律、四倍律,夷則、南呂、無射、應鍾各有倍律。陰陽各八。外形橢圓,大小同制,惟內高、內徑、容積各不同。實體之薄厚,以次遞增。第一倍夷則之鐘,體厚一分三釐三豪,至第十六應鍾之鐘,體厚二分八釐四豪。簨虡塗金,上簨左右刻龍首,中、下二簨俱刻朵雲,系金鉤懸鐘。兩虡承以五採伏獅,下為趺,鏤山水形。

編磬,以靈壁石或碧玉為之,十六磬同虡,應十二正律、四倍律,與編鐘同。陰陽各八。皆為鈍角矩形,大小同制。股長七寸二分九釐,鼓長一尺九分三釐五豪,惟實體之薄厚,以次遞增。第一倍夷則之磬,厚六分六豪八絲,至第十六應鍾之磬,厚一寸二分九釐六豪。簨虡制同編鐘,惟上簨左右刻鳳首,趺飾臥鳧,白羽硃喙。

建鼓,木匡冒革,貫以柱而樹之。面徑二尺三寸四釐,匡長三尺四寸五分七釐,匡半穿方孔,貫柱上出擎蓋,下植至趺。蓋上穹下方,頂塗金,上植金鸞為飾。承鼓以曲木,四歧抱匡,趺四足,各飾臥獅。擊以雙桴,直柄圓首,凡鼓桴皆如之。

篪二,皆截竹為質,間纏以絲,橫吹之。一孔上出為吹口,五孔外出,一孔內出,又二孔並間下出為出音孔。管末有底,中開一孔,吹孔上留竹節以閉音。一姑洗篪,徑八分七釐,自吹口至管末,九寸九分五釐九豪,陽月用之。一仲呂篪,徑八分三釐二豪,自吹口至管末,九寸五分二釐五豪,陰月用之。

排簫,比竹為之,其形參差象鳳翼。十六管,陰陽各八,同徑殊長。上開山口單吹之,無旁出孔。自左而右,列二倍律、夷則,無射。六正律以協陽均。自右而左,列二倍呂,南呂,應鍾。六正呂以協陰均。管面各鐫律呂名,納於一櫝,而齊其吹口。櫝用木,形如幾,虛其中以受管。

塤有二,燒土為之,形皆橢圓如鵝子,上銳下平。前四孔,後二孔,頂上一孔,以手捧而吹之。一黃鍾塤,內高二寸二分三釐,腹徑一寸七分一釐七豪,底徑一寸一分六釐八豪,陽月用之。一大呂塤,內高二寸一分三釐三豪,腹徑一寸六分四釐二豪,底徑一寸一分一釐七豪,陰月用之。

簫二,截竹為之,皆上開山口,五孔前出,一孔後出,出音孔二,相對旁出。一姑洗簫,徑四分三釐五豪,自山口至出音孔,長一尺五寸八分四釐二豪,陽月用之。一仲呂簫,徑四分一釐六豪,自山口至出音孔,長一尺五寸一分五釐二豪,陰月用之。

笛二,截竹為之,皆間纏以絲,兩端加龍首龍尾。左一孔,另吹孔,次孔加竹膜,右六孔,皆上出。出音孔二,相對旁出。末二孔,亦上出。一姑洗笛,徑四分三釐五豪,自吹孔右盡,通長一尺二寸五分一釐七豪,陽月用之。一仲呂笛,徑四分一釐六豪,自吹孔右盡,通長一尺一寸九分七釐二豪,陰月用之。

琴,面用桐,底用梓,魨以漆。前廣、後狹、上圓、下方、中虛。通長三尺一寸五分九釐。底孔二,上曰龍池,下曰鳳池。腹內有天地二柱,天柱圓,當肩下;地柱方,當腰上。凡七弦,皆硃。第一弦一百八綸,第二弦九十六綸,第三弦八十一綸,第四弦七十二綸,第五弦六十四綸,第六弦五十四綸,第七弦四十八綸。軫七,徽十三。其飾岳山焦尾用紫檀,徽用螺蚌,軫結黃絨紃,承以魨漆幾。

瑟體用桐,魨以漆,前廣、後狹、面圓、底平、中高、兩端俯。通長六尺五寸六分一釐。底孔二,是為越。前越四出,後越上圓下平。凡二十五弦,弦皆二百四十三綸。中一弦黃,兩旁皆硃。設柱和弦,柱無定位,各隨宮調。弦孔飾螺蚌,承以魨金幾二。

笙二,截紫竹為管,環植匏中,匏或以木代之。管皆十七,束以竹,本豐末斂,管本近底削半露竅。以薄銅葉為簧,點以蠟珠,其上各按律呂分開出音孔。匏之半施橢圓短嘴,昂其末。中為方孔,別為長嘴如鳳頸,置於短嘴方孔中。末為吹口,氣從吹口入,鼓簧成音。小笙制如大笙而小,亦十七管,惟第一、第九、第十六、第十七管不設簧,有簧者凡十三管,餘均與大笙同。

搏拊,如鼓而小。面徑七寸二分九釐,匡長一尺四寸五分八釐。匡上施金盤龍二,銜小金鐶,以黃絨紃系之,橫置趺上。用時懸於項,擊以左右手。每建鼓一擊,則搏拊兩擊以為節。

柷,以木為之,形如方鬥,上廣下狹,三面正中各隆起為圓形以受擊,一面中為圓孔以出音。以趺承之,擊具曰止。

敔,以木為之,形如伏虎,背上有二十七齟刻,以趺承之。鼓之以籈,以竹為之,析其半為二十四莖,於齟上橫轢之。

麾,黃帛為之,繡九曲雲龍。上飾藍帛,繡紅日,日中繡中和字。上繡三臺星,左北斗,右南斗。帛上下施橫木,上鏤雙龍,下為山水形,皆魨金。硃杠,上曲為龍首以懸麾,麾舉樂作,麾偃樂止。

丹陛大樂,凡御殿受賀及宮中行禮皆用之。其器:戲竹二,大鼓二,方響二,雲鑼二,簫二,管四,笛四,笙四,杖鼓一,拍板一。簫、笛、笙同中和韶樂。

戲竹,析竹為之,凡二,各五十莖。魨硃,承以塗金壺盧,下有柄,亦魨硃。人各執其一,立丹陛上,合則樂作,分則樂止。戲音與麾同,其用亦與麾同。

大鼓,木匡冒革,面徑三尺六寸四分五釐,匡高三尺二寸四分。腹施銅膽,面魨黃,繪五採雲龍。匡魨硃,繪交龍,匡半金鐶四。承以魨硃架,架有鉤,以鉤鐶平懸之。架高六尺,鼓者藉蹈以擊之。

方響,以鋼為之,形長方,十六枚同虡,應十二正律、四倍律,與編鐘、磬同。形質皆同。惟以薄厚為次。倍夷則之厚,三分三豪四絲,遞增至應鍾之厚,六分四釐八豪。後面近上三分之一皆為橫脊,竅其上端,系以黃絨紃,懸於虡而斜倚之,擊以小鋼槌。各部樂皆同,惟馬上凱歌樂分用其八,人各一枚,擎而擊之。

雲鑼,範銅為之,十枚同架,應四正律、六半律,姑洗、蕤賓、夷則、無射四正律,半黃鍾至半無射六半律。皆四旁穿竅,以黃絨紃系於架,中四,左右各三,合三行為九宮形,其一上出。以薄厚為次,下右應姑洗之律,厚二釐五豪二絲。遞增至最上,應半無射之律,厚五釐九豪八絲。

管即頭管,以堅木或骨角為之,大小各一,皆前七孔後一孔,管端設蘆哨,入管吹之。大管以姑洗律管為體,徑二分七釐四豪,哨下口至末,長五寸七分六釐。小管以黃鍾半積同形管為體,徑二分一釐七豪,哨下口至末,長五寸六分二豪。皆間束以絲,兩端以象牙為飾。

杖鼓,上下二面,鐵圈冒革,復楦以木匡,細腰。匡高一尺九寸四分四釐、腰徑二寸八分八釐,兩端徑各八寸一分,上下面徑各一尺二寸九分六釐。面匡俱魨黃,繪五採雲龍,緣以綠皮掩錢。上下邊綴金鉤各六,以黃絨紃交絡之。腰加束焉。腰飾綠皮焦葉文。以魨硃竹片擊之。

拍板,以堅木為之,左右各三片。近上橫穿二孔,以黃絨紃聯之,合擊以為節。

中和清樂,用於冊尊典禮,宴饗進饌,除夕、元夕張燈亦用之。其器:雲鑼二,笛二,管二,笙二,杖鼓一,手鼓一,拍板一。笛、笙同中和韶樂,雲鑼、管、板同丹陛大樂。

杖鼓同丹陛大樂而小,或半之,或為三之二。

手鼓,木匡冒革,面徑九寸一分二豪,腰徑一尺二分四釐。以柄貫匡,持而擊之。

丹陛清樂,用於宴饗進茶、進酒,臨雍賜茶亦用之。樂器均與中和清樂同。

導迎樂、鐃歌樂,用於乘輿出入。鑾駕鹵簿則奏導迎樂,騎駕鹵簿則奏鐃歌之行幸樂,法駕鹵簿、大駕鹵簿則導迎樂間以鐃歌樂,惟大祀詣壇、廟則導迎樂、鐃歌樂設而不作。凡三大節進表及進實錄、聖訓、玉牒,又親耕、親蠶、授時、頒詔、殿試、送榜、迎吻,凡前導以御仗出入者,皆奏導迎樂。鐃歌之樂有鹵簿樂,其部一,曰鐃歌鼓吹。有前部樂,其部一,曰前部大樂。亦曰大罕波。有行幸樂,其部三:曰鳴角,曰鐃歌大樂,曰鐃歌清樂。有凱旋樂,其部二:曰鐃歌,曰凱歌。鹵簿樂與前部大樂並列,亦曰金鼓鐃歌大樂,凡圜丘、祈穀、常雩,用大駕鹵簿,則前部大樂、鐃歌鼓吹、行幸樂三部並陳。方澤,用法駕鹵簿,則陳前部大樂、鐃歌鼓吹。太廟、社稷及各中祀,用法駕鹵簿,則陳鐃歌鼓吹。朝會用法駕鹵簿同。御樓受俘,用法駕鹵簿,則陳金鼓鐃歌大樂。巡幸及大閱,用騎駕鹵簿,則陳鳴角鐃歌大樂、鐃歌清樂。凱旋郊勞,則奏鐃歌。回鑾振旅,則奏凱歌。

導迎樂用戲竹二,管六,笛四,笙二,雲鑼二,導迎鼓一,拍板一。笙、笛同中和韶樂,戲竹、雲鑼、管、板同丹陛大樂。

導迎鼓,制如大鼓而小,面徑二尺四分八釐,匡高一尺六寸二分。繪五採雲龍,腹施銅膽。旁施金鐶四,系黃絨紃。二人舁行,擊以硃槌。

鐃歌鼓吹用龍鼓四十八,畫角二十四,大銅角八,小銅角八,金二,鉦四,笛十二,杖鼓四,拍板四。笛同中和韶樂,板同丹陛大樂。

龍鼓,木匡冒革,面徑一尺五寸三分六釐,匡高六寸四分八釐。面匡繪飾金鐶俱如導迎鼓。鐶系黃絨紃,行則懸於項,陳則置於架。架攢竹三,貫以樞而搘之。

畫角,木質,中虛腹廣,兩端銳。長五尺四寸六分一釐二豪,上下束以銅,中束以藤五就,魨以漆。以木哨入角端吹之,哨長七寸二分九釐。

大銅角,一名大號,範銅為之,上下二截,形如竹筒,本細末大,中為圓球。納上截於下截,用則引而伸之,通長三尺六寸七分二釐。

小銅角,一名二號,範銅為之,上下二截。上截直,下截哆,各有圓球相銜,引納如大銅角,通長四尺一寸四釐。大角體巨聲下,小角體細聲高,不以長短論。

金,範銅為之。面平,徑一尺四寸五分八釐,深二寸二分七釐五豪。旁穿二孔,結黃絨紃貫於木柄,提而擊之。

鉦,範銅為之,形如槃。面平,口徑八寸六分四釐,深一寸二分九釐八豪,邊闊八分六釐四豪。穿六孔,兩兩相比,周以木匡,亦穿孔,以黃絨紃聯屬之。左右銅鐶二,系黃絨紃,懸於項而擊之。

杖鼓,同丹陛大樂,惟面繪流雲,中為太極。

前部大樂,用大銅角四,小銅角四,金口角四。大銅角、小銅角制同鐃歌鼓吹。

金口角,舊名瑣澾,木管,兩端金口,上弇下哆。管長九寸八分九釐。管上金口長二寸一分六釐,為壺盧形,加小銅槃二。管下金口長四寸八分六釐,刻管如竹節相間,前七孔,後一孔,以蘆哨入管端吹之。

鐃歌大樂,用金口角八,銅鼓二,銅點一,金一,鈸一,行鼓一。金口角同前部大樂,金同鐃歌鼓吹。

銅鼓,範銅為之,形如金,面徑九寸七分二釐,中隆起八分一釐,徑二寸六分七釐三豪。邊穿孔二,以黃絨紃懸而擊之。

銅點,制如銅鼓而小。

鈸,範銅為之,面徑六寸四分八釐,中隆起一寸二分九釐六豪,徑三寸二分四釐。穿孔貫紃,左右合擊以和樂。

行鼓,一名紘羅鼓。木匡冒革,上大下小,面匡繪飾如龍鼓。金鐶四,貫以黃絨紃。行則跨於馬上,陳則置於架。

鐃歌清樂,用雲鑼二,笛二,平笛二,管二,笙二,金一,鈸一,銅點一,行鼓一。笛、笙同中和韶樂,雲鑼、管同丹陛大樂,金同鐃歌鼓吹,鈸、銅點、行鼓同鐃歌大樂。

平笛,同中和韶樂,惟不加龍首尾。

行幸樂,合鐃歌大樂、鐃歌清樂之數,益以大銅角八,小銅角八,蒙古角二。大銅角、小銅角同鐃歌鼓吹。

蒙古角,一名蒙古號,木質,中虛末哆,上下二截。角有雌雄二制,雄角上口內徑三分四釐五豪,雌角上口內徑二分八釐五豪,皆於管端施銅口,以角哨納入吹之。雄者聲濁,雌者聲清。

鐃歌用大銅角四,小銅角四,金口角八,金四,鑼二,銅鼓二,鐃四,鈸四,小和鈸二,花匡鼓四,得勝鼓四,海笛四,雲鑼四,簫六,笛六,管六,篪六,笙六。大銅角、小銅角、金同鐃歌鼓吹,金口角同前部大樂,銅鼓、鈸同鐃歌大樂,簫、笛、篪、笙同中和韶樂,雲鑼、管同丹陛大樂。

鑼,制同銅鼓而厚,聲較銅鼓低小。

鐃,範銅為之,面徑一尺二寸。中隆起,穿孔貫紃,左右合擊。

小和鈸,制與鈸同,面徑七寸九分。中隆起,穿孔貫紃,均與鈸同。

花匡鼓,即腰鼓,木匡冒革,面徑一尺五寸二分,匡高一尺六寸,繪花文。座以檀,四柱交趺,以銅鐶懸鼓而擊之。

得勝鼓,木匡冒革,面徑一尺六寸一分,匡高五寸八分,繪雲龍。座為四柱,懸鼓於上而擊之。

海笛,制如金口角而小,通長九寸五分。

凱歌用雲鑼四,方響八,鈸二,大和鈸二,星二,銅點二,鐋二,簫四,笛四,管十二,笙四,杖鼓二,拍板二。簫、笛、笙同中和韶樂,雲鑼、管、杖鼓同丹陛大樂,鈸、銅點同鐃歌大樂。

方響,制同丹陛大樂,分用其八,人各一枚,擎而擊之。

大和鈸,制與鈸同,面徑一尺一寸八分。中隆起,穿孔貫紃,左右合擊。

星,範銅為之,口徑一寸八分,深一寸。中隆起,各穿圓孔,貫以紃,左右合擊。

鐋,範銅為之,面徑二寸七分,口徑三寸一分五釐,深六分。穿孔貫紃,擊以木片。

拍板三片,束其二,以一拍之。

禾辭桑歌樂,親耕、親桑用之。親耕用金六,鼓六,簫六,笛六,笙六,拍板六。親桑用金二,鼓二,簫、笛、笙各六,拍板二。簫、笛、笙同中和韶樂,板同丹陛大樂。

金制同鐃歌鼓吹而微小。槌用黃韋,瓜形,柄魨硃。

鼓,制如龍鼓而微小,懸於項擊之。

慶神歡樂,凡群祀用之。其器云鑼二,管二,笛二,笙一,鼓一,拍板一,惟祀先蠶及關帝、文昌則加隆焉。笛、笙、鼓同中和韶樂,雲鑼、管、板同丹陛大樂。

宴樂凡九:一曰隊舞樂,一曰瓦爾喀部樂,一曰朝鮮樂,一曰蒙古樂,一曰回部樂,一曰番子樂,一曰廓爾喀部樂,一曰緬甸國樂,一曰安南國樂。

隊舞有三:一曰慶隆舞,凡殿廷朝會宮中慶賀宴饗皆用之;一曰世德舞,宴宗室用之;一曰德勝舞,凱旋筵宴用之。三舞同制,皆舞而節以樂。其器用箏一,奚琴一,琵琶三,三弦三,節十六,拍十六。

箏,似瑟而小,刳桐為質,通長四尺七寸三分八釐五豪。十四弦,弦皆五十四綸,各隨宮調設柱。底孔二。前方,後上圓下平,通體魨金,四邊繪金夔龍。梁及尾邊用紫檀,弦孔以象牙為飾。

奚琴,刳桐為質,二弦。龍首,方柄。槽長與柄等。背圓中凹,覆以板。槽端設圓柱,施皮扣以結弦。龍頭下脣為山口,鑿空納弦。綰以兩軸,左右各一,以木系馬尾八十一莖軋之。

琵琶,刳桐為質,四弦,曲首長頸,平面圓背,腹廣而橢。槽面施覆手,曲首中間為山口。設檀軸四以綰弦,左右各二,山口上以黃楊木為四象,下以竹為十三品,按分取聲。中腰兩旁為新月形,腹內以細鋼條為膽,弦自山口至覆手,長二尺一寸六分,第一弦以硃飾之。

三弦,斫檀為質,修柄,方槽,圓角,冒以虺皮。柄貫槽中,柄末槽端覆以木。穿孔貫弦,匙頭下半鑿空納弦,以三軸綰之,左二右一。

節,編竹如箕,魨硃,背為虎形。用圓竹二,劃之以為節。

拍,紫檀皮四片,束其三,以一拍之。

太祖平瓦爾喀部,獲其樂,列於宴樂,是為瓦爾喀部樂舞。用觱篥四,奚琴四。奚琴同隊舞樂。

觱篥,蘆管,三孔,金口,下哆,中有小孔。管端開簧,簧口距管末四寸五分三釐。

太宗時,獲朝鮮國樂,列於宴樂,是為朝鮮國俳。用笛一,管一,俳鼓一。笛同中和韶樂,管同丹陛大樂。

俳鼓如龍鼓而小,懸於項擊之。

太宗平察哈爾,獲其樂,列於宴樂,是為蒙古樂曲。有笳吹,有番部合奏,皆為掇爾多密之樂,掌於什幫處。笳吹用胡笳一,箏一,胡琴一,口琴一。箏與隊舞所用同,惟設六弦。

胡笳,木管,三孔,兩端施角,末翹而哆。自吹口至末,二尺三寸九分六釐。

胡琴,刳木為質,二弦,龍首,方柄。槽橢而下銳,冒以革。槽外設木如簪頭以扣弦,龍首下為山口,鑿空納弦,綰以二軸,左右各一。以木系馬尾八十一莖軋之。

口琴,以鐵為之,一柄兩股,中設簧,末出股外。橫銜於口,鼓簧轉舌,噓吸以成音。

番部合奏,用雲鑼一,簫一,笛一,管一,笙一,箏一,胡琴一,琵琶一,三弦一,二弦一,月琴一,提琴一,軋箏一,火不思一,拍板一。簫、笛、笙同中和韶樂,雲鑼、管同丹陛大樂,箏、琵琶、三弦同隊舞樂。

胡琴,二弦,竹柄椰槽,面以桐。槽徑三寸八分四釐,為圓形,與笳吹之胡琴橢而下銳者不同。山口鑿空納弦,以兩軸綰之,俱在右。弦自山口至柱,長二尺三分五釐二豪,以竹弓系馬尾八十一莖軋之。

二弦,斫樟為質,槽面以桐,形長方,底有孔,槽面施覆手如琵琶。曲首後鑿空納弦,綰以兩軸,左右各一。弦長二尺三寸四釐,設十七品,按分取聲。

月琴,斫檀為質,四弦,槽面以桐,八角曲項,柄貫槽中,槽面施覆手。曲項鑿空納弦,綰以四軸,左右各二。弦長二尺三寸四釐,設十七品,與二弦同。

提琴,四弦,圓木為槽,冒以蟒皮而空其下,竹柄貫槽中,末出槽外。覆木扣弦,柄端鑿空納弦,綰以四軸,俱在右。以竹弓系馬尾,夾於四弦間軋之。

軋箏,似箏而小,刳桐為質,十弦。前後有梁,梁內弦長一尺六寸一分八釐,各設柱,以木桿軋之。

火不思,似琵琶而瘦,四弦,桐柄,刳其下半為槽,冒以蟒皮。曲首鑿空納弦,四軸綰之,俱在右。弦自山口至柱長一尺七寸七分四釐。

拍板,紫檀三片,束其二,以一拍之。

高宗平定回部,獲其樂,列於宴樂之末,是為回部樂技,用達卜一,那噶喇一,哈爾札克一,喀爾奈一,塞他爾一,喇巴卜一,巴拉滿一,蘇爾奈一。

達卜,木匡冒革,形如手鼓而無柄。有大小二制,一面徑一尺三寸六分五釐二豪,一面徑一尺二寸二分四釐,皆魨黃,面繪採獅,以手指擊之。

那噶喇,鐵匡冒革,上大下小,形如行鼓。旁有小鐶,系黃絨紃。兩鼓相聯,左右各以杖擊之。

哈爾札克,形如胡琴,椰槽,冒以馬革。上木柄,下鐵柄。槽底中開一孔,側開三小孔。以馬尾二縷為弦,上自山口穿於後,以兩軸綰之,左右各一,下系鐵柄。馬尾弦下設鋼絲弦十,上系木柄,下擊鐵柄,左右各五軸。另以木桿為弓,系馬尾八十餘莖,軋馬尾弦,應鋼弦取聲。

喀爾奈,狀如世俗洋琴,鋼絲弦十八,刳木中虛,左直右曲。左設梁如琴之岳山,以系鋼弦之本。鋼弦之末施木軸,似琴之軫,入於右端,高下相間作兩層,轉其軸以定弦之緩急。以手冒撥指,或木撥彈之,通體雙弦,惟第一獨弦。

塞他爾。形如匕,絲弦二,綱弦七,木柄通槽,下冒以革。面平背圓,柄有線箍二十三道,如琵琶之品。以九軸綰弦,柄端二軸綰絲弦。二面三軸,左側四軸,綰鋼質雙弦一,獨弦六。以手冒撥指,或木撥彈絲弦,應鋼弦取聲。

喇巴卜,絲弦五,鋼弦二,木柄通槽,槽形如半瓶,下冒以革。曲首鑿空納絲弦,以五軸綰之,左二右三,曲首右側以兩軸綰綱弦。用手冒撥指,或木撥彈絲弦,應綱弦取聲。

巴拉滿,木管,上斂下哆,飾以銅,形如頭管而有底,開小孔以出音。管通長九寸四分,七孔前出,一孔後出,管上設蘆哨吹之。

蘇爾奈,一名瑣澾,木管,兩端飾銅,上斂下哆,形如金口角而小。七孔前出,一孔後出,一孔左出,銅管上設蘆哨吹之。

高宗平定金川,獲其樂,及後藏班禪額爾德尼來朝,獻其樂,均列於宴樂之末,是為番子樂。金川之樂:曰阿爾薩蘭,曰大郭莊,曰四角魯。用得梨一,柏且爾一、得勒窩一。

得梨,似蘇爾奈而小。

柏且爾,範銅二片,圓徑六寸,中隆起,穿孔貫紃,左右合擊。

得勒窩,形似達卜。

班禪之樂:曰札什倫布,用得梨二,巴汪一,蒼清一,龍思馬爾得勒窩四。

得梨同金川樂,形制略大。

巴汪,似喇巴卜,七弦。

蒼清,制同雲鑼。

龍思馬爾得勒窩,似那噶喇而制以銅,面徑一尺三寸,底銳,匡高一尺。

高宗平定廓爾喀,獲其樂,列於宴樂之末,是為廓爾喀樂舞。用達布拉一,薩朗濟三,丹布拉一,達拉一,公古哩二。

達布拉,似那噶喇,一面冒革。有二制:其一面豐底銳,其一底微豐而漸削。四圍俱系韋絳,聯以採縷,懸之腰間,以左右手合擊之。

薩朗濟,刻木為質,韋弦四,鐵弦九。項長三寸,刳其中,面以魚牙刻佛為飾。柄長五寸二分,槽面闊三寸,自上刳之,冒以革。中腰削如缺月,束以黃韋。底橢,鑿空於項以納韋弦,左右各二。軸柄面穿孔九,自右至左,鱗次斜列,各納鐵弦。軸九,俱在右,上五下四。槽面設柱,中為九孔納鐵弦,上承韋弦。以柔木系馬尾軋韋弦,應鐵弦取聲。

丹布拉,刻桐為質,以大匏為槽,直柄,面平背圓,鐵弦四,綰以四軸,上二,左右各一。柄上以鐵片二為山口,一穿孔納弦,一承弦。

達拉,範銅二片,圓徑二寸一分。中隆起,穿孔,系以採縷,左右合擊。

公古哩,範銅為鈴,以採縷聯之,五十枚為一串,凡四串。歌時二人各系於股,雙足騰躍以出聲。

乾隆五十三年,緬甸國內附,獻其樂,列於宴樂之末,是為緬甸國樂。有粗細二制:粗緬甸樂,用接內搭兜呼一,稽灣斜枯一,聶兜姜一,聶聶兜姜一,結莽聶兜布一。

接內搭兜呼,木匡冒革,匡上有紐,系以帛,橫懸於項,以手擊之。

稽灣斜枯,制似雲鑼,其數八,上下各四,同懸於架。架後搘以二木,斜倚而擊之。

聶兜姜,木管銅口,近下漸哆,前七孔,後一孔。管端設銅哨,加蘆哨於上,管與銅口相接處,以銅簽掩之。

聶聶兜姜,形如金口角而小,木管木口,餘與聶兜姜同。

結莽聶兜布,範銅二片,圓徑三寸五分。中隆起,穿孔,貫以韋,左右合擊。

細緬甸樂,用巴打拉一,蚌札一,總稿機一,密穹總一,得約總一,不壘一,接足一。

巴打拉,以木為槽,形如船,通長二尺七寸五分。前後兩端各為山峰形,兩峰之尖,絡以絲繩。排穿竹板二十二片,皆闊一寸。第一片長五寸二分,厚三分五釐,以次則長遞加而厚遞減,至末片則長一尺一寸五分,厚一分。以竹裹綿為槌擊之。

蚌札,木匡冒革,上大下小。面徑六寸一分,底徑四寸,匡高一尺。四圍俱系韋絳,以手擊之。

總稿機,十三弦,曲柄,通槽,柄上曲如蠍尾。槽面冒革,為四圓孔以出音。順槽腹設覆手,穿孔十三,系弦,各斜引至柄束之,彈以手。

密穹總,三弦,木質,為魚形。體長方,腹下通長刳槽,無底,兩旁鐫鱗甲。面設品五,為小圓孔九以出音,前四,中四,後一。首形銳而上出,鐫須角鉅齒圓睛,尾形亦銳。項上以銅為山口,系硃弦三,尾有鐶納弦,旁穿孔,設軸,左二右一,以手彈之。得約總,三弦,木質,中虛,如扇形,中腰兩旁灣曲向內。頸半穿孔納弦,綰以三軸,左二右一,槽末施木以系弦。扣用木弓系馬尾八十餘莖軋之。

不壘,以竹為管,上端以木塞其半為吹口。七孔前出,一孔後出,最上一孔前出,加竹膜。

接足,範銅二片,口徑一寸八分。中隆起,穿孔貫紃,左右合擊。

乾隆五十四年,獲安南國樂,列於宴樂之末,是為安南樂舞。用丐鼓一,丐拍一,丐哨一,丐彈弦子一,丐彈胡琴一,丐彈雙韻一,丐彈琵琶一,丐三音鑼一。安南土語,凡樂器之名,俱以丐字建首。

丐鼓,木匡冒革,空其下,徑八寸四分,承以架。用竹桴二,或左手承鼓,右手以桴擊之。

丐拍,用檀板三:其一上端綴以連錢。其一背刻雁齒,其一右為鋸牙。左手執二板相擊,連錢激響,右手執鋸牙者,引擊雁齒,錯落成聲。

丐哨,即橫笛,截竹為筒,漆飾,二十一節。左第一孔為吹口,次加竹膜,右六孔,末二孔,俱上出,旁二孔對出,兩端飾以角。

丐彈弦子,三弦,斫檀為質,槽方而橢,兩面冒虺皮。匙頭鑿空納弦,以三軸綰之,左二右一。

丐彈胡琴,二弦,竹柄,槽形如筒,底微豐,面冒虺皮。曲首鑿空,兩軸俱自後穿前綰弦,弦自山口至柱,長一尺八寸,餘如番部合樂胡琴之制。

丐彈雙韻,如月琴,四弦,斫檀為質,槽面以桐,形如滿月。徑一尺一寸六分,厚一寸八分。曲項鑿空納弦,綰以四軸,左右各二。槽面覆手,山口下七品,俱以檀為之。

丐彈琵琶,四弦,刳桐為質,通長三尺。項上鑿空納弦,綰以四軸,左右各二。上設四象,下布十品。弦自山口至覆手,長二尺一寸四分。

丐三音鑼,範銅,三面,綰以鐵圈,聯如品字。上一徑二寸四分五釐,右一徑二寸三分八釐,左一徑二寸三分。承以檀柄,槌用角。

賜宴樂,凡經筵禮畢賜宴,文、武鄉、會試賜宴,宴衍聖公,宴正一真人皆用之。其器:雲鑼二,笛二,管二,笙二,鼓一,拍板一。笛、笙、鼓同中和韶樂,雲鑼、管、板同丹陛大樂。

鄉樂,凡府、州、縣學春、秋釋奠皆用之。其器麾一,編鐘十六,編磬十六,琴六,瑟二,排簫二,簫四,笛六,篪二,笙六,塤二,建鼓一,搏拊二,柷一,敔一。制皆同中和韶樂。

鄉飲酒用雲鑼一,方響一,琴二,瑟一,簫四,笛四,笙四,手鼓一,拍板一。琴、瑟、簫、笛、笙同中和韶樂,雲鑼、方響、板同丹陛大樂,手鼓同清樂。

節,中和韶樂用。結旄九重,蓋以金葉,束以綠皮。硃杠,上曲為龍首以銜旄。植架於東西各一,每架二節,司樂者執之以節舞。導文舞曰節,導武舞曰旌,旌亦曰節,制與節同。

千,中和韶樂用。木質,圭首,上半繪五採雲龍,下繪交龍,緣以五色羽文。中為粉地,硃書「雨暘時若,四海永清。倉箱大有,八方敉寧。奉三永奠,得一為正,百神受職,萬國來庭」。凡八語,佾各一語。幹背魨硃,有橫帶二,中施曲木,武舞生左手執之。

戚,中和韶樂用。木質,斧形,背黑刃白,柄魨硃,武舞生右手執之。

羽,中和韶樂用。木柄,植雉羽,銜以塗金龍首,柄魨硃,文舞生右手執之。

籥,中和韶樂用。六孔竹管,魨硃,文舞生左手執之。

舞有二:用於祀神者曰佾舞,用於宴饗者曰隊舞。凡佾舞武用干戚,文用羽籥。干戚曰武功之舞,羽籥曰文德之舞,祭祀初獻以武舞,亞獻終獻以文舞,惟先師廟、文昌廟初獻、亞獻、終獻皆以文舞焉。若大雩,則童子十六人衣皁衣,持羽翳,歌而舞皇舞,凡此皆隸於佾舞者也。隸於隊舞者,初名蟒式舞,亦曰瑪克式舞。乾隆八年,更名慶隆舞,內分大、小馬護為揚烈舞,是為武舞,大臣起舞上壽為喜起舞,是為文舞。是年巡幸盛京,筵宴宗室,增世德舞。十四年,平定金川,凱旋筵宴,又增德勝舞,三舞同制,各有樂章。揚烈舞,用戴面具三十二人,衣黃畫布者半,衣黑羊皮者半。跳躍倒擲,象異獸。騎禺馬者八人,介胄弓矢,分兩翼上,北面一叩,興。周旋馳逐,象八旗。一獸受矢,群獸懾伏,象武成。喜起舞,大臣二十二人,朝服儀刀入,三叩,興,退東位西鄉立。以兩而進,舞畢三叩,退。次隊繼進如前儀。此隊舞之大較也。外此則有四裔樂舞:東曰瓦爾喀、曰朝鮮,北曰蒙古,西曰回、曰番、曰廓爾喀,南曰緬甸、曰安南,皆同列於宴樂之末。瓦爾喀部樂舞,司舞八人,均服紅雲緞鑲壯緞花補袍,狐皮大帽,豫立丹陛之西。將作樂,進前三叩,退。司樂八人,分兩翼上,跪一膝,奏瓦爾喀樂曲。司舞進舞,以兩為隊,每隊舞畢,三叩,退。

朝鮮國俳,笛技、管技、鼓技各一人,均戴氈帽,鏤金頂,服藍雲緞袍,椶色雲緞背心,藍綢帶。俳長一人,戴面具,青緞帽,紅纓,服紅雲緞袍,白綢長袖綠雲緞虎補背心,十字藍綢帶。倒擲技十四人,服短紅衣。立丹陛兩旁。俳長從右翼上,北面立,以高麗語致辭,笛、管、鼓技從右翼上,東北面立,倒擲技從左翼上,自東向西,各呈其藝。

蒙古樂,笳吹,司樂器四人,司章四人,均蟒服,立丹陛旁。番部合奏,司樂器十五人,亦均蟒服,立丹陛旁,與笳吹一班同入。一叩,跪一膝,奏蒙古樂曲。

回部樂,司樂器八人,均錦衣絹裏雜色紡絲接袖衣,錦面布裏倭緞緣回回帽,青緞鞾,綠綢荅膊。司舞二人,舞盤二人,皆衣靠子錦腰襴紡絲接袖衣。倒擲大回子四人,皆衣靠子雜色紡絲接袖衣,戴五色綢回回小帽。小回子二人,雜色綢衣絹里。皆豫立丹陛下,俟朝鮮國俳呈技後,上丹陛作樂。司舞起舞,舞盤人隨舞。畢,倒擲小回子繼進呈技。

番子樂,金川之阿爾薩蘭,司樂器三人,司舞三人,為戲獅,身長七尺,披五色毛,番名僧格乙,引獅者衣雜採,手執繩,系耍球一,五色,番名僧格乙阿拉喀。大郭莊,番名大拉噶地,司舞十人,每兩人相攜而舞,一服蟒服,戴翎,掛珠,斜披黃藍二帶,交如十字;一服藍袍,掛珠,斜披黃紫二帶,交如十字。四角魯,番名得勒布,司舞六人,戴舞盔,番名達帽。插雞翎各六,番名達莫乙。背縛藤牌,番名賽斯丹。帶系腰刀,番名江格乙。左執弓,番名得木尼也。右執箭壺,番名柏拉。盛箭五枝,番名格必乙。相對而舞。班禪之札什倫布,番名柏拉噶,司樂器六人,司舞番童十人,各披長帶,手執斧一,番名沙勒鱉。舞而歌梵曲。

廓爾喀樂舞,司樂器六人,均衣回子衣,著紅羊皮鞾,內二人纏頭以洋錦,餘皆以紅綠布。司歌五人,均以紅綠布纏頭,內一人衣綠綢衣,著紅採履,餘皆回子衣、紅羊皮鞾。司舞二人,均衣紅綠綢衣,戴猩紅氈帽,金銀絲巾,著紅採履,束腰皆用雜色布。舞者每足各系銅鈴一串,曰公古哩,騰躍出聲,歌舞並奏。

粗緬甸樂,司樂器五人,司歌六人,均拖發扎紅,用緬甸衣冠。

細緬甸樂,司樂器七人,均拖發扎紅,衣藍緞短衣。司舞四人,衣閃緞短衣,皆雜色裙,以洋錦束腰,戴扎巾。歌合以粗樂,舞合以細樂。

安南國樂,司樂器九人,均戴道巾,衣黃鸝補服道袍,藍緞帶。司舞四人,衣蟒衣,冠帶與司器同。執採扇而舞。


\end{pinyinscope}