\article{志七十九}

\begin{pinyinscope}
○輿服三

皇帝御寶皇后金寶太皇太后皇太后金寶玉寶附皇貴妃以下寶印

皇子親王以下寶印文武官印信關防條記

清初設御寶於交泰殿,立尚寶司。其後以內監典守,當用則內閣請而用之。乾隆十一年,考定寶譜,藏之交泰殿者二十有五,藏之盛京者十。交泰殿所藏:曰「大清受命之寶」,以章皇序。白玉,方四寸四分,厚一寸。盤龍紐,高二寸。曰「皇帝奉天之寶」,以章奉若。碧玉,方四寸四分,厚一寸一分。盤龍紐,高三寸五分。曰「大清嗣天子寶」,以章繼繩。金,方二寸四分,厚八分。交龍紐,高一寸七分。曰「皇帝之寶」,以布詔赦。青玉,方三寸九分,厚一寸。交龍紐,高二寸一分。曰「皇帝之寶」,以肅法駕。栴檀香木,方四寸八分,厚一寸八分。盤龍紐,高三寸五分。曰「天子之寶」,以祀百神。白玉,方二寸四分,厚八分。交龍紐,高一寸三分。曰「皇帝尊親之寶」,以薦徽號。白玉,方二寸一分,厚七分。盤龍紐,高一寸三分。曰「皇帝親親之寶」,以展宗盟。白玉,方二寸二分,厚一寸二分。交龍紐,高一寸二分。曰「皇帝行寶」,以頒賜賚。碧玉,方四寸八分,厚一寸九分。蹲龍紐,高二寸五分。曰「皇帝信寶」,以徵戎伍。白玉,方三寸三分,厚六分。交龍紐,高一寸六分。曰「天子行寶」,以冊外蠻。碧玉,方四寸八分,厚一寸九分。蹲龍紐,高二寸三分。曰「天子信寶」,以命殊方。青玉,方三寸八分,厚一寸三分。交龍紐,高一寸七分。曰「敬天勤民之寶」,以飭覲吏。白玉,方三寸一分,厚一寸五分。交龍紐,高一寸七分。曰「制誥之寶」,以諭臣僚。青玉,方四寸,厚二寸。交龍紐,高二寸七分。曰「敕命之寶」,以鈐誥敕。碧玉,方三寸五分,厚一寸三分。交龍紐,高一寸八分。曰「垂訓之寶」,以揚國憲。碧玉,方四寸,厚一寸五分。交龍紐,高二寸。曰「命德之寶」,以獎忠良。青玉,方四寸,厚一寸四分。交龍紐,高二寸一分。曰「欽文之璽」,以重文教。墨玉,方三寸六分,厚一寸五分。交龍紐,高一寸六分。曰「表章經史之寶」,以崇古訓。碧玉,方四寸七分,厚二寸一分。交龍紐,高二寸二分。曰「巡狩天下之寶」,以從省方。青玉,方四寸七分,厚二寸。交龍紐,高二寸五分。曰「討罪安民之寶」,以張征伐。青玉,方四寸八分,厚二寸。交龍紐,高二寸五分。曰「制馭六師之寶」,以整戎行。墨玉,方五寸三分,厚一寸四分。交龍紐,高二寸二分。曰「敕正萬邦之寶」,以誥外國。青玉,方三寸八分,厚一寸五分。盤龍紐,高二寸三分。曰「敕正萬民之寶」,以誥四方。青玉,方四寸一分,厚一寸五分。交龍紐,高二寸。曰「廣運之寶」,以謹封識。墨玉,方六寸,厚二寸一分。交龍紐,高二寸。

盛京所藏:曰「大清受命之寶」,碧玉,方四寸八分,厚一寸九分。蹲龍紐,高二寸四分。曰「皇帝之寶」,青玉,方四寸八分,厚一寸九分。交龍紐,高二寸七分。曰「皇帝之寶」,碧玉,方五寸,厚一寸八分。盤龍紐,高三寸。曰「皇帝之寶」,栴檀香木,方三寸八分,厚六分。素龍紐,高五分。曰「奉天之寶」,金,方三寸七分,厚九分。交龍紐,高二寸。曰「天子之寶」,金,方三寸七分,厚九分。交龍紐,高二寸。曰「奉天法祖親賢愛民」,碧玉,方四寸九分,厚一寸五分。交龍紐,高二寸。曰「丹符出驗四方」,青玉,方四寸七分,厚二寸。交龍紐,高二寸二分。曰「敕命之寶」,青玉,方三寸七分,厚一寸八分。交龍紐,高二寸五分。曰「廣運之寶」,金,方二寸四分,厚八分。交龍紐,高一寸五分。

高宗御制國朝傳寶記曰:「國朝受天命,採古制為璽。掌以宮殿監正,襲以重盝,承以魨幾,設交泰殿中,以次左右列,當用則內閣請而用之。其質有玉、有金、有栴檀木。玉之品有白、有青、有碧。紐有交龍、有盤龍、有蹲龍。其文自太宗文皇帝以前,專用國書,既乃兼用古篆。其大小自方六寸至二寸一分不一。嘗考大清會典,載御寶二十有九,今交泰殿所貯三十有九。會典又云:『宮內收貯者六,內庫收貯者二十有三。』今則皆貯交泰殿,數與地皆失實。至謂『皇帝奉天之寶』即傳國璽,兩郊大祀及聖節宮中告天青詞用之,此語尤誕謬。大祀遵古禮,用祝版署名而不用寶。聖節宮中未嘗有告天事,或道籙祝釐,時一行之,亦不過偶存其教耳,未嘗命文人為青詞,亦未嘗用寶。且此璽孰非世之傳守,而專以一寶為傳國璽,亦不經。蓋緣修會典諸臣無宿學卓識,復未嘗請旨取裁,僅沿明時內監所書冊檔,承譌襲謬,遂至於此。甚矣紀載之難也。且會典所不載者,復有『受命於天既壽永昌』一璽,不知何時附藏殿內,反置之正中。按其詞雖類古所傳秦璽,而篆文拙俗,非李斯蟲鳥之舊明甚。獨玉質瑩潔如截肪,方得黍尺四寸四分,厚得方之三。以為良玉不易得則信矣,若論寶,無論非秦璽,即真秦璽,亦何足貴!乾隆三年,高斌督河時奏進屬員濬寶應河所得玉璽,古澤可愛,又與輟耕錄載蔡仲平本頗合。朕謂此好事者仿刻所為,貯之別殿,視為玩好舊器而已。夫秦璽煨燼,古人論之詳矣。即使尚存,政、斯之物,何得與本朝傳寶同貯?於義未當。又雍正年故大學士高其位進未刻碧玉寶,一文未刻,未成為寶,而與諸寶同貯,亦未當。朕嘗論之,君人者在德不在寶。寶雖重,一器耳。明等威、徵信守,與車旗章服何異?德之不足,則山河之險,土宇之富,拱手而授之他人,未有徒恃此區區尺璧,足以自固者。誠能勤修令德,系屬人心,則言傳號渙,萬里奔走,珍非和璧,制不龍螭,篆不斯籀,孰敢不敬信承奉,尊為神明。故寶器非寶,寶於有德。古有得前代符寶,君臣動色矜耀,侈為瑞貺者。我太宗文皇帝時,獲蒙古所傳元帝國寶,容而納之,初不藉以為受命之符。由今思之,文皇帝之臣服函夏,垂統萬世,在德耶?在寶耶?不待智者而知之矣。善夫唐梁肅之言曰:『鼎之輕重,璽之去來,視德之高下,位之安危。』然則人君承祖宗付畀,思以永膺斯寶,引而勿替,其非什襲固守之謂。謂夫日新厥德,居安慮危,凝受皇天大寶命,則德足重寶,而寶以愈重。璽玉自古無定數,今交泰殿所貯,歷年既久,紀載失真,且有重衣復者。爰加考正排次,定為二十有五,以符天數。並著成譜,而序其大恉如此。」又盛京尊藏寶譜序曰:「乾隆十一年春,閱交泰殿所貯諸寶,既詳定位置,為文記之。其應別貯者,分別收貯。至其文或衣復見,及國初行用者,為數凡十。雖不同現用之寶,而未可與古玩並列。因念盛京為國家發祥地,祖宗神爽,實所式憑。朕既重繕列祖實錄,尊藏鳳凰樓上,覲揚光烈,傳示無疆。想當開天之始,凝受帝命,寶符煥發,六服承式,璠興孚尹,手澤存焉。記不云乎,『陳其宗器』,弘璧琬琰,陳之西序,崇世守也。爰奉此十寶,齎送盛京,鐍而藏之,而著其緣起如此。」

乾隆十三年九月,改鐫御寶,始用清篆文,左為清篆,右為漢篆。高宗御題交泰殿寶譜序後曰:「寶譜成於乾隆十一年丙寅,越三年戊辰,始指授儒臣為清文各篆體書。因思向之國寶,官印,漢文用篆書,而清文則用本字者,以國書篆體未備也。今既定為篆法,當施之寶印,以昭畫一。按譜內青玉『皇帝之寶』,本清字篆文,傳自太宗文皇帝時,自是而上四寶,均先代相承,傳為世守者,不敢輕易。其檀香『皇帝之寶』以下二十有一,則朝儀綸綍所常用,宜從新制。因敕所司一律改鐫,與漢篆文相配,並記之寶譜序後云。」

乾隆四十五年八月,高宗七旬聖壽,用杜甫句刻「古稀天子之寶」,並御制古稀說,兼系以詩。四十六年正月,用乾清宮西暖閣貯「敬天勤民寶」之例,貯「古稀天子之寶」於東暖閣。

皇后金寶,清、漢文玉箸篆,交龍紐,平臺,方四寸四分,厚一寸二分。

康熙四年,制太皇太后金寶、玉寶,盤龍紐。餘皆與皇後寶同。玉寶臺高一寸八分,餘同金寶。

皇太后金寶、玉寶,俱盤龍紐。餘與皇後寶同。

皇貴妃金寶,清、漢文玉箸篆,蹲龍紐,平臺,方四寸,厚一寸二分。

貴妃金寶,與皇貴妃同。

妃金印,清、漢文玉箸篆,龜紐,平臺,方三寸六分,厚一寸。

康熙十五年,定皇太子金寶,玉箸篆,蹲龍紐,平臺,方四寸,厚一寸二分。

和碩親王金寶,龜紐,平臺,方三寸六分,厚一寸。親王世子金寶,龜紐,平臺,方三寸五分,厚一寸。多羅郡王鍍金銀印,麒麟紐,平臺,方三寸四分,厚一寸。俱清、漢文芝英篆。

外國王鍍金銀印,清、漢文尚方大篆,駝紐,平臺,方三寸五分,厚一寸。順治十年,以朝鮮國王原領印文有清字無漢字,命禮部改鑄清、漢文金印,頒給該王,仍將舊印繳進。

宗人府、衍聖公銀印,直紐,三臺,方三寸三分,厚一寸。俱清、漢文尚方大篆。

公、侯、伯銀印,虎紐,方三寸三分,厚九分。公三臺,侯、伯二臺。

經略大臣、大將軍、將軍、領侍衛內大臣銀印,虎紐,二臺,方三寸三分,厚九分。俱清、漢文柳葉篆。

軍機事務處銀印,直紐,二臺,方三寸二分,厚八分。宣統三年四月,改軍機處為內閣,舊內閣遂裁。

各部、都察院銀印,直紐,三臺,方三寸三分,厚九分。俱清、漢文尚方大篆。

理籓院銀印,直紐,三臺,方三寸三分,厚九分。兼清、漢、蒙古三體字,清、漢文尚方大篆,蒙古字不用篆。理籓院後改理籓部。

盛京五部銀印,直紐,二臺,方三寸二分,厚八分。盛京五部後裁。

戶部總理三庫事務銀印,直紐,二臺,方三寸二分,厚八分。戶部後改名度支部。

翰林院銀印,二臺,方三寸二分,厚八分。

內務府銀印,二臺,方三寸二分,厚八分。

景陵、泰陵內務府總管,東陵、泰陵承辦事務銅關防,凡關防皆直紐。長三寸,闊一寸九分。鑾儀衛銀印,直紐,二臺,方三寸二分,厚八分。宣統朝因避寫故名鑾輿衛。俱清、漢文尚方大篆。

通政使司、大理寺、太常寺、順天府、奉天府銀印,直紐,方二寸九分,厚六分五釐。通政司後裁,大理寺後改大理院,太常寺後歸並禮部。俱清、漢文尚方小篆。

詹事府銅印,直紐,方二寸七分,厚九分。

光祿寺、太僕寺、武備院、上駟院、奉宸苑銅印,直紐,方二寸六分,厚六分五釐。詹事府後裁,光祿寺後歸並禮部,太僕寺後歸並陸軍部。

內繙書房銅關防,長三寸,闊一寸九分,俱清、漢文尚方小篆。

國子監銅印,直紐,方二寸五分,厚六分。

太醫院銅印,直紐,方二寸四分,厚五分。

各道監察御史、稽察內務府御史、稽察宗人府御史、巡鹽御史銅印,直紐,有孔,方一寸五分,厚三分。

宗人府左、右司,太僕寺左、右司,鑾儀衛左、右司,各部、理籓院各司,銅印,直紐,方二寸四分,厚五分。

內務府各司銅印,直紐,方二寸二分,厚四分五釐。

崇文門稅務管理,坐糧戶部分司,工部木柴監督,工部木廠監督,工部管理街道各倉監督,工部後改並為農工商部。左、右翼管稅,戶部銀庫、緞匹庫,戶部辦理八旗俸餉,戶部辦理八旗現審,順天、奉天府丞,各關稅監督銅關防,長三寸,闊一寸九分。

巡視五城御史、管理古北口驛務,管理獨石口驛務銅關防,長二寸八分,闊一寸九分。

欽天監時憲書銅印,直紐,方二寸一分,厚四分四釐。

暢春園、圓明園、清漪園官房稅庫銅條記,凡條記皆直紐。長二寸六分,闊一寸九分。俱清、漢文鐘鼎篆。

大理寺左、右司,光祿寺四署,五城兵馬司銅印,直紐,方二寸二分,厚四分五釐。

中書科銅印,直紐,方二寸一分,厚四分五釐。

內閣典籍銅關防,長三寸,闊一寸九分。

翰林院典簿,禮部鑄印局,宣統三年印鑄局改屬新內閣,禮部亦改典禮院。理籓院銀庫,工部制造庫,工部料估所,各部、院督催所銅關防,長三寸,闊一寸九分。

順天府府治中、稽察盛京五部將軍衙門、稽察黑龍江等處、稽察寧古塔等處銅關防,長二寸九分,闊一寸九分。

兵馬司副指揮銅關防,長二寸六分,闊一寸六分。

宗人府經歷司銅印,直紐,方二寸四分,厚五分。

都察院經歷司銅印,直紐,方二寸二分,厚四分五釐。

鑾儀衛經歷司,各部、院、寺司務銅印,直紐,方二寸一分,厚四分四釐。

各壇、廟、祠祭署銅印,直紐,方二寸,厚四分二釐。

太醫院藥庫銅印,直紐,方一寸九分,厚四分二釐。

國子監典籍銅印,直紐,方一寸九分,厚四分二釐。

禮部鑄印局大使銅條記,長二寸六分,闊一寸九分。

兵馬司吏目銅條記,長二寸四分,闊一寸四分。俱清、漢文垂露篆。

護軍統領、前鋒統領、火器營統領銀印,虎紐,方三寸三分,厚九分。

提督九門步軍統領,圓明園總管八旗、內府三旗官兵銀印,虎紐,二臺,方三寸三分,厚九分。

總管雲梯健銳營八旗傳事銀關防,直紐,長三寸二分,闊二寸。俱清、漢文柳葉篆。

護軍統領、參領、協領、雲梯健銳營翼長、各處總管銅關防,長三寸,闊一寸九分。俱清、漢文殳篆。

八旗佐領,宗室、覺羅族長銅圖記,凡圖記皆直紐。方一寸七分,厚四分五釐。俱清文懸針篆。

咸安宮官學、景山官學、養心殿造辦處銅圖記,方一寸七分,厚四分。

看守通州三倉首領銅關防,長三寸,闊一寸九分。俱清、漢文懸針篆。

鎮守將軍銀印,虎紐,二臺,方三寸三分,厚九分。

副都統銀印,虎紐,二臺,方三寸二分,厚八分。俱清、漢文柳葉篆。

察哈爾都統銀印,虎紐,二臺,方三寸三分,厚九分。用滿洲、蒙古二種字,滿文柳葉篆。

總統伊犁等處將軍銀印,虎紐,二臺,方三寸三分,厚九分。兼滿、漢、托忒、回子四種字,滿、漢文俱柳葉篆,托忒、回子字不篆。

辦理伊犁、烏魯木齊等處事務大臣銀印,虎紐,二臺,方三寸三分,厚九分。兼滿、漢、托忒三種字,滿、漢文俱柳葉篆。

伊犁分駐雅爾城總理參贊大臣銀印,虎紐,二臺,方三寸三分,厚九分。兼滿洲、托忒、回子三種字,滿文柳葉篆。

辦理葉爾羌、喀什噶爾、阿克蘇諸處事務大臣銀印,虎紐,方三寸三分,厚九分。兼滿、漢、回子三種字,滿、漢文俱柳葉篆。

管理巴里坤等處事務大臣銀印,虎紐,二臺,方三寸三分,厚九分。

辦理哈密糧餉事務大臣銀印,虎紐,二臺,方三寸三分,厚九分。俱柳葉篆。

八旗游牧總管,察哈爾總管、城守尉銅印,方二寸六分,厚六分五釐。殳篆。

興京等城守尉銅關防,長三寸,闊一寸九分。

錦州等城守尉銅關防,長二寸九分,闊一寸九分。

駐防左、右翼長,協領、參領銅條記,長二寸六分,闊一寸六分五釐。俱殳篆。

防守尉銅關防,長二寸八分,闊一寸九分。

駐防佐領銅條記,長二寸六分,闊一寸六分五釐。俱清、漢文懸針篆。

直省總督、巡撫銀關防,直隸總督、陜甘總督、四川總督,鐫兼巡撫字樣。江西巡撫、河南巡撫,鐫兼提督字樣。山西巡撫,鐫兼提督鹽政字樣。長三寸二分,闊二寸,俱清、漢文尚方大篆。

欽差大臣銅關防,如督、撫式。三品以上用之。

各省承宣布政使司銀印,直紐,二臺,方三寸一分,厚八分。

各省提刑按察使司後改提法使。銅印,直紐,方二寸七分,厚九分。

各省鹽運使司銅印,直紐,方二寸六分,厚六分五釐。

各省提督學政後改提學使,並改關防為印信。銅關防,長二寸九分,闊一寸九分。俱清、漢文尚方小篆。

各處管理織造銅關防,長二寸九分,闊一寸九分。

各省守、巡道後於省會地方增設巡警道、勸業道。銅關防,長三寸,闊一寸九分,俱清、漢文鐘鼎篆。

欽差官員銅關防、如道員式。四品以下用之。

各府銅印,直紐,方二寸五分,厚六分。

各府同知、通判銅關防,長二寸八分,闊一寸九分。

各州銅印,直紐,方二寸三分,厚五分。

京縣銅印,直紐,方二寸二分,厚四分五釐。

各縣銅印,直紐,方二寸一分,厚四分四釐。

鹽課提舉司銅印,方二寸四分,厚五分。

淮南儀所監制官銅關防,長二寸八分,闊一寸九分。

布政使司經歷司、理問所銅印,方二寸二分,厚四分五釐。

鹽運使司經歷司銅印,方二寸一分,厚四分四釐。

布政使司照磨所、京府儒學、各府經歷司銅印,方二寸,厚四分二釐。

京府照磨所、司獄司、各府照磨所、司獄司、各府儒學、衛儒學、布政司庫大使、府庫大使、巡檢司、稅課司、茶馬司銅印,方一寸九分,厚四分。

各州、縣儒學銅條記,長二寸六分,闊一寸六分五釐。

縣丞、主簿、吏目、鹽課所、批驗所、各驛丞、遞運所、各局、各倉、各閘銅條記,長二寸四分,闊一寸三分。俱垂露篆。

提督、總兵官銀印,虎紐,三臺,方三寸三分,厚九分。

鎮守掛印總兵官銀印,虎紐,二臺,方三寸三分,厚九分。

鎮守總兵官銅關防,長三寸二分,闊二寸。俱清、漢文柳葉篆。

副將、參將、游擊銅關防,長三寸,闊一寸九分。

宣慰司銅印,方二寸七分,厚九分。俱清、漢文殳篆。

都司僉書銅關防,長三寸,闊一寸九分。營都司,衛、所千總銅關防,長二寸八分,闊一寸九分。

守備銅條記,長二寸六分,闊一寸六分。

衛守備銅印,方二寸六分,厚六分五釐。

宣撫司銅印,方二寸五分,厚六分。

宣撫司副使、安撫司領運千總銅印,方二寸四分,厚五分五釐。

長官司指揮、僉事銅印,方二寸二分,厚四分五釐。俱清、漢文懸針篆。

衛經歷、宣慰司經歷銅印,方二寸一分,厚四分四釐,垂露篆。

土千戶銅印,方二寸三分,厚四分五釐。

土百戶銅印,方二寸,厚四分二釐。俱清、漢文懸針篆。

管理京城喇嘛班第、管理盛京喇嘛班第銅印,方二寸二分,厚四分五釐。俱清、漢文轉宿篆。

正乙真人銅印,方二寸四分,厚五分。清、漢文垂露篆。

乾隆十四年,禮部奉諭:「理籓院印文之蒙古字,不必篆書。外籓扎薩克盟長、喇嘛、並蒙古、西藏,一應滿洲、蒙古、唐古特文,均亦不必篆書。其在京扎薩克、大喇嘛印,滿文俱篆書,蒙古文不必篆書。」又諭:「近因新定清文篆書,鑄造各衙門印信,所司檢閱庫中所藏經略大將軍、將軍諸印,凡百餘顆。皆前此因事頒給,經用繳還,未經銷毀者。會典復有『命將出師,請旨將庫中印信頒給』之文,遂至濫觴。朕思虎符鵲紐,用之軍旅,所以昭信,無取繁多。庫中所藏,其中振揚威武,建立膚功者,具載歷朝實錄,班班可考。今擇其克捷奏凱,底定迅速者,經略印一,大將軍、將軍印各七,分匣收貯。稽其事跡始末,刻諸文笥,足以傳示奕禩。即仍其清、漢舊文,而配以今制清文篆書,如數重造。遇有應用,具奏請旨頒給。一並藏之皇史宬。其餘悉交該部銷毀。此後若遇請自皇史宬而用者,蕆事仍歸之皇史宬。若因遇事特行頒給印信者,事完交部銷毀。將此載入會典。」

高宗御定印譜,欽命總理一切軍務儲糈經略大臣關防一,奉命、撫遠、寧遠、安東、征南、平西、平北大將軍印各一,鎮海、揚威、靖逆、靖東、征南、定西、定北將軍印各一。並御制印譜序曰:「國家膺圖禦宇,神聖代興,赫濯撻伐,光啟鴻業。時則有推轂命將之典,及功成奏凱,還上元戎佩印。載在冊府,藏之史宬。蓋法物留詒,不啻如曩籍所稱玉節牙璋,尚方齊斧者比。乾隆十七年,釐考國書篆字成,因詳加酌定。交泰殿所遵奉世傳御寶,仍依本文,不敢更易。其常行誥敕所鈐用,以及部院司寺以下,外而督、撫、提、鎮以下,咸改鑄篆文,以崇典章、昭法守。而大將軍、經略及諸將軍之印,或存舊,或兼篆,一依交泰殿諸寶之例,各以時代為次。茲西陲武功將竣,爰譜圖系說如左。書曰:『其克詰爾戎兵,以陟禹之跡,方行天下,至於海表,罔有不服。』信大兵可百年不用,不可一日不備。披斯譜也,必將曰:是印也,是我朝某年殄某寇、定某地所用也。又將曰:是印也,鑄自某年,某官既奉以集事,傳至某年,某官復奉以策勛者也。想見一時受成廟算,元老壯猷。豐紐重臺,焜燿耳目。繼自今覲揚光烈,思所以宏此遠謨。弼我億萬世丕丕基,將於是乎在。以視銘績鼎鐘,圖形臺閣者,不尤深切著明也歟?然則觀於寶譜,而一人守器之重可知;觀於印譜,而群才翊運之殷又可知。詩曰:『王之藎臣,無念爾祖。』記曰:『君子聽鼓鼙,則思將帥之臣。』一再披閱,其何能置大風猛士之懷哉!裝潢蕆事,並令守者什襲尊藏。為部凡四:一皇史宬,一大內,一內閣,一盛京也。」


\end{pinyinscope}