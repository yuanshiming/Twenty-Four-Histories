\article{志六十二}

\begin{pinyinscope}
禮六(吉禮六)

昭忠祠賢良祠功臣專祠宗室家廟品官士庶家祭

昭忠祠雍正二年諭曰:「周禮有司勛之官,凡有功者,書名太常,祭於大烝。祭法,『以死勤事則祀之』。於以崇德報功,風厲忠節。自太祖創業後,將帥之臣,守土之官,沒身捍國,良可嘉憫。允宜立祠京邑,世世血食。其偏裨士卒殉難者,亦附祀左右。褒崇表闡,俾遠近觀聽,勃然可生忠義之心,並為立傳垂永久。」於是建祠崇文門內,歲春、秋仲月,諏吉,遣官致祭。王公大臣位正殿,陳案七,羊一、豕一。左三案,共羊豕各一。右如之。每案素帛一、爵三、果盤五。諸臣位兩配樓暨後正室,各設案五,兩廡各設案三,皆羊豕各一,為通數。兵士附祀,案三十有六,案設豕肉一盤、爵三、果品二。太常卿承祭,配樓後室司官分獻。六年,祠成,命曰「昭忠」,頒御書額,曰「表獎忠勛」。

明年,循序定位,前殿正中祀敬謹莊親王尼堪,英誠武勛王揚古利,定南武壯王孔有德,贈忠勇王黃芳度,武襄公巴爾堪,凡五人。東次龕祀安北將軍佟國綱,一等公佟養正、達福、西哈,一等侯馬得功,一等伯巴什太,都統宜理布、巴都里,議政大臣程尼、穆和琳,大學士張泰,議政大臣羅沙,三等伯王之鼎,總督範承謨,額駙托柏,大學士龍西、色思泰,總督額倫特,尚書查弼納、圖捫,太子太保佟濟,倉場侍郎王秉仁,巡撫傅弘烈,都統博波圖,議政大臣雅賚、道禪、名蓋,參贊內大臣馬爾薩,凡二十八人。西次龕祀續順公沈瑞,輔國公巴賽,大學士莫洛,尚書布顏岱,「十六大臣」綽和諾,巡撫柯永升,都統沙裏布,巡撫馬雄鎮,總督甘文焜、佟養甲,侍郎朝哈爾,鹽運使高天爵,參領費揚古,統領圖魯錫、喀爾他拉、喀爾護吉,副都統海蘭、蘇圖,統領胡里布、哈克三,佐領葉喜,侍郎永國,統領阿爾岱,提督孫定遼,凡二十有四人。東又次龕統領劉哈,副都統盧錫、科布蘇、阿喀倪、納爾特、錫密賚、科爾坤、多頗洛、戴豪、渾錦、魏正、羅濟、阿什圖、覺羅阿克善、常祿、阿爾護、吉三、巴雅思虎朗,凡十有八人。西又次龕提督段應舉,副都統穆舒、孟魁、白,原任巡撫賈維鑰,副都統邁圖,參領葛思特,巡撫硃國治、張文衡,侍郎馬如璧,糧道葉映榴,巡道陳啟泰,通政使莫洛渾,一等子穆克覃阿、納達、代音布,巡道陳丹赤,一等子覺羅莫洛渾,數亦如之。東末龕總兵吳萬福、徐勇、費雅達、硃天貴、張存福,都督僉事洪徵,總兵阿爾泰、歐陽凱,兵備道李懋祖,總兵楊佐,統領張廷輔,游擊楊光祖,統領定壽,總兵王承業,侍衛錫喇巴,布政使遲變龍,凡十有六人。西末龕參領郭色,統領新泰,提督康泰,二等子覺羅顧納岱,總兵司九經,二等子拜蘭,總兵郝效忠、劉良臣,三等子巴郎、都爾莽鼐,副將楊虎,參將趙登舉,守備紀法,參將甘應龍,副將蔡隆,二等子拜三,一等男路什,總兵康海,凡十有七人。後室、配樓、左右次龕、又次龕、兩廡暨各次龕,祀官千五百餘人。東西房附祀兵士萬三百有奇。

八年,定制以滿尚書、都統一人承祭,後室、兩廡,太常官分獻。十一年,令子孫居京秩者隨祭。乾隆十三年,諭祀陣亡總兵任舉、侍衛丹泰,旋令征金川陣沒將士並入之。十五年,祀都統傅清、左都御史拉布敦。十八年,追封巴爾堪、巴賽並為簡親王。移巴爾堪位揚古利上,巴賽位孔有德上。初,前室左右各三龕,止序爵秩,不系時代。至是定議,自天命以來,按代序官,同代同官序年月,依賢良祠例,按時班爵為序。其兵士設位,分前、後廡,以橫板隔別之。

中葉以後入祀者,將軍班第、明瑞、溫福,都統滿福、扎拉豐阿,參贊大臣鄂容安,統領觀音保、烏三太、臺斐音阿,提督許世亨,副都統呼爾起阿、第木保、覺羅明善,總兵王玉廷、李全、德福、貴林、張朝龍,而海蘭察以病沒,端濟布以傷,亦並入之。至典史溫模死守通渭,從容就義,特予入祠。且有取義舍生,賞延於世,褒諭流外微官,獲邀血阜廕,茂典也。

嘉慶朝,祀大學士福康安、將軍德楞泰、提督花連布、總兵多爾濟札普、知縣強克捷。先是,康熙間,巡撫曹申吉已入祠,至是以阿附吳三桂按實,奪之。時各省言沒王事者,奏報猥雜,龕位不給,於是詔建各省昭忠祠。其京祠定文三品、武二品以上,及八旗官弁為限,已祀者如故。嗣是卑官預祀,視特旨行。故事,承祭官循例朝服,今改蟒袍補服,示別壇廟也。

道光初元,以國初殉難副將楊祖光等入祀,厥後賡入者,都統巴彥巴圖、烏凌阿、印登額,參贊大臣慶祥,總督裕謙,提督海凌阿、關天培、陳化成,副都統海齡、長喜,總兵萬建功、祥福、葛云飛、鄭國鴻、王錫朋、謝朝恩、江繼蕓、慶和、吳喜,副將烏大魁、馬韜、周承恩、劉大忠、陳連升、硃貴、瑪隆阿、伊克坦布等。其卑秩中,如知縣楊延亮、縣丞方振聲、守備馬步衢、把總陳玉威,亦足多者。

咸豐三年,更定血阜典,文四品、武三品官得再入京祠,並獲祀陣亡所在地。其文五品、武四品以下,凡贈職銜及當例血阜者,並祀之。是時軍興,死事揚烈者踵起,略舉其所入者。都統烏蘭泰、霍隆武,將軍佟鑒、祥厚、蘇布通阿、扎拉芬、和春,總督吳文鎔、陸建瀛,提督長瑞、長壽、董光甲、邵鶴齡、恩長、福珠洪阿、陳勝元、雙福、王錦繡、常祿、雙來、瞿騰龍、佟攀梅、鄧紹良、德安、周天培、史榮椿、張國樑、周天受、王浚、樂善、褚克昌,一等男阿爾精阿,一等子左炘,侍郎呂賢基、戴熙,巡撫常大淳、江忠源、陶思培、鄒鳴鶴、吉爾杭阿、徐有壬,學政孫銘恩、張錫庚,副都統伊勒東阿、哲克東阿、達洪阿、貴升、繃闊、博奇、常壽、西林布、多隆武、托克通阿、格繃額、伊興額、舒明安,頭等侍衛達崇阿,布政使嶽興阿、劉裕珍、塗文鈞、李續賓、李孟群、王友端,按察使李卿穀、周玉衡,贊善趙振祚,郎中宋蔚謙,總兵博春、福諴、馬濟美、玉山、程三光、劉開泰、桂林、王國才、蔣福長、虎坤元、羅玉斌、邱聯恩、田興奇、承惠、陳大富、滕家勝、郭啟元、王之敬,道員羅澤南、硃鎮、金光箸、帥遠燡、溫紹原、何桂珍、王訓、趙印川、郭沛霖、黃淳熙、繆梓,知府謝子澄、劉騰鶴、江炳琳,副將謝升恩、膺保、李成虎、彭三元、周云耀、龍汝元。同治朝,則親王僧格林沁,大學士曾國籓,都統海全、舒通額,將軍多隆阿,統領舒保,參贊大臣錫霖、武隆額,領隊大臣色普詩、惠慶、達春泰、穆克登額,辦事大臣扎克當阿,頭等侍衛隆春、奇克塔善,內閣學士金順,提督占泰、李臣典、向榮、塔齊布、林文察、蕭河清、周顯承、羅朝雲、蕭德揚、楊得勝、曹仁美、毛福益、張仁泗、劉松山、譚玉龍、羅雨春、張紹武、胡良作、姚連升、饒得勝、劉長槐、榮維善、楊春祥、張萬美、魯光明、閻定邦、劉祥發、曹德喜,巡撫王有齡、羅遵殿、鄧爾恆,副都統錫齡阿、蘇倫保、恆齡,按察使黃運昌,總兵郝上庠、雷升、熊建益、林向榮、餘際昌、郎桂芳、江福山、何建鼇、羅應貴、毛芳恆、張樹珊、唐殿魁、周兆麒、李大櫆、陳清彥、鄧鴻超、江登雲、傅先宗,道員福咸、俞焜、趙景賢、張同登、趙國澍、瑞春、周縉、秦聚奎、彭毓橘、葛承霖、鄧子垣,知府硃鈞、姜錫恩、竇天灝、於醇儒,副將劉神山、黃金友、周學貴、羅春鵬、王夢齡、張起鳳、劉勝龍。光緒間,則大學士左宗棠,總督恆春、曾國荃,將軍明緒,領隊大臣崇熙、烏勒德春、托克托布、博勒果素、托克托奈、喀爾莽阿,參贊大臣額騰額、覺羅奎棟,辦事大臣奎英、薩凌阿,提督硃南英、李秀山、湛其英、楊世俊、王子龍、文德盛、陳忠德、滕學義、何明海、魏金闕、文德昌、李登第、王慶福、楊萬義、楊必耀、李大洪、鍾興發、張宗久、楊玉科、劉思河、李其森、梁善明,鹽運使陶士霖,總兵石紹文、陳登雲、鄧仁和、黃應斗、周友山、硃希廣、王茂連、王春和、譚聲俊、達年、剛安泰、向集梧、鄧承恩、韋和禮、劉節高、陳嘉、左寶貴、周康祿、黃鼎、葉維籓、侯雲登,戶部主事玉潤,知府龔秉琳、侯學云、馬椿齡、張瀚中,副將王世晉、李天和、章茂、張定邦、尤正廷、楊隆輝、張玉秋、王碧庭、徐安邦、李啟榮、裕廉、王宗高。二十六年,尚書崇綺,將軍延茂,總督李秉衡,並入祀。尋罷秉衡。凡祠祭諸臣,大都效命戎行,守陴徇義,或積勞沒身。褒忠節,勸來者,會典綦詳。茲錄什一,以見例焉。

雍正初,各省立忠義祠,凡已旌表者,設位祠中,春、秋展祀。乾隆四十一年,定明代殉國諸臣,既邀謚典,並許入祠。又諸生、韋布、山樵、市隱者流,遂志成仁,亦如前例。嘉慶七年,始令各省府城建昭忠祠,或附祀關帝及城隍廟,凡陣亡文武官暨兵士、鄉勇,按籍入祀。八旗二品以上官已祀京祠者,仍許陣亡所在地祠祀,合五十人一龕,位祀正中,兵勇則百人或數十人一位,分列兩旁,駐防位綠營上。春、秋二奠,有司親蒞,用少牢,果品、上香、薦帛、三獻如儀。同治二年,允曾國籓請,江寧建昭忠祠,祀湖南水陸師陣亡員弁。已復抗節官紳亦許崇祀,並建專祠。婦女殉難者,亦別立貞烈祠云。

賢良祠雍正八年詔曰:「古者大烝之祭,凡法施於民,以勞定國者,皆列祀典,受明禋。我朝開國以後,名臣碩輔,先後相望。或勛垂節鉞,或節厲冰霜,既樹羽儀,宜隆俎豆。俾世世為臣者,觀感奮發,知所慕效。庶明良喜起,副予厚期。京師宜擇地建祠,命曰『賢良』,春、秋展祀,永光盛典。」乃營廟宇在地安門外西偏,正殿、後室各五楹,東、西廡,歲春、秋仲月,諏吉,遣官致祭。前殿案各素帛一、羊一、豕一、果五盤。後室果品同,唯牲、帛共案而具一。承祭官蟒服,二跪六叩三獻。餘如常儀。

於是僉議怡賢親王允祥,宗功元祀,宜居首。大學士、公圖海,公賴塔,大學士張英,尚書顧八代、馬爾漢、趙申喬,河道總督靳輔、齊蘇勒,總督楊宗仁,巡撫陳瑸,咸列其選。自是先後賡續入祠者,大學士範文程、巴克什達海、阿蘭泰、李之芳、吳琠、張玉書、李光地、富寧安、張鵬翮、寧完我、魏裔介、額色黑、王熙,領侍衛內大臣福善、費揚古、尹德,尚書勵杜訥、徐潮、姚文然、魏象樞、湯斌,提督張勇、王進寶、孫思克、施瑯,總督趙良棟、於成龍、傅臘塔、孟喬芳、李國英,都統馮國相、李國翰、根特,統領莽依圖,將軍阿爾納、愛星阿、佛尼埒,副都統褚庫巴圖魯。明年祠成,頒御書額曰「崇忠念舊」,設位為祭。前殿內大臣或散秩大臣、尚書、都統主之。後殿用太常寺長官。入祠日,子孫咸與行禮,春、秋遣官陪祀同。

十二年,祀大學士田從典、高其位。乾隆元年,命入祀諸臣未予謚者悉追予。是歲祀尚書銜兼祭酒楊名時,大學士硃軾,內大臣哈世屯,尚書米思翰。五年,祀總督李衛。明年,祀尚書徐元夢,巡撫徐士林。十年,釐定祠位,前殿正中祀怡賢親王,後室諸臣合一龕。首世次最先者,餘分左右行,按世序爵,大學士居前,次領侍衛內大臣、尚書、都統、將軍、總督、前鋒護軍統領、提督、侍郎、巡撫、副都統,以次分列。至世爵有子、男授尚書、都統者,有侯、伯為侍郎、副都統者,仍視官秩為差。

嗣是入祀,則超勇親王策凌,列怡賢親王左次龕。名臣則大學士馬齊、伊桑阿、福敏、黃廷桂、蔣溥、史貽直、梁詩正、來保、傅恆、尹繼善、陳宏謀、劉綸、劉統勛、舒赫德、高晉、英廉、徐本、高斌,協辦大學士兆惠,左都御史拉布敦,尚書汪由敦、李元亮、阿里袞,尚書銜錢陳群,都統傅清,將軍和起、伊勒圖、奎林,總督那蘇圖、陳大受、喀爾吉善、鶴年、吳達善、何煟、袁守侗、方觀承、薩載、提督許世亨,巡撫潘思矩、鄂弼、李湖、傅弘烈。弘烈自雍正時,拉布敦、傅清自乾隆時,並入昭忠祠,今再祀賢良者也。

嘉慶朝,則祀大學士福康安、阿桂、劉墉、王傑、硃珪、戴衢亨、董誥,尚書董邦達、彭元瑞、奉寬,總督鄂輝。道光朝,則祀大學士富俊、曹振鏞、托津、長齡、盧廕溥、文孚、王鼎,協辦大學士汪廷珍、陳官俊,尚書黃鉞、隆文,將軍玉麟,總督楊遇春、陶澍,河道總督黎世序。咸豐朝,則祀大學士潘世恩、文慶、裕誠,協辦大學士杜受田,侍郎杜堮,巡撫胡林翼。同治朝,則祀大學士桂良、祁俊藻、官文、倭仁、曾國籓、瑞常、賈楨,大學士銜翁心存,協辦大學士駱秉章,總督沈兆霖、馬新貽。其光緒朝入祀者,恭忠親王奕。名臣大學士文祥、英桂、全慶、載齡、左宗棠、靈桂、寶鋆、恩承、福錕、張之萬、麟書、額勒和布、李鴻章、榮祿、裕德、昆岡、崇禮、敬信,協辦大學士沈桂芬、李鴻藻,將軍長順,總督沈葆楨、丁寶楨、岑毓英、曾國荃、劉坤一,提督宋慶,巡撫張曜也。宣統初入祀者,止大學士王文韶、張之洞、孫家鼐、鹿傳霖,協辦大學士戴鴻慈五人而已。

各省賢良祠,雍正十年,詔:「各省會地建祠宇,凡外任文武大臣,忠勇威愛,公論允翕者,俾膺祀典,用勸在官。如將軍蔡良,提督張起雲,總兵蘇大有、魏翥國,足稱斯選。」定制,春、秋祭日視京師,以知府承祭,品物儀節亦如之。

功臣專祠順治十一年,詔為孔有德建祠,度地彰義門外三里,曰定南武壯王祠,二妃祔焉。康熙三年,定春、秋展祀,其後建恪僖公祠安定門外,祀一等公遏必隆並縣?舒舒覺羅氏。嗣領侍衛內大臣尹德,尚書阿里袞暨其夫人,乾隆時並祔祀云。

其建自雍正朝者,朝陽門外勤襄公祠,祀定南將軍佟圖賴及其夫人,長子忠勇國綱、次子端純國維,皆以軍功祔祀。德勝門外文襄公祠,祀大學士圖海。安定門外與恪僖祠並峙者,為弘毅公祠,祀光祿大夫額亦都,並以夫人配。

建自乾隆朝者,東安門外恪僖公祠,祀內大臣哈世屯及其夫人,子承恩公米思翰、孫李榮保,其後曾孫大學士傅恆祔祀焉。崇文門內雙忠祠,祀左都御史拉布敦、都統傅清。合昭忠、賢良而復建專祠者,他無與比也。地安門外旌勇祠,祀將軍明瑞,而都統扎拉豐阿,統領觀音保,總兵李全、王玉廷、德福亦先後入祔。睿忠親王祠在朝陽門外,祀多爾袞並福晉六人。嘉慶時,建大學士福康安祠曰「獎忠」,在東安門外,都統額勒登保祠曰「褒忠」,在地安門外。光緒時,建科爾沁親王僧格林沁祠曰「顯忠」,在安定門內。大學士、伯李鴻章祠曰「表忠」,在崇文門內。宣統時,合祀立山、聯元祠在宣武門外。

凡京師專祠,歲春、秋仲月吉日,遣太常卿分往致祭。用少牢一、果品五。唯佟圖賴、哈世屯兩祠,則少牢三,果品十有五。旌勇祠少牢如通常,果品亦十五雲。位各用帛一、爵三,諸祠並同。嘉慶七年,始定承祭官行禮用蟒袍補服。

其在各省者,歲春、秋守土官致祭。茲紀其勛勞最著者。自湖廣建忠節祠以祀左都督徐勇,各省建專祠始此。康熙間,廣西建雙忠祠,祀馬雄鎮、傅弘烈,於是福建祀範承謨、陳啟泰、吳萬福、高天爵,雲南祀甘文焜。

雍正間,清河祀靳輔、齊蘇勒,開封祀田文鏡。盛京祀怡賢親王。乾隆中,詔通達、武功、慧哲、宣獻四郡王,禮烈、饒餘、鄭獻、穎毅四親王並入之,改名賢王祠。已,睿忠、豫宣二親王,克勤郡王,亦均同祀。嵇曾筠、高斌,合祀清河靳輔等祠。伊犁祀班第、鄂容安,而拉布敦、傅清且建祠及西藏矣。

嘉慶時,武威建雙烈祠,祀韓自昌、韓加業,同安祀李長庚,成都祀德楞泰,韓城、滑縣祀強克捷。

道光間,江南祀黎世序,臺灣祀方振聲、馬步衢、陳玉威,趙城祀楊延亮,虎門祀關天培暨陳連升父子,鎮海祀裕謙,定海祀葛云飛、鄭國鴻、王錫朋,京口祀海齡,寶山祀陳化成。

咸豐間,廣西祀長瑞、長壽暨阿爾精阿,西安、蘇州祀林則徐,安慶祀蔣文慶,廬州祀江忠源,瑞州祀劉騰鴻,江寧、蘇州祀向榮、張國樑,京口祀吉爾杭阿,附祀繃闊、劉存厚,揚州祀雙來、瞿騰龍,溧水、滸墅祀李坤元,天津祀佟鑒、謝子澄,長沙、九江祀塔齊布,湖廣、江西、安徽祀李續賓,江西、湖廣祀羅澤南,又與饒廷選合祀廣信,湘鄉復分祀澤南、王珍、劉騰鴻。湖南、江西祀蕭啟江,湖廣祀胡林翼,後安慶亦祀之。遵義祀羅繞典。

同治間,湖北合祀官文、胡林翼,廬州祀李孟群,浙江祀瑞昌、王有齡、張玉良等,杭州祀羅遵殿,富陽祀熊建益,湖州祀趙景賢,陳州、安慶、臨淮、淮安祀袁甲三,南昌、青陽祀江忠義,安徽、湖廣祀李續宜,後復與多隆阿合祀潛山。安慶、蘇州、嘉興祀程學啟,河南、安徽、陜西、吉林祀多隆阿,後與林翼合祀安慶。江寧、安慶、吉安祀李臣典,湖南、福建、廣東祀張運蘭,曹州、天津、蒙城祀僧格林沁,後復祀奉天。湖南、江蘇、安徽祀彭毓橘,湖廣祀曹仁美等,四川、湖南祀駱秉章,陜、甘祀劉松山,江寧、安慶祀馬新貽,江寧、湖南、湖北、安徽、直隸祀曾國籓,後復與國荃合祀開封。長沙合祀張亮基、潘鐸,巴燕岱祀穆克登額,哈密祀扎薩克親王錫伯爾,南豐祀吳嘉賓,貴州祀蔣霨遠、黃潤昌等。於是禮部言:「各省專祠宜擇隙區曠土,毋侵民居,並禁改毀志乘名跡、聖賢祠墓。」報可。

光緒間,揚州、黃州祀吳文鎔,安徽、江西、閩、浙、甘肅祀劉典,江南、江西、福建、臺灣祀沈葆楨,江蘇、建福、山東、湖南祀郭松林,江、浙、直隸、山東、河南祀吳長慶,後復祀朝鮮。閩、浙、陜、甘、新疆、江寧祀左宗棠,四川、湖南、江西、安徽、江蘇祀鮑超,陜、甘、吉林祀金順,大理、鎮南祀楊玉科,江西、廣西、雲南、新寧祀劉長佑,雲、貴、廣西祀岑毓英,安徽、山東祀周盛波,後復與盛傳、戴宗騫合祀濟南。湖廣、江西、江寧、浙江西湖祀彭玉麟,荃,河南、安徽、湖北、直隸、甘、新祀張曜,安慶、江寧、青縣祀周盛傳,山東、江蘇祀陳國瑞,山東、陜西祀閻敬銘,湖南、甘、新祀劉錦棠,安徽、福建祀劉銘傳,山東、四川祀丁寶楨山東、陜西祀州、長沙、蘭州祀楊昌濬,江、浙、河南、直隸、山東祀李鴻章,直隸、奉天、河南、安徽祀宋慶,安徽及蘆臺祀聶士成,湖南、江西、安徽、江寧祀劉坤一,廣西、雲、貴祀馮子材,安徽、湖南祀曾國華,甘、新祀陶模,直隸、安徽祀馬玉昆,安徽祀英翰,湖南、宣城祀鄧紹良,江南祀蕭孚泗,江寧祀陶澍、林則徐、鄒鳴鶴、福珠洪阿,清、淮、徐州祀吳棠,姚廣武等附之。徐州祀滕學義、唐定奎,淮安祀張之萬,杭州祀阮元、蔣益澧,淮、揚祀章合才,南昌祀吳坤修,東鄉祀羅思舉,河南祀倭仁,溫縣祀李棠階,西安祀劉蓉、曾望顏,天津祀怡賢親王、文謙、丁壽昌,靈壽、保定祀成肇麟,順天薊州祀吳可讀,寶坻祀潘祖廕,新疆祀金運昌,奉天建三賢祠,祀文祥、崇實、都興阿,又祀左寶貴、依克唐阿、長順。吉林祀金福、延茂、富俊、希元,福建臺灣祀王凱泰,四川西充祀武肅親王豪格,臨桂祀陳宏謀,貴陽祀曾璧光、韓起、黎培敬。於時各省紛請立專祠,諭毋濫。

宣統享國未久,而湖北、安徽、陜、甘、奉天祀雷正綰,直隸、山東、河南、安徽祀程文炳,安徽及蒙古旗祀潘萬才,合肥祀董履高,渦陽祀牛師韓,杭州西湖祀徐用儀、許景澄、袁昶,號為「三忠」云。昶又祀蕪湖。自是聯元祀寶坻,張之洞祀武昌,王文韶祀長沙,馬維麒祀成都,丁體昌祀秦州,夏毓秀祀昆明,此皆舉其大者。其餘疆吏題請,禮臣議覆,事載實錄,年月可稽者,尚不一而足也。

有清一代,從龍諸佐,蔚起關外。平三籓,漢將西北為多。靖三省教匪,蜀將競興。東南海寇橫,閩帥踵起。湘楚武臣,戡平粵亂。剿捻一役,參以皖將。其間完節死綏,祠祀尤夥。其功臣總祠,世宗朝,建忠勇祠蘭州。仁宗朝,建彰忠祠喀什噶爾。同治中興,湖南有表忠祠,湘鄉、平江有忠義祠,洞庭君山、湘鄉、桂陽有昭忠祠。他如湖口石鐘山水師,金陵湘軍陸師,楚軍水師,吳淞外海水師,臺灣淮楚軍,蘇州、武昌、保定、廬州、巢湖、濟南、無錫各地淮軍,使凡轉戰糜軀者,莫不馨香血食,其為昭忠一也。此外江寧、京口旗營,金陵軍營官紳,武昌武毅軍,成都嵩武軍,錦州毅軍,各昭忠祠,與各州縣忠義、昭忠、慰忠、忠烈等祠,所以血阜死酬勛,不可勝紀。祭禮、祭品如前儀。

宗室家廟崇德元年,定宗室封王者立家廟。順治五年,詔王無嗣,祔饗太廟後殿西廡。有子孫者,立廟別祭。四孟月、歲暮陪祭太廟,畢,歸府第行之。凡薦新,未獻太廟者,不得私獻家廟。於時莊親王立一廟,禮、巽、謙三親王合一廟,饒餘郡王、端重親王合一廟,穎親王、順承郡王合一廟,豫郡王一廟,克勤、衍禧二郡王合一廟。雍正九年,怡賢親王立一廟。

凡親王世子、郡王家祭,建廟七楹,中五為堂,左右墻隔之為夾室。堂後楣北五室,中奉始封王,世世不祧。高、曾、祖、禰依序為二昭二穆,昭東穆西,親盡則祧。由昭祧者,藏主東夾室,升二昭位於一室,以二室奉升祔主。由穆祧者,藏西夾室,升祔亦如之。南為中門,又南廟門,左右側門,庭分東、西廡,東藏衣冠,西則祭器、樂器。廟重簷,丹楹,採桷,綠瓦,紅堊壁。門內焚帛爐。外刲牲房,西鄉。歲以四時仲月諏吉,仲春出祧主合食。

其禮,堂中始封祖專案,正位,南鄉。左東夾祧主共案,次二昭共案,東鄉。右西夾室祧主共案,次二穆共案,西鄉。少西設香帛案一,尊案一,每案羊、豕各一,鉶、簠、簋各二,籩、豆各八。位各帛一、爵三、樂器六。同祖所出子孫,成人以上,屆期會祭,府僚與陪,執事通贊、屬官為之。奉香、帛、爵則用子孫。先三日,主人齋外寢,眾咸齋。祀日昧爽,主人朝服入,位堂簷內正中,與祭伯叔輩位東階上,兄弟子孫位東階下,位以世差,世以齒序。官屬位西階下,序以爵。俱北面。質明,子弟長者二人詣世祖室,四人分詣東西夾室,昭、穆室,各奉主安幾。昭,考右妣左;穆,考左妣右。跪,一叩,興。主人盥,就位,迎神樂作。詣始祖位前三上香,以次詣各祧位前上香,率族屬行二跪六拜禮。奉帛、爵奠、獻、讀祝如儀。三獻訖,詣始祖位前跪受爵、受胙,三拜,徹饌,送神,二跪六拜。詣燎位視燎。禮成,奉主還室,退。分胙頒族屬。

其時祭之禮,堂中設案五,始祖考、妣正位南鄉,高、曾、祖、禰,依昭穆為左右。案各羊一、豕一,餘如合食制。其時節薦新,屆日主人夙興,率子弟盛服入廟,潔堂宇,設案,陳果羞盤各六,每位箸二、■D9三。啟室,以次詣各案前跪上香,三拜,子弟遍獻酒,主人二跪六拜,子弟隨行禮。畢,闔室,退。因事致告,薦果羞各四,禮同薦新。月朔望謁廟亦如之。

貝勒、貝子、宗室公家祭廟五楹,三為堂。後楣北分室五,奉始封祖暨四代。兩旁夾室奉親盡祧主。廟不重簷,門不備採,餘如親王。合食,始祖專案,羊一、豕一,東夾室祧主暨二昭專案,羊豕各一。西夾室祧主暨二穆亦如之。時祭俱專案,昭穆各同牲,籩、豆視親王各減二,不用樂,一跪三拜。時節薦果盤各四,有事則告,朔望則謁。餘如親王儀。

品官士庶家祭凡品官家祭廟立居室東,一至三品廟五楹,三為堂,左右各一墻限之。北為夾室,南為房。庭兩廡,東藏衣物,西藏祭器。庭繚以垣。四至七品廟三楹,中為堂,左右夾室及房,有廡。八、九品廟三楹,中廣,左右狹,庭無廡。篋藏衣物、祭器,陳東西序。堂後四室,奉高、曾、祖,禰,左昭、右穆。妣以嫡配,南鄉。高祖以上,親盡則祧。由昭祧者,藏主東夾室;由穆祧者,藏主西夾室。遷室、祔廟,並依昭穆世次,東西序為祔位,伯叔祖父兄弟子姓成人無後者、殤者,以版按行輩墨書,男東女西,東西鄉。定牲器之數,一至三品,羊一、豕一,每案俎二,鉶、登各二,籩、豆各六。四至七品,特豕,案一俎,籩、豆各四。八品以下,豚肩不特殺,案一俎,籩、豆各二。

歲祭以四時仲月諏吉,讀祝、贊禮、執爵皆子弟為之。子孫年及冠,皆會祭。前三日,主人暨在事者齋。祀日五鼓,主人朝服,眾盛服,入廟。主人俟東階下,族姓俟庭東西,順昭穆世次。主婦率諸婦盛服入,詣爨所視烹飪。羹定,入東房治籩、豆,陳鉶、登、匕、箸、醯、醬以俟。質明,子弟長者啟室,奉主陳之幾,昭位考右妣左,分薦者設東西祔位。主人升自東階,盥訖,詣中簷拜位立。族姓行尊者立兩階上,卑者立階下。咸北面。主人詣香案前跪,三上香,進奠爵,興,復位,率族姓一跪三拜。主人詣高祖案前獻爵,曾、祖、禰案前畢獻如儀,分薦者遍獻祔位酒,讀祝。每獻,主婦率諸婦致薦,一叩興。初獻匕箸醯醬,亞獻羹飯肉胾,三獻餅餌果蔬。卒獻,主人跪香案前,祝代祖考致嘏於主人,主人啐酒嘗食,反器於祝,一叩興,復位,送神,一跪三拜。視燎畢,與祭者出,主人率子弟納神主,上香行禮。徹祭器,闔門,退。日中而餕。

三品以上,時祭遍舉。四至七品,春、秋二舉。八九品春一舉。與祭者,尊卑咸在。主人肅入席,酌尊者酒,子弟年長者離席酌主人,長幼獻酬交錯。已事,咸出。徹席,餕庖人、僕人必盡之。

令節薦新,一至三品,每案果、羞各四,四至七品,減果二,八、九品並減羞二,具羹飯則同。月朔望供茶,食案二器,儀同時薦。庶士家祭,設龕寢堂北,以版隔為四室,奉高、曾、祖、禰,妣配之,位如品官儀,南鄉。服親成人無後者,順行輩書紙為祔位,已事,焚之,不立版。每四時節日,出主以薦,粢盛二盤,肉食果蔬四器,羹二,飯二。先期致齋。薦之前夕,主婦在房治饌,逮明,主人吉服,率子弟奉主陳香案,昭東穆西,設祔位西序案,主人立東階下,眾按行東西立。主人上香畢,一跪三拜,興。主婦率諸婦出房薦匕箸醯醬,跪,叩,退。主人至案前,以次酌酒、薦熟,跪,叩,興。子弟薦祔位,畢,讀祭文。再獻,主婦薦飯羹,三獻薦餅餌時蔬。主人率族姓行禮訖,焚祭文及祭位,納主,徹退,日中而餕。春一舉,月朔望獻茶,有事則告,俱一跪三拜。

庶人家祭,設龕正寢北,奉高、曾、祖、禰位,逢節薦新,案不逾四器,羹飯具。其日夙興,主婦治饌,主人率子弟安主獻祭,一切禮如庶士而稍約。月朔望供茶,燃香、鐙行禮。告事亦如之。


\end{pinyinscope}