\article{列傳三十一}

\begin{pinyinscope}
趙開心楊義林起龍硃克簡成性王命岳

李森先李呈祥魏琯李裀季開生弟振宜張煊

趙開心,字靈伯,湖南長沙人。明崇禎進士,官至兵部員外郎。順治元年,授陜西道監察御史。是歲有自稱故明皇太子者,令故明貴妃袁氏及故東宮官屬內監等視之,皆言不相識。開心及給事中硃徽疏請詳審,下法司,自承為京師民楊玉。以開心疏言「太子若存,明朝之幸」,論死,上命免之。二年,疏言:「刑部治庶獄,數日即結正。惟自別衙門發送者,恆不時讞決,久置獄中。請令所司五日一稽核,當鞫當釋,勿使留滯。並請通飭諸行省撫按遵行。」從之。

尋命巡視南城。滿洲兵初入關,畏痘,有染輒死。京師民有痘者,令移居出城,杜傳染。有司行之急,嬰穉輒棄擲。開心疏請四郊各定一村,移居者與屋宇聚處。旋又疏言:「立政之始,一事之得失,關天下萬世之利害。疏奏不能盡陳,封章不敢頻瀆。乞時賜召對,霽顏聽受。庶用人施政,悉奉宸斷。」睿親王攝政,入朝,朝臣皆跪迎,開心疏請敕禮部詳定儀注。江、浙、湖廣諸行省初定,開心疏請急置撫按,以時綏撫。並得旨俞允。擢左僉都御史。三年,坐事,罷。

八年,召起原官。旋超擢左都御史。開心子而抃,為唐王時舉人。九年,開心疏乞許而抃會試,禮部議不許,開心坐奪職,永不敘用。十年,諭曰:「開心有直名,畀風憲重任。不言國家大事,乃庇子瀆奏,辜朕望實深。朕念開心大臣,一事差謬,遂永棄不用,心終未恝然。召還京。」開心至,疏論湖廣巡撫遲日益、偏沅巡撫金廷獻、鄖襄巡撫趙兆麟所屬寇盜充斥,剿撫無能。得旨,下部察議。又言:「江南諸行省,每因捕治叛逆,株連無辜。如常鎮紳士王期升、路邁、蔣拱辰等,久錮獄中,虛實未辨。就一方一事,可推之他省。」上命確察以聞。時方考察京官,甄別翰林,開心疏論大學士馮銓、陳名夏等,各植門戶,開朋黨之漸,上命開心據實覆奏,未能實指其人,得旨申飭。旋授原官。

十一年,疏陳時政,請御經筵,親奏對,遴賢才,原過誤,許流徙自贖,重法司職掌。上以疏中有「屏斥畋游」語,諭曰:「講武習兵,乃祖宗立國大法,何謂畋游?開心常談淺見,沽名塞責,殊負委任。」尋以名夏獲罪,責言官不先事舉發,降補太僕寺卿。

十二年,遷戶部侍郎。疏言:「畿甸流民載道,有司恐誤留逃人,聽其轉徙。請暫寬隱匿逃人之罪,以免株連,俾流民得邀撫輯。」諭曰:「逃人之多,因有隱匿者,故立法不得不嚴,何謂株連?」因責開心沽譽,降補太僕寺寺丞。尋擢少卿,協理兵部督捕事。十三年,上以逃人多不獲,所司督責不嚴,復降補鴻臚寺少卿。十六年,遷太僕寺少卿。康熙元年,擢總督倉場戶部侍郎,加工部尚書銜。卒官。

楊義,山西洪洞人。明崇禎進士,官山東聊城知縣。順治元年,授河南汝陽知縣。五年,行取,擢江西道御史,巡視兩浙鹽政。義疏請定行鹽掣驗之法,遴選清廉有司照引盤驗,御史親臨監掣。八年,睿親王得罪,義劾工部侍郎李迎晙前官營繕郎中,監造王府,僭擬禁廷,不數年閒,躐升華膴,請敕部治罪。以迎晙事在赦前,寢其議。復巡視長蘆鹽政,劾運使趙秉樞貪酷骫法,削籍逮治。

九年,督學江南,尋掌京畿道事。十一年,大學士陳名夏得罪,義因劾請告侍郎孫承澤黨附名夏,下部,令承澤休致。吏部尚書劉正宗薦降調員外郎董國祥,擬授文選司郎中,義面詰正宗專擅,即具疏劾之,正宗得旨察議,國祥卒以贓敗,謫徙尚陽堡。

十二年,條陳時政,言:「大學士呂宮久疾曠職,宜令歸田,養大臣廉恥。」「巡按既停閱城、審錄諸事,督撫按期巡行,宜令簡隨從,慎關防,毋以擾民。」「兵民匱乏,請令各州縣稟生捐銀準貢,以給滿洲兵備鞍馬器用,餘賑被災貧民。」「諭旨嚴禁加派,有司抗不遵行。如臣籍洪洞,地畝正糧外,又加驛站坐司馬夫、工食、公費等項,幾半正糧。祈敕禁革。」會宮已得旨致仕,飭下所司議行。時議復設巡按,義奏請甄舉才守兼優考試,請簡不拘資俸。是歲四遷至刑部侍郎。十四年,調工部。十七年,調倉場侍郎,擢工部尚書。康熙元年,致仕。卒。

林起龍,順天大興人。順治三年進士,授吏科給事中。疏請嚴禁白蓮、大成、混元、無為等邪教。又疏請重守令,課以十五事,曰:招流亡,墾荒萊,巡阡陌,勸樹藝,稽戶口,均賦稅,輕徭役,除盜賊,抑豪強,懲衙蠹,賑災患,濟孤寡,濬溝池,治橋梁,興學校。考其殿最,而大吏以時訪察。俱如所奏行。四年,劾山東巡撫丁文盛不能弭盜,並薦大理寺卿王永吉可代,部議以起龍有私,降二級外用。又坐劾登州道楊雲鶴婪贓不實,奪官。

世祖親政,召來京。十年,復原官。時軍旅未靖,急轉餉,不遑言積貯,起龍請敕計臣籌畫,先實京倉,次及近輔各直省,務使倉有儲穀,備水旱,應調發。又言:「滿洲兵昔在盛京,無餉而富;今在京師,有餉而貧。時地既迥異,法制宜更定。凡駐守征行,所需馬匹、草束、軍裝、戎器,悉動官帑籌備,毋使拮據。」疏入,諭曰:「滿洲兵建功最多,資生無策,十年來未有言及此者。起龍實心為國,忠誠可嘉!」下部議,以五品京堂用,起龍疏辭。

十一年,轉刑科,加大理寺寺丞銜。疏言:「州縣吏媚事上官,耗費不貲,請禁革;並請遣廉能大臣巡行各直省,體察利弊。」既,疏劾總河楊方興及工部尚書劉昌,召方興、昌相質,所劾皆不實,部議當杖流,上特宥之,左授光祿寺署正。十二年,遷大理寺寺丞。十三年,一歲中三遷,擢工部侍郎。十五年,改戶部侍郎,總督倉場。

十六年,加太子少保。疏請更定綠旗兵制,略言:「有制之師,兵雖少,一以當十,餉愈省、兵愈強而國富;無制之師,兵雖多,萬不敵千,餉愈費、兵愈弱而國貧。今綠營兵幾六十萬,而地方有事,即請滿洲大兵,雖多仍不足用。推原其故,總緣將官赴任,召募家丁,隨營開糧,軍牢、伴當、吹手、轎夫,皆充兵數。甚有地方鋪戶子侄,充兵免徭。其月餉則歸之本管,馬兵剋扣草料,驛遞缺馬,亦供營兵應付。是以馬皆骨立,鞭策不前。又如弓箭、刀槍、盔甲、火器,俱鈍弊朽壞,帳房、窩鋪、雨衣、弓箭罩,則竟闕不具。春秋兩操,不復舉行。將不知分合奇正之勢,兵不知坐作進退之法。徒空國帑,竭民膏,雖★何益?推其病有二:一則營兵原以戡亂,今乃責之捕盜;一則出餉養兵,原以備戰守之用,今則加以剋扣。兵丁所得,僅能存活,又不按月支發,貧乏何以自支?今誠抽練綠旗精兵二十萬,養以四十萬之餉,餉厚兵精,地方有警,戰守有人。不過十年,可使庫藏充溢。」下所司議行。十七年,加太子太保、兵部尚書,巡撫鳳陽。時議懲官吏犯贓,視輕重科罪,不許納贖,起龍疏請如舊例收贖充餉,下廷議,請從之。上曰:「立法止貪,今因濟餉而貸法,如民生何?」絀起龍議不行。

聖祖即位,授起龍漕運總督,迭疏請免濱海移民田地賦額,濬淮城迤南運河,直達射陽湖,修築濟寧、臨清諸處堤閘,並請禁運丁毋病民,運弁毋病丁,條議以上,皆從其請。又疏請禁運丁多攜貨物,稽滯漕運,定分地稽察例。康熙六年,糧艘至濟寧,運丁有多攜貨物者。事覺,總河盧崇峻疏陳起龍言江南漕儲道既裁,總漕不任稽察,御史張志尹糾起龍不引罪。上以詰起龍,起龍謝失職,鐫三級休致。卒。

嘉慶四年,仁宗親政,閱世祖實錄,得起龍更定綠營兵制疏,諭諸行省督撫整飭營伍,並以所言抽練精兵,是否可仿行,飭妥議具奏。諸行省督撫憚改作,議格不行。

硃克簡,字敬可,江南寶應人。順治四年進士,授內閣中書。五年,考授御史。八年,典廣東鄉試。十二年,巡按福建。福建八府一州,其五濱海。鄭成功時入寇,民苦焚掠。克簡至,申明軍政,綢繆防禦,請增兵防仙霞關。時兵部尚書王永吉疏請減兵額,汰營兵老弱,下諸行省。克簡疏言:「福建內防山賊,外御海寇,省兵三萬四千,不可復減。」上如其議。又疏論防海,略言:「用水師不難得其力,難得其心。漳泉為鄭成功故土,沿海多戚屬,宜以連保法察其蹤跡,考其身家,不使入伍;降者令歸耕,或移置他軍,使離舊巢,乃堅歸志。水師戰海中,破浪擒賊,當受上賞,宜著為令。水師用在舟,木、竹、釘鐵、油、麻、★葉,皆海之所無,一物不具,不可以為舟。宜設專官譏察,毋以資敵。」「寧化、崇安濱海要地,今俱為賊踞,當按形勢增兵固守。」又立六規二十四約,與提督馬成功、總兵王之綱等深相結納,諸將咸奉令。

巡汀州,聞成功兵攻福州,即率汀州鎮兵還援。成功兵引退,克簡入城,曰:「寇知我援寡,且復來。」令完城垣、簡卒伍為備。數日,成功兵復至。初,官軍得成功兵輒誅之,克簡令發不過五寸者貸死,編為民,得萬餘人,皆恩克簡,至是助守城,發★擊寇,寇潰,遂出戰,解圍去。至漳州,布政使詳請徵逋賦,克簡力阻之,疏請蠲徵,上從之。至福清,以閩安地當沖,設兵守之,連江、羅源、福清、長樂諸縣要隘皆置汛。至興化,見道有流民,與知府張彥珩議賑,活者萬數千人。至泉州,令崇武、獺戶、大盈諸隘皆置汛。至延平,知其地舟人多通寇,令循江諸州縣設「循環簿」譏察。汀州、延平、建安三郡多伏戎,克簡遣兵破其巢穴,離其黨羽,次第皆就撫。迭疏請汰冗員,蠲鹽課,恤驛困,皆報可。秩滿,乞歸。康熙三十二年,卒。

子約,以副貢生充教習,歷知福安、南豐、費諸縣,擢晉州,所至皆有惠政。

成性,字我存,江南和州人。順治六年進士,授中書科中書。十四年,考授御史,巡按福建。疏言:「福建山海征剿,師旅繁興,民窮地荒。條上四策:一曰嚴汛守。濱海地寥廓,不能遍防。臣愚以為宜設水師,求熟練舟楫、諳識水性之將吏,廣選舵工水手,繚椗招斗,惟其能者,稟餉不為常格。以舟為家,銃械用其長技,操演習熟,庶幾水師可成。泉州近賊巢,水師宜移石湖。崇武、石芝駐陸軍為聲援。惠安北有峰尾司,宜駐兵,為惠州籓籬。同安鄰廈門,當於高浦設屯,劉五店置警砲,時出游騎巡視要隘。此又惠州之脣齒也。一曰分界址。有司禁遏接濟,商阻物貴,民生窮蹙。臣愚以為先定禁例,若竹木、鑌鐵、硝磺、油、麻,毋許通貿。小民日用所需,宜聽轉運。惟濱海大道或捷徑可通者,嚴立疆界。更定勾稽文法,以時比驗。自泉州西出延平上游,去海甚遠,百貨交易,宜聽民便。一曰輯降眾。山海嘯聚之徒,漸次來降。入伍者多,歸耕者少。間有悍氣未馴,凌轢鄉里。居民亦負氣不相下,訐訟不受理,則自相格鬥。臣愚以為宜令解散宿怨,禁止羅織。新附之眾,合者漸分,聚者漸散,近者漸遠,庶可消弭反側。一曰清營伍。府縣編氓,既有保甲,諸營什伍,猶未整齊。臣愚以為當責成兵吏,自為版籍。略仿保甲之制,同居連坐。則軍伍肅、盜源遏矣。」事下兵部議行。

既,又上疏言:「下游四府濱海,海徼無險阻可守,且又兵力所不及。宜令居民築土堡,多備長槍鳥銃,習為團練。賊至,人自為守,家自為戰,馳報附近將領,以兵赴援。久之使賊糧絕勢窮,未有不瓦解者也。」又疏論鹽場利弊,請裁上里、海口、牛田諸場,以福清知縣領其事。十六年,報績,授兵部主事。移疾歸。

康熙七年,始出就官。十一年,授工科給事中。時議招募游民,開墾荒田。性疏言:「民貧不能耕,乃有荒田。游民既失業,安能開墾?請敕督撫令縣官勸民開墾,無力者上布政司給牛種貲錢。以本縣之民,墾本縣之田,官既易於稽察,朝廷本貲亦易於徵收。」又迭疏請獎進廉吏,為國家培元氣,密諭推舉督學,以重人才根本之地。又疏陳民生十害,謂:「州縣胥役挾持長吏,為衙蠹之害;官吏私交舊識,關說曲直,為抽豐之害;鄉民錢糧訟獄,必投在城所主之戶,聽其侵蝕唆使,為歇家之害;大奸巨猾武斷鄉曲,為奸豪之害;督撫及司道胥吏乾託有司,為上官胥吏之害;丞簿佐貳濫受訟牒,為佐貳之害;奸民譸張上控,株連蔓衍,為越訴之害;顏料本色,緣時價低昂,不載由單,任意苛斂,為雜派之害;百姓十室九空,無藉乘急取利,逐月合券,俗謂『印子錢』,利至十之七八,折沒妻孥,為放債之害;郵傳往來,強捉人夫,挽舟負輿,為纖夫之害。請下所在有司,每季書狀,不蹈十害,申大吏按驗。」又請飭督撫嚴飭所司,復社學,講鄉約,舉節孝,立義塚,不力行者,不得與卓異。旋擢掌科。十五年,以疾乞歸,家居三年,卒。

國初循明舊,御史出為巡按。順治七年罷,旋復設。八年,世祖親政,特敕誡諭,並命都察院察訪舉劾。御太和殿,召新命諸巡按入見,賜坐宣諭。十七年,都察院復請罷,王大臣會議,安親王及侍郎石申等議留,別疏上。又以御史陸光旭疏爭,令再議,仍議罷不復設。巡按能舉其職者,又有寧承勛按河南,請塞黃河決口;秦世禎按江蘇,劾罷巡撫土國寶:最知名。承勛大興人,明天啟舉人,自禮部主事考選御史,官至大理寺右寺正。世禎自有傳。

王命岳,字伯咨,福建晉江人。順治十二年進士,改庶吉士。時雲南、貴州未定,策問及之。命嶽言:「李定國貳於孫可望,當緩定國,行間使與可望相疑忌。我兵以守為戰,以屯為守,視隙而動。」上異之,擢工科給事中。上經國遠圖疏,略言:「今國家所最急者,財也。歲入千八百一十四萬有奇,歲出二千二百六十一萬有奇。出浮於入者四百四十七萬。國用所以不足,皆由養兵。各省鎮滿、漢官兵俸米、草豆,都計千八百三十八萬有奇,師行芻秣又百四十萬,其在京王公百官俸薪、披甲俸餉不過二百萬。是則歲費二千二百萬,十分在養兵,一分在雜用也。臣愚以為今日不宜再議剝削以給兵餉,而當議就兵生餉之道。河南、山東、湖廣、陜西、江南北、浙東西、江西、閩、廣諸行省,迭經兵火水旱,田多荒廢。宜令各省駐防官兵分地耕種,稍仿明洪武中屯田之法,初年有司給與牛種、耕具、餼糧,自次年後,兵皆自食其力,便可不費朝廷金錢,此其為利甚溥。古者郡縣之兵,什伍相配,千百成旅,將帥因而轄之。乃者將帥多以僕從、摎役、優伶為兵,其實能操戈殺賊者十不得二三。故食糧有兵,充伍無兵。官去兵隨,難議屯種。今當先定兵額,官有升降,兵無去來。平定各省及去賊二三百里外者,皆給地課耕。因人之力與地之宜,一歲便可生財至千餘萬。群情不為深慮,不過議節省某項、清察某項。譬如盤水,何益旱田?臣見今日因賊而設兵,因兵而措餉,因餉而病民。民復為賊,展轉相因,深可隱憂。要在力破因循,斷無不可核之兵,斷無不可耕之田,斷無不可生之財。」疏下各直省督撫,議格不行。

世祖惡貪吏,令犯贓十兩以上籍沒。命岳疏言:「立法愈嚴,而糾貪不止,病在舉劾不當。請敕吏部,督撫按舉劾疏至,當參酌公論,果有賢者見毀,不肖者蒙譽,據實覆駁。如部臣耳目有限,科道臣皆得執奏。又按臣原有都察院考核甄別,督撫本重臣,言官恐外轉為屬吏,參劾絕少。請特敕責成,簡別精實。每歲終仍命吏部、都察院考核督撫舉劾當否,詳具以聞。庶激勵大法以倡率小廉。」轉戶科。再上疏論漕弊,大要謂:「百姓為運官所苦,運官又自有其苦,不得不苦百姓。請革通倉需索,禁旗丁混搶,倉場督臣親監河兌。」福建方用兵,時又苦旱,命岳疏陳五事,曰:緩徵買,糶勸賑,督催協餉,嚴治奸盜,安置投誠。

十五年,調兵科。師下湖廣,命岳復申屯田之議,請復明軍衛屯田之制,設指揮、千百戶等官,以勞久功多之臣膺其任,子孫世及。無漕之地,專固封疆;有漕之地,即使領運。新附之將,有功亦得拜官。量易其地,勿在本省。尋疏言:「各省除荒之數,歲縮銀五百五十萬有奇。荒地以河南、山東為最多。請選清正御史,督察二省田地,率諸州縣清丈,編造魚鱗圖冊。他省除荒多者,如例均丈。」得旨舉行。命岳又上清丈事宜十餘條。

明桂王既出邊,雲南猶未平。命岳疏言:「雲南歲餉九百萬,而一省正雜賦稅都計十六萬有奇,是以九百萬營十六萬之地也。雲南原有舊屯萬一千一百七十一頃有奇,科糧三十八萬九千九百九十二石有奇。請敕巡撫袁懋功責成原軍,換帖領種。暫發二十萬金,買牛辦種,借給軍民。經年銷算,必無虧損,又可收復科糧舊額。且官收額內,軍餘額外,每粟一石,價可三金,視今年每石十二金,已省餉費四分之三。庶幾兵食兼足,不至竭天下之物力以奉一隅。」上可其奏,命發十萬金買牛辦種,修復舊屯。

命岳乞假歸葬,還朝,疏言:「賊習於海戰,我師皆北人,不諳水性。惟有堵截隘港,禁絕接濟,嚴號令,輕徭賦,與民休息,使民不為賊,賊不得資。久之必有系醜獻闕下者。」吏部以浙江右布政員盡忠遷廣東左布政,命已下,命岳劾其貪穢,盡忠坐罷。康熙初,使廣東還,遷刑科都給事中。時陳豹據南澳,尚為明守,命岳疏請招豹收南澳。尋以議獄未當,奪官。六年,畿輔旱,詔求直言。命岳家居,以天子方沖齡,宜覽古今,廣法戒,撰千秋寶鑒,書垂成,未進,卒。

李森先,字琳枝,山東掖縣人。明崇禎進士。順治二年,自國子監博士考選江西道監察御史。啟睿親王發大學士馮銓貪穢及其子源淮諸不法狀,御史吳達,給事中許作梅、莊憲祖、杜立德,御史王守履、羅國士、鄧孕槐、桑蕓等先後論劾。睿親王於重華殿集大學士,刑部、科道諸臣,召銓等面質,以為無實跡,語詳銓傳,責森先啟請肆市語過當,奪官。世祖既親政,銓罷去。九年十一月,大學士範文程以劾銓諸疏進,上閱之竟,曰:「諸臣劾銓誠當,何為以此罷?」文程曰:「諸臣劾大臣,無非為君國,上當思所以愛惜之。且使大臣而能鉗制言官,非細故也。」越數日,上諭吏部,諸臣以劾銓罷者皆起用,森先補原官。

十三年,巡按江南,劾罷貪吏淮安推官李子燮、蘇州推官楊昌齡,論如律。巡蘇州,杖殺不法僧三遮、優王紫稼並為優張榜少年沈濬,一時震悚。淮安吏張電臣坐侵蝕漕折銀一百二十兩有奇,例當追比,森先為疏請緩之。上責森先徇縱,奪官,逮至京訊鞫,事白,復原官。

十五年,應詔陳言,略曰:「上孜孜圖治,求言詔屢下;而諸臣遲回觀望者,皆以從前言事諸臣,一經懲創,則流徙永錮,相率以言為戒耳。臣以為欲開言路,宜先寬言官之罰。如流徙諫臣李呈祥、季開生、魏琯、李裀、郝浴、張鳴駿等,皆與恩詔因公詿誤例相應。倘蒙俯賜軫恤,使天下昭然知上寬宥直臣,在遠不遺。凡有言責者,有不洗心竭慮而興起者乎?」上責其市恩徇情,奪官,下刑部議,流徙尚陽堡,上仍寬之,復原官。尋命察荒河南,用左都御史魏裔介言,給敕印,未訖事而卒。

十七年,上命吏部開列建言得罪諸臣,其流徙者,舉呈祥、琯、裀、開生及彭長庚、許爾安凡六人。上命釋呈祥,許琯、開生歸葬。餘雖系建言,情罪不同,無可寬免。裀、開生自有傳。長庚、爾安事見睿親王傳。

呈祥,字吉津,山東霑化人。明崇禎進士,選庶吉士。順治初,授編修。累遷少詹事。十年二月,條陳部院衙門應裁去滿官,專用漢人。上諭大學士洪承疇等曰:「呈祥此奏甚不當。昔滿臣贊理庶政,弼成大業。彼時豈曾咨爾漢臣?朕滿、漢一體眷遇,奈何反生異意耶?」副都御史宜巴漢等因劾呈祥,奪官,下刑部,坐呈祥巧言亂政,論斬,上命免死,流徙盛京。居八年,至是命釋還,詣京師疏謝,遂還里。康熙二十七年,卒。

琯,字昭華,山東壽光人。明崇禎進士,官御史。順治二年,以薦起原官,巡按甘肅。請開馬市以柔遠人,下部議行。涼州兵劫參議道廨,捕得倡亂者二十餘人,琯疏言西陲兵驕悍,由明季專事姑息,養奸滋亂,宜用重典。上命悉誅之,並詔後有犯者,首從駢斬,著為令。

四年,授江寧學政。七年,還京,掌河南道。八年,漕運總督吳惟華請輸銀萬,又括諸項羨餘,得九萬三千,請以助餉。琯疏言淮、揚連年水旱,惟華輸餉皆分派屬吏,仍取自民間,乞賜察究,會巡漕御史張中元發惟華貪黷狀,逮治奪官。琯又劾鄖陽撫治趙兆麟,甄別文武屬吏,薦舉多至數十,糾劾僅一二微員,上為責兆麟,並誡諸督撫不得劾微員塞責。九年,授順天府府丞。

十二年,遷大理寺卿。八旗逃人初屬兵部督捕,部議改歸大理寺,琯疏言其不便,乃設兵部督捕侍郎專董其事。又言:「逃人日多,以投充者★。本主私縱成習,聽其他往,日久不還,概訟為逃人。逃人至再,罪止鞭百,而窩逃猶論斬,籍人口、財產給本主。與叛逆無異,非法之平。」下九卿議,改為流,免籍沒。又言窩逃瘐斃,妻子應免流徙,時遇熱審,亦應一體減等。上責其市恩,下王大臣議琯巧寬逃禁,當坐絞,上寬之,降授通政司參議。德州諸生呂煌窩逃事發,州官當坐罪,琯持異議。王大臣劾琯,因追議琯前請熱審減等為煌地,坐奪官,流徙遼陽,卒於戍所。上既許歸葬,並宥其孥還故里。

諸與森先同時劾馮銓者:吳達,江南人。自刑部員外郎授御史。順治二年七月,疏言:「今日用人,皆取材於明季。抗直忤時,山林放棄,此明季所黜而今日當用者也。逆黨權翼,貪墨敗類,此明季所黜而今日不可不黜者也。持祿養交,倒行逆施,此明季未黜而今日不可不黜者也。定鼎初年,藉招徠為名,猶可兼收邪正。江南既定,人材畢集,若復涇渭不分,則君子氣沮,宵小競進。即如阮大鋮、袁宏勛、徐復陽輩,聯袂而至,豈可概加錄用乎?至廣開言路,尤為創業急務。乃動責回奏,是沮敢諫之氣而塞後進之路也。即如趙開心論事爽剴,用其人矣,而所規切時政,果一一用之否耶?」得旨:「朝廷用人,非曰誘之,若先既錄用,後無罪而黜,是有疑心矣。屢飭回奏,欲求其實,非沮言路也。」疏寢不用。旋命巡按山東、湖南,官至太僕寺少卿。

桑蕓,山西榆次人。自行人授御史,巡按順天,累遷光祿寺卿。出為河南汝南道參政,督民墾荒土,除雜派,捕治巨猾斃杖下。累遷廣東左布政。道卒。

又有許作梅,河南新鄉人。亦以劾銓罷,復起官至太僕寺少卿。王守履,山西寧鄉人。自工部郎中授御史,巡按湖北。羅國士,山東德州人。自禮部主事授御史,巡按順天。莊憲祖,直隸東光人。以明進士起戶科給事中。順治三年新進士,除科道,憲祖與吏科都給事中向玉軒疏爭,下刑部,並坐奪官。玉軒,四川通江人。鄧孕槐,失其籍,自順天府推官授御史,巡按江南。

李裀,字龍袞,山東高密人。順治六年,以舉人考授內院中書舍人。擢禮科給事中,轉兵科。劾吏部郎中宋學洙典試河南,宿妓納餽,鞫實,奪官。

八旗以俘獲為奴僕,主遇之虐,輒亡去。漢民有原隸八旗為奴僕者,謂之「投充」,主遇之虐,亦亡去。逃人法自此起。十一年,王大臣議,匿逃人者給其主為奴,兩鄰流徙;捕得在途復逃,解子亦流徙。上以其過嚴,命再議,仍如王大臣原議上。十二年,裀上疏極論其弊曰:「皇上為中國主,其視天下皆為一家。必別為之名曰『東人』,又曰『舊人』,已歧而二之矣。謂滿洲役使軍伍,猶兵與民,不得不分;州縣追攝逃亡,猶清勾逃兵,不得不嚴覈:是已。然立法過重,株連太多,使海內無貧富良賤,皆惴惴莫必旦夕之命。人情洶懼,有傷元氣,可為痛心者一也。法立而犯者眾,當思其何利於隱匿而愍不畏死。此必有居東人為奇貨,挾以為■K1。殷實破家,奴婢為禍,名義蕩盡,可為痛心者二也。犯法不貸,牽引不原,即大逆不道,無以加此。破一家即耗一家之貢賦,殺一人即傷一人之培養。十年生聚,十年教訓,今乃用逃人法戕賊之乎?可為痛心者三也。人情不甚相遠,使其居身得所,何苦相率而逃,況至三萬之多?其非盡懷鄉土、念親戚明矣。不思恩義維系,但欲窮其所往,法愈峻,逃愈多,可為痛心者四也。自逮捕起解,至提赴質審,道路驛騷,雞犬不寧。無論其中冤陷實繁,而瓜蔓相尋,市鬻鋃鐺殆盡。日復一日,生齒彫殘,誰復為皇上赤子?可為痛心者五也。又不特犯者為然,饑民流離,以譏察東人故,吏閉關,民扃戶,無所投止。嗟此窮黎,朝廷方蠲租煮粥,衣而食之,奈何因逃人法迫而使斃?可為痛心者六也。婦女躅躑於郊原,老稚殭僕於溝壑。強有力者,犯霜露,冒雨雪,東西迫逐,勢必鋌而走險。今寇孽未靖,招撫不遑,本我赤子,乃驅之作賊乎?可為痛心者七也。臣謂與其嚴於既逃之後,何如嚴於未逃之先?今逃人三次始行正法,其初犯再犯,不過鞭責。請敕今後逃人初犯即論死,皇上好生如天,不忍殺之,當仿竊盜刺字之例:初逃再逃,皆於面臂刺字。則逃人不敢逃,即逃人自不敢留矣。」疏入,留中。後十餘日,下王大臣會議,僉謂所奏雖於律無罪,然「七可痛」,情由可惡,當論死,上弗許,改議杖,徙寧古塔;上命免杖,安置尚陽堡。逾年,卒。

上深知逃人法過苛重,絀王大臣議罪裀。十三年六月,諭曰:「朕念滿洲官民人等,攻戰勤勞,佐成大業。其家役使之人,皆獲自艱辛,加之撫養。乃十餘年間,背逃日眾,隱匿尤多,特立嚴法。以一人之逃匿而株連數家,以無知之奴僕而累及官吏,皆念爾等數十年之勞苦,萬不得已而設,非朕本懷也。爾等當思家人何以輕去,必非無因。爾能容彼身,彼自體爾心。若專恃嚴法,全不體恤,逃者仍眾,何益之有?朕為萬國主,犯法諸人,孰非天生烝民,朝廷赤子?今後宜體朕意省改,使奴僕充盈,安享富貴。」十五年五月,復諭曰:「督捕逃人事例,屢令會議,量情申法,衷諸平允。年來逃人未止,小民牽連,被害者多。聞有奸徒假冒逃人,詐害百姓,將殷實之家指為窩主,挾詐不已,告到督捕,冒主認領,指詭作真。種種詐偽,重為民害。如有旗下奸宄橫行,許督撫逮捕,並本主治罪。」逃人禍自此漸熄。

季開生,字天中,江南泰興人。順治六年進士,改庶吉士。累遷禮科給事中。明將張名振犯上海,開生疏言防御海寇,宜遠偵探,扼要害,備器械,嚴海禁,杜接濟,密譏察。十一年,因地震,疏言:「地道不靜,民不安也。民之不安,官失職也。官之失職,約有十端:一曰格詔旨,二曰輕民命,三曰縱屬官,四曰庇胥吏,五曰重耗剋,六曰納餽遺,七曰廣株連,八曰閣詞訟,九曰失彈壓,十曰玩糾劾。」分疏其目以上,章下所司。調兵科右給事中。

十二年秋,乾清宮成,發帑遣內監往江南採購陳設器皿,民間訛言往揚州買女子,開生上疏極諫。得旨:「太祖、太宗制度,宮中從無漢女。朕奉皇太后慈訓,豈敢妄行,即太平後尚且不為,何況今日?朕雖不德,每思效法賢聖主,朝夕焦勞。若買女子入宮,成何如主耶?」因責開生肆誣沽直,下刑部杖贖,流尚陽堡,尋卒戍所。十七年,旱,下詔罪己,命吏部察謫降言官,諭曰:「季開生建言,原從朕躬起見,準復官歸葬,廕一子入監讀書。」

弟振宜,字詵兮。順治四年進士,授浙江蘭溪知縣。行取刑部主事,遷戶部員外郎、郎中。十五年,考選浙江道御史。及上以旱下詔罪己,言十二、十三年間,時有過舉。振宜疏言:「伏讀上諭,興革責之部院,條奏責之科道,而內閣諸臣闕焉未及。夫用人行政,其將用未用、將行未行之際,毫釐千里,間不容發。天顏咫尺,呼吸可通者,惟內閣諸臣。皇上親政以來,憂勤惕厲,原未見有過舉。皇上以為有過舉矣,試問其時有言及者乎?則宰相之不言亦可見矣。皇上以心膂股肱寄之內閣諸臣,徒以票擬四五字了宰相事業,皇上縱不譴責,清夜捫心,恐有難以自慰者。」得旨:「閣臣不能盡言,初非其罪。前諭十二、十三年間過舉,皆已行之事。朕心過失,即今豈能盡無,閣臣何由得知?部覆章奏,照擬票發,皆朕親裁,亦非閣臣之咎。朕恆慮此心稍懈,諸臣其各加內省!」

左都御史魏裔介疏劾大學士劉正宗蠹國亂政,振宜亦疏舉正宗樹黨納賄諸罪狀,正宗以是得罪。互見正宗傳。振宜又疏言:「府庫已竭,兵革方興。雲南守御,專任平西王,滿兵抽十之四五駐湖南。鄭成功為閩、浙、江南三省之患,當擇地駐兵,絕其登陸。閩撫徐永楨、浙督趙國祚、浙撫史記功,軍旅皆不嫺習,宜簡賢員以代其任。山東、河南輔翼京師,連年水旱,盜賊實繁。北直八府,白晝公行劫掠。明末流寇,殷鑒不遠。蒙古闌入陜西洮、岷一帶耕種,西寧抵宣、大,長城頹塌,防衛空虛。國家中外一統,疆界原宜分明,何可聽其出入不加譏察?」又請復六科封駁舊制,復以揚、徐近河諸縣加派河夫為民間重累,疏請申禁,下部議行。尋命巡視河東鹽政。乞歸,卒。

順治初以建言名者,又有給事中常若柱、張國憲。若柱疏言:「賊相牛金星弒君殘民,抗拒王師,力盡始降,宜嬰顯戮。乃復玷列卿寺,靦顏朝右。其子銓同父作賊,冒濫為官,任湖廣糧儲道,贓私鉅萬。請將金星父子立正國法,以申公義,快人心。」得旨:「流賊偽官投誠者,多能效力。若柱此奏,殊不合理,應議處。」遂罷歸。國憲疏言:「前朝廠衛之弊,如虎如狼,如鬼如蜮。今易錦衣為鑾儀,此輩無能,逞其故智。乃臣聞有緝事員役在內院門首,訪察賜畫。賜畫特典,內院重地,安所用其訪察?城狐社鼠,小試其端。臣竊謂宜大為之防也。」疏入,下廷臣議禁止,得旨:「鑾儀衛專司扈從,訪役緝事,一概禁止。」廠衛之禍始息。若柱,陜西蒲城人。順治四年進士,自庶吉士改戶科給事中。國憲,順天宛平人。順治三年進士,除吏科給事中。

張煊,山西介休人。明崇禎間進士,自知縣擢河南道御史。為大學士陳演所構,遣戍。順治元年,薦起原官,以憂歸。三年,復補浙江道御史,仍掌河南道事。六年,疏言:「有司朘削小民,督撫徇不以告。言官論劾,乃其職守。乞付廷臣公議,勿遽下獄對理。」上從之,諭:「惟挾仇誣陷,仍奪官治罪。自非然者,雖有不實,不得逕送刑部。」八年,疏言:「文武全才難得。近以武職改任督撫,恐政體民瘼未必曉暢,請還本職。」又言:「貪吏坐贓,多委諸吏役,遇赦輒復原官。請將援免諸人應左降者,調補閒曹;應奪官者,勒令休致。」下部議行。

是年值計典,煊以河南道掌計冊,劾御史李道昌、王士驥、金元正、匡蘭兆、李允嵒等巡方失職。時大學士洪承疇掌都察院,甄別諸御史,議道昌降調,士驥等均奪官,並列煊外轉。煊疏劾吏部尚書陳名夏,以故明修撰,諂事睿親王,驟陟尚書,父為縣民所殺,賜銀歸葬。名夏夤緣奪情,恤典空懸。因舉紊亂銓序,把持計典,列十罪、二不法,並及名夏與洪承疇、陳之遴於火神廟屏左右密議,承疇送母回籍未先奏,亦非法。疏下王大臣勘奏。時上方出獵,巽親王滿達海等召名夏、承疇與煊質,名夏事俱實,承疇言火神廟集議,即為甄別諸御史,送母回籍未先奏,當引罪。上還京,復命王大臣廷鞫,吏部尚書譚泰袒名夏,奏名夏事在赦前;煊奏不多實,且先為御史不言,今當外轉,挾私誣衊,罪當死,因坐絞。九年正月,譚泰得罪,上復發煊疏,命王大臣覆讞,名夏坐奪官。語詳名夏傳。遂下詔雪煊冤,贈太常寺卿,賜祭葬。以贈官官其子基遠,官至禮部侍郎。

論曰:國初言事侃侃,以開心為最。義、起龍皆用言事致顯擢,克簡巡方著聲績,命岳策屯田雖未用,要自有所見。森先、裀、開生以謇直蒙譴,獨森先復起。煊死非罪,世尤哀之;然挾外轉之嫌,授讒人以隙,與森先諸人不同矣。


\end{pinyinscope}