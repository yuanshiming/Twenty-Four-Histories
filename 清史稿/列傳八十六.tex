\article{列傳八十六}

\begin{pinyinscope}
馬會伯從兄際伯際伯弟見伯覿伯路振揚韓良輔弟良卿子勛

楊天縱王郡宋愛

馬會伯,陜西寧夏人。康熙三十九年一甲一名武進士,授頭等侍衛。四十五年,授直隸昌平參將,累遷雲南永北總兵。五十九年,師入西藏,命會伯與總兵趙坤率綠旗兵會都統法喇從征。西藏定,敘功,加左都督。雍正元年,入覲,世宗書榜賚焉,曰「有儒將風」,並賜貂冠、孔雀翎。其從弟覿伯,以山西大同總兵率師駐山丹衛,命會伯代鎮,賜白金五百。二年,還鎮永北。

三年,擢貴州提督,疏言:「貴州土瘠兵貧,臣捐穀千石,所屬四營將備捐千石,貯以濟兵。來歲續捐增貯。」上善之。初,廣順屬長寨仲苗最悍,總督高其倬奏移兵設汛。是歲,建營房,仲苗出阻。會伯會總兵石禮哈率兵捕治,得其酋阿革、阿紀及川販為主謀者李奇,悉誅之,餘眾詣軍前聽命。會伯復赴宗角、者貢、穀隆關、羊城諸地督建營房,得旨嘉獎。

四年,調甘肅,未至,又調署四川,旋授四川巡撫。五年,疏劾按察使程如絲營私網利,遣侍郎黃炳按鞫得實,論罪如律。會伯疏言:「四川巡撫舊有稅規耗銀三萬九千有奇,令並入正項。富順鹽規一萬有奇,令改增引課。仍留丁糧、鹽、茶耗規等一萬七千有奇,為巡撫養廉及犒賞之用。」報聞。又疏請清察隱糧,爭控田地,按名丈量。四川清丈自此起。

調湖北,疏請整飭庶獄,重校刻洗冤錄,頒發州縣,議如所請。七年,命往肅州督西路軍需,並權肅州總兵。上諭之曰:「此任朕屢經斟酌,用滿員,恐與岳鍾琪掣肘;用文吏,則能諳軍機實心任事者甚少。委託於汝,慎毋負任用!」尋擢兵部尚書,仍督兵需,並領肅州總兵如故。八年,上責會伯貽誤,奪職,仍署總兵效力。乾隆元年,卒。

際伯,會伯從兄。初入伍,從勇略將軍趙良棟討吳三桂,復略陽,敗敵陽平關。下四川,奪小關山,克建昌,遂定雲南。敘功,授千總,累加參將銜。又從振武將軍孫思克征噶爾丹,破敵昭莫多。敘功,加副將銜。康熙三十六年,授寧夏鎮標前營游擊。從總兵殷化行擊噶爾丹,至洪敦羅阿濟爾罕。累遷四川建昌總兵。遭母喪,巡撫能泰請留任,上命在任守制。四十六年,入覲,調西寧,賜孔雀翎、鞍馬。五十年,授四川提督。卒,贈右都督,賜祭葬,謚襄毅。

見伯,際伯弟。康熙三十年武進士。洪敦羅阿濟爾罕之役,見伯在行。敘功,授守備。累遷山西太原總兵。上西巡,賜貂褂、蟒袍。母喪,並命在任守制。上復西巡,賜孔雀翎。上命弁兵內通曉文義者得應武鄉會試,見伯疏言武經七書言主解互異,請敕儒臣選定。下部議駁,上諭曰:「見伯此奏亦是。武經七書文義駁雜,朕曾躬歷行間,知用兵之道,七書所言,安可盡用耶?」命再議,乃議武試論二:一以論語、孟子命題,一以孫子、吳子、司馬法命題。見伯並請祭孔子,副將以下皆陪祭,上特允之。旋調天津。五十八年,擢陜西固原提督。五十九年,上命貝子延信為平逆將軍,率兵定西藏,以見伯參贊軍務,屢破敵。師還,次打箭爐,卒,賜祭葬。

覿伯,見伯弟。康熙四十二年武進士,選三等侍衛,授巡捕南營參將。累遷大同總兵。策妄阿喇布坦侵哈密,覿伯率師出駐推河。雍正元年,入覲,賜孔雀翎。命移軍駐山丹衛。二年,還鎮。三年,上諭之曰:「爾前入見,朕命爾受巡撫諾岷教導。近聞爾等俱聽年羹堯指揮,此甚非是。嗣後諸事,當商諸署巡撫伊都立。」尋追議在軍時因事與將軍爭競,奪官,命轄鄂爾坤、圖拉屯田。五年,獻瑞麥,一莖十五穗。上諭曰:「今歲各省產嘉禾,覿伯復獻瑞麥。帝王本不以祥瑞為尚,恐有司借端粉飾,致旱潦不以上聞。雍正五年以後,各省產嘉禾,停其進獻。」乾隆元年,卒。

路振揚,陜西長安人。初入伍,拔補把總。累遷漢中副將。康熙五十一年,擢四川松潘總兵。五十六年,策妄阿喇布坦侵西藏,命四川提督康泰率兵往青海御之。至黃勝關柏木橋,兵譁潰,振揚往鎮撫。事定,以振揚署提督。疏言:「松潘迤南雜穀土司種繁俗悍,土司良爾吉子班第爾吉,臣密令防隘,頗稱勤順,請襲職,並予賞賚。又加渴瓦寺安撫土司桑郎溫愷募眾運糧,漳臘營轄旗命上下包坐司土兵習戰鬥、諳邊情,臣令備兵候調,咸知踴躍,亦請予賞賚。」皆如所請。雍正元年,調重慶總兵。

四年,遷陜西固原提督。疏言:「國家設祿以養廉,立法以懲貪。例定以財行賕,及說事過錢人,審實計贓同科。罪未發而自首者免罪,猶徵正贓。竊思官吏營私,彼此容隱,不易敗露,或有告發,猶必互相掩飾。臣請開自首之路,凡上司保題屬吏,並大計軍政卓異,薦舉人員,以財行賕,彼此皆應治罪。如受者自首,免追贓及應得之罪。如與者自首,則照原贓倍追給主,亦免應得之罪。或說事過錢人自首,免罪給賞。如是,庶彼此皆存顧慮,未事則畏懼不敢為,既事則爭首惟恐後。是或除貪之一法。」奏入,上嘉之,曰:「向聞振揚操守廉潔,今覽此奏,非一塵不染者不敢言也。」下部議行,並命優敘。

六年,上念振揚老,召詣京師,授兵部尚書。振揚以病固辭,上疑其戀外任、懷怨望,命停俸,旋改鑾儀使。八年,署直隸古北口提督。九年,上以古北口、宣化、大同沿邊要地當增兵,獨石口西至殺虎口當增兵,並修邊墻。敕御史舒喜、天津總兵補熙會振揚詳勘。振揚等奏請改設副將以下官,增兵千四百有奇,於各鎮營抽撥;邊墻傾圮,用木柵鹿角堵塞:從之。乾隆元年,回鑾儀使任。旋卒,賜祭葬。

韓良輔,字翼公,陜西甘州人。父成,字君輔,康熙中官重慶總兵。在任十七年,有威惠,民德之。卒,祀名宦祠,葬合州,遂入籍重慶。

良輔,多力有膽氣,年十五,即隨父殺賊。補縣學生員,棄去肄武。康熙二十九年,中式武舉第一。三十年,成一甲三名武進士,選二等侍衛。出為陜西延綏游擊,遷宜君參將。境多盜,有為之主者,捕得必連坐。又多虎,造虎槍,教士卒刺虎法,殺虎百餘,患遂息。遷神木副將,調直隸大名,又移石匣。五十九年,率古北口兵五百赴西寧軍前聽調遣。雍正元年,遷天津總兵,賜孔雀翎。

授廣西提督。廣西多山林,宜藤牌挑刀。良輔令步兵弓箭輭弱者皆改肄牌刀,並增制軍械,買馬以壯易羸。二年,署廣西巡撫。奏言:「廣西土曠人稀,多棄地,其故有六:山谿險峻,瑤、僮雜處,田距村遠,穀熟慮盜割,一也;民樸愚,但取濱江及山水自然之利,不知陂渠塘堰可資蓄洩,二也;不得高卑宜植糧種,三也;不知耕耨,四也;所出祗米穀,納賦必用銀,且徭隨糧起,恐貽後累,五也;良懦墾熟,豪猾勢占,六也。宜選大員督率守令,度地居民,立茅舍,貸牛種,興陂渠塘堰,嚴冒占之禁,寬催科之期,使民知有利無害,皆奮興從事,邊徼可成樂土。」上命李紱為巡撫,令良輔協同料理。三年,良輔以天河三甿瑤、僮時出劫掠,檄柳慶副將孫士魁率兵捕治,並曉以利害,上甿莫旺東等、中甿賈貴翁、下甿覃明甲等皆出降。師還,復撫定宜山屬那隘、三岔諸寨。

四年,復署巡撫。遭嫡母喪,命在任守制。五年,實授巡撫。疏言:「廣西撫、提、鎮三標歲需兵糧七萬六千石有奇,各屬額徵糧數,有無多寡不同。撥運供支,有司既苦繁費,兵士又虞乏食。請酌水道遠近,糧額多少,勻給撥運;並多徵折色,以給舟楫不通之地。」下部議行。上命紱以侍郎奉使,與良輔赴貴州安籠,與總督鄂爾泰議分界,事畢,還廣西。坐前官提督時奉議土民羅文剛抗阻設汛,未早捕治,奪官。七年,卒。

良輔既以兵略顯,子弟多肄武。季弟良卿、長子勛尤知名。

良卿,字省月。康熙五十一年武進士,授侍衛。出為陜西西寧守備,再遷莊浪參將。師討謝爾蘇部土番,從涼州總兵楊盡信擊敵棋子山,功多,賜孔雀翎,賚白金千。累遷寧夏中衛副將、廣西碣石總兵,移肅州。乾隆五年,擢甘肅提督。卒,賜祭葬,謚勤毅。

勛,字建侯。年十九,中式武舉。康熙五十六年,祖成請效力,命在內廷行走。五十九年,師征西藏,勛隨良輔赴噶斯應援。雍正元年,授三等侍衛。出為貴州威寧游擊,未赴,改鎮遠。五年,從提督楊天縱擊仲苗,遷雲南鎮雄參將。八年,烏蒙惈為亂,擾鎮雄、永善。總督鄂爾泰令分兵三道進攻,令提督張耀祖、總兵哈元生各出一路,而以勛將四百人出鎮雄奎鄉,進次莫都都,惈數千出拒,力戰一晝夜,殺二百餘,破寨四。翌日,惈復犯奎鄉,勛擊之。戰三日,殺二千餘,盡焚其寨。時元生已克烏蒙,惈屯魯甸,拒大關以守。耀祖軍次東川不進,鄂爾泰復檄勛自鎮雄夾攻,循途搜斬,破寨百餘。克發烏關,至黃水河,環攻敵壘,大破之,克大關、小關。鎮雄、永善相繼下。捷聞,上諭曰:「參將韓勛,領兵四百,破賊數千。以寡敵眾,鼓三軍之氣,喪賊人之膽,較諸路為獨先。」命優敘。超擢貴州安籠總兵。

九年,移古州,討定稿平苗。十三年,疏言:「古州苗寨接壤郡縣,請視湖廣例,得與內地兵、民聯姻。庶彼此感喻,習知禮義,可底善良。」從之。清江諸苗犯王嶺汛,勛率兵擊之,苗退踞臺拱,勢猶熾,率副將王濤截擊,破烏公、八妹諸寨,進屯朗洞。乾隆元年,從經略張廣泗進攻牛皮大箐,自朗洞旋師,途毀二十餘寨。三年,按治定番州姑盧等寨苗。四年,疏言:「古州西北地名滾縱,臨容江,接牛皮大箐,實為要隘,當設兵防守。」允其請。六年,粵瑤挾黎平黑洞苗入境焚劫,擊走之,擒其首惡石金元等,置之法。擢貴州提督。八年,卒,贈右都督,賜祭葬,謚果壯。

楊天縱,字景聖,陜西渭南人。年十七,父母相繼沒,遂入伍。嘗從勇略將軍趙良棟下雲南,冒矢石,負重創。補四川提標把總,遷瓘邊營千總。康熙三十九年,打箭爐西藏營官喋吧昌側集烈為亂,天縱從提督唐希順討之,易服入敵中數往返,希順用其言為攻取計。四十年,攻二道水、磨岡、磨西面諸地,爭先摧敵,克打箭爐。敘功,加游擊銜,授浙江處州都司。三遷署山東沂州副將。

五十七年,授貴州定廣副將,入覲,上命加總兵銜,留沂州任。山東鹽梟勢張甚,天縱按行各汛,行至費縣,聞有聲自遠至,勢且數百人。正夕,天縱令從騎伏路旁,俟其近,驟出擊之,皆驚潰。逐之,及於柱子村,擒其渠,俘數百。又擊之於蒙陰、於泰安,餘眾悉解散。五十九年,調廣東雷州副將,山東巡撫李樹德以沂州險要,請仍留任,許之,加都督僉事。

雍正元年,遷雲南臨元總兵。魯魁惈夷方景明等恃眾據險,恆出掠。天縱偕布政使李衛率兵捕治,悉殲焉。四年,授貴州提督。五年,疏言:「各省考察軍政,所劾多千總、把總,至一二十員不等。千把總雖微員,有防汛、護餉、解逃、捕盜之責,如有偷惰,應不時斥革,何待此時?蓋緣提鎮以是塞責,且有所劾即有所擢,祗圖可得錙銖。上負君恩,下屈末弁。請敕提鎮,嗣後千把總有劣員,即時斥革。」上韙之,諭兵部著為令。

總督鄂爾泰討平長寨仲苗,環其地東西南皆生苗,獷悍不受約束,內地仲苗以為逋逃藪。天縱從鄂爾泰招撫,遣參將劉成謨率熟苗頭人推誠勸諭,生苗有求見,令薙發,予以衣冠酒食,使轉相化導。受撫者百四十八寨、五千六百餘口。敘功,予拖沙喇哈番世職。

巡撫張廣泗清理苗疆,丹江苗糾眾抗拒,天縱遣兵助剿,疏言:「舊存大砲過重,餘砲力不及遠。臣以己意制砲,大者曰靖蠻大砲,能及數里;小者曰過山鳥,攻遠便捷。選兵送廣泗行營聽用,並調安籠、安南、大定、黔西、長寨諸營兵攜砲赴凱里一路,分布進攻。」上嘉天縱料理合宜。七年,疏劾前署巡撫祖秉圭「不諳事機,廣泗未至日,在教場閱操,言將盡剿諸苗,以致頑苗抗拒,勞師動眾。臣不敢隱諱」。上諭曰:「生苗必經此懲創,方可久安。朕以祖秉圭不勝任,已予罷退。此類情事,焉能逃朕鑒察耶?」九年,以老致仕,加太子太保。十年,請改籍四川成都。旋卒,賜祭葬,謚襄壯。

王郡,陜西乾州人。康熙三十年,陜西饑,就食福建,以李姓入伍,補臺灣鎮標把總,遷延平城守千總。六十年,臺灣民硃一貴為亂,總督滿保檄郡赴援。自廈門渡海,一晝夜至淡水,佐守備陳策固守,與策安集民、番。師至諸羅,往會,從克臺灣。二歲中四遷。雍正元年,擢浙江嚴州副將,奏復姓。尋又遷江西南贛總兵。六年,調臺灣。九年,上以郡在臺灣,三年任滿,例當調內地,命總督劉世明選代郡者。世明舉海壇總兵呂瑞麟,令赴臺灣就郡諮度兵民風土,乃調郡潮州。

十年,擢福建提督。臺灣北部社番為亂,瑞麟與臺灣道劉象愷往剿,郡赴臺灣鎮撫。南路亂渠吳福生等竊發,郡率兵於虎頭山、赤山、碑頭諸地逐捕,擒福生,餘黨悉平,加都督同知。尋北路大甲西、沙轆、牛罵諸社番殺掠兵民,郡自鹿仔港偵知阿束一社有北侖、西侖、東侖、惡馬諸地,為亂番所聚,令游擊邱有章、李科等攻西侖,參將李廕樾、游擊林黃彩等攻東侖、惡馬,而游擊黃貴,守備蔡彬、蔡棨等攻北侖。亂番設伏拒我師,督兵奮擊,悉討平之,加左都督。

十一年,調水師提督。十二年,疏言:「廈門環海,地少人多,需米不貲。加以營兵赴糴,難免匱乏。水師提督公廨舊有官房,魚池賃於民,歲得息五千餘。請買穀貸於兵,俟穀熟買補,數年內可得數萬石。孤島兵民,庶無虞艱食。」上諭曰:「郡將應得租息籌濟兵食,甚可嘉也。」命議敘。尋入覲,途次遘疾,遣太醫診視,賜藥餌。二子:守乾、守坤隨侍,召入見,賜守乾守備銜,守坤戶部主事。乾隆元年,復入覲,賜鞍馬、弓矢。時部議許民間得制鳥槍防盜,郡言:「臺灣遠在海表,番、漢雜處。禁例一開,恐火器充斥。小則侵界擾番,大則偶遇水旱,群不逞藉以為亂。臺灣民居多平衍,山箐中皆生番,各險要皆置兵戍守。民間不需鳥槍,懇仍舊例禁止。」從之。十一年,請老,加太子少保,食全俸。二十一年,卒於家,賜祭葬,謚勤愨。守乾官至南昌總兵。

宋愛,字體仁,陜西靖遠人。父可進,雍正初,以京營參將從撫遠大將軍年羹堯討羅卜藏丹津。敵攻鎮海堡,遣赴援,擊殺六百餘人,敵敗走。敵又攻西寧南川口,圍申中堡,復遣赴援,堡兵出夾擊,敵敗走,擢副將。從提督岳鍾琪攻郭隆寺,毀寨七,焚其屋宇七十餘所。旋與鍾琪分道深入,定青海。擢涼州總兵,授三等阿達哈哈番世職。復從鍾琪攻謝爾蘇土番,戰桌子山,圍之七晝夜,一日數接戰。可進受重創,奮進破其巢,遂討平之,擢甘州提督。

愛,雍正元年武進士,授三等侍衛。二年,命省可進軍中。桌子山之戰,愛從可進奮戰有功。河南河北鎮總兵紀成斌請以愛授河南開封都司,上疑成斌受羹堯指,允其請,即令愛傳諭詰成斌。成斌奏:「開封都司,省會重地,去年剿桌子山,親見愛奮不顧身,極有膽氣。且代可進料理營務,頗有才幹。知其能勝任,故冒昧陳請。臣實未受何人囑託,即可進亦不過同在軍中相識,素無交情。」上諭之曰:「朕原不過揣度之辭。近年年羹堯握兵柄,若爾等蔑國恩,重私誼,甚非朕保全功臣之意。今既無別故,意在為地得人,朕甚嘉賞。」再遷浙江紹興副將,命署總兵,歷南陽、永州、天津、定海諸鎮。

乾隆六年,擢襄陽總兵。七年,調安籠。十年,貴州總督張廣泗奏言:「古州系新闢苗疆,諸鎮中惟愛詳慎周密,年力正壯,請以調補。」上從之。丁母憂,命暫署,服闋後真除。十八年,擢貴州提督。前提督丁士傑奏言古州苗因公役使不從,恐激成驕抗,諭將吏彈壓。愛奏:「古州苗於應備夫役,一呼即至,初未見遲延。所屬新疆苗民,亦不至驕抗。苗性難馴,惟在有司善於約束。平時不煩苛,有事不姑息。務使懷德,兼知畏法。」上褒勉之。十九年,總督碩色劾愛馬政廢弛,又為故鎮遠總兵吳三傑■H0資治喪。會愛卒,寢其奏。

論曰:雍正間文武多通用,高其位以提督逕授大學士為最著。會伯、振揚皆長兵部,然會伯未上官,振揚不久改右班,其績仍在專閫。良輔為疆吏,卓卓有建白,家世出將,與會伯略同。天縱、郡、愛等弭亂綏氓,因事有功。年羹堯部將如宋可進、黃喜林、武正安、周

瑛、王嵩、馬忠孝,岳鍾琪部將如紀成斌、曹勷、張元佐,皆相從轉戰,惟可進以有子愛,名字猶可見,他皆不具始末。成斌、勷且以微罪死,是亦重可哀已!


\end{pinyinscope}