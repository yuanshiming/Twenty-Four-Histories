\article{列傳一百七十八}

\begin{pinyinscope}
倭仁李棠階吳廷棟

倭仁,字艮峰,烏齊格里氏,蒙古正紅旗人,河南駐防。道光九年進士,選庶吉士,授編修。歷中允、侍講、侍讀、庶子、侍講學士、侍讀學士。二十二年,擢詹事。二十四年,遷大理寺卿。文宗即位,應詔陳言,略曰:「行政莫先於用人,用人莫先於君子小人之辨。夫君子小人藏於心術者難知,發於事跡者易見。大抵君子訥拙,小人佞巧;君子澹定,小人躁競;君子愛惜人才,小人排擠異類;君子圖遠大,以國家元氣為先,小人計目前,以聚斂刻薄為務。剛正不撓、無所阿鄉者,君子也;依違兩可、工於趨避者,小人也。諫諍匡弼、進憂危之議,動人主之警心者,君子也;喜言氣數、不畏天變,長人君之逸志者,小人也。公私邪正,相反如此。皇上天亶聰明,孰賢孰否,必能洞知。第恐一人之心思耳目,揣摩者眾,混淆者多,幾微莫辨,情偽滋紛,愛憎稍涉偏私,取舍必至失當。知人則哲,豈有他術,在皇上好學勤求,使聖志益明,聖德日固而已。宋程顥云,『古者人君必有誦訓箴諫之臣』。請命老成之儒,講論道義,又擇天下賢俊,陪侍法從。我朝康熙間,熊賜履上疏,亦以『延訪真儒』為說。二臣所言,皆修養身心之要,用人行政之源也。天下治亂系宰相,君德成就責講筵。惟君德成就而後輔弼得人,輔弼得人而後天下可治。」疏入,上稱其切直,因諭大小臣工進言以倭仁為法。未幾,禮部侍郎曾國籓奏用人三策,上復憶倭仁言,手詔同褒勉焉。

尋予副都統銜,充葉爾羌幫辦大臣。大理寺少卿田雨公疏言倭仁用違其才,上曰:「邊疆要任,非投閒置散也。若以外任皆左遷,豈國家文武兼資、內外並重之意乎?」咸豐二年,倭仁復上敬陳治本一疏,上謂其意在責難陳善,尚無不合,惟僅泛語治道,因戒以留心邊務,勿託空言。候補道何桂珍上封事,言倭仁秉性忠貞,見理明決,生平言行不負所學,請任以艱鉅,未許。三年,倭仁劾葉爾羌回部郡王阿奇木伯克愛瑪特攤派路費及護衛索贓等罪,詔斥未經確訊,率行參奏,下部議,降三級調用。

四年,侍郎王茂廕等請命會同籌辦京師團練,上以軍務非所長,寢其議。尋命以侍講候補入直上書房,授惇郡王讀。五年,擢侍講學士。歷光祿寺卿、盛京禮部侍郎。七年,調戶部,管奉天府尹事,劾罷盛京副都統增慶、兵部侍郎富呢雅杭阿。及頒詔中外,命充朝鮮正使。召回京,授都察院左都御史。同治元年,擢工部尚書。兩宮皇太后以倭仁老成端謹,學問優長,命授穆宗讀。倭仁輯古帝王事跡,及古今名臣奏議,附說進之,賜名啟心金鑒,置弘德殿資講肄。倭仁素嚴正,穆宗尤敬憚焉。

尋兼翰林院掌院學士,調工部尚書、協辦大學士。疏言:「河南自咸豐三年以後,粵、捻焚掠,蓋藏已空,州縣誅求仍復無厭。朝廷不能盡擇州縣,則必慎擇督撫。督撫不取之屬員,則屬員自無可挾以為恣睢之地。今日河南積習,祗曰民刁詐,不曰官貪庸;祗狃於愚民之抗官,不思所以致抗之由。惟在朝廷慎察大吏,力挽積習,寇亂之源,庶幾可弭。」是年秋,拜文淵閣大學士,疏劾新授廣東巡撫黃贊湯貪詐,解其職。

六年,同文館議考選正途五品以下京外官入館肄習天文算學,聘西人為教習。倭仁謂根本之圖,在人心不在技藝,尤以西人教習為不可;且謂必習天文算學,應求中國能精其法者,上疏請罷議。於是詔倭仁保薦,別設一館,即由倭仁督率講求。復奏意中並無其人,不敢妄保。尋命在總理各國事務衙門行走。倭仁屢疏懇辭,不允;因稱疾篤,乞休,命解兼職,仍在弘德殿行走。八年,疏言大婚典禮宜崇節儉,及武英殿災,復偕徐桐、翁同龢疏請勤修聖德,停罷一切工程,以弭災變,並嘉納之。十年,晉文華殿大學士,以疾再乞休。尋卒,贈太保,入祀賢良祠,謚文端。光緒八年,河南巡撫李鶴年奏建專祠於開封,允之。

初,曾國籓官京師,與倭仁、李棠階、吳廷棟、何桂珍、竇垿講求宋儒之學。其後國籓出平大難,為中興名臣冠;倭仁作帝師,正色不阿;棠階、廷棟亦卓然有以自見焉。倭仁著有遺書十三卷。子福咸,江蘇鹽法道,署安徽徽寧池太廣道,咸豐十年,殉難寧國,贈太僕寺卿,騎都尉世職;福裕,奉天府府尹。從子福潤,安徽巡撫。光緒二十六年,外國兵入京師,闔家死焉。

李棠階,字文園,河南河內人。道光二年進士,選庶吉士,授編修。五遷至侍讀。二十二年,督廣東學政,擢太常寺少卿。會巡撫黃恩彤奏請予鄉試年老武生職銜,嚴旨責譴,棠階亦因違例送考,議降三級調用,遂引疾家居。文宗即位,復日講,曾國籓薦棠階醇正堪備講官,召來京。既而日講中輟,棠階以病未赴。

咸豐三年,粵匪北犯,河北土寇蜂起,用尚書周祖培薦,命治河北團練。棠階聯絡村鎮,名曰「友助社」。賊踞溫縣東河灘柳林,四出焚掠,棠階督團練擊之,村民未習戰,且無火器,殺賊數十人,卒不敵。會山東巡撫李僡率兵至,賊引去。賊自渡黃河,始知民間有備,稍稍牽綴。洎河北肅清,敘勞,加四品卿銜,賜花翎。

同治元年,詔起用舊臣,棠階應召至。上疏言:「用人行政,惟在治心。治心之要,莫先克己。請於師保匡弼之餘,豫杜左右近習之漸。暇時進講通鑒、大學衍義諸書,以收物格意誠之效。」又言:「紀綱之飭,在於嚴明賞罰。凡朝廷通諭諸事,務飭疆臣實力奉行,庶中外情志可通,而禍亂可弭。」兩宮嘉納焉。授大理寺卿。先是兩江總督何桂清僨事逮治,部讞從重擬斬決,廷臣有右之者,言部臣有意畸重,仍從本律監候。棠階疏謂桂清貽誤封疆罪大,不當輕比,非公論。後桂清卒伏法。連擢禮部侍郎、左都御史,署戶部尚書。召對,言:「治天下惟在安民,安民必先察吏。今日之盜賊,即昔日之良民,皆地方有司貪虐激之成變。為今日平亂計,非輕徭薄賦不能治本。然非擇大吏,則守令不得其人,亦終不能收令行禁止之效。」因極言河南亂事,及諸行省利病甚悉。命為軍機大臣,具疏力辭,弗許。二年,授工部尚書。

三年,江寧克復,論功,加太子少保。大憝既平,上諭中外臣工以兢業交勉。棠階語恭親王及同直諸大臣,謂當設誠致行,久而不懈,勿徒以空言相文飾,王深然之。翼日召對,王反復陳君臣交儆之義,棠階與同僚繼言之,兩宮改容嘉納。尋調禮部尚書。太后命南書房、上書房諸臣纂輯前史事跡,賜名治平寶鑒,命諸大臣進講。棠階因講漢文帝卻千里馬事,反復推言人主不宜有所嗜好,以啟窺伺之端。自是每進講必原本經義,極論史事,歸於責難陳善。四年,恭親王被劾退出軍機,棠階謂王有定難功,時方多故,不當輕棄親賢,入對,力言王非有心之失。會惇、醇兩王亦奏言奕不可遽罷,乃復命入直。僧格林沁戰歿曹州,棠階以朝廷賞多罰少,疆臣每存藐玩,上疏極言其弊,於是有申飭直省督撫之諭。

棠階自入直樞廷,軍書旁午,一事稍有未安,輒憂形於色。積勞致疾,十一月,卒,年六十八。上震悼,遣貝勒載治奠醊,賜金治喪,贈太子太保,謚文清。

棠階初入翰林,即潛心理學,嘗手鈔湯斌遺書以自勖。會通程、硃、陸、王學說,無所偏主,要以克己復禮、身體實行為歸。日記自省,畢生不懈。家故貧,既貴,儉約無改。嘗曰:「憂患者生之門。吾終身不敢忘忍饑待米時也!」

吳廷棟,字竹如,安徽霍山人。道光五年拔貢,授刑部七品小京官,洊遷郎中。廷棟少好宋儒之學,入官益植節厲行,蹇蹇自靖。咸豐二年,京察一等。時侍郎書元兼崇文門副監督,獲販私釀者三十六人,承審者以漏稅擬滿杖。已而覆訊得書元家人詐贓狀,部臣據以入奏。文宗疑書元孤立,降旨切責,會廷棟召對,上詢是獄。廷棟從容敷奏,且詳陳治道之要,言利之害,君子小人之辨,上首肯,獄竟得解。因詢廷棟讀何書,廷棟以程、硃對。上曰:「學程、硃者每多迂拘。」對曰:「此不善學之過。程、硃以明德為體,新民為用,天下未有有體而無用者。皇上讀書窮理,以裕知人之識;清心寡欲,以養坐照之明。寤寐求賢,內外得人,天下何憂不治?」上韙之。

尋出為直隸河間知府。粵匪北犯畿輔,廷棟練民兵巡防,民倚以為固。內閣學士勝保督師至河間,責供張甚急,知縣王灴迫於應付,自刎不殊。廷棟詣大營陳其事,勝保矍然,飭部下聽命。連擢永定河道、直隸按察使。以河間京師門戶,廷棟善守御,得民心,仍留知府任。四年,軍事定,乃之按察使任。六年,遷山東布政使。時部臣奏請畿內賦稅兼收大錢鈔票各三成,上下交病,總督譚廷襄不敢言。會廷棟入覲,面奏:「大錢鈔票實不流通。立法必先便於民方可行,必先信於民方能久。今條科太多,朝夕更改,國家先不能自信,何以取信於民?」上首肯者再。既而廷襄入朝,遂奏罷前議。山東吏治久窳,廷棟獎廉懲貪。方議海口立局收貨捐,持不可。八年,坐奏銷遲誤,降補直隸按察使。十一年,復調山東。同治二年,入為大理寺卿,尋擢刑部侍郎。

三年,江南平,廷棟上疏,略曰:「萬方之治亂在朝政,百工之敬肆視君心。事不貴文,貴其實;下不從令,從所好。夫治亂決於敬肆,敬肆根於喜懼。自古功成志遂,人主喜心一生而驕心已伏,宦寺有乘其喜而貢諂媚者矣,左右有乘其喜而肆蒙蔽者矣,容悅之臣有因此而工諛佞者矣,屏逐之奸有因此而巧夤緣者矣。諂媚貢則柄暗竊,蒙蔽肆則權下移,諛佞工則主志惑,夤緣巧則宵小升。於是受蠱惑,塞聰明,遠老成,惡忠鯁。從前戒懼之念,一喜敗之;此後侈縱之行,一喜開之。方且矜予智,樂莫違,逞獨斷,快從欲,一人肆於上,群小扇於下,流毒蒼生,貽禍社稷,稽諸史冊,後先一轍。推原其端,祗一念由喜入驕而已。軍興以來,十數省億萬生靈慘遭鋒鏑,即倡亂之奸民,何一非朝廷赤子?大兵所加,盡被誅夷。皇太后、皇上體上天好生之心,必有哀矜不忍喜者。況旗兵乏食,根本空虛,新疆缺餉,邊陲搖動。兼之強鄰偪處,邪教肆行,豈惟不可喜,而實屬可懼。假使萬幾之餘,或有一念之肆,臣工效之,視彰癉為故事,輕告戒為具文,積習相沿,工為粉飾,將仍成為叢脞怠荒之局矣。是非堅定刻苦,持之以恆,積數十年恭儉憂勤,有未易培國脈復元氣者。夫上行必下效,內治則外安,而其道莫大於敬,其幾必始於懼。懼天命無常,則不敢恃天;懼民碞可畏,則不敢玩民。懼者敬之始,敬者懼之終。大智愈明,神武愈彰,紹祖宗富有之大業,開子孫無疆之丕基,是皆由皇心之懼始而敬成也。易曰:『危者使平,易者使傾,懼以終始,其要無咎。』詩曰:『敬之敬之,天維顯思!』可弗以為永鑒歟?」疏上,優詔嘉納,命存其疏於弘德殿以備省覽。皇太后召對時,諭曰:「皇帝沖齡踐阼,國家大事,汝宜直言無隱,以無負先帝知遇。」廷棟感激出涕。五年,以衰病乞休,許之,歸寓江寧。十二年,卒,年八十有一。遺疏入,詔褒其廉靜自持,賜恤如例。直隸、山東皆祀名宦祠。

廷棟學以不欺為本。官臬司時,畿輔連有逆倫獄,總督慮一月頻入奏乾上怒,廷棟曰:「此吾儕不能教化之過,待罪不暇,敢欺飾耶?」及去官,僑居清貧,不受餽遺。著有拙修集十卷。

論曰:倭仁晚為兩宮所敬禮,際會中興,輔導沖主,兢兢於君心敬肆之間,當時舉朝嚴憚,風氣賴以維持。惟未達世變,於自強要政,鄙夷不屑言,後轉為異論者所藉口。李棠階、吳廷棟正色立朝,不負所學,翕然笙磬同音,而棠階尤平實持大體,可謂體用兼備矣。


\end{pinyinscope}