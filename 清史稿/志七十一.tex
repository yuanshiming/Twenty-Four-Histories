\article{志七十一}

\begin{pinyinscope}
○樂三

△樂章一郊廟群祀

圜丘九章郊廟樂,順治元年定,乾隆十一年用舊辭重改。今以順治所制分載句中。中和韶樂,黃鍾宮立宮,倍夷則下羽主調。

迎神始平欽原敬。承純祜原祐。兮,於昭有融。時維永清兮,四海攸同。輸忱元祀兮,從律調風。穆將景福兮,乃眷微躬。淵思高厚兮,期亮天工。原恐負鴻則。聿章彞序兮,夙夜宣通。雲軿延佇原鸞輅。兮,鸞輅空濛。原忽降中壇。翠旗紛裊兮,列缺豐隆。肅始和暢兮,恭仰蒼穹。原慶洽陶匏。百靈祗衛兮,齊明闢公。神來燕娭兮,惟帝時聰。原恭仰顥穹兮,神來燕喜。協昭慈惠兮,逖鑒臣原予。衷。

奠玉帛景平靈旗爰止兮,樂在懸。原奉玉筵。執事有恪兮,奉玉筵。原駿奔前。聿昭誠敬兮,駿奔前。原有美圭璧兮,薦縞纖。嘉玉量幣兮,相後先。原經緯獲理兮,耀瑚璉。來格洋洋兮,思儼然。臣原孔。忱翼翼兮,告中虔。

進俎咸平吉蠲為饎兮,肅豆籩。原升肴珍錯兮,列豆籩。升肴列俎兮,敢弗虔。原吉蠲為饎兮,格乾圜。毛炰繭慄原九州美味。兮,薦膏鮮。致潔陶匏兮,香水泉。原特牲潔敬兮,苾芳筵。原隨原垂。降鑒兮,駐雲軿。錫嘉福兮,億萬斯年。

初獻壽平玉斝肅陳兮,明光。桂漿初醖兮,信芳。臣心迪惠兮,捧觴。醴齊載德兮,馨香。靈慈徽眷兮,矞皇。勤仰止兮,斯徜徉。

亞獻嘉平考鐘拂舞兮,再進瑤觴。翼翼昭事兮,次第肅將。睟顏容與兮,蒼幾輝煌。穆穆居歆兮,和氣洋洋。生民望澤兮,仰睨玉房。榮泉瑞露兮,慶無疆。

終獻永平原雍平。終獻兮,玉斝清。肅秬鬯兮,薦和羹。原微誠。磬管鏘鏘兮,祀孔明。原協氣升。旨酒盈盈原盈盈旨酒。兮,勿替思成。明命顧諟原尚其醉止。兮,群福生。原懷嘉生。八龍蜿蜒兮,苞羽和鳴。

徹饌熙平一陽復兮,協氣升。原盥薦畢兮,精白申。盥薦畢兮,精白陳。原虞燕娭分,勞帝神。旋廢徹兮,敢逡巡。原百闢肅雍兮,傾罍尊。禮將成兮,樂欣欣。瞻九閶兮,轉洪鈞。原無二句。福施下逮兮,佑此原宜佑。人民。

送神清平原太平。升中告成原嘉德夙成。兮,晻靄壇場。穆思回盼兮,靈駕洋洋。原有山河日月兮,朗耀崇深。青龍按節兮,白虎低昂。洪鈞滌蕩兮,妖孽潛消。三句。臣原我。求時惠兮,感思馨香。原有鳴玉鏘金兮,肅若有望。紫壇截嶪兮,赫赫皇皇。臣乘寶歷兮,載須我輔。三句。原蒙博產兮,多士思皇。原作山嶽鍾良。天施地育兮,百穀蕃昌。原不可殫究。殖我嘉師兮,正直平康。原沐浴休光。

望燎太平原安平。隆儀告備兮,誠既將。原雷車電邁兮,飛遠揚。有虔秉火兮,焫越芳。雷車電邁兮,九龍驤。原繁會賁鏞兮,奮龍旂。紫氛四塞兮,雲旗揚。原俾爾昌兮,降蕭光。蒸民蒙福兮,順五常。原富壽康。惟予小子兮,敬戒永臧。原予獲疇祉兮,萬億斯皇。

方澤八章中和韶樂,林鍾清變徵立宮,夾鍾清商主調。

迎神中平吉蠲兮,玉宇開。薰風兮,自南來。鳳馭紛兮,後先;岳瀆藹兮,徘徊。肅展禮兮,報功;沛靈澤兮,九垓。

奠玉帛廣平式時原神州。吉土兮,中壇。毖我郊兆原畤。兮,孔安。原嚴。闢公趨蹌兮,就列;原吉蠲。考鐘伐鼓兮,舞般。原肆筵。黃琮纖縞兮,既奠;原陳列。靈光下燭兮,誠丹。原誠悃宣。

進俎含平原咸平。禮行樂奏原玉俎金奏。兮,未央。嘉肴有踐兮,大房。牲牷告歆兮,惟恪;民力普存兮,肅將。厚載資生兮,無外;幾筵來格原俯鑒。兮,洋洋。

初獻太平原壽平。醴齊融冶兮,信芳。原匏尊泛齊兮,朝踐揚。博碩升庖兮,鼎方。清風穆穆兮,休氣翔。原靈旗張。神明和樂兮,舉初觴。洽百禮兮,禋祀;罄九土兮,豐穰。

亞獻安平一茅三脊兮,縮漿。原江茅兮,縮漿。山罍雲冪兮,馨香。介黍稷兮,芳旨;再滌犧尊兮,敬將。原再展微誠兮,趨蹌。樂成八變兮,綴兆;原樂只。儼皇祇兮,悅康。

終獻時平紫壇兮,嘉氣盈。原方壇兮,豐薦盈。旨酒思柔兮,和且平。原中和平。懍茲陟降兮,心屏營。原陟降從容兮,駐雲軿。禮成三獻兮,薦玉觥。含宏光大兮,德厚;靈佑丕基兮,永清。

徹饌貞平玉俎列兮,庶品該。原尊俎畢兮,誠未虧。黃琮告徹兮,雲翔徊。原儀景暉。晏陰定兮,曦景回。原邀靈錫。南訛秩兮,日恢臺。原奏薰時。肅惟昭明兮,孔邇;覃博厚奠兮,九垓。原載群黎。

送神望瘞寧平靈旗兮,雲路遵。原雲際屯。飛龍蜿兮,高旻。原飛龍兮,逝駸駸。陰儀粹兮,德純。眷四海兮,無塵。配皇穹兮,兩大;原化宣。綏下土原綏百祿。兮,蒸民。

祈穀九章中和韶樂,黃鍾宮立宮,倍夷則下羽主調。

迎神祈平原中平。帝篤祜民原惟帝勤民。兮,求莫匪舒。小民何依兮,飲食惟需。原黍稷與與。莫嘉於穀兮,萬事權輿。原元日有事兮,百闢趨。為民請命兮,豈非在予。原食咸需。日用辛兮,百闢趨。原遙瞻龍駕兮,歷紫虛。暾將出兮,東風徐。原日臨黃道兮,東風徐。惟予小子兮,敬盥陳孚。原臣昭事兮,遑深寧居。皇皇龍駕兮,穆將愉。原原垂嘉惠兮,大有書。

奠玉帛綏平原肅平。念茲稼墻兮,惟民天。原民天惟食兮,農事先。農用八政兮,食為先。原粒我烝民兮,有大田。雨暘時若兮,玉燭全。原風霆流形兮,雨澤霑。粒我蒸民兮,迄用康年。原實穎實慄兮,氣化全。仰三無私兮,昭事虔。原玉帛祗奉兮,禋祀虔。奉璋承帛兮,慄若臨淵。原仰祈寰宇兮,享豐年。

進俎萬平原咸平。鼎烹兮,苾芬。嘉薦兮,無文。升繭慄兮,惟犉。原奉雕俎兮,大武。膻薌達兮,干雲。原氣幹雲。昭民力兮,普存。原昭普存兮,民力。惟明德兮,馨聞。

初獻寶平原壽平。初獻兮,元原旨。酒盈。致純潔兮,儲精誠。原著誠致潔兮,犧尊盛。瑟黃流兮,罍承。原儼對越兮,在上。酌其中兮,外清明。原惟昭明兮,有融。儼對越兮,維清。原瑟黃流兮,玉瓚。帝心歆假兮,綏我思成。原賚嘉禎。

亞獻穰平原景平。犧原著。尊啟兮,告虔。清酤既馨原次第。兮,陳原舉。前。禮再獻兮,祠筵。原肅拜。光煜爚兮,非煙。原列瑤觴兮,秩斯筵。神悅懌兮,僾然。原如在。惠我嘉生兮,大有年。原福便便。

終獻瑞平原永平。終獻兮,奉明粢。原泰尊移。苾芬嘉旨兮,清醴既釃。原圭瓚交馳。神其衎原神其醉止。兮,錫祉;禮成於三兮,陳詞。原灑餘瀝兮,沐群黎。臣拜手兮,青墀。原望雲霓。

徹饌渥平原凝平。俎豆具陳兮,庶品宜。原齊。肸蚃昭鑒兮,荷帝慈。原舉荷昭鑒兮,靡或遺。饌告備原徹。兮,玉幾;登歌洋溢兮,廢徹不遲。原式禮無違。肅微忱兮,告終事;上帝居歆兮,錫純禧。

送神滋平原清平。祗奉天威兮,弗敢康。小心翼翼兮,昭穹蒼。雲垂九天兮,露瀼瀼。翠旗羽節兮,上翱翔。原歸何鄉。臣拜下風兮,肅原意。徬徨。原沛汪澤兮,民多蓋藏。原時其雨暘。

望燎穀平原太平。卬原翹。首兮,天閶。混茫一氣兮,浩無方。原★A0彼雲海兮,何蒼茫。焫蕭束帛兮,薦馨香。精誠感格兮,降福穰穰。四時順序兮,百穀以昌。臣周兆姓兮,咸荷恩光。

社稷壇七章中和韶樂,春夾鍾清商立宮,倍應鍾清變宮主調;秋南呂清徵立宮,仲呂清角主調。

迎神登平原廣平。媼神蕃釐兮,厚德隆。原猗歟土穀兮,功化隆。嘉生繁祉兮,功化同。原蒸民立命兮,九域同。壇壝儼肅兮,風露融。原通。我稷翼翼兮,黍芃芃。原俎豆豐。望雲駕兮,驂鸞龍。植璧秉圭兮,冀感通。原秉圭植璧兮,予親躬。

奠玉帛、初獻茂平原壽平。恪恭禋祀兮,肅且雍。原禋祀黝牲兮,北郊同。清醑原酤。既載兮,臨齋宮。朝踐初舉兮,玉帛共。原鑒予衷。洋洋在上兮,鑒予衷。原錫福洪。

亞獻育平原嘉平。樂具入奏兮,聲喤喤,鬱鬯再升兮,賓八鄉。原兕觥其觩兮,恭再揚。厚德配地兮,佑家邦。綏我豐年原屢豐年。兮,兆庶康。

終獻敦平原雍平。方壇北宇兮,神中央。盈庭萬原帗。舞兮,犮原時低昂。酌酒原酳。三爵兮,桂原綠。醑香。清雖舊邦原新舊邦。兮,命溥將。

徹饌博平原熙平。大房籩豆原籩豆大房。兮,儼成行。歆此吉蠲兮,神迪嘗。原猶回翔。廢徹不遲原椒漿瑤席。兮,餘芬芳。桐生茂豫兮,百穀昌。原黍稷非馨兮,悅且康。

送神樂平原成平。孔蓋翠旌兮,隨風颺。龍輈容與兮,指天閶。咫尺神靈兮,隔穹蒼。原流景祚兮,貺皇章。原流景祚兮,卜世昌。

望瘞徵平原成平。玉既陳原牲玉陳。兮,延景光。禮既洽原百禮既洽。兮,終瘞藏。原神聽兮,時予匡。垂神佑兮,永無疆。原四海攸同兮,惠無疆。

社稷壇祈雨、報祀七章乾隆十八年定。中和韶樂,仲呂清角立宮,大呂清宮主調。初祈用夾鍾清商立宮,報南呂清徵立宮,旋改隨月用律宮譜,舉四月為例。祈晴、報祀同。

迎神延豐九土博厚兮,阜嘉生。方壇五色兮,祀孔明。甿力穡兮,服耕。仰甘膏兮,百穀用成。熙云路兮,瞻翠旌。殷闓澤兮,展精誠。

奠玉帛、初獻介豐神來格兮,宜我黍稷。兩主有邸兮,馨明德。罍尊湛湛兮,干羽飭。油雲澍雨兮,溥下國。

亞獻滋豐奏明兮,申載觴。龍出泉兮,靈安翔。周寰宇兮,滂洋。載神庥兮,悅康。

終獻霈豐帗容與兮,奮皇舞。聲遠姚兮,震靈鼓。爵三奏兮,縮桂醑。號屏來御兮,德施普。

徹饌綏豐協笙磬兮,告吉蠲。神迪嘗兮,禮莫愆。心齋肅兮,增惕乾。咨田畯兮,其樂有年。

送神貽豐撫懷心兮,神聿歸。蓋郅偈兮,驂虯騑。洪釐渥兮,雨祁祁。公私霑足兮,孰知所為。

望瘞溥豐宣祝嘏兮,列瘞繒。貺允答兮,時欽承。高原下隰兮,以莫不興。歌率育兮,慶三登。

社稷壇祈晴、報祀七章嘉慶十一年重定。中和韶樂,仲呂清角立宮,大呂清宮主調。

迎神延和庶匯涵育兮,陽德亨。句萌茁達兮,物向榮。方壇潔兮,展誠。迓休和兮,寰宇鏡清。祈昭格兮,瞻翠旌。沐日月兮,百寶生。

奠玉帛、初獻兆和瑟圭瓚兮,通微合漠。神歆明德兮,鑒誠恪。昭回雲漢兮,噓橐籥。曜靈司晷兮,時暘若。

亞獻布和申獻侑兮,奉明。薦馨香兮,和氣隨。神介福兮,孔綏。耀光明兮,九逵。

終獻協和帗羽舞兮,一風敞。爵三奏兮,告成享。順年祝兮,泰階朗。元冥收陰兮,日掌賞。

徹饌雍和籩俎徹兮,受福多。笙磬同兮,六律和。庶徵協兮,時無頗。熙樂利兮,東作南訛。

送神豐和神聿歸兮,華蓋揚。羲和整馭兮,虯螭翔。遍臨照兮,協農祥。天清地寧兮,黍稷豐穰。

常雩九章乾隆七年定。中和韶樂,黃鍾宮立宮,倍夷則下羽主調。

迎神靄平粒我蒸民兮,神降嘉生。雨暘時若兮,百穀用成。龍見而雩兮,先民有程。臣膺天祚兮,敢不承。念我農兮,心靡寧。肅明禋兮,殫精誠。靈皇皇兮,穆以清。金支五色兮,罨靄蜺旌。

奠玉帛云平玉帛載陳兮,磬管鏘鏘。為民請命兮,惕弗敢康。令清和兮,遂百昌。麥秀歧兮,禾茀稂。日照九兮,時雨滂。俾萬寶兮,千斯倉。

進俎需平越十雨兮,越五風。三光昭明兮,嘉氣蒙。天所與兮,眇躬。予小子兮,懍降豐。紛總總兮,賴皇穹。犉牡■F1亨兮,達臣衷。

初獻霖平酌彼兮,罍洗;飶芬兮,椒香。愧明德兮,維馨。假黍稷兮,誠將。原大父兮,念茲眾子;穆將愉兮,綏以豐穰。

亞獻露平再酌兮,醑清。仰在上兮,明明。庶來格兮,鑒誠。曷敢必兮,屏營。合萬國兮,形神精。承神至尊兮,思成。

終獻霑平三酌兮,成純。備物致志兮,敬陳。多士兮,駿奔。靈承無斁兮,明禋。維蕃釐兮,媼神。雨留甘兮,良苗懷新。

徹饌靈平禮將成兮,舞已終。徹弗遲兮,畏神恫。原留福兮,惠吾農。神之貺兮,協氣融。遂及私兮,越我公。五者來備兮,錫用豐。

送神霮平祥風瑞靄兮,彌靈壇。上帝居歆兮,風肅然。左蒼龍兮,右白虎;般裔裔兮,糺縵縵。仰九閶兮,返御;介祉釐兮,康年。

望燎霈平碧翏翏兮,不可度思。九奏終兮,爟火晳而。神光四燭兮,休氣夥頤。安匪舒兮,抑抑威儀。帝求民莫兮,日鑒在茲。錫福繁祉兮,庶徵曰時。

大雩雲漢詩八章高宗御制。中和韶樂,黃鍾宮立宮,倍夷則下羽主調。

瞻彼硃鳥,爰居實沈。協紀辨律,羽蟲徵音。萬物蕓生,有壬有林。有事南郊,陟降維欽。瞻仰昊天,生物為心。一章維國有本,匪民伊何。維民有天,匪食則那。螻蟈鳴矣,平秩南訛。我祀敢後,我樂維和。鼉鼓淵淵,童舞娑娑。二章自古在昔,春郊夏雩。曰維龍見,田燭朝趨。盛禮既陳,神留以愉。雷師闐闐,飛廉衙衙。曰時雨暘,利我新畬。三章於穆穹宇,在郊之南。對越嚴恭,上帝是臨。繭慄量幣,用將悃忱。惴惴我躬,肅肅我心。六事自責,仰彼桑林。四章權輿粒食,實維后稷。百王承之,永奠邦極。惟予小子,臨民無德。敢解祈年,潔衷翼翼。命彼秩宗,古禮是式。五章古禮是式,值茲吉辰。玉磬金鐘,太羹維醇。玄衣八列,舞羽繽紛。既侑上帝,亦右從神。尚鑒我衷,錫我康年。六章惟天可感,曰維誠恪。惟農可稔,曰維力作。恃天慢人,弗刈弗穫。尚勸農哉,服田孔樂。咨爾保介,庤乃錢鎛。七章我禮既畢,我誠已將。風馬電車,旋駕九閶。山川出雲,為霖澤滂。雨公及私,興鋤利轆。億萬斯年,農夫之慶。八章

朝日七章順治八年定,乾隆七年重改。初制分載句中。夕月同。中和韶樂,太簇商立宮,倍無射變宮主調。

迎神寅曦羲馭兮,寅賓。原於昭兮,旭輪。光煜爚兮,紅輪。原浴虞淵兮,初升。春已融兮,交泰;循典禮兮,明禋。原惟馨。嚴大採兮,祗肅。原焫蕭艾兮,祗肅。神之來兮,如云。原神其聽兮,和平。

奠玉帛朝曦杲黃道兮,暾出;原神來格兮,太乙東。肅將享原統萬國。兮,玉帛同。美齊翼兮,王君公。原肅將享兮,承篚筐。盥以薦兮,昭格通。原盥以薦兮,孚有容。

初獻清曦禦景風兮,下帝扃。原禦景風兮,神式臨。酌黃目原酌清酤。兮,椒其馨。爵方舉兮,歌且舞;漾和盎兮,龍旗青。原憑龍勺兮,吹鳳笙。

亞獻咸曦再舉勺原奠。兮,鬱金香。嘉樂和兮,舞洋洋。德恢大兮,神哉沛;原神飲食兮,意徜徉。澹容與兮,進霞觴。原容貌舒兮,和以康。

終獻純曦式禮莫愆兮,昭清。原式禮未竭兮,還升。終以告虔兮,休成。原醁醽。原神且留兮,鑒茹;以妥以侑兮,忱誠。原以侑以勸兮,至誠。

徹饌延曦物之備兮,希德馨。原儀既成兮,物已饗。神欲起兮,景杳冥。原神欲起兮,運靈爽。徹不遲兮,咸肅穆。原徹不敢遲兮,慎趨蹌。照臨下土兮,瞻曜靈。原照下土兮,常朗朗。

送神歸曦雲車徵兮,風馬翔。猋萬里兮,臨萬方。原馳驅千仞兮,臨萬方。報神功兮,以時享,祈神祐兮,永無疆。原再拜手兮,稱送;神振轡兮,當陽。中天麗兮,徹隱;普天戴兮,恩光。敷和煦兮,成物;錫萬寶兮,永康。報神功兮,時饗;祈神祐兮,悠久無疆。

夕月七章中和韶樂,南宮清徵立宮,仲呂清角主調。

迎神迎光繼日代明兮,象麗天。原猗歟太陰兮,禦望舒。式遵九道兮,臨八埏。原式遵九道兮,游清虛。玉律分秋兮,西顥躔。原駕冰輪兮,行西陸。聿修毖祀兮,樂在懸。原今之夕兮,來饗予。

奠玉帛、初獻升光少採兮,將事;玉帛兮,載陳。原有來雍雍,幣帛在陳。琮璜以嘉,明德維馨。式舉黃流兮,挹犧尊。籩豆靜嘉兮,肴核芬。

亞獻瑤光齊醍兮,載獻;神之來兮,肅然。原二齊載升,維以告虔。歌管鍠鍠,奉神之歡。仰肸蚃兮,鑒顧;原荷亙古兮,麗天。挹清光兮,幾筵。

終獻瑞光戛瑟鳴琴兮,鋗玉鏘。神嘉虞兮,申三觴。金波穆穆兮,珠熉黃。休嘉砰隱兮,溢四方。原一敬畢申,三舉原釂。誠信潔齊,天下有道。鼓鐘簡兮,聲容並茂。象大德兮,厥光皓皓。

徹饌涵光對越在天兮,禮成。徹登豆兮,湛露零。神悅懌兮,德馨。世曼壽兮,安以寧。原其香既歆。對越告成。徹爾登豆,敬用駿奔。神悅懌兮,意欣欣。予翼慎兮,安以寧。

送神保光駕卿雲兮,景星。禦和風兮,霞軿。神留俞兮,壇宇。福率土兮,黃丁。原彩駕霞兮,驂景星。禦和風兮,躡慶雲。神欲起兮,不再停。瞻天衢兮,拜雲程。影蹁躚兮,光澄清。饗予誠兮,意殷勤。予所祝兮,世太平。偃武修文兮,萬世長春。

大享殿合祀天地百神九章順治十七年定,後未施行,故宮譜失載。乾隆十六年,改大享殿為祈年殿,於此行祈穀之禮焉。祈穀樂章見前。

迎神元和乾元資始兮,仰戴元功。坤厚載物兮,率履攸同。亭毒萬匯兮,昭明有融。陰肅陽舒兮,協氣流通。晝夜遞禪兮,二曜在中。群靈畢萃兮,陟降景從。大德普存兮,化著清寧。臣思報本兮,蠲潔粢盛。延佇雲駕兮,屏息臣躬。馨香祗薦兮,爰殫微誠。瞻望歆格兮,瑞色曈曨。至止壇遣兮,式慰欽崇。

奠玉帛景和俯仰覆載兮,殿萬邦。展儀備物兮,舉舊章。良璧在陳兮,介豆觴。束帛戔戔兮,忱可將。對越冥漠兮,念徬徨。臣虔齊明兮,效趨蹌。降鑒無方兮,悅而康。原錫嘉祉兮,慶未央。

進俎肅和和風暢兮,神格思。洽百靈兮,誠無移。潔豆登兮,答洪慈。膋芬達兮,雜菹施。臣仰祈兮,福履綏。房產芝兮,矞雲垂。祝史列兮,敬陳詞。形聲穆兮,鑒在茲。

初獻壽和威光畢煜,肅肅靈旗。壺觴肇啟,用介神禧。普洽和樂,罄無不宜。鏗鍧迭奏,克葉塤篪。駿奔翼翼,進反有儀。臣薦清酤,眷佑弗違。

亞獻安和齋心夙夜,祈答碧虛。洋洋在上,載酌清醑。苾芬式享,秩秩于於。干戚在舞,張弛靡逾。彌歆元旨,臣藎方舒。永言迓惠,戩穀錫餘。

終獻永和肴核既旅,八音克諧。罇罍未馨,慈惠靡涯。肅將三祝,黃流在臺。菁茅既潔,祼獻徘徊。原言醉止,庶展臣懷。於皇錫祉,景福方來。

徹饌協和百福既洽兮,羞明神。蘋藻可將兮,臣悃申。雲軿欲駕兮,彌逡巡。幾筵敬徹兮,不敢陳。

送神泰和敬酬高厚兮,肅秩靈壇。居歆幸孚兮,進止克嫺。群神偕從兮,馭鶴驂鸞。清風穆穆兮,旌旆生寒。遙開閶闔兮,雲路漫漫。六龍前駕兮,劍佩珊珊。百闢相事兮,卿士戒班。臣心益虔兮,立盤桓。式禮莫愆兮,餘忱未殫。惠及黎庶兮,四宇騰歡。萬物咸若兮,遐邇乂安。綿綿衍慶兮,永奠如磐。

望燎、望瘞清和祥光杳靄兮,滿雲端。霓旌揚兮,言還。虔蕭焫兮,祈上達;百執旅進兮,環列紫垣。臣仰止兮,彌切;束躬翹首兮,望元關。天休滋至兮,欽承罔斁;知神永覆兮,濊澤寬。

太廟時饗六章順治元年定,乾隆七年以舊詞重改。初制載句中。奉先殿同。中和韶樂,太簇商立宮,倍無射變宮主調。

迎神貽平原開平。肇茲區夏,世德欽崇。九州維宅,王業自東。戎甲十三,奮起飛龍。維神格思,皇靈顯庸。原皇輿啟圖,世德欽崇。粵庇眇躬,率土攸同。九州維宅,爰止自東,太室既尊,萬國朝宗。翼翼孝孫,對越肅雝。維神格思,皇靈顯庸。

奠帛、初獻敉平原壽平於皇祖考,克配上天。越文武功,萬邦原四方。是宣。孝孫受命,不忘不愆。原達志承前。羹墻永慕,時薦斯虔。原永錫純嘏,億萬斯年。

亞獻敷平原嘉平。毖祀精忱,原神。洋洋如生。尊罍再舉,於赫昭明。原有融昭明,陟降於庭。僾然有容。愾然有聲。我懷靡及,原孝孫虔只。惕原容。若中情。

終獻紹平原雍平。粵若祖德,誕受方國。肆予小子,大猷是式。原越祖宗之德,肇茲天歷。敢曰予小子,享有成績。欲報之德,昊天罔極。殷勤三獻,中心翼翼。原我心悅懌。

徹饌光平原熙平。庶物既陳,九奏具舉。原儀肅樂成,神燕以娭。告成於祖,亦右皇妣。敬徹不遲,用終殷祀。原用終祀禮。式禮如茲,皇其燕喜。原介福綏祿,永錫祚祉。

還宮乂平原成平。對越無方,陟降無跡。原盈溢肅雝,神運無跡。寢祏靜淵,孔安且吉。原恍兮安適。惟靈在天,惟主在室。於萬斯年,孝思無斁。

太廟大祫六章順治十六年定,乾隆七年以舊辭重改。初制載句中。中和韶樂,太簇商立宮,倍無射變宮主調。

迎神開平原貞平。承眷命兮,撫萬邦。嗣丕基兮,祖德昌。溯謨烈兮,唐哉皇。原弗敢忘。虔歲祀兮,式原舉。舊章。肅對越兮,誠悃將。原瀝悃誠兮,迓休光。尚來格兮,仰休光。原祈來格兮,意徬徨。

奠帛、初獻肅平原壽平。粵我先兮,肇俄朵。長白山兮,鵲銜果。綿瓜瓞兮,天所佐。明之侵兮,殲其左。混中外兮,逮乎我。奉太室兮,安以妥。原紛威蕤兮,神畢臨。儼對越兮,抒素忱。陳纖縞兮,有壬林。酌醇酤兮,薦德馨。恪溥將兮,俶來歆。錫嘉祉兮,祐斯民。

亞獻協平原嘉平。紛葳蕤兮,列聖臨。儼對越兮,心欽欽。陳纖縞兮,有壬林。擊浮兮,彈硃琴。恪溥將兮,肅來歆。錫嘉祉兮,天地心。原維肇祥兮,德配天。垂燕翼兮,祚百年。潔豆籩兮,秩斯筵。載陳醴兮,介牲牷。協笙鏞兮,繞雲軿。肅駿奔兮,中彌虔。

終獻裕平原雍平。椒飶芬兮,神留俞。爵三獻兮,旨清醑。萬羽幹兮,樂孔都。禮明備兮,罔敢渝。神原既。醉止兮,咸樂胥。永啟佑兮,披皇圖。

徹饌諴平原熙平。祝幣陳兮,神燕娭。原典儀敘兮,神格思。尊俎將兮,反威儀。原享靡遺。悅且康兮,徹弗遲。不可度兮,矧射思。禮有成兮,釐百宜。原無此二句。鑒精誠原禋。兮,茀祿綏。

還宮成平原清平。龍之馭兮,旋穆清。原孝思展兮,禮告成。神言歸兮,陟在庭。萃龍馭兮,返穆清。三句。神之御兮,式丹楹。原主肅將兮,式丹楹。瞻列聖兮,僾容聲。回靈眄兮,佑丕承。維神聽兮,和且平。繼序皇兮,亶休徵。

祭先農七章順治十一年定,乾隆七年以舊詞重改。初制載句中。中和韶樂,姑洗角立宮,黃鍾宮主調。

迎神永豐先農播穀,克配彼天。粒我蒸民,於萬斯年。農祥晨正,協風滿★F2。曰予小子,宜稼於田。原句芒秉令,土牛是驅。天下一人,蒼龍駕車。念彼田疇,民命所需。生成有德,尚式臨諸。

奠帛、初獻時豐厥初生民,萬匯莫辨。神錫之庥,嘉種乃誕。斯德曷酬,何名可贊。我酒惟旨,是用初獻。原先農神哉,耒耜教民。田祖靈哉,稼穡是親。功德深厚,天地同仁。肅將幣帛,肇舉明禋。厥初生民,萬匯莫辨。神錫之庥,嘉種乃誕。執茲醴齊,農功益見。玉瓚椒醑,肅雍舉奠。

亞獻成豐無物稱德,惟誠有孚。載升玉瓚,神肯留虞。惟茲兆庶,豈異古初。神曾子之,今其食諸。原上原下隰,百穀盈止。粒我蒸民,秀良興起。樂舞具備,吹豳稱兕。再躋以獻,肴馨酒旨。

終獻大豐秬秠穈芑,皆神所貽。以之饗神,式食庶幾。神其丕佑,佑我黔黎。萬方大有,肇此三推。原穈芑秬秠,維神所貽。以神饗神,曰予將之。秉耒三推,東作允宜。五風十雨,率土何私。

徹饌屢豐青祇司職,土膏脈起。日涓吉亥,舉耕耤禮。神安留俞,不我遐棄。執事告徹,予將舉趾。原於皇農事,自古為烈。莫敢不承,今茲忻悅。籩豆既豐,簠簋雲潔。神視井疆,執事告徹。

送神報豐匪且有且,匪今斯今。靈雨崇朝,田家萬金。考鐘伐鼓,戛瑟鳴琴。神歸何所,大地秧針。原麻麥芃芃,粳稻連阡。縱橫萬里,皆神所瞻。人歌鼓腹,史載有年。歲有常典,茀祿綿延。

望瘞慶豐肅肅靈壇,昭昭上天。神下神歸,其風肅然。玉版蒼幣,瘞埋告虔。神之聽之,錫大有年。原玉版蒼幣,來鑒來歆。敬之重之,藏於厚深。典禮由古,予行自今。樂樂利利,國以永寧。

祭先蠶六章乾隆七年定。仲呂清角立宮,大呂清宮主調。先蠶壇樂,以雲鑼代鐘,方響代磬,與中和韶樂微異。樂章正義後編列入先農壇之次,從之。

迎神庥平軒轅御籙時,西陵位正妃。柔桑沃,載陽遲。黼黻玄黃供祀事,稱繭更繰絲。龍精報貺,椒屋宗師。

初獻承平春堤柳綻金,倉庚有好音。衣褘翟,致精忱。後月躬應教織紝。柘館式齋心。黃流初薦,肸蚃如臨。

亞獻均平清和日正長,靈壇水一方。紆香陌,執籧筐。桑葉陰濃風澹蕩,八育普嘉祥。玉鬯再陳,降福穰穰。

終獻齊平神皋接上園,葭蘆翠浪翻。鶯聲滑,■F3花繁。天棘絲絲初引蔓,三薦潔蘋蘩。雲依寶鼎,露浥旌幡。

徹饌柔平公宮吉禮成,有齋奉豆登。僮僮被,肅肅升。廢徹毋遲咸祗敬,法坎不常盈。萬方衣被,百福其朋。

送神洽平神風拂廣筵,靈香下肅然。儀不忒,禮無愆。禺馬流星相焫絢,玉蝀亙平川。彤管司職,瑞繭登編。

祭歷代帝王廟六章順治二年定,乾隆七年以舊詞重改。初制載句中。中和韶樂,春夾鍾清商立宮,倍應鍾清變徵主調。秋南呂清徵立宮,仲呂清角主調。

迎神肇平原雍平。撫原乘。時兮,極隆。造經綸兮,顯庸。總古今兮,一揆;貽大寶兮,微躬。仰徽猷兮,有嚴閟宮。原有儀群帝兮,後先。一句。予稽首兮,下風。

奠帛、初獻興平原安平。莽若雲兮,神之行。原靈之來兮,儼若盈。予仰止兮,在廷。承筐篚兮,既登。偃靈蓋兮,翠旌。原結翠旌。鑒予情兮,歆享。薦芳馨兮,肅成。原有景行兮,六龍。嘉氣兮,曈曨。奠牛羲尊兮,以笙以鏞。群工肅兮,屏營。惠我懿則兮,允中。五句。

亞獻崇平原中平。貳觴兮,酒行。原有諸帝熙和兮,悅成。一句。念昔致治兮,永清。瞻龍袞兮,若英。原自天。原紹錫兮,嘉平。

終獻恬平原肅平。鬱鬯原瑤爵。兮,獻終。萬舞洋洋兮,沐清風。龍鸞徐整兮,企予。原有嗣徽音兮,何從。盼雲車兮,緩移。二句。示周行兮,迪予衷。

徹饌淳平原凝平。饁肴蒸兮,畢升。五音會兮,滿盈。禮將徹兮,虔告。鑒孔忱兮,載翼載登。

送神匡平原壽平。羽原幡。幢繚繞兮,動回風。和鸞並馭兮,歸天宮。五雲擁兮,高馳翔。原回靈眄兮,錫年豐。

望燎同駕群龍原群龍驂駕。兮,一氣中。熏蒿芬烈兮,窅冥通。望神光兮,遙燭;惟終古兮,是崇。

先師廟六章順治元年定,乾隆七年以舊詞重改。初制載句中。中和韶樂,春夾鍾清商立宮,倍應鍾清變宮主調。

迎神昭平原咸平。大哉至聖,德盛道隆。原峻德宏功。生民未有,原敷文衍化。百王是崇。典則昭垂,原典則有常。式原昭。茲闢雍。載原有。虔簠簋,載原有。嚴鼓鐘。

奠帛、初獻宣平原寧平。覺我生民,陶鑄賢原前。聖。巍巍泰山,實予景行。禮備樂和,豆籩嘉原惟。靜。既述六經,爰斟三正。

亞獻秩平原安平。至哉聖師,克明明德。原天授明德。木鐸萬年,原世。維民之則。原式是群闢。清酒既原維。醑,言觀秉翟。太和常流,英材斯植。

終獻敘平原景平。猗歟素王,示予物軌。瞻之在前,師表萬祀。原神其寧止。酌彼金罍。我酒惟旨。原惟清且旨。登獻雖原既。終,弗遐有喜。

徹饌懿平原成平。璧水淵淵,芹芳藻潔。原崇牙岌嶪。既歆宣聖,亦儀十哲。聲金振玉,告茲將徹。鬷假有成,日月昭揭。原羹墻靡愒。

送神德平原咸平。煌煌闢雍,原學宮。四方來宗。甄陶樂育,原胄子。多士景從。原暨予微躬。如土斯埴,原思皇多士。如金在鎔。原膚奏厥功。佐予敷治,俗美時雍。原佐予永清,三五是隆。

直省先師廟六章乾隆七年重定。中和韶樂,宮調同。

迎神昭平大哉孔子,先覺先知。與天地參,萬世之師。祥徵麟紱,韻答金絲。日月既揭,乾坤清夷。

奠帛、初獻宣平予懷明德,玉振金聲。生民未有,展也大成。★F5豆千古,春秋上丁。清酒既載,其香始升。

亞獻秩平式禮莫愆,升堂再獻。響協卉鼓鏞,誠孚罍甗。肅肅雍雍,譽髦斯彥。禮陶樂淑,相觀而善。

終獻敘平自古在昔,先民有作。皮弁祭菜,於論思樂。惟天牖民,惟聖時若。彞倫攸敘,至今木鐸。

徹饌懿平先師有言,祭則受福。四海黌宮,疇敢不肅。禮成告徹,毋疏毋瀆。樂所自生,中原有菽。

送神德平鳧繹峨峨,洙泗洋洋。景行行止,流澤無疆。聿昭祀事,祀事孔明。化我蒸民,育我膠庠。

太歲壇六章順治元年定,乾隆七年以舊詞重改。初制載句中。中和韶樂,太簇商立宮,倍無射變宮主調。

迎神保平協茲五紀,歲日月辰。天維顯思,神職攸分。於赫太歲,統馭百神。承天之德,陰騭下民。原吉日良辰,祀典孔殷。於維太歲,月將百神。乘時秉德,輔國祐民。遙遙龍馭,頓轡九閽。壇壝蠲潔,延佇來臨。

奠帛、初獻定平原安平。禮崇明祀,涓選休成。潔齋滌志,量幣告成。祈福維何,福我蒼生。陳饋奉酎,瞻仰雲旌。原維神至止,螭駕雲旗。洋洋在上,淑景延禧。束帛承筐,展我誠斯。神示昭鑒,尚其無遺。神兮弭節,薦馨敢後。祀事方初,陳饋奉酎。神光熹微,嘉祥承候。百禮不愆,樂具入奏。

亞獻嘏平原中平。百末蘭生,有飶其香。升歌清越,磬管鏘鏘。牲牷肥腯,嘉薦令芳。神其歆止,在上洋洋。原以我齊明,率禮攸行。再拜稽首,旨酒斯盈。牲牷肥腯,交彼神明。尊罍上下,鬷假思成。

終獻富平原肅平。執事有嚴,再拜稽首。三爵既升,以妥以侑。盥薦有孚,肅茲籩豆。神其歆止,人民曼壽。原執事有嚴,品物斯備。非馨黍稷,用宣誠意。硃弦登歌,絲衣揚觶。於胥樂兮,神錫爾類。

徹饌盈平原雍平。王省維歲,有報有祈。六氣無易,平衡正璣。嘉生蕃祉,澤及蜎飛。百禮以洽,承神吉輝。原春祈秋報,歲省惟勤。含醇飲德,莫匪明神。惟神臨御,肸蚃逡巡。獻酬雲畢,誠敬斯伸。

送神豐平原寧平。神兮旋馭,肅瞻景光。靈飆上下,無體無方。嘉承惠和,億兆溥將。歲歲大有,神其迪嘗。原出令明堂,神爽卒度。報功迎氣,崇祀斯作。神人以和,既康且樂。瞻望景光,邈彼寥廓。

太歲壇祈雨、報祀六章乾隆十八年定。中和韶樂,太簇商立宮,倍無射變宮主調。

迎神需豐持元化兮,富媼神。秉歲籥兮,六氣均。馳雲車兮,風旗;殷闐闐兮,天門。情徬徨兮,孔殷。神之來兮,康我民。

奠帛、初獻宜豐薦嘉幣兮,芳醴清。練予素兮,升飶馨。紛肸蚃兮,格歆。甘膏沃兮,神所令。

亞獻晉豐啟山罍兮,攝椒漿。侑神宮兮,靈洋洋。族雲興兮,使我心若;惠嘉生兮,降康。

終獻協豐清斝兮,三;揚翟籥兮,載愉。靈回翔兮,六幕;澤雱霈兮,遍八區。

徹饌應豐禮儀備兮,孔時。音繁會兮,徹不遲。昭靈貺兮,迓蕃祉;田多稼兮,氾濩之。

送神洽豐顧億兆兮,誠求。渥甘澍兮,神之休。慶時若兮,百昌遂。惠我無疆兮,歲有秋。

天神、地祇壇祈雨、報祀六章乾隆七年定。中和韶樂,天神黃鍾宮立宮,倍夷則下羽主調。地祇林鍾清變徵立宮,夾鍾清商主調。

迎神祈豐云車馳兮,風旆征。雷闐闐兮,雨冥冥。表六合兮,穹青。橫大川兮,揚靈。紛總總兮,來會;穆予心兮,齊明。

奠帛、初獻華豐束帛戔戔兮,筐篚將。昭誠素兮,鬯馨香。瘼此下民兮,候有望。神垂鴻祜兮,渠未央。

亞獻興豐疏冪兮,再啟;芳齊兮,載陳。惠邀兮,神貺;福我兮,人民。

終獻儀豐犧尊兮,三滌;旨酒兮,思柔。誠無斁兮,嘉薦;神燕娭兮,降休。

徹饌和豐禮既成兮,孔殷。潔明粢兮,苾芬。廢徹兮,不遲;至敬兮,無文。

送神錫豐流形兮,露生。苞符兮,孕靈。介我稷黍兮,曰雨而雨;神之格思兮,祀事孔明。

巡祭泰山岱廟六章乾隆十三年定。中和韶樂,林鍾清變徵立宮,夾鍾清商主調。

迎神祈豐資元氣兮,鎮青陽。鼓橐籥兮,孕靈祥。行時令兮,東巡;式展禮兮,誠將。

奠帛、初獻華豐金壇肅穆兮,黼帷張。瑟黃流兮,茅縮漿。昭誠素兮,舉初觴。神斯陟降兮,格馨香。

亞獻興豐日觀兮,雞鳴。天門兮,鳳翔。犧尊兮,再獻;維神兮,降康。

終獻儀豐醴齊兮,三薦;金牒兮,輝煌。申至敬兮,無祈;鑒予誠兮,齋莊。

徹饌和豐瞻石閭兮,在望。實籩豆兮,大房。黍稷兮,非馨;明德兮,是將。

送神錫豐禮成兮,孔臧。神駕兮,龍驤。膚寸而合兮,觸石而起;彌於六合兮,降福穰穰。

巡祭嵩山中嶽廟六章乾隆十五年定。中和韶樂,林鍾清變徵立宮,夾鍾清商主調。

迎神祈豐維靈岳兮,鎮中央。展時巡兮,洛之陽。虔望秩兮,懷柔;儼對越兮,神光。

奠帛、初獻華豐石闕岧嶢兮,鳴鳳翔。奏瑤笙兮,肅祼將。初奉斝兮,陳篚筐。至誠昭格兮,福無疆。

亞獻興豐潁水兮,安恬;緱嶺兮,青蒼。黃琮兮,告薦;椒醑兮,芬芳。

終獻儀豐香升兮,華黍;三滌兮,嘉觴。答靈響兮,嵩門;登萬寶兮,咸昌。

徹饌和豐三臺蔚兮,峻極;二室鬱兮,相望。告徹兮,維時;懷德兮,靡忘。

送神錫豐云車兮,龍驤。仰止兮,高閶。玉漿含滋兮,金璧呈瑞;配天作鎮兮,長發其祥。

望祀長白山六章乾隆十九年定。中和韶樂,林鍾清變徵立宮,夾鍾清商主調。

迎神祈豐天作高山兮,作而康。鍾王氣兮,應期昌。巡豐沛兮,來望。躬禋祀兮,虔將。

奠帛、初獻華豐飶黃流兮,進初觴。緬仙源兮,心遫莊。靄佳氣兮,鬱蒼蒼。欣來格兮,惠無疆。

亞獻興豐硃果兮,實蕃;靈淵兮,澤雱。清尊兮,再獻;綿祚兮,純常。

終獻儀豐具薦兮,玉饌;三酌兮,瓊漿。思王跡兮,彌欽;清緝熙兮,敢忘。

徹饌和豐松花水兮,湯湯。鴨綠波兮,泱泱。神飫兮,錫釐;如川至兮,莫量。

送神錫豐祀事兮,孔臧。昭假兮,永明。邁周岐兮,越殷土;萬有千歲兮,長發其祥。

群祀慶神歡樂乾隆七年定,每歲祭先醫於景惠殿,火神廟、顯佑宮、關帝廟、都城隍廟、東嶽廟、黑龍潭龍神祠、玉泉龍神祠、興工祭后土、司工之神、迎晹祭窯神、門神皆用之。三獻三奏。乾隆三十三年又重定關帝廟迎神、三獻、送神各一章。咸豐三年升入中祀,特制樂章,列後。

先醫精氣緣乎理,調劑觀所頤。曰惟古聖,嘗草定醫,似鐵隨磁。沴除吉至,化工出自於指,萬姓永荷恩施。

顯佑宮居所躔星軫,象緯環拱辰。貞元運轉,藏用顯仁。宥密基命,毓和葆順。潔粢醴,以昭信。日襄哉,贊大鈞。

東嶽廟維嶽崧高五,泰岱常祀殊。累朝玉檢,柴望始虞。木德條風,吹萬畢煦,宅東隅以生物。仰天齊,鑒有孚。

都城隍廟佳麗皇都勝,保障神力宏。萬方輻輳,尨夜不驚。正直聰明,癉彰如影,荷靈貺,篤其慶。固金甌,護玉京。

火神廟離正南方位,燭照光九圍。粒民火食,功用不違。瓘斝明粢,我民祈慰,覆祥靄,戢鶉尾。息融風,降福禧。

龍神祠興雨祁祁應,歷歲恩屢覃。湫幽神御,農扈具瞻。寸合崇朝,十千有渰。黍膏溥,牟麥湛。賽神庥,以作甘。

門神和氣嘉祥應,聖日華耀明。仰方泰紫,俯奠泰寧。遼廓紘瀛,此惟表正。食神德,蒙神慶。享明禋,億萬齡。

司工之神仰眺銀河上,閣道如駕梁。儼神宅只,愉矣穆將。揆日鳴櫜,翳神斯掌。奠椒酒,以禋享。荷神庥,澤未央。

關帝廟扶植綱常正,浩氣昭日星。絕倫獨立,英爽若生。俎豆常馨,夏彞胥敬;仰神德莫疇,並助邦家永太平。

乾隆三十三年,重定關帝廟樂五章

迎神青湛湛,玉霄門。神來下,採旄紛。宮墻輪奐,籩豆芳芬。光景動人民。丹心照日。浩氣扶輪。

奠帛初獻調蘭醑,酌桂尊。神來饗,房俎陳。忠貫金石,義炳乾坤。純臣戴一君。力扶王室,不原三分。

亞獻汎盎齊,觴再進。簫鼓諧,聲歌韻。武節絕倫,不辭利鈍。神勇天威震。方知舊史,未符公論。

終獻禮秩秩,樂欣欣。儼威靈,至今存。惟靈惟佑,佑國佑民。典禮極隆文。式揚顯號,時薦明禋。

徹饌、送神司儀告徹,靈風來洎。神聿歸,嘉徵萃。大濟群生,善良胥得意。邪慝無伸喙,皇化所及。咸尊廟食,東西朔南靡弗暨。

咸豐三年,關帝廟樂七章中和韶樂

迎神格平懿鑠兮,焜煌。神威靈兮,赫八方。偉烈昭兮,累禩;祀事明兮,永光。達精誠兮,黍稷馨香。儼如在兮,洋洋。

奠帛、初獻翊平英風颯兮,神格思。紛綺蓋兮,龍旂。桂醑兮,盈卮。香始升兮,明粢。惟降鑒兮,在茲。流景祚兮,翊昌時。

亞獻恢平觴再酌兮,告虔。舞干戚兮,合宮懸。歆苾芬兮,潔蠲。扇巍顯翼兮,神功宣。

終獻靖平鬱鬯兮,三申。羅籩簋兮,畢陳。儀卒度兮,肅明禋。神降福兮,宜民宜人。

徹饌彞平物惟備兮,咸有。明德惟馨兮,神其受。告徹兮,禮終罔咎。佑我家邦兮,孔厚。

送神康平幢葆葳蕤兮,神聿歸。馭鳳軫兮,驂虯騑。降煙煴兮,餘芬菲。原回靈盼兮,德洽明威。

望燎同熏蒿烈兮,燎有煇。神光遙矚兮,祥雲霏。祭受福兮,茂典無違。庶揚駿烈兮,永奠疆畿。

文昌帝君廟七章咸豐六年升入中祀,重定樂章。中和韶樂

迎神丕平秉氣兮,靈躔。文運兮,赫中天。蜺旌兮,戾止。雕俎兮,告虔。迓神庥兮,於萬斯年。

奠帛、初獻俶平神之來兮,籩簋式陳。神之格兮,幾筵式親。極昭彰兮,靈貺;致蠲潔兮,明禋。升香兮,伊始;居歆兮,佑我人民。

亞獻煥平再酌兮,瑤觴。燦爛兮,庭燎之光。申虔禱兮,神座;儼陟降兮,帝旁。粢醴潔兮,齋遫將。綏景運兮,靈長。

終獻煜平禮成三獻兮,樂奏三終。覃敷元化兮,繄神功。馨香達兮,肸蚃通。歆明德兮,昭察寅衷。

徹饌懿平備物兮,惟時。告徹兮,終禮儀。神悅懌兮,監在茲。垂鴻佑兮,累洽重熙。

送神蔚平雲軿駕兮,風旗招。神之歸兮,天路遙。瞻翠葆兮,企丹霄。原回靈眷兮,福我朝。

望燎同煙煴降兮,元氣和。神光爥兮,梓潼之阿。化成耆定兮,櫜弓戢戈。文治光兮,受福則那。

順治元年,皇帝祭祀回鑾二章導迎樂

天地群祀祜平皇天有命,列聖承之。我後配德,文匡武綏。海隅寧謐,神靈燕娭。於萬斯年,流慶降釐。

太廟禧平於皇紹烈,累熙重光。銷鑠群慝,我武奮揚。肅肅清廟,瓘瓘奉璋。奠鬯斯馨,祚命無疆。

乾隆十七年,重定祭祀回鑾祐平十三章樂章乾隆七年制,十七年始定凡祭祀回鑾樂皆曰祐平,而以慶典所奏者為禧平。導迎樂

圜丘崇德殷薦,升燎告虔。惟聖能饗,至諴天眷。駕六龍,臨紫煙。佑命申,圖籙綿。

方澤隤爾而靜,持載廣生。長至修祀,聿來光景。富媼愉,元德升。岳瀆安,民物亨。

祈穀民者邦本,食乃民天。爰卜辛日,大君殷薦。龍角明,祈有年。耒耜親,天下先。

雩祭炎夏初屆,憫我穡夫。為民請命,法駕載塗。明德馨,誠意孚。禾稼登,斯樂胥。

太廟儀若先典,追孝在天。鴻慶遐鬯,烈光丕顯。祝事明,神貺宣。福庶民,千萬年。

社稷壇分職三大,康乂國家。平土蕃穀,降休中夏。薦吉蠲,神不遐。遍九垓,鳥祉嘉。

堂子禋祀隆永,維統百靈。延福儲祉,奠安神鼎。修祀祠,通紫庭。降福祥,昭德馨。

出師、凱旋告祭堂子維文武略,勛業攸崇。欽承睿算,往征不恭。扇仁風,在師中。月三捷,奏膚功。

日壇雝肅音送,暾出自東。兼燭垂曜,與天用同。秩典修,皇敬通。表瑞輝,揚至公。

月壇殷仲嘗酎,華黍若油。興穀繁祉,受符天後。湧桂華,凝彩斿。玉燭調,千萬秋。

歷代帝王廟時序群品,端在一欽。衣德凝命,荷天之任。景軌儀,誠既歆。肅駿奔,顒若臨。

先師廟先聖垂軌,千載是祗。虔奉師表,景行行止。奠兩楹,神降之。啟後人,文在茲。

先農壇翩彼桑扈,仁氣布和。千畝親御,百祥膺荷。保介歆,穜棱多。帝手推,民樂歌。


\end{pinyinscope}