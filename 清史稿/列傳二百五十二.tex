\article{列傳二百五十二}

\begin{pinyinscope}
徐桐豫師子承煜剛毅趙舒翹啟秀英年

裕祿廷雍毓賢李廷簫

徐桐,字廕軒,漢軍正藍旗人,尚書澤醇子。道光三十年進士,選庶吉士,授編修。坐修改中卷乾磨勘,罷職。咸豐十年,特賞檢討,協修文宗實錄。同治初,命在上書房行走,奉懿旨番講治平寶鑒,入直弘德殿,累遷侍講學士。先後疏請習政事、勤修省,成大學衍義體要以進。數擢至禮部侍郎。念外人麕集京師,和議難恃,宜壹意修攘圖自強;因條上簡才能、結民心、裕度支、修邊備四策。光緒初,授禮部尚書,加太子少保。主事吳可讀請豫定大統,以尸諫,桐與翁同龢等謂其未悉本朝家法:「當申明列聖不建儲彞訓,俾知他日紹膺大寶之元良,即為承繼穆宗之聖子。揆諸前諭則合,準諸家法則符。」疏入,詔存毓慶宮備覽。

時崇厚擅訂俄約,下群臣議,乃條摘其不可行者:曰伊、塔各城定界;曰新疆、蒙古通商;曰運貨逕至漢口;曰行船直入伯都訥。六年,廷議徇俄人請,將赦崇厚罪,桐力持不可,謂:「揆度機要在樞廷,折沖俎豆在總署,講信修睦在使臣。赦之而彼就範,猶裨國事;若釁端仍不能弭,反失刑政大權。推原禍始,宜肅國憲。」又言:「今日用人之道,秉忠持正者為上,宅心樸實者次之。若以機權靈警,諳曉各國語言文字,遽目為通才,而責以鉅任,未有不僨且蹶者!」不報。歷充翰林院掌院學士、上書房總師傅。十五年,以吏部尚書協辦大學士,晉太子太保。二十二年,拜體仁閣大學士。

桐崇宋儒說,守舊,惡西學如仇。門人言新政者,屏不令入謁。二十四年政變後,太后以其耆臣碩望,頗優禮,朝請令近侍扶掖以寵之。

有豫師者,字錫之,內務府漢軍。進士。官至烏魯木齊都統,以講學為桐所傾服。方太后議廢帝,立端王載漪子溥俊為「大阿哥」,桐主之甚力,實皆豫師本謀也。既而桐被命照料,益親載漪。各國不慊載漪等所為,漪恚甚,圖報復。二十六年,義和拳起釁仇外,載漪大喜,導之入都。桐謂:「中國當自此強矣!」至且親迓之。然及其亂時,仍被劫掠。袁昶、許景澄之死,舉國稱冤,而桐則曰:「是死且有餘辜!」時其子承煜監刑,揚揚頗自得。

承煜,字楠士。拔貢。以戶部小京官晉遷郎中,累官刑部左侍郎。已,聯軍入,桐倉皇失措,承煜請曰:「父芘拳匪,外人至,必不免,失大臣體。盍殉國?兒當從侍地下耳!」桐乃投繯死,年八十有二矣。而承煜遂亡走,為日軍所拘,置之順天府尹署,與啟秀俱明年正月正法。命下,日軍官置酒為餞,傳詔旨,承煜色變,口呼冤,痛詆西人不已。翼日,備輿送至菜市,監刑官出席禮之,已昏不知人矣,尋就戮。和議成,褫桐職,奪恤典,旋論棄市,以先死議免。

剛毅,字子良,滿洲鑲藍旗人。以筆帖式累遷刑部郎中。諳悉例案,承審浙江餘杭縣民婦葛畢氏案,獲平反,按律定擬,得旨嘉獎。出為廣東惠潮嘉道,遷江西按察使,調直隸;遷廣東布政使,調雲南。光緒十一年,擢山西巡撫。請設課吏館,手輯牧令須知諸書,分講習,詔飭行各省。治套外屯田,建分段、開渠、設官三策。明年,移撫江蘇。蘇患水祲,先後濬蘊藻河、吳淞江,以工代賑,民德之。調廣東。二十年,召授軍機大臣,補禮部侍郎。二十四年,以工部尚書協辦大學士,疏陳實倉廩,嚴保甲,罷不急官。二十五年,按事江南及廣東諸省。迭疏請籌長江防務,籌餉練兵,清理財政,及整頓地方一切事宜,詔皆飭行。

二十六年,拳亂作,命趙舒翹及剛毅馳往近畿一帶查辦解散,及還京覆命,而宣戰詔已先下矣。匪集都城,肆焚殺,時方稱義民,亡敢誰何。載漪等復疏言:「雪恥強國,在此一舉!」又盛推拳民忠勇,有神術,可用。太后愈信之,因命剛毅、載勛統之,比於官軍。然匪專殺自如,勿能問,且擾禁城,日焚劫不止。詔各軍營會拏正法,盡拆所設神壇,並諭責剛毅、董福祥親自開導,勒令解散,卒不能阻。各國聯軍入犯,兩宮西狩,剛毅扈行至太原。車駕欲之西安,又從。道遘疾,還至侯馬鎮,死。其後各國請懲禍首,以先死免議,追奪原官。

趙舒翹,字展如,陜西長安人。同治十三年進士,授刑部主事,遷員外郎。讞河南王樹汶獄,承旨研辨,獲平反,巡撫李鶴年以下譴謫有差。居刑曹十年,多所纂定,其議服制及婦女離異諸條,能傅古義,為時所誦。光緒十二年,以郎中出知安徽鳳陽府。皖北水祲,割俸助賑。課最,擢浙江溫處道,再遷布政使。二十年,擢江蘇巡撫。捕治太湖匪酋葉子春,餘黨股慄;復為籌善後策,弊風漸革。明年,改訂日本條約,牒請總署重民生,所言皆切中。是時朝廷矜慎庶獄,以舒翹諳律令,召為刑部左侍郎。二十四年,晉尚書,督辦礦務、鐵路。明年,命入總理各國事務衙門,充軍機大臣。

拳匪據涿州,舒翹被命馳往解散;匪眾堅請褫提督聶士成職,剛毅踵至,許之。匪既入京,攻使館。聯軍至,李秉衡兵敗,太后乃令王文韶與舒翹詣使館通殷勤,為議款計。文韶以老辭,舒翹曰:「臣望淺,不如文韶!」卒不往。旋隨扈至西安。聯軍索辦罪魁,乃褫職留任,尋改斬監候。次年,各國索益亟,西安士民集數百人為舒翹請命,上聞,賜自盡,命岑春煊監視。舒翹故不袒匪,又痛老母九十餘見此慘禍,頗自悔恨。初飲金,更飲以鴆,久之乃絕,其妻仰藥以殉。

啟秀,字穎之,庫雅拉氏,滿洲正白旗人。以孝聞。同治四年進士,選庶吉士,散館改刑部主事,累遷內閣學士。光緒五年,授工部右侍郎,調盛京刑部。吉林將軍銘安被彈劾,啟秀白其誣,轉戶部。論者以按銘安事多徇芘,攻甚力,命崇綺覆按,無左驗,免議。東省練新軍,倚餉京師,閻敬銘掌戶部,方規節帑,未應也。啟秀力言,始獲請,歲發四十萬濟之。二十年,拜理籓院尚書。中、日和議成,將換約,啟秀疏請:「條約宜緩發,先商諸各國,杜後患。」不報。敖漢王達木林達爾達克鑒朝陽覆轍,自請增練蒙軍。言者論其苛派蒙眾,謀不軌,啟秀為訟其冤。敖漢王雖奪扎薩克秩,而其子獲嗣,以故大得蒙眾心。充總管內務府大臣。二十四年,授禮部尚書,疏陳釐正文體,倡明聖學。命充軍機大臣兼總理各國事務衙門。

啟秀端謹有風操,為徐桐所賞。自政變後,桐最被盻遇,欲引參機務,乃舉啟秀自代。已而拳亂作,董福祥攻使館不下,啟秀薦五臺僧禦敵,頗附和之。逮兩宮狩西安,啟秀以母病弗克從。日本軍拘啟秀及徐承煜嚴守之,承煜,桐子也。朝旨褫職,而各國猶言罪魁不可縱。明年,正法命下,日軍官置酒為餞,席次,傳詔旨,啟秀神色自若,曰:「即此已邀聖恩矣!」肅衣冠赴菜市。啟秀宅近日本權領地,日官與語,當善芘其家,第曰:「厚意可感。」他無復言,遂就戮。

英年,字菊儕,姓何氏,隸內務府,為漢軍正白旗人。以貢生考取筆帖式,累遷郎中兼護軍參領。光緒中,歷奉宸苑卿、左翼總兵、正紅旗漢軍副都統、工部右侍郎,調戶部。拳匪亂作,以英年、載瀾副載勛、剛毅統之。載勛等出示,招致義民助攻使館,英年弗能阻,匪益橫,任意戕殺官民。聯軍既陷京師,兩宮幸西安,英年充行在查營大臣,旋授左都御史。行次猗氏,知縣玉寶供張不備,疏劾之。款成,各使議懲首禍,英年褫職論斬,羈西安獄,尋賜自盡。

裕祿,字壽山,喜塔臘氏,滿洲正白旗人,湖北巡撫崇綸子。以刑部筆帖式歷官郎中。出為熱河兵備道,累遷安徽布政使。同治十三年,擢巡撫,年甫逾三十。前江南提督李世忠本降寇,罷職家居,所為橫恣,裕祿疏請誅之。會以事詣安慶,召飲署中,酒行,出密旨,麾眾縛斬之,而仍恤其家,人以是高其能。光緒十三年,遷湖廣總督,調兩江,復還鄂。廷議修鐵路,起盧溝訖漢口,下群臣議,裕祿力陳不可,忤旨。十五年,徙為盛京將軍。十七年,熱河奸民騷動,毀教堂,殺蒙人,裕祿會師朝陽,擊平之,予優敘。二十年秋,朝鮮亂起,奉天戒嚴,坐安東、鳳凰失守,數被議。明年,調福州,改授四川總督。二十四年,召為軍機大臣、禮部尚書兼總理各國事務衙門。會徵榮祿入樞廷,遂代之督直隸。

義和拳起山東,入直境。初,義和會源出八卦教乾坎二系,聚黨直、魯間,為臨清郜生文餘孽,後稱團,專仇教。裕祿初頗持正論,主剿,捕其酋姚洛奇置之法。逾歲,開州傳舉烽,言匪復至,擒渠率斬以徇。居無何,毓賢撫山東,縱匪,匪散入河間、深、冀,而裕祿承風指,忽主撫。袁世凱方將武衛軍,語裕祿:「盍不請嚴旨捕治?」裕祿曰:「拳民無他伎,緩則自消,激則生變。且此委巢事,何煩瀆天聽邪?」已而毓賢去,世凱代之,自興兵疾擊,以故匪不敢近山東,而紛紛入畿疆矣。吳橋知縣勞乃宣禁傳習,為書上裕祿,格不行。

時直隸官吏多信拳,布政使廷傑獨力主剿辦,嚴定州縣查緝拳匪懲戒辦法。遽奉詔開缺回京,匪愈橫。張德成居獨流,稱「舉國第一壇」,曹福田為津匪魁,二人者炫神術,為妄妖言相煽誘,裕祿不之問。已,復致書請餉二十萬,自任滅外人,裕祿馳檄召之,於是二人出入節署,與裕祿抗禮。當是時,津城拳匪至可三萬人,呼嘯周衢市,又以紅燈照熒眾,每入夜,家家懸紅燈,謂「迎仙姑」。

頃之,各國兵艦大集,匪猶群聚督轅求槍砲,裕祿命詣軍械所任自擇,盡攫以去。而聯軍絡繹登岸,索大沽砲臺,裕祿懼,疏告急,請敕董福祥來援。聯軍索益堅,提督羅榮光不允,戰失利,而裕祿且上天津團民殺敵狀,於是朝廷以團民為可恃,宣戰詔書遂下,而不知大沽已先數日失矣。裕祿又報大捷,盛張拳匪功,發帑金十萬犒團,更薦德成、福田於朝,飾戰狀,獲賞頭品秩、花翎、黃馬褂。事急,官軍戰車站,敗績,裕祿退保北倉。閱三日,城陷,德成、福田挾貲走,卒系而罪之。裕祿飛章自劾,詔革職留任。逾月,北倉失,裕祿又退楊村,遂自殺。和議成,奪職。

廷雍,字邵民,滿洲正紅旗人。以貢生累官直隸布政使。裕祿死,護總督。聯軍入保定,被執,並及諸士紳。各軍訊其事,雍曰:「保紳夙從令,可釋,事皆由我。今至此,斧鉞由汝,奚問為?」遂見殺。郡人尚多哀之。

毓賢,字佐臣,內務府正黃旗漢軍。監生。以同知納貲為山東知府。光緒十四年,署曹州,善治盜,不憚斬戮。以巡撫張曜奏薦,得實授,累遷按察使,權布政使。二十四年,調補湖南,署江寧將軍。裁革陋規萬餘兩,上聞而嘉之。

是時李秉衡撫山東,適有大刀會仇西教,秉衡獎借之,戕德國二教士。廷議以毓賢官魯久,諳河務,擢代之。既蒞事,護大刀會尤力。匪首硃紅燈構亂,倡言滅教。毓賢令知府盧昌詒按問,匪擊殺官軍數十人,自稱義和拳。毓賢為更名曰「團」,團建旗幟,皆署「毓」字。教士乞保護,置勿問。匪浸熾,法使詰總署,乃徵還。至則謁端王載漪、莊王載勛、大學士剛毅,盛言拳民忠勇得神助。俄拜山西巡撫之命,於是拳術漸被山西。平陽府縣上書言匪事,毓賢痛斥之,匪益熾。毓賢更命制鋼刀數百,賜拳童令演習,其酋出入撫署,款若上賓。

居無何,朝旨申命保教民,毓賢陽遵旨,行下各縣文書稠疊,教士咸感悅。未幾,又命傳致教士駐省城,曰:「縣中兵力薄,防疏失也。」教士先後至者七十餘人,乃扃聚一室,衛以兵,時致蔬果。一日,毓賢忽冠服拜母,泣不可止,曰:「男勤國事,不復能顧身家矣!」問之不語。遽出,坐堂皇,呼七十餘人者至,令自承悔教,教民不肯承,乃悉率出斬之,婦孺就死,呼號聲不忍聞。

聯軍既陷天津,毓賢請勤王,未及行,朝旨趣之再。兩宮已西幸,毓賢遇諸塗,遂隨扈行。和議成,聯軍指索罪魁,中外大臣復交章論劾,始褫職,戍新疆。十二月,行抵甘肅,而正法命下。時李廷簫權甘督。

廷簫,籍湖北黃安。以進士累官山西布政使,嘗附毓賢縱匪。至是得旨,持告毓賢,毓賢曰:「死,吾分也,如執事何?」廷簫慮譴及,元旦仰藥死。蘭州士民為毓賢呼冤,將集眾代請命,毓賢移書止之。其母留太原,年八十餘矣。一妾從行,令自裁。逾數日,伏誅未殊,連呼求速死,有僕助斷其頸,為斂而葬之。

論曰:戊戌政變後,廢立議起,患外人為梗,遂欲仇之,而庚子拳匪之亂乘機作矣。太后信其術,思倚以鋤敵而立威。王公貴人各為其私,群奉意旨不敢違,大亂遂成。及事敗,各國議懲首禍,徐桐等皆不能免。逢君之惡,孽由自作。然刑賞聽命於人,何以立國哉?


\end{pinyinscope}