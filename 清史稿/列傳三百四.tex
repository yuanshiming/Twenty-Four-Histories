\article{列傳三百四}

\begin{pinyinscope}
土司六

○甘肅

甘肅,明時屬於陜西。西番諸衛、河州、洮州、岷州、番族土官,明史歸西域傳,不入土司傳。實則指揮同知、宣慰司、土千戶、土百戶,皆予世襲,均土司也。清改甘肅為省,各土司仍其舊,有捍衛之勞,無悖叛之事。楊應琚曰:「按西寧土司計十六家,皆自明洪武時授以世職,安置於西、碾二屬。是時地廣人稀,城池左近水地,給民樹藝,邊遠旱地,賜各土司,各領所部耕牧。內惟土司陳子明系南人,元淮南右丞歸附,餘俱系蒙古及西域纏頭,或以元時舊職投誠,或率領所部歸命。李氏、祁氏、冶氏皆膺顯爵而建忠勛。迨至我朝,俱就招撫。孟總督喬芳請仍錫以原職世襲。今已百年,輸糧供役,與民無異。惟是生息蕃庶,所分田土多鬻民間,與民錯雜而居,聯姻而社,並有不習土語者。故土官易制,絕不類蜀、黔諸土司桀驁難馴也。」今寧郡外亦有土弁,合紀其始末為一卷。

狄道州:

脫鐵木兒,蒙古人。明初,授陜西平章宣慰使司都元帥,隨大將軍徐達招撫十八族鐵城、岷山等處,賜姓趙,更名安,授臨洮衛土官指揮同知。正統十年,卒,子英襲。傳至趙師範,清順治二年,底定隴右,師範率子樞勷歸附,仍令管理臨洮衛指揮使土司事務。同治元年,河回倡亂,趙壇領土兵防守州城。二年,壇赴洮州卓泥調撥鐵布番兵。適州城失守,敕書號紙均毀。四年,回匪圍鞏昌,壇赴陜甘大營請援,行至董家堡遇害。以兄子元銘為繼。光緒二十年,襲職,領兵部號紙。二十一年,河回復叛,渡河攻城,元銘率土兵五百由抹邦河進剿。至城南川,適統領威定軍何建威拔狄道,亦至,遂會軍抵河州。何以元銘勇,委帶威定前營,駐城南黃家灘。於邊家灣、三家集、羅神廟等處屢捷,解河州圍,加二品銜勇號。趙氏世居檜柏莊。

河州:

何貞南,河州人。元授陜西平章宣慰使司都元帥。明初,投誠,賜姓何,授河州衛土官指揮。傳至何永吉,清順治二年,歸附。五年,回變,其子揚威帶兵有功,請給號紙世襲。至乾隆年,趙武襲。撒回叛亂,武同子大臣在老鴉、南岔等關防禦。四十九年,石峰堡之變,父子防禦盡職。嘉慶四年,教匪由川入甘,時武患病,委子大臣在南界景古城瞎歌灘防堵。同治二年,武玄孫何柄繼。兵火倏起,守城有勞,復獲渠魁李法正,賞戴花翎。光緒四年,襲職。

韓哈麻,元、明時,授河州衛土司。清初,歸附。乾隆十四年,河州發給土千戶委牌,子霆襲。四十六年,撒回猖獗,統兵固守。旋因修蓋佛寺,違禁斥革。繼鹽茶回變,防禦有功,總督福康安給土司外委劄付。霆曾孫鈞,同治初,與賊接仗陣亡。子廷俊。同治十年,禦賊八峴山口,身先士卒,刀石弗避,左宗棠賞給養傷銀兩。又有韓完卜者,世襲指揮使。清初,歸附。其後韓千貫以劄印遺失,授為外委土司。雍正間,韓世公因逆夷跳梁,把隘無失,仍授指揮使。雯卒,子成璘襲。乾隆四十六年,陣亡。咸豐十一年,韓廷佐襲。韓氏世居韓家集。

岷州:

馬紀,自云伏波將軍後裔。元至正間,因防守哈達川九族,授指揮使職,家岷州衛。子珍,明洪武間,以功授世襲土官百戶。清順治二年,馬國棟歸附,授原職。馬氏世居宕昌城。

後成,明鎮守指揮能之季子,景泰間,守禦洮州;成子璋,成化間,征烏斯藏有功,授世襲土官百戶。清初,後承慶內附,為外委百戶。康熙三十年,劄委任事。乾隆九年,永慶孫發葵始實授土百戶。後氏世居攢都溝。

趙黨只管卜,岷州衛人。明洪武間,授世襲土官百戶。清初,趙應臣內附,為外委土官。康熙二十一年,授其子之鼎原職。趙氏世居麻霙里。

以上三土司,所轄雖號土民,與漢民無殊,錢糧命盜重案,俱歸州治,土司不過理尋常詞訟而已。

後祥古子,岷州衛人。明洪武二十八年,以功授世襲土官百戶。清順治間,後希魁歸附,授外委百戶。希魁曾孫榮昌,實授土百戶。光緒初,後振興改襲土把總。後氏世居閭井東。

綽思覺,革那族生番也。明宣德間,授土官副千戶。傳至宏基,順治十六年,歸附,因事革配。康熙十四年,其堂弟宏元於吳逆之變,恢復洮、岷有功,靖逆侯張勇題敘,仍授世襲副千戶。二十九年,宏元子廷賢,雍正初,與黃番煽亂,改土歸流。

洮州:

此夕的,洮州衛卓泥族番人。明永樂二年,率疊番、達拉等族投誠。十六年,授土官指揮僉事。正德間,玄孫旺秀調京引見,賜姓名楊洪。傳至楊朝樑,於順治十八年歸附,仍給劄管理土務,為外委土司。康熙十四年,吳三桂亂,助餉,授拜他喇布勒哈番,準襲二次。二十年,朝樑子威襲。四十五年,威子汝松襲。汝松子沖霄,仍襲指揮僉事。五十一年,黑番為亂,助剿有功。前山十八族、後山十九族黑番,俱給令管轄。曾孫宗業襲職。撒拉回變,以功賞三品頂戴花翎。四十九年,鹽茶回變,兩剿石峰堡,賞大緞二疋。嘉慶十九年,宗業弟宗基襲,兼攝禪定寺僧綱。宗基子元,道光二十四年襲。同治中,奉總督左宗棠檄,剿循屬撒匪,收復洮州新舊二城,歷獎至頭品頂戴、志勇巴圖魯。光緒六年,子作霖襲職,亦以軍功得頭品頂戴,領兵部號紙,兼攝護國禪師。日益言誇大,小弱者割地以鬻,遂並有眾土司地。作霖曾孫積慶,光緒二十八年襲。楊氏世居卓泥堡,地最大,南至階文,西至四川松潘界,土司中最強者,自以為楊業之裔。明正德賜姓之事,則已茫如矣。

昝南秀節,洮州衛底古族西番頭目。明洪武十一年,率部落投誠。十二年,督修洮州邊壕城池。十九年,隨指揮馬煜征疊州,以功授本衛世襲中千戶所百戶。子卜爾結,於洪武二十年襲。二十五年,同指揮李凱等招撫番、夷等,認納茶馬。永樂三年,賜姓昝。宣德五年,以護送侯顯功,升本衛實授百戶。傳至昝承福,清順治十年,歸附。奉洮州衛軍民指揮使司劄付,昝天錫於光緒二十年承襲。昝氏居資卜族。

永魯劄剌肖,洮州衛著遜族番人,明永樂間,以功授土官百戶。傳至永子新,清順治間,歸附,襲職。永隆於光緒二十五年承襲。永氏居著遜隘口。

西寧縣:

祁貢哥星吉,元裔。初封金紫萬戶侯,世守西土。洪武元年,歸附。五年,招撫西番,授副千戶。以追剿西番亦林真卉陣亡,子鎖南襲。永樂十年,從西寧侯宋琥追捕番酋老的罕等於討來川,予正千戶。傳至祁廷諫,襲職。崇禎十六年,闖寇賀錦擾西寧,廷諫率子興周與戰,斬錦。已而賊黨愈熾,並被俘送西安。清順治二年,英親王阿濟格至西安,破走逆闖,得廷諫,賞衣帽、鞍馬、採緞、銀兩,令回西寧安撫番族,仍授本衛指揮使,世襲。十年,病休。興周先以戰功授大靖營參將,至是襲職。會吳逆叛,興周子荊璞隨總兵王進寶克復蘭州、臨鞏諸城。同治元年,撒回復亂,祁敘古防堵有功。十一年,為土番拉莫丹所控,革職。

母李氏代理指揮使印。光緒十五年,以巡防功復職。祁氏世居寄彥才溝。

陳義,江蘇山陽人。父子明,元淮安右丞。至正二十三年,明常遇春兵至淮南,率眾投誠。洪武七年,隨李文忠北伐有功,授隨征指揮僉事。十六年,從征陣亡。義襲父職,調任燕山右護衛。靖難兵起,從燕王轉戰,升山西潞州衛指揮同知。永樂元年,隨新城侯張輔征甘、涼。旋扈成祖征木雅失里,逐北至紅山口,遷指揮使。又從耿炳文駐防甘肅,授西寧衛世襲指揮使。崇禎初,陳師堯隨洪承疇守松山,陣亡。清順治二年,陜西總督孟喬芳收甘肅,師堯弟師文歸附。五年,甘州回米喇印、丁國棟反,隨鎮羌參將魯典戰賊烏稍嶺,仍襲西寧衛指揮使。同治元年,撒回作亂,總督沈兆霖率師進剿,檄陳興恩守思觀。光緒四年,子迎春襲。陳氏世居陳家臺。

李文,西番人。父賞哥,元都督指揮同知。明洪武初,投誠。傳至李洪遠,襲指揮同知職。崇禎十六年,李自成黨陷甘州,獨西寧不下。賊將辛恩貴攻破之,洪遠與其妻祁氏暨家丁一百二十人死於難。清順治七年,洪遠子珍品歸附,仍與原官。咸豐八年,李爾昌襲。同治元年,撒拉回作亂,隨大軍進剿,賞藍翎。李氏世居乞塔城。

納沙密,西番人。明洪武四年,投誠,授總旗。清順治二年,納元標歸附,仍襲指揮僉事。同治元年,總督沈兆霖督軍進討撒回,納朝珍奉檄守南川什張加。光緒四年,朝珍子延年襲。納氏世居納家莊。

南木哥,姓汪氏,西寧州土人。明洪武四年,投誠。累除金吾左衛中衛所副千戶,加指揮僉事。傳至汪升龍,清順治二年,歸附,仍襲指揮僉事。同治元年,撒回反,南進善隨大軍前赴巴燕戎格所屬曲林莊防剿。二年,西寧逆回悉叛,奉檄守府城。十一年,回亂平,招集流亡土民復業。光緒十九年,子祖述襲。汪氏世居海子溝。

吉保,西番人。洪武四年,投誠,授百戶。二十三年,調錦衣衛前所鎮撫。子朵爾只襲。清順治二年,吉天錫歸附。十二年,仍襲指揮僉事。吉氏世居迭溝。

循化:

韓寶元,撒拉爾回人。明洪武三年,投誠,授世襲昭信校尉管軍百戶職銜。傳至韓愈昌,清康熙間,歸附,蒙靖寧將軍張劄委都司職銜。子炳,撫番有功,於雍正間奉兵部號紙,襲土千戶,管西鄉上四工韓姓撒拉。

韓沙班,明時,撫番有功,授世襲撒拉族土百戶。清順治間,歸附,管東鄉下四工馬姓撒拉。藏土百戶王國柱,清順治二年,歸附,授原職,管番民。明時防戍小土司也。

大通縣:

曹通溫布,大通川人。乾隆元年,以功補大通川土千戶,世襲。每年應納貢馬二十四匹,共折銀一百七十三兩。後因回亂,番民逃亡,總督左宗棠咨部,暫以半價交納。由大通縣管理。

碾伯縣:

朵爾只失結,蒙古人。元甘肅行省右丞。明洪武四年,投誠。六年,授西寧衛指揮僉事。子端竹襲。旋調守西寧衛。建文元年,從南軍征北平,陣亡。子祁震襲,始以祁為氏。祁秉忠,明史有傳。秉忠侄國屏,襲都指揮同知。崇禎十六年,流寇蹂西寧,力抗之。清順治二年,歸附。五年,甘州回陷甘、涼、肅諸州,國屏隨總督孟喬芳進剿,復甘州。九年,授西寧衛世襲指揮同知。子伯豸襲。吳三桂反,平涼提督王輔臣叛應之。逆黨陷鞏昌、臨洮、蘭州,伯豸統各土司隨西寧鎮總兵王進寶東征,平蘭州,累官至鑾輿使。聖祖親征噶爾丹、仲豸扈從,擢署溫州鎮總兵,回籍以原官署理指揮同知印務。雍正元年,青海酋羅卜藏丹津叛,大將軍年羹堯檄祁在璿守大峽口。撒拉陷河州,璿侄調元率土兵守碾伯城。鹽茶回田五作亂,調元守魯班峽。同治元年,撒回作亂,調元曾孫承誥協同防禦。以勞疾卒,承誥妻劉氏護理印務。光緒十一年,子貴玉襲。祁氏世居勝番溝。

李南哥,西番人。自云李克用裔。元西寧州同知。明洪武初,投誠,授指揮僉事世襲。招撫流散,收捕黑章砸等處番賊。永樂五年,卒,子英襲。獲番酋老的罕,進都指揮僉事。二十二年,中官鄧成等使西域,道安定、曲先,遇賊見殺,掠所齎金幣。仁宗初立,諭赤斤、罕東及安定、曲先詰賊主名,而敕英與指揮康壽等進討。英言知安定指揮哈三孫散哥、曲先指揮散即思實殺使者,遂率兵西入。賊驚走,追擊,逾昆侖山,深入數百里。至雅令闊,與安定賊遇,大敗之,俘斬千一百餘人,獲馬牛雜畜十四萬。曲先賊聞風遠遁,安定王桑爾加失夾等懼,詣闕謝罪。宣宗嘉英功,遣使褒諭宴勞之,令馳驛入朝。既至,擢右府左都督。宣德二年,封會寧伯,祿千一百石,並贈南哥子爵。英恃功驕,所為多不法。寧夏總兵史昭奏英有異志,英上章辯,賜敕慰諭之。英家西寧,招逋逃七百餘戶,置莊墾田,豪奪人產,復為兵部及言官所劾,追逃者入官。傳至李天俞,闖寇餘黨蹂湟中,天俞被執送西安,其家殉難者三百餘人。清順治二年,英親王阿濟格至關中,流寇潰散,天俞謁王,王賜衣冠、鞍馬、銀兩、彩緞,令回西寧招撫番族。五年,甘州回米喇印反。十年,授西寧衛指揮同知,世襲。吳三桂黨陷蘭州,總兵王進寶檄其子澍從征。澍與弟洽預調水夫五百餘名,各造木筏五十餘隻,由新城河口宵濟官軍,並率土兵千餘騎繼進,遂復蘭州、臨鞏諸城,擢游擊。傳至李長年,光緒四年,襲職。李氏世居上川口。

趙朵爾,岷州人。元招藏萬戶。明洪武三年,投誠。傳至趙瑜,清順治二年,歸附。十八年,仍襲指揮同知。同治初,撒回不靖,總督沈兆霖進剿,檄趙永齡率土兵隨官軍搜剿山後巴燕戎格等處逆黨。光緒七年,永齡襲職。趙氏世居趙家灣。

失剌,蒙古人。元甘肅省郎中。明洪武初,投誠,選充小旗。子阿吉襲小旗,始以阿為氏。扈成祖北征阿魯臺,戰魁列兒河有功,遷總旗。傳至阿鎮,清順治二年,歸附,依舊世襲。同治四年,逆回陷老鴉堡,阿文選率土兵禦賊於隘,眾寡不敵,死之,部下熸焉。光緒九年,文選子保衡襲。二十年,保衡子成棟襲。阿氏世居老鴉白崖子。

帖木錄,西寧衛土人。元,百戶。洪武四年,投誠,授原職。子大都,從都督宋晟討西番叛賊,獲捷遷千戶。永樂七年,卒,子甘肅襲職,始以甘為氏。崇禎十六年,流寇擾西寧,甘繼祖家被掠,失承襲號紙。清順治二年,歸附。吳三桂逆黨延及隴右,繼祖子廷建率土兵三百守黃河渡口,復隨王進寶征討,隴右以安。敘功,襲指揮僉事原職。甘鍾英,光緒四年襲。甘氏世居美都川。

鐵木,西寧州土人。明洪武四年,投誠,充小旗。子金剛保,從成祖北征,追木雅失里不及,移征阿魯臺,連戰於玄冥河、於靜慮鎮、於廣漢戎,皆有功。復從指揮李英討番酋老的罕於沙金城,大破之。二十年,再扈成祖北征,敗賊於魁列兒河,擢千戶。子硃榮襲職,始以硃為氏。從都指揮李英追安定賊,與戰,深入,歿於陣。數傳至硃秉權,值明末流寇賀錦之亂,失官誥號紙。清順治二年,秉權偕子廷璋歸附。康熙四十年,仍授指揮僉事,世襲。數傳至硃協,同治四年,湟中群回肆逆,協殉難。光緒十一年,協子廷佐襲。硃氏世居硃家堡。

薛都爾丁,西域纏頭回人。元,甘肅省僉事。明洪武四年,投誠,授小旗。子也裏只補役,洪熙元年,從征安定賊有功,擢所鎮撫。子也陜舍襲。陜舍孫祥,更姓冶氏。順治二年,冶鼎歸附,仍予世襲。冶氏世居米拉溝。

李化鰲,明世襲西寧衛指揮同知化龍之弟,錦衣衛指揮使光先之次子。清順治二年,歸附,授職百戶。光緒十五年,李長庚襲。李氏世居九家巷。

朵力,西寧州土人。明洪武四年,投充小旗。子七十狗補役。孫辛莊奴,始以辛為氏。清順治二年,辛偉鼎歸附,仍授試百戶職。同治四年,回亂湟中,堡塞俱毀,辛德成挈其子裕後避賊居藏地。光緒十二年,歸里。裕後襲。辛氏世居王家堡。

哈喇反,西寧州土人。明洪武四年,投充小旗。子薛帖裏加替役,以功授百戶。子喇苦襲,以功升副千戶,遂以喇為氏。清順治二年,喇光耀歸附,給與指揮僉事劄付。喇氏世居喇家莊。

平番縣:

鞏卜失加,元裔。父脫歡,封武定王,兼平章政事。明洪武四年,率諸子部落投誠,太祖授鞏卜失加為百夫長,俾統所部居莊浪,以功升百戶。永樂初,殉阿魯臺之難,傳子失加,累署莊浪衛指揮同知,賜姓魯氏。子鑒,鑒子麟,麟子經,三世名將,明史有傳。崇禎十年,以經曾孫印昌任西寧副總兵。及闖寇犯河西,印昌散家財享士卒,提兵至西大通,遇賊黨賀錦,揮兵奮戰,部卒殆盡,遂歿於陣。清順治十六年,印昌子宏歸附,襲指揮使,錫之敕印。宏卒,嫡子帝臣幼,以族人魯大誥代理土務。會吳逆叛,宏妻汪氏捐軍糧四百石。宏曾孫璠,乾隆四十六年,撒拉回攻圍蘭州,率土番兵三百人赴援,戰於亂古堆坪。賊悍甚,兵無後繼,璠負重傷,裹創力戰,竟突圍歸營。事聞,加一等職銜、花翎。鹽茶回復反,璠領土番兵防守蘭州城。道光六年,逆回張格爾犯邊,揚威將軍長齡進討,璠子紀勛奉檄購辦駝只、運軍糧。九年,官兵進剿安集延,仍承辦駝只。紀勛娶額駙阿拉善親王女,緣此習尚奢豪,盛極而衰。嫡孫如皋襲。咸豐初,如皋助軍餉。七年,省城修建錢局,捐本管山場木植數萬株,加二品頂戴、花翎。同治初,回亂,以功加副將銜。十三年,西寧肅清,加提督銜、譽勇巴圖魯。光緒十九年,如皋卒,子燾幼,母和碩特氏護土務。二十一年四月,燾嗣職。魯氏自燾以上,世襲掌印土司指揮使,駐扎莊浪,分守連城。

把只罕,元武定王平章政事長男。明洪武四年,隨父來降,授指揮僉事,後賜姓魯氏。數傳至魯典,清順治二年,歸附。陜西總督孟喬芳嘉其功,委署鎮海營參將,隨大軍征剿。數傳至魯緒周。同治三年,回變,緒周率所部御賊,陣亡,子熹襲職。光緒十一年,子服西襲職。自服西以上,世襲掌印土司指揮僉事。

魯鏞,元裔,與魯鑒同族。明時,以官舍隨征,授總旗。清順治二年,魯大誥隨魯希聖等歸附,仍授前職。光緒十九年,魯瞻泰襲。自泰以上,世居古城,襲土指揮使。

魯之鼎,與魯典同族。明時,世襲土指揮副使。清順治二年,隨典歸附。光緒十八年,魯維禮襲職。自維禮以上,世居大營灣,襲土指揮副使。

魯福,魯鑒次子。從鑒征討,屢立戰功。清順治二年,魯培祚隨魯典歸附。光緒十七年,魯應選襲職。世居西大通峽口,襲土指揮同知。

魯國英,元裔。明正千戶。清順治二年,魯大誠投誠,隨魯典剿甘、涼回逆,力戰陣亡。子景成,仍襲正千戶世職。光緒五年,魯福山襲。世居古城。

魯三奇,元裔。明世襲副千戶。清順治二年,三奇隨同族魯典歸附。光緒十六年。魯政襲職。世居馬軍堡。

西坪土官楊茂才,明正百戶。清順治二年,隨魯典投誠。數傳至楊得榮。同治中,逆回叛,得榮避難,不知所終。

西六渠土官何倫,明時,充小旗。清順治二年,何進功隨魯典歸附。數傳至何萬全。同治四年,捍禦逆回,創重而卒。子臣福襲。

楊國棟,明指揮同知。清順治二年,歸附。九年,復襲指揮同知。後無考。

魯察伯,明實授百戶。清初,歸附。康熙十六年,子魯襄,仍襲實授百戶。後無考。

海世臣,明指揮僉事。世臣子龍襲前職。清順治二年,海洪舟歸附。九年,仍襲指揮僉事。後無考。


\end{pinyinscope}