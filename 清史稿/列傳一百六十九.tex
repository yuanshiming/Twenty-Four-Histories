\article{列傳一百六十九}

\begin{pinyinscope}
瑚松額布彥泰薩迎阿

瑚松額,巴岳忒氏,滿洲正黃旗人,西安駐防。嘉慶初,以前鋒從將軍恆瑞剿湖北教匪,後隸那彥成、德楞泰部下,積功擢協領。十八年,滑縣教匪起,瑚松額率馬隊從副都統富僧德戰道口及滑縣城下,屢有功,賜花翎。二十三年,擢福州副都統,署福州將軍。

道光三年,授察哈爾都統。五年,擢成都將軍。乾隆中,西寧玉舒巴彥囊謙千戶分三百戶與其弟索諾木旺爾吉為小囊謙,由德爾格忒土司居間調處,辦事大臣斷定。既而索諾木旺爾吉之子諾爾布不能服其屬戶,大囊謙欲兼並之,諾爾布訴於德爾格忒土司;大囊謙復以土司有欺凌小囊謙情事,互控不已,下瑚松額按之。奏請仍遵原斷,大囊謙不得覬覦屬戶,德爾格忒土司亦毋預鄰封事,以杜爭端,事乃定。七年,署四川總督。九年,調吉林將軍。會宣宗東巡,扈蹕,校射,中三矢,賜黃馬褂。十年,母憂回旗。尋署盛京將軍。

十二年,命偕尚書禧恩督師剿湖南瑤匪趙金龍,至則金龍已就戮,其黨趙青仔率餘匪竄廣東、湖北境,督兵剿平之。廣東連山排瑤亦叛,率提督餘步雲等進剿,擒匪首鄧三、盤文理等,瑤眾投誠,全境肅清,賜雙眼花翎,予一等輕車都尉世職。命署福州將軍,臺灣土匪張丙等作亂,授為欽差大臣,偕參贊哈哴阿赴剿。及抵福建,提督馬濟勝已擒匪首,臺灣略定。十三年春,命仍渡臺搜捕餘黨,擒各路匪首二十餘人,賊黨三百餘人,分別置之法,械送張丙、陳辦、詹通、陳連至京誅之,加太子太保,復調成都將軍。十四年,瓘邊、馬邊夷匪勾結焚掠,提督楊芳擊斃夷目,以肅清入奏。既而夷復滋擾,瑚松額以芳辦理未善,劾罷之,自請議處,降一級留任。

十五年,授陜甘總督。疏陳兵丁驕縱,應加意訓練駕馭;又密陳吏治情形,優詔嘉納。十七年,京察,詔嘉其不露鋒鋩,細心任事,予議敘。西藏堪布入貢,為四川番匪劫掠。瑚松額捕賊數十人,得贓物;奏請貢道改由柴達木,由青海大臣遺兵護送。又以野馬川地連野番,請於大通河北岸立柵,山巖築設墩卡,派兵防守;提標前後二營廠馬合並,以厚兵力:並允行。二十一年,因病請開缺,尋致仕,許食全俸。二十七年,卒,贈太子太傅,賜恤,謚果毅。

布彥泰,顏扎氏,滿洲正黃旗人。父珠爾杭阿,嘉慶初,官鑲黃旗滿洲副都統,以軍功予騎都尉世職。布彥泰由廕生授藍翎侍衛,襲世職,洊升二等侍衛。二十三年,充伊犁領隊大臣。道光初,擢頭等侍衛。歷喀什噶爾參贊大臣、辦事大臣,伊犁領隊大臣,烏什辦事大臣。九年,授喀什噶爾總兵,病歸。十年,予副都統銜、乾清門行走,充哈密辦事大臣,調西寧辦事大臣。將軍玉麟薦其習邊事,調伊犁參贊大臣,再調塔爾巴哈臺參贊大臣。十四年,復以病歸。十八年,署正藍旗漢軍副都統,擢察哈爾都統。

二十年,授伊犁將軍,入覲,命在御前行走。及赴任,授鑲黃旗蒙古都統。二十二年,疏陳開墾事宜,略言:「惠遠城三棵樹地方可墾地三萬餘畝,請就本地民戶承種輸糧。阿勒卜斯地方可墾十七萬餘畝,請責成阿奇木伯克等籌計戶口,酌量勻撥。」至二十四年,疏報塔什圖畢等處開墾疊著成效,詔嘉其「忠誠為國,督率有方」,加太子太保。又命會勘烏魯木齊未墾之地,及各城曠地,一律興辦。尋疏言:「惠遠城東阿齊烏蘇廢地,前任將軍松筠奏撥八旗餘丁耕種,因乏水,不久廢業。今欲墾復,必逐漸開渠,極東且須引哈什河水,方可用之不竭。經營浩費,較前次各案不啻數倍。現委員勘估,又以伊犁歷屆捐墾成案,皆系收工而非收銀。蓋辦工以工為主,計銀不如計工之直捷,亦不如計工之覈實。此次用夫匠五十三萬四千工,實墾得地三棵樹、紅柳灣三萬三千三百五十畝,阿勒卜斯十六萬一千餘畝。荒地之開墾成田,由於渠工之開通水利,故不能劃出某頃某畝為某員所捐辦者,仍請免其造冊報銷。」從之。時前兩廣總督林則徐在戍所,布彥泰於墾事一以諮之,阿齊烏蘇即由則徐捐辦。事既上聞,命布彥泰傳諭則徐赴南路阿克蘇、烏什、和闐周勘。布彥泰疏留喀喇沙爾辦事大臣全慶暫緩更換,與則徐會勘。凡歷兩年,得田六十餘萬畝。事具全慶傳。

二十五年,授陜甘總督。青海番匪連年肆擾,自二十三年總督富呢揚阿奏報進剿,驅回河南,實僅邀番僧賚撫,約不北犯。次年,復擾河北,掠涼州營馬匹,戕守備。富呢揚阿諉稱匪乃四川果克黑番,大雪封山難剿,而西寧鎮總兵慶和出口會哨,又遇賊被戕。惠吉繼任總督。檄提督胡超進剿。肅州兵不聽調,譁噪,胡超不能制。惠吉籌辦未有緒,歿於任,乃以布彥泰代之,未至,命林則徐先署總督,並授達洪阿西寧辦事大臣,同治其事。二十六年,布彥泰抵任,奏劾胡超畏葸,罷之;又論總兵站住攻剿不力,褫職遣戍。達洪阿率兵剿平番莊,惟黑錯寺匪眾抗拒,攻下之。又破果岔賊巢,拉布楞等寺僧收合四溝散番乞降,事乃定。布彥泰以調度有方,被優敘。親巡邊隘,疏陳西寧地勢因河為固,扼險設備,請於哈拉庫圖爾之南山根、南川營之青石坡,移建營堡,黃河北岸頭岱、東信、忙多各渡口設卡;又奏復防河舊章,安置營汛:並如議行。

二十七年,安集延布魯特糾合回子圍喀什噶爾、英吉沙爾,詔布彥泰率兵赴肅州,授為定西將軍,奕山為參贊大臣,將大舉出師。會奕山率邊兵戰捷,賊退,二城解圍,軍事告竣,布彥泰回任。二十九年,因病請罷,許之。時為固原知州徐採饒等所訐,命協辦大學士祁俊藻往會總督琦善按之,坐關防不密、清查歧誤,及失察家人,議降調革任。尋予二等侍衛,充葉爾羌幫辦大臣,調伊犁參贊大臣,偕將軍奕山會議俄羅斯通商事宜,語詳奕山傳。咸豐二年,授正白旗漢軍副都統,仍留邊任。四年,回京,命赴王慶坨軍營,以疾未行,請開缺。光緒六年,卒,年九十。詔念前勞,依都統例賜恤。

薩迎阿,字湘林,鈕祜祿氏,滿洲鑲黃旗人。嘉慶十三年舉人,授兵部筆帖式。擢禮部主事,洊升郎中。道光三年,出為湖南永州知府,調長沙。歷山東兗沂曹道、甘肅蘭州道。七年,就遷按察使。以治回疆軍需,賜花翎。六年,擢河南布政使,未任,予副都統銜,充哈密辦事大臣。調喀喇沙爾辦事大臣。十年,安集延擾喀什噶爾邊卡,薩迎阿赴土爾扈特、霍碩特召兵赴援,又襄治南路糧運。授盛京工部侍郎,兼管奉天府尹事。十一年,留京署鑲白旗漢軍副都統,充烏什辦事大臣。歷哈密辦事大臣、葉爾羌幫辦大臣,仍調哈密辦事大臣。十五年,授盛京禮部侍郎,兼管府尹事,調戶部。二十年,召授禮部侍郎,兼鑲紅旗漢軍副都統,調戶部,兼管錢法堂。二十三年,擢熱河都統。

二十五年,授伊犁將軍。烏魯木齊興辦喀喇沙爾渠道堤壩,下薩迎阿籌議。疏言:「喀喇沙爾城西開都河,道光十七年,築護堤,有屯田頭工、二工兩渠,自裁屯安戶後,又於上游大河開一大渠,嗣頭二工又各添新渠,共有五渠。上年大水,各渠口沖塌,護堤亦壞。今擬挑濬北大渠,接長二千三百丈,共長九千丈;修築龍口石工,外設木閘,自龍口至坡心灘嘴,築碎石長壩四十餘丈,中設洩水閘,隨時啟閉;接長舊堤三十餘里,至北大渠口為止;其餘諸渠挑濬深通,庶期經久。」又言:「吐魯番掘井取泉,由地中連環導引,澆灌高田,以備渠水所不及,名曰閘井,舊有三十餘處。現因伊拉里克戶民無力,飭屬捐錢籌辦,可得六十餘處,共成百處。」尋以開墾挑渠辦有成效,薩迎阿履勘,籌議招種升科。疏言:「墾地在渠水充盈,用有餘裕,升科不必求急,期實有裨益,行之久長。新疆水利,泉水少而雪水多,雪水之遲早無定,收穫之豐歉難齊,請援鎮、迪舊例,減半升科。」下部議行。英吉沙爾領隊大臣齊清額誤聽伯克言,誣指回子胡完為張格爾逆裔,薩迎阿平反之,詔嘉其詳慎。

二十七年,安集延布魯特回眾入卡,圍喀什噶爾、英吉沙爾二城,薩迎阿檄調諸城兵往剿,葉爾羌參贊大臣奕山率諸軍由巴爾楚克進,三戰皆捷。薩迎阿別遣兵扼樹窩子,二城圍尋解。時方命陜甘總督布彥泰督師,未出關而事平。咸豐元年,召授正白旗滿洲都統,會陜甘總督琦善剿青海番匪,言官劾其妄殺,命薩迎阿赴西寧按之。奏調刑部司員梁照、奎椿、武汝清隨同鞫訊,得番子十四名無辜誣服狀,疏陳琦善剿辦黑城撒拉回子及黃喀窪番賊,尚非無故興師,惟將雍沙番族殺斃多名,實系妄加誅戮,並及文武妄拿、刑求逼供,詔褫琦善職,逮京訊治,命薩迎阿暫署陜甘總督。

甘肅營務廢弛,雖議整頓,而番匪時復出擾。新授福建巡撫王懿德途經金縣,士民呈控,奏下薩迎阿察治,屢被詰責。二年,解任回京。自琦善之逮治也,刑部尚書恆春以薩迎阿論劾過當,欲令原訊司員對簿,獨侍郎曾國籓持不可。及廷臣會訊讞上,琦善遣戍吉林,司道以下文武論罪有差,被誣番子免罪,略如原讞。薩迎阿坐未取應議各員供詞,遽行擬罪,又因子書紳與司員同坐問供,下部議,書紳降三級調用,薩迎阿降四級留任。歷署鑲藍旗、正紅旗蒙古都統。六年,出署西安將軍。逾歲卒,詔念回疆軍務曾著勞績,賜恤,謚恪僖。

論曰:瑚松額川、陜舊將,屢任專征,雖無赫赫功,尚持大體,晚膺疆寄,稱厥職焉。布彥泰新疆開墾,西寧平番,胥賴林則徐之擘畫。薩迎阿平反番獄,持正不阿,而治番亦無良策。蓋番族生計無資。營伍廢弛已久,議剿議撫,補苴一時。林則徐謂治番自古無一勞永逸之計,亦慨乎其言之也。


\end{pinyinscope}