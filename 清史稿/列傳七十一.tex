\article{列傳七十一}

\begin{pinyinscope}
覺羅滿保陳策施世驃藍廷珍從弟鼎元林亮何勉陳倫炯

歐陽凱羅萬倉游崇功

覺羅滿保,字鳧山,滿洲正黃旗人。康熙三十三年進士,選庶吉士,授檢討。累遷國子監祭酒,擢內閣學士,直經筵。

五十年,授福建巡撫。疏言福州、興化、泉、漳等屬十六州縣皆瀕海要地,請揀選直省卓異官除授。御史璩廷祜論其不可,部議以為然。詔下九卿等再議,卒從滿保言。五十四年,擢福建浙江總督,命巡海。議自乍浦至南澳,沿海五千餘里,建臺、寨百二十七所,砲位千一百七十有八。別疏言:「鹿耳門為臺灣咽喉,澎湖為廈門籓衛,安平鎮為水師三營重地,及海洋各口岸宜分極沖、次沖,築墩、臺,設汛巡守;並嚴察海船出入,禁漁船私載米糧、軍器。」又言:「淡水、雞籠山為臺灣北界,其澳港可泊巨艦百餘。更進為肩豆門,沃野百里,番社交據。請增置淡水營,設官駐防為後蔽。」皆報可。

六十年,鳳山民硃一貴為亂。臺灣知府王珍苛稅濫刑,鳳山民黃殿、李勇、吳外等集數百人謀變,一貴素販鴨,託明裔以為渠。劫岡山塘、檳榔林二汛,掠軍器,眾益聚,遂破縣城,進陷臺灣。總兵歐陽凱等率兵禦賊,師敗績,死之。臺廈道梁文煊等走澎湖。滿保疏聞,督兵趨廈門,值淫雨,乘竹兜從數騎行泥淖中。比至,籍丁壯剽悍能殺賊者悉充伍,嚴申軍令,禁舟師毋登陸,民以不擾。淡水營守備陳策使詣廈門乞援,滿保移會巡撫呂猶龍,遣兵自閩安渡淡水。未幾,南澳鎮總兵藍廷珍率舟師至,滿保命統水陸軍,會提督施世驃於澎湖,剋期進剿。六月,世驃、廷珍攻鹿耳門,敗賊安平鎮,遂克臺灣。上以臺灣民附亂非本意,敕滿保招撫。尋諸羅民楊旭等密約壯丁六百人,擒一貴及其黨十二人,獻世驃軍前,檻送京師,磔於市。是役,自出師迄事平凡七日。上嘉滿保調度有方,加兵部尚書。尋疏言:「賊起,惟守備陳策鼓勵兵民,堅守汛地,待大兵進援,奮力效忠。」命擢臺灣總兵。復疏劾珍縱役需索,致一貴乘機倡亂;文煊及所屬官吏一無備御,退回澎湖,應奪官逮問,從之,文煊等論罪如律。秋,臺灣颶作,滿保以聞,諭:「臺灣有司平日貪殘激變,及大兵進剿,殺戮之氣上干天和,令速行賑恤。」

上杭民溫上貴往臺灣從一貴得偽元帥札、印,還上杭,煽鄉人從賊。聞一貴誅,走江西,結棚匪數百,謀掠萬載。知縣施昭庭集營汛剿捕,擒上貴及其黨十數人,並伏法。大學士白潢等條奏禁戢棚匪,滿保疏言:「閩、浙兩省棚民,以種麻靛、造紙、燒灰為業,良莠不一。令鄰坊保結,棚長若有容庇匪類,依律連坐。有司於農隙遍履各棚,嚴加稽察。浙江鄞、奉化等二十七縣,福建閩、龍巖等四十州縣,皆有棚民,宜如沿海州縣例,揀員題補。」詔從之。

雍正三年,卒官。遺疏言:「新任巡撫毛文銓未至,總督印信交福州將軍宜兆熊署理,並留解任巡撫黃國材暫緩起程,如舊辦事。」詔嘉其得體,下部議血阜;時尚書隆科多獲罪鞫訊,得滿保餽金交通狀,世宗諭責滿保諂隆科多、年羹堯,命毋賜恤予謚。

策,字鍾侯,福建晉江人。由銅山守備調淡水。一貴陷臺灣,策孤軍力守一隅。奸人苑景文入境煽誘,擒誅之。師下臺灣,滿保檄剿北路,復南嵌、竹塹、中港、後壟、吞霄、大甲諸社。以功擢臺灣總兵,加左都督。卒。

施世驃,字文秉,靖海侯瑯第六子。康熙二十二年,世驃年十五,從瑯下臺灣,委署守備。臺灣既定,以功加左都督銜,授山東濟南城守參將。三十五年,聖祖親征噶爾丹,天津總兵岳升龍薦世驃從軍。召試騎射,命護糧運至奎素,從大將軍馬斯喀追賊至巴顏烏闌。師還,假歸葬。上褒世驃勤勞,命事畢仍還任。累遷浙江定海總兵。四十二年,上南巡,賜御書「彰信敦禮」榜。時海中多盜,世驃屢出洋巡緝,先遣裨將假商船餌盜,擒獲甚眾,斬盜渠江侖。四十六年,上南巡,詢及擒斬海盜事,溫諭嘉獎,賜孔雀翎。四十七年,擢廣東提督。五十一年,調福建水師提督。

六十年,硃一貴為亂,陷臺灣。世驃聞報,即率所部進扼澎湖,總督滿保檄南澳總兵藍廷珍等以師會。眾議三路進攻。世驃謂南路打狗港在臺灣正南,南風盛,不可泊;北路清風隙去府百餘里,運餉艱;度賊必屯聚中路,宜直搗鹿耳門。時臺地諸將吏皆退次澎湖,惟淡水守備陳策堅守汛地。世驃遣游擊張駴等赴援,自統師出中路。選勁卒,乘小舟,載旗幟,分伏南北港。六月,抵鹿耳門。賊踞砲臺以拒。世驃登樓船督戰,發砲中敵貯火藥器,火大熾,賊驚潰。眾軍齊進,兩港悉樹我軍幟。賊不敢犯,揚帆直渡鯤身。鯤身者海沙也,水淺,大舟不能過。是日海水驟漲八尺餘,舟乘風疾上,遂克安平鎮。翌日,戰,破賊。賊悉眾來犯,世驃遣守備林亮等進西港,游擊硃文等越七鯤身,自鹽埕、大井頭分道登陸趨臺灣。世驃督將士指揮布陣擊賊,賊潰,遂復臺灣。一貴走諸羅,諸羅民縛以獻,賊黨擒斬略盡。臺灣南北兩路悉平。詔優敘,賜世驃東珠帽、黃帶、四團龍補服。未幾,以疾卒於軍。遺疏乞從父瑯葬福建,留妻子守墓,上悉許之。贈太子太保,謚勇果。雍正元年,世宗命予一等阿達哈哈番世職,以其子廷旉嗣。

世驃和易謙雅,治軍嚴明。與瑯先後平臺灣,皆以六月乘海潮異漲渡師,遂以成功。

藍廷珍,字荊璞,福建漳浦人。少習騎射,從祖理器之。入伍,自定海營把總累遷溫州鎮標左營游擊。巡外洋,屢獲盜,盜皆畏避。以是為諸將所忌,讒於總督滿保,將劾之。會關東大盜孫森等竊遼陽巨砲、戰艦逸入海。聖祖震怒,責沿海疆吏嚴緝。廷珍出巡海,至黑水外洋與遇,力戰,盡獲森等九十餘人,及其船艦、砲械。滿保按部至溫州,廷珍迎謁以告。滿保嘆曰:「幾失良將!」召入舟,厚撫之,亟疏薦,超擢福建澎湖副將。未幾,遷南澳總兵。

六十年,硃一貴為亂,廷珍上書滿保策破賊狀,滿保令統戰船四百、將弁一百二十、官兵一萬二千,會提督施世驃於澎湖,剋期進剿。廷珍至澎湖,言於世驃曰:「賊皆烏合,不足憂,惟脅從至三十萬人,請檄示止殲渠魁,餘勿問。則人人有生之樂,無死之心,可不血刃平也。」世驃從之。師至鹿耳門,賊扼險拒守。諸將林亮、董芳當前鋒,殊死戰,廷珍率大隊繼之,連戰皆捷。賊大潰,退保府治。世驃遣亮等自西港仔暗度,廷珍以大軍躡其後。賊在蘇厝甲,與亮等決戰,廷珍分兵馳赴之。賊望見旗幟,戰稍卻,乘勝追逐,遂大潰。夜駐犁頭標,設伏以待,賊果至,四面突擊,賊大亂,自相攻殺。追敗之木柵仔,復敗之蔦松溪,遂入府城,秋毫無所犯,民大悅。一貴及其黨李勇、吳外等皆就擒。分遣諸將復南北二路,署臺灣總兵。秋,南路阿猴林餘孽復起,討平之。招降陳福壽等十數人,皆渠魁也。未幾,世驃卒,廷珍攝提督。餘賊黃殿等以次擒滅。

六十一年,授臺灣總兵。雍正元年,擢福建水師提督,加左都督,賜孔雀翎,予三等阿達哈哈番世職。世宗褒廷珍忠赤,惟屢勉以操守。二年,入覲,命赴馬蘭峪謁景陵,賞賚稠疊。七年,病聞,遣醫診視。尋卒,贈太子少保,謚襄毅。子日寵,嗣世職,官銅山營參將。孫元枚,自有傳。

族弟鼎元,字玉霖,力學負才。廷珍統師入臺灣,鼎元參軍事,著平臺紀略。雍正元年,詔舉文行兼優之士,貢入太學,有司以鼎元薦,大學士硃軾器之,用薦得召見。上書陳時政,上嘉納。授廣東普寧知縣。居官有惠政,長於斷獄。性伉直,坐事劾罷。總督鄂彌達白其誣,召詣京師。旋署廣州知府。甫一月,卒。鼎元嘗論臺灣善後策,謂諸羅宜畫地更設一縣,總兵不可移駐澎湖。後諸羅析縣曰彰化,更設北路三營,總兵官仍駐臺灣,皆如鼎元言。

林亮,字漢侯,福建漳浦人。少習騎射擊刺。生長海濱,島澳險夷,舟航利鈍,靡不講求。初授臺灣水師把總,累遷澎湖協守備。硃一貴陷臺灣,官吏渡澎湖,居民洶懼。將吏以孤島難守,僉議撤歸廈門,各遣家屬登舟。亮按劍厲聲曰:「朝廷疆土,尺寸不可棄!今鋒刃未血,相率委去,縱避賊刃,能逃國法乎?請整兵配船,守御要害,賊至,決死戰!戰不捷,亮死,君等去未遲。」乃馳赴海口,申號令,驅將吏家屬登岸,令敢言退廈門者斬。時糧絕餉匱,亮輸貲買穀,碾米給軍,制戰攻器械,俟師至。提督施世驃、總兵藍廷珍以亮忠勇,令當前鋒,領舟師五百七十人抵鹿耳門。一貴黨蘇天威據砲臺以拒,亮率六艦直進,發砲中敵,火起,斃賊無算。乘勝進攻安平鎮,亮先登樹幟,賊潰走。翌日,鏖戰鯤身,駕舟橫沖賊陣,復大敗之。賊退至府城,世驃令亮分兵自西港仔暗度拊其背,廷珍以軍繼進,大戰,賊死傷遍野,遂克府治。亮功最,遷臺灣參將。雍正元年,敘平臺灣功,加都督同知,予一等阿達哈哈番世職。是年秋,入京,上深嘉之,擢水師副將,賜孔雀翎。

二年,授臺灣鎮總兵。亮以臺灣初被兵,加意撫綏,整水陸兵防。又招撫生番一百八社、男婦一萬八百餘人。亮因番嗜色布、鹽、糖,遣吏歷各社齎賜之,因宣布德意,群番悅服。五年,移浙江定海,卒於官,賜祭葬。

何勉,字尚敏,福建侯官人。初授督標把總。康熙五十八年,薛彥文等聚後洋山為匪,勉奉檄捕擒之。六十年,從提督施世驃討硃一貴,勉攻南路,擒其黨杜會三、蘇清等;又於北路獲黃潛等二十六人。明年,遷臺灣鎮標千總。時一貴餘黨王忠等出沒內山,巡視臺灣御史吳達禮督捕治,總兵藍廷珍檄勉偵緝。遣降卒為導,入鳳山深箐中,獲賊黨劉富生,思拒捕,立擒之。擢北路營參將,予拖沙喇哈番世職。雍正四年,水連沙等社叛番蠢動,總督高其倬檄從臺灣道吳昌祚按治。勉攻北港,番請降,水連沙二十五社悉平。

遷湖廣洞庭協副將。十年,貴州九股苗作亂,詔發湖廣兵二千協剿。提督張正興檄勉領兵五百赴貴州,進攻交汪寨。勉乘霧夾擊,苗敗遁,復據蓮花峰築屯。時貴州提督哈元生自臺拱移軍至,令勉攻其東。勉先登奪,賊竄走,掩擊之,陣斬其渠,餘眾就擒。擢雲南鶴麗鎮總兵,調臨元,復調廣東左翼。五年,調臺灣,尋又移南澳,署福建水師提督。乾隆十年,以疾乞休,詔解任回籍調治。尋召詣京師,以篤老,命原品休致。十七年,卒,賜祭葬。子思和,嗣世職。二十七年,復官臺灣總兵。

陳倫蜅,字次安,福建同安人。父昂,字英士,弱冠賈海上,習島嶼形勢、風潮險易。施瑯征臺灣,徵從軍,有功,授游擊。累遷至碣石總兵,擢廣東右翼副都統。嘗上疏言:「西洋治歷法者宜定員,毋多留,留者勿使布教。」又以沿海居民困於海禁,將疏請弛之。會疾作,命倫蜅以遺疏進,詔報可。

倫蜅初以廕生授三等侍衛。雍正初,授臺灣總兵,調廣東高廉。坐事降臺灣副將。復授總兵,歷江南蘇松、狼山諸鎮。擢浙江提督。卒。

昂疏並言:「臣詳察海上諸國,東海日本為大,次則琉球。西則暹羅為最。東南番族文萊等數十小國,惟噶囉吧、呂宋最強。噶囉吧為紅毛一種,中有英圭黎、乾絲蠛、和蘭西、荷蘭、大小西洋各國。和蘭西最兇狠,與澳門種人同派,習廣東情事。請敕督、撫、關差諸臣防備,於未入港之先,取其火砲。另設所關束,每年不許多船並集。」下兵部,但令沿海將吏晝夜防衛,寢昂議。倫蜅為侍衛時,聖祖嘗召詢互市諸國事,對悉與圖籍合。時互市諸國奉約束惟謹,獨昂、倫蜅父子有遠慮,憂之最早云。

歐陽凱,福建漳浦人。起行伍,累官江南蘇松水師營總兵。康熙五十七年,調福建臺灣鎮,以功加左都督。六十年,硃一貴作亂,官軍遇賊於赤山,千總陳元戰死。賊進攻鳳山,把總林富戰死,守備馬定國自殺。凱率所部守備胡忠義、千總蔣子龍、把總林彥御之春牛埔;參將羅萬倉,游擊孫文元,城守游擊許云,守備游崇功,千總趙奇奉、林文煌,把總李茂吉率水師來會,力戰破賊。次日,賊大至,凱力戰,與忠義子龍彥俱沒於陣,賊截凱首去。雲、崇功、奇奉、文煌同日戰死。茂吉被執,不屈,死。賊陷府治,萬倉戰死,文元奔鹿耳門投海死。同死者游擊王九人、守備吳泰嵩。又有把總石琳,自汀州被檄至臺灣,遇變被圍,死之。六月,師克臺灣。一貴既誅,獲其黨黃殿等,械送福州獄。雍正元年二月,賊破械斬關出,至下渡尾,都司閻威、守備楊士虎逐捕,殺數賊,被創死。先後議恤,凱贈太子少保,廕守備;雲以下皆贈官、予世職有差。

羅萬倉,甘肅寧夏人。官北路參將。凱戰死,賊攻府城,萬倉督將卒登埤,發大砲擊賊,僕賊旗。賊大至,萬倉出城與戰,逾溝墜馬,賊以竹篙刺其喉,猶揮刀殺賊乃死。妾蔣聞報,自經殉。

游崇功,字仲嘉,福建漳浦人。材力雄健。從總兵蔡元鎮襄陽。補右營把總,累遷福建長福營守備,分防長樂縣。濱海有磁澳,賊艘所出沒。崇功廉得狀,以兵二百伏隘口,入澳捕之。賊棄舟登岸,伏發,擒十七人。自是島澳肅清。長樂水災,崇功謁巡撫滿保,請發粟平糶,民食以濟。調臺灣北路營守備,巡緝外洋,擒海賊陳阿尾等六十餘人。遷水師游擊。一貴作亂,崇功方出洋巡哨,聞報,率兵還赴安平,至則賊已熾,崇功急登岸赴敵。其婿蔡章琦叩馬請一過家門區處眷屬,崇功不顧,躍馬揮眾,殺賊甚眾。五月朔,賊數萬戰於春牛埔,凱戰死,崇功突圍沖擊,馬被創,遂歿於陣。章琦,國子監生。聞崇功戰沒,赴海死。


\end{pinyinscope}