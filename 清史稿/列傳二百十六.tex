\article{列傳二百十六}

\begin{pinyinscope}
江忠義周寬世石清吉餘際昌

林文察趙德光張文德

江忠義,字味根,湖南新寧人,忠源從弟。咸豐二年,忠源率楚勇援長沙,忠義年十八,從軍,轉戰湖北、江西。忠源殉難廬州,遂分將其軍。五年,從提督和春復廬州,擢知縣。七年,劉長佑援江西,攻臨江不利,時忠義在籍,巡撫駱秉章檄率新練勇千人往助之,至則破石達開於平墟。臨江既克,擢知府,賞花翎。八年,克崇仁,進攻新城,五戰皆捷,加道銜。江西肅清,凱歸。

九年,石達開犯永州,忠義赴援,連戰破之,擢道員。又敗賊新寧摩訶嶺,扼武岡。賊圍寶慶,忠義進援,會諸軍迭戰解圍,賜號額爾德木巴圖魯。十年,駐守綏靖,母病歸。賊遂陷綏寧、城步,圍武岡,忠義聞警,分軍守新寧,自援武岡,破其眾。新寧之賊走踞東安,一戰克之,加按察使銜。又破賊於寧遠四廣橋。十一年春,連破賊於全州白芒營、宜章慄源堡,還軍屯新寧,遣參將江忠朝扼全州,賊目餘成義斬其酋以降。加二品頂戴,特擢署貴州巡撫。石達開復自粵竄楚,眾號十萬,忠義以三千人扼會同,大破之。賊糾湖北來鳳賊黨肆擾,擊走之,遂克來鳳。達開走入四川。十二月,丁母憂,請終制,詔允開署缺,仍在湖南剿賊。

同治元年,移師援黔,克天柱,改授貴州提督。調援廣西,克修仁,殪賊渠張高友。皖南賊熾,曾國籓疏調援皖,廣西巡撫劉長佑請留不遣,命署廣西提督。二年,江西、廣東皆調援,先後報可,忠義以廣東兵有餘糧,他將足辦賊;江西餉絀兵單,賊數十萬,萬一不支,東南全局瓦解,乃奏請力援江西。檄道員席寶田率前部先發,會剿陶家渡,自將攻湖口,逼賊營,屢出奇兵抄擊,斷文橋,攻太平關,賊酋黃文金受重創遁去,賜黃馬褂。進援青陽,分三路進戰,破賊壘,圍解,太平、石埭、寧國諸城賊第出降,詔嘉獎,予優敘。會疾作,返就醫南昌,未至,卒於吳城,年甫三十。優詔悼惜,依總督例賜恤,贈尚書銜,謚誠恪,立功地建專祠。光緒十一年,加贈太子少保。

從弟忠珀,記名提督。同治八年,剿貴州苗,攻克鎮遠、府衛二城,中砲亡,謚武愍。

周寬世,字厚齋,湖南湘鄉人。咸豐初,從湘軍,隸李續賓部下。戰城陵磯、花園、半壁山,皆有功,擢千總。從援江西,攻廣信,戰烏石山,寬世出左路突陣,為諸軍先,復其城,擢守備。破賊義寧,擢都司。回援武漢,戰通城,寬世馳斬馬賊三,生擒七,以游擊補用。從攻武昌,六年,李續賓夜出偵賊,𥫗之雙鳳山,突戰,寬世潛繞山趾橫擊之,賊敗奔;又戰鷹嘴,受砲傷,假歸。累功擢參將,賜花翎。

既而羅澤南卒於軍,續賓代將,召寬世回營。迭破賊於雙鳳山、魯家港、小龜山,克武漢,復大冶、興國,擢副將。七年,從攻九江,破援賊於童司簰,毀其壘,賜號義勇巴圖魯。破小池口賊屯,會克湖口,復彭澤。賊由臨江犯興國,寬世率千六百人擊走之。八年,回援湖北,戰麻城西南斗坡山。賊設伏,以馬隊誘戰,寬世待其近,突擊之,遂破其伏軍,進克黃安,而麻城亦下。大軍克九江,論功,以總兵記名。

從李續賓進軍安徽,戰楓香鋪、小池驛,克太湖、潛山,搗舒城,寬世皆為軍鋒。十月,進攻三河,續賓戰沒,寬世斂餘眾守二日,彈丸俱盡,夜率親卒突圍,受重傷。是年冬,授湖南永州鎮總兵。九年,石達開犯湖南,巡撫駱秉章令寬世募新軍二千援祁陽。破賊長慶橋,又敗之長葉嶺。進援寶慶,屯城東,連敗賊長沖口、五里牌。李續宜援師至,會諸軍內外夾擊,賊解圍走。回剿永州土匪,平之。十一年,擢湖南提督。

同治元年,赴安徽助剿,駐守桐城。二年,捻匪馬融和犯桐城,擊走之,移防六安。皖北漸定,調守安慶。三年,赴援江西,克東鄉。四年,破霆軍叛勇,追賊入廣東,會諸軍殲賊於嘉應。五年,回湖南提督任。傷發,乞休。光緒十三年,卒。

石清吉,字祥瑞,直隸沙河人。道光二十一年武進士,官三等侍衛。咸豐初,出為湖北鄖陽鎮守備,從剿黃陂、崇陽、應城,累擢參將。克安陸、京山皆有功,以勇稱,所統曰飛虎軍。尋隸將軍都興阿軍,常從多隆阿轉戰。七年,援蘄州,拔太湖,攻安慶。八年,由安慶退保宿松,大戰破賊。九年,攻太湖。十年,大戰小池,克太湖,功皆最。十一年,安慶既下,會諸軍克桐城。

同治元年,從攻廬州,清吉屯城西北,破賊壘,擒斬數千。進毀賊柵,樹雲梯攻城,賊方死拒,而陳玉成兵敗遽去,遂由西門攻入,克廬州。清吉累以戰功賜號幹勇巴圖魯,擢總兵,加提督銜。多隆阿督師赴陜西,以清吉統五千人留守廬州。二年,苗沛霖復叛,廬、壽、開土蜂匪起,清吉悉剿平之。粵、捻諸匪合擾豫、楚之交,清吉赴援湖北,屯孝感、黃岡,拔難民近萬。

三年九月,匪酋陳得才、馬融和合犯蘄水,圍副都統富森保於關口。清吉率軍馳援,會大霧,賊馬步數萬麕集。清吉進至藥山,賊渡河抄後路,圍數重,截其四營為二。自辰至午,血戰,被九創,殞於陣。從戰歿者,副將江星南、穀明發,游擊曾占彪、段會元。事聞,詔視提督陣亡例賜恤,入祀京師昭忠祠,予騎都尉世職,謚威毅,建專祠。

餘際昌,湖北穀城人。咸豐初入伍,剿匪積功至守備,署撫標右營游擊,為巡撫胡林翼所識拔。七年,從戰黃梅、廣濟。八年,陳玉成自太湖竄蘄州,際昌奉檄防皖、楚之交,敗賊南陽河,毀賊壘三十餘,擒賊目。賊走英山,追躡之,復其城,擢游擊。又破賊彌陀寺,晉參將。李續賓軍覆三河,潛山、太湖復陷,際昌屯英山,遏潛、太之沖。九年,進拔天堂。賊大舉來爭,際昌敗諸王婆坳,追至雞冠嶺而還。再敗賊槎水畈,斬馘千餘。時大軍圍太湖急,陳玉成糾黨十餘萬相持小池驛。十年正月,際昌偕金國琛由間道出高橫嶺,與諸軍夾擊,大破之,遂復太湖,乘勝會攻克潛山,擢副將,署湖北督標中軍副將。陳玉成自六安回援安慶,霍山復陷。際昌偕總兵成大吉擊破之,復霍山,加總兵銜。十一年,陳玉成入霍山,自黑石渡撲樂兒嶺。際昌軍潰,賊上竄黃州,革職留營。尋從克黃州,率新募昌勝五營援河南。

同治元年,屯陳留。捻匪麕集杞縣,際昌馳擊,大破之,進拔焦、趙二寨,復原官,賜號偉勇巴圖魯。十月,攻捻於汝寧,破平輿寨,生擒賊酋陳文,詔以總兵記名。僧格林沁嘉其勇,令充翼長,從剿渦河,斬賊渠楊興太等。二年春,追破陳大喜於阜陽吳老莊。捻首張總愚竄侯集,際昌會張曜夜襲之,擒其黨獨角虎、周馬,授河北鎮總兵。夏,逐賊楚、豫間,敗之麻城,躡至方家寨,中伏力戰,受三十餘創,死之。贈提督,予騎都尉世職,謚威毅。

林文察,字子明,福建臺灣人。咸豐八年,從剿臺灣淡水土匪,捐餉助軍,以游擊留福建補用。十年,九壟山匪郭萬淙掠建寧、邵武間,汀州、龍嚴匪胡熊擾寧洋、永安。文察隨軍進剿,擒其黨百餘人。郭萬淙遁據邵武上山坊,文察合軍蹙之,降其眾,復破胡熊於東板土寨,擒之,擢參將,賜號固勇巴圖魯。十一年,援浙,克江山,晉副將,晉號烏訥思齊巴圖魯。汀州、連城相繼陷,文察回援,破賊金雞嶺,設伏,敗之江防,遂拔連城,乘勝克汀州,以總兵記名。冬,杭州既陷,調援浙,文察領臺勇二千人駐衢州。同治元年,破處州賊屯,而遂昌陷,文察進軍逼之。李世賢自江山來援,文察設伏大柘、大廟及石練山之前後,賊至,擊走之。夜,賊來劫營,復為伏兵所敗,復遂昌,進克松陽。會總兵秦如虎攻處州,賊棄城遁,並克縉雲,授福寧鎮總兵。尋擢福建提督。

二年,臺灣不靖,總督左宗棠令渡臺號召舊部,統領諸軍。文察分軍攻彰化及斗六,克之。諭降諸莊,賊渠戴萬生、林戇晟遁走。三年,破樵溪口賊莊,斬其酋林傳,毀張厝莊、四塊厝賊巢,戴萬生、林戇晟並伏誅。

粵匪李世賢、汪海洋合陷漳州,文察倉猝率二百人內渡,遇賊萬松關,歿於陣,贈太子少保,予騎都尉世職,謚剛愍。本籍及漳州建專祠。

子朝棟,光緒中,法兵犯臺灣,陷基隆,朝棟率家兵助戰有功,捐鉅貲,賜四品京堂,有聲於時。

趙德光,原姓張,貴州郎岱人。從副將趙德昌轉戰雲南,德昌弟畜之,故冒姓趙氏。拔補千總,擢都司。咸豐十年,自領一軍,戰獨山,屢敗賊,擢游擊。十一年,賊窺省城,德光擊走之。又敗之羊場平寨,設伏於主戎山麓,殪賊無算,擢參將,賜號豪勇巴圖魯。教匪踞玉華、尚大坪,以王卡為屏蔽。德光率所部攻破楊義司、郭家莊、馬籠口賊營,斷其援,又破腰蘿溪、新寨巖要隘,進偪王卡。德光先登,賊大潰,救出男婦數千人,擢副將。

同治二年,壩芒匪首潘明傑由龍裡窺伺省城。德光迎擊三江橋,賊敗走。進攻甲秀閣賊巢,遂克龍裏舊縣,補都勻協副將,以總兵記名。三年,尚大坪匪撲省城,德光與布政使龔自閎等固守,賊尋退,加提督銜,署古州鎮總兵。旋解清鎮圍,克龍里、廣順、定番、長寨,以提督記名。四年,匪首何二久踞開州、尚大坪,擾近省州縣,無寧歲。德光選精銳過清水江剿之。賊糾集苗匪、教匪沿江以拒,乘間過江攻開州。德光固守十餘日,殺賊八九百人,乘勝追擊,克沿江獅子坉、鎮江坉、三龍營賊屯。進克濱江賊巢,斬馘二千餘,何二棄尚大坪而遁,被優敘,署安義鎮總兵。

五年,署貴州提督。攻克永寧,解安順圍。六年,援定番,乘雷雨破賊,斬賊首許八十等,平花山賊屯,拔底季賊巢,晉號博奇巴圖魯。尋剿賊安平蘆荻哨,深入賊伏,中槍陣亡。詔依提督陣亡賜恤,贈太子太保,予騎都尉兼一雲騎尉世職,謚剛節,建專祠。遺腹生子秉鈞,襲世職,復姓張氏。

張文德,湖南鳳凰人。幼育於文氏,從姓文,名龍德。入行伍,隸鎮筸營。咸豐初,從剿江寧、廬州,敘把總。六年,從提督和春攻三河賊壘不下,文德請獨身持檄諭賊降,投誠者相繼至,遂克三河。七年,從復鎮江,擢都司。八年,從援福建,下浦城、松溪、政和、崇安,賜花翎。九年,敘援浙江功,晉游擊。十年,從張國樑解鎮江圍,援賊復至,文德扼水柵七晝夜,賊引去,擢副將。自是從將軍巴棟阿、提督馮子材守鎮江。十一年,補廣東羅定協副將。

同治元年,賊屢攻鎮江,皆擊退。馮子材奏言:「文德力挫賊鋒,重圍疊解,實為特出之材。」授貴州鎮遠鎮總兵,賜號翼勇巴圖魯。文德以生父年七十無子,養父文氏有二子,陳請復姓,更名文德。二年,連破賊牧馬口、薛村,克柏林村賊壘,加提督銜。賊由東路來犯,文德御諸駭溪、諫壁,腹中砲,腸出,裹創而戰,援軍至,賊乃退;又破之博洛村,攻丹陽,毀賊壘,擒賊目。三年,克白堍鎮及寶堰,賊黨紛紛來降。會鮑超攻丹陽,招賊酋蔣鑒為內應,克其城,斬賊酋陳時永,擒賴桂芳,以提督記名。江南平,予一品封典,命赴鎮遠鎮本任。

四年,總督勞崇光令募楚勇規荔波、獨山。丁父憂,解職。六年,署貴州提督。七年,克開州,破鼎照山賊砦,克龍里、貴定,斬賊酋潘名桀,餘賊多降,被珍賚。進攻平越,擒金大五,連克麻哈、都勻,賜黃馬褂,晉號達桑阿巴圖魯。請假歸葬親,文德既去,賊復熾。八年,回貴州,以糧匱軍潰,都勻復陷。詔原之,免議,署古州鎮。十年,授威寧鎮總兵,督軍剿古州苗。由九甲、五臺山、扁擔山及古州、丹江分路雕剿,年餘,苗渠先後伏誅。十三年,全黔肅清,予雲騎尉世職。光緒元年,加頭品頂帶,擢貴州提督,剿平黎平侗匪。七年,卒,賜恤,貴陽建專祠。

論曰:江忠源諸弟並從治軍,忠義最為傑出,將大用而早沒,時論惜之。周寬世為李續賓所倚,無役不從,及自將亦未著奇績。石清吉、餘際昌、陳大富、林文察、趙德光等,皆久歷行間,以死勤事。張文德佐馮子材守鎮江,功最著,底定黔疆,與有勞焉。


\end{pinyinscope}