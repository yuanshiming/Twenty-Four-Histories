\article{列傳二十六}

\begin{pinyinscope}
沈文奎李棲鳳馬鳴佩馬國柱羅繡錦繡錦弟繪錦雷興王來用

丁文盛子思孔祝世昌

沈文奎,浙江會稽人。少寄育外家王氏,因其姓。年二十,為明諸生,北游遵化。天聰三年,太宗伐明,下遵化,文奎降。從貝勒豪格以歸,命值文館。漢軍旗制定,隸鑲白旗。六年六月,上自將伐察哈爾,因略宣府邊外。明文武大吏請盟,上還師。八月丁卯,召文奎及同值文館諸生孫應時、江雲深入宮賜饌,命策和議成否。文奎等皆言明政日紊,中原盜賊蜂起,民困於離亂。勸上宣布仁義,用賢養民,乘時吊伐。文奎等退,各具疏陳所見。

文奎疏言:「先帝用兵之初,勢若破竹,蓋以執北關之釁,名正言順。其後多疑好殺,百姓離心,皆曰利我子女玉帛耳。上寬仁大度,推心置人。今師次宣、大,長驅而入,誰復敢當?乃以片言之故,卷甲休兵。大信已著,宜乘時遣使,略遜其辭,以踐張家口之約。夫不利人之危,仁也;不乘人之亂,勇也;不失舊約,信也:一舉而三美歸焉。或謂南朝首吝王封,次論地土人民,和必不成。臣謂和否不在南朝,在上意定不定耳。且和而成,我坐收其利,以待天時;和而不成,或薊鎮,或宣、大,或山海,乘時深入,誕告於眾曰:『幽、燕本金故地,陵墓在房山,吾第復吾故疆耳。』師行毋殺人,毋劫掠,則彼民必怨其君之不和,而信我無他志矣。大凌河降夷,上赦之刀斧之下,復加以恩育,其所以去者,皆父母妻子牽其念耳。文王王政,罪不及孥。執殺逃亡,已正國法。豈可因兄及弟,因父及子?以一降夷而使眾降夷自危,且使凡自大凌河降者人人坐疑,非上明白宣諭,上下暌違,終不能釋也。我國衣冠無制,貪而富者,即氓隸,冠裳埒王侯;清而貧者,即高官,服飾同僕從。乞上獨斷,定衣冠之制,使主權尊,民志定,賢愚僉奮,國日以強。」

雲深疏言:「南朝未能決和,宜倍道徑取山海。山海既破,八城折入於我,再與畫界議好,和乃可定。」

應時疏言:「用兵當先足民。年來國用不舒,今歲又被災,十室九空,宜乘時究方略,轉虛為盈,此宜急議者也。八門徵稅,正稅外有羨銀,稅一兩非增三四分不收,朘削窮民脂血,此宜嚴覈者也。六部公廨已畢工,人人當盡心力為上治事,否則不惟負上,抑且負此巨室,此宜申飭者也。大凌河新夷,固自取滅亡,然邊防嚴則逋逃何自越,此亦宜申飭者也。」

是歲近明邊蒙古部民逃入沙河堡,明兵索還。文奎、應時疏中曰「降夷」,曰「新夷」,蓋謂是也。

九月,文奎復疏言:「臣自入國後,見上封事者多矣,而無勸上勤學問者。上喜閱三國志,此一隅之見,偏而不全。帝王治平之道,奧在四書,跡詳史籍。宜選筆帖式通文義者,秀才老成者,分任移譯講解,日進四書二章,通鑒一章。上聽政之暇,日知月積,身體力行,操約而施博,行易而效捷。上無曰『此難能』,更無曰『乃公從馬上得之』,烏用此迂儒之常談,而付之一哂也。上用人亦宜詳審,臣第就書房言之。書房出納章奏,即南朝之通政司也。自達海卒,龍什罷,五榜式不通漢字,三漢官又無責成。秀才八九,閧然而來,群然而散。遇有章奏,彼此相諉,動淹旬月。上方求言,而令喉舌不通,是何異欲其入而閉之門乎?宜量才委用,或分任俾責有所專,或獨任俾事有所總。至筆帖式通文義者,惟恩國泰一人,宜再擇一二以助不逮。立簿籍,定期會,使大事不過五,小事不過十,分而任之。課勤惰,察能否,而從以賞罰,則政柄不搖,賢愚並勵矣。」

七年七月,疏言:「圖事功者,以得人為先務。頃聞開科取士,誠開創急事也。然臣以為非掄才之完策,上宜發明諭,不拘族類,不限貴賤,不分新舊,有才能者許自薦,知人有才能者許保舉。自薦者擇有智識之臣,畀以掄選,而嚴挾私徇情之罰;保舉者不避父子兄弟,但令立狀記籍,異日考其功罪,與同賞罰,然後親加省試,量才錄用。有技能則超擢,無才行則責譴。奴隸工商,有善必取。顯官貴戚,有惡必懲。招以真心實意,歆以高爵厚祿,繩以嚴刑重罰。好榮惡辱,人情所同。雖不能拔十得五,於千百中得數人,而已足為用矣。」崇德元年,甄別文館諸臣,文奎列第二,賜人戶、牲畜,授內弘文院學士。七年八月,以醉乘馬犯鹵簿,論死,上宥之,仍命斷酒。

順治元年,世祖定鼎,七月,命為右副都御史,巡撫保定。時畿南未定,保定、大名、真定所屬諸州縣,盜千百並起,焚掠為民害。文奎到官,駐真定,訓練所部兵,與巡按衛周胤謀捕治,盜渠趙崇陽等數百人降。有韓國璧者,為盜寧晉泊,拒官軍。文奎即用崇陽捕斬國璧,殲其徒。遂分部總兵王巇、守備劉文選等將兵逐賊。巇等討滅香爐、喬家二寨,戮其渠錢子亮、趙建英。文選等攻深州,戮其渠於小安;攻晉州,戮其渠馬數全。於是冀州郭世先、保定李庫、內黃李君相、順德袁三才數十渠魁,並就俘戮。散其脅從,錄驍勇置部下。畿南漸定。州縣吏徵賦仍明季舊習,優免多則蝕賦,攤派行則厲民,文奎疏請悉從正額;寧晉泊地肥而賦輕,豪右競占,逋賦為州縣吏累,文奎疏請招民分耕納賦;二年正月,疏言畿南民重困,歲貢綿絲諸品,皆求諸他行省,請改折色;二月,又論諸衛所地納賦丁入保甲,皆當屬州縣吏:並見採擇。李聯芳、張成軒等為盜南皮、鹽山間,四月,遣都司楊澄、守備徐景山捕治,戮聯芳等九十三人。

尋命加兵部右侍郎,總督陜西。五月,改命總督淮、揚漕運。淮、揚群盜,高進忠、魏用通、高升三人者為之魁,復有酆報國、司邦基挾明宗室新昌王,與相應為亂。文奎遣游擊裴應暘等擊斬用通,總兵王天寵亦擊破升,報國、邦基為其徒縛詣江寧以降;進忠走崇明,亦降。十二月,復令總兵孔希貴、蘇希樂逐盜如皋,得其渠於錫籓、劉一雄。三年八月,又與淮徐道張兆熊發兵擊斬邳州盜楊秉孝、王君實等。江、淮間始稍安。十月,疏請禁革蘇、松諸府徵漕積弊,悉去官戶、儒戶、濟農倉諸名,著為令。四年正月,以擅免荒田賦,又瀆請明陵祀典,奪職。

五年十二月,起為內弘文院學士。六年,充會試總裁。八年,大學士剛林、祁充格得罪,文奎以知睿親王多爾袞令改實錄不上言,當坐,上命免議。四月,復命以兵部侍郎、左副都御史,總督漕運,巡撫鳳陽。請復姓沈氏。七月,疏請慎選運官,清核舍餘,合選殷丁,清勾黃快,皆漕政大端,凡四事。十年,率師討膠州叛將海時行。十一年,遣兵捕硃周錤,清通、泰濱海逋寇。江北廬、鳳、淮、揚諸府災,文奎請蠲賦,戶部議未定,冬盡未啟徵。九月,文奎坐督運愆遲,左遷陜西督糧道。尋卒。

與文奎同時以諸生直文館者,雲深、應時同被召對。又有李棲鳳、楊方興、高士俊、馬國柱、馬鳴佩、雷興輩,蓋皆文奎疏中所謂秀才八九者也。棲鳳、方興、國柱、鳴佩、興自有傳。雲深後不著。應時為啟心郎,以祝世昌請毋以俘婦為妓,為改疏稿,坐死。士俊嘗上疏謂:「上定例一丁予田五日,衣食於此出,力役於此出。民已苦不足,況以繩量田,名五日,實止二三日。將吏復占沃地,役民以耕,宜禁革。民間貸金,當視金多寡定取息重輕,其有逾度者,宜坐罪。」日者,滿洲以計田,土俊用當時語也。士俊入關後,嘗為湖廣巡撫,收長沙,克衡州、常德,有勞。

方上召文奎等策議和成否,亦諭吏民令建言。有胡貢明者,疏言:「我國與南朝未嘗無內外君臣之分。今既議和,當遣使修表,姑聽其區畫。如不欲為之下,遂圖大事,必如漢高祖而後可。」因謂鼓舞用人,養百姓,立法令,收人心,皆未若漢高祖。貢明先嘗上疏請更養人舊例,略言:「太祖時方草創,土地、人民、財用皆與諸貝勒均之。今尚沿此習,上名雖有國,實不啻正黃旗一貝勒耳。一人寸土,上與諸貝勒互不相容。十羊九牧,即有中原不可以為治。出師得財,當以三屬上,七分畀諸貝勒。得人聚而贖之,視其賢不賢,厚薄予奪,權得以自操,而人心亦歸於一。」至是又別疏申前說,並反覆言養豪傑當破格,如高祖之於「三傑」。上覽先疏,頗韙其語,謂後出師當用汝議;覽後疏,責其語冗。貢明復上疏抗辨。

七年,又有扈應元者,疏詆漢官但求名利,語近戇,略如貢明。別疏陳七事,謂備荒宜儲糧;編丁宜恤老;築城建關宜不妨農業;出師宜選公正廉能吏,拊循新下郡邑;取士宜尚德行;求言宜置諫官;乘機取天下,在人心不在火器。上覽其疏,至論築城建關,疑勿善也,不竟閱。應元亦上疏抗辨。

貢明隸鑲紅旗,亦諸生;應元隸正白旗,自署「隱士」。

李棲鳳,字瑞梧,廣寧人,本貫陜西武威。父維新,仕明為四川總兵官。嘗官薊、遼,家焉。馬鳴佩字潤甫,遼陽人,本貫山東蓬萊。其先世嘗為遼東保義副將,因占籍遼陽左衛。棲鳳、鳴佩皆以諸生來歸,事太宗,並值文館。崇德元年,甄別文館諸臣,棲鳳、鳴佩俱列二等,賜人戶、牲畜。漢軍旗制定,同隸鑲紅旗。世祖定鼎,授棲鳳山東東昌道,鳴佩山西冀南道。順治二年,收湖廣,移棲鳳上荊南道,鳴佩下湖南道。

方棲鳳值文館,治事勤慎,達海等聞於上。上命司撰擬,移寫國書。達海卒,棲鳳言文館無專責,櫝貯官文書,人得竊視,慮有漏言。上召王文奎等諮和議成否,棲鳳上疏言:「臣侍文館幾七年,今上與南朝議和,謀及群臣。臣愚以為時政有可惜者二,當速圖者六。先帝勞心力、訓練勁旅以遺上,上當法先帝賞罰出獨斷,有功雖賤雖仇必賞,有罪雖貴雖親必罰。若不振奮鼓舞,必且習為洩洩,弛已成之業。此可惜者一也。上天姿英敏,誠大有為之君也。臣見諸臣章奏,輒曰『上寬仁大度』,此則諛耳。創國之君,不欲過刻,亦不欲過寬。用人聽言,審察其可否,中夜而思,如何使人畏,如何使人喜,而後可以驅使。倘信虛譽而毗於仁厚,必誤上英敏矣。此可惜者二也。民以食為天。今歲水且螟,米值驟昂。上宜速出師攻關外八城,八城為我有,豈復慮我民之枵腹耶?一失此機,民無食且流散,國亦稍稍衰矣。當速圖者一也。上舊得人民,兵農工役,物物皆備。惟頻歲役民築城,此毀彼建,不得休,民未必無怨。昨聞大凌河西夷復加誅戮,奈何先與之誓而後又殺之也?今宜罷非時之工,廣養人之惠。當速圖者二也。南朝東西支梧,奔命不遑,勢必且南遷。祖大壽與上嘗有盟約,當急遣使游說,乘機進兵,遲則失時。當速圖者三也。君雖聖,必賴賢臣以調燮之。近雖有二三骨鯁之臣,位卑祿薄,信任未專。如永平道張春,在彼中號有謀略,上宜隆以禮遇,心雖金石,將為我鎔。我國雖邊鄙,未始無才,重賞之下,必有勇夫。當速圖者四也。諸臣多請制定衣冠,尚未允行。夫所謂衣冠,豈必如南朝紗帽圓領而後可?但能別尊卑,差貴賤,即是制度。國體威嚴視斯,人心系戀視斯,綱紀法度,風移俗易,莫不視斯。當速圖者五也。達海竭心力奉上,及其卒,斂乃無鞾,其廉若此,未聞上破格矜恤。總兵布三取遼陽首功,先帝賜敕免死,今以事奪官,且下之獄,不過以愚直得罪。功過貪廉,自古無全才,不可拘於一。當速圖者六也。」調為上荊南道參政。明年六月,遷湖廣右布政使。

十月,命以右副都御史巡撫安徽。吳繼、程國柱等為寇休寧、婺源間,棲鳳檄總兵李仲興、許漢鼎等帥師捕治,獲所置總兵江烏、鄭恩祥,降張天麒、江周等千人。其黨趙正挾明瑞昌王誼貴攻宿松,棲鳳率總兵卜從善、冷允登御之洿池,斬千級,獲誼貴及正子捷應,弟允升。招撫江南大學士洪承疇上其事。旋坐屬縣濫徵賦不舉劾,左遷。

六年,復自浙江嘉湖道參議授右僉都御史,巡撫廣西。明桂王由榔遣兵略廣東諸郡縣,尚可喜、耿繼茂軍駐廣州,棲鳳駐南雄,為具儲糈。七年,合兵克韶州,並破雷州、廉州諸寨。八年,明將曾志建侵韶州,棲鳳令南韶道林嗣琛、游擊張瑋等擊之,斬二千餘級。九年,遣副將先啟玉等攻欽州,獲叛將李成棟子元胤。十年,明將李定國自梧州侵肇慶,棲鳳遣兵敗之龍頂岡;尋分遣總兵徐成功、吳進功等復羅定州東安縣。捷聞,上手書「知方略」三字以賜。又遣副將陳武、李之珍徇高州,至沙江。敵循江岸列寨,師渡江縱擊,獲所置副將姚奇、中軍余元璣等。克化州、吳川縣,焚其壘,殲敵。以功進兵部右侍郎。

十五年三月,考滿,加兵部尚書。六月,命總督兩廣。時明桂王走雲南,其將陳奇策及明江夏王蘊鑰、德陽王儼錦等據上思州,旁掠諸縣,棲鳳令總兵慄養志等討之,獲奇策等;又剿撫那錦、板強諸寨,定太平、思恩諸府。十七年,加太子少保。十八年九月,分設廣東、廣西兩總督,棲鳳督廣東。十二月,以老乞休。康熙三年正月,卒。

鳴佩,天聰三年,授工部啟心郎,仍直文館。六年,與同官羅繡錦疏論輸糧令,語詳繡錦傳。崇德八年,授半個前程。順治三年,自下湖南道參政授戶部侍郎銜,總督江南糧儲兼理錢法。疏言錢法首禁私鑄,犯必誅,並請設錢法道專其責;江南軍餉不足,請留關稅佐之:皆議行。八年,入為戶部侍郎。十年,改總督倉場侍郎。

十一年二月,命以兵部左侍郎兼右副都御史,總督宣、大、山西,勸墾宣府、大同荒地三千餘頃。盜發平陽,鳴佩令副將許占魁等捕治,分兵扼隘,誅其渠張五等二百八十餘人,降其黨九十餘。

十月,加兵部尚書,移督江南、江西。時鄭成功為寇海上,陳其綸、汪龍等為明將,號為侯、伯,據郡縣,遙應成功。鳴佩檄總兵胡有升等攻其綸瑞金,破大柏山寨。其綸走寧都天心寨,寨民獲以獻;復獲龍九江,並擊破成功之徒胡寧等。未幾,明將張名振以舟師侵崇明,鳴佩亦以舟師御之,名振敗走,得其副將林正禮等;復周歷松江、崇明諸郡邑,視形勢,疏陳水陸攻守之策。會給事中張玉治言江寧提督當移駐蘇州,吳淞宜增兵,上令鳴佩覈議。鳴佩請令江寧提督分兵守劉河、福山,蘇松提督駐吳淞,不煩更增兵,但令與江寧提督互策守御為犄角。得旨,如所議。十二月,名振兵復侵崇明,以舟師斷海港,官軍莫能渡,鳴佩密令民束草削★H9,佐軍焚敵舟,俘馘無算,名振夜引去。十三年正月,降所置總兵顧忠,副將黃忠、董禮等百餘人。顧忠故劇盜,號「綱倉顧三」,善水戰,至是降,敵益沮。復率參將吳守祖等出海,至浙江獨山破敵。分兵討吉安、贛州盜,敗之上坪;討徽州盜,剿花橋諸寨。閏五月,以目疾乞罷,進三等阿達哈哈番。康熙五年正月,卒。

鳴佩嘗薦梁化鳳有大將才,及成功入攻江寧,賴化鳳破敵。棲鳳、鳴佩子弟皆才。棲鳳弟棲凰漕運總督加太子太保,棲鵾、棲鸞總兵,棲鳴廣東提督;子鎮鼎,亦官廣東提督,加太子太保。鳴佩子雄鎮,自有傳。

馬國柱,遼陽人。天聰間,以諸生直文館。六年,諸生胡貢明請更養人舊例,語附見沈文奎傳。國柱上疏,謂:「以家喻國,上猶祖父,諸貝勒猶子弟,而人則妻孥也。祖父重持家,子弟喜便嬖,好惡不同,不能迫而從也。我國正直者多貧賤,貪佞者多富貴。正詘而邪申,欲國之興得乎?宜採貢明議,無分新舊人,悉養於上。如疑八家分人而贍為先帝舊例,試思先帝時雖曰分贍,而厚薄予奪操之一人。今昔相較,果何如乎?況善繼志者謂之大孝。先帝至今日,亦當更舊習。茍益於國,何有於小嫌?且利於八家,而上獨擅焉,誠不可也;今養人乃勞事,雖專之,庸何傷?」

先是,國柱與高鴻中、鮑承先、寧完我、範文程等合疏請置言官,是疏並申言之;而諸上書言時事者,扈應元、徐明遠、許世昌、仇震疏中往往及是。應元事見沈文奎傳。明遠,明兵部吏,自永平降,隸鑲黃旗。疏並請禁交結,定法度,立管屯將吏考課黜陟之制,禁管臺將吏掊克士卒,禁八門監榷不得用重秤,豁流亡戶籍,錄閒冗吏,革鬻良人為妓。世昌,正紅旗牛錄章京。疏並請定先帝謚號,建中書府。震,明武進士、都督僉事。疏自署「俘臣」,並請譯書史,申法律,簡賢才,與明通和。

八年,太宗命禮部設科取士,中式為舉人,國柱與焉。直文館如故。崇德初,始置都察院。三年,授國柱理事官。漢軍旗制定,隸正白旗。順治元年,從入關,授左僉都御史。師已定大同、代州,七月,命國柱以右副都御史巡撫山西,道昌平,出居庸關,至代州任事。師自忻州克太原,國柱進駐太原。師行,任策應。汾州、平陽、潞安、澤州諸府以次底定。李自成將李過、高一功走保綏德,國柱疏請分兵東西夾擊,使賊首尾不相應,上韙其議。二年,遣游擊楊捷擊斬陽曲盜閻汝龍,別將討嵐縣盜高九英,降四十餘寨。交城盜梁自雨、河曲盜李俊與九英犄角,國柱復分兵捕治。國柱撫山西年餘,捕誅自成餘孽伏民間者,安集撫循,民漸復業。客軍數往來,苦供億繁,國柱悉心措置,民不知兵。十月,擢兼兵部侍郎,總督宣、大。

四年七月,加兵部尚書,移督江南、江西、河南三行省。五年正月,安慶亂者馮洪圖陷巢縣,掠無為州,國柱令按察使土國寶從侍郎鄂屯帥師討之,獲洪圖及其黨蔣懋修、鍾武等。江西總兵金聲桓叛,其將潘永禧犯徽州,國柱遣滿洲駐防官兵擊破之,復祁門、黟二縣。上命征南大將軍譚泰帥師討聲桓,克九江、南康、饒州等府。明尚書餘應桂據都昌,出沒鄱陽湖,國柱令副將楊捷等從譚泰攻克都昌,獲應桂;復擊敗其將鄧應龍等於武寧。十月,廣東叛將李成棟自南雄侵贛州,國柱遣將與江西巡撫劉武元合兵擊殺之。

六年,有王定安者,為亂於湖廣,陷羅田,結英山盜陳元等掠霍山,國柱遣中軍副將硃運亨等擊之,戰於三尖山,元等引去;又令總兵卜從善剿白雲、梅家、英窠諸寨。明石城王統錡率五千餘人自金紫寨赴援,倚山列陣,從善與戰,俘馘甚眾,獲所置總兵孔文燦、副將方學達等。國柱復率師會江寧昂邦章京巴山、提督張大猷討六安盜,圍將軍寨,擊斬其渠張福寰,降所置總兵王俊、副將霍維倫等。安徽境諸弄兵者,往往依山結寨相望,至是始盡。

明魯王以海在舟山,其將吳凱據大蘭山為聲援,上命國柱策剿撫。國柱知寧波諸生方聖時與以海臣嚴我公友,使為游說,我公遂降,國柱護送京師。上遣齎敕招凱,國柱復寓書焉,凱與其將顧奇勛、姜君獻、陳德芝等降。七年,加太子少保。

九年七月,有張自盛者,為亂於福建,闌入江西境,保大覺巖,國柱檄提督劉光弼擊斬所置總兵李全等,遂獲自盛。十一年正月,明將張名振攻崇明、劉河、吳淞,國柱募水師,遣總兵王璟、副將張恩達分將之,敗之於靖江,復敗之於泰興,毀其舟,名振引去。二月,有賴龍者,為亂於湖廣,號「紅頭賊」,自桂東侵江西境,國柱與湖廣總督祖澤遠合兵攻桂東,得龍,亂乃定,復加太子太保。旋致仕。國柱初至江南,駐防兵與民不相習,國柱善為撫戢,令行禁止,兵民相安。康熙三年二月,卒。

天聰八年,舉人凡十六人,漢人習漢書者,齊國儒、硃燦然、羅繡錦、梁正大、雷興、馬國柱、金柱、王來用,得八人。國柱及繡錦、興、來用入關後,皆至督撫,而國柱、繡錦、興又同值文館。

繡錦,亦遼陽人,以諸生來歸。天聰五年,與馬鳴佩同授工部啟心郎。六年,上以大凌河新附人眾,計國中無問官民,計口儲糧,有餘悉輸官,視市值記籍,徐為之償;有餘糧不輸者,許家人告發。繡錦、鳴佩疏言:「民有餘糧,孰肯輸之官?縱令首告,有仇則訐,無仇則隱,所得必少。且民不敢以糧入市,新人糧不足及舊人之無糧者,皆無所於糴。不若出令,無問滿、漢、蒙古官生軍民,人輸糧一斗。有糧者固易辦,無糧者人出銀二三錢,糴以輸官,亦無大損;其有餘糧原輸官者,獎以升賞:此兩便之術也。」崇德元年五月,授內國史院學士。纂太祖實錄成,得優賚。漢軍旗制定,隸鑲藍旗。七年,兼牛錄額真。

順治元年,從入關,七月,命以右副都御史,巡撫河南。時李自成西走,其黨掠衛輝、懷慶間,而原武、新鄉諸縣盜競起。繡錦至官,與總兵官祖可法等謀防禦。疏言:「自成之眾二萬餘,攻懷慶甚急。明尚書張縉彥等擁兵河上,副將郭光輔、參將郝尚周不應徵調,叛而為寇。明兵在南,流寇在西,請發兵靖亂。」上已令豫親王多鐸為定國大將軍,帥師南征,令取道河南捕治群寇。繡錦亦遣衛輝參將趙士忠等攻破婁兒寺盜寨,擒其渠。繡錦請以河北荒地萬餘畝令守兵屯墾,得旨俞允。

二年十一月,擢兼兵部右侍郎,總督湖廣、四川。湖南諸州縣尚為明守,自成從子錦擁眾降於明,侵湖北。繡錦至荊州,錦率眾來攻。順承郡王勒克德渾自江寧來援,錦敗走。勒克德渾師還,錦又至,繡錦帥師御之,錦復敗走。有胡公緒者,據天門八百洲,四出焚掠,戕署鹽道周世慶,繡錦遣中軍副將唐國臣、署總兵楊文富等分道討之,獲公緒,毀其巢。三年六月,遣總兵官徐勇擊破麻城山寨,獲其渠梅增、周文江;岳州署總兵官高蛟龍等擊斬滿大壯,獲龍見明等。九月,明總督何騰蛟寇岳州,繡錦遣將御之,多所斬獲。十月,遣總兵鄭四維等定夷陵、枝江、宜都三州縣。

四年,定南大將軍恭順王孔有德等略湖廣,取長沙、衡州、寶慶、辰州諸府。繡錦條奏增設鎮協,下部議行。王光泰以鄖陽叛,上命侍郎喀喀木帥師討之,繡錦與合兵克鄖陽,光泰走四川。五年,金聲桓以江西叛,湖南騷動,常德、武岡、辰、沅諸府州復入於明。繡錦疏留喀喀木駐荊州,而分遣總兵徐勇、馬蛟麟等分守要隘,屢敗明將馬進忠等。上復命鄭親王濟爾哈朗共率師徇湖南,漸收諸郡縣。繡錦疏請移降卒腹地,毋使師還後復為餘孽煽誘,上嘉納其言。九年七月,卒,贈兵部尚書。

弟繪錦,自通政司理事官再遷,終貴州巡撫。

興,亦遼東人。太祖時,以諸生選直文館。事太宗,授秘書院副理事官。崇德間,遷都察院理事官。漢軍旗制定,隸正黃旗。順治元年十月,命以右副都御史巡撫天津。李聯芳、張成軒為亂滄州、南皮間,興與總兵官婁光先帥師討之。成軒等將遁出海,師已扼海口,乃驚潰,投水死者強半。興復遣兵捕治,斬渠宥脅,盜盡散。疏言大沽海口為神京門戶,請置戰船為備,下所司議行。二年四月,移巡撫陜西。陜西方被兵,民多流亡,興招徠撫綏,疏述其狀。上旌以冠服、裘馬。三年,肅親王豪格帥師自陜西徇四川,師未至,有孫守法者,為亂於興安;賀珍又以漢中叛。興移潼關兵戍商州,密檄漢羌道胡全才為備,待師至,悉戡定。興疏請隴州置兵,臨洮、鞏昌留屯軍防邊,皆報可。四年四月,以疾乞罷。十年八月,復起巡撫河南。未上,卒,贈兵部侍郎。

來用,亦隸鑲藍旗。授工部啟心郎。順治初,再遷山西布政使。三年,師略四川,三月,授來用戶部右侍郎,總督山西、川、陜糧餉,駐西安。疏言陜西兵後民困,請蠲荒徵熟。山西銅缺,鑄錢多,定值過低,商不前,請酌增。四年,疏言漢南遘賀珍亂,蹂躪荒殘,請恩賑,並敕部儲備肅親王還師餉糈。五年,疏言河西回亂,運河阻,諸軍南討,請發湖廣漕供餉。又言漢中屯軍歲餉數十萬,請專設餉司。皆如所請。六年,疏言兵出鎮,贍其孥如所食糈。司兵者請自離伍日起,司餉者請自到軍日起,持異議,請定例畫一。部議以應徵日起,中途逃亡,不得濫與。八年正月,御史聶玠劾來用專倚中軍王楨,自隳職業,部議左遷,援赦免。七月,裁缺。九年,命巡撫順天。十年,移駐河間。十一年,以定南王孔有德喪歸,其屬吏或格詔書不出迎,坐左遷。十四年,改授河南大梁道。尋卒。

丁文盛,廣寧人。初為明諸生。天命六年,歸太祖。天聰間,授兵部啟心郎。七年正月,偕同官趙福星疏言:「師行戒毋擾民,子女玉帛,秋毫無犯,但發倉庫以佐軍興。攻關東八城,當先其易者,後其難者。舍寧、錦、前衛,但得其他小城,因糧以度師,進攻山海。舊制編民為兵,十丁而取一。當令諸甲喇及領屯將吏,慎選年事盛強、身家相稱者,毋許以他人代。永平砲兵衣食不足,宜擇其技精者授千總,督演習,食糈視鑄砲之工。哈喇沁降者置遼河外,慮且逃亡,宜移屯腹地。」

及孔有德。耿仲明來降,五月,文盛、福星上疏請水陸並進,攻山海,取旅順;並言:「毛帥來歸,令金、漢官吏出羊、雞、鵝、米、肉以贍其兵。臣慮新人未必肥,而舊人已不勝瘠。復使市馬,力尤不能舉。若用八門稅,一二月已足。」孔有德等,毛文龍部曲,文龍嘗使冒其姓,故是時猶稱毛帥。及旅順既下,七月,文盛、福星復請城旅順,加意防守。考績,授世職牛錄章京。

順治初,從入關,授山東登萊兵備道參政。二年六月,授右僉都御史,巡撫山東。濰縣盜張廣為亂,以數千人攻萊州,文盛令游擊馮武鄉等討之,戰三埠,再戰紅山口,斬廣黨尼思齊、趙明春。廣走平度,游擊楊遇明逐之,及於徐里甿,射廣殪,殲其徒。明季馬政弛,驛馬缺,求諸民,文盛疏請以餘存驛站銀市馬。明季增牙稅及他雜稅,文盛疏請罷。臨清、東昌、平山諸衛置兵五千人,虛額逾半,文盛疏請減,留二千人,節餉令州縣募壯丁逐捕盜賊。別疏又請教有司清刑獄,禁獄卒毋虐囚。皆下部議行。三年,盜發茌平、高唐諸縣,文盛請兵,上遣副都統覺善率師捕治。四年,文盛被彈事不勝任,左遷河南按察使,稍遷福建布政使。七年,卒。

文盛子思孔,字景行。順治九年進士,選庶吉士。四遷,授陜西漢羌道副使,康熙二年,巡撫賈漢復劾思孔追胥役蝕糧草逾限,左遷河南開封府同知。思孔詣通政使自列胥役蝕糧草,獄瘐家罄。事上巡撫,巡撫久乃入告未嘗逾限,下總督白如梅勘實。復授直隸通薊道。直隸未設布政、按察兩司,八年,巡撫金世德請增置保定守道領錢穀,以授思孔。再遷江南布政使。時吳三桂亂方定,師行江西、湖廣,思孔主餽運,應期不愆。禁旅還自福建,庀役具舟,科量悉當。修蘇州府學,置育嬰堂、養濟院,諸政皆舉。二十一年,遇大計,總督於成龍以思孔督賦未中程,不得舉卓異,特疏薦廉能,上命準卓異。二十二年,擢偏沅巡撫。偏沅所領七郡,溪山環互,民、僚雜處,反側初定,餘孽每煽亂,思孔撫其渠,群盜漸散。復嶽麓書院,御書旌楣。

二十七年,移撫河南,方上,而有夏逢龍之亂,復移撫湖北。逢龍私自署置千總胡耀乾,參將李廷秀,馬兵周凱、萬金鎰皆號總兵,守備林德號副將。上命振武將軍瓦岱帥師討之,趣思孔詣荊州主餉。思孔以武昌倉庫皆陷賊,諸軍餉乏,乃發河南庫帑,護詣襄陽,諸軍資以濟,疏報稱旨。七月,瓦岱師至,蹙賊黃州,誅逢龍,而耀乾等尚據武昌拒命。思孔至漢口,具舟渡江,單騎叩漢陽門,呼耀乾出見,耀乾等遂降。思孔入武昌,數耀乾等罪而誅之,並戮所置巡撫傅爾學、布政婁方順、驛道金奇功,凡八人,武昌遂定。九月,復設湖廣總督,以命思孔。陳龍越八者,逢龍之徒也,二十八年五月,謀為變,期夜半。思孔晡始聞,執陳龍越八戮於市,他悉不問。設水師,分戍武昌、荊州、岳州、常德。嘗歲饑,便宜發帑市米江西,平值以糴。

三十三年四月,移督云、貴。八月,卒。

祝世昌,遼陽人。先世在明初授遼陽定邊前衛世襲指揮,十數傳至世昌,為鎮江城游擊。天命六年,太祖克遼陽,世昌率三百餘人來降,仍授游擊,統其眾。命董築沈陽、遼陽、海州三城,事竟,授沈陽城守昂邦章京。

天聰五年,從征大凌河。六年,太宗閱烏真超哈兵,賚諸將,世昌與焉。尋遷禮部承政,授世職參將。七年七月,克旅順。世昌疏請大舉伐明,謂:「攻城當專用紅衣砲,國中新舊三十餘具,沈陽留四具,城守已足,餘悉載軍中。砲多則糜藥亦多,藥局制藥,硝丁淋硝慮不足於用。旅順新獲硝磺,宜以其半送水審陽制藥。師行克城邑,當得練達謹慎之吏,不求小利,不貪財賄,乃能戢民心、保疆圉,宜預選令從軍備任使。用兵當兼奇正,輕兵先發,奪人畜,掠儲峙,然後整軍挾紅衣砲自大道徐進。」上尋遣貝勒阿巴泰等將二千人略山海關外,未深入,引還。

崇德七年,疏請禁俘良家婦鬻入樂戶,上諭都察院承政張存仁、祖可法曰:「世昌豈不知朕禁樂戶?而為此疏,不過徇漢人,藉此要譽耳。朕度世昌身在我國,心猶鄉明。世昌果忠於明,明以元功臣田、劉、張三姓之裔隸樂戶,世昌何不聞有言乎?朕視滿、蒙、漢若一體,爾等同心輔國。譬諸五味,貴調劑得宜。若各相庇護,是猶咸苦酸辛不得其和。爾等徇世昌而不舉劾,咎在爾等。曾子曰:『吾日三省吾身。』爾等能如曾之省身,則何過之有?」旋命固山額真石廷柱、馬光遠與諸漢官會鞫,坐世昌死。其弟世廕同居,知其事,啟心郎孫應時為改疏稿,皆死。禮部承政甲喇章京姜新、甲喇章京馬光先見疏稿稱善,當奪職坐罰。上命誅應時,而貸世昌、世廕,徙邊外席北。新解承政,與光先皆貰罪。

順治二年,召還,隸漢軍鑲紅旗。四年七月,授右副都御史,巡撫山西。時盜發盂、五臺、永寧、靜樂諸縣,世昌遣兵捕治。五年十二月,上遣英親王阿濟格等戍大同備邊,總兵官姜瓖疑見誅,遂叛。世昌檄諸縣兵還守省城,瓖遣兵陷朔州、岢嵐,攻代州急,世昌帥師赴援,疏請發禁旅出居庸取大同,分兵出紫荊關,至代州濟師。上命阿濟格等討瓖,別遣敬謹親王尼堪等帥師鎮太原。六年正月,瓖將姚舉等掠平原驛,戕冀寧道王昌齡,下忻州。固山額真庫魯克、達爾漢、阿賴等破舉眾石嶺關,舉棄忻州走。既,復襲陷寧武,萬鍊踞偏關,劉遷破繁峙、靜樂及交城東關。世昌疏趣援,尼堪師至,出攻寧武,逾月未下,移師向大同。瓖黨以其間攻陷保德、交城、石樓、永和諸縣,世昌復請發禁旅守太原、曲沃。李建泰以大學士罷歸,謀應瓖叛,世昌得其手書以聞。會瓖為其將楊振威所殺,以大同降,師討定汾、絳、潞安、永寧、寧鄉諸州縣。建泰與瓖將李大猷等入太平,師從之,建泰等亦降。是歲平陽盜虞允、韓昭宣為亂,攻陷州縣,應瓖,陜西總督孟喬芳將兵擊破之,世昌以聞。山西底定。七年,卒,謚僖靖。

天聰間,有徐明遠者,疏陳時事,因言:「軍中得良家婦,上悉令歸故夫。此誠如天之仁,禹、湯、文、武殆莫能過。臣竊見遵化、永平俘得良家婦,其主貪利,輒鬻入樂戶,得無損上仁聲?且樂戶既多,吏民游冶,損財物,耗精血,於國無益。買良為賤,古著於令甲,今豈可任其所為而不之禁乎?」明遠蓋自永平降者,事互見張存仁傳。世昌繼以為言,乃得罪。

論曰:順治初,諸督撫多自文館出。蓋國方新造,用滿臣與民閡,用漢臣又與政地閡,惟文館諸臣本為漢人,而侍直既久,情事相浹,政令皆習聞,為最宜也。文盛、世昌未嘗直文館,而自太祖朝已來附,抒讜效忱,遂與文奎、棲鳳、國柱輩分領疆圻,各著聲績。天聰間諸言時政者,並以類附見。當時章奏,流傳蓋鮮,經綸草昧,毋俾終湮也。


\end{pinyinscope}