\article{列傳二百七十九}

\begin{pinyinscope}
忠義六

齋清阿童添云彭三元蕭捷三周清元蔡應龍

蕭意文周福昌彭志德李存漢杜廷光等賴高翔畢定邦

劉德亮陳大富陳萬勝郭鵬程王紹羲王之敬

陳忠德劉玉林等黃金友麟瑞蔡東祥鄒上元

郝上庠張遇祥兄張遇清曹仁美毛克寬邢連科

田興奇田興勝馬定國

齋清阿,字竹塍,納喇氏,滿洲鑲黃旗人。早喪父,母氏撫之。家貧,月夕至撤去鐙火。膂力過人,取巨磚置平地,拳擊之,立碎。以善射得名,嘗隨扈盛京,命射,中靶,賜克食。道光六年,發閩、浙,以都司用,補浙江杭州營都司,為總督孫爾準所賞。英吉利船入犯,獻燒船退敵策,不用。遞擢至廣東肇慶協副將。三十年,廣西金田亂起,檄令率兵至兩粵交界開建縣堵御。匪二千餘,船四十餘,從縣北金莊偷越,督兵進擊,斬其酋二人,餘敗竄,自是不敢犯境。

咸豐元年,廣寧屬江穀屯積匪滋擾,廣東兵會剿,竄廣西懷集一帶,至賀縣屯聚。廣西大吏以廣東各官惟知驅賊了事,移文廣東詰之。總督徐廣縉檄肇慶府知府蔡振武、參將左炘赴廣西剿賊,道出開建。齋清阿以越境追賊,須重兵制其死命,原統駐劄開建之師同赴廣西。守備薩國亮以無越境剿賊之責諫,齋清阿奮然曰:「賊勢蔓延,若畫分畛域,何以紓民困而報國恩?吾雖逾七旬,精力未衰,正臣子戮力時也!」遂與振武等督兵入廣西境。

至賀縣鋪門圩,復進至家坪,距賊巢裏許,賊突出數百人撲營,官軍迎擊之,斃賊數十人。賊退回松圩,在圩內施放大砲,官軍避入田中,火藥槍繩盡濕,賊復分隊挑戰,抄官兵後,爇火燒山,齋清阿督兵以槍擊斃賊七十餘人。值日暮,孤軍無援,深入被困。事急,掣佩刀連刃數賊,肩中火箭,猶拔箭作戰,手刃執旗賊目一人,刀折,歿於陣,手握斷刀牢不可開,怒目上視,懍然如生,時咸豐元年四月也。恤贈總兵,賞世職,予謚威烈。

童添雲,字鎮銘,湖南平江人。以貧,偕弟必發走長沙為戰兵。饒膂力,能開五石弓,射必命中。道光二十二年,從提督楊芳出師廣東,一日,夷撲城,有營在城外,芳欲調入,火攻甚熾。募敢死者持令縋城出,添雲應募,少選,兵皆入城,芳奇之。咸豐二年,粵賊圍長沙,與必發從守城,圍解,添雲語人曰:「吾觀諸將中能稱將才者,惟塔都司與彭千總耳。」塔即塔齊布,彭即三元也。會塔齊布練標兵,添云隸麾下,三元時別將一營,深相結納。茶陵土寇起,塔齊布命解火藥,期三日,添雲逾宿至,吒曰:「何速也?」添雲曰:「遲恐有阻,則貽誤大矣。」

四年三月,賊陷湘潭,塔齊布帥標兵等拒戰,添雲與必發從。時賊踞城外民廛,塔齊布好輕騎觀賊,策馬入黃龍巷,必發先驅,巷狹而長,甫入,賊突出刺塔齊布,必發急以背承之,中肩,塔齊布跳而免,必發死之。越二日,師大捷。湘潭平,擢守備,或賀之,添雲憤然曰:「賊戕吾弟,雖官至一品,弗原也!原生啖賊肉耳,何賀為?」遂由湘潭轉戰至岳州,從克岳州,擢都司。克武昌,擢游擊。克興國、大冶、黃梅、廣濟,破田家鎮,擢參將。

添雲身長面赤,額以下痘瘢如錢。橫矛陷陣,槍丸如雨,不少卻。賊見其旗,輒相語曰:「童麻子至矣!」則皆走。五年十二月,攻九江,城砲傷胸,舁歸營,卒。發其笥,衣數領而已,同營皆痛哭之。詔贈副將,謚壯節,附祀塔齊布祠。

彭三元,字春浦,湖南善化人。道光二十五年武進士,用衛守備,借補千總。咸豐二年,粵匪竄湖南道州,勾結會匪犯東安,三元偕署守備周祿兩次迎剿,殲賊多名。三年,敘省城防堵功,用守備。侍郎曾國籓檄寶慶知府魁聯募寶勇千名,分屬三元五百人。旋平江西泰和土匪於茶陵、安仁。四年,隨副將塔齊布剿賊湘潭,復其城。

六月,進攻岳州,是時湘潭潰賊由靖港竄岳州,增壘設卡,為久抗計。巡撫駱秉章暨國籓會督戰船,塔齊布亦統陸路官軍,約期並進。先以疑兵誘賊,賊擁至,觸伏盡殪,擊沉賊船百餘隻,遂復岳州。七月,賊水陸大至,官軍迎擊,悉焚其船,其由陸路來犯者,三元沿岸截擊,殲賊目一、餘匪百餘,生擒四十餘名。嗣賊由高橋撲鳳凰山大營,塔齊布督率將弁進剿,三元出奇抄截,分路迎擊,斃賊六百餘名。八月,匪於崇陽交界設卡抗拒,九月,三元偕候選知府羅澤南分路進攻,抵其壘,痛殲之。

尋隨塔齊布由嘉魚轉戰而前,所向披靡,直抵武漢。塔齊布分軍三路:一攻武昌,一攻漢陽,一由水路進剿。時風勢順利,官軍縱火,焚賊船數十隻,乘勢奮擊,斃賊無數。漢陽賊大懼,棄城遁,武昌賊亦遁,遂復之。三元並截於洪山要隘,斬馘甚多。十月,偕澤南進屯馬嶺坳,直偪半壁山。賊悉眾至,官軍徑搗賊壘,賊狂竄,三元等分途截殺,斬偽丞相林紹璋及偽將軍指揮等。越數日,賊復由田家鎮渡江來犯,塔齊布擊卻之,列隊江幹。賊偵官軍盡赴下游,徑從上游登岸,將掩襲澤南老營,三元馳至,率眾奮擊,追至牛礶磯,毀其船,斃賊百餘,餘眾潰遁。

時三元累功擢至游擊,捷入,進參將。旋隨塔齊布進攻黃梅,時湖北踞匪招安慶援賊並入廣濟,塔齊布擊走之。賊敗竄黃梅,官軍追剿至大河埔。十一月,軍至黃梅,塔齊布偕澤南攻北門,三元列陣橋西以遏賊沖,塔齊布、澤南自城北溝港中取道入,三元等亦由城西越二橋,從柵門躍入。賊驚竄,官軍四面兜圍,其由營壘突出者,殲滅殆盡。克黃梅,移剿九壟驛,擒偽丞相余福勝。大軍復渡南岸,攻九江城,三元戰績最多。

五年二月,武昌復陷。八月,塔齊布病歿,三元副澤南回援武昌。九月,復通城,進師崇陽,賊夜遁,遂克之。國籓疏保堪勝總兵人員,三元得記名以總兵用。會湖南防兵戰蒲圻羊樓峒失利,澤南飭諸營移駐羊樓峒,遏賊上犯,獨率三元及湘副中營官李杏春駐崇陽,於是乘勝攻蒲圻,斃賊數百。賊首石達開率賊大至,三元等分路抵御,鏖戰多時,斃賊百餘。翌日,賊悉眾來攻,繞營三匝,眾寡不敵,遂歿於陣。贈副將銜,附祀塔齊布專祠,謚勤勇。

三元忠勇識大體,嘗戰濠頭堡,忽訛言子瑾光陣亡,左右以告,三元急止之曰:「速擊賊!無以吾子阻士氣。」督戰益急。陣歿之日,將出隊,馬忽踶齧,三上三墜,眾以為不祥。杏春亦同時歿於陣。

蕭捷三,字敏南,湖南武陵人。由武舉投營效力,擢千總。咸豐二年,以守省城功遷守備,署湘陰千總。四年,賊陷湘陰,坐免。曾國籓奇其才,檄領水師。既克岳州,沿江進剿。閏七月,敗賊高橋、城陵磯,進攻擂鼓臺,捷三偕李孟群、楊載福等搜捕兩岸伏賊,俘馘甚眾。乘勝追至六溪口,平賊壘,毀賊艘殆盡,水陸各軍遂進抵嘉魚。以功復職,授永綏協守備。八月,進規武漢,水師分兩隊,捷三率戰艦為前隊,冒砲駛至鸚鵡洲,擲火球焚沿江賊柵,賊不支,揚帆下遁,急駛出賊前,毀其輜重。渡江攻漢陽朝宗門外土城,偕載福等殊死戰,焚漢口以內賊船皆盡。會陸軍破花園賊壘,武昌、漢陽同日復,進都司。時餘賊尚據襄河,乃扼新灘口,溯流進剿,賊艘千餘,連檣下駛,迎擊敗之。追至上游,突有悍賊數舟,用火彈撲營,灼捷三頭面手足幾死,仍裹創力戰,追剿二十餘里。襄河肅清。

尋偕彭玉麟敗賊蘄州釣魚臺、骨牌磯,遂大破田家鎮,逾九江,直趨湖口。先是江西吳城戰艦數百淪於賊,賊實沙石沉湖口,截江路,於對岸梅家洲築城,環列巨砲,拒官軍。十二月,捷三駕火舟徑沖賊柵,燔賊舟百數,乘勝駛入內湖,泊大孤塘。游擊孫昌國、黃翼升等出賊不意,焚內湖賊舟二百餘。賊益囊土塞湖口,水涸,師弗克歸。賊以小艇雜外江巨艦中,潛縱火,水師驚潰,國籓大營泊九江北岸,亦被焚襲。捷三陷入內湖,內外隔絕,以忠義激勵將士,軍心彌固。

五年,國籓入江西,益大治水師,疏薦捷三忠勇,晉游擊。四月,敗賊雞公湖,復都昌。五月,賊由大孤塘上犯,捷三逆擊,屢敗之青山,奪回舊所失帥船及賊魁艨艟巨艦。秋七月,國籓檄平江營陸軍渡湖,約水師夾攻湖口,克之。賊退保石鐘山堅壘,捷三率十七舟銳進,遙見陸軍圍石鐘,氣益奮,方沖越賊艘,上下夾擊而下,石鐘山、梅家洲賊壘砲齊發,捷三中砲死。詔贈副將,謚節愍,賞世職。九年,建石鐘山水師昭忠祠,祀死事將士三千餘人,捷三為之冠。

周清元,字玉泉,湘陰人。世業農。時與群兒角戲於牧場,立表數十步外,飛石命中。掘溝數丈,跳越之,能往復十次,群兒皆出其下。同里左明志以拳勇鳴於鄉,招致門下,傳以技,言:「天下幸無事,有事,則清元暨子光培皆驍將也。」咸豐二年冬,賊自益陽竄臨資口,清元混跡市中,默識其軍卒舟艦糧械之數,聞提督向榮尾追至八字哨,相距三十里,遮道見榮曰:「廣西能戰賊,不過三千餘人,餘皆裹脅也。臨資口四面阻水,湘包其東南,資繞其西北,數十里平原,渺無障蔽。賊所擄民船笨重不易行,一炬可盡也。請以兵扼要路,使不得偷渡,賊糧盡,旬日當餓死,何怯而不為?」榮不省,固請,榮叱之退。賊遂從容駛去。及東南糜爛,清元嘆曰:「賊自走絕地,向公縱之去,能辭咎耶?」

三年,國籓大治水軍,清元與光培同應募,隸千總楊載福部下。載福嘗為湘陰汛外委,夙才清元;捷三官湘陰時,亦知清元驍勇,故戰必與俱。四年,賊踞湘潭,載福等帥水師進剿,時賊掠民船數千,旗幟蔽兩岸。水軍本新募,又經嶽州新挫,望之奪氣。清元言於介眾曰:「民船不能戰,一炬可盡也!」遂隨載福猛擊,逼賊巨艦。賊倉卒以瓷碗來擲,清元手接而回擲之,中賊渠。載福躍登賊舟,清元隨聳身入,用火球分擲左右舟,風烈火大熾,賊爭赴水死。從軍見火起,急槳爭進,分途縱火,燔賊船皆盡。以功拔充哨官,隨攻克岳州、嘉魚。八月,攻克武漢,受重創,力疾進剿蘄、黃、田家鎮皆有功。五年,武昌再陷,隨彭玉麟回援,駐金口,扼上游。每戰必身先,不受創不止。

六年,胡林翼攻武昌,經歲不下,議先斷糧路困賊,乃檄水師清江面賊船。清元時典水師副後營,率先下駛,越武、漢二城,直駐沙口,屢敗賊。駐沙口八閱月,賊糧斷,城賊乃困。十一月二十二日,清元由沙口帥師上擊,先破賊浮橋,斷其鐵鍊,大戰黃鶴樓下,被砲傷,力戰不退,各營繼之,遂克武昌。未幾,以創重卒於軍,年二十有六。清元時已洊保參將,詔視副將例議恤,謚貞愍,賞世職。石鐘山昭忠祠,捷三冠而清元次之。

蔡應龍,江西樂平人。由行伍洊升千總。道光三十年,升廣西永寧營守備。咸豐元年七月,提督向榮擊賊於東鄉,馬中砲斃,應龍以所乘馬授榮,步行接戰,立斃賊三人,榮乃得免。欽差大臣賽尚阿以聞,授梧州都司。二年,晉游擊。

三年五月,江寧賊掠商船,泊觀音門外,時榮官欽差大臣,飭應龍偕知府陳景曾馳往,諭以大義,船戶各憬悟聽命,自焚其船,押船賊無一得脫者,計焚毀及逃竄千餘艘,遣散水手萬餘人。時賊踞城外街,與雨花臺相犄角,應龍潛師過雨花臺,至街尾縱火燒賊壘,賊驚遁,官軍截擊之,斬馘無算。

四年,升全州營參將。五月,大兵圍偪江寧,賊拒守不出,應龍登鍾山,望太平門外賊勢,賊見官軍少,包抄而上,應龍且戰且退,以伏兵擊賊,大敗之。時賊船麕集於江北七里洲,應龍駕小船入,潛薄北岸,射火箭毀其船二十,而大隊賊船適至,應龍舍舟陸戰,燃砲擊沉賊船數只。閏七月,擊賊洪武門,斬首數百級,復連敗賊高橋門等處,三日斃賊數千。賊猝於兩花臺、洪武門突出,撲七星橋營壘,應龍擊卻之。旋升楚雄副將。

十月,賊造木簰,上施木城,列巨砲,沿南岸下駛,至八卦洲擱淺。應龍乘夜發火燒之,賊爭赴水死。官軍突煙上簰,擒斬餘黨凈盡。時浦口九洑洲久為賊踞,以梗官軍,陸軍攻之,賊船來援。應龍統帶紅單、拖罟各船截擊,賊敗遁,官軍遂奪九洑洲。十一月,赴秣陵關查勘地勢,還言於榮,請乘虛襲板橋賊營,既可援應水師,更可抄出雨花臺、上坊橋諸賊營之後。遂率千人間道襲擊,街外賊敗走,餘賊憑壘死守,復急攻之,焚其壘。

五年九月,官軍為蕪湖援賊牽制,應龍攻奪米家嶺賊壘二、廣福磯賊壘四。賊旋於丁橋一帶袤延築壘,其地則外圍塘港,中間小路。應龍率師攻擊時許,遽麾軍退,誘賊過而擊之,殲溺無算。

六年,江寧賊分股至楊家壩、陳莊築壘,欲窺倉頭。應龍與總兵張國樑分路沖擊,斷賊為二,賊敗竄歸巢。三月,督兵攻拔炭渚、下蜀街、太平橋一帶賊壘七,毀沿江賊卡十餘處,殲斃四千餘人。五月,赴援寧國,戰歿。榮以聞,詔以應龍在窯灣力戰身亡,命優恤,謚勇介,給世職。

蕭意文,字章甫,湖南湘鄉人。初隸羅澤南麾下,從征江西、湖北,累以功至參將。咸豐八年,李續賓征皖北,從克潛山、太湖、舒城、桐城,進攻三河鎮。三河鎮者,舒、廬適中地,賊屯糧械以濟廬州、金陵者也。築大城,環以九壘,備嚴甚。續賓銳意攻取,十月,分三路進剿,意文攻河南老鼠夾賊壘,冒砲石先進,各營繼之,縱火焚其壘,賊大亂。意文受砲創,殊死戰,奪柵入,九壘皆下,賊盡殲,無脫者。官軍傷亡千餘,意文以創重歸營卒。詔贈副將,謚剛勇。續賓部將以敢戰著、同死三河之難者,彭友勝、劉神山,均見續賓傳。

周福高,字子祥。亦先從澤南援剿江、鄂。續賓接統湘軍,福高無役不從。以小池口、梅家洲諸戰尤用命,累官至參將。軍抵三河,援賊麕至,諸將知戰必敗,無鬥志。福高憤然曰:「男兒效力疆場,寧可逆計禍福,敗則死耳,吾輩畏死不來矣!既至此,敢惜死隳壯志!」遂慷慨赴敵,力戰而歿。詔贈副將,謚敏烈。

彭志德,字道一。隸湘軍,每戰必為前驅,恥居人後。累官至參將。三河之役,諸營皆潰,志德率所部貫賊陣突出,死者過半,身受重創。走入中右營,與副將李存漢等竭力死守,越三日,營陷,死之。詔贈副將,謚武烈。

李存漢,以鄉勇隨剿廣西、江西、湖北等省,累官至副將。未抵三河鎮之先,進攻舒城者凡五營,並西北面賊壘,獨存漢一軍攻東南城門。壘既破,城賊以存漢故,弗能救,旋棄城遁,追斬無遺。續賓被圍三河,調桐城戍兵未至,事迫,誓必死,存漢等皆跪泣,原從死以報國。續賓陷陣卒,存漢與道員孫守信等堅守待援,力持三晝夜。營陷,存漢率壯士沖賊陣,越壕走保桐城。賊大至,城破,存漢巷戰歿。詔贈總兵,謚果愍。福高、志德、存漢均湘鄉人,並附祀續賓祠。

同時游擊杜廷光、王懷興,均湘鄉人,均以苦戰陣亡。

賴高翔,福建和平人。少入行伍,累功至千總。咸豐三年,潮州小刀會匪糾土匪陷漳州,高翔從總兵饒廷選討平之,擢漳州城守營都司。四年,漳浦古竹社匪戕官擾亂,築石堡自固,官軍久攻未拔。高翔偕龍巖游擊馬至元、漳州鎮左營游擊惠壽等冒雨直搗賊巢。賊固守不下,高翔夜偕勇首畢定邦潛師梯登,克石堡,斬獲無算。餘匪乘夜奔竄,窮追至海汊,皆赴水死,漳州平。

六年,江西邊錢會匪糾粵賊陷新城、貴溪,謀攻廣信。知府沈葆楨以血書告急於廷選,高翔時從廷選駐防玉山,倍道赴之。廷選軍素無部伍,唯高翔與定邦以敢戰名,行不齎糧,止不為屯,故赴急易。軍至廣信,寇旋至,背城擊賊,屢破之。賊來益眾,幕府文員皆懼,慫廷選還軍,高翔、定邦怒曰:「諸君怯,何如勿來?今我在城中,賊不知我虛實,以我能援廣信,後路必有大軍。若棄城遁,賊知吾兵寡,氣益壯,追殲立盡,尚何浙境之可歸耶?當為諸君決死戰,翼日觀吾破賊!」乃偕定邦開城縱擊,自晨至日昃,盡毀城外賊壘,斃賊三千餘,斬渠帥數人,賊駭遁。論功以游擊用。廣信圍既解,廷選還浙,高翔留駐廣信。

明年七月,樂平賊踞縣城,將軍福興檄高翔往剿,賊眾五六千,分道抗拒。高翔督都司馮日坤、勇目刁士樞等迎擊,賊殊死鬥,高翔突陣負創,戰益力。擊斃黃衣賊目,橫沖賊營,賊大潰,乘勝蹙之,生擒偽指揮遜天侯等,餘賊竄景德鎮,遂克樂平。移防弋陽,八年二月,補游擊。臨江餘寇合撫州賊趨廣豐,福興退駐廣信,高翔自弋陽聞警赴援,轉戰至鉛山之石塘,賊勢益盛,兵寡援絕,力戰死。贈副將,給世職。

畢定邦,字康侯,山東淄川人。以武童投效漳州軍營。小刀會匪陷漳州,紳民輸款,游擊饒廷選約內應,定邦率建勇助剿,戰最力,從復府城。以次討平雲霄、漳浦賊匪,斬獲尤眾。復討平仙游會匪,總督王懿德檄定邦率仙游得勝之師,間道馳剿。冬夜四鼓,蛇行進,將賊堡附近釘桶竹簽拔除,黎旦,奮勇梯登,與高翔同有功,復與高翔同解廣信圍,累擢至參將。

七年,粵賊竄圍建寧,分黨陷邵武、浦城,定邦奉檄援閩,率部眾疾趨抵甌寧,直前搏賊。賊由建陽逃竄,復糾鄉團夾擊。賊斷七星橋抗拒,令鄉團伏山腰,張幟以疑之,躬率勁旅迫橋,以輕兵由淺處渡河,前後合攻。賊殊斗,黃衣悍黨數十,屢出蕩決,盡殪之,賊大奔。毀賊壘十一,焚逆舟六十,直逼建寧臨江門。大股賊復來犯,縱擊敗之,斬悍目六,斃賊數千,踏平城外賊壘,遂解建寧之圍。進搗邵武,克之,遷參將。復督鄉團剿平浦城之賊,閩邊肅清,以副將升用。進剿白水墩賊匪,中彈,卒於軍,年二十六。給世職,謚愍烈,與高翔同附祀廷選祠。

劉德亮,湖南長沙人。咸豐四年,投效水師營,隨道員褚汝航等破岳州踞賊,又隨知府彭玉麟克漢口鎮。五年,剿賊武、漢、蘄、黃間,大小數十戰,德亮皆沖鋒陷陣,又隨軍斫斷橫江鐵鎖,擊沙洲爭渡之賊。嗣偕都司胡友亮堵賊童司牌,焚內湖賊艇,並燒浮橋。尋與游擊孫昌凱會剿黃梅踞賊,破其要沖。八年,福建陸路提督楊載福等攻九江,發地雷,轟塌城垣,賊由龍口河傾壁出竄,德亮率所部登岸截擊,殲數百人,復府城。

又隨載福軍進規安慶,先破大通賊壘。趨銅陵,德亮麾隊攻其北,直偪城下,身受七傷,猶裹創仰攻不退。池州賊黨萬餘來救安慶,擄民船渡至樅陽,載福令隨總兵陳金鼇等馳往截剿。師至羅塘洲嘴,樅陽港內木椿鐵鍊層層攔截,泊賊船百餘。副將王明山等登洲轟擊,督勇鳧水過港,賊驚潰,官軍盡焚其船。遂率隊攻樅陽街尾,金鼇攻樅陽街頭。賊排砲抗拒,德亮鼓眾飛槳進截新河鐵金柬,麾隊登岸直攻中路,副將李朝斌抄賊壘後,官軍三路進偪賊壕,平其五壘,逐北二十餘里,賊尸枕藉。累功擢至參將。

十年,再攻樅陽,破鮑家村賊壘,斬晏家塘賊魁。時池州賊以殷家匯為犄角,載福率步隊往攻,而令德亮等以舢板夾擊,斃匪甚多,獲槍械馬匹稱是。殷家匯賊壘既平,乘勝攻池州,德亮由東門外卡緣墻斬關入,破其石壘,盡毀東門外房屋,復分攻南門,獲逆艇八。德亮奮不顧身,執旗先登,中砲,歿於陣。載福上聞,詔令議恤,謚威毅,給世職。

陳大富,字餘庵,湖南武陵人。起行伍。道光末,以外委從提督向榮剿賊廣西,回援長沙,追賊武昌,屢著戰績,洊擢常德協都司。進剿江寧,轉戰蕪湖、鎮江間,以功賞花翎。咸豐七年,隨提督鄧紹良復寧國屬之灣沚、黃池,進游擊。尋援浙江,敗賊金華、處州,除參將。賊竄婺源、石埭、太平,先後擊走之。以從復涇縣,拔南陵,擢副將。八年十一月,灣沚師潰,紹良死,大富左次南陵。明年四月,賊犯南陵,百計環攻,不得逞,十年三月,圍始解。帝嘉其功,除皖南鎮總兵。

五月,偽侍王李世賢圍寧國,分黨攻金壇、南陵,時提督周天受守寧國,總兵蕭知音、參將周天孚等守金壇,大富仍守南陵。賊眾數十萬,官軍勢不敵,各血戰死守待援。七月,金壇陷,賊屠其城,天受知寧城不守,則盡出城中兵民數萬令各逃生,自誓以身殉。寧民扶老攜幼走南陵,大富開門納之。八月,寧國陷,賊圍南陵益急,城中食且盡,大富以忠義激勵軍民,皆誓死弗去。夜遣壯士縋城出,乞援於水師,前後數輩為邏賊遮獲,最後乃得達。

時提督楊岳斌統水師奮袂起,九月,揚帆進泊魯港,聲言攻蕪湖,密飭各營扼要隘。十月,水師驟登陸,出賊不意,悉燔港左右賊屯,圍賊爭馳奔魯港,囂且亂。大富乘城遙望見,拊髀曰:「援師至矣!」遂出城夾擊,賊披靡,追殺十餘里,與援師會殲賊萬餘,撲水死者無算,圍立解。城中兵不食月餘,僅存皮骨,民餓殍相屬。岳斌船粟往哺,歡聲雷動。大富方繕城垣固守,岳斌力言形勢不便,乃帥師屯上游,市民從者十餘萬。大富前後守南陵,始被圍經年,繼六閱月,以蕞爾城抗巨寇,忍死待援,卒熸兇焰,由是以善守名於時。

十一年正月,會水師復建德。二月,李世賢率黨數萬竄景德鎮,大富率兵四千自建德往援。賊銜恨,以計陷之。盡伏悍賊牛角嶺、柳家灣、回龍嶺等處,率隊由鎮南雙鳳橋竄李村,誘官軍,佯敗遁。大富率眾前進,躍馬爭先,參將田應科等繼之,賊突從鎮東抄出,伏賊盡起,大富挺矛力禦,砲洞左乳,血淋漓,仍裹創鏖戰。賊從間道襲焚我營,應科及游擊蕭傳科、胡占鼇,都司胡鳳雝、熊定邦、吳定魁,千總羅廷材皆戰死。大富見營中火起,下馬北鄉叩首,曰:「臣力竭矣!」投李村河死。贈提督,謚威肅,建專祠南陵,應科等並附祀。

陳萬勝,湘潭人。官軍規復江寧,圍攻將四年,用地雷法,穴城三十餘處,皆不就。同治三年六月,提督李臣典請從賊砲最密處重闕隧道,統帥曾國荃韙之。命各軍於城下築砲臺,護地道,別遣軍士刈濕葦蒿■M5積城下,覆以沙土,陽為肉薄登城狀。賊用全力捍爭,砲彈雨下。是月十五日,賊出死黨燒砲臺,官軍血戰竟夕,十六日,地雷發,遂克偽都。萬勝與郭鵬程、王紹羲則於先一日死之。萬勝初隸吉字營,從大軍規江寧,皆為軍鋒,累功擢副將。地道既成,國荃入隧親勘之,悍賊出太平門,直犯地道。別從朝陽門出數百人燒各砲臺及所積蘆蒿,萬勝督隊血戰,殲百餘人,力竭死之。賊裂尸,竿其首於城。

鵬程,湘鄉人。先後隸羅澤南、李續賓營,累以克九江、援寶慶諸役擢副將;又以皖北肅清,以總兵記名簡放。紹羲,同邑人。少入湘軍,累功亦以總兵記名簡放。是役也,賊以砲火轟擊,密如飛蝗,皆奮前督攻,同時歿於陣。城復,以三人死綏事上聞,有詔惋惜,各賞三等輕車都尉世職,謚萬勝武烈,鵬程勇烈,紹羲剛毅。

王之敬,浙江奉化人。道光二十九年,由水勇散目隨捕江蘇洋盜出力,拔補水師千總。咸豐三年,粵匪陷江寧,之敬管帶艇師,接戰甚勇,升守備。五年四月,由浦口會各營連宗剿賊,毀賊船獲勝,擢游擊。尋升太湖協副將。十年,遷江南福山鎮總兵。適值蘇、常淪陷,太湖三面皆賊,之敬孤軍設守,屢挫賊鋒,東西兩山,賴以安堵。十一年正月,賊忽率眾圍撲東山,之敬迎戰失利,東山遂陷,之敬失所在。嗣之敬之子祖培尋父尸至教場之西,見所畜犬臥土堆上,向之哀號,知有異,掘之得之敬尸,卷以席,傷痕遍體,而面目如生。詢居民,系於賊船退後撈獲掩埋者,不知其為總兵也。之敬性忠勇,號能戰,至此以寡不敵眾被害,人爭惜之。贈提督銜,謚果愍,建祠於太湖東西兩山。

陳忠德,字仁山,湖南清泉人。操舟為業。咸豐二年,粵賊圍長沙,掠舟北渡,遂陷賊中。忠德驍勇有智略,偽盡力於賊,久之,大見信任。十一年,道員曾國荃圍安慶,忠德自拔來歸,由是官軍始盡得賊中虛實。五月,從攻菱湖兩岸賊壘平之,從克安慶及平江岸各城隘,擢千總。

李鴻章援上海,選將得程學啟、郭松林於曾軍,忠德亦屬焉。從學啟破柘林、南匯、川沙、金山、青浦各城隘,擊退虹橋大股賊眾。會克嘉定,並解北新涇、四江口之圍。二年四月,昆山、新陽既復,從規蘇州。六月,攻破花涇港、同里鎮,蘇州賊水陸萬餘來援,忠德力戰,負重創,卒敗之,遂收吳江、震澤。

學啟軍益進,逼婁門外石壘,十月十九日,偽忠王李秀成、偽慕王譚紹光率萬人出婁門拒戰,學啟令忠德等擊敗之。李、潭二逆走入城,石壘遂下。賊計窮,其黨郜雲官等殺譚逆以城降,蘇州復,賞勇號。累以功擢副將,加總兵銜。後以克嘉興擢總兵,隨攻湖州,中砲歿於陣,照提督例賜恤。

復吳之役,死於戰者:攻青浦,為都司劉玉林、守備熊得春;攻太倉,為參將王國安;攻長洲、望亭,為把總沈玉德;攻無錫,為游擊汪龍淦。皆奮身陷陣,優恤,給世職。

黃金友,字益亭,湖南人。初從軍廣西,轉戰湖北、江西、安徽,積功至副將,賜勇號。金友躬犯矢石,創遍體。咸豐十年,江蘇巡撫薛煥奏調駐金山衛。十一年,賊犯浙江平湖,陷乍浦,東略姚廊,窺金山,金友迎戰大破之,遂平新倉賊壘,晉總兵。平湖知縣汪元祥乞師規復,金友壯之,檄金山、華亭、奉賢各營同赴援,躬督師進駐平湖之廣陳。元祥率民兵迎勞,請為鄉導,賊連營三十里,一鼓破之。賊會嘉興援賊分道來襲,金友迎御於十字街,賊大集,相持久,金友右肋被槍,猶誓死力戰,士皆奮呼,無不一當百。賊始卻,而金友創發不能騎,舁至明珠菴,卒。贈提督,恤如例。

麟瑞,字靄人,滿洲瓜爾佳氏,乍浦駐防。父觀成官南川知縣,有德政,蜀人為立生祠,稱小關廟,以關、瓜音通也。麟瑞以筆帖式歷印務章京。咸豐十一年,賊犯乍城,從副都統錫齡阿出督戰,偕弟鳳瑞、雲瑞手燃巨砲縱擊,賊驚卻,拔出難民無算。城陷,麟瑞率眾巷戰,力刃數賊,賊環攻,被槍,歿於陣。贈副都統,予世職,祀昭忠祠,謚忠節。雲瑞陷賊不屈死。

鳳瑞出從李鴻章軍,轉戰江、浙,攻和州、含山,以百騎計破賊萬餘,鴻章嘗稱為非常人。克太倉等處皆有功,贈將軍。麟瑞督戰時,本為副都統,護印至死不釋。後其子柏梁官乍浦副都統,蒞任拜印,啟視,斑斑猶見血痕云。柏梁自有傳。

蔡東祥,湖南湘陰人。充湖北水師水勇。咸豐四年,粵匪再陷武昌,與漢陽為犄角。東祥隨攻武漢兩岸賊,多有斬獲。隨攻占魚套,焚賊船,通糧道。湖北提督楊載福追賊田家鎮,賊聯木簰,置砲石,於半壁山拒敵。東祥奉令以火具鎔鐵鎖斷之,水師驟下,燔賊艘無算,遂拔田家鎮。於湖口、望江、九江、東流、建德、樅陽、蕪湖、銅陵諸戰皆有功,累擢至副將。

同治初,布政使曾國荃親率十二營與道員劉連捷分道擊江岸分踞賊,東祥分攻桐城,克雍家鎮。又會攻巢縣、含山、和州與裕溪口、江心洲、梁山各隘,復太平、蕪湖二縣。陸軍進逼金柱關,兵部侍郎彭玉麟率東祥等分水師為三隊,連環轟擊,躍上堤埂,短兵擊刺,積骸滿渠。關破,並劃三汊河、上駟渡賊壘,江岸肅清。先後賞雄勇巴圖魯等勇號,加總兵銜。

東祥勇決有謀,七年湖口之戰,大風,舟入口為賊所抄,不得出。所領長龍艦一,偃旗與賊艦混,須臾風止,急槳貫賊陣出,賊覺,追之不及。是役失長龍艦五、舢板十三,將弁死者二十一人,而東祥舟獨完。十一年安慶之戰,水師屢挫,賊易視之,見輒爭擊。東祥請易戰艦白旗為紅旗,賊疑為援兵,駭愕間,急率所部乘之,賊以敗,軍威復振。

旋調江蘇剿賊,偕副將歐陽利見,率淮陽水師,巡防三江口,戰嘉善。奮勇駛進西塘,援賊猝至,兩岸夾攻,被槍子傷,落水死。東祥轉戰克敵,素稱勇敢,照總兵例議恤,贈提督銜,給世職。

鄒上元,字蘭亭,湖南湘鄉人。咸豐初,投羅澤南營,澤南自江西援湖北,取道崇陽、通城,進攻佛嶺賊卡。上元隨隊由佛嶺北攀巖先進,與諸軍夾擊,破之。賊自崇陽三路來攻,上元從破右路,克崇陽、咸寧諸役均有功,擢千總。五年,賊犯湖南,巡撫駱秉章檄蕭啟江募勇助剿,號湘果營,上元隸焉。從援江西,克萬載,復袁州,果擢都司。秉章督蜀,檄黃純熙率果毅營從,上元方假歸,純熙招偕行,充營官。賊酋何國樑、彭紹福率黨攻定遠急,定遠東北瀕江,賊屯東南,造浮橋江上遏外援。上元從純熙疾赴,師至興學場,賊分黨逆戰。上元從右路襲擊,賊大奔,乘勝追至祖師殿,毀沿途二十餘壘,蹙賊江幹,何國樑鳧水遁,上元追斬之。

彭紹福聞敗,糾黨來援,踞燕子窩、二郎場等處。二郎場者絕地也,四山壁立,鳥道一線,西北阻涪江,純熙恐失寇,不待軍集,率千人追之。上元慮有伏,進至距二郎場二十里,遣諜偵之,不見賊,土人皆言賊去遠。夜半,至燕子窩,突遇賊騎,進逼之,賊繞山竄入場,純熙知中伏,分道搜之,伏盡發。官軍偪處泥淖間,不能成列,上元馳救,突圍入,手刃數賊,賊環刺之,死,純熙亦歿。定遠之捷,上元擢參將。死事聞,命視副將例,賞世職,附祀純熙祠。

郝上庠,直隸沙河人。由武進士授侍衛。道光二十六年,出為山東曹州鎮標守備,累遷至武定游擊。咸豐四年,韓莊盜起,山東巡撫張亮基慮徐州道梗,知上庠饒將略,薦署兗州鎮總兵。賊帥硃廣田寇郯、蘭、沂、莒諸屬,上庠率所部會鄉兵擊走之。賊南竄贛榆,上庠追及,殲其眾,斬廣田於陣,擢參將。五年,山東金鄉賊陶三相為亂,上庠疾馳誅其渠,餘眾驚潰,事得解。時海上多盜,連舟窺諸口,將北犯天津洋。山東巡撫崇恩疏請以上庠攝登州鎮總兵,專司防務。往來策應,先後擒斬賊首李希夢等,以功敘沂州協副將。

上庠每戰,輒身先士卒,不避艱險,以勇武受上知。賊平,手詔褒美,命以總兵記名。十年,署曹州鎮總兵,以疏防捻寇入境,奪記名,留鎮如故。十一年,官軍取濮州,河北肅清。上庠以屢勝功復官,賜提督銜。九月,賊潛渡濮、範,上庠不能御,與賊遇陽穀,又戰不利,為山東巡撫譚廷襄所劾,落職。

十月,克張秋鎮,移兵會營總烏爾棍扎布、游擊緒倫攻堂邑賊,戰於丁家廟,敗之。賊益眾來援,上庠奮擊不退,馬蹶墮地,拔刀殺賊數十人,力竭,戰死。命優恤,謚勤勇。聊城士民念上庠捍賊功,請立祠東昌,從之。

張遇祥,字瑞麟,直隸新樂人。年十五,能開兩石弓。道光十五年,成武進士,授乾清門侍衛。二十一年,選浙江衢州城守營都司,公廉能得士卒心。咸豐二年,在壽張游擊任,匪林鳳祥、李開芳率粵匪圍懷慶,山東巡撫李僡檄遇祥從征當一路,為士卒先,所向披靡。經略勝保嘉其勇,益感奮。嘗夤夜渡河戰,自寅至申,始奉令而返。

三年秋,曹縣捻匪亂起,攜親兵百人入城,捻首陳九千歲、張四大王擁眾擾城市,無敢攖者。遇祥密令伏兵於外,變服入賊巢,詐言他事,賊優禮之。夜分,酒酣,遇祥驟起蹴賊,首腦迸裂死,賊群起,且戰且走,出巢伏起,賊皆駭散。又偵知張四大王所在,託病不出,密令親信軍士夜馳百餘里入賊室擒之。粵匪攻臨清,率部卒二百人,夜砍賊營而入,殺無算。所部五十餘人被圍,復匹馬蕩決者三,攜之出,無一失者。右腿受矛傷,裹傷屢戰,賊不敢當。創劇不能起,巡撫親驗之,諭令歸養,新撫崇恩疑規避,奏參褫職,令解赴山東。既到標,崇恩始知其誣,慰勞備至。時金鄉、魚臺、嘉祥、費縣、鉅野、鄆城、城武七縣被捻匪所陷,遇祥招舊部六百人,自為一隊,復七城,餘孽悉平。復原官,以創發回籍調養。

十一年,山東教匪糾回、捻北犯直隸,勝保久無功,乃肅書聘遇祥,且令募勇自隨。書至,即招募鄉中子弟五百人,星馳而往,一戰敗之。勝保南移館陶,進次尖莊,賊麕集尖莊南,遇祥復馳救,賊奔。民爭奉糗餌漿粥,軍得一飽,督隊回尖莊。勝保又退守館陶,遇祥趨謁之。賊又欲渡河,勝保令往堵河口,遇祥曰:「士卒昨日一飯後,枵腹至今,烏能戰耶?」勝保曰:「汝速往!吾即遣人執釜甑從汝也。」遂率隊趨大河,士卒覓食不得,賊已先渡,遇祥匹馬陷賊陣,戰至日暮,下馬而搏。天明,回顧所部餘數十人,急揮之去,曰:「同死無益。吾身經數百戰,未曾一挫,今勢至此,不斬賊渠,不生還也!」縱馬示不返,士卒益感奮,誓同死,遇祥左右射,當者皆殪。賊以長戟鉤斷其弦,乃舍弓提刀戰,至下堡寺,日近山,從者餘六人,忽見大纛下賊渠至,將聳身刃之,時已戰兩晝夜,饑甚,舊傷皆發,復中矛數十處,力既竭,遂歿於陣,時咸豐十一年十一月七日。館陶、臨清諸村堡,爭建祠以祀。

兄遇清,字芳辰。武舉人。官廣東,洊擢至督標參將,檄援廣西平南縣,提刀巷戰,賊槍刺其腹,腸出,益奮。賊折其刀,手執木棍抵拒,賊攢擊,死之。平南士民亦立祠以祀。

曹仁美,字擇庵,湘潭人。初隸曾國荃軍,援江西,戰吉安天華山,克之。復景德鎮、浮梁,與有功。咸豐十年,賊據黟縣、建德,勢張甚,時仁美改隸曾國籓麾下,官軍屢戰不利,堅壘與持。仁美曰:「兩軍相持日久,當乘其懈而擊之,否則援至難圖也。」會夜大霧,仁美率所部摩其壘,更籌寂然,乃梯而入,手刃司柝者,縱火焚之。眾軍為承,斬數千級,賊大潰,遂克之,獲器械無數。遷都司,賜號勵勇巴圖魯。十一年,國荃攻安慶,久不下,國籓遣仁美往助,偽英王陳玉成合江、淮賊來援,國荃督戰,中流矢,仁美負之登高,揮諸軍奮擊,城遂下。以次從克大江兩岸城隘。同治元年,從圍金陵,仁美屯雨花臺西,國荃以城賊糧將匱,為坐困計,令諸軍毋與戰,凡四十有六日。仁美恚曰:「當賊不擊,將何待?」乃以其軍出毀石壘,賊頗死。國荃以其勇,薄責之,遂引疾歸。

投入李鴻章軍,圍攻常州,鴻章檄劉銘傳等偏師直搗,仁美率眾繼進,大破之。三年,克金陵,餘賊突出鏖戰,湘、淮諸軍屢挫於奔牛。銘傳軍被圍急,議者欲退保丹陽,仁美曰:「賊雖銳,猶困獸之斗也,出奇兵勝之。」次日,與諸軍略其東南,賊眾轟擊以拒,仁美膝行至砲旁,連擲火彈,賊駭走,官軍鼓噪而登,夷東路各壘。賊自隔河來犯,仁美夜乘輕舸,率健卒數人躍輪艦殺賊十餘,以火攻之,船盡裂,奔牛賊平。軍無錫,執游兵擾民者斬以徇。至是詔以總兵記名。

四年,偽侍王李世賢陷漳浦,鴻章遣仁美與郭松林俱航海赴援。既至,甫築壘,賊大至。仁美令諸軍無動,獨率所部三百人迎敵。賊疑有伏,不敢前,仁美伺懈擊之,賊奔南靖。乘勝薄城下,率眾先登,世賢巷戰逾時,啟西門而遁,遂復漳州,下南靖。擢提督。進規漳浦,賊分門堅守,仁美與副將張遵道分路迎戰,賊稍卻,麾軍競進,攻克之。進復雲霄。旋歸,以兄仁賢領其眾。

五年,國荃巡撫湖北,檄仁美與松林募軍進至唐縣,會東捻自信陽竄入,遇於德安,薄而擊之,追至鍾祥臼口。師分三路入,仁美攻其左,抵羅家集,遇伏,與戰,力竭死之。賞世職,予鍾祥及原籍建祠。

毛克寬,湖南漵浦人。咸豐初,兄弟五人同入田興恕虎威營,皆以善戰著,克寬尤驍勇。六年,隨興恕援江西,克萍鄉、萬載,復袁州。後從圍臨江,吉安賊來援,城賊填壕伺夾擊。克寬從拒援寇,大破之太平墟,燒賊屯四十七,遂克臨江。湖南巡撫駱秉章以克寬久經戰陣、勞績甚多聞於朝。時貴州苗、教各匪麕聚,復隨興恕赴援,連克錦屏等處,克寬功最。偽翼王石達開犯湖南,窺寶慶,克寬從興恕回援,破賊寶慶城南黃塘,復敗賊七架坡。賊合圍攻興恕壘,克寬日夜搏戰,援軍既集,內外夾攻,賊敗遁。追及九鞏橋、白楊鋪,復大破之。賊走廣西,遂以參將留湖南。

黔亂復熾,朝命興恕為貴州提督,督辦軍務。克寬再入黔,逆酋安太然及偽元帥韓成龍、偽招討覃國英等圍攻印江、石阡,克寬率虎後營分道進擊平陽等處賊屯百餘,斬韓成龍、覃國英,拔出難民三千餘人,遂解城圍,乘勝復甕安。兩旬之間,蕩平數百里。興恕疏薦克寬「膽識俱優,屢獲奇捷,隨征六載,戰必身先,實屬英勇冠群」。命以副將留黔,賞號銳勇巴圖魯。

石達開從廣西犯黔,陷歸化、定番等城。克寬迎剿,破籠溪、猴坪賊巢,進駐赤土,督軍攻定番、長寨,克之,復解安順、安平城圍。十一年,補大定協副將,移駐大水橋,通運道。賊乘營壘未成,遣悍黨分股來犯,克寬分隊逆戰,敗其左右翼,中路賊死抗不退,克寬策馬轢陣,往來蕩決,刃悍賊數十,賊眾披靡。會飛砲中馬,徒步奮擊,身受數創,歿於陣,年三十三。詔贈總兵銜,建專祠,賞世職。弟克佳,官把總,戰歿臨江。

先死黔苗之亂者,有邢連科,原名正堉,貴州貴陽人。臺拱黃施衛千總。咸豐三年,苗叛,攻城,連科迭乞援,累月兵始至。連科蒐殘卒夾擊出陷陣,而援兵先潰,連科轉戰十家寨,陣亡。

子士義。舉人,主講平越,先聞警,馳赴城守。至是召家人環坐,縱火藥自焚,僕諶年有、婢玉蘭從死。千總署堂皇之下,列尸二十有二。巡撫蔣霨遠、田興恕先後以闔門殉難聞,賜祀,予世職。孫以謙,曾孫端,翰林院編修。

田興奇,湖南鳳凰人。隸田興恕虎威營。咸豐六年,從平郴、桂、茶陵,以功敘外委。興恕援江西,進攻袁州,興恕躍馬突賊陣,興奇隨入,各軍繼之,遂獲大勝,賊潰奔數十里。分宜、袁州復,擢千總,加守備銜,賞藍翎。七年,師次高安陰岡嶺,興奇斬偽監軍姜萬祥、總制艾得勝。攻復臨江,擢游擊,換花翎。八年夏,貴州寇起,隨興恕往援,敗賊黎平,夷其營。轉攻漢砦,斬馘二十餘級,夷賊營十餘處,追北十八江,斬偽侯黃必升等二十一人,擒偽將軍伍雲童。黎平復,以參將留湖南補用,加副將銜。

九年春,石達開率眾十餘萬犯寶慶,興恕軍適自黔還,道其境,駐軍九鞏橋。賊乘其甫安營,悉眾來犯,隨興恕擊卻之。是夜三鼓,興奇率壯士八百人襲賊營,賊驚潰,死亡相屬,餘眾奔逃。寶慶平,擢副將。

十年,從興恕剿貴州苗匪,興奇領虎勇二千人至石阡,戰龍潭,斬賊偽元帥韓成龍、覃國英,盡平其營。越二日再攻,賊走馬坪,斬馘甚眾,並拔出被擄老穉男女三千餘口。捷聞,賜沖勇巴圖魯名號,加總兵銜,仍駐石阡。六月,擊賊雙溪,中伏,死之,時年三十二。詔贈提督銜,謚剛介。

田興勝亦隸興恕部下,平郴、桂,援寶慶及援江西,同有功,累擢至守備。又隨剿貴州各匪,破籠溪,解餘慶圍。偕都司沈宏富等進屯雄黃、小崽等處。賊於老巢立堅壘十八,悍黨萬餘,分布左右山梁,興勝約游擊楊巖寶兩路夾攻,自與沈宏富攻左路各寨;都司田興考由右路繞山後出擊,並設伏後路。計定,率兵直沖首山梁,賊數千迎擊,興勝督隊沖突,鏖戰逾時,賊沿山潰遁,營內賊開卡狂竄,興勝親入賊陣,手刃悍目二。宏富督後隊合圍,先毀其右寨,移攻左寨,破之。乘勝追襲,有黃衣賊目率黨死拒,興勝射之,蹶,擒而梟示,餘賊大潰。追十餘里,墜崖死者無數。是役共破堅寨十餘座,陣斬偽元帥韓進、楊正閏等二十餘人,馘千五百餘級,乘勝平賊營十餘座。

復隨總兵劉吉三等夜攻三角莊,賊猝不及防,驚而潰,毀其連營三座。適松坪賊首石復明糾玉華山匪黨數萬,分六股來撲,興勝偕游擊劉祖得合兵迎剿,都司徐祥太與興考各率所部設伏山麓及民舍中,賊遇伏大敗,死亡枕藉。追至木影頂,地險峻,賊寨負隅難拔,因收隊。明日,同知唐繩武等由間道出松坪之後,先取老巢,興勝奉令偕巖寶等攻木影頂,攀藤而上,賊礌石交下,軍少卻。興勝橫刀躍馬,奮身進,飛登寨墻。賊矛攢刺,捉其矛而上,揮刀連斬悍賊十餘,諸軍繼進,賊散走。官軍四面兜剿,殲除殆盡,擒其渠秦官寶、劉老栘等,誅之。

松坪黃號賊眾五萬,連營三十餘,興勝等即時裹糧疾進,悉銳攻之,賊傾巢出拒,官軍奮擊敗之。有偽扶明王者,悍酋也,手斫敗退賊,挺身來抗。興勝自與搏戰,殪之。餘賊猶相持,宏富等已從後破其巢,賊乃潰竄。官軍夾擊,斬馘四千餘級,松坪賊壘皆平。乘勝攻猴嶺,拔之。

旋偕宏富移剿甕安,先破小山寺營、馬安營賊壘百餘,冒雨分三路進攻紅岡堡,興勝策馬陷陣,連刃數賊,諸軍繼之,立破其巢,躡追數里。會甕安賊來援,敗賊亦返斗,興勝偕宏富等縱擊。射斃賊酋數名,賊亂返奔,官軍急躡之,賊不敢入城,奔至玉華山老巢。甕安遂復,自是入省之路始通。興恕疏稱「興勝每戰單槍陷陣,不計生死,實屬忠勇可嘉」。詔以游擊拔補,給果勇巴圖魯勇號。

十一年,粵匪大股竄貴州境,踞定番州、長寨等處,偪近省垣,巖寶等攻之未下。興勝隨興恕往剿,屢戰皆捷,克定番,又偕副將周學桂進兵拔長寨,其別股踞土地關者,分黨撲安平以牽掣官軍,興勝馳救,立解城圍。匪首張遇恩勾結仲匪圍安順,並撲定南汛城,又隨總兵趙德昌擊退之,省會解嚴。

三月,偕巖寶等進攻土地關,與賊戰於赤土,賊敗走,賊首仰天燕斷後。興勝追之,將及,以乘騎饑疲,馳驟過猛,一蹶而斃。賊回隊圍之,徒步格鬥,殺悍賊十餘人,身受多創,血流如注,猶抵死相持,力竭,歿於陣。照總兵例優恤,謚武烈。

馬定國,四川萬縣人。咸豐六年,投鮑超霆營,從攻九江小池口,回援黃梅,疊破賊孔壟、大河鋪、億生寺、黃蠟山等處,定國功多,委帶霆字左營親兵。復隨剿太湖之楓香驛,破賊壘十餘座。八年,上援麻城,遇賊黃土岡,拔主將出圍。從克麻城、太湖,毀雷公埠、石牌逆壘,進攻安徽省城。賊於北門外及東西山灣,連柵周亙,堅不能拔。定國負楯直入重柵,破其數壘。會巡撫李續賓軍覆三河,賊由舒城、潛山上竄,遂從超退扼宿松之二郎河,賊來犯,擊破之。復破賊於花涼亭,進圍太湖。悍酋陳玉成以大股來援,超移壁小池驛,賊眾圍之。十年正月,大戰,破賊壘數十座,斃賊萬餘,乘勝克太湖、潛山。累擢至游擊,晉參將。從規皖南,收黟縣,大破賊盧村、羊棧嶺,命以副將用,乞假回籍。

同治元年,滇匪擾四川萬縣之紅穀田,定國率鄉兵禦賊,戰歿。詔贈總兵銜,建專祠,賞世職。


\end{pinyinscope}