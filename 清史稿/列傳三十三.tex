\article{列傳三十三}

\begin{pinyinscope}
譚泰何洛會錫圖庫博爾輝冷僧機

譚泰,舒穆祿氏,滿洲正黃旗人,揚古利從弟也。初授牛錄額真。天聰八年,擢巴牙喇章京,與固山額真圖爾格分統左右翼兵,略錦州。還從太宗伐明,自上方堡毀邊墻以入,敗明兵,克保安州。擢巴牙喇纛章京,令關白諸事。九年,揚古利賜第,侍衛宗室濟馬護欲得其舊居,揚古利不可。濟馬護囑譚泰入奏,譚泰匿不以聞,濟馬護訴於上,上責譚泰曰:「爾為朕耳目,凡事當無隱。濟馬護乃朕叔父之子,其言尚不能達,民間勞苦嗟怨,何由得使朕知?爾恃宗族強盛,欺陵愚弱,朕所深惡!」下刑部質訊,奪官。尋復授本旗固山額真。

崇德元年,從武英郡王阿濟格等伐明,克延慶等十二城。進圍定興,先登有功。復與固山額真阿山等設伏,敗明遵化三屯營守兵,盡殲之。師還,宴勞。復從上伐朝鮮,朝鮮王棄城走,譚泰率師入其城,盡收其輜重。從上逐朝鮮王至南漢城,受降而還。四年,從睿親王多爾袞等伐明,與固山額真葉臣自太平寨破青山口,與明兵十三戰,皆捷。輔國將軍鞏阿岱,濟馬護兄也,譚泰與相詬於禁門,坐罰。

六年,從圍錦州,譚泰將四百人自小凌河直抵海濱,絕明兵歸路。與明總督洪承疇兵戰,大敗之。授世職二等參將。七年,從輔國公篇古等攻薊州,擊敗明總兵白騰蛟、白廣恩等,俘馘為諸軍最。八年,命率銳卒與固山額真準塔更番戍錦州。順治元年,從入關,逐破李自成於慶都。復將巴牙喇兵躡擊,至真定,大破之,敘功授一等公。

睿親王攝政,譚泰與巴牙喇纛章京圖賴、啟心郎索尼並見信任。固山額真何洛會誣肅親王豪格怨譚泰等不附己,訐之睿親王,王謂譚泰忠,益信任之。大學士希福忤譚泰,希福欲易賜第,譚泰不可,希福誚之,益怒。其弟譚布以希福述睿親王自言過誤告譚泰,譚泰訐之法司,希福坐黜。二年,英親王阿濟格坐奏軍事不實得罪,命譚泰與鰲拜等集眾宣其罪。譚泰匿諭旨不以示眾,索尼發其罪,降世職昂邦章京,奪官。譚泰怨索尼,訐索尼於內庫牧馬鼓琴及禁門橋下捕魚,索尼亦坐黜。譚泰復起為本旗固山額真。

初,師下江南,譚泰自西安逐捕流寇,慮不與平江南功,使謂圖賴曰:「我軍道迂險,後至。今南京未下,請留待我軍。」圖賴書告索尼,使啟睿親王,或發觀之,懼譚泰得罪,毀其書勿使達索尼。圖賴師還,詰索尼,發其事,王鞫齎書者,得狀。譚泰又坐與婦翁固山額真阿山遣巫者治病。下廷臣議罪,論死,下獄,王使視之,並餽食焉。譚泰曰:「王若拯我,我殺身報王!」乃出之獄。五年,復原官。

金聲桓叛江西,授譚泰征南大將軍,率師討之。聲桓以步騎七萬人抗我師,譚泰督諸軍與戰,次九江,大敗聲桓兵,獲其舟以濟師。攻南昌,為長圍困之,數月,麾將士以雲梯登,聲桓中二矢,投水死;又破其將王得仁。南康、瑞州、臨江、袁州並下。當聲桓叛時,李成棟以廣東應之,南昌圍急,成棟赴援。譚泰師將至贛州,聞成棟入信豐,譚泰遣諸將乘勝襲擊,成棟兵潰,溺水死,克信豐。別將徇撫州、建昌。江西悉平。師還,授一等精奇尼哈番。

七年,睿親王薨,上命吏、刑、工三部增設滿洲尚書各一,授譚泰吏部尚書。八年,世祖親政,追論睿親王罪狀,大學士剛林、祁充格皆坐誅,罪不及譚泰。時圖賴已卒,索尼方罪廢,譚泰毀圖賴墓室,洩舊忿。五月,御史張煊劾大學士陳名夏等,下王大臣會鞫。譚泰袒名夏,讞上,命未下,譚泰前奏,言煊劾皆虛,且所舉諸事皆在赦前,煊以外轉嫌,誣名夏等死罪,當反坐,煊遂見法。

譚泰愈縱恣,嶽爾多其妻弟也,襲一等精奇尼哈番,為奪其族人法喀應襲一等阿思哈尼哈番合並為三等侯;佟圖賴其女弟之夫也,時金礪駐防杭州,妄稱員缺,以佟圖賴擬補。上自譚泰袒陳名夏構張煊,心厭惡之。是歲八月,下詔責其專橫,命執付獄,集廷臣議罪。鰲拜復訐譚泰阿附睿親王及營私擅政諸狀,讞皆實。王大臣議誅譚泰及其子孫,上命誅譚泰,籍其家,子孫貸連坐。

何洛會,失其氏,滿洲鑲白旗人。父阿吉賴,事太祖,從征戰,官牛錄額真。卒,何洛會嗣,兼巴牙喇甲喇章京。天聰八年,從伐明,略錦州。九年,詔免諸功臣徭役,何洛會與焉。崇德五年,授正黃旗蒙古固山額真。從睿親王多爾袞伐明,圍錦州。調滿洲固山額真。七年,錦州既下,追論圍錦州時何洛會匿鄂羅塞臣破陣功,當奪官,上宥之。

何洛會隸肅親王豪格,頗見任使。世祖即位,睿親王攝政,與肅親王有隙。何洛會訐肅親王與兩黃旗大臣揚善、俄莫克圖、伊成格、羅碩將謀亂,肅親王坐削爵,揚善等皆棄市。賞何洛會告奸,籍俄莫克圖、伊成格家畀之,授世職二等甲喇章京。尋從睿親王入關,擊李自成,逐至慶都。還,睿親王令奉表迎世祖,擢內大臣,留守盛京。阿哈尼堪將左翼,碩詹將右翼,並於熊耀城、錦州、寧遠、鳳凰城、興京、義州、新城、牛莊、岫巖城各置城守官,皆統於何洛會。

順治二年,敘功,進世職一等。旋命率師駐防西安,道河南,討定西平土寇劉洪起等。是歲十二月,授定西大將軍,命自陜西徇四川。時自成將劉體純等犯商州,叛將賀珍與其黨孫守法、胡向宸等分據漢中、興安。三年,珍以七萬人犯西安,何洛會督兵迎戰,珍敗走,復逐破之,並破體純商州。

肅親王從入關,破李自成,復爵。至是,上命為靖遠大將軍,下四川,召何洛會還京師。四年,命率師駐防宣府,仍授正黃旗滿洲固山額真。五年,調鑲白旗。命佐譚泰定江西,擊破金聲桓、王得仁、李成棟,事具譚泰傳。師還,賜所獲金銀珠玉,進世職三等精奇尼哈番。

肅親王師還,貝子屯齊等訐王諸悖妄狀,何洛會復從而證之,遂坐奪爵,以幽系終。睿親王取肅親王福金,召肅親王諸子入府校射,何洛會詈之曰:「見此鬼魅,不覺心悸!」尚書譚泰聞其語。及睿親王薨,世祖親政,何洛會語貝子錫翰曰:「兩黃旗大臣與我相惡,我嘗訐告肅親王,今豈肯容我?」八年二月,蘇克薩哈等訐睿親王將率兩白旗移駐永平,且私具上服御,及薨用斂,何洛會、羅什、博爾惠等皆知狀。時羅什、博爾惠已先誅,執何洛會,下王大臣會鞫。譚泰、錫翰各以何洛會語告,又追論誣告肅親王罪,與其兄胡錫並磔死,籍其家。

錫圖庫,烏扎拉氏,滿洲正白旗人,世居烏拉。兄福蘭,當太祖時來歸,授世職備御。卒,錫圖庫嗣,授牛錄額真,兼巴牙喇甲喇章京。天聰四年,師克永平,錫圖庫與甲喇額真圖魯什等率兵循徼,得邏卒二、馬十七。五年,詗敵大凌河,得二人以還。上伐明,圍大凌河城,敗錦州援兵,錫圖庫皆有功。六年,復從伐明,略宣府、大同邊外,多所斬獲。八年,復略蒙古錫爾哈、錫伯圖諸地,斬七十餘級、俘百餘戶及馬駝,賚以所獲,進世職一等甲喇章京。九年,偕噶布什賢噶喇依昂邦勞薩等略明邊,入長城,攻代、朔諸州,多所斬獲。

崇德元年,睿親王多爾袞率師伐明,攻寧遠,錫圖庫以二十人前驅,至中後所及山海關外詗敵,屢得邏卒,並獲其馬,又於前屯衛設伏敗敵。喀木尼堪部葉類等盜科爾沁諸部馬叛走,錫圖庫率巴牙喇壯達八人詣寧古塔,與梅勒額真吳巴海督兵追之。行數十日,及於溫鐸,招降不從,葉類潛遁,盡殲其黨九十四人,俘婦女八十餘,得馬五十六;復逐捕葉類,入山,射之殪。師還,上遣大臣出迎五里,宴勞,進世職一等梅勒章京。

五年,命偕巴牙喇纛章京濟什哈率師並徵蒙古敖漢、柰曼、烏喇特諸部兵伐索倫部,敗敵於甘河,擒部長博木博郭爾,籍千餘戶,得馬數百。師還,賜宴北驛館,進世職三等昂邦章京。旋授本旗梅勒額真。七年,從貝勒阿巴泰伐明,自薊州越明都,下山東。師還,以先出邊,部議當奪官,命寬之,白金百。八年,擢巴牙喇纛章京。

順治元年,從睿親王多爾袞伐明,敗李自成將唐通於一片石,遂入山海關,屢戰皆勝;敗自成游騎於三河,追擊至安肅。旋從固山額真葉臣等取太原,戰於汾州、於絳州,屢破敵。二年,進二等精奇尼哈番。時自成猶據陜西,師自潼關、綏德南北兩路入,錫圖庫率師與北路軍會,敗賊延安。自成走入湖廣,鍚圖庫移兵從之,自安陸至於荊門,屢擊敗自成兵。

三年,復從肅親王豪格下四川,討張獻忠。五年,進世職一等。復從鄭親王濟爾哈朗下湖南。六年,師次長沙,錫圖庫從左翼巴牙喇纛章京努三率兵前驅,攻湘潭。努三軍北門,錫圖庫軍西門,遂克之。進徇永興,斬明將尹舉智、杜貞明等。再進定寶慶,取全州,破明將焦璉。又移兵克永安關,取道州。師還,賜白金三百。

七年,睿親王多爾袞薨。八年春,吳拜、羅什、博爾輝等訐英親王阿濟格將謀亂,鞫實,錫圖庫坐與謀,誅死,籍其家。

博爾輝,他塔喇氏,滿洲正白旗人。初以巴牙喇壯達從征棟奎部,有俘馘。天聰三年,從太宗伐明,自龍井關入攻遵化。明總兵趙率教自山海關赴援,與戰,博爾輝斬其副將,明兵驚潰。五年,擢巴牙喇甲喇章京,兼戶部參政。復從伐明,與明兵遇寧遠,擊殺前隊七人。八年,復從伐明,攻大同,明兵三千自龍門迎戰,博爾輝與噶布什賢章京錫特庫、牛錄額真星訥等奮擊破之。九年,命偕承政馬福塔齎敕諭朝鮮國王。師出邊招察哈爾部眾,自歸化經明邊東還,博爾輝殿。明兵二百三十追擊我師,博爾輝以二十人擊卻之,斬十人,俘一人,得馬三。明兵從我師,有垂為所獲者,博爾輝救之得脫。崇德元年,敘功,授世職牛錄章京。三年,裁參政,專任巴牙喇甲喇章京。

順治元年,兼任刑部理事官。從入關,擊李自成,敘功,進世職二等甲喇章京。旋署巴牙喇纛章京。從順承郡王勒克德渾下湖廣,師至武昌。時自成將馬進忠、王進才既降復叛,據岳州,令博爾輝率師討之,次臨湘,擊敗其兵。進攻岳州,進忠、進才走長沙,逐擊敗之,其將黑運昌以舟師降。師還,優賚。五年,真除巴牙喇纛章京,列議政大臣,進世職二等阿思哈尼哈番。

睿親王攝政,諸王多與忤。鄭親王濟爾哈朗降郡王,旋復爵。初以端重親王博洛、敬謹親王尼堪佐理事,亦以專擅降爵。博爾輝及諸大臣羅什、額克親、吳拜、蘇拜皆謹事睿親王,從王獵喀喇城。王薨,喪還。英親王阿濟格為睿親王同母兄,欲繼王柄政,博爾輝等與阿爾津共發其罪,英親王奪爵幽禁,賞諸告者,博爾輝進世職二等精奇尼哈番。博爾輝等傳睿親王遺言,復理事二王親王爵,以告兩黃旗大臣。居月餘,命未下,博爾輝有疾,穆爾泰往視之,博爾輝以為言。穆爾泰告額爾德赫,額爾德赫告敬謹郡王尼堪,遂與端重郡王博洛訴於鄭親王。八年正月,復二王爵。越八日,執博爾輝等下獄,坐博爾輝、羅什動搖國事,蠱惑人心,論死,籍其家。額克親削宗室籍,及吳拜、蘇拜皆奪官為民。議上,得旨:「朕每聞刑人,殊不忍。二人罪當誅,姑宥死何如?」王大臣復以初議上,乃誅死。

冷僧機,納喇氏,滿洲正黃旗人,葉赫部長金臺石之族也。葉赫亡,來歸,隸正藍旗,屬貝勒莽古爾泰。天聰元年,敖漢部長索諾木來歸,尚公主為額駙,以冷僧機隸焉。莽古爾泰既卒,九年,冷僧機詣法司言莽古爾泰及貝勒德格類與公主及索諾木結黨,設誓謀不軌。冷僧機與甲喇額真屯布祿、巴克什愛巴禮並下法司,鞫實,冷僧機以自首免罪,屯布祿、愛巴禮皆坐誅,籍其家以★K2冷僧機,改隸正黃旗,授世職三等梅勒章京。

崇德二年,固山額真都類坐事下兵部待鞫,兵部參政穆爾泰令諸在系者避都類。或以告冷僧機,聞於上,穆爾泰及同官皆坐降罰,授冷僧機一等侍衛。七年,祖大壽來歸,上幸牧馬所,命內大臣侍衛與大壽等校射,中的者有所賜,冷僧機得駝一。世祖即位,授內大臣。順治二年,進二等阿思哈尼哈番兼拖沙喇哈番。譚泰訐索尼,引冷僧機為證,謝未聞,坐徇庇,當削世職籍沒,上貰之。旋進世職三等精奇尼哈番。

七年,睿親王有疾,怨上未臨視,冷僧機及貝子錫翰等奏請上臨視,睿親王坐以擅請降世職,恩詔復故,進一等伯。睿親王薨,以豫親王多鐸子多爾博為後,襲爵。冷僧機言於上曰:「昔太宗登遐,兩黃旗大臣誓立肅親王。睿親王定策奉上紹統,多爾博宜特見優遇。」又舉侍衛羅什,羅什上為冷僧機乞恩。八年,鄭親王濟爾哈朗等劾羅什蠱惑諸王,坐誅,辭連冷僧機。上因命諸大臣詰誓立肅親王事,冷僧機窮,諸大臣兼發阿諛睿親王諸罪,論斬籍沒,命寬之。九年,追論冷僧機與貝子鞏阿岱、錫翰,內大臣西訥布庫等迎合睿親王,亂國政,下王大臣鞫實,與鞏阿岱、錫翰、西訥布庫等並誅,籍其家。

論曰:定金聲桓、王得仁之亂,譚泰專將,何洛會為之佐。錫圖庫、博爾輝亦久從征戰有勞。睿親王既薨,諸阿附者乃互相傾,何洛會之獄,譚泰證之;錫圖庫之誅,博爾輝等發之:轉相排軋,同就誅夷。若冷僧機者,專事告訐,其及也亦宜矣。


\end{pinyinscope}