\article{列傳五十八}

\begin{pinyinscope}
徐乾學翁叔元王鴻緒高士奇

徐乾學,字原一,江南昆山人。幼慧,八歲能文。康熙九年,一甲三名進士,授編修。十一年,副蔡啟僔主順天鄉試,拔韓菼於遺卷中,明年魁天下,文體一變。坐副榜未取漢軍卷,與啟僔並鐫秩調用。尋復故官,遷左贊善,充日講起居注官。丁母憂歸,乾學父先卒,哀毀三年,喪葬一以禮;及母卒,如之。為讀禮通考百二十卷,博採眾說,剖析其義。服闋,起故官。充明史總裁官,累遷侍講學士。

二十三年,乾學弟元文以左都御史降調,其子樹聲與乾學子樹屏並舉順天鄉試。上以是科取中南皿卷皆江、浙人,而湖廣、江西、福建無一與者,下九卿科道磨勘。樹屏等坐斥舉人。是年冬,乾學進詹事。二十四年,召試翰詹諸臣,擢乾學第一,與侍讀韓菼、編修孫岳頒、侍講歸允肅、編修喬萊等四人並降敕褒獎賞賚。尋直南書房,擢內閣學士,充大清會典、一統志副總裁,教習庶吉士。時戶部郎中色楞額往福建稽察鼓鑄,請禁用明代舊錢,尚書科爾坤、餘國柱等議如所請。乾學言:「自古皆新舊兼行,以從民便。若設厲禁,恐滋紛擾。」因考自漢至明故事,為議以獻。上然之,事遂寢。

詔採購遺書,乾學以宋、元經解、李燾續通鑒長編及唐開元禮,或繕寫,或仍古本,綜其體要,條列奏進,上稱善。時乾學與學士張英日侍左右,凡著作之任,皆以屬之。學士例推巡撫,上以二人學問淹通,宜侍從,特諭吏部,遇巡撫缺勿預推。未幾,遷禮部侍郎,直講經筵。朝鮮使臣鄭載嵩訴其國王受枉,語悖妄。乾學謂恐長外籓跋扈,劾其使臣失辭不敬,宜責以大義。上見疏,獎,謂有關國體。已而王上疏謝罪。二十六年,遷左都御史,擢刑部尚書。二十七年,典會試。

初,明珠當國,勢張甚,其黨布中外,乾學不能立異同。至是,明珠漸失帝眷,而乾學驟拜左都御史,即劾罷江西巡撫安世鼎,諷諸御史風聞言事,臺諫多所彈劾,不避權貴。明珠竟罷相,眾皆謂乾學主之。時有南、北黨之目,互相抨擊。尚書科爾坤、佛倫,明珠黨也,乾學遇會議會推,輒與齟。總河靳輔奏下河屯田,下九卿會議,乾學偕尚書張玉書言屯田所占民地應歸舊業,科爾坤、佛倫勿從。御史陸祖修因劾科爾坤等偏袒河臣,不顧公議,御史郭琇亦劾輔興屯累民,詔罷輔任。湖廣巡撫張汧亦明珠私人,先是命色楞額往讞上荊南道祖澤深婪贓各款,並察汧有無穢跡,色楞額悉為庇隱。御史陳紫芝劾汧貪黷,命副都御史開音布會巡撫於成龍、馬齊覆訊,汧、澤深事俱實,復得澤深交結大學士餘國柱為囑色楞額徇庇及汧遣人赴京行賄狀,下法司嚴議。時國柱已為琇劾罷,法司請檄追質訊,並詰汧行賄何人,汧指乾學。上聞,命免國柱質訊,戒勿株連。於是但論汧、澤深、色楞額如律,事遂寢。乾學尋乞罷,疏言:「臣蒙特達之知,感激矢報,苞苴餽遺,一切禁絕。前任湖北巡撫張汧橫肆汙衊,緣臣為憲長,拒其幣問,是以銜憾誣攀。非聖明在上,是非幾至混淆。臣備位卿僚,乃為貪吏誣構,皇上覆載之仁,不加譴責,臣復何顏出入禁廷,有玷清班?伏冀聖慈放歸田里。」詔許以原官解任,仍領修書總裁事。

二十八年,元文拜大學士,乾學子樹穀考選御史。副都御史許三禮劾乾學:「律身不嚴,為張汧所引。皇上寬仁,不加譴責,即宜引咎自退,乞命歸里。又復優柔系戀,潛住長安。乘修史為名,出入禁廷,與高士奇相為表裏。物議沸騰,招搖納賄。其子樹穀不遵成例,朦朧考選御史,明有所恃。獨其弟秉義文行兼優,原任禮部尚書熊賜履理學醇儒,乞立即召用,以佐盛治。乾學當逐出史館,樹穀應調部屬,以遵成例。」詔乾學復奏,乾學疏辨,乞罷斥歸田,並免樹穀職。疏皆下部議,坐三禮所劾無實,應鐫秩調用。三禮益恚,復列款訐乾學贓罪,帝嚴斥之,免降調,仍留任。

是年冬,乾學復上疏言:「臣年六十,精神衰耗,祗以受恩深重,依戀徘徊。三禮私怨逞忿,幸聖主洞燭幽隱。臣方寸靡寧,不能復事鉛槧。且恐因循居此,更有無端彈射。乞恩終始矜全,俾得保其衰病之身,歸省先臣丘隴,庶身心閒暇。原比古人書局自隨之義,屏跡編摩,少報萬一。」乃許給假回籍,降旨褒嘉,命攜書籍即家編輯。二十九年春,陛辭,賜御書「光焰萬丈」榜額。未幾,兩江總督傅臘塔疏劾乾學囑託蘇州府貢監等請建生祠,復縱其子侄交結巡撫洪之傑,倚勢競利,請敕部嚴議。語具元文傳。上置弗問,而予元文休致。

三十年,山東巡撫佛倫劾濰縣知縣硃敦厚加收火耗論死,並及乾學嘗致書前任巡撫錢鎯庇敦厚。乾學與鎯俱坐是奪職。自是齮齕者不已。嘉定知縣聞在上為縣民訐告私派,逮獄,閱二年未定讞。按察使高承爵窮詰,在上自承嘗餽乾學子樹敏金,至事發後追還,因坐樹敏罪論絞。會詔戒內外各官私怨報復,樹敏得贖罪。三十三年,諭大學士舉長於文章學問超卓者,王熙、張玉書等薦乾學與王鴻緒、高士奇,命來京修書。乾學已前卒,遺疏以所纂一統志進,詔下所司,復故官。

翁叔元,字寶林,江南常熟人。康熙十五年,一甲三名進士,授編修,館試第一。累遷國子監祭酒,洊擢吏部侍郎,遷工部尚書。部例,每有工作,先計其直上之,名曰「料估」。工完多冒破,所司不敢以聞,有十年不銷算者,大工至四十三案。叔元蒞部甫半載,積牘一清。調邢部,移疾歸,卒。叔元愛才而褊隘,何焯在門下,初甚賞之;叔元疏劾湯斌,焯請削門生籍,叔元擯之,竟不得成名。以是為世所誚云。

王鴻緒,初名度心,字季友,江南婁縣人。康熙十二年一甲二名進士,授編修。十四年,主順天鄉試。充日講起居注官。累遷翰林院侍講。十九年,聖祖諭獎講官勤勞,加鴻緒侍讀學士銜。時湖廣有硃方旦者,自號二眉山人。造中說補,聚徒橫議,常至數千人。自詡前知,與人決休咎。巡撫董國興劾其左道惑眾,逮至京,得旨寬釋。及吳三桂反,順承郡王勒爾錦駐師荊州,方旦以占驗出入軍營,巡撫張朝珍亦稱為異人。上密戒勒爾錦勿為所惑。方旦乃避走江、浙,會鴻緒得其所刊中質秘書,遂以奏進,列其誣罔君上、悖逆聖道、搖惑人心三大罪。方旦坐誅。

二十一年,轉侍讀,充明史總裁。累擢內閣學士、戶部侍郎。二十四年,典會試。二十五年,疏請回籍治本生母喪,遣官賜祭。二十六年,擢左都御史。疏劾廣東巡撫李士楨貪劣,潮州知府林杭學嘗從吳三桂反,乃舉其清廉。士楨坐罷,杭學奪職。會靈臺郎董漢臣疏陳時事,以諭教元良、慎簡宰執為言。御史陶式玉劾漢臣摭拾浮言,欺世盜名,請逮治。鴻緒疏言:「欽天監靈臺郎、博士等官,不擇流品,星卜屠沽之徒,粗識數字,便得濫竽。請敕下考試,分別去留。」下部議行。漢臣及博士賈文然等十五人並以詞理舛誤黜。初,以式玉疏下九卿集議,尚書湯斌謂大臣不言,慚對漢臣。漢臣既黜,鴻緒偕左都御史璙丹、副都御史徐元珙合疏劾斌務名鮮實,並追論江寧巡撫去任時,巧飾文告,以博虛譽。上素重斌清廉,置弗問。

鴻緒論各省駐防官兵累民,略言:「駐防將領恃威放肆,或占奪民業,或重息放債,或強娶民婦。或謊詐逃人,株連良善;或收羅奸棍,巧生扎詐。種種為害,所在時有。如西安、荊州駐防官兵紀律太寬,牧放馬匹,驅赴村莊,累民芻秣;百十成群,踐食田禾,所至驛騷。其他苦累,又可類推。請嚴飭將軍、副都統等力行約束。綠旗提、鎮縱兵害民,以及虛冒兵糧者,不一而足,請飭督撫立行指參。」上命議行。

未幾,以父憂歸。二十八年,服闋,將赴補。左都御史郭琇劾鴻緒與高士奇招權納賄,並及給事中何楷、編修陳元龍,皆予休致。語具士奇傳。嘉定知縣聞在上為縣民訐告私派事,按察使高承爵按治。在上言嘗以銀餽舉人徐樹敏,至事發退還,因坐樹敏罪。巡撫鄭端覆訊,在上言嘗以銀五百餽鴻緒,亦事發退還。端乃劾乾學縱子行詐,鴻緒竟染贓銀,有玷大臣名節,乞敕部嚴議。上特諭曰:「朕崇尚德教,蠲滌煩苛。凡大小臣工,咸思恩禮下逮,曲全始終;即因事放歸,仍令各安田里。近見諸臣彼此傾軋,伐異黨同,私怨相尋,牽連報復;雖業已解職投閒,仍復吹求不已,株連逮於子弟,顛覆及於身家。朕總攬萬機,已三十年,此等情態,知之甚悉。媢嫉傾軋之害,歷代皆有,而明季為甚。公家之事,置若罔聞,而分樹黨援,飛誣排陷,迄無虛日。朕於此等背公誤國之人,深切痛恨。自今以往,內外大小諸臣,宜各端心術,盡蠲私忿,共矢公忠。儻仍執迷不悟,復踵前非,朕將窮極根株,悉坐以朋黨之罪。」時鴻緒方就質,詔至,得釋。

三十三年,以薦召來京修書。尋授工部尚書,充經筵講官。四十七年,調戶部。其年冬,皇太子允礽既廢,詔大臣保奏儲貳,鴻緒與內大臣阿靈阿、侍郎揆敘等謀,舉皇子允禩,詔切責,以原品休致。

五十三年,疏言:「臣舊居館職,奉命為明史總裁官,與湯斌、徐乾學、葉方靄互相參訂,僅成數卷。及臣回籍多年,恩召重領史局,而前此纂輯諸臣,罕有存者。惟大學士張玉書為監修,尚書陳廷敬為總裁,各專一類:玉書任志,廷敬任本紀,臣任列傳。因臣原銜食俸,比二臣得有餘暇,刪繁就簡,正謬訂譌。如是數年,匯分成帙,而大學士熊賜履續奉監修之命,檄取傳稿以進,玉書、廷敬暨臣皆未參閱。臣恐傳稿尚多舛誤,自蒙恩歸田,欲圖報稱,因重理舊編,搜殘補闕,復經五載,成列傳二百八卷。其間是非邪正,悉據公論,不敢稍逞私臆。但年代久遠,傳聞異辭,未敢自信為是。謹繕寫全稿,齎呈御鑒,請宣付史館,以備參考。」詔俞之。

五十四年,復召來京修書,充省方盛典總裁官。雍正元年,卒於京。乾隆四十三年,國史館進鴻緒傳,高宗命以郭琇劾疏載入,使後世知鴻緒輩罪狀。

孫興吾,進士,官吏部侍郎。

高士奇,字澹人,浙江錢塘人。幼好學能文。貧,以監生就順天鄉試,充書寫序班。工書法,以明珠薦,入內廷供奉,授詹事府錄事。遷內閣中書,食六品俸,賜居西安門內。康熙十七年,聖祖降敕,以士奇書寫密諭及纂輯講章、詩文,供奉有年,特賜表裏十匹、銀五百。十九年,復諭吏部優敘,授為額外翰林院侍講。尋補侍讀,充日講起居注官,遷右庶子。累擢詹事府少詹事。

二十六年,上謁陵,於成龍在道盡發明珠、餘國柱之私。駕旋,值太皇太后喪,不入宮,以成龍言問士奇,亦盡言之。上曰:「何無人劾奏?」士奇對曰:「人孰不畏死。」帝曰:「若輩重於四輔臣乎?欲去則去之矣,有何懼?」未幾,郭琇疏上,明珠、國柱遂罷相。二十七年,山東巡撫張汧以齎銀赴京行賄事發,逮治,獄辭涉士奇。會奉諭戒勿株連,於是置弗問。事

詳徐乾學傳。士奇因疏言:「臣等編摩纂輯,惟在直廬。宣諭奏對,悉經中使。非進講,或數月不覲天顏,從未干涉政事。不獨臣為然,前入直諸臣,如熊賜履、葉方靄、張玉書、孫在豐、王士禎、硃彞尊等,近今同事諸臣,如陳廷敬、徐乾學、王鴻緒、張英、勵杜訥等,莫不皆然。獨是供奉日久,嫌疑日滋。張汧無端疑怨,含沙污衊,臣將無以自明,幸賴聖明在上,誣構難施。但禁廷清秘,來茲萋斐,豈容仍玷清班?伏乞賜歸田里。」上命解任,仍領修書事。二十八年,從上南巡,至杭州,幸士奇西溪山莊,御書「竹窗」榜額賜之。

未幾,左都御史郭琇劾奏曰:「皇上宵旰焦勞,勵精圖治,用人行政,未嘗纖毫假手左右。乃有原任少詹事高士奇、左都御史王鴻緒等,表裏為奸,植黨營私,試略陳其罪。士奇出身微賤,其始徒步來京,覓館為生。皇上因其字學頗工,不拘資格,擢補翰林。令入南書房供奉,不過使之考訂文章,原未假之與聞政事。而士奇日思結納,諂附大臣,攬事招權,以圖分肥。內外大小臣工,無不知有士奇者。聲名赫奕,乃至如此。是其罪之可誅者一也。久之羽翼既多,遂自立門戶,結王鴻緒為死黨,給事中何楷為義兄弟,翰林陳元龍為叔侄,鴻緒兄頊齡為子女姻親,俱寄以心腹,在外招攬。凡督、撫、籓、臬、道、府、、縣及在內大小卿員,皆鴻緒、楷等為之居停,哄騙餽至,成千累萬。即不屬黨護者,亦有常例,名之曰『平安錢』。是士奇等之奸貪壞法,全無顧忌,其罪之可誅者二也。光棍俞子易,在京肆橫有年,事發潛遁。有虎坊橋瓦房六十餘間,價值八千金,餽送士奇。此外順成門外斜街並各處房屋,令心腹出名置買,寄頓賄銀至四十餘萬。又於本鄉平湖縣置田產千頃,大興土木,杭州西溪廣置園宅。以覓館餬口之窮儒,忽為數百萬之富翁。試問金從何來?無非取給於各官。官從何來?非侵國帑,即剝民膏。是士奇等真國之蠹而民之賊也,其罪之可誅者三也。皇上洞悉其罪,因各館編纂未竣,令解任修書,矜全之恩至矣!士奇不思改過自新,仍怙惡不悛,當聖駕南巡,上諭嚴戒餽送,以軍法治罪。惟士奇與鴻緒愍不畏死,鴻緒在淮、揚等處,招攬各官餽送萬金,潛遺士奇。淮、揚如此,他處可知。是士奇等欺君滅法,背公行私,其罪之可誅者四也。王鴻緒、陳元龍鼎甲出身,儼然士林翹楚;竟不顧清議,依媚大臣,無所不至。茍圖富貴,傷敗名教,豈不玷朝班而羞當世之士哉?總之高士奇、王鴻緒、陳元龍、何楷、王頊齡等,豺狼其性,蛇蠍其心,鬼蜮其形。畏勢者既觀望而不敢言,趨勢者復擁戴而不肯言。臣若不言,有負聖恩。故不避嫌怨,請立賜罷斥,明正典刑,天下幸甚。」疏入,士奇等俱休致回籍。副都御史許三禮復疏劾解任尚書徐乾學與士奇姻親,招搖納賄,相為表裏。部議以所劾無據,得寢。

三十三年,召來京修書。士奇既至,仍直南書房。三十六年,以養母乞歸,詔允之,特授詹事府詹事。尋擢禮部侍郎,以母老未赴。四十二年,上南巡,士奇迎駕淮安,扈蹕至杭州。及回鑾,復從至京師,屢入對,賜予優渥。上顧侍臣曰:「朕初讀書,內監授以四子本經,作時文;得士奇,始知學問門徑。初見士奇得古人詩文,一覽即知其時代,心以為異,未幾,朕亦能之。士奇無戰陣功,而朕待之厚,以其裨朕學問者大也。」尋遣歸,是年卒於家。上深惜之,命加給全葬,授其子庶吉士輿為編修。尋謚文恪。

論曰:儒臣直內廷,謂之「書房」,存未入關前舊名也。上書房授諸皇子讀,尊為師傅;南書房以詩文書畫供御,地分清切,參與密勿。乾學、士奇先後入直,鴻緒亦以文學進。乃憑藉權勢,互結黨援,納賄營私,致屢遭彈劾,聖祖曲予保全。乾學、鴻緒猶得以書局自隨,竟編纂之業,士奇亦以恩禮終,不其幸歟!


\end{pinyinscope}