\article{列傳三十六}

\begin{pinyinscope}
索尼蘇克薩哈蘇納海硃昌祚王登聯白爾赫圖

遏必隆子尹德鼇拜弟穆里瑪班布爾善

索尼,赫舍里氏,滿洲正黃旗人。父碩色,大學士希福兄也,太祖時,自哈達挈家來歸。太祖以其兄弟父子並通國書及蒙、漢文字,命碩色與希福同直文館,賜號「巴克什」。授索尼一等侍衛。從征界籓、棟夔。天聰元年,從太宗攻錦州,偵敵寧遠,並有功。

二年,上親征喀爾喀,徵兵外籓,科爾沁不至。命索尼與侍衛阿珠祜齎諭飭責土謝圖額駙奧巴。初,奧巴為臺吉,入朝,太祖以貝勒舒爾哈齊女妻焉。既而奧巴屢背約,私與明通,復徵兵不至。索尼受方略行,既入境,其部人饋以牲,索尼不受,曰:「爾汗有異心,爾物豈可食耶?」時奧巴病足,索尼與阿珠祜見公主,以諭旨告。奧巴聞之,扶掖至,佯問曰:「此為誰?」索尼曰:「吾儕天使也!爾有罪,義當絕。今特以公主故,使來餽問耳。」奧巴顧左右趣具饌,索尼等不顧而出。奧巴恐,使臺吉塞冷等請其事。索尼出璽書示之,即令從者先行。奧巴得書大驚,令所屬大臣𧾷忌留,索尼責以大義,奧巴叩首悔罪,原入朝。索尼與阿珠祜偕其大臣黨阿賴先歸奏狀,帝甚悅。

三年,從大軍入關,薄燕京,明督師袁崇煥赴援,列營城東南。貝勒豪格突入陣,敵兵蹙之,矢石如雨。索尼躍馬馳入,斬殺甚眾,拔豪格破圍出。四年,諭降榛子鎮、沙河驛,拔永平,守之。五年,擢吏部啟心郎。從圍大凌河。明兵自錦州來援,敗之。六年,從征察哈爾,略大同,取阜臺寨。尋予牛錄章京世職,仍直內院。崇德八年,考績,進三等甲喇章京。

太宗崩後五日,睿親王多爾袞詣三官廟,召索尼議冊立。索尼曰:「先帝有皇子在,必立其一。他非所知也。」是夕,巴牙喇纛章京圖賴詣索尼,告以定立皇子。黎明,兩黃旗大臣盟於大清門,令兩旗巴牙喇兵張弓挾矢,環立宮殿,率以詣崇政殿。諸王大臣列坐東西廡,索尼及巴圖魯鄂拜首言立皇子,睿親王令暫退。英親王阿濟格、豫親王多鐸勸睿親王即帝位,睿親王猶豫未允,豫親王曰:「若不允,當立我。我名在太祖遺詔。」睿親王曰:「肅親王亦有名,不獨王也。」豫親王又曰:「不立我,論長當立禮親王。」禮親王曰:「睿親王若允,我國之福。否則當立皇子。我老矣,能勝此耶?」乃定議奉世祖即位。索尼與譚泰、圖賴、鞏阿岱、錫翰、鄂拜盟於三官廟,誓輔幼主,六人如一體。都統何洛會等訐告肅親王豪格,王坐廢,詔褒索尼不附王,賜鞍馬。

順治元年,從睿親王入關,定京師。二年,晉二等昂邦章京。睿親王令解啟心郎職,仍理部事。睿親王方擅政,譚泰、鞏阿岱、錫翰皆背盟附之,憾索尼不附。李自成之敗也,焚宮殿西走。至是議修建,睿親王亦營第,■H0工庀材,工部給直偏厚,諸匠役皆急營王第。佟機言於王,王怒,欲殺之。索尼力言其無罪,王以是愈憾索尼。英親王阿濟格慢上,目為「八歲幼兒」,索尼以告睿親王,請罪之,王不許。王嘗召諸大臣議分封諸王,索尼持不可。鞏阿岱、錫翰進曰:「索尼不欲王平天下乎?」請罪之,王亦不許。索尼發固山額真譚泰隱匿詔旨,譚泰坐削公爵;因訐索尼以內庫漆琴與人,及使牧者秣馬庫院,傔從捕魚禁門橋下,索尼遂坐罷。

三年,巴牙喇纛章京圖賴劾譚泰怨望,詞涉索尼。順治初,大軍分道剿賊西安,譚泰後至,無功。及移師江南,譚泰慮勿預,語圖賴,甚怏怏。圖賴遺書索尼,使啟睿親王,齎書者私發之,恐譚泰獲罪,沉諸河。圖賴發前事,逮訊齎書者塞爾特,詭雲書巳達索尼。諸大臣論索尼罪當斬,王親鞫之,索尼曰:「吾前發譚泰匿詔旨罪,顧匿圖賴書以庇之乎?」王窮訊齎書者,事得白。尋復世職,然王與譚泰等憾索尼滋甚。五年,值清明,遣索尼祭昭陵,既行,貝子屯齊訐索尼與圖賴等謀立肅親王,論死,末減,奪官,籍其家,即安置昭陵。

八年,世祖親政,特召還,復世職。累進一等伯世襲,擢內大臣,兼議政大臣、總管內務府。十七年,應詔上言,略謂:「小民冤抑,有司不為詳審者,請嚴察,使毋壅於上聞。犯罪發覺,其奉有嚴旨者,有司輒從重比,不無枉濫。請敕法司詳慎。前議福建將士失律罪,在大將軍止削一不世襲之拜他喇布勒哈番,而所屬將領乃盡奪世職,輕重不平,有乖懲勸,請敕更正。開國諸臣,自拜他喇布勒哈番以上皆有功業,宜予世襲;其後恩詔所加,非有戰功,請毋給世襲敕書。在外諸籓,風俗不齊,若必嚴以內定之例,恐反滋擾,請予以優容。大臣奪據行市,奸宄之徒,投托指引,以攘貨財,四方商賈,負擔來京,輒復勒價強買。諸王貝勒及大臣私引玉泉山水灌溉,泉流為之竭。邊外木植,皆商人雇民採伐。今又為大臣私行強占,致商不聊生。大臣不殫心公事,惟飾宅第。皆請申禁。五城審事官,遇世族富家與窮民訟者,必罪窮民,曲意徇私,不思執法。請嚴飭毋得枉屈賄庇。」疏入,上以所奏皆實,飭議行。

十八年,世祖崩,遺詔以索尼與蘇克薩哈、遏必隆、鼇拜同輔政。索尼聞命,跪告諸王貝勒,請共任國政,諸王貝勒皆曰:「大行皇帝深知汝四大臣,委以國家重務,誰敢干預?」索尼等乃奏知皇太后,誓於上帝及大行皇帝前,其辭曰:「先皇帝不以索尼、蘇克薩哈、遏必隆、鼇拜等為庸劣,遺詔寄託,保翊沖主。索尼等誓協忠誠,共生死,輔佐政務。不私親戚,不計怨仇,不聽旁人及兄弟子侄教唆之言,不求無義之富貴,不私往來諸王貝勒等府受其餽遺,不結黨羽,不受賄賂,惟以忠心仰報先皇帝大恩。若各為身謀,有違斯誓,上天殛罰,奪算兇誅。」誓訖,乃受事。

世祖定中國,既親政,紀綱法度,循太祖、太宗遺制;亦頗取明舊典損益之,務使稱國體。四輔臣為政,稱旨諭諸王、貝勒、諸大臣,詳考太祖、太宗成憲,勒為典章。引世祖遺詔,謂:「不能仰法太祖、太宗,多所更張;今當率祖制,復舊章,以副先帝遺意。」乃改內閣翰林院還為內三院,復設理籓院,罷裁太常、光祿、鴻臚諸寺。他舉措皆類是。而鑲黃、正白兩旗互易圈地,興大獄。四輔臣稱旨,亦謂太祖、太宗時,八旗莊田廬舍,依左右翼順序分給。既入關,睿親王多爾袞使鑲黃旗處右翼之末,正白旗圈地本當屬鑲黃旗,今還與相易,亦以復舊制。

索尼故不慊蘇克薩哈,顧見鼇拜勢日張,與蘇克薩哈不相容,內怵;又念年已老,多病,康熙六年三月,遂與蘇克薩哈、遏必隆、鼇拜共為奏請上親政。上未即允,而詔褒索尼忠,加授一等公,與前授一等伯並世襲,索尼辭,不許。六月,卒,謚文忠,賜祭葬有加禮。七月,乃下索尼等奏,上親政,以第五子心裕襲一等伯,法保襲一等公。長子噶布喇官領侍衛內大臣,孝誠皇后父也,十三年,後崩,推恩所生,授一等公,世襲。第三子索額圖,自有傳。

蘇克薩哈,納喇氏,滿洲正白旗人。父蘇納,葉赫貝勒金臺什同族。太祖初創業,來歸,命尚主為額駙,授牛錄額真。累進梅勒額真。天聰初,從太宗征錦州,貝勒莽古爾泰帥偏師衛塔山餉道,蘇納屯塔山西,明兵來攻,擊破之。三年,與固山額真武納格擊察哈爾,入境,降其民二千戶。聞降者將為變,盡殲其男子,俘婦女八千餘,上責其妄殺。蒙古人有自察哈爾逃入明邊者,命蘇納以百人逐之,所俘獲相當。累進三等甲喇章京。坐隱匿丁壯,削職。尋授正白旗蒙古固山額真。崇德初,從伐明,攻雕鶚、長安諸堡及昌平諸城,五十六戰皆捷。又攻破容城。及出邊,後隊潰,坐罰鍰。又從伐朝鮮,擊破朝鮮軍,俘其將。以朝鮮王出謁時亂班釋甲,又自他道還,坐罰鍰。尋以讞獄有所徇,坐罷,仍專管牛錄事。順治五年,卒。

蘇克薩哈初授牛錄額真。崇德六年,從鄭親王濟爾哈朗圍錦州,明總督洪承疇師赴援,太宗親帥大軍蹙之,蘇克薩哈戰有功,授牛錄章京世職,晉三等甲喇章京。順治七年,世祖命追復蘇納世職,以蘇克薩哈並襲為三等阿思哈尼哈番。尋授議政大臣,進一等,加拖沙喇哈番。蘇克薩哈隸睿親王多爾袞屬下,王薨,蘇克薩哈與王府護衛詹岱等訐王謀移駐永平諸逆狀,及殯斂服色違制,王坐是追黜。是年,擢巴牙喇纛章京。

十年,孫可望寇湖廣,命蘇克薩哈偕固山額真陳泰率禁旅出鎮湖南,與經略洪承疇會剿。十二年,劉文秀遣其將盧明臣等分兵犯岳州、武昌,蘇克薩哈邀擊,大敗之。文秀引兵寇常德,戰艦蔽江,蘇克薩哈六戰皆捷,縱火焚其舟,斬獲甚眾,明臣赴水死,文秀走貴州。敘功,晉二等精奇尼哈番,擢領侍衛內大臣,加太子太保。

聖祖立,受遺詔輔政。時索尼為四朝舊臣,遏必隆、鼇拜皆以公爵先蘇克薩哈為內大臣,鼇拜尤功多,意氣凌轢,人多憚之。蘇克薩哈以額駙子入侍禁廷,承恩眷,班行亞索尼;與鼇拜有姻連,而論事輒齟,浸以成隙。鼇拜隸鑲黃旗,與正白旗互易莊地,遂興大獄。大學士兼戶部尚書蘇納海,總督硃昌祚、巡撫王登聯坐紛更阻撓,下刑部議罪,以律無正條,請鞭責籍沒。上覽奏,召輔臣議,鼇拜請置重典,索尼、遏必隆不能爭,獨蘇克薩哈不對,上因不允。鼇拜卒矯命,悉棄市。

鼇拜以蘇克薩哈與相抗,憾滋甚。鼇拜日益驕恣,蘇克薩哈居常怏怏。康熙六年,上親政,加恩輔臣。越日,蘇克薩哈奏乞守先帝陵寢,庶得保全餘生。有旨詰問,鼇拜與其黨大學士班布爾善等遂誣以怨望,不欲歸政,構罪狀二十四款,以大逆論,與其長子內大臣查克旦皆磔死;餘子六人、孫一人、兄弟子二人皆處斬,籍沒;族人前鋒統領白爾赫圖、侍衛額爾德皆斬:獄上,上不允。鼇拜攘臂上前,強奏累日,卒坐蘇克薩哈處絞,餘悉如議。八年,鼇拜敗,詔以蘇克薩哈雖有罪,不至誅滅子孫,此皆鼇拜挾仇所致,命復官及世爵,以其幼子蘇常壽襲。

蘇納海,他塔喇氏,滿州正白旗人。由王府護衛擢弘文院學士,累遷工部尚書,加太子少保。聖祖即位,拜國史院大學士,兼管戶部。時鼇拜擅權,以蘇納海不阿附,嗛之。尋鼇拜欲以薊、遵化、遷安正白旗諸屯莊改撥鑲黃旗,而別圈民地益正白旗,使旗人訴請牒戶部。蘇納海持不可,謂旗人安業已久,且奉旨不許再圈民地,宜罷議,鼇拜益銜之,矯旨遣貝子溫齊等履勘。旋以鑲黃地不堪耕種疏聞,遂遣蘇納海會直隸總督硃昌祚、巡撫王登聯董理其事。昌祚、登聯交章請停圈換,蘇納海亦言屯地難丈量,候明詔進止,鼇拜遂坐以藐視上命,並棄市。鼇拜獲罪,昭雪復官,謚蘇納海襄愍,昌祚勤愍,登聯愨愍。

昌祚,字雲門,漢軍鑲白旗人。順治初,官宗人府啟心郎。十八年,以工部侍郎巡撫浙江,清廉沉毅。平寇盜,撥荒地,給瀕海內徙居民開墾,免其所棄田畝丁糧,戒所司藉端苛斂,浙人德之。康熙四年,擢直隸、山東、河南三省總督。圈地議起,旗民失業者數十萬人。昌祚抗疏力言其不便,卒以冤死。祀直隸、浙江名宦。

登聯,字捷軒,漢軍鑲紅旗人。自貢生授河南鄭州知州,薦擢山東濟寧道,累遷大理寺卿。順治十七年,授保定巡撫。嚴緝捕,盜賊屏息。康熙五年,以京東諸路圈地擾民,疏請停止,言甚痛切。民聞其死,甚哀之。祀直隸名宦。

白爾赫圖,初由噶布什賢壯達授兵部副理事官。崇德間,屢從征有功,擢噶布什賢章京。順治元年,入關,擊李自成,敗賊將唐通於一片石,多斬獲。尋從豫親王多鐸西剿流寇,克潼關。移師江南,徇蘇州,略定浙江、福建。五年,從鄭親王濟爾哈朗征湖南,大破賊於湘潭,平寶慶、武岡。累功,晉一等阿達哈哈番,擢噶布什賢噶喇依昂邦。

十五年,從信郡王多尼徵貴州,屢陷陣,進克雲南。逾年,率兵取永昌府,渡潞江,敗李定國,遂克騰越州。明桂王由榔及定國、白文選俱遁入緬甸。信郡王班師,白爾赫圖留駐雲南。定國入犯,約降將高應鳳內應,以由榔印劄誘元江土司那嵩叛,白爾赫圖往剿,斬應鳳於陣,那嵩自焚死,賜白金、鞍馬。十八年,與定西將軍愛星阿會師木邦,緬人獻由榔至軍中。康熙元年,詔班師。進一等阿思哈尼哈番。

後蘇克薩哈為鼇拜構陷,以白爾赫圖為其族弟,竟被禍。八年,上以白爾赫圖無罪枉坐,追復故官世職。尋其子一等侍衛羅鐸訟其父云南戰功為鼇拜所抑,未予優敘,詔晉三等精奇尼哈番,賜祭葬,謚忠勇。

遏必隆,鈕祜祿氏,滿州鑲黃旗人。額亦都第十六子,母和碩公主。天聰八年,襲一等昂邦章京,授侍衛,管牛錄事。貝勒尼堪福晉,遏必隆兄圖爾格女也,無子,詐取僕婦女為己生。事發,遏必隆坐徇庇,奪世職。崇德六年,從太宗伐明,營松山,築長圍守之。明總兵曹變蛟率步騎突圍,迭敗之。夜三鼓,變蛟集潰卒突犯御營,遏必隆與內大臣錫翰等力戰,殪十餘人,變蛟負創走。論功,得優賚。七年,從饒餘貝勒阿巴泰等入長城,克薊州;進兵山東,攻夏津,先登,拔之:予牛錄章京世職。

順治二年,從順承郡王勒克德渾剿李自成兄子錦於武昌,拔鐵門關,進二等甲喇章京。五年,兄子侍衛科普索訐其與白旗諸王有隙,設兵護門,奪世職及佐領。世祖親政,遏必隆訟冤,詔復職。科普索旋獲罪,以所襲圖爾格二等公爵令遏必隆並襲為一等公。尋授議政大臣,擢領侍衛內大臣,累加少傅兼太子太傅。十八年,受遺詔為輔政大臣。

康熙六年,聖祖親政,加恩輔臣,特封一等公,以前所襲公爵授長子法喀,賜雙眼花翎,加太師。屢乞罷輔政,許之。四大臣當國,鼇拜獨專恣,屢矯旨誅戮大臣。遏必隆知其惡,緘默不加阻,亦不劾奏。八年,上逮治鼇拜,並下遏必隆獄。康親王傑書讞上遏必隆罪十二,論死,上宥之,削太師,奪爵。九年,上念其為顧命大臣,且勛臣子,命仍以公爵宿衛內廷。十二年,疾篤,車駕親臨慰問。及卒,賜祭葬,謚恪僖,禦制碑文,勒石墓道。十七

年,孝昭皇后崩,遏必隆為後父,降旨推恩所生,敕立家廟,賜御書榜額。五十一年,上以遏必隆初襲額亦都世職,命其第四子尹德襲一等精奇尼哈番。

尹德初自佐領授侍衛,從聖祖征噶爾丹,扈蹕寧夏。尋自都統擢領侍衛內大臣,兼議政大臣。雍正五年,以病乞休,許致仕。未幾卒,謚愨敬。尹德恭謹誠樸,宿衛十餘年,未嘗有過。兼襲圖爾格二等公,歲祿所入,以均宗族,人皆賢之。尋祀賢良祠。乾隆元年,詔晉一等公。

鼇拜,瓜爾佳氏,滿州鑲黃旗人,衛齊第三子。初以巴牙喇壯達從征,屢有功。天聰八年,授牛錄章京世職,任甲喇額真。崇德二年,徵明皮島,與甲喇額真準塔為前鋒,渡海搏戰,敵軍披靡,遂克之。命優敘,進三等梅勒章京,賜號「巴圖魯」。六年,從鄭親王濟爾哈朗圍錦州,明總督洪承疇赴援,鼇拜輒先陷陣,五戰皆捷,明兵大潰,追擊之,擒斬過半。功最,進一等,擢巴牙喇纛章京。八年,從貝勒阿巴泰等敗明守關將,進薄燕京,略地山東,多斬獲。凱旋,敗明總督範志完總兵吳三桂軍。敘功,進三等昂邦章京,賚賜甚厚。

順治元年,隨大兵定燕京。世祖考諸臣功績,以鼇拜忠勤戮力,進一等。二年,從英親王阿濟格征湖廣,至安陸,破流賊李自成。進征四川,斬張獻忠於陣。下遵義、夔州、茂州諸郡縣。五年,坐事,奪世職。又以貝子屯齊訐告謀立肅親王,私結盟誓,論死,詔宥之,罰鍰自贖。是年,率兵駐防大同,擊叛鎮姜襄,迭敗之,克孝義。七年,復坐事,降一等阿思哈尼哈番。

世祖親政,授議政大臣。累進二等公,予世襲。擢領侍衛內大臣,累加少傅兼太子太傅。十八年,受顧命輔政。既受事,與內大臣費揚古有隙,又惡其子侍衛倭赫及侍衛西住、折克圖、覺羅塞爾弼同直御前,不加禮輔臣。遂論倭赫等擅乘御馬及取御用弓矢射鹿,並棄市。又坐費揚古怨望,亦論死,並殺其子尼侃、薩哈連,籍其家,以與弟都統穆里瑪。

初入關,八旗皆有分地。睿親王多爾袞領鑲黃旗,定分地在雄、大城、新安、河間、任丘、肅寧、容城諸縣。至是已二十年,旗、民相安久。鼇拜以地確,倡議八旗自有定序,鑲黃旗不當處右翼之末,當與正白旗薊、遵化、遷安諸州縣分地相易。正白旗地不足,別圈民地補之。中外皆言不便。蘇克薩哈為正白旗人,與相抗尤力。鼇拜怒,悉逮蘇納海等,棄市。事具蘇克薩哈傳。又追論故戶部尚書英俄爾岱當睿親王攝政時阿王意,授分地亂序,並及他專擅諸事,奪世職。時有竊其馬者,鼇拜捕斬之,並殺御馬群牧長。怒蒙古都統俄訥、喇哈達、宜理布於議政時不附己,即令蒙古都統不與會議。

鼇拜受顧命,名列遏必隆後,自索尼卒,班行章奏,鼇拜皆首列。日與弟穆里瑪、侄塞本特、訥莫及班布爾善、阿思哈、噶褚哈、瑪爾賽、泰必圖、濟世、吳格塞等黨比營私,凡事即家定議,然後施行。侍讀熊賜履應詔陳時政得失,鼇拜惡之,請禁言官不得陳奏。上親政,加一等公,其子納穆福襲二等公。世祖配天,加太師,納穆福加太子少師。鼇拜益專恣。戶部滿尚書缺員,欲以命瑪爾賽,上別授瑪希納,鼇拜援順治間故事,戶部置滿尚書二,強請除授。漢尚書王弘祚領部久,瑪爾賽不得自擅,乃因事齮而去之。卒,又擅子謚忠敏。工部滿尚書缺員,妄稱濟世才能,強請推補。

康熙八年,上以鼇拜結黨專擅,勿思悛改,下詔數其罪,命議政王等逮治。康親王傑書等會讞,列上鼇拜大罪三十,論大闢,並籍其家,納穆福亦論死,上親鞫俱實,詔謂:「效力年久,不忍加誅,但褫職籍沒。」納穆福亦免死,俱予禁錮。鼇拜死禁所,乃釋納穆福。

五十二年,上念其舊勞,追賜一等阿思哈尼哈番,以其從孫蘇赫襲。蘇赫卒,仍以鼇拜孫達福襲。世宗立,賜祭葬,復一等公,予世襲,加封號曰超武。乾隆四十五年,高宗宣諭群臣,追覈鼇拜功罪,命停襲公爵,仍襲一等男;並命當時為鼇拜誣害諸臣有褫奪世職者,各旗察奏,錄其子孫。

穆里瑪,衛齊第六子。衛齊卒,襲世職牛錄章京,授一等侍衛。順治初,遷甲喇額真。世職累進一等阿達哈哈番兼拖沙喇哈番。從征金聲桓,克饒州,遂下南昌。十七年,擢工部尚書,並授本旗滿洲都統。李自成將李來亨等降於明,竄伏鄖、襄山中,出劫掠為寇。康熙二年,授穆里瑪靖西將軍,圖海定西將軍,率師討之。來亨擁眾據茅麓山,穆里瑪督兵攻圍,九戰皆捷。來亨等夜襲總督李國英、提督鄭蛟麟營,穆里瑪赴援,大破之,來亨自焚死,餘眾降。論功,超進一等阿思哈尼哈番。鼇拜得罪,坐死。

班布爾善,太祖諸孫輔國公塔拜子也。初封三等奉國將軍,累進輔國公。康熙六年,以領侍衛內大臣拜秘書院大學士,諂事鼇拜。及事敗,王大臣劾奏班布爾善大罪二十一,坐絞。

同時坐鼇拜黨罪至死者,吏部尚書阿思哈、侍郎泰必圖、兵部尚書噶褚哈、工部尚書濟世、內秘書院學士吳格塞及鼇拜侄塞本特、訥莫、瑪爾賽,追奪官爵,削謚。

論曰:四輔臣當國時,改世祖之政,必舉太祖、太宗以為辭。然世祖罷明季三餉,四輔臣時復徵練餉,並令並入地丁考成。此非太祖、太宗舊制然也,則又將何辭?索尼忠於事主,始終一節,錫以美謚,誠無愧焉。蘇克薩哈見忌同列,遂致覆宗。遏必隆黨比求全,幾及於禍。鼇拜多戮無辜,功不掩罪。聖祖不加誅殛,亦云幸矣。


\end{pinyinscope}