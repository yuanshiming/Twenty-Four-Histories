\article{志一百一}

\begin{pinyinscope}
○河渠一

△黃河

中國河患,歷代詳矣。有清首重治河,探河源以窮水患。聖祖初,命侍衛拉錫往窮河源,至鄂敦塔拉,即星宿海。高宗復遣侍衛阿彌達往,西逾星宿更三百里,乃得之阿勒坦噶達蘇老山。自古窮河源,無如是之詳且確者。然此猶重源也。若其初源,則出蔥嶺,與漢書合。東行為喀什噶爾河,又東會葉爾羌、和闐諸水,為塔裏木河,而匯於羅布淖爾。東南潛行沙磧千五百里,再出為阿勒坦河。伏流初見,輒作黃金色,蒙人謂金「阿勒坦」,因以名之。是為河之重源。東北會星宿海水,行二千七百里,至河州積石關入中國。經行山間,不能為大患。一出龍門,至滎陽以東,地皆平衍,惟賴堤防為之限。而治之者往往違水之性,逆水之勢,以與水爭地,甚且因緣為利,致潰決時聞,勞費無等,患有不可勝言者。

自明崇禎末李自成決河灌汴梁,其後屢塞屢決。順治元年夏,黃河自復故道,由開封經蘭、儀、商、虞,迄曹、單、碭山、豐、沛、蕭、徐州、靈壁、睢寧、邳、宿遷、桃源,東逕清河與淮合,歷雲梯關入海。秋,決溫縣,命內祕書院學士楊方興總督河道,駐濟寧。二年夏,決考城,又決王家園。方興言:「自遭闖亂,官竄夫逃,無人防守。伏秋汛漲,北岸小宋、曹家口悉沖決,濟寧以南田廬多淹沒。宜乘水勢稍涸,鳩工急築。」上命工部遴員勘議協修。七月,決流通集,一趨曹、單及南陽入運,一趨塔兒灣、魏家灣,侵淤運道,下流徐、邳、淮陽亦多沖決。是年孟縣海子村至渡口村河清二日,詔封河神為顯佑通濟金龍四大王,命河臣致祭。明年,流通集塞,全河下注,勢湍激,由汶上決入蜀山湖。五年,決蘭陽。

七年八月,決荊隆硃源寨,直往沙灣,潰運堤,挾汶由大清河入海。方興用河道方大猷言,先築上游長縷堤,遏其來勢,再築小長堤。八年,塞之。九年,決封丘大王廟,沖圮縣城,水由長垣趨東昌,壞平安堤,北入海,大為漕渠梗。發丁夫數萬治之,旋築旋決。給事中許作梅,御史楊世學、陳斐交章請勘九河故道,使河北流入海。方興言:「黃河古今同患,而治河古今異宜。宋以前治河,但令入海有路,可南亦可北。元、明以迄我朝,東南漕運,由清口至董口二百餘里,必藉黃為轉輸,是治河即所以治漕,可以南不可以北。若順水北行,無論漕運不通,轉恐決出之水東西奔蕩,不可收拾。今乃欲尋禹舊跡,重加疏導,勢必別築長堤,較之增卑培薄,難易曉然。且河流挾沙,束之一,則水急沙流,播之九,則水緩沙積,數年之後,河仍他徙,何以濟運?」上然其言,乃於丁家寨鑿渠引流,以殺水勢。

是年,復決邳州,又決祥符硃源寨。戶部左侍郎王永吉、御史楊世學均言:「治河必先治淮,導淮必先導海口,蓋淮為河之下流,而濱海諸州縣又為淮之下流。乞下河、漕重臣,凡海口有為奸民堵塞者,盡行疏濬。其漕堤閘口,因時啟閉,然後循流而上。至於河身,剔淺去淤,使河身愈深,足以容水。」議皆不果行。十一年,復決大王廟。給事中林起龍劾方興侵冒,上解方興任,遣大理卿吳庫禮、工科左給事中許作梅往按。起龍坐誣,復方興任。十三年,塞大王廟,費銀八十萬。

十四年,方興乞休,以吏部左侍郎硃之錫代之。是年決祥符槐疙疸,隨塞。十五年,決陽柴溝姚家灣,旋塞。復決陽武慕家樓。十六年,決歸仁堤。先是御史孫可化疏陳淮、黃堤工,事下總河。之錫言:「桃源費家嘴及安東五口淤澱久,工繁費鉅。且黃河諺稱『神河』,難保不旋濬旋淤,惟有加意修防,補偏救弊而已。」之錫陳兩河利害,條上工程、器具、夫役、物料八弊。又言:「因材器使,用人所亟。獨治河之事,非澹泊無以耐風雨之勞,非精細無以察防護之理,非慈斷兼行無以盡群夫之力,非勇往直前無以應倉猝之機,故非預選河員不可。」因陳預選之法二:曰薦用,曰儲才;諳習之法二:曰久任,曰交代。又條上河政十事:曰議增河南夫役,曰均派淮工夫役,曰察議通惠河工,曰建設柳園,曰嚴剔弊端,曰釐覈曠盡銀兩,曰慎重職守,曰明定河工專職,曰申明激勸大典,曰酌議撥補夫役。均允行。

十七年,決陳州郭家埠、虞城羅家口,隨塞。康熙元年五月,決曹縣石香爐、武陟大村、睢寧孟家灣。六月,決開封黃練集,灌祥符、中牟、陽武、杞、通許、尉氏、扶溝七縣。七月,再決歸仁堤。河勢既逆入清口,又挾睢、湖諸水自決口入,與洪澤湖連,直趨高堰,沖決翟家壩,流成大澗九,淮陽自是歲以災告。二年,決睢寧武官營及硃家營。三年,決杞縣及祥符閻家寨,再決硃家營,旋塞。四年四月,河決上游,灌虞城、永城、夏邑,又決安東茆良口。

五年,之錫卒,以貴州總督楊茂勛為河道總督。六年,決桃源煙墩、蕭縣石將軍廟,逾年塞之。又決桃源黃家嘴,已塞復決,沿河州縣悉受水患,清河沖沒尤甚,三汊河以下水不沒骭。黃河下流既阻,水勢盡注洪澤湖,高郵水高幾二丈,城門堵塞,鄉民溺斃數萬,遣官蠲賑。冬,命明珠等相視海口,開天妃、石闥、白駒等閘,毀白駒奸民閉閘碑。

八年,決清河三汊口,又決清水潭。副都御史馬紹曾、巡鹽御史李棠交章劾茂勛不職,罷之,以羅多為河道總督。九年,決曹縣牛市屯,又決單縣譙樓寺,灌清河縣治。是歲五月暴風雨,淮、黃並溢,撞卸高堰石工六十餘段,沖決五丈餘,高、寶等湖受淮、黃合力之漲,高堰幾塌,淮陽岌岌可虞。工科給事中李宗孔疏言:「水之合從諸決口以注于湖也,江都、高、寶無歲不防堤增堤,與水俱高。以數千里奔悍之水,攻一線孤高之堤,值西風鼓浪,一瀉萬頃,而江、高、寶、泰以東無田地,興化以北無城郭室廬。他如淥陽、平望諸湖,淺狹不能受水。各河港疏濬不時,範公堤下諸閘久廢,入海港口盡塞。雖經大臣會閱,嚴飭開閘出水,而年深工大,所費不貲,兼為傍海奸灶所格,竟不果行。水迂回至東北廟灣口入海,七邑田舍沈沒,動經歲時。比宿水方消,而新歲橫流又已踵至矣。」御史徐越亦言高堰宜乘冬水落時大加修築。於是起桃源東至龍王廟,因舊址加築大堤三千三百三十丈有奇。臘後冰解水溢,沿河村舍林木剷刷殆盡。

十年春,河溢蕭縣。六月,決清河五堡、桃源陳家樓。八月,又決七里溝。以王光裕總督河道。光裕請復明潘季馴所建崔壩鎮等三壩,而移季太壩於黃家嘴舊河地,以分殺水勢。是歲,茆良口塞。十一年秋,決蕭縣兩河口、邳州塘池舊城,又溢虞城,遣學士郭廷祚等履勘。十二年,桃源七里溝塞。十三年,決桃源新莊口及王家營,又自新河鄭家口北決。十四年,決徐州潘家塘、宿遷蔡家樓,又決睢寧花山壩,復灌清河治,民多流亡。十五年夏,久雨,河倒灌洪澤湖,高堰不能支,決口三十四。漕堤崩潰,高郵之清水潭,陸漫溝之大澤灣,共決三百餘丈,揚屬皆被水,漂溺無算。上遣工部尚書冀如錫、戶部侍郎伊桑阿訪究利病。是歲又決宿遷白洋河、於家岡,清河張家莊、王家營,安東邢家口、二鋪口,山陽羅家口。塞桃源新莊。

十六年,如錫等覆陳河工壞潰情形,光裕解任勘問。以安徽巡撫靳輔為河督。輔言:「治河當審全局,必合河道、運道為一體,而後治可無弊。河道之變遷,總由議治河者多盡力於漕艘經行之處,其他決口,則以為無關運道而緩視之,以致河道日壞,運道因之日梗。河水裹沙而行,全賴各處清水並力助刷,始能奔趨歸海。今河身所以日淺,皆由從前歸仁堤等決口不即堵塞之所致。查自清江浦至海口,約長三百里,向日河面在清江浦石工之下,今則石工與地平矣。向日河身深二三四丈不等,今則深者不過八九尺,淺者僅二三尺矣。河淤運亦淤,今淮安城堞卑於河底矣。運淤,清江與爛泥淺盡淤,今洪澤湖底漸成平陸矣。河身既墊高若此,而黃流裹沙之水自西北來,晝夜不息,一至徐、邳、宿、桃,即緩弱散漫。臣目見河沙無日不積,河身無日不加高,若不大修治,不特洪澤湖漸成陸地,將南而運河,東而清江浦以下,淤沙日甚,行見三面壅遏,而河無去路,勢必沖突內潰,河南、山東俱有淪胥沈溺之憂,彼時雖費千萬金錢,亦難剋期補救。」因分列大修事宜八:曰取土築堤,使河寬深;曰開清口及爛泥淺引河,使得引淮刷黃;曰加築高家堰堤岸;曰周橋閘至翟家壩決口三十四,須次第堵塞;曰深挑清口至清水潭運道,增培東西兩堤;曰淮揚田及商船貨物,酌納修河銀;曰裁並河員以專責成;曰按裡設兵,畫堤分守。廷議以軍務未竣,大修募夫多,宜暫停。疏再上,惟改運土用夫為車運,餘悉如所請。

於是各工並舉。大挑清口、爛泥淺引河四,及清口至雲梯關河道,創築關外束水堤萬八千餘丈,塞於家岡、武家墩大決口十六,又築蘭陽、中牟、儀封、商丘月堤及虞城周家堤。明年,創建王家營、張家莊減水壩二,築周橋翟壩堤二十五里,加培高家堰長堤,山、清、安三縣黃河兩岸及湖堰,大小決口盡塞。優詔褒美。十八年,建南岸碭山毛城鋪、北岸大谷山減水石壩各一,以殺上流水勢。二十年,塞楊家莊,蓋決五年矣。是歲增建高郵南北滾水壩八,徐州長樊大壩外月堤千六百八十九丈。

大修至是已三年,河未盡復故道,輔自劾。部議褫職,上命留任。二十一年,決宿遷徐家灣,隨塞。又決蕭家渡。先是河身僅一線,輔盡堵楊家莊,欲束水刷之,而引河淺窄,淤刷鼎沸,遇徐家灣堤卑則決,蕭家渡土松則又決。會候補布政使崔維雅上河防芻議,條列二十四事,請盡變輔前法。上遣尚書伊桑阿、侍郎宋文運履勘,命維雅隨往。維雅欲盡毀減水壩,別圖挑築。伊桑阿等言輔所建工程固多不堅,改築亦未必成功。輔亦申辨「工將次第告竣,不宜有所更張」。並下廷議。因召輔至京,輔言「蕭家口明正可塞,維雅議不可行」,上是之,命還工。二十二年春,蕭家渡塞,河歸故道。明年,上南巡閱河,賜詩褒美。

二十四年秋,輔以河南地在上游,河南有失,則江南河道淤澱不旋踵。乃築考城、儀封堤七千九百八十九丈,封丘荊隆口大月堤三百三十丈,滎陽埽工三百十丈,又鑿睢寧南岸龍虎山減水閘四。上念高郵諸州湖溢淹民田,命安徽按察使於成龍修治海口及下河,聽輔節制。旋召輔、成龍至京集議。成龍力主開濬海口;輔言下河海口高內地五尺,應築長堤高丈六尺,束水趨海。所見不合,下廷臣議,亦各持一說。上以講官喬萊江北人,召問,萊言輔議非是。因遣尚書薩穆哈等勘議,還言開海口無益。會江寧巡撫湯斌入為尚書,詢之,斌言海口開則積水可洩,惟高郵、興化民慮毀廬墓為不便耳。乃黜薩穆哈,頒內帑二十萬,命侍郎孫在豐董其役。時又有督修下河宜先塞減水壩之議,上不許。召輔入對,輔言南壩永塞,恐淮弱不敵黃強,宜於高家堰外增築重堤,截水出清口不入下河,停丁溪等處工程。成龍時任直撫,示以輔疏,仍言下河宜濬,修重堤勞費無益。議不決。復遣尚書佛倫等勘議,佛倫主輔議。二十七年,御史郭琇劾輔治河無績,內外臣工亦交章論之,乃停築重堤,免輔官,以閩浙總督王新命代之,仍督修下河,鐫在豐級,以學士凱音布代之。

明年,上南巡,閱高家堰,謂諸臣曰:「此堤頗堅固,然亦不可無減水壩以防水大沖決。但靳輔欲於舊堤外更築重堤,實屬無益。」並以輔於險工修挑水壩,令水勢回緩,甚善。車駕還京,復其官。三十一年,新命罷,仍令輔為河督。輔以衰疾辭,命順天府丞徐廷璽副之。輔請於黃河兩岸值柳種草,多設涵洞,俱報可。是冬,輔卒,上聞,嘆悼,予騎都尉世職。以於成龍為河督。

越二年,召詢成龍曰:「減水壩果可塞否?」對曰:「不宜塞,仍照輔所修而行。」上曰:「如此,何不早陳?爾排陷他人則易,身任總河則難,非明驗耶?」三十四年,成龍遭父憂,以漕督董安國代之。明年,大水,決張家莊,河會丹、沁偪滎澤,徙治高埠。又決安東童家營,水入射陽湖。是歲築攔黃大壩,於雲梯關挑引河千二百餘丈,於關外馬家港導黃由南潮河東注入海。去路不暢,上游易潰,而河患日亟。三十六年,決時家馬頭。明年,仍以成龍為河督。三十八年春,上南巡,臨視高家堰等堤,謂諸臣曰:「治河上策,惟以深浚河身為要。河底浚深,則洪澤湖水直達黃河,興化、鹽城等七州縣無汎濫之患,田產自然涸出。若不治源,治流終無裨益。今黃、淮交會之口過於徑直,應將河、淮之堤各迤東灣曲拓築,使之斜行會流,則黃不致倒灌矣。」

明年,成龍卒,以兩江總督張鵬翮為河督。是歲塞時家馬頭,從鵬翮先疏海口之請,盡拆雲梯關外攔黃壩,賜名大清口;建宿遷北岸臨黃外口石閘,徐州南岸楊家樓至段家莊月堤。四十一年,上謂永定河石堤甚有益,欲推行黃河兩岸,自徐州至清口皆修石堤。鵬翮言「建築石工,必地基堅實。惟河性靡常,沙土松浮,石堤工繁費鉅,告成難以預料」。遂作罷。四十二年,上南巡,閱視河工,制河臣箴以賜鵬翮。秋,移建中河出水口於楊家樓,逼溜南趨,清水暢流敵黃,海口大通,河底日深,黃水不虞倒灌。上嘉鵬翮績,加太子太保。四十六年八月,決豐縣吳家莊,隨塞。明年,鵬翮入為刑部尚為,以趙世顯代之。四十八年六月,決蘭陽雷家集、儀封洪邵灣及水驛張家莊各堤。

六十年八月,決武陟詹家店、馬營口、魏家口,大溜北趨,注滑縣、長垣、東明,奪運河,至張秋,由五空橋入鹽河歸海。自河工告成,黃流順軌,安瀾十餘年矣,至是遣鵬翮等往勘。九月,塞詹家店、魏家口;十一月,塞馬營口。世顯罷,以陳鵬年署河道總督。六十一年正月,馬營口復決,灌張秋,奔注大清河。六月,沁水暴漲,沖塌秦家廠南北壩臺及釘船幫大壩。時王家溝引河成,引溜由東南會滎澤入正河,馬營堤因無恙。鵬年復於廣武山官莊峪挑引河百四十餘丈以分水勢。九月,秦家廠南壩甫塞,北壩又決,馬營亦漫開;十二月,塞之。

雍正元年六月,決中牟十里店、婁家莊,由劉家寨南入賈魯河。會鵬年卒,齊蘇勒為總河,慮賈魯河下注之水,山盱、高堰臨湖堤工不能容納,亟宜相機堵閉,上命兵部侍郎嵇曾筠馳往協議。七月,決梁家營、詹家店,復遣大學士張鵬翮往協修,是月塞。九月,決鄭州來童寨民堤,鄭民挖陽武故堤洩水,並沖決中牟楊橋官堤,尋塞。是歲建清口東西束水壩以禦黃蓄清。二年,以嵇曾筠為副總河,駐武陟,轄河南河務,東河分治自此始。六月,決儀封大寨、蘭陽板橋,逾月塞之。

三年六月,決睢寧硃家海,東注洪澤湖。明年四月,塞未竣,河水陡漲,沖塌東岸壩臺,睢寧、虹、泗、桃源、宿遷悉被淹,命兩廣總督孔毓珣馳勘協防,十二月塞。是月河清,起陜西府穀訖江南桃源。五年,齊蘇勒以硃家海素稱險要,增築夾壩月堤、防風埽,並於大溜頂沖處削陡岸為斜坡,懸密葉大柳於坡上,以抵溜之汕刷。久之,大溜歸中泓,柳枝沾掛泥滓,悉成沙灘,易險為平,工不勞而費甚省。因請凡河崖陡峻處,俱仿此行。六年,曾筠內遷禮部尚書,副總河如故,命署廣東按察使尹繼善協理江南河務。

七年,改河道總督為江南河道總督,駐清江,以孔毓珣任,省副總河。以曾筠為山東河道總督,駐濟寧。上以明臣潘季馴有每歲派夫加高堤身五寸之議,前靳輔亦以為言,計歲費不過三四萬,下兩河總督議。毓珣等請酌緩急,分年輪流加倍,約歲需二萬餘金,下部議行。八年,毓珣卒,曾筠調督南河,田文鏡兼署東河總督。五月,敕建河州口外河源神廟成,加封號。是月,河清,起積石關訖撒喇城查漢斯。是歲決宿遷及桃源沈家莊,旋塞。以封丘荊隆口大溜頂沖開黑堈口至柳園口引河三千三百五十丈。十年,增修高堰石堤成。十一年,揀派部院司員赴南河學習,期以三年。授曾筠文華殿大學士兼吏部尚書,督南河如故,命兩淮鹽政高斌就習河務。曾筠旋遭母憂,斌署南河總督。

乾隆元年四月,河水大漲,由碭山毛城鋪閘口洶湧南下,堤多沖塌,潘家道口平地水深三五尺。上以下流多在蕭、宿、靈、虹、睢寧、五河等州縣,今止議濬上源而無疏通下游之策,則水無歸宿,下江南、河南各督撫暨兩總河委勘會議,並移南副總河駐徐州以專督率。旋高斌請濬毛城鋪迤下河道,經徐、蕭、睢、宿、靈、虹至泗州安河陡門,紆直六百餘里,以達洪澤,出清口會黃,而淮揚京員夏之芳等言其不便。明年,召斌詢問,斌繪圖呈覽,乃知之芳等所言失實,令同總督慶復確估定議,並將開濬有利無害,曉喻淮揚士民。初,斌疏濬毛城鋪水道,別開新口塞舊口,以免黃河倒灌。至三年秋,河漲灌運,論者多歸咎新開運口。斌言:「十月後黃水平退,湖水暢流,新淤隨溜刷去,可無虞淺澀。」四年,斌又言「上年清水微弱,時值黃水異漲,並非開新口所致」,而南人言者不已。上遣大學士鄂爾泰馳勘,亦言新口宜開。明年,黃溜仍南逼清口,仿宋陳堯佐法,制設木龍二,挑溜北行。

六年,斌以宿遷至桃源、清河二百餘里,河流湍激,北岸只縷堤六,並無遙堤,又內逼運河,將運河南岸縷堤通築高厚,作黃河北岸遙堤,更於縷堤內擇要增築格堤九。未成,斌調督直隸,完顏偉繼之。先是上以河溜逼清口,倒漾為患,詔循康熙間舊跡,開陶莊引河,導黃使北,遣鄂爾泰會勘。議甫定,以汛水驟漲停工,斌亦去任。至是,完顏偉慮引河不就,於清口迤西、黃河南岸設木龍挑溜北走,引河之議遂寢。厥後四十一年,上決意開之,逾年工竣,新河直抵周家莊,會清東下,倒漾之患永絕。

七年,決豐縣石林、黃村,奪溜東趨,又決沛縣縷堤,旋塞。完顏偉調督東河,改白鍾山南河總督。初豐、沛決時,大學士陳世倌往勘,添建滾水石壩二於天然南北二壩處,以分洩水勢。十年,決阜寧陳家浦。時淮、黃交漲,沿河州縣被淹。漕督顧琮言:「陳家浦逼近海口,以下十餘里向無堤工,每遇水漲,任其散溢。若仍於此堵塞,是與水爭地,費多益少,應於上流築遙堤以束水勢。」事下訥親、高斌,仍議塞舊決口。十一年,鍾山罷,顧琮署南總河,建木龍三於安東西門,逼溜南趨,自木龍以上皆淤灘,化險為平。十三年,琮調督東河,詔大學士高斌管南河事。斌以雲梯關下二套漲出沙灘,大溜南趨,直逼天妃宮辛家蕩堤工,開分水引河,並修補徐州東門外蟄裂石堤。琮亦以祥符十九堡南岸日淤,大溜北趨逼堤根,建南北壩臺,並於壩外卷埽簽椿。十六年六月,決陽武,命斌赴工,會琮堵築,十一月塞。十七年,上以豫省河岸大堤外有大行堤一,連接直、東,年久殘缺,在直隸者,令方觀承勘修,其山東界內,有無汕刷殘缺,令鄂容安查修。鄂容安言曹、單二縣大行堤大小殘缺三千四百三十丈,並加幫卑薄,補築缺口三百三十餘丈,疏濬堤南洩水河以宣坡水。

十八年秋,決陽武十三堡。九月,決銅山張家馬路,沖塌內堤、縷越堤二百餘丈,南注靈、虹諸邑,入洪澤湖,奪淮而下。以尹繼善督南河,遣尚書舒赫德偕白鍾山馳赴協理。同知李焞、守備張賓侵帑誤工,為學習河務布政使富勒赫所劾,勘實,置之法。高斌及協理張師載坐失察,縛視行刑。是冬,河塞。

方銅山之始決也,下廷議,吏部尚書孫嘉淦獨主開減河引水入大清河,略言:「自順、康以來,河決北岸十之九。北岸決,潰運者半,不潰者半。凡其潰道,皆由大清河入海者也。蓋大清河東南皆泰山基腳,其道亙古不壞,亦不遷移。前南北分流時,已受河之半。及張秋潰決,且受河之全,未聞有沖城郭淹人民之事,則此河之有利無害,已足徵矣。今銅山決口不能收功,上下兩江二三十州縣之積水不能消涸,故臣言開減河也。上游減則下游微,決口易塞,積水早消。但河流湍急,設開減河而奪溜以出,不可不防,故臣言減入大清河也。現開減河數處,皆距大清河不遠。計大清河所經,只東阿、濟陽、濱州、利津四五州縣,即有漫溢,不過偏災,忍四五州縣之偏災,可減兩江二三十州縣之積水,並解淮、揚兩府之急難,此其利害輕重,不待智者而後知也。減河開後,經兩三州縣境,或有漫溢,築土埂以御之,一入大清河,則河身深廣,兩岸堵築處甚少,計費不過一二十萬,而所省下游決口之工費,賑濟之錢米,至少一二百萬,此其得失多寡,亦不待智者而後知也。計無便於此者。」上慮形勢隔礙,不能用。

自銅山塞後,月堤內積水尚深七八尺至丈八九尺。上命於引河兜水壩南再開引河分溜,使新工不受沖激。二十一年,決孫家集,隨塞。明年二月,上南巡至天妃閘閱木龍。時鍾山調總南河,偕東河總督張師載言:「徐州南北岸相距甚迫,一遇盛漲,時有潰決。請挑濬淤淺,增築堤工,並堵築北岸支河,為南北分籌之計。」制可。二十三年,命安徽巡撫高晉協理南河。秋七月,決竇家寨新築土壩,直注毛城鋪,漫開金門土壩。晉言:「土壩過高,阻遏水勢,以致壅決,不須再築。」上不許,並令開蔣家營、傅家窪引河仍導入黃。二十六年七月,沁、黃並漲,武陟、滎澤、陽武、祥符、蘭陽同時決十五口,中牟之楊橋決數百丈,大溜直趨賈魯河。遣大學士劉統勛、公兆惠馳勘,巡撫常鈞請先築南岸。上謂河流奪溜,宜亟堵楊橋,鈞言大謬,調撫江西,以胡寶瑔為河南巡撫,並令高晉赴豫協理。十一月塞,上聞大喜,命於工所立河神廟。

三十年,上南巡,祭河神,閱清口東壩木龍惠濟閘。三十一年,決銅沛之韓家堂,旋塞。三十三年,豫撫阿思哈請以豫工節省銀加築堤岸,總河吳嗣爵言:「豫省河面寬,溜勢去來無定,旋險旋平,若將土埽劃為成數,恐各工員視為年例額支,轉啟興工冒銷之弊。」議遂寢。明年,嗣爵言:「銅瓦廂溜勢上是,楊橋大工自四五埽至二十一埽俱頂沖迎溜。請於桃汛未屆拆修,加鑲層土層柴,鑲壓堅實。兩岸大堤外多支河積水,汛發時,引溜注堤,宜多築土壩攔截。」上俱可其奏。三十七年,東河總督姚立德言:「前築土壩,保固堤根,頻歲安瀾,已著成效。請俟冬春閒曠,培築土壩,密栽柳株,俾數年後溝槽淤平,可永固堤根。」上嘉獎之。

三十八年五月,河溢朝邑,漲至二丈五尺,民居多漂沒。三十九年八月,決南河老壩口,大溜由山子湖下注馬家蕩、射陽湖入海,板閘、淮安俱被淹沒,尋塞。四十一年,嗣爵言黃水倒灌洪湖、運河,清口挑挖引河恐於事無濟。會內遷,薩載署南總河,上命偕江南總督高晉勘議。晉等言:「臣晉在工二十餘年,歷經倒灌。惟有將清口通湖引河挑挖,使得暢流,匯黃東注,並力刷沙,則黃河不濬自深,海口不疏自治,補偏救弊,惟此一法。」又言:「清口西所建木龍,原冀排溜北趨,刷陶莊積土,使黃不逼清。但驟難盡刷,宜於陶莊積土之北開一引河,使黃離清口較遠,至周家莊會清東注,不惟可免倒灌,淤沙漸可攻刷,即圩堰亦資穩固,所謂治淮即以治黃也。」明年二月,引河成。上喜成此鉅工,一勞永逸,可廢數百年藉清敵黃之說,飭建河神廟於新口石壩,自制文記之。

四十三年,決祥符,旬日塞之。閏六月,決儀封十六堡,寬七十餘丈,地在諸口上,掣溜湍急,由睢州、寧陵、永城直達亳州之渦河入淮。命高晉率熟諳河務員弁赴豫協堵,撥兩淮鹽課銀五十萬、江西漕糧三十萬賑恤災民,並遣尚書袁守侗勘辦。八月,上游迭漲,續塌二百二十餘丈,十六堡已塞復決。十二月再塞之。越日,時和驛東西壩相繼蟄陷。遣大學士公阿桂馳勘。明年四月,北壩復陷二十餘丈。上念儀工綦切,以古有沈璧禮河事,特頒白璧祭文,命阿桂等詣工所致祭。四十五年二月塞。是役也,歷時二載,費帑五百餘萬,堵築五次始合,命於陶莊河神廟建碑記之。六月,決睢寧郭家渡,又決考城、曹縣,未幾俱塞。十一月,張家油房塞而復開。

四十六年五月,決睢寧魏家莊,大溜注洪澤湖。七月,決儀封,漫口二十餘,北岸水勢全注青龍岡。十二月,將塞復蟄塌,大溜全掣由漫口下注。四十七年,兩次堵塞,皆復蟄塌。阿桂等請自蘭陽三堡大壩外增築南堤,開引河百七十餘里,導水下注,由商丘七堡出堤歸入正河,掣溜使全歸故道,曲家樓漫口自可堵閉。上從其言。明年二月,引河成,三月塞。四十九年八月,決睢州二堡,仍遣阿桂赴工督率,十一月塞。

先是上念豫工連歲漫溢,堤防外無宣洩之路,欲就勢建減水壩,俾大汛時有所分洩,下阿桂及河、撫諸臣勘議。至是,阿桂等言:「豫省堤工,滎澤、鄭州土性高堅,距廣武山近,毋庸設減壩。中牟以下,沙土夾雜,或系純沙,建壩不能保固。至堤南洩水各河,惟賈魯河系洩水要路。經鄭州、中牟、祥符、尉氏、扶溝、西華至周家口入沙河。又惠濟系賈魯支河,二河窄狹淤墊,如須減黃,應大加挑浚,需費浩繁,非一時所能集事。惟蘭、儀、高家寨河勢坐灣,若挑濬取直,引溜北注,河道可以暢行。」上然之。五十一年秋,決桃源司家莊、煙墩,十月塞。明年夏,復決睢州,十月塞。十二月,山西河清二旬,自永寧以下長千三百里。五十四年夏,決睢寧周家樓,十月塞。五十九年,決豐北曲家莊,尋塞。

嘉慶元年六月,決豐汛六堡,刷開運河餘家莊堤,水由豐、沛北注山東金鄉、魚臺,漾入昭陽、微山各湖,穿入運河,漫溢兩岸,江蘇山陽、清河多被淹。南河總督蘭錫第導水入藺家山壩,引河由荊山橋分達宿遷諸湖,又啟放宿遷十家河竹絡壩、桃源顧家莊堤,洩水仍入河下注,並於漫口西南挑挖舊河,引溜東趨入正河,繪圖以聞。上令取直向南而東,展寬開挖,俾溜勢直注正河,較為得力。命兩江總督蘇凌阿、山東布政使康基田會勘籌辦。十一月,復因凌汛蟄塌壩身二十餘丈,時蘇凌阿按事江西,改命東河總督李奉翰赴工會辦。明年二月塞,加奉翰太子太保,調督兩江,兼管南河事。是年七月,河溢曹汛二十五堡。

三年春,壩工再蟄,奉翰自劾,遣大學士劉墉、尚書慶桂履勘,並責問奉翰等因循。墉等言漫口已跌成塘,眴屆凌汛,請展至秋後興工。八月,溢睢州,水入洪澤湖。上游水勢既分,曹工遂以十月塞。明年正月,睢工亦塞。三月,以河南布政使吳璥署東河總督。璥言:「豫東兩岸堤工丈尺加增,而淤墊如故,病在豐、曹、睢疊經漫溢,雖塞後順軌安瀾,然引河不能寬暢,且徐城河狹,旁洩過多,遂成中梗。去淤之法,惟在束水攻沙,以堤束水。聞江南河臣康基田培築堤工,極為認真,應令酌看堤埽情形,守護閘壩,宣洩有度,自可日見深通。」上命與基田商辦。八月,決碭汛邵家壩。十二月,已塞復滲漏,又料船不戒,延燒殆盡,基田奪職留工,調璥督南河,以河南布政使王秉韜為東河總督,移東河料物迅濟南河。

五年冬,邵家壩塞。六年九月,溢蕭南唐家灣,十一月塞。八年九月,決封丘衡家樓,大溜奔注,東北由範縣達張秋,穿運河東趨鹽河,經利津入海。直隸長垣、東平、開州均被水成災。上飭布政使瞻住撫恤,復遣鴻臚卿通恩等治賑,兵部侍郎那彥寶赴工,會同東河總督嵇承志堵築。明年二月塞。

十年閏六月,兩江總督鐵保言:「河防之病,有謂海口不利者,有謂洪湖淤墊者,有謂河身高仰者。此三說皆可勿論。惟宜專力於清口,大修各閘壩,借湖水刷沙而河治。湖水有路入黃,不虞壅滯,而湖亦治。」上嘉其言明晰扼要。「至謂清水敵黃,所爭在高下不在深淺,所論固是,但湖不深,焉能多蓄?是必蓄深然後力能敵黃。俟大汛後,會商南河總督徐端,迅將高堰五壩,及各閘壩支河,酌量施工。」時有議由王營減壩改河經六塘河入海者,鐵保偕南河總督戴均元上言:「新河堤長四百里,中段漫水甚廣,急難施工,必須二三年之久,約費三四百萬。堵築減壩,不過二三月,費只二百餘萬。且舊河有故道可尋,施工較易。」上從之。十一年四月,兵部侍郎吳璥再督東河。六月,復置南副總河,降徐端為之。七月,決宿遷周家樓。八月,決郭家房。先後塞之。

十二年六月,漫山、安馬港口、張家莊,分流由灌口入海,旋塞。七月,決雲梯關外陳家浦,分流強半由五辛港入射陽湖注海。十三年二月,陳家浦塞。鐵保等請復毛城鋪石壩、王營減壩,培兩岸大堤,接築雲梯關外長堤,及培高堰、山盱堤后土坡。遣大學士長麟等馳勘。太僕寺卿莫瞻菉言:「河入江南,惟資淮以為抵御。淮萃七十二河之水匯於洪澤,以堰、盱石堤五壩束之,令出清口匯黃入海,此即束水攻沙之道。今治南河,宜先治清口,保守五壩。五壩不輕啟洩,則湖水可並力刷黃。黃不倒灌,運河自可疏通。今河臣請接築雲梯關外長堤二百餘里,則於坐灣取直處,必須添築埽段以為防護。既設修防,必添建營,多設官兵。是徒多糜費之煩,未必收束刷之效。至謂修復毛城滾壩,挑挖洪、濉,為減黃流異漲,以保徐城則可,若恃此助清濟運則不可。自黃水入湖淤停,水勢奔注,堰、盱五壩且難防守,又何能使之暢出清口?故加培五壩,使湖水暢出,悉力敵黃,順流直下,即可淘刷河身以入海。」御史徐亮言:「鐵保等條陳修防各事,惟於原議高堰石坦坡,未曾籌及蓄清刷黃,專在固守高堰,實得全河關鍵,以柔制剛,其法最善。風浪沖擊,至坡則平。然全堰俱得坦坡外護,則五壩可永閉不開,清水可全力刷黃,淮陽可長登衣任席,此萬世永圖而目前急務也。海口,尾閭也。清口,咽喉也。高堰則心腹也。要害之地,宜先著力。」敬亦以為言。長麟等覆稱:「毛城壩易致沖決,應無庸議。王營減霸積水太深,難以施工。請改建滾壩於其西,並添築石壩。至碎石坦坡,工段綿長,時難猝辦,先築土坡。」餘如鐵保言。均元病免,端復督南河。

初,陳家浦漫溢,由射陽湖旁趨入海。鐵保等以挑河費鉅,徑由射陽湖入海,較正河為近,因有改河道之議。至是,命璥等履勘。璥等言:「前明及康熙間所有灌河入海之路,覆轍俱在。現北潮河匯流馬港口、張家莊漫水尚在,壅積可見。去路不暢,又不能刷出河槽,此外更無可另闢海口之路。仍請修復故道,接築雲梯關外大堤,束水東注。」上如其言。是年六月,決堂子對岸千根棋桿及荷花塘,掣通臨湖磚百餘丈,堂子對岸及千根棋桿隨塞,荷花塘既堵復蟄。端再降副總河,以璥總南河。明年正月塞。是年冬,築高堰碎石坦坡。十五年八月,端復督南河,省副總河。十一月,大風激浪,決山盱屬仁、義、智三壩磚石堤三千餘丈,及高堰屬磚石堤千七百餘丈。端啟高郵車邏大壩及下游歸江各閘壩,並先堵仁、智壩以洩水勢。時璥養病家居,上垂詢辦法。璥言義壩應一律堵築,高堰石工尤須於明年大汛前修竣。上嘉所論切要。未幾,仁、義、智三壩及馬港俱塞,河歸正道入海。

明年四月,馬港復決。五月,王營減壩蟄陷。七月,決邳北綿拐山及蕭南李家樓。十二月,王營減壩塞。十七年二月,李家樓亦塞。十八年九月,決睢州及睢南薛家樓、桃北丁家莊,褫東河總督李亨特職,以均元代之。明年正月,均元內召,起璥再督東河,董理睢工。二十年二月塞。二十三年六月,溢虞城。二十四年七月,溢儀封及蘭陽,再溢祥符、陳留、中牟,奪葉觀潮職,以李鴻賓督東河。璥時為刑部尚書,偕往會籌。未幾,陳留、祥符、中牟俱塞,而武陟縷堤決,觀潮連堵溝槽五。又決馬營壩,奪溜東趨,穿運注大清河,分二道入海。儀封缺口尋涸。上命枷示觀潮河干。均元以大學士偕侍郎那彥寶履勘。那彥寶留督馬營壩工。久之,壩基不定,鴻賓被斥責,遂以不諳河務辭。上怒,奪其職,觀潮復督東河。二十五年三月,馬營口塞,加河神金龍四大王、黃大王、硃大王封號。是月儀封又漫塌,削觀潮及豫撫琦善職。宣宗立,仍命璥及那彥寶赴工會辦,十二月塞。

道光元年,禮部右侍郎吳烜言:「據御史王雲錦函稱,去冬回籍過河,審視原武、陽武一帶,堤高如嶺,堤內甚卑。向來堤高於灘約丈八尺,自馬營壩漫決,灘淤,堤高於灘不過八九尺。若不急於增堤,恐至夏盛漲,不免有出堤之患。」上命河督張文浩偕豫撫姚祖同履勘。三年,江督孫玉庭。河督黎世序加培南河兩岸大堤,令高出盛漲水痕四五尺,除有工及險要處堤頂另估加寬,餘悉以丈五尺及二丈為度。五月工竣。四年十一月,大風,決高堰十三堡,山盱周橋之息浪菴壞石堤萬一千餘丈,奪文浩職,以嚴烺督南河,遣尚書文孚、汪廷珍馳勘。侍講學士潘錫恩言:「蓄清敵黃,相傳成法。大汛將至,則急堵禦黃壩,使黃水全力東趨。今文浩遲堵此壩,致黃河倒灌,釀成如此巨患。且欲籌減洩,當在下游。乃輒開祥符閘,減黃入湖。壩口既灌於下,閘口復灌於上,黃無出路,湖墊極高,為患不可勝言。」尋文孚等亦以為言。文浩遣戍。玉庭褫職留任。十二月,十三堡、息浪菴均塞。

五年十月,東河總督張井言:「自來當伏秋大汛,河員皆倉皇奔走,救護不遑。及至水落,則以現在可保無虞,不復求疏刷河身之策,漸至清水不能暢出,河底日高。堤身遞增,城郭居民,盡在水底之下。惟仗歲積金錢,抬河於最高之處。」上嘉所言切中時弊。初,琦善等有改移海口以減黃,拋護石坡以蓄清之議。至是,井言灌河海口屢改屢決,自不可輕易更張,即碎石坦坡,亦有議及流弊者。尤不可不從長計議。是月增培河南十三、山東漕河、糧河二堤堰壩戧各工,皆從井請也。

六年春,河復漲,命井偕琦善、烺會勘海口。琦善、烺知海口不能改,乃條上五事,皆一時補苴之計。井言:「履勘下游,河病中滿,淤灘梗塞難疏,海口無可移改,請由安東東門工下北岸別築新堤,改北堤為南堤,相距八里十里,中挑引河,導河由北傍舊河行至絲網濱入海。河水高堤內灘丈五六尺,引河挑深一丈,則水勢高下幾三丈,形勢順利。自東門工至御黃壩六十里,去路既暢,上游可落水四五尺。黃落則御壩可啟,束清壩,挑清水,外出刷黃,底淤攻盡,黃可落至丈餘。湖水蓄七八尺,已為建瓴,石工易保。」上善其策。於是烺坐堰、盱新工掣卸,降三品調署東河,而以井督南河,淮揚道潘錫恩副之,使經畫其事。而琦善以改河非策,請啟王家營減壩,將正河挑挖深通,放清水刷滌,再堵壩挽黃歸正河。已允行矣,給事中楊煊言:「嘉慶中王家營減壩開,上下游州縣俱災。如止減黃不奪溜,何必奏籌撫恤?今奏啟減壩,至預及撫血⼙堵口事宜,即與從前情形無異。下壅上潰,不可不防。」事下江督、河督會議。井初議安東改河,時撓之者謂東門工埽外有舊拋碎石,正當咽喉,恐有阻遏。井謂有石處可啟除其吳工碎石千餘方,但上下掣通,亦斷不致礙全河。然議者終以為疑。及井見煊奏,復言:「嘉慶間減壩遇水後,次年黃仍倒灌,今河底淤高丈四五尺,豈如當時深通。兼以洪湖石工隱患甚多,本年二月,存水丈二尺八寸,遇風已多掣卸。秋後湖水止能蓄至三丈,冬令有耗無增,來年重運經行,必黃水止存二丈八九尺,清方高於黃一尺。若黃加高,即成倒灌。禦黃壩外河底墊高,淤運淤湖,為害不小。且海州積水未消,鹽河遙堤地高,去路不暢,啟壩後河必抬高,徒深四邑之災,無補全河之病。請於減壩迤下安東門工上山安李工遙是外築北堤,斜向趨東,仍與前議改河堤工相連,增長七千餘丈,挑河至八套即入正河。李工至八套舊堤長四萬一千丈,取直築堤,僅長三萬二千餘丈,可避東門碎石之阻。河減清高,漕行自利。督臣意以開放減壩已經奏定,不得以旁觀一言輒思變計,並臚列七難駁臣所議。臣已逐條致覆。」疏入,上終以改河為創舉,從琦善議。

十一年七月,決楊河十四堡及馬棚灣,十二月塞。十二年八月,決祥符。九月,桃源奸民陳瑞因河水盛漲,糾眾盜挖於家灣大堤,放淤肥田,致決口寬大,掣全溜入湖。桃南通判田銳等褫職遣戍。是月祥符塞。明年正月,於家灣塞。十五年,以慄毓美為東河總督。時原武汛串溝受水寬三百餘丈,行四十餘里,至陽武汛溝尾復入大河,又合沁河及武陟、滎澤諸灘水畢注堤下。兩汛素無工,故無稭料,堤南北皆水,不能取土築堤。毓美試用拋磚法,於受沖處拋磚成壩。六十餘壩甫成,風雨大至,支河首尾決,而壩如故。屢試皆效。遂請減稭石銀兼備磚價,令沿河民設窯燒磚,每方石可購二方磚。行之數年,省帑百三十餘萬,而工益堅。會有不便其事者,持異議。於是御史李蓴請停燒磚。上遣蓴隨尚書敬徵履勘,卒以溜深急則磚不可恃,停之。十九年,毓美復以磚工得力省費為言,乃允於北岸之馬營、滎原兩堤,南岸之祥符下汛、陳留汛,各購磚五千方備用。

二十一年六月,決祥符,大溜全掣,水圍省城。河督文沖請照睢工漫口,暫緩堵築。遣大學士王鼎、通政使慧成勘議。文沖又請遷省治,上命同豫撫牛鑒勘議。時河溜由歸德、陳州折入渦會淮注洪澤湖,拆展御黃、束清各壩,尚不足資宣洩,並展放禮、智、仁壩,義河亦啟放。八月,鑒言節逾白露,水勢漸落,城垣可無虞,自未便輕議遷移。鼎等言:「河流隨時變遷,自古迄無上策,然斷無決而不塞、塞而不速之理。如文沖言,俟一二年再塞,且引睢工為證。查黃水經安徽匯洪澤,宣洩不及,則高堰危,淮揚盡成巨浸。況新河所經,須更築新堤,工費均難數計。即幸而集事,而此一二年之久,數十州縣億萬生靈流離,豈堪設想。且睢工漫口與此不同。河臣所奏,斷不可行。」疏入,解文沖任,枷示河干,以硃襄繼之。

二十二年,祥符塞,用帑六百餘萬,加鼎太子太師。七月,決桃源十五堡、蕭家莊,溜穿運由六塘河下注。未幾,十五堡掛淤,蕭家莊口刷寬百九十餘丈,掣動大溜,正河斷流。河督麟慶意欲改道,遣尚書敬徵、廖鴻荃履勘。敬徵等言,改河有礙運道,惟有汛堵漫口,挽歸故道,俟明年軍船回空後,築壩合龍,從之。十一月,以吏部侍郎潘錫恩總督南河。二十三年,御史雷以諴言,決口無庸堵塞,請改舊為支,以通運道。下錫恩勘議。錫恩言灌口非可行河之地,北岸無可改河之理,不敢輕議更張,漕船仍由中河灌塘。上然之,更命侍郎成剛、順天府尹李德會勘。六月,決中牟,水趨硃仙鎮,歷通許、扶溝、太康入渦會淮。復遣敬徵等赴勘,以鍾祥為東河總督,鴻荃督工。旋以尚書麟魁代敬徵。二十四年正月,大風,壩工蟄動,旋東壩連失五占,麟魁等降黜有差,仍留工督辦。七月,上以頻年軍餉河工一時並集,經費支絀,意欲緩至明秋興築。鍾祥等力陳不可。十二月塞,用帑千一百九十餘萬。二十九年六月,決吳城。十月,命侍郎福濟履勘,會同堵合。

咸豐元年閏八月,決豐北下汛三堡,大溜全掣,正河斷流。時侍郎瑞常典試江南,命試竣便道往勘,又命福建按察使查文經馳赴會辦。三年正月,豐北三堡塞,敕建河神廟,從河督楊以增請也。五月大雨,水長溜急,豐北大壩復蟄塌三十餘丈。上責以增及承修各員加倍罰賠。

五年六月,決蘭陽銅瓦廂,奪溜由長垣、東明至張秋,穿運注大清河入海,正河斷流。上念軍務未平,餉糈不繼,若能因勢利導,使黃流通暢入海,則蘭陽決口即可暫緩堵築。事下河督李鈞察奏。鈞旋陳三事:「曰順河築墊。東西千餘里築堤,所費不貲,何敢輕議。除河近城垣不能不築堤壩以資抵御,餘擬就漫水所及,酌定墊基,勸民接築,高不過三尺,水小藉以攔阻,水大聽其漫過。散水無力,隨漫隨淤,地面漸高,且變沙磧為沃壤矣。曰遇灣切灘。河性喜坐灣,每至漲水,遇灣則怒而橫決。惟於坐灣之對面,勸令切除灘嘴,以寬河勢,水漲即可刷直,就下愈暢,並可免兜灘沖決之虞。曰堵截支流。現在黃流漫溢,既不能築堅堤以束其流,又不能挑引河以殺其勢,宜乘冬令水弱溜平,勸民築壩斷流,再於以下溝槽跨築土格,高出數尺。漫水再入,上無來源,下無去路,冀漸淤成平陸。」東撫崇恩亦以為言。上令直隸、山東、河南各督撫妥為勸辦。

十一年,御史薛書堂言:「南河自黃水改道,下游已無工可修,請省南河總督及員。」下廷臣議。侍郎沈兆霖言:「導河始自神禹,九河故道皆在山東,入海處在今滄州,是禹貢之河,固由東北入海。自漢王莽時河徙千乘入海,而禹之故道失。歷東漢迄隋、唐,從無變異。宋神宗時,河分南北兩派並行,北派由北清河入海。即今大清河。至元至元間,會通河成,懼河北行礙運,而北流塞。歷今五六百年,河屢北決,無不挽之使南。說者謂河一入運,必挾泥沙以入海,而運道亦淤,故順河之性,北行為宜。乾隆朝,孫嘉淦請開減河入大清河一疏,言之甚詳,足破北行礙運之疑。夫河入大清,由利津入海,正今黃河所改之道。現在張秋以東,自魚山至利津海口,皆築民堰,惟蘭儀之北、張秋之南,河自決口而出,奪趙王河及舊引河,汎濫平原,田廬久被淹浸。張秋高家林舊堰殘缺過多,工程最鉅。如東明、長垣、菏澤、鄆城,其培築較張秋為易。宜乘此時順水之性,聽其由大清河入海,諭令紳民力籌措辦,或應開減河,或應築堤堰,統於水落興工。河慶順軌,民樂力田,缺額之地丁可復,歷年之賑濟可停,就此裁去南河總督及員,可省歲帑數十萬,而歸德、徐、淮一帶地幾千里,均可變為沃壤,逐漸播種升科,似亦一舉而兼數善者矣。」下直督恆福、東撫文煜、豫撫慶廉、東河總督黃贊湯勘議。六月,省南河總督,及淮揚、淮海、豐北、蕭南、宿南、宿北、桃南、桃北各道,改置淮揚徐海兵備道,兼轄河務。

同治二年,復省蘭儀、儀睢、睢寧、商虞、曹考五。六月,漫上南各屬,水由蘭陽下注,直、東境內涸出村莊,復被淹沒。菏澤、東明、濮、範、齊河、利津等州縣,水皆逼城下。署河督譚廷襄上言:「河已北行,攔水惟恃民墊,從未議疏導,恐漸次淤墊,海口稍有捍格阻滯,事更為難。查濮、範一帶舊有金堤,前臣任東撫時,設法修築,未久復被沖缺,上游毗連直隸開州處亦有沖缺。開州不修,濮、範築亦無益。東、長之墊,開、濮之堤,須設法集貲督民修築,庶可以衛城池而保廬墓。此外既未專設河員,要在沿河地方官督率修理,並勸助裒集,以助民力之不逮。請飭下直督、東撫迅將蘭陽下游漫溢地方,揀員會同該州縣妥辦。」從之。十二月又言:「今年夏秋陰雨,來源之盛,迥異尋常。一股直下開州,一股旁趨定陶、曹、單。豫省以有堤壩,幸獲保全。直、東則無,不能不聽其汎濫。迄今半載,直隸未聞如何經畫。開州缺口,亦未興工。至山東被害尤深。或欲培築堤墊,或欲疏濬支河,議無一定。濮州金堤,亦因開未動工。不能興辦。瞬屆春汛,何以御之?臣遣運河道宗稷辰履勘,直至利津之鐵門關,測量水勢,深至六七丈,去路不為不暢,而上游仍到處旁溢,則大清河身太狹不能容納之故。如蒲臺、齊東、濟陽、長清、平陰、肥城民墊缺口,寬數丈或數十丈,不下三四十處,不加修築,則來歲依然漫淹。是欲求下游永奠,必先開支渠以減漲水,而後功有可施。必將附近徒駭、馬頰二河設法疏濬,庶水有分洩,再堵各缺口,並築壩以護近水各城垣,此大清河下游之當先料理者也。至開、濮金堤及毗連菏澤之史家堤,當先堵築,並加培舊堰,擇要接修,此大清河上游之當先經畫者也。」復下直督劉長佑、東撫閻敬銘會籌。明年三月,以濮州當河沖,允敬銘請,移治舊城,並築堤捍禦。

五年七月,決上南胡家屯。長佑言:「溜勢趨重西北,新修金堤,概被沖刷。開州沖開支河數道,自開、滑之杜家寨至開、濮界之陳家莊,險工五段,長九千六百餘丈,均須加厚培高,方資捍禦。惟上游在豫,下游在東,非直隸一省所能辦理。應會同三省統籌全修,再行設汛,撥款備料,庶可一勞永逸。自河流改道,直隸堤工應並歸河督管轄,作豫、直、東三省河督,以專責成。」疏入,命河督蘇廷魁履勘,會同三省督撫籌議。

七年六月,決滎澤十堡,又漫武陟趙樊村,水勢下注潁、壽入洪澤湖。侍郎胡家玉言:「不宜專塞滎澤新口、疏蘭陽舊口,宜仿古人發卒治河成法,飭各將領督率分段挑濬舊河,一律深通,然後決上游之水,掣溜東行,庶河南之患不移於河北,治河即所以治漕。」下直督曾國籓、鄂督李瀚章、江督馬新貽、漕督張之萬,及河督,江蘇、河南、山東、安徽各巡撫妥議。國籓等言:「以今日時勢計之,河有不能驟行規復者三。蘭陽漫決已十四年,自銅瓦廂至雲梯關以下,兩岸堤長千餘里,歲久停修,堤塌河淤,今欲照舊時挑深培高,恐非數千萬金不能蕆事。且營久裁,兵夫星散,一一復設,仍應分儲料物,廂辦埽壩,並預籌防險之費,又歲須數百萬金。當此軍務初平,庫藏空虛,安從籌此鉅款?一也。滎澤地處上游,論形勢自應先堵滎澤,蘭工勢難並舉。使滎口掣動全黃,則蘭工可以乾涸。今滎口分溜無多,大溜仍由蘭口直注利津入海,其水面之寬,跌塘之深,施工之難,較之滎工自增數倍。滎工堵合無期,蘭工更無把握。原奏決放舊河,掣溜東行,似言之太易。且瞬交春令,興工已難。二也。漢決酸棗,再決瓠子,為發卒治河之始。元、明發丁夫供役,亦以十數萬計。現在直、東、江、豫捻氛甫靖,而土匪游勇在在須防。所留勇營,斷難盡赴河干,亦斷不敷分挑之用。若再添募數十萬丁夫,聚集沿黃數千里間,駕馭失宜,滋生事端,尤為可慮。三也。應俟國庫充盈,再議大舉。因時制宜,惟有趕堵滎工,為保全豫、皖、淮揚下游之計。」上然之。八年正月,滎澤塞。

十年八月,決鄆城侯家林,東注南旺湖,又由汶上、嘉祥、濟寧之趙王、牛朗等河,直趨東南,入南陽湖。時廷魁內召,命新河督喬松年會同東撫丁寶楨勘辦。寶楨方以病在告,乃偕護撫文彬至工相度。文彬言:「河臣遠在豫省,若往返咨商,恐誤要工。一面飛咨河臣遴派掌壩,並管理正雜料廠員弁,及諳習工程之弁兵工匠,帶同器具,於年內來東,一面由臣籌購應需料物,以期應手。」上責松年剋期興工,松年言已飭原估委員並熟習工程人員赴東聽遣,並飭購備竹纜,及覓雇綑鑲船只備提用。惟已交立春,春水瞬生,辦工殊無把握。並移書文彬主持其事。文彬不能決。寶楨力疾視事,上言;「河臣職司河道,疆臣身任地方,均責無旁貸。乃松年一概諉之地方,不知用意所在。現在已過立春,若再候其的信以定行止,恐誤要工。且此口不堵,必漫淹曹、兗、濟十餘州縣。若再向南奔注,則清、淮、里下河更形吃重。松年既立意諉卸,臣若避越俎之嫌,展轉遷延,實有萬趕不及之勢。惟有力疾銷假,親赴工次,擇日開工,俟松年所遣員弁到工,即責成該工員等一手經理,剋期完工,保全大局。應請破格保獎,以昭激勸。倘敢陽奉陰違,有心貽誤,一經驗實,應請便宜行事,即將該員弁正法工次,以為罔上殃民者戒。」上嘉其勇於任事,並諭松年當和衷共濟,不遽加責也。

十一年二月,侯家林塞,予寶楨優敘。先是同知蔣作錦條上河、運事宜,朝廷頗韙其議,下河、漕、撫臣議奏。未幾,侯家林決,松年、寶楨意見齟。及寶楨塞侯家林,松年上言:「作錦所陳,卓然有見,可以採取。並稱東境黃水日愈汎濫,運道日愈淤塞,宜築堤束黃,先堵霍家橋諸口,並修南北岸長堤,俾黃趨張秋以濟運。挑濬張秋迤南北淤塞,修建閘壩,以利漕行。」上以松年意在因勢利導,不為無見,令寶楨、文彬詳議,毋固執己見。旋覆稱;「目前治黃之法,不外堵銅瓦廂以復淮、徐故道,與東省築堤即由利津入海兩策。顧謂二者之中,以築堤束黃為優,而上下游均歸緩辦,臣實未見其可。自銅瓦廂至牡蠣嘴,計千三百餘里,創建南北兩堤,相距牽計,約須十里。除現在淹沒不計外,尚須棄地數千萬頃,其中居民不知幾億萬,作何安插?是有損於財賦者一也。東省沿河州縣,自二三里至七八里者不下十餘。若齊河、齊東、蒲臺、利津,皆近在臨水,築堤必須遷避,是有難於建置者二也。大清河近接泰山麓,山陰水悉北注,除小清、溜瀰諸河均可自行入海,餘悉以大清河為尾閭。置堤束黃以後,水勢抬高,向所洩水之處,留閘則虞倒灌,堵遏則水無所歸,是有妨於水利者三也。東綱鹽場,坐落利津、霑化、壽光、樂安等縣,濱臨大清河兩岸。自黃由大清入海,鹽船重載,溯行於湍流,甚形阻滯,而灘地間被漫溢,產鹽日絀,海灘被黃淤遠,納潮甚難,東綱必至隳廢,私梟亦因而蜂起。是有礙於鹺綱者四也。臣寶楨身任地方,於通省大局所關,固宜直陳無隱。然使於治運漕果有把握,則京倉為根本至計,猶當權利害之輕重,而量為變通。臣等熟思審計,實未見其可恃,而深覺其可慮。似仍以堵合銅瓦廂使復淮、徐故道為正辦。」並陳四便。御史游百川亦言河、運並治,宜詳籌妥辦。疏入,廷議不能決。

下直督李鴻章。鴻章因遣員周歷齊、豫、徐、海,訪察測量,期得要領。十二年六月,上言:「治河之策,原不外恭親王等『審地勢,識水性,酌工程,權利害』四語,而尤以水勢順逆為要。現在銅瓦廂決口寬約十里,跌塘過深,水涸時猶逾一二丈。舊河身高,決口以下,水面二三丈不等。如欲挽河復故,必挑深引河三丈餘,方能吸溜東趨。查乾隆間蘭陽青龍岡之役,費帑至二千餘萬。阿桂言引河深至丈六尺,人力無可再施,今豈能挑深至三丈餘乎?十里口門進占合龍,亦屬創見。國初以來,黃河決口寬不過三四百丈,且屢堵屢潰,常閱數年而不成。今豈能合龍而保固乎?且由蘭陽下抵淮、徐之舊河,身高於平地三四丈。年來避水之民,移住其中,村落漸多,禾苗無際。若挽地中三丈之水,跨行於地上三丈之河,其停淤待潰、危險莫保情形,有目者無不知之。歲久堤乾,即加修治,必有受病不易見之處。萬一上游放溜,下游旋決,收拾更難。議者或以河北行則穿運,為運道計,終不能不強之使南以會清口。臣查嘉慶以後清口淤墊,夏令黃高於清,已不能啟壩送運。道光以後,禦黃壩竟至終歲不啟,遂改用灌塘之法,自黃浦洩黃入湖。湖身頓高,運河水少,灌塘又不便,遂改行海運。今即能復故道,亦不能驟復河運,非河一南行,即可僥幸無事。此淮、徐故道勢難挽復,且於漕運無益之實在情形也。至河臣所請就東境束黃濟運一節,查清口淤墊,即借黃濟運之病。今張秋運河寬僅數丈,兩岸廢土如山,若引重濁之黃,以閘壩節宣用之,水勢抬高,其淤倍速。人力幾何,安能挑此日進之沙?且所挑之沙,仍堆積於積年廢土之上,兩淋風蕩,河底日高,閘亦壅塞,久之黃必難引。明弘治中,荊龍口,銅瓦廂屢次大決,皆因引黃濟張秋之運,遂致導隙濫觴。臨清地勢低於張秋數丈,而必以後無掣溜奪河之害,臣亦不敢信也。至霍家橋堵口築是,工尤不易。該處本非決口、乃大溜經行之地,兩頭無岸,一望浮沙,並無真土可取。勉強堆築,節節逼溜下注,恐浮沙易塌,實足攖河之怒,而所耗實多。一遭潰決,水仍別穿運道,而不專會張秋,豈非全功盡棄?至作錦擬導衛濟運,原因張秋以北無清水灌運,故為此議。查元村集迤南有黃河故道,地多積沙,施工不易。且以全淮之水不能敵黃,尚致倒灌停淤,豈一清淺之衛,遂能禦黃濟運耶?其意蓋襲取山東諸水濟運之法。不知泰山之陽,水皆西流,因勢利導,十六州縣百八十泉之水,源旺派多,自足濟運。衛水來源,甚弱最順,今必屈曲使之南行,勢多不便。此借黃濟運及築堤束水均無把握,與導衛濟運之實在情形也。惟河既不能挽復故道,則東境財賦有傷,水利有礙,城池難以移置,鹽場間被漫淹,如寶楨所陳,誠屬可慮。臣查大清河原寬不過十餘丈,今已刷寬半里餘,冬春水涸,尚深二三丈,岸高水面又二三丈,是不汛時河槽能容五六丈,奔騰迅疾,水行地中,此人力莫可挽回之事,亦祀禱以求而不可得之事。目下北岸自齊河至利津,南岸齊東、蒲臺,皆接築民墊,雖高僅丈許,詢之土人,遇盛漲出槽不過數尺,尚可抵御。岱陰、繡江諸河,亦經擇耍築堤,汛至則漲,汛過則消,受災不重。至齊河、濟陽、齊東、蒲臺、利津各城,近臨河岸十九,年來幸防守無患,以後相勢設施。若驟議遷徙,經費無籌,民情難喻,無此辦法。東省鹽場在海口者,雖受黃淤產鹽不旺,經撫臣南運膠濟之鹽時為接濟,引地無虞淡食,惟價值稍昂耳。河在東省固不能無害,但得設法維持,尚不至為大患。昔乾隆中,銅山決口不能成功,孫嘉淦曾有分河入大清之疏。其後蘭陽大工屢敗垂成,嵇璜又有改河大清之請。此外裘曰修、錢大昕、胡宗緒、孫星衍、魏源諸臣議者更多。其時河未北流,尚欲挽之使北。今河自北流,乃欲挽使南流,豈非拂逆水性?大抵南河堵築一次,通牽約七八百萬,歲修約七百餘萬,實為無底之壑。今河北徙,近二十年未有大變,亦未多費巨款,比之往代,已屬幸事。且環拱神京,尤得形勝。自銅瓦廂東決,粵、捻諸逆竄擾曹、濟,幾無虛日,未能過河一步,而北岸防守有所憑依,更為畿輔百世之利。此兩相比較,河在東雖不亟治而後患稍輕,河回南即能大治而後患甚重之實在情形也。近世治河兼言治運,遂致兩難,卒無長策。臣愚以為天庾正賦,以蘇、浙為大宗,國家治安之道,尤以海防為重。今沿海洋舶駢集,為千古創局,已不能閉關自治。正不妨借海運轉輸之便,逐漸推廣,以擴商路而實軍儲。蘇、浙漕糧,現既統由海運,臣前招致華商購造輪船搭運,漸有成效,由海船解津,較為便速。至海道雖不暢通,河務未可全廢,此時治河之法,不外古人『因水所在,增立堤防』一語。查北岸張秋以上,有古大金堤可恃以為固,張秋以下,岸高水深,應由東撫隨時飭將民墊保護加培。至侯家林上下民墊應仿照官堤辦法,一律加高培厚,更為久遠之計。又銅瓦廂決口,水勢日向東坍刷,久必汎濫南趨。請飭松年察看形勢,量築堤墊,與曹州之堤相接,俾資周防而期順軌。至南河故道千餘里,居民占種豐收,並請查明升科,以免私墾爭奪之患。」疏入,議乃定。

是年夏秋,決開州焦丘、濮州蘭莊,又決東明之嶽新莊、石莊戶民墊,分溜趨金鄉、嘉祥、宿遷、水術陽入六塘河。寶楨勘由鄆城張家支門築堤堵塞。旋乞假展墓。十三年春,溜益南趨,潰漫不可收拾,江督累章告災。九月,寶楨回任,改由菏澤賈莊建壩。十二月興工。

光緒元年三月,東明決塞,並築李連莊以下南堤二百五十里。時河督曾國荃請設南岸七。部議俟直、東、豫籌有防汛的款再定。二年春,署東撫李元華言:「黃河南堤,自賈莊至東平二百餘里均完固,惟上游毗連直、豫,自東明謝寨至考城七十餘里,並無堤岸,此工刻不可緩。昔年侯家林塞,後怵於費多,未暇顧問,遂至賈莊決口。此次賈莊以下堤雖完固,上游若不修築,設有漫決,豈惟前功盡棄,河南、安徽、江蘇仍然受害,山東首當其沖無論已。臣擬調營勇,兼雇民夫,築此七十餘里長堤。深恐呼應不靈,已商直督、豫撫協力襄辦。至濮、範之民,自黃河改道,昏墊十有餘年。賈莊決後,稍有生機,及賈莊塞,受災如故。查南堤距北面金堤六七十里,以屏蔽京師則可,於濮、範村莊田畝則不能保衛。該處紳民原修北堤,惟力有未支,請酌加津貼,既成以後,派弁勇一律修防,濮、範、陽穀、壽張、東阿五縣地畝可涸出千餘頃。又查濮、範以上,有黃水二道。擬於壽張、東阿境內新河尾閭,抽挑引河二,冀歸並一渠。於南堤之北、黃河之南,再立小堤以束水,又可涸出地畝千餘頃。至北堤上游內有八里系開州轄,若不一律修築,不惟北堤徒勞無功,即畿輔亦難保不受其患。已商直督遣員協助,妥速蕆功。惟所壓直、豫地畝,該處居民無甚大益,而山東百姓受益無窮,自應由山東折償地價。上游收束既窄,下游水溜勢急,不可不防。自東平至利津海口九百餘里,已飭沿河州縣就民堤加培,酌給津貼,以工代賑。各項通計需費二千餘萬。此黃河大段擬辦情形也。」事下所司。

五年,決歷城溞溝。明年,復決。八年,決歷城桃園,十一月塞。九年,東撫陳士傑創建張秋以下兩岸大堤。時山東數遭河患,朝士屢以為言。上遣侍郎游百川馳往會勘。百川言:「自來論河者,分持南行北行二說。臣詳察形勢,將來遇伏秋盛漲,復折而東,自尋故道,亦未可知。若挽以人力,則勢有萬難。一則北堤決後,已沖刷凈盡,築堤進占,工已甚鉅。且全河正流北行,中流堵御以圖合龍,必震駭非常,辦理殊無把握。一則故道旁沙嶺勢難挑動,且徐、海一帶河身涸出淤地千餘里,民盡墾種,一旦驅而之他,民豈甘心失業?此南行之說應無庸議也。至大清河本汶、濟交會,自黃流灌入,初猶水行地中,今則河身淤墊,既患水不能洩,自濟河上下,北則濟陽、惠民、濱州、利津,南則青城、章丘、歷城至鄒、長、高、博,漫決十一處。竊惟河入濟瀆已二十八年,其始誤於山東無辦河成案,誘民自為堤墊,縱屢開決,未肯形諸奏牘,貽患至斯。今則泛濫數百里,漂沒數百村,遍歷災區,傷心慘目。謹擬辦法三。一,疏通河道。黃初入濟,尚能容納,淤墊日高,至海口尤日形淤塞。沙淤水底,人力難施,計惟多用船隻,各帶鐵篦混江龍,上下拖刷,使不能停蓄,日漸刮深。疏導之方,似無逾此。一,分減黃流。濟一受黃,其勢岌岌不可終日。查大清河北,徒駭最近,馬頰較遠,鬲津尤在其北。大清河與徒駭最近處在惠民白龍灣,相距十許里。若由此開築減壩,分入徒駭河,其勢較便。再設法疏通其間之沙河、寬河、屯民等河,引入馬頰、鬲津,分疏入海,當不復虞其滿溢。一,亟築縷堤。民間自築縷堤,近臨河干,多不合法,且大率單薄,又斷續相間,屢經塌陷,一築再築,民力困竭。今擬自長清抵利津,南北岸先築縷堤,其頂沖處再築重堤,約長六百餘里,仍借民力,加以津貼,可計日成功,為民捍患,民自樂從。至謂治水不與水爭地,其法無過普築遙堤。然濟、武兩郡,地狹民稠,多占田畝,小民失業,正非所原。且其間村鎮廬墓不可數計,兼之齊河、濟陽、齊東、蒲臺、利津皆城臨河干,使之實逼處此,民情未免震駭。價買民田,需款不下四五百萬,工艱費鉅,可作緩圖。臣所以請築縷堤以濟急,而不敢輕持遙堤之議者此也。」士傑持異議。會海豐人御史吳峋言徒駭、馬頰二引河不可輕開,命直督李鴻章偕士傑會勘,亦如峋言。乃定議築兩岸長堤。

是年決利津十四戶,十年三月塞。閏五月,決歷城河套圈、霍家溜,齊河李家岸、陳家林、蕭家莊,利津張家莊、十四戶,先後塞之。是年兩岸大堤成,各距河流數百丈,即縷堤也,而東民仍守臨河墊,有司亦諭令先守民墊,如墊決再守大堤,而堤內村廬未議遷徙,大漲出槽,田廬悉淹,居民遂決堤洩水,官亦不能禁,嗣是只守墊不守大堤矣。十一年,蕭家莊、溞溝再決,又決齊河趙莊。十二月,溞溝、趙莊塞。明年二月,蕭家莊塞。六月,再決河套圈,又決濟陽王家圈、惠民姚家口、章丘河王莊、壽張徐家沙窩,惟王家圈工緩辦,餘皆年內塞。東境河雖屢決,然皆分溜少奪溜,每堵築一次,費數萬或數十萬,多亦不過一二百萬,較南河時所省正多,被淹地畝亦較少,地平水緩故也。

十三年六月,決開州大辛莊,水灌東境,濮、範、壽張、陽穀、東阿、平陰、禹城均以災告。八月,決鄭州,奪溜由賈魯河入淮,直注洪澤湖。正河斷流,王家圈旱口乃塞。鄭州既決,議者多言不必塞,宜乘此復故道。戶部尚書翁同龢、工部尚書潘祖廕同上言:「河自大禹以後,行北地者三千六百餘年,南行不過五百餘年,是河由雲梯關入海,本不得謂故道。即指為故道,而現在溜注洪澤湖,形北高南下,不能導之使出清口,去故道尚百餘里,其勢斷不能復。或謂山東數被水害,遂以河南行為幸。不知河性利北行。自金章宗後,河雖分流。有明一代,北決者十四,南決者五;我朝順、康以來,北決者十九,南決者十一。況淮無經行之渠,黃入淮安有歸宿之地?下流不得宣洩,上游必將復決,決則仍入東境,山東之患仍未能弭。至黃水南注,有二大患、五可慮。黃注洪澤,而淮口淤墊,久不通水,僅張福口引河,闊不過數丈,大溜東注,以運河為尾閭,僅恃東堤為護,已岌岌可危。今忽加一黃河,必不能保。大患一。洪澤淤墊,高家堰久不可恃,黃河勢悍,入湖後難保不立時塌卸。不東沖里下河,即南灌揚州,江、淮、河、漢並而為一,東南大局,何堪設想!大患二。里下河為產米之區,萬一被淹,漕米何從措辦?可慮一。即令漕米如故,或因黃挾沙墊運,不能浮送。或因積水漫溢,纖道無存,漕艘停滯。且山東本借黃濟運,黃既遠去,沂、汶微弱,水從何出?河運必廢。可慮二。兩淮鹽場,胥在範公堤東。範堤不保,鹽場淹沒,國課何從徵納?可慮三。潁、壽、徐、海,好勇鬥狠,小民蕩析,難保不生事端。可慮四。黃汛合淮,勢不能局於湖瀦,必別尋入海之道,橫流猝至,江鄉居民莫保旦夕。可慮五。至入湖之水,亦須早籌宣洩。里下河地勢,西北俯、東南仰,宜順其就下之勢,由興化以北,歷朦朧、傅家塢入舊河,避雲梯關淤沙,北濬大通口,入潮河以達淮河,海口則取徑直,形勢便,經費亦不過鉅。」

上命江督曾國荃、漕督盧士傑籌議。適國荃、士傑亦言:「捍河匯淮東下,其危險百倍尋常。查治水不外宣防二策,而宣之用尤多。洪湖出路二,皆由運入江。今大患特至,不能不於湖之上游多籌出路,分支宣洩,博採群議。桃源有成子河,南接洪湖,北至舊河,又北為中運河。若加挑成子河,使通舊河,直達中運河,兩岸築堤,即可引漫水由楊莊舊河至雲梯關入海,此洪湖上面新闢一去路也。清河有碎石河,西接張福口,引河東達舊河,大加挑挖,亦可引漫水由楊莊舊河至雲梯關入海,此洪湖下面新闢一去路也。詢之耆舊,僉謂舍此別無良法。是以臣等議定即勘估興工,不敢拘泥成規,往返遷延,致誤事機。」上韙之,並遣前山西布政使紹諴、降調浙江按察使陳寶箴、前山東按察使潘駿文迅赴鄭工,隨同河督成孚、豫撫倪文蔚襄理河務。時工賑需款鉅且急,戶部條上籌款六事:一,裁防營長夫;一,停購軍械船隻機器;一,停止京員兵丁米折銀;一,酌調附近防軍協同工作;一,令鹽商捐輸給獎;一,預徵當商匯號稅銀。議上,詔裁長夫、捐鹽商及預徵稅銀,餘不允。九月,命禮部尚書李鴻藻偕刑部侍郎薛允升馳勘,鴻藻留督工。時黃流漫溢,河南州縣如中牟、尉氏、扶溝、鄢陵、通許、太康、西華、淮寧、祥符、沈丘、鹿邑多被淹浸,水深四五尺至一二丈,特頒內帑十萬,並截留京餉三十萬賑撫。而河工需款急,允御史周天霖、李世琨請,特開鄭工新捐例,奪成孚職,以李鶴年署河督。

十月,東撫張曜言:山東河淤潮高,黃流實難容納,請乘勢規復南河故道。下鴻藻、鶴年議。鴻藻等遂請飭迅籌合辦。上以「黃河籌復故道,迭經臣工條奏,但費鉅工繁,斷難於決口未堵之先,同時並舉。此奏於故道宜復,止空論其理,語簡意疏。一切利害之輕重,地勢之高下,工用之浩大,時日之迫促,並未全局通籌,縷晰奏覆。如此大事,朝廷安能據此寥寥數語,定計決疑?故道一議,可暫從緩。至所稱一切工作,先自下游開辦,南河舊道現在情形如何,工程能否速辦,經費能否立籌,有無滯礙,著國荃、士傑、崧駿迅速估奏。」國荃言:「黃流東注,淮南北地處下游,宜籌分洩之策。請就楊莊以下舊河二百餘里挑濬,以分沂、泗之水,騰出中運河,預備洪河盛漲,挾黃北行,堪以容納,是上游籌有去路。而淮由三河壩直趨而東,則運堤極為吃重,勢不能不開壩宣洩,里下河如臨釜底,而枝河頗多,若預先疏導,使水能順軌,則田廬民命亦可保全。同龢、祖廕所言,洵得水性就下之勢,業經遣員履勘,並請調熟悉河工之江蘇臬司張富年督理。」制可。先是侍郎徐郙有通籌黃河全局之疏。文蔚言:「郙所陳口門北岸上游酌開引河,上南以下河內挑川字河,及築排水壩,三者皆河南必辦之事,即前人著效之法。臣前請於河身闊處切灘疏淤,即郙酌開引河及川字河之意。河員以近日河勢略變,須更籌辦法,且有引河不可挑之說。而此項土夫,皆系應賑之人,無論何工,皆系應辦之事,將來或幫挑運河,或幫築河身,應就商河臣隨時調度。」報聞。

十二月,國荃、士傑言:「同龢等所陳二患五慮,不啻身歷其境,將臣等所欲言者,代達宸聰。當經派員分投履勘。自傅家塢入舊黃河,過雲梯關至大通口,測量地勢,北高丈五七尺,揆諸就下之性,殊未相宜。不敢不恪遵聖訓,於興化境內別籌疏淤。查下河入海河道,以新陽、射陽兩河為最,鬥龍港次之。祗以支河阻塞,未能通暢。查興化屬之大圍閘、丁溪場屬之古河口小海,均極淤淺。疏濬以後,如果高郵開壩,可冀水皆順軌,由新陽等河宣暢歸海。其閘門窄狹過水不暢者,另於左右開挖越河,俾得滔滔直注。此外幹支各河,再接續擇要興挑,以期逐節通暢,核與同龢、祖廕之奏事異功同。」

十四年正月,國荃等又言:「徐郙通籌河局疏,稱淮揚實無處位置黃河,宜先籌宣洩之方,再求堵合之法,洵屬確中肯綮。至請挑天然及張福口引河,本系由淮入黃咽喉,昔人建導淮之議,皆從引河入手。祗以張福淤墊太高,挑不得法,且恐沂、泗倒灌。又順清河為清江三閘來源,曩時堵築以資自衛。自河北徙,此壩久廢。今既引淮入黃,仍須堵築順清壩,庶三閘可保無虞。經臣等派員審度河底,雖北高南低,加工挑深,尚可配平。順清河雖水深溜急,多備料土,亦可設法堵築。又經臣士傑履勘,陳家窯可開引河,上接張福口,下達吳城七堡,與碎石河功用相同。已於十月分段興挑,自張福口、內窯河起,至順清河止,開深丈四尺至二丈,冀上游多洩一分之水,下河即少受一分之災。其工段亦間調哨勇幫同挑濬,以補民夫之不足。以上辦法,與該侍郎所陳江南數條,不謀而合。」

先是上以將來河仍北趨,有「趁湍流驟減,挑濬東明長堤,開州河身,加培堤墊」之諭。至是,鴻章言:「直境黃河長八九十里,一律挑濬,工鉅費煩。即酌挑北面數處,亦需二三十萬。兩岸河灘高於中洪一二丈,河身尚可容水。惟東明南堤歷年沖刷,亟應擇要修築,已調派大名練軍春融赴工,並募民夫同時力作。開州全堤殘缺已甚,亦經派員估修。至長垣南岸小堤,離河較遠,尚可緩辦。北岸民墊,飭勸民間修培,不得逼束河流,致礙大局。」

六月,小楊莊塞。是月,鴻藻言鄭工兩壩,共進占六百一十四丈,尚餘口門三十餘丈,因伏秋暴漲,人力難施,請緩俟秋汛稍平,接續舉辦。上嚴旨切責,褫鶴年職,與成孚並戍軍臺。鴻藻、文蔚均降三級留任。以廣東巡撫吳大澂署河道總督。大澂言:「醫者治病,必考其致病之由,病者服藥,必求其對癥之方。臣日在河干,與鄉村父老諮詢舊事,證以前人紀載,知豫省河患非不能治,病在不治。築堤無善策,鑲埽非久計,要在建壩以挑溜,逼溜以攻沙。溜入中洪,河不著堤,則堤身自固,河患自輕。員中年久者,僉言咸豐初滎澤尚有磚石壩二十餘道,堤外皆灘,河溜離堤甚遠,就壩築埽以防險,而堤根之埽工甚少。自舊壩失修,不數年廢棄殆盡,河勢愈逼愈近,埽數愈添愈多,員救過不遑,顧此失彼,每遇險工,輒成大患。河員以鑲埽為能事,至大溜圈注不移,旋鑲旋蟄,幾至束手。臣親督道趕拋石垛,三四丈深之大溜,投石不過一二尺,溜即外移,始知水深溜激,惟拋石足以救急,其效十倍埽工,以石護溜,溜緩而埽穩。歷朝河臣如潘季馴、靳輔、慄毓美,皆主建壩挑溜,良不誣也。現以數十年久廢之要工,數十道應修之大壩,非一旦所能補築竣工。惟有於鄭工款內核實撙節,省得一萬,即多購一萬之石垛,省得十萬,即多做十萬之壩工,雖系善後事宜,趁此乾河修築,人力易施,否則鄭工合龍後,明年春夏出險,必至措手不及。雖不敢謂一治而病即愈,特愈於不治而病日增。果能對癥發藥,一年而小效,三五年後必有大效。」上嘉勉之。

大澂又言:「向來修築壩垛,皆用條磚碎石,每遇大汛急溜,壩根淘刷日深,不但磚易沖散,重大石塊亦即隨流坍塌。聞西洋有塞門德土,拌沙黏合,不患水侵。趁此引河未放,各處須築挑壩,正在河身乾涸之時,擬於磚面石縫,試用塞門德土塗灌,斂散為整,可使壩基做成一片,足以抵當河溜,用石少而工必堅,似亦一勞永逸之法。」報聞。十二月,鄭工塞,用帑千二百萬,實授大澂河督,詔於工次立河神廟,並建黃大王祠,賜扁額,與黨將軍俱加封號。是年七月,決長垣範莊。未幾塞。十五年六月,決章丘大寨莊、金王莊,分溜由小清河入海。又決長清張村、齊河西紙坊,山東濱河州縣多被淹浸。是冬塞。

十六年二月,東撫張曜言:「前南總河轄河工九百餘里,東總河轄五百餘里。自決銅瓦廂,河入山東,遂裁南總河,而東河所轄河工僅二百餘里。今東河縣長九百里,日淤日高,全恃堤防為保衛。本年臣駐工二百餘日,督率修防,日不暇給。請將自菏澤至運河口河道二百餘里,歸河督轄,與原轄之河道里數相等。」部議以此段工程,向由巡撫督率地方官兼管,河督恐呼應不靈。曜又言:「向來沿河州縣,本歸河臣兼轄,員缺仍會河臣題補,遇有功過,河臣亦應舉劾,尚無呼應不靈之患。請並下河督籌議。」先是大澂遣員測繪直、東、豫全河,至是圖成上之。五月,決齊河高家套,旋塞。

十八年六月,決惠民白茅墳,奪溜北行,直趨徒駭入海。又決利津張家屋、濟陽桑家渡及南關、灰壩,俱匯白茅墳漫水歸徒駭河。七月,決章丘胡家岸,夾河以內,一片汪洋,遷出歷城、章丘、濟陽、齊東、青城、濱州,蒲臺、利津八縣災民三萬三千二百餘戶。初,河督許振禕請於歲額六十萬內,提十二萬歸河防局,籌添料石,先事預防,由河督主之,至是部令分案題銷。振禕言:「河工大險,恃法不如用人。如以恃法論,則從來報銷例案,工部知之,河工亦知之,故自每年添款及鄭工報銷之千數百萬,未聞其不合例也。如以用人論,則臣近此改章從事,比年大險橫生,亦均次第搶補,幸奏安瀾,至添料添石,固有不盡合例者矣。原臣立河防局,意有二端。一則恐員遇險推諉,藉口無錢無料,故提此鉅款先事預防之資。一則恐員不實不盡,故添委官紳臨時匡救之用,而限十二萬纖悉到工,不準絲毫入局,並不準開支薪水。河南官紳吏民罔不知之。即如今歲之得保鉅險,就買石一款,已用過十一萬數千兩,餘則補鄭工金門沈裂之堤,此不能分案題銷者也。又多方買石,隨處搶堵,險未平必加拋,險已過即停止,此不能繪圖貼說者也。」上如所請行。是年白茅墳各口塞。

二十一年六月,決壽張高家大廟、齊東趙家大堤。未幾,決濟陽高家紙坊、利津呂家窪、趙家園、十六戶。是冬次第塞。明年六月,決利津西韓家、陳家。御史宋伯魯條上東河積弊:一,冒領矇銷,宜嚴定處分;一,收發各料,宜設法稽查;一,申明賠修舊例,以防隨意改名;一,武弁宜認真巡察。詔東撫嚴除積弊,並令有河務各督撫查察,遇有劣員,嚴參懲辦。二十三年正月,決歷城小沙灘、章丘胡家岸,隨塞。十一月凌汛,決利津姜家莊、扈家灘,水由霑化降河入海。二十四年六月,決山東黑虎廟,穿運東洩,仍入正河。又決歷城楊史道口、壽張楊家井、濟陽桑家渡、東阿王家廟,分注徒駭、小清二河入海。遣鴻章偕河督任道鎔、東撫張汝梅會勘。未幾,省東河總督,尋復置。

二十五年二月,鴻章等言:「山東黃河自銅瓦廂改道大清河以來,時當軍興,未遑修治。同治季年,漸有潰溢,始築上游南堤。光緒八年後潰溢屢見,遂普築兩岸大堤。乃民間先就河涯築有小墊,緊逼黃流。大堤成後,復勸民守墊,且有改為官守者。於是堤久失修,每遇汎漲墊決,水遂建瓴而下,堤亦隨決,此歷來失事病根也。上游曹、兗屬南北堤湊長四百餘里,兩堤相距二十里至四十里,民墊偶決,水由堤內歸入正河,大決則堤亦不保。計南北墊工二十四,同治以來,決僅四五見,此上游情形也。中游濟、泰屬兩岸堤墊各半,湊長五百里,南岸上段傍山無堤,下段守墊,北岸上守堤,下守墊,參差不一,無非為堤內村莊難遷,權為保守計。下游武、定屬南岸全守堤,北岸全守墊,湊長五百餘里,地勢愈平,水勢愈大,險工七十餘處,二十五年來,已決二十三次,此中下游情形也。東省修防事本草創,間有興作,皆因費絀,未按治河成法。前撫臣李秉衡歷陳山東受河之害,治河之難,謂近幾無歲不決,無歲不數決。朝廷屢糜鉅金,閭閻終無安歲。若不按成規大加修治,何以仰答愛養元元之意?臣等詳考古來治河之法,惟漢賈讓徙當水沖之民,讓地於水,實為上策。前撫臣陳士傑建築中下游兩岸大堤,湊長千里,兩堤相距五六里至八九里,就此加培修守,似不失為中策。惟先有棄堤守墊處,如南岸濼口上下,守墊者百一十里,上段近省六十里,商賈輻輳,近險工稍平,暫緩推展;下段要險極多,十餘年來,已決九次,擬遷出墊外二十餘村,棄墊守堤,離水稍遠,防守易固。此南岸酌擬遷民廢墊辦法也。至北岸堤工,自長清至利津四百六十里,墊外堤內數百村莊。長墊逼近湍流,河面太狹,無處不灣,無灣不險。河脣淤高,墊外地如釜底,各村斷不能久安室家。且墊破堤必破,欲保墊外數百村,並堤外數千村同一被災,尤覺非計。但小民安土重遷,屢被沈災,不肯遠去,非可旦夕議定。今擬北岸自長清官莊至齊河六十餘里,河面尚寬,利津至鹽窩七十餘里,地皆斥鹵,不便徙民,均以墊作堤,墊外之民,無庸遷徙。其齊河至利津尚有三百二十里,民墊緊逼河干,竟有不及一里者,勢不得不廢墊守堤。但北堤殘缺多半,無可退守,且需款過鉅,遷民更難,應暫守舊墊,此北岸分別守墊作堤,及將來再議廢墊守堤辦法也。至南北大堤,為河工第一重大關系。既處處卑薄,擬並改墊之堤,及暫定之民墊,照河工舊式,一律修培,總期足禦汛漲。至下口入海尾閭,尤關全河大局。查鐵門關故道尚有八十餘里,愈下愈寬深,直通海口,形勢較絲網口、韓家垣為順,工費亦較省。然建攔河大壩、挑引河、築兩岸大堤,需費頗鉅,下口不治,全河皆病,不得不核實勘估,此又加培兩岸堤工、改正下口辦法也。約估工費需九百三十萬有奇,分五六年可告竣。」朝議如所請,先發帑百萬,交東撫毓賢督修。

毓賢言:「黃河治法,誠如部臣所云,展寬河面、盤築堤身、疏通尾閭三事為扼要。查尾閭之害,以鐵板河為最。全河挾沙帶泥,到此無所歸束,散漫無力,經以風潮,膠結如鐵,流不暢則出路塞而橫流多,故無十年不病之河。擬建長堤直至淤灘,防護風潮,縱不能徑達入海,而多進一步即多一步之益。至堤墊卑薄,擬修培時,土方必足,夯硪必堅,尤加意保守。其坐灣處,一灣一險,如上游賈莊、孫家樓,中流坰家岸、霍家溜、桑家渡,下游白龍潭、北鎮家集鹽窩,均著名巨險,餘險尤多,固非裁灣取直不可,然亦須相度形勢,必引河上口能迎溜勢、下口直入河心方得。蒲臺迤西魏家口至迤東宋莊,約長四十里,河水分流,納正河之溜三分之。若就勢修堤建壩挑溜,使歸北河,正河如淤,蒲臺城垣永免水患。此裁灣取直之最有益者,擬即勘估興辦。」報聞。

二十六年,拳匪亂作,未續請款。嗣時局日艱,無暇議及河防矣。是年凌汛,決濱州張肖堂家。明年三月塞。六月,決章丘陳家窯、惠民楊家大堤,隨塞。黃河之初北徙也,忠親王僧格林沁有裁總河之請。嗣東河改歸巡撫兼轄,河督喬松年復以為請。至是,河督錫良言:「直、東河工久歸督撫管轄,豫撫本有兼理河道之責。請仿山東成案,改歸兼理,而省東河總督。」制可。二十八年夏,決利津馮家莊。秋,決惠民劉旺莊。逾年二月,劉旺莊塞。六月,決利津寧海莊,十二月塞。三十年正月,凌汛,決利津王莊、扈家灘、姜莊、馬莊,隨塞。六月,河溢甘肅皋蘭,淹沒沿灘村莊二十餘。又決山東利津薄莊,淹村莊、鹽窩各二十餘。

先是山東屢遭河患,當事者皆就水立堤,隨灣就曲,水不暢行。張秋以下,堤卑河窄,又無石工幫護。利津以下,尾閭改向南,形勢益不順。巡撫周馥請帑三百萬,略事修培,部臣靳不予。不得已,自籌二十萬添購石料,又給貲遷利津下民之當水沖者,而民徙未盡。又於堤南增建大堤,以備舊堤壞、民有新居可歸。至薄莊決,水東北由徒駭河入海。馥言:「舊河淤成平陸,若依舊堵合,估須九十萬有奇,鉅款難籌。且堵合之後,防守毫無把握,漫口以下,水深丈餘至二三丈,奔騰浩瀚,就下行疾,入徒駭後,勢益寬深,較鐵門關、韓家垣、絲網口尤暢達。與其逆水之性,耗無益之財,救民而終莫能救,不如遷民避水,不與水爭地,而使水與民各得其所。依此而行,其益有三:尾閭通順,流暢消速,益一;舟楫便利,商貨流通,益二;河流順直,險輕費省,益三。所省堵築費猶不計也。然補救之策,費財亦有三:一,遷民之費;二,築墊之費;三,移設鹽垣之費。約需五十萬金,較堵築費省四之三,而受益過之。」制可,遂不堵。嗣是東河安瀾,數年未嘗一決。

宣統元年,決開州孟民莊。明年塞。三年,東撫孫寶琦言:「自黃入東省,河道深通,初無修防。積久淤溢,始築民墊,緊逼黃流。嗣經普築大堤,而復令民守墊。墊有漫決。官無處分,直、東兩省,定例皆然。元年開州決,水循東省上游墊外堤內下注,至中游始歸正河,濮﹑範、壽張受災甚重。臣會商直督,遣員協款堵築,上年始告成功。如能通籌,分別勘治,改歸官守上游橫決,為患何堪設想!臣昔隨李鴻章來東勘河,時比工程司建議築堤伸入海深處為最要辦法,卒以費鉅不果。如由主治者統籌經費,分年築堤,藉束水為攻沙之計,再酌購外洋挖泥輪機,往來疏濬,尾閭可望深通,全局皆受其益。河工為專門之學,非久於閱歷,不能得其奧竅。亟宜仿照豫省定章,改定文武額缺為終身官,三省互相遷調。臣上年設立河工研究所,招集學員講求河務,原為養成治河人材;如設汛,此項人員畢業,即可分別試用,於工程大有裨益。以上四端,必應興辦。臣愚以為宜設總河大員,歷勘會商,將三省常年經費百數十萬,統歸應用,俟議定大治辦法,隨時請撥,俾免掣肘而竟事功。」疏入,詔會商直督、豫撫通籌。未及議覆,而武昌變作,遂置不行。


\end{pinyinscope}