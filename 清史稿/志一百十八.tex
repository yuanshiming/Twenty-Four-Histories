\article{志一百十八}

\begin{pinyinscope}
○刑法二

明律淵源唐代,以笞、杖、徒、流、死為五刑。自笞一十至五十,為笞刑五。自杖六十至一百,為杖刑五。徒自杖六十徒一年起,每等加杖十,刑期半年,至杖一百徒三年,為徒五等。流以二千里、二千五百里、三千里為三等,而皆加杖一百。死刑二:曰斬,曰絞。此正刑也。其律例內之雜犯、斬絞、遷徙、充軍、枷號、刺字、論贖、凌遲、梟首、戮尸等刑,或取諸前代,或明所自創,要皆非刑之正。

清太祖、太宗之治遼東,刑制尚簡,重則斬,輕則鞭撲而已。迨世祖入關,沿襲明制,初頒刑律,笞、杖以五折十,注入本刑各條。康熙朝現行則例改為四折除零。雍正三年之律,乃依例各於本律注明板數。徒、流加杖,亦至配所照數折責。蓋恐撲責過多,致傷生命,法外之仁也。文武官犯笞、杖,則分別公私,代以罰俸、降級、降調,至革職而止。

徒者,奴也,蓋奴辱之。明發鹽場鐵冶煎鹽炒鐵,清則發本省驛遞。其無驛縣,分撥各衙門充水火夫各項雜役,限滿釋放。

流犯,初制由各縣解交巡撫衙門,按照里數,酌發各處荒蕪及瀕海州縣。嗣以各省分撥失均,不免趨避揀擇。乾隆八年,刑部始纂輯三流道里表,將某省某府屬流犯,應流二千里者發何省何府屬安置,應流二千五百里者發何省何府屬安置,應流三千里者發何省何府屬安置,按計程途,限定地址,逐省逐府,分別開載。嗣於四十九年及嘉慶六年兩次修訂。然第於州縣之增並,道里之參差,略有修改,而大體不易。律稱:「犯流妻妾從之,父祖子孫欲隨者聽。」乾隆二十四年,將僉妻之例停止。其軍、流、遣犯情原隨帶家屬者,不得官為資送,律成虛設矣。

斬、絞,同是死刑。然自漢以來,有秋後決囚之制。唐律除犯惡逆以上及奴婢、部曲殺主者,從立春至秋分不得奏決死刑。明弘治十年奏定真犯死罪決不待時者,凌遲十二條,斬三十七條,絞十二條;真犯死罪秋後處決者,斬一百條,絞八十六條。順治初定律,乃於各條內分晰注明,凡律不注監候者,皆立決也;凡例不言立決者,皆監候也。自此京、外死罪多決於秋,朝審遂為一代之大典。雜犯斬、絞準徒五年與雜犯三流總徒四年,大都創自有明。清律於官吏受贓,枉法不枉法,滿貫俱改為實絞,餘多仍之。名實混淆,殊形轇轕。

遷徙原於唐之殺人移鄉,而定罪則異。律文沿用數條,然皆改為比流減半、徒二年,並不徙諸千里之外。惟條例於土蠻、瑤、僮、苗人仇殺劫擄及改土為流之土司有犯,將家口實行遷徙。然各有定地,亦不限千里也。

明之充軍,義主實邊,不盡與流刑相比附。清初裁撤邊衛,而仍沿充軍之名。後遂以附近、近邊、邊遠、極邊、煙瘴為五軍,且於滿流以上,為節級加等之用。附近二千里,近邊二千五百里,邊遠三千里,極邊、煙瘴俱四千里。在京兵部定地,在外巡撫定地。雍正三年之律,第於十五布政司應發省分約略編定。乾隆三十七年,兵部根據邦政紀略,輯為五軍道里表,凡發配者,視表所列。然名為充軍,至配並不入營差操,第於每月朔望檢點,實與流犯無異。而滿流加附近、近邊道里,反由遠而近,司讞者每苦其紛歧,而又有發遣名目。初第發尚陽堡、寧古塔,或烏喇地方安插,後並發齊齊哈爾、黑龍江、三姓、喀爾喀、科布多,或各省駐防為奴。乾隆年間,新疆開闢,例又有發往伊犁、烏魯木齊、巴里坤各回城分別為奴種地者。咸、同之際,新疆道梗,又復改發內地充軍。其制屢經變易,然軍遣止及其身。茍情節稍輕,尚得更赦放還。以視明之永遠軍戍,數世後猶句及本籍子孫者,大有間也。若文武職官犯徒以上,輕則軍臺效力,重則新疆當差。成案相沿,遂為定例。此又軍遣中之歧出者焉。

枷杻,本以羈獄囚。明代問刑條例,於本罪外或加以枷號,示戮辱也。清律犯罪免發遣條:「凡旗人犯罪,笞、杖各照數鞭責,軍、流、徒免發遣,分別枷號。徒一年者,枷號二十日,每等遞加五日。流二千里者,枷號五十日,每等亦遞加五日。充軍附近者,枷號七十日,近邊、沿海、邊外者八十日,極邊、煙瘴者九十日。」原立法之意,亦以旗人生則入檔,壯則充兵,鞏衛本根,未便離遠,有犯徒、流等罪,直以枷號代刑,強榦之義則然。然犯系寡廉鮮恥,則銷除旗檔,一律實發,不姑息也。若竊盜再犯加枷,初犯再犯計次加枷,犯奸加枷,賭博加枷,逃軍逃流加枷,暨一切敗檢逾閑、不顧行止者酌量加枷,則初無旗、民之別。康熙八年,部議囚禁人犯止用細鍊,不用長枷,而枷號遂專為行刑之用。其數初不過一月、二月、三月,後竟有論年或永遠枷號者。始制重者七十,輕者六十斤。乾隆五年,改定應枷人犯俱重二十五斤,然例尚有用百斤重枷者。嘉慶以降,重枷斷用三十五斤,而於四川、陜西、湖北、河南、山東、安徽、廣東等省匪徒,又有系帶鐵桿石礅之例,亦一時創刑也。

刺字,古肉刑之一,律第嚴於賊盜。乃其後條例滋多,刺緣坐,刺兇犯,刺逃軍、逃流,刺外遣、改遣、改發。有刺事由者,有刺地方者,並有分刺滿、漢文字者。初刺右臂,次刺左臂,次刺右面、左面。大抵律多刺臂,例多刺面。若竊盜責充警跡,二三年無過,或緝獲強盜二名以上、竊盜三名以上,例又準其起除刺字,復為良民。蓋惡惡雖嚴,而亦未嘗不予以自新之路焉。

贖刑有三:一曰納贖,無力照律決配,有力照例納贖。二曰收贖,老幼廢疾、天文生及婦人折杖,照律收贖。三曰贖罪,官員正妻及例難的決,並婦人有力者,照例贖罪。收贖名曰律贖,原本唐律收贖。贖罪名為例贖,則明代所創行。順治修律,五刑不列贖銀數目。雍正三年,始將明律贖圖內應贖銀數斟酌修改,定為納贖諸例圖。然自康熙現行例定有承問官濫準納贖交部議處之條,而前明納贖及贖罪諸舊例又節經刪改,故律贖俱照舊援用,而例贖則多成具文。

其捐贖一項,順治十八年,有官員犯流徒籍沒認工贖罪例;康熙二十九年,有死罪現監人犯輸米邊口贖罪例;三十年,有軍流人犯捐贖例;三十四年,有通倉運米捐贖例;三十九年,有永定河工捐贖例;六十年,有河工捐贖例。然皆事竣停止,其歷朝沿用者,惟雍正十二年戶部會同刑部奏準預籌運糧事例,不論旗、民,罪應斬、絞,非常赦所不原者,三品以上官照西安駝捐例捐運糧銀一萬二千兩,四品官照營田例捐運糧銀五千兩,五、六品官照營田例捐銀四千兩,七品以下、進士、舉人二千五百兩,貢、監生二千兩,平人一千二百兩,軍、流各減十分之四,徒以下各減十分之六,俱準免罪。西安駝捐,行自雍正元年,營田例則五年所定也。乾隆十七年,西安布政使張若震奏請另定捐贖笞、杖銀數。經部議,預籌運糧事例,杖、笞與徒罪不分輕重,一例捐贖,究未允協。除犯枷號、杖責者照徒罪捐贖外,酌擬分杖為一等,笞為一等。其數,杖視徒遞減,笞視杖遞減。二十三年,諭將斬、絞緩決各犯納贖之例永行停止。遇有恩赦減等時,其憚於遠行者,再準收贖。而贖鍰則仍視原擬罪名,不得照減等之罪。著為令。嗣後官員贖罪者,俱照運糧事例核奪。刑部別設贖罪處,專司其事。此又律贖、例贖而外,別自為制者矣。

凌遲,用之十惡中不道以上諸重罪,號為極刑。梟首,則強盜居多。戮尸,所以待惡逆及強盜應梟諸犯之監故者。凡此諸刑,類皆承用明律,略有通變,行之二百餘年。至過誤殺之賠人,竊盜之割腳筋,重闢減等之貫耳鼻,強盜、貪官及窩逃之籍家產,或沿自盛京定例,或順治朝偶行之峻令,不久革除,非所論也。

自光緒變法,二十八年,山西巡撫趙爾巽奏請各省通設罪犯習藝所。經刑部議準,徒犯毋庸發配,按照年限,於本地收所習藝。軍、流為常赦所不原者,照定例發配,到配一律收所習藝。流二千里限工作六年,二千五百里八年,三千里者十年。遣軍照滿流年限計算,限滿釋放,聽其自謀生計,並準在配所入籍為民。若為常赦所得原者,無論軍、流,俱無庸發配,即在本省收所習藝。工作年限,亦照前科算。自此五徒並不發配,即軍、流之發配者,數亦銳減矣。二十九年,刑部奏準刪除充軍名目,將附近、近邊,邊遠並入三流,極邊及煙瘴改為安置,仍與當差並行。自此五軍第留其二,而刑名亦改變矣。三十年,劉坤一、張之洞會奏變法第二摺內,有恤刑獄九條。其省刑責條內,經法律館議準,笞、杖等罪,仿照外國罰金之法,改為罰銀。凡律例內笞刑五,以五錢為一等,至笞五十罰銀二兩五錢,杖六十者改為罰五兩。每一等加二兩五錢,以次遞加,至杖一百改為罰十五兩而止。如無力完納,折為作工。應罰一兩,折作工四日,以次遞加,至十五兩折作工六十日而止。然竊盜未便罰金,議將犯竊應擬笞罪者,改科工作一月;杖六十者,改科工作兩月;杖七十至一百,每等遞加兩月。又附片請將軍、流、徒加杖概予寬免,無庸決責。自此而笞、杖二刑廢棄矣。

三十一年,修訂法律大臣沈家本等奏請刪除重法數端,略稱:「見行律例款目極繁,而最重之法,亟應先議刪除者,約有三事:一曰凌遲、梟首、戮尸。凌遲之刑,唐以前無此名目。遼史刑法志始列入正刑之內。宋自熙寧以後,漸亦沿用。元、明至今,相仍未改。梟首在秦、漢時惟用諸夷族之誅,六朝梁、陳、齊、周諸律,始於斬之外別立梟名。自隋迄元,復棄而不用。今之斬梟,仍明制也。戮尸一事,惟秦時成蟜軍反,其軍吏皆斬戮尸,見於始皇本紀。此外歷代刑制,俱無此法。明自萬歷十六年,定有戮尸條例,專指謀殺祖父母、父母而言。國朝因之,後更推及於強盜。凡此酷重之刑,固所以懲戒兇惡。第刑至於斬,身首分離,已為至慘。若命在頃忽,菹醢必令備嘗,氣久消亡,刀鋸猶難幸免,揆諸仁人之心,當必慘然不樂。謂將以懲本犯,而被刑者魂魄何知;謂將以警戒眾人,而習見習聞,轉感召其殘忍之性,實非聖世所宜遵。請將凌遲、梟首、戮尸三項,一概刪除,死罪至斬決而止。凡律例內凌遲、斬梟各條,俱改斬決。斬決而下,依次遞減。一曰緣坐。緣坐之制,起於秦之參夷及收司連坐法。漢高後除三族令,文帝除收孥相坐律,當時以為盛德。惜夷族之誅,猶間用之。晉以下仍有家屬從坐之法,唐律惟反叛、惡逆、不道,律有緣坐,他無有也。今律則奸黨、交結近侍諸項俱緣坐矣,反獄、邪教諸項亦緣坐矣。一案株連,動輒數十人。夫以一人之故而波及全家,以無罪之人而科以重罪,漢文帝以為不正之法反害於民,北魏崔挺嘗曰『一人有罪,延及闔門,則司馬牛受桓魋之罰,柳下惠膺盜跖之誅,不亦哀哉』,其言皆篤論也。今世各國,皆主持刑罰止及一身之義,與『罪人不孥』之古訓實相符合。請將律內緣坐各條,除知情者仍坐罪外,其不知情者悉予寬免。餘條有科及家屬者準此。一曰刺字。刺字乃古墨刑,漢之黥也。文帝廢肉刑而黥亦廢,魏、晉、六朝雖有逃奴劫盜之刺,旋行旋廢。隋、唐皆無此法。至石晉天福間,始創刺配之制,相沿至今。其初不過竊盜逃人,其後日加煩密。在立法之意,原欲使莠民知恥,庶幾悔過而遷善。詎知習於為非者,適予以標識,助其兇橫。而偶罹法網者,則黥刺一膺,終身僇辱。夫肉刑久廢,而此法獨存,漢文所謂刻肌膚痛而不德者,未能收弼教之益,而徒留此不德之名,豈仁政所宜出此。擬請將刺字款目,概行刪除。凡竊盜皆令收所習藝,按罪名輕重,定以年限,俾一技能嫻,得以餬口,自少再犯、三犯之人。一切遞解人犯,嚴令地方官僉差押解,果能實力奉行,逃亡者自少也。」奏上,諭令凌遲、梟首、戮尸三項永遠刪除。所有現行律例內凌遲、斬梟各條,俱改為斬決;其斬決各條,俱改為絞決;絞決各條,俱改為絞監候,入於秋審情實;斬監候各條,俱改為絞監候,與絞候人犯仍入於秋審,分別實緩。至緣坐各條,除知情者仍治罪外,餘悉寬免。其刺字等項,亦概行革除。旨下,中外稱頌焉。

三十二年,法律館奏準將戲殺、誤殺、擅殺虛擬死罪各案,分別減為徒、流。自此而死刑亦多輕減矣。又是年法律館以婦女收贖,銀數太微,不足以資警戒,議準婦女犯笞、杖,照新章罰金。徒、流、軍、遣,除不孝及奸、盜、詐偽舊例應實發者,改留本地習藝所工作,以十年為限,餘俱準其贖罪。徒一年折銀二十兩,每五兩為一等,五徒準此遞加。由徒入流,每一等加十兩,三流準此遞加。遣、軍照滿流科斷。如無力完繳,將應罰之數,照新章按銀數折算時日,改習工藝。其犯該枷號,不論日數多寡,俱酌加五兩,以示區別。自此而收贖銀數亦稍變矣。

故宣統二年頒布之現行刑律,第將近數年奏定之章程採獲修入,於是刑制遂大有變更。其五刑之目,首罰刑十,以代舊律之笞、杖。一等罰,罰銀五錢,至十等罰,為銀十五兩,據法律館議覆恤刑獄之奏也。次徒刑五,年限仍舊律。次流刑三,道里仍舊律,然均不加杖,以法律館業經附片奏刪也。次遣刑二:曰極邊足四千里及煙瘴地方安置,曰新疆當差。以閏刑加入正刑,承用者廣,不得不別自為制也。次死刑二:曰絞,曰斬。時雖有死刑唯一之議,以舊制顯分等差,且凌遲、梟首等項甫經議減,不敢徑行廢斬也。徒、流雖仍舊律,然為制不同。按照習藝章程,五徒依限收入本地習藝所習藝;流、遣毋論發配與否,俱應工作。故於徒五等注明按限工作,流二千里注工作六年,二千五百里注工作八年,三千里注工作十年,遣刑俱注工作十二年。收贖則根據婦女贖罪新章酌減銀數,改為通例。罰刑照應罰之數折半收贖,徒一年贖銀十兩,每等加銀二兩五錢,至徒三年收贖銀二十兩。流刑每等加銀五兩,至三千里贖銀三十五兩。遣刑與滿流同科。絞、斬則收贖銀四十兩。亦分注於各刑條下。然非例應收贖者,不得濫及也。捐贖,據光緒二十九年刑部奏準照運糧事例,減半銀數,另輯為例。其笞、杖雖不入正刑,仍留竹板,以備刑訊之用。外此各刑具,盡行廢除,枷號亦一概芟削,刑制較為徑省矣。

惟就地正法一項,始自咸豐三年。時各省軍興,地方大吏,遇土匪竊發,往往先行正法,然後奏聞。嗣軍務敉平,疆吏樂其便己,相沿不改。光緒七八年間,御史胡隆洵、陳啟泰等屢以為言。刑部聲請飭下各省,體察情形,仍照舊例解勘,分別題奏。嗣各督撫俱覆稱地方不靖,礙難規復舊制。刑部不得已,乃酌量加以限制,如實系土匪、馬賊、游勇、會匪,方準先行正法,尋常強盜,不得濫引。自此章程行,沿及國變,而就地正法之制,訖未之能革。


\end{pinyinscope}