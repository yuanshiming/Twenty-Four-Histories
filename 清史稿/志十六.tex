\article{志十六}

\begin{pinyinscope}
災異二

洪範曰:「火曰炎上。」火不炎上,則為咎徵。凡恆燠、草異、羽蟲之★、羊禍,其災火,赤眚、赤祥皆屬之於火。

順治十三年冬,莊浪燠,無雪。十八年冬,龍門無雪。

康熙二十一年冬,西寧無雪。二十七年冬,天鎮無雪。三十五年冬,臨縣無雪。四十一年冬,平原燠如夏。四十二年冬,咸陽燠,無雪。五十九年冬,浮山無雪。

乾隆四年冬,彭澤、元昌燠如夏,人有衣單衣者。四十九年冬,菏澤無雪。五十七年冬,蘇州無冰雪。

嘉慶三年冬,桐鄉燠。十三年冬,昌黎無雪。十八年冬,鄖縣無雪。二十年二月,湖州大燠。

咸豐十年冬,皋蘭無雪。

同治元年冬,黃縣大燠。

光緒元年冬,望都、撫寧無雪。十四年冬,皋蘭燠。

順治三年五月,丘縣雨麥。六月,潮陽雨豆。十一年六月,商州一蒂兩瓜,大如斗。十二年二月,渭南天雨粟,平樂天雨蕎麥。三月,鳳陽、安西天雨叕麥、豌豆。五月,臨潼、咸陽雨叕麥、豌豆。十四年二月,婺源雨黍。十五年冬,昌化竹生實。十八年十月,高要竹生實。

康熙二年十月,阜陽雨粟,粒若蕎麥,圓小而堅,味辛,厚處盈寸。三年七月,婺源大鄣山竹生實,形如麰,民採而舂食之,厥味甘。二十一年三月,溫州雨豆。二十二年四月,寧都天雨豆,又雨黑黍。二十六年二月,合肥雨黑豆。二十八年正月,含山雨小豆。四十一年二月,湖南竹生實。四十四年三月,葭州雨黑豆。四十五年春,橫州竹生實。六十年夏,安化天雨蕎麥。六十一年正月,大埔竹結實。十一月,岑溪枯竹開花。

雍正五年五月,鍾祥竹開花。十月,當塗雨紅綠豆,形如小麥,無蒂。十年,什邡縣雨蕎麥。十一年二月,山陽、清河雨黑豆,啖之味苦。十二年三月,宜昌竹結實,民採食之。十三年七月,夷陵竹生實如麥,民競採食之。

乾隆二年二月,昌化雨豆。六年十一月,成縣竹生實。十八年九月,陽春竹皆結實枯死。二十二年正月,永嘉東山竹結實如麥。二十三年,池州雨豆。二十六年,安化雨蕎麥,形似而小。四十一年六月,餘姚雨小麥、黃豆。

嘉慶十二年春,黃陂雨豆。

道光二年夏,黃巖天雨菽。四年十月,黃梅雨豆麥穀米。十六年十二月,武寧雨豆。二十年十一月,鍾祥竹開花。二十五年七月,竹生米,可食。

咸豐元年六月,孝義山竹結實,人採食;青浦竹生花。二年十一月,太平雨豆。四年二月,隨州天雨豆。四月,黃岡雨黑豆,食之味苦。冬,武昌縣雨黑豆。五年正月初三日,孝感天雨小豆。二月,又雨豆。三月,武昌天雨黑豆,如槐實;黃安雨豆。夏,黃州、蘄水雨豆,如槐實。十一月,黃岡天雨豆,如槐實;歸州竹結實,人採食之。六年八月,隨州雨豆。八年二月,興國雨豆,色赤。秋,興山竹結實。九年春,麻城民間番瓜成人形。十年,龍泉雨豆,色赤。十一年三月,麻城雨豆。十二月,溪梁雨豆,色赤。

同治元年八月,西寧丹噶爾竹開紅花;灤州瓜窳剖之有血,食者立病。三年正月,永豐天雨豆,五色斑爛。三月,景寧、嵊縣雨豆。五月,京山縣雨豆,其色黑光。六年二月,棲霞雨草子如蕎麥。夏,嵊縣雨豆。七年,★城生豆如人面,五官俱備。九年十月,遂昌雨穀,外黑內紅;德興雨豆,內黑外白。十一年三月,即墨天雨紅豆蕎麥。

光緒二年四月,青田雨豆。九年三月,咸寧雨紅麥。十年八月,孝義竹結實。十一月,洮州山竹開花結實。

順治十四年,武昌鴉卸火,集人廬,輒災,一月始息。

康熙十六年,海豐有異鳥集林中三日,高六七尺,舒吭丈餘,啄雞鳧以食,居民奮擊之,分啖其肉,輒病死。

順治元年七月,商州郊外見大羊,色黃,長丈餘,百姓搏而殺之,肉重五百斤。四年五月,山陰民家羊生羔,三足,前二後一。五年,杭州民家羊生羔,三足。

康熙十二年,北山民家產一羊,一角一目,隨斃。二十四年,順德羊生羔,三足,前一後二。

乾隆元年四月,連州山羊入城,蹄角甚巨,人逐獲之。五十五年,雲夢見三足羊。

道光十七年八月,武進民家產羊,人首羊足,墮地即斃。

咸豐十年八月,江山西山白羊成★,★不見。

同治三年,寧州民家羊產一羔,五角。

光緒九年九月,孝義民家羊產羔,人面羊身。二十五年,寧州民家羊產一羔,兩首。二十七年三月,丹噶爾有一羊兩頭;一羔三足。三十三年,寧州民家羊生羔,人面。

順治元年七月,西鄉文廟火。六年正月初六日,無為州城門大火。八年七月,嵐縣火焚民房。十二年五月十八日,梧州府城外大火;十二月又大火。十四年十月,蘄水火。十五年,連州大火。十八年五月,宜昌大火,延燒民舍千餘間。八月,嵊縣城中大火。

康熙元年五月,黃岡大火,焚民房十之八九。秋,荊州大火,燒民房殆盡。十月,興國火起自大西門,延燒城中,至大東門,男婦死者以千計。二年二月初三日,海陽西郊火起,延燒民房千餘家。七月十五日,黃岡大火,燒民舍殆盡。三年四月,梧州府城外大火,焚八百餘家。五月,海陽大火。六月,含山鼓樓火。四年正月,京山火,焚一百八十餘家。十月,懷遠大火,自西城外至驛前,延燒一百五十餘家。十一月,高州府城火,合浦火焚民舍。十二月,廣州府城火。五年正月,海陽南北二廂火起,延燒民房千餘間。二月十三日,鍾祥火,毀數百家,延及府署,焚死人畜甚多;二十八日,城內外又燔數百家。秋,靈川北廂火起,延燒民房殆盡。十二月,嚴州大火,民房盡毀,延燒城樓。六年正月,海陽城外四廂火起,延燒民房千餘間,死於火者二百餘人。七年三月,鄖陽府火,民舍盡毀。七月,大冶西市火,延燒百餘家。八月,宣化城內火,焚千餘家,次日城外又焚百餘家。八年二月,海陽西北二廂火,焚民房數百間。三月,鄖縣火。十月,獨山州大火,仙居、黃巖二縣火。九年,平樂南關火,延燒四十餘家。是年十二月至次年四月,火災凡四見。

十年五月,錢塘大火。七月,大冶西市火。九月初七日,浦江太極宮大火。十一年三月,縉雲大火,延燒縣署。十二年九月,宣縣西門外沙市被火災四次,毀數百家。十三年五月,靜樂火,毀民舍;興國唐村火,焚死二百三十七人。十五年七月,太平城內火,毀民房過半。十八年正月初三日,望江吉水鎮火災,燔百餘家。六月,順慶府治火。十九年正月十五日,平陽火,毀民居過半。三月初四日,海陽火,延燒百餘家,死者四十餘人。七月,和平城外火,延燒百餘家。

二十年二月,東湖縣署火。五月,蒼梧東廊火。八月,濟寧州大堂火,溫州火,燎民舍五千餘間。九月,永嘉城中大火,燎民舍千餘間。十月,思州府火,延燒五十餘家。二十一年春,濟寧州城內東偏大火,延及西隅,民舍皆盡,關壯繆侯祠亦毀,獨神像香案無恙。八月,池州天火,毀田禾芋苗,葉盡生煙。十月,萬載火,延燒城隍廟,連山西郊火。二十二年四月,囗陽西門火。二十三年七月,長寧城隍廟火。冬,忠州石寶寨火。二十五年四月,萬載火,延燒城隍廟;六合南門火,焚市廛數百間。二十六年十月初一日,平陽城樓火,燔百餘家;忠州石寶寨又火。二十七年八月,婺源火,延燒五十餘家。十二月,合浦西橋火,郡城火。二十八年九月己卯,蒼梧西關火。十二月,松陽火。二十九年七月二十八日,酆都城內大火,民居盡毀。

三十年十月,平樂火,延燒二百餘家;獨山州大火。三十一年九月,平陽城樓火。三十二年夏,鎮安府署火,延燒民居數十間。九月,平陽東門外火,燔數十家。三十四年正月,馬平南川河下火,延燒大南門城樓。三十五年七月,江夏火起自火藥庫,死者無算。八月,陽高南街火。三十七年二月,漢陽漢口鎮火,延燒數千家。

四十年九月,陽山火,延燒二百餘家。四十一年二月,崖州火,傷四人。四十二年七月十六日,桐鄉青鎮火,燔民舍一百七十餘家。四十三年正月二十九日夜,灌陽火,焚東門外民舍殆盡。四十四年三月,婺源太平坊火。十一月,武寧火。四十五年四月,竹溪火,官署民房俱■。四十六年正月初四日,荔浦火,初囗又火。巴縣太平門大火。四十八年三月,獨山州城內大火,居民無得免者。四十九年八月二十五日,嘉定火,延燒七十餘家。

五十年正月,大埔白堠墟火,毀民舍數百家。五月,萬載潭埠火,市店民房蕩然無存。五十二年十一月,宣平火。五十三年九月初八日,宣化沙市火,焚千餘家;獨山州大火。五十四年九月,江陵火,延燒二千餘家。五十五年九月,江陵又火,延燒二十餘家,死十一人;思州府城大火,延燒四十餘家。五十六年五月初三日,丹棱縣大火,延燒數百家。五十七年三月,合肥城內大火,延燒四十餘家。八月初一日,鍾祥城內火,延燒城外民房數百間。五十九年十月,蒼梧西門外大火。六十年四月十八日,鹽山縣城火,自學宮延燒東南北三門,毀民居數千家。六十一年二月,無為州小西門內火,延燒三十餘家。七月,獨山州東門火。冬,麗水縣火。

雍正二年正月,沔陽仙桃鎮大火,焚百餘家,死者甚★。七月,梧州梧城驛火。十月,城內火。十一月,戎墟火。十二月初一日,開化城內火,延燒百餘家。三年六月,梓潼縣文昌廟火。七月,馬平小南門火,延燒三百餘家。四年十二月初四日,平陽西門外火,燔百餘家。五年十二月,北流民舍失火,延燒縣署,案牘皆盡。六年正月十六日夜,蒼梧火,延燒民居一百七十餘間;高州城東火。十月,昆山火,焚朝陽門譙樓。七年九月,蒼梧戎墟火。九年正月,荊州大火。十年五月初三日,阜陽西城火,延燒民舍四千六百十一間。十一年七月初七日,玉屏聞空中有呼救火聲,越半月,鼓樓街災,燒民居數百家。十三年冬,婺源城隍廟災。

乾隆元年四月,通州北郭火,延燒百餘家。十一月,玉屏南門火。二年二月十八日,鎮安府城火,燔數百家。三月乙丑,同官明倫堂大火。五月,沁州大街火。九月,北流典史署火,延燒民舍。三年十月初七日,潮陽南門火。四年正月十七日,瑞安大火,燔百餘家。四月十八日,鎮安城內火,延燒八十餘家。五年二月,嵊縣火,延燒二百餘家。六年正月初六日,梧州府南門外火,延燒民房三百餘家。七年二月十四夜,饒平縣城火,延燒大樓房三十餘間,小屋無數。八年十一月,饒平縣又火。十年二月庚午,泰安縣署火,延燒百餘家。十一年六月,海豐龍津橋火,延燒蓬鋪四十餘間。十二年八月,化州南街火。九月,豐順縣城火。十一月初十日,崖州東街火,延燒七十餘家,傷二人。十五年四月,泰安火。十七年正月朔,漢陽糧船火,焚數十艘。四月,桐鄉南柵大火,毀市廛三百餘家。五月二十二日,保昌孝悌街火,延燒三十餘家。十八年七月,陸川大火,毀民居。十月,梧州府城外大火,傷二十餘人。十九年八月,蒼梧府城外又火。二十年三月,高州府城火,五月又火。二十二年十月,宜昌東湖火,燔民居無數。二十三年三月初一日,重慶太平門外大火。四月,獨山州大火。二十四年十二月初八日,惠來縣署火。二十五年八月二十八日,朝天千斯門內大火。二十七年十月,石門玉溪鎮火,延燒百餘家。二十八年十二月初五日,慶元火,延燒五十餘家。二十九年五月,沂水縣城南綢市街火,延燒數百家。十月,婺源西關外居民失火,延燒數百家。三十年十月,梧州府城外火。三十一年十一月,蒼梧戎墟大火三次,共燒民房六百餘家。三十三年正月二十八日,梧州府城外火,延燒三百餘家。三十八年七月,金華府署火。四十二年十二月,青田城大火。四十四年十一月初四日,桐鄉大火,燔市廛四十餘家。四十六年夏,陸川城南失火,延燒縣署。四十七年六月,寧波府城火,毀鼓樓。四十八年五月庚子,慶元火,延燒百餘家。四十九年四月朔,成都大火,延燒官署民舍殆盡。五十年夏,潛江城外火。五十二年三月,江陵城隍廟火。五十五年三月,義烏縣署火。五十六年十二月,南昌火,延燒千餘家。

嘉慶三年二月丙子,京師乾清宮火。九年七月初三日,定海城中大火,延燒二百餘家。十三年五月十二夜,濟南府西門大火,延燒四百餘家。十六年三月二十九日,石門城西火。十七年春,齊東火,燒死數百人。十八年三月,貴陽城大火。二十年四月,尚山火。十二月二十日,蘭州西門火藥局焚軒轅城樓民舍,死者數十人。二十二年八月,黃陽火,燒民舍一百餘家。二十四年閏四月,青田火,延燒二百餘家。五月,青浦城火,延燒七十餘家。

道光二年六月十一日,大埔南門外火,延燒兩晝夜。九年七月十三日,江山江郎山火,延燒兩晝夜。十年八月,鉛山石塘火,延燒五百餘家;次年又火。十六年十二月十九日,雲和火,毀民舍八十餘家。十九年正月初二日,貴陽府道德橋火,延及貢院頭門。三月二十日,貴陽府學大成殿災;江陵沙市大火,燔數百家。四月,定海道頭港營船火。二十二年十月初三日,麗水火,燔一百四十七家。二十五年冬,黃巖蒼頭街火。二十六年五月,貴陽火,燒民房八百餘家。二十九年十二月初三日,太平城隍廟火災。

咸豐元年十月,太平火,燔百餘家。二年八月,通州西庫火。十月,武昌縣署火。十一月,漢陽火。六年十一月,枝江火,燔市廛八百餘家。七年五月,皋蘭西關火,延燒市廛二百餘間。八年秋,武昌縣左市火。十年二月,青浦火,麗水火。

同治元年冬,黃山石路橋大火。三年十月,黃巖火。五年十一月,漢口火;餘干瑞洪鎮火,延燒四百餘家。六年三月,江夏火藥局災,斃者以千計。五月二十五日,漢陽鮑家巷火,燔船隻,傷人口甚★。七年十月,太平縣城火,燔四百餘家。九年冬,黃巖火。十一年四月,烏程火,延燒十餘里。十三年五月,武昌縣小西門火。

光緒二年七月壬午,青浦城火,延燒三十餘家;九月庚寅又火,東碼頭上下岸俱■。六年十一月,武昌縣北市火。二十八年二月,皋蘭南街大火。二十九年十月,西寧火。三十一年七月,西寧大街火;十一月,孔子文廟災。

宣統元年正月初四日,皋蘭縣災,延燒官舍六十餘間。二月二十六日,蘭州省城院門南街大火,延燒房屋二百零九間。

順治十年二月,曹縣夜間火光遍野。五月,渭南四野火災,見持炬人三尺許,★繞火際,次日焚處拾一折簡,字數行,如人書,其語曰:「土地不寧,天降兇神三位,一收牛,一收馬,一收人。」十四年十月二十七日,清豐空中起火,燒民房數百間。

康熙十二年三月,縉雲曉見黧面人從空中放火,捕之不見。五月,寧波仙鎮廟井中有火光上騰。十四年八月十五日,海豐火光遍野。二十五年二月,兩廣總督衙門兩旗竿忽白晝飛火,焚其右,焦灼過半。三十一年秋,南樂空中有火,著鐵皆明,自申至亥乃止。五十年二月丙寅,東平烈風中有火光。

乾隆十七年四月二十九日,岐山有火光,自西而東。七月,芮城有火光如電。二十年十一月,彭澤江心洲有穴出火,投葦輒燃,久而不息。二十二年二月二日,高平有火球大如斗,其色黃紅綠相間,就地行走,不知所終。二十七年九月,臨縣空中有火光大如斗,墜城南隅。三十三年三月,即墨日夕有火球經天。五十年冬,棗陽有火球如斗,飛半里外。

嘉慶九年二月十二日,滕縣城東石溝見火球飛落。十二年四月,黃安有火大如球,自東而西,落於泮池。十六年夏,撫寧夜遍地起火。

道光三年三月,蘄州、清江水中出火。二十年五月,均州夜見火光。二十二年十一月,鄖西地中出火。二十四年七月,光化遍地綠火。二十六年正月,平鄉火光遍野。

咸豐元年八月二十八日,隨州有天火自西南流東北,其光觸地,有聲如鼓。三年正月,通州有火如星如燐,以千百計,自西南趨東北,凡四五夜始熄。十年冬,肥城既昏,有火從地中起,如燐而火,色赤而青,作二流光,遍地皆燃。

同治二年九月,曲陽有火球自西南飛向東北,或散四方,或聚為一,其象無常。四年,通渭、泰安火光西現如隕星。

光緒元年正月十四日,灤州五聖祠突有火光,俄而火起高矗雲霄,祠竟無恙。五年冬,玉田見火球飛向東北,其聲如雷。二十二年四月戊子,南樂有火光徑尺,明如月,自西南往東北,尾長丈許,忽炸為火星四散。二十三年五月戊午夕,南樂有火光,圓可徑尺,飛向西南。二十五年十一月乙未夜,南樂有火光流空中,其明如月。二十六年七月壬戌夜,南樂有火光流空中。

順治三年六月初四日,鎮洋新安鎮李明家地出血。初五日,俞二家地出血尤甚。五年夏,嘉定見赤氣■東方。七年冬,鶴王鎮鄉民見血從地湧出。九年正月,東昌有赤光,聲如水鴨,往東南而沒。十六年,永年南關外地中湧血,嘶嘶有聲。

康熙十四年四月,萊陽地湧血丈餘,氣腥,久不敢近。五十九年七月十六日,榮成、萊陽有赤氣自東而起,★如匹練,■向西北去,有聲如雷。

雍正七年十二月二十八日夜分,福山見紅光滿野。

乾隆三十五年七月二十八日,肥城有赤光自北方起,夜半漸退;長山西北見赤氣彌天,中有白氣如縷間之,四更後始散。二十九日夜,榮成夜見紅光燭天;東光有氣如火,橫蔽西北,■數十丈,中含紅光,森如劍戟上射。

嘉慶九年正月,歷城天雨血。

道光十一年冬,太平雨血,著人衣皆赤。十七年六月二十八日,嵊縣有赤光如球,高數丈,三日乃滅。二十八年正月,松滋天雨血,以★盛之,作桃花色。

咸豐三年正月十四日,西鄉雨血如注。五年八月,曹縣東方有赤氣如旗桿形。六年七月,武進地出血。

同治五年秋,崇陽雨血。七年正月二十日,光化雨血。

光緒二年二月初四日,曹縣見紅光自天降於八里灣水中。七年四月,襄陽雨血。


\end{pinyinscope}