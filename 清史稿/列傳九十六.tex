\article{列傳九十六}

\begin{pinyinscope}
崔紀喀爾吉善子定長孫鄂雲布雅爾圖晏斯盛瑚寶

衛哲治蘇昌鶴年吳達善崔應階王檢吳士功

崔紀,初名珺,字南有,山西永濟人。年幼喪母,哀毀如成人。事父及後母孝。康熙五十七年,成進士,改庶吉士,授編修。遷國子監司業,以母憂歸。服闋,補故官。三遷祭酒。乾隆元年,提督順天學政。雍正間,採安徽學政李鳳翥、河南學政習俊、浙江學政王蘭生條議:每歲令諸生五人互結,無抗糧攬訟;諸生有事告州縣,當先以呈詞赴學掛號;為人作證及冒認命盜案,先革後審;諸生毆殺人及代寫詞狀,加常罪一等;已斥諸生不許出境;諸生欠糧,必全完乃收考。紀疏請罷之。又定諸生月課三次不到,詳革,紀請改一年;諸生完糧,上戶限十月,中、下戶限八月,紀請改歲底。下部議行。遷詹事,再遷倉場侍郎,署甘肅巡撫。

二年,移署陜西巡撫。疏言:「陜屬平原八百餘里,農率待澤於天,旱則束手。惟鑿井灌田,實可補雨澤之缺。臣居蒲州,習見其利。陜屬延安、榆林、邠、鄜、綏德各府州,地高土厚,不能鑿井。此外西安、同州、鳳翔、漢中四府並渭南九州縣最低,渭北二十餘州縣地較高,掘地一二丈至六七丈,皆可得水。勸諭鑿井,貧民實難勉強。懇準將地丁羨銀借給充費,分三年繳完。民力況瘁,與河泉自然水利不同。請免以水田升科。」上諭曰:「此極應行之美舉,當徐徐化導,實力奉行,自不能視水田升科也。」擢吏部侍郎,仍留巡撫,尋實授。紀疏言:「陜西水利,莫如龍洞渠,上承涇水,中受諸泉。自雍正間總督岳鍾琪發帑修濬,涇陽、醴泉、三原、高陵諸縣資以灌溉。惟未定歲修法,涇漲入渠,泥沙澱閼,泉泛出渠,石罅滲漏。擬於龍洞高築石堤,以納眾泉,不使入涇。水磨橋、大王橋諸泉亦築壩其旁,收入渠內。並額定水工,司啟閉。」均從之。陜西民憚興作,言紀煩擾。上令詳勘地勢,俯順輿情。三年,命與湖北巡撫張楷互調,時報新開井七萬餘,上令楷察勘。楷言民間食其利者三萬二千餘,遇旱,井效乃見。民益私鑿井,歲歲增廣矣。

紀至湖北,自陳不職,部議降調。上諭曰:「紀在陜西鑿井灌田,料理未善,致反貽民累。惟其本意為民,命從寬留任。」五年。總督德沛劾紀以公使錢畀護糧道崔乃鏞,上又聞紀以淮鹽到遲,令民間暫食私鹽,諭紀自列,紀疏辨,下部議,降調。六年,再授祭酒。九年,督江蘇學政。以父憂歸。十四年,起授山東布政使。以東省貧民借官穀累百萬石,請視部定價石六錢,收折色,紓民力。十五年,命以副都御史銜再督江蘇學政,力疾按試。旋卒。

紀潛心理學,上亦聞之,再任祭酒,召見,命作太極圖說。歷官所至,以教養為先。遇事有不可,輒艴然曰:「士君子當引君當道,奈何若是?」

喀爾吉善,字澹園,伊爾根覺羅氏,滿洲正黃旗人。先世居瓦爾喀,有赫臣者,當太祖創業時來歸,授牛錄額真。使葉赫,葉赫部長金臺石使人戕之。太祖滅葉赫,令其子克宜福手刃其仇以祭。克宜福從軍有功,世職至三等阿達哈哈番。克宜福子喀齊蘭,官至正黃旗副都統;孫凱里布,官至吏部尚書:皆襲世職。

喀爾吉善降襲拜他喇布勒哈番,授上駟院員外郎。歷工部郎中,兼襲世管佐領。雍正六年,命偕通政使留保如廣東按署巡撫阿克敦等被劾狀。八年,擢兵部額外侍郎。九年,授侍郎。十三年,以驗馬不實奪官,令往盛京收糧。乾隆元年,起廢籍,命管圓明園八旗兵丁。復往盛京收糧,奏禁八旗臺站官兵與朝鮮貿易。上諭曰:「官兵不暇貿易,亦不諳貿易。當令商民與互巿,務均平交易,毋抑價,毋強索。」三年,擢內閣學士。遷戶部侍郎,協理步軍統領刑名事務。調吏部,四年,命兼管三庫。

五年,授山西巡撫。上聞山西布政使薩哈諒、學政哈爾欽皆貪婪,詢喀爾吉善。喀爾吉善疏劾,命侍郎楊嗣璟會鞫,論如律。上以喀爾吉善不即劾,下部議,奪官,命寬之。又劾河東鹽政白起圖貪婪,白起圖疏辨,命副都統塞楞額往鞫,論如律。七年,調安徽。

八年,復調山東。疏言:「山東歲饑,民多流亡,而鄰省貧民亦有轉入山東覓食者,請飭官吏勸各回故土以待治賑。」上諭曰:「所見甚得體。各省督撫當於平居無事時委曲開導,使知敦本務實,力田逢年;若輕棄其鄉,本業既荒,無所依倚。即國家收養資送,亦不得已之舉,非可恃為長策也。」又以濟南、武定、東昌三府遇旱,濟南、東昌府倉存穀緩急可相通;武定無倉,請撥登、萊二府倉穀以濟民食。九年,疏言:「方春糧價踴貴,貧民艱食,請酌量減糶。」又言:「山東兵米,本折兼支,春季價昂支折色,秋季價減支本色,請春秋二季本折更換。」又請修德州、海豐、惠民、樂陵城工以代賑。復以濟南、武定諸屬縣麥復不登,令於曹、沂諸府豐收之區採買接濟。上皆允之。直隸槁城知縣高崶請開臨淄、即墨、平陰、泰安、沂、費、滕、嶧諸縣銀、銅、鉛、鐵各礦,事下喀爾吉善勘奏,奏言:「東省拱衛神京,地跨四府八縣,形勢聯屬。礦洞久經封禁,未便開採。利之所在,眾必共趨。恐濟、武災區,沂、曹盜藪,別生事端,應仍封禁。」上亦如其請。

十一年,遷閩浙總督。臺灣生番為亂,遣兵討之。奏言:「臺灣流民日多,匪類肆竊,甚或恣行不法,民間謂為闖棍。請令竊案再犯及闖棍治罪後,並逐回內地。」又請在臺人民迎取眷屬,限一年給照過臺。浙江處州總兵苗國琮請於官山種樹,儲戰船桅木之用,下喀爾吉善勘奏。奏言:「令有司種樹,須先糜帑,且必百十年後始中繩墨,日久稽察非易。不若許民自種,在官不費經營,而巨材可獲實用。」從之。疏劾浙江巡撫常安貪婪,命大學士訥親往鞫得實,論如律。詔嘉其公直,加太子少保。疏言:「寧海東湖舊與海通,宋後失修,飭府縣察形勢土性,導士民輸資築堤,撥為世業,定限升科。」上諭曰:「勸課農桑,興修水利,務本之圖也。欣悅覽之!」十五年,加兵部尚書銜。

十六年,上南巡,蠲江南積逋二百餘萬,浙省無逋賦,亦特蠲本年正賦三十萬,制詩褒之。十七年,以年老乞休,溫詔慰留。疏言:「閩省產米少,本歲豐稔,宜為儲備。請現存倉穀不及半者,令購足數;已及半而本地穀賤,亦以原存糶價買補。」上是之。漳州民蔡榮祖謀亂,事洩,捕獲,寘之法,予議敘。十九年,加太子太保。上以八旗生齒日繁,許在京漢軍改入民籍,推行於各省。喀爾吉善與福州將軍新柱疏言:「漢軍原為民,無問世族、閒散,許入民籍。如別無生計,坐補綠營糧缺。所遺馬、步甲,以滿洲兵坐補。」二十二年秋,病瘍,遣醫偕其子定敏馳視,賜人葠。未幾,卒,賜祭葬,謚莊恪。

定長,喀爾吉善子。初授內閣中書,遷侍讀。擢江南徐州知府。四遷至巡撫,歷安徽、廣西、山西、貴州諸省。乾隆十八年,湖廣總督永常奏請於鄰省會哨,定長奏:「貴州與鄰省聯界,苗、夷環處。遽行會哨,苗性多猜,或滋事變。請停止。」從之。二十年,題請原任黔西知州黃秉忠入祀名宦,上以秉忠為總督廷桂父,瞻徇巿恩,降旨嚴斥。二十二年,上南巡,請入覲,命便道省喀爾吉善,賜詩褒寵。尋命與尚書劉統勛按雲貴總督恆文貪婪狀,即命署云貴總督。調山西巡撫,未之任,丁父憂。旋授副都統銜,往西路軍營督屯田事。補兵部侍郎,授福建巡撫,遷湖廣總督。三十三年,卒,諭部議恤。尋署總督高晉劾荊州副都統石亮衰庸,上責定長徇庇,罷恤典。

鄂雲布,喀爾吉善孫。初授筆帖式。三遷工科給事中。嘉慶元年,授陜西漢中知府。上以鄂雲布喀爾吉善孫,家風具在,即擢甘肅西寧道。再遷江蘇布政使,護安徽巡撫。旋以秋審諸案原擬緩決,刑部多改情實,責鄂雲布寬縱,下吏部議降調,命留任。尋遷貴州巡撫,年老召還,鄂雲布聞命即行。上聞之,不懌,下吏部議,奪官,授筆帖式,賞藍翎侍衛,充葉爾羌辦事大臣。旋卒。

雅爾圖,蒙古鑲黃旗人。雍正四年,自筆帖式入貲授主事,分工部。再遷郎中。十三年,授鑲藍旗滿洲副都統。乾隆元年,疏言:「京員無養廉,請將戶部餘平銀給部院辦事官。八旗參佐等員視步軍營例,予空糧。」如所議。師征準噶爾,授參贊大臣。三年,命暫管定邊副將軍印。四年,召授左副都御史,遷兵部侍郎。

河南新鄉民及伊陽教匪為亂,命往按治,就授河南巡撫。疏言:「河南多盜,不逞之民陰為之主,俗謂『窩家』。保甲、甲長等畏窩家甚於官法。大河以南,深山邃谷。民以防鳥獸為名,皆有刀械。惑於邪教,懷私角斗,何所不為。如梁朝鳳、梁周、張位等輩,黨類甚多,愚民易遭煽惑。與其發覺後盡置諸法,何如於未發覺前設法銷散。文武會遣兵役搜查,仍令自首免罪。」又言:「各省提鎮以下官皆有伴擋兵丁及各色工匠,一營有數名虛糧,即少數名額兵。請照官級核定數目,不得虛占兵額。」俱下部議行。

五年,奏報捕得女教匪首一枝花,命議敘。尋諭河南止設河北、南陽二鎮,與巡撫不相統屬,視山西例兼提督銜。疏陳整飭營務:足兵額,勤差操,明賞罰,練技藝,整軍械,重兵食,驗馬匹,謹守望,嚴約束;並請以州縣民壯之半交駐防汛弁操練;並戒兵民和衷,不得偏袒,平時試習騎射,期於嫺熟:俱如所請行。三月,疏言:「河南上年霪雨,省城多積水。臣令淺處濬深,窄處開寬。為合城受水之區通溝建閘,時其蓄洩。養魚植木,以利民用。」又言:「河南上年被水,奉命濬省城乾河涯及淮、潁、汝、蔡各水。目前二麥成熟,農務正殷,餘請概停開浚。」上從之。又奏言:「現獲盜百餘,多系鄰省人,臣迭飭員弁分路訪緝。出省捕盜,例須赴地方官掛號,盜聞而潛逃,請得逕行往捕。」上命勉為之。

六年,又奏言:「河南界連五省,西南伏牛、嵩山、桐柏等山,支幹交錯,地多林木,易於藏盜。請每歲秋冬,與聯界各省文武訂期巡察。」上命如所請行。七年,奏言開、歸等處積水,無妨田畝,上責其掩飾。尋又奏:「河南地平土髟松,水利誠不如東南之通達。開、歸等處地當下游,夏秋大雨,澗水匯注。積水未消,多系鄰近黃河州縣。歷來豁免錢糧,於民生並無妨礙。且土性咸鹵,難以種植。未便一律疏洩,以損田廬。」上諭曰:「實難宣洩,朕不怪汝。若避而為飾辭則不可。」八年,自陳「戇直致被人言。」上諭曰:「汝必欲以豐年為政效,水旱漠不關心。此奏殊屬客氣。」命來京,改授鑲藍旗滿洲副都統。授刑部侍郎,調吏部。

十二年,命往山西按治安邑、萬泉民亂,中途稱病,上責其逗遛,命解任。尋起授內閣侍讀學士,復擢兵部侍郎。十三年,調倉場侍郎,兼正紅旗滿洲副都統。迭署戶部侍郎、步軍統領。十八年,因疾解任。三十二年,卒。

晏斯盛,字虞際,江西新喻人。康熙五十九年,舉鄉試第一。六十年,成進士,改庶吉士。雍正元年,授檢討。五年,考選山西道御史。鑲紅旗巡役,以斯盛從騎驚突,拘辱之。斯盛以聞,命治罪。疏言:「各州縣立社倉,原以通濟豐歉。貧民借穀,石收息十升。如遇歉,當不取其息。」從之。九年,督貴州學政。遷鴻臚寺少卿。乾隆元年,擢安徽布政使。奏言:「各省水旱災,督撫題報,應即遴員發倉穀治賑,仍於四十五日限內題明應否加賑。其當免錢糧,將丁銀統入地糧覈算,限兩月題報。或分年帶徵,或按分蠲免,請旨遵行。」三年,疏言:「安徽被災州縣,倉儲不敷賑糶,請留未被災州縣漕米備賑。」四年,奏言:「江北向多游食之人,每遇歉歲,輕去其鄉。惟寓賑於工,人必爭趨。鳳陽、潁州以睢水為經,廬州以巢湖為緯,六安、滁、泗舊有堤堰,請援淮、揚水利例,動帑修濬。」皆從之。

七年,擢山東巡撫。山東有老瓜賊,巡撫硃定元令汛兵巡大道。斯盛疏言:「賊情狡獪,大道巡嚴,必潛移僻路;或假僧道技流,伏匿村落。應令州縣督佐雜分地巡察。」又奏:「邪教惑民,莫如創立教會,陽修善事。此倡彼和,日傳日廣,大為風教之害。盡法深求,株連蔓延,恐生事端。請將創教授徒為首者如法捕治,被誘者薄懲,出首者免究。」上從其請。尋以萊州被水,請暫禁米出海。上諭曰:「此不過屬吏為一郡一邑之說,汝等封疆大吏,不可存遏糶之心。若無米可販,百姓自不運,何待汝等禁乎?」又言兗、沂等府州被水,而江南饑民復至,疏請無災州縣留養限五百人,有災州縣限二三百人,上命實力料理。八年,調湖北巡撫。九年,遷戶部侍郎,仍留任。

斯盛究心民事,屢陳救濟民食諸疏,以社倉保甲相為經緯,因言:「周禮族師、遂人之法,稽其實則井田為之經。蓋就相生相養之地,而行政教法令於其中。是以習其事而不覺,久於其道而不變。周衰,管子作軌里連鄉,小治而未大效。秦、漢、隋、唐,龐雜無紀。宋熙寧中,編閭里之戶為保甲,事本近古,然亦第相保相受,而未得其相生相養之經。臣前奏推廣社倉之法,請按堡設倉,使人有所恃,安土重遷,保甲聯比,相為經緯。顧欲各堡一倉,倉積穀三千,一時既有難行;而入穀之數,則變通於額賦之中,別分本折,稍覺紛更。雖然,社倉保甲,原有相通之理,亦有兼及之勢。求備誠難,試行或易。加意倉儲,既慮貴糶妨民,停止採買,又慮積貯無資。詳加酌劑,擬請停戶部捐銀之例,令各省捐監於本地交納本色,以本地之穀實本地之倉,備本地之用。不採買而倉儲自充,誠為兼濟之道。竊謂常平之積便於城,未甚便於鄉。城積多,則責之也專,而無能之吏或以為累;鄉積多,則守之者眾,而當社之民可以分勞。且社倉未有實際,以倉費無所出也。名有社倉,而倉不在社,社實無倉,往往然矣。今捐穀多在於鄉,而例又議有倉費。擬請將此項捐納移入社倉,捐多則倉亦多。取鄉保穀數而約舉之,大州縣八十堡,四堡一倉,倉一千二百五十石,總二萬五千石,中小州縣,以此類推。儲蓄之方,莫便於此。方今治平日久,一甲中不少良善,四堡之倉,輪推甲長遞管,互相稽覈,年清年款。則社長累弊自除,而官考其成,隱然有上下相維之勢矣。」奏入,上嘉納之。

十年,進喜雨詩四章,用其韻賜答。京師錢貴,上令廷臣議平巿值,下各督撫仿行。斯盛疏請視京師例,禁民間銅鋪毀錢;又令州縣每歲秋以平糶錢巿穀。時設局令商民以銀平易,又疏請捕私錢,並禁民私剪錢緣,兼限民間用銀二三兩以上、糶米二三石以上,皆不得以錢準銀,下廷臣議行。尋以母老請終養回籍。十七年,卒。

斯盛著楚蒙山房易經解,唐鑒稱其「不廢象數而無技術曲說,不廢義理而無心性空談,在近日易家猶為篤實近理」云。

瑚寶,伊爾庫勒氏,滿洲鑲白旗人。雍正五年武進士,授三等侍衛。補陜西永興堡守備。八年,準噶爾二萬餘犯科什圖卡倫,從總兵樊廷進剿,遇於尖山,獲駝九十。又進敗之於北山,又遇於烏素達阪,擊之退。翌日,分七隊迎戰,瑚寶督兵奮擊,自辰至申,至科什圖,殪敵無算。敵圍峨侖磯,瑚寶赴援,乘夜來襲,領先鋒轉戰雪中七晝夜,奪波羅磚並白墩、紅山、鏡兒泉諸地,得其渠六,敵潰遁。九年,準噶爾復犯吐魯番,瑚寶從廷進剿,以勞賜白金三百。累遷肅州鎮右營游擊。

高宗即位,復累遷山西大同總兵,賜孔雀翎。乾隆十二年,遷固原提督。上諭之曰:「固原兵驕縱,犯上不法。瑚寶當加意整飭,使兵知畏法,漸次轉移。」又諭之曰:「固原城內外兵多民少,回民過半,私立掌教等名。應時時體訪,期杜釁端。回人充標兵,應留意分別:豪悍者懲黜,怯弱者淘汰,使營伍肅清。」旋疏請營兵具互結,以弓箭、鳥槍、技藝三項輪操;冬季借支春餉,次年四季扣除。下部議行。師征金川,調固原步兵二千。瑚寶請馱載軍裝,以二騾代三馬,可省費三分一,從之。

十三年,署甘肅巡撫,兼辦總督。奏言:「陜西歉收,師行採買草料為難。將甘肅倉貯豆石撥用,俟兵過照買還倉。」上以通融協濟,有益軍需,溫諭嘉勉。召授兵部尚書。尋署陜甘總督,調湖廣。又改授漕運總督。坐失察盧魯生偽造奏稿事,奪官,仍留任。尋卒,謚恭恪。

衛哲治,字我愚,河南濟源人。雍正七年,以拔貢生廷試優等,發江南委用。初署贛榆知縣,調鹽城。值蝗災,設六條拊循:優禮德望,饋餉高年,旌獎孝義,經理煢獨,譏警游惰,約束過犯。縣北有司河,匯上游七縣水入海。夏旱水弱,海潮至,咸苦不可食,甚乃浸溢民田;秋水盛,又患河寬流緩,入海不速。哲治建閘立斗門,蓄洩有備。斥鹵化膏腴,歲有涸出地,給無業民承耕。田沉沒而糧未除者,悉請豁免。循海築土墩九十餘,潮大,漁者得就墩逃溺,號「救命墩」。乾隆二年,補長洲,兼攝吳縣。請豁坍荒逋賦十餘萬。八年,遷海州知州。歲歉治賑,全活二十萬人,流民有自山東就食者。擢淮安知府。十年,河決陳家堡,漂溺男女、田廬無算。哲治遣小舟載餅餌救之,躬涉風濤,往來存問。山東復災,流民南下。哲治捐俸,益以勸募,葺草屋,自清江浦屬魚溝以北,銜接二百餘里,所在給粥糜、衣、藥。十三年,山東又災,兩江總督尹繼善令哲治運賑米至臺莊。上聞哲治善治賑,調山東登萊青道。居數月,擢布政使。

十四年,授安徽巡撫。奏言:「歙縣馬田地在休寧,請折徵充餉。」又言:「廣德催糧,每圖有單頭,數圖有經催。前巡撫潘思矩改行順莊,轉有未便,請得仍舊。」皆下部議行。旋召詣京師。十五年,令回任,上諭之曰:「汝不滿朕意。以一時無人,故仍留汝。宜奮勉改過。」調廣西。入覲,哲治具言親老不便迎養,命仍留安徽。尋丁憂。十八年,服闋,署兵部侍郎,暫管戶部事。復授安徽巡撫。疏建歙縣惠濟倉。再調廣西。二十年,內擢工部尚書。因病乞回籍。二十一年,卒。

蘇昌,伊爾根覺羅氏,滿洲正藍旗人,滿丕孫。康熙五十九年,自監生考取內閣中書,遷侍讀。考選浙江道御史。乾隆元年,命巡察吉林。奏言:「船廠、寧古塔、三姓、白都訥、阿爾楚喀等處滿官不知律例,訟案稽延累民,請自京師遣官往理。」三年,轉禮科給事中。屢擢至奉天府尹。十一年,奉天被水,蘇昌請設廠四鄉,增辦賑官吏公費;又請禁止游民往來奉天等處。

十四年,擢廣東巡撫。十六年,署兩廣總督。廣西巡撫舒輅請於思陵土州沿邊種竻竹,杜私越;土目因以侵夷地致釁。蘇昌奏:「鎮安、太平、南寧等沿邊二千餘里,無論種竹難遍。料理稍疏,事端轉啟,請更正。」上責舒輅輕率,寢其事。蘇昌奏:「瓊州海外瘠區,貧民生計艱難,有可墾荒地二百五十餘頃,請招民開墾,免其升科。」從之。召來京。十九年,授吏部侍郎。

二十四年,署工部尚書,授湖廣總督。在籍御史孫紹基稱與按察使沈作朋舊為同官,因以取賕。蘇昌劾奏抵罪,並請定回籍之員與有司交結處分。蘇昌劾湖北巡撫周琬乖張掩飾,上調蘇昌兩廣,命繼任總督愛必達察琬。愛必達發琬匿災徇劣吏狀,奪官,戍巴里坤。蘇昌至廣東,又劾碣石總兵王陳榮貪黷,奪官,論如律。加蘇昌太子太保。二十九年,奏言:「廣東產米不敷民食,宜多貯社穀,以補常平不足。請嗣後息穀統存州縣備賑,免其變價。」從之。

調閩浙總督。在兩廣薦鹽運使王概,概以贓敗,下吏議。御史羅暹春因劾蘇昌瞻徇糊塗,不堪節制海疆。上曰:「蘇昌不能辭失察之咎。節制海疆,乃朕所簡用,非御史所宜言。」蘇昌別疏劾知縣劉紹汜,下刑部。上以暹春與紹汜同為江西人,疑暹春劾蘇昌為紹汜地,詰責暹春,改主事;命蘇昌留任。三十年,臺灣淡水生番為亂,焚鱟殼莊,民死者五十餘。蘇昌檄按察使餘文儀會臺灣總兵督兵討平之。三十三年,入覲。卒,謚恪勤。子富綱,官云貴總督。

蘇昌在兩廣,有巨室橫斃人母,誣其子,獄久具,勾決本已下。蘇昌疑其冤,親鞫之,得實,疏自劾,上獎諭之,寘知縣於法,時論稱焉。

鶴年,字芝仙,伊爾根覺羅氏,滿洲鑲藍旗人。父春山,康熙五十一年進士,選庶吉士,官至盛京兵部侍郎。

鶴年,乾隆元年進士,選庶吉士,授檢討,兼公中佐領。三遷內閣學士。十五年,擢倉場侍郎。以京師米貴,疏請京、通俸餉米先半月支放。十八年,劾坐糧郎中綽克托剛愎自用,遲延徇縱,綽克托坐奪官。又奏:「通州南倉建自明天順間,後並入中倉。雍正間,復分為二,與西倉分貯漕白米。臣見中西倉足敷收貯,請裁南倉歸並中西倉。」從之。

十九年,授廣東巡撫。奏陳平米價,嚴保甲,緝竊盜案,禁私鑄、私雕諸事。上諭曰:「諸凡行之以實,持之以久。勉之!」尋復疏請以化州石城官租穀碾給海安營兵米。又奏海陽蔡家園土堤改築灰墻,出俸倡修。二十一年,奏言:「番禺、花、陽春諸縣徵收兵米,有所謂廚房米、官眷米,相傳起於明代籓府。後為旗營武職俸米,凡萬二千餘石,必細長潔白,產少價昂,甚為民累,應請禁革。」上嘉之。

調山東巡撫。奏言濟寧、魚臺、金鄉、滕、嶧諸州縣積水為災,上命加意賑恤。二十二年,上南巡,迎蹕。奏言:「海豐地處海濱,東北鄉尤低下,易罹水患。積年逋賦請豁免,乾隆十一年至二十年舊欠並改用下則。」復奏濟寧等五州縣積水尚未盡涸。上以江南宿虹、靈壁,河南永城、夏邑,皆有積水,命侍郎裘曰修會諸督撫籌度疏消。

七月,擢兩廣總督。奏:「東省水患頻仍,正與裘曰修商度,擬濬伊家河,洩微山湖水。河自韓莊迤西至江南梁旺城入運河,計程七十里,需銀十三四萬,一切正須經理。又與河臣張師載商濬運河,並及建堤。事不容已,懇留任督辦。」上諭曰:「覽奏,具見良心。然朕以無人,不得不用汝。汝仍遵前命。」

十月,復命以總督銜管山東巡撫事,綜理工程。奏言:「濬運河必先濬伊家河以洩積水,使久淹地畝漸次涸出,然後履勘估修,庶工實費省。請俟春暖鳩工,不致有誤新運。」又偕師載疏言:「運河淤墊日甚,尋常修濬,非經久之策。應自濟寧石佛閘起北至臨清閘,逐一探底,以深八尺為度,俾河身一體平坦。」上韙其言。十二月,伊家河工竟。又奏言:「運河淤淺處分段築壩,測量纖路,多民居。草土屋原售,給價拆除;瓦屋不原售,量將纖路加寬。被水民田速為疏消,俾為種麥;應修橋梁,察有解江餘石應用,不使估報買採。」上以「實心經理,不負任使」嘉之。尋卒,贈太子太保、兵部尚書銜,祀賢良祠,賜祭葬,謚文勤。子桂林,自有傳。

吳達善,字雨民,瓜爾佳氏,滿洲正紅旗人,陜西駐防。乾隆元年進士,授戶部主事。累擢至工部侍郎、鑲紅旗滿洲副都統。二十年,授甘肅巡撫。赴巴里坤督理軍需,以勞賜孔雀翎。二十二年,疏言:「軍糧自肅州運哈密至軍,石需費十二、三兩。凱旋官兵糶口糧制衣履,請改二成本色,八成折價。既得隨時支用,亦可稍省運費。」從之。加太子少保。

二十四年,代黃廷桂為陜甘總督,尋復以命楊應琚,改總督銜管巡撫事。奏言:「寧夏橫城堡河漲城圮。相度水勢,分別添築草壩,俾大溜北注,化險為平。」旋以總督銜調河南巡撫。奏改延津、封丘、胙城、滎澤、盧氏、靈寶諸縣營制,議行。

授雲貴總督。二十七年,奏言:「雲南、貴州各鎮協營每兵千設藤牌兵百,少不適用。請以七成改習鳥槍,三成改習弓箭。」從之。尋兼署云南巡撫。二十九年,奏改都勻、銅仁二協營制。調湖廣總督,兼署湖北巡撫。巴陵民熊正朝偽稱縣人巡撫方顯子,居省城與紳士交結,乘間盜竊,捕得寘諸法。

三十一年,調陜甘總督,奏言:「木壘地廣土沃。請將招集戶民編里分甲,里選里長,百戶選渠長,鄉約保正。訟獄,守備審理;命盜案,守備驗訊。巴里坤同知審解。」從之。三十三年,復調湖廣總督,兼署荊州將軍。命赴貴州,偕內閣學士富察善、侍郎錢維城按巡撫良卿、按察使高積營私骫法,論如律。三十五年,兼署湖南巡撫。

三十六年,復調陜甘總督,值土爾扈特部內附,上命分賚羊及皮衣。吳達善料理周妥,上嘉其能。以病乞解任。尋卒,贈太子太保,祀賢良祠,賜祭葬,謚勤毅。

崔應階,字吉升,湖北江夏人。父相國,官浙江處州鎮總兵。應階,廕生。初授順天府通判,遷西路同知。雍正中,擢山西汾州知府。乾隆十五年,授河南驛鹽道。擢安徽按察使。丁母憂,服闋,補貴州按察使。二十一年,擢湖南布政使,署巡撫。總督碩色劾應階子甘肅東樂知縣琇附驛寄家書,應階不檢舉,上特命降調。二十二年,補江南常鎮揚道。再遷山東布政使。

二十八年,遷貴州巡撫,調山東。疏請濬荊山橋舊河,洩積水。二十九年,疏言:「武城運河東岸牛蹄窩、祝官屯,西岸蔡河陂水匯注,俱為堤隔,浸灌民田,請各建閘啟閉。」均如所議。三十一年,疏言:「各州縣民壯有名無實,飭屬汰老弱,選精壯,改習鳥槍,與營伍無二。不增糧餉,省得精壯三千三百餘名。」得旨嘉獎。三十二年,疏言:「武定濱海,屢有水患:一在徒駭尾閭不暢,一在鉤盤淤塞未開。徒駭上游寬百餘丈,至霑化入海處僅十餘丈,紆回曲折,歸海遲延。徒駭舊有漫口,徑二十五里,寬至四五十丈,水漲賴以宣洩。若就此開濬,庶歸海得以迅速。又有八方泊為眾水所匯,伏秋霖雨,下游阻滯,淹及民田。泊東北為古鉤盤河,經一百三十餘里,久成湮廢。若就此開濬,引水入海,則上游不致停蓄,積水亦可順流而下。」皆如所請。

調福建,三十三年,擢閩浙總督,加太子太保。三十四年,劾興泉永道蔡琛貪鄙,論如律。調漕運總督,奏糧道專司漕務,無地方之責,令親押赴淮,不得轉委丞倅。召授刑部尚書,調左都御史。四十五年,以原品休致。尋卒。

王檢,字思及,山東福山人。父趯,官太常寺卿。檢,雍正十一年進士,改庶吉士。乾隆元年,授編修。大考四等,休致。十三年,上幸闕里,召試,復授編修。十四年,授直隸河間知府,遷甘肅涼莊道。以官河間有政聲,即調直隸霸昌道。累擢安徽按察使。奏:「外任官員眷屬外,定例州縣家人二十名,府道以上遞加十名,違者降級。定額本寬,近則州縣一署幾至百人,毋論招搖滋弊,即養廉亦不足供,請申明定例,違數詳參。」又奏:「皖城濱臨大江,歲多劫案,請加重沿江乘危搶奪舊例,邊海有犯視此。」均得旨允行。調直隸,又調山西。二十八年,遷廣西布政使,調甘肅。奏:「各省大計舉劾,例由籓司主稿。請嗣後籓司新任,得援督撫例展限三月,以重考核。」

二十九年,擢湖北巡撫,署湖廣總督。以前巡撫愛必達請於沔陽新堤設文泉縣治,地處低窪,城倉庫獄俱未興工,且於民情未便,奏請裁撤,移沔陽州同駐新堤,下部議行。

調廣東巡撫。秋審,刑部進湖廣招冊,檢所定擬,多自緩決改情實,或改可矜。上覈刑部九卿所改皆允,諭檢「秋讞大典,宜詳慎持平,失出失入,厥過維均」,傳旨申飭。三十一年,奏:「凡盜出洋肆劫,夥黨、器械,招買皆自內地。如果保甲嚴查,豈能藏匿?請嗣後洋盜案發,詢明由某地出口,將專管及兼轄、統轄各員,照保甲不實力例議處。」從之。廣東有名竹洲艇者,其制上寬下銳,行駛極速。海盜用以行劫,追捕為難。檢令凡船皆改平底。瓊州地懸海外,黎人那隆等劫商骫法,為諸盜最。檢親督剿捕,決遣如律。又以民多聚族而居,置祭田名曰「嘗租」,租穀饒裕,每用以糾眾械斗。奏請「嘗租自百畝以上者,留供每年祭祀,餘田歸本人。其以租利所置,按支均派,俾貧民有田以資生,兇徒無財以滋事」。上諭曰:「所奏意在懲兇息訟,惟恐有司奉行不善,族戶賢否不齊,難免侵漁攘奪。嗣後因恃祠產豐厚,糾眾械斗,按律懲治。即以祠田如檢所請分給族人,俾兇徒知所警懼,而守分善良仍得保其世業。」三十二年,因病請假,有詔慰問。旋卒。

子啟緒,自編修官河南開歸陳許道;燕緒,自編修官侍講;孫慶長,內閣中書,官福建按察使。

吳士功,字惟亮,河南光州人。雍正十一年進士,選庶吉士,改吏部主事。累遷郎中,考選御史。奏言:「部院大臣簡用督撫,調所屬司員以道府題補,恐滋偏聽、交結諸弊,請照雍正舊例停止。」從之。御史仲永檀言密奏留中,近多洩漏。敕王大臣詰問,舉士功劾尚書史貽直疏以對。上出士功疏,戒以不悛改,當重譴。乾隆七年,授山東濟東泰武道,丁憂,服闋,調直隸大名道。改山東兗沂曹道,屬縣饑,上南巡,迎駕,召對,以聞。為截留糧米六十萬石賑之,命士功董其事。旱蝗為災,督吏捕治,晝夜巡閱,未及旬,蝗盡。調湖南糧道,巡撫阿克敦疏留,調山東糧道。再遷湖北按察使。二十二年,護巡撫。河南饑,敕湖北發毗連州縣倉米運河南,即留本年應運漕糧歸倉。士功奏湖北地卑濕,米難久貯,請以一米改收二穀還倉,報聞。

遷陜西布政使,護巡撫。疏言:「宜君、榆林、葭州、懷遠、府谷、神木、靖邊、寧遠諸州縣先旱後潦。撥寧夏米麥五萬石分賑懷遠、靖邊諸縣,中阻黃河,河冰即難挽運,臣飭先期速運;撥綏德等四州縣米二萬石協濟榆林、葭州,山路崎嶇,臣飭添雇騾駝速運,俾民早霑實惠。」諭令竭力妥為之。調直隸,奏請:「撫籓離任,將庫項有無虧空奏明。新任撫籓亦於交代限內另摺奏聞,仍照例出結保題,以除挪借積弊。」上以所奏簡而易行,命著為例。二十三年,復調陜西,護巡撫。疏言:「延安府兵米,各縣運府倉。弁兵赴府支領,路遠費倍,耗損過半。請甘泉、宜川、延川、延長四縣本縣徵收支給。」又奏:「隴州汧陽縣跬步皆山,歲徵屯豆,請改折色解司充餉。」俱從之。

擢福建巡撫。二十四年,奏請捕私鑄,按錢數多寡治罪。又奏獲南洲盜八十餘人,與總督楊廷璋疏請改定南洲塘汛。又奏:「福建九府二州,常平缺額穀三十一萬石有奇;臺灣積年平糶未買穀十五萬石有奇:皆令補足。浙西歉收,請撥臺灣穀十萬石聽浙商販運。風汛不便,先發內地沿海府縣倉穀撥給,俟臺灣穀運到還倉。一轉移間,無妨於閩,有益於浙。」上嘉之。二十五年,奏:「寄居臺灣皆閩、粵濱海之民,乾隆十二年復禁止移眷,民多冒險偷渡,內外人民皆朝廷赤子。向之在臺灣為匪者,均隻身無賴。若既報墾立業,必顧惜身家,各思保聚。有的屬在內地者,請許報官給照,遷徙完聚。」又條奏稽查濱海漁船,令取船主、澳甲保結;出口逾期不還,責成澳甲、船主查報;稽察攜帶多貨,帆檣編字號,書姓名,免匪舟溷跡:均從之。尋以福建民多械斗,由大族欺凌小族,疏請大戶恃強糾眾擬情實,小戶被欺抵御擬緩決。刑部擬駁,上諭曰:「福建械斗最為惡俗。士功乃欲以族大族小分立科條,是使械鬥者得以趨避其詞,司讞者因而高下其手。士功夙習沽名,宜刻自提撕,勿自貽伊戚!」

二十六年,廷璋劾提督馬龍圖挪用存營公項,命士功嚴讞。會奏龍圖借用公項,已於盤查時歸補,援自首例減等擬徒。上以龍圖敗露後始行歸補,且將登記數簿焚毀,又增舞文之罪,不得以自首論,因究詰出何人意,尋覆奏士功主政。上奪士功官,發巴里坤效力自贖。二十七年,廷璋奏閩縣民楊魁等假造敕書承襲世職,投撫標效力。上命巴里坤辦事大臣詰責士功,並令自揣應得處分,贖罪自效。士功輸銀贖罪,命釋回。旋卒。

子玉綸,二十六年進士,自檢討累遷兵部侍郎,督福建學政,復降授檢討。

論曰:疆政首重宜民。紀督鑿井,反貽怨讟。喀爾吉善遂阻開礦、種樹之議,興利誠不易言也。雅爾圖、應階治水,斯盛治社倉,哲治治賑,才有洪纖,效有巨細,要皆有益於民。蘇昌劾大吏,頗見風力,瑚寶等亦各有建樹。自古未有不盡心民事而可以稱善治者也。


\end{pinyinscope}