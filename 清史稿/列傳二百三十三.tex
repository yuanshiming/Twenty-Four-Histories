\article{列傳二百三十三}

\begin{pinyinscope}
郭嵩燾弟昆燾崇厚曾紀澤薛福成黎庶昌馬建忠

李鳳苞洪鈞劉瑞芬徐壽朋楊儒

郭嵩燾,字筠仙,湖南湘陰人。道光二十七年進士,選庶吉士,遭憂歸。會粵寇犯長沙,曾國籓奉詔治軍,嵩燾力贊之出。贛事亟,江忠源乞師國籓,國籓遣之往,從忠源守章門。是時寇艎集饒、瑞,分泊長江,因獻編練水師議,忠源韙之,令具疏請敕湖南北、四川制戰艦百餘艘。嗣以贛被圍久,船非可剋期造,乃先造巨筏,列砲其上,與陸師夾擊,寇引去。厥後用以塞湖口者即此筏也。湘軍名大顯。論功,授編修。還朝,入直上書房。咸豐九年,英人犯津沽,僧格林沁撤北塘備,嵩燾力爭之,議不合,辭去。

同治改元,起授蘇松糧儲道,遷兩淮鹽運使。庫儲竭,諸軍仰餔淮鹺者數十萬,嵩燾躬自掣驗,配置各營。提督李世忠擁重兵行私鹺,亡誰何,益遣入捕治之,運政乃鬯。明年,署廣東巡撫。寇偪陽山,亟使張運蘭擊卻之。詔安陷,饒平、大埔警,與總督瑞麟遣將防邊,追入詔安城,殺數千人,軍稍振。是時金陵克,罷釐捐議起,嵩燾陳說利害凡千餘言,事遂寢。偽森王侯玉山避匿香港,恃英為護符,官吏莫能捕。嵩燾援公法與爭,執以歸,論斬。而瑞麟遽張其功,以率兵往捕聞,嵩燾力止之,不可。英人大恚,數移牒詰責。

初,毛鴻賓督粵,事皆決於幕僚徐灝。瑞麟繼至,灝益橫。嵩壽銜之,上疏論軍情數誤,劾逐灝,並自請罷斥。事下左宗棠,宗棠言其跡近負氣,被訶責。左、郭本姻家。宗棠先厄於官文,罪不測,嵩燾為求解肅順,並言於同列潘祖廕,白無他,始獲免,至是宗棠竟不為疏辨。嵩燾念事皆繇督撫同城所誤,逾歲解職,遂上疏極論其弊,不報。

光緒元年,授福建按察使,未上,命直總署。擢兵部侍郎、出使英國大臣,兼使法。英人馬加理入滇邊遇害,嵩燾疏劾岑毓英,意在朝廷自罷其職,藉箝外人口也。而一時士論大譁,謂嵩燾媚外。嵩燾言既不用,英使威妥瑪出都,邦交幾裂。嵩燾又欲以身任之,上言:「交涉之方,不外理、勢。勢者人與我共,可者與,不可者拒。理者所以自處。勢足而理直,固不可違;勢不足而別無可恃,尤恃理以折。」因條列四事以進。而郎中劉錫鴻者,方謀隨嵩燾出使,慮疏上觸忌,遏之,比嵩燾覺,始補上,而事已無及。既蒞英,錫鴻為副使,益事事齮齕之,嵩燾不能堪,乞病歸,主講城南書院。

未幾,而俄事棘。崇厚以辱國論死,群臣多主戰,徵調騷然。嵩燾於是條上六事:曰收還伊犁,歸甘督覈議;曰遣使議還伊犁,當赴伊會辦;曰直截議駁,暫聽俄人駐師;曰駐英、法公使不宜遣使俄;曰議定崇厚罪名,當稍準萬國公法;曰廷臣主戰,止一隅見,當斟酌情理之平。上嘉其見確。已而召曾紀澤使俄,卒改約。

嵩燾雖家居,然頗關心君國。朝鮮亂作,法越釁開,皆有所論列。逮馬江敗,恭親王奕等去位,言路持政府益亟,嵩燾獨憂之。嘗言:「宋以來士夫好名,致誤人家國事。託攘外美名,圖不次峻擢;洎事任屬,變故興,遷就倉皇,周章失措。生心害政,莫斯為甚!」是疏傳於外,時議咸斥之。及庚子禍作,其言始大驗,而嵩燾已於十七年卒矣。箸有禮記質疑四十九卷,大學中庸質疑三卷,訂正家禮六卷,周易釋例四卷,毛詩約義二卷,綏邊徵實二十四卷,詩文集若干卷。

其弟昆燾,字意城。以舉人參張亮基戎幕,與宗棠俱。李開方擾湖北,自懷慶折而南,武昌夜半得報,亟調師會鵝公頸。驟遇寇,寇出不意,大擾亂,遂斬開方,殲其軍。報至,亮基始知之,昆燾恆以是自喜。駱秉章撫湘,昆燾從國籓東征,宗棠援浙,軍資並倚之。由國子監助教歷加四品卿。後劉昆討黔苗,昆燾久引疾歸,力起贊軍事。苗將平,又辭去。光緒八年,卒。

崇厚,字地山,完顏氏,內務府鑲黃旗人,河督麟慶子。道光二十九年舉人。選知階州,歷遷長蘆鹽運使。咸豐十年,署鹽政,疏請停領餘引,代銷滯引,依永平低價。會僧格林沁治畿輔水田,又勸墾葛沽、鹽水沽沃鹵地四千二百餘畝。明年,充三口通商大臣。又明年,遷大理寺卿,仍留津與英、法重修租界條約。同治改元,以兵部侍郎參直隸軍事,尋署總督。時葡萄牙遣使入京乞換約,崇厚牒請總署摽勿受。法使哥士耆緩頰,治蒞津,朝命崇厚承其事。次年,諭遏冀州竄匪,坐失機,被責。已而丹使踵葡例,拒如初。復命為全權大臣,訂約五十五條,通商章程九款。自是而荷、而日、而比、而意、而奧,皆遣使求取,並為延款,語具邦交志。復建議設北洋機器局城南分局,城堞砲臺與郡城遙相峙。五年,貸款墾海河北岸,首邢家沽訖臥河村,中洩為渠,闢稻田可五百頃,手訂試墾章程,於是兩岸為沃野。九年,津郡民、教失和,被議。事寧,朝廷遣使修好,命充出使法國大臣,是為專使一國之始,然事畢即返。歷署戶部、吏部侍郎。

光緒二年,署奉天將軍,疏請擇地設官,置寬甸、懷仁、通化三縣,增邊關兵備道,升昌圖為府,改八家鎮為縣,徙經歷駐康家屯,改梨樹城為,徙照磨駐八面城;其通判、知縣並加理事同知銜,兼治蒙民,議行。先後疏論吉林積弊,請辦馬賊,懲聚博,清積訟,覈荒地,除金匪。又以私墾圍場者眾,為懇寬其既往,已墾者量丈升科,未墾者擇地安插,仍留隙地以講武,稱旨。

四年,俄界回寇擾邊,與其外部格爾斯合力禁止。其秋,授出使俄國大臣,加內大臣銜,晉左都御史。明年,赴俄。初,左宗棠進兵伊犁,乘俄土戰爭,要俄人退去庫爾札,俄人多所挾求。至是,崇厚抵利伐第亞謁俄皇達使命,貿然與訂和約:一,自嘉峪關逕西安、漢中達漢口,俄有通商權;一,自松花江至伯都訥,貿易自由;一,自蒙古及天山南北輸入商品,不課稅金;一,自西伯利亞至張家口,歸俄敷設鐵道;一,自陜甘至漢口,既榷常稅,其雜稅概免;一,嘉峪關、科布多、哈密、吐魯番、烏魯木齊、庫車置領事官;一,凡俄國臣民旅華,許攜銃器;一,伊犁城及旁近地,凡俄所有土地及建築物,不在還付例。約成,朝野譁然,於是修撰王仁堪、洗馬張之洞等交章論劾。上大怒,下崇厚獄,定斬監候,以徇俄人請,貸死,仍羈禁。更遣曾紀澤往俄更約,爭回伊犁南路七百餘里,嘉峪關諸地緩置官。

十年,崇厚輸銀三十萬濟軍,釋歸。遇太后五旬萬壽,隨班祝嘏,朝旨依原官降二級,賞給職銜。十九年,卒,年六十有七。

曾紀澤,字劼剛,大學士國籓子。少負俊才。以廕補戶部員外郎。父憂服除,襲侯爵。光緒四年,充出使英法大臣,補太常寺少卿,轉大理寺。六年,使俄大臣崇厚獲罪去,以紀澤兼之。

先是俄乘我內亂,據伊犁,及回部平,乃舉以還我,議定界、通商。崇厚不請旨,遽署押,所定約多失權利,因詔紀澤兼使俄,議改前約。俄以崇厚罹大闢,怫甚。紀澤慮礙交涉,請貸崇厚死,上許之,論監禁。紀澤乃疏言:「伊犁一役,辦法有三:曰戰,守,和。言戰者,謂左宗棠等席全勝之勢,不難一戰。臣竊謂伊犁地形巖險,俄為強敵,非西陲比。兵戎一啟,後患滋長。東三省與俄毗連,根本重地,防不勝防。或欲游說歐邦,使相牽制,是特戰國之陳言耳。各邦雖外和內忌,而協以謀我則同,孰肯出而相助?言守者,則謂伊犁邊境,若多糜巨帑以獲之,是鶩荒遠、潰腹心也,不如棄而勿收。不知開國以來,經營西域者至矣。聖祖、世宗不憚勤天下力以征討之,至乾隆二十二年,伊犁底定,腹地始得安枕。今若棄之,如新疆何?說者謂姑紓吾力以俟後圖。不知左宗棠等軍,將召之使還乎?則經界未明,緩急何以應變?抑任其逍遙境上,則難於轉餉,銳氣坐銷。是今日之事,戰、守皆不足恃,仍不外言和。和亦有辦法三:曰分界,通商,償款其小者也。即通商亦較分界為輕。何以言之?西國定約之例,有常守不渝者,亦有隨時修改者。不渝者,分界是也。此益則彼損。是以定約之時,其難其慎。修改者,通商是也。若干年修改一次。條文之不善,商務之受損,正賴此修改之年可以換約,固非彼族所得專也。俄約經崇厚議定,俄君署押,今欲全數更換,勢所不能。臣愚以為分界既屬常守之局,必當堅持力爭。若通商各條,惟當去其太甚,其餘從權應允,俟諸異日之修改,庶和局可終保全。不然,事機決裂,必須聲罪致討,此戰之說也。廟堂勝算,固非使臣所敢議也。不然,暫置伊犁勿論,此守之說也。是邊界不可稍讓,而全境轉可盡捐,臣亦未敢以為是也。再不然,姑先為駁議,俟不得已時酌量允之,此和之說也。是乃市井售物嘗試之術,非所以敦信義、馭遠人也。蓋準駁貴有一定之計,勿致後日迫於事勢,復有後允之條。今臣至俄都,但言兩國和好,自應遣使通誠。至辨論公事,傳達語言,系使臣職分,俟接奉本國文牘,再行商議。如此立言,庶不至見拒鄰邦,貽國羞辱。臣駑下,唯有懍遵聖訓,不激不隨,冀收得尺得寸之功,稍維大局。」

及至俄,日與俄外部及駐華公使布策等反復辨論,凡數十萬言,十閱月而議始定。崇厚原約,僅得伊犁之半,巖險屬俄如故。紀澤爭回南境之烏宗島山、帖克斯川要隘,然後伊犁拱宸諸城足以自守,且得與喀什噶爾、阿克蘇諸城通行無阻。其他分界及通商條文,亦多所釐正焉。七年,遷宗人府府丞、左副都御史。秩滿,留任三載。

法越構釁,紀澤與法抗辯不稍屈,疏陳備御六策。十年,晉兵部侍郎。與英人議定洋藥稅釐,歲增銀六百餘萬。明年,還朝,轉入總理各國事務衙門。調戶部,兼署刑部、吏部各侍郎。十六年,卒,加太子少保,謚惠敏。子廣鑾,左副都御史;廣銓,兵部員外郎。

薛福成,字叔耘,江蘇無錫人。以副貢生參曾國籓戎幕,積勞至直隸州知州。光緒初元,下詔求言,福成上治平六策,又密議海防十事。時總稅務司赫德喜言事,總署議授為總海防司,福成上書力爭,乃止。八年,朝鮮亂,張樹聲代李鴻章督畿輔,聞變,將牒總署奏請發兵。福成慮緩則蹈琉球覆轍,請速發軍艦東渡援之。亂定,以功遷道員。

十年,授寧紹臺道。法蘭西敗盟,構兵越南,詔緣海戒嚴。寧波故浙東要衢也,方是時,提督歐陽利見頓金雞山,楊岐珍頓招寶山,總兵錢玉興分守要隘。諸將故等夷,不相統攝。巡撫劉秉璋檄福成綜營務,調護諸將,築長墻,釘叢舂,造電線,清間諜,絕鄉導與窺伺。其南洋援臺三艦為法人追襲,駛入鎮海口,復令其合力守御。謀甫定而寇氛逼矣,再至,再卻之,卒不得逞而去。十四年,除湖南按察使。

明年,改三品京堂,出使英法義比四國大臣,歷光祿、太常、大理寺卿,留使如故。未幾,坎巨提來乞師。坎故羈縻回部,自英滅克什米爾,遂為所屬。近且築路貫其境,坎拒之,戰弗勝,乃求援,朝旨使福成詰其故。福成晤英外部沙力斯伯里,詗知其防俄心切,遂與訂定會立坎酋,以釋嫌怨。因具選立本末以上,並陳英、俄互爭帕米爾狀,請趣俄分界,冀英隱助。已而被命集議滇緬界線、商務。先是曾紀澤使英,謀將南掌、拈人諸土司盡為我屬,議未決而歸。至是福成繼之,始變前規,稍拓邊界,訂定條約二十款,語具邦交志。

福成任使事數年,恆惓惓於保商,疏請除舊禁,廣招徠。其爭設南洋各島領事官,尤持正義,英人終亦從之。又以英、法教案牽涉既廣,條列治本治標機宜甚悉。其將歸也,復撮舉見聞上疏以陳,大恉謂宜厲人才,整戎備,濬利源,重使職,為棄短集長之策。二十二年,歸,至上海病卒,優詔賜恤。卒後半載,而中英訂附款,致將福成收回各地割棄泰半,論者惜之。

福成好為古文辭,演迆平易,曲盡事理,尤長於論事紀載。著有庸菴文編、筆記,海外文編,出使英法義比日記,浙東籌防錄。

黎庶昌,字蓴齋,貴州遵義人。少嗜讀,從鄭珍游,講求經世學。同治初元,星變,應詔上書論時政,條舉利病甚悉,上嘉之。以廩貢生授知縣,交曾國籓差序。國籓素重鄭氏,接庶昌延入幕,歷署吳江、青浦諸邑;兩筦榷關,稅驟進。光緒二年,郭嵩燾出使英國,調充參贊。歷比、瑞、葡、奧諸邦,箸書以撮所聞見,成西洋雜志。晉道員。

七年,命充出使日本大臣。值議琉球案及華商雜居事,其外部井上馨持甚堅,庶昌翻復辨論,卒如所議。明年,日本將襲朝鮮,庶昌電請速出援師為先發制人計。師至,日艦知有備,還,言歸於好。中國古籍,經戎燼後多散佚,日籓族★L3藏富,庶昌擇其足翼經史者,刊古逸叢書二十六種。中法易約,條列七事進。尋遭憂歸,服闋,仍故官。

十七年,除川東道。川俗故闇僿。既蒞事,設學堂,倡實業,建病院,整武恤商,百廢具舉。中東事起,庶昌曰:「日本蓄謀久矣,朝鮮猶其外府也。戰固難勝,讓亦啟侮。」乃倡布告列邦議,以維持屬國,願東渡排難,當事者弗納。及戰事殷,財詘,庶昌首輸萬金,請按職列等差,亦不報。二十一年,詔陛見。駐渝法領事聞其將去,留辦教案,代者多方困之。遘疾,遂去官。未幾,卒。川東民建祠溈郡祀之。

馬建忠,字眉叔,江蘇丹徒人。少好學,通經史。憤外患日深,乃專究西學,派赴西洋各國使館學習洋務。歷上書言借款、造路、創設海軍、通商、開礦、興學、儲材,北洋大臣李鴻章頗稱賞之,所議多採行。累保道員。光緒七年,鴻章遣建忠赴南洋與英人議鴉片專售事。建忠以鴉片流毒,中外騰謗,當寓禁於徵,不可專重稅收。時英人持正議者,亦以強開煙禁責其政府,引以為恥。聞建忠言,雖未能遽許,皆稱其公。

八年,朝鮮始與美國議約,鴻章奏派建忠往蒞盟。約成,英、法先後遣使至,建忠介之,皆如美例成約。日本駐朝公使屢詗結約事,建忠秘不使預聞,日人滋不悅。建忠歸而朝鮮亂作,庶昌以聞。時鴻章以憂去,張樹聲權北洋大臣,令建忠偕海軍提督丁汝昌率兵艦東渡觀變。建忠抵仁川,日本海軍已先至,建忠設辭緩之,而亟請速濟師代定亂。朝命提督吳長慶率三千人東援。建忠先定誘執首亂之策,偕長慶、汝昌往候大院君李昰應,減騶從,示坦率。及昰應來報謁,建忠遂執之,強納諸輿,交長慶夜達兵輪,而汝昌護送至天津。復擒亂黨,援朝鮮國王復其位。日使雖有言,而亂已定,亦無如何,皆建忠謀也。於是長慶統軍留駐,其隨員袁世凱始來佐營務。及建忠歸,而維新黨之亂又作。日軍先入,交涉屢失機,其後卒致全敗。建忠憤後繼失人,初謀盡毀,譔東行錄以記其事。

建忠博學,善古文辭;尤精歐文,自英、法現行文字以至希臘、拉丁古文,無不兼通。以泰西各國皆有學文程式之書,中文經籍雖皆有規矩隱寓其中,特無有為之比儗而揭示之,遂使學者論文困於句解,知其然而不能知其所以然。乃發憤創為文通一書,因西文已有之規矩,於經籍中求其所同所不同者,曲證繁引,以確知中文義例之所在,務令學者明所區別,而後施之於文,各得其當,不唯執筆學為古文詞有左宜右有之妙,即學泰西古今一切文學,亦不難精求而會通焉。書出,學者皆稱其精,推為古今特創之作。又著有適可齋記言、記行等書。

李鳳苞,字丹厓,江蘇崇明人。少聰慧,究心歷算之學,精測繪。丁日昌撫吳,知其才,資以貲為道員。歷辦江南制造局、吳淞砲臺工程局,繪地球全圖,並譯西洋諸書。日昌為船政大臣,調充總考工。朝議遣生徒出洋,加三品卿,派為監督。光緒三年,率赴英、法兩國,分置肆業。明年,賜二品頂戴,充出使德國大臣,旋兼使奧、義、荷三國,往來數千里,周旋各國間,聯絡邦交。時建議興海軍,並命督造戰艦。

十年,法越構釁,暫署法使。法事決裂,遂奉命回國,歸過澳門。澳門自明中葉久為葡萄牙人稅居,及是葡人私議欲攘為己有。鳳苞寓書部臣,乞請旨與葡人定約,免後患。部臣懼生事,寢其議。後一年,葡人遂據其地,論者惜之。既,覆命,有旨發往直隸交李鴻章差遣,令總辦營務處,兼管水師學堂。未幾,以在德造艦報銷不實,被議革職。十三年,卒。著有四裔編年表、西國政聞匯編、文藻齋詩文集等。其他音韻、地理、數學,皆有論著,未成。

洪鈞,字文卿,江蘇吳縣人。同治七年一甲一名進士,授修撰。出督湖北學政,歷典陜西、山東鄉試。遷侍讀,視學江西。光緒七年,歷遷內閣學士。母老乞終養,嗣丁憂,服闋,起故官。出使俄德奧比四國大臣,晉兵部左侍郎。初,喀什噶爾續勘西邊界約,中國圖學未精,乏善本。鈞蒞俄,以俄人所訂中俄界圖紅線均與界約符,私慮英先發,乃譯成漢字備不虞。十六年,使成,攜之歸,命直總理各國事務衙門。

值帕米爾爭界事起,大理寺少卿延茂謂鈞所譯地圖畫蘇滿諸卡置界外,致邊事日棘,乃痛劾其貽誤狀,事下總署察覆。總署同列諸臣以鈞所譯圖,本以備考覈,非以為左證,且非專為中俄交涉而設,安得歸咎於此圖?事白,而言者猶未息。右庶子準良建議,帕地圖說紛紜,宜求精確。於是鈞等具疏論列,謂:「內府輿圖、一統志圖紀載漏略。總署歷辦此案,證以李鴻章譯寄英圖,與許景澄集成英、俄、德、法全圖,無大紕繆,而覈諸準良所奏,則歧異甚多。欽定西域圖志敘霍爾干諸地,則總結之曰屬喀什噶爾;敘喇楚勒、葉什勒庫勒諸地,則總結之曰屬喀什噶爾西境外:文義明顯。原奏乃謂:『其曰境外者,大小和卓木舊境外也。曰屬者,屬今喀什噶爾,為國家自闢之壤地也。』語近穿鑿。喀地正北、東北毗俄七河,正西倚俄費爾干,其西南錯居者帕也。後藏極西曰阿里,西北循雪山逕挪格爾、坎巨提,訖印度克什米爾,無待北涉帕地。設俄欲躡喀,英欲偪阿里,不患無路。原奏乃謂:『二國侵奪拔達克山、安集延而終莫得通。』斯於邊情不亦闇乎?中俄分界,起科布多、塔爾巴哈臺、伊犁,訖喀西南烏仔別里山口止,並自東北以達西南。原奏乃謂:『當日勘界,自俄屬薩馬乾而東,實以烏仔別里西口為界。今斷以東口,大乖情勢。』案各城約無薩馬乾地名,惟浩罕、安集延極西有薩馬爾干,明史作撒馬兒罕,久隸俄,與我疆無涉。當日勘界,並非自西而東,亦無東西二口之說,不知原奏何以傳訛若此?謹繪許景澄所寄地圖以進。」並陳扼守蔥嶺及爭蘇滿有礙約章狀。

先是坎巨提之役,彼此爭惎其間,我是以有退兵撤卡之舉,英乘隙而使阿富汗據蘇滿。至是,俄西隊出與阿戰,東隊且駸駸偪邊境。總署復具籌辦西南邊外本末以上。鈞附言:「自譯中俄界圖,知烏仔別里以南,東西橫亙,皆是帕地。喀約所謂中國界線,應介乎其間。今日俄人爭帕,早種因喀城定約之年。劉錦棠添設蘇卡,意在拓邊。無如喀約具在,成事難說。唯依界圖南北經度斜線,自烏仔別里徑南,尚可得帕地少半,尋按故址,已稍廓張。俄阿交閧,揣阿必潰。俟俄退兵,可與議界,當更與疆臣合力經營,爭得一分即獲一分之益。」上皆嘉納。十九年,卒,予優恤。

鈞嗜學,通經史,嘗譔元史釋文證補,取材域外,時論稱之。

劉瑞芬,字芝田,安徽貴池人。以諸生從李鴻章軍援上海,檄主水陸軍械轉運。時初用西式槍砲,皆購自外洋,瑞芬考驗精審,應時解濟,淮軍遂以善用西洋利器名。累保道員,督辦松滬釐捐。光緒二年,權兩淮鹽運使。淮北薦饑,流民就食揚州,瑞芬築圩城外,構棚分宿,計口授食,所全活六萬餘人。旋授蘇松太道。租界以黃浦南北分華洋船埠,洋人時侵南岸。瑞芬丈量南北,中分為界,設水利局委員董其事,洋人亦就範焉。擢江西按察使,遷布政使。

十一年,改三品京堂,命充出使英俄等國大臣;授太常寺卿,遷大理寺,仍留使。改駐英、法、義、比。初,俄人覬覦漠河金礦,瑞芬亟達總理衙門,創議先自開辦。英既占緬甸,罷其朝貢,瑞芬執故事與爭,仍如舊。英復侵西藏,瑞芬力爭於其外部,追還印度入藏之師,乃別議藏印條約,事具邦交志。

瑞芬久事外交,有遠見。朝鮮亂初起,即上書言:「朝鮮毗連東三省,關系甚重。中國能收其全土改行省,上策也。次則當約英、美諸國共議保護,庶免強鄰獨占,存籓屬以固邊陲。」總署寢其議不行,其後果如所言。十五年,召授廣東巡撫。十八年,卒,恤如制。

子三。世珩,字聚卿。光緒二十年舉人。累至道員。歷辦江南商務官報、學務工程、湖北造幣等事。旋擢度支部參議,加三品卿。條議幣制,中外稱其精確,未及行而辛亥變起,遂歸寓上海。丙寅年,卒。嗜古,富藏書,校刊古籍尤精。有聚學軒叢書、貴池先哲遺書、玉海堂宋元槧本叢書及曲譜、曲品等。

徐壽朋,字進齋,直隸清苑人,本籍浙江紹興。以廩貢生納貲為主事。諳習外情,佐津海關辦交涉。光緒二年,以道員充美日使館二等參贊。時華人傭於洛士丙冷者多被虐殺,壽朋佐使臣鄭藻如索償,詞錚義屈。未竟,會開秘魯使館,移充駐秘參贊,攝行公使事。秘故虐遇華工,益苛其例,壽朋與秘廷辨論,多所補救。駐外久,辦理交涉,常服遠人。晉二品秩。還國,適李鴻章督畿輔,闢居幕府。疏薦其練吏治,熟邦交。召見,奏對稱旨。

二十四年,授安徽徽寧池太廣道,遷按察使。未半載,徵還,命以三品京堂充韓國全權議約大臣。既至,與其外部樸齊純議定商約十三條,語具邦交志。初,韓本為我屬國,貢獻不絕。自馬關定新約,認為獨立自主,遂以壽朋膺使命,是為中韓立約之始。其秋,除太僕寺卿。約成,改充出使韓國大臣。奏設漢城總領事,惠保僑民,始復自治權。二十六年,聯軍入京,鴻章被命議和,奏調壽朋佐議。壽朋習西國語言文字,徐起應付,卒能不失鴻章本意。逾歲,議定和約十二款。復力請回鑾。遷外務部左侍郎。尋病卒,予優恤。

楊儒,字子通,漢軍正紅旗入。以監生納貲為員外郎,銓兵部。舉同治六年鄉試。久之,出為常鎮道。母憂,服闋,除溫處道,調徽寧池太道。光緒十八年,改四品卿,出使美日秘三國大臣,補太常寺少卿。與英外部葛禮山續定華工條約。歷通政使副使、左副都御史,留使如故。二十二年,調使俄奧和三國。越二年,晉工部侍郎,仍駐俄。

二十六年,拳亂作,聯軍入津沽,電命儒遞國書,乞俄調解。京師陷,車駕幸西安。俄佯議撤兵,而潛使人詣關東,掠吉林、黑龍江地,達營口北。儒至黑海行宮與婉商,俄允還地,而不允撤保路兵。將軍增祺遽與訂密約九款,多失權利,上責其謬妄,下嚴旨,仍令儒與俄議。儒與商更約,俄堅拒,儒正色曰:「既言保我自主,何兵權、利權、命官權而不予畀?既稱不利土地,何以東三省不為中國版圖?」俄窮於應,始允別立正約。上聞而嘉之,授為全權大臣。

逾歲,俄交草約十二款,趣畫押。東南士民甚激昂,各國亦騰口舌,朝旨命再商改。儒責其外部食言,語激切,俄人勉為改數事,而仍未平準。儒數往謁,拒不見,見則第趣畫諾,語竟即起,不容儒致一詞。儒憤出,及階踣,傷右足,乞假赴德、奧療治。俄留之,且因其病篤,命駐華公使戢耳詩與李鴻章在京協定。儒復請代,不許。調戶部。明年正月,卒,予優恤。

論曰:中國遣使,始於光緒初。嵩燾首膺其選,論交涉獨具遠識。崇厚擅定俄約,誤國甚矣。紀澤繼之,抗議改正。其時國勢猶足自申焉。至儒爭密約,竟以憤死,終不能挽救,公理尚可恃乎?福成、庶昌諸人,並嫺文學,各有著述,討論修飾,皆美使才也。馬建忠定亂濟變,策奇制勝,亦有足多,故並附於篇。


\end{pinyinscope}