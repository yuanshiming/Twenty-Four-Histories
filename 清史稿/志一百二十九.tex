\article{志一百二十九}

\begin{pinyinscope}
○邦交二

△英吉利

英吉利在歐羅巴西北。清康熙三十七年置定海關,英人始來互市,然不能每歲至。雍正三年來粵東,所載皆黑鉛、番錢、羽緞、哆囉、嗶嘰諸物,未幾去。七年,始通市不絕。乾隆七年冬十一月,英巡船遭風,飄至廣東澳門,總督策楞令地方官給貲糧、修船舶遣之。二十年,來寧波互市。時英商船收定海港,運貨寧波,逾年遂增數舶。旋禁不許入浙,並禁絲斤出洋。二十四年,英商喀喇生、通事洪任輝欲赴寧波開港。既不得請,自海道入天津,仍乞通市寧波,並訐粵海關陋弊。七月,命福州將軍來粵按驗,得其與徽商汪聖儀交結狀,治聖儀罪,而下洪任輝於獄。旋釋之。二十七年夏五月,英商啗闌等以禁止絲斤,其貨艱於成造,仍求通市。粵督蘇昌以聞,許之,然仍限每船只許配買土絲五千,二蠶湖絲三千斤,至頭蠶湖絲及綢緞綾匹仍禁。

五十八年,英國王雅治遣使臣馬戛爾尼等來朝貢,表請派人駐京,及通市浙江寧波、珠山、天津、廣東等地,並求減關稅,不許。六十年,復入貢,表陳「天朝大將軍前年督兵至的密,英國曾發兵應援」。的密即廓爾喀也。奏入,敕書賜賚如例。

嘉慶七年春三月,英人窺澳門,以兵船六泊雞頸洋,粵督吉慶宣諭回國,至六月始去。十年春三月,英王雅治復遣其臣多林文附商船來粵獻方物。十三年秋九月,復謀襲澳門,以兵船護貨為詞,總督吳熊光屢諭使去,不聽,遂據澳,復以兵船闖入虎門,進泊黃埔。命剿辦絕市,褫熊光職,英人始於十月退師。明年春二月,增築澳門砲臺。夏五月,定廣東互市章程。十九年冬十一月,禁英人傳教。二十年春三月,申鴉片煙禁。

二十一年夏六月,英國遣其臣加拉威禮來粵東投書,言英太子攝政已歷四年,感念純皇帝聖恩,遣使來獻方物,循乾隆五十八年貢道,由海洋舟山至天津赴都,懇總督先奏。時總督蔣攸銛方入朝,巡撫董教增權督篆,許其晉見,援督撫大吏見暹邏諸國貢使禮,加拉威禮不受,再三議相見儀,教增不得已許之。其日總督及將軍、兩副都統、海關監督畢坐節堂,陳儀衛,加拉威禮上謁,免冠致敬,通事為達意,教增離坐起立相問答,允為入告,加拉威禮徑出。比教增奏入,而貢使羅爾美都、副貢使馬禮遜乘貢舟五,已達天津。帝命戶部尚書和世泰、工部尚書蘇楞額往天津,率長蘆鹽政廣惠伴貢使來京,一日夜馳至圓明園,車路顛簸,又衣裝皆落後。詰朝,帝升殿受朝會,時正使已病,副使言衣車未至,無朝服不能成禮,和世泰懼獲譴,詭奏二貢使皆病,遂卻其貢不納,遣廣惠伴押使臣回粵。初英貢使齎表,帝覽表文,抗若敵體,又理籓院迓接不如儀,帝故疑其慢,絕不與通。羅爾美都等既出都,有以實入告者,帝始知非貢使罪,復降諭錫賚,追及良鄉,酌收貢物,仍賜國王珍玩數事,並敕諭國王歸咎使臣不遵禮節謝宴,英使怏怏去。七月,降革蘇楞額、和世泰、廣惠等有差。

道光元年,復申鴉片煙禁。七年,廣東巡撫硃桂楨毀英商公局,以其侵占民地也。十三年,英罷商公司。西洋市廣東者十餘國皆散商,惟英有公司。公司與散商交惡,是年遂散公司,聽商自運,而第徵其稅。明年,粵督盧坤誤聽洋商言,以英公司雖散,而粵中不可無理洋務之人,遂奏請飭英仍派遣公司大班來粵管理貿易。英王乃遣領事律勞卑來粵。尋代以義律。義律議在粵設審判署,理各洋交涉訟事,其貿易仍聽散商自理。

十六年,定食鴉片煙罪。初,英自道光元年以後,私設貯煙大舶十餘隻,謂之「躉船」,又省城包買戶,謂之「窯口」。由窯口兌價銀於英館,由英館給票單至躉船取貨。有來往護艇,名曰「快蟹」,砲械畢具。太常寺卿許乃濟見銀輸出歲千餘萬,奏請弛煙禁,令英商仍照藥材納稅,入關交行後,只許以貨易貨,不得用銀購買,以示限制。已報可,旋因疆臣奏請嚴販賣吸食罪名,加重至死,而私販私吸如故。十八年,鴻臚寺卿黃爵滋請嚴吸食罪,行保甲連坐之法,且謂其禍烈於洪水猛獸。疏上,下各督撫議,於是請禁者紛起。

湖廣總督林則徐奏尤剴切,言:「鴉片不禁絕,則國日貧,民日弱,十餘年後,豈惟無可籌之餉,抑且無可用之兵。」帝深然其言,詔至京面授方略,以兵部尚書頒欽差大臣關防,赴粵東查辦。明年春正月,至粵東,與總督鄧廷楨會申煙禁,頒新律:以一年又六月為限,吸煙罪絞,販煙罪斬。時嚴捕煙犯,洋人泊零丁洋諸躉船將徙避,則徐咨水師提督各營分路扼守,令在洋躉船先繳煙方許開艙。又傳集十三行商人等,令諭各商估煙土存儲實數,並索歷年販煙之查頓、顛地二人,查頓遁走。義律託故回澳門。及事亟,斷水陸餉道,義律乃使各商繳所存煙土,凡二萬二百八十三箱,則徐命悉焚之,而每箱償以茶葉五斤,復令各商具「永不售賣煙土」結。於是煙商失利,遂生觖望。

義律恥見挫辱,乃鼓動國人,冀國王出干預。國王謀於上下議院,僉以此類貿易本干中國例禁,其曲在我。遂有律土丹者,上書求禁,並請禁印度栽種。又有地爾窪,作鴉片罪過論,以為既壞中國風俗,又使中國猜忌英人,反礙商務。然自燒煙之信傳入外洋,茶絲日見翔踴,銀利日長,義律遂以為鴉片興衰,實關民生國計。

時林則徐令各洋船先停洋面候查,必無攜帶鴉片者,始許入口開艙。各國商俱如命。獨義律抗不遵命,謂必俟其國王命定章程,方許貨船入口,而遞書請許其國貨船泊近澳門,不入黃埔。則徐嚴駁不許,又禁絕薪蔬食物入澳。義律率妻子去澳,寄居尖沙嘴貨船,乃潛招其國兵船二,又取貨船配以砲械,假索食,突攻九龍山。參將賴恩爵砲沈其雙桅船一,餘船留漢仔者亦為水師攻毀。義律求澳人轉圜,原遵新例,惟不肯即交毆斃村民之犯;又上書請毋逐尖沙嘴貨船,且俟其國王之命。水師提督關天培以不交犯,擲還其書。冬十月,天培擊敗英人,義律遁。十一月,罷英人互市,英貨船三十餘艘皆不得入。又搜捕偵探船,日數起。英商人人怨義律,義律不得已,復遣人投書乞恩,請仍回居澳門。林則徐以新奉旨難驟更,復嚴斥與之絕。而英貨船皆泊老萬山外洋不肯去,惟以厚利啗島濱亡命漁舟蜑艇致薪蔬,且以鴉片與之市。是月,廣東增嚴海防。

二十年春正月,廣東游擊馬辰焚運煙濟英匪船二十餘。夏五月,林則徐復遣兵逐英人於磨刀洋。時義律先回國請益兵,其國遂命伯麥率兵船十餘及印度兵船二十餘來粵,泊金星門。則徐以火艘乘風潮往攻,英船避去。英人見粵防嚴,謀擾閩,敗於廈門。六月,攻定海,殺知縣姚懷祥等。事聞,特旨命兩江總督伊里布為欽差大臣,赴浙督師。七月,則徐遣副將陳連升、游擊馬辰,率船五艘攻英帥士密於磨刀洋。馬辰一艘先至,乘風攻之,砲破其船。

八月,義律來天津要撫。時大學士琦善任直隸總督,義律以其國巴里滿衙門照會中國宰相書,遣人詣大沽口上之,多所要索:一,索貨價;二,索廣州、廈門、福州、定海各港口為市埠;三,欲敵體平行;四,索犒軍費;五,不得以外洋販煙之船貽累岸商;六,欲盡裁洋商浮費。琦善力持撫議,旋宴其酋目二十餘人,許陳奏。遂入都面陳撫事。乃頒欽差大臣關防,命琦善赴粵東查辦。是月,免浙江巡撫烏爾恭額,以失守海疆,又英人投書不受故也。義律既起椗,過山東,巡撫託渾布具犒迎送,代義律奏事,謂義律恭順,且感皇上派欽差赴粵查辦恩。罷兩廣總督林則徐,上諭切責,以怡良暫署總督事。會義律南行過蘇,復潛赴鎮海。時伊里布駐浙,接琦善議撫咨,遣家丁張喜赴英船犒師。英水師統領伯麥踞定海數月,聞撫事定,聽洋艘四出游弈。至餘姚,有土人誘其五桅船入攔淺灘,獲黑白洋人數十。伊里布聞之,飛檄餘姚縣設供張,委員護入粵。

冬十月,琦善抵廣州,尋授兩廣總督。義律請撤沿海諸防。虎門為廣州水道咽喉,水師提督駐焉。其外大角、沙角二砲臺,燒煙後,益增戍守。師船、火船及蜑艇、扒龍、快蟹,悉列口門內外,密布橫檔暗椿,至是裁撤殆盡。義律遂日夜增船櫓,造攻具;首索煙價,繼求香港,且行文趣琦善速覆。十二月五日,突攻沙角砲臺,副將陳連升等兵不能支,遂陷,皆死之。英人又以火輪、三板赴三門口,焚我戰船十數艘,水師亦潰。英人乘勝攻大角砲臺,千總黎志安受傷,推砲落水,潰圍出,砲臺陷。英人悉取水中砲,分兵戍守,於是虎門危急。水師提督關天培、總兵李廷鈺、游擊馬辰等守靖遠、威遠砲臺,僅兵數百,遣弁告急,不應。廷鈺至省泣求增兵,以固省城門戶。琦善恐妨撫議,不許。文武僚屬皆力請,始允遣兵五百。義律仍挾兵力索煙價及香港。二十一年春正月,琦善以香港許英,而未敢入奏,乃歸浙江英俘易定海。義律先遣人赴浙繳還定海,續請獻沙角、大角砲臺以易之。琦善與訂期會於蓮花城。義律出所定貿易章程,並給予香港全島,如澳門故事,皆私許之。

既而琦善以義律來文入奏,帝怒不許。罷琦善並伊里布,命宗室奕山為靖逆將軍,尚書隆文、湖南提督楊芳為參贊大臣,赴粵剿辦。時義律以香港已經琦善允給,遍諭居民,以香港為英屬埠。又牒大鵬營副將令撤營汛。粵撫怡良聞之,大駭,奏聞。帝大怒,合籍琦善家。遂下詔暴英人罪,促奕山等兼程進,會各路官兵進剿。尋以兩江總督裕謙為欽差大臣,赴浙視師。時定海、鎮海等處英船四出游弈,裕謙遣兵節次焚剿,並誅其酋目一人。二月,英人犯虎門,水師提督關天培死之;乘勝薄烏湧,省城大震。十三日,參贊楊芳抵粵,各路官兵未集,而虎門內外舟師悉被毀。楊芳議以堵為剿,使總兵段永福率千兵扼守東勝寺,陸路總兵長春率千兵扼鳳凰岡水路。英人率師近逼,雖經鳳凰岡官兵擊退,仍乘潮深入,飛砲火箭並力注攻。會美領事以戰事礙各國商船進口,赴營請進埔開艙,兼為英人說和,謂英人繳還定海,惟求通商如舊,並出義律書,有「惟求照常貿易,如帶違禁物,即將貨船入官」之文。時定海師船亦至粵,楊芳欲藉此緩兵退敵,遂與怡良聯銜奏請。帝以其復踵請撫故轍,嚴旨切責不許。三月,詔林則徐會辦浙江軍務,尋復遣戍新疆。

四月,奕山以楊芳、隆文等軍分路夜襲英人,不克。英人遂犯廣州城。不得已,仍議款。義律索煙價千二百萬。美商居間,許其半。議既定,奕山奏稱義律乞撫,求許照舊通商,永不售賣鴉片,將所償費六百萬改為追交商欠。撫議既定,英人以撤四方砲臺兵將擾佛山鎮,取道泥城,經蕭關、三元里,里民憤起,號召各鄉壯勇,四面邀截,英兵死者二百餘,殪其渠帥伯麥等。義律馳援,復被圍。亟遣人突出告急於廣州知府餘葆純,葆純馳往解散,翼義律出圍登舟免。時三山村民亦擊殺英兵百餘。佛山義勇圍攻英民於龜岡砲臺,殲英兵數十,又擊破應援之杉板船。新安亦以火攻毀其大兵船一,餘船遁。義律牒總督示諭,眾始解散。

義律受挫,久之,始變計入閩,攻廈門,再陷。復統兵攻定海,總兵葛云飛等戰沒。裕謙以所部兵赴鎮海,方至,而英人自蛟門島來攻。時鎮海防兵僅四千,提督餘步雲與總兵謝朝恩各領其半。步雲違裕謙節制,不戰先走。英遂據招寶山,俯攻鎮海,陷之。裕謙赴水死,謝朝恩亦戰歿。英人乘勝據寧波。八月,英人攻雞籠,為臺灣道姚瑩所敗。九月,命大學士宗室奕經為揚威將軍,侍郎文蔚、副都統特依順為參贊大臣,赴浙,以怡良為欽差大臣,赴閩,會辦軍務。二十二年春正月,大兵進次紹興,將軍、參贊定議同日分襲寧波、鎮海。豫洩師期,及戰,官軍多損失。是月,姚瑩復敗英人於大安。二月,英人攻慈谿營,金華協副將硃貴及其子武生昭南、督糧官即用知縣顏履敬死之。是月,起用伊里布。先是伊里布解任,並逮其家人張喜入都遣戍。至是,浙撫劉韻琦請起用,報可。旋以耆英為杭州將軍,命臺灣設防。

夏四月,英人犯乍浦,副都統長喜、同知韋逢甲等戰死。時伊里布已來浙,即命家人張喜見英酋,告以撫事有成,令先退至大洋,即還所俘英人。英人如約,遂以收復乍浦奏聞。英人連陷寶山、上海,江南提督陳化成等死之,遂犯松江,陷鎮江,殺副都統海齡。淮揚鹽商懼甚,賂英師乞免。

秋七月,犯江寧。英火輪兵船八十餘艘溯江上,自觀音門至下關。時耆英方自浙啟行,伊里布亦奉詔自浙馳至,遣張喜詣英船道意。英人要求各款:一,索煙價、商欠、兵費銀二千一百萬;一,索香港為市埠,並通商廣州、福州、廈門、寧波、上海五口;一,英官與中國官用敵體禮;餘則劃抵關稅、釋放漢奸等款,末請鈐用國寶。會耆英至,按款稍駮詰。英突張紅旗,揚言今日如不定議,詰朝攻城,遂即夜覆書,一如所言。翼日,遣侍衛咸齡、布政司黃恩彤、寧紹臺道鹿澤長往告各款已代請,俟批回即定約。奏上,許之。時耆英、伊里布、牛鑒以將修好,遣張喜等約期相見。馬利遜請以本國平行禮見。耆英等遂詣英舟,與璞鼎查等用舉手加額禮訂約,復親具牛酒犒師,畫諾於靜海寺,是為白門條約。自此煙禁遂大開矣。而英猶以臺灣殺英俘,為總兵達洪阿、兵備道姚瑩罪來詰,不得已,罷之。

十二月,以伊里布為欽差大臣,赴廣東督辦通商事。二十三年夏,伊里布卒,詔耆英往代。先許英廣州通市。初,英粵東互市章程,各國皆就彼掛號始輸稅。法人、美人皆言「我非英屬」,不肯從,遂許法、美二國互市皆如英例。

二十四年,英人築福州烏石山,英領事官見浙閩總督劉韻珂,請立商埠,欲於會城內外自南臺至烏石山造洋樓,阻之。值交還欠款,照江寧約,已付甲辰年銀二百五十萬,應將舟山、鼓浪嶼退還中國。英公使藉不許福州城內建樓事,不與交還。屢經辯論,始允退還鼓浪嶼,然執在彼建屋如故。

福州既得請,遂冀入居廣州城。廣州民憤阻,揭帖議劫十三洋行,英酋逸去,入城之議遂不行。二十六年秋七月,英人還舟山。十二月,請與西藏定界通商,以非條約所載,不許。二十八年,英酋文翰復請入廣州城互市,總督徐廣縉拒之。越日,英舟闖入省河,廣縉單舸往諭,省河兩岸義勇呼聲震天。文翰請仍修舊好,不復言入城事。

咸豐元年,文宗嗣位,英人以火輪船駛赴天津,稱來吊大行皇帝喪。直隸總督以聞,命卻之。三年,洪秀全陷江寧,英以輪船駛至江寧,迎入城,與通款,英人言:「不助官,亦不助洪。」四年,劉麗川據上海作亂。初,英人阻官軍進兵,江督怡良等詰之。既而英人欲變通貿易章程,聯法、美二國請於粵督葉名琛,不許,遂赴上海見蘇撫吉爾杭阿。九月,赴天津。帝命長蘆鹽政崇綸等與相見,拒其遣使駐京諸條,久之始去。

六年秋九月,英人巴夏裏致書葉名琛,請循江寧舊約入城,不省。英人攻粵城,不克逞,復請釋甲入見,亦不許。冬十月,攻虎門橫檔各砲臺,又為廣州義勇所卻,乃馳告其國。於是簡其伯爵額爾金來華,擬由粵入都,先將火輪兵船分泊澳門、香港以俟。額爾金至粵,初謀入城,不可。與水師提督、領事等議款,牒粵中官吏,俟其復書定進止,名琛置不答。七年冬十二月,英人遂合法、美、俄攻城,城陷,執名琛去。因歸罪粵中官吏,上書大學士裕誠求達。裕誠覆書,令赴粵與新命粵督黃宗漢商辦,不省。

八年夏四月,聯兵犯大沽,連陷前路砲臺。帝命科爾沁親王僧格林沁率師赴天津防剿,京師戒嚴。帝命大學士桂良、吏部尚書花沙納赴天津查辦,復起用耆英偕往。耆英至,往謁英使,不得見,擅自回京,賜自盡。英有里國太者,嘉應州人也,世仰食外洋,隨英公使額爾金為行營參贊。聞桂良至,即持所定新議五十六條,要桂良允許,桂良辭之。津民憤,與英人鬥,擒里國太將殺之。桂良、譚廷襄恐誤撫局,亟遣人釋里國太,送回舟。時廷臣交章請罷撫議,以疆事棘,不得已,始命桂良等與定和約五十六款。六月,遣桂良、花沙納巡視江蘇,籌議諸國通商稅則。冬十月,定通商稅則。時英人以條約許增設長江海口商埠,欲先察看沿江形勢。定約後,即遣水師、領事以輪船入江,溯流至漢口,逾月而返。

是年,議通商善後事。時各國來天津換約,均因桂良原議,改由北塘海口入。獨英船先抵天津海口,俄人繼之,突背前約,闖入大沽口。直隸總督恆福遣人持約往,令改道,不聽。九年夏五月,英船十餘艘駛至灘心。越日,豎紅旗挑戰,拽倒港口鐵金巢、鐵椿,遂逼砲臺,開砲轟擊。時僧格林沁防海口,開砲應之,沈毀其數船。英人復以步隊接戰,又敗之。十年夏六月,復犯天津海口,直隸提督樂善守北岸砲臺,拒戰,中砲死。時僧格林沁尚守南岸砲臺。詔罷兵議撫,乃自天津退軍張家灣,英遂乘勢陷天津。尋復遣僧格林沁進軍通州。帝仍命大學士桂良往天津議撫。桂良抵津,牒洋人商和局。英公使額爾金、參贊巴夏裏請增軍費及在天津通商,並請各國公使帶兵入京換約。桂良以聞,嚴旨拒絕,仍命僧格林沁等守通州。

八月,英人犯通州,帝命怡親王載垣赴通議款。時桂良及軍機大臣穆廕皆在,英使額爾金遣其參贊巴夏裏入城議和,請循天津原議,並約法使會商。翼日,宴於東獄廟。巴夏裏起曰:「今日之約,須面見大皇帝,以昭誠信。」又曰:「遠方慕義,欲觀光上國久矣,請以軍容入。」王憤其語不遜,密商僧格林沁,擒送京師,兵端復作。時帝適秋獮,自行在詔以恭親王奕為全權大臣,守京師,並詔南軍入援。時團防大臣、大學士周祖培,尚書陳孚恩等議籌辦團練城守事。恭親王、桂良駐城外,而英師已薄城下,焚圓明園。英人請開安定門入與恭親王面議和,乃約以次日定和議,而釋巴夏里於獄,遣恆祺送歸。九月,和議成,增償兵費八百萬,並開天津商埠,復以廣東九龍司地與英人。是年,用里國太幫辦稅務。

十一年春二月,英人始立漢口、九江市埠,均設洋關。九月,總署因與英使卜魯士議暫訂長江通商章程十二款,納稅章程五款。是月,交還廣東省城。卜魯士始駐京。同治元年,粵賊陷蘇、松、常、太各城,各國懼擾上海商務,謀自衛。英水師提督何伯隨法、美攻剿,復青浦、寧波諸處。捷聞,嘉獎。九月,與英人續訂長江通商章程。二年春,以英將戈登統常勝軍,權授江蘇總兵。四年秋七月,英交還大沽砲臺。

五年春正月,與英人議立招工章程。七年十二月,臺灣英領事吉必勛因運樟腦被阻,牽及教堂,洋將茄當踞營署,殺傷兵勇,焚燒軍火局庫,索取兵費。事聞,詰英使,久之,始將吉必勛撤任。未幾,英兵船在潮州,又有毀燒民房、殺死民人事,幾釀變。八年九月,與英換新約,英使阿禮國請朝覲,不許。九年,請辦電線、鐵路,不許。既而請設水底電線於中國通商各口,許之。十年,請開瓊州商埠。先是同治七年修新約,英使阿禮國允將瓊州停止通商,以易溫州。至是,英使威妥瑪與法、俄、美、布各國咸以為請,允仍開瓊州。十二年,穆宗親政,始覲見。初因覲見禮節中外不同,各國議數月不決,英持尤力,至是始以鞠躬代拜跪,惟易三鞠躬為五,號為加禮。

光緒元年正月乙卯,英繙譯官馬嘉理被戕於雲南。先是馬嘉理奉其使臣威妥瑪命,以總署護照赴緬甸迎探路員副將柏郎等,偕行至雲南騰越屬蠻允土司地被戕。時岑毓英以巡撫兼署總督。威妥瑪疑之,聲言將派兵自辦。帝派湖廣總督李瀚章赴滇查辦。威妥瑪遂出京赴上海,於是有命李鴻章、丁日昌會同商議之舉。威妥瑪至津見李鴻章,以六事相要,鴻章拒之。政府派前兵部侍郎郭嵩燾使英,威妥瑪亦欲拒議。又駐滬英商租上海、吳淞間地敷設鐵軌,行駛火車,總督沈葆楨以英人築路租界外,違約,飭停工。至是,威妥瑪遣其漢文正使梅輝立赴滬商辦,鴻章乃與約,令英商停工,而中國以原價購回自辦。初上海既通商,租界內仍有釐捐局,專收華商未完半稅之貨。至是,威妥瑪欲盡去釐捐局,界內中國不得設局徵收釐稅,鴻章請政府勿許。

二年五月,諭:「馬嘉理案,疊經王大臣與英使威妥瑪辯論未洽,命李鴻章商辦早結。」六月,命鴻章為全權大臣,赴煙臺,與威妥瑪會商,相持者逾月,議始定。七月,鴻章奏稱:「臣抵煙臺,威妥瑪堅求將全案人證解京覆訊,其注意尤在岑毓英主使。臣與反復駁辨,適俄、德、美、法、日、奧六國使臣及英、德水師提督均集煙臺,往來談宴,因於萬壽聖節,邀請列國公使、提督至公所燕飲慶賀,情誼聯洽。翌日,威使始允另議辦法,將條款送臣查核。其昭雪滇案六條,皆總理衙門已經應允,惟償款銀數未定。其優待使臣三條:一,京外兩國官員會晤,禮節儀制互異,欲訂以免爭端;一,通商各口會審案件;一,中外辦案觀審,兩條可合並參看。觀審一節,亦經總署於八條內允行。至通商事務原議七條:一,通商各口,請定不應抽收洋貨釐金之界,並欲在沿海、沿江、沿湖地面,添設口岸;一,請添口岸,分作三項,以重慶、宜昌、溫州、蕪湖、北海五處為領事官駐扎,湖口、沙市、水東三處為稅務司分駐,安慶、大通、武穴、陸溪口、岳州、瑪斯六處為輪船上下客商貨物;一,洋藥準在新關並納稅釐;一,洋貨半稅單,請定劃一款式,華、洋商人均準領單,洋商運土貨出口,商定防弊章程;一,洋貨運回外國,訂明存票年限;一,香港會定巡船收稅章程;一,各口未定租界,請再議訂。以上如洋藥釐稅由新關並徵,既免偷漏,亦可隨時加增;土貨報單嚴定章程,冀免影射冒騙諸弊;香港妥議收稅辦法,均尚於中國課餉有益。其餘亦與條約不背。英使又擬明年派員赴西藏探路,請給護照,因不便附入滇案、優待、通商三端之內,故列為專條。免定口界、添設口岸兩事,反覆爭論,乃允免定口界,僅於租界免抽洋貨釐金,且指明洋貨、土貨仍可抽收。將來洋藥加徵,稍資撥補,似於大局無甚妨礙。至添口岸一節,總署已允宜昌、溫州、北海三處,赫德續請添蕪湖口,亦經奏準。今仍堅持前議,準添四口,作為領事官駐扎處所。其重慶派英員駐寓,總署已於八條內議準,未便即作口岸,聲明俟輪船能上駛時,再行議辦。至沿江不通商口岸上下客商貨物一節,自長江開碼頭後,輪船隨處停泊,載人運物,因未明定章程,礙難禁阻。英使既必欲議準,似不在停泊處所之多寡,要在口岸內地之分明。臣今與訂『上下貨物,皆用民船起卸,仍照內地定章,除洋貨稅單查驗免釐外,有報單之土貨,只準上船,不準卸賣,其餘應完稅釐,由地方官一律妥辦』等語,是與民船載貨查收釐金者一律,只須各地方關卡員役查察嚴密耳。英使先請湖口等九處,臣與釐定廣東之水東系沿海地方,不準驟開此禁,岳州距江稍遠,不準繞越行走,姑允沿江之大通、安慶、湖口、武穴、陸溪口、沙市六處,輪船可暫停泊,悉照內地抽徵章程。臣復與德國使臣巴蘭德議及德國修約添口,即照英國議定辦理。威妥瑪請半年後,開辦口岸租界,免洋貨釐,洋藥並納釐稅,須與各國會商,再行開辦,因準另為一條。至派員赴西藏探路一節,條約既準游歷,亦無阻止之理。臣於原議內由總理衙門、駐藏大臣查度情形字樣,屆時應由總理衙門妥慎籌酌。迨至諸議就緒,商及滇案償款。英使謂去冬專為此事,調來飛游幫大兵船四隻,保護商民,計船費已近百萬。臣謂兩國並未失和,無認償兵費之例,囑其定數。英使謂吳淞鐵路正滋口舌,如臣能調停主持,彼即擔代,仍照原議作二十萬,遂定議。因於二十六日,將所繕會議條款華、洋文四分,彼此畫押蓋印互換。至滇邊通商,威使面稱擬暫緩開辦,求於結案諭旨之末,豫為聲明。」疏入,報聞。鴻章仍回直督本任。約成互換,是為煙臺條約。約分三端:一曰昭雪滇案,二曰優待往來,三曰通商事務。又另議專案一條。是年,遣候補五品京堂劉錫鴻持璽書往英,為踐約惋惜滇案也。

三年,英窺喀什噶爾,以護持安集延為詞。陜甘總督左宗棠拒之。英人欲中國與喀什噶爾劃地界,又請入西藏探路,皆不行。是年始於英屬地星嘉坡設領事。四年秋八月,福建民毀英烏石山教堂,英人要求償所失乃已。五年,英欲與中國定釐稅並徵確數。總署擬仍照煙臺原議條款,稅照舊則,釐照舊章。

七年十月,李鴻章復與威妥瑪議洋藥加徵稅釐。初,洋藥稅釐並徵之議,始發於左宗棠,原議每箱徵銀一百五十兩。其後各督撫往來商議,訖無成說。滇案起,鴻章乃與威妥瑪議商洋藥加徵稅釐。威妥瑪謂須將進出口稅同商,定議進口稅值百抽十,而出口稅以英商不原加稅為辭,並主張在各口新關釐稅並加,通免內地釐金。鴻章以欲通免釐金,當於海關抽稅百二十兩,須加正稅三倍。如不免釐金,則須增加一倍至六十兩。既,威妥瑪接到本國擬定鴉片加稅章程數條:「一,釐稅並徵增至九十兩;二,增正稅至五十兩,各口釐金仍照舊收;三,擬由中國通收印度鴉片,而印度政府或約於每年減種鴉片,或由兩國商定當減年限,至限滿日停種,至每石定價,或按年交還,或另立付價,時候亦由兩國訂明,其價或在香港撥還,或在印度交兌,其事則官辦商辦均可;四,擬立專辦洋藥英商公司,每箱應償印度政府一定價值,應納中國國家一定釐稅,至繳清此項釐稅後,其洋藥在中國即不重徵,印度政府約明年限,將鴉片逐漸裁止。」初,威妥瑪於進口已允值百抽十,至是因洋藥稅釐未定,又翻。又欲於各口租界外,酌定二三十里之界,免收洋貨釐。鴻章以租界免釐,載在條約,業經開辦有年,何得復議推廣?拒之。威妥瑪又請由香港設電線達粵省,其上岸祗準在黃埔輪船停泊附近之處,由粵省大吏酌定。

九年三月,上諭:「洋藥稅釐並徵,載在煙臺條約,總理衙門歷次與英使威妥瑪商議,終以咨報本國為詞,藉作延宕。威妥瑪現已回國,著派出使大臣曾紀澤妥為商辦,如李鴻章前議一百一十兩之數,並在進口時輸納,即可就此定議。洋藥流毒多年,自應設法禁止。英國現有戒煙會,頗以洋藥害人為恥。如能乘機利導,與英外部酌議洋藥進口、分年遞減專條,逐漸禁止,尤屬正本清源之計。並著酌量籌辦。」紀澤奉旨與英外部議,三年始定。十一年六月,奏曰:「臣遵旨與英外部尚書伯爵葛蘭斐爾,侍郎龐斯茀德、克雷等商論,力爭數目,最後乃得照一百一十兩之數。今年二月,準彼外部允照臣議,開具節略,咨送臣署,且欲另定專條,聲明中國如不能令有約諸國一體遵照,英國即有立廢專約之權。臣復力爭,不允載入專條,彼乃改用照會。詳勘所送節略,即系商定約稿。其首段限制約束等語,緣逐年遞減之說,印度部尚書堅執不允。其侍郎配德爾密告臣署參贊官馬格里雲,照專條辦法,印度每年已減收英金七十萬餘鎊,中國欲陸續禁減洋藥入口,惟有將來陸續議加稅金,以減吸食之人,而不能與英廷豫定遞減之法。遂未堅執固爭,而請外部於專案首段,加入於行銷洋藥之事須有限制約束一語,以聲明此次議約加稅之意,而暗伏將來修約議加之根。至如何酌定防弊章程,設立稽徵總口,煙臺條約第三端第五節固已明定要約。臣此次所定專條第九款又復聲明前說,將來派員商定,自不難妥立章程,嚴防偷漏。其餘各條,核與疊準總理衙門函電吻合。旋承總署覆電照議畫押。時適英外部尚書葛蘭斐爾退位,前尚書侯爵沙力斯伯里推為首相,仍兼外部。六月三日,始據來文定期七日畫押。臣屆期帶同參隨等員前往外部,與沙力斯伯里將續增條約專條漢文、英文各二分,互相蓋印畫押。按此次所訂條約,除第二條稅釐並徵數目,恪遵諭旨,議得百一十兩外,又於第五條議得洋藥於內地拆包零售,仍可抽釐,是內地並未全免稅捐。將來若於土煙加重稅釐,以期禁減,則洋藥亦可相較均算,另加稅釐。臣於專條中並未提及土煙加稅之說,以期保我主權。」疏入,得旨允行。旋兩國派員互換,是為煙臺續約。

秋八月,英人議通商西藏。是歲英窺緬甸,踞其都。滇督岑毓英奏請設防,旋遣總兵丁槐率師往騰越備之。中國以緬甸久為我屬,電曾紀澤向英外部力爭,令存緬祀立孟氏。英外部不認緬為我籓屬,而允立孟氏支屬為緬甸教王,不得與聞政令。紀澤未允,外部尚書更易教王之說亦置諸不議矣。既,英署使歐格訥以煙臺約有派員入藏之文,堅求立見施行。總署王大臣方以藏眾不許西人入境,力拒所請。會歐格訥以緬約事自詣總署,言緬甸前與法私立盟約,是以興師問罪。令若重立緬王,則法約不能作廢,故難從命。今欲依緬甸舊例,每屆十年,由緬甸長官派員赴京,而勘定滇、緬邊界,設關通商,以踐前約。王大臣等以但言派員赴京,並未明言貢獻,辨爭再四,始改為呈進方物,循例舉行,而勘界、通商,則皆如所請。歐格訥始允停止派員入藏,藏、印通商,仍請中國體察情形,再行商議。議既定,總署因與歐格訥商訂草約四條,得旨允行。十二年九月,請英退朝鮮巨文島,不聽。十月,議瓊州口岸。英領事以條約有牛莊、登州、臺灣、潮州、瓊州府城口字樣,謂城與口皆口岸,中國以英約十一款雖有瓊州等府城口字樣,而煙臺續約第三端,聲明新舊各口岸,除已定有各國租界,應無庸議云云。英約天津郡城海口作通商埠,紫竹林已定有各國租界,城內亦不作為口岸,以此例之,則瓊州海口系口岸,瓊州府城非口岸也。十三年秋七月,與英換緬約於倫敦。

十四年春,英人麻葛藟督兵入藏,藏人築卡御之,為英屬印兵所逐。藏人旋又攻哲孟雄境之日納宗,又敗。先是,藏地國初歸附,自英侵入印度後,藏遂與英鄰。乾隆年,英印度總督曾通使班禪求互市,班禪謂當請諸中國,議未協而罷。哲孟雄者,藏、印間之部落也。道光間,英收為印屬。及煙臺訂約有派員入藏之說,而藏人未知,遂築砲臺於邊外之隆吐山,冀阻英兵使不得前。英人以為言,帝諭四川總督劉秉璋,飛咨駐藏大臣文碩、幫辦大臣升泰,傳各番官嚴切宣示,迅撤卡兵。於時升泰尚未抵任,文碩未諳交涉,輒以拒英護藏覆奏。於是嚴旨切責,以長庚代之。仍有旨催令升泰赴藏,傳齊番官,諭以:「上年與英人訂議,緩辦通商,正朝廷護持黃教、覆庇藏番,代籌一永保安全之至計。但令迅速撤卡,印督已言明彼決不越藏、中定界熱勒巴拉山嶺一步。彼此未經開戰,無論此地屬藏屬哲,將來尚可從容辨論。」時十四年正月也。

寄諭未至,英兵已進攻隆吐,毀其壘,藏番悉潰。乃欲藉通商以緩師,文碩復左右之,竟以藏人與英自行立約入奏。四月丁亥,諭曰:「印、藏通商一事,英人約定並不催辦。此次開釁,與通商絕無干涉。文碩始終不明機要,乃欲藉通商為轉圜,不思藏為中國屬地,豈有聽其自行與人立約之理!升泰、文碩接奉此旨,即傳集番官,諭以事須稟明駐藏大臣具奏,由總理衙門核定,候旨遵辦。」五月庚申,又諭曰:「使英大臣劉瑞芬電稱,『印督近又函達藏官,但令藏眾退回原界,便可仍舊和好,絕不欲侵入藏地,致礙兩國睦誼。』向來藏務專歸商上,第穆呼圖克圖人尚和平曉事,現在掌辦商上,責有專歸。升泰接奉此旨,即傳諭第穆,令其妥為了結。」

未幾,升泰抵任受事。九月,奏言:「藏番自作不靖,肇起兵戈。所有隆吐山南北本皆哲孟雄地,英人雖視為保護境內,實則哲孟雄、布魯克巴皆西藏屬籓,每屆年終,兩部長必與駐藏大臣呈遞賀稟,駐藏大臣循例優加賞犒。唐古特自達賴喇嘛以下,均有額定禮物,商上亦回賞緞疋銀茶,與兩部復書草稿,必呈送駐藏大臣批準,始行繕覆。哲、布兩部遇有爭訟,亦稟由藏官酌派漢、番官辦理,此哲、布本為藏地屬籓之實在情形也。兩部長於光緒二年曾各遞番字稟,以英人有窺伺藏地之心,請早為設法辦理。雖經前西藏糧員四望關通判周溱帶同戴琫札喜達結往辦,祗取哲孟雄空結一紙,敷衍了事,並不妥籌善後,貽悮邊疆,其禍實自此始。嗣後哲夷知藏番並無遠慮,始一意與英人交接,又復貪利取租,聽英人修路直至捻納,迄今仍稱租界,又藏中自失籓籬之始末也。藏人不知優待屬籓,哲部偶受欺凌,不為申理,此時漸覺英人有偪己之心,忽又攘奪哲地以為己有,更揚言哲夷私結英人,屢議起兵攻伐,哲夷內不自安,則益句結英人以圖自保,此又藏、印交兵之所由來也。藏人自四月十三日戰敗之後,不思設法弭患,又復添調各路土兵,分由小道至帕克里,沿途騷擾,良民大受荼毒。番官管餉,又多減刻,人有怨言,軍無鬥志。除向隸戴琫之兵三千,及工布兵數百人,差可用命,餘則悉系烏合。現劄帕隘以外者一萬餘人,分布各口又數千人,一旦敗北譁潰,則數千里臺站伏莽增多,此內患之堪虞者也。近時開導之難,實因曩時初與外人交涉,商上辦事諸員邀三大寺僧眾,以護教為名,共立誓詞,云『藏地男女不原與洋人共生於天地,此後藏中男女老弱有違此誓,即有背黃教,人人得而誅之』。此本不肖之徒,為聚眾抗官之謀,三大寺僧眾亦藉此干預政事。今事機危迫,特旨到藏,第穆亦知凜畏。無如遽違初議,即禍在目前,雖掌辦商務之尊,恐亦不免自危,其噶布倫以次更不待言。窺其情形,似非背城一戰,難望轉機。此臣探其隱衷而言,非藏番等自有此語也。此時兵尚未撤,委員不便前往。且委員至彼辦理界務,應與英國何人會議,應請飭詢英使,由總署知照藏中,庶免隔閡。近年藏番異常刁悍,今自開兵釁,尚不自知悔悟,實難姑容。第藏衛距川過遠,餉絀兵單,無事不形掣肘。臣萬不敢不出之審慎,籌慮萬全,相機駕馭,冀紓朝廷西顧之憂。」

是月丁卯,又奏:「臣於五月二十六日抵藏,第穆與大小番官僧俗公同遞稟,譯其情詞,總以隆吐之南日納宗為藏界,藏人設卡系在境內,英人無端恃強動兵侵地為言。臣以經界為地方要政,從前豈無案牘。乃派員將新舊各案卷概行檢閱,始尋出乾隆五十九年前大臣尚書和琳、內閣學士和瑛任內奏設鄂博原案一卷,注明藏內界址,系在距帕克里三站之雅拉、支木兩山,設有鄂博。又有春丕、日納宗兩處,上年雖系藏界,乾隆五十三年廓番用兵,哲孟雄被廓夷追過藏曲大河,哲部窮蹙,達賴喇嘛始將日納宗地賞給哲部筦理,原派委員西藏游擊張志林原稟,即聲敘日納宗不應作為藏界,只在雅拉、支木兩山設立鄂博,稟詞甚為明晰。此圖惜已佚,又覓得舊圖一張,並注明納蕩一地乃哲孟雄邊境,藏圖南面極邊界線之上亦繪有雅拉山,是雅拉山確屬藏地南界。至藏人設卡之隆吐山,考之舊圖,實無此名,以英人所云日納宗在隆吐北數十里,而藏番新圖則日納宗又在隆吐之南,顯系藏人多繪此一段,飾稱藏界。臣既考察明確,即以原卷舊圖發交開導委員,轉給藏番閱看。番人雖有愧色,然終以日納宗本屬藏地,從前雖賞給哲夷,今哲夷已歸英屬,應即收回自筦。旋奉電傳寄諭,臣即面授第穆。臣深慮第穆使將屯兵先行撤入帕克里,並札飭哲、布兩部長親赴英軍,告以藏人畏偪,故兵難先撤,印兵亦宜克踐前言,彼此約期同日撤退,仍由臣致信英官,促其速撤。忽又得報,英人於六月二十八日添兵九百餘名,又益以大砲六門。第穆旋亦稟英人屢次攻撲我營。且廓爾喀前王子果爾雜捻曾出奔印度,今亦由印帶兵五百名前來助戰,聞已過大吉嶺,是以未敢撤兵。伏乞飭下總署詳告英使,轉電印督,約期撤兵,並飭印兵毋得再動。」

疏入,奉上諭:「升泰所陳,頗中肯綮。劉瑞芬八月二十八日電稱:『印兵在熱勒巴拉山近處與藏兵攻戰,藏兵傷亡數百,印兵追入徵畢山岔。』九月十五日電稱:『英外部照覆,雲來攻納蕩之英軍統領拉哈瑪,已遵印度政府之諭,不可占據藏地,故追入徵畢後,立即退回。印督又報告其政府,謂駐藏大臣將以西歷十月三日由拉薩前赴邊界,已派政事官保爾前往會晤。』目前升泰想已接晤保爾。藏、哲界址當已查明,印督又有『甚望速了』之語。著即熟商妥辦。」

升泰先使江孜守備蕭占先馳往開導,又以知縣秀廕繼之。藏兵之敗也,英兵追至仁進岡,將盡焚山上下民舍。會占先至,見英將力爭,乃退屯對邦,而促升泰前往會議。數日,復進據姑布。升泰十一月至,與英員保爾相見於對邦,議經月未就。乃奏言:「英人戰勝而驕,必欲諸事議妥始允撤兵。現議哲孟雄事不下十次,保爾必欲將哲為英屬,註明條約,而畫咱利拉山為界,即歷次奏牘所謂熱勒巴拉山也。臣議以印督前言『藏眾退回原界,仍守二年以前情形,不在隆吐山駐兵,便可照舊辦理,絕不侵入藏界』等語折之,保爾則謂此語當在未開戰前,戰釁既開,自當另議。通商一事,英人開來條款,直欲到藏貿易。臣百端辨說,始允退至江孜。又答以萬不能行,則又意在帕克哩。帕隘乃藏南門戶,其險要在山腰之格林卡,若至帕克哩,則已在高原,為廓爾喀、哲孟雄、布魯克巴三部通衢。目前開導藏番,通商必在界外,始可期其遵從。是以臣堅未允許,保爾意甚怫然。臣惟有平心靜氣,婉與商榷,冀紓目前之急。」是年英定華工往澳大利亞例限。英君主維多利亞登位五十年,中國遣使致賀。

十五年,升泰復與英人接議通商、分界,久不決。十六年二月,朝旨派總稅務司赫德之弟赫政赴藏協商藏、印約事。升泰奏言:「撤兵藏番已原遵旨,所難者分界、通商兩大端耳。臣自到邊,哲部長之母率其親族頭目來營具稟,云:『英人昔年立約,曾經議明,無論如何不得逾日喜曲河一步。哲部租地與英,每年應收租費洋銀十二千圓,英人分毫未給。此次印、藏構兵,以致殃及,實不原再歸英屬。』臣維哲孟雄本屬小邦,僻在極邊。本年印、藏用兵,被英人掠取全土,復遷其部長,安置印度噶倫繃之地,而以重兵駐守扛多,即部長平時治所也。流離轉徙,情實可矜。是以此次會議,但許其保護,而必爭『照舊』兩字,使藏人不至咎臣辦理邊事失去屬籓,並可藉此羈縻布魯克巴。至布魯克巴,地大物博,民俗強悍,其地數倍哲孟雄,實為前藏屏蔽,西人呼為布丹國。上年曾經入貢,其部長向無印信,亦無封號。臣此次到邊,其部長派兵千七百人來營效力。臣方飭藏兵遣撤,豈可留此多人,致貽口實?是以優給賞賚,勉以大義,飭令速回,許事後為之代懇天恩。該部人歡忻鼓舞而去。」

赫政既抵藏,升泰與英官開議,保爾雖奉命印督為議約專員,然不得自主,事事仍請命印督。藏番不原與英接壤,必間哲孟雄於中,乃可定界。英既幽哲酋於噶倫繃,直欲收入印度幅員之內,藏人聞之益憤。升泰嚴飭番官僧俗毋率行干預哲事,而亟使赫政勸阻英官,勿遽更易哲酋,使藏人有所藉口。藏、哲舊界本在雅納、支木兩山間,其後商販往來另闢捷徑,於是有所謂咱利孔道者,即熱勒巴拉嶺之支麓也。升泰議即咱利山立石畫分藏、哲之界,其印、哲舊界在日喜河者,亦擬仍舊,而於條約註明。藏番不原通商,初指對邦附近地為商埠,後始議定後藏之亞東,於其地修建關卡,設漢官治之。藏番甫首肯,而英官又遷延不遽決。升泰亟奏請飭總署促英使迅速議約。總署王大臣旋擬四條,與英使華爾身籌商久之,始議定八款。總署乃上奏,謂:「第一款,藏、哲以咱利山一帶山顛為界;第二款,哲地歸英保護;第三款,兩邊各無犯越;其餘緩議。各條善後應辦事宜,侭可徐與商榷,彼此派員定議。請簡派升泰為全權大臣,與英員先行畫押。」奉旨俞允。是歲秋七月,出使大臣薛福成與英外部互換於倫敦,是為中英會議藏印條約。

是年德宗大婚,英派使臣華爾身齎英主維多利亞國書致賀,並自鳴鐘一座,上刻祝辭云:「日月同明,報十二時,吉祥如意,天地合德,慶億萬年,富貴壽康。」旋命駐使薛福成赴英外部傳旨致謝,並遞國書。是年英開重慶商埠。

十七年春正月,換約限滿,前駐藏大臣升泰遣員黃紹勛、張昉及總務司赫政與英印督蘭士丹所派之保爾在大吉嶺會議,各擬辦法。保爾欲在仁進岡入藏一百五十餘里之法利城即帕克裡設關通商,並俟十年後再定入口貨稅。升泰執定十二年條款「藏、印邊界通商,由中國體察情形」之語,辯駮久不決。十八年夏六月,復與保爾商議辦法九款,續款二條,定於交界之咱利山下亞東境內為英商貿易所。商上等復懷疑慮,堅請於二款內註明「不得擅入關內」字樣,又請禁印茶運藏,一再與英使華爾身辯論,仍不決。至十九年五月,總理衙門奏:「現據赫德稱:『印度已將辦法九款更改商訂,最緊要之第二款內,註明英商在亞東貿易,自交界至亞東而止;第四款內註明進出口稅,俟五年期滿酌定稅則;至印茶一項,現議開辦時不即運藏,俟五年限滿,方可入藏銷售,應納之稅不得過華茶入英納稅之數;此外各款,均照升泰所擬辦理。』臣等查中英通商稅則,茶葉每百斤徵銀二兩五錢,而洋商運華茶至英,每百斤徵銀十兩。現在先與議定,如印茶入藏,應照華茶入英每百斤稅銀十兩,磋議經年,始克就範。竊思藏約未結三端,自十七年開議至今,已屆三年之久,始得印、藏兩情翕然允協,即可就此收束,以綏邊圉。」是為續議中英會議藏印條款。是年十月,在大吉嶺互換。

既又與英議滇、緬界務。初,曾紀澤與英議約,英許中國稍展邊界,擬予以潞江以東南掌、拈人之地。既,紀澤又向英外部要求八募之地,不允。英外部侍郎克蕾謂英廷已飭駐緬之英官勘驗一地,允中國立埠設關收稅,有另指舊八募之說,在八募東二三十里。紀澤因與外部互書節略存卷,暫停不議。旋受代回華。

至是,出使大臣薛福成見英人與暹羅勘界,並有創築鐵路通接滇邊之訊,恐分界、通商事宜不早籌議,臨時必受虧損。於是上書請與英人提議。及福成往促踐前議,英以公法為解,謂:「西洋公法,議在立約之後,不可不遵;議在立約以前,不能共守。」蓋不認讓中國展邊界及以大金沙江為公共江、八募近處勘地、中國立埠設關三端。

薛福成以英既翻前議,因思野人山地綿亙數千里,不在緬甸轄境之內,復照外務部,請以大金沙江為界,江東之境歸滇。而印度總督不允,出師盞達邊外之昔馬攻擊野人,以示不原分地之意。又欲借端停商全約。福成仍促速議。久之,英始允將久淪於緬之漢龍、天馬兩關還中國。又久之,始允讓所據之鐵壁關。惟虎踞關,英人以深入彼境七八十里,與八募相近,不允讓。至於設關,拒尤力。福成以英既不允我地,則英所得於我之權利亦應作廢。相持甚久,始就滇境東南商定於孟定橄欖壩西南邊外讓一地曰科干,又自猛卯土司邊外包括漢龍關在內,作一直線,東抵潞江麻慄壩之對岸止,劃歸中國,約計八百英方里。又車里、孟連土司所屬鎮邊,系為兩屬,亦允全讓,並野人山毗連之昔馬亦允讓。至此界務告一結束。而商務,大金沙江行船、八募立埠設關,英仍不允。福成久與爭論,始於行船一事,於約中另立一條,不許他國援例,而設關仍不肯通融。惟約中於英人所得權利,如緬鹽不準運入滇境,英關暫不徵收貨稅,領事僅設一員、限一定駐所,商貨僅由二路,不準開埠,英亦無詞。遂於二十年正月二十四日在倫敦定約,共二十條:一、二、三、四,劃定各段界線;五,中國不再索問永昌、騰越邊界外隙地,英國於北丹泥及科干照所劃邊界讓與中國,孟連、江洪之地亦歸中國,惟未定議前不得讓與他國;八,各貨物分別應稅不應稅;十、十一,分別各貨物準販運不準販運;十三,中國派領事駐仰光,英國派領事駐蠻允;十五,定交逃犯例;十七,定中、英民在兩國界內相待最優例;又專條內各條款,僅用於兩國所指屬地,不能用於別處。是為中英續議滇緬界務商務條款。

是年又與英議接滇、緬邊界陸路電線條約。尋又議藏、印條款。二十一年夏,中、日和議既成,法索雲南普洱徼外猛烏、烏得兩地。英使歐格訥以兩地屬緬江洪,指為違約,欲中國將八募北野人山地,由薩伯坪起,東南到盞達,西南順南碗河折向瑞麗江,循江至猛卯,向南至工隆、八關、科幹皆在內,讓歸英。不許。英忽請允西江通商,再議野人山地,許之。復要求在肇慶、梧州、桂林、潯州、南寧五府設立領事,佛山、高要、封川、南新墟等處停泊輪船,由廣州澳門出入。中國以野人山地減索無幾,而通商口岸太多,且桂林在北江之北,潯州、南寧在藤江、龔江上游,並非西江,豈能強索?阻止之。英外部又以北丹尼、科干兩地原屬緬,為前薛福成定界時誤畫入華,求索回;又請於騰越、順寧、思茅三處設領事;及緬甸現有及將來續開之鐵路接入中國;又請援照俄、法條約利益,於新疆設領事。再三駮論,始允將新疆設埠及援照俄、法利益一節刪去;滇、緬接路一節,改為俟中國鐵路展至緬界時彼此相接;滇界領事一節,改為將已設之蠻允領事,改駐或順寧或騰越一處,其思茅領事,系援利益均霑之例,非英獨創;其野人山界線,改為南坎一處作為永租,餘俟兩國派員勘定。惟西江通商一節,允至梧州而止,梧州之東,祗開三水縣城、江根墟兩地,商船由磨刀門進口,其由香港至廣州省城,本系舊約所許,仍限江門、甘竹、肇慶、德慶四處,遂定議立中緬條約附款。時二十三年正月也。是年英主維多利亞在位六十年,命張廕桓前往致賀。

二十四年四月,議展香港界址至九龍城,租期九十九年。五月,英租威海衛。初,威海為日本軍占領,英人致書日相伊藤博文,原代繳償款,要求早撤兵。會我償款繳清,北洋大臣派員收回,英使竇納樂遂請租借。政府派慶親王奕劻、尚書廖壽恆與立約,文云:「以劉公島並在威海灣之群島及威海全灣沿岸以內十英里之地租與英國,威海衛城墻以內仍由中國自行管理。又所租於英國之水面,中國兵船無論在局內局外仍可享用。」並另備照會,謂「中國重整海軍,船舶可泊港內,請英人代為訓練」。

是月,英領事因沙市教案,照請開辦湖南通商口岸。張之洞以岳州系奉準開埠,尚須體察詳商辦理,致總署請商緩。總署擬推展兩年,英使不允。總署以湖南系我自開口岸,與他口不同,不許,亦不許牽入沙案。久不決。二十五年五月,駐漢英領事牒鄂督張之洞云:「本國巴管帶欲乘威拉小兵輪往洞庭湖上下游,先至岳州,再往湘陰、長沙,後往沅江、龍陽、常德、安鄉等處。」張之洞以條約並無兵輪準往內地之說,阻之。十二月,英參贊璧閣銜欲由湖南長沙取道常德、永順入川,過酉陽州抵重慶。張之洞復阻之。尋允改由宜昌入川。

二十六年,拳匪起。五月,漢口英領事法磊斯見張之洞,面述沙侯電云:「如長江一帶布置彈壓,英原以水師相助。」張之洞答以當與江督劉坤一力任保護,不須外助,力阻之。時英以保全東南商務為辭,已派水師提督西摩入長江。七月二十日,聯軍入京,英軍從廣渠門入,各據地段。八月,英與德結保護中國商務土地條款,又欲代中國理財、練兵,卻之。西摩欲派小輪入襄河探水道,張之洞阻之。既復議浙衢教案。時湘案未結,英又欲派兵輪往,屢阻之。是年英君主逝,國書致唁,皇太后復專電吊唁之。

二十七年,既與各國議定和約大綱十二條,四月,英人請直隸、山西停考。張之洞以所請與大綱條約第十條不符,辨駮久之,七月,始定議。八月,英商立德欲在川河行駛輪船,沿江購地七處,請地方官註冊。英領事照會到鄂,以條約非通商口岸,無準洋商置買地基產業之條,拒之。

十一月,英使馬凱赴江、鄂,與劉坤一、張之洞商議免釐,答以去年在京與赫德籌議洋貨稅釐並徵,必須稅至值百抽二五方能免釐。馬凱允加進口稅而不欲多加。於是朝命尚書呂海寰為辦理商約大臣,侍郎盛宣懷副之,並命劉坤一、張之洞皆與議。研商數月,海寰等乃會奏:「臣等奉命會辦商約,英使馬凱開送約稿二十四款,聚議六十餘次。加稅免釐一款,業經奏明,允如所請。此外各款,均經臣等隨時會奏。惟第十款內港行輪,續經妥定章程,第十一款通商口岸權利,共議列三條,馬凱自請刪除。統核所索二十四款,駮拒未允者七:曰洋鹽進口,曰內地僑居貿易,曰郵政電報,曰設海上律例,曰整頓上海會審衙門,曰口岸免釐界限,曰貨物同在一河免復進口稅。議定後而又刪除者一:曰通商口岸利權歸入加稅免釐款內並議。藉為抵制者五:曰新開口岸,曰減出口稅,曰三聯單,曰子口單,曰常關歸新關筦理。商允改妥者十一:曰存票,曰國幣,曰廣東民船輪船稅則一律,曰華洋合股,曰整頓珠江、川江,曰推廣關棧,曰保護牌號,曰加稅免釐,曰礦務章程,曰內港行輪,曰米穀禁令。此就馬凱原議款目分別刪改歸並者也。臣之洞等復向馬凱索議,彼允入約者三款:曰治外法權,曰籌議教案,曰禁止嗎啡。皆我補救國計民生要圖,幸就範圍,實有裨益。馬凱於定議後補請入約者二款:曰修改稅則年限,曰約文以英文為憑。查系照舊約辦理,為約中應有之義。共計十六款。臣等按馬凱所請加稅之款,意在不得抵原撥釐金五百萬以外之洋債賠款及挪作別用,恐各省再將貨物收捐,業已先後奏明。本定八月初二日畫押,馬凱又接英廷來電,必欲增敘詳明,以慰加稅洋商之意。駐英使臣張德彞亦稱英外部謂擬加之稅務須降旨歸督撫提用,否則不能畫押,似英廷用意總慮稅加而釐不能撤。臣等詳細審度,彼雖請全數撥還各省,而內敘各省向解北京及應還洋債仍如數照撥。我復照會,聲明應撥各項即留存海關,聽候戶部與各省商定抵解。將來戶部如何商定派撥劃抵,由我自主,彼亦無從過問。且現議償款易金還銀,正以我財力竭蹶為言,則加稅聲明祗抵裁釐,不涉賠款,可見毫無盈餘,藉可杜列國之口實。畫押已延多日,即於八月初四日亥刻,會同英使馬凱在上海畫押蓋印。」疏入,報聞。

同時又續改內港行輪章程十款。自滬蘇、滬杭、蘇杭三線外,江蘇則有海門線自上海東北至海門、蘇鎮線自蘇州至鎮江、鎮寧線自鎮江至江寧、鎮清線自鎮江至清江;浙江則有餘姚線自寧波至餘姚、舟山線自寧波至舟山、海門線自寧波至臺州之海門;安徽則有廬州線自蕪湖至廬州;江西則有南昌線自九江至南昌;湖北則有武穴線自漢口至武穴、襄河線自漢口至仙桃鎮、岳州線自漢口至湖南岳州;湖南則有湘潭線自岳州至湘潭、常德線自岳州至常德;而福建亦有水口、梅花兩線皆發自福州。又議湖南辰州府斃英教士案。是月,英交還關內外鐵路。是年,英皇愛惠將加冕,特命貝子載振為專使往賀。先期遞國書,向例須候各國專使齊集同見,英皇特定單班先見。屆期行鞠躬禮,英主答禮,各述頌詞、答詞。

二十九年春二月,與英訂滬寧鐵路借款合同。初,英於光緒二十四年欲攬自滬至寧鐵路,令英商怡和承辦。已議草約,旋以拳匪亂延緩。久之,始定議以年息五釐,借英金三百二十五萬鎊。張之洞乃上奏,言:「借英金三百二十五萬鎊,虛數九扣,年息五釐,五十年為期,準其分次印售金鎊小票。如中國國家有款撥給,或中國紳富集資原購,借款總數便應照減,撥還淞滬鐵路工價後,即將已成車路暨備造滬寧全路作為借款抵押,所獲餘利,銀公司得五分之一,即照售票應分之數,另給餘利憑票,十二年半後,每百鎊加給二鎊半,隨時可將小票贖還,二十五年後,便照一百鎊原價取贖,毋庸加給。至餘利憑票年期屆滿,分給餘利即時作廢,毋庸取贖。造路期內,就本付息,路成以後,贖票撥本,悉在鐵路進款支給。全路訂定五年全竣。設無事故,逾此期限,銀公司五年內應得餘利全行扣罰。上海設立總管理處一所,本省督撫與督辦大臣會派總辦兩員,會同英員專理工程,另由南洋大臣加派一員,職銜相當,隨時查閱賬目,稟報督撫稽核。洋工司祗管工程,不能干預地方公事。凡所建築,悉應順洽華人意見,尊敬中國官員。借款期內,不收專稅。如日後中國推設各項稅捐,如印花稅之類,別項商稅一律徵收,則滬寧鐵路亦應照準。全路雙軌。地畝總公司自備,仍由銀公司墊款,另須購地於標界之外,預備日後推展商務所必需,一並加售小票,綜計不得逾英金二十五萬鎊,年息六釐,在中國應得餘利項下支給,不能仍由鐵路進款支付。此項加售購地小票,並無年限,隨時可以取贖。造路購用中外材料,按照西例,每百給五,此外別無絲毫加用。漢陽鐵廠自造料件,訂明侭先購用。凡遇調兵、運械、賑饑各事,照核定車價減半給發,侭先載運。侵礙中國主權,概不得經由此路。正約簽定,草約作廢。十二個月不興工,即將正約註銷。中國祗認英國銀公司,不準轉與他國及他國之人民。」報可。十月,又與英訂滬寧路電交接辦法合同。

三十年四月,英新任水師提督率大小兵船十艘抵滬,欲進長江。張之洞聞之,電阻,英提督僅以四艘入江,至江寧而止。是年與英訂保工條約。時英於南斐洲新屬欲招華工開礦,政府援咸豐十年約,與訂專章。至是,約成,遣領事於華工駐在地善視之。三十一年四月,與英續訂滇緬電線約款。英派委印度電務司貝林登為議約專員,電政大臣袁世凱委道員硃寶奎與議。貝林登又請添造江通至思茅副線一條,不許。遂定議簽押。

又與英訂道清鐵路借款行車合同。初,英使向總署索英商承造鐵路五條,不許。英復援礦務合同許有修築鐵路由礦山運送礦產至河口以達長江,欲修澤襄鐵路。嗣以襄陽至漢口水道不能通暢,請改道澤鐵路,欲在河南懷慶府與盧漢銜接;渡河後,折入安徽正陽關以達江蘇江浦縣之浦口,改名懷浦鐵路。總署以懷浦遠跨豫、皖,名為緯路,實已斜亙南北,隱然增一幹路,以為有妨盧漢,仍不許。英使乃請修由澤州至道口鐵路,許之。鐵路大臣盛宣懷等與議借款,為目二十一,行車款十,英金七十萬鎊,五釐行息,九扣交付,折實六十三萬鎊。又同時訂擬設山西鎔化廠及合辦礦務合同,並請修廣州九龍鐵路。英使復請借款合同須由外務部將上諭照會立案,方允畫押,許之。

三十二年四月,與英訂藏印條約。初,中國於光緒十六、十九兩年與英訂藏印條約,然藏、哲界牌既未建立,英人入藏細則又久未定。二十九年,印督遣兵入藏。次年春,度大吉嶺,據江孜;其夏,遂入拉薩。及達賴私與英訂約,駐藏大臣有泰始入告,而英、藏約已成。政府命有泰與英議廢約,無效。復命外務部左侍郎唐紹儀為議約全權大臣,赴印度,與英外部專使費利夏會議。費利夏欲我認印藏新約,方允改訂,紹儀不可,英遂欲停議。紹儀不得已,與商訂約稿六條。外務部王大臣以約內第一款有「英國國家允認中國為西藏之上國」一語最有關系,電紹儀使改「上國」為「主國」,費利夏持不可。約久未定。九月,召紹儀回京,而以參贊張廕棠為大臣,接辦約事。外務部商諸英使薩道義,刪約稿第一條,英政府允諾,而其他條款則不容再改。然費利夏仍堅持初議,數促廕棠畫諾,即第一條亦不能增減一字,廕棠力拒之。會英廷新易政府,繼任者乃飭薩道義在京續商。久之始議訂正約五條。

未幾,片馬交涉又起。片馬處滇、緬交界之間,屬於騰越。英並緬甸,至是兩國會勘境界。至片馬附近,各執為本國土地,久不決。時英又欲遣工程師勘騰越至大理中間道路,請中國保護。滇督丁振鐸照會英領事,以滇現奏設公司自行修造,與前會勘時情形不同,請勿派往。英使硃爾典旋照會外務部,云:「據光緒二十八年二月初七日照會,英得有承造新街至騰越鐵路之權,而承辦此段較短之鐵路,英政府不能視為足抵光緒二十四年三月準法政府或法政府所指之法商修造勞開至雲南府鐵路之利益。」外務部覆,引中緬附約,謂:「第十二條載明中國答允將來審量在雲南修建鐵路與貿易有無裨益,如果修建,即允與緬甸鐵路相接。是該處中國境內鐵路應由中國自行審量。迨光緒二十七年九月十九、十月十六等日,本部先後復薩前大臣照會,均一再守此旨,並聲明法國鐵路由雲南邊界修至雲南,本為條約所準,與滇緬約意不同。緣兩國交涉各有約章可據,固不能相提並論也。逮二十八年二月初三日準薩前大臣照稱本國署理騰越烈領事不日將往雲南府,與滇督面商鐵路邊界各事宜,滇緬鐵路相接為振興商務之舉,凡在滇省,允給法商之利益,應一體允給英商。本部當以原照所稱面商鐵路邊界各事宜,又稱滇緬鐵路相接,曰邊界,曰相接,均系按照原約立論,故於是月初七日以據咨滇督也。嗣於本年正月準滇督文,稱準英務領事照會,接烈領事來電,奉緬政府電,擬由新街達騰越修造一鐵路,以便商人運貨,先派公司勘明可否能修,再議商辦。當復以派員會勘,各修各路、各出各費等語,是滇與英領事所迭次議商者,亦均扼定約章鐵路相接之一語,毫無刺謬。本年五月,滇督奏請修理騰越小鐵路,籌款自辦,奉旨允準,原期中國雲南境內次第修建,以符與緬路相接之權。乃貴大臣來照,以為英政府得有承造新街至騰越鐵路之權,並引二十八年二月初七日之文為據,而以允給法商之利益相比例,實與中緬附約暨本部迭次照會之意不符。」蓋不認英有造騰越鐵路之權也。

三十三年正月,與英訂九廣鐵路借款正合同。初,英既得九龍,即請承修由廣州至九龍鐵路。總署令督辦鐵路大臣盛宣懷與英商怡和洋行議辦,已簽草合同五條,旋因事未行。至是,又以為請。外務部電知粵督岑春煊,以此項草約雖云仿照滬寧辦法,而滬寧路長費鉅,九廣路短費少,情形不同,應查酌第二款,熟權利弊,派員與中英公司研商,以符原議。四月,與英公司代理人羅士、濮蘭德議,岑春煊欲照津榆鐵路辦法。濮蘭德以成議在先,不允,由粵到京,與唐紹儀等接議。久之約成,議借英金一百五十萬鎊,照虛九四折納,年息五釐,以本路作抵押,三十年為期滿,十二年半後按照列表分期還本。二十五年以前,如欲於表額外多還股本,每英金一百鎊加還兩鎊半。中英公司代售此項股票。其股票填明價值若干鎊,由中國駐英大臣與英公司商定,所有建路及一切工需,均由粵督督辦。其重要職司,應用中國人,允當開工時,即於廣州設立總局一所,總理造路行車各事,由總督派中國總辦一人管理,佐以英國總工程司及總管帳各一人,均由總督核準。英公司辦事出力,給予酬金三萬五千鎊,兩期交付,其一切用錢暨酬勞費均在內。並聲明此路確系中國產業。倘自本合同簽定之日起,八個月並未興工,即作廢紙。所載權利,均不得讓給他國,中國亦不得另建一路以奪本路利益。旋簽押。

六月,政府命湖南巡撫岑春蓂查辦雲南與英畫界失地案。先是雲貴總督丁振鐸委候補知府石鴻韶與英領事烈敦會勘騰越北段尖高山以北界,從尖高山起向北勘,越高黎共雪山直抵麗江府所管地。烈敦執定以大啞口為界,石鴻韶執定以小江邊為界。貴州提學使陳榮昌奏參石鴻韶定界有失地事,政府命岑春蓂查辦。春蓂派候補道沈祖燕往勘,旋覆稟云:「卷查烈領事此次所勘之界,系從尖高山起,東至札山,過狼牙山、磨石河頭、搬瓦丫口、姊妹山、大啞口、茨竹丫口、片馬丫口直上高黎共雪山北往西藏。所云大啞口,即為恩買卡河與潞江中間之分水嶺。其照會石道有云,由明光河頭直上高黎共雪山頂,由山頂北往西藏,凡水入金沙江者,概歸緬甸管理等語。若不幸照此定界,則是由滇而蜀而藏,邊界之地所被其割去者,當以數千里計。外務部所謂『直是分割華境,是斷不能允從,可無庸置議』者也。若石道所擬以小江邊為界,系從尖高山起,由磨石河頭直上歪頭山,過之非河,經張家坡,登高良共山,又抵九角塘河,順小江邊,復另行橫出,上至小江源,又至板廠山為止。查其所勘之界,於騰越、保山、雲龍、龍陵各屬土司素所管轄之地,數百年來向化中國者,一旦棄去不少。又言北段界務,自以外務部所言之界線,由尖高山起至石我、獨木二河之間,循恩買卡河至小江西恩買卡河之東之分水嶺為界。按此嶺當是他戛甲大山,最為持平。且英使本有以小江即恩買卡河以東之分水嶺作為定界,又云天然界線系自東流入恩買卡河即小江諸江之分水嶺等語,與此正合。則此次勘界,即於恩買卡河循流而行,至小江止,已足滿意。且所勘滇、緬北段,本祗為騰越與野人山之界,則必執定騰越諸土司之屬地及野人山之分界處以畫界,自是一定不易之理。而與小江即恩買卡河以東之分水嶺,又自東流入恩買卡河,即小江諸江之分水嶺,並與譯出薛星使福成二十年簽押英文圖內之恩買卡分水嶺,其部位亦均相符合。石道並不先自詳審界限,而惟處處曲徇,以致失誤,此真為人意料所不及者也。查此次勘界,英使既言以小江即恩買卡河以東之分水嶺為界,又言自東流入恩買卡河即小江諸河之分水嶺,既明曰以東,又明曰自東流入,何以任烈領事之混為西流,竟勘至狼牙山迤北至大啞口而止?此其誤者一。又外務部覆稱明有『各守邊界』之文,此為甘稗地、茨竹、派賴燒殺之役而起,各守之地,自即在此。何以不實守此小江邊界之說,至小江順流而下,而反另向東行,指鹿為馬,再直上別尋一小江源至板廠山為界?此其誤者二。又英使所言天然界線,乃自東流入恩買卡河即小江諸水之分水嶺,而烈領事所勘,乃指恩買卡河與龍江之分水嶺,謂嶺之東所有溪河均入明光龍江,嶺之西所有溪河均入恩買卡、金沙江,以此嶺之東西為中、緬之分界。石道不能明據小江東流,力為駮斥,而乃以山形水勢則然一語,含混答覆,而竟任烈領事之隨意所指,東西自便。此其誤者三。且即如英使照會恩買卡河與潞江之分水嶺之說,此嶺即為大啞口,亦祗西勘至片馬丫口為止,何以任烈領事直上高黎共雪山,竟偕測繪王生,勘至麗江府屬蘭州邊界始回也?此其誤者四。又小江外如噬戛等寨,系騰越屬之茨竹、大塘土司所轄,籠榜系保山屬之登埂土司所轄,確鑿可據。乃烈領事照會言『貴道來示,謂已摒諸化外』,而石道覆稱又言『業經聲明久在化外』。石道責在勘界,並不援據力爭,而反先自認『久在化外』,實所不解。此其誤者五。又茅貢等寨原系滇灘屬土司所轄,本中國舊有之地,不過英兵曾經至此,並強收門戶稅而已,並非英人實已占為屬地,而中國有允認之明文也。乃石道照會謂『早經貴國辦過案件,不復管理』,竟絕不置辨。此其誤者六。至於大啞口外,如甘稗地等各處,烈領事欲仿三角地成案,作為永租。既欲議租,則已明認為中國之地,正可趁此力駮,使之無辭可遁。計大啞口外共有一十八寨,其地甚廣,豈可輕棄?且既認租,則茨竹、派賴燒殺一百十四命之案,明是入我中國之界,正可提議,使之不能諉卸,何以絕不辨論?此其誤者七。又狼速之地,甚為遼闊,一名狼宋。大理府志:『莪昌散處於狼宋、曹澗、趕馬撒之間,道光十八年準兵部議,以趕馬撒、曹澗等寨歸雲龍州管轄』,則狼速乃大理府屬境。若如石道所勘,另尋一小江源至板廠山為界,則不特噬戛等一十八寨摒諸化外,且並將狼速地一帶地方亦概棄之不問矣。此其誤者八。然此八者,甚害尚祗在滇省也。更有大誤足以為將來之後患者:一則小江外之狼速地一旦棄去,再北而為怒夷,其地踞龍、潞兩江之上流,東接維西、中甸,直通麗江,北與四川之巴塘、里塘諸土司相接,西北即可以通至西藏;一則高黎共雪山之地任其節外生枝,自往履勘,將來若果曲從,則即可從此高黎共雪山之頂,沿潞江、金沙江之上流由北直進,不特球夷、怒夷之地去其大半,即維西屬之鋪拉籠、西藏屬之擦瓦龍一帶皆將被其所侵占,所失之土地豈尚可以數計?」岑春蓂得覆,即據以入奏。上諭革石鴻韶等職,仍不允。

時因津鎮鐵路借款,直隸、山東、江蘇三省商民欲廢約,英不允,允改章。德與英同。英又因鄂境修造粵漢、川漢兩路需款,欲借款於中國,卻之。是年,山西商務局與英福公司議定贖回開礦制鐵轉運合同。初,晉省礦由晉商與福公司商人羅沙第訂立合同。旋於光緒二十四年復由商務局紳商與福公司改訂借款章程二十條。三十一年,又經盛宣懷續立合同四條。案久未結。至是商務局員紳並全省代表各員在京開議,訂定贖回自辦合同十二條,贖款行平化寶銀二百七十五萬兩,由山西商務局擔任,按期交清。

三十四年二月,與英訂滬杭甬鐵路借款合同。先是滬杭甬鐵路已立有草合同四條:一,訂草約章程,與滬寧鐵路章程一樣;二,將來訂正約,仍與嗣後商定核準之滬寧正約一樣;三,從速測勘;四,如有地方窒礙之處,即行更正,俟訂正約,即會同入奏。至是浙江紳士籌辦全省鐵路,欲廢前約,收回自辦。英使不允,因命侍郎汪大燮等與英公司改商借款辦法,久未決。於是政府再命侍郎梁敦彥接議,分辦路、借款為兩事,路由中國自造,除華商原有股本侭數備用外,約仍需英金一百五十萬鎊,即向英公司籌借,按九三折扣交納,年五釐息,以三十年為期;並聲明如所收此路進項不足,由關內外鐵路餘利撥付;凡提用款項,均由郵傳部或其所派之人經理;此鐵路建造工程,以及管理一切之權,全歸中國國家;英公司代購外洋材料機器,以三萬五千鎊作為酬勞,一切用銀均在內;選用英總工程司一人,仍須聽命於總辦等語。遂定議。九月,與英訂藏印通商章程。是年,借英匯豐及法匯理銀行款,收回京漢鐵路。

宣統元年四月,督辦鐵道大臣張之洞與英及德、法、美四國銀行訂粵漢川漢鐵道借款草約,豫定六百萬鎊。會之洞卒,復與盛宣懷立約續成之。又與英及德兩公司續訂津浦鐵路借款合同,共二十四款,借英金五百萬鎊,年息五釐,路工四年造竣。二年,英人以兵力據片馬,設砲臺於高黎貢山,侵踞小江以北茶山土司地。滇人大憤,各省人亦起應之,遂電政府請力爭。滇督李經羲亦請外務部與英使交涉,英卒不退兵。三年,復派員與英劃境,不省。是年度支部尚書載澤與英及德、法、美締結一千萬鎊借款契約,以改革幣制及東三省興業為詞,是為四國借款契約。又與英訂禁煙條件。原議十年遞減,至是中國以為國內栽種吸食漸已減少,欲縮短年限禁絕,與英特訂專條,期印藥不入中國。而第三條又言廣州、上海二口為最後之結束,不能驟禁,於是煙卒不能禁矣。


\end{pinyinscope}