\article{志一百二十六}

\begin{pinyinscope}
○交通三

△電報

電報之法,自英吉利人初設於其國都,推及於印度,再及於上海。同治十三年,日本犯臺灣,兩江總督沈葆楨疏言電報之利,詔旨飭辦,不果行。光緒五年,直隸總督李鴻章始於大沽、北塘海口砲臺設線達天津,試行之而利,明年因有安設南北洋電報之請。先是同治間,英使阿禮國請設電線於中國境內,力拒之,乃已。九年,其使臣威妥瑪復申前議,易陸線為水線,自廣州經閩、浙以達上海,爭之數月,卒如所請。嗣是香港海線循廣州達天津,陸線達九龍。而丹國陸線亦由吳淞至滬上,駸駸有闌入內地之勢。

天津道盛宣懷言於鴻章:「宜仿輪船招商之例,醵集商股,速設津滬陸線,以通南北兩洋之郵,遏外線潛侵之患;並設電報學堂,育人才,備任使。」鴻章韙之。明年,疏言:「用兵之道,神速為貴。泰東西各國於講求槍砲之外,水路則有快輪船,陸路則有火輪車,而數萬里海洋欲通軍信,則又有電報之法。近來俄羅斯、日本均效而行之。故由各國以至上海,莫不設立電報,瞬息之間,可以互相問答。獨中國文書尚恃驛遞,雖日行六百里加緊,亦已遲速懸殊。查俄國海線可達上海,旱線可達恰克圖。欽使曾紀澤由俄國電報到上海,祗須一日。而由上海至京城,輪船附寄,尚須六七日到京。如遇海道不通,由驛必以十日為期。是上海至京僅二千數百里,較之俄國至上海數萬里,消息反遲十倍。倘遇用兵之際,彼等外國軍信速於中國,利害已判若徑庭。且其鐵甲兵船,在海洋日行千餘里,勢必聲東擊西,莫可測度,全賴軍報神速,相機調援,是電報實為防務所必需。現自北洋以至南洋,調兵饋餉,在在俱關緊要,亟宜設立電報,以通氣脈。如由天津陸路循運河以至江北,越長江以達上海,安置旱線,即與外國通中國之電線相接,需費不過十餘萬兩,一半年可以告成。約計正線支線,橫亙三千餘里,沿路分設局棧,常年用費,先於軍餉內墊辦。辦成後,仿照輪船招商章程,擇公正商董,招股集貲,俾令分年繳還本銀。嗣後即由官督商辦,並設電報學堂,雇用洋人教習中國學生,自行經理,庶幾權自我操,歷久不敝。」疏入,報可。逾年,工竣,以宣懷董其事。

未幾,英、法、德、美各使擬設萬國電報公司於上海,增滬至香港各口海線。英使格維納並援案請增上海至寧波、溫州、福州、廈門、汕頭海線。鴻章言:「宜令華商速設沿海陸線,以爭先著,使彼無利可圖,庶幾中止。且從此海疆各省與京、外脈絡貫注,實與洋務海防有裨。即商民轉輸貿易,消息靈通,為利更大。」從之。而蘇州至浙、閩、粵陸線因之告成。其時香港英商方欲設水線至廣州,粵督曾國荃亟造陸線以遏之。於是港線不得侵入粵境,英線不獲造至福州。而上海丹線、九龍英線先後毀去,或貲購之。沿海電線,其權悉操於中國之手。此因外線之侵入而次第創設者也。

當沿海陸線未設之先,海疆萬里,消息阻絕,緩急無以為備。御史陳啟泰上防海六策,其一言:「洋面既派兵輪分駐,即不可不設電線以通消息。議者必以不急之務虛糜鉅款為疑。不知非常之原,斷非省嗇所能集事。即以目前而論,越南情形,每藉各國新聞紙以為耳目。今年朝鮮之變,非由日本發來電信,中國尚不得知。軍情緊急,日夕萬狀,郵傳迂緩,既恐有誤機宜,藉助外人,事體更多窒礙,自不如招雇洋匠自行安設之為愈。中國電報,似宜推廣各省海口,凡兵船寄椗之處,一律開辦。廣東瓊州之線逕達越南,奉天旅順之線逕達朝鮮,總期脈絡聯貫,呼應靈通,遇有警報,瞬息可至。」下所司議行。十年,法、越事起,海防急,設線北塘以訖山海關,遞及於營口、旅順。江督左宗棠則設長江線以通武漢,粵督張樹聲則設廣西線以達龍州。二十一年,中日戰事亟,慮直東一線有阻,接設老河口至西安線。是役江蘇增上海至獅子林、金山衛、乍浦,清江至青口、板浦,揚州至通州、泰州,鎮江至圌山關、天都廟,崇明至吳淞等線,而奉天至仁川電線先成於十一年。臺灣以瀕海要區,十四年亦水陸線並設焉。此因海防緊要而次第安設者也。

滇、桂密邇越南、緬甸,邊備為急。滇省電線,其始僅通鄂通蜀,與南寧接線之議,光緒十一年得請而未果行。十三年,滇督岑毓英復言:「由緬入滇,以騰越為入境門戶,猶蒙自之於越南也。今英國有開辦通商之請,自當先事籌維。擬就粵西工匠到滇之便,即將省城至騰越一路安設電線,以通英緬聲息。」時粵督張之洞亦言:「廣西南界接壤滇邊,桂、滇皆西鄰越南,滇則西接緬甸。若僅恃由鄂入滇一線傳達電音,設有雷雨折斷電桿,阻滯堪虞。且遇有軍務之時,由滇、川、滬、鄂展轉至粵,恐有交會壅滯之患。已商之滇督,自剝隘至蒙自,由粵接造,並增騰越之線。」疏入,報聞。蓋剝蒙設線,所以備越南;騰越設線,所以備緬甸也。

吉林、黑龍江偪處俄疆,邊防尤要。十五年,自吉林省城設線至松花江南岸,歷茂興站、齊齊哈爾、布特哈、墨爾根、興安嶺、黑龍江以達黑河鎮,從練兵大臣穆圖善之言也。十八年,陜甘總督楊昌濬言:「新疆西北鄰俄,西南與英屬部接壤,文報濡滯,貽誤必多。宜由肅州設線至新疆省城,及於伊犁、喀什噶爾。」宣統元年,桂撫張鳴岐疏陳設柳邕電線二千三百餘里。俱得請。此因邊備而增設者也。

初,奏設南北洋陸線,北端僅至天津。法事將起,出使大臣曾紀澤請接營近畿電線,謂可壯聲威以保和局,靈呼應以利戰事。事下所司,與鴻章議展拓之法。鴻章言:「神京為中外所歸鄉,發號施令,需用倍切。前於創辦電報之初,頗慮士大夫見聞未熟,或滋口舌,是以暫從天津設起,漸開風氣。其於軍國要務,裨益實多。今總理衙門與曾紀澤皆以近畿展線為善策,擬暫設至通州,逐漸接展至京。」允行。逾年,津線遂逾通州達京師。自時厥後,各省咸知電報之利。或本無而創設,或已有而引伸。其尤要之區,則陸線、水線兼營,正線、支線並設,縱橫全國,經緯相維。直、蘇、粵、桂、滇、魯、鄂諸省,設局多至二十餘所,餘省亦十餘局或數局有差。其互相銜接者,京師之線所達,曰庫倫、濟南、太原。天津之線所達,曰奉天。奉天之線所達,曰天津、旅順、吉林。吉林之線所達,曰海參崴、齊齊哈爾、奉天。黑龍江之線所達,曰吉林、海蘭泡。江蘇之線所達,曰京師、蕪湖。安徽之線所達,曰江寧、九江。山西之線所達,曰京師、西安。山東之線所達,曰京師、開封、清江浦。河南之線所達,曰京師、濟南、西安。陜西之線所達,曰開封、太原、蘭州、漢口。甘肅之線所達,曰迪化、西安。新疆之線所達,曰蘭州。浙江之線所達,曰上海、福州。江西之線所達,曰廣州、蕪湖、河口。湖北之線所達,曰九江、成都、長沙、鄭州。湖南之線所達,曰漢口、桂林。四川之線所達,曰漢口。福建之線所達,曰杭州、廣州。廣東之線所達,曰福州、梧州、九江。廣西之線所達,曰長沙、廣州。雲南之線所達,曰漢口、重慶、八莫、南寧。貴州之線所達,曰重慶。外蒙則達京師、張家口焉。瀕海之區則設海線。直隸自大沽以通之罘。江蘇自上海東通長崎,北通之罘、大沽,南通廈門、香港。廣東自香港通海防、新嘉坡、廈門、上海、馬尼喇。山東自之罘通大沽、旅順、威海衛、青島、上海。福建自川石山通臺灣淡水,自廈門通上海、香港。蓋總計陸線之設,不下四萬里有奇,而水線不與焉。

電報設局,亦如輪船招商之例,商力舉辦而官董其成,謂官督商辦也。津水扈一線,其始倡以官帑,未幾即歸商局,醵貲至二百餘萬。而各省電線不盡由商辦者,良以商人重利,入貲則權子母、計盈虧,其於海防邊備情勢緩急,國內交通利便與否,不以措意。往往一線,官辦商辦,參互錯綜,大率以官辦補商辦之不足。兩粵電線,廣州至龍州則屬之官,至梧州則屬之商。欽、廉、雷、瓊及鎮南關、虎門,則官商協力。而滇線一自鄂入,一自蜀入,一自桂入。西安迄嘉峪關、甘、新、奉、吉、黑等省,通州至承德,陸線俱官為之。此類是也。然由水扈達粵之線,本為防止外線而設,需費四十餘萬兩,咸由商力措備。其時香港英人並欲引線達廣州,亦賴華合公司預設線至九龍,其謀始戢。方華合公司設線九龍也,華民抗拒,英商撓阻,其勢洶洶。公司商人何獻墀等排眾難而為之,不為所屈,卒底於成。中日戰事棘,引襄陽線千餘里直達西安,俾京、滬軍報不至梗阻。而張家口至恰克圖一線,以俄使援約相促,亦由商局集金六十餘萬兩,接線二千七百餘里,經營至二三年之久,工鉅費繁,為全國最。此外造成之線,不能里數,其所裨殆非淺鮮矣。

二十五年,大學士徐桐言電報局獲利不貲,並無裨益公家之實。廷臣亦有以招商、電報各局假公濟私為言者。俱下協辦大學士剛毅查復。剛毅時以事銜命赴蘇,尋疏陳:「電局自恰線成後,所虧至鉅,俟有贏餘,歲輸南北洋學款十二萬四千兩。」報可。明年,廷臣復言電局利權太重,宜遴員接辦。詔飭宣懷按年冊報收支款目,官電應免收費。宣懷上疏,略言:「電局本系集華商合眾之力,以與洋商爭衡,旁觀每驚為大利所叢,其實析分千百股商,仍皆寸寸銖銖之微利。近年電線開拓日廣,則局用及修線養線之費亦日增。上年因中俄條約,接造恰克圖之線用費六十餘萬兩,未請官款,悉系電商集貲辦成。沙漠荒僻之區,絕少報費,而常年用數尤鉅。至本年應辦之工,因辦理鐵路,盧溝橋至保定線已造成,又須造保定至漢口幹線。因辦理海防,乃須造寧波至溫州之線。總理衙門因洋人之請,則須造山東泰安、沂州之線。此外各路加線要工,絡繹不絕,官款並無可籌,皆借股商之力,以赴公家之急。總局收支各賬,均系按年刊布。各局詳細坐簿,亦任股商隨時查閱。一出一入,眾見眾聞,非如官中所辦報銷,出於一二人之手者可比。原奏所疑各節,似屬不知此中原委。至官報之費,前定章程,擬一半報效,一半給貲,期於官商兼顧,持久不廢,仍宜照舊辦理,以維大局。」報聞。

宣懷時綜司輪、電兩局,疊被指摘。二十八年,言於直督袁世凱:「電報宜歸官有。輪船純系商業,可易督辦,不可歸官。」世凱謀諸執政者,以為然,聞於上。尋命世凱督辦電局,候補侍郎吳和喜副之。明詔發還商股,不遽予行。眾商洶懼,爭欲持券售之外人。宣懷力遏之,乃已。尋詔原有商股一仍其舊,蓋其時僅易一商股官辦之局而已。

三十四年,郵傳設部已二年,將以全國電局為實行部轄之計。郵傳部尚書陳璧疏言:「電報為交通全國機關。各國電報之權皆操諸國家。中國電報,創始原歸商辦。而光緒初年,商股微薄,仍賴官力以為補助,非完全商辦也。歷年獲利,約計五六百萬。果使全國交通推行無阻,則富商即可富國,亦何必別議更張?乃觀商線所至之處,皆屬市鎮都會,而邊遠省分,如雲、貴、廣西、甘肅、新疆,商人以無利可圖,均推歸官辦。雖商力實有未逮,而顧私利、忘遠略,實悖朝廷立部之初心。衡以中國近狀,自非改為官辦,無以定區畫之方,即末由收擴充之效。東西各國,電線如織,策應靈通,故伏莽方生,旋就撲滅。中國電報,無論要荒,即腹地稍僻者,亦多缺而未舉。一旦有事,道途修阻,聲息不通,實於軍務有礙。況當百度維新,外交內政關系非輕,稍滯交通,輒形捍格。近來科布多、川、藏、蒙古、閩、浙、江西、蘇、松紛紛請設電線。本年四月,奉旨迅設貴陽至義興電線。又陸軍部以秋間江、鄂各軍在安徽會操,請設安慶至太湖電線。外務部請設川、藏通印度電線,以為收贖英人江孜線路張本。湖北官電局以賠累不堪,請改歸部辦。紛來沓至,均為不可稍緩之圖。核計各省請設各線,不下萬有餘里,工程當在一百餘萬以上。且此萬餘里,半皆荒村僻壤,報務不多,增一線即賠一線之本,修一里即虧一里之費。前此添設雲、貴一二邊省電線,各股商尚慮虧損。今統籌荒瘠之區,更難著手。至利則歸己,損則歸公,恐亦無此情理。此展線之宜歸官辦者也。各省線路,待修者眾,朽敗難支,而陜、豫、閩三省尤甚。設遇軍興倉猝,何堪設想。現在遴員調查,通盤籌畫,尚有應移近鐵路者,有關系交涉亟須先占者,有文報日多應行添線者。次第修舉,工費浩繁,需銀約五六十萬兩。此項巨費,即盡括商股餘利息項,亦難支抵。此大修之宜歸官辦者也。中國報費昂貴,甲於全球。遠省一二字之費,幾與各國二十字相等。近據寧夏副都統志銳,請核減報費以利交通。又據赴葡部員周萬鵬稱,葡國公會亦以中國報費太昂為詞。自當酌減,使價目與各國略同,為入萬國電政會之預備。惟核減電費,以歲入三百餘萬元計算,若減一二成,即在五六十萬以上。若遞減至四五成,或減至與東西洋相等,為數尤多。此事一行,則商股年息恐不可保,餘利更不待言。此減費之宜歸官辦者也。凡此三事,實為電政今日最要之圖,即為商股今日最損之策。與其茍且因循,日積月累,致官商之兩病,曷若平價收贖,期上下之交益。實見夫今日電報有必須擴充之勢,即有不免折閱之時。在商人祗課贏餘,在國家必求利便。事實不同,斷難強合。臣等擬恪遵光緒二十八年諭旨,改為官辦,籌還商股。即由部備價收贖,於每股股本外特予加價,以示國家恤商之意。」奏入,允行。

八月,電股收贖完竣。陳璧疏言:「臣部收贖商電,酌核市值贖之,每百圓電股,給予一百七十圓。旋復從眾商之請,加價十圓,作為優待費。計共二十二萬圓。自頒發收贖章程後,旬月之間,共收回商股二萬一千四百餘股。其未到之五百餘股,委系外埠及內地僻處,遞寄維艱,擬請寬予限期,照章給價,提存現款,以便續領,仍給優待費,以示體恤。此後即全歸國有,與商無涉。收贖之款三百九十六萬,臣部暫由路款借撥,仍須另行設法歸還,以清款目。」又言:「電政為交通樞機,圖擴充方期發達。今既改歸國有,應將減費、展線、修線諸事次第整頓。而減價為中外眾目所睹,非實行籌辦,尤不足以饜人望而廣招徠。擬自光緒三十五年正月始,酌減電費二成,以所收商報約三百萬圓之額計之,即少收約六十萬圓,不敷在二十萬圓以上。減費之後,報費必增,可供挹注。而一時添線、修線,並擴充電話,在在需款。所增之數,必須抵撥,逐漸推廣工程之用。預算短額,擬暫由臣部各路餘利項下,每年分撥二十萬圓,以三年為限,自第四年起至六年止,每年勻還二十萬圓,一律還清。一轉移間,路款均歸有著,電政亦可漸興,不煩續借他款,實收財政統籌之益。」報可。自時厥後,事權統一,呼應靈通,每歲展拓電線三四千里以為常。而取值之廉,迥異疇昔,此則非商辦之所及也。

中國幅員遼闊,文報稽延,至於變起倉猝,往往因消息遲滯,坐誤機宜,釀成鉅患。歷朝變亂之起,大率以此。自有電報,舉向來音信隔絕之弊,一掃而空。若朝陽教匪之倡亂,雲南猛喇游匪魏名高之滋事,均因電報之告警,與軍事布置之迅速,得以立即剿平。而外則朝鮮之二次內訌,越南事變之先事防禦,亦惟電報是賴。此其明效大驗也。而然當創辦之初,鄉僻囿於見聞,外人多所撓阻,艱難曲折,乃克成功。設線之處,若系邊疆瘴癘、塞外荒涼之地,措手之艱,什伯內地。以故在事人員,得邀獎敘,而近省不得援例以請者此也。

至於意外之損壞,其事尤夥。貴州畢節鄉民之拆線;山西霍山鄉民之毀桿;湘省澧州民誤以電線為外人所設,集眾毀棄;陜之長武、乾州、醴泉、邠州、永壽,甘之涇州、平涼等處,人民謂旱疫為電線所致,拆毀殆盡。俱由地方官出貲修復,首犯有論重闢者。二十六年拳匪之變,京師至保定電線先為所毀,京津、京德繼之,山西、河南又繼之。馴至晉、豫、直隸、山東四省境內,蕩然無一線之遺。南北隔閡,中外阻塞,消息不通者數月。而外兵盤踞京、津,初設行軍電線,嗣擬設大沽至上海水線,以大東、大北兩公司主其事。宣懷密行作價,購其機器料物,屬於中國商局,其謀竟不得逞。宣懷尋請修復已毀各線。其經戰事損壞者,商局任之。晉、豫未有戰事,地方官保護不力,甚且指使拆壞者,援畢節、霍山之例,分別賠修。報可。三十年,東三省線再毀於日俄之戰。迨三十四年,總督徐世昌修復之。此已毀復修各線之大略情形也。

電報之利於交通,與鐵路相輔而行,缺一不可。然鐵路需費過鉅,每有興築,擬假外貲集事,非如電報工省費輕,商力已足舉辦,其借外債而成者,僅滬、煙、沽正副水線而已。光緒二十六年,外兵方據京、津,謀設大沽至滬水線。宣懷以其侵我主權,密向承辦之大東、大北公司購歸商局辦理。方是時,兩公司因利乘便,故昂其值。中國官商交困,復絀於力,於是以購價作為息借,分三十年償還。殆迫於勢之不得已也。前外人在中國設線,由商股購回者,如丹國所設之淞滬旱線、德國所設之京沽幹線、鐵路至天津支線是也。電報非僅達於國內已也,必行馳域外,而其用益宏。於是與外國通線,若法、若英、若俄,既訂通線費之約,並分訂聯合其價攤分之約,以相約束焉。

電局既日漸擴充,尤以培養人才為要。電報學堂創於光緒六年。嗣分設按報、測量、高等諸塾,以宏造就。二十五年,並設電話學科以附益之。

電話初名曰「德律風」。二十五年,宣懷疏言:「德律風創自歐、美。入手而能用,著耳而得聲,坐一室而可對百朋,隔顏色而可親謦欬,此亙古未有之便宜。故創行未三十年,遍於各國。其始止達數十里,現已可通數千里。新機既闢,不可禁遏。日本電報、德律風,統歸遞信省。學生教於一堂,機器出於一廠。中國之有德律風也,自英人設於上海租界始。近年各處通商口岸,洋人紛紛謀設。吳淞、漢口則請借桿掛線矣,廈門則請自行設線矣。電報公司竭力堅拒,但恐各國使臣將赴總理衙門要求,又滋口舌。一經應允,為患甚鉅。況西人眈眈逐逐,欲攘我電報之權利而未得其間。沿江沿海通商各埠,若令皆設有德律風,他日由短線而達長路,由傳聲而兼傳字,勢必一縱而不可收拾。不特中國電報權利必為所奪,而彼之消息更速於我。防備不早,補救何從?現在官款恐難籌措。臣與電報各商董再四熟籌,惟有勸集華商貲本,自辦德律風,與電報相輔而行。自通商各口岸次第開辦,再以次及於省會各郡縣,庶可預杜彼族覬覦之謀,保全電報已成之局。」報可。自是京師、天津、上海、奉天、福州、廣州、江寧、漢口、長沙、太原皆設之,此則連類而及者也。


\end{pinyinscope}