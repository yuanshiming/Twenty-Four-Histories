\article{列傳二百七十七}

\begin{pinyinscope}
忠義四

張錫嶸王東槐曹楙堅等周玉衡王本梧陳宗元

明善覺羅豫立世焜徐榮許上達等郭沛霖王培榮

硃鈞錢貴升徐曾庾蕭翰慶黃輔相福格等

孔昭慈徐曉峰袁績懋楊夢巖鄧子垣羅萱

侯雲登黃鼎陳源兗瑞春鄂爾霍巴許承岳潘錦芳

廖宗元劉體舒李慶福等李保衡徐海等淡樹琪

褚汝航陳輝龍夏鑾儲玫躬李杏春硃善寶

莊裕崧萬年新易舉等

張錫嶸,字敬堂,安徽靈壁人。咸豐三年進士,選庶吉士。四年,安徽巡撫袁甲三奏請總辦靈壁團練,授編修,記名御史。十年,命視學滇南。時回匪作亂,府縣多為賊踞。或勸乞疾,錫嶸毅然曰:「吾奉命之官,寧避賊耶?」叱馭不顧,竟到滇。省城被圍,幫辦防務。以丁母憂回籍。

曾國籓之徵捻也,駐軍臨淮,所部湘勇遣撤殆盡,僅存劉松山老湘營萬人,餘悉倚淮軍辦賊。淮軍新建平吳功,將領多自矜。國籓欲於淮北別募新營,使異軍蒼頭特起,備西北之用,而置將久難其選。值錫嶸服闋來謁,國籓大喜,密疏奏保治軍濠上,謂其誦法儒先,堅忍耐苦,足勝將帥之任。檄募敬字三營,隨湘軍戰守。時湖團有通捻者,國籓下令遷徙,錫嶸分別良莠,聯絡義圩。又以災賑日行泥淖中,圩民得蘇。

捻寇張總愚竄陜西,國籓調劉松山軍赴援,令錫嶸統三營與俱,至則解西安圍。復與賊戰於城西雨花寨,獨率百餘人沖擊,陷入賊陣,被十餘創而殞,時同治六年正月初六日。贈侍講學士,賞世職。

初錫嶸居京時,日鈔書數十紙,雖盛暑不輟。祿薄,日常一餐,無一介乞助於人。著有孝經章句讀、硃子就正錄、孝經問答行於世。陜西巡撫劉蓉奏錫嶸死事,言:「自到營以來,嘗著草履,與士卒同甘苦。文學之臣,能堅苦自奮如此,臣實惜之!」家極寒,國籓賻三千金養其孤,漕運總督吳棠刻其遺書。

王東槐,字廕之,山東滕縣人。生穎異,父病危,命饑寒毋廢學。居喪哀毀,母以遺言勉之,乃忍痛致力群經。屢空,日與昆弟食一餅。道光十八年,聯捷進士,改翰林,散館授檢討。二十四年,轉江西道監察御史,奏劾山東玩盜官吏,得實,升戶科給事中。時議開礦益帑藏,已允行,東槐敬陳列聖封禁成訓,謂:「開採者,上非良吏,下非良民,請緩其令。」事竟寢。巡視北城,王府役車,橫行中逵,懲治不貸。廉獲巨猾曹七,治如律。

三十年,應文宗登極求言詔,奏言:「捐例一開,鹽商輒請捐數十萬,運庫墊發,分年扣還。覈其虧短,都不堪問。即如道光二十年兩淮清查案內,欠至四千三百餘萬,是鹽商捐輸者,掩耳盜鈴之術也。又官員捐輸,現任居多,所捐之項即庫款,所虧之項即捐款。上年山東虧至一百四十餘萬,江、浙更甚,是現任官之捐輸,剜肉補瘡之術也。是事例不停,庫虧不止。若開礦之舉,臣曾疏陳不便,順天已停,而湘、贛等省試辦,驚擾百姓,利害莫測,則尤愈趨愈下也。查戶部歲入之數,四千四百餘萬,歲出之數,三千九百餘萬,經費本自餘裕。督撫整理有方,寇盜不作,則耗財者去一;邊防慎守,無生事以挑外患,則耗財者又去一;河防得要,長流順軌,不使更添別款,則耗財者又去一;州縣之官,斥貪墨,重清廉,陋規力裁,流攤永禁,則耗財者又去一。去此四害,而又罷不急之工,減無益之費,量入為出,而財患不足者,未之有也。」奉諭:「貴州仍令開採,餘省著督撫確查,果不便民,即奏停止。」左都御史王廣廕舉東槐忠鯁,升內閣侍讀學士,旋授湖南衡州知府,陛辭,帝面諭云:「汝樸誠,故任外事。」未至,升福建興泉永道。

廈門濱海,俗又敝,東槐刊硃子試吏泉漳勸俗文揭於衢,傳誦多感發。屬縣蠹役、訟師,嚴鋤治,惟與學舍生徒講析道義,則溫然以和。海上番估好違約放恣,東槐戒毋逾尺寸,為國全大體,尤人所難。咸豐元年,調湖北鹽法道,未赴,署福建按察使。舉行保甲法,竭八晝夜,剖汀州互訐之訟。親歷南臺、閩安各海口,相度形勢,於夷船往來之處設卡樓、築砲臺、資防守。並令澳嶼漁戶盡編保甲,以清盜源。

二年,抵鹽法道任,捐備軍需,優敘。粵匪犯湖南省城,調防岳州,躬勵將帥,夜不解衣臥。剿臨湘縣土匪,獲首逆楊兆勝等。復奉調防蒲、通,丁母憂,奪情留武昌。提督博勒恭武棄岳州,東槐請於巡撫常大淳,全調城外兵勇,亟發庫藏勵士氣,尚可嬰城固守。巡撫吝賞,不能用。城陷,東槐偕妻蕭氏對縊死之,其女投井死,恤世職,謚文直。子四,均賜舉人。

同與此難者:湖北按察使曹楙堅,江蘇吳縣人。豪於詩。道光十二年進士,改庶吉士,散館授主事,官科道時擒治妖道薛執中。江蘇巡撫創議南漕改折,上疏力言其不便,事遂寢。漢黃德道延志,武昌縣知縣何開泰。延志,瓜爾佳氏,滿洲正紅旗人。何開泰,字梅生,安徽鳳陽人。道光三十年進士。

周玉衡,字器之,湖北荊門州人,本鍾祥王氏,依外祖周,遂從姓焉。嘉慶十二年舉人,道光四年,大挑知縣,發江西。署會昌、龍泉、大庾,除龍南,調贛縣。又署寧都、新建,遷義寧知州。湖北崇陽土匪滋事,以協防功擢知府。二十五年,授南康,調贛州。咸豐元年,粵匪起,又以防守毗連粵境地方功進道員。二年,授吉南贛寧道。時廣東土匪竄始興,玉衡飭守備任士魁等協剿,殲擒甚夥。三年,剿泰和竄匪失利,坐褫職留任。以克復萬安、泰和、搜捕龍泉等處餘匪,援剿廣東南雄、韶州勞,復職。

五年,擢按察使,總理吉安軍務。時粵匪由湘入贛,連陷郡邑。玉衡子江寧布政司理問恩慶適奉差至,遂捐貲募勇,率恩慶領兵三千餘分路進剿。先後復安福、分宜。攻萬載,賊眾二萬拒官軍,玉衡身先士卒,奮勇鏖戰,恩慶繼之,斬馘無數。克萬載,軍威大振。賊由間道竄吉安,急率兵馳救,歷數十戰,斬馘數千。賊圍城月餘,糧盡,死守,援不至。地雷發,城陷,猶巷戰,手刃數賊,死之。恩慶亦遇害。

玉衡起家牧令,長聽斷,勤緝捕,有循聲。及身在戎行,與士卒同甘苦,故人思效命。卒,年六十有六。詔視布政使例賜恤,謚貞恪,賞世職,祠祀吉、南、贛三府。子恩慶贈知州銜,賞世職,詔祀荊門。穆宗御極,追念殉難諸臣,各賜祭一壇,玉衡與焉。玉衡第四子炎,知府。剿匪泰和,陣亡,贈太僕寺卿,亦賞世職。

王本梧,字鳳棲,浙江鄞縣人。道光六年,由拔貢朝考用七品小京官分兵部,進主事。遷員外郎,充軍機章京,擢河南道監察御史。奏言:「各省州縣設常平倉,出陳易新,備民間水旱之用。近年州縣乘出借名色,任意侵蝕,新舊交代,捏造冊籍。非以無為有,即折銀代穀。設遇荒歉,倉無顆粒。本年江西、湖北被水,皇上恩膏立沛,共撥銀百數十萬,兩省州縣未聞有碾動倉穀賑濟之處。若非州縣朦蔽轉報,掩飾虧空,何至臨事束手!請敕督撫將所屬倉儲若干,盤查足額,有缺照數買補。直隸、山東、河南等省,本年秋收豐稔,常平倉穀,正可及時採買。民間村鄉有原立義倉者,地方官為倡捐,曉喻紳士,踴躍樂輸,不必官為辦理,致胥吏之擾。」允行。尋掌京畿道,疏陳水師營務廢弛,請飭海疆督撫留意人材,力加整頓,條列六事,曰:戰船宜堅固,戰具宜精良,將弁宜激勸,兵丁宜振作,海岸宜防守,商船宜護送。帝納其言。俸滿,截取知府。

咸豐元年,授江西吉安府。時吉安戒嚴,飭屬團練為備。郴州陷,賊氛逼,籌防益力。三年,賊竄撲南昌,本梧率兵馳援。七月,泰和匪起,聞警折回。偕贛南道周玉衡先後赴剿,行抵倉背嶺,賊直撲吉安。本梧退保郡城,坐褫職留任。賊攻城急,本梧激勵兵勇,登陴固守,相持五晝夜。賊麕集城外,肆焚掠,本梧憤甚,身先士卒,出城迎擊。斃賊百餘,俘十餘人。守備岳殿卿擁兵城內不援。中賊計,兵潰,勢孤力竭,猶手刃數賊,死之。贈道銜,賞世職,祠祀吉安。

陳宗元,字保之,江蘇吳江人。道光十三年進士,吏部主事,歷郎中。咸豐二年,俸滿用知府。三年,記名以道府用,授江西吉安府。吉安當往來之沖,先嘗被陷,宗元至,疆吏以西南保障委之。五年九月,粵匪陷永新、安福,圖犯吉安,宗元力籌堵剿。會按察使周玉衡率兵至,遂同克復二縣,賊竄逸。

十一月,賊自袁州、臨江回竄,別隊更自泰和來犯,號稱五六萬。城中練勇及玉衡所部僅千人,紳民大懼,宗元慰勉之,分兵守要隘。越六日,賊至,撲城。宗元燃砲轟之,賊少卻,知無外援,築長圍,日夜攻撲。宗元語玉衡及諸僚佐曰:「事急矣!非戰無以為守。」會夜風雨大作,開城出擊,毀賊營數座,殺千人,奪旗幟無算。賊恨之,攻益力,屢用梯沖、地道,俱不得逞。

相持半月,城中糧且盡,宗元周巡慰勞,勉以大義,婦孺感憤有泣者。十二月,宗元出與賊戰,身被數創,血至足,屹不為動。城有缺口,宗元督勇填垛,行少疾,失足,自雉堞顛,折左股。蹩躠復上,若無所苦。遣使間道赴省告急,先後十八次,並繪援兵繞道地圖,卒不應。六年正月,逆首石達開遣糾內應之賊,宗元屏左右,面與約,縱之。翌日,賊果偪東門,而宗元命發空槍,賊遂放膽,蟻附城下。宗元突鳴鼓角,槍彈矢石並下,賊不及退,死四五千人。

越兩日,賊復大至,宗元偕玉衡及僚佐分門御之,方馳至東門指揮城守,而西城地雷發,裂數丈,賊蜂擁入,玉衡被戕,城陷。宗元率子世濟揮刀巷戰,與吉安通判王保庸、廬陵知縣楊曉昀等,同時遇害。賊銜宗元深,割宗元父子首,懸東門城樓。計與賊相持者六十五日。其族父陳鈺,姻親周以衡,幕友李鴻鈞、硃芬、硃華、楊福鬯、葉廷樑、蔣志澐及家丁王杞、王慶,並兵勇等四十餘人俱殉焉。宗元照道員例賜恤,予謚武烈。

世濟,監生。城陷之前,宗元遣赴省,囑曰:「此間旦晚不保,汝得我問,即奉母挈弟妹歸奉大母,俱死無益。」世濟既受命,已而復返城,城閉不得入,繞城號哭,乃縋而登之。自此寸步不離父側,遇難時年二十一。

明善,字韞田,富察氏,滿洲鑲藍旗人。父昌宜泰,河南開封知府,以濬賈魯河有功於民,祀名宦祠。明善由筆帖式歷步軍統領、郎中。道光中,出為湖北荊州知府,輸金修萬城堤。繼水災,沿江郡縣皆患潦,荊州獨以堤固得安,眾皆德之。尋調武昌。咸豐二年,粵寇至,登陴助防守,勢不支,城陷,率眾巷戰死。恤如制。妾葉,聞訃自經死。

覺羅豫立,字粒民,隸滿洲鑲藍旗。由戶部筆帖式歷員外郎。道光二十九年,出為江蘇鎮江知府,寬惠有恩,尤重甄拔人才。每遇府試及課書院日,坐堂皇,手自甲乙,至夜不輟,所取多知名士。咸豐三年,以失守府城褫職,仍留治軍需。七年,克鎮江,復原官。十年,浙江巡撫王有齡調總糧臺。

十一年,賊攻省城,豫立偕府縣官籌戰守,城垂陷,豫立督親軍開城決戰,刃及其膚,屹立不動。悍賊以砲擊之,中額死。閩浙總督左宗棠奏請優恤,並祀昭忠祠。豫立工書,善行草,嘗集顏真卿多寶塔字,作詩數十首勒石,論者謂其人其字皆無愧真卿云。

世焜,字顯侯,佚其氏,滿洲正白旗人。初任江蘇常州知府,以愛民稱。咸豐四年,調揚州,當賊亂後,市井蕭然。世焜至,闢草萊,招流亡,還定安集之,民氣少蘇。官廨已毀,借蔣氏園,顏其事曰三十六桂軒而為之記,曰:「百物凋殘,此桂獨盛,原吾民復蘇,欣欣向榮,亦如此也。」明年,賊復渡江至,世焜知城不能守,誓死不去,率鄉兵二百人登城,城破,巷戰被執,勸之降,世焜紿以先釋難民然後可,俟民去遠,遽自刎死。

徐榮,字鐵生,漢軍正黃旗人,廣州駐防。道光十六年進士,以知縣發浙江。歷權遂昌、嘉興等縣,杭州理事同知。授臨安,升玉環同知。保知府,權溫州府事,招降洋盜莊通等二百餘人,授紹興府。咸豐三年,調杭州,並護杭嘉湖道,創議海運章程。時臨安、昌化、於潛土匪趙四喜等謀不軌,榮督兵剿滅之。四年,粵匪竄徽州,浙撫黃宗漢以皖南新隸浙江,中旨亦以「保徽即以保浙」為言,奏派榮督辦徽防。榮扶疾至防,親至箬嶺,開壕遏賊,增設天心洞防勇。七月,剿賊櫸根嶺,斃賊二百餘名。隨諸將克建德、東流兩縣,復敗賊堯渡。十一月,移駐祁門,遍諭居民團練設防,共相保衛。以糧運難繼,撤兵回浙。安徽學政沈祖懋以徽防緊要,奏請留辦。五年正月,升福建汀漳龍道。

先是粵匪沿江上竄,由石埭之流離、霧露兩嶺分竄羊棧嶺,入踞黟縣。時榮尚未赴任,即率師往漁亭防剿。二月,連敗賊,殲二百餘。嗣賊眾紛至,援兵未集,榮率其子慮善與署嚴州同知裕英等出戰,身受刀矛重傷,歿於陣,年六十有四。用正三品例賜恤,於漁亭建專祠,以同時殉難之都司許上達、歙縣知縣廉驥元、候補按察司知事張穎濱及陣亡各員弁附祀。妾伍,迎喪回寓殉難,亦予旌表。

榮律己甚嚴,恆以「行無悔事,讀有用書」二語自勖。守杭時,以時局多警,命鑿井署中,語家人曰:「此即古人止水亭也。有變,吾即死此!」卒踐其言,以剿賊而亡。

郭沛霖,字仲霽,湖北蘄水人。少年即以經濟自負。道光十六年進士,改翰林,授編修,累遷左贊善,記名道府用。官翰詹時,講求河務,時各衙門保送河工人員,沛霖與焉。既抵工,咨詢詳盡,謂治河宜識土性,宜合者合,宜分者分。因勢利導,則不為害而為利。檄管豐工兼引河工程,昕夕在工,與弁卒雜作。凡占數之增減,松纜之尺寸,極微極瑣之事,無不斟酌至當。力主引河寬深,俾掣大溜,濬下游安東二塘、雲梯關、老鸛河等處,先修決口上下之險工,全啟各閘洞,以分水勢。緩進占,緩合龍,以期步步追壓到底,為一勞永逸之計。議不盡用。

咸豐三年,以道員留南河,尋署兩淮鹽運使,授江蘇淮揚道,仍兼署鹽運使。時淮南引鹽道梗,鹽場尚完善,詔兩江總督怡良飭沛霖移駐通、泰適中之地,悉心經畫。沛霖遂駐泰州,督銷引鹽。

六年,賊再陷揚州,泰州戒嚴,沛霖募勇五百,集城、鄉團勇二萬,督屬籌防。建議請江蘇布政使雷以諴移駐灣頭,防賊北竄。幫辦軍務詹事府少詹事翁同書移駐瓦窯鋪,為自守有餘、進攻亦便之策,揚城旋復。淮南旱,沛霖請留淮北折價洎畫提甲寅綱協貼,撫恤各場。招徠殷戶殷灶,赴盱眙等處買米麥平糶。七年,奏派督辦里下河七州縣及通、海二州團練。時江陰靖江水勇經費無出,有議設卡江北各港令自行抽釐者,沛霖力陳其弊,事遂寢。有以淮南稅課造報不實聞者,詔毋庸署理運司,令總督何桂清等查參,以新任未即至,暫緩交卸。

先是淮南之旱也,言者請堵八壩資灌溉,命桂清等詳查酌辦。沛霖力言:「下河七州縣眾水所歸,潦者其常,旱者其偶。上年東南數省大旱,下河盡涸,此數十年一見,不可以常理論也。然如高、寶兩邑,近居運河堤下,並未成災,而田產稻米,猶能以其餘接濟鄰境。咸豐三年,前大臣琦善統兵至揚,盡啟八壩,餘悉緩堵,以為設險禦防之計。是年十一月,揚州東路兵潰,六年三月,逆賊復陷揚州,終不敢越灣頭、萬福橋一步,是未堵各壩足以扼賊之明效大驗。今日賊氛未熄,民力已殫,與其糜無益之費以病民,曷若留可守之險以防寇?現在大兵環攻瓜鎮,奔竄可虞,正宜留八壩以扼逆賊北竄之路。」桂清據以覆奏,詔從之。

桂清等旋以查明淮南稅課無以多報少情事上聞,九月,偕江寧布政使楊能格辦揚州東路團防,自募勇千二百人駐仙女鎮,與毛三元、三岔河營策應。十一月,隨大軍克瓜洲、鎮江,桂清飭沛霖移駐揚州籌善後。八年八月,偽英王陳玉成攻陷浦口,天長,儀徵相繼陷,賊大股徑趨揚州。沛霖督眾迎剿,力不足,遂渡河至仙女鎮,招集潰卒,促援兵為復城計。適提督張國樑渡江來援,沛霖率兵助之,揚州尋復。大臣德興阿劾沛霖先期逃避,詔褫職查辦。又以沛霖專辦揚州善後,與尋常兼轄不同,仍敕刑部擬罪。嗣允大臣勝保、巡撫翁同書疏調,準發安徽充定遠大營總文案。捻逆數萬來攻,偕知縣周佩濂嬰城固守。賊圍數匝,適已革副將盧又熊援兵至,夾擊大捷。

九年六月,捻匪張漋又糾陳玉成眾數十萬再攻定遠,沛霖分守小東門,又熊以賊眾兵單拔營去。總兵惠成出戰不利,沛霖督眾嚴守八晝夜。十八日,力憊回寓,齧指血書「正大光明自盡」六字於壁,復乘馬出,提刀巷戰。賊四面縱火,悍賊從後刺之,傷足墜馬,陣亡。事聞,復原官,恤世職。尋俞定遠士民請,建專祠。沛霖服膺昆山顧炎武之學,兼通術數。嘗言歲在甲子,金陵當復,並自知死難年月。著有日知堂集等書。

時同守定遠者,為候補知縣王培榮。培榮,湖北羅田人。嘗在籍與舉人熊五緯練團剿蘄水土匪,五緯戰死,培榮中二十七創,不退,卒復蘄水縣城。與沛霖同時殉難,尸失,家人即以從前所遺中創血衣葬之。

硃鈞,字筱漚,浙江海寧州人。由廩貢生捐同知,發江蘇,歷辦海運出力,獎擢知府。咸豐七年,奏補蘇州府。十年,護理按察使。時粵匪犯浙江,吳中大震。鈞募勇團練,嚴詰奸宄,人心少定。四月,賊由常州猝逼蘇省,鈞晝夜登陴,誓以身殉,而外無援兵,知事已去,乃先令居民遷避。城陷,率眾巷戰,身受數十創,力竭,投井自盡。恤贈太常寺卿,給世職。後以鈞在官多善政,建祠蘇州。

把總錢貴升,元和人。故業織,入貲竄名尺籍中,檄守婁門。賊破閶門入,貴升未知也,遇二賊城壕,尚衣冠詰之。賊訶之降,拔佩刀斫一賊,賊群至,亂斫死。從者什長張義,同與於難。

時江蘇巡撫徐有壬既殉節,其族弟名曾庾,字裕齋,道光舉人。官工部,來寓巡撫署。建議請兵居城外,民守城內,有壬不能用。城垂陷,有壬促曾庾出避,慨然曰:「兄能死忠,弟獨不能死義耶?況弟亦曾忝一官者耶?」自經死。

蕭翰慶,字黼臣,湖南清泉人。咸豐元年,從都司徐大醇討賊廣西。大醇死綏,翰慶冒險扶櫬返。三年,侍郎曾國籓治水師,翰慶投效營中,屢敘至千總。四年夏,爭紅旗報岳州捷,國籓奇其文雅,詢為讀書士,改敘從九品。以隨剿粵匪功,屢擢至直州判。七年,武昌克復,超晉知府,隨提督楊載福等克九江。鄂督官文疏留鄂省,統帶龍坪以上至漢口水師。九年正月,會陸師援湖南,時賊首石達開自江西道郴、桂圍永州,水師抵祁陽,沿江皆賊壘。翰慶躬入小河,乘舢板督戰,平之。總兵周寬世與賊戰長葉嶺,水師夾擊之,賊大敗。詔以道員記名簡放。

十年,浙撫羅遵殿奏調楚軍援浙,翰慶與遵殿子忠祐有舊,遂請行。倉猝無現兵,得唐訓方舊部訓字營四千人,益以降卒二千馳赴之。抵皖,而杭州已陷。時左都御史張芾方治徽、寧防務,留翰慶辦賊。攻石埭、太平,克之。方進攻池州,而常州促援,乃分降卒圍池,自帥訓字六營、親兵三營行。途次聞湖州被圍,乃改援湖,以湖州為皖、浙咽喉,棄之,則兩浙潰爛。行抵禮義橋,悍賊突出截橋,戰勝之。日暮大雨,所部持仗立風雨中。平旦啟行,距湖州四十里,甫半,賊大至,且戰且進,抵同心橋,賊來愈眾,圍數重。參將吳修考、鄧茂先戰死,翰慶血戰良久,力竭死之,年三十有四,謚壯節。

黃輔相,字斗南,貴州貴築人。道光二十五年進士,用知縣,分廣西。權陸川、博白縣事,以捕盜著能聲。江南提督張國樑者,原名嘉祥,本盜魁也,糾黨寇博白,勢張甚。輔相率練敗之,獲其酋。三十年,權橫州知州。時南寧各州縣盜賊蜂起,輔相招降數股,以賊攻賊計走之。巨盜王斌,號九江三者,與其弟九江四大舉入橫之陶圩,輔相調團勇,會提督向榮合力兜擊,擒九江三兄弟,斃賊三千有奇。博合圩附近十餘村,賊蟻聚,民多從逆者。輔相聲言閱團,召諸生閔麟書等語之曰:「官不能除害,是尸位也;紳不能衛鄉,是虛生也。爾等豈無意乎?」因泣。諸生皆泣,誓殲賊。

咸豐元年二月,輔相從十餘騎至那陽,麟書等以團勇八千一夕至,圍陳山,賊遁獨竹,背倚高山。率死士攀藤下,火其巢,擒斬甚眾。餘黨竄上石,地險,不利仰攻,堅守困之。賊糧盡,突圍出,追擊之,先後殪賊數千。初,賊酋之降也,輔相察其詐,陽與羈縻。至是陰遣諸生殺之,橫境以安。以出奇制勝,擢直隸州知州,旋授鎮安府知府。五月,賊糾眾撲州城,麕集南岸,輔相密令諸團分扼水陸要隘,遣子韶年率練伏村東。夜半,以火具自大道攻入,別遣勁卒五百由小路抄襲,賊奔潰,斃無算。餘匪躍入舟,守者截而焚之,悍賊數千無漏網者。橫州肅清,賞花翎,並賞韶年六品翎頂。十二月,改權南寧,兼權左江道。

二年春,艇匪自梧州連陷桂平、貴縣,圖犯左江,輔相率四百人馳抵橫州,斬其先鋒,賊震懾不敢入境。勇目潘其泰與土賊有隙,賊假殺其泰名攻南寧,輔相堅守百五十日。城中糧垂盡,毀銅錫器為砲子,力戰,圍解。四年秋,權右江道,以巡撫勞崇光薦進道員。

五年,廣東賊李文茂圍潯州,犯武宣,署知縣硃爾輔以澛灘為北河要隘,自督兵屯守,乞濟師。崇光檄輔相統水師駐武宣之碧灘,與澛灘犄角。賊分水陸來撲,迎戰屢勝,艇賊何松亭率黨就撫。八月,文茂陷潯州,屢攻澛灘,擊退之。

六年二月,以餉絀撤澛灘防兵,賊麕至,糧盡援絕,勢岌岌。輔相連牒布政使乞餉,不報。復遺書桂林守李承恩,瀝陳四難四易,使聞於巡撫,有「力竭心殫,惟以一死報國」之語。未幾,兵士果以饑譁,賊黨潛結土匪內應,開城納賊。輔相督外委吳錦蘭等巷戰,格殺數十人,賊乘夜冒雨大至,署潯州營副將福格暨錦蘭皆死之。輔相受創被執,絕粒罵賊,仰藥死,賊棄尸於江。輔相才略足辦賊,時有旨調引見,而殉難事聞,賜恤如例。

孔昭慈,字雲鶴,山東曲阜人,至聖七十一代裔孫。道光十五年進士,改庶吉士,散館授廣東饒平縣知縣。憂歸,服闋,發福建,署莆田、沙縣。攝興化通判,授古田縣。二十八年,調閩縣,進邵武同知,移臺灣鹿港。時南北匪徒洪恭等陷鳳山,知縣王廷幹、高鴻飛相繼死,郡城岌岌不保。昭慈聞警,航海赴援,協力守御,殲擒甚眾。咸豐四年,擢臺灣府知府,督捕餘孽,次第蕩平。進道員,備兵臺、澎,加按察使銜,兼督學政,以助餉加二品銜。在臺五年,威信大著,外裔內番悉畏服。

同治元年,彰化亂民戴萬生等糾眾結會謀亂。昭慈偵知,督兵馳抵彰化,部署未定,變起倉卒,城陷,巷戰,力竭不支,殉節文廟先聖前。

昭慈為政,興利剔弊,不遺餘力。莆田俗好鬥,推誠諭禁,勸以懲忿保身,治正兇不少貸,民憚法罷斗。邑多孔氏寄籍,為創立義學。沙縣土利藝茶,少耕植,游民競逐末,暇則事攘奪。為拔茶禁之,而農桑始興,至今利賴。所至停採買,革津貼,捐粟平糶,多損己益民。尤愛才,重林文察材略,白其復父仇可宥而薦之,殺賊立功,官至提督。治盜嚴明,誅止其魁,盜之良者,或重其賢而避之。歿後,匪為斂殯歸喪,愧嘆曰:「吾輩負孔使君矣!」卒,年六十八,恤世職,謚剛介,於立功地方建祠。

徐曉峰,江蘇東臺人。初由供事隨工部侍郎呂賢基剿辦安徽捻匪,獎六品頂帶。旋署蒙城縣知縣,有惠政。時潁州府捻氛不靖,給事中袁甲三檄曉峰剿辦,先後獲其酋馬文俊、鄧大俊、馬在隴、馬九、陳建中等。餘匪麕聚阜、亳交界,復擒捻首李致文於陣。剿匪渦河,匪眾鳧水來撲,曉峰領隊堵截,賊砲從馬腹過,馬驚蹶,頸背皆傷,復上馬截擊賊渡河者,殲焉。

粵匪撲潁州府城,甲三復檄曉峰赴援,御之南岸河上,殲匪毀船。匪於滁州駐馬河扎營為久踞計。曉峰改裝潛探,隨按察使恩錫分三路進剿,毀賊營三座。竄林母圩,復偕都司劉鶴翔等剿敗之。又隨廬鳳道張光第追剿粵匪於高旺街,賊潰,追敗之。烏江賊分隊襲後路,於大雨中麾眾痛擊,擒偽司馬等五人。匪由江寧鎮下竄,陣斬偽佑天侯富姓、偽右四丞尉張盛林等。

咸豐七年,亳州捻匪劉老淵等竄擾李八莊等處,曉峰督兵攻剿,斃賊百餘,生擒三十餘人。攻破宗圩匪巢,鄧圩賊內訌,被脅男婦閉門乞命,縛匪首李寅、悍賊劉破頭等三十五人。姚圩賊二百餘人亦降,遂平兩圩。著名巨賊,悉數就誅。王圩捻匪復踞河抗拒,曉峰乘夜進攻,難民內應,遂擒匪首王紹堂。乘勝收復東面七圩,宿州以南,一律肅清。五六年間,曉峰於剿辦捻、粵各匪,戰功獨著。由知縣歷保知府,至是擢道員,記名簡放。旋授福建汀漳龍道,同治元年,赴任。

三年二月,檄署按察使,督全省軍務,守延平。粵匪餘孽竄粵,閩防解嚴,七月,還漳州道任。賊復由粵竄閩,守漳者僅練勇二百五十人,賊遂勾結土匪攻城。無備無援,九月十四日城陷,曉峰被執,死之。妻王氏聞城破,知曉峰必死,先絞其女,亦自經。恤贈內閣學士銜,給騎都尉世職,復以「曉峰從戎豫、皖由軍功洊升監司,自軍營回任,甫及旬餘,倉猝被害,最為慘烈。妻女皆以身殉。忠孝節烈,萃於一門」褒之,於死所建專祠。

曉峰初從甲三軍,與馬新貽同幕,新貽謂曉峰殺氣滿面,目光灼灼射人,終當以義烈見。及被賊執,備受凌虐,叱跪不跪,勸降不降,書其禁錮壁間有云:「壯志未酬,君恩莫補。取義成仁,臣心千古。」又絕命辭二章。尋予謚剛毅。

袁績懋,字厚安,順天宛平人,原籍江蘇陽湖。父俊,道光九年進士,官河南知縣。績懋,道光二十七年進士,以一甲二名授編修,散館改主事,分刑部。旋丁父艱,服闋,授例以道員赴閩。時漳、泉初下,好事者欲多殺以邀功,而清查叛產,尤多誣陷,人心洶洶,將復叛。總督慶端檄績懋往治其事,至則集眾於庭,取叛冊焚之,脅從者皆獲免,人情大安。皆曰:「使君活我!」事竣,委赴延平府會辦軍事,即令署延建邵道。

粵匪時竄邵武,勢張甚,績懋親督軍士,夜撲賊營,賊驚潰,追斬悍酋數名。賊大憤,鳩眾出儳道陡截之。我軍既寡,又軍實未備,戰不支,乃退守順昌。防軍僅數百名,相持月餘。有勸之棄順昌、守延平者,績懋以:「順昌為省垣屏蔽。順昌不守,則賊長驅直逼省城,大勢去矣!且數萬生靈,視我進退為存亡,敢輕去耶?」於是守益堅,賊不得逞,乃潛隧城,實火藥,地道發,城陷,賊蜂擁進。績懋知事去,躬率死士戰西門,連刃數賊,賊以騎突之,僕地,引刀自殺,刺不及,賊執而去,刃亂下,醢而死。時咸豐八年九月十二日也。事聞,優恤,贈按察使,入祀京師及陣亡地方昭忠祠,世襲騎都尉。嗣常州、順昌奉特旨建專祠,追謚文節。子學昌,官至湖南提法使。

績懋性通敏,書過目輒成誦,號稱淹雅。著有諸經質疑十二卷,通鑒正誤十卷,漢碑篆額考異二卷,味梅齋詩草四卷。

楊夢巖,湖南鳳凰人。諸生。入田興恕幕。咸豐六年,興恕率虎威軍援江西,勇果名天下,夢巖實贊助之。累功由縣丞擢同知。興恕奉命援黔,以夢巖綜理營務。時苗酋楊龍臺煽惑諸苗,出沒不時,以思南、石阡為尤甚。夢巖與田興奇、沈宏富等焚剿之。會夢巖晉保道員,遂自領一軍扼守思南。

同治元年正月,石阡賊來攻,與副將吳通才犄角扼守,會援至,賊敗走。二月,夢巖帥師次浮橋,賊分道來攻,通才戰死,夢巖極力鏖戰,賊卒敗走。三月六日,賊復數倍來攻,更番迭進,累日夜不休,營忽陷。夢巖奮臂大呼,持矛入賊陣,刺殺數人,身受十九傷,力竭死之。照布政使例賜恤,於思南及原籍建專祠。

鄧子垣,字星階,湖南新寧人。咸豐初,以諸生從同邑劉長佑剿賊江西臨江、撫州、新城,與湖南永州、寶慶,皆有功,累保知縣。石達開走貴州,竄廣西,還窺義寧,子垣與參將江忠朝自武岡趨扼全州,為東安、零陵等屬屏蔽。賊由靈川渡河竄楊梅坪,又偕忠朝壁界首。賊掠道州、永明、江華而東,自藍山趨桂陽、宜章,又從江忠義轉戰數百里,殺賊殆逾萬。忠義病歸,子垣與忠朝代領其眾。十一年,貴州銅仁、石阡、思州、松桃、天柱、工⼙水賊窺湖南,檄子垣赴黔會剿,同治元年,屢破之,晉知府。復檄援廣西,攻桂嶺賊巢,環擊三時,毀其砲臺,擒斬甚眾。賊黨廖永賢懼,原輸誠為內應,官軍分薄內壕,擲火焚燒,永賢開西閘納軍,遂克桂嶺要隘。進搗蓮塘賊壘,逆首張高友、陳士養恃險抗拒。懸重賞,募死士,潛由內貨村僻徑,捫蘿攀葛出賊背,破其西柵。翼日復進攻,悍賊由山澗奪路奔,督軍截擊,斬張、陳兩賊,擒殲及墜巖死者無算,餘眾乞降。蓮塘平,以道員侭先選用。

三年,粵賊竄江西,陷新城,子垣會諸軍破走之,新城復。巡撫沈葆楨以捷聞,命以道員留江西。時賊分踞金谿、東鄉、宜黃、崇仁、南豐各城,而崇仁之賊尤悍,增壘城外,為負嵎計。子垣張兩翼橫沖之,城外賊潰,竄潘橋、秀才埠。城賊出犯,再擊敗之,直逼崇仁城下。賊聚悍黨許灣,遙為聲援,提督鮑超破之。子垣乘勢復崇仁,賜勇號。

五年冬,率精捷營助剿貴州苗匪,駐軍工⼙水,會攻頗洞老巢。寨頭苗黨數萬來援,逆擊敗之,徑搗頗洞,山高寨險,不能下。選趫捷精勇,攜藥逾嶺,入巢縱火,自督軍登山猛攻,斃苗數千,山谷皆平,遂克頗洞。疊破甘林、杉木等屯,由記名巖直趨茂林坡,盡毀碉卡,追擊至傳水寨,赭其巢。黎平、靖州肅清。以糧運不繼,壁清溪。時按察使黃潤昌率軍來會,分餉哺之,約進攻。八年正月,破文德關兩路口各隘,師益進,合攻鎮遠府、衛兩城,克之。連破牙溪、田壩、黃蠟坡等三十餘屯。潤昌欲乘勝由東路疾攻,命營務處羅萱,偕子垣馳謁布政使席寶田請濟師,寶田遣提督榮維善助之。三月,進規施秉,逆酋包大肚據巢死守,力戰,斃苗千餘,拔施秉。復破白洗等寨,進圖黃平。

黃平州,滇、黔孔道也,蜀兵援黔,輒為所阻。潤昌議通此道,時維善軍戰久,疲,請休息,萱亦以苗眾道險勸留屯,潤昌不可,師遂進。道出黃飄山,中伏,子垣蕩決數十次,地險,不能出,中砲死之,諸軍皆敗。語具榮維善、黃潤昌傳。優詔賜恤,原籍及死事地方建專祠,謚壯毅。

羅萱,字伯宜,湖南湘潭人。父汝懷,芷江學訓導,著有湖南褒忠錄。萱少警悟,工詩、善書。弱冠為諸生,總督賀長齡、教諭鄧顯鶴咸器之。咸豐元年,粵賊犯湖南,萱倡鄉團,習技擊。四年,曾國籓帥水師東下,闢掌書記,貽書極推重。從克武漢田家鎮,敘訓導。國籓進圖九江,水師失利,萱僅以身免。國籓重整水師,屯南康,皆策馬相從,調護諸將,各當其意。六年,石達開陷瑞、臨、袁、吉、撫、建諸郡,省垣孤懸,萱從國籓單舸赴南昌,達開稍引去。國籓檄萱領江軍三千人攻建昌,復檄助攻撫州,合攻瑞州。破沿途賊卡,擊走靖安、奉新守隘賊。當是時,城賊數萬日伺隙,九江賊復率二萬來援,萱與劉騰鴻等堅壘嚴陣以待,八戰皆捷,江西軍始振。論功擢知縣。騰鴻喜攻堅,萱引孫子書戒之,不聽。騰鴻克瑞,竟以創死。

假歸,湘撫駱秉章檄治團練,粵撫郭嵩燾囑創水師,皆不肯久留。自以文士,不欲棄科舉,屢應省試,卒不遇,益肆力於學。尋與知府劉德謙領威信軍防郴,會霆軍叛勇掠茶、攸間,萱與德謙敗之,遁入粵。進屯樂昌,當事命增募一營,號威震軍。賊平,累功晉同知。按察使黃潤昌奉檄統萬人援黔,潤昌與萱同邑,邀與俱。萱綜文案,兼營務處。每晝出領隊,夜歸削牘,以克鎮遠府、衛二城功,遷知府。進規施秉,連戰皆捷。黃平之敗,與文武將領十八人同死之。恤贈太常寺卿,附祀黃潤昌祠。

萱性澹泊,從軍十數年,不圖仕進,而耽學弗倦。著有儀鄭堂文箋注二卷,粵游日記一卷,蓼花齋詩詞四卷。

侯雲登,河南商丘人。道光二十一年進士,由內閣中書洊升刑部郎中。咸豐六年,補江南道監察御史。奏言:「皖、豫接壤,向有捻匪,自粵匪北竄蒙、亳,捻匪乘之蜂起。捻首張洛行更句結蘇添福等,合為一股,所過荼毒,蒙、亳迤北,歸德以東,數百里幾無人煙。一誤於張維翰,而永夏受困,馬牧被焚;再誤於武隆額,而賊擾掠歸、陳。武隆額雖撤歸巡撫英桂調遣,並張維翰迄今未聞撤參,且其營勇,多雜匪類。今邱聯恩軍亦潰敗,歸德決河未堵,防備綦難。儻捻匪逾河而北,句結東省災民,其患甚大。查匪逾十萬,擾及四省,惟賴兵力兜剿,而調集需時。莫若以勇濟兵,請於皖、汴、蘇、魯接壤之區,設立勇營,簡員督辦。本年二月間,命已革左副都御史袁甲三隨同英桂剿辦捻匪,請即加以卿銜,責令募勇。其於勸約鄉團、捐辦勇糧,必能悉心籌畫,次第舉行。擬辦法四條:一,酌保文武,勸懲悉照軍營之法;一,審度地勢,擇要安營,與官兵互相策應,遏賊北竄;一,急籌糧餉,請先由糧臺撥給,並四省就近州縣動項奏撥,仍勸捐以資接濟;一,明定賞罰,認真訓練,以嚴紀律。」等語。疏入,朝廷頗韙其議。

九年,掌京畿道事務,授給事中。又言:「捻匪蹂躪豫境二十餘州縣,仍分股四出焚掠,擾及直、東邊境。雖有關保、博崇武等軍,兵力過單,馬隊未能精壯。儻賊再蔓延,非獨豫省全局不保,直、東亦防不及防。救急之法,惟有直、東兩省防兵並力進剿,並請催副都統巴揚阿將所帶馬隊赴豫,與關保合軍剿辦。並請令副都統德楞額統軍由歸德探賊剿擊,必可制勝。再東明、長垣已無匪蹤,請令直督將所派東明、長垣之兵,出境協剿,以壯聲威。豫省肅清,直、東南路,不待設防,均可無虞矣。」十年,授甘肅寧夏道,同治元年,陜回倡亂,靈州被圍,佐領富隆阿援軍戰失利,雲登督兵勇進剿,斬馘無算,圍立解。護督恩麟上其功,加按察使銜,賞花翎。

時寧夏令彭慶章屢請散團,雲登以回性險詐止之。恩麟檄雲登開城納降,慶章暗為回匪內應,變猝起,雲登率兵巷戰,被執不屈,死之。子錫田同遇害。

黃鼎,字彞封,四川崇慶州人。以諸生倡辦團練。同治元年,粵匪犯敘永,鼎率所部,佐官軍擊破之,敘功授教諭。二年,復新寧。松潘番亂,總督駱秉章檄鼎募蜀中驍勇士,得五百人,為蜀軍彞字營。會四川布政使劉蓉巡撫陜西,檄鼎以所部從。時粵寇擾漢中,偽啟王梁成富據南鄭,分兵陷諸州縣,且東侵興安境。鼎會陜軍分道討擊,盡復諸城邑。

三年二月,漢中土寇曹燦章召滇賊藍朝柱自川北進犯陜南,前鋒至松花坪,將越秦嶺而北。檄鼎率所部邀擊,遇賊七里溝,大破之,轉戰八十餘里,擒斬殆盡。是役也,鼎所將才千人,破悍賊數萬,號奇捷。朝柱黨悉平。四月,破燦章於八里坪,獲之。

梁成富南寇襄樊不利,引而北入興安境,山南三郡悉戒嚴。鼎聞警,自漢中東援,而賊已出山,焚掠鄠縣,遂渡渭而北。鼎率師沿渭追擊,賊不得逞。是時蜀寇西北犯階、秦,謀出山窺蘭、鞏,秉章急召鼎屯畢口。四年正月,大會諸軍,進師階州,力戰抵城下,督軍以地雷轟城,諸軍填壕樹梯而上,斬偽昭武王蔡昌榮於陣,賊乞降,遂復階州。

十二月,蓉合諸軍三十餘營,與捻首張總愚戰於滻橋,鼎以所部橫貫賊陣,殲斃甚眾。會天大雪,藥繩皆濕,軍士殭凍,賊突以萬騎穿湘軍陣,統將蕭德揚兄弟三人皆戰死,軍大潰。鼎以千人憑原為異軍,湘軍既熸,賊悉萃於鼎,圍之數十重。夜三鼓,賊少疲,鼎乃結圜陣,騎兵居中,步卒環外,以矛護槍,力戰,突出。鄉晨,賊傅城東關,意鼎已沒,忽睹彞字旗,大驚。鼎麾軍迎戰,敗之,賊始退。是役也,微鼎,西安城幾危。

六年四月,敗賊於大荔、朝邑,捻寇稍衰,而叛回復熾,犯鳳翔,游騎及省城西郊。鼎移師進擊,累破之,斬偽元帥一。賊東走,據富平張家堡,鼎追擊,夜襲其壘,斬馘無算。賊由臨津南渡渭,覬入南山,鼎悉力拒之,賊不得西。十月,會諸軍追賊至三原,旋移援汧陽,率步將韋占雄、徐占彪等先登陷陣,大破賊黃裏鋪,追擊至五里坡,又敗之。

七年,賊竄甘肅之靈臺,犯涇州,西安迤西,汧、隴、乾、邠間,無慮皆為賊據。鼎率所部為游擊軍,隨賊上下,相持數月,大小數十戰,累克堅巢。甘賊與陜回合,悉眾來犯,鼎復大破之。鼎以戰功由教諭累擢至陜西道員,賞二品頂戴,兩賞巴圖魯勇號,至是授陜西陜安道,未之任。

八年,回酋陳林等糾大眾來犯,鼎率所部嚴陣以待,賊不得進,譟而走。鼎追擊十數里,涇、慶賊悉平。初鼎督涇州賑,撫屯田,廣為招徠。至是涇州得民屯十三萬畝有奇,營屯五千有奇,鎮原得民屯十三萬有奇,平涼、崇信各有差,軍益饒富。甘肅土寇張貴為亂,鼎一鼓平之。

左宗棠會諸軍進攻金積堡,堡,回酋馬化隆偽都也。化隆遣將據固原,抗大軍,鼎大破之,復其城。賊走狄道、河州,復擊敗之。捷聞,賞內府珍物。九年,金積堡未下,湘軍大將劉松山新戰歿,軍事方棘,宗棠檄召鼎會固原提督雷正綰赴援。軍抵牛頭山,山峽狹隘,為金積堡第一門戶,賊恃為天險,鼎力攻拔之,連下數十壘。復攻馬家堡,環圍三面,缺其一,設伏以待。賊果由缺處遁,伏發,賊大敗。進傅金積堡,盡毀附近小壘,師集堡下,晝夜環攻,遂克之。化隆父子伏誅,餘黨悉平。以功賞黃馬褂。十三年,移防陜北,旋丁父憂,詔奪情留軍中。光緒二年六月,部將湯秉勛以不給四川咨文之嫌,突起刺之,遂卒於軍。

鼎治軍素嚴,在防所招集流亡,開墾荒蕪,修濬堡渠,興學課士,得軍民心。其屯軍漢中也,曲阜孔廣銘落拓廢寺中,鼎軍行經其寺,睹廣銘題壁詩,異之,召與語,叩所學,大悅,遂延入幕。鼎軍所鄉有功,半廣銘策也。

陳源兗,字岱雲,湖南茶陵州人。道光十八年進士,改翰林,授編修,旋授江西吉安府。先是源兗妻易氏以源兗遘疾幾殆,籥天原以身代,刲臂和藥飲源兗,源兗以愈,易氏旋病卒。同鄉公舉孝婦,請旌於朝。源兗適召對,宣宗垂詢及之,遂有是命。以回避原籍調廣信,母故,去任。服闋,簡放安徽遺缺知府,補池州。

咸豐三年,粵匪自桐城竄撲廬州,巡撫江忠源檄源兗赴廬協守,賊架雲梯薄城而登,源兗守大東門,屢卻之。賊復穴威武門為隧道,伏地雷,官軍迎掘之。尋水西門地雷發,轟塌城垣數丈,急搶築,城卒完。時陜甘總督舒興阿奉命統兵萬五千人來援,屢戰不利,賊連日攻益急,城中餉乏兵疲。十二月,賊復穴水西門隧道攻入,源兗自東城馳救,至則江忠源已戰歿,遂赴文廟自經死。先嘗與所親謁文廟,徘徊庭樹,謂「事亟吾且死此,以無負先師殺身成仁」之訓,蓋死志素定云。

瑞春,字慰農,姓鄂濟氏,蒙古正藍旗人。由筆帖式洊升理籓院郎中、軍機章京,擢湖州府知府。治尚寬平,有瑞佛之稱。湖城危急,與副將鄂爾霍巴、郡紳趙景賢激勵軍民,嬰城固守。景賢主湖郡鄉團,多專擅,瑞春無所忤,嘗曰:「趙兵睢陽之儔,我其為許遠乎?」城陷,西門火起,朝服升堂,賊至脅降,大罵不屈,被害。母章佳氏及妻、妾、二子、子婦皆死於難。

鄂爾霍巴,字斐堂,滿洲正白旗人。起家侍衛,出為湖州協副將。湖州初次解圍,上守城功,鄂爾霍巴以屬邑失守自劾,時論偉之。餉糧久匱,困甚,以衣物質錢自給。每圍急,身出巡城,而閉妻子於後堂,外環火藥,戒家人曰:「有不測,即舉火,無污賊!」如是者屢矣。及城陷,在北城督戰,策馬回署,則賊已入事。手燃火繩,藥發,闔家轟死。

時署烏程縣事者為許承嶽。承岳,字柱山,湖南寧鄉人。由縣丞擢署縣事,誓與瑞春死守。千總熊得勝以搜米擾民,涕泣阻之,得勝開東門降賊。承嶽即騎馬歸署,手刃二女,自縊於官所,妾錢氏從死。

潘錦芳,湖州人。城圍久,趙景賢以江蘇巡撫駐軍上海,作血書乞援,募能犯圍出者。錦芳時老病,家亦賣酒小康,獨激於義憤,請行,展轉得達。議以松江提督曾秉忠率水師絕太湖而西,為外內合攻計。湖賈之在上海者,且聚貲鉅萬餉之。行有日矣,有尼之者,中變,錦芳流涕曰:「老夫出城時,糧將罄矣。兵一日兩粥,民食草根樹皮,空巷敝廬,死亡枕藉。其幸存者,數老夫之行,旦暮待援,懼不相及。城外賊如麻,登高叫呼,兵則憑堞應答,岌岌將為變。鄉人之賈於此者,念在圍親屬,其愁迫何如?獨恨水師無翼而飛耳。彼尼之者,何不仁乎?嗚呼!吾不復見趙公矣。」抵案大呼,嘔血以死。

廖宗元,字梓臣,湖南寧鄉人。道光二十年進士,以知縣分浙江,任仙居、德清等縣,有能名。權歸安,粵逆自廣德進窺湖州,宗元建議:「湖州四面阻山,有險足恃,且城多富室,粟芻無虞。今寧國雖潰,營將田宗升、楊國正皆宗元鄉人,若給以糗糧,可使為我固守。」知府從其言,悉以防務屬之。賊至,出擊。賊知有備,引去。會蘇、常、杭、嘉諸府相繼陷,賊復擾湖。道員蕭翰慶陣亡,宗元收其潰卒,入城餉之。明日出戰,大捷,賊敗走。有以蜚語上聞者,解任聽勘,事得白。

會偽忠王李秀成陷金華、處、嚴諸府,浙撫王有齡因檄宗元署紹興府。時浦江、義烏、東陽皆不守,紹興戒嚴。既受篆,議調外江砲船入內港,勿為賊有;議設水柵,以斷賊道;請徵團防勇丁入城:均為在籍團練大臣王履謙所阻。九月,宗元令營將何炳謙率水師出擊,戰歿,敗卒歸伍。富紳張存浩等挾捐輸之嫌,誣其通賊,毆傷宗元,履謙置不問。賊果由浦江入諸暨,奪外江砲船,渡臨浦,陷蕭山,以撲紹興。履謙率姚勇走上虞,有開門迎賊者,城遂陷。宗元朝服坐公堂,罵賊不屈,死之。詔以:「宗元力籌防守,嚴催富戶捐輸,致被富紳張存浩等誣毆,旋復御賊捐軀,城亡與亡。實屬大節凜然,深堪嘉憫。照知府例優議給恤,並於死所祠祀,以彰忠藎。」給世職。

劉體舒,字雲巖,雲南景東人。道光十三年進士,用知縣,分直隸,授廣宗。二十一年,揀發廣西,署養利知州,除融縣。進直隸州知州,授鬱林。咸豐四年,權潯州府事。時艇賊梁培友、大口昌縱橫水面,聞體舒至,就撫,已而叛去。糾貴縣賊趙洪、李七等眾數千犯郡城,體舒督兵登陴守御,更番出擊,分兵截歸路。戰西關,擒斬千七百餘級,賊遁。追至河邊,毀賊船數十,餘匪仍退據貴縣。巡撫勞崇光奏薦堪勝道府任,進知府,尋授思恩,權潯州如故。

五年,廣東賊季文茂等溯江西上,犯潯州,培友等與之合,賊萬餘,晝夜環攻,絕城中運道。七月,穴地攻小南門,陷其郛,賊蟻附上。官軍奮擊,矢石雨下,斃賊數百,體舒血書乞援。八月,按察使張敬修、參將尹達章自平南督水師至石嘴,戰失利。賊詗知糧盡援絕,攻益急,官軍饑疲不能拒,城陷。體舒暨桂平知縣李慶福、卸縣事舒樺均被執不屈,死。經歷宣元烺自縊,典史沈廉赴水死。體舒贈太僕寺卿銜,賞世職,慶福等賜恤有差。

李保衡,浙江會稽人。由訓導捐同知,分貴州。同治元年,署普定縣知縣。時貞豐回匪陷歸化,延及縣屬白巖、沙子溝,擊敗之。粵賊偪安順,保衡籌防,獲間諜,得賊情,豫為備,賊不得逞。賊何二竄擾,又督團兵兜擊,殲賊數百,境賴以安。三年,調署鎮寧州知州,明年,署興義府知府,時回酋金阿渾據新城,陽反正,陰蓄發,懷異志。保衡率敢死士數十,徑抵城下,呼之出,示以威信,阿渾感服,薙發就撫。降酋馬忠署安義游擊,擁兵驕恣,侵知府權,縱部卒虐民。保衡規之曰:「既反正,當圖晚蓋,奈何若此?」忠為斂跡。流亡歸集數百戶,總督勞崇光疏薦保衡政聲卓著,擢知府。丁父憂,奏請奪情留任。

五年,以貞豐回匪馬沖負隅,檄都司熊忠、守備劉萬勝等進剿。賊分股來拒,進踞距城三十里八達地方,與普坪黑夷王罰傭句結,保衡督忠等設伏截擊,斬馘無算。萬勝亦擊退頂廟賊,合師攻八達。罰傭勢蹙,詐降於忠。忠將至新城受降,保衡力阻,不聽,竟遇害。聞變,亟調興義、普安團練御之。未至,賊偪城下,保衡登陴固守,或勸以「勢急徒守無益,盍逆師境上為兩全計」。保衡曰:「臣子之義,城亡與亡,吾知效死勿去,他非所知也!」三月,忠部降卒與賊通,城陷。保衡巷戰,手刃多賊,力竭被執,罵賊不屈,受鱗傷死。屬紳劉官禮等以重金募人覓其骸,越二年始獲葬。署經歷徐海、州同李善鬥同遇難。詔贈道員,祠祀興義,海、善鬥附祀。

淡樹琪,四川廣安州人。咸豐六年,以知府候補雲南。先是雲南各郡縣漢、回相殺,回人據大理諸州縣。樹琪至滇境,聞變,遣家屬還,間道至省城。次日,城門晝閉,得奸人托福、托壽,搜其家,旗幟刀矛咸具。事既洩,諸回不自安。漢人聞回人之欲相殘也,為先發計,一呼而眾合,城內外火光殺聲兩日不絕。初,樹琪以部曹出守貴州,苗匪亂,辦賊有聲。大吏就問計,樹琪因乘間說曰:「漢、回相仇久矣,直漢者曰回曲,直回者曰漢曲,兩直不相下,是助之攻也。今日之事,誠宜兩曲之,以蓄謀曲回,而以擅殺曲漢。然後宣布天子威德,示禍福利害,使各愛其身家,亂庶幾止。」又請設勸捐籌餉局,不十日,軍民輸錢米者十餘萬,省城事稍定。

各郡縣告急,警報迭至,大吏卒遣樹琪及副將謝周綺防堵碧雞關,屬以練勇三千人。樹琪視所屬練勇不習戰,餉又不能持久,不得已至關。關去城三十里,地狹不能布眾,乃去關八里硃家祠屯駐。時亂回據彩鳳山下,左曰三家村、曰二里坡,皆賊窟,其右則昆陽、安陵地。大吏責樹琪辦賊,樹琪使練目熊載攻三家村,從九品周廷軫攻二里坡,周文舉具船五十號攻賊前,其右則委之安陵州牧,剋日逼賊巢。至日,樹琪與周綺整隊據彩鳳山頂。辰、巳交,大霧滿山谷,數武外不可辨。左右或勸且收隊,樹琪嘆曰:「督戰方急,而諸路兵又分遣,軍令不得失期,今日但有戰耳。」揮隊下山,俄報左路敗,載與廷軫死,樹琪軍遽潰。周綺先走,樹琪據嶺畔一大松立,僕何彬、李秉、劉喜、楊紳皆有力能戰,無何,三僕戰死。紳持矛擁樹琪,樹琪據地呼殺不絕聲,賊從後砍紳墜嶺下,樹琪旋遇害。時六月二十六日,距至雲南僅七十餘日。事聞,贈太僕寺卿。

褚汝航,字一帆,江蘇吳縣人,或曰廣東人。道光二十八年,捐職布政司經歷,發廣西。粵匪倡亂,汝航於金田及新墟等處剿擊出力,累功擢知府。應曾國籓招,至湖南,與夏鑾督造戰艦,練水軍。咸豐四年,率所部復岳州,復湘潭。賊犯城陵磯,汝航偕鑾分路進擊,奪賊前船,殪偽丞相汪得勝等,追殲殆盡。捷聞,以道員選用。尋賊由擂鼓臺上竄,汝航督兵迎擊,敗之。賊復以船伏城陵磯,夾洲為誘敵計。汝航偕鑾暨都司楊載福等督兵直偪城陵磯,賊眾未及抄截,被水陸官軍分途擊潰,夾洲泊船亦被毀。以汝航膽力俱壯,奏獎鹽運使銜。嗣統師船於下游一帶與總兵陳輝龍等水師排陣合攻,多所殲斃,並火其舟。其時群賊下竄,風逆船膠,賊艘復集,官軍陷入重圍,輝龍及游擊沙鎮邦等俱陣歿。汝航等督軍馳救,均被鉅創,死之。汝航條理精密,為國籓所重,及死,尤痛惜焉。

輝龍,廣東吳川人。國籓定水師剿賊策,輝龍實先以廣東兵船從。城陵磯之役,自乘拖罟船先發,而汝航繼之。死事上聞,賜謚壯勇。

夏鑾,字鳴之,江蘇上元人。以附生從九品發廣西。盜匪陳亞貴滋事,鑾捐貲募勇在荔浦、修仁防剿,保府經歷。與汝航治水軍,凡器械之屬及營制,多鑾手定。同復岳州,同復湘潭,歷保府同知。城陵磯之役,汝航統師船進擊,鑾於陸路設伏互應。進剿至白螺磯蘆葦中,賊眾復集,鑾手刃數賊,躍入水中,死之。諸生何南青同戰歿,事聞,均賜恤如例。

儲玫躬,字石友,湖南靖州人。廩生。少有大志,讀書喜講求營陣攻擊之法,嘗於本籍擒治傳習左道倡亂者。道光二十九年,土匪李沅發作亂,踞新寧縣城,玫躬督鄉勇從間道馳截要隘,助官軍討平之,敘功以訓導即選。咸豐三年,選授武陵縣訓導,江西泰和縣土匪闌入茶陵州,巡撫駱秉章檄令募勇討賊。八月,賊竄安仁縣,玫躬偕把總張大楷往援,遇賊於安仁、酃縣交界地,與酃縣團勇合力兜剿,大敗之。常寧土匪圍攻藍山縣城六晝夜,玫躬復偕縣丞王珍等會剿,陣斬六百餘名,賊潰,藍山以全。移剿股匪於道州四眼橋,玫躬繼各營至,逼賊而陣,奮擊敗之,追殲殆盡。玫躬為偏將,兵不滿五百,未嘗出境與大寇戰,馳逐衡、永、郴、桂間者,先後凡三年。粵匪竄擾湖南,逼省城,曾國籓在籍督辦團練,檄玫躬等各統所部遏之。

四年正月,賊攻寧鄉縣,玫躬偕候選同知趙煥聯往援,遂冒雪夜發,身先馳之。抵縣南門,城已破,賊正縱火焚掠。玫躬率勇目喻西林、文生楊英華等奮力奪西門入,轉戰城南北,賊尸填街市。悍賊橫截之,復挺矛入賊隊。圍數匝,身被十餘槍,力竭,與西林、英華等同歿於陣。國籓疏以「玫躬寧鄉一戰,以五百勇敵賊三千,斬馘數百,我兵喪亡止十八名,賊氣奪夜竄,寧鄉卒得保全,合邑感激,欲為建祠。藍山、道州戰績,擬保同知直隸州,撫臣未及匯奏,不料遽爾捐軀,請照進秩議恤。」詔進贈道員,謚忠壯。湖南巡撫駱秉章立忠義專祠,祀安徽巡撫江忠源等,復請以玫躬附祀,從之。

李杏春,字石仙,湖南湘鄉人。少工制藝,神清體弱,而膽識過人。由廩生投效軍營,以功用訓導。咸豐四年,隨寧紹臺道羅澤南軍。義寧州之戰,與縣丞蔣益澧率兵數百,當賊黨七八千。杏春直馳中路,賊潰走,諸軍追殺十餘里,斃賊六百人。復戰鼇嶺,賊多墜崖死。乘勝偪西門,與各軍環攻,克之。至是累功進同知直隸州,進剿湖北通城,督兵攻西北,澤南自將中軍繼之,斃悍賊數十。賊狂奔入城,諸軍疾躡之,奪門入,立復縣城。賊竄蒲圻,杏春敗之道口。賊踞梯木山,率眾攀藤上,焚其巢。

逆首石達開率大股來援,官軍分三路應之,杏春當右路松林之賊,躍馬登山,整隊以待。賊洶湧麕至,官軍突前擊之,斬執旗悍賊酋十餘人,餘眾驚走。明日,賊眾二萬來犯,眾議退師,杏春不可,曰:「大軍在後,退則全軍奪氣。」與參將彭三元扼要堵御,鏖戰五時,斬馘數百。咸寧賊悉眾來援,崇陽土匪響應,眾數萬,圍營三匝。杏春與三元分路馳突,相持兩時許,砲下如雨,三元戰死。杏春勒馬回救,麾下勸之走,弗從,曰:「彭參將死,我何忍獨生?」馳入賊陣,手刃悍賊一人而死。贈知府銜,附祀塔齊布專祠。

硃善寶,字子玉,浙江平湖人。由監生入貲為州判,剿海州、徐州匪,保同知,署江寧府督糧同知。咸豐十年,隨總督何桂清駐常州,江南大營陷,常州大震,桂清以守御事悉任善寶。既,賊陷丹陽,桂清遁,欽差大臣和春亦走無錫,提督張玉良收潰卒營城外,亦戰敗。賊從奔牛鎮來犯,城兵千餘,旦夕垂破。善寶以常州為蘇、浙門戶,常州不守,則蘇、浙瓦解,卒不去。賦絕命詩以見志,與通判岳昌勵眾登陴,殺賊千計。賊麕至,攻益力,城陷,戰青果巷,被十餘創,死之。恤世職。

莊裕崧,陽湖人。以監生輸餉獎通判,銓四川。佐駐藏幫辦大臣恩慶治里塘夷務,晉直隸州知州。初,裕崧幕游蜀,至是例回避,恩慶疏留辦善後。藏事畢,改省甘肅。同治元年,補鹽茶同知,廉慎自持,諳練政治。屬回目王大桂等以平遠回揚言漢民傳帖約期滅教,轉相煽惑,於是群回驚疑,謀起事。裕崧與涼州鎮總兵萬年新馳赴秦家灣敵營,曉以禍福,責以大義,回眾跪道左,咸聽命。裕崧等領赴固原,遣員分赴各莊,回戶皆就撫。獨臣賊馬彪、馬新成等抗拒不服,大桂立殺之,繇是無一敢抗者,事遂定。其年秋,循化、巴燕戎格撒拉回族時出攻剽,分擾西寧、碾伯、隆德、河州,居民苦之。裕崧奉檄與諸軍分道進擊,戰屢捷。撒回勢蹙,相率歸命。

二年,護理總督恩麟狀其績,晉知府。俄而固原回楊大娃子等犯鹽茶,年新戰失利,直逼城,裕崧率文武登陴固守逾月。賊力攻,內奸啟西門,遂長驅入。裕崧率團丁巷戰,矢盡糧絕,被執,擁至禮拜寺,百計威脅,詈賊不屈,遂及於難。前都司高如岡、照磨胡敉皆戰死。賊入署,執幕友四川舉人易舉索印,拒不與,並家丁李暢等十一人同時被殺。

年新,湖南人。固原失陷,馳往查辦,賊偽乞降,率眾潛至襲擊,為所執,不屈,死之。


\end{pinyinscope}