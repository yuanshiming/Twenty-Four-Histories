\article{志一百二十四}

\begin{pinyinscope}
○交通一

有清之世,歐洲諸國以制器相競致強富,路船郵電,因利乘權。道光朝五口通商,各國踵跡至。中外棣通,外舟侵入我江海置郵通商地。大北、大東兩公司海底電線貫太平洋、大西洋而來,亦駢集我海上,駸駸有返客為主之勢焉。李鴻章、郭嵩燾諸臣以國權、商務、戎機所關甚鉅,抗疏論列。其始也阻於眾咻,其繼也卒排群議而次第建設之,開我國數千年未有之奇局。於時鴻章總督直隸,領北洋通商大臣,忍詬負重,卒觀厥成。長江招商輪船局始於同治十三年。逮光緒三年,有唐山胥各莊鐵路之築。四年,設郵政局。五年,設電線於大沽、北塘海口砲臺,西達天津。自時厥後,歲展月拓,分途並進。輪船則有官輪、商輪之別,鐵路則有官辦、商辦之別,電線則有部辦、省辦之別,郵政則有總局、分局之別。宣統初,郵傳部計路之通車者逾萬里,線之通電者九萬餘里,局之通郵者四千餘處。歲之所入,路約銀二千萬,電約一千萬,郵六百餘萬,而歲支外所盈無幾,無乃分其利者眾歟?昔者車行日不過百里,舟則視風勢水流為遲疾,廷寄軍書,驛人介馬俟,盡日夕行不過六七百里已耳。今則京漢之車,津滬之舟,計程各二三千里而遙,不出三日,郵之附舟車以達者如之。若以電線達者,數萬里外瞬息立至。民情慮始難,觀成易,故船、電、路皆有商辦名。顧言利之臣胥欲籠為國有,以加諸電商者加之川漢自辦之路,操之過激,商股抗議者輒罪之。淫刑而逞,以犯眾怒,黨人乘之,國本遂搖。孔子論治,以書同文、車同軌、行同倫為極盛。清之天下,可謂同文同軌矣,惟行殊焉,而理亂頓異。則知伏羲氏所謂通天下之志者,有形下之器,尤貴有形上之道以維系之,未可重器而遺道也。撰交通志。

鐵路

鐵路創始於英吉利,各國踵而行之。同治季年,海防議起,直督李鴻章數為執政者陳鐵路之利,不果行。

光緒初,英人擅築上海鐵路達吳淞,命鴻章禁止,因偕江督沈葆楨,檄盛宣懷等與英人議,卒以銀二十八萬兩購回,廢置不用,識者惜之。

三年,有商人築唐山至胥各莊鐵路八犬里,是為中國自築鐵路之始。

六年,劉銘傳入覲,疏言:「自古敵國外患,未有如今日之多且強也。一國有事,各國環窺,而俄地橫亙東、西、北,與我壤界交錯,尤為心腹之憂。俄自歐洲起造鐵路,漸近浩罕,又將由海參崴開路以達琿春,此時之持滿不發者,以鐵路未成故也。不出十年,禍且不測。日本一彈丸國耳,師西人之長技,恃有鐵路,亦遇事與我為難。舍此不圖,自強恐無及矣。自強之道,練兵造器,固宜次第舉行。然其機括,則在於急造鐵路。鐵路之利,於漕務、賑務、商務、礦務、釐捐、行旅者,不可殫述,而於用兵尤不可緩。中國幅員遼闊,北邊綿亙萬里,毗連俄界;通商各海口,又與各國共之。畫疆而守,則防不勝防,馳逐往來,則鞭長莫及。惟鐵路一開,則東西南北呼吸相通,視敵所趨,相機策應,雖萬里之遙,數日可至,百萬之眾,一呼而集。且兵合則強,分則弱。以中國十八省計之,兵非不多,餉非不足,然此疆彼界,各具一心,遇有兵端,自顧不暇,徵餉調兵,無力承應。若鐵路告成,則聲勢聯絡,血脈貫通,裁兵節餉,並成勁旅,防邊防海,轉運槍砲,朝發夕至,駐防之兵即可為游擊之旅,十八省合為一氣,一兵可抵十數兵之用。將來兵權餉權,俱在朝廷,內重外輕,不為疆臣所牽制矣。方今國計絀於邊防,民生困於釐卡。各國通商,爭奪利權,財賦日竭,後患方殷。如有鐵路,收費足以養兵,則釐卡可以酌裁,裕國便民,無逾於此。今欲乘時立辦,莫如籌借洋債。中國要路有二:南路一由清江經山東,一由漢口經河南,俱達京師;北路由京師東通盛京,西達甘肅。若未能同時並舉,可先修清江至京一路,與本年擬修之電線相為表裏。」

事下直督李鴻章、江督劉坤一議覆。鴻章言:「鐵路之設,關於國計、軍政、京畿、民生、轉運、郵政、礦務、招商、輪船、行旅者,其利甚溥。而借用洋債,外人於鐵路把持侵占,與妨害國用諸端,亦不可不防。」坤一以妨兒民生、釐稅為言。學士張家驤言興修鐵路有三大弊。復下其疏於鴻章,鴻章力主銘傳言。會臺官合疏力爭,侍講張楷言九不利,御史洪良品言五害,語尤激切。以廷臣諫止者多,詔罷其議。嗣是無復有言之者矣。

十一年,既與法國議和,朝廷念海防不可弛,詔各臣工切籌善後。李鴻章言:「法事起後,借洋債累二千萬,十年分起籌還,更無力籌水師之歲需。開源之道,當效西法採煤鐵、造鐵路、興商政。礦藏固為美富,鐵路實有遠利。但招商集股,難遽踴躍,官又無可資助。若輕息假洋款為之,雖各國所恆有,乃群情所駭詫,非聖明主持於上,誰敢破眾議以冒不韙?」大學士左宗棠條上七事,一言宜仿造鐵路:「外國以經商為本,因商造路,因路治兵,轉運靈通,無往不利。其未建以前,阻撓固甚,一經告成,民因而富,國因而強,人物因而倍盛,有利無害,固有明徵。電報、輪船,中國所無,一旦有之,則為不可少之物。倘鐵路造成,其利尤溥。清江至通州宜先設立鐵路,以通南北之樞,一便於轉漕,而商務必有起色;一便於徵調,而額兵即可多裁。且為費僅數百萬,由官招商股試辦,即可舉行,且與地方民生並無妨礙。迨辦有成效,再添設分支。至推廣西北一路,尤為日後必然之勢。」疏下王大臣議,雖善其言而不能用也。是年冬,鴻章復言:「陶城、臨清間二百餘里,運道淤墊,請試辦鐵道,為南北大道樞紐。」上用漕督崧駿等言,格不行。

初,法、越事起,以運輸不便,軍事幾敗。事平,執政者始知鐵路關系軍事至要。十三年春,海軍衙門王大臣奕枻等言:「鐵路之議,歷有年所,毀譽紛紜,莫衷一是。自經前歲戰事,始悉局外空談與局中實際,判然兩途。臣奕枻總理事務,見聞較切。臣曾紀澤出使八年,親見西洋各國輪車鐵路之益。現公同酌覈,調兵運械,貴在便捷,自當擇要而圖。據天津司道營員等稟,直隸海岸綿長,防守不易,轉運尤艱。請將開平至閻莊商辦鐵路,南接大沽北岸八十餘里,先行接造,再由大沽至天津百餘里,逐漸興修。津沽鐵路告成,續辦開平迤北至山海關,則提督周盛波所部萬人,馳騁援應,不啻數萬人之用。此項海防要工,集資不易,應以官款興辦,調兵勇協同工作,以期速成。如蒙俞允,即派員督率開平公司經理。」從之。明年,路成。總理衙門奏言:「新造津沽鐵路,自天津府城經塘沽、蘆臺以至閻莊,長一百七十五里,其自閻莊至灤州之唐山,長八十里,為各商舊造鐵路。新舊鐵路首尾銜接,輪車通行快利,為輪船所不及。通塞之權,操之自我,斷無利器假人之慮。由此經營推廣,一遇徵兵運械,輓粟飛芻,咄嗟可致;商民貿遷,無遠弗屆,榛莽之地,可變通衢,洵為今日自強之急務。」

會粵商陳承德請接造天津至通州鐵路,略言:「現造鐵路,其所入不敷養路之用。如接造此路,既可抽還造路借本,並可報效海軍經費。」直督李鴻章以聞,已如所請矣;於時舉朝駭然,尚書翁同龢、奎潤,閣學文治,學士徐會灃,御史餘聯沅、洪良品、屠仁守交章諫阻。其大端不外資敵、擾民、失業三者,亦有言宜於邊地及設於德州、濟寧以通河運者。命俱下海軍衙門。尋議上,略言:「原奏所慮各節,一在資敵。不知敵至而車已收回,豈有資敵之慮?一在擾民。建設鐵路,首在繞避民間廬舍丘墓,其萬難繞避者,亦給重價,諭令遷徙,可無擾民之事。一在失業。鐵路興而商業盛,謀生之途益廣,更鮮失業之虞。津通之路,非為富國,亦非利商,外助海軍相輔之需,內備徵兵入衛之用。乃議者不察底蘊,不相匡助,或竟道聽途說,或竟憑空結撰,連章論列,上瀆天聽。方今環球諸國,各治甲兵,其往也,非干羽所能格,其來也,非牛餼所能退,全視中華之強弱,為相安相擾之樞機。臣等創修鐵路本意,不在效外洋之到處皆設,而專主利於用兵。不僅修津通之路,而志期應援全局。誠能於江南、趙北、關東、隴西各設重兵,各安鐵路,則軍行萬里無胼胝之勞,糧運千倉有瞬息之效,零星隊伍可撤可並,浮濫餉乾或裁或節。此外如海防河運,裨益實多,而通貨物、銷礦產、利行旅、便工役、速郵遞,利之所興,難以枚舉。而事屬創辦,不厭求詳。請下沿江沿海各將軍督撫,各抒所見。」遂如所請,命各詳議以聞。

臺灣巡撫劉銘傳議由津沽造路至京師,護蘇撫黃彭年議先辦邊防、漕路,緩辦腹地及沿江沿海各省,而試行於津通。粵督張之洞請緩辦津通,改建腹省幹路,疏言:「今日鐵路之用,以開通土貨為急。進口外貨,歲逾出口土貨二千萬兩。若聽其耗漏,以後萬不可支,惟有設法多出土貨、多銷土貨以濟之。有鐵路,則機器可入,笨貨可出,山鄉邊郡之產,悉可致諸江岸海壖,流行於九洲四瀛之外矣。而沿江沿海、遼東三省、秦隴沿邊,強鄰窺伺,防不勝防。若無鐵路應援赴敵,以靜待動,安得無數良將精兵利砲巨餉而守之?宜先擇四達之衢,首建幹路,為經營全局之計。至津通鐵路,則關系甚鉅,不便尤多。設此路創造之時,稍有紛擾,則習常蹈故者,益將執為口實,視為畏途。以後他處續造,集股之官商必裹足,疑沮之愚民必有辭,則鐵路之功終無由成,而鐵路之效終無由見矣。翁同龢請試行於邊地以便運兵,徐會灃等請改設於德州、濟寧以便運漕,均擬緩辦津通,為另闢一路之計。但邊地偏遠,無裨全局,效亦難見;且非商賈輻輳之所,鐵路費無所出,不足以自存。德濟一路,黃河岸闊沙松,工費太鉅。臣以為宜自京城外之盧溝橋起,經河南達於湖北漢口鎮。豫、鄂居天下之腹,中原綰轂,胥出其塗。鐵路取道,宜自保定、正定、磁州,歷彰、衛、懷等府,北岸在清化鎮以南,南岸在滎澤口以上,擇黃河上游灘窄岸堅經流不改之處,作橋以渡河,則三晉之轍下於井陘,關隴之驂交於洛口,西北聲息刻期可通。自河以南,則由鄭、許、信陽驛路以抵漢口,東引淮、吳,南通湘、蜀。語其便利,約有數事。內處腹地,不近海口,無引敵之慮,利一。南北三千餘里,原野廣漠,編戶散處,不似近郊之稠密,一屋一墳易於勘避,利二。幹路袤遠,廠盛站多,經路生理既繁,緯路枝流必旺。執鞭之徒,列肆之賈,生計甚寬,舍舊謀新,決無失所,利三。以一路控八九省之沖,人貨輻輳,貿易必旺。將來汴洛、荊襄、濟東、淮泗,經緯縱橫,各省旁通,四達不悖。豈惟有養路之資費,實可裕無窮之餉源,利四。近畿有事,三楚舊部,兩淮精兵,電檄一傳,不崇朝而雲集都下。或內地偶有土寇竊發,發兵征討,旬日立可蕩平。徵兵之道,莫此為便,利五。中國礦利,惟煤鐵最有把握。太行以北,煤鐵最旺而最精,而質最重、路最艱。既有鐵路,則輦機器以開採,用西法以煎鎔,礦產日多。大開三晉之利源,永塞中華之漏卮,利六。海上用兵,首慮梗漕。東南漕米百餘萬石,由鎮江輪船溯江而上,三日而抵漢口,又二日而達京城。由盧溝橋運赴京倉,道里與通縣相等,足以備河海之不虞,闢飛輓之坦道,而又省挑河剝運之浮糜。較之東道王家營一路礙於黃河下流者,辦理轉有把握,利七。若慮費鉅難成,則分北京至正定為首段,次至黃河北岸,又次至信陽州為二三段,次至漢口為末段。每里不過五六千金,每段不過四百萬內外,合計四段之工,須八年造成,款亦八年分籌。中國之大,每年籌二百萬之款,似尚不至無策。籌款之法,除由鐵路公司照常招股外,應酌擇各省口岸較盛、鹽課較旺之地,由籓運兩司、關道轉發印票股單,設法勸集。鐵料運自晉省,置爐鍊冶,以供取用,庶施工有序,而藏富在民。」

奏上,仍下海軍衙門。尋復議上:「各國興辦鐵路,以幹路為經,以枝路為緯,有事則以路徵兵,無事則以商養路。就五大洲言之,宜於西洋,宜於東洋,豈其獨不宜於中國?就中國言之,或云宜於邊防,或云宜於腹地,豈其獨不宜於臣衙門所奏準之津通?津通,畿東南一正幹也。水路受沿海七省之委輸,陸路通關東三省之命脈。豫鄂則畿西南一正幹也,控荊襄,達關隴,以一道扼七八省之沖。初意徐議中原,而先以津沽便海防,繼以津通擴商利,區區二百里,其關系與豫鄂之千里略同。今張之洞亦設為津通五宜審之說,其中所慮各節,前奏固已剖析無遺。惟事關創始,擇善而從。津通鐵路應即暫從緩辦,而盧漢必以漢口至信陽為首段,層遞而北,並改為盧溝、漢口兩路分投試辦,綜計需銀三千萬兩,以商股、官帑、洋債三者為集款之法。」議上,詔旨允之。

初,鴻章倡津通鐵路之議,舉朝以為不可,鴻章持之甚力。之洞特創盧漢幹路之說,調停其間,而醇親王奕枻復贊之於內,其事始定。然其時廷臣尚多不以盧漢造路為然,但無敢昌言者。故通政黃體芳謂鐵路不可借洋債以自累,而臺臣亦有言黃河橋工難成者,以執政者堅持舉辦,久之浮議始息。鴻章與之洞書,謂局外議論紛歧,宜速開辦,免生枝節,之洞深然之。未幾,之洞總督湖廣。之洞既移鄂,益銳意興辦盧漢鐵路,其所經畫,曰儲材宜急,勘路宜緩,興工宜遲,竣工宜速。以商股難恃,請歲撥帑金二百萬兩以備路用。上如所請。

十六年,以東三省邊事亟,從海軍衙門王大臣及直督李鴻章言,命移盧漢路款先辦關東鐵路。擬由林西造幹路,出山海關至沈陽達吉林,另由沈陽造枝路以至牛莊、營口,計二千三百二十三里,年撥銀二百萬兩為關東造路專★,命李鴻章為督辦大臣,裕祿為會辦大臣,而盧漢路工因之延緩。蓋自光緒初年,內外臣工往往條陳鐵路,當國者亦欲試行以開風氣,而疆吏畏難因循,顧慮清議,莫敢為天下先。盧漢鐵路已定議矣,尋復中輟。至是年,國內鐵路,僅有唐山至閻莊八十五里,閻莊至林西鎮二百三十五里,又基隆至淡水六十里而已。

二十一年,命張之洞遴保人才,及籌議清江至京路事。之洞言鐵路以盧漢為要,江寧、蘇、杭次之,清江築路非宜。上韙其言。時之洞方督兩江,特命移鄂綜其事。以盧漢路長款鉅,諭有招股千萬者,許設公司自辦。粵人許應鏘、方培■J0等咸言集貲如額,遵旨承辦。直督王文韶與之洞言承辦各商舉不足恃,請以津海關道盛宣懷為督辦,允之,命以四品京堂督路事。宣懷條上四事,一請特設鐵路總公司,撥官款,募商股,借洋債。先辦盧漢,次第及於蘇滬、粵漢。上如所請。是年設總公司於上海,而盧漢之始基以立。

自中日戰後,外人窺伺中國益亟,侵略之策,以攬辦鐵路為先。俄索接造西伯利亞幹路,橫貫黑、吉兩省,修枝路以達旅順、大連灣。英則請修五路:一蘇杭甬,自蘇州經杭州以達寧波;一廣九,自廣州以達九龍;一津鎮,自天津以達鎮江;一浦信,自浦口以達信陽;一自山西、河南以達長江。法自越南築路以達雲南省,自龍州築路以達鎮南關。德踞膠州灣,築路以達濟南。葡據澳門,築路以達廣州。日本擅於新民築路達奉天,更獲有奉天至安東鐵道之權。此各國以鐵路侵略中國之大略也。

先是俄人陰結朝鮮窺奉天,建言者請急建關內外路以相鈐制,乃命順天府尹胡燏棻督辦津榆路事;後以續造吉林一路款絀中輟。二十四年,俄事急,燏棻請息借英款為之。疏言:「關外一路,初擬逕達吉林,以無款又落後著。迨歸並津盧,俄即起而爭執。近允其由俄邊直接大連灣,奉、吉兩省東北之利盡為所占。計惟有由大凌河趕造至新民鐵路,以備聯絡沈陽之路,並可兼護蒙古、熱河礦務。一面由營口至廣寧,庶中國海關不致為俄侵占,尚可保全奉省西北之利。現東三省全局已在俄人掌握,幸留此一線之路,堪以設法抵御。若坐失機宜,後悔何及。」從之。

初,英人圖粵路甚亟。王文韶、張之洞、盛宣懷合疏言:「粵漢南幹路,原擬稍緩續籌,無如時局日亟,刻不及待。群雄環伺,輒以交涉細故,兵輪互相馳騁,海洋通塞,靡有定期。今海軍既無力能興,設有外變,隔若異域,必內地造有鐵路,方可聯絡貫通。廣東財賦之區,南戒山河,未可遐棄,此粵漢南路當與北路並舉者也。」又疏言:「德國無理肇釁,占踞膠、墨要害,並獲承辦山東鐵路。俄已造路於黑龍江、吉林,為通奉天、旅順之謀。法已造路於廣西,以為割滇之計。獨英人窺伺最久,尚無所得。今年春,英商屢來攬辦粵路,堅持未允。其所擬急行者,在趕營中國中部,或廣東建築軌道。蓋英所欲者,一借款,一修路,一擬索香港對岸之深水埠,其為覬覦鐵路無疑。現在德已踞膠,俄已留旅,法已窺瓊,英有圖扼長江、吳淞之謀。是中國各海口幾盡為外國所占,僅有內地尚可南北往來。若粵漢一線再假手英人,將來俄路南引,英軌北趨,惟有盧漢一路跼蹐其中,何能展布?甚或為英、俄之路所並。惟有趕將粵漢一路占定自辦,尚足補救萬一。」嘉納之。

初,粵漢路議由鄂入贛達粵。嗣病其迂遠,改道湘之郴、永、衡、長。至是,定議三省紳商自辦,總公司綜其綱領。蓋各省幹路,以關東肇其端,盧漢、粵漢次之。此外則建天津至盧溝橋之津盧路,正定至太原之正太路,鄭州東至開封、西歷滎陽、汜水達洛陽之汴洛路,廣州至九龍之廣九路,上海至江寧之滬寧路,萍鄉至昭山之萍昭路,道口至清化鎮之道清路,京師至張家口之京張路,天津至浦口之津浦路,吉林、長春之吉長路,齊齊哈爾卜魁城至昂昂溪之齊昂路,此屬於官辦者也。若潮汕、新寧、川漢、同蒲、洛潼、西潼、廣廈、歸包、歸新、桂全、滇桂、滇蜀、騰越以及浙、蘇、皖、贛、滇、蜀諸省,咸請自修幹枝等路,悉如所請。至是建造鐵路之說,風行全國,自朝廷以逮士庶,咸以鐵路為當務之急。

趨鄉既定,籌款與辦法最關緊要。籌款有官帑,有洋債,有民股。修路有官辦,有商辦,有官督商辦。自劉銘傳倡借債築路之議,為眾論所尼,借款修路,遂為當時所諱言。故盧漢建議之初,猶以部帑為請,未敢昌言借洋債也。借洋債自津盧、關內外鐵路始。迨盛宣懷督辦路事,首以三路分三國借款之策進。曰盧漢借比款,滬寧借英款,粵漢借美款。上俞其請。由是正太則借俄款,汴洛則借比款,廣九、蘇杭甬則借英款,津浦則借英、德款。貸之者,大率資金什予其九,息金二什而取其一;以路為質,或並及附路之產物。付息、還本、贖路,咸有定程,而還本、贖路未及其時,且勿許。購料、勘路、興工,多假外人為之。故外人多以款為餌,冀獲承辦之利。

盧漢路近三千里,費逾四千萬,黃河橋工糜款尤鉅,官帑僅資開辦而已。借款始擬美,以所望奢,改與比議。英、德、法諸國接踵而至。卒借比款一百十二兆五十萬佛郎。比小國,饒鋼鐵,嫺工事,於中國無大志。三十一年,續借一百二十五萬佛郎。逾年,路成。北端直抵京師,因易名京漢。京漢之枝路曰正太,曰汴洛。正太借款,始二十三年。俄璞科第與晉官紳議定而中止。二十八年,盛宣懷與議借款四千萬佛郎。約成,而俄人挨士巴尼忽索太原至榆次,至成都,至太谷,至西安,石莊至東光、微水、橫澗四岔道,及同蒲諸路。均格部議,而岔道卒如所請。三十三年秋,工竣。

自容閎倡辦津鎮,盛宣懷恐奪盧漢之利,因議辦汴洛、開濟以相鈐制。汴洛借款始於二十五年,至二十八年而約成,借比款二千五百萬佛郎。比人盧法爾主工事。嗣續借千六百萬佛郎。三十四年,路成。津沽用款百三十萬,官帑、商股兼備,以洋債補其不足。津盧假英金四十萬鎊。關內外路借英金二百三十萬鎊。本由商辦,迨胡燏棻為督辦,始官為之。拳匪亂起,關外路為俄踞,關內路為英踞。命袁世凱等與英使立約收回,英人遂攫有百里內不準他人承修之權。三十一年,全路告竣,是為京奉路。道清路為英商福公司所造,長九十里,利微費鉅。初,英商索澤襄、懷浦,俱不獲。遂以借款收回道清為言,內外臣工咸持不可,終借英金六十一萬四千六百鎊贖回。津浦路,因津鎮之議不果行,改議北起天津,南訖浦口,借英、德款五百萬鎊。尚書呂海寰主其事。宣統三年,工竣。

其促成各省鐵路自辦與拒絕外債之機者,則滬寧、蘇杭甬、粵漢借款所致也。滬寧築路,倡於盛宣懷,南北洋大臣據以入告,得請。方從事淞滬工作,而英聲請承辦,宣懷與訂草約。二十九年,正約成,借英金三百二十五萬鎊,五十年為期。商部以借款幾倍於原估之數詰之。而工未及半,款已告罄,復議續借百萬鎊。蘇人群起責難,並疏聞於上。命唐紹儀督辦滬寧、京漢,罷鐵路總公司。紹儀既任事,徇英工程司之請,復議售小票六十五萬鎊。疏言:「盛宣懷移交合同文卷及購地工程帳冊,支款浮濫,當經駁回。滬寧合同吃虧,比京漢幹路為甚。其最棘手者,在設立總管理處。華員二人,洋員三人,每會議時,彼眾我寡,已占低著。議者有添舉監督之說。豈知權在總管理處,合同早已訂明,雖有監督,實不濟事。其尤棘手者,財政之權操於洋人掌握,用款雖由華員簽字,而司帳者為洋員也。分段司帳,其支發權仍在工程司也。購料事宜,向由怡和洋行經手。行車總管、材料總管,皆洋員專司也。本彼眾我寡之因,以成事事掣肘之果。挽回補救之術,惟有改訂總管理處章程,加派華員司帳,並分任各總管,現已分別辦理。至路款不敷,尚擬續售小票六十五萬鎊以資接濟。」下所司議行。方紹儀擬續借英款也,侍郎吳鬱生上疏力爭,略言:「滬寧鐵路由英國銀公司要求承造,盛宣懷與之訂立合同。以長不逾六百里之路工,借款至三百五十萬鎊之鉅,估價多,必至浮濫。自合同宣布後,遠近駭然。上年奉嚴旨改派唐紹儀妥籌辦理。近聞滬寧工程司來京,又以工款不敷,有議續售小票七十萬鎊之說。此項路工,即就業經借定之三百五十萬鎊侭數開支,每里合銀三萬兩以上,視他路浮逾兩倍,公家受虧已多。今若再借鉅債,是唐紹儀接辦以來,於盛宣懷失算之處並無補救之方。請飭按照合同,嚴覈用款,一面自行籌款接濟。不可再令銀公司出售小票,致以九折虛數,受人盤剝。」疏上,下所司知之。而滬寧鐵路終以本息過鉅,收贖無期也。

蘇杭甬鐵路,自二十四年許英商承辦。是年,盛宣懷與訂草約,大要悉本水扈寧。約成而英人置之。三十一年,浙路自辦之局定,御史硃錫恩請廢前約,上命宣懷偕浙撫主其事。英人恃有前約,堅欲承辦,往復辨難要挾,久之不決。侍郎汪大燮與議,分修路、借款為二事。浙人以路股集有成數,一意拒款,聞之大譁,詆大燮甚力。大燮旋使英,以梁敦彥繼之。浙推孫廷翰、蘇推王同愈等議於京,終以成約難廢,由部借英款,貸之兩省而事息。

粵漢借用美款,倡於盛宣懷。駐美使伍廷芳與合興公司議借美金四千萬,期以五年工竣。美以畢來斯司路事。起粵之三水,築路十五里,糜款逾二百萬。畢來斯歿,工事亦輟,而美股多售之比人。鄂督張之洞以比已承修盧漢,粵漢再假之比,兩路相合,非國之利,力倡廢美約之議,湘人助之。上用御史黃昌年言,命之洞妥籌辦理。之洞主廢約益力。宣懷不原,陰撓其事,詔宣懷不得干預。之洞復屬駐美使梁誠與合興公司議,年餘始定,借英金百十萬鎊贖回焉。

方之洞議借英款也,英人乘間請改訂廣九路約。廣九為英人請辦五路之一,二十五年簽訂草約,懸而未定者也。三十年,滬寧約成,英人索議未果。迨蘇杭甬事起,相持方急,部許英人先議廣九,以緩其事,而正約以成。至是議粵漢借款,英人復索合辦廣九全路,粵督持不可。旋索以粵鹽及路質借款,粵人亦不之許。終假英金百五十萬鎊而約成。之洞既借英款贖美約,一時議者以為以英易美,其害相埒,相與詬病。昌年復言路權至重,贖款難擔,亟宜興修,嚴杜干涉。詔以借款修路,流弊滋多,應由三省集股興修,以保利權。自明詔嚴禁借債修路,而商部復有限制借款之條。各省人士亦以外人謀我之亟,咸謀鐵路自辦,以杜外患,鑒外債受虧之鉅,爭欲招集股款,自保路權。此由官辦改為商辦之所由來也。

商辦鐵路,始於唐山至閻莊,更自天津、大沽以達林西鎮,皆開平公司為之。嗣是武舉李福明請修京至西沽路,粵人許應鏘等請辦盧漢路,俱不獲,自此無復有言商辦者。二十九年,粵人張煜南請設公司承辦潮汕鐵路。既得請,而川漢繼之。川督錫良以英、美商人競涎川路,而美商班士復索灌縣富順枝路,奏準由川人籌款自辦。明年,贛人以李有棻總理江西鐵路,以南潯為幹路第一段。三十一年,編修陳榮昌等以法人已修滇越路,滇省內地應自行推廣,以杜口實,請辦滇蜀鐵路,滇督丁振鐸據以入告,報可。黔撫林紹年言黔路不通,滇亦少利,因並及黔。榮昌嗣請展修騰越,以編修吳琨總理其事。皖以李經方為總理,經始於蕪湖,以期北接盧漢、南通贛浙。閩以陳寶琛為總理,築路廈門。浙以湯壽潛為總理,幹路一自杭達蘇,一歷富陽達江西;枝路則南道江山以通閩,西道湖、長以通皖。新寧、廣廈鐵路,粵人陳宜禧、張振勛經辦。西潼路,近聯汴洛,遠達甘新,為西北緯幹之樞紐,陜撫曹鴻勛奏準。三十二年,蘇人以王清穆為總理,規畫江蘇全路,江南自上海經松江以達浙江,北自海州入徐以達豫。桂以於式枚為總理,擬自桂林築路至全州以達湘,經梧州以達粵。粵漢自美約廢後,三省公設路局於鄂,籌款築路,各自為之,不相攙越,先幹後枝,以為要約。湘以袁樹勛為總理,粵人內閣侍讀梁慶桂、道員黎國廉與粵督岑春煊爭粵路商辦,被劾奪官。上命往查,旋起二人原官,路由官督商辦,旬日集股數達四千萬元,以鄭官應為總理。

當其時,以鐵路為救時要圖,凡有奏請,立予俞允。請辦幹、枝各路,經緯相屬,幾遍全國。其籌款,於招集民股外,大率不外開辦米穀、鹽、茶、房屋、彩券、土藥等捐,及銅元餘利、隨糧認股數者。而程功之速,事權之一,首推新寧。陳宜禧者,籍新寧,嫺鐵路學,眾相推戴,始終其事,故二年而路竣。次則潮汕,雖勘路招股,事變屢起,而卒底於成。總理張煜南,獎擢三品京堂。此外,以粵漢路粵人集股為最多,傾軋亦最劇。總理屢易,路工停滯。川省以租股為大宗,數達千餘萬元。浙、閩、皖、贛亦均次第興工。其餘各省,大都集股無多,有名鮮實。西潼一路,以商股難成,奏歸官辦,其見端也。

三十四年,上用蘇撫陳啟泰言,以大學士張之洞督辦粵漢,冀以統一事權,亦無所濟。是年,詔以鐵路為交通大政,紳商集股,各設公司,奏辦有年,多無起色,命所司遴員分往查勘。尋奏上勘路查款辦法。時川漢已派員往查。其餘以洛潼、西潼、同蒲、江蘇、浙江最要,為一起;粵漢、潮汕、新寧、惠潮、廣西、福建次要,為一起;滇蜀、安徽、江西再次,為一起。擬先查洛潼、西潼、同蒲三路,報聞。宣統二年,川路司出納者,虧倒路股百九十餘萬,川人宦京者甘大璋等聞於上,查明飭追,徒託空言而已。

三年,給事中石長信言:「我國興造各省鐵路,事前並未謀定後動。今宜明定幹路、枝路辦法,使天下咸知國家鐵路政策之所在,此後有所遵循,不再如從前之群議龐雜,茫無主宰。當此時事日亟,邊防最要。國家若不趕將東西南北諸大幹路迅速次第興築,則強鄰四逼,無所措手。人民不足責,其如大局何。此中利害,間不容發。惟有仰懇乾綱獨斷,不再游移。在德、奧、法、日本、墨西哥諸國,其鐵路均歸國有,而我分枝路與民,已為優異。況幹枝相輔,上下相維,於理尚順,於事稍易。此路政之大綱,亟宜明定辦法者一也。又東南幹路,以粵漢議辦為最早。光緒二十六年,督辦大臣會同湖廣總督等奏準借美款興造。當時訂定合同後,業已築成粵省之佛山三水鐵路一百餘里,廣州至英德幹路亦已購地開工。乃三十年春間,張之洞忽信王先謙等之言,不惜鉅資,經向美公司廢約,堅持固執,卒至停罷。廢約後,原欲集鄂、湘、粵三省之力以成此路。詎悠忽數年,粵則有款而紳士爭權,辦路甚少,湘、鄂則集款無著,徒糜局費。張之洞翻然悔悟,不護前非,仍擬借款築造,乃向英、德、法三國銀行訂定借款草合同,簽押後正欲入告,因美國援案插入,暫緩陳奏。張之洞旋即病故,此事遂一擱至今。計自廢約以來,已閱七載。倘若無此翻覆,粵漢早已告成,亦如京漢,已屆十年還本之期矣。至川漢集款,皆屬取諸田間,其款確有一千餘萬。紳士樹黨,各懷意見,上年始由宜昌開工至歸州以東,此五百里工程,尚不及十分之二三,不知何年方能告竣。而施典章擅將川路租股之所入,倒帳竟至數百萬之多。此又川、粵、漢幹路之潰敗延誤,亟宜查辦者又一也。近來雲貴督臣李經羲議造滇桂邊路,於國防尤有關系。然不有粵漢幹路自湖南之永興與廣西之全州相接,則滇桂路何能自守?今我粵漢直貫桂滇,川漢遠控西藏,實為國家應有兩大幹路,萬一有事,緩急可恃。故無論袤延數千里之幹路,斷非民間零星湊集之款所能圖成,即使遲以十年或二十年,造成之後,而各分畛域,倘於有事之際,命令不行,仍必如東西洋之議歸國家收買。此幹路之必歸國有者又一也。國家成法,待民寬厚,雖當財賦極困難之時,不肯加賦。四川、湖南現因興造鐵路,創為租股名目,每畝帶徵,以充路款。聞兩省農民,正深訾怨,偶遇荒年,追呼尤覺難堪。但路局以路亡地亡之說驚哧愚民,遂不得不從。川省民力較紓,尚能勉強擔負。湘民本非饒足,若數年之間,強逼百姓出此數千鉅萬之重貲,而路工一日不完,路利一日無著,深恐民窮財盡,欲圖富強而轉滋貧弱。是以幹路收歸國有,命下之日,薄海百姓,必無阻撓之慮。況留此民力以造枝路,其工易成,其資易集,其利易收。使其土貨得以暢行,民間漸資饒富,此枝路之可歸民有者又一也。」

疏上,下所司議行。詔曰:「中國幅員遼闊,邊疆袤延數萬里,程途動需數閱月之久,朝廷每念邊防,輒勞宵旰。欲資控御,惟有速造鐵路之一策。況憲政之諮謀,軍務之徵調,土產之運輸,胥賴交通便利,大局始有轉機。熟籌再四,國家必有縱橫四境諸大幹路,方足以資行政而握中央之樞紐。從前規畫未善,並無一定辦法,以致全國鐵路,錯亂紛歧,不分枝幹,不量民力,一紙呈請,輒行批準商辦。乃數年以來,粵則收股及半,造路無多;川則倒帳甚鉅,參追無著;湘、鄂則開局多年,徒資坐耗。竭萬民之脂膏,或以虛糜,或以侵蝕,恐曠時愈久,民累愈深,上下交受其害,貽誤何堪設想。用特明白曉諭,昭示天下,幹路均歸國有,定為政策。所有宣統三年以前,各省分設公司、集股商辦之幹路,延誤已久,應即由國家收回,趕緊興築。除枝路仍準商民量力酌行外,其從前批準幹路各案,一律取消。至應如何收回之詳細辦法,著度支部、郵傳部悉心籌畫,迅速請旨辦理。」

度支部奏:「粵、川、湘、鄂四省所抽所招之公司股票,盡數收回,由度支、郵傳兩部特出國家鐵路股票,常年六釐給息。嗣後如有餘利,按股分給。倘原抽本,五年後亦可分十五年抽本。其不原換國家鐵路股票者,均準分別辦理,以昭平允。粵路全系商股,因路工停頓,糜費太甚,票價不及五成。現每股從優發給六成,其虧耗之四成,發給國家無利股票。路成獲利之日,準在本路餘利項下,分十年攤給。湘路商本,照本發還。其米捐、租股等款,準發給國家保利股本。鄂路商股,並準一律照本發還。其因路動用賑糶捐款,準照湖南米捐辦理。川路宜昌實用工料之款四百數十萬兩,準給國家保利股票。其現存七百餘萬兩,原否入股,或歸本省興辦實業,仍聽其便。」從之。詔停川、湘兩省租股。起端方以侍郎督辦粵漢、川漢鐵路。其粵漢、川漢,英、德、法三國借款,亦即簽訂。

方幹路收歸國有之詔既頒,湘、粵人士群起譁譟,力謀抗拒,顧未久即定。護川督王人文代陳川諮議局請緩接收川路,詔旨斥之。川人羅綸等言:「部臣對待川民,均以威力從事,毫不持平。」人文復據以上聞,仍嚴斥之。未幾,以趙爾豐署四川總督。川人因路事持久不決,始以罷市、罷課,抗糧、抗捐,發布自保商榷書;繼則集眾圍攻督署,再攻省垣。遂命端方率軍入川。又以川事日棘,命前粵督岑春煊赴川辦理剿撫。春煊既受命,請以現金償川省路股,桂撫沈秉堃亦以為言,部議借英金三百萬鎊,不能決也。春煊至鄂,會成都圍解,稱疾不往。

御史陳善同上章,請罷斥郵傳大臣盛宣懷,以弭巨變。疏言:「竊維國以民為本,自古未有得民心而國不興者,即未有失民心而不危者。傳曰:『眾怒難犯。』書曰:『民可近,不可下。』此中消息至微。此次以鐵路幹線歸國有,政策本極相宜。比者屢詔蠲除各項雜捐,所以恤民者,固已仁至義盡。而湘、粵等省人心惶駭,擾擾不靖,川患且日以加劇者,則以郵傳大臣盛宣懷於此事之辦理實有未善也。各路商辦之局,其始皆歷奉先皇帝諭旨,根據大清商律。如欲改歸官辦,自應統籌全局,劃定年限,分期分段,量力遞收,於國於民,方為兩利。今盛宣懷事前毫無預備,徒仰仗借款,突然將批準各案奏請一律取消。各路以十餘年之經營,千數百萬之籌集,一旦盡取諸其懷而奪之。而所訂借款合同,利率之高,虛折之多,抵押之鉅,債權之重,又著著失敗,予人口實。各省人民,痛念前勞,怵心後禍,宜其奔走駭告,岌岌若不終日也。查給事中石長信之請定幹路、枝路辦法,在四月初七日;郵傳部之覆奏,宣布國有政策,在十一日;而借款合同之簽押,在二十二日。似政策之改定,實緣借款而發生也者。舉辦此等大事,乃平時漫無布置,出以猝遽如此,反使朝廷減輕民累之恉晦沒不彰。而復不能審慎臨機,強令宜歸工程每月工項仍由川款開支,實與五月二十一日上諭『川路仍存七百餘萬,原否入股,或辦實業,並聽其便』等語大相違背。必欲使我皇上體恤商民之恩,壅遏之不使下逮,陷朝廷以不信,示天下以可疑,群起抵抗,何怪其然。幸以國家三百年來深仁厚澤,淪浹人心,故雖眾怨交集於盛宣懷,終無敢有歸怨朝廷者。比聞川省風潮日烈,皆以盛宣懷喪權誤國,欲得而甘心。月餘以來,屢開全省股東大會,每次到者近萬人,誓與路為存亡,在場之人無不為之泣下。合十餘州縣地方,並相約不納錢糧,不上捐輸,學堂停課,商民罷市。各戶恭設先皇帝靈位,朝夕痛哭,人無樂生之心,士懷必死之志,愁慘蕭條,如經大劫,至可憐念。夫今日皇皇失所之窮民,皆國家袁皿々在疚之赤子,情形狼狽至此,我皇上聞之,必有惻然動念者。若不亟為拯救,萬一相持不解,稍延時日,或有不軌之徒,從中鼓煽,強者並命於尋仇,弱者絕望於逃死,眾志一睽,全體瓦解,終非國家福也。現在湘、鄂爭路,餘波尚未大熄,而雨水為災,幾近十省,盜匪成群,流亡遍野。若川省小有風鶴之警,恐由滇、藏以至沿江、沿海,必有起而應之者,其為患又豈止於路不能收而已。頃者我皇上諭派鄂、粵、川、湘等省督撫,令於所轄境內鐵路事宜各得會同辦理。盛宣懷剛愎自用,不洽輿情,已可概見,應如何懲處,以儆將來。至川民爭議,久懸不斷,終慮釀成鉅變。應責成督辦、會辦各大臣,酌度情形,妥速維持,以息眾喙。」時宣懷入為郵傳大臣,幹路收歸國有、及息借外債築路、處分四省路股,實主其事,故善同及之,語至切直。

疏入不省。而川省溫江等十餘州縣民團,每起數千或萬人,所至焚掠,勢極猖獗。大軍擊退之,旋據崇慶、新津、彭山,而嘉定、灌縣相繼失陷。邛州軍隊譁變,汶川縣署被毀,命湘、鄂、陜諸軍赴援。會鄂事起,川亂愈亟,以岑春煊為川督,而川省旋為民軍所據,端方、趙爾豐均及於難。乃罷盛宣懷以謝川人,而國事已不可為矣。

蓋論辦路之優劣,官辦則籌款易、竣工速,自非商辦可及。而外債之虧耗,大權之旁落,弊害孔多,亦遠過於商路。惟京張鐵路,以京奉餘利舉辦,詹天佑躬親其役,絲毫不假外人,允為中國自辦之路。而鄂之鐵廠,制鋼軌以應全國造路之需,挽回大利,尤為不鮮。統計官辦之路:京漢長二千六百三十里,資本金一萬萬零五百六十二萬八千餘元。京奉長二千二百四十六里,資本金五千零八十八萬四千餘元。津浦長一千八百六十三里,資本金八千零四十九萬餘元。京張長五百四十六里,資本金一千零三十二萬餘元。滬寧長七百二十五里,資本金三千六百五十三萬餘元。正太長六百二十三里,資本金二千三百十二萬六千餘元。汴洛長四百零二里,資本金二千零五十萬元。道清長三百三十里,資本金九百五十四萬九千餘元。廣九長三百零三里,資本金一千一百六十六萬二千餘元。吉長長一百四十里,資本金一百二十萬三千七百零四元。萍株長二百零五里,資本金四百六十一萬六千餘元。齊昂長五十六里,資本金四十八萬八千餘元。商辦之路:浙江長三百四十二里,資本金一千二百七十八萬八千餘元。新寧長二百六十里,資本金四百零八萬九千餘元。南潯長七十七里,資本金三百五十萬六千餘元。福建長二十八里,資本金二百四十二萬八千餘元。潮汕長八十三里,資本金三百五十四萬六千餘元。其借外債所築各路,惟京漢屆期贖歸我有,其他則尚未及雲。


\end{pinyinscope}