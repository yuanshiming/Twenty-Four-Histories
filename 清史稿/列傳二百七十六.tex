\article{列傳二百七十六}

\begin{pinyinscope}
忠義三

宗室恆斌倪國正趙文哲王日杏汪時孫維龍吳璜吳鉞等

曹永訚何道深沈齊義陳枚吳躭等溫模邵如椿李南暉

湯大奎周大綸壽同春李喬基熊恩紱宋如椿趙福劉升滕家瓚

蕭水清劉大成王翼孫王行儉王銑汪兆鼎左觀瀾董寧川

韓嘉業葉槐陸維基毛大瀛張大鵬白廷英楊繼曉

楊堂等曾艾曾彰泗羅江泰霍永清強克捷趙綸等

宗室奕湄景興陳孝寬等王鼎銘呂志恆邵用之等

楊延亮師長治王光宇

宗室恆斌,字絅文,太宗第十子輔國公韜塞後。授三等侍衛。父薩喇善,官吉林將軍,緣事命戍伊犁。方臥病,恆斌陳請代奏以身從父往,詔許之,而以沽名褫其職。恆斌在途侍疾,至廢寢食,父每怒其愚,無幾微怨。既抵伊犁,父疾以瘳,將軍阿桂大賢之。會哈薩克新附,遣使入貢,有旨擇賢員伴送,阿桂即命恆斌充伴送官。途間馭陪臣忠信得大體,召見慰藉,復其官,令留京供職。恆斌請畢伴送事仍往伊犁侍父,允之。擢二等侍衛。會烏什回叛,恆斌隨將軍明瑞由伊犁倍道進剿,戰屢捷。領左翼兵陣城南山下,賊麕至,恆斌奮勇擊之,所向披靡。賊懼,隱城壕誘敵。怒馬而前,萬鏃齊發,不及御,陣亡。事聞,褒恤,而宥其父罪還京。

倪國正,字懋功,四川成都人。康熙舉人。雍正十年,揀發廣西,授義寧縣知縣。義寧東北曰雙江,苗、民雜處,與湖南城步、綏寧二邑紅苗接壤,計千餘里。隘口十,堡七十二,大小寨凡數百。不通教化,僅設雙江巡檢羈縻之。乾隆六年,楚匪黃順等煽粵苗,偽稱名號。國正計擒黃順,中道被劫,還合楚苗為奸。國正牒文武諸大府請兵,撥駐四百名,苗稍靖。時大府意在招撫,知府張永熹、巡檢蔡多奇迎合其意,遂撤駐兵,而檄國正與多奇及縣丞吳嗣昌同往。國正將行,嘆曰:「此所謂投虎以肉,徒肆其噬耳!」

行數日,抵苗巢,苗挾兵出迓,氣囂甚,多奇易衣遁,眾失色。或告國正:「不去,禍將及!」國正曰:「吾固知犬羊之性,不先以威,不可以德化也。今日之事,有死而已。」付健役縣印,令間道馳還,正襟坐待之。苗突至,取官弁及隨行隸三十餘人,盡掊殺之。禁國正土窯,絕粒六日,縛至烈日中,去其衣,掘土埋足至膝,脅之降,不屈。授以紙筆,令「省中以萬金為贖,可不死。」國正擲筆裂紙,大罵。苗怒,擊其齒,血流被衣,聲益厲。齒盡,截其舌。國正猶噴血作罵狀,遂擊死,沉尸深潭。事聞,帝為之輟食。國正為諸生時,書押則云「為國盡忠」,案頭玉尺,亦刻「丹心捧日」,蓋報國之志,本素定云。

趙文哲,字升之,江蘇上海人。生有異稟,讀書數行下。同時青浦王昶,嘉定王鳴盛、曹仁虎皆以能詩名,獨心折文哲。為人瘦不勝衣,而意氣高邁,由廩生應乾隆二十七年南巡召試,賜舉人,授內閣中書,在軍機章京上行走。以原任兩淮鹽運使盧見曾查抄案通信寄頓,褫職。時大軍征緬甸,署云南總督阿桂奏請隨軍。阿桂由緬至蜀,將軍溫福方督師征金川,見文哲,與語,大悅之。時溫福與阿桂分兵,文哲遂入溫福幕。溫福重文哲,片時不見,輒令人覘文哲何作。已而連克金川地,三十七年十月,遂剿平美諾。以功復中書,又授戶部主事,仍隨營治事。三十八年,兵至木果木,六月,小金川降者叛,與金川合抄後路,師將潰,在軍者逆知賊大至,相率逃竄,文哲毅然以為:「身為幕府贊畫,且疊荷國恩,詎可舍帥臣而去!」卒與溫福同死。

同時遇害者:刑部主事王日杏、新繁縣知縣徐瓚、酆都縣知縣楊夢槎、合州吏目羅載堂。其在各臺站遇害者:潼川府通判汪時、漢州知州徐諗、內江縣知縣許椿、大竹縣知縣程廕桂、秀山縣巡檢郭良相、納溪縣典史許濟。其沿途被害者:候補從四品王如玉,候補知縣孫維龍、張世永,布政司照磨倪鵬,候補縣丞倪霖,秀山縣典史周國衡。先後殉難者:又有重慶府知府吳一嵩,原任貴州大定府知府鍾邦任,刑部主事特音布,原任湖南澧州知州吳璜,原任浙江雲和縣知縣彭元瑋,四川崇慶州知州常紀,原任廣西越巂通判吳景,納溪縣知縣章世珍,營山縣典史吳鉞。幕客同與難者:硃南仲、楊紹沂、熊應飛、田舒祿、顧佐、岳廷栻、周煒、鄭文、許國、長炳、王鳴鏞十一人。事定,四川成都府、金川崇化屯先後建祠祀之,均建慰忠祠碑。

日杏,字丹宸,江蘇無錫人。善書,於魏、晉以降書跡臨摹畢肖。官中書,行走軍機處。每扈從行圍,遇公事旁午,坐馬上盤一膝,置紙膝上,信筆作小楷,疾如飛。有官中書者,見機要大臣,跽一足請事,日杏怒詈之,謂為非人。知銅仁府,民以王青天稱之。

汪時,浙江錢塘人。時駐岱多喇嘛寺,寺破,罵賊死。官軍收復小金川,見血影濺涅壁間,尚漉漉如濕焉。

程廕桂,浙江仁和人。與其子烈同遇害。

孫維龍,字普田,寄籍宛平。先官安徽黟縣知縣,創立書院,延劉大櫆教士。又建石橋於漁亭鎮,通浙、楚往來,行旅稱之。

吳璜,字鑒南,浙江會稽人。父爚文,舉博學鴻詞科。璜為商盤甥,早以詩名。

常紀,字銘勛,奉天承德人。以進士選授西充縣,有治行。嘗興建關神武祠,殉難後,縣民即關祠肖紀像祀之。

吳鉞,字炳臣,河南固始人。賊犯木果木時,鉞守澤耳多糧站,去大營六十里,大營以東,澤耳多以西,松林溝、赤裏角溝,俱為賊奪。事急,有勸鉞走者,鉞奮然曰:「吾奉命守此,與站存亡,分也!與我共殺賊者,吾骨肉也!」因拔佩刀立木城旁,曰:「敢言走者斬!」眾心稍定。賊至,鉞率兵役禦之,眾寡不敵,火器環擊木城,猶徒手抵賊,卒被戕。

曹永訚,字文甫,浙江金華人。雍正七年武舉人,補江南大河衛千總,洊擢四川海寧營參將。御士卒嚴而有恩,多樂為用。乾隆三十六年,隨溫福征小金川,提督董天弼檄守牛廠石卡,旋為賊據,天弼自劾,並請治永訚罪。上念小金川事棘,置未問。永訚乃與阜和游擊宋元俊獻三策:一自斑斕山探小金川,擊其首;一自美諾趨甲金達,擊其中;一自約咱進取僧格宗,擊其尾。用其言頗效。永訚善謀,謀定而戰,元俊諳地利,進退有度,軍中曹、宋齊名。不數月,悉復明正侵地,前後十餘捷。

三十七年,隨攻克布朗郭宗、底木達,執僧格桑父澤旺以獻。明年,師以賊扼險不得進,別取道攻昔嶺,移營木果木。未嚴備山後要隘,而賊突薄大營,劫糧臺,奪砲局,會運糧士卒數千爭避入營,溫福堅壁不納,轟而潰,賊蹂入,溫福遂遇害。是時,永訚軍距稍遠,聞砲聲,遽嚴甲起,飛騎至,曰:「大營失矣!」問:「大將軍安往?」曰:「不知。」傍一騎至,曰:「宜速退!」叱曰:「大將軍不知所往,吾將焉往?」即蹀血進,殞於陣。同時殉難者,參將惠世溥以下四十七人。

何道深,字會源,山西靈石人。由武進士、乾清門侍衛出為貴州提標游擊。乾隆三十二年,兵部尚書明瑞總督云、貴,進討緬甸,集諸道兵。明瑞聞道深訓練營卒可用,檄至永昌,果整練冠他軍。三路出師,以道深所統自隨。從取木邦,破錫箔,逾天生橋,大戰蠻結。賊立十六寨,豎木柵,列象陣力拒。道深冒矢石,攀柵先登,火槍中右額,紀功第一。

又從入窮乍,去賊巢阿瓦城益近。賊斷木壘石守隘,官軍糧少,火藥鉛丸垂盡,將旋,賊抄其後。道深為之殿,遇山谷險厄,必奮戰,俾全師得度至猛育。未至猛育前二日,道深中鳥槍,夜息,部下校進曰:「傷重矣。賊至日眾,道險,難與敵。盍稱病且逸歸乎?」道深曰:「賊眾,乃將卒致力時也。」叱之退。明日,戰益力。初,明瑞將中軍趨錫箔,別將分左右軍,異路約會師。及至猛育,兩軍渝約,前阻大山,賊盡塞蹊隘,環圍數重,軍殺馬以食。三十三年二月,明瑞令夜拔營,以次沖出。平明,賊來邀截,道深立高岡指揮拒之,他軍士得從旁脫出。道深自晨戰至日中,被數創,始僕。

道深撫士嚴而有恩,其始聞檄調也,令二日即行,凡無子、無兄弟者皆弗從。歿後,軍皆悲涕,以其帶、發還,詔賜葬本邑。

沈齊義,字立人,浙江烏程人。乾隆九年舉人,大挑用知縣,發山東。歷權冠、汶上、費、齊東等縣,題補泗水。齊義有吏能,初往鉅野辦賑,慮吏胥作奸,親自登記,歷數十里皆然。冠有翦辮訛言,謂妖人翦人辮發,能以咒語攝其魂,令移他處錢物入己,被翦者數日即死。訊無實,悉縱去。他縣獄上,皆獲譴,人服其識。汶上為入都孔道,東門外石橋久圮,撤而新之。南旺、蜀山、馬踏三湖,為漕渠水櫃。堤壞,出貲修築,工固而民不擾。泗水多閒田,而民間畜蠶者少,悉令栽桑飼蠶,自此隙地皆桑,繭絲之饒甲他邑。費有冤獄,特緩其事,或以吏議懼之,齊義謂與其令民以冤死,毋寧被劾以去官。

三十七年,改授壽張。縣境趙王河湮三十餘年,大雨至,水無所洩,禾麥皆淹死,民多逃去。請募夫開濬,凡三十餘里,上引範、濮諸水,悉達諸五空橋,自是南鄙無水患,民皆復業。故明籓府私田賦極輕,入清謂之「更名地」。部議加賦,壽張更名地二千四百餘頃,先於雍正間,歸入籽粒地,加賦,而舊名猶存。乃檢尋故牘,以原委達大府,削除之。故事,歲辦河工稭料及解京黃蠟,分里購買,吏用為奸,為往他所買解,民得免累。所至求民利病,若其身事。

三十九年八月二十八日夕,陽穀縣黨家店奸民王倫糾眾突起為亂,入壽張。齊義聞變,即衣冠出蒞宅外,斥曰:「吾非贓吏,爾等劫我何為?」賊伏拜曰:「知公廉,民等亦素沐公恩。但須及早從順,順則生,逆則死!」齊義駭曰:「爾等不顧赤族誅耶?」大罵之。賊謂齊義不知生死,麾眾退,令自為計。齊義即入,解其印,令掘坎埋之。復出,家人及賓友挽其臂,揮去,趨宅外,僕又牽馬至,請上省告急,齊義曰:「若將使我蒙面見上官耶?」批其頰斥之。須臾,賊復至,有泣拜求請者。齊義大怒,拳足交下。賊擬以兵,齊義毒罵不絕口,遂攢殺之。先數日,齊義聞陽穀有妖人聚眾,遣人四出偵刺,賊懼禍及,首劫壽張,故齊義罹於難。賊既破壽張,遂掠陽穀。堂邑縣奸民王聖如亦劫殺村落應倫,權縣事陳枚死之。

枚,字元幹,廣西全州人。由舉人揀發山東,用知縣。聞聖如亂作,即馳往搜捕,盡逮賊黨妻子系獄,而聖如以倫眾數千至。邑無城守具,人情恟懼。枚本攝任,將受代,或勸枚引去,枚指天日自誓,與城存亡。城陷,被執,怒目視賊。賊曰:「攝令為令清,赦勿殺。」枚愈怒,發豎眥裂,罵曰:「汝輩罪不赦,乃敢雲赦吾耶!」脅以刃,不屈。其弟元樑奔救,手刃數賊,賊縛枚及元樑至王倫屯,偪令跪,仍不屈。賊先斷枚兩足,又斷兩手,旋支解元樑,弟兄同時死。

堂邑訓導吳躭,福山人。年七十餘矣,攜侄文秀及僕王忠到官。賊劫學署,見其老,置不問,躭叱之,詞甚厲。賊怒,殺躭及文秀與忠。陽穀縣丞劉希燾、典史方光祀、壽張營游擊幹福、調守陽穀莘縣汛把總楊兆立、堂邑汛把總楊兆相等,亦先後被害。

溫模,字孫朗,福建長樂人。入貲為吏目,發甘肅,借補通渭縣典史。乾隆四十九年,鹽茶逆回田五倡新教作亂,聚石峰堡,遂犯通渭。模以回民馬世雄預告,知賊計,為之備。知縣王慺恇怯不任事,模乃與縣署幕客邵如椿、縣紳李南暉同時城守。模率兵民登陴禦賊,凡七晝夜,士皆用命。糧盡,請開倉給守者,慺持不可。城將陷,馳返官廨,正衣冠北向拜,鍵戶自經死。世雄戰死。

如椿,浙江紹興人。父以申韓術游陜西,因占咸寧籍,補諸生。如椿就慺聘,事急,乃立城闉,袒而大呼曰:「好男子!當從我守城殺賊。」應者數千人。令壯者執刀矛,老弱運甓石,並集城上,而身率猶子曾燮登西墉,以當賊沖。城庳薄,賊蟻附上,手短刀格鬥。良久,力不支,被執。賊方肆戮,猶大言曰:「首議守城者,我也!何多殺他人為?」凡被十三創,曾燮被十一創,均罵不絕口死。

南暉,由舉人於乾隆三十年任四川威遠縣知縣,以疾告歸。先於逆回蘇十三肆擾通渭,有守御功。至是又率子思沆、猶子師沆召募壯夫百五十人助城守,累擲大石殺賊。城陷巷戰,與子思沆同罵賊死。師沆自經死。安定縣典史費元燈,亦以奉檄偵賊被害。

湯大奎,字緯堂,江蘇武進人。乾隆二十八年進士,授福建鳳山縣知縣。五十一年冬,臺灣賊民林爽文作亂,起彰化,其黨曾伯達等應之,南竄鳳山。縣故無城,僅土垣三尺許。時大奎已秩滿候代,屬賊勢蔓延,乃率僚佐募鄉勇,日夜守御。賊來攻,與參將瑚圖裏擊卻之。瑚圖里馳馬逐賊去,大奎聞城北有警,捕內應四賊,斬以徇。方獎勵兵役,賊突進北門,入縣治,典史史謙死之。大奎朝服坐事,手劍擊賊,賊刃交下,猶瞋目詈不止。長子荀業從之官,先以父詩文稿畀其戚,令遠避,身佩刀蔽父不去,同遇害。大奎初喪其元,城復後,有僕識大奎系發線,形容亦約略可辨,因並入棺。孫二,貽汾自有傳。

謙,字昭和,順天宛平人。先遣子善戰奉大母出避,乃與大奎同城守。死後亦喪其元,為百姓竊埋之,賊退始改殮。

周大綸,字理甫,直隸天津人。乾隆二十年,由貢生捐職州同,發福建,補臺灣府彰化縣丞。數年,知民頑,憂形於色,屢言於上官,斥不信。任滿,將引見,假公事滯諸諸羅。亂作,大綸奮入縣治,縣令懦,甘以身殉。大綸曰:「國家建官,命能守,不命能死。坐致民逆,死以塞責,小丈夫也。」激之,弗應,為謀所以禦賊計。夜,賊入,據縣治,有見大綸者,縛去不殺,而勸之降。大綸大罵之,賊摑其頰,撫頰大哭曰:「此污乃為賊污!」首觸柱,額裂。囚數日,卒遇害。大綸僕陳德以護主不去,大綸死,以頭椿賊,支解死。

壽同春,名星,以字行,浙江諸暨人。習文法,客臺灣淡水同知程俊署,年七十餘矣。竹塹城陷,俊先以出捕賊遇害。俊子攜印走,同春為賊執,佯為所用,賊留其黨三十六人守城,而自出掠。同春客淡水久,胥徒皆熟習,士民皆信服,潛為糾合甚眾,出不意,就同知事駢斬留賊,即日閉城門,為朝廷守。賊聞大駴,悉眾返攻,同春部勒其眾,日夜登陴。樵蘇既斷,發屋掘鼠為食,得間,輒出選鋒擊賊。相持數日,賊稍引卻。道通,署同知徐夢麟始以印至,次第招撫附近脅從者,夢麟一切倚同春辦治。是時,首逆負嵎,據大里杙自固,官軍環營其外,疑莫敢入。同春草書與夢麟,令上軍門,速攻之。久乃得報,合六路進剿。同春率官軍從西路入,而鹿港之兵,遷延失期。既入,無援,馬蹶,被獲。賊恨同春久,至是喜得報,攢刃支解之。

又廣東嘉應州人李喬基者,名安善,以字行。善少林拳術。客臺灣,見土豪嘯聚相仇殺,嘆曰:「亂將作矣!」乃簡僑寓南北莊人團練之。亂作,郡城大震。召諸健兒曰:「賊眾一閧而出,遂破彰化、淡水、諸羅三城,所不即取郡城者,懼粵人躡其後耳。吾出兵牽制之,賊至則守,去則擊,相持久,則援師且至,賊不足平矣。」集萬餘人,莊為柵,里為臺,計畝以為糧。一莊有賊,諸臺應之。賊數至,皆不得逞。十二月,率三千人從知縣張貞生復彰化,已而糧盡,士卒多散去,城復陷。明年正月,復從總兵柴大紀復諸羅。自起義兵與賊二十餘戰,斬馘萬計,賊銜之,以萬金購喬基首。二月,喬基與從子舉柏率健兒數百人赴鹿港請火藥,為賊所偵。還至青堈,伏發,御之,殺數百人。賊大至,矢石交下,突圍出,失舉柏。喬基三入賊中,傷左股,被獲,諸健兒皆戰死。賊誘喬基降,罵賊,賊斷其舌,縛而射之,猶不屈,乃磔焉。至是白衣冠哭者萬餘人,皆誓不與爽賊俱生也。是役也,死事之烈,以喬基為最。

熊恩紱,字隆輔,廣西永康州人。乾隆十七年進士。父疾,意不在試,以訛脫列下等,歸本班選用,選授直隸永安縣知縣。累遷永平府知府。四十三年,高宗東巡,召對稱旨,擢霸昌道,改大順廣兵備道。為政務持大體,尤慎刑罰,時語人曰:「慮囚,但久跪索供,感寒濕即病足,或發他疾,皆足致死,豈獨三木能斃人也?」

始單縣有劉某者,習八卦教,煽惑鄉里,官捕而殺之,械其子於獄。人復就獄中傳其術,從者益眾。自山東、河北、直隸境無慮數萬人。而段文經故胥吏,以事斥革家居,性險詐,屢挾數以役人,群服其黠,奉以為帥。立期劫單縣獄,圖攻奪州郡。恩紱聞之,下元城令密捕所在匪黨,而郡縣吏皆通賊,多為耳目者。走白賊云:「將屠滅汝等。」賊駴且恚,突於五十一年閏七月十四日夜半毀道署,入,殺恩紱。恩紱聞讙聲,疑失火,旋知有變,亟還。令人守庫,舉印授妻繆氏,挺身出,大罵。賊攢刃斫之。

賊固與其黨有成約,以先期起事,不及應。戕恩紱後,即散劫郡縣署,皆以有備不獲逞,故鄰境得以次擒獲。恩紱被害,尸面如生,兩手猶作搏賊狀。家人以守庫被殺者六人,印以繆氏匿之,得無失。恩紱逆折賊謀,不至如三省教匪蔓延不已,躬犯大難,論者多之。

宋如椿,漢軍鑲紅旗人。以寶慶通判權乾州同知。乾隆六十年正月,黔、楚苗石柳鄧、石三保等叛,苗響應,居民爭避竄。如椿召諭之曰:「若屬先人丘壟皆在,不可棄。同知地方官,當為若效死守。」皆許諾。已而賊勢張甚,棄去者大半。如椿被發徒跣,周走號呼,勸之守,自旦至夕,不絕聲,訖不聽。賊旋攻西門,如椿仗劍出御,傷左腿,歸,北向再拜自刎。從人張忠在側,固遣之,弗去,亦被創死。方賊攻急,如椿度不能支,呼巡檢江瑤佩印,令赴辰州求援。瑤出城,遽遇賊,死。其子朝棟挈印送辰,歸,覓父尸,與家屬俱遇害。

趙福,湖南零陵人。由行伍隨征金川,有功,累擢至鎮筸中營守備。逆苗滋事,駐守淥溪口,淥溪為鎮筸糧道,約士卒嚴,民安之。五月,官軍從狗腦巖潰歸,賊眾近萬人,謀絕糧道,攻之急。時守兵先抽調其半,民請福避去,福曰:「兵衛民,將統兵,爾輩可去,吾奉命守淥溪,去一步,即失職。苗至,福怒馬奮槊當先拒之,殺數十人。苗分番更戰,民以福不得食,為納橐饘,福揮去之。且曰:「賊之不遽追戮者,以我在也。我死,合力追汝,無焦類矣!」民泣涕去。麾下五十人,感福義,無一逃者。戰一晝夜,溪橋被撤,卒死且盡,手過山槍三發,斃苗數十人,指掌焦爛,不能持,身被數創,投溪死,民隔溪望者,咸痛哭。苗旋散去,難民數千賴之全活,後架數椽祀之,曰趙將軍廟。

劉升,邵陽人。寶慶協把總,從副將某征苗,副將逗留不前,升於眾中出謾語,某銜之。師至狗爬崖,令率百人為前鋒,約舉白旗為後援。升策馬轢陣,賊不能支,偵無後繼,復悉銳搏戰。升連舉白旗,旗失,復解所服白衵招之,某故按兵不發,升戰死,百人殉焉。死極慘,首體糜粉,無可收瘞者。後祀昭忠祠,主入時,旋風暴起,吹氣作血腥,襲眾幾僕。時以鄉團死最烈者,有滕家瓚。

家瓚,湖南麻陽人。諸生。有膂力,能負鐵砲擊賊。捐布政司理問職銜,居高村,與乾州苗接壤。乾隆六十年,逆苗掠麻陽,家瓚同兄監生家瑞、弟武生家瑤,悉出家財鉅萬,設卡堵御,有功。自正月至四月,共打仗十八次,殺賊八十餘名,賊恨之。總督福康安寵異家瓚,家瓚為畫破賊策甚備。一日,家瓚率眾守溪口,賊驟圍其居,曰:「出家瓚,禍可已。」族弟武生家泰挺身出,語其村人曰:「豈可惜一身而害一村?」遂大罵賊,自承為家瓚。賊剝家泰皮,至死不更一辭。又執其家口,始知非家瓚也,全家被害。家瓚聞而馳救,無及,請官兵援助,官軍忌其能,不助一卒,且檄調鄉兵他去。家瓚復往溪口,與眾共守,賊急攻之,力鬥死。

蕭水清,字廣銓,廣東平遠人。以監生納捐,發湖北,補保康縣典史。嘉慶元年二月,白蓮教謀反,姚之富、齊王氏起襄陽,曹海揚、祁中耀起房竹,王蘭、曾世興起保康,眾各數萬。齊王氏掠州郡,與王蘭會保康之白溪溝,賊黨楊昭為內應,水清計擒之,徇於眾,賊銜之。時守城兵以剿苗他調,縣令畏賊他往,城中空虛。水清給印札曉諭四鄉,激以忠義。賊遽至,縣城故庳薄,水清拒守,殺賊過當,歷五日夜不懈。遣勇健詣郡乞援,為賊得,圍益急。水清知不可為,旋署,語其妻曰:「吾義不屈,爾其自為計!」妻誓先殉,子其馨等及家人皆原從死。遂出,城已陷,遇賊縣治前,罵賊,死焉。教官黃義峰、吳珍義,子其馨、其芳,族子祚超,妻弟林鳳良同殉之。妻林氏、子婦韓及孫女與僕婦、婢女等,皆闔戶自刎。水清死後,鄉勇始集,皆頭插小青箬為識,以別賊,從援軍擒賊首王蘭、曾世興。小青箬者,即水清印札之號令也。

賊旋犯竹山。竹山縣知縣劉大成,江西新昌人。乾隆四十六年進士,選授蒞任。縣界萬山中,故有專營駐防,亦以剿苗他調,留者僅百名。大成先捕得賊黨,有「約期搶據竹山」語,即飛牘告急,且與僚屬謀,曰:「吾守具未完,為賊乘,必困。不如出據險要,相機堵御。」方派撥間,賊已據保康。乃以典史吳國華、守備孫掄魁分守縣治及隘口,而自守武陽堡,當其沖。納縣印於懷,據險設伏,遴健足偵探,終夜無少休。賊突越後嶺,入縣焚掠,國華、掄魁俱不支,先後至武陽。大成復率以赴剿,槍斃十數。賊來益眾,遂退往武陽。國華、掄魁方出點兵,大成乃遣親信出探隘口。比反,大成已肅衣冠佩印北向自縊矣。國華、掄魁踵至,愕然,亦殉焉。別股賊犯襄陽呂堰驛,巡檢王翼孫亦以拒戰死。

翼孫,江蘇長洲人。呂堰當驛道之沖,無城可守。翼孫聞變,募鄉勇戒備,而賊已大至。翼孫率眾迎擊,殲先鋒三人,遂登大橋御之。賊來益眾,鄉兵潰,又手刃數賊。賊矛環刺,受傷重,跳而投於水。賊以鉤起之,攢刃毀其尸。翼孫初至任,預立禦賊章程,一鄉勇十,設頭目一,頭目十,設總頭目一,各相鈐制,統於巡檢司。附近村落,單丁獨戶,皆遷於鎮。選壯者充鄉勇。設哨探,定功過,儲糧秣,練刀仗,禁飲博,其區畫為甚備云。

王行儉,江蘇溧陽人。由舉人大挑知縣,發陜西,補南鄭縣。以承審命案不實,褫職。嘉慶元年,投效軍營,二年,教匪竄汝河,以平利縣防守嚴,向東南偪白土路營。時行儉帶兵六百名,偕都司趙禧御之,賊分股前後夾攻,禧中刀傷歿。行儉罵賊不撓,身被矛傷十餘處,陣亡。以離任文員,帶兵協剿,罵賊捐軀,詔深恤之。

王銑,字麗可,江蘇武進人。以四庫館謄錄勞,授華陰縣丞。性介,不合上官。先調守山陽豐陽砦,糾義勇八百餘人,皆鋒銳可用。銑被豐陽知縣檄入城共守御,義勇以所將非人,被殲。銑為建祠山陽南關,勒石志名姓,哭之。三年,調至洵陽佐理撫恤事。縣令圖與銑分吞賑款,嚴斥之。縣令恚,圖中傷銑,以行臺省需餉,急薦銑。行至雒南廟溝坡,坡高二里,銑已北下坡,家人甫押後隊逾坡脊,賊高均德大隊至坡南。探騎二,縱轡馳上。家人大呼,速銑下馬避賊,銑不應。探騎至坡脊,馳下夾銑去,幾一里,復馳回,一騎以矛剔銑面,一騎就刺胸及肋,皆洞穿而死。同以運餉死者,四川省有汪兆鼎。

兆鼎,字子元,武進人。亦以四庫館勞,授直隸棗陽縣丞,以事褫職。赴四川軍營投效,未用。四年,同郡硃向隆為達州巡檢,有解餉之役,邀兆鼎偕。至東鄉縣太平石岸遇賊,向隆逃,眾謂兆鼎非蜀官,盍亟避,兆鼎弗應。乃各奔,兆鼎獨守餉,罵賊被害。

左觀瀾,字繡川,江西永新人。由舉人大挑知縣,發陜西,權五郎通判。五郎扼要川、陜,無城。觀瀾蒞任,既募鄉勇訓練,即牒大府,捐廉雇役,築土城,躬自督之。半月工竣,三日而教匪至,悉精銳啟城追剿,斬獲甚眾。數日,賊突出別道,薄城,眾寡不敵,請援又不至,觀瀾乃召子承廕等勵之,皆泣對曰:「原從死。」即分兵乘城,夜多燃炬束,老弱大呼譟。賊不知虛實,引去。將軍德楞泰、明亮至,詢狀驚嘆,遣守備率兵駐城中,聽觀瀾節制,城守益堅,民樂為用。

以勞補安定縣,西安府啟巡撫留之,巡撫悟,立止毋去任,而賊果悉眾至。見觀瀾立城堙,咸錯愕。觀瀾諭賊降,次日二百餘人至,觀瀾納之;守備欲殲以要功,觀瀾不聽。乃庭集降者曰:「汝等欲終從賊,即聽去。」降者稽首謝不敢。以後至六人,不可信,令降者自別之,果於里衣得賊黨所以為識者,即斬之,投六首城外,賊駭遁去。

三年,賊復大至,觀瀾舁大砲城上為御,手發砲斃賊無算,觀瀾亦以砲裂傷砲,負痛,解佩刀付承廕,舁歸署,亟遣人間道請代,乃卒。後二百,援兵至,承廕泣叩軍門,原復仇,總督那彥成哀而壯之,俾隨官軍剿賊。四年十月,躡賊沙溝口,力戰陣亡,猶手父佩刀不可拔。父子俱歿王事,賜恤尤厚。觀瀾事繼母以孝稱。兄觀海,官上思州知州。時有兄弟爭財者,適得思州書,念弟甚,引蘇軾「世世為兄弟」句,觀瀾讀而泣下,付訟者兄弟令閱。訟者感悔,泣謝去。

董寧川,直隸永寧人。由武舉選授貴州鎮遠鎮標守備,隨剿苗匪。嘉慶元年,累擢至湖北興國營參將。三年,隨總兵諸神保等軍剿教匪,賞健勇巴圖魯名號。復隨副都統額勒登保進剿終報山,偕都司張廷楷等自西入,奮勇奪山隘。官軍魚貫上,並力攻擊,擒首逆覃加耀等。股匪劉成棟、張漢潮、張添倫分擾巫山、荊門及撲鬧楊坪邊隘,先後擊敗之。四年五月,股匪高均德竄雲路溝邊隘,偕游擊姚國棟合攻,賊奔梓桐埡,復偕都司劉應世由瓘坪迎截,殲二百餘名,餘匪潰。寧川見有騎馬二賊目,追益力。至樹林中,賊棄馬遁。寧川令弁兵圍山腰,自率弁兵數十,下馬追入深林,賊並隊轉鬥,寧川中矛傷,仍手刃十餘人,斃騎馬賊一人,力竭,歿於陣。事聞,詔曰:「董寧川下馬擊賊,至被戕害,似此忠勇之臣,不能承受國恩,為之墮淚!」命直隸總督胡季堂贍寧川母,命湖廣總督倭什布送寧川子及家屬歸原籍,皆出異數云。

韓嘉業,字健庵,甘肅武威人。父增壽,官涼標千總,隨征金川戰死。嘉業誓報父仇,入伍有功,累擢至陜甘督標游擊。

嘉慶元年,四川教匪滋事,陜西興安府屬地相接,奸民乘機蠢應,踞安嶺為巢穴,憑高恃險,立木城;又於高廟山設立大卡,形勢陡峻。嘉業奉檄率兵由羊毛子堰進克之。復會他將進偪安嶺,遣健卒潛燒木城,賊驚潰,乘勝取大卡,擒戮無遺,擢參將。四年,嬍敗李樹之股匪,追出班鳩嶺,賊竄六道河,嘉業循河右追賊至廟子壩,賊遁入川境。未幾,賊又由川界老林入南鄭,時嘉業循江防守,聞之,亟率兵前駐法慈院,堵其北竄。賊將就淺涉嘉陵江,而沔縣賊三四千人,由阜川偪近官莊,陜甘總督松筠令嘉業偕直隸守備麻允光擇要迎擊。賊全數出磚峒子,嘉業馳馬首先沖入,賊分兩翼繞馬家嶺自上壓下,四面合圍。嘉業力戰突擊,馬蹶,復箭殺執旗賊。賊以矛直刺,甘肅鎮標把總高騰蛟從旁格之,遂殺持矛賊,而群賊競進,嘉業中矛僕,遂死。騰蛟以身蔽其上,亦死,允光亦戰歿。事聞,優恤,謚武烈。後嘉業兄莊浪協副將自昌,亦陣亡盩厔,命共建一祠,賜名雙烈。

葉槐,字廕階,浙江錢塘人。父文麟,官陜西,權孝義同知。教匪躪秦中,槐聞警省父,即具牒軍門自效。嘉慶二年正月,奉檄率鄉勇剿賊於光頭山,賊旋由河南盧氏竄商州,與孝義接壤,隨父乘障搘拄,賊不敢入。別股賊復由漢中東竄,將由鎮安、五郎逼孝義,復佐父堵御。凡團練首領可用者,必傾身交接,以是豪健依附者甚眾。西鄉急,請援,槐選其鋒赴之。比至,賊即北竄。城固、洋縣有警,又率以往,賊遁入山。部分其眾,守通棧要路,而自逐賊,入虢川等處,陣斬賊,獲騾馬器械均無算。

賊東奔大峪口,孝義在重山中,無城郭,槐慮不能當,請援孝義,大府不許。槐不自安,拔營東追,果遇賊。會別部兵至,謂遇賊得捷,賊未必再犯孝義,阻其返。槐終慮孝義被困,復言於大府,謂「不發兵,即單騎行矣」。詞氣激昂,聞者色變。大府乃許撥鄉勇一千六百人隨槐行,抵孝義,賊果至。乃據險結營,令四山放號火,以張聲勢。西南賊尤勁,鄉勇人人思鬥,遂破賊前隊,斬其酋三人,賊稍卻。大隊來攻,復並力沖殺,賊無可乘,乃解去。

大府調其父權富平,槐亦入貲為縣丞,當就選,戀父不行,留大營司偵候事。會賊渡漢江,偪洋縣,醴泉縣知縣陸維基請行,舉槐為助,慨然偕往。維基帶勇練登手扳崖,至巔,遇賊,罵賊死。槐數突圍不得進,左旋至山梁,力竭,賊矛刺腰,大創死。僕四人皆從死。槐以衛父至,而卒死於兵,時皆壯之。維基,順天大興人。

毛大瀛,字海客,江蘇寶山人。少以能詩名,為「練川十二才子」之一。由附監生充四庫館謄錄,用州同,發陜西,累為河南巡撫畢沅、山東巡撫惠齡調用。大兵征廓爾喀,惠齡督四川,辦理濟嚨糧務,檄大瀛赴西藏差遣,事竣,留川補用。借補潼川府經歷,以軍功擢授中江縣知縣。嘉慶元年,檄赴湖北軍營隨剿教匪,復以軍功擢授四川簡州知州。時惠齡由湖北入川,沿路剿賊,大瀛從之。四年,回簡州任。五年,股匪張子聰竄潼河,擾三臺、中江地,官軍分路截剿,賊復分擾遂寧、樂至等處,由金堂之廣元寺,肆行焚掠,及簡州境。大瀛率鄉勇前往堵御,行抵土橋溝,馬步賊蜂至,力戰遇害。大瀛屢入督撫幕府,工箋奏,業此者二十年。山東巡撫國泰為在京舊交,國泰性暴戾,獨敬事大瀛。國泰被嚴譴,大瀛盡始終之誼,為時所稱。恤世職,孫岳生襲。岳生亦以詩文名一時。

張大鵬,陜西紫陽人,子楚常、希賢、紹堂,孫應朝、應邦、應選,皆諸生,餘皆布衣。家世以忠義為教。嘉慶元年,賊犯紫陽洞、汝二河,官軍未集,大鵬率子孫、出家財,募鄉勇八百餘人,助有司守御。賊掠龍形、響水二溝,楚常率眾進擊之,殺三人,遂前攻賊寨。山峻霧作,中傷歸。後三日,賊至大水溝觀音堂地,紹堂殺賊魁六十餘人。又三日,希賢與賊戰桃園,復殺三十一人。當賊之起,勢猛銳,官軍亦避其鋒。至是運見殺傷,大憤,遂率黨數千人至,希賢首出逆戰,中槍死。紹堂據險隘,復為賊殺。大鵬氣益奮,更率其孫應達、應祿、應愷、應試等持械深入,沖突躍呼,所殺傷甚眾。卒以眾寡不敵,皆戰死。初,張氏父子及孫凡十二人,自賊之興,戰死者七人,溺死者一人,傷者二人。陜西以鄉團死者,又有興平人白廷英。

廷英,縣舉鄉飲賓。嘉慶二年,教匪由蜀渡漢江而北,眾十餘萬,終南近山無完村,廷英督鄉人築村後張家寨避之。三年二月,賊自城東窺寨,寨人不二三百,賊急攻,槍矢雨下,丁壯悉潰。賊蟻附而登,廷英罵賊死,弟廷才、廷揚從死,賊俱焚之。次子筐廷英頭去,賊逐之,筐倒,頭落山下,後得於谷底,尸則焦爛不可辨矣。廷英年七十五,凡以守寨死者八十餘人。

是年,四川各鄉團之死難者,為廣元人楊繼曉,世居高城堡。繼曉姙十三月而生,既壯,以氣力伏一鄉。捐職州同知,隨父璽蘇州督糧同知任所。聞教匪擾蜀,歸省母。時巴州已破,繼曉與同縣貢生楊哻等倡議團練。罄家財,得千餘人,請縣令給劄為守御,縣官不省,散去。賊破南江,距縣境長池數十里,縣令始速繼曉出御。以烏合一散不易集,議先虛聲撓賊,作高城堡、人自相要約語,列名至多,書投賊營,賊果遲疑不遽進。會陜賊姚之富等數萬人穿老林出,將至德山,木門賊亦以數千人將至通坪。通坪居高城後,德山亙其肋,長池枕其前。繼曉謀於眾,攻長池者,縣官自御之,而自任後路。夕漏三下,與族人楊冕率眾出木門之橫江梁,遇賊先鋒,斗之。賊大隊至,不可敵,乃據險趣哻濟師。賊登山,了知兵少,無繼,合圍擊之。繼曉手刃數十人,力竭被執。至九曲坡,欲誘降之,大罵不屈,賊刳其腹而焚之,從戰者皆被戕。哻以三百人來援,至則皆歿,楊氏一門亦盡殲。

楊堂、梁崇、李培秀,皆廣東嘉應州人。堂官四川蒼溪縣典史,崇官陜西咸陽縣典史,培秀官陜西試用典史。嘉慶三年,王三槐擾蜀,大軍追剿急,亡命四竄。堂守永興場,士卒譁曰:「賊至矣!」皆欲走,堂手劍叱曰:「賊未至而棄糧,法當死,孰若守糧而死也!」賊至死之。三年,大軍駐鎮安剿張漢潮,崇率鄉勇剿鳳皇嘴賊,散,解囚回省。至孝義,遇賊。崇釋囚七人,曰:「若曹於法當死,然死於賊則枉,吾不忍也,可速去,毋從賊。餘義不可逃,死其所矣!」賊至,被執,不屈死。五年,培秀從大軍挽粟至四川大寧縣,與賊遇,盡委輜重於河,遣其僕曰:「速報大營,賊不得糧,必掠東郊,截而擊之,可盡覆也。吾死不及見矣!」大軍果破賊,訊俘,言培秀死時,賊不得糧,被二十一創云。初,崇所釋囚七人,皆歸獄,報崇死事狀,曰:「吾不負梁典史也!」至是,七人皆赦。

曾艾,字虎卿,湖南新化人。嘗割左臂療父疾。以例貢考授州同,發江西,署安福等縣。艾夙為嘉勇貝子福康安所知。辰州苗變,隨福康安軍令守麥地汛,從克諸砦有功。嘉慶元年,補貴州永豐州分防州同。州隸南籠,故苗地。州同駐冊亨,在萬山中,尤險遠難治。艾督各砦守本業,民、夷悉安。二年,遣人迎眷口,甫至,而南籠仲苗七柳須等遽叛。艾聞警,約駐防把總外委堅守,並諭四鄉亭目,招集良苗,繕城治械,令出肅然。賊至,部分守御,自出城奮擊,往來策應。城中婦女,亦改裝登埤。相持半月,援兵卒不至。賊眾數萬,圍益急,手發矢斃執旗賊魁。北門火起,率隊趨救,還賊城西隅,巷戰,中槍死。僕九人從死,兩妾聞訊皆自刎。次子為其戚攜出,號泣曰:「吾父母皆死,何以生為?」賊尾及之,亦中槍而殞。事聞,皆予恤。改南籠為興義府,永豐為貞豐州。

艾同族彰泗,字孔林。以拔貢生朝考用知縣,發陜西,授延川縣。嘉慶十年,權洋縣。時教匪被剿勢衰,以終南山為窟穴,搜捕不易。朝議改五通通判為同知,添設寧陜鎮總兵,募兵六千,改十大營鎮之,而以積年立功無業可歸之鄉勇充伍。以善後計,名曰「新兵」。新兵素難御,司儲者又誤扣米折,於是陳先倫、陳達順等於十一年二月作亂,戕官,連破營城十九處,偪洋縣,彰泗拒守七晝夜,援兵阻河不能至,城陷,彰泗死之。民保其眷屬潛出,故不及於難。

羅江泰,字靜波,浙江黃巖人。家貧,習賈。去賈投營,由外委歷擢游擊,皆在浙;由參將至副將在閩,總兵又在浙。前後與提督李長庚相左右,而在閩功特顯。長庚銳意剿海寇蔡牽,專意外洋,凡閫內事均以屬江泰。賊船高大,官軍仰攻失利,檄江泰造霆艇。艇成,陵賊船,賊大困,南走福建。江泰於白犬洋、四礵嶼、頭東各役俱有功,護海壇總兵。遂赴南洋,合金門總兵何定江截牽去路,橫擊於銅山,追至浮鷹洋。賊沖礁走,匿山上。江泰搜山,擒賊目王硃,又焚賊船於仰月橫山,賊皆墮水死。在閩逾年,凡十擊賊,號「敢死軍」。賊見江泰軍,輒引去。擢總兵,鎮金門。九年,移鎮定海。是時牽南竄臺灣,長庚正總閩、浙水軍,同心戮力,誓殺牽。十年九月,牽船泊道頭,忽遁去。江泰從甌洋會八總兵追之,至盡山,失牽所在。黑雲起海上,亟令移港,風驟至,白波山立,群舟相擊觸,頃刻破碎。江泰大船帆重不可下,下及尺,船遽不知所終。朝命沿海各省探訪,久之無得者,葬衣冠黃巖。

霍永清,字肇元,廣東南海人,居瀾石鄉。膂力絕人。嘉慶十四年,海氛未靖,大吏行封港策,海賊無所得食,相率蹂躪傍海各鄉,漸入內地,所過焚掠,怯懦者遂以款賊為得計。八月,賊聯數十艘由陳村、平洲、小圃直抵瀾石,眾議款之,永清曰:「彼恃舟楫為利,今深入重地,自取死耳。好男子從我殺賊,何為低首求免乎?」主款者陽受約,賊至,從壁上觀。永清獨率鄉勇堵御,相持一日夜,賊稍卻。明日,督勇再戰,而款賊者導賊從村後掩入,腹背受敵,力不支,中砲僕地,左右五人並死之。鄉人以永清以死勤事,建祠祀之,名祠曰愍義。

強克捷,陜西韓城人。嘉慶十三年進士,即用知縣,發河南,補滑縣。十八年九月,教匪李文成謀亂,期十五日與伏京城賊林清中外同起事。克捷初蒞滑,有退吏某方訟系,為白其誣出之。吏詗文成等逆謀,告克捷,歷申於守,不應。初六日,突執文成,嚴詰謀叛狀,笞斷其脛,及黨二十四人,鐍之獄。夜半,其黨牛亮臣突劫文成出,攻某吏,屠其家,踞城以叛,克捷及家屬俱死之。後文成焚死輝縣,林清伏誅京城,詔:「克捷首先訪獲逆黨,俾二逆失約敗謀,後先授首,實屬功在社稷。」優恤,謚忠烈,祀京師昭忠祠。於韓城、滑縣皆建專祠,與難者均予附祀。並以前大學士王傑同隸韓城,士風淳茂,永廣文武學額各五名。

在城者老岸鎮巡檢劉斌、教諭呂秉鈞、典史陳實勛同時預難;把總戚明彰以拒賊陣亡:均闔門殉節。逆黨赴硃村說降,諸生硃繼連不屈,率村人戰歿。滑縣變作,黨徐安國起長垣,知縣趙綸;又黨硃成貴起曹縣,知縣姚國旃;陷定陶,知縣賀德瀚:均死之。

綸,浙江錢塘人。國旃,安徽歙縣人。林清將為亂,金鄉縣令廉知其謀,即羽檄各縣,皆不之信。國旃以幕友吳星萃力陳利害,乃為緝捕計。以吏役多通賊,故賊攻縣治,急求星萃甘心,先國旃攢刺數十創死。

德瀚,長沙寧鄉人。事急,令家丁賚印赴府告變,幕友硃樹堂等皆死於難。在籍洙泗學院學錄孔毓俊等則率鄉勇助官剿賊,戰死奮義村。

林清果於九月十五日率逆黨持械闌入禁城,頭等侍衛那倫應值太和門,聞警趨入,有勸其緩行者,不聽,曰:「國家世臣,當此等事,敢不急趨所守耶?」至熙和門,門閉,賊蜂至,被戕。那倫者,前太傅明珠後也。

宗室奕湄,鑲藍旗人。由筆帖式累擢至內閣侍讀學士。道光四年,命以頭等侍衛為和闐辦事大臣。六年七月,回部逆裔張格爾入卡滋事,勾結喀什噶爾回眾為內應。帝以和闐附近,命加意嚴防。八月,賊分擾葉爾羌,命揚威將軍大學士長齡帶兵往剿,取道和闐,奕湄派綠營弁兵前往策應,諭奕湄:「隨時查探彼處實在情形,如葉爾羌現在被圍,當令迅速相機前進,仍嚴防後路,毋墮賊計。否則即留兵和闐防堵,以壯聲威。」旋以葉爾羌失守,賊四出滋擾,奕湄仍回和闐駐守。賊偪城下,援兵未至,城兵僅八十餘名,奕湄晝夜嚴防,力竭城陷,死之。幫辦大臣桂斌同與於難。

景興,李佳氏,滿洲鑲紅旗人,駐防伊犁。官佐領。嘉慶二十五年,喀什噶爾卡倫外布魯特滋事,伊犁將軍慶祥以景興熟悉回情,奏派馳往查看。經參贊大臣永芹奏留署協領事。道光六年六月,張格爾復率布魯特滋事,慶祥又令馳往偵訪,設法進剿。旋與七品伯克帕塔爾生擒奇比勒迪之子侄,縛解來城,伏誅。又探得張格爾與從前滋事汰劣克一處居住,即乘其未備,剿殺逆回百餘名,生擒楚滿一名。奏入,帝嘉之。是年八月,喀什噶爾城陷,與防禦佟善等皆力戰陣亡。喀什噶爾城圍攻兩月有餘,以城中回匪響應,穴地道而進,遂致不守。文員則七品小京官銜陳孝寬,以戍員派辦文案在城,與巡檢陳天錫、未入流陳德隆均死之。

王鼎銘,字新之,山東嶧縣人。由廩貢官中書,除湖南新田知縣。道光九年蒞任,先投城隍廟,誓於神。治事甚勤。夏旱,跪禱烈日中,有應,以是得民心。十二年正月,江華瑤匪趙金龍亂作,湖南提督海陵阿進剿。鼎銘慮煽邑瑤,即冒雪步歷瑤棚戒諭。復召瑤長,曉以國法。與教官率紳士練鄉勇以守。突聞海陵阿等被戕池塘墟,即督眾御賊。城外賊偪甚,將往諭賊,居民泣阻之。或報曰:「賊至!」城民驚竄,鼎銘朝服坐堂皇待之,書於幾曰:「仇我當殺我,勿傷我百姓。」指三尺練曰:「城亡,吾以此死。」以賊蹤尚遠,徐之。近縣寧遠、桂陽民感鼎銘之能死守,集萬人請帶剿,於是四路同進。賊分隊出,斃之無算。越日,桂陽之臨泰、大富等鄉復集二萬人,鼎銘身先策馬出城南,誓大創之。賊突以槍砲抵拒,死甚眾。先是賊密約邑瑤供送藥丸,瑤未肯負鼎銘,不與。賊乘夜脅取,故火器復烈。眾潰,鼎銘殿後,賊追至,大肆殺戮。鼎銘四顧慟曰:「奈何殺我百姓?」中砲落馬,剜兩目,身首異地。邑人得而攢之,越九十二日始改斂,面如生。

鼎銘殿後時,馬蹶,邑武生鄭奇光以所乘馬授之,鼎銘不可,強扶而上,鞭馬使疾馳。回身舞刀捍賊,受重創,死之。

呂志恆,江蘇陽湖人。由監生捐縣丞,發福建,累擢至臺灣府知府。道光十二年,嘉義縣賊匪張丙等糾眾滋事,焚掠各莊,志恆率署知縣邵用之分路剿捕,用之行至店仔口被戕。志恆復帶兵擊賊於大排竹,以眾寡不敵遇害。先是逆匪輒以貪官汙吏妄殺無辜為詞,帝疑有激變事,下福州將軍瑚松額等查奏無據,如例予恤。

方振聲,順天大興人。由供事選授福建巡檢,升嘉義縣斗六門縣丞。賊逼斗六門,振聲樹柵濬渠,率兵勇防堵。賊首黃城率匪黨攻撲,與署守備馬步衢等協力守御。賊夤夜縱火,蜂擁入柵,振聲持刀巷戰,戮數賊,力竭遇害。幕友沈志勇等同死之。妻女皆被戕甚慘。步衢與把總陳玉威亦同時陣歿。

楊延亮,字菊泉,湖南長沙人。嘉慶十六年,舉鄉試第一,成進士,用知縣,發山西。道光元年,補趙城縣。十五年,推升雲南南安州知州。時趙城有奸民曹順,以治病為名,傳習先天教,與其黨謀為不軌。斂錢造械,約八月分往平陽府、霍州、洪洞縣同時起事。三月,延亮尚未謝趙城任,偵得其狀,即飭兵役緝之。賊知謀洩,即糾黨潛入城,夤夜放獄囚,焚縣治,延亮死之,母妻子女及幕友楊成鼎同時遇害。事聞,詔用強克捷例予恤,特謚昭節。

師長治,字理齋,韓城人。由舉人捐內閣中書,改知縣,選浙江上虞。道光二十一年,再選湖北崇陽,蒞任甫百日而及於難。先是,縣胥役催徵錢漕,久為鄉民害。生員鍾人傑、金太和等起而包輸納,不數年皆驟富,與縣胥分黨角立。前令折錦元憒不治事,一惟胥役所為,致兩次閧漕。援巡撫伍長華批牘「漕石加徵一斗」語制扁送縣,毀差房。武昌知府明俊務調停姑息,於是奸民日肆。錦元旋劾罷,以金雲門權縣事,擒太和置武昌獄,勢少戢。

其年九月,長治至,人傑聞上游檄捕急,疑其仇生員蔡紹勛所譖,糾黨數百人篡取之。至則紹勛遁入城,躡追抵城,門閉,內外鼎沸。長治登城諭,不退,持竟夜,質明,人益眾。逾缺入,大索紹勛,不得,迫長治申狀,言紹勛作亂,人傑倡義捕反者,並請釋太和。時明俊以事至蒲圻,距崇陽一日程,長治先期遣長子懷印潛出,請明俊蒞縣鎮撫,而明俊急返武昌。眾益張,長治罵不屈,遂遇害。妾吳氏及侄女皆自經。家丁曹彬被殺。時十二月十二日也。

人傑以長治始至,無可歸罪,乃槥斂而哭祭之,言己以報仇倉卒,誤戕良吏,事不獲已,遂據城叛。脅眾逾萬,陷旁近數縣。明年正月,人傑等伏誅,恤世職。弟長鑣,官參將。於咸豐七年,援剿安徽,與賊戰婺源之橫槎,陣亡。

王光宇,字溥泉,興寧人。以未入流分湖北,歷權典史、巡檢事,治盜有聲,補崇陽典史。變作,衣冠自經死。


\end{pinyinscope}