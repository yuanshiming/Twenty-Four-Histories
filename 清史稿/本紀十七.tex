\article{本紀十七}

\begin{pinyinscope}
宣宗本紀一

宣宗效天符運立中體正至文聖武智勇仁慈儉勤孝敏寬定成皇帝,諱旻寧,仁宗次子。母孝淑睿皇后,乾隆四十七年八月初十日,生上於擷芳殿。幼好學,從編修秦承業、檢討萬承風先後受讀。又與禮部右侍郎汪廷珍、翰林侍讀學士徐頲朝夕講論。

乾隆五十六年八月,高宗行圍威遜格爾,上引弓獲鹿,高宗大喜,賜黃馬褂、花翎。嘉慶元年,娶孝穆成皇后。四年四月戊戌,仁宗遵建儲家法,親書上名,緘藏鐍匣。十三年正月,孝穆成皇后薨,繼娶孝慎成皇后。

十八年九月,從幸秋獮木蘭,上先還京師,而教匪林清黨犯闕之變作。是月,戊寅,賊入內右門,至養心殿南,欲北竄。上御槍斃二賊,餘賊潰散,亂始平。飛章上聞。仁宗欣慰,封上為智親王,號所御槍曰「威烈」。諭內閣曰:「忠孝兼備,豈容稍靳恩施。」上謙沖不自滿假,謝恩奏言:「事在倉猝,又無禦賊之人,勢不由己,事後愈思愈恐。」其不矜不伐如此。

二十五年秋七年,仁宗秋獮熱河,上隨扈。戊寅,仁宗不豫,己卯,大漸,御前大臣賽沖阿、索特納木多布齋,軍機大臣托津、戴均元、盧廕溥、文孚,總管內務府禧恩、和世泰公啟鐍匣,宣示嘉慶四年禦書,立上為皇太子。仁宗崩,即日奉大行皇帝梓宮回京。辛巳,尊母後為皇太后,晉封惇郡王釂愷為惇親王,釂愉為惠郡王。癸未,奉皇太后懿旨:「大行皇帝龍馭上賓,皇次子智親王仁孝聰睿,英武端醇,見隨行在,自當上膺付託,撫馭黎元。但恐倉卒之中,大行皇帝未及明諭,而皇次子秉性謙沖,予所深知。為降諭旨,傳諭留京王大臣,馳寄皇次子,即正尊位。」上奉懿旨,恭摺覆奏,並將御前大臣等啟鐍匣所藏嘉慶四年四月立皇太子硃諭進呈。召在籍翰林院侍講秦承業來京。

八月乙酉,命遵古制行三年之喪,臣民仍照定例持服。免直隸承德府屬及經過宛平等五州縣明年額賦。癸巳,允王大臣請,持服百日。乙未,大行皇帝梓宮還京師。御史袁銑疏陳定規模、正好惡七事。上優詔嘉納之。加方受疇太子太保。戊申,大學士、九卿等奏上大行皇帝廟號尊謚曰仁宗受天興運敷化綏猷崇文經武孝恭勤儉端敏英哲睿皇帝。頒大行皇帝遺詔於朝鮮、琉球、暹羅、越南、緬甸諸國。庚戌,上即皇帝位於太和殿,告祭天地、太廟、社稷,頒詔天下,以明年為道光元年。加恩中外,非常赦不原者,咸赦除之。加黃鉞、劉鐶之、賽沖阿、孫玉庭、蔣攸金舌太子少保。辛亥,停本年秋決。是月,賑河南許州地震災。貸盛京彰武臺邊門等處被淹兵丁一年錢糧,並給修屋費。貸巨流河等處一月口糧。

九月己未,尊大行皇帝陵曰昌陵。庚申,切責軍機大臣,以擬遺詔錯誤,罷托津、戴均元軍機大臣,文孚、盧廕溥仍留軍機大臣,均下部嚴議。斌靜奏沖巴噶什愛曼布魯特比蘇蘭奇糾薩木薩克之子張格爾作亂。命慶祥兼程赴喀什噶爾剿之。命大學士曹振鏞,尚書黃鉞、英和在軍機大臣上行走。壬戌,以那彥成為理籓院尚書。命吏部尚書、協辦大學士吳璥督理河南儀封河工。調劉鐶之為吏部尚書,茹棻為兵部尚書,盧廕溥為工部尚書,黃鉞為戶部尚書,汪廷珍為禮部尚書,顧德慶為左都御史。起松筠為左副都御史。戊辰,以秦承業為翰林院侍講學士,命在上書房行走。庚午,上始御西廠幄次,引見廷臣。詔開鄉會試恩科。命臣工切實言事。丁丑,豫親王裕興以罪奪爵圈禁。壬午,加提督楊遇春太子少保,賞雙眼花翎。是月,賑河南睢州等七州縣水災,並給睢州等四州縣一月口糧。

冬十月戊子,調英和為戶部尚書,那彥成為吏部尚書,穆克登額為工部尚書,普恭為禮部尚書,和世泰為理籓院尚書,松筠為左都御史。辛丑,上大行皇帝尊謚廟號。翌日,頒詔天下,覃恩有差。甲辰,賑江蘇被水江寧等八州縣、安徽被水鳳陽等府所屬州縣。戊申,以德英阿為烏魯木齊都統。是月,賑江南海州、安徽泗州等八州縣及屯衛水旱災。給浙江蕭山等三十三縣貧民口糧。

十一月丙辰,上奉皇太后居壽康宮。戊辰,以魏元煜為江蘇巡撫,左輔為湖南巡撫。庚午,冬至,祀天於圜丘。自是每歲如之。癸酉,以誠安為左都御史,松筠為熱河都統。甲戌,誠安改鑲黃旗漢軍都統。以文孚為左都御史。丁丑,翰林院侍講學士顧蓴奏松筠宜置左右,忤旨,下部嚴議。

十二月甲申,上皇太后徽號曰恭慈皇太后。翌日,頒詔天下,覃恩有差。諭奉皇太后懿旨,立皇帝繼妃佟佳氏為皇后。丙戌,和世泰改福州將軍。以晉昌為理籓院尚書。調慶保為閩浙總督。以史致光為雲貴總督,韓克均為雲南巡撫,顏檢為福建巡撫。庚寅,河南儀封決口合龍。癸巳,加上孝敬憲皇后、孝聖憲皇后、高宗純皇帝、孝賢純皇后、孝儀純皇后尊謚。英和罷軍機大臣,照舊供尚書等職。丙申,以汪廷珍、湯金釗、方受疇、蔣攸金舌言查陋規不便予議敘,孫玉庭奏尤為剴切,溫諭褒之。起李鴻賓為安徽巡撫。召張映漢來京,以陳若霖為湖廣總督,帥承瀛為浙江巡撫。

是歲,朝鮮、琉球來貢。

道光元年春正月癸丑,御太和殿受朝,樂設而不作,不讀賀表。丙辰,賞刑部員外郎初彭齡禮部侍郎銜。裁浙江鹽政,以巡撫兼管。己未,以文孚為禮部尚書,那清安為左都御史。丁卯,越南進香,表賀,貢方物,詔止之。丙子,朝鮮國王李鍚奉表慰唁;廓爾喀王熱尊達爾畢噶爾瑪薩野奏仁宗升遐成服,貢金緞,賜敕嘉賚之。

二月壬午朔,日食。班禪額爾德尼進貢物,賜敕褒嘉賚之。戊戌,協辦大學士吳璥予告。庚子,命孫玉庭為協辦大學士,仍留兩江總督。加陜甘總督長齡太子少保。甲辰,免江西豐城等六縣民借籽種口糧逋穀。

三月辛亥朔,欽天監奏,本年四月初一日,日月合璧,五星聯珠。詔:「益勵寅恭,與內外臣工共圖上理,不必宣付史館。」壬子,以送仁宗睿皇帝梓宮至山陵,命莊親王綿課等留京辦事。癸丑,再免經過地方本年旗租,並給麥田籽種。辛酉,仁宗睿皇帝發引,上奉皇太後送至昌陵。壬戌,廓爾喀進登極表貢,命與道光二年例貢同進。丙寅,上謁泰陵、泰東陵、昌陵、隆恩殿,上孝淑睿皇后尊謚曰孝淑端和仁莊慈懿光天佑聖睿皇后。丁卯,命成都將軍呢瑪善赴雲南幫辦軍務。癸酉,葬仁宗睿皇帝於昌陵。加托津、曹振鏞太子太傅。丁丑,上奉皇太后還京師。戊寅,仁宗睿皇帝、孝淑睿皇后升祔太廟。己卯,以升祔禮成,頒詔天下,覃恩有差。命貴州提督羅思舉赴雲南軍營協剿。是月,貸山西岢嵐等十州縣、甘肅狄道等五州縣上年災民倉穀口糧。

夏四月丙戌,常雩,祀天於圜丘,仁宗睿皇帝配享,自是歲以為常。庚寅,授呢瑪善為欽差大臣,督辦雲南永北軍務。授那清安左都御史。大學士、三等侯明亮致仕。命戴均元、穆克登額、阿克當阿相度萬年吉地。甲辰,雲南大姚拉古賊平。丁未,上詣大高殿祈雨。戊午,撥江蘇海州等州縣賑銀四十五萬六千兩。命伯麟為大學士管兵部。以長齡為協辦大學士,仍留陜甘總督任。癸亥,詔停本年秋決。甲子,授伯麟體仁閣大學士,曹振鏞武英殿大學士。丙寅,封阮福皎為越南國王。以松筠為兵部尚書,慶惠為熱河都統。壬申,夏至,祭地於方澤,仁宗睿皇帝配享,自是歲以為常。癸酉,雲南永北大姚賊平。

六月辛巳,以張師誠為廣東巡撫。甲申,安定門災。庚寅,上御太和門,命鄭親王烏爾恭阿、順承郡王倫柱齎冊寶詣孝穆皇后殯宮行冊謚禮。戊戌,召成齡來京,以李鴻賓為漕運總督,孫爾準為安徽巡撫。除河南新鄉縣地賦。以琦善為山東巡撫。

秋七月庚戌,刑部尚書和瑛卒,調那彥成為刑部尚書,松筠為吏部尚書,晉昌為兵部尚書。以穆克登布為理籓院尚書。己未,嚴烺以三品頂戴署河東河道總督。丁卯,調毓岱為江西巡撫。以楊懋恬為湖北巡撫。庚午,上奉皇太后謁西陵,免經過地方額賦十分之三。壬申,上奉皇太后還京師。是月,賑甘肅寧夏等四縣水旱災,並免上年額賦。

八月庚辰,展順天鄉試於九月舉行。丁亥,命松筠在軍機大臣上行走。以特依順保為烏里雅蘇臺將軍。癸巳,兵部尚書茹棻卒,以初彭齡代之。乙未,霍罕遣使請入覲,卻之。丙午,調張師誠為安徽巡撫,孫爾準為廣東巡撫。

九月戊辰,暹羅國王鄭佛遣使進香、貢方物,溫諭止之。己巳,召長齡來京,以硃勛署陜甘總督。是月,賑安徽宿州等三州縣水災。

冬十月己卯,上禦乾清門聽政,自是歲以為常。丁亥,調孫爾準為安徽巡撫,嵩孚為廣東巡撫。

十一月己未,貴州巡撫陳若霖奏請歲減民、苗佃租二萬二千石,給苗疆會試舉人川費,允之。壬戌,以河防功加黎世序太子太保銜。

十二月戊子,以邱樹棠為山西巡撫。癸巳,吏部尚書劉鐶之卒,調盧廕溥為吏部尚書,免軍機大臣。調初彭齡為工部尚書。以戴聯奎為兵部尚書。

是歲,朝鮮、越南、琉球來貢。

二年春正月丁未朔,方受疇病免,以顏檢為直隸總督,長齡署之。以葉世倬為福建巡撫。辛酉,祈穀於上帝,仁宗睿皇帝配享,自是每歲如之。庚午,召特依順保來京,調奕顥為烏里雅蘇臺將軍,松筠為黑龍江將軍。以晉昌為盛京將軍,那清安署兵部尚書。辛未,以三載考績,予曹振鏞等議敘,罷侍郎那彥寶、善慶、吳芳培,降左都御史顧德慶。以王鼎為左都御史。命長齡回陜甘總督。以松筠署直隸總督,那彥成署吏部尚書。

二月丁亥,以謁陵命莊親王釂課等留京辦事。癸巳,兵部尚書戴聯奎卒,以王宗誠代之。

三月丙午,撥江蘇上元等二十州縣賑銀五十四萬兩。丁未,上謁東陵,免經過地方額賦十分之三。庚戌,上謁昭西陵、孝東陵、景陵、裕陵,詣端慧皇太子園寢奠酒。調穆克登額為禮部尚書,文孚為工部尚書。癸丑,上還京師。甲寅,上奉皇太后謁西陵,免經過地方額賦十分之三。乙卯,以裕陵工程不慎,降莊親王釂課為郡王,解戴均元太子太保及管刑部,褫蘇楞額職,令在工次聽差,仍分成賠繳有差。戊午,上謁泰陵、泰東陵、昌陵。己未,清明節,上詣昌陵行敷土禮。壬戌,上詣孝穆皇后殯宮前奠酒。奉皇太后還京師。

閏三月戊寅,穆克登布免理籓院尚書。乙酉,以禧恩為理籓院尚書。庚子,賜戴蘭芬等二百二十二人進士及第出身有差。是月,蠲緩奉天寧遠等三州額賦。

夏四月辛未,上孝敬憲皇后、孝聖憲皇后、高宗純皇帝、孝賢純皇后、孝儀純皇后尊謚,藏冊寶於太廟、盛京太廟,並藏仁宗睿皇帝、孝淑睿皇后冊寶於盛京太廟。壬午,青海番賊平。以阿霖為江西巡撫。乙酉,以倉場侍郎莫晉奏事妄言,硃批駁斥,降內閣學士。是月,蠲緩河南睢州等十六州縣沙壓、堤占、水占地賦,直隸滄州等五州縣並嚴鎮、海豐二場被水賦課。

六月癸丑,大學士伯麟原品休致。命戴均元仍管刑部。己未,命那彥成署陜西巡撫。調嵩孚為貴州巡撫。以羅含章為廣東巡撫。以那清安署刑部尚書。壬戌,褫松筠吏部尚書、軍機大臣,命以六部員外郎候補。戊辰,命長齡為大學士兼管理籓院。以英和協辦大學士。調文孚為吏部尚書,禧恩為工部尚書。以那清安為兵部尚書,玉麟為左都御史。己巳,以富俊為理籓院尚書,松箖為吉林將軍,德英阿為黑龍江將軍,英惠為烏魯木齊都統。是月,賑山西興縣水災。

秋七月,以程祖洛為河南巡撫,王鼎署之。以程國仁為陜西巡撫。是月,賑直隸霸州等二十一州縣水災。

八月癸卯,召雲貴總督史致光來京,以明山代之。河南新蔡縣教匪硃麻子作亂,命程祖洛捕誅之。戊申,召慶保來京,以趙慎畛為閩浙總督,盧坤為廣西巡撫。庚戌,以盧坤署陜西巡撫。戊辰,賞廓爾喀國王寶石頂戴,噶箕畢穆興塔巴三品頂戴。辛未,召長齡、松廷來京,以那彥成署陜甘總督。是月,給河南安陽等三縣,直隸霸州等十二州縣,山西歸化城、薩拉齊二,山東恩縣等三縣水災口糧。貸土默特被水蒙古口糧。蠲緩山東高唐等四十一州縣衛,雲南鶴慶、劍川二州災歉賦課。

九月壬申朔,允暹羅進本年例貢。甲戌,撥通倉米十萬石賑直隸被水災民。乙酉,四川果洛克番賊平。授嚴烺河東河道總督。庚寅,以蔣攸金舌署刑部尚書。調陳若霖為四川總督,李鴻賓為湖廣總督。以魏元煜為漕運總督,韓文綺為江蘇巡撫。庚子,調盧坤為陜西巡撫。以成格為廣西巡撫。是月,給江西瑞昌縣,河南武陟、原武二縣災民口糧。

冬十月丙午謁陵,命莊親王釂課等留京辦事。授那彥成陜甘總督,蔣攸金舌刑部尚書。乙卯,上以釋服奉皇太后謁西陵,免經過地方額賦十分之三。己未,上謁泰陵、泰東陵、昌陵。庚申,上謁昌陵行釋服禮。癸亥,上奉皇太后還京師。是月,賑甘肅河州、安徽宿州、直隸霸州等四十三州縣,江蘇海州、湖北天門二縣水災。貸給盛京廣寧縣,山東濮州等五州縣,黑龍江城庫木爾等二站水災口糧。蠲緩甘肅靜寧等六州縣、河南儀封等二十三縣、湖北沔陽等十三州縣、山東濮州等五十一州縣衛、直隸通州等十八州縣、江蘇海州等三十四州縣衛被水災新舊額賦,墨爾根、布特哈舊欠糧石,長蘆及江蘇松江府屬正溢鹽課。

十一月辛未朔,以玉麟署禮部尚書。癸未,撫恤廣東省城火災貧民蛋戶。乙酉,以玉麟為禮部尚書,慶保為左都御史。丙戌,立繼妃佟佳氏為皇后。翌日,頒詔天下。覃恩有差。戊子,起松筠為光祿寺少卿。壬辰,上詣大高殿祈雪。丁酉,以冊立皇后禮成,上皇太后徽號曰恭慈康豫皇太后。翌日,頒詔天下,覃恩有差。是月,賑安徽宿州等七州縣及屯坐各衛、河南武陟縣水災旱災,給安徽泗州等八州縣、甘肅河州等十一州縣災民口糧。蠲緩安徽宿州等十七州縣及屯坐各衛,河南武陟、陽武二縣,甘肅狄道等六州縣,江西南昌等七縣並南昌、九江二衛,湖南澧州、浙江海寧等四州縣被災新舊額賦,長蘆被水引地、兩淮板浦等九場被水新舊額賦。

十二月丙午,上詣大高殿祈雪。癸丑,上以祈雪未應,命再禱七日。熱河都統成德卒,以慶保代之。賞松筠二品頂戴為左都御史。調程含章為山東巡撫。以陳中孚為廣東巡撫。甲寅,河南虞城縣匪徒盧照常等作亂,捕誅之。庚申,免江蘇、安徽嘉慶二十三年以前民欠攤徵銀。調德英阿為綏遠城將軍,祿成為黑龍江將軍。乙丑,內閣漢票簽處火。是月,給直隸大城縣水災口糧。貸直隸駐扎災區兵丁餉銀。蠲緩直隸隆平等三縣、江蘇山陽等四縣水災旱災額賦。

是歲,朝鮮、暹羅、琉球來貢。

三年春正月壬申,御重華宮,宴群臣及內廷翰林。調孫爾準為福建巡撫。以陶澍為安徽巡撫。以廓爾喀額爾德尼王遣噶箕達納彭咱邦禮等來賀登極進表貢,賜詔嘉勉,仍優賚之。壬午,幸圓明園。乙未,命大學士長齡在軍機大臣上行走。以史致光為左都御史。是月,賑奉天小黑山白旗堡旗戶、直隸霸州等三十六州縣、江蘇海州水災。給江蘇邳州等八州縣衛水災、安徽宿州等十二州縣衛水災旱災、河南武陟縣水災、山東濮州等六州縣災民一月口糧。貸浙江海鹽、長興二縣旱災,陜西留壩等十一州縣雹災水災,甘肅靜寧等十七州縣地震災,兩淮板浦等九場水災,河南武陟等三縣,黑龍江齊齊哈爾、墨爾根城旗丁水災籽種糧石。

二月辛丑朔,命以原任大學士阿桂配饗太廟。調嵩孚為湖南巡撫。以程國仁為貴州巡撫。丁未,釋奠先師孔子。辛亥,以原任尚書湯斌從祀文廟。癸丑,上詣文廟釋奠,臨闢雍講學,加禮部尚書汪廷珍太子太保銜。是月,加給直隸大城縣口糧。

三月壬申,上御勤政殿聽政。乙亥,上親耕耤田,加一推。丙子,上奉皇太后幸南苑。上行圍。辛巳,上奉皇太后還宮。甲午,上奉皇太后閱健銳營兵。戊戌,調程含章為江西巡撫,以琦善署山東巡撫。是月,加給直隸文安縣災民一月口糧。

夏四月甲辰,召顏檢來京,以蔣攸金舌為直隸總督。調那清安為刑部尚書,玉麟為兵部尚書。以戶部左侍郎穆克登額為禮部尚書。癸亥,上禱雨于覺生寺。甲子,賜林召棠等二百四十六人進士及第出身有差。

五月辛未,賑直隸霸州等州縣災。是月,賑直隸霸州等三十六州縣災民。

六月,命署工部侍郎張文浩會同蔣攸金舌查勘南北運河並永定、大清、滹沱各河。戊午,以果勒豐額為烏里雅蘇臺將軍。永定河決。壬戌,北運河決。是月,加給直隸靜海、青縣二縣災民兩月口糧。貸河南汝陽、正陽二縣倉穀。

秋七月戊辰,以陸以莊為左都御史。己巳,以直隸霸州等十州縣被淹較重,飭撥銀米先行撫恤。飭琦善撲蝗。壬午,以江蘇水災,免各關商米稅銀。免河南應攤川楚及衛案軍需四百六十萬兩。是月,給江西德化縣、湖北黃梅縣、江蘇太倉等十七州縣水災一月口糧。加賑直隸通州等二十一州縣水災。

八月己亥,初舉經筵。乙卯,以浙江杭州等三府屬水災,免海運商米船稅,並留各關稅銀備賑。是月,賑安徽無為等十六州縣水災。給河南濬縣等十三縣水災一月口糧。

九月壬申,以謁陵命托津、英和、盧廕溥、汪廷珍留京辦事。丁丑,永定河決口合龍。壬午,上奉皇太后謁西陵。丙戌,謁泰陵、泰東陵、昌陵。丁亥,免直隸通州二十七州縣水災額賦。己丑,奉皇太后還京師。壬辰,以松筠為吉林將軍,穆彰阿為左都御史。是月,賑直隸通州等四十州縣、山東臨清等五州縣水災。加賑江西德化縣、湖北黃梅縣、河南武陟等五縣水災。給江蘇儀徵等四縣、湖北江陵等三縣水災口糧。蠲緩山東臨清等十六州縣衛、直隸薊州五十州縣水災新舊額賦,河南武陟縣、湖北黃梅縣水災額賦及屯坐各衛應徵新舊額賦,並給修屋費。

冬十月,賑湖北江陵等三縣衛水災,並免新舊額賦,給修屋費。貸奉天錦州旗民、山東武城縣水災一月口糧,直隸天津鎮三營及紫荊關各汛被水兵丁銀米。乙亥,以毓岱為廣西巡撫。是月,貸甘肅靜寧等十六州縣災民口糧。蠲緩湖南澧州等五州縣等水災,甘肅宜禾縣旱災新舊額賦。癸丑,以緝盜功,加陜西陜安道嚴如煜按察使銜。是月,貸江蘇蘇州等五府駐扎災區兵丁銀米。

是歲,朝鮮、琉球、暹羅、緬甸來貢。

四年春正月壬申,命停今歲木蘭秋獮。癸酉,享太廟,命皇子奕緯代行禮。癸未,撥戶部銀八萬兩貸直隸貧民口糧。是月,賑直隸通州等三十八州縣上年雹災,河南武陟縣、濬縣旱災各一月。給江蘇太倉等三十州縣衛,安徽無為等十七州縣衛,浙江海寧等十二州縣、橫浦等四場,兩淮安豐等九場水災,山東臨清等五州縣雹災一月口糧。貸河南武陟等十二縣上年水災籽種口糧倉穀,江西德化等十四縣、湖北黃梅等三縣及各屯衛、湖南澧州等四州縣、甘肅秦州等十州縣、齊齊哈爾等三城被災軍民籽種口糧,江蘇泰興營兵丁兩月錢糧。

二月丁酉,召松筠為都察院左都御史。以富俊為吉林將軍,穆彰阿為理籓院尚書、軍機大臣。江南河道總督黎世序卒,以張文浩代之。己亥,御經筵。甲寅,上奉皇太后幸南苑。丁巳,上行圍。己未,上奉皇太后還宮。是月,給江蘇銅山縣災民一月口糧。調毓岱為江西巡撫,以康紹鏞為廣西巡撫。丁亥,上閱健銳營兵。初彭齡罷,以陳若霖為工部尚書。

夏四月壬戌,貸湖北武昌府屬道士洑營、荊州城守等營兵丁倉穀,江南徐州鎮標中營等駐扎災區兩月錢糧。

五月己巳,上詣黑龍潭祈雨。甲戌,雨。增致祭堂子禮。戊寅,增皇太后萬壽告祭太廟後殿禮。

六月癸巳朔,日食。乙巳,以張師誠為山西巡撫。甲寅,暹羅國王鄭佛卒。

秋七月丙子,韓崶免,以陳若霖為刑部尚書,陸以莊為工部尚書,姚文田為左都御史。辛巳,大學士戴均元致仕。是月,貸湖北衛昌、德安二營兵丁倉穀。

閏七月辛丑,江蘇巡撫韓文綺降調,調張師誠為江蘇巡撫,以硃桂楨為山西巡撫。壬寅,以韓克均兼署云貴總督。丁未,命孫玉庭為大學士,以蔣攸金舌為協辦大學士,均仍留總督任。成都將軍呢瑪善卒。以奕顥為綏遠城將軍。辛亥,以福釂為山西巡撫。乙卯,免安徽無為等三十一州縣上年水災旱災額賦。是月,貸江南二營銀米。

八月壬戌,命江蘇按察使林則徐濬浙江水道。己巳,御試翰林、詹事等官,擢硃方增五員一等,餘升黜有差。戊寅,御經筵。庚辰,以蘇明阿為貴州巡撫。丙戌,予告大學士伯麟卒。丁亥,以成格為江西巡撫。是月,蠲緩長蘆興國等七場、滄州等七州縣上年水災灶課,甘肅宜禾縣早災額賦。

九月壬寅,以黃鳴傑為浙江巡撫。癸卯,免安徽無為等十一州縣被災學田租銀。是月,給陜西寧羌等四州縣災民口糧。貸江蘇瓜州營被災兵丁銀米,陜西安定等縣水災雹災倉穀。

冬十月乙丑,回酋張格爾入烏魯克卡倫,官軍失利,侍衛花山布等陣亡。丙子,巴彥巴圖等率兵剿張格爾,敗之。張格爾奔喀拉提錦。甲申,予告大學士章煦卒。以孫玉庭奏開王營減水壩,命相機速辦。

十一月己酉,以高堰十三堡決口,張文浩交部嚴議。辛亥,命文孚、汪廷珍往江南查看高堰決口。調嚴烺為江南河道總督。以張井署河東河道總督。甲寅,孫玉庭坐徇隱張文浩,免兩江總督,以魏元煜署。命兵部尚書玉麟在軍機大臣上行走。是月,給安徽宿州、靈壁縣及屯坐各衛災民口糧。貸江寧八旗、兩江督標協標兵丁餉銀,甘肅靜寧等十三州縣及東樂縣丞所屬災民口糧。

十二月己未朔,上復詣大高殿祈雪。戊辰,授魏元煜兩江總督,以顏檢為漕運總督。己卯,召明山來京,以長齡為雲貴總督。高堰決口合龍。以慶保為烏里雅蘇臺將軍,那清安為熱河都統,明山為刑部尚書,穆彰阿署。是月,給雲南太和等三縣災民、景東屬鹽井灶戶一月口糧及修屋費,江蘇高郵等五州縣災民並清河災民一月口糧。

是歲,朝鮮、琉球來貢。

五年春正月,授戴三錫四川總督。辛亥,以三載考績,予托津、長齡、曹振鏞、黃鉞、英和、汪廷珍、蔣攸金舌、那彥成、嚴烺議敘,加琦善總督銜。是月,給江蘇高郵等四州縣,安徽天長縣、泗州衛上年水災旱災軍民口糧。貸直隸文安、大城二縣,河南汝陽、淮寧二縣,陜甘寧羌等七州縣,甘肅狄道等四十州縣及肅州州同、莊浪等縣丞所屬水災旱災雹災籽種口糧,兩淮中正場水災灶戶口糧,雲南景東被水鹽井修費,並免上年額課。

二月庚申,御經筵。甲子,以謁陵命莊親王、托津、盧廕溥、汪廷珍留京辦事。戊寅,上奉皇太后謁陵,免經過地方額賦十分之三。上謁昭西陵、孝陵、孝東陵、景陵、裕陵,至寶華峪閱視萬年吉地,回鑾。甲申,幸南苑行圍。是月,給安徽天長縣災民一月口糧。

三月戊子朔,上還京師。以琦善為山東巡撫。甲辰,以程含章為浙江巡撫。壬子,王鼎以一品銜署戶部左侍郎。丙辰,免河南積年民欠並河工加價攤銀。是月,貸直隸寶坻、靜海二縣,甘肅洮州十七州縣及莊浪縣丞所屬災隸籽種口糧,齊齊哈爾被災旗人耕牛銀。

夏四月乙丑,免直隸積年逋賦。辛未,以伊里布為陜西巡撫。是月,貸駐扎歉區山西寧武等二營,湖北安陸等三營,荊州水師營、提標後營兵丁倉穀。

五月甲午,太監馬進喜以在滸墅關偽稱奉旨進香,交刑部治罪。諭各督撫,凡遇通緝太監,當認真緝捕。有偽稱奉差者,迅即奏辦。丁酉,黃鉞以年老免軍機大臣,專辦部務,仍直南書房。命王鼎在軍機大臣上行走。調張師誠為安徽巡撫,陶澍為江蘇巡撫。戊申,孫玉庭、顏檢罷,調魏元煜為漕運總督,以琦善為兩江總督。調伊里布為山東巡撫,以鄂山為陜西巡撫。甲寅,以本年漕運遲誤,諭切責孫玉庭等。玉庭交部嚴議,魏元煜、顏檢議處。是月,賑貴州鎮遠府屬州縣水災,並免額賦,貸兵丁餉銀,給城衙修費。貸湖北荊州駐守等四營駐扎災區兵丁倉穀。

六月,命蔣攸金舌為大學士,仍留直隸總督任。以禮部尚書汪廷珍協辦大學士。丁卯,降魏元煜三品頂戴,仍留漕運總督任。孫玉庭、顏檢均交琦善督令挑濬運河,工費命玉庭、檢、元煜分償。甲戌,魏元煜卒,以理籓院尚書穆彰阿署漕運總督,前江寧將軍普恭署理籓院尚書。乙酉,以陶澍奏,停江南折漕,仍議河海並運。是月,貸福建提標五營、泉州城守營穀價。

秋七月丁未,以德英阿為烏里雅蘇臺將軍,和世泰為察哈爾都統。是月,減免直隸等七州縣積水地額賦。

八月,以嵩孚為刑部尚書,調康紹鏞為湖南巡撫,以蘇成額為廣西巡撫。己未,御經筵。以陳中孚為漕運總督,調成格為廣東巡撫,以武隆阿為江西巡撫。

九月乙酉,召那彥成,以鄂山署陜甘總督。調長齡為陜甘總督,趙慎畛為雲貴總督,以孫爾準為閩浙總督。調韓克均為福建巡撫,以伊里布署云南巡撫。調武隆阿為山東巡撫,韓文綺為江蘇巡撫,以嵩溥為貴州巡撫。庚子,以張井為河東河道總督。甲辰,以德英阿署伊犁將軍,松筠署烏里雅蘇臺將軍,普恭署左都御史。喀什噶爾幫辦大臣巴彥巴圖等率兵剿張格爾,妄殺布魯特部人。其酋汰列克糾眾圍巴彥巴圖等於喀什噶爾,慶祥使穆克登布等援之。命慶祥緩來京。是月,賑陜西綏德等四州縣雹災。蠲直隸開州等十五州縣旱災雹災新舊額賦。

冬十月庚辰,以長齡署伊犁將軍,楊遇春署陜甘總督,鄂山回陜西巡撫。命德英阿赴烏里雅蘇臺。召松筠來京。辛巳,召蔣攸金舌,以那彥成為直隸總督。是月,賑陜西榆林等三縣雹災。

十一月壬辰,以暹羅國貢船漂沒,詔免其補貢,封世子鄭福為暹羅國王。庚子,免托津管刑部,以蔣攸金舌代之,並命為軍機大臣。乙巳,上詣大高殿祈雪。丙午,除直隸昌黎縣捍禦灤河地額賦。丁未,雪。命慶祥以將軍銜署喀什噶爾參贊大臣。壬子,以慶祥為喀什噶爾參贊大臣兼鑲黃旗漢軍都統,未任前,以穆克登布署之。授長齡伊犁將軍。是月,賑甘肅岷州等六州縣水災雹災。

十二月己巳,免山東章丘、鄒平二縣被水逋賦。戊寅,命科爾沁郡王僧格林沁御前行走。是月,賑奉天錦州府旱災蟲災。

是歲,朝鮮、琉球、暹羅、越南入貢。

六年春正月甲申,以雙城堡屯田,加富俊太子太保。是月,賑奉天錦州、中前所等處旗戶水災。給江蘇沛縣災民口糧。貸奉天寧遠州旗民,河南鄢陵等七縣,甘肅岷州等十二州縣,山西襄垣縣,直隸寶坻等三縣水災旱災雹災籽種口糧倉穀。

二月戊午,以謁陵命托津、英和、汪廷珍、盧廕溥留京辦事。甲戌,上謁西陵,免經過地方額賦十分之三。戊寅,謁泰陵、泰東陵、昌陵。辛巳,上還圓明園。

三月癸巳,調張井為江南河道總督。庚戌,賞潘錫恩三品頂戴,為南河副總河。是月,貸山西靈丘縣、湖北荊州等五營被災兵丁倉穀。

夏四月甲子,上詣黑龍潭神祠祈雨。甲戌,以鄧廷楨為安徽巡撫。丙子,賜硃昌頤等二百六十五人進士及第出身有差。是月,給江蘇沛縣災民口糧。貸江蘇徐州鎮三營,湖北德安、宜都二營災區兵丁錢穀。

五月乙未,禮部尚書穆克登額免,以松筠代之。以那清安為左都御史。以明山為熱河都統。戊戌,雲貴總督趙慎畛卒,調阮元代之。以嵩孚為湖廣總督,明山為刑部尚書,慶惠為熱河都統。壬寅。免直隸河間等五縣積水地畝逋賦。是月,給山東堂邑等十二縣旱災口糧。貸直隸廣平等五縣、山東堂邑等十二縣、河南臨漳等十二縣、山西隰州營口糧籽種倉穀。

六月,賑湖北江陵、當陽二縣水災。給河南臨漳等七縣旱災口糧。貸直隸大名鎮標等七營被旱兵餉。

秋七月癸巳,張格爾糾安集延、布魯特回眾入卡。喀什噶爾回眾響應之。命楊遇春為欽差大臣剿之,鄂山署陜甘總督。命武隆阿為欽差大臣赴臺灣。己亥,以德英阿為伊犁參贊大臣,倫布多爾濟署烏里雅蘇臺將軍,庚子,張格爾陷和闐城,領隊大臣奕湄、幫辦大臣桂斌等死之。甲辰,命長齡為揚威將軍,以武隆阿為欽差大臣,與楊遇春參贊軍務。乙巳,以德英阿署伊犁將軍。是月,賑江蘇高郵等六州縣水災。給湖南醴陵等三州縣、山西歸化城水災口糧。貸陜西西鄉、盩厔二縣水災籽種,奉天錦州府屬各驛馬乾銀。

八月,回酋巴布頂等陷英吉沙爾。甲戌,張格爾陷喀什噶爾城,參贊大臣慶祥、幫辦大臣舒爾哈善等死之。進陷葉爾羌,辦事大臣音登額、幫辦大臣多隆武等死之。是月,賑江蘇海州等五州縣水災。給薩拉齊水災口糧。貸山西綏遠城渾津黑河水災口糧。

九月己卯朔,黃鉞免,以王鼎為戶部尚書。辛巳,幸南苑。命固原提督楊芳、甘肅提督齊慎赴阿克蘇軍營。丁亥,還圓明園。戊子,以博啟圖為察哈爾都統。辛卯,召穆彰阿來京,以楊懋恬署漕運總督。乙未,以長清為阿克蘇辦事大臣。己亥,慶廉奏敗賊於阿察他克臺。辛丑,免阿克蘇附近回莊本年應交麥石。癸卯,調格布舍為烏里雅蘇臺將軍。是月,給貴州松桃,山西歸化,江蘇山陽、鹽城二縣,江西蓮花等七縣水災口糧銀穀。

冬十月庚申,贈喀什噶爾死事參贊大臣慶祥太子太保。壬戌,免兩淮富安等十四場水災灶課。甲子,撥江蘇籓關道庫銀一百四十五萬兩賑高郵等二十州縣水災。是月,給安徽宿州等八州縣衛被災口糧。蠲緩江蘇高郵等四十七州縣衛災民新舊額賦。

十一月戊子,長齡等奏敗賊於阿克蘇之柯爾坪。己丑,以臺灣平,加孫爾準太子少保。是月,賑湖南茶陵等三州縣災民。貸甘肅秦州等十三州縣災民口糧。蠲緩盛京牛莊等處水災糧租,湖南茶陵等五州縣水災新舊額賦。

十二月戊申朔,以楊健為湖北巡撫。以訥爾經額為漕運總督。丙辰,四子部扎薩克親王伊什楚克魯布以僭妄削爵。戊午,調英和為理籓院尚書,禧恩為戶部尚書,穆彰阿為工部尚書。

是歲,琉球、朝鮮入貢。

七年春正月丁酉,和闐回眾降,命優賚之。尋復為張格爾所陷。庚子,以惠顯為駐藏辦事大臣。是月,展賑江蘇高郵等二十三州縣衛軍民、兩淮丁谿等九場灶戶水災。給安徽泗州、五河縣及屯坐各衛,奉天白旗堡、小黑山二處災歉口糧。貸直隸開州等十州縣、甘肅秦州等十七州縣、河南原武等四縣、兩淮富安等五場、江西蓮花等五縣災歉口糧籽種,河南修武、封丘二縣,山西薩拉齊災民倉穀,江蘇川沙等三營、青村等八營銀米。

二月甲戌,上詣黑龍潭祈雨。是月,貸江蘇狼山等三營毗連災區兵餉。

三月己丑,賑江蘇高郵等州縣水災。丙申,長齡等奏敗賊於洋阿爾巴特。晉長齡太子太保。丁酉,上詣黑龍潭祈雨。己亥,長齡等敗賊於沙布都爾,獲安集延回目色提巴爾第。命蔣攸金舌、穆彰阿查勘南河。以那清安署工部尚書。癸卯,以惠顯為駐藏大臣。甲辰,雨。是月,賑江蘇高郵等州縣災民。貸甘肅張掖等三縣、直隸開州等六州縣貧民口糧。

夏四月丙午朔,日食。戊申,長齡等奏敗賊於阿克瓦巴特。予長齡紫韁,加楊遇春太子太傅,武隆阿太子少保。壬子,長齡等克喀什噶爾,張格爾遁。辛酉,進克英吉沙爾。以張格爾未獲,褫長齡紫韁、楊遇春太子太傅、武隆阿太子少保。

五月庚辰,楊芳克和闐,獲回目噶爾勒等,誅之。壬午,陸以莊免,以王引之為工部尚書。癸未,琦善、張井、潘錫恩嚴議。琦善免兩江總督,以蔣攸金舌代之。以托津管刑部。丁亥,命穆彰阿在軍機大臣上學習行走。

閏五月乙巳朔,免回疆八城新舊額賦。丙午,命楊遇春回,以楊芳為參贊大臣。戊申,調奕顥為盛京將軍,晉昌為綏遠城將軍。是月,貸湖北黃州協道士洑營兵丁穀石。

六月壬午,上詣黑龍潭祈雨。丙戌,雨。

秋七月壬子,協辦大學士、禮部尚書汪廷珍卒。晉昌免正黃旗領侍衛內大臣,以鄭親王烏爾恭阿代之。丙辰,以姚文田為禮部尚書,湯金釗為左都御史。丁巳,命盧廕溥協辦大學士。己未,英和以失察家丁,褫協辦大學士、理籓院尚書、紫韁。召富俊為理籓院尚書、協辦大學士。以博啟圖為吉林將軍。以安福為察哈爾都統。辛酉,熱河都統升寅免,以那清安代之。癸亥,那清安仍為左都御史。英和褫太子太保,降二品頂戴,為熱河都統。乙丑,以武隆阿為喀什噶爾參贊大臣。以盧坤為山東巡撫。戊辰,免甘肅兵差過境之各州縣額賦,協濟軍需之甘肅、陜西各州縣額賦十分之六。庚午,論喀什噶爾等四城收復功,復楊遇春太子太保,加鄂山、盧坤太子少保。壬申,以再定回疆,晉曹振鏞太子太師,蔣攸金舌、文孚太子太保,加王鼎、玉麟太子少保。是月,給奉天錦州等三府州縣水災旗民口糧。

八月癸未,萬壽節,停筵宴。丙申,調盧坤為山西巡撫,以琦善為山東巡撫。是月,賑陜西略陽縣,湖北江陵、監利二縣水災。給江淮等處被災幫丁月糧。蠲緩江蘇高郵等四十七州縣衛被水新舊額賦。

九月癸丑,以孝穆皇后梓宮移寶華峪,命皇長子奕緯祖奠。丙辰,上詣孝穆皇后梓宮前奠酒。授伊里布雲南巡撫。戊午,免兵差過境之陜西華州等二十二州縣額賦十分之六。庚申,上謁東陵,免經過地方本年額賦十分之五。癸亥,謁昭西陵、孝陵、孝東陵、景陵、裕陵。召長齡,以德英阿為伊犁將軍。晉戴均元太子太師。是日,回鑾。庚午,以楊國楨為河南巡撫。免兵差過境之盛京省城及所屬開原等十四處旗民額賦十分之四。是月,加賑陜西略陽縣水災。

冬十月庚辰,免嘉慶二十五年至道光五年各省民欠正雜錢糧。壬午,皇太后萬壽聖節,奉懿旨停筵宴。丙戌,禮部尚書姚文田卒,以湯金釗代之。以潘世恩為都察院左都御史。庚寅,巴繃阿免,以額勒津為科布多參贊大臣。丁酉,以綸布多爾濟為庫倫蒙古辦事大臣。是月,賑湖北江陵、監利二縣及屯坐各衛水災。給奉天廣寧縣被水站丁口糧。貸山西定襄、潞城二縣旱災雹災倉穀,黑龍江墨爾根城歉收口糧。

十一月乙巳,命長齡督同楊芳辦理回疆善後事宜。丙午,召那彥成。庚戌,授那彥成欽差大臣,會同長齡籌辦回疆善後事宜。以屠之申署直隸總督。己巳,免奉天遼陽等七州縣地丁銀十分之四。是月,賑甘肅岷州等六州縣水災雹災。

十二月,以彥德為烏里雅蘇臺將軍。

是歲,朝鮮、琉球、暹羅入貢。

八年春正月丙午,以松筠署熱河都統,那清安署禮部尚書。戊申,授劉彬士浙江巡撫。壬戌,長齡奏獲張格爾。癸亥,封長齡威勇公,授御前大臣。封楊芳果勇侯。調果齊斯歡為綏遠城將軍。乙丑,晉曹振鏞太傅,文孚太子太傅,玉麟太子太保。加穆彰阿太子少保,並充軍機大臣。授楊遇春陜甘總督。丙寅,晉將攸金舌太子太傅。復英和太子太保。命那彥成仍以欽差大臣赴喀什噶爾,偕楊芳辦善後。丁卯,加禧恩太子少保。是月,給江蘇沛縣貧民口糧。貸直隸滄州等九州縣災歉口糧,湖北江陵、監利二縣及屯坐各衛籽種,山西定襄等四縣倉穀,江蘇江寧、京口駐防修屋費。

二月乙亥,群臣以再定回疆,上尊號,卻之,命議上皇太后徽號。都察院左都御史史致光卒。

三月庚子朔,日食。乙巳,上行圍,至丁巳皆如之。戊申,上還宮。是月,貸直隸開州等六州縣貧民口糧。

夏四月,調果齊斯歡為黑龍江將軍,以特依順保為綏遠城將軍。是月,貸山西代州等二十四州縣歉收、湖北駐兵災區、荊州水師各營倉穀。

五月己酉,以獲張格爾,遣官告祭太廟、社稷,行獻俘禮。庚戌,御午門受俘。晉長齡太保。加楊芳太子太保。壬子,上廷訊張格爾罪,磔於市。丁巳,命圖平定回疆四十功臣及軍機大臣曹振鏞、文孚、王鼎、玉麟像於紫光閣。是月,貸湖北駐扎歉區黃州協兵丁倉穀。

六月癸酉,揚威將軍、大學士長齡凱旋,命鄭親王烏爾恭阿等迎勞。丙子,命長齡管理籓院。

秋七月甲辰,朝鮮國王李鍚以回疆平定,遣使表賀進方物。丙午,以升寅為熱河都統,以那清安署禮部尚書。

八月丁丑,萬壽節,停止筵宴。己卯,以成格為熱河都統。調盧坤為廣東巡撫。以徐炘為山西巡撫。甲申,命奕紹、托津、富俊、陳若霖留京辦事。是月,給浙江淳安等四縣水災口糧。貸長蘆被淹灶戶工本。蠲緩浙江淳安等四縣新舊額賦。

九月戊戌朔,日食。丙午,上謁東陵,免經過地方額賦十分之三。丁未,以寶華峪工程不慎,褫英和職,降戴均元三品頂戴。己酉,謁昭西陵、孝陵、孝東陵、景陵、裕陵,並祭孝穆皇后殯宮。褫戴均元職。庚戌,謁裕陵,行大饗禮。辛亥,下英和於獄,籍其家。癸丑,上還圓明園。甲寅,上謁西陵,免經過地方額賦十分之三。丁巳,謁泰陵、泰東陵、昌陵。戊午,謁昌陵,行大饗禮。庚申,逮戴均元下獄,籍其家。辛酉,上還圓明園。調特依順保為黑龍江將軍。以那彥寶為綏遠城將軍,達凌阿為塔爾巴哈臺參贊大臣。是月,賑兩淮海州屬中正等三場灶戶水災。貸回疆西四城兵丁茶價銀。

冬十月庚午,英和遣戍黑龍江。甲午,復惇郡王綿愷為惇親王。是月,賑江蘇海州等三州縣衛、浙江建德等五縣水災。給江蘇高郵等九州縣、安徽泗州等二十六縣水災旱災一月口糧。貸奉天廣寧等處水災旗民口糧,浙江富陽縣貧民穀石,齊齊合爾等處旗營官莊銀糧。蠲緩江蘇海州等三十六州縣衛、安徽泗州等二十六州縣、浙江海寧等十三州縣旱災水災新舊額賦。

十一月甲辰,上皇太后徽號曰恭慈安豫康成皇太后。乙巳,以加上皇太后徽號禮成,頒詔天下,覃恩有差。己未,釋戴均元。是月,賑浙江富陽縣水災。給盛京寧古塔等處水災口糧。

十二月辛巳,那彥成奏招降附霍罕之額提格訥部落。諭嘉之,令妥為撫馭。

是歲,琉球、朝鮮入貢。

九年春正月丁未,希皮察克愛曼布魯特阿仔和卓來降。壬子,楊芳加太子太傅。是月,給安徽泗州等五州縣並屯坐衛、江蘇海州等十五州縣衛災民口糧。賑兩淮板浦等三場被災貧丁。貸山西代州、解州水災籽種,河南上蔡縣水災倉穀。

二月己巳,御經筵。庚午,上奉皇太后幸圓明園。霍罕西南達爾瓦斯部落遣使內附,諭嘉獎卻之。甲午,命吉林將軍博啟圖為御前大臣,以瑚松額代之。

三月丙午,上幸南苑。丁未,上行圍,至辛亥皆如之。辛亥,西藏徼外拉達克部長呈進奏表。壬子,上還圓明園。甲寅,上御閱武樓閱京營兵。戊午,召琦善,以訥爾經額為山東巡撫,硃桂楨為漕運總督。

夏四月癸酉,召戴三錫,以琦善為四川總督。壬午,屠之申以讞獄錯誤降,松筠署直隸總督。丙戌,奉皇太后御含輝樓閱皇子及侍衛等騎射。戊子,賜李振鈞等二百二十一人進士及第出身有差。是月,貸湖南乾州等五州縣上年旱災口糧籽種、山西朔州等二十三州縣歉收倉穀。

五月丁酉,移孝穆皇后梓宮於寶華峪正殿,神牌於東配殿。是月,貸湖北荊州城守、水師二營及宜都營被水倉穀。

六月乙丑,以福釂為科布多參贊大臣。己巳,免西藏喀喇烏蘇等處雪災番族貢馬銀,並撫恤達木八旗被災官兵戶口。甲戌,伊犁將軍德英阿卒,以玉麟代之。調松筠為兵部尚書。以博啟圖為禮部尚書。丁丑,召安福,以福克精額署察哈爾都統。是月,貸三姓地方上年被水倉穀。

七月己亥,申嚴粵海關官銀出洋、私貨入口禁。以扎隆阿為喀什噶爾參贊大臣。丁巳,越南國王以母老乞葠芪,上嘉賚之。是月,賑廣西雒容、永福二縣水災。免安徽泗州五河縣,鳳陽、泗州二衛上年被水錢糧十分之一。

八月癸亥,以謁盛京祖陵,命奕紹、托津、湯金釗、明山留京辦事。庚辰,上奉皇太后謁盛京祖陵。

九月壬辰朔,日食。免蹕路經過之承德等五州縣本年額賦,及幫辦差務之岫巖等九州縣額賦十分之五。壬寅,朝鮮貢使李相璜等迎覲。乙巳,上親射,並閱盛京官兵等騎射。丁未,上謁永陵。戊申,行大饗禮。閱興京城。己酉,博啟圖降調,以耆英為禮部尚書。上謁福陵,臨奠弘毅公額亦都墓,加恩後裔博克順等。癸丑,行大饗禮。上至盛京,詣太廟寶冊前行禮。乙卯,上詣天壇、堂子。奉皇太后幸嘉廕堂。臨奠克勤郡王岳託墓。朝鮮國王李鍚遣使貢方物。戊午,詣地壇。臨奠直義公費英東墓。己未,上御大政殿,賜扈從王、公、大臣,蒙古王、貝勒、貝子、公及盛京文武官員宴賞有差。

十月,以潘世恩署禮部尚書。辛未,皇太后聖壽節,上率扈從王、公、大臣詣皇太后行宮行慶賀禮。上奉皇太后幸澄海樓。壬午,謁裕陵。甲申,以吳光悅為江西巡撫。乙酉,上奉皇太后還宮。是月,給安徽泗州等五州縣★一月口糧。

十一月丁巳,召英惠,調成格為烏魯木齊都統。以裕恩為熱河都統。是月,賑奉天遼陽等五處被災旗民口糧。

十二月甲子,緬甸國王孟既遣使表賀。乙亥,撫恤西藏三十九族成災番民。是月,賑山東益都、臨朐二縣地震災。蠲直隸隆平、寧晉二縣窪地額賦之五。

十年春正月丁巳,暹羅國王鄭福遣使表賀,並貢方物。是月,賑江蘇沛縣、安徽盱眙等六州縣衛旱災水災。貸直隸滄州、鹽山二州縣,甘肅皋蘭等十四州縣旱災水災銀穀。

二月壬戌,上御經筵。丁卯,命緝捕河南梟匪、捻匪。丁丑,命緝捕江西上猶縣會匪。

三月庚寅,以謁西陵,命奕紹、托津、長齡、盧廕溥留京辦事。己亥,免湖南澧州濱湖淤田額賦並前借籽種銀。壬寅,上奉皇太后謁西陵。以升寅為綏遠城將軍。甲辰,調瑚松額為盛京將軍,以福克精阿為吉林將軍,武忠額為察哈爾都統。丙午,上謁泰陵、泰東陵、昌陵。己酉,上幸南苑。庚戌,上行圍,至壬子如之。壬子,以哈薩克汗阿勒坦沙喇等請遣其子入覲,命至熱河陛見。

四月辛未,申禁外省才不勝任之員改京職。

五月辛酉,河南、直隸毗連十四州縣地震,命加意撫恤。

六月辛卯,蔣攸金舌有疾,以陶澍署兩江總督。乙未,以程祖洛為湖南巡撫。

七月丙子,暹羅遣使賀萬壽貢方物。免江蘇海州四州縣舊欠額賦。

八月乙未,萬壽節,停筵宴。庚戌,召蔣攸金舌來京,授陶澍兩江總督。調盧坤為江蘇巡撫。以硃桂楨為廣東巡撫。命吳邦慶以三品銜署漕運總督。是月,加賑湖北監利等四縣水災。

九月戊午,安集延回匪復入喀什噶爾,幫辦大臣塔斯哈戰敗,死之,遂圍喀什噶爾城。命玉麟等往剿。命楊遇春駐肅州,楊芳、胡超率陜甘兵協剿。以鄂山署陜甘總督。徐炘署陜西巡撫,阿勒精阿署山西巡撫。己未,以楊遇春為欽差大臣,督辦軍務。以英惠署黑龍江將軍。丁卯,命長齡為欽差大臣,率桂輪、阿勒罕保等赴新疆。辛未,以玉英署黑龍江將軍。乙亥,上閱火器營兵。丁丑,大學士蔣攸金舌以讞獄誤,降侍郎。召徐炘來京,以顏伯燾署陜西巡撫。以盧廕溥為大學士,李鴻賓協辦大學士,仍留兩廣總督任。調湯金釗為吏部尚書,王引之為禮部尚書,潘世恩為工部尚書,硃士彥為左都御史。是月,賑直隸磁州等三州縣地震災、四川彭城等二縣水災。

十月,以盧廕溥為體仁閣大學士。戊子,以富呢揚阿為浙江巡撫。乙未,仍授長齡為揚威將軍,命哈哴阿、楊芳參贊軍務。庚子,以樂善為烏里雅蘇臺將軍。辛丑,以軍事遲誤,褫伊犁參贊大臣容安職並所襲子爵。壬寅,以恩銘為烏里雅蘇臺參贊大臣。癸卯,回匪犯葉爾羌,壁昌等擊敗之。丁未,逮容安。壬子,召富呢揚阿來京。是月,賑直隸大城、文安二縣災民。給安徽蕪湖等五州縣衛口糧。貸黑龍江等三處旗民倉穀、甘肅皋蘭等十一州縣貧民口糧。

十一月,以楊懌曾為湖北巡撫。乙亥,申諭李鴻賓等查辦廣東會匪。丁丑,諭吳光悅查辦江西贛南會匪。壬午,嵩孚降調,以盧坤為湖廣總督。調程祖洛江蘇巡撫,蘇成額湖南巡撫。以祁為廣西巡撫。以阿勒精阿為江西巡撫。是月,賑河南安陽等三縣地震災。給江西廬陵縣水災修屋費。

十二月癸巳,托津免管刑部,以盧廕溥代之。丙申,喀什噶爾、英吉沙爾回匪平。予死事喀什噶爾幫辦大臣塔斯哈都統銜。是月,賑雲南習瓘縣水災。貸江蘇蘇州等四府州屬駐近災區兵丁銀米。


\end{pinyinscope}