\article{列傳三十五}

\begin{pinyinscope}
許定國劉良佐左夢庚郝效忠徐勇盧光祖田雄馬得功

張天祿弟天福趙之龍孫可望白文選

許定國,河南太康人。明崇禎間,官山西總兵官。李自成圍開封,趣定國赴援,師次沁水,一夕師潰,逮治論死。尋復授援剿河南總兵官。福王時,駐軍睢州。

順治元年,豫親王多鐸下河南,次孟津,定國使請降。肅親王豪格略山東,復上書請以其孥來附,肅親王命遣子為質。二年,遣其子詣肅親王軍。明督師大學士史可法遣總兵高傑徇河南,次歸德,聞定國已遣子納款,招往會,不赴。傑乃與巡撫越其傑、巡按陳潛夫就定國睢州,定國不得已郊迎。其傑勸傑勿入城,傑輕定國,不聽。既入,定國宴傑,侑以妓。傑酣,為定國刻行期,並微及遣子納款事。定國益懼,中夜伏兵殺傑。明日,傑部將攻定國屠城。定國走考城,遂來降。

豫親王請以定國從征,留其孥曹縣,命河道總督楊方興厚贍之。定國妻邢有疾,乞還鄉里,方興為代奏。命暫居曹縣,俟定國入覲。豫親王師還,定國詣京師,隸漢軍鑲白旗。三年,卒。五年,以來降功,授一等精奇尼哈番,子爾安襲。十二年,詔求言,爾安為睿親王多爾袞訟功德,請修其墓。語詳睿親王傳。坐煽惑,減死流寧古塔。弟爾吉襲。

史可法置江北四鎮,傑與劉澤清、劉良佐、黃得功分領之。傑為定國所殺,得功戰死蕪湖。

劉良佐,直隸人,明總兵,預擁立福王。順治二年,豫親王下江南,良佐以兵十萬來降。江南定,詣京師,隸漢軍鑲黃旗。五年,以來降功,授世職二等精奇尼哈番。從大將軍譚泰討金聲桓。師還,授散秩大臣。十八年,授江南江安提督,加總管銜。尋改直隸提督,改左都督。康熙五年,以病乞休。六年,卒。劉澤清既降復叛,誅死。

左夢庚,山東臨清人。父良玉,明史有傳。良玉初授平賊將軍,及封寧南伯,以平賊將軍印授夢庚。福王時,良玉舉兵自武昌東下,號「清君側」。次九江,病卒。諸將推夢庚為帥。總督袁繼咸御戰,夢庚還駐池州,遣兵間道自彭澤下建德,遂取安慶。總兵黃得功破之銅陵,乃退保九江。

順治二年,英親王阿濟格逐李自成至九江,夢庚率眾降。師還,入覲,宴午門內,命隸漢軍正黃旗。疏言:「部將盧光祖、李國英從入京師,餘若張應祥、徐恩盛、郝效忠、金聲桓、常登、徐勇、吳學禮、張應元、徐育賢俱奉英親王調發防剿江西、湖廣。誠恐諸將在外,蹤跡未定,室家未安,訛惑之事,不可不籌。」命有司安插。五年,敘來降功,授一等精奇尼哈番。六年,從英親王討大同叛將姜瓖,攻左衛,克之。擢本旗固山額真。十一年,卒,謚莊敏。乾隆初,定封一等子。夢庚諸將,李國英最顯,自有傳。

郝效忠,遼東人,隸漢軍正白旗。從英親王定湖南,擢湖南右路總兵,加都督僉事,授世職三等阿達哈哈番。孫可望陷沅州,效忠率師克黎平。可望兵驟至,力戰,馬蹶被執,不屈,遂見殺,贈都督同知。

徐勇,亦遼東人。英親王檄署九江總兵,調黃州,捕治九江、黃州土寇。明唐王使招之,勇斬使以聞,命移鎮長沙。金聲桓叛,招勇,復斬其使。與李錦戰江中,中矢,裹創戰愈奮。賊攻城,設策守御,錦遁去。迎鄭親王師擊破明大學士何騰蛟。復調辰常總兵,授世職一等阿達哈哈番兼拖沙喇哈番。明桂王遣將張光翠、張景春窺辰州,屯荔溪。勇督將士渡江戰,擊殺景春,擒裨將六,馘士卒數百,加左都督,進世職三等阿思哈尼哈番。桂王復遣白文選來攻,驅象為陣,破城,勇巷戰死之,贈太子太保,進世職二等,謚忠節。以其兄子襲,入籍武昌衛。

盧光祖,遼東海州人,隸漢軍鑲藍旗。從肅親王下四川,破張獻忠。授夔州總兵。擊破明桂王將硃天麟等。取順慶,屢捕治土寇。甘一爵、硃德洪據鄰水、大竹為亂。光祖督師討之,戰七晝夜,斬一爵、德洪,降硐寨十餘。以功授世職一等阿達哈哈番。孫可望破敘州,將軍李國翰率師赴援,光祖殿,遇敵,戰敗,命立功自贖。尋改川北總兵。卒。金聲桓既降復叛,誅死。

田雄,直隸宣化人。馬得功,遼東人。仕明皆至總兵。順治二年,豫親王多鐸下江南,明福王由崧走蕪湖。巴牙喇纛章京圖賴督兵截江斷道,雄、得功縛福王及其妃來獻,豫親王令以原銜從征。尋授雄杭州總兵,得功鎮江總兵。

雄佐總督張存仁、梅勒章京珠瑪喇,駐軍杭州。時明魯王以海稱「監國」紹興,乘間渡錢塘江窺杭州,雄與存仁、珠瑪喇等屢擊破之。三年,擢浙江提督。六年,發李成棟逆書,加左都督。八年,敘來降功,授世職一等精奇尼哈番。

明魯王與其臣阮進、張名振等據舟山,雄與固山額真金礪以舟師出海擒進,遂破舟山,隳其城,名振擁魯王入海。十二年,進將阮思、陳六禦等復據舟山,朝命寧海大將軍伊爾德率師南征。雄預治戰艦攻具,分兵遣裨將扼要隘,通聲援,而以舟師會伊爾德擊思,以橫洋、金塘為舟山要路,分兵擊破之。張兩翼夾擊,殲其眾無算,思赴水死。捷聞,加少傅兼太子太傅。十五年,疏請歸旗籍,隸漢軍鑲黃旗。

鄭成功兵擾浙境,陷遂安、平陽諸縣。兵部劾雄,上命寬之。十六年,成功兵攻太平,擊卻之。復攻寧波,雄督戰,分三路進剿,成功兵引退。十八年,進二等侯。康熙二年,卒,贈太傅,謚毅勇。

得功,亦隸漢軍鑲黃旗。江寧初定,明瑞昌王誼泐屯花山、龍潭間。順治三年,謀攻江寧,事洩,走鎮江。得功獲誼泐,誅之。尋以收劫盜入伍,降調。四年,大學士洪承疇請以得功署副將。從浙閩總督張存仁剿建寧、邵武山寇,克松溪、政和、建陽、崇安、光澤諸縣,即令駐松溪。復克慶元、永春、德化諸縣。六年,授右路總兵,加都督僉事。克南安,破海寇林忠。復捕治興化、仙游、惠安諸縣海寇鄭丹國等。

時鄭成功據廈門。巡撫張學聖詗成功方出,令得功攻廈門,克之。成功還救,復陷。遂圍漳州,破海澄。得功退守泉州,與固山額真金礪會師解漳州圍。以得功初克廈門貪取財物為成功所乘,命逮治,援赦免。十一年,敘前功,賜一品頂帶,出鎮泉州。得功自陳與雄同降,援雄例乞世職,授一等精奇尼哈番,加都督同知。

十三年,擢福建提督。林忠復據永春、德化、尤溪、大田諸縣,巡撫宜永貴令得功率師討之。師行,寇自閩安逕攻會城,得功引師還,與城兵夾擊,圍解。十四年,與浙閩總督李率泰等合兵克閩安,成功屢內犯,得功擊卻之。十八年,進三等侯。康熙元年,遷濱海居民內地,擊敗海寇阻民遷者。二年,師進攻廈門,得功克烏沙,以舟師出海。南風起,寇乘上流來戰,得功奮擊,沒於陣。李率泰以聞,進一等侯,謚襄武。子三奇,襲爵,官至潮州總兵。乾隆十四年,定諸侯、伯封號,雄曰順義,得功曰順勤。

張天祿,陜西榆林人。明季與弟天福以義勇從軍,積功至總兵。福王時,大學士史可法督師,令屯瓜洲為前鋒。豫親王師下江南,天祿、天福率所部三千人從趙之龍迎降,豫親王令以原官從征,隸漢軍鑲黃旗。

明僉都御史金聲家休寧,受唐王命,糾鄉勇十餘萬據徽州。貝勒博洛遣固山額真葉臣率師擊之,天祿及總兵卜從善、李仲興、劉澤泳並從。師自旌德入,戰績溪,獲聲及中軍吳國禎、副將成有功、守備萬全等,送江寧殺之。徽州平。

明大學士黃道周率兵犯徽州,天祿擊之,斬其將程嗣聖等十餘人,獲總兵李堯光等。順治三年,戰婺源,獲道周,亦送江寧殺之。分兵出祁門、江灣、街口、黃源,四道逐捕道周餘眾。以功加都督同知,授徽寧池太總兵官。天祿屯徽州城外,依山為營。值雨,父老迎天祿入城,天祿曰:「三軍方在泥塗,何忍獨安?」終不下山。軍民皆稱之。明嵩安王常淇糾眾數千擾婺源,天祿率副將許漢鼎等擊之,獲常淇及監軍江於東等。四年,授江南提督。五年,敘來降功,授世職三等阿達哈哈番。八年,進三等精奇尼哈番。

九年,鄭成功圍漳州,命天祿赴援,成功引退。天祿留駐延平,捕治山寇。十一年,明魯王將張名振攻崇明,天祿還松江御戰。名振既出海,復侵吳淞。我水師與戰,敗績。江南總督馬鳴佩劾天祿失舟師三百餘及砲械,匿未報;閩浙總督佟泰劾天祿與名振通書:逮下刑部,讞通書無據,坐匿失砲械等,奪官,降世職三等阿達哈哈番。十六年,卒。

天福初降,從征昆山、嘉定。民不薙發,據城抗我師,天福與總兵李成棟討平之。順治五年,授陜西漢羌總兵。敘來降及戰功,授世職一等阿思哈尼哈番。明山陰王鼎濟聚兵據毛壩關,署單一涵為元帥。年六,天福自漢中率師入山,獲鼎濟,一涵投崖死。參將王永祥叛延安,山寇劉宏才攻同官,天福先後討平之。以病還京師,授散秩大臣。十七年,授本旗都統。康熙六年,卒。

趙之龍,江南虹縣人。崇禎時,以忻城伯鎮南京。福王立,與擁戴,干政。豫親王師至,與魏國公徐允爵,保國公張國弼,隆平侯張拱日,臨淮侯李祖述,懷寧侯孫維城,靈壁侯湯國祚,安遠侯柳祚昌,永昌侯徐宏爵,定遠侯鄧文囿,項城伯常應俊,大興伯鄒存義,寧晉伯劉允極,南和伯方一元,東寧伯焦夢熊,安城伯張國才,洛中伯黃九鼎,成安伯郭祚永,駙馬齊贊元,大學士王鐸,尚書錢謙益,侍郎硃之臣、梁雲構、李綽等迎降。之龍授世職三等阿思哈尼哈番,允爵等皆置勿用。鐸等詣京師。先是北都降者多授原官,御史盧傳言南都新人不得與舊臣比。鐸至,命以尚書管弘文院學士,累擢至禮部尚書,卒,謚文安。謙益語在文苑傳。

孫可望,陜西延長人。從張獻忠為賊,與李定國、劉文秀、艾能奇並為獻忠養子。獻忠據四川,使分將其眾,可望號平東將軍。順治三年,肅親王豪格師定四川,獻忠敗死。可望與定國等率殘眾南竄,道重慶、綦江、遵義入貴陽。阿迷土司沙定洲亂雲南,可望率眾兼程赴之。定洲方攻楚雄,迎戰大敗,走歸阿迷。可望入雲南會城,遣定國徇迤東,而與文秀率兵西出,得副使楊畏知,相誓扶明室,與俱至楚雄,略迤西諸府。定國亦定迤東諸府。可望遂盡有雲南,自號平東王,以干支紀年,鑄錢曰「興朝通寶」。時能奇已前死,可望並將其眾。定國、文秀故等夷,不為可望下。可望假事杖定國,欲以威眾,隙益深。

明桂王在肇慶,乃遣畏知奉表乞王封,桂王封可望景國公,賜名朝宗。使以敕印往,而桂王諸將爭欲得可望為強援。堵胤錫駐梧州,承制改封平遼王;陳邦傅守泗城,又矯命封秦王;可望乃不受景國公命。會我師克韶州,桂王走梧州。可望復遣使請封,議封澂江王。使者謂非秦不敢復命,大學士嚴起恆持不可,議中寢。可望襲貴陽,復遣文秀攻嘉定,入四川。我師定兩廣,桂王至南寧,乃遣使封可望冀王,可望猶不受,復使畏知詣桂王,而遣其將賀九儀等以五千人先驅,取起恆及諸臣阻秦封者盡殺之。桂王乃真封可望秦王,而留畏知授大學士。可望聞之怒,召至貴陽面數之,畏知以冠擊可望,亦被殺。

桂王遣大學士文安之督師四川,將以招川中諸鎮。可望遣兵伺於都勻,邀止之。可望將移桂王自近,挾以作威。桂王奔廣南,可望遣兵迎入安隆所,改為安龍府,歲供銀八千、米百石,窮迫不可堪;而馬吉祥、龐天壽輩方欲戴可望行禪讓,可望遂自設內閣六部等官,立太廟,定朝儀,改印文為八疊。桂王益憂懼。

初,定國自廣西入湖廣,兵益強,不復稟可望約束。會定國敗於衡州,使召詣沅州議事,將以為罪而殺之;定國辭不赴,又自柳州攻肇慶,下高、廉、雷諸府。至是,桂王聞定國兵強,密詔使入衛。可望聞,使執大學士吳貞毓等,凡預謀者盡殺之。議移桂王貴陽,使其將白文選督行期。文選心不直可望,以情輸桂王,緩其行。俟定國至,奉桂王自安南衛走雲南。時文秀守雲南,亦怨可望,迎桂王入雲南會城。可望舉兵反桂王,以雙禮留守,令文選統諸軍前行。定國、文秀率師御之,次三岔河,夾水而軍。文選輕騎奔定國。可望遣其將張勝、馬寶等自尋甸間道襲雲南,而自率勁卒擊定國等。戰方合,其將馬惟興先奔,遂大潰,定國遣文秀等追之。

可望至貴陽,雙禮紿言追兵且至。可望知事去,將詣經略洪承疇請降,遣使先納款。文秀等遣將楊武追之,及於沙子嶺。承疇援兵至,乃得脫,將妻子詣長沙降,時順治十四年十月也。詔封義王,慰諭之。尋遣學士麻勒吉等齎敕印冊封。十五年,詣京師,命簡親王濟度等郊迎。入覲,宴中和殿,賜白金萬,官其部將陳傑,劉天瑞等百餘人,命隸漢軍正白旗。可望請從討雲南自效,下王大臣議,寢其奏。十七年,疏辭封爵,復慰諭之。尋卒,謚恪順。

子徵淇襲,未幾卒。徵淳襲,卒,謚順愍。徵灝請襲,御史孟飛熊疏言:「可望,獻忠餘黨,久據滇、黔,負固不服。及為定國所敗,窮蹙來歸,濫膺非分。宜即停止,或以次降等。」下部議,降襲慕義公,官至兵部尚書,謚清端。子降襲一等阿達哈哈番。乾隆三十六年,命停襲。

文選,陜西吳堡人,亦從獻忠為賊。獻忠敗,從可望入貴州。其緩桂王使得入雲南也,桂王封為鞏國公,令還貴陽慰諭可望,可望奪其兵,置軍中。及將舉兵,諸將說可望原得文選為大將,可望使將前鋒,遂降定國,可望以是敗。桂王封文選鞏昌王。

順治十六年,我師下雲南,定國戰屢敗,令文選為殿;戰於玉龍關,文選復敗,走木邦。桂王入緬甸,居赭硜。十七年,文選攻阿瓦,弗克,與定國會師孟艮;再攻阿瓦,求出桂王,終不獲,我師益深入。文選據錫箔,憑江拒守。我師出木邦,造筏將渡,文選奔茶山。總兵馬寧將偏師追之,及於猛養,文選降。詔封承恩公,亦隸漢軍正白旗。康熙元年,命予三等公俸。七年,加太子少師。十四年,卒。子繪,降襲一等精奇尼哈番。卒,停襲。

論曰:邦家新造,師行所至,逆者誅,順者庸。雖其人叛故國,賊舊君,茍為利於我,固不能不以為功也。可望獨以臺官言降爵,終見削奪。唐通降自成,既復來歸,授世職,康熙間即停襲,事又在其前;而定國、夢庚、雄輩及他諸降將,皆襲封如故。民間傳雄負福王出,王咬其項,遂潰死。雄死時,明亡已二十年。其言誠無稽,然民之所惡,蓋亦可見矣。


\end{pinyinscope}