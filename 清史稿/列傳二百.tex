\article{列傳二百}

\begin{pinyinscope}
曾國荃弟貞幹沈葆楨劉坤一

曾國荃,字沅甫,湖南湘鄉人,大學士國籓之弟也。少負奇氣,從國籓受學京師。咸豐二年,舉優貢。六年,粵匪石達開犯江西,國籓兵不利。國荃欲赴兄急,與新授吉安知府黃冕議,請於湖南巡撫駱秉章,使募勇三千人,別以周鳳山一軍,合六千人,同援江西。十一月,克安福,連破賊於大汾河、千金坡,進攻吉安,下旁數縣。

七年春,丁父憂回籍。夏,賊麕聚吉安,周鳳山軍敗潰。時王珍、劉騰鴻皆喪亡,士氣衰沮。江西巡撫耆齡奏起國荃統吉安諸軍,軍復振。冬,敗石達開於三曲灘,吉安圍始合。八年春,克吉水、萬安。八月,督水師毀白鷺洲賊船,破城外堅壘,遂克吉安,擒賊首李雅鳳。以功累擢知府,撤軍還長沙。九年,復赴江西,率硃品隆等軍五千餘人援剿景德鎮。時諸軍與賊相持數月,莫肯先進。國荃至,乃合力敗援賊於浮梁南。三戰皆捷,火鎮市,追殲賊及半,克浮梁,擢道員。江西肅清。

國籓出九江,至黃州,與胡林翼議分路圖皖。國荃留軍巴河,自還湖南增募為萬人。多隆阿、鮑超等既大破賊於太湖、潛山,十年閏三月,國荃乃進軍集賢關,規攻安慶。陳玉成來援,擊走之。十一年,陳玉成復糾捻眾至於菱湖,兩岸築堅壘,與城賊更番來犯。國荃調水師入湖,令弟貞幹築壘湖東以御之。會陳玉成在桐城為多隆阿所敗,還趨集賢關,迎擊破之。玉成由馬踏石遁走,仍留黨踞赤岡嶺,與菱湖賊壘犄角。國荃困以長壕,鮑超來,合攻,悉破其壘,擒斬萬餘。進破安慶城外賊營,毀東門月城。惟北門三石壘堅不可下,令降將程學啟選死士緣砲穴入,拔之。陳玉成屢為多隆阿所創,收餘眾,糾合捻匪,復屯集賢關,襲官軍後路,城賊葉蕓來亦傾巢出撲。國荃憑壕而戰,屢擊卻,仍復進,增築新壘,遣貞幹合水師扼菱湖,絕賊糧路。八月,以地雷轟城,克之,殲賊萬餘,俘數千。捷聞,以按察使記名,加布政使銜,賜黃馬褂。尋以追殄餘賊,賜號偉勇巴圖魯。於是國籓進駐安慶,國荃率師東下規江寧,克無為州,破運漕鎮,拔東關,加頭品頂戴。分兵守諸隘,自回湖南增募勇營。

同治元年,授浙江按察使,遷江蘇布政使。詔以軍務緊要,毋庸與兄國籓回避同省。三月,率新募六千人至軍,自循江北岸,令弟貞幹循南岸,彭玉麟等率水師同進,拔銅城徬、雍家鎮諸隘,復巢縣、含山、和州,克裕溪口、西梁山。渡江會攻金柱關,乘間襲太平,克之。回克金柱關,貞幹亦克蕪湖。令彭毓橘截敗賊於薛鎮渡口,大破之。五月,連奪秣陵關、大勝關要隘。水師進扼江寧護城河口,陸師逕抵城南雨花臺駐屯,賊來爭,皆擊卻之。國籓猶以孤軍深入為慮,國荃謂:「舍老巢勿攻,浪戰無益,逼城足以致敵。雖危,事有可為。」會秋疫大作,士卒病者半。賊酋李秀成自蘇州糾眾數十萬來援,結二百餘壘。國荃於要隘增壘,輔以水師,先固糧道。賊環攻六晝夜,彭毓橘等乘其乏出擊,破賊營四。賊悉向東路,填壕而進,前僕後繼。國荃督軍抵御,砲傷頰,裹創力戰,賊始退。李世賢又自浙江率十萬眾至,與秀成合攻,屢掘地道來襲,毀營墻,百計攻襲,皆未得逞。蕪湖守將王可升率援師至,國荃簡精銳分出,焚賊數壘,餘棄壘走,進擊,大破之。先後殲賊數萬,圍乃解。秀成、世賢引去。是役以病餘之卒,苦戰四十餘日,卒保危局,詔嘉獎,頒珍賚。

議者欲令乘勝退保蕪湖,國荃以賊雖眾,烏合不足畏,不肯退。二年春,國籓親至視師,見圍屯堅定,始決止退軍之議。詔擢浙江巡撫,仍統前敵之軍規取江寧。四月,攻雨花臺及聚寶門外石壘,克之。九洑洲為江寧犄角,賊聚守最堅。國荃偕彭玉麟、楊岳斌往覘形勢,合水陸軍血戰,克之,江面遂清。連克上方橋、江東橋,近城之中和橋、雙橋門、七甕橋,稍遠之方山、土山、上方門、高橋門、秣陵關、博望鎮諸賊壘,以次並下。國荃初至,合各路兵僅二萬,至是募圍師至五萬人。十月,分軍扼孝陵衛。李鴻章克蘇州,李秀成率敗眾分布丹陽、句容,自入江寧,勸洪秀全同走,不聽,遂留同城守。

三年春,克鍾山天保城,城圍始合。賊糧匱,城中種麥濟饑。國荃迭令掘地道數十處,賊築月圍以拒,士卒多傷亡。會詔李鴻章移師會攻,諸將以城計日可破,恥借力於人,攻益力。鴻章亦不至。國荃慮師老生變,督李臣典等當賊砲密處開地道。既成,懸重賞募死士,李臣典、硃洪章、伍維壽、武明良、譚國泰、劉連捷、沈鴻賓、張詩日、羅雨春誓先登者九人。六月十六日,日加午,地道火發,城崩二十餘丈,李臣典、硃洪章等蟻附爭登。賊傾火藥轟燒,彭毓橘、蕭孚泗手刃退卒數人,遂擁入。硃洪章、沈鴻賓、羅雨春攻中路,向偽天王府;劉連捷、張詩日、譚國泰攻右路,趨神策門,硃南桂等梯城入,合取儀鳳門;其左路彭毓橘由內城至通濟門,蕭孚泗等奪朝陽、洪武門,羅逢元等從聚寶門入,李金洲從通濟門入,陳湜、易良虎從旱西、水西門入:於是江寧九門皆破。守陴賊誅殺殆盡,猶保子城。夜半,自縱火焚偽王府,突圍走。要截斬數百人,追及湖、熟,俘斬亦數百。洪秀全已前一月死,獲其尸於偽宮。其子洪福瑱年十五六,訛言已自焚死,餘黨挾之走廣德。國荃令閉城救火,搜殺餘賊。獲秀全兄洪仁達及李秀成,伏誅。凡偽王主將大小酋目三千餘,皆死亂兵,斃賊十餘萬,拔難民數十萬。捷聞,詔嘉國荃堅忍成功,加太子少保,封一等伯爵,錫名威毅,賜雙眼花翎。

國荃功高多謗,初奏洪福瑱已斃,既而奔竄浙江、江西,仍為諸賊所擁,言者以為口實,遂引疾求退,遣撤部下諸軍,溫詔慰留;再疏,始允開缺回籍。四年,起授山西巡撫,辭不就。調湖北巡撫,命幫辦軍務,調舊部剿捻匪。

五年,抵任,汰湖北冗軍,增湘軍六千,以彭毓橘、郭松林分統之。時捻匪往來鄂、豫之交,國荃檄鮑超由棗陽趨淅川、內鄉防西路,郭松林由桐柏、唐縣出東路,劉維楨向新野為聲援。賊折而北竄,詔郭松林越境會剿。是年冬,敗賊於信陽、孝感。賊竄雲夢、應城、德安,郭松林擊走之,克應城、雲夢,又敗之皁河、楊澤。松林追賊臼口,中伏受重傷,其弟芳珍戰死。彭毓橘破賊於沙口,又敗之安陸。國荃以賊多騎,難與追逐,欲困之山地。毓橘偕劉維楨屢戰不能大創,賊竄去。總督官文與不協,國荃疏劾其貪庸驕蹇,詔解官文總督任。六年春,賊復犯德安,為劉銘傳、鮑超所敗,遁入河南境,尋復回竄。彭毓橘恃勇輕進,遇賊蘄州,戰歿於六神港。五月,捻匪長驅經河南擾及山東。詔斥諸疆吏防剿日久無功,國荃摘頂,下部議處,尋以病請開缺,允之。

光緒元年,起授陜西巡撫,遷河東河道總督。二年,復調山西巡撫。比年大旱,災連數省。國荃力行賑恤,官帑之外,告貸諸行省,勸捐協濟,分別災情輕重、賑期久暫,先後賑銀一千三百萬兩、米二百萬石,活饑民六百萬。善後蠲徭役,歲省民錢鉅萬。同時荒政,山西為各省之冠,民德之,為立生祠。六年,以疾乞罷,慰留,尋召來京。七年,授陜甘總督,命赴山海關治防,復乞病歸。八年,署兩廣總督。

九年,內召。十年,署禮部尚書,調署兩江總督兼通商大臣,尋實授。時法蘭西兵犯沿海,中朝和戰兩議相持。國荃修江海防務,知上海關系諸國商務,法兵不能驟至,馭以鎮靜。詔遣文臣分赴海疆會辦,福建疆吏遂不能主兵。國荃言權不可分,朝廷亦以其老於軍事,專倚之。命遣兵輪援臺灣,原議五,實遣其三。坐下部議,革職留任。兵輪終不得達,其二折至浙洋,助戰鎮海有功,和議尋定。十一年,京察,以國荃夙著勛勤,開復處分。十五年,皇太后歸政,推恩加太子太保。

國荃治兩江凡六年,總攬宏綱,不苛細故,軍民相安。十六年,卒於官,贈太傅,賜金治喪,命江寧將軍致祭,特謚忠襄,入祀昭忠祠、賢良祠,建專祠。孫廣漢襲伯爵,官至左副都御史。

國荃弟貞幹,原名國葆。諸生。從兄國籓剿平常德、寧鄉土匪。時楊岳斌為把總,彭玉麟為諸生,貞幹亟稱於國籓,謂二人英毅非常,同闢領水師。初敗於岳州,貞幹自引咎,言諸將無罪。國籓東征,貞幹家居未從。及其兄國華戰歿三河,貞幹誓殺賊復仇。胡林翼使領千人,自黃州轉戰潛山、太湖。從國荃攻安慶,設計招降賊將程學啟,克城之功,學啟為多。同治元年,與國荃分路沿江進師,破魯港,克繁昌、南陵、蕪湖,會軍雨花臺。尋染疫,將假歸,援賊至,被圍,強起任戰守,圍解而病劇,卒於軍。初以功敘訓導,加國子監學正銜,賜號迅勇巴圖魯。既破援賊,擢知府,命下而貞幹已歿。事聞,贈按察使。李鴻章為陳戰績,詔依二品議恤,贈內閣學士,予騎都尉世職,建專祠,謚靖毅。

沈葆楨,字幼丹,福建侯官人。道光二十七年進士,選庶吉士,授編修。遷御史,數上疏論兵事,為文宗所知。咸豐五年,出為江西九江知府。九江已陷賊,從曾國籓筦營務。六年,署廣信府。賊酋楊輔清連陷貴溪、弋陽,將逼廣信。葆楨方赴河口籌餉,聞警馳回郡,官吏軍民多避走。妻林,先刺血書乞援於浙軍總兵饒廷選。會大雨,賊滯興安。廷選先入城,賊至,七戰皆捷,解圍去。曾國籓上其城守狀,詔嘉獎,以道員用。七年,擢廣饒九南道,留筦廣信防務。數假客軍擊走竄賊,平弋陽土匪,誅安仁抗糧奸民,加按察使銜。以伉直忤大吏,乞養親去官。

十年,起授吉贛南道。以親老辭,未出,命留原籍治團練。曾國籓屢薦其才,十一年,詔赴安慶大營委用。未幾,超擢江西巡撫,諭曰:「朕久聞沈葆楨德望冠時,才堪應變。以其家有老親,擇江西近省授以疆寄,便其迎養;且為曾經仕宦之區,將來樹建殊勛,光榮門戶,足承親歡。如此體恤,如此委任,諒不再以養親瀆請。」葆楨奉詔,感泣赴官。時浙江淪陷,左宗棠由江西進軍規復。賊酋楊輔清、李世賢合擾江西,冀斷皖、浙運道。同治元年,葆楨親赴廣信籌防,令士民築堡自衛,堅壁清野。倚用湘將王德榜、段起及席寶田、江忠義諸人,客軍並聽指揮,賊至輒擊退。二年,破黃文金於小路口,又破之於祁門。會浙軍克黟縣,賊由太平、石埭、建德擾江西,督軍進擊走之。是年秋,因病請假。

初,曾國籓軍餉多倚江西。葆楨以本省軍事方殷,奏留自給。江寧前敵需餉亟,而江西協解不至,國籓疏爭。御史華祝三亦疏言兩人齟,慮誤大局,詔兩解之,命各分其半,別以江海關撥款濟江南軍。三年,大軍圍江寧急。賊聚擾江西,圖牽後路。詔楊岳斌移師督剿,命葆楨會商機宜。既而江寧、杭州相繼復,黃文金擁洪福瑱由浙、皖竄江西,為入粵計。葆楨令席寶田追剿,至石城,大破之。陣擒洪仁玕、洪仁政、黃文英等,搜獲洪福瑱於荒谷中,皆伏誅。以擒首逆功,予一等輕車都尉世職,加頭品頂戴。葆楨推功諸將,疏辭,詔嘉其開誠布公,將士用命,且江西吏治民風,日有起色,宜膺殊賞,不允所請。尋乞歸養,溫詔慰留。四年,以親病請假省視,因防務急,未行,丁母憂,命治喪百日,假滿仍回任。堅請終制,乃允之。

六年,命為總理船政大臣。初,左宗棠創議於福州馬尾山麓瀕江設船廠,未及興工,宗棠調陜甘,疏言非葆楨莫能任。葆楨釋服,始出任事。造船塢及機器諸廠,聘洋員日意格、德克碑為監督。月由海關撥經費五萬兩,期以五年告成。附設藝童學堂,預募水勇習練駕駛。事皆創立,船材來自外國,煤炭亦購諸南洋,採辦尤易侵漁。葆楨堅明約束,一無瞻徇。布政使周開錫為提調,延平知府李慶霖佐局事,皆為總督所不喜,齮齕欲去之,葆楨疏爭得留,籓署吏玩抗,以軍法斬之,眾咸驚服。

九年,丁父憂,仍請終制,暫解事,服闋始出。當其居憂,內閣學士宋晉疏請暫停船工,詔下酌議。葆楨上疏,略謂:「自強之道,與好大喜功不同,不可以浮言搖動。且洋員合同不能廢,機廠經營不可棄。不特不能即時裁撤,五年期滿,亦不可停。」推論利害切至,詔嘉納之。十一年,再蒞事。先後造成兵艦二十艘,分布各海口。尋以匠徒藝成,議酌改船式,督令自造,不用洋員監督。疏陳善後事宜,並如議行。

十三年,日本因商船避風泊臺灣,又為生番所戕,藉詞調兵,覬覦番社地。詔葆楨巡視,兼辦各國通商事務。日兵已登岸結營,葆楨據理詰之。曉諭番族遵約束,修城築壘為戰備。提督唐定奎亦率淮軍至,日人如約撤兵。乃議善後事宜,疏陳福建巡撫宜移駐臺灣,吏治軍政方能整頓,詔如所請。甫內渡,獅頭社番戕官滋事,光緒元年,復往,督唐定奎等伐山開道,攻破內外獅頭等社,毀其巢,脅從者次第就撫。中路、北路亦分軍深入,諸番皆聽約束。先於瑯玡增設恆春縣,至是奏設臺北府,淡水、新竹、宜蘭三縣隸之;噶瑪蘭通判移駐雞籠山;臺灣府同知移駐卑南;鹿港同知移駐水沙。連疏陳營伍積弊,請歸巡撫節制。購機器,開臺北煤窯,為明遺臣鄭成功請予謚建祠,以作臺民忠義之氣,並報可。遂撤軍內渡,事竣,擢兩江總督,兼通商大臣。

江南自軍事定後,已逾十年。疆吏習為寬大,葆楨精覈吏事,治尚嚴肅。屬吏懍懍奉職,宿將驕蹇者繩以法,不稍假借。尤嚴治盜,蒞任三月,誅戮近百人,莠民屏跡。皖南教案,華教士誣良民重罪,親訊,得其受枉狀,反坐教士,立誅之,然後奏聞,洋人亦屈伏。淮南引地以次歸復,濬河、積穀、捕蝗、禁種罌粟諸政,並實力施行。數以病乞退,五年,入覲,皇太后溫諭勉以共濟時艱,毋萌退志,自此遂不言病。是年十一月,卒於位,贈太子太保,祀賢良祠,立功各省建專祠,謚文肅。子瑋慶,賜舉人,襲一等輕車都尉世職;瑜慶,恩廕主事,官至貴州巡撫。

劉坤一,字峴莊,湖南新寧人。廩生。咸豐五年,領團練從官軍克茶陵、郴州、桂陽、宜章,敘功以教諭即選。六年,駱秉章遣劉長佑率師援江西,坤一為長佑族叔而年少,師事之,從軍中自領一營。長佑既克萍鄉,令進戰蘆溪、宣風鎮,連破賊,逼袁州,招降賊目李能通。於是降者相繼,守城賊何益發夜啟西門,坤一先入,復袁州。累擢直隸州知州,賜花翎。

七年,克臨江,擢知府。八年,長佑以病歸,坤一代將其軍。偕蕭啟江渡贛江規撫州,克崇仁。啟江在上頓渡為賊所困,往援,大破賊,遂復撫州,連克建昌,擢道員。九年,石達開犯湖南,坤一回援,解永州、新寧之圍,加鹽運使銜。賊竄廣西,從劉長佑追躡,復柳州。長佑擢撫廣西,令坤一駐柳州清餘匪,悉平之,加布政使銜。進攻潯州,十一年七月,拔其城,以按察使記名。石達開回趨川、楚,坤一扼之融縣,掩擊敗之,賊潰走入黔,授廣東按察使。

同治元年,遷廣西布政使。劉長佑赴兩廣總督任,命坤一接統其軍,赴潯州進剿。貴縣匪首黃鼎鳳,在諸匪中最狡悍,屢議剿撫,不能下。二年,坤一破之於登龍橋,遂駐守之。鼎鳳老巢曰平天寨,倚山險樹重柵,守以巨砲,覃墟相距十餘里,為犄角。坤一陽議撫,撤軍回貴縣,潛師夜襲覃墟,遂圍平天寨,復橫州,鼎鳳勢蹙。三年四月,擒鼎鳳及其黨誅之。潯州平,賜號碩勇巴圖魯。四年,剿平思恩、南寧土匪,復永淳,擢江西巡撫。令席寶田、黃少春會剿粵匪餘黨於閩邊,五年,聚殲於廣東嘉應州,加頭品頂戴。軍事既定,坤一治尚安靜,因整頓丁漕,不便於紳戶。十一年,左都御史胡家玉疏劾之,坤一奏家玉積欠漕糧,又屢貽書干預地方事。詔兩斥之,家玉獲譴,坤一亦坐先不上聞,部議降三級調用,加恩改革職留任,降三品頂戴。尋復之,命署兩江總督。

光緒元年,擢兩廣總督。廣東號為富穰,庫儲實空,出入不能相抵。議者請加鹽釐及洋藥稅,坤一以加鹽釐則官引愈滯,但嚴緝私販,以暢銷路;又援成案,籌款收買餘鹽,發商交運,官民交便。藥釐抽收,各地輕重不同,改歸一律,無加稅之名,歲增鉅萬。吏治重在久任,令實缺各歸本任,不輕更調。禁賭以絕盜源,水陸緝捕各營,分定地段以專責成,盜發輒獲。

二年,調授兩江總督。六年,俄羅斯以交還伊犁,藉端要挾。詔籌防務,坤一上疏,略謂:「東三省無久經戰陣之宿將勁旅,急宜綢繆。西北既戒嚴,東南不可復生波折。日本、琉球之事宜早結束,勿使與俄人合以謀我。英、德諸國與俄猜忌日深,應如何結為聲援,以伺俄人之後。凡此皆賴廟謨廣運,神而明之。」九年,法越構釁,邊事戒嚴。坤一疏:「請由廣東、廣西遴派明幹大員統勁旅出關,駐扎諒山等處,以助剿土匪為名,密與越南共籌防禦。並令越南招太原、宣光黑旗賊眾,免為法人誘用。雲南據險設奇,以資犄角。法人知我有備,其謀自沮。雲南方擬加重越南貨稅,決不可行。重稅能施之越人,不能施之法人。越人倘因此轉嗾法人入滇通商,得以依託假冒,如沿海奸商故智,不可不慮。越南如果與法別立新約,中國縱不能禁,亦應使其慎重;或即指示機宜,免致再誤。越南積弱,若不早為扶持,覆亡立待。滇、粵籓籬盡失,逼處堪虞。與其補救於後,曷若慎防於先。此不可不明目張膽以提挈者也。」疏入,多被採納。

十二年,丁繼母憂。十六年,仍授兩江總督。十七年,命幫辦海軍事務。二十年,皇太后萬壽,賜雙眼花翎。日本犯遼東,九連城、鳳凰城、金州、旅順悉陷,北洋海陸軍皆失利。召坤一至京,命為欽差大臣,督關內外防剿諸軍。坤一謂兵未集,械未備,不能輕試,詔促之出關。時已遣使議和,坤一以兩宮意見未洽為憂,瀕行,語師傅翁同龢曰:「公調和之責,比餘軍事為重也。」二十一年春,前敵宋慶、吳大澂等復屢敗,新募諸軍實不能任戰,日本議和要挾彌甚,下坤一與直隸總督王文韶決和戰之策。坤一以身任軍事,仍主戰而不堅執。未幾和議成,回任。坤一素多病,臥治江南,事持大體。言者論其左右用事,詔誡其不可偏信,振刷精神,以任艱鉅。坤一屢疏陳情乞退,不許。

二十五年,立溥俊為穆宗嗣子,朝野洶洶,謂將有廢立事,坤一致書大學士榮祿曰:「君臣之分久定,中外之口宜防。坤一所以報國在此,所以報公亦在此。」二十六年,值德宗萬壽,加太子太保。拳匪亂起,坤一偕李鴻章、張之洞創議,會東南疆吏與各國領事訂約,互為保護,人心始定。車駕西幸,議者或請遷都西安,坤一復偕各督撫力陳其不可,籥請回鑾。二十七年,偕張之洞會議請變法,以興學為首務,中法之應整頓變通者十二事,西法之應兼採並用者十一事,聯銜分三疏上之。詔下政務處議行,是為實行變法之始。洎回鑾,施恩疆吏,加太子太保。

二十八年,卒,優詔賜恤。嘉其秉性公忠,才猷宏遠,保障東南,厥功尤著,追封一等男爵,贈太傅,賜金治喪,命江寧將軍致祭,特謚忠誠。祀賢良祠,原籍、立功省建專祠。賜其子能紀四品京堂,諸孫並予官。張之洞疏陳坤一居官廉靜寬厚,不求赫赫之名,而身際艱危,維持大局,毅然擔當,從不推諉,其忠定明決,能斷大事,有古名臣風。世以所言為允。

論曰:曾國荃當蘇、浙未復,孤軍直搗金陵,在兵事為危機,其成功由於堅忍。鏟其本根,則枝蔓自絕,信不世之勛也。屢退復起,朝廷倚為保障,以功名終。沈葆楨清望冠時,力任艱鉅,兵略、吏治並卓然。其手創船政,精果一時無耦。後來不能充拓,且聽廢棄,豈非因任事之難其人哉?劉坤一起家軍旅,謀國獨見其大,晚年勛望,幾軼同儕,房、杜謀斷之功,不與褒、鄂並論矣。


\end{pinyinscope}