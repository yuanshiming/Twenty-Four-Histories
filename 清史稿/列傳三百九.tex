\article{列傳三百九}

\begin{pinyinscope}
籓部五

○青海額魯特

青海額魯特部,在西寧邊外,至京師五千七十里。東及北界甘肅,西界西藏,南界四川,袤延二千餘里,即古西海郡地。分左右二境,左境:東自棟科爾廟,西至洮賚河界,八百餘里;南自博囉充克克河北岸,北至西喇塔拉界,四百餘里;東南自拉喇山,西北至額濟訥河界,四百餘里;東北自永昌縣界,西南至布隆吉爾河界,三千餘里。右境:東自棟科爾廟,西至噶斯池界,二千五百餘里;南自漳臘嶺,北至博囉充克克河南岸,千五百餘里;東南自達爾濟嶺,西北至塞爾騰、西爾噶拉金界,二千餘里;東北自克騰庫特爾,西南至穆嚕烏蘇河界,千五百餘里。

厄魯特舊分四部:曰和碩特,姓博爾濟吉特;曰準噶爾;曰杜爾伯特,姓綽囉斯;曰土爾扈特,姓不著。部自為長,號四衛拉特。金稱厄魯特,即明時所謂阿魯臺也。有輝特者最微,初隸杜爾伯特。後土爾扈特徙俄羅斯境,輝特遂為四衛拉特之一云。青海蒙古分牧而處,有和碩特,有土爾扈特,有準噶爾,有輝特,統以厄魯特稱之。

和碩特設扎薩克二十有一,其始祖為元太祖弟哈布圖哈薩爾,七傳至阿克薩噶勒泰。子二:長阿魯克特穆爾,今內扎薩克科爾沁、扎賚特、杜爾伯特、郭爾羅斯、阿嚕科爾沁、四子部落、茂明安、烏喇特八部,其裔也。次烏嚕克特穆爾,九傳至博貝密爾咱,稱衛拉特汗,子哈尼諾顏洪果爾繼之。有子六:長哈納克土謝圖,次拜布噶斯,次昆都倫烏巴什,次圖魯拜琥,次色棱哈坦巴圖爾,次布雅鄂特歡。哈納克土謝圖為公中扎薩克臺吉車凌納木扎勒一旗祖。拜布噶斯子鄂齊爾圖汗及阿巴賴諾顏,牧西套,後準噶爾滅其部。昆都倫烏巴什號都爾格齊諾顏,今駐牧珠都斯之和碩特部四旗,其裔也。圖魯拜琥號顧實汗,分青海部眾為二翼,子十人領之。居左翼者,曰達延,曰鄂木布,曰達蘭泰,曰巴延阿布該阿玉什。居右翼者,曰伊勒都齊,曰多爾濟,曰瑚嚕木什,曰桑噶爾扎,曰袞布察琿,曰達什巴圖爾。達延號鄂齊爾汗,為扎薩克鎮國公噶勒丹達什,輔國公諾爾布朋素克、車凌三旗祖。別有附察哈爾之和碩特,亦其裔也。鄂木布號車臣岱青,為扎薩克臺吉羅卜藏察罕、濟克濟扎布、達瑪璘色布騰、阿喇布坦四旗祖。達蘭泰為扎薩克郡王額爾德尼額爾克托克托鼐、臺吉車凌多爾濟二旗祖。巴延阿布該阿玉什號達賴烏巴什,為扎薩克臺吉扎布一旗祖。別有阿拉善厄魯特一旗,亦其裔也。伊勒都齊為扎薩克親王察罕丹津、輔國公阿喇布坦札木素、臺吉察罕喇布坦三旗祖。多爾濟號達賴巴圖爾,為扎薩克貝勒朋素克旺扎勒、達什車凌、臺吉伊什多勒扎布三旗祖。瑚嚕木什號額爾德尼岱青,為扎薩克貝子丹巴、臺吉色布騰博碩克圖二旗祖。桑噶爾扎號伊勒登,為扎薩克貝子索諾布達什一旗祖。袞布察琿無嗣。達什巴圖爾子羅卜藏丹津,叛逃準噶爾,後就擒,宥之,隸內蒙古正黃旗。顧實汗弟色棱哈坦巴圖爾,號扎薩克陀音,為扎薩克臺吉哈爾噶斯一旗祖。布延鄂特歡三傳至阿布,子二:長達瓦,次鄂爾奇達遜,隸準噶爾,號扈魯瑪臺吉,後來歸。達瓦封公品級,尋卒。鄂爾奇達遜授伯爵,隸內蒙古正黃旗。

土爾扈特設扎薩克四,其始祖曰翁罕。七傳至貝果鄂爾勒克,為扎薩克臺吉索諾布喇布坦多爾濟、色特爾布木二旗祖。別有土爾扈特部十二旗,亦其裔也。貝果鄂爾勒克弟翁貴,為扎薩克臺吉達爾扎、丹忠二旗祖。

準噶爾設扎薩克二旗,始祖曰孛罕,六傳至額森。子二:長博羅納哈勒,為杜爾伯特所自始,今駐牧烏蘭固木之杜爾伯特部十六旗,自輝特二旗外,皆其裔也。次額斯墨特達爾漢諾顏,為準噶爾所自始,七傳至和多和沁,號巴圖爾琿臺吉,駐牧阿爾臺。子十一:曰車臣,為其弟噶爾丹所殺;曰卓特巴巴圖爾,徙牧青海,為扎薩克郡王色布騰扎勒一旗祖,色布騰扎勒再傳,嗣絕;曰班達哩,孫車木伯勒,襲色布騰扎勒所遺扎薩克;曰卓哩克圖和碩齊,為扎薩克輔國公阿喇布坦一旗祖;曰溫春,子丹濟拉,以來歸,封扎薩克輔國公,附喀爾喀賽因諾顏部;曰僧格,子策妄阿喇布坦,號琿臺吉,再傳,為其本族達瓦齊所篡,嗣絕;曰噶爾丹,以掠喀爾喀,為大軍所敗,竄死;曰布木,號額爾德尼臺吉,其曾孫即達瓦齊,大軍平其部,俘至京,尋釋之,封親王,不列籓部;曰多爾濟扎布,為喀爾喀土謝圖汗察琿多爾濟所戕;曰朋素克達什,孫噶勒藏多爾濟,以來歸,封綽囉斯汗,尋叛,為從子扎納噶爾布所殺;曰噶爾瑪,三傳至三濟札布,以來歸,授侍衛,隸內蒙古正黃旗。

和多和沁弟曰墨爾根岱青,子二:長丹津,號噶爾瑪岱青和碩齊,孫阿喇布坦,以來歸,封扎薩克郡王,附喀爾喀賽因諾顏部;次阿海,三傳至達什達瓦,嗣絕,妻車臣哈屯攜眾來歸,編佐領,置直隸承德府境,不設扎薩克。

輝特設扎薩克一,其始祖曰納木占,再傳至卓哩克圖和碩齊,為扎薩克輔國公貢格一旗祖。

厄魯特諸扎薩克外,設喀爾喀公中扎薩克一。別有察罕諾捫汗,授扎薩克喇嘛,轄四佐領,自為一旗,不列諸扎薩克盟。

天聰初,蒙古諸部內附,厄魯特猶私與明市,上以遠,弗之禁。崇德二年,顧實汗遣使通貢,閱歲乃至。七年,偕達賴喇嘛等奉表貢。八年,遣使存問達賴喇嘛。以顧實汗擊敗唐古特藏巴汗,敕曰:「有敗道違法而行者,聞爾已懲治之。自古帝王致治,法教未嘗斷絕。今遣使敦禮高賢,爾其知之!」並賜甲胄。使未至,顧實汗請發幣使延達賴喇嘛,允之。順治二年,顧實汗子達賴巴圖爾貢馬至,奏:「聞天使召聖僧,臣等自當遵奉。」三年,以厄魯特臺吉等入甘肅境要糧賞,詔所司議剿撫。會顧實汗奉表貢,賜甲胄弓矢,命轄諸厄魯特。嗣間歲輒遣使至,厄魯特臺吉等附名以達。

和碩特族曰都爾格齊諾顏,曰色棱哈坦巴圖爾,曰鄂齊爾汗,曰鄂齊爾圖汗,曰阿巴賴諾顏,曰達賴烏巴什諾顏,曰伊拉古克三班第達呼圖克圖,曰額爾德尼琿臺吉,曰阿哩祿克三陀音,曰噶爾第巴臺吉,曰瑪賴臺吉,曰諾木齊臺吉,曰綽克圖臺吉。土爾扈特族曰羅卜藏諾顏,曰楚琥爾岱青,曰博第蘇克。準噶爾族曰巴圖爾琿臺吉,曰墨爾根岱青,曰杜喇勒和碩齊,曰楚琥爾烏巴什,曰羅卜藏呼圖克圖。以顧實汗為之首。

五年,甘肅巡撫王世功奏青海蒙古駐西寧,需索供應,請定貢使入關額,餘駐關外給口糧,許之。九年,顧實汗導達賴喇嘛入覲,先奉表聞,並貢駝馬方物。十年,詔封遵文行義敏慧顧實汗,賜金冊印。十三年,顧實汗卒。上念其忠勤修貢,遣官致祭。

會青海屬復為邊患,諭顧實汗子車臣岱青及達賴巴圖爾等曰:「分疆別界,向有定例。邇來爾等率番眾掠內地,抗官兵,守臣奏報二十餘次,屢諭不悛。今特遣官赴甘肅、西寧等處勘狀。或爾等親至,或遣宰桑來質,誣妄之罪,各有攸歸。番眾等舊納貢蒙古者聽爾轄,儻系前明所屬,應仍歸中國。至漢人蒙古交界,與市易隘口,務宜詳加察覈,分定耕牧,毋得越境妄行。」十五年,復諭車臣岱青曰:「前因爾等頻犯內地,遣官往勘。據奏爾等入邊,向番取貢,輒肆攘奪。咎自難辭,朕悉宥爾前愆。但中外本無異視,疆圉自有大防。爾等向屬番取貢,酌定人數,路由正口,遣頭目稟告守臣,方準入邊。至市易定所,應從西寧鎮海堡、川北、洪水等口出入,毋得任意取道。如或不悛,國憲具在,朕不爾貸也。」

康熙四年,甘肅提督張勇奏蒙古番眾游牧莊浪諸境,情形叵測,請增甘肅、西寧駐防兵。先是青海蒙古戀西喇塔拉水草饒,乞駐牧。張勇以其地為甘肅要隘,不容偪處,往責之,謝罪去。因設永固營,聯築八寨。至是蒙古等復相繼徙近邊。上以漸不可啟,詔如張勇請。五年,勇復奏:「青海雖通西藏,不過荒徼絕塞,朝廷曲示招徠,準開市,自應鈐束部落,各安邊境。乃邇來蜂屯祁連山,縱牧內地大草灘。曾遣諭徙,復抗拒定羌廟,官軍敗之,猶不悛,聲言糾眾分入河州、臨洮、鞏昌、西寧、涼州諸地。請設兵備。」詔嚴防禦,仍善撫以柔其心。勇等乃自扁都口、西水關至嘉峪關,固築邊墻。六年,川陜總督盧崇峻奏青海諸頭目偵於八月將入寇,因赴莊浪所備之,遣總兵孫思克屯南山隘,相形勢固守。達賴喇嘛尋檄厄魯特諸臺吉毋擾內地,駐牧黃城兒、大草灘。蒙古悉徙去,獻駝馬羊等服罪,請撤駐防兵,允之。

十四年,西寧諸鎮兵屯河東剿叛賊王輔臣,青海蒙古乘隙犯河西。永固營副將陳達御之,陣歿。孫思克屯涼州,宣示朝廷恩威,各引罪出塞。會達賴喇嘛使至,命傳諭達賴巴圖爾等戢部眾,勿為邊患。

十六年,準噶爾臺吉噶爾丹襲殺駐牧西套之鄂齊爾圖汗。青海和碩特諸臺吉懼,挈廬幕數千避居大草灘,撫遠大將軍圖海等飭歸故巢。十七年,西套諸臺吉偵噶爾丹將侵青海,遣使告和碩特臺吉達賴巴圖爾等為防禦計。上聞之,諭張勇曰:「噶爾丹侵青海,如遠從達布素圖瀚海而往,則聽之。若欲經大草灘,則令堅立信約,勿擾內地。」尋噶爾丹以從者異志,且距青海遠,行十一日撤兵歸。遺書張勇,詭稱其祖多克辛諾顏偕顧實汗取青海,和碩特族獨據之,欲往索,以將軍所轄地,故不果。既而懼和碩特諸臺吉襲己,密遣使議婚,以女布木妻博碩克圖濟農子根特爾。張勇諜得狀,奏噶爾丹仇青海、蒙古,議婚後,恐復往侵,甘肅當往來沖,請增防,上報可。有巴圖爾額爾克濟農和囉理者,巴延阿布該阿玉什子也,駐牧西套,以避噶爾丹侵,乞假內地赴青海,許之。會噶爾丹屬額爾德尼和碩齊潛掠烏喇特戶畜,青海墨爾根臺吉聞之,遣使詰歸所掠。喀爾喀臺吉畢瑪里吉哩諦亦以厄魯特掠所部,陰偵之,告額爾德尼和碩齊、和囉理及青海臺吉茂濟喇克等。游牧額濟訥河,則未知其為何厄魯特也。十八年,遣使諭達賴巴圖爾等曰:「爾墨爾根臺吉將被盜劫掠人察護解送,朕甚嘉之。夫勸善懲惡者,國之法也。邇聞厄魯特眾棲處額濟訥河,爾達賴巴圖爾及墨爾根臺吉,其照汝例,嚴加治罪。」使至,稱茂濟喇克、和囉理皆無掠烏喇特事。額爾德尼和碩齊為準噶爾屬,已徙牧去。詔檄噶爾丹收補之,不從。

二十九年,大軍敗噶爾丹於烏蘭布通,青海諸臺吉附達賴喇嘛表上尊號,詔不允。三十年,甘肅提督孫思克奏:「噶爾丹巢距邊月餘,從子策妄阿喇布坦雖交惡,恐復合,有侵青海舉,道必經嘉峪關外。肅州密邇青海,請設兵三千為備。」上報可。三十二年,昭武將軍郎坦奏稱青海諸臺吉私與噶爾丹通問,請屯兵哈密,絕往來蹤。上以噶爾丹自烏蘭布通敗遁後,乏邊警,且青海諸臺吉素恭順,寢議。噶爾丹尋屯牧巴顏烏蘭,偪內汛,詔西寧設戍兵。唐古特部第巴陰比噶爾丹,詭為達賴喇嘛奏稱青海諸臺吉無異志,請撤戍。諭曰:「此為征剿噶爾丹計,非防青海諸臺吉也。」會議剿噶爾丹,詔檄青海眾勿驚懼。

三十五年,上親征噶爾丹,敗之,獲青海通噶爾丹使。以博碩克圖濟農及薩楚墨爾根臺吉為所部長,遣使齎敕諭曰:「爾青海厄魯特尊崇達賴喇嘛法教,敬事本朝,聘問貢獻,恭順有年,朕亦頻加恩賚。乃噶爾丹違達賴喇嘛法教,不遵朕旨,朕統軍至圖拉,剿而滅之。博碩克圖濟農等遣往噶爾丹使,為朕所擒,俱言達賴喇嘛脫緇已久,第巴匿之,且噶爾丹詭言青海諸臺吉謀與彼同犯中國。今噶爾丹亡命西走,青海諸臺吉如欲仍前修睦,其各防守邊界,遇噶爾丹即行擒解。若知而故縱,此後永仇絕之。」我使至察罕托羅海宣諭善巴陵堪布,蓋達賴喇嘛遣理青海蒙古務者也。善巴陵堪布召青海諸臺吉集盟壇言曰:「噶爾丹殺鄂齊爾圖汗,我等與仇。但素奉達賴喇嘛言,應遣議。」時達賴喇嘛示寂久,唐古特達賴汗尋約和碩特八臺吉遣使慶捷。達賴汗即鄂齊爾圖汗子也,世長唐古特。鄂齊爾圖汗弟自袞布察琿無嗣外,餘八人皆居青海,故其裔稱和碩特八臺吉。

三十六年二月,上視師寧夏,詔額駙阿喇布坦、都統都思噶爾、巴林臺吉德木楚克、西寧喇嘛商南多爾濟等攜青海諸臺吉使及賞物往招撫之。復以哈密達爾漢伯克額貝都拉內附,詔青海厄魯特勿擾哈密境。三月,阿喇布坦等至察罕托羅海,察罕諾捫汗迎告曰:「皇上令青海眾得享安樂,永受恩澤,何幸如之!」時顧實汗子惟達什巴圖爾存,阿喇布坦等宣諭之。達什巴圖爾議遣博碩克圖濟農及額爾德尼臺吉代入覲。阿喇布坦等語曰:「皇上駕臨寧夏,爾當率眾往朝,毋自誤!」達什巴圖爾偕察罕諾捫汗、善巴陵堪布及唐古特達賴汗子拉藏等檄諸臺吉議,欲四月起行。達爾寺垂臧呼圖克圖、溫都遜寺達賴綽爾濟喇嘛及囊素通事等咸請從,私向使問獅象狀,且相謂曰:「我等往朝,殆必以所未見文物相示。」閏三月,阿喇布坦、德木楚克自青海歸。議諸臺吉至,若露處,未協朝典,應令秋後入覲京師。詔如議,命都思噶爾、商南多爾濟留駐鎮海堡俟之。扈蹕諸臣奏:「青海厄魯特與準噶爾同部,聞噶爾丹敗竄,咸驚懼。皇上定策安集所部,身至如歸,誠非常舉。請行慶賀禮。」諭曰:「青海職貢有年,來朝亦常事耳。可勿賀。」諸臣固請,因奉表賀曰:「青海向雖修貢,未隸臣屬。今舉部歸誠,噶爾丹益無竄路。皇上安內攘外之心,自此允愜矣。」四月,諭留糧騎及羊九千餘於達希圖海,俟青海眾至給之。十一月,達什巴圖爾偕諸臺吉入覲,諭曰:「朕非威懾爾等前來,不過欲令天下生靈各得其所。朕何物不備,朕之尊不在爾等來否,所望爾等各遂安全,副朕好生至意耳。」詔所從諸宰桑咸列坐預宴,以御用冠服、朝珠賜達什巴圖爾,賞諸臺吉鞍馬、銀幣有差。復傳諭曰:「爾等自祖父來,歲修職貢,故特優錫,以寵爾歸。」十二月,上大閱玉泉山,達什巴圖爾等扈駕往觀,戰慄失色,奏:「天朝兵威若此,何敵不克?」三十七年正月,詔封達什巴圖爾為和碩親王,諸臺吉授貝勒、貝子、公等爵有差。

先是噶爾丹詭與青海★L5,實謀往侵,懼大軍討,乃寢。第巴以策妄阿喇布坦不附噶爾丹,陰間之,偽為達賴喇嘛疏,奏策妄阿喇布坦將侵青海及唐古特,上斥其妄。會噶爾丹使至,諭曰:「青海諸臺吉奉貢久,儻噶爾丹屬犯青海,朕必往討之。」至是噶爾丹就滅,策妄阿喇布坦憾達什巴圖爾等內附,詭請大軍征青海,討前助噶爾丹罪。諭曰:「青海諸臺吉聞朕出師寧夏,遠徙游牧。嗣噶爾丹平定,親來稱慶。伊等並無過端,豈肯遽為加兵?朕統馭天下,惟原宇內群生咸獲安堵,豈有使爾等構釁之理?」二月,上幸五臺山,詔達什巴圖爾等從。將旋蹕,召覲行幄,溫諭遣歸,給駝馬。三十九年,策妄阿喇布坦聲言兵擊第巴,遣使赴青海陰覘強弱。上以策妄阿喇布坦將不靖,詔廷臣留意漢趙充國所議五事,為防禦計。四十二年,上幸西安府,達什巴圖爾等來朝,扈駕閱駐防兵,奏:「禁卒精練,天下無敵。外省軍容復如是。億萬年可永享升平。」賜宴遣歸。

五十四年,策妄阿喇布坦遣兵掠哈密。上以鄰青海左翼牧,詔兵備之,準噶爾敗遁。初,達賴汗子拉藏偕青海諸臺吉定議內附,尋襲唐古特汗,以第巴私立偽達賴喇嘛,襲殺之,而自立博克達之伊什扎穆蘇為達賴喇嘛瑚畢勒罕。青海貝勒察罕丹津等訐其偽,奏里塘之羅卜藏噶勒藏嘉穆錯為真達賴喇嘛瑚畢勒罕,詔內閣學士拉都琥往驗。尋遣侍衛阿齊圖召青海兩翼議徙里塘達賴喇嘛瑚畢勒罕以弭爭端。貝勒色布騰扎勒、阿喇布坦鄂木布、朋素克旺扎勒,臺吉達顏、蘇爾扎等僉請徙。察罕丹津不從,將偕達什巴圖爾子羅卜藏丹津盟,率兵攻異己者。阿齊圖疏至,王大臣等奏察罕丹津若先攻諸部,色布騰扎勒等來奔,應置邊內。察罕丹津牧距松潘僅四五日程,請備兵待。詔西寧、四川松潘諸路設兵備之。

五十五年,察罕丹津畏罪,徙里塘達賴喇嘛瑚畢勒罕至西寧宗喀巴寺。阿齊圖奏請集諸臺吉定盟,以羅卜藏丹津、察罕丹津、達顏等領右翼,額爾德尼額爾克托克托鼐、阿喇布坦鄂木布等領左翼,令永睦,允之。會噶爾丹由沙拉襲青海,掠臺吉羅布藏丹濟卜等牧畜,復謀盜噶斯口官軍駝馬。諭曰:「準噶爾偵噶斯口兵勢稍弱,潛來侵擾青海,不可不嚴籌之。著西安兵會青海左翼,四川督標兵會青海右翼,協力防禦。」

五十六年,遣使赴青海測分野。未幾,靖逆將軍富寧安諜策妄阿喇布坦遣兵赴唐古特,馳疏聞。上以里塘達賴喇嘛瑚畢勒罕事初定,拉藏汗或陰導準噶爾侵青海,詔理籓院尚書赫壽諭拉藏汗勿得與察罕丹津、羅卜藏丹津等構兵。復諭遣侍衛色楞等赴青海,曰:「準噶爾若侵拉藏汗,爾即與青海諸臺吉等定議協剿,務令絕無猜忌,不至滋變方善。或拉藏汗導準噶爾侵青海,爾即諭察罕丹津等曰:『策妄阿喇布坦屢抗大軍,今拉藏汗與同謀,是顯為仇敵也。國家始終仁愛,保護顧實汗子孫,爾等正當奮志報效而行。』」尋察罕丹津等以準噶爾侵拉藏汗告,諭內大臣策旺諾爾布、西安將軍額倫特等分屯青海要地。

五十七年,拉藏汗乞援疏至,詔色楞等會青海王、臺吉議進兵。察罕丹津諜拉藏汗被戕,謀誘準噶爾至青海迎擊之。準噶爾懼,不至。先是哈密伯克額貝都拉獻西吉木、達里圖、西喇郭勒地,詔設赤金、靖逆二衛及柳溝所,聽兵民耕牧。五十八年,以其地錯青海左翼牧,遣官偕貝子阿喇布坦、臺吉阿爾薩蘭等勘定界。阿喇布坦等曰:「青海眾荷厚恩,何惜隙地?可耕者聽給兵民,留我等牧地足矣!」因集所屬宰桑等畫地標識,議勿私越。時撫遠大將軍固山貝子允統兵駐西寧,請自索諾木至柴達木路設站五,站置青海兵十,別令左、右翼兵各三百屯近軍地,防準噶爾賊,從之。允復遵旨集兩翼王、臺吉,以上意宣諭曰:「唐古特部達賴喇嘛、班禪喇嘛法教,原系爾祖顧實汗所設。今準噶爾戕拉藏汗,離散番眾。爾等前稱里塘羅卜藏噶勒藏嘉穆錯為真達賴喇嘛瑚畢勒罕,原置禪榻,廣施法教,今唐古特民人及阿木島喇嘛如爾言。皇上為安藏計,遣大兵送往唐古特,爾等宜率所屬兵或萬或五六千從往,其定議具奏。」兩翼王、臺吉等僉稱原聽命。五十九年,所部兵從大軍敗準噶爾於札卜克河、齊諾郭勒、綽瑪喇等處,因護達賴喇嘛入藏。捷聞,詔留兵二千屯青海偵防準噶爾。

雍正元年,諭曰:「自西陲用兵,青海王以下、臺吉以上各著勞績。皇考曾降旨俟凱旋日計功,今青海王、臺吉等歷年效績,應各酌加封賞。其率兵進藏,至駐防噶斯、柴達木等眾,應令各處將軍分別加賞。」是年羅卜藏丹津叛,命大軍往討,越歲而定。羅卜藏丹津初襲其父達什巴圖爾親王爵,從大軍入藏,歸,覬為唐古特長,陰約策妄阿喇布坦援己,復誘青海臺吉等盟察罕托羅海,令如所部故號,不得復稱王、貝勒、貝子、公等爵,而自號達賴琿臺吉以統之。郡王額爾德尼額爾克托克托鼐不從,偕鎮國公噶爾丹達什來奔。上以和碩特族自相殘,不忍遽加兵,詔撫遠大將軍貝子延信善慰額爾德尼額爾克托克托鼐。時兵部左侍郎常壽駐西寧理青海務,命傳諭羅卜藏丹津罷兵,不從則懲治之。羅卜藏丹津詭言親王察罕丹津、郡王額爾德尼額爾克托克托鼐謀據唐古特,諸臺吉不服,將率兵與決勝負。蓋以察罕丹津、額爾德尼額爾克托克托鼐首不附己,欲誣以罪,因脅諸臺吉奉己,如鄂齊爾汗駐唐古特以遙制青海也。

察罕丹津為羅卜藏丹津所偪,繼額爾德尼額爾克托克托鼐挈眾至。敕川陜總督年羹堯曰:「羅卜藏丹津自其祖顧實汗敬謹恭順,達什巴圖爾慕化來歸,晉封親王,復令其子羅卜藏丹津襲封,自宜仰體寵眷,敬奉法紀。乃妄逞強梁,骨肉相仇,欺凌親王察汗丹津、郡王額爾德尼額爾克托克托鼐等,恣行倡亂。朕甫聞其事,遣使往諭,令伊講和修睦,式好無尤。乃肆意稱兵,侵襲察罕丹津、額爾德尼額爾克托克托鼐,以致投入內境。是其深負朕恩,悖逆天常,擾害生靈,誅戮不可少緩。朕欲大張天威,特命爾為撫遠大將軍,統領大兵,往聲羅卜藏丹津罪。如敢抗拒,即行剿滅。其黨有懼羅卜藏丹津勢,暫為脅從者,果悔罪來歸,即行寬宥。有能擒斬羅卜藏丹津者,分別具奏。有情急來歸者,加意撫恤。其不抗拒者,毋加殺戮。」羅卜藏丹津詭罷兵,誘常壽至察罕托羅海,留之,遣叛黨分掠西寧諸路,煽賊番等為應。副將軍阿喇納自吐魯番馳赴噶斯,斷由穆魯烏蘇往藏路;副將王嵩、參將孫繼宗等擊賊黨於布隆吉爾及鎮海堡、申中堡、北川、新城等處。四川提督岳鍾琪以雜穀土司等兵剿歸德堡外上寺東策卜、下寺東策卜及南川口外郭密諸番,復檄前鋒統領蘇丹等協剿,所至告捷。羅卜藏丹津懼,送常壽歸,請罪。諭年羹堯曰:「伊乃深負國恩、與大軍對敵之叛賊,國法斷不可宥。不得因伊曾封王爵,稍存疑慮。其與羅卜藏丹津同謀之王、貝勒、貝子、公等,既經背叛,即宜削爵。伊等或來歸順,或被擒獲,不必更論封爵,但視行事輕重,可寬宥者從寬,應治罪者治罪。」

二年,詔以岳鍾琪為奮威將軍,參贊軍務。鍾琪奉命進剿,偵從賊之巴爾珠爾阿喇布坦自烏蘭博爾克遁,尾擊之,至伊克喀爾吉,擒其黨阿喇布坦鄂木布。遣西寧總兵黃喜林由西爾哈羅色赴柴達木,斷噶斯路。偵羅卜藏丹津走烏蘭木和爾,鍾琪復分兵馳擊,擒其母阿爾泰,俘戶畜無算。羅卜藏丹津偕賊黨分道竄。侍衛達鼐等擒丹津琿臺吉於華海子,阿布濟車臣臺吉於布哈色布蘇,吹喇克諾木齊、扎什敦多卜等於烏拉克,羅卜藏丹津走準噶爾。逆黨悉檻送京師,詔行獻俘禮。

是役也,以兵拒羅卜藏丹津者,親王察汗丹津、郡王額爾德尼額爾克托克托鼐也。不從羅卜藏丹津逆者,郡王色布騰扎勒,臺吉阿喇布坦、噶勒丹岱青諾爾布、巴勒珠爾、察罕喇布坦、旺舒克喇布坦也。為羅卜藏丹津脅從者,貝勒朋素克旺扎勒、輔國公車凌、臺吉諾爾布也。始附羅卜藏丹津、尋以悔罪宥者,貝勒羅卜藏察罕、車凌敦多布,貝子濟克濟扎布、拉扎布,臺吉袞布、色布騰、納罕伊什也。其附羅卜藏丹津者,首惡曰吹喇克諾木齊、阿喇布坦鄂木布、藏巴扎木,從黨曰巴勒珠爾阿喇布坦、扎什敦多布、格勒克阿喇布坦、巴蘇泰及察罕丹津從子塔爾寺喇嘛堪布諾捫汗也。有中甸者,隸雲南麗江府,羅卜藏丹津給偽劄令附己。大軍至,率戶三千餘請降。洮、岷界外諸番舊為青海屬,悉就撫,其不順者剿誅之。阿岡、多卜藏瑪嘉、鐵布納珠公寺、朝天堂、卓子山、棋子山、先密寺、興馬寺、阿羅、西脫巴、上篤爾素華藏、上扎爾的諸番眾以次底定,青海患始靖。禦制平定青海文,立石太學。

王大臣等遵旨議善後事宜,奏青海王、臺吉等應論功罪定賞罰,游牧地令各分界,如內扎薩克例。百戶置佐領一,不及百戶者為半佐領,以扎薩克領之。設協理臺吉及協領、副協領、參領各一,每參領設佐領、驍騎校各一。歲會盟,令奏選盟長,勿私推。貢期自明年始分三班,九年一周,自備駝馬,由邊入京。市易以四仲月集西寧西川邊外納喇薩喇地,官兵督視,有擅入邊墻者治罪。又羅卜藏丹津之吹宰桑及察罕丹津從子丹衷之宰桑色布騰達什等率眾降,請各授千、百戶等官。又喀爾喀居青海者,勿復隸和碩特旗,令別設扎薩克,土爾扈特及準噶爾、輝特如之。至西番部眾,凡陜西所屬甘州、涼州、莊浪、西寧、河州,四川所屬松潘、打箭爐、里塘,雲南所屬中甸等處,或為喇嘛耕地,或納租青海,但知有蒙古,不知有衛營伍諸官。今番眾悉歸化,應擇給土司千百戶、巡檢等職,令附近道及衛所轄。又青海及巴爾喀木、藏、衛舊稱唐古特四大部,顧實汗侵據之。以青海地廣可牧畜,巴爾喀木糧富,令子孫游牧青海,而巴爾喀木納其賦。藏、衛二地,舊給達賴喇嘛、班禪喇嘛,今以青海叛,取其地,應令四川、雲南諸官管理。又達賴喇嘛遣人赴市打箭爐,馱裝經察木多、乍雅、里塘、巴塘,向喇嘛等索銀有差,名曰鞍租,至打箭爐納稅。請飭達賴喇嘛勿收鞍租,打箭爐免取稅,歲給達賴喇嘛茶五千斤,班禪喇嘛半之。又西寧各寺喇嘛多者數千,少者以五六百,易藏奸,前羅卜藏丹津叛,喇嘛率番眾抗大兵。請於塔爾寺喇嘛選老成者三百給印照,嗣後歲察二次,廟舍不得過二百,喇嘛多者百餘,少者十餘。番民糧賦,令地方官管理,度各寺歲用給之。又陜西邊外河州、西寧、蘭州、中衛、寧夏、榆林、莊浪、甘州等處,水草豐美,林麓茂密,蒙古諸部戀牧大草灘及昌寧湖。請於西寧北川邊外上下白塔等處,自巴爾托海至扁都口築城堡,令蒙古等勿妄據。又肅州西洮賚河、常瑪爾、鄂敦塔拉等處,應募民墾膏腴地,庶漸致富饒。至寧夏險要,無過阿拉善。顧實汗裔舊游牧山後,今或徙至山前。請令阿拉善扎薩克郡王額駙阿寶飭所屬歸阿拉善山後,其山前營盤水、長流水等處,悉為內地。又甘州、西寧界各設營汛,令蒙古等不敢覬覦。又巴爾喀木等部眾,自魯隆宗東察木多、乍雅外,諸番目悉給印照,視內地土司例。又青海屬左格諸番,請徙內地。阿巴土司頭目墨丹桂等從剿有功,請給安撫司銜,不隸青海轄。又西寧邊內可耕地,請發直隸、山西、山東、河南、陜西五省遣犯,能種地者,官給牛具籽種,三年後起科如例。又甘州喀黃番,應招撫為青海籓籬。青海諸部,令各守牧地,不得強據,妄掠商賈。察汗諾捫汗喇嘛廟毋得私聚議事。遣官齎敕往,不論秩崇卑,王公以下跪迎,有背貳者必懲。上從其議。

三年,詔以博羅充克克地給阿拉善郡王阿寶居之,鈐青海族屬,越七載始撤歸。是年,青海和碩特、土爾扈特、準噶爾、輝特、喀爾喀及察罕諾捫汗各授扎薩克,鑄「總理青海蒙古番子事務」關防,遣大臣齎鎮其地,轄所部扎薩克。岳鍾琪復奏:「親王察罕丹津、鎮國公拉扎布等游牧河東,地近河州、松潘各路。前議市納喇薩喇地,地狹,恐不給蒙古需。請改市河州及松潘。河州定於土門關附近雙城堡,松潘定於黃勝關之西河口,二地並有城屋,水草美,互市可久。又郡王額爾德尼額爾克托克托鼐、色布騰扎勒等游牧河西,地近西寧,請改市西寧口外丹噶爾寺。至蒙古歲資牲畜,請每年六月後聽不時當易,庶蒙古商眾獲利益。」允之。

六年,唐古特部噶卜倫阿爾布巴、隆布鼐、扎爾鼐等叛,擾唐古特,謀通準噶爾,大軍誅之。七年,上以準噶爾不靖,必擾青海及唐古特,因決策進討。王大臣等議噶斯為準噶爾通青海及唐古特要隘,請選青海扎薩克兵千五百分屯噶斯及柴達木、得卜特爾、察罕烏蘇諸路,允之。會噶爾丹策凌遣使告將獻羅卜藏丹津,聞大軍就道,懼,仍攜歸。八年,詔暫緩進兵,諭噶爾丹策凌速獻羅卜藏丹津,當宥罪。復命青海扎薩克備兵游牧聽調。準噶爾尋襲科舍圖汛,諭青海兵速赴噶斯,準噶爾遁。

九年,遣二等侍衛殷扎納傳諭左右翼扎薩克選兵萬屯青海適中地,官兵皆賞裝。復命所部採買牲畜,勿滋擾。扎薩克公諾爾布、拉扎布等尋徙牧,叛。詔曰:「朕因準噶爾賊乘西路軍不備,盜駝馬,因念青海各扎薩克人眾恐招逆賊侵害,諭令派兵防護。其採買馬羊者,原欲使伊等所有牧畜得變價值,可獲利益,並非需此區區助也。朕曾諭殷扎納,一切派兵採買,聽蒙古便,不可絲毫勉強。並慮王、臺吉等科派所屬,諭令嚴行禁約,豈肯令遣往人逼迫蒙古從事乎?今拉扎布等無故他徙,或殷扎納不能宣揚朕諭,使眾心共曉,而採買馬羊又不聽從其便,以致拉扎布等心懷疑畏,漸避差徭。特頒旨諭拉扎布等,令其速歸本處,準噶爾賊或由喀喇沙爾前赴噶斯,潛行騷擾,或增人眾窺伺青海。所部蒙古兵丁尚未齊集,器械亦未周備,難望捍禦賊鋒,亦令官兵善為保護。」會拉扎布等不奉命,諸扎薩克擒獻。復集兵七千為備,軍械及馬不給。上憫之,諭廷臣曰:「朕所以聚此兵者,特為保全伊等家口及游牧計,非為征伐調遣用也。今聞其生計情形,朕心深為惻然。俟從容料理,必有加恩之處。所聚七千,著選派三千,照前所降恩旨,官員賞給本年俸銀,兵丁賞銀五兩。戍卒駐防日久,貲斧維艱,著給茶幣等項,及每月所食青稞。遣歸兵四千名,官員等著給三月俸銀,兵丁等著賞銀三兩,令各回游牧。準噶爾賊或潛擾青海,朕意欲將伊等預行從容遷徙,令賊由遠路來一無所得,不待戰而力盡。我官兵與賊交戰時,青海三千兵但追襲賊後,量力驅賊馬匹,所得即賞之,仍計馬匹多寡,加恩議敘。」

十年,以喀爾喀敗準噶爾於克爾森齊老及額爾德尼昭,諭青海扎薩克等曰:「喀爾喀奮勇剿賊,爾等何獨不能?各宜鼓舞振興,踴躍效命。賊眾侵擾青海,止有噶斯一路,爾等須防守隘口,儻準噶爾前來,務期協力追殺,悉行剿除。」十三年,詔撤駐防大軍,所部仍選兵二千屯得卜特爾、伊克柴達木等汛,以臺吉達瑪璘色布騰、色特爾布木領之。

乾隆十一年,辦理青海事務副都統眾佛保遵旨宣諭諸扎薩克歲防汛,議以郡王額爾德尼額爾克托克托鼐之長子索諾木丹津及扎薩克臺吉袞布喇布坦、色特爾布木、多爾濟色布騰、薩喇等防得卜特爾汛,以郡王袞楚克達什、車凌喇布坦,貝子丹巴,輔國公納木扎勒車凌,扎薩克臺吉達瑪璘色布騰等防伊克柴達木汛。十人分為五班,三年一察軍械。十二年,以準噶爾使赴藏煎茶,道噶斯,復議自伊克柴達木、得卜特爾外,設汛哈濟爾、察汗烏蘇。

二十年,大軍征達瓦齊,抵伊犁,羅卜藏丹津就擒。諭曰:「羅卜藏丹津負恩背叛,逃往準噶爾,偷生三十餘載。今兩路大軍至,伊無路奔竄,仍就擒獲,實足以彰國憲而快人心。」羅卜藏丹津俘至,告祭太廟社稷,行獻俘禮,上御午門樓受之。以世宗憲皇帝有羅卜藏丹津至仍宥罪之旨,詔免死。子巴朗及察罕額布根授藍翎侍衛,其戚屬處伊犁者,詔勿內徙。

二十三年,大軍剿瑪哈沁,偵沙拉斯瑪呼斯賊竄呼爾塔克羅卜諾爾。以地近噶斯,通青海,詔副都統濟福赴西寧宣諭所部集兵千為備,復遣識噶斯道者偵賊蹤。既而所部兵集扎噶蘇臺,詔歸牧聽調,勿遽就道。濟福遵旨諭之,請遣近牧者歸,仍量留遠道兵屯烏圖,備不虞。上鑒其誠,詔酌賞遣歸兵。久之,噶斯無賊蹤,乃撤烏圖兵還。二十四年,陜甘總督楊應琚奏:「青海得卜特爾、伊克柴達木等處設汛屯兵,為防準噶爾計。今準噶爾及回部悉底定,請撤青海駐防兵。」從之。先是阿睦爾撒納叛,大軍分道進剿,所部購馬二千、駝四百,送巴里坤軍。詔予值,斃者半。至是復輸馬七百餘、駝三百二十餘,請償斃數,詔仍如值給。

二十七年,以所部翁扎薩克請給羅卜藏丹津舊牧地,楊應琚遵旨往勘,奏:「洮賚河等處系西寧、肅州鎮標馬廠及番族牧地,不便撥給。西喇郭勒及西爾噶拉金東西五百餘里,南北三十餘里,地曠,且距扎薩克等游牧近,請給。其西爾噶拉金逾河即產礦山場,久封禁,請飭扎薩克等就近守視。」詔以西喇郭勒給之,西爾噶拉金河東聽駐牧,河西鉛礦,勿得越界私採。是年復設西寧辦事大臣,轄蒙古、番子事務。

所部扎薩克,自察罕諾們汗外,旗二十有九。爵三十:扎薩克多羅郡王三,一由親王降襲,一由貝勒晉襲;扎薩克多羅貝勒二,一由郡王降襲;扎薩克固山貝子二,一由輔國公晉襲;扎薩克輔國公四,一由鎮國公降襲;扎薩克一等臺吉十六,一由貝勒降襲,二由貝子降襲,一由輔國公降襲;附固山貝子一;公中扎薩克一等臺吉二。

二十九年十一月,命青海各扎薩克每年輪派兵丁設卡防果洛克。三十年九月,以果洛克肆行劫殺,諭青海各扎薩克協力剿之。三十一年六月,青海王、貝子、扎薩克等請留辦事大臣七十五,不許。七月,諭四川禁果洛克土司番人越境掠竊青海蒙古牲畜。九月,移青海附近果洛克之各扎薩克駐牧地方,添設卡兵。十月,以青海扎薩克羅布藏色布騰等游牧為果洛克番賊劫掠,革之。四十年九月,青海扎薩克公禮塔爾以出獵被番賊戕害,諭青海辦事大臣福德查辦。

五十一年九月,禁青海喇嘛不領路引私自赴藏。分青海納罕達爾濟等三旗兵,羅卜藏丹津、袞楚克二旗兵駐奎屯、西哩克等處,設果洛克防卡。五十六年九月,以青海郡王納漢達爾濟屬人勾引番子戕扎薩克沙喇布提,嚴飭之,並諭各於境內游牧,勿容匿番族。十二月,以大軍進藏征廓爾喀,予親往巡查青海新設臺站之貝子羅布藏色布騰貝勒銜、鎮國公達瑪林貝子銜,仍賚預備駝馬之王、公、扎薩克等有差。五十八年,循化等處番族占居蒙古地界,命辦事大臣特克慎以兵驅逐之。

嘉慶四年九月,青海郡王那罕多爾濟等呈番子搶擄六千餘戶,傷害男女二千餘人。詔責辦事大臣奎舒諱匿,革逮,以臺斐廕代之,命廣厚赴西寧查辦。十月,以松筠奏命青海蒙古王公撫綏所屬,毋致勾引番子搶劫。五年六月,青海貝勒克莫特伊什等番子交出牲畜較少,諭臺斐廕下部嚴議。九月,臺斐廕以不準青海蒙古報被番子搶劫免,以臺布為西寧辦事大臣。六年十月,以勘定青海卡倫,禁蒙古擅出,番子擅入。十二月,臺布奏循化番子渡河搶劫。諭飭撥兵防護。

七年二月,臺布令西寧鎮總兵保青署河州鎮總兵,福寧阿撥兵駐守黃河冰橋,防護蒙旗果爾的等,番族均斂跡。諭臺布責成蒙古設法自衛。八月,臺布奏番子格爾吉族縛獻犯事賊番,撤坐卡官兵。四月,以循化、貴德番子擾青海蒙古各旗,劫執貝子齊默特丹巴,諭辦事大臣都爾嘉嚴行查治。五月,諭都爾嘉撫恤青海被擾蒙古,命貢楚克扎布會同都爾嘉查辦番案。六月,都爾嘉奏捕獲劫殺青海貝子夫人兇番齊克他勒,誅之。命陜甘總督惠齡赴西寧查辦野番,撫恤青海被擾蒙古,每口加給官茶一分。七月,命惠寧等妥酌防番卡倫章程。貢楚克扎布等渡河驅逐野番。八月,貢楚克扎布奏野番退出占住蒙古地方,移回番境。命曉諭番目尖木贊交還贓畜,縛獻賊目,並飭定善後章程。九年九月,辦事大臣玉寧復以青海蒙古被番子搶劫之案甚多入告。

十年六月,以青海郡王納罕多爾濟呈蒙古窮困,諭玉寧遇水旱之災,酌量賑濟。七月,諭玉寧飭郡王納罕多爾濟等勿令商人私挖木植、大黃。九月,玉寧奏青海番子尖木贊等占據諾們汗等旗。命貢楚克扎布赴西寧會同驅逐之。十一年二月,辦事大臣貢楚克扎布奏:「貴德、循化番子頭目帶至闇門內,與寧西鎮總兵九十、西寧道慶炆傳見曉諭,番目尖木贊、策合洛等請每年各出羊隻,租住蒙古空閒地方,今年三四月間,劃定界限,設立鄂博,每年春季,再添會哨一次。」六月,貢楚克扎布奏番帳驅逐凈盡,請以青海尚那克空地安插野番,允之。二十二年十月,以青海扎薩克臺吉恩凱巴雅爾捕獲劫奪蒙古果洛克番賊,予花翎。二十四年十二月,護陜甘總督硃勛奏邊外番目縛獻番賊,交出原搶蒙古人口牲畜,予番目尖木贊四品頂戴。

道光二年正月,以硃勛奏河北插帳之循化等處九族野番及鹽池一帶挖鹽番戶抗不回巢,又蘊依、雙勿兩族,勾結循化、貴德及四川野番,盤踞原為貝勒特里巴勒珠爾六旗游牧之克勒蓋、克克烏蘇地方,搶掠蒙旗,請增卡防官兵,允之。命長齡回陜甘總督,會松廷相機辦理,設法驅逐。三月,長齡奏調官兵八千餘名,分途並進,迫令遷移。五月,長齡以剿捕蘊依等二十三族野番全數肅清奏聞。諭飭妥籌善後事宜,並曉諭蒙古王公等勉思振勵,自相保衛。六月,長齡以貝勒特里巴勒等移居青海已久,憚回原牧,請以克勒蓋一帶令察罕諾們汗移居,克克烏蘇一帶令阿里克阿百戶住牧,停向年會哨之兵,免究治諾們汗失察屬下勾結野番搶掠之咎,允之。尋野番復出劫掠貝子喇特納希第游牧。八月,長齡以野番一千數百人過河殺掠聞。命那彥成馳往查辦,署陜甘總督,責長齡辦理不善,撤雙眼花翎。十月,那彥成奏酌設卡隘,嚴捕漢奸。並謂:「野番冥頑成性,蒙古虐其屬下,反投野番謀生,導引搶掠其主。內地歇家奸販,潛住貿易,無事則教引野番漸擾邊境,有兵則潛過報信。近年番勢日張,弊實在此。」十一月,增設西寧鎮鎮海協副將、都司、守備各一,大通營游擊一,哈拉庫圖爾營都司一,哈瑪爾托亥營都司一,雙俄卜營守備一,千、把以下弁兵有差。以那彥成請,以保衛蒙旗,防禦番賊。十二月,那彥成奏:「察罕諾們汗所部夥同野番,勾結漢奸,作賊已久。此次將糧茶斷絕,立見窮蹙,原歸河南游牧,現押令過河。」上以「不勞力、不延歲月、辦理認真」嘉之。定清釐河南、循化、貴德番族,安插河北番族及易換糧茶章程,設千戶、百戶、百總、十總管束之,封閉野牛溝、八寶山等處偷挖金砂窯洞。

三年,賚青海被擾郡王車凌敦多布等二十四旗青稞三萬石。十月,允理籓院議覆那彥成奏,分青海河北二十四旗為左、右翼,每翼設正副盟長各一,每六旗設扎齊克齊一,每三旗設梅楞一,每旗設扎蘭一,承辦巡防事件。每旗出二十五人,以五人為一班,每季更換,隨同官兵巡防。十八年,玉樹熟番內雍希葉布、蒙古爾津尼、牙木錯、卡愛爾四族,以避果洛克番劫掠,奔赴青海,右翼盟長郡王恭木楚克集克默特原讓游牧內空閒地段住牧。西寧辦事大臣蘇勒芳阿派員勘明其地,東至和達素溝,西至奎田口,北至烏蘭麥爾河沿,南至哈利蓋邊界,於四至高阜處設立鄂博,分定界址。雍希葉布等四族計人戶二百有九,男婦大小一千一百有八十名口。議立交納馬貢易換糧茶各章程,盟長等鎮百戶番目謁見蘇勒芳阿,議定應行事宜,額外苛派。九月,奏入,得旨依議。十二月,青海兩翼正副盟長郡王車凌敦多布等呈蘇勒芳阿:「河南察罕諾們汗一旗被各番賊劫掠,人戶失散,現僅存三百餘戶,日不聊生,不及原來人戶四分之一。請將該旗照舊移過河北,與察罕洛亥駐防官兵協同把守渡口,實與蒙古有益。」蘇勒芳阿奏:「即飭貴德文武將該旗安分守法之人移過河北,交車凌敦多布代為管理。仍飭留心稽查,如有滋事作賊之人,不準混淆移過,以昭慎重。」從之。

二十二年,果洛克番賊竄青海,掠蒙古及番族。盟長郡王恭木楚克集克默特率兵剿捕,俘番賊多名,得所掠牲畜,賚緞疋獎之。二十三年七月,以陜甘總督富呢揚阿等奏河北近邊及河南番族畏法,酌撤各路官兵,予出力左翼盟長郡王貝子索諾木雅爾吉獎,分給在事蒙、番牛羊一萬四千六百有奇。二十四年三月,錄斬擒偷渡河北番賊功,予左翼副盟長貝勒羅布藏濟木巴雙眼花翎。五月,番族喀布藏與蒙古挾仇報復,蒙兵敗之。六月,富呢揚阿奏派防兵並蒙古、番兵,按季於出巡前赴青海南適中之貢額爾蓋地方會哨。是年,僑居郡王恭木楚克集克默特旗之雍希葉布等四番族仍回原牧。六平番賊復出劫掠,命甘肅提督胡超赴永固剿之,飭西寧辦事大臣德興駐丹噶爾。六月,陜甘總督布彥泰等奏剿黑錯寺,番族竄遁,酌量撤兵。

咸豐二年,以陜甘總督舒興阿奏,飭暫駐永安城之蒙古郡王等回牧,裁察罕洛亥等處蒙古兵一半。四年,陜甘總督易棠奏於野牛溝三處招募獵戶各一千名開採金砂,堵御番匪。同治三年,飭山西籌解青海蒙古王公等歲俸。以青海剿賊出力,予扎薩克王烏爾琿扎布等獎敘。

光緒元年九月,西寧辦事大臣豫師奏捕獲柴達木搶殺番目之蒙古人犯。諭免扎薩克達什多布吉議處,仍飭認真約束。四年十一月,予青海歷年剿匪出力之副盟長貝勒拉旺多布吉等獎。

二十三年二月,甘肅回匪劉四伏等率潰賊數萬人由南山水峽口竄青海格德地方,貝子納木希哩率蒙兵,右翼副盟長貝勒拉旺多布吉、貝子吹木丕勒爾布、察罕諾們汗旗及剛咱族總千戶均派兵會合堵剿。納木希哩等陣亡,尋贈納木希哩郡王銜,恤之。是月十四、十五等日,匪竄左翼郡王鞥克濟爾噶勒游牧都藍果立地方,鞥克濟爾噶勒派兵進擊,匪遂竄柴達木,勢張甚。陜西巡撫魏光燾派道員嚴金清率馬隊由水峽口尾追,甘肅提督董福祥派馬隊從丹噶爾日月山出口,會兵海南一帶,齊至都藍果力地方前進。劉四伏等竄踞遐力哈凈並腮什唐等地,負嵎死拒。柴達木住牧之左翼盟長貝子恭布車布坦、貝勒車琳端多布、臺吉索木端多布等親率蒙兵迎擊。時口外盛雪嚴寒,回匪無所得食,饑凍斃者大半。劉四伏等見勢不支,遂向西分竄安西、敦煌各境。陜甘總督陶模派道員潘效蘇分兵由扁都口進戰,西寧辦事大臣奎順飭大通住牧之右翼正盟長郡王棍布拉布坦、公齊克什扎布、臺吉丹把、臺吉齊莫特林增,阿里克族百戶格拉哈官布等親督蒙、番兵丁,會合官軍,分途兜剿。公齊克什扎布手帶槍傷,裹創力戰。劉四伏率匪西遁,餘賊降,於貝子恭布車布坦旗安插管束,青海肅清。陶模請獎奏入,於郡王鞥克濟爾噶勒等獎有差。陶模等於丹噶爾設局,以銀布糧茶賑被難各旗。

宣統二年四月,郡王巴勒珠爾拉布坦為資政院欽選議員。三年四月,青海左翼正盟長扎薩克貝勒車林端多布卒,廣恕奏以本翼郡王鞥克濟爾噶勒暫代之。

其地有礦,有鹽,林木亦富。佐領共一百有三。


\end{pinyinscope}