\article{志八十}

\begin{pinyinscope}
○輿服四鹵簿附

皇帝鹵簿太上皇鹵簿皇太子儀衛皇后儀駕太皇太后儀駕皇太后儀駕

皇貴妃以下儀仗採仗親王以下儀衛固倫公主以下儀衛

額駙儀衛職官儀衛

清自太宗天聰六年定儀仗之制,凡國中往來,御前旗三對,傘二柄,校尉六人,其制甚簡。自天聰十年改元崇德,始定御仗數目及品官儀從。迨世祖入關定鼎,參稽往制,量加增飾。原定皇帝儀衛有大駕鹵簿、行駕儀仗、行幸儀仗之別,乾隆十三年,復就原定器數增改釐訂,遂更大駕鹵簿為法駕鹵簿,行駕儀仗為鑾駕鹵簿,行幸儀仗為騎駕鹵簿。三者合,則為大駕鹵簿。而凡皇后儀駕、妃嬪儀仗採仗以及親王以下儀衛,均視原定加詳。茲依乾隆朝所定者標目,而以原定器數及崇德初年所定者附見於後。又太上皇鹵簿、皇太子儀衛,皆一時之制,非同常設,亦並著於篇。庶考因革者,得以沿流溯源,詳稽一代之制焉。

皇帝大駕鹵簿,圜丘、祈穀、常雩三大祀用之。大閱時詣行宮,禮成還宮,亦用之。其制,前列導象四,次寶象五,次靜鞭四。次前部大樂,其器大銅角四,小銅角四,金口角四。次革輅駕馬四,木輅駕馬六,象輅駕馬八,金輅駕象一,玉輅駕象一。次鐃歌樂,鐃歌鼓吹與行幸樂並設,名鐃歌樂。其器金二,銅鼓四,銅鈸二,扁鼓二,銅點二,龍篴二,平篴二,雲鑼二,管二,笙二,金口角八,大銅角十六,小銅角十六,蒙古角二,金鉦四,畫角二十四,龍鼓二十四,龍篴十二,拍板四,仗鼓四,金四,龍鼓二十四,間以紅鐙六。次引仗六,御仗十六,吾仗十六,立瓜、臥瓜各十六,星、鉞各十六,出警、入蹕旗各一,五色金龍小旗四十,次翠華旗二,金鼓旗二,門旗八,日、月旗各一,五雲旗五,五雷旗五,八風旗八,甘雨旗四,列宿旗二十八,五星旗五,五岳旗五,四瀆旗四,神武、硃雀、青龍、白虎旗各一,天馬、天鹿、闢邪、犀牛、赤熊、黃羆、白澤、角端、游麟、彩獅、振鷺、鳴鳶、赤烏、華蟲、黃鵠、白雉、雲鶴、孔雀、儀鳳、翔鸞旗各一。五色龍纛四十,前鋒纛八,護軍纛八,驍騎纛二十四。次黃麾四,儀鍠氅四,金節四,進善納言、敷文振武、褒功懷遠、行慶施惠、明刑弼教、教孝表節旌各二。龍頭幡四,豹尾幡四,絳引幡四,信幡四。羽葆幢四,霓幢四,紫幢四,長壽幢四。次鸞鳳赤方扇八,雉尾扇八,孔雀扇八,單龍赤團扇八,單龍黃團扇八,雙龍赤團扇八,雙龍黃團扇八,赤滿單龍團扇六,黃滿雙龍團扇六,壽字黃扇八。次赤素方傘四,紫素方傘四,五色花傘十,五色壯緞傘十,間以五色九龍團傘十。次九龍黃蓋二十,紫芝蓋二,翠華蓋二,九龍曲柄黃蓋四。次戟四,殳四,豹尾槍三十,弓矢三十,儀刀三十。次仗馬十。次金方幾一,金交椅一,金瓶二,金盥盤一,金盂一,金盒二,金爐二,拂二。次九龍曲柄黃蓋一。前引佩刀大臣十人,提爐二,玉輦在中。後扈佩刀大臣二人,豹尾班執槍佩儀刀侍衛各十人,佩弓矢侍衛二十人,領侍衛內大臣一人,侍衛班領二人。後管宗人府王、公二人,散秩大臣一人,前鋒護軍統領一人,給事中、御史二人,各部郎中、員外郎四人,侍衛班領一人,署侍衛班領一人,侍衛什長二人。次黃龍大纛二,領侍衛內大臣一人,司纛侍衛什長二人,建纛親軍四人,鳴佩螺親軍六人。太宗崇德元年,備大駕鹵簿,玉璽四顆。黃傘五,團扇二。纛十,旗十。大刀六,戟六。立瓜、臥瓜、骨朵各二,吾仗六。馬十。金椅、金杌、香盒、香爐、金水盆、金唾壺、金瓶、樂器全設。嗣復定儀仗數目,用金漆椅一,金漆杌一,蠅拂四,金唾盂一,金壺一,金瓶、金盆各一,香爐、香盒各二。曲柄傘一,直柄傘四,扇二,節四。骨朵二,立瓜、臥瓜各二,吾仗六,紅仗四。鑼二,鼓二,畫角四,簫二,笙二,架鼓四,橫笛二,龍頭橫笛二,檀板二,小銅鈸四,小銅鑼二,大銅鑼四,雲鑼二,鎖吶四。世祖入關,一仍舊制。迨順治三年以後,更定皇帝鹵簿,有大駕鹵簿、行駕儀仗、行幸儀仗之別。大駕鹵簿之制,曲柄九,龍傘四,直柄九龍傘十六,直柄瑞草傘六,直柄花傘六,方傘八。大刀二十,弓矢二十,豹尾槍二十,龍頭方天戟四。黃麾二,絳引幡四,信幡、傳教幡、告止幡、政平訟理幡各四,儀鍠氅八,羽葆幢四,青龍、白虎、硃雀、神武幢各一,豹尾幡、龍頭竿幡各四。金節四。銷金龍纛、銷金龍小旗各二十。金鉞六。馬十。鸞鳳扇八,單龍扇十二,雙龍扇二十。拂子二,紅鐙六,金香爐、金瓶、金香盒各二,金唾壺、金盆、金杌、金交椅、金腳踏各一。御仗六,星六。篦頭八。棕薦三十。靜鞭三十。品級山七十二。肅靜旗、金鼓旗、白澤旗各二,門旗八,日、月、風、雲、雷、雨旗各一,五緯旗五,二十八宿旗各一,北斗旗一,五岳旗五,四瀆旗四,青龍、白虎、硃雀、神武、天鹿、天馬、鸞麟、熊羆旗各一。立瓜、臥瓜、吾仗各六。畫角二十四,鼓四十八,大銅號、小銅號各八,金、金鉦、仗鼓各四,龍頭笛十二,板四串。凡郊祀大典,萬壽、元旦、冬至三大朝會及諸典禮皆用之。

法駕鹵簿,與大駕鹵簿同,惟彼用鐃歌樂,此則用鐃歌鼓吹。其器大銅角八,小銅角八,金鉦四,畫角二十四,龍鼓二十四,龍篴十二,拍板四,仗鼓四,金二,龍鼓二十四,間以紅鐙六,視鐃歌樂為減。又御仗、吾仗、立瓜、臥瓜、星、鉞皆各六,五色金龍小旗二十,五色龍纛二十,九龍黃蓋十,豹尾槍二十,弓矢二十,儀刀二十,佩弓矢侍衛十人,其赤滿單龍團扇、黃滿雙龍團扇及五色莊緞傘皆不設,亦均較大駕為減。又玉輦改設金輦,餘均與大駕鹵簿同。凡祭方澤、太廟、社稷、日月、先農各壇,歷代帝王、先師各廟,則陳之。若遇慶典朝賀,則陳於太和殿庭。其制,九龍曲柄黃華蓋設於太和殿門外正中,次拂、爐、盒、盂、盤、瓶、椅、幾在殿簷東、西。次儀刀、弓矢、豹尾槍親軍、護軍相間為十班,暨殳戟,均在丹陛東、西。次九龍曲柄黃蓋、翠華蓋、紫芝蓋、九龍黃蓋、五色九龍傘、五色花傘,自丹陛三成,相間達於兩階。階下靜鞭、仗馬列甬道東、西。紫、赤方傘、扇,幢、幡、旌、節、氅、麾、纛、旗、鉞、星、瓜、仗,列丹墀東、西。玉輦、金輦在太和門外,五輅在午門外,寶象在五輅之南,鹵簿樂即鐃歌鼓吹。在寶象之南,朝象即導象。在天安門外。若於圓明園行慶賀禮,則陳於正大光明殿階下,至大宮門外,惟輦輅儀象不設。若御樓受俘,則設九龍曲柄黃華蓋於樓簷下,設丹陛鹵簿於午門外左右兩觀下,設丹墀鹵簿於闕左右門至端門北,設仗馬於兩角樓前,設輦輅儀象於天安門外,設靜鞭於兩角樓夾御道左右,設金鼓鐃歌大樂鐃歌鼓吹與前部大樂並列,曰金鼓鐃歌大樂。於午門前。設丹陛大樂於鹵簿之末,其器云鑼二,方響二,簫二,篴四,頭管四,笙四,大鼓二,仗鼓一,拍板一。

鑾駕鹵簿,行幸於皇城則陳之。其制,前列導迎樂,先以戲竹二,次管六,篴四,笙二,雲鑼二,導迎鼓一,拍板一。次御仗四,吾仗四,立瓜、臥瓜、星、鉞各四,次五色金龍小旗十,五色龍纛十。次雙龍黃團扇十,黃九龍傘十。次九龍曲柄黃華蓋一。皆在皇帝步輦前。次前引佩刀大臣十人,後扈佩刀大臣二人,步輦在中,次豹尾班侍衛執槍十人,佩儀刀十人,佩弓矢十人,殿以黃龍大纛。原定行駕儀仗,銷金九龍傘十,銷金龍纛十,銷金龍小旗十,雙龍扇十。金鉞四,星四,御仗四,吾仗四,立瓜、臥瓜各四。凡車駕出入,執事人馬上排列。

騎駕鹵簿,巡方若大閱則陳之。其制,前列鐃歌大樂。間以鐃歌清樂,器用大銅角八,小銅角八,金口角八,雲鑼二,龍篴二,平篴二,管二,笙二,銅鼓四,金二,銅點二,銅鈸二,行鼓二,蒙古角二。次御仗六,吾仗六,立瓜、臥瓜、星、鉞各六。次五色金龍小旗十,五色龍纛十。次單龍赤團扇六,雙龍黃團扇六,五色花傘十。次豹尾槍十,弓矢十,儀刀十。次九龍曲柄黃華蓋一。皆在皇帝輕步輿前,若乘馬則在馬前。次前引佩刀大臣十人,後扈佩刀大臣二人,輕步輿在中。次豹尾班侍衛執槍十人,佩儀刀十人,佩弓矢十人,殿以黃龍大纛。駐蹕御營,朝陳蒙古角,夕陳鐃歌樂。大閱則陳鹵簿於行宮門外。原定行幸儀仗,莊緞傘十,銷金龍纛十,銷金龍小旗十。雙龍扇六,單龍扇四。豹尾槍十,大刀十,弓矢十。金鉞六,星六,御仗、吾仗、立瓜、臥瓜各六。金二,笙二,雲鑼二,管二,篴四,金鉦四,銅鈸四,鼓二,鎖吶八,銅點二,小號、大號各八,蒙古號六。凡車駕行幸,執事人步行排列。

太上皇鹵簿,原定無之。嘉慶元年,因授璽禮成,陳太上皇鹵簿於寧壽宮。其制,引仗六,御仗十六,吾仗十六,立瓜、臥瓜各十六,星、鉞各十六,旗、纛二百二十四,麾、氅、節各四,旌十六,幡十二,幢二十,扇八十六,傘六十六,戟殳各四,豹尾槍、弓矢、儀刀各三十,金交椅、金馬杌各一,拂二,金器八,銀水、火壺各一,雨傘二,盤線鐙二,紅鐙六。樂器備設,笙、管、雲鑼、平篴、鈸、點鼓各二,金及金鉦、銅鼓、扁鼓、仗鼓各四,架鼓、金口角各十二,龍篴十四,大銅角、小銅角、蒙古畫角各二十四,龍鼓四十八。

皇太子儀衛,清自康熙五十二年後不復建儲,故國初雖有皇太子儀仗,幾同虛設。乾隆六十年,以明年將行內禪,九月,議定皇太子出入內朝,用導從侍衛四人,乾清門侍衛二人。如出外朝及城市內外,隨從散秩大臣一人,侍衛十人,領侍衛內大臣一人,乾清門侍衛四人。前設虎槍六,後設豹尾槍八。是年復諭禮臣,以冊立皇太子典禮既不舉行,其一切儀仗制造需時,亦毋庸另行備辦。原定皇太子儀仗,曲柄九龍傘三,直柄龍傘四,直柄瑞草傘二,方傘四,雙龍扇四,孔雀扇四。白澤旗二。金節二。羽葆幢二,傳教幡、告止幡、信幡、絳引幡各二,儀鍠氅二。銷金龍纛十,銷金龍小旗十。豹尾槍十,弓矢十,大刀十,馬八,金鉞四,立瓜、臥瓜、骨朵、吾仗各四。拂二。畫角十二,花匡鼓二十四,大銅號八,小銅號二,金、金鉦、仗鼓各二,龍頭篴二,板二。金香爐、金瓶、金香盒各二,金唾壺,金盆各一,金杌一,金交椅一,金腳踏一。

皇后儀駕,原名鹵簿。吾仗四,立瓜四,臥瓜四,五色龍鳳旗十。次赤、黃龍、鳳扇各四,雉尾扇八,次赤、素方傘四,黃緞繡四季花傘四,五色九鳳傘十。次金節二。次拂二,金香爐二,金香盒二,金盥盤一,金盂一,金瓶二,金椅一,金方幾一。次九鳳曲柄黃蓋一。鳳輿一乘,儀輿二乘,鳳車一乘,儀車二乘。原定太皇太后鹵簿,銷金龍鳳旗八。金節二。吾仗四,立瓜四,臥瓜四。黃曲柄九鳳傘一,黃直柄花傘四,紅直柄瑞草傘二,青黑直柄九鳳傘各二,紅方傘二,黃、紅銷金龍、鳳扇各二,金黃素扇二,紅鸞鳳扇二。拂二,金香爐二,金瓶二,金香盒二,金唾壺一,金盆一,金杌一,金交椅一,金腳踏一。凡萬壽節、元旦、冬至及諸慶典,鑾儀衛先時陳設。皇太后、皇后鹵簿並同。

太皇太后儀駕暨皇太后儀駕,均與皇后儀駕同。惟車、輿兼繪龍鳳文。

皇貴妃儀仗,吾仗四,立瓜四,臥瓜四。赤、黑素旗各二,金黃色鳳旗二,赤、黑鳳旗各二。金黃、赤、黑三色素扇各二,赤、黑鸞鳳扇各二,赤、黑瑞草傘各二,明黃、赤、黑三色花傘各二。金節二。拂二,金香爐、香盒、盥盤、盂各一,金瓶二,金椅一,金方幾一。七鳳明黃曲柄蓋一。翟輿一乘,儀輿一乘,翟車一乘。原定皇貴妃儀仗,紅、黑鳳旗各二,金節二,吾仗二,立瓜二,臥瓜二。紅曲柄七鳳傘一,紅直柄花傘二,紅直柄瑞草傘二,紅方傘二,金黃素扇二,紅繡扇二。拂二,金香爐一,金瓶二,金香盒一,金唾壺一,金盆一,馬杌一,交椅一,腳踏一。貴妃儀仗同。

貴妃儀仗,吾仗二,立瓜二,臥瓜二。赤、黑素旗各二,赤、黑鳳旗各二,金黃、赤、黑三色素扇各二,赤、黑鸞鳳扇各二,赤、黑瑞草傘各二,金黃、赤、黑三色花傘各二。金節二。拂二,金香爐、香盒、盥盤、盂各一,金瓶二,金椅一,金方幾一。七鳳金黃曲柄蓋一。翟輿一乘,儀輿一乘,儀車一乘。

妃採仗,原名儀仗。吾仗二,立瓜二,臥瓜二。赤、黑鳳旗各二。赤、黑素扇各二,赤、黑花傘各二,金黃素傘二。金節二。拂二,銀質飾金香爐、香盒、盥盤、盂各一,銀瓶二,銀椅一,銀方幾一。七鳳金黃曲柄蓋一。翟輿一乘,儀輿一乘,儀車一乘。原定妃儀仗,黑鳳旗二。金節二。吾仗二,立瓜二,臥瓜二。紅直柄花傘二,紅直柄瑞草傘二,金黃素扇二。拂二,銀質飾金香爐、香盒各一,瓶一,唾壺一,盆一,馬杌一,交椅一,腳踏一。

嬪採仗,原名儀仗。視妃採仗少直柄瑞草傘二。餘同。

親王儀衛,原名儀仗。以下並同。吾仗四,立瓜四,臥瓜四,骨朵四。紅羅繡五龍曲柄蓋一。紅羅繡四季花傘二,紅羅銷金瑞草傘二,紅羅繡四季花扇二,青羅繡孔雀扇二。旗槍十,大纛二,條纛二。豹尾槍四,儀刀四。馬六。遇大典禮,則陳於府第,出使用以導從。常日在京,用曲柄蓋一。紅羅傘扇各二。吾仗、立瓜、臥瓜、骨朵全。馬四。前引十人,後從六人。因事入景運門,帶從官三人。原定有紅羅繡花曲柄傘一,豹尾槍二,大刀二。茲改為五龍曲柄蓋一,豹尾槍四,儀刀四。餘同。崇德初年,定親王銷金紅傘二,纛二,旗十,立瓜、骨朵各二,吾仗四。

世子儀衛,吾仗四,立瓜四,臥瓜二,骨朵二。紅羅四龍曲柄蓋一。紅羅繡四季花傘一,紅羅銷金瑞草傘二,紅羅繡四季花扇二,青羅繡孔雀扇二。旗槍八,大纛一,條纛一。豹尾槍二,儀刀二。馬六。常日用紅羅傘、扇各二,吾仗、立瓜、臥瓜、骨朵全。馬四。前引八人,後從六人。原定吾仗二,立瓜二,有紅羅繡花曲柄傘一,無豹尾槍,茲增為吾仗四,立瓜四,改曲柄傘為四龍曲柄蓋,添豹尾槍二。餘同。崇德年所定,無世子儀仗。

郡王儀衛,吾仗四,立瓜四,臥瓜二,骨朵二。紅羅繡四龍曲柄蓋一。紅羅銷金瑞草傘二,紅羅繡四季花扇二,青羅繡孔雀扇二。旗槍八,條纛二。豹尾槍二,儀刀二。馬六。常日用紅羅傘、扇各二,吾仗、立瓜、臥瓜、骨朵全。馬二。前引八人,後從六人。原定有紅羅繡花曲柄傘一,無豹尾槍、儀刀。茲改曲柄傘為四龍曲柄蓋,增豹尾槍二,儀刀二。餘同。崇德初年,定郡王銷金紅傘一,纛一,旗八,臥瓜二,吾仗四。

郡王長子儀衛,原定及崇德年所定均無。吾仗二,立瓜二,臥瓜二,骨朵二。紅羅銷金瑞草傘一,紅羅繡四季花扇二。旗槍六,條纛一。馬四。常日用傘一,吾仗、立瓜、臥瓜、骨朵全。前引六人,後從六人。

貝勒儀衛與郡王長子同。原定紅羅銷金傘二,茲減一。餘同。崇德初年,定貝勒銷金紅傘一,纛一,旗六,骨朵二,紅仗二。自世子以下至貝勒,因事入景運門,帶從官二人。

貝子儀衛,吾仗二,立瓜二,骨朵二。紅羅銷金瑞草傘一,紅羅繡四季花扇二。旗槍六,條纛一。常日用吾仗、立瓜、骨朵全。前引四人,後從六人。原定紅羅銷金傘二,茲減一。餘同。崇德初年定貝子彩畫雲紅傘一,豹尾槍二,旗六,紅仗二。

鎮國公、輔國公儀衛,吾仗二,骨朵二。紅羅銷金瑞草傘一,青羅繡孔雀扇一。旗槍六。常日用吾仗、骨朵全。前引二人,後從八人。原定同。崇德初年,定鎮國公紅傘一,豹尾槍二,旗六,紅仗二。輔國公減豹尾槍一。餘同。自貝子以下、輔國公以上,因事入景運門,帶從官一人。

鎮國將軍儀衛,杳黃傘一,青扇一,旗槍六。常日前引二人,後從六人。原定有金黃傘一,無青扇。茲改為杏黃傘一,增青扇一。餘同。自鎮國將軍以下,原定均照崇德初年定制。

輔國將軍儀衛,與鎮國將軍同。常日前引一人,後從四人。原定常日前引二人,茲減一。餘同。

奉國將軍、奉恩將軍儀衛,原定無奉恩將軍。青扇一,旗槍四。常日後從四人。原定無青扇。

固倫公主儀衛,吾仗二,立瓜二,臥瓜二,骨朵二。金黃羅曲柄繡寶相花傘一,紅羅繡寶相花傘二,青羅繡寶相花扇二,紅羅繡孔雀扇二。黑纛二。前引十人,朝賀日隨侍女五人。原定曲柄傘用紅羅,茲改金黃羅。餘同。崇德初年,定固倫公主清道旗二。紅仗、吾仗各二。銷金紅傘一,青扇一。拂子二,金吐盂、金水盆各一。

和碩公主儀衛,吾仗二,立瓜二,臥瓜二,骨朵二。紅羅曲柄繡寶相花傘一,紅羅繡寶相花傘二,紅羅繡孔雀扇二。黑纛二。前引八人,隨朝侍女四人。原定同。崇德初年,定和碩公主紅仗、吾仗各二。銷金紅傘一,青扇一。拂子二,金水盆一。

郡主儀衛,吾仗二,立瓜二,骨朵二。紅羅繡寶相花傘二,紅羅繡孔雀扇二。前引六人,隨朝侍女三人。原定同。崇德初年,定郡主吾仗二,銷金紅傘一,青扇一,拂子二。

縣主儀衛,吾仗二,立瓜二。紅羅繡寶相花傘一,青羅繡寶相花扇二。前引二人,隨朝侍女三人。原定同。崇德初年,定縣主紅仗二,銷金紅傘一,拂子二。

郡君隨朝侍女二人,縣君隨朝侍女二人,鄉君隨朝侍女一人,俱無儀仗。原定郡君以下無儀仗。崇德初年,定郡君紅仗二,銷金青傘一,縣君紅仗二。

親王福晉視固倫公主,惟曲柄傘用紅色,隨朝侍女四人。世子福晉視和碩公主,郡王福晉視郡主,郡王長子福晉、貝勒夫人均視縣主,隨朝侍女二人。貝子夫人、公夫人隨朝侍女一人。自貝子夫人以下無儀仗。自將軍夫人以下無隨朝侍女。原定同。惟福晉皆稱妃,又別定側妃、側夫人儀仗。其制,親王側妃視嫡妃少青羅花扇二。餘同。世子側妃纛二,吾仗、立瓜、骨朵各二,紅羅繡花傘、紅羅繡孔雀扇各二。郡王側妃吾仗、立瓜各二,紅羅繡花傘一,青羅繡花扇二。貝勒側夫人及貝子夫人均無儀仗。崇德初年,定親王妃清道旗二,紅仗、吾仗各二,銷金紅傘一,青扇一,拂子二,金唾盂、金水盆各一。郡王妃同,惟少紅仗、金唾盂。貝勒夫人紅仗二,銷金紅傘一,拂子二。貝子夫人以下無儀仗。

額駙儀衛,固倫公主額駙,紅仗二,紅傘一,大小青扇二,旗槍十,豹尾槍二。常日前引二人,後從八人。和碩公主額駙,紅棍四,杏黃傘二,大、小青扇二,旗槍十。常日前引二人,後從八人。郡主額駙,紅棍四,杏黃傘一,大、小青扇二,旗槍十。常日前引二人,後從八人。縣主額駙,杏黃傘一,青扇一,旗槍六。常日前引二人,後從六人。郡君額駙,青扇一,旗槍六。常日前引二人,後從六人。縣君額駙,青扇一,旗槍四。常日無前引,惟後從二人。鄉君額駙,青扇一,旗槍二。常日惟後從一人。

職官儀衛,原名儀從。民公視和碩公主額駙。侯,金黃棍四,餘視郡主額駙。其有加級者,棍得用紅。伯,大、小青扇二,餘視侯。子,金黃棍二,杏黃傘一,大、小青扇二,旗槍八。前引、後從視侯。男,金黃棍二,杏黃傘一,大、小青扇二,旗槍六。常日前引二人,後從六人。

京官,一品視子,二品視男。三品,金黃棍二,杏黃傘一,大、小青扇二,旗槍六。常日前引二人,後從四人。四品,杏黃傘一,大、小青扇二,旗槍四。常日無前引,惟後從二人。餘官均用青素扇一。常日惟後從一人。宗室、覺羅之有職者,各從其品,惟扇柄及棍皆魨以硃。以上儀衛,於京外得全設,常日在京,不得用旗、傘、黃棍。文官三品以上,得用甘蔗棍二。武官三品以上,得用棕竹棍二。自一品至九品,均得用扇,扇各用清、漢字書銜。若進皇城,扇棍及前引人均不得入。文武大臣因事入景運門,帶從官一人。

直省文官,總督,青旗八,飛虎旗、杏黃傘、青扇、兵拳、雁翎刀、獸劍、金黃棍、桐棍、皮槊各二,旗槍四,回避、肅靜牌各二。巡撫,青旗八,杏黃傘、青扇、獸劍、金黃棍、桐棍、皮槊各二、旗槍二,回避、肅靜牌各二。凡二品以上大臣陛見到京,入景運門,帶從官一人。布政使、按察使,青旗六,杏黃傘、青扇、金黃棍、皮槊各二,回避、肅靜牌各二。各道青旗四,杏黃傘、青扇各一,桐棍、皮槊各二,回避、肅靜牌各二。知府與道同。府倅、知州、知縣,青旗四,藍傘一,青扇一,桐棍、皮槊各二,肅靜牌二。縣佐,藍傘一,桐棍二。學官,藍傘一。雜職,竹板二。河道、漕運總督視總督,學政、鹽政、織造暨各欽差官三品以上視巡撫。四品以下視兩司。

武官,提督,青旗八,飛虎旗、杏黃傘、青扇、兵拳、雁翎刀、獸劍、刑仗各二,旗槍四,回避、肅靜牌各二。總兵官,青旗八,飛虎旗、杏黃傘、青扇、獸劍、旗槍、大刀各二,回避、肅靜牌各二。副將,青旗六,杏黃傘一,青扇二,金黃棍二,回避、肅靜牌各二。參將、游擊、都司,青旗四,杏黃傘一,青扇一,桐棍二,回避、肅靜牌各二。守備,青旗四,杏黃傘一,青扇一,桐棍二。各省駐防將軍視內都統。副都統以下均與京職同。順治三年,定京官儀從,公,掌扇貼方金一。職官掌扇,一品貼圓金四,二品貼圓金三,三品貼圓金二,四品用灑金掌扇,五品至七品俱用素黑掌扇,八品、九品俱用白掌扇。六年,定公以下四品官以上用大、小灑金扇各一,文官用甘蔗棍二,武官用棕竹棍二。八年,定民公、和碩公主額駙,杏黃傘二,旗十,大、小扇二,貼方金四。侯、郡主額駙,杏黃傘一,旗十,大、小扇二,貼方金三。伯,杏黃傘一,旗十,大、小扇二,貼圓金一。一品官,杏黃傘一,旗八,大、小扇二,貼圓金四。二品官,杏黃傘一,旗六,大、小扇二,貼圓金三。三品官,杏黃傘一,旗六,大、小扇二,貼圓金二。四品官,旗四,灑金大、小扇二。五品以下官如三年例。京城內不得排列旗、傘,惟於外用之。宗室、覺羅各官,扇柄及棍皆丹漆。凡入皇城,惟用小扇。九年,定公以下、漢文官三品以上,皇城外坐暗轎,四人舁之,掌扇各書官銜,兼滿、漢字。康熙初年,定公、和碩公主額駙,旗十,杏黃傘二,金黃棍四。侯、伯、郡主額駙,旗十,杏黃傘一,金黃棍四。都統,鎮國將軍,內大臣,縣主額駙,子,滿、漢大學士,尚書,左都御史,旗八,杏黃傘一,金黃棍二。輔國將軍,郡君額駙,護軍統領,前鋒統領,副都統,男,滿、漢侍郎,學士,步軍總尉及三品官,旗六,杏黃傘一,金黃棍二。四品官旗四。京城內一品官以上惟用傘、棍,二品官並不用傘。四年,定在京文官三品以上、武官散秩大臣以上,一人引馬前導,其餘各官禁之。七年,定在京各官停用傘、棍,民公以下俱照順治八年例,出京用鞍籠閒馬前導。康熙七年,定外官儀從。總督,杏黃傘二,金黃棍二,旗八,扇二,兵拳二,雁翎刀二,飛虎旗二,獸劍二,桐棍二,槊棍二,槍四,回避、肅靜牌各二。巡撫,杏黃傘二,金黃棍二,旗八,扇二,獸劍二,桐棍二,槊棍二,槍二,回避、肅靜牌各二。布政使、按察使,杏黃傘二,金黃棍二,旗六,扇二,槊棍二,回避、肅靜牌各二。各道掌印、都司、知府,杏黃傘一,旗四,桐棍二,槊棍二,回避、肅靜牌各二。同知、通判、知州、知縣,藍傘一,扇一,桐棍一,槊棍一,回避牌二。州同、州判、縣丞,藍傘一,桐棍二。典史、雜職,竹板二。提督,杏黃傘二,金黃棍二,旗八,扇二,兵拳二,雁翎刀二,飛虎旗二,槍二,獸劍二,刑仗二,回避、肅靜牌各二。總兵,杏黃傘二,金黃棍二,旗八,扇二,大刀一,獸劍二,槍二,回避、肅靜牌各二。副將,杏黃傘二,金黃棍二,旗六,扇二,回避、肅靜牌各二。參將、游擊、都司,杏黃傘一,旗四,扇一,桐棍二,回避、肅靜牌各二。守備,杏黃傘一,旗四,扇一,棍二。崇德初年,定固倫額駙,彩畫雲紅傘一,豹尾槍二,紅仗二。超品公、和碩額駙,金黃傘一,豹尾槍二,旗六,後從十人。民公、郡主額駙,金黃傘一,豹尾槍一,旗六,後從八人。都統、子、尚書、縣主額駙,金黃傘一,旗六,後從六人。內大臣、大學士、副都統、護軍統領、前鋒統領、侍郎、郡君額駙,旗六,後從四人。一等侍衛、護衛、參領、前鋒參領、縣君額駙、學士、滿啟心郎、郎中,旗四,後從四人。二等侍衛、護衛、佐領、漢啟心郎、員外郎,旗二,後從二人。三等侍衛、護衛、護軍校、主事以下官員,止用後從一人。


\end{pinyinscope}