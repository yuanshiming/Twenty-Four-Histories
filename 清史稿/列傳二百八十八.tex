\article{列傳二百八十八}

\begin{pinyinscope}
遺逸二

李孔昭單者昌崔周田劉繼寧劉永錫彭之燦

徐枋戴易李天植理洪儲顧柔謙子祖禹冒襄陳貞慧

祁班孫兄理孫汪渢餘增遠周齊曾傅山子眉費密

王弘撰杜濬弟岕郭都賢陶汝鼐李世熊談遷

李孔昭,字光四,薊州人。性孤介,平居教授生徒,倡明理學。崇禎十五年進士,見世事日非,不赴廷對,以所給牌坊銀留助軍餉。奉母隱盤山中,躬執樵採自給。母病,刲股療之。北都陷,素服哭於野者三載。薊州城破,妻王殉難死,終身不再娶。形跡數易,人無識者。

清初,詔求遺老,撫按交章薦,不出。一日,當道遣吏持書幣往,遇負薪者,呼而問之,曰:「若識李進士耶?」負薪者詰得其故,以手遙指而去。吏至其室,虛矣。鄰叟曰:「汝面失之。向所負薪者,李進士也!」後屢物色之,卒不得。時有某孝廉,當上公車,輒止不行,曰:「吾出郭門一步,何面目見李光四乎?」

會值邑中方興役,按戶簽夫,驅孔昭,孔昭曰:「吾力不能任,原出貲以代。」吏持去。閱數日,大學士杜立德聞孔昭在邑,急往候之,吏聞,趨謝罪。孔昭曰:「此間不知有李進士,若勿誤也。」由是跡愈密,或黃冠,或儒服,見者甚稀。惟寶坻單者昌、崔周田、劉繼寧皆高士,與之友善,往來無虛歲。

者昌,字蔚起。才名埒孔昭。早餼於庠,入清不復應試。杜立德招之,不能致,獨與孔昭徜徉田野間,悲歌慨憤,有所作,輒焚之,不以示人。竟以憂死。

周田,字錫齡。順治中,充歲貢,不與試。建一樓,貯古本書及金石刻萬卷,日吟嘯其中。嘗過盤山,與孔昭坐林石間相笑語。孔昭亦時下榻於其家,周田命其子執弟子禮,且迎孔昭母,事之如所生。

繼寧,字兌菴。少負義氣,有古俠士風。嘗出重金贖難女二,為之擇配。歲饑,煮粥食餓者。視周田如手足,有緩急恆資之,周田亦弗謝也。晚年為子擇師游盤山,跡孔昭,得之。邀至其家,令其三子從受業。暇則與周田聚宴歌呼以為樂,然每一念母,雖深夜必馳歸,弗能禁也。晚好陶詩,因又自號潛翁。一日,為門人講孟子盡心章,曰:「此傳心法也!」言訖而卒。其弟子私謚曰安節先生。

劉永錫,字欽爾,號賸菴,魏縣人。崇禎乙亥舉人,官長洲教諭。南都敗,率妻慄隱居相城,大吏造其室,欲強之出,永錫袒裼疾視,曰:「我中原男子,年二十,渡漳河,登大伾,躍馬鳴鞘,兩河豪傑,誰不知我者!欲見辱耶?」取壁上劍自刎。門下士抱持之,得解,謂其妻曰:「彼再至,我與若立決矣!」皆裂尺帛握之。尋移居陽城湖濱,與妻及子臨、女貞織席以食。市中見永錫攜席至,皆呼席先生。食不繼,時不舉火,有遺之粟者,非其人不受,益困憊。其女已許字,未嫁,亂後恐遭辱,絕粒死。其妻哭之成疾,亦死。其僮僕遇水災乏食,相繼餓死,或散走。有老奴從魏縣來,勸之歸,曰:「室廬故在也!」永錫曰:「我非不欲歸,然昔奉君命來,義不可離此一步。」命其子與婦攜老奴還里,曰:「祖宗丘墓責在汝!」麾之去。時歲荒,得食愈艱,每雜糠籺作飯。臨既歸,思父不置,假貸得百金馳獻,中途馬驚,墮地死。

永錫容貌甚偉,至是,毀形骨立,既自悼無家,買一破船往來江湖間。嘗泛舟中流,鼓枻而歌曰:「溯彼中流兮,採其荇矣。呼君與父兮,莫之應矣。身為餓夫兮,天所命矣。中心殷殷兮,涕斯迸矣。」又歌曰;「白日墮兮野荒荒,逐鳧雁兮侶牛羊,壯士何心兮歸故鄉。」歌聲悲烈,聞者哀之。尚書錢謙益念其窮,招之往,永錫曰:「尚書為黨魁,受主眷,枚卜時天子期以伊、傅,彼豈忘之邪?」卻不往,卒窮餓至不能起。一夕,大呼「烈皇帝」者三,遂卒,時順治十一年秋也。弟子長洲徐晟、陳三島,友人常熟陸泓,經紀其喪,葬之於虎丘山塘,以妻、女祔之。

彭之燦,字了凡,蠡縣諸生。甲申後攜妻寓饒陽作村塾師。未幾,妻、子相繼死,至蘇門,與孫奇逢游。然性不諧俗,愛靜坐。有人延於家,以市囂,輒避去。嘗渡河南游,韓鼎業為館之僧舍,年餘,又棄去。獨擔瓢笠圖書,遍游嵩、少、王屋諸名勝。在九山絕粒數日,奇逢挽之夏峰,勸歸老先人墓旁。之燦曰:「某出門時,已誓告先壟不再返,不能蹈東海、入西山而死,即溝壑道路,無恨也!」順治十五年六月,竟死嘯臺東北石柱下。奇逢為鐫石記其事,立墓上,曰「餓夫之墓」。之燦與容城張果中、西華理鬯和,並稱「蘇門三賢。」

徐枋,字昭法,長洲人。父汧,明少詹事,殉國難,事具明史。枋,崇禎壬午舉人。汧殉國時,枋欲從死,汧曰;「吾不可以不死,若長為農夫以沒世可也!」自是遁跡山中,布衣草履,終身不入城市。及游靈巖山,愛其曠遠,卜澗上居之,老焉。枋與宣城沈壽民、嘉興巢鳴盛,稱「海內三遺民」。枋書法孫過庭,畫宗巨然,間法倪、黃,自署秦餘山人。嘗寄靈芝一貞於王士禎,士禎與金孝章畫梅、王玠草書作齋中三詠以記之。然性峻介,鍵戶勿與人接。睢州湯斌巡撫江南,屏騶從,往訪之,枋避不見。斌登其堂,堅坐移晷,為誦白駒之詩,周覽太息而去。川湖總督蔡毓榮自荊州致書求其畫,枋答書而返幣,竟不為作。曰:「明府是殷荊州,吾薄顧長康不為耳。」所往來惟沈壽民與萊陽姜垓、同里楊無咎、門人吳江潘耒及南嶽僧洪儲而已。

家貧絕糧,耐饑寒,不受人一絲一粟。洪儲時其急而周之,枋曰:「此世外清凈食也。」無不受。豢一驢,通人意。日用間有所需,則以所作書畫卷置簏於驢背,驅之。驢獨行,及城闉而止,不闌入一步。見者爭趣之,曰:「高士驢至矣!」亟取卷,以日用所需物,如其指,備而納諸簏,驢即負以返,以為常。卒,年七十三。

時商丘宋犖撫吳,枋預戒曰;「宋中丞甚知我,若我死,勿受其賻也。」犖果使人贈棺槥貲如枋命,終不受。卒,以貧不能葬。一日,有高士從武林來吊,請任窀穹,其人亦貧,而特工篆、隸,乃賃居郡中。鬻字以庀葬具,紙得百錢。積二年,乃克葬枋於青芝山下,而以羨歸其家。語之曰:「吾欲稱貸富家,懼先生吐之,故勞吾腕,知先生所心許也。」葬畢即去,不言名氏。或有識之者,曰:「此山陰戴易也!」

易,字南枝。少從劉宗周學,游吳門,年七十餘矣。有六子,不受其養,獨攜一子及殘書百卷自隨。其售字也,銖積寸累,不妄費一錢。一蒼頭饑不能忍,輒逃去。己寄食僧舍中,語及枋,必流涕。嘗浮七里瀨,登嚴子陵釣臺,賦詩,且歌且泣。或竟日不得食,採野蕨充膳。操瓢量水,坐長松古石間飲之。

李天植,字因仲,平湖人。崇禎癸酉舉人。改名確,字潛夫。甲申後,餘田四十畝、宅一區,乃並家具分與所後子震及女,而與妻別隱陳山,絕跡不入城市,訓山中童子自給。居十年,以僧開堂,始避喧,返蜃園,賣文自食;不足,則與其妻為椶奚竹筥以佐之。好事者約月供薪米,力辭不受。有司慕其高,往訪之,輒逾垣避。所著詩文,皆吊甲申以來殉節者。蜃園者,乍浦勝地,可望見海市者也。

又十年,家益困,鬻其園,寄身僧舍,戚友贖而歸之,始復與妻居,時年七十矣。子震,亦棄諸生,非義一介不取。老夫婦白頭相對,時絕食,則嘆曰:「吾生本贅耳,待盡而已。」有餽食者,非其人,終不受。或問身後,曰:「楊王孫之葬,何必棺也!」

又十年,蜃園僅存二楹,兩耳聾,又苦腹疾,終日仰臥。客至,以粉版書相問荅。魏禧來自江西,造其廬,天植與之粉版,書竟,天植視姓字,則強起張目視之,泣,禧亦泣。時方絕糧,禧探囊得銀半兩贈之,五反不受,固以請,曰:「此非盜跖物也!」始納之。買米為炊,共食而別。禧囑布衣周筼、侍郎曹溶糾同志為繼粟,且謀身後事,徐枋聞之曰:「李先生不食人食,聽其以餓死可也。」已而筼齎粟往,天植果堅拒。禧聞之,曰:「吾淺之乎為丈夫已。」乍浦有鄭嬰垣者,孤介絕俗,與天植稱金石交,先二年,凍死雪中,至是天植亦饑死。臨歾,曰:「吾無愧於老友矣!」時康熙十一年也。年八十有二。葬牛橋。所著有蜃園集、乍浦九山志。

理洪儲,字繼起,興化人。本姓李。父嘉兆與中州理鬯和恥與賊同姓,皆改理氏,天下稱「二理」。洪儲早歲出家,南都覆,明之遺臣多舉兵,洪儲左右之,被逮,獲免,好事如故。人戒之,則曰:「吾茍自反無媿,即有意外風波,久當自定。」又曰:「憂患得其宜,湯火亦樂國也。」枋聞之,嘆曰:「是真能以忠孝作佛事者也!」洪儲在沙門,宏暢宗風,篤好人物,海內皆能道之。枋曰:「此其跡也,但觀其每年三月十九日素服焚香,北面揮涕,二十八年如一日,是何為者?」

顧柔謙,字剛中,無錫人,遷常熟。幼遭家難,貲產皆盡。嘗同兄出門游,有數人擁之行,行乃擠大澤中。母忽心動,急呼老僕往跡之,得不死。補弟子員。甲申之變,柔謙哀憤,往往形諸詩歌,讀者悲之。不妄交游,以父執師事馬士奇,而江陰黃毓祺、嘉定黃淳耀皆一見定交。諸人殉國難,柔謙皆設位以哭盡哀。子祖禹,見父嘗閉門嘿坐,或竟日不食,祖禹叩頭寬譬,柔謙乃曰:「汝能終身窮餓,不思富貴乎?」祖禹跪應曰:「能。」柔謙曰:「汝能以身為人機上肉,不思報復乎?」祖禹復應曰:「能。」柔謙喜曰:「吾與汝偕隱耳!」遂更名隱,署其室曰伐檀。常夜蹴祖禹曰:「汝他日得志,如舊怨何?」祖禹曰:「每憶幼時祖母抱兒置膝上,為言家難,及墮大澤中事,祖禹不敢忘。」柔謙曰:「嘻,汝何見之隘?吾家數傳以來,頗盈盛,以祖、父之才,而竟中折,天也!於彼何尤?同室之中,寧彼以非禮來,吾不可以非禮報,汝謹識之!」著有補韻略、六書考定、山居贅論。

祖禹,字復初。柔謙精於史學,嘗謂:「明一統志於戰守攻取之要,類皆不詳山川,條列又復割裂失倫,源流不備。」祖禹承其志,撰讀史方輿紀要一百三十卷,凡職方、廣輿諸書,承譌襲謬,皆為駮正。詳於山川險易,及古今戰守成敗之跡,而景物名勝皆在所略。創稿時年二十九,及成書,年五十矣。寧都魏禧見之,嘆曰:「此數千百年絕無僅有之書也!」以其書與梅文鼎歷算全書、李清南北史合鈔稱三大奇書。祖禹與禧為金石交,禧客死,祖禹經紀其喪。徐乾學奉敕修一統志,延致祖禹,將薦起之,力亂罷。後終於家。

冒襄,字闢疆,別號巢民,如皋人。父起宗,明副使。襄十歲能詩,董其昌為作序。崇禎壬午副榜貢生,當授推官,會亂作,遂不出。與桐城方以智、宜興陳貞慧、商丘侯方域,並稱「四公子」。襄少年負盛氣,才特高,尤能傾動人。嘗置酒桃葉渡,會六君子諸孤,一時名士咸集。酒酣,輒發狂悲歌,訾詈懷寧阮大鋮,大鋮故奄黨也。時金陵歌舞諸部,以懷寧為冠,歌詞皆出大鋮。大鋮欲自結諸社人,令歌者來,襄與客且罵且稱善,大鋮聞之益恨。甲申黨獄興,襄賴救僅免。家故有園池亭館之勝,歸益喜客,招致無虛日,家自此中落,怡然不悔也。

襄既隱居不出,名益盛。督撫以監軍薦,御史以人才薦,皆以親老辭。康熙中,復以山林隱逸及博學鴻詞薦,亦不就。著述甚富,行世者,有先世前徽錄,六十年師友詩文同人集,樸巢詩文集,水繪園詩文集。書法絕妙,喜作擘大字,人皆藏★M8珍之。康熙三十二年,卒,年八十有三。私謚潛孝先生。

陳貞慧,字定生,宜興人,明都御史陳於廷子。於廷,東林黨魁。貞慧與吳應箕草留都防亂檄,擯阮大鋮。黨禍起,逮貞慧至鎮撫司,事雖解,已瀕十死。國亡,埋身土室,不入城市者十餘年。遺民故老時時向陽羨山中一問生死,流連痛飲,驚離吊往,聞者悲之。順

治十三年,卒,年五十三。著有皇明語林、山陽錄、雪岑集、交游錄、秋園雜佩諸書。子維崧,見文苑傳。

祁班孫,字奕喜,山陰人。父彪佳,明蘇松巡撫。班孫次六,人稱六公子,彪佳嘗受業於劉宗周,宗周將兵江上,班孫與其兄理孫罄家餉之。祁氏藏書甲江左,班孫兄弟以故國喬木自任。豪宕喜結客,家居山陰之梅墅,園林深茂。登其堂,衣復壁大隧,莫能詰也。慈谿布衣魏耕者,狂走四方,思得一當。班孫兄弟與之誓天,稱莫逆。或告變於浙大吏,四道捕耕,並縛班孫兄弟去。既讞,兄弟爭承,祁氏客乃納賂而宥其兄。班孫遣戍遼左,理孫竟以痛弟鬱鬱死,而祁氏家亦破。

旋班孫遁歸,祝發於吳之堯峰,尋主毗陵馬鞍山寺,所稱咒林明大師者也。班孫好議論古今,不談佛法,每語及先朝,則掩面哭,然終莫有知之者。康熙十二年,卒。發其篋,有東行風俗記、紫芝軒集。且得其遺教,命歸祔,乃知為山陰祁六公子,遂得返葬云。

班孫娶少師硃燮元女孫,硃工詩。其來歸也,與其姑商、姒張、小姑湘君,時相唱和。商氏字塚婦曰楚纕,字介婦曰趙璧,以志閨門之盛。班孫既被難,硃盛年,孤燈緇帳,數十年未嘗一出屏。自班孫兄弟歾,淡生堂書星散,論者謂江東文獻一大厄運也。

汪渢,字魏美,錢塘人。少孤貧,力學,與人落落寡諧,人號曰汪冷。舉崇禎己卯鄉試,與同縣陸培齊名。甲申後,培自經死,渢為文祭之,一慟幾絕,遂棄科舉。★L5黨欲強之試禮部,出千金兒其妻,俾勸駕,妻曰:「吾夫子不可勸,吾亦不屑此金也。」嘗獨身提藥裹往來山谷間,宿食無定處。渢故城居,母老,欲時時見渢,其兄澄、弟澐亦棄諸生服,奉母徙城外。渢時來定省,然渢能自來,家人欲往跡之,不可得。

嗣因兵亂,奉母入天臺。海上師起,群盜滿山谷,復返錢塘。當是時,湖上有三孝廉,皆高士,渢其一也,當事皆重之。監司盧高尤下士,一日,遇渢於僧舍,問:「汪孝廉何在?」渢應曰:「適在此,今已去矣。」高悵然,不知應者即渢也。高嘗艤舟載酒西湖上,約三高士以世外禮相見,惟渢不至。已,知其在孤山,以船就之,排墻遁去。渢不入城市,有司或以俸金為壽,不得卻,坎而埋之。里貴人請墓銘,饋百金,拒弗納。徙居孤山,匡床布被外,殘書數卷,鍵戶出,或返或不返,莫可蹤跡。遇好友,飲酒一斗不醉。

晚好道,夜觀天象,晝習壬遁,能數日不食,了不問世事。黃宗羲遇之於孤山,講龍溪調息法。嘗坐月至三更,夜寒甚,止布被一,渢與宗羲背相摩,得少暖氣。魏禧自江西來訪,謝弗見。禧留書曰:「吾寧都魏禧也,欲與子握手一痛哭耳!」渢省書大驚,一見若平生歡。臨別,執手涕下。渢嘗從愚菴和尚究出世法,禧曰:「君事愚菴謹,豈有意為其弟子耶?」渢曰:「吾甚敬愚菴,然今之志士,多為釋氏牽去,此吾所以不屑也。」康熙四年秋,終於寶石山僧舍,年四十有八。臨歾,舉書卷焚之,詩文無一存者。起視日影,曰:「可矣!」書五言詩一章,投筆就寢而逝。渢與陳廷會、柴紹炳、沈昀、孫治人,稱「西陵五君子」。

餘增遠,字謙貞,世稱若水先生,會稽人。明崇禎十六年進士,除寶應知縣。南都授禮部主事,遷郎中。事敗,逃之山中。郡縣逼之出見,乃輿疾城南,以死拒。久之,事得解。草屋三間,不蔽風雨,以鱉甲承漏。聚村童五六人,授以三字經。臥榻之下,牛宮雞,無下足處。晨則秉耒出,與老農雜作。同年生王天錫為海防道,欲與話舊,以疾辭。天錫披帷直入,增遠擁衾不起,曰:「不幸有狗馬疾,不得與故人為禮。」天錫執手勞苦,出間未數武,則已與一婢子擔糞灌園矣。天賜遙望見之,嘆息去。冬夏一皁帽,雖至暱者,不見其科頭。增遠慨世路偪仄,遂疑荀卿性惡之說為確,至欲著論以非孟。康熙八年,卒,年六十有五。蓋二十有四年不出城南一步也。疾革,黃宗羲造其榻前,欲為切脈,增遠笑曰:「某祈死二十年前,反祈生二十年後乎?」宗羲泫然而別。

同時有周齊曾者,字思沂,號唯一,鄞人,增遠同年進士也。知廣東順德縣事,變社倉為義田,而以社倉之法行之。國變後,棄官遯入剡源,盡去其發為發塚,架險立飄榜,曰「囊云」,自稱無發居士。剡源饒水石,與山僧樵子出沒瀑聲虹影間。天錫訪之,拒曰:「咫尺清輝,舉目有山河之異,不原見也。」為詩文,機鋒電激,汪洋自恣,寓言十九。然清苦自立,胸中兀然有所不可,與增遠無二也。黃宗羲嘗為兩人合志其墓云。

傅山,字青主,陽曲人。六歲,啖黃精,不穀食,強之,乃飯。讀書過目成誦。明季天下將亂,諸號為搢紳先生者,多迂腐不足道,憤之,乃堅苦持氣節,不少媕冘。提學袁繼咸為巡按張孫振所誣,孫振,閹黨也。山約同學曹良直等詣通政使,三上書訟之,巡撫吳甡亦直袁,遂得雪。山以此名聞一下,甲申後,山改黃冠裝,衣硃衣,居土穴,以養母。繼咸自九江執歸燕邸,以難中詩遺山,且曰:「不敢媿友生也!」山省書,慟哭,曰:「嗚呼!吾亦安敢負公哉!」

順治十一年,以河南獄牽連被逮,抗詞不屈,絕粒九日,幾死。門人中有以奇計救之,得免。然山深自吒恨,謂不若速死為安,而其仰視天、俯視地者,未嘗一日止。比天下大定,始出與人接。

康熙十七年,詔舉鴻博,給事中李宗孔薦,固辭。有司強迫,至令役夫舁其床以行。至京師二十里,誓死不入。大學士馮溥首過之,公卿畢至,山臥床不具迎送禮。魏象樞以老病上聞,詔免試,加內閣中書以寵之。馮溥強其入謝,使人舁以入,望見大清門,淚涔涔下,僕於地。魏象樞進曰:「止,止,是即謝矣!」翼日歸,溥以下皆出城送之。山嘆曰:「今而後其脫然無累哉!」既而曰:「使後世或妄以許衡、劉因輩賢我,且死不瞑目矣!」聞者咋舌。至家,大吏咸造廬請謁。山冬夏著一布衣,自稱曰「民」。或曰:「君非舍人乎?」不應也。卒,以硃衣、黃冠斂。

山工書畫,謂:「書寧拙毋巧,寧丑毋媚,寧支離毋輕滑,寧真率毋安排。」人謂此言非止言書也。詩文初學韓昌黎,崛強自喜,後信筆抒寫,俳調俗語,皆入筆端,不原以此名家矣。著有霜紅龕集十二卷。子眉,先卒,詩亦附焉。

眉,字壽髦。每日出樵,置書擔上,休則把讀。山常賣藥四方,與眉共挽一車,暮抵逆旅,篝燈課經,力學,繼父志。與客談中州文獻,滔滔不盡。山喜苦酒,自稱老糵禪,眉乃稱小糵禪。

費密,字此度,新繁人。父經虞,明雲南昆明縣知縣。密年十四,父病,醫言嘗糞甘苦,可知生死,密嘗而苦,父病果起。未幾,流賊張獻忠犯蜀,密上書巡按御史劉之勃,陳戰守策,不省。已而全蜀皆陷,密展轉窮山中,會有人傳其父滇中消息,聞之痛哭,遂去家入滇。經歷蠻峒中,奉父自滇歸蜀。至建昌衛,為凹者蠻所得,父賂蠻人,始脫歸。

明將楊展聞密名,遣使致聘,密乃說展曰:「賊亂數年,民且無食,今非屯田,無以救蜀民,且兵不能自立。」展納其言,命子總兵官璟偕密屯田於榮經瓦屋山之楊村,以次舉其法,行諸州縣。後展為袁韜、武大定所殺,密與璟整師為復仇計,嘗與賊戰,躬自擐甲,左手為刃所傷。時璟營於峨眉,裨將有與花溪民毆爭者,言「花溪居民下石擊吾營,勢且反」以怒璟。璟欲引兵誅之,密力爭曰:「花溪,吾民也。方與賊戰而殺吾民,彼變從賊,是益賊也。」璟乃止,全活數百家。

後密還成都省墓,至新津,為武大定兵所掠。知密嘗參展軍事,欲殺之,以計得免。密嘆曰:「既不能報國,又不能庇親及身,不如舍而他去!」遂奉父由成都北行入秦,溯漢江,下吳、越,流寓泰州,老焉。

經虞邃於經學,嘗著毛詩廣義、雅論諸書,以漢儒注說為宗。密盡傳父業,又博證學士大夫,與王復禮、毛甡、閻若璩交,密一足跛,後往蘇門謁孫奇逢,稱弟子。工詩、古文,俯仰取給於授徒、賣文,人咸重其品,悲其遇。州守為之除徭役,杜門三十年,著書甚多。

密謂宋人以周、程接孔、孟,盡黜二千餘年儒者為未聞道,乃上稽古經、正史,旁及群書,作中傳正紀百二十卷,序儒者授受源流,自子夏始。又作弘道書十卷、古今篤論四卷、中旨定錄四卷、中旨辨錄四卷、中旨申感四卷,皆申明弘道書之旨。又有尚書說、周官注論、二南偶說、中庸大學駮議、四禮補篇、史記箋、古史正、歷代貢舉合議、費氏家訓及詩文集。卒,年七十七。子錫琮、錫璜,世其學。

王弘撰,字無異,號山史,華陰人。明諸生。博雅能古文,嗜金石,藏古書畫金石最富。又通濂、洛、關、閩之學,好易,精圖象。學者翕然宗之,關中入士領袖也。與李顒、李柏、李因篤齊名,時以得一言為榮。凡碑版銘志非三李則弘撰,而弘撰工書法,故求者多於三李。弘撰交游遍天下,甲申後,奔走結納,尤著志節。

顧炎武遍觀四方,至華陰,謂秦人慕經學、重處士、持清議,他邦所少;華陰綰轂之口,雖足不出戶,而能見天下之人,聞天下之事。欲定居,弘撰為營齋舍居之。炎武嘗曰:「好學不倦,篤於朋友,吾不如王山史。」當時儒碩遺逸皆與弘撰往還,頗推重之。弘撰嘗集炎武及孫枝蔚、閻爾梅等數十人所與書札,合為一冊,手題曰友聲集,各注姓氏。中有為謀炎武卜居華下事,言:「此舉大有關系,世道人心,實皆攸賴,唯速圖之!」蓋當日華下集議,實有所為也。

康熙間,以鴻博徵,不赴。初與因篤同學,甚密,及因篤就徵,遂與之絕。弘撰所居華山下,有讀易廬,與華峰相向,稱絕勝。卒,年七十有五。著有易象圖說、山志、砥齋集。

杜濬,字於皇,號茶村,黃岡人。明季為諸生,避亂居金陵。少倜儻,嘗欲著奇節,既不得試,遂刻意為詩,然不欲以詩人自名也。於並世人獨重宣城沈壽民、吳中徐枋,自媿不如。其在金陵,與方仲舒善,仲舒,苞父也。金陵冠蓋輻輳,諸公貴人求詩者踵至,多謝絕。錢謙益嘗造訪,至閉門不與通,惟故舊徒步到門,則偶接焉。門內為竹關,關外設坐,約客至,視鍵閉,則坐而待,不得叩關,雖大府至,亦然。及功令有挑門之役,有司按籍欲優免,濬曰:「是吾所服也!」躬雜廝輿夜巡綽,眾莫能止。嗜茗飲,嘗言吾有絕糧,無絕茶。既有花塚,因拾殘茗聚封之,謂之「茶丘」。年七十七,卒於揚州。

喪歸,故人謀卜兆,子世濟曰:「吾有親,而以葬事辱二三君乎?是謂我非人也。」亡何,世濟卒。又數年,陳鵬年來守金陵,始葬諸蔣山北梅花村。

濬詩最富,世所傳不及十一,手定者四十七冊。吳偉業嘗云:「吾五言律得茶村焦山詩而始進。」閻若璩於時賢多所訾謷,獨許濬五律,稱為「詩聖」。已刻者曰變雅堂集。

弟岕,字蒼略,號些山。諸生。與兄同避亂金陵。昆弟行身略同,而趣各異。濬峻廉隅,孤特自遂。遇名貴人,必以氣折之,於眾人未嘗接語言,用此叢忌嫉。然名在天下,詩每出,遠近爭傳誦之。岕則退然自同於眾,所著詩歌、古文,雖子弟弗示也。方壯喪偶,不復娶。所居室漏且穿,木榻敝帷,數十年未嘗易。室中終歲不掃除,每日中不得食,兒女啼號,客至無酒漿,意色間無幾微不自適者。行於途,常避人,不中道與人言,雖兒童廝輿,惟恐或傷之也。後兄七年卒,年七十七。有些山集。

郭都賢,字天門,益陽人。天啟壬戌進士,授行人。分校順天鄉試,得史可法等六人。歷官員外郎,出為四川參議,督江西學政,分守嶺北道,巡撫江西。時張獻忠已逼境,賊騎充斥。都賢晝夜繕守御,兵餉無措,乃大會屬僚,凡官司一應供給,皆捐以助餉。左良玉屯兵九江,驕蹇觀望,都賢惡其淫掠,檄歸之,而募士兵為戍。會有尼之者,遂乞病,棄官入廬山。逾年,北京陷,悲憤不食。南都建號,史可法開閫揚州,薦授以官,辭不赴。桂王立肇慶,以兵部尚書召,而都賢已祝發為僧矣。先是洪承疇坐事落職,都賢奏請起用,至是承疇經略西南,以故舊謁都賢於山中,餽以金,不受;奏攜其子監軍,亦堅辭。都賢見承疇時,故作目瞇狀,承疇驚問何時得目疾,都賢曰:「始吾識公時,目故有疾。」承疇默然。

都賢篤至性,哀樂過人,嚴而介,風骨嶄然。博學強識,工詩文,書法瘦硬,兼善繪事,寫竹尤入妙。僧號頑石,又號些菴。茹苦,無定居。初依熊開元、尹民興於嘉魚,住梅熟菴;已,流寓海陽,築補山堂:前後十九年。歸結草廬桃花江。客死江寧承天寺。

有女名純貞,許字黔國公沐氏,變後,音問梗絕,遂終於家。純貞能詩,自署曰郭貞女。

都督所著有衡岳集、止菴集、秋聲吟、西山片石集、破草奚集、補山堂集、些菴雜著等書。

陶汝鼐,字仲調,一字密菴,寧鄉人。與都賢交最篤。崇禎初,充拔貢生。會帝幸太學,群臣請復高皇積分法,祭酒顧錫疇奏薦汝鼐才,特賜第一,詔題名勒石太學。除五品官,不拜,乞留監肄業。癸酉舉於鄉,兩中會試副榜。南渡後,薙發溈山,號忍頭陀。生平內行篤,父歾,哀慕終身。事母曲盡孝養,處族黨多厚德,嘗為人雪奇冤,冒險難,活千餘人,然不自言也。詩古文有奇氣,著有廣西涯樂府、古集、寄雲樓集、褐玉堂集、嘉樹堂集,都賢為序而行之。有「生同里、長同學、出處患難同時同志」之語。

李世熊,字元仲,寧化人。明諸生。少負奇氣,植大節,更危險,死生弗渝。篤交游,敢任難事。生平喜讀異書,博聞強記。年八十,讀書恆至夜分始休。六經、諸子百家靡不貫究,然獨好韓非、屈原、韓愈之書。其為文,沉深峭刻,奧博離奇,悲憤之音,稱其所遇。縱

論古今興亡,儒生出處,及江南北利害,備兵屯田水利諸大政,輒慷慨欷歔,涔涔泣下不止。年十六,補弟子員,旋中天啟元年副榜,以興化司李佘昌祚得其文,爭元於主司弗得,袖其卷去,曰:「須後作元也。」典閩試者,爭欲物色之為重。

甲申後,自號寒支道人,屏居不見客。徵書累下,固謝卻之。凡守、令、監司、鎮將至其門者,罕能一識面。閩中擁唐王監國,用大學士黃道周、禮部侍郎曹學佺、都察院何楷薦,徵拜翰林博士,辭不赴。嘗上書道周,感憤時事。及道周殉節,走福州請褒恤,時恤問其孤嫠。

順治初,師入閩,有齮齕於郡帥者,帥遣某生移書,逼入都,且言:「不出山,禍不測。」世熊復之曰:「死生有命,豈遂懸於要津之手?且某年四十八矣,諸葛瘁躬之日,僅少一年;文山盡節之辰,已多一歲。何能抑情違性,重取羞辱哉!」時蜚語騰沸,世熊矢死不為動,疑謗旋亦釋。

世熊既以文章氣節著一時,名大震。辛卯、壬辰間,建昌潰賊黃希孕剽掠過寧化,有卒摘其園中二橘,希孕立鞭之,駐馬園側,視卒盡過乃行。粵寇至,燔民屋,火及其園,賊魁劉大勝遣卒撲救之,曰:「奈何壞李公居?」當時雖匹夫匹婦,無不知有寒支子者。

世熊積壘塊胸中,每放浪山水,以寫其牢騷不平之概。嘗詣西江,交魏禧、魏禮、彭士望諸子,相與泛彭蠡,登廬山絕頂。追維闖賊橫行時事,痛悼如絕,淚下如泉湧,不能禁也。耿精忠反,遣偽使敦聘,世熊嚴拒之。自春徂冬,堅臥不起,乃得免。世熊山居四十餘年,鄉人宗之,爭趨決事。有為不善者,曰:「不使李公知也。」晚自號媿菴,顏其齋曰「但月」。所著有寒支集、寧化縣志、本行錄、經正錄、狗馬史記等。年八十五,卒於家。

世熊有三弟,早世,遺子女,撫育裝遣之。饋遺其親戚終身。又獨建祖祠,修祖墓,編述九世以來宗譜。凡祭祀,必親必謹。父母忌日,則減餐絕宴會。元旦,展先人遺像,則泣下沾襟,拜伏不能起,蓋其孝友出於天性云。

談遷,字孺木,原名以訓,海寧人。初為諸生。南都立,以中書薦,召入史館,皆辭,曰:「餘豈以國家之不幸博一官耶?」未幾,歸里。遷肆力經史百家言,尤注心於明朝典故。嘗謂:「史之所憑者,實錄耳。實錄見其表,其在里者,已不可見。況革除之事,楊文貞未免失實;泰陵之盛,焦泌陽又多醜正;神、熹之載筆者,皆逆奄之舍人。至於思陵十七年之憂勤惕厲,而太史遯荒,皇宬烈焰,國滅而史亦隨滅,普天心痛,莫甚於此!」乃汰十五朝實錄,正其是非。訪崇禎十七年邸報,補其缺文,成書,名曰國榷。

當是時,人士身經喪亂,多欲追敘緣因,以顯來世,而見聞窄狹,無所憑藉。聞遷有是書,思欲竊之為己有。遷家貧,不見可欲者,夜有盜入其室,盡發藏橐以去。遷喟然曰:「吾手尚在,寧遂已乎?」從嘉善錢氏借書復成之。陽城張慎言目為奇士,折節下之。慎言卒,遷方北走昌平,哭思陵,復欲赴陽城哭慎言,未至而卒,順治十二年冬十一月也。黃宗羲為表其墓。

明末遺逸,守志不屈,身雖隱而心不死,至事不可為,發憤著書,欲託空文以見志,如遷者,其憂憤豈有已耶?故以附於各省遺逸之末。


\end{pinyinscope}