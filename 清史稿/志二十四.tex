\article{志二十四}

\begin{pinyinscope}
時憲五

△康熙甲子元法下

月食用數

朔策二十九日五三0五九三。

望策十四日七六五二九六五。

太陽平行,朔策一十萬四千七百八十四秒,小餘三0四三二四。

太陽引數,朔策一十萬四千七百七十九秒,小餘三五八八六五。

太陰引數,朔策九萬二千九百四十秒,小餘二四八五九。

太陰交周,朔策十一萬0四百十四秒,小餘0一六五七四。

太陽平行,望策十四度三十三分十二秒0九微。

太陽引數,望策十四度三十三分0九秒四十一微。

太陰引數,望策六宮十二度五十四分三十秒0七微。

太陰交周,望策六宮十五度二十分0七秒。

太陽一小時平行一百四十七秒,小餘八四七一0四九。

太陽一小時引數一百四十七秒,小餘八四0一二七。

太陰一小時引數一千九百五十九秒,小餘七四七六五四二。

太陰一小時交周一千九百八十四秒,小餘四0二五四九。

月距日一小時平行一千八百二十八秒,小餘六一二一一0八。

太陽光分半徑六百三十七。

太陰實半徑二十七。

地半徑一百。

太陽最高距地一千0十七萬九千二百0八,與地半徑之比例,為十一萬六千二百。

太陰最高距地一千0十七萬二千五百,與地半徑之比例,為五千八百一十六。

朔應二十六日三八五二六六六。

首朔太陽平行應初宮二十六度二十分四十二秒五十七微。

首朔太陽引數應初宮十九度一十分二十七秒二十一微。

首朔太陰引數應九宮十八度三十四分二十六秒十六微。

首朔太陰交周應六宮初度三十分五十五秒十四微,餘見日躔、月離。

推月食法

求天正冬至,同日躔。

求紀日,以天正冬至日數加一日,得紀日。

求首朔,先求得積日同月離。置積日減朔應,得通朔。上考則加。以朔策除之,得數加一為積朔。餘數轉減朔策為首朔。上考則除得之數即積朔,不用加一。餘數即首朔,不用轉減。

求太陰入食限,置積朔,以太陰交周朔策乘之,滿周天秒數去之,餘為積朔太陰交周。加首朔太陰交周應,得首朔太陰交周。上考則置首朔交周應減積朔交周。又加太陰交周望策,再以交周朔策遞加十三次,得逐月望太陰平交周。視某月交周入可食之限,即為有食之月。交周自五宮十五度0六分至六宮十四度五十四分,自十一宮十五度0六分至初宮十四度五十四分,皆可食之限。再於實交周詳之。

求平望,以太陰入食限月數與朔策相乘,加望策,再加首朔日分及紀日,滿紀法去之,餘為平望日分。自初日起甲子,得平望干支,以刻下分通其小餘,如法收之。初時起子正,得時刻分秒。

求太陽平行,置積朔,加太陰入食限之月數為通月,以太陽平行朔策乘之。滿周天秒數去之,加首朔太陽平行應,上考則減。又加太陽平行望策,即得。

求太陽平引,置通月,以太陽引數朔策乘之,去周天秒數,加首朔太陽引數應,上考則減。又加太陽引數望策,即得。

求太陰平引,置通月,以太陰引數朔策乘之,去周天秒數,加首朔太陰引數應,上考則減。又加太陰引數望策,即得。

求太陽實引,以太陽平引,依日躔法求得太陽均數,以太陰平引,依月離法求得太陰初均數,兩均數相加減為距弧。兩均同號相減,異號相加。以月距日一小時平行為一率,一小時化秒為二率,距弧化秒為三率,求得四率為距時秒,隨定其加減號。兩均同號,日大仍之,日小反之;兩均一加一減,其加減從日。又以一小時化秒為一率,太陽一小時引數為二率,距時秒為三率,求得四率為秒。以度分收之,為太陽引弧。依距時加減號。以加減太陽平引,得實引。

求太陰實引,以一小時化秒為一率,太陰一小時引數為二率,距時秒為三率,求得四率為秒。以度分收之,為太陰引弧。依距時加減號。以加減太陰平引,得實引。

求實望,以太陽實引復求均數為日實均,並求得太陽距地心線。即實均第二平三角形對正角之邊。以太陰實引復求均數為月實均,★求得太陰距地心線。法同太陽。兩均相加減為實距弧。加減與距弧同。依前求距時法,求得時分為實距時,以加減平望,加減與距時同。得實望。加滿二十四時,則實望進一日,不足減者,借一日作二十四時減之,則實望退一日。

求實交周,以一小時化秒為一率,太陰一小時交周為二率,實距時化秒為三率,求得四率為秒,以度分收之,為交周距弧。以加減太陰交周,依實距時加減號。又以月實均加減之,為實交周。若實交周入必食之限,為有食。自五宮十七度四十三分0五秒至六宮十二度十六分五十五秒,自十一宮十七度四十三分0五秒至初宮十二度十六分五十五秒,為必食之限。不入此限者,不必布算。

求太陽黃赤道實經度,以一小時化秒為一率,太陽一小時平行為二率,實距時化秒為三率,求得四率為秒,以度分收之,為太陽距弧。依時距時加減號。以加減太陽平行,又以日實均加減之,即黃道經度。又用弧三角形求得赤道經度。詳月離求太陰出入時刻條。

求實望用時,以日實均變時為均數時差,以升度差黃赤道經度之較。變時為升度時差,兩時差相加減為時差總,加減之法,詳月離求用時平行條。以加減實望,為實望用時。距日出後日入前九刻以內者,可以見食。九刻以外者全在晝,不必算。

求食甚時刻,以本天半徑為一率,黃白大距之餘弦為二率,實交周之正切為三率,求得四率為正切,檢表得食甚交周。與實交周相減,為交周升度差。又以太陰一小時引數與太陰實引相加,依月離求初均法算之,為後均。以後均與月實均相加減,兩均同號相減,異號相加。得數又與一小時月距日平行相加減,兩均同加,後均大則加,小則減。兩均同減,後均大則減,小則加。兩均一加一減,其加減從後均。為月距日實行。乃以月距日實行化秒為一率,一小時化秒為二率,交周升度差化秒為三率,求得四率為秒。以時分收之,得食甚距時。以加減實望用時,實交周初宮六宮為減,五宮十一宮為加。為食甚時刻。

求食甚距緯,以本天半徑為一率,黃白大距之正弦為二率,實交周之正弦為三率,求得四率為正弦,檢表得食甚距緯。實交周初宮五宮為北,六宮十一宮為南。

求太陰半徑,以太陰最高距地為一率,地半徑比例數為二率,太陰距地心線內減去次均輪半徑為三率,求得四率為太陰距地。又以太陰距地為一率,太陰實半徑為二率,本天半徑為三率,求得四率為正弦。檢表得太陰半徑。

求地影半徑,以太陽最高距地為一率,地半徑比例數為二率,太陽距地心線為三率,求得四率為太陽距地。又以太陽光分半徑內減地半徑為一率,太陽距地為二率,地半徑為三率,求得四率為地影之長。又以地影長為一率,地半徑為二率,本天半徑為三率,求得四率為正弦,檢表得地影角。又以本天半徑為一率,地影角之正切為二率,地影長內減太陰距地為三率,求得四率為太陰所入地影之闊。乃以太陰距地為一率,地影之闊為二率,本天半徑為三率,求得四率為正切,檢表得地影半徑。

求食分,以太陰全徑為一率,十分為二率,並徑太陰地影兩半徑相並。內減食甚距緯之較並徑不及減距緯即不食。為三率,求得四率即食分。

求初虧、復圓時刻,以食甚距緯之餘弦為一率,並徑之餘弦為二率,半徑千萬為三率,求得四率為餘弦,檢表得初虧、復圓距弧。又以月距日實行化秒為一率,一小時化秒為二率,初虧、復圓距弧化秒為三率,求得四率為秒。以時分收之,為初虧、復圓距時。以加減食甚時刻,得初虧、復圓時刻。減得初虧,加得復圓。

求食既、生光時刻,以食甚距緯之餘弦為一率,兩半徑較之餘弦為二率,半徑千萬為三率,求得四率為餘弦,檢表得食既、生光距弧。又以月距日實行化秒為一率,一小時化秒為二率,食既、生光距弧化秒為三率,求得四率為秒。以時分收之,為食既、生光距時。以加減食甚時刻,得食既、生光時刻。減得食既,加得生光。

求食限總時,以初虧、復圓距時倍之,即得。

求太陰黃道經緯度,置太陽黃道經度,加減六宮,過六宮則減去六宮,不及六宮,則加六宮。再加減食甚距弧,又加減黃白升度差,求升度差法,詳月離求黃道實行條。得太陰黃道經度。求緯度,詳月離。

求太陰赤道經緯度,詳月離求太陰出入時刻條。

求宿度,同日躔。

求黃道地平交角,以食甚時刻變赤道度,每時之四分變一度。又於太陽赤道經度內減三宮,不及減者,加十二宮減之。餘為太陽距春分赤道度。兩數相加,滿全周去之。為春分距子正赤道度。與半周相減,得春分距午正東西赤道度。過半周者,減去半周,為午正西。不及半周者,去減半周,為午正東。春分距午正東西度過象限者,與半周相減,餘為秋分距午正東西赤道度。秋分距午東西,與春分相反。以春秋分距午正東西度與九十度相減,餘為春秋分距地平赤道度。乃用為弧三角形之一邊,以黃赤大距及赤道地平交角即赤道地平上高度,春分午西、秋分午東者用此。若春分午東、秋分午西者,則以此度與半周相減用其餘。為邊傍之兩角,求得對邊之角,為黃道地平交角。春分午東、秋分午西者,得數即為黃道地平交角。春分午西、秋分午東者,則以得數與半周相減,餘為黃道地平交角。

求黃道高弧交角,以黃道地平交角之正弦為一率,赤道地平交角之正弦為二率,春秋分距地平赤道度之正弦為三率,求得四率為正弦,檢表得春秋分距地平黃道度。又視春秋分在地平上者,以太陰黃道經度與三宮、九宮相減,春分與三宮相減,秋分與九宮相減。餘為太陰距春秋分黃道度。春秋分宮度大於太陰宮度,為距春秋分前;反此則在後。又以春秋分距地平黃道度與太陰距春秋分黃道度相加減,為太陰距地平黃道度,春秋分在午正西者,太陰在分後則加,在分前則減;春秋分在午正東者反是。隨視其距限之東西。春秋分在午正西者,太陰距地平黃道度不及九十度為限西,過九十度為限東;春秋分在午正東者反是。乃以太陰距地平黃道度之餘弦為一率,本天半徑為二率,黃道地平交角之餘切為三率,求得四率為正切,檢表得黃道高弧交角。

求初虧、復圓定交角,置食甚交周,以初虧、復圓距弧加減之,得初虧、復圓交周。減得初虧,加得復圓。乃以本天半徑為一率,黃白大距之正弦為二率,初虧交周之正弦為三率,求得四率為正弦,檢表得初虧距緯。又以復圓交周之正弦為三率,一率二率同前。求得四率為正弦,檢表得復圓距緯。交周初宮、五宮為緯北,六宮、十一宮為緯南。又以並徑之正弦為一率,初虧、復圓距緯之正弦各為二率,半徑千萬為三率,各求得四率為正弦,檢表得初虧、復圓兩緯差角。以兩緯差角各與黃道高弧交角相加減,得初虧、復圓定交角。初虧限東,緯南則加,緯北則減;限西,緯南則減,緯北則加。復圓反是。若初虧、復圓無緯差角,即以黃道高弧交角為定交角。

求初虧、復圓方位,食在限東者,定交角在四十五度以內,初虧下偏左,復圓上偏右。四十五度以外,初虧左偏下,復圓右偏上。適足九十度,初虧正左,復圓正右。過九十度,初虧左偏上,復圓右偏下。食在限西者,定交角四十五度以內,初虧上偏左,復圓下偏右。四十五度以外,初虧左偏上,復圓右偏下。適足九十度,初虧正左,復圓正右。過九十度,初虧左偏下,復圓右偏上。京師黃平象限恆在天頂南,定方位如此。在天頂北反是。

求帶食分秒,以本日日出或日入時分初虧或食甚在日入前者,為帶食出地,用日入分。食甚或復圓在日出後者,為帶食入地,用日出分。與食甚時分相減,餘為帶食距時。以一小時化秒為一率,一小時月距日實行化秒為二率,帶食距時化秒為三率,求得四率為秒。以度分收之,為帶食距弧。又以半徑千萬為一率,帶食距弧之餘切為二率,食甚距緯之餘弦為三率,求得四率為餘切,檢表得帶食兩心相距之弧。乃以太陰全徑為一率,十分為二率,並徑內減帶食兩心相距之餘為三率,求得四率,即帶食分秒。

求各省月食時刻,以各省距京師東西偏度變時,每偏一度,變時之四分。加減京師月食時刻,即得。東加,西減。

求各省月食方位,以各省赤道高度及月食時刻,依京師推方位法求之,即得。

繪月食圖,先作橫★二線,直角相交,橫★當黃道,★線當黃道經圈,用地影半徑度於中心作圈以象闇虛。次以並徑為度作外虛圈,為初虧、復圓之限。又以兩徑較為度作內虛圈,為食既、生光之限。復於外虛圈上周★線或左或右,取五度為識,視實交周初宮、十一宮作識於右,五宮、六宮作識於左。乃自所識作線過圈心至外虛圈下周,即為白道經圈。於此線上自圈心取食甚距緯作識,即食甚月心所在。從此作十字橫線,即為白道。割內外虛圈之點,為食甚前後四限月心所在。末以月半徑為度,於五限月心各作小圈,五限之象具備。

日食用數

太陽實半徑五百零七,餘見月食推日食法。

求天正冬至,同日躔。

求紀日,同月食。

求首朔,同月食。

求太陰入食限,與月食求逐月望平交周之法同,惟不用望策,即為逐月朔平交周。視某月交周入可食之限,即為有食之月。交周自五宮九度零八分至六宮八度五十一分,又自十一宮二十一度零九分至初宮二十度五十二分,皆為可食之限。

求平朔,

求太陽平行,

求太陽平引,

求太陰平引,以上四條,皆與月食求平望之法同,惟不加望策。

求太陽實引,同月食。

求太陰實引,同月食。

求實朔,與月食求實望之法同。

求實交周,與月食同。視實交周入食限為有食。自五宮十一度四十五分至六宮六度十四分,又自十一宮二十三度四十六分至初宮十八度十五分,為實朔可食限。

求太陽黃赤道實經度,同月食。

求實朔用時,同月食求實望用時。實朔用時,在日出前或日入後。五刻以外,則在夜,不必算。

求食甚用時,與月食求食甚時刻法同。

求用時春秋分距午赤道度,以太陽赤道經度減三宮,不足減者,加十二宮減之。為太陽距春分後赤道度。又以食甚用時變為赤道度,加減半周,過半周者減去半周,不及半周者加半周。為太陽距午正赤道度。兩數相加,滿全周去之。其數不過象限者,為春分距午西赤道度。過一象限者,與半周相減,餘為秋分距午東赤道度。過二象限者,則減去二象限,餘為秋分距午西赤道度。過三象限者,與全周相減,餘為春分距午東赤道度。

求用時春秋分距午黃道度,以黃赤大距之餘弦為一率,本天半徑為二率,春秋分距午赤道度之正切為三率,求得四率為正切,檢表得用時春秋分距午黃道度。

求用時正午黃赤距緯,以本天半徑為一率,黃赤大距之正弦為二率,距午黃道度之正弦為三率,求得四率為正弦,檢表得用時正午黃赤距緯。

求用時黃道與子午圈交角,以距午黃道度之正弦為一率,距午赤道度之正弦為二率,本天半徑為三率,求得四率為正弦,檢表得用時黃道與子午圈交角。

求用時正午黃道宮度,置用時春秋分距午黃道度,春分加減三宮。午西加三宮,午東與三宮相減。秋分加減九宮,午西加九宮,午東與九宮相減。得用時正午黃道宮度。

求用時正午黃道高,置赤道高度,北極高度減象限之餘。以正午黃赤距緯加減之,黃道三宮至八宮加,九宮至二宮減。即得。

求用時黃平象限距午,以黃道子午圈交角之餘弦為一率,本天半徑為二率,正午黃道高之正切為三率,求得四率為正切,檢表得度分。與九十度相減,餘為黃平象限距午之度分。

求用時黃平象限宮度,以黃平象限距午度分與正午黃道宮度相加減,正午黃道宮度初宮至五宮為加,六宮至十一宮為減,若正午黃道高過九十度,則反其加減。即得。

求用時月距限,以太陽黃道經度與用時黃平象限宮度相減,餘為月距限度,隨視其距限之東西。太陽黃道經度大於黃平象限宮度者為限東,小者為限西。

求用時限距地高,以本天半徑為一率,黃道子午圈交角之正弦為二率,正午黃道高之餘弦為三率,求得四率為餘弦,檢表得限距地高。

求用時太陰高弧,以本天半徑為一率,限距地高之正弦為二率,月距限之餘弦為三率,求得四率為正弦,檢表得太陰高弧。

求用時黃道高弧交角,以月距限之正弦為一率,限距地高之餘切為二率,本天半徑為三率,求得四率為正切,檢表得黃道高弧交角。

求用時白道高弧交角,置黃道高弧交角,以黃白大距加減之,食甚交周初宮、十一宮,月距限東則加,限西則減。五宮、六宮反是。即得。如過九十度,限東變為限西,限西變為限東,不足減者反減之。則黃平象限在天頂南者,白平象限在天頂北;黃平象限在天頂北者,白平象限在天頂南。

求太陽距地,詳月食求地影半徑條。

求太陰距地,詳月食求太陰半徑條。

求用時高下差,用平三角形,以地半徑為一邊,太陽距地為一邊,用時太陰高弧與象限相減,餘為所夾之角,求得對太陽距地邊之角。減去一象限,為太陽視高。與太陰高弧相減,餘為太陽地半徑差。又用平三角形,以地半徑為一邊,太陰距地為一邊,用時太陰高弧與象限相減,餘為所夾之角,求得對太陰距地邊之角。減去一象限,為太陰視高。與高弧相減,餘為太陰地半徑差。兩地半徑差相減,得高下差。

求用時東西差,以半徑千萬為一率,白道高弧交角之餘弦為二率,高下差之正切為三率,求得四率為正切,檢表得用時東西差。

求食甚近時,以月距日實行化秒為一率,一小時化秒為二率,東西差化秒為三率,求得四率為秒。以時分收之,為近時距分。以加減食甚用時,月距限西則加,限東則減,仍視白道高弧交角變限不變限為定。得食甚近時。

求近時春秋分距午赤道度,以食甚近時變赤道度求之,餘與前用時之法同。後諸條仿此,但皆用近時度分立算。

求近時春秋分距午黃道度。

求近時正午黃赤距緯。

求近時黃道與子午圈交角。

求近時正午黃道宮度。

求近時正午黃道高。

求近時黃平象限距午。

求近時黃平象限宮度。

求近時月距限,置太陽黃道經度,加減用時東西差,依近時距分加減號。為近時太陰黃道經度。與近時黃平象限宮度相減,為近時月距限。餘同用時。

求近時限距地高。

求近時太陰高弧。

求近時黃道高弧交角。

求近時白道高弧交角。

求近時高下差。

求近時東西差。

求食甚視行,倍用時東西差減近時東西差,即得。

求食甚真時,以視行化秒為一率,近時距分化秒為二率,用時東西差化秒為三率,求得四率為秒。以時分收之,為真時距分,以加減食甚用時,得食甚真時。加減與近時距分同。

求真時春秋分距午赤道度,以食甚真時變赤道度求之,餘與用時之法同。後諸條仿此,但皆用真時度分立算。

求真時春秋分距午黃道度。

求真時正午黃赤距緯。

求真時黃道與子午圈交角。

求真時正午黃道宮度。

求真時正午黃道高。

求真時黃平象限距午。

求真時黃平象限宮度。

求真時月距限,置太陽黃道經度,加減近時東西差,依真時距分加減號。為真時太陰黃道經度。餘同用時。

求真時限距地高。

求真時太陰高弧。

求真時黃道高弧交角。

求真時白道高弧交角。

求真時高下差。

求真時東西差。

求真時南北差,以半徑千萬為一率,真時白道高弧交角之正弦為二率,真時高下差之正弦為三率,求得四率為正弦,檢表得真時南北差。

求食甚視緯,依月食求食甚距緯法推之,得實緯。以真時南北差加減之,為食甚視緯。白平象限在天頂南者,緯南則加,而視緯仍為南;緯北則減,而視緯仍為北。若緯北而南北差大於實緯,則反減而視緯變為南。限在天頂北者反是。

求太陽半徑,以太陽距地為一率,太陽實半徑為二率,本天半徑為三率,求得四率為正弦,檢表得太陽半徑。

求太陰半徑,詳月食。

求食分,以太陽全徑為一率,十分為二率,並徑太陽太陰兩半徑並。減去視緯為三率,求得四率即食分。

求初虧、復圓用時,以食甚視緯之餘弦為一率,並徑之餘弦為二率,半徑千萬為三率,求得四率為餘弦,檢表得初虧、復圓距弧。又以月距日實行化秒為一率,一小時化秒為二率,初虧、復圓距弧化秒為三率,求得四率為秒。以時分收之,為初虧、復圓距時。以加減食甚真時,得初虧、復圓用時。減得初虧,加得復圓。

求初虧春秋分距午赤道度,以初虧用時變赤道度求之,餘與用時同。後諸條仿此,但皆用初虧度分立算。

求初虧春秋分距午黃道度。

求初虧正午黃赤距緯。

求初虧黃道與子午圈交角。

求初虧正午黃道宮度。

求初虧正午黃道高。

求初虧黃平象限距午。

求初虧黃平象限宮度。

求初虧月距限,置太陽黃道經度,減初虧、復圓距弧,又加減真時東西差,依真時距分加減號。得初虧太陰黃道經度。餘同用時。

求初虧限距地高。

求初虧太陰高弧。

求初虧黃道高弧交角。

求初虧白道高弧交角。

求初虧高下差。

求初虧東西差。

求初虧南北差。

求初虧視行,以初虧、東西差與真時東西差相減並初虧食甚同限則減,初虧限東食甚限西則並。為差分,以加減初虧、復圓距弧為視行。相減為差分者,食在限東,初虧東西差大則減,小則加。食在限西反是。相並為差分者恆減。

求初虧真時,以初虧、視行化秒為一率,初虧、復圓距時化秒為二率,初虧、復圓距弧化秒為三率,求得四率為秒。以時分收之,為初虧距分。以減食甚真時,得初虧真時。

求復圓春秋分距午赤道度,以復圓用時變赤道度求之。餘同用時。後諸條仿此,但皆用復圓度分立算。

求復圓春秋分距午黃道度。

求復圓正午黃赤距緯。

求復圓黃道與子午圈交角。

求復圓正午黃道宮度。

求復圓正午黃道高。

求復圓黃平象限距午。

求復圓黃平象限宮度。

求復圓月距限,置太陽黃道經度,加初虧、復圓距弧,又加減真時東西差,依真時距分加減號。得復圓太陰黃道經度。餘同用時。

求復圓限距地高。

求復圓太陰高弧。

求復圓黃道高弧交角。

求復圓白道高弧交角。

求復圓高下差。

求復圓東西差。

求復圓南北差。

求復圓視行,以復圓東西差與真時東西差相減並為差分,復圓食甚同限,則減;食甚限東,復圓限西,則並。以加減初虧、復圓距弧為視行。相減為差分者,食在限東,復圓東西差大則加,小則減。食在限西反是,相並為差分者恆減。

求復圓真時,以復圓視行化秒為一率,初虧、復圓距時化秒為二率,初虧、復圓距弧化秒為三率,求得四率為秒。以時分收之,為復圓距分。以加食甚真時,得復圓真時。

求食限總時,以初虧距分與復圓距分相並,即得。

求太陽黃道宿度,同日躔。

求太陽赤道宿度,依恆星求赤道經緯法求得本年赤道宿鈐,餘同日躔求黃道法。

求初虧、復圓定交角,求得初虧、復圓各視緯,與食甚法同。以求各緯差角。各與黃道高弧交角相加減,為初虧及復圓之定交角。法與月食同。

求初虧、復圓方位,食在限東者,定交角在四十五度以內,初虧上偏右,復圓下偏左。四十五度以外,初虧右偏上,復圓左偏下。適足九十度,初虧正右,復圓正左。過九十度,初虧右偏下,復圓左偏上。食在限西者,定交角在四十五度以內,初虧下偏右,復圓上偏左。四十五度以外,初虧右偏下,復圓左偏上。適足九十度,初虧正右,復圓正左。過九十度,初虧右偏上,復圓左偏下。京師黃平象限恆在天頂南,定方位如此,在天頂北反是。

求帶食分秒,以本日日出或日入時分初虧或食甚在日出前者,為帶食出地,用日出分;食甚或復圓在日入後者,為帶時入地,用日入分。與食甚真時相減,餘為帶食距時。乃以初虧、復圓距時化秒為一率,初虧、復圓視行化秒為二率,帶食在食甚前,用初虧視行;帶食在食甚後,用復圓視行。帶食距時化秒為三率,求得四率為秒。以度分收之,為帶食距弧。又以半徑千萬為一率,帶食距弧之餘切為二率,食甚距緯之餘弦為三率,求得四率為餘切,檢表得帶食兩心相距。乃以太陽全徑為一率,十分為二率,並徑內減帶食兩心相距為三率,求得四率,為帶食分秒。

求各省日食時刻及食分,以京師食甚用時,按各省東西偏度加減之,得各省食甚用時。乃按各省北極高度,如京師法求之,即得。

求各省日食方位,以各省黃道高弧交角及初虧、復圓視緯,求其定交角,即得。

繪日食圖法同月食,但只用日月兩半徑為度,作一大虛圈,為初虧、復圓月心所到。不用內虛圈,無食既、生光二限。

凌犯用數,具七政恆星行及交食。

推凌犯法,求凌犯入限,太陰凌犯恆星,以太陰本日次日經度,查本年心互星經緯度表,某星緯度不過十度,經度在此限內,為凌犯入限。復查太陰在入限各星之上下,如星月兩緯同在黃道北者,緯多為在上,緯少為在下。同在黃道南者反是。一南一北者,北為在上,南為在下。太陰在上者,兩緯相距二度以內取用;太陰在下者,一度以內取用。相距十七分以內為凌,十八分以外為犯,緯同為掩。太陰凌犯五星,以本日太陰經度在星前、次日在星後為入限,餘與凌犯恆星同。五星凌犯恆星,以兩緯相距一度內取用。相距三分以內為凌,四分以外為犯,餘與太陰同。五星自相凌犯,以行速者為凌犯之星,行遲者為受凌犯之星。如遲速相同而有順逆,則為順行之星凌犯逆行之星,皆以此星經度本日在彼星前、次日在彼星後為入限。餘同凌犯恆星。

求日行度,太陰凌犯恆星,即以太陰一日實行度為日行度。凌犯五星,以太陰一日實行度與本星一日實行度相加減,星順行則減,逆行則加。為日行度。五星凌犯恆星,以本星一日實行度為日行度。五星自相凌犯,以兩星一日實行度相加減,順逆同行則減,異行則加。為日行度。

求凌犯時刻,以日行度化秒為一率,刻下分為二率,本日子正相距度化秒為三率,求得四率為分。以時刻收之,初時起子正,即得。

求太陰凌犯視差,五星視差甚微,可以不計。以刻下分為一率,太陽一日實行度化秒為二率,凌犯時刻化分為三率,求得四率為秒。以度分收之,與本日子正太陽實行相加,為本時太陽黃道度。依日食法求東西差及南北差。

求太陰視緯,置太陰實緯,以南北差加減之,加減之法,與日食同。即得。求太陰距星,以太陰視緯與星緯相加減,南北相同則減,一南一北則加。得太陰距星。取相距一度以內者用。

求凌犯視時,以太陰一小時實行化秒為一率,一小時化秒為二率,東西差化秒為三率,求得四率為秒。收為分,以加減凌犯時刻,太陰距限西則加,東則減。得凌犯視時。


\end{pinyinscope}