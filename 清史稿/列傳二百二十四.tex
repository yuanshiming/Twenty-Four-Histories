\article{列傳二百二十四}

\begin{pinyinscope}
榮祿王文韶張之洞瞿鴻禨

榮祿,字仲華,瓜爾佳氏,滿洲正白旗人。祖喀什噶爾幫辦大臣塔斯哈,父總兵長壽,均見忠義傳。

榮祿以廕生賞主事,隸工部,晉員外郎。出為直隸候補道。同治初,設神機營,賞五品京堂,充翼長,兼專操大臣。再遷左翼總兵。用大學士文祥薦,改工部侍郎,調戶部,兼總管內務府大臣。穆宗崩,德宗嗣統。榮祿言於恭親王,乃請頒詔,俟嗣皇帝有子,承繼穆宗。其後始定以紹統者為嗣。光緒元年,兼步軍統領。遷左都御史,擢工部尚書。慈禧皇太后嘗欲自選宮監,榮祿奏非祖制,忤旨。會學士寶廷奏言滿大臣兼差多,乃解尚書及內務府差。又以被劾納賄,降二級,旋開復,出為西安將軍。二十年,祝嘏留京,再授步軍統領。日本構釁,恭親王、慶親王督辦軍務,榮祿參其事。和議成,疏薦溫處道袁世凱練新軍,是曰「新建陸軍」。授兵部尚書、協辦大學士。疏請益練新軍,而調甘肅提督董福祥軍入衛京師。

二十四年,晉大學士,命為直隸總督。是時上擢用主事康有為及知府譚嗣同等參預新政,議變法,斥舊臣。召直隸按察使袁世凱入覲,超授侍郎,統練兵。榮祿不自安。御史楊崇伊奏請太后再垂簾,於是太后復臨朝訓政,召榮祿為軍機大臣,以世凱代之。命查拿康有為,斬譚嗣同等六人於市。以上有疾,詔徵醫。復命榮祿管兵部,仍節制北洋海陸各軍。榮祿乃奏設武衛軍,以聶士成駐蘆臺為前軍,董福祥駐薊州為後軍,宋慶駐山海關為左軍,世凱駐小站為右軍,而自募萬人為中軍,駐南苑。時太后議廢帝,立端王載漪子溥俊為穆宗嗣,患外人為梗,用榮祿言,改稱「大阿哥」。

二十六年,拳匪亂作,載漪等稱其術,太后信之,欲倚以排外人。福祥率甘軍攻使館,月餘不下。榮祿不能阻,載漪等益橫,京師大亂,駢戮忠諫大臣。榮祿踉蹌入言,太后厲色斥之。聯軍入京,兩宮西幸,駐蹕太原。榮祿請赴行在,不許,命為留京辦事大臣。已而詔詣西安,既至,寵禮有加,賞黃馬褂,賜雙眼花翎、紫韁。隨扈還京,加太子太保,轉文華殿大學士。二十九年,卒,贈太傅,謚文忠,晉一等男爵。

榮祿久直內廷,得太后信仗。眷顧之隆,一時無比。事無鉅細,常待一言決焉。

王文韶,字夔石,浙江仁和人。咸豐二年進士,銓戶部主事。累遷郎中,出為湖北安襄鄖荊道。左宗棠、李鴻章皆薦其才。擢按察使,遷湖南布政使。同治十年,署巡撫。黔苗亂熾,桂東淪寇域。文韶條上援黔、防境機宜,以兵事屬按察使席寶田,督其部將蘇元春、龔繼昌等進剿,斬首逆張秀眉烏鴉坡,黔境平。文韶繪苗疆要塞圖,上之朝。十一年,除真。寧遠莠民倡亂,耒陽硃鴻英復妄稱明裔構眾,先後檄道員陳寶箴討平之。光緒元年,遣總兵謝晉鈞平新化、衡、永土寇。撫湘六年,內治稱靜謐焉。入權兵部侍郎,直軍機。會歲旱,各省籥災,中旨罪己。文韶亦自陳無狀,詔革職留任。旋除禮部侍郎,兼總理衙門行走。八年,御史洪良品、鄧承修劾雲南軍需案,文韶坐失察,奪二級。乞養歸,終母喪,還前除。

十五年,授雲貴總督。武定會匪陷富民、祿勸,人心恟懼。文韶斬獲叛將,三日而定。無何,鎮邊夷亂起,檄迤南道劉春霖分道進攻,拓地三百里。徙建城於猛朗,募勇屯墾。改臨安猛丁歸流,移府經歷駐其地。其餘寇亂及土族叛服不常,皆隨時殄滅。

初,英、法並緬、越後,西南緣邊防務益棘。文韶綏靖各路土司,令自為守。會日韓啟釁,詔入都詢方略。既至,奉幫辦北洋之命。鴻章赴日議和,文韶權直隸總督、北洋大臣。和議成,實授。時關內外主客軍四百餘營,酌留湘、淮、豫三十營,餘悉散遣,士卒帖然。建議籌修旅大砲臺,謂:「旅順舊臺密於防前,疏於防後,敵自大連灣入,遂失所芘;大連舊臺,專顧防海,未及防陸,敵自金州登岸,遂不能支。今重整海防,必彌其罅隙。」又請加意水師、武備各學堂,以儲將才,嫺武幹,俟財力稍足,徐圖擴充。又陳河運漕糧積弊,請蘇漕統歸海運,他若勘吉林三姓金礦、磁州煤礦,踵鴻章後次第成之,而京漢鐵路亦興築於是時矣。又奏設北洋大學堂、鐵路學堂、育才館、俄文館,造就甚眾。

二十四年,入贊軍機,以戶部尚書協辦大學士。二十六年,拳匪仇教,文韶力言外釁不可啟,不見納。宮車既出,三日,始追及懷來。自聯軍犯京,事急,兩宮召軍機,惟文韶一人入見,諭必侍行。至是立召對,泣慰之,遂隨扈,自晉入秦,晉體仁閣大學士。明年,改外務部會辦大臣,旋賞黃馬褂。署全權大臣,命先還京,佐辦中俄條約。交還東三省及關外鐵路,事寧,賞雙眼花翎。充政務處大臣,督辦路礦總局。轉文淵閣,晉武英殿。三十一年,免直軍機。明年,稱疾乞休。

文韶歷官中外,詳練吏職,究識大體,然更事久,明於趨避,亦往往被口語。三十四年,鄉舉重逢,賜太子太保。其冬,卒,年七十九,晉贈太保,謚文勤。

張之洞,字香濤,直隸南皮人。少有大略,務博覽為詞章,記誦絕人。年十六,舉鄉試第一。同治二年,成進士,廷對策不循常式,用一甲三名授編修。六年,充浙江鄉試副考官,旋督湖北學政。十二年,典試四川,就授學政。所取士多俊才,游其門者,皆私自喜得為學塗徑。光緒初,擢司業,再遷洗馬。之洞以文儒致清要,遇事敢為大言。俄人議歸伊犁,與使俄大臣崇厚訂新約十八條。之洞論奏其失,請斬崇厚,毀俄約。疏上,乃褫崇厚職治罪,以侍郎曾紀澤為使俄大臣,議改約。六年,授侍講,再遷庶子。復論紀澤定約執成見,但論界務,不爭商務,並附陳設防、練兵之策。疏凡七八上。往者詞臣率雍容養望,自之洞喜言事,同時寶廷、陳寶琛、張佩綸輩崛起,糾彈時政,號為清流。七年,由侍講學士擢閣學。俄授山西巡撫。當大祲後,首劾布政使葆亨、冀寧道王定安等黷貨,舉廉明吏五人,條上治晉要務,未及行,移督兩廣。

八年,法越事起,建議當速遣師赴援,示以戰意,乃可居間調解。因薦唐炯、徐延旭、張曜材任將帥。十年春,入覲。四月,兩廣總督張樹聲解任專治軍,遂以之洞代。當是時,雲貴總督岑毓英、廣西巡撫潘鼎新皆出督師,尚書彭玉麟治兵廣東。越將劉永福者,故中國人,素驍勇,與法抗。法攻越未能下,復分兵攻臺灣,其後遂據基隆。朝議和戰久不決,之洞至,言戰事氣自倍,以玉麟夙著威望,虛己聽從之。奏請主事唐景崧募健卒出關,與永福相犄角。朝旨因就加永福提督、景崧五品卿銜,炯、延旭亦皆已至巡撫,當前敵,被劾得罪去,並坐舉者。之洞獨以籌餉械勞,免議。廣西軍既敗於越,朝旨免鼎新,以提督蘇元春統其軍,而之洞復奏遣提督馮子材、總兵王孝祺等,皆宿將,於是滇、越兩軍合扼鎮南關,殊死戰,遂克諒山。會法提督孤拔攻閩、浙,砲毀其坐船,孤拔殪,而我軍不知,法原停戰,廷議許焉。授李鴻章全權大臣,定約,以北圻為界。敘克諒山功,賞花翎。

之洞恥言和,則陰自圖強,設廣東水陸師學堂,創槍砲廠,開礦務局。疏請大治水師,歲提專款購兵艦。復立廣雅書院。武備文事並舉。十二年,兼署巡撫。於兩粵邊防控制之宜,輒多更置。著沿海險要圖說上之。在粵六年,調補兩湖。

會海軍衙門奏請修京通鐵路,臺諫爭陳鐵路之害,請停辦。翁同龢等請試修邊地,便用兵;徐會灃請改修德州濟寧路,利漕運。之洞議曰:「修路之利,以通土貨、厚民生為最大,徵兵、轉餉次之。今宜自京外盧溝橋起,經河南以達湖北漢口鎮。此幹路樞紐,中國大利所萃也。河北路成,則三晉之轍接於井陘,關隴之驂交於洛口;自河以南,則東引淮、吳,南通湘、蜀,萬里聲息,刻期可通。其便利有數端:內處腹地,無慮引敵,利一;原野廣漠,墳廬易避,利二;廠盛站多,役夫賈客可舍舊圖新,利三;以一路控八九省之衢,人貨輻輳,足裕餉源,利四;近畿有事,淮、楚精兵崇朝可集,利五;太原旺煤鐵,運行便則開採必多,利六;海上用兵,漕運無梗,利七。有此七利,分段分年成之。北路責之直隸總督,南路責之湖廣總督,副以河南巡撫。」得旨報可,遂有移楚之命。大冶產鐵,江西萍鄉產煤,之洞乃奏開鍊鐵廠漢陽大別山下,資路用,兼設槍砲鋼藥專廠。又以荊襄宜桑棉麻枲而饒皮革,設織布、紡紗、繅絲、制麻革諸局,佐之以堤工,通之以幣政。由是湖北財賦稱饒,土木工作亦日興矣。

二十一年,中東事棘,代劉坤一督兩江,至則巡閱江防,購新出後膛砲,改築西式砲臺,設專將專兵領之。募德人教練,名曰「江南自強軍」。採東西規制,廣立武備、農工商、鐵路、方言、軍醫諸學堂。尋還任湖北。時國威新挫,朝士日議變法,廢時文,改試策論。之洞言:「廢時文,非廢五經、四書也,故文體必正,命題之意必嚴。否則國家重教之旨不顯,必致不讀經文,背道忘本,非細故也。今宜首場試史論及本朝政法,二場試時務,三場以經義終焉。各隨場去留而層遞取之,庶少流弊。」又言:「武科宜罷騎射、刀石,專試火器。欲挽重文輕武之習,必使兵皆識字,勵行伍以科舉。」二十四年,政變作,之洞先著勸學篇以見意,得免議。

二十六年,京師拳亂,時坤一督兩江,鴻章督兩廣,袁世凱撫山東,要請之洞,同與外國領事定保護東南之約。及聯軍內犯,兩宮西幸,而東南幸無事。明年,和議成,兩宮回鑾。論功,加太子少保。以兵事粗定,乃與坤一合上變法三疏。其論中國積弱不振之故,宜變通者十二事,宜採西法者十一事。於是停捐納,去書吏,考差役,恤刑獄,籌八旗生計,裁屯衛,汰綠營,定礦律、商律、路律、交涉律,行銀圓,取印花稅,擴郵政。其尤要者,則設學堂,停科舉,獎游學。皆次第行焉。

二十八年,充督辦商務大臣,再署兩江總督。有道員私獻商人金二十萬為壽,請開礦海州,立劾罷之。考鹽法利弊,設兵輪緝私,歲有贏課。明年,入覲,充經濟特科閱卷大臣,釐定大學堂章程,畢,仍命還任。陛辭奏對,請化除滿、漢畛域,以彰聖德,遏亂萌,上為動容。旋裁巡撫,以之洞兼之。三十二年,晉協辦大學士。未幾,內召,擢體仁閣大學士,授軍機大臣,兼筦學部。三十四年,督辦粵漢鐵路。

德宗暨慈禧皇太后相繼崩,醇親王載灃監國攝政。之洞以顧命重臣晉太子太保。逾年,親貴浸用事,通私謁。議立海軍,之洞言海軍費絀可緩立,爭之不得。移疾,遂卒,年七十三,朝野震悼。贈太保,謚文襄。

之洞短身巨髯,風儀峻整。蒞官所至,必有興作。務宏大,不問費多寡。愛才好客,名流文士爭趨之。任疆寄數十年,及卒,家不增一畝云。

瞿鴻禨,字子玖,湖南善化人。同治十年進士,授編修。光緒元年,大考一等,擢侍講學士。久乃遷詹事,晉內閣學士。先後典福建、廣西鄉試,督河南、浙江、四川學政。所行皆本功令,律下尤嚴。

朝鮮戰事起,我師出平壤。鴻禨上四路進兵之策,請兼募沿海漁人蜑戶編為舟師,使敵備多力分,庶可制勝。及和議成,鴻禨方自蜀還,復奏言秦中地形險要,請豫建陪都。日本增兵遼東,鴻禨以敵情叵測,請敕劉坤一、王文韶簡練勁旅,不可專任淮軍。適坤一奏劾山西將賀星明侵餉,革職,鴻禨言:「刑賞治天下之大柄,軍紀廢弛已久,宜嚴懲以儆其餘。」又:「葉志超、龔照嶼等敗軍辱國,罪當死。和約既定,勢不能與勾,宜籍其財產,或令巨款捐贖,然後貸其一死。」皆不報。旋遷禮部侍郎,出督江蘇學政。請罷武科。

兩宮西狩,鴻禨差竣詣行在,道授左都御使,晉工部尚書,仍以西安陪都為言。既至,命直軍機,兼充政務處大臣。請以策論試士,開經濟特科,汰書吏,悉允行。改總理各國事務衙門為外務部,班六部上,以鴻禨為尚書。時方與各國議和,鴻禨治事明敏,諳究外交,承旨擬諭,語中竅要,頗當上意焉。扈蹕回鑾,賞黃馬褂,加太子太保。

自新政議起,興學、通商、勸工諸政,有司多借端巧取。鴻禨請降旨禁革苛派,任民間自辦。又請旨以戶部正雜諸款供地方正用,宮中歲費,遵先朝定例,量入為出,不便自戶部增撥。裁汰內務府冗員,用節糜費。充中日議約全權大臣。是時中外咸以立憲為請,朝廷下詔豫備憲政始基,勖天下以忠君尊孔、尚公尚武尚實,用鴻禨言也。三十二年,協辦大學士。特旨派議改官制大臣,鴻禨以樞廷事冗辭。旋命與大學士孫家鼐復核,頗有裁正焉。

鴻禨持躬清刻,以儒臣驟登政地,銳於任事。素善岑春煊,春煊入朝,留長郵傳部。密疏劾慶親王奕劻,奕劻惡春煊,遂及鴻禨。會鴻禨因直言忤太后旨,侍講學士惲毓鼎劾以攬權恣縱,遂罷斥歸里。辛亥,湘變起,流寓上海,旋卒。後追謚文慎。

論曰:德宗親政,憤於外侮,思變法自強。乃以輔導無人,戊戌黨禍,庚子匪亂,遂相繼而作。太后再出垂簾,初堅復舊,繼勉圖新。宣統改元,議行憲政。政體既變,國本遂搖,而大勢不可問矣。榮祿屢參大變,文韶久達世務。鴻禨後起,參議立憲,終以失寵太后,不免放斥。唯之洞一時稱賢,而監國攝政,親貴用事,欲挽救而未能,遂以憂死。人之云亡,邦國殄瘁,尚何言哉?


\end{pinyinscope}