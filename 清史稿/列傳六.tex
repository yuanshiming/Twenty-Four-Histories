\article{列傳六}

\begin{pinyinscope}
諸王五

太宗諸子

肅武親王豪格子溫良郡王猛瓘猛瓘子延信輔國公葉布舒

承澤裕親王碩塞莊恪親王允祿鎮國?厚公高塞

輔國公品級常舒輔國公韜塞襄昭親王博穆博果爾

世祖諸子

裕憲親王福全榮親王恭親王常寧純靖親王隆禧

太宗十一子:孝莊文皇后生世祖,敏惠恭和元妃科爾沁博爾濟吉特氏生第八子,懿靖大貴妃阿巴海博爾濟吉特氏生襄親王博穆博果爾,元妃鈕祜祿氏生洛博會,繼妃烏喇納喇氏生肅親王豪格、洛格,側妃葉赫納喇氏生承澤親王碩塞,庶妃顏扎氏生輔國公葉布舒,庶妃納喇氏生鎮國公高塞,庶妃伊爾根覺羅氏生輔國公品級常舒,庶妃生輔國公韜塞。洛格、洛博會及第八子,皆殤,無封。

肅武親王豪格,太宗第一子。初從征蒙古董夔、察哈爾、鄂爾多斯諸部,有功,授貝勒。天命十一年,偕貝勒代善等征扎嚕特部,斬其貝勒鄂齋圖。天聰元年,敗明兵於錦州,。三年十月,偕?塔山糧運。二年,偕濟爾哈朗討蒙古固特塔布囊,誅之,收其?復率偏師貝勒莽古爾泰等視通州渡口,師薄明都,豪格迎擊寧、錦援兵於廣渠門外,敵伏於右,豪格以所部當之,沖擊至城壕,明兵大潰,偕岳託、薩哈璘圍永平,克香河。六年,從伐察哈爾,移師入明邊,略歸化諸路。六月,進和碩貝勒。

七年,詔問征明與朝鮮、察哈爾三者何先,疏言:「徵明,如徒得錦州,餘堅壁不下及邊外新舊蒙古從舊道入,諭各屯寨,以我欲和而彼君不?,曠日持久,恐老我師。宜悉我答,彼將自怨其主。再用更番法,俟馬肥,益以漢兵巨?,一出寧遠,一出舊道,夾攻山海關,不得,則屯兵招諭流賊,駐師通州,待其懈而擊之。朝鮮、察哈爾且緩圖焉。」八月,略山海關。八年,從上自宣府趨朔州。豪格偕揚古利毀邊墻,分兵自尚方堡入,略朔州及五臺山,從上視大同,擊敗明援兵。

九年,偕多爾袞等收察哈爾林丹汗子額哲,抵托裏圖,定盟。還抵歸化城,復略山西邊郡,毀寧武關,入代州、忻州。崇德元年四月,進封肅親王,掌戶部事。尋坐黨岳託漏上言有怨心,降貝勒,解任,罰銀千。旋偕多爾袞攻錦州,仍攝戶部。又從征朝鮮,偕多爾袞別自寬甸入長山口,克昌州,敗安州、黃州兵於寧邊城下。復遣將敗其援兵,次宣屯村,村民言:「黃州守將聞國王被圍,遣兵萬五千往援,行三日矣。」我軍疾馳一晝夜,追及於陶山,擊敗之。九月,坐固山額真鄂莫克圖欲脅取蒙古臺吉博洛女媚事豪格,豪格不治其罪,罷部任,罰銀千。

三年九月,伐明,自董家口毀邊墻入,敗明兵於豐潤。遂下山東,降高唐,略地至曹州,還下東光。又遣騎二千破明兵,克獻縣。四年四月,師還,賜馬二、銀萬,復攝戶部,復原封。又偕多鐸敗寧遠兵,斬明將金國鳳。五年六月,偕多爾袞屯田義州,刈錦州禾,克臺九、小凌河西臺二。明兵夜出襲鑲藍旗營,擊敗之。又擊洪承疇杏山,偕多爾袞圍錦州。坐離城遠駐,復遣兵還家,降郡王。六年,再圍錦州,擊松山及山海關援兵,皆敗之,獲馬五百餘。

承疇將兵十三萬援錦州,破其壘三。上至軍,將駐高橋,豪格等恐敵約軍夾攻,請改屯松山、杏山間。七年,松山明將夏承德密遣人請降,以其子舒為質,豪格遣左右翼夜梯城入,八旗兵繼之,旦,克松山,獲承疇及巡撫邱民仰等,斬官百餘、兵千六十有奇。進駐杏山,復偕濟爾哈朗克塔山。?功,復原封,賜鞍馬一、蟒緞百。

順治元年四月,以語侵睿親王多爾袞,為固山額真何洛會所訐,坐削爵。十月,大封諸王,念豪格從定中原有功,仍復原封。其年冬,定濟寧滿家洞土寇,堙山洞二百五十一。

三年,命為靖遠大將軍,偕衍禧郡王羅洛渾、貝勒尼堪等西征。師次西安,遣尚書星訥等破敵邠州,別遣固山額真都類攻慶陽。時賀珍、二隻虎、孫守法據漢中、興安,武大定、高如礪、蔣登雷、石國璽、王可成、周克德據徽縣、階州。師自西安分兵進擊,登雷、國璽、可成、克德俱降,餘潰走,下所陷城邑。陜西平。十一月,入四川,張獻忠據西充,遣巴牙喇昂邦鰲拜先發,師繼進,抵西充,大破之,豪格親射獻忠,殪,平其壘百三十餘所,斬首數萬級。捷聞,上嘉?。四年八月,遵義、夔州、茂州、榮昌、隆昌、富順、內江、寶陽諸郡縣悉定。四川平。五年二月,師還,上御太和殿宴勞。睿親王多爾袞與豪格有夙隙,坐豪格徇隱部將冒功及擢用罪人揚善弟吉賽,系豪格於獄。三月,薨。

睿親王納豪格福晉,嘗召其子富綬至邸校射。何洛會語人曰:「見此鬼魅,令人心悸,何不除之?」錫翰以告,睿親王曰:「何洛會意,因爾不知我愛彼也。」由是得全。八年正月,上親政,雪豪格枉,復封和碩肅親王,立碑表之。十三年,追謚。親王得謚自豪格始。以謚系封號上,曰武肅親王。乾隆四十三年,配享太廟。

豪格子七,有爵者二:富綬、猛瓘。

富綬襲爵,改號曰顯親王。康熙八年,薨,謚曰懿。子丹臻,襲。三十五年,從征噶爾丹。四十一年,薨,謚曰密。子衍潢,襲。乾隆三十六年,薨,年八十二,謚曰謹。富綬孫蘊著,襲。乾隆中,自三等輔國將軍授內閣侍讀學士,歷通政使、盛京戶部侍郎。調兵部侍郎,遷漕運總督。坐受商人餽遺,謬稱上旨籍鹽政吉慶家,坐絞,上寬之,復授副都統,歷涼州、綏遠城將軍,工部尚書。既,襲封。四十三年,復號肅親王。薨,年八十,謚曰勤。丹臻孫永錫,襲。官都統。坐事,罷。道光元年,薨,謚曰恭。子敬敏,襲。咸豐二年,薨,謚曰慎。子華豐,襲,歷內大臣、宗令。以火器營設碓制藥,占用王府地,華豐力拒之,詔責不知大體,罷宗令、內大臣。八年,薨,謚曰恪。子隆懃,襲,官內大臣。光緒二十一年,疏請納正言、裕財用,上嘉納之。二十四年,薨,謚曰良。子善耆,襲。三十三年,授民政部尚書。遜國後,避居大連灣。久之,薨,謚曰忠。

溫良郡王猛瓘,豪格第五子。順治十四年,封。康熙十三年,薨。子佛永惠,襲。三十七年,降貝勒。卒。子揆惠,襲輔國公。坐事,奪爵。

延信,猛瓘第三子。初封奉國將軍。累官至都統。五十七年,從撫遠大將軍貝子允率師討策妄阿喇布坦,駐西寧。五十九年,授平逆將軍,率師徇西藏,道青海,擊敗策妄阿喇布坦將策零敦多卜,遂入西藏。西藏平。詔曰:「平逆將軍延信領滿洲、蒙古、綠旗各軍,經自古未闢之道,煙瘴惡溪,人跡罕見。身臨絕域,殲夷醜類,勇略可嘉!封輔國公。」尋攝撫遠大將軍事。揆惠既奪爵,議以延信襲。進貝子,再進貝勒。授西安將軍。雍正五年,上以延信與阿其那等結黨,又陰結允,徇年羹堯,入藏侵帑十萬兩,奪爵,逮下王大臣按治。讞上延信黨援、欺罔、負恩、要結人心、貪婪亂政、失誤兵機,凡二十罪,當斬,上命幽禁,子孫降紅帶。

輔國公葉布舒,太宗第四子。初封鎮國將軍。康熙八年,晉輔國公。二十九年,卒。子蘇爾登,降襲鎮國將軍。

承澤裕親王碩塞,太宗第五子。順治元年,封。時李自成奔潼關,河以南仍為自成守。碩塞從豫親王多鐸師次孟津,進攻陜州,破自成將張有增、劉方亮,自成迎戰,又大破之。師入關,斬其將馬世堯。尋復從南征,擊破明福王由崧,賜團龍紗衣一襲、金二千、銀二萬。嗣復從多鐸征喀爾喀、英親王阿濟格戍大同。會姜瓖叛,碩塞移師解代州圍,進親王。諭曰:「博洛、尼堪、碩塞皆不當在貴寵之列。茲以太祖孫故,加錫王爵。其班次、俸祿不得與和碩親王等。」七年,以和碩親王下、多羅郡王上無止稱親王者,仍改郡王。八年,復進和碩親王。迭掌兵部、宗人府。十一年十二月,薨,予謚。

第一子博果鐸,襲,改號曰莊親王。雍正元年,薨,年七十四,謚曰靖。無子,宗人府題請以聖祖子承襲,世宗請於皇太后,以聖祖第十六子允祿為之後,襲爵。居數日,上手詔謂:「外間妄議朕愛十六阿哥,令其承襲莊親王爵。朕封諸弟為親王,何所不可,而必藉承襲莊親王爵加厚於十六阿哥乎?」

允祿精數學,通樂律,承聖祖指授,與修數理精蘊。乾隆元年,命總理事務,兼掌工部,食親王雙俸。二年,?總理勞,加封鎮國公,允祿請以碩塞孫寧赫襲。尋坐事,奪爵,仍厚分與田宅,時論稱之。四年,坐與允礽子弘?往來詭秘,停雙俸,罷都統。七年,命與三泰、張照管樂部。允祿等奏:「藉田禮畢,筵宴當奏雨暘時若、五穀豐登、家給時足三章,本為蔣廷錫所撰,樂與禮不符,不能施於燕樂。請敕別撰。」又奏:「中和韶樂,例用笙四、簫笛樂之上。請增笙為八,簫笛為四。」又奏:「漢以來各史樂志,?皆二,金、革二音獨出俱有鎛鐘、特磬。今得西江古鎛鐘,考定黃鐘直度,上下損益,鑄鎛鐘十二。竊以條理宜備始終,請仿周禮磬氏遺法,制特磬十二,與鎛鐘俱為特懸。樂闋擊特磬,乃奏敔;大祭祀、大典禮皆依應月之律,設鎛鐘、特磬各一?。」上悉從之。二十九年,允祿年七十,上賜詩褒之。三十二年,薨,年七十三,謚曰恪。

內大臣,仍管樂部、宗人府。?子弘普,輔國公,前卒。孫永巘,襲,歷都統、領侍內大臣、御前大臣。嘉慶十?五十三年,薨,謚曰慎。無子,以從子綿課襲,歷都統、領侍八年,林清為亂,其徒入宮門,綿課持械拒,射傷一人,得旨議敘。明年,上幸木蘭,綿課奏河橋圮于水,意在尼行,不稱上旨,坐罰俸,並罷諸職。道光二年,坐承修裕陵隆恩殿工草率,降郡王。四年,重修工蕆,復親王。六年,薨,謚曰襄。子奕鎛,嗣。八年,以寶華峪地宮入水,追論綿課罪,降奕鎛郡王,並奪諸子奕貹、奕飀、奕賟、奕賡職。十一年,上五十萬壽,復奕鎛親王。十八年九月,坐與輔國公溥喜赴尼寺食鴉片,奪爵。上聞奕鎛浮薄無行,戍吉林;又娶民女為妾,改戍黑龍江,以允祿曾孫綿護襲。

綿護,允祿次子輔國公弘?孫,輔國將軍永蕃子也。二十一年,薨,謚曰勤。弟綿,襲,二十五年,薨,謚曰質。子奕仁,襲,同治十三年,薨,謚曰厚。子載勛,襲。光緒二十六年,義和團入京師,載勛與端郡王載漪相結,設壇於其邸,縱令侵使館。俄,授步軍統領。上奉太后幸太原,載勛從,為行在查營大臣。既,與各國議和,罪禍首,奪爵,賜自盡。弟載功,襲。

碩塞第二子博爾果洛,封惠郡王。坐事,奪爵。世宗既以允祿襲莊親王,封博爾果洛孫球琳為貝勒,惠郡王所屬佐領皆隸焉。乾隆中,坐事,奪爵。子德謹,襲輔國公。子孫遞降,以奉恩將軍世襲。

鎮國?厚公高塞,太宗第六子。初封輔國公。康熙八年,進鎮國公。高塞居盛京,讀書醫無閭山,嗜文學,彈琴賦詩,自號敬一主人。九年,卒。子孫遞降,至曾孫忠福,襲輔國將軍,坐事奪爵。

輔國公品級常舒,太宗第七子。初封鎮國將軍。康熙八年,進輔國公。十四年,坐事,奪爵。三十七年,授輔國公品級。明年,卒。乾隆元年,高宗命錄太祖、太宗諸子後無爵者,授常舒子海林奉恩將軍,世襲。再傳至慧文,卒,命停襲。

輔國公韜塞,太宗第十子。初封鎮國將軍。康熙八年,進輔國公。三十四年,卒。乾隆元年,授韜塞子諭德奉恩將軍,世襲。

襄昭親王博穆博果爾,太宗第十一子。順治十二年,封襄親王。十三年,薨,予謚。無子,爵除。

世祖八子:孝康章皇后生聖祖,孝獻皇后董鄂氏生榮親王,寧?妃董鄂氏生裕憲親王福全,庶妃巴氏生牛鈕,庶妃陳氏生恭親王常寧,庶妃唐氏生奇授,庶妃鈕氏生純靖親王隆禧,庶妃穆克圖氏生永幹。牛鈕、奇授、永幹皆殤,無封。

裕憲親王福全,世祖第二子。幼時,世祖問志,對:「原為賢王。」世祖異之。康熙六年,封,命與議政。十一年十二月,疏辭,允之。二十二年,上奉太皇太后幸五臺,先行視道路,命福全扈太皇太后行。次長城嶺,上以嶺險不可陟,命福全奉太皇太后先還。二十七年,太皇太后崩。既繹祭,諭曰:「裕親王自太皇太后違豫,與朕同處,殊勞苦。」命皇長子及大臣送王歸第。?領侍

二十九年七月,噶爾丹深入烏硃穆秦,命為撫遠大將軍,皇長子允禔副之,出古北口;而以恭親王常寧為安北大將軍,出喜?口。福全請發大同綠旗兵往殺虎口聽調遣,上令發大同鎮標馬兵六百、步兵一千四百從征,兼命理籓院自阿喇尼設站處量發附近蒙古兵尾大軍置驛。福全又請凡諜報皆下軍中,上從之。師行,上御太和門賜敕印,出東直門送之。上先後遣內大臣阿密達、尚書阿喇尼、都統阿南達等出塞,命各率所部與福全師會。上出塞,駐古魯富爾堅嘉渾噶山,命康親王傑書率師會福全,進駐博洛咎屯。又命簡親王雅布參贊福全軍事。上先遣內大臣索額圖、都統蘇努分道出師,福全奏請令索額圖駐巴林,待師至,與會,上從之,並令蘇努同赴巴林,又趣阿密達、阿喇尼等速率兵內向分駐師所經道中以待。上自博洛和屯還駐舍裏烏硃,遣使諭福全曰:「兵漸與敵近,斥堠宜嚴明。噶爾丹當先與羈縻,以待盛京及烏喇、科爾沁諸部兵至。」

福全遣濟隆胡土克圖等以書喻噶爾丹曰:「我與汝協護黃教,汝追喀爾喀入我界,上命我等來論決此事。汝使言:『我汗遵達賴喇嘛之諭。』講信修禮,所關重大,今將於何地會議?」並遺以羊百、牛二十。蘇努、阿密達師來會,福全疏言:「噶爾丹聲息漸近,臣等分大軍為三隊,三隊當置將。自參贊大臣以下、副都統以上在行間者,皆奮欲前驅,唯上所命。」上命前鋒統領邁圖、護軍統領楊岱、副都統札木素、塞赫、羅滿色、海蘭,尚書吉勒塔布、阿喇尼率前隊,都統楊文魁、副都統康喀喇、伊壘、色格印率次隊,公蘇努、彭春率兩翼,內大臣佟國維、索額圖、明珠、阿密達從王親督指揮,師遂進。八月己未朔,次烏闌布通,厄魯特兵遇。黎明,整隊進,日晡,與戰,發槍?。至山下,厄魯特兵於林內隔河高岸橫臥。師右翼阻??駝以為障。內大臣佟國綱等戰沒。至昏,師左翼自山腰入,大敗之,斬馘頗河崖泥淖,夜收兵徐退。事聞,上深?諭之。

噶爾丹遣伊拉古克三胡土克圖至軍前,請執土謝圖汗、澤卜尊丹巴畀之,福全數其罪,遣還。越日,濟隆胡土克圖率其弟子七十人來言:「博碩克圖汗信伊拉古克三等言,入邊侵掠,大非理。但欲索其仇土謝圖汗及澤卜尊丹巴,迫而致此。彼今亦不敢復索土謝圖汗,原以澤卜尊丹巴予其師達賴喇嘛,榮莫大矣!」福全謂之曰:「土謝圖汗、澤卜尊丹巴即有罪,唯上責之,豈能因噶爾丹之言遣還達賴喇嘛?且汝往來行說,能保噶爾丹不乘間奔逸掠我境內民人乎?」濟隆固言噶爾丹不敢妄行,福全許檄各路軍止勿擊。時盛京及烏喇、科爾沁諸軍未至,厄魯特方據險,故福全既擊敗厄魯特,欲因濟隆之請羈縻之,待諸軍至復戰。

上以福全奏下王大臣集議,僉謂福全不即進軍,明知濟隆為噶爾丹游說以緩我師而故聽之,坐失事機,上嚴旨詰責,又以允禔與福全不協,留軍前必僨事,召先還京師。福全吳丹、護軍參領塞爾濟等偕濟隆諭噶爾丹,噶爾丹跪威靈佛前稽首設誓,復遣伊拉古?遣侍克三齎奏章及誓書詣軍前乞宥罪,出邊待命。上許之,復戒福全曰:「噶爾丹雖服罪請降,但性狡詐,我撤兵即虞背盟,仍宜為之備。」十月,福全率師還,駐哈嗎爾嶺內,疏言:「軍中糧至十月十日當盡,前遣侍郎額爾賀圖偕伊拉古克三諭噶爾丹,月餘未歸,度噶爾丹已出邊遠遁。」上以福全擅率師內徙,待歸時議罪,命即撤兵還京師,令福全及索額圖、明珠、費揚古、阿密達留後。尋奏:「噶爾丹出邊,伊拉古克三等追及於塞外。噶爾丹具疏謝罪。」因並命福全還京師。

十一月,福全等至京師,命止朝陽門外聽勘,諭曰:「貝勒阿敏棄永平,代善使朝鮮,不遵旨行事,英親王以兵譟,皆取口供,今應用其例。」且諭允示是曰:「裕親王乃汝伯父,倘汝供與王有異同,必置汝於法。」福全初欲錄允示是軍中過惡上聞,聞上命,流涕曰:「我復何言!」遂引為己罪。王大臣議奪爵,上以擊敗厄魯特功,免奪爵,罷議政,罰俸三年,撤三佐領。

三十五年,從上親征噶爾丹。四十一年,重修國子監文廟。封長子保泰為世子。四十二年,福全有疾,上再臨視。巡塞外,聞福全疾篤,命諸皇子還京師。福全薨,即日還蹕。臨喪,摘纓,哭至柩前奠酒,慟不已。是日,太后先臨王第,上勸太后還宮,自蒼震門入居景仁宮,不理政事。?臣勸上還乾清宮,上曰:「居便殿不自朕始,乃太祖、太宗舊典也。」越日,再臨喪,賜內?馬二、對馬二、散馬六、駱駝十,及蟒緞、銀兩。予謚。又越日,舉殯,上奉太后臨王第慟哭,殯行,乃已。命如鄭親王例,常祭外有加祭。御史羅占為監造墳塋,建碑。

福全畏遠權勢,上友愛綦篤,嘗命畫工寫御容與並坐桐陰,示同老意也。有目耕園,禮接士大夫。子保泰、保綬。

保泰,初封世子,襲爵。雍正二年,坐諂附廉親王允禩國喪演劇,奪爵。以保綬子廣寧襲,保綬追封悼親王。四年,諭:「廣寧治事錯繆,未除保泰朋黨之習。」奪爵,鎖禁。弟廣祿,襲。乾隆五十年,薨,謚曰莊。子亮煥,襲郡王。嘉慶十三年,薨,謚曰僖。孫文和,襲貝勒。子孫循例遞降,以鎮國公世襲。

榮親王,世祖第四子。生二歲,未命名,薨。追封。

恭親王常寧,世祖第五子。康熙十年,封。十四年,分給佐領。二十二年,府第災,上親臨視。是秋,上奉太皇太后幸五臺,常寧扈從。二十九年,噶爾丹深入烏硃穆秦。常寧為安北大將軍,簡親王雅布、信郡王鄂扎副之,出喜?口;同時,裕親王福全以撫遠大將軍,出古北口。先發,旋令率師會裕親王軍。十一月,以擊敗噶爾丹不窮追,罷議政,罰王俸三年。三十五年,從上親征。四十二年,薨。上方巡幸塞外,命諸皇子經理其喪,賜銀萬,內務府郎中皁保監修墳塋,立碑,遣官致祭。上還京師,臨其喪。第三子海善,襲貝勒。五十一年,坐縱內監妄行,奪爵。雍正十年,復封。乾隆八年,卒,謚僖敏。初奪爵,以常寧第二子滿都護襲貝勒,屢坐事,降鎮國公,又以海善孫斐蘇襲貝勒。子孫循例遞降,以不入八分鎮國公世襲。

純靖親王隆禧,世祖第七子。康熙十三年,封。十四年,分給佐領。十八年七月,隆禧疾篤,上親臨視,為召醫。是日再臨視,日加申,薨,上痛悼,輟朝三日。太皇太后欲臨其喪,上力諫乃止。上復欲臨奠,太皇太后亦諭止之,留太皇太后宮中。越日,上臨奠,命發帑修塋,加祭,予謚。子富爾祜倫,襲,明年,薨,上輟朝三日。又明年,葬純親王隆禧,上臨奠。富爾祜倫無子,未立後,爵除。


\end{pinyinscope}