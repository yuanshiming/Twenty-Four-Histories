\article{志四十八}

\begin{pinyinscope}
地理二十

△廣西

廣西:禹貢荊州南徼之域。元置廣西等處行中書省,明改承宣布政使司。清初建省,置巡撫、布政共治焉。置兩廣總督。康熙二年,廣東西分設總督,四年復故。雍正元年,復分設,明年復合。六年,以雲貴總督兼轄廣西。十二年,仍復故,駐廣東肇慶府,後移治廣州府。初領府九。桂林、柳州、慶遠、思恩、平樂、梧州、潯州、南寧、太平。順治十五年,升泗城土州為府,尋改為軍民府。雍正三年,升鬱林、賓州為直隸州。五年,泗城復為府。七年,置鎮安府。十二年,降賓州隸思恩府,升西隆州為直隸州。乾隆七年,降西隆州隸泗城府。光緒元年,升百色為直隸。十二年,升歸順州為直隸州。十八年,升上思州為直隸。東北距京師七千四百六十里。東至湖南道州;三百七十里。西至貴州普安;二千五百五十里。南至廣東信宜;九百四十里。北至湖南城步縣;三百二十里。廣二千八百十里,袤二千九百六十里。宣統三年,編戶百二十七萬四千五百四十四,口八百七十四萬六千七百四十七。領府十一,直隸二,直隸州二,八,州十五,縣四十九,土州二十四,土縣四,土司十三。在慶遠者曰長官司。其名山:越城、臨賀、句漏、陽海、大容。其巨川:漓江、黔江、鬱江、湘江。驛道:東北逾越城嶠達湖南永州;西南逾昆侖關達龍州;東南達廣東封川。電線:東北達長沙,東南達廣州,西通慶遠。鐵路:自龍州出鎮南關達安南諒山。

桂林府:沖,繁,難。隸桂平梧鬱道。巡撫,布政、提學、提法,勸業、巡警道駐。光緒三十二年,桂平梧鬱兼管鹽法道,徙駐梧州。提督徙駐南寧。明洪武五年,改靜江府為桂林府,領州二,縣七。順治初,因明舊為省治。乾隆六年,析義寧縣地置龍勝。光緒三十二年,析永寧州永福、融、柳城、雒容四縣地置中渡。廣二百五十里,袤三百里。北極高二十五度十三分。京師偏西六度十四分。領一,州二,縣七。臨桂沖,繁,難。倚。明府治因之。內:桂山、獨秀。城東:七星。南:南溪。西:隱山。東北:駮鹿。漓水一曰桂江,省境西江北岸第二大受渠也。自靈川入,西南流,經府治,合陽江。屈而東南,右受靈建水,又東與秋陂江合。屈東南流,乖水西流注之,南入陽朔。西:白石江,上源曰義江,自義寧入,經縣西南入永福。相思江出縣南臥石山,南注分水塘,歧為二:東出者與浪石江合,入漓江;西出者與繞江合,入白石江。南,六塘堡有汛。同知駐大墟,光緒三十二年徙中渡。西南,蘇橋巡司;南,六塘巡司。驛一:東江。興安沖,繁。府東北百三十四里。東北:越城嶺,一曰始安嶠,五嶺在廣西北境者二,此其最西嶺也。東南:龍蟠山。西南:鬱金山。海陽江即湘、漓二水源也,出靈川,右受石梯山水,左受太平堡水,又東北,經治東分水塘,歧為二,西南流者為漓水,東北流者為湘水。湘水自治東東北流,右受莫川,入全州。漓水經治北曰陡河,西流折南,至興隆市,六峒江合黃柏江、華江、川江、反璧江為大融江,自西北來注,西南入靈川。西北:小融江,出戴雲山,東南流,亦入靈川。全州營分防汛駐城。西北有泍水汛。社水巡司。鹽砂、唐家二寨廢司。驛一:白雲。靈川沖,難。府東北五十里。北:北障山。東南:堯山。西:呂仙。北:鳳皇。東南:陽海山,海陽江出,繞興安城西南流為漓水,復經縣東北曰靈江。合小融江,又西南至治東,右受潞江,左受淦水,西南合甘棠江入臨桂。甘棠江自興安入,經縣西北曰東江,東南流,合西江,又南,松木江合流風江東流注之。經龍巖,諸水匯巖下,伏流而南,左受社江,曲流入靈江。有帶融南北二堰,引潞江溉田二千餘頃。西,塘下有鎮。全州營分防汛駐。驛一:大龍。陽朔沖。府南少東百五十四里。北:陽朔山。西北:云源。西:都利。南,古羅。漓水自臨桂入,東南流,興平、熙平二水西流注之。屈曲南流,至治東折東,安樂、歸義二水東北流注之。左受白鶴山水,東南入平樂。西,金寶堡,明置戍。康熙八年,游擊駐防,後廢。桂林營分防汛駐城。東北有鉛寶塘汛,有水汛。驛一:古祚。永寧州簡。府西百四十里。東:百壽巖,東北:都狼。北:銀瓶山,白馬江經其下。東江一曰黃源水,出龍勝西南;經州北,東南流,至江口村,白馬江合大巖江、風門隘、茫洞江諸水,東北流,經治東注之,東南入臨桂境,折入永福,會白石江。富河江出州西南古河山;東南流,經高坡,伏流,至蒲臺寨西復出,合大洋江,西南錯入中渡曰中渡江,即雒容水也。西南桐木、富祿、南常安三鎮。永寧營駐城。西南安良、屯秋,南常安,有汛。南喇峒巡司。永福沖。府西南百里。明初屬府。後屬永寧州。順治初改屬府。西南:金山,與馬芒山對峙,江流經其中。又太和山濱江。西:白石江自臨桂入,北經治東曰永福江,毛江合泗定河西北流注之。東江亦自臨桂入,合西江東流注之。西南流,受★E2容水,入雒容為洛清江。石流江,上源曰四牌溪,自修仁流經縣東南,折而西,亦入雒容。西南:蘭麻山,俗呼攔馬山,攔馬水所出,東入永福江。山路險絕。有蘭麻鎮。桂林營左哨駐城。南寨沙、西南理定、黃冕、鹿寨有汛。縣丞駐鹿寨。驛二:三里、橫塘。義寧難。府西北八十二里。明初屬府。後屬永寧州。順治初改屬。西南:華嚴山、靈鷟。西北:智慧。北:丁嶺,義江出,南流,右受觀音田山、江頭嶺北諸山溪水,又南,智慧江合高家山水東南流注之。經治西,左受石豪江,南入臨桂為白石江。西北舊有桑江口廢司。乾隆六年,以所屬置龍勝通判。東:楊梅關。義寧協左營分防汛駐城。南有大嶺汛。全州沖,繁,難。府東北二百五十四里。東:黃華山。西:覆釜、湘山、禮山。西北有七十二峰,州西諸水多濫觴山麓。湘水自興安入,右受建安鄉水,又東北,四溪源、大朝源水合為長亭江,北流注之。經治南,灌江自灌陽北流注之。羅水承萬鄉、寨墟、大會三水,東南流注之。又東北,宜湘河合橫溪、梓溪、鍋頭、玉升諸源水,西北入湖南零陵。西延水出州西,東北合眾小水,入湖南新寧。西延鎮,州同駐。有山角、山棗二巡司。康熙二十五年,改置全州營,駐城。黃沙、西延並有汛。驛一:零陵。灌陽難。府東北三百六十四里。明屬全州。順治初改屬。東北:麒麟山。東南:三峰山。西南:賚子山。灌江,古觀水,出縣西南仙人掌諸山,合牛江,至黃牛市,鹽川承賚子諸山水,東南流注之。又東北,右受澥江水,經治東,左受龍川水,馬渡江承烏石江、黑巖水,東南流注之,北入全州,合於湘水。東北石櫃、永安,北昭義有關。全州營分防汛駐城。北:巨巖堡汛。西南:崇順裡巡司。龍勝要。府西北二百十七里。明,義寧縣地。乾隆六年,析義寧西北地置,改龍勝理苗通判。東南:龍脊山。西南:大羅山。貝子溪自湖南城步入,西南流,右合芙蓉溪,經治北,左受牛脛溪、平水江,又西北,南平江合西南諸水北流注之。太平溪亦自城步入,合獨車溪,西南流注之,西入懷遠,是為潯江。有義寧協左營駐防副將駐,右營分防。西北廣南、城北小江、獨車,東北龍甸、芙蓉、貝子,西瓢裏,西北石村有汛。西:廣南巡司。又有龍勝巡司,本桑江,改駐城。中渡要。府西南百二十里。光緒三十年,剿平四十八峒匪,以桂林同知帶兵移駐峒內。三十二年,析永寧、永福、融、柳城、★E2容地置,四十八峒俱入轄境。改桂林同知為中渡撫民同知,仍屬府。中渡江上源即富河江,自永寧入,東北受北來一水,折東南,經治東,受西來一水,東至永福曰雒容水,入永福江。有保安營駐防。西:平山巡司,光緒三十二年,以原駐中渡平樂司改置,移駐峒內平山墟。

柳州府:沖,難。右江道駐,府隸焉。明洪武元年為府,領州二,領縣十。順治初,因明舊為府。雍正三年,升賓州為直隸州,以府屬之來賓、武宣、遷江、上林四縣隸之。十二年,降賓州隸思恩府,來賓還屬。有柳慶鎮標左右營、柳州城守營駐防。提督舊駐府,光緒十二年移駐龍州,置柳慶鎮總兵官駐。光緒末年廢。東北距省治三百六十里。廣百五十里,袤五百里。北極高二十四度二十一分。京師偏西六度五十七分。領州一,縣七。馬平沖,繁,難。倚。明府治,因之。城東北隅:鵲山。南:仙奕、石魚。東南:甑山。東北:龍壁山。柳江即黔江,省境西江北岸第一大受渠也。自柳城入,南與五都水合。屈東南,繞府治西、南、東三面,東北經橫瀨山麓,左界雒容。復西南,三江出縣西,伏流,至雞公山北復出,東北流注之。東流,左受洛清江,折南入象州為象江,穿山水從之。東新興、西樟木有鎮。東振柳、東南白沙、西北洛滿塘、南穿山、西南三都墟、雞公山有汛。穿山、三都二巡司。驛二:雷塘、穿山。雒容沖,難。府東北六十里。光緒三十二年,劃長盛坳以北地屬中渡。西:橫瀨山。西北:八角嶺。西南:獨靜山。南有柳江,緣界東流入象州。洛清江自永福流入縣東北,西南流,左受山道江,經治南,屈曲東南流,注柳江。柳慶鎮左營分防汛駐城。西洛垢墟、西南高嶺、南豐軌鄉有汛。東南有江口鎮巡司,因明舊置。羅城難。府西北百四十里。東北:覆鐘。北:青陵、磨盤山。東:大小蒙山。西:九萬山。武陽江舊名歸順水,有二源,一出西北平西里,一出東北高懸里,合於寺門墟,東南入融江。大仁江一曰通道江,源出西北大仁崗,東流,經三防司北,東入融縣曰背江。融懷營分防汛駐城。北:通道汛。通道舊有鎮,當萬山中,多瘴癘。明置巡司,乾隆五十一年改為三防塘主簿。又武陽鎮巡司,因明舊置。柳城簡。府北七十里。南:烏鸞。西南:青鳳。東:伏虎山。北:融江自融縣流入縣北曰柳江,東南流,沙鋪水出融縣思管鎮,西南流注之。洛澰河出中渡黃泥村,西流注之。經治西,會龍江,南入馬平。柳慶鎮右營分防汛駐城。東北山嘴汛、西北古砦鎮二巡司,因明舊置。驛三:馬頭、東江、羅江。懷遠沖,難。府北三百十里。北:白雲、龍頂、九曲。東北:林溪。西北:朝萬山。西:溶江,上源曰黔江,自貴州永從入縣西北曰福祿江。東流,合蔡江、大年河、南江,又東,左受腮江、孟團江,折南受潯江。經治北,歧為二,繞至治南復合,入融縣。潯江,即貝子溪下流也,自龍勝入,西南流,左受斗江,右受石眼江,西南注福祿江。融懷營駐城。西北石牌塘、沈口有汛。東北潯江、西北萬石有鎮。梅寨巡司。古宜甲主簿。驛一:在城。來賓沖,難。府南百八十五里。明屬府。雍正三年改屬直隸賓州,十二年還屬。北:龍鎮山、瑞象。東北:鵝頭。東南:金峰山。城南大江即紅水江,一曰都泥江,西江幹流也,自遷江入,東北流,左受北三江,至城南,白馬溪出白牛峒,北流注之。又東北,右受觀音山水,左受定清水,復折東南,與象江會,曰潭江,入武宣。東:蓬萊鎮。賓州營分防汛駐城。東南:平安汛。有界牌巡司,因明舊置。驛一:在城。融沖,難。府西北二百五十里。西南:真仙巖,一曰老君洞。西北:攬口山。東北:老鴉山。福祿江自懷遠入,左受寶江,曰融江,西南流,浪溪江自永寧來,合南江,西流注之。又西南,背江上承三源,其一即羅城通道江也,合於三江門,東南流注之。經治東,西南流,左受清流江,右受高橋江,合羅城之武陽江,南入柳城。南:清流鎮。融懷營分防汛駐城。東南思管鎮、東北長安鎮二巡司,因明舊置。象州沖。府東南百五十里。西:象山。又西:西山。南:獨傲。東:雷山。東北:聖塘山。北:象江即柳江,自馬平入州。北,運江,上承仁義、下里二江,西流注之。折西而南,經治西,又南,古城江自武宣來,西流注之,南入武宣。甕嶺,城東南,熱水江所出,一曰十里江,溉田其廣,西北入象江。柳慶鎮左營分防汛駐城。東北:大樂汛。龍門寨巡司,因明舊置。驛一:象臺。

慶遠府:繁,難。隸右江道。慶遠協左營駐防副將駐。明洪武三年復為府,領州四,縣五,長官司三。順治初,因明舊。雍正七年,劃分東蘭土州,同升東蘭土州為州,設流官。十年,改河池州屬之荔波縣隸貴州都勻府。光緒三十一年,置安化。東北距省治五百八十里。廣四百七十里,袤二百九十里。北極高二十四度三十分。京師偏西七度四十二分。領一,州二,縣三,土州二,土縣一,長官司三。宜山繁,難。倚。北:北山,一曰宜山,下臨龍江。南:大號、南山。西:羊角。東:小曹、大曹。西北:龍江上源曰勞村江,柳江西系也,自河池入,合東江,東南流,折北,右受馬鬃河,左受中洲小河。又東南,經府治北,合洛蒙江、思吾溪,經永順副土司南,受永順水,東至柳城合於融江。東大曹、西懷遠、東江有鎮。有白土、德勝、龍門三巡司。縣丞駐楞村。水驛二:大曹、宜陽。馬驛:德勝。天河難。府北八十五里。東:東山。北:獨俊山。西北:高寨山。東小江源出羅城,流入縣東北,合數小水,西南流,經治北,思吾溪合西北小水,東南流注之,又東南入宜山,注龍江。西南:福祿鎮。慶遠協左營分防汛駐城。河池州難。府西二百五里。北:鳳儀山,州城半枕山麓。東:都銘。南:天馬。西:吳山。東北:屏風。金城江,龍江上流也,自南丹土州入,右合秀水,經治南,伏而復出,東入宜山。洪龍江出南丹北,為中平溪,流入州西,右受坡旺水,東南入永順土司,下流為刁江。慶遠協右營駐城。州同駐三旺里。思恩難。府北百二十里。明,舊屬府。正德元年,屬河池州。順治初,改屬府。明建治歐家山。順治中,遷治清潭村,十五年仍徙歐家山。北:馬廕、紺山。南:寒山。東:三峰。東南:米嶺。環江,東江上流也,自安化入縣北,屈曲南流,經治西至宜山,注龍江。中洲小河亦自安化入,南流至宜山注龍江。慶遠協右營分防汛駐城。安化要。府北二百里。光緒三十一年,析思恩北境置,以宜山德勝理苗同知移駐五十二峒,改為撫民理苗同知。東北:中洲上里,接貴州古州八萬瑤山界。中洲小河自古州流入東北,中有沙洲,四面水繞,分上、中、下三里,悉為瑤居。環江自貴州荔波來,南流合帶溪,南入思恩。有慶遠防營駐防。東蘭州難。府西南四百四十里。雍正七年以東院內六哨改流建治。東:都夷。東南:霸陵。南:雙鳳。北:福山。西北:紅水江自那地土州入,為隘洞江,右納九曲水,又東南逕那州墟,左合平紬江,逕板馬墟入興隆。南丹土州府西北三百四十里。北:蓮花山、青雲峰。西北:孟英。東:金雞山。東南:三寶、羅侯。東北:勞村江自貴州荔波入,東南流,右受金城江,左界思恩,南流入河池。中平溪出州北十里許,經治東,東南流,入河池,曰洪龍江。那地土州府西北三百四里。北:黃花嶺。西北:翠靈山。又三碧、虎山。紅水江自凌雲流入州西北,左受一水,東南入東蘭。有龍泉溝,出州北黃花嶺,一州水利賴之。東蘭土州府西五百二十里。明,東蘭州地。雍正七年降土知州為州同,分轄鳳山外六哨地。北:交椅。東:十八鶴。東北:九曲。喬英墟水出州西北銀騰隘,東南流,入水雲洞,至州治北復出,經治西,至坡龍村,伏流數里,又南出,復伏流入百色。三里墟水出東北巴華村,東流,右納一水,折入東蘭。忻城土縣府南少東九十里。北:馬鞍山。西北:疊石山。紅水江自安定土司緣南界東流,右為上林界。龍塘江出永定土司,南流注之。又東南,右界遷江,左受古萬墟水,入遷江。永定長官司府南六十里。南:頭盔山。東:羅漢山。司東:鳳凰嶺。西北:龍橋江出司西北,東南流,合北來一水,南入忻城,曰龍塘江。永順正長官司府西南三百里。南:高椅山。北:西龍山。司北:多靈山。東北有泉溉田。刁江,洪龍江下流也,自河池流入司西北,東南流,經司治北,入安定土司。永順副長官司府東北四十里。永順水有二源,一自羅城入,一自柳城入,至司南合流入龍江。

思恩府:繁,難。隸右江道。明正統四年升府,六年改軍民府,領州二,縣二,土巡檢司九。順治初,改明軍民府復為府。康熙二年,鎮安土府改流官來屬。三年,降直隸土田州來屬。五年,升安隆長官司為西隆州,上林長官司為西林縣,並來屬。雍正七年,升鎮安土府為府,以向武、都康、上映三土州隸之,析土田州置百色。八年,改西隆州西林縣隸泗城府。十年,改奉議州判隸鎮安府。十二年,降直隸賓州,並所領遷江、上林二縣來屬。乾隆七年,析土田州置陽萬土州判。同治九年,廢那馬土司,改置那馬。光緒元年,升百色為直隸,廢土田州,置恩隆縣及上林土縣、下旺土司,往屬之。五年,改陽萬土州判為恩陽州判,並屬百色直隸。東北距省治千百五十里。廣三百三十五里,袤二百四十里。北極高二十三度二十七分。京師偏西八度五分。領一,州一,縣三,土司七。土司疆域,華離甌脫。無附郭縣。東北:三臺山,東溪出。西北:筆架山,西溪出。夾城而南,合為府江,入武緣。武緣繁,難。府南七十五里。明正統七年,府遷治喬利,在今治北四十里白山土司境。嘉靖七年,徙縣北止戈里之荒田驛,即今治安山。府治北:蜿蜒山。東北:大名。東:思粦山。縣東:黃道山。北:高峰嶺。府江二源,至府城南合流,左受大攬江,右受仙湖江,經治西,與南流江會。南流江自上林經縣東北,受名山、黃塘各墟水,折西南,馱淺江出縣東南,西流注之,會府江。又西南,左受那楞江,又西,右受三朝水,西南流,入隆安。思恩營駐城。府城有分防汛。西有高井寨巡司,舊駐上林土縣,乾隆十九年移駐羅墟。賓州繁,難。府東二百里。明屬柳州府。順治初因之。雍正三年升直隸州,領遷江、上林、來賓、武宣四縣。十二年降州來屬。南:仙影山。西:古漏山,古漏水所出。鎮龍山,東南七十五里。思覽江,上源曰北江,自上林流經州東北,武陵江出州南,合龍龔江曰李依江,又合丁橋江,北流注之,東入遷江,下流為清水江。丁橋江出州西南,二源合東北流,歧為二,至州治東北復合,入李依江。賓州營駐城。東:安城汛。有安城鎮巡司。驛二:在城,清水。遷江沖,難。府東少北三百二十里。明屬賓州。順治初,隸柳州府。雍正三年改屬直隸賓州,十二年來屬。北:泊鑒山。西北:雲屏。東北:蓮花。東南:牛眠、紗帽。紅水江自上林入,東南流,左受儉排水,經治北,會清水江,東入來賓。思覽江自賓州入縣南,屈曲北流,左受賀水曰清水江,北注紅水江。北三江出忻城土縣,經縣東,北流入來賓,注紅水江。東南有清水鎮。賓州營分防汛駐城北。北四墟、西洛峒有汛。西:平陽墟巡司。上林繁,難。府東百八十里。明屬賓州。順治初,隸柳州府。雍正三年改屬直隸賓州,十二年來屬。北:八角山。東北:雲凌山。東南:張光嶺。南北兩江合流,其下紅水江,經縣北,緣忻城南界,東南流入遷江。北江出縣西北清水隘,東南流,經治北,右受南江,曰鼓江。又東,匯水自縣東北二源合南流注之。又東,右受獅螺江,東南入賓州。縣丞駐三里城。乾隆三年改州同,置三里營駐焉。東鄒墟,北六便,東北喬賢墟、思吉鎮有汛。又東北周安鎮、東南思隴墟二巡司。驛一:思隴。那馬府西北八十五里。明,那馬土巡司。順治初因之。同治九年廢那馬土司,改置通判,仍屬府。東:岊鹿山。東南:岊顏山。西南:蘇韋山,蘇韋水出焉,北流,右受岊馬山水,北入興隆土司。西南:穠企水,出穠企山,北入舊城土司。白山土司府北八十里。舊治西南喬利墟。明末移治隴兔村。吳三桂亂,徙博結村,即今治。南:獨秀山。西南:九兒山。紅水江在北,左界安定土司,東北流,入忻城,合姑娘江,屈東南入上林。興隆土司府北七十里。西南:七首山。西北:天堂嶺。紅水江自都陽土司境南流入司西北,右界恩隆。折東南,經都陽土司南、舊城土司東北,復入司境。又東北,右受那馬水,羅墟、喬利墟水合西流注之,北入白山、安定二土司界。定羅土司府西九十里。北:羅漢山。東北:五更。架溪出舊城土司,東南流,至五更山,右受一水,伏流,經那馬合穠企水,至舊城貢村墟入紅水江。舊城土司府西北百五十里。北:八峰山。東北:崠嶁山。紅水江界司東北境,穠企水流合焉。那感水出治前,南流入武緣。都陽土司府西北二百八十五里。北:岊皁山。南:強山。西:寶珠巖。紅水江界恩隆境入司西南,屈東北,右受北來一水,東南入興隆土司。古零土司府東北八十里。南:紗帽、象山。東北:獅子山。古旺墟水出司東局董村,東南流入龍洞,復出,經古旺墟,至上林注匯水。安定土司府北百六十里。北:大察山。東南:八仙山。紅水江自興隆土司北流,經司東南,合九鄧墟水,逕滅蠻關入,左納刁江,折東南入上林。匹夫關。

百色直隸:要。隸左江道。右江鎮標中左營駐防總兵官,雍正七年由泗城府皈樂墟移駐。明為州,直隸布政司。順治初,因明舊為土田州。康熙三年,改屬思恩府,隸右江道。雍正七年,遷思恩府理苗同知原駐武緣。駐百色,曰百色。光緒元年,田州改土歸流,升百色為直隸,隸左江道,廢土田州,置恩隆縣及思恩府屬之上林土縣、下旺土司並屬焉。五年,又改陽萬土州判為恩陽州判來屬。東北距省治千七百八十五里。廣二百七十五里,袤百五十里。北極高二十三度五十五分。京師偏西九度四十六分。領縣一,州判一,土縣一,土司一。北:探鵝嶺。東:獻寶山。東北:仙橋山。西洋江一曰右江,亦曰鵝江,鬱江北系也,自恩陽來,右岸為恩陽界。東流,經治南,澄碧水自凌雲南流注之。屈南而東,入奉議。寅桑河亦自凌雲南流,經東,入奉議。隆溪亦自凌雲南流,經東北,入恩陽,至奉議,並注右江。篆溪源出東北坡耶墟,東南流,緣界入東蘭,注紅水江,城雍正八年建,亦曰鵝城。恩隆沖,繁。東南百八十里。光緒元年,廢土田州,改流來隸。五年,自來安徙平馬墟,為今治。東:天馬山。北:蓮花山。南:右江自奉議緣界東南流,左受砦桑水,經治南,又東南入上林土縣。東北:紅水江,經縣東北,自東蘭南流,左界興隆土司,右受篆溪,又南,眾水匯合,東流注之。折東北,入都陽土司。恩隆營駐城。東:平馬汛。燕峒在北,縣丞治。東榜墟巡司。恩陽州判西南水程七十里。乾隆七年,析土田州置陽萬土州判,屬思恩府。光緒五年改流,置恩陽州判來隸。西:馬武山。其南:大王山、八角山。西八十里,右江,即西洋江,古牂牁水也,自雲南土富經剝隘入州西北,者郎河北流注之。又東,左右受數水,左界百色境,折南,紫歐溪東北流注之。又東經治北,屈南入奉議。西巴平墟、西北邏村、淥豐墟有汛。西南:東凌寨巡司。上林土縣東南二百五十里。舊屬思恩府。光緒元年來隸。南:那造山。舊治在東岊耀。西北,右江界恩隆、奉議入縣西北,又東南經治北,右納枯榕江,即大含溪,左納小溪,又東南入果化土州。下旺土司東南二百六十里。有甌脫。舊屬思恩府。光緒元年來隸。東:獨秀山。北:魁山。東南:波岌山,波岌水所出,繞司治東北流,入舊城土司。小溪水出舊城南流,經新墟,至上林土縣,入右江。

泗城府:難。隸左江道。右江鎮標右營駐防。雍正五年置右江鎮,駐皈樂墟。七年移駐百色。明為泗城州,與利州直隸布政司。嘉靖二年廢利州。順治初,為泗城土府。十五年為府。尋改為軍民府,屬思恩府。雍正五年復為府,改流官,隸右江道。乾隆五年,置凌雲縣為府治。七年,降直隸西隆州並所屬西林縣來屬。九年,改隸左江道。東北距省治千七百八十里。廣四百二十里,袤二百五十里。北極高二十四度二十五分。京師偏西九度五十分。領州一,縣二。凌雲難。倚。乾隆五年,以泗城府本治置。北:凌雲山、蓮花峰。西:■E3陽。東:三臺坡。西北:青龍山。紅水江為縣北界,自西隆入,東北流,左界貴州貞豐、羅斛境,右受白朗塘、羅西塘水。東南布柳水,上承鞋里、甘田、傘里、巴更各墟水,東北流注之,東南入那地。澄碧水出縣北靈洞,繞治西南流,入百色。東邏樓、農登,東北平蠟,東南龍川,南皈樂,西邏里,西北長隘、百樂,西南汪甸各墟有汛。天峨甲,縣丞治。東有平樂一甲巡司。縣城,嘉慶二年建。西林難。府西南五百十里。明,上林長官司。萬歷中,省入泗城州。康熙五年,改流官升縣,隸思恩府。雍正十二年改屬直隸西隆州。乾隆七年來屬。北:交椅山。東:端峰山。西北:界亭山。右江有二源,南源曰西洋江,北源曰清水河。西洋江自雲南寶寧流入縣南,東北流,與北源會。馱娘江上流即清水河,自西隆來,東南流,右受馱門江,經治東南,者文、那陽、界廷各墟水自縣北合流注之。又東折南,會西洋江,入雲南土富。上林營駐城。東北潞城、東南周馬、南八盤、西八柴有汛。有潞城巡司。縣城,康熙六年建。西隆州沖,難。府西北九百六十里。明,安隆長官司。康熙五年改流官升州,隸思恩府。雍正十二年升直隸州,領西林縣。乾隆七年復為州,來屬。南:三臺山。西:營盤山。西南:金鐘山。南盤江即八達河,西江初源也,下流為紅水江,自雲南寶寧緣界北流,受者扛、羊街二墟水,經州北,東南至北樓墟,冷水河合治西小水,東北流注之。又東北,會北盤江,東南流,入凌雲。清水河即同舍河,自雲南寶寧北流入州西南,折東北入西林。八達城,州同駐。舊州,州判駐。州城有里仁汛。東舊州、東北三隘、東南隆或、西南永靜、古障有汛。州城,雍正七年建。

平樂府:沖,難。隸桂平梧鬱道。平樂協左右營駐防副將駐。明為府,領州一,縣一。順治初,因明舊。宣統三年,析賀縣、懷集暨廣東開建地置信都。西北距省治二百十六里。廣三百八十里,袤二百五十里。北極高二十四度三十五分。京師偏西五度四十七分。領一,州一,縣七。平樂沖,繁。倚。明府治,因之。東:團山、瓜嶺。東南:蓮花。北:目巖。東北:魯溪。桂江一曰府江,自陽朔緣界南流,會修江。屈東,經府治西南,平樂江自北來會,又東南流入昭平。平樂江亦曰樂川,上源曰東江,自恭城入縣。東北納島坪江,又東南,勢江自恭城東南境西北流注之。折西,誕山江合南平江西北流注之,折西南,合於府江。沙子街,縣丞駐。水驛三:昭潭、昭平、龍門。恭城簡。府東北九十里。北:仙姑。西:石盆。東南:五馬山。東北:印山、銀殿。西北:金龍山。東江自湖南永明入,西南流,平川江合平源瑤小河南流注之。折南,右受南江,南錯入平樂。旋復入縣境,經治東,左受下山源、北洞源水,入平樂為樂川。平樂營分防汛駐城。東北:龍虎關汛。有鎮峽寨巡司,因明舊置。富川繁,難。府東北二百六十里。東北:獨秀巖。西南:白雲山。東南:東山。富江出縣西北石鼓山,東南流,左受麥嶺水,經治東南,龍窩水合白源水西南流注之。又南至鍾山渡,折東南,左受白沙水,入賀縣曰臨水。麥嶺,縣北,麥嶺營駐防。雍正八年移同知駐。光緒三十三年徙信都。舊治東南鍾山下,縣徙置鎮,通判駐焉,宣統元年廢。富賀營分防汛駐城。東白沙、東北牛巖、東南鍾山、西北小水峽有汛。西南有白霞寨巡司,因明舊置。賀繁,難。府東南百九十五里。東北:臨賀嶺,即桂嶺,五嶺之第四嶺也,與湖南江華、廣東連山接界。西:瑞雲山。西南:大桂山。臨水自富川經縣西北,東南流,右受馬窩山水,左受裏松墟水,經治北。又東南,右受大桂山水,賀江合桂嶺諸山水西南流來會,東南入信都。富賀營駐城。東龍水、東北大發、大凝墟有汛。縣丞駐桂嶺大會墟。西北:裏松鄉巡司。信都簡。府東南五百七十里。光緒三十四年,析賀縣、懷集暨廣東開建地置,改平樂府分防麥嶺同知為撫民同知,移駐信都。鋪門墟舊隸三縣,撫民耕兵劃歸轄。宣統元年,遷治官潭墟。北:湖頭山。西北:大鼇。西南:雲臺。臨水自賀南流,經治東,又南,右受臨田水,至鋪門墟,深沖源水西南流注之,右受雲臺山水,南入開建。東石牛坡、南鋪門墟有汛。舊信都巡司,光緒三十四年廢,改信都照磨兼司獄。荔浦簡。府西七十五里。東北:三奇、火焰。西北:鏌邪山。東南:鵝翎。修江一曰荔江,自修仁入,左受荔江尾水,東北經治東,左受夾板隘水,右受丹竹江水,又東北,綠水河上承慄江、普陀河、龍坪河東流注之,東北入平樂注桂江。平樂協右營分防汛駐城。北兩江墟、東北馬嶺、西北王瑤隘、西南蓮塘有汛。修仁難。府西南百二十里。東北:羅仁山。西南:凌雲山。南:崇仁大峒。修江出西南瑤山界分水坳,東北流,經治東,至羅仁山東南麓,入荔浦曰荔江。四牌溪出西南文筆山,西北流,經四牌墟,入永福。平樂協右營分防汛駐城。西:石墻堡汛。昭平沖。府南二百里。東:木皮山,其北接米嶺,山高路險。雍正三年,開鑿嶺道,上下四十里。東南:天朝嶺、羊角嶺。桂江自平樂流入縣北,右受歸化江,左受思懃江,經治東,又東南至馬江塘,富郡江出縣東,合招賢水西南流注之,東南入蒼梧。平樂協左營分防汛駐城。東南欖水、東北蓮花、燕塘、山口有汛。東樟木墟巡司,東南馬江塘巡司。永安州簡。府西南百六十五里。東北:石鼓山。東南:石印、古眉、摩天嶺。西南:力山。西北:天堂、馬鬃嶺。眉江,古蒙水,一曰激江,出州西北,東南流,右受濁川水、西江水,經治南,左合銀江,又東南,六樟水東流注之,又東南,榕木嶺水西南流注之,南入藤縣曰濛江。平樂協右營分防汛駐城。

梧州府:沖,繁。桂平梧鬱道治所。梧州協左右營駐防副將駐。明洪武元年為府,領州一,縣九。順治初,因明舊。雍正三年,升鬱林州為直隸州,割博白、北流、陸川、興業四縣往屬焉。西北距省治九百三十五里。廣二百七十里,袤四百六十里。北極高二十三度三十分。京師偏西五度二分。領縣五。光緒二十三年設關通商,以桂平梧鹽法道兼梧州關監督。三十年,由桂林移駐。蒼梧沖,繁。倚。明府治因之。東北:芋莢、古欖嶺。西北:文殊山。南:銅鑊、沖霄。龔江上流即藤江,自藤縣流入,左受安平江,分流夾思化洲、長洲,右受須羅江、長行江,至石磯塘復合,又東會桂江,東入廣州封川曰西江。桂江自昭平入,左受龍江,右受石澗河,東南流,思良江、峽山水出治北南流注之,折西南,至治西注龔江。沿江有水汛。東分界塘,南三角嘴、廣平墟,東南大燕,東北三番,西北卬竹、古欖,西南戎墟有汛。同知駐戎墟。有東安鄉、安平鄉、長行鄉三巡司,因明舊置。水驛二:府門、龍江。梧州關。商埠,光緒二十二年中緬條約開。藤繁,難。府西百六十里。南:靈山。西南:勾刀。西北:穀。藤江上流即潯江,自平南入,流經縣西北,右受都榜江,濛江合牛皮江南流注之,曰龔江。又東南,右受慕寮江,折東,經治北,繡江合思羅江、黃華江、義昌江,自西南經治東來會,曰劍江。又東北,左受四培江,右受黃桶江、白石江,東入蒼梧。梧州協左營分防汛駐城。沿江有水汛。東南糯峒、西白馬有汛。有白石寨、竇家寨二巡司,因明舊置。明故五屯千戶所,稱藤峽左臂,今白石司治。驛四:雙兢、黃甲、金雞、藤江。容簡。府西南四百八十里。東北:朝陽嶺。東南:人文嶺。西北:大容山,亙數百里,潯、鬱分據其麓。容江上承北流之圭江,經縣西南,渭龍江自廣東宜信入,北經治南,左受思登江,曰容江。又東北,右受波羅江,北入藤縣曰繡江。梧州協左營分防汛駐城。有自良墟、粉壁寨二巡司。驛二:自良、繡江。岑溪難。府西南百八十里。東:白石山。西:鄧公山。東南:通天嶺。東北:周公。黃華江自廣東信宜入,流經縣南,西北流,折入藤縣。腰峨嶺,東北義昌江出,西南流,鐵根隘水合黃陵隘水西北流注之。又西,洋羅隘水東北流注之,西北經治南入藤縣。梧州協左營分防汛駐城。東:大洴汛。南:大峒鎮。有平河巡司,因明舊置。懷集繁,難。府東三百里。西:忠讜山。南:天馬。西北:齊岳、牛欄山。西南:白鶴山。懷溪一曰南溪,出縣西北大石屋村,東南流,合古城水、赤水,又東南,右受宿泊水,左受冷坑水、白沙水,經治南,右受甘峒水。又東南,桃花水合東北諸山溪水西南流注之,東南入廣東廣寧,下流曰綏江。永固水出西南南洲山,北流,經永固墟,入廣寧。懷集營駐城。東龍門、東北洽水、西南朋岡有汛。有武城鄉、慈樂塞二巡司,因明舊置。

鬱林直隸州:沖,繁,難。隸左江道。潯州協鬱林營駐防。明為州,屬梧州府,領縣四。順治初,因明舊。雍正三年,升為直隸州。舊隸桂平梧鬱道。光緒十三年,改隸左江道。東北距省治千五百二十五里。廣二百七十里,袤二百九十里。北極高二十二度四十七分。京師偏西六度十分。領縣四。北:寒山。東:信石、峽山。東南:天馬。東北:大容山。西:石人嶺。定川江自興業入,東南流,左受鴉橋江,右受都黃江。又東南,綠藍江自北來,西南流,經治南,合羅望江注之,曰南流江。又南,橋麗江即回龍江,自陸川西南流注之,入博白。東夾山,西平山、石井,北北底,西北蒲塘、楓木有汛。有撫康巡司。西甌廢驛。博白難。州西南九十里。雍正三年,自梧州來隸。南:大荒。東南:蟠龍。西南:九岐、飛雲。西北:綠蘿山。南流江自鬱林入,流經治西,右受綠珠江,左受小白江、大白江。屈西,右合浪馬江,至宴石山西麓,陀角江西北流注之。又西南,左受旺勝江,入廣東合浦。鬱林營分防汛駐城。西南:龍潭汛。有周羅寨、沙河寨二巡司。因明舊置。北流繁,難。州東六十里。雍正三年自梧州來隸。東北:句漏山,山脈自越南來,東入廣東境,鬱江南岸一大系也,東會靈山。西北:大容山。南:綠藍山,綠藍水所出,南入鬱林注南流江。圭江出縣東南,有二源,一石梯水,出大雲嶺,一雙威水,出雙威山,至三口鋪合流曰圭江,西北流,思賀江自陸川東北流注之。右受螭蜍河,經治東入容縣,曰容江。鬱林營分防汛駐城。東南陸靖、善逕有汛。有雙威寨巡司,因明舊置。驛一:寶圭。陸川難。東南九十里。雍正三年自梧州來隸。東:文龍山。西:鳴石。西北:石湖。東南:大這。北:分水山,二水源出焉:一南流,經治東,饅頭嶺水自西北繞城南合流曰烏江,又南,左受水車江、龍化江,曰平南江,入廣東石城;一西北流,曰回龍江,屈西南,合略峒江,入鬱林,曰橋麗江。鬱林營分防汛駐城。北馬坡、南花槎有汛。有溫水寨巡司,因明舊置。有永寧廢驛。興業簡。州西北七十三里。雍正三年自梧州來隸。北:北斗山,與東斗山對峙。西:萬石、白馬巖。西南:夔龍巖。定川江三源,北源曰龍穿江,出縣西北,東南流,通濟江自東北繞城來會,岑江自西來會,三江合曰定川江,東南入鬱林。鬱林營分防汛駐城。北番車、南六纂、西南雷塱、城隍墟有汛。

潯州府:沖,繁。隸右江道。潯州協左右營駐防副將駐。明洪武元年為府。領縣三。順治初,因明舊。雍正八年,武宣來屬,舊隸左江道。乾隆九年,改隸右江道。東北距省治八百七里。廣四百里,袤五百二十里。北極高二十三度二十九分。京師偏西六度十六分。領縣四。桂平沖。倚。明府治,因之。西:西山、石梯。西南:羅叢。東南:白石、大容、紫荊關。西北:大藤峽數百里,跨西江西岸,明韓雍破瑤賊地,咸豐金田之役實肇亂於此。黔江上流即柳江,一曰右江,自武宣入縣西北,左出支津曰南淥江,東南經治北,與鬱江會。鬱江一曰左江,自貴縣經縣西南,右受繡江,左受蓬閬江,屈曲東北流,至治東會黔江,曰潯江。又東北,大江嶺水合東南諸水西北流注之。南淥江,合相思江東流注之,東入平南。有南北河水汛。北靜安、東北武靖有鎮。有大黃江、穆樂墟二巡司。平南沖。府東九十里。東北:勾崖嶺。東南:黃花山。西北:閬石山。潯江自桂平入,左受思旺江,東南流,烏江合數小水南流注之。經治南,又東南,左受秦川河、白沙江自桂平東北流注之,東入藤縣。大同江出西北龍軍瑤地,東南流,亦入藤縣。潯州協左營分防汛駐城。東南樟木墟、丹竹墟有汛。有大同鄉、秦川鄉二巡司,因明舊置。水驛:烏江。貴繁,難。府西南百七十里。北:大北山。東:小北山。龍山,北五十里,藤峽右臂也。西:金雞峽。鬱江自橫州入縣西南,武思江自廣東合浦來,北流注之。東北流,左受思繳江,寶江自賓州來,東流注之。經治南而東,左受沙江、東津江,右受橫眉江,入桂平。潯州協右營分防汛駐城。東南三塘、西覃塘、西北五山有汛。有五山鎮巡司。通判駐木梓墟,宣統二年廢。水驛二:東津、香江武宣沖。府西北百九十里。明宣德六年更名,屬象州。順治初,隸柳州府。雍正三年改屬直隸賓州,八年來屬。南:大藤峽。東北:金龍山。西南:仙巖山。柳江緣象州、來賓界,流經縣西北,受古城水,屈曲南流,經治南,勾樓山水西南流注之。又東南,右受古豪江、武賴水,左受陰江、新江水,入桂平。潯州協左營分防汛駐城。南寺村、西南大樟有汛。有縣廓鎮巡司。驛一:仙山。

南寧府:沖,繁,難。左江道治。左江鎮中左右營、南寧城守營駐防總兵官駐。光緒三十二年,提督由龍州移駐。明洪武元年為府,領州七,縣三。順治初,因明舊。雍正十年,下雷土州改隸鎮安府。光緒十三年,上思州改隸太平府。東北距省治千十里。廣三百里,袤百五十里。北極高二十二度五十四分。京師偏西七度五十六分。領州二,縣三,土州三。宣化沖,繁,疲,難。倚。明府治,因之。北:高峰山。西北:聖嶺。東北:昆侖山。鬱江即左江,省境西江南岸一大受渠也,上承左、右二江。左江自新寧入,東北流,右江自隆安入,東南流,至合江鎮合流曰鬱江。屈折東流,左受星盈江,經府治南,右受烏水江、八尺江。又東,左受大沖江、伶俐江,入永淳。東八尺江、西三江口、南那曉墟、北宣賓陸路有汛。有八尺寨、三官堡、金城寨、遷隆寨、墰落墟五巡司。驛四:建武、黃花、陵山、大淮。南寧關。商埠,光緒三十二年自開。新寧州簡。府西百十里。北:青雲山。南:獨秀山。東北:六合山。麗江一曰左江,亦曰定淥江,鬱江南系也,自土江州入,受響水,東北流,右受旺莊河,經治北,左受淥甕水,曰左江,入宣化。左江鎮右營分防汛駐城。隆安簡。府西北二百八十五里。東:馬王山。東南:金榜、梅龜山。右江自果化北、歸德南東南入,左納塘河水,南,右受佛子溪、曲霞溪,經治北,東南流,綠絳水自萬承來,合羅興江,東北流注之。又東南,南流江亦自武緣來會,折南入宣化。左江鎮左營分防汛駐城。西北有果化卡汛。橫州沖,繁。府東南二百四十里。東:橫嶺。北:震龍山。東北:定祥山。西北:平天嶺。鬱江自永淳入,東流,右受橫槎江、平南江、鹿江,經治南,東北流,左受清江,右受武流江,折北,古江自永淳東南流注之,東北入貴縣。南寧營分防汛駐城。有大灘巡司。水驛二:烏蠻、川門。永淳簡。府東二百五里。明屬橫州。順治初,改屬府。東:雷峰嶺。東南:火煙。東北:鎮龍山。鬱江自宣化入,東南流,經治北,東班江自賓州來注之。繞城東南,秋風江自廣東靈山來注之。又東南入橫州。左江鎮中營分防汛駐城。西南:那懷汛。北有武羅、南里鄉二巡司。水驛二:永淳、火煙。土忠州府西南二百二十里。北:芭仙山。旺莊河出州南,東北流,經治南,折北入新寧,注定淥江。歸德土州府西北三百二十五里。北:九兒山。右江自上林流經白山南、果化北入,緣界東南流,至馱灣村,入隆安。果化土州府西北三百六十里。南:青秀山。東南:獨石山。右江為州北界,東南流,經旺墟,右界歸德,入隆安。

太平府:沖,難。太平思順道治所。新太協左營駐防副將駐。明洪武二年為府,領州十七,縣三。順治初,因明舊。雍正三年,置上龍、下龍二土司。七年,廢下龍司,置龍州。十年,改思明土府為土思州,並所屬下石西土州來屬。十一年,改思明州為寧明州,置明江,兼管上石土州事,降直隸江州為土州,及所屬羅白縣為土縣,又降思陵、憑祥二直隸為土州,並來屬,省思城土州入崇善縣。光緒十三年,上思州來屬。十八年,升上思州為直隸。宣統二年,憑祥土司改流官,置憑祥。舊隸左江道。光緒十三年,改隸太平思順道。提督駐龍州,督辦邊防。光緒十二年由柳州移駐。二十九年,改置督辦邊防大臣。三十一年廢。以太平思順道辦理邊防事務。自光緒十一年越南淪陷,法人逼處西南一隅,與越南諒山、高平、宣光等省接壤;邊防處處險要。分三路:自鎮南關口及關以內憑祥所轄各關前隘為中路;關以東,自明江寧明州暨下石西、思陵土州,至土思州屬派遷山止,所轄各隘為東路;關以西,自龍州、歸順州即下凍、下雷土州,至鎮邊縣屬各達村巖峒橋頭止,所轄各隘為西路。沿邊千八百九十四里,隘卡百五十有六,有防兵二十五營,分扎沿邊對汛及各砲臺。東北距省治千二百八十里。廣五百七十里,袤六百六里。北極高二十二度二十五分。京師偏西八度五十分。領二,州四,縣一,土州十六,土縣二,土司一。崇善沖,難。倚。明府治,又思城州地。順治初因之。雍正十一年省思城土州入,以縣丞分駐。北:青連。東:將軍。東南:銀山。西北:翠微、馬鞍山。西南:麗江自上龍流經縣西南,左受邏水,東北流,繞治西、南、東三面,納邏水,舊名歸順河,一曰舊縣江,自安平南流,通利江自養利來注之,又經太平流入縣西北,右受多烈水,注麗江。東北有崩坎汛。有馱盧巡司。左州沖。府東北九十五里。南:天燈。東:雲巖。西北:金山。西南:華父山。麗江自崇善緣界,經崇善馱盧司,左受橋龍江,東北入新寧。新大協左營分防汛駐城。驛一:馱林。養利州難。府西北百四十里。東:武陽山。南:無懷。西:印山。西北:通利江自龍英緣界,東南流入州,西經響水橋,大嶺墟水東北來,至迎恩橋注之,南入崇善。新太協左營分防汛駐城。永康州難。府東北百八十里。明萬歷二十八年升州,省思同州入,與陀陵縣並屬府。順治初因之。康熙三十八年,陀陵縣並入州。南:鳳凰山。西南:天馬山。西:吞臼、星游、月獅、連吸諸嶺。西北:淥倥山,淥倥江所出,一曰綠甕江,屈曲東南流,經治西南,淥零水東流注之,東南入羅陽。新太協左營分防汛駐城。寧明州沖,難。府西南百二十里。明,思明州。順治初為土州。康熙五十八年改流官。雍正五年罷知州,以思明府同知兼管州事。十一年,以思明四寨、六團改置寧明州。乾隆元年移治思明土府舊城。東北:風門嶺。西北:龍勝山。西南:伏波山。明江自土思州入,西流,經明江西南,交趾河自越南來,左合下石州水注之,西北經治北,曲流三十餘里,入上龍土司會龍江。馗纛營駐防風門嶺。西南羅隘有汛。明江沖,難。府西南百十里。明,思明府又上石西州地。順治初為思明土府同知。雍正十一年改為明江理土督捕同知、兼管上石西州事,駐思明土府舊城。北:珠峽。西南:伏波。東北:風門嶺。東南:白馬山。明江自土思州入,屈西,逕明江至城東,又西北逕那關山,入上龍。馗纛營分防汛駐城。龍州沖,難。府西百八十里。明,龍州,直隸布政司。順治初來屬。雍正三年罷州,析其地為上龍司、下龍司,置土巡檢。七年廢下龍司,移太平府通判駐劄。乾隆五十六年改同知。東:獨山。北:軍山。西南:秀嶺。龍江有南北二源:北源曰平定溪,自越南流入西北水口關,東南流,經上下凍土州,至治西南,會南源;南源曰豈宜溪,自越南流經西南平南關,界憑祥北境,屈東北至治西南,與北源合,東流曰龍江,入上龍。光緒十三年,提督來駐。二十九年,督辦邊防大臣駐。尋廢,提督移駐南寧。有提標中營、龍州城守營駐防。西北:水口關、斗奧隘有汛。龍州關。商埠,光緒十三年法越商務條約開。有東西關砲臺。有鐵路。憑祥府西南二百三十里。明成化十八年升州,直隸布政司。順治初,為土州來屬。宣統二年改流官,置撫民同知,並明江同知兼攝之。舊上石土州入焉,並兼轄承審下石土州,仍舊治。東:白石山。南:叫穀山、馬鞍山。龍江南源界北境,憑祥水自治南合澗水北流注之,東南入龍州。西南:鎮南關,一曰界首關,越南入境第一門戶也。有左右輔山砲臺。東受降城、北平南關、南由隘南關、西南咘沙卡有汛。太平土州府西北百十里。東:九峰山。東南:龍蟠山。西北:邏水自安平入,右受五橋水,東南入崇善。多烈水亦自安平入,東南流入上龍。安平土州府西北百三十七里。南:會仙巖,西南:星山。邏水自下雷入,東南經州署北,左界崇善、思城境,入太平。多烈水自越南流入,經巖昆山南麓,又東南亦入太平。五橋水出州西北要村隘,東南流,經五橋,至太平土州,注邏水。萬承土州府東北二百五十里。東北:金童山。西北:蓮花山。西南:雲門、紫洞。綠降水一曰玉帶水,出州西南玉屏山,經州署南,東北入隆安,注右江。茗盈土州府西北百七十里。南:岊懷山。東北:觀音巖,澗水出焉,西來一澗水,至州署南合流,曰茗盈水,西南經養利入龍英,注通利江。全茗土州府西北百六十里。北:州望山。西北:猛山。通利江自龍英東南流,至仙橋入境,又東南經州署西,合布顯水,屈西南,復入龍英。龍英土州府西北二百里。北:筆架山。西南:通山。通利江自都康東流入州,西北,左受寧墟水,東南流,屈西南,經全茗境,復折入州,州署前諸水合穠茗水東流注之,又東南,納茗盈水,入養利。結倫土州府西北二百三十里。東:高峙山。東北:陽果嶺。西北:斗牟山。咘吝水出山澗中,流繞州前,南有咘畢水自都結來,堰水上流也。結安土州府西北二百二十里。東:馬鞍山。南:窟井山。北:飛鼠山。西南有堰水,即澗水也,出都結山澗中,流入境,伏而復出,土人堰水灌田,曰堰水。鎮遠土州府西北三百十里。南:筆架山。西:天馬山。北:揚山。西北:布腰巖。磨水出東南,入結安。都結土州府東北三百六十里。北:青雲山。南:觀音山。西:陽果嶺,沛水出焉,曲折東北流,受二小水,東南經州署北曰綠水江,東入隆安。咘畢水一曰澗水,西南流;經結倫,至結安南為堰水。思陵土州府西南二百四十里。明,思陵州,直隸布政司。順治初,為土州來屬。東:天馬山。東北:東陵山。南:角硬山,角硬水所出,東流,右受板邦隘、叫荒隘二水,又東北,折西,經東陵山南麓,又西南,經州署南,入越南。土江州府南二十五里。明,江州,直隸布政司。順治初,為土州來屬。南:波巖山。州東:掛榜山。東南:榕樹嶺。麗江自上龍流經州西北,左界崇善境,東北流,屈東南,入左州、新寧界。土思州府南百二里。明,思明府,直隸布政司。順治初,為土府來屬。雍正十年改土州,更名,移治伯江哨。西:飛仙巖,西南:摩天嶺。東南:派遷山。明江自遷隆峒入,經州署北,又西入寧明。東有海淵墟汛。驛一:明江。下石西土州府西南百六十里,明屬思明府。順治初來屬,歸寧明州兼轄。宣統二年,改歸憑祥。西北:白樂山。西:獨山。東有一水流合交趾河,東北注明江。上下凍土州府西二百二十里。南:湖山。北:岊梱山。西南:八峰山。龍江北源自越南入龍州西北轄境,東南流,經州署北,咘局隘水合咅花隘水,東流至州南注之,東入龍州。有龍州營分防汛。羅白土縣府東南五十里。明屬江州。順治初來屬。東南:龍洞山。西南:羅高山。北:獨龍山。隴水出,西北入江州。羅陽土縣府東北二百里。東:青龍山。西北:白虎山,一曰白面山。綠甕水自永康流經縣署西南,又東南,沙房墟水合一水東流注之,入新寧。上龍土司府西百八十里。明,龍州地。雍正三年析置。西北:武德山。北:古甑山,古甑泉出焉,南流經司署,西入龍州。龍江自龍州入司南,東北至三江口會明江。屈東南流,合邏水,入崇善。多烈水自太平入司東北,亦東南入崇善,注邏水。

上思直隸要。隸左江道。提標上思營駐防。明,上思州,屬南寧府。順治初,因明舊。光緒十三年,改屬太平府。十八年,升為直隸,以南寧府屬之遷隆土司隸之。東北距省治千二百八十里。廣百二十五里,袤七十三里。北極高二十二度十一分。京師偏西八度十三分。領土司一。北:望州山。西南:營盤山。十萬大山環列東、南、西三面,延袤百餘里,接廣東欽州訖越南祿州界,游匪出沒所也。沿山有八隘。明江出西南十萬山中,東北流六十餘里,屈西北,經治南,又西南,平弄隘、平寨隘二水合會遷隆峒、板蒙隘水北流注之,西北入遷隆。遷隆峒土司西七十里。明土巡檢司,屬上思州。順治初因之,與上思並屬南寧府。光緒十八年來屬。北:分界嶺。東南:那馬。明江自入,西北逕城東,屈西南入土思州。

鎮安府難。隸左江道。明洪武二年為府。順治初,為土府,隸思恩府。康熙二年,改設流官通判。雍正七年,升為府,隸右江道,以向武、都康、上映三土州隸之,歸順州改流來屬。十年,思恩府屬之奉議土州,改設流官州判,又南寧府屬之下雷土州並來屬。改隸左江道。乾隆三年,置天保縣為府治。三十一年,置小鎮安。光緒元年,升奉議為州。十二年,升歸順州為直隸州,改小鎮安為鎮邊縣,並下雷土州往屬之。東北距省治千六百八十五里。廣百三十里,袤百六十里。北極高二十三度十九分。京師偏西九度四十三分。領州一,縣一,土州三。天保難。倚。明,鎮安府地。乾隆三年置。北:天保山,東北:扶蘇山。西:鑒山。歸順江一曰浤渰江,自歸順入縣西,伏流,至鑒山前復出,東流經治南,右受馱命江,天保泉自北來注,又東,左受咘來河,右受歸順之武平河,東北入奉議。鎮安協右營分防汛駐城。奉議州沖。府東北二百十里。明洪武二十八年改衛,尋復為州,直隸布政司。嘉靖中,改屬思恩府。順治初為土州。雍正十年改掌印州判來屬。光緒元年升為州。東北:獅子山。東南:三齊山。西:大小蓮花山,有蓮花關。右江自百色南流入州,西北折東,經治北。又東北,寅桑河自百色來,隆溪自恩隆來,南流注之。東南流,右受歸順江,左界恩隆入上林。鎮安協左營分防汛駐城。西南:古眉墟汛。東南:作登墟巡司。向武土州府東南百六十里。明初屬田州府,尋廢。建文中復置,直隸布政司。順治初,隸思恩府。雍正七年來屬。東南:天臺。東北:向陽山,山上有關。西北:上旱山,下有上旱溪,出天保山中,東北流,入奉議。枯榕江一曰大乃溪,出上映州山中,東北曲流合勞溪,經州署西北,又東北入上林。有鎮安協右營分防汛。都康土州府東南百九十里。明初沒於夷僚,建文初復置,直隸布政司。康熙三年,改隸思恩府。雍正七年來屬。東:崇山。北:映秀。南:翠屏山。通利江自上映州入,東流,經州署南,左受岊營水,入龍英。上映土州府東南百八十五里。明初廢為峒。萬歷中復置,隸思恩府。順治初因之。雍正七年來屬。南:錦屏山。西:嶔岫山。西南:八字咅。通利江上源為秀泉,出州西北山中,東南流,經州署南,至仙橋入都康。

歸順直隸州繁,難。隸太平思順道。鎮安協左營駐防副將駐。明,歸順州,直隸布政司。順治初,為土州,隸思恩府。雍正七年,改流官,隸鎮安府。乾隆十二年,省湖潤寨土巡檢司入焉。光緒十二年,升為直隸州,隸太平思順道,改鎮安府屬之小鎮安為鎮邊縣,及下雷土州來屬。東北距省治千八百六十里。廣二百二十里,袤百六十里。北極高二十三度六分。京師偏西九度五十四分。領縣一,土州一。南:獅子。西:嶺衛。西南:叫鵝山。西北:三臺、照陽山。龍潭水出城東北里許,南流經治東,穠黎水出州西北,東南流注之,入越南。穠那水出州西,亦東南入越南。歸順江出西北耑雷墟,武平河出東北小龍潭,並東流入天保。邏水出州東,東南流,左受立咅水,入下雷。西:榮勞墟。南:隴邦、壬莊、頻峒各隘有汛。東南:湖潤寨巡司。鎮邊繁,難。州西北二百三十里。明永樂中分置鎮安土州,屬思恩府,尋廢。乾隆八年設土巡檢。三十一年改流官。通判駐轄曰小鎮安,光緒十二年改置縣,更名來屬。北:感馱巖。又北:末山,有水西北流,入雲南土富。勞山,勞水所出,西北流,經治西,合大魁水、弄內水,折東北,伏流復出,亦入土富。德窩水出縣南,東南流,經百合墟。折西南,茍華水、坡芽水、百都水並出縣西南,合流注之,入越南。那摩水自越南入州西南邊境,合坡酬水,復西入越南。鎮安協右營駐防。下雷土州府東南二百二十里。本下雷峒。明萬歷十八年,升為州,屬南寧府。順治初,因之為土州。雍正十年,屬鎮安府。光緒十二年來屬。北:天關山。南:地軸山。又南:神農山。邏水一曰西北河,自歸順入州西北,北河自向武來,伏流復出,西南流注之,經署東,又東南,西南河自越南緣界東流注之,入安平。有鎮安營分防汛。


\end{pinyinscope}