\article{本紀十}

\begin{pinyinscope}
高宗本紀一

高宗法天隆運至誠先覺體元立極敷文奮武欽明孝慈神聖純皇帝,諱弘歷,世宗第四子,母孝聖憲皇后,康熙五十年八月十三日生於雍親王府邸。隆準頎身,聖祖見而鍾愛,令讀書宮中,受學於庶吉士福敏,過目成誦。復學射於貝勒允禧,學火器於莊親王允祿。木蘭從獮,命侍衛引射熊。甫上馬,熊突起。上控轡自若。聖祖御槍殪熊。入武帳,顧語溫惠皇太妃曰:「是命貴重,福將過予。」

雍正元年八月,世宗御乾清宮,密書上名,緘藏世祖所書正大光明扁額上。五年,娶孝賢皇後富察氏。十一年,封和碩寶親王。時準噶爾役未竟,又有黔苗兵事,命上綜理軍機,諮決大計。

十三年八月丁亥,世宗不豫。時駐蹕圓明園,上與和親王弘晝朝夕謹侍。戊子,世宗疾大漸,召莊親王允祿,果親王允禮,大學士鄂爾泰、張廷玉,領侍衛內大臣豐盛額、訥親,內大臣戶部侍郎海望入受顧命。己丑,崩。王大臣請奉大行皇帝還宮。莊親王允祿等啟雍正元年立皇太子密封,宣詔即皇帝位。尋諭奉大行皇帝遺命,莊親王允祿、果親王允禮、鄂爾泰、張廷玉輔政,並令鄂爾泰復任,以鄂爾泰因病請假也。以遺命尊奉妃母為皇太后,復奉懿旨以上元妃為皇后。召大學士硃軾回京。命大學士嵇曾筠總理浙江海塘工,趙弘恩署江南河道總督。大行皇帝大殮,命以乾清宮南廡為倚廬。庚寅,命總理事務王大臣議行三年喪。命履郡王允祹暫管禮部事務。召張照回京,以張廣泗總理苗疆事務,大學士邁柱署湖廣總督。諭大將軍查郎阿駐肅州,與劉於義同掌軍務,北路大將軍平郡王福彭堅守。飭揚威將軍哈元生等剿撫苗疆。癸巳,頒大行皇帝遺詔。

九月丁酉朔,日食。高起、憲德俱罷,仍帶尚書銜。以鄂爾泰總理兵部事,果親王允禮總理刑部事,莊親王允祿總理工部事,甘汝來為漢兵部尚書,傅鼐署滿兵部尚書。己亥,上即位於太和殿,以明年為乾隆元年。庚子,定三年喪制,卻群臣以日易月之請。命大學士硃軾協同總理事務王大臣辦事。辛酉,召史貽直來京。壬寅,止進獻方物。禁內廷行走僧人招搖。頒乾隆元年時憲書。鑄乾隆通寶。遣官頒詔朝鮮。丙辰,賑甘肅蘭州、平涼等處旱災。丙午,命慶復往北路軍營,代回福彭。手敕額駙策凌勿離軍營。丁未,大行皇帝梓宮安奉雍和宮。戊申,上詣雍和宮行禮。自是日至乙卯以為常。己酉,賞莊親王允祿、果親王允禮雙俸,鄂爾泰、張廷玉世襲一等輕車都尉,硃軾世襲騎都尉。庚戌,召楊名時來京。辛亥,命海望署戶部尚書,傅鼐署刑部尚書。乙卯,上詣雍和宮行大祭禮。奉皇太后居永仁宮。是日,上移居養心殿。命廷臣輪班條奏,各舉所知。戊午,賞李紱侍郎銜,命管戶部三庫事。己未,上詣雍和宮梓宮前行月祭禮。自是迄奉移,每月如之。再免民欠丁賦,並諭官吏侵蝕者亦免之。逮傅爾丹下獄。庚申,開鄉會試恩科。免貴州被擾州縣之額賦,未擾者停征。辛酉,上詣田村孝敬皇后梓宮前致祭。以本年鄉試弊多,逮治考官顧祖鎮、戴瀚。大學士馬齊乞休,允之。癸亥,召署河東鹽政孫家淦來京,以侍郎用。

冬十月丙寅朔,饗太廟,遣裕親王廣保代行。命副將軍常德赴北路軍營。丁卯,申禁各省貢獻。以張廣泗為征苗經略,揚威將軍哈元生、副將軍董芳以下俱聽節制。庚午,命履郡王允祹管禮部,召原任尚書塗天相來京。辛未,以任蘭枝為禮部尚書。壬申,免江南等省漕糧蘆課及學租雜稅。命治曾靜、張熙罪。加左都御史福敏太子太保。以王大臣辦事遲延疏縱,申諭嚴明振作,毋與用寬之意相左。調徐本為刑部尚書,塗天相為工部尚書。丙子,以劉勷為直隸河道總督。丁丑,起彭維新為左都御史。命徐本軍機處行走。癸未,停諸王兼管部院事。甲申,授海望戶部尚書。己丑,命來保署工部尚書,兼管內務府。癸巳,傅爾丹、岳鍾琪、石云倬、馬蘭泰論斬。甲午,改訥親、海望、徐本為協辦總理事務,納延泰行走,如班第等例。豐盛額、莽鵠立罷。庚子,張照下獄鞫治。壬寅,湖北忠★等十五土司改土歸流,分置一府五縣,於恩施縣建府治,名曰施南府,分設縣治,名曰宣恩、來鳳、咸豐、利川。乙巳,申諭薦舉博學鴻詞。丁未,上大行皇帝尊謚曰敬天昌運建中表正文武英明寬仁信毅大孝至誠憲皇帝,廟號世宗,次日頒詔覃恩有差。免四川巴縣等旱災額賦。戊申,召邁柱來京,以史貽直署湖廣總督。庚戌,以孫嘉淦為左都御史。癸丑,命慶復為定邊大將軍,赴北路軍營。命孫嘉淦仍兼管吏部。諭赦降苗罪。免貴州三年內耗羨。丙辰,上詣田村上孝敬憲皇后尊謚曰孝敬恭和懿順昭惠佑天翊聖憲皇后,次日頒詔覃恩有差。改河東總督仍為河南巡撫,以傅德為之。丁巳,授鍾保湖南巡撫,俞兆嶽江西巡撫。命岱林布為右衛將軍。己未,以平郡王福彭協辦總理事務。董芳、元展成、德希壽褫職逮問,奪哈元生揚威將軍,命經略張廣泗兼貴州巡撫。癸亥,賞阿其那、塞思黑子孫紅帶,收入玉牒。甲子,以王大臣會刑部夾訊李禧、耿韜,命審訊大臣宜存大體。

十二月丙寅朔,以博第為吉林將軍,吳禮布為黑龍江將軍。復設川陜總督,裁四川總督。戊辰,賑安徽泗州、湖北潛江水災。癸酉,免浙江、山東、福建、廣東鹽場欠課。戊寅,上皇太后徽號曰崇慶皇太后,次日頒詔覃恩有差。己卯,以準噶爾遣使請和,命喀爾喀扎薩克等詳議定界事宜。庚辰,調傅鼐為刑部尚書,仍兼管兵部。甲申,磔曾靜、張熙於巿。都統李禧以贓,尚書高起以欺罔,俱論斬。丙戌,命嵇曾筠兼管浙江巡撫。以高斌為江南河道總督。設歸化城將軍及副都統。辛卯,晉封訥親一等公,世襲。

乾隆元年春正月丙申朔,上詣堂子行禮。至觀德殿更素服,詣雍和門行禮畢,率諸王大臣詣慈寧宮行禮。御太和殿受朝,不作樂,不宣表。戊戌,命北路參贊大臣薩木哈回京。辛丑,祈穀於上帝,親詣行禮。自是每年如之。癸卯,建京師先蠶壇。準噶爾臺吉噶爾丹策零遣使貢方物。丁未,準噶爾貢使吹納木喀入覲。召大將軍慶復回京。命伊勒慎、阿成阿、哈岱為參贊大臣,協同額駙策凌辦事,駐鄂爾坤。命都統王常、侍郎柏修往鄂爾坤勘屯田。丙辰,以顧琮署江蘇巡撫。己未,署湖南永州鎮總兵崔起潛妄劾鄂爾泰、張廣泗,褫職逮治。南掌入貢。庚辰,上啟蹕謁陵。癸亥,上謁昭西陵、孝陵、孝東陵、景陵。賑臺灣諸羅縣地震災民。賑甘肅固原、四川忠州等州縣旱災。

二月丙寅,上還京師。戊辰,祭大社、大稷,上親詣行禮。自是每年如之。以補熙署漕運總督。甲戌,遣準噶爾來使歸,詔以遵皇考諭旨,酌定疆界,齎示噶爾丹策零。乙卯,賜準噶爾臺吉噶爾丹策零敕書,斥所請以哲爾格西喇呼魯蘇為界,及專令喀爾喀內徙。庚辰,命邁柱兼管工部。申飭陳奏謬妄之謝濟世、李徽、陳世倌等。加楊名時禮部尚書銜,管國子監祭酒事。辛酉,朝鮮國王李昑遣使進香,賞賚如例。甲申,命改嵇曾筠為浙江總督,兼管兩浙鹽政。郝玉麟以閩浙總督專管福建事。戊子,定世宗山陵名曰泰陵。己丑,達賴喇嘛及貝勒頗羅鼐遣使貢方物。辛卯,以程元章為漕運總督。癸巳,尹繼善奏克空稗、臺雄等寨。張廣泗奏克大小丹江等處。

三月庚子,釋汪景琪、查嗣庭親族回籍。乙巳,加上太祖尊謚曰太祖承天廣運聖德神功肇紀立極仁孝睿武端毅欽安弘文定業高皇帝,孝慈皇后尊謚曰孝慈昭憲敬順仁徽懿德慶顯承天輔聖高皇后;太宗尊謚曰太宗應天興國弘德彰武寬溫仁聖睿孝敬敏昭定隆道顯功文皇帝,孝端皇后尊謚曰孝端正敬仁懿哲順慈僖莊敏輔天協聖文皇后,孝莊皇后尊謚曰孝莊仁宣誠憲恭懿至德純徽翊天啟聖文皇后;世祖尊謚曰世祖體天隆運定統建極英睿欽文顯武大德弘功至仁純孝章皇帝,孝惠皇后尊謚曰孝惠仁憲端懿慈淑恭安純德順天翼聖章皇后,孝康皇后尊謚曰孝康慈和莊懿恭惠溫穆端靖崇天育聖章皇后;聖祖尊謚曰聖祖合天弘運文武睿哲恭儉寬裕孝敬誠信中和功德大成仁皇帝,孝誠皇后尊謚曰孝誠恭肅正惠安和淑懿儷天襄聖仁皇后,孝昭皇后尊謚曰孝昭靜淑明惠正和安裕欽天順聖仁皇后,孝恭皇后尊謚曰孝恭宣惠溫肅定裕慈純贊天承聖仁皇后。丁未,免四川涼山等處番民額賦。己酉,免肅州威魯堡回民舊欠。庚戌,以固原提督樊廷為駐哈密總督。乙卯,免廣東歸善等四縣加增漁稅及通省逋賦。

夏四月丙寅,免江南阜寧等州縣緩征漕糧。壬申,命王常、海瀾為參贊大臣,協同額駙策凌辦事。以高其倬為湖北巡撫,暫署湖南巡撫。戊寅,以王士俊為四川巡撫。辛巳,貴州提督哈元生褫職逮問。裁直隸副總河,以總督兼管河務。戊子,賜金德瑛等三百三十四名進士及第出身有差。壬辰,布魯克巴部諾顏林沁齊壘喇布濟至西藏請上安,並貢方物。

五月丁未,賑河南永城縣水災。壬子,命江南副總河移駐徐州。甲寅,免四川南溪等州縣被風雹額賦。乙卯,朝鮮國王李昑表賀登極及尊崇皇太后,並進方物。乙巳,暹羅國王參立拍照廣拍馬噓六坤司尤提雅菩挨表謝賜扁,並貢方物。庚辰,免甘肅伏羌等州縣地震傷亡缺額丁銀。

六月戊辰,賑江蘇蕭縣等州縣水災。己巳,以慶復署吏部尚書,仍兼署戶部事。癸酉,授張廣泗貴州總督,兼管巡撫事。以尹繼善為雲南總督。

秋七月癸巳朔,以貴州流民多就食沅州,免沅州額賦。甲午,召總理事務王大臣九卿等,宣諭密書建儲諭旨,收藏於乾清宮正大光明扁額上。己亥,免貴州通省本年額賦。辛丑,除古州等處苗賦。甲辰,免崔起潛罪。丙午,賑江西安福水災。辛亥,追謚明建文皇帝為恭閔惠皇帝。賑江南蕭、碭等州縣衛水災。丁巳,賑甘肅隴西等州縣水雹災。戊午,調鍾保為湖北巡撫,高其倬為湖南巡撫。賑湖北漢川等五州縣衛水災。癸酉,逮問王士俊,尋論斬。賑廣東南海、潮陽等縣水災。

八月戊辰,祭大稷、大社,上親詣行禮。自是每歲如之。準噶爾部人孟克來降。庚午,尚書傅鼐有罪免。乙卯,賑河南南陽等五縣水災。乙酉,賑喀喇沁饑。丁亥,兵部尚書通智免,以奉天將軍那蘇圖代之。調博第為奉天將軍。以吉爾黨阿為寧古塔將軍。賑陜西神木、府穀雹災。辛卯,賑浙江蘭溪等六縣、江南溧水等二十四州縣、湖北潛江等九州縣衛水災。

九月丙申,免張照、哈元生、董芳、元展成、德希壽貽誤苗疆罪。丁酉,禮部尚書楊名時卒。戊戌,以慶復為刑部尚書,兼管吏部。命傅鼐暫署兵部尚書。庚子,停本年秋決。癸卯,賑浙江安吉等四縣水災。丙午,上臨大學士硃軾第視疾。免江西安福水災額賦。庚戌,大學士硃軾卒,上親臨賜奠。壬子,賑安徽宿州等二十州縣衛水災。致仕大學士陳元龍卒。乙卯,賑江蘇蕭縣等三州縣水災。己未,御試博學鴻詞一百七十六人於保和殿,授劉綸等官。賑江蘇無錫等十三州衛水災。準噶爾臺吉車林等來降。

冬十月壬戌,以邵基為江蘇巡撫。乙丑,除浙江仁和等州縣水災額賦。庚午,調岳濬為江西巡撫,以法敏為山東巡撫。辛未,上奉皇太後送世宗梓宮至泰陵。庚辰,上奉皇太后還京師。

十一月甲午,上始御乾清門聽政。加嵇曾筠太子太傅。命徐本為東閣大學士,仍兼管刑部。以孫家淦為刑部尚書,楊汝穀為左都御史。以額爾圖為黑龍江將軍。丙申,免雲南楚雄等四府州縣額賦。丁酉,賑安徽霍丘等三縣衛、湖北漢川等十三縣一水災。己酉,冬至,祀天於圜丘,上親詣行禮。自是每年如之。己未,賑陜西定邊雹災,江南長洲等十二州縣衛水災。

十二月辛酉,賑巴林郡王等四旗旱災。甲子,賑江蘇婁、溧水等十三州縣水災。乙丑,改江南壽春協為鎮,設總兵。己巳,免陜西府谷、神木本年雹災額賦。移南河副總河駐徐州。丁丑,免安徽泗州衛屯田、長蘆、廣云灶地水災額賦。丁亥,岱林布改江寧將軍。以王常為建威將軍,雅爾圖為參贊大臣。免兩淮莞瀆等三場水災額賦。

是歲,朝鮮、南掌、暹羅、安南來貢。

二年春正月庚寅朔,免朝賀。庚子,召趙弘恩來京。以慶復為兩江總督。調那蘇圖為刑部尚書。以訥親為兵部尚書。乙巳,以楊超曾為廣西巡撫。丙午,釋王士俊。戊子,李衛劾治誠親王府護衛囑託。上嘉之,賞四團龍褂。

二月丙寅,安南國王黎維祜卒,嗣子黎維禕遣使告哀,並貢方物。癸酉,賑江蘇高郵水災。戊寅,遣翰林院侍讀嵩壽、修撰陳倓冊封黎維禕為安南國王。庚辰,孝敬憲皇后發引,上奉皇太後送至泰陵。

三月庚寅,葬世宗於泰陵,孝敬憲皇后祔。壬辰,上還京師。癸巳,世宗憲皇帝、孝敬憲皇后升祔太廟,頒詔覃恩有差。辛丑,命保德等頒升祔詔於朝鮮。甲辰,塗天相罷。以趙弘恩為工部尚書。以顧琮協辦吏部尚書。戊申,命翰林、科道輪進經史奏議。庚戌,移右衛將軍駐歸化新城,增副都統二。辛亥,調碩色為四川巡撫。壬子,調楊永斌為湖北巡撫。

四月甲子,以旱命刑部清理庶獄。乙卯,訓飭建言諸臣。己巳,疏濬清口並江南運河。賑江蘇江寧、常州二府旱災。甲戌,祀天於圜丘,奉世宗配饗,次日頒詔覃恩有差。是日,雨。釋傅爾丹、陳泰、岳鍾琪。丙子,免順天直隸額賦。己卯,召尹繼善來京。以張允隨署云南總督。甲申,免湖北漢川等五州縣衛水災額賦。南掌入貢。丁亥,免江蘇蕭、碭二縣水災額賦。

五月壬辰,賜於敏中等三百二十四人進士及第出身有差。癸巳,免湖北荊州、安陸二府水災額賦。乙未,賑河南南陽等十二州縣水災。戊戌,御試翰林、詹事等官,擢陳大受等三員為一等,餘各升黜有差。準本年新進士條奏地方利弊。戊申,免山東正項錢糧一百萬兩。辛亥,祭地於方澤,奉世宗配饗。除廣東開建、恩平二縣米稅。乙卯,除湖南永州等處額外稅。免安徽宿州水災額賦。免浙江仁和等四州縣水災額賦。賑陜西商南、膚施等縣雹災。甲戌,以御門聽政,澍雨優渥,賜執事諸臣紗疋有差。辛酉,命直隸試行區田法。戊戌,賑安徽石埭等六州縣水災。

秋七月戊子,以永定河決,遣侍衛策楞等分赴盧溝橋、良鄉撫恤災民。癸卯,命侍衛松福等往文安、霸州等處撫恤災民。乙未,命顧琮勘永定河沖決各工。丙申,賑山東德平、陽穀等州縣旱雹各災。壬寅,賑順直宛平、清苑等八十一州縣衛旱災。御試續到博學鴻詞於體仁閣,授萬松齡等官。丙辰,命各省蠲免額賦,已輸者抵作次年正賦,著為令。賑安徽黟縣等十四州縣水災。

八月丁巳朔,賑陜西安塞等三縣雹災。湖南城步縣瑤匪平。賑撫甘肅平番等四縣旱災。命巡漕御史四員分駐淮安、濟寧、天津、通州。甲戌,命鄂爾泰詳勘直隸河道水利。丙子,以顧琮署直隸河道總督。丁丑,免江蘇碭山水災未完額賦十分之七。壬午,復設貴州威寧鎮總兵官。築浙江魚鱗大石海塘。免山東歷城等二十八州縣衛本年旱災額賦。甲申,賑甘肅會寧旱災,福建霞浦等州縣水災。

九月辛卯,調北路參贊大臣哈岱回京,以瑪尼代之。乙未,準噶爾回民米爾哈書爾來降。乙未,以楊永斌為江蘇巡撫。己亥,賑福建閩縣等沿海風災。甲辰,訓飭科道毋挾私言事。召史貽直入都。以德沛為湖廣總督,元展成為甘肅巡撫。賑山西興縣等十二州縣旱災。辛亥,賑甘肅寧夏縣水災。癸丑,免雲南寧州上年夏稅。乙卯,以那蘇圖署兵部尚書。

閏九月癸亥,免河南西華等四縣本年水災額賦。丁卯,以尹繼善為刑部尚書,兼辦兵部事。調慶復為雲南總督。以那蘇圖為兩江總督。甲戌,賑長蘆、蘆臺等場水災灶戶。除江西袁州、饒州二府雜稅。丙子,馬蘭峪陵工竣。辛巳,賑福建霞浦等二縣風災。壬午,賑奉天小清河驛水災。以雲南布政使陳宏謀瀆奏本省墾務,下部嚴議。賑江蘇上元等二十五州縣水災,並加賑有差。賑貴州安順等府縣雹災。

冬十月乙酉朔,賑山西永濟等三縣霜災。丁亥,修盛京三陵。戊子,上詣東陵。辛卯,上謁昭西陵、孝陵、孝東陵。乙未,上還京師。丙申,安西鎮總兵張嘉翰坐剝削軍需論斬。以崔紀為陜西巡撫,尹會一為河南巡撫,張楷為湖北巡撫。己亥,大學士尹泰乞休,溫諭留之。癸卯,賑山東齊河等二十八州衛水災。免江南淳縣本年蟲災額賦,桃源等三縣未完銀米。丁未,賑黑龍江水災。戊申,修奉先殿。辛亥,免甘肅平番旱災額賦。

十一月乙卯,賑安徽壽州、霍丘旱災。免陜西靖邊等八州縣本年水災額賦。丁巳,朝鮮國王李昑請封世子李愃,禮部言年未及歲,上特允之。癸亥,賑貴州郎岱等三縣雹災。乙丑,除山西河津被水額賦。丙寅,賑安徽太平等十一州縣衛水災。辛未,上詣泰陵,改總管為副都統。免江南銅山、碭山二縣逋賦。壬寅,祭告泰陵,上釋服。乙亥,賑甘肅環縣、蘭州,廣東三水等十縣旱災。上還京師。戊寅,皇太后聖壽節,御慈寧宮,上率諸王大臣行慶賀禮。自是每年如之。己卯,免山西興縣等四州縣旱災丁銀。庚辰,命仍設軍機處,以大學士鄂爾泰、張廷玉,尚書訥親、海望,侍郎納延泰、班第為軍機大臣。

十二月甲申朔,漕運總督補熙免,以查克丹代之。以來保為工部尚書。免江南阜寧上年水災額賦。丁亥,上御太和殿,冊立嫡妃富察氏為皇后。戊子,奉皇太后御慈寧宮,上率諸王大臣行慶賀禮畢,上御太和殿,群臣慶賀,頒詔覃恩有差。辛卯,免江蘇溧水等十二州縣水災額賦。壬辰,賑陜西府穀等三縣雹災。甲午,以冊立皇后禮成,加上皇太后徽號曰崇慶慈宣皇太后。奉皇太后御慈寧宮,上率諸王大臣行慶賀禮,次日頒詔覃恩有差。己亥,免直隸本年旱災灶課。免甘肅寧夏水災額賦。壬寅,鄂爾泰封三等伯。賑福建閩縣等六縣、廣東海康等七縣風潮災。大學士邁柱乞病,許之。琉球貢方物。癸卯,張廷玉封三等伯。辛亥,賑涿州水災。

三年春正月甲寅朔,上初舉元正朝賀,率王以下文武大臣詣壽康宮慶賀皇太后,禮成,御太和殿受賀。自是每年元正如之。乙卯,以福敏為武英殿大學士,馬爾泰為左都御史。辛酉,祈穀於上帝,奉世宗配享。癸亥,命舉行經筵。甲子,上初幸圓明園,奉皇太后居申昜春園。戊辰,御正大光明殿,賜朝正外籓及內大臣、大學士宴。癸酉,以硃藻為直隸河道總督,顧琮協理河道事。丁丑,準噶爾噶爾丹策零遣使奉表至京,並進貂皮。遣侍郎阿克敦充正使,御前侍衛旺扎爾、乾清門臺吉額默根充副使,齎敕往準噶爾議定界。己卯,上自圓明園還宮。辛巳,以謁泰陵,命鄂爾泰在京總理事務。

二月丁亥,釋奠先師孔子。戊子,幸圓明園。癸巳,準噶爾使入覲,賞銀幣有差。戊戌,上謁泰陵。己亥,上祭泰陵。辛丑,上幸南苑行圍。壬寅,上還京師。丙午,舉行經筵。自是每季仲月舉行一次,歲以為常。丁未,免山東齊河等三十二州縣衛水災額賦。辛亥,上親耕耤田,加一推。自是每年如之。壬子,趙弘恩以納賄奪職,以高其倬為工部尚書,張渠為湖南巡撫。

三月癸丑朔,賑福建閩縣等八縣颶風災。甲寅,上詣太學釋奠,御彞倫堂,命講中庸、尚書。乙卯,調崔紀為湖北巡撫,張楷為西安巡撫。己未,免江蘇六合等十二州縣水災額賦,廣東三水等十州縣旱災額賦。辛酉,賑江蘇上元等二十五州縣衛水災,並免額賦。丁卯,上詣黑龍潭祈雨。辛未,免甘肅蘭州等處旱災額賦。壬申,以旱命刑部清理庶獄。癸酉,免安徽太平等十一州縣衛水災額賦。丁丑,免湖北沔陽州逋賦。

夏四月甲申,以旱申命求言。停督撫貢獻。理籓院尚書僧格休致,以納延泰代之。己丑,調孫嘉淦為吏部尚書,以趙國麟為刑部尚書,孫國璽為安徽巡撫。壬辰,命顧琮往直隸會同硃藻辦理河工。免長蘆蘆臺等場、衡水等州縣水災額賦。

五月癸丑,賑陜西蒲城等十州縣雹災。己未,賑山東章丘等州縣衛雹災。庚申,賑陜西雒南等八州縣雹災。壬戌,貴州定番州苗阿沙等作亂,張廣泗討平之。辛未,調額爾圖為奉天將軍,博第為黑龍江將軍。乙亥,免江南松江府額賦。辛巳,賑陜西靖邊等八州縣旱災。

六月庚寅,賑山東東平等四州縣雹災。丙午,左都御史楊汝穀乞休,允之。

秋七月壬子,起前左都御史彭維新為原官。丁巳,免福建詔安縣旱災額賦。癸亥,免浙江溫州等衛漕欠。乙丑,調史貽直為工部尚書,高其倬為戶部尚書。丁卯,命查郎阿入閣辦事。調鄂彌達為川陜總督。以馬爾泰為兩廣總督,查克丹為左都御史,託時為漕運總督。大學士尹泰乞休,允之。

八月丙戌,江蘇海州、山東郯城等州縣蝗。賑湖南石門縣、甘肅武威等三縣水災。己丑,海望丁憂,以訥親暫署戶部尚書。己亥,奉皇太后謁泰陵。癸卯,上詣泰陵行三周年祭禮。丙午,上奉皇太后駐蹕南苑,上行圍。戊申,賑安徽望江等四十八州衛旱災。

九月庚戌朔,上奉皇太后還宮。免陜西長安等十五州縣雹災額賦。賑山東招遠縣雹災。戊午,免福建漳浦上年旱災額賦。辛酉,命嵇曾筠入閣辦事,兼理永定河務。裁浙江總督,復設巡撫,以郝玉麟仍為閩浙總督,盧焯為浙江巡撫。甲子,硃藻解任,遣訥親、孫嘉淦往鞫之。以顧琮管總河印務。安南入貢。己巳,大學士尹泰卒。編修彭樹葵進十思箴,上嘉賚之。賑甘肅碾伯等處旱災。丁丑,免江蘇江寧等五十二州縣衛水災額賦,並賑之。戊寅,賑臺灣旱災。

冬十月庚辰朔,賑陜西安定等六州縣雹災。辛巳,免山東鄒平等八州縣本年雹災額賦。壬午,免直隸被水州縣逋賦。免江蘇、安徽被災各州縣逋賦。辛卯,皇次子永璉薨,輟朝五日,以御極後,親書永璉為皇太子密旨,一切典禮如皇太子儀。賑安徽懷寧等五十州縣衛旱災。壬辰,戶部尚書高其倬卒。丙申,調任蘭枝為戶部尚書,趙國麟為禮部尚書,史貽直為刑部尚書,以趙殿最為工部尚書。丁酉,謚皇太子永璉為端慧皇太子。直隸總督李衛以病免,命孫嘉淦署之。己亥,賑浙江吉安等州縣旱災。庚子,朝鮮國王李昑表賀上皇太后徽號並冊封皇后,又表謝恩封世子,附進方物。壬寅,上幸田村,奠端慧皇太子。癸卯,免江南、江西、河南漕欠。乙巳,授孫嘉淦直隸總督,以甘汝來為吏部尚書兼兵部,楊超曾為兵部尚書。丙午,授顧琮直隸河道總督。

十一月己酉朔,復廣東海南道為雷瓊道,改高雷道為高廉道。庚戌,以孫嘉淦劾貝勒允祜,上嘉之,予議敘。允祜下宗人府嚴議。壬子,賑江蘇華亭等六縣衛旱災。賑湖南石門縣旱災。癸丑,免奉天寧遠等四州縣蟲災額賦。賑浙江歸安、烏程,陜西綏德等四州縣雹災,湖北孝感等六州縣旱災。癸丑,免河南信陽等八州縣旱災額賦。賑湖北應山、四川忠州等三州縣旱災。乙丑,免江南淮安、徐州二府湖灘額租。免山東招遠縣雹災額賦。庚午,大學士嵇曾筠以病乞休,允之。壬申,甘肅寧夏地震,水湧新渠,寶豐縣治沈沒,發蘭州庫銀二十萬兩,命兵部侍郎班第往賑之。乙亥,吏部尚書性桂乞休,允之。丁丑,免直隸宣化各府州逋賦。

十二月乙卯朔,調訥親為吏部尚書。庚辰,賑四川射洪等六縣水災。賑兩淮鹽場本年旱災。丙戌,彭維新褫職,以魏廷珍為左都御史。丁亥,甘肅寧夏地震。甲午,賑甘肅平番蟲災。命大理寺卿汪★往江南總辦河工。琉球國王尚敬遣使表賀登極,入貢。戊戌,準噶爾臺吉噶爾丹策零遣哈柳等從侍郎阿克敦等至京師,進表。乙巳,準噶爾使哈柳等入覲,諭曰:「所奏游牧不越阿爾臺,朕甚嘉之。托爾和、布延圖卡倫內移,不可行。」

四年春正月己酉,上禦乾清宮西暖閣,召王、大臣、翰林、科道及督、撫、學政在京者九十九人賜宴,賦柏梁體詩。丁卯,免甘肅寧夏等五縣地震被災額賦。壬申,大學士嵇曾筠卒。趙國麟為大學士,調任蘭枝為禮部尚書,以陳德華為戶部尚書。

二月己卯,調張渠為江蘇巡撫,以馮光裕為湖南巡撫。丙戌,免直隸滄州等四州縣、興國等四場水災灶地額賦。免貴州郎岱等四州縣雹災額賦。乙未,免甘肅靖遠風災額賦。丙申,準噶爾部人孟克特穆爾等來降。免陜西咸寧、鎮安水災,甘肅柳溝衛蟲災額賦。戊戌,免湖南永順、永綏新闢苗疆鹽課。免浙江上虞等縣逋賦。庚子,準噶爾臺吉噶爾丹策零請以阿爾泰山為界,許之。免湖北鍾祥等五縣衛旱災額賦。

三月丁未朔,己酉,召雅爾圖來京,以阿蘭泰為北路參贊大臣。免安徽宿州等四州縣逋賦。吏部奏行取屆期,上命尚書、都御史、侍郎保舉如陸隴其、彭鵬者。免湖北應山上年旱災額賦。甲子,設熱河兵備道,駐承德州。命訥親協辦大學士。戊辰,以旱災特免直隸、江蘇、安徽三省額賦。壬申,以魏廷珍為工部尚書。賑直隸文安等六縣水災。

夏四月丁卯,免安徽壽州上年旱災額賦。戊寅,免江蘇丹陽等七縣旱災額賦。辛巳,賜莊有恭等三百二十八人進士及第出身有差。壬午,免長蘆上年旱災逋賦。丙戌,以旱申命求言。命刑部清理庶獄,減徒以下罪。甲午,免四川忠州等三州縣旱災額賦。乙未,以陳世倌為左都御史。癸卯,西藏巴勒布部庫庫木、顏布、葉楞三汗入貢。

五月甲子,朝鮮國王李昑謝賜本國列傳,進方物。戊辰,改築浙江海寧石塘。辛未,致仕大學士馬齊卒。癸酉,加鄂爾泰、張廷玉、福敏太保,徐本、訥親太子太保,甘汝來、海望、鄂善、尹繼昌、徐元夢、孫嘉淦、慶復太子少保。

六月庚辰,調碩色為山東巡撫,方顯為四川巡撫。甲辰,免甘肅赤金所上年被災額賦。山東濟南等七府蝗。曹縣河決,仍賑被水六州縣災民。甘肅秦安等六州縣雹災。

秋七月戊申,額駙策凌奏率兵駐鄂爾海西拉烏蘇,並分兵駐鄂爾坤河、齊齊爾里克、額爾德尼招、塔密爾、烏里雅蘇臺附近,防範準噶爾。庚戌,以甘肅秦安等十五州縣雹災,命無論已未成災,悉免本年額賦。辛酉,賑河南祥符等四十七州縣水災。壬戌,賑山東海豐等縣場灶戶。甲子,賑江蘇睢寧等十三州縣衛水雹各災,湖北房縣旱災。丙寅,吏部尚書甘汝來卒。以郝玉麟為吏部尚書,宗室德沛為閩浙總督,以班第為湖廣總督。己巳,賑安徽宿州雹災。庚申,安南馬郎叛人矣長等來降。賑山東利津等二縣雹災。壬申,賑直隸開州等州縣、江蘇海州等州縣水災。江蘇淮安、安徽鳳陽等府州蝗。

八月丙子,御史張湄劾諸大臣阻塞言路。上斥為漸染方苞惡習,召見滿、漢奏事大臣諭之。辛巳,賑河南商丘等州縣水災。壬午,敘張廣泗經理苗疆功,授三等輕車都尉,黃廷桂等加銜、加級有差。戊子,賑山東歷城等六十六州縣衛所水災,停征新舊額賦。庚寅,江蘇金壇縣貢生蔣振生進手鈔十三經,賜國子監學正銜。

九月乙巳朔,署廣西提督譚行義以安南鄭氏專柄,清化鎮邵郡公及黎鷟起兵與鄭氏內閧,奏聞。丙午,免江蘇海州、贛榆二州被水漕糧。戊申,賑河南祥符等三十七州縣水災有差。丁巳,上奉皇太后謁陵。庚申,上謁昭西陵、孝陵、孝東陵、景陵。賑山東臨邑等縣水災。癸亥,賑甘肅張掖東樂堡水災。賑河南鄧州等四州縣水災,山西榆次等三縣旱災。命停征江蘇、安徽漕糧。上奉皇太后還宮。庚午,上以疾命和親王弘晝代行孟冬時饗禮。免甘肅秦安等十五州縣糧草三分之一,及靈州、碾伯等州縣本年水雹各災額賦。

冬十月丁丑,準噶爾回人伊斯拉木定來降。庚辰,以江蘇海州等四州縣水災,免逋賦。甲申,端慧皇太子周年,上幸田村奠酒。乙酉,賑山東歷城等六十六州縣水災,給葺屋銀。丁亥,免陜西興平等十六州縣雹災額賦。己丑,莊親王允祿、理親王弘晰等緣事,宗人府議削爵圈禁。上曰:「莊親王寬免。理親王弘晰、貝勒弘昌、貝子弘普俱削爵。弘升永遠圈禁。弘★王爵,系奉皇考特旨,從寬留王號,停俸。」丙申,釋馬蘭泰。己亥,額魯特札薩克多羅郡王、和碩額駙阿寶之妻和碩格格進顧實汗所傳玉璽,諭還之。壬寅,召定邊左副將軍額駙策凌來京。封弘勩郡王,襲理親王爵。癸卯,上幸南苑行圍。

十一月丙午,上行大閱禮,連發五矢皆中的,賜在事王大臣銀幣有差。戊申,以郝玉麟署兩江總督。庚戌,召尹會一來京,以雅爾圖為河南巡撫。賑江蘇安東等十五州縣水災有差。壬申,免寧夏次年額賦。

十二月癸酉朔,免山東金鄉等六州衛水災額賦。丙子,免浙江安吉等州縣漕糧,河南羅山旱災額賦。戊寅,弘晰坐問安泰「準噶爾能否到京,上壽算如何」,擬立絞。諭免死,永遠圈禁,安泰論絞。免陜西榆林等十一州縣逋賦。癸未,免河南祥符等四十四州縣水災額賦。乙酉,晉封貝勒頗羅鼐為郡王。庚寅,免河南商丘等十州縣水災額賦。壬辰,哈柳等入覲。甲午,召車臣汗達瑪林等賜茶。

五年春正月丁未,賑安徽宿州等八州縣,廬江等十州縣衛旱災有差。丁卯,朝鮮入貢。辛未,命烏赫圖、巴靈阿護準噶爾人赴藏熬茶。湖南綏寧苗作亂,命馮光裕等剿之。

二月,琉球入貢。乙亥,命額駙策凌等定各部落接準噶爾游牧邊界。哈柳歸,召入賜茶,以和議成,嘉獎之。辛巳,以伊勒慎為綏遠城將軍。癸未,工部尚書魏廷珍罷。申諭九卿,毋蹈模棱覆轍。免山東章丘等六十州縣衛水災額賦。戊子,免湖北襄陽縣衛上年額賦。壬辰,免上年安徽宿州雹災、山東滕縣等五縣水災額賦。戊戌,以韓光基為工部尚書。辛丑,免湖北漢陽等四縣上年旱災額賦。

三月庚戌,以尹繼善為川陜總督,鄂善署刑部尚書。壬子,免直隸雄縣上年水災額賦。甲子,免山東霑化等縣場水災額賦。庚午,湖南慄林、鬼沖各寨苗匪平。

夏四月丙戌,賑兩淮板浦等場災。戊子,御史褚泰坐受賄論斬。免陜西葭州、懷遠旱災額賦。己丑,以那蘇圖為刑部尚書。甲午,以旱召九卿面諭,直陳政事闕失。改山東河道為運河道,兗沂曹道為分巡兗、沂、曹三府,管河工。戊戌,任蘭枝及太常寺卿陶正靖坐朋比,下部嚴議。

五月甲寅,上詣黑龍潭祈雨。丙辰,命刑部清理庶獄。甲子,以楊超曾署兩江總督。丁卯,諭馮光裕及湖廣提督杜愷剿捕城步、綏寧瑤匪。

六月癸酉,命阿里袞、硃必階查勘山東沂州等處水旱災。戊寅,命山東、江蘇、安徽捕除蝻子。召張廣泗來京。壬辰,賑甘肅秦州水災。戊戌,福州將軍隆升坐收餽遺,褫職鞫治。

閏六月甲辰,廣西義寧苗作亂,諭馬爾泰赴桂林調度兵事。辛亥,以喀爾吉善為山西巡撫。命杜愷統率湖南兵至軍前。乙卯,命張廣泗赴湖南會辦軍務。甲子,準噶爾臺吉噶爾丹策零遣使進表。

秋七月癸酉,調張渠為湖北巡撫。以徐士林為江蘇巡撫。調方顯為廣西巡撫,碩色為四川巡撫,硃定元為山東巡撫。乙亥,賜噶爾丹策零敕書,諭準噶爾使以阿爾泰山為界,山南游牧之人,仍居舊地。設甘肅安西提督,駐哈密。丁丑,以補熙為綏遠城將軍。辛巳,詔停今年秋決。甲申,張廣泗留辦湖南善後。賑安徽宣城衛饑。己丑,免安徽鳳陽等十九州縣衛水災、無為等四州縣旱災額賦。甲午,賑山西徐溝饑。丁酉,賑甘肅武威等三縣饑。戊戌,班第奏總兵劉策名等連克長坪各苗寨,獲首倡妖言黎阿蘭等。

八月己亥朔,廣西宜山縣蠻匪平。庚子,諭曰:「朕閱江省歲額錢糧雜辦款目,沿自前明,賦役全書亦未編定,官民交受其累,其悉予豁免。」庚戌,班第奏剿平鹽井口苗匪各寨。壬戌,上奉皇太后駐南苑。賑福建永定饑。免河南中牟等十四州縣水災額賦。戊辰,譚行義奏安南人立龍彪為王,僭元景興。癸酉,調楊超曾為吏部尚書,仍署兩江總督,史貽直為兵部尚書,韓光基為刑部尚書,陳世倌為工部尚書。辛巳,協辦大學士禮部尚書三泰乞休,慰留之。賑福建上杭饑。賑浙江餘杭等十六州縣衛所水災。丙戌,江蘇宿遷縣硃家閘河決,命築挑水霸。丁亥,築江蘇寶山縣吳家濱海塘石壩。賑陜西葭州等州縣饑。以王安國為左都御史。永定河復歸故道。

冬十月戊戌朔,以常安為漕運總督。壬寅,上謁泰陵。乙巳,上還京師。賑四川綿竹等三縣水災。甲寅,免甘肅平羅本年水災額賦,仍免寧夏、寧朔半賦。丙辰,僉都御史劉藻奏請停減圓明園營造,上嘉納之。賑福建臺灣、諸羅風災。丁卯,張廣泗奏獲苗匪慄賢宇等,及附瑤匪之戴名揚等,克平溪等寨。

十一月己巳,以那蘇圖署湖廣總督。庚午,調來保為刑部尚書,哈達哈為工部尚書。丙子,楊超曾劾江西巡撫岳濬,命高斌往會鞫之。己卯,召王★來京。命王安國以左都御史管廣東巡撫事。命阿里袞同高斌勘鞫岳濬。以劉吳龍為左都御史。乙酉,命廷臣各舉所知,如湯斌、陸隴其、陳瑸、彭鵬諸人。賑陜西葭州等六州縣饑。

十二月壬寅,張廣泗進剿湖南城步、綏寧,廣西義寧苗、瑤,悉平之。免安徽宣城、宣州二縣衛雹災額賦。免托克托城等處雹災額賦。壬子,免山東蒲臺逋賦。

六年春正月甲戌,裁安西總兵,設提督。丙子,免福建閩縣等五縣逋賦。甲申,命鄂爾泰、訥親會同孫嘉淦、顧琮勘視永定河工。命參贊大臣阿岱駐烏里雅蘇臺。以慶泰為北路軍營參贊大臣。戊子,免霸州、雄縣額賦。甲午,命班第仍在軍機處行走。

二月,御史叢洞請暫息行圍,上以飭兵懷遠之意訓之。丙午,以完顏偉為南河副總河。免湖北鍾祥等四縣雹水災額賦。甲寅,免陜西葭州等三州縣雹災額賦。庚申,增設山西歸化城分巡道。

三月壬申,命侍郎楊嗣璟往山西會鞫山西學政喀爾欽賄賣生員之獄。甲申,以御史仲永檀劾鄂善受賄,命怡親王等鞫之。鄂善褫職逮問。辛卯,擢仲永檀為僉都御史。

夏四月乙未朔,大學士趙國麟乞休,不允。免江蘇豐縣等十州縣衛水災、蟲災、民屯蘆課。甲辰,免順天直隸霸州等十州縣上年水災額賦。以慶復署兩廣總督,張允隨署云貴總督。己酉,賜鄂善自盡。

五月戊寅,免福建臺灣逋賦。賑江西興國等縣水災,貴州仁懷、平越水災。

六月甲午朔,免陜西葭州等六州縣上年水災額賦。丙申,江蘇巡撫徐士林給假省親,調陳大受署之。改張楷為安徽巡撫。庚子,命王安國勘廣東徵糧積弊。乙巳,以御史李刬劾甘肅匿災,命會同尹繼善勘之。己酉,浙江巡撫盧焯解任,命德沛及副都統汪扎勒鞫之。賑安徽宿州等十二州縣水災,江蘇山陽等州縣水災。趙國麟以薦舉非人,降調。

秋七月,免江蘇蘇州等府屬逋賦。甲子,喀爾欽處斬。丙子,薩哈諒論斬。戊寅,甘肅巡撫元展成以御史胡定劾,解任,命副都統新柱往會尹繼善鞫之。癸未,詔停今年秋決。戊子,上初舉秋獮。奉皇太后幸避暑山莊,免經過額賦十分之三。自是每年皆如之,減行圍所過州縣額賦。辛卯,賑江西武寧等二縣水災。壬辰,上至古北口閱兵。賑廣東永安、歸善二縣饑。

八月癸巳,賑安徽宿州等十九州縣衛水災。庚子,上駐蹕張三營。辛丑,上行圍。賑江蘇山陽等十八州縣、莞瀆等場水災。己酉,召楊超曾回京。調那蘇圖為兩江總督,孫嘉淦為湖廣總督。以高斌為直隸總督,完顏偉為江南河道總督。裁直隸河道總督,命高斌兼理直隸河務。辛亥,召寧古塔將軍吉黨阿來京,以鄂爾達代之。

九月癸亥朔,以陳宏謀為甘肅巡撫。乙丑,上奉皇太后回駐避暑山莊。賑廣東南海等二十六州縣饑。上奉皇太后回蹕。壬申,授王恕福建巡撫,楊錫紱廣西巡撫。甲戌,調陳宏謀為江西巡撫,黃廷桂為甘肅巡撫。免江蘇、安徽乾隆三四年被災漕糧。己卯,調韓光基為工部尚書。以劉吳龍為刑部尚書。辛巳,原任江蘇巡撫徐士林卒。授陳大受江蘇巡撫,張楷安徽巡撫。賑福建福清等八縣及長福等鎮營饑。丁亥,以劉統勛為左都御史。

冬十月庚子,賑廣東瓊山等二十四州縣颶災。丁未,賑安徽宿州等三十一州縣衛水災,並免宿州等三州縣額賦漕糧。己酉,賑甘肅靈州等處饑。丙辰,賑熱河四旗丁水災。

十一月甲子,賑兩淮灶戶饑。乙丑,南掌國王島孫遣使入貢。丙寅,賑甘肅平番等十四州縣雹水災。己巳,御史李刬陳奏甘肅饑饉情形不實,部議革職。上曰:「與其懲言官而開諱災之端,寧從寬假以廣耳目。」命革職留任。戊寅,免江蘇山陽等十五州縣衛水災額賦。賑句容等三十四州縣衛饑。丙戌,皇太后五旬聖壽節,御慈寧宮,上率諸王大臣等行慶賀禮。

十二月乙未,劉統勛請停張廷玉近屬升轉,減訥親所管事務,上嘉之。丙申,大學士張廷玉請解部務,不許。辛丑,免甘肅武威等二縣五年被水額賦。賑江蘇江浦等州縣旱災。免湖南湘鄉等二縣被水額賦。乙巳,免浙江仁和等十九州縣本年額賦。丁未,免山東歷城等十六州縣衛旱災額賦。庚戌,免甘肅永昌等三縣旱災額賦。琉球入貢。調常安為浙江巡撫,顧琮為漕運總督。命劉統勛往浙江會勘海塘。賑浙江嵊縣等十七州縣、仁和等場水旱災。

七年春正月壬戌,調史貽直為吏部尚書,任蘭枝為兵部尚書。以趙國麟為禮部尚書。庚午,定綏遠城、右衛、歸化城土默特、察哈爾共挑兵四千名,內札薩克首隊兵四千五百名、二隊兵六千五百名,援應北路軍營,並於額爾德尼昭沿途置駝馬備用。戊寅,以那克素三十九部番民備辦準噶爾進藏官兵駝馬,免本年額賦。甲申,賑安徽鳳陽、潁州二府,泗州一州屬饑民。庚寅,準噶爾入貢。

二月辛卯朔,上詣泰陵。乙未,上謁泰陵。是日,回蹕。丙申,朝鮮入貢。戊戌,上幸南苑行圍。己亥,琉球入貢。己酉,禮部尚書趙國麟乞休,不允。乙卯,以吉黨阿為歸化城都統。

三月庚申朔,上憂旱,申命求言,並飭九卿大臣體國盡職。丁卯,命大學士、九卿、督、撫舉如馬周、陽城者為言官。乙亥,以旱命刑部清理庶獄,各省如之。以晏斯盛為山東巡撫。辛巳,準噶爾臺吉噶爾丹策零遣使吹納木喀等奉表貢方物,乞勿限年貿易。壬午,以噶爾丹策零表奏狡詐,諭西北兩路軍營大臣加意防之。戊子,上詣黑龍潭祈雨。以兩江總督那蘇圖辦賑遺漏,切責之。

夏四月庚寅朔,準噶爾貢使吹納木喀等入覲。裁八溝、獨石口副都統各一,增天津副都統一。以古北口提督管獨石口外臺站。免河南永城等三縣上年被水額賦。甲午,賜金甡等三百二十三人進士及第出身有差。調德沛為兩江總督,那蘇圖為閩浙總督。乙未,撥安徽賑銀三十萬兩有奇,並準採買湖廣米備糶。辛丑,賑安徽宿州等州縣衛水災。甲辰,賜準噶爾臺吉噶爾丹策零敕書,申誡以追論舊事,屢違定約,並諭將此次奏請貿易、改道噶斯等事停止,仍賞賚如例。甲寅,除河南洧川等十一縣水沖地賦。免福建福清等七縣颶災額賦。丙辰,刑部尚書劉吳龍卒,以張照為刑部尚書。

五月己未朔,以順天、保定等八府,易州等五州缺雨,命停征新舊錢糧。定移駐滿兵屯墾拉林、阿勒楚喀事宜,設副都統,以巴靈阿為之。戊辰,以御史胡定劾,寢趙弘恩補刑部侍郎之命。癸酉,定雩祭典禮,禦制樂章。免江蘇沛縣昭陽湖水沈田畝額賦。丙戌,禁奏章稱蒙古為「夷人」。以琉球國王資送江南遭風難民,嘉獎之。張允隨奏猛遮界外孟艮酋長召賀罕被逐,遁入緬甸。

六月甲寅,諭督撫董率州縣經畫地利。戊申,訓飭地方官實心經理平糶。

秋七月己未,命資送日本遭風難民歸國。免廣西梧州等三府屬逋賦。辛酉,除山西繁峙、廣西武緣荒地額賦。乙丑,禮部尚書趙國麟乞休,上責其矯飾,褫職。調任蘭枝為禮部尚書,陳德華為兵部尚書,徐本兼管戶部尚書。丙寅,命大學士鄂爾泰兼領侍衛內大臣。命賑江蘇山陽等州縣水災。命撫恤江蘇阜寧等州縣水災。癸未,命高斌、周學健往江南查辦災賑、水利。甲申,賑湖北漢川、襄陽等州縣衛水雹災,並停徵額賦。丙戌,賑江蘇江浦等十八州縣衛、安徽臨淮等州縣衛。撫恤江西興國等州縣、浙江淳安等州縣、湖南醴陵等八州縣、山東嶧縣等十州縣衛、甘肅狄道等四州縣災民。

八月戊子,江南黃、淮交漲,命疆吏拯救災黎,毋拘常例。訓飭慎重軍政。撥江蘇、安徽賑銀二百五十萬兩有奇。庚寅,免江蘇、安徽被水地方本年額賦。辛卯,定皇后親蠶典禮。戊戌,免直隸、江蘇、安徽、福建、甘肅、廣東等省雍正十三年逋賦,並免江南、浙江未完雍正十三年漕項。庚子,諭河南等省撫恤江南流民。壬寅,上奉皇太后幸南苑,上行圍。癸卯,賑江西興國水災。乙巳,上奉皇太后幸晾鷹臺閱圍。

九月丁巳朔,撥江蘇運山東截留漕米十萬石,備淮、徐、鳳、潁各屬賑糶。賑湖北潛江等十州縣水災。辛酉,免廣東崖州等二州縣風災額賦。免安徽鳳、潁、泗三府州本年水災地方漕賦,不成災者折徵之。賑湖南湘陰等九縣水災。丁卯,上詣東陵。庚午,上謁昭西陵、孝陵、孝東陵、景陵。免江蘇山陽等二十一州縣本年被水漕賦。壬申,上幸盤山。賑恤江蘇、安徽災銀二百九十萬兩、米穀二百二十萬石各有奇。命再撥鄰省銀一百萬兩備明春接濟。乙亥,上幸棽髻山。戊寅,上回蹕。

冬十月丙戌,撥山東、河南明年運漕米各五萬石備江南賑,仍由直隸赴古北口外如數採買補運。己丑,免山東歷城等十九州縣旱災額賦。庚寅,命江南截留癸亥年漕糧二十萬石,仍撥山東漕糧二十萬石,河南倉米二十萬石,運江南備賑。癸巳,浙江提督裴鉽等以侵欺褫職鞫治。壬辰,賑江蘇山陽等二十八州縣衛饑。甲午,命清理滯獄。乙未,命撥山東沿河倉穀十萬石運江南備賑。丁酉,賑安徽鳳陽二十四州縣衛水災。甲辰,朝鮮國王李昑表謝國人金時宗等越境犯法,屢荷寬典。上曰:「此朕柔遠之恩。若恃有寬典,犯法滋多,非朕保全外籓之本意。王其嚴加約束,毋俾干紀。」以塞楞額為陜西巡撫。己酉,賑河南永城等十三州縣饑。辛亥,上詣順懿密太妃宮問疾。壬子,賑江蘇山陽等七州縣衛水災。

十一月丙辰朔,大學士等奏纂輯明史體例。上曰:「諸卿所見與朕意同。繼春秋之翼道,昭來茲之鑒觀,我君臣其共勉之。」賑湖北漢川等十二州縣水災饑。戊午,賑浙江瑞安等縣場、湖南湘陰等九縣水災。庚申,福建漳浦縣會匪戕殺知縣,命嚴治之。壬戌,賑山東膠州十州縣衛水災。癸亥,賑甘肅狄道等州縣水雹災。乙亥,命持法寬嚴,務歸平允。命陳世倌會同高斌查勘江南水利。戊寅,諭明春奉皇太后詣盛京謁陵。庚辰,以初定齋宮禮,是日詣齋宮。

十二月丙戌朔,賑山東濟寧等七州縣衛饑。丁亥,命考試薦舉科道人才。周學健舉三人皆同鄉,諭飭之。命左副都御史仲永檀會同周學健查賑。壬辰,上奉皇太后幸瀛臺。丙子,仲永檀、鄂容安以漏洩機密,逮交內務府慎刑司,命莊親王等鞫治。免福建尤溪等四縣荒田溢額銀。己亥,召安徽巡撫張楷來京,調喀爾吉善代之。命寬鄂爾泰黨庇仲永檀罪。免直隸薊州等三州縣水災額賦。丁未,撥運吉林烏拉倉糧接濟齊齊哈爾等處旱災。庚戌,賑奉天承德等五州縣饑。免山東膠州等十州縣衛水災額賦。辛亥,調完顏偉為河東河道總督,白鍾山為江南河道總督。乙卯,諭曰:「江南水災地畝涸出,耕種刻不容緩。疆吏其勸災民愛護田牛,或給貲飼養,毋得以細事置之。」

八年春正月丁巳,免鄂容安發軍臺,命仍在上書房行走。仲永檀死於獄。召孫嘉淦來京。以阿爾賽為湖廣總督。甲子,陳世倌等奏修江蘇淮、徐、揚、海,安徽鳳、潁、泗各屬河道水利,下大學士鄂爾泰等大臣議行之。己卯,命軍機大臣徐本、班第、那彥泰隨往盛京。辛巳,召參贊大臣阿岱、塔爾瑪回京,以拉布敦、烏爾登代之。壬辰,內閣學士李紱致仕陛辭,以慎終如始對,賜詩嘉之。辛卯,以考選御史,杭世駿策言內滿外漢,忤旨褫職。調劉於義為山西巡撫。命孫嘉淦署福建巡撫。丙申,命尹繼善署兩江總督,協同白鍾山料理河務。癸卯,命侍講鄧時敏、給事中倪國璉為鳳、潁、泗宣諭化導使,編修塗逢震、御史徐以升為淮、徐、揚、海宣諭化導使。乙巳,免湖北漢川等十一州縣衛水災額賦。準趙國麟回籍。癸丑,遣和親王弘晝代祀先農壇、用中和韶樂,與上親祭同,著為例。賑山東滕縣等六州縣饑。庚午,調喀爾吉善為山東巡撫,晏斯盛為湖北巡撫,範璨為安徽巡撫。丙子,上詣壽祺皇太妃宮問疾。

夏四月甲申朔,壽祺皇太妃薨,輟朝十日。上欲持服,莊親王等祈免。訓飭九卿勤事。申命各督撫陳奏屬員賢否。乙酉,上詣壽祺皇貴太妃宮致奠。辛卯,命奉宸苑試行區田法。丁酉,賑安徽鳳陽六府州屬水災饑。免湖北襄陽等三縣水災額賦。庚子,裁江蘇海防道,設淮徐海道,駐徐州府。以蘇松巡道兼管塘工。揚州府隸常鎮道。原設淮徐、淮揚二道專管河工。

閏四月甲寅朔,琉球入貢。丁巳,御試翰林、詹事等官,擢王會汾等三員為一等,餘各升黜有差。辛酉,免河南鄭州等十三州縣本年水災額賦。甲戌,除江蘇吳江等二縣坍沒田蕩額賦。

五月癸未朔,諭鑾輿巡幸,令扈從護軍等加意約束,不得踐踏田禾。乙酉,御史沈懋華以進呈經史講義召見,已去,下部嚴議。丁亥,命河南停徵上年被水地方錢糧。己亥,免江蘇山陽等十三州縣牙稅。免臨清商民運徵米船料及銅補商補。辛丑,賑山東歷城等十八州縣衛饑。丙午,以碩色為河南巡撫,紀山為四川巡撫。戊申,調慶復為川陜總督。以馬爾泰為兩廣總督。授張允隨為雲南總督,兼管巡撫事。辛酉,蘇祿國王麻喊末阿稟漻寧表請三年一修職貢。命仍遵五年舊例。

六月壬子朔,御史陳仁請以經史考試翰詹,不宜用詩賦,上嘉之。甲寅,改南掌為十年一貢。乙卯,除江蘇沛縣水沈地賦。丙辰,以旱求言。戊午,命阿里袞暫署河南巡撫。丁卯,以御史胡定劾湖南巡撫許容一案,究出督撫誣陷扶同,予敘。壬申,諭督撫率屬重農。

秋七月乙酉,上詣順懿密太妃宮問疾。丙戌,以安南不靖,擾及雲南開化都霙廠,命張允隨等嚴防之。開化鎮總兵賽都請討安南,不許。戊子,上奉皇太后由熱河詣盛京謁陵,免經過之直隸、奉天地方錢糧。撥通倉米四十萬石賑直隸旱災。壬辰,免山東歷城等十六州縣衛旱災額賦。乙未,停今年勾決。上奉皇太后駐避暑山莊。丙申,除福建連江等二縣水沖地賦。己亥,上奉皇太后詣盛京。癸卯,上行圍於永安莽喀。乙巳,上行圍於愛里。丙午,上行圍於錫拉諾海。命嚴除州縣徵漕坐倉之弊。戊申,免直隸滄州被雹灶戶額賦。上奉皇太后駐蹕嗎嗎塔喇。己酉,上行圍,至己卯皆如之。嚴督撫等漏洩密奏之禁。賑湖北興國等三州縣水災,並免額賦。癸亥,萬壽節,上詣皇太后行幄行禮。御行幄,扈從諸王以下大臣官員暨蒙古王以下各官慶賀。賜諸王、大臣、蒙古王等宴。甲子,上駐蹕巴雅爾圖塔剌。乙丑,上行圍。戊辰,上行圍。壬申,上駐蹕伊克淖爾,上行圍,至丙子如之。甲戌,賑四川西昌水災。定直隸被旱州縣賑恤事宜。賑廣東始興等十六州縣水災。己卯,上行圍於巴彥,親射殪虎。

九月庚辰朔,上行圍於伍什杭阿,親射殪虎。辛巳,上行圍威準。壬午,上行圍黃科。癸未,上行圍阿蘭。以哲布尊丹巴呼圖克圖未奏往額爾德尼招禮拜,土謝圖汗敦丹多爾濟均下理籓院議處。甲申,賑陜西商州水災饑。乙酉,上行圍舍裏。丙戌,上行圍善顏倭赫。丁亥,上行圍巴彥。鄂彌達改荊州將軍。調博第為吉林將軍,富森為黑龍江將軍。戊子,上行圍尼雅滿珠。己丑,上行圍珠敦。庚寅,上行圍英額邊門外。是日,駐蹕烏蘇河。甲午,許容以劾謝濟世貪縱各款皆虛,孫嘉淦以扶同定案,均褫職。署糧道倉德以通揭鞫實,予敘。上駐蹕穆奇村。乙未,上奉皇太后謁永陵。丙申,行大饗禮。命停顧琮議限民田。賑河南祥符等二十一州縣、山東齊東等十八州縣衛旱災,並免額賦有差。辛丑,謁福陵。壬寅,行大饗禮。謁昭陵。癸卯,行大饗禮。上奉皇太后駐蹕盛京。朝鮮國王李昑遣陪臣至盛京貢方物。甲辰,上率群臣詣皇太后宮行慶賀禮。御崇政殿受賀。賜群臣及朝鮮使臣宴。御大政殿賜酺。頒詔覃恩有差。乙巳,上詣文廟釋奠。幸講武臺大閱。諭王公宗室大臣等潔蠲禮典,訓導兵民,毋忘淳樸舊俗。丙午,上親奠克勤郡王岳託及武勛王揚古利墓。遣官望祭長白山、北鎮醫巫閭山及遼太祖陵。戊申,上親奠弘毅公額宜都、直義公費英東墓。免河南帶征乾隆七年以前民欠。

冬十月庚戌朔,上御大政殿,賜扈從王大臣宴於鳳凰樓前。諭王公宗室等革除陋習,恪守舊章。免盛京、興京等十五處旗地本年額賦及乾隆七年逋賦。禦制盛京賦。辛亥,上奉皇太后回蹕。乙丑,賑廣東南海等七縣水災。是日,上登望海樓,駐文殊菴。丁卯,命直隸被災各屬減價平糶。己巳,命部院大臣京察各舉賢自代。以劉於義為戶部尚書,阿里袞為山西巡撫。命徐本仍兼管戶部。調陳宏謀為陜西巡撫,塞楞額為江西巡撫。庚午,賑河南祥符等十四州縣旱災。甲戌,上奉皇太后還京師。丁丑,上以謁陵禮成,率群臣詣皇太后宮行慶賀禮。御太和殿,王大臣各官進表朝賀。

十一月,賑安徽無為水災,並免額賦。壬午,賑甘肅狄道等二十四州縣水蟲風雹災。庚寅,安南國王黎維禕表謝賜祭及襲封恩,進貢方物。辛丑,賑廣東萬州等十四州縣水災,福建臺灣等三縣旱災。壬寅,貸黑龍江被旱被霜兵丁等倉糧。賑山西曲沃等十一州旱災。癸卯,賑直隸天津等二縣旱災。丁未,賑安徽壽州等九州縣衛旱災。己酉,免謁陵經過額賦十分之三。

十二月庚戌朔,賑廣東吳川縣旱災。辛亥,命史貽直協辦大學士。乙卯,賑山東陵縣等十二州縣衛旱災。葬端慧皇太子於硃華山寢園。辛酉,大學士福敏乞退。溫諭慰留。甲子,準噶爾遣貢使圖爾都等至京,謝進藏人由噶斯路行走,賜助牲畜恩,並貢方物。乙丑,以陳德華隱匿其弟陜西按察使陳德正申辨參案密奏,下部嚴議。德正褫職鞫治。丁卯,以星變示儆,詔修省。

九年春正月辛巳,以徐本病,命史貽直為大學士。以劉於義為吏部尚書、協辦大學士,張楷為戶部尚書。陳德華罷,以王安國為兵部尚書。壬午,幸瀛臺。御大幄次,賜準噶爾使圖爾都宴,命立首班大臣末。以噶爾丹策零恭順,圖爾都誠敬可嘉,召圖爾都近前,賜飲三爵,錫賚有加。訓飭各省州縣教養兼施。丁亥,賑直隸天津等十一州縣災。庚子,王安國憂免,以彭維新為兵部尚書。以許容署湖北巡撫。授史貽直文淵閣大學士。朝鮮入貢。給訥親欽差大臣關防。癸卯,上奉皇太后詣泰陵。丙午,上詣泰陵。是日,奉皇太后回蹕。

二月,上奉皇太后幸南苑。丙辰,以給事中陳大玠等奏,寢許容署湖北巡撫之命,留晏斯盛任,仍申誡言官扶同糾論。免安徽桐城等九州縣上年水災額賦。免福建臺灣等三縣旱災額賦,並賑之。甲子,陳德華降調。丁卯,賑雲南霑、益二州縣水災。丁丑,戶部尚書張楷卒,以阿爾賽代之,鄂彌達為湖廣總督。

三月癸未,以汪由敦為工部尚書。丁亥,免江蘇沛縣、河南中牟等六縣旱災額賦。丁酉,調博第為西安將軍。以巴靈阿為寧古塔將軍。乙巳,賑山東德州等五州縣衛旱災。以訥親奏查閱河南、江南營伍廢弛,上曰:「可見外省大吏無一不欺朕者,不可不懲一儆百。」

四月戊申朔,始建先蠶壇成。乙卯,上詣圜丘行大雩禮,特詔貶損儀節,以示虔禱。以旱命省刑寬禁。辛未,賑山東德平等八州縣旱災。己卯,諭曰:「一春以來,雨澤稀少。皇太后以天時久旱,憂形於色,今日從寢宮步行至園內龍神廟虔禱。朕惶恐戰慄,即刻前往請安,諄懇謝罪,特諭內外臣工知之。」戊子,祭地於方澤,不乘輦,不設鹵簿。庚寅,雨。壬寅,大學士、九卿議覆御史柴潮生請修直隸水利,命協辦大學士劉於義往保定會同高斌籌畫。

六月己酉,大學士徐本以病乞休,允之。癸丑,賑山東歷城等三十二州縣旱災,蘭山等六州縣雹災。

秋七月丙子朔,諭直隸災重之天津等十六州縣,本年停征新舊錢糧。丙戌,免江蘇、安徽雍正十三年逋賦。壬辰,額爾圖以不職免,以達勒黨阿為奉天將軍。

八月己酉,撫恤安徽歙縣等二十州縣水災。戊申,免江蘇淮安、安徽鳳陽二府雍正十三年逋賦。癸丑,賑四川成都等州縣水災。乙丑,予告大學士徐本回籍,上賜詩寵行,賞賚有加,並諭行幸南苑之日,親臨慰問。丙寅,免直隸天津等三十一州縣上年逋賦。己巳,上奉皇太后幸南苑,上行圍。

九月己亥朔,以翰林院編修黃體明進呈講章,牽及搜檢太嚴,隱含諷刺,下部嚴議褫職。乙未,免山西清水河本年雹災額賦。癸卯,賑山東博興等縣旱災。丁未,改明年會試於三月舉行。己酉,以陳世倌假滿,命入閣辦事。賑山西文水等縣水災。庚戌,以四川學政蔣蔚實心教士,命留任。乙卯,上奉皇太后幸湯山。江南、河南、山東蝗。癸亥,上幸盤山。丁卯,上奉皇太后還宮。庚午,重修翰林院工竣。上幸翰林院賜宴,分韻賦詩,復御制柏梁體詩首句,群臣以次賡續。賜掌院大學士鄂爾泰、張廷玉御書扁額,及翰林、詹事諸臣書幣有差。是日,幸貢院,賜御書聯額。復幸紫微殿、觀象臺。賑直隸保定等十八州縣水蟲雹等災。賑江蘇靖江等十二州縣衛潮災,安徽歙縣二十一州縣衛水災。庚辰,起孫嘉淦為宗人府府丞。辛巳,除直隸涿州等三州縣水沖地賦。丙戌,山東登州鎮總兵馬世龍以科派兵丁,鞫實論絞。賑甘肅河州等三十五州縣衛雹水各災。辛卯,以江西學政金德瑛取士公明,命留任。己亥,以貴州學政佟保守潔士服,命留任。丙午,鄂爾泰議覆劉於義奏勘直隸水利,命撥銀五十萬兩興修。丁未,免浙江仁和等三十一州縣所旱災額賦,並賑之。辛亥,賑成都等三十州縣水災。壬子,允準噶爾貢使哈柳等隨帶牛羊等物在肅州貿易。甲子,免山東歷城等三十二州縣衛本年旱雹等災額賦。乙丑,免直隸保定等十一州縣本年水旱蟲雹災額賦。丙寅,賞雷鋐額外諭德,食俸。戊辰,張照丁憂,調汪由敦為刑部尚書,以趙弘恩為工部尚書。免安徽歙縣等二十一州縣衛水災額賦。辛未,以福建閩縣等縣火災,諭責疆吏不嚴火備。羅卜藏丹怎就獲。

十年春正月丙子,召大學士、內廷翰林於重華宮聯句。改會試於三月,著為令。乙未,大學士鄂爾泰以病乞解任,溫諭慰留。己亥,準噶爾遣使哈柳貢方物。庚子,召高斌來京,以劉於義署直隸總督。己酉,賑浙江淳安等四縣上年水災。朝鮮入貢。辛亥,上幸內右門直廬視鄂爾泰疾。己未,上謁昭西陵、孝陵、孝東陵、景陵。庚申,免廣東海陽等二縣上年水災額賦。甲子,免江蘇丹徒等十州縣衛上年水災額賦。丁卯,上還京師。己巳,免山東博興等二縣乾隆九年旱災額賦。庚午,高斌回直隸總督。

三月癸酉朔,日食。乙亥,改殿試於四月,著為令。賑雲南白鹽井水災。庚辰,上幸鄂爾泰第視疾。辛巳,加鄂爾泰太傅。己丑,協辦大學士、禮部尚書三泰乞休,允之。庚寅,命訥親協辦大學士,調來保為禮部尚書,以盛安為刑部尚書。癸巳,免浙江仁和等三十州縣上年旱災額賦。甲午,以安南莫康武作亂,攻陷太原、高平等處,命那蘇圖等嚴防邊隘。乙未,加史貽直、陳世倌、來保、高斌太子太保,劉於義、張允隨、張廣泗太子少保。

夏四月癸卯朔,發江南帑銀五十六萬兩濬河道。己巳,免山東海豐等二縣被旱額徵灶課。乙卯,大學士鄂爾泰卒,上臨奠,輟朝二日,命遵世宗遺詔,配饗太廟。召那蘇圖來京,以策楞為兩廣總督。調準泰為廣東巡撫。以魏定國為安徽巡撫。庚申,召蔣溥來京,以楊錫紱為湖南巡撫。壬戌,飭沿海各省訓練水師。癸亥,以旱命刑部清理庶獄。戊辰,策試貢士,詔能深悉時政直言極諫者聽。己巳,慶復、紀山奏進剿瞻對番。

五月壬申朔,賜錢維城等三百三十三人進士及第出身有差。丁亥,除江蘇蘇州等九府坍沒蘆課。頒禦制太學訓飭士子文於各省學宮,同世祖臥碑文、聖祖聖諭廣訓、世宗朋黨論朔望宣講。命訥親為保和殿大學士。辛卯,戶部尚書阿爾賽為家奴所害,磔家奴於市。以高斌為吏部尚書,那蘇圖為直隸總督。命高斌、劉於義仍辦直隸水利河道。以梁詩正為戶部尚書。己亥,命劉於義兼管戶部事務。

六月丁未,普免全國錢糧。諭曰:「朕臨御天下,十年於茲。撫育蒸黎,躬行儉約,薄賦輕徭,孜孜保治,不敢稍有暇逸。今寰宇敉寧,左藏有餘,持盈保泰,莫先足民。天下之財,止有此數,不聚於上,即散於下。我皇祖在位六十一年,蠲租賜復之詔,史不絕書,普免天下錢糧一次。我皇考無日不下減賦寬徵之令,如甘肅一省,正賦全行豁免者十有餘年。朕以繼志述事之心,際重熙累洽之後,欲使海澨山陬,俱沾大澤,為是特頒諭旨,丙寅年直省應徵錢糧,其通蠲之。」庚戌,免安徽鳳陽等州府連年被災地方耗羨。命戶部侍郎傅恆在軍機處行走。辛酉,御史赫泰請收回普免錢糧成命。上斥其悖謬,褫職。癸亥,上詣黑龍潭祈雨。

秋七月辛未朔,免甘肅寧夏等三縣逋賦。癸酉,以順直宛平等六十四州縣缺雨,命停徵錢糧。乙酉,命高斌仍兼直隸河道總督。戊子,賑安徽壽州等十八州縣衛水災雹災。壬辰,上奉皇太后幸多倫諾爾,免經過州縣額賦十分之四。戊戌,上奉皇太后駐避暑山莊。賑安徽宿州等州縣衛水災。

八月癸卯,賑兩淮莞瀆等三場水災。停征湖北漢川等十七州縣水災、光化等二縣雹災額賦,並賑之。上奉皇太后幸木蘭行圍。甲辰,上駐波羅河屯。賜青海蒙古王公宴,並賚之。丁未,上行圍永安莽喀。戊申,上行圍畢雅喀拉。己酉,上行圍溫都裏華。辛亥,上行圍額爾袞郭。賜蒙古王、額駙、臺吉等宴。癸丑,上行圍布爾噶蘇臺。甲寅,上行圍巴彥溝。乙卯,上行圍烏里雅蘇臺。賜王、大臣、蒙古王、額駙、臺吉等宴。丙辰,上行圍畢圖舍爾。賑直隸宣化府屬旱災。丁巳,上行圍阿濟格鳩和洛。戊午,上行圍僧機圖。己未,上行圍永安湃。庚申,上行圍英圖和洛。辛酉,上行圍薩達克圖口。壬戌,賑湖北宜城等三州縣衛水災。癸亥,上行圍老圖博勒齊爾。乙丑,上行圍庫爾奇勒。丙寅,賑甘肅安定等三縣、廣東電白等二縣旱災,海豐蟲災,南澳風災。上駐多倫諾爾。丁卯,賜王、大臣、蒙古王、額駙、臺吉等宴。賑山西曲沃等十二州縣水災。

九月庚午朔,上行圍額爾托昂色欽。辛未,上行圍多倫鄂博圖。壬申,遣祭明陵。上行圍古哲諾爾。癸酉,張允隨以猛緬土司奉廷徵等通緬莽,請改土歸流,命詳議。上行圍塔奔陀羅海。乙亥,賑河南永城等五縣水災。上行圍札瑪克圖。丙子,上行圍峞爾呼。丁丑,賑直隸故城等十五州縣衛旱災。癸未,上駐宣化府。甲申,上閱宣化鎮兵。丁亥,賑山東濟寧等六州縣衛水災,海豐旱災。癸巳,上奉皇太后還京師。甲午,授鄂彌達湖廣總督。賑兩淮廟灣場水災。丁酉,以普免錢糧,命查各省歷年存餘銀,以抵歲需。戊戌,授尹繼善兩江總督。命修明愍帝陵。賑江蘇淮、徐、海被災州縣。慶復奏收撫上瞻對,進剿下瞻對班滾,克加社丫等卡及南路各寨。賑陜西長安等六縣水災。

冬十月丁未,以甘肅甘山道歸並肅州道。戊申,賑河南商丘等五縣水災。辛亥,裁通政使司漢右通政一。丙辰,命塞陳家浦決口。戊午,命四川嚴查啯匪。禮部尚書任蘭枝乞休,允之。癸亥,免江蘇海州等七州縣漕糧。甲子,給江南災民葺屋銀。賑江蘇江浦等二十一州縣衛水災。乙丑,賑湖南湘陰等三縣、湖北漢川等二十一州縣衛旱災。丙寅,除湖北當陽等二縣衛水沖地賦。

十一月庚午,賑順直香河等四十八州縣旱災,陜西興平等六縣水災。辛未,賑山東滕縣等七州縣衛水災。壬申,以王安國為禮部尚書。甲戌,賑兩淮廟灣等場水災。乙亥,傅清奏準噶爾臺吉噶爾丹策零與阿卜都爾噶裏木汗構兵。丁丑,賑山西大同等十八州縣旱霜雹災。湖北巡撫晏斯盛乞養,以開泰代之。辛巳,賑廣西思恩等縣旱災。壬午,準噶爾臺吉噶爾丹策零卒。命西北兩路籌備邊防。乙酉,賑廣東海矬等四場風災。戊子,免安徽宿州等五州縣水災地方漕糧。庚寅,陳家浦決口合龍。癸巳,賑直隸宣化府屬及慶雲縣旱災。

十二月辛亥,大學士福敏乞休,優詔允之,加太傅。壬子,命慶復為文華殿大學士,留川陜總督任。命高斌協辦大學士。賑陜西隴西等州縣旱災。賑淮北板浦等場水災。乙卯,命協辦大學士高斌、侍郎蔣溥均在軍機處行走。


\end{pinyinscope}