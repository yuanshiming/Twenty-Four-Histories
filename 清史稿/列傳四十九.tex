\article{列傳四十九}

\begin{pinyinscope}
魏裔介、熊賜履、李光地

魏裔介,字石生,直隸柏鄉人。順治三年進士,選庶吉士。四年,授工科給事中。五年,疏請舉經筵及時講學,以隆治本。又言:「燕、趙之民,椎牛裹糧,首先歸命。此漢高之關中,光武之河內也。今天下初定,屢奉詔蠲賦,而畿輔未霑實惠,宜切責奉行之吏,彰信於民。」俱報聞。

轉吏科,以母憂歸。服闋,九年,起故官。應詔疏言:「上下之情未通,滿、漢之氣中閼。大臣闒茸以保富貴,小臣鉗結以習功名。紀綱日弛,法度日壞。請時御正殿,召對群臣,虛心諮訪。令部院科道等官面奏政事,仍令史官記注,以求救時之實。」時世祖親政,裔介疏言:「督撫重臣宜慎選擇,不宜專用遼左舊人。」又言:「攝政王時,隱匿逃人,立法太嚴,天下囂然,喪其樂生之心。後以言官陳說,始寬其禁,責成州縣,法至善也。若舍此之外別有峻法,竊恐下拂人心,上干天和,非尋常政治小小得失而已。」上韙之。

河南巡撫吳景道援恩詔薦舉明兵部尚書張縉彥。裔介疏言:「縉彥仕明,身任中樞,養寇誤國,有盧杞、賈似道之奸,而庸劣過之。宜予擯棄,以協公論。」疏下部議,以事在赦前,予外用。又疏言:「州縣遇災荒,既經報部,其例得蠲緩錢糧,即予停徵,以杜吏胥欺隱。並就州縣積穀及存貯庫銀,先行賑貸。」下所司議行。時直隸、河南、山東諸省災,別疏請賑。上命發帑金二十四萬,分遣大臣賑之,全活甚眾。

十一年,遷兵科都給事中。東南兵事未定,疏言:「今日劉文秀復起於川南,孫可望竊據於貴築,李定國伺隙於西粵,張名振流氛於海島,連年征討,尚稽天誅。為目前進取計,蜀為滇、黔門戶,蜀既守而滇、黔之勢蹙,故蜀不可不先取。此西南之情形也。粵西稍弱,昨歲桂林之役未大創,必圖再犯,以牽制我湖南之師。宜令籓鎮更番迭出,相機戰守。此三方者,攻瑕宜先粵西。粵西潰則可望膽落,滇、黔亦當瓦解。」又疏劾湖南將軍續順公沈永忠擁兵觀望,致總兵官徐勇、辰常道劉升祚力竭戰死。永忠坐罷任奪爵。復劾福建提督楊名高玩寇,致漳州郡縣為鄭成功淪陷,名高坐罷任。

尋遷太常寺少卿,擢左副都御史。十三年,疏劾大學士陳之遴營私植黨,之遴坐解官,發遼陽閒住。十四年,遷左都御史,上諭之曰:「朕擢用汝,非繇人薦達。」裔介益感奮,盡所欲言。四月,因欽天監推算次月日月交食,疏請廣言路,緩工作,寬州縣考成,速頒恩赦,釋滯獄,酌復五品以下官俸,減徵調之兵,節供應之費。上嘉之,下部詳議以行。嘗侍經筵,講漢文帝春和之詔,因舉仁政所宜先者數事。正陽門外菜園為前朝嘉蔬圃地,久為民居,部議入官。裔介過其地,民走訴,即入告,仍以予民。十六年,加太子太保。十七年,京察自陳。以御史巡方屢坐貪敗,責裔介未糾劾,削太子太保,供職如故。

時可望猶據貴州,鄭成功亂未已。裔介疏言:「可望恃峒蠻為助,宜命在事諸臣加意招徠,予以新敕印,舊者毋即收繳,則歸我者必多。成功作亂海上,我水師無多,惟於沿海要地增兵築堡,使不得泊岸劫掠,然後招其攜貳,散其黨與,海患可以漸平。」下部議行。未幾,疏劾大學士劉正宗、成克鞏欺罔附和諸罪,命正宗、克鞏回奏,未得實,下法司勘訊,並解裔介官與質。讞定,正宗獲罪籍沒,克鞏奪職視事,復裔介官。時以雲南、福建用兵,加派錢糧。裔介疏請敕戶部綜計軍需足用即停止,上命未派者並停止。康熙元年,雲南既定,疏言:「雲南既有吳三桂籓兵數萬,及督提兩標兵,則滿洲兵可撤。但滇、黔、川、楚邊方遼遠,不以滿洲兵鎮守要地,倘戎寇生心,恐鞭長莫及。荊、襄乃天下腹心,宜擇大將領滿兵數千駐防,無事則控制形勢,可以銷奸宄之萌;有事則提兵應援,可以據水陸之勝。」疏下部,格不行。復請以湖廣總督移駐荊州,從之。

進吏部尚書。三年,拜保和殿大學士。時輔臣柄政,論事輒爭執,裔介調和異同,時有所匡正。預修世祖實錄,充總裁官。九年,典會試。是年內院承旨會吏、禮二部選新進士六十人,試以文字,擬上中下三等入奏,上親定二十七人為庶吉士。御史李之芳劾裔介所擬上卷二十四人,先使人通信,招權納賄;並謂與班布爾善相比,引用私人。班布爾善官大學士,黨鼇拜,伏法。上命裔介復奏,裔介疏辨,並言:「臣與班布爾善同官,論事輒齟齬。以鼇拜之執焰,足跡不至其門,豈肯附班布爾善?臣服官以來,彈劾無所避忌。前劾劉正宗,其黨切齒於臣者十年於茲。之芳,正宗同鄉,今為報復。」因自請罷斥,疏下吏部會質。之芳力爭,裔介自引咎。部議以之芳劾奏有因,裔介應削秩罰俸,上寬之,命供職如故。

十年,以老病乞休,詔許解官回籍。世祖實錄成,進太子太傅。二十五年,卒,賜祭葬如制。

裔介居言路最久,疏至百餘上,敷陳剴切,多見施行。生平篤誠,信程、硃之學,以見知聞知述聖學之統。著述凡百餘卷,大指原本儒先,並及經世之學。家居十六年,躬課稼穡,循行阡陌,人不知其為故相也。雍正間,祀賢良祠。乾隆元年,追謚文毅。

熊賜履,字敬修,湖北孝感人。順治十五年進士,選庶吉士,授檢討。典順天鄉試,遷國子監司業,進弘文院侍讀。

康熙六年,聖祖詔求直言。時輔臣鼇拜專政,賜履上疏幾萬言,略謂:「民生困苦孔亟,私派倍於官徵,雜項浮於正額。一旦水旱頻仍,蠲豁則吏收其實而民受其名,賑濟則官增其肥而民重其瘠。然非獨守令之過也,上之有監司,又上之有督撫。朝廷方責守令以廉,而上官實縱之以貪;方授守令以養民之職,而上官實課以厲民之行。故督撫廉則監司廉,守令亦不得不廉;督撫貪則監司貪,守令亦不得不貪。此又理勢之必然者也。伏乞甄別督撫,以民生苦樂為守令之賢否,以守令貪廉為督撫之優劣。督撫得人,守令亦得人矣。雖然,內臣者外臣之表也,本原之地則在朝廷。其大者尤在立綱陳紀、用人行政之間。今朝廷之可議者不止一端,擇其重且大者言之:一曰,政事極其紛更,而國體因之日傷也。國家章程法度,不聞略加整頓,而急功喜事之人又從而意為更變,但知趨目前尺寸之利以便其私,而不知無窮之患已潛滋暗伏於其中。乞敕議政王等詳議制度,參酌古今,勒為會典,則上有道揆、下有法守矣。一曰,職業極其隳窳,而士氣因之日靡也。部院臣工大率緘默瞻顧,外託老成慎重之名,內懷持祿養身之念。憂憤者謂之疏狂,任事者目為躁競,廉靜者斥為矯激,端方者詆為迂腐。間有讀書窮理之士,則群指為道學,誹笑詆排,欲禁錮其終身而後已。乞申飭滿、漢諸臣,虛衷酌理,實心任事,化情面為肝膽,轉推諉為擔當。漢官勿阿附滿官,堂官勿偏任司員。宰執盡心獻納,勿以唯諾為休容,臺諫極力糾繩,勿以鉗結為將順,則職業修舉,官箴日肅而士氣日奮矣。一曰,學校極其廢弛,而文教因之日衰也。今庠序之教缺焉不講,師道不立,經訓不明。士子惟揣摩舉業,為弋科名掇富貴之具,不知讀書講學、求聖賢理道之歸。高明者或汎濫於百家,沉淪於二氏,斯道淪晦,未有甚於此時者也。乞責成學院、學道,統率士子,講明正學,特簡儒臣使司成均,則道術以明,教化大行,人才日出矣。一曰,風俗極其僭濫,而禮制因之日壞也。今一裘而費中人之產,一宴而糜終歲之糧,輿隸被貴介之服,倡優擬命婦之飾,習為固然。夫風俗奢、禮制壞,為饑寒之本原,盜賊、訟獄、兇荒所由起也。乞明詔內外臣民,一以儉約為尚,自王公以及士庶,凡宮室、車馬、衣服,規定經制,不許逾越,則貪風自息、民俗漸醇矣。雖然,猶非本計也。根本切要,端在皇上。皇上生長深宮,春秋方富,正宜慎選左右,輔導聖躬,薰陶德性,優以保衡之任,隆以師傅之禮;又妙選天下英俊,使之陪侍法從,朝夕獻納。毋徒事講幄之虛文,毋徒應經筵之故事,毋以寒暑有輟,毋以晨夕有間。於是考諸六經之文,監於歷代之跡,實體諸身心,以為敷政出治之本。若夫左右近習,必端其選,綴衣虎賁,亦擇其人。佞幸不置於前,聲色不御於側。非聖之書不讀,無益之事不為。內而深宮燕閒之間,外而大廷廣眾之地,微而起居言動之恆,凡所以維持此身者無不備,防閑此心者無不周,主德清明,君身強固。由是直接二帝三王之心法,目足措斯世於唐、虞、三代之盛,又何吏治之不清,民生之不遂哉?」疏入,鼇拜惡之,請治以妄言罪,上勿許。

七年,遷秘書院侍讀學士。疏言:「朝政積習未除,國計隱憂可慮。年來災異頻仍,饑荒疊見,正宵旰憂勤、徹懸減膳之日,講學勤政,在今日最為切要。乞時御便殿,接見群臣,講求政治,行之以誠,持之以敬,庶幾轉咎徵為休徵。」疏入,鼇拜傳旨詰問積習、隱憂實事,以所陳無據,妄奏沽名,下吏議,鐫二秩,上原之。八年,鼇拜敗,命康親王傑書等鞫治,以鼇拜銜賜履,意圖傾害,為罪狀之一。方鼇拜輔政擅威福,大臣稍與異同,立加誅戮。賜履以詞臣論事侃侃無所避,用是著直聲。上即位後,未舉經筵,賜履特具疏請之,並請設起居注官。上欲幸塞外,以賜履疏諫,乃寢,且嘉其直。

九年,擢國史院學士。未幾,復內閣,設翰林院,更以為掌院學士。舉經筵,以賜履為講官,日進講弘德殿。賜履上陳道德,下達民隱,上每虛己以聽。十四年,諭★其才能清慎,遷內閣學士,尋超授武英殿大學士,兼刑部尚書。十五年,陜西總督哈占疏報獲盜,開復疏防官,下內閣,賜履誤票三法司核擬。既,檢舉,得旨免究。賜履改草簽,欲諉咎同官杜立德,又取原草簽嚼而毀之,立德以語索額圖。事上聞,吏部議賜履票擬錯誤,欲諉咎同官杜立德,改寫草簽,復私取嚼毀,失大臣體,坐奪官。歸,僑居江寧。

二十三年,上南巡,賜履迎謁,召入對,御書經義齋榜以賜。二十七年,起禮部尚書。未幾,以母憂去。二十八年,上復南巡,賞賚有加。二十九年,起故官,仍直經筵。命往江南讞獄,調吏部。會河督靳輔請豁近河所占民田額賦,命賜履會勘。奏免高郵、山陽等州縣額賦三千七百二十八頃有奇。三十四年,弟編修賜瓚以奏對欺飾下獄,御史龔翔麟遂劾吏部銓除州縣以意高下,賜履偽學欺罔,乞嚴譴。下都察院議,賜履與尚書庫勒納,侍郎趙士麟、彭孫遹當降官,上不問,賜瓚亦獲赦。

三十八年,授東閣大學士兼吏部尚書,預修聖訓、實錄、方略、明史,並充總裁官。典會試者五。以年老累疏乞休。四十二年,溫旨許解機務,仍食俸,留京備顧問。四十五年,乞歸江寧。比行,召入講論累日。賜履因奏巡幸所至,官民供張煩費,惟上留意,上頷之,給傳遣官護歸。四十六年,上閱河,幸江寧,召見慰問,賜御用冠服。四十八年,卒,年七十五,命禮部遣官視喪,賜賻金千兩,贈太子太保,謚文端。五十一年,上追念賜履,知其貧,迭命江寧織造周恤其家,諭吏部召其二子志契、志夔詣京師,皆尚幼,復諭賜履僚屬門生醵金佽之。

賜履論學,以默識篤行為旨,其言曰:「聖賢之道,不外乎庸,庸乃所以為神也。」著閑道錄,嘗進上,命備省覽。雍正間,祀賢良祠。

李光地,字晉卿,福建安溪人。幼穎異。年十三,舉家陷山賊中,得脫歸。力學慕古。康熙九年成進士,選庶吉士,授編修。十二年,乞省親歸。

十三年,耿精忠反,鄭錦據泉州,光地奉親匿山谷間,錦與精忠並遣人招之,力拒。十四年,密疏言:「閩疆褊小,自二賊割據,誅求敲撲,民力已盡,賊勢亦窮。南來大兵宜急攻,不可假以歲月,恐生他變。方今精忠悉力於仙霞、杉關,鄭錦並命於漳、潮之界,惟汀州小路與贛州接壤,賊所置守御不過千百疲卒。竊聞大兵南來,皆於賊兵多處鏖戰,而不知出奇以搗其虛,此計之失也。宜因賊防之疏,選精兵萬人或五六千人,詐為入廣,由贛達汀,為程七八日耳。二賊聞急趨救,非月餘不至,則我軍入閩久矣。賊方悉兵外拒,內地空虛,大軍果從汀州小路橫貫其腹,則三路之賊不戰自潰。伏乞密敕領兵官偵諜虛實,隨機進取。仍恐小路崎嶇,須使鄉兵在大軍之前,步兵又在馬兵之前,庶幾萬全,可以必勝。」置疏蠟丸中,遣使間道赴京師,因內閣學士富鴻基上之。上得疏動容,嘉其忠,下兵部錄付領兵大臣。時尚之信亦叛,師次贛州、南安,未能入福建。康親王傑書自衢州克仙霞關,復建寧、延平,精忠請降。師進駐福州,令都統拉哈達、賚塔等討鄭錦,並求光地所在。十六年,復泉州,光地謁拉哈達於漳州。拉哈達白王,疏稱「光地矢志為國,顛沛不渝,宜予褒揚」,命優敘,擢侍讀學士。行至福州,以父喪歸。

十七年,同安賊蔡寅結眾萬餘,以白巾為號,掠安溪。光地募鄉勇百餘人扼守,絕其糧道,賊解去。未幾,錦遣其將劉國軒陷海澄、漳平、同安、惠安諸縣,進逼泉州,斷萬安、江東二橋,南北援絕。光地遣使赴拉哈達軍告急,值江水漲,道阻,乃導軍自漳平、安谿小道入。光地從父日蚃率鄉勇度石珠嶺,芟荊棘,架浮橋以濟。光地出迎,具牛酒犒軍。又使弟光垤、光垠以鄉兵千度白鴿嶺,迎巡撫吳興祚軍於永春。師次泉州,擊破國軒,竄入海。拉哈達上其功,再予優敘,遷翰林學士。光地上疏推功將帥,辭新命,不允;並官日蚃,後積功官至永州總兵。

十九年,光地至京師,授內閣學士。入對,言:「鄭錦已死,子克塽幼弱,部下爭權,宜急取之。」且舉內大臣施瑯習海上形勢,知兵,可重任,上用其言,卒平臺灣。

陳夢雷者,侯官人。與光地同歲舉進士,同官編修。方家居,精忠亂作,光地使日蚃潛詣夢雷探消息,得虛實,約並具疏密陳破賊狀,光地獨上之,由是大受寵眷。及精忠敗,夢雷以附逆逮京師,下獄論斬。光地乃疏陳兩次密約狀,夢雷得減死戍奉天。

二十一年,乞假奉母歸。二十五年,還京,授翰林院掌院學士,直經筵,兼充日講起居注官,教習庶吉士。逾年,以母病乞歸省。二十七年,至京。初,光地與侍讀學士德格勒善,於上前互相稱引。上召德格勒與諸詞臣試乾清宮,以文字劣,鐫秩。旋掌院庫勒訥劾其私抹起居注事,下獄論罪。詔責光地,光地引罪,乞嚴譴,上原之。尋擢兵部侍郎。三十年,典會試。偕侍郎博霽、徐廷璽,原任河督靳輔勘視河工。三十三年,督順天學政。聞母喪,命在任守制。光地乞假九月回裡治喪。御史沈愷曾、楊敬儒交章論劾,上令遵初命。給事中彭鵬復疏論光地十不可留,目為貪位忘親,排詆尤力。乃下九卿議,命光地解任,在京守制。三十五年,服闋,仍督順天學政。三十六年,授工部侍郎。

三十七年,出為直隸巡撫。初,畿輔屢遭水患,上以漳河與滹沱合流易汎濫,命光地導漳自故道引入運河,殺滹沱之勢。光地疏言:「漳河現分為三:一自廣平經魏、元城,至山東館陶入衛水歸運;一為老漳河,自山東丘縣經南宮諸縣,與完固口合流,至鮑家嘴歸運;一為小漳河,自丘縣經廣宗、鉅鹿合於滏,又經束鹿、冀州合於滹沱。由衡水出獻縣完固口復分為兩支:小支與老漳河合流而歸運,大支經河間、大城、靜海入子牙河而歸澱。今入衛之河與老漳河流淺而弱,宜疏濬;其完固口小支應築壩逼水入河,更於靜海閻、留二莊挑土築堤,束水歸澱,俾無汎濫。」詔報可。尋奏霸州、永清、宛平、良鄉、固安、高陽、獻縣因濬新河,占民田一百三十九頃,請豁免賦額,從之。通州等六州縣額設紅剝船六百號,剝運南漕,每船給贍田,遇水旱例不蠲免,光地奏請援民田例概蠲免之。三十九年,上臨視子牙河工,命光地於獻縣東西兩岸築長堤,西接大城,東接靜海,亙二百餘里;又於靜海廣福樓、焦家口開新河,引水入澱:由是下流益暢,無水患。四十二年,上褒其治績,擢吏部尚書,仍管巡撫事。四十三年,給事中黃鼎楫、湯右曾、許志進、宋駿業、王原等合疏劾光地撫綏無狀,致河間饑民流入京畿,並寧津縣匿災不報狀。光地疏辨,引咎乞罷,詔原之。再疏辭尚書,不許。尋疏劾雲南布政使張霖假稱詔旨,販鬻私鹽,得銀百六十餘萬,霖論斬,籍沒。

四十四年,拜文淵閣大學士。時上潛心理學,旁闡六藝,御纂硃子全書及周易折中、性理精義諸書,皆命光地校理,日召入便殿揅求探討。四十七年,皇太子允礽以疾廢,命諸大臣保奏諸皇子孰可當儲位者。尚書王鴻緒等舉皇子允禩,上切責之。詢光地何無一言,光地奏:「前者皇上問臣以廢太子病,臣奏言徐徐調治,天下之福,臣未嘗告諸人也。」光地被上遇,同列多忌之者,凡所稱薦,多見排擠,因以撼光地。撫直隸時,御史呂履恆劾光地於秋審事任意斷決,上察其不實,還其奏。給事中王原劾文選郎中陳汝弼受贓,法司論絞,汝弼,光地所薦也。上察其供證非實,下廷臣確核,得逼供行賄狀,汝弼免罪,承讞官降革有差,原奪官。

光地益敬慎,其有獻納,罕見於章奏。江寧知府陳鵬年忤總督阿山,坐事論重闢,光地言其誣,鵬年遂內召。兩江總督噶禮與巡撫張伯行互訐,遣大臣往訊,久不決。嗣詔罷噶禮,復伯行官,光地實贊之。桐城貢士方苞坐戴名世獄論死,上偶言及侍郎汪霦卒後,誰能作古文者,光地曰:「惟戴名世案內方苞能。」苞得釋,召入南書房。其扶植善類如此。

五十二年,與千叟宴,賜賚有加。頃之,以病乞休,溫旨慰留。越二年,復以為請,且言母喪未葬,許給假二年,賜詩寵行。五十六年,還朝,累疏乞罷,上以大學士王掞方在告,暫止之。五十七年,卒,年七十七,遣恆親王允祺奠醊,賜金千兩,謚文貞。使工部尚書徐元夢護其喪歸,復諭閣臣:「李光地謹慎清勤,始終一節,學問淵博。朕知之最真,知朕亦無過光地者!」雍正初,贈太子太傅,祀賢良祠。

弟光坡,性至孝,家居不仕,潛心經術。子鍾倫,舉人,治經史性理,旁及諸子百家,從其叔父光坡治三禮,於周官、禮記尤精,稱其家學。從子天寵,進士,官編修,有志操,邃於經學,與弟鍾僑、鍾旺俱以窮經講學為業。鍾僑進士,官編修,督學江西,以實行課士,左遷國子監丞。鍾旺,舉人,授中書,充性理精義纂修官。

論曰:聖祖崇儒重道,經筵講論,孜孜聖賢之學,朝臣承其化,一時成為風氣。裔介久官臺諫,數進讜言,為憂盛危明之計,自登政府,柴立不阿,奉身早退,有古大臣之風。賜履剛方鯁直,疏舉經筵,冀裨主德,庶乎以道事君者歟?光地易又歷中外,得君最專,而疑叢業集,委蛇進退,務為韜默。聖祖嘗論道學不在空言,先行後言,君子所尚。夫道學豈易言哉?


\end{pinyinscope}