\article{列傳六十五}

\begin{pinyinscope}
慕天顏阿山噶禮

慕天顏,字拱極,甘肅靜寧人。順治十二年進士,授浙江錢塘知縣。遷廣西南寧同知,再遷福建興化知府。康熙九年,擢湖廣上荊南道。總督劉兆麒疏言天顏習邊海諸事,請調福建興泉道。尋擢江蘇布政使。十二年,喪母。總督麻勒吉、巡撫瑪祜疏言:「天顏廉明勤敏,清積年逋賦,釐剔挪移,事未竟,請令在官守制。」十三年,入覲,疏言:「江南田地錢糧有隱占、詭寄諸弊,臣飭州縣通計田額,均分里甲;又因科則不等,立徵收截票之法,每戶實徵錢糧分十限,於開徵日給限票,依限完納截票。逾限未截,按數追比,吏不能欺民。」下部,著為令。

十五年,擢江寧巡撫。疏進錢糧交代冊,上嘉其清晰,命布政使交代當以此為式。尋以節減驛站錢糧,加兵部侍郎。師征吳三桂,大將軍貝勒尚善請造船濟師,下天顏督造送岳州。敘勞,加太子少保、兵部尚書,仍兼右副都御史。時諸道兵應徵發,舳艫蔽江,夫役牽挽,動以千萬計。天顏疏言:「纖夫募諸民間,夫給銀一錢。民爭逃匿,計里均派,先期拘集,饑寒踣頓。及兵既到,計船給夫,兵與船戶橫索財物,鞭撻死傷。臣擬軍赴前敵,仍給纖夫;其凱旋還京,並各省調遣歸標官兵,每船應夫若干,以其直給船戶,令雇水手。」上從之,命下直省,著為令。

江南水道交錯,天顏為布政使時,請於巡撫瑪祜,濬吳淞江、劉河淤道。十九年,江南困霪雨,疏言:「附近吳淞江、劉河諸州縣水道通暢,旋溢旋消。宜興、常熟、武進、江陰、金壇諸縣水無出路,或要口湮塞,致積雨成壑。常熟白茆港為長洲、昆山、無錫諸水出海要道,武進孟瀆河為丹陽、宜興、金壇諸水歸江要道,請動帑疏濬。」上從之。於是濬白茆港四十三里達海,濬孟瀆河四十八里達江,皆建閘以時啟閉,費帑九萬有奇。又嘗疏請減浮糧,除版荒、坍沒公占田地,部議坍沒許豁除,版荒令覆勘。二十年,疏請募民墾版荒,六年後起科。

揚州知府高德貴虧帑數萬,既劾罷,旋卒;天顏疏銷草豆價,戶部覈減七千有奇,天顏檄追德貴家屬。京口防御高騰龍,德貴族也,與參領馬崇駿以天顏奏銷浮冒訐於將軍楊鳳翔,鳳翔格不行。總督阿席熙劾崇駿、騰龍婪取,上遣郎中圖爾宸、鍾有德會天顏勘治。崇駿、騰龍叩閽訟天顏奏銷浮冒,惡其訐告構罪狀,唆總督劾奏。上命圖爾宸、鍾有德具獄,崇駿、騰龍婪取罪至死,天顏以草豆價戶部覈減諉罪德貴,當左遷。得旨,如議。

天顏將去官,疏列成勞,且言:「夙夜冰兢精白,不意遭誣訐,蒙鑒宥不加嚴譴。」上以天顏未聞有廉名,乃自言「冰兢精白」,非是,命嚴飭。二十三年,起湖北巡撫,復諭之曰:「爾前為巡撫,未能潔己率屬。今宜痛改前非,廉謹自持,以副任使。」旋移貴州。

二十六年,授漕運總督,疏言:「京口至瓜洲,漕船往來,風濤最險。請仿民間渡生船,官設十船,導引護防。」部議非例,不允。上曰:「朕南巡見京口、瓜洲往來人眾,備船過渡,有益於民。其如所請行。」天顏疏陳江南、江西累年未完漕項銀米請恩貰,上命盡免康熙十七年以前積逋。江南揚州、淮安所屬運河東瀕海諸州縣地卑下,謂之下河,頻歲被水。上先用湯斌議,遣侍郎孫在豐疏濬下河。河道總督靳輔議起翟家壩迄高家堰築重堤,束堤堰溢出之水北出清口,謂疏濬無益。天顏仍主疏濬,並修築高家堰,與不協。上遣尚書佛倫、熊一瀟,給事中達奇納、趙吉士會勘,佛倫等主用輔議,天顏、在豐議與輔異。天顏密疏力爭,輔疏劾天顏與在豐有連,欲在豐建功,故堅阻上游築堤。下部議,奪天顏職,而輔亦為御史郭琇、陸祖修,給事中劉楷交章劾罷。初,輔請於仲家莊建閘,引駱馬湖水,別鑿中河,俾漕船避黃河之險,天顏亦議為無益。上命學士開音布、侍衛馬武往視,還奏天顏令漕船毋入中河,上以責天顏,逮下獄。天顏反覆申辨,副都御史噶爾圖舉天顏訴辭先後互異,坐奏事上書不以實論罪,上追錄天顏造舟濟師,特寬之。三十五年,卒。

天顏歷官有惠績,嘗疏請有司虧帑雖逾限,於發遣前清償,仍貰其罪。獄囚因逸犯株連,待質已三年者,於秋審時開釋;獄囚無親屬饋食,月給米三斗:皆恤下之政。在江南,興水利,蠲積逋,而請免纖夫,甦一時之困,江南民尤頌之。獨劾嘉定知縣陸隴其不協於輿論,左都御史魏象樞疏言:「天顏劾隴其,稱其操守絕一塵,德有餘而才不足。今之有司,惟操守為難;既知之矣,何不留以長養百姓?請嚴飭諸督撫大破積習,勿使廉吏灰心,貪風日長。」會詔舉清廉,象樞遂以隴其應,語具隴其傳。

阿山,伊拉哩氏,滿洲鑲藍旗人。初自吏部筆帖式歷刑部主事、戶部員外部。康熙十八年,授翰林院侍講,七遷至戶部侍郎。三十年,命治賑西安、鳳翔二府,明年還京。上聞流民有至襄陽者,以問阿山。阿山言正月已得雪,民無流亡。上曰:「正月雖雪,二、三月雨不時,麥收未可望。流民至襄陽甚多,汝未之知耳。」坐奉使不盡心,左授郎中。三十三年,擢左副都御史。三十五年,上親征噶爾丹,阿山從。授阿密達為將軍,逐噶爾丹,阿山為參贊。師還,授盛京禮部侍郎。三十六年,授翰林院掌院學士。

三十九年,授江南江西總督。安徽布政使張四教以憂去官,巡撫高永爵劾四教擅動庫帑,下阿山察奏。阿山言四教動庫帑為公用,請免議,上復命具實狀以聞。阿山乃言:「三十八年上南巡,四教發庫帑十一萬供辦,議令各官扣俸抵補。各官皆自承,臣不敢隱。」上責阿山徇情沽譽,命漕運總督桑額鞫四教,論如律。阿山當奪職,上寬之,命留任。

四十三年,阿山劾江西巡撫張志棟大計不公,志棟及布政使李興祖、按察使劉廷璣、道員韓象起等皆奪職。阿山又言大計志棟主之,請復興祖等官。給事中許志進劾阿山恩威自擅,阿山疏辯,且詆志進為淮安漕標營卒子,素行不端,為志棟報復。志進亦追論阿山庇張四教,並收屬吏賄賂,盜倉穀不問,貪淫惡跡,縱妾父生事。疏並下部議,部議皆奪職。上復寬阿山,命留任如故。四十四年,疏劾江寧知府陳鵬年貪酷,並以妓樓改建講堂,瀆聖諭,大不敬。命會桑額及河道總督張鵬翮集讞,坐鵬年罪至斬,上特命來京,事具鵬年傳。

阿山與桑額、鵬翮議自泗州開河築堤,引淮水至黃家堰,入張福口,會出清口,是為溜淮套,疏請上臨視。四十五年,授刑部尚書。四十六年,上南巡,臨視溜淮套,諭曰:「阿山等奏溜淮套別開一河,分洩淮水,繪圖進呈。朕策騎自清口至曹家廟,見地勢甚高,雖成河,不能直達清口,與所進圖不同。且所立標竿多在民塚上,朕何忍發此無限枯骨耶?」命鵬翮罷其事。下九卿議,阿山及桑額、鵬翮皆奪職;上以阿山主其議,命但坐阿山,遂奪職。五十一年,江蘇布政使宜思恭以虧帑坐譴,因列訴總督噶禮等頻向需索,阿山亦受節餽,下部議,上以阿山老,寬之。五十二年,萬壽,復原品。逾年,卒。

阿山故精察,上嘗問大學士李光地:「阿山在官何若?」光地奏:「臣嘗與同僚,廉幹,果於任事。其失民心,獨劾陳鵬年一事耳。」上頷之。

噶禮,棟鄂氏,滿洲正紅旗人,何和哩四世孫也。自廕生授吏部主事,再遷郎中。康熙三十五年,上親征噶爾丹,次克魯倫河。噶禮從左都御史於成龍督運中路兵糧,首達行在,召對,當上意。尋擢盛京戶部理事官。歲餘三遷,授內閣學士。三十八年,授山西巡撫。噶禮當官勤敏能治事,然貪甚,縱吏虐民。撫山西數年,山西民不能堪。會潞安知府缺員,噶禮疏薦霍州知州李紹祖,紹祖使酒自刎,噶禮匿不以奏。上聞之,下九卿議罪,擬奪噶禮職,上寬之。御史劉若鼐疏論噶禮貪,得贓無慮數十萬,太原知府趙鳳詔為其腹心,專用酷刑以濟貪壑事。下噶禮復奏,得辨釋。

平遙民郭明奇等以噶禮庇貪婪知縣王綬,走京師詣巡城御史袁橋列訴。橋疏聞,並言「噶禮通省錢糧加火耗十之二,分補大同、臨汾等縣虧帑,餘並以入己,得四十餘萬;指修解州祠宇,用巡撫印簿勒捐;令家伶赴平陽、汾州、潞安三府迫富民饋遺;又以訟得臨汾、介休富民亢時鼎、梁湄金;縱汾州同知馬遴;庇洪洞知縣杜連登,皆貪吏;隱平定雹災」,凡七事。上命噶禮復奏,山西學政鄒士聰代太原士民疏留噶禮。御史蔡珍疏劾士璁「職在衡文,乃與巡撫朋比。且袁橋疏得旨二日後,太原士民即具呈,顯為誣偽。噶禮與士璁同城,委為不知,是昏憒也;知而不阻,是幸恩也。請並敕部議處」。尋噶禮復奏,以明奇等屢坐事走京師誣告,並辨橋、珍所言皆無據。下九卿察奏,明奇等下刑部治罪,橋、珍坐誣譴罷。

四十八年,遷戶部侍郎,旋擢江南江西總督。噶禮至江南,益恣肆,累疏劾江蘇巡撫於準、布政使宜思恭、按察使焦映漢,皆坐罷。知府陳鵬年初為總督阿山劾罷,上復命守蘇州;及宜思恭罷,署布政使。鵬年素伉直,忤噶禮。噶禮續劾宜思恭虧帑,又論糧道賈樸建關開河皆有所侵蝕,遂及鵬年覈報不實,鵬年復坐罷。噶禮復密疏鵬年虎丘詩怨望,上不為動。

巡撫張伯行有廉聲,至則又與噶禮忤。五十年,伯行疏言本科江南鄉試取士不協輿論,正考官副都御史左必蕃亦檢舉同考官知縣王曰俞、方名所薦士有不通文字者。上命尚書張鵬翮如揚州會噶禮及伯行察審。鵬翮至,會讞,既得副考官編修趙晉及曰俞、名諸交通狀,伯行欲窮其獄。噶禮盛怒,刑證人,遂罷讞。伯行乃劾噶禮,謂輿論盛傳總督與監臨提調交通鬻舉人;及事發,又傳總督索銀五十萬,許不竟其事:請敕解任就讞。噶禮亦劾伯行,謂:「方會讞時,臣正鞫囚,伯行謂臣言不當,臣恐爭論失體,緘口結舌。伯行遂陰謀誣陷,以鬻舉人得銀五十萬汙臣,臣不能與俱生。」因及伯行專事著書,猜忌糊塗,不能清理案牘。時方有戴名世之獄,又言:「南山集刻板在蘇州印行,伯行豈得不知?進士方苞以作序連坐,伯行夙與友,不肯捕治。」並羅列伯行不職數事。

疏入,上並命解任,令鵬翮會漕運總督赫壽察奏。獄具,晉、曰俞、名及所取士交通得賄,當科場舞弊律論罪;噶禮劾伯行不能清理案牘事實,餘皆督撫會銜題咨舊事,苞為伯行逮送刑部,南山集刻板在江寧,皆免議;伯行妄奏噶禮鬻舉人,當奪職。上切責鵬翮、赫壽瞻徇,又命尚書穆和倫、張廷樞覆讞,仍如鵬翮等議。上諭曰:「噶禮才有餘,治事敏練,而性喜生事,屢疏劾伯行。朕以伯行操守為天下第一,手批不淮。此議是非顛倒!」下九卿、詹事、科道察奏,復諭曰:「噶禮操守,朕不能信;若無張伯行,江南必受其朘削且半矣。即如陳鵬年稍有聲譽,噶禮欲害之,摘虎丘詩有悖謬語,朕閱其詩,初無他意。又劾中軍副將李麟騎射皆劣。麟比來迎駕,朕試以騎射,俱優。若令噶禮與較,定不能及。朕於是心疑噶禮矣。互劾之案,遣大臣往讞,為噶禮所制。爾等皆能體朕保全廉吏之心,使正人無所疑懼,則海宇蒙升平之福矣。九卿等議噶禮與伯行同任封疆,互劾失大臣禮,皆奪職;上命留伯行任,噶禮如議奪職。

五十三年,噶禮母叩閽,言噶禮與弟色勒奇、子幹都置毒食物中謀弒母,噶禮妻以別戶子幹泰為子,縱令糾眾毀屋。下刑部鞫得實,擬噶禮當極刑,妻論絞,色勒奇、幹都皆斬,幹泰發黑龍江,家產沒入官。上令噶禮自盡,妻從死,餘如部議。

論曰:廉吏往往不獲於上,豈長官皆不肖,抑其強項固有所不可堪歟?隴其之廉,天顏知之而不能容。鵬年初扼於阿山,繼挫於噶禮,皆欲中以危法,抑又甚矣。伯行與噶禮互劾,再讞不得直。幸賴聖祖仁明,隴其復起,鵬年致大用,伯行亦終獲全。二三正人詘而得申,人心風氣震蕩洋溢,所被至遠。噶禮不足以語此,蓋天顏、阿山亦弗能喻也。


\end{pinyinscope}