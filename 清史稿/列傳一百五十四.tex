\article{列傳一百五十四}

\begin{pinyinscope}
長齡那彥成子容安容照玉麟特依順保

長齡,字懋亭,薩爾圖克氏,蒙古正白旗人,尚書納延泰子,惠齡之弟也。乾隆中,由繙譯生員補工部筆帖式,充軍機章京,擢理籓院主事。從征甘肅、臺灣、廓爾喀,累擢內閣學士,兼副都統。嘉慶四年,授右翼總兵。五年,赴湖北剿教匪,為領隊大臣,數敗高天升、馬學禮於川、楚交界,授宜昌鎮總兵。又敗徐天德、茍文明等。六年,擢湖北提督,署總督。七年,敗樊人傑、曾芝秀等,予雲騎尉世職。以病回京,歷左翼總兵,出為古北口提督。九年,授安徽巡撫,擒蒙城教匪餘連。十年,調山東。十二年,擢陜甘總督,討平西寧叛番。十三年,坐在山東供應欽差侍郎廣興動用庫帑,褫職,戍伊犁。尋予藍翎侍衛,充科布多參贊大臣。十六年,授河南巡撫。十八年,復授陜甘總督,剿擒南山匪首萬五等,晉騎都尉世職。

二十一年,予都統銜,充伊犁參贊大臣,命察治回匪圖爾邁善獄,劾罷將軍松筠,遂代之。二十二年,復授陜甘總督。道光元年,加太子少保,協辦大學士,留總督任。二年,署直隸總督。會青海野番滋事,命回陜甘,遣總兵穆爾泰、馬騰龍討平之,賜雙眼花翎,拜文華殿大學士,管理籓院事,召還京。尋以青海奏凱後,野番復渡河劫掠,奪雙眼花翎。三年,授軍機大臣,管理戶部三庫,充總諳達。四年,出為雲貴總督,五年,調陜甘,改授伊犁將軍。

初,回疆自乾隆中戡定後,歲徵貢稅頗約。旋懲於烏什之亂,由辦事大臣縱肆激變,益慎選邊臣,回民賴以休息。久之,法漸弛,蒞其任者,往往苛索伯克,伯克又斂之回民。嘉慶末,參贊大臣斌靜尤淫虐,失眾心。張格爾者,回酋大和卓木博羅尼都之孫也。博羅尼都當乾隆中以叛誅,至是張格爾因眾怨糾安集延、布魯特寇邊。道光二年,逮治斌靜,代以永芹,亦未能撫馭。四年秋、五年夏兩次犯邊,領隊大臣巴彥圖敗績,遂益猖獗。

六年六月,張格爾大舉入卡,陷喀什噶爾、英吉沙爾、葉爾羌、和闐四城,命陜甘總督楊遇春駐哈密,督兵進剿。長齡疏言:「逆酋已踞巢穴,全局蠢動。喀城距阿克蘇二千里,四面回村,中多戈壁,非伊犁、烏魯木齊六千援兵所能克。請速發大兵四萬,以萬五千分護糧臺,以二萬五千進戰。」詔授長齡揚威將軍,遇春及山東巡撫武隆阿為參贊,率諸軍討之。十月,師抵阿克蘇。時提督達凌阿等已敗賊渾巴什河,張格爾以眾三千踞柯爾坪,令提督楊芳襲破之。大雪封山,兵止未進,疏言:「前奉旨兵分二路,正兵由中路臺站、奇兵由烏什草地,繞出喀城,斷其竄遁。惟烏什卡倫外直抵巴爾昌,山溝險狹,戈壁數百里,所經布魯特部落,半為賊煽,未可孤軍深入。且留防阿克蘇、烏什、庫車兵八千餘,其延、綏、四川兵尚未到。進剿之步騎止二萬二千,兩路相距二十餘站,聲息不通。喀城賊眾不下數十萬,非全軍直搗,反正為奇,難期無失。喀城邊外凡十卡,皆接外夷,恐賊敗遁,已諭黑回約眾邀截。」

七年二月,師至巴爾楚軍臺,為喀、葉二城分道處,復留兵三千以防繞襲。進次大河拐,賊屯洋阿爾巴特,夜來犯營,卻之。遂由中路進,殲賊萬餘,擒五千。越三日,張格爾拒戰於沙布都爾,多樹葦,決水成沮洳,賊數萬臨渠橫列。乃令步卒越渠鏖斗,騎兵繞左右橫截入陣,賊潰,追逾渾水河,擒斬萬計。又越二日,進剿阿瓦巴特,分三路掩殺,俘斬二萬有奇。追至洋達瑪河,距喀城僅十餘里,賊悉眾十餘萬背城阻河而陣,亙二十餘里,選死士夜擾其營。會大風霾,用楊遇春策,遣索倫千騎繞趨下游牽賊勢,大兵驟渡上游蹙之,賊陣亂,乃大奔,乘勝抵喀什噶爾,克之。時三月朔日也。張格爾已先遁,獲其侄與甥,及安集延酋推立汗、薩木汗。分兵令遇春下英吉沙爾、葉爾羌,芳下和闐,於是四城皆復。

上以元惡漏網,嚴詔詰責,限速捕獲。六月,遇春、芳率兵八千出塞窮追,遇春屯色勒庫,芳屯阿賴,諭各部落擒獻。浩罕遣諜誘官軍入伏,鏖戰幾殆,僅得出險。詔斥諸將老師糜餉,留兵八千,餘命遇春率兵入關,芳代為參贊。當大軍之出,密詔詢將軍、參贊:事平後,西四城可否仿土司分封。至是,長齡疏言:「愚回崇信和卓,猶西番崇信達賴,即使張逆就擒,尚有兄弟之子在浩罕,終留後患。八千留防之兵難制百萬犬羊之眾。博羅尼都之子阿布都哈里尚羈在京師,惟有赦歸,令總轄西四城,可以服內夷、制外患。」武隆阿亦以為言。上切責其請釋逆裔之謬,並革職留任,命那彥成為欽差大臣,代長齡籌善後。

張格爾傳食諸部落,日窮蹙。長齡等遣黑回誘之,率步騎五百,欲乘歲除襲喀城。芳嚴兵以待,賊覺而奔,追至喀爾鐵蓋山,擊斬殆盡。張格爾僅餘三十人,棄騎登山,副將胡超、都司段永福等擒之。八年正月,捷聞,上大悅,錫封長齡二等威勇公,世襲罔替,賜寶石頂、四團龍補服、紫韁,授御前大臣。諸將封賞有差。五月,檻送張格爾於京師,上御午門受俘,磔於市。晉長齡太保,賜三眼花翎,圖形紫光閣。尋回京,命親王大臣迎勞,行抱見禮於勤政殿。授閱兵大臣,管理籓院及戶部三庫,正大光明殿賜凱宴,賜銀幣,授領侍衛內大臣。恩禮優渥,並用乾隆朝故事,時稱盛焉。

十年秋,浩罕以內地安集延被驅逐,貲產皆鈔沒,積怨憤,遂挾張格爾之兄玉素普及其黨博巴克等復入邊,圍喀什噶爾、英吉沙爾二城,且犯葉爾羌。復命長齡為揚威將軍,往督師。會葉爾羌辦事大臣璧昌連破賊,長齡令參贊哈哴阿、提督胡超分路進援喀、英二城,賊聞風解圍遁出塞。於是偕伊犁將軍玉麟合疏陳善後事,略曰:「此次入寇,與張格爾不同,不過烏合夷眾,挾驅逐鈔沒之憾,虜掠取償,無志於土地人民。各白回畏賊騷掠,助順守御,亦非上年甘心從逆之比。此時戰緩而守急。惟兵未至而賊已先逃,兵久駐而賊無一獲,戰守俱無長策。諸臣條奏增兵廣屯,以省徵調,言之似易,行之實難,即收效亦在數十年之後。若仿土司以西四城付阿奇木伯克,回性懦弱,非浩罕敵;茍無官兵守御,賊至必如入無人之境。臣等再四籌商,統兵之人宜立不敗之地,斯能制人而不為人制,惟有移參贊大臣於葉爾羌,其地本回疆都會,距喀什噶爾六站,在不遠不近之間。再移和闐領隊大臣備調遣。喀什噶爾留換防總兵一,與英吉沙爾領隊為犄角。巴爾楚克駐守總兵一,為樹窩子咽喉鎖鑰。六城相距均不過數百里。於西四城額兵六千之外,留伊犁騎兵三千,陜甘綠營兵四千,量分駐守,而以重兵隨參贊居中調度。新兵糧餉,請於各省綠營兵額內裁百分之二,歲省三十餘萬,以為回疆兵餉。俟屯田有效,即以回疆兵食守回疆,仍撤回內地餉額。」又疏請招民開墾西四城閒地以供兵糈。又請添設同知二、巡檢五,由陜、甘選勤能之員任之。並下廷議,往復再三,罷設文員,減滿、漢兵二千五百名,新增餉需不過十萬兩,各城額徵糧科可敷供支,乃允行。以璧昌為參贊大臣,各城聽節制。其辦事、領隊各大臣,命長齡等保奏任用。

浩罕懼大軍出討,乞援俄羅斯,俄人拒之,乃遣頭人詣軍求通商。長齡責縛獻賊目,釋還兵民,來報原還俘虜,復乞免稅,並給還所沒貲財。上方欲示以寬大,且謂獻犯亦不足信,一切允之。浩罕喜過望,進表納貢通商如故,邊境乃安。

長齡駐回疆凡兩載,十二年,回京,晉太傅,管理兵部,調戶部,賜四開衣契袍。十七年,以病乞休,上親視其疾,溫詔慰留。以八十壽,晉一等公爵。次年,卒,上震悼,親奠,賜金治喪,入祀賢良祠、伊犁名宦祠,謚文襄。十九年,命每次謁陵後,賜奠其墓。子桂輪,襲公爵,官至烏里雅蘇臺、杭州將軍,謚恪慎。孫麟興,襲爵,亦官烏里雅蘇臺將軍。

那彥成,字繹堂,章佳氏,滿洲正白旗人,大學士阿桂孫。乾隆五十四年進士,選庶吉士,授編修,直南書房。四遷為內閣學士。嘉慶三年,命在軍機大臣上行走。遷工部侍郎,調戶部,兼翰林院掌院學士。擢工部尚書,兼都統、內務府大臣。那彥成三歲而孤,母那拉氏,守志,撫之成立,至是三十載,仁宗御書「勵節教忠」額表其門。

時教匪張漢潮久擾陜西,參贊大臣明亮及將軍慶成、巡撫永保同剿之,互有隙,師行不相顧。是年秋,命那彥成為欽差大臣,督明亮軍,褫慶成、永保職,逮治。那彥成以樞臣出膺軍寄,意銳甚。明亮聞其將至,急擊賊敗之,漢潮伏誅。帝嘉其先聲奪人,特詔褒美。漢潮黨冉學勝亦狡悍,猶在陜。冬,敗之五郎。竄秦嶺老林,又迭敗之高關峪、夾嶺、鳳皇山。賊乘間逸入湖北、河南境。五年春,進兵漢中,遂入棧剿川匪,追出棧,大破之隴州隴山鎮,俘斬甚眾,授參贊大臣。會經略額勒登保病,上以那彥成隴山捷後,軍威已振,命兼督各路兵。高天升、馬學禮陷文縣,踞卡郎寨。乘夜渡河破之,賊南竄,趨松潘、岷州。額勒登保病起,合擊敗之,餘賊將竄川境,即陰平入蜀道也。那彥成以地險不利騎兵,檄總兵百祥迎擊於農安,自率師回陜。初,那彥成西行,以南山餘賊付巡撫臺布。繼而川賊五家營至與合,欲東犯,臺布遣將扼之。賊趨鎮安,張世龍、張天倫為經略大兵所驅,亦奔鎮安,群賊皆注漢北山內。額勒登保追入老林,賊向商,雒,為楊遇春所破,始不敢東。那彥成與會師鎮安。商、雒賊折犯楚境。上以軍事不得要領,召回京面詢方略,而高、馬二賊入川後益張,總兵施縉戰歿,詔斥那彥成縱賊,罷軍機、書房一切差使。及至,召對,忤旨,再斥在陜漫無布置,面詢兵事餉事,惟諉諸劫數未盡,且有忌額勒登保戰功意,褫尚書、講官、花翎,降翰林院侍講。歷少詹事、內閣學士。

七年,赴江西按巡撫張誠基被劾事,未定讞,兩廣總督吉慶以剿會匪被譴自戕,命往鞫。八年,率提督孫全謀平會匪,條上善後,署吏部侍郎。擢禮部尚書。九年,復授軍機大臣,赴河南鞫獄,未畢,命署陜甘總督,治搜捕餘匪善後事宜,手詔戒之曰:「汝誠柱石之臣,有為有守。惟自恃聰明,不求謀議,務資兼聽並觀之益,勿存五日京兆之見。」未幾,調授兩廣總督。廣東土匪勾結海寇為患,久不靖。那彥成以兵不足用,乃招撫盜首黃正嵩、李崇玉,先後降者五千餘人,獎以千總外委銜及銀幣有差。巡撫孫玉庭劾其賞盜,降藍翎侍衛,充伊犁領隊大臣。既而李崇玉檻送京師,訊得與正嵩皆受四品銜守備劄,褫職戍伊衛。十二年,復予二等侍衛,充領隊,調喀喇沙爾辦事大臣,又調西寧,平叛番,擢南河副總河。以荷花塘漫口合而復決,降二等侍衛。歷喀喇沙爾、葉爾羌辦事大臣,喀什噶爾參贊大臣。十四年,復授陜甘總督。

十八年,河南天理會教匪李文成等倡亂,陷滑縣,直隸、山東皆響應,林清糾黨犯禁門。初,命總督溫承惠往剿,清既誅,乃發京兵,授那彥成欽差大臣,加都統銜,督師率楊遇春、楊芳等討之,迭詔責戰甚急。那彥成以小醜不足平,惟慮遁入太行,勢且蔓延,十月,至衛輝,合師而後進。賊踞桃源集、道口,與滑縣為犄角,連敗之於新鎮、丁欒集。遇春擊破道口,殲賊萬餘,焚其巢;尋破桃源集,追道口餘賊,抵滑縣。文成遁輝縣司寨,楊芳、德英阿追破之,文成自焚死。親督遇春等圍滑城數旬,以地雷攻拔之,獲首虜二萬餘。山東賊亦平。捷聞,加太子少保,封三等子爵,賜雙眼花翎,授直隸總督,賜祭其祖阿桂墓。

二十一年,坐前在陜甘移賑銀津貼腳價,褫職逮問,論大闢;繳完賠銀,改戍伊犁。會丁母憂,詔援滑縣功,免發遣。二十三年,授翰林院侍講。歷理籓院、吏部、刑部尚書,授內大臣。道光二年,青海野番甫定復擾,命那彥成往按,遂授陜甘總督。驅私住河北番族回河南原牧,嚴定約束,緝治漢奸,乃漸平。五年,調直隸。七年,回疆四城既復,命為欽差大臣,往治善後事。先後奏定章程,革各城積弊。諸領隊、辦事大臣歲終受考覈於參贊大臣,又總考覈於伊犁將軍,互相糾察;增其廉俸,許其攜眷,久其任期。印房章京由京揀選,不用駐防。除伯克賄補之弊,嚴制資格,保舉回避。五城叛產歸官收租,歲糧五萬六千餘石,支兵餉外,餘萬八千石為酌增各官養廉鹽米銀之用,有餘則變價解阿克蘇採買儲倉。改建城垣,增卡堡,練戍兵。浩罕為逋逃藪,所屬八城,安集延即其一。嚴禁茶葉、大黃出卡。盡逐內地流夷,收撫各布魯特,待其款關求貢,然後撫之。詔悉允行。張格爾既誅,加太子太保,賜紫韁、雙眼花翎,繪像紫光閣,列功臣之末。

浩罕匿張格爾妻孥,詐使人投書伺隙。那彥成禁不使與內地交接,絕其貿易。九年,使人出卡搜求逆屬,上慮其邀功生事,召還京,仍回直隸總督任。未及兩歲,西陲復不靖。論者謂那彥成驅內地安集延,沒貲產、絕貿易所致。十一年,詔斥誤國肇釁,褫職。十三年,卒,宣宗追念平教匪功,賜尚書銜,依例賜恤,謚文毅。

那彥成遇事有為,工文翰,好士,雖屢起屢躓,中外想望風採。子容安、容照。

容安,廕戶部主事,襲子爵。歷侍衛、副都統。從長齡徵回疆有功,歷伊犁參贊大臣。亂事再起,容安率兵四千五百赴援,抵阿克蘇,遷延不進。由和闐繞道,又分兵烏什,致喀、英二城圍久不解。褫職逮治,讞大闢。尋以二城未失,從寬改監候,罰繳和闐軍需,貸死戍吉林。父喪,釋還。數年卒。

容照,以大臣子予侍衛。累擢內閣學士。亦從征回疆,隨父治善後。擢理籓院侍郎。容安既獲罪,襲子爵。繼因那彥成被譴,同褫職。起,歷馬蘭鎮總兵。治獄失入,復褫爵職。以侍衛從揚威將軍奕經防廣東。充庫倫辦事大臣,復為馬蘭鎮總兵。咸豐中,從尚書恩華剿捻匪有功,加副都統銜。以疾回京,卒,賜血⼙。孫鄂素,襲爵。

玉麟,字子振,哈達納喇氏,滿洲正黃旗人。乾隆六十年進士,選庶吉士,授編修。嘉慶初,三遷為祭酒。歷詹事、內閣學士。纂修實錄久,特詔充總纂,奏事列名總裁後。入直上書房。歷禮部、吏部侍郎,典會試。奉使鞫安徽壽州獄,及湖北官銀匠侵虧錢糧事,大吏並被嚴譴。後歷赴湖南、江西、直隸、河南按事,時稱公正。十二年,督安徽學政,調江蘇。十六年,兼右翼總兵。坐吏部銓序有誤,奪職。未幾,授內閣學士,兼護軍統領、左翼總兵,遷戶部侍郎。十八年八月,車駕自熱河回蹕,迎至白澗,先還京。會林清逆黨犯禁門,率所部擊捕;坐門禁懈弛,褫職。十九年,予三等侍衛,赴葉爾羌辦事。二十二年,加副都統銜,充駐藏大臣。歷左翼總兵、鑲白旗漢軍副都統,遷左都御史,禮部、吏部、兵部尚書。

道光四年,命在軍機大臣上行走。六年,回疆亂起,西四城皆陷。阿克蘇辦事大臣長清獨能固守卻賊,先由玉麟論薦,詔特嘉之,賜花翎。七年,兼翰林院掌院學士,充上書房總師傅,加太子少保。八年,回疆既定,晉太子太保,繪像紫光閣。

上方廑顧西陲,以玉麟悉邊務,九年,特命出為伊犁將軍。疏言:「浩罕將作不靖,請緩南路換防。阿坦臺、汰劣克屢請投順,包藏禍心,添巡邊兵以備御。伊薩克忠勇能事,責令乘機謀之。近夷布呼等愛曼恭順,重賞以固其心,則卡外動靜俱悉。」詔如議行,並令喀什噶爾參贊大臣札隆阿為之備。札隆阿誤信汰劣克等,不之疑也。十年秋,安集延果引浩罕內犯,喀什噶爾幫辦大臣塔斯哈率兵出御,遇伏陷歿。札隆阿將棄城退守阿克蘇,玉麟急疏聞,請責長清等速籌糧儲,哈豐阿速進攻,發伊犁兵四千五百名,令容安率之赴援。容安至阿克蘇,與長清議,中途有朵蘭回子梗阻,令哈豐阿、孝順岱由和闐草地進兵。玉麟疏劾曰:「喀、英兩城被困兩月,賊勢尚單,易於援剿,由大路直赴葉爾羌,二城之圍自解。迂道和闐,須一月方至,賊勢漸厚,哈豐阿軍未必得力。阿克蘇現集兵不下萬人,僅以三千人繞路進發,留兵坐糜餉糧,實屬非計。札催十數次,該大臣等始以糧運遷延,後又稱蒙兵、民遣皆不足恃。計程裹糧二十日足用,後路轉運已源源而來。前年克復四城,民遣得力,渾巴什河之捷,土爾扈特出力較多。近日璧昌以少勝眾,豈沿邊零匪轉不能就地殲除?請將長清等嚴行申飭。」上韙其言,仍促哈豐阿進兵。及長齡督楊芳、胡超等大兵至喀、英二城,賊已遠遁。玉麟疏言:「賊勢渙散,現調官兵不止四萬,月需糧萬五千石,運費十餘萬兩。請停止續調四川、陜、甘兵,並飭回疆各城採買糧餉,較之戈壁轉輸,節省不止倍蓰。」從之。

初張格爾之就擒也,回子郡王銜貝子伊薩克實誘致,諸夷忌之,亂起,兵民謀劫掠,事洩,誅首犯,逐流民。怨者譌言伊薩克通賊,遂圍劫其家,並殺避亂回眾二百餘人。札隆阿不能制,反附和劾囚之。玉麟以伊薩克身膺王封,助亂得不償失,子孫在阿克蘇,家業在庫車,豈無顧慮?疏陳其可疑,命偕長齡會鞫,得札隆阿懼罪欲殺之以掩跡,及委員章京等捏奏迎合誣證狀,札隆阿以下坐罪有差,復伊薩克爵職,回眾大服。

時諸臣議回疆事宜,玉麟上疏曰:「閱固原提督楊芳添兵招佃奏稿,稱四川總督鄂山有請西四城改照土司之議。伏思回疆自入版圖,設官駐兵,不惟西四城為東道籓籬,南八城為西陲保障,即前後藏及西北沿邊蒙古、番子部落,皆賴以鞏固。若西四城不設官兵,僅令回人守土,誠恐回性無恆,又最畏布魯特強橫,轉瞬即為外夷所有,則阿克蘇又將為極邊矣。其迤東之庫車、喀喇沙爾、吐魯番、哈密等城,必至漸不安堵。以形勢論,脣亡則齒寒;以地利論,喀什噶爾、葉爾羌、和闐三處為回疆殷實之區。舍沃壤而守瘠土,是藉寇兵而齎盜糧也。楊芳所謂守善於棄,實不易之論。至請將喀什噶爾參贊移遷阿克蘇,殊非善計。該處幅員狹隘,不足為重鎮。且距喀城二千里,有鞭長不及之患。其所陳招佃通商各條,則為治邊良法,請用之。」於是詔發長齡密陳十條及中外奏議,交玉麟悉心籌畫。十一年,偕長齡會疏,上定以參贊大臣移駐葉爾羌,暨善後諸政,具詳長齡傳。十二年,事定,回伊犁,調劑番戍官兵以均勞逸。惠遠城南瀕河,定歲修之例;以待種之地租給回民,收租充兵食,並為贍孤寡備差操諸用。拓敬業官學學舍,創建文廟。宣宗特頒扁額以重其事,邊徼士風漸蒸蒸焉。十三年,命回京,以特依順保代之。行至陜西,卒於途次。上聞震悼,優詔賜恤,贈太保,入祀賢良祠。柩至京,親臨賜奠,謚文恭。伊犁請祠祀,允之。

特依順保,鈕祜祿氏,滿洲正白旗人。由吉林前鋒長從征廓爾喀,有功。嘉慶中,從長齡剿教匪,屢破高天升、馬學禮,賜號安成額巴圖魯。累擢甘肅西寧鎮總兵。十八年,從那彥成討滑縣教匪,力戰,數破賊,克司寨,殲首逆李文成,克滑縣,執賊渠,予雲騎尉世職。移剿陜西三才峽匪。事平,擢黑龍江將軍。調烏里雅蘇臺將軍、塔爾巴哈臺參贊大臣、葉爾羌辦事大臣。召授正白旗蒙古都統。張格爾之亂,命赴阿克蘇。尋署甘肅提督,兼西寧辦事大臣。歷綏遠城、黑龍江、寧夏、西安將軍。調伊犁,承玉麟之後,休息邊氓,撫馭夷部。巴爾楚克諸地屯田漸興,酌撤防兵。在任五年,邊疆無事。道光十八年,入覲,詔嘉其治邊措施悉當,加太子太保,授內大臣,留京供職。尋授領侍衛內大臣。二十年,病,請解職。未幾,卒,賜恤如例。

論曰:回疆之役,削平易而善後難。長齡持重於始,老成之謀。那彥成力袪積弊,善矣,而操切肇釁,未竟厥功。玉麟以樞臣自請治邊,補救綢繆,西陲乃得乂安無事。紫閣銘勛,蓋非幸已。


\end{pinyinscope}