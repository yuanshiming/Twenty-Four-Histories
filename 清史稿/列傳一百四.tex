\article{列傳一百四}

\begin{pinyinscope}
王無黨吳進義譚行義李勛樊廷武進升馬負書

範毓皛

王無黨,直隸萬全人。康熙五十一年武進士,授藍翎侍衛。累遷廣西梧州協副將。貴州臺拱九股苗為亂,無黨率師討定之,擢左江鎮總兵。九股苗復為亂,無黨馳抵古州,分兵赴八寨督剿。經略張廣泗檄無黨分攻臺拱大臺雄,克之。平交上等三十餘寨,擒其渠巴利,會收牛皮大箐。乾隆元年,署貴州提督。從廣泗撫定上下九股諸苗從為亂者。二年,真除。疏陳黔省急務,請籌積貯,築城垣,整墩臺塘房,禁掠賣人口,下部議行。定番州屬姑盧寨苗視險強肆,廣泗與無黨遣漢、土官兵三千餘,分道毀寨搜箐,擒其渠老排,十餘日而定,上褒其妥協。四年,陛見,賜孔雀翎。

六年,移湖廣提督。黑峒苗為亂,大學士鄂爾泰以無黨在貴州久,熟苗事,留使戡定乃上官。八年,上以湖廣軍政廢弛,無黨至官未有所整理,下詔詰責。十三年,坐提標兵救火攘衣物,兵部論無黨徇庇,當奪官,命詣京師引見,左授湖南沅州協副將。遷雲南楚姚鎮總兵。內擢鑾儀使。復外授福建漳州鎮總兵。遷浙江提督。以目疾乞罷。卒,謚壯愨。

吳進義,字子恆,陜西寧朔人。父開圻,康熙二十七年一甲三名武進士,官至雲南元江副將。進義入伍,從振武將軍孫思克征噶爾丹,劄署守備,發江南借補千總。累遷江南壽春鎮總兵。擢江南提督,疏言:「太湖界江、浙,漁船奸良難辨。請照海洋例巡哨,支河小汛,飭兩省陸路兵巡查,則聲勢聯絡,奸宄斂跡。」有旨嘉獎。久之,移浙江,再移福建,復還浙江。時有偽為孫嘉淦疏稿語訐上,進義與浙閩總督喀爾吉善以聞。上令究所從來,語連提督廨胥吏,喀爾吉善劾進義隱諱,命解官聽讞。進義力辨未嘗隱諱,其幕客證進義已見稿。浙江巡撫雅爾哈善論進義當重闢,上愍其老,命貰罪。復以疏稿未得作偽主名,令江蘇巡撫莊有恭會鞫。有恭疏陳進義實未見稿,浙江承審諸吏牽合附會。事下軍機大臣覆訊,得實。上以進義無辜廢斥,召來京,命以提督銜署直隸宣化鎮總兵。未幾,授古北口提督。進義請限操演火藥,增設河屯協弓兵,皆允行。二十三年,加太子少保。二十七年,卒,年八十四,加太子太保,謚壯愨。

進義家世多武功,從祖坤,貴州永北總兵,嘗徵四川苗及金川有功。坤子開增,自武舉官至浙江溫州總兵。

譚行義,四川三臺人。康熙時,以武舉授陜西西寧衛千總。雍正初,從軍平青海,再遷河南城守營參將。河東總督田文鏡劾行義送陜西軍馬疲瘦,奪官,上令來京引見,召對稱旨,賜編刻上諭、貂皮、香珠,復原官。再遷廣東高雷廉總兵。總督鄂彌達檄行義將五千人協剿貴州亂苗,進擊滾縱、高表諸寨。經略張廣泗令赴援上江,攻烏婆、擺吊諸險要地,搜牛皮大箐,獲其魁。歷福建漳州、湖南鎮筸諸鎮。

乾隆四年,授廣西提督,帥師會討楚、粵亂苗。宜山縣土蠻恃險劫掠,行義與總督馬爾泰、署巡撫安圖令游擊楊剛討之。破白土、丘索二村,執其渠,斬以徇。忻城土縣外八堡有劇盜曰藍明星,恃險焚劫。行義檄副將畢映捕治,明星遁入山,搜捕得之。有黃順者,匿湖北、廣東錯壤處,謀為亂。貴州人黎阿蘭與相應,散旗印,將起事。行義詗知之,督兵攻克賊巢,擒斬首從七十餘,事乃定。柳州兵皆居草舍,患火。行義請發白金四千貸兵建瓦屋,分三年還帑,從之。又有李彩者,糾眾聚遷江石版村謀犯縣城,行義既捕治,請城北設汛。尋以擅發倉穀貸於兵,左授登州鎮總兵。十一年,遷江南提督。十四年,移浙江提督。十六年,再移福建陸路提督。十八年,卒,謚恭愨。

李勛,貴州鎮遠人。入伍,稍遷守備。從征臺拱九股生苗,廣泗檄同剿羊吊、洞里、羊色諸地,搜牛皮大箐,勛亦在行間。累遷湖廣提督。緬甸亂,移雲南提督。疏請自普洱馳往孟艮捕治亂渠召散,上以其老,不勝瘴癘,命還普洱。勛已至孟艮,督總兵劉德成、華封等葺堡寨,防要隘,得召散兄猛養等。勛還,卒於途。加太子太保,謚莊毅。

樊廷,陜西武威人。初入伍,更姓名王剛。從征烏蒙、青海、西藏,積功累遷甘肅肅州鎮總兵。自陳復姓名,改籍四川潼川縣。準噶爾犯科舍圖卡倫,盜駝馬,其眾二萬餘。廷率副將冶大雄等將二千人御之,轉戰七晝夜,與總兵張元佐等軍合,殺賊無算,盡還所盜。時提督紀成斌護寧遠大將軍印,聞上,詔褒廷以寡敵眾,忠勇冠軍,賜白金萬,一等輕車都尉世職。授陜西固原提督、都督僉事。入覲,請從軍,命從署寧遠大將軍查郎阿出師屯南山。副將軍張廣泗偵賊伏烏爾圖水,檄廷將千五百人自鹼泉子進剿,至哈洮遇賊,奪據山梁,連敗之。越噶順抵鄂隆吉大阪,殺賊四百,擒三十六,收其糧械。

乾隆初,上從查郎阿請,發甘、涼諸鎮兵五千人駐哈密,置總統提督,以授廷。廷至軍,疏言:「烏爾克為極西第一要隘,兵出偵洮賴大阪北蘆草溝、噶順溝東亂山子及烏爾圖水,夜輒有火光。守隘兵寡,請量增。」又疏言:「哈密兵在山南煙墩溝諸地牧駝馬,請分山北防兵巡護。」皆用其議。在邊二年,以病乞罷,命還固原治疾,遣醫往診。尋卒於哈密。遺疏論防邊事甚切,上深愍之。命查郎阿經紀其喪,歸葬涼州。贈都督同知,謚勇毅。

子經文,官至廣東右翼總兵。經文子繼祖,官湖北副將。繼祖子從典,請改籍湖北恩施。從典子燮,官湖南永州鎮總兵,同治中,坐事罷。

武進升,山西寧鄉人,其後改籍江南江寧。初以張姓入伍。稍遷浙江溫州鎮標守備。雍正初,閩浙總督滿保疏薦,引見,授三等侍衛,屬怡親王允祥。尋外授江寧游擊。累遷福建陸路提督。言:「閩省不習騎射,加意督率,弓力漸增。馬兵出馬收馬較前改觀。」高宗諭以「如此方不負任使,然亦不可欲速,尤貴為之以實,要之以久」。進升與總督喀爾吉善忤,疏言:「喀爾吉善外似和平,心實剛愎。令臣密察水師提督張天駿營伍,臣辭以水師非所轄。督臣正言厲色,必令臣密察。及察知水師陋規,告之督臣,督臣置不問,反與天駿契合。臣察漳州營馬值,總兵馬負書為督臣舊部,巧為徇私。令臣無地自容。」又疏言喀爾吉善衰憊狀,上斥進升支離狂率。喀爾吉善亦劾進升徇所屬,縱兵行竊。因左授江南狼山鎮總兵,進升疏謝,諭曰:「汝無他過,祗好勝多事,故左授示薄懲。若不知改,或遂委靡,一切姑息,皆不可也。」居數月,擢江南提督,以老罷。再起,終浙江提督。卒,年八十餘,謚良毅。

馬負書,漢軍鑲黃旗人。乾隆元年一甲一名武進士,授頭等侍衛。累遷福建漳州鎮總兵。疏言:「漳州民好鬥,有所謂『闖棍』,結黨肆行,土豪養為牙爪,請嚴治之。」上下其章喀爾吉善,令體察懲治。歷瓊州、金門、臺灣、狼山諸鎮。署古北口提督,疏言:「兵習陣法,無濟實用。應於秋冬收穫後,擇地成列,為仰攻旁擊勢。分合進退,以金鼓為節。常月教場演習,仍依營制。」得旨允行。授福建陸路提督。卒,謚昭毅。

範毓皛,山西介休人。範氏故巨富。康熙中,師征準噶爾,輸米餽軍,率以百二十金致一石。六十年,再出師,毓皛兄毓馪請以家財轉餉,受運值視官運三之一。雍正間,師出西北二路,怡親王允祥薦毓馪主餉,計穀多寡,程道路遠近,以次受值,凡石米自十一兩五錢至二十五兩有差,累年運米百餘萬石。世宗特賜太僕寺卿銜,章服同二品。寇犯北路,失米十三萬餘石,毓馪斥私財補運,凡白金百四十四萬。師既罷,米轉運近地,戶部按近值核銷,故所受遠值,責毓馪追繳,凡白金二百六十二萬,復出私財採蓡,市銅供鑄錢以償。

毓皛以武舉授衛千總,以駝佐軍,擢守備。累遷直隸天津鎮總兵。自河南河北鎮移廣東潮州,疏請令潮州營兵如河北例,兼習長槍、短棍、連接棍諸藝。世宗命與總督鄂彌達、提督張溥商榷。鄂彌達等上言:「廣東山海交錯,軍械惟鳥槍最宜,次則弓箭、藤牌、挑刀、大砲。毓皛所議與廣東不甚宜。」上韙鄂彌達等議,仍諭毓皛初至,當嘉其肯言。嘉應、潮陽遇颶,海岸決。毓皛以聞,命加意撫綏。乾隆初,署廣東提督。故事,市舶至,詣海關納稅。或遇風未至所往地,中道暫泊,亦論稅如例。毓皛慮民避屢稅,遇風不敢泊,致傾覆,疏請商舟寄泊,非即地市易不徵稅,上命待審察。毓皛以憂歸,服終,授直隸正定鎮總兵。湖廣總督阿爾賽請移任苗疆,上不允,諭以「毓皛富家子弟,謹慎無過。苗疆事重,不能勝也」。上巡五臺,毓皛言兄毓馪子清注具羊千、馬十備賞賚,上卻之。尋以老罷。卒。

論曰:提鎮雖專閫,然受制於督撫,所轄兵散處諸營汛,都試肄武,虛存其制耳。無黨、進義皆能勤其官者,行義捕盜,廷屢從戰,皆有勞。進升齗齗不欲曠其職。毓皛與其兄出私財助軍興,幾傾其家而不悔,求諸往史,所未有也。

列傳一百四

王無黨吳進義譚行義李勛樊廷武進升馬負書

範毓皛

王無黨,直隸萬全人。康熙五十一年武進士,授藍翎侍衛。累遷廣西梧州協副將。貴州臺拱九股苗為亂,無黨率師討定之,擢左江鎮總兵。九股苗復為亂,無黨馳抵古州,分兵赴八寨督剿。經略張廣泗檄無黨分攻臺拱大臺雄,克之。平交上等三十餘寨,擒其渠巴利,會收牛皮大箐。乾隆元年,署貴州提督。從廣泗撫定上下九股諸苗從為亂者。二年,真除。疏陳黔省急務,請籌積貯,築城垣,整墩臺塘房,禁掠賣人口,下部議行。定番州屬姑盧寨苗視險強肆,廣泗與無黨遣漢、土官兵三千餘,分道毀寨搜箐,擒其渠老排,十餘日而定,上褒其妥協。四年,陛見,賜孔雀翎。

六年,移湖廣提督。黑峒苗為亂,大學士鄂爾泰以無黨在貴州久,熟苗事,留使戡定乃上官。八年,上以湖廣軍政廢弛,無黨至官未有所整理,下詔詰責。十三年,坐提標兵救火攘衣物,兵部論無黨徇庇,當奪官,命詣京師引見,左授湖南沅州協副將。遷雲南楚姚鎮總兵。內擢鑾儀使。復外授福建漳州鎮總兵。遷浙江提督。以目疾乞罷。卒,謚壯愨。

吳進義,字子恆,陜西寧朔人。父開圻,康熙二十七年一甲三名武進士,官至雲南元江副將。進義入伍,從振武將軍孫思克征噶爾丹,劄署守備,發江南借補千總。累遷江南壽春鎮總兵。擢江南提督,疏言:「太湖界江、浙,漁船奸良難辨。請照海洋例巡哨,支河小汛,飭兩省陸路兵巡查,則聲勢聯絡,奸宄斂跡。」有旨嘉獎。久之,移浙江,再移福建,復還浙江。時有偽為孫嘉淦疏稿語訐上,進義與浙閩總督喀爾吉善以聞。上令究所從來,語連提督廨胥吏,喀爾吉善劾進義隱諱,命解官聽讞。進義力辨未嘗隱諱,其幕客證進義已見稿。浙江巡撫雅爾哈善論進義當重闢,上愍其老,命貰罪。復以疏稿未得作偽主名,令江蘇巡撫莊有恭會鞫。有恭疏陳進義實未見稿,浙江承審諸吏牽合附會。事下軍機大臣覆訊,得實。上以進義無辜廢斥,召來京,命以提督銜署直隸宣化鎮總兵。未幾,授古北口提督。進義請限操演火藥,增設河屯協弓兵,皆允行。二十三年,加太子少保。二十七年,卒,年八十四,加太子太保,謚壯愨。

進義家世多武功,從祖坤,貴州永北總兵,嘗徵四川苗及金川有功。坤子開增,自武舉官至浙江溫州總兵。

譚行義,四川三臺人。康熙時,以武舉授陜西西寧衛千總。雍正初,從軍平青海,再遷河南城守營參將。河東總督田文鏡劾行義送陜西軍馬疲瘦,奪官,上令來京引見,召對稱旨,賜編刻上諭、貂皮、香珠,復原官。再遷廣東高雷廉總兵。總督鄂彌達檄行義將五千人協剿貴州亂苗,進擊滾縱、高表諸寨。經略張廣泗令赴援上江,攻烏婆、擺吊諸險要地,搜牛皮大箐,獲其魁。歷福建漳州、湖南鎮筸諸鎮。

乾隆四年,授廣西提督,帥師會討楚、粵亂苗。宜山縣土蠻恃險劫掠,行義與總督馬爾泰、署巡撫安圖令游擊楊剛討之。破白土、丘索二村,執其渠,斬以徇。忻城土縣外八堡有劇盜曰藍明星,恃險焚劫。行義檄副將畢映捕治,明星遁入山,搜捕得之。有黃順者,匿湖北、廣東錯壤處,謀為亂。貴州人黎阿蘭與相應,散旗印,將起事。行義詗知之,督兵攻克賊巢,擒斬首從七十餘,事乃定。柳州兵皆居草舍,患火。行義請發白金四千貸兵建瓦屋,分三年還帑,從之。又有李彩者,糾眾聚遷江石版村謀犯縣城,行義既捕治,請城北設汛。尋以擅發倉穀貸於兵,左授登州鎮總兵。十一年,遷江南提督。十四年,移浙江提督。十六年,再移福建陸路提督。十八年,卒,謚恭愨。

李勛,貴州鎮遠人。入伍,稍遷守備。從征臺拱九股生苗,廣泗檄同剿羊吊、洞里、羊色諸地,搜牛皮大箐,勛亦在行間。累遷湖廣提督。緬甸亂,移雲南提督。疏請自普洱馳往孟艮捕治亂渠召散,上以其老,不勝瘴癘,命還普洱。勛已至孟艮,督總兵劉德成、華封等葺堡寨,防要隘,得召散兄猛養等。勛還,卒於途。加太子太保,謚莊毅。

樊廷,陜西武威人。初入伍,更姓名王剛。從征烏蒙、青海、西藏,積功累遷甘肅肅州鎮總兵。自陳復姓名,改籍四川潼川縣。準噶爾犯科舍圖卡倫,盜駝馬,其眾二萬餘。廷率副將冶大雄等將二千人御之,轉戰七晝夜,與總兵張元佐等軍合,殺賊無算,盡還所盜。時提督紀成斌護寧遠大將軍印,聞上,詔褒廷以寡敵眾,忠勇冠軍,賜白金萬,一等輕車都尉世職。授陜西固原提督、都督僉事。入覲,請從軍,命從署寧遠大將軍查郎阿出師屯南山。副將軍張廣泗偵賊伏烏爾圖水,檄廷將千五百人自鹼泉子進剿,至哈洮遇賊,奪據山梁,連敗之。越噶順抵鄂隆吉大阪,殺賊四百,擒三十六,收其糧械。

乾隆初,上從查郎阿請,發甘、涼諸鎮兵五千人駐哈密,置總統提督,以授廷。廷至軍,疏言:「烏爾克為極西第一要隘,兵出偵洮賴大阪北蘆草溝、噶順溝東亂山子及烏爾圖水,夜輒有火光。守隘兵寡,請量增。」又疏言:「哈密兵在山南煙墩溝諸地牧駝馬,請分山北防兵巡護。」皆用其議。在邊二年,以病乞罷,命還固原治疾,遣醫往診。尋卒於哈密。遺疏論防邊事甚切,上深愍之。命查郎阿經紀其喪,歸葬涼州。贈都督同知,謚勇毅。

子經文,官至廣東右翼總兵。經文子繼祖,官湖北副將。繼祖子從典,請改籍湖北恩施。從典子燮,官湖南永州鎮總兵,同治中,坐事罷。

武進升,山西寧鄉人,其後改籍江南江寧。初以張姓入伍。稍遷浙江溫州鎮標守備。雍正初,閩浙總督滿保疏薦,引見,授三等侍衛,屬怡親王允祥。尋外授江寧游擊。累遷福建陸路提督。言:「閩省不習騎射,加意督率,弓力漸增。馬兵出馬收馬較前改觀。」高宗諭以「如此方不負任使,然亦不可欲速,尤貴為之以實,要之以久」。進升與總督喀爾吉善忤,疏言:「喀爾吉善外似和平,心實剛愎。令臣密察水師提督張天駿營伍,臣辭以水師非所轄。督臣正言厲色,必令臣密察。及察知水師陋規,告之督臣,督臣置不問,反與天駿契合。臣察漳州營馬值,總兵馬負書為督臣舊部,巧為徇私。令臣無地自容。」又疏言喀爾吉善衰憊狀,上斥進升支離狂率。喀爾吉善亦劾進升徇所屬,縱兵行竊。因左授江南狼山鎮總兵,進升疏謝,諭曰:「汝無他過,祗好勝多事,故左授示薄懲。若不知改,或遂委靡,一切姑息,皆不可也。」居數月,擢江南提督,以老罷。再起,終浙江提督。卒,年八十餘,謚良毅。

馬負書,漢軍鑲黃旗人。乾隆元年一甲一名武進士,授頭等侍衛。累遷福建漳州鎮總兵。疏言:「漳州民好鬥,有所謂『闖棍』,結黨肆行,土豪養為牙爪,請嚴治之。」上下其章喀爾吉善,令體察懲治。歷瓊州、金門、臺灣、狼山諸鎮。署古北口提督,疏言:「兵習陣法,無濟實用。應於秋冬收穫後,擇地成列,為仰攻旁擊勢。分合進退,以金鼓為節。常月教場演習,仍依營制。」得旨允行。授福建陸路提督。卒,謚昭毅。

範毓皛,山西介休人。範氏故巨富。康熙中,師征準噶爾,輸米餽軍,率以百二十金致一石。六十年,再出師,毓皛兄毓馪請以家財轉餉,受運值視官運三之一。雍正間,師出西北二路,怡親王允祥薦毓馪主餉,計穀多寡,程道路遠近,以次受值,凡石米自十一兩五錢至二十五兩有差,累年運米百餘萬石。世宗特賜太僕寺卿銜,章服同二品。寇犯北路,失米十三萬餘石,毓馪斥私財補運,凡白金百四十四萬。師既罷,米轉運近地,戶部按近值核銷,故所受遠值,責毓馪追繳,凡白金二百六十二萬,復出私財採蓡,市銅供鑄錢以償。

毓皛以武舉授衛千總,以駝佐軍,擢守備。累遷直隸天津鎮總兵。自河南河北鎮移廣東潮州,疏請令潮州營兵如河北例,兼習長槍、短棍、連接棍諸藝。世宗命與總督鄂彌達、提督張溥商榷。鄂彌達等上言:「廣東山海交錯,軍械惟鳥槍最宜,次則弓箭、藤牌、挑刀、大砲。毓皛所議與廣東不甚宜。」上韙鄂彌達等議,仍諭毓皛初至,當嘉其肯言。嘉應、潮陽遇颶,海岸決。毓皛以聞,命加意撫綏。乾隆初,署廣東提督。故事,市舶至,詣海關納稅。或遇風未至所往地,中道暫泊,亦論稅如例。毓皛慮民避屢稅,遇風不敢泊,致傾覆,疏請商舟寄泊,非即地市易不徵稅,上命待審察。毓皛以憂歸,服終,授直隸正定鎮總兵。湖廣總督阿爾賽請移任苗疆,上不允,諭以「毓皛富家子弟,謹慎無過。苗疆事重,不能勝也」。上巡五臺,毓皛言兄毓馪子清注具羊千、馬十備賞賚,上卻之。尋以老罷。卒。

論曰:提鎮雖專閫,然受制於督撫,所轄兵散處諸營汛,都試肄武,虛存其制耳。無黨、進義皆能勤其官者,行義捕盜,廷屢從戰,皆有勞。進升齗齗不欲曠其職。毓皛與其兄出私財助軍興,幾傾其家而不悔,求諸往史,所未有也。


\end{pinyinscope}