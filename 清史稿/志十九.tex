\article{志十九}

\begin{pinyinscope}
災異五

洪範曰:「土爰稼■。」不成則為咎徵。凡恆風霾、晦冥、花妖、蟲孽、牛禍、地震、山頹、雨毛、地生毛、年饑、黃眚、黃祥皆屬之於土。

順治二年七月,湖州大風拔木。三年二月,孝感大風拔木。五年六月,無為州大風,壞屋拔木。八月,海豐颶風,毀廬舍無算。六年正月,潞安飆風大作。五月,五河狂風晝夜不息,大木盡拔。八月,惠來颶風大作,四晝夜不息,毀官署民舍。七年二月,阜陽、襄陽、漳南大風,拔木覆屋。九年五月,東陽大風拔木。十年八月,澄海颶風大作,舟吹陸地,屋飛空中,官署民房盡毀,壓斃男婦不計其數,從來颶風未有如此甚者。十一年二月,太湖大風,毀城內牌坊。六月,全椒颶風大作,屋瓦皆飛。十二年六月,石門大風拔木。十三年五月,章丘大風拔木。十四年三月,平樂颶風大作,飛石拔木,民房多傾頹。六月,石門大風毀民居。十六年正月二十八日,嘉應州大風拔木。十七年五月,慶元颶風拔木。

康熙元年八月初三日,寧海颶風三晝夜,宜興大風雨拔木。二年,遂溪颶風拔木。三年四月,臨城大風傷人。七月,清河颶風壞廬舍無算;慈谿大風雨,大木盡拔。八月,嘉興颶風大作,拔木飛瓦。四年二月,江陰大風拔木。五月,東陽大風雷雨★至,拔木壞屋。七月,嘉應州大風拔木。八月,長樂大風拔木。五年五月,海陽颶風拔木。八月,澄海颶風傷稼。六年四月,信宜大風,墻垣皆頹。七年四月,東陽大風雨,壓倒民居七所,拔木無算。六月,太平大風拔木。七月,瑞安大風,毀城垣廬舍。八年四月,大冶大風拔木。六月,海寧大風拔木。九年春,崇陽大風拔木。五月,全椒大風拔木。六月,安縣大風拔木,三晝夜乃息。七月,武定大風拔木。

十年正月,平遠大風拔木。十一年七月,榆社大風殺稼;瓊州颶風大作,官署民房悉圮無存,毀城垣十五丈。九月,吳川颶風,壞城垣廬舍。十二年正月,海陽颶風,拔木壞屋。八月十六日,澄海颶風大作。十三年二月,桐廬大風拔木。十四年二月,武強大風殺稼。三月,玉田大風,揚沙拔木。六月,新城大風拔木。十五年四月至六月,澄海颶風屢作,壞屋拔木。十六年四月,宜城大風拔木。六月,東陽大風,屋瓦皆飛。十七年六月,武強大風拔木。十八年六月,惠州大風,壞文星塔頂。十九年秋,瓊州大風拔木。

二十一年三月,望江大風拔木。七月,信宜大風拔木。二十二年二月初五日,單縣大風,揚塵蔽天,■忽變幻五色。二十三年正月,清河大風拔木。二十四年三月,文登大風拔木。二十五年四月,漢中、定遠大風拔木。五月,西充、南充大風拔木。六月,岳陽大風拔木。二十六年三月,太湖大風拔木。六月,平樂、蒼梧大風拔木壞屋。七月,蘇州、昆山、武進大風傷禾。二十七年五月,昌樂、壽光大風拔木。六月,沁水大風拔木。二十八年五月,恩縣異風,損壞城樓,吹倒石坊。二十九年四月,鄖陽大風拔木。五月二十六日夜,六安狂風暴起,屋瓦皆飛,大木盡拔。八月,黃巖大風拔木。

三十年三月,寧陽大風拔木。四月,江浦大風,屋瓦皆飛。三十一年正月,蓬萊大風,拔木毀屋。二月,沛縣大風,拔木毀屋。五月,東昌、丘縣大風拔木。六月,高密大風拔木。三十二年六月,營山大風拔木,風過,草木如焚。三十三年十月十六日,鄒平怪風,吹倒城★六座。三十四年秋,長寧大風拔木。三十五年七月二十二日,青浦、澤州大風拔木。二十三日,桐鄉、石門、嘉興、湖州颶風大作,民居傾覆,壓傷人畜甚多。八月十一日,海州大風雨,民舍盡傾。三十七年四月,濟南大風拔木。七月,蘇州大風拔木。三十八年春,青州大風拔木。六月,南樂大風拔木。

四十一年八月,開平颶風,拔木倒墻。十月乙酉,東明大風拔木。四十二年五月,枝江大風拔木。六月,潮陽颶風傷稼。四十四年四月,崇陽大風拔木。五月,歷城、霑化、丘縣大風拔木。四十五年六月二十夜,什邡大風自東北來,飛瓦拔木。四十六年三月,鄒平、長山大風拔木。四十七年五月,惠民颶風大作,毀民舍。七月,臺州大風拔木。四十八年四月,太原大風毀牌坊。八月,定海大風雨,孔子廟及御書樓皆圮。四十九年三月,中★大風拔木。

五十年三月,祁州大風毀南城樓。五月,安丘、諸城大風拔木。五十一年四月,香山颶風拔木。八月,寨城、富川大風拔木。九月,北流大風拔木。五十二年三月,全州大風雨雹,屋瓦皆飛,大木盡拔。六月,潮陽大風壞北橋。五十三年五月,固安大風拔木。六月,順義大風,樹木盡拔。五十四年六月初一日,潮陽颶風拔木。五十五年閏三月朔,解州大風拔木。四月辛亥,靜寧州大風拔木。五十六年七月十九日,掖縣暴風雨一晝夜,大木盡拔。五十七年五月,澄海颶風拔木。六月,湯溪大風,拔巨木,壞廬舍。七月,日照、黃縣大風雨一晝夜,大木盡拔。五十八年五月,乾州大風拔木。六月,寶坻大風拔木。八月十九日夜,揭陽颶風大作,風中如燐火,樹木皆枯;澄海颶風大作,民房傾覆,壓倒男婦無算。五十九年正月,陽春颶風傷稼。六十年八月,澄海颶風大作,如燐火,毀城垣。六十一年四月,甘泉大風拔木。五月,慶雲大風拔木。十二月,欽州大風雨,吹塌城垣二十餘丈。

雍正元年四月,平鄉大風拔木。六月,岑溪大風拔木。冬,武寧大風拔木。二年二月,陽信、霑化大風,風中帶火。四年五月,高淳、當塗大風拔木。六月,濰縣大風雨,壞民舍十二家。五年七月,鎮海颶風大作,毀縣署大堂。九年二月,連州大風雨,拔木壞屋。六月,陽信大風拔木。十年七月,南匯大風拔木。八月,海陽大風拔木。十月,泰州大風拔木。十一年八月,沂州大風四晝夜。十二年七月,泰州大風,拔木壞屋。十三年八月,高淳大風三晝夜。

乾隆元年五月,翼城大風拔木。二年八月十五日,平陽大風。三年七月,鍾祥大風拔木。五年三月,通州大風拔木。六月,掖縣大風拔木。六年四月,平定、樂平、孟縣大風拔木。八年五月,光化大風拔木。十年三月,棲霞大風拔木。十一年七月十五日,高郵大風拔木。十二年七月,昆山、鹽城、清河、福山、棲霞、文登大風,拔木覆屋。十三年三月,鶴慶大風拔木。四月,清河大風雨,民舍傾圮無算。五月,泰州、通州大風拔木。十四年四月,池州大風拔木。六月,高邑大風拔木。十五年三月,武昌暴風起江中,覆舟無數。六月,武寧大風拔木。十六年七月,鶴慶大風。十七年五月十一日,長子縣王婆村大風雷,田禾如爇,屋瓦車輪有飛至數里外者。十八年六月,潮陽大風拔木。七月,雞澤大風拔木。十九年七月,陵川大風害稼。二十年三月,蘄州大風,壞民舍二百餘間,壓斃十餘人。五月,高平大風拔木。七月,昌樂大風拔木。二十二年六月,吳川颶風,拔木壞屋。七月,孟縣、樂平大風傷稼。二十三年六月二十九日,即墨大風,一夜,大木盡拔。二十四年八月,平定大風害稼。二十六年三月,潛山大風,拔木壞屋。二十七年三月十八日,潯州颶風毀城樓。七月,嘉善大風,拔木壞屋。二十八年二月,歙縣大風,拔木覆屋,壓斃人畜甚多。三十年三月,臨邑大風拔木。三十一年七月,黃縣大風拔木。三十二年三月,文登、榮成大風拔木。五月,濟寧州大風拔木。三十三年二月,安丘大風損麥。六月十八日,瓊州颶風大作,毀官署民房無算。三十四年五月,東平大風拔木。秋,嘉善大風,禾盡偃。三十五年六月,祁縣大風拔木。三十六年二月,文登、榮成大風拔木。三十七年八月十七日,慶雲夜起異風,拔木無算。三十八年秋,永年、薊州大風雨拔木,熟禾盡偃。三十九年二月,黃縣、文登、榮成大風連日,麥苗盡損。七月,滎陽大風拔木。四十一年,安丘大風蔽日,風內有火光。四十二年四月,山陽大風拔木。四十三年二月,光化大風拔木。四十四年五月,南宮烈風雷雨,樹木多拔。四十六年六月,金華、嘉善大風拔木。四十七年六月,新城大風拔木。四十八年二月,文登、榮成大風拔木。六月二十四日,吳川颶風大作,壞官署民房及城垣。四十九年二月,平陰大風拔木。五十年二月,永昌大風拔木。五十一年正月,文登、榮成大風拔木。五十七年七月壬戌,蘇州大風毀民舍。五十九年七月,桐鄉大風雨竟夜,拔去大成殿前柏二株;湖州、嘉善大風,拔木壞屋。六十年六月,石門大風拔木。

嘉慶元年八月朔,瑞安大風,傾覆民舍,壓斃男婦九十一人。二年六月,武進大風拔木。三年四月,宜城大風拔木。四年七月,文登大風拔木。五年四月,黃縣大風,拔木壞屋。六年二月初五日,滕縣大風,色黃,既而如墨。八年二月,黃縣大風,拔木壞屋。九年二月,文登大風損麥。十年六月,慶雲大風拔木。十二年二月十七日,肥城暴風,天色忽紅忽黑,一夜方止。八月,邢臺大風拔木。十六年六月十二日,靜海大風拔木,摧折運糧船桅無算。十七年二月,麗水大風拔木。二十二年六月,棗陽大風拔木。二十三年四月,臨榆大風拔木。六月,永嘉大風拔木。二十四年七月初八日,平谷有怪風兼雨自南來,房舍皆摧折,禾盡偃,其平如掃。二十五年七月,樂清大風拔木。

道光二年六月,金華大風壞屋。七月,蘄州大風,拔木壞民舍。四年十一月十二日,泰州大風拔木,兩晝夜不止。五年六月,羅田大風拔木。六年二月二十六日,黃縣大風拔木。五月,肅州烈風拔木。七月,黃巖大風,拔木折屋。八年五月二十六日,黃縣大風拔木,屋瓦皆飛。十二年夏,公安大風三晝夜,拔樹無算。十四年四月,臨朐大風傷禾。六月,黃巖大風拔木,民居多壞。十五年七月,蓬萊、黃縣、棲霞、招遠大風三日,大木盡拔。八月,曲陽大風害稼。十六年六月二十九日,灤州怪風,毀南城樓。十七年八月,昌平大風拔木。二十年六月十九日,滕縣大風自西北來,拔大木數百株。二十二年八月,潛江狂風大作,飛石拔木,壞民居無算。二十三年七月,寧海暴風傷禾。二十六年六月,青浦大風拔木。二十七年三月朔,蓬萊大風拔木。六月,日照大風拔木。二十八年六月壬戌,通州颶風大作,毀屋。七月十四日,永嘉大風兼雨連旬,毀孔子廟及縣署。十八日,縉雲大風拔木。十月,武昌大風起江中,覆舟,人多溺死。三十年春,灤州大風傷稼。

咸豐二年五月初五日,肅州大風,拔木千餘株。六月,霑化大風拔木。三年三月初三日,宜昌大風拔木,民舍折損無算,牛馬有吹去失所在者。五月,隨州大風拔木。七月,蓬萊、黃縣大風拔木。七年四月,清苑、望都大風拔木。六月,寧津大風傷禾稼。八年四月,華縣大風拔木。十年二月,昌平怪風傷人。六月,房縣大風拔木。十一年四月,西寧大風拔木。七月,襄陽大風拔木。

同治元年二月初七日,宜都大風拔木。三月戊申,惠民大風拔木。二年二月,枝江大風,覆舟無算。五月,寧津狂風拔木。三年五月,房縣大風拔木。六月,嘉興、桐鄉大風拔木。四年正月,宜城大風,覆屋拔木。六年五月,高淳大風拔木。七月,菏澤、曹縣大風拔木。九年三月,嘉興府大風毀屋。四月,柏鄉大風毀屋。十年三月,湖州狂風驟雨,拔木覆舟。十一年六月二十七日夜,日照大風雨,偃禾拔木。秋,唐山大風,拔木損禾。十二年五月初六日,固原大風,壞城中回回寺。十三年五月,安陸大風拔木,府學墻頹。

光緒元年六月,皋蘭、均州大風拔木。七月,日照、臨朐大風傷稼。二年六月,黃巖大風拔木。三年八月,菏澤大風拔木。四年四月,臨江大風,覆舟無算。五年五月,蘄州大風拔木。六月十四日,寧海、文登、海陽、榮成大風,拔木壞屋。二十四日,萊陽怪風突起,屋瓦皆飛,民房被揭去樑棟椽柱,不知所之,拔大樹無算。七年七月,永嘉大風拔木。八年三月,孝義大風拔木。九年三月初八日,安陸大風拔木。十一年五月,光化大風拔木。十二年六月,涇州大風拔木。十五年六月十三日,灤州大風,拔木壞屋。十六年八月十五日,固原大風拔木。二十二年五月,南樂大風拔木。二十三年八月,靖遠大風拔木。二十七年六月,金縣大風拔木。二十八年四月初四日,曲陽大風拔木。二十九年六月十七日,洮州大風拔木。三十年七月二十二日,東樂大風拔木。

順治元年七月,平原狂風晝晦。二年十月,全椒晝晦。五年九月,漢陽大風晝晦。六年四月,莊浪風霾殺禾。九月,府谷風霾晝晦。七年十月,東明晝晦。十二年春,樂亭風霾晝晦。十三年七月,高邑大風霾晝晦。八月,邢臺風霾。十四年二月,陽城黃霾蔽天,屋瓦皆飛。十六年四月朔,萬州晝晦。

康熙元年正月朔,長興晝晦。九月,昌黎風霾。二年正月,蘄州晝晦。四年正月朔,蕭縣晝晦。四月辛亥,臨邑晝晦。七年二月,咸寧晝晦。十二年七月,樂亭風霾。十三年二月,咸陽大風霾十餘日。三月,朝城晝晦。十四年三月二十六日,冀州起異風,自巳至戌,黃霾蔽天,屋瓦皆飛;懷安、西寧大風霾晝晦;玉田大風,揚沙拔木,陰霾竟日。十五年五月,貴州晝晦如夜。十六年春,清河風霾四十餘日。二十三年四月朔,朝城晝晦。七月望,壁山晝晦。二十四年正月二十三日,文安大風霾,晝晦如夜;武邑黑風晝晦。二十五年二月二十七日,鄆城黑風晝晦。二十七年四月朔,西寧、龍門、延安、文縣同日晝晦。二十九年三月十九日,廣宗風霾,紅、黃、黑、白互變。四月初五日,西寧晝晦。三十年三月初四日,寧陽大風晝晦。三十一年正月朔,廣宗晝晦,青州、沛縣、丘縣大風晝晦。二月朔,丘縣大風赤霾晝晦,廣宗晝晦。三十二年二月十七日,丘縣大風霾,空中望之如火。十八日,桐鄉大風霾。十九日,湖州大風霾。三十三年四月朔,保安州晝晦。三十四年四月,肅州晝晦。三十五年正月,靜樂晝晦。二月十八日,定陶黑風,觸器有光,行人不辨咫尺。三十六年三月朔,靖遠晝晦。三十七年四月,龍門晝晦。四十二年五月二十二日,鞏縣大風晝晦。四十四年五月十八日,利津、陽信晝晦。四十五年正月十二,商河狂風晝晦。四十七年六月二十五日,涼州晝晦如夜。四十八年六月,東昌大風霾蔽天。四十九年三月初七日,中★晝晦者四日。六月初二日,什邡晝晦。五十年五月壬子,諸城晝晦。五十一年二月癸亥,東平、東阿大風,色紅黑,自申至亥方止;陽穀黑風晝晦;郯城、莘縣大風霾。三月十六日,鉅鹿風霾如火,晝晦如夜。六月十日,恩縣赤霾蔽天,咫尺不辨人物。十一月二十一日,宿州晝晦。五十三年二月二十一日,井陘風霾蔽天,晝晦。五十五年五月,壽光、臨朐大風晝晦。五十七年五月二十二日,新樂大風晝晦。五十九年五月二十六日,青城大風晝晦。六月,太平大風霾。六十年夏,丘縣大風霾連日。六十一年七月,元氏、沁州大風霾。

雍正元年三月,青州風霾。四月初七日,獻縣風霾晝晦;恩縣夜起大風,飛石拔木,有頃黑霾如墨,良久復變為紅霾,乍明乍暗,逮曉方息;泰安大風霾晝晦。十一日,高密、高苑大風霾晝晦。十七日,邢臺、元氏大風霾拔木。八月初八日,掖縣大風霾晝晦。二年二月初六日,元氏大風霾。八年正月十一日,高苑大風霾晝晦。

乾隆二年二月初五日,濟寧、鉅野風霾晝晦。三年正月十四日,武寧晝晦。五年五月,高郵大風霾。八年三月,贛州晝晦。十年三月,蒲臺大風晝晦。十七年四月十八日,祁州、新樂風霾損禾。十九年三月朔,慶陽晝晦。二十四年秋,芮城大風霾。二十五年二月初十日,宜昌晝晦。五月朔,昌樂晝晦。二十九年五月二十八日,南陵晝晦。三十二年二月初二日,範縣晝晦。二十四日,南宮大風晝晦。三十三年二月,潞安大風晝晦。三十六年二月朔,太原大風晝晦。初二日,高邑大風霾晝晦。三十八年二月初八日,滕縣大風霾五色,晝晦。三十九年春,南宮多風霾。四十九年二月初二日,菏澤風霾晝晦。五十年二月二十五日,臨清晝晦。四月十八日,南宮、棗強大風霾晝晦。

嘉慶元年三月二十六日,宜城晝晦。二年四月十四日,灤州大風霾晝晦。三年二月二十九日,灤州、昌黎晝晦。十一年十一月,滕縣大風五色,晝晦。十二年二月,武強大風霾,色黃,復黑赤。三月十二日,東光大風霾。十五年正月十七日,臨邑、章丘、新城風霾晝晦。二十七日,滕縣晝晦;南樂大風霾,平地積沙二寸許。二十三年四月,清苑、定州、武強、無極、唐山、臨榆大風霾晝晦。二十四年四月朔酉刻,京師晝晦。

道光三年六月朔,棗陽晝晦。四年六月癸巳,沂水晝晦。六年二月二十二日,武強大風霾,晝晦如夜。二十四日,南宮大風霾晝晦凡三日,濟南風霾晝晦。十年三月二十八日,中★晝晦。十一年七月十八日,曹縣晝晦。十四年五月十二日,即墨大風霾。十六年正月朔,樂亭風霾。十七年二月甲子,灤州晝晦。十九年三月初六日,元氏大風霾。二十年六月,撫寧晝晦。二十二年六月朔,太平、黃巖、湖州晝晦。二十九年,雲夢自正月至五月晝晦凡五閱月。三十年正月,嵊縣風霾十餘日。

咸豐元年五月丙午,灤州大風晝晦。二年二月,蓬萊大風晝晦。三年三月十四日,靈州晝晦,翼日始明。五年四月,灤州狂風晝晦。六年四月,南樂晝晦。七年四月初二日,景寧大風晝晦。十一年四月初四日,曹縣紅霾晝晦。

同治元年二月二十六日,霑化風霾日曀。三月初三日,武強風霾晝晦。二年二月,崇陽大風晝晦。三年六月,菏澤大風晝晦。四年正月十四日,棗陽晝晦。五年正月二十八日,霑化大風霾。九年正月二十五日,霑化大風霾日曀。十一年七月,灤州大風霾。十三年四月,曹縣大風晝晦。

光緒三年八月十五日,菏澤、曹縣大風晝晦。十年五月十三日,興山晝晦。二十年二月二十七日,甘州大風晝晦。二十八年四月初四日,曲陽大風晝晦。三十一年,邢臺晝晦。

順治四年九月,新安桃李華。五年秋,太谷桃李華。六年冬,德平桃李華。七年九月,階州桃再華。十月,銅陵桃李華。十一年九月,廣平桃李華。十三年冬,湖州桃李華。十五年十月,寧鄉桃李華。十七年冬,唐山牡丹華。

康熙二年十月,通州桃李華。四年十二月,德清吉祥寺牡丹開數莖。六年冬,寧津桃李華。七年秋,肥鄉桃李華。八年十一月,西充桃李華。十年八月,唐山海棠華,高邑丁香華。十一年九月,湖州桃李華。十七年十月,陽高桃李華。二十六年八月,新城桃李海棠華。三十年冬,潛江桃李華。三十六年七月,孟縣、平定桃李華。三十八年三月,石阡府學宮桂再華。四十三年冬,曲沃桃杏華。四十六年十月,瓊州海棠再華。五十四年冬,蒲臺李華。五十六年十月,寧津牡丹華。六十年冬,揭陽桃李華。

雍正三年冬,順德桃李華。八年冬,通州桃李華。九年冬,高淳桃李華。十年八月,通州桃李華。十三年七月,清河李再華。

乾隆三年秋,曲沃桃李華。七年冬至日,崇明牡丹開。九年冬,桐鄉桃李華。十年八月,寧津桃李華。十三年五月,玉屏梅花盛開。十四年八月,鎮海杏再華。十六年九月,分宜高林寺牡丹開。十八年九月,新安縣署牡丹開花一★,十月又開十★,歷月不萎。九月,太原桃李華。二十年春,普安州桂花盛開。二十四年九月,潞安桃李華。三十年九月,高邑桃李華。三十三年九月,和順桃李華。四十三年九月,新城桃李華。冬,石門桃李華。四十六年九月,臨邑桃李華。四十九年十月,桐鄉鳳鳴寺牡丹開二花,單瓣紫色。十一月,金華桃李華。五十年秋,通州杏再華。六十年十二月,樂清桃李華。

嘉慶四年九月,邢臺桃李華。六年八月,陸川桃李華。

道光三年九月,興國桃花盛開。九年十月,宜城桃李華。十七年冬,望都、清苑桃杏華。二十二年十月,崇陽桃李華。二十四年九月,滕縣桃李華。二十五年十月,鍾祥桃再華。二十九年秋,餘姚桃花盛開。三十年九月,竹山桃李牡丹華。

咸豐元年秋,貴溪桃李華。十月,鄖縣桃李華。四年冬,松滋桃李華。五年十一月,武昌桃李華。九年秋,崇陽桃再華。十月,宜昌桃李華。十年九月,嘉興桃李華。十一月,麻城桃再華。

同治元年十月,襄陽桃李華。二年冬,通州桃李華。四年冬,房縣桃李華。八年冬,黃安桃再華。十二年九月,惠民桃李華。

光緒元年十一月,莊浪桃杏華。四年冬,武昌、光化桃李華。五年冬至時,高淳★花齊放,宛如春色。六年七月,歸州桃李華。九月,蒼溪桃再華。九年冬,興山桃李華。十二年九月,南樂杏再華。二十四年十月,南樂桃李華。三十二年秋,靖遠桃李華。三十四年八月,固原桃李華。

順治十七年八月,玉屏黑蟲蔽山,草木皆盡。

康熙十年秋,潮州蟲生五色,大如指,長三寸,食稼。十一年七月,杭州雨蟲,食穗。十二年七月,萬載蟲食禾。十三年三月,寧都屋上有生黑蟲者,★人甚痛。十七年七月,崇明出兩頭蟲,首尾皆喙,嚙草如刈。十九年六月,婺源青蟲害稼。二十年二月,鄖陽蟲災。二十一年五月,金華蟲災。二十二年四月,恩施蟲災。二十三年五月,渠縣有蟲數萬斛,似蝗,黑色,頭銳,有翅,嗅之甚臭。二十七年七月,蘇州、青浦蟲災。二十九年四月,沁水白黑蟲食禾,結繭。三十年三月,萬載青蟲食禾。三十六年,遵化州生蟲,似槐蟲而黑,食稼幾盡。三十九年,貴縣生蟲,食豆。四十二年,昭化有蟲如蠶,食禾。四十五年二月,房縣蟲食禾。夏,霑化有蟲似螳螂而金色,識者曰,此蒼諸也,見則歲兇。四十九年五月,井陘五色蟲生。五十六年,鶴慶蟲食禾。五十七年夏,新樂生蟲,青色,傷禾。

雍正二年七月,鎮海麥莖生蟲,頭紅身黑,狀如蠶。十年秋,清河禾生蟲,形似蛆,有毛,紅色。

乾隆十七年八月,仁和蟲食稼。二十年春,臨安蟲災。二十一年六月,景寧有白蟲無數自南來。二十三年秋,海寧雨蠶。二十四年八月,武邑有蟲食禾根。三十年十月,嘉興蟲災。三十八年春,青浦河水生蟲,色紅,狀如蜈蚣,長三四寸,昏暮始見。六十年正月,平度蟲災。

嘉慶九年夏,洛川蟲傷禾。

道光五年七月,滕縣生五色蟲,食禾殆盡。

咸豐元年六月,崇陽蟲災。九年五月,蘇州禾田中出蟲,名曰稻★。

同治四年秋,秀水有青蟲如蠶,喙黑,卷葉作網。十三年九月,嘉興田禾生蟲,食根,似黑蟻,蜂腰,六足,有須。

光緒二年八月,寧津蟲傷稼。十四年春,泰安蟲災。

順治元年二月,萊陽民家牛產犢,一體二首。二年二月,交城民家牛產一犢,遍體鱗甲。十年,文縣民家牛產兩麟。十六年,定州民家牛產麟。

康熙五年,南昌民家牛產麟。十三年七月,巫山民家牛產一犢,三目四耳,舌端有鈌,胸列四蹄,脊後分為二身,各二蹄一尾。十五年,池州民家牛產犢,二首八足。十七年六月,鎮洋民家牛產犢,兩頭。二十八年九月,餘姚北鄉民家牛產麟,狼項、馬足、★身,遍體鱗甲,金紫相錯。三十八年,景寧民家牛產麟。

雍正七年,鎮海民家牛生一犢,遍體鱗紋,色青黑,頷下有髯,項皆細鱗。十一年五月,鹽亭民家牛產一麟,高二尺五寸,肉角一,長寸許,目如水晶,鱗甲遍體,兩脊傍至尾各有肉粒如豆,黃金色,★身,八足,牛蹄,產時風雨交至,金光滿院,射草木皆黃。十三年二月,綿州民家牛產一犢,首形如龍,身有鱗紋,無毛,落地而殤。

乾隆四年,盛京民家牛產麟。五年,壽州民家牛產麟,一室火光,★以為怪,格殺之,剝皮,見周身鱗甲,頭角猶隱ю也;荊州民家牛產麟,遍體鱗甲。二十二年,崇明見三足牛,前一後二。

嘉慶元年,遂安民家牛產麟。二年,平度州民家牛產麟。五年,白河縣民家牛產一犢,兩首雙項,剖腹視之,心赤有二。

道光十二年,永嘉民家牛產犢,兩首。

咸豐二年,潛江民家牛產犢,兩首。七年,黃巖民家牛產犢,四首。

同治九年,莘縣民家牛產犢,兩首。

光緒十九年,太平民家牛產麟。七年,京山民家牛產犢,三足,前二後一,識之者謂之豲。三十四年,皋蘭民家牛產犢,兩首。

順治元年九月,翼城地震。冬,石首地震。二十二年,祁縣地震三次。三年十月十日,石埭地震。四年四月,全椒地震。五年三月甲辰,涇陽、三原、臨潼、鳳翔地震;戊辰又地震。四月二十四日,榆社地震。八月,潞安地震有聲。六年正月,南樂地震。二月初六日,陸川地震。四月,高平、陽城地震。七年八月初十日,高淳地震。八年正月丁卯,蘇州、昆山地震。六月,高平地震。九年正月元旦,潛江、太湖地震。十五日,貴池地震,屋瓦皆飛,江波如蕩。二月十五日,池州、潁上、阜陽、五河、全椒地震。二十四日,宿松地震。二十六日,銅陵地震。七月,贛榆地震。九月,霍山、六安地震。十年正月,廬江地震。六月乙卯,鎮洋地震。七月,海豐地震。九月,樂陵地震。十月二十一日,貴池地震;二十四日復震。十一月二十三日,五河地震。十一年正月朔,潛山、望江、石樓、貴池、銅陵、舒城、廬江地震。五月,廬江又震。四月初六日,蕭山地震。五月初八日,寶雞、定遠、沔縣地震,壞屋壓人。六月,興安、安康、白河、紫陽、洵陽、蘭州、鞏昌、慶陽等處地震,聲如雷,壞民舍,壓死人畜甚★。八月初五日,陽穀、東昌地震,次日又震。初八日辰刻,朝城地震,申刻復震。十二年正月初七日,陽湖、營山地震。二月庚申,昆山、婁縣地震。十三年三月初八日,中部地震。十四年三月朔,成都、威州、汶川地震。二十五日,西充地震,次日復震。七月,富陽地震。十五年二月二十四日,惠來地震。五月二十三日,武進地震。八月二十三日,蘇州、昆山、上海、青浦地震。十一月,安塞地震有聲。十六年正月二十八日,鎮平地震。二月初八日,揭陽地震。七月十七日,石埭、貴池地震,聲如雷。十七年八月,曹縣、兗州地震。十二月二十三日,雒南、商南地震。十八年正月,兗州地屢震。

康熙元年正月二十五日,伏羌地震。三月初四日,西寧、龍門、宣化、赤城、保安州等處地大震,人皆眩僕。六月十七日,太平地震。七月十一日,蒼梧、容縣、岑溪地震。十一月二十二日,威縣地震。二年正月二十五日,鍾祥地震,次日復震。五月二十一日,咸寧地震。六月望日,東安地震。十二月,鶴慶地震。三年三月初二日,保安州、龍門地震。初三日,懷來、灤州地震。五月,開平地震。八月十七日,萊陽地震。二十三日,安邑、解州地震。九月丙子,昆山地震。十一月二十一日,順德地震。四年二月初四日,平陰地震。三月初二日,京師地震有聲。初四日,景州地震。四月十五日,灤州、東安、昌平、順義地震二次,房垣皆傾。七月十七日,大城地震。五年二月二十二日,開平地震,次日又震。三月初八日,交河地震。七月十七日,虹縣地震,城傾數十丈,民舍悉壞。九月二十六日,揭陽地震。十二月丁未,蘇州地震。六年正月初四日,陽春地震。四月十二日,揭陽地震。六月十七日,慶雲地震。八月十四日,邢臺、內丘地震,聲如雷。九月二十三日,永年、威縣地震。七年四月,金華地震。五月癸醜子時,京師地震;初七、初九、初十、十三又震。六月十七日,上海、海鹽地震,窗廊皆鳴;湖州、紹興地震,壓斃人畜,次日又震;桐鄉、嵊縣地震,屋瓦皆落。十八日,香河、無極、南樂地震,自西北起,戛戛有聲,房屋搖動。十九日,清河、德清地震有聲,房舍皆傾。七月二十日,錢塘地震。二十五日,潛江地震。八年九月甲午寅時,京師地震有聲。九年四月初六日,安縣地震。五月初七日,揭陽地震。七月己未,吳江、震澤地震有聲。八月初七日,開建、安丘地震。十一月冬至前一日,鄒縣地震。

十年九月初九日,保安州地震。十一年三月初三日,陽曲地震。五月丙寅,沛縣、高密地震。六月二十四日,高唐地震。七月二十八日,廣平地震。八月癸亥,蘇州地震。九月丁亥,平樂地震。十二年二月十二日,廬州地震,聲如雷,屋舍傾倒。四月初四日,臨縣、高淳地震。七月二十三日,寶坻、霸州、萬全地震。九月初九日,懷安、赤城、西寧、天鎮、紹□、德陽地震。十二月,湖州地震。十三年二月初七日,保德州地震。八月三十日,隴州、懷安地震。九月初九日,府穀地震。十四年六月十二日,曹州府屬各州縣同時地震。十五年七月十五日,婺源地震。十一月初四日,蘇州地震有聲。十六年五月十四日,合浦地震。六月,階州地震,數日乃止。十七年四月初五日,蘇州、鎮洋、上海、青浦、崇明地震。初七日,海鹽地震,屋瓦傾覆。七月二十八日,京師地震。十月初五日,安平地震。十八年三月二十三日,鎮洋地震。六月朔,榮成、寧海、文登地震。二十八日,濱州、信陽、海豐、霑化地震。七月初九日,京師地震;通州、三河、平谷、香河、武清、永清、寶坻、固安地大震,聲響如奔車,如急雷,晝晦如夜,房舍傾倒,壓斃男婦無算,地裂,湧黑水甚臭。二十八日,宣化、鉅鹿、武邑、昌黎、新城、唐山、景州、沙河、寧津、東光、慶雲、無極地震。八月,萬全、保定、安肅地屢震。九月,襄垣、武鄉、徐溝地震數次,民舍盡頹。十月,潞安地震。十一月,遵化州地震有聲如雷。十九年四月二十五日,瓊州地震。十一月,廬陵地震,有聲如雷。

二十年春,永嘉、樂清地震。七月十七日,瓊州地震。八月十一日,東流地震有聲。九月,貴州地震。十月初十日,平遠州、潞城地震。十一月初三日,東流、府穀地震。二十一年十月初五日,襄垣地震。初六日,潞安地震。初十日,介休地震,民舍多傾倒。二十二年五月十五日,龍門地震。七月初五日,定襄地震,壓斃千餘人。十月初五日,保德州地震,人有壓斃者。十一月朔,瓊州地震。二十三年五月,封川地震。十月初五日,普州地震。十一月初七日,合浦地震。二十四年二月庚子,永安州、平樂地震。十二月二十四日,蓬萊、福山、文登地震,越二日又震。二十五年七月十七日,宜都、宜昌地震。十月初五日,井陘地震。二十六年九月丁亥醜時,京師地震。十月十六日,蓬萊、棲霞地震,聲如雷,月餘乃息。十二月朔,無為地震。二十七年四月,臨潼、咸陽地震。二十八年正月十八日,瓊州、陸川地震;三月初十日又震。六月朔,榮成、文登地震。二十九年二月,杭州地震。七月,臨汾、襄垣地震。九月,襄陵地震。

三十年三月十六日,慶雲地震。三十二年三月十九日,海豐地震;二十日又大震,壞民舍。三十三年二月初八日,巢縣地震。八月,雞澤地震。三十四年正月朔,瓊州、雷州、全州、柳城地震。十五日,巢縣地震。四月初六日,光化、滕縣、恩縣、丘縣、徐溝、太平、真陽、盂縣、交城地大震;臨汾、翼城、浮山、安邑、平陸震尤甚,壞廬舍十之五,壓斃萬餘人。八月,平原地震。三十五年正月二十一日,巢縣地震。三月十五日,南陵地震。四月甲午,沛縣地震。九月辛巳,京師地震。三十六年正月朔,巢縣地震;三月三日又震。三十九年三月十六日,貴州地震。十八日,黃岡地震。四月,陽江地震。

四十年三月十二日,長子地震。四十一年正月,鶴慶地震。十一月乙酉,京師地微震。十二月二十三日,瓊州地震。四十三年七月十三日,涇陽地震,壓斃人畜無數。八月二十日,東光地震。四十四年正月十三日,平遙地震。八月丁酉,京師地震。九月十六夜,慶雲地震。四十五年二月丙辰,京師地微震。四十六年七月初四日,蘇州地震。十月,興寧地震;十一月又震。四十七年正月朔,曲沃地震。五月十七日,嘉定地震。六月十二日,鳳翔地震。九月十二日,寧陜地震。十三日,永年地震。十月十一日,丘縣地震。四十八年九月初二日,保德州地震。十二日,涼州、西寧、固原、寧夏、中★地震傷人;靖遠大震,塌民舍二千餘間,城墻倒六百六十餘丈,壓斃居民甚多。四十九年三月十四日,靈臺、環縣地震。八月初三日,黃岡地震。

五十年九月十一日,景寧地震。十月十一日,平樂地震。五十一年九月十二日,慶元、江浦地震;十一月又震。二十五日,高淳、儀徵、丹陽地震。五十二年四月,棲霞地震。七月,全蜀地震。五十三年三月十四日,寧州地震。九月初八日,湖州地震。五十四年正月十四日,鎮海地震。五月二十一日,岐山地震。五十五年二月,曲沃地震。八月,枝江地震。五十六年五月二十八日,公安、石首、枝江地震。五十七年五月二十七日,翼城地震。六月初八日,海陽地震。五十八年春,泰安地震。五月十一日,盩厔、丹陽地震。六月朔,德州、陽信、霑化、廣靈地震。七月十六日,榆次、懷來地震;八月復震,民居倒壞無數;密雲、東安地震,有聲如雷。六十年六月初八日,青城地震。六十一年八月初四日,江安地震。十一月,順德地震。

雍正三年十月,環縣地震,壞廬舍。四年六月二十一日,宜昌地震。八月,平鄉地震。五年五月二十日,鍾祥地震。六年二月初五日,吳川地震。五月,橫州地震。八月,蔚州地震。七年六月十七日,德州地震。八月十三日,富川地震聲如雷。八年四月十六日,宜昌地震。八月十九日,京師、寧河、慶雲、寧津、臨榆、薊州、邢臺、萬全、容城、淶水、新安、東光、滄州同時地震。十月二十六日,上海地震。二十八日,蘇州、震澤、婁縣、青浦地震。十一月二十八日,嘉興、湖州、桐鄉地震。九年九月初二日,海陽地震;二十二日又震。十月,泰州地震。十一月初八日,海州地震。二十一日,普宣地震。十年正月初三日,西昌縣,會理州、德昌、河西、迷易三所地震。十一月,通州地震。十一年七月十八日,海陽地震,十一月,黃岡地震。十二月,宜昌地震。十二年二月,浦江地震。六月十六日,銅陵地震。十月十三日,潮陽、海陽地震。十二月初一日,清遠地震。十三年七月十七日夜,富川地震。二十日,桐鄉地震,有聲如雷。九月,慶遠府屬地震。十一月,光化地震。

乾隆元年七月朔,臨清地震。初七日,定陶地震。十五日,平原、夏津地震。十一月二十四日,黃山、福山、文登、榮成地震。二年五月初十日,宜昌地震有聲。七月二十五日,雞澤地震有聲。九月初七日,高平地震。十月二十四日,長子地震。三年十一月二十四日,芮城、襄垣、安邑、安定、綏德州、天鎮地震。二十五日,靖遠、慶陽、寧夏、平羅、中★地震如奮躍,土皆墳起,地裂數尺或盈丈,其氣甚熱,壓斃五萬餘人。四年三月二十四日,昌化地震。十一月二十四日,岐山地震。五年三月,萬全地震。八月,赤城、懷安地震。十一月二十四日,清潤地震聲如雷,是夕連震八九次,屋舍傾圮。六年十一月,正寧地震有聲。八月二十四日,昌化地震。九年正月,光化地震。十年四月初四日,浮山地震。五月初六日,高淳地震。十一年五月,增城地震有聲。六月丁丑,京師地震。十月,廣濟地震有聲。十二年三月初九日,鶴慶地震。十月壬午,同官地震。十三年五月,歷城、長山地震。十月,環縣地震。十四年正月初三日,鶴慶地震;十三日又震;二十九日復震。三月二十八日,蒼梧地震。十五年十二月庚午,同官地震。十六年二月,奉議州地震。十七年二月,崖州地震;四月又震。四月初四日,嘉興、湖州、桐鄉地震。九月十二日,惠來地震。十八年八月,兗州地震。十九年四月,慶元、太原地震。五月,蒼梧地震。二十年十一月,婁縣、青浦地震。十二月庚子,蘇州、湖州、桐鄉地震,屋瓦皆鳴。二十一年二月二十二日,荊門州地震,聲如雷。五月十四日,青城地震,聲如雷。九月朔,陽信地震。十月十六日,青浦、桐鄉地震。二十二年十一月十六日,歙縣地震;次日復震。二十三年三月二十七日,永平地震,聲如雷。二十四年九月初五日,象州地震。二十五年十一月二十日,潞安、長子地震。二十六年三月十一日,嘉興地震有聲。二十八年五月甲申,蘇州、湖州地震。二十九年正月丁巳,蘇州、湖州地震。五月二十八日,溧水地震。十月初二日,南宮地震。三十年正月甲寅,蘇州地震。二月十一日,文登、榮成地震。七月初一日,鳳翔地震;十八日又震。伏羌地大震,倒塌屋舍二萬八千七百餘間,壓斃七百七十餘人。三十一年十一月初二日,南宮地震。三十二年五月二十二日,臨潼地震。六月二十日,文登、榮成地震。七月二十五日,南宮地震。十月十六日,婺源地震。三十三年二月,南陵地震。三十四年六月二十五日,蒼梧地震。七月十一日,吳川地震。八月初七日,蒼梧地又震。十二月二十日,武進、潛山、合肥地震。三十五年正月,溧水地震。十二月二十二日,麻城地震。三十六年七月十五日,慶雲地震。三十八年七月二十八日,臨清地震。二十九日,陵川地震。三十九年九月,青浦地震。十月,東阿地震。四十年十一月十一日,陵川地震。十二月二十八日,屏山地震。四十二年四月初七日,祁縣地震。四十三年三月,光化地震。九月九日,吳川地震,有聲如雷。初十日,陸川地震,次日又震。四十四年八月二十日,湖州地震。四十六年三月十六日,樂清地震。四月十六日,瑞安地震。四十七年六月庚寅,蘇州地震。四十九年十一月,光化地震。五十年三月初八日,永昌地震。六月初五日,武城地震。八月初十日,黃縣、文登地震。五十一年五月十一日,鹽亭、遂寧地震。五十四年三月十七日,嘉興地震;二十日又震。九月二十日,潼關地震,壞民舍,人有壓斃者。五十五年正月初八日,濟南地震。八月二十四日,樂清地震。十月初六日,文登、榮成地震。五十六年正月初九日,濟南地震。二月二十一日,吳川地震有聲。五十七年五月癸卯,蘇州、湖州地震。五十九年正月,武強地震。三月,臨邑地震。六十年十一月二十五日,嘉善地震。

嘉慶元年正月,樂清地震,地裂,湧黑水。二月,諸城地震。二年六月十三日,灤州地震。三年八月,嘉善地震。四年正月二十五日,文登、榮成地震。十二月,臨榆地震。五年二月二十六日,昌黎地震。七年九月,崇陽地震。八年二月,紫陽地震。十一月,宜春地震。十年二月十二日,灤州地震。六月六日,邢臺地震,有聲如雷。十一年十一月十四日,黃縣地震。十八日又震。十二年四月初十日,寧津、東光地震。九月十二日,麻城地震。十三年九月,慶元地震。十五年十月十五日,縉雲地震。十六年二月二十三日,永嘉、樂清地震。四月初九日,文登地震。八月初十日,打箭爐、百利、甘孜、綽倭地方地震,震斃夷、民四百八十一人。十八年六月,安定地震。八月,鄖縣地震。九月十一日,永嘉地震。十月初二日,平樂地震有聲。二十年四月十九日,光化地震。七月朔,寧津、東光地震。九月十一日,樂陵地震;越十日又震;宣平、三原地震。十月二十一日,湖州地震。二十一年六月十四日,東光地震。秋,均州地震。二十二年四月初八日,文登、榮成地震,聲如雷。二十三年十一月十五日,滕縣地震。二十四年七月二十五日,貴陽地震。九月,縉雲地震。十月十二日,黃縣地震;十六日又震。二十五年正月十九日,鎮番地震,聲如雷。四月,貴陽地震。六月二十二日,南宮地震。

道光元年三月晦日,撫寧地震。六月,紫陽地震。三年三月,宜都地震。六月,文登地震。七月,定遠地震。四年十一月十四日,枝江地震。五年六月,保康地震。六年正月晦日,章丘地震。二月二十四日,枝江地震。四月初四日,宜昌閻家坪裂五尺許,廣四丈餘。六月,貴陽地震。七年二月,鄖縣地震。七月,寧津地震。十月,章丘、新城、長清地震。八年八月,興山地震。十月二十三日,黃縣地震。九年五月初四日,宜城地震。十月二十二日,博平、莘縣地震;青州、臨朐地震,十餘日方止,民舍傾倒,壓死數百人。二十三日,黃縣、即墨、平度、滕縣、長清、章丘地震。十年四月二十二日,南宮、平鄉地震。閏四月二十二日,元氏、新樂、荷澤、曹縣等處同時地震,房舍傾圮,人有壓斃者。十月十六日,武定地震;二十四日又震。十一月朔,黃安地震有聲。十一年三月,撫寧地震。四月,臨邑地震。九月,武進地震。十二年九月二十三日,臨邑地震。十三年四月十八日,定遠漁渡壩陷十餘丈。十月二十四日,鄖縣地震。十五年七月初三日,高淳地震。十七年十月辛亥,臨朐地震。十八年二月,興安地陷,水湧如塘。十九年九月乙卯,青浦地震。戊戌,武進地震。二十年正月二十三日,隨州地震,屋瓦皆動。二十二年正月十四日,高淳地震。九月,即墨地震。二十三年三月初八日,棲霞地震。二十四年八月二十五日,寧海地震。十月壬戌,青浦地震。二十五年六月辛丑,青浦、蘇州地震。十月十四日,嵊縣地震,屋舍搖動。二十六年五月十一日,嵊縣地震。六月十二日,湖州、定海地震。十月丁巳,青浦地震。二十七年十月辛亥,蘇州地震。二十八年六月十四日,永嘉地震。十一月初七日,縉雲地震。二十九年三月初五日,撫寧地震。三十年三月二十八日,枝江、松滋地震。

咸豐元年正月甲辰,青浦地震。二月,江陵、公安地震。五月,黃安地震。六月朔,瀘溪地震。二年四月十二日,應山地震。十八日,中★地震,湧黑沙,壓斃數百人。十月初六日,黃巖、太平、嵊縣地震。十一月壬子,蘇州、青浦地震。三年正月,黃巖地震,是年屢震。三月辛亥、壬子,蘇州地震;辛酉又震。四月初五日,通州地屢震。二十三日,元氏地震。七月,景州地震。四年五月,安福地陷,廣數丈,深不可測。九月朔,江陵地震。十二月初四日,鍾祥地震。五年正月辛酉,青浦地震;九月戊寅又震;十月辛卯又震。十二月朔,棲霞地震。初五日,黃縣地震。六年五月初六日,來鳳地震,武昌縣百子畈地裂。七年四月,興國地震。九月,鉛山地震。十月,永豐地震。十二月二十六日,蓬萊地震,有聲如雷,自是屢震。八年正月二十七日,蓬萊地復震,十餘日始止;自七年至八年,凡震三十餘次。十二月,宜黃地震。九年三月,恩施地震。十年七月初八日,枝江地震。十一年五月二十五日,棲霞地震。八月朔,寧遠地震。

同治元年六月十二日,應城地震有聲。三年三月庚午,青浦地震。四年正月二十九日,鍾祥地震;二月初四日復震。五年八月十三日,景寧地震。九月十四日,青田地震。六年二月初一日,鍾祥地震。三月十五日,江陵地震。八月,太平地震;十二月又震。七年六月初三日,均州、光化、鄖縣地震。七月初三日,隨州安全巖地陷水湧。十年四月,襄陽地震。十一年六月十九日,高淳地震。八月十九日,嘉興、柏鄉地震。十二年正月二十六日,肅州地震。十三年三月二十日,霑化地震。

光緒元年九月,皋蘭地震。三年六月丁亥,青浦地震。四年十二月二十八日,襄陽地震。五年五月初十日,隴右諸州縣同時地震。十二日,光化地震。十三日,京山地震。六年十月,光化地震。七年四月,太平地震;五月又震。十月二十日,東光地震;二十五日復震。禮縣地震,震斃四百八十人,傾倒民房四千有奇,牲畜無算。十一月初二日,西寧丹噶爾地震。八年二月初八日,西利地震。七月,南樂、望都地震。九年十二月二十二日,寧津地震。十年十月二十二日,東光地震。十一月二十九日,西利地震。十一年九月二十七日,武昌地震。十三年十二月甲戌,河州地震。十四年五月初五日,霑化、灤州地震。十五年八月,靈川地震;九月又震。十六年正月二十八日,西寧地震。十九年四月十九日,西寧地震,傾圮民房二百餘間,人多壓斃。二十一年十二月初四日,山丹地震。二十三年正月二十四日,蘭州地震。二十七年春,靜寧州地震。二十八年十二月除夕,永昌地震。二十九年五月二十九日,曲陽地震。

順治元年十一月十二日,鹽亭山崩。三年四月,河源桂山崩。六年四月,兩當山崩,壓斃人畜無算;蘭溪大慈山崩。七年六月,武昌馬山崩。八年四月二十六日,黃縣萊山巨石崩,聲聞數里。六月,安丘土山裂丈餘,廣二尺餘,深不可測,翼日乃合。八月己巳,同官王益山崩。九年五月,馬平槎山崩。十六年秋,成都霪雨,錦屏山崩。

康熙元年秋,蕭山大雨,小山崩;平陸山崩;霪雨,四明山崩;兩當暴雨,山崩。二年七月,河州大雷雨,井溝山崩,壓死居民二十餘口。九月,灌陽大營山崩。六年四月二十一日,開建大紫山崩;臺州臨海大雨,山崩。十五年七月辛丑,同官濟塞山崩,壓死四十餘人。十九年八月初二日,平湖雅巖裂。二十年正月,天臺方山崩。五月十二日,宣平大萊山陷。三十五年十二月二十七日,保德州康家山崩。四十一年秋,寶雞霪雨,山崩。四十二年四月初六日,太原奉聖寺山移數步。四十七年,保縣熊耳山崩。

雍正七年三月,崖州南山崩。八年五月,興安大雨,山崩;狄道鳳臺山崩。十年六月,富川西嶺山崩數處。

乾隆四年十一月十四日,泰安縣北山崩。十年七月十二日,百泉山崩,壓斃二十五人。十五年五月,英山巖崩裂。六月,棠陰大雨,西北山崩。十六年二月,奉議州東咘露村山崩裂,有聲如雷。六月十二日,秦州仁壽山崩。十七年二月,忻城山崩,有聲如雷。二十一年八月,秦州邽山崩。三十八年五月,慶元白馬山崩。三十九年六月,雲和大雨,山崩,壓斃四人。四十一年十二月,雲和五樹莊山裂數百丈。五十七年五月,宜黃山崩,壓斃數十人。六十年四月,慶元蓋竹山崩。

嘉慶五年六月二十一日,義烏霪雨,山崩。二十三日,金華大雨,山崩。九年正月,新城北屯山崩。二十三年七月初五日,狄道州東山崩,壓陷田地三十餘畝。八月,永嘉大雨,西山崩,陷地丈許。二十四年五月,東湖山崩。

道光元年夏,新昌上方源山裂。三年七月甲戌,蘇州玉遮山裂。四年六月,定遠五塊石山崩,壞市廛民舍。六年六月,宜昌大雨,山崩。十一年六月,狄道州黎家窪山崩,壓斃二十餘人。十二年七月,漢城槐木溝巖崩。十三年四月十八日,招遠羅巖崩一角,聲聞數里。十四年正月十五日,麻城磨石岡巨石裂數塊,有聲如雷。十五年六月,定遠霪雨,母★硐山崩。十八年十一月,恩施山崩。二十四年九月,星子五老★右巖崩墜,有聲如雷。二十七年六月十七日,西寧縣北川郭家塔爾山崩,南川田家寨山崩。七月,皋蘭縣山崩。十月,宜山崩。二十九年五月二十四日,黃岡大崎山裂數十丈,年餘漸合。

咸豐元年六月,禮縣霪雨,山崩;袁家崖山崩裂,聲震如雷,縱二尺許,橫二百丈。十月,興山仙侶山崩。二年六月朔,狄道馬銜山裂;平河大雨,山崩,壓倒民房無數。三年三月十六日,雲和山裂二百丈。六月二十六日,景寧大雨,山崩,壓斃七十三人。鄖縣青巖崩裂十餘丈;保康大山崩移十里許,毀田廬無算;永嘉大雨,龍泉村山圮覆屋,壓傷十九人。四年七月,雲和山崩,壓斃三十餘人。五年四月,大通縣塔破山崩。六年五月初八日,來鳳大壩路猲甚山崩,壓斃三百餘人。九月,松陽大雷雨,山崩數十丈。

同治四年七月,固原山崩;漢陽鋪有平石寬長丈餘,高四尺,忽自行里許始止。十三年七月十二日,宣平北門山崩。八月十一日,西和西山崩,走入城中,壓倒城垣二百四十餘丈,民房九十餘處,壓死四十九人。

光緒元年正月朔,西寧西川陰山崩。七月,舊洮州東明山崩。三年六月,河州紅崖山崩,壓斃二百餘人,牲畜無算。五年五月,文縣山崩。九年三月,光化馬窟山裂。十二年六月,河州草領山崩。十九年五月,狄道州皇后溝山崩,壓斃十三人。二十年二月二十七日,河州東八部蘭山崩。二十二年二月,河州哈家山崩。二十三年八月,寧遠大夫溝山崩。二十六年六月,漳縣還山崩,靜寧州南五臺山崩,河州王家山崩。二十七年六月,皋蘭五泉山、三臺閣山崖崩。三十一年七月二十四日,洮州泉古山崩。三十二年五月,洮州莽灣山崩。七月,芽坡山崩。三十三年五月,寧遠小村槽山崩。

宣統元年六月十五日,秦州雒家川南山崩。

順治五年三月,上海遍地生白毛。四月,婁縣地生白毛。六年六月,杭州、嘉興地生白毛。八月初三日,萊陽雨白毛。七年六月,蘇州、鎮洋、震澤、青浦地裂,生白毛。九年十月初四日,永嘉雨絮。

康熙七年六月,上海、海鹽、湖州、平湖、寧波地生白毛,長尺許。七月,臨安、餘姚地生白毛,長尺許。八月,永嘉、桐鄉地生白毛。八年八月,開化縣地生白毛。十月,義烏地生白毛。十四年三月,瓊州地生白毛,長寸餘。十七年十月二十六日,鎮洋雨白毛如雪片。四十六年,太平地生毛。

乾隆二十八年,南陵地生毛,白質黑穎。二十九年五月,武進地生白毛,長數寸。四十一年,婺源地生白毛。

嘉慶十九年七月,青浦遍地生白毛如發。二十三年三月,宜城地生毛,或白或黑,長尺餘。

咸豐元年,江陵地生白毛,長三寸許。二年五月,青浦地生白毛。三年四月,武進地生毛。六年夏,青浦地生毛。七月,武進地生毛。九月,桐鄉地生白毛。九年十月,武昌地生毛。

同治元年七月,高陽地生毛。四年六月,羅田遍地生蒼白毛,長三寸許;即墨地生毛。五年十一月,潛江地生黑毛,長三寸;江陵地生毛。六年三月,德安地生毛。八月,京山地生毛,或黑或白,長尺餘。十月,隨州地生白毛。七年春,應山地生毛。夏,黃安地生毛。

光緒四年冬,光化地生毛。

順治元年春,荊門大饑。冬,鄖縣大饑。二年,棗陽、襄陽、光化、宜城大饑,人相食。三年,太平、瑞安、崇陽大饑。四年,蘇州、震澤、嘉定、太湖、潛山、石埭、建德、宿松、江山、常山大饑。五年春,廣州、鶴慶、嵩明大饑,人相食。夏,惠來、大埔、嘉應州、興寧、陽春、梧州、北流大饑,斗米可易一子。冬,全蜀饑;六年,全蜀仍饑;灌陽、平陽大饑。七年夏,榆林、青田饑。秋,永寧州、襄垣、萍鄉大饑。冬,阜平饑。八年春,平湖、袁州、萍鄉、萬載饑。夏,壽陽、靜樂饑。九年春,蘇州大饑。夏,黃陂、孝感、天門饑,民多為盜。十年夏,興寧、長樂、博羅、陽江、陽春饑。冬,六安饑。十一年,臨榆、樂亭、新樂饑。十二年夏,臨川、沁州饑。秋,武邑、寧晉饑。冬,金華、東陽、永康、武義、湯溪五縣饑。十三年春,瓊州饑。秋,東安饑。冬,烏程、壽光饑。十四年,樂亭饑。十五年,永年、撫寧、昌黎、慶雲、雞澤、威縣饑。十六年春,陽信、海豐、莒州大饑。夏,膠州饑。十七年夏,遵化州饑。秋,獨山州大饑,民多餓斃。冬,灤州饑。十八年春,興寧饑。夏,南籠府大饑。秋,臨安饑。

康熙元年,吳川大饑。二年,合肥饑。三年春,揭陽饑。秋,交河、寧晉饑。四年春,曹州、兗州、東昌大饑。夏,惠來饑。秋,懷遠饑。冬,烏城饑。六年,應山饑。七年,無極大饑。十年夏,海鹽大饑。秋,臨安、東陽大饑。十一年,永康、峽江、大冶饑。秋,遂安、湯溪大饑。十二年,樂亭大饑。十三年春,興寧、鎮平、京山大饑。十四年,東光饑。十五年春,大冶饑。夏,連平饑。十六年春,嘉應州大饑。夏,鄖縣、鄖陽、鄖西大饑。十七年秋,曲江饑。十八年春,真定府屬饑。夏,興寧、長樂、嘉應州、平遠饑。秋,無為、合肥、廬江、巢縣、博興、樂安、臨朐、高苑、昌樂、壽光大饑。冬,滿城饑。十九年春,江夏大饑。夏,大同、天鎮饑。冬,萬泉、遵化州、滄州饑。二十年夏,儋州、永嘉饑。二十一年春,桐鄉饑。冬,信宜、真定、保安州饑。二十二年春,宜興饑。秋,單縣饑。二十三年春,濟寧州、剡州、費縣饑。秋,巴縣、江安、羅田饑。二十四年春,沛縣饑。二十五年秋,恭城大饑。冬,★城大饑。二十六年,博興大饑。二十七年秋,蔚州饑。二十八年春,高邑、文登饑。夏,潛江大饑。秋,龍門饑。二十九年夏,黃岡、黃安、羅田、蘄州、黃梅、廣濟饑。秋,襄垣、長子、平順饑。三十年春,昌邑饑。秋,順天府、保安州、真定饑。三十一年春,洪洞、臨汾、襄陵饑。夏,富平、盩厔、涇陽饑。秋,陜西饑。三十二年夏,慶陽饑。秋,湖州饑。三十三年,沙河饑。三十四年,畢節饑。三十五年夏,長寧、新安、★城饑。秋,大埔饑。三十六年夏,廣寧、連平、龍川、海陽、揭陽、澄海、嘉應州大饑。秋,慶元、龍南、潛江、酉陽、江陵、遠安、荊州、鄖西、江陵、監利饑。三十七年春,平定、樂平大饑,人相食。夏,濟南、寧陽、莒州、沂水大饑。三十八年春,陵川饑。夏,婺源、費縣饑。秋,金華饑。三十九年秋,西安、江山、常山饑。四十年,靖遠饑。四十一年春,吳川大饑。夏,沂州、剡城、費縣大饑。冬,慶雲饑。四十二年夏,永年、東明饑。秋,沛縣、亳州、東阿、曲阜、蒲縣、滕縣大饑。冬,汶上、沂州、莒州、兗州、東昌、鄆城大饑,人相食。四十三年春,泰安大饑,人相食,死者枕藉;肥城、東平大饑,人相食;武定、濱州、商河、陽信、利津、霑化饑;兗州、登州大饑,民死大半,至食屋草;昌邑、即墨、掖縣、高密、膠州大饑,人相食。四十四年,鳳陽府屬饑。四十五年春,漢川、鍾祥、荊門、江陵、監利、京山、潛江、沔陽、鄖縣、鄖西饑。四十六年秋,東流、宿州饑。四十七年,平鄉、沙河、鉅鹿饑。四十八年春,無為、宿州饑。夏,沂城、剡城、邢臺、平鄉饑。秋,武進、清河饑。四十九年,阜陽饑。五十年,通州饑。五十一年,古浪饑。五十二年春,蒼梧饑,死者以千計。夏,長寧、連平、合浦、信宜、崖州、柳城饑。五十三年春,陽江饑。冬,漢陽、漢川、孝感饑。五十四年夏,臨榆饑;遵化州大饑,人食樹皮。五十五年春,順天、樂亭饑。五十六年春,天臺饑。五十七年,廣濟饑。五十八年春,日照饑。夏,靜寧、環縣饑。五十九年春,臨潼、三原饑。夏,蒲縣饑。六十年春,平樂、富川饑。夏,邢臺饑。秋,咸陽大饑。冬,兗州府屬饑。六十一年夏,井陘、曲陽、平鄉、邢臺饑。夏,蒙陰、沂水饑。秋,嘉興、金華饑。冬,懷集饑。

雍正元年夏,通州饑。秋,嘉興饑。二年春,蒲臺大饑。夏,樂清、金華、嵊縣饑。冬,英山饑。三年夏,順德、膠州饑。冬,惠來饑。四年春,嘉應州饑。秋,澄陽江饑。五年冬,江陵、崇陽饑。七年,壽州饑。八年夏,肥城、武城饑。冬,銅陵大饑。九年春,肥城大饑,死者相枕藉;莒州、範縣、黃縣、招遠、文登饑。夏,章丘、鄒平大饑。冬,濟南大饑。十年,崇明、海寧饑。十一年冬,上海、嘉興饑。十二年秋,武進大饑。十三年秋,慶遠府屬大饑。冬,垣曲饑。

乾隆元年夏,海陽饑。三年秋,平陽饑。四年春,葭州饑。夏,碭山饑。五年,鞏昌、秦州、慶陽等處饑。六年,甘肅隴右諸州縣大饑。七年春,山陽饑。夏,宜都饑。秋,亳州饑。八年春,南昌、饒州、廣信、撫州、瑞州、袁州、贛州各府大饑。夏,天津、深州二十八州縣饑。九年,高邑大饑。十年,正定、贊皇、無極、★城、元氏等縣饑。十一年春,霑化饑。夏,慶雲、寧津饑。十二年,曹州、博山、高苑、昌樂、安丘、諸城、臨朐饑。十三年春,曲阜、寧陽、濟寧、日照、沂水饑。夏,福山、棲霞、文登、榮成饑,棲霞尤甚,鬻男女。十四年春,安丘、諸城、黃縣大饑,餓殍載道,鬻子女者無算。十五年秋,廣信饑。十六年春,福山、棲霞饑,民多餓死。夏,南昌、廣信饑。冬,建德饑。十七年春,全州饑。夏,同官、洵陽、白河饑。冬,房縣饑。十八年春,慶元饑。秋,鄖縣饑。十九年,羅田饑。二十年,溧水、通州饑。二十一年春,青浦、東流、湖州、石門、金華饑。夏,沂州、武城饑。冬,濟南府饑。二十二年夏,博白饑。秋,掖縣饑。二十三年春,翁源、蒼梧饑。夏,日照饑。二十四年秋,隴右諸州縣大饑。二十五年,平定、潞安、長子、長治、和順、天門饑。二十六年,江夏、隨州、枝江饑。二十七年春,濟南饑。夏,棗強、慶雲饑。二十八年夏,永年、永昌大饑。二十九年秋,東光大饑。三十年春,桐廬饑。秋,吉安、廣信、袁州、撫州饑。冬,威遠饑。三十一年,濟南、新城、德州、禹城饑。三十二年冬,池州大饑。三十三年夏,沂水、日照大饑。三十四年,溧水、太湖、高淳饑。三十五年,蘭州、鞏昌、秦州各屬大饑。三十六年夏,會寧、肥城大饑。秋,新城、寧陜饑。三十八年秋,文登、榮成饑。三十九年秋,秦州、鎮番大饑。四十年,溧水、武進、高郵、南陵大饑。四十二年秋,陸川饑。四十三年,全蜀大饑,立人市鬻子女;江夏、武昌等三十一州縣饑。四十四年春,南漳、光化、房縣、隨州、枝江饑。夏,秦州屬饑。四十五年秋,江陵、保康饑。四十七年,灤州、昌黎、臨榆饑。四十八年春,黃縣饑。秋,綏德州饑。四十九年春,葭州饑。夏,來鳳饑。五十年春,宜城、光化、隨州、枝江大饑,人食樹皮。夏,章丘、鄒平、臨邑、東阿、肥城饑。秋,壽光、昌樂、安丘、諸城大饑,父子相食。五十一年春,山東各府、州、縣大饑,人相食。五十二年,臨榆大饑。五十三年秋,文登、榮成饑。五十四年夏,宜都饑。五十五年秋,禹城饑。五十六年,邢臺等八縣饑。五十七年,唐山、寧津、武強、平鄉饑,民多餓斃。五十八年春,常山饑。五十九年,清苑、望都、蠡縣饑。六十年春,蓬萊、黃縣、棲霞饑。夏,麻城饑。

嘉慶五年夏,海陽饑。六年,文登、榮成饑。七年冬,樂亭饑。八年夏,秦州各屬大饑。九年春,滕縣饑。十年夏,黃縣、邢臺饑。十一年春,中部、通渭饑。冬,安陸饑。十二年,薊州、昌黎、永安州饑。十三年夏,黃縣饑。十五年秋,寧津、東光、章丘饑。十六年夏,霸州、保定、文安、大城、固安、永清、東安、宛平、涿州、良鄉、雄縣、安州、新安、任丘、灤州、薊州饑。十七年春,登州府屬大饑,秦州各屬及鎮番、永昌等處大饑。夏,臨榆饑。冬,樂清饑。十八年春,肥城、東阿、滕縣、濟寧、曹縣、諸城饑。冬,宜城、房縣、竹溪、均州、保康饑。十九年春,宜城、安陸、保康、麻城、鄖縣饑。夏,襄陽、漢陽、棗陽、南漳饑。秋,高淳饑。二十年,清苑饑。二十一年,武昌縣饑。二十二年,固安、武強、內丘饑。二十五年秋,樂清、永嘉饑。

道光元年秋,榮成饑。二年夏,灤州饑。三年春,東阿饑。秋,曲陽饑。四年,皋蘭、靜寧、西寧、鞏昌、秦州等處大饑。五年秋,南樂、靜海、文安、大城、寶坻饑。七年春,日照大饑。八年,太平饑。十年冬,江陵饑。十二年春,昌平饑。夏,紫陽大饑,人相食。冬,鍾祥、潛江、漢城、蘄川、黃梅、江陵、公安、監利、松滋饑。十三年春,諸城、日照大饑,民流亡。夏,保康、鄖縣、房縣饑,人相食。秋,灤州、撫寧饑。十四年春,歸州、興山大饑,人相食。夏,莊浪及秦州各屬饑。秋,青浦饑。冬,定海饑。十五年春,諸城饑。秋,孝義大饑。十六年春,登州府屬大饑。冬,太平饑。十七年冬,即墨饑。十八年夏,永年饑。二十年冬,灤州、樂亭、撫寧饑。二十一年夏,高淳饑。冬,枝江饑。二十二年冬,蘄州饑。二十三年秋,湖州饑。二十六年秋,平涼縣饑。二十七年,南樂饑,人相食。二十九年夏,江陵、公安、石首、松滋、枝江、宜都大饑,餓死者無算。冬,青浦饑。三十年春,湖州、咸寧、崇陽饑。

咸豐二年春,日照大饑。夏,全縣大饑。六年,黃縣、臨朐饑。七年春,肥城、東平大饑,死者枕藉;魚臺、日照、臨朐亦饑,人相食。夏,清苑、元氏、無極、邢臺大饑。八年秋,興山饑。

同治元年春,樂亭饑。二年春,孝義饑。秋,江山、常山饑。三年,保康饑。四年春,蘄水饑,民有[B185]子女者。五年,蘭州饑,人相食。六年春,莊浪、金縣、皋蘭饑。七年春,即墨、孝義、藍田、沔縣饑。夏,涇州大饑,人相食。冬,平涼、靜寧、古浪、固原、靈臺、秦州、永昌等處大饑。八年春,日照饑。九年夏,上饒饑。十年秋,望都、樂亭饑。十三年秋,雄縣饑。冬,山丹饑。

光緒元年冬,海州饑。二年春,日照、海陽、灤州饑。三年,高陵大饑,餓斃男婦三千餘人;靖遠、平涼、涇州、靈臺、禮縣、文縣、合水大饑。四年,唐縣等四十州縣饑,莊浪、階州、成縣、靈州、鞏昌、秦州各屬饑。六年秋,邢臺饑。七年,通州等州縣饑。九年秋,鶴★州大饑。十一年夏,霑化饑。十三年冬,洮州、永昌饑。十五年春,魚臺饑。二十一年春,邢臺、灤州饑。二十二年夏,太平饑。二十三年,寧津饑。二十四年冬,靖遠、靜寧、莊浪、丹噶爾饑。二十五年秋,文縣饑。二十六年夏,靖遠饑。二十七年冬,洮州、靜寧、靈臺饑。二十九年,洮州仍饑。三十三年秋,皋蘭饑。

順治十五年六月,遂安雨黃沙。

康熙元年十一月,曹縣雨土數日。三十一年正月,襄垣雨土。三十七年四月,龍門雨黃沙。四十八年九月,丘縣黃埃障天。六十年春,安定雨土。

乾隆四年三月,甘泉雨土。十六年三月十五日,忠州夜雨黃土,★人物皆黃。二十四年二月初七日,薊州雨黃土。三月,永年雨黃土。四十八年三月十四日,寧陜雨土。五十年二月十五日,臨清雨土。五十一年正月,文登、榮成雨土。五十九年二月二十六日,翼城雨土。

嘉慶十四年冬,泰州雨土。二十三年四月,唐山雨土二寸許。

道光四年春,霑化雨土。

咸豐三年二月,棲霞雨土。三月,宜昌雨土。六年三月二十三日,咸寧雨土。

同治三年春,麻城雨土。

光緒四年二月二十九日,宜城雨黃沙。三月,蓬萊雨土。


\end{pinyinscope}