\article{列傳九十五}

\begin{pinyinscope}
那蘇圖楊超曾徐士林邵基王師尹會一王恕

方顯子桂馮光裕楊錫紱潘思矩胡寶瑔

那蘇圖,戴佳氏,字羲文,滿洲鑲黃旗人。康熙五十年,襲拖沙喇哈番世職,授藍翎侍衛。雍正初,四遷兵部侍郎。四年,出為黑龍江將軍。八年,調奉天將軍。乾隆元年,擢兵部尚書。二年,調刑部,授兩江總督。協辦吏部尚書顧琮請江、浙沿海設塘堡,復衛所,下督撫詳議。三年,那蘇圖奏:「明沿海衛、所武事廢弛,我朝裁衛改營,江南有金山、柘林、青村、南匯、川沙、吳淞、劉河諸營,提督駐松江控制。崇明、狼山二鎮對峙海口,塘汛聲勢連絡,無庸復設衛、所。瀕海砲臺,應改建者一,華亭漴缺墩;應增建者二:柘林南門,福山挑山嘴;應移建者一,吳淞王家嘴;應修者一,劉河北七丫口。」並請改舊制,撤墻設垛,置木蓋,留貯藥之屋;並請城茜涇,設兵崇明西南二條監河、顧四房溝、堂沙頭港諸地。下部議行。江南旱,上命撥福建倉穀三十萬石治賑。那蘇圖奏言:「江、廣諸省買米,次第運至,無災州縣,本年漕糧全數截留,兩江不患無米。福建海疆重地,且不產米,請留十萬石分撥災區,以二十萬石運還福建。」上嘉其得封疆大臣之度。四年,詔免兩江地丁錢糧。奏言:「向例蠲免不分貧富,但富戶遇歉,未傷元氣;貧民素乏蓋藏,多免一分,即受一分之惠。請以各州縣實徵冊為據,額根五錢以下者全蠲,五錢以上者酌量蠲免,五兩以上者無庸議蠲。」上諭曰:「卿能如此酌議,如此擔當,誠為可嘉。古人云『有治人無治法』,當訪察胥役,毋令因事擾民,則全美矣。」以憂去。

五年,授刑部尚書。旋出署湖廣總督。六年,調兩江。七年,調閩浙。疏裁闔省鹽場浮費,場員受年節規禮,以不枉法贓論罪。八年,疏言:「溫、臺二洋,漁船汛兵,向有陋規。總督李衛奏改塗稅,稽曾筠又請減半徵收。漁船出洋,海關徵樑頭稅,有司徵漁課,不當復加塗稅。」命永遠革除。九年,疏言:「臺灣孤懸海外,漳、泉、潮、惠流民聚居,巡臺御史熊學鵬議令開荒。臣思曠土久封,遽行召墾,恐匪徒滋事,已令中止。」報聞。

旋調兩廣。十年,條奏:「兩廣鹽政,請以商欠鹽價羨餘分年帶徵。商已承替,令承替者償;官或侵漁,令侵漁者償。埠商占引地,逋成本,斥逐另募。鹽課外加二五加一,並屬私派,悉行禁革。」又調直隸。十一年,條奏八旗屯田章程。十二年,上東巡,那蘇圖從至通州,賚白金萬。條奏稽察山海關諸事,並如所奏議行。加太子少傅。十三年,加太子太保,授領侍衛內大臣,仍留總督任。那蘇圖請赴金川軍前佐班第治事,上不許。十四年,命暫署河道總督。卒,賜祭葬,謚恪勤。

楊超曾,字孟班,湖南武陵人。康熙五十四年進士,改庶吉士,授編修。雍正四年,直南書房。時湖南北甫分闈,命充湖北鄉試考官。旋督陜西學政。再遷左庶子。六年,疏陳:「鎮安、山陽、商南、平利、紫陽、石泉、白河諸縣士風衰落,西安、漢中各屬冒考,號為寄籍,諸弊叢生。請就本籍量取,寧缺無濫。並改寄籍者歸本籍,廩增俱作附生。」議行。調順天學政。遷侍讀學士。九年,擢奉天府尹。疏言:「奉天各屬科派多於正供,造冊有費,考試有費,修廨宇、治保甲有費。長官取之州縣,州縣取之民間,衙蠹里胥,指一派十,嬰害尤劇。已嚴檄所屬檄鑱石禁。」上韙之,下其奏永為例。十年,疏言:「秋收稍歉,明春米穀勢必騰貴,請停商運。」下部議行。十一年,疏言:「州縣所收加一耗羨,自錦州、寧遠外,俱留充州縣養廉。府尹以下養廉,以中江等稅羨支給。」部議即以是年始,著為令。內務府準御史八十條奏,增錦州莊頭百戶撥民種退圈地畝。超曾奏:「地給民種,立業已久。今增莊頭百戶,戶給六百五十晌,晌六畝,都計三十九萬畝。民間萬戶,無地可耕,一時斷難安輯。且正值春耕,清丈動需時日,舊戶新莊俱不能播種,本年賦必兩懸。請緩俟秋收查丈。」事遂寢。遷倉場侍郎。十二年,擢刑部額外侍郎,仍督倉場如故。旋授刑部侍郎。

乾隆元年,署廣西巡撫,二年,實授。疏請豁除桂林等府縣各墟及賀縣花麻地租雜稅。初,巡撫金鉷奏令廢員官生墾荒報捐,有司因以為利,搜民間有餘熟田,量給工本,即作新墾。雲南布政使陳宏謀疏陳其弊,下總督鄂彌達及超曾覈覆。會疏陳捐墾不實田畝、應減應豁及官生短給工本諸事,上命豁加賦虛田凡數萬畝,鉷及布政使張鉞皆奪官。三年,召授兵部尚書。

五年夏,署兩江總督。秋,授吏部尚書,仍署總督。疏劾江西巡撫岳濬及知府董文偉、劉永錫徇情納賄,遣侍郎阿里袞會江南河道總督高斌按治,濬等坐譴。六年,疏請裁太通道、揚州鹽務道,以通州隸常鎮道轄,餘如舊,可其奏。兼署安徽巡撫。秋,大風雨,濱江、海諸州縣皆被水。超曾令先以本州縣所存銀米撫恤,並發司庫銀八萬、未被水諸州縣倉米十萬,賑上江各州縣;又發司庫銀十萬、各縣穀百餘萬,賑下江各州縣。疏入,上諭曰:「料理賑恤,頗為得宜。當以至誠惻怛為之,庶可稍救災黎也。」通州鹽河亦以水發輟工,督治水利大理寺卿汪漋、副都御史德爾敏令開唐家閘洩水。民慮淹及麥田,紛集欲罷巿。侍郎楊嗣璟疏劾,命超曾按其事。超曾奏:「民無挾制阻撓情狀,似可無事深究。」上從之。復疏薦江蘇巡撫徐士林處己儉約,安徽巡撫陳大受虛中無滯,江西巡撫包括性情和平,惟吏玩民刁,鮮所整頓。上諭曰:「此至當至公之論,與朕見同也。」尋內召視部事,以父憂歸,籍喪次。病作,七年,卒,賜祭葬,謚文敏。

徐士林,字式儒,山東文登人。父農也,士林幼聞鄰塾讀書聲,慕之,跪母前曰:「原送兒入塾。」乃奮志勵學。康熙五十二年,成進士,授內閣中書。再遷禮部員外郎。雍正五年,授江南安慶知府。十年,擢江蘇按察使。坐在安慶失察私鑄,左遷福建汀漳道。漳州俗好鬥,殺人,捕之,輒聚眾據山拒。或請用兵,士林不可。命壯丁分扼要隘,三日,度其食且盡,遣人深入,好語曰:「垂手出山者免!」如其言,果逐隊出。伏其仇於旁,仇舉為首者,擒以徇,眾驚散。自此捕殺人者,無敢據山拒。

乾隆元年,遷河南布政使。以父病乞歸侍,旋居父喪。命署江蘇布政使,士林以母病、父未葬,辭。四年,命以布政使護江蘇巡撫,復奏母病篤不能行。是年夏,詣京師,高宗召對,問:「道所經山東、直隸,麥收若何?」曰:「旱且萎。」問:「得雨如何?」曰:「雖雨無益。」問:「何以用人?」曰:「工獻納者,雖敏非才;昧是非者,雖廉實蠹。」上深然之。真除江蘇布政使。五年,湖廣遣山東流民還里,道經江南,恃其眾擾民。士林疏言:「真確災民,或有田可耕,或無田而佃,素皆力穡。時值春融,自當資送復業。至若游惰無業,漂泊日久,彼固非能耕之人,亦不盡被災之民,應請停資送。或謂無籍窮民,恐流而為匪,終年搜查遞送不得休。臣未聞不為匪於本籍,獨為匪於鄰封者;亦未聞真為匪者遞回本籍,即能務本力田而不復潛至鄰封者也。安分則撫之,犯法則懲之,在地方官處置得宜而已。」上是其言,下九卿議行。

秋,授江蘇巡撫。湖北巡撫崔紀以湖廣食淮鹽,自雍正元年定值,遞年加增,為民累,疏請核減,命士林會鹽政準泰核議。士林奏:「鹽為民食所資,貴固累民,賤亦累商。今確核成本,每引賤價以五兩三錢餘為率,貴價以五兩七錢餘為率。商人計子母,若令按本出售,恐商力日絀,轉運不前,民亦所未便。請每引酌給餘息二三錢。」疏下戶部議,成本如所定,至餘息已在成本內,無庸酌給。士林奏:「商人牟利,運鹽不時至,巿值即因之而長。鹽政三保原議每引賤至六兩三錢餘,貴至六兩五錢餘,是實有餘息。今臣所議已將餘息減除,僅加息二三錢。計售於民,每斤增不過以毫計,利已至薄。祗以商本饒裕,常年通算,積少成多。今不給餘息,商情必生退阻。倘漢口運鹽不繼,恐淮商困而楚民亦病也。」上特從之。是歲徐、海水災,士林疏請治賑。六年春,復疏請酌借貧民穀麥。沛縣災最重,請發籓庫餘平銀糴米續賑。別疏言:「江蘇社穀積貯無多,去年秋成,惟徐、海被災,餘俱豐稔。臣飭諸州縣勸捐十餘萬石,仍戒勿強派,勿限數,勿差役滋擾。」上深嘉之。尋以病請告,溫旨慰留,遣醫診視。又疏言:「淮北被水,二麥無收,急宜撫恤。臣不敢泥成例,已先飭發庫帑賑濟,俟察實成災分數具題。」上諭曰:「如此料理,甚副朕視民如傷之念。」

及秋,病益甚,疏請乞假,且言:「母年八十三,未能迎養,暌違兩載,寢食靡寧。」上允之。行至淮安,卒。遺疏入,上諭曰:「士林忠孝性成,以母老遠離,不受妻孥之養,鞠躬盡瘁,遂致沉痾。及得假後,力疾旋里,以圖侍母。臨終無一語及私,勸朕以憂盛危明之心為長治久安之計。此等良臣,方資倚任。乃今溘逝,朕實切切含悲不能自已者也!」命祀京師賢良祠,賜祭葬。遺疏言:「故父之淮,母鞠氏,孝養祖父母,侍病二十餘年,歷久不懈。懇賜表揚。」命予旌如例。

士林善治獄。為巡撫,守令來謁,輒具獄命擬判,每誡之曰:「深文傷和,姑息養奸。夫律例猶本草,其情事萬端,如病者之經絡虛實,不善用藥者殺人,不善用律者亦如之。」凡讞定必先摘大略牌示,始發繕文冊,吏不得因緣為奸。日治官文書,至夜坐白木榻,一燈熒然,手批目覽,雖除夕、元辰弗輟。愛民憂國,惟日不足。江南民尤德之。九年,請祀蘇州名宦祠。鄞縣邵基、臨汾王師與士林先后撫江蘇,有清名。

基,字學阯。康熙六十年進士,改庶吉士。雍正三年,授編修。考選福建道御史。巡中城,止司坊官饋遺商巿月椿錢,釐積案,奸宄惕息。巡直隸順德、大名、廣平三府,以廉勤飭使事。遷戶科給事中,命在上書房行走。四遷國子監祭酒,立教術五條,勉生徒以正學。歷右通政、左僉都御史,並仍兼祭酒。十二年,遷右副都御史,擢吏部侍郎。疏言:「強梗屬員,以上官將予參劾,輒先發制人。往往參本未到,揭帖已至。質訊虛誣,按律治罪,上官已被其累。請嗣後上官恃勢,屬員受屈,仍許直揭部科;其有誣揭者,於本罪外加重科斷。」議行。尋兼翰林院掌院學士。

乾隆元年,充博學鴻詞閱卷官。出為江蘇巡撫。二年,疏言:「江蘇各屬,江、海交錯,全資水利。運道、官河及湖海鉅工,自當發帑官修。其支河汊港,蓄水灌田,向皆民力疏濬。近悉請官帑,似非執中無弊。請將運河及江、河、湖、海專資通洩之處,仍發庫帑估修;其餘河港圩岸,令有司勸民以時疏濬修築,庶公私兩益。」下部議,從之。時以治賑收捐,基疏爭,略言:「天下傳皇上新政,首罷捐例。今為樂善好施之例,是開捐而巧更其名也。周官荒政十二,未聞乞靈於貲郎。」上命停止,戶部持不可,卒行之。上以基題補按察使戴永椿,知府王喬林、石傑皆同鄉,道員李梅賓、盧見曾皆同年,不知避嫌,嚴旨詰責。基旋卒。子鐸,官檢討,早卒。孫洪,賜舉人,官至禮部侍郎,亦有清名。

師,字貞甫。雍正八年進士,以知縣發直隸。十一年,授元城知縣。王勝甿蕪田數百畝,歲有徵,請除其累。導民樹蓺,沙壤成沃,歲祲不待請而賑。調清苑,遷冀州知州。州民被誣為殺人,已定讞,民所聘女誓同死。廉得實,覆鞫,雪其枉,俾完娶。累遷清河道,從大學士高斌等規畫直隸水利,周歷保定、河間、天津、正定諸地,所擘畫多被採用。擢直隸按察使。乾隆十一年,遷浙江布政使,調江蘇,巡撫安寧劾,解任。又以按察使任內失察邪教,降補天津道。再授浙江布政使。十五年,擢江蘇巡撫,免沛縣昭陽湖淹地老荒麻地徵課。尋卒。子亶望,自有傳。

尹會一,字元孚,直隸博野人。雍正二年進士,分工部學習,授主事,遷員外郎。五年,出為襄陽知府。漢水暴漲,壞護城石堤。會一督修建,分植巡功,民忘其勞。創八蠟廟,表諸葛亮所居山,復為茅廬其上。署荊州,石首饑民聚眾,揚言將劫倉穀。會一單騎往諭,系其強悍者,發倉穀次第散予之,眾悅服。九年,調江南揚州知府,濬新舊城巿河通舟楫,濬城西蜀岡下河灌田疇。十一年,遷兩淮鹽運使。新安定書院,士興於學。高宗即位,就加僉都御史銜,擢兩淮鹽政。

乾隆二年,入覲,命署廣東巡撫,以母老辭。調署河南巡撫。河南方閔雨,疏請緩徵,並發倉平糶,不拘存七糶三舊例,視緩急為多寡,上從之。尋疏言:「力田貴乘天時。河南民時宜播種,尚未舉耜;時宜耘耔,始行播種。臣擬分析種植先後,刊諭老農,督率勸勉。如工本不敷,許借倉穀,秋後補還。北方地闊,一夫所耕,自七八十畝至百餘畝,力散工薄。臣勸諭田主,授田以三十畝為率。分多種之田給無田之人,則游民亦少。河南多咸鹼沙地,犁去三尺,則咸少而潤澤。臣責成鄉保就隙地植所宜木,則地無曠土。河南產木棉,而商賈販於江南,民家有機杼者百不得一。擬動公項制造給領。廣勸婦女,互相仿效。」上諭之曰:「酌量而行,不可欲速,不可終怠。若民不樂從,尤不可繩以法也。」旋命實授。三年,上以河南歲稔,敕籌備倉穀。會一疏言:「河南歲豐,直隸、江南歲歉,商販紛集,米價日昂。臣飭有司,本地價高,於鄰縣買補;鄰縣價高,報明不敷銀,在各屬盈餘款內均撥。河南民食麥為上,高粱、蕎麥、豆次之。臣並令參酌糴貯,來春先侭糶借。」上嘉之。

四年,黃河、沁水共漲,瀕河四十七州縣成災。會一定賑恤規條十六,無食者予一月之糧,無居者予葺屋之資,緩徵減糶,留漕運貸倉米,米不足,移他郡之粟助之,富民周濟;並假餘屋以棲貧窶,建棚舍,安流亡,免米稅,興工代賑,種蔓菁助民食,助耔種,施藥餌,延諸生稽察;又令離鄉求食者,有司隨在廩給,開以作業,俟改歲東作資送還鄉。御史宮煥文劾會一本年報盜百六十餘案,秋審招冊駁改至三十餘案,疲玩貽誤,上以會一忠厚謹慎,非有心誤公,召授左副都御史。疏陳:「人主一言,天下屬耳目。今方甄別年老不勝任之員,而饒州知府張鍾以年老改部屬,旬日間前後頓殊,群下無所法守。」上嘉納之。

會一母年七十餘,疏請終養。上知會一孝母,母李先以節孝旌,有賢名,賜詩褒之。會一在官有善政,必歸美於母。家居設義倉,置義田,興義學,謂皆出母意。母卒,會一年已逾五十,居喪一遵古禮。十一年,服闋,召授工部侍郎,督江蘇學政。

十二年,上敕各省學政按試時,以御纂四經取與舊說別異處發問,答不失指者,童入學,生補廩。會一請令生童冊報考試經解,別期發問,不在冊報者,不概補經解。下部議行。會一以江南文勝,風以質行。嘗謁東林道南祠,刻小學頒示士子。處士是鏡廬墓隱舜山,親訪之,薦於朝。侍郎方苞屏居清涼山,徒步造訪,執弟子禮。校文詳慎,士林悅服。十三年,轉吏部,仍留學政任。力疾按試,至松江,卒。遺疏請任賢納諫。巡撫雅爾哈善奏準入名宦祠。

子嘉銓,自舉人授刑部主事,再遷郎中。授山東濟東道,再仙甘肅布政使。改大理寺卿,休致。乾隆四十六年,上巡幸保定,嘉銓遣其子齎奏,為會一乞謚;又請以湯斌、範文程、李光地、顧八代、張伯行及會一從祀孔子廟。上責其謬妄,逮至京師親鞫之,坐極刑,改絞死。上以嘉銓自著年譜,載與刑部簽商緩決,並稱大學士為「相國」,又編本朝名臣言行錄,屢降旨深斥之。

王恕,字中安,四川銅梁人。康熙六十年進士,改庶吉士。雍正元年,吏部以員外郎缺員,請以庶吉士揀補,恕與焉。旋自員外郎遷郎中。考選廣西道御史。轉兵科給事中。出為江南江安糧道,再遷廣東布政使。乾隆五年,署福建巡撫。上諭之曰:「勉力務實,勿粉飾外觀。封疆大吏不可徒自立無過之地,遂謂可保祿全身也。」旋奏:「臣到任數月,官方民俗,積儲兵防,已得其大略。漳、泉素刁悍,已嚴諭有司勤為聽斷,力行整刷。民俗尚華靡,督臣德沛以儉樸化民,臣更當倡導為助。合省常平倉穀,至四年歲終,共存一百三十四萬,又收捐監穀十五萬,委道府切實察覈。」報聞。六年,奏言:「臺灣各縣最稱難治。於繁缺知縣內揀選調補,多以處分被駁。請嗣後調臺官員,雖有經徵承追各案,準予題調。」上諭曰:「用此定例則不可,隨本奏請則可。」又奏:「各鄉社穀向俱借存寺廟,請於四鄉村鎮適中處分建倉房,工費即以社穀撥充,俟將來續收補項。」又奏免崇安無田浮賦一千二百五十一頃,及閩縣加徵無著學租。又奏:「福建多山田,零星合計成畝。嗣後民間開墾不及一畝,與雖及一畝而地角山頭不相毗連者,免其升科。」均從之。實授巡撫。

江蘇布政使安凝條奏賑務,上發各督撫察閱。恕疏言:「救災之法有三:曰賑,曰糶,曰借。此三者,實心辦理則益民,奉行不善則害政。以賑而論,地方有司於倉猝查報時,分極貧、次貧。一有差等,便啟弊端。里甲於此酬恩怨,胥役於此得上下,而民之冀幸而生觖望者,更不待言。蓋貧富易辨,極次難分。如以有田為次貧,無田為極貧,一遇旱澇,顆粒皆無,有田與無田等也。如以有家為次貧,無家為極貧,則無從得食,相忍守饑,完聚與煢獨同也。與其倉猝分別開爭競之門,莫如一視同仁絕覬覦之望。臣愚以為初賑似應一律散給,加賑再行分別,庶杜爭端。以糶而論,定例石減時價一錢,俾小民升斗易求,牙商居奇無望,誠接濟良法。乃有司每多請過減,倘輕聽準行,勢必希圖多糶,規利者雲集喧囂。且米價太賤,商販不前。臣請嗣後平糶,仍照定例斟酌辦理,使災民實沾升斗之惠,而棍徒囤戶難行冒濫之奸。以借而論,動公家之銀,為百姓謀有無、通匱乏,此周官恤貧遺法也。然使辦理未協,則官民交累。假如荒年田土無力耕種,有司借給耔種,猶可穫時即償。若告貸銀米以給口食,則必計其能還而後與之,狡黠之流遂謂官有偏私,不免造謗生事。有司不得已略為變通,而無力還官,差拘徵比,民無安息。是始則借不能遍,因爭閧而被刑;繼則還不能清,迫追呼而更困。名為利民,實為病民。且年久不清,蒙恩豁免,帑項終歸無著。臣以為與其借而無償,莫如賑而不借。此皆當先事而熟籌者也。」報聞。旋以官按察使時刪改囚供,下吏部,召詣京師。上以恕居官賢否詢閩浙總督策楞,又命新任巡撫劉於義考察。策楞言「恕操守廉潔,老成持重,惟識力不能堅定」;於義亦言「恕廉潔,百姓俱稱安靜和平,絕無擾累。惟不能振作」。上謂兩奏皆至公之論。尋補浙江布政使。旋卒。

恕治事不茍。初授湖北糧道,押運赴淮,以船戶挾私鹽,自請總督糾劾。任江安糧道,整飭漕務尤有聲。充福建鄉試監臨,武生邱鵬飛以五經舉第一,士論不平,奏請覆試。尋察出實使其弟代作,吏議降調,上特寬之。

子汝璧,字鎮之。乾隆三十一年進士,授吏部主事。累遷郎中。出為直隸順德知府,調保定。因承審建昌盜馬十未親鞫,奪官戍軍臺。尋準贖罪,降授同知,署直隸宣化府同知。累擢大名道。嘉慶四年,擢山東按察使。五年,遷江蘇布政使。六年,護理巡撫。旋授安徽巡撫。七年,請增設潁州督捕同知。湖廣總督吳熊光等奏湖廣需兵米,請於安徽糴十萬石。上以安徽方缺雨,令酌量。汝璧奏:「湖廣軍需事要,當如數撥運。請視嘉慶二年例,先運六萬石。」如所請。尋奏太湖續報成災,請緩徵,並劾府縣勘報遲延。上以督撫查辦災賑,於奏報後續行查出災區,往往回護屬吏,將小民疾苦置之不問。汝璧獨據實參奏,因深嘉之。八年,召授內閣學士,擢禮部侍郎。旋復授安徽巡撫。九年,召授兵部侍郎,調刑部。因病,請解任。十一年,卒。

汝璧兄汝嘉,後汝璧六年成進士,官檢討。

方顯,字周謨,湖南巴陵人。自歲貢生授湘鄉教諭,稍遷廣西恭城知縣。雍正四年,詔諸行省舉賢能吏,布政使黃叔琬以顯應,超擢貴州鎮遠知府。值歲饑,捐俸煮粥食饑民,民頌之。總督鄂爾泰議開苗疆,改土歸流,雲南東川、烏蒙、鎮雄諸土府既內屬,貴州苗未服。貴州苗大者,南曰古州、曰八寨,西南曰丹江,東北曰九股、曰清水江。九股、清水江界鎮遠,丹江界凱里,八寨界都勻,古州界黎平,參錯萬山中,地方三千里,眾數十萬,恆出剽掠。鄂爾泰召顯問狀,顯力言宜如雲南例改土歸流。問剿與撫宜孰施,對曰:「二者宜並施。第先撫後剿,既剿則仍歸於撫耳。」因條上十六事,曰:別良頑,審先後,禁騷擾,耐繁難,防邀截,戒姑息,宥脅從,除漢奸,繳軍器,編戶口,輕錢糧,簡條約,設重兵,建城垣,分塘汛,疏河道,各為之說甚備,鄂爾泰韙之。檄按察使張廣泗招撫古州、丹江、八寨諸苗,而以九股、清水江諸苗屬顯。

六年,顯自梁上進次挨磨、者磨,再進次柏枝坪,宣諭諸苗,撫定清水江生苗十六寨、九股屬臺拱生苗數寨。冬,廣泗已戡定丹江,顯續招清水江生苗七寨、九股屬陶賴生苗十三寨。施秉有盜匿臺拱農二寨,副將張尚謨捕不得,欲屠之。苗懼,逃林谷,將為變。顯聞之,曰:「如此則諸苗人人自危。」獨馳入苗寨,寨空無人,顯則宿寨中。翌旦,張蓋出,令從者繞林谷呼苗出,撫諭之曰:「汝曹速歸寨即良民,天子必不殺良民。」苗感泣,相率歸寨。顯益宿寨中三日,苗縛施秉盜以獻。七年三月,廣泗以清水江南岸諸寨尚懷觀望,檄顯與尚謨率兵循北岸徼巡。次柳羅,南岸公鵝、柳利、雞擺尾諸寨苗渡江來攻,顯督兵御之,殺數十人。苗眾師寡,尚謨欲引退,顯不可,固守待援。廣泗師至,圍乃解。廣泗用顯議,散諸寨,專攻公鵝,破之,諸寨皆聽命。鄂爾泰奏置貴東道,即以命顯,仍駐兵清江。顯申軍令,誓將士毋掠,毋淫,毋踐田穀,苗民有來愬者,為處其曲直;乃益築城郭,建官廨,治砲臺營房,苗民競來助役。九年,諸工竟。顯巡行視塘汛,黔、楚商船上下相接,苗民皆悅服。事粗定,尋授顯按察使。

臺拱者,苗中扼要地也,鄂爾泰議置營於此。十年,巡撫張廣泗奏請顯董其事。秋,羊翁、烏羅、桃賴諸寨苗為亂,九股諸苗附之,攻臺拱。顯與總兵趙文英嚴為備,擊走之。進破羊翁寨。苗夜至,顯以兵少,令人爇兩炷香手之為火繩狀以怖苗,苗走,退踞排略。排略者臺拱隘,我師餉道所必經。臺拱師僅二千五百人,苗數萬,援兵再敗。自賊始攻,或欲棄之走,顯拒之。及圍久糧盡,宰馬以食,迫冬寒,眾洶洶不自保,議潰圍退保下秉。顯曰:「臺拱失,古州、清江諸寨皆煽動。茍免,失臣節;撓敗,損國威。事急,死此耳。」眾感奮,會總兵霍升援至。苗奪我後山,樵路絕,顯夜出兵奪以還。苗攻益急,顯怒馬擊之,眾殊死戰,苗敗走。乘勝拔烏孟、井底二寨,取米穀餉軍。升兵亦克大關入,顯率兵出夾擊,苗大潰。凡堅守六十九日而圍解。提督哈元生師繼至,破蓮花悍苗。九股苗復定。自鄂爾泰議開貴州苗疆,事發於廣泗,而策決於顯,卒終始其事,崎嶇前後七年而事集。

乾隆元年,丁母憂,去官。三年,服除,授四川布政使。四年,署巡撫。大小金川、雜穀、梭磨、沃日、革布什咱諸土司相仇殺,顯遣人諭之,事稍解。議者欲乘此視雲南、貴州例,令改土歸流。顯疏言:「雜穀、梭磨,吐番後裔,其巢穴在唐為維州,戶口十餘萬。金川與接壤,戶口不過數萬。雜穀憚金川之強,金川則畏雜穀之眾,彼此鉗制,邊境乂安。固不可任其爭競,亦不可強其和協。沿邊生番,留之可為內地捍衛。從前川省調用土兵,亦供徵發。至其同類操戈,原未擾及內地。前經化誨,亦尚凜遵。設欲改土歸流,非惟彈丸土司無裨尺寸,且所給印信號紙,一經追取,即成無統屬之生番。稍有違抗,又費經營。」奏入,上以所見甚是,褒之,寢前議未行。旋與總督鄂彌達、提督鄭文煥疏言小金川與雜穀、梭磨畫界,以所侵必色多六寨歸雜穀、梭磨;又與沃日畫界,以隴堡等三寨隸沃日,美因等二寨隸小金川。大金川與革布什咱二土司構爭,檄建昌道李學裕開諭,革布什咱建轉經樓詛大金川,令即毀除,大金川亦歸所侵蓋古地。邊外諸土司亂悉平。

郭羅克番為亂,走匿色利溝,遣兵圍捕,土酋蒙柯縱使走。顯令總兵潘紹周按治,奏聞,上諭曰:「此等事汝固應就近料理,亦當與總督熟商。」總督,黃廷桂也。四川亂民號侂嚕子,為民害。顯疏言:「四川自明末兵燹,屠戮殆盡。我朝戡定後,各省移民來者多失業之民,奸頑叢集。有所謂侂嚕子,結連黨羽,暗藏刀斧,晝夜盜劫。臣嚴諭捕治,並令編保甲,整塘汛,以清其源。」得旨:「實力奉行,毋視為虛文。」

五年,授廣西巡撫。時顯方病目,聞命赴新任,上嘉其急公。旋請回籍調理,上慰留之。六年,顯病目未愈,命太醫院選眼科馳往醫治。尋以疾亟,請告回裡。卒。

顯蒞政明而恕。文煥嘗奏顯「爽直坦白,政治勤敏,遇事彼此悉心商榷,推誠共濟」。上嘉文煥論甚正。顯嘗奏薦學裕,因及夔州知府崔景俊「賦性巧滑,以其悛改,姑從寬恕」。上諭曰:「似此考察屬吏,且宥過錄長,得用人之要矣。」

桂,顯子,字友蘭。從顯平貴州苗有功,議敘。父喪終,以知縣發廣東,補英德,調潮陽。以善折獄名。舉卓異,擢雲南昆陽知州,署安寧。乾隆二十年,擢臨安知府,署澂江。調東川,丁母憂。服除,授甘肅鞏昌知府。鞏昌及平涼、慶陽三府饑,詔發西安籓庫銀六十萬治賑,大吏檄桂任其事。至平涼,饑民待食急,適部撥城工銀三十萬先至,桂以便宜留治賑,饑民賴以全。三十三年,遷浙江寧紹臺道。故事,定海戰艦九歲更造,則移致寧波船廠,取其值輸之官,名曰「折變」。奉檄裁戰艦,桂請視時值倍之,部駁坐短估,戍伊犁。三十七年,放還。卒。

馮光裕,字叔益,山西代州人。康熙五十年舉人。雍正元年,以薦授雲南大姚知縣。大姚賦少而耗重,積逋數萬。光裕不取耗,視負尤多者薄責之,逋賦悉清。民以耗重故,輒寄大戶造偽券占田,吏毀其籍。光裕檢毀未盡者藏之,按牒辨其偽,歸田故主,民尤頌之。遷貴州銅仁同知,赴闕引見。時古州苗方亂,世宗詢及之,光裕對苗不可盡殺,宜隨機化導,令歸版圖,上韙其言。既行,擢思州知府,未任,改雲南永北。永北介金沙江外,與四川連界,苗、惈窟其中,有事則兩界相諉。總督鄂爾泰命往勘,光裕輕騎往,惈從谷中出,挺刃相向。光裕策馬前,諭以利害,惈羅拜聽命,各散去。鄂爾泰疏請改知麗江,仍兼理永北事。未幾,擢驛鹽道。八年,東川、烏蒙惈叛,鄂爾泰檄光裕會鎮將討平之,擢按察使。烏蒙俘七千人,語不可通,譯者面謾莫能詰。光裕集群譯於使院,分室居之,訊一人,經數譯乃得其情。惈姓名多同,為編次年貌,驗決無誤,省釋者甚眾。廣西州民李天保以邪教聚眾殆千人,檄光裕按治。光裕曰:「愚民茹蔬奉佛,非有異志。」薄其罪,焚籍,置不問。

十一年,擢貴州布政使。十三年,古州苗叛,都江、清江、八寨、丹江、臺拱諸新附苗皆應。師討之,光裕督餉,令民應役,厚與直,行得持械自衛。募熟苗為助,畀以木符,戒官兵無妄殺,皆踴躍應募。師集十餘萬,皆得宿飽。軍罷,民被兵者無所棲止,給草舍居之,賦以衣食,復業者二十餘萬戶。貴州賦銀八萬八千、米十五萬五千,光裕奏請蠲免。高宗即位,命被兵地停徵三年。又奏:「古州、丹江諸苗剿除殆盡,荒田空寨,遠近相望。當募民居苗寨,墾苗田,設屯置衛,行保甲法,授降苗所納軍器,俾農隙講肄,以壯聲援、省餽餉。」得旨允行。

乾隆四年,擢湖南巡撫。鎮筸紅苗叛,光裕督兵捕治,不三月而平。疾,乞假,聞城步、綏寧苗復勾結粵瑤為亂,密咨兩廣總督籌協捕。尋卒。遺疏猶言:「二縣困於兵,請免今年租。」上從其請。

子祁,乾隆二年進士,官編修。孫廷丞,舉人,以廕生授光祿寺署正,官至湖北按察使。

楊錫紱,字方來,江西清江人。雍正五年進士,授吏部主事。累遷郎中。考選貴州道御史。十年,授廣東肇羅道。肇慶瀕海,藉圍基衛田。歲親蒞修築,終任無水患。乾隆元年,署廣西布政使,尋實授。請禁州縣以土產餽上官。六年,授廣西巡撫。貴州土苗石金元為亂,焚永從縣治。會貴州、湖廣兵剿擒之。既而遷江土苗復為亂,謀犯思恩府。檄兵往捕,得其渠李尚彩及其黨八十餘。七年,奏言:「廣西未行保甲。苗、僮雖殊種,多聚族而居,原有頭人,略諳事體。請因其舊制,寓以稽覈。苗、瑤、伶、僮各就其俗為變通。」詔嘉之。尋又奏言:「設兵以衛民,乃反以累民:城守兵欺凌負販,攫取薪蔬;塘汛兵驅役村莊,恣為飲博。臣於撫標訪察懲治,請敕封疆大臣共相釐剔。」得旨允行。八年,梧州知府戴肇名餽人葠,詭其名曰「長生果」,卻之,具以聞,上諭曰:「汝可謂不愧四知矣。」廣西民有逃入安南者,捕得下諸獄,疏聞,上命重處,錫紱即杖殺之。上諭曰:「朕前批示,令其具讞明正典刑。乃錫紱誤會,即斃杖下。此皆當死罪人,設使不應死者死,則死者不可復生矣。」下部議處。九年,授禮部侍郎。

十年,授湖南巡撫。奏言:「周禮:遂人治野,百里之間,為澮者一,為洫者百,為溝者萬,捐膏腴之地以為溝洫。誠以蓄洩有時,則旱潦不為患,所棄小、所利大也。後世阡陌既開,溝洫雖廢,然陂澤池塘尚與田畝相依,近水則腴,遠水則瘠。湖南濱臨洞庭,愚民昧於遠計,往往廢水利而圖田工。甚至數畝之塘,培土改田;一灣之澗,絕流種蓺。彼徒狃於雨暘時若,以為無害;不知偶值旱澇,得不償失。且溪澗之水,遠近所資,若截墾為田,則上溢下漫,無不受累。官吏以改則升科為勸墾之功,亦復貪利忘害,溝洫遂致盡廢。臣以為關系水利,當以地予水而後水不為害,田亦受益。請敕各省督撫,凡有池塘陂澤處所,嚴禁改墾。」上以各省米價騰貴,諭各督撫體察陳奏,錫紱疏言:「米貴由於積漸。上諭謂處處積貯,年年採買,民間所出,半入倉庾,此為米貴之一端。臣生長鄉村,世勤耕作,見康熙間石不過二三錢,雍正間需四五錢,今則五六錢。戶口多則需穀多,價亦逐漸加增。國初人經離亂,俗尚樸醇。數十年後,漸習奢靡,揭借為常,力田不給。甫屆冬春,農糴於巿,穀乃愈乏。承平既久,地值日高,貧民賣田。既賣無力復買,田歸富戶十之五六。富戶穀不輕售,巿者多而售者寡,其值安得不增?臣以為生齒滋繁,無可議者。田歸富戶,非均田不可,今難以施行。風俗奢靡,止可徐徐化導,不能遽收其效。至常平積貯,當以足敷賑濟而止,不必過多。目今養民之政,尤宜專意講求水利,使蓄洩有備,偏災不能為患。以期產穀之多,未必非補救米貴之一道也。」疏入,上均嘉納焉。丁父憂,服闋,十五年,授刑部侍郎,仍授湖南巡撫。丁母憂,服闋,十八年,仍授湖南巡撫。擢左都御史。十九年,署吏部尚書。禮部侍郎張泰開保同部侍郎鄒一桂子志伊為國子監學錄,下吏部議處,議未當,責錫紱曲庇,下都察院,議奪官,命留任。二十年,復署湖南巡撫,授禮部尚書。二十一年,署山東巡撫。

二十二年,授漕運總督,疏請豁興武、江淮二衛旗丁欠繳漕項,上責其沽名,命以養廉代償。二十三年,疏言:「屯田取贖,宜寬年限。價百金以上,許三年交價,價足田即歸船。旗丁交兌不足,名曰『掛欠』。應由坐糧限追懲治,督運官以下有一丁掛欠,即停其議敘,旗丁改僉。新丁但交篷桅槓索價值;舊丁公私欠項,不得勒新丁接受。水次兌漕,令倉役執斛,旗丁執概。江淮、興武二衛運丁運糧,快丁駕船。應循例並僉,不得避運就快。」上諭曰:「此奏確有所見。」下部議,從之。二十五年,疏言:「自開中河,漕艘得避黃河之險。獨江北、長淮等幫,以在徐州交兌,不能避險。請令改泊皁河,弁丁詣徐州受兌。州縣代雇剝船轉運過壩。」上從之。尋以錫紱實心治事,命免以養廉代償漕項。二十六年,疏言:「運薊州糧船自寧河轉入寶坻,由白龍港、劉家莊達薊州。水道淤淺,請責成官為疏濬。」又疏言:「板閘、臨清、天津三關,尚沿明制,漕艘給發限單,應請裁革。州縣收漕如有攙雜潮潤,糧道察出,本管知府視徇庇劣員例議處。軍丁兼充書役,一體句僉。頭舵水手受雇,領費輒復潛逃,請發邊遠充軍。」上諭曰:「所奏俱可行。」從之。加太子少師。二十八年,加太子太保。二十九年,疏言:「軍、民戶籍各分,既隸軍籍,即應聽僉辦運。乃宦家富戶百計圖避,所僉皆無力窮民,情理未得其平。嗣後如僉報後辨訴審虛,參劾治罪。」上諭曰:「錫紱此奏,破瞻徇之習。如所議行。」並下部議敘。又疏言:「糧艘例禁私鹽。道經揚州,總督、鹽政及臣各專委督察。乃又有淮揚道,揚州游擊、守備,江都、甘泉兩縣,各差兵役搜查,糧艘因之羈阻。如江廣幫為通漕殿後,過揚州已在冬令,尤為苦累。臣思事權宜歸於一,請專聽總督、鹽政委員督察,餘悉停止。」上諭曰:「所奏是。」下部議行。三十年,疏言:「駱馬湖蓄水,相傳專濟江廣重運。今歲幫船阻滯,先開柳園堤口,運河水長,江浙幫遂得遄行。次開王家溝口,江廣幫至,湖水未嘗告竭。每歲沂水自湖而下,為海州、沭陽水患。若於四五月間引湖濟運,亦減海州、沭陽水患,一舉兩利。」從之。三十三年,卒,賜祭葬,謚勤愨。

錫紱官漕督十二年,編輯漕運全書,黃登賢代為漕督,表上之。自後任漕政者,上輒命遵錫紱舊章。

潘思矩,字絜方,江南陽湖人。雍正二年進士,改庶吉士。三年,分刑部學習。六年,補主事。累遷郎中。八年,授廣東南雄知府。驟雨水溢,郊野成巨浸,露宿於野。督吏卒治筏拯溺,出金瘞死贍生,活民無算。十三年,遷海南道。濬瓊州西湖。深入五指山,安輯黎眾,劾守將之殘黎民者。調糧驛道。乾隆四年,遷按察使。懲貪鉏猾,理冤獄尤多。民以旱糾眾入巿掠奪,思矩方被疾,強起坐堂皇,立捕數十人杖以徇,事乃定。疏言:「廣東有俍、瑤、黎三種:俍世居茂名,今附民籍,讀書應試如平民。瑤亦輸稅歸誠,設瑤童義學為訓課。惟黎僻處海南,崖、儋、萬、陵水、昌化、感恩、定安七州縣為最多。生黎居深山,熟黎錯居民間相往來,語言相習,請於此七州縣視瑤童例設義學,擇師教誨,能通文義者許應試。」部議從之。

七年,遷浙江布政使。八年,疏言:「常平倉穀春發秋斂。但收成有遲早,俗所謂青黃不接。有司不揆緩急,甫春開糶,牙行囤積,吏胥侵漁。民未霑實惠,而穀已出逾額,且減價過多。迨秋成買補,非存價觀望,冀省耗折;即抑派爭買,致昂巿價。請定浙東諸府以四月、浙西諸府以六月發糶,價平即止。」上以因時制宜,許之。又疏言:「浙江土狹民稠,全資溪湖容蓄灌溉,乃民間占墾甚多。如餘杭南湖,會稽鑒湖,上虞夏蓋湖,餘姚汝仇湖,慈谿慈湖,向稱巨浸,今已彌望田疇,殊妨水利。嗣後報墾田地,當責有司親勘,果非官湖,方準升科;查勘不實,嚴定處分。」下廷臣議行。秋,金、衢、嚴三府被水,旁溢杭、湖、紹三府,漂流人畜無算。思矩出臨江幹,處分賑事。蕭山民洶洶欲渡江,思矩曰:「民饑當哺,閧則亂民耳。」嚴治之,自是無敢譁者。思矩再疏聞,上諭曰:「今歲浙江災,巡撫常安有諱災之意,汝為其難矣。」

十一年,授安徽巡撫。河決鳳陽,潁、泗諸府州災尤重。思矩請加賑,按行督察,犯風渡洪澤,舟幾覆。十二年,疏請調濟災區,略言:「鳳、潁民習惰窳。臣上年遍歷查勘,方冬水落,二麥已播種,而民不知耰鋤培壅。所過村落,林木甚稀,蔬圃亦少。臣令有司審察桑麻、蔬蓏,凡可佐小民日食之用者,隨宜試種。鳳、潁地分三等,岡地最高,湖地稍低,灣地最下。灣地連大河,水發難施人力。湖地則外仰中低,積潦為湖,下流疏洩,即可涸出栽種。岡地水雖不及,而絕少溝池,交秋缺雨,即患又乾。間有傍山麓而為陂塘,如壽州安豐塘、懷遠郭陂塘、鳳陽六塘,均應及時修築。與其因災動帑鉅萬,何如平時酌動數百金陸續培治。民間減荒歉,多收成,朝廷亦省帑金。縱遇偏災,亦可以工代賑。鳳、潁民好轉徙,豐年秋成事畢,二麥已種,輒攜家外出,春熟方歸。遇災留一二人在家領賑,餘又潛往鄰境。俗謂在家領賑為大糧,在外留養為小糧,沿途資送為行糧,至有一家領三糧者。本業拋荒,人無固志。應令有司嚴察,流民過境,實系被災,方準資送;藉端生事者究懲。」奏入,上諭曰:「此乃固本之事,歷來無有言及此者。朕甚嘉悅焉!」

尋調福建巡撫。未行,疏請安徽學田、囚田、義田三項,視江蘇免學租例,予以蠲免。下軍機大臣察議,以江蘇無免學租例,上責思矩沽名干譽,博去後之思,命出資修涿州城工示罰。十三年,疏言:「福建自乾隆元年至十一年積欠錢糧,正設法清釐。民間田業授受,往往不及推糧過割。糧從田出,既有賠糧之戶;即有無糧之田,豈可使得業者任其脫漏,無業者代其追比?當飭有司確察,務使糧歸於田。」十四年,復疏言:「臣清察積欠,一在屯田戶名不清,一在寺田租賦不一。自順治間裁並衛所,名雖軍戶,實系民耕,乃糧冊仍列故軍姓名,致難催比,應令覈實更正。寺田始自明季,僧、民相雜,輒稱寺廢僧逃,藉詞逋賦,應令分析寺已廢者,官為經理。」上命實力為之。別疏言:「福州城外西湖為東晉郡守嚴高所開,周二十餘里,蓄水溉田,年久淤墊。臣勸導疏濬,並築堤建閘。又福清郎官港、法海埔俱有海灘淤地,臣令築堤招墾,得地二千一百餘畝。」上獎諭之。

思矩蒞政精勤,晝見官屬,夜披案牘。旱潦必撫恤。民獷,以鬥訟相尚,多去為盜,廉得主名,飭有司捕治。又以農隙巡行海防,周閱戰艦。朔望入書院與諸生講說經藝,如是者以為常。積勞疾作不少止。十七年,卒。上命用江蘇巡撫徐士林例,祀京師賢良祠。予恤視一品,賜祭葬,謚敏惠。

胡寶瑔,字泰舒,江南歙縣人。父廷對,嘗官婁縣訓導,因居青浦。寶泉,雍正元年舉人。乾隆二年,考授內閣中書,充軍機處章京。六年,大學士查郎阿、侍郎阿里袞清察黑龍江、吉林烏喇開墾地畝,以寶瑔從。八年,遷侍讀,考選福建道御史。是歲直隸旱,上命治賑。寶瑔疏言:「直隸被旱,民多流亡,請敕總督宣示上意,使民靜以待賑。流民原歸耕而無力得歸者,資送還里,俾及時藝麥,於來歲民食有益。」九年,上命大學士訥親閱河南、山東、江南諸省營伍,寶瑔疏言:「營伍積玩,器械堅脆,糧馬盈虧,各處不一。勢必聞風修整買補,不肖營員或藉端苛派,或坐扣月糧,請敕督撫提鎮嚴飭查察。」十年,山東、江南水災。寶瑔疏言:「方冬水涸,應勸諭農民引流赴壑,俾田不久浸,以便春耕,尤當預防蝻子。」諸疏皆議行。十一年,轉戶科給事中,遷順天府府丞。大學士傅恆視師金川,以寶瑔從。授府尹,歷宗人府丞、左副都御史。擢兵部侍郎,兼府尹如故。河南民傅毓俊告張天重謀逆,遣寶瑔按治,毓俊服誣,論如律。

十七年,署山西巡撫,十八年,實授。撫饑民,理冤獄,劾貪吏,整關隘是防,諸政並舉。尋調湖南。十九年,奏言:「郴、桂二州銅鉛礦委員董理,一年而代。礦為弊藪,代者必數月乃能明察。此數月中,欺蒙隱漏,已自不少。請仿臺灣、瓊州例,令新舊協辦數月。」得旨允行。

二十年,調江西。二十一年,疏言:「廣信銅塘山勘明無可墾之地,無可用之材,無可煎之礦,請永行封禁。」二十二年,疏言:「豐城堤工最要,石是官修,土堤民修,向設里夫,行之已久。黠者避役,貧者誤工,復改為折徵。請按田均堤,附漕糧徵收。有田始有糧,有糧始有夫。圩長無從侵冒,工程乃可永固。」均如所請行。

復調河南。河屢決,山東、河南、安徽諸州縣多積水。上遣侍郎裘曰修會諸省督撫疏治。寶瑔與曰修會勘,疏言:「河南幹河有四:賈魯、惠濟、渦河、巴溝。巴溝在商丘為豐樂河,在夏邑為響河,在永城為巴河。今擬疏濬加寬深,以最低處為率。惠濟上游在中牟、祥符諸縣,下游在柘城、鹿邑諸縣,今亦擬加寬深,以六七丈為率。賈魯自中牟以下有惠濟分流,自硃仙鎮以下,截沙灣,塞決口,拓舊堤。渦河自通許青岡為燕城河,上游應加寬,下游應加深。鹿邑以下本已寬深,當增築月堤。支河應濬者,商丘北沙、洪溝二河為支中之幹,餘大小支河,分要工、次工、緩工,次第興修。」二十三年,上諭曰:「河南災區積困,寶瑔不辭勞瘁,能體朕意,盡力調劑,以蘇窮民,甚可嘉也!」尋加太子少傅。諸工皆竟,上禦制中州治河碑,褒寶瑔、曰修,語並見曰修傳。

二十五年,疏言:「河北諸水,衛河為大。雍正間,河督嵇曾筠於汲、淇、濬、湯陰、內黃諸縣建草壩二十六,今已漸次淤墊。臣相度疏築,俾一律深通。請定為三年一小修、五年一大修。」上可其奏。是冬,調江西。二十六年,河決楊橋。復調還河南。疏言:「賈魯、惠濟二河在中牟境內,逼近楊橋。賈魯受黃水南徙,至祥符時家岡仍入故道,今已成河。當將分者截之使合,淺者疏之使深,兩岸多挑渠港,增築堤堰,自成河道。惠濟自兩閘至岡頭橋已淤斷,而岡頭橋至十里坡賈魯河不過四五里。即於十里坡建滾水壩,導由岡頭橋入惠濟,以分賈魯之勢,而惠濟亦復故道。」上褒為事半功倍。

二十七年,寶瑔疾作,請解任。上諭曰:「此奏甚非朕之所望,安心靜攝,以慰廑念。」遣醫馳驛診視。疏言:「溝渠與河道相為表裏,臣於二十三年河工告竣,即督令州縣經理溝洫,每一州縣中開溝自十數道至百數十道,長自里許至數十里,寬自數尺至數丈,皆以足資蓄洩為度。驛路通衢,並就道傍開濬,雖道里綿亙,而分戶承挑,民易為力。自是每歲或春融,或農隙,隨時加濬寬深。」上深嘉之,並令直隸總督方觀承仿行。二十八年,卒,加太子太保、兵部尚書,賜祭葬,謚恪靖。遺疏請入籍青浦,許之。

論曰:那蘇圖、士林、恕、思矩皆以清節著,而超曾、寶瑔又濟之以勤敏。恕論救災,寶瑔善行水,皆以民事為急。顯佐定苗疆,有拊循之績。錫紱督漕運,所修舉似若瑣細,然皆當官之急務也。會一澤以道學,但微近名,遂貽後嗣之禍,恫哉!


\end{pinyinscope}