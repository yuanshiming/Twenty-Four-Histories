\article{列傳一百四十六}

\begin{pinyinscope}
岳起荊道乾謝啟昆李殿圖張師誠王紹蘭

李奕疇錢楷和舜武

岳起,鄂濟氏,滿洲鑲白旗人。乾隆三十六年舉人,議敘,授筆帖式。累擢戶部員外郎、翰林院侍講學士、詹事府少詹事。五十六年,遷奉天府尹。前官貪黷,嶽起至,屋宇器用遍洗滌之,曰:「勿染其污跡也!」與將軍忤。逾年,擢內閣學士,尋出為江西布政使。殫心民事,值水災,行勘圩堤,落水致疾。詔嘉其勤,許解任養痾。

嘉慶四年,特起授山東布政使。未幾,擢江蘇巡撫。清介自矢,僮僕僅數人,出屏騶從,禁游船聲伎,無事不許宴賓演劇。吳下奢俗為之一變。疏陳漕弊,略曰:「京漕積習相因,惟弊是營。米數之盈絀,米色之純雜,竟置不問。旗丁領運,無處不以米為挾制,即無處不以賄為通融。推原其故,沿途之抑勒,由旗丁之有幫費;旗丁之索幫費,由州縣之浮收。除弊當絕其源,嚴禁浮收,實絕弊源之首。請下有漕各省,列款指明,嚴行禁革,俾旗丁及漕運倉場,無從更生觀望冀幸之心。」詔嘉其實心除弊。常州知府胡觀瀾結交鹽政徵瑞長隨高柏林,派捐修葺江陰廣福寺。嶽起疏言觀瀾、柏林雖罷逐,尚不足服眾心,請將錢二萬餘串責二人分償,以修蘇州官塘橋路。丹徒知縣黎誕登諷士紳臚其政績保留,實不職,劾罷之。

五年,署兩江總督。劾南河工員莊剛、劉普等侵漁舞弊,莫澐於任所設店肆運貨至工居奇網利,並治如律。揚州關溢額稅銀不入私,盡以報解;覈減兩籓司耗羨閒款,實存銀數報部:並下部議行。六年,疏請濬築毛城鋪以下河道堤岸、上游永城洪河、下游蕭、碭境內河堰,並借帑舉工,分五年計畝徵還,允之。

八年,入覲,以疾留京,署禮部侍郎。會孝淑皇后奉移山陵,坐會疏措語不經,革職留任。尋命解署職,遂卒。帝深惜之,贈太子少保,賜恤如例。

無子,詔問其家產,僅屋四間、田七十六畝。故事,旗員歿無嗣者產入官。以岳起家清貧,留贍其妻;妻歿,官為管業,以為祭掃修墳之資。異數也。妻亦嚴正,嶽起為巡撫時,一日親往籍畢沅家。暮歸,飲酒微醺。妻正色曰:「畢公躭於酒色,不保其家,君方畏戒之不暇,乃復效彼耶?」嶽起謝之。及至京,居無邸舍,病歿於僧寺,妻紡績以終。吳民尤思其德,呼曰嶽青天,演為歌謠,謂可繼湯斌云。

荊道乾,字健中,山西臨晉人。乾隆二十四年舉人,大挑知縣,官湖南,歷麻陽、龍山、東安、永順、慈利、靖州。所至有惠政,屏陋規,平冤獄。在靖州賑饑,尤多全活,屢膺上考。四十七年,遷甘肅寧夏同知,入覲,大學士劉墉曾官湖南巡撫,稱之曰:「第一清官也。」名始著。尋署石峰堡同知,時方用兵,治事不廢,修復水利,復薦卓異記名。五十四年,擢安徽池州知府,屢署徽寧池太道,筦蕪湖關,贏餘不入己,以充賑恤。調安慶,硃珪為巡撫,尤信任之,疏薦,擢山東登萊青道,攝布政使。以激濁揚清為己任,薦廉吏崔映淮、李如珩等,而劾不飭者。

嘉慶二年,遷按察使。四年,遷江蘇布政使。先是州縣存留俸薪役食及驛站經費,改解籓庫,俟奏銷後請支,始則防吏侵挪,久之解有浮費,發有短平。或勒抵前官虧空,佐雜教官不能得俸,驛傳領於臬司;或苛駁案牘,因索餽遺,郵政日弛廢。道乾入覲時,面陳其弊,請悉依定章,於州縣徵收時開支,省解領之繁。仁宗俞可;至是疏上施行,天下便之。上方欲整飭漕政,以巡撫嶽起及道乾皆有清名,責其肅清諸弊。到官三閱月,擢安徽巡撫,疏請禁徵漕浮收舊耗米一斗,給運丁五升,加給二升。運丁所得,有據可考;其所用沿途浮費,採訪知之,應禁革。詔下所奏於有漕各省永禁。又言:「屯田所以贍運,每丁派田若干及應得租耔,新僉旗丁不能了然。令糧道刊刻木榜,俾僉丁認田收租。運船領款,刻易知單,由丁正身親領,以杜包領欺壓之弊。田冊歸糧道收管,另造副冊發各衛以備查驗。」並允行。宿州、靈壁、泗州水災,道乾親往監視賑廠。六年,以病乞罷,詔許解任調理,俟病痊來京候簡。次年三月,詔詢道乾病狀,已先卒於安慶,帝悼惜,賜祭,賜其孫炆舉人。

道乾由監司不三年擢至巡撫,求治益急,不避嫌怨,自處刻苦。臨歿,呼舊僚至寢所,指床下金示之曰:「吾受重恩,積養廉數千兩,足以歸喪。諸君素愛我,勿為斂賻。」又呼其兄曰:「兄仁弱,勿聽人慫恿受賻,違吾意。」兄如其言。

謝啟昆,字蘊山,江西南康人。乾隆二十六年進士,朝考第一,選庶吉士,授編修。典河南鄉試,分校禮闈,均得士。三十七年,出為江蘇鎮江知府,調揚州。明於吏事,所持堅正,上官異意不為奪。治東臺徐述夔詩詞悖逆獄遲緩,褫職戍軍臺。尋捐復原官,留江南。父憂,奪情署安徽寧國知府;復遭母憂,服闋,稱病久不出。五十五年,特擢江南河庫道,遷浙江按察使。六十年,遷山西布政使。州縣倉庫積虧八十餘萬,不一歲悉補完。高宗異其才,以浙江財賦地虧尤多,特調任。歷三歲,亦彌補十之五。

嘉慶四年,擢廣西巡撫。上疏,略曰:「各省倉庫積弊有三變。始則大吏貪婪者利州縣之餽賂,僨事者資州縣之攤賠。州縣匿其私橐,以公帑應之,離任則虧空累累。大吏既餌其資助,不得不抑勒後任接收。此虧空之緣起也。繼則大吏庸闇者任其欺蒙,姑息者又懼興大獄,以敢接虧空為能員,以稟揭虧空為多事。州縣且有藉多虧挾制上司升遷美缺者。此虧空之濫觴也。近年不職督撫相繼敗露,諸大吏共相濯磨,州縣亦爭先彌補。但彌補之法,寬則生玩,胥吏因緣為奸;急則張皇,百姓先受其累。各省貧富不同,難易迥別,一法立即一弊生,惟在因地制宜。率定章程,又多窒礙。請飭下各省先查實虧之數、原虧之人,如律論治。其無著者,詳記檔案,使猾吏無可影射。多分年限,使後任量力補苴,不必展轉株求,亦不必程功旦夕。責成督撫裁陋規以清其源,倡節儉以絕其流,講求愛民之術以培元氣,獎擢清廉之員以勵官常。日計不足,月計有餘。不數年間,休養生息,不徒倉庫充盈,吏治民生亦蒸蒸日上。廣西自孫士毅經營安南,軍需供億,所費不貲,米銀裝械,毀棄關外,令州縣分賠,遂致通省皆虧。本非州縣侵蝕,且人已去任,接收者正在補苴,一經參追,難保不勸捐派累。惟率司、道、府、州省衣節食,革去一切陋規,俾州縣從容彌補,進廉去貪,無累百姓,計三年之內,庫項必可補足。惟是數十人補之而不足,一二人敗之而有餘。是又在知人善任,大法小廉,不愛逢迎,不存姑息,庶不致後有續虧之患。」又言:「彌補虧空,初不為一身免累之計,乃有實際。臣前歷山西、浙江,皆未咨部,亦未咨追原籍。蓋當日之員,大半死亡遣戍,子孫貧乏者多,咨追徒滋紛擾,如數完繳者實無二三,現任反置身事外。廣西庫項未完者三十九州縣,覈其廉數多寡,分限三年,按月交庫,於交代時有不足者,即以虧空論劾。」疏入,仁宗嘉納焉。時詔買補倉穀,取諸豐稔鄰縣,禁於本境採買。啟昆言廣西跬步皆山,轉運不減於穀價,恐不肖者因採買之難,或為勒派,請仍聽本境買補便,詔如所議。

廣西土司四十有六,生計日絀,貸於客民,輒以田產準折。啟昆請禁重利盤剝,違者治罪。田產給還土司,其無力回贖者,俟收田租滿一本一利,田歸原主,五年為斷;其不禁客民入苗地者,廉土民馴愚,物產稀少,藉販運以通有無也。仿浙江海塘竹簍囊石之法,修築興安陡河石堤,以除水患。河流深通,舊銅船過陡河必一月,至是三日而畢。七年,卒於官,詔嘉其廉潔,於所節省潯、梧兩關盈餘項下賜銀三千兩治喪。廣西士民請祀名宦祠。

啟昆少以文學名,博聞強識,尤善為詩。著樹經堂集、西魏書、小學考,晚成廣西通志,為世所稱。

李殿圖,字桓符,直隸高陽人。乾隆三十一年進士,選庶吉士,授編修。典湖南鄉試,遷御史。督廣西學政,遷給事中。

四十九年,甘肅回亂,從阿桂、福康安赴軍治糧餉、臺站,授鞏秦階道。軍事初竣,民、回相仇,焚掠報復,訛言時起。殿圖處以鎮靜,叛黨緣坐,婦稚量情釋宥;罹害戶口,隨宜賑恤,流亡漸安。卓泥土司與四川松潘、漳臘各番爭噶噶固山界,殿圖輕騎履勘,歷小洮河、丈八嶺、鸚哥口,皆人跡罕到,群番導行,片語判決,立石達魚山頂而還。高宗幾餘考涇、渭清濁源流,命殿圖親勘,自秦州溯流至鳥鼠、崆峒,繪圖附說以進,詔嘉其詳實。

六十年,遷福建按察使,嘉慶三年,就遷布政使。疏言:「乾隆中,業農家必畜騾馬三四以任耕種,嗣後官吏借用應差,漸形滋擾,應嚴行革除。獄訟必速為審結,開釋無辜,小民始得安業。常平倉穀積久弊生,民未受益,官倉已受其虧。無災之年,不宜貸假。吏役例有定額,近則人思託足,藉免役徭。關津稅口,官署長隨,呼朋引類,並為奸藪,宜並禁止。」詔下直省一體察禁。閩俗售田,田面田根,糾纏不決。蠹吏影射,佃戶頑抗,錢糧日多脫欠,徵收不敷,每以虛出通關而致虧缺,殿圖奏請嚴治。在任逾年,庫儲大增。

擢安徽巡撫,七年,調福建。有林、陳、藍、胡諸大姓糾眾械斗,治如律。治海盜三腳虎及蔡牽羽黨,請祀海洋陣亡官兵,緝匪死事者一體入祀,從之。十一年,蔡牽久未平,仁宗以臺灣剿捕事殷,殿圖操守尚好,軍務未嫺,調江西巡撫。尋詔斥殿圖於軍事無所陳奏,又不能禁止海口偷漏水米火藥,降四五品京堂;又以所屬久羈案犯,以中允、贊善降補。尋遷翰林院侍講,引病歸。十七年,卒。光緒初,閩浙總督文煜疏陳殿圖前任福建政績昭著,謚文肅。

張師誠,字蘭渚,浙江歸安人。乾隆中,南巡,召試賜舉人,授內閣中書,充軍機章京。遷吏部主事,忤和珅,緣事降中書。得應會試,五十五年,成進士,改庶吉士,授編修。嘉慶元年,出為山西蒲州知府,歷雁平道,河南、江蘇按察使,遷山西布政使。州縣倉庫多虧,師誠知清查有名無實,特嚴於交代之際,有虧必完,在任三年,庫儲充裕。十一年,擢江西巡撫,以兼提督賜花翎,遂著為令。尋調福建,清治淹牘,疏陳整頓積弊事宜,詔嘉勉。

時海盜蔡牽、硃濆方猖獗,總督玉德廢弛黜去,阿林保繼任,復與提督李長庚不協;師誠至,始嚴防海口,杜岸奸接濟,籌備船械,長庚得盡力剿捕。是年冬,長庚追蔡牽於粵洋,以傷殞。牽犯臺灣後山噶仔蘭,為生番擊退,請收其地入版籍,免為賊踞。十三年,硃濆與牽有隙,獨竄閩洋,總兵許松年擊斃之。其弟渥,勢蹙思投首,會道員德華由臺灣內渡,遇牽黨圍劫,渥救之,藉以通款,尋復拒敵粵師不果降。十四年,阿林保調兩江,師誠暫署總督。聞蔡牽竄浙洋,親駐廈門,提督王得祿、邱良功合剿,毀盜舟,牽墮海死。硃渥尋率三千餘人歸誠,赦其罪,海疆以安,閩人刊石烏石山以紀功。海寇稽誅久,由閩、浙不能合力,自師誠治閩,而阮元復蒞浙,始告成功。仁宗嘉其嚴斷接濟,為殄寇之本。京察特予獎敘。

十九年,調江蘇。百齡為總督,諸巡撫皆承望風旨,師誠獨舉其職。初彭齡奉命同查虧帑,意與百齡、師誠不合,遂劾兩人皆受餽遺,而不得實,詔原之。會百齡窮治逆書獄,閭閻悚息,巡撫所主五府州得無擾。川沙民有燒香傳徒者,有司密捕解江寧,師誠遣標弁要於途,交按察司依律鞫治,免辜磔者數十人,時以稱之。二十一年,父病篤,不俟代回籍,被嚴議褫職。尋予編修,服闋,遷中允。歷江西、安徽布政使。道光元年,擢廣東巡撫,調安徽,繼母憂去官。復歷山西、江蘇巡撫。六年,召授倉場侍郎。以病乞歸,卒於家。

師誠警敏綜覈,在當時疆吏中有能名,治福建最著,繼之者為王紹蘭。

紹蘭,字南陔,浙江蕭山人。乾隆五十八年進士,授福建南屏知縣,調閩縣。巡撫汪志伊薦其治行,仁宗曰:「王紹蘭好官,朕早聞其名。」召入見,以知州用,擢泉州知府。漳、泉兩郡多械斗,自紹蘭治泉州,民俗漸馴,而漳州守令以械斗獄獲罪,詔舉紹蘭以為法。擢興泉永道,捕獲蔡牽養子蔡三及其黨蔡昌等,予議敘。遷按察使,母憂去,服闋,起故官,就遷布政使。嘉慶十九年,擢巡撫,始終未出福建。尋汪志伊來為總督,與布政使李賡蕓不合,因訐告受賂,劾治,屬吏希指羅織,賡蕓憤而自縊。志伊獲譴,紹蘭坐不能匡正,牽連罷職。

少嗜學,究經史大義。去官後,一意著述,以許慎、鄭康成為宗,於儀禮、說文致力尤深,著書皆可傳。

李奕疇,字書年,河南夏邑人。乾隆四十五年進士,選庶吉士,授檢討。大考改禮部主事,典貴州鄉試,洊遷郎中。五十七年,出為山西寧武知府,調平陽,有政聲。歷江蘇糧道、山東按察使。嘉慶十一年,坐巡撫保薦屬吏違例,牽連被議,左遷江南河庫道。

十三年,遷安徽按察使,治獄明慎,多平反。霍丘民範受之者,贅於顧氏,與妻反目,外出久不歸。縣令誤聽訛言,謂其妻私於鄰楊三,鍛鍊成獄,當顧氏、楊三謀殺罪,其母與弟及傭工某加功,實無左證,五人者不勝刑,皆誣服。奕疇閱供詞,疑之,驟詰曰:「爾曹言骨已被焚,然尚有臟腑腸胃,棄之何所?」囚不能對,惟伏地哭。奕疇慨然曰:「是有冤!」使幹吏偵之,至陳姓家,言正月十五夜受之曾過宿,而讞曰被殺在十三日,乃緩系諸囚,嚴緝受之。久之,受之忽自歸,則以負博遠避,不敢使家人知所在,今始聞大獄起,乃歸投案也。事得白。奕疇故無子,獄既解,乃生子銘皖。民間傳頌,至演為劇曲。就遷布政使。

十八年,擢浙江巡撫。時近畿教匪未靖,或言嚴、衢兩郡匪徒傳習天罡會,詔奕疇嚴治。奕疇逮訊葉機、姚漢楫等,實止愚民相聚誦經祈福,無逆跡,坐罪首犯數人,株連皆省釋。安徽、江西游民來浙租山墾種者日眾,言官請禁。奕疇疏陳勢難遽逐,請分年遣令回籍。上悟曰:「茲事不易言。游民皆無恆產,驅之此省,又轉徙他省,斷不能復歸鄉里。」命徐謀教養,俾流亡者變為土著,乃得安。

尋授漕運總督,在任五年,運務無誤。奕疇固長者,待下寬,坐濫委運弁降四級,命以吏、禮二部郎中用。復以運弁縱容幫丁索費,被劾,降主事。二十五年,宣宗即位,命奕疇以尚書守護昌陵。道光二年,原品休致。十九年,重宴鹿鳴,加太子少保。明年,會榜重逢,子銘皖適登第,同與恩榮宴,稱盛事焉。二十四年,卒,年九十有一。

錢楷,字裴山,浙江嘉興人。乾隆五十四年進士,選翰林院庶吉士,散館改戶部主事,充軍機章京。嘉慶三年,典四川鄉試,督廣西學政,回京,仍直軍機。遷禮部郎中,調刑部,甚被眷遇。截取京察當外用,予升銜留任。十一年,詔嘉楷久直勤勉,以四五品京堂用。歷太常寺少卿、光祿寺卿。十二年,京師旱,疏請循漢書求雨閉陽縱陰之說,停止正陽門外石路工程,詔「修省在實政,無事傅會五行」,罷其奏。迭命往河南、山西鞫獄,次第奏結,無枉縱。授河南布政使,十四年,護理巡撫,暫署河東河道總督。擢授廣西巡撫,尋調湖北。

十六年,疏言:「外洋鴉片煙入中國,奸商巧為夾帶。凡粵東西兩省匪類糾結,多由於此,以致盜風益熾。請飭閩、粵各關監督並近海督撫,嚴督關員盤檢,按律加等究辦。內地貨賣一經發覺,窮究買自何人,來從何處,不得含糊搪塞,將失察偷漏監督委員及地方官一體參處,務使來蹤盡絕,流弊自除,乃清理匪源之一端也。」詔下沿海督撫認真察辦。授戶部侍郎,兼管錢法堂事。奏陳湖北地方事宜應酌劑者四端:請附近荊州糧米供支滿營兵食,餘俱改歸北漕;沿江契買洲地,準其耕種納糧,無契者作為官地,召佃承種;新設提督,移駐襄陽府城;楚北均食淮鹽,襄陽、宜昌等府籌議減價。下所司會議,惟沿江洲地一事照行,餘以窒礙置之。

復出署河南巡撫。匪徒王胯子句結南陽饑民滋事,成大獄。楷至任,疏言:「前任巡撫恩長於南陽匪徒一案,前後具奏情節與原報不符,辦理過當。府、州、縣等緝犯並未廢弛,平日聲名尚好,現擬絞監候之二十餘犯,明年秋審,均應情實,不敢知而不言。」詔以「句決與否,臨時自有權衡,非臣下所可豫定。地方官咎有應得,豈能開復?」斥楷敷陳未當,近於喜事。調補工部侍郎。尋授安徽巡撫。以歙縣監生張良璧採生斃命,命楷親訊,讞擬未依凌遲律,失於輕比,部議降一級調用,改降二級留任。十七年,卒。詔以「楷直樞曹久,有勞,自簡封圻,治理安靜。母程年逾七旬,嗣子尚幼,深憫之,特賜恤。」

和舜武,伊拉里氏,滿洲鑲藍旗人。官學生,考授太常寺筆帖式。累遷步軍統領衙門員外郎。以治獄明獲議敘,遷兵部郎中,兼公中佐領。嘉慶十五年,出為江蘇鹽法道。累遷山東布政使,整飭吏治,輿論歸之。二十二年,擢山西巡撫,調河南。會布政使吳邦慶疏請於漳、衛合流之處建閘壩,和舜武謂:「漳河盛漲湍悍,非一閘所能御,越閘旁趨,且停蓄泥沙,塞衛水宣洩之路。」疏請罷之,仍舊章每年挑濬竇公河以資鹽運,如所議行。逾年,調山東。仁宗聞其前為布政使有聲,故有此授。山東民俗好訟,又近畿,輒走訴京師。和舜武再蒞,訟頓減,特詔褒勉。疏請清理京控積案,責巡撫、籓、臬分提鞫訊,月定課程,各自陳奏;又請酌改竊盜窩匪條例,加重定擬,俟盜風稍戢,復舊:並從之。至年終,審結積案千餘起,予優敘。京察復予議敘。二十四年,卒,上甚惜之,優詔賜恤,贈總督銜,謚恭慎。

論曰:仁宗初政,特重廉吏。嶽起、荊道乾清操實政為之冠;謝啟昆、張師誠才猷建樹,卓越一時:並專圻碩望矣。李殿圖、李奕疇、錢楷亦各以明慎慈惠見稱,和舜武課最簿書,遂邀易名曠典;王紹蘭一眚坐廢,晚成經學:殆有幸有不幸哉?


\end{pinyinscope}