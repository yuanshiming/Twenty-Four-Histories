\article{志四十一}

\begin{pinyinscope}
地理十三

△江西

江西省:禹貢揚州之域。明置江西巡撫,承宣布政使司,南贛巡撫。清初因之。順治四年,置江南河南江西總督。治江寧。六年,罷河南不轄。九年,移治南昌。尋還舊治。十八年,置江西總督。康熙四年復故。先於三年裁南贛巡撫,為永制。乾隆八年,吉安增置蓮花。十九年,升贛州寧都縣為直隸州。三十八年,升贛州定南縣為。光緒二十九年,改贛州觀音閣通判為虔南。三十三年,改銅鼓營為,屬瑞州。東至安徽婺源縣;六百五十八里。南至廣東連平州;一千三百里。西至湖南瀏陽縣;四百二十里。北至湖北黃梅縣。三百四十里。廣九百七十七里,袤一千八百二十里。北極高二十四度十七分至二十九度五十八分。京師偏東一度五十五分至偏西二度二十四分。宣統三年,編戶三百四十三萬九千八百七十三,口一千三百五十二萬七千二十九。凡領府十三,直隸州一,四,州一,縣七十四。驛道達各省者五:一,北渡江達湖北黃梅;一,東南逾杉關達福建光澤;一,東逾懷玉山達浙江常山;一,南逾大庾嶺達廣東南雄;一,西出插嶺達湖南醴陵。鐵路擬築者四:曰南潯鐵路,自九江而南昌而吉安而贛南,凡三段,備與廣東接,此外道瑞、袁通湘,道撫、建通閩,道廣、信通浙。為支路亦三。航路則九江為江輪停泊之埠。電線自南昌北通九江,南通廣州;又自九江東通蕪湖,西通漢口。

南昌府:沖,繁,難。隸糧儲道。江西巡撫,布政、提學、提法三司,糧儲、巡警、勸業三道駐。東北距京師三千二百四十五里。廣四百四十里,袤四百二十五里。北極高二十八度三十七分。京師偏西三十七分。領州一,縣七。南昌沖,繁,難。倚。東南:麥山、漸山。南:斜山、虎山。西:贛江,一曰章江,自豐城入,經巿汊汛,歧為二。一東北行,會撫河,仍合經流與東湖通,東北逕蛟溪入新建。東:武陽水,即旴水,西北行,入進賢。萬公堤。竿韶鎮。三江口、巿汊二巡司。一驛:巿汊。新建沖,繁,難。倚。西:西山,古曰散原,亙奉新、建昌諸縣境。西南:逍遙。北:松門。西北:銅山。西南:贛江,上承劍江,自豐城入,流逕瑞河口,蜀江自高安來會。東北行,逕吳城,修水自建昌來會,流合鄱陽湖。經星子,出湖口,入大江。湖即彭蠡也。全湖跨南昌、饒州、南康、九江四府境,為省境諸水所匯。南潯鐵路起沙井。巡司駐生米鎮,同知駐吳城鎮。樵舍、海口、吉山、望湖亭、後河、白馬六汛。新興廢驛。豐城沖,繁,難。府南一百三十里。西南:豐城山,縣以此名,道書「三十七福地」之一。東:鍾山。西南:澄山。西北:馬鞍山。南:羅山,富水出;桮山,豐水出。贛江自清江入,流逕縣西,東北行,豐水南來合富水會焉。又東北入南昌、新建。東:雩韶水,來自臨川,西北行,入贛江。松湖、港口、曲江三鎮。大江口巡司。一驛:劍江。進賢沖。府東南一百二十里。城內三臺山。西南:麻山。東:棲賢。西:烏英山、金山。旴水自南昌入,東北行,逕縣西,抵八字,入鄱陽湖。西有軍山湖、日月湖、青嵐湖,俱流會三陽水入鄱陽湖。巡司駐梅莊。有鄔子廢司。奉新沖。府西北一百十五里。北:登高山。西南:華林。東南:岐山。西北:藥王山。西:百丈山,馮水出,左合側潭水,右合金港源,又東南至九梓鋪,納龍頭溪、白水、華林水,至城南。又東納鳴溪三溪,入安義界。巡司駐羅坊。靖安簡。府西北一百五十五里。北:吳(甜心)山。西北:金城、葛仙山;桃源山,桃源水所出,流合雙溪。雙溪一曰南河,源出義寧毛竹山,合龍頭坳、管家坳、委源諸水,逕高湖,歧為二,環城南北,至鴨婆潭復合。一曰北河,出雙坑洞,合爛草湖、大橫溪,逕象湖入安義,會馮水。龍頭坳鎮。武寧繁,難。府西北三百五十里。西北:遼山。西南:大孤。東:遼東山。修水自義寧入,右合洋湖水,左合青坪水,至城南,納楊浦、鳳口水,又西合磧溪、箬溪、魯溪、中黃、三水,入建昌。縣丞駐木高。箬溪水汛。太平陸汛。高坪巿巡司。義寧州繁,疲,難。府西北三百五十五里。原名寧州,嘉慶三年改。東南:毛竹山。東:旌陽。西:九龍。西北:黃龍、幕阜二山。修水出東南,左合百菖水、杭口水,右東津水,至城西。武寧鄉水出大溈山,合東鄉水及鹿源水自南來會,又西折北至城東,合安平水、鶴源水,入武寧。查津,同知駐。八疊嶺鎮有巡司,與排埠塘、杉巿為三。

饒州府:沖,繁,疲,難。隸饒九道。西南距省治三百六十里。廣四百八十七里,袤三百四十里。北極高二十八度五十九分。京師偏東一十一分。沿明制,領縣七。鄱陽沖,繁,難。倚。北:芝山。東:郭璞山。西:堯山。南:關山。鄱陽湖,西南,鄱江匯焉。有二源,一自安徽祁門來,歷浮梁為昌江,一自安徽婺源來,歷德興、樂平為樂安江。流會城東,環城北出,歧為雙港,分注鄱陽湖。東有東湖,一名督軍,流合鄱江。汛八:八字、團轉、強山、館驛前、黃龍廟、樂安河、螺螄嘴、棠陰。石門巡司。芝山驛。餘干沖,難。府東南一百十里。西:藏山。東北:萬重、武陵。西南:李梅。東南:黃山。西北有康郎山,在鄱陽湖中,湖因名康郎。龍窟河一曰安仁江,自安仁入,流經潼口灘,歧為二,西北行,三餘諸水入焉,流抵饒河口,俱入鄱陽湖。康山、黃丘埠、瑞洪三鎮。康山、梅溪、表恩、高溪四汛。瑞洪,縣丞駐。一驛:龍津,裁。樂平繁,難。府東一百十七里。東:康山。北:鳳游。西:吳溪。東南:石城山。婺江自德興入,為樂安江,合長樂水、建節水、吳溪、殷河,流逕縣南樂安鄉,因名。西南流入萬年界。仙鶴、八澗二鎮。康山驛。浮梁沖,繁,難。府東北一百八十七里。北:孔阜山。東:芭蕉。西:金魚。西南:陽麻山。東南:大游、小游山。昌江自安徽祁門入,合小北港、苦竹坑水、磨刀港,流逕城南,西南行,會歷降水、柳家灣水,入鄱陽。景德、桃樹二鎮。巡司駐景德。德興沖,繁。府東二百三十七里。東:銀山、銅山。西北:洪雅。東南:大茅山。大溪自安徽婺源入,建節水自弋陽入,合樂平之桐山港、洎山之洎水,為樂安江,並入樂平。白沙巡司。銀峰驛。安仁沖。府東南一百八十里。東:張古山。北:蟠象山。東北:青山。西南:積煙。西北:華山。安仁江上源為上饒江,自貴溪入,合玉石澗,逕城南,西北行,與白塔河會,至城西北合藍溪,入餘干。萬年難。府東南一百二十里。城北萬年山,縣以此名。西南:團湖。西:託里。西北:軸山。東南:百丈嶺,殷河出,合文溪、南溪,入樂平,注樂安江,又西入鄱陽。巡司駐石頭街。

廣信府:沖,繁,難。隸廣饒九道。西北距省治五百六十里。廣四百二十五里,袤三百六十里。北極高二十八度二十七分。京師偏東一度三十八分。沿明制,領縣七。上饒沖,繁,疲,難。倚。北:茶山。西:銅山。南:銅塘山。東南:鐵山、南屏山。上饒江自玉山入,流逕城南,左納永豐溪,右合櫧溪,其南有岑陽關,自福建崇安入,又西北並入鉛山。鄭家坊、八房場二巡司。葛陽驛。玉山沖,繁,疲,難。府東一百里。西:回龍。南:武安山。北:三清、懷玉山。上乾溪有二源,一出三清山冰玉洞,一自浙江常山來會,合下乾溪,逕城南,為玉溪。又西行,右合侖溪、沙溪,入上饒,為上饒江。縣丞駐營盤要口,巡司駐太平橋。一驛:懷玉,裁。弋陽沖,難。府西一百三十里。南:龜峰、軍陽山。東:搗藥山。上饒江自鉛山入,逕黃沙港,合大洲溪,右瀆自興安入,合弋溪,縣以此名。又西逕城南,左合軍陽水,右納葛溪,西行入貴溪。大橋汛。縣丞駐漆工鎮。一驛:葛溪,裁。貴溪沖,繁,難。府西二百五里。南:龍虎山。西:自鳴山。西南:象山。北:百丈嶺。貴溪一名薌溪,上流為弋陽江,自弋陽入,流逕縣南,納須溪、箬溪,又西合惠安溪、橫石港,會上清溪,西行入安仁。有火燒關,與福建光澤界。縣丞駐江滸山。上清鎮、鷹潭鎮巡司二。薌溪廢驛。鉛山沖,繁。府西南八十里。西:鉛山,縣以此名。北:鵝湖。南:鳳凰。西南:銅寶。西北:芙蓉山。上饒江自上饒入,流逕縣西,至汭口,桐木、紫溪諸水合為汭口水注焉,西行入弋陽。其東大洲溪自上饒入從之。分水、溫林、桐木、雲際並有關。紫溪、河口二鎮。湖坊鎮巡司。河口,縣丞駐。鵝湖驛。廣豐繁,難。府東南五十五里。舊名永豐,雍正九年改。西:鶴山。西北:覆泉。東:雙門、三巖山、念青嶺。南:平洋山。永豐溪自福建浦城入,合銅鋍、封禁諸山水,又北,左合永平溪,折西南,逕城南,流至水南渡,合西橋諸水,入上饒江。拓陽鎮。巡司駐洋口。興安簡。府西八十五里。北:橫峰山、重山。南:赭亭山。西:仙巖。葛溪自上饒來,南行注弋陽江,合黃藤港水,又西入弋陽。

南康府:沖。隸饒廣九南道。南距省治二百四十里。廣三百里,袤一百十里。北極高二十九度三十一分。京師偏西二十五分。沿明制,領縣四。星子簡。倚。西南有黃龍山。西北有廬山,硃子知南康軍講學處。北:吳章。東北:定山。鄱陽湖在縣城外,贛江經焉,北行至都昌。又北入德化。南落星湖,東宮亭湖,鄱陽湖之隨地異名者也。谷簾水自德安入,東南行,入鄱陽湖。諸磯、青山、謝師塘、岡窯四汛。渚溪、青山二巡司。都昌疲,難。府東六十里。西:元辰山,道書「五十一福地」。東:陽儲。北:檀樹。東北:篁竹山。鄱陽湖在西,其中有強山、四望、松門諸山,北流入湖口。北通後港河,自左蠡石流嘴引入,至徐家埠,又北匯西洋橋水入湖口。柴棚、左蠡二鎮。棠陰、黃金嘴、豬婆山、左蠡四汛。縣丞駐張家嶺。巡司駐周溪。一驛:團山,裁。建昌沖,繁。府西南一百二十里。北:將軍。西:越山。西南:長山。馮水自安義入,至城南曰南河,流合修水。修水自武寧入,至縣西曰西河,右合桃花水、雲門水,左合白楊港水、白水,東北入新建注贛江。蘆埠、河滸二鎮。南潯鐵路。安義沖,繁。府西南二百里。南:文山。東:西山。西:臺山。北:馬山。馮水自奉新入,左納雙溪,右合兆州水,至閔房分流復合,東北匯洪泉水,入建昌。龍江水、東陽、新逕水俱自靖安入,流注修水。

九江府:沖,繁,難。饒廣九南道治所。九江鎮總兵駐。南距省治三百里。廣四百十里,袤七十里。北極高二十九度五十二分。京師偏西二十四分。沿明制,領縣五。德化沖,繁,疲,難。倚。南:廬山,道書「第八洞天」,又與虎溪為「七十二福地」之二。西南:柴桑。東南:天花井山。東南:鄱陽湖,大孤山在其中。湖東北流,行至德化界。大江右瀆自瑞昌入,曰潯陽江,流逕城西北,東行合湓水,其支津瀦為城門、金雞、鶴向諸湖,會龍開河,東逕白石磯入湖口。西南:黃河、潘溪入德安。東:女兒港,源出廬山,東北流,入鄱陽湖。城西有鈔關。商場:咸豐十一年開,在西門外。南潯鐵路止龍開河。巡司三:大姑塘、小池口、城子鎮。汛三。一驛:通遠。潯陽驛,裁。德安沖,繁。府西南一百二十里。南有博陽山,古敷淺原,前有博陽水。北:孤山。西北:望夫山。谷簾水出康王谷,東南流,入星子。廬山河一名東河,出廬山烏龍潭,西北流。黃河一名北河,出高良山,南流。西河出苦竹源,東南流。三河合於縣東北之烏石門,曰三港口。南潯鐵路所經。瑞昌簡。府西七十里。南:榜山。西:白龍。西北:蘇山、鴉髻。西南:清湓山,清湓水出焉,流逕城西,南行入德化。西:瀼溪,西南流入大江。江在縣北,自湖北興國入,納下曹湖、赤湖,東行入德化。西南:傅陽水,出小坳,東南入德安。黃土巖水出大坳,西北入興國。北有梁公堤。巡司駐肇陳口。湖口沖,繁,難。府東六十里。長江水師總兵駐。東:武山。西南:旗山。南:上石鐘山。北:下石鐘山。山西岸為梅家洲,鄱陽湖挾贛江由此入大江。江來自德化,納清水港、太平關水,東北行,入彭澤。北有長虹堤,水師中營駐。汛十一:上下石鐘、洋港、大王廟、馬家灣、梅家洲、龍潭、柘磯、八里江、白滸塘、老洲頭。鎮四:流撕橋、湖口、柘磯、茭石磯。一驛:彭蠡,裁。彭澤沖,繁。府東北一百五十里。南:龍游山。東南:浩山。北:小孤山,山在江中,江畔彭浪磯,與山對峙。東北:馬當山,枕大江。江自湖口入,納馬埠水,瀦為筲箕港、黃土港,其支津太平關水,入湖口。又東合六口水,至馬當山麓,入安徽望江,東流。巡司駐馬當。汛六:馬當、小孤洑、北風套、芙蓉墩、陸口、金剛料。一驛:龍城,裁。

建昌府:繁,疲,難。隸督糧道。西北距省治三百六十里。廣二百二十五里,袤三百七十里。北極高二十七度三十四分。京師偏東一十一分。沿明制,領縣五。南城沖,繁,難。倚。城內高空山。東:斂山。西:雲蓋。東北:白馬。西南:麻姑山。旴江一曰建昌江,自南豐入,納彭武溪、斤竹澗,逕城東,合黎灘水、飛猿水,為東江,東北行,入金谿。巡司駐新豐。藍田、洑牛二鎮。新城沖,繁。府東南一百二十里。西:日山。南:福山。東:飛猨嶺。西南:紅水嶺,黎灘水出,一名中川,合九折水,至南津,左合七星澗,右合九龍潭,逕城南。又西北至港口,左會龍安水,至公口入南城。飛猨水一名東川,源自濟源杉嶺,周湖並下,西北行,入南城,合黎灘水,為東江。龍安水一名西川,出會仙峰,東北流,注黎灘水。石峽、龍安、五福三鎮。極高、同安二巡司。南豐繁,疲,難。府南一百二十里。西:軍山。東:大龍。南:石龍。東南:百丈嶺。旴江自廣昌入,左合瞿溪、灑溪,右九劇水,逕城南而東,合蔓草湖、雙港、梓港,入南城。盤州、黃沙、白舍、龍池、仙居五鎮。龍池、羅坊二巡司。廣昌難。府南二百四十里。西北:金嶂。西南:望軍山。東:中華。南:翔鳳。東南:血木嶺,旴水出焉,西北,右合庚坊、文會、大凌諸港,至南門外,亦曰西大川。又北逕城東,左合學溪、青銅港,入瀘溪。白水、頭陂二汛。白水、秀嶺二巡司。瀘溪簡。府東北百五十里。南:蓮花山。東:石筍。西:魚山。西南:雲溪。東南:五鳳山。瀘溪自福建光澤入,屈西北至石陂渡,合見溪,逕城北至三溪口,左合南港水,折東北逕高阜,右合嚴槎港、稅溪,入貴溪為上清溪。有瀘溪鎮。

撫州府:繁,疲,難。隸督糧道。北距省治二百十里。廣三百七十五里,袤三百里。北極高二十七度五十六分。京師偏西十分。沿明制,領縣六。臨川沖,繁,難。倚。城內香柟山。西:銅山。東:英巨。北:筆架。南:戚姑。東南:靈谷山。汝水即旴江,一曰撫河,自金谿入,西北行,合臨水,入南昌界注贛江。臨水自崇仁入,東北行,會宜黃水,逕縣北注汝水。北有千金陂。航步鎮。巡司一駐溫家圳,一駐東館。孔家驛,裁。金谿繁。府東南一百三十里。南:官山。西:柘岡。東:銀山、金窟山;雲林山,跨撫、建、信三府境。旴水自南城入,逕明山港,合清江,亦曰石門水,會金溪、苦竹水,入臨川。清江水出福建光澤,至縣之清江橋,曰清江水,西北流,合旴水。三港水出崖山,合青田港、仙巖港,入東鄉。一鎮:許灣,縣丞駐。崇仁繁,難。府西九十里。西有崇仁山,縣以此名。東南:沸湖。南:華蓋山、相山。東:仙游。北:櫟山。臨水即寶塘水,自樂安入,逕嚴陀寨,巴水會焉,折東合羅山右水。西:寧水,逕城南,又東合羅山左水及青水,右孤嶺水,至白鷺渡入臨川,與宜黃水會。巡司駐鳳岡墟。宜黃繁,難。府西南一百二十里。城北隅鳳臺山。南:黃山。北:曹山。西南有黃土嶺,黃水出焉。東南有軍山,宜水出焉。二水合流曰宜黃水。合藍水、曹水,合於城東,入臨川,注臨水。巡司駐棠陰。樂安簡。府西南九十里。東南:鼇頭。西:仕山。北:萬靈山。東:芙蓉山,鼇溪水出,西合西華山水,至城東,合載興山、甑蓋山水,逕城南至負陂,合遠溪、大溪,入永豐。西北:大盤山,寶唐水出,東北合河源、蛟河等水,入崇仁為臨水。龍義鎮。招攜巡司。東鄉難。府東北八十里。東:七寶。北:五彩。西:槲山。東北:三港口水,匯花山港、太平橋水,西逕新陂,納齊岡水,入臨川注汝水。其金谿三港入為田埠水,緣安仁界入之。潤溪亦三源,合於巖前陂,北入餘干。鎮二:白玕、平塘。

臨江府:沖,繁。隸鹽法道。東北距省治二百十里。廣二百五十里,袤一百七十五里。北極高二十七度五十八分。京師偏西一度三分。沿明制,領縣四。清江沖,繁,疲。倚。西:章山。東:閤皁山。南:瑞筠山。贛江自新淦入,袁江自新喻入境,合上下橫河,繞郡城而北,為清江。自贛江北沖蛇溪江不復合,至城北二十里始復會焉。北行會蕭、淦諸水,入豐城。東有梅家畬堤。東北白公堤。樟樹鎮汛,通判駐。一驛:蕭灘,裁。新淦沖,繁。府南六十里。東湓山。南:楓岡山。西:百丈峰。東北:小廬。贛江自峽江入,合沂江、蘆嶺水、逆口溪,逕城西,又流逕縣西南,北行,左合桂湖,右金水,入清江。杯山巡司。金川廢驛。新喻繁,難。府西一百二十里。西南:銅山。北:蒙山。南:虎瞰山。袁江一曰渝水,自分宜入,左合嚴塘江,右板陂水,逕縣南,東行入。逕嚴潭,至城南,亦曰秀水。左合畫水、睦宦水,右納七里山水、麻田水,又東北入清江。西南:黃金水自廬陵入,入峽江。巡司駐水北墟。峽江沖。府西南一百三十里。西:鳳凰山。東:玉笥山。南:刀劍山。贛江自吉水入,納黃金水,北逕城南,亦曰峽江。又東北,右合玉澗水,左亭頭水,至烏石渡,納水匽水、象水、蓮花潭水,入新淦。沂江自新淦入,河源頭水、南源水環蜈蚣山仍入之。有峽江廢司。玉峽廢驛。

瑞州府:沖。隸鹽法道。順治初,沿明制。光緒三十三年,改銅鼓營為來屬。東北距省治一百二十里。廣二百二十五里,袤一百五里。北極高二十八度二十五分。京師偏西一度十一分。領縣三,一。高安沖,繁。倚。城內碧落山。東:大愚。西:鳳嶺。南:羊山。北:米山。西北:華林山。蜀江一名錦水,自上高入,東行至瑞河口之象牙潭,與贛江會。曲水出蒙山,南入贛江。岡嶺鎮。灰埠巡司。新昌疲,難。府西北一百二十里。西:黃。北:大姑嶺,相連為八疊嶺。西北:黃岡山。西南:錦水自上高入,左合長塍,東南至凌江口入上高。凌江一名鹽溪,源出逍遙、八疊二山,流逕城西,合滕江,注蜀江。巡司二:大姑嶺、黃岡洞。上高難。府西南一百二十里。北:敖山。南:蒙山。西南:米山。蜀江自萬載入,左合益樂水,右合雲江。又東北合凌江,又東南合六口水、斜口水,逕城南,又東北至洞口腦入高安。有離婁橋鎮。銅鼓簡。府西北二百二十里。光緒三十三年,裁都司營改置。西:大溈山,寧鄉水出,一名西河,下流注修水。排埠塘巡司。磉頭汛。

袁州府:沖,繁。分巡道治所萍鄉。原為鹽法道兼巡袁州、瑞州、臨江三府,駐南昌。光緒三十三年改專分巡加兵備銜,擬由南昌移治,南昌並屬焉。東北距省治四百八十里。廣三百里,袤二百八十里。北極高二十七度四十九分。京師偏西二度五分。沿明制,領縣四。宜春沖。倚。南:仰山。北:喝斷山。東:震山、萬勝岡。西南:望鳳山。袁江,古牽水,自萍鄉入,合鸞溪為稠江,至城北為秀江,右合清瀝江、九曲水,左渥江,入分宜。西北:滄江嶺水,入萍鄉。太平關鎮汛。黃圃巡司。分宜沖。府東八十里。東:鐘山。西:昌山。北:貴山。東北:雞足山。袁江西南自宜春入,逕城南為縣前江,東行出鐘山峽,入新喻。東北:竹橋水、麻田水、硯江,入安福、廬陵。貴山鎮。安仁廢驛。萍鄉沖,繁。府西一百四十里。西:徐仙山。南:筆架。東南:羅霄山,羅霄水出焉。分二支:一東流合牽水、渝水,折東逕宣鳳汛,入宜春,為袁江;一西流合泉江,逕城南,會羅霄西北水,折西北,逕湘東鎮,右合平溪嶺水,入湖南醴陵,注淥江。四鎮:宣風、蘆溪、上慄、插嶺關。巡司二:蘆溪巿、安樂。有草巿廢司。鐵路達湖南醴陵。萬載繁,難。府北九十里。北:龍山。東:東岐山。西:鐵山。西北:紫蓋山。龍江,古蜀水,源出縣西劍池,會別源缽盂塘水,東匯於金鐀湖。其西流者入湖南瀏陽。又東流至楮樹潭,合野豬河,至城北,會龍河。又西,左合康樂水,入上高界。巡司駐珠樹潭。

吉安府:沖,繁,疲,難。隸吉贛南寧道。順治初因之。乾隆八年,析永新西北境、安福西境,置蓮花。東北距省治四百八十里。廣五百里,袤三百九十里。北極高二十七度八分。京師偏西一度三十五分。沿明制,領縣九。廬陵沖,繁,疲,難。倚。西:天華。北:瑞華。東南:青原。南:神岡。北:螺子山。贛江自泰和入,納義昌水,東北逕廟前汛,廬水西自安福入,會永新禾水、邕水,逕神岡山來會。又北逕城東,合真君山,逕白鷺洲至螺子川,曰螺川。又東北合橫石水、西岡嶺水,並入吉水。固江、永陽、富田三巡司。泰和沖。府東南八十里。東:王山。西:武山。西北:傳擔山。贛江西南自萬安入,曰澄江,流抵縣之珠林口,雲亭江西北流來會,左合清溪,逕磯頭,右納仙槎江,其西北邕水並入廬陵。白羊凹鎮。馬家州巡司。吉水沖,繁。府東北四十五里。東:東山。南:天岳。北:王嶺。西北:朝元。東南:觀山。義昌水自永豐入,合盧江,至張家渡入廬陵。贛江又東北流,逕城西南合恩江,為吉文水,東北行,入峽江。盧江源出永豐,入境匯為盧陂,下流注贛江。阜田巡司。永豐疲,難。府東北一百三十五里。南:龍華。西:西華。北:巘山。東:郭山。西北:王嶺。恩江一名永豐水,出寧都及樂安、興國,流逕城東南,合葛溪、白水,會龍門江、義昌水,入吉水,注贛江。巡司三:層山、沙溪、表湖。有上固汛。安福繁,難。府西北一百二十里。東:蒙岡。南:南岡。北:鵝湖。西北:武功山,瀘水出,即古廬水,至城北,又東合智溪,折南至洋口,與王江合,入廬陵,會永新江入贛江。石鎮。蘿塘巡司。龍泉繁,疲,難。府西南二百五十里。西:石含。東:銀山。南:五峰。東南:錢塘山。遂水一曰龍泉江,源出左右溪。左溪一自湖南桂陽入,一自上猶入,至縣之左安而合。右溪出石含山,至李派渡合左溪,為遂水,東行入萬安。蜀水一名禾蜀,出縣北黃坳,東行至太和之瓦窯,入贛江。三鎮:禾源,其北鄉、秀州並有巡司,與左安三。萬安沖,繁。府東南一百八十里。西:芙蓉。東:蕉源。南:朝山。贛江自贛入,合梁口、造口及油田溪,逕城西南,合龍溪、檜溪,又北逕黃公灘納韶水,為韶口。又東北合城江,至窯塘入泰和。其西蜀水自龍泉入,亦入泰和界,西北有梅陂。巡司二:武索、灘頭。一驛:皁口,裁。永新繁,難。府西二百二十里。東:東華山。東北:高士。南:義山。西北:禾山,禾水出,一曰永新江,自蓮花入,東行繞縣,至白堡入廬陵。上坪寨巡司。永寧簡。府西二百八十里。東:旗山。西:漿山。西北:七谿嶺,勝業水出焉,西會拐湖山水,逕城南,又西會漿山水,折北逕小江山入永新,注禾川。巡司駐升鄉寨。蓮花沖,疲,難。府西二百六十里。乾隆八年置。西:關城山。東南:壺山。北:黃暘。東北:玉屏山。文匯江西北出萍鄉及湖南攸縣界,經高天巖,合上西、礱溪二鄉水,匯於龍陂。環城而西,合琴亭水,自馬江至礱山口注永新。東北:水雲洞水入萍鄉。西南:茶水出書堂嶺,入湖南。富漢村巡司。蓮花橋汛。

贛州府:沖,繁,疲,難。吉贛南寧道治所。南贛吉袁臨寧總兵駐。順治初,因明制,置南贛巡撫。康熙三年裁巡撫。乾隆十九年,升寧都縣為直隸州,割瑞金、石城隸之。三十八年,升定南縣為。光緒二十九年,改觀音閣通判為虔南。東北距省治九百三十里。廣三百三十里,袤五百六十里。北極高二十五度五十二分。京師偏西一度四十一分。領縣八。贛沖,繁,難。倚。南:崆峒山。東南:玉房。西南:九峰。北:儲山、黃唐。東北:金螺山。章水自南康入,東北行,逕城西,貢水自雩都入,西行逕城東,至魚尾潭,與章水合,是為贛江,贛闕在焉,古稱湖漢水,北行入萬安。十八灘,九隸縣境。鈔關在治北。長興、桂源、大湖江三巡司。水口、官村、良富、東塘四汛。雩都難。府東一百五十五里。東:峽山。北:雩山。西南:藥山。東南:柴侯山。貢水自會昌入,北逕齊茅汛,右合雷公嶂水,又西合垇腦廡水,至白石塘,合寧都水,入贛。興仁巡司。信豐繁,疲,難。府南一百六十里。西:木公山。東:長老。西北:廩山。桃江自龍南入,北行入境,為信豐江。東北行,合三江水,入贛注貢水。楊溪堡巡司。興國難。府東北一百八十里。西:玉山。東:崖石山。北:覆笥。東北:蜈蚣山。瀲江一名興國江,會平川,折南逕城東,又西,左合程水,右菏嶺廖屋溪、烏山嵊水,入贛注貢水。西:義昌水出虔公山,入永豐。北:雲亭江入泰和。衣錦寨巡司。均村、崖石二汛。會昌沖,繁。府東南三百二十里。南:四望山。北:明山。東:古方。東南:盤古山。貢水自瑞金入,會綿、濂、湘水,西行入雩都。東南:榮陽水出筠門嶺,入武平。有湘鄉、承鄉二鎮。筠門嶺巡司。安遠簡。府東南三百三十里。西:源華。北:鐵山。東南:馬鞍山。濂江一名安遠江,出長寧仰天湖,西逕城南,西北合欣山安遠水,縣以此名。東北行,逕古田,會上濂水,又北,左合里仁、小華江,右雲雷水,入會昌,匯湘水。三百坑水出三百山,西南流入定南。縣丞駐羅塘市,巡司駐板石鎮。長寧簡。府東南四百四十里。西:大帽山。西南:鈐山。北:官谿。東南:項山。尋鄔水出尋鄔堡新窖路山,屈東南,合馬伏崠水,又西南至城東大陂角,會馬踶江、河嶺水、太湖洞水,入廣東龍川。雙橋鎮。新坪、黃鄉堡二巡司。龍南簡。府南三百五十里。南:歸美。西:祿馬。東南:清修。西南:冬桃山,桃水出焉。東北行,逕城北,與濂、渥二水合,為三江口。又北合灑源堡水,逕龍頭山,入信豐,為桃江。定南繁,疲,難。府南四百三十五里。舊為縣。乾隆三十八年改置。城內文昌山。南:三臺。北:楊梅山。東北有劉■D3山。鶴子水上源即三百坑水,出自安遠,入為九曲河,逕九洲,合劉■D3隘水,至水口。右合汶嶺水,又北逕三溪口,合三坑水,入廣東龍川。咸水出南坑諸山,流抵龍南,會濂水,注桃江。下歷鎮有巡司。虔南繁,疲,難。府南四百五里。舊為觀音閣,通判治。光緒二十九年改置。桃水自龍南來,東行入三江口。巡司駐楊溪堡。

寧都直隸州繁,疲,難。隸吉贛南寧道。順治初,因明制,贛州為縣。乾隆十九年,升直隸州,並割瑞金、石城隸之。北距省治七百二十里。廣二百十五里,袤四百五里。北極高二十六度二十七分。京師偏西三十八分。領縣二。西:金精山。東:翠屏。南:螺石。北:凌雲山。東北:梅嶺,梅江水出焉,南行合諸水為東江,抵州東北,合西江,為三江口。又南,合白沙、白鹿水,為寧都水,入雩都。下河寨巡司。蕭田、蘆畬、黃陂、固村四汛。瑞金繁,疲,難。州東南一百七十里。東北:陳石。西:石門。南:雲龍。北:瑞雲山。貢水由福建長汀入,至城東南,會綿水、羅漢水,至水東渡,會北壩水,入會昌。東北:琴江,自寧都緣界入雩都。瑞林寨、湖陂二巡司。瑞林寨汛。石城簡。州東一百十里。東:筍石。東北:牙梳。西南:八卦。西:西華山。琴水出牙梳山之鷹子岡,西南會壩水,至城東,又西南,右合蝦公磔,左楓樹坳、蓮花水,逕古樟潭,合梨子崠、黃株潭水,入州。捉殺寨巡司。

南安府:沖,繁。隸吉贛南寧道。東北距省治一千一百三十里。廣三百五十里,袤三百六十五里。北極高二十五度二十九分。京師偏西二度三十分。沿明制,領縣四。大庾沖,繁。倚。西南有大庾嶺,縣以此名,一曰梅嶺,上有關曰梅關,相連為小梅嶺。東:獅子。西:西華。北:鐵岡。東北:玉泉。章江自崇義入,逕東北徒峰山,合李洞碧、赤嶺水,又東南,合平政水及涼熱水,又東納浮江,逕城南,又東合大沙河、湛口江,入南康。赤石嶺、鬱林鎮二巡司。小城、新城二鎮。一驛:小溪。其橫浦驛,裁。南康沖,繁。府東北一百三十五里。北:旭山。東北:丫髻。東南:獨秀。西南:龍山。西北:禽山。芙蓉江即章江,自大庾入,東流折北納南埜水,又東北,上猶江自其縣入,合西符水,左合禽水、過水、梅江來入,是為三江口。又東入贛會貢水。潭口、相安二巡司。南埜廢驛。上猶簡。府東北二百五里。東:資壽。西:書山。南:方山。北:飛鳳山。章水西南自崇義合西北琴江及禮信水,逕蜈蚣峽,左合鬥水、米潮水、料水,折東南,逕城南,曰縣前水。又東,左合猶水,曰上猶江。復合九十九曲水,又東南,合感坑水,與城南稍水並入南康,注章水。浮龍巡司。營前,縣丞駐。崇義簡。府北一百二十里。北:崇山。南:觀音。西北:桶岡。西南:聶都山,章水出,南逕師子巖,歧為二:南派亦曰池江,入大庾;北派東北逕城西,其西源流為益漿水,東納琴江入上猶,東南至坪江。西:符水,合南源水,右納義安水,至符江口。又南浮江,並入南康。橫水出大嶂山,繞城北出,會東溪水,入上猶江。上保、文英二鎮。金坑、鉛廠、長龍三巡司。


\end{pinyinscope}