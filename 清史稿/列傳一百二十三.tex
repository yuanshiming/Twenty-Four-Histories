\article{列傳一百二十三}

\begin{pinyinscope}
葉士寬陳夢說介錫周方浩金溶張維寅顧光旭

沈善富方昂唐侍陛張沖之

葉士寬,字映庭,江蘇吳縣人。康熙五十九年舉人,授山西定襄知縣。求民隱,滌煩苛,不假胥吏,事辦而民不擾。雍正八年,擢沁州知州,署潞安知府。除無名諸稅,復四門集以便商民。歷署平陽、太原,治行為山西最。十二年,舉卓異,擢浙江紹興知府。有惰民格殺士人,眾譁,將罷試,士寬方勘三江閘,馳歸,數言諭解之。風潮陷海塘,躬任堵築,三月而工完。乾隆初,調金華。東陽饑民求賑者以萬計,士寬曰:「按冊施賑,是賑冊非賑民也。」乃召饑者前注名於冊,而斥二人,眾乃定。二人者:一婦人,曾以訟至官,服華服,至是易敝衣乞賑,士寬識之,令褫其敝衣,內華服如故;一男子,容甚澤,令飲癰莢汁,嘔出酒肉。眾驚服,冒賑者潛散去。在金華三年,多善政,郡人為立生祠。擢杭嘉湖道,調金衢嚴道。衢州地高,西安、龍游諸縣,素築壩蓄水溉田?木商入山者,私開壩,水日涸,士寬嚴禁之,民皆稱便。八年,調寧紹臺道。紹興大水,蕭山、諸暨民多挾眾詣縣求食,巡撫惡之,不欲賑。士寬曰:「某來時,民饑幾欲死。何忍坐視其悉填溝壑耶?」繼以泣請,乃得上聞給賑。士寬以待饑而賑常不及,議濬紹興之鑒湖、寧波之廣德湖,會去官,未果。著浙東水利書,冀後有行之者。父憂歸,遂不出。

陳夢說,字曉巖,山西絳縣人。乾隆十三年進士,授刑部主事。讞決,執法不阿上官;兼提牢,役不能為奸。累遷禮部郎中。出為浙江寧紹臺道。臺州素犬廣悍,寧海梅村民拒捕,提督將以兵往,旁村皆驚竄。夢說輕騎臨縣,縣令已糸累系竄者數十人,盡釋之,曰:「吾來捕梅姓數人而已。」獲誅拒捕者,而釋其少子一人。臺人感之,謠其事為存孤記。修鄞縣錢湖徬。值上南巡,召見,素知其在刑部有能名,賜綺貂。尋以失察屬吏不職罣議,仍以道員用。授督糧道,卻餽金,漕政肅然。時訛言妖人翦發,蕭山捕僧了凡等四人,誣服,夢說平反之。後或言事由浙見,解京訊治無驗,抵妄捕者罪,以夢說輕比,降秩。修餘杭南湖堤。署嘉興、嚴州、處州、湖州諸府,復原官。夢說官浙十二年,所至有聲。尋乞歸。

介錫周,字鼎卜,山西解州人。康熙六十年進士。雍正初,授貴州畢節知縣。烏蒙土司叛,督運軍糧,遇逆苗,徒役欲棄糧走,錫周厲聲曰:「失糧法當死,犯苗亦死。死法毋寧死賊!」策馬徑前,千夫擁糧而進,逆苗眙愕,鳥獸散。遷平遠知州。烏蒙惈夷復叛,川、滇苗、惈應之。錫周先往撫大定苗,平遠得無患。十三年,擢大定知府。古州苗亂,陷黃平、清平,驛路俱梗。塘兵妄報土酋安國賢通古州苗,剋期犯貴陽。大吏發川兵將至,國賢轄地九百里,眾惶駭。錫周甫蒞郡,立召國賢至,諭以禍福。國賢伏地陳無交通古州狀,錫周曰:「汝率眾苗就撫,我以百口保汝不死,且止川兵。」時丹江亦被圍,乃請以川兵往援,丹江圍解而大定安堵。

南籠民王祖先素無籍。以書符惑眾,播為逆詞。又粵西儂人王阿耳為寨長王文甲所執,竄入苗寨,誣文甲將糾合冊亨諸寨叛。二獄同時起,株連千餘人,南籠獄不能容。滇、粵錯壤,寨苗多逃。錫周奉檄往會鞫,蔽罪悉當,釋文甲及系累者,逃亡並歸,邊境以靖。攝貴東道,筦糧運。時軍興,歲餽餉金二百四十餘萬兩、米八十餘萬石,調馬三千、夫五千,麕集鎮遠,漫無紀,夫縻廩食,馬累裏戶;復於上游南籠諸府役民夫加運九站,下游銅仁諸府則增雇調二千人助役。錫周畫三策:以馬設臺站,運凱里、丹江諸路;夫按期日運臺拱諸路,楚、粵米皆由水運;分清江及古州、都江兩路,輓輸迅速,糧乃集。上游之加運,下游之調夫,皆止之,省帑數十萬,民間亦減勞費之累。補貴西道,調糧道。兵米折色,不收餘羨,兵民交頌之。乾隆中,擢按察使。

錫周在黔中久,吏治、風土、民苗疾苦皆熟習,蒞之以誠,慎刑獄,興教化。性素耿介,不諧於時,以老乞休。上念其勞勩,召入覲,授太僕寺少卿。閱三年,告歸。

方浩,字孟亭,安徽桐城人。雍正八年進士,授山西太原知縣。嘗知隰、平定二州。隰民有茹素號為大乘教者,浩召至庭,啖以酒肉,人莫知其故。其後逮捕大乘教人連數郡,而隰民獨免。平定旱,奸民煽譁呶求糶,捕渠魁一人置之法,餘悉不問。遷潞安知府。會上西巡,取道澤、潞,吏平道,及道旁民田。浩以鑾輿未出而民廢耕作,非上愛民之意,令耕如平時。民得收穫,而事亦治。擢江西廣饒九南道按察副使,兼攝九江府事。歲旱,米商未至,他郡縣乏食,大吏檄運倉糧往濟。浩以郡民咸待食,而移粟他往,恐生事,請獨輸九江倉,而屬縣停運,違大吏意。未幾,安仁以阻運成大獄,大吏以此重浩。旋調吉南贛道。奸民據險為亂,馳詣捕緝。比大吏至,謀主已就擒,其敏捷如此。坐事罷,循例復職。方需次吏部,以疾卒。

金溶,字廣蘊,順天大興人。雍正八年進士,以刑部員外郎擢山東道監察御史。高宗即位,詔求直言,溶上疏言安民五事:一曰開墾之地緩其升科;二曰帶徵之項宜加豁免;三曰關稅正額之外免報盈餘;四曰州縣殿最首重民事,不以辦差為能;五曰巡狩之地崇尚樸素,不以紛華取媚。當是時,上命翰詹科道各進經史摺子,溶又上疏曰:「頭會箕斂以裕囊櫝者,匹夫之富也;輕徭薄稅使四海咸寧者,天子之富也。易卦:損下益上,上益矣而反名損;損上益下,上損矣而反名益。蓋謂百姓足君孰與不足,百姓不足君孰與足,聖人制卦之意可深長思也。」乾隆九年,湖廣總督孫嘉淦因徇巡撫許容奪職,命修順義城。溶上疏論曰:「賞罰者,人主御世之大權。臣工有罪,有罰鍰一例,因其素非廉吏,使天下曉然知所得者終不能為子孫計留也。孫嘉淦操守不茍,久在聖明洞鑒之中,而罰令出貲效力,恐天下督撫聞之,謂以嘉淦之操守,尚不免於議罰,或一不得當,而罰即相隨,勢必隳廉隅預為受罰之地。是罰行而貪風起,不可不慎也。臣為嘉淦所取士,不敢避師生之嫌而隱默不言。」奏上,部議奪職。

未幾,特起為福建漳州知府。漳俗強悍,胥吏千餘交結大吏家奴,勢力出長官上。有吳成者,設局誘博,擒治之,民稱快。華葑村距縣治二百里,康熙時嘗議設縣丞,以不便於胥吏,格不行。溶復以請,布政使文不下府而直行縣,溶大怒,嚴訊縣胥,得其交通狀,乃詳請治罪而設官。其父老嘆曰:「微金公,吾儕奔馳道路死矣!」十三年春,閩省旱,斗米千錢,大府檄溶平糶。溶勸富家出糶,給印紙令商人赴糴;又請寬臺灣米入內地之禁;民情帖然。其他脩文廟樂器,增書院膏火,皆次第舉行。遷臺灣道。補陜西鹽驛道。署布、按兩司事。調浙江糧道,與巡撫陳學鵬牴牾,學鵬論溶迂緩不任事,原品休致。卒,年七十三。

張維寅,字子畏,直隸南皮人。乾隆元年進士,授戶部江南司主事。江南賦役甲天下,支銷留解,端緒毛櫛。維寅綜覈精密,猾吏不能欺。遷吏部員外郎,考選監察御史,補掌貴州道。劾奏閩督誘人受賕而坐之罪,失政體,上是之,為通行飭戒。簡雲南迤東道,至,改補驛鹽。滇鹽無成法,維寅一一調之,使井官、煎戶、運夫、鋪商無偏累,滇人稱便。歲節縮歸公銀七千兩。以前官累,左遷知府。於時東川官設牛馬站,通百色,銅往鹽返,謂可省費。既奏行,而路險阻,車摧折,牛馬多死,銅鹽耗失。維寅奉勘得實,以事不可已,請夷路用車,險雇夫役,貲出爐息,無溢費,且不擾民,從之,獲濟。署鶴慶、永北,補臨安,調首郡,兼楚雄。值地震為災,躬勘鶴慶、劍川、浪穹、麗江、昌門賑,活災氓每數萬計。遷督糧道,整頓銅廠,代償前官虧帑,待罪得脫。調浙江鹽道,未數月,調福建汀漳龍道。閩俗獷悍,痛懲以法,擒巨猾,散夥黨,健訟鬥狠之風為息。察冤決疑,人稱神明。舉卓異,入覲,上獎慰甚至。復之官,病卒。

顧光旭,字晴沙,江蘇無錫人。乾隆十八年進士,授戶部主事。晉員外郎,主鹽筴,兩淮解銀,輒掛欠百之十五。光旭謂:「各省庫平皆部較頒,何獨兩淮歷久如是?是銀庫多索也。」白於長官除免之。擢御史。二十四年,直隸、山東大水。次年春,疏曰:「上年兩省災,截漕發帑加賑。近見流民扶老攜幼入京,春來尤甚。五城米廠飯廠人倍增,詢之,近京數百里,毀屋伐樹,賣男鬻女,老弱踣頓,不可勝計。耳目所及如此,其外可知。伏思救荒無奇策,惟督撫及有司親民之官實心實力方克有濟。各州縣未嘗不施賑,或委任佐貳,或假手胥吏,或設廠遠離村鎮,窮民奔走待食,或得或不得。良法美意,一入俗吏之手,沾實惠者十不及五。一二賢有司撫循周至,則他境流民聞風畢集,轉難措手。此督撫不能真實愛民,下亦以應付塞責,一切皆屬具文。請敕下隨地撫綏,毋致流移失所。疏導積水,以工代賑,借給牛種,以資耕作。有流民有礦土,尉即重治督撫州縣之罪。來京饑民,已領廠賑。一年之計,在於東作。無力自回者給貲遣送,其本籍無倚賴者歸大興、宛平安輯,勿令棲流無著。又每遇水旱,司、道、府親勘,先以供應煩州縣,所委佐貳,亦滋擾累,請嚴參重處。」奏入,上善之。命赴京畿察勘,疏消文安、大城積水。樂亭民擁閧縣門,撫定之,馳章請加賑。歷寶坻、灤州,盧龍,兩月竣事。遷給事中。

尋出為甘肅寧夏知府,調平涼。三十五年,大旱,請賑,初為上官所格。光旭親察災戶,亟發銀米,煮粥以賑,鄰縣饑者率就之。時災黎鬻妻子,道殣相望,光旭巡視山僻,賦詩曰:「輪蹄鳥道羊腸路,溝壑鳩形鵠面人。」又曰:「產破妻孥賤,腸枯草木甘。」誦者感動。自夏至次年三月始雨。平涼、隆德、固原、靜寧各設粥廠二,饑民日增。慮入夏疫作,給每口兩月糧,遣使歸耕。時已擢涼莊道,總督文綬任以河東賑事,一切錢糧聽支取,知府以下聽調遣。分八路比戶清勘,刊發三連票備考覈。發奸摘伏,官吏惕息。竟事無中飽,民獲更生。

三十七年,金川用兵,文綬調四川總督,疏請光旭隨往,司三路餽餉,署按察使。蜀民失業無賴者,多習拳勇,嗜飲博,浸至劫殺,號為啯嚕子,至是益眾。嚴捕治之,改悔者發為運丁,頗收其用。以秋審失出,罷職,留治糧餉。四十年,金川平,駐西路臥龍關經理凱旋兵十餘萬,帖然無擾。事竣,乞病歸,年未五十。

里居遇災,助賑一如在官時。主東林書院數十年,聚生徒講論道義,繼其鄉顧憲成、高攀龍之緒。著響泉集。

沈善富,字既堂,江蘇高郵人。乾隆十九年進士,選庶吉士,授編修。典江西、山西鄉試。撰制誥,辦院事,纂修國史、續文獻通考,勤於其職。出為安徽太平知府,在官十有六年,尤盡心災賑。三十四年,大水,坐浴盆經行村落,得賑者五十萬口。當塗官圩決,密勸富家出糶,禁轉掠,使各村自保。有告某家不糶者,笞之,曰;「汝奉何明令使富家出粟耶?」民乃定。三十六年,泗州水,大吏檄善富往賑之,釐戶口之弊,民受其惠。值大疫,設局施藥施瘞,絕葷祈禳。前後課屬縣種柳數百萬株,官路成陰。埋暴十餘萬棺。時傳妖人割發,搜捕令下,諸郡騷然,獨太平不妄捕一人。兄弟訟,察其詞出一手,杖主訟者。兄弟悔悟如初。師弟互訐陰事,取案前文卷盈尺火之。曰:「爾詞必有稿,可上控郡守焚案,不汝靳。」兩造皆泣,訟乃息。貴池有爭地訟於部者,視舊牘,得成化二十一年閏四月官契,念愚民安知閏,檢明史七卿表,得是年閏四月文,據以定讞。

四十六年,擢河東鹽運使。鹽池受淡水,歉產,商運蒙古鹽多勞費。及鹽產復盛,弊多商困。善富曰:「鹽池自古為利,不當廢革。若聽民自販,必致蒙鹽內侵。商人之力,不患寡,患不均。其弊有三:奸商棄瘠據肥,一也;費浮地遠,伙攫其利,二也;僉代之期,貧富倒置,三也。」乃總三省引地為三等均之。復以道路遠近順配為五十六路,鬮分簽掣之,於是賂絕弊清。後乾隆末廢商運,蒙鹽果內侵,至嘉慶十一年,仍復舊制,皆如所預計。所至興學愛士,人文蔚起。以母老乞終養,居鄉多善舉。著味鐙齋詩文集。

方昂,字坳堂,山東歷城人。乾隆三十六年進士,授刑部主事,累晉郎中。會秋讞更新例,凡金刃殺人,概為情實。昂分別其輕重,固爭不得,後高宗特旨改正。坐是為同僚所忌,淹滯十年。又數上書與長官爭,長官慍之,卒重其人。以薦出為江西饒州知府。安南阮光平入覲,驛傳所經,多飾供帳。昂曰:「國家以威德服四夷,非言誇以靡麗。」戒所屬勿與。擢江蘇蘇松道,已受代將行,營弁緝鹽,波及良善,眾洶洶不平。營弁以民變告,且徵兵,昂曰:「新守與民未習,民勿信。」自出曉諭,捕倡首者置法,申請上官褫營弁職,事即定。至任,有尼之者,遂謝病去官。

病痊,復出署松太道。閩,廣洋盜竄入吳淞,總督、巡撫、提督會師於寶山。昂建議曰:「衢山與大小羊山,江、浙之分界,港汊叢雜,盜船隨處可寄椗。一得風潮之便,倏忽出沒,猝不及防。當其乘風而來,迎擊之時,彼順而我逆;及其趁潮而退,追擊之,則我後而彼先:是使盜常憑勝勢也。請於要隘多設伏,俟其至,則縱使過,而躡其後;遇其退,則扼不使前,以待後隊之追剿。盜雖黠,無能為也。」從其議,盜果大摧。補江寧鹽巡道。緝訟師,剔衙蠹,戢強暴,弭盜賊,尤以砥礪風俗為先,屏絕酬酢。同官聞其風採,咸重之。嘉慶三年,擢貴州按察使,八閱月,遷江寧布政使。未久,以病乞歸。

昂剛勁勤職。其歸也,上曰:「此人可惜!」尋卒。

唐侍陛,字贊宸,江蘇江都人,巡撫綏祖孫。乾隆中,以廕生授南河山盱通判。歷任宿虹、銅沛、里河、外河同知。以治河績考最,擢湖北鄖陽知府,母憂去官。四十七年,服闋,會河決青龍岡,屢築屢圮,大學士阿桂督治,以侍陛習河事,疏調赴工。阿桂方與總河議改河之策,決計於侍陛,侍陛曰:「今全河下注,非土埽所能當;欲逆挽歸正道,難矣。但於南岸上游百里外開引河,則不與急流爭,其全勢易掣。以逸待勞,此上策也。」於是定計開蘭陽引河,至商丘歸正河,以侍陛總其事。工成,被詔嘉獎。

擢開歸道。時新引河堤初成,溜逼甚險,復於儀封十六堡增開引河。夏汛水至,果分為二派:一由新引河,一由儀封舊城之南達所增引河。又於毛家寨增築月堤,睢汛七堡建挑水壩,水勢乃暢下,無潰決。五十三年,署彰衛懷道。測河勢將有變,請於銅瓦廂大堤後增築撐堤,總河蘭錫第以無故興大工難之,固請乃可。次年夏,銅瓦工內塌,勢岌岌。總河李奉翰新至,視河、曰:「奈何?」侍陛曰:「待其塌多,必大決。今當於堤之下口新築撐堤內掘開數丈,使水回溜而入。入必淤,淤則大堤撐堤合為一。河直注之力已殺,堤乃可保。」從之,堤合險平。錫第曰:「君之出奇制勝者,在前之預築撐堤也。」

侍陛前官銅沛時,亦用放淤平險之法;又在宿虹時,夏家馬路黃、運交逼,里河淤淺,水將沒堤,效黃河清水龍法,疏其淤而堤安;於徐州城外增築石工,石磯嘴增爛石,城乃無患。衛河水弱,漕艘不利,掘地引沁挾濟以助衛。其應變弭患多類此。嘗論治河之道曰:「河行挾沙,治法宜激之使怒而直以暢其勢,曲以殺其威。無廢工而不可偪,無爭土而不可讓。守此岸則慮彼岸,治上游則慮下游。」世以為名言。尋補山東運河道,調兗沂曹濟道。以失察,左遷。遂乞病歸。

侍陛歷官皆有聲,有功於河、淮者為多。先是南汝光道張沖之以治河著。

沖之,字道淵,順天宛平人。雍正初,以諸生舉孝廉方正,授工部主事。遇事奮厲,於總理果親王前持議無避忌。各行省奏追虧帑積數千萬,牘冗無實,請分別覈免之。尋以事被謫。乾隆初,復原官,改刑部。累遷戶部郎中。治事平恕。二十六年,擢河南南汝光道。是年秋,河決楊橋,大學士劉統勛、兆惠奉命往塞之,調沖之襄河事。時徵秸,價騰至一莖兩錢,既大集,河員猶以多備請,官吏在事者群附和之。沖之曰:「計工需料若干萬,今已贏矣。災民搜括脂髓來供用,忍復乘以為利耶?」亟白使臣,請及時楗塞,期以某日合龍,當有餘料若干萬,力持其議。卒聽沖之減徵秫稭六千萬、麻六百萬,即責沖之董其役,果如期合龍,仍有餘料,殫數給還,以紓民力。巡撫胡寶瑔喜曰:「吾為國家得一良總河矣!」在官三年,治羅山獄,活誣服者四人;修城工務覈實,有司不得緣為蠹;民德之。以商城獄坐徇庇,奪職,效力軍臺。逾年放歸。

論曰:諸道本以佐布政、按察二使分領郡、縣;乾隆中,罷參政、參議、副使、僉事,道始為專官。士寬等皆觥觥能舉其職,侍陛尤以治河著。觀其所設施,益於國,澤於民,雖古循吏,不是過也。


\end{pinyinscope}