\article{列傳七十九}

\begin{pinyinscope}
高其倬金鉷楊宗仁子文乾孔毓珣裴幰度子宗錫

唐執玉楊永斌

高其倬,字章之,漢軍鑲黃旗人。父廕爵,官口北道。其倬,康熙三十三年進士,改庶吉士,散館授檢討。尋兼佐領。五遷內閣學士。五十八年,河南南陽鎮兵挾忿圍辱知府沈淵,命偕尚書張廷樞按治,誅首事者,總兵高成等論罪有差。

五十九年,授廣西巡撫。鄧橫苗叛,其倬親撫之降。六十一年,世宗即位,擢雲貴總督。疏言:「士司承襲,向有陋規,已嚴行禁革。咨部文冊,如無大舛錯,請免駁換。」得旨嘉獎。青海臺吉羅卜藏丹津叛侵西藏,其倬以中甸為入藏要道,檄諸將劉宗魁、劉國侯等嚴為備。並遵上指,令提督郝玉麟將二千人自中甸進駐察木多,副將孫宏本將五百人赴中甸為聲援。雍正二年,師定青海,中甸喇嘛、番酋等率三千五百戶納土請降。上嘉其倬能,予世職拜他喇布勒哈番。其倬規畫安撫中甸,疏「請設同知以下官:番酋營官外,又有神翁、列賓諸號,聽堪布、喇嘛指揮,請改授守備、千把總劄付,聽將吏統轄。僧寺喇嘛以三百為限,收兵械入官。沿江數百里及山谷曠土,招民開墾。舊行滇茶,視打箭爐例,設引收課」。魯魁山者,自國初為盜藪,夷、惈雜處,推楊、方、普、李四姓為渠。有方景明者,挾惈、夷掠元江。其倬遣兵擊破之,擒景明,殲惈、夷數百,疏請於其地駐兵,號普威營。參將駐普洱,守備駐威遠、茶山,改威遠歸流,設同知以下官。土官刁光煥及其孥移置會城,而以新開二鹽井充新設兵餉。設義塾,教夷人子弟。元江府學額外增額二名,待其應試。勸夷人墾田,旱田十年後、水田六年後升科。貴州仲家苗酋阿近及其弟阿臥為亂,其倬使撫定傍近諸苗寨。阿近等失援,遣兵擒戮之,並按治定番、廣順諸苗酋不順命者。疏請改設定廣協,分置營汛,防定番、廣順及西孟、青藤、斷杉樹、長寨、遮貢、羊城諸地。又移都勻守備駐獨山,改湖廣五開衛為縣,移隸黎平。並言貴州地連川、楚,奸人掠販貧家子女為民害,請飭地方官捕治,歲計人數為課最。貴州民間陋俗,被人劫殺,力不能報,則掠質他家人畜,令轉為報仇;不應則索贖,謂之「拏白放黑」。請加等治罪。土司貧困,田賦令屬苗代納,請清察,責執業者完賦。土司下設權目人等,請令報有司,有罪並懲。詔悉如所請。

三年,進兵部尚書銜,加太子少傅,調福建浙江總督。瀕行,疏言:「鄧川、嵩明、騰越、太和、浪穹諸州縣土軍丁銀,起明嘉靖、萬歷間,遣民防夷,立太和、鳳梧二所,丁徵賦一兩。是於本貫已完民賦,請豁除軍糧。」詔從之。四年,疏言:「福、興、漳、泉、汀五府地狹人稠,無田可耕,民且去而為盜。出海貿易,富者為船主、為商人,貧者為頭舵、為水手,一舟養百人,且得餘利歸贍家屬。曩者設禁例,如慮盜米出洋,則外洋皆產米地;如慮漏消息,今廣東估舟許出外國,何獨嚴於福建?如慮私販船料,中國船小,外國得之不足資其用。臣愚請弛禁便。」下怡親王會同大學士九卿議行。五年,臺灣水連社番為亂,其倬遣兵討之,擒其渠骨宗等,諸社悉降。尋以李衛為浙江總督,命其倬專督福建。迭疏請整飭鹽政,改造水師戰船,釐定營汛,並下部議行。入覲,加太子太保。

上以其倬通堪輿術,命詣福陵相度。其倬還奏:「陵前左畔水法,因溢流更故道,弓抱之勢微覺外張。當順導河流,方為盡善。」下大學士等,如所議修濬。八年,調江南江西總督。復召至京師,令從怡親王勘定太平峪萬年吉地,進世職三等阿思哈尼哈番。命署云貴廣西總督。十一年,普洱屬思茅土把總刁國興糾苦蔥蠻及元江夷為亂,攻普洱,通關大寨夷復附苦蔥蠻,渡阿墨河攻他郎。其倬檄提督蔡成貴等分道捕治,擒其酋並所屬五百餘,亂乃定。是歲春,命其倬回兩江總督。秋,命以總督銜領江蘇巡撫。十二年,坐徇知縣趙昆珵償海塘工款,部議降調,即授江蘇巡撫。

乾隆元年,召還京師,復授湖北巡撫,調湖南。討平城步、綏寧二縣瑤亂。三年,擢工部尚書,調戶部。其倬詣京師,過寶應,疾作,卒於舟次,賜祭葬,謚文良。

金鉷,字震方,漢軍鑲白旗人,世居登州。父延祚,從世祖入關,官至工部侍郎。鉷初自監生授江西廣昌知縣,洊升山西太原知府。雍正五年,擢廣西按察使,尋遷布政使。六年,就擢巡撫。討平西隆州八達寨叛苗。以汛兵少,粵土蕪不治,奏開屯田,與民牛,招之耕,教以技勇。每名給水田十畝,一畝為公田;旱田二十畝,二畝為公田:存公田租於社倉。行之數年,闢田數萬畝,倉廩亦實。又奏請召商開桂林屬諸礦,及採梧州金砂供鼓鑄。乾隆元年,提督霍升劾鉷言躁氣浮,失封疆大臣之體,高宗召入京,授刑部侍郎。鉷瀕行,裝不治,以印券囑蒼梧道黃岳牧借銅務充公銀千二百,巡撫楊超曾論劾,奪官,交刑部嚴訊。上以非正項錢糧,鉷以印券支借,岳牧以印冊申解,非侵蝕比,命免罪,毋追所借銀。五年,授河南布政使,而鉷已卒。

鉷才通敏。自太原入覲,方議耗羨歸公,鉷奏曰:「財在上不如在下。州縣親民官,寧使留其有餘,養廉不能胥足,一遇公事,動致侜張。上意豈不曰凡是官辦,皆許開支正供?但從司院按覈以至戶部,層層隔閡,報銷甚難,從此州縣恐多茍且之政。上意在必行,臣請養廉外多增公費,或存縣,或存司,庶於事有濟。」上乃敕直省覈定公費。及為廣西布政使,奏請州縣分沖、繁、疲、難四項,許督撫量才奏補,上嘉納之。州縣缺分四項自此始。

楊宗仁,字天爵,漢軍正白旗人。監生。康熙三十五年,授湖廣慈利知縣。苗酋虐,其眾走縣境,苗酋求之,不與。上官檄與之,宗仁持不可,乃止。調藍山。八排苗為亂,巡撫趙申喬遣兵討之,將不恤兵,兵將為變,宗仁單騎撫定之。舉卓異,四遷甘肅西寧道。五十三年,授浙江按察使,丁父憂歸。五十七年,起廣西按察使,署巡撫。旋擢廣東巡撫。聖祖以各直省錢糧多虧空,諭督撫清理。宗仁疏言:「廣東虧空現正嚴飭追完。至防杜將來,惟有督撫、司道、府交相砥礪,勿藉事勒索。州縣正雜錢糧,當責知府不時察覈,毋許虧缺。倘敢徇縱,本官治罪,上司從重議處,庶上下皆知儆惕。地方有不得已事,當以督撫等所得公項抵補。不敷,則濟以公捐,必不使課帑虛懸。」下部議,如所請。

六十一年,世宗即位,授湖廣總督。雍正元年,丁母憂,命在任守制。宗仁疏停本身封廕,為父母求諭祭,許之,仍給封廕。尋賜孔雀翎。疏言:「湖廣舊習,文武大吏收受所屬規禮,致州縣橫徵私派,將弁虛兵冒餉,兵民挾比逞私,不敢過問。臣今概行禁革,庶驕兵玩吏錮習潛消。各官貪得鹽規,鹽價增長,民間嗟怨,總督鹽規漸次加至四萬。臣亦行禁革,令商平價以惠窮民。」上深嘉之。又疏言:「官有俸,役有工,朝制也。湖廣州縣以上,俸工報捐已十餘年,官役枵腹,安能禁其不擾民?請自雍正元年起,俸工如額編支。從前有公事,令州縣分捐,實皆轉派於民。令州縣於加一耗羨內,節省二分,交籓庫充用,此外絲毫不得派捐。」上諭曰:「所言皆是。勉之!」尋薦廣東南海知縣宋瑋擢湖南寶慶知府,廣州左衛守備範宗堯改湖北漢陽知縣,上允之,命後勿踵行。

宗仁病作,請以子榆林道文乾自侍,上加文乾按察使銜,馳驛速往,並遣御醫診視。宗仁力疾視事,飭諸州縣編保甲,立社倉,罷荊州關私設口岸百五十處。三年,加太子少傅。尋卒,贈少保,予拜他喇布勒哈番世職,賜祭葬,謚清端。

宗仁砥節矢公,始終一節,上為制像贊,謂「廉潔如冰,耿介如石」。嘗言:「士當審其所當為,嚴其所不可為。」其馭屬吏寬平忠厚,務安上全下,使各稱其職而止。

文乾,字元統。以監生效力永定河工。康熙五十三年,授山東曹州知州,遷東昌知府。舉卓異,遷陜西榆林道。雍正元年,加按察使銜,命侍宗仁任所。三年,宗仁病有間,入謝。上問湖廣四鎮營制及設鎮始末,文乾具以對,上嘉其詳審,擢河南布政使。未幾,遷廣東巡撫,入謝,賜孔雀翎、冠服、鞍馬。宗仁卒,命在任守制。

廣東省城多盜,文乾令編保甲,以滿洲兵與民連居,會將軍編察,疏聞,上嘉之。廣東歲歉米貴,文乾令吏詣廣西買穀平糶。滿洲兵閻尚義等群聚掠穀,文乾令捕治。將軍李枚庇兵,文乾請遣大臣按治。上命侍郎塞楞額、阿克敦往勘,枚及尚義等論罪如律。文乾蒞政精勤,多所釐正。疏言:「廣東民納糧多用老戶,臣令改立的名,杜詭寄、飛灑諸弊,民以為便。丁銀隨糧辦者十四五,餘令布政使確核,盡歸地糧。」得旨嘉獎。又疏言:「廣東地狹人眾,現存倉穀一百六十餘萬石,為民食久遠計,應加貯二百餘萬石,擇地建倉貯穀。」下廷議,令於海陽、潮陽、程鄉、饒平、海豐、瓊山加貯穀三十四萬石,從之。又疏言:「廣東公使銀歲六七萬,取諸火耗。臣為裁省,歲計需四萬餘。擬以民間置產推糧易戶例納公費及屯糧陋規兩項充用。州縣火耗,每兩加一,實計一錢三四分有奇,十之五六留充州縣養廉,十之七八為督撫以下各官養廉。」上諭之曰:「但務得中為是。民不可令驕慢,屬吏亦不可令窘乏。天下事惟貴平,當徹始終籌畫,慎毋輕舉。」

五年,乞假葬父。福建巡撫常賚劾文乾徵粵海關稅,設專行六,得銀二十餘萬;又疏劾文乾匿粵海關羨餘銀五萬餘,縱綢緞出洋,得銀萬餘,番銀加一扣收,得銀四萬餘,選洋船奇巧之物入署,令專行代償,又銀二萬餘,又以銀交鹽商營運。上嚴諭文乾,令愧悔痛改。尋以福建倉庫虧空,命文乾與浙江觀風整俗使許容等往按,而移常賚署廣東巡撫。文乾令分路察核官虧民欠,分別追納,不敷,責前巡撫毛文銓償補。上獎文乾秉公無瞻顧。文乾疏言:「福建府、州、縣各官都計八十員,前後劾罷五十餘員。新補各官,守倉庫有餘,理繁劇不足。請選熟諳民事者,詣福建補繁要州縣。」上為敕各督撫各選謹慎敏練之吏咨送福建。

文乾強幹善折獄。初知曹州,有婦告夫為人殺者。文乾視其屨白,問曰:「若夫死,若預知之乎?」曰:「今旦乃知之。」曰:「然則汝何辦白屨之夙也?」婦乃服以奸殺夫。五人者同宿,其一失金,訟其四,文乾令坐於庭,視久之,曰:「吾已得盜金者,非盜聽去。」一人欲起,執之,果盜金者。曹民有偽稱硃六太子者,挾妖術惑愚民,朝命侍郎勒什布、湯右曾按治。檄至,文乾秘之,密捕得送京師。在東昌,請運糧饋軍出西寧,先期至,以是受知於世宗。

然頗與同官多齟。赴廣東,途中疏劾布政使硃絳倚總督孔毓珣有連,虧帑三萬餘。毓珣疏先入,上命文乾毋聽屬吏離間。既上官,疏言盜案塵積,請概為速結。上諭曰:「孔毓珣緝捕盜賊甚盡力。彼擒之,汝縱之,恐汝不能當此論。縱虎歸山,豈為仁政?宜加意斟酌。」在福建,毓珣入覲,上命侍郎阿克敦署兩廣總督。文乾疏言盜劫龍門營軍器,阿克敦令從寬結案;將軍標兵窩盜,將軍石禮哈袒兵,謂告者誣良。既,上命常賚還福建,而以阿克敦署廣東巡撫。六年,文乾還廣東,劾阿克敦勒索暹羅商船規禮,布政使官達縱幕客納賄,皆奪官。命文乾與毓珣會鞫,未及訊,文乾卒,賜祭葬。子應琚,自有傳。

孔毓珣,字東美,山東曲阜人,孔子六十六世孫。父恩洪,福建按察使。康熙二十三年,上幸曲阜釋奠,毓珣以諸生陪祀,賜恩貢生。二十九年,授湖廣武昌通判。舉卓異,遷江南徐州知州。徐州民敝於丁賦,毓珣在官七年,拊循多惠政。三十九年,河道總督張鵬翮以毓珣熟於河務,薦授邳睢同知。四十三年,遷山西平陽知府,未上,改雲南順寧。四十六年,調開化,以母憂去官。五十年,服終,除四川龍安。毓珣歷守邊郡,皆因俗為治,弊去其太甚,邊民安之。再舉卓異。五十五年,遷湖廣上荊南道。築堤捍江,民號曰孔公堤。

五十六年,遷廣西按察使。廣西地瘠民悍,瑤、僮為民害。靈川僮酋廖三屢出焚掠,毓珣白巡撫陳元龍,遣兵捕得置諸法,諸苗讋服。五十七年,授四川布政使。西藏方用兵,毓珣轉餉出察木多,不以勞民。重築灌江口堰,四川民尤德之。六十一年,擢廣西巡撫。雍正元年,加授總督。廣西提鎮標空糧,毓珣飭募補。疏言:「各官俸不足自贍,請於定例外量加親丁名糧。」上命酌中為之。廣西諸州縣舊有常平倉,毓珣議:「春耕借於民,秋收還倉,年豐加息,歉免息,荒緩至次年還本。日久穀多,分貯四鄉,建社倉,擇里中信實者為司出入。」又言:「地多盜,瑤、僮雜處,保甲不能遍立。諸鄉多有團練,令選誠幹者充鄉勇,得盜者賞,怠惰者罰。」又言:「廣西邊遠,鹽商多滯運,民憂淡食。請發籓庫銀六萬,官為運銷。行有贏餘,本還籓庫,並可量減鹽價。」並從之。柳州僮莫貴鳳出掠馬平、柳城、永福諸縣,毓珣遣兵捕治,毀其寨,置貴鳳於法。來賓僮覃扶成等出掠,未傷人,毓珣令予杖荷校,滿日,充撫標兵,散其黨類。疏聞,上嘉其寬嚴兩得。

二年,授兩廣總督。上諭之曰:「廣東武備廢弛,劫掠公行,舉劾官吏,百無一公,爾當盡心料理。」毓珣疏請釐定鹽政,灶丁鹽價、船戶水腳增十之一,並免埠商羨餘;設潮州運同、鹽運司經歷。大金、蕉木兩山產礦砂,東隸開建、連山,西隸賀縣、懷集。舊制,懷集汛屬潯州協,毓珣請改屬梧州協,賀縣、開建、連山並增兵設汛。廣東香山澳西洋商舶,毓珣請以二十五艘為限。皆下部議行。潮州田少米貴,民賴常平倉穀以濟。毓珣請提鎮各營貯穀借兵,散餉時買還,概免加息,上特允之。三年,加兵部尚書銜。

四年,毓珣請入覲,上以毓珣習河事,令詳勘黃、運諸河水勢,協同齊蘇勒酌議。毓珣疏言:「宿遷縣西,黃河與中河相近,舊有汰黃壩。運河水大,引清水刷黃,黃河水大,引黃水濟運。舊時黃水入中河不過十之一二,今河南岸沙漲,逼水北行,水流甚急。齊蘇勒議收小汰黃壩口以束水勢。臣詳勘南岸漲沙曲處,宜濬引河以避此險。仍俟齊蘇勒相度定議。」又陳江南水利,言:「吳淞、劉河、七浦、白茆諸閘,宜令管閘官役隨潮啟閉。江蘇地形四高中下,宜令力勸築區立圩。濱河諸地民占為田廬,其無甚害者,姑從民便,餘宜嚴禁。支河小港,宜令於農隙深濬,即取土培圩。」並敕部議行。又言:「道經宿州靈壁,見溝洫不通,積雨成潦,請飭安徽巡撫疏濬。」上嘉毓珣實陳。

五年,還廣東,巡撫楊文乾劾署巡撫阿克敦、布政使官達,上命通政使留保等往按。毓珣失察,當下吏議,上命寬之。尋調江南河道總督。上以天然壩洩水,慮溢浸民田,命毓珣相度築堤束水歸湖。毓珣疏言:「天然南、北二壩分洩水勢,年年開放,堤口殘缺。當如上指築堤束水,請於南岸王家庵至趙家莊築新堤一道。舊堤尾距湖尚二十餘里,請於南岸馬家圩至應家集、北岸周家圩至李艮橋,各築新堤一道,並將南北舊堤加培高廣,庶兩堤夾束湍流,無患旁溢。」上又以高家堰為蓄清敵黃關鍵,發帑百萬,命毓珣籌畫。毓珣疏言:「高家堰石堤,自武家墩至黃莊,地高工固,惟侯二門等四壩,及小黃莊至山盱古溝東壩,當一律加高。」又言:「各堤加培高廣,宜視地勢緩急、舊堤厚薄,分年修增,期三年而畢。嗣後仍按年以次加培。」又請修築宿遷鈔關前、桃源沈家莊河堤,瓜洲由閘上游濬越河一道,並建草壩束水。諸疏入,並報可。毓珣積瘁遘疾,上賜以藥餌,命其子刑部郎中傳熹偕御醫馳驛往視。未至,毓珣卒,賜祭葬,謚溫僖。

裴幰度,字晉武,山西曲沃人。少為諸生,工詩,能書畫。入貲為主事。康熙三十五年,授刑部主事。洊擢戶部郎中。四十九年,授雲南澂江知府,調廣南。以大計入覲,聖祖聞其能詩,命題應制,稱旨。五十五年,遷河東鹽運使,尋改兩浙。海寧築塘,巡撫徐元度檄幰度董其事。潮大至,撼塘,塘欲裂,幰度據地坐督役力護,久之乃定。幰度自是中濕,病重膇,終其身。五十九年,遷湖北按察使。六十年,遷貴州布政使。

雍正元年,擢江西巡撫。九江舊設關榷稅,後徙湖口。湖口當江、湖沖,水急,商舟時覆溺。幰度疏言:「九江舊關,上有龍開河、官牌夾,下有老鶴塘、白水港,地勢寬平,泊舟安穩。離湖四十里曰大姑塘,為商舟所必經,水漲則有女兒港、張家套,皆可泊舟;水落則平湖一線,夾岸泥沙,無風濤礁石之險。請仍移關九江,而於大姑塘設口分抽。」上令會同總督查弼納料理。南昌、袁州、瑞州三府賦額,明沿陳友諒之舊,視他府偏重。順治間、減袁、瑞二府賦額,而南昌未及。幰度疏言:「常賦未易屢更,同省實難歧視。請將南昌賦額視袁、瑞二府同予核減。」下部議減南昌浮額七萬五千五百兩有奇。

福建、廣東流民入江西,就山結棚以居,蓺靛葉、煙草,謂之「棚民」,往往出為盜。萬載溫上貴、寧州劉允公等,皆以棚民為亂,幰度捕治論如律。上令編保甲,幰度疏言:「棚民良莠淆雜,去留無定,或散居山箐,或為土民傭工墾地。臣飭屬嚴察,凡萬五千餘戶,編甲造冊,按年入籍。」上獎勉之。上聞江西里長催徵累民,民多尚邪教,諭幰度禁革。幰度疏言:「臣察知里長累民,已勒石永禁,令糧戶自封投櫃。距城較遠畸零小戶,原輪雇交納者聽其便,仍嚴防不得干累。邪教自當捕治,醫卜星相往往假其術以惑民,雖非邪教,亦當以時嚴懲。」上深嘉之。

總督查弼納議開廣信封禁山,諭幰度酌度。幰度疏言:「封禁山舊名銅塘山,相傳產銅,然有名無實,故自明封禁至今。順治間有議採木者,郡縣力陳不便,勒碑永禁。臣揆查弼納意,或以棚民巢穴在此山中,故為破巢搗穴之計。此山荊榛充塞,稔毒滋藏,並非有梗化頑民盤踞在內。臣詳度此山開則擾累,封則安寧,成案俱存,確有可據。」諭曰:「當開則不得因循,當禁則不宜依違。但不存貪功之念,實心為地方興利除害,何事不可為?在卿等秉公相度時宜而酌定之。」仍封禁如初。

四年,遷戶部侍郎,擢左都御史。上遣侍郎邁柱勘江西諸州縣倉穀,命幰度留任。邁柱疏言:「倉穀虧空甚多,例定穀一石折銀二錢,州縣交代,按此數接收,不敷糴補。」上奪幰度及歷任布政使張楷、陳安策官,命以所存折價買穀還倉。十年,事畢,釋還里。乾隆五年,卒。

子宗錫,入貲為同知。十五年,授山東濟南同知,屢遷轉。二十八年,授直隸霸昌道,遷直隸按察使。疏言:「古北口外山場產菠蘿樹,此即橡樹,葉可飼蠶。臣在濟東,飭屬通栽,頗有成效。請令用東省養蠶法,廣栽試養。」命交總督方觀承試行。三十二年,以母憂去官。宗錫在任,誤應驛站車馬,部議當降調。總督楊廷璋咨部,言宗錫當自行檢舉。上諭曰:「宗錫,朕知其為人,頗可造就。按察使管理驛站,偶有一二誤應,原屬公過。今已丁憂,安得自行檢舉?廷璋乃令作此趨避,愛之適以害之也。」三十五年,宗錫服將闋,仍授直隸按察使。

俄擢安徽布政使,就遷巡撫。疏言:「安慶瀕江舊有漳葭港,上通潛山、太湖、望江三縣,下達江,漕艘商舶往來停泊,淤久漸成平陸。前巡撫張楷於上游別開新河,地高水急,重載逆上,遇風每虞覆溺。請仍濬漳葭港故道。」命總督高晉履勘,如宗錫議行。又疏言:「鳳、泗所屬州縣,高地宜多作池塘,低地宜厚築圩圍,以備灌溉、資捍禦。鳳陽地多高岡曠野,不宜五穀,令視土宜種樹。」諭獎其留心本務。

四十年,調雲南。旋命署貴州。疏言:「貴州地處邊圉,請敕部撥銀三十萬貯司庫。」從之。又疏請增設鎮遠稅口,上嚴斥不許。又疏言:「貴州額輸京師及湖廣白鉛歲七百餘萬斤,鉛廠僅三處,年久產絀。臣察知松桃巴壩山、遵義縣新寨產鉛,近水次,已飭設廠,歲各得鉛百餘萬斤。分撥京師、湖廣,歲節省運費銀四萬三千有奇。」得旨嘉允。又疏言:「貴州古州有牛皮大箐,亙數百里,列屯置軍,應將箐內平曠之土開墾成田,寓防於屯,安屯養軍。丹江雷公院地平衍,可墾四五百畝,歐收、甬荒高箐二地畸零,可墾三四百畝,應令附近震威堡屯軍派撥試墾,並於丹江營移撥千總一、兵五十,入箐設卡駐守。」時上已命宗錫還雲南,命交後政圖思德如所議行。四十四年,以病乞解任。旋卒,賜祭葬。

唐執玉,字益功,江南武進人。康熙四十二年進士,授浙江德清知縣。德清盛科第,多鉅室,執玉執法無所撓。將編審,吏以例餽金,執玉卻之,而罪其吏。召縣民親勘,有田無糧者令自首,有糧無田者除之,富無隱糧,貧無賠累。行取工部主事,考選戶科給事中。五十八年,疏言:「戶部錢糧款項最易作弊,當先驅除作弊之人。乃有所謂『缺主』者,或一人占一司,或數人共一省,占為世業,句通內外書吏,舞文弄法,當嚴行查禁。」因劾山西司缺主沈天生包攬捐馬事例,下九卿議,逮治。六十年,遷鴻臚寺卿。歷奉天府府丞、大理寺少卿。雍正二年,歲三遷禮部侍郎。五年,擢左都御史。

七年,命署直隸總督。執玉治事勤,州縣稍歉收,必籌畫賑恤。隆平報產瑞禾三十三本,執玉於報秋成摺附奏,上嘉之。適貢荔支至,命以賜執玉,方有疾,治事如常。時宗人府府丞冀棟以醫進,上命視執玉疾,賜人葠,諭令:「愛養精神,量力治事。若欲棟料量方藥,保定咫尺,可再命之來也。」熱河徵落地稅,司其事者議增歲額,並於榜什營等地設口徵稅。下執玉議,執玉言:「商稅多寡,視歲收豐歉,故止能折中定額。榜什營距一百八十餘里,已收落地稅,又抽進路鈔銀,恐商賈不前,正稅反缺,請如舊便。」議乃寢。長蘆巡鹽御史鄭禪寶以商人虧帑,請增鹽價,上以詢執玉。執玉言:「上於商民無歧視。諸商不謹身節用,先公後私,乃至虧帑。欲增鹽價厲民,臣以為非宜。」亦罷不行。

八年春,入覲。灤、盧龍、遷安、撫寧、昌黎、樂亭諸州縣米貯喜峰口倉,虧二千五百餘石,執玉請視通州中、西二倉例免追償。部議不許,上特允之。密雲城臨白河,舊築土木堤壩盡圮,僅存石堤。上游有積土斜出,激水使怒,俗謂之「土嘴」。執玉疏請疏治,使水得暢流;仍築土堤,務堅厚,用榆囤載石為基,使輔石堤護縣城。上褒其妥協,命於夏月水漲前竟工。遷兵部尚書,仍署總督。是歲秋,積雨,永定、滹沱諸水皆盛漲。執玉疏報災,上命侍郎牧可登、副都統阿魯等分往治賑。執玉奏言:「諸州縣被水,消長不一。有上諭所及,而水消未成災者;有上諭所未及,而水大成災,田廬被淹,急須拯恤者:請飭治賑諸臣勘實。」上特允之。

國初以民地予滿洲將士,謂之「圈地」。民地既圈,以鄰近州縣地撥補,糧額從舊貫,於是有寄糧;佃租戶移新地,於是有寄莊。歷年既久,百弊叢起。上令執玉勘察,更除改正,並舉懷安、宣化、萬全、寶坻、豐潤、三河諸縣為例。執玉奏言:「此外所在皆有,如晉州武丘村、孔目莊,趙州馬圈村糧有在贊皇者;蔚縣夾道溝、細賢莊糧有在宣化者;宣化井頭莊糧有在西寧者:官苦追呼,民勞跋涉。凡地在此處,糧寄彼處,皆令從地所在,糧隨產轉,此收彼除,不使有交錯之病,亦無庸存代徵之名,經界各正,田賦悉清。」直隸驛馬一,每歲雜支大率至十兩。執玉奏定馬一每歲雜支三兩六錢。昌平、延慶、宣化諸驛事煩,撥僻地馬協濟,而牧養仍責原驛。執玉奏請改隸受協州縣牧養。皆下部議行。

直隸耗羨歸公,自雍正三年始。部議元、二年耗羨在三年補納者,州縣充公用,仍當追償。霸、文安等七州縣民借倉穀,逋米二萬一千石、穀一萬六千石各有奇,部議責州縣追償。執玉言:「元、二年耗羨在未著令歸公以前,前督臣許州縣充公用。今欲追償,是為小費而失大言。」又言:「倉穀民欠歷年已久,人產胥絕。今欲追償,此數十年官州縣者無慮百數,悉逮其子孫而加以追比,於情可憫。」上並如執玉議,寬之。

九年,以病甚乞解任,許之。十年,病少瘳,命領刑部尚書。十一年春,復命署直隸總督,力辭,上勉之行。三月,卒於官,賜祭葬。

執玉重民事,每請從寬大,疏入輒報可。執玉嘗曰:「吾才拙,政事不如人,可自力者勤耳。勤必自儉始。」養廉歲用十三四,餘歸之司庫。

楊永斌,字壽廷,雲南昆明人。康熙三十八年舉人。以知縣發廣西,補臨桂知縣,以廉能聞。遭喪去,服除,授直隸阜平知縣,署平山,調大城,皆有惠政。以捕治內監陳永忠未即獲,奪官。大城民乞巡撫疏留,會世宗即位,知永斌賢,許復官。遷涿州知州。

雍正三年,特諭永斌才守俱優,授貴州威寧知府。威寧界滇、蜀,諸土司虐使其眾,時出掠境外。烏蒙祿萬鍾、鎮雄隴慶侯尤強悍。永斌被檄定界,單騎入諭其渠,陰使人偽為商賈,分道圖地形。鄂爾泰督云、貴,永斌以圖上,且曰:「二酋不懲,終為邊患。萬鍾幼,諸土司未附。今四川總督劾萬鍾不職,請發兵壓境,召萬鍾出就質。不出,以兵入。烏蒙平,鎮雄勢孤,亦且降。」鄂爾泰從之,召萬鍾不至,令游擊哈元生與永斌督兵入。萬鍾走鎮遠,與慶侯同詣四川降。凡三十三日而事定。米貼土婦陸氏為亂,鄂爾泰遣兵討之,永斌語元生曰:「賊以冕山、巴補為後路,事急則渡金沙江而逸。以重兵扼其前,奇兵越江攻之,賊可殲也。」元生用其策,克米貼。

鄂爾泰疏薦永斌可大用,擢貴東道,旋調糧驛道,署按察使。朝議加稅軍田畝五錢,永斌議曰:「軍田糧以屯租為準,已數倍於民田。且今轉相授受,與民田交易無異。名為軍屯,實皆民產,而畝稅之,是重科也,民必不服。當多事之秋,增剝膚之患,驅之為亂耳。」鄂爾泰以聞,事乃寢。七年,遷湖南布政使。湖南方議清察軍田計畝,未定,永斌援貴州議以請,亦得免。

九年,調廣東。十年春,命署巡撫,是秋真除。廣東生齒繁,民不勤稼穡,米值高。永斌飭諸州縣勸墾,高亢不宜禾,令藝豆麥,諸山坡麓栽所宜木。又以惠、潮兩府民最悍,招墾官田,租入充粵秀書院膏火。奏聞,嘉獎,命勘明墾地畝數。尋又奏言:「勘明可墾地六千八百餘頃,此外或山深箐密,或夾沙帶鹵,體察民情,恐磽地薄收,糧賦無出。臣思瘠田產穀雖少,若多墾數十萬畝,年豐可得數十萬石,即歉歲亦必稍有所獲,事益於民。察通省糧額,新寧斥鹵,輕則畝徵銀四釐有奇、米四合有奇。擬請凡承墾磽瘠之地,概準此例,十年起科。」下部議行,於是墾田至百十八萬餘畝。

乾隆元年,兼署兩廣總督。上命除落地稅,因請並免漁課、埠稅,革粵海關贏餘陋例未盡汰者,上悉從之。永斌在廣東數年,坦懷虛己,淬厲諸將吏。獲劇盜餘猊、陳美倫數十輩置之法,收曲江乳源諸峒瑤歸化。西洋估舶互市至者,悉令寄椗澳門,不得泊會城下。粵民頌其績。二年,調湖北,兼署湖廣總督。令嚴保甲,繕城堡,課農桑,實社倉,興學校,諸政畢舉。

未幾,調江蘇。按行奉賢、南匯、上海、寶山四縣海塘,以築塘取土成渠,塘根浸損,議於塘內開河,南接華亭運河,北達寶山高橋。又察華亭金山嘴、倪家路,寶山楊家嘴地當沖要,議視地所宜,或增築石壩,或就舊塘加築寬厚,或改築石塘。又請於寶山建海神廟。並從之。三年,以老病乞休,召詣京師,署禮部侍郎。尋授吏部。四年,致仕。五年,卒。孫,廕生,初授主事,官至江蘇按察使。

論曰:其倬、宗仁、毓珣,皆聖祖所擢用,丕著勛勩;世宗畀以兼圻,忠誠靡懈,恩禮始終,宜矣!幰度居官不擾民,執玉、永斌尤懃懃施惠,文乾、宗錫能濟其美。世宗治尚明肅,諸臣皆以開敏精勤稱上指,為政持大體,與夫急功近名,流於谿刻,重為世詬病者,固大異矣。


\end{pinyinscope}