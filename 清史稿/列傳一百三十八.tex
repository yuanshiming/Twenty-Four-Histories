\article{列傳一百三十八}

\begin{pinyinscope}
沈初金士松鄒炳泰戴聯奎王懿修子宗誠黃鉞

沈初,字景初,浙江平湖人。少有異稟,讀書目數行下,同郡錢陳群稱為異才。乾隆二十七年,南巡,召試,賜舉人,授內閣中書。明年,成一甲第三名進士,授編修。三十二年,直懋勤殿,合寫經為皇太后祝釐。逾年,大考翰詹,以直內廷未與試,詔褒初學問優美,特晉一秩,擢侍講。三十六年,直南書房,督河南學政,未赴任,丁祖母承重憂。服闋,遷右庶子。累擢禮部侍郎,督福建學政。遭本生父憂,服闋,起兵部侍郎。尋以母病乞歸終養。後起故官,督順天學政,調江蘇。任滿回京,調吏部,又督江西學政。

初以文學受知,歷充四庫全書館、三通館副總裁,續編石渠寶笈、秘殿珠林,校勘太學石經。嘉慶元年,與千叟宴,充會試知貢舉。擢左都御史,授軍機大臣,遷兵部尚書,歷吏、戶二部。四年,以老罷樞務,免直內廷,充實錄館副總裁。未幾卒,謚文恪,祀賢良祠。

金士松,字亭立,江蘇吳江人,寄籍宛平。舉順天鄉試,改歸原籍。乾隆二十五年,成進士,選庶吉士,授編修。遷侍讀,直懋勤殿寫經。典福建鄉試,督廣東學政。直南書房,累遷詹事,以生母憂歸。服闋,會高宗南巡,迎鑾道左,回京督順天學政。以寄籍辭,詔免回避,聯任凡七年。累擢禮部侍郎,調兵部。五十年,帝禦乾清宮,賜千叟宴。士松年五十七,未得與,特命試詩,賞賚同一品。調吏部,直講經筵,校勘石經,遷左都御史。嘉慶元年,再與千叟宴,遷禮部尚書。二年,調兵部,罷直書房。五年,扈蹕謁裕陵,途次嬰疾,遣御醫診視。還京,卒,謚文簡,祀賢良祠。

鄒炳泰,字仲文,江蘇無錫人。乾隆三十七年進士,選庶吉士,授編修,纂修四庫全書,遷國子監司業。國學因元、明舊,未立闢雍,炳泰援古制疏請。四十八年,高宗釋奠禮成,因下詔增建闢雍。逾兩年,始舉臨雍禮,稱盛典焉。尋超擢炳泰為祭酒。累遷內閣學士,歷山東、江西學政。嘉慶四年,授禮部侍郎,調倉場,剔除積弊。坐糧顏培天不職,劾去之。六年,京察,特予議敘。軍船交糧掛欠,已許抵補,後至者復然。炳泰慮年年積欠,與同官達慶意不合,自具疏奏,詔斥其偏執使氣,鐫級留任。又奏監督輪值宿倉,倉役出入滋弊,宜令於倉外官房居住,從之。十年,擢左都御史,遷兵部尚書,兼署工部,管理戶部三庫。十一年,兼管順天府尹事。十二年,調吏部。十四年,加太子少保。倉吏高添鳳盜米事覺,坐久任倉場無所覺察,褫宮銜,降二品頂戴,革職留任,久乃復之。十六年,署戶部尚書。尋以吏部尚書協辦大學士。

炳泰在吏部久,尤慎銓政。十八年,銓選兵部主事有誤,同官瑚圖禮徇司員議,回護堅執。炳泰力爭曰:「吾年已衰,何戀戀祿位?不可使朝廷法自我壞!」自具疏白其故,上韙其言,卒罷瑚圖禮。既而有降革官捐復者二人,準駁不當,侍郎初彭齡論與不合,疏聞,上斥炳泰無定見,鐫級留任。又盜劫兵部主事姚堃於昌平八仙莊,詔以地近京畿,官吏闒茸,不能治盜,罷炳泰兼管府尹事。及教匪林清變起,逆黨多居固安及黃村,追論炳泰在官不能覺察,以中允、贊善降補。尋休致,歸。二十五年,卒。

柄泰自初登第,不登權要之門,浮沉館職,久之始躋卿貳。屢掌文衡,稱得士。立朝不茍,仁宗重之,而終黜。

戴聯奎,字紫垣,江蘇如皋人。乾隆四十年進士,選庶吉士,授編修。聯奎少從邵晉涵受經學,既通籍,以清節自厲,在翰林久不遷。大學士嵇璜掌院事,將保送御史,列聯奎名,滿掌院學士曰:「吾未識其人,何以論其才否?」璜以語聯奎,使往見,聯奎漫應之,不往。及京察舉一等,又列聯奎名,復言如前,終不得與,璜乃益重之。和珅為掌院,訪時望傅其子豐紳殷德,或薦晉涵及聯奎,晉涵移病歸,聯奎亦堅辭。循資累遷至內閣學士。嘉慶九年,遷兵部侍郎,歷禮部、兵部、吏部。二十一年,擢左都御史。逾年,擢禮部尚書,調兵部。二十五年,失行印,坐降三品京堂,補太常寺卿,督浙江學政。道光元年,擢禮部侍郎,又擢兵部尚書。召還京,未至,卒。

王懿修,字仲美,安徽青陽人。乾隆三十一年進士,選庶吉士,授編修。入直上書房,授慶郡王永璘讀。典陜西、廣東、江西鄉試,督廣西、湖北學政,水存擢少詹事。五十四年,引病歸,終父母喪始出,復乞病在告。嘉慶元年,舉行千叟宴,懿修與焉,被禦制詩刻、玉鳩杖、文綺之賜。七年,起授通政司副使,歷光祿寺卿、內閣學士。八年,擢禮部侍郎,督順天學政。十年,擢左都御史,回京供職。尋擢禮部尚書,管戶部三庫事。十二年,充上書房總師傅。十四年,萬壽慶典,加太子少保,典會試。

懿修持躬端謹,制作雅正,甚被仁宗眷遇。十八年,以老致仕。逾二年,年八十,賜壽,謁宮門謝,逢上出御經筵,親解佩囊賜之。二十一年,卒。謚文僖。

子宗誠,字廉甫。乾隆五十五年一甲三名進士,授編修。嘉慶中,歷典云南、四川、陜西鄉試,督河南、山東、江西學政,洊擢禮部侍郎,歷工部、兵部,典會試。道光二年,擢兵部尚書,歷署禮部、工部尚書,兼管順天府尹。當懿修為侍郎時,宗誠已官學士,尋隨父扈蹕東巡,侍宴翰林院,父子同席。高宗實錄成,賜宴禮部,懿修以尚書主席。懿修致仕後,宗誠繼直上書房,海內推為榮遇。上亦以其兩世官禁近,皆能清慎,特優睞焉。道光十七年,卒。

黃鉞,字左田,安徽當塗人。乾隆五十五年進士,授戶部主事。時和珅管部務,鉞不欲趨附,乞假歸,不出。嘉慶四年,仁宗親政,硃珪薦之,召來京。入見,上曰:「朕居籓邸時,知汝名久矣,何以假歸不出?」鉞以實對,荷溫諭,尋直懋勤殿。九年,改贊善,入直南書房,未補官,命與考試差,典山東鄉試。十年,督山西學政,累遷庶子。十五年,差滿,仍直南書房,遷侍講學士。十八年,復典山東鄉試,留學政,擢內閣學士。是年,滑縣教匪起,蔓延山東,劾罷失察武生習教之菏澤訓導宋璇,請恤擊匪陣歿之曹州學錄孔毓俊、生員孔毓仲,獎勵手擒賊渠之金鄉生員李九標。十九年,召回京,仍內直,擢戶部侍郎,尋調禮部。充秘殿珠林、石渠寶笈續編總閱、全唐文館總裁,書成,並邀賞賚。復調戶部。二十四年,擢禮部尚書,加太子少保。二十五年,命為軍機大臣,尋調戶部尚書。

鉞受仁宗特達之知,久直內廷,書畫並被宸賞。習於掌故,持議詳慎。宣宗即位,始畀樞務,甚優禮之。道光四年,以年老罷直軍機。累疏乞休,六年,始許致仕,在籍食半俸。二十一年,卒,年九十二,贈太子太保,謚勤敏。

論曰:國家優禮詞臣,回翔禁近,坐致公卿。沈初、金士松,高宗舊臣,獲恩禮終。王懿修父子同朝,尤稱盛事。黃鉞以不附和珅,特邀殊遇,改授館職,馴參機務。鄒炳泰、戴聯奎皆有耿介之操,晚節枯菀乃殊,要不失為端人焉。


\end{pinyinscope}