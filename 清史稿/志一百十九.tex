\article{志一百十九}

\begin{pinyinscope}
○刑法三

太祖始創八旗,每旗設總管大臣一,佐管大臣二。又置理政聽訟大臣五人,號為議政五大臣。扎爾固齊十人,號為理事十大臣。凡聽斷之事,先經扎爾固齊十人審問,然後言於五臣,五臣再加審問,然後言於諸貝勒。眾議既定,猶恐冤抑,親加鞫問。天命元年,諭貝勒大臣曰:「國人有事,當訴於公所,毋得訴於諸臣之家。茲播告國中,自貝勒大臣以下有罪,當靜聽公斷,執拗不服者,加等治罪。凡事俱五日一聽斷於公所,其私訴於家,不執送而私斷者,治罪不貸。」十一年,太宗以議政五大臣、理事十大臣不皆分授,或即以總管、佐管兼之,於是集諸貝勒定議裁撤。每旗由佐管大臣審斷詞訟,不令出兵駐防。其每旗別設調遣大臣二員,遇有駐防調遣所屬詞訟,仍令審理。天聰七年,設刑部承政、參政、啟心郎等官,聽訟始有專責。

世祖入主中夏,仍明舊制,凡訴訟在外由州縣層遞至於督撫,在內歸總於三法司。然明制三法司,刑部受天下刑名,都察院糾察,大理寺駮正。清則外省刑案,統由刑部核覆。不會法者,院寺無由過問,應會法者,亦由刑部主稿。在京訟獄,無論奏咨,俱由刑部審理,而部權特重。刑部初設十四司。雍正元年,添置現審左右二司,審理八旗命盜及各衙門欽發事件。後復改並,定為十八清吏司:曰直隸,曰奉天,曰江蘇,曰安徽,曰江西,曰福建,曰浙江,曰湖廣,曰山東,曰山西,曰陜西,曰四川,曰廣東,曰廣西,曰雲南,曰貴州。凡各省刑名咨揭到部,各司具稿呈堂,以定準駮。吉林、黑龍江附諸奉天,甘肅、新疆附諸陜西,京曹各署關涉文件,亦分隸於十七司。現審則輪流簽分。順治十年,設督捕衙門,置侍郎滿、漢各一員,其屬有前司、後司。初隸兵部,專理緝捕逃旗事宜。康熙三十八年裁撤,將前後司改隸刑部。嗣復並為督捕一司,不掌外省刑名,亦不分現審。刑部收受訟案,已結未結,每月匯奏。設督催所,而督以例限。審結尋常徒、流、軍、遣等罪,按季匯題。案系奏交,情雖輕,專案奏結。死罪既取供,大理寺委寺丞或評事,都察院委御史,赴本司會審,謂之會小法。獄成呈堂,都察院左都御史或左副都御史,大理寺卿或少卿,挈同屬員赴刑部會審,謂之會大法。如有翻異,發司覆審,否則會稿分別題奏。罪幹立決,旨下,本司派員監刑。監候則入朝審。各省戶、婚、田土及笞、杖輕罪,由州縣完結,例稱自理。詞訟每月設立循環簿,申送督、撫、司、道查考。巡道巡歷所至,提簿查核,如有未完,勒限催審。徒以上解府、道、臬司審轉,徒罪由督撫匯案咨結。有關人命及流以上,專咨由部匯題。死罪系謀反、大逆、惡逆、不道、劫獄、反獄、戕官,並洋盜、會匪、強盜、拒殺官差,罪乾凌遲、斬、梟者,專摺具奏,交部速議。殺一家二命之案,交部速題。其餘斬、絞,俱專本具題,分送揭帖於法司科道,內閣票擬,交三法司核議。如情罪不符及引律錯誤者,或駮令覆審,或徑行改正,合則如擬核定。議上立決,命下,釘封飛遞各州縣正印官或佐貳,會同武職行刑。監候則入秋審。

朝審原於明天順三年,令每歲霜降後,但有該決重囚,三法司會同公、侯、伯從實審錄。秋審亦原於明之奏決單,冬至前會審決之。順治元年,刑部左侍郎黨崇雅奏言:「舊制凡刑獄重犯,自大逆、大盜決不待時外,餘俱監候處決。在京有熱審、朝審之例,每至霜降後方請旨處決。在外直省,亦有三司秋審之例,未嘗一麗死刑輒棄於市。望照例區別,以昭欽恤。」此有清言秋、朝審之始。嗣後逐漸舉行,而法益加密。初制分情實、緩決、矜、疑,然疑獄不經見。雍正以後,加入留養承祀,區為五類。截止日期,雲南、貴州、四川、廣東、廣西以年前封印日,福建以正月三十日,奉天、吉林、黑龍江、陜西、甘肅、湖北、湖南、浙江、江西、安徽、江蘇以月初十日,河南、山東、山西以三月初十日,直隸以三月三十日。然遇情重之案,雖後期有聲明趕入秋審者。刑部各司,自歲首將各省截止期前題準之案,分類編冊,發交司員看詳。初看藍筆句改,覆看用紫,輪遞至秋審處坐辦、律例館提調,墨書粘簽,一一詳加斟酌,而後呈堂核閱。朝審本刑部問擬之案,刑部自定實緩。秋審則直省各督撫於應勘時,將人犯提解省城,率同在省司道公同會勘,定擬具題。刑部俟定限五月中旬以前,各省後尾到齊,查閱外勘與部擬不符者,別列一冊。始則司議,提調、坐辦主之。繼則堂議,六堂主之,司議各員與焉。議定,刑部將原案及法司督撫各勘語刊刷招冊,送九卿、詹事、科道各一分,八月內定期在金水橋西會同詳核。先日朝審,三法司、九卿、詹事、科道入座,刑部將監內應死人犯提至當堂,命吏朗誦罪狀及定擬實、緩節略,事畢回禁。次日秋審,憑招冊審核,如俱無異議,會同將原擬陸續具題;有異,前期簽商。若各執不相下,持異之人奏上,類由刑部回奏聽裁。茍攻及原審,則徑行扣除再訊。二百餘年來,刑部歷辦秋、朝審,句稽講貫,備極周密,長官每以此校司員之優劣。究之人命至重,死者不可復生,其所矜慎,尤在實、緩。乾隆以前,各司隨意定擬,每不畫一。三十二年,始酌定比對條款四十則,刊分各司,並頒諸各省,以為勘擬之準繩。四十九年,復行增輯。嗣刑部侍郎阮葵生別輯秋讞志略,而後規矩略備,中外言秋勘者依之,並比附歷年成案,故秋、朝審會議,其持異特奏者,每不勝焉。

秋審本上,入緩決者,得旨後,刑部將戲殺、誤殺、擅殺之犯,奏減杖一百,流三千里,竊贓滿貫、三犯竊贓至五十兩以上之犯,奏減雲、貴、兩廣極邊、煙瘴充軍,其餘仍舊監固,俟秋審三次後查辦。間有初次入緩,後復改實者,權操自上,非常例也。入可矜者,或減流,或減徒。留養承祀者,將該犯枷號兩月,責四十板釋放。案系鬥殺,追銀二十兩給死者家屬養贍。情實則大別有三,服制、官犯、常犯是也。本下,內閣隨命欽天監分期擇日。句到,刑部按期進呈黃冊。至日,素服御殿,大學士三法司侍,上秉硃筆,或命大學士按單予句。服制冊大都殺傷期功尊長之案,既以情輕而改監候,類不句決;情實二次,大學士會同刑部奏請改緩。官犯則情重者,刑部從嚴聲敘,未容幸免;輕則一律免句,十次改緩。常犯之入情實,固罪無可逭者;其或一線可原,刑部粘簽聲敘,類多邀恩不句,十次亦改緩。向例句決重囚,刑科三覆奏,自乾隆十四年簡去二覆,第於句到前五日,覆奏一次。句到時,將原本進呈覆閱,一俟批發,在京例由刑科給事中、刑部侍郎各一人赴西市監視。官犯無論句否,俱綁赴行刑場候決。在外則刑部各司將句單連同榜示釘封送兵部發驛,文到之日行刑。如恭逢慶典或國家有故,則下旨停句。

順治十三年,諭刑部:「朝審秋決,系刑獄重典。朕必詳閱招案始末,情形允協,令死者無冤。今烸期伊邇,朝審甫竣,招冊繁多,尚未及詳細簡閱,驟行正法,朕心不忍。今年姑著暫停秋決,昭朕矜恤至意。」自是列朝於秋讞俱勤慎校閱。康熙二十二年,聖袓御懋勤殿,召大學士、學士等入,酌定在京秋審情實重犯。聖袓取罪案逐一親閱,再三詳審,其斷無可恕者,始定情實。因諭曰:「人命事關重大,故召爾等共相商酌。情有可原,即開生路。」雍正十一年,世宗御洞明堂,閱秋審情實招冊,諭刑部曰:「諸臣所進招冊,俱經細加斟酌,擬定情實。但此內有一線可生之機,爾等亦當陳奏。在前日定擬情實,自是執法,在此刻句到商酌,又當原情,斷不可因前奏難更,遂爾隱默也。」高宗尤垂意刑名,秋審冊上,每乾飭責。乾隆三十一年,湖南官犯饒佺,以其回護己過予句。迨閱浙省招冊,知府高象震亦以承審回護,原題僅擬軍臺效力。急諭湖南巡撫將饒佺暫停處決,令刑部查明兩案情節不同,始行明諭處分。其慎重讞典如此。仁宗亦嫻習法律。嘉慶七年,御史廣興會議秋審,奏請將鬥殺擬緩之廣東姚得輝改入情實,援引乾隆十八年「一命必有一抵」之旨。仁宗謂:「一命一抵,原指械斗等案而言,至尋常鬥毆,各斃各命,自當酌情理之平,分別實緩。若拘泥『一命必有一抵』之語,則是秋讞囚徒,凡殺傷斃命之案,將盡行問擬情實,可不必有緩決一項。有是理乎?」命仍照原擬入緩。其剖析法意,致為明允。自後宣宗、文宗遵循前軌,罕可紀述。穆宗、德宗兩經垂簾,每逢句到,命大學士一人捧單入內閣恭代,後遂沿為故事。

而前行之秋審條款,因光緒季年死刑遞有減降,法律館重加釐定,奏頒內外焉。

熱審之制,順治初賡續舉行。康熙十年,定每年小滿後十日起,至立秋前一日止,非實犯死罪及軍、流,俱量予減等。四十三年,諭刑部停止。雍正初復行。乾隆以後,第準免笞、杖,則遞行八折決放,枷號漸釋,餘不之及。且惟京師行之,外省笞、杖自理,無從考核,具文而已,列朝無寒審,而有軍、流、遣犯隆冬停遣之例。未起解者,十月至正月終及六月俱停遣。若已至中途,至十一月初一日準停。倘抵配不遠,並發往東南省分,人犯有情原前進者,一體起解。

又有停審之例,每年正月、六月、十月及元旦令節七日,上元令節三日,端午、中秋、重陽各一日,萬壽聖節七日,各壇廟祭享、齋戒以及忌辰素服等日,並封印日期,四月初八日,每月初一、初二日、皆不理刑名。然中外問刑衙門,於正月、六月、十月及封印日期、每月初一二等日不盡如例行也。其農忙停審,則自四月初一日至七月三十日,一應戶、婚、田土細故,不準受理,刑事不在此限。又有停刑之例,每年正月、六月及冬至以前十日,夏至以前五日,一應立決人犯及秋、朝審處決重囚,皆停止行刑。

凡審級,直省以州縣正印官為初審。不服,控府、控道、控司、控院,越訴者笞。其有冤抑赴都察院、通政司或步軍統領衙門呈訴者,名曰京控。登聞鼓,順治初立諸都察院。十三年,改設右長安門外。每日科道官一員輪值。後移入通政司,別置鼓。其投擊鼓,或遇乘輿出郊,迎駕申訴者,名曰叩閽。從前有擅入午門、長安門、堂子跪告,及打長安門內、正陽門外石獅鳴冤者,嚴禁始絕。即迎車駕而沖突儀仗,亦罪至充軍。京控及叩閽之案,或發回該省督撫,或奏交刑部提訊。如情罪重大,以及事涉各省大吏,抑經言官、督撫彈劾,往往欽命大臣蒞審。發回及駮審之案,責成督撫率同司道親鞫,不準復發原問官,名為欽部事件。文武官犯罪,題參革職。道府、副將以上,遴委道員審理。同知、游擊以下,遴委知府審理。巡按御史,順治初猶常設。四年,從大理寺卿王永吉奏,差官往直省恤刑,然皆不久停罷。外省刑名,遂總匯於按察使司,而督撫受成焉。京師笞、杖及無關罪名詞訟,內城由步軍統領,外城由五城巡城御史完結,徒以上送部,重則奏交。如非常大獄,或命王、大臣、大學士、九卿會訊。自順治迄乾隆間,有御廷親鞫者。律稱八議者犯罪,實封奏聞請旨,不許擅自句問。在京大小官員亦如之。

若宗室有犯,宗人府會刑部審理。覺羅,刑部會宗人府審理。所犯笞、杖、枷號,照例折罰責打;犯徒,宗人府拘禁;軍、流、鎖禁,俱照旗人折枷日期,滿日開釋。屢犯軍、流,發盛京、吉林、黑龍江等處圈禁;死刑,宗人府進黃冊。閹寺犯輕罪,內務府慎刑司訊決,徒以上亦送部。八旗地畝之訟,屬諸戶部現審處,刑事統歸刑部。清初有都統會審之制,有高墻拘禁之條,至乾隆時俱廢。旗營駐防省分,額設理事同知。旗人獄訟,同知會同州縣審理。熱河都統衙門特設理刑司,刑部派員聽訟,三年一任。同治三年,以吉林獄訟繁多,詔依熱河設立刑司例,令刑部揀派滿、漢郎中、員外、主事各一員,分別掌印主稿,統歸將軍管轄。嗣吉林建省裁撤,而熱河如故。

蒙古刑獄,內外扎薩克王公、臺吉、塔布囊及協理臺吉等承審。康熙三十七年,曾遣內地官員教導蒙古王等聽斷盜案,後不常設。沿邊與民人交涉案件,會同地方官審理,死罪由盟長核報理籓院,會同三法司奏當。在京犯斬、絞,刑部審訖,會理籓院法司亦如之。盛京刑部掌讞盛京旗人及邊外蒙古之獄。秋審,會同四部侍郎、奉天府尹酌定實、緩匯題,蓋皆特別之制。

凡檢驗,以宋宋慈所撰之洗冤錄為準,刑部題定驗尸圖格,頒行各省。人命呈報到官,地方正印官隨帶刑書、仵作,立即親往相驗。仵作據傷喝報部位之分寸,行兇之器物,傷痕之長短淺深,一一填入尸圖。若尸親控告傷痕互異,許再行覆檢,不得違例三檢。如自縊、溺水、事主被殺等案,尸屬呈請免驗者,聽。京師內城正身旗人及香山等處各營房命案,由刑部當月司員往驗。街道及外城人命,無論旗、民,歸五城兵馬司指揮相驗。檢驗不以實者有刑。

凡訊囚用杖,每日不得過三十。熱審得用掌嘴、跪鍊等刑,強盜人命酌用夾棍,婦人指,通不得過二次。其餘一切非刑有禁。斷罪必取輸服供詞,律雖有「眾證明白,即同獄成」之文,然非共犯有逃亡,並罪在軍、流以下,不輕用也。

凡審限,直省尋常命案限六閱月,盜劫及情重命案、欽部事件並搶奪掘墳一切雜案俱定限四閱月。其限六月者,州縣三月解府州,府州一月解司,司一月解督撫,督撫一月咨題。其限四月者,州縣兩月解府州,府州二十日解司,司二十日解督撫,督撫二十日咨題。如案內正犯及要證未獲,或在監患病,準其展限或扣限。若隔屬提人及行查者,以人文到日起限。限滿不結,督撫咨部,即於限滿之日起算,再限二、三、四月,各級分限如前。如仍遲逾,照例參處。按察司自理事件,限一月完結。州縣自理事件,限二十日審結。上司批發事件,限一月審報。刑部現審,笞杖限十日,遣、軍、流、徒二十日,命盜等案應會三法司者三十日。每月奏報,聲明曾否逾限。如有患病及查傳等情,亦得依例扣展。速議速題,均限五日覆。死罪會核,自科鈔到部之日,立決限七十日,監候限八十日。會同題覆,院寺各分限八日。由咨改題之案,展限十日。系清文加譯漢十日或二十日,逾限附參。盜賊逾月不獲,捕役汛兵予笞,官罰俸。吏兵兩部處分則例,尚有疏防及初、二、三、四參之分。命案兇犯在逃,承緝、接緝亦按限開參。然例雖嚴,而巧於規避者,蓋自若也。

凡解犯有三:一、定案時之解審。徒犯解至府州轉報,軍、流、遣及死罪,自府州遞省,逐級訊問無異,督撫然後咨題。一、秋審時之解勘。死罪非立決,發回本州縣監禁,逮秋審,徑行解司審勘。官犯自定案即拘禁司監待決。常犯緩決者,二次秋審,即不復解。其直省各邊地離督撫駐處窵遠,有由該管巡道審勘加結轉報者,非通例也。一、發遣時之解配。徒囚問發隔縣,軍、流起解省分,預行咨明應發省分督撫,查照道里表,酌量州縣大小遠近、在配軍流多寡,先期定地,飭知入境首站州縣,隨到隨發。遣犯解至例定地方安插。犯籍州縣僉差,名曰長解。沿途州縣,派撥兵役護送,名為短解。罪囚視罪名輕重,定用鐵鎖杻金道數。若中途不覺失囚,訊明有無賄縱,分別治罪。隔屬關提及發交各地方官管束者,視此為差。京師現審,徒犯發順天府充徒。流囚由刑部定地,劄行順天府起送。五軍咨由兵部定地提發,外遣亦咨兵部差役起解。綜計訴訟所歷,自始審迄終結,其程序各有定規,毋或逾越。

迨光緒變法,三十二年,改刑部為法部,統一司法行政。改大理寺為大理院,配置總檢察,專司審判。於是法部不掌現審,各省刑名,畫歸大理院覆判,並不會都察院,而三法司之制廢。題本改為摺奏,內閣無所事事。秋、朝審專屬法部,其例緩者隨案聲明,不更加勘,而九卿、科道會審之制廢。京師暨各省設高等審檢,都城省會及商埠各設地方及初級審檢,改按察使為提法司。三十二年,法部奏定各級試辦章程。宣統二年,法律館奏頒法院編制法,由初級起訴之案不服,可控由地方而至高等,由地方起訴之案不服,可控由高等而至大理院,名為四級三審。從前審級、審限、解審、解勘之制,州縣行之而不行於法院。審判分民事、刑事。民律艱於成書,所據者第舊律戶役、田宅、錢債、婚姻各條,而法未備。司法事務有年度,判斷有評議,刑事有檢察官蒞審,人命由檢察官相驗,法院行之而不能行於州縣。刑訴制度,蓋雜糅矣。

然爾時所以急於改革者,亦曰取法東西列強,藉以收回領事裁判權也。考領事裁判,行諸上海會審公堂,其源肇自咸豐朝,與英、法等國締結通商條約,約載中外商民交涉詞訟,各赴被告所屬之國官員處控告,各按本國律例審斷。嗣遇他國締約,俱援利益均霑之說,群相仿效。同治八年,定有洋涇濱設官章程,遴委同知一員,會同各國領事審理華洋訴訟。其外人應否科刑,讞員例不過問。華人第限於錢債、鬥毆、竊盜等罪,在枷杖以下,準其決責。後各領擴張權限,公堂有逕定監禁數年者。外人不受中國之刑章,而華人反就外國之裁判。清季士大夫習知國際法者,每咎彼時議約諸臣不明外情,致使法權坐失。光緒庚子以後,各國重立和約,我國齗齗爭令撤銷,而各使藉口中國法制未善,靳不之許。迨爭之既亟,始聲明異日如審判改良,允將領事裁判權廢棄。載在約章,存為左券。故二十八年設立法律館,有「按照交涉情形,參酌各國法律,務期中外通行」之旨。蓋亦欲修明法律,俾外國就範也。夫外交視國勢之強弱,權利既失,豈口舌所能爭。故終日言變法,逮至國本已傷,而收效卒鮮,豈法制之咎與?然其中有變之稍善而未竟其功者,曰監獄。有政體所關而未之變者,曰赦典。

監獄與刑制相消息,從前監羈罪犯,並無已決未決之分。其囚禁在獄,大都未決犯為多。既定罪,則笞、杖折責釋放,徒、流、軍、遣即日發配,久禁者斬、絞監候而已。州縣監獄,以吏目、典史為管獄官,知州、知縣為有獄官,司監則設按司獄。各監有內監以禁死囚,有外監以禁徒、流以下,婦人別置一室,曰女監。徒以上鎖收,杖以下散禁。囚犯日給倉米一升,寒給絮衣一件。鎖杻常洗滌,席薦常鋪置,夏備涼漿,冬設暖床,疾病給醫藥。然外省監獄多湫隘,故例有輕罪人犯及干連證佐,準取保候審之文。無如州縣懼其延誤,每有班館差帶諸名目,胥役藉端虐詐,弊竇叢滋。雖屢經內外臣工參奏,不能革也。刑部有南北兩監,額設司獄八員、提牢二員,掌管獄卒,稽查罪囚,輪流分值。每月派御史查監,有瘐斃者亦報御史相驗。年終並由部匯奏一次,防閑致為周備。自光緒三十二年審判畫歸大理院,院設看守所,以羈犯罪之待訊者,各級審檢亦然,於是法部犴狴空虛。別設已決監於外城,以容徒、流之工作,並令各省設置新監,其制大都採自日本。監房有定式,工廠有定程。法律館特派員赴東調查,又開監獄學堂,以備京、外新監之用。然斯時新法初行,措置未備,外省又限於財力,未能遍設也。

赦典有恩赦、恩旨之別。歷朝登極、升祔、冊立皇后、皇上五旬以上萬壽、皇太后六旬以上萬壽及武功克捷之類,例有恩赦。其詔書內開:一、官吏軍民人等有犯,除謀反、大逆、子孫謀殺祖父母父母、內亂、妻妾殺夫、奴婢殺家長、殺一家非死罪三人、採生折割人、謀殺故殺真正人命、蠱毒魘魅毒藥殺人、強盜、妖言、十惡等真正死罪不赦外,軍務獲罪、隱匿逃人及侵貪入己亦不赦外,其餘已發覺未發覺、已結未結者,咸赦除之。若尋常萬壽及喜慶等事,則傳旨行赦。恩赦死罪以下俱免,恩旨則死罪已下遞減。詔書既頒,刑部檢查成案,分別準免不準免,開單奏定,名為恩赦條款。恩旨則分別準減不準減,名為減等條款。部設減等處,專司核駮。其巡幸所經,赦及一方,及水旱兵災、清理庶獄者,則視詔旨從事焉。明制,徒、流已至配,不復援赦。清自康熙九年準在配徒犯會赦放免。乾隆二年恩詔,軍、流在配三年,安靜悔過,情原回籍,查明準釋。迨嘉慶二十五年,始將到配未及三年人犯一體查辦,尤為曠典。昔人有言:「赦者小人之幸,君子之不幸。」意第謂赦恩之不可濫耳。若夫非常慶典,特頒汗號,使之蕩滌瑕穢,灑然自新,未始非仁政之一端。有清一代,赦典屢頒,然條款頗嚴,毋虞濫及。且行慶施惠,王者馭世之大權,非茍然也。故光緒三十四年宣統登極,猶循例大赦雲。


\end{pinyinscope}