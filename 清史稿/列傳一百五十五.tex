\article{列傳一百五十五}

\begin{pinyinscope}
楊芳胡超齊慎郭繼昌段永福武隆阿哈★阿巴哈布

長清達凌阿哈豐阿慶祥舒爾哈善烏凌阿穆克登布多隆武

壁昌恆敬

楊芳,字誠齋,貴州松桃人。少有幹略,讀書通大義。應試不售,入伍,充書識。楊遇春一見奇之,薦補把總。從征苗疆,戰輒摧鋒。洊擢臺拱營守備。

嘉慶二年,從額勒登保剿教匪,敗張漢潮於南漳,賜花翎。轉戰川、陜,常充偵騎,深入得賊情地勢,額勒登保連破劇寇,賴其鄉導之力。四年,殲冷天祿於人頭堰。大軍追餘賊,芳以九騎前行,至石筍河,見賊數千爭渡,後逼陡崖,左右無路,芳遣二騎回報,自將七騎大呼馳下,賊驚潰,陷淺洲中,其先渡者無由回救。五舟離岸,群賊蟻附,舟重,每發一矢覆一舟,五發五覆。俄,楊遇春、穆克登布至,浮馬渡,追擊賊盡,軍中稱為奇捷。連擢平遠營都司、下江營游擊、兩廣督標參將。

五年,楊開甲、張天倫趨雒南,芳以千騎扼東路,繞出賊前。賊折而西,黎明追及,見馬跡中積水猶潢,急馳之。甫轉山灣,見賊擁塞平川,芳率數十騎沖突,後騎至,乘勢蹂躪,賊倉卒奔潰,擒斬無算。賜號誠勇巴圖魯,擢廣西新泰協副將。尋從穆克登布擊伍懷志,連敗之成縣、階州。賊渡白水河窺四川龍安,旁入老林,冒雨追擊,及之於磨刀石,手刃十餘賊,傷足墜馬,徒步殺賊,復傷臂,射傷伍懷志,大軍乘之,大破賊眾。仁宗聞而嘉之,詔問傷狀。六年,冉學勝趨甘肅,偕札克塔爾要擊於固原,賊反奔,芳輕騎摧其後隊,又敗之於漢江南岸,賊由平利走洵陽。時張天倫踞高唐嶺,芳破之,餘賊與學勝合,東出楊柏坡,芳先至,設伏敗之,而李彬、茍文明、高見奇、姚馨佐合竄平利。彬走南江,天倫隨之,見奇、馨佐入寧羌。額勒登保自追之,囑芳以南江之賊,擊天倫,擒其黨張良祖、馬德清、劉奇;復破見奇、馨佐於桂門關,追及黑洞溝,擒其黨辛斗:擢陜西寧陜鎮總兵。又敗李彬於太平,賊棄老弱逸,獲彬妻及其悍黨冉天璜。七年,茍文明犯寧陜,其黨劉永受、宋應伏分布秦嶺北。芳由五郎口進,殲應伏之眾過半,永受遁,為寨民所殺,文明尋亦授首。額勒登保入楚,檄芳剿陜境餘匪,先後擒郭士嘉、茍文學等,賊黨潰散。

八年,總督惠齡檄芳還剿南山賊,芳由洵陽壩深入,冒雨捫崖攀葛,獮薙無遺,遂大搜秦嶺南北,陜西賊垂盡。忽有李彪者,自太白山突出,合茍文潤擾洋縣。芳截剿勿及,坐奪翎頂。賊逼川境,德楞泰至,令芳歸防山內。茍文明餘黨自竹谿竄陜,芳嚴守漢江,卻之,復翎頂。是年秋,三省悉平,凱撤諸軍。

寧陜鎮標皆選鄉勇精銳充伍,凡五千人,號新兵,芳馭之素寬。十一年,芳代楊遇春署固原提督,去鎮,副將楊之震攝。以包穀充糧,又鹽米銀未時給,眾鼓噪,營卒陳達順、陳先倫遂倡亂,戕之震,其黨蒲大芳護芳家屬出而復從賊。芳聞變,馳赴石泉,詔德楞泰率楊遇春等討之。秋,賊大掠洋縣、留壩,脅眾盈萬,推大芳為魁。攻孝義,窺子午谷,圍鄠縣急。芳馳救,鏖戰終夜,傷臂。旦日,賊辨為芳,自引去。遇春督諸軍戰於方柴關,不利。芳與遇春計,賊尚感舊恩,可勸諭,單騎入賊,曉以順逆利害,猶倔強,與語數年共生死情,聲淚俱下,眾感泣原降,遂宿賊壘。大芳縛達順、先倫以獻;復率大芳追斬不聽命者硃貴等數百人,乃定。德楞泰疏請降兵歸伍,被譴責,大芳等二百餘人免死戍伊犁。芳坐馭兵姑息,亦褫職遣戍。明年,釋還,以守備、千總用。十五年,授廣東右翼鎮總兵,調陜西西安鎮。母憂,去官。

十八年,服闋,入都,至河南,會教匪李文成踞滑縣,總統那彥成留之剿賊,授河北鎮總兵。偕楊遇春克道口,進薄滑縣。巡撫高杞有兵六千,與總統不協,戰不力,芳說杞,盡領其眾。文成走踞輝縣司寨,偕特依順保追擊之,賊死鬥,芳手刃退卒,大捷,以火攻破碉樓,文成自焚死,予雲騎尉世職。大兵隧地攻滑城,賊多方御之,歷四十日不得下。芳復於西南隅穿穴深入,九日而成。地雷發,城圮,殄賊二萬餘。蕆功優敘,調西安鎮。移師剿平三才峽匪,復勇號,調漢中鎮。二十年,擢甘肅提督。

道光初,歷直隸、湖南、固原提督。六年,回疆軍事急,芳自請從征,許之。十月,會軍阿克蘇。柯爾坪為要沖,芳先進,一鼓破之,焚回莊,斬賊酋伊瞞及安集延偽帥約勒達什,大軍無阻。七年二月,偕參贊楊遇春、武隆阿進師,三戰皆捷,抵喀什噶爾渾河北,合擊大破之,遂復其城;率兵六千趨和闐,三月,戰於毗拉滿,分軍繞賊後夾擊,擒賊酋噶爾勒,復和闐:加騎都尉世職,授乾清門侍衛。張格爾已遁,命楊遇春偕芳出卡掩捕,芳軍阿賴,檄諸夷部縛獻。芳言賊遁愈遠,道險餉艱,諸夷貪賞妄報不足信,至秋,詔班師。會芳追博巴克之眾,入險遇伏,數戰始拔全軍出,協領都凌阿死之。遇春先入關,芳代為參贊,遣黑回用間言大兵全退。張格爾俟歲將除,率五百騎來襲,中途覺而反奔。芳急馳一晝夜,追及於喀爾鐵蓋山,殲其從騎殆盡。餘賊擁張格爾登山,棄騎走,芳率胡超、段永福等擒之,錫封三等果勇侯,賜紫韁、雙眼花翎,晉御前侍衛,賜其子承注舉人。張格爾械京伏誅,加太子太保。九年,入覲,晉二等侯,加太子少傅。十年,浩罕、安集延復擾喀什噶爾、葉爾羌等城,偕長齡往剿,仍為參贊。兵至,賊已遁。疏言移城屯田事,下長齡等議行。尋回鎮。

十三年,四川清溪、越巂、瓘邊諸夷叛,提督桂涵卒於軍,以芳代之。至則清溪、越巂皆平,進攻瓘邊賊巢,斬其酋,十二姓熟夷皆降,山內惈夷亦就撫。與按察使花傑籌治善後,晉一等侯。逾年,諸夷復時出擾,降二等侯,褫御前侍衛,以甘肅總兵候補。引疾歸。十六年,起為湖南鎮筸總兵,撫定變兵。歷廣西、湖南提督。

二十年,海疆事起,定海既陷,琦善赴廣東議撫,英吉利要挾,攻奪砲臺。二十一年春,命奕山為靖逆將軍,芳及隆文為參贊,率師防剿。奕山等不知兵,惟倚芳。先至廣州,英兵入犯虎門、烏湧,提督關天培戰死。敵兵逼省城,嚴備守御。芳見兵不可恃,而洋商久停貿易,亦原休戰,美利堅商人居間,請通商,詔不許;又偕巡撫怡良疏請準港腳商船貿易,詔斥有意阻撓,怠慢軍心,嚴議奪職,改留任。奕山至,戰亦不利。四月,英艦退,收復砲臺,奕山等遂請班師。芳以老病乞解職,溫諭慰之,命回湖南本任。二十三年,許致仕,在籍食全俸。二十六年,卒於家,詔念前勞,賜金治喪,依例賜恤,予其諸孫官有差,謚勤勇。子承注先卒,孫恩科襲侯爵。

芳自剿三省教匪,勛名亞於楊遇春。至回疆之役,以生擒首逆,先封侯,繪像紫光閣,論功超列遇春上。漢臣同列者凡九人:署固原提督胡超,貴州提督餘步雲,直隸提督齊慎,安徽壽春鎮總兵郭繼昌,陜西西固營都司段永福,陜西馬兵升甘肅寧遠堡守備楊發,陜西馬兵升撫標左營守備田大武。發、大武並從擒張格爾,以伍卒躋列,異數也。

胡超,四川長壽人。初讀書應試不售,入伍,從征苗疆有功。嘉慶中,川、楚、陜教匪起,率鄉勇轉戰,屢殲悍賊,以勇健名。累擢都司,坐事奪職。入都,考充國史館供事。十八年,林清逆黨犯禁城,手殺數賊,大學士勒保薦赴河南軍營。從楊遇春剿賊,單騎入賊壘,與數十賊搏戰,殲其二,搴旗而出;又敗賊於中市,率勁騎前驅,克道口,復原官。克滑城,擒賊首,上功居最。十九年,從遇春平三才峽匪,殪賊目麻大旗、劉二,擒龔貴等,賜號勁勇巴圖魯。累擢陜西循化營參將。

道光元年,從征叛番,戰博洛托亥、烏蘭哈達皆捷,夜襲凍雪嶺賊帳,擢甘肅永昌協副將,駐防西寧。六年,回疆事起,楊遇春檄赴軍。從楊芳攻柯爾坪,先破賊於和色爾湖,次日攻北莊,持矛步戰,殺賊過半,陣斬賊首伊瞞,加總兵銜。七年,連戰皆捷,抵渾河,賊夜來襲,擊敗之,遂渡河薄賊壘,賊大潰。四城既復,追和闐逸賊,出卡至瑪雜敗之,截擊於新地溝,盡殲其眾,擢四川重慶鎮總兵。是年冬,追張格爾至喀爾鐵蓋山,舍騎步躡山巔,張格爾窮蹙欲自剄,超與段永福奪其刀,生縛之,予騎都尉世職,授乾清門侍衛。與功臣宴,禦制贊有「雄勇超群,名實克稱」之褒。歷署古北口、固原提督,授甘肅提督。

十年,浩、安集延復犯邊,超率兵四千馳剿,至英吉沙爾,賊已遁,遂解喀什噶爾圍。分兵追薩漢莊竄匪,俘戮殆盡。凱旋,調固原提督。十六年,入覲,命在御前行走。二十一年,命率兵二千赴山海關駐防。尋以浙江海防急,授參贊大臣赴援,未行,留防天津。從郡王僧格林沁視直隸、山東海口防務,逾年撤防歸伍。尋調甘肅提督。二十六年,以西寧番叛,調援不力,褫職,仍留騎都尉。乞病歸,食半俸。二十九年,卒。

齊慎,河南新野人。以武生率鄉團擊教匪。入伍,隸慶成部下,轉戰三省,以勇聞。比教匪平,洊擢至陜安鎮右營游擊,楊遇春甚器之。嘉慶十八年,滑縣亂,檄慎從征。賊踞道口,遇春初至,直前搏戰,慎從之,賊氣奪,入巢。明日,慎獨破賊於衛河西岸。賊掠中市,率騎斷其歸路,夾擊,毀浮橋,遂克道口,破桃源集援賊。進薄滑縣,駐營未定,賊萬餘由西北門出來犯,力戰,相持竟夜;遲明,城賊二千餘復出,慎躍馬沖賊陣中斷,乃大潰。又破賊新鄉牛市,首逆李文成走踞司寨,慎由淇縣大廟山右進,鏖戰白土岡,會攻司寨,克之。自道口至此凡十三戰,敘功最,賜號健勇巴圖魯。克滑城,先登受傷,擢副將;遂從遇春平三才峽匪,授神木協副將。歷西安、陜安兩鎮總兵。

道光元年,擢甘肅提督。二年,西寧插帳番擾河北,慎率本標兵迭戰於烏蘭哈達、哈錫山、落它灘,擒斬數百,番眾乞降,放還河南。詔褒獎,被珍賚。六年,從征回疆,長齡令充翼長,駐守阿克蘇。父喪,留軍。特奇里克愛曼布魯特助逆擾烏什,慎戰屢捷,擒其酋庫圖魯克。七年,出哈蘭德卡倫,駐倭胡素魯,遏賊內犯。事平,調古北口提督,改號強謙巴圖魯。十二年,病歸。起授甘肅提督,調四川。十七年,平雷波叛夷,調雲南,復調四川。

二十一年,命率川兵五百赴廣東參贊靖逆將軍奕山軍務,守佛山鎮。楊芳病,移守省城,會罷戰。二十二年,赴湖北剿崇陽亂民,未至已定,命赴浙江會辦揚威將軍奕經軍務,駐上虞,扼曹娥江。移防江蘇鎮江。英兵來犯,力戰卻敵。城卒陷,退守新豐。奕山、奕經先後被譴,慎奪職留任,回四川。二十四年,出閱伍,卒於馬邊,贈太子太保,謚勇毅。

郭繼昌,直隸正定人。以行伍從慶成剿教匪於襄陽,繼從恆瑞入川,擊羅其清、冉文儔等於龍鳳坪,殲冉文富於馬鞍山,功皆最。又赴陜、甘剿張漢潮,擢龍固營都司。累遷陜西宜君營參將。道光元年,赴喀什噶爾換防,授定邊協副將,調安西協。六年,換防葉爾羌,抵阿克蘇,值亂起,駐守托什罕,擊敗渡河賊。協領都倫布被圍,繼昌兵少不能救,借調額爾古倫騎隊三百,夜率馳往,突賊營,殲其酋庫爾班素皮,追及河上,擒斬千餘,擢總兵,賜號幹勇巴圖魯。七年,從大軍戰大河拐,夜襲賊營,破之。從復喀什噶爾城,追賊至塔裏克達坡,分兵繞山後狙擊,賊驚潰,授壽春鎮總兵。調陜西延榆綏鎮。十年,再赴喀什噶爾剿餘孽,還署固原提督。十七年,調廣東陸路提督。洎海防急,往來廣、惠間籌守御。二十一年,以勞卒。

段永福,陜西長安人,原籍四川。以鄉勇從征教匪,積功至千總。嘉慶十八年,滑縣教匪起,從楊遇春轉戰直隸、河南,克道口、司寨,復滑縣,皆有功。復從遇春剿陜西郿縣賊,率騎兵追至柏楊嶺,殲賊目麻大旗、劉二於陣。累擢甘肅張義營都司。道光七年,從楊芳征回疆,洋阿爾巴特、沙布都爾、阿瓦巴特三戰皆力,賜號利勇巴圖魯。張格爾就擒於喀爾鐵蓋山,永福從胡超步上山嶺,直前奪其刀,手縛之,予騎都尉世職。擢參將,歷甘肅永固協副將,陜西寧夏鎮總兵,調貴州安義鎮。二十年,命赴廣東防海,英吉利兵艦初至,永福扼虎門,砲擊退之。二十二年,命赴浙江佐揚威將軍奕經軍,寧波、鎮海已陷,令永福分路往攻,漏師期,他路先挫,永福師不得進,遂無功。擢廣西提督,未赴,調浙江。未幾,卒,謚勇毅。

武隆阿,瓜爾佳氏,滿洲正黃旗人,提督七十五子。嘉慶初,以健銳營前鋒從征湖北教匪,後隨父剿賊四川,功多,累擢副都統。七十五以病去,武隆阿代領所部留川,為勒保所忌,父喪,乃還京。十年,授廣東潮州鎮總兵。時海盜充斥,仁宗以武隆阿勇敢,故使治之。既而總督那彥成招降盜首李崇玉,予四品銜守備劄,而以武隆阿捕獲聞。事覺,坐降二等侍衛,赴臺灣軍營效力。十一年,偕王得祿等擊蔡牽於鹿耳門,敗之,遷頭等侍衛,授臺灣鎮總兵。二十五年,母憂,回旗。尋充喀什噶爾參贊大臣。道光元年,疏陳八旗生計,請以綠營兵半為旗額,由駐防子弟挑補,詔斥紊言亂政,降二等侍衛,調西寧辦事大臣。三年,召還,授內閣學士。出為直隸提督,授江西巡撫,調山東。

六年,臺灣奸民張丙作亂,詔武隆阿往督師,未行而回疆亂急,授欽差大臣,與楊遇春同參贊揚威將軍長齡軍務,率吉林、黑龍江騎兵三千出關。七年二月,戰於洋阿爾巴特,武隆阿將右軍,扼其前,賊敗走,追至排子巴特,又敗之,進克沙布都爾回莊,乘勝至渾水河,悍賊數千來援,迎擊破之,斬其酋色提巴爾第等。進次阿瓦巴特,賊伏精銳以待,遣羸師挑戰,佯敗,武隆阿整隊進,以連環槍聚擊,別遣藤牌軍由山谷間道沖出,賊馬驚卻走,伏賊自林中出,不復成列,縱擊之,殪賊萬餘,斬其酋阿瓦子邁瑪底、那爾巴特阿渾等。捷聞,加太子少保。賊壘踞渾河南岸,列大砲山穴,死守以拒,武隆阿軍至不得進。日暮,偕楊遇春乘風潛渡上游襲賊後,賊數進數退,卒不支,始潰走,遂復喀什噶爾城。

張格爾聞敗先遁,詔斥將軍、參贊不能生致首逆,並被譴,奪武隆阿宮銜,責擒張格爾以自贖。武隆阿病留喀城,授喀什噶爾參贊大臣。詔詢善後方略,長齡請以逆裔阿布都哈裡管西四城回部事。武隆阿亦疏言:「留兵少則不敷戰守,留兵多則難繼度支。前此大兵進剿,幸克捷迅速,奸謀始息。臣以為西四城環逼外夷,處處受敵,地不足守,人不足臣,非如東四城為中路不可少之保障。與其糜有用兵餉於無用之地,不如歸並東四城,省兵費之半,即可鞏如金甌,似無需更守此漏卮。」詔切責其附和長齡。會諜報張格爾潛居達爾瓦,武隆阿率師往擊之,侍衛色克精阿等歿於陣,上愈怒,議革職,從寬留任。尋以病亟請解職,允之,命在喀城調理,病愈仍署原官。八年,張格爾就擒,免前後吏議。尋實授喀什噶爾參贊大臣,奏招撫歸順部落額提格訥布魯特,安置依劣克達阪地。詔以「受降易,安撫難」勉之。召回京。

九年,陜、甘兵凱撤,給鹽糧銀依內地防軍舊例,軍士意不滿,譁噪。那彥成疏言:「武隆阿戰陣勇敢,而多疑少斷,未洽人心。陜軍囂爭,實其意存節省、拘泥成例所致,慮不勝參贊任。」及至京召對,語復掩飾,降頭等侍衛。尋充和闐辦事大臣。十年,召還。逾年,卒。

武隆阿回疆戰功與二楊相埒,以言棄地獲譴,未膺優賞。宣宗念前勞,仍列功臣,繪像紫光閣。八旗諸將同列者:都統威勇侯哈哴阿,護軍統領阿勒罕保,庫爾烏蘇領隊大臣副都統巴哈布,副都統蘇清阿,阿克蘇辦事大臣副都統長清,塔爾巴哈臺參贊大臣達凌阿,察哈爾都統安福,頭等侍衛巴清德,吉林副都統吉勒通阿,喀什噶爾幫辦大臣副都統銜額爾古倫,頭等侍衛塔爾巴哈臺辦事大臣德勒格爾桑,頭等侍衛華山泰,寧夏副都統伊勒通阿,吉林協領壽昌,黑龍江協領鄂爾克彥、全凌阿,黑龍江總管副都統銜舒凌阿,伊犁察哈爾總管烏齊拉爾,三等侍衛得勝額,吉林佐領烏凌額、德成額,黑龍江佐領占布、阿勒吉訥,伊犁錫伯佐領德克精阿,伊犁索倫副總管哈丹保,伊犁錫伯馬甲防禦銜驍騎校訥松阿、舒興阿,而回子郡王伊薩克亦與焉。

哈哴阿,瓜爾佳氏,滿洲正黃旗人。由世襲雲騎尉為伯父額勒登保嗣,襲一等威勇侯,授頭等侍衛、乾清門行走。嘉慶十八年,從剿滑縣教匪有功,賜號繼勇巴圖魯。二十一年,晉御前侍衛,兼副都統、武備院卿,歷護軍前鋒統領。

道光六年,從長齡赴回疆,充領隊大臣,將騎兵。連戰洋阿爾巴特、沙布都爾、阿瓦巴特,擒安集延頭目阿瓦子邁瑪底等,復喀什噶爾,擒逆屬及從逆伯克阿布都拉、安集延頭目推立汗。從楊芳破玉努斯於毗拉滿,復和闐,擢鑲紅旗蒙古都統。八年,檻送張格爾至京,獻俘闕下,禮成,賜蟒袍、大緞。十年,喀什噶爾復被圍,授參贊大臣,從長齡視師,至則賊已遁,命偕楊芳察各城戰守及回眾助逆者,捕誅百餘人,被脅免罪,獎賞有功,並如議行。留回疆駐守,訓練屯兵。十二年,浩罕遣使進表,送還所掠回民,率貿易人進卡,哈哴阿受之,宣示通商免稅恩詔,賜予筵宴,事畢還京。

臺灣匪起,授參贊大臣,偕將軍瑚松額往剿,未至,事平,旋師。十五年,命赴山、陜閱兵,擢領侍衛內大臣。尋以閱兵不慎,降二等侍衛。累遷都統。二十一年,海疆戒嚴,駐防山海關,復授參贊大臣,偕奕經赴浙江防剿。未幾,仍回山海關防守。和議成,回京,授領侍衛內大臣。二十五年,以病請解職,食侯爵全俸。二十九年,卒,贈太子少保,謚剛恪。子那銘,孫榮全,襲爵。榮全官至副都統,自有傳。

巴哈布,伍彌特氏,蒙古正黃旗人。以健銳營前鋒、藍翎長從征教匪,又赴臺灣剿賊,累遷前鋒參領。以克滑縣功,授右翼翼長,擢鑲藍旗蒙古副都統。道光五年,出為哈喇沙爾辦事大臣。六年,率土爾扈特、和碩特、蒙古兵援阿克蘇,賊潛渡渾巴什河犯阿城,迎擊,殲其渠庫爾班素皮,被優敘。偕提督達凌阿援烏什,敗賊於沙坡樹窩。尋撤蒙古兵,自請留軍前。七年,和闐回眾縛賊酋乞降,往撫之。洋阿爾巴特之戰,偕哈哴阿率勁騎進擊,所向披靡。沙布都爾、阿瓦巴特連戰皆力,署葉爾羌幫辦大臣。凱旋,予雲騎尉世職。九年,授塔爾巴哈臺參贊大臣。十二年,召還京。尋擢江寧將軍,治軍有聲。十七年,卒於官,優恤,謚勤勇。

長清,鈕祜祿氏,滿洲鑲紅旗人,內大臣策楞孫,副都統特成額子也。以廕生入貲,銓授兵部主事。累遷郎中。嘉慶二十四年,出為廣西左江道。母憂去官。仍為兵部郎中。道光五年,加副都統銜,充阿克蘇辦事大臣。六年,張格爾入寇,西四城相繼陷。長清截留各城換防,又發銅廠錢局官兵,扼渾巴什河。參將王鴻儀戰歿於都齊特,賊糾眾五六千自葉爾羌來犯,屢撲渡,皆擊退。踞城百餘里,波斯圖拉、哈爾塔兩地多朵蘭回莊,附逆抗拒,分兵進剿。賊復由托什罕渡河,逼城二十里,長清令數十騎馳騁揚塵,鼓噪東來,賊疑大軍至,退走河南。乃進軍,渡河結營,賊來攻,連敗之,擒斬千餘,賊始不敢窺河北。阿克蘇城小,擴關廂,開壕築壘為外郭,民、回安堵。遣兵五百助守烏什為犄角,東四城恃以無恐。宣宗初慮長清未諳軍事,命特依順保往領其職而長清副之,猶未至,至是詔嘉長清防剿深合機宜,賜花翎,予優敘,遂寢前命。大軍進討,滿、漢兵三萬數千皆集阿克蘇,長清置局供支運輸,鑄錢增驛,規畫甚備,授鑲白旗蒙古副都統,仍留任。七年,四城復,詔:「長清於大軍未到,力捍孤城,厥功甚偉,予雲騎尉世職,擢其子富春為主事。」八年,疏言:「長齡議於阿克蘇添兵一千,柯爾坪添兵五百。柯爾坪距阿城三百里,回眾數萬,兵少無益,請歸並阿克蘇,練成勁旅,可以總治兩路所屬。乃塔爾達巴罕及阿爾通霍什皆有小路可通伊犁,請並封禁。」從之。張格爾就擒,械送至京。予優敘。

十年,喀什噶爾諸城復告警,容安率伊犁兵赴援,命至阿克蘇與長清會商進兵。疏請分兵和闐、烏什,待哈豐阿、胡超兩路兵至進剿,詔斥容安畏葸,長清並下嚴議。尋原之,降二等侍衛,仍留任。十二年,加提督銜,充葉爾羌辦事大臣,馭夷開屯,措施並稱職。十四年,授烏魯木齊都統。逾年,召回京。尋授福州將軍,加太子太保。十七年,卒,晉太子太傅,賜金治喪,謚勤毅。

達凌阿,佟佳氏,滿洲鑲黃旗人。以健銳營前鋒從永保剿湖北教匪,繼隨楊遇春戰川、陜,數有功。累擢靜寧協副將,署西安鎮總兵。三才峽匪起,率兵四百御之澇峪、八里坪,大敗其眾。追尤九餘黨至黑水峪,攻克之,又敗之傅家河;擊萬五於辛峪口,連敗之,萬五率殘卒遁,尋就擒:加總兵銜,擢巴里坤總兵,調西安鎮。

道光二年,擢烏魯木齊提督。六年,率兵四千援阿克蘇,軍次庫車,遣錫伯兵扼柯爾坪,分守庫車、烏什。九月,與賊夾渾巴什河而軍,持數日,賊分走烏什,偕巴哈布迎擊,敗之於阿拉爾,追至沙坡樹窩,破伏賊。其自托什罕渡河者,方圍協領都倫布營,遏副將郭繼昌援路。達凌阿還軍馳救,奮擊敗之,賊爭渡,死者相藉,河水為之不流。迨長齡至,河北已無賊,被優敘。七年,從大軍三戰復喀城,駐守葉爾羌,署辦事大臣,予雲騎尉世職。是年秋,聞邊警,調防烏什,張格爾就擒,回本鎮。歷塔爾巴哈臺參贊大臣、西安將軍。十年,卒,優恤,謚武壯。

哈豐阿,富察氏,滿洲鑲黃旗人。嘉慶初,以健銳營前鋒從剿襄陽教匪,轉戰川、陜,累遷前鋒侍衛。搜捕南山餘匪甚力,事平,授貴州定廣協副將。擢威寧鎮總兵,歷浙江處州,陜甘涼州、漢中諸鎮。道光八年,擢烏魯木齊提督。十年,回疆復警,命馳赴阿克蘇,偕長清防剿。十一月,進攻葉爾羌賊營,賊潰,潛伏哈拉布扎什軍臺,分道要擊,破之。進圍黑色爾,擒其酋巴拉特,乘勝至英吉沙爾,喀什噶爾圍亦解,予雲騎尉世職,賜號進勇巴圖魯。初詔哈豐阿倍道馳援葉爾羌,聽容安計,繞道和闐,失期,議奪職,原之,責償軍費十之二,仍留任。

擢廣州將軍。疏請鑄巨砲百,選精銳五百人,嚴守望以重海防。十四年,英吉利兵船二,號稱護商,入廣州海口,縱砲擊之。船停黃埔,調兵建閘,制其出入,英酋謝罪,事乃解。調黑龍江將軍,舉發御前大臣高克鼐囑託私書,詔獎其持正,授內大臣,加太子少保。請添練馬隊,增置官吏,補助布特哈生計,並允行。調西安將軍。二十年,卒,謚愨勤。

慶祥,圖博特氏,蒙古正白旗人,大學士保寧子。授藍翎侍衛。嘉慶十三年,襲三等公爵,授散秩大臣、鑲白旗蒙古副都統,兼正藍旗護軍參領。尋授理籓院侍郎,調工部。十八年,率京營兵從那彥成剿滑縣教匪,凱旋,擢正黃旗漢軍都統,歷熱河、烏魯木齊都統。二十五年,授伊犁將軍。八月,逆回張格爾擾喀什噶爾,官軍剿捕,乃引去。參贊大臣斌靜以聞,不言釁由,宣宗疑之,命慶祥往勘,得斌靜縱容家奴凌辱伯克、交通奸利狀,褫逮論罪。疏陳善後六事,又密請羈縻浩罕部落,許遣使入覲,以安夷心,詔俞之。

道光五年夏,張格爾復擾邊,內地回戶多與通。幫辦大臣巴彥巴圖率兵出塞掩之,不遇,即縱殺游牧布魯特而還。其酋汰列克追覆官軍於山谷,賊遂猖獗,褫參贊大臣永芹職,命慶祥代之。慶祥至,誤信奸回阿布都拉,反為賊耳目。六年夏,張格爾遣其黨赫爾巴什潛赴綽勒薩雅克愛曼,糾合夷眾,復令奇比勒迪至巴雅爾開渠占地,遣兵擒斬之。張格爾率眾五百由開齊山路突至回城,拜其先和卓木之墓,回人所謂「瑪雜」也。慶祥令幫辦大臣舒爾哈善及領隊大臣烏凌阿往剿,夜雷雨,張格爾潰圍走,白帽回眾紛起應之。張格爾復由大河沿合眾數萬進犯喀城,慶祥盡調各營卡兵為三營,令烏凌阿、穆克登布分率之,迎戰,先後沒於陣。先是張格爾求助於浩罕,約四城破,分所掠,且割喀城以報。及見官軍無援,悔欲背約,浩罕酋怒,自以所部攻城未下,尋引去;張格爾追擊之,收其降眾數千,遂益強。八月,圍喀城凡七十日,城陷,慶祥自經死。事聞,贈太子太保,晉封一等公,兼雲騎尉世職,以子文煇嗣,謚壯直,祀昭忠祠。逾年,回疆平,詔於喀什噶爾建昭忠祠祀之,舒爾哈善、烏凌阿、穆克登布俱從祀,禦制憫忠詩勒諸石。八年,張格爾伏誅,命其子文煇看視行刑,摘心於墓前致祭。

舒爾哈善,葛哲勒氏,滿洲鑲白旗人。以驍騎校從征川、陜教匪有功,予巴圖魯勇號。累擢布特哈烏拉協領。克滑縣,加副都統銜。坐事褫職。道光初,予三等侍衛,充庫爾喀喇烏蘇領隊大臣。六年,張格爾入犯,調喀什噶爾幫辦大臣。與賊戰,身先士卒,受槍傷,仍麾兵前進,殺數百人。城陷,被戕,予騎都尉世職。

烏凌阿,瓜爾佳氏,滿洲鑲白旗人。由前鋒從征教匪,累擢頭等侍衛。道光三年,授伊犁領隊大臣、正紅旗蒙古副都統。六年,賊逼喀城,慶祥檄令回援,遇賊於渾河,力戰至晡,沒於陣。贈都統銜,謚壯武,予騎都尉兼雲騎尉世職。

穆克登布,季氏,滿洲鑲紅旗人,伊犁駐防。由委前鋒校累擢協領。道光元年,慶祥密令誘捕張格爾於托雲山內,獲其黨蒙達拉克等,予議敘。二年,充庫爾喀喇烏蘇領隊大臣,調伊犁。五年,率兵至喀什噶爾,駐防圖舒克塔什卡倫。張格爾犯喀城,撤兵回戰於七里河,死之。贈都統銜,謚壯節,予騎都尉兼雲騎尉世職。

多隆武,烏素爾氏,滿洲鑲白旗人。由筆帖式補驍騎校,累擢協領。道光四年,加副都統銜,充葉爾羌幫辦大臣。六年,喀什噶爾被圍急,遣兵赴援。賊由阿色爾布依岳坡爾湖而南,分兵防禦。奸回阿布都拉等潛通賊,多隆武盡誅之。喀、英兩城相繼陷,賊趨葉爾羌,參將吳亨佑扼單板橋,戰歿;遂由黑子鋪入,防師盡熸,回兵半為賊脅。伊犁道梗不能救,葉城乃陷,多隆武死之。依都統例賜恤,於葉爾羌建專祠,予騎都尉兼雲騎尉世職。

葉爾羌辦事大臣印登、英吉沙爾領隊大臣蘇倫保、和闐領隊大臣奕湄、幫辦大臣桂斌同殉難,追論死事諸臣,並贈恤有差,惟喀什噶爾幫辦大臣巴彥巴圖坐濫殺陷師,奪其恤典。

壁昌,字東垣,額勒德特氏,蒙古鑲黃旗人,尚書和瑛子。由工部筆帖式銓選河南陽武知縣,改直隸棗強,擢大名知府。道光七年,從那彥成赴回疆,佐理善後。壁昌有吏才,以父久官西陲,熟諳情勢,事多倚辦。九年,擢頭等侍衛,充葉爾羌辦事大臣。壁昌至官,於奏定事宜復有變通,清出私墾地畝新糧萬九千餘石,改徵折色,撥補阿克蘇、烏什、喀喇沙爾俸餉,餘留葉城充經費,以存倉二萬石定為額貯,歲出陳易新,於是倉庫兩益。葉爾羌喀拉布札什軍臺西至英吉沙爾察木倫軍臺,中隔戈壁百數十里,相地改驛,於黑色熱巴特增建軍臺,開渠水,種苜蓿,士馬大便。所屬塔塔爾及和沙瓦特兩地新墾荒田,皆回戶承種,奏免第一年田賦,以恤窮氓。新建漢城,始與回城隔別,百貨輻輳,倍於往時。以回城官房易新城南門外曠土,葺屋設肆,商民便之。訪問疾苦,聯絡漢、回,人心益定。

十年八月,浩罕糾諸部寇邊,圍喀什噶爾、英吉沙爾兩城,遂犯葉爾羌。容安率援師遷延不至,壁昌撫諭回酋,同心守御,分扼科熱巴特、亮噶爾諸要隘。賊萬餘撲城,迎戰於東門外,擊破之,賊宵遁,詔嘉其援師未至之先即獲全勝,加副都統銜,尋授鑲黃旗漢軍副都統。自九月至十一月,賊復三次來犯,迭擊敗走之。最後賊攻城,相持五日,而哈豐阿援兵至,賊望風遁,追破之於哈拉布札什。越數日,進兵英吉沙爾,而喀什噶爾之賊已飽颺出塞,大軍至,則無賊矣。壁昌素得回眾心,是役尤得阿奇木伯克阿布都滿之助,賴以戰守。事定,奏請仍襲其祖郡王封爵。長齡、玉麟奉命會籌善後事,盡諮於璧昌。

十一年,擢參贊大臣,改駐葉爾羌,遂專回疆全局。興喀拉赫依屯田,招練民戶五百人,修渠築壩,以牌博為界,不侵回地,凡墾屯地二萬二百四十畝。十二年,和闐回民塔瓦克戕伯克多拉特、依斯瑪伊勒等為亂,捕其黨盡置諸法。疏言:「長齡等奏增南路防兵三千屯巴爾楚克,因其地築城未竣,遂以二千人分屯葉、喀二城。二城形勝較巴爾楚克尤要,請以暫時分屯之兵永為定額。喀城更增綠營兵三千五百,分屯七里河為犄角,葉城增烏魯木齊滿洲兵五百、綠營兵一千。」詔從之。十三年,召還京。十四年,復出為烏什辦事大臣。歷涼州副都統、阿克蘇辦事大臣、察哈爾都統。緣事降調,充伊犁參贊大臣。授陜西巡撫,擢福州將軍。

二十三年,署兩江總督,尋實授。英吉利和議初成,壁昌奏設福山鎮水師總兵,沿江形勢,扼險設防,請於五龍、北固兩山及圌山關、鵝鼻嘴修築砲臺砲堤,是為籌江防之始。言官請團練鄉兵,以窒礙無益,奏寢其議。淮北已改票鹽,御史劉良駒疏請推廣於淮南試行。疏言其不便,略謂:「淮南地廣引多,價昂課重,行銷之不齊,堵緝之難易,與淮北迥別。灶戶成本不能驟減至三四倍,民販更非一時可集,而課項皆常年要需。如改票議行,應納課銀孰肯再繳?應追積欠亦當豁除。此後攤帶錢糧亦將盡停,利猶未見,害已先形。為今之計,但能肅清場灶以杜偷漏之源,整飭口岸以廣行銷之路,嚴禁浮濫以除在官之蠹,顧惜成本以冀商力之紓,庶淮鹺漸有起色。」疏入,如所請。二十七年,入覲,留京授內大臣,復出為福州將軍。數月,以疾請回旗。咸豐三年,粵匪北犯,逼近畿,命為巡防大臣。四年,卒,贈太子太保,謚勤襄。子恆福,直隸總督。孫錫珍,同治七年進士,由翰林院編修歷官吏部尚書。

當壁昌初蒞葉爾羌,實繼恆敬之後。恆敬原名恆敏,伊爾根覺羅氏,滿洲正藍旗人。嘉慶初,為四川打箭爐同知。治軍需糧餉有功,擢綏定知府。累遷江寧布政使。道光初,授光祿寺卿,充哈密辦事大臣。大軍征張格爾,命督辦轉運,鑄錢購糧,增設臺站,供軍無缺。七年,調烏什辦事大臣。命赴喀什噶爾幫辦善後,授葉爾羌辦事大臣。遷建新城於罕那裏克,勘墾官荒田,歲增糧供防兵二千口食,復於西北隅跴荒地一百餘里,水土肥饒,疏請試墾。壁昌至,始墾成。八年,乞病歸。尋授正白旗漢軍副都統,出為西寧辦事大臣。十二年,卒。

論曰:平定回疆,多用川、楚、陜舊將,百戰之餘,以臨犬羊烏合,摧枯拉朽,旬月而告功成,何其易哉!及後海疆事起,授鉞分麾,莫能禦侮,蓋所當堅脆不同,而勝之不可以狃也。楊芳一時名將之冠,差知彼己,晚伍庸帥,依違召譏,其以恩禮終,猶為幸焉。慶祥心知危局,身殉孤城,壁昌力捍寇氛,卒安邊徼,回疆安危之所系也,並著於篇。


\end{pinyinscope}