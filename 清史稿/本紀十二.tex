\article{本紀十二}

\begin{pinyinscope}
高宗本紀三

二十一年春正月庚午,以額駙科爾沁親王色布騰巴勒珠爾貽誤軍機,褫爵禁錮。喀爾喀親王額琳沁多爾濟以疏縱阿睦爾撒納,處斬。己卯,以準噶爾故總臺吉達什達瓦之妻率眾來降,封為車臣默爾根哈屯。命尹繼善往浙江會審鄂樂舜。丁亥,阿巴噶斯得木齊哈丹等來降。乙未,命哈達哈由阿爾泰進兵協剿。原任副將軍薩喇勒由珠勒都斯來歸,命與鄂勒哲依同掌副將軍印。命協辦大學士達勒黨阿由珠勒都斯進兵協剿。丁酉,致仕協辦大學士阿克敦卒。

二月癸卯,授巴里坤辦事大臣和起欽差大臣關防。戊申,以楊廷璋為浙江巡撫。辛亥,上啟蹕謁孔林。以策楞奏報獲阿睦爾撒納,命改謁泰陵。甲寅,上謁泰陵。免直隸、山東經過州縣錢糧十分之三,歉收地方免十分之五。乙卯,上幸山東,詣孔林。免山東海豐等三縣潮災額賦。壬辰,賑山東蘭山等州縣水災。癸亥,賑浙江仁和等十五州縣場水災。甲子,工部尚書衛哲治病免,以趙弘恩代之。策楞以誤傳獲阿睦爾撒納奏聞。丁卯,命薩喇勒以副將軍駐特訥格爾。戊辰,授碩色為湖廣總督,郭一裕為雲南巡撫。

三月己巳朔,上至曲阜,謁先師孔子廟。授清保為盛京將軍。庚午,釋奠禮成。謁孔林、少昊陵、元聖周公廟。免曲阜丁丑年額賦。辛未,賑山東鄒縣等十七州縣衛水災。丙戌,免江蘇宿遷被災河租,湖北潛江等五州縣上年水災額賦。丁亥,命哈達哈進兵烏梁海布延圖,以青滾雜卜、車布登為參贊大臣。策楞等奏復伊犁。戊子,免安徽宿州等二十一州縣衛、江蘇阜寧等七十二州縣衛上年水災額賦。壬辰,上謁昭西陵、孝陵、景陵,詣孝賢皇后陵奠酒。丙申,賜鄂樂舜自盡。丁酉,上還京師。

夏四月壬子,免山東鄒縣等十九州縣衛上年潮災額賦。命達勒黨阿由西路、哈達哈由北路進徵哈薩克,以哈寧阿、鄂實為參贊大臣。癸丑,命大學士傅恆赴額林哈畢爾噶整飭軍務。策楞、玉保逮問。以烏勒登疏縱阿睦爾撒納處斬。甲寅,命尚書阿里袞在軍機處行走。丁巳,召傅恆回京。富德奏敗哈薩克於塞伯蘇臺。壬戌,免山西岢嵐州二十年霜災額賦。癸亥,軍機大臣雅爾哈善、劉綸罷。命裘曰修在軍機處行走。乙丑,召劉統勛回京。

五月戊辰朔,玉保降領隊大臣,以達勒黨阿為定邊右副將軍,巴祿為參贊大臣。乙亥,免浙江仁和等十三州縣上年被災額賦。庚辰,上詣黑龍潭祈雨。乙酉,以莽阿納、達什車凌為參贊大臣。丁亥,免甘肅甘州等三府本年民屯額賦。賑甘肅皋蘭等二十州縣上年霜雹災。辛丑,噶勒雜特宰桑根敦等來降。壬子,以莽阿納為歸化城都統。癸丑,何國宗降調,以趙弘恩為左都御史,調汪由敦為工部尚書,劉統勛為刑部尚書。丙辰,伯什阿噶什屬宰桑賽音伯克來降。癸亥,杜爾伯特臺吉伯什阿噶什遣使來降,命封親王。乙丑,封杜爾伯特臺吉烏巴什為貝子。

秋七月戊辰,免安徽無為等三十二州衛上年水災額賦。壬申,特楞古特宰桑敦多克及古爾班和卓等於濟爾瑪臺詐降,哈達哈等率兵殄之。授哈達哈領侍衛內大臣,車布登扎布郡王,唐喀祿、舒赫德副都統,三都布多爾濟公爵,餘議敘有差。庚辰,漕運總督瑚寶卒,以張師載代之。丁亥,上幸清河,至班第、鄂容安喪次賜奠。壬辰,以青滾雜卜叛跡已著,諭舒明、成袞扎布等捕剿之。癸巳,庫車伯克鄂對等來降。

八月壬寅,以綽爾多為黑龍江將軍。乙巳,命喀爾喀親王成袞扎布為定邊左副將軍,舒明、阿蘭泰、桑齋多爾濟、德沁扎布、塔勒瑪善為參贊大臣。辛亥,命納木扎勒、德木楚克為參贊大臣。以保德署綏遠城將軍。癸丑,上奉皇太后秋獮木蘭。磔阿巴噶斯等於市。戊午,賑車臣汗部落扎薩克輔國公成袞等六旗旱災。額魯特達瑪琳來降。庚申,上奉皇太后巡幸木蘭,行圍。授瑚圖靈阿、富昌、保德、哲庫納、阿爾賓為參贊大臣,隨成袞扎布辦事。以保雲署綏遠城將軍。壬戌,臺吉伯什阿噶什入覲,召見行殿,賜宴。癸亥,予成袞扎布等議敘。甲子,以喀爾喀貝勒品級車木楚克扎布接續臺站,封為貝勒。乙丑,哈達哈等征哈薩克,大敗之。授扎拉豐阿為貝子,明瑞為副都統。賑陜西長安等十三州縣雹災。

九月甲戌,達瓦齊近族臺吉巴里率人戶來降,命附牧扎哈沁地方。丁丑,土爾扈特臺吉敦多布達什遣使臣吹扎布入貢,上召見於行幄,賜宴。戊子,免甘肅乾隆元年至十五年積年欠賦,及寧夏安西等二十二州縣衛本年額賦有差。庚寅,上奉皇太后回駐避暑山莊。授杜爾伯特親王伯什阿噶什為盟長。乙未,暹羅國王遣使貢方物。賑山東魚臺等縣水災。

閏九月癸卯,封羅卜藏車楞之子塔木楚克扎布為貝勒。戊申,上奉皇太后回蹕。庚戌,授阿桂為北路參贊大臣。準借黑龍江被水人戶籽種口糧。甲寅,上奉皇太后還京師。賑安徽宿州等十二州縣衛水災。辛酉,免江蘇清河十二州縣衛被災漕項。

冬十月戊辰,命哈達哈以參贊大臣隨同成袞扎布辦事,阿里袞、富德回京。壬申,以富勒赫未能豫防河決,召來京。命愛必達為河道總督,劉統勛署之。調鶴年為山東巡撫,授尹繼善兩江總督,兼管河務。癸酉,以滿福為巴里坤都統。丙子,兆惠以回部霍集占叛狀聞,遣阿敏道等進兵。戊寅,輝特臺吉巴雅爾叛掠洪霍爾拜、扎哈沁,命寧夏將軍和起討之。己卯,賑直隸延慶等八州縣衛本年水災雹災。乙酉,致仕大學士福敏卒。

十一月丁未,賑甘肅皋蘭等二十六州縣水雹災。辛亥,調陳宏謀為陜西巡撫,圖勒炳阿為湖南巡撫。甲寅,命仍逮問策楞、玉保。降封扎拉豐阿公爵。以達勒黨阿為定西將軍,兆惠為定邊右副將軍,永貴為參贊大臣。庚申,哈薩克錫喇巴瑪及回人莽噶里克率眾襲將軍和起於闢展。和起力戰死之,命如傅清、拉布敦例恤。己未,黃廷桂奏備馬三萬匹,增兵駐哈密等處。上以「明決擔當」嘉之。賞黃廷桂雙眼花翎、騎都尉世職。壬戌,王安國病免。以汪由敦署吏部尚書,趙弘恩署工部尚書,何國宗署左都御史。

十二月甲子朔,策楞、玉保逮京,途次為額魯特人所害。庚午,賑山西汾陽等縣水災。辛未,諭哲布尊丹巴胡圖克圖加號敷教安眾喇嘛。壬申,以盧焯為湖北巡撫。賑山東金鄉等二十一州縣衛水災。甲戌,免陜西盩厔等四縣本年水災民屯額賦、馬廠地額賦之半。戊寅,獲青滾雜卜於杭噶獎噶斯,賞成袞扎布黃帶,封子一人為世子,封納木扎勒一等伯。己卯,召瑚圖靈阿等回京。以獲青滾雜卜功,晉貝勒車木楚克扎布郡王品級,賞貝勒旺布多爾濟等雙眼花翎。丙戌,達勒黨阿罷協辦大學士,以鄂彌達代之。

二十二年春正月甲午,以南巡免江蘇、安徽、浙江累年逋賦。以成袞扎布為定邊將軍,由巴里坤進剿,車布登扎布署北路定邊左副將軍,舒赫德、富德、鄂實為參贊大臣,色布騰巴勒珠爾、阿里袞、明瑞等為領隊大臣。乙未,賑江蘇清河等十九州縣水災。戊戌,命嵩椿為荊州將軍。以莽古賚為參贊大臣,赴北路軍營。己亥,命哈達哈為參贊大臣,駐科布多。庚子,以哈寧阿、永貴為參贊大臣。癸卯,上奉皇太后南巡。甲辰,授汪由敦吏部尚書,調何國宗為禮部尚書,秦蕙田為工部尚書,趙弘恩仍回左副都御史,白鍾山為江南河道總督,張師載為河東河道總督,楊錫紱為漕運總督,授愛必達江蘇巡撫。丙午,免直隸靜海等三州縣逋賦。丁未,免經過直隸、山東地方本年錢糧十分之三,被災地方十分之五。壬子,賑山東濟寧五州縣衛水災。癸丑,以阿思哈為北路參贊大臣。己未,以嵇璜為江南副總河。命阿桂留烏里雅蘇臺辦事。壬戌,噶勒藏多爾濟、達什車凌等叛。

二月癸亥朔,免經過江南、浙江地方本年錢糧十分之三,被災地方十分之五。甲子,賑江蘇清河十四州縣衛、安徽宿州等四州縣衛災民。丙寅,兆惠全師至烏魯木齊,封一等伯,世襲。丁卯,上奉皇太后渡河至天妃閘,閱木龍。免江南乾隆十年以前漕項積欠。免兩淮灶戶乾隆十七年至十九年未完折價銀兩。乙亥,上奉皇太后渡江。癸未,幸宋臣範仲淹高義園。甲申,上奉皇太后幸蘇州府。乙酉,上奉皇太后臨視織造機房。調富森為吏部尚書,以納木札勒為工部尚書。降阿里袞為侍郎,以兆惠為戶部尚書、領侍衛內大臣,舒赫德為兵部尚書。命成袞扎布、兆惠分路捕剿額魯特叛眾。丙戌,上閱兵於嘉興府後教場。丁亥,上閱兵於石門鎮。己丑,上奉皇太后幸杭州府。庚寅,上閱兵。辛卯,免山東齊河等三州縣民欠,及山西汾陽等二縣、江蘇清河等十二州縣水災額賦。

三月丁酉,噶勒藏多爾濟陷伊犁,命成袞扎布討之。庚子,上奉皇太后駐蹕蘇州府。己酉,上奉皇太后幸江寧府。免江南之江寧、蘇州,浙江之杭州三府附郭諸縣本年額賦。庚戌,上奠明太祖陵。輝特臺吉車布登多爾濟叛,哈達哈討獲之。命盡誅丁壯,以女口賞喀爾喀。辛亥,以哈達哈為兵部尚書。癸丑,上奉皇太后渡江。甲寅,召原任大學士史貽直入閣辦事,黃廷桂仍以大學士兼管陜甘總督。丙辰,免陜西潼關等州縣上年水雹災額賦。召劉統勛赴行在。己未,上奉皇太后渡河。

夏四月壬戌朔,直隸總督方觀承劾奏巡檢張若瀛擅責內監僧人。上斥為不識大體,仍諭內監在外生事者聽人責懲。乙丑,免江蘇淮安等三府州地畝額賦。命劉統勛督修徐州石工,侍郎夢麟督修六塘以下河工,副總河嵇璜督修昭關滾壩支河,均會同督、撫、總河籌辦。召成袞扎布、兆惠、舒赫德等來京,以雅爾哈善為參贊大臣,掌定邊右副將軍印,命阿里袞駐巴里坤辦事。丙寅,上至孫家集閱堤工。唐喀祿獲車布登多爾濟,以普爾普部人賞烏梁海。丁卯,上渡河,至荊山橋、韓莊閘閱河工。戊辰,免直隸延慶等州縣衛二十一年雹災水災額賦。庚午,減山東海豐縣屬黎敬等五莊糧額,並免十一年至二十年逋賦。以松阿里為綏遠城將軍。獲普爾普。辛未,上至闕里釋奠先師孔子。上奉皇太后駐蹕靈巖。命史貽直仍以文淵閣大學士兼吏部尚書。乙亥,改松阿里為涼州將軍,以保德為綏遠城將軍。戊寅,免山東濟寧等五州縣逋賦。己卯,調蔣炳為河南巡撫,以阿思哈為湖南巡撫。庚辰,免河南夏邑等四縣逋賦。辛巳,以夏邑生員段昌緒藏吳三桂偽檄,命方觀承赴河南會同圖勒炳阿嚴鞫之。乙酉,何國宗罷。丁亥,上還京師。命秦蕙田署禮部尚書。戊子,以前布政使彭家屏藏明末野史,褫職逮問。以歸宣光為禮部尚書。庚寅,福建廈門火。丁酉,上詣藍靛廠迎皇太后居申昜春園。乙巳,賜蔡以臺等二百四十二人進士及第出身有差。丁未,霍集占叛,副都統阿敏道死之。

六月辛酉朔,以胡寶瑔為河南巡撫,阿思哈署江西巡撫。壬戌,免甘肅及河南夏邑等四縣明年額賦。癸亥,以愛必達為雲貴總督,調陳宏謀為江蘇巡撫,明德為陜西巡撫,定長為山西巡撫。甲子,賑河南鄢陵等州縣水災。戊辰,彭家屏論斬。丁丑,賞達什達瓦部落兩月口糧。癸未,喀爾喀達瑪琳叛,命桑寨多爾濟討之。己丑,賑安徽宿州等十六州縣衛水災、甘肅碾伯等三十八州縣水雹災。

秋七月辛卯朔,賑山東館陶等州縣水災。壬辰,以劉藻為雲南巡撫。癸卯,賜彭家屏自盡。命史貽直仍兼工部。乙巳,賑安徽宿州等十州縣水災雹災。丙午,賑山東東平州等五州縣水災。以獲巴雅爾授富德內大臣,封貝勒羅布藏多爾濟為郡王。丁未,以楊應琚為閩浙總督,以鶴年為兩廣總督,蔣洲為山東巡撫,塔永寧為山西巡撫。哈薩克汗阿布賚遣使入貢。戊申,上奉皇太后巡幸木蘭。癸丑,額魯特臺吉渾齊等殺札那噶爾布,以其首來降。戊午,賑山東濟寧等三十二州縣衛水災、福建龍嚴等二州縣水災。

八月丙寅,哈薩克霍集伯爾根等降。丁卯,以薩喇善為吉林將軍,傅森署之。戊辰,賑甘肅柳溝等三衛旱災。乙亥,上奉皇太后巡幸木蘭,行圍。賑山西汾陽水災。辛巳,巴雅爾、達什車凌伏誅。

九月癸巳,克埒特、烏嚕特各部俱平。甲午,上御行殿,哈薩克阿布賚等使臣入覲,賜宴。戊戌,以富勒渾為湖南巡撫。琿齊等復叛。庚子,額魯特沙喇斯、瑪呼斯二宰桑叛,命都統滿福討之。以雅爾哈善為兵部尚書。辛丑,上奉皇太后回駐避暑山莊。壬寅,磔尼瑪等於故將軍和起墓前。丁未,命劉統勛赴山東、江南辦理河工。辛亥,上奉皇太后還京師。

冬十月壬戌,上幸南苑,行圍。癸亥,琉球入貢。乙丑,以雅爾哈善署定邊右副將軍。丁卯,召車布登扎布來京,以納木扎勒署定邊左副將軍。阿桂赴科布多,以莽古賚為北路參贊大臣。辛未,以兆惠為定邊將軍,車布登扎布為定邊右副將軍。丙戌,以永貴為陜西巡撫。

十一月丙申,以喀爾喀親王德沁扎布為北路參贊大臣。壬子,以吳拜為左都御史。戊午,賑甘肅皋蘭等二十二州縣霜雹等災。

十二月癸亥,以陳宏謀為兩廣總督,李侍堯署之,託恩多為江蘇巡撫,阿爾泰為山東巡撫。己巳,大學士陳世倌乞休,許之。乙亥,封車木楚克扎布為郡王。丁丑,賑扎嚕特、阿嚕、科爾沁三旗災。庚辰,舒赫德以失機褫職。甲申,加史貽直、陳世倌太子太傅,鄂彌達、劉統勛太子太保。

是歲,朝鮮、暹羅、琉球入貢。

二十三年春正月己丑,賑河南衛輝等府屬災民一月。免甘肅乾隆十六年至二十二年逋賦。庚寅,命兆惠、車布登扎布剿沙喇伯勒,雅爾哈善、額敏和卓徵回部。辛卯,賑江蘇清河等十八州縣、安徽宿州等十州縣災民有差。癸酉,賑直隸大名等州縣災民。丙午,以俄羅斯呈驗阿睦爾撒納尸及哈薩克稱臣納貢,宣諭中外。己酉,吏部尚書汪由敦卒,上親臨賜奠。壬子,以劉統勛為吏部尚書,調秦蕙田為刑部尚書,以嵇璜為工部尚書,調鍾音為廣東巡撫,周琬為福建巡撫,周人驥署貴州巡撫。癸丑,命雅爾哈善為靖逆將軍,額敏和卓、哈寧阿為參贊大臣,順德訥、愛隆阿、玉素布為領隊大臣,徵回部。命永貴、定長以欽差大臣關防辦理屯田事務。

二月庚申,朝鮮入貢。癸亥,賑陜西葭州等八州縣旱災。乙丑,賑德州等三十七州縣衛所災民。

三月庚寅,上謁西陵。癸巳,上謁昭西陵、孝陵、孝東陵、景陵。庚子,上謁泰陵。辛丑,兆惠等進兵沙喇伯勒,獲扎哈沁哈拉拜,盡殲其眾。舍楞遁,命和碩齊、唐喀祿追捕之。壬寅,免江蘇山陽等二十五州縣衛額賦有差。乙巳,御試翰林、詹事等官,擢王鳴盛等三員為一等,餘升黜有差。試由部院改入翰林等官,擢德爾泰為一等,餘升黜有差。丁未,以吳士功為福建巡撫,鍾音為陜西巡撫,托恩多為廣東巡撫,莊有恭署江蘇巡撫,馮鈐為湖北巡撫。

夏四月壬戌,免甘肅蘭州等六府州縣乾隆三年至十年逋賦。戊辰,復封額駙色布騰巴勒珠爾為親王。免直隸霸州等三十三州縣乾隆十年至二十年逋賦。庚午,致仕大學士陳世倌卒。壬申,命李元亮兼署戶部尚書。免直隸魏縣等二十九州縣上年水災額賦。丙子,命陳宏謀回江蘇,以總督銜管巡撫事。以馮鈐為湖南巡撫,莊有恭署湖北巡撫,李侍堯署兩廣總督。庚辰,上詣黑龍潭祈雨。壬午,以旱命刑部清理庶獄,減徒以下罪,直隸如之。

五月戊子,免甘肅通省二十四年額賦。癸丑,賑陜西延安等三府州旱災。

六月辛未,免陜西榆林等八州縣逋賦。癸未,免陜西靖邊等八州縣上年額賦。直隸元城等州縣蝗。

秋七月丁亥,免甘肅安西等三衛二十二年風災額賦。己丑,毛城鋪河決。庚寅,霍集占援庫車,雅爾哈善等擊敗之。免福建臺灣縣旱災額賦。丙申,加黃廷桂少保,楊應琚、開泰太子太保,楊錫紱太子少師,陳宏謀、高晉、胡寶瑔太子少傅,白鍾山、愛必達、吳達善太子少保。戊戌,賑山西靜樂等州縣水雹災。庚子,上奉皇太后秋獮木蘭。壬寅,舍楞奔俄羅斯。召阿桂還。癸卯,右翼布魯特瑪木特呼裡比米隆遣其弟舍爾伯克入覲。諭縛獻哈薩克錫喇。乙巳,以納木札勒為靖逆將軍,三泰為參贊大臣。諭兆惠赴庫車,丙午,上奉皇太后駐避暑山莊。戊申,賞車布登扎布親王品級。壬子,賑陜西延安等十七州縣旱雹災。

八月丙寅,雨。己巳,上奉皇太后幸木蘭行圍。甲戌,以都賚為兵部尚書。丁丑,賑甘肅皋蘭等二十四州縣旱災。壬午,緬甸國王莽達喇為得楞野夷所害,木梳鋪土官甕藉牙自立。

九月己丑,賜布魯特使臣舍爾伯克宴。提督馬得勝以攻庫車失機,處斬。庚寅,右部哈薩克圖裏拜及塔什乾回人圖爾占等來降。丙申,奉皇太后駐避暑山莊。戊戌,調歸宣光為左都御史,以嵇璜為禮部尚書,命梁詩正署工部尚書。命駐防伊犁大臣兼理回部事務。己亥,賑浙江仁和等縣水災。甲辰,哈喇哈勒巴克回部來降。庚戌,和闐城伯克霍集斯等來降。壬子,烏什城降。

冬十月癸亥,賑浙江錢塘等十六縣場水災,山西朔平府屬霜災。丁卯,賑直隸大城等九縣水雹霜災。兆惠自巴爾楚克進兵葉爾羌。甲戌,吳拜病免,以德敏為左都御史。賑直隸滄州等六州縣場水災。

十一月甲申朔,右部哈薩克遣使來朝,賜宴。乙酉,上回蹕。丙戌,上幸南苑行圍。戊子,上大閱。己丑,以阿里袞為參贊大臣,赴兆惠軍營。辛卯,賑江蘇海州等五州縣水旱潮災。丁酉,兆惠至葉爾羌城外,陷賊圍中。授富德為定邊右副將軍,阿里袞、愛隆阿、福祿、舒赫德為參贊大臣,往葉爾羌策應。己亥,以十二月朔望日月並蝕,諭修省。辛丑,克里雅伯克阿裏木沙來降。甲辰,以兆惠深入鏖戰,封一等武毅謀勇公,晉額敏和卓郡王品級,霍集斯貝子加貝勒品級。丁未,納木扎勒、三泰、奎瑪岱策應兆惠,途次遇賊,死之。加贈納木扎勒公爵、三泰子爵、奎瑪岱世職。以舒赫德為工部尚書。庚戌,富德赴葉爾羌。

十二月癸丑朔,日蝕。左副都御史孫灝奏請明年停止巡幸,上斥其識見舛繆,改用三品京堂,並以「效法皇祖練武習勞」諭中外。賑福建臺灣等四縣風災。加賑浙江仁和等七縣所水災。壬戌,裘曰修罷軍機處行走。丁卯,除甘肅張掖等四縣水沖田畝額賦。戊辰,晉封喀爾喀扎薩克郡王齊巴克雅喇木丕勒為親王。壬申,免浙江錢塘等七縣本年水災額賦。

二十四年春正月甲申,免甘肅通省明年額賦及積年各項積欠。癸巳,雅爾哈善處斬。己亥,大學士伯黃廷桂卒,以吳達善為陜甘總督,明德為甘肅巡撫,暫護總督。授李侍堯兩廣總督。癸卯,命蔣溥為大學士,仍管戶部尚書,梁詩正為兵部尚書,歸宣光為工部尚書,陳德華為左都御史,李元亮兼管兵部滿尚書,蘇昌署滿工部尚書。

二月壬戌,哈寧阿論斬。癸亥,賑車都布等三旗旱災。甲子,富德、阿里袞與霍集占戰呼爾璊,大敗之。封富德為三等伯,予舒赫德、阿里袞、豆斌等世職。命舒赫德回阿克蘇辦事。己巳,富德兵至葉爾羌,會兆惠兵進攻。晉封富德一等伯。命車布登扎布為副將軍,福祿、車木楚克扎布為參贊大臣。鄂斯滿等陷克里雅。諭巴祿援和闐。庚辰,以兆惠、富德回阿克蘇,嚴責之。

三月癸未,命舒赫德同霍集斯駐和闐,截賊竄路。己丑,以頭等侍衛烏勒登、副都統齊努渾為北路參贊大臣。壬辰,召楊應琚來京,以楊廷璋署閩浙總督。甲午,彗星見。己亥,明瑞晉封承恩毅勇公。江蘇淮安等三府州蝗。

夏四月辛亥,富德等援和闐。癸丑,以阿桂為富德軍營參贊大臣。丁巳,常雩,祀天於圜丘。上以農田望澤,命停止鹵簿,步行虔禱。以楊應琚為陜甘總督,吳達善以總督銜管巡撫事。戊午,以楊廷璋為閩浙總督,莊有恭為浙江巡撫。庚申,免浙江錢塘等十六縣場上年風災額賦。辛酉,展賑甘肅河州等處旱災。命刑部清獄減刑,甘肅亦如之。甲子,賑甘肅狄道等二十三州縣衛旱災雹災。丁卯,上臨原任大學士黃廷桂喪。癸酉,免山西陽曲等五州縣上年水災雹災額賦。丁丑,禁織造貢精巧絺繡。命舒赫德仍回駐阿克蘇。

五月辛巳,免陜西潼關等六十五州縣本年額賦有差。辛卯,上詣黑龍潭祈雨。丁酉,賑陜西咸寧等州縣旱災。己亥,詔諸臣修省,仍直言得失。辛丑,上素服詣社稷壇祈雨。丁未,上以雨澤未沛,不乘輦,不設鹵簿,由景運門步行祭方澤。己酉,賑甘肅皋蘭等州縣被旱災民。

六月庚戌,緩常犯奏請處決。甲寅,以恆祿為綏遠城將軍。戊午,賑陜西榆林等十一州縣旱災。庚申,上以久旱,步至圜丘行大雩禮。是日,大雨。命兆惠進兵喀什噶爾,富德進兵葉爾羌。甲戌,江蘇海州等州縣、山東蘭山等縣蝗,諭裘曰修、海明捕蝗。丙子,英吉利商船赴寧波貿易,莊有恭奏卻之。諭李侍堯傳集外商,示以禁約。

閏六月丙戌,免福建臺灣等三縣上年風災額賦。丁酉,賑甘肅皋蘭等州縣旱災。庚子,布拉呢敦棄喀什噶爾遁。甲辰,霍集占棄葉爾羌遁。丙午,以劉綸為左都御史。戊申,以甘肅旱,停發本年巴里坤等處遣犯。

秋七月己酉朔,兆惠等奏喀什噶爾、葉爾羌回眾迎降。布拉呢敦、霍集占遁巴達克山。命阿里袞等率兵攻巴爾楚克。庚戌,諭兆惠等追捕布拉呢敦、霍集占。命車布登扎布駐伊犁,防霍集占等入俄羅斯。辛亥,以捕蝗不力,奪陳宏謀總督銜。壬子,上奉皇太后啟蹕,秋獮木蘭。己未,上奉皇太后駐蹕避暑山莊。停徵山西陽曲等三十九州縣旱災額賦。丁丑,改西安總督為川陜總督,四川總督為四川巡撫,甘肅巡撫為甘肅總督管巡撫事。以開泰為川陜總督,楊應琚為甘肅總督。山西平定等州縣蝗。

八月己卯,明瑞追剿霍集占等於霍斯庫魯克嶺,大敗之。壬午,賑甘肅皋蘭等四十州縣本年旱災。己丑,申禁英吉利商船逗遛寧波。壬辰,富德等奏追剿霍集占於阿勒楚爾,大敗之。癸巳,上奉皇太后幸木蘭,行圍。庚子,富德奏兵至葉什勒庫勒諾爾,霍集占竄巴達克山。

九月庚戌,賑浙江江山等縣水災。論剿賊功,晉封回人鄂對為貝子,阿什默特、哈岱默特為公,復敏珠爾多爾濟公爵。癸丑,定西域祀典。命阿桂赴阿克蘇辦事。晉封玉素布為貝勒。丙寅,改甘肅安西鎮為安西府。上奉皇太后還京師。以蘇昌為湖廣總督。除回城霍集占等苛斂。

冬十月己卯,頒給阿桂欽差大臣關防。癸未,賑山西陽曲等五十六州縣旱災。丁亥,賜哈寧阿自盡。戊子,禁州縣捕蝗派累民間。癸巳,免山西助馬口莊頭本年旱災額賦十分之七。乙未,以鄂弼為山西巡撫。賑盛京開原等城、承德等七州縣旱災,撫恤長蘆滄州等六州縣、嚴鎮等五場被水灶戶,均蠲額賦有差。免甘肅狄道等二十二州縣上年水災雹災額賦。丙申,賑順天直隸固安等四十七州縣水霜雹蟲災,並蠲額賦有差。丁酉,諭:「國家承平百年,休養滋息,生齒漸繁。今幸邊陲式廓萬有餘里,以新闢之土疆,佐中原之耕鑿,又化兇頑之敗類為務本之良民,一舉而數善備。各督撫其通飭所屬,安插巴里坤各城人犯,分別懲治,勿以縱釋有罪為仁,使良法不行。」己亥,賑江蘇上元等十九州縣衛水蟲風潮災。庚子,富德奏巴達克山素勒坦沙獻霍集占首級,全部投誠。命宣諭中外。將軍兆惠加賞宗室公品級鞍轡。將軍富德晉封侯爵,並賞戴雙眼花翎。參贊大臣公明瑞、公阿里袞賞戴雙眼花翎。舒赫德以下,均從優議敘。晉封額敏和卓為郡王,賞玉素布郡王品級。辛丑,以平定準、回兩部用兵本末,制開惑論,宣示中外。賑浙江嘉興等二十州縣衛所、雙穗等九場水災蟲災。壬寅,卻諸王大臣請上尊號。賑陜西定邊等九縣旱雹霜災。癸卯,召喀爾喀、杜爾伯特諸部落汗、王、公等赴太平嘉宴。

十一月辛亥,以平定回部,上率諸王大臣詣皇太后壽康宮慶賀。御太和殿受朝賀。頒詔中外,覃恩有差。辛西,楊應琚加太子太師。乙丑,除山東濟寧州、魚臺縣水淹地賦。癸酉,命各回城伯克等輪班入覲。哈爾塔金布魯特來降。

十二月甲子,賑甘肅皋蘭等十四州縣及東樂縣丞屬本年旱災。癸巳,免兩淮丁溪等七場被災應納折價十分之七。甲午,賑山東海豐等十六州縣衛、永阜等三場本年水災潮災。丁酉,免浙江江山等三縣本年水災額賦。

二十五年春正月戊申,以西師凱旋,再免來歲甘肅額賦。己酉,賑甘肅皋蘭等州縣旱災。庚戌,命烏魯木齊屯田。乙卯,霍罕額爾德尼伯克遣使陀克塔瑪特等入覲。丙辰,巴達克山素勒坦沙遣使額穆爾伯克等及齊哩克、博羅爾使入覲。定邊將軍兆惠等以霍集占首級來上,並俘酋捫多索丕等至京。丁巳,上御午門行獻俘禮。命霍集占首級懸示通衢,宥捫多索丕等罪。己未,布魯特阿濟比遣使錫喇噶斯等入覲。

二月丁丑,命侍郎裘曰修、伊祿順清查甘肅各州縣辦理軍需。賑扎薩克圖汗等四旗部落饑。癸未,上啟蹕詣東陵。乙酉,賑山西陽曲等州縣上年旱災。丙戌,上謁昭西陵、孝陵、孝東陵、景陵。丁亥,以清馥遷延諱匿,命正法。辛卯,免盛京等十九驛旱災額賦,並賑之。癸巳,上還京師。丙申,命車布登扎布以副將軍統兵剿捕哈薩克巴魯克巴圖魯,以瑪巘、車木楚克扎布為參贊大臣。上詣泰陵。己亥,上謁泰陵。以兆惠、富德為御前大臣。壬寅,兆惠等凱旋,上至良鄉郊勞。癸卯,上還京師。甲辰,賜哈密扎薩克郡王品級、貝勒玉素布等冠服有差。

三月丙午朔,上御太和殿受凱旋朝賀。丁未,試辦伊犁海努克等處屯田。設烏魯木齊至羅克倫屯田村莊。免安徽懷寧等十七州縣衛上年水蟲災額賦。壬子,以阿布都拉為烏什阿奇木伯克,阿什默特為和闐阿奇木伯克,噶岱默特為喀什噶爾阿奇木伯克,鄂對為葉爾羌阿奇木伯克。甲寅,頒阿桂關防,駐伊犁辦事,常亮等協同辦事。丁巳,免浙江仁和等十州縣衛所、雙穗等九場上年水災蟲災額賦。辛酉,賑江蘇上元等五十五州縣衛上年水災。甲子,上臨和碩和婉公主喪次,賜奠。丙寅,上幸皇六子永瑢第。戊辰,命新柱往葉爾羌辦事。己巳,晉封純貴妃為皇貴妃。以巴圖濟爾噶勒為內大臣。庚午,免山東海豐等十六州縣、永阜等三場上年潮水災額賦。

夏四月戊子,以山東蘭山等縣蝻生,命直隸豫防之。己亥,內大臣薩喇勒卒。

五月甲辰朔,日食,詔修省。丙午,諭陜甘總督轄境止烏魯木齊,飭楊應琚仍回內地。壬子,詔曰:「內地民人往蒙古四十八部種植,設禁之,是厲民。今烏魯木齊各處屯政方興,客民前往,各成聚落,汙萊闢而就食多,大裨國家牧民本圖。無識者又疑勞民。特為宣諭。」癸丑,賜畢沅等一百六十四人進士及第出身有差。丁巳,免安徽懷寧等十七州縣衛上年水災蟲傷額賦。乙丑,裁陜西榆葭道,改延綏道為延榆綏道,移駐榆林府,以鄜州隸督糧道。己巳,哈薩克阿布勒巴木比特遣使入覲,賜敕書,卻所請游牧伊犁,及居住巴爾魯克等地。前掠烏梁海之巴魯克巴圖魯服罪,獻還所獲,仍錫賚之。

六月乙亥,免甘肅徵本年及來年耗羨。丁酉,召阿里袞回京。命海明赴喀什噶爾辦事。

秋七月癸卯朔,諭熱河捕蝗。甲辰,山西寧遠等、直隸廣昌等州縣蝗。甲寅,伯什克勒木等莊回人邁喇木呢雅斯叛,阿里袞剿平之。以阿思哈為江西巡撫。乙卯,賑江蘇高郵等州縣水災。戊辰,以楊寧為喀什噶爾提督。己巳,以俄羅斯駐兵和寧嶺、喀屯河、額爾齊斯、阿勒坦諾爾四路,聲言分界,諭阿桂、車布登扎布等來歲以兵逐之。

八月丙戌,命烏魯木齊駐劄大臣安泰、定長、永德為總辦,列名奏事。其大臣侍衛等,均如領隊大臣例,專任一事,咨安泰等轉奏。己丑,上奉皇太后秋獮木蘭。壬辰,以阿桂總理伊犁事務,授為都統。丙申,上奉皇太后駐避暑山莊。戊戌,上奉皇太后幸木蘭,行圍。己亥,增設江蘇江寧布政使,駐江寧府,分轄江、淮、揚、徐、通、海六府州。以蘇州布政使分轄蘇、松、常、鎮、太五府州,安徽布政使回駐安慶。命託庸調補江寧布政使。命戶部侍郎於敏中在軍機處行走。

九月乙卯,喀爾喀車臣汗札薩克旺沁扎布,以不能約束屬人,革札薩克,降貝子為鎮國公。丙辰,恆祿引見,以舒明署綏遠城將軍。丁巳,三姓副都統巴岱以挖葠人眾滋事,不能捕治,反給牌票,上以畏懦責之,命正法。庚申,命德爾格駐闢展辦事。癸亥,哈薩克汗阿布賚使都勒特克埒入覲。

冬十月壬申朔,上奉皇太后回駐避暑山莊。乙亥,以蘇州布政使蘇崇阿刑求書吏,妄奏侵蝕七十餘萬,劉統勛等鞫治皆虛,革發伊犁。戊寅,以恆祿為吉林將軍,如松為綏遠城將軍。乙酉,賑安徽宿州等十三州縣衛本年水災。辛卯,上奉皇太后還京師。以阿里袞為領侍衛內大臣。癸巳,免直隸宣化等七州縣本年水雹災額賦。己亥,賑湖南常寧等十二州縣衛旱災。

十一月癸卯,免江蘇山陽等二十五州縣衛本年水災額賦有差。丁未,除山東永利等二場並海豐縣潮沖灶地額賦。庚申,賑甘肅洮州等二十七州縣衛本年水災。丙寅,以常鈞署江西巡撫。庚午,允墾肅州鄰邊荒地,開渠溉田。

十二月丙戌,西安將軍松阿哩以受屬員餽遺,褫職論絞。命甘肅總督仍改為陜甘總督。以伊犁、葉爾羌等處均駐大臣,無須更置道員,歸總督轄。停四川總督兼管陜西。調胡寶瑔為江西巡撫,吳達善為河南巡撫,以明德為甘肅巡撫。丁亥,大學士蔣溥以病乞休,溫諭慰留。壬辰,上幸瀛臺,賜入覲葉爾羌諸城伯克薩裡等食,至重華宮賜茶果。壬辰,阿思哈論絞。丙申,德敏遷荊州將軍。以永貴為左都御史,命赴喀什噶爾辦事,代舒赫德回京。

是年,朝鮮、南掌入貢。

二十六年春正月壬寅,紫光閣落成,賜畫像功臣並文武大臣,蒙古王公等宴。賑湖南零陵等七州縣、江蘇清河等六州縣水災。丙午,以愛必達、劉藻兩年所出屬員考語相同,下部嚴議。浙江提督馬龍圖以挪用公項,解任鞫治。甲寅,尹繼善陛見,高晉護兩江總督。調海明赴阿克蘇辦事。命舒赫德赴喀什噶爾辦事,永貴赴葉爾羌辦事。癸亥,以傅森署左都御史。癸酉,上臨大學士蔣溥第視疾。鄂寶以回護陸川縣縱賊一案,下部嚴議。以託庸為廣西巡撫,永泰署湖南巡撫。庚辰,上奉皇太后西巡五臺。壬午,免所過州縣額賦十分之三。甲申,上奉皇太后謁泰陵。乙酉,安南國王黎維禕卒,封其侄黎維為安南國王。丁亥,免直隸宣化、萬全等八州縣乾隆八年至十八年逋賦。癸巳,上奉皇太后駐臺麓寺。己亥,免山東濟寧等三州縣上年水災額賦。貸甘肅淵泉等三縣農民豌豆籽種,令試種。

三月庚子,希布察克布魯特額穆爾比自安集延來歸,遣使入覲。乙巳,上幸正定府閱兵。戊申,江南河道總督白鍾山卒,以高晉代之。調託庸為安徽巡撫,以熊學鵬為廣西巡撫。己酉,設喀什噶爾駐劄辦事大臣,命伊勒圖協同永貴辦事。庚戌,賑安徽宿州等十三州縣衛水災。壬子,上幸平陽澱行圍。乙卯,免直隸宣化等二縣上年雹災額賦。丁卯,授阿桂內大臣。改綏遠城建威將軍曰綏遠城將軍。己巳,南掌國王蘇嗎喇薩提拉準第駕公滿遣使表賀皇太后聖壽、皇上萬壽,並貢方物。

夏四月庚午,上臨莊親王第、大學士蔣溥第視疾。辛未,莊有恭奏劾參將安廷召,不以保舉在前,姑容於後,諭嘉之。己卯,大學士蔣溥卒。命旌額理、阿思哈赴烏魯木齊辦事,達桑阿赴阿克蘇辦事,代安泰、定長、納世通回京。戊子,免湖南常寧等十二州縣上年旱災額賦有差。庚寅,上閱健銳營兵。壬辰,以李侍堯為戶部尚書,調蘇昌為兩廣總督,愛必達為湖廣總督。以吳達善為雲貴總督,常鈞為河南巡撫。癸巳,命劉藻暫署云貴總督。甲午,賜王傑等二百一十七人進士及第出身有差。

五月丁未,以劉統勛為東閣大學士,兼管禮部事,梁詩正為吏部尚書、協辦大學士,劉綸為兵部尚書,金德瑛為左都御史。戊午,以定長為福建巡撫,楊廷璋兼署之。

六月癸未,賑雲南新興等二州縣地震災。壬辰,免江蘇句容等十八州縣衛坍地額賦。

秋七月辛丑,協辦大學士鄂彌達卒,命兆惠協辦大學士。調舒赫德為刑部尚書,兆惠署。以阿桂為工部尚書,阿里袞署。癸丑,上啟蹕,秋獮大蘭。命諴親王允祕扈皇太后駕。壬戌,上駐避暑山莊。以皇太后巡幸木蘭,直隸沿途地方文武玩忽規避,飭下部嚴議。丙寅,河南祥符等州縣河溢。

八月丁丑,賑湖北漢川等十三州縣衛水災。戊寅,以湯聘為湖北巡撫,胡寶瑔為河南巡撫,常鈞為江西巡撫。庚辰,命高晉赴河南協辦河工。辛卯,上奉皇太后幸木蘭。壬辰,察噶爾、薩爾巴噶什兩部伯克之兄子孟克及雅木古爾齊入覲。

九月丁酉,停今年勾決。辛丑,命明瑞赴伊犁辦事,代阿桂回京。癸卯,山東曹縣二十堡黃河及運河各漫口均合龍。丙午,賑湖南武陵等州縣水災。戊申,河南懷慶府丹、沁二河溢入城,沖沒人口千三百有奇,賑被災人民。壬子,賑湖北沔陽等十一州縣衛水災。乙卯,以竇光鼐於會讞大典,紛呶謾詈,下部嚴議。己未,命素誠赴烏什辦事,代永慶回京。以札拉豐阿為烏里雅蘇臺參贊大臣,雅郎阿赴科布多辦事,代札隆阿、福祿回京。庚申,命傅景赴西藏辦事,代集福回京。乙丑,賑山東濟河等四十五州縣水災,河南祥符等五十四州縣本年水災。

冬十月戊辰,除甘肅皋蘭等三十二州縣水沖田畝額賦,並免山丹等五縣水沖撥運糧米。辛未,上奉皇太后還京師。壬辰,召裘曰修回京。賑江蘇銅山等縣水災。周人驥奏仁懷等處試織繭紬,各屬仿行,上嘉之。

十一月乙未朔,賑順直固安等六十九州縣本年水災。丁酉,以英廉為總管內務府大臣。己亥,河南楊橋漫口合龍。辛丑,調嵩椿為察哈爾都統,以舒明為綏遠城將軍。癸卯,免山西陽曲等三十八州縣、大同管糧等十四二十四年水災隨徵耗銀。丁未,免河南祥符等四十三州縣漕糧漕項有差。辛亥,減江蘇山陽等二十一州縣衛水沈地畝,並除民屯、學田、湖蕩、草灘額賦。癸丑,禮部尚書五齡安以讀表錯誤,褫職。甲寅,上奉皇太后御慈寧宮,加上徽號曰崇慶慈宣康惠敦和裕壽純禧恭懿皇太后,翌日頒詔覃恩有差。以永貴為禮部尚書,阿里袞署之。丙辰,上奉皇太后御慈寧宮,率王大臣行慶賀禮。進製聖母七旬萬壽連珠,奉皇太后懿旨,停止進獻。以勒爾森為左都御史。

十二月丁卯,以雲南江川等二州縣地震成災,命加倍賑之,仍免本年額賦。辛未,免江蘇南匯等六州縣二十三年水旱災額賦。甲戌,賑山西文水等十三州縣水災。甲申,賑湖北漢川等二縣衛水災。

二十七年春正月丙申,以奉皇太后巡省江、浙,詔免江蘇、安徽、浙江逋賦。賑河南祥符等州縣災民有差。丁酉,以科爾沁敏珠爾多爾濟旗災,貸倉穀濟之。丙午,上奉皇太后南巡,發京師,免直隸、山東經過地方本年錢糧十分之三,上年被災處十分之五。戊申,左都御史金德瑛卒,以董邦達代之。賑順直文安等二十八州縣上年水災。甲寅,賑山東曹、齊河等二縣水災有差。召多爾濟回京,命容保駐西寧辦事。丁巳,綏遠城將軍舒明卒,調蘊著代之。戊午,免山東惠民等十五州縣衛歷年民欠穀銀。己未,以周人驥固執開南明河,荒農累民,罷之。命喬光烈為貴州巡撫。癸亥,命清查俄羅斯疆界。

二月己巳,賑江蘇高郵等十一州縣、安徽太和等五州縣水災。庚午,命尹繼善為御前大臣。壬申,上奉皇太后渡河,閱清口東壩、惠濟閘。命阿里袞為御前大臣,高晉為內大臣。丙子,朝鮮入貢。丁丑,哈薩克使策伯克等入覲行在,賜冠服有差。庚辰,上奉皇太后渡江,閱京口兵。辛巳,上幸焦山。乙酉,上奉皇太后臨幸蘇州府。丙戌,免河南祥符等四十三州縣上年水災額賦。戊子,上謁文廟。

三月甲午朔,上奉皇太后臨幸杭州府。乙未,上幸海寧閱海塘。丁酉,賑湖北潛江等九州縣衛水災。戊戌,上閱兵。庚子,免江、浙節年未完地丁屯餉、漕項,並水鄉灶課銀。辛丑,賑山東齊河等五州縣上年水災。壬寅,上幸觀潮樓。賜浙江召試貢生沈初等二人舉人,與進士孫士毅等二人並授內閣中書。癸卯,上奉皇太后臨視織造機房。丙午,回蹕。丁未,加錢陳群刑部尚書銜。甲寅,上奉皇太后渡江。乙卯,命濬築直隸各河堤,以工代賑。丙辰,移山西歸綏道駐綏遠城。己未,上祭明太祖陵。閱兵。幸兩江總督尹繼善署。庚申,免江蘇江寧、蘇州,杭州附郭諸縣本年額賦。辛酉,賜江南召試諸生程晉芳等五人舉人,與進士吳泰來等三人並授內閣中書。壬戌,上奉皇太后渡江。

夏四月庚午,上閱高家堰,諭濟運壩至運口接建磚工。上奉皇太后渡河。以大理寺少卿顧汝修奉使安南,擅移書詰責國王,褫職。癸酉,命莊親王允祿等由水程奉皇太后回蹕。上登陸由徐州閱河。甲戌,免浙江仁和等十縣、湖州一所、仁和等五場上年水災額賦。庚辰,上祭孟子廟,謁先師廟。辛巳,上謁孔林。賑甘肅安定等十州縣上年雹災。壬午,免山東齊河等四十四州縣衛所上年水災額賦。戊子,皇太后登陸,駐蹕德州行宮。己丑,上送皇太后登舟。庚寅,命劉統勛會勘景州疏築事宜。辛卯,免順直大興等十州縣逋賦。

五月甲午,以乾清門行走額魯特鄂爾奇達遜奮勉勇往,賞三等伯爵。賑安徽壽州等十州縣衛上年水災。乙未,上至涿州。哈薩克陪臣阿塔海等入覲,賜冠服有差。賑長蘆屬滄州等七州縣及嚴鎮等七場上年水災灶戶,並免賦有差。辛丑,上詣黃新莊迎皇太后居申昜春園。賑湖南武陵等四州縣上年水災,並免額賦有差。癸卯,除安徽虹縣等四州縣衛水占窪地額賦。戊申,調鄂弼為陜西巡撫。以扎拉豐阿為正白旗領侍衛內大臣。癸丑,以倭和為總管內務府大臣。

閏五月癸亥朔,以清保年老,召來京。調格舍圖為盛京將軍,朝銓署之。丁卯,免湖北潛江等九州縣衛上年水災額賦。辛巳,籍沒納延泰財產。辛卯,命西安將軍如松襲封信郡王,以德昭之子修齡襲如松公爵。改察哈爾都統嵩椿為西安將軍,以巴爾品代之。

六月丁酉,免直隸固安七十四州縣上年水災額賦。壬寅,召此次南巡接駕休致之編修沈齊禮來京,及因事降革之馮鎬等十三員引見。乙巳,以庫爾勒伯克等進貢,諭計直頒賞,仍通諭各城,非盛典進方物者皆止之。己酉,以原任將軍班第、參贊大臣鄂容安在伊犁竭忠全節,命於伊犁關帝廟後設位致祭。

秋七月壬戌,以朝鮮三水府滋事逃人越境,命恆祿等赴邊境查勘。癸亥,免安徽壽州等十六州縣衛上年水災額賦。戊辰,上奉皇太后巡幸木蘭,免經過地方本年錢糧十分之五。乙亥,霍罕侵據額德格訥阿濟畢布魯特之鄂斯等處,諭永貴檄霍罕還之。

八月庚子,建伊犁之固勒札、烏哈爾里克兩城,賜名綏定、安遠。上奉皇太后回駐避暑山莊。甲辰,託恩多丁憂,調明山署廣東巡撫,蘇昌兼署,湯聘為江西巡撫,以宋邦綏為湖北巡撫,愛必達兼署。壬子,免順直文安等十七州縣逋賦及寧河等五縣本年水災額賦。丙辰,賜察哈爾都統敕書。黑龍江將軍綽勒多卒,調國多歡代之。

九月癸亥,賞自哈薩克來投之塔爾巴哈沁額魯特巴桑銀綺。庚午,上奉皇太后回蹕。辛未,巴達克山素勒坦沙遣使入覲。丁丑,命乾清門侍衛明仁帶御醫馳視胡寶瑔疾。賑山東齊河等三十五州縣衛水災,並免額賦。甲申,建烏魯木齊城堡,賜城名曰寧邊、輯懷,堡名曰宣仁、懷義、樂全、寶昌、惠徠、屢豐。戊子,理籓院尚書、領侍衛內大臣富德以索取蒙古王公馬畜,褫職逮問。己丑,以新柱為理籓院尚書,明瑞為正白旗領侍衛內大臣。

冬十月辛卯,調陳宏謀為湖南巡撫,宋邦綏署之,莊有恭為江蘇巡撫,熊學鵬為浙江巡撫,馮鈐為廣西巡撫,顧濟美護之。癸巳,緬目宮裏雁以焚殺孟連土司刀派春全家,命處斬,傳首示眾。癸卯,以愛烏罕汗愛哈默特沙遣使入貢,諭沿途督撫預備筵宴,並命額勒登額護送。乙巳,設總管伊犁等處將軍,以明瑞為之。命築科布多城。己酉,賑順直霸州等六十三州縣水雹霜災,免江蘇清河等十七州縣衛本年水災額賦。甲寅,賑浙江仁和等二十八州縣衛場水災。丁巳,奉天府府尹通福壽以徇縱治中高錦勒索商人,解任鞫治。

十一月己未朔,濬山東德州運河。庚申,設伊犁參贊大臣,以愛隆阿、伊勒圖為之。辛酉,設伊犁領隊大臣。命明瑞等率兵驅逐塔爾巴哈臺山陰之哈喇巴哈等處越牧哈薩克。戊辰,以薩魯布魯特頭目沙巴圖交還所掠霍罕貿易人等馬匹,諭永貴等酌賞之。呼什齊布魯特為霍罕所侵來投,命移於阿拉克圖呼勒等處游牧。庚午,命博斯和勒為杜爾伯特盟長,設副將軍二員,以車凌烏巴什為右翼副將軍,巴桑為左翼副將軍。辛未,建喀什噶爾新城。壬申,改山西平魯營參將為都司,裁原設中軍守備及井坪營都司。丙子,哈薩克努爾賚、烏爾根齊城哈雅克等遣使入覲。甲申,諭方觀承仿河南濬道路溝洫。賑甘肅皋蘭等二十州縣本年冰雹霜雪災。戊子,濬山東壽張等州縣河道溝渠。

十二月庚寅,大學士史貽直以老病乞休,優詔慰留,命不必兼攝工部,以示體恤。丙申,克什密爾呢雅斯伯克請入覲,允之。霍罕呈書,以布魯特鄂斯故地為己有,諭永貴等嚴檄令給還。辛丑,以霍罕伯克復永貴等書謂前遣使人奉旨稱為汗,欲以喀什噶爾為界,諭嚴檄斥駁之。丁未,工部尚書歸宣光卒,以董邦達代之。壬子,命納世通赴喀什噶爾辦事,代永貴回京。癸丑,巴達克山侵圍博羅爾,諭新柱等嚴檄責令息兵,並索獻布拉尼敦妻孥。

二十八年春正月庚申,賑順直屬之霸州等三十五州縣、山東齊河等三十州縣衛水災有差。甲子,上御紫光閣,賜愛烏罕、巴達克山、霍罕、哈薩克各部使人宴。丁卯,上大閱申昜春園之西廠,命各部使人從觀。以法起為歸化城都統。壬申,命阿桂在軍機處行走。壬午,河南巡撫胡寶瑔卒,以葉存仁為河南巡撫。甲申,以納世通為參贊大臣,駐喀什噶爾,總理回疆事務。壬辰,命方觀承赴河南會勘漳河工程。戊戌,改西安滿洲、漢軍副都統為左右翼副都統。壬寅,裁西寧辦事大臣。庚戌,上謁昭西陵、孝陵、孝東陵、景陵。是日,回蹕。改烏魯木齊副將為總兵。乙卯,命侍郎裘曰修督辦直隸水利。

三月己未,上還京師。壬戌,免山東齊河等三十一州縣衛水災額賦。丁卯,上謁泰陵。是日,回蹕。賞寧津縣百有三歲壽民李友益及其子侄孫銀牌緞疋有差。丁丑,設伊犁額魯特總管三員,副總管以下員額有差。戊寅,命福德赴庫倫,同桑齋多爾濟辦事。丙戌,免江蘇清河等十四州縣衛水災額賦。

夏四月壬辰,賑浙江錢塘等十七州縣場上年水災。癸卯,上詣黑龍潭祈雨。乙巳,雨。戊申,法起以贓免。以傅良為歸化城都統。壬子,賜秦大成等一百八十八人進士及第出身有差。甲寅,裁歸化城都統。

五月辛酉,圓明園火。癸亥,命尚書阿桂往直隸霸州等處,會同侍郎裘曰修、總督方觀承督辦疏濬事。以舒赫德署工部尚書。甲子,封朝鮮國王孫李算為世孫。己巳,果親王弘適以干與朝政削王爵,仍賞給貝勒。和親王弘晝以儀節僭妄,罰俸三年。庚午,大學士史貽直卒。壬申,上試翰林、詹事等官,擢王文治等三員為一等,餘各升黜有差。甲戌,上奉皇太后秋獮木蘭。以李侍堯為湖廣總督,輔德為湖北巡撫,陳宏謀兼署之。調劉綸為戶部尚書,仍兼署兵部。以陳宏謀為兵部尚書。調喬光烈為湖南巡撫,來朝署之。乙亥,以崔應階為貴州巡撫。己卯,調明德為江西巡撫。以和其衷為山西巡撫。丙戌,命福德往庫倫辦事,仍帶署理籓院侍郎銜。以額爾景額為參贊大臣,往葉爾羌辦事。

六月庚寅,山東歷城等州縣蝗。壬辰,賑甘肅狄道等三十州縣水旱霜雹災。戊戌,開泰以恇怯規避免。以鄂弼為四川總督,明山為陜西巡撫,阿里袞署之,阿思哈為廣東巡撫,蘇昌兼署,命阿思哈先署廣西巡撫。壬寅,四川總督鄂弼卒。以阿爾泰為四川總督,崔應階為山東巡撫,圖勒炳阿為貴州巡撫,吳達善兼署云南巡撫。以梁詩正為東閣大學士,劉綸協辦大學士。調陳宏謀為吏部尚書,彭啟豐為兵部尚書,張泰開為左都御史。甲辰,上幸簡親王第視疾。壬子,簡親王奇通阿卒。

秋七月庚申,英廉丁憂,命舒赫德兼署戶部尚書,劉綸留部治事。戊辰,仍設西寧辦事大臣,以七十五為之。己巳,順直大城、滄州等州縣蝗。庚辰,履親王允祹卒。

八月癸巳,賜烏魯木齊城名曰迪化,特訥格爾城名曰阜康。辛丑,上奉皇太后幸木蘭,行圍。

九月乙卯朔,日食。乙丑,上奉皇太后回駐避暑山莊。庚午,上奉皇太后回蹕。癸酉,改甘肅臨洮道為驛傳道,兼巡蘭州府,洮岷道為分巡鞏秦階道。丙子,上奉皇太后還京師。

冬十月甲申,加梁詩正、高晉太子太傅,兆惠、劉綸、阿里袞、舒赫德、秦蕙田、阿桂、陳宏謀、楊錫紱、楊廷璋、李侍堯、蘇昌、阿爾泰太子太保,莊有恭、劉藻太子少保。丙戌,上臨奠履親王允祹。丁未,免江蘇銅山等九州縣水災額賦。

十一月甲寅朔,召成袞扎布來京,以扎拉豐阿署烏里雅蘇臺將軍,雅郎阿留科布多。辛酉,河東河道總督張師載卒,以葉存仁代之。調阿思哈為河南巡撫,明山為廣東巡撫,明德為陜西巡撫,輔德為江西巡撫,常鈞為湖北巡撫。以楊應琚兼署甘肅巡撫。丁卯,大學士梁詩正卒。己卯,以楊廷璋為體仁閣大學士,仍留閩浙總督任。

十二月乙酉,免直隸延慶等十州縣雹旱災額賦。丁亥,賑甘肅皋蘭等十二縣旱災饑民。辛卯,賑山東濟寧等八州縣衛水災。乙未,召國多歡來京,調富僧阿為黑龍江將軍。庚子,休致左都御史梅成卒。丁未,命綽克托赴烏魯木齊辦事,代旌額里回京。

二十九年春正月癸丑朔,賑山東濟寧等七州縣衛、甘肅永昌等二十四州縣災民。甲戌,加賑雲南江川等五州縣地震災民,並免額賦。己卯,朝鮮入貢。

二月丁亥,命阿敏爾圖駐藏辦事,代福鼐回京。甲午,上謁泰陵。乙未,命觀音保赴伊犁,代愛隆阿回京。己亥,上還京師。己酉,免上年直隸蔚州雹災、萬全縣旱災額賦。辛亥,免湖北沔陽等三州縣衛上年水災額賦。

三月癸丑,太子太傅、大學士來保卒。乙卯,移陜甘總督駐蘭州,兼管甘肅巡撫事,裁甘肅巡撫。移固原提督回駐西安。改河州鎮總兵為固原鎮總兵。免山東濟寧等七州縣衛上年水災額賦。庚申,上臨故大學士來保第賜奠。免江蘇銅山等二十八州縣衛上年水災額賦。壬戌,命兆惠署工部尚書,阿桂赴西寧會同七十五及章嘉呼圖克圖選派郭羅克頭目。

夏四月甲午,賑甘肅金縣等縣旱災。

五月壬子朔,諭粵海關官貢毋進珍珠等物。辛酉,以託恩多署兵部尚書。

六月癸未,賑湖南武岡等州縣水災。甲申,命玉桂赴北路,代扎拉豐阿回京。丁亥,河東河道總督葉存仁卒,以李宏代之。庚寅,奉天寧遠等州縣蝗。丁酉,賑廣東英德等縣水災。甲辰,調蘇昌為閩浙總督,李侍堯為兩廣總督,明山署之。調吳達善為湖廣總督。以劉藻為雲貴總督。乙巳,調常鈞為雲南巡撫。以王檢為湖北巡撫。丁未,命阿爾泰回四川總督。

秋七月辛亥朔,以楊應琚為大學士,留陜甘總督任,陳宏謀協辦大學士。壬子,命常鈞暫兼署湖廣總督,劉藻兼署云南巡撫。甲子,湖北黃梅等州縣江溢,命撫恤災民。丙寅,湖南湘陰等州縣湖水溢,命賑恤災民。丁卯,上奉皇太后秋獮木蘭。癸酉,上奉皇太后駐蹕避暑山莊。丁丑,賑安徽當塗等州縣水災。

八月辛巳,免甘肅皋蘭等三十二州縣本年旱災額賦。壬辰,諭阿爾泰等曉諭綽斯甲布九土司會攻金川。戊戌,上奉皇太后巡幸木蘭,行圍。秦蕙田以病解任,以劉綸兼署禮部尚書。庚子,增伊犁、雅爾等處領隊大臣各二員。以綽克托為塔爾巴哈臺參贊大臣。命伍彌泰等仍留烏魯木齊辦事。

九月己未,命刑部侍郎阿永阿會同吳達善讞湖南新寧縣民傳帖罷市獄。癸亥,賑江西南昌等八縣水災,並免額賦。丙寅,刑部尚書秦蕙田卒,以莊有恭代之,暫留江蘇巡撫任。己巳,上奉皇太后回駐避暑山莊。

冬十月癸巳,喬光烈以新寧罷市獄褫職,調圖勒炳阿為湖南巡撫。以方世俊為貴州巡撫。丙申,以託恩多為理籓院尚書。辛丑,山東進牡丹。壬寅,賑江蘇上元等六州縣災民。癸卯,召鍾音回京。調富明安赴葉爾羌辦事。甲辰,賑安徽懷寧等十九州縣衛水災。

十一月壬子,賑甘肅皋蘭等二十州縣旱災。癸丑,築呼圖壁城成,賜名曰景化。丙辰,免湖南武岡等二州縣水災額賦。賑甘肅皋蘭等十五州縣水雹災。乙丑,協辦大學士、戶部尚書兆惠卒,上臨奠。丁卯,以阿里袞為戶部尚書、協辦大學士。調託恩多為兵部尚書。以五吉為理籓院尚書,兆德為正黃旗領侍衛內大臣。

十二月戊寅朔,以常復為烏里雅蘇臺參贊大臣。戊子,賑湖北黃梅等州縣水災。甲午,禮部尚書陳德華病免,調董邦達代之。以楊廷璋為工部尚書。

三十年春正月戊申,以皇太后四巡江、浙,免江蘇、安徽、浙江歷年因災未完丁漕。賑甘肅皋蘭等二十九州縣旱災、湖北監利等四縣水災有差。癸丑,劉綸丁憂,命莊有恭以刑部尚書協辦大學士。以於敏中為戶部尚書。調明德為江蘇巡撫,和其衷為陜西巡撫。以彰寶為山西巡撫,文綬護之。壬戌,上奉皇太后啟蹕南巡。癸亥,免直隸、山東經過州縣額賦十分之三。

二月戊子,上奉皇太后渡河。閱清口東壩木龍、惠濟閘。命阿桂赴伊犁辦事。壬辰,免江蘇州縣乾隆二十八年以前熟田地丁雜款舊欠,並經過州縣本年額賦之半。丙申,上奉皇太后渡江。己亥,朝鮮入貢。

閏二月丙午朔,上奉皇太后臨幸蘇州府。上謁文廟。己酉,免江寧、蘇州、杭州附郭諸縣本年丁銀。免浙江經過州縣本年額賦之半。辛亥,醜達改葉爾羌辦事。命索琳赴庫倫辦事。以額爾景額為喀什噶爾參贊大臣。壬子,上奉皇太后臨幸杭州府。乙卯,烏什回人作亂,戕辦事大臣素誠。丁巳,加沈德潛、錢陳群太子太傅。命明瑞進剿烏什。庚申,命明瑞、額爾景額總理烏什軍務,明瑞節制各軍。命阿桂、明亮赴伊犁辦事。辛酉,舒赫德留京辦事。以託恩多署工部尚書。戊辰,調明山為江西巡撫,王檢為廣東巡撫,李侍堯兼署。以李因培為湖北巡撫。己巳,賜伊犁新築駐防城名曰惠遠,哈什回城曰懷順。乙亥,免江蘇上元等五縣上年水旱災額賦。

三月丙子朔,賑湖北漢陽等七州縣上年水災。上幸焦山。戊寅,上奉皇太后駐江寧府。壬午,上詣明太祖陵奠酒。幸尹繼善署。觀音保剿烏什逆回失利。甲申,以馮鈐為湖南巡撫,宋邦綏為廣西巡撫。丙戌,上奉皇太后渡江。丁亥,果郡王弘適卒。甲午,以京察予大學士傅恆等敘。乙未,上閱高家堰堤,奉皇太后渡河。召尹繼善入閣辦事。以高晉為兩江總督。調李宏為江南河道總督,以李清時為河東河道總督。壬寅,追論素誠貪淫激變罪,籍產,戍其子於伊犁。以納世通、卡塔海諱匿敗狀,籍產治罪。命永貴赴喀什噶爾辦事。以託恩多署禮部尚書。癸卯,上渡河。

夏四月丙午朔,賑甘肅河州等三十六州縣上年雹水旱霜災。庚戌,免湖北漢陽等十二州縣衛上年水災額賦。辛亥,追予故刑部尚書王士禎謚文簡。丁巳,上奉皇太后駐德州。庚申,裁江蘇淮徐海道。丙寅,上還京師。庚午,上迓皇太后居申昜春園。辛未,哈薩克使臣鄂托爾濟等入覲。

五月乙亥,晉封喀爾喀郡王羅布藏多爾濟為親王。乙酉,上臨果郡王弘適殯所,及簡勤親王奇通阿園寢賜奠。以和闐辦事大臣和誠婪索回人,奪職逮問。命伊勒圖赴塔爾巴哈臺辦事。辛卯,京師地震。丁酉,免安徽懷寧等十九州縣衛上年水災額賦。甲辰,納世通、卡塔海貽誤軍務,正法。

六月己酉,以楊廷璋署兩廣總督,明山暫署,董邦達署工部尚書。乙卯,晉封令貴妃魏氏為皇貴妃。己巳,諭明瑞勿受烏什逆回降。

秋七月辛巳,上奉皇太后秋獮木蘭。戊子,以官保為左都御史。乙未,前和闐辦事大臣和誠以貪婪鞫實,正法。丁酉,奪喀爾喀親王桑齋多爾濟爵。

八月甲辰朔,減朝審、秋審緩決三次以上刑。己未,上幸木蘭行圍。庚申,賑甘肅靖遠等十一縣旱災。甲子,甘肅寧遠等州縣地震,命賑恤,並免本年額賦。

九月丙子,賑山東章丘等二十一州縣水災。戊寅,命尹繼善管兵部,劉統勛管刑部。烏什叛回以城降。乙酉,以高恆為總管內務府大臣。辛卯,以明瑞等未將烏什叛人殄誅,送往伊犁,下部嚴議。辛丑,以李侍堯署工部尚書。

冬十月己酉,明瑞、阿桂以辦烏什事務錯繆,褫職留任。賑長蘆屬滄州等三場水災。己巳,楊應琚陛見。命和其衷署陜甘總督,湯聘署陜西巡撫。

十一月癸酉,免江蘇海州等六州縣本年旱災額賦。乙酉,以吏部尚書傅森年老,授內大臣,調託恩多代之。以託庸為兵部尚書。調馮鈐為安徽巡撫。庚寅,醜達以扶同桑齋多爾濟私與俄羅斯貿易,正法。明瑞等以盡誅烏什附逆回眾奏聞。辛卯,賑山東章丘等十八州縣水災,甘肅狄道等十二州縣雹霜災。甲午,以阿桂為塔爾巴哈臺參贊大臣,代安泰回京。丁未,解阿桂工部尚書,以蘊著代之。以嵩椿為綏遠城將軍。戊申,賑甘肅靖遠等十一縣旱災,並免額賦。乙卯,賑山東齊河等十五州縣水災。丁卯,命託恩多兼署兵部尚書。壬辰,封皇五子永祺為榮親王。

十二月戊午,以陜西涇陽縣貢生張璘七世同居,賜禦制詩章、緞匹。


\end{pinyinscope}