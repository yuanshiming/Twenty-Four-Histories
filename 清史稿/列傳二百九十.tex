\article{列傳二百九十}

\begin{pinyinscope}
藝術二

王澍蔣衡徐用錫王文治梁巘梁同書

鄧石如錢伯坰吳育楊沂孫吳熙載梅植之楊亮

王澍,字若林,號虛舟,江南金壇人。績學工文,尤以書名。康熙五十一年進士,入翰林,累遷戶科給事中。雍正初,詔以六科隸都察院。澍謂科臣掌封駁,品卑任重,儻隸臺臣,將廢科參,偕同官崔致遠、康五端抗疏力爭。世宗怒。立召詰之,從容奏對,上意稍解,遂改吏部員外郎。越二年,告歸,益耽書,名播海內。摹古名搨殆遍,四體並工。於唐賢歐、褚兩家,致力尤深,輒跋尾自道所得。後內閣學士翁方綱持論與異,謂其篆書得古法,行書次之,正書又次之。所著題跋及淳化閣帖考正,並行於世。

自明、清之際,工書者,河北以王鐸、傅山為冠,繼則江左王鴻緒、姜宸英、何焯、汪士鋐、張照等,接踵而起,多見他傳。大抵淵源出於明文徵明、董其昌兩家,鴻緒、照為董氏嫡派,焯及澍則於文氏為近。澍論書尤詳,一時所宗。

蔣衡,改名振生,字湘帆,晚號拙老人。與澍同里。鍵戶十二年,寫十三經。乾隆中,進上,高宗命刻石國學,授衡國子監學正,終不出。衡早歲好游,足跡半海內,觀碑關中,獲晉、唐以來名跡,臨摹三百餘種,曰拙存堂臨古帖。晚與澍相期斫勝,每臨一書,相從質證。子驥,孫和,並以書世其家。

驥尤精分隸,著漢隸譌體集、古帖字體、續書法論各一卷,兼工畫。其言曰:「漢、魏字體不同,性情各異。書須懸臂中鋒,而用力以和平為主。作畫之提頓逆折,參差映帶,其理一爾。」皆闡明其先說。

徐用錫,字壇長,宿遷人,占籍大興。登鄉舉,康熙四十八年進士,官翰林院編修。從李光地游,究心樂律、音韻、歷數、書法。五十四年,分校會試,嚴絕請託,銜之者反嗾言官劾其把持闈事,聖祖原之,終以浮議罷歸。乾隆初,起授翰林院侍讀,年已八十。尋告歸,卒於家。用錫鄉舉出姜宸英之門,與何焯同為光地客,論書多與二家相出入。精於鑒別古人,言筆法亦多心得,著字學劄記二卷,載圭美堂集中。

王文治,字禹卿,江蘇丹徒人。生有夙慧,十二歲能詩,即工書。長游京師,從翰林院侍讀全魁使琉球,文字播於海外。乾隆三十五年,成一甲三名進士,授翰林院編修。逾三年,大考第一,擢侍讀。出為雲南臨安知府,因事鐫級,乞病歸。後當復官,厭吏事,遂不出,往來吳、越間,主講杭州、鎮江書院。高宗南巡,至錢塘僧寺,見文治書碑,大賞愛之。內廷有以告,招之出者,亦不應。

喜聲伎,行輒以歌伶一部自隨,辨論音律,窮極幽渺。客至張樂,窮朝暮不倦。海內求書者,多有餽遺,率費於聲伎。然客散,默然禪定,夜坐,肋未嘗至席。持佛戒,自言吾詩與書皆禪理也。卒,年七十三。

所著詩集外有快雨堂題跋,略見論書之旨。文治書名並時與劉墉相埒,人稱之曰「濃墨宰相,淡墨探花」。與姚鼐交最深,論最契,當時書名,鼐不及文治之遠播;後包世臣極推鼐書,與劉墉並列上品,名轉出文治上。

梁巘,字聞山,安徽亳州人。乾隆二十七年舉人,官四川巴縣知縣。晚辭官,主講壽春書院,以工李北海書名於世。初為咸安宮教習,至京師,聞欽天監正何國宗曾以事系刑部,時尚書張照亦以他事在系,得其筆法,因詣家就問。國宗年已八十餘,病不能對客,遣一孫傳語。巘質以所聞,國宗答曰:「君已得之矣。」贈以所臨米、黃二帖。

後巘以語金壇段玉裁曰:「執筆之法,指以運臂,臂以運身。凡捉筆,以大指尖與食指尖相對,筆正直在兩指尖之間,兩指尖相接如環,兩指本以上平,可安酒杯。平其肘,腕不附幾,肘圓而兩指與筆正當胸,令全身之力,行於臂而湊於兩指尖。兩指尖不圓如環,或如環而不平,則捉之也不緊,臂之力尚不能出,而況於身?緊則身之力全湊於指尖,而何有於臂?古人知指之不能運臂也,故使指頂相接以固筆,筆管可斷,指鍥痛不可勝,而後字中有力。其以大指與食指也,謂之單勾;其以大指與食指中指也,謂之雙勾;中指者,所以輔食指之力也,總謂之『撥鐙法』。王獻之七、八歲時學書,右軍從旁掣其筆不得,即謂此法。舍此法,皆旁門外道。二王以後,至唐、宋、元、明諸大家,口口相傳如是,董宗伯以授王司農鴻緒,司農以授張文敏,吾聞而知之。本朝但有一張文敏耳,他未為善。王虛舟用筆祗得一半,蔣湘帆知握筆而少作字樂趣。世人但言無火氣,不知火氣使盡,而後可言無火氣也。如此捉筆,則筆心不偏,中心透紙,紙上颯颯有聲。直畫粗者濃墨兩分,中如有絲界,筆心為之主也。如此捉筆,則必堅紙作字,輭薄紙當之易破。其橫、直、撇、捺皆與今人殊,筆鋒所指,方向迥異,筆心總在每筆之中,無少偏也。古人所謂屋漏痕、折金義股、錐畫沙、印印泥者,於此可悟入。」巘少著述,所傳緒論僅此。當時與梁同書並稱,巘曰「北梁」,同書曰「南梁」。

梁同書,字元穎,晚號山舟,浙江錢塘人,大學士詩正子。乾隆十七年,會試未第,高宗特賜與殿試,入翰林,大考,擢侍講。淡於榮利,未老,因疾不出。晚年重宴鹿鳴,加侍講學士銜。卒,年九十三。好書出天性,十二歲能為擘窠大字。初法顏、柳,中年用米法,七十後乃變化。名滿天下,求書者紙日數束,日本、琉球皆重之。

嘗與張燕昌論書,略曰:「古人云『筆力直透紙背』,當與天馬行空參看。今人誤認透紙,便如藥山所云『看穿牛皮』,終無是處。蓋透紙者,狀其精氣結撰墨光浮溢耳,彼用筆如游絲者,何嘗不透紙背耶?用腕力使極輭之筆自見,譬如人持一彊者,使之直,則無所用力;持一弱者,欲不使之偃,則全腕之力,自然集於兩指端。其實書者只知指運,而不知有腕力也。藏鋒之說,非筆如鈍錐之謂,自來書家從無不出鋒者,只是處處留得筆住,不使直走。筆耍輭,輭則遒;筆要長,長則靈;筆耍飽,飽則腴;落筆耍快,快則意出。書家燥鋒曰渴筆,畫家亦有枯筆,二字判然不同。渴則不潤,枯則死矣。今人喜用硬筆故枯。帖教人看,不教人摹。今人只是刻舟求劍,將古人書摹畫如小兒寫仿本,就便形似,豈復有我?字耍有氣,氣須從熟得來。有氣則有勢,大小、長短、高下、欹整,隨筆所至,自然貫注,成一片段,卻著不得絲毫擺布,熟後自知。中鋒之法,筆提得起,自然中,亦未嘗無兼用側鋒處,總為我一縷筆尖所使,雖不中亦中。亂頭粗服非字也,求逸則野,求舊則拙,此處不可有半點名心在。」同書平生書旨,與梁巘之異同,具見於此。

鄧石如,初名避仁宗諱,遂以字行,改字頑伯,安徽懷寧人。居皖公山下,又號完白山人。少產僻鄉,鮮聞見,獨好刻石,仿漢人印篆甚工。弱冠孤貧,游壽州,梁巘見其篆書,驚為筆勢渾鷙,而未盡得古法。介謁江寧梅鏐,都御史成子也。家多■L3藏金石善本,盡出示之,為具衣食楮墨,使專肄習。

好石鼓文,李斯嶧山碑、泰山刻石,漢開母石闕,燉煌太守碑,吳蘇建國山碑,皇象天發神讖碑,唐李陽冰城隍廟碑、三墳記,每種臨摹各百本。又苦篆體不備,寫說文解字二十本。帝搜三代鐘鼎,秦、漢瓦當、碑額。五年,篆書成。乃學漢分,臨史晨前、後碑,華山碑,白石神君,張遷,潘校官,孔羨,受禪,大饗諸碑,各五十本。三年,分書成。石如篆法以二李為宗,縱橫闢闔,得之史籀,稍參隸意,殺鋒以取勁折,字體微方,與秦、漢當額為近。分書結體嚴重,約嶧山、國山之法而為之。自謂:「吾篆未及陽冰,而分不減梁鵠。」

客梅氏八年,學既成,遍游名山水,以書刻自給。游黃山,至歙,鬻篆於賈肆。編修張惠言故深究秦篆,時館修撰金榜家,偶見石如書,語榜曰:「今日得見上蔡真跡。」乃冒雨同訪於荒寺,榜備禮客之於家。薦於尚書曹文埴,偕至京師,大學士劉墉、副都御史陸錫熊皆驚異曰:「千數百年無此作矣!」時京師論篆、分者,多宗內閣學士翁方綱,方綱以石如不至其門,力詆之。石如乃去,客兩湖總督畢沅,沅故好客,吳中名士多集節署,裘馬都麗,石如獨布衣徒步。居三年,辭歸,沅為置田宅,俾終老。瀕行,觴之,曰:「山人,吾幕府一服清涼散也!」石如年四十六始娶,常往來江、淮間,卒,年六十三。

子傳密,初名廷璽,字守之。從李兆洛學,晚客曾國籓幕。能以篆書世其家。

當乾、嘉之間,嘉定錢坫、陽湖錢伯坰,皆以書名。坫自負其篆直接陽冰,嘗游焦山,見壁間篆書心經,嘆為陽冰之亞。既而知為石如所作,摭其不合六書者以為詆。伯坰故服石如篆、分為絕業,及見其行、草,嘆曰:「此楊少師神境也!」復與論筆法不合,遂助坫詆之尤力。坫見儒林傳。

伯坰,字魯斯,自號僕射山人,尚書維城從子。少孤,力學,工詩嗜酒,廣交游,以國子監生終。書學顏平原、李北海,嘗曰:「古人用兔毫,故書有中線,今用羊毫,其精者乃成雙鉤。吾躭此五十年,才十得三四。」論者謂自劉墉歿,正、行書以伯坰為第一。其執筆,虛小指,以三指包管外,與大指相拒,側毫入紙,助怒張之勢。指腕皆不動,以肘來去,斥古今相承撥鐙之說。石如作書,則懸腕雙鉤,管隨指轉,兩家法大殊。

吳育,字山子,江蘇吳江人。與包世臣、李兆洛游,能文,工書。謂:「下筆須使筆毫平鋪紙上,乃四面圓足,此陽冰篆法,書家真秘密語。」世臣取其說。育篆書尤工,法與石如差近。

楊沂孫,字詠春,江蘇常熟人。道光二十三年舉人,官安徽鳳陽知府。父憂歸,遂不出,自號濠叟,少學於李兆洛,治周、秦諸子。耽書法,尤致力於篆、籀,著文字解說問譌,欲補苴段玉裁、王筠所未備。又考上古逮史籀、李斯,折衷於許慎,作在昔篇。篆、隸宗石如,而多自得。嘗曰:「吾書篆、籀,頡頏鄧氏,得意處或過之;分、隸則不能及也。」光緒七年,卒,年六十九。沂孫同時工篆、籀者,又推吳大澂,自有傳。

吳熙載,初名廷颺,以字行,後又字讓之,江蘇儀徵人。先世居江寧,父明煌,始游揚州,善相人術。熙載為諸生,博學多能,從包世臣學書。世臣創明北朝書派,溯源窮流,為一家之學。其筆法兼採同時黃乙生、王良士、吳育、硃昂之、鄧石如諸人之說。執筆,食指高鉤,大指加食指、中指之間,中指內鉤,小指貼名指外拒,管向左迤,後稍偃,若指鼻準。運鋒,使筆毫平鋪紙上,筆筆斷而後起。結字計白當黑,使左右牝牡相得,自謂合古人八法、九宮之旨。熙載恪守師法,世臣真、行、槁草無不工,嗜篆、分而未致力,熙載篆、分功力尤深。復縱筆作畫,亦有士氣。咸豐中,卒。

與熙載同受包氏法者,江都梅植之蘊生,甘泉楊亮季子,高涼黃洵修存,餘姚毛長齡仰蘇,旌德姚配中仲虞,松桃楊承汪挹之。配中詳儒林傳。

植之,道光十九年舉人。通經,以詩鳴,世臣尤稱其書。謂其跌宕遒麗,段煉舊搨,血脈精氣,奔赴腕下,熙載未之敢先。又得琴法於吳思伯之女弟子顏夫人,獨具神解。糾正思伯傳譜,於古操制曲之故,輒能知之。自署所居曰嵇庵。配中與有同嗜,著琴學二卷。植之五十而卒,琴法未有傳書。

亮,世為將家,襲騎都尉世職。篤學敦行,江、淮間士大夫多稱之。書亞於熙載。

合肥沈用熙最後出,至光緒末始卒,年近八十。畢生守師法,最為包門老弟子。

世臣敘次清一代書人為五品,分九等:「平和簡靜,遒麗天成,曰神品;醖釀無跡,橫直相安,曰妙品;逐跡尋源,思力交至,曰能品;楚調自歌,不謬風雅,曰逸品;墨守跡象,雅有門庭,曰佳品。神品一人,鄧石如隸及篆書。妙品上一人,鄧石如分及真書;妙品下二人,劉墉小真書,姚鼐行草書。能品上七人,釋邱山真及行書,宋玨分榜書,傅山草書,姜宸英行書,鄧石如草書,劉墉榜書,黃乙生行榜書;能品下二十三人,王鐸草書,周亮工草書,笪重光行書,吳大來草書,趙潤草榜書,張照行書,劉紹庭草榜書,吳襄行書,翟賜履草書,王澍行書,周於禮行書,梁巘真及行書,翁方綱行書,於令行書,巴慰祖分書,顧光旭行書,張惠言篆書,王文治方寸真書,劉墉行書,汪庭桂分書,錢伯坰行及榜書,陳希祖行書,黃乙生小真行書。逸品上十五人,顧炎武正書,蕭雲從行書,釋雪浪行書,鄭簠分及行書,高其佩行書,陳洪綬行書,程邃行書,紀映鍾行書,金農分書,張鵬翀行書,袁枚行書,硃筠槁書,硃珪真書,鄧石如行書,宋鎔行書;逸品下十六人,王時敏行及分書,硃彞尊分及行書,程京萼行書,釋道濟行書,趙青藜真及行書,錢載行書,程瑤田小真書,巴慰祖行書,汪中行書,畢涵行書,陳淮行書,姚鼐小真書,程世淳行書,李天澂行書,伊秉綬行書,張桂巖行書。佳品上二十二人,沈荃真書,王鴻緒行書,先著行書,查士標行書,汪士鋐真書,何焯小真書,陳奕禧行書,陳鵬年行書,徐良行書,蔣衡真書,於振行書,趙知希草書,孔繼涑行書,稽璜真書,錢澧行書,桂馥分書,翁方綱小真書,張燕昌小真書,康基田行書,錢坫篆書,穀際岐行書,洪梧小真書;佳品下十人,鄭來行書,林佶小真書,方觀承行書,董邦達行書,華嵒行書,秦大士行書,高方小真書,金榜真書,吳俊行書,陳崇本小真書。」九品共九十七人,重見者六人,實九十一人。復增能品上一人,張琦真、行及分書;能品下三人,於書佃行書,段玉立小真及草書,吳德旋行書。佳品上六人,吳育篆及行書,方履籛分書,梅植之行書,硃昂之行書,李兆洛行書,徐準宜真書。

其後包氏之學盛行,咸、同以來,以書名者,何紹基、張裕釗、翁同龢三家最著,並見他傳。紹基宗顏平原法,晚復出入漢分;裕釗源出於包氏;同龢規模閎變,不為諸家所囿,為一代後勁云。


\end{pinyinscope}