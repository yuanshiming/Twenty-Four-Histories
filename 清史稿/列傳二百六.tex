\article{列傳二百六}

\begin{pinyinscope}
劉長佑劉岳昭岑毓英弟毓寶

劉長佑,字印渠,湖南新寧人。道光二十九年拔貢。與同縣江忠源友。咸豐二年,忠源率鄉勇赴廣西助剿,長佑從。粵匪自桂林走湖南,忠源破之於蓑衣渡,長佑有贊畫功,獎敘教諭。又從破瀏陽徵義堂會匪,擢知縣。三年,平衡山土匪,擢同知直隸州。忠源援湖北,遇賊崇、通間,長佑自長沙馳援,戰於通城,大破之,自是獨領一軍。忠源守南昌,長佑偕羅澤南赴援,解吉安圍,分兵克泰和,擢知府。忠源殉廬州,長佑偕忠源弟忠濬率千人馳援弗及,大憤,誓滅賊。

五年,江忠淑剿東安賊不利,駱秉章以長佑兼統其眾,所部始盛。克東安,追破之新寧。六年,復郴州,擢道員。江西賊方熾,秉章奏以長佑率蕭啟江等諸軍赴援,克萍鄉,加按察使銜。遣啟江復萬載,進圍袁州,屢擊敗援賊。十一月,降賊李能通為內應,克袁州。七年二月,進屯太平墟。賊由吉安大舉來襲,列陣二十餘里,以驍騎沖突,將士多死亡,全軍敗潰。長佑下馬引佩刀欲自裁,營務處劉坤一擁之上馬,退保分宜。近縣士民爭運糧械濟之,潰卒皆來歸,軍勢復振。

進規臨江,八月,石達開自撫州率二十萬眾來援,總兵普承堯戰峽江不利,賊薄太平墟。長佑乘其營壘未定,約蕭啟江、田興恕合戰,江忠義、李明惠先陷陣,盧秀峰繞其後,縱擊,大破之,遂圍郡城。捷聞,詔嘉其奮勇,賜號齊普圖巴圖魯。十二月,克臨江,殲賊酋張發紀,加布政使銜。八年,長佑病歸,以劉坤一代統其眾,蕭啟江自為一軍,合克新淦、崇仁,進克撫州。是年夏,長佑復至軍,屯建昌,迭敗賊於新城、金谿,敗入福建界。江西邊境肅清,記名遇江西道員缺簡放。

九年,回軍湖南剿郴、桂賊,解永州圍,記名以按察使題奏。石達開圍寶慶,長佑與李續宜分扼東西兩路,賊敗走,長佑追破之九鞏橋、白楊埔、大臨橋、蘆洪司,遂竄廣西,陷興安、靈州,直撲桂林。長佑倍道赴援,賊不虞其驟至,走慶遠,追擊之,所向皆捷,授廣西按察使,逾月,擢布政使。攻柳州,拔之。

十年,擢廣西巡撫。四月,克慶遠,破達開於思恩,又破之興安,乃遁竄。時廣西土匪猶蔓延,大者踞郡縣,小者千百為群,倏兵倏賊。長佑蒞任,整飭吏治,興練水師,匪氛漸戢。商貨流通,稅釐增倍。軍事餉事差能自固,不盡仰資鄰省。十一年,遣劉坤一剿柳州土匪,斬其渠伍聲揚,餘黨就撫。調水陸軍剿潯州艇匪,克府城,斬其渠陳開。貴州匪首黃金義投誠復叛,擒斬之。同治元年,長佑親赴潯州督防,分軍進剿,迭克要隘。尋擢兩廣總督,以所部楚軍付劉坤一接統,留剿廣西諸匪。

未幾,調直隸總督。時降捻張錫珠、宋景詩先後叛,畿輔騷動。二年春,長佑航海至天津,即赴衡水督師。三月,破賊束鹿,殲張錫珠。命督辦直隸、山東、河南三省交界剿匪事宜。宋景詩踞劉貫寨、甘官屯,合山東軍攻之,以遲延降級留任。九月,破賊張秋鎮,殲賊目楊殿一,景詩逸走,乃罷軍。

四年,僧格林沁戰歿曹州,捻匪益熾,畿南戒嚴,長佑遣兵自開州至張秋扼河防。奉天馬賊入喜峰口,坐疏防議處。八月,捻匪竄山東濮、範南岸,長佑馳赴大名,擊走之。疏請直隸分練六軍,議定營制,加練二軍,下部議行。

六年,滄州梟匪張六等劫慶雲、鹽山、寧津、南皮四縣鹽場,山東教匪應之,保定、天津、河間三府屬皆騷動。長佑檄前籓司唐訓方屯齊河,臬司張樹聲屯張秋,防捻。自率兵剿捕鹽梟,賊乘虛北走,過滹沱河,眾增至千餘,竄擾涿州、固安、永清、霸州,逼近京師,詔褫長佑職,以大學士官文代之。命下數日而梟匪平,予三品頂戴,率所部回籍。尋東捻平,詔念前勞,晉二品頂戴。

十年,起授廣東巡撫,尋調廣西。初,奸民出關劫掠越南,官兵不能制。悍酋吳終伏誅,而蘇國漢復起。九年,廣西提督馮子材進軍龍州,國漢旋乞撫於兩廣總督瑞麟,仍招納亡命,匪首鄧建新、曾亞日,分路肆擾。至是總兵劉玉成擒亞日於上林社,誅之。復會廣東軍攻克舊街,乘勝抵海寧,匪多散亡,國漢奔東興,亦就擒。長佑奏言:「論越南大局,則宜直搗河陽,一勞永逸。然河陽距關二千餘里,窮兵勞費,討捕為難。今擬芟蕩海陽、太原,即回師列戍,以固籓籬。庶可分助越之眾,協剿黔苗;抽出關之兵,先清土莽。」十月,副將陳得貴、游擊李揚才克越南從化府,遂會劉玉成克通化、白通,破瓊山、北山匪巢。十一年正月,復敗匪於三星山,擒其酋何三等,餘黨悉平。長佑檄劉玉成暫屯鎮撫,咨越南國王遣兵換防,久之不至;又以營弁滋事,暴兵非計,七月,乃撤入關內,搜捕沿邊伏匪。

時匪酋黃崇英猶踞越南河陽,結白苗攻保樂,擾我鎮安邊。十二年春,長佑檄關內外軍擊走之,密奏:「越南貧弱,版章日蹙,法國蠶食於濱海,黎裔虎視於橫山,桶岡則白苗跳梁,峒奔則黃酋雄踞。近聞其國君臣輸款法人,黃崇英受職黎裔,雖系道聽之言,亦系意中之事。臣竊謂黎裔為患,越南受之;法國為患,不僅越南受之。今欲拯敝扶衰,必須大舉深入。若合兩粵之力,寬以數年之期,步步設防,節節進剿,庶交夷可期復振,而他族不至生心。否則惟有慎固邊防,嚴杜勾結而已。」是時防越諸軍尚八千人,長佑檄劉玉成引軍北還,以六營屯關外諸隘,四營屯歸順、龍卅,令覃遠璡八營分駐關內。

十月,法人攻陷河內,黃崇英等乘機襲太原,潛與之通。山西奸民響應,北寧戒嚴。越南乞援,乃令劉玉成統十營進太原為左軍,道員趙沃統十營分部鎮安為右軍。法人尋與越南議和,黃崇英為越將劉永福所敗,潛伏河陽,遣黨陳亞水攻保樂。十三年十月,長佑閱兵南寧,令趙沃、劉玉成進軍。光緒元年二月,趙沃右軍由龍闌渡河克同文,白苗棄巢遁,沃撫之為助,遂攻底定、襄安,皆克之。劉玉成左軍敗賊白通,陣斬鄧志雄。崇英聞師至,嗾周建新拒左軍,陳亞水守猛法,自當右軍,憑險拒守。五月,沃軍克淰臺,直薄河陽,崇英敗走。右軍復敗陸之平援眾,進攻猛法。陳亞水惶懼,乞為內應,河陽、安邊同日降。崇英遁走,捕獲誅之。劉玉成左軍亦克通化、白通,斬周建新,合攻者巖,克之。陸之平遁,宣光、金沙江上下肅清。凱撤入關。

擢雲貴總督,二年,抵任。先是,滇邊野番殺英人馬加理,為交涉鉅案,及議定,允於雲南設埠通商。詔下其議,長佑疏言:「雲南山川深阻,種人獷悍成性,剽掠行旅。本地紳練,恃眾橫行,挾制官長。上下猜忌,法令不行。萬一防護不及,致有同於前案,或更甚於前案,其有害於雲南一隅猶小,其有撓於中夏全局甚大。且洋人知前案難辦,有免其既往之議;知後患難防,有保其將來之議。臣恐滇省官民,於已往者不以為幸免,而以為得計;將來者不引為前鑒,或敢於效尤。洋人通商,意在圖利,亦斷無不思遠害之理。應俟三五年內外官民稍稍安定,遣員商辦。」長佑以滇事漸定,屢疏引病乞罷,優詔慰留。

四年,騰越徼外土目耿榮高等攻陷耿馬,長佑遣將討之,榮高降;又剿平臨安、開化、廣南土匪。初,騰越蘇關先之亂,其黨劉寶玉逃之野山。野山在滇、緬之交,其夷自為君長,不隸羈屬。劉寶玉糾野貫十三種及盞達夷伏羅坤山,時出劫掠。會緬甸遣官詣騰越,持圖說約由野山通道列戍。長佑檄熊昭鏡赴騰越,召諸土司、野貫申禁約,誘誅寶玉於千崖,諸野夷皆解散。

七年,法兵窺越南東京,詔滇、粵備邊。長佑疏言:「法人自據嘉定以來,越南四境皆有商埠、教堂,脅其君臣,漁其財力。取越與否,非有甚異。其所以處心積慮,乃在通商雲南。與其既失越境,為守邊之計,不若乘其始動,為弭釁之謀。滇、粵三省,與越接壤,東西幾二千里,要害與共,勞費殊甚。若自三江口以至海陽,東西僅數百里,以中國兵力為之禦敵,兵聚而力省。以視防守滇、粵邊境,勞逸懸殊。請以廣西兵二萬為中路,廣東、雲南各以萬人相犄角。廣東之兵自欽、連而入,雲南之兵出洮江而東。別以輪船守廣東順化港口,斷其首尾,法人必無自全之理。」又力言劉永福可禦敵,請密諭越王給其兵食。疏入,詔下廷議。

八年,法兵陷東京,越匪紛起,廣西援兵至太原,長佑檄道員沈壽榕率軍出關,與為聲援。長佑屢以病乞罷,慰留未許。八月,入覲,予假兩月,九年,乃許開缺回籍。尋坐雲南報銷失察,降三級。十三年,卒於家。詔念前功,嘉其端謹老成,開復處分,仍依總督例議恤,謚武慎。廣西、雲南、湖南並立專祠。

劉岳昭,字藎臣,湖南湘鄉人。以文童投效湘軍。咸豐六年,從蕭啟江援江西,轉戰積功,累擢以知縣用。啟江器其才,使領果後營。七年,破賊高安鶯哥嶺,連拔彭家村賊巢。進攻臨江,擊敗援賊於太平墟。尋克臨江府城,擢同知。八年,從剿撫州賊,大捷於何家村、香溪諸處。崇仁賊踞白陂墟,又破之。由上頓渡進偪撫州,賊開東門逸,復其城,擢知府,賜花翎。九年,援南康,克新城墟,進搗池江。前軍潰,岳昭殿後,斃賊甚眾,克南安,援信豐,解其圍,加道員銜。石達開由江西擁眾犯湖南,岳昭移軍茶陵備之,而賊已趨寶慶,奉檄馳援。至柳家橋,遏東路,賊六萬餘撲營,岳昭偕副將餘星元、楊恆升等鏖戰三日,斃賊數千,援軍大集,賊解圍而遁。是役岳昭戰最力,名始顯。

十年,屯江華,賊酋陳金剛踞廣西賀縣,阻山為固,岳昭招降其黨。進拔蓮塘縣,破河東街賊屯,合蔣益澧軍克縣城,以道員記名,加按察使銜。是年冬,連破竄匪於道州、宜章,湘境肅清,賜號鼓勇巴圖魯。

十一年,駱秉章赴四川督師,疏請岳昭率所部從行。中途聞粵匪陳玉成犯湖北,陷隨州,秉章令岳昭回軍赴援,會諸軍克之,以按察使記名。石達開由龍山犯宣恩,窺伺施南,岳昭迎擊走之。而黔匪陷來鳳,同治元年春,岳昭進軍克其城,分軍截剿,迭捷於散毛河、白蘭壩兩河口,抵黑洞,斬馘尤多。石達開竄四川,圍涪州,岳昭會知府唐蜅、副將唐有耕破之仰天窩。渡江重慶截擊,解涪州圍。賊敗踞長寧,攻克之,復追敗之先市寨、得用壩、丁子場。賊尋踞敘州雙龍場,約降賊郭集益內應,破其營,殪賊近二萬。貴州巡撫張亮基疏薦其才,請擢用,二年,授雲南按察使;三年,遷布政使;皆未之任,留四川治軍。

駱秉章奏遣援黔,九月,克仁懷,連敗馬氾灘踞匪。四年,克正安,追賊至清溪河,斬其渠。五年,擢雲南巡撫,進規綏陽。天臺山最為城北險隘,列陣綴其前,從山後攻入,平其壘,投誠者三百餘寨。綏陽城賊吳元彪乞降,黔西北路始通。由溫水進剿,平菉竹山老巢,收降鐵匠坪、九倉壩及被脅巖洞二十餘處。六年,破沙窩踞賊,解大定圍。拔大屯朵壩賊壘,會滇軍平豬拱箐苗,又拔平遠牛場坉苗巢。黔西肅清。

七年,疏陳云南軍事,命赴本任。尋擢雲貴總督,駐軍曲靖。進攻尋甸,破七星橋木城,扼文筆山、法鼓山要沖,剷平附近村莊賊壘。收復果馬,疊捷於塘子、張徐灣諸處。援賊大至,圍攻果馬,各營皆陷,革職留任。八年,解馬龍圍,進逼尋甸,賊首馬天順、李芳園乞撫,遂復其城。

雲南捻亂已久,各軍惟布政使岑毓英所部最強,而毓英素尚意氣,岳昭開誠專任,調發進止悉聽之。毓英尋擢巡撫,和衷無牽制,軍事日有起色。九年,克麗江,復威遠、姚州,復永北、鶴慶、鎮南、鄧川、浪穹,拔鳳羽白米莊賊巢,平彌勒縣竹園踞匪。十年,平永善蠻匪,拔賓州賊巢,平香爐山槓匪,連克河西之大東溝、小東溝及臨安之五山夷寨。十一年,復貴州興義新城,先後克永平、雲南及趙州、蒙化各城。攻大理上下兩關,復大理府城,誅大酋杜文秀,詔復原職。十二年,滇省肅清,賜黃馬褂,疏請陛見。

光緒元年,以入覲遷延,御史李廷簫劾其規避,下部議褫職。九年,卒。署湖南巡撫龐際雲疏陳:「岳昭統兵十餘年,建功之地,黔屬為多;任事之艱,雲南為最;請復原官。」詔允之。

岳昭之規尋甸也,杜文秀遣黨萬餘,戰不利。從弟岳晙請岳昭速還曲靖,以固根本。賊果分黨往襲,以有備不得逞。岳晙守馬龍,賊圍之,伺懈出擊,走之。固守數月,練兵得三千人,會攻尋甸,破七星橋要隘,賊蹙乞降,猶懷反側,岳晙率三十人入城,示以坦白,人心始定。次日,毓英兵亦至,服其膽略。岳晙先以積功擢至道員,岳昭至滇後,專任毓英滇軍,其舊部多遣去云。

岑毓英,字彥卿,廣西西林人。諸生。治鄉團,擊土匪,以功敘縣丞。咸豐六年,率勇赴雲南迤西助剿回匪。九年,克宜良,權縣事。十年,克路南,署州事,擢同知直隸州。進攻澂江,兼署知府。十一年,克澂江賊壘,破昆陽海口賊,迤西回匪連陷楚雄、廣通、祿豐,省城戒嚴。毓英赴援,同治元年,破賊大樹營。時總督張亮基引疾去,巡撫徐之銘主撫,回酋馬如龍通款,毓英往諭順逆,如龍獻所踞新興等八城,之銘奏以毓英攝布政使。尋以安撫功,加按察使銜,賜花翎。二年,回弁馬榮叛,戕總督潘鐸,毓英率所部粵勇一千,與弟毓寶等守籓署。之銘微服詣毓英,司道皆集,分兵守東、南門,密召馬如龍入援。如龍至,誅亂黨,馬榮跳走南寧,合馬聯升踞曲靖八屬。詔嘉毓英守城功,擢道員。

率師西剿,復富民、安寧、羅次、高明、祿豐、武定、祿勸、廣通、陸涼、南安諸城,及黑、元、永三鹽井,進搗楚雄。會東路有警,之銘檄回省,分兵克霑益、平彞。赴楚雄督攻,克其城。進復大姚、雲南、趙州、賓川、鄧川、浪穹、鶴慶,分道進規大理上下關。三年,克定遠,圍攻鎮南,大破援賊於普棚。馬聯升復陷霑益,犯馬龍,回軍破之於天生關。進攻曲靖,復馬龍、霑益。進克尋甸,擒馬榮、馬興才,克曲靖,擒馬聯升,並誅之。尚書趙光疏呈滇紳公啟,言毓英所向有功,特詔嘉勉,下總督勞崇光據實保奏。四年,肅清迤東,加布政使銜,賜號勉勇巴圖魯。

西路自毓英軍移去,所克諸城多復陷,僅存楚雄未失。毓英駐軍曲靖,護省城運道。五年,命署布政使,勞崇光至是始至滇受事,奏以提督馬如龍專辦西路,令毓英督剿豬拱箐苗。豬拱箐隸貴州威寧州,與海馬姑相犄角,山溪阻深,苗酋陶新春、陶三春分據之。糾聚苗、教諸匪及粵匪石達開餘黨,凡十數萬人,迭擾滇之鎮雄、彞良、大關、昭通,黔之大定、黔西、威寧、畢節,且及川疆,三省會剿久無功。毓英上書駱秉章,謂權不一則軍不用命,原率滇軍獨任,期百二十日覆其巢,授迤西道,署布政使如故。

六年,擢布政使。二月,師抵豬拱箐,令張保和、林守懷領二千人,由大溜口出二龍關後,掩襲吳家屯,自督三千人攻關。賊傾巢出戰,關後砲發,賊回救,毓英揮軍夾擊,三隘皆下,遂奪吳家屯,擒斬數千。賊自海馬姑來援,截擊之,斬其酋,餘賊反奔。令蔡標、劉重慶分軍圍剿海馬姑,克紅巖、尖山,賊援乃斷,遂逼豬拱箐老巢。賊以巨石自山顛墜下,驅牛馬突營,將士多傷亡,毓英督軍搏戰,斬悍酋,賊始卻。於營前掘深坎,賊所發石盡陷坎內,誘降惈人,得賊虛實,選敢死士二千,填壕以進,連破木城二,直搗其巢,縱火焚之,斬馘二萬,擒陶新春及其死黨,磔之,拔山男婦四萬餘人。乘勝合攻海馬姑,伏兵山前後,進毀賊壘三十餘,以噴筒環燒,擒陶三春及悍酋二百餘人,皆斬之,賊悉平。計自進兵至是,僅逾期四日,加頭品頂戴。

馬如龍剿迤西屢失利,勞崇光病歿,杜文秀大舉東犯,連陷二十餘城,省垣告急。是年冬,毓英自豬拱箐凱旋曲靖,先遣弟毓寶助省防。七年春,揚言師出陸涼,而取道宜涼、七甸,連破大小石壩、小板橋、古庭庵、金馬寺賊壘,進屯大樹營。馬如龍來會,人心始定。昆陽匪首楊震鵬夜渡昆明池襲省城,毓寶擊敗之,震鵬負創遁。進攻楊林,毓英鼻受槍傷,回軍省城,連破石虎關賊壘,擒賊渠李洪勛,擢授巡撫。附省賊壘猶繁,與之相持。總督劉岳昭初至滇,由馬龍進剿尋甸,失利,賊勢復熾。

毓英疏陳軍事、餉事,略曰:「杜文秀竊踞迤西十有三載,根深蒂固。今擬三路進兵,一出迤南牽賊勢,一出三姚、永北斷賊援,大軍由楚雄、鎮南直搗中堅,使賊面面受敵,不能兼顧。臣選精銳六萬,更番戰守,既無停兵之時,亦免師老之患。兵勇無須外募,以本省兵剿本省賊,既習地利,復熟賊情。現在滇省兵勇鄉團已調集八萬有奇,擬俟附省逆壘肅清,認真裁汰,選定精銳,以資得力。滇省綠營額設馬步兵三萬七千數百名,承平日久,訓練多疏,將不知兵,兵不知戰。倉卒有事,則募勇以代兵;餉需支絀,不能不後兵而先勇。於是兵丁愈困,營務益弛。通省營兵所存不及十一,臣擬即此六萬人中,擇補營額,目前仍令隨征,事竣再飭歸伍。既有常業,自有恆心,責以成功,收效必速。滇省近年用兵,多藉鄉勇之力,擬按州縣之大小,定徵調之多寡,共編鄉勇四十營,分兩班隨營征討,餉銀仍由各地籌捐。兩年之內,迤西肅清,即可裁撤歸農。滇省兵勇,向於餉銀之外,每名月支米三斗。現擬用兵六萬,每年共需米二十餘萬石,為數甚鉅。歷年皆按成熟田畝酌抽釐穀,約十分取其一二,資助軍食,與川之津貼,黔之義穀,名異實同。今請照舊抽收,並將近年可徵地丁抽糧,全數改徵糧米,如不敷用,再行籌價採買接濟,一俟軍事肅清,分別裁止。滇省綠營官兵俸餉,有閏之年,需銀七十萬兩有奇,無閏需銀六十四萬數千兩。現既易勇為兵,則餉銀較勇糧稍厚。倘因籌餉維艱,每月先給半餉,加以賞需軍火各費,約共需銀八萬兩。鹽課、地丁、釐稅之外,每月所短不過三四萬兩,應由外省協撥,較之向例協餉,有減無增。若發全餉,則每月應由外省撥銀六萬,較常例所增亦屬無幾。現在部臣指撥各省協滇軍餉,如浙江、廣東、江西,距滇較遠,籌撥起解,往返經年,緩難濟急。請飭改作京餉,另由川、楚等省應解京餉,改撥濟滇,兩無窒礙。至於選任鎮將,宜不拘資格,不惜情面,凡有能將三千兵以上,才當一面者,雖其名位尚卑,亦宜委署要職。其謀勇平常,僅止熟習營務,縱系實缺,另予差遣,勿使幸位。」疏入,下部如所議行。

八年春,賊酋楊榮率眾數萬踞楊林長坡,分黨踞小偏橋、十里鋪、羊芳凹、牛街、興福寺,省城大震。毓英督諸軍分剿,奪回小偏橋諸處,復連敗之於蕭家山、鸚鵡山,擒斬逾萬,劃除省東賊壘百餘。西北兩方賊仍負隅拒守,毓英令副將楊玉科、總兵李維述等規迤西,與騰越義兵約期並進。於是副將張保和等克富民、昆陽,總兵馬忠等克呈貢、晉寧、易門、澂江、祿豐,玉科等克武定、祿勸、元謀、羅次、定遠、大姚,維述等克廣通、楚雄、南安及黑瑯、元水諸井。凡悍酋劇匪,擒斬殆盡,省城解嚴,被詔嘉獎。

九年,澂江回復叛,踞府城,毓英率軍往剿,圍其郛,十年二月,克之。並拔竹園、江那諸賊巢,迤西軍亦克麗江、劍川、永北、鶴慶、賓川、姚州、鎮南諸城。疏言:「滇省前事之誤,東南未定,遽議西征,屢致喪師失地。今通籌全局,必先掃蕩東南兩迤,然後全軍西上,方無後顧之憂。」

十一年,迤東、迤西兩路悉平,西軍亦先後克復永昌、鄧川、浪穹、趙州、雲南、永平、蒙化及上下兩關,而大理賊猶堅守,恃騰越、順寧互為應援。十一月,毓英親往督戰,先斷賊援,直薄城下,掘隧道,陷城垣數十丈,奪東南兩門入。賊守內城,晝夜環攻,守陴賊多死。杜文秀窮蹙服毒,其黨舁之出城詐降,斬首傳示,勒繳軍械,賊黨猶請緩期。毓英令楊玉科率壯士二百入城受降,布重兵城外夾擊之,斬酋目三百餘名,生擒楊榮、蔡廷棟、馬仲山,磔於市。大理肅清,賜黃馬褂,予騎都尉世職。十二年,順寧、雲州、騰越皆下,全滇底定,加太子少保,晉一等輕車都尉世職。

十三年,兼署云貴總督。光緒二年,丁繼母憂。五年,服闋,授貴州巡撫,加兵部尚書銜。七年,調福建督辦臺灣防務,開山撫番,濬大甲溪,築臺北城。八年,署云貴總督,九年,實授。

法越兵事起,自請出關赴前敵,屯興化。十年,命節制關外粵、楚各軍。會廣西軍潰於北寧、太原,毓英全師退屯保勝,以未奉命,降二級留任。七月,命進軍決戰,連復越南館司、鎮安、清波、夏和諸縣,屯館司關,規取河內諸省。令丁槐、何秀林攻宣光,以地雷毀其城,擒斬甚眾。十一年,京察,開復降級處分,令覃修綱攻克緬旺、清水、清山。法兵援宣光,掘地營延袤十餘里扼之。破法兵於臨洮府,奪梅枝關。連克不拔、廣威、永祥,進搗山西、河內,廣西軍亦收復諒山。越南興安、寧平、南定、興化、太原各省聞風響應。會和議成,詔班師。五月,回駐邊關。十二年,會勘邊界,兼署巡撫。十三年,剿順寧惈黑夷匪張登發,平之。十四年,京察,議敘。十五年,皇太后歸政,晉太子太保。尋卒,贈太子太傅,入祀賢良祠,雲南、貴州建專祠,謚襄勤。子春煊,官至四川總督。

弟毓寶,從毓英轉戰雲南,功最著,累擢道員,賜號額圖琿巴圖魯。光緒十年,出關援剿宣光、臨洮,旋克廣威府、不拔縣、梅枝關,賜黃馬褂。十四年,授福建鹽法道,擢雲南按察使,權布政使,護巡撫,兼護總督。二十一年,調貴卅布政使,未行,復調雲南。毓寶勇於戰陣,不諳文法,御史溥松劾其護總督時,任用私人,政刑失當,坐奪職,卒於家。雲貴總督崧蕃疏陳毓寶戰功,詔復原官。

論曰:劉長佑樸誠廉毅,老於軍事,時病其失之慈柔。自言:「於是非邪正,不自欺以欺人。」非飾辭也。滇、粵籌邊,尤有遠見。劉岳昭治滇,能屈己以聽岑毓英。毓英與滇事相終始,跋扈霸才,竟成戡定偉績,信乎識時之傑,能自樹立者已。


\end{pinyinscope}