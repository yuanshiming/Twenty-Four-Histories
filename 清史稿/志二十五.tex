\article{志二十五}

\begin{pinyinscope}
時憲六

△雍正癸卯元法上

日躔改法之原:

一,更定歲實以衡消長。歲實古多而今少,故授時有消長之術。西人第穀所定,減郭守敬萬分之三。至奈端等屢加測驗,謂第穀所減太過,定為三百六十五日二四二三三四四二0一四一五,比第穀所定多萬分之一有奇。以除周天三百六十度,得每日平行,比第穀所定少五纖有奇。本法用之。

一,更定黃赤距緯以徵翕闢。黃赤大距,古闊而今狹,恆有減而無增,西人利酌理、噶西尼測定黃赤大距二十三度二十九分,比第穀所定少二分三十秒,比刻白爾所定少一分。本法用之。一,細考清蒙氣差以祛歧視。西人第穀悟得蒙氣繞地球之周,日月星照蒙氣之外,人在地面為蒙氣所映,必能視之使高。而日月星之光線入蒙氣之中,必反折之使下。故光線與視線蒙氣之內合而為一,蒙氣之外,歧而為二。二線所交,即為蒙氣差角,然未有算術。噶西尼反覆精求,謂視線光線所歧雖有不同,相合則有定處。自地心過所合處作線抵圓周,即為蒙氣割線。視線與割線成一角,光線與割線亦成一角,二角相減,得蒙氣差角。爰在北極出地高四十四度處,屢加精測,得地平上最大差為三十二分一十九秒,蒙氣之厚為地半徑千萬分之六千零九十五,視線角與光線角正弦之比例,常如一千萬與一千萬零二千八百四十一。用是推得逐度蒙氣差。本法用之。如圖甲為地心,乙為地面,丙乙為蒙氣之厚,醜甲為割線,癸乙為視線,子戊為光線,癸戊子為蒙氣差角,癸寅、子卯為兩正弦。

一,細考地半徑差以辨蒙雜。康熙十一年壬子秒分前十四日夜半,火星與太陽沖,西人噶西尼於富郎濟亞國測得火星距天頂五十九度四十分一十五秒,利實爾於同一子午線之噶耶那島測得火星距天頂一十五度四十七分五秒,同時用有千里鏡能測秒微之儀器,與子午線上最近一恆星,測其相距。噶西尼所得火星較低一十五秒,因恆星無地半徑差以之立法,用平三角形,推得火星在地平上最大地半徑差二十五秒,小餘三七。又據歌白尼、第穀測得火星距地與太陽距地之比,如一百與二百六十六,用轉比例法,求得太陽在中距時地平上最大地半徑差一十秒,其逐度之差,以半徑與正弦為比例。本法用之,以求地半徑與日天半徑之比例,中距為一與二萬零六百二十六,最高為一與二萬零九百七十五,最卑為一與二萬零二百七十七,地平上最大地半徑差最高為九秒五十微,最卑為一十秒一十微。

一,用橢圓面積為平行以酌中數。西人刻白爾以來,屢加精測,盈縮之最大差止一度五十六分一十二秒。以推逐度盈縮差,最高前後,本輪失之小,均輪失之大;最卑前後,本輪失之大,均輪失之小。乃以盈縮最大差折半,檢其正弦,得一六九000為兩心差。以本天心距最高卑為一千萬,作橢圓,自地心出線,均分其面積,為平行度,以所夾之角為實行度,以推盈縮。在本輪、均輪所得數之間,而逐度推求,苦無算術。噶西尼等乃立角積相求諸法,驗諸實測,斯為菂合。本法用之。如圖甲為地心,乙為本天心,丁為最高,丙為最卑,戊己為中距,瓜分之面積為平行,所對之平圓周角度為黃道實行。一,更定最卑行以正引數。西人噶西尼等測得每歲平行一分二秒五十九微五十一纖零八忽,比甲子元法多一秒四十九微有奇。本法用之。

一,更定平行所在以正歲首。用西人噶西尼所定,推得雍正癸卯年天正冬至為丙申日醜正三刻十一分有奇,比甲子元法遲二刻。次日子正初刻最卑過冬至八度七分三十二秒二十二微,比甲子元法多十七分三十五秒四十二微。

月離改法之原:

一,求太陰本天心距地及最高行,隨時不同,以期通變。自西人刻白爾創隋圓之法,奈端等累測月離,得日當月天中距時最大遲疾差為四度五十七分五十七秒,兩心差為四三三一九0。日當月天最高,或當月天最卑,則最大遲疾差為七度三十九分三十三秒,兩心差為六六七八二0。日歷月天高卑而後,兩心差漸小;中距而後,兩心差漸大;日距月天高卑前後四十五度,兩心差適中。又日當月天高卑時,最高之行常速,至高卑後四十五度而止;日當月天中距時,最高之行常遲,至中距後四十五度而止;與日月之盈縮遲疾相似,而周轉之數倍之。因以地心為心,以兩心差最大最小兩數相加折半,得五五0五0五,為最高本輪半徑。相減折半,得一一七三一五,為最高均輪半徑。均輪心循本輪周右旋,行最高平行度;本天心循均輪周起最遠點右旋,行日距月天最高之倍度。用平三角形,推得最高實均。又推得逐時兩心差,以求面積。如日躔求盈縮法,以求遲疾,名曰初均。本法用之。如圖戊為地心,甲壬癸子為本輪,乙丁丑丙為均輪,丙丁皆本天心,丙為最遠,丁為最近,戊丙兩心差大,己庚橢圓面積少,戊丁兩心差小,辛申橢圓面積多。

一,增立一平均數以合時差。西人刻白爾以來,奈端等屢加測驗,得日在最卑後太陰平行常遲,最高平行、正交平行常速。日在最高後反是。因定日在中距,太陰平行差一十一分五十秒,最高平行差一十九分五十六秒,正交平行差九分三十秒。其間逐度之差,皆以太陽中距之均數與太陽逐度之均數為比例,名曰一平均。本法用之。

一,增立二平均數以均面積。西人奈端以來,屢加精測,得太陽在月天高卑前後太陰平行常遲,至高卑後四十五度而止。在月天中距前後反是。然積遲、積速之多,正在四十五度,而太陽在最高與在最卑,其差又有不同。因定太陽在最高,距月天高卑中距後四十五度之最大差為三分三十四秒;太陽在最卑,距月天高卑中距後四十五度之最大差為三分五十六秒。高卑後為減,中距後為加,其間日距月最高逐度之差,皆以半徑與日距月最高倍度之正弦為比例。太陽距地逐度之差,又以太陽高卑距地之立方較與太陽本日距地同太陽最高距地之立方較為比例,名曰二平均。本法用之。

一,增立三平均數以合交差。西人奈端以來,定白極在正交均輪周行日距正交之倍度,因定太陽在黃白兩交後,則太陰平行又稍遲;在黃白大距後,則太陰平行又稍速;其最大差為四十七秒。兩交後為減,大距後為加。其逐度之差,皆以半徑與日距正交倍度之正弦為比例,名曰三平均。本法用之。

一,更定二均數以正倍離。西人噶西尼以來,屢加測驗,定日在最高朔望前後四十五度,最大差為三十三分一十四秒;日在最卑朔望前後四十五度,最大差為三十七分一十一秒。朔望後為加,兩弦後為減。其間月距日逐度之二均,則以半徑與月距日倍度之正弦為比例。其太陽距最高逐度二均之差,又以日天高卑距地之立方較與本日太陽距地同太陽最高距地之立方較為比例,與二平均同。本法用之。

一,更定三均數以合總數。西人噶西尼以來,取月距日與月高距日高共為九十度時測之,除末均之差外,其差與月距日或月高距日高之獨為九十度者等。又取月距日與月高距日高共為四十五度時測之,亦除末均差外,其差與月距日或月高距日高之獨為四十五度者等。乃定太陰三均之差,在月距日與月高距日高之總度半周內為加,半周外為減。其九十度與二百七十度之最大差為二分二十五秒。其間逐度之差,以半徑與總度之正弦為比例。本法用之。

一,增立末均數以合距度。西人噶西尼以來,測日月最高同度或日月同度兩者只有一相距之差,則止有三均。若兩高有距度,日月又有距度,則三均之外,朔後又差而遲,望後又差而速。及至月高距日高九十度、月距日亦九十度時,無三均,而其差反最大。故知三均之外,又有末均。乃將月高距日高九十度分為九限,各於月距日九十度時測之,兩高相距九十度,其差三分;八十度,其差二分三十九秒;七十度,其差二分一十九秒;六十度,其差二分;五十度,其差一分四十三秒;四十度,其差一分二十八秒;三十度,其差一分一十六秒;二十度,其差一分七秒;一十度,其差一分一秒。其間逐度之差,用中比例求之。其間月距日逐度之差,皆以半徑與月距日之正弦為比例。朔後為減,望後為加。本法用之。

一,更定交均及黃白大距以合差分。西人奈端、噶西尼以來,測得日在兩交時,交角最大為五度一十七分二十秒;日距交九十度時,交角最小為四度五十九分三十五秒。朔望而後,交角又有加分。因日距交與月距日之漸遠,以漸而大,至日距交九十度、月距日亦九十度時,加二分四十三秒。交均之最大者,為一度二十九分四十二秒。乃以最大、最小兩交角相加折半,為繞黃極本輪;相減折半,為負白極均輪。分均輪全徑為五,取其一,內去朔望後加分,為最大加分小輪全徑,設於白道,餘為交均小輪全徑。與均輪全徑相減,餘為負小輪全徑,與均輪同心,均輪負而行,不自行。均輪心行於本輪周,左旋,為正交平行。交均小輪心在負小輪周,起最遠點,右旋,行日距正交之倍度。白極在交均小輪周,起最遠點,左旋,行度又倍之。而白道上之加分小輪,其周最近。黃道之點,與朔望之白道相切,其全徑按日距正交倍度為大小,常與最大加分小輪內所當之正矢等。又按本時全徑內取月距日倍度所當之正矢為所張之度,驗諸實測,無不菂合。本法用之。如圖甲為黃極,乙為本輪,丙為均輪,丁為負小輪,戊己皆為交均小輪,庚辛皆為白極,壬為黃道,醜、癸皆為朔望時白道,寅、子皆為兩弦時白道,卯、辰皆為白道上加分小輪。

一,更定地半徑差以合高均。求得兩心差最大時,最高距地心一0六六七八二0,為六十三倍地半徑又百分之七十七;最卑距地心九三三二一八0,為五十五倍地半徑又百分之七十九。兩心差最小時,最高距地心一0四三三一九0,為六十二倍地半徑又百分之三十七;最卑距地心九五六六八一0,為五十七倍地半徑又百分之一十九;中距距地心一千萬,為五十九倍地半徑又百分之七十八。又用平三角形,求得太陰自高至卑逐度距地心線及地平上最大差。其實高逐度之差,皆以半徑與正弦為比例。

一,更定三種平行及平行所在。太陰每日平行,比甲子元法多千萬分秒之二萬二千三百一十六,最高每日平行,比甲子元法少百萬分秒之七千二百五十一,正交每日平行,比甲子元法少十萬分秒之一百三十七。雍正癸卯天正冬至,次日子正,太陰平行所在,比甲子元法多二分一十四秒五十七微,最高平行所在,比甲子元法少三十六分三十七秒一十微,正交平行所在,比甲子元法多五分六秒三十三微。

交食改法之原:

一,用兩時日躔、月離黃道度求實朔、望。先推平朔、望以求其入交之月,次推本日、次日兩子正之日躔、月離黃道經度以求其實朔、望之時,又推本時次時兩日躔、月離以比例其時刻。與甲子元法止用兩日及用黃白同經者不同。一,用兩經斜距求日、月食甚時刻及兩心實相距。以黃白二道原非平行,而日、月兩經常相斜距。若以太陽為不動,則太陰如由斜距線行,故求兩心相距最近之線,不與白道成正角,而與斜距線成正角。其距弧變時,亦不以月距日實行度為比例,而以斜距度為比例。如圖甲乙為黃道,戊乙為白道,甲戊為實朔、望距緯,甲癸為太陽一小時實行,戊丑為太陰一小時實行。設太陽不動而合癸與甲,則太陰不在醜而在寅。戊寅為一小時兩經斜距線,甲卯與戊寅成正角,即為兩心相距最近之線,戊卯為食甚距弧,皆借弧線為直線,用平三角形求之。初虧、復圓,則以並徑為弦作勾股。一,更定日、月實徑與地徑之比例。西人默爵制造鏡儀,測得日視徑最高為三十一分四十秒,中距為三十二分一十二秒,最卑為三十二分四十五秒;月視徑最高為二十九分二十三秒,中距為三十一分二十一秒,最卑為三十三分三十六秒。用此數推算日實徑為地徑之九十六倍又十分之六,月實徑為地徑百分之二十七,小餘二六強,太陽光分一十五秒。本法用之。

一,更定求影半徑法及影差。以日、月兩地半徑差相加,內減去日半徑,餘即為實影半徑。又月食時日在地下,蒙氣轉蔽日光,地影視徑大於實徑約為太陰地半徑差六十九分之一,是為影差。如圖甲丁辛三角形,丁辛二內角與壬甲辛一外角等,丁角即太陽地半徑差,辛角即太陰地半徑差,甲丁線略與甲丙日天半徑等,甲辛線略與甲己月天半徑等,其角皆與地半徑甲乙相當故。壬甲己對角丙甲丁即日半徑。故以丁角、辛角相加,即得壬甲辛角,內減壬甲己角,餘己甲辛角,即實影半徑。

圖形尚無資料

一,更定求日食食甚真時及兩心視相距。借弧線為直線,用平三角形,以食甚用時兩心實相距為一邊,用時高下差為一邊,用時白經高弧交角為所夾之角,求得對角之邊,為兩心視相距,並求得對兩心實相距角。復設一時,限西向後設,限東向前設。求其兩心實相距及高下差為二邊。白經高弧交角與對設時距弧角相減,餘為所夾之角,求得對角之邊,為設時兩心視相距,亦求得對兩心實相距角。乃取用時、設時兩白經高弧交角較,與用時對兩心實相距角相減。又加設時對兩心實相距角,又與全周相減為一角,用時、設時兩視相距為夾角之二邊,求其對邊為視行,求其中垂線至視行之點,為食甚真時所在,垂線為真時視相距。以上加減,據向後設而言。然後以所得真時,復考其兩心視相距果與所求垂線合,即為定真時。如圖乾為日心,乾子為用時兩心實相距,乾壬為高下差,壬子為兩心視相距,乾午為設時兩心實相距,乾己為高下差,己午同壬未為兩心視相距,壬丑中垂線為真時視相距。初虧、復圓法同,但以並徑為比考真時之限。至帶食則以地平為斷,亦逕求兩心視相距,不用視行。

恆星改法之原,見天文志。

土星改法之原,見推步因革篇。

羅★、計都更名,乾隆五年,和碩莊親王等援古法奏請更正,下大學士、九卿議奏,乾隆九年更正。

紫氣增設之原,大學士、伯訥爾泰等議覆,更定羅★、計都名目,★援古法增入紫氣,約二十八年十閏而氣行一周天,每日行二分六秒,小餘七二0七七七。以乾隆九年甲子天正冬至,次日子正在七宮十七度五十分十四秒五十三微為元。

日躔用數,雍正元年癸卯天正冬至為法元。壬寅年十一月冬至。

周歲三百六十五日二四二三三四四二。

太陽每日平行三千五百四十八秒,小餘三二九0八九七。

最卑歲行六十二秒,小餘九九七五。

最卑日行十分秒之一又七二四八。

本天橢圓大半徑一千萬,小半徑九百九十九萬八千五百七十一,小餘八五,兩心差十六萬九千。

宿度,乾隆十八年以前,用康熙壬子年表,十九年以後,用乾隆甲子年表,俱見天文志。

各省及蒙古、回部、兩金川土司北極高度、東西偏度,見天文志。

黃赤大距二十三度二十九分。

最卑應八度七分三十二秒二十二微。

氣應三十二日一二二五四。

宿應二十七日一二二五四。

宿名,乾隆十八年以前,同甲子元,十九年以後,易觜前參後,餘見甲子元法。

推日躔法求天正冬至,同甲子元法。

求平行,同甲子元法。

求實行,先求引數,同甲子元法。乃用平三角形,以二千萬為一邊,倍兩心差為一邊,引數為所夾之角,六宮內用內角,六宮外與全周相減用其餘。求得對倍兩心差之角,倍之為橢圓界角。又以本天小半徑為一率,大半徑為二率,前所夾角正切為三率,求得四率為橢圓之正切,檢表得度分秒。與引數相減,餘為橢圓差角。最卑前後各三宮與橢圓界角相加,最高前後各三宮與橢圓界角相減,自初宮為最卑後,以此順計。為均數。置平行,以均數加減之,引數初宮至五宮為加,六宮至十一宮為減。得實行。

求宿度。

求紀日值宿。

求節氣時刻。

求距緯度。

求日出入晝夜時刻。★同甲子元法。

月離用數太陰每日平行四萬七千四百三十五秒,小餘0二三四0八六。

最高每日平行四百零一秒,小餘0七0二二六。

正交每日平行一百九十秒,小餘六三八六三。

太陽最大均數六千九百七十三秒。

太陰最大一平均七百一十秒。

最高最大平均一千一百九十六秒。

正交最大平均五百七十秒。

太陽最高立方積一0五一五六二。

太陽高卑立方大較一0一四一0。

太陽在最高,太陰最大二平均二百一十四秒。

太陽在最卑,太陰最大二平均二百三十六秒。

太陰最大三平均四十七秒。

本天橢圓大半徑一千萬。

最大兩心差六六七八二0。

最小兩心差四三三一九0。

最高本輪半徑五五0五0五,即中數兩心差。

最高均輪半徑一一七三一五。

太陽在最高,太陰最大二均一千九百九十四秒。

太陽在最卑,太陰最大二均二千二百三十一秒。

太陰最大三均一百四十五秒。

兩最高相距一十度,兩弦最大末均六十一秒。

相距二十度,兩弦最大末均六十七秒。

相距三十度,兩弦最大末均七十六秒。

相距四十度,兩弦最大末均八十八秒。

相距五十度,兩弦最大末均一百零三秒。

相距六十度,兩弦最大末均一百二十秒。

相距七十度,兩弦最大末均一百三十九秒。

相距八十度,兩弦最大末均一百五十九秒。

相距九十度,兩弦最大末均一百八十秒。

正交本輪半徑五十七分半。

正交均輪半徑一分半。

最大黃白大距五度一十七分二十秒。

最小黃白大距四度五十九分三十五秒。

黃白大距中數五萬八千五百零七秒半。

黃白大距半較五百三十二秒半。

最大交角加分一千零六十五秒。

最大距日加分一百六十三秒。

太陰平行應五宮二十六度二十七分四十八秒五十三微。

最高應八宮一度一十五分四十五秒三十八微。

正交應五宮二十二度五十七分三十七秒三十三微。餘見日躔。

推月離法求天正冬至,同甲子元法。

求太陰平行,同甲子元法。

求最高平行,同甲子元法求月孛行。

求正交平行,同甲子元法。

求用平行,以太陽最大均數為一率,太陰最大一平均為二率,本日太陽均數化秒為三率,求得四率為秒。收為分,後皆同。為太陰一平均。又以最高最大平均為二率,一率、三率同前。求得四率為本日最高平均。又以正交最大平均為二率,求得四率,為本日正交平均,隨記其加減號。太陰正交與太陽相反,最高與太陽同。各加減平行,得太陰二平行及用最高用正交。於太陽實行內減去用最高,為日距月最高。減去用正交,為日距正交。次以半徑千萬為一率,太陽引數內加減太陽均數為實引,取其餘弦為二率,太陽倍兩心差為三率,求得四率為分股。又以實引正弦為二率,一率、三率同前。求得四率為勾;以分股與全徑二千萬相加減,實引三宮內九宮外加,三宮外九宮內減。為股弦和;求得弦。轉與全徑相減,為日距地心數。自乘再乘得立方積,與太陽最高立方積相減,為本時立方較。又以半徑千萬為一率,高卑最大二平均各為二率,日距月最高倍度正弦為三率,各求得四率,為本時高卑二平均。又以高卑立方大較為一率,本時立方較為二率,本時高卑二平均相減餘為三率,求得四率與本時最高二平均相加,為本時二平均,記加減號。日距月最高倍度不及半周為減,過為加。復以半徑千萬為一率,最大三平均為二率,日距正交倍度正弦為三率,求得四率,為三平均,記加減號。日距正交倍度不及半周為減,過為加。乃置二平行,加減二三平均,得用平行。

求初實行,用平三角形,以最高本輪半徑為一邊,最高均輪半徑為一邊,日距月最高倍度與半周相減,餘為所夾之角,求得對均輪半徑之角,為最高實均,記加減號。日距月最高倍度不及半周為加,過為減。又求得對原角之邊,為本時兩心差。以最高實均加減用最高為最高實行,以最高實行減用平行為太陰引數,復用平三角形,以半徑千萬為一邊,本時兩心差為一邊,太陰引數與半周相減餘為所夾之角,求得對兩心差之角。與原角相加,復為所夾之角。求得對半徑千萬之角,為平圓引數。乃以本天大半徑為一率,本時兩心差為正弦,對表取餘弦為二率,平圓引數之正切線為三率,求得四率為正切,檢表為實引,與太陰引數相減為初均數。置用平行,以初均數加減之,引數初宮至五宮為減,六宮至十一宮為加。得初實行。

求白道實行,置初實行,減本日太陽實行,為月距日。乃以半徑千萬為一率,高卑最大二均數各為二率,月距日倍度正弦為三率,各求得四率,為本時高卑二均數。又以高卑立方大較為一率,本時立方較為二率,本時高卑二均數相減餘為三率,求得四率,與本時最高二均數相加,為本時二均數,記加減號。月距日倍度不及半周為加,過為減。又置月距日,加減二均,為實月距日。置太陽最卑平行,加減六宮,為日最高太陰最高實行。內減日最高,為日月最高相距。與實月距日相加,為相距總數。以半徑千萬為一率,最大三均為二率,相距總數正弦為三率,求得四率,為三均數,記加減號。總數不及半周為加,過為減。又以半徑千萬為一率;日月最高相距度用中比例,取本時兩弦最大末均為二率,實月距日正弦為三率,求得四率,為末均數,記加減號。實月距日不及半周為減,過為加。乃置初實行,加減二均、三均、末均,得白道實行。

求黃道實行,用平三角形,以正交本輪半徑為一邊,正交均輪半徑為一邊,日距正交倍度為所夾之外角,倍度過半周,減去半周,用其餘。求得對兩邊二角之半較。與日距正交相減,餘為正交實均。以加減日距正交倍度不及半周為加,過為減。用正交,為正交實行。置白道實行,減正交實行,為月距正交。又以半徑千萬為一率,日距正交倍度正矢為二率,倍度過半周,與全周相減,用其餘。黃白大距半較為三率,求得四率,為交角減分。又以最大距日加分折半為三率,一率、二率同前。求得四率,為距交加差。又以半徑千萬為一率,實月距日倍度正矢為二率,倍度過半周,與全周相減,用其餘。距交加差折半為三率,求得四率,為距日加分。置最大大距,減交角,減分加距日加分,為黃白大距。乃以半徑千萬為一率,黃白大距餘弦為二率,月距正交、正切為三率,求得四率為正切,檢表為黃道距交度。與月距正交相減,餘為升度差。以加減白道實行,月距正交初、一、二、六、七、八宮為減,三、四、五、九、十、十一宮為加。得黃道實行。

求黃道緯度,同甲子元法。

求四種宿度,月孛用最高實行,羅★用正交實行加減六宮,計都用正交實行,餘同甲子元法。

求紀日值宿。

求交宮時刻。

求太陰出入時刻。

求合朔弦望。

求正升、斜升、橫升。

求月大小。

求閏月,並同甲子元法。

求月令,日躔娵訾,為建寅正月,東風解凍,蟄蟲始振,魚陟負冰,獺祭魚,候雁北,草木萌動,凡六候。日躔降婁,為建卯二月,桃始華,倉庚鳴,鷹化為鳩,玄鳥至,雷乃發聲,始電,凡六候。日躔大梁,為建辰三月,桐始華,田鼠化為鴽,虹始見,萍始生,鳴鳩拂其羽,戴勝降於桑,凡六候。日躔實沈,為建巳四月,螻蟈鳴,蚯蚓出,王瓜生,苦菜秀,靡草死,麥秋至,凡六候。日躔鶉首,為建午五月,螳螂生,鵙始鳴,反舌無聲,鹿角解,蜩始鳴,半夏生,凡六候。日躔鶉火,為建未六月,溫風至,蟋蟀居壁,鷹始摯,腐草為螢,土潤溽暑,大雨時行,凡六候。日躔鶉尾,為建申七月,涼風至,白露降,寒蟬鳴,鷹乃祭鳥,天地始肅,禾乃登,凡六候。日躔壽星,為建酉八月,鴻雁來,玄鳥歸,★鳥養羞,雷始收聲,蟄蟲坯戶,水始涸,凡六候。日躔大火,為建戌九月,鴻雁來賓,雀入大水為蛤,菊有黃華,豺乃祭獸,草木黃落,蟄蟲咸俯,凡六候。日躔析木,為建亥十月,水始冰,地始凍,雉入大水為蜃,虹藏不見,天氣上升,地氣下降,閉塞而成冬,凡六候。日躔星紀,為建子十一月,鶡鴠不鳴,虎始交,荔挺出,蚯蚓結,麈角解,水泉動,凡六候。日躔元枵,為建丑十二月,雁北鄉,鵲始巢,雉雊,雞乳,征鳥厲疾,水澤腹堅,凡六候。每五度為一候,按宮度推之即得。

五星用數,推五星行,並同甲子元法,惟土星平行應減去三十分。

恆星用數,見天文志,推恆星法,同甲子元法。

紫氣用數,乾隆九年甲子天正冬至為法元。癸亥年十一月冬至。

紫氣日行一百二十六秒,小餘七二0七七七。

紫氣應七宮十七度五十分十四秒五十三微。

推紫氣法,求紫氣行,與日躔求平行法同。

求宿度,與太陽同。


\end{pinyinscope}