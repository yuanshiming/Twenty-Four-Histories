\article{志八十九}

\begin{pinyinscope}
○職官一

太祖肇基東土,國俗淳壹,事簡職專,置八旗總管大臣、佐管大臣董統軍旅,置議政五大臣、理事十大臣釐治政刑,任用者止親貴數臣,官稱職立,人稱官置,興也勃焉。太宗厲精為治,設三館,置八承政,論功料勤,翕斯郅治。世祖入關,因明遺制,內自閣、部以迄庶司,損益有物。籓部創建,名並七卿,外臺督撫,杜其紛更,著為令甲。綠營提鎮以下,悉易差遣為官,旗營御前領衛,年宿位重,意任隆密。都統旗長,軍民合治,職視專圻駐防,分翰外畿,規撫京制。西北邊陲,守以重臣,綏靖蒙、番,方軌都護,斯皆因俗而治,得其宜已。世宗綜覈,罷尚寶、行人、僉都諸目。高宗明哲,損參政、參議、副使、僉事諸銜,沙汰虛冗,奉職肅然。嘉、道以降,整釐如舊。日久頹弛,精意浸失,日革月易,百職相侵。光緒變法,宣統議制,品目張皇,掌寄紛雜,將以靖國,不益囂乎!夫一國事權,操自樞垣,匯於六曹,分寄於疆吏。自改內三院為內閣,臺輔拱袂。迨軍機設,題本廢,內閣益類閒曹,六部長官數四,各無專事。甚或朝握銓衡,夕兼支計,甫主戎政,復領容臺,一職數官,一官數職,曲存稟仰,建樹寧論。時軍機之權,獨峙於其上,國家興大兵役,特簡經略大臣、參贊大臣,親寄軍要。吏部助之用人,戶部協以巨饟,用能藉此雄職,奏厥膚功。自是權復移於經略,督撫儀品雖與相埒,然不過承號令、備策應而已。厥後海疆釁起,經略才望稍爽,權力漸微。粵難糾紛,首相督師,屢僨厥事。朝廷間用督撫董戎,多不辱命,猶復不制以文法,故能霈施魄力,自是權又移於督撫。同治中興,光緒還都,皆其力也。洎乎末造,親貴用事,權削四旁,厚集中央,疆事遂致不支焉。初制內外群僚,滿、漢參用,蒙古、漢軍,次第分布。康、雍兩朝,西北督撫,權定滿缺,領隊、辦事大臣,專任滿員,累朝膺閫外重寄者,滿臣為多。逮文宗兼用漢人,勛業遂著。大抵中葉以前,開疆拓宇,功多成於滿人。中葉以後,撥劇整亂,功多成於漢人。季世釐定官制,始未嘗不欲混齊畛域,以固厥根本也。而弊風相仍,一物自為鴻乙,徒致疑駴,雖危亡之政,無關典要,亦必輯而列之,以著一時故實,治亂之跡,庶皎然若覽焉。

宗人府師傅保內閣稽查欽奉上諭事件處中書科軍機處內繙書房方略館

吏部戶部三庫倉場關稅各差禮部會同四譯館樂部兵部刑部工部火藥局河道溝渠盛京五部

宗人府宗令,左、右宗正,左、右宗人,俱各一人。宗室王、公為之。府丞,漢一人。正三品。其屬:堂主事,漢主事,經歷司經歷,並正六品。左、右二司理事官,正五品。副理事官,從五品。主事,委署主事,俱各二人;筆帖式,效力筆帖式,各二十有四人。俱宗室為之。

宗令掌皇族屬籍,顯祖宣皇帝本支為宗室,系金黃帶。旁支曰覺羅,系紅帶。革字者,系紫帶。以時修輯玉牒,奠昭穆,序爵祿,宗室封爵十有二:曰和碩親王,曰多羅郡王,曰多羅貝勒,曰固山貝子,曰奉恩鎮國公,曰奉恩輔國公,曰不入八分鎮國公,曰不入八分輔國公,曰鎮國將軍,曰輔國將軍,曰奉國將軍,曰奉恩將軍。嫡子受封者二等:曰世子,曰長子。福晉、夫人之號,各視夫爵以為差。公主之等二:曰固倫公主,曰和碩公主。格格之等五:曰郡主,曰縣主,曰郡君,曰縣君,曰鄉君。不入五等曰宗女。額駙品級,各視公主、格格等級以為差。麗派別,申教誡,議賞罰,承陵廟祀事。宗正、宗人佐之。府丞掌校漢文冊籍。左、右二司分掌左、右翼宗室、覺羅譜牒,序錄子女嫡庶、生卒、婚嫁,官爵、名謚;並覈承襲次序,秩俸等差,及養給優恤諸事。堂主事掌清文奏稿。漢主事掌漢文典籍。經歷掌出納文移。筆帖式掌繙譯文書。各部同。筆帖式為滿員進身之階。國初,大學士達海、額爾德尼、索尼諸人,並起家武臣,以諳練國書,特恩賜號「巴克什」,即後之筆帖式也。厥後各署候補者紛不可紀矣。其兼領者:左、右翼宗學,總理學務王二人,稽察京堂官三人,並請旨簡派。總管四人,食七品俸。副管十有六人,食八品俸。並以宗室中分尊年長者引見補授。清書教習、騎射教習各六人,漢書教習八人。所轄銀庫,以本府堂官及滿洲大臣各一人領之,請旨簡派。司官二人,由府引見補授。筆帖式四人。空房,司官、筆帖式亦如之。黃檔房,司官、筆帖式無員限。

初制,列署篤恭殿前,置八和碩貝勒共議國政,各置官屬。順治九年,設宗人府,置宗令一人;親王、郡王為之。左、右宗正,貝勒、貝子兼攝。宗人,鎮國公、輔國公及將軍兼攝。後擇賢,不以爵限。俱各二人。啟心郎,覺羅一人,漢軍二人,初制,秩視理事官。九年,改視侍郎。始以滿臣不諳漢語,議事令坐其中。後多緣以為奸,康熙十二年省。與府丞並為正官。其郎中六人,康熙三十八年省二人。員外郎四人,主事三人,以覺羅為之,嗣改覺羅、滿洲參用。堂主事二人,經歷三人,宗室、滿洲二人,漢一人。康熙三十八年省漢缺。乾隆二十九年改用宗室。筆帖式二十有四人。後增★無恆。初為他赤哈哈番、筆帖式哈番,尋改六、七、八品及無頂戴筆帖式。各部同。康熙十二年,省啟心郎,增滿洲主事一人,分隸左、右二司。雍正元年,增漢主事二人。用進士出身者。明年,改郎中為理事官,員外郎為副理事官,並定為宗室、滿洲參用。乾隆二十九年,允府丞儲麟趾奏,始專用宗室人員。五十三年,增置委署主事四人。筆帖式改。

太師、太傅、太保為三公。正一品。少師、少傅、少保為三孤。從一品。太子太師、太子太傅、太子太保,從一品。太子少師、太子少傅、太子少保,正二品。俱東宮大臣,無員限,無專授。

初沿明制,大臣有授公、孤者。嗣定為兼官、加官及贈官。

內閣大學士,滿、漢各二人。初制,滿員一品,漢員二品。順治十五年,改與漢同。雍正八年,並定正一品。協辦大學士,滿、漢各一人。尚書內特簡。正一品。學士,漢洲六人,漢四人。初制,滿員二品,漢員三品。順治十五年,並改正五品,兼禮部侍郎者正三品。雍正八年,定從二品。後皆兼禮部侍郎銜。典籍典籍,滿、漢、漢軍各二人。正七品。侍讀學士,滿洲四人,蒙、漢各二人。初兼太常寺卿銜,尋罷。雍正三年,定從四品。中書,正七品。滿洲七十人,蒙古十有六人,漢軍八人。貼寫中書,滿洲四十人,蒙古六人。

大學士掌鈞國政,贊詔命,釐憲典,議大禮、大政,裁酌可否入告。協辦佐之。修實錄、史、志,充監修總裁官。經筵領講官。會試充考試官。殿試充讀卷官。春秋釋奠,攝行祭事。學士掌敷奏。侍讀學士掌典校。侍讀掌勘對。典籍掌出納文移。內閣為典掌絲綸之地,自大學士以下,皆不置印,惟典籍置之,以鈐往來文牒。中書掌撰擬、繙譯。分辦本章處凡五:曰滿本房,漢本房,蒙古本房,滿簽票處,漢簽票處。又誥敕房,稽察房,收發紅本副本處,飯銀庫,俱由大學士委侍讀以下官司之。惟批本處額置滿洲翰林官一人,請旨簡派。中書七人。滿中書內補授。

初,天聰二年,建文館,命儒臣分直。十年,更名內三院。曰國史,曰秘書,曰弘文。始亦沿承政名,後各置大學士一人。順治元年,置滿、漢大學士,不備官,兼各部尚書銜。學士,滿洲、康熙九年改置二人,十年增四人,通舊為六人。漢軍康熙十年改置二人,十二年並入漢缺。各三人,漢學士無員限。康熙十年改置二人,明年增二人,十二年省漢軍入漢缺,通舊為四人。典籍,滿、漢、漢軍各三人。康熙九年改置二人。侍讀,滿洲十有一人,清文五人,清漢文六人。康熙三十八年省清文一人,清漢文二人。尋復增二人,通舊為十人。蒙古、漢軍、康熙九年各置二人。漢康熙九年省。雍正四年置二人。各三人。中書,滿洲七十有五人,蒙古十有九人,漢軍十有三人,漢三十有六人。康熙三十八年省滿洲、漢軍各五人,蒙古三人,漢四人。乾隆十三年復省漢三人。二年,定為正二品衙門,以翰林官分隸之。三院上並系「內翰林」字。八年,置侍讀學士,滿、蒙、漢軍各三人。十八年增滿洲二人,蒙古三人。康熙九年增滿洲四人,餘改置二人。乾隆十七年省漢軍入漢缺。十年,置三院漢大學士各二人。十五年,更名內閣,別置翰林院官,以大學士分兼。殿閣曰中和殿、保和殿、文華殿、武英殿、文淵閣、東閣,諸大學士仍兼尚書,學士亦如之。十八年,復三院舊制。康熙九年,仍別置翰林院,改三院為內閣,置滿、漢大學士四人。雍正九年,禮部尚書陳元龍、左都御史尹泰特授額外大學士。置協辦自此始。厥後多至六人,少或一二人。乾隆十三年,始定大學士、協辦大學士員限,省中和殿,增體仁閣,以三殿、三閣為定制,唯保和殿不常置。嗣後授保和者止傅恆一人。凡遇歲時慶節朝會,漢員列滿員下。自光緒間李鴻章系文華殿銜,而寶鋆時系武英殿,班轉居其右。五十八年,停兼尚書銜。宣統三年,改組內閣,別令大學士序次翰林院。

先是世祖親政,日至票本房,大學士司票擬,意任隆密。康熙時,改內閣,分其職設翰林院。雍正時,青海告警,復分其職設軍機處,議者謂與內三院無異。顧南書房翰林雖典內廷書詔,而軍國機要綜歸內閣,猶為重寄。至本章歸內閣,大政由樞臣承旨,權任漸輕矣。

稽察欽奉上諭事件處,兼理大臣無員限。滿、漢大學士、尚書、左都御史內特簡。掌察諸司諭旨特交事件,督以例限。委署主事,滿洲一人。行走司官,漢四人。並於吏、兵、刑、工四部選補。筆帖式四十人。額外筆帖式八人。

中書科,稽察科事內閣學士,滿、漢各一人,由內閣學士內特簡。掌稽頒冊軸。掌印中書,滿洲一人。掌科中書,漢一人。中書,並從七品。滿洲一人,漢三人,掌繕書誥敕。筆帖式十人。

初制,置滿洲中書舍人一人,乾隆十四年增一人。漢中書舍人八人。雍正十三年派兼內閣行走。乾隆十三年省四人。順治九年,置滿洲記事官,同掌科事。康熙九年,改記事官為中書舍人。乾隆三十六年,置管中書科事漢內閣學士一人。明年,改管科事為稽察科事;增置滿洲內閣學士一人;改中書舍人為中書科,置掌印中書,滿、漢各一人。宣統三年省。

軍機處軍機大臣,無定員,由大學士、尚書、侍郎內特旨召入。區其名曰大臣,曰大臣上行走。其初入者加「學習」二字。掌軍國大政,以贊機務。常日侍直,應對獻替,巡幸亦如之。明降諭旨,述交內閣。諭本處行者,封寄所司。並冊藏存記人員,屆時題奏。其屬曰章京,滿洲十有六人,漢二十人,名曰行走,分頭班、二班。初無定額,嘉慶四年定每班八人。後增★無恆。光緒三十二年定三十有六人,復定領班秩視三品,幫領班秩視四品,餘並以原官充補。三十四年,改領班為從三品,幫領班為從四品。分掌清文、漢字。

初設議政處,令鞏阿岱等為議政大臣,參畫軍要。雍正十年,用兵西北,慮儤直者洩機密,始設軍機房,後改軍機處,而滿洲大學士尚有兼議政銜者。乾隆五十六年停。高宗涖政,更名總理處,尋復如初。時入直者皆重臣。故事,親王不假事權。至嘉慶四年,始命成親王入直,旋出之。咸豐間,復命恭親王入直,歷三朝領班如故。嗣是醇賢親王、禮親王、慶親王等踵相躡。光緒二十七年,設政務處,以軍機大臣領督辦事。參預大臣無定員。提調、幫提調、總辦、幫總辦,俱各二人,章京八人,並以本處員司兼充。二十八年,附設財政處,尋罷。三十二年更名會議政務處,隸內閣。宣統三年省。三十一年,定署名制。越二年,設憲政編查館,復命軍機大臣領之。先是設考察政治館,命度支部尚書載澤等考察各國政治,至是更名。置提調四;總核、參議各二;庶務處總辦一;一、二等諮議官,無恆額。設編制、統計、官報三局,局長、副局長各一,科員視事酌置。又考核科總辦一,幫辦正科員各二,副科員八,調京、外官兼充。宣統三年省。宣統三年,改責任內閣,以軍機大臣為總協理大臣。

內繙書房管理大臣,滿洲軍機大臣兼充,掌繙諭旨、御論、冊祝文字。提調、協辦提調,各二人。收掌官、掌檔官,俱各四人。並於本房行走官內酌派。繙譯四十人。宣統初,改隸翰林院。

方略館總裁,軍機大臣兼充。掌修方略。提調、收掌,俱滿、漢二人。纂修,滿洲三人,漢六人。俱由軍機章京內派充。漢纂修缺內由翰林院咨送充補一人。校對,無員限。六部司員、內閣中書兼充。有事權置,畢乃省。

吏部尚書,初制,滿洲一品,漢人二品。順治十六年改滿尚書二品。康熙六年復故,九年仍改正二品。雍正八年俱定從一品。各部同。左、右侍郎,初制,滿洲、漢軍二品,漢員三品。順治十六年改滿侍郎三品。康熙六年復故,九年仍改正三品。雍正八年俱定從二品。各部同。俱滿、漢一人。其屬:堂主事,初制四品。順治十六年改六品。康熙六年升五品,九年定正六品。各部同。清檔房滿洲二人,漢本房滿洲二人,漢軍一人。司務司務,初制從九品。乾隆三十年定正八品。各部同。滿、漢各一人。繕本筆帖式,十有二人。文選、考功、驗封、稽勛四清吏司:郎中,初制三品。順治十六年改五品,尋升四品。康熙六年仍改三品,九年定正五品。各部同。滿洲九人,文選四人,考功三人,驗封、稽勛司各一人。蒙古一人,文選司置。漢五人。文選二人,餘各一人。員外郎,初制四品。順治十六年改五品。康熙六年復故,九年定從五品。各部同。宗室一人,稽勛司置。滿洲八人,文選三人,考功、驗封各二人,稽勛一人。蒙古一人,考功司置。漢六人。文選三人,餘各一人。主事,宗室一人,稽勛司置。滿洲四人,司各一人。蒙古一人,驗封司置。漢七人。文選三人,考功二人,餘各一人。筆帖式,宗室一人,滿洲五十有七人,蒙古四人,漢軍十有二人。學習行走者,有額外司員、七品小京官。各部同。

尚書掌銓綜衡軸,以布邦職。侍郎貳之。堂主事掌文案章奏。司務掌出納文移。以上二員各部同。文選掌班秩遷除,平均銓法。官分九品,各系正從,級十有八,不及九品曰未入流。選人並登資簿,依流平進,踵故牒序遷之。考功掌考課,三載考績。京察、大計各聽察於長官,著跡計簿。凡論劾、釋免、引年、稱疾,並覈功過處分。交議者,辨公私輕重,條議以聞。稽勛掌勛級、名籍、喪養,兼稽京朝官廩祿,稽俸隸之。漢司官員數,八旗世職繼襲。驗封掌廕敘、正一品子正五品敘,從一品子從五品敘,其下以是為差。封贈、階十有八:正一品授光祿大夫,從一品授榮祿大夫,正二品授資政大夫,從二品授奉政大夫,正三品授通議大夫,從三品授中議大夫,正四品授中憲大夫,從四品授朝議大夫,正五品授奉政大夫,從五品授奉直大夫,俱授誥命。正六品授承德郎,從六品授儒林郎,吏員出身者宣德郎,正七品授文林郎,吏員出身者宣義郎,從七品授徵仕郎,正八品授修職郎,從八品授修職佐郎,正九品授登仕郎,從九品授登仕佐郎,俱授敕命。命婦之號九:一曰一品夫人,二品亦曰夫人,三品曰淑人,四品曰恭人,五品曰宜人,六品曰安人,七品曰孺人,八品曰八品孺人,九品曰九品孺人,不分正從。因其子孫封者加「太」字,夫在則否。一品封贈三代,二、三品二代,四品至七品一代,以下止封本身。一品四軸用玉,二品三軸用犀,三品三軸、四品二軸用抹金,五品以下二軸用角。凡嫡母在,生母不得並封。又兩子當封,從其品大者。酬庸、獎忠。覈贈、廕死難官員,有贈、有廕。當否。襲封則辨分合,別宗支等。其世流降除,勘土官世職,移文選司注擬。推恩外戚,加榮聖裔,優恤勝國,並按典奏聞。別設督催所,趣各司交議事,督以例限。當月處,主受事、付事,兼監堂印。遴司員分司之。各部同。

初,天聰五年,詔群僚議定官制,建六部,各以貝勒一人領之。順治元年罷。八年復以親王、郡王兼攝,九年罷。置承政四人,滿二人,蒙、漢各一人。唯工部滿一人,漢二人。參政八人,唯工部置蒙、漢各二人。共十有二人。啟心郎一人。工部置漢二人。順治九年定秩視侍郎。崇德三年,六部定承政一人,左參政二人,右參政三人,戶部四人。啟心郎三人,滿一人,漢二人。理事官四十有三人,吏、禮二部各四人,戶、兵二部各十人,刑部六人,工部九人。副理事官六十有五人,吏部六人,戶、兵二部各十有六人,禮部七人,刑部八人,工部十有二人。額哲庫二人。

順治元年,改承政為尚書,參政為侍郎,理事官為郎中,副理事官為員外郎,額哲庫為主事。初置增減無恆。時滿洲尚書,滿、漢左、右侍郎,亦無員限。漢右侍郎兼翰林院學士銜。非翰林出身者不兼。尋罷。本部郎中,滿洲四人,十二年增四人。光緒十三年增文選一人。漢軍二人,雍正五年省。滿、蒙員外郎八人,十二年省蒙古缺。十八年復置蒙古八人,康熙元年省,五十七年復置一人。漢軍六人。康熙三十八年省四人。雍正五年並省。滿洲堂主事、清文、清漢文各二人。司主事光緒十三年增文選一人。各四人,漢軍一人,漢司務二人。四年省一人。十五年定滿、漢各一人。各部同。文選司,漢郎中、員外郎各一人,雍正五年增員外郎一人。光緒十三年各增一人。主事二人。光緒十三年增一人。考功、稽勛、驗封三司,漢郎中、員外郎、主事各一人。雍正五年增考功主事一人。並置筆帖式,分隸堂司。各部同。五年,定滿、漢尚書各一人。七年增滿洲一人,十年省。十五年,省啟心郎,定滿、漢左、右侍郎各一人。康熙五十七年,增置蒙古郎中、主事各一人。雍正元年,以大學士領部事。嘉慶四年,更命親王綜之,尋罷。改滿洲員外郎、主事各一人為宗室員缺。六年,復以大學士管部,自是為定制。光緒二十三年,澄汰書吏,增文選、考功二司郎中、員外郎、主事各一人。滿、漢參用。三十二年,定尚書,左、右侍郎,左、右丞、參各一人。丞、參品秩,詳新官制外務部。

初制,滿、蒙、漢軍司官,六部統為員額,不置專曹,後始分司定秩如漢人。季世詔泯滿、漢畛域,各部復參用矣。吏部班次曩居六部上。各司郎官,非科甲出身者,不得注授。禮部、宗人府、起居注主事同。自外務部設,班位稍爽,改組內閣,設銓敘、制誥等局,吏部遂廢。

戶部尚書,左、右侍郎,俱滿、漢一人。其屬:堂主事,南檔房滿洲二人,北檔房滿洲、漢軍各二人。司務司務,滿、漢各一人。繕本筆帖式二十人。江南、江西、浙江、湖廣、福建、山東、山西、河南、陜西、四川、廣東、廣西、雲南、貴州十四清吏司:郎中,宗室一人,江西司置。滿洲十有七人,江南、浙江、河南、山西、陜西、四川、廣東、廣西、貴州司各一人,福建、湖廣、山東、雲南司各二人。蒙古一人,山西司置。漢十有四人。司各一人。員外郎,宗室二人,廣東、廣西司置。滿洲三十有六人,山西司一人,浙江、江西、河南、四川、廣東、湖廣司各二人,江南、陜西、廣西、山東、雲南、貴州司各三人,福建司五人。漢十有四人。主事,宗室一人,浙江司置。蒙古一人,福建司置。滿、漢各十有四人。筆帖式,宗室一人,滿洲百人,蒙古四人,漢軍十有六人。

尚書掌軍國支計,以足邦用。侍郎貳之。右侍郎兼掌寶泉局鼓鑄。十四司,各掌其分省民賦,及八旗諸司稟祿,軍士饟糈,各倉,鹽課,鈔關,雜稅。江南司兼稽江寧、蘇州織造支銷,江寧、京口駐防俸餉,各省平餘地丁逾限未結者。江西司兼稽各省協餉。浙江司兼稽杭州織造支銷,杭州、乍浦駐防俸餉,及各省民數、穀數。福建司兼稽直隸民賦,天津海稅,東西陵、熱河、密雲駐防俸餉,司乳牛牧馬政令,文武鄉會試支供,五城賑粟。湖廣司兼稽奉省廠課,荊州駐防俸餉,各省地丁耗羨之數。河南司兼稽開封駐防俸餉,察哈爾俸餉,及報銷未結者。山東司兼稽青州、德州駐防俸餉,東三省兵糈出納,參票畜稅,並察給八旗官養廉,長蘆等處鹽課。山西司兼稽游牧察哈爾地畝,土默特地糧,喀爾喀、回部定邊左副將軍辦事官屬,張家口、賽爾烏蘇臺站俸餉,烏里雅蘇臺、科布多屯田官兵番換,並各省歲入歲出之數。陜西司兼稽甘肅民賦,行銷鹽引,西安、寧夏、涼州、莊浪各駐防俸餉,並匯覈在京支款,新疆經費。四川司兼稽本省關稅,兩金川等處、新疆屯務,成都駐防俸餉,並京城草廠出納,各部院紙硃支費,入官戶口,贓銀兩,凡各省郡縣豐歉水旱,歲具其數以上。廣東司兼稽廣州駐防俸餉,八旗繼嗣戶產更代,凡壽民、孝子、節婦受旌者,給以坊直。廣西司兼稽本省礦政廠稅,及京省錢法,內倉出納。雲南司兼稽本省廠課,山東、河南、江南、江西、浙江、湖廣漕政,京、通倉儲,及江寧水次六倉考覈。貴州司兼稽各關稅課,並覈貂貢。所轄內倉監督,滿洲二人。司員內派委。寶泉局監督、各部司員內保送補用。主事,本部司員內派委。俱滿、漢各一人;局大使,東、西、南、北四廠大使,俱滿洲一人。筆帖式充。初置漢一人。雍正四年增四人,七年改滿洲員缺。各省錢局監鑄官,十有八人。外官兼充,並受法式法部。其別領者三:曰井田科,典八旗土田、內府莊戶;曰俸饟處,覈八旗俸饟丁冊;曰現審處,平八旗戶口田房諍訟。又飯銀處、減平處、捐納房、監印處、則例館,俱派司屬分治其事。

初,天聰五年,設戶部。順治元年,置尚書、侍郎。右侍郎管錢法堂事。郎中,滿洲十有八人,蒙古四人,康熙三十八年省。五十七年復置一人。漢軍二人。康熙三十八年省。員外郎,滿洲三十有八人,蒙古五人,康熙三十八年省,五十七年復置一人。漢軍六人。康熙三十八年省。滿洲堂主事四人,主事十有四人,漢軍堂主事二人。十四司,漢郎中、員外郎各一人,主事各三人。六年,司各增一人。十一年省增額。康熙六年省江南、浙江、江西、湖廣、福建、河南、陜西、廣西、四川、貴州各一人。三十八年省山東、山西、廣東、雲南各一人。五年,定滿、漢尚書各一人。七年增滿洲一人,十年省。康熙六年復置,八年又省。康熙五十七年,增置福建司蒙古主事一人。雍正初,始令親王、大學士領部事。嘉慶四年,以川省用兵,銷算務劇,復令親王永瑆綜之。尋罷。並改滿洲郎中一人、員外郎二人為宗室員缺。十一年,仍令大學士管部。光緒六年,增浙江司宗室主事一人。三十二年,更名度支部。初制,按省分職,十三司外,增設江南一司,凡銅、關、鹽、漕,及續建行省,別以司之事簡領之。

管理三庫大臣,滿、漢各一人,三年請旨更派。掌庫藏出納,月會歲要,覈實以聞。其屬:檔房主事一人,銀、緞疋、顏料三庫郎中各一人,員外郎各二人,司庫五人,正七品。銀庫一人,餘各二人。大使四人,銀庫二人,餘各一人。各部司員內補授。筆帖式四人,庫使十有一人。未入流。以上俱為滿缺。

順治初,設後庫,在部署。置郎中四人,員外郎二人。康熙二十九年定三庫俱各一人。雍正二年增員外郎各一人。十三年,分建三庫,改後庫為銀庫。緞疋庫在東華門外,即舊裏新庫。顏料庫在西安門內,即舊甲字庫。置理事官綜其事。雍正元年,改命王公大臣領之。明年,置大使各一人,乾隆三年增銀庫一人。並增主事一人,稽覈檔案。光緒二十八年省。

總督倉場侍郎,滿、漢各一人,分駐通州新城。掌倉穀委積,北河運務。其屬:筆帖式四人。所轄坐糧,滿、漢各一人,滿員由六部、理籓院郎員,漢員由六部郎員內簡用。掌轉運輸倉,及通濟庫出納。大通橋監督,滿、漢各一人,十一倉監督內補用。掌轉大通陸運。十一倉監督,曰祿米、曰南新、曰舊太、曰富新、曰興平、曰海運、曰北新、曰太平,俱清初建。曰本裕,康熙四十五年建。曰儲濟,雍正六年建。曰豐益,七年建。舊有萬安、裕豐,後省。其恩豐倉,乾隆二十六年建,隸內府。俱滿、漢各一人,各部院保送補用。掌分管京倉。中、西二倉監督,沿明制建。舊有南倉,後省。滿、漢各一人,十一倉監督內調補。掌分管通倉。

順治元年,置漢侍郎一人。康熙八年省,十八年復。京、通各倉,戶部員司分理之。通州坐糧,十二年設京糧。十五年並入大通橋。康熙二年置滿、漢監督各一人,尋省。四十七年復。以戶部官一人承其事。九年,置滿洲、漢軍侍郎各一人。尋省漢軍缺。十五年,定滿、漢各一人。康熙五十年,定京、通倉監督滿、漢各一人。雍正二年置副監督,尋省。其缺由內閣中書、部院監寺官番選。又初有總理,滿洲侍郎一人,與總漕並理漕務。順治八年省,十二年復,十八年又省。

京師崇文門,正監督、副監督,左翼、右翼各一人。內府大臣及尚書侍郎兼充。其各常關,或部臣題請特簡,或由京掣差部司官,或改令外官兼轄。天津關,長蘆鹽政兼管。通州,坐糧兼管。張家口、殺虎口,部院司員兼充。潘桃口,多倫諾爾同知兼理。龍泉、紫荊、喜峰、五虎、固關、白石、倒馬、茨溝、插箭嶺、馬水口,提督兼管,委參將、都司、守備、把總監收。三座塔、八溝、烏蘭哈達,理籓院司員兼充。奉天牛馬稅,部院司員兼充。中江,盛京將軍衙門章京及五部司員番選,後歸興鳳道兼理。臨清,巡撫兼管,委知州監收。歸化城,巡撫兼管,委道員監收。潼關,道員兼理。滸墅關,蘇州織造監理。淮安關兼廟灣口,內府司員兼充。揚關,巡撫兼管,委淮揚海道兼收。西新關,江寧織造兼理,後改歸巡撫。鳳陽關,皖北道兼理。贛關,巡撫兼管,委吉南贛寧道監收。閩安關,巡撫兼管,後改歸總督,委福州府同知監收。北新關,杭州織造兼管,後改歸巡撫。武昌廠、荊關,巡撫兼管,後改歸總督委員監收。夔關,總督兼管,委知府監收。打箭爐,同知兼理。太平關,巡撫兼管,委南韶連道監收。梧廠、潯廠,巡撫兼管,委梧、潯二知府監收。

初制,榷百貨者曰戶關,榷竹木船鈔者曰工關,為戶、工二部分司,後改今制。宣統三年,工關多改稱常關,唯直隸等省名稱如故。並隸度支部。往例以內府官簡充。乾隆間,改令內務府大臣為之。後部院大臣並得簡充,定為滿洲員缺。

禮部尚書,左、右侍郎,俱滿、漢一人。其屬:堂主事,清檔房滿洲二人,漢本房滿洲、漢軍各一人。司務司務,滿、漢各一人。筆帖式,宗室一人,滿洲三十有四人,蒙古二人,漢軍四人。典制、祠祭、主客、精膳四清吏司:郎中,滿洲六人,典制、祠祭,各二人,餘俱一人。蒙古一人,主客司置。漢四人。司各一人。員外郎,宗室一人,主客司置。滿洲八人,典制、祠祭司各三人。餘俱一人。蒙古一人,祠祭司置。漢二人。典制、祠祭司各一人。主事,宗室、蒙古各一人,精膳司置。滿洲三人,典制、祠祭、精膳司各一人。漢四人。司各一人。印鑄局,漢員外郎、滿洲署主事、漢大使,未入流。各一人。堂子尉,滿洲八人。七品二人,八品六人。

尚書掌五禮秩敘,典領學校貢舉,以布邦教。侍郎貳之。典制掌嘉禮、軍禮。稽彞章,辨名數,頒式諸司。三歲大比,司其名籍。四方忠孝貞義,訪懋旌閭。祠祭掌吉禮、兇禮。凡大祀、中祀、群祀,以歲時辨其序事與其用等。日月交食,內外諸司救護;有災異即奏聞。凡喪葬、祭祀,貴賤有等,皆定程式而頒行之。勛戚、文武大臣請葬祭、贈謚,必移所司覈行。並籍領史祝、醫巫、音樂、僧道,司其禁令,有妖妄者罪無赦。主客掌賓禮。凡蕃使朝貢,館餼賜予,辨其貢道遠邇、貢使多寡、貢物豐約以定。頒實錄、玉牒告成褒賞。稽霍茶歲額。精膳掌五禮燕饗與其牲牷。賜百官禮食,視品秩以為差,光祿供膳羞,會計其數而程其出納,匯覈各司。鑄印局題銷鑄印,掌鑄寶璽,凡內外諸司印信,並範冶之。用銀質直鈕三臺者:宗人府、衍聖公,清、漢文尚方大篆,方三寸三分,厚一寸;六部、戶部鹽茶、都察院、行在部院,清、漢、蒙三體字,清、漢文尚方大篆,蒙文不篆,方三寸三分,厚九分。直鈕二臺者:盛京五部、戶部三庫,清、漢文尚方大篆,方三寸三分,厚八分;軍機處、內務府、盛京內務府、翰林院、鑾輿衛,清、漢文尚方大篆,方三寸二分,厚八分。虎鈕三臺者:提督、總兵。虎鈕二臺者:侯、伯、領侍衛內大臣、都統、前鋒統領、護軍統領、步軍統領、總管火器營神機營、圓明園總營八旗包衣三旗官兵、經略大臣、大將軍、鎮守將軍、科布多參贊大臣、鎮守掛印總兵,清、漢文柳葉篆;西寧辦事大臣、駐藏辦事大臣,清、漢、回三體字;伊犁將軍,清、漢、回、托忒四體字;定邊參贊大臣,清、漢、托忒三體字,清、漢文柳葉篆,塔爾巴哈臺參贊大臣,清文、托忒二體字,清文柳葉篆;庫倫辦事大臣,清、漢、蒙三體字,清、漢文柳葉篆;外籓扎薩克各盟長,清、蒙二體字,不篆,並方三寸三分,厚九分;鄉導總領、駐防副都統,清、漢文柳葉篆,方三寸二分,厚八分。直鈕者:布政使司,清、漢文小篆,方三寸一分,厚八分;通政使司、大理寺、太常寺、順天府、奉天府,清、漢文小篆,方二寸九分,厚六分五釐。用銅質直鈕者:詹事府、按察使司,清、漢文小篆;額魯特總管,清、漢、蒙三體字,清文殳篆;宣慰使司、指揮使司,清、漢文殳篆,並方一寸七分,厚九分;光祿寺、太僕寺、武備院、上駟院、奉宸苑,清、漢文小篆;鹽運使司,清、漢文鐘鼎篆;旗手衛、城守尉,清、漢文殳篆;衛守備,清、漢文懸針篆;察哈爾總管,清、蒙二體字,清文殳篆,並方二寸六分,厚六分五釐;府,清、漢文垂露篆,方二寸五分,厚六分;宗人府左右司、左右春坊、司經局、六部理籓院各司、鑾輿衛各所、欽天監、太醫院、盛京五部各司,清、漢文鐘鼎篆;宗人府經歷、鹽課提舉司,清、漢文垂露篆,並方二寸四分,厚五分;宣撫使司副使、安撫使司、領運千總,清、漢文懸針篆;方二寸四分,厚五分五釐;州,清、漢文垂露篆,方二寸三分,厚四分五釐;土千戶,清、漢文懸針篆,方厚如州;內務府各司、鑾輿衛馴象等所,清、漢文鐘鼎篆;吏戶二部稽俸、都察院經歷、大理寺太僕寺左右司、光祿寺四署,樂部和聲署、五城兵馬司、大興宛平二縣、盛京承德縣、布政使司經歷、理問,清、漢文垂露篆;旗手衛左右司、九姓長官司、指揮僉事,清、漢文懸針篆,並方二寸二分,厚四分五釐;六科、欽天監時憲書,清、漢文鐘鼎篆;中書科太常寺光祿寺典簿、詹事府太僕寺主簿、部寺司務、懸鑾輿衛通政使司按察使司鹽運使司各衛宣慰使司諸經歷,並方二寸一分,厚四分四釐;國子監三、鴻臚寺欽天監各主簿、京府儒學、壇廟祠祭署、布政使司照磨、府經歷,清、漢文垂露篆,方二寸,厚四分二釐;刑部司獄、國子監典簿、神樂觀犧牲所、光祿寺銀庫、太醫院藥庫、寶泉寶源二局,清、漢文垂露篆,方一寸九分,厚四分二釐;京府照磨、司獄、布政使司司庫、按察使司照磨、司獄、府照磨、司獄、庫大使、府衛儒學、巡檢司、都稅司、稅課司、茶馬司,清、漢文垂露篆,並方一寸九分,厚四分。直鈕有孔者:監察御史、稽察宗人府內務府御史,清、漢文鐘鼎篆,方一寸五分,厚三分。喇嘛、呼圖克圖,或金質,或銀質,扎薩克大喇嘛,銅質,並云鈕,用清文、蒙古、唐古忒三體字,不篆,或清、漢文轉宿篆、正一真人,銅質直鈕,清、漢文鐘鼎篆,方二寸六分,厚六分五釐。僧錄司、道錄司,銅質直鈕,清、漢文垂露篆,方二寸二分,厚四分五釐。餘用關防或圖記、條記也。別設書籍庫、板片庫、南庫、養廉處、地租處,俱遴員司分治其事。

天聰五年,設禮部。順治元年,置尚書、侍郎各官。十五年省漢軍侍郎。郎中,滿洲四人,十八年增二人。員外郎六人,十二年增四人。堂主事二人,司主事四人;蒙古章京二人;康熙九年改郎中、員外郎各一人。三十八年省。漢軍郎中八人,康熙九年省七人。雍正五年俱省。員外郎五人,康熙三十八年省二人。雍正五年俱省。堂主事一人。儀制、祠祭、主客、精膳四司,漢郎中、員外郎、主事各一人。二年省主客、精膳員外郎各一人。滿洲讀祝官六人。九年省四人。康熙十年改隸太常寺。皇史宬尉,正七品。滿洲三人。司牲官,正七品。蒙古二人。鑄印局,滿洲員外郎一人,以上三員尋省。漢大使一人。五年,定滿、漢尚書各一人。康熙五十七年,增置蒙古郎中、主客司。員外郎、祠祭司。主事精膳司。各一人。雍正元年,以親王、郡王、大學士領部事,隨時簡任,不為常目。乾隆三年,增置鑄印局漢員外郎、筆帖式、署主事各一人。十三年,省行人司入之。嘉慶四年,改滿洲員外郎、主事各一人為宗室員缺。光緒二十四年,省光祿、鴻臚二寺入之,尋復故。三十一年,詔罷科舉,各省學政改隸學務大臣,自是釐正士風之責,不屬本部矣。三十二年,以光祿、太常、鴻臚三寺同為執禮官,仍省入。更精膳司曰光祿,主客司曰太常,並各置郎中、員外郎、主事各一人。鴻臚事稍簡,歸入典制司,增員外郎一人,並滿、漢參用。是歲定尚書,侍郎,左、右丞、參員額如吏部。設禮器庫,置郎中、員外郎各一人,贊禮官、讀祝官亦如之。俱六品。太常寺丞改充。簿正、光祿寺署正改充。典簿太常寺博士改充者三人,光祿寺典簿改充者一人。各四人,司庫二人,太常、光祿兩寺司庫改充。以上品秩俱如舊。筆帖式十有四人。三寺內揀選酌留。宣統元年,避帝諱,改儀制司曰典制。

初制,禮部設馬館,置正、副監督各一人。正監督,本部司員充。副監督,理籓院司員充。乾隆二十七年,省入理籓院。又初置滿洲宣表官四人,後減二人,尋並入太常寺。

會同四譯館,滿洲稽察大臣二人,部院司寺堂官內簡派。提督館事兼鴻臚寺少卿一人,禮部郎中內選補。掌治賓客,諭言語。漢大使一人,正九品。正教、序班漢二人,朝鮮通事官八人。六品、七品各二人,八品四人。

順治元年,會同四譯分設二館。會同館隸禮部,以主客司主事滿、漢各一人提督之。四譯館隸翰林院,以太常寺漢少卿一人提督之。分設回回、緬甸、百夷、西番、高昌、西天、八百、暹羅八館,以譯遠方朝貢文字。置序班二十人,十五年定正教、協教各八人。康熙間省至九人,以一人管典務事。乾隆十三年,省典務一人,序班六人,額定二人。朝鮮通事官六人。後增十人。凡六品十人、七品六人。乾隆二十三年省六品四人、七品二人,增八品二人。後俱省。十四年,置員外郎品級通事一人,掌會同館印。尋省。乾隆十三年,省四譯館入禮部,更名會同四譯館,改八館為二,曰西域,曰百夷,以禮部郎中兼鴻臚寺少卿銜一人攝之。光緒二十九年省。

樂部,典樂大臣無員限,禮部滿洲尚書一人兼之。後改各部侍郎、內務府大臣兼理。又滿洲王大臣知樂者,亦曰管理大臣。掌考樂律樂均度數,協之以聲歌,播之以器物。辨祭祀、朝會、燕饗之用,以格幽明,和上下。神樂署,署正一人,正六品。左、右署丞各一人,從八品。協律郎五人,正八品。司樂二十有五人。正九品。凡樂生百八十人、舞生三百人屬之,俱漢員,兼隸太常寺,掌郊廟、祠祭諸樂。和聲署,署正、署丞,俱滿、漢各一人。滿員,內務府郎中、員外郎兼充。漢員,禮部郎中、員外郎兼充。凡供用官三十人,本署八人。禮部筆帖式兼充二人,內務府贊禮郎兼充六人,筆帖式及各項有品級者兼充十有二人,鴻臚寺鳴贊官兼充二人。署史長十有六人,署史百四十有八人屬之,掌殿廷朝會、燕饗諸樂。其宮廷之樂,內務府掌禮司中和樂處典之。鹵簿之樂,鑾輿衛、旗手衛校尉典之。並隸以部。

什傍處,掇爾契達一人,兼三等侍衛。六品銜達、七品銜達各二人。拜唐阿六十人,兼隸侍衛處。掌奏掇爾多密之樂,燕饗列之。

順治元年,置教坊司,奉鑾一人,左、右韶舞,左、右司樂各一人,協同十人。以上並正九品。俳長無定員。未入流。太常寺神樂觀,漢提點一人,正六品。左、右知觀各一人,正八品。漢協律郎五人。康熙三十八年省。雍正元年復故。乾隆二年增三人,九年省六人。嘉慶四年增二人。道光元年增二人。咸豐二年增二人。雍正七年,改教坊司為和樂署,省奉鑾各官。乾隆七年,設樂部,簡典樂大臣領之。置和聲署官,以內府、太常、鴻臚各官兼攝,侍從、待詔為加銜。並詔禁太常樂員習道教,不原改業者削籍。先是依明制,凡樂官祀丞概用道流。明年,改神樂觀為所,知觀為知所。十三年,復改神樂所為署,更提點曰署正,知所曰署丞。

兵部尚書,左、右侍郎,俱滿、漢一人。其屬:堂主事,清檔房滿洲二人,漢本房滿洲二人,漢軍一人。司務司務,滿、漢各一人。繕本筆帖式十有五人。武選、車駕、職方、武庫四清吏司:郎中,宗室一人,車駕司置。滿洲十有一人,武選三人,職方五人,車駕一人,武庫二人。蒙古一人,武選司置。漢五人。職方二人,餘俱一人。員外郎,宗室一人,車駕司置。滿洲九人,武選四人,職方、車駕各二人,武庫一人。蒙古三人,職方、車駕、武庫各一人。漢四人。武選、職方各二人。主事,滿、漢各四人。司各一人。筆帖式,宗室一人,滿洲六十有二人,蒙古、漢軍各八人。

尚書掌釐治戎政,簡覈軍實,以整邦樞。侍郎貳之。武選掌武職選授、品級、階十有八:正一品授建威將軍,公、侯、伯同;從一品授振威將軍;正二品授武顯將軍;從二品授武功將軍;正三品授武義都尉;從三品授武翼都尉;正四品授昭武都尉;從四品授宣武都尉;正五品授武德騎尉;從五品授武德佐騎尉;正六品授武略騎尉;從六品授武略佐騎尉;正七品授武信騎尉;從七品授武信佐騎尉;正八品授奮武校尉;從八品授奮武佐校尉;正九品授修武校尉;從九品授修武佐校尉。高下各如其級。命婦之號視文職。封贈、襲廕,俱同文職。並典營制,暨土司政令。職方掌各省輿圖。綠營官年老三載甄別,五年軍政,敘功覈過,以待賞罰黜陟,並典處分、敘恤、關禁、海禁。車駕掌牧馬政令,以裕戎備。凡置郵曰驛、曰站、曰塘、曰臺、曰所、曰鋪,馳驛者驗郵符,洩匿稽留者論如法。武庫掌兵籍、戎器,鄉會武科,編發、戍軍諸事。有征伐,工部給器仗,籍紀其數。制敕下各邊徵發,或使人出關,必驗勘合。其分攝者,會同館管理館所侍郎一人,本部侍郎簡派。滿、漢監督各一人,司員內補授。典京師驛傳,以待使命。又捷報處司官無定額。駐京提塘官十有六人。直隸、山東、山西、河南、江西、福建、浙江、湖北、湖南、四川、廣東各一人,陜甘、新疆一人,雲南、貴州一人,漕河一人,由督撫保送本省武進士、舉人及守備咨補。後改隸郵傳部。

初,天聰五年,設兵部。順治元年,置尚書、侍郎各官。郎中,滿洲八人,十二年增三人。雍正五年增一人。蒙古四人,康熙三十八年省。五十七年復置一人。漢軍二人,雍正五年省。漢四人。雍正五年增一人。員外郎,滿洲八人,十二年增五人。康熙三十八年省三人。蒙古四人,康熙三十八年省。五十七年復置三人。漢軍六人,康熙三十八年省四人,雍正五年俱省。漢四人。十一年省二人。雍正五年增一人。堂主事、司主事,俱滿洲四人;漢軍堂主事一人,漢主事五人。會同館大使一人。康熙三十八年省。五年,定滿、漢尚書各一人。八年,以諸王、貝勒兼理部事。尋罷。

十一年,增置督捕。滿左侍郎、漢右侍郎各一人。漢協理督捕、太僕寺少卿,二人。尋改。左右理事官,滿洲、漢軍各一人。後改滿、漢各一人。滿、漢郎中各一人。員外郎,滿洲七人、漢軍八人,漢一人。堂主事,滿洲三人,司主事一人,十四年增一人。漢主事六人,司獄二人。郎中以下亦有兼督捕銜者。分理八司掌捕政。三營將弁隸之。十二年,增置督捕員外郎八人。旗各一人。時八旗武職選授處分,並隸銓曹,康熙二年始來屬。三十八年,省督捕侍郎以次各官,並入刑部。雍正元年,命大學士管部,自後以為常。嘉慶四年,省滿洲郎中、員外郎各一人,為宗室員缺。光緒三十二年,更名陸軍部。

刑部尚書,左、右侍郎,俱滿、漢一人。其屬:堂主事,清檔房滿洲二人,漢本房滿洲三人,漢軍一人。司務司務,滿、漢各一人。繕本筆帖式四十人。直隸、奉天、江蘇、安徽、江西、福建、浙江、湖廣、河南、山東、山西、陜西、四川、廣東、廣西、雲南、貴州十七清吏司:郎中,宗室一人,湖廣司置。滿洲十有五人,除奉天、湖廣兩司外,司各一人。蒙古一人,奉天司置。漢十有九人。湖廣、陜西司各二人,餘俱一人。員外郎,宗室二人,廣東、雲南司各一人。滿洲二十有三人,江蘇、湖廣、河南、山東、陜西、廣東司各二人,餘俱一人。蒙古一人,直隸司置。漢十有九人。直隸、浙江司二人,餘俱一人。主事,宗室一人,廣西司置。滿洲十有五人,除奉天、湖廣二司外,司各一人。蒙古一人,山西司置。漢十有七人。司各一人。督捕清吏司:郎中,滿、漢各一人。員外郎,滿洲一人。主事,滿、漢各一人。筆帖式,宗室一人,滿洲百有三人,蒙古四人,漢軍十有五人。提牢主事,滿、漢各一人。由額外及試俸主事引見補授。司獄,從九品。滿洲四人,漢軍、漢各一人。贓罰庫,正七品。滿洲一人。庫使,未入流。滿洲二人。

尚書掌折獄審刑,簡覈法律,各省讞疑,處當具報,以肅邦紀。侍郎貳之。十七司各掌其分省所屬刑名。直隸司兼掌八旗游牧、察哈爾左翼所屬,並理京畿道御史、順天府、東西陵、熱河都統、圍場總管、密雲副都統、山海關副都統、古北口、張家口、獨石口、喜峰口、蘆峰口、塔子溝、三座塔、八溝、烏蘭哈達、喀拉河屯、多倫諾爾文移。奉天司兼掌吉林、黑龍江所屬,並理宗人府、理籓院文移。江蘇司兼掌各省減免之案,凡遇恩赦,審詳具奏。並理江南道御史、江寧將軍、京口副都統、漕運總督、南河總督文移。安徽司兼理鑲紅旗、宣武門文移。江西司兼理江西道御史、中城御史、正黃旗、西直門文移。浙江司兼理都察院刑科、浙江道御史、南城御史、杭州將軍、乍浦副都統文移。並司條奏匯題,及各司爰書駁正者,會其成,比年一奏。福建司兼理都察院戶科、倉場衙門、左右兩翼監督、鑲藍旗、阜成門、福州將軍文移。湖廣司兼掌湖北、湖南所屬,並理湖廣道御史、荊州將軍文移。河南司兼理禮部、都察院禮科、河南道御史、太常寺、光祿寺、國子監、鴻臚寺、欽天監、太醫院、東城御史、正紅旗、德勝門文移。凡夏令熱審,頒行各省欽恤如制。山東司兼理兵部、都察院兵科、山東道御史、太僕寺、青州副都統、東河總督文移。凡步軍營捕獲盜賊,歲登其數請敘。山西司兼理察哈爾右翼、綏遠城將軍、歸化城副都統、定邊左副將軍、科布多參贊大臣、庫倫辦事大臣所屬,並理軍機處、內閣、翰林院、詹事府、起居注、中書科、內廷各館、內務府、山西道御史、北城御史、鑲白旗、崇文門文移,及各省年例咨報之案。陜西司兼掌甘肅、伊犁、烏魯木齊、塔爾巴哈臺、葉爾羌、喀什噶爾、烏什、阿克蘇、庫車、吐魯番、哈喇沙爾、和闐、哈密所屬,並理陜西道御史、大理寺、西城御史、西安將軍、寧夏將軍、涼州副都統、伊犁將軍文移。囚糧則以時散給。四川司兼理工部、都察院工科、四川道御史、成都將軍文移。凡秋審,會九卿、詹事於朝房以定爰書,並收發刑具。廣東司兼理鑾輿衛、正白旗、廣東道御史、安定門,廣州將軍文移。廣西司兼理通政司、廣西道御史文移。凡朝審,具題稿,囚衣則以時散給。雲南司兼理鑲黃旗、雲南道御史,東直門文移。並司堂印封啟。貴州司兼理吏部、都察院吏科、正藍旗、貴州道御史、朝陽門文移。並定各司漢員升補。督捕司掌八旗及各省逃亡。提牢掌檢獄圄。司獄掌督獄卒。贓罰庫掌貯現審贓款,會數送戶部。別設律例館,由尚書或侍郎充總裁。提調一人,纂修四人,司員兼充。校對四人,收掌二人,翻譯、謄錄各四人。司員及筆帖式充。掌修條例。五年匯輯為小修,十年重編為大修。秋審處,主覈秋錄大典。初以四川、廣西二司分理。雍正十二年,始別遣滿、漢司員各二人,曰總辦秋審處。尋佐以協辦者四人。錄各省囚,謂之秋審;錄本部囚,謂之朝審。歲八月,會九卿、詹事、科道公閱爰書,覈定情實。凡大闢,御史、大理寺官會刑司錄問,案法隨科,曰會小三法司。錄畢,白長官。都御史、大理卿詣部偕尚書、侍郎會鞫,各麗法議獄,曰會大三法司。讞上,復召大臣按覆,然後麗之於闢。初制,刑部會擬朝審,俱本部案件。其外省之案,康熙十六年始命刑部覆覈,九卿會議。

初,天聰五年,設刑部。順治元年,置尚書、侍郎各官。設江南、浙江、福建、四川、湖廣、陜西、河南、江西、山東、山西、廣東、廣西、雲南、貴州十四司,置滿洲郎中六人,五年增八人。員外郎八人,五年增十人。堂主事五人,司主事十有四人;漢軍郎中四人,雍正五年省。員外郎十有二人,康熙三十八年省八人,雍正五年俱省。堂主事一人;漢郎中、雍正五年,增江南、湖廣、陜西司各一人。員外郎、十五年省湖廣、廣西、雲南、廣東司各一人。雍正三年復故,並增四川司一人。五年增浙江、山東司各一人。主事,十五年省河南、四川、陜西、貴州司各一人。雍正三年復故。各十有四人。滿洲司庫一人,漢司獄四人。康熙五十一年增滿洲四人。乾隆六年定漢軍、漢各二人。五年,定滿、漢尚書各一人。七年增滿洲一人,十年省。十八年,置蒙古員外郎八人。康熙元年省。康熙三十八年,增設督捕前、後司,為十六司。由兵部並入。五十七年,增置蒙古郎中、員外郎、主事各一人。雍正元年,設現審左、右二司,主鞫訊囚系。十二年,析江南司為江蘇、安徽二司,定滿、漢郎中俱各一人,滿洲員外郎三人,江蘇司二人,安徽司一人。漢員外郎二人,滿、漢主事司各一人,並督捕前、後司為一。自時厥後,親王、郡王奉命管部,無常員。乾隆六年,更現審左司為奉天司,右司為直隸司,定滿洲、直隸司置。蒙古奉天司置。郎中各一人,漢郎中各一人,滿洲員外郎二人,蒙古一人,直隸司置。漢三人,奉天司一人,直隸司二人。滿、漢主事俱各一人,是為十七司。嘉慶四年,以大學士領部事,改滿洲郎中、員外郎、主事各一人為宗室員缺。光緒六年,增置雲南司宗室員外郎一人。三十二年,更名法部。

工部尚書,左、右侍郎,俱滿、漢一人。其屬:堂主事,清檔房滿洲三人,漢本房滿洲、漢軍各一人。司務司務,滿、漢各一人。繕本筆帖式,宗室一人,滿洲十人。營繕、虞衡、都水、屯田四清吏司:郎中,宗室一人,屯田司置。滿洲十有六人,營繕、虞衡各四人,都水五人,屯田三人。蒙古一人,營繕司置。漢四人。司各一人。員外郎,宗室一人,虞衡司置。滿洲十有六人,營繕、虞衡各四人,都水五人,屯田三人。蒙古一人,營繕司置。漢四人。司各一人。主事,宗室一人,屯田司置。滿洲十有一人,營繕、屯田各二人,虞衡三人,都水四人。蒙古一人,營繕司置。漢六人。營繕、都水各二人,虞衡、屯田各一人。筆帖式,宗室一人,滿洲八十有五人,蒙古二人,漢軍十人。制造庫,郎中,滿洲二人,漢一人;司庫、正七品。司匠,初制七品,康熙九年定從九品。俱滿洲二人;庫使,未入流。滿洲二十有一人。節慎庫,滿洲郎中、員外郎各一人,司庫二人,庫使十有二人。硝磺庫、鉛子庫,滿洲員外郎、主事俱各一人。

尚書掌工虞器用、辨物庀材,以飭邦事。侍郎貳之。右侍郎兼掌寶源局鼓鑄。營繕掌營建工作,凡壇廟、宮府、城郭、倉庫、廨宇、營房,鳩工會材,並典領工籍,勾檢木稅、葦稅。虞衡掌山澤採捕,陶冶器用。凡軍裝軍火,各按營額例價,計會覈銷,京營則給部制。頒權量程式,辦東珠等差。都水掌河渠舟航,道路關梁,公私水事。歲十有二月,伐冰納窖,仲夏頒之;並典壇廟殿廷器用。屯田掌修陵寢大工,辦王、公、百官墳塋制度。大祭祀供薪炭,百司歲給亦如之;並檢督匠役,審覈海、葦、煤課。節慎掌主帑藏,司出納。制造掌典五工:曰銀工、曰鍍工、曰皮工、曰繡工、曰甲工;凡車輅儀仗,展採備物,會鑾儀衛以供用。所轄寶源局,滿、漢監督各一人,滿員由宗人府、六部、步軍統領衙門司員內保送。漢員由六部司員內保送。大使二人,正九品。本部筆帖式內保送。初置筆帖式一人,雍正七年改置。職視寶泉局。其皇木廠,琉璃窯,木倉,軍需局,官車處,惜薪廠,冰窖,採紬庫,滿、漢監督俱各一人。砲子庫,滿洲監督一人。皇差銷算處,滿、漢司員各二人。料估所,滿、漢司員各葈人。黃檔房無定員。以上各員,並由本部司員內選用。

初,天聰五年,設工部。順治元年,置尚書、侍郎各官。右侍郎兼管錢法。康熙十八年增滿洲一人兼管。郎中,滿洲八人,內一人管節慎庫。十二年增八人。雍正五年增一人。蒙古一人。康熙三十八年省,五十七年復故。員外郎,滿洲九人,十二年增八人。康熙五十七年增一人,雍正五年增一人。道光十六年,改營繕司員外郎一人專司鉛子庫,都水司員外郎一人專司硝磺庫。蒙古三人。康熙三十八年省,五十七年復置一人。滿洲堂主事三人,清文二人,清漢文一人。司主事四人;康熙二十三年增八人。漢軍郎中二人,雍正五年省。員外郎六人,康熙三十八年省四人,雍正五年俱省。堂主事一人。節慎庫,滿洲員外郎一人,司庫二人,漢大使一人。十五年省。制造庫,滿洲郎中一人,員外郎二人,尋省。司庫、司匠各二人。營繕、虞衡、都水、屯田,漢郎中五人,營繕二人,餘各一人。十五年省營繕一人。十六年增虞衡一人。十八年復置營繕一人。康熙元年增額仍省。員外郎七人,屯田一人,餘各二人。十五年省營繕、都水、虞衡各一人。康熙十一年,增都水二人。三十年,增額仍省。主事二十人。營繕、虞衡、屯田各三人,都水十有一人。十四年增營繕三人。十五年省都水一人。明年省營繕一人。康熙元年又省營繕一人。六年省營繕、虞衡、屯田各一人,都水四人。十二年又省都水四人。道光十六年,改營繕一人專司鉛子庫,都水一人專司硝磺庫。營繕司所正、所副各一人。文思院,廣積庫,柴炭司,通州抽分竹木局,各大使俱一人。十五年並省。寶源局監督三人。康熙十七年定二人。五年,定滿、漢尚書各一人。十四年,增置營繕司所丞二人。分管清江廠、臨清磚廠。十五年省臨清廠一人。康熙六年省清江廠一人。九年復置清江一人。雍正四年俱省。康熙五十七年,增置蒙古主事一人。雍正元年,命親王、郡王、大學士攝部事。尋罷。七年,增置寶源局大使二人。初置筆帖式一人,至是改置。嘉慶四年,改滿洲員外郎、主事各一人為宗室員缺。十年,改令大學士管部。光緒六年,增置宗室郎中一人。屯田司置。三十二年,更名農工商部,省節慎庫,並土木工程入民政部,木稅、船政入度支部,軍械、兵艦入陸軍部,內外典禮分入內府與禮部。初制,置柴薪正、副監督各一人,本部司員充。煤炭監督二人。一以部員兼攝,一以內府司員兼攝。乾隆四十六年,亦改隸內府。

管理直年火藥局大臣二人,欽派一人,本部侍郎一人。掌儲火藥。監督無恆額。本部司員、筆帖式內派委。

直年河道溝渠大臣四人,本部堂官一人,奉宸院、頤和園、步軍統領衙門堂官各一人,每歲並由工部奏請。掌京師五城河道溝渠。督理街道衙門御史,滿、漢各一人。本部司員、步軍統領衙門司員各一人,掌道路溝瀆。

盛京五部戶部,侍郎一人,自侍郎以下,俱滿缺。品秩視京師。各部同。掌盛京財賦。宗室郎中、堂主事各一人。經會、糧儲、農田三司,郎中三人,農田司一人,乾隆八年增。員外郎六人,司各二人。主事五人。經會、糧儲各二人,農田一人。經會典泉貨。糧儲典穀糈。農田典畝數。管銀庫,正關防郎中、副關防員外郎,各一人。管莊,六品官二人。管喇嘛丁銀委,六品官一人。司庫二人,庫使八人。筆帖式二十有二人。內漢軍二人。外郎九人。漢軍六缺,候補筆帖式內挨補。六年期滿,除授州同、州判、縣丞。

禮部,侍郎一人,掌盛京朝祭。宗室主事一人,堂主事二人。左、右兩司,郎中各一人,員外郎各二人。左司典祭物,司關領。右司典祭物,贍僧道。讀祝官初制五品。後改九品。八人,贊禮郎初制四品。後改九品。十有六人。管千丁,六、七品官各一人。管學,助教四人。筆帖式十人。庫使八人。外郎二人。僧錄、道錄二司視京師。

兵部,侍郎一人,掌盛京戎政。宗室員外郎一人,堂主事二人。左、右兩司,郎中各一人,員外郎各二人,主事各一人。筆帖式十有二人。外郎四人。內漢軍二缺。左司典郵政,右司司邊禁。

刑部,侍郎一人,掌盛京讞獄。邊外蒙古隸之。宗室員外郎一人,堂主事二人。漢軍一人。肅紀前、後、左、右四司,郎中各一人,員外郎六人,前司、左司各二人,餘俱一人。主事六人。右司蒙古三人,餘俱一人。司獄二人。漢軍一人。司庫一人,庫使二人。筆帖式三十有一人。內蒙古二人,漢軍五人。外郎二人。漢軍缺。前司、左司典十五城獄訟,右司典蒙古獄訟,後司典★J5庫禁令。

工部,侍郎一人,掌盛京工政。宗室主事一人,堂主事二人。左、右兩司,郎中各一人,員外郎各二人,主事各一人。左司治木稅,右司治葦稅。管千丁,四品官一人。世襲。大政殿,六品官一人。滿洲、漢軍參用。黃瓦廠,五品官一人。侯姓世襲。司匠役,六品官一人。司庫二人,庫使八人。筆帖式十有七人。漢軍一人。外郎九人。漢軍四人。

初,締造沈陽,建六部,置承政、參政各官。世祖奠鼎燕京,置官鎮守,戶、禮、兵、工四曹隸之。十五年,設禮部;明年,設戶、工兩部;康熙元年,設刑部;三十年,復設兵部;並置侍郎以次各官,五部之制始備。舊制各置理事官正四品。一人,六十年省。雍正三年,定每歲差御史一人稽察五部。嘉慶四年停。五年,允御史傅色納請,增置漢郎中等官。乾隆八年省。復定鳳凰城迎送官三人。正五品。乾隆三十六年省。八年,置尚書領其事。尋省。光緒初,定將軍兼理兵、刑二部,佩金銀庫印鑰,稽覈戶部。餘悉如故。四年,增置宗室司員。如前所列。三十一年,復命將軍趙爾巽兼管五部。尋以政令紛歧,疏省之。報可。


\end{pinyinscope}