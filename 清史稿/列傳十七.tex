\article{列傳十七}

\begin{pinyinscope}
武理堪子吳拜蘇拜蘇拜子和託武納格子德穆圖齊墨克圖

阿什達爾漢蘇納固三泰固三泰子明阿圖明阿圖子賽弼翰瑚什布

瑚什布子穆徹納鄂莫克圖喀山喀山子納海安達立綽拜布丹

孫達哩吉思哈弟吉普喀達吳巴海康喀勒從兄子和托瑪拉

兄孫通嘉薩璧翰

武理堪,瓜爾佳氏,世居義屯。父伊蘭柱,徙居哈達費德里。太祖初起,武理堪來歸。歲癸巳,葉赫糾九部之師,三道來侵,上遣武理堪出東路偵敵。武理堪出虎攔哈達新城,行將百里,方度嶺,群鴉競噪,若阻其行者,武理堪心異之,度行且與敵左,馳歸告上,上命改道自札喀路向渾河部。武理堪行,薄暮至渾河,敵方屯北岸會食,爨火密如星。武理堪得葉赫邏卒一,言敵兵三萬,將夜度沙濟嶺而進,遂挾以還報,時夜方半,上命旦日出師。武理堪慮我軍怵敵眾,言曰:「敵雖眾,心不一,誰能御我?」及戰,遂破諸路兵。

旗制定,武理堪隸滿洲正白旗,分轄丁戶,為牛錄額真。出從征伐,率選鋒前驅,為噶布什賢噶喇昂邦。天命四年,明經略楊鎬合諸鎮兵四道來侵,太祖督諸貝勒帥師御之。既,敗其三道,獨總兵李如柏出鴉鶻關,未與我師遇,鎬檄使引還。武理堪方率二十騎邏虎攔山,見如柏軍行山麓;乃令諸騎立馬山巔,鳴螺,脫帽系弓末,揮且噪,若指揮伏兵者,如柏軍望而愕顧。武理堪遂縱騎疾馳下擊,斬四十人,獲馬五十匹,如柏軍奪路走,相蹂藉死者復千餘。武理堪尋卒,太祖嘆曰:「武理堪從朕摧鋒陷陣,幾死者數矣!」乃錄其二子吳拜、蘇拜。

吳拜,年十六,從太祖伐明,略撫順,遇敵輒奮斗,矢中顙不顧。嘗從太祖獵,有熊突圍出,躍上峻嶺,太祖遙望見一人躍馬射熊,貫胸而墮。上顧侍臣雅蓀曰:「是非吳拜不能。」遣視之,吳拜也。因諭諸皇子曰:「吳拜之勇,今共見之矣!」遂授侍衛。天命四年,從伐葉赫,負重創,力戰不退,師還,賜良馬。明總兵毛文龍誘我新附之眾實皮島。吳拜循徼三日,獲逋八十餘,射殺文龍使者,還告上。時吳拜已代父為牛錄額真,上命以所獲隸所轄牛錄。六年,從伐明。破明軍於南壽山,授備御。既克遼陽,以俘獲分隸諸將,上以吳拜能繼父志,年少建功,命視一等大臣,隸千人。十一年,蒙古巴林部貝勒囊努兔背盟掠境上,上遣將討之,吳拜從,諜者為敵困,援之出,殪敵百人。

太宗即位,列十六大臣,佐鑲白旗。命逐蒙古亡去者,至都爾弼。蒙古亡去者十五人,拒戰,吳拜既被創,仍奮擊,盡斬之。太宗諭諸大臣曰:「是固先帝數嘉許者!」賞特厚。天聰四年,伐明,取永平、灤州等四城,吳拜從貝勒阿敏守永平。阿敏引還,吳拜當坐罪逮系,以嘗率擺牙喇兵援灤州,夜入敵營,太宗命貰之,釋其縛。尋授噶布什賢甲喇章京。五年,從伐明,圍大凌河城,與甲喇章京蘇達喇詣錦州偵敵。六年,從伐察哈爾,率精騎前驅,道遇蒙古亡去者,擊殺之,察哈爾林丹汗西奔土默特部。師還,取歸化城,上命吳拜撫輯降者。

八年,伐明,攻大同,多爾濟將中軍,圖魯什將左軍,吳拜將右軍,明總兵曹文詔迎戰,擊敗之。復與甲喇章京席特庫設伏宣府,獲明守備一,殲其游騎。尋與承政阿什達爾漢等招林丹汗子額哲來歸。九年五月,明屯軍大凌河西,吳拜與固山額真阿山、石廷柱、圖賴要其歸路,斬明副將劉應選,獲游擊曹得功及守備三,殲步騎五百餘,復攻克松山城南堡。師還,進三等甲喇章京。是時,上遣諸貝勒分道伐明,命吳拜等帥師駐上都城舊址,偵軍事。崇德元年,復命與勞薩等齎書投明邊吏。

冬,征朝鮮,命與承政馬福塔等率兵三百為前驅,襲朝鮮都城,朝鮮王倧走南漢山城。師進,吳拜與勞薩擊破朝鮮援兵,斬二百餘級。二年,授噶布什賢噶喇昂邦,列議政大臣。甲喇章京丹岱、阿爾津等如土默特互市,將還,命吳拜率將校至歸化城迎護,遇明邏卒十六人,斬其十五,獲馬十九,俘一,以還。

三年四月,略寧遠,逐敵墮壕,斬馘甚眾。八月,率兵八十人至洪山口,遇明兵,斬其裨將;復擊走羅文峪騎兵五百,奪其纛,獲馬四十,殲密雲步兵百餘。五年,與勞薩率兵過中後所,略海濱,斬級二百,獲馬騾牲畜。我師攻錦州,命吳拜駐軍要隘為策應,屢敗敵兵。六年春,以攻錦州勿克,論統師王貝勒罪,吳拜坐罰鍰。秋,上自將攻松山,明兵敗走,吳拜未邀擊,逮系,旋命釋之。七年,從貝勒阿巴泰入明邊,敗敵豐潤、三河、靜海,至於青州。八年,從鄭親王濟爾哈朗取明中後所、前屯衛。十一月,復授正白旗梅勒額真。順治初,從入關,擊李自成。二年,解梅勒額真,授內大臣。三年,從豫親王多鐸討蘇尼特部長騰機思。四年,與輔國公鞏阿岱、內大臣何洛會帥師戍宣府。論功,遇恩詔,累進二等伯。

蘇拜,年十五,從太祖伐蒙古有功,授侍衛,兼領牛錄額真。天聰間,從軍收察哈爾林丹汗子額哲,遂入明邊,攻代州,明兵三百自崞縣赴援,蘇拜爭先當敵,明兵潰走。崇德元年,從伐朝鮮,破敵桃山村。三年,授擺牙喇甲喇章京。從貝勒岳託伐明,自墻子嶺入,越明都,擊敗明太監馮永盛。四年,圍錦州,蘇拜屢擊敗明援兵自松山、杏山至者;又與固山額真圖爾格等伏兵烏忻河口,多所俘獲,敵千餘躡師後,擊卻之,獲其輜重。六年,復圍錦州,敗松山騎兵,又敗明總督洪承疇所將步兵,予世職牛錄章京,兼半個前程。七年,從貝勒阿巴泰伐明,敗敵,克樂安、昌邑。八年,師還,進三等甲喇章京。

順治初,從入關,擊李自成。世祖既定鼎,命將分道討自成:以豫親王多鐸為定國大將軍,出山西、河南;英親王阿濟格為靖遠大將軍,道塞外土默特、鄂爾多斯諸部入邊,南取西安,蘇拜佐阿濟格軍。方冬,渡黃河,鑿冰以濟。明年春,至榆林。自成兵夜襲蒙古軍,蘇拜與擺牙喇纛章京徹爾布赴援,賊敗走,還軍遇伏,復擊卻之。攻延安,七戰皆勝。自成走湖廣,追之至安陸,屢破賊壘,俘馘無算。三年,攝擺牙喇纛章京。從肅親王豪格討張獻忠,敗獻忠將高汝礪於三寨山,進擊獻忠於西充。賊攻正藍旗營,蘇拜與阿爾津共援之,大破賊兵。五年,師還,授擺牙喇纛章京。論功,遇恩詔,累進二等精奇尼哈番。

八年正月,吳拜、蘇拜及內大臣洛什、博爾輝發英親王阿濟格罪狀,吳拜進三等侯,蘇拜進一等精奇尼哈番加拖沙喇哈番。二月,洛什、博爾輝以諂媚諸王、造言構釁,論死;吳拜兄弟坐削爵,奪官,籍沒。蘇拜又坐阿徇睿親王多爾兗,論死,上特宥之。九年,起蘇拜為正白旗梅勒額真。十三年,擢內大臣。十五年,上念吳拜兄弟事太祖、太宗有戰功,復授吳拜世職一等精奇尼哈番,蘇拜一等阿思哈尼哈番。尋授蘇拜領侍衛內大臣。康熙三年十二月,蘇拜卒,謚勤僖。四年四月,吳拜卒,年七十,謚果壯。吳拜子郎談,自有傳。

蘇拜第三子和託,康熙間以侍衛從討王輔臣,戰平涼城北,殺賊甚眾;從討吳三桂,戰攸縣,敗三桂將王國佐等;戰永興,敗三桂將胡國柱等。十九年,自廣西進兵攻石門坎、黃草壩,薄雲南省城,敗吳世璠將胡國柄、劉起龍等,皆有功。官至護軍參領,予世職騎都尉加一雲騎尉。五十二年,卒。

武納格,博爾濟吉特氏,隸蒙古正白旗。其先蓋出自蒙古,而居於葉赫。太祖創業,武納格以七十二人來歸。有勇略,通蒙、漢文,賜號「巴克什」。歲癸丑,從伐烏喇有功,授三等副將。天命十一年,太祖伐明,圍寧遠城未下,命武納格別將兵攻覺華島。明參將姚撫民將兵四萬,倚島列屯,鑿冰為壕,袤十五里,衛以盾。武納格督軍爭壕,首排盾逕入,盡殲其眾,焚所儲芻糗及舟二千餘,進三等總兵官。

太宗即位,武納格總管蒙古軍,位亞揚古利、李永芳,在八大臣上。旋以蒙古軍益眾,分左、右二營,武納格與鄂本兌同為固山額真。天聰三年春,與額駙蘇納等率蒙古軍,益以滿洲驍卒八十人,伐察哈爾,降其邊境二千戶。軍中流言降者瞷我師寡將為變,於是盡殲其男子,惟二臺吉得免,俘其孥八千。太宗責武納格等殺降非義,奪所給牲畜,命以所俘分隸二翼,贍之毋失所。冬,從太宗伐明,入龍井關,克遵化,進薄明都。明督師袁崇煥自寧遠來援,左翼蒙古兵迎戰不能勝,武納格麾右翼蒙古兵繼進,遂敗敵。賜俘獲之半以犒其軍。尋克固安。四年春,克永平。明將以三千騎自玉田至,武納格遣兵擊之走,獲馬百餘。行略豐潤,還,聞明兵四千攻大安口城急,與察哈喇赴援,解其圍。又就軍士行樵,設伏致敵,斬獲無算。

五年秋,復從伐明。明總兵祖大壽守大凌河城,杏山守將與大壽書,謀攜軍棄城相就,武納格獲以獻,得其情,於是環城築壘鑿壕,為久困計。武納格統蒙古兵屯城東南,大壽縱兵出攻我所下臺堡,武納格與貝勒阿濟格等率兵夾攻,殲敵過半,自是城兵不復出。六年夏,與阿濟格招撫大同、宣府邊外察哈爾部眾。七年秋,與貝勒阿巴泰等侵明,攻山海關,有所俘獲。師還,明兵追襲,武納格為殿,力戰卻之。太宗諭諸貝勒大臣曰:「武納格所在建功,今又為殿敗敵。人臣為國,當如是也!」

八年五月,改蒙古軍左、右營為左、右翼,以武納格為左翼固山額真。定諸將功次,武納格以一等昂邦章京世襲,旋進三等公。是年,太宗復率諸貝勒分道伐明,命武納格統蒙古軍為策應,入獨石口,越興安嶺,經保安州,至應州,與大軍會,道收察哈爾千餘戶,所過諸州縣,或攻或撫,悉稱上意。閏八月,自得勝堡班師,收蒙古逃人自陽和入者四百七十人。九月,喀爾喀部眾為察哈爾所襲殺,命將百人往詗,斬二十餘人而還。九年二月,卒。子德穆圖、齊墨克圖、廣泰。

德穆圖,武納格長子也。初任牛錄額真。崇德三年正月,擢戶部承政。七月,更定官制,改右參政。四年,從上伐明,圍松山,樹雲梯攻城。會明兵自錦州赴援,德穆圖度不能克,棄雲梯引還,罪當死,上特貰之,論罰。尋兼任梅勒額真。六年,從鄭親王濟爾哈朗伐明,圍錦州。蒙古貝勒諾木齊等守外城,約降,鄭親王令德穆圖迎之。諾木齊方率所部與明兵戰,德穆圖以其子阿桑喜出我師克外城,諾木齊始來歸。德穆圖詭言諾木齊父子皆所拔出,論罰,籍家產之半,罷參政、梅勒額真,俾專領牛錄。七年,從貝勒阿巴泰伐明,自薊州入邊,薄明都,略山東。順治元年,從入關擊流賊,授拜他喇布勒哈番。二年,從豫親王多鐸攻潼關,遂定江南。敗明將鄭鴻逵於瓜洲,與都統馬喇希徇常州,與明將黃蜚等遇,再戰皆捷。分兵下宜興、昆山諸縣,加拖沙喇哈番。復任本旗蒙古副都統,三進一等阿達哈哈番。九年,卒。

齊墨克圖,武納格次子。早歲屢從行陣,略寧遠,敗明兵。武納格既卒,以廣泰襲世職,從伐明,坐違令不前,奪世職,以齊墨克圖降襲一等阿思哈尼哈番。復從伐明,與沙爾虎達等率邏卒至錦州,明兵五百來追,還擊敗之,獲馬六十及其纛。太宗伐明,三圍錦州,齊墨克圖皆在軍中,遇城兵出戰,驟馬截擊,陣斬十人。攻洪承疇所將步卒,掩殺甚眾,又敗敵援兵。崇德八年三月,與阿爾津、哈寧阿等伐黑龍江,圍都里屯,克之;又降大小噶爾達蘇、能吉爾三屯師:賚貂皮、銀幣。十一月,擢梅勒額真,佐本旗。順治初,從入關,加拜他喇布勒哈番,合為三等精奇尼哈番。三年,從定西大將軍何洛會擊破叛將賀珍。五年,卒,復以廣泰襲一等阿思哈尼哈番,別以齊墨克圖子薩哈炳分襲拜他喇布勒哈番。廣泰遇恩詔,進二等精奇尼哈番。乾隆初,定封三等子。

阿什達爾漢,納喇氏,與葉赫貝勒金臺石同族,為兄弟,太宗諸舅也。太祖滅葉赫,阿什達爾漢率所屬來歸,授牛錄額真,隸滿洲正白旗。天命六年二月,從伐明,攻奉集堡,圍其城,阿什達爾漢先諸將奮進,三月,攻遼陽,復先登,克之,授一等參將,敕免死一次。

太宗嗣位,以阿什達爾漢典朝鮮、蒙古諸屬部,嘗奉使宣諭。天聰六年,明邊吏遣使議和,上命阿什達爾漢及白格、龍什等報聘。既盟而歸,白格言阿什達爾漢及龍什等受明邊吏餽,命奪入官。六年,從貝勒濟爾哈朗、薩哈璘如蒙古鞫獄,賚敕二十道,失其九,論罰。十一月,復以定律令頒布蒙古諸部。

八年五月,上自將伐察哈爾林丹汗,命徵兵科爾沁部,會於宣府左衛。林丹汗西遁,道死。所屬額爾德尼囊蘇等以其眾降。上命阿什達爾漢及吳拜等挾額爾德尼囊蘇詗林丹汗子額哲所在。九月,率來降臺吉塞冷等還,並報復有祁他特等率千人而來者,踵相接也。旋命至舂科爾大會蒙古諸部,分畫牧地,使各有封守,復與諸貝勒亭平其獄訟。十一月,還報稱旨,令專轄一牛錄。九年二月,從貝勒多爾袞等將萬人取額哲。四月,師至托裏圖,多爾袞等遵上所授方略,遣阿什達爾漢及金臺石孫南褚諭額哲母。額哲母,金臺石女孫也,阿什達爾漢為其族尊行,額哲遂從其母舉部來降。當我軍未至,有鄂爾多斯濟農圖巴者招額哲,與盟而去。阿什達爾漢偵知之,追及圖巴,令悉歸額哲之餽。又率兵入明邊,略宣府、大同,入山西境,多所俘獲。師還,上親迎勞之。

崇德元年六月,授都察院承政。上御崇政殿,侍臣巴圖魯詹、額爾克戴青後至,阿什達爾漢責其慢,叱出之。十月,與希福使察哈爾、喀爾喀、科爾沁諸部,申明律令。十二月,從伐朝鮮,國王李倧走保南漢山城,豫親王多鐸帥師追之,圍城。朝鮮諸道援兵合萬八千人,樹二柵城外,悉眾出戰,阿什達爾漢及貝子碩託率精騎銳進,大破其軍。朝鮮別將以五千人屯山麓為聲援,復分兵百,循河而南,阿什達爾漢馳擊盡殲之,攻破其壘,餘眾皆潰。二年正月,倧請降。論功,進三等副將,世襲。尋復使科爾沁、巴林、扎魯特、喀喇沁、土默特、阿祿諸部,頒赦,且讞獄。明年五月,部議阿什達爾漢讞獄失平,受蒙古諸部餽,命罷承政,奪所餽入官。七月,復授都察院承政。

五年,與參政祖可法等疏論時事,略言:「皇上欲恢張治道,深思篤行之。今諸國景附,朝廷清明,而諸王以下至諸固山額真,彼此瞻顧,第念身家,莫肯一心為國,有所論列。不知果無可言耶,抑有所畏忌而不敢言耶?夫刑所以防民之奸,骫於法則麗於刑,此不可宥也。今刑部斷獄不依本律,諸臣有坐者,或從重論,輒削其職。臣思諸臣歷戰陣,出死力,蒙恩授官;一旦有過,豈可不論重輕而遽削其職乎?臣等竊思先時簡選議事十人,今皆不稱職,宜罷斥。令甲,戰死者將吏得世職,兵則恤其妻孥。今又未盡行,惟皇上裁察。」疏入,上嘉納之。

六年,從伐明,上督諸軍圍松山。明總兵曹變蛟屯乳峰山,乘夜棄寨,率步騎直犯御營,諸將力戰卻之。阿什達爾漢未至,論罪,罷承政,降世職為牛錄章京。尋卒。

太祖諸臣自葉赫來歸者,蘇納、固三泰、瑚什布皆與金臺石同族。

蘇納當葉赫未亡,棄兄弟歸太祖,太祖妻以女,為額駙。編所屬人戶為牛錄,使領牛錄額真,隸正白旗。天命四年,太祖滅葉赫,命蘇納收其戚屬隸所領牛錄。十年,授甲喇額真。錄戰功,賜敕免死四次。尋擢梅勒額真。

天聰元年,太宗自將伐明,攻錦州,以貝勒莽古爾泰等將偏師屯塔山,衛餉道;命蘇納選八旗蒙古精銳別屯塔山西路,截明兵。明兵二千人至,蘇納領纛進擊,敗之,乘勝逐敵,多所俘斬,獲馬百五十。三年春,命與武納格將兵伐察哈爾,以殺降見詰責。十月,復與武納格將兵逐蒙古亡去者。語並見武納格傳。五年,授擺牙喇纛章京,擢兵部承政。從伐明,圍大凌河城,敗城兵及錦州援兵,授備御世職。八年,考滿,進三等甲喇章京,免徭役。九年,以隱匿壯丁,削世職。七月,定蒙古旗制,以蘇納領鑲白旗。

崇德元年,從武英郡王阿濟格伐明,薄明都,攻雕鶚、長安、昌平諸城隘,五十六戰皆捷;復與薩穆什喀共攻容城,克之。師還,以先出邊,後隊為敵乘,潰敗,奪所俘獲。十二月,從伐朝鮮,朝鮮將以步騎兵千餘御戰,蘇納及吳塔齊等邀擊,大破之,俘其將。二年,吏議蘇納坐朝鮮國王朝行在,亂班釋甲,又離大軍先還,論罰。三年,又坐有所徇隱,論罰,罷固山額真,仍領牛錄。順治五年,卒。世祖追錄蘇納舊勞,復原職。子蘇克薩哈,自有傳。

固三泰歸太祖,太祖妻以女,為額駙。領牛錄,隸滿洲鑲藍旗。從伐明,戰於廣寧,單騎入敵陣,身被數創,戰愈力,師乘之,遂敗敵,授副將世職。太宗即位,為八大臣,領本旗。天聰元年三月,從貝勒阿敏伐朝鮮有功,師還,上郊勞。三年,上自將伐明,攻遵化,固三泰率本旗兵攻其西南,克之。四年,上命固三泰與達爾漢等助攻昌黎。語詳達爾漢傳。復命與高鴻中、庫爾纏等下灤州,籍其倉庫銀穀以聞。五年,上幸文館,覽達海所譯武經,因諭群臣曰:「為將當恤士。朕聞額駙固三泰與敵戰,士有死者,以繩系其足曳歸,蔑視若此,何以得其死力乎?」尋命解固山額真。九年,詔免徭役,並增賜人戶,俾專領牛錄。順治初,卒。

子明阿圖。睿親王多爾袞帥師入關,明阿圖攝梅勒額真為殿。累官都察院理事官、鑲藍旗蒙古副都統,授三等阿達哈哈番。順治八年,卒。

明阿圖子賽弼翰,初為簡親王濟度護衛。康熙四年,授護軍參領。從護軍統領瑚里布西御吳三桂將吳之茂,克陽平、朝陽諸關;趨保寧討王輔臣,克秦州。從平南將軍賴塔南討鄭錦,戰漳州,敗錦將劉國軒等。誅吳世璠,定雲南。累官鑲藍旗滿洲副都統,授拜他喇布勒哈番。二十九年,卒。

瑚什布,與固三泰同隸鑲藍旗,領牛錄。尋任侍衛,兼甲喇額真。天聰二年,從伐通古索爾和部,身被七創,戰益力,斬敵將,授備御世職。八年,從伐明,攻大同,與圖魯什等擊敗明總兵祖大弼;攻萬全左衛,擊敗明總兵曹文詔;復設伏邀擊,斬三十餘級,俘四人。九年,定蒙古旗制,瑚什布領鑲藍旗。崇德元年,從武英郡王阿濟格伐明,越明都,克定興。師還,部議出邊時不為殿,為敵所乘,士卒戰死者十人,罰白金六百,奪世職,罷固山額真,專領牛錄。三年,授理籓院副理事官。順治四年,復世職。遇恩詔,進二等阿達哈哈番。七年,卒。

子穆徹納。順治間,官護軍參領。從豫親王多鐸征蘇尼特部騰吉思,敗喀爾喀兵。從武英親王阿濟格討姜瓖,敗其將劉偉思等;攻寧武關,敗宜孟臣援兵;至左衛城,戰於吳家峪。從靖南將軍珠瑪喇定廣東,敗李定國於新會。累進三等阿達哈哈番。十三年,卒。

鄂莫克圖,納喇氏。自葉赫歸太祖,隸滿洲正藍旗。初為擺牙喇壯達。天聰元年正月,從伐朝鮮,克義州。五月,上自將伐明,攻寧遠。明總兵滿桂陣於城東,鄂克莫圖從諸將進戰,殪敵。三年,從貝勒岳託伐明,攻保安州,先登,克之,賜號「巴圖魯」,授備御世職,任甲喇章京。八年,從伐黑龍江虎爾哈部,計俘,為諸甲喇章京冠。崇德二年,復從伐卦爾察部,計俘如伐虎爾哈部時。

三年七月,授兵部理事官。九月,從睿親王多爾袞伐明,自青山口入邊,越明都,擊敗明太監馮永盛,克臨潼關,略地至濟南。四年七月,上遣使如明,命與努山等率兵護使者以行。五年,授噶布什賢噶喇昂邦。從伐明,圍錦州,擊敗明總兵祖大壽。六年,復圍錦州,擊敗明經略洪承疇。語見喀山傳。上軍松山、杏山間,明軍自松山潰遁,騎兵走杏山,步兵走塔山,鄂莫克圖先後邀擊,並有斬獲。七年,復從圍杏山,分兵略寧遠,掠牲畜。明總兵吳三桂以兵躡我師後,我師擊之,敗走,復益兵覘我師壘,鄂莫克圖與戰,窮追至連山,敵騎自沙河犯我師牧地,復奮擊破之。錦州既下,進二等參將。

順治元年,從入關,敗賊安肅,追之至慶都。尋率前鋒兵徇山西,敗賊絳州渡口。二年,從英親王阿濟格定陜西,敗賊延安。李自成走湖廣,追之至安陸,屢破賊壘,得戰艦三十,授一等甲喇章京。三年,從肅親王豪格下四川,敗賊漢中。逐張獻忠至西充,與護軍統領白爾赫圖等屢戰皆捷,加授半個前程。遇恩詔,進一等阿思哈尼哈番。十一年,授正藍旗滿洲副都統。十三年,致仕。康熙十二年,卒,年七十八。乾隆初,定封一等男。

喀山,納喇氏,世居蘇完。當葉赫未滅,挈家歸太祖,隸滿洲鑲藍旗,授牛錄額真。屢從伐明,下遼、沈有功,予游擊世職。天命九年,明總兵毛文龍以兵百人劫額駙康果禮莊,喀山率所部御之,斬二裨將,殲其眾。天聰六年,從伐察哈爾,與勞薩、吳拜率精銳前驅。林丹汗遁走。八年,進三等梅勒章京。目失明,辭牛錄。順治初,進二等昂邦章京。尋改二等精奇尼哈番。十二年,卒,謚敏壯。

子納海。初以喀山病目,命代領牛錄。旋授噶布什賢甲喇額真。從伐明,與席特庫等以步兵四千擊敗明陽和騎兵,斬級二百,獲馬六十餘;復設伏宣府,捕明邏騎。天聰九年,復從伐明,攻大同,命與布丹等駐上都城故址,詗軍事。尋命與鄂莫克圖等齎書諭明邊守將,歷喜峰口、潘家口、董家口諸隘,及還,斬邏卒百餘。

崇德二年,命與席特庫齎書諭明錦州守將祖大壽,自廣寧入邊,獲邏卒十二,斬其九,縱二人使齎諭以往,俘一人以還。四年,從武英郡王阿濟格伐明錦州,還報捷。復從上攻松山,明兵出戰,擊卻之。祖大壽遣兵自寧遠乘舟趨杏山,將入城;納海與瑚密色、索渾將兵擊其後,斬級五十,獲甲四十、舟一。又與瑚密色、席特庫等行略地,俘採薪者二十二人,牛、羊、騾馬無算。五年,從伐明,圍錦州,敵築臺城外,納海與色赫、布丹、蘇爾德將騎兵馳擊,斬四十人,復逐斬刈草者四十二人,敵來犯,屢擊卻之。與色赫等略小凌河,斬祖大壽所遣蒙古十七人。

六年,明總督洪承疇集各道兵赴援,次松山,與吳拜擊敗其騎兵。上自將攻松山,敵自杏山走塔山,與鄂莫克圖帥師邀擊,追至筆架山,斬級四百,俘二十八,得纛六,獲馬二百餘。七年,錦州下。敘功,予半個前程,命攝噶布什賢噶喇昂邦。從貝勒阿巴泰伐明,自黃崖口入長城,趨薊州,敗明總兵白騰蛟、白廣恩,遂略山東。明年,師還。以右翼諸將不俟左翼軍至,先出邊,功不敘。順治初,遇恩詔,進三等阿達哈哈番。及喀山卒,兼襲二等精奇尼哈番,例進二等伯。雍正中,從孫奇山,降襲一等阿思哈尼哈番。乾隆元年,定封一等男。

安達立,納喇氏。自葉赫歸太祖,隸滿洲正紅旗。太祖遣兵徇鐵嶺,刈其禾,有蒙古人降於明,出拒,安達立擊之走。事太宗,從貝勒薩哈璘駐牛莊。師攻永平,葉臣率二十四人冒矢石先登,安達立其一也。師還,從圖魯什偵敵建昌,夜戰,甲士有中矢墜馬者,援之出,擢噶布什賢章京。從伐明,攻崞縣,率所部先登;復以四十人伏忻口,敗敵,得纛三、馬五十餘。出邊,圖爾格擊敵潰竄,安達立邀擊,迫敵入壕,所殺傷過當。天聰九年,授牛錄章京世職,擢正紅旗蒙古梅勒額真。

崇德三年,從貝勒岳託伐明,將至墻子嶺,聞明軍備甚固,安達立與固山額真恩格圖率所部趨嶺右,陟高峰間道入邊,擊敗明軍。越燕京,略山東。明年,師自青山口出邊,復擊敗明軍。五年,圍錦州,屢戰皆捷。六年,復圍錦州,洪承疇援師至,與戰,破三營,至暮,敵潰,翌日復戰,又擊卻之。敘功,加半個前程。尋卒。

子阿積賴,襲職。順治初,從入關,逐李自成,戰於慶都。又從葉臣徇山西,署正紅旗蒙古梅勒額真。又從英親王阿濟格攻延安,逐自成至武昌,竄入九宮山,率師搜剿,殲其徒甚眾。四年,兼任刑部理事官。五年,署巴牙喇纛章京。從鄭親王濟爾哈朗征湖南,分兵徇道州,攻永安關。敘功,進一等阿達哈哈番兼拖沙喇哈番。卒。

綽拜,巴林氏。自葉赫歸太祖,隸蒙古鑲白旗,為牛錄額真。天聰八年,授世職牛錄章京。九年,與吳巴海伐瓦爾喀部,深入額赫庫倫、額埒岳索諸地,進三等甲喇章京。崇德三年,兼任戶部理事官。從睿親王多爾袞伐明,徇山東,至濟南,敵騎千餘拒戰,何洛會先眾奮擊,遂克其城。七年,從肅親王豪格圍明總督洪承疇於松山,承疇遣兵夜越壕攻鑲黃旗營,擊卻之。八年,進二等。順治初,從入關,破流賊,進一等。四年,加拖沙喇哈番。五年,授參領。從征南大將軍譚泰討叛將金聲桓,克饒州、南昌,師還,賚白金千、馬四十。七年,遷倉場侍郎。八年,授鑲白旗蒙古梅勒額真,兼工部侍郎。擢本旗固山額真,進一等阿思哈尼哈番。九年十二月,卒。

布丹,富察氏。自葉赫歸太祖,隸滿洲正紅旗,授牛錄額真。尋遷甲喇額真,領擺牙喇兵。天聰八年,從伐明,克萬全左衛城,先登,授半個前程。九年,從貝勒多鐸伐明,攻錦州,師還,明兵驟至,固山額真石廷柱所部有陷陣不能出者,布丹破陣援之出。旋命與納海等詗軍事。崇德元年,從武英郡王阿濟格伐明,破雕鶚、長安二隘,皆先登,與蘇納同功。轉戰至涿州,師還,明兵出居庸關,設伏邀我軍輜重,擊破之。四年,與沙爾瑚達等將土默特兵二百,略寧遠北境,以數騎挑戰,敵堅壁不出,乃俘其樵者以歸。五年,圍錦州,殺敵。語見納海傳。六年,與明兵戰松山、杏山,屢勝。錦州下,進牛錄章京世職。七年冬,復與

納海等從貝勒阿巴泰伐明。順治初,從入關,破流賊。敘功,並遇恩詔,進一等阿達哈哈番。九年,擢正紅旗蒙古副都統。十一年,卒,謚毅勤。

孫達哩,魯布哩氏。太祖取葉赫,以其民分屬八旗,孫達哩隸正黃旗。選充驍騎,遇戰必先,中創不為卻,屢得優賚。崇德三年,從睿親王多爾袞伐明,入自青山口,越明都,轉戰至山東,攻濟南,先登第一,賜號「巴圖魯」,授二等參將,領牛錄額真。順治間,累進二等阿思哈尼哈番,遷擺牙喇纛章京。從穆里瑪、圖海討李自成餘黨李來亨、袁宗第等,破茅麓山,有功。十二年,加太子少傅。十四年四月,卒,謚果壯。

吉思哈,烏蘇氏,世居瓦爾喀馮佳屯。初屬烏喇,見其貝勒不足事,與弟吉普喀達歸太祖,並授牛錄額真,隸滿洲正白旗。旋改隸鑲白旗。天命四年,從伐明有功,授游擊世職。六年,以甲喇額真帥師圍遼陽,樹雲梯先登。天聰八年,太宗追錄其功,進二等參將。是年,與甲喇額真吳巴海伐東海虎爾喀部,俘一千五百有奇,及牲畜輜重。九年,與梅勒額真巴奇蘭等伐黑龍江,收二千人以還,進一等參將。崇德元年,太宗自將伐朝鮮,聞明兵入鹼場,遣吉思哈率兵躡其後,擊敗之。二年,師既克朝鮮都城,上命旗出甲士十,並簡科爾沁、敖漢、柰曼、扎魯特、烏拉特諸部兵,俾吉思哈及理籓院承政尼堪為將,自朝鮮伐瓦爾喀,因擊破朝鮮軍,斬平壤巡撫,進略瓦爾喀,奏捷稱旨。語詳尼堪傳。累遷至吏部參政。三年四月,卒。子吉瞻,襲。

吉普喀達,吉思哈弟也。天命四年,授游擊。六年,任甲喇額真。從伐明,攻奉集堡,明總兵李秉誠赴援,師與戰,明兵走入城,師從之,至壕,城上發巨砲,吉普喀達中砲卒。天聰八年,贈二等參將。子瓜爾察,襲。

吳巴海,瓜爾佳氏,自烏喇歸太祖。太祖討尼堪外蘭,吳巴海實從。隸滿洲鑲藍旗,授牛錄額真。天聰元年四月,從貝勒阿敏伐朝鮮,攻義州,與梅勒額真阿山、穆克譚等先登,克之。五月,從太宗伐明,攻錦州,敵來犯,我師少卻,吳巴海為殿,督戰敗敵。五年,與梅勒額真蒙阿圖伐瓦爾喀,略額黑庫倫、額勒約索二部,收降人數千,上郊勞,賜宴,賜號「巴圖魯」。六年,從伐察哈爾,林丹汗西遁,上命吳巴海逐逋逃,斬察哈爾兵五,獲其馬及牲畜。旋率師伐烏扎喇,部署所將兵四道並進,會敵方漁於握黑河,吳巴海揮騎直前,斬三百餘人,得其輜重。七年,與牛錄額真郎格如朝鮮互市,得瓦爾喀部長族屬十五人以歸。八年,與吉思哈伐東海虎爾哈部。語詳吉思哈傳。十二月,復與牛錄額真景固爾岱將四百人伐瓦爾喀,降屯長分得裏,收阿庫里尼滿部眾千餘。師還,上命大臣迎勞,以所獲賚之。

九年,從貝勒岳託率師鎮歸化城。土默特人訐部長博碩克圖,謂其子陰遣使與明通,岳託遣吳巴海及甲喇額真阿爾津等四人要諸途,毛罕私以告,喀爾喀人潛遁,吳巴海追獲之,並得明使。毛罕者,博碩克圖子乳母之夫也,初從土默特來降,既而有叛志,號博碩克圖子為汗,自號貝勒。吳巴海既執喀爾喀使人,遂殺毛罕。十年,授梅勒額真,世職一等甲喇章京。

崇德元年六月,進三等梅勒章京,移鎮寧古塔。十二月,喀木尼漢部葉雷等叛,將其孥俱亡,吳巴海部兵逐之。行數十日無所見,見宿雁三,射之,一雁負矢飛且墮,往取之,見遺火,知逃者自此過。躡其跡,及之於溫多,獲其孥。葉雷入山,追及圍之,諭使降,不可,射之。葉雷將注矢,有狐起於前,觸葉雷弓,弓墜,遂射殺葉雷及其從者。師還,太宗命諸固山額真迎勞。二年,敘功,進三等昂邦章京,賜衣服、僕、馬、莊田。三年,與梅勒額真吳善帥師戍歸化城。旋坐匿罪人、徇廝養卒盜米,罷梅勒額真,論罰,四年卒,分世職為一等甲喇章京者一,為牛錄章京者二,授其子弟。

康喀勒,納喇氏,輝發貝勒王機砮之孫也。太祖時,偕從兄通貴率族屬來歸,隸滿洲鑲紅旗,授牛錄額真。天聰六年,從伐察哈爾部。八年正月,上以察哈爾林丹汗西遁,其部眾流散錫爾哈、錫伯圖,命康喀勒與岱青塔布囊等率蒙古及諸部駐牧兵往取以歸。五月,授世職牛錄章京。崇德三年,兼刑部副理事官。五年,擢鑲紅旗蒙古梅勒額真。六年,從伐明,圍錦州,並攻松山城。七年,松山、錦州皆下,復克塔山城。尋追論攻松山避敵、克塔山與固山額真葉臣爭功,罪當死,太宗特貰之。

順治初,從入關,擊李自成,加半個前程。尋從豫親王多鐸下江南,與固山額真準塔自徐州水陸並進,次清河。明總兵劉澤清遣部將馬花豹、張思義等率戰艦千餘、兵數萬,屯黃淮口。康喀勒與游擊範炳、吉天相等發砲毀其舟,分兵追擊,澤清走,淮安下。復與梅勒額真譚布擊明總漕田仰,仰方屯湖口橋,以三千人迎戰,擊破之;又戰於三里橋,逐至海岸,獲舟八十;又戰於如皋,攻通州,以雲梯克其城,旁近諸縣皆下。二年十一月,授鎮守江寧梅勒額真。時江北未定,群相聚為亂,江寧有諜為內應者,康喀勒與駐防總管巴山先期捕治,殺三十人而定。已而明潞安王硃誼石集眾二萬餘,分三道來攻,康喀勒等擊卻之。三年,以功進三等甲喇章京,世襲。四年,改三等阿達哈哈番。旋卒。子洛多,襲職。

和托,康喀勒從兄之子也。順治元年,以噶布什賢甲喇章京從入關,破李自成潼關,移兵下江寧。復從貝勒博洛徇浙江,破明總兵方國安等於杭州。復略福建,所向克捷。攻汀州,先登,克其城。論功,並遇恩詔,授拜他喇布勒哈番兼拖沙喇哈番,世襲。十一年,從征雲南,擊敗明將白文選,進取永昌,奪瀾滄江鐵索橋。康熙九年,卒。

瑪拉,和托弟也。順治十二年,以三等侍衛署甲喇額真。從固山額真伊勒德攻舟山,從擺牙喇纛章京穆成額破鄭成功兵於泉州;十六年,從安南將軍達素擊成功廈門:皆有功。康熙二十二年,卒。

通嘉,康喀勒兄孫。初襲其父莽佳三等阿達哈哈番。順治十八年,以護軍參領從靖東將軍濟什喀討山東賊於七。於七據棲霞、岠褭山為亂,其黨呂思曲、俞三等以數千人拒戰,通嘉擊敗之,賊遂以平。康熙六年,改前鋒參領。十四年,從信郡王鄂托討察哈爾布爾尼,師至達祿,布爾尼為伏山谷間,通嘉督所部盡擊殺之,布爾尼以三十騎遁。以功加拖沙喇哈番。旋坐事削。十八年,以護軍統領從討吳三桂,破譚弘於雲陽。二十三年,遷本旗蒙古都統。二十四年,卒。

薩璧翰,亦納喇氏。父三檀,自輝發率屬歸太祖,授牛錄額真,隸滿洲正藍旗。卒,薩璧翰與其兄薩珠瑚並授牛錄額真。太宗即位,以薩璧翰列十六大臣,佐正藍旗。天聰五年,擢戶部承政。八月,上自將伐明,圍大凌河,城兵出御,薩璧翰與戰,舍馬而步,逐敵薄壕。城上發砲矢,甲士巴遜沒於陣,薩璧翰力戰,入敵陣,取其尸還。八年五月,上自將伐察哈爾,貝勒濟爾哈朗居守,薩璧翰與梅勒額真蒙阿圖副之。考滿,授世職甲喇章京。崇德二年,從伐朝鮮,取皮島。師還,薩璧翰與其兄薩珠瑚發貝子碩託以廝役冒甲士請恤,坐論罰,薩璧翰初隸碩託,至是命改隸饒餘貝勒阿巴泰。旋以薩璧翰從子侍衛吳達禮從伐朝鮮,私役甲士,坐奪世職。三年,改吏部右參政。四年,授議政大臣。六年八月,從伐明,攻錦州,明援兵自松山至,誘戰,薩璧翰被創,卒於軍。

子漢楚哈、哈爾沁,皆授牛錄額真。哈爾沁從討吳三桂,從討噶爾丹,皆有功,授拖沙喇哈番。漢楚哈子哈爾弼,授一等護衛,從擊鄭成功,戰廈門,歿於陣,亦授拖沙喇哈番。

論曰:太祖初起,扈倫四部與為敵,四部之豪俊,先後來歸。武理堪等自哈達,武納格、阿什達爾漢、鄂莫克圖等自葉赫,吉思哈等自烏喇,康喀勒等自輝發,皆能效奔走,立名氏。武納格其先出自蒙古,遂為「白奇超哈」統帥,勛績尤著。四部有才而不能用,太祖股肱爪牙取於敵有餘。國之興亡,雖曰天命,豈非人事哉?


\end{pinyinscope}