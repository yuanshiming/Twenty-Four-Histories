\article{列傳二百六十二}

\begin{pinyinscope}
洪秀全

洪秀全,廣東花縣人。少飲博無賴,以演卜游粵、湘間。有硃九疇者,倡上帝會,亦名三點會,秀全及同邑馮雲山師事之。九疇死,眾以秀全為教主。官捕之急,乃往香港入耶穌教,藉抗官。旋偕雲山傳教至廣西,居桂平。時秀全妹壻蕭朝貴及楊秀清、韋昌輝皆家桂平,與相結納。貴縣石達開亦來入教。秀全嘗患病,詭云病死七日而蘇,能知未來事。謂:「上帝召我,有大劫,惟拜上帝可免。」凡會中人男稱兄弟,女稱姊妹,欲人皆平等,託名西洋教。自言通天語,謂天父名耶和華,耶穌其長子,己為次子。嗣是輒臥一室,禁人窺伺,不進飲食,歷數日而後出。出則謂與上帝議事,眾皆駭服。復造寶誥、真言諸偽書,密為傳布。潛蓄發,藏山菁間。嗾人分赴武宣、象州、藤縣、陸川、博白各邑,誘眾入會。

初,粵西歲饑多盜,湖南雷再浩、新寧李沅發復竄入為亂。粵盜張家福等各率黨數千,四出俘劫。秀全乘之,與楊秀清創立保良攻匪會,練兵籌餉,歸附者益眾。桂平知縣誘而執之,搜獲入教名冊十七本;巡撫鄭祖琛不能決,遂釋之。秀全既出獄,秀清率眾迎歸,招集亡命,貴縣秦日綱、林鳳祥,揭陽海盜羅大綱,衡山洪大全皆來附,有眾萬人。馮雲山讀書多智計,為部署隊伍、攻守方略。以歲值丁未,應「紅羊」之讖,遂乘勢倡亂於金田。褫鄭祖琛職,起前雲貴總督林則徐為欽差大臣往督師。則徐薨於途,以兩廣總督李星沅代之,赴廣西剿寇。寇竄平南恩旺墟,副將李殿元擊卻之,復回撲,巡檢張鏞不屈死;仍遁金田,星沅檄清江協副將伊克坦布往攻,被圍陣亡。星沅檄鎮遠總兵周鳳岐往援,戰一晝夜,斃寇數百,圍始解。上以寇勢日熾,命前漕運總督周天爵署廣西巡撫,乃請提督向榮專剿金田。

咸豐元年,秀全僭號偽天王,縱火焚其墟,盡驅眾分擾桂平、貴、武宣、平南等縣,入象州。上命廣州副都統烏蘭泰會討,以大學士賽尚阿為欽差大臣,率都統巴清德、副都統達洪阿馳防。烏蘭泰至象州,三戰皆捷,疏言:「粵西寇眾皆烏合,惟東鄉僭號設官、易服蓄發有大志,兇悍過群盜,實腹心大患。」周天爵主滾營進偪,驅諸羅淥洞盡殲之,向榮不謂然。檄貴州鎮總兵秦定三移營大林,堵北竄象州道,定三亦不奉命。四月,秀全自大林逸出走象州,犯桂平新墟。賽尚阿增調川兵,募鄉勇,合三萬人,分兵要隘。一日戰七勝,斬捕二千,寇仍遁新墟。七月,竄紫金山,山前以新墟為門戶,後以雙髻山、豬仔峽為要隘。巴清德與川、楚鄉勇出其後,上下奪雙髻山,寇大潰,屯風門坳。向榮率諸軍三路攻之,陣斃韋昌輝弟韋亞孫、韋十一等,始遁走。我軍追之,會大雨,軍仗盡失。

閏八月,寇分二路東走藤,北犯永安,陷之,遂僭號太平天國。秀全自為天王,妻賴氏為後,建元天德。以秀清為東王,軍事皆取決,蕭朝貴西王,馮雲山南王,韋昌輝北王,石達開翼王,洪大全天德王;秦日綱、羅亞旺、範連德、胡以晃等四十八人任丞相、軍師偽職。時官軍勢勝,寇知不可敵,有散志。秀清獨建策封王以羈縻之,勢燼而復熾。九月,大軍移陽朔,會攻永安,賊分屯莫家村。烏蘭泰建中軍旗於秀才嶺,上植一紅蓋,下埋地雷,誘敵燔殺四千,大軍乘之,遂克莫家村。

二年正月,大兵圍永安,毀東、西砲臺。二月,石達開分兵為四,敗我軍於壽春營,進破古束沖、小路關。偽丞相秦日綱由水竇屯仙回嶺。烏蘭泰分兵夾擊,斃寇數千,擒偽天德王洪大全,檻送京師,磔之市。時大雨如注,烏蘭泰提精卒入山,山路濘滑,寇乘我軍陣未定,短兵沖突,遂大敗。秀全從楊秀清謀,由瑤山、馬嶺間道徑撲桂林。烏蘭泰率敗卒追之城南將軍橋,受重創,卒於軍。三月,賊徑趨廣西省城。向榮先一時繞道至省,寇踵至,已有備,相持不能拔,解圍而北。

馮雲山、羅大綱先驅陷興安、全州,將順流趨長沙。浙江知縣江忠源御之蓑衣渡,馮雲山中砲死,寇退走道州。道州俗悍,多會匪,所至爭為效死,勢復張。六月,陷江華、寧遠、嘉禾。七月,陷桂陽州,江忠源躡至,一戰走之,趨郴州。蕭朝貴以膽智自豪,謂群寇遲心耎,又詗長沙守兵單,可襲而取也,乃率李開芳、林鳳祥由永興、茶陵、醴陵趨長沙,設幕城南。八月,蕭朝貴攻南門,官兵擊之,殪,尸埋老龍潭,後起出梟其首。秀全聞朝貴死,自郴州至,督攻益急,九月,掘隧道攻城,屢不獲逞。

十月,秀全於南門外得偽玉璽,稱為天賜,脅眾呼萬歲。遂夜渡湘水,由回龍塘竄寧鄉,抵益陽。擄民船數千,出臨資口,渡洞庭,陷岳州。城中舊儲吳三桂軍械,至是盡以資寇。寇入長江,旬日間奪五千艘,婦孺貨財盡驅之滿載。秀全駕龍舟,樹黃旗,列巨砲,夜則張三十六燈,他船稱是,數十里火光不絕如晝,遂東下,十一月,陷漢陽。十二月,攻武昌。時楊秀清司軍令,李開芳、林鳳祥、羅大綱掌兵事。值武漢二江屆冬水涸,乃擄船作浮橋,環以鐵索,直達省城,分門攻之。向榮馳至,約城內夾攻,巡撫常大淳慮城啟有失,不許。地雷發,城遂陷。秀全出令,民人蓄發束冠巾,建高臺小別山下,演說吊民伐罪之意。

三年,上以賽尚阿久無功,授兩廣總督徐廣縉為欽差大臣。時石達開攻武昌,廣縉逗留岳州不敢進,上責其罪,更以向榮為欽差大臣,日夜攻之急,寇棄武昌駕船東下,眾號五十萬,資糧、軍械、子女、財帛盡置舟中,分兩岸步騎夾行,進向九江,下黃州、武昌、蘄水等十四州縣;抵廣濟縣,下武穴鎮。兩廣總督陸建瀛率兵二萬餘、船千五百艘上溯,遇寇不戰而走,前軍盡覆,建瀛狼狽還金陵。寇薄九江而下,收官軍委棄砲仗,破安慶,巡撫蔣文慶死之。寇奪銀米無算,水陸並進,抵金陵,沿城築壘二十四,具戰船,起新州大勝關迤邐至七洲里止,晝夜環攻,掘地道壞城,守兵潰亂。建瀛易服走,為寇所戕。將軍祥厚偕副都統霍隆武等守滿城,二日城陷,皆死之。城中男女死者四萬餘,閹童子三千餘人,水曳守城之忿。

秀全既破金陵,遂建偽都,擁精兵六十餘萬。群上頌稱明代後嗣,首謁明太祖陵,舉行祀典。其祝詞曰:「不肖子孫洪秀全得光復我大明先帝南部疆土,登極南京,一遵洪武元年祖制。」軍士夾道呼漢天子者三,頒登極制誥。大封將卒,王分四等,侯為五等。設天、地、春、夏、秋、冬六官丞相為六等,殿前三十六檢點為七等,殿前七十二指揮為八等,炎、水、木、金、土正副一百將軍為九等,炎、水、木、金、土九十五總制為十等,炎、水、木、金、土正副一百監軍為十一等,前、後、左、右、中九十五軍帥為十二等,前、後、左、右、中四百四十五師帥為十三等,前、後、左、右、中二千三百七十五旅帥為十四等,前、後、左、右、中一萬一千八百七十五卒長為十五等,前、後、左、右、中四萬七千五百兩司馬為十六等;又自檢點以下至兩司馬,皆有職同名目。其制大抵分朝內、軍中、守土三途:朝內官如掌朝門左右史之類,名目繁多,日新月異;軍中官為總制、監軍、軍帥、師帥、旅帥、卒長、兩司馬,凡攻城略地,嘗以國宗或丞相領軍,而練士卒,分隊伍,屯營結壘,接陣進師,皆責成軍帥,由監軍總制上達於領兵大帥以取決焉,其大小相制,臂使指應,統系分明,甚得馭眾之道;守土官為郡總制、州縣監軍、鄉軍帥、鄉師帥、鄉旅帥、鄉卒長、鄉兩司馬,凡地方獄訟錢糧,由軍帥、監軍區畫,而取成於總制,民事之重,皆得決之。

自都金陵,分兵攻克府、、州、縣,遂即其地分軍,立軍帥以下各官,而統於監軍,鎮以總制,監軍、總制受命於偽朝。自軍帥至兩司馬為鄉官,鄉官者以其鄉人為之也。軍帥兼理軍民之政,師帥、旅帥、卒長、兩司馬以次相承,皆如軍制。此外又有女官,曰女軍師、女丞相、女檢點、女指揮、女將軍、女總制、女監軍、女軍帥、女卒長、女管長,即兩司馬也,共女官六千五百八十四人。女軍四十,女兵十萬人,而職同官名目亦同。總計男女官三十餘萬,而臨時增設及恩賞各偽職尚不在此數也。

其軍制,每一軍領一萬二千五百人,以軍帥統之,總制、監軍監之。其下則各轄五師帥,各分領二千五百人。每師帥轄五旅帥,各分領五百人。每旅帥轄五卒長,各分領百人。每卒長轄四兩司馬,每兩司馬領伍長五人,伍卒二十人,共二十五人。

其陣法有四:曰牽陣法。凡由此至彼,必下令作牽陣行走法。每兩司馬執一旗,後隨二十五人。百人則間卒長一旗,五百人則間旅帥一旗,二千五百人則間師帥一旗,一萬二千五百人則間軍帥一旗,軍帥、監軍、總制乘輿馬隨行。一軍盡,一軍續進。寬路則令雙行,狹路單行,魚貫以進。凡行軍亂其行列者斬。其牽線行走時,一遇敵軍,首尾蟠屈鉤連,頃刻岔集。敗則聞敲金方退,仍牽線以行,不得斜奔旁逸。曰螃蟹陣。乃三隊平列陣也。中一隊人數少,兩翼人數多。其法視敵軍分幾隊,即變陣以應之。如敵軍僅左右隊,即以中隊分益左右,亦為兩隊。如敵軍前後各一隊,則分左右翼之前鋒為一隊,以後半與中一隊合而平列,為前隊接應。如敵軍左右何隊兵多,則變偏左右翼以與之敵。如敵軍分四五隊,亦分為四五隊次第迎拒。其大陣包小陣法,或先以小隊嘗敵,後出大陣包之;或詐敗誘敵追,伏兵四起以包敵軍,窮極變化。至於損左益右,移後置前,臨時指揮,操之司令,兵士悉視大旗所往而奔赴之,無敢或後。曰百鳥陣。此陣用之平原曠野,以二十五人為一小隊,分百數十隊,散布如星,使敵軍驚疑,不知其數之多寡。敵軍氣餒,即合而攻之。曰伏地陣。敵兵追北至山窮水阻之地,忽一旗偃,千旗齊偃,瞬息千里,皆伏地不見。敵軍見前寂無一卒,詫異徘徊。賊伏半時,忽一旗立,千旗齊立,急趨撲敵,往往轉敗為勝。

其營壘或夾江、夾河、浮筏、阻山、據村市,及包敵營,為營動合古法。每數營必立一望樓了敵。守城無布帳,每五垛架木為板屋。木墻、土墻亦環庋板屋。地當敵沖,則浚重壕,築重墻,壕務寬深,密插竹簽。重墻用雙層板片,約以橫木,虛其中如衣復壁,中填沙石專土。築二重墻築物無定,或密排樹株,或積鹽包、糖包,及水浸棉花包,異常堅固。其攻城專恃地道,謂之鼇翻。土營而外,又有木營、金營。組織諸匠,各營以指揮統之。其總制至兩司馬皆如土營之制。立水營九軍,每軍以軍帥統之。但未經訓練,不能作戰,專以船多威敵而已。

其旗幟亦有差等,偽東王黃綢旗,紅字綠緣,方一丈;以下皆黃綢旗、紅字,而以緣別。如偽西王白緣,偽南王紫緣,偽北王黑緣。偽翼王藍緣,其尺丈長闊則以五寸遞減。豫王、燕王皆黃綢尖旗、紅字、水紅緣,國宗黃綢尖旗、紅字。其緣視何王國宗,即從何色,皆長闊八尺。侯,黃綢尖旗,長闊七尺八寸。丞相,黃綢尖旗,長闊七尺五寸。檢點,黃綢尖旗,長闊七尺。以上皆紅字、水紅緣。指揮,黃綢尖旗,黑字、水紅緣,長闊六尺五寸。將軍至兩司馬,皆黃旗無緣,形尖,黑字,自長闊六尺以下遞減至二尺五寸。每一軍大小黃旗至六百五十六面之多。

軍中號令,惟擊鼓、敲金、吹螺、搖旗。凡起行出隊,俱擂鼓、吹螺、搖旗以集眾。打仗則擊鼓吶喊,收隊則鳴鉦。有老軍、新軍、童子軍。尤善用間諜,混入敵營。又能取遠勢,聲東擊西,就虛避實,其以進為退,以退為進,往往令人不測,墮其術中。此其行軍之大略也。

其服色尚黃。偽天王金冠,雕鏤龍鳳,如圓規沙帽式,上繡滿天星斗,下繡一統山河,中留空格,鑿金為「天王」二字。東王、北王、翼王冠如古兜鍪式,冠額繡雙龍單鳳,中立金字職銜。國宗略同諸王式。自檢點至兩司馬,皆獸頭兜鍪式,帽上龍以節數分等差。如諸王九節,侯相七節,檢點、指揮、將軍五節,總制、監軍、軍帥三節是也。袍服則黃龍袍、紅袍、黃紅馬褂。偽天王黃緞袍,繡九龍。自諸王以下至侯相,遞減至四龍。檢點素黃袍,指揮至兩司馬皆素紅袍。自偽王至兩司馬,皆繡職銜於馬褂前團內。儀衛輿馬,諸王皆黃緞轎繡雲龍,侯、相、檢點、指揮皆紅緞轎,將軍、總制、監軍綠轎,軍帥、師帥、旅帥藍轎,卒長、兩司馬黑轎。

至金陵,始建宮室,毀總督署,復擴民居以廣其址,役夫萬餘,窮極奢麗。雕鏤螭龍、鳥獸、花木,多以金為之。偽王皆建偽府,馮雲山、蕭朝貴早授首,其子亦襲封建府。其宗教制度,半效西洋。日登高殿,集眾演說,與人民以自由權,解婦人拘束。定偽律六十二條,最為慘酷。然行軍嚴搶奪之令,官軍在三十里外,始準擄劫;若官軍在前,有取民間尺布、百錢者,殺無赦。於安慶大星橋設榷關,撥砲船十艘,環以鐵索,木筏橫截江濱,以防偷漏。九江、蕪湖,及沿江州縣岔河、小港地當沖要者,一律設立偽卡,徵收雜稅。此其建國大略也。

既都金陵,欲圖河北,羅大綱曰:「欲圖北必先定河南。大駕駐河南,軍乃渡河,至皖、豫一出。否則先定南九省,無內顧憂,然後三路出師:一出湘、楚;一出漢中,疾趨咸陽;以徐、揚席卷山左,再出山右,會獵燕都。若懸軍深入,犯險無後援,必敗之道也。且既都金陵,宜多備戰艦,精練水師,然後可戰可守。若待粵之拖罟咸集長江,則運道梗矣。今宜先備木筏,堵截江面,以待戰監之成,猶可及也。」秀清方專政,不納。乃遣偽丞相林鳳祥、李開芳、羅大綱、曾立昌率眾東下,秀全詔之曰:「師行間道,疾趨燕都,無貪攻城奪地糜時日。」大綱語人曰:「天下未定,乃欲安居此都,其能久乎?吾屬無類矣!」

二月,林鳳祥等陷鎮江、揚州,令吳如孝等留守,分據浦口、瓜洲諸隘。向榮既復武昌,躡寇而東,抵金陵,軍孝陵衛,是謂江南大營。都統琦善亦以欽差大臣率直隸、陜西、黑龍江馬步諸軍軍揚州城外,是謂江北大營。三月,向榮破通濟門寇壘,襲占七橋甕,奪獲鍾山圍,殲寇無算,遂移大營逼城而軍。四月,漕運總督楊殿邦進攻揚州,毀城外木城土壘,東路寇悉避入城。琦善、勝保先後督攻,五戰皆捷。鳳祥留立昌踞揚城,驅婦女及所劫貨財運回金陵;率三十六軍北竄,分擾滁州,踞臨淮關,陷鳳陽府。其酋硃錫錕、黃益蕓等別率悍黨犯浦口,攻六合,知縣溫紹原率鄉團拒之,夜火其營,寇遁回金陵。五月,大兵圍揚州,殺敵逾萬。勝保自揚州躡其後,力攻鳳陽,寇遁河南。

楊秀清遣偽丞相吉文元由浦口竄亳州,偕林鳳祥陷永城,犯開封。省官兵擊破之,又敗之汜水。寇奔黃河渡口,溺死無算。楊秀清遣偽豫王胡以晃陷安慶,又遣偽丞相賴漢英、石祥禎攻九江、湖口,進圍南昌。湖北按察使江忠源馳援江西,入城固守。鳳祥等自汜水敗退,犯鄭州、滎陽。六月,圍懷慶,以地道攻城,不克。鎮江寇出城撲我軍,戰北固山下,伏寇縱火,七營皆被焚。鄧紹良退守丹陽,都司劉廷鍈等督潮勇馳援。寇退入城,復擾丹徒鎮,劉廷鍈復擊退之。向榮檄總兵和春與劉廷鍈扎徒陽運河之新豐鎮,寇始不敢南竄,常州獲安。寇之圍懷慶也,立木柵為城,深溝高壘,我兵相持幾至六旬。訥爾經額親督諸將分五路攻壘,毀其木柵,斃敵酋吉文元。鳳祥受重創,解圍而遁,河北肅清。

八月,鳳祥竄山西,陷平陽,直抵洪洞;竄直隸,踞臨洺關,擾至深州。賴漢英等解南昌圍,入湖北,踞田家鎮之半壁山。九月,踞入楚要隘,水陸並進,陷黃州。其竄深州者,旁擾欒城。十月,竄天津,踞靜海,屯獨流、楊柳青諸鎮。漢陽之寇,分股北竄:一陷孝感、黃陂諸縣,一由應城犯德安府,為防兵所遏,合眾退黃州。秦日綱等陷安徽桐城、舒城,侍郎呂賢基死之。舒城既失,賊遂徑撲廬州,陷之。廬州者,安徽文武大吏之所僑寓以為省治者也。十一月,秀全以揚州、鎮江攻圍急,遣賴漢英等領江西眾,糾合儀徽黨援揚州;又令由安徽寧國灣沚進薄高淳湖,窺伺東壩,圖解鎮江之圍,我軍均擊退之。寇復由三汊河進撲,死戰不退。揚州寇曾立昌突出,與賴漢英同竄瓜洲。

上以寇擾長江,非立水師不能制其死命,乃命在籍侍郎曾國籓練鄉勇、創水師討寇。初,寇圍南昌,城外寇壘僅文孝廟數座,官軍屢攻不能克。郭嵩燾偶獲諜訊之,則寇皆舟居,其壘則環三面築墻而虛其後,專蔽舟楫而已。嵩燾因與江忠源議曰:「東南州縣多阻水,江湖遇風,一日可數百里。官軍率由陸路躡寇,其勢常不及。長江數千里之險獨為敵有。且寇上犯以舟楫,而官軍以營壘御之,求與一戰而不可得,宜寇勢之日昌也。」忠源即具疏請飭湖南北、四川仿廣東拖罟船式,各造戰艦數千,飭廣東制備砲位以供戰艦之用,並交曾國籓督帶部署,奉旨允行。國籓遂治戰船於衡湘,至是始成。共募水勇四千,分為十營;募陸勇五千,亦分十營。以塔齊布為軍鋒。國籓親統大軍發衡州,水陸夾江而下。

初,鎮江、揚州、儀徵、瓜洲四處寇互相應援,不得破。十二月,琦善以揚州寇退,瓜洲勢孤,督軍攻復儀徵,乘勝追抵瓜洲。楊秀清遣胡以晃率黨十餘萬攻廬州,巡撫江忠源晝夜抵御,以眾寡不敵,城陷,死之。四年正月,黃州寇張燈高會,總督吳文鎔出其不意襲之,會大雪罷戰。越數日,秀清分兵設伏山崗,命其黨率城軍撲營,文鎔拒戰,伏起火發,十三營皆潰,文鎔死之。賊乘勝遂陷漢陽。二月,揚州軍進剿瓜洲,總兵瞿騰龍陣亡。寇遣偽將孫寅山陷太平府,踞為巢。自瓜洲結壘屬於江,以達金陵,往來不絕。秀清復遣石祥禎會漢、黃寇黨溯江直上,陷岳州,溯流至銅官渚,逼近長沙。曾國籓邀之靖港,而寇已由間道襲湘潭,副將塔齊布率兵千三百同水師血戰五晝夜,斃寇數萬。論者謂微此戰,寇溯湘源以達粵,直下金陵,首尾一江相貫注,大局不可支矣。

是月,參贊大臣僧格林沁攻克獨流寇巢,靜海寇復竄踞阜城。僧格林沁攻毀堆村、連村、林家場三處寇壘,擒殺偽指揮、監軍以下一百餘人,悉遁入城。秀全念河北不能支,遣皖黨由豐豆工偷渡黃河,竄山東金鄉,進撲臨清州,冀抒阜城之困。三月,寇以地雷陷城,尋為我軍攻復,竄冠縣、鄆城,復據曹縣,築木城拒守。四月,勝保破其巢,追至漫口支河,逼溺水,偽丞相曾立昌、許宗揚皆溺死。偽副丞相陳世保已先於冠縣燒斃,悉數殲除。踞阜城者即於是日全股南竄入連鎮。僧格林沁及勝保會軍合剿破之,誅林鳳祥;復破之高唐州馮官屯,生擒李開芳,磔之京市。河北肅清,是後不復北犯,我軍遂無後顧憂。

初,長江為寇往來道,荊州當四路之沖,至省道梗,特召荊州將軍官文統軍討寇。時沔陽、安陸、荊門、監利、京山、天門均陷,進窺荊州。雲南普洱營游擊王國才奉調至,一戰敗之,重鎮始安。並克復監利、宜昌,寇遁洞庭湖,合股犯常德府。先是李侍賢常與陳玉成、李秀成謀解金陵圍,犯江西、福建。偽啟王梁成先犯陜西,後與捻合,欲犯湖南、河南,而陳玉成志在武昌、漢陽,乃領一隊入梁子湖達武昌,渡江分犯,以全力圖武昌,六月,陷之,並踞漢陽。巡撫青麟自縊不死,棄城走,尋正法。秀全以秦日綱留守武昌,授玉成偽殿右十八指揮;還陷田家鎮,破廣濟、黃梅,連陷九江,升偽殿右三十檢點。

楊秀清雖在軍,而金陵之事一決於己,驛騎絡繹,多稽時日。向榮軍孝陵衛,稱江南勁旅,秀清憂之,既克武昌,遂馳還金陵,命石達開代守武漢。官文自荊州下剿,克沔陽。初,寇欲先取長沙,踞上游為破竹之勢,而韋志俊略湘潭不得志,退踞岳州,築壘毀橋,意圖久抗。我軍水師設伏誘敗之,七月,復岳州。寇由城陵磯來犯,我軍分五路迎擊,斃偽丞相汪得勝等二人,獲船七十六,殲賊千餘人。塔齊布陣斬悍酋偽丞相曾天養。閏七月,寇奔城陵磯,塔齊布策馬率湘勇直入,毀營十三,斃二千人。陸軍既勝,曾國籓飭李孟群率水師追剿,荊河東、西兩岸寇壘悉夷。自此由荊入川,由岳州入湘,門戶始固。初,武昌失陷,上以楊霈代總督,臺勇克京山、安陸,復天門,生擒孔昭文等正法。餘皆下竄,踞沔陽州仙桃鎮。是月破其巢,並收復下游孝感、黃州、麻城諸縣。寇悉竄黃州。

時金陵寇分股嘯聚於太平府,與鎮江遙為應援。向榮分兵四隊擊之,斃其偽國宗韋得玲、偽檢點陳贄見、偽將軍李長有、偽總制吳春和,遂復府城。楊秀清自率戰,圍軍不利,三路皆潰;退入城,謂韋昌輝等曰:「江南大營不走,吾輩無安枕日矣!現其勢方銳,不可敵也。當乘其罷徐圖之。」金陵寇以乏糧,驅婦女之老而無色者出城,聽其自散。盡取年十五以上、五十以下之婦女,指配給眾,不從則殺之,守志者多自盡,死者萬計。八月,總督楊霈收復黃州府屬蘄水、廣濟、羅田諸縣。曾國籓自克岳州後,議乘勝東下,先與塔齊布會攻崇陽,克之,生擒偽丞相金之亨等十一人。惟廖二逃竄,復句結餘黨,重失縣城。國籓親督水陸諸軍攻武昌、漢陽。寇守城之法,不守陴而守險,洪山、花園兩路皆精銳所在。大軍自螺山下剿,楊載福等率水師,羅澤南率陸師,三路同進,連克寇壘,焚毀敵船數千。李孟群、塔齊布進薄武昌,寇宵遁。楊昌泗亦攻漢陽,克之。黃州府城、武昌縣均收復。九月,提督和春敗寇廬江,擒偽監軍任大綱等十七名。

下游知官軍分路進剿,乃由田家鎮糾黨六千餘,一由興國分抄大冶以拒武昌軍,一踞興國以拒金牛軍。羅澤南馳至興國,敗之,克州城。塔齊布赴大冶,擊斃千餘。彭玉麟、楊載福抵蘄州,燒寇船九十餘艘。十月,楚軍攻半壁山,寇置橫江鐵鎖四道,攔以木牌,遍列槍砲。楊載福等率水師至田家鎮,會陸師進攻,乘風縱火,破其壘,燔舟一萬有奇。陳玉成棄蘄州竄陷廣濟,聯合秦日綱、羅大綱等分扼要隘。塔齊布渡江追之,收復廣濟。寇退踞黃梅,黃梅為湖北、江西、安徽三省總匯之區。寇死拒,以萬餘守小池口抗水師,以數萬拒大河埔,以萬餘扎北城外,又以數千游弋聯絡之。塔齊布與羅澤南登山下擊兜殺,陳玉成縋城而逸,遂克黃梅。玉成自請罪,而秀成反加偽勛號曰成天裕。

時捻匪蜂起,粵寇與之聯合,或令分擾,或令前驅,以牽制我軍。秀成由廬州踞舒城,並扼桐城大、小二關,阻我南路之師。二關為安慶通衢,屢復屢失。京堂袁甲三檄參將劉玉豹、舉人臧紆青等戰奪兩關,斬其目吳鳳珠等十二名,進抵桐城。廬江寇糾安慶黨來援,我軍兜戮殆盡,而潛山援寇復至,臧紆青歿於陣。

十一月,國籓進軍九江。玉成自黃梅敗後,復糾安慶新到之眾踞孔壟驛、小池口,與對岸九江相句連。李孟群七戰七捷。塔齊布與羅澤南等由北岸進至濯港,進攻孔壟驛,破土城,縱火焚街市,寇無得脫者。小池口寇聞之,亦遁。乃調陸軍攻九江,水師乘勝攻湖口,大綱趨救,大戰梅家洲,毀小河簰船、沙洲橋壘。十二月,蕭捷三率水師馳入鄱陽湖內,追至大姑塘。石達開聯船為卡斷其後,捷三不能返,遂與外江水師隔絕。達開潛以小舟馳襲國籓坐船,國籓跳入羅澤南營以免。大軍之攻九江也,敗寇收合潰散,分三道東陷黃梅。值歲除,潛至廣濟,火楊霈大營,霈突圍出,不敢入武昌,走保德安。

五年正月,江蘇巡撫吉爾杭阿克上海縣,縣自三年秋陷於賊,至是始復。秀全令皖寇大舉犯湖北,中道自小池口沿江陷黃、蘄;復分黨從富池渡江西,陷興國、通城、崇陽、咸寧、通山,且掠江西武寧,所至脅眾以行。湖北巡撫陶恩培甫蒞任數日,時總督在外,未及議守備。城中兵僅二千,徵兵半途聞警皆潰去。湖北、江西方千里,旬日騷然矣。始寇之起,所行無留難。其踞省府,脅取民財米。行道掠人夫,不用則遣還,未嘗增眾。及屢敗,乃結土寇屯城鎮,頗收拔悍鷙者,而任用石達開、陳玉成等,極稱得人之盛。自漢口進襄河,上犯漢川,擾沔陽,進犯武昌,踞漢陽府城。沿江築壘,並於漢陽下南岸嘴高築砲臺,以阻下游之師。時江西寇入腹地陷饒州。國籓親至南昌,修整內湖水師,與羅澤南陸軍相依。

二月,韋國宗等攻陷武昌,巡撫陶恩培等死之。寇溯漢江而上,以岳家口、仙桃鎮為老巢。上以胡林翼巡撫湖北,國籓進吳城鎮,屢書與議東南大勢,以武昌據金陵上游,宜厚集兵力為恢復計。四月,陷德安府,楊霈退走襄陽,上褫其職,以官文為湖廣總督。國籓屯南康,思整軍出江謀進取,然寇已由都昌陷饒州,別由東流、建德窺樂平,屯景德鎮,東犯祁門、休寧諸處。而湘軍僅萬餘人,水陸分為四:李孟群等水師回援武昌,塔齊布留攻九江,羅澤南入江西攻饒州,國籓收蕭捷三水師三營屯南康。羅澤南奔走往來,克廣信府,收景德。寇之踞徽州者,與土匪相結,據險以抗我軍。浙軍出境擊寇,復徽州,乘勝克休寧、黟縣、婺源,生擒偽將軍、兩司馬等八名。秀全命北固山、鎮江、瓜洲、金山四路,約期進犯揚州。托明阿伏兵九洑洲,迎擊破之,斷鐵鎖船鍊,焚船三百。諸路寇被創而遁。饒州之寇分據樂平、德興、弋陽,江西軍率水陸師往剿。寇出五隊來撲,不克而奔,郡城立復。秀全以金陵山三山為濱江要區,以精卒守之,水師不能上駛。托明阿督水師總兵吳全美沿江掃蕩,焚船二百餘艘,獲拖罟、快蟹等船二十五艘、大小砲八十餘尊,生擒偽先鋒陳長順等六十一名。吳全美乘勢上山,蹋平營卡。江西肅清,水師始棹行無阻。

五月,秀全謀襲金口,斷楚軍糧道。林翼督軍屢戰,斬其偽丞相陳大為等,進屯紙坊,逼省城小東門。寇潛自他門出掠。林翼建議先攻漢陽,扼溳口、蔡店要隘,絕竄湘之路;開濬江堤,以水師腹背攻之,則漢陽可破,而鄂省咽喉已通,不難並力於武昌矣。初,寇由府河入湘,所過州邑悉殘破無完土,復為官文伏軍所狙擊,分途潰退。六月,收復云夢、應城,二城者府河出入要道也,寇失之,大恐。我軍進攻德安,斷其出入,寇始不敢窺伺荊襄。七月,塔齊布卒於軍。寇陷義寧,國籓遣羅澤南出奇兵復之。寇嚴守襄河蔡店,上通德安,下達漢鎮,互為應援。十二月,彭玉麟克蔡店,水陸並進,毀襄河鐵索浮橋,蹋平南岸敵巢,而下游塘角、漢陽、大別山營壘焚毀殆盡,德安之寇益蹙。林翼既克蔡店,而漢川為蔡店後路,寇據此游行沖突,德安亦資以通聲氣。林翼與官父會軍克復漢川,武漢首尾始聯絡一氣。

蕪湖之陷幾二載矣,江、皖往來道梗。寇以為上下關鍵,水則聯艦,陸則砌臺,我軍屢攻不能拔。是月,向榮督軍分道擊之,縣城始復。楚南軍亦攻復湖口、都昌。八月,按察使李孟群守金口,崇寧寇句結武昌城黨分道來撲,陸營失守,林翼亦敗於奓山,退保大軍山。寇勢復熾,分擾漢陽,並繞道襲陷漢川。九月,官文、林翼檄調羅澤南援武漢,澤南上書請率所部以行,謂:「得武昌乃可控制江、皖,屏蔽江西,而後內湖外江聲息可通,攻九江始操勝券。」國籓從之,乃部署援師五千人,自義寧趨通城。寇聞我軍至桂口,分眾來援,設木城重壕自固。澤南會軍克之,進攻崇陽。桂口寇退入崇陽,密約通山來援。桂口與湖南、江西、湖北交界,形勢奧衍,米糧充足。曩偽丞相鍾酋義寧敗後,踞此修土木城,跨山引澗,袤斜六里,欲踞一隅以掣三省之師,伺隙而動。澤南移得勝之兵先奪是隘,進克崇陽,焚寇壘,馳赴羊樓司扼敵上竄。

十月,克復廬州。廬州陷已三年矣,守之者為偽豫王胡以晃,與我軍大小數百戰,死傷萬餘,皆受創而去,是月始克之。其據德安者,眾不過數千,恃武漢為下游奧主,襄、府二河群蟻聚。我軍勝東挫西,疲於奔命。至是官文督兵力戰,守城寇黨陸長年、馬超群潛赴大營投誠,約為內應。值大風雨,放火開門納師,遂復其城。時寇之牽制我軍者三路:自隨、棗至襄陽為北路,武昌上下為南路,漢川中路。尾潛、沔,首德安為尤要,屢收屢陷。官文督軍分四道齊集漢川,克之,遂率兵東下,與林翼合謀武漢。石達開自安慶率三萬人上援武昌。澤南會林翼夾攻,連克蒲圻、咸寧;至金口,會攻武昌,破城外敵壘,駐軍洪山。寇之踞武昌者,城外大壘八、小壘二。林翼與戰,澤南襲之,破大壘一、小壘二。李孟群亦薄攻漢南,與官文軍相聲援。水師往來南北燒敵船,都興阿以馬隊護之。群帥輯和,寇益不得逞。漢陽城外自龜山沿河而下,敵船林立。上游入江之梁子湖,下游金牛鎮,群寇赴援。水陸各軍督團勇犁巢掃穴,武漢外患至此盡除。

秀全以瓜、鎮屢挫敗,圖往援,十一月,出龍脖子等處。向榮飭張國樑敗之仙鶴門、甘家巷,又戰七霞街,斃偽丞相周少魁等四十名;追至石埠橋,馘二千餘,逃入城不敢復出。秀全於對江九洑洲築石壘,浚深壕,悉銳守之,為金陵屏蔽。寇竄江北,以此為出路,屢攻之未下也。六合知縣溫紹原克其壘,後復為寇據,同治二年始復之。瓜洲、鎮江一水相望,兩城往來無阻,並時有合竄揚州圖北犯意。揚州軍與瓜洲相持已二年餘,托明阿以日戰無效,諭士民築長圍於瓜洲之北以扼之,至是圍成。寇水路分撲,大敗去,奪其簰船,生擒偽參護鄭金柱等十名。吉爾杭阿既克上海,移軍鎮江,是月營小九華山。又於黃鶴山、京畿嶺築城置砲臺以逼之,並為地道轟城,寇死拒不得入。

十二月,無為寇糾合安慶、蕪湖諸黨東下,圖解鎮江之圍。蕪湖下至揚州,沿江汊河套港皆寇通藪。向榮檄水師溯江會攻,敗之神塘河,又敗裕溪口援寇於陶陽浦,生擒偽檢點趙元發、偽將軍王化興等數十名。十二日,秀全遣李秀成等援鎮江,我軍御之石埠橋,尋由江州下竄下蜀街。先是楊秀清調上游蕪湖,江北和、含及廬州眾還江寧,統以李秀成及偽丞相陳玉成、偽春官丞相塗鎮興、偽夏官副丞相陳仕章、偽夏官正丞相周勝坤,取道棲霞、石埠,而豫遣城寇四出絓我軍。向榮大營存兵不敷分布,檄蕪湖鄧紹良分軍為張國樑、秦如虎應援,令吳全美以師船攻大勝關,以分敵勢。明安泰嚴堵秣陵關,咨吉爾杭阿等守丹陽,以固蘇、常要隘。初,澤南既去江西,石達開乘虛復入義寧,敗江西官軍,陷新昌、瑞州、臨江、袁州、安福、分宜、萬載。江西、湖北隔絕,軍勢不能復振。曾國籓飛調副將周鳳山統九江全軍往援,遇寇樟樹鎮,以鉤連槍敗其藤簰手,並會水師毀敵船,新淦寇聞風竄走,遂復其城。

六年正月,石達開陷吉安,乃由湖北入通城。達開悍而多詐,肆擾江西,不急犯省城,不直指南康,先旁收郡縣,遍置偽官,迫其土民,劫以助逆,因糧因兵,愈蔓愈廣。其陷瑞州者為偽檢點賴裕新,先陷袁州者為偽豫王胡以晃,先攻臨江後攻吉安者為偽春官丞相張遂謀。廣東土匪入江西者,以周培春黨為眾。又匪目葛耀明、鄧象等均於瑞州入達開大股之中,匪目王義潮、劉夢熊分屯吉安、泰和,亦與達開合並為一。達開久居臨江,為上下適中之地,兇悍之眾,皆萃於此。南則窺伺贛州、南安以通粵匪,北則踞守武寧、新昌以通九江。達開進攻南昌,周鳳山以九江全軍守樟樹鎮。時達開眾才數千餘,乃張燈火山谷間為疑兵,率敢死士乘夜來襲,我軍不戰而潰。鳳山走南昌,國籓亦移軍省城。秀全益以皖、贛諸事付達開,尋陷進賢、東鄉、安仁,破撫州。未幾,建康、南昌相繼失。澤南念國籓艱危,日夜憂憤,督戰益急。秦日綱嬰城待援,士卒多傷亡,陰穴城為突門。會達開率九江援黨至,開城迎之。澤南要之突門,寇出直沖澤南軍。澤南三退三進,軍幾潰,槍丸中左額,收軍還,創發而歿。以李續賓領其眾。

續賓初建議分屯窯彎絕寇糧,既代澤南,仍屯洪山,以游兵巡窯彎、塘角間。古隆賢率萬人來援武昌,約城寇舉燧為識,林翼諜知,佯舉火,城寇出,陷伏大敗。達開援眾號十萬,林翼分水陸力戰,焚敵船七十,平八十壘。武昌寇大窘,城守益固。而江西請師日數至,義寧寇復犯崇通,九江寇合興國、大冶土寇自武昌縣進至葛店,謀襲巡撫大營。林翼以江西待援,分軍四千一百人,以國籓弟國華統之,率劉騰鴻、劉連捷等道義寧,收咸寧、蒲圻、崇陽、通城、上高。湖南所遣援軍將劉長佑收萍鄉,蕭啟江收萬載。國籓命李元度收東鄉,周鳳山等收進賢,劉子淳收豐城。五月,畢金科將千人防饒州,陷,旋收復。黃虎臣將三千五百人攻建昌,遇寇死。六月,彭玉麟收復南康。七月,劉騰鴻至瑞州,戰寇,走之。

是時江西列縣陷者四十餘城,廣東和平土寇犯定南、安遠、信豐、長寧、上猶、崇義、雩都,省城不能救,軍報數月不相聞。瑞州居江、湘之沖,有南北城,中隔一河。劉騰鴻援南城,韋昌輝自臨江來援,至北城,遽挑戰,騰鴻乘其弊攻之,從北岸渡河抄其後,南城兵角其前,寇大敗。至是江湖路通,自長沙以至南昌無道梗憂。寇自陷吉、袁、瑞、臨諸府,大修戰船,議秋間圍攻省城。瑞、臨寇船出而下,湖口寇船入而上,困我水師,復於生米口築立堅壘。七月,由松湖帶戰船三十餘艘、陸寇千餘,將抵瑞河口,我水師偵知,豫釘排椿。寇甫至,我軍沖入,縱火焚之,復堵城寇於臨江口,焚其船壘。生米口之寇聞之亦遁。八月,劉騰鴻等敗臨川偽指揮黃某,收復靖安、安義。寧都土寇襲陷建昌、鉛山、貴溪,圍廣信。浙將饒廷選赴援,寇遁走。時江西寇勢浩大,黨類眾多,欲以全力困江西。自去年十一月至本年二月,以石達開為主;三、四、五月,以黃玉昆為主;六、七月,以韋昌輝為主。九江則林啟容,瑞州則賴裕新,湖口則黃文金,撫州則三檢點,建昌則張三和,袁州則李能通,皆劇寇也。統計江西境內近十萬人。

九月,國籓視師瑞州,李元度以撫州不克,餉益絀,乃分軍略旁縣募糧,且分寇勢,遂收宜黃,復崇仁。是日城寇出攻江軍,林原恩敗死,元度突圍免。撫州軍俱潰,元度移屯貴溪。十月,復陷宜黃、崇仁,分陷金溪。福建援軍將張從龍援建昌,軍潰,特詔起黃冕知吉安府,率軍往,以國籓弟國荃為軍主。當是時,江西軍分為四,湘軍最強。國籓居水軍中,劉長祐屯袁州,派隊攻克分宜,援寇路絕。十一月,偽將李能通啟西門納官軍,袁州復。國荃收安福。江西諸軍稍振。

初,武昌久不下,林翼謂戰易攻難,以分兵牽寇斷其援路為要。是月,唐訓方等敗石達開於葛店。寇增召戰艦復犯葛店,蔣益澧總六營往,逆戰,克之。追奔至樊口,合水師燔其船,入武昌縣城。石達開憤樊口之敗,大集黨萬餘,由廣濟、蘄水、黃岡至漢鎮,密約偽丞相鍾某堅守以待。官文獲其偽諜,令都興阿、多隆阿馬步兜擊,寇大潰。我軍乘勝攻黃州,不能克。舒興阿、舒保等將馬隊四百人渡江,寇於青山、魯港間增十三壘相持。水陸合擊破之,追奔至葛店。寇懾於騎軍,乃大奔。自是水陸馬步相輔,軍勢日盛,益募陸軍五千、水軍十營,增長圍困之。武昌、漢陽同克復,擊斃偽丞相鍾某、偽指揮劉滿,生擒偽將軍、師帥、旅帥、兩司馬五百餘名。武昌寇分七隊突門出,生擒偽檢點古文新等四人,斃先鋒悍黨八百餘,死兩萬有奇。蓋武漢自五年三月失守,至是已二十餘月矣。尋復武昌縣、黃州、興國、大冶、蘄州,民兵復蘄水、廣濟、黃梅。陳師九江城下。十二月,國籓至九江勞軍,議統水師決取九江,以聯絡內外。乃派千總張金璧等復建昌。李續賓追寇東下,復瑞昌。進攻九江,派軍復德安。劉長祐由袁州赴分宜,寇退踞新喻溪,遇之寶山,降將李能通匹馬沖陣,寇退入城,我軍隨之人,寇出東門遁。湖南援軍將劉拔元等收永寧、永新、蓮花、崇義、上猶。

寇陷鎮江至是四年矣,是年京口為張國樑所迫,秀清命四偽丞相李秀成、陳玉成、陳仕章、塗鎮興往援。秀成欲一人渡江,潛往京口,約兵夾擊,無敢應者。玉成乃夜乘小舟潛越水寨,縱兵擊國樑軍,秀成登高見城中兵出,遣鎮興、仕章當敵,而自率奇兵繞國樑軍後擊之。乘勝擊丹徒,和春敗走,遂渡瓜洲攻揚州,陷之。托明阿軍潰退北路,詔德興阿代領其軍。偽顧王吳如孝守鎮江,分兵踞高資。古爾杭阿檄知府劉存厚扼之,金陵寇大恐。秀清遣悍黨數萬出句容來援,吉爾杭阿中砲死。存厚翼其尸不得出,亦戰死。向榮急遣張國樑會救,克之。秀成以揚州孤懸江北,留守不便,遂棄去,竄回金陵。

當是時,向榮、張國樑負眾望,稱江南勁旅。然頻年征戰,餽餉乖時,士卒常忍饑赴敵,頗缺望,又分兵四出,所部兵力過單。楊秀清知可乘,請於秀全,定夾攻大營之策。五月,密約吳如孝率鎮江寇自東而西,拊大軍之背;金陵寇自西而東與相應,更命溧水、金柱關諸寇旁出橫截。秀清自率勁旅出廣濟門,先遣賴漢英率紫荊山諸黨攻七橋甕以挑之。向榮、張國樑狃常勝,並力截殺,漢英忽少卻,向榮益策大軍赴敵。吳如孝以鎮江黨突薄之,大營空虛,守兵驚散。向榮見大營火起,退無所據,軍立潰。寇數路乘之,大軍死傷遍地。國樑獨以身翼榮出,稍收敗卒退保丹陽。寇築壘圍之,向榮以病不能進,乃以軍事付國梁,一慟而絕。

向榮既死,寇舉酒相慶,頌秀清功。秀全益深居不出,軍事皆決於秀清,文報先白其府,刑賞黜陟皆由之,出諸偽王上。如韋昌輝、石達開雖同起草澤,比於裨將。大營既潰,南京無圍師。秀清自以為功莫與京,陰謀自立,脅秀全過其宅,令其下呼萬歲。秀全不能堪,因召韋昌輝密圖之。昌輝自江西敗歸,秀清責其無功,不許入城;再請,始許之。先詣秀全,秀全詭責之,趣赴偽東王府請命,而陰授之計,昌輝戒備以往。既見秀清,語以人呼萬歲事,昌輝佯喜拜賀,秀清留宴。酒半,昌輝出不意,拔佩刀刺之,洞胸而死。乃令於眾曰:「東王謀反,吾陰受天王命,誅之。」因出偽詔,糜其尸咽群賊,令閉城搜偽東王黨殲焉。東黨恟懼,日與北黨相鬥殺,東黨多死亡逃匿。秀全妻賴氏曰:「除惡不盡,必留後禍。」因說秀全詭罪昌輝酷殺,予杖,慰謝東黨,召之來觀,可聚殲焉。秀全用其策,而突以甲圍殺觀者。東黨殆盡,前後死者近三萬人。

時石達開在湖北洪山,黃玉昆在江西臨江,聞亂趨歸。達開頗誚讓昌輝,昌輝怒,將並圖之。達開縋城走寧國,昌輝悉殺其母妻子女。秀全責以太過,昌輝負誅秀清功大,不服,率其黨圍攻偽天王府,秀全兵拒敗之。昌輝遁,渡江為邏者所獲,縛送金陵磔之,夷其族,傳首寧國。甘言召達開回,既至,或謂達開兵眾功高,請留之京師,解其兵柄,否則又一楊秀清也。秀全心動,乃命如秀清故事輔朝政。達開危懼不自安,其黨張遂謀曰:「王得軍心,何鬱鬱受人制?中原不易圖,曷入川作玄德,成鼎足之業?」達開從之,乃還走安徽,約陳玉成、李秀成偕行,二人不從,益不能還金陵。於是始起事諸悍黨略盡,乃以偽春官正丞相蒙得恩為正掌率,調度軍事;偽成天豫陳玉成為右正掌率,偽合天侯李秀成為副掌率,兵事專屬秀成、玉成,均聽蒙得恩節制;而內政則秀全兄弟偽安王洪仁發、偽福王洪仁達操之。

時我軍自克復廬州,寇黨竄踞三河,分營金牛,一路壁壘相望,屢抗我師。八月,和春督軍乘夜逾壕火其藥局,梯城而入,寇倉皇奪門出,追斃之巢湖。生擒偽指揮張大有、偽將軍秦標盛等十一名,殲賊五千餘。江南軍克復高淳。九月,擊敗句容、溧水,二城近金陵為犄角。金陵聞其敗,氣阻,大營始安。巢縣者,寇之老巢也,其水陸連營無數,所掠糧餉悉輸金陵。巡撫福濟與編修李鴻章督軍攻復之。廬州所屬州邑以次肅清。

七年正月,湖南援軍吳坤修克安義、靖安,與民團會攻奉新,寇棄城遁。武昌之陷也,曾國籓遣彭玉麟援鄂;及石達開躪江西,連陷瑞、臨、袁、吉、建、撫諸郡,又檄玉麟赴援。尋國籓以父喪歸,上命彭玉麟協同載福調度軍事。九江為江西重鎮,皖、楚咽喉,寇力爭天險,匯踞九江,而以對岸黃梅之小池口為外蔽,進以犯湖北,退以擾贛、皖,游行掉臂,防不勝防。大軍自達九江、宿松,諸酋聚眾數十萬,城於小池口,以遏官軍,圖上竄。是月寇分三路入犯,距黃梅縣城數十里,知縣單瀚元請空城誘入,都興阿從其計,伏軍四起殲之。寇棄城走,截斬其偽搗天侯陳某,偽天王壻鍾某、曾某三名。小池口寇聞之喪膽,乃築堅城為固守計;復於段窯、楓樹坳、獨山鎮諸處依山砌石,為壘數十,引水浚壕,阻我軍東下。都興阿遣鮑超、多隆阿、王國才等分攻,悉平其壘。

四月,玉成犯湖北,眾號十萬。李續賓壁小池,鮑超移屯黃梅,遏其沖,分途迎擊,大破之,軍威始振。五月,李續賓攻九江,掘長壕困寇,設伏敗之馬宿嶺、茶嶺諸處。越旬餘,安慶寇來援,合城寇三萬,蜿蜒數里。我軍水陸會剿,連戰皆捷。閏五月,玉成復犯湖北,大小二十五戰,亡七千有餘。時蘄、黃一路寇猖甚,蘄州諸軍並挫,賴舒保力戰,水師左光培扼巴河,得免上竄。官文令唐訓方增軍守要,約都興阿力扼黃梅,嚴防後路。以是黃州上下烽火不絕,而武漢帖然無恙。六月,續賓浚長壕合水師力攻九江,宿松、太湖群寇糾合皖省饑民十餘萬乘虛圖武漢,且解九江之圍。寇據黃梅、廣濟、蘄州、蘄水,分四路進,大小五十餘戰,死萬餘而勢不稍衰。

初小池口之捷也,潯陽、湖口立望廓清;及皖寇上援九江,陸軍梗阻,而上游水師又難驟撤,楚軍馬隊不及萬,寇所竄伏,崎嶇泥淖,馬隊幾無可施。惟將士一心,屹然不為所撼。楊載福、李續賓督水陸上援,多隆阿、鮑超攻賊童司牌,敗之十里鋪。寇造浮橋河中,東通北湖,西達武穴。續賓渡江平南岸寇屯,水師復焚寇艇,毀浮橋,寇不得逞。七月,黃梅寇以弱兵守壘,而以強悍驍勇者遍伏村落。多隆阿偵知,約鮑超直沖村落,斃五千餘,而其在蘄、黃者仍不下數萬。官文督軍五路進攻,杜其上竄,擒渠掃穴,蘄、黃路通。尋又大破皖寇於黃岡、蘄水界,克復瑞州,我軍直抵小池口。小池口與潯城隔江對峙,為江、皖入楚沖途。寇壘石為城,深溝高壘。胡林翼以寇焰正衰,約諸軍先拔小池口,親督唐訓方、李續宜等由蘄水達黃梅坡下,建碉以塞宿松上竄之路。偵知城內爨具已毀於砲,炊煙斷熄,乃令水陸環攻,射火入城。我軍乘亂而登,寇盡殄滅。全楚始一律肅清。

江西軍隨復東鄉。東鄉隸撫州,寇踞之以為撫州保障,復陷萬年諸縣。八月,將軍福興冒雨進攻,縱火平塘,絕寇竄路。平塘者,附城往撫州之沖也。寇果棄城而遁。初,寇踞石鐘山,守湖口兩岸,致水師隔絕。九月,克湖口,連破梅家洲,燔石鐘山寇巢,殪萬餘。內湖外江至今三載始合。載福以取九江當先援彭澤,彭澤南有小孤山,寇築城其上以守彭澤,為九江聲援。載福會軍攻克縣城,盡掃小孤山寇巢。下游巨險悉夷。大軍回向九江。十二月,長祐會攻臨江府,拔之。寇竄湖北興國州,復為續賓所殲。餘眾僅二百,皆鳧水而逸。

寇之自楚北敗竄回皖也,糾合河南捻匪,撲廬州及巢縣、柘皋。我軍進平柘皋寇壘,火巢湖派河兩子鋪寇船,寇蹤遂絕。先是江南水師提督李德麟率紅單船入皖,寇遏之繁昌縣峽口,不得上,七閱月矣。載福督師東下,焚奪陳玉成所派戰船略盡。連日焚華陽鎮,復望江、東流,疾趨安慶,破樅陽大通鎮,進克銅陵,馳入峽內,與紅單船合。寇望風瓦解,逼泥汊偽城,李成謀擲火焚之,斬戮過當。時江西寇糾黨二萬餘,由浮梁、建德、都昌、鄱陽竄湖口,而宿松、太湖寇憤九江之敗,糾黨五六萬,麕集於楓香驛、仙田鋪等處,聲勢相依。官文檄唐訓方壁陳園,固蘄州門戶。多隆阿、鮑超等迎剿太湖,李續宜會水師分三路直搗,斃寇二萬餘,寇勢大挫。

江南大營之退駐丹陽也,秀成踞句容,屢出窺伺。正月,國樑獨率精卒間道抵城下,毀其外壘,斃寇千數百名,寇不敢復出。二月,金陵、安慶寇偵溧水勢蹙,糾眾至鄔山,築壘為援。和春乘寇營未定,邀而敗之。寇渡河復結四壘,江南軍三路敗其眾,合兵攻溧水城,前後平寇壘二十六座,殲三千,斃偽靠天侯以下十餘名。移營已及一年,戰功此為最烈。四月,瓜洲以我軍圍攻久,勢頻危,乃出背城計,水陸並撲,戰土橋西里鋪,不勝;復以戰艦分兩路進:一沿港助勢,一渡江他擾,均為我水師所殲。寇之聚溧水者,屢招援黨攻大營,死萬餘,復於鄔山築壘數十以抗我軍。五月,總兵傅振邦破其外壘,繼以火攻城。副將虎坤元乘內亂,斬守城悍酋而入,遂復其城。溧水既克,和春進規句容,與溧水相犄角。寇結外援,聲勢尚壯。國樑會軍圍攻,而自帥親兵沖入,刺黃衣悍目數名,寇奔潰。和春進沖內壕,國樑首先登城,寇尸山積。閏五月二十五日,收復縣城。九月,鎮江寇出城至甘露寺,逕撲大營,和春迎剿敗之。寇欲西竄接應金陵,國樑密於高資增營扼塞,寇亦築壘,運糧河北。國樑遣參將余兆青等毀其砲臺,而自率親兵渡河,會水陸諸軍鏖戰六晝夜,沉巨船十餘艘,削壁壘二,生擒魏長仁等六名,斬俘無數。

寇之踞瓜洲者,遙聯金陵,近接鎮江,阻官軍進剿之路,歷五年矣。適南岸寇援創於和春,德興阿乘其隙,檄大軍逾城而入,遂下瓜洲。十二月,國樑大捷於瓜洲南岸,陣斬偽王,奪壘十七,遂圍鎮江。秀全四遣眾援,均為虎坤元所破。國樑督軍攻四門壞垣,奪復其城,逸出者沿江搜殺近萬人。惟吳如孝潰圍遁入金陵,復竄聚安慶。而潛山太湖之寇又陷霍山,旋退出,欲從羅田、麻城上竄,踞獨山、西河口為營。官文調馬步軍兼程馳防豫、皖交界之處,以固楚疆。八月,皖寇糾豫捻謀援金陵,犯商、固,擾光州、六安,窺伺隨、棗一路。而太湖、渡石牌等處寇黨連營三十里,眾六七萬,乘我軍度歲,竄近蘄州,尋又竄荊橋、好漢坡諸處。多隆阿迎剿,敗之仙田鋪、風火山,追抵太湖,連營宿、太以扼寇沖。而秦日綱遣其黨北趨,避實擊虛,謀犯蘄州。蘄水、六安之寇亦並力上竄,陷英山縣,分七路竄羅田。羅田知縣崔蘭馨連日鏖戰,收復英山。守備梁洪勝等督楚軍擒偽丞相韋朝綱。寇出黃花嶺,竄楚境楓樹坳等處。都興阿遣將往南陽河迎擊。寇築壘北岸,我軍潛伏北岸山谷中,而列陣南岸。寇渡河而南,我軍邀擊之,乘勝北渡。寇陣山腰,潰寇踵至,伏兵起,斃寇無數。楚軍勢大振,宿、太諸營始紓後顧憂。

時秀全大會諸黨,飭陳玉成為前軍主將,以潛、太、黃、宿為根據,敵我上游楚師;楊輔清為中軍主將,以殷家匯、東流為根據,敵我中路曾軍;李侍賢為左軍主將;李秀成為五軍主將。二月,和春攻破秣陵關,關為金陵南面外蔽,寇所嚴守者也。三月,和春率張國樑等圍攻金陵。會秀全張筵飲群黨酒,流丸墜秀全膝下,群駭愕。秀全曰:「予已受天命,縱敵兵百萬,彈丸雨下,又將如予何!況和春非吾敵也,諸將弄彼如小兒,特供一時笑樂耳,奚恐為?」初,寇屢伺我軍懈,悉銳出犯,冀解其圍,而雨花臺爭之尤力。和春嚴為防儆,寇果由雨花臺攻大營,大敗之。和春、張國樑作長圍困寇,度地勢險夷,溝而垣之,鑿山越水,周城百餘里。諸營大小相維,絕寇應援,秀全大懼,誡各門嚴備。潛結壘於壽德州,屢突長圍,不克,死者枕藉。當是時,石達開在蜀,楊輔清竄閩,林紹璋敗於湘,林啟榮圍於九江,黃文玉坐困於湖口,張朝爵、陳得才孤守皖省,陳玉成坐守小孤山、華陽鎮一帶,秣陵又陷,金陵老巢聲援殆絕。而糧食尚充足,上游諸州縣皆為寇據,呼吸可通,故寇雖危蹙而未遽顛覆。

我軍屢圍金陵,玉成多方抗拒,而秀成出陷杭州,以掣圍師之肘,我軍不動。玉成乃自潛山、太湖下江浦,伺官軍之虛,悉眾攻大營,以冀解圍。蘇、常相繼而陷。四月,李續賓、楊載福會攻九江,九江為金陵犄角,南岸肅清,專力攻九江。城寇被圍久,以數千人攖城,植蔬種麥供軍食,其守愈暇,頻傷攻城軍士。嗣地道成,城破而復完。楊載福督水陸十六營攻四門,地雷再發,城崩百餘丈,諸軍躍登,斃寇萬六七千。出城者水師扼之,俘斬無遺。林啟榮、李興隆均敗死,磔其尸。九江既克,寇黨無固志。楚南軍先後收復新淦、崇仁,下撫州,克安樂、宜黃、安豐、新城諸縣,收復建昌。國荃攻吉安,旁克吉水、萬安二縣。於是江西陷城收復八九矣。寇黨畏懾,金陵寇亦窮蹙。

秀全力圖外擾,乃命寇將竄皖南北及閩、浙諸省,冀大軍分援,以牽我師。玉成勾結捻首張洛行、龔瞎子。眾號十萬人,踞麻城,四門築五十八壘,溝塹重疊,據險自固。而安慶暨英、霍諸寇又陷黃安,冀窺漢陽、德安,取道北竄。官文檄續賓上援,以紓麻城之患。先是秀全命賴漢英掠江西,皖寇入福建,陷政和縣、邵武府,遂陷浦城,分擾建寧。五月,我軍克復黃安、麻城,斬偽丞相指揮數十人;追至商城,並進剿太湖、潛山、英山、霍山諸寇。其黨竄踞東安者,圖為江南北聲勢。和春督軍立復縣城,金陵寇愈形危蹙,急思潰竄。和春派水師分剿繁昌,毀其堅壘土橋,進破峨橋、魯港等處。城寇憤恚,出太平、神策門分犯大營,張玉良、馮子材等陷陣敗之,寇退。遂攻金川門,悉毀東北城外壘柵。

石達開乃自廣豐陷江山縣,金華、衢州、處州三府屬邑焚掠殆遍。浙軍敗之壽昌七里亭。六月,寇竄全椒,踞滁州、九洑洲等處,浙軍大敗之,進克武義、永康、常山、江山、開化、縉雲、宣平,衢圍亦解。寇悉竄處州,陷之,周天受督軍克復。會閩寇蜿蜒猖獗,所復各城旋失,又陷松溪、崇安、建陽等縣,建寧府亦被圍。浙江巡撫晏端書檄將馳援閩省,又出師江山界,剿浦城寇巢。

是時,上以浙、閩寇並起,乃起曾國籓率江西湘軍援浙,旋命改援閩。國籓自鉛山進軍,寇大懼,圖牽制之計,分萬餘人犯江西,圍廣豐、玉山,入踞安仁。閩軍遂克光澤,收建陽,解順昌圍,連復松溪、政和、寧化、崇安,破浦城老巢。復邵武府,閩省肅清。國籓移軍弋陽,親督水陸各軍克復安仁縣城。八月,克吉安,擒偽先鋒李雅鳳、偽丞相翟明海,正法。江西列城皆復。進攻太湖,前月寇陷廬州,巡撫翁同書告急於續賓,官文以太湖方血戰有功,疏留之。時寇於東岸及楓香鋪、小池驛、東山頭各築營壘,續賓等分段攻城,焚其火藥庫,寇眾駭散,遂克太湖,乘勝抵潛山。潛山石牌為南北要沖,寇屢集黨與援應,抗我東征之師。都興阿等營北門彰法山,馬步並進,寇敗潰,斃七八千,遂復縣城。我軍分二路平上下石牌老巢。

九月,玉成自潛、太會九洑洲群寇下江浦,伺官軍之虛,疾攻浦口,以冀解金陵之圍。我軍進退爭一橋,遂大挫。和春派兵來援,寇分軍綴之,仍力撲浦口。江北大營遂失陷。迭陷江浦、天長、儀徵。並分攻六合,德興阿遁。揚州賊破南門入,揚州陷。進犯邵伯縣,國樑率軍渡江。會北軍克復府城,移攻儀徵,亦克之。亟引兵救六合,阻於寇,不得驟進。寇穿地道陷城,補用道溫紹原赴水死。寇渡江陷溧水,築壘江藍埠諸處,為扼要持久計。十月,和春遣總兵張玉良攻復溧水。寇夾攻高古山大營,國樑怒馬陷陣,斃寇五六千。合兵追抵江寧鎮,毀卡壁數十座。小丹陽以至採石磯老巢悉平。

初,勝保率皖軍攻天長,捻首李昭壽以部眾二千降,勝保奏請賞給花翎三品銜,賜名世忠,使為內應,遂克縣城。大軍之入皖也,克復桐城、舒城二縣,寇悉遁三河。都興阿會水師盡掃安慶城外寇壘。續賓追至三河,玉成、秀成、侍賢連江浦、六合、廬江眾,又乞援捻匪,招潁、壽、光州群盜,合十餘萬,圍官軍三重,眾寡不能敵,續賓死之。潰軍至桐城,前留防四城軍潰,不旬日,桐、舒、潛、太復陷。都興阿收潰卒,由石牌駐軍宿松,進剿黃泥營寇眾,敗之;復督鮑超、多隆阿大戰荊橋、陳家大屋,平三十餘壘,軍勢復振。玉成退還太湖,以為舒、桐已得而宿松不破,則安慶之守不固,與秀成謀再舉。秀成知不可敵,不欲從,而玉成屢言有妙策,始與分道來犯,卒受大創而退。玉成留軍太湖,而自還安慶。秀成率黨還巢縣、黃山。

是時江西寇復闌入閩界,蹂將樂縣,並陷浦城、永吉、建陽、順昌、寧化、長汀等城。國籓入閩,軍建昌。諸陷城以次復。寇復竄回江西,惟連城尚聚萬餘,復陷景德、東流,謀竄湖口、九江等處。國籓檄調道員張運蘭倍道馳赴景德鎮,屢戰皆捷。初,寇踞景德鎮,勢焰薰熾,江右要沖之區,恣行無阻。國籓添派其弟國荃率湘軍五千八百赴鎮,助運蘭攻剿。寇夜襲艇師劉於淳,燃火彈拋燒簰卡無數。寇棄鎮竄浮梁,國荃等水陸進攻,復浮梁。寇走建德北去,江西稍定。

十一月,江南大營援軍直隸通永鎮總兵戴文英戰死寧國灣沚。次日,幫辦皖南軍務浙江提督鄧紹良,大營陷,死之。寧郡設防三百餘里,皆鄰寇巢,近則蕪湖、青陽、繁昌、銅陵,遠則無為、和州、滁州,渡江即至。而祿口、秣陵、溧水敗寇,勾合太平金柱關、東西梁山黨眾,潛山、太湖、舒、桐及樅陽土橋敗黨,皆以寧國為通藪;防軍僅七千有奇,又多調援他處,寇眾兵單,故及於敗。國籓疏陳目前緩急,宜先攻景德鎮,保全湖口,上是其議。胡林翼先以丁母憂回籍,會三河變起,朝旨迫起督師,十二月,渡江駐黃州。時寇之踞南安者有五支:一為偽翼府宰制陳亨容、傅忠信、何名標,一為偽渠帥蕭壽璜、蔡次賢,一為偽尚書周竹坡,一為偽軍略賴裕發,一為偽承宣劉逸才、張遂謀,眾七八萬,將由南康犯贛州,築偽城於新墟,設卡壘,踞村莊,綿亙二十餘里。

九年正月,國籓檄蕭啟江設伏赤石塘,敗寇,克新墟,進破南康池江、小溪、鳳凰城、長江墟寇壘,並克崇義、南安,進解信豐之圍。二月,江浦薛三元獻城降,進克浦口,陣斬偽天福洪方、偽立天豫莫興。寇覘李世忠擊高旺,乘虛再陷浦口。世忠回軍再克之,浦口肅清。李秀成急率悍賊七八萬來犯,踞烏衣鎮汊河。秀成復要陳玉成自廬州來援。烏衣鎮屬滁州、江浦交界鎖鑰,寇意在斷絕浦營餉道,為張國樑擊敗。寇與閩、浙餘寇皆趨郴、桂,所謂石達開三十萬眾後圍寶慶者也。玉成由六合犯廬州,布政使李孟群被執,不屈,死之。三月,糾安慶黨圍撲定遠護城營,築堅壘數十以困我師。勝保襲破其壘,秀成東走,而黨眾日增。國樑於定遠縣西築十里長墻御之,其北路自九里山至浦口,三四十里,寇壘殆遍。我軍日戰,副將鄭朝棟、張占魁皆歿於陣。時浦口後路滁州、來安皆困於寇。世忠自浦口繞道回援勝保,撤烏衣汊口防軍還定遠,其地復為寇踞。和春慮江北軍單,遣馮子材渡江援應。玉成度江浦、浦口未可力爭,分黨援六合;又謀趨天長、揚州,渡江攻南營後路,並襲北營。於是寇眾四五萬東趨六合,蔓延來安、盱眙諸境。

四月,玉成圍揚州。提督德安擊寇天長,失利,歿於陣。勝保率軍進戰石梁,互有死傷,還屯舊鋪,扼盱眙前路捍北犯。駐汊澗軍為寇困,先後突圍出。和春遣張玉良、安勇分六合軍赴防揚州,以固清、淮門戶。時池州、青陽寇逼石硊,窺灣沚。當塗、蕪湖寇分壁青山、亭頭逼黃池。我軍敗盱眙、汊澗及天長寇,天長寇分竄六合,並踞儀徵江幹東溝,圖撲紅山窯。其地距六合二十里,旁通瓜埠,為大營餉道咽喉。五月,鞠殿華督軍破平六合東路王子廟、太平集寇壘。初,六合、儀徵連界二十里,寇壘四十餘,阻糧道。至是六合廓清。時六合以北、天長以南,寇麕集數萬,餉道危急,由烏家集繞犯各軍之背,世忠退保滁、來。寇趨舊鋪,直犯盱眙,圍勝保於桑樹,都興阿力戰解之。尋舊鋪寇犯紅子橋,勝保及穆騰阿馳援,而寇已分犯盱眙,盱眙故無城,倉猝遂失。

六月,勝保攻克盱眙,追創之磨臍、天臺諸山。揚州諸軍安勇等聞天長寇回竄六合,赴儀徵截擊,大破於沙河、大小銅山。玉成憤甚,圖報復,率死黨攻來安。世忠守城,伏壯勇於兩門外,自督軍沖入寇營。寇乘虛襲城,伏起攔擊,世忠返隊夾攻,寇大敗,夜走滁州。世忠由水口焚燒寇壘,寇大潰,糾合捻匪圍定遠,再敗再進,我軍眾寡不敵,遂失陷。七月,玉成率死黨攻來安,犯滁州,世忠擊之,稍卻;尋復糾眾圍來安,並分屯城西北卓家巢等處,寇壘幾遍。世忠偵寇志已驕,潛伏兵挑戰,偽敗,寇笑官軍怯;而世忠又環譟之,寇不為意,惟槍聲絕續作備而已。世忠驟起鳴角而前,火其營,破二十八壘。會勝保解其圍,世忠還滁州。八月,敗寇西竄陷霍山,江長貴等擊敗太平郭村、宏潭踞寇,尋竄石埭,陷烏石隴,防營游擊黃金祥退屯楊谿河。自去歲三河失陷,寇造偽城高二丈餘,砲眼星列,環以深壕,椿簽密布,與太湖互相援應,兼通糧道。

石牌鎮隸安徽懷寧,當宿、望、潛、太之交,為由皖入楚要沖。官文以偽城不拔,終礙東征,乃令多隆阿統馬步軍會攻,拔偽城,擊斬霍天燕、石廷玉等四十七名,並拒敗潛山、安慶援寇。偽顧王吳如孝者,寇之最悍者也,自鎮江逸出,至皖北,糾捻沿淮肆擾;尋撲盱眙之清壩,為格蘭額等槍斃,斷其首。眾南潰,九月,擾霍山下符橋。六安防軍盧又熊等擊敗之,破毛坦廠寇壘,而廬州、安慶寇同犯六安,乃引軍還盱眙。天長寇犯揚州,參將艾得勝、雙喜等敗死司徒廟。玉成率大股自甘泉山西竄儀徵陳板橋,進援六合,圍李若珠壘。馮子材御之失利,退屯段要口。寇踞紅山窯,斷李若珠營後路,餉運不通。

十月,若珠自八埠墻、陳家集潰圍出,中數創,退屯揚州,死傷馬步軍二千八百餘人。石埭夏村寇分股糾青陽寇萬餘,竄踞涇縣查村,防軍副將石玉龍敗死南山嶺。適周天受至自寧國,督天孚等力擊之,寇退還查村。王浚破平陶美鎮寇壘,陣斬偽丞相孫瑞亨,鎮距秣陵關二十餘里。盧又熊克霍山,寇自太平、蕪湖犯寧國,陷黃池,高州鎮總兵蕭知音敗退新豐鎮。玉成及秀成自天長、六合糾大股窺伺江浦,分屯南北兩岸。張國樑渡江遣水師破壽德州寇壘,水師曹秉忠破六合、紅山窯、瓜埠寇七壘,彭常宣敗寇於儀徵泗源溝。時寇眾悉踞揚州西北,尋陷江浦防軍壘,周天培死之,大軍退保江浦。寇乘勢東伺揚、儀,西逼江浦,南窺溧水,勢復熾。

寇自洪、楊內亂,鎮江克復,秀全兇焰久衰,徒以陳玉成往來江北,句結捻匪,擾廬州、浦口、三河等處,迭挫我師。曾國籓以為廓清諸路,必先攻破江寧;欲破江寧,必先駐重兵於滁、和,而後可去江寧之外屏,斷蕪湖之糧道。欲駐滁、和,必先圍安慶,以破陳玉成之老巢,兼搗廬州,以攻陳所必救。誠能攻圍兩處,略取旁縣,備多力分,不特不敢悉力北竄齊、梁,並不敢一意東顧江浦、六合,蓋寇未有不悉力以護其根本者也。於是定四路進兵之策:國籓任第一路,由宿松、石牌以窺安慶;多隆阿、鮑超任第二路,由太湖、潛山以取桐城;胡林翼任第三路,由英山、霍山取舒城;調回李續宜任第四路,由商、固以規廬州。以後平寇之策,皆不出此。

十一月,涇縣查村寇犯吳正熙壘,不利,而章家渡亦為我軍所挫。揚州寇踞甘泉山,馬德昭破其壘。國樑督軍攻江浦寇壘不下,寇掘地道攻城,玉良遣將縋城出,焚其壘,填塞地道。寇築壘磨盤洲,我軍四路蹙之,寇眾大敗狂奔,北門寇營亦同時攻破。其陳家集等處之寇竄回天長,南路之寇潛窺溧水,皆為防軍擊退。江長貴克太平,郭村、查村敗寇竄涇縣北路。副將榮升連破石柱坑、盤臺寇卡,寇竄踞董家村、白茅塘,犯萬級、黃柏兩嶺。榮升會徽軍破之,覆其巢。寇又竄擾河西,為參將硃景山等所敗。副將吳再升遂乘勝進剿黃池南岸牛頭山寇壘,北岸寇糾眾來援,分兵拒之,寇多死傷;北岸寇潰走渡河,我軍遂收南岸。池州守城寇韋志俊獻城於楊載福,其部下古隆賢等不從,回撲府城,城復陷。桐、潛寇援太湖,將襲天堂後路,餘繼昌會軍團分路敗之槎水畈,陣斬偽漢天侯、拱天豫二名,寇奔潰。

十二月,侍賢由蕪湖金柱關率大股犯寧國,與黃池北岸寇合勢,連日分擾黃岡橋、牛頭山等處,再犯西河,蕭知音、熊廷芳退走寒亭。寇圍游擊冉正祥壘,都司李培基馳援始解。玉成以定遠、舒城、廬州寇眾北犯壽州,翁同書令副將尹善廷率精銳馳援,挫寇於東、南兩路。時玉成以楚師甚盛,欲圖西竄六合拒楚師,因北犯壽州以牽掣我軍。尋自江浦回援安慶、太平,糾合捻首龔得樹、張洛行等分道上犯,眾號十餘萬。多隆阿、鮑超、蔣凝學御之潛山,連破靈港寇壘。蕪湖寇進犯宣城、灣沚,周天受御之,不得逞;乃分眾四竄,我軍亦分拒於海南渡、浮橋口、清水潭、鹽官渡。寇退踞許村埠,進犯西河,硃景山等創之,增軍守東西岸。寇迭窺灣沚,我軍渡河擊之,寧國西北寇鋒稍斂。先是銅陵、青陽寇常犯南陵、涇縣之交,我軍扼守雲嶺、蘇嶺,而設伏朝山要、三里甸,參將方國淮出奇擊之。寇屢犯三里甸,陷國淮壘,復竄越雲嶺,陷觀嶺防營。天受調金友堵清弋江,寇北走南陵,陳大富擊之,寇復退入涇境。

自玉成回援安慶後,秀成獨屯浦口,寇勢已孤。時金陵困急,援兵皆不至。秀成以玉成兵最強,請加封王號寄閫外,秀全乃封玉成英王,賜八方黃金印,便宜行事。然玉成雖專閫寄,而威信遠不如秀成,無遵調者。李世忠因致書秀成曰:「君智謀勇功,何事不如玉成?今玉成已王,而君尚為將,秀全之憒憒可知矣。吾始反正,清帝優禮有加。以君雄才,胡為鬱鬱久居人下?盍從我游乎!」時偽兵部尚書莫仕葵以勘軍在秀成營,書落其手,閱之大驚,以示秀成。秀成曰:「臣不事二君,猶女不更二夫。昭壽自為不義,乃欲陷人耶?」仕葵曰:「吾知公久矣。」乃代奏之,秀全命封江阻秀成兵,並遣其母、妻出居北岸,止其南渡。仕葵曰:「如此,則大事去矣!」乃偕蒙得恩、林紹璋、李春發入偽宮切諫曰:「昭壽為敵行間,王奈何墮其計,自壞長城?京師一線之路,賴秀成障之。玉成總軍數月,不能調一軍,其效可睹矣。今宜優詔褒勉,以安其心。臣等原以百口保之。」秀全悟,召秀成入,慰之曰:「如卿忠義,而誤信謠傳,朕之過也。卿宜釋懷,效力王室!」即進封偽爵為忠王榮千歲。寇自楊、韋構殺,秀全以其兄弟仁發等主持偽政,偽幼西王蕭有和,蕭朝貴子也,秀全尤倚任之,而以一偽將畜秀成,不與聞大計。至是晉偽爵為王,乃大悅,以為秀全任己漸專,不料其疑己也。

浦口當金陵咽喉要地,迫於大軍,而糧援無措;南渡時,見秀全問計,秀全語以「事皆天父排定,奚煩計慮?」又與仁發等謀留其助守金陵,秀成不可,曰:「官軍既以長圍困我,當謀救困法,俱死於此無益也。」渡還,以黃子隆、陳贊明屯浦口,親赴上游糾合皖南蕪湖、寧國死黨,謀間道犯浙江,分江南大營兵力,還解長圍之困,其志固不在浙也。連日援太湖寇、捻攻鮑超潛山小池驛營壘不克,楊輔清、古隆賢用內應陷池州。韋志俊突圍屯泥灣,收合散亡,移屯香口;迭敗寇於八都阪、慄樹街,俘斬偽將軍陳松克等三十餘人。

是年,秀全大封諸王。初,秀全定都金陵,一切文武之制,悉由偽東王楊秀清手定。是時為秀全建國極盛時代,其宮室制度:第一,為龍鳳殿,即朝堂也,主議政、議戰諸大事。每有大事,鳴鐘擊鼓,會議,秀全即升座,張紅幔。諸王丞相兩旁分坐,依官職順列。賊將則侍立於後。議畢,鳴鐘伐鼓退朝。第二,說教臺,每日午,秀全御此,衣黃龍袍,冠紫金冕,垂三十六旒。後有二侍者持長旗,上書「天父、天兄、天王、太平天國」。臺式圓,高五丈,階百步。說教時,官民皆入聽。其有意見者,亦可登座陳說。文左上,武右上。士民由前後路直上,立有一定之位。第三,軍政議事局,軍事調遣、糧餉、器械總登所。秀全自為元帥,當日偽東王為副元帥,北王、翼王為左、右前軍副元帥,六官左、右副丞相為局中管理。各科員中,分軍馬、軍糧、軍械、軍衣、軍帳、軍船、軍圖、軍俘、軍事諸科。又有糧餉轉運局、文書管理局、前鋒告急局、接濟局,皆屬軍政議事局。內以六官左、右副丞相領之。其最尊者為軍機會商局長,初以偽東王領之。遇有戰事,籌畫一切,則偽東王中坐,諸王、丞相、天將左右坐立,各手地圖論形勢,然後出師。秀清死,偽翼王領之。石達開去後,李秀成領之。秀成東入蘇、杭,則有名無實,虛懸其位矣。其時寇之武備頗詳盡。自諸偽王內訌,人心解體,秀全以為非不次拔擢,無以安諸將之心。然自此大封之後,幾至無人不王,而丞相、天將之職多攝行。於是各持一軍,勢不相下,而調遣諸王者,僅陳玉成一人。故八年以前,寇之用兵,攻守並用。八年以後,不過用攻以救守,戰局遂至日危,以底於亡。

十年正月,偽匡王、偽奉王、偽襄王糾合偽攝王自南陵犯涇縣灣灘,游擊王熊飛退走,寇遂蔓延黃村、焦石埠,進攻副將李嘉萬,援師為楊名聲所敗,斬偽岡天燕、賴文禾。寇竄踞黃柏嶺,其黨尋大至,陷涇縣。楊名聲等退走旌德,寇踵至,明日亦陷。我軍還守寧國。是時秀成自率悍黨數千,已由寧國縣間道犯廣德。張國樑督水陸諸軍渡江期大舉,克浦口八壘,黃子隆、陳贊明遁;攻九洑洲,克其老巢,焚之。寇自咸豐四年築壘九洑洲,內蔽江寧,外通大江,踞為南北水陸要區。江寧長圍成後,浦口、九洑洲皆克,勢大困。

秀成由皖犯浙,分我兵勢,而諸將又以寇在陷阱,無能為役,習為驕佚,戰志漸消,故有閏三月大營失敗之禍。太湖寇、捻分四股來犯我軍,知府金國琛會集諸軍敗之仰天庵、高橫嶺,生擒悍目藍承宣,向擾害蘄、黃者,寸磔之。金國琛等復敗寇、捻於潛山廣福寨。玉成率龔得樹、張洛行來援,乘霧移營於羅山沖、白沙畈,冀與城寇相通,以圖牽綴我軍。諸軍會擊,寇大敗,擒斬偽庶天侯麥烏宿、偽軍師汪遂林等。明日,鮑超等進攻小池驛,當東路;蔣凝學等攻羅山沖,當西路;多隆阿居中路策應。羅山沖寇蜂擁來撲,凝學連破沖口,攻入內山,馬隊繼之,寇大敗。值東南風作,以火焚之,毀壘百有數十。寇奪路狂奔,斃偽丞相葉榮發、偽將軍舒春華等。城寇謀宵遁,伏軍四起擊之。是役也,殲寇二萬餘,益惶懼,竄入潛山。多隆阿督軍尾擊,克其城。

秀成、侍賢等至廣德,詐為清軍,陷之,杭、湖、蘇、常並震。巡撫羅遵殿調徽、寧防軍援剿廣德,以保兩浙門戶。張芾遣周天孚馳防長興四安鎮,鎮距廣德四十里,當蘇、浙之交。和春遣水陸軍來會,秀成留陳坤書、陳炳文守廣德,自率譚紹光、陳順德、吳定彩等馳攻四安鎮,陷之。和春遣水師會攻江寧上下兩關,七里洲寇謝茂廷、壽德州寇秦禮國遣使詣大營乞降。江寧西北各門皆瀕大江,洲堵錯互,寇踞上、中、下三關,築壘於壽德、七里各洲,與北岸九洑洲遙相倚藉。九洑洲既克,茂廷、禮國約舉火為號,於是上下關同日而克。國樑增八壘於江東門,增四壘於安德門,毛公渡南北岸關隘悉為我奪,寇益大困。

秀全檄諸寇解金陵圍。時秀成在皖,與其部下謀曰:「清軍精銳悉萃金陵城下,其餉源在蘇、杭。今金陵城外長壕已成,清軍內圍外御。張國樑又嚄唶善戰,攻之難得志,不如輕兵從間道急搗杭州。杭州危,蘇州亦必震動。清軍慮我絕其餉源,必分師奔命以救。我瞷大營虛,還軍以破圍師,則蘇、杭皆我有也。」乃自率數千精卒以行,連陷安吉、孝豐、長興諸縣。以其弟侍賢犯湖州,自率悍黨陷武康,間道逾嶺犯杭州。預結捻首張洛行、龔瞎子等,使內擾清、淮,以分江、皖兵力。

上命和春兼辦浙江軍務,而以張玉良總統援浙諸軍。玉良分大營兵勇五分之二御之。秀成攻杭州,以地雷崩清波門,陷之,巡撫羅遵殿等均死難。秀成之破杭州也,祗一千二百五十先鋒。諸處援兵不知虛實,聞城破,皆潰走。迨張玉良援軍至,屯武林門,秀成曰:「中吾計矣!」自以兵少,乃多制旗幟作疑兵,潛退出城,委之而去。玉良與將軍瑞昌會擊,立復省城。

三月,秀成回竄餘杭,陷臨安。旋為李定泰克復。孝豐、武康寇亦退走。時秀成及侍賢回廣德,楊輔清亦自池州來會。李定泰等會圖廣德,寇已分走建平,陷之,連陷東壩、高淳,復詐為官軍陷溧陽。自是江南大營後路驟急,蘇、常俱大震。和春馳檄張玉良等還救常州,熊天喜等克廣德,而楊輔清陷溧水,詐為官軍襲金壇,為周天孚等所敗,棄壘西竄。句容亦陷,句容當大營後路,餉道所必經,且與丹陽、鎮江接壤,為常州門戶。和春遣副將梁克勛赴援,不及,續遣副將張威邦由淳化進剿。何桂清遣將分防丹陽、鎮江、瓜洲,冀通大營至蘇、常水陸道路。馬德昭等出屯郡城三十餘里下弋橋,堵溧陽、宜興各路寇內犯。米興朝自廣德進軍克建平。

閏三月,寇自昌化出於潛,分犯分水,陷而旋復,進陷淳安。秀成約會諸酋同議救金陵之策,秀成與侍賢由淳化、輔清由溧水退秣陵關,玉成亦自江浦渡江來會,江寧寇爭出築壘接應。斯時大營四面受敵,而良將勁兵調援浙西者一萬三千人,淳壩、宜興防軍又調去一千有奇,大營空虛,糧路又截斷,乃改月餉積四十五日始一發。兵勇皆怨,心漸攜貳。時群寇麕集,和春急調張玉良回援,何桂清留之不遣。寇至雄黃鎮,我軍御之不克。輔清由秣陵關至南門,玉成由江寧鎮至頭關,板橋、善橋諸寇皆集南岸。秀成由姚巧門進紫荊山尾,陳坤書、劉官芳由高橋門而來,侍賢由北門紅山而至,輔清由雨花臺,玉成由板橋、善橋,連日攻撲長圍。國樑與王浚分督諸將力禦,十五日夜,雷雨雹雪,大寒,總兵黃靖、副將馬登富、守備吳天爵戰死。大營火起,全軍潰陷。和春、許乃釗退走鎮江,再退丹陽,旋馳書趣國樑亦至,留馮子材守鎮江。國樑語和春曰:「六年向帥大營失陷,退扼丹陽。彼時京口未復,今東門之限在於鎮江。舍此不守,是導寇而東也。」和春卒不能用,而宜興同時亦失陷。

寇勢大張,而秀全於戰士不及獎敘,終日亦不問政事,只教人認實天情,自有升平之局。仁達、仁發忌秀成功,嗾秀全下嚴詔,飭秀成率所部限一月取蘇、常。寇掠金壇四鄉,大書於壁曰:「攻野不攻城,野荒城自破。」我軍屯六門,日與賊戰,互有勝負。秀成自句容攻丹陽,國樑開南門酣戰,秀成命力士溷入我軍潰卒中,猝擊國樑,被創大呼,入尹公橋下而死。秀成入丹陽,命收國梁尸,曰:「兩國交兵,各忠其事。生雖為敵,死尚可為仇乎?」以禮葬之下寶塔。和春奔常州,寇躡其後。何桂清聞變跳走。是月,楚軍援皖南,會克太平、建德、石埭三縣。涇縣張芾會同周天受等進毀白華、宴公堂一路寇壘,直抵城下,斬關直入,遂復縣城。

四月,天長、六合寇乘金陵大營退守,分三路進犯:一由陳家集圖揚城,一由東溝窺瓜洲,一由僧道橋編筏偷渡襲邵伯,皆為我軍所截擊,不敢逞,乃築壘僧道橋圖久踞。我軍分左、右、中三路疾趨會攻,毀二壘,焚木城,積尸枕藉。寇合股退踞陳家集。揚州與鎮江相為脣齒,李若珠咨艇師陳泰國等分扼各口。寇大逼常州,張玉良由杭郡率軍先至,築營寨大小四十餘,悉為所破。常州陷,玉良敗走無錫。秀成率所部精卒潛出九龍山,拊高橋之背。玉良軍大敗,無錫陷,敗走蘇州。和春創胸,至蘇州濟墅關而卒。玉良連敗之師不能復戰,寇薄蘇州,玉良退走杭州。長洲、元和兩縣廣勇李文炳、何信義開門迎秀成入踞之。巡無徐有壬等同殉難。

秀成踞蘇後,改北街吳氏復園為偽府。秀成踞蘇十有一日,出偽示安民。城廂內外凡收尸八萬三千餘具,而從者猶盛稱秀成愛人不嗜殺也。寇踞蘇城,復恣意擄掠,民競團練為自保計。江、皖援浙諸軍以次克復諸城,遂會剿淳安,寇敗遁入徽州境。蘇寇陷吳江,犯平望,浙江防軍潰,江長貴負傷還走仁和塘棲鎮,副將張守元亦潰於清杉徬。嘉興危急,杭省大震。侍賢燒嘉興南門入踞之。玉良攻嘉興西、南兩門,陳坤書、陳炳文求救於蘇。適青浦周文嘉與洋軍戰,來告急,秀成乃先援青浦,擊退洋軍,直攻上海,不克,遂應嘉興之援,由松江、浦邑而回戰,取嘉興、平湖,順至嘉興,連戰五日,分一股上石門,斷玉良來路,兵多降者,玉良回杭州。

五月,貴池、青陽寇犯涇縣,總兵李嘉萬等敗死,楊名聲退至太平黃花嶺。寇陷廣德,米興朝軍潰,奔孝豐,再退歙北篛嶺外。初,涇縣、廣德同時告警,周天受遣援皆不及,而參將丁文尚守涇,又退走,寇遂由三谿竄旌德孫村。廣德寇窺伺寧國,天受擊卻之。寇由寧國縣東岸至旌德,與涇縣合勢,嘉定陷,薛煥尋克之,收太倉。寇攻鎮江,陷青浦,陷松江。

寇之守江寧也,以安慶、廬州為犄角,以太平、蕪湖為衛護。蕪湖之南,有固城南漪、丹陽白臼諸湖,上可通寧國之水陽江、清弋江,下則止於東壩。掘東壩而放之,則可經太湖歷蘇州以達於婁江。蕪湖孤懸水中,寇守之則易,官軍攻之則難。是以踞五年血戰不退,而黃池、灣沚屢次失利,皆以我無水師,寇堅忍善守。官軍圍攻屢年,往往因水路無兵,不能斷其接濟。今蘇州既失,面面皆水,寇若阻河為守,陸軍幾無進攻之路,城外幾無立營之所。則欲攻蘇州,須立太湖水師,使太湖盡為我有,而後西可通寧國之氣,東可拊蘇州之背。因建淮陽、寧國、太湖速立水師之策。

寇陷江陰,玉良連以砲艇破嘉興三塔、普濟二寺,平新塍寇壘,移營逼西門、南門,破壘七。平望鎮者,浙江之嘉興、湖州,江蘇之吳江總匯處也。寇踞沿河六里橋、梅堰諸處,遍築堅壘,密釘排椿,扼險以阻江、浙之路。湖州趙景賢毀沿河寇壘,分軍進克平望,會軍於米市湖,盡毀砲臺巢穴,進圍嘉興。寇既陷松江,遣其黨窺上海。薛煥乘其不備,直搗南門而入,殺黃衣目十三名,奪船七十餘艘,立復府城。自松江至上海,沿途團練截殺殆盡。

六月,楊輔清糾旌德、太平大股犯寧國。寇自長興竄陷安吉,王有齡遣彭斯舉赴援,遇於孝豐,失利,退走昌化。寇直犯於潛,陷之,杭省大恐。寇復由黃渡再陷嘉定,糾土匪進踞南翔鎮,逼上海四十餘里,再陷平望。蘇州、嘉興寇勢復合。於潛寇連陷臨安、餘杭,分擾富陽,吳雲會洋將華爾攻青浦急,偽寧王周文嘉乞援於蘇州,秀成率大股親援,我軍敗績。寇收槍砲乘船再犯松江,陷之。江陰寇分黨築壘申港,掠船謀北渡,李若珠飭艇師破毀之,偽丞相方得勝遁。玉良以地雷崩嘉興南門城垣,寇嚴拒不得進。劉季三等連克餘杭、臨安。浙西寇回竄孝豐,突犯建德。

七月,秀成毀松江城堞,率偽會王蔡元隆、偽納王郜永寬北犯上海,號十萬,焚掠泗涇,七寶民團御之,多死傷。寇屯徐家匯,薛煥督文武登陴固守。寇詐為官軍賺城,城上詗知,創卻之。洋輪之泊黃浦江者,升開花砲於桅發之,寇始敗退。孝豐寇陷廣德,游擊黃占起、江國霖戰死。江長貴突圍退至安吉,米興朝奔四安。未幾,趙景賢復廣德,寇再陷踞之。寇復陷江陰楊厙汛城,逼常熟二十餘里。黃浦輪船洋兵以開花砲測擊上海寇壘,六發,創及秀成。是夜秀成解圍還青浦。時善興寇告急,遂趨浙江。

初,副將陳汝霖率民團救松江,迨上海解圍,洋將華爾會守松江,賜號常勝軍。秀成陷嘉善,陷平湖,錫齡阿兵勇皆潰,寇旋去,收之。寇陷金壇,知縣李淮守百四十餘日,糧盡援絕,川兵通寇,殺參將周天孚,陷之,李淮等皆戰死。丹陽寇糾黨六七千由新豐等處分道撲水師,謀掠舟北渡,並沿河築壘,架砲轟射,周希濂督艇師乘煙霧對擊,寇不支,遁回丹陽。玉良攻嘉興兩月不下,先後集兵三萬有奇,而蘇、常以北無牽掣之師,松江、青浦之寇可直入嘉興,常州、宜興之寇可直入長興,建平、廣德之寇可直入安吉,寧國、涇縣之寇可直入於潛。

寇前自長興逕逼省垣,雖經擊退,並立復數城,而廣德遂至不守。迨收復廣德,而嘉善、平湖又復失陷。寇處處牽掣我軍,近復添築營壘砲臺,又偷劫五龍橋頭卡,多方誤我軍,實有罷乏不堪之勢。秀成以嘉興圍急,率大股來援。玉良督戰五日,勝負未決。秀成分股上趨石門,謀斷大營後路。地形多支河,塘路絕,無可歸。我軍懼奔,玉良負創,疾馳還杭省。賊既解嘉興之困,復陷石門,分兩路直逼杭省:一趨塘棲,民團御之,退掠新市;一趨臨平,吳再升敗之,轉走海寧。彭斯舉等擊斬頗眾,寇悉退還石門。未幾,石門寇亦退,再升進駐石門。馬德昭由臨平、長安相繼前進。

八月,寇陷昭、常,再攻平湖、嘉善,陷之。是時秀成自嘉興還蘇州,奉秀全偽詔,趨還江寧,令經營北路。初,咸豐三年,林鳳祥、李開芳北犯不返,秀成未敢輕舉。適江西、湖北匪目四十餘人具降書投秀成,邀其上竄,自稱有眾數十萬備調遣。秀成覆書允之,留陳坤書駐守蘇州,自返江寧,請先赴上游招集各股,再籌進止。秀全大怒,責其違令。秀成反復爭辯,堅執不從,秀全卒不能強。於是取道皖南,上竄江、鄂。

秀成之在偽京與諸黨會議也,曰:「曾國籓善用兵,將士聽命,非向、張可比。將來七困天京,必屬此人。若皖省無他故,尚不足慮。一旦有失,則保固京城,必須多購糧食,為持久之計。」秀全聞之,責秀成曰:「爾怕死!我天生真主,不待用兵而天下一統,何過慮也?」秀成嘆息而出,因與蒙得恩、林紹璋等再三計議,僉以秀成之策為然。因議定自偽王侯以下,凡有一命於朝者,各量其力出家財,廣購米穀儲公倉,設官督理之。俟缺乏時,平價出糶,如均輸故事,以為思患預防之計。洪仁發等相謂曰:「此亦一權利也。」因說秀全用鹽引、牙帖之法,分上、中、下三等:上帖取米若干石,中、下以次遞減。此帖即充偽樞府諸偽王祿秩。收入後無須撥解,而稍提其稅入公,大半皆入私橐。商販非執有帖者,粒米不得入城,犯者以私販論罪。如是,則法可行而利可獲矣。洪氏諸偽王乃分售帖利,上帖售價有貴至數千金者。及商販至下關,驗帖官皆仁發輩鷹犬,百端挑剔,任意勒索。商販呼籥無門,漸皆裹足;而諸偽王侯又因成本加重,售價過昂,不原多出貲金,米糧反絕。秀成言之秀全,請廢洪氏帖。秀全以詰仁發,仁發以:「奸商每借販米為名,私代清營傳遞消息。設非洪氏,誰能別其真偽?此實我兄弟輩之苦心,所以防奸,非以罔利也。」秀全信其言,置之不問,秀成憤然而去。

寇陷寧國,提督周天受等死之。寧國之陷也,玉成與賴裕新、古隆賢、楊輔清四面圍擊。周天受戰守七十餘日,軍中食乏,餉阻不能達,寇破竹塘、廟埠諸壘,副將硃景山等皆戰死。旌德、太平兩軍力單不能救,寇乘勢盡掃城外諸壘,城陷,天受遂遇害。寧國既失,南陵孤懸。總兵陳大富苦守閱半載,國籓檄令自拔出城,遣水師迎之,難民從者十餘萬。寇再陷太倉。玉成糾合江寧、丹陽、句容寇十餘萬,自九洑洲、新江頭掠船二百餘,日夜更番,意圖乘虛下竄,為軍團所敗,竄六合。鎮江寇船駛入丹徒、諫壁兩鎮港口,為水師李新明擊退。馮子材尋進解鎮江城圍。侍賢率寇四萬出廣德攻陷徽州,署皖南道李元度潰走開化。寇趨祁門甚急,國籓檄邀運蘭屯霍縣,趣鮑超自太平還屯漁亭,以捍大營。徽州既失,杭、嚴兩府防務益急。

寇之踞蘇城也,同時城邑陷者數十。江陰居大江尾閭狼山對岸再陷於寇,寇踞之以窺江北,人心惶然。九月,通州知州張富年等會水師攻復其城。上月玉成率悍寇二十餘萬進陷白爐橋尹善廷壘,旋至馬廠集,犯東津渡,黃鳴鐸擊卻之。至是圖取壽州內東肥河,跨山越谷,盤行抵淮河岸,聯營櫛比。餘黨竄入姚家灣,擄船,欲水陸會攻。巡撫翁同書派砲艇沿河截擊,寇乘霧鳧水入小港,為黃慶仁圍殺;復以步騎撲北關,城上彈丸雨下,夜縱火焚寇壘皆燼,城圍立解。寇竄南路:一還定遠山,一走廬江,一赴六安。徽州寇自淳安竄陷嚴州,進踞烏龍嶺。江寧寇糾九洑洲寇船二百餘艘下竄儀徵,我軍大敗之東溝。副將格洪額會破盱眙竹鎮集寇屯,擒斬偽檢點汪王發等,曾秉忠破青浦寇於米家角,攻城三日不下,秉忠中創。參將李廷舉攻寶山羅店,寇敗並嘉定,旋再竄羅店踞之。寇攔入壽昌、金華,兵團復之。旋再陷再復。

十月,寇自淳安擾及威坪,兵團御之,回竄蜀口。徽州北路寇竄至杞樟里,逼昌化昱嶺二十里。先是江西瑞金、廣昌、新城、瀘溪大股寇窺伺福建,汀州、邵武防軍力禦之,遂折竄建昌,而瑞金一股竄踞福建之武平,尋陷汀州,兇焰甚張。句容寇至鎮江湯岡築壘,馮子材擊之不下。寧國寇直趨四安,破長興長橋卡防,分竄廣坤、梅谿。嚴州寇連陷桐廬、新城。蘇州寇分股撲金山,我軍擊敗之,遂克楓涇鎮。寇復糾蘇、常大股襲廣富林,圖犯松江。守將向奎軍單,敗退。曾秉忠回援,寇竄寶山羅店,都司姜德設伏敗之,還青浦。玉良克嚴州。

新城寇竄陷臨安。初,壽昌被陷,金華知府程兆綸督民團復之,桐廬亦同時收復。於是寇眾悉趨富陽,副將劉貴芳、總兵劉季三敗死,城遂陷。旋收復,毀江口浮橋。侍賢復糾集臨安寇陷餘杭,逼杭州省城。侍賢由嚴州還顧徽州,瑞昌等敗寇秦山亭、古蕩、觀音橋,追至留下,寇棄壘走。省城解圍,遂復餘杭。侍賢不得志於杭州,自餘杭直犯湖州。建昌寇間道犯鉛山河口鎮踞之。福建浦城、崇安,浙江衢州、常山、開化邊防皆急。時徽州寇自深渡街口下竄天長,防軍會水師大破於三河、衡陽等處。是股為天長葵天玉、陳天福會合秀成黨三萬餘眾,將謀渡河分擾淮陽。秀成竄皖南,逾羊棧嶺,陷黟縣,鮑超大創之,城立復,再破之盧村,陣斬偽丞相吳桂先,秀成受傷遁徽州。國籓飭將屯守盧村,村距黟縣二十五里。是時侍賢自嚴還徽,輔清盤踞旌德,環二百里皆寇。秀成復由江蘇上犯,越嶺肆擾。我軍疾馳百餘里,力戰兩日,驅出嶺,祁門大營始安。

趙景賢大破賊,解湖州城圍。湖州自三月以來,迭被賊困,此趙景賢第三次解圍也。先是寇踞楊家莊為老巢,以礪山、仁黃山為犄角,焚掠雙林諸村鎮,蔓延長興、四安、太湖。景賢會軍先攻礪山、仁黃,以孤老巢之勢。我軍踞仁黃,毀楊家莊,敗寇西竄。天長寇掠下五莊舟船數百艘,欲犯湖路;我軍克河口鎮,復創之石谿;竄廣豐,道員段起御之,寇間道走玉山。多隆阿、李續宜會軍大破桐城寇陳玉成、龔得樹於掛車河、鶴墩、香鋪街等處,平寇壘四十餘,寇退奔舒城。

安慶者,江表之咽喉,實平吳之根本也。寇援安慶,水陸阻梗,不能直抵江寧。玉成眷屬悉在安慶城中,邀合發、捻十餘萬人,圖解城圍。多隆阿、李續宜雖力挫之,仍分屯廬江、桐城,復糾集下游江寧、蘇、常援寇並力上犯,逼近樅陽、桐城鄉村,眈眈以伺我軍之隙,將挾江南寇勢全力謀楚軍。時屆冬令,安徽城河水涸,道路紛歧。我軍四路告急之書應接不暇,皖南、浙江之寇分三大枝竄入江西,祁門各營圍裹於中,勢頗危急。湖南道州寇亦竄江西。寇既陷吳,勢必全力犯楚,此其深謀詭計。故安慶一城,寇以死力爭之。

左宗棠之入景德也,聞南贛寇分黨由貴溪過安仁,直撲饒、景,遣軍迎敗之周坊,寇竄陷德興踞之。十一月,宗棠進克德興,寇奔婺源,又克之。十日內轉戰三百餘里,寇驚為神速。彭斯舉解玉山城圍,寇竄衢境,犯常山,與兵團戰,敗走開化埠。楊輔清自池州率黨竄陷東流,進陷建德。水師收東流,而建德防軍潰退。國籓遣唐義訓馳擊,至利涉口,寇築壘河洲,列隊以待,並以馬隊扼拒各卡。我軍分東西兩路緣山上,立破其卡;前軍夾擊河洲寇,後軍抄其背,寇敗走。我軍復分為三進攻,寇出東門逸,遂復其城。寇復陷彭澤,闌入浮梁,越一日復之。寇趨馬影橋,逼湖口。玉麟督水陸軍力擊之,遂收彭澤。寇宵遁,陷都昌、鄱陽。我師馳至都昌,擊退踞寇,復之。

休寧寇犯上谿口,陷副將王夢麟壘。屯谿寇犯江灣,陷副將楊名聲壘。古隆賢、賴裕新糾大股犯羊棧、桐林二嶺,張運蘭會軍擊之,寇由新嶺退去,犯婺源。國籓督飭鮑超大破於黟縣盧村,別軍繞出羊棧,斷寇歸路。寇沿崖逃走,追軍反出其前,迫之,墜崖死無算。而休寧城寇以鮑超回剿景德,由藍田擾及小溪一帶,張運蘭擊敗之,別股屯鄭家橋者進逼漁亭。我軍兩路抄擊,寇狂奔,斃黃世瑚等;復敗上溪口寇,追至馬全街而還。自是嶺外寇不敢輕入。玉成率眾萬餘犯桐城、樅陽,我軍鎮靜固守。寇踞七里亭,韋志俊扼樅陽街口。李成謀舁三版入蓮花池護衛營卡,寇不得逞。

寇再犯景德鎮,宗棠敗之。鮑超進扼洋塘,宗棠進扼梅源橋。寇自下游糾大股屯洋塘對岸,我軍大破之,偽定南主將黃文金負創西奔。時祁門三面皆寇,僅留景德鎮一路以通接濟,寇盡銳攻撲,欲得甘心焉。時國荃圍安慶,寇勢漸窮。十二月,玉成糾約秀成、輔清及捻匪並力西犯,其大股寇、捻俱從南岸渡江而北,會於無為、廬江,以圖急援懷寧、桐城,勢甚猖迍。多隆阿等會於樅陽一帶,布署戰守。皖南寇大股竄孝豐,又別股由昌化竄分水。嘉興踞寇備具砲船,意圖南下,寇勢蔓延,浙東西同時告警。

十一年正月,傅忠信、譚體元、汪海洋、洪容海各挾眾數萬,棄石達開歸秀成。秀成驟增眾二十萬,勢大熾,由石埭分兩路趨祁門,防軍皆敗。江長貴援大洪,唐義訓迎戰歷口,斬偽麟天豫古得金,寇潰走常山。富陽、新城、臨安皆為我軍所克,解廣信之圍。寇竄鉛山、弋陽、貴溪、金谿,漸逼建昌。秀成自去冬犯皖南黟縣羊棧嶺不得志,竄浙江常山、江山等處;今春以全力攻玉山,轉圍廣豐,犯廣信,志在踞守要地,以通徽、浙之路。犯建昌,作浮橋渡河,以大股屯水東,環城築二十餘壘,以浮橋通往來。江西寇黃文金犯景德鎮,左宗棠、鮑超敗之石門、洋塘,斃許茂材、林世發。文金踉蹌宵遁,銅陵援寇與敗匪合,復入建德,分踞黃麥鋪諸處。鮑超督諸軍乘勝壓之,斃寇萬計,追至建德,會水師收復縣城,誅林天福。秀成梯攻建昌,參將富安等縱火具創退之,復潛為地道修子城備之。寇船犯太湖,陷東西山。全湖失陷,湖州北路七十二漊港橫被竄擾。太湖,巨浸也,襟帶蘇、常、湖三郡,港口紛歧,多至百餘。自蘇、常陷後,沿湖要隘多為寇有。去冬間寇船自湖州出湖,迭犯西山、角頭等處,為副將王之敬砲船所敗。然所部不滿十艘,募民船佐之,卒以眾寡不敵敗死。

二月,玉成圖援安慶,糾合捻首龔瞎子,率五萬人攻松子關。成大吉兵僅二千五百,寇多二十倍,分兩路抄官軍後。大吉令參將王名滔從左側山橫截而出,陣斬龔瞎子,寇驚潰,復選悍黨分五路進,再戰再敗,捻散亡三萬人。初,玉成嗾龔瞎子犯松子關,而自率悍黨十餘萬,從霍山之黑石渡,襲餘際昌營於樂兒嶺,相持四晝夜,力竭而潰,遂抵英山,入蘄水,襲陷黃州。分黨取蘄州,擾麻城,闌入黃安、黃坡、孝感、雲夢諸縣,並陷德安府、隨州,勢益猖獗,武昌戒嚴。

秀成聞玉成攻國荃久不下,分攻蘄、黃、廣濟,欲國籓赴援以分兵力。秀成嘆曰:「英王誤矣!正使國籓得全力以攻皖,彼豈暇救此閒城哉?彼有長江之利,而我無戰艦,安能絕其糧道?不能以我攻浙救京師為例也。」玉成屯孝感,而以德安、雲夢、隨州三處為長蛇陣,窺伺荊、襄。官文飛調李續宜、舒保、彭玉麟率水陸諸軍回救。

侍賢竄踞休寧城,築壘上溪口、河村、石田、小當等處,與休寧屯溪之寇互為犄角。國籓以休城不克,徽郡難圖,祁門終屬危地,檄硃品隆等進攻,焚諸壘,寇夜遁,收復縣城。左宗棠進剿婺源寇,破侍賢於清華街,而城寇忽分黨由中云竄入樂平界。宗棠親率數營屯柳家灣,扼其沖,寇敗退;而援寇漫野至,返旆截殺,復大潰而去。侍賢糾徽州悍黨圍王開林於婺源甲路,越三日,潰圍出,還景德。

秀成率黨竄撫州,為知府鍾峻等擊敗,竄宜黃,復糾土匪陷遂安。國籓遣大富防景德。宗棠進軍鄱陽,次占魚山,聞寇偷渡昌江,圖合圍景德,旋移駐金橋。寇竄平湖,分由西路櫸根嶺、北路禾黍嶺進犯。副將沈寶成當西面,江長貴出北路拒之。國籓復檄硃品隆由祁門馳援,殲,寇越嶺而逃;宗棠擊破於樂平範家村,陣斬偽謝天義黃勝才、偽嬈天福李佳普等。

侍賢率數萬眾潛匿於牛嶺、柳家灣、回龍嶺,翌日,齊進景德鎮,陳大富戰死,陷之。金魚橋坐營後路已絕,遂移屯樂平。初,國籓至皖南,設糧臺於江西,以景德鎮為轉運。寇之窺祁門者,屢遭挫敗,遂悉銳再犯景德,冀絕大軍餉道,至是陷之。國籓度糧路已斷,惟急復徽州,可通浙米,親至休寧攻徽寇不克,仍屯祁門;而寇環攻不已,誓以身殉。宗棠大破寇於樂平,斬馘數萬。侍賢遁,圍建昌、撫州,攻之不下,遂陷吉安,大軍旋復之,乃進陷瑞州。於是祁門之路始通。

三月,我軍克新淦,解麻城圍。繁昌荻港、蕪湖魯港寇皆敗走。民團克雲夢,收應城、黃安、黃坡。金國琛會水師克孝感、進攻德安,逼城為壘。嘉興寇竄陷海鹽、平湖,宗棠敗寇於龍珠、桃嶺。寇渡吉水海灘,陷吉安,復為知府曾詠等攻復。寇由吉安東犯,分股竄峽江,與新喻賊合並,屯陰岡嶺,臨江告警。玉成分股守德安、隨州,牽綴我軍,率悍黨由蘄州、黃、廣回宿松,進太湖大營後路,繞趨宿松桃花鋪,逕竄石牌,逼安慶集賢關築壘。未幾,桐城、廬江偽章王林紹璋、偽干王洪仁玕等率二萬自新安渡至橫山鋪、練潭一帶,連營三十餘里。至馬踏石,竄安慶,與玉成會解城圍,多隆阿分擊敗之。玉成闌入集賢關,攻圍師各壘;復於菱湖兩岸築壘,阻水師進攻。楊載福遣軍舁砲船入湖,毀寇船筏,復立壘湖嘴,使寇不敢逼水陸大營。多隆阿自桐城掛車河進攻安慶援寇林紹璋等於練潭、橫山,逼溺菜子湖無數。黃文金糾蕪湖寇及捻二萬,築壘天林莊二十餘座,謀入安慶城,多隆阿誘斃二千,國荃圍師不少動。水師焚奪寇船,斷其接濟,以困城寇。

侍賢自廣信竄常山,陷之,入踞江山,旋由常山分股犯衢州。先是福建援師自衢防回援汀州,軍勢驟孤,故寇乘虛回竄。又別股由開化白沙關竄玉山童家坊,偽為難民呼城,砲創之;常山寇復至,會攻城,詐為援兵,奮力擊之,退屯三里街、七里街,潛掘地道,道員王德榜縋城出,大破之,寇還常山。遂安寇直抵淳安港口,副將餘永春中創敗退;茶園寇踵至,再退桐關。嚴州大震。

四月,國籓自祁門移駐東流,多隆阿擊敗玉成,棄壘遁,屯集賢關。國籓復撥鮑超一軍,胡林翼撥成大吉一軍,同赴安慶。初,玉成即菱湖北岸築壘十三,城寇葉蕓來出城接應,亦築五壘於南岸,以隔國荃與其弟貞幹之師。國荃掘長壕,包寇壘於長壕之內。玉成前阻圍師,後受鮑、成兩軍夾攻,計窮遁去,猶死守關內外寇壘,又於隨州、德安各留悍寇牽掣我兵。官文派軍攻德安,築長圍困之。

秀成踞義寧州武寧縣,逼近湖南北邊境。官文派軍分守興國及崇、通、山、冶四縣。寇裹脅七八萬人,一由苦竹、南樓二嶺犯通城,一由蛇箭嶺犯通山。我軍眾寡不敵,均被闌入,直抵崇陽之白霓橋。其窺伺興國之寇,撲餘際昌營,官軍戰失利,退大冶。寇隨至大冶,並擾武昌。官文咨調李續宜等屯東湖、跕紙坊一帶,相機進剿。秀成自孝豐竄四路,陷長興、壽昌,分犯三里亭、千家村;復自瑞州分竄西路,連陷上高、新昌,北路陷奉新擾義安,以阻援師。連日曾秉忠等水陸諸軍破走乍浦、平湖寇於金山各隘。金山與浙之平湖水陸交錯,薛煥與秉忠商籌,平湖一日不復,松江屬一日不安;謀越越境會攻平湖,再圖乍浦。

當陳玉成之退走也,多隆阿已進軍磨盤山,遣溫德勒克西、曹克忠、金順等分途尾追。玉成復糾合林紹璋、洪仁玕、黃文金及格天義陳時永、捻首孫葵心共三萬餘,並力上犯,築八壘於掛車河、夬峔尖迤西釭盤嶺;率黨破山內黃山鋪團卡,仍出山外調黃文金四千餘人伏山內,自率悍黨分道進犯我壘。多隆阿分軍設伏於釭盤嶺、老梅樹街,而自率馬步各軍分道拒戰,寇後隊忽自亂,老梅樹街伏騎乘之,勢不支。玉成督敗黨抵敵,而項家河寇壘為舒亮伏兵襲焚,煙焰突起,寇大驚,敗奔桐城。八壘悉平,燒山內寇館數十處,斃八千餘。上命左宗棠幫辦軍務。

寇越衢州陷龍游,連陷湯谿、金華,紹興、寧波皆大震。寧、紹為浙東完善之區,寇垂涎已久。金華既陷,勢將內犯。寇犯丹徒,水軍敗之,毀寇浮橋。曾秉忠自金山攻青浦,寇堅壁不出,敗嘉善援寇於章練塘。寶山防軍姜德攻嘉定以分寇勢。都興阿敗天長、六合竄寇於揚州西北鄉,盡毀甘泉山寇壘。秉忠自金山洙涇率砲船進破白虎頭、金澤鎮寇巢,直抵浙境,敗西塘援寇,進毀俞匯卡,寇退入嘉善。金華寇分股陷蘭谿、武義。

五月,鮑超、成大吉破集賢關外赤岡嶺寇壘三,殲寇三千餘。偽屈天豫賈仁富、偽傅天安李仕福、偽垂天義硃孔棠等皆伏誅。大吉回援武昌,餘一壘超獨破之,擒斬劉瑲林。瑲林陷蘇、常為前鋒,自恃其勇,欲以孤壘遏官軍,既伏誅,國荃軍勢自倍。國籓之移東流也,皖南寇度嶺內空虛,糾眾由方干嶺樟樹衛防軍而入,潛陷黟縣,築壘西武嶺等處,窺伺祁門。張運蘭等克黟縣,寇並入盧村十都,增壘抗拒。我軍克其七壘,寇悉駢誅。徽州寇聞之竄走。

宗棠追剿侍賢至廣信,以建德再陷,竄入鄱陽視田街,急回景德。寇宵遁,宗棠截之,大戰㭎樹嶺,寇走建德後河,遂復縣城。運蘭進攻徽州,復之。汀州寇由江西瑞金回竄。時江西寇竄江山,進陷遂昌。秀成以一股踞瑞州、義寧、武寧,分三路犯湖北,連陷南岸興國、崇陽、通城、大冶、通山、武昌、咸寧、蒲圻,寇鋒逼武昌省城。官文、李續宜會遣水陸軍分道進剿,胡林翼亦自太湖移軍還省,先援南岸,再圖黃、蘄。

寇之竄擾江西者,自去冬以來,前後凡五大股,其由皖境竄入,自北而南者三股:一曰黃文金,連陷建德、鄱陽六縣;一曰李侍賢,連陷浮梁、景德等處。此二股均經左宗棠擊退,未能深入江西腹地。一曰李秀成,連圍玉山、廣信、廣豐三城,又深入內地,圍建昌,撲撫州,均未破,竄入崇仁樟樹鎮、吉安峽江,並踞瑞州府城,分竄奉新、靖安、武寧、義寧各州縣,又竄入湖北之興國、大冶、蒲圻、崇、通等處。此北三股也。其由兩廣竄入,自南而北者二股:一曰廣東股,其渠有周姓、許姓,上年由仁化、樂昌闌入江西,與李秀成聯合,圍攻廣信、南豐、建昌各城,連陷湖口、興安、婺源,經左宗棠攻克德、婺兩城,遂歸並徽州。一曰廣西股,其渠為硃衣點、彭大瞬,本石達開之餘黨也。由江西竄出湖南,經過南贛,陷福建之汀州,回竄江西,蹂躪寧都、建昌、河口等處。其前隊已由婺源竄浙,後隊尚留撫州。此南二股也。五大股中,又分為三支、四支,忽分忽合,時南時北。

國籓令鮑超回援江西,由九江直搗建昌,先保江西省城。瑞州及各縣踞寇逼近南昌,毓科留張運桂等扼屯城外,劉於潯屯安義堡,後營屯生米,丁峻屯臨江,皆為省垣西路屏蔽。李續宜克武昌,寇陷松陽、處州、永康、縉雲。縉雲既復,寇竄永康。宗棠遣軍克建德,國籓移屯婺源,婺源者,界江、皖、浙三省之沖也。賴裕新合汀州寇犯德興,分黨踞九都之新建,遣軍敗之。寇渡江竄浙江開化華埠,德興、婺源肅清。游擊黃載清克遂昌,松陽踞寇亦聞風遁,載清進攻宣平克之,寇竄武義,處屬肅清。金華知府王桐等克永康,寇並趨金華。江陰、常熟寇由海壩竄壽星沙,大肆焚掠。

六月,曾國荃會水師破菱湖北岸十三壘、南岸五壘,斬馘九千餘。寇之踞湖北咸寧、蒲圻、通城者,我軍均克復。寇由武寧犯建昌。金華寇出擾曹宅等處,李元度克義寧。張玉良等率水陸軍圍攻蘭谿不下。寇偵嚴州軍單,築壘女埠,下竄嚴州,陷之,尋為張玉良所復。寇陷上高,擾萬載。知縣翁延緒等克復武寧。水師李德麟等擊毀黃山、黃田、石牌三港寇船,壽星沙寇退回江陰。蘇州寇擾青浦,李恆嵩自北簳山移屯塘橋,以固松江門戶。嘉定寇糾蘇州寇犯上海,我軍御於真如。寇渡河竄華漕,奪踞參將王占魁壘,薛煥遣軍奪還。寇走南翔,再敗退嘉定。時江水盛漲,國籓檄楊載福圖池州,牽掣南岸。載福攻十日不能下,乃率李成謀三營至舊縣,偵北岸無為瀕江壘寇避水移神塘里河,駛擊破之,進攻州城。陳玉成、楊輔清死力抗拒。還軍次大通,敗青陽寇,還屯黃石磯。

七月,玉成糾輔清眾十餘萬自無為州犯英山,繞宿松,徑攻太湖,為救援安慶計。寇排隊山岡作長蛇勢。復有寇數萬自龍山宮對岸至塔下,袤延二十餘里,分路誘我軍,我軍堅守不動。夜大雨,賊洶湧潮進,城中飛丸隨雨落,至晨圍始解。玉成乃自小池驛進至清河高樓嶺,欲結桐城寇包裹我軍,以解安慶之圍。攻撲六晝夜,玉成、輔清援桴鼓督軍,揮刀砍不前者。我軍奮擊,大挫其鋒。玉成、輔清率大隊竄至高河鋪、馬鞍山,桐城圍復合。安慶圍師悉平城外諸壘。秀成自竄踞瑞州,分陷上高、新昌、奉新等縣,以瑞州為老巢。官文檄元度會軍攻克新昌、奉新、上高,敗寇均趨瑞州。乃遣游擊賀接華等會軍直搗郡城,攻復之。官文以德安寇謀窺荊襄,派水陸諸軍節次嚴剿,寇堅守,有冒死突出者,諸軍擊潰,寇不得入城。寇復出抄我軍後以援前寇,乃斷其歸路,寇奔河西,伏兵四合,架梯入,立復府城。

程學啟克安慶北門外三石壘,北門寇路已絕。德安寇竄河南信陽,轉犯羅山、光山、商城,兵團截擊,退奔皖境。鮑超督諸軍大破豐城西北岸寇壘,東岸屯寇驚潰,劉於潯乘之,收樟樹鎮。初,超自九江進軍,秀成聞風遠遁,率瑞州、奉新、清安、安義之寇,先分萬人擾撫州;令玉成率悍黨二萬攻豐城,而自領大隊由臨江踞樟樹、沙湖、豐城一帶,綿亙百餘里。先一日,超至豐城對河,值寇在樟樹為浮橋,陣山岡,超分中、後、左、右四隊齊進,寇搖旗迎拒,戰一時許,大敗,馘八千餘人。

八月,克安慶城,城外四偽王竄集賢關。安慶既復,東南之勢益促。水軍進克池州,乘勝下剿,復銅陵縣。時偽右軍劉官才方盤踞池州,與安慶相犄角。內則堅守石埭、太平,阻徽師進兵之路;外則上犯德、建、鄱陽,為江省北邊之患。今與安慶相繼而下,皖南軍勢益張。國荃與多隆阿會議,以桐城為七省要道、安慶咽喉,寇死守待援;玉成尚擁眾數萬,徘徊於集賢關內外,謀與桐城合並。乃會軍進擊,玉成、輔清皆大敗,越山而逸,遂復其城。宿松、黃梅、蘄州、廣濟相繼下。多隆阿進屯蘄州曹家渡,扼下游敗寇,以絕黃州寇援。李續燾等會同水師進攻黃州,寇築壘浚壕以抗我軍。蔣凝學令投誠劉維楨服寇衣,偽為援眾;復造玉成偽文,誘寇出,而設伏以待,寇果出,為我軍所殲,立復黃州。

時秀成竄出豐城,踞白馬寨,遣黨攻撫州。鮑超馳至,撫圍立解。寇走貴谿,得廣東新寇,合大眾據湖防河口,勢甚洶湧。超分五路應之,蹋毀七十餘壘,進復鉛山。國籓移駐安慶省城。初,侍賢攻嚴州,兩月不能下,乃於烏石、方門二灘連環築壘,逼近外壕。城內糧盡援絕,副將羅大春受重傷,率將士突北門出,城遂陷。秀成自桐廬、新城進陷餘杭。寇之竄貴谿者,聞鮑超自撫州至,豫遁走。初,閩寇三起,與花旗廣匪先後由建昌竄至廣信,與秀成並為一路。鮑超會屈蟠大破廣信寇壘,立解城圍。秀成敗走鉛山,築七壘,與城寇相守御。超踵至,悉覆其壘,渡河攻城,克之。秀成竄圍廣豐,不克,分竄玉山,築壘十餘,復為道員王德榜所破。

秀成全股悉自江西竄犯浙境,一由玉山陷常山,一由廣豐犯江山,龍游踞寇同時出擾,衢州危迫。是月知府張詩華克復瀘谿、興安各城。福建援軍張啟煊擊寇浦江,失利,陷之。寇直犯五指山,金華大股踵至,米興朝等迎擊失利。義烏、東陽相繼不守。啟煊退守諸暨闢水嶺,寇至再潰。九月,寇陷處州。

大軍之破安慶也,無為州寇馬玉棠妻子居安慶,曾國荃生致之,密諭玉棠獻無為城。無為居皖北形勢,控金陵,引蕪湖,為寇必爭之路。附近泥汊口、神塘河諸處,石壘星列,以阻我軍。曾國荃會同水師抵泥汊口,壘高難仰攻,乃令築寨安營,而自率勁旅迅赴楊家橋、鳳凰頸,決堤斷寇歸路。寇大恐,遁入城。越日,攻神塘河,寇亦遁歸,乘勝直抵城下。至是玉棠事洩,偽頂王硃王陰幽之。玉棠黨舉兵攻王陰,我軍乘之,寇大潰。斃偽豫侯、丞相等,城立復。

秀成由臨浦陷蕭山,再由蕭山塘路竄杭州,陷諸暨、紹興府城,分竄新昌、嵊縣。上虞、餘姚均先後失守。國荃連破運漕鎮及東關鎮,鎮在無為、含山之界,外瀕大江,內連巢湖,寇糧皆屯於此,上濟安慶、廬州,下輸金陵,為南北鎖鑰。偽巨王洪某率眾五六千,並砲船數十,守之。寇失此益膽落。

十月,楚軍克復隨州。湖北自黃州、德安復後,惟隨州以孫捻援應,擾及襄陽,憑堅死抗。官文擊退豫捻,復用降將劉維楨取黃州計,誘寇出城敗之,克復州城。多隆阿收舒城、廬江,李續宜部將蔣凝學屯六安、霍山,宗棠屯婺源,張運蘭等屯徽州,李元度新軍出廣信,寇悉赴浙江,而皖南寇聚保廬州。宗棠議大舉援浙,浙寇猖獗,全省糜爛,逼近省垣。

上命曾國籓管轄江蘇、安徽、江西、浙江軍務。秀全見各省攻討嚴劇,迭克名城,金陵唾手可下,乃大懼,令秀成、侍賢分途竄擾,分我兵力。秀成竄浙江,疊陷各郡邑,直至餘杭,浙省戒嚴。寇從塘路抵杭州,撲武林門外賣魚橋,踞我營卡。寇隊尋大至,運糧道阻。提督張玉良來援,與城軍夾守。寇乃自海潮寺至鳳凰山,環木柵實土其中為堅壁,使城外隔絕,日以槍砲轟城。玉良攻木柵,中砲死。內外兵益懼,而城中久乏糧,人多餓死。自蕭山、諸暨等城陷後,援兵路絕。寇尋陷奉化、臺州,十一月,由慈谿犯陷鎮海。臺州寇分股陷黃巖、寧波,會同輔清由浙江嚴州遂安逾嶺回竄徽州,復蔓延衢屬開化,其謀在深入江、皖腹地,阻我援浙之師。

宗棠於廣德奉督辦浙江軍務之命,以寇圍徽郡,當入浙後路,遣軍至婺源,會防軍援剿。輔清大舉分犯徽州、休寧,兩敗於屯谿、篁墩,遂趨南路,逕逼嚴郡;而秀成已陷杭州,滿城亦相繼失守。輔清率寧國寇圍攻徽州、休寧兩城,偽成天安竄休寧之屯谿,尋又竄篁墩。十二月,總兵張運桂堅守徽城待援,乘間出擊破寇壘。寇復踞屯谿、市街、潛口一帶,以絕徽軍糧道。國籓調總兵硃品隆馳赴休寧,與唐義訓先破屯谿街口寇卡,毀河邊四壘,進平石橋、潛口寇巢,尋派軍護糧赴徽。寇由萬安街分股包抄,我軍奮擊之,寇不敢遏。張運桂以嚴市街為寇踞,糧道梗阻,與休寧軍會商兜剿,毀寇十餘壘,輔清受傷。值除夕大雪,寇掠無所得,悉遁去,圍立解。

是月提督李世忠克復天長、六合。寇自八年踞六合,屢攻不下。黃雅冬思反正,潛約世忠先削其外壘,而己為內應。世忠自滁州至六合,遂大破寇壘,斬馮匊林,直造城下。黃雅冬倒戈,縱火拔關,我軍擁入城,擒斬沖天福林國安、頂天燕江玉城、攀天福魏正福五十餘人,皆悍黨也。詔黃雅冬更名朝棟。世忠以黃朝棟密約天長寇陳世明為內應,朝棟所部未盡薙發,詐為援寇進城下,世明拔關納軍,遂克之。

是歲汀州寇竄陷連城,分竄上杭。江西寇復闌入武平境,時寇謀分股:一圖窺龍、巖,一由清流、寧化擾延平、邵武。總督慶端以延平為全省關鍵,馳往駐札糸;克連城,進攻汀州,創洪容海,克之。而江西續竄武平之寇,亦為副將林文察等擊敗。國籓疏請飭慶端嚴守浦城,俾寇不得由閩境竄江西。

同治元年正月,是時浙、蘇兩省膏腴盡為寇有,全浙所存,尚有湖州、海寧兩城,又孤懸賊中,獨衢州一府尚可圖存。國籓疏薦福建延邵建寧道李鴻章署理江蘇巡撫,別立一軍,由滬圖蘇;以圍攻金陵屬曾國荃,以浙事屬左宗棠。於是東南寇勢日就衰熄。世忠攻江浦,約劉元成、單玉功為內應,殺偽報王、偽宗王、操天福等七十餘名,立復江浦;至浦口,復克其城。自江、浙兩省寇糾眾數十萬,並力東犯,連陷奉賢、南匯、川沙等縣,烽火遍浦東,逼上海。各隘防軍遇寇輒潰走。寇既為法船所擊退,踞天馬山陳防橋,復為李恆嵩所破,敗入青浦城中。其浦東大股踞高橋,欲斷我要隘。美人華爾、白齊文撲寇巢,毀其壁,進攻浦東、浦南,大破之。尋嘉定、青浦寇進逼七寶,以窺上海圍防軍郭太平營。薛煥督軍解其圍,甫還軍,寇復大至。

李鴻章率湘淮軍援江蘇,營上海城南。黃翼升統水師相繼至。宗棠出嶺攻開化,楊輔清糾眾踞張村、銀坑、石佛嶺,窺伺衢州,連戰走之,陣斬藍以道,開化肅清。九洑洲寇竄江浦、浦口、和州,分黨犯橋林。二月,嘉善、平湖寇水陸犯金山、洙涇,七日,陷之。松江、上海俱震。東梁山偽愛王黃崇發、西梁山偽親王某、裕谿口偽善王陳觀意糾合雍家鎮寇,分股上竄五顯廟、水家村、湯家溝,水師李成謀登岸破之,斬黃崇發。初,宗棠既克開化,進軍常山璞石,偵李侍賢嗾壽昌蘭谿寇糾遂安恭天義賴連繡犯開化馬金,謀長圍困我。宗棠回攻遂安,乘援寇未集,先剿克之。

三月,李元度敗寇於江山碗窯,再敗小青湖援寇。江山賊進陷峽口閩軍曾元福壘,踞之,連營數十里。宗棠自常山遣軍進剿。林文察之克遂昌也,寇並聚松陽,蔓延雲和、景寧,尋以次擊退,進規處州,破寇松陽平港頭寇壘,寇奔處州碧湖。上年國籓檄鮑超平青陽城外寇壘,作長圍困之。偽奉王古隆賢潛糾浙江死黨三萬餘人進撲銅陵,分我軍勢。超率援軍盡掃橫塘等處百餘卡三十六壘。古隆賢偵知大軍北出,陰統涇、太悍寇築九壘於青陽豬婆店阻我師,以通糧道。適超凱旋,進逼青陽,克之。

國荃既募湘勇回安慶,國籓飭令攻取巢縣、含山、和州、西梁山等處,以為欲制寇死命,先自巢始,遂進屯縣城東北。劉連捷等赴望山岡以扼南路,寇乘我壘未定來犯,拒敗之。初,寇踞江岸以北,上援廬州,下衛江寧,分布堅城,拒守要塞,通上下之氣,杜我進兵要路。至是攻破巢縣銅城閘、雍家鎮,旋收巢縣,復含山,再會水師大破裕溪口寇屯。

寇之踞金陵也,以全力扼東西梁山,兩山對峙,大江至此一束,水急流洄,視小孤山尤為形勝。國荃以此關為金陵鎖鑰,循江上逼西梁山而陣。寇走江州,水陸爭起搏擊,遂克之。自是金陵重鎮已失其半。偽匡王賴文鴻竄繁昌縣,糾約踞寇撲我三山峽營。曾貞幹乘寇未逼,督軍分道馳剿,寇大亂,殪其渠吳大嘴,遂克繁昌。蕪湖援寇潛屯魯港以圖抗拒,貞幹會同水師奪其船百餘艘,夷三壘十餘卡,餘悉遁蕪湖;貞幹督各軍馳抵南陵,立復其城。鮑超既克青陽,議先取石埭、太平以固徽州之防,繼取涇縣以達寧國之路;乃分軍五路,由龍口直抵石埭,寇矢石交下,總兵婁云慶從西、北二門攻入,立復縣城。寇大懼,糾集南陵援寇大至,屯甘棠鎮阻我師。我軍分三路進擊,破十七壘,遂復太平,擒斬偽主將徐國華等三十七人。先是超攻青陽時,有偽佐將張遇春率眾萬餘乞降,超納之,令暫屯景德三谿。至是太平賊寇將北走,張遇春驟起殲之。超回軍挾之,進攻涇縣,復其城,遂東渡清弋江,進規寧國。

李侍賢自陷常山,與龍游寇同撲衢州,總兵李定泰等敗之,復常山,解衢州之圍。宗棠馳至常山,攻克招賢關,以通衢州糧道。其撲江山之寇,檄李元度會攻石門花園港寇巢,毀十四壘,寇勢不支,乘夜向臺州潛遁。是時楊輔清由淳化犯遂安,宗棠自常山進屯開化圖之,尋遣劉典進攻遂安;輔清退走淳安、昌化,竄皖南犯寧國。四月,儀徵寇敗後,竄擾沙漫洲等處,水師擊走之,揚境肅清。浙軍會民團克復臺州仙居、黃巖、太平、寧海,縉雲、樂清、慈谿等縣,擒斬延天義李元徠,偽王李洪藻、李遇茂,偽主將李尚揚。尋進克寧波府鎮海、青田二縣。臺、處寇上竄溫州,踞太平嶺及任橋、瞿谿,尋分竄瑞安,張啟煊營陷,退守縣城。楚皖軍會克廬州,秀成竄擾蘇、常,玉成則盤踞皖、楚之交。

自大軍克復安慶,玉成率黨自石牌而上,調宿松、黃梅之寇同至野雞河,欲赴湖北德安、襄陽招集其黨,群酋不從,乘夜由六安走廬州,眾漸攜貳。秀全復督責甚切,玉成懼,力守廬州不敢走,皖、楚諸軍困之,日盼外援。而潁郡解圍後,偽扶王陳得才西竄,偽天將馬融和隨張洛行遠遁,外援遂絕。多隆阿與皖軍張得勝設伏誘賊出戰,兩軍合擊,寇大敗,遂克府城,誅偽官二百十三名。玉成奔壽州,尚有死黨二千,以苗沛霖陰受偽封,往乞援;沛霖縛之,獻潁州勝保大營,並擒偽導王陳士才、偽從王陳德、偽統天義陳聚成、偽天軍主將向士才、偽虔天義陳安成、偽禱天義梁顯新,及親隨偽官二十餘,並正法。秀成聞玉成死,頓足嘆曰:「吾無助矣!」玉成兇狠亞楊秀清,而戰略尤過之,軍中號「四眼狗」。自玉成伏誅,楚、皖稍得息肩,而金陵勢益孤矣。

民團克寧海、象山、奉化。李世忠大破寇於六合八步橋,寇竄滁州來安,又大敗。寇勢大餒。華爾等克柘林,謀搗金山衛。知府李慶琛攻太倉,秀成率偽聽王陳炳文、偽納王郜雲官來援,逼青浦。李恆嵩失利,退走塘橋。嘉定、寶山皆震。寇別隊由婁塘攻陷嘉定,英、法二提督及我軍突圍走上海。鴻章飭軍駐法華鎮,扼滬西,寇遂逾青浦,逕逼松江。秀全自竄踞金陵,以東西梁山為鎖鑰,以蕪湖為屏蔽,而尤以金柱關為關鍵。自國荃破太平府,彭玉麟破金柱關,黃翼升往襲東梁山一戰而下,貞幹循江進克蕪湖,提督王明山等復攻破烈山石壘,未逾三日,上下要隘悉為我有。從此上而寧國,下而江寧,寇均失所恃矣。

國荃進軍江寧鎮,屯板橋,潛襲秣陵關,進破大勝關、三汊河,直抵雨花臺軍焉。江北浦口、六合敗寇悉聚江邊,李世忠蹙之,渡還九洑洲。江北肅清。五月,偽什天安建瀛統眾聚南匯,與淋天福劉二林屢為秀成養子所凌,至是來降,我軍整隊入城。秀成養子方踞金山衛,來犯,復糾川沙寇回撲,復為我軍所敗,直逼川沙。寇由海塘竄出,遂復川沙,進攻松江。寇逼泗涇,防軍游擊林叢文敗退北門。華爾由青浦回援,鴻章令程學啟扼虹橋,分青浦、松江後路。寇陷湖州,福建糧儲道趙景賢拔刀自刎,賊奪之,囚一年餘,秀成禮待之極厚,終日罵不絕口,譚紹光舉槍一擊而殞。

初,寇圍逼松江,鴻章以松江扼青浦東、西之中,為最要地,自赴新橋,令程學啟時出兵綴寇。寇初營西門妙嚴寺土城,華爾以砲毀之。寇復攫據之,增築砲臺,環合四門。常勝軍戰寇竇福濱,城軍乘夜分門出擊,寇竄遁。乃簡精卒破天馬山寇營,突入青浦,盡焚輜重。寇死戰,並力守松江。其分屯廣福林及泗涇之寇,鴻章進擊敗之。寇遁入營,斷橋以拒。劉銘傳等克奉賢,陳炳文、郜雲官等率眾數萬圍新橋程學啟營,填壕拔鹿角,學啟不及以槍砲御,擲磚石擊之,寇藉尸登壕,學啟開壁突擊,寇始卻,而分股逾新橋逼上海。鴻章將七營往援,大破之,追至新橋,學啟大呼夾擊,寇解圍遁。陳炳文、郜雲官皆負傷竄走。進軍泗涇,寇大潰,盡燒其壘。廣福林、塘橋寇亦退。上海、松江俱解嚴。

初,李世忠遣軍自六合通江集南渡,連破石埠橋、龍潭、東陽寇壘,寇悉遁句容。自是九洑洲寇外援盡絕。秀全遣江寧寇大攻石埠橋曾玉梁壘,世忠遣義子李顯發往援,入壘會守。陳坤書自句容進攻龍潭、東陽諸壘,守軍黃國棟等退並石埠,而寇攻益急,顯發會水師力戰,盡平其壘,解石埠圍。時李秀成自松、滬敗還,謀連合杭、湖寇眾救援江寧。秀全遣悍黨二萬攻大營,國荃設伏敗之。連日宗棠督軍攻衢州,東、南、北三路寇壘皆盡。

初,秀全以大軍驟逼城下,日出撲犯,輒被創,趣浙酋李侍賢、蘇酋李秀成還救江寧,而宗棠攻衢州,與李侍賢相拒遂安、龍游間。鴻章新克松江縣,秀成奔命未遑,乃與諸黨議曰:「曾國荃兵力厚集於金陵,為久困之計。我勢日蹙,不如先圍寧國、太平,斷其後路。我軍勢既振,敵乃可圖也。」秀全以久困,慮糧不繼,仍促其入援。秀成不得已,乃先遣悍黨數萬自蘇西援。時寧國餘寇竄並江寧,屯於淳化鎮者亦不下二萬餘。六月,宗棠自衢州進軍龍潭。侍賢自遂安敗後,復糾合金華及溫、處悍黨,分屯南岸湖鎮、羅埠,北岸蘭谿之永昌、太平、祝家堰、諸葛村、孟塘、油埠、裘家堰。宗棠駐潭石望,距城十五里,遣眭金城會劉培元駐城西圭塘山,屈蟠、王德榜駐紫金旺,崔大光駐城北對河茶圩,劉典駐高橋。先是李侍賢偵我軍平衢州寇壘,將進軍龍游,糾黨潛趨蘭谿、嚴淳,乘虛襲遂安,為就糧江、浙斷我餉道計。宗棠遣軍大破之,寇由壽昌退還金華。七月,我軍進敗龍游寇,毀蘭谿、油埠寇屯,陣斬偽駿天義鄧積士等。宣平、處州、餘姚、壽昌均以次收復。

是時楊輔清糾眾十餘萬竄踞寧國府城,復分黨屯聚團山、寒亭等處,阻我進兵之路。鮑超飭將卒撲寒亭,寇出巢猛拒,總兵宋國永橫躍入陣,伏起扼歸路,寇驚潰,平寒亭管家橋、楠家甸、獅子山寇館數十處,寇壘三十五座。偽衛王楊雄清糾合餘眾遁回寧郡。輔清聞寒亭戰敗,即糾黨繞城結壘,延三十餘里。鮑超進壁烏紗鋪,飭婁云慶設伏望城岡,以輕兵誘寇。寇以我兵寡,直壓山岡而下,我軍張兩翼卻之。寇見旌幟遍山谷,誤為援寇,反斗中伏,我軍復斷其後,斃無算。望城岡及抱龍岡十數村皆平。寇復陰結別股築壘堅拒,鮑超率各軍逼壘而營。各偽王出大隊於南、北兩門夾攻,鮑超分軍進搏,寇敗走浮橋,我軍焚橋截殺,無得脫者。寇復收餘燼,再戰再敗,輔清單騎脫走。立收寧國府、縣二城。初,輔清聞超軍至,數遣使乞援於江寧,秀全遣偽保王洪容海率悍寇赴之,容海者石達開死黨也,既至,懾超威,乞降,超許之,而郡城已克。容海奔廣德,襲獻州城,率眾六萬就撫,復本姓童。

江寧援寇大舉犯壘,分二十餘隊牽掣各軍,而以銳卒突雨花臺。國荃拔卡縱擊,大破之,解圍去。潘鼎新等會克金山衛城,地界江、浙,為浦東門戶,至是一律肅清。是月鴻章督諸軍會攻青浦,克其城。譚紹光方踞湖州,聞青浦已失,恐官軍躡其後,乃合嘉、湖、蘇、昆寇犯松、滬西北,進窺青浦,學啟會水師擊卻之。寇攻北簳山壘不獲逞,遂東趨攻北新涇,北新涇為上海西路之蔽,防軍大戰卻之。寇分竄法華,逼上海。鴻章調諸軍自金山衛、青浦、松陵回援,悍寇二萬圍官軍,學啟戰逾時,寇大潰。北新涇之寇憑河據壘,伏左右以待我軍。鴻章親督陣,與學啟軍合,盡毀寇營,紹光遁嘉定。上海危而復安。

八月,寇陷慈谿,華爾復之,受創,尋卒。嵊縣、新昌寇陷奉化。閏八月,奉化寇窺寧波,宗棠飭蔣益澧等進平蘭谿裘家堰寇壘,斃偽元天福萬興仁、偽疌天福劉茂林等。羅埠踞寇偽戎天義李世祥乞降,益澧攻羅埠,世祥應之,破五壘。湖鎮寇聞之遁,我軍渡河破五星街。此皆龍游、湯谿要沖也。益澧屢攻湯谿不下,軍多死傷。宗棠進營新涼亭,逼龍游城五里,飭益澧由羅埠進攻湯谿。劉典屯扼油埠、湖鎮,以堵蘭谿、金華援寇。

九月,鴻章會常勝軍攻嘉定,克之。紹光及偽聽王陳炳文復糾援寇十餘萬,分道自太倉、昆山來犯,北由蟠龍鎮至四江口,圖據黃渡以當青浦;南由安亭至方泰鎮,圖入南翔。尋我軍卻寇南翔,寇乃於三江口、四江口立左右大寨,設浮橋潛渡,困我水師;而青浦西北洋新涇、趙屯橋、白鶴江寇益蔓延,擾及重固鎮張堰,距青浦十餘里。黃翼升率水師自青浦出沖敵舟,寇扼白鶴江不得進,別隊犯黃渡,李鶴章會擊敗之。時四江口久被圍,紹光屯吳淞江口,炳文踞南岸。鴻章督諸軍至黃渡,分三路進擊,自辰至未,屢沖不動。鴻章督戰益急,諸軍逾壕直逼寇營,學啟砲傷胸,復裹創疾戰,寇由南岸潰而北。四江營守將皆沖圍而出,寇退昆山。我軍毀其浮橋石卡殆盡,斃數萬,夷壘二百座。寇自是不復窺松、水扈,悉力堅守昆山、太倉,尤為蘇州門戶,寇所必爭者也。

寧國再陷,寇復由句容進薄鎮江,壁湯岡。馮子材督軍破湯岡九壘,寇歸青山老巢,乘勝拔之,竄還句容。初,偽護王陳坤書糾眾四五萬圖犯金柱關,彭玉麟御之花津,五戰皆捷。尋寇以戰艦數百從東壩拖出,我軍毀其浮橋,寇乃不敢渡河。至是悍寇結筏偷渡,屢逼金柱關,我軍水陸大舉,敗之花山。寇遁上駟坡,而水師已先毀浮橋,寇回戈轉鬥,諸軍合擊,殲萬餘。其窯頭等處尚延袤百餘里,我軍環攻,焚其壘。花津、清山、象山、採石磯諸寇巢悉數平毀。自是蕪湖、金柱關六十里之間寇蹤以絕。

時大營軍士患疫方稍止,秀成親率十三偽王,號稱六十萬,麕集金陵,東自方山,西至板橋鎮,旗幟林立,直逼我軍營壘,尤趨重於東西兩隅。曾貞幹等擊敗之小河邊城寇援,寇尋由東西兩路進攻,分黨趨洲上,抄出猛字等營後,我軍分路擊退之。寇之圍逼西路者,歷六晝夜,為我軍擊敗。寇悉向東路,逼營而陣,潛通地道,百計環攻。各軍將士負墻露立,擲火球擊寇。寇負板蛇行而進,填壕欲上。我軍叢矛擊刺,寇拽尸復進,抵死不退。飛彈傷國荃頰,血流交頤,仍裹創上壕守御。侍賢自浙東來援,急攻吉後營砲臺。國荃引軍馳救,寇來益眾,用箱匱實土排砌壕間,暗鑿地道。我軍以火箭攢射,隨出銳卒擊之,賊鋒稍挫,遂毀西路寇壘。東路之寇環逼不已,嘉字、吉後兩營地道轟發,寇擁入塌口,我軍分路沖出決戰,塌口以內之寇誅戮無遺。壕外寇復舉旗督戰,各營同出抄殺,寇精銳悉挫折。復於東路別開地道,西路決江水淹絕運糧之路。貞幹在高坡增築小營,令水師駐雙徬護餉道。我軍凡破寇地道五處,寇計益窮。國荃乘勢進拔十餘卡,破東路四壘,西南諸壘望風驚潰,追至南路牛首山一帶,平壘數十座,搜剿至方山之西。雨花臺守眾句結城寇絕我軍歸路,我軍左右蕩決,寇分路而遁,重圍始解。是役也,秀成自蘇,侍賢自浙,先後圍攻大營四十六晝夜。國荃率諸將居圍中,設奇破之,弟貞幹力顧餉道,將士獰目髹面,皮肉幾盡。

大營解圍後,秀成仍屯秣陵關、六郎橋一帶。侍賢謂秀成曰:「今江北方空虛,出其不料,馳攻揚州、六合,括其糧以濟軍;復分兵攻國籓於安慶,彼必分軍馳救。我今屯秣陵、溧水之師,乘虛擊之,鮮不濟矣。」秀成納之,別遣偽納王郜永寬、偽對王洪元春等自九洑洲渡江,竄越江浦、浦口者五六萬。洪元春陷巢縣、含山、和州,遂踞運漕鎮、銅城徬、東關各隘,知無為州米足兵單,徑撲州城。提督蕭慶衍攻運漕、銅城,會彭玉麟水師焚毀三石卡,進破運漕鎮,連覆陶家嘴、昆山岡寇壘,繞出銅城徬後。徬口寇沖圍而出,岡東、徬西寇皆遁走。國荃遣軍守東西梁山顧江隘,令李昭慶帶五營自蕪湖北渡援無為,以保皖南各軍運道。國籓調李續宜、毛有銘移防廬州。賴文鴻、古隆賢等自廣德、寧國竄入旌德,總兵硃品隆敗之,解圍去。其攻涇縣之寇,亦被援軍擊退。先是昌化寇率眾數萬竄入績溪,冀絕旌德防軍糧路。唐義訓會浙軍克之。

十月,我軍連復上虞、嵊縣、新昌。宗棠屢攻龍游、湯谿、蘭谿、嚴州諸處,破壘卡三十餘,惟附城諸壘不可破。寇以死守城,穴墻開砲,軍士多傷亡。龍游、湯谿兩城為金華要道,必兩城下,後路清,而後可攻金華。蘭谿一水直達嚴州,必蘭谿下,餉道通,而後可收嚴郡。此三城者,所謂如骨之梗在喉也。十一月,寇竄太平、黟縣,進陷祁門,將窺伺江西饒州、景德。宗棠恐阻糧路,檄軍助剿,未至而祁門已克。魏喻義攻嚴州,嚴州形勢,外通懷寧,內達杭州。宗棠援浙,謀首下嚴州,而寇在三衢,圖犯江西,斷我餉道。乃先清衢郡,飭劉典攻蘭谿以分寇勢,踞寇偽朝將譚富與蘭谿譚星為兄弟,互相首尾。喻義屯銅關,據險設卡。譚富糾桐廬、浦江諸寇屢來犯,喻義伏兵鐘嶺腳,殲寇前鋒,寇驚還,閉城固守。是夜我軍梯城而入,殲寇萬餘,立克府城,焚船三百餘艘,獲偽印二百九十三顆,餘棄械投誠。

是時紹興偽首王範汝增、偽戴王黃呈忠、偽梯王練業紳率大股由諸暨、東陽、義烏、永康西竄金華,號稱十萬,以援湯谿、龍游。分黨竄武義,林文察敗之。丹陽、句容寇竄鎮江,馮子材督軍大破於丁村、薛村諸處。寇踞常熟、昭文二縣以窺江北。距城十八里有福山者,為江南重鎮,與江北狼山鎮對峙,由江入海之鎖鑰也。二縣守寇錢桂仁、駱國忠、董正勤與太倉酋錢壽仁密通款我軍,李鶴年攻城,寇約內應。國忠夜飲桂仁酒,就座斫殺偽馮天安錢嘉仁、偽逮天福姚得時,以城降。明日,會水師周興隆破平福山滸,白、徐六涇諸海口寇壘,進規太倉,而蘇州內應事洩。譚永光悉眾爭常熟,招江陰、無錫寇六七萬來會,又令楊舍寇乘隙陷福山。官軍固守常、昭。

十二月,寇圍而攻之,團勇潰。周興隆告急,鴻章遣援,而江陰楊厙寇已竄福山各口阻援師;遣常勝軍及水師攻福山口河西寇壘不下。寇攻常熟急,西北門營壘已失,常勝軍阻寇福山不得達。李鶴年攻太倉,寇援甚眾,亦難驟進。鴻章增調浦東軍由海道繞赴福山,會師援剿。鶴年等自望仙橋進攻太倉,敗之。時蔣益澧攻湯谿不下,劉璈會攻蘭谿寇壘亦失利,劉典合水陸進攻,寇堅伏不出。龍游、湯谿援寇適大至,乘間西趨擾江、皖,益澧等迎擊湯谿援寇於金華白龍橋,大創之。偽扶王陳得才、偽端王藍成春、偽增王賴文光、偽顧王梁成富、偽主將馬融初,皆陳玉成悍黨也。玉成命北犯牽掣官軍,而得才見廬州被圍急,欲南援,為防軍所扼,奔竄於河南、湖北、山、陜之間,與諸捻合,遂成流寇。

二年正月,秀成調集常州、丹陽諸寇屯江寧下關、中關,號二十萬,自九洑洲陸續渡江,意欲假道皖北,竄擾鄂疆,截斷江、皖各軍運道,圖解江寧之困,蓋近攻不如取遠勢也。既渡江,陷浦口,李世忠退入江浦。偽匡王賴文鴻、偽奉王古隆賢、偽襄王劉官方糾合花旗廣寇數萬圍涇縣。鮑超自寧國馳援,誘之入伏,寇敗還壘,而壘已為我軍所焚,寇大奔,立解城圍。及還軍,而西河寇乘虛犯壘,見超幟,倉皇遁去。蔣益澧、康國器克湯谿,金華寇恃湯谿、龍游、蘭谿三城為犄角,我軍攻湯谿,寇勢漸蹙。偽朝將彭禹蘭詣營乞降,益澧令內應,誘誅偽天將李尚揚等八名。是夜彭禹蘭啟西門納軍,殺九千餘,城寇遂盡。偽戴王黃呈忠、偽首王範汝增、偽梯王練業紳自白龍橋退奔金華,龍游寇聞風而遁,左宗棠收之,劉典收蘭谿,高連升等收金華,而武義、永康、東陽、義烏、浦江皆相繼收復。

浙東敗寇從於潛、昌化越叢山關,竄皖南績溪,復逾箬嶺,歸旌德,並句容、太平大股麕集石埭,謀西上。建德大震,江西饒州、九江邊亦急。劉典等克諸暨,譚星自浦江敗後,竄踞桐廬,宗棠飭劉培元會水軍合擊之,遏其西趨。蔣益澧乘勢進攻紹興,提督葉炳忠會英、法軍克之。敗寇萬餘,與桐廬踞寇沿江築壘抗拒,我軍水陸合攻,遂復桐廬。蕭山寇亦竄走,浙東肅清。

左宗棠遣益澧進攻杭州。先是楊輔清糾合群黨,嘯聚西河、紅楊樹、麒麟山一帶,十餘萬人,以一大股抄出高祖山,先以一小隊繞過山背,揚言上犯涇縣,實欲圖鮑超老營。二月,寇分三四萬眾圍高祖山八營,鮑超分軍三路,伏兵茯苓山傍以斷其後。戰逾時,寇譁亂奔,近茯苓山,伏起,寇駭懼,遂平高嶺、周家橋、馬家園、小淮窯諸壘。寇之由河西遁入灣沚者,為水師擊敗,並入梅嶺、麒麟山。鮑超遣將分攻之,積尸若阜,並收復仰賢圩各處,餘分道竄逸。偽懷王周逆等糾眾竄至句容城外,會合丹陽偽效天義陳酋,圖由九洑洲北渡。馮子材扼橋據險,分隊進攻,直趨牧馬口,沿村十餘里敵卡林立,官軍直突,屹不稍動。守備李耀光陣斬執旗寇,搗中堅,官軍無不以一當百,立毀牧馬口敵卡。東湖寇亦敗潰,進毀南路柏林村老巢,斬陳酋馬下,即四眼狗玉成之叔也,寇駭奔,向西南竄走。

皖西寇犯休寧,分掠建德,西侵江西鄱陽、彭澤,東擾池州,圍青陽,續由江浦縣新河口迤邐西竄巢、含、全椒之間。南岸則金柱關,時踞皖南寇約有三起:一為胡、黃、古、賴諸寇,即踞寧國、太平、石埭、旌德者也;一為花旗,此廣東匪,前由廣東、湖南、江西入浙、皖者也;一為譚星,即蘭谿抗官軍者也。徽防諸軍紛紛告警。當曾國籓之視師東下也,寇攻常熟益急,譚紹光又益以砲船二百艘,突地攻城,降將駱國忠悉力扼守。鴻章遣軍攻太倉、昆山分寇勢,別遣英將戈登助剿福山,會潘鼎新等水陸軍奪石城,夜毀城壘,翌日寇入西山,而福山火起,乃開門悉銳出擊,寇盡潰,擒斬悍酋孝天義硃衣點。常熟、昭文城圍立解。太平踞寇圖祁門。江西軍王沐敗寇於徽州屯谿。草市寇再敗於嚴寺街、長林、潛口等處,死近萬人,退奔休寧、藍田一帶,西通漁亭。未幾,寇復進踞潛口,祁門防軍御之黟縣漁亭,大破之,陣斬偽天將劉官福。

寇之初起也,禁令嚴明,聽民耕種,故取江南數郡之糧出金柱關,江北數郡之糧出裕谿口,並輸江寧。今耕者廢業,煙火斷絕,寇行無人之境,而安慶、蕪湖、廬州、寧國、東西梁山、金柱關、裕谿口,暨浙之金華、紹興,山川筋絡必爭之地,寇悉喪失,我軍足制其死命。昔年寇之所至,築壘如城,掘濠如川,近乃日近草率,群酋受封至九十餘王之多,各爭雄長,敗不相救,識者知其亡無日矣。

寇復由寧國繞出青陽,分擾建德、東流。三月,由東流犯江西彭澤,進逼祁門;由建德窺饒州,犯梅林營壘。劉典督軍敗之,遂大破潛口寇屯。徽州、休寧解嚴。乃赴漁亭,會克黟縣,斬偽絇天義古文佑。追寇出嶺外,平寇壘二十餘,嶺內一律肅清。太倉踞寇偽會王蔡元隆詐降郊迎,我軍至城下,伏起,槍傷李鶴章,程學啟殿軍而退。鴻章檄戈登會攻太倉,克之。黃文金合許家山各處寇十餘萬,由祁門進逼,與參將韓進春血戰四時,陣斬偽孝王胡鼎文,群寇奪氣。

寇攻廬州,犯舒城。李秀成將由舒城、六安上竄,一出黃州,一出漢口,擾犯湖北,掣我南岸之師以援北岸,掣我下游之師以援上游,皆為解金陵圍計也。湖北為數省樞紐,曾國籓調成大吉回屯灄口,檄水師赴武漢嚴防。秀成來犯石澗埠,進逼我軍,晝夜猛攻,相持不下。寇復於前營增百壘,層層合圍。彭玉麟派隊來援,會軍夾擊,盡平群壘,秀成遁走。其犯廬江、舒城者悉敗走。悍寇馬融和自豫間道犯桐城,我軍敗之三里街,遁往孔城,與秀成合而為一。秀成遣偽富天豫張承得等圍六安,敗死,六安者淮南要沖也。餘寇走廬州,鮑超追擊,會攻巢縣,先破東關、銅城徬二隘,遂克其城。金山、和州相繼皆下。

李侍賢自金陵敗遁,糾悍黨數萬屢犯金柱關。花津、上駟渡、萬頃湖、塗家渡及燕子磯、伏龍橋、護駕墩、灣沚、黃池諸處寇壘,皆為我軍所覆。自是寇不敢輕渡西岸,遁溧水、丹陽一帶。四月,水師楊政謨襲破杭州閘口寇船,登岸進毀望江門寇壘,寇大震,急招新城寇還救。蔣益澧進攻富陽,富陽一城為杭州上游關鍵,賊嚴防禦,船壘相輔。杭州援寇屯新橋,與城寇相為犄角。秀成令陳炳文等舍蘇州、常、昭,急援富陽,並糾蘇、常、嘉興悍黨由餘杭趨臨安,竄新城,擾富陽軍後路。魏喻義等督兵進擊,寇復乘霧分道攻撲新城,大敗,向臨安遁走。偽慕王譚紹光、偽來王陸順德等率大股犯太倉雙鳳鎮,攻昆山後路,圖解城圍。我軍鏖戰三晝夜,破之。偽天將夏天義率悍黨數萬久踞昆山、新陽縣城,鴻章督率程學啟、戈登會水師大破昆山寇壘二十四,斃萬餘。有正義鎮者,為蘇城援昆山必由之路,學啟攻之,破石壘二。寇見歸路已斷,奪路狂奔,遂進克同城昆山新陽。王沐自黟縣回援景德,寇敗於陳家畈、包家踠,竄安寧嶺外。

時江寧攻圍久,百計欲解城圍,既分股由徽、寧窺伺江西,由含、和一帶圖犯湖北,而由湖北下竄之捻,自蘄水分為四枝,一回竄黃州,一撲宿松,越潛、太,以撲廬、桐。寇、捻句合,兇焰甚張。此皆李秀成所規畫也。我軍克復福山後,江陰縣屬揚厙汛為江邊險要,寇糾眾死守,以蔽江陰。我軍水陸會攻,斬趙尚林等,立復汛城,而漍北、漍西、塘市屯寇均棄壘遁還無錫、常州。我軍克復建德,連復巢、含、和三城。於是皖北寇全遁,皖南寇勢亦衰。

初,秀成自六安敗後,率眾東竄,聲言回救蘇州。國荃急爭江寧老巢,攻其必救,使城下之寇不暇遠趨蘇郡,而北岸之寇亦不敢專注揚州;乃率軍分六路並進,潛襲雨花臺及聚寶南門石壘,肉薄登城,遂奪雨花臺,乘勝猛攻東、西、南各卡九壘,皆克之。群寇潰奔,我軍追擊於長幹橋,蹙入水者無數。未幾,城寇出,又敗退,斃六千餘,寇勢從此衰減。秀成在江北,聞雨花臺失,益惶懼,又以昆山新克,蘇州亦受逼,乃與諸偽王改圖南渡。於是天長、六合、來安次第解圍。而寇之分踞喬林小店者,冒雨掠舟,喧闐不絕。五月,浦口寇棄城遁走,而江浦寇忽獻書乞降,鮑超等察其詐,引軍急進,水師次江浦。寇聞風亦宵遁,九洑洲偽城踞寇閉門不納,寇駭竄蘆葦中,溺死者無算。江浦、浦口兩城寇,盡蹙之入江,江北肅清。

曾國荃連日破平下關、草鞋峽、燕子磯,收寶金圩,距蕪湖、金柱關百里內已無寇蹤。進攻克九洑洲,寇之在中關者,附城為壘,卒不稍動。其堅踞九洑洲者,下有列船,上有偽城,群砲轟發不息;復於東、西、南三面分伏洋槍隊,伺間出擊,我軍多損傷。彭楚漢等負創角戰,乘風縱火,夜二鼓,撲墻而入,聚殲無一脫者。九洑洲既克,謀者謂浙軍攻富陽,滬軍攻蘇州,江寧亦宜速合圍,使備多力分。國籓亦主合圍制敵為上策。秀成南渡後,連營於江陰、無錫數十里,聲言援江陰攻常熟。鴻章督諸軍攻破七十五壘,顧山以西寇皆盡。

寇自失九洑洲,下關江上接濟已斷,糧食漸乏,諜赴蘇州、嘉興,力圖接濟。秀全以城圍日逼,留秀成共守老巢,緩援蘇州。六月,秀全遣黨出儀鳳門犯鮑超營,出太平門犯劉連捷營,不克而退。七月,犯下關,亦為我軍擊卻。八月,國荃攻印子山,破其石壘,陣斬偽佩王馮真林。明日,破七橋甕石壘一、土壘三,偽梯王練茶發伏誅。國荃調江浦、浦口防軍,別募萬人,為火舉圍城之計。是日程學啟會水師逼婁、葑,規取蘇州。

初,建德南竄寇敗於汪村,偽匡王賴文鴻創而墮馬,群寇衛之遁。越二日,復敗於分流木塔曹家渡,自是浮梁北路稍靖。先是黃文金糾合諸酋由皖入江,分擾鄱陽、浮梁、祁門、都昌境內,每為我軍所扼,不得深入;乃折而西趨湖口,分三路:上路由文橋,中路由梧桐嶺,下路由太平關,而文橋寇勢最盛,文金親踞其中。尋自文橋撲犯堅山大營,江忠義會諸軍直前迎擊,破其七壘,文金竄皖南,江西肅清。

文金繞越池州圍青陽。八月,富陽寇與新橋寇互相犄角,抗我圍師。蔣益澧督諸軍日夜轟擊,先破寇援,毀倚城大壘及大小諸卡,城寇不支,逃入新橋,城立復。我軍復由雞籠山繞出新橋,並力追殺,寇壘悉數芟夷。江陰踞寇日久負嵎,我軍攻之,勢漸蹙。是月陳坤書及潮武齊區五大股眾十餘萬分道來援,亙數十里,西自江邊,東至山口,沿途扎木城十餘,其中營壘大小百餘,守御堅固。我軍水陸分攻,郭松林潛自山後噪而入,縱橫沖突。銘傳直搗中堅,寇大潰。有內應者,夜三鼓,梯城而入。偽廣王李愷順墜水死,遂克其城。

秀成自江寧返蘇,謀解城圍,與程學啟、戈登戰蘇州寶帶橋,敗北,奔至盤門。我軍毀沿途諸卡,秀成率大股來爭,我軍力擊敗之。初,寇於婁門外附城築十九壘,學啟屯外跨塘,效力不能及寇壘,乃移壁永安橋,城中出夷人百餘,發炸砲助之。未幾,寇分門大出,水陸軍力禦,寇敗退。學啟以寇壘既多且固,不得前;而城東南寶帶橋為太湖鎖鑰,寇立石營一、土營三,悉力拒守,遂謀先破之以挫其勢。乃分水陸軍為三路,先破土營,寇棄壘走,石營亦旋潰。秀成親率援師抵御,學啟督軍卻之。我軍攻無錫賊,敗之芙蓉山。偽潮王黃子隆出拒,再敗走,刃及其肩,幾成擒;郭松林追及城下,破平西北兩城壘,燒寇船百餘艘。

蔣益澧既克富陽,移師杭州,康國器趨餘杭。偽歸王鄧光明、偽聽王陳炳文、偽享王劉酋及偽朝將汪海洋,於附城要隘築壘樹棚,自前倉橋、女兒橋、老人畈、東塘、西谿埠、觀音橋、三墩,直至武林門、北新關,橫至古蕩,連營四十餘里,以拒我軍。海洋自杭州上援餘杭,為我軍擊敗。寇旋由前倉渡河,結壘西葛村,我軍再擊走之。我軍攻杭州江幹十里街,破街口寇壘。我水陸軍大舉攻青陽。初,黃文金在都昌、湖口等處戰敗而東,遂略地池州,直薄青陽城外,近城半里,環築六十六壘;又數里,築七十餘壘。曾國籓調水陸軍進攻,江忠義督所部渡河,從山後緣巖而上,驟攻寇壘。寇糾眾抄我軍後路,忠義揮眾蕩決,寇敗若潮湧,平一百三十餘壘,殄萬餘,寇遁歸石埭一帶,城圍立解。

李侍賢、林紹璋等合股內犯,由無錫南門至坊前梅村三十里;高橋大股亦分眾七八千人擾至西高山,出芙蓉山後;城寇出北門犯塘頭東亭。官軍分路迎剿,設伏誘擊,寇大亂,敗走,陸寇殲誅殆盡,奪寇船六十餘、民船五百餘。秀成自蘇州率偽納王郜雲官、偽來王陸順德、偽趨王黃章桂、偽祥王黃隆蕓、偽紀王黃金愛來援,進逼大橋角營。夷酋白齊文以輪船大砲為寇前驅,李鶴章以連珠噴筒破之。寇水陸皆敗,斃萬餘。別股犯緱山,亦敗走。秀成子及宿祥玉、黃隆蕓皆溺死,秀成頓足大哭。程學啟再敗寇於齊、婁、葑三門,追至護城河邊始斂軍。

九月,杭州城寇大出,由蠻頭山、鳳凰山、九耀山、雷峰塔犯我軍新壘,蔣益澧督諸軍迎擊,大破之。我軍進壁天馬山、南屏山、翁家山。時杭寇兇狡者,以鄧光明、汪海洋為最,陳炳文次之。其計以杭州為老巢,以餘杭為犄角,均賴嘉興、湖州之援,便資其接濟。嘉興入杭之路,則在餘杭,故我軍議先克餘杭,扼截嘉、湖之路,以並合圍。蔣益澧一軍逼扎鳳山、清波各門,扼其西面;餘杭一城已圍其東南,而北路無重兵,兩城之寇往來如故。

寇自大橋角戰敗,勢漸蹙。秀成糾無錫、溧陽、宜興賊八九萬、船千餘艘,泊運河口;而自率悍黨踞金匱縣後宅,連營互進。李鶴章謂寇以河為固,不宜浪戰,宜結營制之。我軍疊敗坊前、梅村、安鎮、鴻山之寇,而寇之大股全集西路,志在保無錫以援蘇州。郎中潘增瑋進攻蠡口、黃埭之策,程學啟乃與戈登攻破蠡口,進擊黃埭,毀其四壘,擒斬偽天將萬國鎮。五龍橋者在寶帶橋西五里,由澹臺湖占魚口達太湖以通浙之要隘也。學啟率戈登會水師先後破寇六營。於是我軍扎永安橋而婁門路斷,扎寶帶橋而葑門路斷,克五龍橋而盤門、太湖之路又斷。寇乃句結浙黨,圖撲吳江,以擾我後。學啟率水陸軍擊破嘉、湖援寇,擒斬偽貴王陳得勝及悍黨四十餘,追至平望,斷其橋。從此攻蘇之軍無牽掣之患。

偽平東王何明亮等以劉典屯績溪,不克上犯徽、歙,遂由寧國千秋關竄浙江,陷昌化,擾於潛。其前竄廣德者,復折踞孝豐。劉典遣軍出績溪昱嶺關援剿。江寧軍自攻克江東橋、上方橋,而城東數隘未下。近城者曰中和橋,曰雙橋,曰七橋甕,稍遠者曰方山、土山,曰上方門、高橋門,迤南則為秣陵,以至博望鎮,皆金陵外輔也。國荃以東路未平,不能制寇死命,令諸軍東渡。提督蕭衍慶過河破五壘,城寇出爭,擊退之,遂克上方門、高橋門、雙橋門。右路方山、土山之寇亦棄壘而奔。七橋甕踞寇倉皇欲遁,而城中忽出大股來援,兩軍相搏,總兵蕭孚泗乘夜縱火,寇冒火突出,遂克七橋甕。其博望鎮,總兵硃南桂已先五日襲取之。博望鎮既失,則秣陵關之勢孤;七橋甕既失,則中和橋之勢孤。總兵伍維壽等南略秣陵關,寇棄壘奔潰。自是鍾山西南無一寇巢。

偽奉王古隆賢詣硃品隆降,收復石埭、太平、景德三城,徽州肅清。餘寇竄踞寧國、廣德、孝豐之間,勢甚渙散。宗棠飭劉典由昌化、於潛趨臨安,進剿孝豐,為規取湖州之地。十月,易開俊克寧國縣城。自是東壩黎立新上書請為內應,建平張勝祿上書請獻城池。鮑超等遂合趨東壩,繞壘環攻。楊輔清從亂軍中逸出,寇立獻偽城。東壩既克,建平張勝祿等即於是日斬偽跟王藍仁得,舉城降,而溧水寇楊英清亦繳械降,遂收二城。國荃克淳化鎮、解谿、龍都、湖墅、三岔鎮等隘,毀寇壘二十餘。江寧城東南百里內寇巢略盡。

蘇州軍自黃埭攻滸墅關,破王瓜涇、觀音廟寇壘,直抵滸墅,擊走偽來王陸順德,進毀十里亭。虎丘、楓橋寇皆遁。躡至閶門,寇大恐。李鶴章敗寇無錫鴨城橋,破西倉寇壘,直抵茅塘橋。李侍賢調常州陳坤書來援,城寇黃子澄出迎,我軍縱擊敗之。秀成聞滸關已失,退屯北望亭,謀返蘇州老巢。坤書還走常州,侍賢遁宜興、溧陽,我軍乘之,下寇壘百餘。鴻章督軍攻婁門寇。蘇州四年自我軍連克要隘,乃於胥、葑、婁等門憑河築壘數十,婁門外石壘尤堅。至是我軍由南北岸而進,秀成等突出婁門拒戰,程學啟與常勝軍分隊以應,援寇遁入城。水師會攻婁葑門外寇壘二十餘皆下。我軍連克齊盤門各壘,三面薄城,寇眾恟懼,而秀成及譚紹光猶圖固守,他酋郜雲官等皆有貳心,密請於鴻章乞反正,許之。學啟、戈登單舸見雲官等,命斬秀成、紹光以獻;而雲官不忍殺秀成,許圖紹光。秀成覺之,涕泗握紹光手為別,乘夜率萬人自胥門出走嘉善。郜雲官殺譚紹光,率偽比王伍貴文,偽康王汪安均,偽寧王周文佳,偽天將範起發、張大洲、汪朅武、汪有為,開齊化門迎降,鴻章受之。雲官等未薙發,要總兵、副將等官,並請自領其眾屯守盤、齊、胥、閶四門。程學啟慮其不可制,密請於鴻章誅之,立復蘇省。

秀成以輪船炸砲越無錫水師北竄。十一月,李鶴章克同城無錫、金匱,追擒黃子隆與子德懋並誅之。劉秉璋等攻浙西,平湖寇陳殿選獻城乞降。連日乍浦賊熊建勛、海鹽䍦浦賊皆反正。十二月,秀成留蘇州敗黨分布丹陽、句容間,自率數百騎潛入江寧太平門,苦勸秀全棄城同走。秀全侈然高座曰:「我奉天父、天兄命,令為天下萬國獨立真主,天兵眾多,何懼之有?」秀成又曰:「糧道已絕,饑死可立待也!」秀全曰:「食天生甜露,自能救饑。」甜露,雜草也。秀成以秀全戀老巢不肯去,非口舌所能爭,乃貽書溧陽約李侍賢,銳意走江西。

初,寇自咸豐十年破江寧長圍,迭陷蘇、常、嘉、湖,上竄江西、湖北,手虜脅潰兵、游匪以百萬計,盡得東南財賦之區,日以強大。自去歲屢戰屢敗,各城精銳散亡不下十萬。今年春夏間竄皖北,我軍截殺解散又十數萬。其自九洑洲過江,僅存四五萬人。秀全驚惶失措,賴秀成回江寧主持守局;而秀成以蘇州為分地,事急回援。今巢穴已失,黨羽又孤,踉蹌而走,隨行僅兩萬餘人,欲赴金陵,解圍無術。力勸秀全突圍上竄回粵,以圖再舉。常州陳坤書、溧陽李侍賢皆聽秀成為進止,而杭州陳炳文系安徽人,鄧光明湖南人,聞秀成有回粵之謀,皆不原從,秀全亦屢勸不聽。

國荃自四月間掘地道,至是始成,而寇附城築墻號「月圍」,下穿橫洞以防隧道。故城崩,而猶阻月圍橫洞不能克。劉銘傳進攻常州西路,奔牛踞寇邵志倫,羅墅灣踞寇夏登山、萬錫階,石橋灣踞寇張邦振皆詣我軍乞降,收眾萬六千人。惟孟河賊尚踞汛城,輒出犯降人壘,銘傳一鼓下之。潘鼎新克嶼城,李鴻章等破常州東門、南門石壘。明日,張樹聲傍城東北築壘,破小門、土門。連日嘉興、桐鄉、石門寇犯嶼城,海寧寇犯澉浦、海鹽,偽章王林紹璋自句容援常州,均為我軍擊走。

程學啟克平望鎮。平望東連嘉興,西接湖州,南通杭郡,為蘇、浙樞紐,浙寇精銳多聚守此,今被我軍攻克,嘉興籓籬已失,踞寇奪氣。寇犯鎮江甘棠橋張文德壘,馮子材等助擊破之,斬李秀成養子偽岡天義黃酋。十二月,偽會王蔡元隆獻城降,左宗棠受之,改名元吉。海寧瀕海,為杭州東北屏蔽。元吉擁眾,致攻杭之師未能合圍;今幡然附款,杭州之勢益孤,東北兩面圍漸合。嘉興賊偽榮王廖發受呈降書,程學啟慮有狡謀,誡軍嚴備。杭州踞寇偽聽王陳炳文遣人詣降,而無降書,宗棠趣蔣益澧攻城益急。

秀成會李侍賢犯江西,既以溧陽至饒州浮梁數百里處處乏食,慮裹糧疾趨為難;因趣侍賢持二十日糧,道長興、廣德、寧國入江西,先踞腴區待己。於是侍賢遣黨西竄,行甚疾。曾國籓遣軍屯休寧,沈寶楨遣軍屯婺源、玉山拒之。四月,川督駱秉璋破賊天全,生擒偽翼王石達開,磔於市。自洪秀全倡亂,封五偽王:馮雲山、蕭朝貴皆敗死,楊秀清、韋昌輝自相賊殺,石達開避禍出奔,自樹一幟,歷犯浙江、福建、兩湖、兩廣諸省,並擾及滇、黔,蓄意入川,以圖竊據,至是為川軍擒戮。凡偽五王前後皆誅滅矣。

三年正月,蔣益澧飭降人蔡元吉襲桐鄉不下。桐鄉為杭、嘉要道,益澧遣將分屯東北門,偽朝將何培章來降,遂收桐鄉,令培章率降眾屯烏鎮、雙橋,阻杭、嘉道,絕寇糧。湖、杭寇皆來爭,擊走之。我軍進屯嘉興,聯絡蘇師,規復郡城。廣德、寧國寇竄犯浙江昌化,副將劉明珍不知寇眾寡,進擊,創矛,退扼河橋。寇尋分黨:一竄徽州績溪,一竄淳安,謀渡威坪河進窺遂安。時侍賢上犯,冀沖過徽州,就食江西。其大勢趨重遂安,擾及開化馬金街,其地與休寧、婺源、常山、玉山接壤。王開琳自徽州進遂安,連破寇於中州昏口。黃少春馳入縣城,敗寇遂安城北,蹙入郭村,殲之。寇後隊仍由章村竄昏口,少春擊破之新橋,開琳復繞出昏口敗之。其竄往開化華埠僅千餘。

侍賢句合黃文金及廣德餘黨由寧國上竄,陷績溪,退踞雄路、孔靈等處,圖撲徽州。唐義訓自徽州出扎吳山鋪,敗之,毀雄路寇館。援寇至,我軍奮擊,再破之,縣城立復。鴻章派郭松林、戈登等攻宜興、荊溪,寇開城出拒,槍子傷松林右肘。我軍屯三里橋,與常勝軍會擊,偽代王黃精忠由溧陽來援,拚死相撲,我軍屹立不動,以洋槍排擊之,寇死傷相繼,無退志。我軍三路包抄,寇始奪路狂奔,城寇勢益不支,開西門而逸,遂復兩縣城。是城瀕太湖西岸,當江、浙沖道,為常郡後路。自是常郡寇益蹙,蘇州、無錫之防益固。

戈登進圍溧陽。溧陽為李侍賢老巢,又江寧後路要隘。常州、金壇在其北,句容、丹陽值其西,長興、廣德當其南,面面寇巢,前與鮑超東壩、溧水之師相隔絕,後距李鴻章常州軍亦稍遠。孤軍深入,李鴻章戒戈登慎進止。尋溧陽酋吳人傑降,戈登復其城。郭松林等進攻金壇,敗偽列王古宗成、偽襄王劉官方於楊巷。是時常州陳坤書合丹陽、句容之寇十餘萬,由西路繞出常州之北,日犯我軍,李鶴章督軍擊敗之。寇以我軍城圍日緊,分犯江陰、常熟、無錫,以圖分我兵勢。李鶴章撤圍師,堅守勿戰,別調軍援三縣。李鶴章等尋解無錫城圍。江陰守將駱秉忠與楊鼎勛等內外夾攻,寇亦敗走;乃並趨常熟,北自楊舍、福山,南自顏山、王莊,數十里皆寇。黃翼升會同城軍夾擊,克王莊、顏山、陳市寇壘,追殺二十餘里。寇由大河回撲常熟,水師截之,首尾不能相顧,掩殺無算。常熟之圍立解。

初,秀成入江寧說秀全出走,不聽,秀成憂糧食不繼,遣黨百計偷運。國荃約楊岳斌水陸巡邏,遇奸民運米入城,輒奪之。秀成遣養子李士貴率黨數千出太平門赴句容接糧入城,伏兵要之,寇棄糧走。國荃銳意合圍,江寧城延亙百餘里,自我軍駐師雨花臺,奪取附近諸隘,東、西、南三面為官軍所據,惟鍾山石壘未克,城北兩門尚未合圍。秀成自將出鍾山南,攻硃洪章營,敗退登山。沈鴻賓等挾火球箭擲壘中,寇突火跳,遂克鍾山石壘,寇所署偽天保城者也。國荃分檄諸將屯太平門、洪山、北固山,塞神策門,餘玄武湖阻水為圍。於是江寧四面成包舉之勢,寇援及糧路皆絕。

二月,寇以運糧路絕,日驅婦孺出城以謀節食。城西北多園圃,豫種麥濟饑。初,程學啟進攻嘉興,破小西門、北門寇營,殲除凈盡,擒偽天將劉得福、慕天義賈慕仁。而湖州寇屢竄南木、壇丘、四亭子、新塍,思犯盛澤、平望,以圖嘉興,不得逞;又犯盛澤,圍王江涇後路營,亦為我軍所敗,而城守甚固,我軍多傷亡;又遇雨不能進攻,學啟急思復嘉興,分門攻擊,增築月墻,寇拚死抗拒;又以地雷巨砲轟蹋城垣百餘丈,擊毀砲臺,賊爭負土填城缺。湖州賊又自新塍來援,學啟會軍猛攻,肉薄登城,丸創其首,部將大憤,縱橫刺射,寇眾潰,遂克嘉興。偽榮王廖發受、偽挺王劉得功皆伏誅。援寇黃文金還湖州。

寇之分竄江西者,疊經我軍截殺,闌入金溪。道員席寶田由安仁馳擊,復其縣城。寇由瀘溪趨建昌,寶田會軍敗之,寇蔓延於新城、南豐。提督黃仁翼進攻新城,克之,餘竄入福建建寧縣境。南豐之寇亦經寶田擊敗,斬偽天侯張在朋,寇棄壘狂奔,退至城下,合圍攻之。

蔣益澧克杭州,康國器等同日收餘杭,寇分竄德清、武康等處。宗棠飭軍分路進取。三月,羅大春等各率所部撲壘環攻,降人楊蕓桂開門迎納官軍。援賊回斗,砲擊之,敗走。德清寇鏖戰四時亦大敗。武康、德清皆復。我軍進逼石門,踞寇鄧光明降。其圖竄孝義之寇,亦截殺無算。鮑超等會克句容,偽漢王項大英、偽列王方成宗皆伏誅。偽守王方海宗遁金壇寶堰,寶堰南距金壇城四十五里,北達丹陽。方海宗與偽顯王袁得厚合謀阻進兵之路,鮑超攻之,閉壘不出;乃負草填壕,一躍而入,寇向金壇、丹陽遁去。

鮑超進攻金壇,設伏茅山,大敗追寇,城寇喪膽,啟南門遁走,遂復其城。敗寇二三千,屯踞南渡,偽植王林得英約會常州西路孟河、呂城諸寇,欲由金壇歸並廣德,同踞南渡。鴻章檄道員吳毓芬等會水陸軍分三路夜襲其巢,陣斬林得英及秫天安黃有才等,殄寇殆盡。其攻丹陽者,為鎮江、揚州防軍,援寇一由常州運河,一由江陰孟河大至,詹啟綸、張文德會擊敗之,援寇退屯丹陽東北一帶。丹陽一城多聚巨酋,偽然王陳時永為陳玉成叔父,偽來王賴桂芳為洪秀全妻弟。因其內閧,我軍乘之,陳時永創僕,斬之。其黨自縛賴桂芳及偽廣王李愷瞬、偽列王金友順、偽梁王凌郭鈞、偽鄒王周林保,並偽義安福、燕、豫諸目,獻軍前乞降,皆駢誅,遂克其城。至是常州、鎮江各屬俱告肅清。

寇自德清、武康、石門克後,李侍賢及偽聽王陳炳文、偽康王汪海洋等仍堅踞湖州。是時浙江惟湖州、長興、安吉三城未下,湖州寇於附城二三十里修築堅壘,復於長興、安吉各隘連營數十里,相為犄角。高連升擊之於湖州境,寇分竄昌化、分水,防軍劉明珍截斬數百。李侍賢繞越老竺嶺,竄皖南績溪,復間道走歙西,竄屯溪;陳炳文、汪海洋由歙北竄浙境,分犯淳化、遂安。遂安防軍截擊歙南小川,竄寇敗還楊村。唐義訓等進剿失利,於是嶺內遍地皆寇,徽、休、祁、黟岌岌難保。寇前隊由龍灣、婺源竄江西,後之續至者絡繹不絕。國籓調石埭、青陽防軍入嶺援徽,檄鮑超率全軍援江西。

陳坤書之踞常州四年矣,自蘇軍力攻,以炸砲毀城,寇死守,取舊棺敗船堵城缺,以槍砲拒我軍。時城西寇壘二十里夾河環列,劉銘傳等攻破十四壘,餘壘皆不戰而潰,而河干寇壘二十餘又為張樹聲等所破,於是寇西道皆絕,惟小南門、西門附城十餘壘,我軍復擊平之。陳坤書塞門不納敗黨,恐官軍奪入,悍賊皆死城下。城圍既合,築長墻,伏奇兵,備大舉,水陸軍發砲轟城,風煙迷漫,寇如墜霧中。俄,城壞數十丈,寇以人塞缺口,炸丸迸裂,人與磚石齊起天際,然旋散旋集,蓋蘇省各路敗寇,積年麕聚於此,猶圖萬死一生計也。鴻章益揮軍迫登,我軍偕藤牌噴筒直前,寇傾火藥,以長矛格刺,軍士十墜六七。龔生陽突入,擒陳坤書,周盛波擒偽列王費天將,戰城上良久,寇大潰,縋城出者復為我軍所殲,我軍亦死亡千數百人。常州之失以咸豐十年四月初六日,越四年而復,日月皆不爽,亦一奇也。陳坤書凌遲處死,梟示東門。

時常州敗寇竄徽州,我軍擊破之。餘竄江西,圍玉山,副將劉明珍等陣斬珊天安等,斃寇二千。李侍賢越金谿犯撫州河東灣,猛攻東門,為我軍擊退。忽突起攻橋,環呼城中內應,冀亂我軍,劉於淳砲斃多名,餘遁金溪,城圍立解。偽列王林彩新竄江西弋陽,我軍追抵湖西,揮軍抄擊,寇敗竄黃沙港。對岸楊家坡寇黨從上游渡寇千餘來援,諸軍沿河截擊,鏖戰逾時,寇始蹙,多落水死。

侍賢等先後由浙犯徽,由徽入江西。江陰楊舍及常州城外之寇,由丹陽、湖州上竄徽南,其酋為林彩新及偽麟王硃某、偽爵潘忠義等,從昌化進老竹嶺,闌入歙境。唐義訓伏兵鍾塘嶺後,以五營進,遇寇大戰,伏起,殲其前鋒。時金國琛等行抵富陽,隔河而屯,寇眾列山岡上下,我軍俟寇涉水將半,突出奮擊;唐義訓尾而至,夾擊之,寇不支,遁向黃山小路。我軍馳抵五弓橋,再敗寇,因循河埋伏兵。寇正渡時,伏突起,寇大驚亂,生擒林彩新等十名,陣斬潘忠義等十四酋,死亡者二萬餘人。

是月洪秀全以金陵危急,服毒死。群酋用上帝教殮法,繡緞裹尸,無棺槨,瘞偽宮內,秘不發喪。其子年十六,襲偽位。秀全生時即號其子為幼主,或曰本名天貴福。其刻印稱洪福,旁列「真王」二文,誤合為「瑱」,其稱洪福瑱以此。然諦觀印文,實「真主」二小字,非真王也。

時湖州寇方竄湖濱楊漊大錢口,潘鼎新分軍屯南潯,而寇恃長興為聲援。長興在湖州西,毗連宜興、溧陽及廣德州。宗棠貽書鴻章,移嘉興之師助攻長興。鴻章遣諸軍分道往取,水師入夾浦口。五月,鼎新進扎吳漊,水師破平夾浦石壘。鼎新連破吳漊、殷瀆村,毀其卡壘。郭松林毀長興城東上莘橋、跨塘橋寇壘。湖州、廣德、四安寇率數萬分路來援,依山築壘,綿亙林谷。松林等擊湖州援寇,劉士奇擊廣德、四安援寇,寇壘悉平毀,殲溺萬餘人。而湖北廣德寇復添撥大股折回,我軍乘賊眾喘息未定,合力痛剿,追殺二十餘里,寇乃遠遁。我軍乘夜攻城,以炸砲轟塌城垣十餘丈,松林等首先沖入,遂復長興縣城。

江西貴溪盛源洞等處寇分踞小巷一帶,築壘修卡,我軍飭砲船馳赴黃土墩,以槍砲擊之,寇倉皇敗竄,水陸諸軍撲卡而入,擒斬多名,寇大潰。貴溪寇壘一律肅清。浙軍進規湖州,攻復孝豐縣城,生擒偽感王陳榮,斃千餘。是月偽扶王陳得才等合捻竄擾孝感、雲夢等縣。

上以江寧垂克,而河南捻犯麻城、皖城,深入江西,恐掣全局,趣國荃迅取金陵。國荃進攻鍾山龍脖子,寇所稱地保城也。我軍自得偽天保城後,城寇防守益嚴。是城扼建要害,尋為李祥和所破。國荃築砲臺其上,日發巨砲轟擊,居高臨下,全城形勢皆在掌中。六月十六日,國荃飭諸軍發太平門地雷,塌城垣二十餘丈,前敵總兵李臣典、硃洪章等九人先登,諸將分門合力,攻克江寧省城,獲偽玉璽二方、金印一方。是夜,寇自焚偽天王府,秀成攜秀全幼子從城垣倒口遁去,並以己馬與之乘以行。國荃令閉門封缺口,搜殺三日,斃寇十餘萬,凡偽王以下大小酋目約三千餘。最後城西北隅清涼山伏寇數千出與官軍死戰,卒殲之。其偽天王府婦女多自縊,及溺城河而死。國荃派馬隊追至淳化鎮,生擒偽列王李萬才。其自城破後逸出者,洪秀全之兄偽巨王、偽幼西王、偽幼南王、偽定王、偽崇王、偽璋王悉為馬隊殺斃。蕭孚泗搜獲李秀成及洪仁發、洪仁達於江寧天印山,搜掘洪秀全尸於偽宮,戮而焚之。國籓親訊秀成等,供讞成,駢誅於市。

七月,鮑超連克東鄉、金谿,楊岳斌等連克崇仁、宜黃,潘鼎新會克湖州,楊昌濬等克安吉,斬偽駙馬列王徐朗。寇並竄廣德、長興,守軍吳毓芬克四安鎮,劉銘傳克廣德。初,秀全幼子自江寧出亡,悍黨衛至州城;至是城克,偽昭王黃文英等挾之走寧國。八月,唐義訓等敗湖州餘寇於歙縣,殲酋目偽幼孝王等九人。連日劉光明等大破寇於昌化、淳安,偽堵王黃文金、偽偕王譚體元伏誅。李遠繼挾秀全幼子奔江西廣信,於是浙江平。

初,秀全幼子及黃、李諸酋由寧國趨昌化之白牛橋,譚酋及偽樂王之子莫桂先、偽首王範汝增等由寧國趨淳安威坪,約同竄徽境,眾尚十數萬。劉光明擊白牛橋寇,黃酋中砲死,弟文英代領其眾,踉蹌西奔。黃少春斬譚體元於洪橋,並誅莫桂先等酋百五十餘人。劉明珍率所部由淳化上趨,值黃文英、李遠繼來犯,明珍偕魏喻義等分兵御之,創黃文英於陣。李遠繼挾秀全幼子遁至徽、歙之交。寇由建口渡河,我軍乘其半濟擊之,大潰,斬偽列王邱國文等,收降卒六千餘。餘黨向積溪而逸。其已度建口者,竄至遂安,黃少春等復擊走之。偽列王劉得義、蕭雅泗等率二萬人投誠。洪氏勢益孤,乃由遂安昏口遁走開化,竄入江西。

九月,鮑超擊偽康王汪海洋於寧郡,大破之,擒斬偽朝將王金瑞等百二十餘人。席寶田追剿湖州逃寇,大殲於廣昌白水嶺,擒偽干王洪仁玕、偽恤王洪仁政及偽昭王黃文英,皆伏誅。沈葆楨克雩都、會昌,練勇克瑞金。寶田追寇於石城,游擊周家良搜獲秀全幼子於黃谷,檻致南昌省城,誅之。各路官軍截剿餘寇殆盡,於是洪氏遂滅。

論曰:秀全以匹夫倡革命,改元易服,建號定都,立國逾十餘年,用兵至十餘省,南北交爭,隱然敵國。當時竭天下之力,始克平之,而元氣遂已傷矣。中國危亡,實兆於此。成則王,敗則寇,故不必以一時之是非論定焉。唯初起必託言上帝,設會傳教,假「天父」之號,應「紅羊」之讖,名不正則言不順,世多疑之;而攻城略地,殺戮太過,又嚴種族之見,人心不屬。此其所以敗歟?


\end{pinyinscope}