\article{志六十七}

\begin{pinyinscope}
禮十一(兇禮一)

皇帝喪儀皇后喪儀貴妃等喪儀

五曰兇禮。三年之喪,自天子以至於庶人,無貴賤一也。有清孝治光昭,上自帝後喪儀,下逮士庶喪制,稱情立文,詳載會典與通禮。茲依次類編,累朝損益,皎然若鑒焉。

皇帝喪儀天命十年,太祖崩。遠近臣民,號慟如喪考妣。越五日,奉龍轝出宮,安梓宮沈陽城中西北隅。國制,除夕、元旦備陳樂舞,至是悉罷。時東邦甫建,制闕未詳。

崇德八年,太宗崩。男自親王訖牛錄章京、朝鮮世子,女自公主訖奉國將軍妻,集清寧宮前,詣幾筵焚香,跪奠酒三,起立,舉哀。固山額真、昂邦章京、承政以下官及命婦集大清門外,序立舉哀。次日,奉梓宮崇政殿,王公百官朝夕哭臨三日。其齋所,王、貝勒、貝子、公歸第,部、院官宿署,閒散諸臣赴篤恭殿,固山額真等官及命婦,翌日暮還家。

世祖登極,年甫六齡,會天大寒,侍臣進貂裘,卻弗御,帝曰:「若黃裏,朕自衣之。唯紅,故不服耳。」是日不設鹵簿,不作樂。王大臣等謂已即位,冠宜綴纓,於是軍民皆綴纓。服官者暫停婚嫁宴會,民間不禁。乃頒哀詔朝鮮、蒙古,制曰:「我皇考盛德弘業,侯服愛戴。本年某月日,龍馭上賓,中外臣民,罔弗哀悼。屬在籓服,咸使聞知。祭葬禮儀,悉從儉樸。仍遵古制,以日易月,二十七日釋服。」詔到,國王以下舉行喪禮如故,時猶在關外也。

順治十八年,世祖崩,聖祖截發辮成服,王、公、百官、公主、福晉以下,宗女、佐領、三等侍衛、命婦以上,男摘冠纓截發,女去妝飾翦發。既大斂,奉梓宮乾清宮,設幾筵,朝、晡、日中三設奠,帝親詣尚食祭酒,三拜,立,舉哀。王、公、大臣、公主、福晉、縣君、宗室公夫人詣幾筵前,副都統以上序立乾清門外,漢文官赴景運門外,武職赴隆宗門外,咸縞素,朝夕哭臨,凡三日。外籓陪臣給白布制服。至四日,王公百官齋宿凡二十七日。過此則日哭臨一次,軍民服除。音樂、嫁娶,官停百日,軍民一月。百日內票本用藍筆,文移藍印。禁屠宰四十九日。京城自大喪日始,寺、觀各聲鐘三萬杵。越日頒遺詔天安門,★臣素服,三跪九拜。宣畢,舉哀。禮部謄黃,頒行各省。聽選官、監生、吏典、僧道,咸素服赴順天府署,朝夕哭臨三日。詔至各省,長官帥屬素服出郊跪迎,入公廨行禮,聽宣舉哀,同服二十七日除,命婦亦如之。軍民男女十三日除。餘俱如京師。

殷奠,列饌筵二十一,酒尊十一,羊九,楮幣九萬。讀文。帝詣幾筵哭,內外傳哭,奠酒,率眾三拜,舉哀,焚燎。設啟奠如殷奠儀。屆日奉梓宮登大升轝,三祭酒,並祭所過門、橋。帝號泣從,群臣依次隨行。將至景山,內外集序,俟靈駕至,跪舉哀。奉安壽皇殿訖,設幾筵,帝三祭酒,每祭一拜,哀慟無已。皇太后再三撫慰,始還宮。明日行初祭,帝釋服。又明日行繹祭,周月行月奠,自是百日內月奠,期年內月奠,儀並視殷奠,唯所陳品幣有差。期年滿月致祭,不讀文。

上尊謚廟號,祗告郊廟社稷。屆日殯宮外陳鹵簿,作樂,大學士奉冊寶陳案上,三叩,退。帝素服御太和門,閱訖,一跪三拜,退立東旁。大學士詣案前,復三叩,奉冊、寶列採亭內,如初禮。校尉舁行,御仗前導,車駕從。王公百官先集協和門外,跪迎,隨行壽皇殿大門外,冊寶亭入,至簷前,帝入自左門,禮部長官先奉絹冊寶陳中案,退。大學士詣亭前三叩,奉香冊寶陳左案。帝就位,率眾三跪九拜,大學士從左案奉冊跪進,帝獻冊,授右旁大學士,跪受,陳中案上。進寶亦如之。乃宣冊,宣冊官奉絹冊宣訖,三叩,退。宣寶儀同。帝率眾行禮如初。復詣幾筵前致祭,奠帛,讀文,三獻爵,如儀。焚絹冊寶,禮成。翌日頒詔如制。百日內外集序,讀文、哭奠如初祭。

是日題神主,大學士一人進觀德殿,詣祔廟神主前上香,奉主至壽皇殿外陳案上,並三叩。滿、漢大學士各一人,詣香案前復三叩。填青訖,行禮如初。奉主登黃轝,至觀德殿前止。大學士進殿,詣祔奉先殿神主前三叩,奉主登安轝,隨黃轝後,出景山東門,入東華門,帝素服跪迎景運門內,從至乾清門,轝止。帝詣兩神主前各三叩,先後陳案上,三獻,九拜,禮成。諏吉升祔,祥吉禮。

大祭如初祭儀。畢,帝升殿,延見群臣。清明、中元、冬至、歲除,並以時致奠。

既卜葬吉,將奉移山陵,前三日,遣告天地、宗社。前一日,設祖奠,儀如啟奠。先是王大臣援引古禮,止駕遠送,不許。至是奉太后懿旨,不獲已,勉遵慈命。屆日內外齊集,帝詣梓宮奠酒,盡禮盡哀。輔臣率執事官奉梓宮登轝啟行,鹵簿前導,冊寶後隨,帝攀號。俟過,步至東安門外泣奠,群臣從之。所過門、橋皆致祭。途中宿次,朝夕奠獻,親王行禮,群臣舉哀。百里內守土官素服跪迎道右。至陵,奠獻如在途。

大葬前期,遣輔臣及三品以上官詣陵陳祭。先三日,祗告如常告儀。屆日輔臣詣梓宮告遷,三奠酒,奉梓宮登轝,群臣序立,跪舉哀。俟轝過,哭從。至地宮,王大臣奉梓宮入,冊寶陳左右,掩石門。輔臣率眾三奠酒,舉哀,鹵簿儀仗焚。饗殿成,奉安世祖神位,致祭如時饗。屆二十七月,詣太廟祫祭,如歲暮祫祭禮。

康熙六十一年,聖祖崩,大斂,命王公大臣入乾清門瞻仰梓宮,並命皇子、皇孫行禮丹墀上,公主、福晉等咸集幾筵殿前,帝及諸皇子成服。以東廡為倚廬,頒遺詔,諭禮臣增訂儀節。屆時帝立乾清宮外,西鄉,大學士奉遺詔自中道出,帝跪,俟過,還苫次。大學士出乾清門,禮部尚書三拜跪受,餘如故時遺詔。

二十七日釋服,帝曰:「持服乃人子之道,二十七日服制,斷難遵從。」群臣以萬幾至重,請遵遺詔除服。不允。復疏云:「從來天子之孝,與士庶不同。孝經曰,天子以德教加於百姓、施於四海為孝。書稱高宗諒陰,晉杜預謂釋服後心喪之文。蓋人君主宗廟社稷,祭為吉禮,必除服後舉行。若二十七日不除,祀典未免有闕。」復叩首固請,始俞允。既釋服,仍移御養心殿,齋居素服三年。靈駕奉安壽皇殿,日三尚食。退觀德殿席地坐,有事此進奏。晡奠畢,始還倚廬。

群臣議進尊謚,帝親刺指血圈用「聖祖」字。禮臣進儀注未愜意,更定。前期並祗告奉先殿,至日閱冊、寶訖,帝行一跪三拜禮,東次西鄉立,俟冊寶亭行始還宮,豫至殯殿倚廬恭俟。會朝鮮貢祭品,設幾筵前。群臣咸集,鴻臚寺官引來使入,立儀仗南,北鄉,三跪九拜。遣官讀文,三祭酒,每祭一拜,眾及來使咸舉哀行禮。來使復行二跪六拜禮,焚燎,退。外籓敖漢王請謁梓宮,報可。自是蒙籓使者皆得入謁以為常。

雍正初元,將奉移景陵饗殿,廷臣援宋、明二代禮,謂嗣皇帝不親送梓宮,帝不允。禮臣議奉安地宮後,題太廟神主,令親王敬奉還京。帝曰:「明季帝王不親送梓宮,故令王大臣代行。朕既親往,自宜親奉以還焉。」先奉移二日,並遣告後土、昌瑞山神。

屆日,帝詣梓宮祭酒,率眾三拜,舉哀畢,趨立大門東旁。梓宮出,跪,舉哀。登大升轝,帝跪左。禮臣祭轝,三叩。靈駕發,帝步隨。至景山東門,俟宿次。至景陵,帝跪迎紅門外,舉哀。徒步從,抵三洞橋,跪俟。降大升轝登小轝,安奉饗殿,設幾筵,列冊、寶。三祭酒,三拜,禮成。帝不忍別,群臣以皇太后為言。無已,翌日還蹕。王大臣請御門聽政,帝以梓宮未永安,命暫緩。固請之,始行。

既卜葬,屆日晨帝詣景陵奠獻,躃踴哀慟,祭酒三拜,趨陵寢門外跪哭以俟。龍輴入地宮,復祭酒三拜,出俟幄次。題主、虞祭如常儀,歸奉主升祔太廟。二十七月將屆滿,允吏部尚書硃軾請,祫祭太廟,頒示臣民。

世宗崩,喪禮悉依景陵故事。越日朝奠,特簡王、貝子、公數人入內瞻仰,餘集乾清宮廊下行禮。嗣後王公大臣、額駙暨臺吉初至者,均得請旨瞻仰。又命宗室三十人、覺羅二十人番上奠獻申哀慕。

頒遺詔,大學士奉至乾清宮簷下,帝親受之,陳案上,三拜。大學士詣黃案前亦三拜。詔出中門,帝跪迎,俟過,始還苫次。詔至直省,軍民男女改素服二十七日。梓宮奉移雍和宮,帝徒步隨行,群臣諫阻不獲,遂留居是宮。至二十七日後始還。月內日叩謁,月外間日一次,二月外三日一次。

時帝欲行三年之喪,廷臣請以日易月,不許。命詳稽典禮。尋議上:「一,祭祀,按禮記王制『喪三年不祭,唯祭天地社稷,越紼行事』。註謂『不敢以卑廢尊』。是知三年內本應親行。明呂坤謂祖宗不輕於父母,奉祭不緩於居喪,何可久廢?誠以天親一理,宗廟之祭,亦當並舉。謹議:凡遇郊廟、社稷、奉先殿大祀,皇帝躬詣行禮,或遣官恭代,皆作樂。先期齋戒,素服,冠綴纓緯,視祝版,御禮服。朝日,夕月,饗帝王、先師、先農,遣官行禮,咸禮服作樂。屆日冠服如齋期。宮內祭神,百日後舉行。經筵、耕耤,釋服後舉行。一,朝會,典禮攸關,元旦朝正,萬國瞻仰,朝儀最重。謹議:二十七月內,遇元旦朝賀,吉服升太和殿,不宣表,不作樂,常朝亦然。一,御門聽政,典制至鉅。昔宋仁宗行三年喪,臨朝改服。孝宗時,二十七日後,百官請聽政,援書被冕服出應門語固請,乃許。稽之史冊,自古為然。謹議:常事及引見俱在便殿,百日後乃御門。一,冠服,按諒陰之制,先儒謂古無可考。史載魏孝文帝、唐德宗釋服後仍素服練巾聽政,宋仁宗雖用以日易月制,改服臨朝,宮中實行三年之喪。蓋縞素不可以臨朝。前代行三年喪者,亦唯宮中素服而已。謹議:百日內服縞素,百日外易素服,詣幾筵仍服縞素,御門蒞官聽政或詣皇太后宮俱素服,冠綴纓緯。升殿受朝則易吉。祭祀及一切典禮俱禮服。二十七月服滿,如百日禮,致祭釋服。一,宮中服制,帝後齊體,服制不容有異。二十七日後後素服,遇典禮易禮服,詣幾筵仍縞素。妃嬪亦如之。皇子與諸王同。一,在京王公百官,二十七日除服。遇典禮及朝會、坐班吉服,在署治事、入朝奏事俱素服,冠綴纓緯。詣幾筵去冠纓。各署進本章用硃印。」制可。

乾隆元年正旦,以御極初元,御太和殿常朝,次年仍罷,著為例。將移泰陵,帝詣梓宮行禮畢,皇太后亦三祭酒,餘如故。向例清明、中元、歲暮、國忌皆朝服行禮畢,素服舉哀,唯冬至不更素服。帝以梓宮未葬,且在服內,允禮臣請。承祭執事各官不綴冠纓,仍用素服。

嘉慶四年,居高宗喪,如泰陵故事,唯遺詔到直省,文武官率紳耆摘纓素服出郊跪迎,入公署行禮。聽宣畢,舉哀,始成服,哭臨三日。官吏軍民自大事日始,百日不薙發。大葬,帝躬引梓宮御龍輴入地宮。復以朝正大禮元年已行,二十七月內不再舉。

仁宗崩熱河,越六日,梓宮至京,始大斂,奉安淡泊敬誠殿。又四日,頒遺詔,禮官奉安龍亭,驛送入都。舊制,自太后以下二十七日後俱素服,孝和睿皇后改服縞素,百日後始易。喪將至,群臣出郊哭迎,帝先返,至安定門、東華門,並祗俟哭迎。步隨入大內,奉安乾清宮。允禮臣議,喪服已屆二十七日,改大祭後除服。又幾筵前奠獻,陳法駕鹵簿,百官會集暨各署用藍印,俱大祭后停罷。

宣宗崩,梓宮奉移圓明園,安正大光明殿。會衍聖公至京,遇二周月致祭,命赴園隨行禮。

文宗崩熱河,依宣宗故事,梓宮移東陵。穆宗年尚幼,群臣援康熙二年例,止帝遠送。同治二年釋服,奉兩宮皇太后懿旨,諸慶典及筵宴,俟山陵事畢再行。穆宗、德宗崩,並循斯例。

自世宗親營泰陵吉壤,工需動用內帑,並諭毋建石像,惜人力。宣宗葬慕陵,規制簡約。至同治時,侍郎宋晉言定陵工程宜法慕陵,雖廷臣囿於成憲,而制度毋稍逾侈,時稱其儉。宣統初,為德宗營崇陵,頒帑數百萬,親貴主其事,移以營私第,致逾三年未成。遜國後,當道撥款營治,及葬,工甫半,故較舊制為略云。

皇后喪儀太祖癸卯年九月,皇後葉赫納喇氏崩。越三載,葬尼雅滿山。天聰三年,與太祖合葬福陵,制甚簡也。入關後,凡遇列後大事,特簡大臣典喪儀,會禮臣詳議。

順治六年四月,太宗皇后博爾濟吉特氏崩,梓宮奉安宮中,正殿設幾筵,建丹旐門外右旁。首親王訖騎都尉,公主、福晉、命婦咸集。世祖率眾成服,初祭、大祭、繹祭、月祭、百日等祭,與大喪禮同。七年,上尊謚曰孝端文皇后,葬昭陵。

聖祖母慈和皇太后佟佳氏,康熙二年二月崩。初違豫,帝時年十一,朝夕侍。及大漸,廢餐輟寐。至是截發成服,躃踴哀號,水漿不入,近侍感泣。日尚三食,王公大臣二次番哭。停嫁娶,輟音樂,軍民摘冠纓,命婦去裝飾,二十七日。餘凡七日。四日後,入直官摘冠纓,服縞素。五日頒詔,文武官素服泣迎,入公署三跪九拜,聽宣舉哀,行禮如初。朝夕哭臨三日,服白布,軍民男女素服如京師。上尊謚曰孝康章皇后。梓宮移壩上,帝祭酒行禮攀號,太皇太后、皇太后念帝沖齡,止親送。與世祖合葬孝陵,升祔太廟。

十二年五月,皇后赫舍里氏崩,輟朝五日,服縞素,日三奠,內外會集服布素,朝夕哭臨三日。移北沙河鞏奉城殯宮,帝親送。自初喪至百日,亦躬親致祭。時用兵三籓,慮直省舉哀制服易惑觀聽,免治喪,餘如故。冊謚仁孝。三周後,致祭如陵寢。後葬昌瑞山。世宗登極,謚曰孝誠仁皇后。

十七年二月,皇后鈕祜祿氏崩,喪葬視仁孝後,冊謚孝昭。世宗加謚曰仁。

二十六年,世祖母博爾濟吉特氏崩。先是太皇太后違豫,帝躬侍,步禱南郊,原減算益慈壽。親制祝文,詞義墾篤。太常宣讀,涕泗交頤。既遭大喪,悲號無間。居廬席地,毀瘠過甚,至昏暈嘔血。自是日始,內外咸集,日三哭臨,四日後日二哭臨。官民齋宿凡二十七日。寺、觀各聲鐘三萬杵。文移藍印,題本硃印,詔旨藍批答。值除夕、元旦,群臣請帝暫還宮,不許。唯令元旦輟哭一日。禮臣議上尊謚曰孝莊文皇后。帝以升遐未久,遽易徽號為尊謚,心實不忍。諭俟奉安寢園,稱謚以祭。及梓宮啟攢夕,攀慕不勝,左右固請升輦,堅不就駕,斷去車靷,慟哭步送。遇舁校番上,輒長跽伏泣,直至殯宮,顏悴足疲,淒感衢陌。又傳旨還宮日仍居乾清門外幕次。並定志服三年喪,不忍以日易月。群臣交章數請除服,國子生五百餘人咸以節哀順禮為請,帝骨立長號,勉釋衰絰,而有觸輒痛,閱三年不改。

初太皇太后病篤時,諭帝曰:「太宗梓宮奉安已久,卑不動尊,未便合葬。若別營塋域,不免勞費。我心戀汝父子,不忍遠去,必安厝遵化為宜。」帝遂相孝陵南建饗殿,奉安梓宮,稱暫安奉殿,設官奉祀如孝陵制。至世宗改建地宮,號昭西陵,始大葬。

聖祖仁皇后佟佳氏,二十八年七月崩,時由妃立後第二日也。帝輟朝親臨,制四詩悼之,謚曰孝懿,喪儀如孝昭。

世祖皇后博爾濟吉特氏,五十六年十二月崩。先是疾大漸,禮臣請如孝康後喪禮。帝言:「孝康升遐,朕甫十歲,輔臣治喪,禮恐未備。後見仁孝後喪儀,條理頗晰,如遇大事,其悉議以行。」及崩,會帝病足,舁近幾筵,就榻成服。哭而暈,有間蘇。群臣環跽叩勸,乃勉舁側殿。將移殯宮,設啟奠,禮臣請遣皇子代。帝曰:「此初祭,朕必親奠,寧壽宮中豈能復行此禮耶?」至日遣代奠爵,仍舁幾筵旁榻上行禮。梓宮啟行,舁榻哭送,出寧壽宮西門,仰望不見靈駕,乃止哀,還苫次。大祭,足疾少愈,即親詣殯宮行禮。謚曰孝惠章皇后,葬孝東陵。

雍正元年,世宗母仁壽皇太后烏雅氏崩,喪禮如孝惠,謚曰孝恭仁皇后,與聖祖合葬景陵。時帝遭聖祖喪,齋居養心殿。服竟,仍終太后喪。輔臣援聖祖喪禮請服闋行祫祭,帝曰:「父母之喪,人子之心則一,帝後之禮,國家之制迥殊。今屆皇妣釋服期,諏日祭告奉先殿,無頒諭中外為也。」

九年九月,世宗皇後那拉氏崩,帝服縞素十三日除,奉移田村,三周年後,殯宮時奠與沙河殯宮禮同,唯承祭各官改補服。高宗立,上尊謚曰孝敬憲皇后。乾隆二年,與世宗合葬泰陵。

十三年三月,帝奉皇太后東巡,皇後富察氏從,還至德州崩,親制悼亡篇。喪將至,王公大臣詣通州蘆殿會集,皇子祭酒,舉哀行禮。既至,群臣素服跪迎朝陽門,公主近支王福晉集儲秀宮,諸王福晉及命婦集東華門外,咸喪服跪迎梓宮,奉安長壽宮。帝親臨成服,輟朝六日。

中宮之喪,自孝誠仁皇后後,直省治喪儀制久未舉行。至是王大臣言:「周禮為王後服衰,註謂諸臣皆齊衰,是內外臣工無異也。明會典載後喪儀,十三布政使司暨直隸、禮部請敕差官訃告。外省官吏軍民,服制與京師同。今大行皇后崩逝,正四海同哀之日,應令外省文武官持服如制。」從之。冊謚孝賢。

五月,廷臣奏言:「後雖儷體,禮統所尊,升殿視朝,事關典制。孝賢皇后喪儀,應遵祖制,百日後皇帝升殿,文武百官及外籓使臣朝服行禮如常儀。帝兩月除沐禮,御門聽政,群臣朝服不掛珠,禮畢仍素服。百日後如御門,群臣常服掛珠,庶協禮制分義。」帝曰:「孝賢皇后喪儀,朕皆斟酌古今,不參私意。考明嘉靖七年孝潔陳皇后之喪,張璁援引古禮,謂『喪服自期以下諸侯絕,特為旁期言。若妻喪本三年報服,殺為期年,固未嘗絕。上宜為後服期喪』云云。今據議奏,如升殿作樂,凡大朝祀典,自當如例。唯常日視朝,但鳴鐘鼓,樂懸而不作。至明年正月,將屆期年,一切典禮如常儀。」

時沂州營都司姜興漢、錦州知府金文醇國恤期內薙發,所司以聞,下部逮治。並申明祖制,禁百日內薙發,違者處斬。諭載入會典。

三十一年,皇後那拉氏薨,時帝幸熱河,留京王大臣以聞。詔言:「後自冊立以來,尚無失德。去年侍太后南巡,性忽改常,未盡孝道,理應廢黜。今仍存其名號,喪儀依貴妃例,內務府大臣承辦。」

仁宗母魏佳氏,四十年正月在貴妃位崩,詔稱令懿皇貴妃,命皇八子、十二子、十五子、皇孫綿德等穿孝,葬勝水峪。嗣立仁宗為皇太子,遂贈謚孝儀皇后,升祔奉先殿,後復上廟謚為純皇后,乃升祔太廟。

高宗母崇慶皇太后鈕祜祿氏,四十二年正月崩,帝衰服百日,如世宗喪,餘仍素服。親擬尊謚曰孝聖憲皇后。禮臣上喪儀,援雍正九年例,二十七日內遇郊廟大事,素服致祭,樂設不作。帝曰:「郊廟典重,不應因大喪而稍略。」復下軍機大臣議。旋議上:「遇郊廟大祀,遣官致祭,仍作樂,朝服行禮,常祀素服致祭,樂設不作。」制可。頒遺詔,自到省會日始,停嫁娶,王公百官百日,軍民一月。輟音樂,王公百官一年,軍民百日。餘如故。

先是歷代喪禮,百日後服色禮制,未載會典,至是命軍機大臣會典喪儀王大臣詳議。議上御殿視朝儀注。得旨:「元正朝會,二十七月內不必舉行。其常日視朝,百日後行之。」

又議定御用服色:「一,百日內縞素。百日釋服後,二十七月內素服。詣幾筵,冠摘纓。一,百日內遇祭郊、社、日壇,遣官將事。齋戒日,素服冠綴纓。百日外,親詣行禮。又齋期,常服不掛珠。閱祝版,先期宿壇,常服掛珠。祭日朝服作樂,還宮樂設不作。一,百日外祭事御龍袍褂。百日內祭奉先殿冠綴纓、青袍褂,百日外珠頂冠、藍袍、金龍褂。一,二十七月內祭月壇、帝王、先師、先農,俱遣官行禮。一,宮中祀大神,百日後親詣行禮,龍袍、藍褂、掛珠。一,二十七日外,遇元旦,前後七日貂褂掛珠,百日外,御門聽政,常服不掛珠。一,二十七日外百日內,召見及引見俱在便殿,服縞素。遇萬壽節。七日常服。一,閱視大行皇太后冊、寶,素服冠綴纓,先期齋戒帶牌。一,閱視玉牒,朝服。一,十二月封寶,正月開寶,御龍褂。一,文武傳臚不升殿。一,經筵、耕耤,二十七月後舉行。一,山陵禮制,二十七月內謁陵,青袍褂,冠摘纓,其往返在途,冠並綴纓。一,內廷主位,二十七日釋縞素後,二十七月內常服。遇元旦萬壽,俱七日吉服。百日內遇親蠶,遣王福晉恭代。朝服,百日外二十七月內,依舊行禮,吉服。其文武百官,二十七日縞素,百日內素服,冠綴纓,夏用雨纓冠,詣幾筵仍摘纓。一,百日內祭郊廟、社稷、日壇,遣官恭代。先期省牲、視牲咸素服。祭日,承祭、執事各官咸朝服。作樂。百日外二十七月內,親詣行禮。齋戒日常服掛珠,閱祝版、省視牲、宿壇並補褂。冬貂褂掛珠。祭日,朝服作樂。一,百日外祭堂子,俱蟒袍、補褂、掛珠。百日內祭奉先殿,青袍褂,冠綴纓。百日外補褂、掛珠。一,百日外祭月壇、帝王、先師、先農,遣官行禮,皆素服齋戒。祭日,朝服,作樂。百日內素服行禮,樂設不作。一,二十七月內遇元旦謁堂子,百官皆蟒袍、補褂、掛珠。其前後三日及萬壽前後七日皆常服掛珠。一,二十七日外百日內引見官,青袍褂。百日外青褂。一,百日外二十七月內,遇升殿、常朝、坐班俱朝服。遇朔、望常服掛珠。一,奉移山陵,隨從官在途青袍褂、冠摘纓。禮成後,神主還京,並百日後隨從謁陵,在途俱青袍褂,冠綴纓。謁陵日如之。還京時,仍短襟袍、馬褂。一,百日內雨衣、雨冠均青色。百日外雨冠按品級,雨衣仍青色。皇子以下同。」制可。

四月,葬泰東陵,梓宮逕泰陵,命暫停道旁,帝代向陵寢行禮,著為令。

至陵翼日行饗奠禮。初,帝以會典舊稱「遣奠」,稱名未當,命儒臣稽所自昉。大學士言:「遣奠之稱,禮經並無明文,唯見諸孔穎達士喪禮疏,唐以後相沿用之。蓋穎達第用儀禮葬日將行苞牲體之車名為遣車,遂取遣字為奠名,牽合無當。復考儀禮,將行之祭,『徹巾苞牲。』鄭康成註:『象既饗而歸賓俎也。』又禮記雜記:『大饗既饗,卷三牲之俎歸於賓館,所以為哀也。』鄭註:『既饗歸賓俎,言孝子哀親之去也。』是將行之祭,本用饗禮,舊稱遣奠,似不若作饗奠為長。」敕下部更正從之。

四十四年四月,帝詣陵釋服。諭曰:「朕昔遭皇考大故,思持服三年,因遵聖母慈諭,斷以百日。然縞素雖釋,其服仍存。嗣值聖母大喪,百日後即不存,非厚前薄後也。蓋彼時年力正壯,可終三年喪制。今春秋望七,設存之而弗克盡禮,於心轉不安也。」

仁宗皇后喜塔臘氏,嘉慶二年二月崩,奉太上皇敕旨,喪儀如皇后。改為輟朝五日,素服七日。奠醊時,皇子等成服如制。官民俱素服七日,不摘纓,不蓄發。尋諭輟朝期內,仍進章疏,毋廢引見諸事。其奏事官暨引見官,俱常服不掛珠。凡停嫁娶、輟音樂,官二十七日,軍民七日,餘如儀。冊謚孝淑,嗣葬太平峪。

十三年正月,宣宗皇后鈕祜祿氏崩,時在福晉位,暫安王佐村園寢,二十五年帝即位,追封孝穆皇后。擬改園寢為陵寢,禮部言:「園寢規制未備,忌辰大祭,朔、望小祭,請如孝淑後殯宮例舉行。」制可。遂命大學士戴均元等勘定寶華峪,嗣以地宮滲水,道光十一年,改葬龍泉峪。

越二年,宣宗皇后佟佳氏崩,帝輟朝九日,素服十三日,冊謚孝慎。又越二年,卜葬,與孝穆后同吉壤。

二十年正月,皇后鈕祜祿氏崩,帝服青袍褂十三日除,臨奠仍素服。謚孝全。亦葬龍泉峪。

二十九年十二月,仁宗皇合鈕祜祿氏崩,謚曰孝和睿皇后。時帝年七十,二十七日釋縞素,數日而崩。咸豐三年,葬昌西陵。

方孝和後崩次日,文宗後薩克達氏崩福晉位,內府治喪,殯田村。次年正月帝即位,追封孝德皇后,其喪儀先期豫改,如大喪禮。同治四年,與文宗合葬定陵。

康慈皇貴太妃,宣宗皇貴妃也。咸豐五年七月,尊為皇太后。俄崩,帝持服百日如制。加謚孝靜,升祔奉先殿,改慕陵妃園為慕東陵。同治初元,加廟謚曰成,升祔太廟。

光緒元年二月,嘉順皇後蒙古阿魯特氏崩,去穆宗喪未百日,帝釋縞素後,率群臣服喪二十七日,儀如故事。謚曰孝哲毅皇后。五年,與穆宗合葬惠陵。

慈安皇太后,鈕祜祿氏,文宗後也。七年二月崩,謚曰孝貞顯皇后,葬定東陵。

三十四年十月,慈禧太皇太后後德宗一日崩,詔禮部從優具議。尋議百日內上諭用藍筆,章疏十五日後具奏。王、公、百官、公主、福晉、命婦二十七日內日三哭臨。官停嫁娶期年,輟音樂二十七月,京師軍民二十七日罷祭祀,餘如大喪禮。謚曰孝欽顯皇后,葬定東陵。

貴妃等喪儀順治初,定制,妃、殯之喪,內務府掌行,臨時請旨。

康熙四年,壽康太妃博爾濟吉特氏薨,帝輟朝三日,大內及宗室咸素服。王、公、大臣、公主、福晉、命婦畢集。初祭,陳楮幣十四萬,畫緞萬,饌筵三十有一,牛一,羊十八,酒九尊,讀文致祭。次日繹,陳楮幣萬,饌筵五,羊三,酒三尊。大祭同初祭。奉移豫祭,陳楮幣二萬,饌筵十三,羊五,酒五尊。歲時致祭如例。

九年,慧妃博爾濟吉特氏薨,輟朝三日,大內、宗室咸素服。三日不祀神。妃宮中女子翦發,內監截發辮,成服,二十七日除。又定金棺至殯宮,初祭陳楮幣十四萬,畫緞千,帛九千,饌筵二十一,羊十九,酒十九尊,設採仗行禮。奉移則陳楮幣三萬,饌筵十三,羊、酒各五。不直班官員跪迎十里外,俟過隨行。次日行奉安禮,如奉移儀。

十三年,太宗懿靜太貴妃博爾濟吉特氏薨,帝摘冠纓,躬詣致祭,餘同太妃儀。

三十五年,溫僖貴妃鈕祜祿氏薨,輟朝五日。命所生皇子成服,大祭日除,百日薙發,餘如制。

雍正三年,敦肅皇貴妃年氏薨,輟朝五日。特簡王公大臣典喪儀,遣近支王公七,內務府總管一,散秩大臣二,侍衛九十,內府三旗佐領,官民男女咸成服。大祭日除,薙發。日三設奠,內外齊集,百日後至未葬前,日中一設奠,朔望仍三奠,命內管領妻祭酒三爵。奉移日,禮部長官祭轝。金棺啟行,王公百官從。禮部長官祭所過門、橋。初祭陳楮幣十八萬,帛九千,畫緞千,饌筵三十五,羊、酒各二十一。大祭同。

又定貴妃晉封皇貴妃,未受冊封前薨,罷制金冊寶,以絹冊寶書謚號。遣正、副使讀文致祭,先期遣告太廟後殿、奉先殿。屆日內外會集,正、副使赴內閣詣冊寶案前一跪三叩,奉冊寶出,至午門外陳採輿內,復三叩。校尉舁至殯宮大門外,正、副使行禮如初。奉冊寶入中門,陳案上。正使詣香案前三上香,宣訖,讀文致祭如儀。乾隆二年,奉移金棺從孝敬後葬泰陵。

八年,壽祺皇貴太妃佟佳氏薨,禮部以輟朝五日請,詔改十日。摘冠纓,親詣行禮,謚愨惠,餘同貴妃儀。

二十九年,忻妃戴佳氏薨,詔加恩如貴妃例治喪。先是,晉封時金冊寶已鐫字,未授受,至是陳設金棺前,其絹冊寶增書貴妃字焚之。又諭:「嗣後貴妃以上薨逝,王公大臣俱步送暫安處,妃、嬪豫往,滿大臣年老艱步履者如之。」故事,皇貴妃金棺至園寢,始制神牌,甚稽時日,三十三年諭:「嗣後遇大祭,即往園寢制造,俟金棺至,刻字填青,大學士等監視。奉安後,陵寢官朝服行禮,奉設饗殿。著為令。」

四十年,奏定皇貴妃以下五等喪。凡請輟朝、素服日期,傳行內外齊集,請遣承祭大臣,奉安地宮前期祭告陵寢及金棺前,並所過門、橋奠酒諸事,均禮部掌行。其追封贈謚制牌,會同二部奏辦,餘歸內府掌儀司牒禮、工二部襄治之。

四十九年,裕皇貴太妃耿氏薨,詔罷朝,仍親詣奠酒行禮,謚純懿,餘如故。

嘉慶四年,慶貴妃陸氏薨,帝念其撫育如生母,特追封慶恭皇貴妃,下所司議贈謚典禮。尋議上,豫期工部制絹冊寶,寢陵官制神牌,遣告太廟、奉先殿暨高宗幾筵,俟高宗梓宮移山陵次日,遣正、副使詣園寢配殿致祭。九年,議定皇貴妃喪,罷坤寧宮致祭酌減為五日,貴妃二日,妃、嬪不停止。

道光十三年,仁宗諴僖皇貴妃劉氏薨,不輟朝,不素服,命僧格林沁穿孝,謚和裕。

同治五年十一月初七日,琳皇貴太妃烏雅氏薨,會初十日慈禧太后萬壽,命大內、宗室王公百官展期十二日素服一日。


\end{pinyinscope}