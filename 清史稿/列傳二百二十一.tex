\article{列傳二百二十一}

\begin{pinyinscope}
沈棣輝鄧仁堃餘炳燾慄燿硃孫貽史致諤

劉郇膏硃善張子之榛黃輔辰子彭年

沈棣輝,字奏篪,浙江歸安人。少游淮上,為河督麟慶司章奏。道光中,納貲為廣東通判,補廣州永寧通判。擢黃岡同知,以功晉知府,補韶州。咸豐二年,調署廉州。時嶺嶠群盜並起,李士奎、顏品瑤、黃春晚等分踞欽州之那彭,靈山之林墟,眾數十萬。棣輝至,出賊不意,率兵二千掩入那彭,殲之。急分千人趨林墟,賊空壁出關,棣輝已由間道入其巢,遂連克旁近諸賊壘。博白賊劉八伺隙襲廉州,馳還,遇賊五里亭,令列陣以待。賊疑有伏,稍引去,呼噪乘之,賊大潰。休兵十日而進,又殲賊靈山早禾湧,追至廣西橫州,斬劉八。廉州平。

總督徐廣縉駐梧州,剿艇匪,檄棣輝出鬱林,援潯州。賊舟數百圍城,攻甚急。遣卒梯而入,約期會戰,伏兵兩岸,縱火焚賊舟,與城兵夾擊,大破之。督諸軍窮追,梧州水師邀擊,沉賊舟無脫者。論功,加按察使銜。時廣西賊竄湖南,徐廣縉督師移剿,棣輝隨參軍事。廣縉罷,葉名琛督兩廣,調棣輝回廣東治軍需。先已授廣西左江道,至是調肇羅道。四年,署廣東鹽運使。

陳開者,廣州匪首,倡亂踞佛山。群賊何子海、豆皮春、李文茂等應之,踞石門金官窯為犄角。連陷數十州縣,環省皆賊壘。賊渠陳光龍屯河南岸,何博奮海艇千餘踞省河,道路梗塞,外援皆絕。名琛悉以軍事付棣輝。選精銳四千人,以二千駐流橋、西山廟,為兩翼;以千人伏城中,出小西門分布要害,多張旗幟為疑兵。賊四面薄城,城內發砲中賊,陣亂,縱兵擊之,斬級千,賊自是不敢近城。至十一月,圍未解。棣輝謀於眾曰:「今外無援兵,內無積儲。聞賊中因爭食內攜,急擊不可失!」乃自將千人出攻小港橋賊壘。日晡未下,忽見賊營火起,大呼曰:「賊破矣!」士卒皆奮,克之。乘勝進攻佛山,值大霧,賊不虞其至,連戰皆捷,遂復佛山。

聞東莞水賊由石門犯省城,還軍救之。至黃竹岐,賊船數千,官軍僅數百艘,又居下風,勢甚危。棣輝禱於南海神,俄而反風,令裨將何高漢駕艨艟沖入賊陣,碎其乘舟,大破之,殺賊萬餘,俘數千,溺死者無算。危城獲全,又分兵殲賊酋黃福於潭州。五年,復順德、清遠、英德。賊圍韶州城已年餘,至是聞援兵至,遁。南北路悉平,擢按察使。六年,擢貴州布政使,未之任,卒。賜恤,贈內閣學士。

棣輝以文吏治軍,明賞罰,均甘苦,尤能知人。剿劉八時,招撫馮子材,後立大功為名將。何高漢乃賊何博奮之弟,推誠馭之,賴以成省河之功。廉州、潯州、廣州三戰,皆履險犯難,卒得大捷,尤為時稱云。

鄧仁堃,字厚甫,湖南武岡人。道光五年拔貢,以知縣用,發四川,歷署梁山、江油、洪雅。補綦江,調富順。薦卓異,以憂歸。服闋,入貲為知府,補江西南安,調署廣信。所至皆有政聲。署督糧道。咸豐二年,粵匪趨湖南,仁堃請修省城,籌守御。三年春,賊由武漢蔽江下,九江不守。巡撫出防,民爭遷徙,仁堃諭令安堵。上守江議,請增兵扼湖口,又條上城守事宜。實授督糧道。五月,賊犯江西,會江忠源師抵九江,仁堃請巡撫疏調,且遣使迎其師。忠源至,入任城守,與仁堃語合。巡撫張芾傾心倚任,曰:「戰問江君,守問鄧君!」地雷屢發壞城,皆以力戰獲完。仁堃欲出奇計焚賊舟,以鄱陽知縣沈衍慶忠果有謀,令率所部千人備草船藏火藥,約期襲賊,議阻未果。仁堃改糧船數十艘為砲艇,募卒扼守進賢門以保餉道。自夏徂秋乃引去。仁堃曰:「賊未受大創去,禍未已也!」亟請大修城以備,乃督工建砲城、砲臺,城上官房、營棚、軍器庫、硝磺庫、了臺、望樓皆備,浚環城壕深廣各三丈,築臨壕砲臺,甃石為堤閘,用銀十四萬有奇,守禦之具可恃。

五年,賊自湖北犯義寧,仁堃令道勇五百人往援。會贛州知府率勇二千至,仁堃請令駐義寧;巡撫陳其邁令防饒州,仁堃曰:「義寧扼三省要沖,官民頻年固守。團防為江省最,若棄不救,後將不能責官以守城,責民以團練。」力請改援,不許,僅以二百五十人往助守。中道遇賊,潰,義寧尋陷。未幾,羅澤南師至,仁堃固請往攻義寧,為措餉十萬濟之,澤南尋克其城。

十月,賊陷瑞州、臨江,圍吉安,下游賊復萃九江、湖口,南昌大震。仁堃添募捍衛、保衛軍,城備益嚴。曾國籓令副將周鳳山率三千五百人規臨江、瑞州,戰勝樟樹鎮。時按察使周玉衡孤軍守吉安,仁堃請檄鳳山乘勝援吉安。眾議倚鳳山蔽省城,仁堃爭曰:「賊知城高池深難卒攻,必為翦枝及本之計,先擾郡縣,使會城孤立,然後大舉而攻之。若懸賞二萬金,周軍必賈勇以解吉安圍,瑞、臨皆可復。吉安失,則撫、建必相繼不保,馴至全省糜爛,會垣且坐困矣。」終不聽。六年正月,吉安陷,周玉衡死之,鳳山軍潰於樟樹鎮,撫州、建昌亦陷,南昌屬縣並為賊躪。仁堃兼署按察使、布政使。

子輔綸,偕同知林源恩同率平江勇三千餘人,益以寶勇、志同軍進規撫州,復進賢。國籓亦檄李元度率勇四千自湖口移師會之,復東鄉,兩軍合破賊河東灣。攻撫州久不下,援賊驟至,營陷,林源恩死之。學政廉兆綸劾輔綸臬司子,不應與兵事,並劾仁堃辦城工不實,事下國籓及巡撫文俊按治,坐修城時未先請勘估,降五級調用。國籓疏言:「仁堃所承修為南數省第一名城。七郡並陷,省垣終保,不為無功。」仁堃既歸,輸穀三千石助軍。十年,協守武岡,以功議敘。同治五年,卒。

餘炳燾,字吟香,浙江會稽人。道光元年舉人,充景山官學教習。期滿,以知縣用,分發陜西。補清澗,調盩厔,又調渭南。回人馬得全等謀不軌,親入其巢捕之,置諸法,擢河南懷慶知府。咸豐三年,粵匪北犯窺開封,遂渡河圍懷慶。時郡城兵僅三百,炳燾選團勇三千人登陴固守,募敢死士縋城下砍賊營,又潛毒城外汲道使自斃。賊以地雷隳城者三,皆擊退。一日,雷雨中砲火蝟集,危甚,天忽反風,賊燔死者眾,勢頓沮。賊於近城樹木柵,以斷內外,為久困計。山東巡撫李僡先赴援,既而援軍四集,詔大學士訥爾經額督師。圍久,城中糧漸不支,炳燾素得民心,激以忠義,括糧節食,人心不渙。屢詔促戰,都統勝保、將軍托明阿等迭敗賊,賊始入山西竄,凡被圍五十八日乃解。特詔褒獎,賜花翎,以道員用,擢陜西鳳邠道。尋改授河南南汝光道。未幾,就遷按察使。

大河南北以防匪倡聯莊會,遇警相救;及賊去,聚而不散,莠民恃眾抗官。四年,禹州、鄭州、密縣疊肇變,圍城、焚署、縱囚、掠紳民。巡撫英桂出防信陽,咸請兩司奏聞待命。炳燾曰:「賊雖眾,皆烏合,志在剽掠,無紀律。速臨以兵,必驚潰,解散其黨,不久魁渠可縛也。若請朝命,遲將蔓延!」遂親率兵七百、勇五百馳往,剿撫兼施,事即定。尋署布政使。

捻首張洛行擾歸德,命炳燾往剿,攻雉河集,解亳州圍,又潛入永城,擊走之。既而歸德又有警,炳燾馳救,而他軍遽退,賊遂東逸。炳燾染病,特旨予假治理,不開缺。七年,卒。懷慶請祀名宦祠。

慄燿,字仲然,山西渾源人,東河總督毓美子。道光十五年舉人,以父血⼙廕,特賜進士。咸豐三年,授湖北漢陽知府,至則漢陽再陷,行省未復,督撫皆寄治軍中,委燿綜理營務。四年,從大軍復武漢,未幾,賊大至,城復陷,六年,始復。敘功,晉秩道員。燿以廉幹為巡撫胡林翼所器,令筦釐稅糧臺。八年,署荊宜施道。尋加按察使銜,授武昌道,仍留署任,兼督鈔關。軍餉皆仰資鹽榷,燿綜核嚴密,稅入羨餘,悉籍入公。修戰艦,增軍屯,水陸戰守皆有備。

十一年,賊逼施南,燿請重兵,復集民團,守山險。賊合川匪分掠宣、咸諸縣,施南協副將御之,遇伏,一軍盡沒。會劉岳昭軍至,與郡兵夾擊,賊大創,竄歸。松滋人馬鉦者,挾左道惑人,眾至數千,密通賊,官軍擒斬之。燿料賊不知鉦死,必復至,集水陸軍密為備。賊果趣夔州,遇官軍輒敗,及知馬鉦已誅,遂大潰。水陸合擊,俘斬萬餘,自是川匪無敢犯楚境。會大雨,荊江暴漲,齧攻萬城堤。燿督兵民備畚挶,儲土石,立泥淖間躬視板築,信宿堤上,事定乃還。

在荊州四年,政教大行。署按察使,兼攝布政使,甫逾月,授湖北按察使。燿以其父毓美曾任是職,乃顏其堂曰誦芬。同治元年,擢布政使,未任,卒。

硃孫貽,字石翹,江西清江人。入貲為刑部主事。改知縣,發湖南,歷署寧鄉、長沙,皆有聲。道光三十年,署湘鄉。漕務積弊,屢釀巨獄,孫貽蒞任,鄉民方麕集環譟。孫貽令曰:「新漕限迫,驟改章,弗及。來年當為若剔朘削弊,敢煽動浮言者罪之。會匪切近災也,亟縛獻!」眾唯唯散。疊捕盜魁陳勝祥、劉福田等置之法。稔知邑士之賢者,舉羅澤南孝廉方正;縣試拔劉蓉冠其曹;延王珍襄幕;於康景暉、李續賓、續宜皆獎勖之。廣西匪熾,孫貽集眾曰:「賊勢未易殄,北竄,湖南當其沖,欲衛閭里,非團練鄉兵不可。」王珍等曰:「謹奉令!」總督程矞採防衡州,孫貽以策干之,不省。會匪驟起,偕劉蓉、康景暉往捕。孫貽中彈,裹創戰於湖洞,擒賊目王祥二、熊聰一,王珍復捕賊百餘,檻致總督行營,前後七百餘人。

咸豐二年,洪秀全連陷道州、江華、永明、桂陽、郴州。孫貽集團丁分三營,以羅澤南領中營,易良幹副之;王珍領左營,揚虎臣、王開化、張運蘭隸焉;康景暉領右營。羅信南綜糧糈,謝邦翰治兵械。推古人陣法,制為起伏分合,湘軍紀律自此始。長沙圍未解,王珍、康景暉、趙煥聯分駐要隘;羅澤南、易良幹防縣城,伏莽蠢動,即時捕滅,縣境肅然。三年,巡撫張亮基聞湘鄉團丁名,調防省城,孫貽令王珍、羅澤南、羅信南、劉蓉率之往。四年,孫貽率團破安化藍田賊,擢郴州直隸州。

江忠源奉幫辦軍務之命,與曾國籓議援江西,令孫貽率湘軍赴之。羅澤南領中營,易良幹領前營,謝邦翰領右營,康景暉領左營,揚虎臣領後營,羅信南領親兵營,共三千人,至南昌,戰永定門外,大破賊。謝邦翰、易良幹、羅信東窮追被戕,孫貽哭之慟,以李續賓代領右營,羅信南兼領前營。吉安土匪鄒恩隆應賊,孫貽扼樟樹鎮,分軍令澤南、續賓及劉長佑剿平之。南昌圍解,凱旋,加知府銜,擢寶慶知府。諏才俊,嚴保甲,懲積匪,一如治湘鄉時。捐寺觀貲產制旗械軍火,募戰士千人,發義倉、常平儲穀充餉,親歷各鄉訓練,捕新寧山門團匪誅之。五年,粵匪陷東安,率千人偕副將聯霈馳扼五峰鋪,賊不敢犯。衡陽土匪起,出境平之。

六年,駱秉章疏薦人才,記名以湖南道員簡放。尋以治防功被優敘。八年,勞崇光調赴廣西,假滿未出,降一級調用,仍治湘、寶團防。十年,會劉長佑克廣西柳州,開復處分,賜花翎,加按察使銜。駱秉章赴四川督師,奏調孫貽總理營務。同治元年,擢授浙江鹽運使。秉章奏治川省團練,孫貽與秉章左右議不合,引疾請罷。命力疾赴陜西佐理多隆阿營務,以病辭,終不復出。光緒五年,卒。

史致諤,字士良,順天宛平人,原籍江蘇溧陽。道光十八年進士,選庶吉士,授編修。道光末,出為江西廣信知府。咸豐元年,署南昌。三年,粵匪犯江南,九江戒嚴,南昌訛言四起,城門晝閉,致諤請開城以安人心。尋回廣信任。賊陷饒州,致諤募勇號信新軍,因險設防,與浙軍為犄角。四年,調南昌。江西諸郡行淮鹽,惟廣信行浙鹽。軍興,淮鹽不至,致諤議借銷浙引,以餘息充餉,名曰「餉鹽」,從之,即以致諤襄其事。年餘,銷引逾常額,江、楚及浙皆利之。賊陷武寧,致諤率信新軍赴剿,迭挫賊於紫鹿嶺、巾口、火爐坪、箬田,復武寧。是年冬,南昌戒嚴,援師大集,主客軍不相下。致諤協和將吏,客軍二卒持刀擾質庫,立斬以徇。五年,兼署鹽法道。尋以母憂去官,留襄軍事。九年,服闋,命赴浙江交巡撫王有齡差遣。

同治元年,署寧紹臺道。寧波自前歲陷於賊,資洋兵之力復城,方謀畫曹娥江而守。尋以法總兵馬籌思所部與廣勇互斗,廣勇潰,賊乘間竄慈谿、奉化。致諤至,慈谿已陷,激厲民團登陴固守。與英總兵欻樂克、稅務司法人日意格推誠相結;以美兵官華爾忠勇可用,介以相見,令攻慈谿,以駐餘姚之洋兵及同知謝採璋團勇應之。慈谿賊分擾鄞縣境,及半浦,而嵊縣、新昌賊復大舉犯陳公嶺。華爾克慈谿,中砲歿於軍。陳公嶺不守,奉化復陷,郡城又警。致諤乞餉於上海,令都司楊應龍募忠勇軍,紳士李諤招大嵐山義勇,又以廣勇潰散,慮為賊用,招之回,令洋將布興有、布良帶,守備張其光分統之。部署甫定,賊由間道犯郡城,天雨陰霾,勒兵以待,伺賊懈出擊之,分兵兜剿,連捷於橫溪、石橋。進薄奉化,楊應龍率死士以梯登城,下之。時致諤已實授寧紹臺道。奉化竄賊復勾結上虞賊分道犯慈谿、餘姚。致諤以賊眾兵寡,分援則力弱,議直搗上虞,賊必還救,因出師漸遠,郡城餉事不能兼顧,乃請巡撫疏免前署道張景渠罪,責其專任兵事。連復上虞、嵊縣、新昌,增軍萬人,進規紹興。二年,復之。進克蕭山,與大軍會於錢塘江,浙東以平。巡撫左宗棠奏減杭、嘉、湖三府漕賦,致諤上書言:「蠲賦惠政,減正額尤當革浮收,各縣情形各異。當擇大者奏咨,餘並著為省例,以盡通變之宜。」三年,以籌餉功,加按察使銜,賜花翎。先以衰老乞歸,未允,至是原品休致,卒於家。

劉郇膏,字松巖,河南太康人。道光二十七年進士,江蘇即用知縣。咸豐元年,署婁縣,有政聲。三年,粵匪陷江寧,揚州、鎮江相繼失守。會匪劉麗川倡亂踞上海,附近川沙、南匯、嘉定、寶山、青浦諸縣並陷。巡撫吉爾杭阿檄郇膏隨營剿賊,郇膏率漕勇三百復嘉定,權知縣事,選丁壯嚴守望,稽保甲,籍游民,民心大定。敘功,加同知銜,賜花翎。補青浦。

八年,調上海。租界華洋雜處,數構釁,郇膏爭執是非,不為撓屈。有招工誘逼出洋者,親登舟搜獲,並追回已去者,民感之,洋人亦帖服。蘇、杭既陷,上海孤懸賊中,郇膏練民兵,四鄉設二十局,以資保衛。賊首李秀成陷松江,進犯上海。登陴堅守十餘日,賊不得逞而去。時大吏萃居上海,或議他徙。郇膏曰:「滬城據海口,為餉源所自出,異日規復全省,必自此始。奈何舍而去之?」十一年冬,賊復陷浦東諸縣,大吏檄郇膏往援,郇膏曰:「賊勢張甚,宜守不宜戰。」弗聽,率練勇、鄉團出戰,果敗,乃專議守。治行上聞,加道銜,以知府用。擢海防同知,超署按察使。尋實授,命署布政使,異數也!

李鴻章督師至,命總理營務,饋運無缺,兼協濟江寧大營,兩軍月餉二十萬,悉取給於上海。濬吳淞江以通運道,招集流亡,通商惠工,善後諸事,次第舉行。尋命護理巡撫,丁母憂。同治五年,卒。贈右都御史,上海建專祠,祀蘇州名宦。

硃善張,字子弓,浙江平湖人。諸生。授桃南通判,升里河同知。咸豐九年,擢淮徐揚海道。粵匪、捻匪時擾江北,奸民乘時蜂起。善張常在行間,剿幅匪於海州、沭陽,殲其渠,賜花翎,加鹽運使銜。捻首張隆據浮山,令水師伏臨淮焚其舟,又卻之小溪。粵匪陷天長,撲蔣壩,善張馳援,殪其酋,賜號庫木勒濟特依巴圖魯。善張方駐揚州,陳玉成來犯,攻城,發巨砲擊之,賊結堅壘為久困計。援師集,敗之七里店,追越儀徵以西,揚州獲安。尋賊復麕至,連營至司徒廟。善張晝夜守陴,時出殺賊,賊卒不得逞,引去。十年,捻匪陷清江浦,率師克之,築圩寨為善後計。

同治元年,調徐州道,兼筦糧臺,用堅壁清野法防捻匪。從僧格林沁攻孫甿老巢,破之棗溝。二年,苗沛霖叛,陷壽州,圍蒙城。善張知蒙城餉絕,輸粟助之。苗沛霖伏誅,湖團之亂起。湖團者,始議招流民開微山湖,自沛縣至魚臺,戶數萬,爭利亡命,遷跡其中。三年,新團𥫗匪殺掠沛縣劉民寨圩,善張會兵剿之,未竟,疽發背卒。贈右都御史,賜恤。

子之榛,以廕授官,補蘇州府總捕同知。歷以海運敘勞,晉秩道員。官江蘇凡四十年,筦釐務最久。精於綜覈,以剔除中飽為職志,地方利病,無不洞悉。署督糧道。歷署按察使十二次、布政使二次,大吏倚之。忌者眾,屢被彈劾,按治皆得白。光緒二十五年,清釐田賦,歲增漕糧十五萬石、丁銀二十萬兩。二十六年,海防戒嚴,省城獄囚謀變,之榛方署臬篆,出情實者駢誅之,事乃定。宣統元年,授淮揚道,未任,卒。

黃輔辰,字琴塢,貴州貴築人,原籍湖南醴陵。道光十五年進士,授吏部主事,累遷郎中。遇事侃侃持正論,屢忤上官,不少屈,時稱「硬黃」。咸豐初,以知府分山西。會貴州亂作,遽歸倡團練,修碉堡,積穀省城二萬餘石,撫清水江諸苗,平巴香亂,以功晉道員。尋赴山西,署冀寧道。餉絀,議加釐捐,輔辰謂晉人皆賈於外,山多地瘠,非他行省比,不宜病民。爭之不得,則請蠲苛細,取大宗,及不切民生日用者。戶部設寶泉分局於平定州,就鑄鐵錢。滯不行,則令分銷諸郡縣,歲收息銀三萬解部。輔辰謂:「京師用鐵錢以濟銅幣之乏,山西勿便也。今行各縣,議令交納錢糧,以三萬之微利,妨數百萬之正供,利一而害百。即專行平定一州,日積日滯,其患滋大。」議上,遂罷之。九年,調赴直隸軍營,察海口形勢,請以重兵扼北塘,當事迂其言,不用。尋乞假去。至四川,依總督駱秉章。

陜西自回亂,地多荒蕪,巡撫劉蓉議興營田。輔辰書陳方略,採官私書為營田輯要三卷,大旨在用民而不用兵,與民興利,不與民牟利,蓉疏薦之。五年,授陜西鳳邠鹽法道,任以西安、同州、鳳翔、延安、乾州、邠州、鄜州七屬營田事。輔辰建議謂:「關中土曠人少,非招徠客民,事末由濟。然耕牛、耔種、農具、棚舍,官不能給,民不樂趨也。莫若即以地畀之,薄收其租,畝二斗為差,六年後給券,使世其業。慮田無限制,賦無定則,吏得以意高下為民患,當先正經界,略如古井田法,量地百畝為區,編列次第,書賦額於券,視土肥瘠別等則上下授之。凡領墾者以先後為次,十區為甲,十甲為里,置長焉。里長總十甲租課,歲輸之官,凡移徙更替事皆責之。別授田六畝,俾食其入,為庶人在官者之祿,而官總其成。」令下,民皆便之。復定考課舉劾法,策奉行不力者。期年,凡墾田十八萬餘畝。時軍事急,賴所入租麥以餉之。又撥產給書院、義學、養濟院、育嬰堂、種痘局及灞岸堤工、渠工,諸廢皆舉。尋卒,祀名宦。

子彭年,字子壽。舉道光二十五年會試,逾兩年,改庶吉士,授編修。咸豐初,隨父在籍治團練,後入駱秉章四川戎幕,數有贊畫功,不受保薦。同治初,劉蓉延主關中書院。久之,李鴻章聘修畿輔通志,兼主蓮池書院。當光緒中,法、俄邊事迭起,侍從近臣多慷慨建言,彭年雖不在朝,負時望,中外大臣密薦之。八年,擢授湖北襄鄖荊道,遷按察使。屏餽遺,禁胥吏需索,年餘,結京控案四十餘起,平反大獄十數。調陜西,署布政使。

十一年,遷江蘇布政使。連歲水旱,米踴貴,屬縣請加漕折,巡撫欲許之,彭年謂:「定例漕糧一石,隨徵水腳錢一千,所費僅數百,獨不可以有餘補不足耶?今增漕折,民間多出二十萬緡,與國計無關,盡歸中飽。」持不可。十五年,護理巡撫,請以賑餘三十萬緡濬吳淞江、白茆河、蘊藻濱,工未及舉,十六年,調湖北布政使,總督張之洞尤倚重之。然守正不阿,遇庫款出入,齗齗以爭,雖忤其意,勿顧也。未幾,卒。

彭年廉明剛毅,博學多通。所至,以陶成士類為國儲才為己任。主講蓮池及在吳時設學古堂,成就尤眾。著有陶樓詩文集、三省邊防考略、金沙江考略、歷代關隘津梁考存、銅運考略。子國瑾,光緒二年進士,官編修。嗜學能文,甚有時譽。父喪,以毀卒。

論曰:軍興以來,監司賢者,保障一方,其功與疆吏等。軍政財政,各行省多有專任之人。沈棣輝平廣匪,餘炳燾守懷慶,其最著也。鄧仁堃殫心籌防,不盡見用。硃孫貽提倡團練,振興人材,實為湘軍肇基。劉郇膏主守上海以待援軍。皆以一縣令有裨大局。史致諤用外兵定寧波,硃善張保障淮、揚,功皆可紀。慄耀筦湖北稅釐,黃輔辰興陜西營田,並為兵食根本。黃彭年名父之子,久負時望,晚達未盡其用,時論惜之。


\end{pinyinscope}