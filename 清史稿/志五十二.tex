\article{志五十二}

\begin{pinyinscope}
地理二十四

△內蒙古

內蒙古:古雍、冀、幽、並、營五州北境。周,獫狁、山戎。秦、漢,匈奴盡有其地。漢末,烏桓、鮮卑薦居。元魏,蠕蠕及庫莫奚為大。隋、唐屬突厥,後入回紇、薛延陀。遼、金建都邑城郭同內地。元,故蒙古,起西北有天下。明,阿裕實哩達喇遁歸朔漠,復改號,遺踵曼衍,北陲多故。清興,蒙古科爾沁部首內附。既滅察哈爾,諸部踵降,正其疆界,悉遵約束。有大征伐,並帥師以從。定鼎後,祿爵世及,歲時朝貢,置理籓院統之。部落二十有五,旗五十有一,並同內八旗。乾隆間,改歸化城土默特入山西,仍有部落二十四,旗四十九。其貢道:由山海關者,科爾沁、郭爾羅斯、杜爾伯特、札賚特四部,旗十;由喜峰口者,阿嚕科爾沁、札嚕特、土默特、喀喇沁、喀爾喀左翼、奈曼、翁牛特、敖漢八部,旗十三;由獨石口者,阿巴噶左翼、阿巴哈納爾左翼、浩齊特、烏珠穆沁、巴林、克什克騰六部,旗九;由張家口者,阿巴噶右翼、阿巴哈納爾右翼、蘇尼特、四子部落、喀爾喀右翼、茂明安六部,旗七;由殺虎口者,歸化城土默特、烏喇特、鄂爾多斯三部,旗十二。是為內札薩克蒙古。袤延萬餘里。東界吉林、黑龍江,西界厄魯特,南界盛京、直隸、山西、陜西、甘肅,五省並以長城為限。北外蒙古。面積百四十八萬一千七百六十方里。北極三十七度三十分至四十七度十五分。經線京師偏東九度至偏西九度三十分。

科爾沁部旗六旗:在喜峰口東北八百七十里。西南距京師一千二百八十里。秦、漢,遼東郡北境。後漢為扶餘、鮮卑地。隋、唐為契丹、靺鞨地。遼為上京東境、東京北境。金分屬上京、北京及咸平路。元,開元路北境。明置三衛,自黃泥窪逾鐵嶺至開原曰福餘衛,以元後烏梁海酋領為都指揮,後自立國號曰科爾沁。清初以接壤聯姻。其後臺吉奧巴為察哈爾所侵,率先來降,太祖賜以土謝圖汗之號,後封親王四、郡王三、貝勒三、貝子一、鎮國公一、輔國公五,掌旗世襲。所部廣八百七十里,袤二千一百里。東界札賚特,西界札魯特,南界盛京邊墻,北界索倫。貢道由山海關。科爾沁右翼中旗札薩克駐巴音和碩南,曰塔克禪,在喜峰口東北一千二百里。西南距京師一千六百十里。本靺鞨地。遼為黃龍府北境。金屬上京路。元廢。牧地當哈古勒河、阿魯坤都倫河合流之北岸。東界那哈太山,南界察罕莽哈,西界塔勒布拉克,北界巴音和碩。廣一百五十里,袤四百五十里。北極高四十六度十七分。京師偏東四度三十分。其山:東曰烏蘭峰、那哈太山。南,阿達金察漢陀羅海坡、漢惠圖坡。西,鮮卑山,土名蒙格。北,溫山,蒙名哈祿納。東南,巴朗濟喇坡。西南,烏爾圖岡。其水:西北,郭特爾河,上承哈爾古勒河,自札魯特東南流入界,逕科爾沁左翼中旗。南,阿魯坤都倫河、鄂布爾坤都倫河,並自札魯特東來合流注之。又東南,逕右翼中旗南、左翼中旗北,屈曲流,至翁袞山東南,匯為察罕泊。北:阿爾達河,源出溫山,逕榆木山,東南流入右翼前旗;海拉蘇臺河,一名榆河,源出興安山,逕火山,東南流,皆與貴勒爾河會。鶴午河源出伊克呼巴海山,逕磨爾託山,東南流入右翼前旗,入榆河。科爾沁左翼中旗札薩克駐西遼河之北伊克唐噶里克坡,在喜峰口東北一千六十五里。西南距京師一千四百七十五里。本契丹地。遼為黃龍府北境。金屬上京路。元廢。牧地當吉林赫爾蘇邊門外昌圖界,跨東西二遼河。東界鄂拉達干,南界小陀果勒濟山,西界唐海,北界博羅霍吉爾山。廣一百八十里,袤五百五十里。北極高四十三度四十分。京師偏東六度四十分。其山:東南曰伊克圖虎爾幾山,一名牛頭山、巴漢圖虎爾幾山、巴漢哈爾巴爾山。西北,巴顏朔龍山、吉爾巴爾山一名水精山、巴漢查克朵爾山一名小房山。北,五峰山蒙名他奔拖羅海、伊克查克朵爾山一名大房山。東北,大石山蒙名葛倫齊老、太保山蒙名圖斯哈爾圖。西南,吉里岡。東南:遼河自永吉州入,逕額爾金山,西北流,入左翼後旗,又西南會潢河入邊。潢河自札魯特左翼入,逕噶爾岡東南來注之。卓索河源出邊內,西北流入左翼後旗,會尹幾哈臺河,入遼河。西北:和爾河,一名合河,自札魯特左翼入,東逕右翼中旗、前旗、後旗地,入因沁插漢池。阿祿昆都倫河自札魯特左翼入,逕葛勒圖溫都爾山,東流,會額伯爾昆都倫河,入右翼中旗,西北經魁屯山,東南流,會於合河。西北:中天河蒙名都穆達圖騰葛里,東天河蒙名準騰葛里,源均出吉爾巴爾山,東南流,會幾伯圖泉,入佟噶喇克插漢池,幾伯圖泉、他拉泉從之。科爾沁左翼後旗札薩克駐雙和爾山,在喜峰口東北一千四十里。西南距京師一千四百五十里。本契丹地。遼置鳳州。金廢。牧地當法庫邊門北,東西二遼河於此合流。東界碩勒和碩,南界柳條邊,西界伊柯鄂爾多,北界格爾莽噶。廣二百里,袤一百五十里。北極高四十二度。京師偏東六度二十分。其山:東,得石山。西界曰巴漢巴虎山。東北,得石拖羅海山。東,奚王嶺,土名蒙古爾拖羅海。東南:羊城濼,蒙名尹兀哈臺,源出邊內,流入境,北流,會卓索河,入邊河。科爾沁右翼前旗札薩克駐錫喇布爾哈蘇,在喜峰口東北一千三百五十里。西南距京師一千七百六十里。本靺鞨地。金置肇州,隸會寧府。海陵改屬濟州。承安三年升鎮。元,遼王乃顏分地。牧地當索嶽爾濟山南,洮爾河、歸喇里河於是合流注嫩江。東界岳索圖濟喇,南界達什伊哈克,西界那哈太山,北界索嶽爾濟山。廣一百二十里,袤三百八十里。北極高四十六度。京師偏東五度三十分。其山:西北曰喀喇阿幾爾漢山、魁勒庫山。北,神山、火山。東北,羊山蒙名衣馬圖、駱駝山蒙名特門。南,插漢碧老岱坡。西:洮兒河,源出西北興安山,東南流,合貴勒爾河,又東北折,逕右翼後旗南,又東逕札賚特南,匯為納藍撒藍池,入嫩江。北:貴勒爾河,自右翼鶴五河東北流,會榆河,為貴勒爾河,逕魁勒庫山,東南流,會阿爾達爾入洮兒河。駱駝河,蒙名特門河,源出葛爾濟隆山,東流,會戳兒河,東入嫩江。科爾沁右翼後旗札薩克駐額木圖坡,在喜峰口東北一千四百五十里。西南距京師一千八百六十里。本靺鞨地。遼置衍州安廣軍。金,州廢。元為乃顏分地。牧地跨洮兒河,即陀喇河。東界查巴爾太山,南界拜格臺陀博,西界博達爾罕山,北界慶哈山。廣一百二十里,袤三百七十里。北極高四十六度。京師偏東五度三十分。其山:東北曰西伯圖山、納幾山。北,硃爾噶岱山、卓索臺山。西南,鼐滿烏里堵坡。東南:因沁插漢池。科爾沁左翼前旗札薩克駐伊克岳里泊,在喜峰口東北八百七十里。西南距京師一千二百八十里。本契丹地。遼置長青州。金降為縣,隸泰州。元廢。牧地當法庫邊門外養息牧牧場東。東界霍雅斯,南界柳條邊,西界伊拉木圖,北界阿木塔克。廣一百里,袤一百二十里。北極高四十三度。京師偏東六度四十分。東南:龍門山蒙名阿會圖。東南:布敦山、寬山、朔龍峰。南:鴨子河,蒙名沖古爾,其地有二泉,並名沖古爾,西南流入養息牧河。東南有巴漢岳里泊。

扎賚特部一旗:附科爾沁右翼。扎薩克駐圖卜新察汗坡,在喜峰口東北一千六百里。西南距京師二千十里。本契丹地。遼,長春州。金,泰州北境。元為遼王分地。明為科爾沁所據,後分與其弟阿敏,是為札賚特。天命中,臺吉蒙袞來降,後封貝勒,世襲掌旗。牧地在齊齊哈爾城西。東界嫩江,南界鍾奇,西界烏蘭陀博,北界鄂魯起達巴哈山。廣六十里,袤四百里。北極高四十六度三十分。京師偏東七度四十五分。貢道由山海關。東北:阿敏山,蓋以所部之祖名其山也。西北:赤房山蒙名烏蘭格爾、雕窠山蒙名岳樂。北:朵雲山、塞肯山。西南:阿揚噶爾坡。東:嫩江,自黑龍江入,又南入郭爾羅斯前旗。北:綽爾河,源出西北興安山,東南流,至旗西,分數歧,又東南折入嫩江。西北:佗新河,自右翼後旗入,逕托額貴山,東南流,會綽爾河。西南:洮兒河,自右翼後旗入,東南流,匯為日月池,同入嫩江。以上統盟於哲裏木。盟地在科爾沁右翼中旗境內。

杜爾伯特部一旗:附科爾沁右翼。札薩克駐多克多爾坡,在喜峰口東北一千六百四十里。西南距京師二千五十里。本契丹地。遼,長春州。金,泰州北境。元,遼王分地。明為科爾沁所據,後分與弟愛納噶,是為杜爾伯特。天聰中,臺吉阿都齊來降,後封其子賽冷貝子,世襲掌旗。牧地當嫩江東岸、齊齊哈爾城東南。東界哈他伯齊坡,南界阿蘇臺札噶,西界嫩江,北界布臺格爾池、烏柯爾鄂克達。廣一百七十里,袤二百四十里。北極高四十七度十五分。京師偏東七度十分。貢道由山海關。東:富峪蒙名巴雅鼐。東南:哈他伯齊坡。西南:和幾蒙克坡。東北:阿拉克阿幾爾漢坡。北:疊翠巖蒙名磨朵圖。西:嫩江,自黑龍江境南流入,西與札賚特分界,又南入郭爾羅斯後旗。東:烏葉爾河,源出黑龍江境,西南入,逕黨納坡,又南入郭爾羅斯後旗。

郭爾羅斯部二旗:附科爾沁左翼。在喜峰口東北。本契丹地。遼置泰州昌德軍,屬上京。金大定中廢,移州於長春縣,以故地為金安縣,隸之。元為遼王分地。明為科爾沁所據,後分與其弟烏巴什,是為郭爾羅斯。天聰七年,臺吉古木及布木巴來降,後封古木弟桑阿爾賽輔國公,世襲掌前旗,布木巴鎮國公,世襲掌後旗。其所部東界盛京永吉州,南界盛京邊墻,西及北界科爾沁。貢道由山海關。郭爾羅斯前旗札薩克駐固爾班察漢,在喜峰口東北一千四百八十七里。西南距京師一千八百九十七里。牧地當嫩江與松花江合流之西岸,在吉林伊通邊門外長春之西。東界烏拉河,南界柳條邊,西界博果圖,北界拜格臺和碩。廣二百三十里,袤四百里。北極高四十五度三十分。京師偏東八度十分。其山:西南曰巴顏硃爾克山,一名牛心山。東南,衣馬圖峰。北,他奔拖羅海坡。東北,巴吉岱坡。西,巴顏布他岡。東:混同江,土名吉林江,自奉天永吉州西北入,東北流,會嫩江。又東折入後旗地,東北流,會黑龍江,東入海。南:一禿河,源出奉天永吉州境,北流出邊,逕龍安城,又東北流,會伊爾們河,入混同江。東南:伊爾們河,源出永吉州境,北流出邊,受南來之烏蘇土烏海河,會一禿河,入混同江。郭爾羅斯後旗札薩克駐榛子嶺,在喜峰口東北一千五百七十里。西南至京師一千九百八十里。牧地當混同江北岸、嫩江東岸。東界阿勒克巴魯,南界嫩江,西界嫩江,北界烏魯勒圖。廣二百二十里,袤二百六十里。北極高四十六度十分。京師偏東八度二十分。其山:東曰常峽坡。東南:阿祿布克色坡、阿拉克碧老坡。西北:拜拉喇齊坡。東北:布拉克臺坡。西:烏葉爾河,自杜爾伯特入,分流為西訥河,西南流,同入嫩江。嫩江分流為牛川,蒙名烏庫爾,東南流,會烏葉爾河。

喀喇沁部二旗,新增一旗曰中旗:在喜峰口東北。秦、漢,遼西郡境。唐,饒樂都督府,後入契丹。遼置中京大定府。金,北京。元,大寧路。明洪武中,封子權寧王。永樂初,盡以大寧地賜朵顏、泰寧、福餘三衛。朵顏時陰附韃靼為邊患,後為察哈爾所滅,以其地予其塔布囊,是為喀喇沁。天聰七年,部長蘇布地率昆弟塞冷等來降,後封蘇布地之子古魯思起布為貝子,主右翼,塞冷為鎮國公,主左翼,並世襲。康熙中,增設一旗,授喀寧阿一等塔布囊,加公銜,襲封。所部東界土默特及敖漢,西界察哈爾正藍旗牧場,南界盛京邊墻外,北界翁牛特。廣五百里,袤四百五十里。貢道由喜峰口。喀喇沁右翼札薩克駐錫伯河北,在喜峰口北三百九十里。西南距京師八百里。牧地在圍場東,跨老哈河。東界鄂博噶圖,南界霍落蘇泰,西界察罕鄂博,北界霍爾哈嶺。廣三百里,袤二百八十里。北極高四十一度五十分。京師偏東二度四十分。其山:東曰和爾坤都倫喀喇山、烏爾圖納蘇圖波羅山、伊瑪岱山、七金山蒙名和爾博勒津、大紅螺山蒙名巴顏烏蘭。東南,大斧山蒙名喀喇和邵、柞山蒙名巴圖插漢、大青山蒙名巴顏喀喇。南,和爾和克阿惠山、常吉爾岱山、拉克拉哈爾山。西南,昆都倫喀喇山。西,昆都爾圖山。北,鄂通臺和羅圖山、綽和羅漠林嶺。南:老河,蒙名老哈,源出明安山,東北流,會諸小水,逕敖漢北、翁牛特左翼南,又經奈曼、喀爾喀二部,納奇札帶河,北流與潢河會。南,虎查河、和爾和克河、上神水河、呼魯蘇臺河、巴爾漢河、納林昆都倫河,東,落馬河,同入老河。西:木睿喀喇克沁河,源出卯金插漢拖羅海山,西北流,會布墩河,又西流,合宜孫河,南入灤河。淘金圖河,西南流,會烏喇林河,亦南入灤河。東南:土河,蒙名土爾根,源出西默特山,東南流入土默特右翼。西:卯金溫泉有二,一出卯金河東,西流會卯金河,一出卯金河西,東南流,亦會卯金河。卯金河源出卯金嶺,西南流,會熱河。賽因阿拉善溫泉,即熱河之源也。喀喇沁左翼札薩克駐牛心山,在喜峰口東北三百五十里。西南距京師七百六十里。牧地當傲木倫河源。東界烏蘭哈達圖和碩,南界寧遠邊墻,西界烏里蘇太梁,北界唐奇鼐陀羅海。廣二百三里,袤一百七十里。北極高四十一度十分。京師偏東三度四十分。其山:東北曰峨倫和歌諾忒山。東,柏樹山蒙名邁拉蘇臺喀喇。東南,阿布察山、噶海圖博羅山。南,翁噶爾圖山、拖和喀喇山、他奔拖羅海圖山。西南,樺山蒙名韋蘇圖、柞子嶺、貴石嶺。西,佗蘇圖喀喇山。西北,察爾契山、庫葛會山。西北:青龍河,蒙名顧沁河,源出長吉爾岱山,西南流,會湯圖河,逕額倫碧老嶺入邊城,逕永平府,北入灤河。南:額類河,源出額類嶺,南流會寬河,至奉天寧遠州西入邊,為黑水河,入六州河。北:大凌河蒙名敖木倫河,源出尾蘇圖山,東流,至西喇哈達圖山東北,折入土默特右翼,又東南入邊。西:和爾圖河,源出陀蘇圖喀喇山,東流,會敖木倫河。森幾河、賽因臺河、石塔河、神水河、清水河皆從之。喀喇沁中旗在左右翼二旗界內。札薩克駐珠布格朗圖巴顏哈喇山。牧地跨老哈河源。東與北若西皆右翼,南左翼,東界博勒多克山,北界岳羅梁,西界霍爾果克。北極高四十一度三十分。京師偏東二度。其山:東曰博勒多克山。山南拉克篤爾山。

土默特部二旗,左翼附一旗:在喜峰口東北。古孤竹國。漢,遼西郡治柳城縣地。燕,慕容皝建都,改龍城縣。元魏為營州治。隋復置柳城縣。唐為營州都督府治。遼置興中府。元,大寧路興中州。明以內附部長為三衛,自錦、義歷廣寧至遼河曰泰寧衛,後為蒙古土默特所據。天聰三年,臺吉鄂木布、塔布囊善巴來降,後封善巴貝勒,主左翼,鄂木布貝子,主右翼,世襲。所部東界養息牧牧場,西界喀喇沁右翼,南界盛京邊墻,北界喀爾喀左翼及敖漢。廣四百六十里,袤三百一十里。貢道由喜峰口。乾隆中停貢。土默特左翼札薩克駐哈特哈山,左喜峰口東北八百二十里。西南距京師一千二百三十里。牧地當錫哷圖庫倫喇嘛游牧之南,養息牧牧場之西。東界岳洋河,南界什巴古圖山,西界巴噶塔布桑,北界當道斯河。廣一百六十里,袤一百三十里。北極高四十二度十分。京師偏東四度三十分。其山:南,達離山蒙名刻特俄爾多和碩。西,膜衣達摩山、青金山蒙名博羅蒙魁。北,淘金圖山、伊克翁山、巴漢翁山。北:庫昆河,或作呼渾河,自喀爾喀左翼入,會烏訥蘇臺河、阿哈里河,入養息牧河。西北:羖羊河,蒙名衣馬圖河,源出彌勒山,西南流,逕青山,又南會馬鞍河,入邊,逕義州東北為細河,會清河入大凌河。土默特右翼札薩克駐巴顏和碩,亦名大華山,在喜峰口東北五百九十里。西南距京師一千里。牧地在九關臺、新臺邊門外,跨鄂木倫河。東界訥哷遜山,南界魏平山,西界鄂朋圖山,北界什喇陀羅海。廣二百九十里,袤一百八十里。北極高四十一度四十分。京師偏東四度二十分。其山:東曰衣達摩山、五鳳山、蓮花山。東南,喀喇七靈圖山。南,神應山蒙名蘇巴爾噶圖。西南,土祿克臺山、卓常吉爾山。西,釜山蒙名喀喇拖和多、青山蒙名博羅和邵、鳳凰山蒙名兆馨喀喇。西北,布祿爾喀喇山。北,回賀爾山。東北,赤山蒙名五藍。西:大凌河,自喀喇沁左翼入,東流,逕古興中城,南折,東南流,納柳河,入邊。土爾根河,一自喀喇沁右翼東流入,一自奈曼南流入,均南流入大凌河。北:卓索河,源出回賀爾山。老寨河、土河、柞河,東北楊河,皆南流入土爾根河。西:小凌河,蒙名明安河,源出明安喀喇山,東北流,會木壘河、哈柳圖河,入邊,會烏馨河入海。土默特左翼附旗初,喀爾喀臺吉巴勒布冰圖,康熙元年自杭愛山率屬來歸,詔附左翼札薩克達爾漢貝勒皁哩克圖牧。四年,封多羅貝勒。牧地在錫哷圖庫倫喇嘛游牧之西。東界霍濟勒河,南界庫昆河,西界布圖昆地,北界愛篤罕山。以上統盟於卓索圖。盟地在土默特右翼境內。

敖漢部一旗:札薩克駐固班圖勒噶山,在喜峰口東北六百里。西南距京師一千一十里。本古鮮卑地。隋,契丹地。唐屬營州都督府。遼、金為興中府北境。元為遼王分地。明為喀爾喀所據,後分與其弟,號曰敖漢,役屬於察哈爾。天聰元年,貝勒塞臣卓禮克圖舉部來降,後封郡王,世襲。牧地跨老哈河。東界奈曼,南界土默特,西界喀喇沁,北界翁牛特。廣一百六十里,袤二百八十里。北極高四十三度十五分。京師偏東四度。貢道喜峰口。其山:東,哈達圖拖羅海山。東南,白石山蒙名插漢齊老臺、富泉山。南,二天山蒙名騰格里、小蟠羊山蒙名巴漢衣馬圖。西南,韋布爾漢山、庫爾奇勒山。西,森幾拖羅海山、棗山蒙名齊巴噶。西北,巴雅海山。北,寬山蒙名鄂達博羅、兆虎圖插漢拖羅海山。東北,庫爾奇勒峰、梨穀蒙名阿里馬圖。其水:北,老河,蒙名老哈,自喀喇沁右翼入,東北流,逕噶察喀喇山,又東入翁牛特。西南,落馬河,蒙名百爾格,自喀喇沁右翼入,東北流,入老河。南,杜母達納林河,源出天山,北流入七老臺池。南,衣馬圖泉,下流入沙池。東北,昆都倫喀喇烏素泉,南流入老河。

奈曼部一旗:札薩克駐章武臺,在喜峰口東北七百里。西南距京師一千一百十里。古鮮卑地。隋,契丹地。唐屬營州都督府。遼、金為興中府北境。明為喀爾喀所據,分與親弟,號曰奈曼。天聰元年,酋長袞楚克巴圖魯為察哈爾所侵,來降,後封郡王,世襲。牧地當潢河、老哈河合流之南岸。東界科爾沁,南界土默特,西界敖漢,北界翁牛特。廣九十五里,袤二百二十里。北極高四十三度十五分。京師偏東五度。貢道由喜峰口。其山:南曰馬尼喀喇山、五鳳山蒙名他奔拖羅海。西,呼原博塔蘇爾海岡。東南,大黑山蒙名巴顏喀喇。東北,哈納岡。北:潢河自敖漢入,合老哈河,東北流,入喀爾喀左翼。南:圖爾根河,亦名土河,源出塔本陀羅海山,南入土默特右翼。西:固爾班和爾圖泉,東南流,會圖爾根河。

巴林部二旗:在古北口東北七百二十里。南距京師九百六十里。遼,上京臨潢府。金,大定後,並屬北京路。元屬廣寧路,為魯王分地。明初為廣寧衛,後屬烏梁海北境,後為順義王諳達五子巴林臺吉所據,役屬於察哈爾。天命十一年,以巴林叛盟,征之,戮其貝勒。天聰二年,為察哈爾所破,貝勒塞特哩、臺吉滿硃習禮來歸,改封塞特哩之子塞布騰郡王,主右翼,滿硃習禮為貝子,主左翼,襲封。右翼、左翼同游牧地,當潢河北岸。東界阿嚕科爾沁,南界翁牛特左翼,西界克什克騰,北界烏珠穆沁。廣二百五十里,袤二百三十三里。北極高四十三度三十六分。京師偏東二度十四分。貢道由獨石口。其山:東有鄂拜山、石雞山蒙名伊韜圖。南,巴爾達木哈喇山、勃突山蒙名巴爾當。遼五代祖勃突生於此山,因以名焉。西,碧柳圖山、清金山。東南,特墨車戶山。東北,僧機圖。南:潢河,自克什克騰入,東流,會黑河,入翁牛特左翼。黑河即古慶州黑水。東北:布雅鼐河,源出僧機圖山,東南流,會烏爾圖綽農河,東入阿嚕科爾沁,注于達布蘇圖池。有哈爾達蘇臺河,西自克什克騰來注之,東南流入潢河。巴林左翼札薩克駐阿察圖陀羅海。巴林右翼札薩克駐托★山。

札魯特部二旗:在喜峰口東北。漢,遼東郡北境。唐屬營州都督府。遼,上京道地。金屬北京路。元屬上都路。明為蒙古札魯特所據,後屬喀爾喀。清初與札魯特內齊汗結親。後貝勒色本引兵助明,太祖擊擒之,旋釋歸。天聰二年,色本等為察哈爾所侵,與內齊舉部來降,封內齊貝勒,主左翼,色本貝勒,主右翼,世襲。左、右同游牧地,當哈古勒河、阿魯昆都倫河之源。東界科爾沁,南界喀爾喀左翼,西界阿嚕科爾沁,北界烏珠穆沁。廣一百二十五里,袤四百六十里。北極高四十五度三十分。京師偏東三度。貢道由喜峰口。札魯特左翼札薩克駐齊齊靈花陀羅海山北,在喜峰口東北一千一百里。西南距京師一千五百一十里。牧地當哈古勒河、阿魯坤都倫河之源,達布蘇圖河於此流入於沙。其山:北曰野鵲山蒙名巴顏喀喇、巴噶查克朵爾山。東北,屈劣山蒙名布敦花拖羅海。西南,噶海岡、車爾百湖岡。西,獨石岡。東南,貴勒蘇臺。其水:南,潢河自阿嚕科爾沁入,逕車爾百湖岡,東流,入科爾沁,蒙名西拉木倫河,即遼河之西源也。北,沙河、阿祿昆都侖河,東流入科爾沁。額百里昆都侖河,源出愁思嶺,東流,亦入科爾沁。札魯特右翼札薩克駐圖爾山南,在喜峰口東北一千二百里。西南距京師一千六百四十里。牧地同。其山:南曰嵬石山蒙名札拉克。西南,托幾山。西,小白雲山蒙名巴哈插漢拖羅海山。西北,色爾奔山、幾祿克山、大青羊山蒙名伊克特黑。北,花山、蛇山、小青羊山蒙名巴漢特黑。其水:西北曰魁屯河,一名陰涼河,源出賀爾戈圖五藍山,東南流,會天河。北,阿里雅河,源出大赤峰,西流逕花山,入阿嚕科爾沁。他魯河源出大青羊山,南流,合阿里雅河。

翁牛特部二旗:在古北口東北。唐,饒樂都督府地。遼置饒州匡義軍節度,屬上京道。金,北京路地。元為魯王分地。明初以烏梁海置衛為外籓,後自稱翁牛特,本服屬於阿嚕科爾沁。天聰七年,濟農索音、貝勒東率所部來降,後封索音郡王,主右翼,東貝勒,主左翼,並襲封。所部東界阿嚕科爾沁,南界喀喇沁及敖漢,西界熱河禁地,北界巴林及克什克騰。廣三百里,袤一百六十里。北極高四十三度十分。京師偏東二度五十分。貢道由喜峰口。翁牛特左翼札薩克駐札喇峰西綽克溫都爾,在古北口東北六百八十里。西南距京師九百二十里。牧地介潢河、老哈河之間。東界阿嚕科爾沁,南界敖漢,西界克什克騰,北界巴林。廣三百里,袤九十里。北極高四十三度十分。京師偏東二度五十分。其山:東曰小華山蒙名巴哈哈爾占、大松山蒙名伊克納喇蘇臺。南,兆呼圖插漢拖羅海山。西,勃突山蒙名布墩、吐頹山蒙名巴爾哈岱。西北,古爾板土爾哈山。東南,阿爾齊土插漢岡。東北:兔麛山。其水:北曰潢河,自克什克騰入,東流逕巴林,又東流入境,又東北流,老河自敖漢來會,逕札魯特南、喀爾喀北,入科爾沁。翁牛特右翼札薩克駐哈齊特呼朗,在古北口外五百二十里。西南距京師七百六十里。牧地在熱河圍場東北,老哈河南岸。東界敖漢,南界喀喇沁右翼,西界圍場,北界克什克騰。廣二百四十里,袤一百里。北極高四十三度十分。京師偏東二度五十分。其山:東曰烏蘭布通山、夏屋山蒙名伊克布庫圖爾。東南,花和博圖山、阿爾渾查克插漢拖羅海山、棗山蒙名齊巴哈。南,古爾板拖羅海山、遮蓋山蒙名阿惠喀喇。西南,巴倫桑噶蘇臺山、大黑山、額類蘇圖山。西,徒古爾喀喇山、博多克圖山。西北,巴顏布爾噶蘇臺山、黃山蒙名洪戈爾峨博。北,馬鞍山蒙名西喇得伯僧、海他漢山。其水:南曰錫伯河,自喀喇沁北流入境,東北流,會麞河入老河。麞河,蒙名西爾哈,亦自喀喇沁流入境,東北流,逕巴顏喀喇山,東北會英金河,又東逕五藍峰北入老河。西北,烏拉岱河,源出楊木嶺,南流,經博多克圖山,折東北流,會麞河。西,巴倫撒拉河,源出葛爾齊老東北,東南流,逕巴爾圖山,折東北流,會烏拉岱河。西,車爾伯呼河,源出奴克都呼爾山,東南流,會麞河。英金河,源出嘏蟆嶺,東南流,亦會麞河,又東入老河。奴古臺河、珠爾河、拜拉河,皆與英金河會。北,卓索河,源出海他漢山,東流會麞河,入老河。

阿嚕科爾沁部一旗:札薩克駐琿圖爾山東托果木臺,在古北口東北千一百里。西南距京師一千三百四十里。遼,臨潢府地。金,大定府北境。元為遼王分地。明初於烏梁海地置衛為外籓,後自號阿嚕科爾沁。天聰六年,部長達賴為察哈爾所侵,率其子穆章來降,後封穆章貝勒,世襲,掌旗。牧地哈奇爾河、傲木倫河於此合流為達布蘇圖河。東界巴彥塔拉,南界翁牛特左翼什喇木蘭,西界蘇布山,北界烏蘭嶺。廣三百三十里,袤四百二十里。北極高四十度三十分。京師偏東三度五十分。貢道由喜峰口。其山:東北曰渾圖山。東,伊克陀惠山。東南,峨博圖山。南,庫格圖山、連山蒙名賀爾博拖羅海。西北,棗山蒙名齊巴哈圖。西南,巴漢阿拍札哈山、伊克阿拍札哈山。西,珍珠山蒙名蘇布、樂游山蒙名得訥格爾。南:潢河,蒙名西喇木倫河,自巴林入,逕他木虎噶察岡,入札魯特。西南:烏爾圖綽農河,自巴林入,逕刻勒峰,東南流,會哈喜爾河。又西北有和戈圖綽農河,源出西喇溫都爾山,南流,會烏爾圖綽農河,入哈喜爾河。哈喜爾河源出薩碧爾漢山南流逕庫格圖山,折而東流入札魯特。東北:阿里雅河,自札魯特右翼入,西南流,會哈喜爾河。西北:枯爾圖河,源出白石山,西流入巴林,會烏爾圖綽農河。尹札漢河,北流入烏珠穆沁。

克什克騰部一旗:札薩克駐吉拉巴斯峰,在古北口東北五百七十里。南距京師八百十里。遼,上京道地。金屬北京路。元屬上都路及應昌路地。明為蒙古所據。天聰八年,滅察哈爾,克什克騰索諾木戴青來歸,授掌旗一等臺吉,世襲。牧地在圍場北,當潢河之源。東界畢勒固圖和嶺,南界布圖坤,西界克勒特格伊場,北界烏蘇池。廣三百三十四里,袤三百五十七里。北極高四十三度。京師偏東一度。貢道由獨石口。其山:東曰蜘蛛山蒙名阿爾札、高澱山蒙名音納哈喀喇。東南,寧楚渾杜爾賓山。西南,恩都爾花山。西,烏素圖杜爾賓山、大黑山蒙名巴顏喀喇。西北,巴漢衣色裏山、博爾多克山。北,黃山蒙名巴顏洪戈爾、木葉山蒙名幾幾恩都爾。東北,馬尾山蒙名叟幾。西:潢河,大遼水西一源也,蒙名西喇木倫,源出百爾赫賀爾洪,東北流,會諸水,逕旗北,又東流入巴林。又東,逕阿嚕科爾沁南、翁牛特北,又東北流,會老河,逕札魯特南、喀爾喀北,折東南流,逕科爾沁左翼,又南會大遼水,入邊城,是為遼河。西:薩里克河,源出烏素圖杜爾賓山,東北流,入潢河。西北:衣爾都黑河,源出烏素圖杜爾賓山,西流,會伊黑庫窩圖河,東北流,入潢河。西北:格類河,源出興安山東,南流,會穆名河入潢河。東北:釜河蒙名陀惠,源出嶽碧爾山,北流入黑河。西南:高涼河,蒙名拜查,源出拜查泊,東北流,入潢河。東北:阿爾達圖河,源出興安山,西北流入烏珠穆沁,北流會葫蘆谷爾河。西北:捕魚兒海,蒙名達爾,公姑、野豬等四河流入其中,周數十里。

喀爾喀左翼部一旗:札薩克駐察罕和碩圖,在喜峰口東北八百四十里。西南距京師一千二百五十里。古鮮卑地。唐屬營州都督府。遼,上京道南境。金屬北京路。明為喀爾喀所據,後屬於西路札薩克圖汗。元太祖十六世孫格埒森札居杭愛山,始號喀爾喀,其孫巴延達喇為西路札薩克圖汗之祖,即今外蒙古四部之一。清初酋長古木布伊爾登與札薩克圖汗來降,後封貝勒,世襲,主左翼。牧地當養息牧河源。東界科爾沁,南界土默特左翼,西界奈曼,北界札魯特。廣一百二十五里,袤二百三十里。北極高四十三度四十二分。京師偏東五度二十七分。貢道由喜峰口。其山:東曰喀海拖羅海山。南,達祿拖羅海山、巴漢哈伯他海山。西南,五灰山蒙名烏尼蘇臺、大黑山蒙名巴顏喀喇、青山蒙名博羅惠博羅溫都爾,與奈曼東南接界。東南,他木虎岡。北:潢河,自翁牛特流入,又東流入科爾沁。西北:老河,蒙名老哈,自奈曼入,東北流,會潢河。東南:養息牧河,源出旗南,東北流,逕喀海拖羅海山,又東南,會庫昆河,逕養息牧牧廠,東流入彰武臺邊門,西至廣寧,又東南流入遼河。南:庫昆河,源出五灰山,東流入土默特。以上統盟於昭烏達。盟地在翁牛特左翼境內。

烏珠穆沁部二旗:在古北口東北。遼,上京道北境。金屬北京路。元屬上都路。明為蒙古所據,自號烏珠穆沁,察哈爾汗族也。林丹汗暴虐,貝勒多爾濟偕塞楞往依喀爾喀。天聰八年來歸,封多爾濟親王,主右翼,塞楞貝勒,主左翼,並世襲。其地東界索倫,西界浩齊特,南界巴林,北界瀚海。廣三百六十里,袤四百二十五里。貢道由獨石口。烏珠穆沁右翼札薩克駐巴克蘇爾哈臺山,在古北口東北九百二十三里。南距京師一千一百六十三里。牧地有音札哈河流入於沙,有胡蘆古爾河,瀦於阿達克諾爾。東界左翼,南界巴林,西界浩齊特左翼,北界車臣汗中右旗。廣三百六十里,袤二百一十里。北極高四十四度四十五分。京師偏東一度十分。其山:東曰瑞鹿山蒙名布虎圖。西,大小黃鷹山、黑山蒙名喀喇圖。西北,雙山蒙名賀嶽爾俄得、烏里雅臺山。東北,賽音恩都爾山。水則東南:賀爾洪河,源出噶木爾站,西流入蘆水。禿河一名葫蘆古爾,源出古什克騰東北,名阿爾達圖河,西北流入右翼,為葫蘆古爾河,又北流入阿達可池。烏珠穆沁左翼札薩克駐鄂爾虎河之側奎蘇陀羅海,在古北口東北一千一百六十里。南距京師一千四百里。牧地當索嶽爾濟山之西。有鄂爾虎河,繞其游牧,匯於和里圖諾爾。東界霍尼雅爾哈賴圖,南界庫冽圖,西界達賴蘇圖,北界額裡引什里。廣二百五十六里,袤二百一十五里。北極高四十六度二十分。京師偏東二度二十分。其山:東南曰哈爾站五藍峰。北,色爾蚌峰。水則東北:色野爾齊河,源出哈老圖泊,西南入蘆水。東南:音札哈河,自阿嚕科爾沁入,西北亦入蘆水。

阿巴哈納爾部二旗:在張家口東北。漢,上谷郡北境。晉屬元魏。隋、唐為突厥地。遼為上京道西境。金為北京路西北境。元屬上都路。明為蒙古所據,號所部曰阿巴哈納爾,本役屬於喀爾喀車臣汗。崇德間,臺吉塞冷、董夷思拉布來降,後封董夷思拉布貝子,主左翼,塞冷貝勒,主右翼,並襲封。所部東界浩齊特,西界阿巴噶右翼,南界正藍旗察哈爾,北界瀚海。廣百八十里,袤四百三十六里。貢道:右翼由張家口,左翼由獨石口。阿巴哈納爾右翼札薩克駐永安山,在張家口東北六百四十里。東南距京師一千五十里。牧地有達里岡愛諾爾。東界希爾當山,南界博羅溫都爾岡,西界哈喇堂,北界華陀羅海山。廣六十里,袤三百一十里。北極高四十三度三十分。京師偏東二十分。其山:南曰巴爾達木山。東,特爾墨山。北,哈納峰、僧機圖山。西,賀爾賀山。東南,大熊山蒙名巴賴都爾。東北,床山蒙名席勒。西北,雙山蒙名和嶽爾察罕陀羅海山。其水:南曰韭河,蒙名郭和蘇臺,自正藍旗察哈爾入,逕博羅岡,西北入阿巴噶。南,息雞澱,蒙名哈雅。東,葦澱,蒙名呼魯蘇臺布祿都。西南,褒勒泊。西北,袞布祿都泊。北,葛都爾庫泉、和幾葛爾泉。阿巴哈納爾左翼札薩克駐烏爾呼拖羅海山,在獨石口東北五百八十里。東南距京師一千一百里。牧地同上。東與北皆界浩齊特,南界阿巴噶,西界右翼旗。廣一百二十里,袤三百一十八里。北極高四十三度五十三分。京師偏東二十八分。其山:西曰色爾騰洪戈爾山,一名黃山。西北,布爾漢山、觸寶山、覆舟山蒙名呼里翁戈春。其水:北有黑勒泊。西北,達藍圖裏泉。

浩齊特部二旗:在獨石口東北。遼,上京道西境。金屬北京路。元屬上都路。明為蒙古所據。察哈爾汗族也。林丹汗暴虐,其貝勒博羅特、臺吉噶爾瑪色旺往依喀爾喀。天聰八年來降,以博羅特主左翼,噶爾瑪色旺主右翼,並郡王,襲封。所部東北界烏珠穆沁,南界克什克騰,西界阿巴噶。廣一百七十里,袤三百七十五里。貢道由獨石口。浩齊特右翼札薩克駐特古力克呼圖克湖欽,在獨石口東北六百九十里。東南距京師一千一百九十里。牧地當錫林河下游,北瀦為達母鄂謨。東界布爾勒吉山,南界札哈蘇臺池,西界布爾色克陀羅海,北界哈魯勒陀羅海。廣七十五里,袤三百七十五里。北極高四十四度。京師偏東三分。其山,右翼主山:東南,古爾板賀老圖山、古爾板俄得山。東,伊爾伯都山。南,布當圖山。北,胡呂山蒙名阿拉忒。西北,阿拍達蘭圖山。水則東:白濼蒙名柴達木。東南:大魚濼。南:松子泉蒙名和爾多。東北:察得爾泉。西北:昆都侖泉、布哈泉。浩齊特左翼札薩克駐烏默黑塞里,在獨石口東北六百八十五里。東南距京師一千一百八十五里。牧地濱大小吉里河。東界額爾起納克登,南界小吉里河源,北界奇塔特哈覃陀羅海,西界瑪齊布勒克烏蘭哈達。廣九十五里,袤三百一十里。北極高四十四度五分。京師偏東四分。其山:東南曰薩爾巴山。西北,野狐山蒙名烏納格忒。北,蘇門峰。西北,五藍峰。水則東南:天鵝濼、庫魯爾圖泉。北,沖戈爾泊。西南,阿祿布裏都泊。西北,賀老圖泉。

阿巴噶部二旗:在張家口東北。漢,上谷郡北境。晉為拓跋氏地。隋、唐為突厥地。遼,上京道西境。金屬北京路。元屬上都路。明為蒙古所據,號所部曰阿巴噶。本役屬於察哈爾。林丹汗暴虐,濟農都思噶爾、貝勒多爾濟往依喀爾喀。天聰九年來降,後以多爾濟主右翼,都思噶爾主左翼,並封郡王,世襲。所部東界阿巴哈納爾,西界蘇尼特,南界正藍旗察哈爾,北界瀚海。廣二百里,袤三百十里。右翼貢道由張家口。左翼貢道由獨石口。阿巴噶左翼札薩克駐巴顏額倫,在獨石口東北五百五十里。南距京師一千七十里。牧地環錫林河。東界巴爾啟臺之哈喇鄂博噶圖,南界烏蘇圖土魯格池,西界什爾登山,北界哈布塔噶陀羅海。廣一百二十里,袤一百八十里。北極高四十三度五十分。其山:東南曰哈爾塔爾山、喀喇得伯僧山、邵龍山。西南,武歷山蒙名哲爾吉倫、察里爾圖山。南,哈斯胡雅斯坡。其水:東南,陰涼河,蒙名魁屯,源出卓索圖站,流入旗界。東南,鶴壘斗勒泊。北,金河泊。西南,西喇布裏都泊。阿巴噶右翼札薩克駐科布爾泉,在張家口東北五百九十里。南距京師一千里。牧地有庫爾察罕諾爾,為固爾班烏斯克河所瀦。東界哈畢喇噶泉,南界伊柯什噶,西界庫庫勒,北界華陀博。廣八十里,袤三百一十里。北極高四十三度三十分。京師偏西二十分。其山:東南曰色幾庫山。南,硃爾哈臺拖羅海山。西北,馬尼圖拖羅海山、白石山蒙名插漢七老圖。北,阿拍濟哈山、霸特山蒙名克色克拖羅海、羖羊山蒙名特克拖羅海。其水:東南,韭河,蒙名郭和蘇臺,自阿巴哈納爾入,逕色幾庫山,西流入白海子。南,噶爾圖泊。東南,渾圖泊。西南,呼爾泊、鴛鴦濼蒙名昂吉爾圖。東,硃爾克額勒蘇圖泉。北,赤泉。東北,哈碧爾漢泉。

蘇尼特部二旗:在張家口北。漢,上谷、代二郡北境。後漢,烏桓、鮮卑地。隋、唐為突厥地。遼置撫州。金因之,屬西京路。元為興和路地。明為蘇尼特所據,察哈爾汗族也。天聰九年,其濟農叟塞、貝勒滕吉思來朝,後封叟塞郡王,主右翼,滕吉思弟滕吉泰郡王,主左翼,襲封。東界阿巴噶右翼,西界四子部落,南界察哈爾正藍旗牧廠,北界瀚海。廣四百六里,袤五百八十里。貢道由張家口。蘇尼特右翼札薩克駐薩敏錫勒山,在張家口北五百五十里。東南距京師九百六十里。牧地在瀚海北。東界額爾蘇霍吉爾,南界烏科爾齊老,西界特莫格圖,北界吉魯格。廣二百四十六里,袤二百八十里。北極高四十三度二分。京師偏西二度一分。其山:南曰布爾色克山、福山蒙名克什克、和爾和山。西南,烏克爾硃爾克山、俄爾綽克山。西,德林山。東北,巴輪明安拖羅海山、嵬名山蒙名札喇。東南,努倫坡。其水:西南曰長水,蒙名烏爾圖,源出和爾和山。東南,占木土鹽泊。南,西喇布祿泊、滾泊。電局在西蘇尼特王府東北七十里。蘇尼特左翼札薩克駐和林圖察伯臺岡,在張家口北五百七十里。東南距京師九百八十里。牧地當固爾班烏斯克河。東界庫庫勒山,南界察罕池,西界色柯爾山,北界阿爾噶里山。廣一百六十里,袤三百里。北極高四十三度三分。京師偏東一度二分。其山:東南曰巴顏特克山一名羖䍽山。西北,喀爾他和邵山。北,博錐拖羅海山、拜音拖羅海山一名祥古山。其水:東南曰努克黑忒水,一名兔園水,自察哈爾正藍旗入,逕福山北流入呼爾泊。西,古爾板馬潭泊。東南,呼爾泊。西南,黑山濼。以上統盟於錫林郭勒。盟地在阿巴噶左翼、阿巴哈納爾左翼兩旗界內。

四子部落一旗:札薩克駐烏蘭額爾濟坡,在張家口西北五百五十里。東南距京師九百六十里。漢,雁門、定襄二郡北境。晉為拓跋氏地。唐為振武軍地。遼為豐州地,屬西京道。金屬西京路。元屬大同路。明為阿祿喀爾喀所據,分與四子,號四子部。天聰八年,貝勒鄂木布來朝,後敘功封郡王,襲封。牧地有錫喇察漢諾爾,錫喇木倫河瀦之。東北界蘇尼特,西界歸化城土默特,南界鑲紅旗察哈爾。廣二百三十五里,袤二百四十里。北極高四十二度四十一分。京師偏西四度二十二分。貢道由張家口。其山:東曰博濟蘇克山。東南,陰山。南,白爾白狼山一名新婦山、爾多斯山。西南,納札海山、阿祿蘇門峰。西北,獨牛山蒙名烏克爾圖祿。東北:陽山蒙名北蘭。西,富峪蒙名巴顏鄂坡蘇。西北:黃水河,蒙名西喇木倫,自喀爾喀右翼入,東北流,出喀倫邊。西:希巴爾臺泉、雅孫哈柏濟爾泉。南:噶爾哈圖泉。西南:德本得泉、青堿泉蒙名博羅虎濟爾。西北:白石泉蒙名插漢齊老。

茂明安部一旗:札薩克駐徹特塞里,在張家口西北八百里。東南距京師二千二百四十里。漢,五原郡地。元魏,懷朔鎮地。唐,振武軍地。遼,東勝州地,屬西京道。金因之。元屬大同路。明初設衛戍守,蒙古據之,號曰茂明安。天聰八年,舉部來降。康熙三年,授僧格掌旗一等臺吉,襲封。牧地當愛布哈河源。東界喀爾喀,西界烏喇特,南界歸化城土默特,北界瀚海。廣百里,袤一百九十里。北極高四十一度十五分。京師偏西六度九分。貢道由張家口。其山:東曰伊克哈達圖山。東南,和嶽爾白爾克山、插漢峨博山。西南,哈拉海圖山、官山。西,羖羊山蒙名喀喇特克。西北,齊齊爾哈插漢七老山。東北,古爾板喀喇山、郭岳惠插漢七老山。南:昆都倫河,源出和嶽爾白爾克山,西流,逕官山,入烏喇特。東北:布祿爾托海河,源出伊克哈達圖山,北流,會愛畢哈河。愛畢哈河源出刻勒峰,東流,逕古爾板喀喇山,入喀爾喀。南:拜星圖泉,源出哈拉海圖山,西南流,會昆都倫河。

烏喇特部三旗:三札薩克同駐哈達瑪爾,在歸化城西三百六十里。東南距京師一千五百二十里。漢,五原郡。元魏,懷朔鎮。唐,中西受降城地。遼置雲內州,屬西京道。金因之。元為大同路。明為瓦喇所據。天聰七年,瓦喇臺吉鄂板達爾漢來朝,率圖巴額爾赫及塞泠伊爾登二旗歸附。順治五年,敘從征功,以圖巴掌中旗,鄂木布子鄂班掌前旗,色棱子巴克巴海掌後旗,同封鎮國公,授札薩克,世襲。前、中、後三旗同牧地,當河套北岸噶札爾山南。東界茂明安,南界鄂爾多斯左翼前旗,西界鄂爾多斯右翼後旗,北界喀爾喀右翼。廣二百一十五里,袤三百里。北極高四十度五十二分。京師偏西六度三十分。貢道由殺虎口。其山:東曰昆都倫山一名居延山、狼山蒙名綽農拖羅海山。西,木納山。北,河套山、雪山蒙名叉蘇臺。東北,敖西喜山、白石山蒙名插漢七老圖。西北,大青山蒙名漠喀喇、烏蘭拜星山一名赤城山。西南,席勒山一名床山。東南,漠惠圖坡。南:黃河,自鄂爾多斯西北境入,東流逕旗南,又東折南入歸化城土默特。西北:柳河,蒙名布爾哈圖,源出陽山東平地,西南流,會敖泉入黃河。哈柳圖河,源出席勒山北,南流會席勒河,逕馬神山,又西南折入黃河。北:東哈柳圖河,源出麥垛山,西南逕東西德爾山南、拜星圖北,為席漢河,又西南入黃河。烏爾圖河,源出雪山,西南流入黃河。帷山河,源出帷山,西南會黑河。黑河,蒙名喀喇木倫,自茂明安所屬地流入,西南流,逕帷山入黃河。齊齊爾哈納河,自茂明安入,西南流,逕白石山,亦會黑河。蘇爾哲河,源出雪山,西流會舍忒河。舍忒河源出敖西喜山,西流逕大青山入黃河。東:昆都倫河,東南五達河從之。

喀爾喀右翼部一旗:札薩克駐塔爾渾河,在張家口西北七百十里。東南距京師一千一百三十里。漢,定襄、雲中二郡北境。唐,振武軍地。遼,豐州地,屬西京道。金因之。元屬大同路。明為喀爾喀所據,臺吉本塔爾,喀爾喀土謝圖汗親屬,世為臺吉。順治中,與土謝圖汗有隙,來歸,封親王,主右翼。牧地在愛布哈、塔爾渾河合流處。東界四子部落,西界茂明安,南界歸化城,北界瀚海。廣百二十里,袤一百三十里。北極高四十一度四十四分。京師偏西五度五十五分。貢道由張家口。其山:東曰拜音拖羅海山、西神山。西南,哈達圖山、罽嶺蒙名毛德爾。北,白雲山蒙名插漢和邵。東北,插漢峨博山、摩禮圖峨博岡。東南,烏蘭峨博山、翁公峨博岡。西,西巴爾圖峨博岡。東南:黃水河,自歸化城土默特入境,逕翁公峨博岡,東北流,入四子部落。西北:愛畢哈河,自茂明安逕白雲山、喀喇峨博岡間,東流,出喀倫邊。以上統盟於烏蘭察布。盟地在四子部落境內,歸化城南百二十里。有五藍叉拍山,即此。

鄂爾多斯舊六旗,又增設一旗,共七旗:在綏遠西二百八十五里河套內。東南距京師一千一百里。秦,新秦中。漢,朔方郡地。晉,前後趙、前後秦、赫連夏地。元魏為夏州北境。隋於其地東置勝州、西置豐州,後改榆林、五原二郡。唐置州,復改郡。五代、宋、金屬西夏。元立西夏、中興等路。後廢,其地東屬東勝、雲內二州,延安、寧夏等路。明初置東勝等州,立屯戍,耕牧其中。嘉靖中,套西吉納部落擊破和實居此,是為鄂爾多斯。天聰九年,額林臣來歸,賜濟農之號。順治六年,封郡王等爵有差,七旗皆授札薩克,自為一盟於伊克昭。東界歸化城土默特,西界喀爾喀,南界陜西長城,北界烏喇特。東、西、北三面距河,自山西偏頭關至陜西寧夏街,延長二千餘里。貢道由殺虎口。乾隆元年裁。鄂爾多斯左翼中旗正中近東。札薩克駐敖西喜峰,在札拉谷西一百六十里,本隋、唐勝州地。牧地有納瑪帶泊,喀錫拉河出旗界東北流瀦焉。東至袞額爾吉廟,接左翼前旗,南至神木縣邊城,西至察罕額爾吉,接右翼前旗,北至喀賴泉,接右翼後旗。廣一百一十五里,袤三百二十里。北極高三十九度三十分。京師偏西七度。其水:東曰紫河,蒙名五藍木倫,源出臺石坡西平地,西南流入陜西邊境。東,袞額爾吉河,源出袞額爾吉坡南平地,西南流,會哈楚爾河。哈楚爾河源出喀楚爾坡西平地,西南流,會紫河,入神木,為屈野河。鄂爾多斯左翼前旗套內東南。古榆林塞。札薩克駐札拉谷,在湖灘河朔西百四十五里。明,榆林左衛地。牧地當偏關西。左倚黃河,東界湖灘河朔,南界清水河,西界左翼中旗,北界左翼後旗。廣二百四十五里,袤二百一十里。北極高三十九度四十分。京師偏西五度四十分。東南:和嶽爾喀喇拖羅海山一名夾山、黑山蒙名喀喇和邵。北:巴漢得石峰。西北:得石峰。東北:昆兌河,源出平地,東南流入黃河。東南:小昆兌河,亦東南流入黃河。東:布林河,源出查木,塔爾奇爾河,源出噶克插冒頓;哈岱河,源出賀爾博金坡南平地,均東南流入黃河。芹河,蒙名伊克西喇爾幾臺,源出杜爾伯特拜坡東平地,南流入邊城,為陜西府谷縣清水川。小芹河,源出得勒蘇臺坡南平地,克丑河,源出噶克插冒頓東平地,南西河,源出科爾口,俱東入芹河。西南:麞河,蒙名西爾哈,源出常樂堡,合葫盧海南流入紅石峽。鄂爾多斯左翼後旗套內東北。札薩克駐巴爾哈遜湖,在黃河帽帶津西百四十里。隋、唐,勝州、榆林郡治。牧地當山西五原南、薩拉齊西。東界薩拉齊,南界左翼前旗,西界左翼中旗,北界烏喇特。廣二百八十里,袤一百五十里。北極高四十度四十分。京師偏西八度。東南:退諾克拖羅海山,山西為拜圖拖羅海山。南:伊克翁公岡、巴漢翁公岡。東南:插漢拖羅海岡。西北:車根木倫河,源出撒爾奇喇地,東流入黃河。烏爾巴齊河,源出平地,黑河蒙名伊克土爾根,源出虎虎冒頓地;西:兔毛河,蒙名陶賴昆兌,源出敖柴達木,柳河,蒙名布爾哈蘇臺,源出插漢拖羅海岡,喀賴河,源出硃爾漢虎都克,西都喇虎河,源出吳烈泉,東坎臺河,源出布木巴泉,均北流入黃河。鄂爾多斯右翼中旗正西近南。札薩克駐錫拉布裏多諾爾,在鄂爾吉虎泊西南二百六十里。漢朔方郡南境。牧地當寧夏東北騰格里泊。東北皆界右翼後旗,南界右翼前旗,西界賽音諾顏左翼後旗。廣三百二十里,袤四百八十里。北極高三十九度四十分。京師偏西九度。南:蘇海阿祿山、賀佟圖山。西:色爾騰山。西北:黃草山蒙名庫勒爾齊、鄂藍喀喇陀羅海山、色爾蚌喀喇山。西南:庫葛爾黑河,源出庫葛爾黑泉,南流入邊,又西折出邊,入黃河。西北:伊克托蘇圖河,源出布海札剌克地,西流會黃河。西:巴漢托蘇圖河,源出巴惠泉,西北流,會依克托蘇圖河,入黃河。鄂爾多斯右翼前旗套內西南。札薩克駐巴哈諾爾,在敖西喜★西九十里。隋、唐,夏、勝二州地。牧地當陜西懷遠西北大鹽濼。東界左翼中旗,南界懷遠,西界右翼中旗,北界右翼後旗。廣一百八十里,袤二百七十里。北極高三十八度二十分。京師偏西九度。其山:南曰恩多爾拜山、巖靈山一名錦屏山。東南,總材山蒙名磨多圖。西南,巴音山。東南:上稍兒河,源出鯀布裏都,南流入邊城。南:席伯爾河,源出蟒喀圖虎爾虎地,南流會西克丑河入邊城,為榆林之榆溪。阿爾塞河,源出恩多爾拜山南平地,西南流,會席伯爾河。西南:金河蒙名西喇烏素,源出磨虎喇虎地,南流會哈柳圖河,東南流,合細河、金河二水,入榆林邊,至波羅營,會西來之額圖渾,為無定河。細河,蒙名納林河,源出托裏泉,南流亦會哈柳圖河。石窯川河,蒙名額圖渾,源出賀佟圖山北平地,東南流,合數小水,入懷遠邊,為恍忽都河,又折而東北,至波羅營,會海克圖河,為無定河。東:忒默圖插漢池,一名大鹽濼。西南:烏楞池,一名紅鹽池。南:長鹽池,蒙名達布蘇圖。鄂爾多斯右翼後旗套內西北。札薩克駐鄂爾吉虎諾爾河,在巴爾哈孫泊西一百七十里。隋、唐,豐州、九原郡治地。牧地當山西五原西、甘肅寧夏東北。右倚黃河,東界左翼後旗,南界左翼中旗,西界右翼中旗。北界烏喇特。廣一百八十里,袤一百六十里。北極高四十度四十分。京師偏西八度。西:馬陰山蒙名阿克塔賀邵。東南:吳烈鄂博拖羅海岡。西南:達爾巴漢岡。西:赤沙河,蒙名烏藍,源出赤沙泉,東北流,入鍋底池。西南:黃水河,蒙名西喇木倫,源出馬陰山北平地,東北流,入古爾板泊。鍋底池,周二十餘里,產鹽。兔河、赤沙河二水注其中,土名喀喇莽奈。鄂爾多斯右翼前末旗順治六年授二等臺吉。康熙十四年晉一等。乾隆元年,以族繁增旗一,授札薩克,世襲,掌右翼前末旗,附右翼前旗游牧。札薩克駐所,距綏遠城七百二十里。內蒙古驛凡五道:曰喜峰口,古北口,獨石口,張家口,殺虎口。自喜峰口至札賚特為一路,計千六百餘里,設十六驛。自古北口至烏珠穆沁為一路,計九百餘里,設九驛。自獨石口至浩齊特為一路,計六百餘里,設六驛。自張家口至四子部落為一路,計五百餘里,設五驛。自殺虎口至烏喇特為一路,計九百餘里,設九驛。自歸化城至鄂爾多斯計八百餘里,設八驛,仍為殺虎口一路。各驛站均設水泉佳勝處。以上自為一盟於伊克昭,與上五盟同列內札薩克。


\end{pinyinscope}