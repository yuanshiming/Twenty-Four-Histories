\article{列傳二百四}

\begin{pinyinscope}
都興阿弟西凌阿福興富明阿舒保伊興額滕家勝關保

都興阿,字直夫,郭貝爾氏,滿洲正白旗人,內大臣阿那保孫。父博多歡,正黃旗蒙古都統。都興阿由廕生授三等侍衛,晉二等。咸豐三年,從僧格林沁赴天津剿粵匪,破之於杜家嘴,擢頭等侍衛。四年,克獨流,追賊阜城,破運河濱賊壘。五年,克連鎮,賊首林鳳祥就擒,加副都統銜、乾清門行走。尋授京口副都統。

弟西凌阿督師湖北,都興阿率馬隊往助剿,復德安,從總督官文進規武漢。時官文軍北岸,趨漢陽,巡撫胡林翼軍南岸,攻武昌。都興阿率騎兵護水師,敗賊沙口,薄漢陽西門。六年,賊由金鋪山上竄,都興阿揮步隊迎擊,分馬隊抄其後,斬馘甚眾,焚團風鎮屯糧,斬其酋。林翼燔漢陽城外賊艇,賊登岸遁,都興阿以馬隊遮殲之,擢江寧將軍。襄樊土匪方熾,都興阿馳援襄陽,解其圍。進圍武昌,賊糧盡援絕,棄城遁,復武昌、漢陽,乘勝克黃州、興國、大冶、蘄水、蘄州、廣濟、黃梅諸城,賜號霍欽巴圖魯。

大軍進規九江,南路李續賓主之,北路都興阿主之。七年,賊由太湖竄窺黃梅,都興阿空城誘之,盡殲騎賊千餘,其由獨山鎮來襲者,馬步合擊,擒斬數千。進攻小池口,令多隆阿等破段窯、楓樹坳、獨山鎮賊巢。陳玉成大舉麕至,都興阿令多隆阿出黃梅,鮑超屯孔壟,自督馬步攻渡河橋,平二十餘壘,俘斬數千。會合楊岳斌、李續賓水陸軍攻童司牌,盡平賊壘。進克黃蠟山,先後殲賊萬餘,玉成遁走。詔都興阿幫辦官文軍務。攻小池口,燔其城,遂會克湖口,破賊彭澤,下望江、東流。八年,會克九江,被優敘。復黃安、麻城,分軍破賊彌陀鎮、南陽河,復太湖,偕李續賓軍會攻石牌,克之,授荊州將軍。會水師進規安慶,奪集賢關,薄安慶北門,破賊壘環攻。而李續賓戰歿三河,桐城、舒城再失,都興阿率軍退保宿松。多隆阿偕鮑超大破賊於花涼亭,楚師復振。

九年,曾國籓奏請於安徽上游北岸添馬步三萬人,以都興阿領其軍,會病足,薦多隆阿自代,詔赴荊州本任。十年,江南大營潰,上命都興阿帥馬步援江北,而以曾國籓總督兩江。時英法聯軍犯京師,都興阿備北援,馳抵壽州。和議成,命赴揚州督辦江北軍務。十一年,令總兵吳全美率師船攻和州江下關,毀賊壘,破內江口賊船。

都興阿樂用楚軍,胡林翼分其軍以畀國籓,揚州兵單,留徐州鎮總兵詹啟綸從剿,令提督黃開榜焚三河賊船。賊由儀徵犯揚州,都興阿遣總兵王萬清防湖西,自率三百騎出覘賊,賊眾萬餘,列陣待。都興阿令騎皆下,自席地坐,賊疑有伏,不敢逼,後軍至,奮擊破之。賊又糾蘇州、句容悍黨分犯瓜洲、鎮江,都興阿乘其壘未成,令營總杜嘎爾率馬步軍沖擊,自督隊繼之,賊大潰。詹啟綸乘勢踏毀甘泉山賊壘,鎮江圍解。

調江寧將軍,仍駐揚州督江北軍,文武悉聽調度。令副都統海全等破後石橋賊營,賊由常州窺鎮江,總兵黃彬統水師擊敗之。都興阿馳抵天長城下,平其壘卡,賊酋龔長春遁走,沿途截殺殆盡。黃彬等破賊船小河口、太平港,平瓜埠賊巢,尋會江南提督李世忠收降六合、天長二城。同治元年,江浦、浦口復陷,賊進犯揚州,北營甘泉山,南亙樸樹灣,都興阿親督諸軍連擊,敗走之。

時上游諸軍連克沿江要隘,進薄江寧,都興阿令總兵李起高駛至浦口,襲攻觀音門、燕子磯為聲援,曾國荃大營為援賊所圍,遣副將楊心純率二千五百人赴援,入壕助守,又令黃彬率水師援九洑洲。二年,賊謀入里下河,都興阿遏之不得逞,別遣副將梁正源會江南軍焚中關、下關賊舟,李起高會收江陰。

三年,江寧合圍,江北無警,而陜、甘回亂益熾,詔都興阿赴綏遠城督防。時甘肅寧夏漢城陷於賊,滿城待援,召都興阿入覲,調西安將軍,督辦甘肅軍務,署陜甘總督。江寧克復,論功,予騎都尉世職。

六月,都興阿至定邊,奏言回酋馬化隆起靈州金積堡,占踞城堡,蔓延千里。定邊距離尚遠,宜進兵花馬池,三路合攻,方期得力。令杜嘎爾等由草地繞石嘴山渡河,攻克姜家村、紅柳溝賊巢,追至寶豐,賊三路出撲,擊敗之,復寶豐,解平羅圍。軍進渠公堡,都興阿慮深入無繼,奏調荊州將軍穆圖善會剿。賊首馬叱吽踞通成堡,突出戰,為杜嘎爾等所敗,退踞清水堡。都興阿移營進逼,絕其糧道,攻克之。進金貴堡、王格莊,去寧夏城二十里,敗西路援賊。城賊抄官軍後,都興阿督諸軍迎擊,賊大潰。四年,列陣城東誘賊出,擊敗之,拔南路賊圩。鹽池、固原竄匪踞安化元城鎮,窺寧條梁糧路,都興阿遣軍分防花馬池、定邊、寧條梁,而靖遠南山賊焚堡據壩修堰,將決渠困官軍,乃移屯城東南。賊又踞堤築壘斷水道,並擊退,不得逞。都興阿親督隊敗賊於金貴堡,分軍屯定邊、花馬池,賊由固原趨平羅、寶豐,截擊於金貴堡,敗之。杜嘎爾擊賊於磴口,斃其酋馬生顏,花馬池、定邊兩路同捷,馘賊首孫義和。寧夏賊勢漸蹙,詭辭乞撫,計緩兵,而潛決西河水灌官軍。都興阿拒其降,益修戰備,進解滿城圍,克城東賊圩,敗之西門橋,分軍擊走大水坑、吳中堡踞賊,斬回酋馬有富,軍威頗振。

會奉天馬賊猖獗,調都興阿盛京將軍移剿,而以穆圖善代之。穆圖善主撫,寧夏賊尋降,納砲械縛渠以獻。五年,穆圖善劾都興阿受降入城,仍戮回酋章保立,部下殺掠。詔斥都興阿剿撫無定見,下議褫職,改留任。都興阿至奉天,馬賊漸平,奏定緝捕章程,搜剿餘匪,尋定。

七年,西捻張總愚竄畿輔,李鴻章、左宗棠率兵入衛,賊流竄直隸、河南、山東,數月未定。詔召都興阿入京,管理神機營,授欽差大臣,以副都統春壽,提督張曜、宋慶,侍衛陳國瑞四軍隸之,列名在鴻章、宗棠上。視師天津,捻匪尋蕩平,仍回本任。光緒元年,卒於官,賜恤,贈太子太保,謚清愨。奉天士民請與大學士文祥、將軍崇實合建三賢祠,揚州亦請與將軍富明阿合祠。

西凌阿,都興阿弟。由拜唐阿授侍衛。道光中,從揚威將軍奕經援浙江,迭晉頭等侍衛,累擢察哈爾都統。咸豐三年,率黑龍江騎兵從琦善防浦口,因不能阻★匪北竄,褫職留營,責令追賊。偕將軍托明阿等馳解開封圍,又敗之汜水。賊渡河圍懷慶,援軍會集,西凌阿戰最力,圍解,復原官。追賊,迭戰王屋、邵原、平陽、洪洞,由山西入直隸,命幫辦勝保軍務。至靜海,賊蹤始定,會軍圍攻。四年春,賊走阜城,西凌阿追至後康莊,破之。從僧格林沁連破城外賊屯,賊走踞東光、連鎮,攻戰數月,西凌阿常為軍鋒,五年正月,克之,擒賊首林鳳祥,予二等輕車都尉世職,賜號伊精阿巴圖魯。又從僧格林沁克馮官屯,俘李開芳,錫封三等男爵,授欽差大臣,督辦湖北軍務。初至隨州,戰不利,命其兄都興阿往助,尋褫職,以官文代督師。從復德安府城,復原官,原駐以固北防。

六年,僧格林沁薦之,率馬隊赴河南剿捻匪。七年,復以屢挫,褫職留任,破張洛行白龍王廟老巢,復之。八年,命駐防山海關。十年,授鑲藍旗蒙古都統,從僧格林沁赴山東剿捻匪,尋命幫辦軍務。十一年,戰菏澤失利,下部嚴議。破賊於東昌,焚賊巢,克七級鎮,進克張秋。又破曹州紅川口匪圩,進敗賊於劉家橋、大張寺。同治元年,以腿疾回京,授鑲藍旗漢軍都統。五年,卒,賜恤,謚勇毅。

福興,穆爾察氏,滿洲正白旗人,都統穆克登布曾孫。以一品廕生授三等侍衛,出為直隸懷安路都司,累擢督標中軍副將。咸豐元年,擢廣東高州鎮總兵。二年,平羅鏡匪凌十八及鬱林、博白土匪,賜號剛安巴圖魯,擢廣西提督。命援湖南,偕向榮分路追賊湖北,以遷延,奪職留營。三年,從援江寧,屯朝陽門外,屢擊賊,予三品頂戴,充翼長。偕提督鄧紹良破賊東壩,復高淳,會克太平,回軍江寧,迭擊賊於高橋門、上方橋、通濟門、雨花臺,晉二品頂戴,署江寧將軍。母憂,奪情留軍。

六年,授西安將軍,幫辦向榮軍務。偕張國樑援鎮江,敗賊丁卯橋。江寧大營潰,向榮退保丹陽,上切責諸將,福興革職留任。榮病卒,命偕張國樑同任防剿。上聞福興與國樑不和,諭怡良察之,遂命福興赴江西會辦軍務。七年,復樂平,攻東鄉、金谿。石達開自安慶竄浮梁、樂平,圍貴溪。八年正月,福興至弋陽,賊來犯,福興兵少,多為疑兵,賊不敢偪,擊走之,竄浙江,福興進屯衢州東關,賊迭來撲,皆擊退。回駐玉山,防賊復竄廣信。尋又赴衢州,攻東關賊營。福興右腿受傷,尋以傷重乞假,召回京。十一年,署鑲紅旗漢軍都統。

同治四年,從尚書文祥會剿奉天馬賊,戰大凌河、北井子,擒斬甚眾。進援吉林,迭破賊於張登、望城岡,署盛京將軍。五年,擒賊首馬傻仔於黃旗堡,誅之。事平,凱旋,授察哈爾都統,調綏遠城將軍。六年,以舊傷發乞休,光緒四年,卒,賜恤,謚莊愨。

富明阿,字治安,袁氏,漢軍正白旗人,明兵部尚書崇煥裔孫。崇煥裔死,家流寓汝寧,有子文弼,從軍有功,編入寧古塔漢軍。五傳至富明阿,以馬甲從征喀什噶爾,授驍騎校,洊升參領。

咸豐三年,從欽差大臣琦善軍揚州,戰於洞清鋪,受槍傷,裹創奮斗,斬馘數十,擢協領,特賜玉。四年,破賊瓜洲,賜花翎,管帶寧古塔兵。五年,戰虹橋,戒所部距賊二十步始發矢,射斃賊酋,分兩翼搜伏賊,賊潰走,加副都統銜。六年,署寧古塔副都統,迭敗賊於徐家集、硯臺山。攻瓜洲,又率隊及六合練勇攻江浦,敗賊於十里橋,賜號車齊博巴圖魯。又敗賊於樸樹灣、土橋、五新橋。七年,會攻瓜洲,連敗賊富家井、白廟,以副都統記名。是年冬,克瓜洲,詔以副都統侭先題奏。充江北軍翼長,進攻江浦。八年春,迭破援賊,復其城。進屯六合,攻滁州,克來安,加頭品頂戴。八月,德興阿兵敗浦口,富明阿馳援失利,傷亡幾半。退儀徵,收集散卒,復成軍,扼萬福橋,破賊於運河東,授寧古塔副都統。偕張國樑克揚州、儀徵,又破賊於冶鋪橋。

九年,德興阿以失律罷黜,江北軍不署統帥,命歸和春節制,別選謀勇可當一面者,和春以富明阿薦,詔幫辦和春軍務。時六合、浦口皆未復,富明阿督軍進攻,迭戰百龍廟、李家營及六合城外。既而賊數萬撲營,分股繞襲後路,遂大挫。富明阿身被十二創,詔許開缺回旗醫治,傷已成殘,命以原品休致,食全俸。十一年,召至京,命訓練京營。

同治元年,授正紅旗漢軍都統,管理神機營。尋命赴揚州幫辦都興阿軍務。江北里下河十餘縣未被賊擾,鹽場之利如故,偕都興阿疏請運鹽濟餉,軍用得給。長江下游南北岸要口四十餘處,排椿駐船,分撥水師扼要駐防,疏陳部署情形,詔特嘉其諳悉地勢。賊屢糾捻匪窺伺江北,迭擊走之。分軍渡江助馮子材守鎮江。是年秋,親率精銳援臨淮,會僧格林沁剿苗沛霖,詔幫辦軍務,令部將詹啟綸、克蒙額會陳國瑞等進攻,連破賊,沛霖伏誅。傷發,請假就醫清淮,疏陳皖北圩練之弊,詔下僧格林沁、曾國籓議加整頓。

三年春,都興阿赴陜、甘剿回,詔促富明阿回揚州坐鎮,署江寧將軍,尋實授。遣詹啟綸率兵渡江助剿,克丹陽,賜黃馬褂。江寧克復,予騎都尉世職,仍督所部水陸諸軍留防江北。於是裁撤紅單船,由提督吳全美率回廣東,酌裁陸軍數千。疏言:「江寧駐防,亂後僅存男婦六百餘人,現設官二十七員,兵二百五十八名,稍存營制。京口駐防,尚存四千餘人,官兵挑補足額,俸餉不能全支,房屋均已焚毀。請飭撥餉修蓋房屋,使有依歸。」從之。

四年,因腿傷未痊,請開缺,予假赴京醫治,許坐肩輿,至京,仍命管理神機營。傷病久不愈,詔允回旗。五年,起授吉林將軍,督剿馬賊。力疾進搜山險,遣將分捕,數月肅清。招撫金匪,開闢閒田至數萬頃。不及十年,遂開建郡縣焉。在任四年,復以傷病陳情乞罷,允之,仍在家食全俸。光緒八年,卒,優恤,謚威勤。吉林、揚州請建祠。

子壽山,官至黑龍江將軍,光緒中,俄羅斯犯邊,殉難;永山,官三等侍衛,亦於鳳皇城拒日本,力戰死事:皆自有傳。

舒保,字輔廷,舒穆魯氏,滿洲正黃旗人。由護軍累擢護軍參領。咸豐四年,從僧格林沁剿粵匪,攻圍連鎮,賊乘大風出竄,舒保截殲之。五年,竄踞馮官屯,引水灌之,功最,賊渠俘獲,加副都統銜。荊州將軍綿洵奏調赴湖北,率馬隊破賊德安。六年,迭破黃州李先集、團風賊壘。胡林翼之圍武昌也,官文令舒保率馬隊三百渡江助戰。城賊、援賊分八路來犯,舒保以勁騎馳突,賊大奔。平魯家港賊壘,又敗之沙子嶺、小龜山、雙鳳山。旬日之間,大小二十八戰,胡林翼奏稱舒保馬隊之力特多,賜號倭什洪額巴圖魯。偕知府唐訓方合剿襄陽土匪,迭敗之黃龍橋、餘山店,解襄陽圍。克樊城、老河口賊巢,復光化、房、竹山三城。雪夜擒匪首高二張家集,誅之,襄陽平。餘匪遁入河南境,陷內鄉,七年,舒保躡至,會豫軍殲之。

八年,授鑲黃旗漢軍副都統。舒保方駐防商城,而賊由六安進犯湖北,陷麻城,急回軍趨黃州。南勇敗於望天畈,為賊追逼,舒保迎擊,戰一晝夜,賊始退。又偕李續宜破諸蘄水。

時欽差大臣勝保援固始,兵未利,而商城又告警。勝保嚴檄舒保助剿,胡林翼疏言:「舒保樸訥忠勇,在楚有年,洪山之戰,襄陽之役,蘄、黃之捷,實能為人所難為,從無就易避難之意。今以特簡二品大員,勝保乃嚴札驅迫,加以苛辭。師克在和,古有明訓。束縛馳驟,必誤戎機。挾權任術,馭不肖之將,或可取快一時,若忠良之士,不煩督率而自奮也。臣謂舒保一軍,應審楚、豫各路賊勢,相機進剿,毋庸強歸鄰省節制調遣。」上命舒保仍回羅田、麻城剿賊。

固始圍解,陳玉成復犯湖北,舒保偕續宜破之麻城。李續賓既克九江,會師攻黃安,下之。既而續賓戰歿三河,楚邊大震,舒保以所部四百騎自武昌東下。林翼次黃州,增舒保軍千人,以新補西丹游牧蘄水、上巴河,而令率舊部赴太湖,為多隆阿聲援。會別賊又陷德安、黃州、孝感諸府縣,將軍都興阿檄調舒保未至,奏劾其觀望,下部議。十一年,偕道員金國琛會攻德安,先克孝感,復會水陸軍圍攻德安,克之,加都統銜。

同治元年,授護軍統領。粵、捻諸匪分兩路竄湖北,總兵穆正春擊其西路,舒保擊其東路,連敗之於黃陂、廣濟、應山,賊竄回豫境,賜黃馬褂。賊復回應山,撲孝感城,舒保啟南門奮擊,賊已卻,突別賊數千潛由北城入,副都統德克登額、署知縣韓體震等死之。舒保還戰城中,賊復敗逸,追殺三十餘里。二年,賊由應城圖襲漢口,為官軍所卻。轉撲孝感,舒保迭戰李家灣、倉子埠,陣斬老捻千餘,遂引去,被珍賚。三年,擊捻匪於德安西,追抵壽山,日暮,層岡深澗,不利騎戰,賊來益眾,舒保陷重圍中,越坎落馬,力竭陣亡,贈太子少保,予騎都尉兼雲騎尉世職,入祀昭忠祠,湖北建專祠,謚貞恪。

伊興額,原名伊清阿,字松坪,何圖哩氏,蒙古正白旗人,吉林駐防。從征喀什噶爾,除驍騎校,選授侍衛。入京召對,宣宗以原名不合清語,命改名伊興額。道光十九年,擢三等侍衛,改隸滿洲。

咸豐三年,自請從軍,發揚州大營。琦善令援江浦,初至,示弱不戰,斫賊營,大敗之柳樹壩,破九洑洲,累擢頭等侍衛。賊圍和州急,伊興額不待令,督軍進擊,解其圍。駐江浦三年。六年,剿捻匪夏白、任仲勉於澮河北岸,殲賊二千餘,仲勉斃於陣。夏白糾雉河賊黨圍宿州,伊興額率千騎往援,四戰皆捷,解其圍。分軍防徐州、宿州,張洛行來犯,偕總兵傅振邦擊走之。時潁、亳、蒙、宿諸捻蜂起,徐、宿百里內宴然,耕穫不輟。賊首王廣愛、梁振貴眾數萬聚張七家樓,圖北竄。伊興額選精銳數百,疾馳掩入賊壘,擒王、梁二賊,賊黨來援,擊走之,以副都統記名,賜花翎。

七年,招降王家墟捻黨陳保元五千人,斬其渠李月,賜號額圖渾巴圖魯。因病回徐州,勝保劾其不遵調度,報捷不實,褫翎頂。尋率馬隊攻喬家廟,擒斬捻首梁思住,夜進攻酆家墟,誅賊渠劉大淵,偕總兵史榮椿破賊趙家屯。渦河兩岸肅清,復翎頂。八年,授正紅旗蒙古副都統。破捻匪於紀家莊,解蒙城圍。攻薛家湖賊巢,砲傷股,裹創力戰,毀其墟,加頭品頂戴。尋捻首劉添祥由六安北竄,眾號數萬,伊興額以孤軍無援,退屯蕭縣。賊陷豐縣,坐奪職。

九年春,起佐傅振邦剿捻,接統袁甲三所部兵,詔復職,督辦河南軍事。伊興額率騎千三百赴援,躡賊商水老湖坡。賊眾數倍,列車陣拒戰,潛分兵繞賊後夾擊之,賊潰走,窮追三晝夜,歷沈丘、項城至太和孫家圩,先後斃賊二萬餘,被旨嘉獎。時總兵邱聯恩戰歿舞陽,舞陽去商水二百里,及戰勝,舞陽賊聞風遁。

勝保復奏劾:「伊興額性情乖僻,商水之役,僅擊退別股,並未迎剿。舞陽賊眾僅六千,而疏報輒稱二萬三千。」詔奪頭品頂戴,交勝保差遣。所請獎老湖坡戰勝員弁,下署漕督袁甲三察奏。伊興額遂謝病回徐州,距復起僅三月。尋詔飭赴甲三營剿賊,稱疾不赴。蕭縣民鄭立本等以伊興額去,賊復熾,叩閽請還鎮。德楞額復代奏:「徐州紳民以伊興額在徐養病,請飭就近治軍。」先後諭詢伊興額病狀,伊興額固以篤疾辭,上怒,褫職,勒令回旗。都察院奏上安徽監生張鴻文呈,言伊興額前功,懇令總辦討賊事宜,不報。

十年,僧格林沁疏薦,予六品頂戴,尋加三品,敕辦徐、宿團練。伊興額再起,其舊部多不隸麾下,所將五百騎未及訓練,以賊竄曹州,僧格林沁趣援。十一年春,偕徐州鎮總兵滕家勝率騎二千馳往,擊賊於東平、汶上,累捷。追至臥虎岡,風霾忽作,急退楊柳集。伏起,家勝馬躓,歿於陣。伊興額揮百餘騎沖入賊中,索之不得,突圍出,從騎僅隨者十餘人,賊圍之數匝,力竭死之。詔復原官,予騎都尉世職,謚壯愍,建祠徐州、汶上、宿州、永城。

滕家勝,湖南乾州人。由行伍從江忠源剿賊湖北,累擢游擊。繼從袁甲三剿捻於皖、豫之間,擢參將,賜號伊博格巴圖魯。勝保薦其少年勇敢,超授四川川北鎮總兵,調徐州鎮,幫辦徐、宿軍務。家勝舊隸伊興額部下,至是同戰歿,贈提督,予騎都尉並雲騎尉世職,謚武烈。

關保,烏扎拉氏,滿洲正黃旗人,吉林駐防。道光初,從征喀什噶爾有功,洊升三姓佐領。咸豐三年,隨侍郎恩華剿粵匪,解懷慶圍,追敗之平陽,屯正定。勝保檄充營總,剿賊於深州、靜海,攻獨流鎮,擢協領。四年,從僧格林沁戰阜城三里莊,槍傷額,奮擊破賊,賜號年昌阿巴圖魯。從勝保援臨清,追賊至豐縣,殲之,以總管升用。五年,從僧格林沁克馮官屯,從西凌阿赴湖北,尋調河南,又調安徽,從和春克廬州,加二品銜。六年,偕副都統麟瑞破賊五河,斃黃衣賊目二人,殲賊千餘,敗邳州援賊,解壽州圍。擊賊潁上,五戰皆捷,所部馬隊,各省爭欲得之助剿。尋隸河南巡撫英桂軍,敗賊馬村橋,進亳州雙溝,遇賊姬橋,殲之。又連敗賊三丈口、舊縣集,安徽巡撫福濟疏調赴蒙城,英桂仍請留河南,詔令和衷商榷,先赴所急。其冬,率軍趨懷遠,越境敗捻匪於徐州。漕運總督邵燦疏請留徐州,報允。

七年,偕總兵史榮椿攻永城嶽家集,捻首李月先遁,焚其巢,尋以病歸。八年,命率吉林、黑龍江、察哈爾兵千八百赴袁甲三徐州營。九年,會攻澮北捻首曹金斗,破其圩,乘勝擊捻首張寶全,破之。張洛行陷泗州,踞草溝民寨,關保率民團奪圩外砲臺,毀其寨,賊分竄,自相踐踏。餘賊竄五河雙渡口,奪船爭渡,追殪過半,擒賊目張起等,以副都統記名。命幫辦傅振邦三省剿匪事,斬捻首張添福,進搗任乾畢圩,圩民內應斬乾,餘黨盡殲。接統伊興額軍,命督辦河南防務,佐振邦剿匪三省如故。

授黑龍江副都統。破亳州竄匪,捻首孫葵心聚黨永城,圖分竄,詔勿令攔入山東邊境。飭關保截賊西路,逼之歸巢。已而賊眾二萬分擾商丘、柘城,圍睢州,開封戒嚴,上命由鹿邑赴援。賊趨蘭儀,分擾通許、尉氏,關保馳抵陳留,合諸軍夾擊,賊南走。馳援許州,遣副將王鳳翔率騎兵敗賊洪河北岸,又敗之臨潁城下,陣擒葵心親屬孫套。夜,簡精騎劫賊營,斬馘無算,拔難民千餘,賊東奔,偵別賊竄扶溝、太康,要擊之王隆集。沿途搜捕,豫境肅清。十年,命勝保督辦河南軍務,關保仍副之。賊擾虞、夏邑、鹿邑,遣將擊走之,俄又大至,逼近省垣,詔詰責。尋轉戰汝寧、確山皆捷,分兵破賊鹿邑劉集,解丘集圍。賊復糾黨來攻,擊走之。傷發,予假調理。同治元年,赴黑龍江任,八年卒。

論曰:都興阿雅量寬閎,知兵容眾,胡林翼稱其有豐、鎬故家遺風,當時滿洲諸名將,半出部下。舒保亦以樸勇為林翼所倚重,及林翼歿,無人善用,倉卒殞寇,世咸惜之。富明阿始終江北軍事,其勛勞出托明阿、德興阿之上,晚膺邊寄,亦稱賢帥。伊興額剿捻盡瘁,最得民心,為驕帥齮齕,未竟其用。關保善將騎,群帥爭相引重,其遭際為獨幸焉。


\end{pinyinscope}