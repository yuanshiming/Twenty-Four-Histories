\article{志二十一}

\begin{pinyinscope}
時憲二

△推步算術

推步新法所用者,曰平三角形,曰弧三角形,曰橢圓形。今撮其大旨,證立法之原,驗用數之實,都為一十六術,著於篇。

平三角形者,三直線相遇而成。其線為邊,兩線所夾空處為角。有正角,當全圓四分之一,如甲乙丙形之甲角。有銳角,不足四分之一,如乙、丙兩角。有鈍角,過四分之一,如丁戊己形之戊角。圖形尚無資料

角之度無論多寡,皆有其相當之八線。曰正弦、正矢、正割、正切,所有度與九十度相減餘度之四線也,如甲乙為本度,則丙乙為餘度。正弦乙戊,正矢甲戊,正割庚丁,正切庚甲,餘弦乙己,餘矢丙己,餘割辛丁,餘切辛丙。若壬癸為本度,則丑癸為餘度,正弦癸辰,正矢壬辰,餘弦癸卯,餘矢丑卯,餘割子寅,餘切丑寅。以壬癸過九十度無正割、正切,借癸午之子未為正割,午未為正切。若正九十度丑壬為本度,則無餘度,醜子半徑為正弦,壬子半徑為正矢,亦無正割、正切,並無餘弦、餘矢、餘割、餘切。

古定全圓周為三百六十度,四分之一稱一象限,為九十度。每度六十分,每分六十秒,每秒六十微。圓半徑為十萬,後改千萬。逐度逐分求其八線,備列於表。推算三角,在九十度內,欲用某度某線,就表取之,算得某線。欲知某度,就表對之。過九十度者,欲用正弦、正割、正切及四餘,以其度與半周相減餘,就表取之。欲用正矢,取餘弦加半徑為之。既得某線,欲知某度,就表對得其度與半周相減餘命之。

圖形尚無資料

算平三角凡五術:

一曰對邊求對角,以所知邊為一率,對角正弦為二率,所知又一邊為三率,二三相乘,一率除之,求得四率,為所不知之對角正弦。如圖甲乙為所知邊,丁角為所知對角,乙丁為所知又一邊,甲角為所不知對角也。此其理系兩次比例省為一次。如圖乙丁為半徑之比,乙丙為丁角正弦之比。法當先以半徑為一率,丁角正弦為二率,乙丁為三率,求得四率中垂線乙丙。既得乙丙,甲乙為半徑之比,乙丙又為甲角正弦之比。乃以甲乙為一率,乙丙為二率,半徑為三率,求得四率,自為甲角正弦。然後合而算之,以先之一率半徑與後之一率甲乙相乘為共一率,先之二率丁角正弦與後之二率乙丙相乘為共二率,先之三率乙丁與後之三率半徑相乘為共三率,求得四率,自為先之四率乙丙與後之四率甲角正弦相乘數,仍當以乙丙除之,乃得甲角正弦。後既當除,不如先之勿乘。共二率內之乙丙與三率相乘者也,乘除相報,乙丙宜省。又共三率內之半徑與二率相乘者也,共一率內之半徑又主除之,乘除相報,半徑又宜省。故徑以甲乙為一率,丁角正弦為二率,乙丁為三率,求得四率,為甲角正弦。

二曰對角求對邊,以所知角正弦為一率,對邊為二率,所知又一角正弦為三率,求得四率,為所不知對邊。此其理具對邊求對角,反觀自明。

三曰兩邊夾一角求不知之二角,以所知角旁兩邊相加為一率,相減餘為二率,所知角與半周相減,餘為外角,半之,取其正切為三率,求得四率,為半較角正切。對表得度,與半外角相加,為對所知角旁略大邊之角;相減,餘為對所知角旁略小邊之角。此其理一在平三角形。三角相並,必共成半周。如圖甲乙丙形,中垂線甲丁,分為兩正角形。正角為長方之半,長方四角皆正九十度,正角形兩銳角斜剖長方,此角過九十度之半幾何,彼角不足九十度之半亦幾何,一線徑過,其勢然也。故甲右邊分角必與乙角合為九十度,甲左邊分角必與丙角合為九十度。論正角形各加丁角,皆成半周,合為銳角形。除去丁角,三角合亦自為半周。故既知一角之外,其餘二角雖不知各得幾何度分,必知其共得此角減半周之餘也。一在三角同式形比例。如圖丙庚戊形,知丙庚、丙戊兩邊及丙角。展丙庚為丙甲,連丙戊為甲戊,兩邊相加。截丙戊於丙丁,為戊丁,兩邊相減餘。作庚丁虛線,丙庚、丙丁同長,庚丁向圓內二角必同度,是皆為丙角之半外角,與甲辛、辛庚之度等。而庚向圓外之角,即本形庚角大於戊角之半,是為半外角。以庚丁為半徑之比,則甲庚即為丁半外角正切之比。半徑與正切恆為正角,甲庚與庚丁圓內作兩通弦,亦無不成正角故也。又作丁己線,與甲庚平行,庚丁仍為半徑之比,丁己又為庚向圓外半較角正切之比。而戊甲庚大形與戊丁己小形,戊甲、戊丁既在一線,甲庚、丁己又系平行,自然同式。故甲戊兩邊相加為一率,戊丁兩邊相減餘為二率,甲庚半外角正切為三率,求得四率,自當丁己半較角正切也。

四曰兩角夾一邊求不知之一角,以所知兩角相並,與半周相減,餘即得。此其理具兩邊夾一角。

五曰三邊求角,以大邊為底,中、小二邊相並相減,兩數相乘,大邊除之,得數與大邊相加折半為分底大邊,相減餘折半為分底小邊。乃以中邊為一率,分底大邊為二率,半徑為三率,求得四率,為對小邊角餘弦。或以小邊為一率,分底小邊為二率,半徑為三率,求得四率,為對中邊角餘弦。此其理在勾股弦冪相求及兩方冪相較。如圖甲丙中邊、甲乙小邊皆為弦,乙丙大邊由丁分之,丁丙、丁乙皆為勾,中垂線甲丁為股。勾股冪相並恆為弦冪,今甲丁股既兩形所同,則甲丙大弦冪多於甲乙小弦冪,即同丙丁大勾冪多於乙丁小勾冪。又兩方冪相較,恆如兩方根和較相乘之數。如圖戊寅壬庚為大方冪,減去己卯辛庚小方冪,餘戊己卯辛壬寅曲矩形。移卯癸壬辛為癸寅丑子,成一直方形,其長戊丑,自為大方根戊寅、小方根卯辛之和;其闊戊己,自為大方根戊庚、小方根己庚之較。故甲乙丙形,甲丙、甲乙相加為和,相減為較。兩數相乘,即如丙丁、丁乙和較相乘之數。丙乙除之,自得其較。丙午相加相減各折半,自得丙丁及乙丁,既得丙丁、乙丁,各以丙甲、乙甲為半徑之比,丙丁、乙丁自為餘弦之比矣。

此五術者,有四不待算,一不可算。對邊求對角,令所知兩邊相等,則所求角與所知角必相等。對角求對邊,令所知兩角相等,則所求邊與所知邊必相等。兩邊夾一角,令所知兩邊相等,則所求二角必正得所知外角之半。三邊求角,令二邊相等,即分不等者之半為底邊;三邊相等,即平分半周三角皆六十度,皆不待算也。若對邊求對角,所知一邊數少,對所知一角銳;又所知一邊數多,求所對之角,不能知其為銳、為鈍,是不可算也。諸題求邊角未盡者,互按得之。

弧三角形者,三圓周相遇而成,其邊亦以度計。九十度為足,少於九十度為小,過九十度為大。其角銳、鈍、正與平三角等。算術有七:

一曰對邊求對角,以所知邊正弦為一率,對角正弦為二率,所知又一邊正弦為三率,求得四率,為所求對角正弦。此其理亦系兩次比例省為一次。如圖甲乙丙形,知甲乙、丙乙二邊及丙角,求甲角。作乙辛垂弧,半徑與丙角正弦之比,同於乙丙正弦與乙辛正弦之比。法當以半徑為一率,丙角正弦為二率,乙丙正弦為三率,求得四率,為乙辛正弦。既得乙辛正弦,甲乙正弦與乙辛正弦之比,同於半徑與甲角正弦之比。乃以甲乙正弦為一率,乙辛正弦為二率,半徑為三率,求得四率,為甲角正弦。然乘除相報,可省省之。

二曰對角求對邊,以所知角正弦為一率,對邊正弦為二率,所知又一角正弦為三率,求得四率,為所求對邊正弦。此其理反觀自明。

三曰兩邊夾一角,或銳或鈍,求不知之一邊。以半徑為一率,所知角餘弦為二率,任以所知一邊正切為三率,求得四率,命為正切。對表得度,與所知又一邊相減,餘為分邊。乃以前得度餘弦為一率,先用邊餘弦為二率,分邊餘弦為三率,求得四率,為不知之邊餘弦。原角鈍,分邊大,此邊小;分邊小,此邊大。原角銳,分邊小,此邊小;分邊大,此邊大。此其理系三次比例省為二次。如圖甲丙丁形,知甲丙、甲丁二邊及甲角,中作垂弧丙乙,半徑與甲角餘弦之比,同於甲丙正切與甲乙正切之比。先一算為易明。既分甲丁於乙,而得丁乙分邊,甲乙餘弦與半徑之比,同於甲丙餘弦與丙乙餘弦之比。法當先以甲乙餘弦為一率,半徑為二率,甲丙餘弦為三率,求得四率,為丙乙餘弦。既得丙乙餘弦,半徑與乙丁餘弦之比,同於丙乙餘弦與丁丙餘弦之比。乃以半徑為一率,乙丁餘弦為二率,丙乙餘弦為三率,求得四率,為丁丙餘弦。然而乘除相報,故從省。兩邊夾一角若正,則徑以所知兩邊餘弦相乘半徑除之,即得不知邊之餘弦,理自明也。所知兩邊俱大俱小,此邊小;所知兩邊一小一大,此邊大。

四曰兩角夾一邊,求不知之一角。以角為邊,以邊為角,反求之;得度,反取之;求、取皆與半周相減。

五曰所知兩邊對所知兩角,或銳、或鈍,求不知之邊角。以半徑為一率,任以所知一角之餘弦為二率,對所知又一角之邊正切為三率,求得四率,命為正切,對表得度。復以所知又一角、一邊如法求之,復得度。視原所知兩角銳、鈍相同,則兩得度相加;不同,則兩得度相減;皆加減為不知之邊。乃按第一術對邊求對角,即得不知之角。原又一角鈍,對先用角之邊大於後得度,此角鈍;對先用角之邊小於後得度,此角銳。原又一角銳,對先用角之邊小於後得度,此角鈍;對先用角之邊大於後得度,此角銳。此其理系垂弧在形內與在形外之不同,及角分銳鈍,邊殊大小,前後左右俯仰向背之相應。如圖甲乙丙形,甲乙二角俱銳,兩銳相向,故垂弧丙丁,從中取正,而在形內。己丙庚形,己庚二角俱鈍,兩鈍相向,故垂弧戊丙亦在形內。庚丙乙形,庚乙兩角,一銳一鈍相違,垂弧丙丁,從外補正,自在形外。在形內者判底邊為二,兩得分邊之度,如乙丁、丁甲,合而成一底邊如乙甲,故宜相加。在形外者,引底邊之餘,兩得分邊之度,如庚丁、乙丁,重而不揜,底邊如庚乙,故宜相減。銳鈍大小之相應,亦如右圖審之。所知兩邊對所知兩角有一正,則一得度即為不知之邊,理亦自明。

六曰三邊求角,以所求角旁兩邊正弦相乘為一率,半徑自乘為二率,兩邊相減餘為較弧,取其正矢與對邊之正矢相減餘為三率,求得四率,為所求角正矢。此其理在兩次比例省為一次。如圖甲壬乙形,求甲角,其正矢為醜丁。法當以甲乙邊正弦乙丙為一率,半徑乙己為二率,兩邊較弧正矢乙癸與對邊正矢乙卯相減餘癸卯同辛子為三率,求得四率為壬辛。乃以甲壬邊正弦戊辛為一率,壬辛為二率,半徑己丁為三率,求得四率為醜丁。甲角正矢亦以乘除相報,故從省焉。

七曰三角或銳、或鈍求邊,以角為邊,反求其角;既得角,復取為邊;求、取皆與半周相減。此其理在次形,如圖甲乙丙形,甲角之度為丁戊,與半周相減為戊己,其度必同於次形子辛午之子辛邊,蓋丑卯為乙之角度丑點之交,甲乙弧必為正角,丁戊為甲之角度戊點之交,甲乙弧亦必為正角。以一甲乙而交丑辛、戊辛二弧皆成正角,則二弧必皆九十度,弧三角之勢如此也。戊辛既九十度,子己亦九十度,去相覆之戊子,己戊自同子辛,於是庚癸必同子午,卯未必同午辛,理皆如是矣。而此形之餘角既皆為彼形之邊,彼形餘角不得不為此形之邊,故反取之而得焉。若三角有一正,除正角外,以一角之正弦為一率,又一角之餘弦為二率,半徑為三率,求得四率,為對又一角之邊餘弦。此其理亦系次形,而以正角及一角為次形之角,以又一角加減象限為次形對角之邊,取象稍異。

凡茲七術,惟邊角相求,有銳鈍、大小不能定者,然推步無其題,不備列。此七題中求邊角有未盡者,互按得之。

橢圓形者,兩端徑長、兩腰徑短之圓面。然必其應規,乃可推算。作之之術,任以兩點各為心,一點為界,各用一針釘之,圍以絲線,末以鉛筆代為界之。針引而旋轉,即成橢圓形。如圖甲己午三點,如法作之,為醜午巳未橢圓,寅丑、寅巳為大半徑,寅午、寅未為小半徑,寅甲為兩心差,己甲為倍兩心差。甲午數如寅巳,亦同寅丑,己午如之;二數相和,恆與醜巳同。令午針引至申,甲申、申己長短雖殊,共數不易。甲午同大半徑之數如弦,兩心差如勾,小半徑如股,但知兩數,即可以勾股術得不知之一數。若求面積,以平方面率四00000000為一率,平圓面率三一四一五九二六五為二率,大小徑相乘成長方面為三率,求得四率為橢圓面積。若求中率半徑,大小半徑相乘,平方開之即得。然自甲心出線,離丑右旋,如圖至戌,甲丑、甲戌之間,有所割之面積,亦有所當之角度。

角積相求,爰有四術:

一曰以角求積,以半徑為一率,所知角度正弦為二率,倍兩心差為三率,求得四率為倍兩心差之端,垂線如己酉。又以半徑為一率,所知角度餘弦為二率,倍兩心差為三率,求得四率為界度積線,引出之線如甲酉,倍兩心差之端垂線為勾自乘。以引出之線,與甲戌、己戌和如巳丑大徑者相加為股弦和,除之得較。和、較相加折半為己戌弦,與大徑相減為甲戌線。又以半徑為一率,所知角正弦為二率,甲戌線為三率,求得四率為戌亥邊。又以小徑為一率,大徑為二率,戌亥邊為三率,求得四率為辰亥邊。又以大半徑寅辰同寅丑為一率,半徑為二率,辰亥邊為三率,求得四率為正弦,對表得度。又以半周天一百八十度化秒為一率,半圓周三一四一五九二六為二率,所得度化秒為三率,求得四率為比例弧線。又以半徑為一率,大半徑為二率,比例弧線為三率,求得四率為辰丑弧線,與大半徑相乘折半,為寅辰丑分平圓面積。又以大半徑為一率,小半徑為二率,分平圓面積為三率,求得四率為寅戌丑分橢圓面積。乃以寅甲兩心差與戌亥邊相乘折半,與寅戌丑相減,為甲戌、甲丑之間所割面積。此其理具本圖及平三角、弧三角,其法至密。

二曰以積求角,以兩心差減大半徑餘得甲丑線自乘為一率,中率半徑自乘為二率,甲戌、甲丑之間面積為三率,求得四率為中率面積,如甲氐亢。分橢圓面積為三百六十度,取一度之面積為法除之,即得甲戌、甲丑之間所夾角度,此其理為同式形比例。然甲亢與甲氐同長,甲戌則長於甲丑,以所差不多,借為同數。若引戌至心,甲丑甲心所差實多,仍須用前法求甲戌線,借甲戌甲心相近為同數求之。

三曰借積求積,以所知面積,如圖之辛甲丑,用一度之面積為法除之,得面積之度。設其度為角度,於倍兩心差之端如庚己丑。以半徑為一率,己角正弦為二率,倍兩心差為三率,求得四率為甲子垂線。又以半徑為一率,己角餘弦為二率,倍兩心差為三率,求得四率為己子分邊。甲子為勾自乘,己子與大徑相減餘為股弦和,除之得股弦較。和、較相加折半得甲庚線。又以甲庚線為一率,甲子垂線為二率,半徑為三率,求得四率為庚角正弦,得度與己角相加為庚甲丑角。乃用以角求積法,求得庚甲醜面積,與辛甲醜面積相減餘如庚甲辛,又用以積求角法,求得度,與庚甲丑角相加,即得辛甲丑角。

四曰借角求角,以所知面積如前法取為積度,如醜甲丁。設其度為角度,於橢圓心如丁乙辛。以小半徑為一率,大半徑為二率,所設角度正切為三率,求得四率為丁乙癸角正切。對表得度,乃於倍兩心差之端丙作丙丑線,即命丑丙甲角如癸乙丁之角度,乃將丙丑線引長至寅,使醜寅與甲丑等,則丙寅同大徑。又作甲寅線,成甲寅丙三角形,用切線分外角法求得寅角,倍之為甲丙醜形之醜角,與丙角相加為醜甲丁角。此其理癸乙甲角度多於丑甲丁積度,為子乙癸角度。即以此度當前之補算辛甲庚者,蓋所差無多也。

此四術內凡單言半徑者,皆八線表一千萬之數。圖形尚無資料


\end{pinyinscope}