\article{志六十一}

\begin{pinyinscope}
禮五(吉禮五)

宗廟之制時饗祫祭加上謚號東西廡配饗醇賢親王廟

謁陵

宗廟之制清初尊祀列祖神御,崇德建元,立太廟盛京撫近門東。前殿五室,奉太祖武皇帝、孝慈武皇后。後殿三室,奉始祖澤王、高祖慶王、曾祖昌王、祖福王,考、妣俱南鄉。並設床榻、衾枕、楎椸、帷幔,如生事儀。太宗受尊號,躬率群臣祭告,其太牢、少牢色尚黑。復嗣考祭儀,定祭品,牛一,羊一,豕一,簠、簋各二,籩、豆各十有二,爐一,鐙二,各帛一,登、鉶、尊各一,玉爵三,金匕一,金箸二。帛共篚,牲共俎。尊實酒,疏布冪勺具。階前設樂部,分左、右懸。祀日陳法駕鹵簿。

世祖定燕京,建太廟端門左,南鄉。硃門丹壁,上覆黃琉璃,衛以崇垣,周二百九十一丈。凡殿三,前殿十一楹,階三成,陛皆五出。一成四級,二成五級,三成中十一,左、右各九。中奉太祖、太后神龕。中殿九楹,同堂異室,奉列聖、列後神龕。後界硃垣,中三門,左、右各一。為後殿,亦九楹,奉祧廟神龕,俱南鄉。前殿兩廡各十五楹,東諸王配饗,西功臣配饗。東廡前、西廡南燎爐各一。中後殿兩廡庋祭器。東廡南燎爐一。戟門五,中三門內外列戟百二十,左、右門各三。其外石梁五。橋北井亭三,南神庫、神廚。西南奉祀署,東南宰牲亭。其盛京太廟尊為四祖廟云。

順治四年,定盛京守廟首領馬法秩視拖沙喇哈番,餘馬法視護軍校。

五年冬,追尊澤王為肇祖,慶王為興祖,昌王為景祖,福王為顯祖,與四后並奉後殿,致祭如時饗儀。

八年,孝端文皇后祔廟,奉神主祗見太祖、太后暨太宗,代行三跪九拜禮,位次太宗,復一跪三拜。畢,遂行大饗。祀後殿則遣官。凡升祔,先一日遣告,至日祗見、奉安、大饗,著為例。十八年,世祖祔廟,位次太祖西旁,東鄉。康熙九年,孝康章皇后祔廟,位次世祖。二十七年,孝莊文皇后祔廟,屆期世祖及章後神主避立於旁,始行祗見禮,位次文後。凡祔廟主,以卑避尊,後仿此。五十七年,孝惠章皇后升祔,議者以孝康祔廟久,欲位其次。大學士王掞議曰:「陛下聖孝格天,曩時太皇太后祔廟,不以躋孝端上,今肯以孝康躋孝惠上乎?」議者不從,帝果以為非是,令改正焉。

雍正元年,禮臣言:「古帝王升祔太廟,必以皇后配饗。周祀閟宮,漢於別寢,唐、宋有坤儀、奉慈殿以展孝思。自是配廟者,皇后字上一字與廟謚同,祀別廟者,但有謚無廟號。其配位或一帝一後,或一帝二後。宋太宗、徽宗則四後先後升祔,禮制不同。本朝太祖三後,唯孝慈祔廟稱高後,太宗二後,孝端、孝莊並稱文後,世祖三後,孝惠、孝康並稱章後,孝獻但祀孝陵饗殿,定制然也。今聖祖祔廟,仁孝作配,允宜同饗。第廟謚曰仁,與尊謚衣復,改題孝誠,與孝恭體備母儀,並宜同祔。其孝昭、孝懿,應集廷臣詳議。」尋議定:「夏、商逮六朝,皆一帝一後,唐睿宗二後,宋太祖三後,太宗四後。祔廟之制,硃子諸儒咸無異說。謹按前典,孝昭、孝懿應與孝誠、孝恭並稱仁皇后,同祔太廟。」從之。

案儀,一元後,一繼立,一本生,並列如序。首孝誠,次孝昭,次孝懿,次孝恭。於此奉帝、後神主,以次安東旁,西鄉,位次太宗。

乾隆二年,世宗暨孝敬後祔廟,位西旁,東鄉,居世祖次。四十二年,孝聖後升祔,次孝敬。

明年,高宗詣盛京,徙建四祖廟大清門東,南北袤十一丈一尺五寸,東西廣十丈三尺五寸。正殿五楹,東、西配廡各三楹。正門三,東、西門各一。敕大臣監視落成。

嘉慶四年,高宗暨孝賢、孝儀二後祔廟,位東旁,西鄉,次聖祖。道光元年,仁宗暨孝淑後祔廟,位西序,東鄉,次世宗。

三十年,宣宗遺諭及祔廟事,略謂:「禮經天子七廟,周禮小宗伯辨廟祧昭穆,漢七廟六室,唐九代十一室,宋九世十二室,議禮紛紛,不一而足。我朝首太祖訖仁宗,巍然七室,不參酌今古,必至貽笑後嗣。朕薄德承基,何敢上擬祖考,祔廟斷不可行。其奉先殿、壽皇殿、安佑宮為古原廟,制可仍舊。」乃下廷臣議,於是禮親王全齡等主遵成憲。侍郎曾國籓亦言:「萬難遵從。古者祧廟,為七廟親盡言,有親盡不祧者,則必世德作求,不在七廟數。若殷三宗,周文、武是也。大行皇帝於皇上為禰廟,非七廟親盡比,而功德彌綸,又當與列祖、列宗同為百世不祧之室。且諸侯大夫尚有廟祭,況尊如天子,敢廢祔典?」帝俞其請。詔曰:「天子七廟,特禮之常制,非合不祧之室言也。皇考祔廟稱宗,於制為允。」遂於咸豐二年,奉宣宗暨孝穆、孝慎、孝全三後祔廟,位東序,西鄉,次高宗。明年,奉孝和睿皇后升祔,次孝淑。

文宗少時為康慈太后撫育,十一年帝崩,穆宗體大行遺志,上尊謚曰孝靜。同治建元,祔廟次孝全。四年,文宗暨孝德後祔廟,位西序,東鄉,次仁宗。於時太廟中殿,九楹咸序。

洎穆宗崩御,而祔次尚虛。光緒三年,惇親王奕脤等躬往相度,集議所宜。侍講張佩綸請仿殷、周制,立太宗世室,百世不祧。展後殿旁垣左右各建世室。侍郎袁保恆謂周制世室在太祖廟旁,居昭穆上,後世同堂異室,以近祖為尊。請以中殿太祖左右為世室九楹,東西各展兩楹,別建昭穆六代親廟。太祖居中,兩旁各六楹,為左右世室。太祖至穆宗同為百世不祧,不必俟親盡遞升。其左右隙地,更建兩廟,各三楹,為三昭三穆,循次繼入,藉省遷移。鴻臚寺卿徐樹銘言:「古者廟前寢後,廟以祭饗,今前殿是;寢以藏衣冠,今中殿、後殿是。茲所當議者,藏衣冠寢殿耳。應就中殿左建寢殿,祭饗仍在前殿。列祖、列宗,百世不祧,若建世室後殿旁,反嫌居太祖上。唯增寢室,則昭穆序矣。」其他條議,大率主世室者多。有謂後殿宜增殿宇,移四祖神主其中。改為世室,移太宗居中一室。穆宗祔廟,奉安中殿西第四室者,通政使錫珍說也。有謂中殿兩旁建世室,東二西一,中奉太祖主,七廟東一廟奉太宗,二廟奉聖祖;西一廟奉世祖。前殿兩旁建六親廟,世宗以下奉之,斯昭穆不紊。少詹事文治說也。有謂中殿兩旁建昭穆二世室,但建方殿,縱橫各五楹,移太宗居昭世室,世祖居穆世室,皆北面中一楹。聖祖居昭世室,東面第一楹。中殿仍奉太祖。昭穆各四楹,列聖神位依序上移。穆宗升祔,居昭第三楹。司業寶廷說也。已,閣議以紛更廟制,未可從。

禮親王世鐸等謂:「與其附會古典,不如恪守成規。太廟中殿九楹,中楹仍舊,東西各四楹,請如道光初故事,增修改飾。東次楹又次楹為昭位,太宗暨二后、聖祖暨四后、高宗暨二后、宣宗暨四后神主序焉。西次楹又次楹為穆位,世祖暨二后、世宗暨二后、仁宗暨二后、文宗、孝德後神主序焉。將來穆宗、孝哲後升祔,位居宣宗次。」議上,醇親王奕枻韙之,奏言:「寓尊崇於變通,較諸說為當。第廟楹有限,國統無窮,增修尚非至計。祧廟為歷朝經制,無可避忌。請敕自今以往,毋援百世不祧之文,當循親盡則祧之禮,庶鉅典與天地常存。」於時徐樹銘力主宣宗遺諭,以漢、唐增室為非,今用奉先殿增龕成案,億萬年後,勢難再加。宜遵祖訓,豫定昭穆。內閣學士鍾佩賢亦以為言,鴻臚寺少卿文碩且請建穆宗寢廟,而文治、寶廷尤力爭並龕簡陋,非永制。兩宮太后不獲已,再下王大臣議,兼詢直隸總督李鴻章。鴻章言:「周官,匠人營國,世室、明堂,皆止五室。鄭注,五室並在一堂。據此,則硃子所圖世室、親廟以次而南,未盡合制。至建寢殿、增方殿,古制所無,禮親王等所言,未為無見。我朝廟制,祖宗神靈,協會一室,一旦遷改,神明奚安?太廟重垣,庭墀殿陛,各有恆式。準古酌今,改廟非便。因時立制,自以援奉先殿增龕例為宜。議者或嫌簡略,準古禮祔廟迭遷,亦止改塗易簷,並不大更舊廟。今之龕座,猶晉、宋時坎室,晉華垣建議廟堂以容主為限,無拘常數。王導、溫嶠往復商榷,始增坎室。宋增八室,蔡襄為圖。今之增龕,何以異是?」又謂:「奉先殿即古原廟,與太廟殊。然雍正時奏定奉先殿神牌與太廟■A2若畫一。成憲可循,不得謂增龕之制獨不可仿行太廟也。至祧遷雖常典,而藏主之室,禮無明文。鄭康成言周祧主藏於太廟及文武世室,是已祧之主與不遷之祖同處一廟,故廟亦名祧。晉藏西儲夾室,當時疑其非禮,後世緣為故事。儒家謂古祧夾室,殆為肊辭。廟既與古不同,祧亦未容輕議。唯醇親王所陳,為能導皇上以大讓,酌廟制以從宜。」自此議遂定。

五年,穆宗暨孝哲後祔廟,位東序,西鄉,次宣宗。七年,孝貞後升祔,次孝德。宣統元年,孝欽後升祔,次孝貞。是歲考議德宗祔廟事,禮臣言:「兄弟同昭穆,但主穆位空一室。」其餘議禮諸臣,重宗統者,以為異昭穆不便;重皇統者,復以為同昭穆不合。而大學士張之洞獨主:「古有祧遷之禮,則兄弟昭穆宜同。今無祧遷之禮,則兄弟昭穆可異。」議乃定。其秋,詔曰:「我朝廟制,前殿自太祖以下七世皆南鄉,宣宗以下三世分東西鄉,與古所謂穆北鄉、昭南鄉不同。穆、德二廟,同為百世不祧,宜守硃子之說,以昭穆分左右,不以昭穆為尊卑。禮緣義起,毋因經說異同,過事拘執。德宗祔廟,中殿奉西又次楹又五室穆位,前殿位次西旁文宗坐西鄉東穆位。體先朝兼祧之旨,慰列聖在天之靈,垂為定制。奉先殿位序亦如之。」

時饗太宗建國初,遇清明、除夕,躬謁太祖陵,即時饗所由始。崇德元年,建太廟成,凡四孟時饗,每月薦新,聖誕、忌辰、清明、中元、歲暮俱致祭。五月獻櫻桃,命薦太廟。凡新進果穀,皆先薦乃進御,著為令。順治元年,定時饗制,孟春擇上旬日,三孟用朔日,樂章六奏。二年,命祭太廟如奉先殿儀,讀祝、致祭。遣官祭福陵、昭陵、四祖廟,止上香燭、供酒果,不讀祝。七月朔,秋祭太廟、四祖廟,中元祭陵,並用牛、羊。尋定四祖廟祭例視京師,牲用生。又饗太廟用熟牛,罷晉胙。八年,定親饗制,飲福、受胙如圜丘。奏樂備文、武佾舞。康熙十二年,從禮臣言,祭太廟,質明將事。二十四年春,親饗畢,諭曰:「往見贊禮郎宣祝,至朕名,聲不揚。禮稱父前子名,子孫通名祖父,豈可慢易?嗣後垂為戒。」

雍正十一年,世宗以廟饗無上香,奠帛、爵無跪獻,命大學士禮臣議增。尋議言:「大祀莫重郊壇,孝享莫大配天。宗廟典禮,宜視社稷。祭社稷日,皇帝親詣上香,太廟自宜一例。至帛、爵俱不親獻,皇帝立拜位前,所以亞郊壇也。仍舊儀便。」報可。

又定太廟神牌如奉先殿制,供奉居中。請牌用太常官,獻帛、爵用侍衛,尋改用宗室官。

高宗嗣位,定三年持服內,饗廟御禮服作樂如故,唯齋戒用素服,冠綴纓。乾隆二年,用禮臣言,祝版書列聖尊謚。香帛送燎時行中路,帝轉立東旁,俟奉祝帛官出,復位,如祀郊壇式。尋定每日上香,守廟官行禮。朔望用太常官。嗣改宗室王公番行。十二年,諭太廟獻帛、爵用宗室官,俾習禮儀、鎔氣質。敕宗人府王公監視,後復定後殿獻帛、爵用覺羅官。

向例,饗廟,帝乘輿出宮,至太和門外改乘輦。入街門,至神路右,步入南門,詣戟門幄次。入升東階,進前殿門,就拜位。禮成,出如初。凡入門皆左。三十七年,帝年漸高,略減儀節。入廟時,改自闕左門輦入西北門,至廟北門外,輿入。至戟門外東階下。步入門,升階進殿。行禮畢,出亦如之。

嘉慶四年,定時饗前殿座次。太祖、太宗、世祖皇考、妣皆南鄉,聖祖皇考、妣東位西鄉,世宗皇考、妣西位東鄉,高宗皇考、妣東次西鄉。以後帝、後位次仿此。八年孟春時饗,禮臣卜吉初六日,仁宗以前三日致齋。會逢高宗忌辰,服色未協,命改初八日。嗣是春饗皆擇正月初八、九、十等日行之。

道光四年,諭廟饗謝福胙如祀社稷儀,王公百官隨行三跪九拜禮。穆宗、德宗初立,時饗、祫祭遣親王代,逮親政始躬蒞。宣統朝攝政王攝行。

祫祭歷代禘、祫分祭,禮說繽紛,罔衷古訓。清制有祫無禘。除夕饗廟,實始太宗,世祖本之,著為祭典。順治十六年,左副都御史袁懋功請舉祫祭,以彰孝治。乃定歲除前一日大祫,移後殿、中殿神主奉前殿。四祖、太祖南鄉,太宗東位西鄉。先一日遣官告後殿、中殿,致齋視牲。屆日世祖親詣,禮如時饗,自是歲以為常。尋定祫祭樂舞陳殿外。

康熙時,御史李時謙請行禘祭。禮臣張玉書上言:「考禮制言禘不一,有謂虞、夏禘黃帝,殷、周禘嚳,皆配祭圜丘者;有謂祖所自出為感生帝,而祭之南郊者;有謂圜丘、方澤、宗廟為三禘者:先儒皆辯其非。而宗廟之禘,說尤不一。或謂禘止及毀廟,或謂長發詩為殷禘,雍詩為周禘,而親廟、毀廟兼祭者。唯唐趙匡、陸淳以為禘異於祫,不兼群廟。王者立始祖廟,推祖所自出之帝,以始祖配之,故名禘。至三年一祫,五年一禘,說始漢儒,後人宗之。漢、唐、宋禘禮,並未考始祖所自出,止五歲中合群廟之祖,行祫禘於宗廟而已。大抵夏、商以前有禘祭,而厥制莫詳。漢、唐以後有禘名,而與祫無別。周以後稷為始祖,以帝嚳為所自出,而太廟中無嚳位,故祫祭不及。至禘祭乃設嚳位,以稷配焉。行於後代,不能盡合。故宋神宗罷禘禮。明洪武初或請舉行,眾議不果。嘉靖中,乃立虛位,祀皇初祖帝,以太祖配,事涉不經,禮亦旋罷。國家初定鼎,追上四祖尊稱,立廟崇祀,自肇祖始。太祖功德隆盛,當為萬世廟祖,而推所自出,則締造大業,肇祖最著。今太廟祭禮,四孟分祭前、後殿,以各伸其尊。歲暮祫饗前殿,以同將其敬。一歲屢申祼獻,仁孝誠敬,已無不極。五年一禘,可不必行。」遂寢其議。

乾隆三十七年大祫,帝親詣肇祖位前上香,餘遣皇子親王分詣,復位行禮如常儀。詣廟節文減之如時饗。六十年將屆歸政,九廟俱親上香。嘉慶四年,定歲暮祫祭,前殿座位視時饗。咸豐八年,文宗疾甫平,親王代行祫祭,然先祭時猶親詣拜跪焉。其因時祫祭者,古禮天子三年喪畢,合先祖神饗之,謂之吉祭。雍正二年,吏部尚書硃軾言:「皇上至仁大孝,喪三年如一日,今服制竟,請祫祭太廟,即吉釋哀。」制可。明年二月,帝詣廟行祫祭,如歲暮大祫儀。自後服竟行祫祭仿此。

加上謚號崇德元年,太宗受尊號,追封始祖為澤王、高祖慶王、曾祖昌王、祖福王,上太祖武皇帝、孝慈皇后尊謚。即日躬祀太廟。翼日,百官表賀。順治元年,進太祖、孝慈後、太宗玉冊、玉寶,奉安太廟。冊長八寸八分,廣三寸九分,厚四分。冊數十,面底二頁鐫升降龍。寶方四寸二分,厚一寸五分,紐高二寸七分,長四寸二分,廣三寸五分,寶盝金質。凡太廟冊、寶皆用玉,色青白,冊文用驪體,寶文如謚號,曰「某祖某宗某皇帝之寶」,後曰「某皇后之寶」。

五年,追尊澤王肇祖原皇帝,妣原皇后;慶王興祖直皇帝,妣直皇后;昌王景祖翼皇帝,妣翼皇后;福王顯祖宣皇帝,妣宣皇后。奉安訖,致禮如時饗。越三日,慶賀如儀。七年,上孝端文皇后尊謚。九年,進四祖帝後冊寶。十八年,上世祖尊謚,前期齋戒,遣官祭告天地、宗廟、社稷。

屆日,帝素服御太和門,閱冊、寶訖,大學士奉安採亭,校尉舁行,導以御杖,駕從之。王公百官各於所立位跪俟,隨行。至壽皇殿大門外降輦,入左門,採亭入右門。大學士二人跪奉冊寶陳案上,帝就位,率群臣行三跪九叩禮。贊引奏「跪」,奏「進冊」,奉冊大學士跪左,進帝跪獻。畢,授右跪大學士陳中案。奏「進寶」,如初。奏「宣冊」,宣冊官跪宣:「上尊謚曰體天隆運英睿欽文大德弘功至仁純孝章皇帝,廟號世祖。」宣冊訖,奏「宣寶」,儀亦如之。行禮三跪九叩,致祭同時饗。畢,奉絹冊、寶、祝帛如燎所焚之。大學士二人,奉香冊、寶導梓宮奉安,一跪三叩,翼日頒詔天下。凡上大行帝後尊謚,香冊、香寶獻幾筵後,奉安山陵,絹冊、寶送燎,玉冊、玉寶卜吉藏之太廟,後仿此。

初太祖尊謚曰承天廣運聖德神功肇紀立極仁孝武皇帝,太宗曰應天興國弘德彰武寬溫仁聖睿孝文皇帝。聖祖纘業,加太祖「睿武弘文定業」六字,更廟號高皇帝;太宗「隆道顯功」四字,廟號如故。用禮臣言,俟世祖祔饗後行禮。明年,上慈和皇太后尊謚。二十七年,上孝莊太皇太后尊謚。五十七年,上孝惠皇太后尊謚,後,聖祖嫡母也。祔廟日,命安神位慈和上。

六十一年冬,世宗諭廷臣:「皇考繼統,本應稱宗,但經云:祖有功,宗有德。皇考手定太平,論繼統為守成,論勛業為開創,宜崇祖號,以副豐功。其確議之。」議言:「按禮經:有虞氏禘黃帝而郊嚳,祖顓頊而宗堯。而舜典云:舜格文祖。注曰堯廟。歸格藝祖,復釋為堯之祖。合之祖顓頊,則有三祖矣。宋陳祥道云:凡配天者皆得稱祖。國語展禽謂有虞氏祖高陽而郊堯,堯所以稱文祖也。顓頊至堯,並黃帝子孫,故皆稱祖。又周禮大宗伯:祫、禘、追享、朝享。解云:古者朝廟合群祖而祭焉,故祫曰朝饗,以合群祖為不足,復禘其所自出,故禘曰追饗。夫祖所自出,始祖也,其下曰群祖,則自始祖以下皆可稱祖矣。」又謚議:「帝王功業隆盛,得援祖有功古義稱為祖。竊謂唯聖可揚峻德,唯祖可顯隆功。」議上,稱旨。雍正初元,遂上尊謚,廟號聖祖。復諭:「太祖、太宗、世祖三聖相承,功高德盛;孝莊、孝康、孝惠翼運啟期,懿徽流慶;宜並加謚,俾展孝思。」於是加謚太祖曰端毅,太宗曰敬敏,世祖曰定統建極,而孝慈、孝端及三後並尊謚焉。

於時工部奉神主廟室,魨漆飾金,中書、翰林官各一人書新謚。奏遣大學士二人行填青禮,先期祗告天地、社稷。至日,世宗禮服詣太廟行上尊謚禮。畢,還宮,易袞服,詣奉先殿致祭,後仿此。六年,鐫造列聖、列後玉寶、玉冊暨聖祖皇考、妣冊、寶成,奉之太廟。其儀,太廟潔室設黃案,張採幔兩旁,中陳冊、寶,王大臣朝服將事,帝御禮服恭閱,一跪三拜,安奉採亭,輿導如前儀。供案訖,帝入行禮如初。冊、寶集中殿,分藏金匱。帝以次上香,一跪三拜,禮成。

高宗踐阼,加列聖、列後尊謚,諭言:「宗廟徽稱有制,報本忱悃靡窮。藉抒至情,不為恆式。」

乾隆四十五年,以列朝冊、寶玉色參差,命選工琢和闐精璆。越二年工竣,祗閱訖,奉太廟如禮。其舊藏十六分,命賚送盛京太廟,尊藏玉檢金繩。自是帝、後祔廟,皆別備冊、寶送盛京,永為制。

嘉慶四年,仁宗守遺訓,著制,凡列聖尊謚已加至二十四字、列後尊謚已加至十六字不復議加。

功臣配饗,所以顯功,宗親郡王配東廡,文武大臣配西廡。崇德元年,追封皇伯祖禮敦巴圖魯為武功郡王,巴圖魯其名也,配東廡,福晉與焉。並以直義公費英東、弘毅公額亦都配西廡。順治元年,西廡增祀武勛王揚古利,位直義上。九年,復增祀忠義公圖爾格、昭勛公圖賴,昭勛為直義子,忠義為弘毅子,父子配侑,世尤榮之。十一年,東廡增祀通達郡王雅爾哈齊、慧哲郡王額爾袞、宣獻郡王界堪,通達位武功上,而慧哲、宣獻兩福晉亦並侑云。

康熙九年,定祀東廡用太牢,歲以為常。

雍正二年,西廡增祀文襄公圖海。定功臣配饗儀,前期告太廟。屆日陳採亭,列引仗,奉主至廟西階。拜位在階下,三跪九拜。奉主大臣攝行,還納龕位,一跪三拜。

八年,怡親王允祥配東廡。定王配饗儀,奉主以郡王,迎主用採亭吾仗,至廟東階,拜位在階上。代行禮畢,降自東階,餘如西廡。

九年,進加費英東信勇公,圖爾格果毅公,圖賴雄勇公,圖海忠達公。乾隆中,西廡增祀襄勤伯鄂爾泰,超勇親王策凌,大學士張廷玉,蒙古王、漢大臣侑食自此始。

四十三年,詔:「祖宗創業艱難,懿親藎臣,佐命殊功,從古未有。當時崇封錫爵,酬答從優。以後有及身緣事降削者,有子孫承襲易封者,不為追復舊恩,心實未愜。」於是睿親王多爾袞以元勛懿戚,橫被流言,特旨昭雪。禮烈親王代善,後人改封為巽,已復改為康,鄭獻親王濟爾哈朗改為簡,豫通親王多鐸改為信,肅裕親王豪格改為顯,克勤郡王岳託改為衍禧,又改為平,均非初號。悉命復舊,並配祀東廡。禮王位宣獻下,睿王等以次列序,位怡王上,而徙策凌列怡王次。

嘉慶元年,西廡增大學士傅恆、福康安、協辦大學士兆惠。福康安即傅心互子,並封郡王,異姓世臣,被恩最渥。

道光三年,復增大學士阿桂,功臣凡十有二人。

同治四年,東廡增科爾沁親王僧格林沁,功王凡十有三人。

凡時饗,帝上香時,分獻官上香配位前,各分獻不拜。三獻畢,退。祫祭同。

醇賢親王廟光緒十六年,醇賢親王奕枻薨,中旨引高宗濮議辨,應稱所生曰「本生父」,沒稱「本生考」,立廟不祧,祀以天子之禮,合乎「父為士,子為大夫,葬以士,祭以大夫」古義,斯尊親兩全矣。乃定稱號曰「皇帝本生考」。復定廟祀典,建廟新賜邸第,額曰醇賢親王廟。正殿七楹,東、西廡殿,後寢室,各五楹。中門三。門內焚帛亭、祭器亭,其外宰牲亭、神庫、神廚。大門三。殿宇正門中覆黃琉璃,殿脊及門四周上覆綠琉璃。其祀儀、樂舞、祭器、祭品視天子禮。凡時饗以四仲月朔,襲王承祭。帝親行,則襲王陪祀。誕辰、忌日,帝親詣行禮。

謁陵有清肇跡興京,四祖陵並在京西北,稱興京陵。太祖定遼陽,景祖、顯祖二陵徙盛京東南,稱東京陵。嗣是太祖陵當盛京東北,稱福陵;太宗陵當盛京西北,稱昭陵。崇德間,定歲暮、清明祭興京陵,用牛一,遣守陵官行禮。東京陵用牛二,遣宗室、覺羅大臣行禮。福陵用牛一、羊二,遣大臣行禮。國忌、誕辰、孟秋望日,燃香燭,獻酒果,奠帛、讀祝、行禮。朔、望用牛一,具香燭、酒果,遣守陵官致祭,不讀祝、奠帛。

順治八年,封興京陵山曰啟運,東京陵山曰積慶,福陵山曰天柱,昭陵山曰隆業,並從祀方澤,置陵官、陵戶。定祀儀,冬至用牛一、羊一、豕一,餘同前。清明、歲暮、孟秋望日亦如之。十三年,詔立界碑,禁樵採。十五年,移東京陵改祔興京,罷積慶山祀。明年,尊稱為永陵,饗殿、暖閣如制。

康熙二年,相度遵化鳳臺山建世祖陵,曰孝陵。先是世祖校獵於此,停轡四顧曰:「此山王氣蔥鬱,可為朕壽宮。」因自取佩鞢擲之,諭侍臣曰:「鞢落處定為穴。」至是陵成,皆驚為吉壤。歲以清明、中元、冬至、歲暮為四大祭。並改建福陵、昭陵地宮。工竣,以奉安祗告,致祭如大饗。安神位隆恩殿,制龕座、寶床、帷幔、衾褥、楎椸如太廟式。

凡因公謁陵,三品以上官羅城門外行禮。遇祭日,二品以上許入城隨守陵官陪祭。歸,謁辭。

凡謁陵,東逕石門,王、貝勒在隆恩門外三跪九拜,當直官啟門,貝子以下、三品官以上則否,皆奉祀官為導,遇祭日免。是時三陵建功德碑,嗣凡起陵,皆立碑,如故事。

八年,定四時大祭,遣多羅貝勒以下,奉國將軍、覺羅男以上行禮。

明年秋,奉太皇太后、皇太后率皇后謁孝陵。前一日,躬告太廟,越日啟鑾。陳鹵簿,不作樂。既達陵所,太皇太后坐方城東旁,奠酒舉哀。皇太后率皇后等詣明樓前中立,六肅、三跪、三拜,隨舉哀、奠酒,復三拜。還行宮。凡皇太后謁陵仿此。次日,帝復謁隆恩殿,行大饗禮。又次日,殿前設黃幄,焚楮帛,讀文致祭,禮成。還京,仍告太廟。越二日,御太和殿,百官表賀。

明年秋,車駕至盛京,謁福陵、昭陵畢,召將軍等賜以酒,並諭守陵總管、副總管曰:「爾等職司典祀,凡祭品必親虔視,務盡誠敬,副朕孝思。」還御大清門受賀,燕賚群臣,頒守陵官軍。其永陵遣王大臣致祭,復遣官分詣穎親王、克勤郡王、直義公費英東諸勛貴墓酹酒。還京日,仍告廟如儀。

二十一年,滇平,詣兩京謁陵,如初禮。還京,祗告奉先殿。自是靖寇難,謁陵告祭以為常。

六十年,御極周甲,命世宗率皇子、皇孫詣盛京,皇子祭昭陵,皇孫祭永陵,帝親往福陵大祭。

雍正元年,定聖祖陵曰景陵。其明年,清明謁祭如典。八年冬至,會聖祖忌辰,禮臣言準陵寢大祭,用太牢,獻帛、爵,讀祝文。遣官承祭具朝服。十三年清明、冬至大饗,改遣公爵番行。七月望日,將軍、侍郎等承祭,其朔、望、忌辰,則定總管掌關防承祭,行三跪九叩禮。

乾隆元年,命宗室輔國將軍等六人徙駐沈陽,給田廬,歲時致祀。二年,諭改朔、望承祭貝勒、公、大臣番行。復慮儀節不齊,增贊禮郎二人導引退,仍不贊。三年清明,謁世宗泰陵。

六年,定三陵四時大饗。忌辰祭饗,題派移駐將軍二人行禮。七年,增置三陵爵墊,備禮儀。

八年,定謁陵如太廟親祀儀,載入儀注。已,奉皇太后謁祖陵,禮節準康熙時例。自後三謁皆如之。

四十三年秋,先後謁永陵、福陵,因諭:「睠懷遼沈舊疆,再三周歷,心儀舊緒,蘄永勿諼。夫奕升平景運,皆昔日艱難開創所貽。後世子孫,當覽原巘而興思,拜松楸而感悟。默念天眷何以久膺,先澤何以善繼。知守成之難,兢業無墜。庶熙洽之盛,億萬斯年。不然,輕故都,憚遠涉。或偶詣祖陵,漠不動心,視同覽古,是忘本也。盛京根本重地,發祥所自,後世不可不躬親閱歷,其毋負朕言!」

嘉慶五年清明,詣昌瑞山謁高宗裕陵,先敷土,次大饗。陵寢官豫取潔土儲筐,俟帝如更衣次易縞素,執事從官素服,冠去纓,隨至方城。有司進黃布護履,帝納履,從臣亦如之,自東磴道升至寶城石欄東,陵寢大臣合土以筐,隨駕至敷土處跪進。帝拱舉,敷畢,授筐,降,脫履。於是更袍服,冠綴纓,執事官俱易。禮臣請行大饗,帝詣隆恩殿行禮。讀祝,三獻。

凡清明日謁陵敷土,在喪服期,帝親行。十年,帝初謁永陵,御素服,詣啟運殿後階,三跪九拜,有司進奠幾,三拜三奠爵。訖,舉哀。翼日朝服行大饗。謁福陵、昭陵亦如之。後復以祭器乖誤,革盛京禮部侍郎世臣職。因諭「豐沛舊都,大臣不應忘卻」。下其諭各公署,其重祀如此。

道光八年,謁裕陵、昌陵,軍機大臣隨入門,命著為例。九年,奉皇太后詣盛京謁三陵,如儀。

咸豐元年謁東陵,五年謁西陵,孝貞皇后謁泰陵,陵寢女官為導,入門皆由左,至明樓前行禮,六肅三跪三拜。女官進奠幾,後三拜三奠爵,西饗舉哀。次謁昌陵、慕陵如初禮。同、光間悉依此行。

凡孝陵、景陵以下,世宗曰泰陵,高宗裕陵,仁宗昌陵,宣宗慕陵,文宗定陵,穆宗惠陵,並在直隸易、遵化二州,稱東西陵,東陵鳳臺山,封昌山;西陵太平峪,封永寧山;並祀方澤。設奉祀官,置莊園。

隆恩殿大饗用祝幣,其日燃明鐙,用牛一、羊二、尊四,帝、後同案位,設奉先制幣一,羹飯脯醢器十八,餅果器六十五。牲實俎,帛實筐。酒實尊,承以舟。疏布冪勺具。皇貴妃祔祀,則西旁東鄉,素帛一,減餅果十一器。

凡冬至暨慶典不舉哀。遣官祭饗用朝服。升降自西階,出入皆門右。皇子謁陵,至下馬碑降騎,至隆恩門外升左階。三跪九拜,不贊,不奠酒。

妃園寢設官如制,建饗殿,設神位。四時遣官尊酒,二跪六拜,不贊。出人殿左門。朔、望則奉祀官行禮。光緒間,帝謁西陵,詣莊順皇貴妃寢園,一跪三拜三奠酒。並諭禮臣,祭品儀節從優。是後清明、中元、冬至、忌辰遣王公致祭,餅果增至六十五器。

宣統初,德宗葬興隆峪,號崇陵。

皇太子園寢與妃園寢同。嘉慶間,帝親臨端慧皇太子園寢,三奠三爵,從臣隨行禮,每奠一拜。載其儀入會典云。


\end{pinyinscope}