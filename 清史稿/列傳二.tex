\article{列傳二}

\begin{pinyinscope}
諸王一

景祖諸子

武功郡王禮敦孫色勒慧哲郡王額爾袞宣獻郡王齋堪

恪恭貝勒塔察篇古

顯祖諸子

誠毅勇壯貝勒穆爾哈齊子襄敏貝子務達海莊親王舒爾哈齊

子阿敏鄭獻親王濟爾哈朗靖定貝勒費揚武阿敏子溫簡貝子固爾瑪琿

固爾瑪琿子鎮國襄敏公瓦三濟爾哈朗子簡純親王濟度輔國武襄公巴爾堪

濟度子簡親王喇布簡修親王雅布雅布從孫簡儀親王德沛巴爾堪子輔

國襄愍公巴賽費揚武子尚善惠獻貝子傅喇塔舒爾哈齊孫輔國公品級札

喀納鎮國公品級屯齊鎮國將軍洛託通達郡王雅爾哈齊篤義

剛果貝勒巴雅喇

有明諸籓,分封而不錫土,列爵而不臨民,食祿而不治事,史稱其制善。清興,諸子弟但稱臺吉、貝勒;既乃?明建親、郡王,而次以貝勒、貝子,又次以公爵,復別為不入八亦益廣,下此則有將軍,無中尉,又與明小異;諸王?分。蓋所以存國俗,而等殺既多,屏不錫土,而其封號但予嘉名,不加郡國,視明為尤善。然內襄政本,外領師干,與明所謂不臨民、不治事者乃絕相反。

國初開創,櫛風沐雨,以百戰定天下,繄諸王是庸。康熙間,出討三籓,勝負互見,而卒底蕩平之績。其後行師西北,仍以諸王典兵。雍正、乾隆諒闇之始,重臣宅揆,亦領以諸王。嘉慶初,以親王為軍機大臣,未幾,以非祖制罷。穆宗踐阼,輟贊襄之命,而設議政王,尋仍改直樞廷。自是相沿,爰及季年,親貴用事,以攝政始,以攝政終。論者謂有天焉,誠一代得失之林也。

今用諸史例,以皇子為宗,子孫襲爵者從焉;子孫別有功績復立爵者亦從焉。其爵世,備書之;其爵不世,則具詳於表。表曰皇子,傳曰諸王,亦互文以見義焉。自公以下,別被除拜具有事實者,及疏宗登追列高位著名績者,皆散與諸臣相次。清矯明失,宗子與庶姓並用,通前史之例以存其實也。

景祖五子:翼皇后生顯祖;諸子,武功郡王禮敦,慧哲郡王額爾袞,宣獻郡王齋堪,恪恭貝勒塔察篇古,皆不詳其母氏。

武功郡王禮敦,景祖第一子也。肇祖而下,世系始詳,事跡未備,四傳至興祖。興祖六子:長,德世庫;次,劉闡;次,索長阿;次,景祖;次,包朗阿;次,寶實:號「寧古塔貝勒」。景祖承肇祖舊業,居赫圖阿拉,德世庫居覺爾察,劉闡居阿哈河洛,索長阿居河?洛噶善,包朗阿居尼麻喇,寶實居章甲,環赫圖阿拉而城,近者五里,遠者二十里,互相。寶實子阿哈納渥濟格與董鄂部長克轍巴顏有隙,屢來侵。索長阿子吳泰,哈達萬汗?也,乞援於哈達,攻董鄂部,取數寨,董鄂部兵乃不復至。「寧古塔」亦自此稍弱。及太祖兵起,德世庫、劉闡、索長阿、寶實等子孫惎其英武,屢欲加害;其後益強大,謀始戢。索長阿、寶實子孫皆從攻戰,包朗阿云孫拜山尤有功,自有傳。帝業既成,景祖諸兄弟追封皆未及。

禮敦生而英勇,景祖討平碩色納、奈呼二部,禮敦功最,號曰「巴圖魯」。太祖兵起,禮敦卒久矣。太祖既定東京,葬景祖、顯祖於楊魯山,以禮敦陪葬。崇德四年八月,進封武功郡王,配享太廟。子貝和齊,太祖伐明,攻廣寧,留守遼陽。

孫色勒,事太祖,授牛錄額真。事太宗,自十六大臣進八大臣,授正藍旗固山額真。從太宗圍大凌河,軍城南,屢擊敗明兵。又從太宗略宣府、大同,與貝勒德格類入獨石口,敗明兵於長安嶺,攻赤城,克其郛。尋坐事,降鑲黃旗梅勒額真。崇德初,從伐朝鮮,朝鮮國王李倧遣妻子入江華島,走保南漢山城。豫親王多鐸圍之急,朝鮮將赴援,色勒與甲喇額真阿爾津擊敗之。分兵攻江華島,色勒率右翼兵渡海,越敵艦,近躍登島,破其守兵,得倧妻子。倧既降,論功,授牛錄章京世職,兼吏部右參政。順治元年,擢內大臣。錄禮敦諸孫席賚、阿濟賚、阿賚等,並授拜他喇布勒哈番。色勒進一等阿思哈尼哈番,再進二等精奇尼內大臣。卒,謚勤?。阿賚子吉哈禮,自有傳。?哈番,擢領侍

慧哲郡王額爾袞,景祖第二子。順治十年,追封謚,配享太廟。

宣獻郡王齋堪,景祖第三子。當族人與太祖構難,齋堪與額爾袞皆不與。順治十年,追封謚,配享太廟。

恪恭貝勒塔察篇古,景祖第五子。順治間,追封謚。天聰九年,詔德世庫等子孫以「覺羅」為氏,系紅帶。乾隆四十年,詔國史館:「禮敦等傳列諸臣之首,以別於宗室諸王。」

顯祖五子:宣皇后生太祖、莊親王舒爾哈齊、通達郡王雅爾哈齊;繼妃納喇氏生篤義剛果貝勒巴雅喇;庶妃李佳氏生誠毅勇壯貝勒穆爾哈齊。

誠毅勇壯貝勒穆爾哈齊,顯祖第二子。驍勇善戰,每先登陷陣。歲乙酉,從太祖伐哲還,留八十人:被棉甲者五十,被鐵甲者三十,行略地。加哈人蘇枯賴?陳部,值大水,遣虎密以告,於是托漠河、章甲、把爾達、撒爾湖、界凡五城合兵禦我。後哨章京能古德馳告,上出他道,弗遇。上深入,遙望見敵兵八百餘,陣渾河至於南山。包朗阿孫札親桑古里懼敵,解其甲授人,上呵之。穆爾哈齊及左右顏布祿、兀浚噶從上馳近敵陣,下馬奮擊,射?殺二十餘人,敵渡渾河走。穆爾哈齊復從上躡敵後,至吉林崖,遙見敵兵十五自旁徑來。上去胄上纓,隱而待,射其前至者,貫脊殪。穆爾哈齊復射殪其一,餘皆墜崖死。上曰:「今日以四人敗八百人,天助我也!」穆爾哈齊屢從征伐,賜號青巴圖魯,譯言「誠毅」。天命五年九月,卒,年六十。上臨祭其墓。順治十年,追封謚。

子十一,有爵者六:達爾察、務達海、漢岱、塔海、祜世塔、喇世塔。達爾察、塔海、祜世塔、喇世塔皆封輔國公,達爾察謚剛毅,喇世塔謚恪僖。

襄敏貝子務達海,穆爾哈齊第四子。事太宗,授牛錄章京。崇德元年,從睿親王多爾袞伐明,攻沙河、南和及臨洺關、魏縣並有功。三年,授刑部左參政。從貝勒岳託敗明兵開平,復偕固山額真何洛會等敗明兵沙河、三河,又敗之渾河岸,至趙州。復攻山東,克臨清、安丘、臨淄。還次密雲,俘四千餘。五年,授鑲白旗滿洲梅勒額真。從攻錦州,夜略杏山、塔山。七年,擢刑部承政。從伐明,分軍略登州,未至先歸,坐奪俘獲入官。順治元年,從定京師,逐李自成至延安,城兵夜出,擊破之。復從豫親王多鐸徇江南。三年,又從討蘇尼特騰機思,敗土謝圖汗、碩雷汗援兵。五年,偕固山額真阿賴等戍漢中。累進爵,自三等輔國將軍至貝子。六年,偕鎮國公屯齊哈、輔國公巴布泰代英親王阿濟格討叛將姜瓖。八年,攝都察院事。十一年,從鄭親王世子濟度討鄭成功,中道疾作,召還。十二年,卒,謚襄敏。

務達海子托克托慧,封鎮國公。托克托慧子揚福,事聖祖,官黑龍江將軍久,聖祖屢稱之,命襲不入八分鎮國公。卒,謚襄毅。揚福子三官保,聖祖褒其孝,即命繼揚福署黑龍江將軍,襲爵。

漢岱,穆爾哈齊第五子。事太宗,與務達海同授牛錄章京。崇德六年,從上圍松山,擊破明總兵吳三桂、王樸。七年,從貝勒阿巴泰伐明,攻薊州、河間、景州,進克兗州,即軍前授兵部承政。順治元年,從入關擊李自成,又從多鐸西征,破自成潼關。二年,與梅勒額真伊爾德率兵自南陽趨歸德,克州一、縣四;渡淮克揚州。賜金二十五兩、銀千三百兩。三年,授鑲白旗滿洲固山額真,與貝勒博洛徇杭州,進攻臺州,擊明魯王以海。分兵略福建,攻分水關,破明唐王將師福,入崇安,斬所置巡撫楊文英等,下興化、漳州、泉州。五年,從貝子屯齊將兵討陜西亂回。亂定,與英親王阿濟格合軍討叛將姜瓖。六年,從巽親王滿達海克朔州、寧武。移師攻遼州,下長留、襄垣、榆社、武鄉諸縣。七年,授吏部尚書、正藍旗滿洲固山額真。八年,調刑部。累進爵,自一等奉國將軍至鎮國公。九年,復調吏部。從定遠大將軍尼堪下湖南,尼堪戰沒,坐奪爵。十二年,復授吏部尚書,加太子太保,授鎮國將軍品級。十三年四月,坐依阿蒙蔽,奪官爵。卒。

漢岱子海蘭、席布錫倫、嵩布圖,均封輔國公。海蘭謚?厚,席布錫倫謚悼敏,嵩布圖謚懷思。

莊親王舒爾哈齊,顯祖第三子。初為貝勒。蜚悠城長策穆特黑苦烏喇之虐,原來附。扈爾漢、納齊布,將?太祖令舒爾哈齊及貝勒褚英、代善,諸將費英東、揚古利、常書,侍三千人往迎之。夜陰晦,軍行,纛有光,舒爾哈齊曰:「吾從上行兵屢矣,未見此異,其非吉兆耶?」欲還兵,褚英、代善不可。至蜚悠,盡收環城屯寨五百戶而歸。烏喇貝勒布占泰發兵萬人邀於路,褚英、代善力戰破之。舒爾哈齊以五百人止山下,常書、納齊布別將百人從焉。褚英、代善既破敵,乃驅兵前進,繞山行,未能多斬獲。師還,賜號達爾漢巴圖魯。既,論常書、納齊布止山下不力戰罪,當死。舒爾哈齊曰:「誅二臣與殺我同。」上乃宥之,罰常書金百,奪納齊布所屬。自是上不遣舒爾哈齊將兵。舒爾哈齊居恆鬱鬱,語其第一子於人哉?」移居黑扯木。上怒,誅其二子?阿爾通阿、第三子扎薩克圖曰:「吾豈以衣食受。舒爾哈齊乃復還。歲辛亥八月,薨。順治十年,追封謚。子九,有爵者五:阿敏、圖倫、寨桑武、濟爾哈朗、費揚武。以歸。歲癸醜?阿敏,舒爾哈齊第二子。歲戊申,偕褚英伐烏喇,克宜罕山城,俘其,上伐烏喇,布占泰以三萬人拒,諸將欲戰,上止之。阿敏曰:「布占泰已出,舍而不戰,將奈何?」上乃決戰,遂破烏喇。天命元年,與代善、莽古爾泰及太宗並授和碩貝勒,號「四大貝勒」,執國政。阿敏以序稱二貝勒。四年,明經略楊鎬大舉來侵,阿敏從上擊破明兵薩爾滸山。復御明總兵劉綎於棟鄂路,代善等繼之,陣斬綎。還擊明將喬一琦,一琦奔固拉庫崖,與朝鮮將姜弘烈合軍。阿敏攻之,弘烈降。一琦自經死。尋從上破葉赫。六人,從上取沈陽、遼陽。鎮江城將陳良策叛附明將毛文龍,阿敏遷其民於內地。文龍屯兵朝鮮境,阿敏夜渡鎮江,擊殺其守將,文龍敗走。十一年,伐喀爾喀巴林部,取所屬屯寨。伐扎嚕特部。?,俘其

天聰元年,太宗命偕貝勒岳託等伐朝鮮,瀕行,上命並討文龍。師拔義州,分兵攻鐵山,文龍所屯地也,文龍敗走。進克定州,渡嘉山江,克安州、平壤。復進次中和,朝鮮國王李倧使迎師。阿敏與諸貝勒答書數其罪有七,力拒之。師復進次黃州,倧復遣使來。阿敏欲遂破其都城,諸貝勒謂宜待其大臣至,蒞盟許平。總兵李永芳進曰:「我等奉上命仗義而行,前已遺書言遣大臣蒞盟即班師,背之不義。」阿敏怒,叱之退,師復進次平山,倧走江華島,遣其臣進昌君至軍,阿敏令吹角督進兵。岳託乃與濟爾哈朗駐平山,遣副將劉興祚入江華島責倧。倧遣族弟原昌君覺等詣軍,為設宴。宴畢,岳託議還師,阿敏曰:「吾恆慕明帝及朝鮮王城郭宮殿,今既至此,何遽歸耶?我意當留兵屯耕,杜度與我叔侄同居於此。」杜度變色曰:「上乃我叔,我何肯遠離,何為與爾同居?」濟爾哈朗亦力阻,諸貝勒乃定議許倧盟。阿敏縱兵掠三日,乃還。上迎勞於武靖營,賜御衣一襲。復從上伐明,圍錦州,攻寧遠,斬明步卒千餘。

四年,師克永平、灤州、遷安、遵化,上命阿敏偕貝勒碩託將五千人駐守。阿敏駐永平,分遣諸將分守三城,諭降榛子鎮。明經略孫承宗督兵攻灤州,阿敏遣數百人赴援,收遷安守兵入永平。明兵攻灤州急,灤州守將固山額真圖爾格等不能支,棄城奔永平,明兵截擊,師死者四百餘。阿敏令遵化守將固山額真察哈喇等亦棄其城,遂盡殺明將吏降者,屠城民,收其金帛,夜出冷口東還。

上方遣貝勒杜度赴援,聞阿敏等棄四城而歸,上御殿,集諸貝勒大臣宣諭,罪阿敏等。阿敏等至,令屯距城十五里,復宣諭詰責。上念士卒陷敵,感傷墮淚。越三日,召諸貝勒曰:「阿敏怙惡久矣。當太祖時,嗾其父欲移居黑?大臣集闕下,上御殿,令貝勒岳託宣於扯木,太祖坐其父子罪,既而宥之。其父既終,太祖愛養阿敏如己出,授為和碩貝勒。及上嗣位,禮待如初。師征朝鮮,既定盟受質,不原班師,欲與杜度居王京,濟爾哈朗力諫乃止。此阿敏有異志之見端也。俘美婦進上,既,復自求之。上察其觖望,曰:『奈何以一婦人乖兄弟之好?』以賜總兵冷格里。伐察哈爾,土謝圖額駙背約與通好,上怒,絕之。阿敏遺以甲胄鞍轡,且以上語盡告之。諸貝勒子女婚嫁必聞上,阿敏私以女嫁蒙古貝勒塞特爾,及宴,上不往,常懷怨憤。太祖時,守邊駐防,原有定界,乃越界移駐黑扯木。上責以擅棄汛地,將有異志,阿敏不能答。上出征,令阿敏留守,惟耽逸樂,屢出行獵。岳託、豪格出師兵?先還,坐受其拜,儼如國君。及代守永平,妄曰:『既克城,何故不殺其民?』又明告曰:『我既來此,豈能令爾等不飽欲而歸?』略榛子鎮,盡掠其財物,又驅降人分給八家為奴。明兵圍灤州三晝夜,擁兵不親援,屠永平、遷安官民,悉載財帛牲畜以歸。毀壞基業,議其罪。僉曰:「當誅。」命幽之。留莊六所、園二所、奴僕二十、?故令我軍傷殘。」命羊五百、牛二十,餘財產悉畀濟爾哈朗。崇德五年十一月,卒於幽所。

阿敏子六,有爵者五:愛爾禮、固爾瑪琿、恭阿、果?、果賴。愛爾禮、果?、果賴皆封鎮國公,愛爾禮坐罪死,果?謚端純。

溫簡貝子固爾瑪琿,崇德間,從多爾袞伐明,自京師入山西境,復東至濟南,克城四十餘,封輔國公。阿敏得罪,奪爵,削宗籍。順治五年,復封輔國公。上以其貧乏,賜白金三千。從濟爾哈朗徇湖廣,破何騰蛟。師復進攻永興,奪門入,敗明兵,進貝子。康熙二十年,卒,謚溫簡。

鎮國襄敏公瓦三,固爾瑪琿子。事聖祖,初封輔國將軍。從岳託定湖廣,襲輔國公。二十一年,授右宗人。追論攻長沙退縮罪,奪官,仍留爵。復授鑲藍旗滿州固山額真。俄羅斯侵據雅克薩,上遣瓦三偕侍郎果丕,會黑龍江將軍薩布素按治。尋命固山額真朋春等率師討之,以瓦三統轄黑龍江將士。二十四年,卒,謚襄敏。瓦三子齊克塔哈,襲輔國公。事聖內大臣。坐事,奪爵。以固爾瑪琿孫鄂斐襲。?祖,徵噶爾丹在行。歷右宗人、都統、領侍征噶爾丹亦在行。卒,以子鄂齊襲。事世宗,嘗奉使西藏,宣諭達賴喇嘛,進鎮國公。授天津水師都統,坐不能約束所部,奪爵。復起授都統,坐納賂,再奪爵。

恭阿,亦以阿敏得罪,與固爾瑪琿同譴,尋同還宗籍。順治五年,同徇湖廣,克六十餘城,封鎮國公。六年,卒於軍。

鄭獻親王濟爾哈朗,舒爾哈齊第六子。幼育於太祖。封和碩貝勒。天命十一年,伐喀爾喀巴林部、扎嚕特部,並有功。天聰元年,伐朝鮮,朝鮮國王李倧既乞盟,阿敏仍欲攻其國都。岳託邀濟爾哈朗議,濟爾哈朗曰:「吾等不宜深入,當駐兵平山以待。」卒定盟而還。五月,從上伐明,圍錦州,偕莽古爾泰擊敗明兵。復移師寧遠,與明總兵滿桂遇,裹創力歸。三年八月,伐明錦?。二年五月,偕豪格討蒙古固特塔布囊,戮之,收其?戰,大敗其州、寧遠,焚其積聚。十月,上率師自洪山口入,濟爾哈朗偕岳託攻大安口,夜毀水門以進,擊明馬蘭營援兵。及旦,明兵立二營山上,濟爾哈朗督兵追擊,五戰皆捷,降馬蘭營、馬蘭口、大安口三營。引軍趨石門寨,殲明援兵,寨民出降。會師遵化,薄明都,徇通州張家灣。四年正月,從上圍永平,擊斬叛將劉興祚,獲其弟興賢。既克永平,與貝勒薩哈璘駐守,察倉庫,閱士卒,置官吏,傳檄下灤州、遷安。三月,阿敏代戍,乃引師還。

五年七月,初設六部,濟爾哈朗掌刑部事。從上圍大凌河,濟爾哈朗督兵收近城臺堡千餘人。七年三月,城岫巖。五月,明將孔?。六年五月,從征察哈爾,還趨歸化城,收其有德、耿仲明自登州來降,明總兵黃龍以水師邀之,朝鮮兵與會,濟爾哈朗與貝勒阿濟格等勒軍自鎮江迓有德等,明兵引去。

崇德元年四月,封和碩鄭親王。三年五月,攻寧遠,薄中後所城,明兵不敢出。移師二千有奇。五年三月?克模龍關及五里堡屯臺。四年五月,略錦州、松山,九戰皆勝,俘其,修義州城。蒙古多羅特部蘇班岱、阿爾巴岱附於明,屯杏山五里臺,請以三十戶來歸。上命率師千五百人迎之,戒曰:「明兵見我寡,必來戰,可分軍為三隊以行。」夜過錦州,至杏山,使潛告蘇班岱等攜輜重以行。旦,明杏山總兵劉周智與錦州、松山守將合兵七千逼我師,濟爾哈朗縱師入敵陣,大敗之,賜御?良馬一。九月,圍錦州,設伏城南,敵不進,追擊破之。六年三月,復圍錦州,立八營,掘壕築塹,以困祖大壽。大壽以蒙古兵守外郛,臺吉諾木齊等遣人約獻東關,為大壽所覺,與之戰。濟爾哈朗督兵薄城,蒙古兵縋以入,據其郛。遷蒙古六千餘人於義州,降明將八十餘。上御篤恭殿宣捷。四月,敗明援兵。五月,又敗之,斬級二千。六月,師還。九月,復圍錦州。十二月,洪承疇自松山遣兵夜犯我軍,我。七年,再圍錦州。三月,大壽降,隳松山、塔山?軍循壕射之,敵敗去,不得入,盡降其、杏山三城以歸,賜鞍馬一、蟒緞百。

八年,世祖即位,命與睿親王多爾袞同輔政。九月,攻寧遠,拔中後所,並取中前所。順治元年,王令政事先白睿親王,列銜亦先之。五月,睿親王率師入山海關,定京師。十月,封為信義輔政叔王,賜金千、銀萬、緞千疋。四年二月,以府第逾制,罰銀二千,罷輔政。五年三月,貝子屯齊、尚善、屯齊喀等訐王諸罪狀,言王當太宗初喪,不舉發大臣謀立肅親王豪格。召王就質,議罪當死,遂興大獄。勛臣額亦都、費英東、揚古利諸子侄皆連染,議罪當死,籍沒。既,改從輕比,王坐降郡王,肅親王豪格遂以幽死。

閏四月,復親王爵。九月,命為定遠大將軍,率師下湖廣。十月,次山東,降將劉澤清以叛誅。六年正月,次長沙,明總督何騰蛟,總兵馬進忠、陶養用等,合李自成餘部一隻虎等據湖南。王分軍進擊,拔湘潭,擒騰蛟。四月,次辰州,一隻虎遁走,克寶慶,破南山坡、大水、洪江諸路兵凡二十八營。七月,下靖州,進攻衡州,斬養用。逐敵至廣西全州,分軍下道州、黎平及烏撒土司,先後克六十餘城。七年正月,師還,賜金二百、銀二萬。

八年二月,偕巽親王滿達海、端重親王博洛、敬謹親王尼堪奏削故睿親王多爾袞爵,語詳睿親王傳。三月,以王老,免朝賀、謝恩行禮。九年二月,進封叔和碩鄭親王。十二年二月,疏言:「太祖創業之初,日與四大貝勒、五大臣討論政事得失,咨訪士民疾苦,上下交孚,鮮有壅蔽,故能掃清?雄,肇興大業。太宗纘承大統,亦時與諸王貝勒講論不輟,崇?忠直,錄功棄過,凡詔令必求可以順民心、垂久遠者。又慮武備廢弛,時出射獵,諸王貝勒置酒高宴,以優戲為樂。太宗怒曰:『我國肇興,治弓矢,繕甲兵,視將士若赤子,故人爭效死,每戰必克。常恐後世子孫棄淳厚之風,沿習漢俗,即於慆淫。今若輩為此荒樂,欲國家隆盛,其可得乎?』遣大臣索尼再三申諭。今皇上詔大小臣工盡言,臣以為平治天下,莫要於信。前者軫恤滿洲官民,聞者懽忭。嗣役修乾清宮,詔令不信,何以使民?伏祈效法太祖、太宗,時與大臣詳究政事得失,必商榷盡善,然後布之詔令,庶幾法行民信,紹二聖之休烈。抑有請者,垂謨昭德,莫先於史。古聖明王,進君子,遠小人,措天下於太平,垂鴻名於萬世,繄史官是賴。今宜設起居注官,置之左右,一言一行,傳之無窮,亦治道之助也。」疏上,嘉納之。

五月,疾革,上臨問,奏:「臣受三朝厚恩,未及答,原以取雲貴,殄桂王,統一四海為念。」上垂涕曰:「天奈何不令朕叔長年耶!」言已,大慟。命工圖其像。翌日薨,年五十七。輟朝七日。賜銀萬,置守園十戶,立碑紀功。康熙十年六月,追謚。乾隆四十三年正月,詔配享太廟,復嗣王封號曰鄭。

濟爾哈朗子十一,有爵者四:富爾敦、濟度、勒度、巴爾堪。

富爾敦,濟爾哈朗第一子,封世子。順治八年,卒,謚?厚。

簡純親王濟度,濟爾哈朗第二子。初封簡郡王。富爾敦卒,封世子。十一年十一月,命為定遠大將軍,率師討鄭成功。十二年九月,次福州。久之,進次泉州。十三年六月,成功將黃梧、蘇明、鄭純自海澄來降,移軍次漳州。俄,成功犯福州,遣梅勒額真阿克善等赴援,擊敗之,斬二百餘級。復擊斬其將林祖蘭等,奪其舟十有四。又分軍攻惠安、閩安、漳浦,獲舟數百,斬二千餘級。十四年三月,師還,上遣大臣迎勞盧溝橋,始聞鄭獻親王之喪,令入就喪次,上臨其第慰諭之。五月,襲爵,改號簡親王。十七年,薨。

濟度子五,喇布、德塞、雅布先後襲爵簡親王。

喇布,濟度第二子。濟度初薨,以第三子德塞襲。康熙九年,薨,謚曰惠。是年,喇布襲爵。吳三桂反,十三年九月,命為揚威大將軍,率師駐江寧。十四年九月,移師江西,鎮南昌,屢遣兵援東鄉,擊鄱陽,破金谿、萬年。三桂將高得捷、韓大任陷吉安,詔趣進師。喇布駐南昌,不出師攻吉安,屯螺子山,敵來攻,師敗績。上遣侍郎班迪按敗狀,喇布乃督師圍吉安。十六年三月,敵引走,喇布入吉安,疏稱大任等屯寧都請降,詔報可。既而大任自寧都出擾萬安、泰和,喇布復請增兵。上諭曰:「簡親王喇布自至江西,無尺寸之功,深居會城,虛糜廩餉。迨赴吉安,以重兵圍城,而韓大任竄逸,竊踞寧都,復擾萬安、泰和,不能擊滅。喇布所轄官兵為數不為少,乃一大任不能翦除,宜嚴加處分,俟事平日議罪。」十七年正月,護軍統領哈克三等敗大任於老虎洞,毀其壘,斬六千級。大任奔福建,詣康親王傑書軍降。二月,移師湖南,駐茶陵。八月,三桂死於衡州,詔令自安仁進師。十八年正月,進復衡州。二月,分軍復祁陽、耒陽、寶慶。九月,進次廣西,駐桂林。十九年正月,馬承廕以柳州叛。五月,進攻柳州,承廕降。八月,移駐南寧。十月,詔選所部付大將軍賚塔進攻雲南。二十年八月,召還京師。十月,薨。二十一年,追論吉安失機罪,奪爵。

雅布,濟度第五子。二十二年,襲。二十七年,命赴蘇尼特防噶爾丹。二十九年,噶爾丹深入烏珠穆沁地,以恭親王常寧為安北大將軍,雅布與信郡王鄂扎副之,出喜?口。既而罷行,詔赴裕親王福全軍參贊軍務。八月,擊敗噶爾丹於烏闌布通,噶爾丹遁,未窮追。師還,議不追敵罪,當奪爵,詔罰俸三年。三十五年,從上親征。三十八年,掌宗人府事。四十年,薨,謚曰修。子十五,雅爾江阿、神保住先後襲爵。

飲廢?雅爾江阿,雅布第一子。初封世子。四十二年,襲。雍正四年,詔責雅爾江阿事,奪爵。神保住,雅布第十四子。初封鎮國將軍。雅爾江阿既黜,世宗命襲爵。乾隆十三年,詔責神保住恣意妄為,致兩目成眚,又虐待兄女,奪爵。以濟爾哈朗弟貝勒費揚武曾孫德沛襲。

德沛字濟齋,貝子福存子。雍正十三年,授鎮國將軍。以果親王允禮薦,世宗召見,問所欲,對曰:「原廁孔廡分特豚之饋。」上大重之。授兵部侍郎。乾隆元年,改古北口提督。二年,授甘肅巡撫,奏言:「甘肅州縣多在萬山中,遇災,民入城領賑,路窵遠。宜於鄉鎮設廠散糧,並許州縣吏具詳即施賑。」旋擢湖廣總督,奏言:「治苗疆宜勸墾田,置學校,並諭令植樹。」四年,調閩浙總督。御史硃續?劾福建巡撫王士任贓私,上疑不實,命續?會鞫。德沛自承失察,直續?而奪士任官,時服其公。福州將軍隆升貪縱,劾去之。奏,宜酌移鎮將營汛,預弭爭端。」五年十二月,諭曰:「德沛屢任封?言;「海濱居民恆械疆,操守廉潔,一介不取,逋負日積,致蠲舊產。賜福建籓庫銀一萬,以風有位。」六年,兼署浙江巡撫。七年,調兩江總督。淮、揚大水,令府縣發倉庫,奏撥地丁、關稅、鹽課銀十萬兩治賑。尋議河事與高斌不合。八年,轉吏部侍郎。十二年五月,署山西巡撫。十二月,擢吏部尚書。十三年七月,以疾解任。神保住既黜,上以德沛操履厚重,特命襲爵,曾祖貝勒費揚武、祖貝子傅喇塔、父福存,並追封簡親王。十七年,薨,謚曰儀。以濟爾哈朗曾孫奇通阿襲。

奇通阿,輔國公巴賽子。初授輔國將軍。襲輔國公。乾隆元年,授正紅旗滿州都統。內大臣。十七年,襲。祖輔國公巴爾堪、父巴賽,並追封簡親王。二十一年?三年,授領侍。事高宗,從師?,掌宗人府事。二十八年,薨,謚曰勤。子豐訥亨襲。豐訥亨初授三等侍討準噶爾,將健銳千人屯呼爾璊。霍集占以五千人來犯,合諸軍擊?之,逐北十餘里。師進,敵踞塹以拒戰,奪塹,所乘馬中創,易馬再進,敗敵沁達勒河渡口,再敗敵葉爾羌河岸。,擢鑲白旗滿洲副都統。移軍伊犁,授領隊大臣。擊破瑪哈沁及哈薩?詔嘉其勇,遷二等侍克部人,收其馬。二十七年,師還,賜雙眼孔雀翎。遷護軍統領,管健銳營。二十八年,襲爵。授都統,掌宗人府事。四十年,薨,謚曰恪。子積哈納,襲。四十三年正月,復號鄭親王。四十九年,薨,謚曰恭。子烏爾恭阿,襲。

烏爾恭阿初名佛爾果崇額,襲爵,詔改名。道光二十六年,薨,謚曰慎。

子端華,襲。授御前大臣。宣宗崩,受顧命。文宗即位,迭命為閱兵大臣、右宗正。內大臣。端華弟肅順用事,文宗崩,?京師戒嚴,令督察巡防。十年,扈上幸熱河,授領侍再受顧命,與怡親王載垣及肅順等並號「贊襄政務王大臣」。穆宗還京師,詔責端華等專擅跋扈罪,端華坐賜死。肅順自有傳。爵降為不入八分輔國公。同治元年二月,以濟爾哈朗八世孫岳齡襲。三年七月,克復江寧,復還鄭親王世爵,以奇通阿五世孫承志襲。

承志,輔國公西朗阿子。初襲輔國公。既襲王爵,曾祖輔國公經訥亨、祖輔國公伊豐額、父西朗阿,並追封鄭親王,而以岳齡改襲輔國公。四年二月,御史劉慶劾承志品行不端玉壽毆殺主事福珣,奪爵,圈禁。以積哈納孫慶至襲。?,詔令力圖湔濯。十一年,坐令護慶至,奉恩將軍松德嗣子。既襲王爵,松德追封鄭親王。慶至,光緒四年,薨,謚曰順。子凱泰,襲。二十六年,薨。謚曰恪。子昭煦,襲。

勒度,濟爾哈朗第三子。封敏郡王。薨,謚曰簡。無子,爵除。

輔國武襄公巴爾堪,濟爾哈朗第四子。初授輔國將軍。康熙十三年,吳三桂據湖南,令巴爾堪率師赴兗州,署梅勒額真。進次江寧,耿精忠遣兵犯徽州,詔巴爾堪進剿。九月,次旌德,聞績溪陷,疾趨過徽嶺,破敵。江寧將軍額楚繼至,合師逐北,斬三千餘級,克徽州。復破敵黟縣董亭橋,進攻婺源。復破敵於奇臺嶺、於黃茅新嶺,復婺源。進克樂平,擊破叛將陳九傑,乘勝下饒州。十四年,攻萬年石頭街,敵四萬人禦渡口,水陸並進,破五十七營,斬五千級,擒九傑,克安仁,敵焚舟走。五月,復貴溪,進略弋陽,攻永豐。十六年正月,敗於螺子山,議奪官。偕額楚徇廣東。九月,戰韶州蓮花山,陷陣,中流矢,裹創力戰,大破敵。十九年八月,喇布師次廣西,上命以巴爾堪從。病作,語固山額真額赫納等曰:「吾不能臨陣而死,今創發,勿令家人以陣亡冒功也。」遂卒於軍。喪還,上命內大臣輝塞往奠,下部議恤。雍正元年,追封謚。子巴賽,襲。

輔國襄愍公巴賽,事聖祖,授鑲藍旗漢軍副都統。從征噶爾丹,遷正紅旗蒙古都統,署黑龍江將軍。世宗即位,授寧古塔將軍。既,襲爵,召還。雍正四年,授振武將軍,軍阿爾臺。五年,當代還,以喀爾喀郡王丹津多爾濟言巴賽治事整飭,命留防。七年,靖邊大將軍傅爾丹率師討噶爾丹策零,授巴賽副將軍。八年,傅爾丹入覲,護大將軍印。九年,偕傅爾丹駐科布多。六月,噶爾丹策零以三萬人來犯,傅爾丹信間言噶爾丹策零兵寡,遂出師,次庫列圖嶺。敵據險,攻之不克,移軍和通呼爾哈諾爾。敵伏山谷,突起截戰,蒙古兵潰,收滿洲兵四千作方營,保輜重,退渡哈爾哈納河。敵追至,傅爾丹還科布多,巴賽與副將軍旌黃帶示我師曰:「汝宗室為我所?查弼納率殘兵越嶺至河濱禦敵,沒於陣。噶爾丹策零之殺矣!」賜恤謚,祀昭忠祠。子奇通阿,襲。尋改襲簡親王,公爵當除。高宗以巴爾堪、巴賽仍世有戰功,以奇通阿次子經訥亨襲。四傳至曾孫承志,復改襲鄭親王。

靖定貝勒費揚武,一名芬古,舒爾哈齊第八子。天聰五年,授鑲藍旗固山額真。從上伐明,攻大凌河城,費揚武率本旗兵圍其西南。上幸阿濟格營,城兵突出,費揚武擊敗之。上令諸軍向錦州,幟而馳,若明援兵至者,以致祖大壽。費揚武迎擊,大壽敗入城,遂不敢出。八年,再從伐明,師進獨石口,克長安嶺,攻赤城,克其郛。九年,師入山西,上命費揚武等攻寧、錦,緩明師。大壽軍大凌河西,擊敗之。崇德元年,伐明,克城十。是冬,伐朝鮮。?功,封固山貝子。四年,坐受外籓蒙古賄,削爵。尋復封輔國公。七年,伐明,敗明總兵白騰蛟等於薊州,克其城。八年,代戍錦州。十二月,卒。順治十年,追封謚。

費揚武子七,有爵者三:尚善、傅喇塔、努賽。努賽封貝子,謚悼哀。

尚善,初襲輔國公。順治元年,進貝子。二年,從多鐸南征擊李自成,敵以騎兵三百沖我師,尚善擊敗之。平河南,下江南,並有功,賜圓補紗衣一襲、金百、銀五千、鞍馬一。五年,戍大同。六年,進貝勒,掌理籓院,為議政大臣。十五年,從多尼征雲南。明桂王由榔奔永昌,尚善進鎮南州,破其將白文選於玉龍關,渡瀾滄江,下永昌,由榔先遁,乘勝取騰越,進南甸,至孟村而還。十六年,賜蟒袍一、玲瓏刀一、鞍馬一。十七年,追論尚善撤永昌守兵致軍士入城傷人罪,降貝子。康熙十一年,復爵,任右宗正。以疾罷。

吳三桂反。授安遠靖寇大將軍,率師之岳州。尚善至軍,移書三桂曰:「王以亡國餘生,乞師我朝,殄殲賊寇,雪國恥,復父仇,蒙恩眷禮,列爵分籓,富貴榮寵,迄今三十餘年矣;而晚節末路,自取顛覆,竊為王不解也。王今藉口興復明室,曩者大兵入關,不聞請立明裔;天下大定,猶為我計除後患,翦滅明宗,安在其為故主效忠哉?將為子孫創大業,則公主、額駙入滇之時,何不即萌反側?至遣子入侍,乃復稱兵,以陷子於戮,可謂慈乎?若謂光耀前人,則王之投誠也,祖考皆膺封錫,今則墳塋毀棄,骸骨遺於道路,可謂孝乎?為人臣僕,身事兩朝,而未嘗忠於一主,可謂義乎?躬蹈四罪,而猶逞志角力,謬欲收拾人心,是厝薪於火而雲安,結巢於幕而云固也。聖朝寬大,如輸誠悔罪,應許自新,毋蹈公孫述、彭寵故轍,赤族湛身,為世大僇。」三桂得書,不報。

尚善疏請發荊州綠營兵、京口沙唬船五十,進攻岳州。十四年,遣舟師絕敵餉道。十五年,敗賊於洞庭,取君山,分兵助攻長沙。十六年四月,三桂奔衡州,復出湘潭,分遣其侵兩粵。十七年,詔責尚善師無功,令率所部駐長沙,而以岳樂統大軍取岳州。尚善請率?舟師克岳州自效,上從之。三桂將杜輝等犯柳林嘴,師迎擊,舟師來會,合戰,輝敗走。八月,卒於軍。十九年,追論退縮罪,削貝勒。聖祖念尚善舊勞,授其子門度輔國公,世襲。

惠獻貝子傅喇塔,費揚武第四子。初封輔國公。順治二年,從勒克德渾徇湖廣,有功,賜金五十、銀千。五年,復征湖廣,逐敵至廣西,賜銀六百。六年,進貝子。十六年,以朝參失儀,降輔國公。十八年,復爵。

康熙十三年,耿精忠反,授寧海將軍,佐康親王傑書討之。師至浙江,溫州、處州皆陷。傅喇塔師進臺州,戰黃瑞山,擊斬精忠將陳鵬等,復破敵天臺紫雲山。十四年,精忠將曾養性再犯臺州,師自仙居襲其後,破之,乘勝圍黃巖,養性遁,城降。先後復太平、樂清、青田諸縣,進攻溫州,破敵南江。十五年,精忠兵四萬水陸來犯,師分路迎擊,斬其將三百、兵二萬有奇。

初,傅喇塔之攻溫州也,以待紅衣?為辭,繼言須戰船,傑書疏聞。上責其言先後歧,命?期取溫州。傅喇塔疏言:「臣奉康親王檄催,心思皇惑,語言違謬。臣前駐臺州,王云:『待破臺州,進福建。』臣得黃巖,又云:『必取溫州。』以是責臣,臣將無辭。今蒙恩刻期下溫州,敢不戮力,但環溫州皆水,我軍不能猝入。」上命康親王留兵圍溫州,而趣傅喇塔率師自衢州規福建。諭曰:「王、貝子皆朕懿親,受命討賊,師克在和,宜同心合力,以奏膚功。」於是傅喇塔亦留兵圍溫州,而自率師攻處州,溯江抵得勝山。養性等以數百艘泊江中,復立兩營對江及得勝山下古溪,阻我師。傅喇塔遣攻古溪,伏林中,敵敗,伏起截殺,並發?碎敵舟及對江營。師進次溫溪渡口,敗精忠將馬成龍。尋會傑書師於衢州。精忠兵屯雲和石塘嶺,擊之,破其壘二十八,克雲和。九月,師入福建,精忠降。浙江諸寇悉平。十一月,卒於軍。喪歸,賜祭奠,謚惠獻。

子富善,仍襲貝子。授左宗人。以病解任。諭責富善乖亂,奪爵。弟福存,襲。卒,子德普,襲鎮國公。授左宗人。卒,子恆魯,襲輔國公。事高宗,歷工部侍郎、左宗人,綏遠城、盛京將軍,授內大臣。卒,謚恭懿。子興兆,襲輔國公。事高宗,從征金川,為領隊大臣。歷右宗人、荊州將軍。攻當噶拉、得黑、絨布寨、卡卡角諸地,有功。金川平,畫像紫光閣。歷西安、綏遠城將軍。坐事,奪官。復授荊州將軍。苗石柳鄧、吳半生、吳八月等為亂,與提督花連布擊吳半生,降;與內大臣額勒登保等擊吳八月,復擊石柳鄧,殲焉:,?屢荷恩賚。嘉慶初,討教匪姚之富、齊王氏等,師久無功,奪官,戍烏魯木齊。復授侍駐和闐、塔爾巴哈臺。坐事,復奪官。子孫仍以輔國公世襲,錄傅喇塔功也。

舒爾哈齊諸孫,札喀納、屯齊、洛托皆有功,受封。

札喀納,扎薩克圖子。崇德三年八月,睿親王多爾袞率師伐明,毀邊墻,至涿州,分,所向有功。四年,師還,賜駝?軍八道入。札喀納趨臨清州,渡運河,破濟南,還破天津馬各一、銀二千,封鎮國公。上命追蒙古、漢人之逃亡者,札喀納以泥淖,不追而還,降輔國公。六年,從上攻錦州。明總督洪承疇以兵犯鑲紅旗營,擊敗之。罷戰,敵襲我後,距百步而近,札喀納奮力轉戰,敵驚遁。復偕輔國公費揚武,追擊明將吳三桂、白廣恩、王樸等於塔山。七年,戍錦州。追論敏惠恭和元妃喪時札喀納從武英郡王阿濟格歌舞為樂,大不敬,削爵,黜宗籍,幽禁。

順治初,釋之。從多爾袞敗李自成,復宗籍,授輔國公品級。偕鎮國公傅勒赫戍江南,復從平南大將軍勒克德渾徇湖廣。師還,賜金五十、銀千。五年,從郡王瓦克達赴英親王阿濟格軍戍大同。六年,進貝子。九年,從定遠大將軍尼堪征湖南,賜蟒衣、鞍馬、弓矢。至衡州,尼堪戰歿,上以貝勒屯齊與札喀納合領其軍。敗明兵於周家坡。十一年,追論衡州敗績罪,奪爵。十二年,復授輔國公品級。十五年,從定遠靖寇大將軍多尼徇雲南,克永昌。十六年閏三月,卒於軍。子瑪喀納,襲。

三等鎮國將軍品級屯齊,圖倫子。圖倫,舒爾哈齊第四子,追封貝勒,謚恪僖。屯齊,事太宗,從英親王阿濟格伐明,有功。從鄭親王濟爾哈朗略錦州、松山、杏山,九戰九勝。屯齊受創,加賜銀百,封輔國公。五年,從睿親王多爾袞圍錦州,明兵夜襲鑲藍旗營,擊敗之。坐不臨城及私遣兵還,議削爵,命罰銀千。六年,從上攻錦州、塔山,敗明兵,復從多爾袞圍錦州。

順治元年,進貝子。尋從豫親王多鐸破流寇,平陜西、河南並有功,賜圓補紗衣一襲。從多鐸下江寧,明福王由崧走太平,與貝勒尼堪追至蕪湖,獲之。師還,賜金百、銀五千、鞍馬一,授鑲藍旗滿洲固山額真。三年,從肅親王豪格西征,破賀珍,解漢中圍。會一隻虎、孫守法陷興安,進師漢陰,擊走之。五年,陜西回亂,命為平西大將軍,率師討之。總督孟喬芳已擊斬回酋米喇印、丁國棟等,還赴英親王阿濟格軍,戍大同。六年,進貝勒。

張獻忠將孫可望、李定國等降於明桂王由榔,擾湖南。九年,屯齊從定遠大將軍尼堪南征。尼堪戰歿,以屯齊代將。時定國及別將馬進忠率兵四萬餘,屯永州。定國聞師至,度龍虎關先遁。可望在靖州,別將馮雙禮在武岡。屯齊進師寶慶,至周家坡,雙禮、進忠據險號十萬,屯齊分兵縱擊,大破?抗我師,會暮天雨,列陣相拒。其夜可望自寶慶以兵來會,之。十一年,追坐衡州敗績,削爵。十二年,授鎮國公品級。十五年,從多尼徇雲南,定國。還。康熙二年,卒。?挾由榔奔永昌,降其餘

子溫齊,初封貝子,授右宗人、鑲藍旗滿洲都統。吳三桂反,上命定西大將軍董額自陜西徇四川,溫齊從。陜西提督王輔臣叛應三桂,師駐漢中。十四年,進次隴州,克仙逸、關山二關,復秦州禮縣,逐敵至西和,清水、伏羌並下。十六年,詔責董額師久無功,溫齊亦坐降輔國公,奪官。三桂陷湖南,安遠靖寇大將軍尚善規岳州,上發禁旅,命溫齊率以往,參贊軍務。十七年,敗賊於柳林嘴、於君山、於陸石口,進克岳州。十八年,溫齊追三桂將吳應麒,以未攜爨具,引還,且妄報陣斬五千餘級。時尚善已卒,察尼代將。事聞,命察尼按鞫之,溫齊坐削爵。

洛託,寨桑武子。寨桑武,舒爾哈齊第五子,追封貝勒,謚和惠。

洛託,天聰八年,從太宗伐明。上駐師大同南山西岡,洛託籍所俘以獻。崇德元年,封貝子。從伐朝鮮,偕貝勒多鐸圍南漢山城。朝鮮將以八千人赴援,盡殲之;又以五千人赴援,擊之,敗走。二年,與議政。四年,從英親王阿濟格圍塔山、連山。五年,從睿親王多爾袞屯田義州。錦州兵夜襲我鑲藍旗營,與屯齊共擊敗之。六年,坐圍錦州不臨城,且私遣兵還,議削爵,詔罰銀千。上征松山,大破明總督洪承疇兵。洛託橫擊潰兵於塔山,復圍錦州。七年,從鄭親王濟爾哈朗攻塔山,克之,授都察院承政。偕博洛、尼堪駐錦州。八年,坐事,削爵,幽禁。

順治初,釋之。八年,復封三等鎮國將軍。十三年,授鑲藍旗滿洲固山額真。十四年,孫可望、李定國、馮雙禮等擾湖南,命為寧南靖寇將軍,駐防荊州,佐經略洪承疇討之。遣兵取心潭隘,斷巴東渡口,可望將趙世超、譚新傳、趙三才皆降。俄,可望與定國內訌,戰不勝,亦來降。上命偕都統濟席哈自湖南進取貴州。十五年,與承疇會師常德,次辰州。復沅陵、瀘溪、麻陽、黔陽、?浦諸縣,進次沅州。檄偏沅巡撫袁廓宇徇靖州,屯鎮遠二十里山口以禦敵。雙禮部將馮天裕、閻廷桂等先後自平越降。四月,師至貴州,明將羅大順收,擊敗之,洛託與承疇守貴陽。十六年,師還。?功,加授拖沙喇哈番,進一?潰卒襲新添等鎮國將軍。十七年,命為安南將軍,徵鄭成功,大破之。十一月,還。康熙四年,卒。

子富達禮,襲拜他喇布勒哈番世職。旋改襲奉恩將軍。八年,進一等輔國將軍。坐諂索額圖,為其從弟所訐,削爵。

通達郡王雅爾哈齊,顯祖第四子,太祖同母弟。其生平不著。順治十年五月,追封謚,配享太廟。

篤義剛果貝勒巴雅喇,顯祖第五子。初授臺吉。歲戊戌正月,太祖命偕褚英伐安楚拉庫路,夜取屯寨二十,降萬餘人,賜號卓禮克圖,譯言「篤義」。歲丁未五月,伐東海窩集部,取赫席赫、鄂謨和蘇魯、佛訥赫托克索三路,俘二千人。天命九年,卒。順治十年,追封謚。

子拜音圖,事太宗,授三等昂邦章京、鑲黃旗固山額真。崇德元年五月,從武英郡王阿濟格略保定,攻安肅,克之。十月,獻所獲於篤恭殿,上以拜音圖戰不忘君,深嘉之。從伐朝鮮,騎入城,收其輜重。三年,從睿親王多爾袞伐明,偕固山額真圖爾格敗敵董家口,毀邊墻入,克青山關下城。六年,拜音圖弟鞏阿岱從大軍圍錦州,臨陣退撓,下王大臣鞫其罪,拜音圖拂袖出,坐徇庇,論死,命奪爵職,罰鍰贖罪。尋率師助多爾袞攻錦州,復偕多鐸圍松山。七年,復授固山額真。順治二年,從多鐸西征,敗敵潼關,封一等鎮國將軍,賜繡服一襲。復從南征,克揚州,又以舟師破其兵於江南岸,偕貝子博洛下杭州。?功,賜金八十、銀四千、鞍馬一。三年,授三等公。五年,進貝子。從阿濟格戍大同。叛將姜瓖既,進貝勒。鞏阿岱事?死,餘黨猶分據郡邑。六年,拔沁州,復圍瓖將胡國鼎於潞安,殲其多爾袞,最見信任,累進封貝子。多爾袞既薨,坐黨附罪,死。拜音圖亦牽連,削爵,幽禁,削宗籍。嘉慶四年,仁宗命復宗籍,賜紅帶。鞏阿岱裔孫伊里布,自有傳。


\end{pinyinscope}