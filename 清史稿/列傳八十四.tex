\article{列傳八十四}

\begin{pinyinscope}
查郎阿傅爾丹馬爾賽李杕慶復李質粹張廣泗

查郎阿,字松莊,納喇氏,滿洲鑲白旗人。曾祖章泰,以軍功授拖沙喇哈番。祖查爾海,復以軍功進一等阿達哈哈番。父色思特,死烏闌布通之戰。查郎阿襲世職,兼佐領,遷參領。雍正元年,授吏部郎中。二年,超擢侍郎,署鑲黃旗滿洲都統。五年,遷左都御史,仍治吏部事。是歲冬,西藏噶布倫阿爾布巴等為亂,戕總理藏務貝子康濟鼐,扎薩克臺吉頗羅鼐馳聞,上命查郎阿偕副都統邁祿率兵入藏。六年,擢尚書。秋,師至藏,駐藏副都統馬喇等已擒阿爾布巴,即按誅之,並殲其餘黨。查郎阿奏移達賴喇嘛暫居里塘,留兵二千交駐藏大臣調遣;又奏請以頗羅鼐總理後藏,而前藏達賴喇嘛未還,畢昭新授噶布倫,慮未妥協,並令頗羅鼐兼領:皆從之。

七年,命查郎阿至西安,留佐川陜總督岳鍾琪,專理軍需。鍾琪授大將軍,出師,令署川陜總督兼西安將軍,加太子少保。八年,命往肅州專理軍需。九年,析置四川、陜西兩總督,查郎阿改署陜西總督。十年,召鍾琪還京師,以查郎阿署寧遠大將軍,命大學士鄂爾泰馳驛往肅州授方略,並賜白金萬。十一年,疏劾副將紀成斌防廋集、總兵張元佐防無克克嶺,敵入掠糧車,漫無偵察。上命斬成斌,元佐坐降調。又劾總兵曹勷防哈密,縱賊妄報,上命斬勷。又劾副都統阿克山、觀音保牧馬多死,玩悮軍事,下部議當斬。查郎阿復奏阿克山、觀音保所部兵久居南方,不知牧馬法,視退縮竊換者有間,請暫免死,今於通衢荷校,遍示諸軍。

十三年,噶爾丹策凌乞和,命查郎阿撤兵。奏請留兵戍哈密及三堡沙棗爾、塔勒納沁諸城,並於南山大阪、無克克嶺、塔勒納沁河源分設斥堠,又奏於安西及赤金、靖逆、柳溝、布隆吉爾、橋灣五處分兵駐防,部議如所請。授文華殿大學士,兼兵部尚書,仍改陜西總督為川陜總督。乾隆元年,疏言甘肅地瘠,請撥陜西倉糧預籌協濟,命會巡撫劉於義確議。尋請撥陜西倉糧八萬石運貯慶陽、涇州、靜寧、固原諸處,從之。疏劾甘肅巡撫許容匿災營私,上命奪容官逮治。秋,入覲,諭速回任。奏言:「軍中馬駝被竊,當責大將軍償補。雍正十年以前,岳鍾琪任之;十一年,臣任之。惟鄂爾多斯牧廠所失及歷年馬駝多斃,請免追償。」上許之。三年,奏劾肅州道黃文煒、軍需道沈青崖等侵帑,並及於義徇庇,遣左都御史馬爾泰會鞫論罪。

章嘉呼圖克圖請以里塘、巴塘畀達賴喇嘛,查郎阿奏:「聖祖時克西藏,收里塘、巴塘內屬。章嘉呼圖克圖以日用不敷為辭,藏中大小廟千餘,常住喇嘛四十餘萬,需用良鉅。請視里塘、巴塘諸地每歲徵收數目,以打箭爐商稅撥予達賴喇嘛,地仍內屬如故。」上嘉納之。寧夏地震,查郎阿馳往賑撫。五年,命還京入閣治事,加太子太保。六年,命與侍郎阿里袞清察黑龍江、吉林烏喇開墾地畝。十二年,以衰病,命致仕。尋卒。

傅爾丹,瓜爾佳氏,滿洲鑲黃旗人,費英東曾孫,倭黑子也。康熙二十年,襲三等公,兼佐領,授散秩大臣。四十三年,上西巡,駐蹕祁縣鄭家莊,於行宮前閱太原城守兵騎射。有卒馬驚逸近御仗,傅爾丹直前勒止之,捽其人下。上悅,諭獎傅爾丹,賜貂皮褂。尋授正白旗蒙古都統。四十八年,授領侍衛內大臣。五十四年,以託疾未入直,罷領侍衛內大臣。命率土默特兵千赴烏蘭固木等處屯田。五十六年,復授領侍衛內大臣。

師討噶爾丹,授富寧安靖逆將軍,出西路;傅爾丹振武將軍,出北路:駐軍阿爾泰。五十七年,疏請與富寧安分路進兵,諭定師期。傅爾丹請與征西將軍祁里德將萬二千人,以七月出布魯爾,直抵額爾齊斯河。會策妄阿喇布坦使來乞和,令暫停進取,繕兵防守。上欲於烏蘭固木、科布多築城衛喀爾喀游牧,命傅爾丹相度具奏。五十八年春,傅爾丹疏請築城鄂勒齊圖郭勒,上以鄂勒齊圖郭勒距師遠,命更於科布多築城。傅爾丹復疏言:「科布多阻大河,材木難致。請築城察罕廋爾,距鄂勒齊圖郭勒千里,中設十一站。」上從之。五十九年,將八千人自布拉罕進次格爾額爾格,準噶爾兵潰,擊斬二百餘級,擒宰桑等百餘,盡降其眾。又焚烏蘭呼濟爾敵糧,引還。雍正元年,命兼統祁里德軍,分兵駐巴里坤。三年,召還,授內大臣。四年,授黑龍江將軍。六年,授吏部尚書,賜雙眼孔雀翎。

初,青海羅卜藏丹津敗走,準噶爾策妄阿喇布坦納之。上屢遣使索獻,策妄阿喇布坦亦遣使請和,上罷兩路兵,久之議未決。策妄阿喇布坦死,子噶爾丹策零嗣,屢犯邊。七年二月,上命廷臣集議,大學士硃軾、左都御史沈近思皆言天時未至,副都統達福亦言不可,惟大學士張廷玉贊用兵,上意乃決,復出師。命傅爾丹為靖邊大將軍,出北路;發京師八旗兵六千、車騎營兵九千、奉天等處兵八千八百,以巴賽為副將軍,順承郡王錫保掌振武將軍印,陳泰、袞泰、石禮哈、岱豪、達福、覺羅海蘭為參贊。定壽將前鋒,魏麟、閃文繡將車騎營,納秦將奉天兵,塔爾岱、西彌賴將索倫兵,費雅思哈將寧古塔兵,阿三將右衛兵,素圖將寧夏兵,承保、常祿將察哈爾兵,馬爾齊、袞布將土默特兵,丹巴、沙津達賴將喀喇沁、土默特兵,法敏、伊都立、巴泰、西琳、傅德理餉,永國護印。上祭告太廟,幸南苑閱車騎營兵,御太和殿行授鉞禮,賜傅爾丹御用朝珠、黃帶、紫轡、白金五千,加少保。出駐阿爾泰。八年,噶爾丹策零表請執羅卜藏丹津以獻,上命緩進兵。尋召與兵鍾琪同詣京師議軍事,遣還軍。九年,疏言科布多為進兵孔道,請仍於此築城,下廷議,如所請。

五月,傅爾丹移軍科布多,噶爾丹策零遣所部嗒蘇爾海丹巴為間,為守卡侍衛所獲,詰之,曰:「噶爾丹策零發兵三萬,使大策零敦多卜、小策零敦多卜分將犯北路。小策零敦多卜已至察罕哈達,大策零敦多卜以事宿留未至。」傅爾丹信其語,計及其未集擊之。令選兵萬人,循科布多河西以進,素圖、岱豪為前鋒,定壽等領第一隊,馬爾薩等領第二隊,傅爾丹舉大兵繼其後,令袞泰護築城,陳泰屯科布多河東,斷奇蘭道。六月庚子,師發科布多,定壽等進次扎克賽河,獲準噶爾邏卒,言距察罕哈達止三日程,準噶爾兵不過千人,未立營。傅爾丹命乘夜速進,行數日不見敵。戊申,獲諜,言準噶爾兵二千屯博克托嶺。傅爾丹遣素圖、岱豪將三千人往擊之。敵出羸兵誘師,而伏二萬人谷中。己酉,定壽師次庫列圖嶺,遇敵,斬四百餘級,敵驅駝馬逾嶺遁。

庚戌,傅爾丹師至,素圖、定壽皆會。辛亥,逐敵入谷,伏發,據高阜沖擊。傅爾丹督戰,殺敵千餘,塔爾岱、馬爾齊督兵奪西山,敵據險,師攻之不能克。壬子,傅爾丹令移軍和通呼爾哈諾爾,定壽、素圖、覺羅海蘭、常祿、西彌賴據山梁東,塔爾岱、馬爾齊據其西,承保居中,馬爾薩出其東,達福、岱豪當前,舒楞額、沙津達賴等護後。師甫移,敵力攻山梁東西二軍,定壽等奮戰。大風雨雹,師為敵所圍。傅爾丹遣兵援塔爾岱出,又令承保援定壽,日暮,圍未解。癸丑,海蘭突圍出,定壽、素圖、馬爾齊皆自殺;西彌賴令索倫兵赴援,兵潰,亦自殺。甲寅,敵環攻大營,傅爾丹督兵御之,殺敵五百餘。科爾沁兵潰,沙津達賴奮戰入敵陣,師望見其纛,曰:「土默特兵陷賊矣!」遂大潰。乙卯,永國、海蘭、岱豪皆自殺。傅爾丹雜士伍中以出。敵大集,查弼納、巴賽、達福、馬爾薩、舒楞額皆戰死。傅爾丹率殘兵渡哈爾噶納河,敵追至,擊殺五百餘人。七月壬戌朔,還至科布多,收餘兵僅存二千餘。

方戰,科爾沁蒙古兵先敗,傅爾丹聞人言,謂先敗者土默特兵也。劾沙津達賴,論斬。歸化城土默特副都統袞布降敵,戮其孥。傅爾丹疏請罪,上諭曰:「損兵誠有罪,朕因爾等竭蹶力戰,特寬恕之。痛惻難忍,不覺淚下!解朕親束帶賜傅爾丹。爾等毋妄動,敵至能堅守,即爾等之功。科布多不能守,可還軍察罕廋爾。」傅爾丹復疏請罪,上諭曰:「輕信賊言,冒險深入,中賊詭計,是爾之罪。至不肯輕生自殺,力戰全歸,此爾能辨別輕重。事定,朕自有處置。」尋命以錫保為靖邊大將軍,傅爾丹掌振武將軍印,協辦軍務。十年七月,準噶爾侵烏遜珠勒,錫保令傅爾丹將三千人御之,敗績。錫保疏劾,罷領侍衛內大臣、振武將軍,削公爵。十一年,錫保再疏劾傅爾丹,上察傅爾丹兵寡,原其罪,命留軍效力。

十三年,伊都立等侵軍餉事發,辭連傅爾丹,命侍郎海望逮詣京師下獄,並追論和通呼爾哈諾爾及烏遜珠勒失機罪,王大臣等依律擬斬。命未下,世宗崩,高宗即位,命改監候。乾隆四年,與岳鍾琪並釋出獄。十三年,師討大金川未下,授內大臣、護軍統領,赴軍,尋命署川陜總督,與鍾琪治軍事。大學士傅恆出為經略,奏傅爾丹衰老,惟熟於管理滿洲兵,請專治營壘諸事。十四年,命為參贊。大金川師罷,授黑龍江將軍。十七年,卒,賜祭葬,謚溫愨。子兆德,襲爵;哈達哈,自有傳。

傅爾丹頎然岳立,面微赬,美須髯。其為大將軍,廷玉實薦之。鍾琪嘗過其帳,見壁上刀槊森然,問:「安用此?」傅爾丹曰:「此吾所素習者,懸以勵眾。」鍾琪出曰:「為大將,不恃謀而恃勇,敗矣!」

馬爾賽,馬佳氏,滿洲正黃旗人,大學士、三等公圖海孫。馬爾賽,襲爵。康熙間,迭授護軍統領、鑲黃旗蒙古都統、領侍衛內大臣,掌鑾儀衛事。雍正二年,加贈圖海一等公,號曰忠達,仍以馬爾賽襲。調鑲藍旗滿洲。六年,授武英殿大學士,兼吏部尚書。八年,命與大學士張廷玉、蔣廷錫詳議軍行事宜。尋以翊贊機務,加一等阿達哈哈番世職。

九年,靖邊大將軍傅爾丹討噶爾丹策零,師敗績。授撫遠大將軍,調西路副將軍覺羅伊禮布為參贊,率師駐圖拉。馬爾賽師行,聞準噶爾將犯科布多,奏請暫駐第十五臺。俄聞準噶爾兵屯科布多近處,又奏請進駐察罕廋爾;既又聞準噶爾兵至奎素,復奏請調蒙、漢兵七千人赴推河。上責馬爾賽展轉不定,命駐第十四臺待命。旋命將蒙、漢兵五千人駐翁袞。上解傅爾丹靖邊大將軍印授順承郡王錫保,諭馬爾賽,蒙古諸扎薩克俱遵靖邊大將軍調遣,不得以撫遠大將軍印有所徵發。尋改授撫遠將軍,駐扎克拜達里克。

十年秋,準噶爾大舉內犯,掠喀爾喀諸部。喀爾喀親王策棱與戰額爾德尼昭,大破之,餘眾循鄂爾昆河源走推河。錫保劄馬爾賽,令與建勛將軍達爾濟合軍截擊,喀爾喀親王丹津多爾濟亦馳報,促馬爾賽發兵。馬爾賽集諸將議,諾爾琿曰:「我等當速發兵迎截,遲且將不及。」諸將皆和之,獨都統李杕以為但當守城,馬爾賽以杕言為然。諾爾琿、博爾屯等力請,傅鼐至跪求,馬爾賽持不可。達爾濟遣使約會師,馬爾賽終不應。士卒登城見敵過,奮欲出擊,參贊胡琳、傅鼐不待馬爾賽令,將所部以出,馬爾賽乃與偕行。至博木喀拉,令欽拜將七百人逐敵,馬爾賽引還。準噶爾兵去已遠,欽拜等亦無所獲而返。胡琳、欽拜、博爾屯、諾爾琿等先後疏報,上命奪馬爾賽官爵治罪,錫保等請誅馬爾賽及杕,部議當貽誤軍機律斬。十二月,遣副都統索林赴扎克拜達里克,斬馬爾賽。

李杕,漢軍鑲藍旗人,李國翰四世孫。降襲三等伯,累擢至廣州將軍。坐駐防兵閧巡撫官廨,逮京師論斬,上貸之,復授都統,仍令襲爵。至是,責其一言僨事,罪與馬爾賽等,奪官爵,論斬。

慶復,字瑞園,佟佳氏,滿州鑲黃旗人,佟國維第六子。雍正五年,襲一等公,授散秩大臣。遷鑾儀使,兼領武備院事。七年,授正白旗漢軍副都統。八年,遷正藍旗漢軍都統。九年,列議政大臣。十一年,授工部尚書,署刑部,調戶部。十二年,授領侍衛內大臣。十三年,高宗即位,命代平郡王福彭為定邊大將軍,出北路。乾隆元年,準噶爾乞和,罷兵。慶復請沿邊設卡倫,以侍衛或護軍一專管,喀爾喀臺吉一協理;發土謝圖、賽因諾顏、扎薩克圖、車臣四部兵合三千人,歲六月集鄂爾坤出巡,九月罷歸牧:詔如所請。召還京,署吏部尚書,兼戶部,尋真除刑部。二年,授兩江總督。劾江西巡撫俞兆岳貪鄙營私,奪官,論如律。疏言蘇、常、揚、鎮、通、泰諸屬例徵麥二萬餘石,請改徵米,從之。

移督云、貴。四年,加太子少保。五年,疏言:「雲南府屬縣引南汁等六河溉田,山溪箐澗水發不常,沙石壅遏,堤埂易決。請以時修治。」上嘉之。又言:「滇、黔、粵、蜀四省接壤,瑤、苗雜處,往往爭界構訟,積案莫結。如廣西鎮安屬小鎮安土州與雲南廣南屬土目爭剝頭、者賴二村,臣令詳勘,以村入廣西境應歸廣西;而廣西又議以小鎮安土州歸雲南,畫昭陽關為界。雲南、四川於金沙江分界,雲南屬江驛、七戛、則補、晉毛諸地越在江外,兩省駐汛分防,犬牙互制,而四川又欲劃江分界。現在民、夷寧帖,應仍舊貫,不必紛更。其或田在彼境,糧在此境,當以糧從田,俾免牽混。」下軍機大臣議行。又疏言錢價日昂,請省城增十爐,臨安增五爐,發餉銀七錢三。下部議行。又分疏請開姚州鹽井,南安州屬咢嘉、大小猛光、回子門諸地招墾,濬治金沙江。

旋移督兩廣,疏劾粵海關監督鄭伍賽需索侵蝕,擬罪如律。又疏言:「瓊州四面環海,中有五指山,黎人所居。請設義學,俾子弟就學應試,別編『黎』字,州縣額取一名。」八年,又疏言:「廣西東蘭州自雍正初改土為流,置兵二百戍守。水土毒惡,山路崎嶇,民病於運糧。請以其半改駐三旺。」均從之。

復移督川、陜。郭羅克土番處青海界上,地寒不能畜牧,屢出為「夾壩」,夾壩,華言盜也。慶復令捕其酋林噶架立誅之,番眾頂經誓奉約束。慶復令貧番三百餘戶授地課耕,歲五六月許出獵,限一次,寨限十五人。要隘設汛置兵,松潘鎮總兵歲出巡,駐阿壩。番人訟不決,詣總兵剖晰。上中下三部置土千戶一、土百戶二,種人為盜,責三土目捕治。疏聞,下軍機大臣議行。又有瞻對土司在打箭爐邊外,處萬山中,恃險肆劫,掠及臺站兵,有司捕治。上瞻對土目四朗、下瞻對土目班滾匿罪人不出。

十年,慶復偕巡撫紀山、提督李質粹疏請發兵進剿,上命宜妥協周詳,毋少疏忽。慶復遂發兵,質粹進駐東俄洛,扼兩瞻對總隘;夔州副將馬良柱出里塘為南路,松潘總兵宋宗璋出甘孜為北路,建昌總兵袁士弼出沙晉隆為中路,刻期並發,四朗詣宗璋軍降。士弼自擴城頂趨納爾格,與番人戰加社袨卡諸地,屢勝。良柱攻嚓嗎所,焚其寨三,地雷發,番人死甚眾。上下瞻對夾江而居,四朗居江西地,曰撒墩,其從子肯硃居江東地,曰孺耳,班滾亦居江西地,曰如郎。江東木魯工為要隘。四朗既降,宗璋兵越撒墩駐阿賽,去如郎數十里,良柱亦逼進如郎,質粹發兵往應,班滾力拒。宗璋分兵自然多會士弼,克臘蓋,破底硃。良柱亦撫定番寨四十六。班滾請降,慶復不許。疏入,上命毋恃勝輕敵。尋授慶復文華殿大學士,仍留總督。

十一年春,慶復進駐東俄洛,奏言:「前克底硃,班滾母率頭人至軍前請降,質粹遣令歸。臣咨詢質粹,令速進兵。」上責質粹失機,慶復又疏劾士弼意主招降,請奪官,仍戴罪效力。尋自東俄洛進駐靈雀,以明正土司汪結及降人騷達邦、俄木丁等為導,自茹色以皮船渡,破十餘卡,逼如郎,攻泥日寨,圍之數日,焚碉。質粹咨慶復,言班滾已焚死,又言焚碉時,火光中望見番酋懸縊。慶復詢於眾,俄木丁於燼中得鳥槍銅捥,謂班滾物也,遂以班滾焚斃疏聞。上察慶復師逼如郎時,嘗奏班滾走沙加邦河,土目姜錯太迎入寨,未言至泥日;諭慶復,班滾渠魁斷不可漏網,毋留遺孽,毋墮狡計。尋加慶復太子太保。慶復又劾士弼怯懦乖張,奪官,逮下刑部論罪。

十二年,大金川土酋莎羅奔為亂,上授張廣泗川陜總督,召慶復入閣治事,命兼管兵部。尋廣泗奏言訊土司汪結,言班滾尚匿如郎未死,慶復得班滾子沙加七立,為更名德昌喇嘛,令仍居班滾大碉,冒稱經堂。上責慶復欺罔,奪官待罪。欽差大臣尚書班第奏言師克如郎,班滾已逃,僅得空寨。上逮質粹下刑部獄,召宗璋與質。質粹言:「曩報班滾焚斃,實未親見;後聞藏匿山洞,亦未告慶復追捕。」上命下慶復刑部獄,令軍機大臣會訊,按律定擬,坐貽誤軍機律論斬。十四年九月,賜自盡。

李質粹,漢軍正白旗人。雍正初,自把總擢藍翎侍衛。嘗從年羹堯出師,累擢陜西、固原提督。丁憂,命署四川提督。附和慶復妄言班滾死,慶復死之明年,斬質粹。

張廣泗,漢軍鑲紅旗人。以監生入貲授知府。康熙六十一年,選貴州思州。雍正四年,調雲南楚雄。雲貴總督鄂爾泰討亂苗,以廣泗佐其事,奏改調黎平。五年,擢貴州按察使。六年,廣泗率兵赴都勻、黎平、鎮遠、清平諸地化導群苗,相機剿撫,超授巡撫。清平屬丹江苗最悍,廣泗遣兵分道攻克小丹江、大丹江及雞溝等寨。鎮遠屬上九股諸寨與接壤,亦次第降。下九股、清水江、古州諸苗悉定。疏聞,上命與鄂爾泰詳議善後諸事,語詳鄂爾泰傳。十年,廣泗疏言:「清水江及都江為黔、楚、粵三省通流,當設哨船聯絡聲勢。古州應貯米,責成同知以下董理。譯人分別勤惰予糈,並授土官劄付,宣布條約,化導苗民。」下部議行。敘功,授拜他喇布勒哈番世職。

準噶爾擾邊,寧遠大將軍岳鍾琪率師出西路。上授廣泗副將軍,召詣京師授方略。廣泗至軍,鍾琪方自巴爾廣爾移軍穆壘。廣泗將四千人出鄂隆吉,與鍾琪會於科舍圖,至穆壘。上召鍾琪還京師,命廣泗護大將軍印。廣泗疏言:「穆壘地處兩山間,築城其中,形如釜底,非屯兵進取之地。今築城未竟,臣與副將軍常賚兩營當要沖,兵止二三百,即鍾琪營亦僅數百,遇警何以抵禦?準噶爾專用馬,我兵必馬步兼用,而鍾琪立意用車,沙磧殊非所宜。至馬步兵弓箭、鳥槍之外,止攜木梃,全無刀戟,官兵莫不竊議。穆壘又無牧地,鍾琪留馬二千餘,悉就牧烏蘭烏蘇、科舍圖兩地,敵人窺伺可虞。駐兵數萬人,糧運最要。地多叢山大嶺,車駝分運,必繞出沙磧。鍾琪聞寇至,輒令停運,以此遲緩。鍾琪張皇剛愎,號令不明。題奏奉到諭旨,臨時宣傳,莫測誠偽。」上奪鍾琪官,命廣泗還軍巴爾庫爾。廣泗奏軍還巴爾庫爾,分兵防洮賚、無克克嶺,斷敵南走道,防廋集察罕、哈馬爾,斷敵西來道;巴爾庫爾北為鏡兒泉、噶順、烏卜圖克勒克諸地,東北為圖古里克、特爾庫勒諸地,敵自沙磧來,處處可通,皆置兵守。他諸要隘並設卡倫,巡護牧廠,哈密、塔勒納沁皆增兵為備。尋以查郎阿為大將軍,授廣泗正紅旗漢軍都統,留軍。十一年,廣泗將萬餘人分駐北山。十二年,詗寇至烏爾圖河,檄副都統班第達什、降調總兵張元佐及提督樊廷逐捕,越噶順至鄂隆吉大阪,擊破之,斬四百餘人,獲三十六人。捷聞,命議敘。十三年,準噶爾乞和,師還。授湖廣總督。

自鄂爾泰定苗疆,至是九股苗復為亂。尚書張照偕將軍哈元生、副將軍董芳率兵討之,久無功。高宗即位,授廣泗經略,赴貴州,將軍以下聽節制。廣泗疏劾照阻撓軍機,徵集兵數萬,元生沿途分布,用以攻剿者不過三千,顧此失彼。芳駐守八弓,僅事招撫。巡撫元展成治賑,條款紛錯,官民並困。上為奪照、芳、展成等官,命廣泗兼領貴州巡撫;罷元生將軍,以提督聽廣泗驅策。十二月,廣泗至凱里,分兵三道進剿:副將長壽出空稗,總兵王無黨出臺營,廣泗督兵出清江地曰雞擺尾,刻期並進。破上九股卦丁等寨,毀其巢,餘苗走入牛皮大箐。乾隆元年正月,廣泗令諸軍合圍,獲其渠包利等,斬萬餘級,諸苗悉定。授廣泗云貴總督,兼領巡撫,進三等阿達哈哈番世職。奏定鎮遠、安順、大定、平遠諸營制,增貴州兵額,都計二千九百有奇。三年,復請濬治清水江、都江,增爐鑄錢。皆下部議行。五年,請入覲,會湖廣城步橫嶺等寨紅苗糾粵瑤為亂,命廣泗往勘。九月,授欽差大臣,楚、粵提鎮以下受節制。十一月,亂定。六年正月,至京師,乞歸葬,賜其父母祭。貴州黎平黑苗復糾粵瑤為亂,命廣泗還貴州按治,獲苗酋石金元等置之法。十年,加太子少保。

十一年,大金川土司莎羅奔為亂,調川陜總督。廣泗至軍,小金川土司澤旺土舍良爾吉來降。八月,遣總兵宋宗璋、許應虎分道攻勒烏圍,副將馬良柱攻噶拉依,副將張興、參將買國良繼進。山險碉堅,轉戰逾二年,師無功。十三年,疏劾良柱自丹噶撤軍失砲械,命逮詣京師。上授大學士訥親經略,出視師,並起岳鍾琪赴軍,詔責廣泗師老氣怯,調度失機宜。廣泗奏報攻克戎布寨五十餘碉,諭曰:「此亦小小攻克耳。佇待捷音,以慰西顧。」訥親初至,督攻碉,師敗績。總兵任舉為驍將,戰沒。乃議令官軍築碉,謂與賊共險。上以為非策,責廣泗附和推諉,嚴諭詰難。訥親劾廣泗分十道進兵,兵力微弱,老師糜餉;鍾琪亦劾廣泗玩兵養寇,信用良爾吉及漢奸王秋,洩軍事於敵。上責廣泗貽誤軍機,奪官,逮至京師,上御瀛臺親鞫。廣泗極言其枉,命用刑,辨不已。上諭曰:「金川用兵,張廣泗、訥親前後貽誤。廣泗初至軍,妄為大言,既久無成效,則諉過於部將。及訥親往,乃復觀望推諉,見訥親種種失宜,無一語相告。見其必敗,訕笑非議,備極險忮。蓋恐此時奏聞,猶或譴責,不若坐視決裂為得計也。朕詳悉推勘,如見肺肝。訥親且在其術中而不覺矣。廣泗熟嫻軍旅,與訥親並為練達政事之大臣,乃自逞其私,罔恤國事。今朕明正其罪,以彰國憲。」下軍機大臣會刑部議罪,當失誤軍機律斬。十二月,斬廣泗。後十日,諭並誅訥親。

論曰:為三軍擇將,豈易言哉?查郎阿臨邊未遇敵,按殺成斌、勷。世謂與查廩有連為修怨,甚矣其枉也!傅爾丹中敵間,師徒撓敗,世宗特寬之;高宗時復起,至與岳鍾琪同視,何其幸歟!若馬爾賽之畏縮,慶復之欺誑,譴當其罪。廣泗傾鍾琪,劾照,知訥親不可撼,乃坐視其敗,以忮殺其身,雖有勞不能逭。籲,可畏哉!


\end{pinyinscope}