\article{志二}

\begin{pinyinscope}
天文二

△儀象

漢創渾天儀,謂即璣衡遺制,唐、宋皆仿為之。至元始有簡儀、仰儀、闚幾、景符等器,視古加詳焉。明於北京齊化門內倚城築觀象臺,仿元制作渾儀、簡儀、天體三儀,置於臺上。臺下有晷景堂、圭表、壺漏,清初因之。康熙八年,聖祖用監臣南懷仁言,改造六儀,曰黃道經緯儀、赤道經緯儀、地平經儀、地平緯儀、紀限儀、天體儀。五十二年,復將地平經、緯合為一儀。乾隆九年,高宗御制璣衡撫辰儀,並安置臺上。今考各形制用法,悉著於篇。

黃道經緯儀,儀之圈有四,各分四象限,限各九十度。其外大圈恆定而不移者,名天元子午規,外徑六尺,規面厚一寸三分,側面寬二寸五分,規之下半夾入於雲座。仰載之半圈,前後正直子午,上直天頂,中直地平。從地平上下按京師南北兩極出入度分,定赤道兩極。次內為過極至圈,圈周平分處,各以鋼樞貫於赤道二極。又依黃赤大距度,於過極至圈上定黃道南北極。距黃極九十度安黃道圈,與過極至圈十字相交,各陷其中以相入,令兩圈為一體,旋轉相從。黃道圈之兩側面,一為十二宮,一為二十四節氣。其兩交,一當冬至,一當夏至。次內為黃道經圈,則以鋼樞貫於黃極焉。圈之徑為圓軸,圍三寸。軸之中心立圓柱為緯表,與經圈側面成直角,而黃道圈經圈上各設游表,儀頂更設銅絲為垂線。全儀以雙龍擎之,復為交梁以立龍足。梁之四端,各承以獅,仍置螺柱以取平。垂線有偏側,則轉螺柱,垂線正,則儀正矣。用法,欲求某星黃道經緯度,須一人於黃道圈上查先所得某星之經緯度分,其上加游表,而過南北軸中柱,表對星定儀;又一人用游表於經圈上過柱,表對所測之星,游移取置,則經圈上游表之指線定某星緯度。又定儀查黃道圈兩表相距之度分,即某星之經度差。或測日月,以距星為比,亦如之。

赤道經緯儀,儀有三圈,外大圈者,天元子午規也。以一龍南向而負之。規之分度定極,皆與黃道儀同。去極九十度安赤道圈,與子午規十字相交,恆定不動。圈內規面及上側面皆鋟二十四時,時各四刻。外規面分三百六十度,內安赤道經圈。以南北極為樞,而可東西游轉,與赤道圈內規面相切。經圈徑為圓軸,軸中心立圓柱,以及游表、垂線、交梁、螺柱等法,皆同黃道儀。用法,若測日時刻,則赤道圈上用時刻游表,即通光耳,對於南北軸表,視赤道圈內游表所指,即時刻分秒。若諸曜經度,用兩通光耳,即兩徑表,在赤道圈上一定一游。一人從定耳窺南北軸表,與先得星相參測之;一人以游耳轉移遷就,而窺本軸表與所測參相直,視兩耳間應赤道圈上之度分,即兩經度之差也。緯度亦以通光耳於經圈上轉移而遷就焉。務欲令目與表與所測相參直,視本耳下經圈在赤道或南或北之度分,即所測距赤道南北之度分也。

地平經儀,儀只一地平圈,全徑六尺,其平面寬二寸五分,厚一寸二分。分四象限,各九十度。以四龍立於交梁以承之。梁之四端,各施取平之螺柱。梁之交處安立柱,高與地平圈等,適當地平圈之中心。又於地平圈上東西各立一軸,約高四尺,柱各一龍,盤旋而上,從柱端各伸一爪,互捧圓珠。下有立軸,其形扁方,空其中如窗欞,以安直線。軸之上端入於珠,下端入立柱中心,令可旋轉。而軸中之線,恆為天頂之垂線焉。又為長方橫表,長如地平圈,全徑厚一寸,寬一寸五分,中心開方孔管於立軸下端,使隨立軸旋轉。復剡其兩端令銳,以指地平圈之度分。又自兩端各出一線,而上會於立軸中直線之頂,成兩三角形。凡有所測,則旋轉游表,使三線與所測參相直,乃視表端所指,即所測之地平經度也。

地平緯儀,即象限儀,蓋取全圓四分之一以測高度者也。其弧九十度,其兩邊皆圓半徑,長六尺。兩半徑交處為儀心。儀架東西立柱,各以二龍拱之。上架橫梁,又立中柱上管於橫梁,令可轉動儀心,上指儀之兩邊,一與中柱平行,一與橫梁平行。又於儀心立短圓柱以為表,又加窺衡,長與半徑等,上端安於儀心,剡其下端,以指弧面度分,更安表耳。有所測,乃以窺衡上下游移,從表耳縫中窺圓柱,令與所測相參直。其衡端所指度分,即所測之地平緯度也。

紀限儀,弧面為全圓六分之一,分六十度。一弧一幹,幹長六尺,即全圓之半徑。弧之寬二寸五分,幹之左右,細雲糾縵纏連,所以固之。幹之上端有小衡,與幹成十字。儀心與衡兩端皆立圓柱為表,弧面設游表。承儀之臺,約高四尺,中植立柱,以系儀之重心,則左右旋轉,高低斜側,無所不可,故又名百游儀焉。用法,測兩曜,不論黃赤經緯,而求大圈相距之度,一人從衡端耳表窺中心柱表,對定此曜;又一人從游耳表向中心柱表窺彼曜相參直,視衡端至游耳表下度分,即兩曜相距度分也。

天體儀,儀為圓球,徑六尺,宛然穹象,故以天體名之。中貫鋼軸,露其兩端,以屬於子午規之南北極,令可轉運。座高四尺七寸。座上為地平圈,寬八寸。當子午處各為闕,以入子午規。闕之度與子午規之寬厚等,則兩圈十字相交,內規面恰平,而左右上下環抱乎儀。周圍皆空五分,以便高弧游表進退。又安時盤於子午規外,徑二尺,分二十四時。以北極為心,其指時刻之表,亦定於北極,令能隨天體轉移,又能自轉焉。座下復設機輪,運轉子午規,使北極隨各方出地升降,各方天象隱見之限,皆可究觀矣。

地平經緯合儀,經儀中心立柱安緯儀。用法,旋轉緯儀,對定所測游表,於緯儀上得緯度;視緯儀邊切經儀之處,即得經度:一測而兩得焉。

璣衡撫辰儀,儀制三重,其在外者,即古六合儀,而不用地平圈,其正立雙環為子午圈。兩面皆刻周天三百六十度,自南北極起,初度至中要九十度,是為天經。斜倚單環為天常赤道圈,兩面皆刻周日十二時,以子正午正當子午雙環中空之半,而結於其中要,是為天緯。其南北二極皆設圓軸,軸本實於子午雙環中空之間,而軸內向,以貫內二重之環。其下承以雲座,仰面正中開雙槽以受雙環,東面正中開雲窩以受垂線。下面置十字架,施螺旋以取平。架之東西兩端各植龍柱,龍口銜珠,開孔以承天常赤道卯酉之兩軸,依觀象臺測定南北正線,將座架安定,則平面之四方正。又依京師北極出地度分,上數至成一象限,即天頂。依南極入地度分,下數成一象限,即地心。於天頂施小釘懸垂線,而垂適當地心,又適切於雙環之面。線末垂球,又適當云窩,不即不離,則上下正立面之四方亦正,而地平已在其中。

次其內即古三辰儀,而不用黃道圈。其貫於二極之雙環,為赤極經圈。兩極各設軸孔,以受天經之軸,兩面皆刻周天三百六十度。結於赤極圈之中要,與天常赤道平運者,為游旋赤道圈,兩面皆刻周天三百六十度,與宗動天赤道旋轉相應。自經圈之南極,作兩象限弧以承之,使不傾墊。

次最內即古四游儀,貫於二極之雙環,為四游圈,兩面皆刻三百六十度。定於游圈之兩極者為直距,綰於直距之中心者為窺衡。游圈中要設直表,以指經度及時,窺衡右旁設直表,以指緯度。別設借弧指時度表、立表、平行立表、平行借弧表,以濟所測之窮。又設綰經度表、綰時度表、平行線測經度表,以期兩測之合。

其數,子午圈外徑六尺三寸,內徑五尺六寸六分,環面闊三寸二分,厚九分,中空一寸。天常赤道外徑六尺一寸二分,內徑五尺六寸四分,環面闊二寸四分,厚一寸四分。赤極經圈外徑五尺五寸六分,內徑五尺一寸二分,環面闊二寸八分,厚八分,中空一寸二分。游旋赤道外徑五尺五寸六分,內徑五尺一寸二分,環面闊二寸二分,厚一寸二分。四游圈外徑五尺,內徑四尺六寸八分,環面闊一寸六分,厚七分,中空一寸四分。直距長如圓之通徑,闊一寸六分,厚七分,中空一寸四分。窺衡長四尺七寸二分,方一寸二分,中空一寸。上下兩端施方銅★,厚五分,內三分,方一寸,入於管中,外二分,方一寸二分,齊於管面,中心開圓孔。

指時度表,通長七寸三分,本長一寸六分,形如方筒,入於四游雙環中空之間,闊一寸四分,橫帶長三寸二分,闊五分,兩端各金句回二分,扣於環面之外。表長五寸二分,闊一寸。其指時度之邊線,對方筒之正中,下端二寸四分,厚三分,切於游旋赤道之面,以指度分。上端二寸八分,厚二分,切於天常赤道之面,以指時刻。

指緯度表,其形兩曲,安於窺衡之右面。底長三寸,闊九分,曲橫七分,與四游環之厚等。又曲長一寸七分,切於四游環之外面,從中線減闊之半,所以指緯度也。

借弧指時度表,其本方筒及橫帶長闊並與前指時度表同。橫帶之下,自左向右,立安弧背一道,長九寸三分,闊一寸二分,厚一分六釐。弧背之末,平安指時度表,除弧背之厚,長五寸二分,闊一寸。計自表本方筒之中線至指時度表之內邊,長六寸七分,當游旋赤道之十五度,當天常赤道之一小時。

立表二座,形直底平,表高底長各三寸二分,闊九分,厚一分。一表向上開長方孔,長一寸,中留直線,又上五分開圓孔,徑四分,中留十字線,安於窺衡之上端。一表依前度下開直縫,上開小圓孔,安於窺衡下端,各對衡面中線,以螺旋結之。

平行立表二座,形曲底平,底盤長四寸,闊一寸二分,厚一分,中空三寸二分,闊九分。表曲如勾股。股直如立表,高三寸二分,闊九分。勾橫連於股末,長五寸,闊九分,橫植於底盤之末。底盤中空,冒於立表底盤之外,以掐表固之。

平行借弧表,制如平行立表,而倒正異。一表上植於衡面,高四寸一分零八毫,一表自衡面下垂,長六寸二分零八毫。距表端下六分開圓孔,又下五分開長方孔,皆與立表制同。

綰經度表,通長四寸,闊一寸四分。其本方筒長一寸六分,高一寸八分,入於四游雙環之間,以左右螺旋固之。其末上下二面,以夾游旋赤道,上面闊七分,減本之半,與窺衡中線相直,下面以螺旋固之。

綰時度表,內外二截,內截上下內三面,綰於游旋赤道之內規。上面之末,承於外截之下,開二方孔,以受外截之方足,下面以螺旋固之。外截上下外三面,綰於天常赤道之外規。上面之末,覆於內截之上,下面以螺旋固之。

平行線測經度表,於直距南北極之兩端,各安銅版,如工字形,正方二寸八分,與直距二面之分等。兩要各缺一長方,長一寸六分,闊七分,扣於直距中空之間。中心開圓孔,貫於天經之軸。四隅距中心一寸九分,各安立柱,圓頂開孔,以穿直線,與直距中徑平行。下安小環,以為結赤經平行線之用。又按距星宮度,於游旋赤道安赤經平行線表,其制上畫半圓,內容半方,自對角斜線起,初度至橫徑為四十五度,其中直徑與指度表之邊線相參直。半圓中心安二游表,各長二寸,距中心一寸九分。邊留小臍,中開小圓孔,以線穿之。上端系於北極銅版對角之兩環,下端貫於南極銅版對角之兩環,各以垂球墜之。

用法,測日時刻,以四游圈東西推轉,窺衡南北低昂,令日光透孔圓正,視四游圈下指時度表臨天常赤道某時刻,即得。若日景為赤道所礙,則用窺衡上立表測之,令表兩孔正透,仍於指時度表視時刻。或為龍柱所礙,則用平行立表測之,亦於指時度視時刻。若指時度表為子午圈所礙,則易用借弧指時度表,次用平行立表。測定日景,視借弧指時度表所指時刻,加一小時,即得。測經度,取所知正午前後一恆星,以其赤道經度之對沖,用綰經度表於游旋赤道綰定四游圈。又任設一時,用綰時度表,於其時之對沖,綰天常赤道。乃將四游圈帶定游旋赤道,用窺衡測準距星,隨之左旋。候至所設時刻,視綰時度表對游旋赤道某宮度分,即日赤道經度。或以本時太陽赤道經度,用綰時度表於游旋赤道綰定,又以所設時刻之對沖,於天常赤道綰定。候至所設時刻,用四游窺測月星,乃視指時度表所指游旋赤道宮度,加半周,即得所測月星赤道經度。測兩曜相距經度,用平行線測經度表於游旋赤道初宮初度安定,令一人用此平行線表、左兩線、右兩線,並窺定距西之曜,隨之左旋;一人用四游窺衡測距東之曜,視指時度表所指游旋赤道之度分,即所測兩曜相距赤道經度也。測緯度,凡得經度時,隨察指緯度表所指四游圈之度分,即得所測赤道緯度。其有所礙,皆如測時刻法易之。其近北極之星,則以平行借弧表測之。


\end{pinyinscope}