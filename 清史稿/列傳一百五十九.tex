\article{列傳一百五十九}

\begin{pinyinscope}
裕謙謝朝恩重祥關天培陳連升祥福江繼蕓

陳化成海齡葛云飛王錫朋鄭國鴻硃貴

裕謙,原名裕泰,字魯山,博羅忒氏,蒙古鑲黃旗人,一等誠勇公班第曾孫,綏遠城將軍巴祿孫。父慶麟,京口副都統。

裕謙,嘉慶二十二年進士,選庶吉士。散館改禮部主事,遷員外郎。道光六年,出為湖北荊州知府,始改今名。調武昌,歷荊宜施道、江蘇按察使。十九年,就遷布政使,署巡撫,尋實授。

二十年,英兵陷定海,伊里布奉命往剿,裕謙代署兩江總督。時英艦游奕海門外洋,江南戒嚴。裕謙赴寶山、上海籌防,檄徐州鎮總兵王志元,佐提督陳化成防海口。疏陳規復定海之策,可無慮者四,難緩待者六,謂各省皆可言守,浙江必應議戰,且應速戰。又疏劾琦善五罪,略曰:「英人至天津,僅五船耳,琦善大張其事,遽稱:『畿疆、遼、沈處處可虞,後來之艦尚多,勢將遍擾南北』。冀聳聽聞,以掩其武備廢弛之咎。張皇欺飾,其罪一。英酋回粵以來,驕桀日甚,琦善惟責兵將謝過,別未設籌,將士解體,軍心沮喪。彼軍乘敝,遂衄我師。我船砲縱不如彼,兵數何啻十倍。琦善不防後路,事敗委過前人。試思琦善未至粵時,未聞失機,其又何說?弛備損威,其罪二。沙角、大角砲臺既失,自應迅駐虎門,乃其奏中不及剿堵事,惟以覆書緩兵為詞,且囑浙省勿進兵。旋以給香港、即日通商定議,不俟交還定海後奏允奉行。違例擅權,其罪三。既畀香港換出定海,而英人仍欲通商寧波,銷售鴉片。何以不在粵翦斷葛藤?將就茍且,其罪四。義律僅外商首領,向來呈牘,自稱遠商遠職。上年在天津、浙江僭稱公使大臣,琦善不之詳,假以稱號。失體招釁,其罪五。琦善已為英人藐玩,各國輕視,不宜久於其任。」疏上,宣宗憤琦善受紿,斥伊里布附和,信裕謙忠直可恃。二十一年春,罷伊里布,以裕謙代之。

裕謙至鎮海,英艦已去定海,渡海往治善後事宜。尋實授兩江總督,以浙事付巡撫劉韻珂、提督餘步雲,自回江南部署防務。初,英兵在定海,殘虐人民,既退,猶四出游奕。裕謙捕獲兵目,剝皮抽筋而懸之,又掘敵尸焚於通衢。英人遂藉口復仇,大舉再犯浙洋,裕謙率江寧駐防及徐州鎮兵千,馳至鎮海督戰,令總兵葛云飛、鄭國鴻、王錫朋率兵五千守定海,手緘密諭,付臨陣啟視,退者立斬。

八月,敵艦二十九艘、兵三萬來攻,分三路並進,血戰六晝夜,三鎮並死之,定海陷。越數日,敵由蛟門島進犯鎮海,招寶山為要沖,餘步雲守之,別遣總兵謝朝恩守金雞嶺為犄角。裕謙疑步雲懷兩端,乃集將士祭關帝、天後,與眾約:「毋以退守為詞,離城一步;亦毋以保全民命為詞,受洋人片紙。不用命者,明正典刑,幽遭神殛!」步雲知其意,不預盟誓。及戰,裕謙登城,手援枹鼓,步雲詣請遣外委陳志剛赴敵艦,暫示羈縻,裕謙不許。有頃,敵登招寶山,步雲不戰而退。敵復分兵攻金雞嶺,謝朝恩中砲殞,兩山同陷,鎮海守兵望風而潰。裕謙先誓必死,一日經學宮前,見泮池石鐫「流芳」二字,曰:「他日於此收吾尸也!吾曾祖於乾隆二十一年八月殉難,今值道光二十一年八月,非佳兆。」預檢硃批寄諭、奏稿送嘉興行館,處分家事甚悉。臨戰,揮幕客先去,曰:「勝,為我草露布;敗,則代辦後事。」至是果投泮池,副將豐伸泰等拯之出,輿至府城,昏憊不省人事。敵且至,以小舟載往餘姚,卒於途,遂至西興,劉韻珂等視其斂。事聞,贈太子太保,予騎都尉兼一雲騎尉世職,附祀京師昭忠祠,於鎮海建立專祠,謚靖節。柩至京,遣成郡王載銳奠醊。

當初敗,餘步雲疏報鎮海大營先潰,裕謙不知所往。韻珂等奏至,上始釋疑,予優恤。幕客陳若木從兵間代裕謙妻草狀,詣闕訟冤,逮步雲論治伏法。嗣子德崚襲世職,以主事用,官至山東候補知府。

謝朝恩,四川華陽人。由行伍從將軍德楞泰剿教匪,積功至都司。累擢閩浙督標副將,從平臺灣張丙亂。道光十四年,擢狼山鎮總兵。從伊里布防鎮海,充翼長。裕謙令守金雞嶺,力戰禦敵。敵別出一隊由沙蟹嶺繞出山後夾攻,遙見招寶山威遠城已為敵踞,兵遂潰。朝恩扼砲臺,中敵砲,墮海,尸不獲。浙人有親見其死者,歌詠傳其事,與葛云飛等同稱四鎮云。賜恤,予騎都尉世職。

重祥,張氏,漢軍正黃旗人。世襲一等輕車都尉,金華協副將。從葛云飛戰定海受傷,復佐守金雞嶺,力戰死之。處州營游擊托云保,卞氏,亦漢軍旗人,偕重祥同殞於陣,並予雲騎尉世職。

關天培,字滋圃,江蘇山陽人。由行伍洊升太湖營水師副將。道光六年,初行海運,督護百四十餘艘抵天津,被優敘。七年,擢蘇松鎮總兵。十三年,署江南提督。十四年,授廣東水師提督。時英吉利通商漸萌跋扈,兵船闌入內河,前提督李增階以疏防黜,天培代之。至則親歷海洋厄塞,增修虎門、南山、橫檔諸砲臺,鑄六千斤大砲四十座,請籌操練犒賞經費。十八年,英人馬他倫至澳門,託言稽察商務,投函不如制,天培卻之。禁煙事起,偕總督鄧廷楨偵緝甚力。

十九年,林則徐蒞廣東,檄天培勒躉船繳煙二萬餘箱焚之,於是嚴海防,橫檔山前海面較狹可扼,鑄巨鐵練橫系之二重,阻敵舟不能逕過,砲臺乃得以伺擊。則徐倚天培如左右手,常駐沙角,督本標及陽江、碣石兩鎮師船排日操練。七月,英艦突犯九龍山口,為參將賴恩爵擊退。九月,二艦至穿鼻洋,阻商船進口,挑戰。天培身立桅前,拔刀督陣,退者立斬。有擊中敵船一砲者,立予重賞,發砲破敵船頭鼻,敵紛紛落海,乃遁。

敵艦久泊尖沙嘴,踞為巢穴。迤北山梁曰官湧,俯視聚泊之所,攻擊最便,天培增砲駐營,敵屢乘隙來爭,不得逞。十月,敵以大艦正面來攻,小舟載兵從側乘潮撲岸,殲之於山岡;復於迤東胡椒角窺伺,砲擊走之。乃調集水陸兵守山梁,參將陳連升、賴恩爵、張斌,游擊伍通標、德連等為五路,合同進攻。敵乘夜來犯,五路大砲齊擊,敵舟自撞,燈火皆滅。侵曉了望,逃者過半,僅存十餘舟遠泊。次日,復有二敵艦潛進,隨者十數,復諸路合擊,毀其頭船,遂散泊外洋。捷聞,詔嘉獎,賜號法福靈阿巴圖魯。二十年春,英艦雖不敢復進,猶招奸民分路載煙私售。天培沿海搜捕,一日數起,復飭漁船蟹艇乘間焚毀敵舟,英人始改計他犯。

及林則徐罷,琦善代之,一意主撫,至粵,先撤沿海防禦,僅留水師制兵三分之一,募勇盡散,而英人要索甚奢,久無定議,戰釁復起。十二月,英船攻虎門外沙角砲臺,副將陳連升死之,大角砲臺隨陷,並為敵踞,虎門危急。天培與總兵李廷鈺分守靖遠、威遠兩砲臺,請援,琦善僅遣兵二百。二十一年正月,敵進攻,守臺兵僅數百,遣將慟哭請益師,無應者。天培度眾寡不敵,乃決以死守,出私財餉將士,率游擊麥廷章晝夜督戰。敵入三門口,沖斷椿練,奮擊甫退,南風大作,敵船大隊圍橫檔、永安兩砲臺,遂陷。進攻虎門,自巳至酉,殺傷相當,而砲門透水不得發,敵自臺後攢擊,身被數十創。事急,以印投僕孫長慶,令去,行未遠,回顧天培已殞絕於地,廷章亦同死,砲臺遂陷。長慶縋崖出,繳印於總督,復往尋天培尸,半體焦焉,負以出。優恤,予騎都尉兼一雲騎尉世職,謚忠節,入祀昭忠祠,建立專祠。母吳年逾八十,命地方官存問,給銀米以養餘年。子從龍襲世職,官安徽候補同知。

陳連升,湖北鶴峰人。由行伍從征川、楚、陜教匪,湖南、廣東逆瑤,數有功。累擢增城營參將。道光十九年,破英兵於官湧,擢三江協副將,調守沙角砲臺。及英艦來犯,連升率子武舉長鵬以兵六百當敵數千,發地雷扛砲斃敵數百,卒無援,歿於陣,長鵬赴水死。敵以連升戰最猛,臠其尸。事聞,詔嘉其父子忠孝兩全,入祀昭忠祠,並建專祠,加等依總兵例賜恤,予騎都尉世職,子展鵬襲,起鵬賜舉人。

祥福,瑪佳氏,滿洲正黃旗人。由親軍累擢冠軍使。出為湖南寶慶協副將。從提督羅思舉平江華瑤有功。歷綏靖、寧夏、鎮筸諸鎮總兵。二十年,率本鎮兵援廣東。二十一年,守烏湧砲臺,與虎門同時陷,祥福死之,予騎都尉世職,祀昭忠祠。尋詔與關天培同建專祠。子喜瀛,襲世職。

天培等皆以琦善不欲戰,無援,故敗,海內傷之,而福建總兵江繼蕓又以顏伯燾促戰而亡。

繼蕓,福建福清人。由行伍拔補千總。道光六年,臺灣張丙之亂,戰枋樹窩、小雞籠,以擒賊功擢守備。累遷臺灣副將。二十年,署南澳鎮總兵。總督鄧廷楨薦其才,尋擢海壇鎮總兵,調金門鎮,從顏伯燾守廈門。二十一年,廣東方議款,英艦游奕閩洋。伯燾素主戰,庀船砲備出擊,而新裁水勇未散,軍心不堅,繼蕓以為言,伯燾不聽。七月,英艦泊鼓浪嶼,集水陸師禦諸嶼口,砲毀敵舟,而敵已撲砲臺登岸,陸師先潰,繼蕓急赴援,中砲落海死。護理延平協副將凌志、淮口都司王世俊同殉。凌志,富察氏,滿洲鑲黃旗人。

陳化成,字蓮峰,福建同安人。由行伍授水師把總。嘉慶中,從提督李長庚擊蔡牽,數有功,以勇聞。累擢烽火門參將。總督董教增薦其久歷閩、粵水師,手擒巨盜四百八十餘人,勤勞最著,請補澎湖副將,以籍隸本省,格不行。遷瑞安協副將。道光元年,乃調澎湖。歷碣石、金門兩鎮總兵。十年,擢福建水師提督。十二年,英吉利船駛入閩、浙、江南、山東洋面,命化成督師巡邏,以備不虞。同安潘塗、宦潯、柏頭諸鄉素為盜藪,掩捕悉平之。

二十年,英艦犯閩,化成率師船擊之於梅林洋,尋退去。調江南提督。江南水師素怯懦,化成選閩中親軍教練,士氣稍振。籌備吳淞防務,修臺鑄★,沿海塘築二十六堡。化成枕戈海上凡二年,與士卒同勞苦,風雨寒暑不避,總督裕謙、牛鑒皆倚為長城。當定海三總兵戰歿,裕謙亦殉,化成哭之慟,謂所部曰:「武臣死於疆場,幸也。汝曹勉之!」吳淞口以東西砲臺為犄角,化成率參將周世榮守西臺,參將崔吉瑞、游擊董永清守東臺,而徐州鎮王志元守小沙背,以防繞襲。

二十二年五月,敵來犯,泊外洋,以汽舟二,列木人兩舷,繞小沙背鄉西臺,欲試我效力。化成知之,不發,敵舟旋去,以水牌浮書約戰。牛鑒方駐寶山,慮敵鋒不可當。化成曰:「吾經歷海洋四十餘年,在砲彈中入死出生,難以數計。今見敵勿擊,是畏敵也。奉命討賊,有進無退。扼險可勝,公勿怖!」鑒乃以化成心如鐵石,士卒用命,民情固結入告,詔特嘉之。越數日,敵艦銜尾進,化成麾旗發砲,毀敵艦三,殲斃甚眾。鑒聞師得力,親至校場督戰,敵以桅砲注擊,毀演武,鑒遽退。敵攻壞土塘,由小沙背登岸,徐州兵先奔,東臺亦潰,萃攻西臺,部將守備韋印福,千總錢金玉、許攀桂,外委徐大華等皆戰死。尸積於前,化成猶掬子藥親發砲,俄中彈,噴血而殞。砲臺既失,寶山、山海相繼陷。越八日,鄉民始負其尸出,殮於嘉定。事聞,宣宗震悼,特詔優恤,賜銀一千兩治喪,予騎都尉兼一雲騎尉世職,謚忠愍,於殉難處所及原籍並建專祠。子廷芳,襲世職;廷棻,賜舉人。

海齡,郭洛羅氏,滿洲鑲白旗人。由驍騎校授張家口守備。累擢大名、正定兩鎮總兵。以事降二等侍衛,充古城領隊大臣。歷西安、江寧、京口副都統。英兵既陷吳淞,由海入江,六月,犯鎮江,提督齊慎、劉承孝敗退,遂攻城,海齡率駐防兵死守二日,敵以雲梯入城屠旗、民,海齡與全家殉焉。予騎都尉兼一雲騎尉世職,謚昭節,入祀昭忠祠,並建祠鎮江,妻及次孫附祀。當城破時,海齡禁居民不得出,常鎮道周頊棄城走,事後訐海齡妄殺良民,為眾所戕,言官亦論奏,下疆吏究勘得白,詔以闔門死難,大節無虧,仍照都統例賜恤,治頊罪如律。子宜蘭泰,襲世職。

葛云飛,字雨田,浙江山陰人。道光三年武進士,授守備,隸浙江水師。勤於緝捕,常微服巡洋,屢獲劇盜,有名。洊擢瑞安協副將。十一年,署定海鎮總兵,尋實授。以父憂歸。

二十年,英兵犯定海,總兵張朝發戰敗失守,巡撫烏爾恭額、提督祝廷彪強起雲飛墨絰從軍,總督鄧廷楨亦薦其可倚,署定海鎮。雲飛議先守後戰,扼招寶、金雞兩山,列砲江岸,築土城,集失伍舊兵訓練,軍氣始振。英人安突得出測量形勢,以計擒之,敵始有戒心。雲飛乘機圖恢復,未果。二十一年,廣東議款,以香港易定海,欽差大臣伊里布令雲飛率所部渡海收地,然後釋俘,以二鎮帥偕往。二鎮者,壽春鎮王錫朋、處州鎮鄭國鴻也。既而裕謙代伊里布,改議戰守,雲飛以定海三面皆山,前臨海無蔽,請於道頭築土城,竹山、曉峰嶺增砲臺,而道頭南五奎山、吉祥門、毛港悉置防為犄角。裕謙以費鉅未盡許,則請借三年廉俸興築,益忤裕謙。尋至定海,見雲飛青布帕首、短衣草履,奔走烈日中;又聞其巡洋捕盜傷臂,奪盜刃刺之,始服其忠勇。迨英兵復來犯,砲擊敵艦於竹山門、東港浦,迭卻之,加提督銜。於是雲飛屯道頭土城,錫朋、國鴻分防曉峰、竹山。雲飛獨當敵沖,敵連檣進突,登五奎山,砲擊紅衣夷目,乃退。次日,敵蔽山後發砲仰擊,亦隔山應之。夜,敵乘霧至,直逼土城,砲中載藥敵船,轟殲甚眾。越日,乃肉搏來奪曉峰嶺,分攻竹山門,錫朋、國鴻皆戰歿,縣城遂陷。敵萃攻土城,雲飛知不可為,出敕印付營弁,率親兵二百,持刀步入敵中,轉鬥二里許,格殺無算。至竹山麓,頭面右手被斫,猶血戰,身受四十餘創,砲洞胸背,植立崖石而死。定海義勇徐保夜負其尸,浮舟渡海。是役連戰六晝夜,斃敵千餘,卒以眾寡不敵,三鎮同殉。事聞,宣宗揮淚下詔,賜金治喪,恤典依提督例,予騎都尉兼一雲騎尉世職,謚壯節。賜兩子文武舉人,以簡襲世職,官至甘肅階州知州;以敦官守備。

雲飛兼能文,著有名將錄、制械制藥要言、水師緝捕管見、浙海險要圖說及詩文集。事母孝,母亦知大義,喪歸,一慟而止,曰:「吾有子矣!」

錫朋,字樵傭,順天寧河人。以武舉授兵部差官,遷固原游擊。從陜甘總督楊遇春徵回疆,大河拐、洋阿爾巴特、沙布都爾、渾河諸戰並有功,賜花翎,擢湖南臨武營參將。十二年,從剿江華瑤趙金龍,賜號銳勇巴圖魯,擢寶慶協副將。又平廣東連州瑤,功最。擢汀州鎮總兵,以憂歸。十八年,起授壽春鎮總兵。

二十年,偕提督陳化成防吳淞,伊里布調援寧波。尋偕葛云飛等守定海。敵至,錫朋初守竹山門,為諸軍應援,數獲勝。及敵乘霧登曉峰嶺,以無巨砲不能御,率兵奮擊,並分援竹山,所部裨弁硃匯源、呂林環、劉桂五、夏敏忠、張魁甲先後陣歿,眾且盡,錫朋手刃數人,遂遇害。久之始得其尸,面如生,耳際有創。巡撫劉韻珂驗實,為改殮,恤典加等,予騎都尉兼一雲騎尉世職,謚剛節。子承泗、承瀚,並賜文舉人,承泗襲世職,官山西溫州知州;承瀚工部主事。

國鴻,字雪堂,湖南鳳凰人。父朝桂,貴州副將。伯父廷松,鎮筸千總,殉苗難,無子,以國鴻嗣,襲雲騎尉。從傅鼐剿苗,授永綏屯守備,洊擢寶慶副將。

道光二十年,擢處州鎮總兵,調防鎮海,充翼長。定海既還,移兵分守要隘。敵艦初犯竹山門,國鴻發巨砲斷其桅,遂以竹山為分汛地。戰連日,久雨,往來泥淖。及敵分三路同時來撲,國鴻奮擊,槍砲皆熱不可用,短兵拒戰,而土寇導敵奪曉峰嶺,險要盡失,國鴻單騎沖陣,被數十創而殞,依總兵賜恤,予騎都尉世職,追謚忠節。子鼎聲已歿,賜其孫鍔、銛並為舉人,鍔襲騎都尉,七品小京官;銛襲雲騎尉。出繼之子鼎臣,批驗大使,從軍中,揚威將軍奕經令募水勇攻敵海山港,賜花翎、四品頂戴。三鎮死事最烈,並入昭忠祠。定海收復,建立專祠,合祀云飛、錫朋,並許原籍各建專祠。

當定海之初陷也,總兵張朝發戰於港口,兵敗,身受砲傷,知縣姚懷祥、典史全福皆死之。時咎朝發不專守陸路,巡撫烏爾恭額疏劾逮治。朝發已以傷殞,恤典不及焉。浙中戰事以定海為最力。後揚威將軍奕經督師,將帥多闒茸,戰事如兒戲,惟金華協副將硃貴稱忠勇。

貴,字黻堂,甘肅河州人。以武生入伍,從征川、陜教匪,剿藍號賊於盧家灣。賊渠冉學勝伏密箐中,以長矛刺傷主將,貴奪其矛而擒之,勇冠軍中。滑縣、三才峽諸役,皆在事有功,累擢涼州守備。道光初,從楊遇春戰回疆,擢游擊,歷陜西西安參將、署察漢托洛亥副將。二十一年,擢浙江金華協副將。揚威將軍奕經督師,貴率陜甘兵九百以從。時兵多新募,惟貴所部最號勁旅。

二十二年春,奕經規復寧波、鎮海,令貴當鎮海一路,行未至,寧波已失利,止勿進,調赴長谿嶺大營,遂屯慈谿城西大寶山。敵乘勝以二千人自大西壩登岸,貴率所部迎擊,斃敵四百餘人。再卻再進,自辰至申,軍中不得食,猶酣戰。鄉勇忽亂隊,敵由山後鈔襲,增者幾倍。又三艦自丈亭江直逼山下,長谿大營驚潰。貴腹背被攻,怒馬斫陣,中槍馬倒,躍起奪敵矛奮斗,傷要害,乃踣。子武生昭南,以身障父,同時陣亡。部下游擊黃泰,守備徐宦、陳芝蘭,浙江候補知縣顏履敬等,兵卒三百餘人,同死。詔嘉其忠勇,依總兵例賜恤,予騎都尉世職,子廷瑞襲。昭南予雲騎尉世職,子輶甫四歲,命及歲襲職。

阿木穰,世襲土司,大金河千總,加副將銜、巴圖魯勇號。哈克里,瓦寺土守備,率金川屯練赴軍,皆趫捷奮勇,戰輒爭先。冠虎形,奕經占有虎頭之兆,令赴前敵,從提督段永福攻寧波。敵已為備,至則城門不閉。阿木穰率土司兵先入,中地雷同歿。哈克里攻奪招寶山,猱升而上,搶入威遠城。敵艦自金雞山翦江至,用砲仰擊,遂不支而退,後亦殉難,浙人哀之。自硃貴大寶山之戰,敵受創甚鉅,遂戒深入,慈谿縣城獲完。士民思其功,為建祠報賽,阿木穰、哈克里亦附祀焉。

論曰:海疆戰事起,既絀於兵械,又昧於敵情,又牽掣於和戰之無定,畏葸者敗,忠勇者亦敗。專閫之臣,忘身殉國,義不返踵,亦各求其心之所安耳。嗚呼,烈已!偏裨授命者,附著於篇。


\end{pinyinscope}