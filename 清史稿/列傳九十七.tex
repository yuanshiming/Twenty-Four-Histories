\article{列傳九十七}

\begin{pinyinscope}
齊蘇勒嵇曾筠子璜高斌從子高晉完顏偉顧琮白鍾山

齊蘇勒,字篤之,納喇氏,滿洲正白旗人。自官學選天文生為欽天監博士,遷靈臺郎。擢內務府主事,授永定河分司。康熙四十二年,聖祖南巡閱河,齊蘇勒扈蹕。至淮安,上諭黃河險要處應下挑水埽壩,命往煙墩、九里岡、龍窩修築。齊蘇勒於回鑾前畢工,上嘉之。洊擢翰林院侍講、國子監祭酒,仍領永定河分司事。河決武陟,奉命同副都御史牛鈕監修堤工。疏言:「自沁河堤頭至滎澤大堤十八里,擇平衍處築遙堤。使河水趨一道,專力刷深,不致旁溢。」六十一年,世宗即位,擢山東按察使,兼理運河事。命先往河南籌辦黃河堤工。時河南巡撫楊宗義請於馬營口南舊有河形處濬引河。齊蘇勒同河道總督陳鵬年疏言:「河不兩行,此洩則彼淤。馬營口堤甫成,若開引河,慮旁洩侵堤。」事乃寢。

雍正元年,授河道總督。既上官,疏言:「治河之道,若瀕危而後圖之,則一丈之險頓成百丈,千金之費糜至萬金。惟先時豫防,庶力省而功易就。」又言:「各堤壩歲久多傾圮,弊在河員廢弛,冒銷帑金。宜嚴立定章示懲勸。」並允行。乃周歷黃河、運河,凡堤形高卑闊狹,水勢淺深緩急,皆計里測量。總河私費,舊取給屬官,歲一萬三千餘金,及年節餽遺,行部供張,齊蘇勒裁革殆盡。舉劾必當其能否,人皆懍懍奉法。

陽武、祥符、商丘三縣界黃河,北岸有支流三,逼堤繞行五十餘里;南岸青佛寺有支流一,逼堤繞行四十餘里。齊蘇勒慮刷損大堤,令築壩堵御,並接築子堤九千二百八十八丈,隔堤七百八十丈。又以洪澤湖水弱,慮黃水倒灌,奏築清口兩岸大壩,中留水門,束高清水以抵黃流。及淮水暢下,壩在波濤中,又慮壩為水蝕,遣員弁駐工,湖漲下埽防壩,黃漲則用混江龍、鐵篦子諸器,駕小舟往來疏濬,不使沙停,水患始緩。詔豫籌山東諸湖蓄洩以利漕運,疏言:「兗州、濟寧境內,如南旺、馬蹋、蜀山、安山、馬場、昭陽、獨山、微山、稀山等湖,皆運道資以蓄洩,昔人謂之『水櫃』。民乘涸占種,湖身漸狹。宜乘水落,除已墾熟田,丈量立界,禁侵越。謹渟蓄:當運河盛漲,引水使與湖平,即築堰截堵;如遇水淺,則引之從高下注諸湖。或宜堤,或宜樹,或宜建閘啟閉,令諸州縣量事程功,則湖水深廣,漕艘無阻矣。」

二年,廣西巡撫李紱入對,上諭及淮、揚運河淤墊年久,水高於城,危險可慮。紱請於運河西別濬新河,以其土築西堤;而以舊河身作東堤,東岸當不至潰決。上命與齊蘇勒商度,齊蘇勒奏言:「淮河上接洪澤,下通江口。西岸臨白馬、寶應、界首諸湖,水勢汪洋無際。若別挑新河,築西堤於湖水中,不惟糜費巨金,抑且大工難就。」上是其言。是秋颶風作,海潮騰踴丈餘。黃河入海之路,二水沖激,歷三晝夜,而濱海堤岸屹然。上嘉其修築堅固,賜孔雀翎,並予拜他喇布勒哈番世職。

三年,副總河嵇曾筠奏於祥符縣回回寨濬引河,事將竣,齊蘇勒奉命偕總督田文鏡察視。齊蘇勒奏言:「濬引河必上口正對頂沖,而下口有建瓴之勢,乃能吸大溜入新河,借其水力滌刷寬深。今所濬引河,與現在水向不甚相對。當移上三十餘丈,對沖迎溜。復於對岸建挑水壩,挑溜順行,以對引河之口。俟水漲時相機開放,庶河流東注,而南岸堤根可保無虞。」上命內閣學士何國宗等以儀器測量,命齊蘇勒會勘。齊蘇勒奏:「儀器測度地勢,於河工高下之宜甚有準則。今洪澤湖滾水壩舊立門檻太高,不便於洩水。請敕諸臣繞至湖口,用儀器測定,將門檻改低,庶宣防有賴。」又奏言:「治河物料用葦、柳,而柳尤適宜。今飭屬於空閒地種柳,沮洳地種葦。應請凡種柳八千株、葦二頃者,予紀錄一次,著為例。」均稱旨。尋又奏言:「供應節禮,並已裁革。河標四營舊有坐糧,歲千餘金,以之修造墩臺,制換衣甲、器械;鹽商陋規歲二千金,為出操驗兵賞功犒勞之用。每年往來勘估,伏秋兩汛,出駐工次,車馬舟楫,日用所需,拮據實甚。河庫道收額解錢糧,向有隨平餘銀五千餘,除道署日用工食,請恩準支銷。」上允之。四年,以堵築睢寧硃家口決口,加兵部尚書、太子太傅。五年,疏言:「黃河斗岸常患沖激,應改斜坡,俾水隨坡溜,坡上懸密柳抵之。既久溜入中泓,柳枝霑泥,並成沙灘,則易險為平。」從其請。是年,齊蘇勒有疾,上遣醫往視。尋入覲,命歲支養廉萬金。

六年,兩江總督範時繹、江蘇巡撫陳時夏濬吳淞江,上命齊蘇勒料理。築壩陳家渡,松江知府周中鋐、千總陸章乘舟督工下埽,潮回壩陷,溺焉。齊蘇勒往視察,下為土埂,中有停沙,因督令疏濬,壩工乃竟。復偕曾筠會勘河南雷家寺支河,是秋事畢。於是黃河自碭山至海口,運河自邳州至江口,縱橫綿亙三千餘里,兩岸堤防崇廣若一,河工益完整。

七年春,疾甚,上復遣醫往視。尋卒,賜銀三千兩為歸櫬資,進世職三等阿達哈哈番,賜祭葬,謚勤恪。上又以靳輔、齊蘇勒實能為國宣勞,有功民社,命尹繼善等擇地,令有司春秋致祭。

齊蘇勒久任河督,世宗深器之,嘗諭曰:「爾清勤不待言,而獨立不倚,從未聞夤緣結交,尤屬可嘉。」又曰:「隆科多、年羹堯作威福,攬權勢。隆科多於朕前謂爾操守難信,年羹堯前歲數詆爾不學無術,朕以此知爾獨立也。」又曰:「齊蘇勒歷練老成,清慎勤三字均屬無媿。」八年,京師賢良祠成,復命與靳輔同入祀。

嵇曾筠,字松友,江南長洲人。父永仁,諸生,從福建總督範承謨死事;母楊守節,撫曾筠成立:事分見忠義、列女傳中。

曾筠,康熙四十五年進士,選庶吉士,授編修。累遷侍講。雍正元年,直南書房,兼上書房。擢左僉都御史,署河南巡撫,即充鄉試考官。遷兵部侍郎。河決中牟劉家莊、十里店諸地。詔往督築,逾數月,工竟。二年春,奏言:「黃、沁並漲,漫溢銚期營、秦家廠、馬營口諸堤。循流審視,窮致患之由。見北岸長沙灘,逼水南趨,至倉頭口,繞廣武山根,逶迤屈曲而下。官莊峪又有山嘴外伸,河流由西南直注東北,秦家廠諸地頂沖受險。請於倉頭口對面橫灘開引河,俾水勢由西北而東南,毋令激射東北;並培釘船幫大壩,更於上下增築減水壩,秦家廠諸地險勢可減。」又與河督齊蘇勒會奏培兩岸堤,北起滎澤,至山東曹縣;南亦起滎澤,至江南碭山:都計十二萬三千餘丈。皆從之。

授河南副總河,駐武陟。疏言:「鄭州大堤石家橋迤東大溜南趨,應下埽簽椿,復於埽灣建磯嘴壩一。中牟拉牌寨黃流逼射,應下埽護岸,建磯嘴挑水壩二。穆家樓堤工坐沖,亦應下埽加鑲。陽武北岸祥符珠水、牛趙二處堤工,近因中牟迤下,新長淤灘,大溜北趨成沖,應順埽加鑲。」又言:「小丹河自辛句口至河內清化鎮水口二千餘里。昔人建閘開渠,定三日放水濟漕,一日塞口灌田。日久閘夫賣水阻運,請嚴飭。仍用官三民一之法,違治其罪。」又言祥符南岸回回寨對面淤灘直出河心,致河勢南趨逼省城。請於北岸舊河身濬引河,導水直行。上諮齊蘇勒用曾筠議。四年,奏衛河水盛,請於汲、湯陰、內黃、大名諸縣築草壩二十七。又請培鄭州薛家集諸處埽壩。

五年,命兼管山東黃河堤工。尋轉吏部侍郎,仍留副總河任。六年,疏言:「儀封北岸因水勢沖急,雷家寺上首灘崖刷成支河。請將舊堤加幫,接築土壩,跨斷支河,以防掣溜侵堤。青龍岡水勢縈紆,將上灣淘作深兜,與下灣相對。請乘勢開引河,導水東行。」尋擢兵部尚書,調吏部,仍管副總河事。奏請培蘭陽耿家寨北堤,下埽簽椿築壩。

七年,授河南山東河道總督,疏請開荊隆口引河。八年,署江南河道總督,疏言:「山水異漲,匯歸駱馬湖,溢運浮黃,河、湖合一。請於山盱周橋以南開壩洩水,並啟高、寶諸堰,分水入江海。高堰山盱石工察有椿腐石欹,順砌卑矮者,應築月壩,加高培實。其年久傾圮者,全行改築。興工之際,築壩攔水,留舊石工為障。俟新基築定,再除舊石,仍留舊底二層,以御風浪。」又奏:「禹王臺壩工為江南下游保障。沭水源長性猛,壩工受沖。請於現有竹絡壩二十七丈外,依頂沖形勢,建石工六百餘丈。接連岡阜,仍築土堤,並濬沭河口門,使循故道直趨入海。」十年,奏揚州芒稻河閘商工草率,改歸官轄,並增設閘官。十二月,加太子太保。十一年四月,授文華殿大學士,兼吏部尚書,仍總督江南河道,予一品封典。十二月,丁母憂,命在任守制。曾筠奏懇回籍終制,溫詔許之。以高斌暫署,仍諭曾筠本籍距淮安不遠,明歲工程,就近協同經理。十二年四月,同高斌奏增築海口辛家蕩堤閘。同副總河白鍾山奏修清江龍王閘,濬通鳳陽廠引河。十三年,諭曾筠葬母事畢赴工。高宗御極,命總理浙江海塘工程。

乾隆元年,兼浙江巡撫。尋命改為總督,兼管鹽政。曾筠條奏鹽政,請改商捕為官役,嚴緝私販,定緝私賞罰。地方有搶鹽奸徒,官吏用盜案例參處。又疏請於海寧築尖山壩,建魚鱗石塘七千四百餘丈。入覲,加太子太傅。二年,疏請築淳安淳河石磡。三年,疏請修樂清濱海堤;又疏請發省城義倉運溫、臺諸縣平糶:並從之。尋召入閣治事,以疾請回籍調治。上令其子璜歸省,又遣醫診視。卒,贈少保,賜祭葬,謚文敏,祀浙江賢良祠。又命視靳輔、齊蘇勒例,一體祠祀。

曾筠在官,視國事如家事。知人善任,恭慎廉明,治河尤著績。用引河殺險法,前後省庫帑甚鉅。第三子璜,亦由治河有功,官大學士,繼其武。

璜,字尚佐。幼讀禹貢,曰:「禹治水皆自下而上。蓋下游宣通,水自順流而下。」長老咸驚異。雍正七年,賜舉人。八年,成進士,選庶吉士,年裁二十。授編修,再遷諭德。乾隆元年,命直南書房。三年,丁父憂,服闋,擢庶子。兩歲四遷左僉都御史。九年,奏:「督撫閱兵,祗就趨走應對定將弁能否。請近省命大臣,邊省命將軍、副都統,簡閱行伍。」是歲令大學士訥親閱河南、山東、江南三省行伍,璜此奏發之也。

璜侍曾筠行河,習工事。奏河工疏築諸事:請浚毛城鋪壩下引河,並於順河集諸地開河引溜,修築黃河岸,留新黃河、韓家堂諸地舊口,洩盛漲,議行。授大理寺卿。累遷戶部侍郎。十八年十月,黃、淮並漲。璜疏請濬銅山以下、清口以上河身,並仿明劉天和制平底方船,用鐵耙疏沙,修補高堰石工、歸仁堤閘,酌復江南境內減水閘壩。尚書舒赫德等被命視河,奏請派熟諳工程大員董理堤防,因令璜偕工部侍郎德爾敏督修。璜奏:「高堰工程有磚石之殊,年分有新舊之異。今當修砌石工,堤外築攔水壩,並將舊有磚工盡改石工。石較磚重,椿木應培增。舊修石堤用石二進,石後用磚二進,磚與土不相融結,久經風浪,根空基圮,令於磚石後加築灰土三尺,以御沖刷。」又奏:「串場河為諸水總匯。請自石閘南更建閘二,並就舊河道疏濬,直達海口。」十九年,奏:「高堰、高澖、龍門、古溝四處深塘兜灣,請修復草壩。」皆從之。是年堤工竟,議敘,轉吏部。二十年,以母病,乞假歸。

二十二年春,上以璜母病愈,授南河副總河,並諭曰:「璜侍父曾筠久任河工,見聞所及,諳練非難。母雖年近八十,常、淮帶水,侭可輕舟迎養,固無異在家侍奉也。」四月,上南巡,臨視高堰、清口及徐州諸工。以伏汛將至,近河諸地歲頻歉,貧民甚多,諭疏築諸工同時並舉,以工代賑。因璜前奏請於昭關增滾壩、濬支河,南關舊壩改建滾水石壩,即命璜董其事。璜奏:「運河東堤減水入下河,經劉莊、伍祐、新興諸場,分注鬥龍、新洋二港歸海。但劉莊大團閘至新興石閘相距較遠,請於伍祐沿窪口、蔡家港各增建石閘,引水出新洋港。並疏射陽湖港口,使之徑直。濬串場河以西孔家溝、岡溝河、皮家河支流凡三。此皆下河歸海之路也。湖河諸水,歸海紆回,歸江逕直。多一分入江,即少一分入海。應挑河築壩,使湖河水勢相平,乃將各壩開放。則湖水既減,可為容納來水地。伏秋水盛,洩高郵湖引入運河,出車邏、南關二壩,則歸海水少,下河田廬可無慮矣。」上諭曰:「璜此奏分別緩急,因勢利導,會全局而熟籌之。改紆為直,移遠為近,濬淺為深,具有條理。即令尹繼善、白鍾山等會璜次第興舉。」十一月,高郵運河東堤新建石壩工成,奏請酌定水則,車邏、南關二壩過水至三尺五寸,開五里中閘;至五尺,開新建石壩。又奏:「車邏、南關壩脊高於高郵湖面二尺七寸。芒稻閘為湖水歸江第一尾閭,請常年啟放,俾江、湖脈絡貫通。」上深嘉之,從所請,並降旨命勒石閘畔。

二十三年正月,擢工部尚書。五月,上下江諸工皆竟。九月,調禮部。二十四年四月,請在籍終養。二十五年,詣京師祝上壽。歸至清江浦,奏言:「歸江之路,尚有應籌。請於金灣壩下開引河,並濬董家溝。又以廖家溝、石羊溝、董家溝三壩改低三尺,使與芒稻閘相準。」上命交尹繼善等勘議。二十九年,丁母憂。三十二年,服闋,署禮部尚書,旋實授。七月,授河東河道總督,奏:「楊橋大壩為河南第一要工,雖已堵閉,時輒滲漏。而北岸河灘順直,既不能挑引河分溜,大壩迤東又遍地飛沙,不能建越堤。請將壩身裹戧培厚,用資完固。」璜每巡河,不避艱險,身先屬吏。一夕聞虞城工險,馳往。天甫曉,雨雹交下,下埽岌岌欲崩,從者失色,勸璜姑退。璜立堤上叱曰:「埽去我與俱去!」雨雹息,堤卒無恙。

三十三年九月,召授工部尚書,罷直南書房。尋以在河督任未甄別佐雜,左遷左副都御史。三十六年,遷工部侍郎。三十八年,擢尚書,調兵部。四十年,復調工部。四十四年,調吏部,協辦大學士。初,璜議挽黃河北流仍歸山東故道,入對嘗及之。是歲河決青龍岡,大學士阿桂視工。上以璜議諮阿桂及河督李奉翰,僉謂地北高南低,水性就下;欲導河北注,揣時度勢,斷不能行。上復命廷臣集議,仍謂黃河南徙已久,不可輕議改道,寢其事。

四十七年,加太子太保,在上書房總師傅上行走。並以璜年老,諭冬令日出後入朝,賜玄狐端罩。五十年正月,與千叟宴,為漢大臣領班。五十一年,以老乞休,賜詩慰留。上幸避暑山莊,命留京辦事。五十五年四月,以璜成進士逾六十年,重與恩榮宴。璜年八十,與高宗同歲生,生日在六月,奏改萬壽節後。上嘉其知禮,代定八月十九日,賜詩及聯榜、上方珍玩寵之。五十六年,復賜肩輿入直。五十九年七月,卒,年八十有四,命皇八子奠醊,贈太子太師,賜祭葬,謚文恭。

子八,長承謙,進士,官至侍讀,先璜卒。族子承恩,舉人,累官至河東河道總督。

高斌,字右文,高佳氏,滿洲鑲黃旗人,初隸內務府。雍正元年,授內務府主事。再遷郎中,管蘇州織造。六年,授廣東布政使,調浙江、江蘇、河南諸省。九年,遷河東副總河。十年,調兩淮鹽政,兼署江寧織造。十一年,署江南河道總督。十二年,回鹽政任。復署河道總督,培範公堤六萬四千餘丈。十三年,回鹽政任。旋授江南河道總督。

乾隆元年,疏請河工搶修工段需用土方,令河兵挑運十之四,用民工十之六。又請葦蕩營採柴均歸廠運。又請各州縣河工外解各項悉歸河庫道。河南永城、江南蕭縣頻年被河患,上命高斌會兩江總督趙弘恩、河南巡撫富德籌疏通之策。高斌等奏:「黃河南岸碭山毛城鋪向有減水石壩一,蕭縣王家山有天然減水石閘一,睢寧縣峰山有減水閘四,建自康熙間,誠分黃導淮以水治水之善策。年久淤淺,水發為患。毛城鋪舊有洪溝、巴河二河,為減洩黃水故道。閘下地勢,東北偏高,水向南行,漫入祝家口。請俟水涸疏濬二河,並於二河上游開蔣溝河,築祝家口、潘家口二壩。漳水南流,使盡入蔣溝、洪溝、巴河分流下注,永城、碭山諸縣當無水患。王家山天然閘減水會入徐溪口,舊有引河,間有淤淺;峰山減水四閘,歷年既久,引河亦有淤淺:均應疏濬。」又奏:「淮揚運河自清口至瓜洲三百餘里,其源為分洪澤湖水入天妃閘,建瓴而下,經淮安、寶應、高郵、揚州以達於江,惟借東西漕堤為障。請於天妃、正越兩閘之下,相距百餘丈,各建草壩三。壩下建正石閘二,越河石閘二。又於所建二閘尾各建草壩三。重重關鎖,層層收蓄,則水平溜緩,可禦洪澤湖異漲,亦可減運河水勢。湖水三分入運,七分會黃。山盱尾閭天然南北二壩,非洪澤湖異漲不可輕開,使清水全力禦黃;而高、寶諸湖所受之水,循軌入口,不至泛溢下河。則高、寶、興、鹽諸縣民田可免洪湖洩水之患。」疏入,均議行。

御史夏之芳等疏言:「毛城鋪引河一開,則高堰危,淮、揚運道民生可慮。」命高斌會大學士嵇曾筠、副總河劉永澄等詳度。安徽布政使晏斯盛、廣東學政王安國復請濬海口,又命高斌與宏恩及江蘇巡撫邵基會勘。二年三月,高斌請入覲。趙弘恩內擢戶部尚書,亦詣京師。上命王大臣集議,並召之芳等皆與。高斌言:「毛城鋪減水壩康熙十七年靳輔所建,減水歸洪澤湖,助清刷黃。六十年來,河道民生,均受其益。現濬毛城鋪,乃因壩下舊河量加挑濬,使水有所歸,並非開壩。況減下之水,紆回曲折六百餘里,經徐、蕭、睢、宿、靈、虹諸州縣,有楊甿等五湖為之渟蓄。入湖時即已澄清,無挾沙入湖之患,亦無湖不能容之慮。」之芳等仍執所見,議未決,御史甄之璜奏:「毛城鋪開河,淮、揚百萬之眾,憂慮惶恐。」鍾衡條奏亦及之。上卒用高斌議,斥之璜、衡、之芳等。

高斌復請別開新運口,堵塞舊運口,以避黃河倒灌。三年正月,淮、揚運河工竟,有旨嘉獎。四年,上聞時論議高斌所改新運口離黃稍遠,而上游水勢遇黃河異漲,仍不見倒灌,命大學士鄂爾泰乘驛往勘。鄂爾泰仍主開新運口,如高斌議。八月,高斌入覲,命便道與直隸總督孫嘉淦、總河顧琮會勘直隸河道。六年,奏言:「黃河自宿遷下至清河,河流湍急,內逼運河,脣齒相依。請培運河南岸縷堤,作為黃河北岸遙堤。」又言:「江都瓜河地勢卑下,請量改口門,別濬越河,以減淮水入瓜河分數。」又言鎮江南岸埽工宜改磚工。均下部議行。

調直隸總督,兼管總河。奏言:「永定河惟在尾閭通暢,請於三角澱旁開引河,下接大清河老河頭,上接鄭家樓水口。挑去積土,即於北岸圈築坡墊,以防北軼。南岸亦量為接築,以遏南溜。下口河脣,隨時疏通。至上游應籌分洩,請於南岸雙營,北岸胡林店、小惠家莊各增建三合土滾壩一;並減堤高,使卑於壩。南岸郭家堤舊草壩應一律修築如式。」七年,淮、揚水災,上命高斌及侍郎周學健會總督德沛等治賑。事畢,還直隸,復奏言:「永定河上游為桑乾河,自山西大同至直隸西寧,兩岸可各開渠灌田。自西寧石閘村入山,經宣化黑龍灣、懷來和合堡、宛平沿河口,兩山夾峙,一線中趨。若於山口取巨石錯落堆疊,仿竹絡壩之意,為玲瓏水壩,以殺其洶湧,則下游河患可減。」疏上,均議行。十年三月,加太子太保。五月,授吏部尚書,仍管直隸水利、河道工程。十二月,命協辦大學士、軍機處行走。

十一年,御史楊開鼎劾南河河道總督白鍾山河決匿災不報,命高斌往江南會總督尹繼善按治,白鍾山坐奪官。疏言:「淮、黃二瀆,每年伏秋水漲,以老壩口水志為準則。乾隆七年最大,水志連底水一丈四尺七寸,當以此較量每年水勢。各處閘壩開閉,應以就近石工水漲尺寸為度。」運河水漲,又命高斌往勘。疏陳培六塘河謝家莊、龍溝口諸處堤堰,濬中墩河、項家沖東門河;又疏請豁免海州、沭陽、贛榆諸縣逋賦,及板浦、徐瀆、中正、莞瀆、臨洪、興莊諸場折價帶徵銀:並從之。高斌嘗謂黃水宜合不宜分,清水宜蓄不宜洩,惟規度湖河水勢,視其縮盈以定蓄洩,方不至泛溢阻礙為民害。諸所籌畫,皆可循守。十二年三月,授文淵閣大學士。四月,命往江南同河道總督周學健督理防汛。五月,直隸水利工竟。

十三年,命偕左都御史劉統勛如山東治賑。又命偕總督顧琮如浙江按巡撫常安婪賄狀,高斌等頗不欲窮治。上又遣大學士訥親往按,責高斌模棱,下吏議,奪官,命留任。閏七月,周學健得罪,命兼管江南河道總督。尋以籍學健家產徇私瞻顧,奪大學士,仍留河道總督。十六年三月,上南巡,命仍以大學士銜管河道總督事。閏五月,暫管兩江總督。八月,盱南陽武漫工未合龍,詔往相度修築,命未下,高斌奏請馳赴協辦。上獎其急公任事,得大臣體。十一月,工竟,命同侍郎汪由敦勘天津諸處河工。十七年,年七十,賜詩。

十八年,洪澤湖溢,邵伯運河二閘沖決,高郵、寶應諸縣被水,下部嚴議。學習河務布政使富勒赫奏劾南河虧帑,命署尚書策楞、尚書劉統勛往按。策楞等疏發外河同知陳克濟、海防同知王德宣虧帑狀;並及洪澤湖水溢,通判周冕未為備,水至不能御,不即奏劾狀。上責高斌徇縱,與協辦河務張師載並奪官,留工效力贖罪。九月,黃河決銅山張家路,南注靈、虹諸縣,歸洪澤湖,奪淮而下。上以秋汛已過,何至沖漫河堤,責高斌命往銅山勒限堵塞。策楞尋奏同知李敦、守備張賓侵帑誤工狀,上命斬燉、賓,縶高斌、張師載使視行刑,仍傳旨釋之。二十年三月,卒於工次。予內大臣銜,發內庫銀一千治喪。

二十二年,上南巡,諭曰:「原任大學士、內大臣高斌,任河道總督時頗著勞績。即如毛城鋪所以分洩黃流,高斌設立徐州水志,至七尺方開。後人不用其法,遂致黃弱沙淤,隱貽河患。其於黃河兩岸汕刷支河,每歲冬季必率汛填築。近年工員疏忽,因有孫家集奪溜之事。至三滾壩洩洪湖盛漲,高斌堅持堵閉,下游州縣屢獲豐收。功在民生,自不可沒。癸酉張家路及運河河閘之決,則其果於自信,抑且年邁自滿之失。在本朝河臣中,即不能如靳輔,較齊蘇勒、嵇曾筠有過無不及。可與靳輔、齊蘇勒、嵇曾筠同祀,使後之司河務者知所激勸。」二十三年,賜謚文定。禦制懷舊詩,列五督臣中。命祀賢良祠。

子高恆,高恆子高樸,皆坐事獲譴,自有傳。上復錄高斌孫高杞授內務府郎中。從子高晉。

高晉,字昭德。父述明,涼州總兵。高晉初授山東泗水知縣,累遷安徽布政使,兼江寧織造。乾隆二十年,擢安徽巡撫。二十二年,上南巡視河,命高晉協辦徐州黃河兩岸堤工。高晉奏言:「鳳、潁災區諸工並舉,米價日昂,動工程銀三萬兩購米,尚慮不敷。上念淮徐海道諸工,截漕二十萬石平糶。請分五萬濟上江各工。」從之。工竟,加太子少傅。

二十六年,遷江南河道總督。奏言:「高、寶、興、泰積年被水,上命封南關、車邏等壩,於金灣壩下濬引河,洩水歸江,使洪澤湖、運河之水不致漫壩東注。下河各縣支河汊港及田間積水,均匯入串場河,北至鹽城石、天妃等閘,出新洋港。又自興化白駒、青龍、八社、大團等閘出鬥龍港,分二道歸海。惟下河形如釜底,積澇驟難消涸。請浚興化迤南丁溪、小梅二閘引河使出王家港,興化迤北上岡、北草堰、陳家沖三閘引河,使匯射陽湖,增二道歸海,俾數州縣積水節節流通,沮洳漸成沃壤。」從之。二十七年,授內大臣,奏言:「運河歸江,邵伯以下舊設六閘。自鹽河分流下注,請將六閘金門量為展寬。又鹽河舊設中、南、北各二閘,應留北二閘以濟鹽、運。南、中二閘過水遲滯,應添建石壩,接長土堤,酌挑引河,俾高、寶湖水歸江益暢。」二十八年,加太子太傅。二十九年,奏言:「清口以上桃、宿等,專受黃水;清口東壩以下,淮、黃合流,至雲梯關迤東歸海。北岸五套、南岸陳家浦頂沖入溜,議培築舊堤。臣以雲梯關外近海,與其築堤束水,不若於舊堤上首作斜長子堰,使水匯正河入海。」上均是之。

三十年,遷兩江總督,仍統理南河事務。三十一年,按蘇州同知段成功縱僕擾民,高晉以成功方病,擬寬之,上責其袒庇。三十三年,署湖廣總督,兼攝荊州將軍事。三十四年,回任,兼署江蘇巡撫。上命採洋銅鑄錢,高晉請收小錢,並運云南銅供鑄,費省於洋銅,上用其議。三十六年,兼署漕運總督,授文華殿大學士,兼禮部尚書,仍任總督如故。尋命同侍郎裘曰修、總督楊廷璋籌勘永定河工。事竟,還江南。

四十年,河東河道總督姚立德奏請以蜀山湖收蓄伏秋汛水,工部以舊例蜀山湖於十月後收蓄汶河清水議駁,上命高晉會勘。尋奏:「蜀山湖周六十五里,在汶河南、運河東,為第一水櫃。向定蓄水限九尺七八寸,請改以一丈一尺為率,兼蓄伏秋汛水。」從之。四十一年,河督吳嗣爵奏黃河淤高,命高晉與總督薩載籌議。請浚清口以內引河停淤,使清水暢出,與黃河匯流東注,並力剔沙,則黃河不濬自深,海口不疏自治。」上諭曰:「此奏甚合機宜形勢,為治淮、黃一大關鍵。屆時妥為之。」是冬,入覲,上以高晉年七十,書榜以賜。

四十三年,命赴浙江會巡撫王亶望相度海塘,又命赴河南堵築儀封漫口。秋,河決時和驛,高晉請議處,命寬之。冬,時和驛工竟。儀封新修埽工蟄陷,部議奪官,仍命留任。十二月,卒,賜祭葬,謚文端。懷舊詩並列五督臣中。子書麟、廣興,自有傳。

完顏偉,完顏即其氏,滿洲鑲黃旗人。雍正間,自內務府筆帖式累遷戶部員外郎。命往江南學習河務。乾隆二年,授浙江海防道。調江南河務道,尋擢浙江按察使。方建尖山壩工,巡撫盧焯奏以偉督工,歲賚銀五百。六年,命為江南副總河,就擢河道總督。高郵南關、五里、車邏三壩,值河、湖盛漲,洩水輒浸下河州縣民田。上命閉洪澤湖天然壩及三壩,不使水入下河。知州沈光曾以上河濱湖灘地被水,議以濟運餘水由三壩減洩,並易芒稻河閘為壩,疏寶應、高郵、甘泉諸湖南注之路。偉劾其擾亂河工,光曾坐奪官。

初,上以黃河大溜逼清口,命循康熙舊跡,開陶莊引河,導使北注。大學士鄂爾泰與河道總督高斌合勘,甫定議,會暴汛積淤,工遂停。高斌亦去任,復命偉相度。偉議自清口迤西黃河南岸設木龍挑溜,使漸趨而北。七年,疏言:「淮源上游雨多水發,賈魯河盛漲,由渦達淮,匯於洪澤湖。三石滾壩減歸高、寶、邵伯等湖,而古溝、東壩漫刷過水又自白馬湖來會,水勢益大。臣督築子堰捍禦,並開高郵老土壩及南關等三壩,水勢始定。」上嘉之。

是歲黃河亦盛漲,石林口減水過多,沛縣及山東魚臺、滕、嶧諸縣皆被水。偉具疏請罪。御史吳煒劾偉用人不得當,偉疏辨,上不深責,調河東河道總督。九年,奏言:「山東歷年被水,由於上游散漫,下游梗阻。運河東接汶、泗、沂、濟諸水,洩入微山、蜀山、南旺、馬踏諸湖;北接漳、衛二水,洩入鹽河、徒駭、馬頰、鉤盤諸河。遇伏秋異漲,宣洩不及,應於運河內增閘壩以分其勢,疏下河以暢其流。其經由各州縣,凡溝渠淤狹者浚之,堤堰殘缺者修之。」報可。十年,以母老乞回京,有旨慰留。十三年,授左副都御史。旋卒。

顧琮,字用方,伊爾根覺羅氏,滿洲鑲黃旗人,尚書顧八代孫。父顧儼,歷官副都統。顧琮,以監生錄入算學館,修算法諸書,書成議敘。康熙六十一年,授吏部員外郎。雍正三年,授戶部郎中,遷御史。四年,巡視長蘆鹽政。八年,遷太僕寺卿。九年,授霸州營田使。十一年,協理直隸總河,遷太常寺卿,署直隸總督。尋授直隸河道總督。十二年,奏報:「永定河口深通,上流始得暢注入澱。近因淤,議濬引河,自然開刷,不勞民力,號為天賜引河。」上令報祀。疏請更定管河汛,增設員缺,下部議行。

乾隆元年,署江蘇巡撫。丁父憂回旗。二年,命協辦吏部尚書事。永定河決,命偕總督李衛督修。旋署河道總督。三年正月,改授硃藻,命協同辦理。奏畿輔西南諸水匯於東西兩澱,淤墊漫溢為患。請設垡船撈泥,以三角澱通判、清河同知司其事。藻罷去,復授河道總督。五年,濬青縣興濟、滄州捷地兩減河,疏陳善後諸事,請疏海口,築遙堤,多設涵洞。六年,請改定子牙河管河官制。尋以裁缺回京。是年,授漕運總督。七年,奏言:「清江以上,運河兩岸,向來只知束水濟運,未知借水灌田,坐聽萬頃源泉,未收涓滴之利。同此田畝,淮南、淮北,腴瘠相懸。或疑運河洩水,於濟運有妨。不知漕艘道經淮、徐,五月上旬即可過竣。稻田須水,正在夏秋間。若屆時始行宣導,是祗借閉蓄之水為灌溉之資,於漕運初無所妨。況清江左右所建涵洞,成效彰彰。推此仿行,萬無疑慮。請特遣大臣總理相度,會同督、撫、河臣詳酌興工。」議未及行。八年,以督運詣京師。入對,請行限田,上斥其擾民。

十年六月,疏請於馬莊集、曹家店各建石閘,束上游之水,並將駱馬湖入運處改在皁河以上車頭,建閘挑渠,引水濟運。十字河竹絡壩開放後,黃水湍激,橫截運河,糧艘提溜為難。當於竹谿壩下束黃壩迤東接堤堵截,別於蘇家閘南濬河越黃入運,從之。十一年,署江南河道總督。十二年,命偕大學士高斌按浙江巡撫常安貪婪狀。坐未窮治,奪官,命留任。尋調河東河道總督。十七年,疏言:「運河堤未設堡房。請視黃河例,每二里建堡房,都計四百餘座。」十九年,坐江南總河任內浮費工銀,奪官。旋卒。

顧琮內行嚴正,嘗入對,值旱多風,世宗以為憂。顧琮引洪範謂「蒙恆風若,慮臣或蔽君」,上為之動容。世宗崩,顧琮方喪偶,逾三年乃續娶。方苞以為合禮。

白鍾山,字毓秀,漢軍正藍旗人。雍正初,自戶部筆帖式遷江南山清里河同知。累擢江蘇布政使。奏:「狼山、蘇松二鎮駐地距蘇州俱遠,軍糈輓運維艱,請就所駐及附近州縣配給。崇明孤懸海外,地不產米,請由江、廣採運,撥萬石貯崇明倉,備平糶。海濱漲出沙洲,民人占居,當築土墩以避潮患。」從之。十二年,授南河副總河,旋擢河東河道總督。

乾隆元年,奏:「河標兵駐濟寧,無倉儲,每稱貸貴糴。請以生息銀二千七百有奇買穀四千石,設倉存貯,春借秋收。」又奏:「豫東河防,水落時,當堵塞支河。伏秋水漲,購料募夫,每慮不及。請發河南、山東司庫銀分存鄭州及武陟、封丘、曹、單諸縣,永遠貯備。」皆從之。四年,疏言:「漳水舊自直隸入海,康熙四十五年,引漳入衛濟運,故道漸淤,全歸衛河,勢難容受。嗣於德州哨馬營建滾水壩,開引河洩衛水,由鉤盤河達老黃河入海。然漳、衛二水隨時淤塞,虛糜帑金。漳水舊有正河、支河,應擇易浚者復其故道。於館陶建閘,衛水大,聽漳入海以防漲;衛水小,分漳入衛以濟運。」奏入,命大學士鄂爾泰詳議,議在丘縣東和爾寨村承漳河北折之勢,接開十餘里,至漳洞村入舊河;因於新河東流入衛處建閘,以時啟閉,上從之。時漕運總督補熙請造十丈大船,運河當以水深四尺為則。白鍾山謂:「徬河無源之水,雨至而後泉旺,泉旺而後河盈。上徬閉、下徬啟,則下徬倍深,上徬倍淺。各徬相距遠近不均,水近者深,則遠者必淺。以人役水,以水送舟,必不能均深四尺。」侍郎趙殿最又請於館陶、臨清各立衛河水則,白鍾山謂:「尺寸不足,將衛輝民田渠徬盡閉,致妨灌溉,事既難行,尺寸既足,將官渠官徬盡閉,來源頓息。下流已逝,運河之水亦立見消涸。二者均屬非計。」議並寢。

八年,調江南河道總督,疏言:「石林口堵築堅固,大溜直趨下流。黃村、韓家塘等處新築子堰,恐不足抵御,於對岸濬引河,導溜南注,並加厚子堰,派兵駐防。」又奏言:「葦蕩左右兩營,歲輸柴二百二十五萬束。積久生弊,輪運不齊。請禁兵民雜採,定採葦期限,濬運柴溝渠,編柴船幫號。」皆允行。

十一年,御史楊開鼎劾:「白鍾山出納慳吝,任情駁減,用損工偷,縱僕役婪索。陳家浦決七百餘丈,止稱二十餘丈。興築延緩,阜寧、鹽城二縣受其害。」命高斌會尹繼善按治,以開鼎從。尋覆奏駁減、婪索無實據,惟陳家浦漫口沖刷,貽害累民。上召白鍾山詣京師,奪官,效力河工。總河顧琮復論白鍾山措處失當,上命籍其貲逾十萬以償。

十五年,授永定河道。十八年,河決張家路,命從尚書舒赫德往勘。旋命以按察使銜協辦南河事。十九年,復授河東河道總督。二十年,署山東巡撫。請罷孔氏世襲曲阜知縣,上命改授世襲六品官。尋奏濟寧以南積水未消,請緩開汶河大壩,疏濬下游河道。上命白鍾山往勘南河,文武各官聽調遣。

二十二年,調江南河道總督,疏言:「自河決張家路,沙停河淤,下流不暢,南高北窪。迨孫家集復決,河底益高。黃河受病,率由水勢側注北岸,沖刷溝槽。惟有南北分籌,南宜疏,北宜築。築則支河不致奪溜,疏則稍分有餘之水勢,庶徐州得以少安。臣與河臣張師載商榷,以為南岸長灘較北岸更險,必於橫亙處濬引河,導溜歸中,岸堤益加高厚。北岸無堤,漫水如梁家馬路、徐家莊等處支河數十道,及黃家莊、郭家堂等處漫槽矮灘,宜築土壩。水平則收束以刷正河,水漲則平漫平消,不至沖槽奪溜。並於孫家集培堤增壩,以為重障。駱馬湖北受蒙陰山水,西受微山湖水,其尾閭在六塘河。上游湖堤在在殘缺,亟應修補捍防。」皆從之。

荊山橋工竟,議敘。奏言:「寧夏上游河水陡漲,急報下游防範。正陽關為淮水上下關鍵,應仿寧夏水報法,派員專司其事。」又奏:「上江諸水皆歸安河以達洪澤湖。安河間段淤淺,連年水患由此。宜多募漁船,伐蘆撈泥,俾尾閭一通,上游皆有去路。又歸仁堤下舊有涵洞,穿鮑家河以達安河,久經湮塞。擬開浚分林子河一支,則安河進水之地亦有所分,患可漸減。」報聞。二十三年,加太子少保。二十六年,卒,贈太子太保,賜祭葬,謚莊恪。

論曰:自靳輔治河、淮,繼其後者,疏濬修築,守成法惟謹。世宗朝,齊蘇勒最著,嵇曾筠、高斌皆仍世繼業,與靳輔同祠河上,有功德於民,克應祭法。完顏偉、顧琮、白鍾山隨事補苴,不負當官之責。高斌任事二十年,疏毛家鋪引河,排眾議行之,民蒙其利。奪淮之役,縛赴工次待決。雷霆不測之威,赫矣哉!


\end{pinyinscope}