\article{列傳二百八十四}

\begin{pinyinscope}
孝義一

硃用純吳蕃昌從弟謙牧沈磊周靖耿燿弟炳兄子於彞

耿輔李景濂汪灝弟晨日昂日升黃農曹亨黃嘉章

鄭明允劉宗洙弟恩廣恩廣子青藜何復漢許季覺

吳氏四孝子雷顯宗趙清榮漣薛文弟化禮

曹孝童丁履豫鍾保覺羅色爾岱翁杜佟良克什布

王麟瑞李盛山李悃奚緝營周士晉黃有則

王尚毅胡鍈李三張夢維樂太希董盛祖

徐守仁李鳳翔卯觀成葛大賓呂斅孚

王子明馮星明張元翰俞鴻慶姜瑢湯淵魏興

戴兆笨潘周岱張淮張廷標胡其愛方其明鄧成珠

張三愛楊夢益閻天倫夏士友白長久郭味兒聶宏

董阿虎張乞人席慕孔張長松崔長生榮孝子

無錫二孝子啞孝子

清興關外,俗純樸,愛親敬長,內愨而外嚴。既定鼎,禮教益備。定旌格,循明舊。親存,奉侍竭其力;親歿,善居喪,或廬於墓;親遠行,萬里行求,或生還,或以喪歸。友於兄弟,同居三五世以上,號義門,及諸義行,皆禮旌。親病,刲股刳肝;親喪,以身殉:皆以傷生有禁,有司以事聞,輒破格報可。所以教民者,若是其周其密也。國史承前例,撰次孝友傳,亦頗及諸義行。合之方志甄錄、文家傳述,無慮千百人。採其尤者,用沈約宋書例,為孝義傳。事親存沒能盡禮;或遘家庭之變,能不失其正;或遇寇難、值水火,能全其親。若殉親而死,或為親復仇,友於兄弟,同居三五世以上,及凡有義行者,各以類聚。事同,以時次。孝為二卷,友與義合一卷。

硃用純,字致一,江南昆山人。父集璜,明季以諸生死難。用純慕王裒攀柏之義,自號曰柏廬。棄諸生,奉母。其學確守程、硃,知行並進,而程於至敬。來學者授以小學、近思錄。仿白鹿洞規,設講約,從者皆興起。居喪哀毀,嘗曰:「宰我欲短喪,吾黨皆以為怪,然可見古人喪禮之盡,必蔬水饘粥哭泣哀毀無茍弛。若今人飲酒食肉不改其常,雖更三年,豈謂久哉?」晚作輟講語,又為治家格言,語平易而切至。病將革,設先人位,拜於堂,告無罪,顧弟子曰:「學問在性命,事業在忠孝。」乃卒。用純與徐枋、楊無咎稱「吳中三高士」,皆明季死事之孤也。

吳蕃昌,字仲木,浙江海鹽人。父麟徵,明季死難,蕃昌事所後母查孝,居喪,水漿不入口。既殯,啜粥,不茹蔬果。寢苫,不脫衰絰。比葬,嘔血數升,逾小祥遂卒。

從弟謙牧,字裒仲。為程、硃之學。事母硃孝,居喪,杖不能起。疾稍間,手編父遺集,復困。治窀穸,哀動行路。謙牧體素羸,益不自勝,遂卒。蕃昌、謙牧皆交於張履祥,履祥稱之。

時以孝著者,復有歸安沈磊,亦履祥友也。磊事母嚴,母不御酒肉,磊力請,終不聽。有疾,醫為言,乃御酒肉。磊客授於外,弟子具時食,不忍食,以為母未嘗也。弟子乃先以饋母,曰:「太君食矣。」乃食,率以為常。

周靖,江南吳縣人。父茂蘭,刺血上書明父順昌冤,事具明史。靖少補諸生,事親能盡力。茂蘭卒,擗踴哭泣,喪葬悉如禮。三年不脫衰絰,不飲酒食肉。小祥,有疾作,或謂在禮得飲酒食肉,靖不可。靖素善作篆,或請題榜,亦以喪辭。

耿燿,河南太康人。世農。父應科,好施與,七世同居,顏其堂曰「效藝」。兄光,明諸生,孝後母而教諸弟嚴,燿從之學,事必諮而後行。明末,流寇屠太康,燿與弟炳舁母避河北,貿布以養。母病,朝出暮歸,不解帶累月。母卒,挽車歸母喪。炳亦純謹,定興耿權與弟極以孝友聞,炳慕其為人,分田舍處之,孫奇逢為作三耿傳焉。方寇至,光前卒,未葬,子於彞號泣守其柩不去,寇執之,推隕城下,傷腰膂,幾死。寇退,歸掬土掩柩乃去。縣饑,知縣餽以粟,散贍貧乏。督僮蔬,任饑者刈以食。

同時有耿輔,虞城人。奉母避寇開封,寇決河灌城,倚浮木負母以渡。母卒,哀毀,緇衣粗食終其世。

李景濂,字亦周,浙江鄞縣人。幼喪母,父再娶於何而卒。何年少,媒氏欲奪之,景濂聞,伺於道,出椎擊之,歸告何。何相與慟哭,誓相依終身。何教景濂嚴,景濂事何甚謹。何嗜酪,景濂日入市求之,端捧急趨,如鳥張翼。市人怪而求其故,則皆嘆其孝,為讓道。何老病,景濂侍疾七年不怠。何卒,景濂亦六十,廬墓三年,作孺子泣。景濂明諸生,明亡,棄諸生去為醫。

汪灝,江南休寧人。晨、日昂、日升,其弟也。父病咯血,灝年十六,割股和藥進,良愈。後數年病足,晨割股煉為末,敷治亦愈。又數年復咯血,晨復割臂以療。更數年,疾大作,灝復割臂,勿瘳。晨病,日昂泣曰:「吾兄割臂愈父,吾不能割以愈吾兄乎?」眾尼之。懵且僕,匠治棺,日升持匠斧斷指,血淋漓,調藥以飲晨。有司表其門曰「一門四孝友」。

錢塘吳瑗及弟琦、璠、琰相友愛,年皆逾九十。江蘇華亭姜應龍,應龍子世璜,世璜子文樞,文樞子超萃,超萃子懷權,懷權子栻,六世皆以孝行旌,人尤以為難。

黃農,江南元和人。父袞,諸生。農年十餘,母吳病六年,農侍疾不懈。母卒,慟屢絕,坐臥母柩側。袞客授於外,攜農俱。久之,察其枕漬淚若膏,貌臒然如初喪。袞客授稍遠家,農歸,五日一往省,袞止之,則私伺門外問安否,衣服器用,時其寒暑具以往。一夕,心悸,走省,袞得暴疾,舁以歸。會除夕禱神,原減算益父,袞愈。農三十餘而卒,妻金,亦賢孝。

曹亨,陜西鎮安人。年十一喪母,不能具棺,號泣於路,乞自鬻為斂。或與之金,葬母畢,即詣其家執役終身。

黃嘉章,湖南桂陽人。吳三桂之亂,從父避兵連珠崖。父歿,兄嘉林年十六,嘉章亦年十一,自鬻以葬父。嘉林稍長,力為傭,得錢贖嘉章還,兄弟相友愛。

鄭明允,江南歙縣人。康熙間,耿精忠兵至,明允侍母抱譜牒及先世遺筆入山。賊大索山中,明允夜負母匿僻塢,還挈二子,未至,霧溢山,虎聲震林木,納二子石穴中,疾趨侍母。賊退,二子亦無恙。兄病,視湯藥不去側。及亡,每慟輒絕。與其戚同賈,失其貲,明允發橐金盡與之。族子縊客舍,明允為坐守達曙,白於官,出私財以斂。有友蕩其貲,困甚,明允罄所有佽之,無難色。明允世業醫,精而不試,曰:「十得九,猶有一誤。」業賈終其身。

劉宗洙,字長源;弟恩廣,字錫三:湖北襄城人。父漢臣,明季從軍。襄城破,被數創,幾殆。恩廣兩耳斷,號泣負以歸。宗洙方走避寇,聞父難,往赴,賊截其耳鼻。居數年,父病,嘗糞,時稱襄城「嘗糞孝子」。父歾,與季弟宗泗同居,俄與恩廣皆得官,以母老不出。母歾,恩廣嘔血至篤疾。或慰解,曰:「勿復言,五內裂矣!」遂卒。宗洙積哀兼痛弟,亦嘔血卒。

恩廣子青藜,康熙四十五年進士,選庶吉士。遭父喪,哀毀嘔血,事母不復出。

何復漢,江西廣昌人。十五而喪父,哭淚皆血。長事母孝,母疾作,嘗糞苦甘以測病深淺,不解帶者數月。母歾,寢苫三月,淚漬苫左右盡血痕。葬,乃廬墓側,日夜悲號,喪終猶廬居。耿精忠兵至,復漢守墓不去,親知毀其廬,乃哭而行。著古今粹言示子孫。子人龍,康熙五十二年進士,入翰林。

許季覺,浙江海寧人。少尚俠,既折節讀書。居親喪,水漿不入口者七日,杖而後起。含殮、殯葬、虞祔、卒哭、祥禫皆用古禮。葬,躬負土,廬於側,朝夕哭不輟。季覺故與同縣查氏交密,查氏貴,營葬侵許氏墓地。季覺曰:「吾不能以友賣親。」訟連年不決,親朋居間,季覺終不讓。查氏誣季覺通海,逮獄,有為辨者,獄稍解,避地山陰。查氏復誣以他事,再逮獄。季覺度不免,獄中碎瓷盎吞之,死。

吳氏四孝子,江南崇明人,失其名。父壯年家貧,鬻子為富家奴。及長,皆能自贖。娶婦列肆居,養父母,兄弟議奉父母膳,月而易。諸婦曰:「翁姑老矣!月而易,必三月後方為翁姑具膳,太疏。」復議日而易,諸婦又曰:「翁姑老矣!日而易,必三日後方為翁姑具膳,仍太疏。」乃議伯具早餐,仲午,叔脯,次日季具早餐,周而復始。越五日,諸子合具饌奉父母,子孫皆侍,諸婦以次上酒食,以為常。室置★,兄弟各具錢五十,父食畢,取錢入市嬉,易果餌,歸畀諸孫,錢將盡,復具。父或從博徒戲,兄弟潛以錢畀博徒,令陽負與其父以為歡。行之數十年,父母皆將百歲,奉事不衰。陸隴其為之傳。

雷顯宗,河南陳州人。諸生。父病瘓,顯宗摩掌熱拊父四支,二十七晝夜不倦,父良愈。居數年,復病劇,侍湯藥兩月餘,竟卒,哀毀柴立。居母喪亦如之。康熙中,歲饑,出米粟濟貧乏,代償其逋賦。有鬻其孥者,贖以歸。佽婚葬者三百餘家。顯宗年九十,朔望集家人講孝經、曲禮、內則諸篇,里閈稱其家範。

趙清,山東諸城人。生有至性,嗜酒,與同縣李澄中、劉翼明輩遍陟縣中山,縱飲,輒沉頓。喪父,廬墓側百日,母往攜以歸。喪母,復廬墓側,麻衣躬畚鍤,負土為墳,毀幾殆。客有勸者,清曰:「清所以為此者,蓋下愚居喪法耳。清狂蕩如湍水,不居墓側,將食旨,久而甘;聞樂,久而樂;居處,且久而安。不一期,沉湎不可問矣。不孝孰甚!」居廬久,或傳有狼與犬為守廬,狎不相齧也。

榮漣,江南無錫人。少孤,多病,母令為道士。善詩畫。事母孝,出游得珍玩、良藥必以奉母。游倦歸,晨昏侍母側。母卒,廬墓不復出。漣與縣人杜詔及僧妙復號「三逸」。

薛文,江南和州人。弟化禮。貧,有母,兄弟一出為傭,一留侍母,迭相代。留者在母側絮絮與母語,不使孤坐。日旰,傭者還,挾酒米魚肉治食奉母,兄弟舞躍歌謳以侑。寒,負母曝戶外,兄弟前後為侏儒作態博母笑。母篤老,病且死,治殯葬畢,毀不能出戶。傭主跡至家,文與化禮骨立不能起,哭益哀,數日皆死,時康熙四十二年也。知州何偉表其閭。偉勤於民,卒,民祠焉。乾隆間,學政硃筠令以文、化禮附韋祠。

曹孝童,江南無錫人。居南郭,父為圬者。童五歲,父或扃戶出,則竟日不食。鄰或哺之,泣不食,俟父歸同食。父死,童嗚咽匍匐死父側,鄰市棺為斂。

丁履豫,江南婁縣人。少孤,事母孝。兄二、弟一皆出游,以歲所入畀履豫,使營甘旨。母卒將斂,畫師貌母像絕肖,履豫諦視久之,大慟,僕地遽絕。

鍾保,滿洲鑲黃旗人。父希晉,以步軍校從討吳三桂,積功當遷,鍾保以父老,力勸請休奉養。康熙間,自刑部筆帖式累遷刑部郎中,居父喪哀慟,水漿不入口。事母尤謹,歸必侍母側。兄蕩產,撫其孤,祖遺田宅悉推與之。弟貧,周之甚力。雍正二年,舉孝子,賜金,旌其門。官至工部侍郎。

覺羅色爾岱,滿洲鑲紅旗人,德世庫七世孫也。性篤孝。年十七,父病,醫不效,乃割左臂為糜以進,病稍間,旋歾。事母益謹,母病飲食減,亦減飲食;飲食不能進,憂之,亦輟飲食;母能飲食,乃復常。雍正元年,命舉忠孝節義,以色爾岱應,詔賜白金,旌其門,授銀庫主事,勤其官,遷郎中。

康熙間,以割臂療親旌者,有翁杜、佟良,與色爾岱同時有克什布。翁杜,滿洲鑲白旗人;佟良,蒙古鑲黃旗人:官防禦。克什布,滿洲鑲紅旗人,官三等侍衛。

王麟瑞,福建南靖人。諸生。八歲喪母,事後母如所生。母病暍,非時思食梅,麟瑞繞樹呼號,不食三日,梅夜華,結實奉母,母良愈。父喪,廬墓三年,遇虎,虎為卻避。雍正初,詔舉孝廉方正,縣以麟瑞上。四年,授陜西道監察御史,出為直隸永平知府。

李盛山,福建羅源人。母病,割肝以救,傷重,卒。巡撫常賚疏請旌,下禮部,禮部議輕生愚孝,無旌表之例。雍正六年三月壬子,世宗諭曰:「朕惟世祖、聖祖臨御萬方,立教明倫,與人為善。而於例慎予旌表者,誠天地好生之盛心,聖人覺世之至道,視人命為至重,不可以愚昧誤戕;念孝道為至弘,不可以毀傷為正。但有司未嘗以聖賢經常之道,與國家愛養之心,明白宣示,是以愚夫愚婦救親而捐軀,殉夫而殞命,往往有之。既有其事,若不予以旌表,無以彰其苦志。故數十年來雖未定例,仍許奏聞,且有邀恩於常格之外者。聖祖哀矜下民之盛心,如是其周詳而委曲也。父母愛子,無所不至,若因己病而致其子割肝刲股以充飲饌、和湯藥,縱其子無恙,父母未有不驚憂惻怛慘惕而不安者,況因此而傷生,豈父母所忍聞乎?父母有疾,固人子盡心竭力之時,儻能至誠純孝,必且感天地、動鬼神,不必以驚世駭俗之為,著奇於日用倫常之外。婦人從一之義,醮而不改,乃天下之正道,然烈婦難,節婦尤難。夫亡之後,婦職之當盡者更多,上有翁姑,則當代為奉養。他如修治蘋蘩,經理家業,其事難以悉數,安得以一死畢其責乎?朕今特頒訓諭,有司廣為宣示,俾知孝子節婦,自有常經,倫常之地,皆合中庸,以毋負國家教養矜全之德。倘訓諭之後,仍有不愛軀命,蹈於危亡者,朕亦不概加旌表,以成激烈輕生之習也。」盛山仍予旌表。

李悃,河南開封府人,失其縣。貧為木工,父病痺,奉侍惟謹。歲歉,不能養,乃行乞於市,歸啖父。後得賑穀一石,慮不能繼,日舂升許供父,而以秕自咽。父病劇,夜中鄰人時聞悃撫摩嗟泣聲,遲明則悃抱父足死矣,父亦一慟而絕。鄰人愍其孝,收而葬之。

奚緝營,字聖輝,江蘇寶山人。父士本,以孝旌。緝營幼讀論語,至「父母之年,不可不知」,輒隕涕簌簌,師奇之,謂真孝子子也。母病,刲臂以療。士本老,惡寒,緝營夜抱父足眠,以為常。兩弟早卒,撫其孤如所生。女兄嫁而貧,從妹寡,皆依以居,為營婚嫁。

周士晉,江蘇嘉定人。母病久,醫言惟飲人乳可生,士晉子生方九月,謀於妻李,棄道旁,以乳乳母。母病已,問兒,以殤對,後李不復姙,亦無怨。越十二年,有僧為殷氏子推命,年月日與士晉兒同,詰之,則得諸道旁者也,父子得復合。

黃有則,湖南邵陽人。四歲喪父,母孫劬苦育以長。遣就傅,或迂之,孫曰:「吾忍死,不欲兒廢學也。」有則大感慟,奮學,客授養母。夏無帳,主人以進,命撤之,曰:「吾母無此也。」寒為制棉衣,又卻之,曰:「家貧,無以暖母,不忍享奇溫。」一夕風雪,既寐,復起,行三十里歸省母。母喜曰:「吾正思兒。」是時母逾九十,有則亦六十矣。母喪,以毀卒。

王尚毅,陜西郃陽人。為人傭。母佞佛,欲鑿山造佛像,力不逮,將死,以命尚毅。尚毅傭,嗇衣食積錢,買山闢洞,琢石為佛像,洞六,像十二,皆手造。或愍而助之,謝曰:「力不己出,非敬母命也。」錢盡乃輟,復出傭,得錢更為之,如是三十餘年。山植柏,圍以紫荊,洞上下蒔迎春,洞成方冬,花盡開,山人怪之,名曰九華洞。山無水,鑿池而雨至,遂不涸,名曰青龍池。

胡鍈,浙江上虞人。鍈九歲從母汲,母墮井,鍈呼救未至,亦躍入井,救至,引以出,俱不死。中歲游陜西,一夕忽心痛,曰:「殆吾父病耶?」馳還,父正病,旋卒,哀慟盡禮。方冬母病,求醫,途遇盜,衣盡褫,冒寒行數十里,與醫俱歸。

李三,江蘇宜興人。一目眇,一足跛。父死,二兄皆娶,析產,有田六畝、屋四椽、舟一,二兄分田、屋,而畀三以舟。迭養母,三奉母食必有肉,母至二兄所,三輒私致甘旨。二兄死,嫂一前死、一嫁,三獨奉母。晨爨畢,乃以舟應客,或當出五十里外,度盡日不能返,雖重雇不之許。事母三十年,鄰里稱其孝,撫兄子慈,而教之嚴。母將死,呼孫執手泣曰:「兒學好,毋累汝叔怒!」自是不復怒其兄子。

張夢維,直隸元城人。縣諸生。父晚病風痺,夢維日侍左右,臥起飲食溲溺皆躬自扶持。父愍其勞,呵之去,少退,復前,數年不少懈。事母如事父。居喪哀毀,準家禮,屏俗習。弟病疽,為剪發灼艾,日數省視,及卒,慟甚,幾喪明。弟妻或詬誶,待之有加,撫孤女逾己出,弟妻卒悟且悔。少師郡人衛鶴鳴,治程、硃之學。鶴鳴卒,心喪三年。授弟子孝經、小學,以力行為本。

樂太希,湖北通山縣人。幼慧,三歲母負以嬉,墮地傷額。祖母問,詭對,恐祖母見憐而怒母也。父疾,抑搔澣濯,晝夜不去側。居喪盡哀,既葬,恆繞墓悲痛。母疾及喪亦如之,廬墓側居五年。早為諸生,以事親不應試,或延使授經,輒辭,慮違親也。親既終,益篤學。

董盛祖,雲南黑鹽井人。盛祖不知書,早失父,事母謹,起居飲食侍視不少懈。一妹嫁里中,盛祖出負販,呼妹還侍母,妹亦善事母如盛祖。盛祖行遇蛇當道,驚曰:「母得無病乎?」歸則母方病,呼祖,人皆怪之。母喪,哭甚哀,或慟絕,鄰里驚救之,乃甦。盛祖有妻早亡,不更娶。或勸之,曰:「娶婦以事親,顧賢者實難。脫不賢,將戾吾母,吾能安乎?」卒不娶。未終喪,遂卒。

徐守仁,安徽青陽人。世為農,未嘗讀書。四歲而孤,事母孝。得傭直,市酒肉奉母,母呼共食,輒以持齋謝,實不忍分甘也。母歿,哀慟。既葬,露處墓側,蛇虺不避,里人哀之,為廬舍飲食焉。守仁並奉其父木主以居,四年,乃還其室,須發皆白。

李鳳翔,直隸武強人。善事父母。鳳翔以父老,自請佐家事,而督諸弟讀書、習射,應文、武試。父將終,遺命析產,心憐幼子而未有言。鳳翔察父意,益以所分三之一。父歾,事母益謹。道光初,滹沱連歲氾溢,閭里蕩析,負鳳翔債者二千餘緡,悉焚其券,復散錢濟貧者。又遇旱,所藝蔬果任饑者採食。族子早孤,他縣人以迎喪遇盜,皆厚周之。或將屠馬,鳳翔贖以歸,馬馴異常畜,鄉人感之,遂無屠馬者。

卯觀成,雲南恩安人。父漢而母夷。烏蒙亂,父死,母被掠,鬻為婢。亂定,觀成無所依,為昭通禁卒。父母嘗為聘婦,舅促觀成娶,娶而不與婚。三年,舅詰之,曰:「吾非不欲婚也,行將嫁吾未婚之妻,取所直歸吾母。與之婚,情不能割,義亦不可出也。」語且泣。有義之者,募得六十金,以半贖其母,半為營廬舍,成婚,仍為禁卒以養母。

葛大賓,字興森,湖南湘鄉人。諸生。四歲喪父,哀慟如成人。喪終,值忌日,出主祭,主僕,粉落「葛」字脫,露「周」姓,蓋木工飾周氏廢主為之。大賓痛哭引咎,告墓易主。事母鉅細必躬,疾嘗藥,生徒有餽則獻。嘗出客授,獨坐心動,亟還呼母,母出,屋後山遽頹,壓母坐處。母歿,飲不入口者五日。既葬,不脫衰,腰以下縷皆盡。喪終,祭必哀,兄弟既分居,財盡,大賓復與同居,通財無所私。歿則庀其喪,無子,為立後。

呂斅孚,湖南永定人。父孟卿,貧,以客授自給。母病將殆,思肉食,斅孚方七歲,貸諸屠,屠不可,泣而歸。聞母呻吟,益痛,內念股肉可啗母,取廚刀礪使利,割右股四寸許,授其女弟,方五歲,令就爐火炙以進。母疾良已,孟卿歸,察斅孚足微跛,得其狀,與母持以哭。斅孚曰:「毋然,兒固無所苦也。」鄉人皆嗟異稱孝童。長為諸生,學政溫忠翰疏聞,尋除華容訓導。孟卿亦嘗刲股愈父病,然斅孚割股時,初不知父有是事也。

王子明,甘肅通渭人。諸生。事母孝。出為客,蔬果新出,必遙獻乃食。嘗赴試,母聞桃香久不散,女曰:「此必吾兄所獻。」記其日,歸驗之,果然。

馮星明,甘肅秦安人。為營卒,戍龍山。食新韭,置諸案,叩首。同伍問之,曰:「以獻母。」咸以為迂。或歸候其母,母曰:「他日吾假寐,夢兒以韭食我,覺,猶有餘香。」叩其日,星明獻韭時也。

張元翰,直隸南皮人。光緒五年舉人,除獲鹿教諭,遷知縣。方謁京師,父嗣陶時為萬全教諭,卒官。元翰奔赴慟哭,幾不能勝。居喪三年,悉用古禮。喪終,以知縣待缺河南,奉母赴官,攝澠池、寧陵諸縣。方有事於考城,而母遽卒,元翰以父母卒皆不克視終事,大痛。將歸葬,自為文祭告,憑棺一慟而絕。

俞鴻慶,湖南善化人。光緒十八年進士,改庶吉士,授編修。事父母篤孝。官京師,歲必乞假歸省。二十七年,母歾,鴻慶方自西安還京師,聞喪奔還,哀慟若不欲生。父年已八十,衰病,鴻慶跬步不去側,婉容愉色,依慕如少時。冬夜必數起省視,或竟夕不眠。二十九年,父歾,鴻慶慟甚,以毀卒,距父勿方匝月。

姜瑢,雲南習峨人。父文柄,嘗遠游,瑢裹糧行求,得以歸。貧,析薪治圃以養。父嗜飲,日必具酒,家益貧,父為罷飲。命子跪而請,翌日偕樵於山,買酒歸,共勸酣飲,日以為常。父歾,輒提父嘗飲壺沽酒,哭於墓,人稱其圃為「孝子圃」。

湯淵,江蘇常熟人。八歲喪父。母茅紡織不稍休,淵見輒淚下。少長,為負販,勸母暫休,母曰:「休,不且餒死耶?」淵大慟。客至,母擎茗椀呼淵持以出,淵跪而受,自責貧不能具僕婢也。娶,生子而婦亡,或勸再娶,曰:「吾已有子,何忍分養母力以養婦?」竟以鰥終。母卒,哀號動行路。其後家稍裕,方冬,有被而無褥,曰:「吾母昔無此也。」將卒,命市棺視殯母之費。

魏興,直隸新城人。早喪父,興與弟繼宗皆入伍。繼宗戰死,興以母老,出伍為樵以養。歲饑,米貴,興以米奉母,而自食糟糠,恆不飽。興亦老,樵不足,毀屋,伐屋後樹以鬻。安康諸生張鵬翼聞其事,過興,見興侍母左右扶持如童子,因問其鄰魏叟:「與其母日何食?」鄰曰:「興敢包穀,母食麥。」鵬翼大嗟異,以其事白知府,月予以粟,興母子始得飽。

戴兆笨,安徽旌德人。少從父業縫紉,十三喪母,盡禮,事後母如母。父病噎,亦減飲食,百方療父,不得,則刲肱糜以進,終不愈。慟甚,廬墓側,朝夕稽顙。時歸省後母,呼妻出,戒以善侍養,不入其室。

潘周岱,安徽涇縣人。為竹工,與父同傭,必躬其勞而遺父易且逸者。父創足,負以往返。老廢,周岱獨應傭,得酒肉時蔬懷歸,燂以進。家食,必父母食乃食。歲饑,奉父母必豐,次以食弟,躬與妻子飽糠覈。父母疾,左右侍養無須臾去側。母家山下泉洌,母病篤,夜半思得泉以飲,周岱挈瓶往,行四十餘里,鄉晨以泉至。居喪,旦暮悲號,先後廬墓三年。喪既終,夕必詣墓爇香燃燈,如是終其身。妻吳亦孝,無違命。

張淮,浙江秀水人。貧,粗識字,為人收田租。父有心疾,思食羊,非特殺則不食,淮買羊殺以食父。思出游,則賃肩輿侍以出,窮日乃還。父疾數年,凡所思,百方致之,不稍怠。疾篤,刲肱進,卒不治。

同時張廷標,為衣工。奉母,常效市中兒嬉戲以娛母。一日鄰家火,負母出,遷祀先之具,而不及他器用。節所入為弟娶婦,而終身不自娶。縣人與淮稱「二孝子」,道光初年事也。

胡其愛,江南桐城人。為人傭而養母。母病疲癃,其愛日夕在左右,視臥起飲食。出就傭,具晨餐,度午不能歸,出勺米付鄰媼,囑代爨,必拜。鄰媼止之,行數里外,復遙拜。夜必歸,為母滌中裙廁牏。在傭家得肉食,即請歸遺母。母出觀優,負以往,夜則負以還。欲往戚黨家,亦如之。母歿,負土為墳,居悒悒而卒。

方其明,亦桐城人。亦為傭而養母,母亦病疲癃。其明慮出傭母饑渴,乃棄傭為丐,負母以出,得食必先母。母卒,乃為圃,時荷鋤而泣曰:「昔為乞,苦饑寒,不離母側;今稍足衣食,思母不可得矣!」

鄧成珠,福建泰寧人。亦為傭而養母。傭所距家遠,日乞米一合,昧旦送母所,還執傭。母盲不能炊,乃負母依主家傍舍,朝夕為具食。主或以為言,成珠曰:「成珠自減餐奉母,不敢重累主人也。」居五年,母卒,葬畢,辭主人,不知所之。

張三愛,江南歙縣人。為人役。事母孝,母病,不能具藥物。或謂之曰:「汝欲愈母病,盍刲肝?」三愛禱於叢祠,破腹,肝墮出,以右手劙肝,得指許,左手納於腹,束以白麻。歸以肝和羹飲母,母良愈,三愛創亦合。三愛所事主,故嘗為知縣,貧,逋賦,三愛輒代承,被笞,不少懟。主病且死,命三愛去,三愛勿聽,事主之子如事主。

楊夢益,陜西郃陽人。賣菜傭也,事母孝,妻賈力紡織以佐養。乾隆中,歲饑,夢益與妻食糠籺,盛米於囊,置其中,熟以奉母。米盡,將鬻子,族人感而周之,乃止。

閻天倫,甘肅隴西人。貧,父居僧寺,天倫與妻楊,雞鳴起磨面,及明入市,求父所嗜往饋,午若晡皆然,夜則從父寢。父失明,天倫為茹素,年餘,目復明。天倫先父卒,楊賣漿為養,如天倫在時。翁卒,力營葬,忌日必祭,終其身。

夏士友,湖北江夏人。事母孝,傭力以養,不足,則減己食食母。鄰或邀食,必先為母具食,然後往。寒,語母勿早起,自執炊置食床前,又丁寧囑母善自護,乃出,如是以為常。年四十未娶,或愍之,助其娶婦。居半載,士友自外歸,婦與姑詬於室,流涕責婦,即日出之。或曰:「出婦,如無後何?」士友曰:「有婦,欲其孝;有子孫,亦欲其孝。茍不孝,安用婦?安用子孫?」年餘,士友疾卒,母哭之慟,鄰有張某感士友孝而不得終事母,月供薪米,終其身。

白長久,甘肅平番人。幼孤,貧,負販奉母,具甘旨。母或不怡,以首抵母,引手披其頸,俟解乃止。里社演劇,負母往觀,侍側說劇中事。母年八十,長久亦六十,未嘗稍懈。光緒中,青海辦事大臣豫師餽以金,不受。母卒,朝夕詣墓,饋食三年。

郭味兒,甘肅禮縣人。賣漿,出必拜母,歸亦然。母嚴,稍不當意即恚,味兒為孺子狀悅母。母苦脛痛,或言瘞枯骨,母當愈,黎明輒攜長鑱徘徊丘隴間,寒暑不間。母卒,飲不入口,五日毀卒。

聶宏,陜西鄠縣人。賣酒,事親孝,得錢易甘脆奉親。母卒,臥父榻側,時省視。畜犬,得餅銜飼母,人以為孝感。

董阿虎,江南山陽人。少喪父,為人擔水,得值養母。稍有餘,必具甘旨。積十餘年,構茅屋奉母。一日,鄰被火,阿虎負母避,還跪戶外,乞神佑。俄左右盡爇,獨阿虎茅屋存。

張乞人,順天永清人,失其名。父死,行乞以養母。穴土為居,天大雪,知縣魏繼齊過其處,聞歌聲出地中,怪而呼問之,曰:「今日母生日,歌以勸餐耳。」繼齊命車載其母子至縣,繼齊母畀其母粟及布,繼齊與銀十緡。乞人叩頭曰:「官母賜我母,不敢不受;官賜我,我不敢受。」繼齊問其故,曰:「民愚,不知此十緡官何所受之?我母年八十,我年六十一,為清白百姓足矣。」繼齊不復強,將為營室,乞人負其母去,不知所終。

席慕孔,廣東三水人。善養母。嘗娶妻生子。歲饑,田數畝盡鬻,妻怨其貧,求去,遂遣之。夏秋助人耕穫為傭,冬則乞食以養。得餅餌歸食母,得餘羹,啜湆,以肉歸。

張長松,山東棲霞人。母瞽,長松出為傭,主人與之食,輒不盡,歸遺母。無所事則乞諸鄰里,母食已,乃食其餘。冬大雪,長松病不能出,呼母涕泣言曰:「兒不肖,不能養吾母,乃乞食,母賴以活。今疾憊,母老,可若何?」遂死。

崔長生,江南邳州人。生而瘖,手又攣。為傭養父母,出入必面。歲大祲,乞食於市,得糟糠,上父母,自食草根木實以活。拾字紙,得遺金,待失者逾月不得。乃易母彘飼之,茁壯蕃息,為父母治送死之具。喪父母,舁葬於中野,遂去,不知所終。

榮孝子,河南遂平人。幼癡聾,無名。家本饒,後中落,貧甚。父卒,無所居,奉母居棲流鋪。出乞食,擇所得供母,自食其餘。得少,則但供母,而自忍饑歸。見母必叩頭,食必跪進。母食則起而舞,食減則泣。母或故減食以食子,則泣不受。母七十餘卒,縣人為具斂,朝暮泣,終其身。吏以孝子旌其楣,亦不知孝子為何名也。卒亦七十餘。

無錫二孝子,皆失其姓氏。其一瞽,磨粉為業,事母至孝,竭力供甘旨。年至四十餘復明,人皆異之。其一啞,行乞得錢以養母,必具酒脯。母卒,食必祭,祭必伏地號痛。既葬,哭於墓,見者皆感。

啞孝子,無姓氏,或曰雲南昆明人。家有母,老矣,行乞以養。得食必奉母,母食然後食。母或怒,嬉戲拜且舞,必母樂乃已。得錢密投諸井,母卒,鄉人有欲醵錢以助斂者,與如井,數數指水中,鄉人為出錢,營殮且葬。事畢,遠游不知所終。


\end{pinyinscope}