\article{列傳一百九十七}

\begin{pinyinscope}
彭玉麟楊岳斌王明山孫昌凱楊明海謝濬畬

彭玉麟,字雪琴,湖南衡陽人。父鳴九,官安徽合肥梁園巡檢。玉麟年十六,父卒,族人奪其田產,避居郡城,為協標書識以養母。知府高人鑒見其文,奇之,招入署讀書,為附生。新寧匪亂,從協標剿捕。敘功,大吏誤以為武生,拔補臨武營外委,不就。至耒陽,佐當商理事。粵匪至,罄所有資助縣令募勇籌防。賊知有備,不來攻,城獲全。玉麟不原敘功,但乞償所假錢,以是知名。

咸豐三年,曾國籓治水師,成十營,闢領一營。其九營多武員,白事悉倚玉麟,隱主全軍,草創規制多所贊畫。四年,初出師規岳州,不利,退長沙。玉麟偕楊岳斌援湘潭,會塔齊布陸師夾攻,賊舟連檣十里,分三隊合擊,同時縱火焚其輜重皆盡。賊棄城走,復湘潭,敘功以知縣選用。六月,再進岳州,賊據南津以拒。玉麟伏君山,岳斌伏雷公湖,遣小舟挑戰,賊舟爭出,兩翼鈔之,毀百餘艘,賊來,迭敗之。進攻擂鼓臺,賊舟多於官軍十倍。玉麟偕岳斌各乘舢板冒砲煙沖入,燒其坐船,賊還救,陣亂,大破之,玉麟傷指,血染襟袖,軍中推二人勇略為冠。既而總兵陳輝龍至,率新軍出戰,軍容甚盛,玉麟偕諸營從觀戰,拕罟膠淺,為賊所乘,急往救,水急風利,陷賊屯中,遂大敗。輝龍等戰歿,玉麟單舸退,自是水師專任彭、楊。

時陸軍累捷,賊退走,水師並進。八月,屯沌口,規武昌。玉麟與諸軍議,請渡江先破城外賊屯。賊自塘角至青山,緣岸列砲,丸發如雨。將士皆露立舢板,棹船徐進,無一俯側避砲者。賊望見奪氣,沿江賊屯盡潰,悉燒屯壘及其舟。武昌、漢陽同日皆復,論功擢同知。群賊麕聚田家鎮,夾江為五屯,依半壁山,連舟斷江,纜以鐵索,布竹木為大筏,施大砲。筏外護以舟,後列輜重,望之如大城。武昌既克,水師欲下攻,而為蘄州江岸賊所撓。玉麟掠江直下,十月,進逼田家鎮。與楊岳斌議分四隊,約陸師同時合擊。頭隊皆小船,具爐備椎斧,融炭以待。順流急趨,至筏下,斷鎖纜得隙,擠而過,後者從之。大呼曰:「鐵鎖開矣!」賊驚噪,爭走相踐墮水。玉麟率二隊順流而下,岳斌率三隊乘風而上,風起火烈,燒毀賊舟四千餘艘,奪獲五百餘艘。玉麟慮軍士互爭,盡焚之。捷入,以知府記名。詔採其戰法頒下江南北諸水軍。遂會諸軍進攻九江,連破賊於小池口、湖口。賊於九江夜襲水師大營,帥舟被燔,曾國籓移駐陸軍。玉麟部將蕭捷三追賊入鄱陽湖,賊斷湖口。玉麟往救不利,乃還新堤籌濟師。

五年,武、漢復陷,玉麟更募士造船,立新軍,合三千人,與楊岳斌分統之。胡林翼約同攻漢口,玉麟自金口進,敗賊占魚套;北岸陸軍為賊所挫,玉麟率眾登岸截擊,破之,攻塘角,焚賊船二百餘:授浙江金華知府。七月,自沌口進拔蔡店,及南北兩岸石城。五顯廟者,賊堅巢也。阻湖而屯,玉麟攻之不下,曰:「已入虎穴,非血戰不能成功。」張兩翼急槳而進,沖賊船尾,摧其卡,奪其船。復督隊徑越賊船,循兩岸包鈔。出襄河口,斷鐵鎖浮橋,毀北岸火藥庫,仍入襄河。乘夜撲漢陽,擒賊酋蕭朝富、吳會元。麾軍攻拔五顯廟,毀晴川閣木城,又破之葉家洲,燒賊船二百餘。初由沙口移軍沌口,過經賊壘,砲如雨下,所乘船桅折覆水。玉麟援橫枚漂江中流,楊岳斌舟掠過,掉舢板拯之還。胡林翼疏陳稱其忠勇冠軍,膽識沉毅,詔以道員記名。

時曾國籓在江西,水軍頻挫,迭召往助。玉麟乞假回長沙,急赴之。袁、瑞兩郡並陷賊,水陸道絕,易衣裝為賈客,徒步數百里達南昌。重整內湖水師為十營,船六百艘。六年,擢廣東惠潮嘉道。敗賊樟樹鎮,又連破之於臨江吳城、塗家埠,克南康。七年,國籓還籍治父喪,玉麟與楊岳斌同領其軍。其秋,武、漢再克,水陸並下,圍九江。玉麟約岳斌夾攻湖口,賊扼石鐘山、梅家洲,力遏內湖軍不得出。玉麟分軍為三以進,賊穴山腹置巨砲,直船沖,舢板先出,前鋒中砲,後船繼進,傷十餘艘。玉麟憤曰:「此險不破,萬不令將士獨死,亦不使怯者獨生!」鼓棹急赴,賊砲忽裂,船銜尾下,與外江水師合,歡聲雷動。陸軍由城背山下應之,賊大奔,乘勝奪小孤山,加按察使銜。八年,連破樅陽、大通、銅陵、峽口賊屯,合圍九江,克之,晉布政使銜。楊岳斌進軍黃石磯,自九江至武昌,置十二屯。

十年,玉麟移營與合屯。賊復上犯彭澤、湖口,分兵赴援,克都昌。十一年,授廣東按察使。賊犯蘄、黃、德安,玉麟會陸軍克孝感、天門、應城、黃州、德安,擢安徽巡撫。命幫辦袁甲三軍務,潁、壽各軍悉歸調遣,累疏固辭,謂:「久居戰艦,草衣短笠,日與水勇、舵工馳逐於巨風惡浪之中。一旦身膺疆寄,進退百僚,問錢穀不知,問刑名不知,勉強負荷,貽誤國家。」又謂:「從軍八年,專帶水師,棄舟而陸,無一旅一將供其指揮,倉猝召募,必致僨事。」詔嘉其不欺,以李續宜代之,改職水師提督。

同治元年,授兵部右侍郎,節制鎮將。軍中重文輕武,玉麟與楊岳斌威望久埒,一旦名位超越,而相處終始無間,論者謂其苦心協和不可及。別立太湖水師十營,並歸統轄。曾國荃由安慶進規江寧,水師助之。克銅城閘,復巢縣、含山、和州,襲破雍家鎮、裕溪口,奪東西梁山,進攻採石,又克金柱關。諸將沖鋒,玉麟每乘小船督戰,以紅旗為識,或前或後,將士皆惴惴盡力。間入陸軍察戰狀,往來飄忽無定蹤,所經行軍民莫敢為奸宄。

二年,與楊岳斌合兵攻九洑洲。賊於洲築壘數十,外作大城,眾舟環之,與江寧相犄角;而攔江磯、草鞋峽、七里洲、燕子磯、中關、下關皆賊屯。玉麟列舟上流,南隊向下關,北隊向草鞋峽,岳斌攻燕子磯,破之。陸軍亦分三隊,掘洲埂攻中關,舢板環洲而陣。賊以槍砲相持,不能進。玉麟督諸軍更番夜攻,下令曰:「洲不破,不收隊。」選死士從火叢登岸,噪曰:「洲破矣!」諸軍歡呼,騰踔而上,立破洲邊屯舟,賊爭潰走。自田家鎮以來,是戰為最烈。於是賊黨由江西犯池州,謀撓官軍。玉麟還救青陽,解其圍,復高淳,克東壩,並論九洑洲功,賜黃馬褂。會楊岳斌赴江西督師,自是玉麟專統水師。三年,江寧復,論功,以創立水師為首,加太子少保,予一等輕車都尉世職。四年,命署漕運總督,再疏辭,允之,命籌商水師善後事宜。

七年,會同曾國籓奏定長江水師營制,自荊州至崇明五千餘里,設提督一員、總兵五員,以六標分汛;營、哨官七百九十八員,兵丁一萬二千人,歲餉六十餘萬兩,以長江釐稅供支,不煩戶部。初,軍事未定,軍餉奇絀,而淮鹽積滯。玉麟議定捆鹽自賣,供水師月餉。及江路大通,曾國籓設三省督銷局,招商領票,水師鹽票大小數百,至是軍餉有額支的款。餘銀及票本巨萬,玉麟一不私取,以五之一取息,助水師公費,且備外患倉猝之需。餘分解雲、貴助餉二十萬,甘肅助餉二十萬,以十萬廣本縣學額,而以鹽票犒諸將有大功者。

事既竣,疏請回籍補行終制,略曰:「臣墨絰從戎,創立水師,治軍十餘年,未嘗營一瓦之覆,一畝之殖;受傷積勞,未嘗請一日之假;終年風濤矢石之中,未嘗移居岸上求一日之安。誠以親服未終,而出從戎旅,既難免不孝之罪,豈敢復為身家之圖乎?臣嘗聞士大夫出處進退,關系風俗之盛衰。臣之從戎,志在滅賊,賊己滅而不歸,近於貪位;長江既設提鎮,臣猶在軍,近於戀權;改易初心,貪戀權位,則前此辭官,疑是作偽;三年之制,賢愚所同,軍事已終,仍不補行終制,久留於外,涉於忘親。四者有一,皆足以傷風敗俗。夫天下之亂,不徒在盜賊之未平,而在士大夫之進無禮,退無義。伏惟皇上中興大業,正宜扶樹名教,整肅紀綱,以振起人心。況人之才力聰明,用久則竭,若不善藏其短,必致轉失所長。古來臣子,往往初年頗有建樹,而晚節末路隕越錯謬,固由才庸,亦其精氣竭也。臣每讀史至此,竊嘆其人不能善藏其短,又惜當日朝廷不知善全其長。知進而不知退,聖人於易深戒之,固有由矣。臣本無經濟之學,而性情褊躁,思慮憂傷。月積年累,怔忡眴暈,精力日衰,心氣日耗。若再不調理,必致貽誤國事。懇請天恩開臣兵部侍郎本缺,回籍補行終制。報國之日正長,斷不敢永圖安逸也。」優詔從之。

八年春,還衡陽,作草樓三重,布衣青鞋,時往母墓,廬居三年不出。自設長江水師,東南無事,將士漸耽安逸,事多廢弛。十一年,詔起玉麟簡閱,疏陳整頓事宜,諷提督黃翼升自退,薦李成謀、彭楚漢二人,即以成謀代之,劾罷營哨官百數十人。入覲,命署兵部侍郎,復陳請開缺,仍命巡閱長江,專摺奏事。別飭兩江、湖廣為籌經費,玉效力辭不受。自築別業於杭州西湖,曰退省庵。每巡閱下游,事畢,居之。自是水師皆整肅,沿江盜蹤斂戢,安堵者數十年。朝廷有大政,及疆吏重案,輒諮詢,命按治。

光緒七年,命署兩江總督,再疏力辭,乃以左宗棠代之。留督江、海防如故。言者議長江提督宜駐吳淞口外,玉麟疏言:「江南提督責在海防,請多畀兵輪,使立一軍於海上。長江提督責在江防,請仍由臣督同巡閱,改駐吳淞,會操兵輪,以通江、海。」九年,擢兵部尚書,以衰病辭。

會法、越構兵,命赴廣東會籌防務。玉麟募四千人從行,駐大黃。遣部將王之春、黃得勝等防瓊州、欽州、靈山,婁云慶、王永章等駐沙角、大角,與粵軍聯合。增兵設壘,編沙戶漁舟,分守內沙港汊。法兵竟不至。十一年春,粵軍大捷於鎮南關,進攻諒山。和議旋成,停戰撤兵。玉麟疏請嚴備戰守,以毖後患,陳海防善後六事。是秋,以病乞休,溫詔慰留。十四年,扶病巡閱。至安慶,巡撫陳彞見其病篤,以聞,詔允開缺回籍,仍留巡閱差使。十六年,卒,年七十五,贈太子太保,依尚書例賜恤,建專祠立功地,謚剛直。

玉麟剛介絕俗,素厭文法,治事輒得法外意。不通權貴,而坦易直亮,無傾軋倨傲之心。歷奉命按重臣疆吏被劾者,於左宗棠、劉坤一、塗宗瀛、張樹聲等,皆主持公道,務存大體,亦不為谿刻。每出巡,偵官吏不法輒劾懲,甚者以軍法斬之然後聞,故所至官吏皆危慄。民有枉,往往盼彭公來。朝廷傾心聽之,不居位而京察屢加褒獎,倚畀蓋過於疆吏。生平奏牘皆手裁,每出,為世傳誦。好畫梅,詩書皆超俗,文採風流亦不沫云。

楊岳斌,原名載福,字厚庵,湖南善化人,原籍乾州。祖勝德,乾隆末,從剿苗,戰歿永綏。父秀貴,以廕官至直隸獨石口副將。岳斌幼嫺騎射,補湘陰外委,從剿新寧匪。

咸豐二年,守湘陰有功,擢宜章營千總。三年,曾國籓創立水師,拔為營官。戰岳州,水陸皆潰,獨岳斌一營力拒不敗。四年,戰湘潭,焚賊舟數百,復其城,擢守備,賜花翎。國籓重整水師,進規岳州。岳斌與彭玉麟為前鋒,伏船雷公湖,誘賊舟至,夾擊,連戰皆捷;賊再至,沿東岸斜擊之,手挺矛刺殺賊酋在汪得勝,奪其舟,賊無還者:擢都司,賜號彪勇巴圖魯。進戰擂鼓臺,乘舢板沖賊屯縱火,賊陣亂,大破之,克岳州,擢游擊。總兵陳輝龍率後隊至,狃前勝,欲乘風攻城陵磯。岳斌曰:「順風難收隊,不可行也。」不從,遇賊伏,竟敗。輝龍及知府褚汝航、同知夏鑾、游擊沙鎮邦皆戰死,岳斌軍獨完。既而賊為陸師所敗,將遁,要擊之,平兩岸砲臺,搜螺山、倒口賊舟。尋夜襲嘉魚黃蓋湖,岳斌先入,被火傷,舟覆落水,躍上別船,大呼陷陣,焚賊舟數十。遂會湖北軍進屯金口,破漢陽關賊營,攻塘角,至青山,焚其壘,賊遁,焚其輜重。武昌、漢陽皆復,擢參將,授湖南常德營副將。諸軍進規田家鎮,岳斌由中路先發,克黃州及武昌縣,破援賊於蘄州,逼田家鎮,偕彭玉麟分隊毀橫江鐵鎖,焚賊船四千餘皆盡,漂尸數萬,遂拔田家鎮,蘄州賊亦遁去。岳斌晝夜進戰,積勞嘔血,詔嘉其勞勩最著,加總兵銜。

五年,水陸會攻九江,岳斌以疾留武穴,尋假歸。水師恃勝銳進,前隊舢板入鄱陽湖,賊樹柵湖口扼之,不得出,而留九江者,亦屢為賊所襲。岳斌聞敗,馳救不及。賊復上犯,武、漢再陷。曾國籓分水師回援,令岳斌回岳州,增募為十營,會屯金口,屢敗賊。秋,退屯新堤,修船,汰疲卒十之三,簡練以圖大舉。自武、漢為賊踞,長江商旅皆絕。及水師駐新堤,流亡歸之,市廛始興,漸為重鎮。授鄖陽鎮總兵,兼署湖北提督。六年,進屯沙口,距武昌三十里。岳斌念賊舟往來長江,停則依壘,行皆乘風,恆避戰,難得大創,乃謀襲燒之。募壯士駕千石大船,實硝黃蘆荻,施火線。約曰:「近賊而發,急登舢板退。」應募者三百人,懸重賞。夜逼賊舟,於南岸嘴縱火,於是賊舟能戰者多燼。前軍直至黃州,旬日間轉戰數百里,擊毀賊舟六百餘,奪其資糧火藥,哨船掠巴河、蘄州、耀兵九江城下而還。武、漢水路援絕,乃益困。十一月,與李續賓陸師合攻。值大風揚沙,波濤洶湧,水師上下環擊,賊大潰敗走。二城同日克復,捷聞,加提督銜。

進規九江,曾國籓以憂歸,薦岳斌接統其軍,彭玉麟副之。分兵扼蘄州,破援賊。秋,會陸軍克小池口,密與彭玉麟約期會攻湖口,克之。於是內湖外江水師始復合。乘勝奪小孤山,克彭澤,留軍屯之。自率前鋒至望江,賊望風遁,遂復東流。過安慶,攻樅陽、大通賊壘,克之。復銅陵,至蕪湖魯港,與江南師船會。詔嘉其轉戰千里,謀略過人,尋授福建陸路提督,許專摺奏事。八年四月,與李續賓會攻九江。岳斌當北門,臨江地雷發,奮呼齊登,擒賊首林啟榮,逸出之賊,盡為水師所殲,賜黃馬褂。

詔促東下,疏言楚境肅清後始能會師,遂移屯黃石磯。連攻安慶、樅陽、大通,奪其壘,分兵復建德,調福建水師提督。九月,會都興阿克集賢關,賊自池州來援,迎擊於樅陽,破之。時李續賓三河師潰,賊復謀上犯湖北。岳斌遣兵分扼龍坪、鄔穴、田家鎮。九年,督剿南北兩岸援賊,時出隊薄安慶城,以牽賊勢。十二月,賊酋韋志俊以池州降,令攻蕪湖。其部下有叛者,還陷池州。岳斌察志俊無異志,分別遣留,得精銳二千五百人,令率以助戰。陳玉成、李侍賢率眾分竄楚、皖,水師移屯觀音洲以備之。十年四月,大破賊於蟂磯,令韋志俊拔殷家匯,進攻池州,毀城外石壘,潛襲樅陽,拔其城。秋,遣將攻池州,奪青溪關。李秀成循江岸上竄,連敗之三山、光穴、子橋、白茅嘴、運漕鎮。分兵入內湖,攻神廟山、鎮山,斷松林口浮橋。冬,由魯港潛行百里,解南陵圍,拔出總兵陳大富一軍,及難民十餘萬,被珍賚。十一年,合攻安慶,偕陸軍破赤岡嶺援賊。戰無為州神塘河,平其壘,焚賊船,劃菱湖兩岸賊屯。集攻安慶東門,乘勝拔城北諸壘,城賊窮蹙。八月,克安慶,遣總兵王明山、黃翼升克池州、桐城,予雲騎尉世職。岳斌屢乞假省親,至是始歸。

同治元年,以母病請展假,不允。五月,至軍,移屯烏江。進攻金柱關,戰龍山橋,殲賊萬餘。,賊尋復來犯陣,斬賊酋陳緒賓,破護駕墩、石垝賊壘。自是江寧大營後路始固。二年春,從曾國籓赴前敵大勝關、雨花臺察視,與曾國荃定合圍之策。三月,克黃池,悉收內河三里埂、伏龍橋、花津、護駕墩諸隘,以通寧國、蕪湖之路。五月,克巢縣、含山、和州及江浦、浦口,破下關、草鞋峽、燕子磯,趨九洑洲,力戰拔之。自是長江無賊舟。十月,克高淳、寧國、建平、溧水,奪東壩要隘,江寧遂合圍。岳斌因親病請歸養,詔賜其父母人葠四兩,慰留之。

三年,命督辦江西、皖南軍務,援軍悉歸節制。尋授陜甘總督,命俟江、皖賊氛凈盡後赴任。江寧平,加太子少保,予一等輕車都尉世職。六月,岳斌抵南昌,遣諸將克崇仁、東鄉、金谿、宜黃、南豐,解寧都圍。秋,赴贛州,克瀘溪、新城、雩都,先後收降賊十餘萬,防境肅清。復疏陳傷病親老,請開缺,不允,乃回籍募兵。四年,率彭楚漢等新軍十營從行,抵西安。會僧格林沁戰歿曹州,詔岳斌移兵入衛京畿。自請開缺,專任剿匪,不許,仍命速赴甘肅,六月,履任。

時甘回方熾,通省糜爛。雷正綰、曹克忠新敗於金積堡,都興阿、穆圖善攻寧夏未下,且奉命將出關;本省兵皆疲弱,疏調各省援兵,無一至者,僅自率新募之數千人;又因兵荒耕作久廢,饋運道塞,庫空如洗。岳斌迭疏乞協餉,僅川、陜鄰省稍稍接濟,無以遍給。議進軍先搗靈州,繼規河、狄。未幾,陶茂林、雷正綰兩軍相繼潰變。五年春,岳斌親赴涇州、慶陽視師。蘭州標兵遽變,圍署戕官,逼迫布政使林之望上疏,言糧餉獨厚楚軍,眾心不服。岳斌聞警,先令曹克忠移師鎮撫,尋自回省城,按誅首犯百餘人,餘不問。以在途拆閱林之望奏摺,自請議處,革職留任,降三品頂戴。迭疏請罷,詔以左宗棠代之,未至,六年春,復陳病劇,乃命穆圖善暫署總督,許嶽斌回籍。

光緒元年,命偕彭玉麟巡閱長江,整頓水師,屢以親病請罷,五年,始允之。九年,法越戰事起,詔岳斌會辦福建軍務,未至,復命赴江南幫辦軍務。十一年,率十二營赴援臺灣,和議成,仍乞養歸。

十六年,卒於家,贈太子太保,照總督例賜恤,建專祠,謚勇愨。岳斌與玉麟始終長江軍事,所部以功敘擢至提、鎮者不可勝數,實膺專閫者亦數十人。

王明山,湘潭人。初隸岳斌營,積功至守備。彭玉麟調領一營。戰鸚鵡洲,登陸破賊,攻金口先登,累擢游擊。咸豐六年,補乾州協都司。攻漢陽,焚東南門賊船,連破賊於黃州樊口、富池口。戰武穴,伏蘆洲,伺賊登岸,突擊殲之。回擊武昌援賊,累捷,擢參將。戰蘄州,焚賊舟七十餘。登岸誘敵,賊聚攻,別隊乘虛襲城,克之,擢副將,賜號拔勇巴圖魯。進克黃州,會攻九江。八年,授浙江金華協副將。克東流,薄安慶,毀城外賊壘,以總兵記名。九年,乞假回籍。會石達開犯湖南,率隊自衡州趨祁陽要擊之,破賊於毛家埠。十年,授安徽壽春鎮總兵,破賊蕪湖蟂磯、義橋。十一年,破賊練潭鎮,斬其渠龔天福。復會陸師克赤岡嶺,遂下安慶。楊岳斌假歸,令明山代統其軍。連復池州、銅陵、破泥汊口、神塘河諸壘。克無為州,別遣將遏巢湖口,克運漕鎮,進拔東關。同治元年,擢福建陸路提督。克銅城徬,復和州、含山、巢縣,殲逸賊於木橋、沙洲,又破之江心洲、西梁山。尋以傷病乞假歸。明山在軍十餘年,屢當大敵。江南平,遂不出。光緒中,圖功臣像於紫光閣,明山與焉。十六年,卒於家,賜血⼙。

孫昌凱,清泉人。入水師,積功擢千總。昌凱舊業鐵工,田家鎮之戰,領小舟為頭隊。冒槍砲鼓備斷鐵鎖,纜開,大呼猛進,筏上賊潰走。後隊縱火,賊舟盡焚。功最,擢守備。咸豐五年,破賊漢口,擢都司。六年,從攻武昌,焚賊舟,授廣東陸路提標游擊。七年,從平蘄、黃賊巢,克小池口、湖口,擢參將。克九江,加副將銜,補兩廣督標參將。九年,回援湖南,防祁陽、衡州,擢惠州協副將。以母病乞養開缺。光緒中,彭玉麟疏薦昌凱誠實篤毅,驍果善戰,授浙江海門鎮總兵。丁母憂,改署任,留襄海防。事定,請終制。後仍補原官,調署處州鎮。二十一年,卒,賜恤,附祀彭玉麟祠。

楊明海,長沙人。入水師,洊擢守備。咸豐十年,戰樅陽、殷家匯、池州、蟂磯,迭破賊,擢都司。十一年,克南陵,擢游擊。克安慶,擢副將。同治元年,從攻東梁山、金柱關,裹創血戰,功最,以總兵記名,賜號忱勇巴圖魯。二年,大捷於九洑洲,以提督記名。戰江寧小沙口,先登陷陣,砲子穿右股,率哨船渡江,從陸軍進剿蘇州,授山東兗州鎮總兵。蘇州復,遂留防。三年,楊岳斌赴甘肅,調明海偕彭楚漢率所募兵從行,破回匪於金縣夏官營,晉號格洪額巴圖魯。軍食久乏,明海奉檄治糧運。八年,赴兗州鎮本任。光緒元年,母憂去官。七年,授狼山鎮總兵。十一年,卒,賜恤。

謝濬畬,原名得勝,長沙人。充水師哨長,進攻武昌,濬畬自請為前鋒。突鹽關賊壘,薄鸚鸝洲,與陸師夾擊,克武、漢,戰蘄州田家鎮,累功擢守備。克九江,擢都司。破賊赤岡嶺,擢游擊。同治元年,從彭玉麟克太平及金柱關、東梁山、秣陵關、九洑洲諸要隘,擢副將。江寧平,以總兵記名,授提標中軍副將。光緒十八年,擢瓜洲鎮總兵,兼署水師提督,調署漢陽鎮。二十七年,卒於官,賜恤,附祀彭玉麟祠。

論曰:彭玉麟、楊岳斌佐曾國籓創立水師,為滅賊根本。兩人勛績,頡頏相並。岳斌後為朝旨強促西征,用違其才,僨事損望。玉麟終身不任官職,巡閱長江,為國家紓東顧之憂。其疏論古人晚節之失,由於不能自藏其短,且惜朝廷不善全其長,洵至言也。後盛昱劾其辭尚書之命,乃謂抗詔鳴高,殆淺之乎測玉麟矣。


\end{pinyinscope}