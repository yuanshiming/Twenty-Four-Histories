\article{列傳二十九}

\begin{pinyinscope}
覺羅果科覺羅阿克善敦拜哈寧阿碩詹碩詹孫達色

濟席哈弟費雅思哈噶達渾費揚武愛松古興鼐興鼐兄孫哈爾奇

達素喀爾塔喇喀爾塔喇子赫特赫

覺羅果科,滿洲鑲白旗人,未詳其屬籍。事太宗,授巴牙喇甲喇章京。崇德六年,從伐明,圍錦州,分兵屯杏山河岸。明兵自寧遠至,果科與噶布什賢噶喇依昂邦努山擊破之,逐至連山,斬級三十,得馬三十二。七年,與努山略寧遠,明兵自中後所犯我牧地,擊之潰遁。八年,復與努山至界嶺口詗明兵,與戰,斬裨將一、步騎三百餘。

順治元年,從入關,擊李自成,追至慶都。二年,從英親王阿濟格下陜西,克綏德。自成兄子錦據延安,果科與巴牙喇纛章京希爾根三戰皆捷。自成奔湖廣,師從之,次安陸,迭擊敗之,得舟八十。三年,從肅親王豪格討張獻忠,經漢中,擊叛將賀珍,進次西充,破獻忠,復與希爾根搜剿餘寇。五年,從鄭親王濟爾哈朗下湖南,授巴牙喇甲喇章京。攻湘潭,明總督何騰蛟城守,果科與噶布什賢章京瑚沙破西門入。尋與固山額真佟圖賴率兵趨衡州,擊破明兵,攻拔石橋寨。又擊破明將周進唐、胡一清等,逐一清至全州。師還,授刑部理事官。

十一年,授工部侍郎。敘功,遇恩詔,並以監修壇廟,世職累進二等阿達哈哈番。十七年,擢工部尚書。十八年,卒。追坐修倉糜費,罰鍰,降世職拖沙喇哈番。聖祖親政,其子薩爾布訴枉,復拜他喇布勒哈番。

覺羅阿克善,滿洲正黃旗人,景祖兄索長阿三世孫。事太宗,授甲喇額真。崇德六年,圍錦州,與果科同在行,擊敗明總兵吳三桂及松山、杏山援軍。師還,明兵襲梅勒額真索海軍,阿克善與巴牙喇纛章京伊爾德赴援,擊卻之,又屢擊敗總督洪承疇軍,授半個前程。八年,從鄭親王濟爾哈朗伐明,攻寧遠,分兵攻前屯衛,先登,克其城。

順治元年,從入關。七年,擢正黃旗滿洲梅勒額真,兼工部侍郎。八年,調兵部。敘功,並遇恩詔,進世職一等阿達哈哈番。九年,與固山額真噶達渾征蒙古鄂爾多斯部,殲其眾於賀蘭山。總兵任珍殺其孥,賄兵部寢勿治,事發,阿克善罷侍郎,降世職拜他喇布勒哈番兼拖沙喇哈番。十一年,暫署都察院左都御史。從征湖廣,戰湘潭、常德、龍陽,屢捷。

十三年,從鄭親王世子濟度討鄭成功,師次烏龍江,水險不可渡,又間道趨福州,分兵令牛錄額真褚庫先驅擊成功,署巴牙喇纛章京伊色克圖擊成功舟師,遂至福州。諜言成功舟三百泊烏龍江,阿克善等水陸合擊,逐敵至三江口,斬其將林祖蘭等,俘獲甚眾。十四年,成功兵侵羅源,阿克善督兵赴援,力戰死之,進世職三等阿達哈哈番。

敦拜,富察氏,滿洲正黃旗人,先世居沙濟。父本科理,歸太祖。嘗從鄂佛洛總管達賴討硃舍理部長尤額楞,有功,賜號蘇赫巴圖魯,授牛錄額真。卒,敦拜嗣。天命十一年,從太祖攻寧遠,先驅,敗城兵。師還,敵騎追射,敦拜還擊卻敵,殿而歸。

天聰八年,授世職牛錄章京。崇德五年,擢巴牙喇纛章京。從鄭親王濟爾哈朗圍錦州,城兵出誘戰,敦拜突入敵隊中,斬三人,眾悉遁。明兵自杏山再來犯,皆戰卻之。六年,復圍錦州,明兵自松山攻兩紅旗及蒙古軍,敦拜御敵力戰,斬二百餘級,得雲梯十四。七年,加半個前程。八年,與巴牙喇纛章京阿濟格尼堪率師駐錦州。

順治元年,從入關,擊李自成,逐之至慶都。二年,進世職二等甲喇章京。大將軍豫親王多鐸南征,敦拜將巴牙喇兵從。次陜州,破自成將劉方亮,方亮兵夜襲營,復擊敗之。克潼關,定西安。自成走商州,入湖廣,敦拜與巴牙喇纛章京阿爾津等追斬三百餘級。從豫親王下江南,克揚州,薄明南都。追明福王至蕪湖,與阿爾津、圖賴等截江口,擊破明將黃得功,得明福王以歸。三年,進世職一等。從端重親王博洛自浙江徇福建,與梅勒額真珠瑪喇合軍破敵。五年,從大將軍譚泰討金聲桓,攻九江,破王得仁軍,克之,撫臨江郡縣。

六年,剿畿南土寇,斬其渠,獻、雄、任丘、寶坻諸縣悉定。七年,進世職三等阿思哈尼哈番。尋從睿親王畋於中後所,坐私出獵,降世職一等阿達哈哈番。八年,上親政,復世職。九年,進二等。

十一年,明將李定國犯廣東,命佐將軍珠瑪喇討之,克新會,逐之至橫州江岸,定國引去。師還,晉世職一等精奇尼哈番。以病乞休,加太子太保。十四年,起為盛京總管。十七年,卒,謚襄壯。乾隆初,定封一等男。

富察之族,有哈寧阿、碩詹、濟席哈、費雅斯哈,皆以武功顯。

哈寧阿,滿洲鑲白旗人,世居額宜湖。父阿爾圖山,率其族攻薩齊庫城,殺其部長喀穆蘇尼堪,撫降三百餘人,以歸太祖,授牛錄額真。既,復分其眾別編一牛錄,以命哈寧阿。天聰二年,從貝勒岳託等伐明,略錦州,攻松山、杏山、高橋諸臺堡,戰甚力,授巴牙喇纛章京。三年,從伐明,薄明都,與袁崇煥戰廣渠門外,以功授世職備御。五年,從攻大凌河。八年,從攻大同,哈寧阿先驅,至小西城,樹雲梯以攻,克之,復將二十人出戰,敗敵兵三百。九年,與承政圖爾格入明邊。師還,道平魯衛,明兵躡師後,還擊敗之,逐薄壕,多所斬馘,進二等甲喇章京。

崇德元年,從攻皮島。二年,授議政大臣。三年,從豫親王多鐸如錦州會師,道中後所,祖大壽以輕騎掩我師,甲喇額真翁克及土默特兵先奔,哈寧阿且戰且退,士卒有死者,論罪當死,上貸之,命奪世職,籍家產之半。四年,復以庇牛錄額真阿蘭太失律,論罪當死,上復貸之。六年,從圍錦州,屢敗敵。明總督洪承疇赴援,上督諸軍環松山而營,度明師且遁,遣諸將分地為伏以待。哈寧阿與巴牙喇纛章京鰲拜陣於海濱,夜初更,明師循海走,哈寧阿等起掩擊,明師蹂藉,死者甚眾。尋進攻松山,屢敗敵。八年三月,與巴牙喇纛章京阿爾津伐虎爾哈部,俘男婦二千五百有奇,獲牲畜、貂皮無算。師還,上厚賚之。

順治元年,從入關,擊李自成,戰慶都,再戰真定,自成焚輜重走。二年,復授世職三等甲喇章京。逐賊綏德,徇延安,戰破城兵。南逐自成,戰安陸,得舟八十。復與譚泰合兵下江南,戰江上,奪敵舟。逐敵至富池口,敵據江岸為陣,復擊之敗。三年二月,從順承郡王勒克德渾略湖廣,破明將吳汝義,降甚★。四月,進二等甲喇章京。五月,從肅親王豪格討叛將賀珍,取漢中,逐賊至秦州。珍黨武大定據三寨山,山勢峻不可攻,師圍之。會其將周克德、石國璽皆乞降,克德遣其子導師自僻徑登,國璽為內應,哈寧阿與梅勒額真阿拉善、署巴牙喇纛章京噶達渾將六百人破壘入,賊皆自投崖下,斬殺略盡。進討張獻忠,徇夔州、茂州、資州、遵義,皆下。五年,師還,進一等阿達哈哈番。尋卒。

碩詹,滿洲正紅旗人,世居訥殷。父舒穆祿,歸太祖,授牛錄額真。卒,碩詹嗣,尋兼甲喇額真。天聰五年,與甲喇額真杭什木、沙爾虎達等略明邊,遇邏卒,斬其三,俘其五及邏卒長。八年,授世職牛錄章京。崇德元年,從伐朝鮮,攻江華島,碩詹舟越朝鮮戰艦,繼牛錄額真阿哈尼堪以登,率眾合圍,降其城,加半個前程。三年,兼刑部理事官。從伐明,深入山東,克禹城、平陰。四年,師還,明兵襲我後軍,與巴圖魯尼哈裡等擊卻之,進世職三等甲喇章京。擢戶部參政。五年,師伐明,命碩詹如朝鮮徵糧及水師助戰。從圍錦州,甲喇額真禧福率甲士二十四駐守駱駝山,明兵四百夜劫營,碩詹赴援,斬二百餘級,得馬十六。七年,領本旗梅勒額真。

順治元年,從入關,改侍郎。上將遷都燕京,命碩詹統右翼兵留守盛京。尋復命從豫親王多鐸南征,自河南徇陜西,遂移師定江南。敘功,世職累進一等阿達哈哈番兼拖沙喇哈番。五年,從鄭親王濟爾哈朗征湖南,偕都統佟圖賴等師出湘潭,明兵阻橋立寨,與固山額真伊拜、巴牙喇甲喇章京覺羅果科共擊下之,斬其將陶養用,衡州平。師還,賚白金三百,進世職一等阿思哈尼哈番。

八年,坐戶部給餉不均,降世職一等阿達哈哈番。九年,以老病罷。十年,命復世職。康熙二年,卒,謚明敏。以其孫達色、法色分襲世職,並授二等阿達哈哈番。

達色以參領從征福建,戰屢捷。鄭錦將劉國軒眾萬餘犯海澄,達色赴援,冒槍砲力戰,聞城陷,自經死,加拖沙喇哈番。法色兼襲,復合為一等阿思哈尼哈番兼拖沙喇哈番。子明寶,雍正間從征西藏,有功,進三等精奇尼哈番。乾隆初,改三等子。子德成,降襲三等男。

濟席哈,亦富察氏,滿洲正黃旗人。父本科里,官牛錄額真。濟席哈初亦授牛錄額真。崇德四年,擢巴牙喇纛章京。五年,從伐明,圍錦州。明兵自松山至,邀戰,與甲喇額真布丹、希爾根等擊卻之。尋駐義州護屯田,上誡諸將固守營壘,勿與明兵戰。明兵犯鑲藍旗營,濟席哈越鑲紅旗營助戰,以擅離汛地,奪官,籍其家三之一。旋與梅勒額真席特庫伐索倫部,得其部長博穆博果爾以歸。六年,師還,與宴勞。七年,授正紅旗蒙古梅勒額真。八年,兼戶部參政。

順治元年,從入關,擊李自成,追之至慶都。敘功,授世職拜他喇布勒哈番。二年,從端重親王博洛下浙江,既克杭州,以梅勒額真駐守。明大學士馬士英、總兵方國安據嚴州,屢來犯,濟席哈督兵御之,五戰皆捷。還京,授工部侍郎,加世職拖沙喇哈番。

五年,命率兵駐東昌。尋以鄭彩寇福建,命從將軍陳泰南征,克長樂、連江、同安、平和諸縣,進世職二等阿達哈哈番。七年,調刑部,擢尚書,進世職三等阿思哈尼哈番。九年,授正紅旗蒙古固山額真。十年,解尚書。膠州總兵海時行叛,命與梅勒額真瑚沙討之,未至,時行走宿州降。詔移兵鎮湖南。十一年,召還。

十四年,命率梅勒額真四、巴牙喇甲喇章京八,從大將軍貝子羅託征雲南。十五年,命佐將軍卓布泰,師進次都勻,擊敗明將李定國。會師,克雲南。十七年,以勘從征將士功罪不實,降一等阿思哈尼哈番。十八年,授靖東將軍,討棲霞土寇於七,擊破所據岠嵎山寨,七竄入海。康熙元年,卒。六十年,以其子西安副都統阿祿疏請,追謚勇壯。

費雅思哈,濟席哈弟也。初以巴牙喇壯達事太宗。天聰六年,從伐察哈爾,分兵略大同,至朔州,城兵出戰,費雅思哈與甲喇額真道喇等擊敗之。崇德三年,署巴牙喇纛章京,從貝勒岳託伐明,敗密雲步兵。五年,師圍錦州,明兵自松山、杏山赴援,費雅思哈御戰皆捷。六年,復圍錦州,同甲喇額真哈寧阿擊敵城下,射殪三人,明總督洪承疇步隊自松山至,費雅思哈力戰卻敵。

順治元年,從入關,擊李自成,追敗之慶都,授巴牙喇甲喇章京。從英親王阿濟格西討,二年春,次榆林,自成兵夜襲營,與巴牙喇纛章京車爾布等擊之走,追自成至武昌,屢破其壘;又以舟師邀擊富池口,得舟三十。三年,從肅親王豪格討張獻忠,道西安,分兵徇邠州。其渠胡敬德以千餘人據三水西北山岡,費雅思哈與巴牙喇纛章京噶達渾破其壘,復與固山額真巴哈納擊叛將賀珍於雞頭關。師下四川,屢戰皆捷。正藍旗兵為賊困,與噶達渾趨援,賊走。敘功,授世職拜他喇布勒哈番兼拖沙喇哈番。

六年,從英親王討叛將姜瓖,掘塹圍城,瓖兵步騎萬餘來犯,費雅思哈先眾迎戰,瓖兵不得入城。瓖兵分踞左衛,陷汾州,窺太原,費雅思哈率巴牙喇兵伺擊,會師圍大同,瓖黨斬以降,進世職一等阿達哈哈番。

十三年,擢巴牙喇纛章京。尋命率兵駐防湖南。明將孫可望據辰州,費雅思哈與固山額真卓羅、梅勒額真泰什哈等,自澧州、常德進徵,可望棄城遁,縱火焚舟,阻我師。費雅思哈取其未焚者以濟師,躡擊至瀘溪,殲敵甚眾。十八年,從將軍愛星阿入緬甸,得明桂王以歸。師還,進世職三等阿思哈尼哈番。康熙十一年,卒,謚僖恪。子素丹,自有傳。

噶達渾,納喇氏,滿洲正紅旗人,世居哈達。其先有約蘭者,當太祖時,率其子懋巴里等來歸。天聰二年,噶達渾以巴牙喇甲喇章京從太宗伐多羅特部,有功。八年,從伐明,略山西,克應州。崇德五年,從伐明,略中後所。睿親王多爾袞等率師圍錦州,令領纛先進,敗杏山騎兵,設伏松山,斬十餘級,明兵營嶺上,擊破之;又從噶布什賢噶喇依昂邦勞薩追擊至北岡。七年,從豫親王多鐸攻寧遠,明兵躡我後,噶達渾先眾還擊,明兵潰走。師還,有巴牙喇兵達哈塔者,被創,僕,掖以歸。

順治元年,擢巴牙喇纛章京。從入關,擊李自成,授世職拜他喇布勒哈番。二年,從英親王阿濟格擊自成至九宮山,三敗之。三年,從肅親王豪格下四川,次西安,分兵討叛將賀珍,徇邠州,其黨胡敬德屯三水,噶達渾與梅勒額真和託直入,破其壘。高汝礪、武大定等屯三寨山,復與巴牙喇纛章京蘇拜、哈寧阿,梅勒額真阿拉善擊敗之,督步卒搜剿巖谷。大定等據山巔,其徒左右迎戰,噶達渾與巴牙喇纛章京阿爾津奮戰,挫其鋒。大定等兵攻正藍旗營,哈寧阿陷圍中,噶達渾與阿爾津、蘇拜疾馳赴援,圍乃解。擢戶部侍郎,五年,調吏部,進世職三等阿達哈哈番。

英親王阿濟格討叛將姜瓖,噶達渾與阿拉善濟師,七戰皆捷。克代州,進復渾源。六年,兼本旗蒙古固山額真。七年,世祖親政,擢戶部尚書,進世職二等。改都察院左都御史,尋還為尚書。率師征鄂爾多斯部,獲部長多爾濟,殲其眾於賀蘭山,進世職三等阿思哈尼哈番。調滿洲固山額真、兵部尚書。十年,進世職二等。世職呂忠行賕事發,部議引赦例貸其罪,坐降世職一等阿達哈哈番。

大將軍鄭親王世子濟度討鄭成功,命噶達渾佐之,敕濟度調遣官兵,毋令噶達渾離左右。克海澄,水陸並進,復福州,遂下泉州,攻惠安海港衛套及閩安鎮,大捷。十四年,師還。卒,贈太字太保,謚敏壯。同族有費揚武、愛松古、興鼐。

費揚武,滿洲正藍旗人。初自巴牙喇壯達累遷甲喇額真。崇德七年,從饒餘貝勒阿巴泰伐明,入塞,擊敗明總兵馬科。越明都,略山東,次膠州,明兵千餘屯城外,費揚武力戰破之;攻濱州,以雲梯先登。出塞,明總督範志完、總兵吳三桂等分道要我師,費揚武先後與戰皆勝,護所俘獲還。

順治初,從入關,擊李自成,敗其騎兵。尋署巴牙喇纛章京。從豫親王多鐸西討自成,次潼關,破自成將劉宗敏。二年,從定江南,攻揚州,得舟二百餘。攻明南都,敗其步兵。逐明福王至蕪湖,與明總兵黃得功戰,得舟三十有一。旋從端重親王博洛下浙江,破明馬士英軍於杭州,生致明總兵一,分兵定海寧、平湖土寇;又與明總兵王之仁戰,得舟十有六:授議政大臣,予世職甲喇章京,加半個前程。四年,從軍福建。卒。

愛松古,滿洲鑲白旗人。太祖時,自葉赫來歸,屢從征伐。崇德元年,命與察漢喇嘛等赴明邊殺虎口互市。復遣往科爾沁徵兵。三年,初設理籓院,授副理事官。尋自歸化城導厄魯特部長墨爾根戴青來歸。再坐事鞭責。

順治元年,授牛錄額真。從固山額真葉臣徇山西。時李自成西走,其將陳永福據太原,發砲攻城圮,永福突圍走,愛松古以蒙古兵戰,多斬馘,得馬千餘。又逐自成將馬驥至河濱,得舟十五。二年,從圍延安,城兵出戰,擊卻之,以八騎躡自成,獲其孥。

三年,從豫親王多鐸討蘇尼特部長騰機思,將蒙古兵三百先驅扼隘,師繼進,騰機思遁走,從侍郎尼堪、梅勒額真明安達里乘夜追擊,得其輜重;斬臺吉茂海,遂渡圖喇河,土謝圖汗以二萬人拒戰,從鎮國將軍瓦克達等敗其騎兵。敘功,授世職拖沙喇哈番。

五年,命率蒙古兵六百駐太原,擊斬涇陽寇李陽,敗交城寇王豪明。時叛將姜瓖據大同,其黨劉遷以萬餘人犯代州,愛松古馳往守御。遷眾傅雲梯乘城,鉤致其梯九,斬級三百;遷眾穴城,城上發矢石,遷眾多殪,乃走繁峙。六年,復來襲,有為應者,引入郛,愛松古嬰城守十餘日,端重親王博洛師至,擊斬其渠郭芳,遷遁去。乃還駐太原,瓖黨十餘萬來犯,愛松古與巡撫祝世昌謀遣兵赴清源徐溝防禦,不使逼城下。端重親王師自晉陽至,破賊。累擢鑲白旗蒙古梅勒額真,世職累進二等阿達哈哈番。九年,從敬謹親王尼堪南征,王沒於陣,愛松古不及救,降世職拜他喇布勒哈番兼拖沙喇哈番。十六年,致仕。康熙十四年,卒。

子訥青,以三等侍衛從討鄭成功,至廈門,卒於軍。

興鼐,滿洲鑲白旗人。父素巴海,自哈達率二百人來歸,太祖編牛錄,授其長子莽果,興鼐其第三子也。事太宗,天聰八年,授世職牛錄章京。崇德元年,從英親王阿濟格伐明,佐固山額真達爾罕攻順義,先登,加半個前程。三年,授工部理事官。考滿,進世職三等甲喇章京。順治元年,從入關,西討李自成。自成之徒自延安出犯,截擊,大破之。逐自成至武昌,躡之至富池口,列陣河岸,與巴牙喇纛章京哈寧阿、甲喇額真希爾根擊之潰。移軍江南,與巴牙喇甲喇額真布克沙敗明將黃蜚於池州,斬級二百,得舟十二。三年,從討蘇尼特部長騰機思,戰敗土謝圖汗、碩類汗二部兵。擢工部侍郎,累進世職二等阿思哈尼哈番。十五年,以勘羅源戰敗將士有所徇,奪官,削世職。十八年,聖祖即位,復授一等阿達哈哈番兼拖沙喇哈番。康熙三年,卒。

哈爾奇,莽果孫也。順治十六年,以巴牙喇壯達從軍。鄭成功內犯,自荊州援江寧,破成功將楊文英。署巴牙喇甲喇章京。討耿精忠,迭戰敗其將楊益茂於九江、邵聯登於建昌,又敗吳三桂將夏國相於萍鄉、謝勝先於瀏陽、吳國貴於武岡。敘功,授拖沙喇哈番。卒。

達素,章佳氏,滿洲鑲黃旗人,先世居費雅朗阿。天聰五年,以巴牙喇壯達從伐明,圍大凌河。明兵來援,與巴牙喇壯達鰲拜同擊卻之。略明邊,斬敵騎。師還,擢巴牙喇甲喇額真。

崇德五年,從圍錦州,敗杏山明兵。六年,復圍錦州,明兵數十人據塔山,列火器拒守。達素率六騎馳而上,盡斬之;復率兵邀擊,明兵走海岸,溺死者無算。七年,從徇寧遠,敗明騎兵。八年,從巴牙喇纛章京阿爾津等伐虎爾哈部,克博和理城,又招降能吉爾、大噶爾達蘇諸屯。

順治元年,從入關,擊李自成。從固山額真巴哈納等徇山西,克絳州,逐賊至黃河。賊以舟濟,達素督兵射之,賊多墮水死。二年,從英親王阿濟格下湖廣,討自成,克安陸、武昌,逐之至富池口,賊營對岸,達素先諸將沖擊,多所俘獲。三年,從肅親王豪格討張獻忠,道漢中,擊破賀珍,下四川,屢戰皆捷。積戰功,授世職拜他喇布勒哈番兼拖沙喇哈番。

六年,從英親王阿濟格討姜瓖,戰於右衛,賊大至,達素奮前搏擊,飛矢及其喉,手足皆創,墮馬。軍校欲負以退,叱曰:「死則死耳,何避為?」裹創督兵復戰,瓖兵敗卻。世職累進一等阿達哈哈番。

九年,從敬謹親王尼堪征湖南,次衡州。貝勒屯齊令別將兵詗敵寶慶,遇敵,擊敗之,進攻全州,破寨五,斬所置文武吏九及其徒四千餘,復興安、灌陽,復斬定國將倪兆龍。敬謹親王沒於陣,將佐俱坐罪,達素以別將兵克敵,得免議。十一年,擢巴牙喇纛章京。十三年,擢內大臣。十六年,鄭成功內犯江寧,授達素安南將軍,同固山額真索渾、巴牙喇纛章京賴塔等率師赴援,至則成功已敗走,移師赴福建。十八年,召還。康熙八年,鰲拜敗,達素為所引用,坐罷官。尋復世職。卒。同族有喀爾塔喇。

喀爾塔喇,滿洲鑲白旗人,先世亦居費雅朗阿。父圖爾坤詹,當太祖時,率五子及所部百餘戶來歸,授牛錄額真。卒,喀爾塔喇嗣,事太宗。崇德三年,以巴牙喇甲喇章京從豫親王多鐸伐明,略寧遠,將入邊,破明兵;及還,又連敗之。六年,從圍錦州,城兵出犯鑲黃旗分守壕塹,坐退避,罪當死,上命罰鍰以贖。

順治元年,從入關,擊李自成,將本旗敗其騎兵,逐之至慶都,盡殲其後隊。旋從固山額真巴哈納等徇懷慶,入山西境,破賊黃河渡口,逐之至榆林。二年,自成走湖廣,移師從之。與巴牙喇甲喇章京鰲拜攻克安陸,督兵進剿,毀其壘,得舟六十餘。

五年,從大將軍譚泰討金聲桓,師次童子渡。聲桓兵據水而陣,方舟為梁,喀爾塔喇奪以渡師,分兵趨饒州。聲桓遣別將以三千人迎戰,喀爾塔喇與甲喇額真巴朗等擊敗之,克饒州。進次南昌,營甫定,葲桓兵出戰,奮戰,挫其鋒。師合圍,喀爾塔喇屯江岸,聲桓兵以舟運糧入城,喀爾塔喇邀擊,得舟八,又縱火焚舟七百餘,師次城北。喀爾塔喇與甲喇額真艮泰分兵攻城南,六年春,克南昌。

九年,擢巴牙喇纛章京。從敬謹親王尼堪征湖南衡州,乘勝疾進,遇伏,力戰,與王同沒於陣。喀爾塔喇積戰功,世職累進一等哈達哈哈番,恤進三等阿思哈尼哈番,謚忠壯。

子赫特赫,襲。十六年,以甲喇額真從討鄭成功,攻廈門,戰死,予世職拜他喇布勒哈番。

論曰:滿洲諸大家多以地為氏,往往氏同而所自出異。戰績既著,門材遂張。濟席哈、達素嘗專將,雖所當非大敵,或未與敵遇,要其才望必有足以勝此任者。果科等皆以裨佐樹績行間,勛閥所存,亦不得而略焉。


\end{pinyinscope}