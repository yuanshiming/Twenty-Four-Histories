\article{列傳二百四十五}

\begin{pinyinscope}
徐延旭唐炯何璟張兆棟

徐延旭,字曉山,山東臨清人。咸豐十年進士,出知廣西容縣。師克潯州,與有功,累晉知府。同治九年,除知梧州。光緒三年,遷安襄荊鄖道。八年,晉廣西布政使,命督辦海防,得專奏事。時法人謀占全越,巡撫張之洞、侍讀張佩綸先後疏薦堪軍事。會南定陷,朝命出鎮南關,與提督黃桂蘭、道員趙沃籌防,未行,越官劉永福戰勝懷德府紙橋,狀其績以上。

九年,出關,至北寧而還,頓龍州,被命為巡撫,敕趣永福規河內。延旭上部署防守狀,略云:「固廣西邊疆,必守北寧;固雲南邊疆,必守山西。左軍前鋒分駐北寧、湧球,去城止十二里。一旦有事,援之則無辭於法,聽之則有慚於越。不如徙軍入城,城固我儲糧屯戍所也。並簡銳扼浪泊湖北岸,為山西聲援;別募勇百人扼月德江,與陸軍相表裏。」附請吏部主事唐景崧留軍。

初,法人犯順安,越未敗,遽乞和。延旭奏言:「越人倉卒議和,或謂因故君未葬,冀緩須臾;或謂因廢立嫌疑,朋興黨禍。越臣黃佐炎等錄寄和約,越誠無以保社稷,中國又何以固籓籬?劉永福現駐山西,法人擬益師往攻,請毋撤兵,用警戎備。」越王阮福升嗣位,遣使告哀,並懇允其詣闕乞封;復具和約二十七條及黃佐炎稟,上之樞府。左宗棠檄前布政使王德榜募勇扼桂邊,朝命受延旭節度。

其冬,力疾再出關,駐諒山,趣軍進取,分襲海陽、嘉林綴敵勢;並請撥船嚴扼海口,斷其出入:諭仍力守北寧。於是令左軍黃桂蘭、右軍趙沃協防其地。適山西陷,延旭猶慮兵力薄,復遣使入關募勇,通舊五十餘營,厚集於此。隨令廣間諜,安地營,禁擾民,嚴冒餉;然沃等皆寡識,桂蘭尤侈汰,與越官張登壇日事宴樂。登壇故通法,嗣以有郤洩其事。上命延旭罷登壇,或囚而殺之,延旭以力不能制而止。日唯籌軍火濟師,以為兵力厚,可恃以無恐。桂蘭復希風指,侈談部下能戰,延旭益信之,遂六上書請戰。上不許,敕保守未失陷地,毋貪功。

十年,法軍陷扶良,三路攻北寧,桂蘭潰奔太原。李鴻章電奏失守,延旭猶上言:「西聯滇軍,東防江口,北寧斷無他虞。」上責其飾詞。會岑毓英抵保勝,部署邊外各軍,遂命延旭軍屬之。初,延旭之任西撫也,未及兩月,亦知桂蘭等未可恃。嗣以臨敵易將,操之急,易生變,以故誥誡備至;而桂蘭等且縱兵剽奪,越民不堪命,忿滋甚。是役也,群反噬,城乃陷。延旭上其欺飾狀,並自糾請治罪。上怒,詔革職留任。

法軍乘勝入芹驛關,復命力捍之,毋再失。延旭以景崧護軍收殘兵,更約束,令駐屯梅。時諒江、朗山、狼甲相繼屠潰,諒山教民且蠢焉思動。延旭鑒覆轍,嚴禁防軍向越官索夫米,有伐一草一木者斬,越民仍不知感。適德榜至,勸延旭勉自支振,圖再舉。於是更嚴勒粵軍,仿楚勇制,力求後效。而逮問之命下,吏議斬監候,改戍新疆。追論舉主,之洞、佩綸,均被訶責。延旭未出都,病卒。子坊,自有傳。

唐炯,字鄂生,貴州遵義人。道光二十九年舉人,訓方子。訓方督師金口,炯馳數千里省視。越夕難作,倉皇奉遺疏謁曾國籓,得代奏。武昌復,求遺骸歸葬。桐梓亂民起,治鄉團御之。服闋,入貲為知縣,銓四川。

咸豐六年,署南溪。值滇寇李永和蠢動,藍朝柱應之,陷敘州,吏士皆恐。炯乃訓練兵壯,晨夜徼循,人心稍靖。有為寇所獲者,縱之還,曰:「為我語唐青天,決不犯南溪一草一木!」炯領兵偪吊黃樓,單騎入營,諭以利害,朝柱款附。永和改犯犍為,炯馳救,壁不動,俟其懈乘之,寇狂奔,自相轔藉。旋與楚軍解成都圍。八年,檄署綿州事。時郫、彭軍事棘,調還省防守。炯詗得黑窩盜虛實,請限八日畢乃事,果如所言。除知夔州,未上,逾月,永和圍綿城,炯掘壕登陴,民助貲糧。炯居城三月,不下,誓死守;援至,圍始解。已而湘、黔軍閧州署,駱秉章劾罷之。事白,仍治軍。

同治改元,統安定營。會石達開圍涪州,與劉岳昭期會師,擊走之。其夏,石黨窺綦江。炯聞警馳援,燔其壁,寇潰,大破之長寧。以疾還成都。秉章詢寇勢,時寇退滇邊,聲入黔,炯曰:「此誘我軍東下耳。彼必走夷地,乘虛入川,寧越宜警備。」俄而寇入紫地,復請遣唐友耕軍大渡河扼之。達開返西岸,退為惈夷所窘,食盡乞降,梟誅之。明年,權綏定府,區邑為八路,路若干場,場若干寨,置寨總,行記善惡法,月朔上其簿親判之;又立書院二、社學八十餘:境內稱治,下其法他縣。越二年,赴陜佐治營田。捻首張總愚犯新豐,大敗之。

六年,四川總督崇實命率師入黔。黔患貧瘠,崇實先問以理財策,炯曰:「理財莫若節用,節用莫若裁勇,裁勇莫若援黔。」崇實然其言,遂以軍事屬之。連破偏刀、水上、大平、黃飄、白堡,擒斬王超凡、劉儀順,降潘人傑、唐天佑,皆積寇也;又克平越、甕安、黃平、清平、麻哈:遷道員,賜號法克精阿巴圖魯。嗣為吳棠所劾,還蜀。

光緒四年,丁寶楨督四川,令佐治鹽筴,旋補建昌道。六年,署鹽茶道,條上善後六事,謂:「發引必先新後舊;徵稅必先課後引;收發鹽引,責成鹽道;改代引張,責成州縣;繳殘則嚴定限期;辦公則優給公費。」議行,凡百餘年引目渾殽、款項轇轕諸弊,至是盡革,語具鹽法志。八年,張之洞、張佩綸先後奏薦堪軍事,於是擢雲南布政使。炯率川軍千人駐關外,滇軍悉歸節度。既蒞事,裁夫馬,治廠務,並釐卡,清田糧,民困少蘇。

法人奪我越南,被命赴開化防守,即於軍前除巡撫。誤聞將議和,亟還省履任。上大怒,褫職逮問,刑部定讞斬監候。久之,上意解,三歷秋審,赦歸。左宗棠臚其治行上於朝,命戍雲南,交岑毓英差序。十三年,賞巡撫銜,督辦雲南礦務,偕日本礦師躬履昭通、東川、威寧銅鉛各廠,疏陳變通章程,又歷請減免貴州鉛課,豁免雲南礦廠官欠民欠,並報可。惟經營十五年,僅歲解京銅百萬斤,為時論所譏。三十一年,謝病歸。三十四年,以鄉舉重逢,晉太子少保。逾歲卒,年八十,恤如制。

何璟,字小宋,廣東香山人。父曰愈,見循吏傳。璟,道光二十七年進士,選庶吉士,授編修,轉御史。咸豐七年,英人陷廣州,總督葉名琛獲譴罷,而巡撫柏貴等罪相埒,譴弗及,時論譁然,璟乃分別上其謬誤狀。明年,英艦入津沽,疏陳戰守要略,先後抗論外務,疏凡八上。遷給事中。十年,出為安徽廬鳳道。同治二年,遷按察使。捻至,與總兵喻吉三隨方應御,寇不得逞。四年,晉湖北布政使。逾歲,到官,值黃陂饑嗛,民就食江、漢,便宜發帑金濟之。九年,擢福建巡撫,歷山西、江蘇。遭父憂,服闋,起閩浙總督。

光緒三年,備日本議起,治海防,飭戎政。其夏患水祲,日坐城上督拯難民,凡閱七晝夜,醵金恤之。水退,濬洪塘江,導支流入海,後患稍殺。五年,兼署巡撫。時日本議廢琉球,數以兵艦浮閩、滬。璟以臺灣地當要沖,基隆尤扼全臺形勝,乃調集輪舶,增募兵勇,建築砲臺,備不虞。

九年,法越事起,海防戒嚴。璟令總兵張得勝等分扼諸郡,提督孫開華等分扼臺、澎,並檄楊在元署臺灣鎮,助防守。明年,又上福、廈、臺益船募卒狀,上皆勖勵之。已而會辦軍務,張佩綸至,事皆專決,視璟等若屬吏;又嚴劾在元貪謬,璟坐疏忽,幹吏議。以是益畏事之,不敢為異同。佩綸調舟師砲船局,璟亦以砲布衙署自衛。廷旨以閩事亟,諄諄諭固守。逮戰書至,璟告佩綸曰:「明日法人將乘潮攻馬尾矣!」佩編弗聽。舟師大挫,璟欲馳援,而臨浦無舟可濟,株守省城,卒致閩事日壞;然猶左袒廣勇,雖逃不問,頗為時訾議。乃飛章自劾,而廷旨已先召還京。尋御史亦劾其闒冗,部議褫職。十四年,卒。

張兆棟,字友山,山東濰縣人。道光二十五年進士,銓刑部主事,累遷郎中。出知陜西鳳翔府,蒞事三月,而回寇竊發,乃募鄉兵捍之。無何,城被圍,寇且掘長壕圖久困,兆棟晨夕登陴慰勞守者。寇轟潰西南城,蟻附上,兆棟躬冒矢石,戰甚力,寇不得逞。益固結紳民,誓堅守,閱十有六月,援師至,圍始解。超授四川按察使。咸豐四年,調廣東,遷布政使。左宗棠治軍嘉應,餽運阻絕,兆棟殫心籌畫,給食不乏。歷安徽、江蘇,皆稱職。

九年,擢漕運總督。時運河久廢不緝,兆棟慮海警阻漕,為上治河濟運狀,稱旨。十一年,再調廣東。粵俗嗜博,闈姓害尤烈,疏請禁止,報可,而總督英翰曲徇商人請,弛其禁,兆棟劾之,落職,遂兼攝總督事,禁益嚴,終其任,粵吏無敢言闈姓者。光緒四年,母憂歸。服闋,起福建巡撫。

十年,法越事起,法艦窺臺、閩。張佩綸銜命會辦閩防軍務,兆棟畏其焰,曲意事之,日謁如衙參。佩綸虛飾勝狀,詔發萬金犒兆棟軍。兆棟且疏劾大員謀遁,意指何璟也,朝旨令據實以聞。已而事亟,己亦微服匿民間,數日略定,復出任事。璟罷,兆棟兼總督,朝廷論馬尾失守罪,褫職。十三年,卒於閩。宣統元年,復官。

論曰:法越初構釁,號識時務者爭上書忼慨言戰。未及旬日,延旭敗退廣西,炯棄關外新安行營。何璟、兆棟懾張佩綸之氣勢,怯懦而無所主,事急皆遁。方其互相汲引,不恤舉疆事以輕試;及其敗也,其黨益肆言熒聽,而此數人者,遂得保首領以沒。朝廷固寬大,亦失刑甚矣。


\end{pinyinscope}