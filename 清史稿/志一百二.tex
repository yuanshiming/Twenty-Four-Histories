\article{志一百二}

\begin{pinyinscope}
○河渠二

△運河

運河自京師歷直沽、山東,下達揚子江口,南北二千餘里,又自京口抵杭州,首尾八百餘里,通謂之運河。

明代有白漕、衛漕、閘漕、河漕、湖漕、江漕、浙漕之別。清自康熙中靳輔開中河,避黃流之險,糧艘經行黃河不過數里,即入中河,於是百八十里之河漕遂廢。若白漕之藉資白河,衛漕之導引衛水,閘漕、湖漕之分受山東、江南諸湖水,與明代無異。嘉慶之季,河流屢決,運道被淤,因而借黃濟運。道光初,試行海運。二十八年,復因節省幫費,續運一次。迨咸豐朝,黃河北徙,中原多故,運道中梗。終清之世,海運遂以為常。

夫黃河南行,淮先受病,淮病而運亦病。由是治河、導淮、濟運三策,群萃於淮安、清口一隅,施工之勤,糜帑之鉅,人民田廬之頻歲受災,未有甚於此者。蓋清口一隅,意在蓄清敵黃。然淮強固可刷黃,而過盛則運堤莫保,淮弱末由濟運,黃流又有倒灌之虞,非若白漕、衛漕僅從事疏淤塞決,閘漕、湖漕但期蓄洩得宜而已。至江漕、浙漕,號稱易治。江漕自湖廣、江西沿漢、沔、鄱陽而下,同入儀河,溯流上駛。京口以南,運河惟徒、陽、陽武等邑時勞疏濬,無錫而下,直抵蘇州,與嘉、杭之運河,固皆清流順軌,不煩人力。今撮其受患最甚、工程最鉅者著於篇。

順治四年夏久雨,決江都運堤,隨塞。六年夏,高郵運堤決數百丈。七年,運堤潰,挾汶水由鹽河入海。八年,募民夫大挑運河。十四年,河督硃之錫言:「南旺南距臺莊高百二十尺,北距臨清高九十尺,應遵定例,非積六七尺不準啟閘,以免瀉涸。閉下閘,啟上閘,水凝亦深;閉上閘,啟下閘,水旺亦淺。重運板不輕啟,回空板不輕閉。」從之。十五年,董口淤。之錫於石牌口迤南開新河二百五十丈,接連大河,以通飛輓。先是漳水於九年從丘縣北流,逕青縣入海。至十七年春夏之交,衛水微弱,糧運澀滯,乃堰漳河分溉民田之水,入衛濟運。時河北累年亢旱,部司姜天樞言:「昔僉事江良材欲導河注衛,增一運道,今獨不可借其議而反用之導衛以注河乎?」之錫從其言,並置衛河主簿,著為令。

康熙元年,定運河修築工限:三年內沖決,參處修築官;過三年,參處防守官;不行防護,致有沖決,一並參處。四年秋,高郵大水,決運堤。五年,運河自儀徵至淮淤淺,知縣何崇倫募民夫濬之。漕督林起龍言:「糧艘北行,處處阻閘阻淺,請飭河臣履勘安山、馬踏諸湖,暨各櫃閘子堤斗門堤岸,及東平、汶上諸泉,有無堵塞,務期濬泉清湖,以通運道。」六年,決江都露筋廟。明年,塞之。十年,決高郵清水潭。明年,再決,十三年始塞。十四年,決江都邵伯鎮。十五年夏,久雨,漕堤崩潰,高郵清水潭、陸漫溝,江都大潭灣,共決三百餘丈。

十六年,以靳輔為河督。時東南水患益深,漕道益淺。輔言:「河、運宜為一體。運道之阻塞,率由河道之變遷。向來議治河者,多盡力於漕艘經行之地,其他決口,以為無關運道而緩視之,以致河道日壞,運道因之日梗。是以原委相關之處,斷不容歧視也。又運河自清口至清水潭,長約二百三十里,因黃內灌,河底淤高,居民日患沈溺,運艘每苦阻梗。請敕下各撫臣,將本年應運漕糧,務於明年三月內盡數過淮。俟糧艘過完,即封閉通濟閘壩,督集人夫,將運河大為挑濬,面寬十一丈,底寬三丈,深丈二尺,日役夫三萬四千七百有奇,三百日竣工。並堵塞清水潭、大潭灣決口六,及翟家壩至武家墩一帶決口,需帑九十八萬有奇。」又言:「向因河身淤墊,阻滯盤剝,艱苦萬端。若清口一律浚深,則船可暢行,省費甚多。因令量輸所省之費,作治河之用,請俟運河浚深,船艘通行,凡過往貨物船,分別徵納剝淺銀數分,一年停止。」均允行。

十七年,築江都漕堤,塞清水潭決口。清水潭逼近高郵湖,頻年潰決,隨築隨圮,決口寬至三百餘丈,大為漕艘患。前年尚書冀如錫勘估工費五十七萬,夫柳仍派及民間,猶慮功不成。輔周視決口,就湖中離決口五六十丈為偃月形,抱兩端築之,成西堤一,長六百五丈,更挑繞西越河一,長八百四十丈,僅費帑九萬。至次年工竣。上嘉之,名河曰永安,新河堤曰永安堤。是歲挑山、清、高、寶、江五州縣運河,塞決口三十二。輔又請按裡設兵,分駐運堤,自清口至邵伯鎮南,每兵管兩岸各九十丈,責以栽柳蓄草,密種菱荷蒲葦,為永遠護岸之策。又言:「運河既議挑深,若不束淮入河濟運,仍容黃流內灌,不久復淤。請於高堰堤工單薄處,幫修坦坡,為久遠衛堤計。」均如所議行。

十八年,決山陽戚家橋,隨塞。明初江南各漕,自瓜、儀至清江浦,由天妃閘入黃。後黃水內灌,潘季馴始移運口於新莊閘,納清避黃,仍以天妃名。然口距黃、淮交會處僅二百丈,黃仍內灌,運河墊高,年年挑濬無已。兼以黃、淮會合,瀠洄激蕩,重運出口,危險殊甚。至是,輔議移南運口於爛泥淺之上,自新莊閘西南挑河一,至太平壩,又自文華寺永濟河頭起挑河一,南經七里閘,轉而西南,亦接太平壩,俱遠爛泥淺。引河內兩渠並行,互為月河,以舒急溜,而爛泥淺一河,分十之二佐運,仍挾十之八射黃,黃不內灌,並難抵運口。由是重運過淮,揚帆直上,如履坦途。是歲開滾水壩於江都鰍魚骨,創建宿遷、桃源、清河、安東減壩六。

十九年,創建鳳陽廠減壩一,碭山毛城鋪、大谷山,宿遷攔馬河、歸仁堤,邳州東岸馬家集減壩十一。康熙初,糧艘抵宿遷,由董口北達。後董口淤塞,遂取道駱馬湖。湖淺水面闊,纖纜無所施,舟泥濘不得前,挑掘舁送,宿邑騷然。輔因創開皁河四十里,上接泇河,下達黃河,漕運便之。是歲霪兩,淮、黃並漲,決興化漕堤,水入高郵治,壞泗州城郭,特築滾壩於高郵南八里,及寶應之子嬰溝。

二十年七月,黃水大漲,皁河淤澱,不能通舟。眾議欲仍由駱馬湖,輔力持不可,親督挑掘丈餘,黃落清出,仍刷成河。隨閉皁河口攔黃壩,於迤東龍岡岔路口至張家莊挑新河三千餘丈,使出皁河,石磡之清水盡由新河行,至張家莊入黃河,是為張莊運口。是歲增築高郵南北滾水壩八,對壩均開越河,以防舟行之險,凡舊堤險處,皆更以石。二十二年九月,黃河由龍岡漫入,新河又淤。隨於石磡築攔黃壩,復設法疏導,旬餘,新河仍暢行。二十三年,上南巡閱河,至清口,以運口水緊,令添建石閘於清河運口。

二十五年,輔以運道經黃河,風濤險惡,自駱馬湖鑿渠,歷宿遷、桃源至清河仲家莊出口,名曰中河。糧船北上,出清口後,行黃河數里,即入中河,直達張莊運口,以避黃河百八十里之險。議者多謂輔此功不在明陳瑄鑿清口下。而按察使於成龍、漕督慕天顏先後劾輔開中河累民,上斥其阻撓。二十七年,復遣尚書張玉書、圖納,左都御史馬齊等往視,亦稱中河安流,舟楫甚便。但逼近黃流,不便展寬,而里運河及駱馬湖之水俱入此河,窄恐難容,應於蕭家渡、楊家莊、新莊各建減壩,俾水大可宣洩;仲家閘口大直恐倒灌,應向東南斜挑以避黃流。詔俟臨閱時定奪。是歲大雨,中河決,淹清河民田數千頃。

明年春,上南巡,閱視河工,至宿遷支河口,謂諸臣曰:「河道關系漕運民生,地形水勢,隨時權變。今觀此河狹隘,逼近黃岸,萬一黃堤潰決,失於防禦,中河、黃河將溷為一。此河開後,商民無不稱便,安識日後若何?」圖納、馬齊言:「臣等勘河時,正值大水,懼河隘不能容諸水,故議於迤北遙堤修減壩三,令由舊河形入海。」輔言:「臣意開此河,可束水入海,及濬畢觀之,漕艘亦可行。今若加增遙堤,以保固黃河堤岸,當可無慮。」河督王新命言:「支河口止一鎮口閘,微山湖諸水甚大,遇淫潦不能支,必致潰決。若於駱馬湖作減壩,令漲水入黃,再修築郯城禹王臺,以禦流入駱馬湖之水,令注沭河,則中河無慮。」上謂可仍開支河,其黃河運道,並存不廢。先是玉書等請閉攔馬河,事下總河,至是新命言:「攔馬河原以宣黃水異漲,似應仍留,水漲則開放,水平則閉,以免中河淤墊。至駱馬湖三減壩,玉書等議留二座於堤內,減水入中河,又恐中河不能容,擬於迤東蕭家渡、楊家莊、新河口量建減壩宣洩。臣謂既以中河不能容,何必留此二壩之水減入中河,復從蕭家渡等處建壩,多此曲折?不若將三壩俱留遙堤外,令由舊河形入海,於蕭家渡三處量留缺口二,酌水勢以宣塞之為愈。郯城沭水口舊有禹王臺,障遏水勢,會白馬河、沂河之水入駱馬湖,愈覺泛溢不可遏,應於臺舊基迎水處堵塞斷流,令仍由故道入海。」下扈從諸臣確議。惟駱馬湖減壩用玉書等原議,餘如新命言。

三十二年,直隸運河決通州李家口等五口,天津耍兒渡等八口。衛河微弱,惟恃漳為灌輸,由館陶分流濟運。明隆、萬間,漳北徙入滏陽河,館陶之流遂絕。至是三十六年,忽分流,仍由館陶入衛濟運。三十八年,廷議改高郵減壩及茆家園等六壩均為滾水壩,增加高堰石工五尺。三十九年,上以清口日淤,恐誤糧艘,海道運津又極艱險,擬以沙船載糧,自江下海,至黃河入海之口,運入中河,則海運不遠。下河督張鵬翮籌議。鵬翮言運河決口已塞,清水又已引出,糧船當可暢達。若改載沙船,雇募水手,徒滋糜費。且由江入海,從黃河海口入中河,風濤不測,實屬難行。從之。初,河督於成龍以中河南逼黃河,難以築堤,乃自桃源盛家道口至清河,棄中河下段,改鑿六十里,名曰新中河。至是,鵬翮見新中河淺狹,且盛家道口河頭灣曲,輓運不順,因於三義壩築攔河堤,截用舊中河上段、新中河下段合為一河,重加修濬,運道稱便。

四十年,以湖口清水已出,宜籌節宣之法,允鵬翮請,於張福口、裴家場二引河間,再開引河一,合力敵黃。若黃漲在糧艘已過,堵攔黃壩,使不得倒灌;漲在行船時,閉裴家場引河口,引清水入三汊河至文華寺濟運。是歲建中河口南岸石閘。四十二年,以仲莊徬清水出口,逼溜南趨,致礙運道,詔移中河運口於楊家莊,即大清水故道,由是漕鹽兩利。逾年,又命建直隸運河楊村減壩以分水勢。

四十四年,上言高堰及運河減壩不開放,則危及堤堰,開洩又潦傷隴畝,宜於高堰三滾壩下挑河築堤,束水入高郵、邵伯諸湖,其減壩下亦挑河築堤,束水由串場溪注白駒、丁溪、草堰諸河入海。令江、漕、河各督勘估,遣官督修。自是淮、揚各郡悉免漫溢之患。四十五年,鵬翮於中河橫堤建草壩二,鮑家營引河處建草壩一,相機啟閉,免中河淤墊。又以運河水漲,堤岸難容,於文華寺建石閘,閘下開引河,自楊家廟、單楊口迄白馬湖,長萬四千八百丈有奇,水漲開放入湖,水涸堵閉。是年,濟寧道張伯行請引漳自成安柏寺營通漳之新河,接館陶之沙河,古所謂馬頰河者,疏其淤塞,使暢流入衛。議未及行。越二年,全漳入館陶,漳、衛合而勢悍急,恩、德當沖受害,乃於德州哨馬營、恩縣四女寺建壩,開支河以殺其勢。

六十年,東撫李樹德請開彭口新河。先是濟寧道某言,彭口一帶有昭陽、微山、西湖,噴沙積於三洞橋內,屢開屢塞,阻滯糧艘,應挑新河、避噴沙,以疏運道。至是,樹德以為言。上曰:「山東運河,自西湖之水流入。前此百姓以為宜開即開,以為宜閉亦閉。開者何意?堵者何意?務悉其故,方可定其開否。不然,虛耗矣。」又曰:「山東運河,全賴湖、泉濟運。今多開稻田,截上流以資灌溉,湖水自然無所蓄瀦,安能濟運?往年東民欲開新河,朕恐下流泛濫,禁而弗許。今又請開新河。此地一面為微山湖,一面為嶧縣諸山,更從何處開鑿耶?張鵬翮到東,將此旨詳諭巡撫,申飭地方,相度泉源,蓄積湖水,俾漕運無誤,自易易耳。」

雍正元年,河督齊蘇勒偕漕督張大有言:「山東蓄水濟運,有南旺、馬踏、蜀山、安山、馬場、昭陽、獨山、微山、郗山等湖,水漲則引河水入湖,涸則引湖水入槽,隨時收蓄,接應運河,古人名曰『水櫃』。歷年既久,昭陽、安山、南旺多為居民占種私墾。現除已成田不追外,餘俟水落丈量,樹立封界,永禁侵占,設法收蓄。至馬踏、蜀山、馬場、南陽諸湖,原有斗門閘座,加以土壩,可收蓄深廣,備來年濟運之資。惟獨山一湖,濱臨運河,一線小堰,且多缺口。相度水勢,河水盛漲,聽其灌入湖中;湖、河平,即築堰堵截;河水稍落,不使湖水走洩涓滴。或遇運河淺塞,則引湖水下注,庶幾接濟便捷。至諸湖閘座,仍照舊例,灌塘積水,啟閉以時,則湖水深廣,運道疏通矣。」下所司議行。

二年,齊蘇勒以駱馬湖東岸低窪易洩,舊壩不足抵御,於湖東陸塘河通寧橋西高地築攔河滾壩,再築攔水堤六百丈,口門寬三十丈,以便宣洩。又幫築運河西岸地洞口堤身五百十丈,高、寶、江東西岸堤工五千二十四丈,寶應西堤七里閘迤南至柳園頭埽工五百七十丈。

四年,齊蘇勒改種家渡南之舊彭口於十字河,而彭口沙壅積如故。先是侍郎蔣陳錫疏陳漕運事宜,上命內閣學士何國宗等勘視豫東運道,至是覆稱:「山東運河必賴湖水接濟,請將安山湖開濬築堤;南旺、馬踏諸堤及關家壩俱加高培厚,建石閘以時啟閉;其分水口兩岸沙山下,各築束水壩一;汶水南戴村壩應加修築;建坎河石壩於汶水北;恩縣四女寺應建挑壩一;專平運河西岸修復進水關二,東岸建滾壩一;濮州沙河會趙王河處,舊有土壩引河,應修築開濬,其河西州縣,聽民開通水道,匯入沙河,於運道民生,均有裨益;武城及恩縣北岸,各挑引河一。河南運河自北泉而下,歷仁、義、禮、智、信五閘,遏水旁注,愚民不無截流盜水之弊。請拆去五閘,於泉池南口建石堰一,開口門三,分為三渠,築小堤使無旁洩;東西各開一渠,渠各建五閘,分溉民田。小丹河自清化鎮下應開濬築小堤,河東一里開水塘一,石閘三,分為三渠,以小丹河為官渠,東西各一為民渠。其洹河石壩皆已湮廢,宜增修為挑壩。諸泉源應各開深廣,入衛濟運。」下所司議行。五年,東撫塞楞額以柳長河日見淤淺,雖一帶相連,而中有金錢嶺分隔,特開引河二,一從嶺北注安山入湖,一從嶺南出閘口濟運。

八年,河督嵇曾筠言:「宿遷駱馬湖舊有十字河口門,引湖濟運,兼以刷黃。嗣湖水微弱,恐黃倒灌,堵閉河口,又於西寧橋迤西建攔湖壩,因是湖水不通,專資黃濟運,致中河之水挾沙淤墊。今秋山水暴漲,去路遏塞,漫溢橫出。請復十字河舊口門,俾湖水入中河,刷深運道,攔湖壩酌量開寬,俾上游之水,由六塘河入海。」從之。是年始設黃、運兩岸守堤堡夫,二里一堡,堡設夫二,住堤巡守,遠近互為聲援。

九年,兼總河田文鏡言:「汶南流濟運,向有玲瓏及亂石、滾水三壩。伏秋盛漲,水由滾壩入鹽河,沙由玲瓏、亂石洞隙隨水滾瀉。自何國宗於三壩內增建石壩,涓滴不通,既無尾閭洩水,又無罅隙通淤,致汶挾沙入運,淤積日高。請改壩為徬,建磯心五十六,中留水門五十五,安徬板以資宣洩。又以不能啟閉,別築土堤,名春秋壩。」如所請行。十一年,東撫岳濬言:「東省水櫃,舊有東平之安山湖廢閘四。自國宗議復安山湖水櫃,重築臨河及圈湖堤,修通湖、蛇溝二閘,並於八里灣、十里鋪兩廢閘間建石閘一,曰安濟閘,俱經修竣,仍不能蓄水濟運。緣湖底土疏,非圈堤所能收蓄,均宜修防。其圈湖堤缺,概停補築,以免糜費。」從之。十二年,直督李衛以故城與山東德州、武城毗連,系河流東注轉灣處,向無堤墊,水漲漫溢,勸諭民間儹修土墊,量給食米,以工代賑。東撫岳濬以德州河溜頂沖,於東岸挑新河、建滾壩,兩岸各築遙堤,酌開涵洞,以資宣洩。

乾隆二年,御史馬起元言:「直、東運河,近多淤塞。」尚書來保言:「衛水濟運灌田,請飭詳查地勢,使漕運不阻,民田亦資灌溉。」上命侍郎趙殿最、侍衛安寧,會同直、漕、河三督,豫、東兩撫勘奏。經部議:「東省泉源四百三十九,無不疏通,閘壩亦完固,惟戴廟、七級、柳林、新店、師莊、棗林、萬年、頓莊各閘,或雁翅潮蟄,或面石裂縫,兩岸斗門涵洞,有滿家三空橋雁翅低陷,石閘面太低,應交河督興修。又馬踏、蜀山、馬場、獨山、微山諸湖,嚴禁占種蘆葦,南旺、南陽、昭陽諸湖水櫃,僅堪洩水,小清河久淤塞,均宜次第修治。至衛水濟運灌田,宜於館陶、臨清各立水則一,測驗淺深,以時啟閉。」起元又言,通州至天津河路多淤淺,糧艘不便。命殿最偕顧琮勘議。尋議天津溯流而上,設有兵弁,無官管轄。應增置漕運通判一,駐張家灣,專司疏濬;把總二,外委四,聽通判調遣。又普濟寺等四閘屬通州,增置吏目一,慶豐等七閘屬大興,增置主簿一,遇應開挑處,報坐糧覈實修濬。用鄂爾泰言,建獨流東岸滾壩,並開引河,注之中塘窪,以免靜海有羨溢之虞,並減天津三汊口爭流之勢。是歲,大挑淮、揚運河,自運口至瓜洲三百餘里。

三年,河督白鍾山言:「衛河水勢,惟在相機啟閉。殿最前奏設館陶、臨清二水閘,可不必立。嗣雨水調勻,百泉各渠閘照舊官民分用。儻值水淺澀,即暫閉民渠民閘以利漕運。或河水充暢,漕艘早過,官渠官閘亦酌量下板以灌民田。」是年,修復三教堂減壩,挑濬淤填支河,使洩水入馬頰河。又於三空橋舊址修減壩,仍挑通支河,使洩水入徒駭河。增建裴家口東南涵洞二,修築房家口上下堤岸、馬家閘土堤,及自嶧縣臺莊迄臨清板閘運堤八百里纖道,亦資障護瀕河田廬。

先是疏濬毛城鋪河道時,高斌以黃流倒灌,移運口於上游七十餘丈,與三汊河接。次年,黃仍灌運,論者多謂新開運口所致,特命大學士鄂爾泰相度。旋言:「運口直對清口,湖水由裴家場引河東北直趨清口,入運之水仍系回流平緩;惟新口外挑水壩稍短,清水盛旺,或恐溜寬,宜再築長壩,不必仍舊開口。惟舊河直捷,新河紆曲,今新建閘壩未開,漕船應行舊河,以利挽運。新河於天妃閘下重建通濟、福興二閘,隨時啟閉。每歲漕船過後,河水充溢,則開放新河以分水勢,湖水漲溢,則閉舊河及新河閘以待水消,庶新舊兩河可以交用。」

鄂爾泰又言:「詳勘漳河故道,一自直隸魏縣北,經山東丘縣城西,至效口村會滏陽河,入大陸澤,下會子牙河,由天津入海。一由魏縣北老沙河,自潘爾莊經丘縣城東,歷清和、武城、景州、阜城各地,過千頃窪,入運歸海。丘縣城西故道去衛河較遠,舊跡既淹,開通匪易。且滏陽河下會子牙河,全漳之水亦難容納。惟老沙河即古馬頰河,河形寬闊,於此挑復故道,自和爾寨村東承漳河北折之勢,開至漳洞村,歸入舊河,勢順工省。即於新挑河頭下東流入衛處建閘,如衛水微弱,則啟以濟運,衛水足用,則閉閘使歸故道;再於青縣下酌建閘壩,臨清以北運道可免淤墊,青縣以下田廬永無浸淹。應飭直、東兩省會勘估修。」五年,改山東管河道為運河道,專司蓄洩疏濬閘壩事,仍管河庫,從白鍾山請也。

二十二年,添建高郵東堤石壩,酌定水則,視水勢大小以為啟閉。巡漕給事中海明言:「江南運河,惟桃源之古城砂礓,溜灘灣沙積,黃河以南,惟揚州之灣頭閘至範公祠三千三百餘丈間段阻淺,均應挑濬。鎮江至丹徒、常州,水本無源,恃江潮灌注,冬春潮小則淺。加以每日潮汐易淤,兩岸土松易卸,應六年大挑一次,否則三年亦須擇段撈淺。丹徒兩閘以下,常州之武進等縣,亦間段淺滯,均應一律挑濬。」詔:「挑河易滋浮冒,宜往來查察,毋得屬之委員。」

二十四年,命海明及河督張師載、東撫阿爾泰會勘直、東運河。初,運河水漲,漫溢德州等處,景州一帶道路淤阻。至是,海明等言:「漳、衛二河,伏秋盛漲,宜旁加疏洩。自臨清至恩縣四女寺二百五十餘里,河身盤曲,臨清塔灣東岸原有沙河一,即黃河遺跡,由清平、德州、高唐入馬頰河歸海。請開挑作滾水石壩,使汶、衛合流,分洩水勢。四女寺、哨馬營兩支河,原系旁洩汶、衛歸海之路,請將狹處展寬,以免下游德州等處沖溢。」二十五年,巡漕給事中耀海偕師載言:「南旺以北僅馬踏一湖,水患不足。獨山湖有金線閘,水祗南流,利濟閘水可北注。請移金線閘於柳林閘北,使獨山諸湖全注北運河。」制可。二十七年,以魚臺辛莊橋北舊有洩水口二,口門刷深,難以節制,允師載等請改建滾壩一。是歲,挑德州西方菴對岸引河,自魏家莊至新河頭,長四十丈,建築齊家莊挑溜埽壩,接築清口東西壩,修李家務石閘。二十八年,用阿爾泰言,於臨清運河逼近村莊處開引河五,以分水勢。

三十三年,黃水入運,命大學士劉統勛等往開臨黃壩,以洩盛漲,並疏濬運河淤淺。三十七年,河督姚立德言:「泗河下流董家口向建石壩分洩,今泗水南趨,轉為石壩所累。請拆去,並展寬孟家橋舊石橋。」如所請行。五十年,命大學士阿桂履勘河工。阿桂言:「臣初到此間,詢商薩載、李奉翰及河上員弁,多主引黃灌湖之說。本年湖水極小,不但黃絕清弱,至六月以後,竟至清水涓滴無出,又值黃水盛漲,倒灌入運,直達淮、揚。計惟有借已灌之黃水以送回空,蓄積弱之清水以濟重運。查本年二進糧艘行入淮河,全藉黃水浮送,方能過淮渡黃,則回空時雖值黃水消落,而空船吃水無多,設法調劑,似可銜尾遄行。」借黃濟運,自此始也。五十一年,運河盛漲,致淮安迤下東岸涇河洩水石閘墻蟄底翻,難資啟閉。越五年,山陽、寶應士民修復之。

嘉慶元年,河決豐汛,刷開南運河佘家莊堤,由豐、沛北注金鄉、魚臺,漾入微山、昭陽各湖,穿入運河,漫溢兩岸。是冬,漫口塞,凌汛復蟄陷。次年,東西兩壩並蟄,二月工始竣。自豐工決後,若曹工、睢工、衡工,幾於無歲不決。九年,因山東運河淺塞,大加濬治;又預蓄微山諸湖水為利運資。然自是以後,黃高於清,漕艘轉資黃水浮送,淤沙日積,利一而害百矣。十二年,倉場侍郎德文等請挑修張家灣正河,堵築康家溝以復運道,御史賈允升請挑濬減河,均下直督溫承惠勘辦。承惠請濬溫榆河上游。上命侍郎托津、英和偕德文等覆勘。尋奏言:「頻年漕運皆藉溫榆下游倒漾之水,以致泥沙淤積。若從上游深挑,直抵石壩,實為因勢利導。惟地勢高下,須逐細測量,俾全河毫無滯礙方善。」制可。

十三年,通州大水,康家溝壩沖決成河,張家灣河道遂淤。倉場侍郎達慶請來年糧艘由康家溝試行一年,暫緩挑復張家灣河身。上命尚書吳璥往勘,與達慶議合,遂允之。明年,御史史祜言,康家溝河道難行,請復張家灣正河。下直督溫承惠。承惠言:「康家溝溜勢奔騰,漕船逆流而上,大費纖輓。該處地勢正高,恐旱乾之歲,河水一瀉無餘,漕行更為棘手。惟張家灣兩岸沙灘,壩基難立,而正河積淤日久,挑濬亦甚不易。」上復遣工部尚書戴均元往勘,亦言壩基難立,且時日已迫,恐河道未復,漕運已來,請仍由康家溝行,再察看一年酌定。如所請行。時淮、揚運河三百餘里淺阻,兩淮鹽政阿克當阿請俟九月內漕船過竣,堵閉清江三壩,築壩斷流,自清江至瓜洲分段挑濬。下部議。覆稱:「近年運河淺阻,固由疊次漫口,而漫口之故,則由黃水倒灌,倒灌之故,則由河底墊高,清水頂阻,不能不借黃濟運,以致積淤潰決,百病叢生。是運河為受病之地,而非致病之原。果使清得暢出敵黃,並分流濟運,則運口內新淤不得停留,舊淤並可刷滌。若不除倒灌之根,而亟亟以挑濬運河為事,恐旋挑旋淤,運河之挑濬愈深,倒灌之勢愈猛,決堤吸溜,為患滋多。」命尚書托津等偕河督勘辦。十八年,漕督阮元以邳、宿運河閘少,水淺沙停,請于匯澤閘上下添建二閘。下江督百齡核奏。

道光元年,山東河湖山水並發,戴村壩迤北堤墊漫決六十餘丈,草工刷三十餘丈,四女寺支河南岸汶水旁洩處三。用巡撫姚祖同言,於正河旁舊河形內抽溝導水濟運,兼顧湖瀦。三年,漫直隸王家莊,由各汛賠修。是歲添築戴村壩北官堤碎石壩四。四年,侍講學士潘錫恩陳借黃濟運之弊,略言:「蓄清敵黃,為相傳成法。今年張文浩遲堵禦黃壩,致倒灌停淤,釀成巨患。若更引黃入運,河道淤滿,處處壅溢,恐有決口之患。」下尚書文孚等妥議。

自嘉慶之季,黃河屢決,致運河淤墊日甚,而歷年借黃濟運,議者亦知非計,於是有籌及海運者。五年,上因漕督魏元煜等籌議海運,群以窒礙難行,獨大學士英和有通籌漕、河全局,暫雇海船以分滯運,酌折漕額以資治河之議,下所司及督撫悉心籌畫。卒以黃、運兩河受病已深,非旦夕所能疏治,詔於明年暫行海運一次。

新授兩江總督琦善言:「臣抵清江,即赴運河及濟運、束清各壩逐加履勘。自借黃濟運以來,運河底高一丈數尺,兩灘積淤寬厚,中泓如線。向來河面寬三四十丈者,今只寬十丈至五六丈不等,河底深丈五六尺者,今只存水三四尺,並有深不及五寸者。舟只在在膠淺,進退俱難。濟運壩所蓄湖水雖漸滋長,水頭下注不過三寸,未能暢注。淮安三十餘里皆然,高、寶以上之運河全賴湖水,其情大可想見。請飭河、漕二臣將河面淤墊處展挑寬深,再放湖水,藉資輓送,以期不誤北上期限。」上以「借黃濟運,原系權宜辦理,孫玉庭察看漕艘挽運艱難,不早陳奏變計,魏元煜舊任漕督,及與顏檢坐觀事機敗壞,隱忍不言,糜帑病民,是誠何心?令將運河淤墊一律挑深,費由玉庭、元煜、檢分賠。」琦善又言,自御黃壩堵閉,運河淤墊不復增高,而洪湖清水蓄至丈餘,各船可資浮送,不敢冒昧挑濬。工費至省在百萬外,玉庭等罄其所有,斷無如許家資。更可慮者,欲濬運河,必先堵束清壩,阻絕來源,而後可以涸底挑辦。現湖水下注湍急,束清壩外跌塘甚深,又系清水,不能掛淤閉氣。設正事興挑,而束清壩臌開,則工廢半途,費歸虛擲。請停止里、揚運河挑工,以免草率而節糜費。」允之。是年,築溫榆河上游果渠村壩埽。七年,東河總督張井、副總河潘錫恩請修復北運河劉老澗石滾壩、中河南纖堤、揚糧二東西纖堤及堤外石工,移建昭關壩。上遣英和等馳勘,乃定移昭關壩於其北三元宮之南,餘如所請行。

十一年,高郵湖河漫馬棚灣及十四堡,湖河連為一。江督陶澍請依嘉慶間故事,運河決口,重空糧艘均繞湖行。八月,十四堡塞。冬,馬棚灣塞。先是澍撫蘇時,以鎮江運河並無水源,祗恃江潮浮送,下練湖湮塞已久,移建黃泥閘於張官渡以當湖之下流,俾得擎托湖流,使之回漾,稍濟江潮之不逮,曾著成效。至十四年遷江督,復偕巡撫林則徐相度,於湖頂沖之黃金壩及東岡築兩重蓄水壩,培圩埂二千八百八十丈,使水得入湖。又建減水石壩二於湖之東堤,俾可宣洩暴漲。於入運處修復念七家古涵,以作水門,並建石閘以放水濟運。是冬工竣,由涵引水出,竟能倒漾上行數十里,軍船得銜尾而南。越二年,溜勢變遷,河形灣曲,復移建黃泥閘於迤上二百丈,改為正越二閘,中建磯心,並改張官渡迤下六十里呂城閘為正越二閘,以利漕行。十五年,移築囊沙引渠沙壩於西河漘外,以資收蓄,從東河總督吳邦慶請也。

十八年,運河淺阻,用河督慄毓美言,暫閉臨清閘,於閘外添築草壩九,節節擎蓄,於韓莊閘上硃姬莊迤南築攔河大壩一,俾上游各泉及運河南注之水,並攔入微山湖。定收瀦濟運章程六。十九年,毓美以戴村壩卑矮,致汶水多旁洩,照舊制增高之。初,給事中成觀言淮、揚芒稻閘、人字河不宜堵壩,阻水去路,下陶澍等議。至是覆稱:「此壩蓄水由來已久,並不攔阻眾水歸江,不得輕議更張。」從之。時衛河淺澀,難以濟運。東撫經額布請變更三日濟運、一日灌田例。詔將百門泉、小丹河各官渠官閘一律暢開,暫避民渠民閘,如有賣水阻運盜挖情弊,即行嚴懲。明年,漕督硃澍復言:「衛河不能下注,有妨運道。」命河督文沖、豫撫牛鑒察勘。文沖等言:「衛河需水之際,正民田待溉之時。民以食為天,斷不能視田禾之枯槁置之不問。嗣後如雨澤愆期,衛河微弱,船行稍遲,毋庸變通舊章。倘天時亢旱,糧船阻滯日久,是漕運尤重於民田,應暫閉民渠民閘,以利漕運。」從之。

咸豐元年,甘泉閘河撐堤潰塌三十餘丈,河決豐縣,山東被淹,運河漫水,漕艘改由湖陂行。先是戶部尚書孫瑞珍言十字河為全漕之害,若於河西改寬新河,以舊河為囊沙,於彭口作滾壩,納濁水而漾清流,漕船無阻,可省起剝費二十萬。下東河總督顏以燠議。至是以燠言:「改挖新河事無把握,無庸輕議更張。」報聞。二年,決北運河北寺莊堤,命尚書賈楨、侍郎李鈞勘堵,並改次年漕糧由海道運津。自是遂以海運為常。同治而後,更以輪舶由海轉運,費省而程速,雖分江北漕糧試行河運,然分者什一,藉保運道而已。五年,銅瓦廂河決,穿運而東,堤墊沖潰。時軍事正棘,僅堵築張秋以北兩岸缺口。民墊殘缺處,先作裹頭護埽,黃流倒漾處築壩收束,未遑他顧也。十年,決淮揚馬棚灣。

同治五年,決清水潭。八年,河決蘭陽,漫水下注,運河堤墊殘缺更甚。自張秋以北,別無來源,歷年惟借黃濟運而已。九年,漕督張之萬請於黃流穿運處堅築南北兩堤,酌留運口為漕船出入門戶,並築草壩,平時堵閉以免倒灌。已下所司議,之萬旋改撫江蘇,繼任張兆棟以「既築堤束水留口門,又築壩堵閉,恐過水稍滯,而上游一氣奔注,新築堤閘難當沖激。設奪運北趨,則東昌、臨清暨天津、河間,淹沒在所必至,北路衛河亦將廢壞。惟有於鄆城沮河一帶遏黃東流,即以保南路之運道,於張秋、八里廟等處疏運河之淤墊,即以通北上之漕行,較之築堤束水,稍有實際」。制可。

十年,侯家林河決,直注南陽、昭陽等湖,鄆城幾為澤國。漕督蘇鳳文言:「安山以北,運河全賴汶水分流,至臨清以上,始得衛水之助。今黃河橫亙於中,挾汶東下,安山以北毫無來源,應於衛河入運及張秋清黃相接處,各建一閘,蓄高衛水,使之南行,俟漕船過齊,即啟臨清新閘,仍放衛北流,以資浮送。並於張秋淤高處挑深丈餘,安山以南亦一律挑濬,庶黃水未漲以前,運河既深,舟行自易。」江督曾國籓言:「河運處處艱阻,如嶧縣大泛口沙淤停積,水深不及二尺,必須挑深四五尺,並將近灘石堆劃除,與河底配平,方利行駛。北則滕縣郗山口入湖要道,淺而且窄,微山湖之王家樓、滿家口、安家口,獨山湖之利建閘,南陽湖北之新店閘、華家淺、石佛閘,南旺閘分水龍王廟北之劉老口、袁口閘,處處淤淺,或數十丈至百餘丈,須一律挑深。此未渡黃以前,阻滯之宜預為籌辦者。至黃水穿運處,漸徙而南,自安山至八里廟五十五里運堤,盡被黃水沖壞,而十里鋪、姜家莊、道人橋均極淤淺,宜一面疏濬,一面於缺口排釘木椿,貫以巨索,俾船過有所依傍牽挽。此渡黃時運道艱滯,宜預為籌辦者。渡黃以後,自張秋至臨河二百餘里,河身有高下,須開挖相等,於黃漲未落時,閉閘蓄水,以免消耗,或就平水南閘迤東築挑壩,引黃入運。此渡黃後運道易涸,宜預為籌辦者。東平運河之西有鹽河,為東省鹽船經行要道。若漕船由安山左近入鹽河,至八里廟仍歸運道,計程百餘里,較之徑渡黃流,上有缺口大溜,下有亂石樹舂者,難易懸殊。如行抵安山,遇黃流過猛,宜變通改道,須先勘明立標為志。此又渡黃改道,宜預為籌辦者。」下河、漕督及東撫商籌。

十一年,河督喬松年請在張秋立閘,借黃濟運。同知蔣作錦則議導衛濟運。上詢之直督李鴻章,鴻章言:「當年清口淤墊,即借黃濟運之病。今張秋河寬僅數丈,若引重濁之黃以閘壩節宣用之,水勢抬高,其淤倍速。至作錦導衛,原因張秋北無清水灌運,故為此議。以全淮之強,不能敵黃,尚致倒灌停淤,豈一清淺之衛,遂能禦黃濟運耶?其意蓋襲取山東諸水濟運之法。不知泰山之陽,水皆西流,因勢利導,百八十泉之水,源旺派多,自足濟運。衛水微弱,北流最順,今必屈曲使之南行,一水兩分,勢多不便。若分沁入衛以助其源,沁水猛濁,一發難收,昔人已有明戒。近世治河兼言利運,遂致兩難,卒無長策。事窮則變,變則通。今沿海數千里,洋舶駢集,為千古以來創局,正不妨借海道轉輸,由水扈解津,較為便速。」疏入,詔江、安糧道漕米年約十萬石仍由河運,餘仍由海運。光緒三年,東撫李元華條上運河上中下三等辦法,並言量東省財力,擬用中等,將北運河一律疏通,復還舊址,並建築北閘。時值年荒,寓賑於工,省而又省,需費三十萬有奇。下所司議。

五年,有請復河運者。江督沈葆楨言:「以大勢言之,前人之於河運,皆萬不得已而後出此者也。漢、唐都長安,宋都汴梁,舍河運無他策。然屢經險阻,官民交困,卒以中道建倉,伺便轉餽,而後疏失差少。元則專行海運,故終元世無河患。有明而後,汲汲於河運。遂不得不致力於河防。運甫定章,河忽改道。河流不時遷徙,漕政與為轉移,我朝因之。前督臣創為海運之說,漕政於窮無復之之時,藉以維持不敝。議者謂運河貫通南北,漕艘藉資轉達,兼以保衛民田,意謂運道存則水利亦存,運道廢則水利亦廢。臣以為舍運道而言水利易,兼運道而籌水利難。民田於運道勢不兩立。兼旬不雨,民欲啟涵洞以溉田,官必閉涵洞以養船。迨運河水溢,官又開閘壩以保堤,堤下民田立成巨浸,農事益不可問。議者又太息經費之無措,舳艫之不備,以致河運無成。臣以為即使道光間歲修之銀與官造之船,至今一一俱存,以行漕於借黃濟運之河,未見其可也。近年江北所雇船隻,不及從前糧艘之半,然必俟黃流汛漲,竭千百勇夫之力以挽之,過數十船而淤復積。今日所淤,必甚於去日,而今朝所費,無益於明朝。即使船大且多,何所施其技乎?近因西北連年亢旱,黃河來源不旺,遂乃狎而玩之。物極必返,設因濟運而奪溜,北趨則畿輔受其害,南趨則淮、徐受其害,如民生何?如國計何?」

八年,伏秋大汛,張家灣運河自蘇莊至姚辛莊沖開新河一段,長七百餘丈,上下口均與舊河接,形勢順直,大溜循之而下。舊河上口至下口,長六千四百餘丈,業已斷流,惟新河身系自行沖開,不能一律深通。明年,直督李鴻章飭制新式鐵口刮泥大板,在兩岸拖拉,使一律通暢。十二年,通州潮白河之平家甿漫口,東趨入箭桿河。未幾,堵復運河故道。十三年六月,復漫刷平家甿新工下之北市莊東小堤,並老堤續塌百數十丈,連成一口,奪溜東趨十之八。尋堵塞之。是年,河決鄭州,山東黃水斷流,漕船不能南下,向之借黃濟運者,至是束手無策。旋將臨口積淤疏挑,空船始得由黃入運。十五年,東撫張曜言:「河運未能久停,請改海運漕米二十萬仍歸河運。」從之。

十六年,用江督曾國荃言,修揚屬南運河堤閘涵洞,及附城附鎮專工。又用漕督松椿言,濬邳、宿運河。十九年,潮白河漲溢,運堤兩岸決口七十餘,上游務關決口七。是冬均塞。二十年,濬濟寧、汶上、滕、嶧、茌平、陽穀、東平各屬運河。明年,濬陶城埠至臨清運河二百餘里。二十四年,侍讀學士瑞洵言南漕改折,有益無損,請每年提折價在津購米以實倉庾。御史秦夔揚亦言河漕勞費太甚,請停江北河運。皆不許,仍飭認真疏濬,照常起運。二十六年,聯軍入京師,各倉被占踞,倉儲粒米無存,江北河運行至德州,改由陸路運送山、陜。二十七年,慶親王奕劻、大學士李鴻章言:「漕糧儲積,關於運務者半,因時制宜,請詔各省漕糧全改折色,其採買運解收放儲備各事,分飭漕臣倉臣籌辦。」自是河運遂廢,而運河水利亦由各省分籌矣。


\end{pinyinscope}