\article{志一百十三}

\begin{pinyinscope}
○兵九

△海防

國初海防,僅備海盜而已。自道光中海禁大開,形勢一變,海防益重。海防向分南北洋。山東煙臺歸北洋兼轄。閩、浙、粵三口,歸南洋兼轄。茲取沿海各省有海防者分述之:曰東三省,曰直隸,曰山東,曰江南,附江防,曰浙江,曰福建,曰廣東。

奉天沿海,南自牛莊至金、蓋各州,轉東至鴨綠江口,西則自山海關至錦州,地皆濱海,口岸凡三十九處。康熙初,廷議錦州一帶籌備沿邊。旋定金州旅順口設水師戰船,隸金州副都統督率訓練,戰船皆木質舊式。雍正四年,將軍噶爾弼以自旅順海口至鳳凰城,水程千里,僅恃旅順水師一營,議增二營,聯絡巡哨。道光二十一年,耆英以奉省海防經營不易,有移民內徙之議,海防漸重。咸豐之季,歐艦北犯津、沽,奉天亦嚴海防。同治四年,崇厚調天津洋槍隊千人赴營口屯駐。五年,以奉天兵船拙重,調天津輕利兵船十餘艘赴長島駐防。復增新練洋槍隊五百人於營口。十一年,瑞麟以南洋自制兵艦告成,砲械咸備,乃商撥兵艦一艘,巡防牛莊海口。光緒初年,以俄羅斯有窺北邊,沿海亦有俄艦游弋,乃於制兵外加餉練兵,凡選練馬步隊四千二百餘人。又增綠營兵四千人,調撥吉林、黑龍江、蒙古馬隊各二百餘人駐營口,與宋慶豫軍協同防守。其東邊道之練軍馬步隊一千三百人,則分駐鳳凰城、大孤山、北河、長甸河口及安東等處。額設正兵,幾同虛設,海上有警,全恃客軍。金州與海參威毗連之處,尤為重要。李鴻章遣鎮東等四砲艦巡防奉省海口。八年,鴻章以北洋迤東口岸,惟奉天旅順口為首沖,乃在旅順之黃金山頂,仿築德國新式砲臺,設巨砲多尊,並建築兵房、子藥庫,近山要路,復設行營砲壘,海口內則布置水雷,沿海岸可登陸處,擇要埋藏地雷,陸路則有護軍營八哨,毅軍十一營,水路則有快砲船、蚊砲船各二艘,表裏依護。其次為營口,海灘平衍,敵易抄襲,復調勁旅接應後路。十年,將軍定安於營口創設水雷營,電線火藥,建雷庫十間存儲。十七年,李鴻章以大連灣為渤海門戶,築老龍頭等處砲臺六座,仿西洋曲折式,兵房、藥庫皆備。二十六年,將軍增祺以岫巖、安東沿海,雖有北洋兵艦巡防,而海濱港汊紛歧,乃增造大號水師船八艘,布列於沙河、大孤山、太平溝等處。

至吉林、黑龍江之海防,以有松花、黑龍二江,貫省境而趨海,舊制二省各設水師營巡防,水師船止運船三十艘,槳船二十艘,每為運糧及採東珠、取樺皮之用,亦稱水手營,非戰艦也。

吉林海防,首重琿春。松花江雖可行海舶,而江水淺處為多。同治四年以後,屢有俄羅斯兵船,乘江水漲時,駛入至阿勒楚喀及伯都訥境內。將軍岐元擬於三姓設水師營,不果。光緒六年,府丞王家璧有整頓東省水師改造戰艦之議。李鴻章以長江水師船不適用於松花、黑龍二江,宜於吉林、三姓左近,擇水深溜大之處設船廠,造小號兵輪船,如廣東蚊砲船之式,入水不深,上可行駛伯都訥、省城附近,下可巡行黑河口,轉入黑龍江,以佐陸軍,備俄船侵入。並撥開花砲、新式馬槍、快槍等,為吉省練兵之用。時將軍銘安、及督辦寧古塔等處防務吳大澂、喜昌,以俄患未平,於吉省沿江沿邊,增練防軍馬步隊五千人,各旗及西丹又募千五百人,練成即赴琿春駐守,並設護江關,防範水路。又慮俄國海軍船堅力猛,水關不能阻其沖突,乃擇要依山建築砲臺,以御俄艦。

黑龍江省於光緒三年始籌辦海防,通省額兵及西丹共一萬人,增鄂倫春兵五百人,兼習新式槍砲。黑省近俄,俄人環黑龍江左岸盤踞,達二千餘里,每相間百餘里,輒有俄兵屯駐之所,刁鬥相聞。故黑省防務,重在陸而不在海。其江流入海之口,在省境東北隅,雖額設師船三十餘艘,僅循例操演。

東三省海防,奉天尤重。自日占旅、大,遼東半島籓籬盡撤。而吉、黑二省,向受俄患,北海屢警,防務益形棘手云。

直隸津、沽口,為南北運河、永定、大清、子牙五河入海處,北連遼東,有旅順、大連以為左翼,南走登、萊,有威海衛以為右翼,為北洋第一重鎮。順治初,天津巡撫雷興疏言,大沽海口為神京門戶,請置戰船以備海防。下所司議行。雍正四年,於海口蘆家嘴創設天津水師營,令滿洲兵丁駐扎,學習水師,特簡都統大員,守御海口。復自天津城南門外起至慶雲縣止,所有沿海各州縣,設立海撥二十五處,分置守兵,扼要防範。

乾隆四年,直隸總督高斌請拓天津水師營、汛,增駐滿兵一千,合舊額為三千。及道光六年,那彥成奏請裁撤海口官兵,改歸大名鎮。十二年,琦善奏天津地處海隅,與山東登州、奉天錦州遙相拱衛,沙線分歧,非熟習海徑者,無由曲折而至。且海口二十里外,有攔港沙一道,融結天成,儼若海口外衛。總兵陸路營伍,足資捍衛,所有天津水師,無庸復設。於是水師營遂裁。二十年,又據琦善奏英艦到粵,難保不分投竄擾,天津密邇京畿,尤宜慎重防堵。遂復嚴旨派員駐扎要隘,協同防禦。二十一年,天津海口增駐官兵,建砲臺營房,近海村落,招集團練,修築土堡,互為策應。二十二年,令直隸沿海營兵,善於鳧水及諳習風濤駕駛之技者,飭統兵官訓練,並增設巡哨兵船,以蘆臺為北塘後路,設通永鎮標十五營駐守。二十三年,令天津水師營每年撥戰船六艘,分三路巡防,與奉天、山東師船,定期會哨,以登州、岫巖城、錦州三處為呈票考驗之地。有畏避風浪,巡哨貽誤者,嚴懲之。三十年,令訥爾經額察視海防。

咸豐八年,令僧格林沁在大沽口及雙港修築砲臺,設水路木筏,及沿岸營壘,調宣化鎮兵會大沽協兵,守護海口砲臺。又令史榮椿等由天津赴山海關履勘海防要隘。同治元年,令曾國籓、薛煥等購買外洋兵船巨砲,統以鎮將,酌分數艘,駐泊天津海口。九年,山東巡撫丁寶楨以大沽、北塘等處為京師門戶,慮直隸兵力不足,調山東舊部十八營,赴直隸邊境候調。十年,直隸總督李鴻章增設大沽協海口六營,酌定營制。修築大沽口南北兩岸砲臺,與北塘相犄角,調遵化練軍千人移駐。十三年,又以北洋海防,僅恃大沽、北塘二海口砲臺,後路尚恐單薄,乃就運河北岸,用三合土建築新城,四圍設大小砲臺,護以金剛墻,引海河為城濠,屯駐重兵,與大沽防營相應。

光緒元年,李鴻章復於大沽、北塘、新城各處,增築洋式砲臺營壘,購置鐵甲快船、碰船、水雷船,以備攻守。二年,令總兵周盛傳率淮軍馬步二十餘營,建築新城砲臺。三年,成之。六年,李鴻章以北塘迤東至山海關,延長數百里,調宋慶、郭松林二軍,分駐沿海蒲河口、秦皇島等處,並增建砲臺。又以淮、練各軍駐天津,防守大沽、北塘各口。以鮑超全軍三十營駐昌黎、樂亭,防守大清河、洋河各口。以山海關防軍,兼顧金山嘴、秦皇島、老龍頭各處。時曾國荃建議直隸海防,不宜遠守營口,宜以重兵守山海關。乃命曾國荃統率安徽、湖北、山西各軍赴山海關駐守。八年,李鴻章於大沽、北塘砲臺下埋伏水雷,大沽口內設攔河木筏,山海關內外築三合土大砲臺一,土砲臺二,瀕海營墻,均仿砲臺建築。又寧海城臨海受敵,於砲臺墻外,悉以沙土掩護。其時大沽南北岸砲臺大小共數十座,輔以水雷鐵艦,沿岸以陸軍駐守。十一年,因京東沿海空虛,調練軍各營,移駐灤州、昌黎等處。二十三年,直隸總督王文韶以武毅軍訓練初成,率前後二軍及馬隊一軍,周歷山海關沿海一帶,以重防務。自歐艦來窺,僧格林沁戰敗,廷議始專津、沽之防。中日之役,旅順、威海相繼淪陷,而津海未開戰事。及拳匪肇釁,聯軍北犯,沽口砲臺,毀於一旦,北洋沿海防務,遂日形懈弛云。

山東海岸綿亙,自直隸界屈曲而南以達江蘇,其間大小海口二百餘處。東北境之登、萊、青三府,地形突出,三面臨海。威海、煙臺島嶼環羅,與朝鮮海峽對峙,為幽、薊屏籓。海禁既開,各國商帆戰艦,歷重洋而來,至山東成山而折入渤海,以達沽口。故創練海軍,以威海、旅順為根據地。欲守津、沽,先守威、旅。齊、魯關山,遂與畿疆並重矣。

順治十一年,令蘇利為水軍都督,駐軍碣石,為山東防海之始。乾隆五十五年,以膠州、文登、即墨等營,兼防海口,以總兵駐登州,統水師三營,戰船十二艘,修治各海口砲臺。道光二十一年,以芝罘島扼東海之口,撥兵防守。蓬萊、黃縣、榮城、寧海、掖縣、膠州、即墨所屬之十三島,編練民團,互為防衛。三十年,以瀕海之三汛師船,四縣水勇,合並防守海口,並扼要安設大砲。咸豐元年,登州總兵陳世忠以海寇奪掠官船,山東水師無多,會閩、粵大號師船,合力截捕。三年,於登、萊、青三府舉辦聯莊團練,給以兵械。八年,飭天津鎮總兵赴山東,詳勘海豐一帶海口。九年,以海豐縣之大沽河有防營故址,飭崇恩等撥兵防守。十年,文煜令青州等沿海各城滿、綠營兵,勤加訓練,分守城官兵之半以守海口。同治九年,丁寶楨以東境海口紛歧,惟有扼要防守。其文登縣屬之馬頭石島,福山縣屬之煙臺,蓬萊縣屬之廟島,掖縣屬之小石島,為洋船北來所必經,第居險要,共撥兵六千餘人分守。十一年,撥大號兵船一艘,駐泊登州洋面。光緒元年,丁寶楨以山東之東三府,三面環海,外寇隨處可登,宜扼要屯守。其要地有三:一、煙臺,於通申岡設防營,駐兵三千。煙臺山下及八蠟廟、芝罘島之西,共建浮鐵砲臺三座。芝罘島之東,築沙土曲折砲臺一座。一、威海衛,於劉公島之東口,建浮鐵砲臺一座,而於島口內築沙土曲折砲臺,於口外海面密布水雷,其北口內亦建沙土浮鐵砲臺,可作兵輪船水寨之用。一、登州,於城北建沙土高式砲臺,城內建沙土圓式砲臺。長山之西,建沙土曲折砲臺,與郡城相犄角。砲臺用克魯伯後膛大砲,參用阿姆司脫朗前膛大砲。兵丁用格林砲、克魯伯四磅砲、亨利馬悌尼快槍,請求行陣攻守之法。六年,以新購外洋蚊砲船駐防煙臺海面。十二年,許景澄建議山東膠州灣當南北洋之中,東為浮山,西為靈山,口狹而水深,宜規畫形勢,為新練海軍屯港,與旅順口東西相應。是年,李鴻章於威海衛南北岸築砲臺,布水雷。十七年,於威海黃泥巖增築新式砲臺,又於南岸龍廟嘴砲臺外,增築趙北口砲臺。劉公島新築地阱砲臺,設後膛巨砲於隧道。其西之黃島,水中之日島,亦設砲臺,與南岸相應。劉公島又設大鐵碼頭,為海軍寄椗上煤之所,防務益周密。

東省形勢,以威海、膠州為要口,於海軍屯泊尤宜。乃甲午一役,威海水陸之防,既毀於日本,而德因教案,曾以大隊鐵艦奪踞膠州灣,闢商埠,開鐵路,浸窺腹地。東省海防,遂無所藉手云。

江南海防,自海州南歷長江、吳淞江二口,稍折而西,至松江奉賢縣境之海灣,南接浙江洋面,其間港口羅列。惟江陰、吳淞二處,一為長江之筦鍵,一為蘇、松之門戶,防務尤重。至江陰以上,以江流深廣,外海兵艦商船,溯流而上,西達夔、渝,三千里流域,雖皖、贛、楚、蜀各有江防,實以江南當下游之沖。自狼、福山以迄京口、金陵,砲壘防營,星羅棋布。上游防衛,與下游繁簡迥殊。而江蘇轄境,長江千里,兵艦砲臺,無異海防,水陸營汛,亦與海疆聯絡。故安徽省以上江防,即隸於蘇省海防焉。

自海州南抵江口,乃昔年黃河入海處,泥沙積久,凝結內海,稱五條沙,海潮甚急。海舶北赴燕、齊者,必東行一晝夜以避其沙,故淮、海州郡,得稍寬海防者,以五條沙為之保障也。自狼、福山口南抵吳淞,沙凝潮急,略同北境。惟長江、吳淞二口,水深溜大,巨艦可直駛內江,故海口防務,視海濱倍重。

清初平定江南,分八旗勁旅駐京口,以鎮海大將軍統之,設水師營,造沙唬船以習水戰。旋以沙唬船難涉大洋,乃改造鳥船。時鄭成功據臺灣,以師船進窺江表,由京口薄金陵,梁化鳳擊敗之。順治十四年,命梁化鳳為水軍都督,率軍萬人,駐防崇明、吳淞。以松江府三面臨海,設提督,駐重兵。康熙六年,因崇明孤懸大海,嚴出海之禁。十四年,以提督統八營駐崇明。二十三年,減存四營,列汛六十有八。太倉州為元代海運放洋之口,明代置兵屯守,清初設游擊,以劉河營移駐茜涇鎮。雍正四年,分設寶山縣,列汛五十有七。上海縣當黃浦江之沖,原有墩臺十七座,康熙二年,以墩臺距海較遠,乃建外塘斥堠。其南為金山縣,踞青浦、南匯之上游,設參將駐守,列汛七十有八。常熟之福山,與隔江之狼山對峙,常熟、昭文瀕海之口,為許浦、徐陸涇、白第港,康熙間,設墩堡戍守,列汛二十有四。通州為狼山營汛地,如皋為掘港營汛地,皆近海要區也。其北境之海州,為南北襟要,海口之大者凡九,最北為荻水口,其東北雲臺山,清初曾徙民內地,阻塞入海之道,康熙二十年復開通,設通海營,列汛五十有五。淮安府昔為淮河入海之處,設廟灣、鹽城二營,會哨巡防,列汛四十有二。揚州府北之興化,南之泰州,為濱海之縣,清初設守備,康熙十一年,設游擊鎮之,列汛凡十。雍正八年,以福山營為江海門戶,於江蘇鎮標四營內分兵船二艘隸之,與狼山營會哨。此清初至雍正年江南之海防也。

乾隆至道光,江海清平,防汛率循舊制。及道光中葉,海警驟起,東南戒嚴。二十一年,以寶山海口為江南要區,屯駐大營,分設游緝之兵。吳淞亦屯兵,增設濠壘。二十二年,令耆英等周歷吳淞、狼山、福山、圌山關各處,整頓戰船砲械。二十三年,以江陰鵝鼻嘴為由海入江要口,設險守御。又防堵瓜洲及南河、灌河、射陽湖之口。令璧昌等察沿海城邑,聯絡保障。所用砲位,設局開鑄,並造水師舢板船,築砲臺於江岸南北。二十四年,璧昌因狼、福山江面太寬,於劉聞沙、東生洲、順江洲、沙圩等處,修築砲堤。水師各營,增大小戰船一百三十餘艘,分廠制造。二十七年,李星沅籌防泖湖,貯石沈船,增置木牌,並存儲砲位,分布重兵。而其時所築砲臺,實止因土為堤,且器械窳舊,布置多疏,非特不足禦歐洲巨艦,咸豐間,粵寇東下,沿江防戍,咸望風奔靡。及湘軍底定東南,軍勢始振。

同治元年,諭薛煥等購西洋兵艦,在上海等要口防守。四年,曾國籓於狼山鎮標,每營增造大舢板船二十號,仿紅單船之式,多設砲位,巡緝內洋。海門設綏海營,置大舢板船二十號,酌設兵輪,分防北岸海汊。七年,更定內洋水師五營,外洋水師六營之制。以兵輪四艘,分隸蘇松、狼山、福山三鎮總兵,駐防海口。九年,南洋初設兵輪統領,駕駛出洋,周歷島嶼。十三年,調陜防武毅軍馬步二十二營,赴山東、江南沿海適中之地駐防日本。時臺灣告警,李宗羲以蘇、松之門戶,吳淞為要,長江之關鍵,江陰為先,而鎮江府屬之焦山、象山,對岸之天都廟,江寧府屬之烏龍山,省城外之下關,均為扼要。以大木方石為基,搗三合土,築砲臺砲門,護以鐵柱鐵板,空其下以藏砲兵。先築烏龍山砲臺十六座,以次江陰、天都廟、象山、焦山、下關各築明暗砲臺,置巨砲。北岸之沙州圩、吳淞口,及江陰北岸之劉聞沙,亦一律增建砲臺,以嚴防務。

光緒元年,劉坤一於江陰鵝鼻嘴砲臺外,復於下游增築砲臺。其北岸之十圩港,亦增築砲臺,與南岸相犄角。又修改焦山、圌山關、烏龍山等處砲門,以期合法。五年,以外海兵輪統領駐吳淞口,凡沿海各省兵輪,悉歸調遣。七年,令彭玉麟籌辦江陰至吳淞口一帶海防。重修圌山關、東生洲兩岸舊築砲堤,並築營壘,置大砲。又改天都廟舊式砲臺為明砲臺。八年,左宗棠舉辦沿海漁團,選漁戶精壯者五千人隸吳淞鎮,給以糧械,隨時操練。彭玉麟以狼、福山為長江總口,長江下游雖修治砲臺,而江面空虛,鐵甲大戰艦無多,止有海防,未能海戰。議造鐵甲小兵艦十艘,專顧內洋,與砲臺相掩護。十年,令安徽疆臣籌備上游江防。乃於安慶城外,築明暗砲臺各一座,石營一座。攔江磯北岸,建明砲臺二座,石營一座,南岸建明砲臺、石營各一座。西梁山建明砲臺四座,石營一座,土營二座。東梁山就其形勢,築石城、砲堤各一道,以控制江面。十年,曾國荃以新購西洋十四口徑八百磅子大砲及開花子彈,分置江陰、吳淞二口砲臺。又購馬梯尼快槍二千枝,分給各營。又於吳淞砲臺增兵八營,江陰砲臺增兵十二營,扼守江海總路。十三年,又增建吳淞、江陰砲臺,以鐵木石土各料築成,各設新式後膛大砲,其旁佐以哈乞開司砲。江陰之四門大砲臺,分建於小角山、黃山二處。黃山舊砲臺所存之八十磅子後膛砲,移設於大石灣明砲臺。凡砲臺之門,各建砲房,護以三合土墻。又田雞砲為軍中利器,於江幹要隘,建砲房,置田雞砲,以資操練。二十二年,張之洞以江南各砲臺分為四路,南路獅子林、南石塘各臺為一路,南北岸各臺為一路,象山、焦山、圌山關、天都廟各臺為一路,江寧之獅子山、幕府山、鍾山、下關各臺為一路,設總管砲臺官四員,以新購外洋四十餘磅子快槍砲三十具分置各砲臺。二十五年,以長江水師兵力單弱,皖省防軍尤少,令沿江督撫,不分畛域,節節設防。

三十一年,以東南各省新軍,次第練成,命兵部侍郎鐵良至江南考察江海防務。旋鐵良覆陳江南之沿江海砲臺,分為四路,曰吳淞,曰江陰,曰鎮江,曰金陵。第一路吳淞砲臺,在寶山縣南,分設三臺,置前後膛大小砲三十四具,砲勇三百餘人,水旱雷營二哨,雷勇一百餘人,以盛字五營駐防。第二路江陰砲臺,在縣城北,於長江南北岸分設砲臺,南岸置前後膛大小砲三十七具,北岸置砲二十具,砲勇共四百餘人,水旱雷營三哨,雷勇二百餘人,以合字、南字等八營分兩岸駐守。第三路鎮江砲臺五處,曰圌山關,曰東生洲,曰象山,曰焦山,曰天都廟。南岸各臺置砲十五具,北岸各臺置砲六具,砲勇二百餘人,以新湘二旗駐防。溯江至鎮江府城,南岸象山,北岸天都廟,中流焦山,分設三臺,象山置砲十八具,焦山六具,天都廟九具,砲勇三百餘人,以武威六營、新湘三旗駐防。金陵城外砲臺七處,曰烏龍山,曰幕府山,曰下關,曰獅子山,曰富貴山,曰清涼山,曰雨花臺。烏龍山在省城外四十里,於南岸分設五臺,置砲十二具,砲勇一百餘人。幕府山在北門外,砲臺依次置砲七具,迤西老虎山置砲四具,砲勇一百餘人。下關砲臺在城外東面對岸,東岸置砲二具,西岸置砲十具,砲勇一百餘人。獅子山在城內,分設東西二臺,置砲八具,砲勇九十人。富貴山在鍾山之麓,置砲六具,砲勇四十餘人。清涼山在西門內,依城為砲臺,置砲二具,砲勇十四人。雨花臺在聚寶門外,置砲二具,砲勇十四人。

安徽省砲臺分為四路,曰東西梁山,曰攔江磯,曰前江,曰棋盤山。梁山夾江對峙,東臺置砲十四具,西臺十二具,以精銳營步兵三哨為砲兵。攔江磯砲臺在省城外四十里西岸,置砲十五具,以續備步隊中營駐臺為砲兵。前江口砲臺在上游十餘里,踞東岸高阜,分上下二臺,置砲十二具,由續備中營撥兵分駐。棋盤山砲臺在安慶東門外北岸,置大小砲六十八具,以步兵前營駐防。

江西省砲臺分為四路,曰馬當,曰湖口,曰金雞坡,曰嶽師門。馬當在彭澤縣東南岸,分設五臺,置砲五具,砲勇六十人。湖口砲臺在縣城北之東西岸,分設二臺,置砲十具,砲勇七十人。金雞坡砲臺在九江府十里外東西岸,分設三臺,列東西北三面,置砲十二具,砲勇二百人。嶽師門砲臺在九江東門外,分上下二臺,沿江岸建築,置砲二十一具,砲勇七十人。

湖北省砲臺,僅田家鎮一路,分中南北三臺,置砲三十一具,砲勇五十人。

自同治間,經營江海防務,歷四十餘年,始稱完密雲。

浙江東南境瀕海者,為杭、嘉、寧、紹、溫、臺六郡,凡一千三百餘里。南連閩嶠,北接蘇、松。自平湖、海鹽西南至錢塘江口,折而東南至定海、舟山,為內海之堂奧。自鎮海而南,歷寧波、溫、臺三府,直接閩境,東俯滄溟,皆外海。論防內海,則嘉興之乍浦、澉浦,海寧之洋山,杭州之鱉子門,紹興之沙門為要。論防外海,則定海縣與玉環皆孤峙大洋。定海為甬郡之屏籓,玉環為溫、臺之保障,尤屬浙防重地。定海之東,其遠勢羅列者,首為海中之馬跡山。山北屬江蘇境,山南屬浙江境,而五奎山亦為扼要。陳錢山則在馬跡之東北,山大而隩廣,可為舟師屯泊之所。迤南經岱山、普陀山,出落迦門,至東霍山,與陳錢山南北相為犄角。其南有昌國外之韭山,均可駐泊舟師。自寧波而南,內有佛頭、桃渚、松門、楚門諸山,外有茶盤、牛頭、積穀、石塘、大小鹿山,為溫、臺所屬水師會哨之所。由玉環而更南,歷漁山、三盤、鳳凰、北屺、南屺而至此關,則接閩省防地矣。

清初平定浙江後,沿明制嚴海防。順治八年,令寧波、溫州、臺州三府沿海居民內徙,以絕海盜之蹤。康熙二年,於沿海立椿界,增設墩堠臺寨,駐兵警備。四年,以欽差大臣巡視浙江海防。七年,命偕總督出巡沿海,直至福建邊境,提督則每年必巡歷各海口,增造巨艦,備戰守。二十九年,命江、浙二省疆臣,會勘轄境海面,分界巡哨,勒石於洋山,垂為定制。雍正五年,以提標之游擊、守備二員,統率兵丁,改隸水師。六年,定沿海商船漁船之帆檣符號,以別奸良,並增設汛弁。選福建之精練水兵至浙,教練浙軍十二營水戰諸務,巡游海口。七年,增建沿海要口臺,增設巡船,及防汛移駐之區,總兵官出巡之制。乾隆五十九年,以五奎山為浙洋扼要之地,撥定海標兵駐守。道光二十年,奇明保等以杭州之鱉子門,為錢塘通海要口,於潮神廟江狹之處,屯兵防守。二十一年,令沿海疆臣,仿定海土堡之法,凡近海村落,招募團練,築土堡,互相聯絡。三十年,以漁山孤懸海外,令黃巖鎮總兵以舟師靖盜。光緒六年,譚鍾麟以浙省沿海各口,巨艦之可深入者,距省最近為乍浦,次則寧波之鎮海、定海、石浦,臺州之海門,溫州之黃華關,舊有砲臺三十餘座,惟海門鎮砲臺建築合法。其澉浦之長山,乍浦之陳山,定海之舟山,海門鎮之小港口各砲臺,咸加修改。鎮海之金雞、招寶二山,於原有砲臺外,增築金雞山嘴砲臺一座。十三年,劉秉璋以浙江海防,首重舟山,次以招寶、金雞二山為要塞。乃酌度形勢,分建宏遠、平遠、綏遠、安遠砲臺四座,置克魯伯後膛大小銅砲,東禦蛟門海口。十四年,砲榮光以浙江原有之營勇砲兵,已陸續汰弱留強,加以整練,鎮海新築砲臺,及改造舊式砲臺,皆已竣工,增置新購後膛巨砲,以新練之軍駐守。十九年,譚鍾麟以浙江水師船僅五十餘艘,增紅單船八艘,助巡洋面。二十五年,劉樹棠以浙江武備新軍左營操法最精,其陸軍水師前敵駐防洋槍隊各營,步伐分合進退,亦均嫻熟,飭分駐寧、臺、三門灣各隘,並澉浦、乍浦沿海口岸。三十三年,張曾易又建言,浙江象山港在定海之南,深入象山境六十六里,口寬而水深,群山環繞,作海軍根據地最宜。尋諭南北洋大臣勘度經營。

浙江海岸綿長,省垣據錢塘江上游,外恃龕、赭二山為口門,江狹沙橫,儼如天塹,敵艦卒難闌入。道光以後,海疆屢警,雖寧、臺戒嚴,而不致牽動全局。中法之役,法艦曾至寧波洋面,招寶山砲臺卻之。此後遂無歐艦之蹤。惟象山港天然形勝,與膠澳、旅順鼎峙而三,惜築港未成云。

福建東南沿海凡二千餘里,港澳凡三百六十餘處,要口凡二十餘處。額設水師二萬七千七百餘人,分三十一營,大小戰船二百六十六艘。自清初以迄乾隆,削平鄭氏,三定臺灣,及嘉慶間靖海之役,福建用兵海上,較他省為多。島嶼星羅,處處與臺、澎相控制,故海防布置,尤為繁密。其州郡濱海者,為福寧、福州、興化、泉州、漳州五府,而臺灣障其東方。五府防務,各有注重之處。福寧重在各港口,自北境之南關山、沙埕港口迤邐而西南,為烏岐港口、鹽田港口、白馬門口、金垂港口、飛鸞江口、東沖總口,海舶之輕利者,隨處可入。其外海島嶼較大者,為東西臺、七星礁、浮瀛、大小崳山,足資屏衛。此福寧之防也。福州重在閩江,以江口內為省治所在。其外自北境松崎、江戶,經東西洛、南北竿塘、鼇江口,至閩江近口之瑯崎島、金牌、五虎門,皆扼要之所。入口經大小嶼、羅星塔,乃同、光間所創建之海軍船廠、軍械制造局,咸在於是。出口沿海而南,經梅花江口、龍江口,少東即海壇島,水師重鎮所在。其外海之島,若猴嶼受閩江之沖,東永當長樂之臂,較白大、東沙諸島為要。此福州之防也。興化重在海濱諸島,自三江口經鹿耳、大小丘,循平海衛、湄洲嶼,至雙溪港口,乃沿海之境。其外海島嶼,為平海、南日二島,列汛置官,視為重地,而湄洲亦興郡屏籓。此興化之防也。泉州重在金、廈二島。自北境惠安峰、崎港口,經雒陽江、晉江、安海港三口。其南為金州鎮。又西經大登、小登,即廈門島。島北為同安港口。金、廈二門,遠控臺、澎,近衛泉、漳,為海防重地。其外海之永寧、定安、烏潯諸島,亦設汛置兵。此泉州之防也。漳州重在南澳,鼓浪嶼為南境盡處,尤擅形勢。其沿海之境,自九龍江口折而西南,經六鼇港、漳江二口,循銅山而南,為詔安港口。其南隔海為南澳鎮,南疆要地,與粵海共之。其外海島嶼,首為烏丘,最當沖要。而鼓浪嶼當海門之口,與鎮海城砲臺同為重地。此漳州之防也。中國沿海各省,自浙洋而北,海濱淤沙多而島嶼少,其海岸徑直,故防務重在江海總口,而略於海岸。自浙洋而南,島嶼多而淤沙少,其海岸紆曲,故防務既重海口,而巨島與海岸亦並重焉。

順治十七年,王命岳以閩省之海門與廈門相望,左為鎮海衛,乃漳州府之門戶,同安縣之高浦城等處,地近廈門,為泉州府屏衛,乃屯兵於鎮海、高浦二城,而分營以防鄰近隘口。雍正四年,浙閩總督高其倬奏陳操練沿海水師,並令閩洋水師巡視本省各口,兼赴浙洋巡緝。嘉慶四年,令閩省水師仿商船式改造戰船八十艘,編為兩列。自泉州之崇武,分南北犄角。由崇武而南,令南澳、銅山、金門及提標後營各鎮將率船巡緝。崇武而北,令海壇、閩安及金門右營各鎮將率船巡緝。道光二十年,諭鄧廷楨招募練勇,嚴守澎湖,以扼閩省赴臺灣之路。二十二年,諭怡良等屯兵福州金牌各要口。其距省二十里外之洪塘河及少岐,均沈船布椿設防。閩省門戶在外洋者,為五虎、芭蕉二口。入口為壺江,水勢稍狹,無險可扼。進至金牌、長門,有巨石橫亙中流,扼守較易。又進乃閩安之南北岸,為水路總匯,兩山夾峙,可稱天險。光緒六年,於南岸建鐵門暗砲臺六、明砲臺八,北岸建鐵門暗砲臺七。七年,又於長門建暗砲臺四、明砲臺六,悉仿洋式。二十四年,增祺因閩省濱海,屯戍空虛,增練旗、綠各營,以厚兵力。二十五年,許應騤以漳州之鼓浪嶼設防尚未周備,增建砲臺,置新式砲。

綜閩省海防,所注重者,隨時異宜。當康熙間,以鄭氏由臺、澎據海壇、金、廈,故海防獨重泉、漳。其時水師以沙唬船不適於海戰,改造鳥船。施瑯之平臺灣,即藉鳥船之力。及嘉慶間,海盜蔡牽竄擾浙、閩、粵三省洋面,而閩省當其中,寧、福、興、泉、漳五郡皆剽掠經由之境,故列郡咸重海防。其時水師利用巨艦,李長庚造霆船三十艘,置大砲四百餘具,屢敗牽於閩海,卒合閩、浙水師之力,圍而殲之。最後為光緒中法之戰,法人以大隊鐵艦專攻福州,故海防獨重閩江口,而各郡無驚。同治以後,創船廠,造鐵艦,築砲臺,制槍砲,海防漸臻嚴密。乃馬江失律,盡隳前功,良足慨耳。

臺灣西與福、興、泉、漳四府相值,距澎、廈各數百里。其山脈北起雞籠,南盡沙馬碕。東西沃野,一歲三熟。宋稱毗舍那國。明季日本、荷蘭人迭踞之。順治間,鄭成功占臺灣、金、廈,時犯泉、漳。康熙初,姚啟聖以閩省水師三百艘討之,先克金、廈。二十二年,施瑯以水師二萬克臺灣,乃置臺灣府,設縣各官,鑄鐵幣,開學校,築城垣,逐生番,戍兵萬有四千,遂為海外重鎮。康熙六十年,硃一貴之叛,施世驃由廈門率水師六百艘進攻,七日而克之。乃以總兵官鎮臺灣,副將守澎湖。乾隆間,福康安平林爽文之亂,臺灣北境乃漸展拓。其山後之地,至嘉慶間始闢之。光緒十三年,開臺灣為省治,設巡撫以下各官,為中國海南右臂。及中日之戰,割讓於日本,而疆事益不可問云。

廣東南境皆瀕海,自東而西,歷潮、惠、廣、肇、高、雷、廉七郡,而抵越南。其東境始於南澳,與閩海接界。潮郡支山入海,有廣澳、赤澳諸島,皆水師巡泊所在。迤西為惠州,民性剽悍,與潮郡無異,設碣石鎮總兵以鎮之。又西為廣州境,其海灣深廣。自新安折而北,又折而南,至香山,是為內海,群島環羅,為廣州省治之外護。又西為金州、馬鞍諸山,則肇郡陽江之屏障也。又西為高州海,多暗礁暗沙,海防較簡。又西為雷州,其南幹突出三百餘里,三面皆海。渡海而南為瓊州。又西為廉、欽,與越南錯壤。廉州多沙,欽州多島,襟山帶海,界接華夷。瓊州孤懸海表,其州縣環繞黎疆,沿海多沈沙,行舟至險,水師可寄泊港口僅有六七處。此全境海防之形勢也。

廣州海防,自零丁洋過龍穴而北,兩山斜峙,東曰沙角,西曰大角,由此入內洋,為第一重隘。進口七里有山曰橫當,前有小山曰下橫當,左為武山,亦曰南山,為海船所必經,乃第二重隘。再進五里曰大虎山,西曰小虎山,又西曰獅子洋,乃黃埔入省城之路,為第三重隘。歷朝於此雖築壘駐兵,而設備未周。歐艦東來,粵東首當其沖。道光禁煙之役,英艦進薄廣州內海,林則徐督粵,屢戰卻之。其時布防較密,而壁塢皆循舊式。至光緒間,彭玉麟、張之洞守粵,始有曲折掩護之砲臺,後膛連珠之槍砲,防務益嚴矣。

清初規制,設大小兵船一百數十艘,僅能巡防內洋,不能越境追捕,遇有寇盜,則賃用民船。康熙五十六年,始建廣州海濱橫當、南山二處砲臺。乾隆五年,以廣東戰船年久失修,諭疆吏加意整頓。四十六年,巴延三以各海口時有寇船出沒,於石棋村總口設立專營,與虎門營汛聯絡。五十八年,吳俊以東莞米艇堅固靈捷,便於追捕海寇,造二千五百石大米艇四十七艘,二千石中米艇二十六艘,一千五百石小米艇二十艘,分布上下洋面,配置水兵,常年巡緝。嘉慶五年,於沙角建砲臺。九年,倭什布以粵海窮漁伺劫商船,遇水師大隊出巡,輒登陸肆擾,遂無寧歲,乃規畫水陸緝捕事宜。十五年,設水師提督駐虎門,扼中路要區,以二營駐香山,一營駐大鵬,為左右翼。二十年,就橫當砲臺加築月臺,又於南山之西北,增建鎮遠砲臺,置砲多具。二十二年,建大虎山砲臺,置砲三十二具。

道光十年,於大角山增建砲臺一,置砲十六具。十五年,在虎門砲臺置六千斤以上大砲四十具。又於南山威遠砲臺前環築月臺,亦置砲位於橫當之陰,及對岸蘆灣山,增建永安、鞏固二砲臺,沙角、大角並增建了望臺。十九年,林則徐籌防粵海,以零丁洋入口之要隘數重,歷年雖增築砲臺,而武山、橫當海面較狹,設大木排八千排,分為二道,大鐵練七百丈,臨以砲臺,輔以水兵,以阻敵船來路。時鄧廷楨因虎門當粵海中路,亦於橫當山前海狹之處,增設練排。又於武山下威遠、鎮遠二砲臺之間,增大砲臺一座,置砲六十具,以護排練。二十年,林則徐以大鵬營所轄尖沙嘴一帶海門島嶼,為海舶東赴惠、潮,北往閩、浙所必經,乃於尖沙嘴之石腳上官湧偏南之處,皆建砲臺,並藥庫兵房。二十三年,祁等以廣東民風宜於團練,招集已得十萬人,以升平社學為團練總匯之所,推及韶州、廉州等處。二十七年,增築高要縣屬琴沙砲臺,並虎門廣濟墟兵卡。同治十年,瑞麟以欽州海面與越南接界,調撥兵輪,會同舟師巡洋。時閩、滬二廠兵輪次第告成,粵省亦仿造兵輪,以備巡防。

光緒六年,劉坤一修整大黃窖及中流砥柱、虎門各砲臺,威遠及下橫當共築砲臺六十餘座,沙角及浮舟山各砲臺亦依次建築。八年,曾國荃以瓊、廉二郡洋面,與越南沿海相通,撥兵輪八艘,拖船二艘,赴北海駐防。九年,國荃以虎門為省城門戶,而黃埔、長洲、白兔、輪岡、魚珠、沙路尤為要區,乃於南岸屯重兵,為砲臺犄角,兼顧後路。十年,彭玉麟辦理廣東軍務,就粵省原有各砲臺,修整改造,並於砲臺後闢山開路,以藏弁兵。築綿亙墻濠,聯絡各砲臺聲勢。自虎門、大角、沙角以次各隘,節節設防。其新會、香山、順德等縣,選練精壯漁團,及新編靖海營兵,防堵各口。十一年,玉麟以省城要口雖已嚴防,而橫門、磨刀門、厓口皆可由海口互達,窺伺後路,淺水兵輪尚未造成,先造舢板船百艘,編為水師,以散御整,藉固內口。十二年,張之洞於廣州駐防兵內,選千五百人,習洋槍洋砲,以旗營水師並入,編為兩翼,分防海疆。十四年,張之洞、吳大澂以瓊州一島,內綏黎族,外通越南,就瓊州原有制兵,酌設練軍,並加練餉,一洗綠營積弊,舊額四千九百餘人,按七底營抽練,共編練一千七百五十人。崖州等處水師,加以整頓,原有拖船,亦配撥練軍,以二艘駐崖州,二艘駐儋州,二艘駐海口,二艘駐海安。其守兵二千人,勻撥緊要塘汛。三十三年,以廣東民風不靖,已裁之廣東水師提督,復其舊制,以資鎮懾。此粵海防務之概略也。

歷朝海疆有警,若大沽,若吳淞,若馬江,迭遭挫敗。惟林則徐、彭玉麟先後守粵,忠勇奮勵,身當前敵,將士用命,敵艦逡巡而退云。


\end{pinyinscope}