\article{列傳二百九十六}

\begin{pinyinscope}
列女二

張延祚妻蔡陳時夏妻田傅光箕妻吳

鄭哲飛妻硃李若金女王師課妻硃秦甲祐妻劉

艾懷元妻姜周子寬妻黃李有成妻王

楊方勖妻劉鄒近泗妻邢胡源渤妻董林國奎妻鄭

陳仁道妻龐張某妻秦李氏女何某妻韓張榮妻吳張萬寶妻李

沈學顏妻尤王賜紱妻時王某妻張子曰琦妻魏李學詩妻趙

學書妻高高明妻劉鄧汝明妻劉魏國棟妻龐

呂才智妻王許爾臣妻駱原某妻馬張揚名妻彭沈萬裕妻王

盧廷華妻沈李豁然妻楊曾經佑妻林梁曇妻李

姜吉生妻木曹某妻王潘思周妻傅倪存謨妾方硃

楊震甲妻楊楊三德妻馬張壺裝妻牛陳大成妻林溫得珠妻李

賈國林妻韓孫云妻白圖斡恰納妻王依氏吳先榜妻鄭

王元龍妻李蔡庚妻吳韓某妻馬李鳴鑾妻黃金光炳妻倪

徐嘉賢妻劉冒樹楷妻周曾廣垕妻劉馮丙煐妻俞

袁績懋妻左子學昌妻曾俞振鸞妻傅周懷伯妻邊

吉山妻瓜爾佳氏張某妻錢戚成勛妻廖曾惟庸妻譚

謝萬程妻李李殿機妻王長清婦程允元妻劉

楊某妻樊劉柱兒妻魯李國郎妻蘇趙惟石妻張鍾某聘妻吳

岳氏姚氏張氏袁氏楊某妻張周士英聘妻張

藺壯聘妻宋沈煜聘妻陳王國隆聘妻餘於天祥聘妻王

方禮祕聘妻範姚世治聘妻陳何秉儀聘妻劉

沈之螽聘妻唐貝勒弘暾聘妻富察氏濰上女子

吳某聘妻林雷廷外聘妻侯程樹聘妻宋張氏子聘妻姜

錢氏子聘妻王王志曾聘妻張李家勛聘妻楊

李家駒聘妻硃賈汝愈聘妻盧袁進舉聘妻某李應宗聘妻李

何其仁聘妻李王前洛聘妻林節義縣主李承宗聘妻何

吳某聘妻硃徐文經聘妻姚李煜聘妻蕭劉戊兒聘妻王

硃某聘妻李武稌聘妻李陳霞池聘妻錢汪榮泰聘妻唐

季斌敏聘妻藺董福慶聘妻馮喬湧濤聘妻方

張氏女粉姐闞氏女趙氏婢

張延祚妻蔡,漳浦人。國初,師既下福建,濱海數百里,猶群起負固。有方祐者,謀舉兵,延祚與語,不合,被殺。子才十餘歲,蔡哀慟,謀復仇。一日,聞祐將其徒至,方夕,易男子服,挾刃詣祐壘。未至,顧見其子踉蹌來,念母子並命,斬張氏祀,乃與俱歸。既,祐降為民,娶於蔡,其婦,蔡大母行也,因得常見祐。祐甘語謝蔡,蔡益憤,夜輒握刃刺壁,壁穿,刃猶擊。

順治五年春,蔡伺祐有所過,度道所必經,將其子止松林中,挾刃俟。日午,祐雄服怒馬來,蔡自林中出叱祐,祐驚呼從者,從者駭走。蔡持刀斫祐,祐墜馬,負創走,蔡疾追之。行人聚而譁,蔡且奔且言曰:「吾夫為此賊害,有助者,吾與俱死!」追及祐,祐攀松枝與鬥,中蔡額,血被面,鬥益力。遂迫祐,左手捽祐,右手奮刃,斷其首,擲道旁,觀者皆大驚。

蔡持祐首告於延祚墓,將其子詣巡按御史臺門請死,巡按御史霍達異其事,問:「有主者乎?」蔡哭對曰:「夫死,所以不即死者,以有子耳。今子且不顧,安肯受他人指耶?然殺人當死,公毋撓國法。」達乃釋不問。

陳時夏妻田,長樂人。時夏父超鵬早卒,母高守節。田讀書,知大義。時夏貧,事王姑及姑高,朝夕扶持,不去左右。病不能食,輒以口哺。時夏卒,督諸子讀,嘗自述與夫論學語,為敬和堂筆訓,以授諸子,粹然儒家言。其自序略曰:「余茍延性命,祗以三子一女,冀其能自立,不至辱泉下耳!大兒今十一,猶有童心,況諸幼孤,未亡人心力垂盡。恐旦暮死,而夫子之學行,與餘之出肝膽,忍艱苦以冀其有成者,將誰為余告之耶?爰述先訓,書之於冊。嗟乎!小子異日讀此,其能自省,使餘生不負於子女,死不愧於夫子否耶?」居十餘年,卒。

傅光箕妻吳,宣城人。吳歸於傅,光箕已病矣,逾年卒。吳父母欲嫁之。吳歸,留吳而訟傅氏,衣食吳。吳還傅氏,以訟故勿納。吳復歸,請自食,無累父母。力紡,聞有媒至,輒求死,乃別居。明季,饑,恆餓。鄰饋之,勿受。族姊歸於魏,亦嫠也,遺之米,乃半易糠覈。或怪問之,曰:「雜糜之,可一月不死也。」久之,紡有餘錢,得婢曰春蘭,拾籜供爨事。里媼或呼春蘭食,吳必審所自,戒勿輕受食。春蘭自是即不受里媼食。

鄭哲飛妻硃,哲飛,南安人;硃,明魯王以海女也。嫁哲飛,生丈夫子一,女子子三,而哲飛卒。會以海亦殂,渡海至臺灣,依明宗室寧靖王術桂以居。康熙二十二年,師克臺灣,術桂自殺,硃奉姑育諸孤,以女紅自給。居五十餘年乃卒,年八十餘。初師下舟山,以海妃陳入井死,以海謚之曰貞,而以海女又以節終。

李若金女,名訚,餘干人。明季,字淮王世子由桂。入國初,由桂出亡,訚誓不更字,嘗詠金環曰:「紅爐經百鍊,不失本來真。」事父母孝,年五十九卒。

王師課妻硃,蕭山人。師課,明天啟中官太醫院院判,卒。明亡兵亂,硃率二子避九里坳,嘗遇賊,脅以刃,硃奪刃剺面,哭且詈。賊欲殺之,二子號慟求代,得不死。事平,歸老於家。嘗為勖子歌五章,其三章曰:「我生之後逢世亂,白頭兵起蒼黃竄,膚血染點叢麻紅,母子支離宵不旦。飛雷聚驚鼓鼙,秋雨淋漓斷薪爨。嗚呼,九里坳邊真瓦全,爾曹性命天所憐。」五章曰:「庭闈肅潔辭親族,薄田聊許資饘粥,震蕩扁舟波復風,兒才卻聘家回祿。此身直緣正氣生,機杼猶能活枵腹。嗚呼,但原長作太平民,何嘗俯仰慚天人。」

秦甲祐妻劉,三原人。甲祐病瘓,劉侍疾甚謹,筦家政甚飭。越十年,甲祐卒,時歲饑,兵未定。劉撫二子四符、四採。嘗訓之曰:「年荒,眾人之荒;學荒,則吾兒之荒也。兵亂,眾人之亂;心亂,則吾一家之亂也。」聞者以為名言。四符,甲祐前婦子也,劉愛之,均於所生。

艾懷元妻姜,米脂人。懷元父穆,兄懷英,在明皆官參將。穆卒,國初懷英降,入鑲藍旗,授牛錄章京,居京師。順治八年,懷元往省其兄,既歸,仇家誣為逃人,遂亡命。官收其孥,穆妻馬,老矣,妾金請代,姜方娠,皆就逮。明年,事雪,西還。姜襁稚子,金與相扶持,行數千里。又明年,馬與金皆卒,懷元遣信至,言母死不得奔喪,誓畢生不歸。姜食貧撫子,居四十餘年乃卒。

周子寬妻黃,順德倫教村人。子寬刺船,與其侶戲,侶溺,坐減死戍貴定。黃求從夫行,譁縣門,吏為注官書。乃盡鬻嫁時物畀舅姑,制竹擔荷具從夫行。夫道病,黃行經村市,操土音歌,求錢,得藥物酒食奉夫。夫瘳,達戍所。居十七年,舉一子、二女,而夫死。黃求以夫骨歸,跪縣門搏顙二十餘日,吏許之,畀以牒。

黃懷牒裹夫骨,筥負小兒女,獨身以行。其長女已嫁農家子,牽衣泣,黃斥不顧。黔多虎,而黃負夫骨,逆旅禁不納。日汲於澗,拾樹枝以爨,夜宿道旁廢廟,恆見虎殘人,餘骼狼藉,無所怖。及至村,黃齒既長,黧黑醜惡,又雜羅施語。有叟獨識之,指道旁塚曰:「此而翁也,而姑殭墻陰,不食已一日。」

黃求得姑,姑兩目眊,黃引其手拊裹中骨,及筥中兒女。姑抱而噎,黃大號,筥中兒女亦號。鄉里皆走視,義之,畀以金,僦屋奉姑居。黃行逮歸十九年,順德人號曰「女蘇武」。

李有成妻王,常寧人。寡,悉散奩飾於族鄰貧者。將卒,呼諸婦曰:「吾寡居四十餘年,耳目如聾瞶,未嘗妄視聽,汝曹其識之!」

楊方勖妻劉,宣城人。嫁五日而寡,剪發自誓。鄰婦或微諷,劉出刀以示,曰:「吾晝以是為鏡,夜以是為枕。」鄰婦懾,不敢復言。

鄒近泗妻邢,昆明人。寡而貧,或諷之嫁,邢曰:「吾能忍饑寒,不能忍恥。」卒以節終。

胡源渤妻董,臨清人。源渤卒,董年十五,為嫠八十年,年九十五乃卒。里婦或問:「守節易乎?」曰:「易。」「如無夫何?」曰:「如未嫁。」「如無子何?」曰:「如有子而死若不孝。」曰:「何以制此心?」曰:「饑而食,倦而寢,不饑不倦,必有事焉,毋坐而嬉。吾嘗為人傭,治女紅,必求其工。求工,則心專;心專,則力勤;力勤,則勞而易倦。倦即寢,寤即興,毋使一息閒,久之則習慣矣。」

林國奎妻鄭,閩人。國奎卒,有子二。鄭將殉,姑誡以存孤,乃已。一子殤,遂自沉於江,漁者拯以還。姑疾,刲肝雜糜進,疾良已。族有亡賴子嘗中夜至,告族人杖於宗祠。亡賴子為嫚書汙鄭,鄭恚,取刀斷左耳,訟於縣,縣笞亡賴子。亡賴子出,益妄語,鄭復割右耳。巡撫卞永譽聞其事,坐轅門讞其獄,令隸以兩耳示觀者,械亡賴子至,閱嫚書一行,輒撻其面,復重榜荷校論戍邊。居數月,鄭兩耳復生,永譽復坐轅門,召而察之,左耳完且晰,右耳赤如血,下廓乃微赬而短於左。文武吏及諸觀者皆驚嘆,一時稱異事云。

陳仁道妻龐,博白人。康熙十九年,吳三桂將程可任掠博白,仁道將與鄰人拒之,為所殺。龐自經,家人救之,甦,乃斥產購得殺仁道者,殺諸仁道墓前。

張某妻秦,三原人。康熙三十一年,仍歲大祲,縣民多流亡。秦內外無所依,至龍橋河北,河岸坼有隙,自匿其中,有老人憫之,遺以食。明日復往,則昨所遺故在,勸之食,且問故,秦曰:「謝翁厚,然不可為常,先後等死耳,我坐岸隙,令死不至暴露足矣。」遂餓而死,年二十餘。老人為封焉。

同時李氏女,從父母逐食至漢口,父母皆疫死。女年十六,美,儈聘焉,將鬻使為妓,女得其情,力求死。三原人賈漢口者群詰儈,儈陰殺之。

何某妻韓,張榮妻吳,張萬寶妻李,皆濰縣人。韓早寡,求疏屬子為後。康熙四十三年,濰大饑,韓晝抱子拾薪,夜則紡績,日一食。久之,有所蓄,非甚饑則不食。卒買宅娶婦生孫,年七十三卒。

吳嫁三日,夫死,貧甚,轉役自活,夜必歸其室。得米雜糠粃樹葉為食,贏一日食,則一日閉戶。年九十二,病將死,呼其侄,謂曰:「我有銀紉衣帶,猶昔吾夫物。我死,以此市棺埋我夫墓側。」

李嫁生子,方晬,而喪夫。舅、姑謂曰:「汝不幸,我曹老,子幼,汝當如何?」李泣曰:「婦非為舅姑老子幼,夫死何所不得?猶忍活至此,婦自審已決,原舅姑無疑。」舅賣漿,暮出戶,聞鐸聲,必趨往代其擔。抱子力作,人未嘗見其啟齒。既喪舅、姑,娶婦生孫乃卒。疾革,謂其子曰:「我死得見汝父,我甚喜,汝勿悲也!」

韓居縣東南草廟村,吳居縣西張家村,李居縣北長甿村。

沈學顏妻尤,仁和人。學顏卒,無子,以從子時吉為後。時吉生子大震,又卒。尤撫孤孫,其兄侮之。秋將穫,以眾刈其禾,尤置針於髻末,外向踴而號,兄提其發,針創手乃去。常恨其孫弱,曰:「我安得見曾孫,見曾孫,死不恨。」大震娶婦舉子,尤乃卒。既卒,大震復舉子近思,自有傳。

王賜紱妻時,黃平人。賜紱出行,宿於翁丙,為苗所殺,棄尸箐中。時行求得之,告官,得苗五,俱伏罪,時年二十一。母欲令更嫁,剪發、烙左頰,毀容矢不行。

王某妻張,灤州人。早寡,無子。以族子曰琦後,亦早卒,妻魏,亦州人。所居村曰柳河,地卑濕,食不足,掇草根木葉,拾蘋藻,雜糠粃以食其孤,復殤。復以族子後。張卒,族人諷魏嫁,魏不可。居十餘年,為所後子娶婦,乃語所親曰:「吾乃今志始遂,使嫁,不過溫飽死耳。人恆苦貧,吾獨不自覺。苦皆自樂生,吾生不知為樂,又焉知有苦?」

州又有李學詩妻趙,學書妻高,娣姒以節著。學詩、學書生友愛,行涉水,學書誤就深,學詩拯之,相抱持俱死。趙生二女,高無出,食貧堅守,年皆逾八十。

高明妻劉,秦安人,早寡,子步雲幼。貧甚,嘗伺鄰家炊,乞餘熱為兒煁餅。步雲稍長,就學歸,則燃燈讀。劉縫紉,夜必盡數線。一夕,線未盡,步雲倦臥,撫之有淚跡,問曰:「兒耶病?」曰:「無之,但饑耳!」劉泫然曰:「兒不慣餓,我則常耳!」步雲為賈,家漸起。

鄧汝明妻劉,崇善人。康熙四十一年,歲大無,官煮粥食饑民,劉不食五日。鄰家招偕赴,劉恥之,三出三返,終不行。因投水,漁人拯之,坐岸側,漁人去,復入水死。

魏國棟妻龐,蠡縣龐家莊人。祖姑徐、姑董,皆節婦。國棟卒,無子,龐力女紅以養。織日一疋,或授以纑,織成必增重,曰:「糨所滋也。」或與值多一錢,不受。祖姑八十餘,目昏,向曝、如廁,躬負以出入。姑亦至八十,負出入如之。再居喪,有周之者,龐曰:「吾貧,幸相貸,然必償。如不使我償,是視我非人也。」日夜織,不期月皆償。當葬,衰而前柩,或請代,龐曰:「我祖姑、我姑無子孫,我在,即其子孫也,可代乎?」姑葬以夏,方雨,龐涉潦號踴,見者皆流涕。雍正三年,縣大水,歲無。有縣治賑役自戶外呼告之,龐曰:「婦固饑,然食朝廷米,償否?」曰:「賑也,何償?」龐曰:「償則食,不償,則我孱婦何功報朝廷而徒食乎?不可!」遂鍵戶,復呼之,不應。縣使役具刺歸之米一石,龐復辭。役曰:「此喬令君所以旌節義,毋辭!」乃拜而受。縣上其事,得旌,族人為立後。

呂才智妻王,博興人。才智病人區僂,杖而行,鬻餅於市。歲祲,才智將鬻王,王曰:「汝病廢,我去,汝不得生!且我身值幾何?汝不過得數日飽。食盡,終當死。等死,不如相依死也。」乃令才智守舍,而出行乞。生一子,才智死,終不嫁。

許爾臣妻駱,肅寧人。家奇貧。爾臣及其父母相繼卒,駱號於市,得柳棺瘞焉。或勸:「盍嫁?」駱曰:「乞食雖辱,猶勝於再嫁!」卒以窮餓死。

原某妻馬,河津人。康熙六十年,饑,行乞食。泣語人曰:「乞食至辱,不如死,顧安得死所無累人耶?」或漫應曰:「去此十餘里,有紅石崖,死此,可無累。」馬明日徑至其所,脫耳環易餅,遲鄰人過者,囑以畀其母,曰:「為我語母,無復望我,我今死此矣!」即投崖下死。

張揚名妻彭,臨江人。早寡,貧,或謂行乞可得食,彭唾之,曰:「我亦書生婦,有餓死張氏舍耳,安能為丐?」日夜操作,立後,娶婦,持門戶。

沈萬裕妻王,浙江山陰人。萬裕早失母,王事後姑謹。萬裕卒,子幼,後姑虐使之。舅予田數畝,使別居。後姑使嫁,王不可。後姑陰取犬子胞擲王室,陽出之,曰:「寡婦室,何乃有此?」迫嫁益厲。或語王:「當以死自明。」王曰:「吾當死。吾死孤不得生,夫且無祀,事終當白。吾死,又誰吾明也?」藏其胞,事後姑愈謹。後姑有少子訟於縣,知縣姚仁昌察胞非人,杖少子,而表王節。其後少子死,王收其孤,為娶婦。

盧廷華妻沈,永定人。廷華好狹邪游,擯沈異居。姑溺愛,亦惡沈。沈晨必謁姑,為理井臼。或私具甘旨,姑不善也。施鞭撻,無懟。廷華得惡疾,沈乃歸侍。廷華死,以節終。

李豁然妻楊,永年人。康熙十五年,豁然卒,楊年二十一。事舅姑孝。撫子尊賢,娶婦王,生子而尊賢卒,姑、婦共撫孤孫至成立。楊以乾隆四十二年卒,壽百二十,守節百有一年。王前一年卒,年亦九十八。

曾經佑妻林,惠安人。早寡。所居濱海,為漁家補網,夜無燈,隨月升落為作輟。積數十年,目因以盲,而手甚習,操作如故。舅姑資以老,復為夫立後。

梁曇妻李,臨汾人。曇卒,時子生方兩月,貧,啖野菜以活。曇嘗蒔槐於庭,李日紡其下,護之甚謹。曰:「此吾夫手植,見之如見吾夫矣!」鄉人因稱「節婦槐」。

姜吉生妻木,東川人。雍正八年,東川屬夷叛,從吉生逃山中。賊至,殺吉生及其子,木忍哭伏林間。師至,賊降,木躡賊至城西,手搏殺吉生賊以告官,請得手刃之。提督張耀愍而許焉,遂磔賊以祭吉生。

曹某妻王,興縣人。早寡,子喑,鄰婦亦早寡,相與約不嫁。居十五年,王詣其戚,或自外至,曰:「鄰婦嫁矣!」王曰:「信有之乎?」曰:「信,我所目見也!」王乃大慟,曰:「不意此婦,乃有此事!」遂絕。

潘思周妻傅,名五芳,會稽人。思周父為田州吏目,傅氏亦僑居廣西。嫁年餘,生一女,思周卒。或欲聘焉,傅截發矢曰:「所不終於潘者,如此發!」未幾,母與兄死,兄公及娣又死,舅亦死,傅持六喪還。出郭門,身衰絰,徒步號泣以從。僮民皆感嘆,稱孝婦。歸營葬,撫叔及其女畢婚嫁。

倪存謨二妾方、硃,富順人。存謨為英山知縣,坐事戍伊犁,方、硃皆從。存謨死,方、硃慟不食。伊犁將軍為徵賻,俾持喪歸。至富順,嫡子出郭迎,方、硃相謂曰:「我二人不死者,懼主人骨不歸。今歸矣,請死。」相攜躍入江,救不死,嫡子及孫死,撫曾孫二成立。

楊震甲妻楊,楊三德妻馬,張壺裝妻牛,皆秦州人。夫皆出客游,久不歸。皆善事孀姑。馬姑尤嚴,日被箠楚,奉之愈謹。楊撫子女成立。馬、牛皆無子,立後。州人為之語曰:「馬牛羊,立人綱。夫遠客,姑在堂。胸中冰,頭上霜。」蓋借「羊」目楊也。

陳大成妻林,連江人。大成坐事戍黑龍江。將行,遣林別嫁,林不可,從大成戍所。居二十八年,大成死,林裹其骨,襁兒女,乞食跣行萬餘里,還故鄉。灌園自給,葬大成祖墓側。

溫得珠妻李,永清人。得珠早喪母,父娶後妻,生二子,遂惡得珠,並憎李。得珠病狂易,一日逃其叔杖,投井死。父母聞,不哭,李力請,乃得斂。遺腹生子經元,舅姑迫李嫁,謂李嫁,則田廬皆二少子產也,因虐之百端。李度終不可留,抱經元辭舅姑還母家,賃地以耕,勞苦自食力。經元娶婦生孫,而舅及二少子皆死,遺田亦殆盡,姑衰病無所依。李乃率子婦還,起居床下。姑執手流涕,道其悔也;而得珠叔故助虐者,亦前死,其嫠仰食於經元。經元有四子,皆力田,能孝養。

賈國林妻韓,國林,扶溝人;韓,淮寧人。乾隆五十一年,大饑,民為盜。國林有族子二,行無賴,執國林及韓,綁於庭之槐,而盡取其室所有,已乃斫綁釋之。國林將指傷,越三日死。韓欲告官,無人焉為之佐。有子二,皆幼。其弟日負薪米贍姊,夜執梃伺門戶。居數年,無賴又至,徹其屋茅,擲大磚中韓手,遂奪田伐樹,一不與較。二人者死,乃稍稍得安。嘉慶二十三年,又大饑,無賴有子鬻其嫂,夜出走,韓為召其夫婦之。因泣告其子曰:「害爾父者,某也。今其子又鬻嫂,不仁哉此父子也!顧為賈氏婦,即餓死,豈可失清白,汝曹當死守之!」此婦竟得免。

孫云妻白,興縣人。生十四年而嫁,嫁十三年而云卒。又二十年,子長娶婦,白挈以拜雲墓,指而言曰:「此君子也,此君婦也,吾事畢,可以從君矣!」慟而僕,遂絕。

圖斡恰納妻王依氏,滿洲人,乍浦駐防。圖斡恰納,瓜爾佳氏,早喪母,尋亦卒,無子,嗣絕矣。父查郎阿謀為立後,王依氏曰:「子他人子,終非骨肉,不足奉大宗,原翁娶繼室。」查郎阿感其意,娶於邵,生子觀成。觀成生七月,而查郎阿卒,王依氏哀姑少寡,奉養甚謹,躬操作助姑撫孤。既遘疾,猶不自逸,事輒代其姑。卒時觀成已舉鄉試,以子鳳瑞為兄嗣,未百年而子孫繁衍至百餘人。

吳先榜妻鄭,陜西山陽人。先榜卒,鄭誓殉。家人慰喻之,曰:「兩兄公皆無子,若方有身,男也,吳氏幸有後。」逾數月生男,撫以成立,吳氏得有後。

王元龍妻李,嘉興人。元龍悍,嗜酒,稍拂意,輒呵斥。既,傷於酒而病,李斥嫁時所媵田供藥餌。元龍病,益悍,稍間,則日夜博。怒李,故以非禮虐使,或加以鞭楚,李安之,無幾微忤也。元龍病三年而死,李朝夕上食,輒號慟。服除,會兄公之官福建,姑老不能赴,李往奉姑,七年而姑卒。李泣謂諸從子曰:「我當從汝叔於地下矣!」會火發,李整衣坐樓上,有梯而援者,李戒毋上樓,燼死焉。

蔡庚妻吳,合肥人。早寡,立從子為後,以事姑。嘗為辭自序曰:「父母生我時,惟原得其所。十六歸君子,同心祀先祖。歸時舅已歿,姑老誰為主?嗟嗟夫質弱,終朝抱疾處。十八幸生男,朝夕姑欣睹。無端因痘殤,姑泣淚如雨。堂上節姑哀,入幔痛肝腑。二十再生男,視若擎天柱。兒生甫一載,忽然夫命殂。始婦並時啼,眷屬群相撫。死者不復生,弱息堪承父。那知天奪兒,骨肉又歸土。姑祗有哭時,我豈無死所!還念朽姑存,我死誰為哺?隱痛斂深閨,衰顏原長護。奇災偏遇火,焦爛姑肌膚。和血以丸藥,年餘乃如故。災退宜多壽,云何復病殂!送姑歸黃泉,夫缺我今補。我今補夫缺,一死何所顧?哀哀我父母,惸惸將泣訴!」卒,年八十有八。

韓某妻馬,萊蕪人。貧,夫商於遼陽,馬出為傭。聞夫死,其父欲嫁之,馬曰:「歸夫骨其可。」乃乞食行五千里,得夫骨,負以歸。日行一二十里,夜或露宿,犯風雪,行歲餘,乃至家。既葬,其父終欲嫁之,馬執白刃自誓,乃已。

李鳴鑾妻黃,騰越人。咸豐間,雲南回亂,鳴鑾以千總戰,負傷卒。黃截發,撫二子。同治初,寇至,轉徙為人縫紉浣濯,日率一粥,仍督子讀不輟。嘗曰:「人不讀書,與禽獸何異?」

金光炳妻倪,金華人。光炳卒,倪殉,救免。洪秀全兵至,攜二子竄山谷。亂定,力作自給。貧甚,督子讀,不少假。

徐嘉賢妻劉,嘉賢,天津人;劉,桐城人。嘉賢少從軍河南,嘗單騎入賊壘,拔陷賊婦女數百人出。旋卒。劉貧,輒數日不舉火,嚴督其子讀。族有為令者招使往,劉曰:「今不自立,而託於人,懼吾子之不振也!」謝不往。

冒樹楷妻周,樹楷,如皋人;周,祥符人。樹楷以知縣待缺福建,早卒。周挈子女從舅廣州,舅亦卒。僑居,日食率百錢,翼子女以長。子得官,將請旌,周拒之曰:「婦節常耳,人子於其母,奈何欲假以為名哉?」父星詒,諸父星譼、星虓,並有文行,周刻其遺著,為父營葬,置墓田焉。

曾廣垕妻劉,衡陽人。歸廣垕,舅老,姑前卒。兄公初喪,舅痛子,幾失明,出入需人。劉侍舅謹,日執炊,一飯三起視舅起居衣食。雖貧,必具酒肉。舅病,奉侍七晝夜不就枕。舅卒,棄田廬治喪。劉方產,徙陋巷,艱苦冰雪中。廣垕又卒,乃與姒李同居,以子為之後。李亦苦節,劉事之如姑。晝治針黹,夜則紡績,節衣食,命子熙就學,卒成進士。方極困,老稚或乞食,必分食與之。晚少豐,年饑,必出穀以賑貧者。

馮丙煐妻俞,丙煐,大興人;俞,婺源人。丙煐為世父後,俞事兩姑,維護調和。迭遘諸喪,丙煐亦卒,喪葬皆盡禮。光緒二十六年,京師被兵,俞市米數十石與貧者,戚友相依者六十餘家,衣食之,亂定始去。亂後多暴骨,募貲為收斂。死難者,求其姓名為請旌恤。獄囚衣糧主者不能給,斥銀米畀之。其後直隸、安徽災,輒募貲至鉅萬。京師恤嫠會、八旗工廠,皆輸金以助其成。

袁績懋妻左,績懋見忠義傳。左名錫璇,字芙江,陽湖人。事親孝,父病,刲臂和藥進。工詩善畫,書法尤精,著有卷葹閣詩集。

績懋子學昌妻曾,名懿,字伯淵,華陽人。通書史,善課子,著有古歡室詩集、醫學篇、女學篇、中饋錄。

俞振鸞妻傅,振鸞,餘杭人;傅名宛,號青泉,大興人,以禮女。能承父學,工詩,著有山青雲白軒詩集。教子嚴,建宗祠,立條教,示子孫。光、宣間,江、浙遇災,屢蠲金賑之。

周懷伯妻邊,懷伯,餘杭人;邊,諸暨人。邊事姑孝,懷伯卒,有女子子三。邊恃女紅養姑,營喪葬,嫁三女,貸於人以舉。節衣縮食,數十年乃畢償。年六十九,知將死,辭親族,啟夫墓右生壙,坐臥其中,遂死。堅囑毋具棺,重以累人。親族哀其志,樏梩而掩之。

吉山妻瓜爾佳氏,名惠興,滿洲人,杭州駐防。早寡,事姑謹,嘗刲肱療姑疾。光緒季年,創立女學。逾年,貲不足,校將散,乃飲毒具牘上將軍,自陳以身殉校。且言曰:「雁過留聲,人過留名,我非樂死,不得已耳!」既死,將軍瑞興與巡撫張曾易又奏聞,賜「貞心毅力」額,眾為集貲擴校,以「惠興」名焉。

張某妻錢,嘉興人。生一女而嫠,還依父母居。姑貧,計鬻之,度錢剛,言無益,陽攜以省戚。先期告鬻婦家,待郭外,舟出郭,別有舟來並艤,則鬻婦家人也。姑乃告錢,錢即起,躍入水。鬻婦家人大驚,而姑已得錢,強婦往,趣舟行。錢屢躍入水,持之不能止,至三。眾皆懼,乃送還父母家,而錢為救者搤胸傷,咯血,數月卒。

戚成勛妻廖,江津人。成勛家萬山中,張獻忠之亂,成勛出避寇,廖弱不能從,閉重門獨居。家故有餘粟,粟將盡,就池畔種稻以食。衣敝,綴草自蔽。居四十餘年,山徑塞,與世隔絕。成勛竄黔中,聞亂定,乃還,行求故山,斧竹木得道,見其宅盡圮,隱隱起炊煙。呼且入,廖自樓上問誰何,成勛道姓名,廖乃泣曰:「我夫今得還耶?我無衣,君以餘衣畀我,乃得下相見。」成勛解衣擲樓上,廖衣以下,面目黧黑,發如蓬,相持大慟。其居又十餘年,年各至九十餘。

曾惟庸妻譚,衡陽人。順治五年,譚歸惟庸,方四閱月,惟庸為游騎掠去。亂定,有言惟庸死者,譚召族人,分授以田宅。康熙二年,惟庸還,詐稱行賈,過譚,音容已盡變,譚不能識。求食,與之;求借宿,不可。越日再至,乃自名惟庸,譚未敢信,問臨別時事,嘗授三鑰,鐵奇銅偶,語皆驗。譚乃泣而言曰:「君別十六年,謂物故久,今幸生還,當告諸宗族。」惟庸召族人,置酒,具白其事,為夫婦如初。

謝萬程妻李,唐縣人。萬程父儀,順治間諸生,貧,卒無棺,萬程將鬻妻以為斂,不忍言。李知萬程意,哭請行。南陽民王全以二十四金鬻李歸,將以為妾。李至全家,日涕泣,但原供織紝,不肯侍全,全亦聽,不強。居一年所,全兄大有與全隙,詣南汝道告全匿逃人。事下南陽府同知張三異,三異漢陽人,嘗為陜西延長知縣,有惠政。詰大有,辭遁。召全,並以李至,問何為匿逃人,全目李妾,因言:「妾至日涕泣,但原供織紝,居一年所,不我從也。」問得自何所,乃復召萬程,具得賣妻葬父狀。三異驚嘆,問萬程:「欲復合否?」萬程言:「妻故無失德,聞其至王氏日涕泣,但原供織紝,居一年所,艱難以守身。我豈不欲合,而無其貲,則奈何?」三異出俸二十四金償全,而使吏以金幣送萬程夫婦還。

李殿機妻王,名素貞,亳州人。幼喪母,父以字殿機,殿機父範同,順治初坐法,妻張及殿機沒入象房,殿機方三歲。稍長,自鬻於鑲紅旗護軍厄爾庫為奴,厄爾庫妻以婢蕭。王從其父居二十餘年,其父病且死,以簪珥授女,泣曰:「此李氏物也!」又數年,或傳殿機死,王氏諸父兄迫女別嫁,女原為殿機死。久之,詗殿機猶在,欲走京師求殿機。鄰有範一魁者,其父友也,王乞為導,諸父兄不欲,令處於樓,去其梯。王以夜縋而下,從一魁至京師,求諸象房,有知者導至厄爾庫家,殿機荷畚拾馬通自廄出。一魁前與語,王出父故所授簪珥,相向哭,行路聚觀,皆流涕。厄爾庫義之,許放殿機及蕭,不督自鬻值。巡視南城御史阿爾賽疏聞,下禮部。禮部議:「八旗家奴不得復為民,惟王氏守節求夫,有裨風化,應如所題。」康熙二十八年四月乙未,疏上,聖祖可其議,王年已三十有四,猶處女也。

長清婦王氏,父王三,農也。未行,歲祲,父母舅姑議鬻之,而均其值。販挾以去,至饒陽,入妓家,矢死不肯汙。轉至孔店村,村諸生孔繼禹、繼淳兄弟好義,愍其志,以五十金贖焉。問所居地,曰焦家臺。問戚屬,以父王三對。當春,村民祠泰山,具榜書女始末畀行者,誡使入長清界則揭榜。焦家臺農有見者,以告王三,詣孔氏以女歸,復歸所字壻。

程允元妻劉,名秀石,允元,江南山陽山;秀石,平谷人也。秀石父登庸,康熙間為山西蒲州知府。初謁選,允元父舉人光奎,亦在京師。相與友,申之以婚姻。時允元二歲,秀石生未期也。光奎歸,尋卒。乾隆初,登庸罷官,居天津北倉,亦卒。秀石年二十二,母前卒,諸兄奔走衣食,弟崇善為童子師,徙廢宅。姊妹姑侄猶五六人,食不得飽,寒無衣,相倚坐取暖。崇善死,益貧,恆數日不得食。屋破,群殭坐雨中,乃徙依比丘尼照震。無何,家人相繼死,惟秀石存,力針黹自活。照震徙天津,秀石從。嘗有求婚者,介照震道意,秀石恚,不食,照震力謝乃已。

允元既喪父,亦中落,聞登庸卒,家且散,顧不知女存亡。或傳女死,勸別娶,允元不可,且曰:「女即死,必酹其墓乃別娶。」乾隆四十二年,附運漕舟至北倉求劉氏,有舟人為言:「劉氏家已散,其孥殆盡死,惟第四女存,是嘗字淮安程氏,傳程氏子已死,而女矢不他適。昔居準提菴,今徙天津,不知菴何名也。」允元因言己即程氏子,舟人又言:「劉氏有故僕,瘖而義,歲時必問女起居。」允元求得僕,偕詣照震,言始末,照震疑,且憚秀石,未敢以通。允元言於監漕吏,牒天津縣知縣金之忠,之忠召允元問之,信。使告女,且勉之嫁,女猶辭。復使謂曰:「女不字五十七年,豈非為程郎?程郎至,天也,復何辭?」乃成婚。

大學士兩江總督高晉以其事上聞,下禮部,禮部議:「義夫貞婦,例得旌表。至幼年聘定,彼此隔絕,經數十年之久,守義懷貞,各矢前盟,卒償所原,實從來所未有,應旌表以獎節義。」上從之。

楊某妻樊,字正,撫寧人。既字而楊氏子病且廢,使辭於樊,樊母乃為正改字。行有日,正請於母曰:「兒奚嫁?」母曰:「嫁某氏。」正曰:「兒幼非受楊氏聘乎?」母曰:「然,楊氏子病且廢,使辭於我。我憐兒,故為兒改字也。」正不語,夜潛出,度山林數十里,晨至楊氏。翁姑未即許,父母亦至,相與慰勉。正曰:「夫病,天也,我為病夫婦,亦天也,違天不祥。欲別嫁,我請死。」乃卒歸於楊,楊氏子病良已。

同縣又有劉柱兒妻魯,字春。柱兒先為李氏義子,聘於魯,既復還劉氏。李富而劉貧,於是李氏之人,嗾魯使罷婚,劉不敢爭也。春聞,亡之劉氏,魯氏劫春歸。訟於縣,縣判歸劉氏。時乾隆十九年,先樊氏女事一歲。

李國郎妻蘇,南安人。未行,父以國郎貧,為女別字富家子,焚李氏書幣。蘇縊,未絕,父招富家子贅於家,以死拒,撻之不悔。富家子自去。國郎聞,訟於官,乃歸於李。婚夕,泣曰:「吾父以吾故在系,何得遽言婚!」國郎為請於有司,出其父。

同縣蔡登龍妻林,其父母亦以婿貧欲別字,不從,令別居。積女紅得十五金,使以遺登龍佐聘錢,父母少之。乃日減餐,治女紅益勤,逾年又得十餘金,卒歸登龍。父母既喪,孤弟貧無依,乃收撫之。

又有黃元河妻戴,吳恆妻陳,婿皆有廢疾,父母議毀盟,力請行。戴勤儉起其家,吳以節終。

趙維石妻張,小字瑤娃,寧羌人。年十七,未行。嘉慶初,教匪掠州,賊渠得之,以畀其妻。其妻以瑤娃慧,畜為女,渠累欲汙之,賴其妻以免。尋竄徽縣,一夕渠醉,召瑤娃,瑤娃拒之力,渠使其下將出殺之。其妻知不可救,戒勿過創,棄諸野,而以死告。次日賊引去,村婦舁之歸,藥其創良愈,將以為子婦。會縣吏過門,瑤娃拔銀釵賄吏,使告縣。瑤娃至縣庭,陳始末,乃召維石,為合婚,與俱歸。

鍾某聘妻吳,武岡人。待年於鍾氏。鍾氏子從父賈四川,久不歸,或傳已死。鍾母卒,吳紡績奉其祖母。祖母卒,為營喪葬。年四十餘,鍾氏子始歸,欲與婚,吳曰:「君出游久,安用就木老處子為!」出貲為買妾,而自居別室。鍾氏子以不婦訟於官,吳曰:「若祖母,吾奉之;若妾,吾畜之。吾齒長,不能育子女,請以貞終。」官判從之。

岳氏,安平人。嫁可仁言,病癇。仁言以禮去惡疾,遂大歸。居數年,病已,而仁言已別娶。或諷其嫁,岳不應,以針線遍綴衣履投井死。仁言聞,乞李恭銘其墓。

姚氏,通州人。嫁同州張維垣。維垣移家湖北,歸既娶,復去。逾年,遺書絕姚,令改嫁。姚持書泣告鄉黨曰:「我無故見絕,死無以自白,原終守以明志。」居五十餘年乃卒。張氏之族高其義,持喪葬張氏兆,為立後。

張氏,江南華亭人。字金景山。年十二,喪父母,待年於姑氏。張莊而無容,景山憎焉。稍長,當婚,景山故遲之。既而病作,張奉湯藥,斥不使近,輒泣而退。景山將死,指而語母曰:「彼非吾偶,兒死必嫁之。」景山死,張矢不嫁。或以夫不見答勸,曰:「我知夫死婦節而已,不知其他。且祖姑及姑誰為養者?若必強我,我請死。」是歲姑卒,越八年,祖姑卒,張為營葬。日夕紡績,足不喻閾,又三十餘年乃卒。

袁氏,名機,字素文,仁和人。兄枚,見文苑傳。機幼字如皋高氏子,高氏子長而有惡疾,其父請離婚,機曰:「女從一者也,疾,我侍之;死,我守之。」卒歸於高。高氏子躁戾佻蕩,游狹邪,傾其奩具;不足,抶之,且灼以火。姑救,則毆母折齒。既,欲鬻機以償博負,乃大歸,齋素奉母。高氏子死,哭之慟,越一年卒。

楊某妻張,名荷,寧國人。某貧,無行,令張以非義,不應。樓居,潛去床前板,紿使墮,折足,匍匐歸母家。某鬻子,張積金贖之。將卒,命子以喪歸楊氏。

周士英聘妻張,泰州人。士英喪父母,叔狡,利其有,箠殺之。時順治九年,張年十九,未行,聞其事,哭,不食。遂自髡為尼,具牘丐母舅偕訴有司。巡按為上其事,誅殺人賊,張乃理士英家財,葬士英及其祖若父,為廬奉佛,祀周氏三世。張既為尼,名曰明貞,表其志也。

藺壯聘妻宋,名典,蔚州人。典家西崖頭,壯居千字村,皆農家也,以羅帕為聘。壯死,典方從母舂穀,聞,輟舂,慟不食。父母喻之,意若稍解者,數日,以羅帕自經死。時康熙四年正月庚辰。

沈煜聘妻陳,名三淑,錢塘人。幼能詩。康熙間,訛言官中閱選,民間女子倉卒嫁娶殆盡,三淑父以許煜。煜故貧,客松江,久不歸,三淑父從軍雲南,戰死。其母欲改字富人子,揚言煜已他娶,以絕三淑意。三淑聞,慟哭,自髡其發,矢不字,遂病,時時哭,極悲。鄰生有聞而哀之者,求煜告以故,煜請婚,母持不可。二十二年春二月,三淑病篤,其母以媒言召煜,煜至,使入省三淑。三淑方寐,告以沈郎至,遽寤,手下帷自蔽。煜問:「可有言乎?」三淑徐曰:「既有成言,何為又他娶?」煜辨其誣,三淑都無言,惟以袂掩淚。煜辭出,三淑泣不已。已而嘆曰:「彼不負我,我死可。」遂不飲藥,越日卒。

王國隆聘妻餘,懷遠人。國隆游不歸,或言在含山,餘父母挈餘行求不遇,遂僑居焉。餘母死,從父灌園,紡績自活,恆以巾冪首,鄰女罕見其面。康熙二十八年,父死,斂畢,女自經。

韋思誠聘妻宣,廣德人。思誠遠行,母以貧,欲令改字,宣不可,遂歸夫家。慮有強暴竊伺,夜懸柝於床,微風柝有聲以警。一夕,語諸姑、姊,夢夫告以死。遂哀泣,不食死。

於天祥聘妻王,名秀女,祥符人。天祥嘗育於陽武王氏,王氏為娶妻,生子,妻死,還於氏。繼室以王,王未行,而天祥死,王父母秘不使知。久之始聞,力請奔喪,天祥喪已小祥矣。王請於陽武王氏,原得子天祥前室子,王氏靳不許。及大祥,具奠,即夕自經。於氏故有刈麥刀二,俄失其一,至是得諸王枕下。

方禮祕聘妻範,名二妹,建水人。幼事父可望孝,字禮祕,未行。禮祕父良佐死,妻改嫁蕭伸,居方氏,禮祕及其兄、妹皆死。範聞,哭之慟,請於父母歸方氏。居久之,聞姑詬伸,始知禮祕非良死,以質姑,姑內慚,不復言。範度事無證,禮祕冤不得白,恆時時號痛。伸憚範,欲以妻其從子,百方強之,範不許。伸怒揮範僕,手點額。範怒曰:「奴汙我額!」刀剜伸手所點處,血淋漓被面。其弟訟諸吏,吏笞伸,以其室屬範,使奉方氏祀。

姚世治聘妻陳,會稽人。兩家皆居京師。既定約,世治歸,陳父欲別嫁,陳易服行求世治,遇諸濟寧。曰:「女違父非孝,得見君子,事畢矣!」遂入水死。

何秉儀聘妻劉,昆明人,農家女也。秉儀卒,女請於父母,欲奔喪,不許。乃竊出,兄追及之,度金汁河,將赴水,兄力持曳以歸。秉儀父使迎女,女哀慟泣血,日夕力作。父母畀田四畝,女為夫弟婚鬻半,喪舅又鬻半。父母怒,使告姑,誣女有所私,當遣之嫁。姑以責女,女不能自白,心疾作,縊死。

沈之螽聘妻唐,之螽,普安人;唐,武進人。之螽父文鬱,唐父元聲,康熙季年,同游高州,相友善,約為婚姻,於是唐生三年矣。元聲卒,喪歸,文鬱亦還普安。普安去武進且萬里,而文鬱貧,慮不能為之螽娶,詭言之螽殤,使謝唐,唐矢死。久之,文鬱將如京師求官,迂道至常州,唐出拜,涕泣慷慨陳所志。文鬱心悔,則請為養女,期得官迓以歸。既,文鬱以病還,唐聞大慟,遂不食,七日竟死。後三十餘年,之螽以事過常州,始聞唐死狀,感痛求其墓,已火葬矣。唐死時年十六。

貝勒弘暾聘妻富察氏,弘暾,怡親王允祥第三子。上命指配富察氏,雍正六年,未婚卒。富察氏聞,大慟,截發詣王邸,請持服,王不許;跪門外,哭,至夕,王終不許,乃還其家持服。越二年,王薨,復詣王邸請持服,王邸長史奏聞,上命許之。諭王福晉收為子婦,令弘暾祭葬視貝勒例,以從子永喜襲貝勒。諭謂:「俾富察氏無子而有子,以彰節女之厚報焉。」

濰上女子,不知其氏,雍正間,濰田家女也。未行而夫死,其母往吊,女請從,母止之不可。衣紅而襲以素,濰俗婦吊喪不至殯,女陽為如廁,因問得殯室,潛入,去襲,縊柩側。

吳某聘妻林,漳浦人。未行,夫坐罪當死,林欲入獄與訣,夫丐獄卒勿納,林晝夜哭不食。夫使畀以錢三百,且曰:「速擇佳婿,毋自苦!」越日,聞夫已決,以所畀錢易絙縊。

雷廷外聘妻侯,南安人。廷外母黃,早寡,貧,慮不能娶,乞貧家女撫之,期長以為婦,故侯四歲而育於黃。十一黃卒,十六廷外卒,死而不瞑,侯慟屢絕。廷外有從兄,以其子震

為後,侯乃笄,抱以拜祖。侯母欲令別嫁,拒以死。身自耕,跪而耨,十指皆胼。嘗誡震曰:「婦人不可受人憐,況孀乎!」震亦早卒,其妻傅,從姑織席以育子。

程樹聘妻宋,名景衛,長洲人。樹十三補諸生,喪母,復喪大父,旋亦卒。景衛年二十,請於父,歸程。以素服拜舅,見於廟;謁其大父喪,成孫婦服;謁其母喪,成婦服;乃哭其夫,持服三年;終,復補行姑服三年。同縣陳氏女淑睿,未行而婿殤,有請婚者,遂自經。景衛為作詩,於詩共姜用劉向說,於春秋伯姬用何休說,旁採硃彞尊、汪琬、彭定求諸家言,申女子子未嫁守貞之義。貫穿賅洽,八百餘言,以破俗說,白己志。景衛通經義,好讀先儒論學書,娣、侄皆從講說。病女教不明,乃會通古訓,括聖賢修身盡倫之要,復作詩九百餘言,授娣、侄,令歌習之。

張氏子聘妻姜,名桂,元和人。年十九,婿與舅、姑先後卒,依其母以居,不嫁。

錢氏子聘妻王,吳人,亦年十九而婿卒,女絕食,大父母強起之。居三年,有請婚者,復絕食,死復蘇。母哭之,女曰:「先年兒私吞金環不死,食銀硃又不死,頃復吞金環。兒死原得葬錢氏之兆。」遂卒。

王志曾聘妻張,亦吳人。年二十,志曾卒。居六年,聞姑喪,因歸於王,奉佛以終。

三女皆與景衛同時,而桂能詩善畫,嘗為柏舟圖,賦詩贈景衛。

景衛有二婢:曰衛喜,字於張,張死,不更字;曰陳壽,嫁硃氏,寡,無子。皆依景衛以老。

李家勛聘妻楊,海寧人。楊富而李貧,家勛父為楊氏佃。楊父行田,見家勛慧,問之,九歲,使入所立塾,資令讀。年十五入學為諸生,家勛父來謝,楊年十四,呼令出拜。楊母及兄皆恚曰:「是老顛!豈患女無家,而棄諸佃人子乎?」父旋卒,楊氏之人薄家勛。一夕,呼燈,無應者,楊自帷言曰:「丈夫不自處高明,何依人受慢為!」家勛遂辭楊氏去。乾隆十五年,舉浙江鄉試,楊氏請婚,家勛以試禮部辭。留京師數年,病卒。楊知母將為議婚他氏,請於母:「原得迎家勛喪,臨奠,然後聽母。」母許之。楊迎喪於郊,奠竟,要母,遂歸李氏。家勛父老而瞽,楊請於姑,為買妾生子。家勛父八十,目復明,德楊甚,命其子呼「嫂母」也。楊或曰徐氏。

李家駒聘妻硃,高安人,大學士軾女。家駒,乾隆三十六年舉人,早卒。硃事父母孝,性和以肅,自諸弟妹及內外臧獲,咸敬憚之。生惡華採,寸金尺帛不以加身。及聞家駒訃,欲奔喪,飲泣不食。時軾督學陜西,大母喻其意,誡當待父命,始復食。軾還,越半載,乃以請,遂歸於李。事祖姑及姑,如事父母。軾有父喪,聖祖命奪情視事,疏請終喪,戚友或尼之。硃泣曰:「吾父不得歸,雖官相國,年上壽,猶無與也。彼姑息之愛何為者?聖主當鑒吾父之誠矣!」卒得請。鄰火且及,硃坐室中不肯出,曰:「死,吾分也!宋共姬何人哉?」姑破扃挾以避。病不肯藥,兩弟來省,曰:「吾死無恨,但恨不得終事吾父及吾舅姑!」又曰:「我生惡華採,寸金尺帛不以加身,死毋負我!」遂卒。

賈汝愈聘妻盧,汝愈,故城人;盧,德州人,協辦大學士廕溥女。汝愈卒,盧矢不嫁,賈氏迎以歸,為立後。

袁進舉聘妻某,天津梁進忠養女也。進忠負薪行水次,有大舟泊焉,或抱女嬰出,授進忠曰:「此女生八月矣,父之官,卒於舟,母繼殞,其善視之!」進忠撫以為女。而進忠有長女悍甚,女稍長,貌端好,長女將鬻以為人妾,女不可,長女益恚。進舉故無藉,長女咻父母使字焉。進舉行不歸,又使告其母謀罷婚,女復不可。進忠病,瘍生於脛,女刲股以療,家人皆不知,而長女虐愈甚。進舉母憐之,迎之歸。進忠及其長女皆死,女為營葬,迎義母進忠妻同居。長女有子,失所,召為鞠之。為進舉弟娶婦,生子為進舉後。終姑及其義母喪,女遂自經死。有司葬之天津西郭外五烈墓傍。

五烈墓者,先為三婦墓,葬譚應宸妻陳、阮某妻諸、趙某妻裴,陳、諸皆以捍強暴死,裴以節終。乾隆元年,金振妻丁殉夫,附葬,稱節烈四婦墓。七年,又有殷氏女誤嫁倡家,為所迫,箠楚砲烙,沃以沸湯,死,葬墓側,稱五烈墓。五十六年,復葬女,更為六烈墓云。

李應宗聘妻李,昆明人。所居曰廟前鋪大河埂,父春榮。未行,應宗卒。其明年,應宗大母語春榮,將改字女,女聞,遂縊。縊之夕,裂綾二尺許,刺血書九十四字。民家女未嘗讀書,字多訛易,嘉興錢儀吉為之句讀。曰「呈天子前」,曰「忠孝節烈」,曰「二月初九日」,二月初九日蓋女死日,事在乾隆末。

何其仁聘妻李,路南人。嘉慶十一年,年十六,未行。其仁及其父皆病篤,李割股畀叔母使送婿家。至,則其仁及其父皆已卒,其仁母燖以奠。李欲奔喪,母尼之,遂縊。

王前洛聘妻林,潛山人。前洛病,林父餽藥,林潛刲股入藥。前洛卒,固請奔喪,引刀誓不嫁。

節義縣主,成郡王綿懃第七女,選文緯為婿。文緯,費莫氏,內閣學士英綬子。未婚,嘉慶十八年文緯卒,主時年十六,詣文緯家守節,仁宗詔封節義縣主。二十二年,卒。

李承宗聘妻何,巢縣漁家女也。兩家居濱溪,相違半里餘,而李氏廬當上流。承宗卒,女年二十,請奔喪,父母不許。不食四日,不死;自經,或拯之。越日自沉於溪,求其尸不得。後三日,尸見溪上流,正值李氏門。

江亨昭妻楊,侯官人,二氏皆漁家。楊未嫁,與亨昭舟相值,必引避。或遇水次,則自匿蘆葦中。其母非之,女曰:「漁家獨不當有恥乎?」既嫁,強暴窺其有色,潛逼之,楊擠使墮水。亨昭死,殉焉。

吳某聘妻硃,海鹽人。吳某年十八,喪父母,遂出游不歸。硃貧,父老,闢纑織屨。其兄悍,屢辱之。硃曰:「兄貧不能食我父,我父衰,無所營,不得不就兄食。我留,乃助兄耳。」及父死,硃年五十八,吳不知其存亡,吳之族愍硃節,迎以歸,為立後。

徐文經聘妻姚,名淑金,侯官人。文經卒,淑金屢求死,乃歸於徐。貧,舅歿,姑疾作,刲股以療。姚掇芹供姑,自食其棄莖。無何,姑亦歿,嗣子以貧去。淑金目昏,不能治女紅,以缽為釜,以草為衾。僦屋不償值,見逐,泣路隅。有負擔者,憐而周之,里人醵金助衣食,僅得不死。猶朝夕拜徐氏祏,祝其嗣子歸也。居十餘年乃卒。

李煜聘妻蕭,秀水人。煜酒家子,居郭南萬螺濱。蕭未行,煜死。蕭無母,請於父,原歸李,翁姑遣媒止之,勿聽,遂歸李。視煜斂,即奉侍姑,執爨濯衣甚謹。姑悍,既不欲李來,又見其貧也,晝夜詈,李唯唯無一言,鄰勿善也。或勸姑,姑亦詈焉。士大夫眾至,誡翁:「毋虐貞女,貞女光爾門,宜善視之!」姑終不欲李同居,眾乃於室後闢小樓居貞女,醵金以佽之。

劉戊兒聘妻王,名孝,武陟人。未嫁,歲大無,戊兒行六年不歸。父母欲別嫁,孝間出,如劉氏。值老嫗,問劉戊兒母,嫗曰:「我即戊兒母也。」孝拜且泣曰:「我王氏女,姑之子婦也!」嫗驚未信,孝探懷出物示嫗曰:「此非姑家聘物耶?吾竊持以來為信。」嫗視之亦泣,復以貧無食辭。曰:「吾夙知姑貧,翁歿,兩叔幼,安得所食?我能女紅,茲固為養姑來也。生未嘗一時離吾母,計無所出而後來。」因復泣曰:「如不見容,我無歸理,惟赴水死耳!」嫗告孝父母許焉。孝勤紡績,夜磨作蒸餅,使叔鬻之。姑病,日夜侍。居數年,鄉里感其義,率錢周其姑。葺舊屋,為叔娶婦生子。姑卒,合葬於舅墓,乃授家事於叔,夜入室,扃戶,寂無聲。翌晨叩戶不應,毀牖入,則自經死,衣履皆易新制者。時嘉慶九年二月乙酉。孝年二十四至劉氏,事姑十二年,姑死乃死。

硃某聘妻李,字容,東安人。父大純,幼字硃氏。硃氏子遠游十餘年不歸,或傳已死。女既喪父母,無昆弟,獨與其婢春華居,誓不嫁。春華稍長,其父謀嫁之,春華義不去,容亦誓不嫁。其父不聽,春華乃告容,俱赴水死。

武稌聘妻李,伊陽人。年十一,喪母,育於武。從娣婦事舅姑謹,姑羸臥,調醫藥,治家事日勤。姑卒,撫叔弟及二女妹。年十七,猶未婚。稌墮井死,誓從井,舅止之,幼弟妹環而哭,李大慟。遂總發為紒,曰:「吾當終婦事。」請於舅,立後,紡織以佐家。舅娶後姑,又有疾,調醫藥,治家事如前時。久之,叔弟補縣學生,兩女妹皆嫁。又數年,為所後子娶婦,則語其兄曰:「妹曩不即死,誠不敢死也。今吾家奉舅姑宗祏幸有人,井中人待我久,我將從之!」晨起,從容問姑安,出行汲,自投稌所墮井死。道光二十一年八月壬寅,稌生日也。

後稌死二十有一年。

陳霞池聘妻錢,桐城人,居東鄉。未行而霞池卒,錢請奔喪。東鄉俗以為子死婦奔喪,於家兇,辭之。錢毀容矢不嫁。久之,陳氏之族迎以歸,為立後。居數十年,縣有士人往存問,為言:「朝廷旌貞女,與節烈並重,當請於有司。」錢聞大驚,蓋初不知其行應旌也。

汪榮泰聘妻唐,名鳳鸞。榮泰,歙人;唐,淳安人。父以許榮泰,未聘而父卒,母更許他姓。他姓來聘,唐自所居樓裂所制衣履擲於庭,俄砉然躍而出,遂墮地死。榮泰請迎喪,母不許;母卒,乃迎喪以歸。

季斌敏聘妻藺,斌敏,正藍旗漢軍;藺,滄州人。斌敏未婚卒,藺年十八,矢不嫁。居二年,聞有媒妁至,截右耳,逾三日,又截左耳。其父春以告季氏,迎以歸。女事姑甚孝,為夫補行喪服。喪終,歸訣父母,謂當死從夫,父母力勸喻之。女復還,見姑,言笑如平時,即夕飲毒死。啟篋封所割兩耳,識曰「全歸」。

董福慶聘妻馮,福慶,固安駐防漢軍;馮,霸州人也。福慶貧,餓猶耕,死於田。女年二十,請奔喪,福慶父往沮之,曰:「子餓至死,復忍餓汝家女耶?」女出拜,伏地哭不起,福慶父乃諾之,遂奔喪。執婦禮以終,寒餒皆無懟。

喬湧濤聘妻方,桐城人。湧濤卒,湧濤母丁亦病,方請於父母,歸於喬。以姑病寒疾,亦薄其衣當風雪。刲股以進姑,病良已。乃營葬湧濤,以衣負土,三日不食。為湧濤立後,淡食布衣,深自刻苦。病將革,戒子婦毋以寸絲斂。

張氏女,名有,鄒平人。歲饑,鬻為高唐硃氏婢。及長,主母為議婚,有泣言幼已字人,不敢負。主母使求得所許字者,則已別娶有子女矣。以語有,有曰:「雖別娶,身不原更事他人。」主母憐而聽之。有終不別字以死。

粉姐,失其姓,高郵人。父為迮氏蒼頭,字某氏子。歲饑,某氏子行乞,轉徙十餘年。女父遇之江都市上,某氏子曰:「我終不能娶,還我聘錢,聽別嫁。」女父喜,還聘錢,與析券。歸告女,女嗚咽不語,夜自經。

闞氏女,名玉,浙江仁和人。玉端麗,能詩文。父亡,與母及兄嫂居。年十三,福王由崧帝南京,選民間女子,玉母匿諸賣菜傭家。玉父亡時,留百金畀玉兄備玉嫁,玉兄蕩其貲,遂與傭謀字傭子。玉在傭家尚待年,號泣求還,不可得,疾作,始遣歸。玉垂絕,語其母曰:「兒今且死,原埋父棺側,不作傭家鬼也。」復嚼齒曰:「兄陷我!」遂卒。

玉嘗作怨歌,好事者以琴譜其聲,曰闞玉操,辭曰:「父生我兮中道逝,母煢煢兮門衰瘁。兄嫂難與居,抉我如目中之塵沙。伊又遘此佻巧兮,胡迋我之實多。彼六禮之已愆兮,曾貞女之貺從。矧要予以桑中兮,夫豈其為予之匹雙。我有母兮,癙思泣血。我父而有知兮,怒沖發。我兄摩挲傭之金兮,骨肉相蔑。嫂旁睨兮,笑言啞啞。我忽憤氣兮,如云。指漆室女以為正兮,又告夫司命與湘君。予不愛一死兮,弗忍速阿母之下世。原死而有憑兮,為兇之厲。嗚呼哀哉,我終死兮,魂獨歸去。明告我母兮,幽告我父。匪我夙夜兮,胡然遭此行露也。縱謂行多露兮,寧能我之汙也。重曰:嘉名為玉,父之命兮。幽辱糞壤,終保貞兮。憂思悄悄,淚淫淫兮。蒙恥忍詬,日當心兮。」

趙氏婢,失其名,為杭州趙氏婢。趙氏嘗有客,言珞琭子之學,使為婢算,曰:「是當七易其夫。」婢恚曰:「吾嫁則有夫,有夫則有死。吾今且不嫁,誰為之夫者?」自是蓬首垢面,矢不嫁。趙氏有婚嫁輒避匿,媒氏至,詬誶不可近。主誨之,搶首乞終役。年至七十餘,死於趙氏。


\end{pinyinscope}