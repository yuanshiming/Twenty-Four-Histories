\article{志一百五}

\begin{pinyinscope}
○兵一

有清以武功定天下。太祖高皇帝崛起東方,初定旗兵制,八旗子弟人盡為兵,不啻舉國皆兵焉。太宗征籓部,世祖定中原,八旗兵力最強。聖祖平南服,世宗征青海,高宗定西疆,以旗兵為主,而輔之以綠營。仁宗剿教匪,宣宗御外寇,兼用防軍,而以鄉兵助之。文宗、穆宗先後平粵、捻,湘軍初起,淮軍繼之,而練勇之功始著,至是兵制蓋數變矣。道、咸以後,海禁大開,德宗復立海軍,內江外海,與水師並行。而練軍、陸軍又相繼以起,擾攘數年,卒釀新軍之變。以兵興者,終以兵敗。嗚呼,豈非天哉!今作兵志:一曰八旗,二曰綠營,三曰防軍,附陸軍,四曰鄉兵,五曰土兵,六曰水師,七曰海軍,八曰邊防,九曰海防,十曰訓練,十一曰制造,十二曰馬政,並分著於篇。

八旗

清初,太祖以遺甲十三副起,歸附日眾,設四旗,曰正黃、正白、正紅、正藍,復增四旗,曰鑲黃、鑲白、鑲紅、鑲藍,統滿洲、蒙古、漢軍之眾,八旗之制自此始。每旗三百人為一牛錄,以牛錄額真領之。五牛錄,領以札蘭額真。五札蘭,領以固山額真。每固山設左右梅勒額真。天命五年,改牛錄額真俱為備御官。天聰八年,定八旗官名,總兵為昂邦章京,副將為梅勒章京,參將為甲喇章京,各分三等。備御為牛錄章京。什長為專達。又定固山額真行營馬兵為阿禮哈超哈,其後曰驍騎營。巴雅喇營前哨兵為噶布什賢超哈,其後曰護軍及前鋒營。駐防盛京兵為守兵,預備兵為援兵。各城寨兵為守邊兵。舊蒙古左右營為左右翼兵。舊漢兵為烏真超哈。孔有德之天祐兵,尚可喜之天助兵,並入漢軍。九年,以所獲察哈爾部眾及喀喇沁壯丁分為蒙古八旗,制與滿洲八旗同。崇德二年,分漢軍為二旗,置左右翼。四年,分為四旗,曰純皁、曰皁鑲黃、曰皁鑲白、曰皁鑲紅。七年,設漢軍八旗,制與滿洲同。世祖定鼎燕京,分置滿、蒙、漢八旗於京城。以次釐定兵制。

禁衛兵大類有二:曰郎衛,曰兵衛。郎衛之制,領侍衛內大臣六人,鑲黃、正黃、正白旗各二人。內大臣六人。散秩大臣無定員。侍衛分四等。更有藍翎侍衛。凡御前侍衛、乾清門侍衛由三旗簡用,漢侍衛由武進士簡用,皆無定員。初,鑲黃、正黃、正白三旗,天子自將,選其子弟曰侍衛,凡值殿廷,以領侍衛內大臣統之。宿衛乾清門、內右門、神武門、寧壽門為內班,宿衛太和門為外班。行幸駐蹕咸從。其扈從,後扈二人,前引十人,豹尾班侍衛六十人。凡佐領親軍,鑲黃旗滿洲八十五佐領,蒙古二十八佐領,每佐領親軍二人;正黃旗滿洲九十三佐領,蒙古二十四佐領;正白旗滿洲八十六佐領,蒙古二十九佐領。三旗親軍選六十人隨侍衛行走,餘皆值宿。巡幸則御前大臣侍衛、乾清門侍衛咸從。行營則列兩廂,餘於幔城之隅,環拱宿衛。康熙二十九年,以武進士技優者拔置侍衛,偕三旗值宿。雍正十一年,以親軍未滿十年者,挑選前鋒。滿、漢八旗左右翼各設前鋒統領一人,備警蹕宿衛。侍衛班內有上駟院侍衛,司轡、司鞍。其兼尚虞、鷹鷂房、鶻房、十五善射、射鵠、善撲等侍衛,統在三旗額內,俱無定員。鑾儀衛亦侍從武職。設掌衛司內大臣一人,鑾儀使三人,冠軍使十人,雲麾使、治儀正、整儀尉各有差,專司乘輿鹵簿。校尉由內府選者為旗尉,由五城選者為民尉。此八旗郎衛制也。

兵衛之制,定鼎初,即以上三旗守衛紫禁宮闕,以護軍統領、參領、前鋒統領率之。噶布什賢超哈滿洲、蒙古八旗分左右翼備宿衛。內務府三旗,各設佐領三人,旗鼓佐領四人,正黃旗設朝鮮佐領一人,每二丁設馬甲一,每佐領各設領催六、護軍十五,以領侍衛內大臣率之。內務府官兵守護行宮者,分東西北三路,設千總等官、兵額不等。熱河行宮亦如之。其守護陵寢者,順治初,永陵、福陵、昭陵各設雲騎尉、騎都尉。嗣後盛京三陵,增設總管、防禦、驍騎校。京師東西陵制亦如之。所屬各旗驍騎有差。八年,制定親王至輔國公等,以次設長史、護衛等官。十七年,定八旗漢字官名,固山額真曰都統,梅勒章京曰副都統,甲喇章京曰參領,牛錄章京曰佐領,昂邦章京曰總管,烏真超哈曰漢軍。凡滿、蒙、漢各旗共選四千八百人為養育兵,訓練技藝。嗣後兵額屢增。乾隆中,滿、蒙養育兵至二萬三百餘人。盛京打牲烏拉,設總管、協領、佐領等員,轄打牲兵丁。吉林之參戶,蜜戶,漁戶,獵戶,鷹、狐、獺、鸛諸戶,咸隸內府三旗。其巡捕營汛守外七城門,上設步兵汛二十五所,城外分中南北三營,馬步兵汛額各有差,統以參將、游擊等。暢春、圓明、靜明等園守兵,統以守備。康熙初,定駐蹕之地,八旗護軍分左右翼巡宿,啟蹕則三旗營總、護軍參領隨行。十三年,定八旗步兵二萬一千餘名,鳥槍步兵凡千七百三十七名。又定內九門外七門設城門校,轄十六門門軍。其步軍營汛守皇城內各汛專用滿洲,城外各汛兼用蒙古、漢軍。尋定上駐園,則八旗兩翼,翼分七汛,更番宿衛。每日當值之前鋒、鳥槍護軍共七百二十人。二十一年,定田獵每年三舉,八旗各簡前鋒軍校以從。二十二年,定車駕巡幸期。八旗驍騎營於內外城並增汛所。二十三年,以黑龍江所進精騎射、善殺虎者編虎槍營。三十年,設火器營。雍正元年,設巡捕營,馬兵汛十五,步兵汛五十二。凡朝會期,協尉、副尉率步軍巡警。二年,諭各旗共選四千八百教養兵,習長槍挑刀各藝。四年,令八旗前鋒習射,月六次。其專司防火者曰防範兵。九年,令五旗門汛護軍、馬甲均歸本營操演。令三旗增訓練兵二千,編為二營。十三年,額定馬甲五千二百五十,春秋二季合操。乾隆十四年,設雲梯兵一營。又於昆明湖設趕繒船,以前鋒軍習水戰。二十五年,令來京回人編一佐領,以和卓為佐領統轄之,後皆準此。三十九年,定大閱頭隊前鋒八旗,分為八隊,每隊小旗八,海螺四,為殿後兵。四十一年,以來京之番子視回人例,編一佐領,統於內務府正白旗。四十六年,增京師步軍左右二營,合南北中為五營,分二十三汛,領兵一萬,於八旗漢軍鄂爾布、步甲、閒散內擇壯丁充補。嘉慶四年,令巡捕五營以中營作提標,管圓明園五汛,參將四人,分管南北左右四營,共十八汛,兩翼總兵分轄之。十七年,以增設之健銳營歸左翼,外火器營歸右翼,合八旗前鋒、內火器營、驍騎營凡三十六營。咸豐三年,諭京師各旗營兵十四萬九千有奇,統兵大臣分班親閱,馬步火器,務令精整,不得以臨時召募濫充。十年,從勝保請,令八旗兵加練槍砲抬槍。同治四年,諭醇親王訓練神機營,旗、綠各營,亦隨時校閱。光緒二十四年,選練神機營馬步隊,以萬人為先鋒隊,習槍砲及行陣戰法。此八旗兵衛制也。

八旗駐防之兵,大類有四:曰畿輔駐防兵,其籓部內附之眾,及在京內務府、理籓院所轄悉附焉;曰東三省駐防兵;曰各直省駐防兵,新疆駐防兵附焉;曰籓部兵。

畿輔駐防兵制,順治初,獨石口、張家口、山海關、喜峰口、古北口並設防御一人或二人,採育里、固安縣設防守尉、防禦有差。康熙十四年,察哈爾八旗,每旗設總管一人,副總管一人,參領三人,佐領、驍騎校、護軍校各有差。捕盜官每旗二人,親軍、前鋒各二,護軍十七,領催四,驍騎二十五。在京蒙古都統兼轄之。山海關總管一人,防禦八人,滿、蒙、漢兵七百有奇。尋設張家口總管一,防禦七,兵百三十有奇。獨石口、古北口增防御各二,喜峰口防禦二,冷口、羅文峪防御各一,兵多則六十八,少則十二人。雍正三年,設天津水師營都統一,協領六,佐領、防禦、驍騎校各三十二,旗兵千六百人,蒙古兵四百人,分左右兩翼。乾隆三年,增熱河駐防兵二千人,委前鋒校、前鋒、領催、鳥槍領催、馬甲、鳥槍馬甲、砲甲、弓矢匠各有差,以千四百人駐熱河,四百駐喀喇河屯、二百駐樺榆溝。八年,改山海關總管為副都統,增協領、佐領諸屬,滿、蒙、漢兵共八百人,分左右翼。二十六年,設察哈爾都統一人,駐張家口,理八旗游牧,兼轄防兵,副都統二人,駐左右翼游牧邊界。四十五年,設駐防密雲滿、蒙兵二千。嘉慶三年,增熱河圍場副都統。九年,改總管。十五年,改設都統一人。以厄魯特達什達瓦降眾徙居科布多,旋分其屬為三旗,設總管、副總管、佐領、驍騎校等。尋移至熱河,作為官兵。先是康熙中,建避暑山莊於熱河,設總管、守備、千總分守各行宮。乾隆間,增建行宮,設千總、委署千總一二人,兵自六人至九十八人不等。木蘭圍場總管一人。康熙季年,設有防禦八及滿、蒙兵百餘。迨乾隆中年,增左右翼長二,驍騎校八,駐兵共八百人。每一兵給地一頃二十畝,或地不宜耕種,則改給牛羊。木蘭之地,周遭樹柵為界,設營房八,卡倫四十,八旗各分五卡倫,各以旗兵守之。道光四年,諭駐京旗兵,遇閏月賞給甲米,他省不得援例。此畿輔駐防制也。

東三省駐防兵制,共駐四十四所,兵三萬五千三百餘人。凡前鋒、領催、馬甲、守門庫等兵,步甲、夜捕手、匠役、養育兵、鳥槍馬甲、領催、水手之屬,或設或否,名額多寡,各視駐地所宜,損益區置之,初無定限。

其在盛京,天聰間始設駐防於牛莊、蓋州,兵九十六人。順治元年,世祖將遷燕京,設盛京八旗駐防兵,以正黃旗內大臣和洛會總統之,以鑲黃旗梅勒章京統左翼,正紅旗梅勒章京統右翼。每旗設滿洲協領一,佐領四,蒙古、漢軍佐領各一。設熊嶽城守官,其下滿洲佐領三,漢軍佐領一,錦州、鳳凰城、寧遠城守官,其下各設滿洲佐領各二,漢軍佐領一,興京、遼陽、牛莊、岫巖、義州城守官,滿洲佐領各一人,蓋州、海州滿、漢佐領各一,統駐防兵。康熙元年,改盛京昂邦章京為鎮守遼東等處將軍,梅勒章京二人為副都統,統轄協領、佐領、驍騎校。四年,改遼東將軍為奉天將軍。十四年,設錦州、義州城守尉各一,佐領、驍騎校各有差。各邊門皆置防御一。尋設開原防禦三,金州防御一,兵弁各有差。五十五年,設金州駐防水師營,船十號,兵五百,水手一百。雍正五年,設熊岳副都統一人,廣寧、義州、錦州、寧遠至山海關設副都統一,復州、南金州、鳳凰城、岫巖、旅順等處設副都統一,分轄旗兵。乾隆十二年,改奉天將軍為鎮守盛京將軍。盛京各額兵都一萬五千有奇。

其在吉林,順治十年,設寧古塔昂邦章京一,梅勒章京二,佐領、驍騎校各八。十八年,設吉林水師營。康熙元年,改寧古塔昂邦章京為將軍,梅勒章京為副都統。三年,設水師營總管各員。七年,增寧古塔協領二。十年,以寧古塔副都統一,佐領、驍騎校各十一,兵七百,移駐吉林。又增吉林協領八,佐領、防禦、驍騎校各十二,兵六百人。尋增防禦十五人。十五年,移寧古塔將軍駐吉林,留副都統於寧古塔,增吉林副都統一人。三十一年,設伯都訥協領二人,佐領、驍騎校各三十,防禦八。五十三年,設三姓、琿春協領一,佐領、驍騎校、防禦有差。雍正三年,設阿勒楚喀協領一人,佐領、驍騎校、防御各五。十年,設三姓副都統一人。尋設吉林鳥槍營參領一人,佐領、驍騎校各八,鳥槍兵千。乾隆十三年,令打牲烏拉兵歸吉林將軍兼轄。先是順治時,設打牲烏拉協領二,又設總管一人,統轄珠軒頭目,及參、蜜、漁、獵諸戶,專司採捕諸役。後遞增佐領、防禦八,驍騎校十或八,額兵千。至是以在吉林境,命兼統於吉林將軍。二十一年,設阿勒楚喀副都統一人。道光六年,以雙城堡移駐京旗分左右翼,各設總、副屯達二人。嗣又分一旗五屯,增總、副屯達各六人。

其在黑龍江,當康熙初年,自吉林移水師營來駐齊齊哈爾等處,水手一千有奇。盛京壯丁散處者,隨時編入八旗。巴爾呼人、錫伯人居近吉林,卦勒察人居近伯都訥,庫爾喀人居近琿春,並設佐領、驍騎校等分駐。其東北最遠者,索倫、達呼爾二部,天命、天聰間,相率內附,其後分充各城額兵。至鄂倫春所居益遠,使馬、使鹿部分處山林,業捕貂,皆審戶比丁,列於軍伍。二十二年,初置黑龍江將軍,原水師營總管等並屬之,設副都統二,協領四,佐領、驍騎校各二十四,防禦八,滿洲兵千,索倫、達呼爾兵五百,駐愛琿城。二十三年,設打牲處總管一,副總管二,以索倫、達呼爾壯丁編設佐領、驍騎校。尋於墨爾根城設駐防兵。二十九年,移將軍駐墨爾根,又增協領四,佐領、驍騎校各七,索倫、達呼爾兵四百餘,以副都統一人統兵駐愛琿。尋設兵千餘駐防齊齊哈爾。三十八年,將軍復自墨爾根移駐齊齊哈爾。四十九年,設墨爾根副都統一人。雍正六年,增設打牲處總管三,滿洲、索倫、達呼爾副總管十六,索倫、達呼爾佐領、驍騎校各六十二。十年,設呼倫貝爾統領一,索倫、巴爾呼總管、副總管各二,佐領、驍騎校各五十,兵三千,尋增兵二千有奇。厄魯特總管、副總管各一。乾隆八年,改呼倫貝爾統領為副都統。嘉慶九年,以齊齊哈爾等處承種官田馬甲歸各本旗,所墾新田,改增養育兵耕種。咸豐八年,增黑龍江馬甲千。光緒八年,將軍文緒請由黑省至茂興設七站,由茂興至呼蘭設五臺,共臺站六十人,置掌路記防御一,驍騎校二,領催六,分隸鈐束。黑龍江八旗兵約分五類:曰前鋒,共百四十六人,佩橐鞬,負旗幟,為先導;曰領催,供會計書寫,馬甲之長也,共七百四十八人;曰馬甲,又稱披甲,共九千二百十三人;曰匠役,為鳥槍、弓、鐵、鞍諸匠,共一百五十二人;曰養育兵,康熙季年,始以旗兵屯田,至嘉慶中,改屯田馬甲為養育兵,共八百人。別有未入伍者曰西丹,譯言控馬奴,不得預征伐之事。此東三省駐防制也。

各直省駐防制,順治二年,始設江南江寧左翼四旗,陜西西安右翼四旗,皆置滿、蒙兵二千,弓匠二十八,鐵匠五十六。六年,於山西太原設正藍、鑲藍二旗滿、蒙駐防兵,暨游牧察哈爾兵。初,太宗親征察哈爾,降土默特之眾,後編為二旗,設左右翼,都統部眾得同辦事。旋裁都統,以旗務掌之將軍、副都統,與內八旗等。至是,游牧察哈爾遂列於山西駐防。十一年,設山東德州鑲黃、正黃二旗滿、蒙領催、馬甲暨弓、鐵匠。十五年,增設西安佐領、驍騎校二十八,驍騎一千。設浙江杭州滿、蒙八旗馬甲、步甲、弓匠,漢軍馬甲、步甲、鐵匠,滿、漢棉甲兵,共四千有奇。其後每旗並增佐領、驍騎校、驍騎。十六年,改設京口駐防鎮海將軍一,副都統二,協領、參領、防禦、佐領、驍騎校有差。尋增江寧、西安步甲各一千。

康熙十三年,增西安右翼四旗滿、蒙馬甲千,弓、鐵匠十四,漢軍馬甲等,江寧馬甲千。後又各增兵二千及弓、鐵匠等。是年增京口步甲千人。十五年,設陜西寧夏八旗滿、蒙領催,馬甲,步甲,弓、鐵匠。十九年,設福建福州左翼四旗漢軍領催、馬甲、步甲、鐵匠,及滿、蒙步甲。二十年,設廣東廣州鑲黃、正黃、正白上三旗漢軍領催、馬甲、砲甲、弓匠。二十二年,設湖廣荊州八旗滿、蒙領催,馬甲,步甲,弓、鐵匠,共二千八百有奇,尋增至四千人。是年又增西安將軍,增滿洲左右翼副都統各一,漢軍左右翼亦如之,八旗滿、蒙協領各八,漢協領、佐領、防禦、驍騎校不等,滿、蒙、漢兵共七千,滿、蒙步軍七百,暨弓、鐵匠等。二十三年,續設廣州鑲白、正紅、鑲紅、正藍、鑲藍五旗漢軍兵,設將軍一人,副都統二,協領、參領各八,防禦、驍騎校各四十,八旗鳥槍領催、鳥槍驍騎、領催、驍騎、砲驍騎、弓、鐵匠共三千有奇,兼置綠旗左右前後四營,將領八,兵三千四百有奇。尋於福州、荊州、寧夏、江寧、京口、杭州並分設鳥槍領催、鳥槍驍騎、領催、驍騎各有差。京口步軍內兼設鳥槍、弓、箭、長槍、藤牌等兵額。是年增設杭州駐防八旗滿、蒙、漢兵共三千二百人。三十二年,設山西右衛八旗滿、蒙、漢護軍、領催、馬甲、鐵匠共五千六百有奇,以將軍統之,設隨甲四十八,筆帖式六。三十六年,裁京口綠旗水師總兵,改設京口副將,分左右二營,設游擊以下將領八人,兵一千九百人。五十九年,設河南開封滿、蒙領催,鳥槍領催,馬甲,鳥槍馬甲,弓、鐵匠。六十年,設四川成都副都統一,協領四,佐領、防禦、驍騎校、鳥槍領催、鳥槍驍騎、驍騎暨步軍,弓、箭、鐵匠。

雍正元年,福州駐防漢軍步兵悉改馬兵。二年,增太原、德州駐防兵各五百人。六年,設福州駐防水師營協領一人,佐領、防御各二,驍騎校六,水師五百。七年,設駐防浙江乍浦水師營。設青州駐防將軍、副都統各一人,協領四,佐領、防禦、驍騎校十六,暨八旗滿、蒙兵弓、鐵匠。設廣州駐防水師營協領一人,佐領、防御各二,驍騎校、八旗漢軍水師領催有差。八年,以各省駐防漢軍營伍廢弛,令所在將軍訓練之。設駐防青州八旗滿洲兵二千人。增右衛駐防兵五百人,自將軍及兩翼副都統以下,設協領,佐領,防禦,驍騎校,滿、蒙前鋒,滿、蒙、漢領催等,及驍騎三千有奇。十三年,設甘肅涼州八旗滿、蒙、漢兵凡二千人。設駐防莊浪八旗滿、蒙、漢兵凡千人。

乾隆二年,設駐防綏遠城,以征準噶爾之滿、蒙、漢開戶家丁二千四百,熱河駐防兵千,及右衛蒙古兵五百,凡三千九百人。設涼州將軍、副都統各一人,滿、蒙、漢佐領、防禦、驍騎校、步軍尉及八旗驍騎二千人,步軍六百人。又設莊浪駐防副都統一人,滿、蒙、漢協領、佐領、防禦、步軍尉及八旗驍騎一千人,步軍四百人。四年,改寧夏駐防步甲六百為養育兵。增荊州養育兵四百人。十年,設江寧駐防養育兵。二十一年,定開封城守尉歸巡撫統轄。二十二年,裁京口將軍,以綠旗左右營改隸江寧將軍。二十五年,改綏遠城將軍駐防兵額,步軍、養育兵各四百,共領催、前鋒、驍騎實二千四百人。二十八年,以土默特二旗歸綏遠城將軍統轄。設歸化城副都統一人。三十九年,改杭州駐防步軍一百二十八人為養育兵。四十一年,設成都駐防將軍一人。四十九年,增西安副都統一人。嘉慶十二年,飭各將軍不得以老弱充兵額。此各直省駐防制也。

新疆駐防兵制,乾隆二十五年,始議於新疆設兵駐守。命阿桂率滿洲、索倫驍騎五百,綠營兵百,回人二百,至伊犁搜捕馬哈沁,招撫厄魯特,並築城屯墾。其後陸續由內地增調屯田兵至二千五百人,五年更替,以五百人差操,二千人屯種,分二十五屯,設屯鎮總兵。其明年,阿桂奏定卡倫侍衛十五人,增伊犁駐防馬兵千五百,合原額兵凡二千五百人。二十七年,以涼州、莊浪駐防兵五千,並戶口移駐伊犁。旋以新疆底定,設駐防兵制。凡卡倫兵以侍衛領之,屯田兵以督屯武職領之,駐防馬兵以佐領領之,綠旗兵以營員領之,而特設將軍為之總轄。侍衛、章京等皆按年番替。二十九年,調綠營兵千,在伊犁河岸築惠遠城。其管理築城兵,設副將一,守備二,千總二,把總八。以察哈爾移駐兵一千八百戶編兩昂吉,領隊大臣統之,設十二佐領,分左右二翼,每佐領設兵二百。以黑龍江移駐戶千編一昂吉,設六佐領,領隊大臣統之。又撥錫伯兵、熱河滿、蒙兵各一千,及達什達瓦厄魯特兵五百,俱攜眷駐伊犁。定馬兵永遠駐守,綠旗兵五年番換。三十年,以投出之厄魯特人編一昂吉,與達什達瓦部眾俱為厄魯特昂吉,以領隊大臣統之。原厄魯特兵作厄魯特右翼。自領隊大臣以下,二三等侍衛、藍翎侍衛無定員。三十一年,定烏魯木齊駐辦事大臣及協辦大臣,統駐防兵及工作官兵,置經理新疆貿易、稽察卡倫臺站各官。三十二年,定左翼厄魯特六佐領為上三旗,右翼厄魯特共十佐領編為下五旗。三十四年,增惠寧城滿兵領隊大臣一人。三十七年,以投誠之沙畢納爾人等歸入下五旗厄魯特,增設四佐領統之。嘉慶二十年,於沙畢納爾四佐領內增副總管一人。道光十年,以惠遠城滿兵四千六百有奇,巴燕岱滿兵二千一百有奇,諭將軍等不得議增兵額。同治六年,以哈薩克人東犯,飭李云麟訓練厄魯特、蒙古兵以防之。增布倫托海辦事大臣,督率喇嘛,建署治事,並設幫辦一人。此屬新疆北路者也。

其在南路防兵,烏什駐總理回務參贊大臣、協辦大臣各一人,統轄滿洲、綠旗及屯田各官兵,兼轄阿克蘇、賽裏木、拜城各駐防兵。所屬有侍衛、章京等官。滿洲營領隊侍衛二,駐轄翼長、參領各一,副參領、委署參領各二,前鋒校六,綠旗營游擊以下、屯田副將以下各十八人。阿克蘇駐章京一,綠旗營游擊一。賽裏木駐翼長一,兼統拜城駐防。葉爾羌駐辦事參贊大臣及領隊大臣,統轄滿洲營領隊副都統、侍衛、參領、副參領等,如烏什例。和闐、喀什噶爾並駐辦事大臣及領隊大臣,統轄滿洲營侍衛、章京、領隊侍衛、參領、副參領等,暨綠營總兵、參將等官。庫車駐辦事官,統轄綠營都司以下官,兼轄沙雅爾事。哈喇沙爾駐辦事官,統轄綠旗營城守,及屯田駐防兵。闢展駐領隊大臣一,統協領、佐領以下暨步兵、綠旗兵。

乾隆二十四年以後,於烏什駐辦事大臣,阿克蘇駐辦事大臣、協辦大臣各一人,葉爾羌設辦事大臣二人,及章京、卡倫侍衛等。滿洲營設副都統一人,統健銳營前鋒參領、副參領等,安西滿洲營佐領五品官、索倫五品官、察哈爾佐領等,綠營總兵、游擊以下各官。又於和闐駐領隊總兵官及游擊以下。又喀什噶爾駐總兵、理回疆事務大臣、協辦大臣各一。滿洲營設副都統一,領隊侍衛二。領隊侍衛兼統索倫兵。索倫設委署副總管及佐領各二,察哈爾總管一,副總管二,及護軍校以下。綠營設提督及都司以下官。英阿薩爾駐領隊總兵官一,兼統索倫、察哈爾、綠旗兵。又於庫車、哈喇沙爾、闢展並駐辦事大臣。初臺站之改,屬闢展者凡六。每臺置外委千、把總一人。葉爾羌西路南北路卡倫六,各置坐卡侍衛一人,東西南三路凡二十一臺,各置筆帖式一人。沙雅爾南路卡倫一。庫車東路至哈喇沙爾西凡十臺。臺置筆帖式一人。每臺、卡俱置防守兵,多至十人,少或一人,俱有供役回人十戶。尋各官兵歸並烏什、阿克蘇,止駐一章京及游擊以下,旋改駐協辦大臣及領隊侍衛等。喀什噶爾之總理大臣移駐烏什之永寧,尋改設辦事大臣二人。三十一年,撤回索倫兵,改遣健銳營兵九百人換防,並令健銳營翼領一人,正副委署參領十八人,護軍校二十四人,統兵分駐各回城。四十四年,裁闢展辦事大臣,改設領隊大臣。旋設吐魯番屯田都司以下官。

道光八年,以阿克蘇為南路適中之地,增兵一千,移柯爾坪防兵五百歸阿克蘇,裁拜城參將以下弁兵,共新舊防兵二千二百人,守卡借差兵外,得練兵一千三百人,控制各路。九年,於喀什噶爾邊增八卡倫弁兵。尋以八卡倫內喀浪圭、圖舒克塔、烏拍拉特三處通霍罕要路,於明約洛建堡,設都司一人,綠營兵二百人駐守。阿爾瑚馬廠三處建堡,置兵二百或六十人。葉爾羌屬卡倫七,以亮葛爾、庫庫雅爾為通夷要隘,英吉沙爾屬卡倫五,惟烏魯克為要路,皆建土堡兵房,設千總官,其次設把總、外委,駐守兵多者六十人,少者十五或十人。

咸豐三年,以新疆南北兩路駐兵四萬餘人,歲餉一百四十五萬,軍興後饋食軍艱難,諭陜、甘赴口外駐防官自是年始,即行停止。其喀什噶爾、英吉沙爾、葉爾羌、和闐八城防兵,由烏魯木齊駐防滿洲兵、綠旗兵酌撥。四年,改定新疆南路換防兵制。增伊犁滿洲兵二百人,烏魯木齊綠營兵千二百人,滿洲兵三百人。裁葉爾羌、喀什噶爾、烏什、阿克蘇四城防兵一千人。七年,以喀城肅清,撤回土爾扈特蒙兵,留伊犁官兵防守。八年,令南路換防官兵自是年始,分六年抽換,以節繁費。天山以南,為回部所居,自設臺站、卡倫,無俟重兵防守。烏什、葉爾羌、喀什噶爾、英阿薩爾咸以滿、漢兵協力守邊。他如和闐、阿克蘇、庫車、哈喇沙爾、闢展則守以綠旗兵。凡滿洲營駐防兵,以三年更換,綠旗營駐防兵,以五年更換。此南路之制也。

同治以來,回疆不靖,欽差大臣左宗棠次第殄平之,新疆漸歸版籍。光緒初年,改省議起。左宗棠擬令將軍率旗營駐伊犁,塔爾巴哈臺改設都統,並統綠、旗各營。迨八年收復伊犁,從譚鍾麟、劉錦棠言,於南北兩路增設額兵,其舊有參贊、辦事、領隊各大臣悉予裁汰。即自哈密至伊犁都統暨諸大臣名額亦酌撤之。巴里坤、古城、烏魯木齊、庫爾喀拉烏蘇等處所餘旗丁,歸並伊犁滿營,均改從各省駐防將軍營制。十一年,行省制成。伊犁旗營實存勇七千,留其精壯,改馬隊九旗,步隊十三旗,以提督、總兵分領之。伊犁開屯由此始,而旗屯居其一焉。蓋新疆自籓部迄於設行省,綜其駐防旗兵制度,約略如此。

其籓部兵制,曰內外蒙古,曰青海,曰西藏。內外蒙古之兵,設旗編次,略同內八旗。每旗設札薩克一人,汗、王、貝勒、貝子、公、臺吉為之。協理旗務二或四人,亦臺吉以上充任。按丁數編為佐領。設佐領一,驍騎校六。每六佐領設參領一人。佐領較多者,設章京、副章京。各率所屬以聽於札薩克。內札薩克蒙古凡二十四部、四十九旗。科爾沁六旗,分左右二翼,二翼又各分前後旗。崇德元年,設左翼旗、左翼前旗、右翼旗、右翼前後旗。順治六年,設左翼後旗。郭爾羅斯前後二旗,杜爾伯特一旗,扎賚特一旗,皆順治五年設。扎魯特二旗,左翼崇德元年設,右翼順治五年設。喀爾喀左翼一旗,康熙三年設。奈曼一旗,敖漢一旗,皆崇德元年設。土默特二旗,左翼崇德元年設,右翼順治二年設。喀喇沁三旗,右翼崇德元年設,左翼順治五年設,康熙中增設一旗。翁牛特左右二旗,阿魯科爾沁一旗,皆崇德元年設。巴林左右二旗,順治五年設。克什克騰一旗,順治三年設。烏珠穆沁二旗,右翼崇德六年設,左翼順治三年設。浩齊特二旗,順治三年設左翼,十年設右翼。阿巴哈納爾二旗,康熙四年設左翼,六年設右翼。阿巴噶二旗,崇德六年設右翼,順治八年設左翼。蘇尼特二旗,崇德六年設左翼,七年設右翼。四子部落一旗,順治八年設。烏喇特右翼一旗,順治十年設。茂明安一旗,順治元年設。烏喇特前中後三旗,順治五年設。鄂爾多斯七旗,兩翼、中旗、前旗、後旗皆順治六年設,雍正九年,增設一旗。歸化城土默特左右二旗,崇德元年設,後置副都統,隸綏遠城將軍轄之。是為內蒙古兵制。

外札薩克蒙古,喀爾喀四部,凡八十六旗。喀爾喀土謝圖汗部二十旗為中路。康熙三十年,設十七旗。逮雍正間,遞增至三十八旗。尋分二十旗屬三音諾顏部,存十八旗。乾隆初,復增二旗,於本旗外分十九札薩克掌之,仍統於土謝圖汗部。車臣汗部二十三旗為東路。康熙三十年,設十二旗。其後增至二十一旗。乾隆間,遞增二旗,於本旗外分二十二札薩克掌之,仍統於車臣汗。札薩克圖汗部十七旗為西路。康熙三十年,設八旗。逮雍正間,遞增至十五旗。乾隆時,遞增二旗,於本旗外分十六札薩克掌之,仍統於札薩克圖汗。三音諾顏親王部二十二旗,雍正十年設,即於土謝圖汗部內分轄二十旗。乾隆初,增二旗,於本旗外分二十一札薩克掌之,仍統於三音諾顏札薩克親王。烏蘭烏蘇厄魯特部二旗,康熙二十五年分設。乾隆間,隸移烏蘭烏蘇並隸三音諾顏部。賀蘭山厄魯特一旗,康熙三十六年設。青海厄魯特部二十一旗,雍正三年設二十旗,乾隆十一年增設一旗。青海游牧綽羅斯部二旗,輝特部一旗,土爾扈特部一旗,喀爾喀部一旗,皆雍正三年設。哈密一旗,康熙三十六年設。吐魯番一旗,雍正十年設。都爾伯特十四旗,乾隆十八年編設。土爾扈特部,乾隆三十六年編設。康熙十三年,定每年春季,王、貝勒以各旗下臺吉兵丁合操。乾隆元年,諭內札薩克六會,防秋兵丁各備牧馬器械,分二班,錫林郭勒、烏蘭察布、伊克昭三會為一班,哲裏木、昭烏達、卓索圖三會為一班,以大札薩克為盟長,每年遣大臣會同盟長,按旗察閱兵丁。其喀爾喀四部游牧防守兵萬人,遣參贊大臣同喀爾喀將軍、貝勒、公等分年簡稽軍實。三年,命賞六會防秋牧馬之兵,視康熙間成例,分給弓矢、衣服、銀兩有差。五十一年,諭蒙古兵丁應習圍場者,車臣汗、土謝圖汗二部,由庫倫辦事王、大臣,三音諾顏、札薩克圖汗二部,由烏里雅蘇臺將軍、大臣等分領練習,並令各部落汗、王、公選大臺吉各四人,小臺吉十人,赴木蘭圍場。道光三年,從陜甘總督那彥成言,以青海二十四旗分左右二翼,每翼設盟長、副盟長,每六旗設霍碩扎爾噶齊,每三旗設一梅勒,每旗設一甲喇,各旗兵按人數之多寡,隨官兵番值巡防。十一年,允楊遇春請,以蒙古兵五百人析為二班,分防八卡。十五年,諭令察哈爾兵丁選補缺額,與札薩克游牧共衛北邊。同治十年,諭邊外各路臺站,都統或盟長分任管轄。每臺額定駱駝百頭,馬五十匹,戈壁地備駱駝百五十頭。此內外蒙古及青海兵制也。

蒙古各盟,當雍、乾時,征討準、回,資其兵力以集事。自俄人闌入,烏蘭海南北並受羈牽,喀魯倫東西侵為田牧,雜居無限,卡倫鄂博,蓋同虛設矣。

西藏旗兵,自乾隆五十七年始。前後藏各設番兵千。定日、江孜各設五百。前藏領兵者曰戴琫,其下如琫,又下甲琫、定琫。原置戴琫三人,二駐後藏,一駐定日,復增戴琫一人駐江孜。前藏番兵,游擊統之。後藏及江孜、定日,都司統之。原有唐古特兵,歸戴琫督練。初制,每番兵千,弓箭三之,鳥槍七之。嗣選唐古特兵三千,鳥槍、刀矛各半。至是新設額兵三千,每千人五成鳥槍,三成弓矢,二成刀矛。其唐古特兵,由駐防將領督同番目教練。前藏駐游擊、守備各一,千總二,把總三,外委五。後藏駐游擊、都司各一,守備三,千總二,把總七,外委九。是年,以福康安疏請江孜增守備一,外委一,兵三十人,定日增守備一,把總一,外委一,兵四十人。尋用和琳疏言,定日要隘曰轄爾多,曰察木達杏嶺,曰古喇噶木洞,曰宗喀,每處各設定琫一人,番兵二十五人。此西藏兵制大略也。

當乾隆十五年,始除西藏王爵,設駐藏大臣,以達賴喇嘛統前藏,班禪統後藏。前後藏凡設四汛,游擊、都司、守備、千把總、外委十六人,兵丁六百六十人,戴琫、如琫、甲琫、定琫百六十六人,番兵三千人,騎兵五百人,駐藏大臣與達賴、班禪參制之。咸、同以後,廓爾喀崛強於西,英吉利侵軼於南,中朝威力羈縻而已。

八旗官兵額數,代有增減,舉其最近者以見例。光、宣之季,實存名數,職官約六千六百有奇,兵丁十二萬三百有奇。八旗各營印務參領雖設專職,大率參領、副參領兼之。印務章京、印務筆帖式亦兼職。親軍校、親軍、拜唐阿等在各旗支餉,實於他所供差。其醇王園寢守護兵,光緒間始增設前鋒、護軍統領諸職,雖已汰去,而設官已久,職亦較崇,仍序列之。其他不具錄云。

鑲黃旗滿洲,都統一,副都統二,印務參領二,參領、副參領各五,印務章京八,佐領八十六,驍騎校八十六,印務筆帖式八,凡二百有三人。領催四百二十八,馬甲千五百六十二,隨甲八十六,養育兵二千二百二十七,親軍校十一,親軍百五十八,弓匠長七,弓匠七十八,倉甲二十五,通州十九,清河六。餘如通州領催,備宴馬甲,盔、鏇、鞍、頭、箭、鐵諸匠,拜唐阿分網戶、粘桿、備箭,一人至九人,陸軍部承差三人,凡四千六百三十人。

正黃旗滿洲,自都統至印務章京及筆帖式並同鑲黃旗,惟佐領九十三、驍騎校九十二為小異,凡二百十六人。領催四百六十二,馬甲千六百二十八,隨甲九十三,養育兵二千三百九十三,親軍校十一,覺羅親軍四,親軍百七十一,南苑驍騎校一,弓匠長八,弓匠八十四,餘如南苑馬甲,備宴馬甲,倉甲,盔、金旋、鞍匠,庫使、守吏、酒吏、鷹手、鞭子手、亭兵,網戶、粘桿拜唐阿等一至六人,陸軍部承差一人,凡四千九百十二人。

正白旗滿洲,都統以下並同上,佐領、驍騎校亦同鑲黃旗,凡二百有三人。領催四百三十,馬甲千四百十四,隨甲八十六,養育兵二千二百四,親軍校十一,覺羅親軍五,親軍百五十六,弓匠長十,弓匠七十六,倉甲三十,通州二十,清河十。餘如南苑馬甲,備宴馬甲,頭、鞍、箭、盔諸匠,鞭子手,網戶、備箭拜唐阿,傳事兵等一至十二人,陸軍部承差三人,凡四千四百八十八人。

正紅旗滿洲,都統以下並同上,惟佐領、驍騎校各七十四,凡一百七十九人。領催三百七十,馬甲千二百八十七,隨甲七十四,養育兵一千八百八十八,親軍校十六,親軍百三十二,弓匠長二,弓匠七十二,倉甲二十七,通州十九,清河八。餘如南苑馬甲,守吏,庫使,傳事兵,粘桿、宰牲拜唐阿等一至九人,凡三千八百九十五人。

鑲白旗滿洲,都統以下並同上,惟佐領、驍騎校各八十四,凡一百九十九人。領催四百二十,馬甲千四百十四,隨甲八十四,養育兵二千一百八十,親軍校十三,親軍百五十四,覺羅親軍二,弓匠長二,弓匠七十二,帳房頭目二,倉甲二十七,通州二十,本裕倉七。餘如鏇、盔諸匠,鞭子手,傳事,渡吏,亭兵,備箭、宰牲拜唐阿等一至四人,陸軍部承差三人,凡四千三百九十七人。

鑲紅旗滿洲,都統以下並同上,佐領、驍騎校亦同鑲黃旗,凡二百有三人。領催四百三十,馬甲千五百四十八,隨甲八十六,養育兵二千二百四,親軍校十九,覺羅親軍三,親軍百五十,弓匠長六,弓匠八十,倉甲二十七,通州二十,本裕倉七。餘如盔匠、金旋匠、鞭子手、南苑馬甲、承差、傳事兵、亭兵、宰牲拜唐阿等一至四人,凡四千五百七十七人。

正藍旗滿洲,都統以下並同上,惟佐領、驍騎校各八十三,凡一百九十七人。領催四百十七,馬甲千四百九十一,隨甲八十三,養育兵二千一百三十九,親軍校十七,覺羅親軍十一,親軍百四十,弓匠長二,弓匠八十三,倉甲十九,通州十七,清河二。餘如金旋匠、盔匠、鞭子手、承差兵、傳事兵、亭兵、南苑馬甲、守吏、拜唐阿、宰牲拜唐阿等一至五人,凡四千四百三十三人。

鑲藍旗滿洲,都統以下並同上,佐領、驍騎校俱同鑲白旗,凡一百九十九人。領催四百三十九,馬甲千五百九十,隨甲八十六,公缺馬甲二十四,恩缺馬甲一,養育兵二千二百四十九,親軍校十五,覺羅親軍六,親軍百五十五,弓匠長六,弓匠八十八,餘如南苑馬甲、南苑領催、帳房頭目、金旋匠、鞭子手、酒醋局吏、庫使、傳事兵、亭兵、宰牲兵等一至八人,陸軍部承差一人,凡四千六百九十人。

鑲黃旗蒙古,都統一,副都統二,印務參領一,參領二,副參領二,印務章京四,佐領、驍騎校各二十八,印務筆帖式四,凡七十二人。領催一百四十,馬甲四百九十七,隨甲二十八,養育兵五百九十二,親軍校四,親軍五十二,弓匠長一,弓匠二十七,餘如長號達、長號、盔匠、鞍匠、網戶、苑甲、承差、傳事兵、亭兵等一至六人,凡千三百六十三人。

正黃旗蒙古,自都統以下至印務章京及筆帖式,並同鑲黃旗,惟佐領、驍騎校各二十四,凡六十四人。領催百二十,馬甲四百五十二,養育兵五百八,親軍校四,親軍四十四,弓匠二十四,餘如長號、拜唐阿、茶拜唐阿、鞍匠,一至七人,凡千一百七十一人。

正白旗蒙古,都統以下並同上,惟佐領、驍騎校各二十九,凡七十四人。領催百四十五,馬甲四百八十七,隨甲二十九,養育兵六百九,親軍校四,親軍五十四,弓匠長二,弓匠二十七,餘如長號、拜唐阿達、拜唐阿、網戶拜唐阿、南苑馬甲、盔匠、鞍匠、亭兵等一至七人,凡千三百七十八人。

正紅旗蒙古,都統以下並同上,惟佐領、驍騎校各二十二,凡六十人。領催一百十,馬甲三百八十一,隨甲二十二,養育兵四百六十,親軍校六,親軍三十八,弓匠長三,弓匠十八,餘如南苑馬甲、哈那器馬甲、盔匠、粘桿拜唐阿、亭兵等一至五人,凡一千五十人。

鑲白旗蒙古,都統以下並同上,佐領、驍騎校俱同正黃旗,凡六十四人。領催一百二十,馬甲四百四十,養育兵五百八,親軍校二,親軍四十八,凡千一百十八人。

鑲紅旗蒙古,都統以下並同上,佐領、驍騎校如正紅旗,凡六十人。領催一百十,馬甲三百八十八,隨甲二十二,養育兵四百五十九,親軍校三,親軍四十一,弓匠長一,弓匠十八,承差、盔匠各一,凡一千四十五人。

正藍旗蒙古,都統以下並同上,惟佐領、驍騎校各三十,凡七十六人。領催一百五十,馬甲五百四十四,隨甲三十,養育兵六百三十,親軍校九,親軍五十一,弓匠長二,弓匠二十八,承差、盔匠、馬甲、亭兵、蒙古通事兵各一,凡一千四百四十八人。

鑲藍旗蒙古,都統以下並同上,惟佐領、驍騎校各二十五,凡六十六人。領催百二十五,馬甲四百四十二,隨甲二十五,養育兵五百二十七,親軍校五,親軍四十四,包衣護軍校二,弓匠長一,弓匠二十二,鞍匠、盔匠、恩缺馬甲、聽差馬甲、亭兵各一,凡千一百九十八人。

鑲黃旗漢軍,都統一,副都統二,印務參領二,參領、副參領各五,印務章京六,佐領、驍騎校各四十一,印務筆帖式六,凡一百有九人。領催二百五,馬甲千六百八十一,隨甲四十一,敖爾布三百二十八,養育兵九百三十七,藍甲三十九,弓匠長六,弓匠三十一,砲手四十,餘如更夫、承差兵、拜唐阿、銅匠、盔匠、鞍匠、亭兵等一至五人,凡三千三百三十二人。

正黃旗漢軍,自都統以下至印務章京及筆帖式,並同鑲黃旗,惟佐領、驍騎校各四十,凡一百有七人。領催二百,馬甲、隨甲千六百八十,敖爾布三百二十,養育兵九百十四,藍甲三十一,弓匠長三,弓匠三十六,砲手四十,餘如更夫、承差兵、拜唐阿、備箭拜唐阿、銅匠、盔匠、鞍匠、聽差兵、亭兵一至十二人,隨印外郎一人,凡三千二百六十人。

正白旗漢軍,都統以下並同上,佐領、驍騎校亦同鑲黃旗,凡一百有七人。領催二百,馬甲千六百四十,隨甲四十,敖爾布三百二十,養育兵九百十四,藍甲五十二,弓匠長二,弓匠三十八,砲手四十,餘如更夫、承差兵、拜唐阿、銅匠、盔匠、鞍匠等一至六人,隨印外郎三人,凡三千二百六十八人。

正紅旗漢軍,都統以下並同上,惟佐領、驍騎校各二十八,凡八十三人。領催百三十八,馬甲千一百五十三,隨甲一,敖爾布二百二十,藍甲五,養育兵六百四十一,弓匠長八,弓匠十四,砲手三十九,餘如更夫、拜唐阿、盔匠、鞍匠、亭兵、承差兵等一至五人,凡二千二百三十二人。

鑲白旗漢軍,都統以下並同上,惟佐領、驍騎校各三十,凡八十七人。領催百五十,馬甲千二百三十,隨甲三十,敖爾布二百四十,養育兵六百九十九,弓匠長四,弓匠十五,砲手四十,餘如更夫、備箭拜唐阿、承差兵、盔匠等一至五人,隨印外郎一人,凡二千四百二十四人。

鑲紅旗漢軍,都統以下並同上,惟佐領、驍騎校各二十九,凡八十五人。領催百四十五,馬甲千一百八十七,隨甲二十九,敖爾布二百三十三,養育兵六百七十四,弓匠長二,弓匠二十,砲手四十,餘如拜唐阿、更夫、承差兵、盔匠、亭兵,一至四人,隨印外郎二人,凡二千三百四十二人。

正藍旗漢軍,都統以下並同上,佐領、驍騎校俱同鑲紅旗,凡八十五人。領催百四十五,馬甲千一百九十四,隨甲二十二,敖爾布二百三十二,養育兵六百七十六,弓匠長四,弓匠二十二,砲甲、砲手各二十,餘如盔匠、馬甲盔匠、公主門甲、更夫、拜唐阿、承差兵、亭兵等一至七人,凡二千三百六十二人。

鑲藍旗漢軍,都統以下並同上,佐領、驍騎校亦同鑲紅旗,凡八十五人。領催百四十五,馬甲千二百十八,敖爾布二百三十二,養育兵六百七十五,藍甲十八,弓匠長五,弓匠二十四,砲手四十,餘如更夫、拜唐阿、盔匠、匠役、亭兵等一至五人,隨印外郎二人,凡二千三百七十六人。

圓明園隨同辦事營總二,營總六,護軍參領八,副護軍參領十六,委護軍參領三十二,護軍校、副護軍校各百二十八,包衣營總一,包衣護軍參領、副護軍參領各三,包衣護軍校九,凡三百三十六人。護軍三千六百七十二,馬甲三百,槍甲四百,養育兵千八百二十六,包衣護軍一百二十,包衣馬甲三十,包衣養育兵六十,凡六千四百八人。

健銳營翼長四,正參領八,副參領十六,委參領三十二,番子防御一,前鋒校、副前鋒校各七十,凡百有二人。前鋒千九百六十,委前鋒一千,領催四,馬甲八十一,養育兵八百三十三,凡三千八百七十八人。

內火器營管營長官二,正翼長、委翼長各一,營總四,正參領四,副參領八,委參領十六,護軍校一百十二,凡一百四十八人。鳥槍護軍二千五百十二,砲甲五百二十八,養育兵八百八十,凡三千九百二十人。

外火器營全營翼長一,委翼長一,營總三,正參領四,副參領八,委參領十六,護軍校一百十二,凡一百四十五人。鳥槍護軍二千五百三十,槍甲三百五十二,養育兵八百十八,凡三千七百人。

左右翼前鋒營,左右翼前鋒統領二,前鋒參領、前鋒侍衛各十六,委前鋒侍★八,空銜花翎十六,前鋒校九十六,空銜前鋒校八,藍翎長四十八,委藍翎長十六,印務筆帖式四,凡二百三十人。前鋒兵千六百六十八人。

八旗護軍營,護軍統領八,護軍參領、副護軍參領各一百十二,委護軍參領五十六,空銜花翎一百十二,護軍校八百八十二,空銜護軍校五十六,藍翎長一百十二,門筆帖式三十六,印務筆帖式十六,凡一千五百有二人。護軍萬四千八十一人。

八旗包衣屬鑲黃旗者,參領、副參領各五,佐領十一,管領十,章京一,護軍參領、副護軍參領各五,護軍校三十五,驍騎校十一,凡八十八人。領催七十九,護軍四百,披甲千六百八十九,隨甲十一,養育兵八十八,拜唐阿四百二十一,凡二千六百八十八人。屬正黃旗者,參領、副參領各五,佐領十三,管領十,護軍參領、副護軍參領、委護軍參領各五,護軍校三十三,前鋒校二,驍騎校十三,凡九十六人。領催九十五,護軍四百七十八,披甲千八百九,隨甲十三,養育兵八十九,拜唐阿等三百四十七,凡二千八百三十一人。屬正白旗者,參領、副參領各五,佐領十二,管領十,護軍參領、副護軍參領、委護軍參領各五,護軍校三十三,前鋒校二,驍騎校十二,凡九十四人。領催八十八,護軍三百六十,前鋒四十,披甲等千七百三十八,隨甲十二,養育兵八十五,拜唐阿等六百三十五,凡二千九百五十八人。屬正紅旗者,參領五,佐領十一,管領十九,包衣達等十六,護軍校六十,驍騎校十二,凡一百二十三人。領催三十四,護軍八十五,馬甲八百四十六,藍甲三百三十二,蒙古護軍七十,凡千三百六十七人。屬鑲白旗者,參領五,佐領十四,管領十一,包衣達等三十二,親軍校一,護軍校八十,驍騎校十三,凡一百五十六人。領催七十四,護軍百四十二,藍甲五百六十六,白甲千一百三十一,拜唐阿三,凡一千九百十六人。屬鑲紅旗者,參領五,佐領十七,管領六,包衣達等六十三,護軍校五十八,驍騎校十二,凡一百六十一人。領催四十七,護軍一百八,紅甲千一百十八,藍甲五百四十五,凡千八百十八人。屬正藍旗者,參領五,佐領六,管領七,包衣達等五十九,護軍校一百三,驍騎校十六,凡一百九十六人。領催七十八,護軍二百二十六,馬甲千六百二十四,藍甲七百六十一,拜唐阿十五,凡二千七百四人。屬鑲藍旗者,參領五,佐領二十一,管領三十八,司庫等九十二,護軍校一百三十七,驍騎校十六,凡三百有九人。領催七十八,護軍百八十九,馬甲千三百八十六,藍甲千二百八十二,凡二千九百三十五人。

醇賢親王園寢翼領一,防御一,驍騎校一,凡三人。領催二,披甲四十六,凡四十八人。

以上凡職官六千六百八十人。兵丁十二萬三百有九人。


\end{pinyinscope}