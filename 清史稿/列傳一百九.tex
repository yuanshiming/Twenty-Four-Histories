\article{列傳一百九}

\begin{pinyinscope}
竇光鼐李漱芳範宜賓曹錫寶謝振定錢灃尹壯圖

竇光鼐,字元調,山東諸城人。乾隆七年進士,選庶吉士,散館授編修。大考四等,罰俸。高宗夙知光鼐,居數月,擢左中允。累遷內閣學士。二十年,授左副都御史。督浙江學政。上南巡,臨海縣訓導章知鄴將獻詩,光鼐以詩拙阻之。知鄴欲訐光鼐,光鼐以聞。上召知鄴試以詩,詩甚拙,且言原從軍。上斥其妄,命奪職戍闢展。後數年,上欲赦知鄴還,而知鄴妄為悖逆語,欲以陷光鼐,上乃誅之。

光鼐學政任滿,還京師。秋讞,光鼐以廣西囚陳父悔守田禾殺賊,不宜入情實;貴州囚羅阿扛逞兇殺人,不宜入緩決:持異議,簽商刑部,語忿激。刑部遽以聞,上命大學士來保、史貽直,協辦大學士梁詩正覆覈,請如刑部議,且言光鼐先已畫題,何得又請改擬。上詰光鼐,光鼐言:「兩案異議,本屬簽商,並非固執。因會議時言詞過激,刑部遽將簽出未定之稿先行密奏。臣未能降心抑氣,與刑部婉言,咎實難辭,請交部嚴加議處。」上以「會讞大典,光鼐意氣自用,甚至紛呶謾罵而不自知。設將來預議者尤而效之,於國憲朝章不可為訓」。命下部嚴議,當左遷,仍命留任。光鼐疏言:「事主殺竊盜,律止杖徒。近來各省多以竊盜拒捕而被殺,比罪人不拒捕而擅殺,皆以鬥論,寬竊盜而嚴事主,非禁暴之意。應請遵本律。」議行。

二十七年,上以光鼐迂拙,不勝副都御史,命署內閣學士。授順天府府尹。坐屬縣蝗不以時捕,左遷四品京堂,仍留任。旋赴三河、懷柔督捕蝗,疏言:「近京州縣多旗地,嗣後捕蝗,民為旗地佃,當一體撥夫應用。」上從所請,以諭直隸總督楊廷璋。廷璋言自方觀承始設護田夫,旗、民均役。上復以詰光鼐,召還京師,令從軍機大臣入見。問:「民為旗地佃,不肯撥夫應用,屬何人莊業?」光鼐不能對,請徵東北二路同知及三河、順義知縣質證。退又疏請罷護田夫,別定派夫捕蝗事例。上以光鼐所見迂鄙紕繆,下部議,奪職。

居數月,諭光鼐但拘鈍無能,無大過,左授通政司副使。再遷宗人府府丞。復督浙江學政,擢吏部侍郎。浙江州縣倉庫多虧缺,上命察覈。光鼐疏言:「前總督陳輝祖、巡撫王亶望貪墨敗露,總督富勒渾未嚴察。臣聞嘉興、海鹽、平陽諸縣虧數皆逾十萬,當察覈分別定擬。」上嘉其持正,命尚書曹文埴、侍郎姜晟往會巡撫伊齡阿及光鼐察覈。

旋疏劾永嘉知縣席世維借諸生穀輸倉;平陽知縣黃梅假彌虧苛斂,且於母死日演劇;仙居知縣徐延翰斃臨海諸生馬寘於獄;並及布政使盛住上年詣京師,攜貲過豐,召物議;總督富勒渾經嘉興,供應浩煩,餽閽役數至千百。上命大學士阿桂如浙江按治。阿桂疏言盛住詣京師,附攜應解參價銀三萬九千餘,非私貲;平陽知縣黃梅母九十生日演劇,即以其夕死;仙居諸生馬寘誣寺僧博,復與鬥毆,因下獄死。光鼐語皆不仇。光鼐再疏論梅事,言阿桂遣屬吏詣平陽諮訪,未得實,躬赴平陽覆察。伊齡阿再疏劾光鼐赴平陽刑迫求佐證諸狀,上責光鼐乖張瞀亂,命奪職,逮下刑部。光鼐尋奏:「親赴平陽,士民呈梅派捐單票,田一畝捐大錢五十;又勒捐富戶數至千百貫;每歲採買倉穀不予值。梅在縣八年,所侵穀值及捐錢不下二十萬。母死不欲發喪,特令演劇。」上以光鼐呈單票有據,時阿桂已還京師,令復如浙江秉公按治,並命江蘇巡撫閔鶚元會讞,以光鼐質證。阿桂、鶚元疏言梅婪索事實,論如律。上以光鼐所奏非妄,命署光祿寺卿,阿桂、文埴、晟、伊齡阿皆下部議。旋擢光鼐宗人府府丞。遷禮部侍郎。復督浙江學政。再遷左都御史。

六十年,充會試正考官,榜發,首歸安王以鋙,次王以銜,兄弟聯名高第。大學士和珅素嫉光鼐,言於上,謂光鼐迭為浙江學政,事有私。上命解任聽部議,及廷試,和珅為讀卷官,以銜復以第一人及第,事乃解。命予四品銜休致。卒。

李漱芳,字藝圃,四川渠縣人。乾隆二十二年進士,授吏部主事。再遷郎中。三十三年,授河南道監察御史。巡視中城,尚書福隆安家奴藍大恃勢縱恣,挾無賴酗酒,橫行市肆間。漱芳捕治,論奏,高宗深嘉之,命戍藍大,以福隆安下吏議。尋擢工科給事中。三十九年,壽張民王倫為亂。漱芳疏陳奸民聚眾滋事,為饑寒所迫;又言近畿亦有流民扶老攜幼,遷徙逃亡,有司監盧溝橋,阻不使北行。給事中範宜賓亦以為言,請增設粥廠。上命侍郎高樸、袁守侗率宜賓、漱芳往盧溝橋及近畿諸城鎮省視,初無流民。倫亂定,俘其徒檻致京師廷鞫,命漱芳旁視,無言為饑寒迫者。問歲事,對秋收尚及半。上責漱芳妄言,代奸民解說,心術不可問,不宜復居言路,為世道人心害,宥罪,降禮部主事。四十三年,禮部請以漱芳升授員外郎。故事,郎中、員外郎員缺,選應升授者,擬正、陪上請。至是,獨以漱芳請。上不懌,責尚書永貴擅專邀譽,涉明季黨援朋比之習,奪其職。漱芳久之乃遷員外郎。卒。

範宜賓,漢軍鑲黃旗人,大學士文程後也。以廕生官戶部郎中,歷御史給事中,累遷太常寺少卿。出為安徽布政使,與巡撫胡文伯不相能,兩江總督高晉以聞。上召宜賓還,授左副都御史。宜賓奏言屬縣蝗見,屢請捕治,文伯執不可。上為黜文伯,而宜賓亦以捕蝗不力下吏議,當左遷。上以宜賓舊為御史尚黽勉,命仍為御史。宜賓疏言籓臬有所陳奏,輒呈稿督撫,當禁飭。上以整飭吏治,要在朝廷綱紀肅清,自無扶同蒙蔽之事,不在設法峻防,置其議不行。及與漱芳同被譴,上以宜賓漢軍世僕,乃敢妄言干譽,特重其罰,奪職,戍新疆。

曹錫寶,字鴻書,一字劍亭,江南上海人。乾隆初,以舉人考授內閣中書,充軍機處章京。資深當擢侍讀,錫寶辭。大學士傅恆知其欲以甲科進,乃不為請遷。二十二年,成進士,改庶吉士。以母憂歸,病瘍,數年乃愈。三十一年,散館,改刑部主事。再遷郎中。授山東糧道。衛千總寧廷言子惠以索逋殺千總張繼渠,錫寶下部議。上巡山東,召見,命來京以部屬用。以大學士阿桂奏,令入四庫全書館自效。書成,以國子監司業升用。

居三年,上以錫寶補司業無期,特授陜西道監察御史。時協辦大學士和珅執政,其奴劉全恃勢營私,衣服、車馬、居室皆逾制。錫寶將論劾,侍郎南匯吳省欽與錫寶同鄉里,聞其事,和珅方從上熱河行在,馳以告和珅,令全毀其室,衣服、車馬有逾制,皆匿無跡。錫寶疏至,上詰和珅。和珅言平時戒約嚴,或扈從日久漸生事,乞嚴察重懲。乃命留京辦事王大臣召錫寶問狀,又令步軍統領遣官從錫寶至全家察視,無跡,錫寶自承冒昧。上召錫寶詣行在面詰,錫寶奏全倚勢營私,未有實跡,第為和珅「杜漸防微」,乃有此奏。復諭軍機大臣、大學士梁國治等覆詢,錫寶又承「杜漸防微」語失當,請治罪。下部議,當左遷。上手詔略言:「平時用人行政,不肯存逆詐億不信之見。若委用臣工不能推誠布公,而猜疑防範,據一時無根之談,遽入人以罪,使天下重足而立、側目而視,斷無此政體。錫寶未察虛實,以書生拘迂之見,託為正言陳奏。姑寬其罰,改革職留任。」五十七年,卒。

仁宗親政,誅和珅,並籍全家,乃追思錫寶直言,諭曰:「故御史曹錫寶,嘗劾和珅奴劉全倚勢營私,家貲豐厚。彼時和珅聲勢薰灼,舉朝無一人敢於糾劾,而錫寶獨能抗辭執奏,不愧諍臣。今和珅治罪後,並籍全家,貲產至二十餘萬。是錫寶所劾不虛,宜加優獎,以旌直言。錫寶贈副都御史,其子江視贈官予廕。」錫寶,一士從子,再世居臺省,敢言名。家有甕,焚諫草,江嘗乞諸能文者為詩歌,傳一時云。

謝振定,字一齋,一字薌泉,湖南湘鄉人。乾隆四十五年進士,改庶吉士,散館授編修。五十九年,考選江南道監察御史。巡視南漕,漕艘阻瓜洲,振定禱於神,風轉順漕艘,人稱「謝公風」。六十年,遷兵科給事中。巡視東城,有乘違制車騁於衢者,執而訊之,則和珅妾弟也,語不遜,振定命痛笞之,遂焚其車。曰:「此車豈堪宰相坐耶?」居數日,給事中王鍾健希和珅意,假他事劾振定,奪職。和珅敗,嘉慶五年,起授禮部主事。遷員外郎,充坐糧,監收漕糧,裁革陋規,兌運肅然。十四年,卒。

道光中,振定子興嶢,官河南裕州知州。以卓薦引見,循例奏姓名、裏貫。宣宗問:「爾湖南人,乃能為京師語,何也?」興嶢對言:「臣父振定官御史,臣生長京師。」上曰:「爾乃燒車御史子耶?」因褒勉甚至。明日,語軍機大臣:「朕少聞燒車御史事,昨乃見其子。」命擢興嶢敘州知府。

錢灃,字東注,雲南昆明人。乾隆三十六年進士,改庶吉士,散館授檢討。四十六年,考選江南道監察御史。甘肅冒賑折捐事發,主其事者為甘肅布政使王亶望,時已遷浙江巡撫,坐誅,總督勒爾謹及諸府縣吏死者數十人,事具亶望傳。陜西巡撫畢沅嘗兩署陜甘總督,獨置不問。灃疏言:「冒賑折捐,固由亶望骫法,但亶望為布政使時,沅兩署總督,近在同城,豈無聞見?使沅早發其奸,則播惡不至如此之甚;即陷於刑闢者,亦不至如此之多。臣不敢謂其利令智昏,甘受所餌,惟是瞻徇回護,不肯舉發,甚非大臣居心之道。請比捏結各員治罪。」上為詰責沅,降秩視三品,事具沅傳。

四十七年,灃疏劾山東巡撫國泰、布政使於易簡吏治廢弛,貪婪無饜,各州縣庫皆虧缺,上命大學士和珅、左都御史劉墉率灃往按。和珅庇國泰,怵灃,灃不為撓。至山東,發歷城縣庫驗帑銀。故事,帑銀以五十兩為一鋌,市銀則否。國泰聞使者將至,假市銀補庫。灃按問得其狀,召商還所假,庫為之空。復按章丘、東平、益都三州縣庫,皆虧缺如灃言。國泰、易簡罪至死,和珅不能護也。上旌灃直言,擢通政司參議。四十八年,遷太常寺少卿。再遷通政司副使。出督湖南學政,灃持正,得士為盛。五十一年,任滿,命留任。湖北荊州水壞城郭,孝感土豪殺饑民。上責灃在鄰省何不以聞,下部議。諸生或匿喪赴試,又有上違禁書籍者。灃按治未竟,聞親喪去官,以事屬巡撫浦霖。霖遂並劾灃,坐奪職。上命左授六部主事。

五十八年,灃服除,詣京師,授戶部主事。引見,即擢員外郎。復除湖廣道監察御史。時和珅愈專政,大學士阿桂、王傑,尚書董誥、福長安與同為軍機大臣,不相能,入直恆異處。灃疏言:「我朝設立軍機處,大臣與其職者,皆萃止其中,庸以集思廣益,仰贊高深。地一則勢無所分,居同則情可共見。即各司咨事畫,亦有定所。近日惟阿桂每日入止軍機處;和珅或止內右門內直廬,或止隆宗門外近造辦處直廬;王傑、董誥則止於南書房;福長安則止於造辦處。每日召對,聯行而入,退即各還所處。雖亦有時暫至軍機處,而事過輒起。各司咨事畫,趨步多歧。皇上乾行之健,離照之明,大小臣工戴德懷刑,浹於肌髓,決不至因此遂啟朋黨角立之漸。然世宗憲皇帝以來,及皇上御極之久,軍機大臣萃止無渙,未嘗纖芥有他。由前律後,不應聽其輕更。內右門內切近禁寢,向因有養心殿帶領引見事,須先一兩刻預備。恩加大臣,不令與各官露立,是以設廬許得暫止。不應於未辨色之前,一大臣入止,而隨從軍機司員亦更入更出。為日既久,不能不與內監相狎。萬一有無知如高雲從者,雖立正刑闢,而所絓已多,杜漸宜早。至南書房備幾暇顧問,俟軍機事畢,入直未遲;若隆宗門外直廬及造辦處,則各色應差皆得覘聽於外,大臣於中治事,亦屬過褻。請敕諸大臣仍照舊規同止軍機處,庶匪懈之忱,各申五夜;協恭之雅,共勵一堂。其圓明園治事,和珅、福長安止於如意門外南順墻東向直廬,王傑、董誥止於南書房直廬,並請敕更正。」上為申誡諸大臣,並命灃稽察軍機處。

和珅素惡灃,至是尤深嗛之。上夙許其持正,度未可遽傾,凡遇勞苦事多委之。灃貧,衣裘薄,宵興晡散,遂得疾。六十年,卒。或謂灃將劾和珅,和珅實酖之。

尹壯圖,字楚珍,雲南昆明人。乾隆三十一年進士,改庶吉士。散館,授禮部主事。再遷郎中。三十九年,考選江南道監察御史,轉京畿道。三遷至內閣學士,兼禮部侍郎。

高宗季年,督撫坐譴,或令繳罰項貸罪,壯圖以為非政體。五十五年,上疏言:「督撫自蹈愆尤,聖恩不即罷斥,罰銀若干萬充公,亦有督撫自請認罰若干萬者。在桀驁者藉口以快其饕餮之私,即清廉者亦不得不望屬員之佽助。日後遇有虧空營私重案,不容不曲為庇護。是罰銀雖嚴,不惟無以動其愧懼之心,且潛生其玩易之念,請永停此例。如才具平常者,或即罷斥,或用京職,毋許再膺外任。」上諭曰:「壯圖請停罰銀例,不為無見。朕以督撫一時不能得人,棄瑕錄用,酌示薄懲。但督撫等或有昧良負恩,以措辦官項為辭,需索屬員;而屬員亦藉此斂派逢迎,此亦不能保其必無。壯圖既為此奏,自必確有見聞,令指實覆奏。」壯圖覆奏:「各督撫聲名狼藉,吏治廢弛。臣經過地方,體察官吏賢否,商民半皆蹙額興嘆。各省風氣,大抵皆然。請旨簡派滿洲大臣同臣往各省密查虧空。」上復諭曰:「壯圖覆奏,並未指實。至稱經過諸省商民蹙額興嘆,竟似居今之世,民不堪命。此聞自何人,見於何處,仍令指實覆奏。」壯圖再覆奏,自承措詞過當,請治罪。上命戶部侍郎慶成偕壯圖赴山西察倉庫,始大同府庫,次山西布政使庫,皆無虧。壯圖請還京治罪。上命慶成偕壯圖再赴直隸、山東、江南諸省。慶成所至,輒游宴數日,乃發倉庫校覈,歷直隸布政使及正定、蘭山、山陽諸府縣,皆無虧。上寄諭壯圖,問途中見商民蹙額興嘆狀否。壯圖覆奏,言目見商民樂業,絕無蹙額興嘆情事。上又令慶成傳旨,令其指實二三人,毋更含糊支飾。壯圖自承虛誑,奏請治罪。尋復察蘇州布政使庫,亦無虧。還京,下刑部治罪,比挾詐欺公、妄生異議律,坐斬決。上謂壯圖逞臆妄言,亦不妨以謗為規,不必遽加重罪,命左授內閣侍讀。繼又以侍讀缺少,改禮部主事。

壯圖以母老乞歸。嘉慶四年,仁宗親政,召詣京師。壯圖仍以母老乞歸,上賜其母大緞兩端,加壯圖給事中銜,賜奏事摺匣,命得上章言事。壯圖未行,復上疏請清覈各省陋規,明定科條,上以為不可行。既歸,疏請拔真才,儲實用,大要謂:「保舉未定處分,當下吏部嚴立科條;科埸或通關節,當將房考落卷送主司搜閱。其尤要者,謂六部滿洲司員案,文義多未曉暢,當嚴督令習經書通文理;鄉會試加廣名額,司員先侭科甲挑補。」下軍機大臣議,奏謂惟房考落卷送主司搜閱,事近可行,補入科場條例。

雲南巡撫初彭齡乞養歸,壯圖疏請留,上不允。別疏復申前議,謂滿洲子弟十五六歲前專責習經書通文理,再習騎射繙譯。上謂:「壯圖以前嘗駁飭之事復行瀆陳,更張本朝成法。下雲南巡撫伊桑阿傳旨申飭。」八年,疏言:「天下萬幾,皆皇上獨理。內外諸臣不過浮沉旅進旅退之中,無能匡扶弼亮。請於內之卿貳、翰詹、科道,外之籓、臬、道、府,慎選二十人,輪直內廷。每日奏章諭旨,盡心檢校,有疏忽偏倚之處,許就近詳辨可否。」上責:「壯圖言皆迂闊紕謬,斷不可行。若如所奏,直於軍機大臣外復設內軍機,成何政體?」因及雲南布政使陳孝升、道員薩榮安方以冒銷軍需被罪,令巡撫那彥寶詰壯圖,何無一言奏及。壯圖言以不得孝升等確據,未敢入告,仍請議處,上命寬之。十三年,卒。

論曰:高宗中年後,遇有言事者,遣大臣按治,輒命其參與。光鼐既將坐譴,卒得自白,阿桂之賢也。灃劾國泰發庫藏掩覆,論者謂劉墉密與灃商榷,蓋亦有力焉。漱芳、錫寶、壯圖皆不能實其言,大臣怙寵亂政,民迫於饑寒,卒成禍亂。嗚呼,古昔聖王兢兢,重畏民碞,良有以也!


\end{pinyinscope}