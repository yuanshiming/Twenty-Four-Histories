\article{志七十四}

\begin{pinyinscope}
○樂六

△樂章四筵宴舞曲大宴笳吹樂番部合奏

元旦、冬至、萬壽三大節,慶隆舞樂九章

於鑠皇清,受命於天。光延鴻祚,億萬斯年。天開令節,瑞啟階蓂。共球萬國,聖壽千齡。粵自我先,肇基俄朵。長白之山,鵲銜硃果。綿綿瓜瓞,長發其祥。篤生列祖,積慶重光。式廓舊疆,東訖海表。聖聖相承,永世克紹。一章赫赫神功,龍飛崛起。戎甲十三,奮跡伊始。復仇靖難,首克圖倫。天戈一指,震懾強鄰。九姓潛侵,滅跡如埽。蒙古五部,來奉大號。明師四路,五日而殲。定鼎遼沈,都城巖巖。陣雲五色,江冰夜凝。天助有德,應運而興。二章天亶聰明,覆幬下國。遠服邇歸,誕敷文德。四方來附,雲集景從。建長命官,庶職是綜。爰創國書,頡文羲畫。聲協元音,萬古不易。爰定軍制,採旐央央。或純或間,永奠八方。締造鴻謨,創成大業。錫福無疆,慶鍾千葉。三章佑啟哲嗣,光闡前猷。昭崇駿德,誕迓天庥。攜貳綏懷,朝鮮歸款。世奉東籓,釐爾圭瓚。三十五郡,厥角稱臣。西被佛土,重譯來賓。濯征有明,耀師齊魯。電埽郊圻,有而弗取。略地松杏,屢殪敵軍。百戰百克,用集大勛。四章帝德廣運,昭受鴻名。建國紀元,永定大清。敦睦九族,彞倫式敘。尚德親賢,股肱心膂。三館是闢,鑒古崇儒。郊社禘嘗,式賁皇圖。爵秩以班,六曹承政。百工允釐,萬邦表正。威鑠函夏,德配蒼穹。敬承無斁,駿烈豐功。五章帝授神器,統一寰瀛。翦滅巨寇,乾坤載清。一著戎衣,若雨甘雨。大告武成,作神人主。躬親大政,飭紀整綱。制禮作樂,昭示典常。納諫任賢,慎微慮遠。定律省刑,萬世垂憲。克勤克儉,忠厚開基。景命維新,兢業自持。六章聖神建極,道冠百王。六十一年,福祚久長。天縱聰明,沖齡御宇。孝奉兩宮,德隆千古。三孽蠢動,一舉蕩平。海氛永靖,浪息長鯨。親御六師,三征沙漠。威肅惠懷,鋤頑扶弱。禹功底績,虞典時巡。敷天率土,莫不尊親。七章惟天行健,神聖則之。典學勤政,作君作師。無逸為箴,宵衣旰食。一人憂勞,綏此萬國。文經武緯,地平天成。中和立極,玉振金聲。億萬斯年,覲光揚烈。敬天勤民,體元作哲。天鑒孔彰,翼翼後王。儀型皇祖,帝祚遐昌。八章瑤圖炳煥,六合雍熙。星輝雲爛,風雨以時。翕受嘉祥,調和玉燭。治靄皇風,道光帝籙。東漸西被,北燮南諧。梯航琛贐,畢致堯階。日升月恆,萬拜蒙福。擊壤歌衢,嵩呼華祝。秩秩盛儀,洋洋頌聲。紹休列祖,永慶升平。九章

嘉慶元年,太上皇筵宴,慶隆舞樂九章

洪惟太上,景福自天。紀元周甲,席瑞循環。丙辰肇歲,寶命躬膺。昊慈默籥,祖武敬繩。初原克符,弗逾前紀。誕畀元良,丕承宗祀。孟陬朔旦,端啟重光。大廷授受,申錫無疆。精一執中,心傳欽守。媲美勛華,世躋仁壽。一章維聖握符,祗嚴昭事。肅肅泰壇,惇稱殷禮。兆南就位,有舉必躬。祈辛卜稼,祭雩占龍。陟降靈祗,精禋肸蚃。四序鈞調,百神歆享。雨暘寒燠,曰風曰時。八徵敬念,九醇熙。壽增上耋,弗懈益虔。茂膺多祜,翕應蕃駢。二章謨烈顯承,福基萬億。對越在天,升馨昭格。羹墻申慕,彞訓式欽。晨興惕若,寶籙披尋。四蒞陪都,珠丘展謁。締構艱難,緬維開國。威宣弧矢,化肅冠裳。祗循前典,曰篤不忘。繼繼繩繩,昭哉嗣服。於萬斯年,錫茲祉福。三章聖人敕政,綜攬萬幾。至誠悠久,維日孜孜。恭乃壽徵,健為乾體。洞照八埏,勵精十晷。丹毫批奏,彤陛延英。劭農履耤,展義巡行。相度塘堤,於河於海。肄武習勤,賚籓宣愷。純一不已,用介大年。貞恆保泰,往牒孰肩。四章民應如草,聖澤如春。大鈞默運,宙合同仁。五免丁糧,三蠲庾米。豁欠寬徵,恩覃肌髓。爰諮稼穡,普及封圻。偏隅有告,大賚龐施。民隱燭微,吏猷選最。岸獄平反,拊循攸賴。乾坤幬載,日月照臨。莫名帝力,允享天心。五章聖謨廣運,文德誕敷。道隆金鏡,象麗瑤樞。禮正圖編,詩釐樂府。四庫分排,七閣崇庋。講筵著論,史鑒宣評。圜橋集鼓,泐石橫經。文薈三千,詩裒五萬。雲漢倬章,日星炳煥。作人敷教,恩榜駢聯。楩柟杞梓,良材蔚然。六章皇猷赫濯,載纘武功。殊方奉朔,逖裔從風。準部回疆,天戈疊指。二萬輿圖,宅畋斯啟。金川再定,工⼙僰犁庭。樓船震懾,海嶠敉寧。緬孟句又關,交南羾闕。衛藏安禪,徼夷向日。鴻勛十告,馳驟禹湯。苗頑率服,纘繼紫光。七章範九五福,惟帝時承。億齡瑞啟,奕葉祥凝。寶篆「十全」,堂顏「四得」。甲子稽撓,貞元衍易。洪開壽,疊舉耆筵。珍府闡繹,景緯昭宣。惟德之基,惟福之積。芝檢文輝,蘿圖慶溢。九如曼羨,八表蕃釐。重輪繼照,光我皇儀。八章太上立德,咸五登三。崇稱卻讓,湛澤均覃。動植蕃昌,裨瀛照洽。子帝承顏,來昆引牒。祥源益溥,慶祚洪延。瑤圖汁紀,珠鬥輝躔。純嘏緝熙,康強逢吉。光啟帝期,永綏皇極。會元章蔀,正載經亹。原齊聖算,長頌臺萊。九章

乾隆二十六年,皇太后七旬萬壽,慶隆舞樂二十章

皇太后萬壽彌增,皇帝至孝以承。洪福同山海,歡聲率土騰。一章歲當辛巳建,壽屆七旬隆。怡愉太平日,舞蹈遍寰中。二章太后宏溥慈仁,宮闈式維均。綿延百世澤,長此樂長春。三章盛際集禎祥,祈年日正長。臣民敬申輿頌,慶覃恩敷八方。四章聖節開嘉燕,雍容舞疊獻。允茲祺壽綏,神人共歡忭。五章金鑰曈曨曉開,成行彩仗先排。臣工大小陪位,會朝稱慶無涯。六章皇帝仁孝兼至,聖母福壽同綿。載考古史所紀,罕得於斯盛焉。七章皞皞熙熙盛世,氤氤氳氳元氣。皇太后聖壽無疆,皇帝孝思不匱。八章如日之升,如月之恆。慈壽綿綿,如南山是徵。九章皇帝聖治臻隆,庶匯被澤胥濃。以茲悅懌聖母,允宜福壽攸崇。十章大孝章矣,茀祿長矣。敷天啿啿,樂時康矣。十一章聖時文教昌,萬匯欣解阜。多福允受茲,清寧並悠久。十二章異域輸誠,稱臣奉琛。屬國以萬數,雲集合歡心。十三章殊方一以平,德威既遐布。長治而久安,慈懷同增豫。十四章回部偕來賀,德化漸以深,戴恩永無極,送喜承皇心。十五章乃陟金陛兮,慶筵載陳。乃展舞採兮,至德洽於無垠。十六章鐘鼓既宿懸,和聲娛愷樂。九重進版圖,王會式增廓。十七章曰雨曰暘時若,省歲實維屢豐。泰宇既安既阜,嬉游何幸風同。十八章浹兮淪兮,沐湛恩兮。九州萬國,戴逾殷兮。十九章聖母延洪康且頤,行慶施惠人無遺。原集多祜綏維祺,勿替引之億萬斯。二十章

乾隆三十六年,皇太后八旬萬壽,慶隆舞樂十八章

聖母萬萬歲,既壽而康。皇帝逾六旬,孝治彌光。絪縕化宇兮,太和翔洽。諸福畢至兮,純嘏爾常。一章重光冒卯歲序新,萬壽八秩啟今辰。景祜自茲以永,慶日引而月升。二章皇帝舞採,聖母燕喜。協氣充周,福祿萃止。三章璇宮丹雘新增,卿云糺縵交凝。瑞符翕集,如松柏之茂承。四章奉安輿以時邁,乃東至於岱宗。陟喬嶽而行禮,百神衛祐咸來從。五章普陀肇靈剎,宗乘宣祝延慈釐。群籓諸部長,咸來膜拜瞻威儀。六章敷天合歡忭,共球萬邦獻。玉冊揚徽稱,煇煌晉萱殿。七章慶筵樂備,孫曾效舞。成文協節,嘏辭疊舉。八章壽如南山崇,福如瀛海廣。運會超郅隆,亙古實無兩。九章振振繩繩樂含飴,撙撙總總福履綏。奉進如意肩相隨,同祝聖壽徵攸宜。十章金門詄蕩開,彩仗棽麗陳。群工忭賀,莫不尊親。十一章八方太平日,負戴來紛闐。僉曰盛哉乎斯世,維申慶於萬年。十二章湛恩汪濊,中外禔福。臣庶戴德,久而彌篤。十三章鴻化迪矣,昭舄奕矣。群黎百姓,洽教澤矣。十四章慈訓式於九圉,維聖母之貽。醲膏浹無外,勿替引令祺。十五章歲功屢告豐穰,五風十雨兆祥。荷昊穹兮錫佑,協皇心兮降康。十六章遠籓內面誠殷,土爾扈特原歸我幅員。率戶口以數萬計,呼嵩鞠鯱來如云。十七章慈顏有喜安以愉,德洋恩普周寰區。純休永永慶那居,億萬斯年樂於胥。十八章

道光二十五年,皇太后七旬萬壽,慶隆舞樂九章

日升月恆兮,天行不息。惟聖母之壽,與天無極。一章淵渟岳峙兮,地道有常。惟聖母之壽,應地無疆。二章辰維良,月維吉。玉觴陳,金奏列。動六琯之春陽,七旬之慶節。三章皇帝奉爵,龍袞以侑。左撫舜琴,右酌堯酒。合薄海臣庶,為聖母壽。四章聖母燕喜,悅豫且康。乃稽慶典,載考彞章。覃恩闓澤,用錫祉於萬方。五章萬方有慶,四海同春。凡我髦士,以逮蒸民。仰思齊之盛化,咸蹈德而詠仁。六章和風習習,甘雨祁祁,嘉穀六穗。瑞麥雙歧。祥源福緒,惟聖母之貽。七章萬姓香花,千衢歌舞。敬祝聖母,誕膺多祜。樂意遍八埏,歡聲騰九土。八章皇帝聖德,惟聖母是承。龍池春麗,鳳液祥凝。永介慈福,延億萬齡。九章

同治十三年,皇太后四旬萬壽,喜起舞樂二十章

皇太后萬壽無疆,孝思共仰當陽。多福符三祝,維天降百祥。一章歲逢甲戌,恭遇四旬。普天祝嘏,身々辰又々。二章延洪綿寶籙,佳節慶長春獻升恆頌,歡聲遍九垠。三章九垠溥被崇釐,大化式於璇帷。萬姓瞻依切,千秋統緒垂。四章垂簾十一年,夾輔任親賢。長治久安歌永賴,武功懋兮文治宣。五章皇躬資撫育,訓政昭嗣服。燕翼荷貽謀,鴻庥多景福。六章生民遂,元化濡。遏朘削,寬租逋。普樂利,醉醍醐。熙熙皞皞,懌懌愉愉。七章殊方如砥平,聲教既遐布。建此丕丕基,慈懷信增豫。八章皇帝仁孝兼隆,聖母萬福攸同。盛世人民樂愷,清時景物照融。九章閶闔千門啟,安輿駕鳳來。德暉欣普照,歡喜上春臺。十章鬱鬱蔥蔥氣佳,行行採仗齊排。鷺序鵷班陪位,會朝稱慶無涯。十一章萱室開嘉燕,萬年觴疊獻。良辰壽而康,神人共歡忭。十二章我皇舞採,聖母情怡。太和翔洽,福祿來為。十三章奏五英,親九族。麟定歌,鴻恩沐。本支百世感深仁,天潢一派益敦睦。十四章籓王部長咸來賓,重譯殊方職貢陳。赤子之慕抑何深,凡有血氣同尊親。十五章奔走偕來趨闕廷,鞠鯱忭舞祝遐齡。錫之冠帶列籓屏,小懷大畏懍威靈。十六章皇帝治洽重熙,垂裳恭己無為。以茲怡悅壽母,允宜福履綏之。十七章大孝章矣,茀祿康矣。勿替引之,純嘏長矣。十八章五風十雨兆嘉祥,頻書大有告豐穰。盛哉斯世泰而昌,神功炳焯煥珠囊。十九章同游化宇戴高厚,躋金陛兮介眉壽。慈顏有喜安以愉,與天地兮同悠久。二十章

光緒十年,皇太后五旬萬壽,喜起舞樂二十章

至哉坤極,悠久無疆。猗歟令德,合撰含章。堯門疊瑞,姒幄披祥。一人有懌,萬壽彌臧。一章星麗南弧,日躔北陸。緹室葭飛,彤階蓂續。太史占云,伶倫候玉。上下和同,受天百福。二章天心復旦,聖節長春。罄壤齊慶,亙霄奉珍。月儀外宙,雲瑞中宸。運隆禮樂,感極天人。三章其禮伊何,璗冊璆章。鸞回寶勢,鳳欱奇香。金支景聚,璇衛雲張。肅雍長樂,艾熾耆昌。四章其樂伊何,韶韺亮希。天歌抗律,雲舞蹌儀。重華縵縵,八風回回。玉節金和,嗣音之徽。五章福以德昌,慶因善積。齊莊有臨,幾康無斁。訓流繭觀,風光椒掖。畢管書儀,傾璜奉式。六章助隆庶政,啟佑我皇。櫜威弓矢,輯瑞梯航。氈鄉即敘,島譯賓王。慈和遍服,保畏彌光。七章於維廣運,上媲昊穹。春育夏養,內帡外幪。薰琴解慍,嘉玉祈豐。萬年翔洽,百室熙隆。八章鄧林翹秀,昆岫搜奇。旁求俊乂,分職官師。四聰明達,三宅周咨。卷耳進賢,如歌風詩。九章聖人在上,席鬯綏和。慈云廕遠,愛日暉多。仁獸歸藪,靈禽在柯。士朝而忭,民野而歌。十章帝隆孝治,躬奉天經。凝旒沖幄,鳴玉慈庭。文容屬屬,舜慕蒸蒸。百禮既至,四海其承。十一章丕顯宗親,蕆時嘉會。溯月瞻星,編珠貫琲。璇源益濬,玉葉知芘。訓儉示恭,行慶施惠。十二章濟濟卿士,將將會朝。人瞻丹扆,天臨慶宵。縟文炳藻,睿孝圖瑤。呼嵩祝華,頌魯歌姚。十三章亦有籓長,綴於朝儀。酎珍溢阼,貢嘏駢墀。來賓璋馬,往賚金犀。既荅皇祉,咸歡壼彞。十四章帝命重申,如綸如綍。劭農賜租,勸位詔祿。熙熙春臺,渠渠夏屋。慶賞雲興,歡聲雷速。十五章昭禮告秩,慈顏愷康。旋輝玉墄,鳴豫瑤觴。矞皇聖孝,菴藹休昌。鉤鈐既朗,延嘉亦芳。十六章黃屋長今,彤箴自古。媯汭嬪虞,塗山贊禹。徽德孔明,重規襲矩。以祉元吉,宜延遐緒。十七章甲子章蔀,循環無端。易圖大衍,羲畫先天。箕疇演福,軒策調元。維天佑聖,於斯萬年。十八章嶟嶟五岳,而岱之宗。濊濊百川,而海斯容。兩儀清穆,八表熙雍。湛恩波沛,峻算山崇。十九章靈貺便蕃,昌期綏茂。帝暉緝熙,母儀純佑。遐邇壹體,小大稽首。壽考維祺,克昌厥後。二十章

光緒二十年,皇太后六旬萬壽,喜起舞樂二十章

皇太後福壽同綿,皇帝仁孝兼全。天佑聖母,錫之大年。一章閼逢歲之陽,其陰在敦牂。其日維吉,其月曰良。二章王會大同,星紀五復。萬國萬年,以介景福。三章猗歟母儀,翼我聖主。曰仁曰智,允文允武。四章其武維何,謐謀璇幄。龕靖神州,威讋殊俗。五章其文維何,崇儒禮賢。奎章藻耀,雲漢在天。六章其智維何,明燭萬里。中外一家,宮府一體。七章其仁維何,如湯如堯。蠲租發帑,以恤民勞。八章民勞休止,庶優游止。雖休勿休,民瘼求止。九章自普天而率土兮,咸浹髓而淪肌。聖皇之德兮,聖母之慈。十章茂矣美矣,薦嘉祉兮。唐矣皇矣,純嘏爾常矣。十一章累印若綬,颺拜稽首。壤歌衢謳,逮及童叟。十二章累譯而至,屬國以萬計。咸含和而吐氣,頌曰盛哉乎斯世。十三章大矣孝熙,聖皇之思。以天下養,永奠此丕丕基。十四章行慶施惠,湛恩汪濊,而熾而昌。眉壽無有害。十五章乃鏤璆冊兮彖瑤章,乃展瓊筵兮奉玉觴。乃瞻金陛兮穆穆皇皇,乃奏雅樂兮喈喈將將。十六章琴瑟在御,鐘罄在虡。(鼓長)乎而鼓,軒乎而舞。十七章蕩蕩八荒,惠問所翔。原聖母壽,應地無疆。十八章圜穹戴笠,徽音四塞,原聖母壽。與天無極。十九章荷天衢,提地釐,迄於期頤。萬有千歲,福履綏之。二十章

乾隆四十五年,高宗七旬萬壽,慶隆舞樂九章

皇帝萬萬壽,福如大海源。浩元氣兮春和溫,澤洋溢兮彌乾坤。一章歲維庚子,恭遇七旬。太平有象,鴻禧日新。二章班禪覲後藏,十方皈依舉延企。瑞靄集豐年,廣法輪,宗風被。三章麗正門,開詄蕩。王公大臣,拜舞瞻天仗。旗罨靄,芬芒芒。五雲朗,爐煙上。四章敬天勤民久,純德四海敷。皇帝壽,萬萬年,孔固南山如。五章蒙古眾臺吉,青海衛拉特。愛之如赤子,傾心世歸德。六章土爾扈特歸順,武義金川威震。如天大德曰生,蹈舞揚休入覲。七章曼壽多福,臚歡無疆。如松柏茂,萬葉純常。八章皇子及孫曾,稱觴介眉壽。祝鴻禧兮歲其有,與天地兮同悠久。九章

乾隆五十五年,高宗八旬萬壽,慶隆舞樂十八章

皇帝萬萬壽,福如大海源。亭育德恩普,休和暢八埏。一章歲維庚戌,恭遇八旬。神人祝嘏,景福益臻。二章洞開九重,闢公呼嵩。祥光罨靄,歌舞攸同。三章敬天勤民,歲書大有。萬壽無疆,山嶽悠久。四章文光炳二曜,武烈宣萬方。義正以仁育,遐邇胥來王。五章燕千叟兮嘉龐眉,賜筇帛兮拜鴻施。臚歡介祉兮叩彤墀,群登壽兮祝蕃釐。六章篤天潢,賜章服。燦五採,親九族。本支百世感殊恩,歡洽群情益敦睦。七章臨雍釋菜,文教振興。人材樂育,為國之楨。八章輯四庫書,譽髦鼓舞。惠茲藝林,上下今古。九章修藏兮譯金經,廣善緣兮福群生。慈雲布濩兮光晶瑩,和風甘雨兮彌八紘。十章籓王部長咸來賓,遐荒重譯職貢陳。依光慕化同尊親,赤子之慕中外均。十一章安南國王趨闕廷,鞠鯱忭舞祝億齡。寵以冠帶列翰屏,聲教遠暨海國寧。十二章緬甸來庭,寵膺綸詔。頒印錫封,永綏炎徼。十三章生番鄉化,傾心太平。恩浹肌髓,威畏惟誠。十四章聖明四照,福綏綿綿。吉祥屢臻,億萬斯年。十五章永承天庥,祥徵滋至。載頌九如,祚延萬世。十六章子孫曾玄戲採舞,壽而康兮祝純嘏。歲歲年年福履增,天地合德同博溥。十七章中和舞樂邁韶韺,普天率土歡同聲。慶萬壽兮茀祿膺,受天佑兮莫不承。十八章

嘉慶十四年,仁宗五旬萬壽,慶隆舞樂九章

皇帝萬萬壽,壽與天無疆。秉德貞恆篤鴻祜,珍符曼羨恩滂洋。一章歲維己巳,聖節五旬。六合昌阜,啿啿陽春。二章我皇功德冒八極,埽除群慝登衣任席。磑磑即即師象山,永綏生民偃兵革。三章民生遂,元化濡。遏朘削,寬惸逋。饜餐粥,襲裾襦。樂皞皞,安愉愉。四章作之君,作之師。孔容保,誕教思。厚莫厚,訓宗支。仁莫仁,箴八旗。五章繼統恭勤兮儉德先,有孚惠心兮靡回延。泉府充羨兮軫民艱,馮蠵輯和兮功濬川。六章重民耕織,雨暘寒燠。圖輯授衣,纂志祖考。大孝備矣,養民為寶。七章闢公卿士,俊髦龐蒙。雲施山應,降福屢豐。綿綿瓜瓞,上怡皇衷。八章承昊佑兮撫八紘,洪景命兮方升恆。率土臚歡兮,重譯職貢,於萬斯年兮,福祿永膺。九章

嘉慶二十四年,仁宗六旬萬壽,慶隆舞樂九章

九有嘉吉萬匯昌,貞冬嫗煦日載陽。帝承昊貺錫兆庶,聖壽曼羨長無疆。一章十幹十二枝,紀歲周復始。皇帝壽齊天,循環萬甲子。二章北暨窮發南雕題,耕桑直過昆侖西。黃河安恬日東注,波澄鏡海騰朝曦。三章景風翔兮卿雲升,民游壽宇兮化日恆。世龐鴻兮多耇耋,生逢太平兮由高曾。四章厚民生,省厥慝。立嘉禾,稂莠茀。崇正教,恥且格。惠元元,遍帝德。五章繼皇統兮承祖澤,王業艱難兮孝思靡極。眷遼沈兮夙法駕,式考訓兮永無斁。六章帝廑烝民,拯之德政。曰雨曰暘,天心協應。熙熙春臺,豐年屢慶。七章日月方升恆,川嶽咸效順。祝嘏萬方同,梯航集琛贐。八章歲己巳兮恩普錫,今茲己卯兮六旬聖節。帝澤汪濊兮,海樂康,原逢旬慶兮萬有千億。九章

乾隆四十八年,乾清宮普宴宗親,世德舞樂九章

天開聖清,覺羅肇興。列祖繼緒,統一寰瀛。一章溯祥長白,垂統發跡。幅員廣大,景附悅懌。二章聖皇立極,與天比崇。純常茀祿,昌後隆宗。三章篤親九族,錫恩單厚。金黃帶垂,峨冠品授。四章枝蕃萼榮,皇情則怡。嘉承天和,方春載熙。五章璇宮肆筵,宗人爰集。黼繡盈庭,班行辨級。六章皇睠有喜,便蕃賚予。侍衛賜茶,恩涵露湑。七章肫仁溥澤,大府頒金。皇慈既渥,以洽壬林。八章宗人拜舞,臚歡忭祝。億萬斯年,永綏多福。九章

乾隆初,巡幸盛京,筵宴,慶隆舞樂一章

皇天明命,篤生太祖。錫之聖智,奄有東土。於聖太祖,開基創業。始制國書,同文六合。曰若太宗,嗣承天命。肇造區夏,仁育義正。太宗如天,丕冒純德。於鑠大清,懋建皇極。興京聿興,盛京斯盛。惟其至仁,九有託命。欽惟聖皇,追慕深思。率祖攸行,惠我嘉師。敬觀實錄,日星為昭。祖業艱難,中心忉忉。乃頒明詔,播告臣氓。恭謁祖陵,旋軫陪京。皇帝篤誠,珠丘展覲。文武從臣,駿奔效藎。我皇聖哉,細大不遺。從臣文武,體恤周知。乃出邊關,乃經蒙古。閱七愛曼,蕃部悅舞。禦光遠臨,旃裘畢來。宸衷軫念,錫賚恩恢。受我皇恩,祝我聖皇。合十膜拜,恩膏溥將。蒞克爾素,駕言行狩。手格虎殪,馬射熊僕。英哉我皇,舍矢如破。獲獸孔多,萬人腹果。爰蒞舊邦,爰謁三陵。既躬既親,我心則平。豈敢憚遠,豈敢畏險。至止禮成,心猶繾綣。皇帝大孝,承祭至敬。肅將明禋,萬邦為鏡。皇仁懋哉,重齒敬老。清問殷勤,德施浩浩。盛京蒞止,臨朝閱武。御崇政殿,恩敷率土。

乾隆八年,巡幸盛京,大宴,高宗御制世德舞樂十章

粵昔造清,匪人伊天。天女降思,長白闥門。是生我祖,我弗敢名。乃繼乃承,逮我玄孫。一章玄孫累葉,維祖之思。我西雲來,我心東依。歷茲故土,仰溯始謀。皇澗過澗,締此丕基。二章於赫太祖,肇命興京。哈達輝發,數渝厥盟。如龍田見,有虎風生。戎甲十三,王業以興。三章爰度爰遷,拓此沈陽。方城周池,太室明堂。不寧不靈,匪居匪康。事異放桀,何心底商。四章丕承太宗,允揚前烈。倬彼松山,明戈耀雪。以寡敵眾,杵漂流血。惜無故老,為餘詳說。五章餘來故邦,瞻仰橋山。慰我追思,夢寐之間。崇政清寧,載啟南軒。華而不侈,鞏哉孔安。六章維我祖宗,欽天敬神。執豕酌匏,咸秩無文。帷幔再張,尊俎重陳。弗渝弗替,遵我先民。七章先民宅茲,載色載笑。今我來思,聖日俯照。爵我周親,藎臣並召。亦有嘉賓,歡言同樂。八章懿茲東土,允維天府。土厚水深,周原膴膴。南陽父老,於是道古。有登其歌,有升其舞。九章我歌既奏,我舞亦陳。故家遺俗,曷敢弗因。渾灝淳休,被於無垠。勿替引之,告我後人。十章

乾隆十四年,金川凱旋,筵宴,慶隆舞樂一章

乾隆聖世,瀛乂康。元首惟明,股肱惟良。景運鴻昌,休德茂著。統馭八埏,惠液遐布。金川小醜,蠢爾冥頑。惟時弗率,跳梁窮邊。用申天討,聲罪執言。長驅驛駕,油雲斯屯。聖謨廣運,決機萬里。聿簡賢臣,良弼是倚。曰忠曰勇,經略戎功。心堅金石,誠格蒼穹。身先烝徒,跋履巖阻。晨夕盡瘁,均勞共苦。奸宄是殛,逆諜是攘。國憲孔昭,我武孔揚。窟穴梨止,嵒巢圮止。堅卡隳止,峻碉毀止。爰褫其魄,爰喪其膽。震懾股慄,潛伏於坎。如鼠竄穴,如鱗游釜。號呼哀籥,再三求撫。元臣執義,憤欲蕩除。帝德好生,曰免駢誅。六條攸約,虔慄遵循。恩綸祗奉,解網施仁。丕革厥心。匍匐奔赴。除道築壇,香雲擁路。帝仁覃敷,六合滂洋。蘇其枯朽,賜以再生。展也滿兵,彌月功成。云何其速,皇猷是憑。神功炳焯,乾坤軒豁。九重勝算,明並日月。勒諸瓊玖,昭諸汗青。禮成鉅典,樂奏升平。膚功克奏,慶筵是侑。太和氤氳,翔洽宇宙。卿云糺縵,景緯珠聯。梯航琛贐,億萬斯年。

乾隆二十五年,西域平定,筵宴,德勝舞樂一章

祖志繼成,翦滅遠叛。籌畫從容,疆闢二萬。川原式廓,乃經土田。廟算宏深,天心契焉。車楞內訌,丐恩臣服。爵錫王公,周恤其屬。鬼蜮阿酋,匍匐帝庭。寵以籓服,秉鉞專征。俘達瓦齊,再生曲宥。念彼軍勞,崇封晉授。阿酋狡獪,將伏天誅。妄冀非分,叛於中途。反覆二心,棄厥妻子。役屬離散,巨惡宜爾。汲水萬里,欲息燎原。似彼狂酋,徒然自燔。竄俄羅斯,疫戕其命。遐方尊王,爰獻於境。滿洲索倫,凌波飛渡。奮勇莫當,峻嶺爰度。俘厥逋逃,收彼牲畜。取彼子女。如摧朽木。弓矢所加,賊壘莫御。急思兔脫,震駭無措。蠢爾賊眾,作亂變更。帝德涵濡,惸獨遂生。豈曰窮兵,豈曰黷武。乘時遘會,忍弗遠撫。回首在囚,解其禁錮。甫還庫車,流言煽布。惟彼兇渠,負君莫比。罔念聖恩,能弗切齒。伊犁既戢,諸部賓服。豫策久長,悉收回族。二賊潰逸,命將追遏。逾其穴巢,直抵巴達。爰遣侍衛,乃得其情。回長搏顙,獻馘輸誠。邃古莫稽,列史具在。殲寇如斯,未有儗類。欃槍凈掃,寰宇升平。師出以正,中外永清。睿謨惟誠,宵旰無逸。宏奏膚功,聖心斯懌。順我者昌,逆我者亡。旌別恐遺,天語孔彰。酬庸封爵,表勇錫名。昭茲懋賞,章採聿明。誅鋤元惡,大功告成。如春育物,德合清寧。

道光八年,重靖回疆,筵宴,德勝舞樂二十章

道光聖世,德洽紘埏。獻琛奉贐,有翼有虔。一章蠢茲逆回,逋誅小醜。喙伏荒裔,敢為戎首。二章麌々其群,驛驛其氛。涉卡潛煽,不戢自焚。三章皇赫斯怒,爰命揚威。汝往討亂,執訊以歸。四章渾巴什河,先聲克振。進殲柯坪,靡有遺燼。五章沙崗鷁鷁,我兵既攻。三莊埽穴,十日奏功。六章奏功一月,四城迅復。伯克跪迎,額頌神速。七章莫赤匪狐,莫黑匪烏。星弧所指,並伏其辜。八章帝軫八城,蠲厥租賦。耄齡歡慶,九天甘澍。九章於鑠宸謨,十條誕敷。罪人務得,勿俾稽誅。十章稽誅勿俾,徵師勿俟。次第凱還,以息勞勩。十一章蒼莽四山,偵騎周環。妖鳥攸投,庭弓攸彎。十二章歲既宴矣,烝徒驤驤。制梃撻賊,阿圖什莊。十三章我追彼竄,自昏徂旦。抵鐵蓋山,去路倏斷。十四章爰喪其馬,爰曳其兵。奪彼短刃,縶以長纓。十五章城柳初荑,驛騎載馳。都人夾道,遙望紅旗。十六章櫜弓錫組,告於天祖。歸善慈闈,孔曼受祜。十七章勒碑志事,御門受俘。聿啟喜宴,露湛雲需。十八章旨酒既嘉,隊舞入侑。小大稽首,我皇萬壽。十九章聖武維揚,聖恩維長。饜仁飫義,萬壽無疆。二十章

大宴笳吹樂六十七章乾隆七年定。

牧馬歌人君之樂,恃此紀綱。兆民之樂,恃我君王。室家孔宜,夫君之力。朋友有成,和輯之德。

古歌八種成壞兮,實人世之常。墮迷網中兮,欲鎖與情韁。愚人無識兮,樂茲殊未央。執空為有兮,謬語其奚當。

如意寶不澡心於群經,具本性而無明。不服膺於佛乘,說妙行而聽熒。

佳兆一人首出,萬國尊親。湛恩汪濊,普被生民。百花敷榮,一日悅目。灌頂寶光,

萬眾所伏。

誠感辭良胡畏哉,襄以至誠。良胡過哉,竭己所能。良胡偽哉,語無文飾。良胡怠哉,罔敢休息。

吉慶篇有君聖明逾戴天,有臣靖共勝後嗣。健婦持家過丈夫,如意寶珠惟孝子。

肖者吟滅除己罪,仗佛真言。如欲療病,惟良藥存。菩提鐙兮,出眾生於黑暗;智慧梳兮,櫛六欲之糾纏。

君馬黃大海之水不可量,天府寶藏奚渠央。良朋和睦益無方,聖有謨訓垂無疆。

懿德吟人君能仁,烝黎之父。君子和平,群相肺附。懿厥哲人,實維師傅。匿智懷私,乃民之蠹。

善哉行惟安惟和,心意所欲。無貳無虞,朋友式穀。

樂土謠分人以財,惠莫大焉。施人以慧,寧不逾旃。

踏搖娘日將出兮,明星煌煌。壽斯徵兮,秀眉其龐。三十維壯,五十遲暮。莫親祖母,莫尊祖父。

頌禱辭我馬蹀躞,行如流水。俊英滿座,交親悅喜。族黨★I2婭,咸富且貴。酌酒為歡,既多且旨。

慢歌十五歡娛八十衰,壯容華茂遲暮悲,祖妣最親祖尊哉。

唐公主遵王之路兮,愆尤希。素位而行兮,夫奚疑。

丹誠曲罔有敗事兮,遵道而行。長無離析兮,順親之情。

明光曲瞻彼日月,虛空發光。聖君聖母,焜燿萬拜。

吉祥師日月之明兮,容光必照。聖君之明兮,烝黎咸造

聖明時際聖明時,良我福只。橫被恩澤,良我祿只。

微言倏忽變遷,順其自然。如彼蜃樓,餘生渺焉。

際嘉平諸惡莫作,菩提薩多。暝曚妄行,用墮三塗。

善政歌經何本,本於宗。身何本,媼與翁。罪何本,嗔蟲々。福何本,和雍雍。

長命辭靡言不適於道兮,水萬派而朝宗。夫惟外道之妄語兮,井自畫而不通。

窈窕娘惴惴原獸,思全其身。兢兢庶士,思庇後昆。

湛露維彼愚人,惟知己身。維此哲人,心周萬民。

四賢吟六欲相牽,微生是戀。嘆彼駒光,如夢如電。

賀聖朝慈悲方便,永斷疑情。極樂凈土,不滅不生。

英流行知之而作兮,明哲所由。不知而作兮,庸愚之儔。慮而後動兮,卓彼先覺;率而妄動兮,是乃下流。

堅固子馬蹀躞兮,身不獲康。念此身兮,本自無常。馬騰驤兮,生不獲寧。念此生兮,本自無生。

月圓良馬之德,於田可徵。良朋之行,相交乃明。

緩歌良馬云何,乘者所思。良朋雲何,久而敬之。

至純辭惟帝力兮勞來,父母力兮免懷。乘騏驥兮馳驟,仗巨擘兮弓開。

美封君貢高專美,曰惟不仁。擁貲自厚,不久四分。惟不惺惺,乃不戒懼。兇心常萌,誰與共處。

少年行嗟棄捐於巖穴兮,盍遠播夫芳聲。嗟終老於草莽兮,盍永垂夫令名。

四天王吟悲哉北邙,令聞宣揚。北邙悲矣,青史不渝。

宛轉辭瞻彼中林,芃芃萬木。旃檀有香,生是使獨。萬類咸若,攘攘蕓蕓。民之父母,首出一人。

鐵驪載飛載翔,惟翮是憑。為聲為律,惟心是經。射之能中,惟指是憑。交之能善,惟和斯恆。

木珠鷇之成雛兮,孚化之功。羽用為儀兮,賦命之隆。迪彼愚蒙兮,惟聖之功。明厥本性兮,實在己躬。

好合曲維勤斯哲,安不可懷。溺茲小樂,至樂難期。

章阜乾照無私,聖教無類。謨訓洋洋,鑒茲不昧。

天馬吟騏驥不群蹇驢,鴻鵠不偕斥鷃。騶虞不邇狐貍,聖哲不暱愚賤。

大龍馬吟疇知幻軀,祕此佛性。疇不退轉,佛恩來證。上德墮落,疇其知病。下士頓起,疇其知競。

始條理福慧天亶,誠哉難覯。通人達士,豈奚易逅。

追風赭馬蔥兮蒨兮,山有芳蘭。僮兮祁兮,首有妙鬘。

回波辭元首明哉首出,股肱良哉罕匹。賢夾輔兮王室,莫執左道兮蟊賊。

長豫景行行止,下民堪憐。宜汎愛眾,毋逆忠言。

平調騏驥適我體,櫜鞬衛我身。嘉言資我道,經史沃我心。

游子吟升彼高阜兮,思我故鄉。有懷二人兮,莫出戶堂。陟彼崔嵬兮,思我故鄉。有懷二人兮,莫出垣墻。

平調曲帝王無逸,天地和寧。闢公膚敏,兆民阜成。

高士吟日之升,天為經。民之行,君為程。水之流,隨坎盈。牝之游,駒之情。

哉生明非冒於貨賄也,感兄弟之敬心。非貪於飲食也,感父老之誠忱。

高哉行雲何致太平,睪然望皇衢。人生夫何常,善保千金軀。民之不能忘,令名照神區。子孫振繩繩,百千萬億餘。

三章敬尊佛敕,如滋甘雨。莫行邪惡,種茲罪苦。

圓音身無常,花到秋。名無常,雷不留。財無常,蜂釀蜜;水無常,海發漚。

欄桿賢者斯賢賢,不賢不賢賢。蜜蜂見花駐,蜻蜒去翩翩。

思哉行千金寶馬,不如先人之畀遺。嘗盡諸果,不如母乳之甘兮。

法座引電可畏兮,時屆硃明。霜可畏兮,五穀將登。禍可畏兮,歡樂所成。罔不畏兮,憶神魂之初降生。

接引辭火宅無凊涼,苦塗無安樂。鳥路誰能攜,閻浮難駐腳。

化導辭閻浮提界,如彼高山,越之維艱。盡卻今時,大海漫漫,欲渡良難。

七寶鞍瞻彼堤岸,水則不濫。有君牧民,當無畔散。飛鳥雖疲,寧甘墮地。君子固窮,之死不二。

短歌嗟餘生之歡樂兮,似黃離之盈昃。感韶光之荏苒兮,似葉上之青色。及芳華之當齒兮,且喜樂以永日。

夕照時乎時乎,時外無時。時其逝矣,奚與樂為。黃離既昃,定少溫暾。天光既暮,曀曀其陰。

歸國謠皇矣聖世,藹如仁君。懷哉懷哉,日遠日分。亦有良朋,如兄如弟。日遠日離,不能遙跂。

僧寶吟投誠皈命,既安且吉。如佛塔廟,云胡遠別。和樂且耽,手足提攜。如姊如娣,云胡遠離。

婆羅門引酪必成醐,父將成祖。沙必成丘,母將成嫗。

三部落試觀三界,漚起漚滅。如彼秋雲,乍興乍沒。

五部落流水何湯湯,吾生如是游。雖有聖賢人,誰能少滯留。

乾隆二十五年,西域平定,筵宴,笳吹樂一章

閶闔煌煌,鐘鏞鏘鏘。鳴鞭祗肅,帝用燕康。荷天純嘏,祖德凝庥。從容底定,允升大猷。聖德宏敷,光被遐邇。如拱北辰,諸部歸止。慈恩覆幬,滄海無量。爭先效順,奔走來王。憲章斯備,勝算克成。跳梁群醜,魚貫輸誠。天威震疊,小腆惕厲。大君惟仁,莫不臣隸。聖教宏敷,額手格心。月竁同風,越邁古今。遠謨是協,絕徼安康。撫綏之德,遍於遐荒。乾元功懋,滂洽垓埏。巍巍盛德,億萬斯年。

道光八年,重靖回疆,筵宴,笳吹樂九章

於赫皇威,式於九圍。回疆耆定,飲至勞歸。一章有截回疆,純皇所綏。畏神服教,鞏我籓籬。二章蠢茲逆裔,逋誅再世。燎原自焚,法不可貰。三章戎車爰西,如雲如霓。一月三捷,四城其徯。四章四城既治,醜黨既夷。彀弣張,妖鳥安之。五章回莊歲邇,有鴞萃止。躡跡窮追,鐵蓋孔巋。六章縶以白組,報以紅旗。新春送喜,皇心載怡。七章昔賦出車,今歌採薇。受釐天祖,歸善慈闈。八章嘉獻允儀,溥哉恩施。奉觴稽首,萬壽維祺。九章

大宴,番部合奏三十一章乾隆七年定,惟大合曲、染絲曲、公莫、雅政辭、鳳凰鳴、乘驛使六章有辭;無辭者有宮譜:曰兔置,曰西鰈曲,曰政治辭,曰千秋辭,曰鴻鵠辭,曰慶君侯,曰慶夫人,曰羨江南,曰救度辭,曰大番曲,曰小番辭,曰游逸辭,曰興盛辭,曰艷冶曲,曰慶聖師,曰白鹿辭,曰合歡曲,曰白駝歌,曰流鶯曲,曰君侯辭,曰夫人辭,曰賢士辭,曰舞辭,曰兆鼓鼓曲,曰調和曲。不載。

大合曲元縡是依,明神是祗。一心至誠,昭事勤只。巍巍大君,永底蒸民。中心愛戴,稽首來臣。念人生之無常兮,合勤修夫善行。信百行之咸善兮,終和平而神聽。

染絲曲大君至聖,教敷率土。敉寧萬邦,拜跪奉主。

公莫丕顯元後,惠懷萬方。國彥棐恭,協贊邦常。率土之濱,誠意溥將。咸拜稽首,依戴聖皇。

雅政辭皇皇聖明,無遠弗燭。林林眾庶,無思不服。元化惠心,為善去惡。聖人之邦,長生永樂。

鳳凰鳴承乾體元,惟我聖君。光開草昧,惟我聖君。綱紀庶政,惟我聖君。父母萬國,惟我聖君。惟我聖君兮,覆幬如天。惟我聖君兮,自新新民。惟我聖君兮,中外乂安。惟我聖君兮,群慝消淪。拜手稽首兮,頌溢兆民。

乘驛使大地茫茫,大海滄滄。豈伊無寶,求之奚方。自古在昔,為君為王。膺圖禦宇,命不於常。實心實政,惠此萬邦。聖御大寶,繄惟我皇。繄惟我皇兮,疇可與之頡頏。

回部樂曲一章律呂後編回部樂曲國書用漢對音而旁注宮譜,今以漢對音載其辭。

思那滿塞勒喀思,察罕珠魯塞勒喀思。


\end{pinyinscope}