\article{列傳一百八十}

\begin{pinyinscope}
李星沅周天爵勞崇光

李星沅,字石梧,湖南湘陰人。道光十二年進士,選庶吉士,授編修。十五年,督廣東學政。粵士多健訟,檄通省籍諸生之乾訟者,牒報詗治之,士風以肅。任滿,授陜西漢中知府,歷河南糧道,陜西、四川、江蘇按察使。在川、陜嚴治刀匪、啯匪,屢擒其魁置之法。遷江西布政使,調江蘇。二十二年,擢陜西巡撫,署陜甘總督。二十五年,調江蘇巡撫。二十六年,擢雲貴總督,兼署云南巡撫。

先是,永昌回亂,迤西道羅天池濫殺,不分良莠,眾回益擾。總督賀長齡、提督張必祿急於主撫,降者輒復叛。至是,緬寧匪首馬國海被剿亡走,潛結雲州回馬登霄、海連升等復起事,迤西大震。星沅追論肇亂之由,長齡、天池並獲譴。二十七年,遣兵進剿,解散被脅回眾,首逆就殲,餘匪肅清。詔嘉其功,加太子太保銜,賜花翎。尋調兩江總督。

星沅未第時,客陶澍幕中,為掌章奏。又歷官江南,習於鹽、漕、河諸利弊。時度支告匱,廷臣主南漕改徵折色解部,於北省採買。星沅謂折多徵收不易,折少採買不敷。穀賤銀貴,民間展轉虧折。且州縣藉端浮勒,胥吏高下其手,防之皆難。迭疏論列,議遂寢。

淮鹽自陶澍整頓之後,歷年又多積欠。星沅疏陳引鹽壅積、課款支絀情形:「揆厥所由,官以畏難而因仍,商以畏難而取巧。成本增於雜費,行銷滯於售私,年復一年,幾同痼疾。先當以內清場私,外敵鄰私,為急則治標之計。本年回空糧私,奏請查禁。其川私、墾私、潞私、浙私,均咨行堵緝。又引船夾帶,為害最鉅,扼要搜查,於揚州仙女廟及江寧下關緝獲百餘萬斤,提省審辦。他如慎出納,提緩課,派懸引,刪繁文,配運殘引,提售新鹽,裁浮巡費,禁捏報淹銷,酌議章程八條,以圖整理。」疏入,下部議行。

舊制,總督兼管河務,自道光二十二年後停止,至是復命兼管。會兼署河督,疏請嚴禁員聚處清江,飭各歸工次。奏籌外海水師事宜,曰磨厲人才,曰變通營巡,曰覈實會哨,曰扼要堵緝,曰配兵足數;又請添造戰船,勸捐給獎:並允行。俄羅斯通商舊由陸路,忽有商船至上海,執約拒之。在任兩年,宣宗甚加倚任。因久病,請解職回籍,允之。

三十年,宣宗崩,赴京謁梓宮,復以母老陳請歸養。會廣西匪亂方熾,起林則徐督師,卒於途,命星沅代為欽差大臣。是年十二月,抵廣西,駐柳州。時左右江匪氛蔓延,諸賊尤以桂平金田洪秀全為最悍。巡撫鄭祖琛、提督閔正鳳皆以貽誤黜去,周天爵、向榮繼為巡撫、提督。二人者並有重名,負意氣,議輒相左,星沅調和之,仍不協,軍事多牽掣。咸豐元年春,向榮進剿,賊由大黃江、牛排嶺竄新墟、紫荊山。星沅檄總兵秦定三、李能臣率滇、黔兵追躡,賊復竄武宣。榮、天爵各進擊,賊踞東鄉,兩軍攻之不克。星沅以事權不一,奏請特簡總統將軍督剿,詔斥其推諉。尋命大學士賽尚阿率總兵達洪阿、都統巴清德赴湖南防堵,將以代之。賽尚阿至湖南,遂授欽差大臣,赴廣西督師,命星沅回湖南治防。四月,星沅力疾赴武宣前敵督戰,至則已憊甚,數日卒於軍。遺疏言:「賊不能平,不忠;養不能終,不孝。歿後斂以常服,用彰臣咎。」文宗覽而哀之,依總督例賜恤,賜金治喪,存問其母,子二人命俟服闋引見,謚文恭。子桓,官至江西布政使。

周天爵,字敬修,山東東阿人。嘉慶十六年進士,歸班銓選。道光四年,授安徽懷遠知縣,調阜陽。天爵少以堅苦自立,篤信王守仁之學。及為令,盡心民事,廉介絕俗,皖北盜賊橫恣,與胥吏通,天爵極刑痛懲之。有劾其殘酷者,總督蔣攸銛奏言:「天爵愛民如子,嫉惡如仇,古良吏也。」由是受宣宗之知,諭曰:「不避嫌怨之員,最為難得,小過可宥之。」連擢宿州知州、廬州知府、廬鳳潁泗道。所至捕盜魁,無漏網者。十五年,擢江西按察使,仍調安徽,遷陜西布政使。

十七年,署漕運總督,尋實授。時漕務積弊,運丁水手尤恣悍,特用天爵嚴馭之,劾衛官十二員以儆眾,詔褒勉之。

十八年,調署湖廣總督,尋授河南巡撫,擢閩浙總督,皆未行,調授湖廣總督。漢口鎮為商船所聚,苦盜。川匪充鉛船水手,每行劫殺人;陜、楚交界奸徒掠販婦女,並為民害:天爵捕治如律,劾失察有司及承審縱延者,悉褫其職。荊州沿江舊於冬季委員巡緝盜賊,天爵謂屬具文,罷之;遴幹吏暗偵,與地方官掩捕,以獲盜多寡定功過。襄陽匪徒傳習牛八邪教,又有天主、十字各教,捕誅數十人。每有疏陳,宣宗輒手詔褒嘉。連年水災,濱江、濱漢堤垸多壞,疏請依治黃河法,遇險立挑壩,並以草護堤;飭治河州縣,有大工解任專治,立限保工,限內失事者罰,紳董亦如之;漢水多灣曲,立磚石斗門以備蓄洩:並如議行。

天爵馭吏嚴,多怨者。二十年,己革大冶知縣孔廣義揭訐多款,天爵置不問。事上聞,嚴斥之,議革職留任。尋言官劾天爵酷刑,與廣義言略同,命侍郎麟魁、吳其濬往按,得天爵信任候補知縣楚鏞用非刑,外委黃雲邦誣執良民諸狀,上震怒,褫天爵職,戍伊犁。二十一年,命赴廣東交靖逆將軍奕山差遣,尋免罪,留粵效力。二十二年,予四品頂戴,以知府候補,調江蘇辦理清江防務。海防事竣,留治淮、揚善後事宜,尋予二品頂戴,署漕運總督,兼署南河總督。二十三年,因濫刑及失察漕書私鐫關防,連被吏議,疏請去職,命以二品頂戴休致。

久之,廣西賊起,日益熾。文宗御極,求知兵大臣,尚書杜受田以天爵對,遂起廣西巡撫,偕欽差大臣李星沅辦賊。咸豐元年春,親率兵與向榮會剿金田匪洪秀全等。賊竄武宣東鄉,合擊於東嶺村,力戰,兵有退者,天爵手刃之,援桴鼓而前,賊始卻。時懷集、賀縣及都康、下雷土司,凌雲、東蘭、橫州、博白並有匪踞,檄各屬力行團練,合力防剿。詔加天爵總督銜,專辦軍務,以布政使勞崇光攝巡撫事。天爵年近八旬,每戰親臨前敵,惟與李星沅、向榮皆不協。星沅既疏請特簡總統督師,尋病歿,命天爵暫署欽差大臣。賊由武宣竄象州,詔斥天爵等相持日久,不能制賊,褫總督銜,解軍務,回省暫署巡撫。洎賽尚阿至軍,議復不合,自陳衰病,詔命來京。既至,連召對十一次,極言軍事,文宗為之動容,然方倚賽尚阿,亦未盡用其言。

二年,粵匪擾及兩湖,天爵僑居宿州,命偕安徽巡撫蔣文慶治防務。三年,疏陳廬、鳳為江淮要區,赴正陽關撫舊捻張鳳山等一千二百人用之,請江蘇、山東、安徽、河南舉行團練。未幾,安慶陷,文慶死之。命天爵署安徽巡撫,尋實授。江寧亦陷,天爵請扼黃河杜賊北竄,辭巡撫專任兵事。命以兵部侍郎銜督師剿宿州、懷遠、蒙城、靈壁捻匪。北路漸清,進規廬、鳳,擒定遠捻首陸遐齡,散其眾四千餘,被褒賚。疏論廬州知府胡元煒劣跡,請革職逮治,巡撫李嘉端置不問。元煒通賊內應,廬州陷,江忠源死之。粵匪踞臨淮關,天爵外遏來賊,內清土匪,孤軍支拄。方奉命往援廬州,以疾卒於軍。

上震悼,詔嘉其秉性忠直,勇敢有為,心地品行迥超流俗,追贈尚書銜,依贈官賜恤,特謚文忠,不由內閣擬上;擢其子光碧都司,賜光嶽舉人。

勞崇光,字辛陔,湖南善化人。道光十二年進士,選庶吉士,授編修。二十一年,出為山西平陽知府。調太原,擢冀寧道,遷廣西按察使。

二十八年,奉使赴越南冊封。事竣入關,值匪亂,駐思恩、南寧,督軍進剿。二十九年,遷湖北布政使,未行而湖南賊李沅發起新寧,仍留廣西治防。沅發平,敘功賜花翎。三十年,就授廣西布政使。慶遠賊竄武緣、賓州,崇光偕提督向榮會剿。擒賊首陳勝,又平上林、遷江竄匪,設方略解散匪黨凡數十起。撫張家祥收隸部下,改名國樑,後以戰功顯。尋署巡撫,副將伊克坦布戰歿於桂平,檄總兵周鳳岐赴援。時命李星沅督師,周天爵為巡撫專治軍。崇光仍攝巡撫事,會辦軍務。

咸豐元年,大學士賽尚阿代星沅,而鄒鳴鶴繼為巡撫,崇光會辦如故,平西林、博白、懷集竄賊。廣東賊顏品瑤擾南寧、太平,崇光駐兵南邕,與廣東軍合擊,屢戰皆捷,品瑤就殲,又平貴縣賊,被優敘。偕左江鎮總兵穀韞燦平白山賊,舉行南、太、泗、鎮四府團練,殲顏品瑤餘黨於靈山,加頭品頂戴。二年,駐梧州,會廣東軍剿艇匪。尋金田賊洪秀全等永安突圍出犯桂林,命崇光回援,至則賊已北竄,連陷興安、全州,偕總兵和春追擊之,賊遂入湖南。會雲貴總督吳文鎔疏稱崇光有膽略血性,請重其事權,就擢巡撫。上疏略曰:「桂林雖解圍,賊氛不遠,群情尚復驚疑,增兵置防,皆非倉卒能辦。惟就現有兵力布置,省標調赴各處者,次第撤回,駐防城內,遴選練丁分扼城外要隘。激勵團練以作民氣,招撫流亡以復民力,訓練兵勇以肅軍紀,搜緝土匪以靖內奸。各屬游匪、土匪不時蠢動,額兵不敷分撥,鼓舞團練,以資捍衛而備援剿。」

時賽尚阿既黜,崇光專任廣西軍務,詔以匪雖已出粵境舊巢穴,慮渠魁踞之為回竄地步,責以搜捕黨羽。三年,洪秀全等既踞江寧,分黨北犯中原。兵事日棘,朝廷不暇顧及邊遠,廣西伏莽時起,旋滅旋萌,餉絀兵單,惟恃團練,不能大創賊。崇光且剿且撫,支拄數載。洎英人踞廣州後,廣東賊氛復熾。艇匪竄擾廣西,潯州、柳州、慶遠、梧州、南寧相繼陷。近地土匪益起,屢逼桂林。軍中多降將,心皆叵測。崇光乞師於湖南,七年,駱秉章令蔣益澧率湘軍赴援,屢破賊,復興安、靈川,入屯省城,乃誅反側,易守軍,桂林始安。八年,奏留益澧在廣西剿賊,連擊艇匪於平樂令公渡、五塘,大破之,斬馘萬餘,由是艇匪始衰,慶遠、柳州相繼復。

九年,調廣東巡撫,兼署兩廣總督。英軍猶踞省城,前任總督黃宗漢、巡撫耆齡等,皆駐外縣不敢入。崇光至,坦然入城,與敵軍狎居。尋實授總督,迭遣將禦湖南、江西竄匪,擊走之。本境土寇時起,皆不久撲滅。與廣西軍會剿艇匪,梧州、潯州賊匪漸清。至十一年,英法聯軍犯京師,和議成,廣州敵軍始退。同治元年,以失察都司陶昌培、知縣許慶鎔營私納賄,降三級調用,命仍以一品頂戴赴貴州按事。前巡撫耆齡、御史華祝三復劾崇光任用非人,調度乖方,詔命自陳,下署總督晏端書、提督昆壽察按,得免議。

尋授雲貴總督。雲南自總督潘鐸被戕,巡撫徐之銘結回酋以自保,張凱嵩繼署總督,久不至,以規避黜,命崇光代之。崇光至貴州,會粵匪石達開餘黨陷綏陽,督兵擊走之,遂駐貴陽。三年春,土匪、苗匪屢來犯,偕巡撫張亮基勒兵固守,賊敗退。時雲南叛回猶雜處省城,議者皆言不可遽往。崇光逕行,軍民父老喜,迎於郊,回眾始稍斂。逆首馬榮、馬連升踞曲靖為巢穴。崇光知候補道岑毓英、降回總兵馬如龍可用,四年春,令參將馮世興與二人合師攻克曲靖,擒榮、連升等斬以徇,遂收馬龍、尋甸,迤東肅清,遣提督趙德光克平江外賊巢,復廣順,進克貴州,黔西大定。五年,復普洱及思茅,雲南軍事漸利。

六年,卒,優詔賜恤。嘉其「沉毅有為,歷官兩廣、雲貴,皆不避艱險,俾地方日有起色」,贈太子太保,謚文毅。廣西請建專祠,雲、貴祀名宦祠。

論曰:粵匪之起也,始由疆臣玩誤,繼復將帥不和。李星沅、周天爵皆素以忠勤著,文宗採時譽而付以重任,於軍事皆不得要領。及易以賽尚阿,而敗壞益甚,虎兕出柙,遂不可制矣。勞崇光久在兵間,洪秀全北竄後已不顧舊巢,然伏莽四起,終賴湘軍之力,數年而後克定;其於廣東、雲南皆受事於萬難措置之時,履虎不咥,權略有足稱焉。


\end{pinyinscope}