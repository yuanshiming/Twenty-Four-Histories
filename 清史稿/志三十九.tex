\article{志三十九}

\begin{pinyinscope}
地理十一

△甘肅

甘肅:禹貢雍州南兼梁州。之域。明陜西布政使司及陜西行都指揮使司地。清順治初,因明制,設甘肅巡撫,駐寧夏。寧夏巡撫旋裁。五年,徙甘肅巡撫駐蘭州。康熙三年,分陜西為左、右布政使司,以右布政使司駐鞏昌,領四府如故。六年,改陜西右布政使司為鞏昌布政使司。七年,又改甘肅布政使司,徙治蘭州。雍正三年,裁行都指揮使司及諸衛所,改置甘州、涼州、寧夏、西寧,升肅州及秦、階二州為直隸州。乾隆三年,廢臨洮府徙蘭州,因更名。二十四年,置安西府。二十九年,裁巡撫,以陜甘總督治蘭州,行巡撫事。三十八年,置鎮西府於巴里坤、迪化直隸州於烏魯木齊。三十九年,降安西府為直隸州。四十二年,升涇州為直隸州。同治十一年,置化平川直隸。十二年,升固原州為直隸州。光緒十二年,新疆改建行省,割迪化、鎮西往屬。東至陜西;及鄜州、邠州。南至四川;保寧、龍安。西南至青海;北至阿拉善、額濟納二旗。及喀爾喀札薩克圖汗部。廣二千一百二十里,袤一千四百十里。宣統三年,編戶九十萬六千六百三十九,口四百六十九萬一千六百二十。領府八,直隸州六,直隸一,州六,八,縣四十七。其名山:隴、嶓塚、崆峒、西傾、積石。其大川:黃河、西漢、渭、涇、洮、湟。其重險:蕭關、嘉峪、玉門。其驛道:一,東南逾六盤達陜西長武;一,西北渡河出嘉峪關達新疆哈密。電線:西北通迪化,東南通西安。

蘭州府:沖,繁,難。陜甘總督,布政、提學、提法三使,巡警、勸業道駐。明為州,屬臨洮府,領金、渭源、河州。乾隆三年,徙臨洮府來治,更名,以所隸河州、狄道、渭源三州縣改屬,升狄道為州,置皋蘭縣為府治,兼割鞏昌府屬之靖遠隸之。東北距京師四千四十里。廣千二百二十五里,袤八百里。北極高三十六度八分。京師偏西四十二度三十四分。領州二,縣四。皋蘭沖,繁,疲,難。倚。城南五里,皋蘭山,五泉出其下。百四十里,康狼山。北:九州、臺山、松山。西:沃干嶺、馬銜山。黃河,西南自河州入,西流,至孔家寺,折而東北,復東流,迤南,逕城北,至東坪,與金縣分界。又東北,逕烏金峽,入靖遠。洮水南自狄道入,西北流,至毛龍峽入黃河。邊墻,西自平番來,起縣西北毛牛圈,東南迤至小蘆塘,入靖遠鹽池。邊墻外,北與蒙古分界,有界碑六。鎮一:納米。驛三:蘭泉、沙井、摩雲。縣丞駐紅水堡。金沖,疲。府東南八十里。南:龕山。西南:馬銜山。東北:北巒山、車道嶺。東:駝項。西北:豬嘴。黃河自皋蘭入,南新營河自狄道入,至大營川,右合瓦家河,左合清水河,合龕峪、徐家峽、大峽諸水,西北至皋蘭入於河。黃河又東北過烏金峽入靖遠。邊墻,西自皋蘭來,逾黃河南至索橋,合舊邊墻,東北入中衛。驛二:定遠、清水。狄道州繁,疲,難。府南二百里。南:抹邦山、煤山。北:馬銜山,故關原。西:西平山。西南:十八盤山。洮水,南自洮州入,合抹邦、東峪、三岔、留川四水及諸小水,屈曲北注,逕沙泥州判境,沙泥水出摩雲嶺西麓,西流入之。洮水又北,入皋蘭。州北河渠,雍正三年濬,引洮流溉田三百頃。趙土司駐所,州東南六盤山麓。驛四:沙泥、洮陽、窯店、慶平。州判駐沙泥堡。渭源沖,疲。府南少東二百五十里。西北:七峰山。南:露骨。西南:五行。西:鳥鼠山,渭水出其北麓,東南流,逕城北,合清源、鍬峪兩水,東入隴西。驛一:慶平。靖遠疲,難。府東北二百里。明,靖虜衛。雍正二年省衛,設同知,屬鞏昌。八年置縣,裁同知。乾隆三年來屬。東:紅山、屈吳山。南:烏蘭。北:雪山。黃河,西南自金縣入,至城北,祖厲河南自會寧來,會左關川水,北流入之。黃河又納縣境諸水,北流,迤西逾邊墻,東北入中衛。邊墻,自皋蘭紅水堡來,接中衛。河州繁,疲,難。府西南二百里。陜西河州鎮總兵駐,雍正四年,省河水衛並入州屬。北:鳳林山。西南:石門山。西北:小積石山,即水經注唐述山。黃河,西自循化入,至積石關入州境,右納樣卑、吹麻、銀川三水,東逕城南,又東至蓮花寺。大夏河,西南自循化來,會州境諸小水,屈曲北流入之。黃河又東入皋蘭。韓土司及土番、老鴉、端言、紅巖、牙黨、川撒諸族,分居州西境。驛五:長安、鳳林、銀川、和政、定羌。太子寺,州判駐。

平涼府:中,沖,繁。平慶涇固化道治所。明為府,領州三,縣七。順治初,因明制。乾隆四十三年,升涇州為直隸州。同治十一年,割平涼、華亭、固原、隆德四州縣屬地置化平川直隸。十二年,升固原州為直隸州。西北距省治七百六十里。廣五百里,袤五百八十里。北極高三十五度三十五分。京師偏西九度四十八分。領州一,縣三。平涼沖,疲,難。倚。西北:天壇。東南:石馬。西南:可藍。西:空同山。其支麓為笄頭、馬屯山、彈箏峽。涇水北源西自固原州來,至沙溝門入境;南源西自化平川來,至味子溝入境。合流城西,右納大峽河,左納小蘆、大蘆、潘陰澗諸水,東南入涇州。汭水西自崇信入,逕縣東南王家寺,東入涇州。東:利民渠,明濬,縣南諸水匯焉。峽石、安國二鎮。驛一:高平。華亭疲。府南九十里。東:義山。西:隴山。北接大漠,南抵汧隴。西北:美高、朝那山。汭水北源出縣西鍬頭津,南源出縣西大關山,東流夾城匯為一,又東,右納策底河,左納五村川水,東迤北入崇信。盤口河出縣西南山中,東流,支津左出為五村川水,入崇信。惠民渠,明濬,遏汭流引入城。制勝、六盤關、三鄉、黃石河鎮。驛一:瓦亭靜寧州沖,疲,難。府西二百三十里。東:隴山、上峽、東山。南:石門。西:西巖。北:橫山。苦水河即長源河,北自隆德入,環城南注,納甜水河及州境諸小水,屈曲南入秦安。西:興隴渠,明濬。驛一:涇陽。隆德沖,難。府西北百四十里。乾隆四十三年,省莊浪縣,以其地來屬。東:六盤山。若水河,北自固原入;納馬蓮川、濫泥河諸水,南入靜寧。其東支甜水河,即隴水,亦出六盤,逕城北,西合底堡川、南源溝水,並從之。驛一:隆城。縣丞駐莊浪故城。

鞏昌府:沖,繁,疲,難。隸平慶涇固化道。明置府,領州三,縣十四。順治初,因明制。雍正七年,升秦、階二州為直隸州,降徽州為縣,及清水、秦安、禮、兩當隸秦州,以文縣、成縣隸階州。八年,增置岷州及靖遠縣。乾隆三年,割靖遠隸蘭州。十三年,改洮州衛為來屬,旋並漳縣入隴西,隸鞏秦階道。西北距省治四百二十里。廣二百九十五里,袤千二百三十里。北極高三十四度五十七分。京師偏西十一度四十三分。領一,州一,縣七。隴西沖,繁。倚。東:三品石、仁壽。北:赤亭。西南:首陽。西:西傾。西北:八角山。渭水,西自渭源入,合廣陽水為山河口,左合援陽河,右合科陽,逕城北,納縣境諸水,東南入寧遠。漳水一曰清水河,西南自岷州入,逕漳縣故城南,東入寧遠。鎮一:天衢。驛二:通遠、三岔。縣丞駐漳縣故城。安定沖,難。府北百六十里。南:南安山。東:照城、鳳凰。西:西巖山。東南:溫泉山。北:車道峴。關川水東源出縣南禪牧山麓,一曰東河,西北流,西源出縣西南胡麻嶺,一曰西河,東北流,逕縣城北,匯為一川,一曰北河,北入會寧。鎮一:龜兒嘴。驛四:延壽、通安、西鞏、秤鉤。會寧沖,難。府東北二百里。東南:桃花。北:烏蘭山,烏蘭關在其下。南:鐵木山。東北:屈吳山。祖河出東南王家山,西流,厲河出南米家峽,北流,匯於城南,曰祖厲河。左納西鞏驛水,右合倉下什子川,西北入靖遠。關川水西南自安定入,逕縣境西北入靖遠。鎮一:翟家。驛四:保安、乾溝、郭城、青家。通渭簡。府東北二百里。西北:筆架山。東北:玉狼。南:十八盤山。華川水出會寧華川嶺,入縣境。東南逕西河灣;左合南家河,右龍尾溝,又東錯秦安,關川河從之。再錯復入,為散渡河,合青石峽水、清溪,入伏羌。鎮二:雞川、安遠。寧遠沖。府東南九十里。南:銀觀峪。西:廣吳山。南:董墨。東北:石門。西南:武城。渭水,西北自隴西入,逕鴛鴦嘴,合漳水及廣吳河,迤東逕城北,納縣境諸水,東入伏羌。縣境濬渠二十七。鎮六:馬務、威遠、來遠、落門、納泥、榆盤。伏羌沖,難。府東南百九十里。南:天門山。西:三都谷。西南:硃圉山。渭水,西自寧遠入,納南來諸水,逕城北,華川水北自通渭來注之,東入秦安。藉水一曰烏油江,出縣南山中,東入秦州。廣濟、陸田、通濟三渠,皆明濬。西和疲。府東南三百里。東南:太祖山。北:寶泉。東北:雞峰。西北:祁山。西南:仇池山。西漢水,東自禮縣入,逕縣北,橫水河逕城東,合葉家河、白水,仍西入禮縣。復東流入境,逕縣南,江底河出縣西南香山,東南流注之,又東南入階州。東北:鹽井。鎮一:長道。岷州疲,難。府西南二百四十里。明衛。雍正八年改置。北:岷山。東南:岷瓘山。洮水自西洮州入,東流過城北,疊藏河西南自楊土司境來,合多邦、綠園二水,北流注之。洮水西北復入洮州。岷瓘江一曰良恭河,出岷瓘山東麓分水嶺;南流迤東入禮縣。白龍江上源曰阿塢河,出岷瓘山西麓分水嶺,東南流,合數小水,曰岷江,又東南逕臨鋪江西入階州。驛三:岷山、西津、酒店。土司二:麻童、百林口堡。番族一:沙莊。洮州繁,難。府西南三百六十里。西南:西傾山。山脈東迤,曰陰得爾圖塔拉山、綽那搜爾山、多克第山、阿穆尼恰珠溫恭山、多噶爾山。洮水出西傾東麓,一曰巴克西河,南流迤東,納庫庫烏蘇、波爾波河、多克第河,合拉爾河、底穆唐河諸水,逕城南,東入岷州,折而西北,復入境,逕東北入狄道。白水江,即禹貢桓水,一曰墊江;西自四川松潘入,逕城西南,東南入階州。邊墻,南起洮州衛故城南峪口,北入河洲。鎮一:廣思。土司三:著遜、卓泥楊氏、資卜馬氏。諸土司皆貧弱,地什九入卓泥楊氏,幅員千餘里,南與松潘接。南路隘口七,通四川番地。西路隘口六,通青海。北路隘口三,通循化番地。

慶陽府:中,疲,難。隸平慶涇固化道。順治初,因明制,並置慶陽衛。雍正五年省衛。西距省治千一百八十里。廣三百十里,袤四百二十里。北極高三十六度三分。京師偏西八度四十六分。領州一,縣四。安化疲,難。倚。東北:太白、青沙嶺。西北:鐵邊山。環河一曰馬蓮河,西北自環縣入,東南流,逕城南,鐵邊河納境內諸水南流注之。又東南,合教子川,入合水,東北荔原川亦入焉。又北,白豹川,入陜西保安。縣北大小鹽池。鎮五:槐安、五交、業樂、馬嶺、董志。驛一:驛馬關。縣丞駐董志原。合水簡。府東七十里。西:錦屏。東:橋山、子午山。環河西北自安化入,至板橋鎮合建水,一曰東河,西南逕城東,右東北川為合水,納馬蓮河,南入寧州。故城川出子午從之。鎮四:華池、鳳川、平戎、太白。驛三:華池、邵莊、宋莊。環簡。府西北百八十里。東:尖山。西:青山。西北:青岡峽,環河出其南麓,東南流,逕城西,左右納小水十餘,又東南入安化,清水、蕭家河並從之。西南,寡婦川亦入焉。鎮三:馬嶺、木缽、石昌。驛三:靈武、靈祐、曲幹。正寧簡,府南二百四十里。本真寧,乾隆初更名。羅水出縣東羅山;西南流,逕城南,納馬造溝水,西入寧州。鎮三:湫頭、平子、山河。寧州中,疲,難。府南百四十里。東:雕嶺。南:雲寂。東北:五掌山。涇水,西自涇州入,納茹水河,南流迤東,環河北自合水來會,納境內諸水,逕城西南流注之,又東納羅水,入陜西長武。鎮八:襄樂、政平、早社、焦村、大昌、新莊、南義井、鳳皇。驛二:彭原、焦村。

寧夏府:沖,繁,疲,難。寧夏道治所。將軍、副都統、總兵駐。明,寧夏五衛。初因明制。順治十五年,並前衛入左衛、中衛入右衛。雍正三年,省衛、所,置府及寧夏、寧朔、平羅、中衛四縣,以靈州直隸州來屬。五年,置新渠縣。七年,置寶豐縣。乾隆三年,省新渠、寶豐入平羅。同治十一年,置寧靈。西南距省治九百四十里。廣五百三十里,袤六百六十里。北極高三十八度三十二分。京師偏西十度二十分。領一,州一,縣四。寧夏沖,繁,疲,難。倚。治府東偏。本前、左二衛地。雍正四年置縣。黃河,西南自靈州入,東北至昌潤渠口入平羅。河入中國,寧夏獨食其利,支渠釃分,灌溉府境。惠農渠,雍正四年濬,漢延渠,雍正九年重修,皆南自寧朔入。唐渠,雍正九年重修,西自寧朔入。皆東北入平羅。東:高臺寺湖。北:月湖。東北:金波湖、三塔湖。驛三:寧夏、王鋐、橫城口。寧朔沖,難。倚。治府西偏。本中、右二衛地。雍正三年置縣。西北:賀蘭山,山脈綿褫,北抵大漠,南訖中衛,山外蒙古阿拉善、額濟納地。黃河,南自寧靈、中衛入,東北至葉升渡入寧夏。惠農渠於縣南上馬家灘承黃河支流,東北入寧夏。漢延渠於縣南下馬家灘承黃河支流,東北納數小渠,入寧夏。大清渠,康熙四十九年濬,於漢渠南承河流,北過雙塔湖合唐渠。唐渠於縣南青銅峽首受河流,東北納支渠十餘,入平羅。南:長湖。西:觀音湖。吉蘭泰鹽池在賀蘭山麓。邊墻,沿山自北而南,逾分守嶺入中衛。定遠城在打臺溝,雍正間,阿拉善遷博羅克科克於此,築城設守。阿拉善王旋還舊游牧,仍以定遠城賜之。平羅疲,難。府北少東百二十里。故平羅所。雍正三年置縣。乾隆三年省新渠、寶豐二縣,以其地來屬。黃河,西南自寧夏入,分為二派,東北流百餘里復合流,北入鄂爾多斯。唐渠、惠農渠西南自寧朔入,東北至石嘴子,復入於河。昌潤渠,雍正六年濬,即故六羊河故瀆,疏流建閘;起縣東南,北流逕寶豐故縣東,復入於河。邊墻,縣北,西起賀蘭山麓,東訖河干。縣丞駐寶豐故城。靈州要,繁,疲,難。府東南九十里。初因明制為直隸州。雍正三年來屬,並省後衛,以其地入州境。黃河,西南自寧靈來,東岸旁州西境。山水河出州南山中,西北流,入平遠,復北入州境。苜蓿渠首受黃河,自西來會,支渠右出曰秦渠。山水河又北流,迤西北入黃河。支流北出曰澇河;北至三道橋又分二瀆,一西北入黃河,一北流會秦渠入河。黃河又東北至橫城口入寧夏。東南有蒲草湖、東湖。南、北、中三鹽池,花馬池,紅柳池,俱州東南。邊墻,起橫城堡,東入陜西延安。鎮一:耀德。驛三:靈州、紅山、沙泉。州同駐花馬池。鹽捕通判駐惠安堡。中衛沖,繁,疲。府西南三百六十里。故中衛地。雍正三年置縣。黃河,西自靖遠入,逕城西南,支渠左釃為美利渠、太平渠,右釃為羚羊角渠,過城東南,右釃為羚羊店渠,又東,左釃為永興渠、勝水渠,右釃為羚羊峽渠。清水河,東南自平遠來,北流注之。黃河又東,迤北,右釃為七星渠,左釃為順水渠、豐樂渠。諸渠皆東北復入於河。黃河又東北入寧靈。邊墻,旁黃河南岸,逾河東入寧靈。驛三:中衛、渠口、長流水。巡司駐渠寧。縣北阿拉善旗界有漢、蒙分界碑。寧靈要。府南二百里。故金積堡,屬靈州。同治十一年,總督左宗棠督師克復,奏設,改寧夏水利同知為撫民同知駐焉。南:金積山。東南:大蠡。東北:紫金。西南:青銅峽。黃河,南自中衛入,行峽中;東北入寧朔、靈州。清水河,西南自海城入,左合邊墻溝、紅溝,入中衛注河。漢渠自城西首受黃河,下流匯山水河。

西寧府:最要,沖,繁,疲,難。西寧道治所。辦事大臣、總兵駐。明,西寧衛。初因明制。雍正二年,省衛,置府及西寧、碾伯二縣。乾隆九年,置巴燕戎格。二十六年,置大通縣。五十七年,置貴德、丹噶爾二,割蘭州之循化來屬。東南距省治六百二十里。廣三百五十里,袤六百五里。北極高三十六度三十九分。京師偏西十四度十三分。領四,縣三。西寧沖,繁,疲,難。倚。東:峽口山,嘆隍[A112]地;紅崖子山。西:土樓山、金山。南:拔延山。西南:南禪山、積石山、拉脊山。西北:北禪山。黃河,西自貴德逕城南,東入巴燕戎格。湟水西自丹噶爾入,逕城北,北川河西北自大通來注之,又東南入碾伯。大通河逕縣東北入平番。縣西:伯顏川渠。縣南:那孩川渠。驛二:西寧、平戎。土司四:陳氏、吉氏、祁氏、李氏。番莊二:上朵壤爾、乜亥加。番族三:上郭必山、松巴、巴哇。碾伯沖,繁。府東百三十里。故守御千戶所,屬西寧衛。雍正二年置縣。南:雪山。西:四望山。東北:阿剌古山。湟水,西自西寧入,東南流,逕城南,曰碾伯河。納縣境諸川,東南至蓮花臺;大通河北自平番來會。河北、河南兩渠,引湟溉田,釃支渠三十。驛三:嘉順、老鴉、巴州。土司三:九家港、勝番溝、老鴉堡。他番族十餘,分居縣境。大通難。府西北百三十里。故番地。雍正二年,以番族效順,置大通衛。乾隆二十六年省衛置縣。西北:大雪山。北:大寒。東:五峰。南:元朔山。大通河古浩亹水,西自青海入,東南入平番。北川河西自青海入,有二源,北曰布庫克河,南曰沙庫克河,合流至城北為北川河,又東南入西寧。東峽川、峽門堡二渠。長寧驛。土司六:起塔鎮、迭溝、大通川、王家堡、硃家堡、美都溝。西北與青海分界,有界碑。貴德要。府南。故歸德千戶所,屬河州衛。雍正四年,省衛所隸河州。乾隆三年改隸西寧。二十六年設縣丞。五十七年升,設撫番同知。東:郭圖。南:莫曲山、圖爾根山。東南:圓柱。南:南山。黃河,南自青海改西北流,折而東北,恰克圖河東來注之。又東北,環西境,至隴羊峽西折而東南,合龍池河及烏蘭石爾廓爾河,並諸小水,入循化、巴燕戎格。番族分生、熟、野番三種。熟番五十四族,畊賦視齊民。生番十九族,畜牧資生。野番八族,其汪食代克一族,乾隆末北徙丹噶爾,餘七族咸居東境,插帳黃河南岸。循化要。府東南。舊屬蘭州,為河州同知駐所。乾隆末,移隸西寧。西南:多噶爾群山,不一名。黃河,西自貴德入,北岸為巴燕戎格地。保安大河南自丹噶爾北流注之,又東納境諸水,至積石關入河州。大夏河,古漓水,出南邊外山中,北流,逕拉布楞寺,屈曲東南入河州。青海和碩特游牧地錯入南境。番族:上隆布西番十六寨,南番二十一寨,阿巴那西番八寨,多奈錯勿日二寨,素呼思記二寨,邊都溝西番十寨,東鄉西番五寨。回民撒拉族所居,曰上八工、下八工。丹噶爾府西南。撫番同知駐。東:翠山。南:日月。北:北極山。湟水出青海噶爾藏嶺,東流,至札藏寺入境,逕城南,東入西寧。清水河出貴德南速古山,東北流。隆武河出循化西南番地,北流,匯為保安大河,北入循化。韓土司轄地在東南。東科爾寺在西南。西寧、青海孔道。沙喇庫圖爾番族聚居處。巴燕戎格府東南,通判駐。明,西寧、碾伯、洮州地。乾隆三年,以鞏昌裁缺通判徙改。北:雪山。西:小積石。東南:拉札山。黃河,西自貴德入,南岸為循化境,巴燕戎格河出小積石山東麓,納境諸小水,南入黃河。

涼州府:沖,繁,疲,難。甘涼道治所。副都統、總兵駐。明,涼州衛。順治初,因明制。雍正二年升府,置、縣。東南距省治五百六十里。廣九百三十里,袤五百二十里。北極高三十七度五十九分。京師偏西十三度四十八分。領一,縣五。武威沖,繁,疲,難。倚。故涼州衛地。雍正二年置縣。南:祁連山,一名大雪山,綿亙千里,西北抵甘州境。沙溝水出山麓,屈曲北注,會黃羊渠為白塔河,又西北迤,逕城北,會雜木河、大七河、金塔寺渠、海藏大河、炭山河、北沙河諸川,為郭河,北入鎮番。東北:邊墻,起鎮番境蔡旗堡,南至土門關入古浪。驛三:武威、懷安、大河。鎮番繁,疲,府東北二百里。故鎮番衛。雍正二年置縣。南:亦不剌山,環東北三面。郭河,南自武威入,西北出邊墻,釃支渠四,又西北出塞,瀦為大澤,蒙古謂之哈剌海謨,古休屠澤也。青鹽池、鴛鴦白鹽池、小白鹽池皆在西北邊墻外。邊墻,西接永昌,東至縣城北,折而南,逾郭河入武威。永昌沖,繁,疲。府西北百六十里。故永昌衛。雍正二年置縣。北:金山。西:燕支。東北:馬氾。東南:炭山。水磨川出縣西南祁連山北麓,四源並導,匯為一川,北流折東,又東北出邊墻,瀦為昌寧湖;今涸。炭山河出縣南,北流至永豐堡南,折而東南入武威。邊墻,西起水泉堡,東訖鎮番境紅崖堡。驛二:永昌、水泉。古浪沖,疲。府東南百三十里。故古浪所。雍正二年置縣。西:白嶺。東南:黑松林。古浪河出縣南烏鞘嶺北麓,納縣境諸水;東北出邊墻,瀦為澤,曰白海。邊墻,自武威南境逾古浪河,迤東南入平番。驛二:古浪、黑松。巡司駐大靖。平番沖,繁,疲,難。府東南三百三十里。故莊浪所。雍正二年置縣。東:松山。北:炭山。西:卓子山。西北:分水嶺。北為萱麻河,入古浪。莊浪河出嶺南麓,納金羌、石門、清水諸小河,至城南,又南至頭道河入皋蘭。大通河,西北自大通入,逕城西入碾伯注湟水。大鹽溝,東南。邊墻,起縣西北,東南入皋蘭。驛五:莊浪、大通、通遠、鎮羌、平城。土司二:古城、連城。縣丞駐西大通。莊浪簡。府東南。同知、理事通判同駐。莊浪河,北自平番入,南至皋蘭境入於河。大通河,西北自平番入,東南至皋蘭、河州境入於河。土司一:大營灣。

甘州府:沖,繁,疲。隸甘涼道。提督駐。明,陜西行都司治。順治初,因明制。雍正二年,罷行都司,置府及張掖、山丹、高臺三縣。七年,割高臺隸肅州。乾隆間,增置撫彞。東南距省治千五百里。廣三百二十里,袤二百里。北極高三十九度。京師偏西十五度三十一分。領一,縣二。張掖要,沖,繁,疲。倚。故甘州左、右衛。雍正二年置縣。北:合黎山。西南:祁連山,綿亙府境,與青海分界。山丹河,東自山丹入,洪水河出縣東南金山北麓,北流注之。又西北逕城北,張掖河古羌谷水,出祁連山中,匯縣境諸渠,北流來會。山丹河自此蒙黑河之稱。又西北,入撫彞。張掖河東岸黑番牧地,西岸黃番牧地。邊墻,傍山丹河北岸,東入山丹。驛二:甘泉、仁壽。縣丞駐東樂。山丹沖,繁,疲。府東百二十里。故山丹衛。雍正二年置縣。山丹河即禹貢弱水,出縣南祁連山麓,四源並導,匯於城南,東入張掖。紅鹽池在縣北,白鹽池濱居延澤。大草灘,東南與涼州、西寧、青海分界。邊墻,起合黎山南,逕縣城北,東入永昌。驛四:山丹、東樂、新河、峽口。撫彞府西北百五十里。舊隸甘州後衛。雍正二年衛省,屬高臺。乾隆十八年來屬,置設通判。南:祁連。響山河出東南,黑河自張掖入合之,西北逕北,左合三清渠,右出支渠,北自魯墩灣入高臺。邊墻,傍黑河北岸東入張掖。驛一:同名。

涇州直隸州:要,沖,疲。難。隸平慶涇固化道。明隸平涼府,領靈臺。順治初,因明制。乾隆四十二年,升直隸州。割崇信、鎮原來屬。西距省治九百五十九里。廣百一十里,袤三百五里。北極高三十五度二十三分。京師偏西九度七分。領縣三。北:兼山。西:回山。西南:弇耳山、青溪嶺。涇水西自平涼入,逕城北,汭水西南自崇信來注之。又東至唐長武故城,洪河西北自鎮原來注之。又東至寧州界,茹水西北自鎮原來注之,南入陜西長武。盤口河,西自靈臺入,旁州南境,東入長武。鎮一:盤口。驛一:安定。崇信難。州西南百二十里。城據錦屏山北麓。西南:箭筈山。西北:峽口。汭水,西自華亭入,匯五龍、斷萬、五馬三山及九峪水,屈東逕城北,東入平涼。盤口河即黑河,亦自華亭入,傍縣南境,東北入靈臺。新柳灘旁汭水,順治中疏為渠。鎮原疲。州西北二百里。東:東山。北:潛夫、孝山。茹水,西北自固原入,逕城南,納交口河、蒲河暨縣北境諸水,東南入寧州。洪河,西北自固原入,合平泉水,西南潘陽澗,入州。鎮二:新城、柳泉。驛一:白水。靈臺疲,難。州南二百里。北:臺山。東:蒼山。東北:書臺。西南:離山。達溪水,西自陜西隴州入,左合鎮川口河,至百里鎮,右合妲己,左小建河,逕城南,東北入陜西長武。盤口河,西自崇信入,逕縣東北,合槐樹溝水,東入州。鎮七:東朝那、良原、百里、邵寨、石塘、上良、西屯。

固原直隸州:沖,繁,難。隸平慶涇固化道。陜西提督駐。明隸平涼府。順治初因之。同治十二年,升直隸州,置平遠、海城二縣屬焉。西距省治八百九十里。廣五百二十里,袤三百十里。北極高三十六度四分。京師偏西十度七分。領縣二。西北:石城山。北:須彌。西南:隴山,一曰六盤山,綿跨平涼化平川境。清水河出隴山開城嶺北麓,古高平川,二源並導,匯為一川,逕城東,納州境諸水,北入平遠。涇水北源出開城嶺南麓,為大小南川,會於瓦亭驛;東逕蒿店,曰橫河,出彈箏峽,入平涼。茹水出開城嶺東麓,洪河出州東南陶家海子,並東入鎮原。驛三:永寧、三營、瓦亭。州判駐硝河城。平遠沖,難。州北二百四十里。故平遠所。同治十二年置縣,又割海城之下馬關西地及靈州同心城來屬。西北:羅山。南:打狼。西北:麥朵。西南:白楊林。清水河,南自州境入;甜水河自東來注之,又納縣境諸水,西北入中衛。山水河,東自靈州入,逕縣北境,復西入靈州。海城沖,疲,難。州西北二百十里。平涼府屬海剌都地。乾隆十四年徙鹽茶同知駐此。同治十二年省同知置縣。西:天都山。西南:蓮花。南:五橋山。北:大黑河、紅井堡水、相洞川,並東入州,注清水河。清水河逕紅古堡,合石峽水,又北合興仁堡水,入寧靈。西北:乾鹽池堡水,逕打拉池,縣丞駐。

階州直隸州:疲。隸鞏秦階道。明隸鞏昌府,領文縣。順治初,因明制。雍正七年,升直隸州,割鞏昌之成縣來屬。西北距省治千一百五十里。廣二百九十里,袤五百五十里。北極高三十三度二十三分。京師偏西十一度二十三分。領縣二。北:鳳凰山。白水江,西北自洮州入,南流,迤東逕西固城南,白龍江北自岷州來注之。又東南,逕城西,納數小水,南入文縣。西漢水,西北自禮縣入,屈曲東南入成縣。鎮四:平洛、安化、角弓、石門。驛三:階州、官城、殺賊橋。州同駐西固城。州判駐白馬關。文簡。州西南二百里。白水江,北自州境來,逕縣東南,清江水一曰文縣河,西北自四川松潘,上承察岡公河,東南流入境,納縣西諸水來會。白水江又東南納縣東諸水,入四川昭化。南:陰平隘。驛二:文縣、臨江。成疲。州東北二百里。西:泥功山、仇池山。東:木皮嶺。西漢水,西北自州境入,逕縣西南,黑峪河出縣北山中,納縣境諸水,西南流注之。西漢水至此蒙犀牛江之稱,東南入陜西略陽。鎮三:泥陽、橫川、拋沙。驛一:小川。

秦州直隸州:要,沖,繁,難。鞏秦階道治所。明隸鞏昌府,領秦安、清水、禮三縣。順治初,因明制,雍正七年,升直隸州,降鞏昌屬之徽州為縣,與所領兩當縣來屬。西北距省治七百三十里。廣三百九十里,袤四百五十里。北極高三十四度三十五分。京師偏西十度四十分。領縣五。西:刑馬山。西北:邽山。東南:麥積。西南:嶓塚。渭水,西自伏羌入,右納藉水,左納牛頭河,東逕城南,又東納諸小水,過三岔城北,迤南入陜西隴州。西漢水出嶓塚山南麓,西入禮縣。駱駝川水出嶓塚山東麓,流合數小水,南入徽縣。鎮四:關子、高橋、社樹坪、董城。州判駐三岔鎮。秦安疲,難。州北八十里。東:九龍山。北:顯親峽。南:新陽崖。東北:青龍。羅玉河古隴水,北自靜寧州入,上承苦水河,南逕縣西,至新陽崖入州境注渭。略陽川水東自清水入,西合石版泉,入靜寧注苦水河。鎮六:金城、川口、郭嘉、太平、隴城、大寨。清水沖,疲。州東北百二十里。東:隴山,大震關在其下。牛頭河一曰東亭河;古橋水,出隴山西麓,眾源並導,匯為一川,逕城北,東流,迤南入州境。略陽川水亦出隴山西麓,西流,納縣境諸水,逕龍山鎮入秦安。鎮八:白沙、巖年、清水、百家、玉屏、松樹、龍山、恭門。驛一:長寧。禮疲。州西南二百里。東:祁山。東南:仇池山。西南:岷瓘山。西漢水,東自州境入,納縣境諸水,逕城東折南,又西入階州。鎮二:石頭、崖城。徽難。州南二百八十里。北:鸞亭。東:赤玉。南:鐵山、青泥嶺。西:慄亭山、木皮嶺。東南:殺金坪,仙人關在其上。故道河,東自兩當入,駱駝川入北自州境來入之,西逕縣南,納小水二,西南入陜西略陽,嘉陵江上游也。慄水出慄亭山,南流為泥陽河,南入略陽。鎮三:永寧、粟亭、火鉆。兩當簡。州南百七十里。東:鸑鷟。南:天門。東北:申家,古南大夫山。故道河,東自陜西鳳縣入,河即兩當水,逕縣南,納小水二,西南逕秦岡山為琵琶湖,入徽。鎮二:廣鄉、兩當。有驛。

肅州直隸州:沖,繁,疲。安肅道治所。總兵駐。明,肅州衛。順治初,因明制。雍正二年,省衛並入甘州府。七年,置直隸州,割甘州之高臺縣來屬。東南距省治千四百六十里。廣百九十里,袤百五十里。北極高三十九度十六分。京師偏西十七度十二分。領縣一。東南:觀音山。南:祁連山。東跨高臺,與青海分界。西:嘉峪山。其西麓設關,俄羅斯通商孔道,稅務司駐焉。洮賴河出州西南祁連山北麓,古呼蠶水,北流東迤,支渠旁出,左播為四,右播為三。又東為北大河,至臨水堡,臨水河出祁連山最高處,東北流注之,折而北,逕金塔寺,西出邊墻為北大河,至古城,右會紅水,左合清水河,曰白河,東北入高臺。豐樂川出州東南祁連山天澇池,北流釃十數渠。南:金廠。邊墻,自嘉峪關迤西北逾洮賴河,折而東南,入高臺。驛二:酒泉、臨水。州同駐金塔寺。巡司駐嘉峪關。高臺沖,繁,疲。州東南二百七十里。故守御千戶所。雍正三年置縣。西:崆峒。南:榆木。東北:合黎山。黑河,東自撫彞入,西北流,逕城北,左出支渠五。又西北逕深溝驛,復釃為數小渠,又北至鎮夷營。出邊墻,右釃為雙樹子屯渠,左釃為毛目渠,白河西南自州來會,北入額濟納旗界,匯於居延海。縣西北鹽池。邊墻,西自州境來,逾黑河,東南入撫彞。驛四:雙井、深溝、黑泉、鹽池。縣丞駐毛目屯。

安西直隸州:沖,繁,疲,難。隸安肅道。明,赤斤、沙州二衛。後以番擾內徙,空其地。康熙五十七年,番族內附,置靖逆、赤斤二衛,設靖逆同知領之,尋增設通判,治柳溝。雍正元年,復置沙州所,築布隆吉城,設安西同知治焉。三年,省靖逆同知,徙通判治其地,仍領二衛,旋升沙州所為衛。六年,徙安西治大灣。乾隆二十四年升府,置淵泉縣附郭,省靖逆通判,並赤斤衛置玉門縣。二十五年,以沙州衛為敦煌縣,省淵泉入府治。二十八年,降直隸州,隸安肅道。東距省治二千一百二十里。廣六百二十里,袤六百里。北極高三十九度四十分。京師偏西十八度五十二分。領縣二。雪山自蔥嶺支分,迤邐東趨,綿跨州境,山外皆大戈壁,與青海分界。其北連山無極,與哈密及札薩克圖汗分界。疏勒河,古南極端水,一曰布隆吉河,其西源昌馬河出,東入玉門,與東源合,復入,右合支渠。鞏昌河西北逕橋灣營南,左納小水七,迤北西流,逕城南,支渠左出為南工渠、北工渠,經流西入敦煌。邊墻,西起布隆吉城東疏勒河北岸,東訖橋灣營入玉門。驛七:柳溝、小宛、瓜州口、白打子、紅柳圈、大泉、馬連井。敦煌繁,難。州西南二百七十里。東南:三危山、鳴沙山。西南:龍勒山。西:白龍堆流沙磧。疏勒河,東自州境入,西至城北雙河岔,黨河自南來注之。黨河,古氐置水,蒙古謂之西拉噶金,出縣南山中,兩源並導,匯為一川,北流逕城西,釃分十數渠,又北入疏勒河。疏勒河又西瀦為哈剌泊。東南:鹽池。玉門關、陽關,皆縣西南。玉門沖,繁。州東二百九十五里。金山環東、西、北三面,綿亙二百餘里。西北:赤金峽。疏勒河出縣南山中,北流,納昌馬河、鞏昌河,又北逕城西,迤東入州境。阿拉克湖即延興海。又東白楊河。有石油泉,古石脂水。邊墻,西自州境來,東入肅州。驛二:赤金湖、赤金峽。

化平川直隸:繁,疲,難。隸平慶涇固化道。平涼、華亭、固原、隆德四州縣地。同治十一年,隴東戡定,置設通判。西北距省治七百四十九里。廣袤各百餘里。北極高三十五度有奇。京師偏西南十度有奇。東:觀山。西南:大關山。涇水南源出山麓老龍潭,東逕白崖山,合白巖河,又東逕飛龍撻銀,左納聖女川、龍江峽水,東入平涼。


\end{pinyinscope}