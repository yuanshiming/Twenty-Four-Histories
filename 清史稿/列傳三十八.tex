\article{列傳三十八}

\begin{pinyinscope}
○圖海李之芳

圖海,字麟洲,馬佳氏,滿洲正黃旗人。父穆哈達,世居綏芬。圖海自筆帖式歷國史院侍讀。世祖嘗幸南苑,負寶從,顧其舉止,以為非常人。擢內秘書院學士,授拜他喇布勒哈番,遷弘文院大學士、議政大臣。順治十二年,加太子太保,攝刑部尚書事。與大學士巴哈納等同訂律例。侍衛阿拉那與公額爾克戴青兩家奴鬥於市,讞失實,坐欺罔,免死,削職。世祖崩,遺命起用。聖祖即位,授正黃旗滿洲都統。

李自成餘眾郝搖旗、劉體純、李來亨嘯聚鄖、襄間。康熙二年,命圖海為定西將軍,副靖西將軍都統穆里瑪,將禁旅,會湖廣、四川諸軍討之,屢破賊。未幾,郝搖旗為副都統杜敏所擒,劉體純亦破滅,惟李來亨據茅麓山,恃險負固,圖海圍之,絕其外援。來亨窮蹙,自焚死,其下以眾降。執斬明新樂王及所署置官屬,俘三千餘以還。六年,復為弘文院大學士,進一等阿達哈哈番。頃之,以兼都統乞解機務,不許。九年,改中和殿大學士,兼禮部尚書。

十二年,平南王尚可喜請老。七月,吳三桂繼之,實探朝旨。廷議移籓狀,莫洛、米思翰、明珠等皆主如所請,惟圖海持不可。上意決,遂黜圖海議。三桂既反,命攝戶部,理餉運。

十四年,察哈爾布爾尼劫其父阿布柰以叛。命信郡王鄂扎為撫遠大將軍,圖海副之,討布爾尼。時禁旅多調發,圖海請籍八旗家奴驍健者率以行,在路騷掠,一不問。至,下令曰:「察哈爾元裔,多珍寶,破之富且倍!」於是士卒奮勇,無不一當百。戰於達祿,布爾尼設伏山谷,別以三千人來拒。既戰,伏發,土默特兵挫。圖海分兵迎擊,敵以四百騎繼進,力戰,覆其眾。布爾尼乃悉眾出,用火攻,圖海令嚴陣待,連擊大破之,招撫人戶一千三百餘。布爾尼以三十騎遁,科爾沁額駙沙津追斬之,察哈爾平。師還,聖祖御南苑大紅門,行郊勞禮。敘功,進一等阿思哈尼哈番。

陜西提督王輔臣以平涼叛應三桂,定西大將軍貝勒董額督諸軍攻之,久未下。三桂遣王屏籓、吳之茂等犯秦、隴,欲與平涼合。十五年,以圖海為撫遠大將軍,八旗每佐領出護軍二名,率以往。臨發,上御太和殿賜敕印,命諸軍咸聽節制。既至,明賞罰,申約束。諸將請乘勢攻城,圖海宣言曰:「仁義之師,先招撫,後攻伐。今奉天威討叛豎,無慮不克。顧城中生靈數十萬,覆巢之下,殺戮必多。當體聖主好生之德,俟其向化。」城中聞者,莫不感泣,思自拔。五月,奪虎山墩,虎山墩者,在平涼城北,高數十仞,賊守以精兵,通餉道。圖海曰:「此平涼咽喉也。」率兵仰攻,賊萬餘列火器以拒師。圖海令兵更迭進,自巳至午,戰益力,遂奪而據之,發大砲攻城,城人洶懼。圖海用幕客周昌策,招輔臣降。

昌,字培公,荊門諸生。好奇計。佐振武將軍吳丹有勞,以七品官錄用。圖海次潼關,以策干之,客諸幕。輔臣所署置總兵黃九疇、布政使龔榮遇皆昌鄉人,屢勸輔臣反正,以蠟丸告昌,昌白圖海。圖海即令昌入城諭降,輔臣遣其將從昌出謁,圖海聞上,上許之。乃假昌參議道,賚詔往撫。輔臣使榮遇上軍民冊,子繼貞繳三桂所授敕印,顧猶觀望,復命昌偕兄子保定諭之,乃薙發降。因令吳丹入城撫定。

吳之茂聞平涼下,自秦州遁,遣將軍佛尼勒敗之於牡丹園,又敗之於西和縣北山。將軍穆占進攻王屏籓於樂門,敗賊於紅崖,復禮縣。輔臣所署置巡撫陳彭,總兵周養民、王好問等相繼降。秦地略定。敘功,進三等公,世襲。

圖海疏請遣兵赴湖廣,會征三桂,上命圖海親率精銳以行。圖海疏陳陜西初定、反側未安狀,乃授穆占征南將軍,率滿洲兵及平涼降卒往,圖海仍留鎮。時平涼、慶陽雖下,漢中、興安猶為賊據。圖海奏調綠旗兵,期明年正月檄提督孫思克赴秦州,趙良棟赴鳳翔,與張勇、王進寶會師進取,勇等謂須俟夏秋。上慮克漢中、興安轉餉難,令守諸要隘,分兵赴荊州攻三桂。十六年,圖海招撫韓城等縣偽官,又遣兵逼禮縣、益門,先後敗賊五盤山、喬家山、塘坊廟、芭蕉園、沙窩諸處,復塔什堡。十七年,復疏請分兵下漢中、興安,上密諭止之。將軍佛尼勒等又敗賊牛頭山香泉,四川總督周有德亦敗賊秦嶺,復潼關堡五寨。慶陽賊袁本秀受三桂劄,謀亂。圖海發慶陽、宜君、延安三營兵,會王進寶討平之,斬本秀衛遠溝。頃之,入覲。十八年,還鎮。

湖南、廣西平。上命亟攻寶雞,規取漢中、興安,定四川。圖海乃厲師攻益門鎮,破之。會賊毀偏橋,兵不得進,狀聞,詔嚴責。乃決策期分四路:圖海親率將軍佛尼勒等趨興安,總兵官程福亮為後援,屯舊縣關;將軍畢力克圖、提督孫思克等自略陽進,總兵官硃衣客為後援,駐西河;將軍王進寶、總兵官費雅達自棧道進,總兵官高孟為後援,駐寶雞;提督趙良棟自徽縣進。十月,師次鎮安,分兵為二隊,進敗三桂將王遇隆,渡乾玉河,奪梁河關。三桂將韓晉卿遁。進寶亦復漢中。良棟復徽縣、略陽。畢力克圖復成縣,又復階州,遣參將康調元復文縣。於是平利、紫陽、石泉、漢陰、洵陽、白河、竹山、竹溪、上津諸縣皆下。興安既克,圖海統大軍之半屯鳳翔,尋移漢中,護諸軍餉。會降將譚洪復叛,陜西總督哈占溯江討之,詔圖海遙為聲援。

二十年,以疾徵還。卒,謚文襄。太宗實錄成,贈少保兼太子太傅。雍正初,追贈一等忠達公,配享太廟。子諾敏,襲爵,歷刑、禮二部尚書,正黃旗蒙古都統。諾敏子馬爾賽,自有傳。

周昌初入城,自陳父明季死流寇,母孫剜目破面觸棺死,原捐軀表母烈。及輔臣降,圖海以聞。上命旌其母,遣官致祭,授昌布政使參政。昌復參蔡毓榮軍事,事平,授山東登萊道,攝布政使,以與總兵互訐罷。昌既罷,猶喜言兵。噶爾丹擾邊,數上書當事陳利害。後卒於家。

李之芳,字鄴園,山東武定人。順治四年進士,授金華府推官。卓異,擢刑部主事。累遷郎中,授廣西道御史。疏請革錢糧陋規,禁州縣官迎送。十七年,巡按山西。聖祖即位,裁巡按,召回。康熙元年,乞假歸。二年,復授湖廣道御史。五年,巡視浙江鹽政。入掌河南道事。

大學士班布爾善坐鰲拜黨誅,之芳疏言:「昔大學士俱內直,諸司章奏,即日票擬。自鰲拜輔政,大學士皆不入直,疏奏俱至次日看詳。請復舊制,杜任意更改之弊。」又疏言:「世祖時賞罰出至公,督撫不敢恣睢無忌。十八年以後,督撫率多夤緣而得,有恃無恐。勒索屬員,擾害百姓。夫直省億萬之眾,皆世祖留遺之群黎,我皇上愛養之赤子,何堪此輩朘削?自與受同罪之法嚴,與者不承,則言者即涉虛,非特不敢糾督撫,且不敢糾司道守令。有貪之利,無貪之害,又何憚而不怙惡自恣?今皇上親政,乞親裁,罷黜溺職督撫,以肅吏治。」疏下部,尋甄別各省督撫,黜其尤者數人。進秩視四品,擢左副都御史。之芳數上封事,請嚴巡鹽考績,慎外官罰俸,皆關治體。遷吏部侍郎。

十二年,以兵部侍郎總督浙江軍務。會吳三桂反,十三年,奏請復標兵原額,督習槍砲。疏甫上,耿精忠亦叛,遣其將曾養性、白顯忠、馬九玉數道闚浙,浙大震。之芳檄諸將扼仙霞關,調總兵李榮率副將王廷梅、牟大寅、陳世凱、鮑虎等分道御寇。時上命都統賴塔率師入浙,五月,偕賴塔率滿洲兵千、綠旗兵二千、鄉勇五百,進駐衢州。眾皆謂會城重地,不宜輕委。之芳曰:「不然。衢踞上游,無衢,是無浙也。今日之事,義無反顧。」顯忠自常山陷開化、壽昌、淳安,養性自處州犯義烏、浦江、東陽、湯溪,沿河阻餉道。溫州鎮總兵祖弘勛叛,召寇陷平陽,再進陷黃巖,集悍卒數萬窺衢州。

七月,之芳與賴塔閱兵水亭門,率總兵官李榮、副都統瑚圖等薄賊壘,軍坑西。之芳手執刀督陣,或請少避,之芳曰:「三軍司命在吾,退即為賊乘。今日勝敗,即吾死生矣!」守備程龍怯戰,斬以徇。麾眾越壕拔柵,敗之。遣陳世凱乘勝復義烏、湯溪,鮑虎復壽昌、淳安,牟大寅破常山,王廷梅敗賊於金華石梁、大溝源,李榮亦復東陽,復敗賊於金華壽溪,馘賊將,毀寨十八。參將洪起元復嵊縣。詔嘉之芳調度有方。

十月,賊將桑明等五萬眾由常山逼衢州西溝溪,倚山為營,覬聯南路賊巢。之芳與賴塔議,出不意,遣廷梅與參領禪布夜趨溝溪,分隊進攻,又大破之,賊棄營遁。

十四年,康親王傑書破曾養性金華,復處州;貝子傅拉塔亦復黃巖,進圍溫州。惟九玉踞江山、常山、開化,連寨數十,與之芳相持。五月,乘大雨河溢,由南塘搗賊前嶺,陣斬七百餘級。十五年,遣將自遂安連破賊寨,遂復開化。

會鄭錦入漳、泉,耿繼祚方攻建昌潰營遁。上知閩中有變,命王撤溫州之圍取福建,之芳乃建議直搗仙霞關,曰:「進取之路,不在溫、處而在衢。雖九玉死守河西難猝破,然其南江山,西則常山,皆間道可襲。我兵一進,使彼首尾受敵,即河西之壘不能獨完。」王至衢州,從之芳議。遂進兵大溪灘,復江山,九玉走,欲別取道奪仙霞。諸將受之芳密檄,急據關夾擊,其將金應虎等窮蹙降。

王師下福建,臨行,之芳啟曰:「王但飭諸軍勿虜掠,即長驅入,兵可不血刃也。」未幾精忠降,溫、處賊皆潰散。精忠所署置總兵馬鵬、汪文生、陳山,將軍程鳳等猶踞玉山、鉛山、弋陽、德興,之芳請會剿。時吳三桂兵寇吉安、袁州,江西兵不能東,乃獨遣兵復玉山,文生遁;自白沙關趨德興,擒鵬;遣游擊郭守金等復鉛山、興安、弋陽、貴溪諸縣。上嘉之芳剿賊鄰省有功,加兵部尚書銜。

十六年,遣參將蔣懋勛等敗賊玉山椒巖,山降。先是文生、鳳皆乞降,而鳳病死,其妻王玉貞籍所屬六萬八千餘人就撫,而精忠將林爾瞻猶擁眾石壟。之芳令懋勛等扼要隘,自以數十騎入寨,往撫慰之,爾瞻乃降。十七年,擊賊子午口,克八仙、老鼠諸洞,賊寨悉平。鄭錦寇瀕海,遣將嚴守御,敗之於廟嶺湖,又敗之於溫州。錦將詹天樞詣世凱降。十八年,檄定海總兵牟大寅斬錦將童耀等孝順洋,奪獲船隻、器械以還。

之芳練世故,沉幾善謀。康親王師將行,問之芳:「所策固萬全乎?」之芳曰:「軍已發,猶豫則士氣沮。」乃詣王曰:「虜在吾目中久,明日捷書至矣!」前軍捷書果至,傑書大喜,以為神。在杭州,與將軍圖喇約為兄弟。精忠既叛,語圖喇勿縱兵暴民。有滿兵犯法,之芳縛詣圖喇,以軍法治之,一軍肅然。浙亂平,疏請蠲被兵州縣額賦,安輯流亡,甚有威惠。所拔偏裨,皆累功至方鎮,而之芳以督臣不敘。久之,追論大溪灘破賊功,授拖沙喇哈番,準襲一次。

入為兵部尚書,調吏部。二十六年,授文華殿大學士。二十七年,御史郭琇疏劾大學士明珠,謂內閣票擬,皆聽明珠指揮,上既罷明珠,並命之芳休致。三十三年,卒於家,謚文襄。

之芳既卒,聖祖思其功,嘗諭群臣曰:「人能效命,即為勇士。耿精忠叛,時之芳為總督,雖不諳騎射,執刀立船首,率眾突前破敵。彼時同出征者,還京皆稱其勇。今承平久,善射,能約束士卒,尚不乏人。若屢經戰陣者,甚難得也!」世宗命立賢良祠,諭曰:「德若湯斌、功若之芳者,祀之。」乾隆間,錄勛臣後,命予恩騎尉,世襲。

論曰:圖海始阻撤籓之議,及其鷹揚西土,綏靖秦隴,卒收底川之績。川軍入滇,遂竟全功。之芳力扼三衢,敵雖東略,終不能得志仙霞。下閩之功,與有勞焉。雖曰遭時盤錯,抑亦聖祖馭材之效哉?並踐綸輔,易名曰襄。嗚呼,偉矣!


\end{pinyinscope}