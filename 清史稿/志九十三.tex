\article{志九十三}

\begin{pinyinscope}
○職官五內務府

△內務府行宮園囿御船處等官學武英殿修書處上駟院武備院

奉宸苑盛京內務府宦官

內務府總管大臣,無員限。滿洲大臣內特簡。初制從二品。乾隆十四年定正二品。其屬:堂郎中,主事,各一人。筆帖式三十有六人。廣儲司總管六庫郎中四人。內二人由各部員司兼攝。銀、皮、磁、緞、衣、茶六庫郎中四人。銀庫二人,兼司皮、磁二庫。緞庫二人,兼司衣、茶二庫。員外郎十有八人。庫各二人,兼攝各一人。六品司庫六人,庫各一人。八品司匠六人,銀、磁、衣三庫各二人。副司庫十有二人,庫使八十人。俱無品級。織造,蘇州,杭州各一人,司員內奏簡。六品司庫各一人,庫使、筆帖式各二人。會稽、掌禮、都虞、慎刑、營造、慶豐六司,郎中十有二人,司各二人。員外郎三十有二人。會稽、都虞、慶豐各五人,掌禮、營造各六人,慎刑四人。主事各一人。催長二十有三人,廣儲八人,會稽五人,都虞四人,掌禮、慎刑、營造各二人。自八品至無品級不等。副催長十有三人,廣儲、都虞各四人,會稽三人,掌禮、慎刑、營造各二人。自九品至無品級不等。委署催長一人,司匠二人。俱無品級。營造司置。錢糧衙門亦曰管理三旗銀兩處。郎中一人,員外郎四人,催長、副催長各三人。俱九品。司俎官四人,正六品。讀祝官四人,學習三人。贊禮郎十有三人,學習四人。俱六品銜食七品俸。八品催長一人,果房掌果、副掌果各二人,果上人十有二人,催長一人。俱九品。自司俎以下隸掌禮司。木、鐵、房、器、薪、炭六庫庫掌,副庫掌,各三人。庫守五十有五人。木、房二庫各十有一人,炭庫八人,鐵庫四人,器、薪二庫、圓明園薪炭庫各七人。無品級。鐵作、漆作司匠,八品銜。委署司匠,俱各一人。爆作庫掌、副庫掌各一人。俱未入流。隸營造司。牛羊★牧值年委署主事一人。六品銜食筆帖式原俸。隸慶豐司。官房租庫庫掌一人,庫守三人。內管領掌關防一人,郎中充。協理二人。員外郎充。內管領、初制正五品。道光三十五年改從五品。副內管領正六品。各三十人。庫掌十有五人,菜庫六人,車庫五人,酒、醋、房、器庫各二人。倉長十有三人。官三倉六人,外餑餑房三人,內餑餑房、器倉、糖倉、米倉各一人。俱無品級。養心殿造辦處郎中、員外郎各二人,主事一人,六品庫掌六人,副庫掌十人,八品催長十有四人。其兼轄者:圓明園活計處副庫掌四人,副司匠九人。俱無品級。中正殿員外郎、副內管領三十額內題補。各二人,無品級催長一人。寧壽宮郎中、員外郎各二人,主事、委署主事各一人。武英殿修書處正監造員外郎、副監造副內管領、六品庫掌、委署主事各一人,七品銜庫掌二人。御書處正監造司庫六品銜食七品俸。一人,副監造庫掌六品銜食八品俸。二人,七品銜副庫掌六人。茶膳房一、二、三等侍衛,一等三品,二等四品,三等五品。尚膳正各三人,四品。尚茶正各二人,四品。尚膳副、尚茶副、俱五品。主事俱各一人。膳上侍衛十有三人,茶上侍衛八人,俱六品。主事、委署主事各一人,承應長十有三人,庖長八人,庫掌五人,庫守十有六人。承應長以下,給虛銜金頂。御藥房初以總管首領太監管理。康熙三十年始來隸。主事一人,七品銜庫掌二人,委署主事、催長各一人。火藥庫庫掌二人。各處筆帖式二百有七人。自養心殿以下,並簡大臣領之,與內府大臣同為內廷右職。其兼攝者:升平署、官房租庫、犧牲所司員各二人。保和、太和、中和三殿司員、內管領各一人。壽康宮、慈寧宮花園司員各二人。御藥房內管領一人,副內管領二人。總理工程處司員無恆額。查覈房、督催房、匯稿處,並遴司員分涖其事。

總管大臣掌內府政令,供御諸職,靡所不綜。堂郎中、主事掌文職銓選,章奏文移。廣儲掌六庫出納,織造、織染局隸之。會稽掌本府出納,凡果園地畝、戶口徭役,歲終會覈以聞。掌禮掌本府祭祀與其禮儀樂舞,兼稽太監品級,果園賦稅。都虞掌武職銓選,稽覈俸饟恩血阜,珠軒歲納,佃漁歲輸,並定其額以供。慎刑掌本府刑名,依律擬罪,重讞移三法司會訊題結;番役處隸之。營造掌本府繕修,庀材飭工,帥六庫三作以供令。慶豐掌牛羊群牧,嘉薦犧牲。錢糧衙門掌三旗莊賦,治其賞罰與其優血阜。內管領處掌承應中宮差務,並稽官三倉物用、恩豐倉餼米。官房租庫掌收房稅。養心殿造辦處掌制造器用。中正殿各司員掌喇嘛唪經。武英殿修書處掌監刊書籍。雍和、寧壽兩宮司員掌陳設氾埽,兼稽宮監勤惰。御書處掌鐫摹御書。御茶膳房掌供飲食。御藥房掌合丸散。犧牲所掌牧養黝牛。總理工程處掌行營工作。凡遇工程,簡勘估大臣、承修大臣,事畢簡查驗大臣。

初制,設內務府,以舊屬司其事。入關後,明三十二衛人附之,設內管領處,置內管領八人。順治三年增四人,十一年增八人,分隸三旗。康熙二十四年增四人,三十年增三人,三十四年增三人。設茶飯處,置總領各三人,飯上人三十有五人,茶上人十有七人,康熙二十年置飯上人委署總領一人。雍正元年定總領授二等侍衛;飯上人授三等侍衛六人,藍翎侍衛七人;茶上人三等侍衛三人,藍翎侍衛四人;復置茶房侍衛內委署總領一人。乾隆八年定三等侍衛內各授一等侍衛一人。十五年改飯房為外膳房。二十四年改總領為尚膳正、尚茶正,副總領為尚膳副、尚茶副。承應長十人,康熙六十一年增一人。雍正元年增一人。庖長三人,康熙五十六年增六人,六十一年增一人。雍正元年增二人。及蘇州、江寧、杭州織造官。光緒三十年省江寧一人。順治十一年,命工部立十三衙門,設司禮、御用、御馬、內官、尚衣、尚膳、尚寶、司設八監,尚方、惜薪、鐘鼓三司,兵仗、織染二局;並三旗牛羊群牧處,置員外郎六人。管理牛只、羊只各三人。康熙二十三年各增二人。乾隆十四年省入寧壽宮一人。咸豐三年省入慎刑司二人。光緒三十年省一人。明年,改尚方司為院,置郎中三人,康熙三十一年省一人。員外郎六人,康熙三十八年省一人,六十一年省一人。光緒三十年省四人。催總一人。雍正二年增一人。乾隆二十四年更名催長。下同。十三年,改鐘鼓司為禮儀監,尚寶監為司。時猶舊臣、寺臣兼用也。十七年,改禮儀監為院,置郎中三人,康熙三十八年省一人。員外郎八人,光緒三十年省一人。贊禮郎十有二人,雍正五年增五人。司胙官四人,康熙三十七年增一人。乾隆二十四年改「胙」為「俎」。光緒三十年省一人。喇嘛唪經處催總一人。乾隆三十三省。改內官監為宣徽院,置郎中三人,康熙二十八年省一人。雍正元年增一人。乾隆四十年改隸寧壽宮一人。員外郎六人,光緒三十年省一人。催總八人。康熙間屢有增損。嘉慶十一年定留頂戴催長五人。十八年,御用監設銀、皮、緞、衣四庫,置郎中三人,員外郎八人,庫使四十人。康熙九年增二十人,十四年增二十有四人,明年省四人,二十八年增十有二人。乾隆十二年升十二人為副司庫。

康熙元年,誅內監吳良輔輩,復以三旗包衣設內務府,改尚膳監為採捕衙門,置郎中三人,三十八年省一人。員外郎六人,六十一年省一人。催總四人。並改惜薪司為內工部,置郎中三人,三十八年省一人。員外郎六人,十六年增二人。光緒三十年省二人。無品級庫掌十有二人,三十五年增二人。雍正三年增三人,明年增一人。復增置庫守、內副庫掌八人。尋又改為庫掌、副庫掌各十有二人,砲作庫掌、副庫掌各一人。八品催總一人,雍正四年增。無品級。催總一人,復於領催內增委署三人。乾隆二十四年改委署催總為委署司匠。庫守五十有九人。三十五年增八人。並置總管大臣,兼以公卿,無專員。三年,置錢糧衙門員外郎六人。咸豐二年省入慎刑司四人。九年,四庫各置六品司庫二人。十二年,置御藥房庫掌二人。明年,總管大臣兼轄內三院。十六年,置堂主事一人。改御用監為廣儲司,宣徽院為會稽司,禮儀院為掌儀司,省牛羊群牧處入之。置掌果二人,果上人十有二人。尚方院為慎刑司,採捕衙門為都虞司,內工部為營造司。二十三年,又分掌儀司立慶豐司,置郎中二人。乾隆四十年省入寧壽宮一人。五十七年復增一人。是為七司。至是奄宦之權悉歸於府矣。是歲置內副管領二十人。二十四年增四人,三十年增三人,三十四年增三人。二十五年,茶飯房設乾肉庫,置庫掌一人。三十年增一人,五十八年增二人。雍正五年增一人,十二年增一人。二十八年,廣儲司設瓷、茶二庫,各置員外郎二人,司庫二人,六庫通舊十有二人。光緒三十年省六人。匠役催總六人,乾隆二年增買辦催總二人,二十四年改買辦催總為催長,匠役催總為司匠。無品級催總四人,乾隆二十四年改副催長。是為六庫。明年,改文書館為武英殿修書處,置監造官六人。雍正二年省,四年復故。乾隆四十七年定正監造為員外郎,副監造為副內管領。御書處監造官四人。四十六年增二人。雍正二年省。八年置一人。乾隆四十七年定監造為司庫。二十五年,暢春園設柴炭庫,置無品級庫掌二人,庫守八人。四十二年,置堂郎中一人。授永定河分司齊蘇勒,升後未補。四十五年,置掌儀司副掌果二人。六十年,設官房租庫,置庫掌一人。

雍正元年,設錢糧衙門,置郎中一人,堂司委署主事十人。十二年省。乾隆二十二年復故。嘉慶四年增堂上一人。光緒三十年省。留慶豐司一人。明年,設養心殿造辦處,置六品庫掌四人,乾隆三十年增二人。嘉慶四年增四人。光緒三十年省四人。御書處庫掌一人,乾隆四年增二人。四十七年改一人為副監造。稽查御史一人。十一年省。乾隆三年改由都察院派員稽查。三年,置錢糧衙門無品級催總一人。七年增一人。乾隆四年增一人。復於領催內增副催總三人。二十四年改副催長。嘉慶三年留頂戴催長、副催長各三人。四年,置茶飯房主事一人。改都虞司承辦鮮魚歸掌儀司,增催總一人。八年增一人。乾隆八年增置承辦姜蒜領催、內副催總二人。二十四年更名副催長。十三年,復置坐辦堂郎中,省督催所入之。乾隆元年,置錢糧衙門主事一人。四十年改隸寧壽宮。五年,置造辦處專管庫務官、造辦事務官各一人,御藥房主事一人。七年,置御書處庫掌二人,八年增一人,十五年增一人,四十四年增二人。官房租庫委署主事一人。尋省。十二年,六庫置委署司庫各二人,尋改為副司庫。二十三年,改造辦處庫務事務官為郎中,各置一人,員外郎二人,主事、委署主事各一人,御藥房委署主事一人。二十六年,置總理工程處委署主事一人。後改司員兼管。四十年,置寧壽宮郎中、員外郎各二人,主事一人。咸豐六年,增置讀祝官四人。故事,內府讀祝官咨取太常寺贊禮郎為之,至是始定員缺。宣統元年,避上諱改掌儀司曰掌禮。

初制,司吏、宣徽、禮儀、尚方諸院,置總理,左、右協理各一人。御用、御馬、尚衣、尚膳諸監,置都管,左、右副管各一人。尚寶、惜薪二司,置都知,左、右參知各一人。司設、兵仗二局,置總轄,左、右佐轄各一人。文書館,置承制,左、右僉承各一人。後俱省。

東陵所屬盤山總管一人。從五品。乾隆二十九年置。內圍千總、六品。委署千總七、八品兼用。各七人。外營千總一人,把總七人。分駐盤山、燕郊、白澗、桃花寺、隆福寺、大興莊、棽髻山。

西陵所屬黃新莊總管一人。乾隆二十九年置。內圍千總、委署千總、外營把總各四人。分駐黃新莊、半壁店、秋蘭村、梁格莊。

湯泉所屬總管一人。康熙五十四年置八品總領。乾隆六年改置。苑丞、六品銜食八品俸。嘉慶十七年置。苑副未入流。各一人。內圍千總、委署千總各六人,外營把總九人。分駐石槽、三家店、密雲縣、要亭、羅家橋、懷柔縣。自盤山以下各千總,俱乾隆間置。

熱河所屬總管、康熙四十二年置。乾隆十六年定為本府額外郎中。二十一年改佐領職銜。三十五年給四品職銜。光緒三十年省歸都統管。副總管乾隆二十一年置,定為郎中職銜。三十五年增三人,秩定五品,後改苑副。光緒三十年省。各一人。苑丞、乾隆五十四年改苑副置。苑副乾隆三十五年後置三人,四十五年增一人。五十四年改苑丞。嘉慶十八年後,復以千總十人改置。二十年增一人。二十四年定與千總互為轉補。自是員額無恆制。道光十八年省四人,二十八年又省四人。各四人。內圍千總十有八人,乾隆九年置。道光十二年省二人,十八年又省二人。委署千總二十有八人。道光九年省七品一人。十八年省七品、八品各十有二人。千總、委署千總分駐兩間房、巴克什營、長山峪、王家營、喀喇河屯、釣魚臺、黃土坎、中關、十八里臺、汰波洛河屯、張三營、吉爾哈郎園。

總管以下掌翊衛行宮,稽察陳設。千總以下掌典守器物,稽察內圍,董帥氾埽。

圓明園總管事務大臣,無員限。特簡。其屬:郎中、主事各一人,員外郎二人,苑丞六人,六、七品兼用。苑副十有六人,七、八品兼用。委署苑副十有三人。九品銜。銀庫、器皿庫委署庫掌一人,庫守十有六人,筆帖式十有四人。雍正元年,置總管大臣。有協理事務官,或奏派,或簡授,無恆額。明年,置總領六人,乾隆十六年,長春園建成,置六品一人。二十四年改苑丞。三十二年增熙春園六品一人。四十六年增春熙院七品一人。嘉慶七年省春熙院一人入熙春園。十六年改暢春園七品一人為本園苑副。咸豐十年省六品二人。光緒三十年省六品一人。副總領十有二人。乾隆八年增七品、八品各一人。十六年增長春園七品、八品各一人。二十四年改苑副。三十九年增綺春園七品一人。四十五年增春熙院八品一人。嘉慶七年省春熙院一人,改為本園額缺。十六年復省暢春園八品一人,改為本園額缺。道光二年省暢春園四人入綺春園。咸豐十年省七品二人、八品三人。光緒三十年省七品一人、八品二人。七年,定總領為六品戴藍翎,後六、七品兼用。副總領七、八品半之。乾隆六年,置委署副總領二人。十六年增五人。三十二年改委署苑副,復增九人。嘉慶十六年增二人。咸豐十年省二人。光緒三十年省三人。八年,置主事一人。十四年,置庫掌一人,三十八年定為六品,增七品一人。光緒三十年俱省。委署庫掌一人,三十二年增一人,三十八年省一人。庫守六人。四十六年增十有二人。咸豐十年省二人。二十二年,增置委署主事一人。光緒三十年省。明年,定協理事務郎中、員外郎各一人。道光二年,改暢春園郎中為綺春園郎中,咸豐十年省。並省其員外郎一人,令專司長春園事。

暢春園總管大臣,無員限。特簡。其屬:苑丞三人,六、七品兼用。苑副五人,八品。委署苑副六人,九品銜。筆帖式三人。康熙間,置郎中一人,道光二年省入綺春園。八品總領三人,四十三年增西花園二人。乾隆五年省一人入靜明園。二十四年改苑丞。三十二年改授六品一人,七品三人。嘉慶十六年省七品一人。無品級總領十人。四十三年增西花園一人。乾隆五年省一人入靜明園。三十二年改委署苑副,額定十有六人。嘉慶十二年省二人入圓明園。道光二年省四人入綺春園。二十九年,置總管大臣。乾隆三十二年,置八品苑副六人。嘉慶十六年省一人入圓明園。

頤和園、靜明園、靜宜園總管大臣,無員限。特簡。其屬:郎中一人,員外郎三人,苑丞十有七人,頤和園十有一人,靜明園、靜宜園各三人,並六、七品兼用。苑副二十有三人,頤和園十有三人,靜明園六人,靜宜園四人,並六、七品兼用。委署苑副七人,靜明園三人,靜宜園四人,俱九品銜。筆帖式十有四人。乾隆十五年,甕山命名萬壽山,建行宮,改金海為昆明湖。明年更名清漪園。光緒十四年更名頤和園。置八品銜委署總催一人。四十八年升六品苑丞。十六年,置總理大臣兼領靜明園、靜宜園事,並六品總領一人,十九年增六品二人。二十四年改苑丞。嘉慶五年省一人入靜明園。十年增六品二人。光緒十四年後,移靜明園六品四人、七品六人,賡續置為本園員額。三十年省六品、七品二人。七品、八品副總領各二人,十八年增七品六人。二十四年改苑副。咸豐十年省八品一人。光緒十四年後,移靜明園八品八人,賡續置為本園員額。三十年省八品四人。八品催總一人。二十四年改催長。四十六年升六品銜苑丞。四十八年定六品秩。十八年,置委署副總領十有二人。尋省六人。咸豐十年省二人。光緒三十年省四人。二十二年,置員外郎一人,兼司靜明園事。二十六年增一人。嘉慶四年,置郎中一人,協理三園事務。明善堂、觀妙堂、西爽村並隸之。其園外鑒遠堂、藻鑒堂、暢觀堂、景明樓、鳳凰墩、治鏡閣、耕織圖,又功德寺,並由大臣遴本處官承其事。玉泉山靜明園初為澄心園,康熙三十一年更名。置無品級總領一人,乾隆五年增一人,八年定秩七品。二十四年改苑丞。三十四年增六品一人。嘉慶四年增六品一人,明年又增六品一人。道光二十三年省七品一人。光緒十三年增六品四人,七品六人。後省入頤和園。副總領二人。康熙三十年增一人。乾隆五年增一人。九年省入靜宜園一人。十八年定秩八品。二十四年增八品一人,改為苑副。三十四年增八品二人。嘉慶五年增置七品二人。道光二十三年省八品一人。咸豐十年省八品一人。光緒十三年增八品八人。後省入頤和園。乾隆二十四年,置委署副總領二人。三十四年增二人。嘉慶五年增一人。道光二十三年省二人。靜宜園初為香山行宮。乾隆十二年更名。乾隆九年,置員外郎一人,道光二十三年省。副總領二人。二十四年改苑副。十年,置八品總領一人,十二年增一人。十六年定秩七品,復增一人。二十四年改苑丞。三十四年增七品一人。四十六年增宗鏡大昭廟六品一人。嘉慶四年增七品二人;尋又增一人。道光二十三年省六品一人,七品二人。無品級副總領一人。十二年增一人。十六年定秩八品,復增一人。二十四年改苑副。三十四年增八品一人。四十六年增宗鏡大昭廟七品一人。四十八年增普覺寺七品一人。道光二十三年省八品三人。咸豐十年省八品一人。二十六年,置委署苑副六人。三十四年增二人。四十年增二人。道光二十三年省四人。咸豐十年省二人。

御船處統領大臣,無員限。兼管司員一人,筆帖式二人,八品司匠一人,八品水手催長四人,八品網戶催長二人。乾隆十六年,改圓明園清漪園御舟事務設御舟處,置統領大臣以次各官。明年,置八品水手催總三人,三十一年增一人。八品網戶催總一人。嘉慶四年增一人。二十四年,改催總為催長。

管理養鷹狗處大臣,無員限。養鷹鷂處統領二人。侍衛內揀補。藍翎侍衛頭領、副頭領各五人。六品冠戴。養狗處統領二人。藍翎侍衛頭領五人,副頭領十人,六品冠戴九人。七品一人。筆帖式六人。初設養狗處及鷹房、鴉鶻房。乾隆十一年改房為處。三十一年裁養鴉鶻處。其員額並入鷹上。

咸安宮官學管理事務大臣,本府大臣內特簡。協理大臣,各部院滿尚書內特簡。各一人。總裁,滿洲二人,漢四人。翰林院讀講學士、詹事府少詹以下兼充。繙譯教習六人。八旗滿、蒙、漢軍舉貢生監考充。清語教習,滿洲三人。弓箭教習,滿洲四人。本府內挑補。漢書教習,漢九人。進士、舉人考補。筆帖式一人。雍正七年,置蒙古官學管理事務大臣一人。理籓院尚書簡充。總裁三人。理籓院司員充。教習,蒙古二人,額外一人。乾隆十三年,置景山官學總管四人。本府司員兼充。繙譯教習,滿洲九人。本府內考補。漢書教習,漢十有二人。舉貢內考補。康熙二十四年置以上三學,俱光緒三十年後省。又,初制有回、緬官學總管二人,本府司員兼充。教習回子、回子佐領下派充。緬子緬甸人派充。各二人。長房官學教習,滿洲二人,本府筆帖式內揀補。蒙古一人。理籓院筆帖式內咨補。先後俱省。

武英殿總裁,滿、漢各一人。尚書侍郎內簡。提調二人,纂修內奏充。纂修十有二人,協修十人,翰林官充。筆帖式四人。

上駟院兼管大臣,無員限。卿二人。正三品。其屬:堂主事二人,委署主事一人,左、右二司郎中一人,掌左司印。右司,員外郎管。員外郎各二人。主事、委署主事各一人,內張家口值年一人。筆帖式十有一人。阿敦侍衛十有五人。司鞍長三人,正六品。副長二人。六品銜。蒙古醫師長三人,正六品。副長二人。八品。牧長二人,初無品級。雍正元年定正七品。副長五人。八品。廄長、署主事各一人。雍正元年各增一人,十二年省,乾隆二十二年復故。光緒三十年省。雍正六年,卿秩定三品。乾隆十一年,置蒙古醫生頭目二人。四十三年額定三人。十四年,定卿額二人,一用侍衛,一用內府官。二十三年,置八品頂戴司鞍長二人。三十九年定拜唐阿補放者給六品銜,戴藍翎。四十五年額定三人。嘉慶六年,依左、右司例,堂上令侍衛兼司。

武備院兼管大臣,無員限。卿二人。正三品。郎中一人,主事二人。南鞍、北鞍、甲、氈四庫員外郎,六品庫掌,各二人;委署六品庫掌各一人。傘房掌蓋、正六品。乾隆四十四年賞戴藍翎。副掌蓋,八品。帳房處司幄、三等侍衛銜食六品俸。副司幄,六品職銜食七品俸。各三人。備弓處司弓、六品職銜食七品俸。乾隆四十四年賞戴藍翎。副司弓,八品職銜。備箭處司矢、副司矢,各二人。職銜同備弓處。箭匠、頭、鞾皮、熟皮、鞍板、染氈、沙河氈作諸司匠,及穿甲官頭目,各一人。■J4作司匠二人。俱八品。無品級庫掌六人,庫守三十有二人。筆帖式二十有四人。

卿掌四庫工作,修造器械,陳設兵仗。凡車駕出入,官屬服櫜鞬以從。郎中、主事掌庫帑出納,章奏文移。北鞍庫掌御用鞍轡、傘蓋、幄幕,傘房、帳房、鞍板作隸之。南鞍庫掌官用鞍轡、皮張、雨纓、絳帶,熟皮作隸之。甲庫掌盔甲、刀仗、旗纛、器械,■J4作隸之。氈庫掌弓箭、鞾鞋、氈片,頭作、鞾皮作、作、沙河氈作、帽作、雜活作帽作以下置領催各一人。隸之。

初名鞍樓,置三旗侍衛三人綜其事。所屬:員外郎四人,康熙十五年、四十五年俱增三人。庫掌三人,順治十一年定六品。康熙十五年增三人,四十五年增二人。庫守二十有四人。康熙十五年增十有八人。三十六年增四人。四十五年增十人。氈庫、弓匠固山達,委署固山達,各三人。亦曰弓箭協領。康熙十一年增置備箭固山達一人,亦曰備箭協領。二十一年定弓匠固山達七品,三十八年定備箭固山達八品。乾隆二十九年,更名司弓、司矢,委署者曰副司弓、副司矢。四十四年定司弓、司矢六品職銜,副司弓、司矢八品職銜。光緒三十年各省一人。掌傘總領二人。康熙三十三年增一人。乾隆二十四年更名掌蓋。帳房頭目,委署帳房頭目,各三人。康熙二十七年定頭目為七品。乾隆三年定委署頭目八品職銜。二十四年改頭目為司幄,委署者為副司幄。三十六定司幄六品職銜,副司幄七品職銜。順治十一年,更名兵仗局。十八年,更名武備院。康熙九年,沙河氈作置催總一人。乾隆二十四年改司匠。下同。十五年,分設鞍、甲、氈三庫,置無品級庫掌三人。四十三年增二人,四十五年增四人。明年,以職掌事務侍衛一人掌印。二十一年,置郎中一人,並定■J4作、亮鐵作、原置■J4作、亮鐵作催總六人。二十七年省■J4作三人。光緒三十年省亮鐵作一人。氈作催總秩八品。三十七年,分鞍庫為南、北,增置頭作催總一人。亦曰鳴鏑長。三十九年,置鞾皮作催總。明年,置熟皮作催總,並定其品秩。復置穿甲官頭目一人。由拜唐阿內委放。乾隆八年定八品職銜。六十一年,置委署主事一人。雍正十二年省。乾隆二十二年復故。雍正六年,以職掌事務侍衛為三品卿。乾隆十四年,定卿額二人,仍管以大臣。

奉宸苑兼管大臣,無員限。卿二人。正三品。郎中一人,員外郎四人,主事一人,苑丞十人,六品。苑副十有九人,九品。委署苑副十人,筆帖式十有五人。天壇齋宮苑丞、六品一人。六品銜一人。苑副各二人。稻田廠庫掌,六品。無品級催長,委署催長,各一人。筆帖式三人。南苑郎中一人,員外郎二人,主事一人,苑丞七人,六品銜。苑副十有三人,八品銜。委署苑副六人,九品銜。委署催長三人,筆帖式五人。

卿掌苑囿禁令,以時修葺備臨幸。郎中以下各官掌分理苑囿河道。齋宮掌陳設氾埽。稻田廠掌供內庭米粟,兼徵田地賦稅。南苑各官掌徵南苑地賦,並治園庭事務。其兼攝者:齋宮兼理郎中,值年員外郎,稻田廠值年員外郎,各一人。

初紫禁城後山、西華門外臺,隸尚膳監管理,置八品催總二人。雍正二年增二人。順治十二年,更名景山、瀛臺。明年,改令內監管理,玉泉山、南苑並隸之。十八年,改南苑隸採捕衙門,置員外郎二人。雍正元年增一人。康熙八年,省南苑員外郎一人,改授郎中。十年,命內務府總管海喇孫、侍衛布喇兼司景山、瀛臺事。十六年,改歸都虞司管理。二十三年,始設奉宸苑,置郎中一人,乾隆十六年增一人,輪管長河行宮事。員外郎四人,主事一人。三十年,置南苑八品催總二人,乾隆四年增一人,十八年復增一人,分隸三旗。無品級總領一人,三十六年增南紅門行宮一人。副總領二人。三十六年增南紅門行宮二人。五十二年增南紅門新行宮一人。雍正元年,置奉宸苑、南苑委署主事各一人。十二年省。乾隆二十二年復故。光緒三十年又省。別命大臣領稻田廠,舊派司官二人兼理。三年始來隸。乾隆二十年,命會同清漪園大臣管理。置玉泉山六品庫掌一人兼司之。三年,增置稻田廠無品級催總一人。明年,兼轄下清河以上徬口。置徬官司之。六年,定卿秩三品。乾隆元年,置南苑主事一人。十一年,增置闡福寺八品催總一人。十四年,依上駟院例,定卿額二人,仍簡大臣領苑事。十六年,增置樂善園、永安寺八品催總各一人,十七年增樂善園一人。樂善園無品級副總領二人,明年增一人。南苑委署催總一人。原置一人。明年復增一人。分隸三旗。是歲依各行宮園囿例,改瀛臺、永安寺等處催總為總領,副催總為副總領。二十四年,復改總領曰苑丞,副總領曰苑副,催總曰催長。二十六年,兼轄正覺寺,置苑副一人,令萬壽寺、倚虹堂苑丞分司之。並令闡福寺苑丞兼管宏仁、仁壽二寺,置委署苑副二人。積水潭置苑副、委署苑副各二人。是歲省各處委署苑副,酌留南苑三處行宮二人。析置瀛臺、永安寺、樂善園及河道四人,並給八品職銜。三十五年,極樂世界、萬佛樓建成,置委署苑副一人。明年,定奉宸苑苑丞品秩。先是苑丞秩八品,與各園庭體制不一,至是俱給六品虛銜。仍食八品原俸。三十八年,復定南苑苑丞品秩,食俸同上。改三旗八品催長三人為苑丞,副催長為苑副。四十一年,置釣魚臺苑丞、苑副各一人,新挖旱河、徬座、蓮花池、河泡、岔河並隸之。四十二年,南苑、團河新行宮告成,省新舊各行宮苑丞一人、苑副二人、委署苑副一人,置為本園額缺。新舊各行宮原置苑丞二人,苑副、委署苑副各四人,南苑行宮苑副三人。四十六年,省樂善園苑丞一人入團河行宮。嘉慶六年,定奉宸苑苑丞食六品俸、苑副食九品俸,各二人,餘悉如故。九年,復省樂善園苑丞、苑副額缺,析置中海苑丞、苑副,倚虹堂苑丞,釣魚臺苑副,北海及長河委署苑副各一人。十二年,復析三海等處苑丞、苑副各二人,令司天壇齋宮。故事,齋宮隸太常寺,歸奉祀壇戶典守。雍正間,置八品催總治其事。至是,額置苑丞各官,定苑丞食六品俸一人。以郎中、員外郎兼領之。

盛京內務府總管大臣一人。盛京將軍兼。後改東三省總督。佐領,驍騎校,各三人。堂主事,委署主事,各一人。廣儲司司庫三人,庫使十有六人。會稽、掌禮、都虞、營造四司,及文溯閣九品催長,無品級催長,各一人。織造庫催長,內管領處內管領,六品虛銜。倉領長,無品級。各一人。牧掌,隸都虞司。倉長,隸內管領處。各三人。俱無品級。筆帖式十有五人。順治元年,盛京包衣三旗置佐領三人,簡一人掌關防,並置司庫三人,乾隆十九年省一人。四十二年增一人。及催總、筆帖式各官。尋置庫使十人。乾隆九年增一人。康熙十七年,置領催下驍騎校一人。乾隆十七年,置總管。明年,置堂主事一人。司庫內改置。二十四年,改催總為催長。二十九年,定各催長員數。如前所列。增置內管領、委署主事筆帖式內改置。各一人。四十八年,文溯閣建成,置九品催長,無品級催長,各一人。光緒三十年,省主事各官。

宦官四品總管太監銜曰宮殿監督領侍。五品總管銜曰宮殿監正侍。亦有以七品執守侍充者。六品副總管銜曰宮殿監副侍。亦有系執守侍銜者。首領太監銜二:七品曰執守侍,八品曰侍監。又有副首領,八品侍監充。亦有無品級者。筆帖式。八品侍監充。敬事房置。自四品至八品凡五等。升遷降調,由內府移咨吏部。

敬事房。兼讀清字書房,漢字、蒙字書房,總管三人。宮殿監督領侍一人。宮殿監正侍二人。宮殿監副侍總管六人。委署總管無定額,執守侍充。專司遵奉諭旨,承應宮內事務與其禮節,收覈外庫錢糧,甄別調補內監,並巡察各門啟閉、火燭關防。執守侍、首領、侍監、筆帖式各二人,專司掌案辦事,承行內府文移,並司巡防坐更。乾清宮。首領四人,執守侍、侍監各二人。專司供奉實錄、聖訓,江山社稷殿香燭,收貯賞用器物,並司陳設氾埽,御前坐更。後省二人。正首領,執守侍充。副首領,侍監充。乾清門。侍監首領二人。專司御門聽政,寶座黼扆,晨昏啟閉,稽察臣工出入,登載南書房翰林入直、侍衛番宿。昭仁殿,兼龍光門。弘德殿。兼鳳彩門。侍監首領各二人。專司陳設氾埽,御前坐更。故事內廷重坐更,御前更尤重。更頭、更二惟首領及執事內監方充是差。以下同。端凝殿。兼自鳴鐘執守侍首領一人。專司近禦隨侍賞用銀兩,並驗鐘鳴時刻。懋勤殿。兼本房首領二人,執守侍、侍監各一人。專司承直御筆,收掌文房書籍,並登載內起居注。四執事。執守侍首領一人。專司上用冠袍帶履,隨侍執傘執爐,承應上用武備,收貯備賞衣服。後增置首領一人,以侍監充之。四執事庫。侍監首領一人。專司上用冠袍帶履,鋪設寢宮帷幔。奏事處。初制隸四執事。後置侍監首領一人,專司傳宣綸綍,引帶召對人員,承接題奏事件。乾隆三十九年,太監高雲從洩漏硃批記載,自後惟軍機奏事由此進呈。各部院奏摺及內府奏家事,並由奏事處官轉上。日精門。兼上書房侍監首領一人。專司啟閉關防,及至聖先師位前香燭。月華門。兼南書房侍監首領一人。專司啟閉關防,承應內廷翰林出入。尚乘轎。侍監首領二人。專司承應請轎隨侍。御藥局。兼太妃、太嬪以次各位下藥房,侍監首領二人。專司帶領御醫各宮請脈,及煎制藥餌。交泰殿。侍監首領二人。專司尊藏御寶,收貯勛臣黃冊,並驗鐘鳴時刻。坤寧宮。兼坤寧門侍監首領二人。專司祭神香燭,啟閉關防,後改置執守侍首領、侍監副首領各一人。東暖殿。兼永祥門。西暖殿。兼增瑞門。執守侍首領、侍監副首領俱各一人。專司陳設氾埽,關防坐更。後省副首領各一人,首領改侍監為之。景和門,隆福門,基化門,端則門。侍監首領各二人。後基化、端則二門各省一人。內左門,內右門。侍監首領俱各二人。內右門兼稽膳房眾太監出入,每晚具單報無事送敬事房。景仁,兼近光左門及御書房收貯書畫。御書房初置侍監首領一人,後始改隸。永壽,兼近光右門。承乾,翊坤,鍾粹,儲秀,延禧,啟祥,永和,長春,景陽,兼大寶殿。景陽初置侍監首領一人。後省,始來隸。咸福十二宮。侍監首領俱各二人。專司承應傳取,餘同各處。養心殿,重華宮,建福宮。首領四人。執守侍、侍監各二人。專司收貯賞用物品。後省執守侍首領一人。養心殿內,兼吉祥門宮殿監副侍副總管一人。執守侍首領、侍監副首領各二人。專司近禦隨侍,收掌內庫錢糧及古玩書畫。古董房,侍監首領一人。專司收貯古玩器皿。御茶房,執守侍首領三人。侍監副首領四人。專司上用茗飲果品,及各處供獻,節令宴席。後省總管一人。御膳房,執守侍總管三人。侍監首領十人。專司上用膳羞,各宮饌品,及各處供獻,節令宴席。後省總管一人、首領二人。鳥槍處,執守侍首領一人。專司隨侍上用鳥槍。弓箭處、按摩處隸之。後改為侍監。南果房。侍監首領一人。專司收貯乾鮮果品。毓慶宮,侍監首領二人。嘉慶元年,青宮臨御始置。蒼震門,遵義門。侍監首領、副首領各二人。專司啟閉關防。蒼震門首領兼稽祭神房眾人出入。後省首領,增副首領一人。齋宮。侍監首領一人。御花園。侍監首領、副首領各二人。專司園內鬥壇四神祠香燭,培灌花木,飼養仙鶴池魚。後改置執守侍首領、侍監副首領各一人。祭神房。侍監首領二人。無品級副首領一人。專司祭神省牲。後省首領一人。中正殿,英華殿。無品級首領各一人。專司香燭。欽安殿。兼城隍廟侍監首領三人。專司唪誦經懺,焚修香火。後省二人。壽皇殿。兼永思殿侍監首領一人。專司御容前香燭。後增置無品級副首領一人。雍和宮。執守侍首領、侍監副首領各一人。後俱省,改置無品級首領一人。兆祥所。兼遇喜處無品級首領一人。打掃處。侍監首領一人。專司運水添缸,並承應雜務。後省柴炭、燒坑二處侍監各二人隸之。熟火處。侍監首領三人。專司各處安設熟火,抬運柴炭,並承應雜務。造辦處。侍監首領一人。專司帶領外匠制造物件。做鐘處。侍監首領一人。所司同造辦處。北小花園。無品級首領一人。專司培灌花木。皇太后宮。執守侍副總管二人。侍監首領五人。茶房、膳房、藥房首領各一人。後省宮首領一人,增置茶、膳、藥三房首領一人。太妃,太嬪,侍監首領各一人。膳房執守侍首領一人。侍監首領二人。太妃以次位下膳房。統設執守侍首領一人,侍監首領二人。慈寧宮佛堂。無品級首領十人,內充喇嘛者二人。後改為首領五人,充喇嘛者三人。副首領二人。壽康宮。無品級首領四人。後改置執守侍首領、侍監副首領各二人。皇子,侍監首領一人。公主,皇孫,皇曾孫。無品級首領各一人。瀛臺。兼武成殿侍監首領、無品級副首領各一人。後增副首領一人。畫舫齋。兼蠶壇侍監首領一人,無品級副首領二人。初未置,後增。永安寺。兼承先殿侍監首領、無品級副首領各一人。後增置副首領一人。景山。執守侍總管一人,侍監首領二人。委署首領無品級,無恆額。南府。執守侍總管一人,侍監首領四人。委署首領與景山同。圓明園。兼長春園靜寄山莊宮殿監副侍總管一人,執守侍總管二人,執守侍首領十人,無品級首領四十有二人。後增置執守侍總管一人,首領四人,無品級首領九人,內恩賞侍監首領二人。頤和園,靜明園,靜宜園,盤山,暢春園,泉宗廟,聖化寺。俱圓明園總管首領等承應差務。內務府所屬掌禮司,侍監首領五人,無品級副首領八人。後省首領二人、副首領四人。司樂,無品級副首領二人。初未置,後增。營造司。侍監首領二人,無品級副首領四人。後省首領一人,副首領三人。陵寢及妃園寢。無品級首領二人。後省一人。南花園。無品級首領一人。永安寺、大西天。無品級首領各一人。兼充喇嘛。簾子庫。兼門神庫無品級首領一人。後增一人。太廟。無品級首領一人。後改置執守侍首領一人,侍監副首領二人。鑾輿衛。無品級首領四人。後省二人。又傳心殿、萬善殿、番經廠、漢經廠、奉宸苑、武備院、尚衣監、酒醋局各首領太監,後俱省。親王、郡王、固倫公主、和碩公主並有定制。首領俱各一人。親王七品,郡王、公主俱八品。

順治元年,按十三衙門給太監品級。十八年省,以內務府大臣總管。康熙十六年,設敬事房,置總管、副總管。定太和、中和、保和、文華四殿三作首領太監員數,給八品職銜。乾隆二十六年,省文華殿員額。四十七年,三大殿直殿太監俱省。六十一年,定五品總管一人,五品太監三人,六品太監二人。太監授職官自此始。雍正元年,定總管秩四品,副總管六品,隨侍首領七品,宮殿首領八品。四年,定敬事房正四品總管為宮殿監督領侍銜,從四品副總管為宮殿監正侍銜,尋改五品。六品副總管為宮殿監副侍銜,七品首領為執守侍銜,八品首領為侍監銜。八年,復定四品至八品,不分正、從。乾隆七年,定內監受爵制不使逾越。故事,寺人不過四品,至是纂為令甲。五十一年,定親王、郡王、公主太監首領員數,並給八品銜。嘉慶間增親王首領秩七品。嘉慶六年,賞慶郡王七品太監三人,儀親王、成親王、定親王增置八品太監一人,不為恆制。

太祖、太宗鑒往易軌,不置宦官。世祖入關,依明宮寢舊制,裁定員額,數止千餘。諭曰:「朕稽考官制,唐、虞、夏、商未用寺人。周始具其職。秦、漢以後,典兵干政,流禍無窮。」敕官員毋與內官交結。復於交泰殿鑄鐵碑,文曰:「以後有犯法干政,竊權納賄,屬託內外衙門,交結滿、漢官員,越分擅奏外事,上言官吏賢否者,凌遲處死。」未幾,吳良輔輩煽立十三衙門,擅竊威福,世祖遺詔發奸。聖祖嗣統,殲厥大憝。時明季內監猶有在宮服役者,綱紀肅然。雍正間,防範內監家屬,敕內官約束,直督具題。高宗立法峻厲,太監高雲從稍豫外事,張鳳盜毀金冊,並正刑書。車駕幸灤陽時,巡檢張若瀛杖責不法內監,特擢七階,並頒則例,俾永遵守。又諭:「明代內監多至數萬人,蟒玉濫加。今制宮中苑囿,綜計不越三千。」爾時並隸內府,蓋猶有塚宰統攝奄人之義。然其員數視世祖時已倍之。至教字停派漢員,報充弗由禮部,奏事改易王姓,屢加裁抑,以清風軌。故終高宗六十餘年,宦官不敢為惡。嘉慶初年,以內外交結,降吳天成七品總管,復以常永貴驕縱無法,革去六品總管,蕭得祿坐濫保罪,並革去督領侍。洎劉得財、劉金輩崇信邪教,謀納叛人,釀成林清巨變,兇悖滋甚。其後曹進喜向吏兵曹長索道府職名冊,馬長喜冒濫名器,曹得英私放鳥槍,張府且私藏軍械。同治元年,御史賈鐸疏聞內監演劇,裁貢緞為戲衣,乃未聞糾厥罰。八年,遂有安得海冒名欽差,織辦龍衣,船颺旗幟,居民惶駭。他如蓄養優伶,馳馬沖仗,累蠹法度,不可殫紀。光緒十二年,御史硃一新疏陳李蓮英隨醇親王巡閱海口,易蹈唐代覆轍,詔降主事。二十七年,總督陶模疏陳近日宦官事微患烈,弊政宜除,書上不報。宦官遂與國相終云。


\end{pinyinscope}