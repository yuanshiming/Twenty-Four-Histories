\article{列傳一百九十九}

\begin{pinyinscope}
左宗棠

左宗棠,字季高,湖南湘陰人。父觀瀾,廩生,有學行。宗棠,道光十二年舉人,三試禮部不第,遂絕意仕進,究心輿地、兵法。喜為壯語驚眾,名在公卿間。嘗以諸葛亮自比,人目其狂也。胡林翼亟稱之,謂橫覽九州,更無才出其右者。年且四十,顧謂所親曰:「非夢卜夐求,殆無幸矣!」

咸豐初,廣西盜起,張亮基巡撫湖南,禮闢不就。林翼敦勸之,乃出。敘守長沙功,由知縣擢同知直隸州。亮基移撫山東,宗棠歸隱梓木洞。駱秉章至湖南,復以計劫之出佐軍幕,倚之如左右手。僚屬白事,輒問:「季高先生云何?」由是忌者日眾,謗議四起,而名日聞。同裏郭嵩燾官編修,一日,文宗召問:「若識舉人左宗棠乎?何久不出也?年幾何矣?過此精力已衰,汝可為書諭吾意,當及時出為吾辦賊。」林翼聞而喜曰:「夢卜夐求時至矣!」

六年,曾國籓克武昌,奏陳宗棠濟師、濟餉功,詔以兵部郎中用,俄加四品卿銜。會秉章劾罷總兵樊燮,燮構於總督官文,為蜚語上聞,召宗棠對簿武昌,秉章疏爭之不得。林翼、國籓皆言宗棠無罪,且薦其才可大用。詹事潘祖廕亦誦言總督惑於浮辭,故得不逮。俄而朝旨下,命以四品京堂從國籓治軍。初,國籓創立湘軍,諸軍遵其營制,獨王珍不用。宗棠募五千人,參用珍法,號曰「楚軍」。十年八月,宗棠既成軍而東,偽翼王石達開竄四川,詔移師討蜀。國籓、林翼以江、皖事急,合疏留之。時國籓進兵皖南,駐祁門,偽侍王李世賢、忠王李秀成糾眾數十萬圍祁門。宗棠率楚軍道江西,轉戰而前,遂克德興、婺源。賊趨浮梁景德鎮,斷祁門餉道。宗棠還師擊之,大戰於樂平、鄱陽,殭尸十餘萬,世賢易服逃,而徽州賊亦遁浙江。自是江、皖軍勢始振。

十一年,詔授太常寺卿,襄辦江南軍務,乃率楚軍八千人東援浙。朝命國籓節制浙江,國籓薦宗棠足任浙事。宗棠部將名者,劉典、王開來、王文瑞、王沐,數軍單薄,不足資戰守;乃奏調蔣益澧於廣西,劉培元、魏喻義於湖南,皆未至,而宗棠以數千人策應七百餘里,指揮若定,國籓服其整暇。已而壕州陷,復疏薦之,遂授浙江巡撫。

時浙地唯湖、衢二州未陷賊,國籓與宗棠計,以保徽州,固饒、廣為根本。奏以三府屬縣賦供其軍,設婺源、景德、河口三稅局裨之,三府防軍悉隸宗棠。賊大舉犯婺源,親督軍敗之。同治元年正月,詔促自衢規浙。宗棠奏言:「行軍之法,必避長圍,防後路。臣軍入衢,則徽、婺疏虞,又成糧盡援絕之勢。今由婺源攻開化,分軍扼華埠,收遂安,使饒、廣相庇以安,然後可以制賊而不為賊制。」二月,克遂安。世賢自金華犯衢州,連擊敗之。而皖南賊復陷寧國,遣文瑞往援,克績溪。十一月,喻義克嚴州。二年正月,益澧及高連升、熊建益、王德榜、余佩玉等克金華、紹興,浙東諸郡縣皆定。

杭州賊震怖,悉眾拒富陽。時諸軍爭議乘勝取杭城,宗棠不喜攻堅,謂皖南賊勢猶盛,治寇以殄滅為期,勿貪近功。乃自金華進軍嚴州,令劉典將八千人會文瑞防徽州,以培元、德榜駐淳安、開化,而益澧攻富陽。劾罷道府及失守將吏十七人,舉浙士吳觀禮等賑荒招墾,足裕軍食。四月,授浙閩總督,兼巡撫事。劉典軍既至皖南,遂留屯。益澧攻富陽,軍僅萬餘人,皆病疫,宗棠亦患瘧困憊,富陽圍久不下,乃簡練舊浙軍,兼募外國軍助之攻。七月,李鴻章江蘇軍入浙攻嘉善,嘉興寇北援,於是水陸大舉攻富陽,克之。益澧等長驅搗杭州,魏喻義、康國器攻餘杭。宗棠以杭賊恃餘杭為犄角,非先下餘杭,收海寧,不能斷嘉、湖援濟,躬至餘杭視師。是時皖賊古隆賢反正,官軍連下建平、高淳諸邑。金陵賊呼秀成入謀他竄,獨世賢踞溧陽,與廣德賊比,中梗官軍。鴻章既克嘉善,上言當益軍攻嘉興。會浙師取常州,而廣德賊已由寧國竄浙。宗棠慮賊分擾江西、福建,乃檄張運蘭率所部趨福建,召劉典防江西。海寧賊蔡元隆以城降,更名元吉,後遂為驍將。三年二月,元吉會江蘇軍克嘉興。杭州賊陳炳文勢蹙約降,猶慮計中變,乘雨急攻之,夜啟門遁,杭州復,餘杭賊汪海洋亦東走。捷聞,加太子少保銜,賜黃馬褂。

移駐省城,申軍禁,招商開市,停杭關稅,減杭、嘉、湖稅三之一。益澧為布政使,亦輕財致士,一時翕然稱之。群賊聚湖州,乃移軍合圍,先攻菱湖。三月,江蘇軍克常州,賊敗竄徽、婺,趨江西。世賢踞崇仁,海洋踞東鄉,宗棠以賊入江西為腹心患,奏請楊岳斌督江西、皖南軍,以劉典副,從之。六月,曾國荃克江寧,洪秀全子福瑱奔湖州,俄復潰走,磔於南昌。七月,克湖州,盡定浙地。論功,封一等恪靖伯。

餘賊散走徽、寧、江西、廣東,折入汀州,福建大震。乃奏請之總督任,以益澧護巡撫,增調德榜軍至閩。四年三月,江蘇軍郭松林來會師,賊棄漳州出大埔。五月,進攻永定。世賢、海洋既屢敗,傷精銳過半,歸誠者三萬。宗棠進屯漳州,躡賊武平。於是賊竄廣東之鎮平,而福建亦定。

乃檄康國器、關鎮平兩軍入粵,王開琳一軍入贛防江西,劉典軍趨南安防湖南,留高連升、黃少春軍武平,伺賊進退。六月,賊大舉犯武平,力戰卻之。世賢投海洋,為所戕,賊黨益猜貳。詔以宗棠節制三省諸軍。十月,賊陷嘉應,宗棠移屯和平琯溪。德榜慮帥屯孤懸,自請當中路。劉典聞德榜軍趨前,亦引軍疾進。猝遇賊,敗,賊追典,掠德榜屯而過,槍環擊之,輒反走。是夜降者逾四萬,言海洋中砲死矣,士氣愈奮。時鮑超軍亦至,賊出拒,又大敗之。合閩、浙、江、粵軍圍嘉應。十二月,賊開城遁,扼諸屯不得走,跪乞免者六萬餘,俘斬賊將七百三十四,首級可計數者萬六千,詔賜雙眼花翎。

五年正月,凱旋。宗棠以粵寇既平,首議減兵並餉,加給練兵。又以海禁開,非制備船械不能圖自強,乃創船廠馬尾山下,薦起沈葆楨主其事。會王師征西陲回亂久無功,詔宗棠移督陜、甘。十月,簡所部三千人西發,令劉典別募三千人期會漢口,中途以西捻張總愚竄陜西,命先入秦剿賊。

陜、甘回眾數至百萬,與捻合。宗棠行次武昌,上奏曰:「臣維東南戰事利在舟,西北戰事利在馬。捻、回馬隊馳騁平原,官軍以步隊當之,必無幸矣。以馬力言,西產不若北產之健。捻馬多北產,故捻之戰悍於回。臣軍止六千,今擬購口北良馬習練馬隊,兼制雙輪砲車。由襄、鄧出紫荊關,徑商州以赴陜西。經營屯田,為久遠之規。是故進兵陜西,必先清關外之賊;進兵甘肅,必先清陜西之賊;駐兵蘭州,必先清各路之賊:然後餽運常通,師行無阻。至於進止久速,隨機赴勢,伏乞假臣便宜,寬其歲月,俾得從容規畫,以要其成。」

六年春,提兵萬二千以西。議以砲車制賊馬,而以馬隊當步賊。捻倏見砲車,皆不戰狂奔。時陜西巡撫劉蓉已解任,總督楊岳斌請歸益急。詔寧夏將軍穆圖善署總督,宗棠以欽差大臣督軍務。分軍三道入關,而皖南鎮總兵劉松山率老湘軍九千人援陜,山西按察使陳湜主河防,其軍皆屬焉。松山既屢敗捻,又合蜀軍將黃鼎、皖軍將郭寶昌,大破之富平。捻掠三原,沿渭北東趨,回則分黨西犯,麕集北山。宗棠以捻強於回,當先制捻。檄諸軍憑河結營,期蹙而殲之涇、洛間。捻乘軍未集,又折而西渡涇、渭,窺豫、鄂。已而大軍進逼,勢不復能南,乃趨白水。乘大風雨,鋌走入北山。宗棠防捻、回合勢,且北山荒瘠,師行糧不繼,因急扼耀州。十月,捻敗走宜川,別黨果竄耀州,合回匪攻同官。留防軍不能御,典、連升軍馳救,大破之。諸軍將雖屢敗捻,終牽於回,師行滯;而捻大眾在宜川者益北擾延長,掠綏德,趨葭州,回亦自延安出陷綏德。宗棠自以延、綏迭失,上書請罪,部議革職。時北山及扶、岐、汧、隴、邠、鳳諸回,所在響應。捻自南而北,千有餘里,回自西而東,亦千有餘里。陜西主客軍能戰者不及五萬,然回當之輒敗。松山等克綏德,回走米脂,捻復分道南竄。於是劉厚基出東北追回,松山等循西岸要捻。師抵宜川,回大出遮官軍,留戰一日,破之;而捻遂取間道逾山至壺口,乘冰橋渡河。宗棠奉朝旨,山右毗連畿輔,令自率五千人赴援,以劉典代督陜甘軍。

是年十二月,捻自垣曲入河南,益北趨定州,游騎犯保定,京師戒嚴。詔切責督兵大臣,自宗棠、鴻章及河南巡撫李鶴年、直隸總督官文,皆奪職。宗棠至保定,松山等連破賊深、祁、饒、晉。當是時,捻馳騖數百里間,由直隸竄河南、山東,已復渡運越吳橋,犯天津。鴻章議築長圍制賊;宗棠謂當且防且剿,西岸固守,必東路有追剿之師,乃可掣其狂奔之勢:上兩從其議。於是勤王師大集,宗棠駐軍吳橋,捻徘徊陵邑、濟陽,合淮、豫軍迭敗之,總愚走河濱以死,西捻平。入覲,天語褒嘉,且詢西陲師期。宗棠對以五年,後卒如其言焉。

七年十月,率師還陜,抵西安。時東北土寇董福祥等眾十餘萬,擾延安、綏德,西南陜回白彥虎等號二十萬,踞甘肅董志原。松山至,破土寇,降福祥;而回益四出剿掠,其西南竄出者,並力擾秦川,黃鼎破之。宗棠進軍乾州,諜報回巢將徙金積堡,分軍擊之,遂下董志原,連復鎮原、慶陽,回死者至三萬。督丁壯耕作,教以區田、代田法。擇嶮荒地,發帑金巨萬,悉取所收饑民及降眾十七萬居焉。遂以八年五月進駐涇州。

甘回最著者,西曰馬朵三,踞西寧;南曰馬占鰲,踞河川;北曰馬化隆,踞寧夏、靈州。化隆以金積堡為老巢,堡當秦、漢兩渠間,扼黃河之險,擅鹽、馬、茶大利。環堡五百餘寨,黨眾嘯聚。掠取漢民產業子女。陜回時時與通市,相為首尾。化隆以新教煽回民,購馬造軍械,而陽輸誠紿穆圖善。董志原既平,陜回竄靈州,化隆上書為陜回乞撫。宗棠察其詐,備三月糧,先攻金積堡,以為收功全隴之基。及松山追陜回至靈州,扼永靈洞。化隆懼,仍代陜回乞撫,謀緩兵,穆圖善信之,日言撫,綏遠城將軍至劾松山濫殺激變。然化隆實無意降也,密召諸回並出劫軍餉。十一月,宗棠進駐平涼。九年,松山陣歿,以其兄子錦棠代之,戰屢捷,而中路、南路軍亦所向有功,陜回受撫者數千人。及奪秦壩關,化隆益窘,詣軍門乞降,誅之,夷其城堡。遷甘回固原、平涼,陜回化平,而編管鈐束之,寧、靈悉定。奏言進規河湟,而是時有伊犁之變,詔宗棠分兵屯肅州,乃遣徐占彪將六千人往。

十年七月,自率大軍由平涼移駐靜寧。八月,至安定。寇聚河州,其東出,必繞洮河三甲集,集西太子寺,再西大東鄉,皆險要。諸將分擊,悉破平之。時回酋朵三已死,占鰲見官軍深入,西寧回已歸順,去路絕,遂亦受撫。河州平。

十一年七月,移駐蘭州。占彪前以伊犁之變率師而西也,於時肅州阻亂,回酋馬文祿先已就撫,聞關外兵事急,復據城叛。及占彪軍至,乃嬰城固守,而乞援西寧。陜回白彥虎、禹得彥亦潛應文祿。會錦棠率軍至,西寧土回及陜回俱變,推馬本源為元帥。西寧東北阻湟水,兩山對峙,古所稱湟中也。賊據險而屯,俄敗走,遺棄馬騾滿山谷,竄巴燕戎格。大通都司馬壽復嗾向陽堡回殺漢民以叛。十二年正月,錦棠攻向陽堡,奪門入,斬馬壽,遂破大通,搗巴燕戎格,誅本源,河東、西諸回堡皆降。文祿踞肅州,詭詞求撫,益招致邊外回助城守,連攻未能下。八月,宗棠來視師,文祿登城見帥旗,奪氣。請出關討賊自效,不許。金順、錦棠軍大集,文祿窮蹙出降,磔之。白彥虎竄遁關外,肅州平。以陜甘總督協辦大學士,加一等輕車都尉。奏請甘肅分闈鄉試,設學政。十三年,晉東閣大學士,留治所。自咸豐初,天下大亂,粵盜最劇,次者捻逆,次者回。宗棠既手戡定之,至是陜、甘悉靖,而塞外平回,朝廷尤矜寵焉。

塞外回酋曰帕夏,本安集延部之和碩伯克也。安集延故屬敖罕,敖罕為俄羅斯所滅,安集延獨存。帕夏畏俄逼,闌入邊。據喀什噶爾,稍蠶食南八城,又攻敗烏魯木齊所踞回妥明。妥明者,西寧回也,初以新教游關外。同治初,乘陜甘漢、回構變倡亂,據烏城。帕夏既攻敗妥明降之,遂並有北路伊犁諸城,收其賦入。妥明旋被逐,走死,而白彥虎竄處烏城,仍隸帕夏。帕夏能屬役回眾,通使結援英、俄,購兵械自備。英人陰助之,欲令別立為國,用捍蔽俄。當是時,俄以回數擾其邊境,遽引兵逐回,取伊犁,且言將代取烏魯木齊。

光緒元年,宗棠既平關隴,將出關,而海防議起。論者多言自高宗定新疆,歲糜數百萬,此漏卮也。今至竭天下力贍西軍,無以待不虞,尤失計。宜徇英人議,許帕夏自立為國稱籓,罷西征,專力海防。鴻章言之尤力。宗棠曰:「關隴新平,不及時規還國家舊所沒地,而割棄使別為國,此坐自遺患。萬一帕夏不能有,不西為英並,即北折而入俄耳。吾地坐縮,邊要盡失,防邊兵不可減,糜餉自若。無益海防而挫國威,且長亂。此必不可。」軍機大臣文祥獨善宗棠議,遂決策出塞,不罷兵。授宗棠欽差大臣,督軍事,金順副之。

二年三月,次肅州。五月,錦棠北逾天山,會金順軍先攻烏魯木齊,克之。白彥虎遁走托克遜。九月,克瑪納斯南城,北路平,乃規南路。令曰:「回部為安酋驅迫,厭亂久矣。大軍所至,勿淫掠,勿殘殺。王者之師如時雨,此其時也。」三年三月,錦棠攻克達阪城,悉釋所擒纏回,縱之歸。南路恟懼,翼日,收托克遜城,而占彪及孫金彪兩軍亦連破諸城隘,合羅長祜等軍收吐魯番,降纏回萬餘。帕夏飲藥死,其子伯克胡里戕其弟,走喀什噶爾。

白彥虎走開都河,宗棠欲遂擒之,奏未上,適庫倫大臣上言西事宜畫定疆界,而廷臣亦謂西征費鉅,今烏城、吐魯番既得,可休兵。宗棠嘆曰:「今時有可乘,乃為畫地縮守之策乎?」抗疏爭之,上以為然。時俄方與土耳其戰,金順請乘虛襲伊犁。宗棠曰:「不可。師不以正,彼有辭矣。」八月,錦棠會師曲會,遂由大道向開都河為正兵,餘虎恩等奇兵出庫爾。白彥虎走庫車,趨阿克蘇,錦棠遮擊之,轉遁喀什噶爾。大軍還定烏什,遂收南疆東四城,何步雲以喀什漢城降。伯克胡裏既納白彥虎,乃效力攻漢城。大軍至,復遁走俄。西四城相繼下,宗棠露布以聞,詔晉二等侯。布魯特十四部爭內附。

四年正月,條上新疆建行省事宜,並請與俄議還伊犁、交叛人二事。詔遣全權大臣崇厚使俄。俄以通商、分界、償款三端相要。崇厚遽定約,為朝士所糾,議久不決。宗棠奏曰:「自俄踞伊犁,蠶食不已,新疆乃有日蹙百里之勢。俄視伊犁為外府,及我索地,則索償盧布五百萬元。是俄還伊犁,於俄無損,我得伊犁,僅一荒郊。今崇厚又議畀俄陬爾果斯河及帖克斯河,是劃伊犁西南之地歸俄也。武事不競之秋,有割地求和者矣。茲一矢未加,遽捐要地,此界務之不可許者也。俄商志在貿易,其政府即廣設領事,欲藉通商深入腹地,此商務之不可許者也。臣維俄人包藏禍心,妄忖吾國或厭用兵,遂以全權之使臣牽制疆臣。為今之計,當先之以議論,委婉而用機,次決之以戰陣,堅忍而求勝。臣雖衰慵無似,敢不勉旃。」上壯其言,嘉許之。崇厚得罪去,命曾紀澤使俄,更前約。於是宗棠乃自請出屯哈密,規復伊犁。以金順出精河為東路,張曜沿特克斯河為中路,錦棠經布魯特游牧為西路;而分遣譚上連等分屯喀什噶爾、阿克蘇、哈密為後路聲援:合馬步卒四萬餘人。

六年四月,宗棠輿櫬發肅州,五月,抵哈密。俄聞王師大出,增兵守伊犁、納林河,別以兵船翔海上,用震撼京師,同時天津、奉天、山東皆警。七月,詔宗棠入都備顧問,以錦棠代之。而俄亦懾我兵威,恐事遂決裂。明年正月,和議成,交還伊犁,防海軍皆罷。

宗棠用兵善審機,不常其方略。籌西事,尤以節兵裕餉為本謀。始西征,慮各行省協助餉不時至,請一借貸外國。沈葆楨尼其議,詔曰:「宗棠以西事自任,國家何惜千萬金。為撥款五百萬,敕自借外國債五百萬。」出塞凡二十月,而新疆南北城盡復者,饋運饒給之力也。初議西事,主興屯田,聞者迂之;及觀宗棠奏論關內外舊屯之弊,以謂掛名兵籍,不得更事農,宜畫兵農為二,簡精壯為兵,散願弱使屯墾,然後人服其老謀。既入覲,賜紫禁城騎馬,使內侍二人扶掖上殿,授軍機大臣,兼值譯署。國家承平久,武備弛不振,而海外諸國爭言富強,雖中國屢平大難,彼猶私議以為脆弱也。及宗棠平帕夏,外國乃稍稍傳說之。其初入京師,內城有教堂高樓,俯瞰宮殿,民間讙言左侯至,樓即毀矣,為示諭曉,乃止。其威望在人如此。然值軍機、譯署,同列頗厭苦之。宗棠亦自不樂居內,引疾乞退。九月,出為兩江總督、南洋通商大臣。嘗出巡吳淞,過上海,西人為建龍旗,聲砲,迎導之維謹。

九年,法人攻越南,自請赴滇督師。檄故吏王德榜募軍永州,號「恪靖定邊軍」,法旋議和,止其行。十年,滇、越邊軍潰,召入都,再直軍機。法大舉內犯,詔宗棠視師福建,檄王珍子詩正潛軍渡臺灣,號「恪靖援臺軍」。詩正至臺南,為法兵所阻,而德榜會諸軍大捷於諒山。和議成,再引疾乞退。七月,卒於福州,年七十三,贈太傅,謚文襄。祀京師昭忠祠、賢良祠,並建專祠於湖南及立功諸省。

宗棠為人多智略,內行甚篤,剛峻自天性。穆宗嘗戒其褊衷。始未出,與國籓、林翼交,氣陵二人出其上。中興諸將帥,大率國籓所薦起,雖貴,皆尊事國籓。宗棠獨與抗行,不少屈,趣舍時合時不合。國籓以學問自斂抑,議外交常持和節;宗棠鋒穎凜凜向敵矣,士論以此益附之。然好自矜伐,故出其門者,成德達材不及國籓之盛云。子四人:孝威,舉人,以廕為主事,先卒,旌表孝行;孝寬,郎中;孝勛,兵部主事;孝同,江蘇提法使。孫念謙,襲侯爵,通政司副使。

論曰:「宗棠事功著矣,其志行忠介,亦有過人。廉不言貧,勤不言勞。待將士以誠信相感。善於治民,每克一地,招徠撫綏,眾至如歸。論者謂宗棠有霸才,而治民則以王道行之,信哉。宗棠初出治軍,胡林翼為書告湖南曰:「左公不顧家,請歲籌三百六十金以贍其私。」曾國籓見其所居幕★C7小,為別制二幕貽之,其廉儉若此。初與國籓論事不洽,及聞其薨,乃曰:「謀國之忠,知人之明,自媿不如。」志益遠矣。


\end{pinyinscope}