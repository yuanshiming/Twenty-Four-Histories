\article{列傳二十三}

\begin{pinyinscope}
陳泰阿爾津李國翰子海爾圖桑額卓布泰弟巴哈

卓羅四世孫永慶愛星阿子富善遜塔子馬錫泰從弟都爾德

陳泰,滿洲鑲黃旗人,額亦都孫,徹爾格子也。初授巴牙喇甲喇章京。從伐明,攻錦州,明兵自寧遠來援。陳泰先眾直入敵陣,斬執纛者,得纛以歸。天聰三年,從太宗伐明,薄明都,屯德勝門外,攻袁崇煥壘,遇伏,奮擊,多所俘馘。五年,從圍大凌河城,明監軍道張春赴援,陳泰設伏,擒其邏卒,復以步軍戰,殲敵。

崇德元年,從伐朝鮮,與梅勒額真薩穆什喀夜襲破黃州守將營。三年,伐明,敗明兵於豐潤,攻太監馮永盛、總兵侯世祿營,拔之;又以巴牙喇兵三十敗明騎卒百餘。五年,從圍錦州,攻杏山,敗敵兵,獲牲畜。六年,復圍錦州,敗松山兵。我兵出樵採,為敵困,陳泰率兵六援之出,敵襲我後隊,迭戰破敵,遂克其郛。予世職,自牛錄章京進三等甲喇章京。七年,復圍錦州,掘塹困松山。明兵夜犯正黃旗蒙古營,赴援,擊之走。八年,從伐明,敗總兵馬科於渾河,築浮橋濟師。明總督範志完拒戰,擊敗之。下山東,陳泰以偏師克東阿、汶上、寧陽三縣,進世職二等。

順治元年,從入關,擊破李自成,進世職一等。四年,授禮部侍郎。從平南大將軍孔有德征湖廣,戰荊州,擊破流賊一隻虎。時明魯王遣其將鄭彩、阮進等寇福建,先後陷府三、州一、縣二十七。上授陳泰靖南將軍,與梅勒額真棟阿賚率師討之,擊破魯王將曹大鎬、張耀星,克同安、平和二縣。五年三月,復克興化。彩遁入海,復克長樂、連江,獲所置總督顧世臣等十一人,斬之。魯王所陷諸府州縣以次盡復,福建平。師還,授巴牙喇纛章京,進二等阿思哈尼哈番。遇恩詔,累進三等精奇尼哈番兼拜他喇布勒哈番。遷刑部尚書。八年,移吏部尚書,授國史院大學士。以加上皇太后尊號恩詔誤增赦款罷任,並以吏部覃恩升襲過濫,降世職一等阿達哈哈番。九年,起禮部尚書,充會試主考官,授鑲黃旗滿洲固山額真,特命進世職二等精奇尼哈番。

十年,上以湖廣未定,大學士洪承疇再出經略,至軍,疏言:「孫可望等戰湖南,郝搖旗、一隻虎等擾湖北。湖南駐重兵,各郡窵遠,不免首尾難顧。」上授陳泰寧南靖寇大將軍,與固山額真藍拜、濟席哈,巴牙喇纛章京蘇克薩哈等統兵鎮湖南。臨行,上諭之曰:「師行有一定紀律,大小將佐,為國盡力,豈致挫釁?上毋藐視主帥,下當撫勵士卒,能愛眾而得其心,遇敵未有不爭先效命者也。」十一年,復授吏部尚書。十二年,孫可望遣其將劉文秀、盧明臣、馮雙禮等以舟師六萬分犯岳州、武昌。文秀引精兵攻常德,陳泰遣蘇克薩哈設伏以待。甲喇額真呼尼牙羅和當前鋒,挫敵;甲喇額真蘇拜、希福等以舟師迎擊;大軍繼進,三合三勝。敵復列艦拒戰,伏起,縱火焚其舟,敵大敗,別遣兵擊敵德山下。師進次龍陽,敵集二千人來犯,我兵奮擊,潰奔,明臣赴水死。雙禮被創,與文秀並遁。降所置裨將四十餘、兵三千餘。論功,進一等精奇尼哈番兼拖沙喇哈番。未幾,卒於軍。

師還,明年正月,世祖宴諸將,追悼陳泰,揮淚酌酒,諭學士麻勒吉、侍衛覺羅塔大曰:「大將軍班師還,朕將親酌酒以慰勞之。不幸中道棄捐,不復相見。爾等以此觴奠大將軍靈次,抒朕追悼。」諸將及侍從皆感涕。賜祭葬,謚忠襄。乾隆初,定封一等子。

阿爾津,伊爾根覺羅氏。父齊瑪塔,與從子阿山歸太祖,官侍衛。旗制定,與阿山同隸正藍旗。阿山自有傳。

阿爾津積戰功,授甲喇額真,世職二等參將。天聰四年,從固山額真納穆泰等守灤州。納穆泰等引還,論罪,上以阿爾津力戰殺敵,特貰之。九年,伐察哈爾,阿爾津從貝勒岳託駐歸化城。博碩克圖汗子陰結喀爾喀等部貳於明,阿爾津獲其使者,進世職一等甲喇章京。

崇德元年,上自將伐朝鮮,朝鮮國王李倧走南漢山城,阿爾津簡精騎追躡,破其援兵。二年,略鐵山,獲明邏卒。授議政大臣,領巴牙喇纛章京。三年,從豫親王多鐸伐明,過中後所,明將祖大壽發兵追襲,阿爾津為殿,戰不力,所部多戰死者,又棄其骨不收,坐削世職,籍家產之半,仍領巴牙喇纛章京如故。五年,從圍錦州,以離城遠駐,坐罰鍰。六年,攻松山,擊明總督洪承疇軍,克臺一、壘三,殲守臺敵兵,出我師被圍者。上嘉其善戰,復授三等甲喇章京。七年,從伐明,攻寧遠。八年,與巴牙喇纛章京哈寧阿等伐虎爾哈部,下七屯,俘獲無算。

順治元年,從入關,擊李自成,追及於安肅,大破之,進二等甲喇章京,兼半個前程。尋從豫親王多鐸西破潼關,還定江南,進三等梅勒章京。三年,與巴牙喇纛章京鰲拜等徇漢中,擊叛將賀珍,破流賊張獻忠,進攻敘州,所向克捷。五年,進一等阿思哈尼哈番。尋率師定宣化叛兵。八年,與固山額真額克青等發武英親王阿濟格罪狀,語詳阿濟格傳。敘功,遇恩詔,進一等精奇尼哈番兼拖沙喇哈番。九年,授西安將軍,鎮漢中。尋改授定南將軍,移師徇湖廣。十一年,自巴牙喇纛章京遷固山額真。

十二年八月,授寧南靖寇大將軍,與固山額真卓羅等率師駐荊州。時土寇姚黃等據歸州,出沒宜昌、襄陽間,阿爾津督兵搜捕,安集兵民,枝江、松滋諸縣悉定。十三年,與卓羅等率師渡江,十月,克辰州。寶慶、永順諸土司率官吏,具版籍,詣軍前降。時雲南、貴州尚為明守,阿爾津議移常德鎮兵守辰州,別移兵屯常德為應援,自辰州下沅、靖,進取滇、黔。經略大學士洪承疇與異議,事聞,上召阿爾津還京師,以宗室羅託代之。

十五年正月,授信郡王多尼為安遠靖寇大將軍,征雲南,命阿爾津率本旗兵以從。五月,卒於軍,贈太子太保,謚端果。乾隆初,定封三等子。

李國翰,漢軍鑲藍旗人,其先居清河。父繼學,初為商,從明經略楊鎬軍,嘗通使於我。天命六年,克遼陽,繼學來歸,授都司。以副將劉興祚婪賄,劾罷之。屢獲明諜,捕逃人,授世職三等副將。請老,國翰襲世職。事太宗,授侍衛,賜號「墨爾根」。

天聰三年,從伐明,薄明都。還攻永平,戰先眾。五年,圍大凌河,城兵突出,國翰督兵擊之退;明兵自錦州赴援,又擊之,敗走。九年,以善拊循所領人戶,進世職二等梅勒章京。崇德三年,授刑部理事官。從伐明,入邊,明兵千餘據山列陣,國翰督兵奮擊,敗之,獲馬四十;進克墻子嶺,轉戰至山東,克濟南。師還,攻慶都、獲鹿,發砲毀其垣。四年,授鑲藍旗漢軍梅勒額真。五年,從攻錦州,克呂洪山臺。七年,攻下塔山、杏山,擢鑲藍旗漢軍固山額真。八年,從克前屯衛、中後所。世職累進三等昂邦章京。

順治元年,從入關,國翰與固山額真劉之源、祖澤潤等率兵剿饒陽土寇康文斗、郭壯畿等,師進征山西。時李自成走陜西,其黨猶分據太原、平陽諸府,國翰與固山額真葉臣謀曰:「自成新敗,賊無固志,當以大兵直搗太原。太原既下,分道略定諸郡縣,餘賊非降即就馘耳。」乃合兵進拔太原,分道略定諸郡縣。師還,賚白金五百。尋又從大將軍英親王阿濟格征陜西,自成走湖廣,師從之,戰應山,進攻武昌,與固山額真金礪等奪舟數百。

二年,命偕固山額真巴顏率兵下四川,次西安,叛將賀珍自漢中來犯,國翰與駐防西安內大臣和洛輝分兵夾擊,大破之,進世職二等。三年,大將軍肅親王豪格師至,令國翰與巴顏逐捕延安餘寇,寇保張果老崖,掘壕困之,乘夜攻克其寨,殲其渠,獲馬二百餘。遂從肅親王下四川,殲張獻忠,復率兵渡涪江,擊破獻忠將袁韜,進世職一等。

五年四月,授定西將軍,同平西王吳三桂鎮漢中。六年,明宗室硃森滏與其將趙榮貴以萬餘人犯階州,國翰督兵赴援,戰必先眾陷陣。諸將請曰:「將軍任討賊之重,柰何輕身犯鋒鏑?脫有不戒,憂及全軍。」國翰曰:「吾固知此。然賊鋒頗銳,戰不利,勢將蔓延。吾故以力戰挫其鋒。明之失機,率由主兵者怯戰耗時,賊以坐大。覆轍可復蹈耶?」遂戰,陣斬森滏、榮貴;復擊破其將王永強,斬級數千,獲駝馬數百,復宜君、同官、蒲城、宜川、安塞、清澗等縣。上深嘉其勇略,諭以「自後但發縱指示,不必身先士卒」。叛將姜瓖據大同,其將劉登樓、張鳳翼、任一貴、謝汝德、萬鍊等分據附近諸郡縣,國翰遣兵會剿,殲賊甚眾,撫定河東;進克府谷,擒斬所置經略高有才以下三百餘人,降其將郝自德等:進一等伯。

九年,與三桂督兵復成都、嘉定,遣將徇重慶、敘州,皆下。明將王復臣等糾惈儸萬餘人犯保寧,列象陣攻城,國翰自綿州赴援,督兵橫擊敵,陣斬復臣,殲其眾。捷聞,進三等侯,賞紫貂冠服、■H6金甲胄、櫜鞬、鞍馬。十年,以四川平,命與三桂還鎮漢中。十四年,明將譚文等與自成餘黨劉二虎等為寇,陷重慶,使所置都督杜子香守之。十五年,國翰與三桂進討之,自西充下合州,子香迎戰,敗遁,復重慶,道桐梓,趨遵義。明將李定國遣其將劉正國等據險拒戰,擊之潰,自水西走雲南,取遵義及所屬州縣;復進克開州,並招降水西土司。時大將軍羅託、經略洪承疇已取貴陽,國翰還駐遵義,策會師取雲南。七月,卒於軍。喪至京,命內大臣致奠,贈太子太保,謚敏壯,侯爵襲三次,循例改襲三等伯。乾隆中,加封號懋烈。

海爾圖,國翰長子。初從國翰軍擊賀珍,破袁韜,皆在行,授兵部理事官、牛錄額真。擢鑲藍旗漢軍梅勒額真,授戶部侍郎,坐事罷。遷本旗固山額真。康熙初,襲三等侯爵。定西將軍貝勒董鄂討叛將王輔臣,命海爾圖運砲赴軍前,並參贊軍務。尋以運砲遲誤,解參贊,留駐鳳翔。從征雲、貴,二十年,卒於軍。

桑額,國翰第三子。康熙初,自參領擢寧夏總兵。遷雲南提督,未上官,吳三桂反,留駐荊州。改湖廣提督,移守武昌。從攻岳州,師進城陵磯,發砲沈敵艦,加右都督。三桂兵自洞庭湖出,桑額擊之卻,逐至岳州城下,三桂兵引去,收萬容、石首、安鄉諸縣,加左都督。詔趣進師,復以桑額為雲南提督,奏改湖廣提標兵為雲南提標,率之進克辰龍關,克辰州、沅州;復進克鎮遠、平越,下貴陽,趨雞公背。三桂兵焚鐵索橋走,桑額督土司沙起龍等築浮橋濟師。旋從大將軍貝子彰泰攻下雲南省城,其將馬寶、胡國柱自四川還救,桑額與副都統託岱等破寶於楚雄,寶走降;又與都統希福困國柱於永昌,國柱自經死。雲南平。

初,桑額標兵中道有潰散者,上遣左都御史哲勒肯按治,疏言標兵家口在武昌,無資養贍,逃回者千餘人。上切責桑額不恤士卒,部議奪職,命留任,敘功復職。二十五年,卒。

卓布泰,瓜爾佳氏,滿洲鑲黃旗人。父衛齊,費英東第九弟。事太祖,從特爾晉等率兵伐虎爾哈,得五百戶以歸,授世職備御。天聰初,從太宗伐明,略遵化,進世職游擊。上統大軍出征,每令衛齊留守盛京,任八門提督。卒。順治間,追謚端勤。子鰲拜,自有傳。

卓布泰,其次子也。事太宗,授牛錄額真。崇德四年,從承政薩穆什喀、索海伐瑚爾哈部,鐸陳、阿薩津二城以兵四百逆戰,卓布泰與牛錄額真薩弼圖率甲士九十人擊敗之,斬級五十。敵復與索倫部長博穆博果爾合兵以拒,卓布泰率先邀擊,俘六十餘人。五年,擢甲喇額真。六年,從伐明,圍錦州,明總督洪承疇屯山口拒守。卓布泰與梅勒額真翁阿岱迎戰,明兵敗走,大軍合擊,復與翁阿岱力戰破敵。師還,敵躡我後,翁阿岱中創僕,卓布泰還殲敵,掖翁阿岱乘馬歸。七年,從伐明,徇山東至青州,屢敗明兵。明將張登科、和應薦等合八鎮兵來拒,卓布泰率兵奮擊,大破之,復乘夜襲破餘兵。八年,師還,賚白金,兼任兵部理事官。順治元年,偕甲喇額真沙爾瑚達略黑龍江,克圖瑚勒禪城,俘二百餘人。

是冬,從大將軍豫親王多鐸西討李自成,次潼關,與固山額真恩格圖等迭戰破敵。二年,進克西安,自成走湖廣,與巴牙喇纛章京敦拜、阿爾津等追擊,殲敵騎三百。移師下江南,從貝勒博洛徇浙江,敗敵於杭州、於海寧、於平湖,得戰艦百餘。三年,復從徇福建,署梅勒額真。次延平,明唐王聿鍵走汀州,師從之,卓布泰別將兵攻克福州。敘功並考滿,進世職三等阿達哈哈番。

五年,從鄭親王濟爾哈朗下湖廣。六年,復署梅勒額真,與固山額真佟圖賴等自湘潭進克衡州。明將胡一清以步騎萬餘踞城南山岡,列七營,與佟圖賴合攻之,潰走;復進克道州、靖州。師還,優賚,授刑部侍郎。累擢內大臣、鑲黃旗滿洲固山額真,進世職二等阿思哈尼哈番。

十四年,授征南將軍,率師至廣西會湖南、四川兩軍規取雲、貴。十五年九月,師次獨山,與信郡王多尼及吳三桂會約師期,語詳洪承疇傳。卓布泰率兵自都勻進次盤江,明兵聞師至,沉舟,潛匿山谷中。卓布泰用土司岑繼魯言,渡下流取所沉舟,中夜濟師。明將李承爵以萬餘人屯涼水井,師進擊破之,攻雙河口山寨。明將李定國以象陣拒戰,擊潰之。定國悉眾為三十營,列柵固守,卓布泰分軍為三隊,張左右翼以進,再戰皆勝,追奔四十餘里,獲其象、馬。聞明兵尚堅守鐵索橋,乃自普安間道進羅平,會信郡王等軍攻克雲南省城,明桂王奔永昌。十六年二月,從貝勒尚善等進軍鎮南,破白文選於玉龍關,渡瀾滄江,取永昌,明桂王奔騰越,師復進,渡潞江。定國以六千人伏磨盤山,卓布泰分兵為八隊,以火器發其伏,掩擊,斬殺過半,遂克騰越。明桂王奔緬甸,卓布泰乘勝追擊,越南甸至猛卯而還。捷聞,賚蟒服、鞍馬。

康熙元年二月,師還,上命內大臣迎勞。尋追論在軍勘將士功罪不實,與議政王貝勒爭辨語怨望,論絞籍沒,上命寬之,奪世職,罷都統。三年,復世職。八年,復以弟鰲拜得罪,奪世職。十六年,再復世職。十七年,卒,謚武襄。

巴哈,卓布泰弟。事太宗,以一等侍衛授議政大臣。順治初,入關,從肅親王豪格征張獻忠有功,世職累進一等甲喇章京。睿親王討姜瓖,巴哈請從征,王勿許,拂衣起,坐論死,命罰鍰以贖。睿親王攝政,巴哈兄弟獨不附。肅親王卒於獄,子富綬尚幼,尚書宗室鞏阿岱議殺之,巴哈及內大臣哈什屯持不可,乃止。鞏阿岱因與弟錫翰及內大臣西訥布庫等欲構陷以罪,聞上嘉其勤勞,議乃寢。世祖親政,使證鞏阿岱等罪狀,皆坐誅。復命為議政大臣,世職累進一等阿思哈尼哈番,加少傅兼太子太傅,授領侍衛內大臣。鰲拜得罪,坐罷官奪世職。卒。

蘇勒達,巴哈子。事聖祖,授侍衛。累遷鑲黃旗蒙古都統、領侍衛內大臣。上親征噶爾丹,從行,贊議進擊,復扈上巡行塞北,賜內廝馬。卒,謚恪僖。

卓羅,滿洲正白旗人,巴篤理子也。卓羅襲三等副將,兼任牛錄額真。崇德三年,從伐明,薄明都,明太監楊永盛出戰,卓羅以三百人擊敗之,遂進略山東。四年,圍錦州,入其郛,獲守備一。六年,復圍錦州,擊敗明總督洪承疇。八年,授刑部參政。

順治初,從入關,破李自成,進世職一等梅勒章京,擢正白旗梅勒額真。三年,從大將軍順承郡王勒克德渾下湖廣,敗自成黨一隻虎於荊州。師還,賚黃金十兩、白金三百兩。是時明桂王由榔駐武岡,其將王進才等分守長沙、衡州、寶慶。大將軍恭順王孔有德等收湖南諸郡縣,命卓羅及梅勒額真藍拜率師益有德。四年,自岳州趨長沙,進才棄城走,卓羅等追擊敗之。遂與智順王尚可喜共擊敗明總兵徐松節,率舟師還長沙。遣甲喇額真張國柱、札蘇藍等以偏師擊敗明總兵楊國棟於天心湖。卓羅會有德下祁陽,道熊羆嶺,克其城。進攻武岡,擊敗明將劉承胤於夕陽橋,承胤降。明桂王走桂林,遂取武岡。五年,師還,上賚如自荊州還時。累擢吏部尚書,兼鑲白旗滿洲固山額真,進一等精奇尼哈番兼拖沙喇哈番。九年十一月,授靖南將軍,下廣東。旋以廣東垂定,罷。

十二年八月,命與固山額真阿爾津帥師屯荊州,時張獻忠將孫可望、李定國、白文選等降於明,屯辰州。十三年八月,卓羅與阿爾津道澧州、常德,下辰州,可望焚舟夜遁,卓羅與梅勒額真泰什哈、巴牙喇纛章京費雅思哈等率兵渡江攻之,遂克辰州。十四年,可望詣長沙降,定國、文選等從明桂王入雲南。

十五年,規取雲南,吳三桂自四川,征南將軍卓布泰自廣西,卓羅從信郡王多尼自湖南,三道並進。十六年正月,合攻雲南,克之,屢敗文選、定國兵,收永昌、騰越,追擊至南甸。命卓羅守雲南,賚蟒服、鞍馬。明桂王奔緬甸,定國屯孟艮,以印劄招元江土司那嵩。十月,卓羅與噶布什賢噶喇昂邦白爾赫圖等共擊之,克其城,那嵩自焚死。十八年,定西將軍愛星阿與三桂帥師入緬甸,卓羅仍守雲南。緬甸執明桂王詣軍,雲南平。康熙元年,召卓羅振旅還京,進二等伯。七年,卒,謚忠襄。乾隆間,定封號曰昭毅。

永慶,卓羅四世孫。乾隆間,以護軍參領降襲三等伯。旋擢副都統。從征準噶爾有功,加雲騎尉,仍進二等伯。出為烏魯木齊副都統。遷江寧將軍,移綏遠城將軍。召還,擢禮部尚書。罷,授內大臣。嘉慶十年七月,卒,謚敬僖。旋以在綏遠城嘗受賕,事露,奪謚。

愛星阿,滿洲正黃旗人,揚古利孫也。父塔瞻,襲封一等公,卒,愛星阿襲封。世祖念揚古利舊勞,命加給三等阿達哈哈番俸。順治八年,授領侍衛內大臣。

明桂王由榔與其將沐天波等奔緬甸,李定國居孟艮,白文選屯木邦,皆在雲南邊外。上命吳三桂鎮雲南,三桂疏請發兵入緬甸取由榔。十七年,授愛星阿定西將軍,與都統卓羅、果爾欽、遜塔,護軍統領畢力克圖、費雅思哈,前鋒統領白爾赫圖率禁旅會三桂南征。十八年,師行,聞世祖大行,三桂猶豫不進。愛星阿曰:「君命不可棄。」督兵先行,三日,三桂乃發。九月,師次大理,休兵秣馬。逾月,出騰越,道南甸、隴川、猛卯。十一月,至木邦,獲文選將馮國恩,訊知文選屯錫箔江濱,定國與不協,走景線。愛星阿令白爾赫圖等簡精銳,疾馳三百餘里至江濱,文選已毀橋走茶山。大軍至,結筏以濟,遣總兵馬寧、沈應時追之。愛星阿與三桂督師趨緬甸,時緬甸酋盡殺桂王從官天波以下數十人,密使人守之,謀擒以歸我師。十二月,師次舊晚坡,去其庭六十里,緬甸使詣軍前請遣兵薄城,當以桂王獻。愛星阿遣白爾赫圖將前鋒百人進,次蘭鳩江濱;復令畢力克圖等將護軍二百人繼其後,緬甸以舟載桂王及其孥並故從官妻女獻軍前。寧、應時追文選及於猛養,文選度不能脫,遂降。定國走死猛獵。捷聞,聖祖諭嘉獎,命以愛星阿所俘獲畀三桂區處,振旅還京師。加太保兼太子太保,敕書增紀軍功。

康熙三年二月,卒,謚敬康。

子富善,襲。授領侍衛內大臣。聖祖親征噶爾丹,富善將鑲紅旗兵扈上出中路,進次克魯倫河,閱選駝馬,徵輸芻粟,皆當上意。師還,加太子太保。卒。乾隆初,追謚恭懿。

遜塔,滿洲鑲藍旗人,安費揚古孫也。父碩爾輝。安費揚古既卒,太祖以所屬人戶分編牛錄,授碩爾輝牛錄額真。卒,遜塔嗣。太宗嘉其能,予世職牛錄章京。崇德三年,授戶部副理事官。是冬伐明,貝勒岳託將右翼自墻子嶺入邊,遜塔署甲喇額真,從噶布什賢噶喇依昂邦席特庫等擊破明總督吳阿衡軍,遂越明都,略山東。明年春,師出邊,明兵躡我後,遜塔從巴牙喇纛章京圖賴等奮戰卻之。明兵侵喀喇沁營,遜塔移兵赴援,明兵潰走。六年,圍錦州,明總督洪承疇赴援,屯松山,遜塔與甲喇額真藍拜率兵擊之,破三壘。明兵乘陰雨犯我師右翼,復與藍拜步戰卻敵。八年,授甲喇額真。

順治元年,從入關,破李自成,進世職三等甲喇章京。三年,從大將軍肅親王豪格西討張獻忠,道漢中,與固山額真巴哈納等擊破叛將賀珍,進次西充。獻忠率其徒拒戰,遜塔與固山額真李國翰等迭擊破之。五年,師還,兼任刑部理事官。命率師駐防淮安。六年,莒州土寇曹良臣破海州,知州張懋勛、州同李士麟死之。遜塔督兵赴援,良臣走保馬髻山,進擊破之。時設浙淮鹽務理事、兼戶部侍郎銜,上以命遜塔,駐揚州。七年,改督理漕運戶部侍郎,仍駐淮安。八年,官裁,遜塔還京,授鑲藍旗滿洲梅勒額真。遇恩詔,進世職三等阿思哈尼哈番。

十三年,授工部尚書。十五年,監修壇殿工成,進世職二等。尋兼授鑲藍旗蒙古固山額真。十七年,罷尚書,專任都統。旋命從定西大將軍愛星阿率師下雲南,明年十一月,會師木邦,趨緬甸,得明桂王以歸。敘功,進世職一等拖沙喇哈番。四年,調本旗滿洲都統。十二月,卒,謚忠襄。

子馬錫泰,襲世職,授佐領,兼前鋒參領。康熙間,從信郡王鄂札徵察哈爾布爾尼,師次達祿,布爾尼屯山岡,列火器拒戰,馬錫泰率前鋒薄險,四戰皆捷,進世職三等精奇尼哈番。又從討吳三桂,遷本旗滿洲副都統。自湖廣出廣西,下雲南,石門坎、黃草壩諸戰,皆在行間。進破雲南省城,逐賊楚雄,降三桂將馬寶、巴養元等。師還,進世職一等。卒,孫德彞,降襲一等阿思哈尼哈番。乾隆初,定封一等男。

都爾德,亦安費揚古孫。父阿爾岱,以牛錄額真事太宗,駐耀州,御明兵有功。從攻大凌河,戰死,贈世職備御,都爾德襲。順治初,授刑部理事官。從入關,擊李自成,署巴牙喇纛章京。從豫親王多鐸西征,戰陜州,督兵陟山拔其壘,復破敵潼關。尋自河南下江南,逐明福王由崧至蕪湖,截江而戰,大敗之。復從端重親王博洛定浙江,徇福建,偕巴牙喇纛章京阿濟格尼堪攻汀州,破明唐王聿鍵。復從鄭親王濟爾哈朗略湖廣,討李自成餘黨李錦等。師還,真除巴牙喇纛章京,授議政大臣,世職累進一等阿思哈尼哈番。康熙三年,卒,賜祭葬,謚忠襄。

論曰:順治初,取福、唐二王,不再期而定。桂王勢更蹙,以有闖、獻餘眾死寇力戰,支拄十餘年。陳泰定湖北,兵力至常、岳,阿爾津繼之,奄有湖南。李國翰略四川、貴州,卓布泰下雲南,卓羅從信郡王為之佐;愛星阿繼之,遜塔為之佐;與吳三桂合軍,深入緬甸取桂王:明宗至是始盡熸矣。


\end{pinyinscope}