\article{列傳十五}

\begin{pinyinscope}
額爾德尼噶蓋噶蓋子武善布善布善子誇扎達海尼堪

庫爾纏弟庫拜英俄爾岱滿達爾漢弟馬福塔明安達禮

額爾德尼,納喇氏,世居都英額。少明敏,兼通蒙古、漢文。太祖時來歸,隸正黃旗滿洲。從伐蒙古諸部,能因其土俗、語言、文字宣示意旨,招納降附。賜號「巴克什」。

滿洲初起時,猶用蒙古文字,兩國語言異,必移譯而成文,國人以為不便。太祖起兵之十六年,歲己亥二月辛亥朔,召巴克什額爾德尼、扎爾固齊噶蓋使制國書。額爾德尼、噶蓋辭以夙習蒙古文字,未易更制。上曰:「漢人誦漢文,未習漢字者皆知之;蒙古人誦蒙古文,未習蒙古字者皆知之。我國語必譯為蒙古語,始成文可誦;則未習蒙古語者,不能知也。奈何以我國語制字為難,而以習他國語為易耶?」額爾德尼、噶蓋請更制之法,上曰:「是不難。但以蒙古字協我國語音,聯屬為句,因文以見義可矣。」於是制國書,行於國中。滿洲有文字自此始。

躡我師後,額爾德尼偕?天命三年,從伐明,取撫順,師還,明總兵張承廕自廣寧率諸將還擊,斬承廕。?功,授副將。太宗時,額爾德尼已前卒,嘗諭文館諸臣,嘆為一代傑冠軍使。賜姓赫舍裏,改入大學士希福?出。順治十一年,追謚文成。子薩哈連,官至鑾儀族中。

噶蓋,伊爾根覺羅氏,世居呼納赫。後隸滿洲鑲黃旗。太祖以為扎爾固齊,位亞費英東。歲癸巳閏十一月,命與額亦都、安費揚古將千人攻訥殷佛多和山寨,斬其酋搜穩塞克什。歲戊戌正月,命與臺吉褚英、巴雅喇及費英東將千人伐安褚拉庫路,降屯寨二十餘。歲己亥,受命制國書。是年九月,命與費英東將二千人戍哈達。哈達貝勒孟格布祿貳於明,將執二將。二將以告,太祖遂滅哈達,以孟格布祿歸。孟格布祿有逆謀,噶蓋坐不覺察,並誅。子武善。

武善年十六,太祖念噶蓋舊勞,授牛錄額真。天命九年,明將毛文龍遣兵入海島屯耕。語詳冷格里傳。文龍復遣兵三百登海岸掠,武善與滿?,太祖命武善與冷格里擊之,殲其都里率兵追擊,斬裨將三,還所掠。太宗即位,列十六大臣,佐鑲紅旗。天聰八年,上遣諸將伐明,武善與阿山為後隊,遵上方略,設伏敗敵,授三等甲喇章京。崇德元年,詗知明兵襲濱海鹺場,上命武善與吏部參政吉恩哈馳援,擊走明兵。三年正月,喀爾喀扎薩克圖窺歸化城,上自將御之,武善與吳巴海從。吳巴海?卒盜軍糈,武善坐徇隱,奪世職。八月,授工部參政。時蒙古、瓦爾喀諸部皆附,使至,每以武善典其事。順治元年,卒。

布善,武善弟。事太宗,授巴牙喇甲喇章京,兼牛錄額真。尋護巴牙喇纛章京,列議政大臣。崇德五年,從伐明,攻錦州,擊敗杏山騎兵。六年,復從伐明,攻松山,洪承疇以而餉不繼,必引去,命諸將比翼列營,直抵?力戰卻敵。上度明兵?十三萬人赴援,布善先海濱。入夜,明兵果引去,諸將截擊,布善率兵窮追,斬獲無算。八年,復從伐明,攻克前、中前所。順治初,從入關,予牛錄章京世職。二年,從征江南,卒於軍。?屯

誇札,布善子,襲職。遇恩詔,進二等阿達哈哈番。十七年,授護軍參領,兼佐領。康熙十三年,從定南將軍希爾根討耿精忠,圍撫州,屢破賊,賊棄城走。四年,從大將軍安親王岳樂討吳三桂,其將夏國相屯萍鄉,依山結寨。誇札率兵奮擊,大破之,國相等棄資械走。十七年,遷護軍統領。十八年,擢鑲紅旗蒙古都統。從安親王攻武岡,軍器輜重自水道進,賊截溪,誇札率兵馳擊,賊卻走。綠旗兵屯溪岸,賊舟坌集逼屯,誇札自陸赴援,道險,?,馬不能行,乃率兵步行。賊據山梁,設鹿角,列火器以拒,誇札督兵直前,斬獲甚賊水陸皆潰。十九年,命將湖廣兵詣廣西,參贊大將軍簡親王喇布軍務,討叛將馬承廕,克武寧,進取象州,圍柳州,承廕降,進復慶遠。廣西平,還京。二十一年,卒。?功,進一等阿達哈哈番。

達海,先世居覺爾察,以地為氏。祖博洛,太祖時來歸。父艾密禪。旗制定,隸滿洲正藍旗。

達海幼慧,九歲即通滿、漢文義。弱冠,太祖召直左右,與明通使命,若蒙古、朝鮮聘問往還,皆使屬草;令於國中,有當兼用漢文者,皆使承命傳宣:悉稱太祖旨。旋命譯明會典及素書、三略。太宗始置文館,命分兩直:達海及剛林、蘇開、顧爾馬渾、托布戚譯漢字書籍;庫爾?、吳巴什、查素喀、胡球、詹霸記注國政。

天聰三年,上伐明,既擊破滿桂等四總兵軍,遣達海齎書與明議和,明閉關拒勿納;復命達海為書二通,一置德勝門外,一置安定門外,乃引師還。四年,復伐明,至沙河驛,命達海以漢語諭降。克永平,命達海持黃旗登城,以漢語諭軍民,城中望見,皆羅跪呼「萬歲」。降將孟喬芳、楊文魁、楊聲遠從貝勒阿巴泰入見,命達海以漢語慰勞。三屯營、漢兒莊既降,明兵襲三屯營。上慮漢兒莊復叛,命達海以漢語撫定之。是年,所譯書成,授游擊。五年七月,賜號「巴克什」。九月,復伐明,破大凌河,命達海以漢語招總兵祖大壽。上賜宴,復命傳諭慰勞。十二月,定朝儀。

達海治國書,補額爾德尼、噶蓋所未備,增為十二字頭。六年三月,太宗諭達海曰:「十二字頭無識別,上下字相同。幼學習之,尋常言語,猶易通曉;若人姓名及山川、土地,無文義可尋,必且舛誤。爾其審度字旁加圈點,使音義分明,俾讀者易曉。」達海承命尋繹,字旁加圈點。又以國書與漢字對音,補所未備,謂:「舊有十二字頭為正字,新補為外字,猶不能盡協,則以兩字合音為一字,較漢文翻切尤精當。」國書始大備。是年六月,達海病,逾月病亟。上聞,垂涕,遣侍臣往視,賜蟒緞,並諭當優恤其子。達海聞命感愴,已不能言,數日遂卒,年三十八。時方譯通鑒、六韜、孟子、三國志、大乘經,皆未竟。

無完者。七年二月,以其長子雅?達海廉謹,在文館久,為領袖。其卒也,當斂,求秦降一等襲職,授備御。國初文臣無世職,有之自達海始。十年,賜謚文成。康熙八年五月,聖祖從其孫禪布請,立碑紀績。

達海子四,長子雅秦,以備禦兼管佐領。崇德三年,從伐明,毀董家口邊墻入,略明畿內,下山東,所向克捷。還,出青山口,遇明軍,雅秦率步兵擊敗之。四年,從攻松山。六年,從圍錦州,城兵突出犯我軍,雅秦率所部兵禦敵,皆有功。旋授吏部理事官。八年,調戶部理事官。順治元年四月,從入關,擊敗李自成。迭遇恩詔,進世職至二等阿思哈尼哈番。八年三月,授吏部侍郎。七月,擢國史院大學士。十月,卒。九年,上以恩詔進世職過濫,命改為一等阿達哈哈番兼拖沙喇哈番。予其子禪布襲職。康熙二十一年,聖祖巡方,命從官祭雅秦墓。

達海次子辰德,太宗嘗召其兄弟,賜饌予幣,命辰德勤習漢文,其後仕未顯。

三子喇捫,康熙間,以前鋒統領從討吳三桂,戰衡州,陣沒,贈拖沙喇哈番。

四子常額,雅秦卒後,世祖特授學士,而雅秦子禪布,康熙初亦官秘書院學士,為達海請立碑。三桂既平之明年,聖祖諮諸大學士:「達海巴克什子孫有入仕者乎?」明珠對:「聞有孫為鴻臚寺官。」因下吏部錄達海諸孫陳布祿等十二人引見,命授陳布祿刑部郎中。其後國子監祭酒阿理瑚請以達海從祀孔子廟,禮部尚書韓菼議不可,乃罷。

達海以增定國書,滿洲?推為聖人。其子孫:男子系紫帶,亞於宗姓;女子不選秀女。

尼堪,納喇氏,世居松阿里烏喇。太祖時來歸,賜號「巴克什」。旗制定,隸滿洲鑲博爾晉等率師伐虎爾哈部,收五?白旗。初以說降蒙古科爾沁部,授備御。天命十年,偕侍百戶以還,上郊勞賜宴。

。從太宗伐明,攻錦州,有功。七年,從諸貝勒按獄蒙古諸部,?天聰初,擢一等侍牛錄額真阿什達爾漢以所齎敕二十道付尼堪,尼堪以授從者,失其九。所司論劾,罰如律。蒿齊忒部臺吉額林等來歸,命尼堪往迎。八年正月,收其部落戶口、牲畜以還。七月,上伐來歸,命尼堪還盛京安置。時鄭親王濟爾哈朗留守,使尼堪偕卦爾察、?明,道遇察哈爾部席特庫率兵十二人偵明兵。明兵適至,奮擊敗之,逐至遼河,凡三戰,斬馘百餘,明兵引退。九年,從貝勒岳託戍歸化城,土默特部私與明通,岳託使尼堪及參領阿爾津伺塞上,得明使四輩、土默特使十輩,皆執以歸。尋與英俄爾岱等使朝鮮。

崇德元年六月,授理籓院承政。二年正月,太宗伐朝鮮,既克其都,命尼堪及吉思哈、葉克舒帥師?護科爾沁、扎魯特、敖漢、柰曼諸部兵伐瓦爾喀,將出朝鮮境,朝鮮兵屯吉木海,阻師行,尼堪督兵進擊,大破之,斬平壤巡撫。既,朝鮮兵二萬餘人復來追襲,尼堪等設伏誘敵,殲萬餘人。敵遁,據山巔立柵拒守,師圍之三日,遂下。降哈忙城巡撫及總兵副使以下官,獲牲畜、布帛諸物無算。進略瓦爾喀部,以所獲畀蒙古諸部兵,尋引師還。復偕阿什達爾漢使科爾沁、巴林、扎魯特、喀喇沁諸部頒赦詔,會諸部王貝勒清庶獄。三年五月,坐讞獄科爾沁失實,解任。七月,授理籓院右參政。四年,伐明,徵蒙古諸部兵,兵至不如額,命尼堪使科爾沁、喀喇沁、土默特諸部詰責。五年四月,上以尼堪充副任使,授三等,編為八牛錄。七月,復命徵蒙古諸部?甲喇章京。復命安集索倫、郭爾羅斯兩部新附之兵伐索倫,簡其軍實。

世祖定鼎,論功,進二等。順治二年,從豫親王多鐸下河南,將蒙古兵自南陽趨歸德。四年,論功,進三?,降州一、縣四。論功,進一等。三年,從多鐸討蘇尼特部,大破其等阿思哈尼哈番。遷理籓院尚書。六年,喀爾喀使至,餽睿親王多爾袞馬,巽親王滿達海以為言,尼堪啟王,王曰:「如例云何?」尼堪曰:「外籓職貢,例不當餽諸王。」王惡其語侵己,令內大臣議罪,奪其俸。三遇恩詔,進三等精奇尼哈番,世襲。十年,上以尼堪老,進二等,致仕。十七年,卒。無子,以其弟阿穆爾圖、阿錫圖,從子瑪拉、兆資分襲世職。瑪拉自有傳。

庫爾?,鈕祜祿氏,世居長白山。祖曰賴盧渾,父曰索塔蘭。賴盧渾先為哈達都督,索塔蘭及所部來歸。旗制定,隸屬滿洲鑲紅旗。太祖以女妻索塔蘭,生子四,庫爾?其次子也。天命元年,召直左右。十一月,蒙古喀爾喀五部來議和,庫爾?齎書蒞盟;九年二月,復將命如科爾沁修好:皆稱旨,授牛錄章京。

太宗即位,伐扎魯特部,庫爾?從,師還,上勞諸貝勒。飲至,達海承旨問諸貝勒行軍勝敵始末,庫爾?為諸貝勒具對,成禮。天聰元年,伐朝鮮,庫爾?從,朝鮮王李倧請行成,庫爾?及副將劉興祚將命宣撫。倧既約降,庫爾?等還報,朝鮮諸將不知倧已約降也,以步騎兵千人邀諸平壤,庫爾?集從者環甲突圍出。朝鮮兵躡其後,庫爾?令從者前行,而以十騎殿,殺朝鮮兵三,疾馳六十里。朝鮮兵三百騎繼至,庫爾?率十騎憑?隘為伏,擊敗之,斬朝鮮將四、兵五十餘,獲馬百,卒達沈陽。上復命齎諭至軍中申軍令,定盟誓而還。

三年四月,定文館職守,命記注時政,備國史。四年正月,伐明,庫爾?偕游擊高鴻中先至灤州,設謀使啟城門,師遂入。二月,師還,庫爾?從諸將戍焉。五月,明監軍道張春等來攻,庫爾?與牛錄額真覺善等勒兵出戰,奮逾塹,直趣敵陣。春等稍卻,旋發火器焚城樓,壞睥睨,庫爾?與覺善還兵御之,敵不能登。都統圖爾格等以孤軍無援,退保永平,敵圍益急,庫爾?且守且戰,屢有斬馘。旋從貝勒阿敏等棄諸城,還都待罪。上以在灤州時能力戰,特貰之。

庫爾?先以口語被訐。五年十一月,使朝鮮,以漢文作書遺朝鮮,受私餽。六年六月,使明得勝堡議和,以其人來,上召入見,屢失期。七年二月,上發庫爾?諸罪,並追議庇劉興祚罪,論死。興祚者開原人,見辱開原道,遂率其諸弟興治等以降,太祖以國語名之曰愛塔。克遼東,授副將,領蓋、復、金三州。興祚婪,索民財畜,被訐解任,遂有叛志。事屢敗,太宗屢覆蓋之。興祚使其弟興賢逃歸毛文龍,作書遺庫爾?,詭言且死,託以營葬,誑瞽者醉而縊殺之,焚其室逸去。庫爾?得書,視興祚,見瞽者尸,以為興祚也,持之慟,告於上,以其子五十襲職,為營葬。既而其弟興治亦遁,詐漸露。興祚、興治去事文龍,文龍薦為參將。袁崇煥殺文龍,使興治及陳繼盛分將其兵。天聰四年,上攻永平,興祚在敵中,襲我軍中喀喇沁兵,殺數十人。使貝勒阿巴泰、濟爾哈朗將五百人求興祚。興祚將趨山海關,阿巴泰遮其前,濟爾哈朗迫其後,遂戰,甲喇額真圖魯什獲興祚,殺之,執興賢以歸。庫爾?解衣瘞興祚,上命發而磔之,庫爾?復竊收其遺骼。時興治將兵駐皮島,諸弟興基、興梁、興沛、興邦皆為偏裨。興沛以游擊守長山島,上遣使招興治等,諱言邏卒誤殺興祚;且令興賢附書述上恩,贍其母及妻。使屢返,復遣護其妻以往,興治亦屢答上書,自署「客國臣」,枝梧不得要領。會興治為興祚發喪,而繼盛信諜言,疑未死,興治忿,執殺繼盛,因縱掠。明使黃龍鎮皮島,興治復為亂,被殺。上亦殺興賢及其諸子。庫爾?與興祚善,未叛,屢為上言,終收其骨,卒以此及。上猶念其有勞,命毋籍其家。世祖定鼎燕京,詔視一品大臣例,予宅地、奴僕。

庫爾?弟庫拜,初以小校事太祖,從伐明,取撫順,戰敗追兵,復下遼、沈,命為牛錄額真。天聰五年,從伐瓦爾喀,手被創,猶力戰,克堡一。是年七月,初設六部,授吏部參政。?功,授牛錄章京世職。復以吏部考滿,授三等甲喇章京。八年,從伐黑龍江諸部。九年,進二等甲喇章京。崇德元年,從伐朝鮮。追論伐瓦爾喀時奪部卒俘,復令部卒私獵,論罰,罷牛錄章京。三年七月,更定官制,改吏部理事官。五年正月,卒。

英俄爾岱,他塔喇氏,世居扎庫木。太祖時,從其祖岱圖庫哈理來歸,授牛錄額真,隸滿洲正白旗。天命四年,從攻開原。有蒙古巴圖魯阿布爾者,素以驍勇名,降明為邊將,出戰,英俄爾岱馳斬之。六年,從克沈陽,授游擊。從克遼陽,授二等參將。

天聰三年,從伐明,克遵化,太宗督諸軍向明都,而令英俄爾岱及李思忠、範文程以兵八百守遵化。師既行,所下諸城堡石門驛、馬蘭峪、三屯營,大安口、羅文峪、漢兒莊、郭家峪、洪山口、潘家口、灤陽營皆復為明守。明兵夜薄遵化,英俄爾岱率兵擊卻之。平旦悉銳奮擊,明兵退,斬殿?,明將以騎兵列陣待,英俄爾岱出戰,明兵驟至,英俄爾岱麾其者五人,俘材官一,明兵宵奔。英俄爾岱以師從之,復殲騎卒百、步卒千餘,以書諭諸城,羅文峪、三屯營、洪山口、漢兒莊、灤陽營五城復降。

五年七月,定官制,始設六部,以英俄爾岱為戶部承政。七年,明故毛文龍部將孔有德、耿仲明自登州來降,使英俄爾岱及游擊羅奇齎書徵糧於朝鮮,朝鮮國王李倧使其臣樸祿報聘,言毛氏舊為敵,不原輸糧。太宗復以書諭,略言:「毛氏將今歸我國,以兵守其舟,當就便輸以糧。」遣英俄爾岱及備御代松阿齎書復往,朝鮮乃輸糧如指。八年五月,改進一等甲喇章京。

太宗自將伐察哈爾,察哈爾林丹汗走圖白特,所部潰散。或得俘,言同行凡千餘戶,方苦無所歸,上命英俄爾岱及梅勒額真覺羅布爾吉將二千人往跡之。英俄爾岱等行遇蒙古頭人侯痕巴圖魯率千戶將來歸,遣使謁上;復遇臺吉布顏圖,縱兵擊殺之,斬二百餘,俘四十奔愬於上,言:「我曹自察?以還。上以駝馬及所俘,賚英俄爾岱及諸將士。既,布顏圖部哈爾來歸,遇大軍,乞降不見允,橫被屠戮。」上怒,命盡奪所賚。英俄爾岱尋以考滿進三等梅勒章京。

十年春,諸貝勒及蒙古諸部以太宗功德日隆,議上尊號,令英俄爾岱齎書使朝鮮喻意。既至,倧謝不延納,令英俄爾岱詣所置議政府陳說,設兵晝夜環守使邸。英俄爾岱率諸從者奪民間馬,突門而出。朝鮮王遣騎持報書追付英俄爾岱而別,以書誡其邊臣令守界,英俄爾岱並奪之以聞。又遇明皮島兵遮歸路,擊走之。

崇德改元,討朝鮮,師克王都,倧出奔南漢城。二年春,上使英俄爾岱及馬福塔齎敕詰責,朝鮮以書謝。師益進,薄南漢城,復使英俄爾岱、馬福塔招倧出城相見,倧答書始稱臣,然猶逡巡不敢出。上詗知倧寄孥江華島,命睿親王多爾袞以偏師下之,獲其妃及諸子。倧乃出降,上留其二子為質,命英俄爾岱、馬福塔送其妃及諸戚屬還王都。二月,班師,倧出送,命英俄爾岱、馬福塔宣諭,仍送之還。旋授議政大臣。十月,復命英俄爾岱、馬福塔齎敕印使朝鮮,封倧仍為朝鮮國王。四年,授固山額真。五年,上以倧繕城郭,積芻糧,欺罔巧飾,使英俄爾岱及鄂莫克圖齎敕詰責,倧上表謝罪。

六年六月,睿親王多爾袞復攻錦州,九月,貝勒多鐸等圍松山,英俄爾岱皆在行間。七年,復使朝鮮鞫獄,還奏稱旨。八年,考滿,進三等精奇尼哈番。順治元年,從睿親王多爾袞入關。是年,改承政為尚書,英俄爾岱仍任戶部。二年,?功,封三等公。三年,奏請、軍械、火器,以杜盜源,從之。四年,考滿,進二等公。五年二月,卒。?禁民間私售馬

英俄爾岱娶饒餘郡王阿巴泰女,授多羅額駙。領戶部十餘年,既領固山,仍綜部政。屢坐事論罰,而恩顧不稍衰。太宗嘗諭?臣曰:「英俄爾岱性素執拗,其於本旗人亦偶有徇庇。朕思人鮮有令德,英俄爾岱能殫心部政,治事明決,朕甚嘉之。視諸部大臣不如英俄爾岱者多矣!」及睿親王薨,得罪,奪英俄爾岱公爵,降精奇尼哈番。康熙間,輔臣鰲拜專政,陷大學士蘇納海等於死,以英俄爾岱與蘇納海同族,追論初授地不平、附睿親王諸罪狀,奪官。子宜圖,官至內大臣,襲爵降三等精奇尼哈番。乾隆初,定封三等子。

滿達爾漢,納喇氏,先世居哈達。父雅虎,率十八戶歸太祖,太祖以為牛錄額真,隸滿洲正黃旗。擢扎爾固齊。與哈穆達尼伐東海卦爾察部,俘二千人以歸,太祖郊勞,與宴。又克舒桑哈達,賜俘百。既乞休,滿達爾漢繼為牛錄額真。從太宗伐虎爾哈部,降五百餘戶。

天聰五年五月,上將伐明,規取海中諸島,使滿達爾漢與董納密聘於朝鮮,且徵舟焉。時朝鮮初附,未敢開罪於明,滿達爾漢等至朝鮮,國王李倧謝不見,且以兵守館。越三日,滿達爾漢謂守者曰:「我奉命至此,何慢我不相見?我歸矣!」遂與諸從者佩弓矢,策騎奪門出。倧使侍臣追及,請見,滿達爾漢等乃入見,致使命而還。七月,授禮部參政。閏十一月,復與庫爾?等使朝鮮,誡毋縱其民越境採獵,毋匿逃人,並令歲餽當如例,倧乃引咎,原如約。

八年,太宗自將伐明,攻大同,滿達爾漢分兵克堡四、臺一,又拔王家莊。以功,授世職牛錄章京。尋擢禮部承政。復使朝鮮。崇德二年,從武英郡王阿濟格伐明,克皮島,賜白金、裘、馬。順治初,世祖定鼎京師,滿達爾漢以老解部任,專領牛錄。恩詔,進二等甲喇章京。三年,卒,謚敬敏。子阿哈丹,襲職。恩詔,進一等阿達哈哈番兼拖沙喇哈番。從征福建,擊鄭成功?門,戰死,恤贈三等阿思哈尼哈番。

馬福塔,滿達爾漢弟也。初授牛錄額真,與滿達爾漢分轄所屬人戶。天聰五年,授戶部參政。八年三月,與戶部承政英俄爾岱如朝鮮互市。五月,太宗自將伐明,馬福塔從貝勒濟爾哈朗等居守。九月,齎奏詣行營,道明鐵山,明兵邀戰,斬五人,俘一人;又刵一人,縱使還。尋擢戶部承政。九年,與參政博爾惠使朝鮮。自是通使朝鮮,馬福塔輒與。

崇德元年,復與英俄爾岱等使朝鮮,明皮島兵遮道,擊走之。九月,復如朝鮮義州監互市,得明邏卒,知明兵入??,因率百人躡其後,明兵引去。值武英郡王阿濟格等伐明還,渡遼,具舟以濟師。十二月,太宗自將伐朝鮮,命馬福塔與勞薩率兵先驅。語詳勞薩傳。朝鮮國王李倧走保南漢山城,二年正月,師克朝鮮都,進攻南漢山城。馬福塔兩奉敕入城數倧罪,且諭降。倧先使其臣謝罪,尋率?僚出城謁上。二月,上班師,倧出送,命馬福塔與英俄爾岱送倧還城。倧餽金,?之,以聞。四月,從武英郡王阿濟格攻明皮島,馬福塔攻其齎還,又令朝鮮?北隅,督戰敗敵。六月,吏議馬福塔從伐朝鮮,私以其子往,得俘獲,先將與貝子碩託交結,罪當死,命罰鍰以贖。十月,復命與英俄爾岱使朝鮮,冊李倧為朝鮮國王。三年七月,更定官制,改戶部左參政。四年六月,命與刑部參政巴哈納使朝鮮,冊倧妃趙氏為王妃。八月,其兄甲喇額真福爾丹從軍退縮,伏法,籍其家畀馬福塔。九月,復為戶部承政。十一月,倧疏言立碑三田渡頌上恩,命與禮部參政超哈爾等往察視。五年二月,卒。

明安達禮,西魯特氏,蒙古正白旗人,世居科爾沁。父博博圖,率七十餘戶歸太祖,即授牛錄額真,領所屬。天聰元年,從伐明,攻錦州,戰死,予世職游擊,以明安達禮襲,仍兼領牛錄額真。

崇德三年,遷巴牙喇甲喇章京。從貝勒岳託伐明,自密雲東北毀邊墻以入,與固山額真伊拜共擊敗明太監馮永盛兵,克南和縣。六年,復從伐明,圍錦州。明兵陣山巔,明安達禮率所部巴牙喇兵陷陣,明兵敗走。既,又有騎兵自松山至,復擊敗之。師阻壕,以守城兵出爭橋,明安達禮迫明兵使引入城。上自將擊洪承疇,明安達禮戰尤力,又敗敵騎,進二等甲喇章京。七年冬,從貝勒阿巴泰伐明,攻薊州,薄明都,擊破明總督趙光抃。又與噶布什賢噶喇依昂邦阿山共擊明兵自三河至者,遂進略山東。八年春,與明總兵白廣恩、張登科、和應薦等戰螺山,又與巴牙喇纛章京鰲拜共擊明總督範志完,屢破敵。師還,賚白金。擢禮部參政,兼正白旗蒙古梅勒額真。

順治元年,從入關,擊李自成。二年,從英親王阿濟格西討,戰延安,七遇皆捷。撫鳳翔等府三十餘城,悉下。三年,調兵部侍郎。蘇尼特騰機思叛,從豫親王多鐸帥師討之,別將兵屯險要。騰機思遁走,明安達禮夜帥師乘之,及諸鄂特克山,戰大勝,斬臺吉茂海,復與鎮國將軍瓦克達等逐北,手斬十一人,獲其輜重。復擊敗土謝圖汗、碩類汗。

五年,擢正白旗蒙古都統。七年,授兵部尚書,九年,列議政大臣。論功,遇恩詔,累進二等精奇尼哈番。十年,坐徇總兵任珍擅殺,罷尚書,降一等阿達哈哈番兼拖沙喇哈番。十一年,帥師伐鄂羅斯,敗敵黑龍江。十三年,授理籓院尚書。

十五年十二月,命為安南將軍,帥師駐防荊州。十六年,鄭成功入攻江寧,明安達禮帥師赴援。成功將楊文英等以舟千餘泊三山峽,明安達禮擊之,斬副將一,獲其舟及諸攻具,成功引入海。上命明安達禮移師駐防舟山。十七年,召還,授兵部尚書。康熙三年,加太子太保。六年,調吏部尚書。引疾,致仕。八年,卒,謚敏果。

子都克,襲。從征噶爾丹有功,授拖沙喇哈番,合為三等阿思哈尼哈番。都克孫永安,降襲一等阿達哈哈番兼拖沙喇哈番。乾隆間,從征甘肅石峰堡亂回。官至山海關副都統。永安孫憲德,憲德子夢麟,自有傳。

論曰:國必有所與立,文字其一也。因蒙古字而制國書,額爾德尼、噶蓋創之,達海成之。尼堪等皆兼通蒙、漢文字,出當專對。造邦之始,撫綏之用廣矣。英俄爾岱領戶部,調兵食最久,見褒於太宗。明安達禮以折沖禦侮之才,屢長兵部。蓋皆有功於創業者,故比而次之。


\end{pinyinscope}