\article{列傳二百七十五}

\begin{pinyinscope}
忠義二

硃國治楊應鶚馬弘儒等周岱生楊三知孫世譽翟世琪

劉嘉猷高天爵李成功張善繼等嵇永仁王龍光等

葉有挺蕭震等戴璣劉欽鄰崔成嵐黃新德

柯永升隨光啟等道禪李茂吉

劉?馬秉倫劉鎮寶羅鳴序

硃國治,漢軍正黃旗人。順治四年,由貢生授固安知縣,屢擢至大理寺卿。十六年,外簡江寧巡撫。時鄭成功盤踞外洋,出沒江南濱海州縣,國治疏言:「欲破狡謀,先度形勢。賊眾負險,我師遠涉風濤,其勞逸不同。賊眾熟識海道,我師弓馬便捷,其素習不同;水師舟楫,較之賊船大小懸殊,其攻取不同。臣謂宜以守寓戰,凡海邊江口,多設墩臺,待賊勢困援絕,乘間攻之,自能擒渠獻馘。」下所司議行。又以蘇、松、常、鎮四府錢糧抗欠者多,分別造冊,紳士一萬三千五百餘人,衙役二百四十人,請敕部察議。部議現任官降二級調用,衿士褫革,衙役照贓治罪有差。以是頗有刻覈名。

康熙十年,補雲南巡撫。時吳三桂謀叛久矣,十二年,詭請移籓錦州,並期以十一月二十四日啟行。國治方請增設驛堡,協撥夫馬待之,三桂遽踞關隘起事。先期三日,邀國治及按察使李興元、雲南知府高顯辰、同知劉昆,脅之從逆,皆不屈。國治罵賊尤烈,即時遇害。先後殉難者,雲貴總督甘文焜,廣西巡撫馬雄鎮、傅弘烈及李興元,均自有傳。

楊應鶚,鑲黃旗漢軍。任貴州貴陽府同知。吳三桂初叛,執應鶚送雲南,錮之順寧。應鶚密謀舉義,偽將軍趙永寧縊死之。時有馬弘儒者,順治十八年武進士。三桂素重之,迫共叛,不聽,以鐵椎椎其齒,齒盡落,囚昆明,不屈死。

先後在滇殉節者,為鄭相等:相,江寧人,以中書隨軍至雲南。大兵入滇,檄權石屏州事,有惠政。偽總兵高應鳳作亂,相死之。昆明人楊樹烈,以四川南川縣知縣告休家居,寇至,北面再拜,自縊死。土酋龍韜等分道剽掠,寧州知州曹誠嬰城守,城陷死之。原任曲靖府教授周起元,則以被執不屈死。生員有唐方齡、張鵾羽。

三桂既叛,所在蠢應,死貴州者:陳上年,直隸清苑人。順治六年進士。時官分巡右江參議道,三桂既執巡撫傅弘烈,乃脅上年降,幽縶死。都勻縣防兵謀應賊,譁譟焚掠,知縣薛佩玉諭以順逆,眾不聽,偪受偽職。佩玉北面再拜,自縊死。

死湖廣者:祝昌,河南固始人。順治六年進士,由中書累擢至辰沅道。三桂叛聞,即流涕諭眾大義,皆感泣。賊大至,城潰,北面再拜自縊死。生員有李廷員、張一元、徐翹楚。

死江西者:為饒州府知府郭萬國,萬年縣知縣王萬鎰。萬國,河南許州人。由諸生從經略洪承疇官貴州,撫苗、蠻有功。萬鎰,浙江錢塘人。由貢生官福建,平土賊有功。賊黨圍廣信急,覘饒州備虛,由間道薄城。萬國令萬鎰赴省請援,甫出北門,賊猝至,與家僮六人中砲歿。饒州營參將趙登舉聞警馳救,沖賊營,擒前隊數人,伏起,陣亡。賊黨環城招降,萬國集其屬同知範之英、鄱陽縣知縣陸之蕃、石門巡檢翁鳳翥、饒州稅課大使李崇道,謂之曰:「文臣不習戰,然守土吏當死,不可徒手就戮。」皆應之。賊偪靈芝門,攀堞登,萬國率家丁巷戰,身先之,中十六創,與之蕃、鳳翥、崇道俱戰歿。文英亦被執不屈死。萍鄉民人彭程淑,亦以三桂餘黨擾其鄉,裂眥怒罵,被亂刃死。

死廣東者:金世爵,鑲藍旗漢軍。由舉人任合浦縣知縣。高州總兵祖澤清叛附三桂,世爵圖城守。偽將王弘勛率賊數萬犯廉州,世爵登陴力禦,城陷,與守備杜嶠同死之。又侯進學者,隸平南王尚可喜籓下。先為三桂所脅,為遞逆書,至廣州自首,可喜以聞,嘉之,授世職。至是為賊所得,囚木籠送常德,三桂臠之於市。逆黨馬雄攻新會,籓下諸將多附逆,誘左翼游擊文天壽同降,天壽叱之曰:「背主不忠!吾錚錚丈夫,豈鼠輩可脅?」遂被害,沉尸海中。

死川、陜者:波羅營副將張國彥,聞提督王輔臣叛,城守。兵變,逼獻印,自刎死。漢中城陷,同知汪化鼇不受偽職,賊縶之,復給偽劄令攝縣事。化鼇痛哭,望闕遙拜,自縊死。漢鳳參將蘇興亦叛附三桂,將襲殺詗賊筆帖式布格爾以滅口。千總魯仁圻憤甚,度無以制之,朝衣拜父像告訣,叩營力爭,觸興怒,殺之。仁圻畜一犬,護尸不去,故吏梁玉收而葬之。又廣安州知州徐盛、劍州知州向榮、商南縣知縣盧英、渠縣知縣王質、綦江縣知縣王無荒、營山縣知縣廖世正及典史劉廷臣、西安府知事張文選、司獄周勝驤、白水縣典史趙煥文,並以被脅不屈死。

其以招諭死者:三桂未叛時,主事辛柱、筆帖式薩爾圖、隨侍郎哲爾肯★詔至滇。既叛,辛柱、薩爾圖將詣闕告變,賊殺之。後則漢川巡檢章啟周,浙江會稽人。從順承郡王勒爾錦軍,以劄委通判往招諭三桂,被戕。及吳世璠時,又遣四品銜董重民往諭以順逆,至鎮遠,逆黨以弓弦縊殺之。又揚威大將軍簡親王喇布檄益陽縣知縣徐碭往衡州招撫三桂餘黨,至泉溪渡,為偽將軍吳國貴所殺。鄖陽降調通判許文耀、阿迷州吏目郭維賢,亦均以招撫三桂餘黨遇害。郎中祝表正隨經略大學士莫洛討叛鎮王輔臣,莫洛戰歿,輔臣幽表正於營,尋復具疏附表正還奏。聖祖即遣表正諭輔臣,至,則百方曉譬,留弗遣,卒為偽總兵巴三綱所殺。甘肅靜寧州知州王札亦以單騎諭輔臣禍福,被脅不屈,死。

又撫叛鎮王屏籓死者,為四川鎮標副將徐升耀,劄付通判王官表、沈日章,劄付參將吳子騄等。

周岱生,字青嶽,江西德化人。由拔貢生除貴州餘慶縣知縣,改廣西平南縣知縣。康熙十三年,吳三桂叛,六月,其黨破梧州,攻平南,岱生練鄉團俍兵拒之大峽口,鏖戰三日,斬其魁。七月,復大至,岱生奮身拒戰,攻益急,鄉團皆戰死,退保城。圍固援絕,自寅戰至午,城陷。賊縶岱生往潯州,脅降,罵不絕口。妻楊氏,於路先自剄死。旋又甘言誘岱生降,卒遇害。長子儒且哭且罵,死尤慘。

岱生令餘慶時,有老賊陳四者,盤踞大同山垂三十年,剿捕不能得,出奇計招之,親至其巢,曉譬利害,曰:「王師且至,吾生汝!」賊感泣,誓終其身無反。岱生曰:「盍隨我至縣城乎?」賊諾之。於是至縣署,賜之食,厚為之裝而遣之。其後吳逆之變,他縣賊皆響應,惟陳四不受偽職。平南市荒民少,岱生捐俸招集,始至,城內草屋九間。未幾,商民大集。俗窳不產蔬菜,岱生教以播種灌溉之方,畦畝鱗次相屬。田皆老荒弗闢,又招粵東流民後先千餘家,報墾升科,其他善政尤多。

楊三知,字知斯,直隸良鄉人。順治三年進士,授山西榆次縣知縣。榆次經流賊殘破後,井里蕭條,三知以恩義安輯,戶口日增。康熙五年,大同鎮總兵姜瓖叛,連陷州縣,攻榆次。三知勵吏民,募鄉勇守城。夜遣人斫賊營,間有斬獲,賊不退。三知令偃旗鼓,示弱。賊徑薄城,攀堞欲登,三知急起,麾眾發矢石,斃甚眾。賊憤,益兵圍之。相持逾六月,敬謹親王尼堪分兵來援,賊始敗走。三知設保甲、練屯聚,復捐俸、立社學,置膳田以資膏火,士民感之。擢兵部主事,累遷郎中,外擢四川松龍道、上東道。上東道屬經張獻忠慘戮,存者在絕峒密箐中,招徠千數百家,築堡渝東,民名之曰楊公堡。

十一年,補陜西神木道。十三年,入覲,還至保德,聞提督王輔臣叛附吳三桂,從者勸遲行,勿渡河,不聽,疾馳還署,圖城守。曩三桂剿闖賊殘孽,過神木,市恩,民謬德之,立生祠,三知即毀之。察知縣孫世譽忠實可倚,時輔臣播偽劄,將弁多為所誘,分據城堡,惟韓城知縣翟世琪與神木通聲援。

世譽,隸鑲紅旗;世琪,山東益都人,順治十六年進士:關中並稱賢令者也。叛黨硃龍犯神木,民恟懼。三知適受檄赴京師代賀,有諷可攜眷行者,謝之。赴闕事竣,抵署三日,延安、吳堡相繼陷。賊至,乘城死守,親挽強弩,發無不中。柳溝營游擊李師膺受偽劄,鼓眾譟餉,世琪出諭賊,先被戕,及其二子。神木守將孫崇雅亦通賊,城遂陷。崇雅合賊將環說三知,以延綏開府啗之,不應;脅交敕印,不與;賊迭以甘言誘三知,且擁回署,三知過井,厲聲諭家人勿作兒女態,躍而入,賊遽縋出之,臂已折。力以三知「在官廉平,初未相迫,毋自苦」為詞,三知大罵不絕口,乃舁置別室,環守之,載脅載誘。一夕,忽合扉,不知何以賊之。其妻妾及二女俱赴井以殉。城復後,家人始於淺土中獲三知遺骸,經長夏,面色如生。世譽亦抗節不屈,賊羈之深室,輔臣後降,卒害世譽以滅口。

劉嘉猷,字憲明,江西金溪人。由明舉人順治初署興國、新建教諭,以正誼明道為教,士多化之。秩滿,改福建侯官縣知縣,為閩浙總督範承謨所賞。撤籓命下,嘉猷度平南王耿精忠必應吳三桂叛,謂家人以「既宰茲土,義不汙賊」。康熙十三年三月,精忠紿文武赴籓府計事,嘉猷從承謨後。見鋒刃交戟脅承謨降,不屈,縛以去,嘉猷歷階而上,厲聲叱精忠,福州府知府王之儀、建寧府同知喻三畏同發憤罵賊。精忠喝武士殺三人,眾股慄。嘉猷戟手作搏擊勢,芒刃亟下,與之儀、三畏同時被害。城守千總廖有功見逆殺三人,發憤大呼,亦死之。

高天爵,字君寵,漢軍鑲白旗人,後改隸鑲黃旗。由廕生於順治四年任山東高苑縣知縣,累官至江西建昌府知府。先是廣昌山賊踞羊石、滴水二砦為巢穴,官軍仰攻,輒為滾木礧石所傷,罷攻,招降,賊佯就撫,仍伺隙煽亂。官軍斃之獄,餘賊益負固。適風雨交作,漂流樹木,沖斷橋梁,賊保巢不出。天爵會巡道參將出不意直搗之,擒斬,盡毀其巢。

耿精忠據閩叛,縱黨入江西,犯建昌,時天爵已擢兩淮鹽運使,或勸之速行,天爵以「守此土十六年,雖受代,不可遽離」答之,率家丁數十人禦賊萬年橋。城守副將趙印已降賊,乘天爵力戰,從後縛之,獻賊,載送入閩,再四誘降,不屈,囚之。越歲餘,與副將王進,武舉胡守謙,把總楊起鵬、姜山等同謀,遣千總徐得功出仙霞嶺迎大軍入關,陰結死士為內應。賊黨偵而訐之,十五年九月四日遇害。

後以福建巡撫卞永譽請,以天爵與原任福寧總兵吳萬福、福州府知府王之儀、邵武府知府張瑞午、建寧府同知喻三畏、邵武府同知高舉、侯官縣知縣劉嘉猷、尤溪縣知縣李員、福州城守千總廖有功等合建一祠於省城西門外,復以子其佩請扁,書「藎忱義烈」四字以額其家祠。長子其位自有傳。

李成功,奉天鐵嶺人。順治六年武進士。歷官至廣東潮州參將。康熙十三年,總兵劉進忠應耿精忠叛。成功潛與游擊張善繼等謀誅進忠,事覺,進忠以兵脅同叛,曰:「汝為我中軍,我視汝猶子,何無義至此?」成功曰:「祿山叛國,死於豬兒;硃泚叛國,死於韓旻;汝今叛國,不知死之將至!我何為從汝?」進忠命斬之,罵不絕口而死。

善繼,直隸彭城衛人。習儒,通孫吳兵法。康熙六年第二名武進士,授潮州城守營游擊。進忠陰遣腹弁赴精忠獻款,弁歸,與進忠謀曰:「善繼剛方固執,深得眾心,宜亟散其卒。」進忠遂令所部分隸私黨。善繼麾下虛無人,謁進忠曰:「公不聞晉王敦乎?威勢未嘗不赫也,兵敗身死,發瘞斬尸,未有叛國而克全終者!」進忠怒,羈之馬王廟,貢生林應璧同被羈,日談古忠孝事。進忠屢遣人諭降,終不屈,令斬之。

白虎,陜西秦州人。康熙十一年,官澄海協右營都司,有「虎將」名。進忠將叛,調虎與其子崇質入郡。至,則知進忠有異志,★M4焉涕下。進忠令虎易帽,虎曰:「頭可斷,帽不可易!」令翦辮,虎曰:「頸可截,辮不可翦!」且責進忠,詞甚厲。左右以搖惑軍心,慫進忠斃之。進忠愛其勇,不忍,曰:「此愚人,不識時務耳!」遂羈之。篡取虎妻張、虎孫士俊為質。虎與同志密遣人赴省請兵,約內應。謀洩,將就刃,謂崇質曰:「死,吾分也!委身存祀則在汝。」崇質對曰:「父為忠臣,子從叛賊,烏乎可?」縛至西市,虎望北叩首,大言曰:「君臣大義盡於此,父子至情,亦盡於此矣!」觀者皆泣下。

何亮,潮州人。官澄海協千總。虎以心腹待之,亮隨虎赴郡,進忠羈虎,旋以內應事洩,並將斬亮。進忠叱之,亮謂當訴於天,同時遇害。其兄弟妻子被殺者尤眾。

於國璉,奉天人。為續順公沈瑞旗員。進忠亂作,瑞命偕都統宋文科、鄧光明攻之。戰太平街,三日,國璉身先士卒,射傷進忠左臂,賊披靡,以眾寡不敵,終為所敗。瑞縛光明及國璉以降,國璉獨不屈,斬於市,尸殭立不僕,數日面如生,眾咸異之。

嵇永仁,字留山,江南無錫人。用長洲籍入學為諸生。入閩浙總督範承謨幕。耿精忠應吳三桂叛,執承謨,脅永仁與同幕王龍光、沈天成及承謨族弟承譜降,不從,被執。永仁少好從士大夫游,討論國家典故,六曹章奏,條分件系,著有集政備考一書。以範、嵇世交,故相從至閩。時精忠蓄謀未發,屢陳弭變策,如請撥協餉、補綠旗兵、安插逃弁、條議屯田諸端,冀固民心、殺賊勢。又請借巡視沿海為名,提輕兵駐上游制賊。以文武吏皆預中賊餌,號令格不行。在獄凡三年,賊害承謨,乃痛哭自經死。永仁知醫,著有東田醫補。工詩詞,有竹林集、葭林堂詩。獄中又著詩二卷、文一卷。與龍光相倡和者,又有百苦吟。

龍光,字幼譽,浙江會稽人。諸生,屢躓鄉闈。年五十餘,已倦游,承謨撫浙,延課其子。擢閩督,龍光以父老不欲行,父以承謨有德於浙,義不可辭,遂往。既被執,脅草安民檄,誘以官爵,皆不從。與永仁誼最合,嘗語龍光曰:「死之日,魂魄原無相離!」在獄著養花說及雜詩五十餘首以見志。

天成,字上章,江南華亭人。變作時,與永仁約同死。偶外出,俄傳同難諸子死訊,遂出踐宿諾,為逆黨縛獻。時鞫者方窮究章奏,將歸罪永仁,天成厲聲辯曰:「承謨心事,青天白日,承謨無他志,書生更何與焉?」乃同系獄。著詩一卷,曰聽鵑。又纂花譜一卷以自遣。三人在獄,有書名和淚譜者,龍光為永仁撰一首,永仁為龍光、天成撰各一首,詩詞皆燒桴煤畫墻上,賴義士林可棟者,或云泰寧人許鼎,時往獄中探視,默識之,得以傳世。

承謨初被難,部曲有張福建者,手雙刀,大呼奪門,衛承謨,群攢刃死之。精忠令三十二人監守承謨,中有蒙古人嘛尼,欲免承謨,事洩,被磔。

葉有挺,字貞夫,福建壽寧人。康熙九年進士,甫釋褐,即徒步南歸。耿精忠以閩叛,檄郡邑,凡在籍搢紳悉坐名,勒限起送,有挺恥之,潛入江西界,佯言已死。逾年,以念母潛返,偽縣令偵知之,持檄促赴召。有挺告母曰:「兒得進士,思有以報君父。今以進士被偽檄,是得一進士反為從逆之資。兒死不赴,如母何?」母以大義勉之,乃抱母大號,遁匿山寺。僧知其為葉進士也,微拒之,有挺仰天嘆曰:「有挺豈以儒者七尺軀茍延旦夕,為釋氏恐怖?又豈以身死蕭寺,貽主僧禍?」夜起,北向九叩,南望母再叩,出走山下,自經古木死。亂平,無以上聞者,故褒贈皆不及。

同時閩中殉難者:蕭震,侯官人。順治九年進士,任山西道監察御史,丁父艱,回籍。精忠叛,謀討之,事洩,遇害。張松齡,莆田人。順治十二年進士,由庶吉士屢遷四川參議。時川省彫敝,松齡加意撫綏,流亡漸復。裁缺歸里,耿逆迫以偽職,羈數月,終不屈死。施大晁,福清人。康熙十二年進士。聞變,匿金芝山,募壯士,助大兵進討,賊執之,嚼舌罵賊,嘔血數升死。莆田舉人劉渭龍、建寧舉人謝邦協、南平舉人原任丹徒令鄒儀周,皆不受偽職。渭龍匿深山絕粒死。邦協舉家避村落中,逆黨以火攻之,不出,闔門遇害。儀周為所執,不屈死。光澤縣民毛錦生,素有力,賊躪其村,邑當事飭為練總,導大兵進剿,遇伏,死雲際關。清流縣諸生李亭,隨邑令守城,並集鄉勇拒戰,旋被執,詈賊死。

又有張存者,順昌人。精忠亂作,存糾義旅保元坑鄉,脅授總兵劄,令率眾出江西,分大軍兵勢,存不從。時和碩安親王岳樂駐師南昌,存潛使赴軍前乞援,並條上攻賊機宜。岳樂授存總兵劄,令捍禦建昌、邵武、汀州等地,且為內應。賊偵知之,急攻元坑。地平,無險可扼,存以忠義激眾,屢敗賊,賊恚甚,分三路夾攻,卒以不支,存被執,死之。

戴璣,字利衡,福建長泰人。順治六年進士,授主事,例轉湖廣按察司僉事。時滇、黔未入版圖,上江防道尤要。璣遍履所部,自岳州至嘉魚,立七汛,造哨船巡邏,萑苻無警。又於洞庭湖接立三汛,行者尤便之。洪承疇正經略五省,以「韓、範儔」稱之。尋遷陜西西寧道,未行,丁父艱。服除,補廣西右江道,駐柳州。東闌土酋構禍日久,璣以恩意調解之。大酋黃應元煽亂,則斬渠魁以徇。諸蠻用是懷德畏威,頑梗盡化。柳堡屯田,寄佃於民,既輸軍租,復應民役,為申請督撫,具奏獲免。復修葺文廟及羅池司戶二賢祠。會朝命裁人並監司,解任歸里,督課諸子,教以忠孝大義。

耿精忠亂作,臺灣賊圍漳州,時璣次子鏻為海澄公將,守東門。賊劫至城下,使招鏻降。璣大聲呼鏻堅守,勿以老人為念。賊怒,牽去。城破,鏻巷戰死,闔門為俘。大兵復漳州,賊遁,璣與子鈃等乘間入山,而妻葉人並諸幼子為賊執赴臺灣,璣置不為意。賊復犯海澄及長泰,璣再被執,脅之降,不從。幽之密室,歷年餘,終不為屈,朝夕誦文信公正氣歌以自壯。一日,顧謂子銑曰:「吾久辱,不死何為?」遂絕粒。數日,病甚,衣冠,命銑扶掖北向再拜,曰:「臣死,命也,當為厲鬼殺賊!」索紙筆,大書「惟忠惟孝,可以服人」數字,嘔血數升死,年七十有四。

劉欽鄰,江南儀徵人。順治十八年進士。康熙八年,授廣西富川縣知縣。十三年,廣西將軍孫延齡叛應吳三桂,遣偽將陷平樂府,旋圍富川。欽鄰募鄉勇城守,與賊相拒五十餘日。同城把總楊虎受延齡偽劄,勾土賊千餘助攻,虎夜引賊入,欽鄰率家丁力戰,殺賊三十餘,家丁死者七,欽鄰被執。賊加以毒刑,縛送桂林。延齡誘降,不屈,羈之。欽鄰賦絕命詞死,追謚忠節。

崔成嵐,鑲藍旗漢軍。由官學生任鬱林州州判,署藤縣知縣。十四年,孫延齡黨吳鳳等率賊數千犯藤縣,水陸夾攻,成嵐與守備劉志高、汪云龍,典史黃新德守御。賊暫退,巳而復合。延齡軍數千,攻城西南,抵禦益力。巡撫洪陳明復遣援兵,協力剿殺,賊不退。偽將軍緱成德復率賊萬餘由賀縣來,勢益熾。成嵐等相持七晝夜,城陷,成嵐手刃二賊,歿於陣,志高等均死之。

新德,廣東海陽人。讀書不多,好遣文,人皆笑之。事亟時,命其子日禱扶母歸養。既被執,賊欲授以偽官,新德曰:「王彥章且不肯降唐,況天朝臣子從賊乎?」賊欲屠城,新德曰:「倡守城者,官也,殺尉足矣,於百姓何與?」賊怒,斮之,新德罵不絕口。刀斧交下,碎其尸。家人四,婢一,皆死焉。微官死事,世尤重之。

柯永升,漢軍鑲紅旗人。由員外郎出任湖南糧道,累擢至湖廣巡撫。康熙二十七年,飭裁湖廣總督,令標兵分別存撤。五月,裁兵夏逢龍,同夥呼為夏包子者,結眾作亂。二十二日,突入巡撫署,拒者輒刃之。傷永升臂,奪其印,復傷永升足,僕地。悉驅其親屬家人出走,搜掠財物。永升乘間自縊死。賊四出剽略,永州錦田衛守備隨光啟嬰城守,力竭,死之。武昌永定營中軍守備孟泰鏖戰金口,亦中砲歿。守備李國俊陽附逢龍,從圍應城。夜半,賊潛梯登城,國俊遽鳴鉦大呼,城中驚起,擊敗之。脫還武昌,卒死樊口。時署布政使者為葉映榴,自有傳。

道禪,滿洲鑲黃旗人,姓戴佳。初為王府長史。康熙中,厄魯特噶勒丹犯喀爾喀,朝命中外備兵。三十五年,大兵三路進剿,道禪奉敕往諭噶勒丹。先是,三十一年,員外郎瑪第奉使策妄阿喇布坦,為噶勒丹掠執,不屈死。至是賊復誘降,道禪抗聲罵賊,死之。

李茂吉,福建漳浦人。臺灣水師營把總,平日不以官小自卑。康熙六十年,土賊硃一貴亂作,自請於副將許云。戰敗被擒,賊渠怪其不跪,叱之,茂吉舉足踢其案,案翻,奮力斷縛,直前奪刀殺賊。賊共斫之,頭腦破裂,尚罵不絕口,賊碎其尸。

劉昆,字玉巖,四川保寧人。由武舉從軍有功。雍正八年,擢權云南東烏營游擊,佐總兵劉起元守城。烏蒙夷祿萬福者,舊土知府萬鍾族弟也。先是,府隸四川,萬鍾數擾雲南邊界,雲貴總督鄂爾泰擒鞫伏法,使萬福父鼎坤襲職,移隸雲南。時改土歸流,既設東川府,次及烏蒙,改授鼎坤守備,趣赴闕。鼎坤怏怏行,密使萬福煽諸蠻為亂。未發,昆密告起元為備,起元蔑視之,檄萬福來見。萬福懼,遂嗾眾反,圍府城。昆聞變,解所佩刀與妻張氏訣,出與起元商禦賊策,皆不應。而游擊汪仁獨以撫賊說起元,起元從之,登城被賊辱。昆遂開城,率數十騎大呼赴賊,游擊馬秉倫與之俱。斬數百級,賊稍卻。野夷數萬蜂至,昆遂與秉倫相失,勢益孤。轉戰至次日,弩穿左脅,創甚,北向再拜,割襟蘸血,大書石壁曰「淋漓鮮血透征衣,報國丹心總不移」十四字,拔刀自刎死。賊嘆其忠,以土覆之而去。昆妻聞變,則以昆佩刀手斫二女及妾,乃引刀椿喉,一門同殉焉,語見列女傳。

秉倫既失昆,亦轉山箐間,鏢貫其頤,猶手剸數賊,力竭,跳崖死。

時官烏蒙通判者為劉鎮寶。鎮寶,字楚善,江西彭澤人。由舉人考授中書舍人,發雲南用知縣。鄂爾泰器其材,奏擢通判。鎮寶既蒞任,駐大關鎮,鎮距府三百里,為苗疆新闢地。苗警既急,以鎮寶熟諳苗情,檄往招諭。至則開陳禍福,詞甚備。苗逆抗之,反執鎮寶。鎮寶罵賊烈,爭斫之,支體糜碎。事平,滇人以鎮寶與昆受害尤酷,為立廟祀,稱二劉公祠。

羅鳴序,湖北漢陽人。康熙五十年舉人,任貴州麻哈知州、兼署黃平州事。雍正十三年春,古州苗叛,脅清平、黃平、施秉、鎮遠四州縣,生熟苗皆應。四月,陷清平縣之凱里汛,去黃平新州三十里。鳴序時在黃平,聞變,趨新州謀守御。環州苗皆起,馳報府縣急援,不應。苗大焚掠,鳴序以城亡與亡自誓。客陳憲者,請與俱,鳴序卻之。憲以「君能為忠臣,我獨不能為義士」為對,相與尋後山有樹可援系者,各默識之。鳴序乃解兩州印付健僕送省,出公帑千付書吏藏某處,曰:「可以死矣!」或曰:「此署事也,有本州在,何不去此而保麻哈?」或曰:「此新州也,何不去此而保舊州?」皆置不聽。或告曰:「城陷矣!」即趨向所識處,將自經。俄又告賊猶未入,則又徐與憲還,登城守。迨矢石器械盡,城中火起,無可再守,乃卒與憲至後山絓樹以死。從死者數人,諸生初震、周大任兩家皆死之。憲,浙江山陰人。


\end{pinyinscope}