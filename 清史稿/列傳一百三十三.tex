\article{列傳一百三十三}

\begin{pinyinscope}
恆瑞慶成七十五富志那亮祿

恆瑞,宗室,隸正白旗,吉林將軍薩喇善子。乾隆中,授侍衛,赴西藏辦事,擢熱河都統,遷福州將軍。五十二年,臺灣林爽文作亂,命率駐防兵往剿,參贊軍務,偕總督常青赴南路。鳳山賊勢方熾,高宗知常青、恆瑞不可恃,命福康安督師。賊圍總兵柴大紀於諸羅,恆瑞駐軍鹽水港,逗留不進,詔解任。福康安至,屢為疏陳戰績,帝益怒,斥其徇護,逮恆瑞論罪。事平,減死戍伊犁。尋予副都統銜,充伊犁參贊大臣。歷定邊左副將軍、綏遠城將軍,調西安。

嘉慶元年,命率駐防兵三千,偕都統永保會剿湖北教匪。三月,與總兵文圖破賊竹山。永保至,合師由房縣進剿,文圖分剿三里坪、喇叭洞諸賊悉盡;而心互瑞追賊至保康,未大創之。賊首姚之富踞襄陽,勢甚熾,命恆瑞進剿。五月,偕明亮進次呂堰,擊賊岳家溝、劉家集,擒斬二千餘。賊圍棗陽,設伏王家岡,誘賊敗之;又敗之於蔣家垱、曲家灣,棗陽圍解。賊偽降,潛襲官軍後路;急以後隊為前隊,擊退之。賊走丫兒山,與張家垱賊相犄角,奮擊一晝夜,破賊營十餘,殲賊甚眾,被獎賚。七月,破賊隨州龍門山,與永保會攻鍾祥賊巢,連破之鄧家河、黑沙河、雙溝。賊乃分竄唐縣、呂堰,追至滹沱鎮,復竄倉臺。尋,之富渡滾河,圍景安於鄧州。詔斥諸將玩誤,逮永保,責恆瑞戴罪立功。

二年正月,偕惠齡等剿襄陽賊,賊首劉起榮就擒;又與慶成敗賊鄭家河,擒賊目李潮;進剿泰山寺、龍鳳溝,擒賊目姚爽等:賜花翎。於是賊分竄,由河南入陜,恆瑞追賊至山陽,遇王廷詔、李全等,擊走之。五月,追賊陜南,與惠齡夾攻於黃龍垱,殲賊三千餘。廷詔、全復與賊之富合趨紫陽,渡漢江,恆瑞坐縱賊,奪花翎。賊遂分路入川,廷詔竄開縣、雲陽、萬縣,犯夔州,西與大寧賊響應,恆瑞追及,連敗之,乃竄太平。八月,當陽逸匪掠白河、洵陽,命馳赴興安扼剿,偕慶成擊賊於張家灘,由牛氾嶺繞出賊前,奪賊營九。廷詔等奔紫陽,與惠齡夾擊敗之。恆瑞率師還漢中,敗賊西鄉,又敗之褒城黃沙鋪。十一月,之富等西奔,將渡漢北竄。偕慶成蹙諸半渡,賊西趨寧、沔。師進,遇高均德於桑樹灣,乃議四面設伏,恆瑞令撒拉爾回兵假鄉勇旗幟誘之,自由山梁馳下,慶成等分路夾擊,俘斬甚眾。捷聞,被優賚。十二月,破王廷詔於保寧,進解營山圍。

三年,川匪羅其清犯順慶,偕慶成往援,因賊勢蔓延,請勒保、宜綿遣兵會剿。賊竄蓬州,潛結冉文儔擾儀隴,恆瑞扼磨盤寨,與惠齡等合擊之,文儔敗走,陜匪龍紹周與合,敗之楊家寨。六月,與德楞泰夾擊高均德於石人河,復偕惠齡攻老林場賊卡,進逼大神山,均德、文儔踞險死拒,分路進攻,賊奔箕山;而徐天德、樊人傑為將軍富成追擊,窮蹙,亦入焉。惠齡、德楞泰攻其前,恆瑞攻其後,盡破山寨,先後斬馘近萬。其清,李全、王廷詔奔大鵬山,進圍,十一月,克之。命赴陜與宜綿等會剿張漢潮。未幾,李全、樊人傑竄西鄉。帝以心互瑞未迎擊,嚴斥之。

四年,署陜甘總督,赴寧羌擊藍、白兩號賊。張應祥等竄秦州、兩當,又擊走張漢潮、冉學勝股匪。五月,解署任,剿白號賊於白馬關,地與川西龍安接壤,遣將冒雨掩擊,賊竄西和、禮縣;令布政使廣厚、總兵吉蘭泰截剿,自趨賈家店、黑馬關抄擊藍號賊,敗之於老柏樹,復花翎。賊竄川北,至秋,折回陜境,擊走之。乃赴城固、洋縣,會明亮剿張漢潮,破之東西叉河,賊從馬埡道遁老林,要之於清水溝,復乘霧雨徐渡三渡水。帝疑諸將縱賊,又以恆瑞前剿藍號賊垂盡,舍之回陜,下尚書那彥成察劾。那彥成,恆瑞之婿也,覆陳回師出總督松筠意,得免罪。尋明亮殲漢潮,恆瑞自五郎追擊,餘黨李得士等由大建溝入老林,趨秦嶺,與那彥成會剿冉學勝等,賊奔澇谷;扼兩岔河,追擊於山陽東溝,敗之。

五年,川匪二萬餘由略陽寇兩當、徽縣,恆瑞自褒城入棧,賊竄隴州、清水、秦安,偕那彥成追至汪家山,大敗之。總兵凝德戰歿秦安,恆瑞赴援,復偕那彥成敗賊於龍泉溝、深都堡,總兵多爾濟、札普戰歿洵陽。詔促恆瑞赴鎮安、五郎剿賊,三月,抵唐藏。楊開甲、高均德方擾南星,留總兵觀祥駐守,自赴商州。帝疑其趨避,累詔詰責,乃赴鎮安剿冉文勝等,敗之於大中溪。會額勒登保破開甲於輝峪,恆瑞自龍駒寨抄截,開甲逸走,圍副將李天林於漫川關,馳援,斬賊目羅貴等,賊乃分路西竄。敘功,予雲騎尉世職。六月,率總兵德忠駐守太渠、唐藏。時伍金柱、高天德、馬學禮犯西鄉,提督王文雄戰歿,乃進兵大石川,賊奔灘口,為楊遇春所破。

恆瑞自教匪起,久在行間,以偏師數臨大敵,至是老病,久無顯功。帝慮其不任戰,詢額勒登保,上其狀,命回鎮西安。逾年卒。

慶成,孫氏,漢軍正白旗人,提督思克曾孫,都統五福孫也。由鑾儀衛整儀尉,累遷廣東督標副將。乾隆五十三年,從總督孫士毅征安南,屢擒敵有功,賜花翎、錫郎阿巴圖魯勇號。內擢正白旗漢軍副都統、戶部侍郎、御前侍衛、正紅旗護軍統領。五十七年,出為古北口提督。

嘉慶元年,率兵赴南陽、襄陽剿教匪,偕恆瑞迭敗姚之富、劉之協於雙溝、張家集。賊屯棗陽丫兒山,分踞張家垱,連營十餘里,遮官軍,慶成先進,襲其寨,大破之,擒宋廷貴、陳正五,追敗餘匪於紅土山,擒黃玉貴,加太子少保。之富竄鍾祥,合劉起榮、張富國等眾五六萬,偕永保等冒雨攻克之,晉太子太保。賊遁雙溝,擾唐縣滹沱鎮。慶成等以久戰兵疲,不能圍剿,詔嚴斥之。賊竄棗陽太平鎮,四路合攻,斬數千級,慶成受矛傷,被優賚。十一月,賊潛渡滾河北竄,與永保等並被嚴譴,盡奪宮銜、花翎、勇號,易惠齡為總統。尋偕惠齡連破賊於王家城、梓山。二年正月,大戰興隆集,斬二千餘級。分路追賊,慶成射中賊首劉起榮,擒之,在諸將中戰最力。高宗以慶成為五福孫,不次擢用;自縱賊滾河,慮其少年自用,不能服眾,命惠齡察奏,至是詔免前罪。二月,擊賊曾家店,胸中矛,裹創而戰。賊敗竄河南境,分數路,慶成追李全,連破之確山五里川、盧氏火焰溝。四月,李全、王廷詔合陷鄖西,馳復其城,賊不戰分遁。未幾,之富竄渡漢江,降二品頂戴,暫留提督任。襄匪竄開州,偕惠齡追敗之南天峒、火焰壩,復花翎。賊趨大寧,與川匪合,慶成與川軍會剿。九月,偕恆瑞截擊湖北回竄之賊於洵陽,而李全、王廷詔沿漢東走,慶成登舟下漢以要其前;惠齡、心互瑞從陸躡其後,至紫陽夾攻之,賊竄興安,慶成一晝夜追及,大破之司渡河。

川匪王三槐擾保寧,羅其清、冉文儔分掠川東,命移兵赴川,與宜綿合剿。三年,截擊其清,腿中槍,創甚,解任回旗就醫。四年,創愈,仍在御前侍衛行走。尋授成都將軍,命赴陜西與永保協剿張漢潮。會明亮訐奏永保、慶成失機,命那彥成、松筠按治,褫職逮問;又以在湖北受軍需饋遺,籍其家。漢潮既殲,宥罪戍伊犁,未行,五年正月,命仍赴陜軍效力。額勒登保檄剿高天德、馬學禮,連敗之禮辛鎮、何家衢,擒斬數千,予三等侍衛。協剿伍金柱、曾柳,授陜安鎮總兵。七月,金柱與冉學勝、張天倫合犯陜,扼之渭河,賊分竄;追天倫於教場壩、麻池溝,殲其黨宋麻子,又敗金柱餘黨曾芝秀於南山:兼署固原提督。時經略赴川,陜、甘兵三萬餘皆歸慶成節制,川匪冉天元、冉學勝、樊人傑先後渡漢江,詔斥慶成疏防,責戴罪立功。六年,徐天德、樊人傑復至江岸,欲偷渡鄖西,擊卻之,實授提督。擊楊開甲餘匪於廣元,獲其子麟生,加頭品頂戴。茍文明潛入甘肅境,擊走之,復勇號。追川匪辛聰等於寧沔,擒其黨曾顯章、張添潮。七年,敗張天倫餘黨於鳳縣、兩當,擒張喜、魏洪升,賊竄紫柏山老林,裹糧入捕,悉殄其眾,復太子太保。

先是慶成父歿,軍事方亟,不得去;至是南山匪漸少,乃許回旗守制。尋署湖北提督,服闋實授,遷成都將軍。十一年,入覲,帝睠其勞,問:「曾戴雙眼花翎否?」慶成對:「征安南蒙賜,和珅禁勿用;獲劉起榮,先帝欲賜,復為和珅所阻。」命軍機處檢檔無之,遂以欺罔褫職,戍黑龍江。逾年,授圍場總管,歷馬蘭鎮總兵、湖北提督、福州將軍。十七年,卒,謚襄恪。

七十五,瓜爾佳氏,滿洲正黃旗人。乾隆中,以護軍從征緬甸,繼赴金川,戰輒力,累遷護軍參領,授貴州大定協副將。總督福康安薦其才,四十九年,擢宜昌鎮總兵。父憂去官,坐事降秩,起為健銳營前鋒統領。五十七年,從征廓爾喀,克濟嚨,又克熱索橋,追賊東覺山、雍雅山,攻甲爾古拉,並有功,擢翼長。

嘉慶元年,赴湖北剿教匪,二年四月,追賊入陜,敗之山陽周家河,授西安右翼副都統,兼領健銳營。其冬,王三槐回竄四川,追擊於達州崖峰尖,傷右臂;逾日,賊復至,裹創力戰,斬獲甚眾。三年,擢四川提督,敗賊巴州。七月,戰廣木山,克險隘,受傷,被優賚。九月,擊冷天祿於木瓜坪,右股中槍,創甚,就夔州療治,四年,始瘳。六月,連破賊於寶塔、蓮花池,扼其入楚之路。會卜三聘竄大寧,追敗之。八月,擒龔建於開縣火峰寨。十月,與穆克登布夾擊樊人傑於通江、巴州界上,賊走太平,他賊自湖北回竄,偕硃射鬥迎擊於雲陽,遂追賊川東。

時賊聚川北,而東路久無軍報,適侍郎廣興疏言七十五駐兵夔州,仁宗疑其逗留,下經略察狀,七十五方以攻麂子坪受重傷,額勒登保為疏辯,得白。五年二月,鮮大川擾螞蝗坪,創發,不能騎,舁至軍前督戰。冉天元擁眾渡嘉陵江,重慶戒嚴,魁倫檄令回守,病不能軍,遣李紹祖率兵赴川西,自就醫順慶。帝疑其飾辭,詔解任,命松筠、勒保察驗得實,以提督銜留營差遣。五月,高天德、馬學禮由陜犯川,折入番地,偕阿哈保夾擊於舊關摩天嶺,克新寨,進圍鐵爐寨。賊乘雨宵遁,追擊之,賊棄牲畜、仗械,驚竄山谷,由草泥土司地走岷州,又走秦州。七月,兵經新寧,偵馬驛溝有賊,設伏,敗之,仍授四川提督。賊勢趨重川境,德楞泰、勒保方進剿,七十五分擊之。至冬,諸賊相繼窺漢江,德楞泰議擊之南岸,而以七十五出廣元三家壩攻其西北。七十五不聽調,曰:「兵深入,將逼賊入陜,非計也。」帝聞,切責之。

六年正月,率子武隆阿由廣元趨南江,擊張世龍於三臺山、後河嶺、北溪河,陣斬世龍,擒其黨趙建功、李大維;又追賊至太平華尖山,擒邱天富、周一洪:被優敘。三月,攻竹園坪。五月,賊分竄陜、楚,七十五追冉天士至平利大渝河,間道據後山,偪其出隘,伏起邀擊,擒斬二千餘,特詔嘉賚。乘勝追賊入湖北境,六月,破湯思蛟、劉朝選於羊耳河;又敗之於保康,殲賊首王鎮賢,遂與德楞泰追龍紹周入川。七月,偕李紹祖敗樊人傑於鄰水,追至開縣,復遇思蛟、朝選,連敗之於馬家亭、桑樹坪,由通城進剿茍文懷,擒之。餘賊與茍文明合,將竄陜,八月,擊之於大寧山,殲擒及半,文明僅身免,俘其家屬。

是年冬,留防川北,敗賊於南江;又與德楞泰合擊於廣元、蒼溪,進搜老林,賊多散匿,百十為群,時有斬獲。十二月,茍文明糾各路餘匪二千餘人,乘間西奔。七十五與勒保不和,追賊入山,餉半載不至,兵饑疲,就糧太平,六日,賊已渡嘉陵江上游,直趨階州,亟偕慶成馳擊。額勒登保、德楞泰先後劾其頓兵縱寇,未幾,賊復自廣元渡江入甘肅,帝益怒,嚴詔褫職逮問。

七十五故宿將,勇而訥,臨陣輒死鬥,身被重創十五次。將弁畏其苦戰,不樂相隨。自領偏師當艱險,數以軍報後時遭譴;至是,復失機就逮,一軍皆慟哭。額勒登保等為疏陳戰狀,乞恩,許留營自贖。七年,剿張長庚、陳自得殘匪於夔州,留防川東。舊創發,予護軍校,還京。逾年,卒,贈副都統銜,賜恤如例。子武隆阿,自有傳。

富志那,赫舍哩氏,滿洲正紅旗人。起健銳營前鋒,從征葉爾羌、緬甸、金川,授副前鋒參領,出為湖南永綏協副將。乾隆六十年,苗叛,駐守永綏。苗踞張坪、亞保阻糧運,悉眾來犯,富志那擊卻之。追至獅子山,詗知有伏,預為戒備,夾攻,多所斬獲。越日,苗復以數千人撲營,殊死戰,簡精銳迎擊,大敗之,賜花翎。永綏被圍久,糧芻且盡,居民隨官軍晝夜登陴,城賴以固。大軍至,圍乃解。從福康安克高多寨,吳半生就擒。福康安薦其老成明幹,苗民感畏,擢總兵。迭攻高斗山、擒頭坡、吉吉寨,皆捷,賜蟒衣一襲。

嘉慶元年,湖北教匪聶傑人、張正謨於枝江、宜都倡亂,巡撫惠齡駐軍太和山,富志那馳赴之,進擊鳳凰山,擒傑人。餘賊乘雨撲營,擊卻之,又敗之於楊白堰。正謨踞灌灣腦,四面環山,富志那自蔡家坡進,冒雨奪卡,而伏隊於深箐,賊至,左右夾擊,多墜巖澗死;山前設疑兵,別由徑道深入,出不意擊之,大捷:賜號法福禮巴圖魯。迭克雞公山、王母峒,進攻筲箕垱,正謨勢蹙,四出求救,富志那與副都統成德分路設伏,偽樹白幟為援兵,誘賊出,大破之,遂克筲箕垱;乘勝取灌灣腦,擒正謨。枝江、宜都悉平。

命回苗疆治善後。二年,議闢永綏北路,留兵二萬分防黔、楚,授富志那為總兵,駐鎮筸,與提督分領其軍。苗疆自同知傅鼐築碉屯田,邊備漸嚴,而苗未遽服,構眾抗阻,大吏諉過於鼐,將劾之,富志那力爭乃止。移軍需助其建設,後屯田利興,苗患遂息。人稱鼐功,兼頌富志那不置雲。五年,鎮筸曬金塘黑苗出掠,與鼐並力禦擊;又要擊苗黨於狗嵒,焚其寨,苗懼,乞降。八年,永綏苗龍六生擾動,擒之。署湖南提督,調授貴州提督,軍政肅然,時稱名將。十五年,卒於官。

亮祿,伊爾根覺羅氏,滿洲正紅旗人。襲輕車都尉世職,授密雲協領。嘉慶初,以參將發河南,署游擊。三年,教匪窺河南,巡撫吳熊光駐防盧氏,兵多他調。寶豐、郟縣賊起,掠汝州。布政使馬慧裕不嫺軍事,亮祿曰:「兵貴神速。今賊初起,烏合易滅,請兼程往剿。」賊屯寶豐翟家集,東阻大溝,恃險不退,亮祿聲言京兵且至,樹八旗大纛,鞭馬腹,俾騰踔嘶號,聲震數里,賊懼;夜吹角而進,躍馬逾壕,火其寨,一鼓殲之,擒其渠李岳等。奏入,仁宗大悅,立擢副將。累遷雲南開化鎮總兵。七年,卒,帝甚惜之。

論曰:恆瑞、慶成戮力襄陽,剿匪最久,後皆獨當一面,功過不掩,故仁宗始終保全。七十五孤軍苦戰,徒以失懽群帥,未奏顯功,論者惜之。富志那獨平枝江、宜都一路,移鎮苗疆,與傅鼐和衷弭亂,有足稱焉。


\end{pinyinscope}