\article{志三十五}

\begin{pinyinscope}
地理七

△山西

山西:禹貢冀州之域。清初沿明制為省,置總督、巡撫。順治末,總督裁。康熙四年,並冀南、冀北置雁平道。雍正元年,置歸化。二年,增直隸州八。平定、忻、代、保、解、絳、吉、隰。三年,增府二。寧武、朔平。六年,升蒲、澤二州並為府,置歸綏道。乾隆四年,增置綏遠同知。二十五年,又以歸綏所屬地增置五通判。歸化城、清水河、薩拉齊、和林格爾、托克托城。與歸、綏二並屬歸綏道。二十九年,裁歸化城通判。三十七年,吉州改屬平陽府,霍州為直隸州。今領府九,直隸州十,十二,州六,縣八十五。東界直隸井陘;三百七十五里。西界陜西吳堡;五百五里。南界河南濟源;七百三十里。北界內蒙古四子部落草地。一千一百里。廣八百八十里,袤一千六百二十里。北極高三十四度五十七分至四十一度五十分。京師偏西三度四分至五度四十五分。東北距京師一千二百里。宣統三年,編戶一百九十九萬三十五,口九百二十一萬九千九百八十七。其名山:管涔、太行、王屋、雷首、底柱、析城、恆、霍、句注、五臺。其巨川:汾、沁、涑、桑乾、滹沱、清漳、濁漳。鐵路:正太。驛道:西達蒙古、陜西潼關,東北至京師。電線達京師,西南西安。

太原府:沖,繁,難。隸冀寧道。巡撫,布政、提學、提法司,巡警、勸業道駐。初沿明制,領州五,縣二十。雍正中,平定、忻、代、保德直隸,割十縣分入之;尋興還隸。乾隆二十八年,省清源入徐溝。距京師千二百里為省治。廣六百里,袤七百里。北極高三十七度五十四分。京師偏西三度五十六分。領州一,縣十。陽曲沖,繁,難。倚。東北:阪泉山。西北:崛唅。北:梁鴻。西南:汾水自交城入,逕冽石口,左合埽谷水,折東南,左合洛陰及石橋、真穀水。水經注「逕盂縣、狼孟故城南」者。至城西北,左合石河、南社河,又南入太原。天門關、石嶺關二巡司駐。王封鎮,同知駐。埽峪村、楊興寨。城晉、陵井驛。太原沖,繁。府西南四十里。西南:尖山。西北:蒙山,其南風峪、懸甕,晉水出焉。東北:駝山。汾水自陽曲入,左納澗河,逕城東,至南張村與合,又西南入徐溝。東:洞渦水自徐溝來,西南流,逕縣南,仍入徐溝界。榆次沖,繁,難。府東南六十里。北:罕山。東南:麓臺。東北:小五臺。洞渦水自壽陽入,左納金水河,古塗水,即水經注蒲水,合八賦嶺、鷹山水今所謂大小塗,即水經注蒲谷水注之。右合原過水四派,唐貞觀中,令孫淇引以溉田,逕城南,西南入徐溝。其澗水入蒜谷,又西入太原。源渦、什帖二鎮。鳴謙、王胡二驛。太谷繁。府東南百二十里。南:鳳皇山。北:壁穀。東南:鳳巢;大塔,大塗水出焉,西北流入榆次。西:烏馬河自榆社入,右合奄谷水,左咸陽谷水,逕城北入祁。象穀水即古蔣谷水,入徐溝。有馬嶺關、杏林寨。主簿駐範村鎮。祁沖,繁。府西南百四十里。東南:竭方、幘山。侯甲水自武鄉入,逕龍舟峪,為龍舟水。又盤陀水,西北為昌源渠,逕城北入平遙。東北:烏馬河自太谷入,又西入徐溝。子洪、盤陀、團柏、賈令四鎮。安寨、盤陀二驛。徐溝沖,繁,難。府南八十里。乾隆二十八年省清源為鄉入。訓導及巡司駐。西:壺屏山。其北,白石、中隱。汾水自太原入,逕孔村至西堡。東北:洞渦水自榆次入,錯太原,復入縣西,左納烏馬及象河入焉。故驛鎮。同戈驛。交城簡。府西南百二十里。交城山,北百二十五里,相近羊腸。西北:狐突。汾水自靜樂入,逕火山村,右合孔河,折東入陽曲。西北:孝文山,文水出,會渾谷、西谷,屈東南,左合酸水,為文谷水,入文水界,從之下流,並達之。故交村,巡司駐。文水繁,難。府西南百六十里。西:陶山。西北:熊耳。西南:隱泉。東有汾水,自徐溝入,西南入平遙。西北:文谷河自交城入,逕文谷口。唐柵城廢渠在焉。至城北,又東南,左合磁窯河、步渾水,折西南入汾陽。有孝義鎮。岢嵐州簡。府西北三百二十里。岢嵐山,東北百里,一曰管涔。迤西南,蘆芽、荷葉坪、雪山。東南:直道村,嵐漪水出東北,右合黃道川、三角城二水,折西北,逕城南,又西逕大澗河,左合砂河,又西南逕巨麓山入興。水峪關。嵐簡。府西北二百六十里。西南:黃嶮山。西:野雞山,蔚汾水出,入興。南:赤堅嶺,嵐水出,東北逕桃尖山,左合乏馬嶺、雙松山水,折東南入靜樂。有東村鎮。興簡。府西北四百里。雍正二年隸保德州。八年仍來隸。東:桃花山。西南:採林。西北:黃河自保德入。東北:嵐漪水自岢嵐入,逕石樓山。東南:蔚汾水自嵐入,逕合查山,至縣西,合南川水並入焉。又南合紫荊山水入臨。蔚汾、合河二關,皆要隘。

汾州府:沖,繁,難。隸冀寧道。康熙六年,省明冀南道入。東北距省治二百二十里。至京師千三百八十里。廣三百五十里,斜。袤三百二十里。沿明制。北極高三十七度十九分。京師偏西四度四十五分。領州一,縣七。汾陽繁,疲,難。倚。西:將軍山、黃蘆嶺。北:謁泉。東北:文谷水自文水入,循汾水故道,右合原公、金鎖關水,至府治東為文湖。又南,右納義水,入孝義。郭柵、陽城二鎮。冀村,巡司駐。有驛。孝義繁。府南少東三十五里。西:上殿山。西北:龍門;薛頡嶺,古狐岐山,禹貢「治梁及岐」。其南,盤重原,勝水出焉,俗名孝河,會南川、陽泉水,逕城南而東。東北:文谷河自汾陽合義水入,逕鹽鍋頭入介休。溫泉、鳳尾二鎮。平遙繁,沖,難。府東八十里。西北:汾水自文水入,逕長壽村。東北:侯甲水自祁入,左合謁戾山、嬰澗、過嶺、魯澗,超山、中都及亭岡水入焉。又南入介休。上殿鎮。洪善驛。介休沖,繁,難。府東南七十里。南,介山,一曰綿山,綿水出。東:天峻,石河出,又東石桐水出。東北:汾水自平遙入,後先合之,入靈石。張蘭鎮,同知駐。義棠驛。石樓簡。府西少南百八十里。東南:石樓山。西:九重。西北:團圓山。黃河自寧鄉入,合屈產泉,古牧馬川,復合溫泉,即石羊水,入永和。臨簡。府西北三百二十里。東南:漢高山。西南:招賢、馬頭。河水左瀆自興入,合紫金山水,又南逕曲峪鎮入永寧。其湫水亦自興入,逕赤壁山,合連枝、積翠、黃龍、漢高、雲山凡十六水入焉,又南入永寧。有三交鎮。永寧州沖,繁,難。府西北百七十里。東:九鳳。東北:呂梁。西:匾斗、南山。西北:馬頭。河水左瀆自臨入。東北:赤堅嶺,離石水出,曰北川。南:步佛山,合蘆子山水,逕城西,合東川,納南川,清水河入焉,又南入寧鄉。吳城、柳林、永安三鎮。柳林、方山堡二巡司。玉亭、吳城、青龍三驛。寧鄉簡。府西百四十里。南:雲集嶺。北:寧陽山。東:柏窊、蕉山。西南:泉子,清水河出,北合屏風山水,逕城東,又西北入永寧注河。河水左瀆復入,逕三交鎮,合石口、牛尾泉水,入石樓。有鋤鉤鎮。

潞安府:繁,疲,難。隸冀寧道。初沿明制,領縣八。乾隆二十九年,省平順,分入潞城、壺關、黎城。西北距省治四百五十里。至京師千三百里。廣三百里,袤二百七十里。北極高三十六度七分。京師偏西三度二十八分。領縣七。長治繁,難。倚。東:壺口山。東南:五龍。東北:柏谷。西南:福泉。濁漳水自長子入。東南陶水出雄山,北合八諫、雞鳴山水,右會淘清河入。又北至秦村,左會藍水,右石子河入,又西北入屯留。鎮四:韓店、高河、太義、西火。分防同知駐太義鎮。縣丞駐西火鎮。驛二:龍泉、太平。長子沖。府西少南五十里。東南:紫雲山、羊頭。西南:發鳩山,水經注鹿谷。濁漳水出其東麓四星池,東會傘蓋、陽泉水,逕城南,右合堯水、慈林水及梁水,入長治。西北:藍水自屯留入,逕河右會雍水,亦入長治。鮑店鎮,縣丞駐。漳澤驛。屯留沖。府西北六十里。東北:良材山。西北:五巑。西南:盤秀,藍水出其陽摩河嶺,古絳水,東入長子注濁漳。至長治北流,逕縣東入潞城。今絳水出其陰,東逕石田山,左會高麗水,又東北,右合霜澤、左三嵕山水,逕城北,右合疑水。雞鳴水乃古諫水,逕餘吾故城南、屯留故城北者。鎮二:寺底、豐儀。驛一:余吾。襄垣沖。府北少西九十里。西南:五巑山。北:五音、仙堂。西北:紫巖。東南:鹿臺。濁漳水自潞城入,逕其北,左會銅鞮水,又北逕城東。東北:涅水自武鄉入,右會臨水,史水自左注之,為合河口,入黎城。鎮二:東周、下良。驛一:虒亭。潞城簡。府東北四十里。南:盧山、大禹。東南:伏牛、葛井。東:靜林。西:三垂。西北:黃阜。西南:濁漳水自屯留入,左合絳水,為交漳,即禹貢降水。又西北入襄垣,至黎城錯入,逕潞縣故城,是濁漳兼有潞浸之稱。又東復錯黎城,仍入境。東出馬塔口入河南涉縣。西南有三垂岡。東南有虹梯關,即魯般門,巡司駐。鎮三:神頭、黃碾、羊圍。東南平順鄉城,鄉學訓導駐焉。壺關簡。府東南三十里。壺關山,西北六里。東北:風穴山。東:馬駒、麥積、安公。壺水出西北,逕城北為石子河,左合清流河,東南大王、抱犢,又東赤壤。其陰東井嶺,淘清河出,西北逕黃山,並入長治。嶺東五指河,東南為沾水,逕紫團山入河南林縣。東有玉峽關。馮坡鎮。黎城簡。府東北百十里。東南:潞祠山。西北:積布、錏峪。濁漳水自襄垣入,東南逕聯珠山,錯潞城復入,左合黃須水,東逕赤壁山,仍入之。東北:繡屏,清漳水自遼入,逕吾兒峪,古壺關在焉,入河南涉縣。玉泉水從之。

澤州府:沖,難。隸冀寧道。初沿明制,為直隸州。領縣四。雍正六年為府,增附郭。西北距省治六百二十里。至京師千六百里。廣三百四十里,袤二百三十里。北極高三十五度三十一分。京師偏西三度三十七分。領縣五。鳳臺沖,繁,難。倚。南:太行山,其巔黑石嶺,其北天井關;西南,小口,即太行陘馬牢。東南:硤石、浮山。北:司馬。東北:丹水自高平入,左合蒲水,南逕高都故城東。其南源澤水,出西北二仙掌,合塔河來會。又南,左納丈水,逕八盤、壘石、石人山。白水合西沙河,逕城南,合轆轤水,天井溪右注之,入河南河內。西北:吳山,陽阿水出,南逕蟠龍、聖王山,入陽城注沁水。沁水復入,入濟源碗子。鎮三:橫望、攔車、周村。驛二:太行、星軺。丞兼巡司駐星軺。高平沖,繁。府北少東八十五里。北:韓王山。東:七佛。西:髑髏、浩山。西南:空倉。西北:發鳩,漳水出其陰。其巔鳳頭,丹水北源出,左會白水,右絕水,實泫水。東南,右合長平水,逕城北。又東南,左合西東長河,至杜村。右合五龍山水,俗亦曰泫水,入鳳臺。東有蒲水,自陵川入從之。東有石壁關。西北有長平關。鎮四:米山、丁壁、野川、時莊。喬村、長平二驛。陽城難。府西八十里。西南:王屋,其東析城,有三峰,亦曰底柱,濝水出。東南:莽山,溴水出,北源,並入河南濟源。東北:沁水自其縣入,左合史山河,右合陽泉水,東南逕南莊。其澗河入為南河,右合濩澤水,逕閻家津,右合桑林水,左納陽阿水,入鳳臺。東南有白雲隘,路通濟源。縣境十七隘,此為最要。東冶鎮,同知駐。陵川簡。府東北百二十里。西南:九仙山。西北:寶應。聖宮山,蒲水出,屈西,左會龍門山、鳳山水,入高平。東北:堯莊,丈河出,西南逕靈泉、六景、佛兒諸山,入鳳臺。東南:王莽嶺,源水出。洪水村,平田水出,並入河南輝縣。南:雙頭泉,屈南,逕瘦驢嶺入修武。東北:淅山,淇水出,俗淅水,逕熊耳,即沮洳山,入壺關。沁水簡。府西北百七十里。西:阜山。西南:輔山。東北:隗山。北:大尖,至河頭寨,右合梅河、杏河。沁水自岳陽右會東河,即水經黑嶺水。又東南逕紫金山至端氏故城,左合秦川及熊耳山水,即水經注簳簳水。又東南,左合潘河,入陽城。西南:鹿臺山,蘆河出,古陽泉水。其南澗河並從之。鎮四:郭壁、武安、固鎮、端氏。端氏,巡司駐。

遼州直隸州:繁,隸冀寧道。西北距省治三百四十里。至京師千三百里。廣三百三十里,袤百七十里。北極高三十七度三分。京師偏西三度一分。領縣二。遼陽山,城東三里。東:東云。南:武軍。城西:★山。東北:摩天嶺。清漳水自和順入,逕黃張鎮,屈南,右會西源西漳河,為交漳口。左合箕山水即洗耳泉東六十里,此附會為河南登封山者,逕黎城東,入河南涉縣,至林縣與濁漳合。長城、黃張、芹泉、桐峪、麻田、韓王、拐兒,凡七鎮。黃澤關之十八盤,巡司駐。南關驛。和順簡。府東北九十里。東北:合山。西南:斷孤。西北:九京。北:麻衣。清漳水自平定入,逕石猴嶺,屈折至城東南,右會南源梁餘水,又東南,左合清水,古黃萬水,逕首陽山入州。西南,八賦嶺,其西源遼陽河出其北轑山。水經注轑水,亦西漳水,東南逕儀城鎮,從之。武鄉水,出其南武山,入。榆社、松煙、寒湖、馬嶺、青城、虎峪、馬坊、橫嶺、溫泉,凡八鎮。八賦嶺巡司。榆社簡。府西九十里。東南:秀容山。東:清涼、箕山。北:北泉。東北:武鄉水自和順入,西逕其故城北三十里即地形志榆社城,折南,逕城西,又南納縣之西川、儀川等水,入武鄉界。西北:黃花嶺,烏馬河出焉,西北流,入太谷界。北有馬陵關,西有石會關。雲簇鎮。

沁州直隸州:沖,繁。隸冀寧道。西北距省治三百三十里。廣三百二十里,袤百三十里。北極高三十六度四十一分。京師偏西三度四十二分。領縣二。東:麟山。西:堯山。西南:銅鞮。西北:伏牛。迤東漳源鎮,小漳水出,左會花山、爛柯山,逕城西,又南,右合後泉、上官泉,至萬安山北,右會銅鞮水,入襄垣,亦通目之。郭村、西湯二鎮。沁陽驛。沁源簡。州西少南百二十里。西北:綿山,其異名曰謁戾,曰羊頭,沁水出焉。東南逕仁霧山,右會湱水,左琴谷水,至交口折西南,逕城東,合芹泉山水,至南石,左會青龍山水,右西川、大南川,入岳陽。柏子、郭道、官車三鎮。武鄉簡。州東北五十里。城東北:鞞山。東南:三原。西:麓臺。西北:侯甲山。有分水嶺,侯甲水出其陰,北入祁。涅水出其陽,實水經注湯谷五泉水。左會高砦寺河,古白雞水,逕城西,左會武鄉水,又東至城南,左合南亭水,折南,左合鍋窯嶺水,入襄垣。鎮二:盤龍、墨鐙。驛二:權店、南關。

平定直隸州:沖,繁。隸冀寧道。初沿明制,為太原屬州。雍正二年升,仍領,並割盂、壽陽來隸。嘉慶元年,省樂平入。西北距省治二百七十五里。至京師八百七十里。廣二百七十里,袤二百九十五里。北極高三十七度五十分。京師偏西二度四十八分。領縣二。東:皋落山。東北:蒙山。東南:松子嶺。西南:沾嶺;冶水南源沾水出,會小松鳴水,東入直隸井陘。其北甘桃河,西北桃水自壽陽入,匯保安河、平潭、陽泉水,逕城北,又東,右合南川,逕交原村,左納文谷水,至古承天軍。左合畢發水,並從之。清漳三源。北源出其西大要谷,山海經所謂「出少山」者,南入和順。洞渦水出其北陡泉嶺,西逕馬尾嶺,左合浮化山,納木瓜嶺水。水經注,南路西入壽陽。東有故關,東北有娘子關,並接井陘界為要隘。有正太鐵路。一鎮:靜陽。三驛:測石、甘桃、柏井。樂平鄉城,州判及鄉學教諭駐。柏井,巡司駐。其甘桃,裁。盂沖。州西北百里。南:石艾山。東:白馬。東北:原仇。北:牛道嶺。滹沱水自五臺入,逕其西,右合烏河,又東,右合龍花河,入直隸平山。西南:秀水出南上社,合香水,夾城東南,從行千二百餘里,下至天津入海。東北黃安、十八盤、榆棗諸關,並通平山,東橫河槽通井陘,並要隘。芹泉驛舊設巡司,後裁。壽陽沖,繁。州西百里。初隸太原府。雍正二年來隸。北:方山。西北:雙鳳、罕山。東:桃源溝,冶水北源桃水出。地理志,綿蔓水會芹泉水東入其州。南:洞渦水自州入,至縣南過山。西南:要羅山,壽水出,東會黑水,龍門河注之,西入榆次。有正太鐵路。一鎮:遂成。一驛:太安。驛丞兼巡司駐。

平陽府:沖,繁,難。太原鎮總兵駐。初沿明制,領州六,縣二十八。雍正二年,蒲、解、絳、吉、隰直隸,割臨晉二十縣分隸太平、襄陵、汾西,尋復。乾隆中,霍直隸,割趙城、靈石屬之,吉州及鄉寧復。東北距省治六百十里。至京師千八百里。廣二百七十里,袤百八十里。北極高三十六度五分。京師偏西四度五十六分。領州一,縣十。臨汾沖,繁,難。倚。東南:浮山。北有汾水自洪洞入。東南:潏水自浮山入,逕其東,左合金水河,右澇水注之,南逕城西。有姑射山,一名平山。平水東注之。其南出者並入襄陵。西北分水嶺,大東河出,入蒲。泊莊、礬山二鎮。建雄驛。洪洞沖,繁。府東北五十五里。東:九箕山、霍山。北有汾水自趙城入,逕城西,右合北澗,屈西南,左納南澗,右合婁山、禹門山水,入臨汾。郭盆鎮。普潤驛。浮山簡。府東少南七十里。浮山,西南三十里,金水河出。東南:銀洞。東北:堯山;烏嶺,澇水出,西入臨汾。東:天壇,南河出,西南:壺口,實蜀山,潏水出。東北:橫嶺,即中條,東河出,入沁水。東張鎮。岳陽簡。府東北百二十里。北:雪白。西北:尖陽。東南:刁黃。東北:沁水自沁源入,右合和川河,左納橫河,屈南入沁水。東北:安吉嶺,澗河出。其一源出西北金堆裏水,逕城東屏風山,又南,左合永樂里水,其南南澗出郭店,並西入洪洞。東北有潼關隘。曲沃沖,繁,難。府南百二十里。西南:絳山。西北:橋山。西有汾水自太平入,左納釜水,入絳。東有水會水自翼城入,左納絳水,亦入絳。鎮二:柴村、侯馬。驛二:侯馬、蒙城。巡司駐侯馬。翼城難。府東南百三十里。北:丹山、蜀山。東南:歷山。東北:烏嶺、佛山。澮水南北源出,合逕城東而南,左會東源絳高山水,今灤水。烏嶺,霍東支,故說文「澮出霍山」,水經則統曰「出澮交東高山」。又西南逕澮交,錯絳復入,入曲沃。西北:小綿山,滏水出,西南流,亦入曲沃。有隆化鎮。太平沖,繁,難。府西北九十里。雍正二年隸絳,七年復。南:汾陰。西南:九泉。東北:汾水自襄陵入,南入曲沃。鎮三:清儲、趙康、史村。一驛:史村,驛丞兼巡司駐。襄陵難。府西南三十里。雍正二年隸絳,七年復。東南:崇山。西南:三嶝。東有汾水自臨汾入,右合平水。又諸山澗水三派東注,入太平。趙曲鎮。汾西簡。府西北百九十里。雍正二年隸隰,九年復。汾陰山,西南五十五里。東南:汾水自霍入,右合轟轟澗、勍香河,逕商山入趙城。鄉寧簡。府西少南二百三十里。雍正二年隸吉州,乾隆三十六年仍來隸。東北:柏山、秦山。西南:兩乳。東南:馬頭。西北:香爐巖。河水自吉入,逕其麓,有師家灘。東:鄂山,鄂水出,會北源高天山水。又西合羅谷水,逕城南,又西北入焉。又東南,入河津。營裏鎮。吉州繁。府西百七十里。雍正二年直隸;乾隆三十七年復。吉山,治北。東北:雞山、石門。北:庖山、風山。河水自大寧入,逕龍王池,禹貢壺口在焉,即孟門山。至小船窩。東南:高天山,清水河出。水經注,羊求水合放馬嶺、雲臺山水,西逕城南入焉。又東南入鄉寧。三垢鎮。

蒲州府:沖,繁,難。隸河東道。明,平陽屬州。雍正二年直隸。仍明所領臨晉、榮河、猗氏、萬泉,惟河津削。六年為府,置附郭。尋增虞鄉。東北距省治千一百里。至京師二千二百里。廣百三十里,袤百十里。北極高三十四度五十二分。偏西六度十五分。領縣六。永濟沖,繁,難。倚。明州治,省河東入。雍正六年置。東南:中條,即禹貢雷首,其南阜堯山、首陽,迤東歷山。東北:河水自臨晉入,西逕蒼陵谷,至韓家營,錯陜西郃陽、朝邑。其涑水會姚暹渠於東五姓湖入,又西從之,至鹽灘復入。迤東南逕風陵渡,溈汭入焉,揚雄賦所謂「河靈矍踢,掌華蹈襄」者。又東入芮城。鎮四:趙伊、匼河、栲栳、永樂。同知駐永樂。河東驛。臨晉沖,繁,難。府東北七十里。東北:嶷山。西北:河水自榮河入,逕吳王渡。東南:涑水自猗氏緣虞鄉界注五姓湖從之。樊橋鎮又驛。虞鄉難。府東六十里。明沿元制,省入臨晉。雍正八年復析置。南:中條山,有王官穀。西南:五老、蔥聾、方山。其北檀首,其北五姓湖。水經注,張陽池有鴨子池,合中條水。東北涑水自臨晉入,會姚暹渠,並匯焉。又西入永濟。故市鎮。湯家驛。榮河難。府東北百二十里。城東:峨眉嶺。西北:河水自河津入,汾水入焉。古汾睢湮。即春秋葵丘。南逕城西入臨晉。胡壁、孫吉二鎮。陽陵驛。猗氏沖,繁。府東北一百十里。東有涑水自安邑入,西南入臨晉。有張岳鎮。萬泉難。府東北百六十里。東:介山,其西峰孤山。城南山陰暖泉。又東澗。解店鎮。

解州直隸州:繁,難。河東道兼鹽法駐安邑運城。明,平陽屬州。領縣五。雍正二年升,割聞喜易其垣曲,尋並隸絳。東北距省治九百五十里。至京師千四百五十里。廣二百二十里,袤百四十里。北極高三十四度五十八分。京師偏西五度三十八分。領縣四。東南:中條山,其脊橫嶺,又白徑、分雲。其北鹽池。又北鹽水,今姚暹渠,自安邑入,逕其北,西入虞鄉。城西北硝池,濁澤。長樂鎮。長樂、鹽池二巡司。安邑沖,繁,難。府東北五十五里。東南:吳山。南:中條。北:鳴條。西南:鹽池。池北運城。河東道及州判駐。東有苦池。東北:姚暹渠,即鹽水,自夏入,逕城北,又西南逕運城北入州。又東北,涑水自夏入,西入猗氏。鎮二:東郭、聖惠。有巡司。浤芝驛。夏沖,繁。府東北百里。南:柏塔山。東北:翠巖、稷山。東南:溫泉。巫咸山,鹽水出,今姚暹渠,西北逕雲谷至城南折西。西北:涑水自聞喜入,南逕夏后陵,並入安邑。曹張、胡張、尉郭、水頭、裴介五鎮。平陸簡。府東南九十里。東北:虞山,上有虞城。其西傅巖、清涼山。西北:天井、卑耳。西南:河水自芮城入,逕洪池,至茅津渡。中條山諸澗,迤東北至砥柱。砥柱禹鑿,六峰、三門山在焉。納劉家溝、後溝、積石水,入垣曲。鎮六:洪池、張店、張谷、常樂、葛趙、茅津。縣丞駐茅津。有廢巡司。芮城難。府西南七十里。北:橫嶺、漱水嶺,洪源澗出,會葡萄澗、地皇泉。西南:河水自永濟入,逕魚鱗磧,至城南。又東,水豆水入,迤北入平陸。

絳州直隸州:繁,難。隸雁平道。明,平陽屬州。領稷山、絳、垣曲。雍正二年升,並割太平、襄陵、河津來隸,以絳屬平陽,垣曲屬解。七年,又割聞喜、絳,垣曲復,而太平、襄陵還舊隸。東北距省治七百十里。至京師千八百里。廣四百里,袤百里。北極高三十五度三十八分。京師偏西五度十三分。領縣五。南:峨眉嶺,即晉原。西北:馬首山。北:九原;鼓山,古水出,即清濁二泉。東北:汾水自太平入,至城南。左會澮水,水經注「逕王橋,澮水入焉」者。又西南,合古水入稷山。南:重興關。西:武平關。垣曲繁。州東南二百十里。雍正二年隸解,七年復。東北:諸馮山、王屋。其北教山,教水出。水經注「南歷鼓鐘、上峽、下峽、馬頭山」者,亦曰沇水。清廉,俗風山,清水出其西嶺,東逕皋落鎮,會亳水及白水,曰亳清河。西南:河水自平陸入,逕鷹嘴山,入河南濟源。河水入晉境,冷行二千七百餘里。鼓鐘鎮、迎駕、六郎鎮。聞喜沖,繁,難。州南七十里。初隸平陽。雍正七年改。東:鳳皇原。東南:香山。湯寨,古景山,景水出,實水經注沙渠水。其北美良川。東北:紫金,古三。涑水自絳縣入,逕董泊,右合甘泉,復左合景水,逕城南;又西入夏。詩「揚之水,不流束薪」者。鎮八:上東、下東、橫水、裴社、宋店、慄村、郭店、蘭德。涑川一驛。絳簡。州東南八十里。初隸平陽。雍正七年改。絳山,西北二十里。北:牛村。東北:備窮。澮水自翼城錯入,合故郡水,又西北入之。東南:回馬嶺,絳水出。水經注所謂「出絳山東,寒泉奮湧,揚波北注」者。其西華池有陳村峪水,實注所謂乾河。西逕大陰山,合紫谷水,又西會煙莊冷口水。水經「出聞喜黍葭谷,逕存雲嶺入聞喜南絳故城」。鎮曰澮交。稷山難。州西五十里。稷山,南五十里。北:姑射、聖王。東南:汾水自州入,逕城南,又西,華水故道出焉,入河津。小河、翟店、下迪、大杜四鎮。河津沖,繁。州西百里。初隸平陽。雍正二年改。東北:黃頰山。西北:河水自鄉寧入,逕龍門山。禹貢「自積石至」者。韓原在焉,所謂少梁。又南入榮。東南:汾水自稷山入,逕疏屬、仙掌山,又西南從之。鎮四:方平、禹門、東張、僧樓。禹門,巡司駐。

隰州直隸州:繁。隸河東道。明,平陽屬州。領大寧、永和。雍正二年升,並割汾西。九年,又割吉之蒲屬之,而汾西還舊隸。東北距省治五百五十里。至京師千七百里。廣二百六十里,袤二百三十里。北極高三十六度三十九分。京師偏西五度三十一分。領縣三。北:妙樓山。東:五鹿。東北:蒲子。其界石樓者有水頭村,蒲川水出,西南合回龍、交口水,逕城西,又東南會義泉河於仵城鎮北。水經注所謂「出石樓山,南逕蒲城蒲子縣,得黃櫨谷水」者,俗曰隰川,入大寧。義泉、張家川、羅真、蒿城、康成、大麥、辛莊、西曲、回龍九鎮。又廣武鎮,巡司駐。大寧簡。州西南九十里。城南:翠微山。西南:石子。西北:孔山。河水自永和入,逕馬鬥關。東北:隰川,即蒲川水,自州入,逕羅曲鎮,折西,逕城出,至藍公山。其南源第一河東南自蒲入,實紫川水,合縣底河入焉,又東南入吉。蒲川水莽灌數百里,元和志日斤水,寰宇記日斤水,明志因誤昕水,方乘從之,非也。一鎮:安阜。蒲簡。州東南百二十里。舊屬平陽。雍正二年屬吉,九年來隸。東:東神山。西南:翠屏。東北:姑射。有分水嶺。蒲水南源第一河出,水經注「紫川西會南川所謂合江水」者。逕城東南,右合東小河,又西入大寧。鎮六:化樂、張村、喬麥灣、薛關、古驛、松峪。永和簡。州西北九十里。東:雙山。南:樓山。西:烈鳳、馬脊。東北:佶北。其南仙芝谷,古域谷。西北:河水自石樓入,逕老牛灘,仙芝河合索陀川、榆林河,水經注「域谷水東啟荒原,西歷長溪」者。至城西南,合甘露河入焉,又南入大寧。桑壁、岔口、劉臺三鎮。

霍州直隸州:沖,繁,難。隸河東道。明,平陽屬州。領靈石。乾隆三十七年升,復割趙城來隸。東北距省治四百六十里。至京師千五百五十里。廣八十里,袤二百三十五里。北極高三十六度三十五分。京師偏西四度四十四分。領縣二。霍山,東南四十里,禹貢太嶽。彘水出石鼻穀。西北:汾水自靈石入,逕靈佛巖合之。水經注「逕觀阜北」者。入汾。辛置鎮。霍山驛。趙城沖。州南五十里。乾隆三十七年自平陽來隸。東北:霍山,霍水出。西:羅雲。西北:汾水自汾西入,逕城西,西北合之,南入洪洞。有驛。靈石沖。州北百里。乾隆三十七年自平陽來隸。東:孝文山。東南:尖陽、十八盤。東北:靜巖、綿山,有五龍泉,俗小水河。汾水自介休入,至城西北,左合之,屈南,右合石門峪、新水峪;左仁義河,逕陰地關入州。水經注「又南過冠爵津,俗雀鼠谷」者。其南高壁嶺,今韓信嶺。鎮二:夏門、仁義。驛二:石、仁義。驛丞兼巡司駐仁義鎮。

大同府:沖,繁,難。總兵駐。初因明制,領州四,縣七。雍正中,增陽高、天鎮,改朔及馬邑隸朔平,蔚及黃昌分隸直隸宣化、易州。南距省治六百二十里。至京師七百二十里。廣二百五十里,袤二百六十里。北極高四十度五分。京師偏西三度十二分。領州二,縣七。大同沖,繁,難。倚。順治五年徙西安堡,九年復。北:紇干山。東:白登,其東牛皮嶺。迤北少咸,敦水出。西南:採掠。桑乾水自應入,逕其南,右合馬耳山水,左有玉河如渾水,自豐鎮入,右逕方山合卷子,左鎮川河,又南逕孤山村,右納小泗水,至城東,又南,右納肖畫河,水經注所謂「右會武周川」者,又南來會。又東,敦水出少咸山;逕西堰頭,並入陽高。甕城、聚樂二驛。懷仁沖。府西南八十里。西:清涼山、錦屏。西南:蘆子。新莊子河出其村,逕大於口入山陰。有安宿峒鎮。西安驛。渾源州難。府東南百二十里。順治十六年,安東中、前二所省。西南:龍山。西北:晝錦。北:龍角。東南:恆山,北嶽,順治十七年自曲陽移祀於此。山高三千九百丈,周回數千里,橫跨燕、趙,屏蔽京師。曲陽其趾,阜平其脊,州境其主峰也。其別阜南曰槍峰嶺,古高氏山,唐河上源滱水出,周禮所謂「嘔夷,並州川」。左會別源翠屏山水,水經注所謂侯塘川,東逕蔡家峪入靈丘。其溫泉堙。嶺之西北渾河出,一曰崞川,西北匯別源亂嶺關及瓷窯峽、李峪、神谷、橫山諸水,入應。王家莊堡,巡司駐。上盤驛。應州沖,府南百二十里。順治十六年安東中屯衛省入。雍正八年罷所隸故城州。東南:茹越山。東北:龍首。西南:龍灣。西:桑乾河自山陰入,逕州東北,渾源河自州來會。水經注「逕巨魏亭北,又東,崞川注之」者。亦通曰渾河。又東北,入大同。一鎮曰安邊。安東衛巡司。安銀子驛。山陰沖。府西南百八十里。南:復宿山、香山。西:桑乾河自朔入,至城北,折東南,逕黃花山,即黃瓜埠,右合黃水河入應。岱嶽站,巡司駐。有驛。陽高沖。府東北百二十里。雍正三年,以陽高衛降置。西:斷頭山、龍混。北:虎頭、雲門。西南:白登山,敦水自大同入,逕其麓。南洋河自豐鎮入,南流,逕守口堡入邊。右合馬邑水,逕城北,又東南會白登河入天鎮。西南:桑乾水自大同入,逕黃土梁,又東並入天鎮。天鎮沖。府東北百八十里。雍正三年以天鎮衛改置。北:環翠山。東:陽門,其幹神頭,其支豐稔。西南:牽牛。桑乾水自大同入,逕嘴兒圖,左合五泉河、石門溝。五泉古安陽水,陽原故城在焉。又東,入直隸西寧。其北南洋河自陽高入,逕福祿山。水經注「雁門水東北入陽門山,謂之陽門水」者。右合三沙河,古醄水,逕城北,又東北逕摺兒嶺入懷安。又北,西洋河自豐鎮入,右合南溝水,逕暖泉墩,及東南小溝口河,亦入懷安。廣靈簡。府東南二百四十里。東南:加斗山。北:千福。西北:九竫。西:望狐;白羊,壺流河出。莎泉,祁夷水,東南逕石梯嶺,合作甿池。枕頭河逕城南,又東逕壺山,入直隸蔚州,達桑乾,為南支。直峪、林關、火燒、樺澗四鎮。馬廠驛。靈丘沖。府東南二百七十里。南:太白山。西北:漫山,其東枚回嶺,古滋水出焉。滱水自渾源入,左合黑龍河,逕城南,又東南逕隘門山、銀釵嶺入直隸廣昌。驛一:太白。

朔平府:沖,繁,難。明,右玉林、左雲川、平虜三衛地,屬山西行都司。清初為右玉、左雲、平魯三衛。雍正三年,於右玉衛置府,並改三衛為縣,屬雁平道。南距省治六百七十里。至京師九百六十里。廣二百十里,袤二百九十里。北極高四十度十一分。京師偏西四度十一分。領州一,縣三。右玉沖,繁。倚。雍正三年以右玉衛改置。玉林山,西二十里。東南:石堂山、紗帽。西南:滄頭河自平魯入;右合牛心山,左孫家川、雲石堡水,屈北,逕府治西。右合範家堡水、馬營河,又北,右會兔毛河。西北有邊墻,西南接平魯,東北至右玉,有殺虎、水柵、鐵山、大沙、雲石等口。威遠堡、殺虎口二巡司。朔州沖,繁,難。府東南二百四十里。明屬大同府。雍正三年來隸。嘉慶元年,所領馬邑省入為鄉。有鄉學訓導。東北:契吳。東:洪濤,其支阜雷山。左黃道泉,右金龍池,桑乾水出,水經注所謂「洪源七輪即溹涫水」者。東南匯於臘河口,古馬邑川水南源。恢河,古★水,自寧武入,逕城南,折東,右合七里河,左沙棱水,又東北至下館故城北來會,入山陰。城東、廣武二驛。左雲沖。府東南七十五里。雍正三年,左云衛改置。東北:彌陀山。東:雕嶺。東南:龍王堂。南:南石,肖畫河出,北逕城西南,右合溫泉,又北折東,左合龍泉,逕焦山,又東南入懷仁。舊有助馬堡巡司,裁。平魯沖。府西南百十五里。雍正三年以平魯衛改置。南:十二連山。西南:迎恩。西:小青。西北:七介、西平、磨兒。清水河出,入其,古樹頹水。城內北固山。北:尖山。東南:天門,相近奎星臺。北嶽峰,蓋水經注大浴真山,滄頭河出。古中陵水,西北貫城出,折東逕碧峰山入右玉。樂寧、伏遠二鎮。

寧武府:沖。隸雁平道。明置寧武關並所。嘉靖中置三關鎮,駐寧武。又偏寧道駐偏頭,後改岢嵐、寧武二道分駐。清初,前後並廢。雍正三年,改所為府,置附郭,偏關所、神池堡、五寨堡為縣。南距省治三百四十里。至京師九百五十里。廣二百九十里,袤三百六十里。北極高三十九度六分。京師偏西四度十一分。領縣四。寧武沖。倚。明置寧化所。雍正三年為府,並置。西南:管涔山,其東天池,其下分水嶺。西出者汾水,左會林溪,樓子山別源,折西南,逕寧化堡,入靜樂。東出者恢河,一曰渾河,古★水,水經注「出累頭山」,地理志謂之治水者,東北逕城南,又東北逕陽方口,出邊入朔,為桑乾南源。有陽方堡。寧化所巡司。偏關沖。府北百八十里。明置守御所。雍正三年改。東:丫角山。北:蠶虎。西北:河水左瀆自清水河入,逕老牛灣西,又西南,東有關河自平魯入,合紅水溝,逕南,又西北入焉。又西南入河曲。老營堡有廢巡司。神池沖。府北三十里。明置神池口巡司,後增神池堡營。雍正三年改。南:黃花嶺。西南:旗山、虎北、洪佛。北:達沐河,西逕磨石山,左合義井河。河本渭流,康熙三十六年聖祖西征,飲馬駝於此,賜名興隆。折北入五寨。五寨沖。府西百里。明建五寨城。雍正三年改。西南:蘆芽山,管涔絕頂也,高三千丈,上有彌連池,即彌澤,下注清漣河,東北達沐河自偏關入會之,為大澗河,折西入河曲。有三岔堡。

忻州直隸州:沖,繁。明,太原屬州。雍正二年升,仍領定襄,割太原之靜樂來隸。西南距省治百四十里。至京師千三百里。廣三百六十里,袤百里。北極高三十八度二十五分。京師偏西三度四十三分。領縣二。南:系舟山。西南:雲母。西:九原。西北:雲中,相屬雙尖,雲中水出,東北入崞,會忻川,注滹沱。滹沱復錯入,入定襄。西南:白馬山,牧馬河出,古三會水,合陀羅、大嶺、清水諸山水,東北逕城南從之。九原一驛。定襄繁。州東五十里。東南:七巖山。東北:聖阜。西北:橫山。滹沱水自州入,東南逕城北,又東北,右會牧馬河,入五臺。南有叢蒙山,三會泉出,北流注牧馬河。西北有滹沱渠,資灌溉。一鎮:芳蘭。靜樂沖。州西百八十里。雍正二年自太原來隸。東:兩嶺山。東南:天柱。西北:管涔,汾水出其陰,自寧武入,逕馬頭山,至城西南,左合碾河,右納嵐水,又東南逕樓煩鎮,右合石樓、臨春山水,入交城。西南:離石水入永寧。鎮三:故鎮、窟谷、永安。又有樓煩鎮巡司。康家會驛。

代州直隸州:沖,繁,難。雁平道駐。明,太原府屬州。雍正二年升,仍所領。西南距省治三百二十里。至京師七百七十里。北極高三十九度六分。京師偏西三度三十二分。領縣三。西北:句注,其嶺太和,唐置雁門關,古曰西隃,隘有十八。其東夏屋,中峰曰復宿。東南:舜山、圭峰。東:滹沱河自蘩峙入,左納峨水,右合夏屋、雁門水,逕城南。又西南,右合羊頭神河入崞。雁門關,一驛。五臺難。州東南百四十里。五臺山,東北百二十里,一名清涼山。聖祖、高宗、仁宗前後十三巡幸。中臺有太華池水,西北流,會縣北峨嶺水,出峨口入繁峙。北:錦屏。西北:鐵嶺。西:紫羅。滹沱河自定襄入,逕東冶鎮,左合慮虒水、清水河,又東南入盂。東:烏牛山,滋水出,東流入直隸平山。鎮三:竇村、東冶、臺懷。巡司駐臺懷。崞沖。州西南八十里。崞山,西南四十五里。其西黃嵬。南:前高。西北:柏枝。東北:滹沱河自州入,逕城東,又南,右合羊虎谷水,又西南,雲中河自忻入,會忻川入忻。原平、鬧泥二驛。繁峙簡。州東六十里。北:茹越山。東南:憨山、小五臺。東:泰戲。滹沱水出泰華池,一曰派水,並州川。說文「起雁門郡葰人縣戊夫山」者。西會三泉,伏流,匯華巖諸水,復出,逕沙澗驛,至城南入州,峨水從之。其東巖頭有白坡,沙河出,南入直隸阜平,古恆水支。平刑關,巡司駐。平刑、沙澗二驛。

保德直隸州:沖,繁。隸雁平道。明為太原屬州。雍正二年升,並割河曲、興來隸。八年,興還隸太原。東南距省治四百六十五里。至京師千七百十五里。廣二百十里,袤百十一里。北極高三十九度四分。京師偏西五度四十分。領縣一。城南:蓮花山。東南:馬頭。西南:羊頭。東北:石梯。河水左瀆自河曲入,逕城北,屈西至花園堡,壺廬山水入,為硃家川。又西南,合裴家川入興。河曲沖。州東北百二十里。明隸太原。雍正二年改。乾隆二十九年徙河保營為今治,東阻險山。南:翠峰。西南:火山。東北:河水左瀆自偏關入,逕城西大辿渡。又西南,東有清漣河自五寨入,為六澗河入焉,古彌澤,入州。壺廬山水從之。河邑巡司駐舊縣。乾隆二十九年徙治河保營,即今治。

歸化城直隸:沖,繁,疲,難。歸綏道鎮守副都統駐。明嘉靖中,蒙古據豐州,是為西土默特,駐牧建城,後封順義王,名其城曰歸化。天聰八年內附。順治三年置左右翼及四副都統。雍正元年置理事同知,駐西河,隸朔平府。乾隆元年增協理通判二,增綏遠。六年置歸綏道,及二協隸。二十五年省協理,徙同知駐城。裁左右翼及副都統。餘副都統一,同駐。光緒十年改撫民同知。南距省治九百六十里。至京師千一百八十里。廣百八十里,袤二百九十里。北極高四十度四十九分。京師偏西四度四十八分。北:大青山,即陰山,古白道川。其支阜,西石綠,西北克壽,東北烏蘭察布、喀喇克沁、鍾山。金河,古芒干水,俗大黑河,西南逕南。左合小黑河,即紫河,古武泉水。又西南,右合哈爾幾河,入托克托。克魯庫河,古白道,中溪水從之。卡倫二十有二。臺站四。有巡司,一在城,兼司獄,一在畢齊克齊。有遞。

薩拉齊直隸:沖,繁,疲,難。隸歸綏道。明初,雲內州,後為雲內縣,屬豐州,尋廢。乾隆四年,置薩拉齊及善岱二協理通判。六年,隸歸綏道。二十五年,改理事,以善岱協理通判省入。同治四年,改置同知。光緒十年,改撫民。東南距省治千二百里。至京師千四百二十里。廣二百五十里,袤百里。北極高四十度三十九分。京師偏西五度十六分。又兼轄鄂爾多斯左翼後旗地。廣四百三十里,袤二百二十里。西北:牛頭、朝那山、夾山。北:宿嵬。東:拜轟克兒。河水左瀆自五原南界東流入境,包頭、五當河並北來注之,逕沙爾沁村,又東至南,合蘇爾哲、帽帶河,入托克托。察蘇河入托克托。卡倫五。臺站一,在治。有巡司兼司獄一,駐包頭鎮。有遞。

清水河直隸:繁,疲,難。隸歸綏道。明,置東勝衛千戶所。乾隆元年,置協理通判。六年,隸歸綏道。二十五年,改理事。光緒十年,改撫民通判。東南距省治九百二十里。至京師千又二十里。廣百三十五里,袤百四十里。北極高四十度六分。京師偏西四度四十八分。東:鄂博圖山、連嶺。東南:吐頹,有君子津。西北:河水左瀆自托克托入,逕紅山口,東南清水河自平魯入,右合湯溪河,西北逕三叉河至南。又西北出古長城,左會兔毛河,亦稱紅河,古中陵水入焉;又南入偏關。有巡司兼司獄在治。有遞。

豐鎮直隸:繁,疲,難。隸歸綏道。明,大同及陽和、天成二衛邊外地。康熙十四年,徙察哈爾蒙古部駐。雍正十二年,置豐川衛及鎮寧所,大朔理事通判統之。乾隆十五年改置,大同、陽高通判徙駐。三十三年,還故治,增置大同理事同知。光緒十年,改撫民。南距省治六百七十里。至京師八百六十里。廣二百三十里,袤二百二十里。北極高四十度三十分。京師偏西三度十二分。北:尖子山、狼頭。西北:留雲。東:盤羊。東北:大青、牛心。其西南壺盧海,如渾水出,今曰玉河。屈西南,左合大科莊水,古旋鴻池,逕古慶梁,右合尖子山水,至東南新城灣。右會得勝河,古羊水,入大同。東北:五祿戶灘,東洋河出,古修水,亦於延水,東逕碾房窯,入直隸張家口。其南胡魯蘇臺,西洋河出,古延鄉水,及南溝水,逕馬市口入天鎮。又西清涼嶺,南洋河出,古雁門水,南逕守口堡入陽高,並達之。自東洋外,並逾邊。巡司三:一駐城,兼司獄;一二道河;一張皋爾。二道河後改設興和。有遞。

托克托直隸:繁,疲,難。隸歸綏道。明,東勝左衛。嘉靖中入土默特。曰脫脫,亦曰托克托。乾隆元年,增協理通判。二十五年,改理事。光緒十年,改撫民通判。東南距省治八百六十里。至京師千一百里。廣八十五里,袤一百三十里。兼轄河西鄂爾多斯右翼後旗地。廣百三十里,袤百五十里。北極高四十度三十分。京師偏西四度四十分。南:紅山,古緣胡。西北:河水左瀆自薩拉齊入。大黑河東自歸化入,左會黃水,又西,右會克魯庫,至東北會察蘇河。逕北,舊匯為黛山湖,古芒干水,合南源白道中溪塞水注沙陵湖者,又西入焉。又南入清水河。有巡司兼司獄。有遞。

寧遠直隸:沖,疲,難。隸歸綏道。明,宣德衛。後為大同邊外地。康熙十六年,察哈爾部析駐。雍正十二年,置寧朔衛及懷遠所,大朔理事通判統之。乾隆十五年省改,徙朔平通判駐。二十一年,改理事通判。光緒十年,改撫民。南距省治八百十里。至京師千又二十里。廣百八十里,袤二百九十里。北極高四十度二十一分。京師偏西三度五十二分。東:猴山。北:倉盤、汗漫、平頂。黑河南源永興河出,古白渠水。其南參河陘,今西溝門,古沃水出,今曰寧遠水。南逕將軍梁,左合寧遠堡水,古可不泥,逾長城。東北:平頂永興河東四道凹,得勝河出,逕豐城溝入豐鎮。其北大海,古諸聞澤,周百餘里。其南小海,地理志鹽澤,古通目曰參合陂。有巡司。有遞。舊有科布爾巡司,後改設陶林。

和林格爾直隸:繁,疲,難。隸歸綏道。明置玉林、雲川二衛。後為蒙古西土默特據。康熙中,置站曰二十家子,蒙語和林格爾。乾隆元年,置協理通判。二十五年,改理事。光緒十年,改撫民通判。南距省治八百四十里。至京師千六十里。廣百七十里,袤百八十里。北極高四十度二十分。京師偏西四度二十四分。東:九峰山。西:摩天嶺。南:大松。東南:玉林。兔毛河自右玉入,逾邊逕殺虎口,右會寧遠河,逕其麓,西北至。西南折西入清水河。東北黃水自寧遠入,西逕北入托克托。有巡司兼司獄。有遞。

興和直隸:明初,天城衛邊北地。光緒二十二年,以豐鎮之二道河巡司置,隸歸綏道。西南距省治八百九十里。至京師千七十里。廣袤闕。北:大青山。東南:水泉入。西北:東洋河自察哈爾旗入。二源合,東西逕北,入直隸張家口。有遞。

陶林直隸:要。隸歸綏道。明,大同邊外地。光緒二十九年,以寧遠之科布爾巡司置。西南距省治千三百里。至京師千四百五十里。廣袤闕。北:伊馬圖山。南:回頭梁。大黑河南源黃水河,古白渠水,出大東溝,西南逕五壩入寧遠。有遞。

武川直隸:要。隸歸綏道。明,西土特默牧場。光緒二十九年,以其北境翁滾置,治烏蘭花,寄治歸化城。南距省治千百七十里。至京師千二百九十里。廣袤闕。北:托克圖山。西北:克壽。東有烏蘭察布源泉,治。一遞。

五原直隸:要。隸歸綏道。光緒二十九年,析薩拉齊西境興盛旺置撫民同知,治隆興長,寄治包頭鎮。東南距省治千七百九十里。至京師二千一百十里。北極高四十度三十九分。京師偏西五度十六分。西北:陽山。北:陰山。河水自甘肅邊外環內蒙鄂爾多斯,折東自烏拉特循其南麓入。有鄂博口,古稒陽道。又東逕南,合博托河入薩拉齊。有遞。

東勝直隸:要。隸歸綏道。明初,東勝衛西界、陜西榆林衛河套地,後為元裔所居。光緒三十二年,以鄂爾多斯左翼中郡王右翼前末扎薩克旗墾地置,治羊壕廠,寄治薩拉齊之包頭鎮。北極高四十度四十九分。京師偏西四度四十八分。西北:河水自鄂爾多斯循五原入北,折東南入薩拉齊。邊墻西自陜西榆林入。又東有遞。


\end{pinyinscope}