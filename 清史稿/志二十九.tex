\article{志二十九}

\begin{pinyinscope}
地理一

有清崛起東方,歷世五六。太祖、太宗力征經營,奄有東土,首定哈達、輝發、烏拉、葉赫及寧古塔諸地,於是舊籓札薩克二十五部五十一旗悉入版圖。世祖入關翦寇,定鼎燕都,悉有中國一十八省之地,統御九有,以定一尊。聖祖、世宗長驅遠馭,拓土開疆,又有新籓喀爾喀四部八十二旗,青海四部二十九旗,及賀蘭山厄魯特迄於兩藏,四譯之國,同我皇風。逮於高宗,定大小金川,收準噶爾、回部,天山南北二萬餘里氈裘湩酪之倫,樹頷蛾服,倚漢如天。自茲以來,東極三姓所屬庫頁島,西極新疆疏勒至於蔥嶺,北極外興安嶺,南極廣東瓊州之崖山,莫不稽顙內鄉,誠系本朝。於皇鑠哉!漢、唐以來未之有也。

穆宗中興以後,臺灣、新疆改列行省;德宗嗣位,復將奉天、吉林、黑龍江改為東三省,與腹地同風:凡府、、州、縣一千七百有奇。自唐三受降城以東,南衛邊門,東湊松花江,北緣大漠,為內蒙古。其外涉瀚海,阻興安,東濱黑龍江,西越阿爾泰山,為外蒙古。重之以屏翰,聯之以昏姻,此皆列帝之所懷柔安輯,故歷世二百餘年,無敢生異志者。

太宗之四征不庭也,朝鮮首先降服,賜號封王。順治六年,琉球奉表納款,永籓東土。繼是安南、暹羅、緬甸、南掌、蘇祿諸國請貢稱臣,列為南服。高宗之世,削平西域,巴勒提、痕都斯坦、愛烏罕、拔達克山、布哈爾、博洛爾、塔什干、安集延、浩罕、東西布魯特、左右哈薩克,及坎車提諸回部,聯翩內附,來享來王。東西朔南,闢地至數萬里,幅員之廣,可謂極矣。洎乎末世,列強環起,虎睨鯨吞,凡重譯貢市之國,四分五裂,悉為有力者負之走矣。

清初畫土分疆,多沿明制,歷年損益,代有不同。其川瀆之變易,郡邑之省增,疆界之分合,悉詳稽圖志,並測斗極定高偏度,以畫中外封域廣輪曲折之數,用備一朝之掌焉。

△直隸

直隸:禹貢冀、兗二州之域。明為北京,置北平布政使司、萬全都指揮使司。清順治初,定鼎京師,為直隸省。置總督一,曰宣大。駐山西大同,轄宣府。順治十三年裁。巡撫三:曰順天,駐遵化,轄順天、永平二府。康熙初裁。曰保定,駐真定,轄保定、真定、順德、廣平、大名、河間六府。順治十六年裁。曰宣府。駐宣府鎮,轄延慶、保安二州。順治八年裁。五年,置直隸、山東、河南三省總督。駐大名。十六年,改為直隸巡撫。明年移駐真定。康熙八年,復移駐保定。雍正二年,復改總督。而府尹舊治順天,為定制。先是順治十八年增置直隸總督,亦駐大名。康熙五年改三省總督,八年裁。唐熙三十二年,改宣府鎮為宣化府。降延慶、保安二州隸之。雍正元年,置熱河,改真定為正定。二年,增置定、冀、晉、趙、深五直隸州,張家口。三年,升天津衛為直隸州,九年為府。十年,置多倫諾爾。十一年,熱河、易州並為直隸州。十二年,置獨石口。降晉州隸正定。乾隆七年,承德仍為熱河。八年,遵化升直隸州。四十三年,復升熱河為承德府。光緒二年,置圍場。隸承德。三十年,置朝陽府。明年置建平隸之。三十三年,升赤峰縣為直隸州。置開魯等二縣隸之。今京尹而外,領府十一,直隸州七,直隸三,散州九,散一,縣百有四。北至內蒙古阿巴噶右翼旗界;一千二百里。東至奉天寧遠州界;六百八十里。南至河南蘭封縣界;一千四百三十里。西至山西廣寧縣界。五百五十里。廣一千二百三十里,袤二千六百三十里。宣統三年,編戶共四百九十九萬五千四百九十五,口二千三百六十一萬三千一百七十一。其山:恆山、太行。其川:桑乾即永定、滹沱即子牙、衛、易、漳、白、灤。其重險:井陘、山海、居庸、紫荊、倒馬諸關,喜峰、古北、獨石、張家諸口。交通則航路:自天津東南通之罘、上海,東北營口,東朝鮮仁川、日本長崎。鐵路:京津,津榆,京漢,正太,京張。郵道:東出山海關達盛京綏中,西出紫荊關達山西靈丘,南涉平原達山東德州,北出古北口達熱河。電線:西北通庫倫,西南通太原;由天津東北通奉天;海線自大沽東通之罘。

順天府:明初曰北平府。後建北京,復改。自遼以來皆都此。正統六年,始定曰京師。領州六,縣二十五。順治初,京師置府尹、府丞、治中。其順天巡撫駐遵化,康熙初裁。十五年,升遵化為州。二十七年,置四路同知,分轄所屬州、縣。分隸通永、霸昌二道。並兼統於直隸總督。雍正元年,復以部院大臣兼管府事,特簡,無定員。九年,置寧河。乾隆八年,遵化復升直隸州,以玉田、豐潤屬之。廣四百四十里,袤五百里。北極高三十九度五十五分。領州五,縣十九。遼,南京,今城西南,唐幽州籓鎮城也。金增拓之。至元而故址漸湮。元之大都,則奄有今安定、德勝門外地。明初縮城之北面,元制亦改。永樂初,重拓南城,又非復洪武之故矣。今皇城周十八里。自正陽門之內曰大清門;東南曰長安左門;西南曰長安右門;東曰東安門;西曰西安門;正北曰地安門,舊曰北安門,順治九年更名。大清門之內曰天安門,舊曰承天門,順治八年改。左太廟,右社稷壇。鄉明而治,於茲宅中焉。其內端門,左闕左門,右闕右門。其內紫禁城在焉。北枕景山,西衽西苑,苑有瀛臺,太液池環之。南與端門屬者曰午門。北神武門,東東華門,西西華門。午門之內,東協和門,東出為文華殿;西熙和門,西出為武英殿,舊曰雍和門,乾隆元年更名。其正中太和門,左昭德門、體仁閣,右貞度門、宏義閣;其內則太和、中和、保和三殿,至乾清門止。東景運門,西隆宗門。凡此皆曰外朝,制也。外則京城,周四十里,為門九:南為正陽門,南之東崇文門,南之西宣武門,東之南朝陽門,東之北東直門,西之南阜成門,西之北西直門,北之東安定門,北之西德勝門。皆沿明舊。而八旗所居:鑲黃,安定門內;正黃,德勝門內;正白,東直門內;鑲白,朝陽門內;正紅,西直門內;鑲紅,阜成門內;正藍,崇文門內;鑲藍,宣武門內。星羅棋峙,不雜廁也。外城長二十八里,為門七:南為永定門,左左安門,右右安門,東廣渠門,西廣寧門;在東、西隅而北向者,東東便門,西西便門。並明嘉靖中築。鼓樓在地安門外,明永樂中毀,乾隆十二年重建。大興沖,繁,疲,難。倚。府東偏,隸西路。北有榆河,自昌平入,納清河。西北:玉河,自宛平入。歧為二:一護城河,至崇文門外合泡子河;一入德勝門為積水潭,即北海子,流為太液池,分為御溝。又合德勝橋東南支津,復合又東,為通會河。涼水河亦自宛平入,逕南苑,即南海子,龍、鳳二河出焉。龍河淤。南路駐黃村。縣丞駐禮賢莊。有青雲店、鳳河營、白塔村三鎮。有採育營巡司。有驛。鐵路。宛平沖,繁,疲,難。倚。隸西路。西山脈自太行,為神京右臂。西北二十里甕山,其湖西海。乾隆十五年賜山名曰萬壽,湖曰昆明。有清漪園,光緒十五年改曰頤和。相近玉泉山,清河、玉河源此。玉河逕高梁橋,一曰高梁河。永定河自懷來入,至盧師山西,亦曰盧溝河,錯出復入。有灰壩、減河。汛十二,石景山有南北岸同知:全轄者七,石景山、盧溝橋二、北頭工上、北頭工中、南頭工上、北二工下;分轄者五,南頭工下、北頭工下、北二工上、南三工、北三工。自順治八年至同治三年,改道十有六,截北流歸中泓,逕魚壩口、三鳳眼入海。蓋道光二十二年以來,雖小潰徙,無害。又涼水、檿牛、龍泉三河兼出西南。西有海澱,有暢春、圓明二園,咸豐末毀。西路駐盧溝橋,有巡司。縣丞駐門頭溝。又龐各莊、青白口、東齋堂巡司三。沿河口、磨石口、榆垡、平羅營、五里坨、趙村、王平口、天津關鎮八。鐵路。良鄉沖,繁,難。府西南七十里。隸西路。永定河自宛平入。汛四,並分轄,隸石景山南岸同知:北頭工下、北二工上、南頭工下、南二工。康熙四十六年建金門石閘,後廢。乾隆三年移建南二汛,改減水石壩仍曰金門閘。永定減水壩十有七。公村河自房山入,為檿牛河,復合茨尾河。盧河自房山入,逕琉璃鎮曰琉璃河,納挾活河。北有黃新莊行宮,南有郊勞臺。縣丞駐趙村。固節、長辛店二驛。鐵路。固安繁,難。府西百二十里。隸南路。永定河道南北岸同知、石景山同知駐。永定河自宛平入。汛六,隸南北岸同知,三角澱通判:全轄者二,南四工、北四工上;分轄者四,南三工、北三工、北四工下、南五工。拒馬岔河自涿入,舊有金門閘。減河亦自涿入,納太平河,曰檿牛河,歧為黃家河,其西蜈蚣河,並淤。東南十八里韓城。南七十里四鋪頭。有牛坨鎮。縣驛一。永清簡。府南少東百四十里。隸南路。三角澱通判駐。永定河自固安入。汛七,隸北岸同知:其通判全轄者三,南六工、北五工、北六工;分轄者四,北四工下、南五工、南七工、北七工。有信安鎮巡司,兼隸霸。東安簡。府東南百四十里。隸南路。永定河自永清入。汛三,並分轄,隸三角澱北通判:南七工、南八工上、北七工。其故道淤。鳳河自大興入。有舊州鎮。縣驛一。香河簡。府東南百二十里。隸東路。西有北運河,自通入。有王家務減河,雍正九年濬,長百四十里。北窩頭河亦自通入。縣驛一。通州沖,繁,疲,難。府東四十里。隸東路。通永道、倉場總督駐。順治十六年省漷縣入之。管河州判駐。白、榆、漒々三河並自順義入。榆納通惠河,與白會,是為北運河,納涼水河。漒々逕窩頭村曰窩頭河。鳳河自東安入。北門外石壩,州判掌之,十五京倉所漕。其東土壩,州同掌之,州西中二倉所漕。馬頭店、永樂店、馬駒橋三鎮。潞河、和合二驛。鐵路。三河沖,繁,難。府東少北百十里。隸東路。西北盤龍山有行宮,乾隆十九年移大新莊。北有泃河,自平谷入,側城東南。西南:窩頭河,自通緣界入。鮑丘河,古巨浸,源自塞外,淤。今出西北田各莊,晴為枯渠,雨則洶注,俗曰瀉肚河。有馬坊鎮。縣驛一。武清沖,繁,疲。府東南九十里。隸東路。西南:永定河自東安入。汛三,隸三角澱北岸通判:南八工上、南八工下、北七工。東北:北運河自香河入。康熙三十八年決筐兒港,明年濬為減河,後淤。同治末,復濬新減河。寶坻北有鳳河自通入,雍正四年改自堠上村折南,下至天津雙口入澱。三角澱一曰東澱,古雍奴藪,亙霸、文、東、武、靜、文、大七州縣境。雍正四年,放永定於澱,塞且半,僅王慶坨一角耳。乾隆十六年後,導河支貫澱而東,平蕪彌望。管河同知駐河西務,通判楊村,並有驛。八鎮:王慶坨、安平、桐柏、崔黃口、三里淺、南蔡村、筐兒港、黃花店。寶坻繁,疲,難。府東少南百八十里。隸東路。北:薊運河自薊右會泃河緣界入,逕江寬村,鮑丘河自三河入,納窩頭河,褒針河注之。又南有筐兒港新減河。其北王家務減河淤。知縣劉枝彥濬自大白莊至俵口,並修窩頭、褒針堤。有玉甫營鎮。縣驛一。寧河沖,繁,難。府東南三百里。隸東路。雍正九年改明寶坻之梁城千戶所置。海,東南九十里為北塘口。薊運河自寶坻入,屈曲環城而南,有七里海,匯王家務、筐兒港二減河,播為罾口,寧車、沽二河分注之,復納金鐘河。東南:大沽口界天津,海沙緣界入。其北北塘口。東南:盧臺鎮,天津河捕通判、通永鎮總兵駐。有巡司、鹽大使。北塘口、新河莊、營城三鎮。昌平州沖,繁,難。府北九十里。霸昌道駐。北路駐鞏華城,州隸之。北:天壽山,明十三陵在焉。西北:榆河自延慶入,伏而復出,左合山水,右納南沙河。又東,龍泉河會絳州營河注之。七渡河亦自延慶入。其南九渡河、檿牛河,並出東北。邊墻西首廟兒港口,東至糜子峪口。汛四:橫嶺路、鎮邊城、常峪城、白羊口。又訖慕田峪口,汛一:黃花路。湯山、藺溝行宮二。港泉營、牛房,𥫗蓌屯、沙屯、高麗營、藺溝、前營、前屯、皁角屯,凡九鎮。榆河驛,州治,及回龍觀,二。順義沖,難。府東北六十里。隸北路。北:牛欄山。白河自懷柔入,逕東麓,合懷河。其東狐奴山,漒々河出焉,一名箭桿河。絳州營河出縣西,納檿牛河。又榆河自大興入。三家店、南石槽行宮二。二鎮:漕河營、楊各莊。縣驛一。密雲沖,繁,難。府東北百三十里。隸北路。縣南:密雲山。東:九松山,舊曰九莊嶺。西有沽河,自灤平入,合白馬關河,是為白河。右出一支津。潮河亦自灤平入,合湯河,又納乾塔河,側城西南來會,俗亦曰潮白河。潮河營,提督駐。古北口關,副都統、巡司駐。西營二:石塘路、石匣城。汛二:潮河川、白馬關口。東營二:曹家路、墻子路。汛五:司馬臺、黑峪關、吉家營、楊家堡、鎮羅關。有劉家莊、羅家橋、要亭莊三行宮。鳳皇、石匣二驛。懷柔沖,繁。府東北百里。隸北路。棽髻山、祗園寺行宮二。石河出其東,下流為洳河。白河自密雲入,其支津亦自縣入,納雁溪水,復合。西:七渡河自昌平入,合九渡河,側城東南,合小泉河,曰懷河。有汛。縣驛一。涿州沖,繁,難。府西南百四十里。隸西路。西:獨鹿山。東北:永定河自良鄉入。其金門閘引河,淤。西北:拒馬岔河自房山分入而合,胡良河合杖引泉注之。至浮洛營東,挾活河錯入復出,注琉璃河。又東納檿牛河,淤,歧溝。西南:督亢陂。東南:古涿水,湮。有王家店、松木店、柳河營、馬溝村、長溝五鎮。涿鹿驛。房山繁,難。府西南九十里。隸西路。西南:大房山,一曰大防山,有溝山峰。雍正八年,鳳凰集此。又石經山。龍泉河,古防水,二源,出西北大安山,東南流,曰盧河。有沙河,環城,合壩兒河注之,是為琉璃河。拒馬河自淶水入,緣界逕鐵鎖崖,岔河出焉。歧為二。其東杖引泉。胡良河、挾活河並出西南,而茨尾河、雅河出東北。又順水河自宛平入。有磁家務巡司。有吉陽驛。霸州沖,繁。府南百八十里。隸南路。玉帶河自保定入為大清河。河南支徑苑家口曰會同河。中支中亭河,亦自保定入,逕栲栳圈,納檿牛河,又歧為北支,下流為辛張河,復錯入檿牛、黃家河,視永定為盈涸。北支,古運糧河。光緒初,游擊陳本榮濬之,復修蒼兒澱堤,植柳六萬一千株。行宮二:一太堡村,一蘇橋鎮。有主簿,兼隸文安。又信安鎮巡司,兼隸永清。有益津驛。文安繁,難。府南少東三百四十里。隸南路。大清河三支並自霸入,趨東澱。其北、中二支合於勝芳西,曰辛張河。文安窪周三百里,有火燒、牛臺、麻窪諸澱。光緒八年,濬臺頭以下河道,長千九百二十丈。左家莊有行宮。縣驛一。大城繁,難。府東南三百九十里。隸南路。西北:會同河自文安入,逕臺頭村,有行宮。大清河、辛張河並自文安入。子牙河自河間入,舊納古洋河,光緒中,改自獻之硃家口,故渠久湮。又黑龍港西支自青入,合東支河。保定簡。府南少西二百里。隸南路。西南:大清河自雄入,曰玉帶河,逕張青口,口西西澱,東東澱,乾隆二十八年界之。又北合趙王河,至盧各莊,康熙中,導為中亭河,合十望河入霸。縣驛一。薊州沖,繁。府東少北百八十里。隸東路。西北:盤山與桃花山、葛山,有行宮三。薊運河自明天順初引潮河溯今州,後廢。順治初復濬,以豐陵粢其上源。梨河東自遵化入,合淋河,至城南五里橋,始曰薊運河。折南,泃河出州北黃崖口外,錯出至三河,復緣界來會。汛四:黃花店、青山嶺、黃崖關、將軍石關。有漁陽驛。平谷簡。府東北百五十里。隸北路。東北:泃河自薊入,合獨樂河,側城西南,會石河,即洳河。縣驛一。

保定府:沖,繁,疲,難。隸清河道。明,領州三,縣十七。康熙八年,自真定移巡撫於此,為直隸省治。雍正二年,改總督。布政使、清河道等同駐。十二年,升易州為直隸州,以淶水屬之。又改深澤屬定州。道光中,省新安。東北距京師三百五十里。廣三百五十里,袤四百里。北極高三十八度五十一分。京師偏西五十二分。領州二,縣十四。清苑沖,繁,疲,難。倚。清苑河即府河,古沈水上游。奇村河自滿城入,合白草溝,環城,左納徐河溝,又東合金線河。唐河自望都入,合陽城河,納齊賢莊河,今淤;咸豐中,南徙;同治末,益南入蠡,至安州,復緣界入,下與府河會,為大清河中支。有大激店鎮,張登店巡司,金臺驛。鐵路。滿城沖。府西少北四十里。西南:抱陽山。西有渝河,自易州入而伏,至縣東湧為一畝、雞距二泉,合申泉,為奇村河。方順河自完縣入,歧為白草溝、金線河。徐河自易州入,一曰大冊河,東入安肅。千里長堤,首縣境,訖獻縣臧家橋,亙順、保、河三府。河丞駐方順橋鎮。有陘陽驛。安肅沖。府北少東四十五里。西有黑山。西南:益村嶺。雹河自易州入,合曲水河,至城北納雞爪泉河,下至新安入澱。其北萍泉河自定興入,東入容城,其支津自城西右出,與曹河並入安州。有梁門陂、白溝驛。鐵路。定興沖,繁。府北少東百二十里。北有拒馬自淶水入,逕城西而南,納中、北二易水及馬村河,緣界入容城、新城為界水。北又有界河。西南:雞爪河。東南:藍溝。有範陽陂、固城鎮、宣化驛。鐵路。新城沖,繁。府東北百五十里。南有拒馬,自定興緣界,其岔河北自固安入,至十九垡左導為蘆僧引河,今淤。又西南合紫泉河、斗門河,納藍溝河,即界河錯出復入者。又南曰白溝河,入容城復合。有方官、新橋、白溝三鎮。汾水驛。唐簡。府西少南百二十里。北有堯山。東北:望都山。西北:大茂山。西有唐河,古滱水,自廣昌入,錯出,左合倒流河。西:雹水,右納恆河、馬泥河、唐河。又東北有放水河。倒馬關西北有嶽嶺、柳角安、軍城鎮、周家堡四口。橫河口巡司。縣驛一。博野疲。府南九十五里。東南:豬龍河自安平緣界入,一曰蟾河,屈南逕白塔村入蠡。唐河自清苑入。縣驛一。望都沖,難。府西南八十里。舊曰慶都,乾隆十一年改。東南:唐河自定州入。有九龍泉,環城珠湧,東出為龍泉河。有翟城驛。容城簡。府東北九十里。北有拒馬河,西支自定興緣界入,與東支白溝河合。西清而弱,東濁而強。又雹河自安肅入,其萍河涸。縣驛一。完簡。府西少南七十里。西:伊祁山,祁水出焉,即曲逆河。圖經惡其名,改方順。納放水河。其舊所合蒲河,涸。唐河自其縣再錯入,合清水河。蠡繁,難。府南少東九十里。南:豬龍河自博野入,一曰楊村河。唐河自博野入,自道光初北徙。河丞駐仉村。縣驛一。雄沖,繁,難。府東北百二十里。西澱,縣南。亙安州、高陽、任丘,周三百三十里,匯府境諸水,所謂「七十二清河」。趙北口扼其中。橋十二。四角河自安州入,出第五橋,曰大清河,錯出復入。白溝河自容城入,南及大港、柴禾二澱。大清河乃改由藥王行宮北與會。有歸義驛。祁州簡。府南少西百二十里。南有滹沱北支,自深澤緣界。其北豬龍河,匯定州滱、沙、滋三水。滱即唐,嘉慶初徙,孟良河奪之。是為豬龍河。又南逕程各莊入博野。縣驛一。束鹿繁,難。府南少西二百四十里。西北:滹沱自晉州入深州為南支,其支津入安平,同治十年所徙。其故道七。縣丞駐小章村。縣驛一。安州簡。府東少北六十里。道光十二年以新安省入。府河、唐河自清苑入而合,納曹河,逕城北為依城河,右注白洋澱,與豬龍河自高陽入者相望也。左注雜澱,復合為四殳河,亦曰四角河。西澱都九十有九,白洋最廣,次燒車,雜澱最★。新安鄉行宮二。州驛一。高陽簡。府東南六十五里。西北:唐河自蠡入,亦曰土尾河。東南:豬龍河亦自蠡入,順治中,復決布裏村,故亦曰布裏河。舊合泔河,即高河,縣氏焉,淤。縣驛一。

定定府:沖,繁。隸清河道。總兵駐。明曰真定。領州五,縣二十七。雍正元年曰正定。二年,升冀、趙、深、定、晉為五直隸州,以南宮等十七縣屬之。十二年,降晉州,並所屬無極、城與定州、新樂還來隸。東距省治二百九十里。廣二百七十里,袤三百八十里。北極高三十八度十一分。京師偏西一度四十八分。領州一,縣十三。正定沖,繁,難。倚。舊曰真定,雍正元年改。西有滹沱,自平山入。有冶河故道二。其北林濟河,合西北諸泉及旺泉河。又北,滋河自新樂入,伏而東。滹沱性善徙,滏北滋南,百數十里沖漫幾遍。今河乃同治七年改決,為康熙中東入深、安、饒故道。有恆山、伏城二驛。獲鹿沖。府西南六十里。南有封龍山。北:五峰山,洨水出焉。合小沙、左金河。西有鹿泉水,東至大要舍納冶河。今淤。有鎮寧驛。井陘簡。府西南百三十里。井陘山東北有關。北:綿蔓河自山西平定州入,合甘淘河,一曰微水。折北,左得金珠泉,至東冶村曰冶河。西南:固關,寄平定州,置參將。其北:娘子關。有汛。邊墻西北首達滴巖,南訖楊莊口。有陘山驛。阜平簡。府西北二百十里。順治末,省。康熙二十二年,復置。大茂山東北,平陽河出焉。沙河自山西繁峙入,納靈丘北流、鷂子諸河曰派河,又東合班峪、燕支諸河。又汊河出縣南白蛇嶺。邊墻東北首落路口,西南訖當城河口。有龍泉關、長城嶺。汛東有王快鎮。康熙中,縣寄此。又茨溝營鎮。縣驛一。欒城簡。府南六十里。西有洨河,自獲鹿入,納北沙、金水二河。南、西有故城二。關城驛。行唐簡。府北七十五里。西北:箕山,郜河出其北兩嶺口,合甘泉河、龍門溝,側城東南,合賈木溝。北:派河自曲陽入,合曲河。西:滋河自靈壽入而伏。靈壽簡。府西北六十里。南:滹沱自平山緣界合松陽河、衛河。衛河,禹貢衛水也。西北:滋河自山西五臺入,納汊河。又東南合慈峪河,亦曰慈河,入行唐。邊墻北首白草溝口,南訖車孤駝口。有叉頭鎮巡司。乾隆中移慈峪鎮。平山簡。府西少北八十里。西北有房山,濊河出焉,古石臼水,今湮。滹沱自山西五臺首入縣西北,始出山。又納冶河,始湍悍。邊墻北首合河口,南訖清風口。有洪子店巡司。元氏簡。府南少西九十里。西北:封龍山,北泜水所出,下流入胡盧河。無極水南入贊皇會南源,復入而合,錯出復入,至紙屯村與槐河會。豬龍河自縣西匯諸山水,北沙河出割髭嶺,今並涸。其南金水河,東入欒城。縣驛一。贊皇簡。府西南百二十里。雍正三年自趙州來隸。西南:贊皇山,河出焉。其北泜河,南源二,出可蘭、四望二山。槐河二源,一黃沙嶺,一紙糊套山,今並涸。王家坪鎮,咸豐末改汛。縣驛一。晉州簡。府東少西南九十里。西北:滹沱自無極入。同治十年,改自城入。又故道二。有驛。無極簡。府東七十里。雍正二年改屬晉州,十二年復。滹沱河自城入,再錯出,復入,逕東漢村,復歧為二。其滋河入逕縣南,屈東又北。木刀溝自新樂入,合護城河,錯出復入,並入深澤。縣驛一。城簡。府東南五十里。雍正二年改屬晉州,十二年復。滹沱自正定入,合西韓、旺泉二河。順、康中再決,並東南過周頭入白牧河。滋河自正定、木刀溝自新樂入,與王莽溝並涸。縣驛一。新樂沖,疲。府東北七十五里。雍正二年改屬定州,十二年復。派河自行唐入,合郜河。木刀溝出平山之濊河,滋河奪之。順治中,知縣林華皖濬自西南閔泉鎮。嘉慶初,滋之支津復自正定入奪之,錯出復入,合浴河。縣驛一。

大名府:沖,繁,難。總兵駐。順治初,置大順廣道。雍正初,改清河道,十一年,復置。初沿明制,領州一,縣十。雍正三年,割內黃、濬、滑分隸河南彰德、衛輝。乾隆二十三年,省魏縣分入大名、元城。東北距省治八百里。廣二百里,袤三百七十里。北極高三十度二十一分三十秒。京師偏西一度六分。領州一,縣六。大名沖,繁,難。倚。府南偏。明徙府南八里南樂鎮。乾隆二十二年已於漳,復故,惟縣丞駐。衛河自河南內黃入。其新衛河自清豐入,錯出復入來會。漳河自臨漳分入,一入衛,一至府治南為漳河引河。東有縣故城三。東北:小灘鎮,嘉慶中置河主簿。縣驛一。元城繁。倚。府北偏。故城三。東南:衛河自大名入。其漳水引河,古漳河入,逕北張莊而合,並東入館陶。東南:馬頰河自南樂入。縣驛一。南樂難。府東南五十里。嘉慶二十一年,新開衛河始自大名入。光緒十四年後,漳河始自其縣來會。西有硃龍河、岳儒固河,東六塔廢河,並自清豐入。又東:龍窩河自山東觀城入,至龍窩村止。夏秋霪潦,輒復彌漫。然六塔平壤故有順水溝,康熙中,知縣王培宗濬;光緒二十一年,原思瀛再濬,命曰永順,邑賴之。清豐難。府南少東九十里。西有廣陽山。衛河自河南內黃緣界。西有古馬頰河。硃龍河自開入。有順河堡鎮。縣驛一。東明繁,疲,難。府南二百二十里。西有黃河自長垣入。自明以來,在縣境者三徙:嘉慶八年奪洪河,二十四年奪漆河,咸豐五年奪賈魯河,後復北徙為今瀆。南有杜勝集鎮。雍正十年改守備置都司,明年置巡司。舊有通判,道光中裁。開州繁,疲,難。府南百二十里。同、光中,黃河自東明潰入者六道,合而復分。北支古瓠子河,一曰毛相河,故小渠,康熙中決荊隆口,始大。南支古濮渠,並入山東濮州。又有黃河故道二,曰古馬頰河、古硃龍河。又硝河自河南滑縣入,亦曰馬頰河。徐鎮堡、兩門集、井店集、柳下屯四鎮。呂丘堡,州判駐。古定鎮有廢巡司。州驛一。長垣繁,疲,難。府西南二百九十里。東有黃河自河南蘭封入,舊逕盤岡里,咸豐八年徙蘭岡,同治二年復折西自蘭通集至舊城口為今瀆。縣丞駐大黃集。有大岡廢巡司。縣驛一。

順德府:沖。隸大順廣道。東北距省治五百七十里。廣二百八十里,袤百五十里。北極高三十七度七分。京師偏西一度四十九分。領縣九。邢臺沖,繁,難。倚。西:封山。野河出西北馬嶺口,淤。今自內丘入,會稻畦、漿水、路羅三川為洪河。北有達活河,合沙應河。又有百泉河,右會七里河。西:黃村巡司。有龍岡驛。鐵路。沙河沖。府南三十五里。沙河自河南武安入,會邢臺之洪河。右出支津,逕城南而東,納西狼溝水,其東即東狼溝。縣驛一。鐵路。南和繁,疲。府東南四十里。西:百泉河自邢臺入。沙河支津亦自其縣入,合東狼溝。其正渠曰乾河。又東洺河、劉壘河,自雞澤入。有驛。鐵路。平鄉疲,難。府東八十里。東:滏陽河自雞澤入。西:劉壘河自南和入。縣驛一。廣宗疲。府東百二十里。漳河故道二,康熙二十六年溢,知縣吳存禮增築東西堤萬九千餘丈。縣驛一。鉅鹿疲,難。府東百十里。鉅鹿藪即大陸澤。滏陽河自任入。老漳河,康熙中徙,廢。縣驛一。唐山簡。府東北八十里。有宣務山。泜河、李陽河、柳林河,並自內丘入。有驛。內丘沖。府北六十里。鵲山一曰龍騰山,龍騰水出焉,匯西山九龍水,東流為柳林河。其西麓姑腦,泜河南源出焉,錯出復入,其泜河第二川、第三川合為野河。有中丘驛。鐵路。任簡。府東北四十里。滏陽河自平鄉入。有大陸澤,納九河八水,東溢為雞爪河來會。澤舊亙鉅鹿、隆平、寧晉境,滹、漳、滏湊焉。今滹北,漳南,滏亦東徙。大陸在任者南泊,即張家泊,在寧晉者北泊,即寧晉泊。縣驛一。

廣平府:簡。隸大順廣道。明,領縣九。雍正初,怡賢親王以滏河故,奏割河南彰德之磁州來隸。東北距省治六百八十里。廣三百五十里,袤百八十里。北極高三十六度四十六分三十秒。京師偏西一度三十五分。領州一,縣九。永年沖,繁,難。倚。西北:婁山。東北:沙河,自沙河入。南:洺河,自河南武安入。乾隆中,決入牛尾河,同治末,復故。東南:滏陽河,自邯鄲入,歧為劉壘河,即牛尾河。有八閘,並引滏溉田萬九千餘畝。臨洺關通判,道光中裁,移河務同知駐此。縣驛一。曲周繁。府東北四十里。西南:滏陽河自永年入。漳河故道東南,自明萬歷初挾滏而北,康熙十年始南徙,四十七年益南,逕大名、元城。縣驛一。肥鄉簡。府東南四十里。東西漳河故道二。東有舊店營。康熙中,縣寄此。縣驛一。雞澤疲,難。府東北六十里。東滏陽自曲周入,右導為興隆河。西有沙、洺、牛尾,自永年入。廣平簡。府東南六十里。漳河故道舊自成安入,其支津拳壯河,並湮。縣驛一。邯鄲沖,繁,難。府西南五十里。西北:紫山。西:靈山。東北:滏陽河自磁入,合渚河、沁河、輸黿河。有叢臺驛。鐵路。成安簡。府南少西六十里。洹、漳故道並自河南臨漳入。順、康中,漳河再毀城垣。乾隆末,改自其縣三臺入衛。威難。府東北百一十里。南有漳河故道。張臺村廢巡司。縣驛一。清河簡。府東百八十里。清河故瀆,縣西。衛河自山東臨清緣界入。其武城,古屯氏別河。西北:漳河故道。雍正中,移縣丞駐油房口,兼巡司事。縣驛一。磁州沖,繁,難。府西南百二十里。雍正四年,自河南彰德來隸。西有神麕山。釜山,滏水南北源出焉。合羊渠河、泥河,東播為五爪渠。環城,復歧為三,合檿牛河、澗水。漳河自河南涉縣入。州判駐彭城鎮。有滏陽驛。

天津府:沖,繁,疲,難。初隸天津道。明,衛,河間地。雍正三年為直隸州,以順天之武清,河間之青、靜海來屬。武清尋還舊隸。九年升府,置附郭縣。降滄州並所屬三縣來隸。天津道、總兵、長蘆鹽運司、通永鎮總兵駐。咸豐十年,海禁洞開,置三口通商大臣。同治九年,廢為津海關道,以總督兼北洋欽差大臣,駐保定,半歲一移節。府城,三岔口西南。光緒庚子,拳匪亂,夷為平地。西距省治四百六十里。廣二百二十里,袤三百八十里。北極高三十九度十分。京師偏東四十七分。領州一,縣六。天津沖,繁,疲,難。倚。雍正九年置。海,東南百二十里。北運河自武清入,匯大清、永定、子牙、南運為海河,逕紫竹林,歷二十一沽,左右引河以十數,至大沽口入焉。大沽鎮有協及同知。雍正初,置天津水師營。同治初,置機器局。後建新城砲臺,與大沽砲臺相聲勢。新城有海防同知。長蘆場八,自山海關至山東樂陵,袤八百餘里。豐財場東南葛沽與西沽、楊青巡司三。大沽、三河、頭水旱溝、蒲溝、咸水沽、雙港、北馬頭、趙家場八鎮。楊青水、陸二驛。航路:東南駛之罘、上海,東北駛營口,東駛朝鮮仁川與日本長崎。鐵路:京津,津榆,津保,津浦焉。青沖,繁,疲,難。府西南百六十里。順治末,省興濟入之。雍正三年自河間來隸。南運河自滄州入,有興濟減河。西:黑龍港河自河間入,東南:滹、漳故渠二。長蘆鎮,縣南七十里,有鹽運司,今移天津。有流河管河主簿。興濟、杜林二鎮巡司。河東、馬廠二汛。流河、乾平二水驛。靜海沖,繁,疲,難。府西南七十里。雍正三年自河間來隸。南:南運河自青入,右出為靳官屯減河。西:子牙河自大城入,納黑龍港河。西北:大清河亦入,納支津辛張河。有獨流鎮巡司。有奉新驛。滄州沖,繁,疲,難。府西南二百里。明屬河間。雍正七年升直隸州,尋降來隸。海,東百三十里。南運河自南皮入,右出為捷地減河。其北興濟減河自青入。其南石碑河上承王莽河,自南皮入,匯為母瀦港,至歧口入焉。東南:宣惠河亦自南皮入。有嚴鎮場鹽大使。磚河、祁口、捷地、舊州四鎮。風化店、孟村、李村三巡司。磚河水、陸二驛。南皮繁,難。府西南二百七十里。雍正中,自滄州來隸。南運河自東光緣界。宣惠河自東光入,歧為王莽河。津河自寧津數錯入。有薛家窩、馮家口二鎮。新橋驛。鹽山繁。府南二百六十里。雍正中,自滄州來隸。海,東北百二十里。宣惠河自州入。古黃河鬲津自南皮入,錯出復入,並入山東樂陵。東有廢無棣溝。海豐場在羊兒莊,與舊縣置巡司二。狼坨子、韓村、高家灣三鎮。慶云簡。府東南三百二十里。雍正中,自滄州來隸。鬲津自鹽山錯入,納胡蘇、覆釜二河。馬頰河自樂陵入,入山東海豐。縣驛一。

河間府:沖,繁,難。隸清河道。明,領州二,縣十六。雍正三年,升天津衛為直隸州。順治末,省興濟入青。至是以青、靜海屬之。七年,復升滄州,以東光、南皮、鹽山、慶雲屬之。九年,東光還隸。北距省治百四十里。廣二百里,袤三百八十里。北極高三十八度三十分。京師偏西十七分。領州一,縣十。河間沖,繁,難。倚。子牙河、黑龍港河自獻入。西有古洋河,合唐河。同治末,滹沱逕此,後廢。縣丞駐東城鎮。又二十里鋪、臥佛堂、沙河橋、崇仙、新村五鎮。景和鎮、北魏村二巡司。有瀛海驛。獻沖,繁,疲,難。府南少東五十五里。西南:滏陽自武強入,歧為滹沱別河。東北:三黑龍港河與南亭子河並湮。淮、商家林二鎮。有樂成驛。阜城沖。府南少東百四十里。西:漳河自景州入。東南:古沙河,即屯氏河,亦自景入,亦曰漫河。有漫河驛。肅寧簡。府西四十里。古唐河自饒陽入,涸。古洋河自獻入。豬龍河舊自高、蠡間溢入為中堡河,又東歧為玉帶河,今並湮。有阜城驛。任丘沖,繁,難。府北六十七里。四角河自安州入,出趙北口。東:大港引河。同治末,復濬為趙王新河,下注清苑玉帶河,並移鄚州東汛縣丞駐此。有廢洋河。古州鎮。鄚城驛。交河繁,疲,難。府東南百一十里。南運河自東光緣界。其西漫河、漳河、亭子河、滹沱別河,並涸。有泊頭鎮河主簿及廢巡司。高川鎮。富莊驛。有丞,裁。寧津簡。府東南二百三十里。古黃河鬲津自吳橋入。南有土河,舊自山東德州入,下至慶雲為限河。或亦曰馬頰河。有包頭鎮。有驛。景州繁,難。府東南百九十里。南運河自山東德州緣界。古沙河自故城入,曰大洋河。曲流河自故城入,曰江江河,合為漫河。又西北有廢漳河。劉智廟、安陵、連窩三鎮。龍華鎮巡司。有東光驛。吳橋繁,難。府東南二百四十里。西:南運河自山東德州緣界入。東:宣惠河。又東:沙河,古黃河鬲津,今四女寺減河,鉤盤河,今哨馬營減河,自德州入而合。有龍華鎮巡司。連窩鎮河丞。分隸景州。有水驛丞,裁。東光繁,疲,難。府東南百六十里。南運河自吳橋入。東:宣惠河,合沙河、漫河自景、阜城分入而合。有燈明寺村、夏口二鎮。馬頭驛。故城疲,難。府南少東二百八十里。南運河自山東入。武城緣界入。德州西北屯氏二支曰古沙河、曲流河,並出縣西。有廢漳河,即黃瀘河。縣丞駐鄭家口。有營。甘陵驛。

承德府:沖,繁,難。隸熱河道。明,諾音、泰寧二衛。天順後,烏梁海居,又並於察哈爾。順治初,內屬。康熙四十二年,建避暑山莊於熱河,歲巡幸焉。五十二年,城之。雍正元年,置。十一年,置承德直隸州。乾隆七年,仍為。四十三年為府。置州一,縣五。嘉慶十五年,置熱河道都統。並轄內蒙古東二盟十六旗,又附西勒圖庫倫喇嘛一旗。光緒初,置圍場。三十年,朝陽升府。以建昌隸之。隸宣化。三十三年,赤峰復升直隸州。西南距省治七百八十里。廣一千二百里,袤八百里。北極高四十一度十分。京師偏東一度三十分。領州一,縣三。府東:天橋山。西:廣仁嶺,本墨斗嶺,康熙末更名。熱河,古武列水。西源固都爾呼河,自豐寧入,納中源茅溝河即默沁河,東源賽音河,逕磬錘峰,合溫泉,始曰熱河。灤河自灤平入合之。又東合白河、老牛河,折南納柳河。其西黃花川、黑河,其東瀑河自平泉再錯入。瀑河並入遷安。伊遜河出圍場伊遜色欽,南入豐寧。又西有乾塔河,入密雲。有釣魚臺、黃土坎、中關、張三營四行宮。邊墻北首漢兒嶺,南訖黑塔關口。有唐三營、中關、下板城、新漳子、六溝、二溝、三溝、茅溝八鎮。石片子巡司。熱河驛。灤平沖,難。府西南六十里。明,諾音衛。乾隆七年,置哈喇河屯,四十三年改。西:棽髻山。西南:青石梁。西北:灤河自豐寧入,合興州河。左伊遜河入府界。潮河自豐寧入。西南:沽河自獨石口入,與湯河、紅土峪、馮家峪、黃崖口、水峪、白道峪、大水峪諸河並入密雲。其西雁溪河入懷柔。有喀喇河屯、王家營、常山峪、兩間房、巴克什營五行宮。邊墻東首漢兒嶺,西訖幵連口。喀喇河屯、大店子、三道梁、馬圈子、紅旗、呼什哈、喇嘛洞七鎮。安匠營巡司。縣驛一。平泉州沖,繁,難。府東百五十里。明,諾音衛。雍正七年置八溝,為南境。乾隆四十三年改置。西有納喇蘇臺山、察罕陀羅海山。錫伯河出其東。熱河東源賽音河。中源默沁河並出西北入府界。瀑河一曰柳河,四源合於元惠州故城西,曰察罕河,逕寬城西曰寬河,入遷安。老哈河古託紇臣水,俗省曰老河,出喀喇沁右翼南百九十里永安山,亦曰察罕河,與奇札爾臺河會,又北合霍爾霍克河、布爾罕烏蘭善河、烏魯頭臺河,又東北合昆都倫河,入建昌。大寧城東北八十里,州判駐。有七溝營、丫頭溝、暖泉、櫻桃溝、龍須門、波羅樹、他拉波羅窪、臥佛寺八鎮。八溝稅務司。州驛一。豐寧繁,難。府西北二百六十里。明,諾音衛。乾隆元年置四旗。四十三年改。西北:赫山、苔山,玲瓏峰舊曰興隆山,乾隆十九年更名。東有熱河西源,自圍場入,逕固都爾呼嶺,曰固都爾呼河,入府界。北:上都河自多倫入,納小灤河,曰灤河。其西興州河,出西北呼爾山。潮河,古洫水,一曰鮑丘水,出縣西大閣北七十里城根營。又湯河出十八盤嶺。東北:伊遜河自府界入,納伊瑪圖河,並入灤平。有波羅河屯、黃姑屯、什巴爾臺、濟爾哈朗圖四行宮。荒地、鄧家柵、上黃旗、林家營、森吉圖、白虎溝六鎮。郭家屯、大閣兒、黃姑屯、土板四巡司。縣驛一。隆化光緒三十年以張三營子置。有巡司管典史事。與郭家屯、黃姑屯二。

朝陽府:繁,疲,難。隸熱河道。明,營州衛。後入泰寧衛。乾隆三年,置塔子溝,為東境。三十九年,析置三座塔。四十三年,置朝陽縣。光緒三十年,以墾地多熟,升府,以建昌隸之。又置縣三。西南距省治一千四百二十里。北極高四十一度四十五分。京師偏東四度二十三分。領縣四。西北:潢河自內蒙古阿魯科爾沁旗入。西南:大凌河自建昌入,合南土河,逕西平房西,左合卑克努河,察罕河,又東合布爾噶蘇臺河,又東至龍城,一曰三座塔城。左合固都河、涼水河,至金教寺東北,左合土河,入盛京義州。小凌河出縣屬土默特右翼明安喀喇山。三源:中明安河,南穆壘河,北參柳水,東南流,合哈柳圖河,入奉天錦縣。養息牧河二源,並出喀爾喀左翼,東南流,合好來昆德河、鴨子河,入奉天廣寧。柳邊南首建昌,北訖科爾沁左翼。門五:新臺、松嶺子、九官臺、清河、白土廠。有六家子、波羅赤、三道梁、青溝四鎮。三進塔稅務司。縣驛一。建昌繁,難。府西南二百六十里。明,營州廢衛。乾隆四十三年以塔子溝西境置。光緒三十年自承德來隸。北有固爾班圖勒噶山。東南:巴顏濟魯克山。東有布祜圖山,漢白狼山,白狼水出焉,今曰大凌河。南源出喀喇沁右翼南土心塔,會中源克爾、東源牛錄,入朝陽。北:漆河自灤平逕縣西入遷安。蒐濟河出喀喇沁左翼東南毛頭泊,入奉天錦州。北有潢河自赤峰入,會老哈河。河自平泉入,合伯爾克河,錯出復入。英金河亦自縣來會,復合落馬河,東北至谷口。乾隆八年,更名敖漢玉瀑,與潢河會,又東入朝陽。柳邊北首朝陽,南訖臨榆。門一:梨樹溝。有貝子口琴、波羅索他拉、胡吉爾圖、大城子四鎮。縣丞駐東北四家子鎮。塔子溝稅務司。蟒莊巡司。縣驛一。

赤峰直隸州:繁,難。明,諾音衛。雍正七年置八溝,為北境。乾隆二十九年,析置烏蘭哈達。四十三年,置赤峰縣,隸承德府。光緒三十三年,升直隸州。增置林西。西南距省治千三百二十里。北極高四十二度三十分。京師偏東二度四十五分。領縣一。潢河自圍場入州北二百餘里之巴林旗。東南:老哈河,自平泉逕東南隅,納伯爾克河,北入建昌。英金河,古饒樂水,三源自圍場入,合於色哷,圍場西南折東,合巴顏郭河、色哷河、壘爾根烏里雅蘇河,入翁牛特右旗,合奇布楚河、鴨子河,又南會使力戛河,其上游納林錫爾哈河。木蘭東北諸水,匯於英金,東南諸水,匯於錫爾哈,三源合北流,合克依呼河,入平泉合克勒河,始入州,西北會烏拉臺河。錫伯河亦自平泉來,與英金河會。英金河又東合卓索河,入建昌。烏拉臺河三源,亦木蘭諸水所匯,東流合默爾根精奇尼河,阿濟格赴河、噶海圖河、布獲圖河。有杜梨子溝、哈拉木頭、四道梁、音只戛梁四鎮。縣丞駐西北大廟鎮。有烏蘭哈達稅務司。有驛。林西州西北四百十八里。光緒三十三年以巴林察罕木倫河西北地置。

宣化府:沖,繁,難。隸口北道。明,宣府鎮。順治八年,裁宣府巡撫。十年,並衛所官。領宣府等十縣。降延慶、保安屬之。康熙三年,改懷隆道為口北道,與總兵並駐此。四年,隸山西,尋復。七年,裁萬全都司。三十二年為府。巡撫郭世隆疏改,置縣八。後割山西蔚州來隸。光緒三十年,復割承德之圍場來隸。東南距省治七百里。廣四百四十里,袤三百二十里。北極高四十度三十七分十秒。京師偏西一度二十一分三十秒。不與。領一,州三,縣七。宣化沖,繁,難。倚。明,宣府前衛。順治中,省左右衛入之,為宣府鎮治。康熙三十二年,改置為府治。北有東望山,西西望山。西有洋河自懷安入,左納清水河、柳河川、泥河,東南入懷來。其南桑乾河自西寧入,數錯出,於懷來合洋河,復入,逕府境。鎮二:雞鳴堡、深井堡。有守備,康熙中裁。有華稍營巡檢司。宣化、雞鳴二驛。又遞二。軍站五。赤城簡。府東北七十里。明,赤城堡。舊為上北路。康熙三十二年改置。又以滴水崖、雲州、鎮安、馬營、鎮寧五堡入之。赤城山城。東北:白河自獨石入,南流出龍河峽,一曰龍門川,側城東南,合大石門水,亦曰赤城河。又得翦子嶺東、浩門嶺西水,屈東南,右納龍門河,左得紅沙梁水,入延慶。營二:獨石左、獨石右。口七:鎮寧、松樹、馬營、君子、鎮安五堡,龍門所、滴水崖。順治中,改參將置守備滴水崖。雍正中,改守備置都司。鎮十一:新鎮樓、雲州堡,及北柵、東柵、西柵、盤道、塘子、清平鎮嶺、四望、磚墩、野雞九口。驛二:雲州、赤城。萬全沖,繁,難。府西北七十五里。明,萬全右衛。舊為西路,康熙三十二年改置。西北有野狐嶺、蕁麻嶺,今★洗馬林。西有洋河自懷安入,左納孫才溝,西沙河、新河、東沙河,仍入之。西有愛陽河。東有清水河自張家口入,合臭灘、黃土梁水,南入宣化。營二:萬全、張家口。有副將。光緒七年,移多倫,惟都司駐。口五:鎮口臺、神威臺、洗馬林、新河、膳房堡。有軍站五。龍門簡。府東北百里。明,龍門衛。舊為下北路。康熙三十二年改。又以葛峪、趙川、雕鶚、長安嶺四堡入之。西有龍門山,龍門河出其北麓,逕城南而東,左得翦子嶺西、浩門嶺南水,入赤城。西有小清水河,自張家口分入而合,曰柳河川。又有泥河,並入宣化。營一:龍門路。口二:葛峪堡、趙川堡。鎮八:安邊、靜樓、墩鎮、沖臺、盤道、宜臺六口,常峪鎮、雕鶚堡。長安嶺堡並有驛,雍正中,嶺置都司,後裁。有軍站二。懷來沖,繁。府東南百五十里。明,懷來衛。舊為東路。康熙三十二年改。又以保安衛及土木、榆林二堡入之。南有軍都山。西有桑乾河,自宣化入,再錯出復入,會洋河,北支也。折東南,右得礬山水,左有右河,至合河口會媯河,其東支也。又南入宛平,為盧溝河。二鎮:保安城,雍正中改參將置都司;礬山堡,守備駐。有沙城堡巡司。土木、榆林二驛。軍站四。蔚州沖,疲,難。府西南二百四十里。雍正六年自山西大同來隸。有衛。康熙三十二年改。乾隆二十二年省入。東南:笄頭山,一曰磨笄山。西有壺流河,自山西廣靈入,再錯出復入。左右得乾沙河,九折,北合定安河、會子河、扶桑泉諸水,入西寧。三鎮。黑石嶺即飛狐岌,有神道溝巡司,康熙中裁,以吏目兼理。又岔口、桃花堡,三遞。西寧簡。府西南二百里。康熙三十二年以明順聖東、西二城置。東南有榆林山、月神山。西有桑乾河,古濕水,自山西天鎮入。有小莊渠,乾隆十年導。又東,左納虎溝河,合五里河、汊河、西沙河,至小河口會壺流河。有順聖川鎮。東城、西城二遞。懷安沖,繁。府西少南百二十里。明,懷安衛。康熙三十二年改。又省萬全左衛及所轄柴溝堡、西洋河堡入之。西北:花山。南:託臺穀。水溝口河自山西天鎮入,合谷水,自洪塘溝東北注洋河。東洋河自張家口入,會西洋河、南洋河,曰洋河,亦曰燕尾河,錯出復入,合水溝口河。營一:柴溝堡,巡司駐。口二:東洋河、西洋河。有左衛城、西洋河堡、水關臺、鎮口臺四鎮。懷安、萬全二驛。軍站四。延慶州沖,難。府東少南二百里。舊隸宣府鎮為東路。順治末,省永寧縣入衛。康熙三十二年改。乾隆二十六年,又省延慶衛及所轄五千戶所入之。北:阪泉山。東北:獨山。南:八達嶺。北:白河自赤城入,復入獨石口。媯河出州東北,伏流復出為黃龍潭,合龍灣水,環城,合沽河、蔡河、黑龍河,入懷來。鎮五:石硤峪、營盤口、小水口、鎮安堡、千家店。口四:周四溝堡、四海冶堡、柳溝城、八達嶺。東有永寧城巡司。居庸驛。軍站一。保安州簡。府東南六十里。舊隸宣府鎮為東路。康熙三十二年改。南:涿鹿山、橋山。西南:釜山、歷山。東南:羹頡山。有泉湛而不流,古阪泉也。西:桑乾河自宣化錯入,再錯懷來入之,導為五渠。有馬水口鎮。有遞。圍場沖,繁,疲,難。西北三十二里,正副總管駐。本內蒙古卓索圖、昭烏達東二盟地。康熙中,進為圍場,曰木蘭,國語「哨鹿」也。光緒二年置。三十年自承德來隸,兼有府、赤峰西北、豐寧東北境。在內蒙古各部落之中,周千三百里,廣三百里,袤二百里,並有奇。四界表識曰「柳條邊」。道二,並自波羅河屯入。東崖口,一曰石片子,西濟爾哈朗圖。舊制以八月秋獮,東入則西出,西入則東出,歲以為常。場都六十有九,以八旗分守於內,旗各營房一、卡倫五。鑲黃旗營房在奇卜楚高,為北之東,其卡倫曰賽堪達巴罕色欽,曰阿魯色埒,曰阿魯呼魯蘇臺,曰英格,曰拜甡圖。正白旗營房在納林錫爾哈,為東之南,其卡倫曰巴倫昆得伊,曰烏拉臺,曰錫拉諾海,曰諾林錫爾哈,曰格爾齊老。鑲白旗營房在什巴爾臺,為南之東西間,其卡倫曰噶海圖,曰卓索,曰什巴爾臺,曰麻尼圖,曰博多克。正藍旗營房在石片子,為南之東,其卡倫曰木壘喀喇沁,曰古都古爾,曰察罕扎克,曰汗特穆爾,曰納喇蘇圖扎巴。正黃旗營房在錫拉扎巴,為北之西,其卡倫曰庫爾圖陀羅海,曰納喇蘇圖和碩,曰沙勒當,曰錫拉扎巴,曰錫拉扎巴色欽。正紅旗營房在扣肯陀羅海,為西之北,其卡倫曰察罕布爾噶蘇臺,曰阿爾撒朗鄂博,曰麻尼圖布拉克,曰齊呼拉臺,曰布哈渾爾。鑲紅旗北營房在蘇木溝,為西之南,其卡倫曰海拉蘇臺,曰姜家營,曰西燕子窩,曰郭拜,曰和羅博爾奇。鑲藍旗營房在海拉蘇臺,為南之西,其卡倫曰硃爾噶岱,曰蘇克蘇爾臺,曰卜克,曰東燕子窩,曰卓索溝。有西圖巡檢司。驛一。

口北三:隸口北道。直宣化府,張、獨二口北。明季,韃靼諸部駐牧地。康熙十四年,徙義州察哈爾部宣、大邊外,壩內農田,壩外牧廠,順治初置,在張、獨者六,其一奉天彰武臺。及察哈爾東翼四旗、西翼半旗。雍正中,先後置三理事同知。光緒七年,並改撫民同知。廣六百里,袤六百五十里。

張家口:要。明初,興和守御千戶所。順治初,為張家口路,隸宣府鎮。西北六十里。康熙中,置縣丞。雍正二年,改理事。轄官地,及察哈爾東翼鑲黃一旗、西翼正黃半旗,並口內蔚、保安二州,宣化、萬全、懷安、西寧四縣旗民。光緒七年改撫民,復。東南距省治七百五十里。北極高四十度五十分四十秒。京師偏西一度三十五分。北有東山、高山、大小烏鴉山。東洋河二源,自山西豐鎮分入而合,左得蘇祿計水。清水河出東北,合毛令溝、太子河、驛馬圖河,曰正溝,合大西溝、大東與新河、東西沙河,並入萬全。其東小清水分入龍門。西北有昂古里泊。又諾莫渾博羅山有正黃等四旗牧廠,查喜爾圖插漢地有禮部牧廠,並明天成衛邊外地。齊齊哈爾河有太僕寺右翼牧廠,廣百五十里,明大同邊外地。東北喀喇尼墩井有太僕寺左翼牧廠,明,宣府邊外地。北控果羅鄂博岡,有鑲黃等四旗牧廠,明廢興和千戶所。自雍正十年與俄定恰克圖約為孔道。光緒二十八年,劃地五百萬方尺為租界。三鎮:興和城、太平莊、烏里雅蘇臺。有站。

獨石口:要。明初為開平衛。順治初為上北路,隸宣府鎮。東北二百五十里。康熙中置縣丞,曰獨石口,並衛入赤城。雍正十二年置理事。轄官地,及察哈爾東翼正藍、鑲白、正白、鑲黃四旗,並口內延慶一州,赤城、龍門、懷來三縣旗民。光緒七年改撫民。副將防守尉。駐。南距省治七百九十里。北極高四十度五十四分四十秒。京師偏西四十分。東南有大小石門山、太保山。白河,古沽水,正源堤頭河,出西北狗牙山,合東西柵口水,與別源獨石泉會,南入赤城。復自延慶州入,與黑河並入灤平,下流會潮、榆諸水,為北運河。上都河,古濡水,出東北巴顏屯圖固爾山,合三道河,西北入多倫,下流為灤河,至樂亭入海,行二千一百里有奇。有金蓮川、伊克勒泊。東北:博羅城,有御馬廠,隸上駟院。四鎮:丁莊灣、黑河川、東卯鎮、千家店。有站。

多倫諾爾:要。明,開平衛地。順治初,置上都牧廠,屬宣府鎮。東北五百五十里。康熙三十年,喀爾喀為準逆所破,車駕蹕此受降焉。雍正十年,置理事。轄察哈爾東翼正藍、鑲白、正黃、鑲黃四旗,及蒙古內札薩克與喀爾喀旗民。光緒七年,改撫民。西南距省治千一百里。北極高四十二度二十八分二十秒。京師偏西六分。西南有駱駝山。北有錫拉穆楞河,自內蒙古克什克騰旗入,合碧七克、碧落、拜察諸河,北入巴林旗。東南有上都河,自獨石口入,合石頂、克伊繃、額爾通、伊札爾、什巴爾臺諸河。七星潭在上都牧廠北,一曰多倫泊,氏焉。蒙語謂止水曰「泊」,大者「諾爾」,次「鄂模」、「庫勒」、「科爾昆」有差。北布珠、博碩岱等泊以十數。西北又有堿池。興化鎮在喇嘛廟南,張家口副將駐。有白岔司。又興盛鎮、二道泉、閃電河、土城子四汛。驛一。

永平府:要。隸通永道。明,領州一,縣五。乾隆初,廢山海衛置臨榆。先是雍正初,以順天之玉田、豐潤來隸。乾隆八年,復改屬遵化。西距省治八百三十里。廣三百三十里,袤三百八十里。北極高三十九度五十五分三十秒。京師偏東二度二十八分三十秒。領州一,縣六。盧龍沖,繁,難。倚。東南:陽山。西南:孤竹山。灤河自遷安入,合青龍河。東有飲馬河。東北:燕河。營一:燕河路。有燕河莊、夷齊廟二鎮。灤河驛。鐵路。遷安繁,疲,難。府西北四十里。西北:九山,康熙中改五虎山。灤河自承德府入,合黃花川河、瀑河,又南,左得鐵門關水,入潘家口,古盧龍塞也。右納潵河,折東逕城西。漆河自建昌入,合白洋、冷口二河,為青龍河。巨梁水出西北黃山,一曰還鄉河。又沙河、石河、館水、徐流營、泉莊諸營田。營二:喜峰路、建昌路。汛八:龍井關、潘家口、李家峪、青山口、榆木嶺、擦崖子、冷口關、桃林口。三屯營、沙河堡、喜峰口三巡司。道光中,移三屯副將大沽口。太平寨、漢兒崖、沙河三鎮。七家嶺、灤陽二驛。撫寧沖,難。府東七十里。海,東南五十里。戴家河三源合於榆關南,為渝河,合獅子河,緣界。又西洋河二源納燕子河入焉。乾溝河起河東,自臨榆入。沙河西自遷安入,合為會河。汛二:界嶺口、臺頭營。鎮三:蒲河營、洋河口、深河堡。蘆峰口、榆關二驛。昌黎繁,難。府東南七十里。北:碣石山。海,東南三十餘里,突北出七里,一曰七里海。灤河自灤州入,左出,支津入焉,為甜水溝口。飲馬河自盧龍入,為沙河。四鎮:姜各莊、蒲河口、沙崖口、蛤泊堡。有鐵路。灤州難。府西南四十五里。海,南百三十里。有劉家河口,清河合沂河緣界入。西蠶沙口,小清河入。灤河自盧龍入。沙河自遷安入。館水亦自其縣入,曰陡河,亦曰檿牛河,合石溜河。州判駐胡各莊。三鎮:劉河口、稻地、開平。榛子鎮,巡司駐。鐵路。樂亭簡。府南少東百二十里。海,南四十五里。灤河自昌黎入,歧為二:東胡盧河,至老米溝;西曰定流,至清河口入灤。入海處五十里內凝碧,一曰綠洋溝。都行二千一百里。石碑場,西南。二鎮:西關裏、馬頭營。臨榆沖,繁,難。府東北百七十里。奉天奉錦道寄此。乾隆二年,以明山海衛置山海關。今東門古榆關。順治時置副將,後改游擊。道光末,與永平副將互徙。北有角山,長城枕其上。石河,古渝水,縣氏焉,譌「榆」。合鴨子河,帥府河入焉。故道在行宮西。其西湯河口。大清河出東北,入奉天寧遠。乾溝河、起河並出西北。汛四:義院口、大毛山口、寧海城、黃土嶺。小河口東曰柳邊。門二:鳴水塘、白石嘴。三鎮:海陽、乾溝、白塔嶺。西有陽化場。石門寨巡司。遷安驛。鐵路。

遵化直隸州:沖,繁,難。隸通永道。明,縣,屬薊州。康熙十五年,以陵寢隩區,升州,改隸順天。乾隆八年,復援易州例升直隸州,割永平之二縣來隸。西南距省治六百三十里。廣百六十里,袤三百七十里。北極高四十度十三分。京師偏東一度三十二分三十秒。領縣二。昌瑞山,西北七十里,本豐臺嶺,改鳳臺山,康熙初復改,東陵在焉。又西北霧靈山,淋、柳、潵橫四河源此。橫即潵右源,合東入遷安,與左源之黑河會。梨河古浭水,出東北蘆兒嶺,自遷安入,一曰果河,合沙河。又有雙女河、車道峪水。馬蘭峪、洪山口,總兵駐;與占魚口、大安口、羅文峪為五鎮。石門鎮,州判駐。又大窪汛、窩哨子、窄道子、老廠四鎮。西:半壁山。巡司二:駐州及石門。有丞。玉田沖,繁,難。州西南九十五里。雍正二年,自順天改屬。乾隆八年來隸。燕山,西北二十五里。北有黎河自州入,曰漳泗河,入薊曰沽河,復緣界曰薊運河。小泉河出東北,嘉慶末,建行宮其上,更名縈輝河,合藍泉、螺山水注之。還鄉河自豐潤入,合沙流河,逕雅鴻橋,合黑龍河,又西來會。雙城河出縣北黃家山,亦南來會。雅鴻橋,河主簿駐。嘉慶十二年,以河丞改。有陽樊驛。鐵路。豐潤沖,繁,難。州東南百里。改隸同玉田。海,南二百里。陡河自灤入,錯出復入,合倍河,分流復合,入為澗河口。東支金沱泊,支津西南合王家河。薊運河自玉田緣界。還鄉河自遷安入,納雙女河、車道峪水。同治中南決,至黑馬甸,於是有黑龍河,合泥河,並注薊運河。沙流河出西北。豐臺鎮西南,有河主簿、巡司。越支場,南百里,大使駐,今移宋家營。小集、畢家圈、開平營三鎮。又義豐驛。鐵路。

易州直隸州:繁,難。隸清河道。明屬保定,領縣一。雍正十一年,升直隸州。割山西大同之廣昌來隸。南距省治百四十里。廣二百六十里,袤二百二十里。北極高三十九度二十三分。京師偏西初度五十分三十秒。領縣二。西有行宮二:一、良各莊;一、泰寧鎮,總兵駐。有永寧山,西陵在焉。北:易濡水,出州西益津嶺,合安河、五里河,其東北即迎紫河。中易、白澗河,出西北武峰嶺,南易、雹水,出西南石虎岡,其南有徐河、澗河、界河。拒馬河自廣昌入,錯出復入,合小水以十數,入邊。口十八,飛狐最險。有塔崖、奇峰二廢巡司。鎮二:烏龍溝、紫荊關。康熙中,移副將真定,改置參將,轄白石口、廣昌營、浮圖峪、烏龍溝、凝靜菴五營。二驛:清苑、上陳。有丞,兼巡司。又州判駐。有鐵路。淶水沖,繁。州東北四十里。西北:檀山。拒馬河自州入,右出支津合鐵嶺水,又北東緣界復合。左出支津復入,合清水河。西南:北易亦自州入,合迎紫河,又東合遒欄河。口七。鎮二:大龍門、馬水口。舊稱京師右輔,有都司,轄大龍口、金水口諸汛。二鎮:水東營、秋瀾汛。黃莊鎮巡司。在城、石亭二驛。鐵路。廣昌簡。州西八十里。雍正十一年自山西大同來隸。城西淶水,譌「漆」,又借「七」,拒馬西源出焉。會東源,錯出復入。湯河自山西靈丘入。口八。鎮八。浮圖峪古銀防路,最險;插箭嶺口、白石口、胡核嶺口、黃土嶺口,又黑石嶺鎮,古飛狐口。縣驛一。鐵路。

冀州直隸州:繁,疲。隸清河道。明屬真定。領縣四。雍正二年,升直隸州。割正定之衡水來隸。北距省治三百里。廣百六十里,袤二百五十里。北極高三十七度三十八分五十秒。京師偏西初度四十七分三十秒。領縣五。滹沱、滏陽,舊自束鹿會縣西,入衡水。雍正初,滹北徙,與滏離,遂橫潰,後卒合滏順軌焉。北有枯洚渠。州驛一。南宮簡。州西南六十里。漳河故道三,中洚瀆,東南古漳,西北新漳。今復南徙,邑遂無水患。縣驛一。棗強繁,疲,難。州東南三十里。東:古漳河,一曰黃瀘河,自南宮入。西:索盧河。衛支津自州入。並涸。新河簡。州西少南六十里。西有滏陽河,自寧晉再入。有胡盧灣,舊與漳合處。縣驛一。武邑疲,難。州東北九十里。西:滏陽河自衡水入。又廢龍治河、老漳河。有水驛。衡水簡。州東北九十里。漳河衡流,古亦曰衡水。隋以氏縣。後為新漳河,乾隆中南徙。其滹沱今北徙。惟滏陽河自州入。古鹽河湮。縣驛一。

趙州直隸州:沖,繁。隸清河道。明屬真定。領縣六。雍正二年,升直隸州。改贊皇隸正定。東北距省治三百九十里。廣二百里,袤百四十里。北極高三十七度四十八分三十秒。京師偏西一度三十三分三十秒。領縣五。西北:洨河自欒城入,納豬龍河、冶河、新桃河。槐河自高邑入,綿蔓既合甘淘、冶河,而洨逕其故道,故即斯洨。太白渠下流亦被冶河目也。有滹沱故道,咸豐初淤。鄗城驛。柏鄉沖,繁。州南六十里。午河自臨城入。河及支津並自高邑入。而納新溝河。有槐水驛。隆平簡。州南九十里。東有滏陽河。灃河自任入。灃有九閘,雍、乾中建。北有泜河,自唐山入,合新溝水。河自柏鄉入,合支津及午河,曰槐午河。有驛。高邑簡。州西南五十里。北有槐河,自元氏入。南新溝河。河自贊皇入。縣驛一。鐵路。寧晉簡。州東南四十里。滏陽河自隆平入。有寧晉泊,周百餘里,匯其灃、泜、午及州之洨、槐諸水,自十字河來會,錯出復入。邑故澤國,康熙末,漳南徙,雍正初,滹東徙,怡賢親王復濬各水口,築堤設斗門,閼內外水出入,積潦始消。光緒中,滹沱復淤塞,半為平陸。有百尺口廢巡司。縣驛一。

深州直隸州:簡。隸清河道。明屬真定。領縣一。雍正二年升,以正定之武強、饒陽、安平來隸。衡水還屬正定。北距省治二百八十里。廣百四十里,袤百六十里。北極高三十八度三分四十秒。京師偏西初度四十七分。領縣三。州境自古病河、漳二水。河、漳先後他徙,滏、滋亦不甚橫。惟滹沱於乾隆十九年自束鹿分支潰入,同治七年復北徙,自安平入,諸故道並淤。有驛。武強簡。府東五十里。南:武強山,下有淵。滏陽自武邑入,至小範鎮北,奪滹沱故道。道光初,滏、滹同溢。有廢亭子、龍冶二河。有驛。驍陽疲,繁,難。州東北六十里。乾隆初,知縣侯鎯以滹為患,濬新溝七。同治中,唐世祿復疏經流三、支渠八,並注獻之古洋河。逾年復決安平。知縣吳恩慶築堤,首郭村,訖秦王莊,滹、滋始分。今滹沱中、南二支自州入,而古唐河自蠡入,半淤。有驛。安平簡。州西北五十里。滹沱中、南二支並自深澤入。豬龍河自祁入。其支津量石河,湮。有驛。

定州直隸州:沖,繁,疲,難。隸清河道。明領二縣。雍正二年升。十二年,以保定祁州之深澤來隸。新樂還屬正定。東北距省治百五十里。廣百四十里,袤二百里。北極高三十八度三十二分三十秒。京師偏西一度二十一分。領縣二。中山,城內,今設鐘鼓樓。北有唐河自唐入,始為患。乾隆中,南奪小清河。嘉慶中,復北奪小清河為今瀆。南有嘉河自曲陽入。沙河自新樂入資河。同治十年南徙,錯出復入會資河,自深澤緣界。唐、沙各故道及木刀溝並涸。有永定驛。鐵路。曲陽簡。州西北六十里。西北:恆山,古北嶽。順治末,改祀於山西渾源。恆水出其北谷,合三會河。唐河納縣北馬泥河,錯入。西北:沙河自阜平入,合平陽河,左得圓覺泉諸水。長星溝出西北孔山,側城東南,合曲逆溪、靈河,自是曰孟良河。縣驛一。深澤簡。州東南九十里。雍正十二年,自祁州來隸。滹沱、滋並自無極入。滹歧為三,北為經流。滋舊納支津木道溝,涸。乾隆初,決趙八莊,尋塞。復濬官道溝,導城西瀝水東注安平。縣驛一。


\end{pinyinscope}