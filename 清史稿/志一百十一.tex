\article{志一百十一}

\begin{pinyinscope}
○兵七

△海軍

中國初無海軍,自道光年籌海防,始有購艦外洋以輔水軍之議。同治初,曾國籓、左宗棠諸臣建議設船廠、鐵廠。沈葆楨興船政於閩海,李鴻章築船塢於旅順,練北洋海軍,是為有海軍之始。而甲申馬江,甲午東海,師船盡毀。嗣後兵艦歲有購置。自光緒中葉迄宣統初,南北洋海軍僅有船五十餘艘,舊式居半。其能出海任戰者,止海籌、海圻等巡洋艦四艘,楚泰、楚謙、江元、江亨等砲艦十餘艘而已。爰紀開創之漸,修繕之規,廠塢之建築,兵艦之購造,咸列於篇。

道光二十二年,文豐疏言購呂宋國船一艘,駕駛靈便,足以禦敵。旋諭隸水師旗營操演,並諭紳商多方購置。是為海軍購艦之始。

咸豐六年,怡良疏言,允英國司稅李泰國之請,置買火輪船,以剿粵匪。旋隸向榮調遣。

十一年,曾國籓疏請購買外洋船砲。奕等請以關稅買購外洋小兵輪十餘艘。飭廣東、江蘇各督撫募內地人學習駕駛。以已租之美利堅輪船二艘,一名土只坡,一名可敷本,為護運之用,配以砲械,駛赴安慶,隸曾國籓調遣。

同治元年,曾國籓於安慶設局,自造小輪船一艘。二年,令容閎出洋購買機器。四年,曾國籓、丁日昌於上海設鐵廠造槍砲。

五年,左宗棠疏請於福建省擇地設廠,購機器,募洋匠,自造火輪兵船。聘洋員日意格等,買築鐵廠船槽及中外公廨、工匠住屋、築基砌岸一切工程。開設學堂,招選生徒,習英、法語言文字、算學、圖畫。採辦鋼鐵木料。期五年內成大小輪船若干,均仿外洋兵船之式,需銀三百萬兩。並陳船政事宜十則,請簡重臣督理。旋以沈葆楨為船政大臣。

六年,李鴻章遷虹口制造局於高昌廟,建船塢,名曰江南制造局。沈葆楨疏言:「福州馬尾山,為省垣奧區天險,設船塢於馬尾之中歧。塢周四百五十丈有奇。鐵船槽長三十丈,寬十五丈,可修造二千五百頓之船。塢內濱江者為鐵廠、輪廠、斫木匠架木棧房。塢東北為船政大臣駐所,紳員公所及外國匠房。其左為法文、英文學堂及生徒住舍。江幹為煤廠,山麓為中國匠房,山之左駐楚軍一營,山之右為洋員駐所。傍江岸為官街,以便貿易。」旋派洋員日意格回國採辦器具,選用工匠。是年,瑞麟向英國訂購六兵船。

七年,沈葆楨疏言:「於船塢之右,創建船臺四座,臺長二十餘丈,船成入水,順推而下。其旁增五廠,曰鐵廠,曰水缸廠,曰打鐵廠,曰鑄鐵廠,曰合攏鐵器廠。規模既具,次第興工。」尋疏陳:「外洋機器到閩,復運煤運木於臺灣,運專於廈門。廠內增設轉鋸廠、木模廠、銅廠、風洞、繪事廠、廣儲廠、儲材廠、東西考工所,先後告竣。」疏入,諭英桂、馬新貽、李福泰、卞寶第等籌給經費,俾蕆要工。是年,曾國籓疏言:「上海設廠,自造第一號輪船告成。汽爐船身,皆考究圖說,自出機杼。長十八丈,寬二丈餘,命名恬吉。請續造二十餘丈之大艦。」旋諭兩江總督馬新貽等,從曾國籓、李鴻章所請,制器設廠,增建譯館諸端,悉心講求。是年,福州船廠自造安瀾等小輪船十艘告成,濟安、永保、海鏡等輪船亦告成。

八年,沈葆楨疏言:「廠中自制第一號大輪船下水,長二十三丈八尺,寬二丈七尺八寸,每小時行八十里,以副將率弁兵水手管駕,安置巨砲,駛出大洋,暫名曰萬年青。第二號暫名曰湄云,俟駛赴天津,再請錫名,以光海宇。」是年,購法國澄波兵船。江南制測海、操江兵船成。又購建威、海東雲二船。

九年,沈葆楨疏言:「第三號福星船,第四號伏波船告成,本屬戰艦,利於巡洋,以學堂上等學生移處船中,令洋員教其駕駛,由近而遠,以收實效。」是年,江南威靖兵船成。

十年,令學生十八人駕建威練船,巡歷南北各海口。是年,曾國籓疏請仿英國小鐵船式,令滬廠制造,為守海口之用。

十一年,船政制安瀾、鎮海、揚武、飛雲、靖遠五兵船成。文煜、宋晉等以造船費重,疏請暫罷,不許。是年,李鴻章疏言:「滬廠造成第五號船,長三十丈,鍋爐均在水線之下,置大砲二十六尊,系仿外洋三枝桅兵船式,英、法人稱為中國最巨之船。請飭沿江海各省,不得自向外洋購船,如有所需,向閩、水扈二廠商撥訂制,以節度支。」

十二年,江南制海安兵船成。沈葆楨疏言:「閩廠七號揚武、八號飛雲兵船下水。揚武用英國前膛砲,飛雲用布國後膛砲。以後十三、十四、十五號兵船,請兼仿外洋商舶之式,運載貨物,以裕經費。九號靖遠、十號振威、十一號濟安、十二號永保、十三號海鏡兵船已告成,以都司、游擊等管駕出洋。其建威練船,巡歷浙江、上海、天津、牛莊及香港、新加坡、檳榔嶼等處。在船學徒,練習風濤,成績甚優。來年遣散洋匠,以中國學徒自造。然能守已成之法,不能拓未竟之緒。請選擇學生,分赴英、法二國,深究造船駕駛之方,練兵制勝之理。」

十三年,船政制濟安、琛航、大雅三運船成。福建善後局購美國二砲船,曰福勝、建勝。李鴻章疏言:「中國東南北三洋,請各設大兵船六艘,根缽小兵船十艘,合成四十八艘。三洋各需大鐵船二艘。北洋駐煙臺、旅順等處,東洋駐長江口外,南洋駐廈門、虎門等處。鐵甲船每艘需銀百萬兩外,分年向外洋購置。餘船由閩、滬二廠仿造,以足四十八艘之數。請飭沿江沿海各省,裁並新舊紅單、拖罟、艇船、舢板等船,以節省之款,專練海軍。」是年,沈葆楨疏言:「續辦船工,尚有三端:一、挖土大機船,一、船土鐵脅,一、新式輪機。鐵脅須購自法國,以閩船皆法匠所造也。臥機、立機須購自英國,以其制精無弊也。」

光緒元年,制造局制馭遠兵船成。船政制元凱兵船成。以揚武練船令學生游歷南洋各處,至日本而還。尋諭南北洋大臣籌辦海防。令總稅務司赫德赴天津,與李鴻章商訂購英國二十六頓半、三十八頓半之砲船各二艘,專備海防之用。是年,沈葆楨購法國威遠兵船。

二年,沈葆楨會同李鴻章奏派學生,分赴英、法各國,入大學堂、制造局練習。此為第一屆出洋學生。是年,船政制登瀛洲、蓺新兩兵船成。制造局制金甌小鐵甲船成。

三年、四年,泰安、威遠、超武兵船亦成。沈葆楨疏請各省協款,每年解南北洋各二百萬,專儲為籌辦海軍之用,期十年成南洋、北洋、粵洋海軍三大枝,猶恐緩不濟急,請以四百萬先解北洋,俟成軍後,再解南洋。

五年,李鴻章疏言,外洋訂購之四砲船來華,以福建船政局員管駕,名飛霆、策電、龍驤、虎威,砲射甚遠,輪機亦精,請再購四艘。沈葆楨疏言,續購蚊砲船四艘到華,以留學英國畢業生管駕,名鎮東、鎮西、鎮南、鎮北,分防吳淞、江陰二口,為夾護砲臺之用。何璟疏言:「閩廠制造各兵船,惟揚武、威遠、濟安較為得力,其餘止供巡緝內洋之用。」旋諭沿江海各督撫整頓海軍。沈葆楨旋卒,海軍屬李鴻章。設海軍營務處於天津。

六年,江督劉坤一疏言:「蚊砲船購自外洋,費鉅而砲位過重。請由粵自造木殼船,丈尺與包鐵者同,砲位改用三萬餘斤之後膛砲,先造二艘,以備守口之用。」李鴻章疏請購外洋每半時行十五海里之快船及碰船、蚊子船。又疏言:「購辦鐵甲船之舉,倡議已歷七年。福建已定購蚊子船四艘,碰船二艘。請移二碰船之價一百三十萬兩,先購鐵甲船一艘,專歸臺灣防剿。以原有之福勝、建勝二蚊子船及船廠自造兵輪之堅利者,合為一軍,則臺防可固。南洋擬購之快碰船二艘,亦請抵購鐵甲船一艘。當與南洋大臣會商,合原有之各兵輪,編練海軍,互為應援。」旋以滬廠因鑄造槍砲經費過重,停造輪船。閩廠亦以財絀停造木質兵船,專造快船與鐵甲船。是年,吉林將軍銘安請於三姓一帶造舢板戰船。諭李鴻章籌度。鴻章覆陳俄國於庫葉島設廠造兵輪,輒由混同江駛入松花江等處,非舢板所能敵。請於三姓水深之處,設廠造蚊子船,可巡駛及黑龍江,以佐陸軍。鴻章尋向英訂造新式鐵甲船,並飭閩廠仿造。彭玉麟亦請飭閩廠分造十七、八丈之小兵輪十艘,以長江任戰之員為管帶,巡緝洋面。諭兩江、福建、廣東各省籌辦。是年,在英廠訂造之超勇、揚威快碰船來華,令提督丁汝昌管駕,與鎮中、鎮邊蚊砲船二艘,同泊旅順。又於閩廠訂造快船二艘,專為朝鮮口岸之用。李鴻章設水師學堂於天津。旋以在德國船廠定購之定遠、鎮遠二鐵艦,濟遠穹甲監將成,令管輪學生赴德國練習。令洋教習率鎮東等四兵船赴渤海一帶梭巡。是年,船政制澄慶兵船成。

七年,李鴻章在大沽口建船塢。九月,超勇、揚威二艦制成來華。鴻章乘赴旅順,察看形勢,籌備建築船塢、砲壘。大沽設水雷營、水雷學堂。旅順設水雷魚雷營、挖泥船。威海設魚雷局、機器廠,並設屯煤所。以丁汝昌統領北洋海軍。定兵艦國旗質地章色之制。會同福建船政派學生赴歐洲肄業。

八年,北洋、粵督各購德國雷艇數艘。以英人瑯威理司海軍訓練,與各國兵艦相遇,始有迎送交接之禮。李鴻章疏言英、法、美、德國近年所造船,曰穹面鋼甲快船,入水十五尺八寸,馬力二千八百匹,壓水力二千三百頓,每時行十五海裏,合中國五十里,機艙等鋼面厚三寸半,砲臺周圍鋼面厚至十寸,每艘需銀六十二萬兩,與鐵艦相輔,最為海軍利器。閩廠自造快船不及其精,已由出使大臣訂購一艘,與鎮遠鐵艦同駛來華。

九年,船政制開濟快碰船成。南洋向德國購南琛、南瑞二巡洋艦。

十年,船政制鏡清快碰船、橫海兵船成。李鴻章疏言,自光緒元年至六年,經營北洋海防,有龍驤等蚊砲船八艘,水雷小艇一艘。其龍、虎、霆、電四船,於六年撥赴南洋調遣。七年以後,先後購到超勇、揚威快碰船二艘,鎮中、鎮邊蚊砲船二艘。由閩廠調至北洋,修改練船之威遠、康濟兵輪二艘。調赴朝鮮、旅順等處,海鏡一艘。在滬制造之快馬小輪船一艘。在津制造之利順小輪船一艘。守雷、下雷所用之暗輪包鋼小輪船二艘。察看船只之大小,頓載之輕重,機器、砲位、桅帆器械之繁簡,配定人數、餉章,與水師統領、教習洋員分別損益,務使利器可得實用。是年五月,以長江水師提督李成謀總統南洋兵輪。總督曾國荃疏言:「江南購買兵輪蚊、快等船及自造者,為數無多。所有登瀛洲、靖遠、澄慶、開濟、龍驤、虎威、飛霆、策電、威靖、測海、馭遠、金甌大小兵輪,及新購之南琛、南瑞,上海機器局所造之鋼板保民兵輪,各船大小不齊,兵額不一,以之海戰則不足,以之扼守江海門戶,與砲臺相輔,藉固江防。」八月,法國海軍犯福建。駐防福州海口之揚武、振威、飛雲、伏波、濟安、福星、蓺新兵船七艘,蚊砲船二艘,琛航、永保商輪二艘,與法國兵船戰於馬江,悉數沈毀,存者惟伏波、蓺新二船。時李鴻章令德國武員率快船五艘,與曾國荃所部開濟、南琛、南瑞、澄慶、馭遠五船援閩,未至而閩師已覆,澄慶、馭遠二船亦沉於石浦。是年,總理衙門請設海軍專部。出使大臣許景澄在德國訂購之定遠、鎮遠鐵甲艦二艘,濟遠鋼甲艦一艘均告成。粵督向德國訂購雷艇八艘。

十一年,曾國荃疏言:「於福建、廣東、浙江三省增設鐵艦、快艦、雷艇。嗣後各兵船專事操練巡洋,不得載勇拖船。」與北洋大臣會奏,派第三屆學生出洋。同時,左宗棠疏請開採鐵礦,擇吳、楚扼要處,立船政砲廠,專造鐵甲兵船、後膛巨砲。制造局制保兵鋼板船成。九月,海軍衙門成立。以醇親王總理海軍,慶郡王、李鴻章為會辦,曾紀澤、善慶為幫辦。

十二年,粵省造淺水兵輪,曰廣元、廣亨、廣利、廣貞,防護海口。向德國購福龍魚雷艇一艘。三月,南洋兵船赴北洋會操。命醇親王、李鴻章校閱海陸軍及沿海臺壘。丁汝昌率兵船巡歷朝鮮。船政大臣裴廕森於福州增設練船,造鐵甲船,疏言:「江蘇之上海,廣東之黃埔,雖有船塢,而港道狹淺。福建羅星塔之下、員山寨之上,兩山間有天成巨港,請建大船塢,備定遠等鐵艦修理之處。」

十三年,閩廠寰泰快碰船、廣甲兵船造成,並造雙機鋼甲輪船及穹式快船、淺水兵輪。是年,北洋向英國購左一出海魚雷大快艇一艘,向德國購左二、左三、右一、右二、右三魚雷艇五艘,挖泥船一艘。北京設水師學堂於昆明湖,廣東設水師學堂於黃埔。

十四年,海軍衙門奏定官制,設提督、總兵、副將、參將、游擊、都司、守備、千總、把總、經制外委等官。是年,在英、德廠所造致遠、靖遠、經遠、來遠四快船來華。英百濟公司所造出海魚雷快艇亦告成。六月,臺灣番民叛,命致遠、靖遠二艦往剿平之。

十五年,船政制平遠鋼甲船、廣庾兵船成。

十六年,裴廕森疏言,閩廠修整龍威鋼甲兵輪,更名平遠,廣乙魚雷快船亦告成,並入北洋艦隊操演。又言石船塢告成,請簡專員董理。八月,北洋設水師學堂於劉公島,南洋設水師學堂於南京。十月,李鴻章疏言旅順口船塢工竣,堪為修理鐵艦之用,並築劉公島、青島等處沿海砲臺。北洋所聘海軍總查英人瑯威理,以爭提督升旗,辭職回國。英政府遂拒我海軍學生在英留學。

十七年,船政制廣丙魚雷快船成。二月,命直隸總督李鴻章、山東巡撫張曜出海閱海軍操。北洋之定遠等十二艦,廣東之廣甲等三艦,南洋之寰泰等六艦,畢會於旅順口,操演船陣槍砲魚雷,並勘砲臺、船塢。四月,戶部請停購外洋槍砲船隻機器二年,以所節價銀解部充餉。六月,提督丁汝昌率兵艦六艘赴日本東京。七月,威海增設魚雷三營。

十九年,船政制福靖魚雷快船成。粵督改水師講堂為水師學堂。

二十年,船政制通濟練船成。訂購英國砲艦一艘,命名福安。二月,鎮遠、定遠二艦置新式克鹿卜快砲十二尊。四月,朝鮮內亂,北洋遣兵艦往剿。五月,與日本兵船戰於牙山口外,濟遠船傷,廣乙船沈,操江船失,載兵之高升商船亦沈。九月,丁汝昌率北洋兵艦與日本戰於大東溝,失致遠、經遠、超勇、揚威四艦。

二十一年,日本以師船攻威海,定遠、鎮遠各艦亦失,丁汝昌敗死。冬,南洋訂購之辰、宿、列、張四雷艇來華。飛霆、飛鷹二驅逐艦在英、德廠造成。以康濟、飛霆、飛鷹、建靖各艦駐防北洋。以南洋之開濟、鏡清、寰泰、南琛,福建之福靖兵艦往來調防。

二十二年,福州羅星塔石塢成。閩浙總督邊寶泉請設法擴充船政。總理衙門疏陳:「船政始於大學士左宗棠、兩江總督沈葆楨。嗣後十餘年,泰西制造日精,閩廠雖有出洋畢業學生,而財力短絀,既不能增機拓廠,復不能制料儲材。自光緒八、九年後,以購買之機器,就廠合攏,成寰泰、鏡清、平遠、開濟各快艦。而新出之法,以無機無廠,不能急起謀新。同治年間所制之琛航、靖遠木質各監,馬力微者,又不適於用。凡一船之成,材居其七,工居其三。各材之中,屬煤、鐵、土、木等為生料,有產自中國者,有產自外洋者;屬鋼甲、鐵甲、帆、纜等為熟料,有能自制者,有必待增機廠而制者。請簡用重臣督辦,開採礦產,增購機械,獎勵學生,籌度經費,以期日起有功。」四月,在德國訂造海容、海籌、海琛三巡洋艦。五月,在英國訂造海天、海圻二巡洋艦。是年,以福州將軍裕祿兼船政大臣,諭加整頓。

二十三年,德國據山東膠州灣,法租廣州灣,英租威海衛,俄租旅順、大連灣。是年,船政制福安運船成。

二十四年,船政制吉雲拖船成。諭各督撫於船政原有經費外,別籌專款,以振海軍。

二十五年,在德訂購之海龍、海華、海青、海犀來華。諭沿海疆臣,增設海軍學堂,講求駕駛戰術。

二十六年,拳匪亂作,北洋各艦悉赴南洋。

二十七年,和議成,海容軍艦回防。

二十八年,船政制建威、建安魚雷快船、建翼魚雷艇成。又制淺水巡洋兵船二艘,一曰安海,一曰定海。是年,船政會辦魏瀚,以監督杜業爾不職,遣回法國。

二十九年,張之洞疏言:「南洋各兵艦年久,咸不適用,徒費國帑。各監惟寰泰、鏡清二兵輪,威靖、登瀛二運船,尚可備巡緝之用。其南瑞、南琛、保民三兵輪,龍驤、虎威、飛霆、策電四蚊船,請一律裁停。鈞和一船,令商人自養,為護商之用。以所節之款,積之十年,可購長江淺水新式快船六、七艘。」允之。是年,煙臺設海軍學校。江督向日本訂造江元淺水快船。

三十年,端方疏請選擇水師學生,由駐滬英國水師總兵,分派在英艦學習,較出洋游歷,費少而收效同。報可。南洋大臣周馥等,疏請以提督葉祖珪督辦南洋水師學堂、上海船塢。湖廣總督張之洞在日本廠購雷艇四艘,曰湖鵬、湖鶚、湖鷹、湖隼;淺水砲艦六艘,曰楚泰、楚同、楚豫、楚有、楚觀、楚謙。兩廣總督岑春煊開辦魚雷局於黃埔。

三十一年,以薩鎮冰總理南北洋海軍。江督在日本廠購淺水快艦三艘,曰江亨、江利、江貞。

三十二年,政務處王大臣疏言:「振興海軍,首重軍港。沿海惟象山港形勢合宜。請飭南北洋大臣勘度經營,以重戎備;並飭各省選派學生四十人,赴日本留學海軍。」

三十三年,設海軍處附於陸軍部內,設正副二使,機要、船政、運籌、儲備、醫務、法務六司。北洋大臣令海籌、海容二艦巡歷西貢、新加坡等處。商部令海圻、海琛二艦巡歷菲律賓島、爪哇島、蘇門答拉等處。粵督令廣亨、廣貞、安香、安東四艦巡歷九洲洋等處。

三十四年,江南船塢制甘泉、安豐二船成。派學生赴日本習航海各技術。

宣統元年,以貝勒載洵、提督薩鎮冰為籌辦海軍事務處大臣,度支部撥開辦費七百萬兩,各省每年分籌海軍費五百萬兩。六月,事務處成立,設參贊及八司,統一南北洋各艦為巡洋艦隊、長江艦隊。八月,載洵等赴歐洲各國考察海軍。令學生留學英國。

二年,江南船塢制聯鯨兵船成。日本訂購之二砲艦亦成。七月,載洵等赴日、美二國考察。尋在英造應瑞、肇和,在德造建康、豫章、同安、江鯤、江犀,在日本造永翔、永豐,在江南船塢造永建、永績,在揚子江造船公司造建中、拱辰、永安,在膠州船塢造舞風各軍鑒。冬,改海軍事務處為海軍部,以載洵、譚學衡為海軍部正副大臣,薩鎮冰為海軍統制,定九級官制。

三年,令海琛軍艦赴南洋各埠,撫慰華僑。六月,查察沿海砲臺。令海圻軍艦赴英賀加冕禮,旋赴美國。八月,江南船塢造澄海砲船成。是月,武昌變起,江海各兵艦悉附民軍。此建置海軍之概略也。

北洋海軍規制,北洋海軍,設於光緒中葉,直隸總督李鴻章實總之。其時有鎮遠、定遠鐵甲船二艘,濟遠、致遠、靖遠、經遠、來遠、超勇、揚威快船七艘,鎮中、鎮邊、鎮東、鎮西、鎮南、鎮北蚊砲船六艘,魚雷艇六艘,威遠、康濟、敏捷練船三艘,利運運船一艘。鎮遠、定遠弁兵各三百二十九人。致遠、濟遠、靖遠、來遠、經遠弁兵各二百二人。超勇、揚威弁兵各一百三十七人。左隊一號魚雷艇,弁兵二十九人。二號魚雷艇,三號魚雷艇,右隊一號魚雷艇,二號魚雷艇,三號魚雷艇,弁兵各二十八人。鎮中、鎮東蚊砲船弁兵各五十五人。鎮邊、鎮西、鎮南、鎮北弁兵各五十五人。威遠、康濟練船弁兵各一百二十四人。敏捷夾板練船弁兵六十人。利運運船弁兵五十七人。練勇學堂弁兵十四人。砲目練勇二百七十人。凡弁兵四千餘人。

其員弁之目:曰管帶,曰幫帶大副,曰魚雷大副,曰駕駛二副,曰槍砲二副,曰船械三副,曰舢板三副,曰正砲弁,曰水手總頭目,曰副砲弁,曰巡查,曰總管輪,曰二、三等管輪,曰水手正、副頭目,曰一、二、三等水手,曰一、二等管旗,曰魚雷頭目,曰一、二、三等升火,曰二等管艙,曰一、二等管油,曰一等管汽,曰油漆匠,曰木匠,曰電燈、鍋爐、洋槍、魚雷等匠,曰夫役,曰文案,曰支應官,曰醫官,曰一、二等舵工,曰一、二等雷兵,曰一、二、三等練勇,曰教習,曰學生。

其官制,設海軍提督一員,統領全軍,駐威海衛。總兵二員,分左右翼,各統鐵艦,為領隊翼長。副將以下各官,以所帶船艦之大小,職事之輕重,別其品秩。總兵以下各官船居,不建衙署。副將五員,參將四員,游擊九員,都司二十七員,守備六十員,千總六十五員,把總九十九員,經制外委四十三員。

其升擢之階,分為三途:曰戰官,由水師學堂出身,兼備天算、地輿、槍砲、魚雷、水雷、汽機諸學,及戰守機宜,充各船管帶,暨大、二、三副職事。曰藝官,由管輪學堂出身,充各船管輪,專司汽機者。曰弁目,由練勇水手出身,充砲弁、水手等,專司槍砲、帆繩者。各歸各途,論資升轉。提鎮大員等,請旨簡放。弁目等咨選海軍衙門送兵部帶領引見。統由北洋大臣節制調遣。

其考選海軍官學生也,一、英國語言文字,二、地輿圖說,三、算學至開平方諸方,四、幾何原本前六卷,五、代數至造對數表法,六、平弧三角法,七、駕駛諸法,八、測量天象推算經緯度諸法,九、重學,十、化學格致。肄業期四年,學成錄用。

其考選練勇也,招沿海漁戶年壯者充之。在練船練習帆繩蕩槳泅水及輪砲之操法,洋槍刀劍之操法。由三等遞升至一等,以備充補水手。水手以上各級,核其才藝勞績,以次遞擢。

其俸餉規制,曰官弁俸銀,兵匠錢糧,船上差缺薪糧,各船俸餉,官弁傷廢俸,兵丁加賞,行船公費,醫藥費,酬應公費,歲需銀一百七十六萬八千一百餘兩。

其定儀制也,曰冠服,曰相見禮節,曰國樂,曰軍樂,曰王命旗牌,曰印信。

其立軍規也,由提督秉公酌擬,呈報北洋大臣核辦,輕者記過,重者降級、革職、撤任。其餘不法等事,由提督援引會典雍正元年軍規四十條,參酌行之。

其簡閱巡防也,逐日小操,按月大操。立冬以後,各艦赴南洋,與南瑞、南琛、開濟、鏡清、寰泰、保民等艦合操,巡閱江、浙、閩、廣沿海要隘,至新加坡以南各島,保護華商,兼資歷練。每逾三年,欽派王大臣與北洋大臣出海校閱,以定賞罰。

水師後路,儲備有資,應時取給。船政由本境駐防提督主之。槍砲藥彈,收發考驗,則總管軍火專員主之。兵弁衣糧,因公用費,總管糧餉專員主之。他若學堂專員,測候譯書畫圖專員,醫藥專員,皆受命於海軍部,以專責任。旅順口大石船塢,及海口操防,特命文武大員董理。其大沽木船塢,海防支應局,旅順、天津軍械局、制造局,旅順魚雷營,威海機器廠、養病院,由北洋大臣簡員董理,規模略備。

自光緒二十一年海軍挫敗,所餘南洋各兵艦,新舊大小不齊,僅備巡防之用。後雖復設北洋統領及幫統官,董理海軍事宜,名存而已。

福州船廠,同治五年,創於閩浙總督左宗棠、船政大臣沈葆楨。閩縣馬尾江,距省會四十里,海口六十里。船塢,光緒十三年,創於船政大臣裴廕森,十九年告成。羅星塔距船廠三里,費二千餘萬,實為中國海軍之基。

其船廠所分隸者:一曰工程處辦公所,以洋員領辦公所,華員入工程處。

一曰繪事院,承繪船身、船機、鍋爐以及鑲配等總圖、分圖,圖成,乃按圖造船,兼精測算之學。院廣六千八百方尺,繪生三十九人。

一曰模廠,專任制造船模、汽鼓模各機件,以及細木雕刻各工。其能力須審圖理,諳折算,悉模型奧竅,辨五金冷熱漲縮之度。廠廣一萬五千一百二十方尺,設各種鋸機、刨機,各種旋機,凡二十具。工程繁時,匠額一百六十人,恆時四十七人。

一曰鑄鐵廠,專任船上所需之鑄銅鐵機件。其能力須諳圖理,明算術,仿木模制土模,及鼓鑄之時,辨明火候,研考銅鐵原質。曾鑄成重大鐵件達三萬斤,銅件達一萬斤。廠廣二萬八千八百餘尺,設鑄銅鐵大小爐凡十一座,轉運重件之將軍柱、碾機、風箱、風櫃凡二十三具。工程繁時,匠額一百六十餘人,恆時五十餘人。

一曰船廠,凡舢板、皮廠、板築廠咸屬之,專任船身工程。設石制船臺一座,長二百九十七英尺,木制船臺一座,長二百七十六英尺。凡船身長短廣狹,桅舵、艙位、頓載、速率、中心點度數,咸均算之。先繪經寸總圖,後繪全船,按圖造船。曾造木質、鐵質、鋼質、穹甲、鋼甲各式兵船四十餘艘。其能力可制四、五千頓之船。所有起蓋鑲配,亦廠中經理。設有鋸木機八架。所轄之皮廠,則制皮帶及各式皮件。舢板廠則制桅舵及大小舢板船。板築所則造船上爐灶,及各煙筒爐灶一切泥水修築各工。廠廣十五萬六千四百餘尺。工程繁時,匠額一千三百餘人,恆時一百五十人。

一曰鐵脅廠,專任制造鋼鐵船脅、船殼、龍骨橫梁、及船上鋼鐵件、拗彎鑲配各工廠。於光緒元年增設。其能力須審識船身圖理制度、鋼鐵原質各法。曾制成鋼甲鋼鐵船身二十餘艘。廠廣七萬九千八百餘尺,配設鋸機、翦機、鉆機、卷機、碾機、刨機三十五具。工程繁時,匠額七百人,恆時六十八人。

一曰拉鐵廠,專任拉制銅鐵,為制船所必需。能拉制重大之銅鋼鐵板、銅鐵槽條,及重大之輪機、轉輪軸、車軸、轉輪臂、汽餅桿、活軌、鐵錨各件。廠廣九萬四千四百餘尺,設汽鎚七架,汽鎚力大者至七頓。此外拉機翦旋床、鉆機刨床、並轉運重機之將軍柱,凡大小五十一具。拉銅鐵打鐵各爐,凡大小五十七座。工程繁時,匠額三百八十餘人,恆時八十七人。

一曰輪機廠,附屬有合攏廠,專任全船大小機器,制成,先在廠試驗,故合攏廠屬之。須較準中線,尤須審明圖理,通曉進脫冷暖壓助噓吃機關各竅汽力。廠廣三萬三千二百餘尺,設車光機、削機、刨機、礪石機、螺絲床、鉗床等,凡大小二百二十三具。工程繁時,匠額三百六十人,恆時一百二十人。

一曰鍋爐廠,專任鍋爐、煙筒、煙艙、汽表、向盤各工。其能力須審辨鋼鐵原質,究汽機之理由,天氣之漲力,以及鑲配法度。廠廣二萬九千六百尺,配設卷鐵床、水力泡丁機翦床、鉆床、船床,凡四十一具。工程繁時,匠額三百五十人,恆時一百十七人。

一曰帆纜廠,專造船上之風帆、天遮、帆索、桅上鑲配繩索,及起重搭架各工。其能力須審帆纜制度,登高工作,及風帆面積、繩索力度。廠廣一萬八千五百尺,不設機器,以手制為多。工程繁時,匠額七十人,恆時四十人。

一曰儲砲廠,專備收儲各船砲械、砲彈、魚雷各件。廠廣二千六十尺,恆時守兵二人。

一曰廣儲所,附設儲材所,專任收發銅鐵煤炭機件油雜各料件,儲材之所,專任收發各種木料。凡船政料件到工,先由兩廠驗收。其職任須審料質之良窳,慎重存儲。凡儲料棧房九座,廣四萬二千一百尺。儲煤廠廣一萬五千一百尺。廣儲所夫役,工程繁時六十人,恆時四十人。儲材所簰夫,工程繁時三十六人,今存二人。

一曰船槽,各國自建船塢後,多不設船槽,此槽乃初興船政時所設,可修一千頓以上之船。年久多損,僅能修整小船,較入塢為易。槽長三百二十二尺,上設機房,凡廣一萬七千三百尺。設拖船機四十架,大螺絲四十條,四十匹馬力一副。工程繁時,匠額六十人,恆時三十七人。

一曰船塢,建築費五十萬,塢身純用石砌,長四百二十尺,廣一百十尺,足容定遠等鐵艦,閩、粵、江、浙各兵艦,及外國兵艦,咸得入塢修整。並建抽水機廠、機器廠、丁役水手房、木料棧房等。面積凡二十九萬三千尺。有船入塢,由各廠飭匠修之。恆時匠額二十七人。

船政經費,同治十三年,首次報銷造船購費、蓋廠各費達五百十六萬兩,養船費十九萬兩。光緒二年後,船政常年費為六十萬兩。自同治五年至光緒三十三年,造船四十艘,用銀八百五十二萬兩。營造廠屋,用銀二百十一萬兩。裝造機器,用銀六十四萬兩。洋員歲俸,及修機器、置書籍,用銀五百五萬兩。學堂費六十七萬兩。養船費一百四十六萬兩。經營船政四十餘年,凡用銀一千九百萬兩有奇。此福州船廠、船塢之概略也。又制火磚練鐵,亦具規模。至光緒三十三年以後,洋監工全數遣散,遂無續制之船云。

旅順船塢,創議於光緒七年直隸總督李鴻章。時值外洋訂購兵艦到華,鴻章疏言,奉天金州旅順口形勢險要,局廠、船塢各工,當陸續籌興。九年二月,續陳旅順工程,開山濬海,工大費鉅,實難預為估定。旋由法國人德威尼承攪,鴻章派員督同興辦,並增築攔潮石壩。

十六年秋,全工告成,派員赴旅順驗工。所築大石船塢,長四十一丈三尺,寬十二丈四尺,深三丈七尺,石階鐵梯滑道俱全。塢口以鐵船橫攔為門。全塢石工,俱用山東大方石,堊以西洋塞門德土,凝結堅實,堪為油修鐵甲戰艦之用。其塢外停艦大石澳,東南北三面,共長四百十丈六尺,西面攔潮大石壩,長九十三丈四尺,形如方池。潮落時,尚深二丈四尺。西北留一口門,為兵船出入所由。四周悉砌石岸。由岸面量至澳底,深三丈八尺。周岸泊船,不患風浪鼓蕩。凡兵艦入塢油底之後,即可出塢傍岸,鑲配修整,至為便利。塢邊修船各廠九座,占地四萬八千五百方尺,為鍋爐廠、機器廠、吸水鍋爐廠、吸水機器廠、木作廠、銅匠廠、鑄鐵廠、打鐵廠、電燈廠。又澳南岸建大庫四座,塢東建大庫一座。每座占地四千八百七十八方尺,備儲船械雜料。以上廠庫,概用鐵梁鐵瓦,高寬堅固,足防風雪火患。又於澳塢之四周,聯以鐵道九百七丈,間段設大小起重鐵架五座,專起重大之物,以濟人力之窮。又於各廠庫馬頭等處,設大小電燈四十六座,為並作夜工之用。慮近海咸水之不宜食用也,遠引山泉,束以鐵管,由地中穿溪越隴,曲屈達於澳塢四旁,使水陸將士,機廠工匠,不致飲水生疾。又慮臨海遠灘之不便起卸也,建丁字式大鐵馬頭一座,使往來兵艦上煤運械,不致停滯。其餘如修小輪船之小石塢,藏舢板之鐵棚,系船浮標鐵臼,以及各廠內一應修船機器,設置完備。於是年九月二十七日工竣。由是日起,限一年,系代德威尼擔保之銀行照料。限滿,再保固十年,均與包工監工洋人訂明。此項工程,共用銀二百餘萬兩。甲午後,遂迭為日、俄所踞云。

沿海軍港,旅順、威海既失,海軍無駐泊之所,於是籌邊者起議築港。宣統初,命親籓南下,建築未遑。而沿海七千里,港灣鱗次,就海軍部所預籌,分為四區。第一區,營口在奉天遼河左岸錦州灣,為渤海兩岸之良港。大沽口為直隸諸水入海所匯。秦皇島東控山海關,為不凍之港。長山列島分內外三層,為旅順外援。大連灣在遼東半島南。芝罘港在山東福山縣,三面負山,北臨渤海。第二區,揚子江口為沿江七省之門戶,沙灘連亙,多喑礁。舟山在定海縣,諸山環列,為杭州海灣之屏蔽。象山港海深可泊巨艦,為寧波後路。三門灣在臨海縣,有三門列島,海水甚深。第三區,永嘉灣即甌江口,三都澳即三沙灣,在福建霞浦縣,港口水淺,港內水深,容大軍艦。福州灣即閩江口,群島林立,淺岸交錯,為完固之港。海檀島為閩省海岸中樞。廈門港有廈門、金門二島,近接臺灣。汕頭港在廣東澄海縣,崖岸峻險。番禺灣即廣州灣,巨石環列,擅天然形勢。第四區,海口島在廣東瓊山縣北,與雷州島對峙,為扼隘之所。榆林港在瓊州島南,背負崖壁,前臨東京灣。以上各港,惟象山港、三都澳確定為修築軍港之地。他如北塘口、榮成灣、靖海灣、葫蘆島、大鵬灣、廟島等處,亦由漸擴充云。

外國訂購各兵艦,始於咸豐十年,廷議購船艦砲位助剿粵寇。十一年,總理各國事務衙門與總稅務司會商購買。自同治、光緒朝迄宣統初年,歷五十年,得船不及百艘,爰依次歲月列其船名。凡所購之國,所造之廠,及丈尺、馬力、頓數、砲位、兵弁咸詳之。其應瑞巡洋艦一艘,永豐、永翔砲艦二艘,建康、豫章、同安驅逐艦三艘,建中、永安、拱辰淺水快船二艘,告成於宣統三年後者不與焉。

金臺船、原名北京。一統船、原名中國。廣萬船、原名廈門。得勝船、原名穆克德恩。百粵船、原名廣東。三衛船、原名天津。鎮吳船,原名江蘇。同治元年,在英國訂購。二年到華,價銀八十萬兩。以英國總兵阿思本為總統,以長江水師武員分統各船。旋議以武職大員為漢總統,阿思本副之。是年六月,李鴻章以金陵垂克,勿庸外國兵船助剿,疏請所購七船,令阿思本駕駛回英,變價售賣,款歸中國。所募水兵,一律遣散。

天平船,同治二年,由總稅務司購買。

安瀾船、定濤船、澄清船、綏靖船、飛龍船、鎮海船,同治五、六年間,兩廣總督瑞麟自英國購置,價銀二十四萬兩。

恬波船,同治七年,兩廣總督瑞麟自法國購置,價銀四萬兩。

海東雲船,原名五雲車。同治九年,閩浙總督英桂自洋商購置,以武員管駕,巡緝臺灣洋面。

建威練船,同治九年,閩浙總督英桂購自德國,為駐練學生之用。

福勝砲艦、建勝砲艦,同治十三年,福建善後局購自美國,光緒二年到華,價銀二十四萬兩。

龍驤砲艦、虎威砲艦、飛艇砲艦、策電砲艦,光緒元年,直隸總督李鴻章自英國阿摩士莊廠訂購,每艘價銀十五萬兩,撥歸南洋調遣。

鎮東砲艦、鎮西砲艦、鎮南砲艦、鎮北砲艦,光緒元年,兩江總督李宗羲自英國阿摩士莊廠訂購,每艘價銀十五萬兩,撥歸北洋調遣。

鎮中砲艦、鎮邊砲艦,光緒七年,李鴻章代山東省自英國船廠訂購,每艘價銀十五萬兩。

超勇巡洋艦、揚威巡洋艦,光緒五年,李鴻章自英國阿摩士莊廠訂購。六年,令提督丁汝昌率員弁二百餘人,赴英國駕駛回華。二艦均木身鋼板。

定遠鐵甲艦、鎮遠鐵甲艦,光緒六年,李鴻章自德國伏爾鏗廠訂購,價銀六百二十萬馬克。十一年來華,附小魚雷艇三艘,魚雷筒三具,小輪船一艘。

濟遠鋼甲艦,光緒六年,與定遠船同廠訂購。

單雷艇二艘,光緒八年,由德國訂購,歸北洋調遣。

雷龍魚雷艇、雷虎魚雷艇、雷中魚雷艇,光緒八年,兩廣總督張之洞由德國訂購。

雷乾魚雷艇、雷坤魚雷艇、雷離魚雷艇、雷坎魚雷艇、雷震魚雷艇、雷艮魚雷艇、雷巽魚雷艇、雷兌魚雷艇,光緒十年,兩廣總督張之洞由德國訂購。

南琛巡洋艦、南瑞巡洋艦,又名運送艦。光緒九年,兩江總督左宗棠由德國伏爾鏗廠訂購。

福龍魚雷艇,光緒十二年,由德國訂購。十六年隸北洋海軍。

致遠巡洋艦、靖遠巡洋艦,光緒十二年,由英國訂購,船價及砲位,凡銀一百六十九萬有奇。經遠巡洋艦、來遠巡洋艦,光緒十二年,由德國訂購,船價及砲位,凡銀一百七十三萬有奇。光緒十三、四年,與致遠、靖遠先後到華,均隸北洋海軍。

左隊一號魚雷大快艇,光緒十二年,直隸總督李鴻章由英國百濟公司訂購,價銀八萬有奇,十三年到華。

左隊二號魚雷大快艇、左隊三號魚雷大快艇、右隊一號魚雷大快艇、右隊二號魚雷大快艇、右隊三號魚雷大快艇,以上魚雷艇六艘,光緒十二、三年,先後由德國船廠購買材料,到華配合,以德員教授。

辰字魚雷艇、宿字魚雷艇,由德國伏爾鏗廠訂購,光緒二十一年到華。

列字魚雷艇、張字魚雷艇,由德石效廠訂購,光緒二十一年到華。

福安砲艦,光緒二十年,由英國阿摩士莊廠訂購。

飛霆驅逐艦,光緒二十一年,由英國阿摩士莊廠訂購。

飛鷹驅逐艦,光緒二十二年,由德國伏爾鏗廠訂購。

海天巡洋艦、海圻巡洋艦,即穹甲快船。光緒二十二年,由總稅務司在英國阿摩士莊廠訂購,每艘價值三十二萬八千二百四十二鎊。

海籌巡洋艦、海容巡洋艦、海琛巡洋艦,光緒二十二年,由總理衙門在德國伏爾鏗廠訂購,每艘價值十六萬三千鎊。二十四年,與海天、海圻巡洋監先後到華。

江元砲艦、江亨砲艦、江利砲艦、江貞砲艦,由兩江總督在日本川畸廠訂購。江元於光緒三十三年告成。江亨於三十四年告成。江利、江貞於宣統元年告成。先造一艘,價日本金三十一萬五千元。續造三艘,每艘價日本金二十九萬三百二十五元。

湖鵬魚雷艇、湖鶚魚雷艇、湖鷹魚雷艇、湖隼魚雷艇,由湖廣總督張之洞在日本川畸廠訂購。湖鵬、湖鶚二艇,於光緒三十三年到華。湖鷹、湖隼二艇,於三十四年到華。每艘合日本金三十八萬元。

楚泰砲艦、楚同砲艦、楚豫砲艦、楚有砲鑒、楚觀砲艦、楚謙砲監,均航海砲艦。由湖廣總督張之洞在日本川畸廠訂購。楚同、楚泰、楚有三砲艦,於光緒三十三年二月到華。楚豫、楚觀、楚謙三砲艦,於十月到華。每艘合日本金四十五萬五千元。

海龍魚雷艇、海青魚雷艇、海華魚雷艇、海犀魚雷艇,在德國實碩廠訂購,於光緒三十四年到華。

舞鳳航海砲艦,宣統三年,在青島德國船廠訂購。

江犀砲艦、江鯤砲艦,均淺水砲艦,原名新璧、新珍。江犀艦在德國克魯伯廠訂購,江鯤艦在德國伏爾鏗廠訂購,均以材料運華,宣統三年,在江南造船所配合,每艘價值一萬八千九百八十鎊。

肇和巡洋艦,宣統三年,在英國阿摩士莊廠訂購,價值二十一萬鎊。

福州船廠,自造各兵艦。始建船廠,聘工師於法,延教員於英。建船臺,購機器。同治八年秋,第一號萬年清輪船成。十二年冬,華匠漸諳制造,廠機亦稍備,乃遣散洋員。凡九年,成大小兵船、商船十五艘,成於洋員者十二,成於華匠者三。光緒三年,始遣學生、藝徒至英、法二國留學。六年歸國,制造、駕駛,悉以任之。其制船之質,始皆以木,繼易木脅為鐵脅,易木板為鐵板,更進則純用鋼脅、鋼板,且護以鋼甲。船機則由立機改臥機。船式則由常式為快船、為穹甲、為鋼甲。至光緒三十三年,成船達四十艘。凡商船八艘,木質兵船十四艘,鐵脅木質兵船五艘,鋼脅木質兵船一艘,鐵甲雙重木質快碰船三艘,鋼甲兵船一艘,鋼甲鋼脅魚雷快船六艘,鋼脅拖船一艘,鋼脅練船一艘。已失者二十六艘。存者十四艘,曰湄雲、曰伏波、曰靖遠、曰琛航、曰元凱、曰登瀛洲、曰鏡清、曰通濟、曰福安、曰吉雲、曰建威、曰建安、曰建翼、曰淺水江船。備列船制於後:

湄云,木質兵船,船價銀十六萬三千兩,同治八年八月造成。

福星,木質兵船,船價銀十萬六千兩,同治九年九月造成。

伏波,木質兵船。船價銀十六萬一千兩,同治十年二月造成。

安瀾,木質兵船,船價銀十六萬五千兩,同治十一年十一月造成。

鎮海,木質兵船,船價銀十萬九千兩,同治十一年六月造成。

揚武,木質兵船,船價銀二十五萬四千兩,同治十一年十一月造成。

飛雲,木質兵船。船價銀十六萬三千兩,同治十一年九月造成。

靖遠,木質兵船,船價銀十一萬兩,同治十一年十一月造成。

振威,木質兵船,船價銀十一萬兩,同治十二年二月造成。

濟安,木質兵船,船價銀十六萬三千兩,同治十三年三月造成。

永保,木質武裝商船,船價銀十六萬七千兩,同治十二年九月造成。

元凱,木質兵船,船價銀十六萬二千兩,光緒元年八月造成。

藝新,木質兵船,船價銀五萬一千兩,光緒二年閏五月造成。

登瀛洲,木質兵船,船價銀十六萬二千兩,光緒二年七月造成。

泰安,木質兵船,船價銀十六萬二千兩,光緒三年三月造成。

威遠,鐵脅木殼兵船,船價銀十九萬五千兩,光緒三年八月造成。

超武,鐵脅木殼兵船,船價銀二十萬兩,光緒四年八月造成。

澄慶,鐵脅木殼兵船,船價銀二十萬兩,光緒六年十一月造成。

開濟,鐵脅雙重快碰船,船價銀三十八萬六千兩,光緒九年八月造成。

橫海,鐵脅木殼兵船,船價銀二十萬兩,光緒十年二月造成。

鏡清,鐵脅雙重木殼快碰船,船價銀三十六萬六千兩,光緒十年七月造成。

寰泰,鐵脅雙重木殼快碰船,船價銀三十六萬六千兩,光緒十三年七月造成。

廣甲,鐵脅木殼兵船,船價銀二十二萬兩,光緒十三年十月造成。

平遠,鋼甲鋼殼兵船,船價銀五十二萬四千兩,光緒十五年四月造成。

廣乙,鋼脅鋼殼魚雷快船,船價銀二十萬兩,光緒十六年十月造成。

廣庚,鋼脅木殼兵船,船價銀六萬兩,光緒十五年十月造成。

廣丙,鋼脅鋼殼魚雷快船,船價銀十二萬兩,光緒十七年十月造成。

福靖,鋼脅鋼殼魚雷快船,船價銀二十萬兩,光緒十九年十月造成。

通濟,鋼脅鋼殼練船,船價銀二十二萬六千兩,光緒二十年八月造成。

吉雲,鋼脅鋼殼拖船,船價銀五萬六千兩,光緒二十四年八月造成。

建威,鋼脅鋼殼魚雷快船,船價銀六十三萬七千兩,光緒二十八年十一月造成。

建安,鋼脅鋼殼魚雷快船,船價銀六十三萬七千兩,光緒二十八年十一月造成。

建翼,鋼脅鋼殼魚雷艇,船價銀二萬四千兩,光緒二十八年五月造成。

廣東船廠,自造各兵艦,光緒十二年,兩廣總督張之洞於省河設廠,選募華工,採用香港英國船廠圖說,自造淺水兵輪四艘,曰廣元、廣亨、廣利、廣貞。

直隸大沽船塢,自造拖船,遇順暗輪鋼拖船,光緒十四年造成,又守雷暗輪包鋼小輪船一艘,下雷暗輪包鋼小輪船一艘。

江南船廠,自造各兵艦。咸豐十一年,曾國籓始有購買船砲及中國試造輪船之疏。同治二年,於安慶設局,不用洋員,自造一小輪行駛。令容閎出洋購買機器。四年,國籓於上海虹口奏設制造局。李鴻章撫蘇,偕丁日昌於上海鐵廠專造槍砲,以供征伐。六年四月,國籓疏請撥留洋稅一成,為專造輪船之用。汽爐、機器、船殼三者,咸研究圖說,自出機杼。先造汽爐廠、機器廠、熟鐵廠、洋槍樓、木工廠、鑄銅鐵廠、火箭廠、庫房、棧房、工務房、工匠室,以應要需。復築船塢以整破舟,建瓦棚以儲材料,立學館以譯圖說。建築既堅,規模亦肅。同治六年,李鴻章建江南制造局,從事制船。八年,測海、操江兩兵船制成。九年,威靖兵船成。以萬金購德國船為練船之用。十二年,海安兵船制成。光緒元年,馭遠兵船制成。二年,金甌小鐵甲船制成。五年,兩江總督沈葆楨疏言:「江南船廠所制兵船,五百匹馬力以下者五艘,其兵數餉章,與福州所造各兵船相等。」八年,購外洋商船一艘,改造為防緝之用,名曰鈞和。嗣後未有造作。

光緒三十年,南洋大臣周馥等,以南洋近年以來,舊有兵船,日益窳朽,徒糜餉項,無裨實際,亟應分別裁留,認真整理。非定章程,不能革除舊習;非專派大員督辦,不能造就將才。因奏派現統北洋海軍廣東水師提督葉祖珪督辦南洋水師學堂、上海船塢,凡餉械支應一切事宜,有與海軍相關者,均歸考核。嗣復奏稱江南制造機器總局內舊有船塢,本為制造官商輪船並修理船械而設,日久弊生,多糜經費,而辦理之員,類無造船專門之學,以致承修船隻,工價高昂。近年以來,商船裹足不前,兵船反入洋塢,非認真整理,無由振興。經與北洋大臣會商,定議船塢別簡大員經理,仿商塢辦法,掃除舊習,妥籌改良船塢,與海軍事相表裏。廣東水師提督葉祖珪,系總理南北洋海軍,往來津、滬,則上海船塢事宜,自應歸其督察,以一事權。遂將船塢與制造局劃分,名曰江南船塢,制造局歸陸軍部轄,船塢歸海軍部轄,以專責任。

此後制造復興,三十四年,甘泉、安豐二兵船成。宣統二年,聯鯨兵船成。三年,澄海砲船成。

海軍自甲午戰後,所餘南洋各艦,不復成軍。嗣後逐漸購置,其編制非復北洋舊章。每艦設艦長一員,副長一員,協長一員,航海正一員,航海副一員或二員,槍砲正一員,槍砲副一員或二員,魚雷正、魚雷副一員或二員,輪機長一員,輪機正一員或二員,輪機副一員至八員,軍需正一員,軍需副一員或二員,軍醫正一員,軍醫副一員或二員,書記官一員。

其戰艦約分新舊二類,新式而有武力者,巡洋艦四,曰海圻、四千三百頓。曰海容、曰海琛、曰海籌。各二千九百五十頓。砲艦十一,曰楚泰、曰楚謙、曰楚觀、曰楚豫、曰楚有、曰楚同、各七百八十頓。曰江元、曰江亨、曰江利、曰江貞、曰江鏡。各五百頓。水雷砲艦一,曰飛鷹。八百五十頓。其屬於舊式者,巡洋艦五,曰通濟、一千九百頓。曰南琛、一千九百零五頓。曰鏡清、一千一百頓。曰保民、一千四百七十七頓。曰登瀛洲。一千二百五十八頓。水雷砲艦二,曰建威、曰建安。各八百十七頓。砲艦二十,曰泰安、曰甘泉、曰廣玉、曰廣戌、曰靖海、曰廕洲海、曰並徵、曰海鏡清、曰廣金、曰廣己、曰廣庚、曰策電、曰第電、曰海長清、曰清海、曰鈞和、曰飛虎、曰靖遠、曰綏遠、曰鎮濤。共一萬零八百二十七頓。報知艦四,曰超武、曰琛航、曰元凱、曰伏波。共五千一百七十七頓。雷魚艇八,曰湖鵬、曰湖隼、曰湖鶚、曰湖鷹、曰辰、曰宿、曰列、曰張。共一千頓。新舊大小各艦凡五十五艘。


\end{pinyinscope}