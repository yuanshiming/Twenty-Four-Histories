\article{列傳二百三十八}

\begin{pinyinscope}
李朝儀段起丁壽昌曾紀鳳儲裕立鐵珊桂中行

劉含芳陳黌舉游智開李用清李希蓮李金鏞

金福曾熊其英謝家福童兆蓉

李朝儀,字藻舟,貴州貴築人。道光二年進士,授直隸平谷縣,歷署饒陽、三河。咸豐初,遷大興京縣,署南路同知,補東路同知,皆有治聲。時海防戒嚴,築寧河、北塘、大沽諸砲臺,工堅費核,平餘銀鉅萬,悉以入官,晉秩知府。十年,署順德。捻匪北竄,朝儀率鄉勇出御,嚴陣以待,砲折大旗,迄不動。益使游騎左右馳突為疑兵,賊來則擊之,退則寂守,久之,賊引去。同治四年,署廣平,敗賊馬甿橋,悉收難民入城,料賊必復至,儲糧械為城守備。已而賊眾數萬果逼城,不敢犯,城獲全。五年,補大名。馬學孟者,故捻黨也,善戰,有勇力。既投誠,充團總,濬、滑、內黃數縣民多附之,其黨有殺人者,遠近因傳學孟叛矣。朝儀馳入其居,曉譬利害,學孟悟而泣,原繳械請罪,遂夷其寨,赦勿問。後朝儀與賊戰,得學孟死力,故不敗。

八年,授永定河道,署按察使。先後任河道八年,勤於職守,痛革河工積弊,課兵種柳,資工用焉。遷山東鹽運使,尋擢順天府尹。京畿靡薄,朝儀廉勤率之,捕劇盜,抑豪強,絕請託,期年風習一變。光緒七年,卒官。朝儀治河績尤著,民立祠固安祀之。

段起,字小湖,湖南清泉人。初入貲助餉,敘道員。咸豐初,佐廣西左江道王普相幕,數陳兵事。普相薦諸巡撫勞崇光,俾將百人,從解全州圍。別寇鄧正高乘虛襲永州,窺衡州,起單騎馳諭降其眾。貴州叛苗犯懷遠,起討平之。奉檄率所部援江西,謁曾國籓於軍中,國籓未之奇也。時賊踞建昌,久不下。起夜率四百人撲其壘,克之,乘勝復德安,國籓乃納其軍。七年,從劉騰鴻、李續宜攻瑞州,騰鴻戰死,起亦被重創,卒克之。八年,援浙,解衢州圍,還攻景德、浮梁,並克之。明年,陳玉成犯景德,起扼其沖,賊不得逞。出家財募勇,遣別將率以援浙,數有功。巡撫王有齡疏調起赴浙將水陸軍,會以前功加鹽運使銜,留江西以道員補用。十一年,李秀成犯廣豐,遂圍廣信。起嬰城固守,伺間出擊賊,敗之,賊遂引去,加布政使銜。同治元年,授江西督糧道,仍留治軍。二年,克鄱陽、彭澤,給瑚松額巴圖魯名號。

三年,始赴任。時軍事漸定,議撤兵。起條上兵弁安置之策,巡撫沈葆楨疏請頒行,武職借補及收標考課,著為令。四年,鮑超軍索餉譁潰,起聞變馳視,遇前隊,傷頤,有識者大呼曰:「段糧道也!」皆棄兵拜,起反覆開譬,變乃定。尋兼署按察使。江西、閩、浙之交,有山綿亙千里,故為盜藪,久封禁。賊未平時,民往往入山避亂,久之生息日繁。至是或頗言粵寇餘孽窟穴其中,詔三省會剿,起疑之,輕騎周歷詢訪,悉其狀,牒大吏疏請弛禁,民德之,立生祠祀焉。六年,以疾歸。邑大饑,傾貲賑贍,全活逾萬家。光緒二年,再授江西督糧道,調江南徐州道。六年,兩廣總督張樹聲調治海防,擢廣東鹽運使。八年,卒於官。

丁壽昌,字樂山,安徽合肥人。少為里塾師,粵寇擾淮南,遂集里中子弟勒以兵法,築寨自保。同治初,率偏師從李鴻章東征,轉戰蘇、松間,由知縣晉秩知府。隨潘鼎新攻浙江,克乍浦,攝乍浦同知。又隨克嘉興,晉道員。進攻湖州,戰於晟舍鎮,賊憑河為險。壽昌鳧水破其兩壘,諸軍隨擊,立克之,湖州賊遂不振。論功,加按察使銜。六年,率師從劉銘傳剿捻,迭敗之黃安、鄧州。賊南竄沭陽,霖雨,平地水數尺,捻酋任化邦竄渡沭水而西。壽昌先解衣率將士徒涉,伐木為梁濟軍,既濟,乃斷梁。眾知無退路,奮擊破賊,追斬化邦贛榆城下。詔以道員簡放,加布政使銜。又戰濰縣,擒捻酋李蕓等,給西林巴圖魯勇號,記名按察使。

八年,天津民、教構釁,命壽昌率銘軍四千馳赴津、沽備非常。遂署天津道,尋實授。時人情洶懼,譌言繁興。壽昌處以鎮靜,扶良詰奸,屬境安堵。救火會董積憤西教,適大火,相約不救教堂。壽昌聞警奔赴,略無畛域。會董感其誠,乃施救。梁家園河堤將圮,壽昌親執畚立水中,眾益奮築,堤獲全。設廠以賑流民,廬灶籓溷悉有程式。會遭父喪,士民奔走籥留者萬人,堅請終制。服除,詔赴天津總理營務,兼充海防翼長。光緒四年,署津海關道,擢按察使,署布政使,以勤慎稱。六年,卒官。賜恤,贈太常卿,於天津建立專祠。

曾紀鳳,字摯民,湖南邵陽人。以諸生從軍,洊保知縣。駱秉章督四川,調領湘果後營。同治元年,石達開竄踞敘州雙龍場,分軍陷高縣。紀鳳從按察使劉岳昭赴援,戰城下,克之。又迭敗之吊黃樓、羅家坳,涉水先驅奪賊壘。達開連營三十,與橫江為犄角。紀鳳毀橫江西岸賊巢,遂薄雙龍場。計招賊黨為內應,而潛軍襲其後。達開奔燕子灘,邀於橫河,半渡,擊之,遂竄滇境。三年,從克正安,進圍綏陽,屢戰有功,晉知府。尋調廣東,又調貴州,並任軍事。十年,與總兵鄧千勝克麻哈,擒楊阿保,晉道員。

十一年,會諸軍剿平苗民之梗化者。貴州下游東西驛道,苗在其南,漢民在其北。自咸豐時,行旅阻隔,垂二十年,至是始通。紀鳳辦理善後,自黃平以上歷清平、平越、麻哈、貴定二百餘里,建碉七十,分立四屯,各設屯官,戍卒六百分守之。墾荒供餉,責以巡緝。奸宄無所容,流民聞風踵至。十二年,古州苗叛,擾清江,旁寨響應。紀鳳率碉兵會諸軍進剿,擒其酋長,撫良苗百數十寨。黔疆略定,賜黃馬褂。光緒元年,授貴西道,巡撫黎培敬深倚之,薦可大用。擢按察使,晉布政使。十二年,調雲南布政使,剿惈黑及大戛寨夷,加頭品頂戴。因請以其地改土歸流,邊隅以安。十五年,乞養歸,尋卒。

儲裕立,字鶴樵,湖南靖州人。從軍貴州,累保知縣。同治初,苗亂熾,迭克天柱、清江,晉知府。十年,署古州同知。兵後彫攰,群苗伺釁出沒。裕立修戰備,撫遺黎,民氣漸復。仍統軍先後收復臺拱、丹江、凱里諸城,擢道員。光緒三年,下游肅清,論功,賞黃馬褂。督治善後,築城堡百二十七,建義塾百三十九。八年,思南災,裕立往賑,遍歷災區,日稽錢粟出入,無假借,實惠及民。時遵義焚教堂,民情洶洶。裕立馳往撫諭,與法人往復詰難,事得解。尋署貴西道,再歷貴東糧儲。二十一年,卒,賜恤如例。

鐵珊,字紹裴,徐氏,漢軍正白旗人。咸豐中,由筆帖式議敘知縣。從欽差大臣勝保討捻山東,單騎入賊壘,招降捻匪劉占考,散其黨數萬。敘功,以直隸州選用。同治初,發甘肅,署通渭。值回亂,一歲九被圍,嬰城固守,卒得全。日供軍糧萬斤,民不堪命。鐵珊規減其半,民感德。及去任,攀轅不得行。迭攝平番、皋蘭、中衛諸邑,所至輒輕賦役,輯流亡,修城堡,除蠹胥。總督上其治狀,擢寧夏知府,未之任,調蘭州。議建貢院,與陜西分試,自光緒紀元始。是年,署甘涼道,武威、永昌、鎮番三邑共一渠,民爭水械斗,久不決。鐵珊為開支渠,別子母水,設徬刊石,立均水約,輪日灌溉,民大悅,為立祠渠上。地宜牧,因畜羊三千頭,歲以蕃息,用給貧民無告者。十三年,擢河陜汝道,擒巨盜李復岐等,置諸法。建陜州書院以課士,文風始振。閿鄉城北濱河,南臨澗水,歲屢圮,議築石壩殺水勢,艱於鑿運,竟得石閿底鎮激湍中,工遂成。十六年夏,淫雨河漲,陜城不沒者數版。民謂官能捍患,恃以不恐。鐵珊復築石堤,四月畢工,身親其役,竟以勞卒。士民請建專祠,詔賜恤。

桂中行,字履真,江西臨川人,先世賈貴州,遂占籍鎮遠。為諸生。咸、同間,積軍功,為知縣安徽,署合肥、蒙城、阜陽。曾國籓率師征捻,檄中行察勘蒙城圩寨。蒙城故捻藪也,中行單騎歷諸圩,曉以利害,擇良幹者為圩長。堅壁清野,寇無所掠。禮接耆老賢士,從詢方略。得通捻奸民簿記之,誅其魁桀數十人,豪猾斂跡。歲餘,威化大行。民陷賊及遠徙者,相率還歸。以功晉知府。調江蘇,筦揚州正陽釐榷。光緒元年,署徐州,以祖母憂去官。

三年,宣城、建平民教閧,焚毀教堂。總督沈葆楨強起中行往治,中行謂:「民倡亂當治如律,然民所以亂,由教堂侵其地。今當令民償教堂財,而教堂還民地。」持數月,卒如中行議。內艱歸,服闋,檄治皖南墾務。皖南兵燹後,客民占墾不輸賦,至是清丈田畝,無問主客。客民噪,捕斬其魁,乃聽命。三歲事竣,增賦鉅萬。

九年,補徐州。值水災,興工賑,修堤墊二百餘里。又濬邳州艾山河,築宿遷六塘墊,水患除,民以不饑。治徐十二年,課農勸士,盜賊衰息。擢岳常澧道,數月,遷廣西按察使,復調湖南。二十年,卒。中行所至有聲,官江南最久,民尤愛戴之。附祀徐州曾國籓祠。

劉含芳,字薌林,安徽貴池人。同治初,李鴻章率師東征,從克蘇州,司運糧械。後隨征捻,積功至道員。鴻章督直隸,命含芳治軍械天津。得西洋利器,省覽機括,久之悉通其意。鴻章方拓北洋軍備,於西沽建武庫,廣收博儲,以肄將士,擴充機器、制造兩局,募工仿構,創設電氣水雷學堂,編立水雷營,皆以含芳董其役。

光緒七年,詔求人才,以鴻章薦,交軍機處存記。時海軍初立,造船塢旅順,含芳兼領沿海水陸營務處。十四年,署津海關道,授甘肅安肅道,留治海防。尋調山東登萊青道,監督東海關,十九年,始之任。含芳自隨鴻章至天津,凡十四載,屯旅順十一載,至是雖領一道,猶隸於北洋。

二十年,遼東兵事起,海陸軍屢挫,旅順、威海相繼陷。登萊青道駐煙臺,敵軍日逼。俄報軍艦沒於劉公島,寧海亦陷,敵前鋒距煙臺十餘里。時巡撫李秉衡亦駐師煙臺。西國諸領事言巡撫在,則敵攻之急,於租地不便,巡撫乃退萊州。領事復言含芳,含芳曰:「巡撫大臣也,可去。某守土吏,去何之?今死此矣!」因置鴆二盂案上,與其妻郝冠服坐待,意氣堅定,民恃無恐。有潰卒數千,持兵噪呼求食。含芳單騎馳諭,處以空營,重為編伍,不原留者厚給遣之,皆出私財。初,西人聞潰兵,甚戒嚴,俄而散遣,殊出不意,咸稱道之。和議成,奏派渡海勘收還地。始威海、旅順、大連灣皆荒島,含芳瘁心力營構十餘年,所成險塞,至是見盡毀矣,因憤慨流涕,以疾乞歸。卒,贈內閣學士。

陳黌舉,字序賓,安徽石埭人。少從其鄉陳艾游,以諸生為曾國籓所識拔。李鴻章督師,令主辦行營支應。或謂「大軍轉餉關天下,往者輒命大臣,今以諸生任耶」?卒用不疑。自粵亂作,海內困軍餉。黌舉曰:「餉糜則斂重,戰久則餉虧,兵不潰,民且寇矣。」乃釐訂條款,杜絕冒濫。軍行數載,餉節民和,平捻之功實基此。鴻章移直隸籌海防,凡砲臺、船塢、制造、電報及疏河、屯田諸役,需費尤鉅,皆倚之以辦。先後綜軍糈二十餘年,一介不茍。將吏服其廉潔,雖被裁抑,無怨言。直、晉大災,兼籌賑務,廢寢忘食,稽核勤摯,人不忍欺。以私款歸實濟,全活以億萬計,眾皆德之。旋以積勞病卒。初由訓導累功至知府,詔贈道員。與含芳同附祀鴻章祠,入祀淮軍昭忠祠,並祀鄉賢。

黌舉子惟彥,亦見重於鴻章,命繼司軍計。由大理寺丞累保知府,官貴州,歷開州、婺川,調守黎平。首革票差催糧,遏龍世渭逆謀,破鴨販彭三等血案,遠近驚為神明。鄰邑有訟,往往越境就訴。興學育才,並創立體仁堂養老恤孤,勸工習藝,政聲頗著。巡撫疏為良吏第一,以道員改江蘇,總釐捐,任督銷。去弊化私,以廉直稱。旋授湖南財政監理官,復委辦兩淮鹽政,創設淮南公所,歲增至二百萬。歸,與弟惟壬於縣境修巨橋跨舒溪,亙六十餘丈,便行旅。邑人私謚曰慈惠。[一]

游智開,字子代,湖南新化人。咸豐元年舉人,揀選知縣。同治初,李續宜巡撫安徽,調司釐榷,以廉平稱。四年,署和州知州,日坐堂皇決事。又時出巡四境,延見父老,問其疾苦。親為諸生考校文藝,剖析經旨,教以孝弟廉讓。期年,治化大行。州舊由胥吏墊完糧賦,最為民病,禁絕之。築瀕江堤防,自督工役,費節而堤堅,免水患。補無為州,署泗州,治盜尤嚴。曾國籓稱其治行為江南第一,移督直隸,調智開署深州。興義學,減浮征,民大悅。補灤州,民苦兵車,為別籌輸送,免擾累。俗健訟,奸民居間交構,痛懲之,其風漸息。

十一年,擢知永平府,一車一蓋,周歷下邑,得其情偽。遇有事,牧令未及報,輒已聞知。一日侵晨,馳至遷安獄,獄吏方私系囚求賂,即拘吏至縣庭笞之。令始驚,起謝。葺書院,築城垣,修郡志,皆事舉,無濫費。瀕海產鹽,貧民資為衣食。部牒禁私販,疏官引。智開上言民間少一私販,即地方多一馬賊。鹽本宜行官引,惟永平則仍舊為便,事得寢。有巨室以析產構訟,久不決。智開坐便室,呼兩造至,不加研鞫,自咎治郡無狀,變起骨肉,望族如此,況齊民乎?訟者流涕請罷。李鴻章疏陳智開清勤端嚴,足勵末俗。光緒六年,擢永定河道。河患夙稱難治,智開每當搶護險工,立河干親指揮,日周巡兩岸以為常,員弁無敢離工次者。左宗棠議將永定河南岸改北岸以紓水患。智開以上下游數百里,城市廬墓,遷徙不便,力爭而止。兩以三汛安瀾邀優獎。

十一年,擢四川按察使。攜一僕乘箯輿入蜀,密訪吏治得失,民情愛惡。督屬清釐積案,常躬自訊結,獄訟為清。兩權布政使。十二年,護理總督。重慶教案起,智開奏言是案當以根究起釁之由,先收險要及預定款目為關鍵。非贖回險要,無以服渝民之心;非嚴誅首犯,無以制洋人之口;非議賠銀兩,無以為結案之具。諗知教首羅元義激成眾怒,幾釀大變,飛檄拘之入省,民團始散。又以元義身雖入教,仍是中國子民,自應治以中國法律。請敕總理衙門據理與爭,勿許公使干預。時中外皆恐以肇釁端,智開持之益力,卒置元義於法。薄給賠償,而案遂結。

十四年,遷廣東布政使,署理巡撫。劾貪墨吏,不避權要,嚴賭禁,卻闈姓例餽三十萬金。僧寺匿匪,廢改義塾。十六年,以老乞休。二十一年,起廣西布政使。為政務持大體,事有不可行,力持不變。痛除官場積習,僚屬化之。靈川鬧糧,省令發兵剿辦。智開以事由激變,辦理不善,責歸縣令,民獲保全。又念粵西地瘠,向鮮蓋藏,捐廉儲糧石,通飭各屬積穀備荒。凡廉俸所入,悉以辦公益,無自私。閱三年,因病罷歸,卒於家。所至各省俱請祀名宦祠。

李用清,字澄齋,山西平定州人。同治四年進士,改庶吉士,出大學士倭仁門,散館授編修。安貧厲節,日研四子書、硃子小學,旁稽掌故,於物力豐瘠,尤所留意。大婚禮成,加侍讀銜。十二年,丁父憂,徒步扶櫬返葬。服闋,入都,仍課生徒自給。

光緒三年,記名御史。會山西奇荒,巡撫曾國荃、欽差大臣閻敬銘奏調用清襄賑務,騎一驢周歷全境,無間寒暑,一僕荷裝從。凡災情輕重、食糧轉輸要道,悉紀之冊。深窮病源,以為晉省罌粟花田彌望無際,必改花田而種五穀,然後生聚有期,元氣可復,上書國荃詳論之。國荃疑晉新荒,禁煙效緩,且全國未禁,徙斂怨,說竟不行。賑竣,卻保獎。還京,傳補御史,引見有日矣。法越事萌芽,張樹聲以廣西邊防奏調。樹聲督兩廣,復調廣東任海防釐榷,洗手奉職。七年,授惠州知府。境故多盜,喜博,喜私斗。用清推誠化之,俗乃稍革。

八年,遷貴州貴西道。明年,超擢布政使,署巡撫。實倉儲,興農利,裁冗員,劾缺額之提鎮,擒粵匪莫夢弼等置諸法。巡閱所至,召士子講說經傳,將吏環聽,相與動容。黔地土瘠,多種罌粟,暢行湘、鄂、贛、粵諸省,用清奏陳禁種之法,分區限年,時自出巡,刈剷煙苗。言者疑其操之過急。十一年秋,有旨來京候簡。召對,猶痛陳罌粟疚國殃民狀,冀可挽回萬一。旋命署陜西布政使,荒燹之後,休養生息,仍嚴煙禁。十四年,復命來京候簡,遂以疾歸,主講晉陽書院凡十年。用清嚴於自治,勇於奉公。籓黔時,庫儲六萬,年餘存十六萬,陜庫三十萬,再期六十餘萬矣。所至尤措意桑棉織組。嘗濬三源縣龍渠,溉田千餘畝。俸入不以自潤,於黔以購粟六千石,於陜購萬石,備不虞。鄭州河決,捐工需二萬兩。二十四年,卒。子貴陽扶柩歸,以毀殤。

同縣李希蓮,字亦青。咸豐十年進士,授戶部主事,再遷郎中。性節儉,官京曹三十年,車馬羸敝,不顧訕譏。英、法兵入都,曹司多走避,希蓮昕夕詣署無間。以忤肅順,乞假歸。同治元年,起原官。時軍興餉絀,希蓮條陳開源節流數端,恭親王奕韙之。雲南報銷案發,同僚有褫職遣戍者,希蓮獨無所染。光緒中,出為江西廣饒道,除濫稅,復徵額。擢山東鹽運使,調長蘆。累遷貴州按察使、陜西布政使。戊戌政變,希蓮頗憂大亂將起,與總督陶模議籌建陪都。及兩宮西幸,入始服其先見雲。

李金鏞,字秋亭,江蘇無錫人。少為賈,以試用同知投效淮軍。光緒二年,淮、徐災,與浙人胡光鏞集十餘萬金往賑,為義賑之始。後遂賑直隸、山東,皆躬其役。五年,晉秩知府。調直隸,修西澱堤。吳大澂督防吉林,金鏞任琿春招墾事。界外蘇城溝墾戶數千,苦俄人侵略,相率來歸,咸得奠居。海參威既通商,俄人援例要請東三省要地設領事,嚴拒之。又力爭八道河民被俄焚掠,抵俄官於法。將軍銘安以為才,疏留吉林任用。中俄界約,自瑚布河口循琿春河至圖門江口,以海中之嶺為界嶺,以西屬中國,距江口二十餘里立土字碑。界圖疏略,致嶺西之罕奇、毛琛崴等鹽場置線外。俄復於黑頂子地私設卡倫,距江口幾百里矣。大澂使金鏞會勘,據約爭還侵地,重立界碑。署吉林知府,整錢法,開溝洫,攤丁於地,以蘇民困。

九年,署長春通判。境為蒙古郭爾羅斯地,初招流民領墾納租,久之墾逾所領,謂之「夾荒」。民懼增稅,因出錢免丈量,刻石紀之。至是蒙旗復牒理籓院請丈,金鏞挾碑文謁將軍為民請命,曰:「誠知清丈則公與某各有所得,然如民何?」將軍聞之愕然,奏罷其事。創建書院,厚其廩餼,購書數千卷,資學者誦習。捕斬劇盜苗青山等,境內乂安。不時巡歷鄉僻,呼召父老,為講孝弟力田。金鏞性坦易,口操南音,所至民愛而憚之。以功晉道員。

俄侵占精奇里河四十八旗屯地,在黑龍江岸東。金鏞爭還補丁屯至老瓜林百七十餘里,劃河定界。漠河者,在璦琿西,三面界俄,地產金,俄人覬覦之。北洋大臣李鴻章議自開採,以金鏞任其事。陸路由墨爾根入,水運由松花江入,各行千餘里,僻遠無人。披斬荊棘,於萬山中設三廠,兩年得金三萬。事事與俄關涉,艱阻百端。又開廠於黑龍江南岸札伊河旁之觀音山,皆為北徼名礦。集商貲立公司,流冗遠歸,商販漸集,收實邊之利焉。十六年,病卒工所。贈內閣學士,予漠河建祠。

金福曾,字苕人,浙江秀水人。以諸生從軍,先侍祖父衍宗溫州教授,任籌團練助城守。旋隨官兵肅清金、處,協守獨松關,解杭州圍。李鴻章器其才,克蘇州,檄辦善後。捻事起,往贊徐州道張樹聲軍務。捻眾北竄,出防大名。丁憂歸,福曾積功已至知縣,服闋,赴江蘇,歷署婁、南匯、吳江諸邑。所至興學校,課農桑,理冤獄,禁溺女,勸墾沙田,開濬河道,多善政,民有去思。光緒初,河南、山西大祲。吳人謝家福等倡義賑,集四十餘萬金,推福曾董其事。四年秋,至河南分賑洛陽等十二州縣。新安、澠池災尤重,福曾創立善堂,恤嫠掩骼,收贖子女,購車馬若干輛,代疲民應役。開渠澗,制龍骨車,興水利。又濬洛陽、宜陽廢渠,貫通伊、洛,灌田二萬頃。五年,賑山西虞鄉等十縣。事竣,移賑直隸。時直隸水患方急,持以工代賑之策。

七年,疏大清河,濬中亭河,培千里堤。福曾先援例以道員候選,至是鴻章督直隸,奏留總辦籌賑局。福曾以澱地淤塞為清河受病之源,清丈東澱無糧地,釐定葦租,規復垡船。八年,濬東澱河道,修築天津三河頭堤。九年,築子牙河堤,展寬正河,又別開支河王家口以洩盛漲。十年,畿東大水,福曾疏青龍灣減河入七里海,疏筐兒港減河入塌河澱,並出北塘海口。又開瀝水各河以水曳武清、寶坻窪區積水。十一年,濬饒陽滹沱河。十三年,濬四女寺南運減河。兩署永定河道,塞決口,於下口別闢新道。又就大清河合流處別濬新河,永定河水始直達天津海河。山東河決數為災,鴻章輒檄福曾往助工賑,親至蘇、浙募貲。會浙西大水,巡撫崧駿復疏留福曾治工賑。於是杭、嘉、湖三府各河次第疏瀹。會廷議濬餘杭南湖,以福曾董其役。明年工竣,直、魯又告災,福曾已臥病,猶力疾籌賑濟。十八年,卒。鴻章等疏請優恤,贈內閣學士。福曾廉公好義,歷辦工賑十餘年,無日不勞身焦思,治行卓然。及其歿,士民同聲惜之。

熊其英,字純叔,江蘇青浦人。以貢生就訓導。家福集金賑河南,其英請行,始事濟源。濟源山僻小縣也,災尤劇,多方補苴,次第以及他邑。其英親履窮僻,稽察戶口,不避風雪,食惟麥粥、面餅、菜羹,與饑民同苦。初頭病瘍,足病濕,醫少愈,仍從事不肯休,遂卒於衛輝。巡撫上聞,詔許被賑各州縣立祠祀之。

家福,字綏之,吳縣人。世以行善為事。聞豫、晉災,呼籥尤切。義聲傾動,聞者風起。自上海、蘇、揚及杭、湖,原助賑者眾。日賚錢至家福門,或千金,或數千金,不一年得銀四十三萬有奇。凡賑二十七州,繼其英往者七十四人。家福才識為時重,於創辦電報及推廣招商輸船局事多所策畫。李鴻章尤賞之,嘗疏薦稱有「物與民胞」之量,體國經野之才。金福曾亦聞而嘆許焉。家福歷保至直隸州知州,卒不仕。時又有吳江紳富沈中堅,鬻田三十頃,親往山西賑災。亦義行之尤著者。

童兆蓉,字少芙,湖南寧鄉人。同治六年舉人,從軍陜西,積功晉知府。光緒三年,署榆林。歲祲,便宜發倉,復運粟於包頭、寧夏,單騎臨賑。既而大疫,延榆綏道及榆林令皆遽歿,代者不至。兆蓉一身兼攝三官,比戶存問,為具醫藥,全活甚眾。六年,署延榆綏道。屬郡荒僻,土僿民貧。為拓學舍,購書勸課,教民樹藝畜牧。治榆溪河,開渠溉田,民利之。八年,授興安知府。汰胥役,禁私錢。總兵餘虎恩販錢為利,獲而毀之。稅胥索賈人金,榜治幾死。民間婚娶茍簡,為定禮制,禁淫祀,葺昭忠、節孝祠,以正民志。安康令徵糧苛急,民聚而譁,兆蓉往撫諭。總兵及釐局挾前嫌,誣為激變,遂解任。尋得白,署漢中,逾年還本任。川匪擾境,擒斬其渠,賊潰走。調西安,攝督糧道,定徵糧改折,上下稱便。

二十六年,擢浙江溫處道,先署杭嘉湖,明年乃之任。值拳匪亂後,瑞安民楊茂奶與教堂積釁。浙東法國主教趙保祿尤橫,挾兵船至溫州,必欲殺楊。兆蓉力爭曰:「彼法不當死,我不能殺人以媚人。」卒拒之,以此名聞。颶風為災,賑糶並舉,民不乏食。三十一年,卒於官。

論曰:光緒初,各省重吏治,監司大吏下逮守令,皆一時之選。朝儀以下諸人,或御亂保民,或治盜清訟,或興學勸業,或救災恤患,莫不以民生為重。承兵燹後,辛苦凋殘之人,得生存以至今日者,實賴於此。「民亦勞止,汔可小休,惠此中國,以為民逑」。誠知本哉![一]按:劉含芳傳所附陳黌舉傳,關內本與關外一次本無。


\end{pinyinscope}