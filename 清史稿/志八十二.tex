\article{志八十二}

\begin{pinyinscope}
○選舉二

△學校二

學校新制之沿革,略分二期。同治初迄光緒辛醜以前,為無系統教育時期;辛醜以後迄宣統末,為有系統教育時期。自五口通商,英法聯軍入京後,朝廷鑒於外交挫衄,非興學不足以圖強。先是交涉重任,率假手無識牟利之通事,往往以小嫌釀大釁,至是始悟通事之不可恃。又震於列強之船堅砲利,急須養成繙譯與制造船械及海陸軍之人才。故其時首先設置之學校,曰京師同文館,曰上海廣方言館,曰福建船政學堂及南北洋水師、武備等學堂。

京師同文館之設,從總理各國事務衙門之請,始於同治元年。初止教授各國語言文字。六年,議於同文館內添設算學館。時京僚瞢於時務,謗讟繁興,原疏排斥眾議,言之剴切。謂:「西人制器之法,無不由度數而生。中國欲講求制造輪船、機器諸法,茍不藉西士為先導,師心自用,無裨實際。疆臣如左宗棠、李鴻章等,皆深明其理,堅持其說,詳於奏牘。且西人之術,聖祖深韙之矣,當時列在臺官,垂為時憲,本朝掌故,不宜數典而忘。若以師法西人為恥,其說尤謬。中國狃於因循,不思振作,恥孰甚焉。今不以不如人為恥,獨以學其人為恥,將安於不如而終不學,遂可雪恥乎?學期適用,事貴因時,物議雖多,權衡宜定。原議招取滿、漢舉人,恩、拔、副、歲、優貢生,並由此出身之正途人員。又擬推廣,凡翰林院庶吉士、編修、檢討,與五品以下進士出身之京、外各官,年在三十歲以內者,均可送考。三年考列高等者,按升階優保班次,以示鼓勵。」詔從其議。

上海廣方言館,創設於同治二年。江蘇巡撫李鴻章言:「京師同文館之設,實為良法。惟洋人總匯地,以上海、廣東兩口為最。擬仿照同文館例,於上海添設外國語言文字學館,選近郡年十四歲以下資稟穎悟、根器端靜之文童,聘西人教習,並聘內地品學兼優之舉、貢生員,課以經、史、文藝。學成送本省督、撫考驗,作為該縣附學生。其候補、佐雜等官,年少聰慧者,許入館一體學習,學成酌給升途。三五年後,有此一種讀書明理之人,精通番語,凡通商、督、撫衙署及海關監督,應設繙譯官承辦洋務者,即於館中遴選派充。庶關稅、軍需可期核實;無賴通事,亦稍斂跡。且能盡閱西人未譯專書,探賾索隱,一切輪船、火器等巧技,由漸通曉,於自強之道,不無裨助。」上諭廣州將軍查照辦理。

福建船廠,同治五年,左宗棠督閩時奏設,並設隨廠學堂。分前、後二堂。前堂習法文,練習造船之術;後堂習英文,練習駕駛之術。課程除造船、駕駛應習常課外,兼習策論,令讀聖諭廣訓、孝經以明義理。首總船政者為沈葆楨,規畫閎遠,尤重視學堂。十二年,奏陳船工善後事宜:「請選派前、後堂生分赴英、法,學習制造駕駛之方,及推陳出新、練兵制勝之理。學生有天資傑出,能習礦學、化學及交涉、公法等事,均可隨宜肄業。」尋葆楨任南洋大臣。光緒二年,奏派華、洋監督,訂定章程。船政學堂成就之人材,實為中國海軍人材之嚆矢。學堂設於馬尾,故清季海軍將領,亦以閩人為最多。

天津水師學堂,光緒八年,北洋大臣李鴻章奏設。次年招取學生,入堂肄業。分駕駛、管輪兩科。教授用英文,兼習操法,及讀經、國文等科。優者遣派出洋留學,以資深造。厥後海軍諸將帥由此畢業者甚夥。

鴻章又於光緒十一年奏設天津武備學堂,規制略仿西國陸軍學堂。挑選營中精健聰穎、略通文義之弁目,入堂肄業。文員原習武事者,一並錄取。其課程一面研究西洋行軍新法,如後膛各種槍砲,土木營壘及布陣分合攻守各術。一面赴營實習,演試槍砲陣勢及造築臺壘。惟學生系挑選弁目,雖聘用德國教員,不能直接聽講,仍用繙譯,展轉教授,與水師學堂注重外國文者不同。初制,學習一年後,考試及格學生,發回各營,由統領量材授事。其後逐漸延長年限,選募良家年幼子弟肄業。迨庚子之變,學堂適當戰區,全校淪為灰燼矣。

此外廣東水陸師學堂,則粵督張之洞於光緒十三年奏設。之洞調任鄂督,二十一年又奏設湖北武備學堂,其辦法課程,水師分管輪、駕駛兩項,陸師分馬、步,槍、砲,營造等項,大略參照北洋成法。洎海軍成立,新軍改建,此類學堂,南洋及各省增設日盛,不具述。

至湖北自強學堂,亦之洞創設。初分方言、格致、算學、商務四門。惟方言一齋,住堂肄業,餘三齋按月考課。其後算學改歸兩湖書院教授,格致、商務停課,本堂專課方言,以為西學梯階。方言分英、法、德、俄四門,亦類似同文館之學堂也。

光緒丙申、丁酉間,各省學堂未能普設,中外臣工多以變通整頓書院為請。詔飭裁改,禮部議準章程,並課天算、格致等學。陜西等省創設格致實學書院,以補學堂之不逮焉。

大抵此期設學之宗旨,專注重實用。蓋其動機緣於對外,故外國語及海陸軍得此期教育之主要,無學制系統之足言。惟南洋公學雖亦承襲此期教育之宗旨,而學制分為三等,已寓普通學校及豫備教育之意旨。

先是光緒二十一年,津海關道盛宣懷於天津創設頭、二等學堂。頭等學堂課程四年,等一年習竣,欲專習一門者,得察學生資質酌定。專門凡五:一工程學,二電學,三礦務學,四機器學,五律例學。二等學堂課程四年,按班次遞升,習滿升入頭等。意謂二等擬外國小學,頭等擬外國大學。因初設,採通融求速辦法。教員既苦乏才,學生亦難精擇,無甚成效。

二十三年,宣懷又於上海創設南洋公學,如津學制而損益之,經費取給招商、電報兩局捐助。奏明辦理,因名公學。分四院:曰師範院,曰外院,曰中院,曰上院。外院即附屬小學,為師範生練習之所。中、上院即二等、頭等學堂,寓中學堂、高等學堂之意。課程大體分中文、英文兩部,而注重法政、經濟。上院畢業生,擇尤異者咨送出洋,就學於各國大學。意謂內國大學猝難設置,以公學為豫備學校,而以外國大學為最高學府。論者謂中國教育有系統之組織,此其見端焉。後改歸郵傳部管轄,定名高等實業學堂。其課程性質,非復設立之初旨。此第一期無系統教育之大略也。

自甲午一役,喪師辱國,列強群起,攘奪權利,國勢益岌岌。朝野志士,恍然於鄉者變法之不得其本。侍郎李端棻、主事康有為等,均條議推廣學堂。光緒二十四年,德宗諭曰:「邇者詔書數下,開特科,改武科制度,立大、小學堂。惟風氣尚未大開,論說莫衷一是。國是不定,則號令不行。特明白宣示中外,自王公至士庶,各宜努力發憤,以聖賢義理之學植其根本,博採西學切於時務者,實力講求,以救空疏迂謬之弊。京師大學為各省倡,應首先舉辦。凡翰林編、檢,部、院司員,各門侍衛,候補、候選道,府、州、縣以下各官,大員子弟,八旗世職,各省武職後裔,均準入學肄業,以期人材輩出,共濟時艱。」下軍機大臣、總理各國事務王、大臣,妥議奏聞。尋議覆籌辦京師大學堂。擬定章程,要端凡四:一寬籌經費,二宏建學舍,三慎選管學大臣,四簡派總教習。詔如所擬。命孫家鼐管理大學堂事務,經費由戶部籌撥。

五月,又諭各直省督、撫,將各省府、、州、縣大、小書院,一律改為兼習中、西學之學校,其階級,以省會之大書院為高等學,郡城之書院為中學,州、縣之書院為小學。頒給京師大學章程,令仿照辦理。各書院經費,侭數提作學堂經費。紳民如能捐建學堂,或廣為勸募,準奏請給獎。有獨立措捐鉅款者,予以破格之賞。民間祠廟不在祀典者,一律改為學堂,以節糜費而隆教育。是時管學大臣之權限,不專管理京師大學堂,並節制各省所設之學堂。實以大學校長兼全國教育部長之職權。

又以同文館及北洋學堂多以西人為總教習,於中學不免偏枯。且外國文不止一國,學科各有專門,非一西人所能勝任。必擇學貫中、西,能見其大之中國學者,為總教習,破格錄用,有選派分教習之權。蓋以管學大臣必大學士或尚書充任,而總教習則不拘資格,可延攬新進之人才也。學生分兩班,已治普通學卒業者為頭班,現治普通學者為二班,猶是南洋公學之舊法。課程分普通、專門兩類。普通學,學生必須通習;專門學,人各占一門或二門。普通學科目為經學,理學,掌故學,諸子學,初級算學,初級格致學,初級政治學,初級地理學,文學,體操學,語言文字學。專門學科目為高等算學,高等格致學,高等政治學、法律屬之,高等地理學、測繪屬之,農學,礦學,工程學,商學,兵學,衛生學、醫學屬之。考驗學生,用積分法。學生月給膏火銀兩有差。上海設編譯局,各學科除外國文外,均讀編譯課本。籌辦大學章程之概要如此。

未幾,八月政變,由舊黨把持朝局,卒釀成庚子之禍。逮二十七年,學校漸有復興之議。其首倡者,則山東巡撫袁世凱也。初,世凱奏陳東省開辦大學堂章程,有旨飭下各省仿辦,令政務處會同禮部妥議選舉鼓勵章程。尋議言:「東西各國學堂,皆系小學、中學、大學以次遞升,畢業後始予出身,擬請按照辦理。小學畢業生考試合格,選入中學堂。畢業考試合格,再選入大學堂。畢業考試合格,發給憑照。督、撫、學政,按其功課,嚴密扃試。優者分別等第,咨送京師大學堂覆試,作為舉人、貢生。其貢生留下屆應考,原應鄉試者聽。舉人積有成數,由京師大學堂嚴加考試,優者分別等第,咨送禮部。簡派大臣考試,候旨欽定,作為進士,一體殿試,酌加擢用,優予官階。查世凱辦法,以通省學堂一時未能遍舉,先於省城建立學堂,分齋督課,其備齋、正齋,即隱寓小學、中學之規制。既經諭令各省仿辦,應酌照將來選舉章程,用資鼓勵。」報可。所議混合科舉、學制為一事,謂之學堂選舉鼓勵章程,各省多未及實行而罷。

辛丑,兩宮回鑾。以創痛鉅深,力求改革。十二月,諭曰:「興學育才,實為當今急務。京師首善之區,尤宜加意作育,以樹風聲。前建大學,應切實舉辦。派張百熙為管學大臣,責成經理,務期端正趨鄉,造就通才。其裁定章程,妥議具奏。」旋諭將同文館並入大學堂,毋庸隸外務部。二十八年正月,百熙奏籌辦大學堂情形豫定辦法一條,言:「各國學制,幼童於蒙學卒業後入小學,三年卒業升中學,又三年升高等學,又三年升大學。以中國準之,小學即縣學堂,中學即府學堂,高等學即省學堂。目前無應入大學肄業之學生,通融辦法,惟有暫時不設專門,先設立一高等學為大學豫備科。分政、藝二科,以經史、政治、法律、通商、理財等事隸政科,以聲、光、電、化、農、工、醫、算等事隸藝科。查京外學堂,辦有成效者,以湖北自強學堂、上海南洋公學為最。此外如京師同文館,上海廣方言館,廣東時敏、浙江求是等學堂,開辦皆在數年以上,不乏合格之才。更由各省督、撫、學政考取府、州、縣高材生,咨送來京,覆試如格,入堂肄業。三年卒業,及格者升大學正科。不及格者,分別留學、撤退。大學豫科與各省省學堂卒業生程度相同,由管學大臣考驗合格,請旨賞給舉人。正科卒業,考驗合格,請旨賞給進士。惟國家需材孔亟,欲收急效而少棄才,則有速成教員一法。於預備科外設速成科,分二門:曰仕學館,曰師範館。凡京員五品以下、八品以上,外官道員以下、教職以上,皆許考入仕學館。舉、貢、生、監,皆許考入師範館。仕學三年卒業,擇尤保獎。師範三年卒業,擇優異者帶領引見。生準作貢生,貢生準作舉人,舉人準作進士,分別給予準作小學、中學教員文憑。蓋豫科生必齲年歲最富、學術稍精者,再加練習,儲為真正合格之才。速成生則取更事較多、立志猛進者,取其聽從速化之效。至增建校舍,附設譯局,廣購書籍、儀器,尤以寬籌經費為根原。經費分兩項:一,華俄道勝銀行存款之息金,全數撥歸大學堂;一,請飭各省籌助經費,每年大省二萬金,中省一萬金,小省五千金,常年撥解京師。」從之。

七月,百熙遵擬學堂章程,疏言:「古今中外,學術不同,其所以致用則一。歐、美、日本諸邦現行制度,頗與中國古昔盛時良法相同。禮記載家有塾,黨有庠,州有序,國有學。比之各國,則國學即大學,家塾、黨庠、州序即蒙學、小學、中學。等級蓋甚分明。周以前選舉、學校合而為一,漢以後專重選舉,及隋設進士科以來,士皆殫精神於詩、賦、策、論,所謂學校,名存而巳。今日而議振興教育,必以真能復學校之舊為第一要圖。雖中外政教風氣原本不同,然其條目秩序之至賾而不可亂,不必盡泥其跡,不能不兼取其長。謹上溯古制,參考列邦,擬定京師大學暨各省高等學、中學、小學、蒙學章程,候欽定頒行各省,核實興辦。凡名是實非之學堂及庸濫充數之教習,一律從嚴整頓。」詔下各省督撫,按照規條實力奉行。是為欽定學堂章程。教育之有系統自此始。

京師大學堂分大學院、大學專門分科、大學豫備科。附設者,仕學、師範兩館。大學院主研究,不講授,不立課程。專門分科凡七:曰政治科,曰文學科,曰格致科,曰農業科,曰工藝科,曰商務科,曰醫術科。政治科分目二:政治,法律。文學科分目七:經學,史學,理學,諸子,掌故,詞章,外國語言文字。格致科分目六:天文,地質,高等算學,化學,物理,動植物。農業科分目四:農藝,農業化學,林學,獸醫。工藝科分目八:土木,機器,造船,造兵器,電氣,建築,應用化學,採礦冶金。商務科分目六:簿記,產業制造,商業語言,商法,商業史,商業地理。醫術科分目二:醫學,藥學。豫備科分政、藝兩科。政科課目:倫理,經學,諸子,詞章,算學,中外史,中外輿地,外國文,物理,名學,法學,理財,體操。藝科課目:倫理,中外史,外國文,算學,物理,化學,動植物,地質及礦產,圖畫,體操。為入專理某科便利計,得增減若干科目。各三年卒業。仕學館課目:算學,博物,物理,外國文,輿地,史學,掌故,理財,交涉,法律,政治。師範館課目:倫理,經學,教育,習字,作文,算學,中外史,中外輿地,博物,物理,化學,外國文,圖畫,體操。

各省高等學堂為中學卒業之升途,又為入分科大學之豫備。分政、藝兩科。課程與大學豫科同。三年卒業。高等學外,得附設農、工、商、醫高等實業學堂,亦中學卒業生升入。教授用專科教員制,各任一門。中學堂,為高等小學卒業之升途,即為入高等學之豫備。課目:修身,讀經,算學,詞章,中外史,中外輿地,外國文,圖畫,博物,物理,化學,體操。四年卒業。中學外,得設中等農、工、商實業學堂,高小卒業生不原治普通學者入之。又附設師範學堂,課目視中學,惟酌減外國文,加教育學、教授法。得合兩班或三班,以兩三教員各任數科目,分教之。小學堂分高等、尋常二級。兒童自六歲起,受蒙學四年。十歲入尋常小學,修業三年。此七年定為義務教育。十三歲入高等小學,三年卒業。得附設簡易農、工、商實業學堂,尋常小學卒業者入之。尋常小學課目:修身、讀經、作文、習字、史學、輿地、算術、體操。高等小學課目,增讀古文辭、理科、圖畫,餘同尋常小學。教授採用級任制。正教習外,得置副教習。蒙學堂屬義務教育,府、、州、縣、城、鎮、鄉、集均應設立。凡義塾或家塾,應照蒙學課程,核實改辦。課目同尋常小學,惟作文易以字課。蒙學宗旨,在於改良私塾,故章程規定,頗注重教授法之改善,於兒童身心之體察,三致意焉。至學生出身獎勵,小學卒業,獎給附生;中學卒業,獎給貢生;高等學卒業,獎給舉人;大學分科卒業,獎給進士。各省師範卒業,照大學師範館例給獎。其大較也。欽定章程雖未臻完備,然已有系統之組織。頒布未及二年,旋又廢止。

先是百熙招致海內名流,任大學堂各職。吳汝綸為總教習,赴日本參觀學校。適留日學生迭起風潮,諑謠繁興,黨爭日甚。二十九年正月,命榮慶會同百熙管理大學堂事宜。二人學術思想,既各不同,用人行政,意見尤多歧異。時鄂督張之洞入覲。之洞負海內重望,於川、晉、粵、鄂,曾創設書院及學堂。著勸學篇,傳誦一時;尤抱整飭學務之素志。閏五月,榮慶約同百熙奏請添派之洞會商學務,詔飭之洞會同管學大臣釐定一切學堂章程,期推行無弊。

十一月,百熙、榮慶、之洞會奏重訂學堂章程,言:「各省初辦學堂,難得深通教育理法之人。學生率取諸原業科舉之士,未經小學陶鎔而來,言論行為,不免軼於範圍之外。此次奉諭會商釐定,詳細推求,倍加審慎。博考外國各項學堂課程門目,參酌變通,擇其宜者用之,其於中國不相宜者缺之,科目名稱不可解者改之,過涉繁重者減之。無論何等學堂,均以忠孝為本,以中國經史之學為基,俾學生心術壹歸於純正。而後以西學瀹其智識,練其藝能,務期他日成材,各適實用。擬成初等小學、高等小學、中學、高等學各章程,大學附通儒院章程。原章有蒙學名目,所列實即外國初等小學之事。外國蒙養院,一名稚園,參酌其意,訂為蒙養院章程及家庭教育法。此原章所有,而增補其缺略者也。辦理學堂,首重師範。原訂師範館章程,系僅就京城情形試辦,尚屬簡略。另擬初級、優級師範學堂章程,並任用教員章程,京城師範館改照優級師範辦理。此外仕學館屬暫設,不在各學堂統系之內,原章應暫仍舊。譯學館即方言學堂;進士館系奉特旨,令新進士概入學堂肄業,課程與各學堂不同,並酌定章程課目。又國民生計,莫要於農、工、商實業,興辦實業學堂,有百益而無一弊,另擬初等、中等、高等農、工、商實業學堂章程,附實業補習普通學堂、藝徒學堂、實業教員講習所各章程。此原章未及,而別加編訂者也。又中國禮教政俗與各國不同,少年初學,胸無定識,哤雜浮囂,在所不免。規範不容不肅,稽察不容不嚴。特訂立規條,申明禁令,為學堂管理通則。並將設學宗旨、立法要義,總括發明,為學務綱要。果能按照現定章程認真舉辦,民智可開,國力可富,人才可成,不致別生流弊。至學生畢業考試,升級、入學考試及獎勵錄用之法,亦經詳定專章,伏候裁定。」

又奏:「奉旨興辦學堂,兩年有餘。至今各省未能多設者,經費難籌也。經費所以不能捐集者,科舉未停,天下士林謂朝廷之意並未專重學堂也。科舉不變通裁減,人情不免觀望,紳富孰肯籌捐?經費斷不能籌,學堂斷不能多。入學堂者,恃有科舉一途為退步,不肯專心鄉學,且不肯恪守學規。況科舉文字多剽竊,學堂功課務實修;科舉止憑一日之短長,學堂必盡累年之研究;科舉但取詞章,學堂並重行檢。彼此相衡,難易迥別。人情莫不避難就易,當此時勢阽危,除興學外,更無養才濟時之術。或慮停罷科舉,士人競談西學,而中學無人肯講。現擬章程,於中學尤為注重。凡中國向有之經學、史學、文學、理學,無不包舉靡遺。科舉所講習者,學堂無不優為;學堂所兼通者,科舉皆所未備。是取材於科舉,不如取材於學堂,彰彰明矣。或又慮學堂雖重積分法,分數定自教員,保無以愛憎而意為增損。不知功課優絀,當堂考驗。教員即欲違眾徇私,而公論可憑,萬難掩飾。臣等尚恐偶有此弊,故於中學考試,歸學政主持,督同道、府辦理。高等學畢業,請簡放主考,會同督、撫、學政考試。大學畢業,請簡放總裁,會同學務大臣考試。不專憑本學堂所定分數。凡科舉掄才之法,已括諸學堂獎勵之中,實將科舉、學堂合並為一。就事理論,必須科舉立時停罷,學堂辦法方有起色,經費方可設籌。惟此時各省學堂,未能遍設,已設學堂,辦理未盡合法,不欲遽議停罷科舉。然使一無舉動,天下未見朝廷有遞減以至停罷之明文,實不足風示海內士民,收振興學堂之效。請查照臣之洞會同袁世凱原奏分科遞減之法,明降諭旨,從下屆丙午科起,每科遞減中額三分之一。一面照現定各學堂章程,從師範入手,責成各省實力舉行,至第三屆壬子科應減盡時,尚有十年。計京、外開辦學堂,已逾十年以外,人才應已輩出。天下士心專注學堂,籌措經費必立見踴躍。人人爭自濯磨,相率入學堂,求實在有用之學,氣象一新,人才自奮。轉弱為強,實基於此。」詔悉如所請。是為頒布奏定章程之期,時科舉未全廢止也。迨三十一年,世凱、之洞會奏:「科舉一日不停,士人有僥幸得第之心,以分其砥礪實修之志。民間相率觀望,私立學堂絕少。如再遲十年甫停科舉,學堂有遷延之勢,人才非急切可求。必須二十餘年後,始得多士之用。擬請宸衷獨斷,立罷科舉。飭下各省督、撫、學政,學堂未辦者,從速提倡;已辦者,極力擴充。學生之良莠,辦學人員之功過,認真考察,不得稍辭其責。」遂詔自丙午科始,停止各省鄉、會試及歲、科試。尋諭各省學政專司考校學堂事務。於是沿襲千餘年之科舉制度,根本劃除。嗣後學校日漸推廣,學術思想因之變遷,此其大關鍵也。

是時學務之組織,尚有一重要之變更,則專設總理學務大臣也。二十九年,之洞言:「管學大臣既管京城大學堂,又管外省各學堂事務。當此經營創始,條緒萬端,專任猶虞不給,兼綜更恐難周。請於京師專設總理學務大臣,統轄全國學務。另設總監督一員,專管京師大學堂事務,受總理學務大臣節制考核,俾有專責。」詔允改管學大臣為學務大臣,並加派孫家鼐為學務大臣,命大理寺少卿張亨嘉充大學堂總監督。奏定章程,規定學校系統,足補欽定章程所未備。

其分科及課目,較舊章亦多有變更。大學設通儒院及大學本科。通儒院不講授,無規定課目。大學本科分科八。曰經學科,分十一門:周易、尚書、毛詩、春秋左傳、春秋三傳、周禮、儀禮、禮記、論語、孟子,附理學。曰政法科,分二門:政治、法律。曰文學科,分九門:中國史、萬國史、中外地理、中國文學、英國文學、法國文學、俄國文學、德國文學、日本國文學。曰醫科,分二門:醫學、藥學。曰格致科,分六門:算學、星學、物理、化學、動植物、地質。曰農科,分四門:農學、農藝化學、林學、獸醫。曰工科,分九門:土木、機器、造船、造兵器、電氣、建築、應用化學、火藥、採礦冶金。曰商科,分三門:銀行及保險、貿易及販運、關稅。各專一門。經學原兼習一兩經者聽。各學科分主課、補助課。三年畢業。惟政治、醫學四年畢業。

高等學與大學豫備科性質相同。學科分三類:第一類為豫備入經學、政法、文學、商科等大學者治之,第二類為豫備入格致、農、工等科大學者治之,第三類為豫備入醫科大學者治之。學科除人倫道德、經學大義、中國文學、外國語、體操各類共同外,第一類課歷史、地理、辨學、法學、理財,第二類課算學、物理、化學、地質、礦物、圖畫,第三類課蠟丁語、算學、物理、化學、動物、植物。其有志入某科某門者,得缺科目或加課他科目,分通習、主課。三年畢業。中學科目:修身、讀經、講經、中國文學、外國語、歷史、地理、算學、博物、物理及化學、法制及理財、圖畫、體操。五年畢業。高等小學科目:修身、讀經、講經、中國文學、算術、中國歷史、地理、格致、圖畫、體操。視地方情形,可加授手工、農、商業等科目。四年畢業。初等小學科目:修身、讀經、講經、中國文學、算術、歷史、地理、格致、體操,為完全科。視地方情形,可加授圖畫、手工之一二科目。其鄉民貧瘠、師儒缺少地方,得量從簡略,修身、讀經合為一科,中國文學科,歷史、地理、格致合為一科,算術、體操,為簡易科。五年畢業。

中、小學科目,不外普通教育之學科。其特殊者,則讀經、講經一科也。學務綱要載中、小學宜注意讀經以存聖教一節,其言曰:「外國學堂有宗教一門,中國之經書即是中國之宗教。學堂不讀經,則是堯、舜、禹、湯、文、武、周公、孔子之道,所謂三綱五常,盡行廢絕,中國必不能立國。無論學生將來所執何業,即由小學改業者,必須曾誦經書之要言,略聞聖教之要義,以定其心性,正其本源。惟學堂科學較繁,晷刻有限,概令全讀十三經,精力日力斷斷不給。茲擇切要各經,分配中、小學堂。若卷帙繁重之禮記、周禮,止選讀通儒節本,儀禮止選讀最要一篇。自初等小學第一年日讀約四十字起,至中學日讀約二百字為止,大率小學每日以一點鐘讀經,一點鐘挑背淺解。中學每星期以六點鐘讀經,三點鐘挑背講解。溫經每日半點鐘,歸自習時督課。學生並不過勞,亦無礙講習西學之日力。計中學畢業,已讀過孝經、四書、易、書、詩、左傳及禮記、周禮、儀禮節本十經,並通大義。較之向來書塾、書院所讀所解,已為加多。不惟聖經不至廢墜,且經學從此更可昌明。」其立論甚正,可考見當時之風氣焉。

蒙養院意在合蒙養、家教為一,輔助家庭教育,兼包括女學。

直系學堂外,並詳訂師範及實業學堂專章。其大異於舊章者,為優級師範學堂。學科分三節:一曰公共科,以補中學之不足,為本科之豫備。科目:人倫道德、群經源流、中國文學、東語、英語、辨學、算學、體操。一年畢業。二曰分類科,凡四類:第一類以中國文學、外國語為主。第二類以地理、歷史為主。第三類以算學、物理、化學為主。第四類以動植物、礦物、生理為主。科目除人倫道德、經學大義、中國文學、教育心理、體操各類共同外,第一類課周秦諸子、英語、德語或法語、辨學、生物、生理。第二類課地理、歷史、法制、理財、英語、生物。第三類課算學、物理、化學、英語、圖畫、手工。第四類課植物、動物、生理、礦物、地學、農學、英語、圖畫。分通習、主課,均三年畢業。三曰加習科,於分類科畢業,擇教育重要數門,加習一年,以資深造。科目:人倫道德、教育學、教育制度、教育政令機關、美學、實驗心理、學校衛生、專科教育、兒童研究、教育演習,並增入教授實事練習。優級師範附屬中學堂、小學堂。初級師範學科程度,與中學略同。完全科學科,於中學科目外,增教育學、習字。視地方情形,可加外國語,手工,農、工業之一科目或數科目。五年畢業。初級師範附屬小學堂。

實業學堂之種類,曰實業教員講習所,曰高等農、工、商實業學堂,曰中等農、工、商實業學堂,曰初等農、工、商實業學堂,及高等、中等、初等商船學堂,曰實業補習普通學堂,曰藝徒學堂。實業教員講習所,以備教成各項實業學堂之教習。分農、商、工三種,農業、商業教員講習所,除人倫道德、英語、教育、教授法、體操為共同學科外,農業課算學及測量氣象、農業汎論、農業化學、農具、土壤、肥料、耕種、畜產、園藝、昆蟲、獸醫、水產、森林、農產制造、農業理財實習;商業課應用化學、應用物理、商業作文、商業算術、商業地理、商業歷史、簿記、商品、商業理財、商業實踐。均二年畢業。工業教員講習所,置完全科及簡易科。完全科凡六:曰金工科、木工科、染織科、窯業科、應用化學科、工業圖樣科。除人倫道德、算學、物理、化學、圖畫、工業理財、工業衛生、機器制圖實習、英語、教育、教授法、體操為共同學科外,金工科課無機化學、應用力學、工場用具及制造法、電氣工業大意、發動機。木工科課無機化學、應用力學、工場用具及制造法、構造用材料、家具及建築流派、房屋構造、衛生、建築制圖及意匠。染織科課一切器用化學、應用機器、定性分析、工業分析、染色配色、機織及意匠。窯業科課一切應用化學、應用機器、定性分析、工業分析、窯業品制造。應用化學科課一切應用化學、機器、電鑄及電礦。工業圖樣科課圖樣、材料。均三年畢業。簡易科分金工、木工、染色、機織、陶器、漆工六科。課目較略。一年畢業。高等實業學堂程度視高等學堂,分豫科、本科。豫科授以各科普通基本功課。一年畢業。高等農業本科凡三:曰農學科,曰林學科,曰獸醫學科。高等工業分科十三:曰應用化學科,曰染色科,曰機織科,曰建築科,曰窯業科,曰機器科,曰電器科,曰電氣化學科,曰土木科,曰礦業科,曰造船科,曰漆工科,曰圖稿繪畫科,各授以本科原理、原則、應用方法及補助科目,多者至三十餘門,得斟酌地方情形,擇合宜數科設之。均三年畢業。中等實業學堂程度視中學堂,亦分豫科、本科,課目較高等為略。初等實業學堂程度視高等小學堂,分普通、實習兩種科目。均三年畢業。商船學堂亦分三等,以授航海機關之學術及駕運商船之知識技術。五年或三年畢業。實業補習普通學堂,以簡易教法授實業必須之知識技能,並補習小學科目。藝徒學堂,授平等程度之工築技術,俾成良善工匠,均可於中、小學堂便宜附設。

其不在學堂系統內者,曰譯學館,曰進士館。先是同文館並入大學堂,設英、法、俄、德、日本五國語文專科,後由大學分出,名譯學館。仍設英、法、俄、德、日本文各一科,無論習何國文,皆須習普通及專門學。普通科目:人倫道德、中國文學、歷史、地理、算學、博物、物理及化學、圖畫、體操。專門科目:交涉、理財、教育。五年畢業。進士館令新進士用翰林部屬、中書者,入館肄業,講求實用之學。課目:史學、地理、教育、法學、理財、交涉、兵政、農政、工政、商政、格致。得選習農、工、商、兵之一科或兩科。西文、東文、算學、體操為隨意科。三年畢業。

各學堂管理通則之規定,與舊章大體相同。月朔,監督、教員集諸生禮堂,宣讀聖諭廣訓一條。皇太后、皇上萬壽節,至聖先師孔子誕日,春、秋上丁釋奠,為慶祝日。堂中各員率學生至萬歲牌前或聖人位前行三跪九叩禮。畢,各員西鄉立,學生向各員行三揖禮,退。開學、散學或畢業,率學生至萬歲牌前、聖人位前行禮如儀。學生向監督、教員行一跪三叩禮。監督等施訓語,乃散。月朔,率學生至聖人位前行禮如儀。每日講堂授課,多者不得過六小時。房、虛、星、昴日為休息例假,慶祝日、端午、中秋節各放假一日。每年以正月二十日開學,至小暑節散學,為第一學期。立秋後六日開學,至十二月十五日散學,為第二學期。學生賞罰,由教員、監學摘出,監督核定。賞分三種:曰語言獎勵,曰名譽獎勵,曰實物獎勵。罰分三種:曰記過,曰禁假,曰出堂。學生以端飭品行為第一要義,監督、監學及教員隨時稽察,詳定分數,與科學分數合算。

學堂考試分五種:曰臨時考試,曰學期考試,曰年終考試,曰畢業考試,曰升學考試。臨時試無定期,學期、年終、畢業考試分數與平日分數平均計算。年考及格者升一級,不及格者留原級補習,下屆再試,仍不及格者退學。評定分數,以百分為滿格,八十分以上為最優等,六十分以上為優等,四十分以上為中等,二十分以上為下等,謂之及格,二十分以下為最下等,應出學。

畢業考試最重,視學堂程度,由所在地方官長會同監督、教員親蒞之,照鄉會試例。高等學畢業,簡放主考,會同督、撫、學政考試。大學分科畢業,簡放總裁,會同學務大臣考試。分內、外二場:外場試,就學堂舉行。擇各科講義精要一二條摘問,令諸生答述。內場試,擇地扃試。分兩場:首場以中學發題,經、史各一,經用論,史用策。二場以西學發題,政、藝各一,西政用考,西藝用說。通儒院畢業,不派員考試,以平日研究所得各種著述,評定等第,進呈,候欽定。其獎勵章程,比照獎勵出洋游學日本學生例,通儒院畢業,予以翰林升階,或分用較優京、外官。大學分科畢業,最優等作為進士出身,用翰林院編修、檢討。優等、中等均作為進士出身,分別用翰林院庶吉士、各部主事。大學選科,比照分科大學降等給獎。大學豫備科及各省高等學畢業,最優等作為舉人,以內閣中書、知州用。優等、中等均作為舉人,以中書科中書、部司務、知縣、通判用。中學畢業,分別獎以拔貢、優貢、歲貢。高等小學畢業,分別獎以廩、增、附生。初等小學屬義務教育,不給獎。優級師範畢業,最優等、優等、中等均作為舉人,分別以國子監博士、助教、學正用。初級師範畢業,分別獎以拔貢、優貢、歲貢,以教授、教諭、訓導用。高等實業學堂畢業,最優等、優等、中等均作為舉人,分別以知州、知縣、州同用。中等實業學堂畢業,獎勵視中學。奏定章程規定之概要如此。

三十一年,詔以各省學堂次第興辦,必須有總匯之區,以資董率而專責成。特設學部,命榮慶為尚書,熙瑛、嚴修為侍郎。裁國子監,歸並學部。明年,學部奏請宣示教育宗旨,略言:「今中國振興學務,宜注重普通教育,令全國之民無人不學。尤以明定宗旨,宣示天下,為握要之圖。中國政教所固有,亟宜發明以距異說者有二:曰忠君,曰尊孔。中國民質所最缺,亟宜箴砭以圖振起者有三:曰尚公,曰尚武,曰尚實。」上諭照所陳各節通飭遵行。尋奏定學部官制,於本部各司、科分掌教育行政事務外,設編譯圖書局、調查學制局、京師督學局。又擬設高等教育會議所,屬學部長官監督。其議員選派部員,及直轄學堂、各省中等以上學堂監督,暨京、外官紳,學識宏通,於教育素有經驗者充任。又擬設教育研究所,延聘精通教育之員,定期講演,以訓練本部員司焉。先是直督袁世凱奏陳學務未盡事宜,以裁撤學政為言。雲南學政吳魯奏請裁撤學政。至是學部會同政務處復議,言:「各省教育行政及擴張興學之經費,督飭辦學之考成,與地方行政在在皆有關系。學政位分較尊,事權不屬,於督、撫為敵體,諸事不便於稟承,於地方為客官,一切不靈於呼應。且地方寥闊,官立、公立、私立學堂日新月盛,勢不能如歲、科試分棚調考之例。而循例按臨,更日不暇給。勞費供張,無裨實事。擬請裁撤學政,各省改設提學使司提學使一員,統轄全省學務,歸督、撫節制。於省會置學務公所,分曹隸事。選派官紳有學行者,別設學務議紳四人,延訪本省學望較崇之紳士充選。議長一人,學部慎選奏派。」從之。嗣是各省學務始有確定之執行機關矣。

勸學所之設,創始於直隸學務處。時嚴修任學務處督辦,提倡小學教育,設勸學所,為、州、縣行政機關。仿警察分區辦法,採日本地方教育行政及學校管理法,訂定章程,頗著成效。三十二年,學部奏定勸學所章程,通行全國,即修呈訂原章也。勸學所由地方官監督,設總董一員,以縣視學兼充,綜核各學區事務。區設勸學員一人,任一學區內勸學之責,以勸募學生多寡,定勸學員成績之優劣。其章程內推廣學務一條,規定辦法凡五:曰勸學,曰興學,曰籌款,曰開風氣,曰去阻力。又奏定各省教育會章程,省會設立者為總會,府、州、縣設立者為分會,以補助教育行政,與學務公所、勸學所相輔而行。皆普及教育切要之圖也。

學部設立後,於各項學堂章程多所更正。其要者,如改訂考試辦法,詳定師範獎勵義務,變通中、小學課程,中學分文科、實科之類,然大致不外修正科目,確定限制,其宏綱細目,不能出奏定章程之範圍。所增定者,則女學堂章程也。先是學部官制已將女學列入職掌。三十三年,奏定女子師範、女子小學章程,以裨補家計,有益家庭教育為要旨。師範科目:修身、教育、國文、歷史、地理、算學、格致、圖畫、家事、裁縫、手藝、音樂、體操。四年畢業。音樂得隨意學習。小學分兩等,高等科目:修身、國文、算術、中國歷史、地理、格致、圖畫、女紅、體操,得酌加音樂,為隨意科。初等科目:修身、國文、算術、女紅、體操,得酌加音樂、圖畫二隨意科。均四年畢業。其授業鐘點,較男子小學減少,與男子小學分別設立,不得混合。宣統三年。奏設中央教育會議,以討論教育應行改進事宜及推行方法。則根據學部原奏,擬設高等教育會議所之規定行之。此為第二期有系統之教育制度也。

至考驗游學畢業生,光緒二十九年,鄂督張之洞奏準鼓勵游學章程。三十一年,學務大臣考驗北洋學生金邦平等,援照鄉、會試覆試例,奏請在保和殿考試,給予出身,分別錄用。迨三十二年,學部奏定,自本年始,每年八月舉行一次。並為綜覈名實起見,妥議考驗章程。將學成試驗與入官試驗分為兩事,酌照分科大學及高等學畢業章程,會同欽派大臣,按所習學科分門考試。酌擬等第,候欽定分別獎給進士、舉人等出身。仍將某科字樣加於進士等名目之上,以為表識。考試分兩場:第一場就所習學科擇要命題;第二場試中國文、外國文,罷廷試。明年,學部憲政編查館會奏游學畢業廷試錄用章程,仍暫照三十一年成案。於欽派大臣會同學部考試請予出身後,廷試一次,分別授職。廷試用經義、科學、論、說各一,其醫、工、格致、農等科大學及各項高等實業學堂畢業者,免試經義。時游學日本、歐、美畢業回國者,絡繹不絕,歲舉行考驗以為常,終清世不廢。


\end{pinyinscope}