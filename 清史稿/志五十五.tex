\article{志五十五}

\begin{pinyinscope}
地理二十七

△西藏

西藏:禹貢三危之地。在四川、雲南徼外,至京師萬有四千餘里。周為西戎,漢為西羌。唐為吐蕃,其君長號贊普。至宋朝貢不絕。元憲宗始於河州置吐蕃宣慰司都元帥府,四川徼外置碉門、魚通、黎雅、長河、西寧等處宣撫司。世祖時,復置烏斯藏,郡縣其地。明為烏斯藏,賜封號,設指揮、宣慰等司,以示羈縻。宣德、成化間,又累加封號。其地有僧號達賴喇嘛,居拉薩之布達拉廟,號為前藏;有班禪喇嘛,居日喀則城之札什倫布廟,號為後藏。太宗崇德七年,有達賴喇嘛及班禪,重譯來貢。未幾,為蒙古顧實汗所據。四傳至曾孫拉藏汗,而準噶爾並之。康熙五十九年,官兵西討,殲偽藏王,以西藏地賜達賴喇嘛,使蒙古舊臣頗羅鼐等五人分守。乾隆四年,敕封頗羅鼐為郡王,領藏事。至其子襲封,以罪誅,遂除西藏王爵。所有輔國公三,一等臺吉一,噶布倫四,戴琫五,碟巴三,堪布一。設駐藏辦事、幫辦大臣,分駐前後藏以轄之。其俗稱國曰圖伯特,又曰唐古忒。近因藏民不遵光緒十六年與英所定條約,辱其邊務大臣,致英兵入拉薩,要挾西藏立約十條,主權盡失。光緒三十一年,特派員至印度與英協商,其新改條約:一,西藏路礦電線,由中、英兩國妥議辦理,他國不得干預;二,西藏用人權,概歸英員與駐藏大臣會議辦理;三,西藏有亂,中政府須與英協商後派兵彈壓;四,西藏增設商埠,由中、英兩國會同辦理;五,西藏土地,非得中、英兩國承辦,不得租借轉賣。據條約觀之,西藏蓋為兩屬之國矣。

境內分四部:曰衛,曰康,曰藏,曰阿里。東界四川,東南界雲南,西界西域回部大沙海,北界青海及回部。廣六千餘里,袤五千餘里。北極高三十度三十五分,京師偏西二十四度十五分。

衛:一曰前藏,即古之危,亦稱中藏,即烏斯藏也。乾隆十五年,設大臣鎮守。其城曰布達拉城。有坐床,為達賴喇嘛所駐,協理藏事。東界喀木,西界後藏,南界不丹,北界青海及新疆。轄城二十八。喇薩城即布達拉,在打箭爐西北三千四百八十里。札什城在喇薩南七里。德慶城在喇薩東南三十八里。柰布東城在喇薩東南二百二十里。桑里城在喇薩東南二百五十一里。垂佳普郎城在喇薩東南二百六十里。野而古城在喇薩東南三百一十里。達克匝城在喇薩東南三百三十七里。則庫城在喇薩東南三百四十里。滿撮納城在喇薩東南四百四十里。拉巴隨城在喇薩東南四百四十里。札木達城在喇薩東南五百四十里。達喇馬宗城在喇薩東南五百六十里。古魯納木吉牙城在喇薩東南六百二十里。碩噶城在喇薩東南六百四十里。硃木宗城在喇薩東南七百五十里。東順城在喇薩東南七百七十里。則布拉岡城在喇薩東南八百七十里。納城在喇薩東南九百六十里。吉尼城在喇薩東南九百八十里。日噶牛城在喇薩西南三十里。楚舒爾城在喇薩西南百十五里。日喀爾城在喇薩西南百二十里。公喀爾城在喇薩西南百四十里,為衛地最大之城。岳吉牙來雜城在喇薩西南三百三十里。多宗城在喇薩西南四百二十里。僧格宗城在喇薩西南四百三十里。董郭爾城在喇薩西二十五里。第巴達克匝城在喇薩東北九十二里。倫硃布宗城在喇薩東北一百二十里。墨魯恭噶城在喇薩東北百五十里。蓬多城在喇薩東北百七十里。設大汛為護防。藏地凡大汛四,一在前藏城,一在後藏。又臺站二,自打箭爐至此有站五。魚通即打箭爐、里塘、巴塘,均屬四川有。前藏者二。曰察木多,曰拉里。城西南:巴則山。西:招拉筆洞山。又布達拉山,高百餘丈。又西:東噶爾山,高約四百餘丈,為西藏要隘。南:牛魔山,高二百餘丈。東北:郎路山、薩木多嶺。北:布克沙克河,源出噶爾占古察嶺,南流,西合沙克河。又東南流,西受庫蘭河,北受布克河。又東南,入喀喇烏蘇。雅魯藏布江,即大金沙江,古之跋布川也。源出藏西界卓書特部西北達木楚克哈巴布山,三源,俱東北流而合,折東流,枯木岡前山水自西南來會。又東北,江加蘇木拉河自西北沙苦牙拉麻拉山東南流來會。又東,阿拉楚河北自沙拉木岡前山水會而南流,又東稍北,拉烏克藏布必拉自東北桑里池西南流,合數水來會。又東南,郭永河自東南昂則嶺東北流,合數水來會。又東,薩楚藏布河自東北合諸水來會。又東,甕出河、式爾的河、滿楚藏布河、薩克藏布河,合諸水來會。又東南,加木租池水北自章阿布林城合東一水南流來注之。又東南,受西南來一水,又正北流,折向西北,受西北隆左池水。又東北,莽噶拉河南自那拉古董察山來注之。又東北,鐘裏山水自東南來注之。又東北,經章拉則城北,又東北,鄂宜楚藏布河自西北札木楚克池合諸水東南流來注之。又北流,戒忒楚河、札克北朋楚河自北來注之。又東南,會薩普楚河。又東逕普冬廟前,烏雨克河自拉公山來注之。又東過薩喇硃噶鐵索橋,逕林奔城北,龍前河自南合二水來注之。又東北,捏木河自西北來注之。折東南流,逕拜的城北岸山北,受西北來一小水,東北過鐵索橋,逕楚舒爾城南,東南至日喀爾公喀爾城北,噶爾招木倫江自東北合諸水,西南流逕衛地喇薩來會,疑即古吐蕃之臧河也。雅魯藏布江既會噶爾招木倫江,東南流,至打格布衣那城北,共八百里。年褚河自北合諸水來會。又東經叉母哈廟北,受東北薩母龍拉嶺水,南流入羅喀布占國。穆楚河合柰楚河,南流入哲孟雄。滕格里池,在境西北,藏地日喀則城東北,隔山即潞江源之布喀諸池。其北隔山即大流沙也。池廣六百餘里,周一千餘里,東西甚長,南北稍狹,蒙古呼天為「滕格里」,言水色同天青。其東有三水流入,皆名查哈蘇太河。西有二水流入,北曰羅薩河,南曰打爾古藏布河,合西來數池水,東流入此池。次曰牙母魯克於木卒池,中有三山,水成五色。曰馬品木達賴池,郎噶池,即狼楚河也。次曰布喀池,潞江源也。東噶爾山上有關。

康:一曰喀木。要寨曰察木多。在前藏東千二百五十里,東界四川,南界珞瑜境及英屬阿薩密,西界衛地,北界青海。喀木今曰昌都,亦稱前藏,本屬呼圖克圖。康熙五十八年始納款。設臺站,置糧員一。有土城。西南有羅隆宗、舒班多、達隆宗,西北有類伍齊等部落,其南有乍丫。康熙五十八年招撫。又南有江卡,雍正元年招撫,設有官寨。東:達蓋喇山、沖得喇山。南:安靜大山,與川、滇分界。西:嘉松古木山。東南:奪布喇山、鼎各喇山。西南:魚別喇山、裏角大山,冬春積雪。又巴貢山、蒙堡山、擦瓦山、雲山、雪山、白奪山、納奪山、黃雲山、隱山、喇貢山。東有列木喇嶺。羅隆宗東有得貢喇山,山勢陡峻。西:得噶喇山。舒班多東有章喇山。西南:吾抵喇山、巴喇山。西:朔馬喇山,即賽瓦合山。達隆宗西有必達喇山,沙貢喇山、魯貢喇山,兩山相連。類伍齊西南有瓦合大山,山大而峻,冬春積雪。又有擦噶喇山、葉達喇山。察木多左有昂楮河,源出中壩,因通云南,亦名云河。右有雜楮河,源出九茹,因通四川,亦名川河。二水合流,入雲南。瀾滄江二源,一源發於匝坐里岡城西北格爾吉匝噶那山,名匝楚河,一源發於匝坐里岡城西北巴喇克拉丹蘇克山,名鄂穆楚河,俱東南流,至匝坐里岡城東北察木多廟前,二水合流,名拉楚河,南流至包敦入乍丫。又南流至察瓦寺,甲倉河東北來會。又東南,左受色爾恭河,折南流,至角占,受左貢河。又東南流,逕茶利大雪山入雲南,始名瀾滄江。潞江在瀾滄西,發源於衛地之布喀大澤,淵澄黝黑,又多伏流,蒙古呼黑為「喀喇」,水為「烏蘇」,故名喀喇烏蘇。逕拉薩北,有池名布喀,橢圜形,廣六十里,袤一百五十里,從此池西北流出,入額爾吉根池,轉東北,入衣達池,又折東南流,入喀喇池。三池俱縱廣五六十里。中有三山,四池環抱。復從喀喇池東南出,納布倫河,又東受北來二小水,折南轉東,至喀喇烏蘇,為西寧進藏大道,皮船為渡。轉東北流,逕蒙古三十九族地,至伊庫山,沙克河西北來會。又東北流,逕蘇圖克土司,索克河自北來會。折南流,左右各受一小水,轉西南會衛楚河。折而東,受雄楚河。又東納沙隆錫河,轉東南流,類烏齊河自北來會。又東南逕必蚌山,至嘉玉橋,為滇、蜀入藏之大道。又東南流,江陽為巴克碩游牧,江陰為波密野番。又東南流,逕桑昂曲宗入江卡。江之外為怒夷,故名怒江。又東南,入雲南維西,折而南下,逕雲龍州西徼,右納捍江,入保山乃名潞江。南流逕潞江安撫司。又南流少東,左納沙河,轉西南至遮放土司,從此出滇境入緬甸。羅隆宗西有偶楮河,源出噶爾藏骨岔海子,海合瀾滄江南峽。隆喜楮河源出噶喇山,東流,合偶楮河。舒班多有納碩布楚河,源出中義溝,北流,逕舒班多城西,合三溪,東北流,入喀喇烏蘇。又柱嗎郎錯河,源出噶喇山,胄楮河,源出吾抵山,均流歸偶楮河。達隆宗北有撒楮河,源出朔馬喇山。東南:邊楮河,流合胄楮河。有俄楮河,源出沙貢喇山,流合葉楮河。類烏齊東北有扎楮河,即昂楮河下流。乍丫有勒楮河,源出昂喇山。樂楮河,源出作喇山。又有甲倉河,源出官角,西南流,逕草裏工,又西南,至洛隆宗,合洛楚河,又西南至乍丫寺前。與猛楚河合。有色楮河,源出上納,奪流入察木多大河。

拉里:一名喇里。在前藏東五百九十里,察木多西六百六十里,達隆宗西北。康熙五十五年,其地有黑喇嘛,附於準噶爾。尋討平之,以地屬前藏。設臺站,置糧員一。無城。西南有工布江達。江達稱沃壤。亦平西藏時就撫。又其南有達克,東北有西藏大臣所屬三十九土司。亦有入甘肅西寧界者,皆喀喇烏蘇番眾也。拉里有拉里大山,勢如龍,上下險峻,四時積雪。西南有瓦子山,番人呼為卓拉大山,延亙數百里,多積雪。江達西有鹿馬嶺,高約四十里,為西藏要隘。拉里東有同妥楮河,源出魯貢喇山,流合得楮河。有熱水塘,四時常溫,番人呼為擦楮卡。江達有岡布藏布河,自衛地東納東北察拉嶺水,又東南,有危楚河自東北來會。又東南,有牛楚河自西北來會。東流過打克拉崩橋,又東,受東北二水,又南,逕的牙爾山西,入岡布部落。至撇皮唐他拉東喀木境內,有薄藏布河自東北來會,土人曰喀克布必拉。逕噶克布衣書裏東城西,又南,逕塞母龍拉嶺東,朵格拉岡里山西,出岡布境。逕公拉岡里山西,又南入羅喀布占國,下流入雅魯藏布江。匝楚藏布江,即年渚必拉江,源出沙羽克岡拉山,即喀爾靼廟東南山也。有水東流,曰馬木楚河,與南來巴拉嶺之巴隆楚河會。又東北,與北來烏山之烏斯江會。又東,逕鹿馬嶺,至順達,有水曰佳囊河,發源過拉松多,東南流,逕江達城東,折而南流,合東二小水來會。又東南曲曲流,至工布什噶城南,有水東北自巴麻穆池南流,合東一水來會。又折而西南,有西來齊布山之牛楚河,合而南流,至工布珠穆宗城東、底穆宗城西,又東南至布拉岡城東,合於雅魯藏布江。又桑楚河,南流,有雅隆布河出舒班多南境來注之,是為薄藏布河,又南入羅喀布占國,注雅魯藏布江。

藏:即後藏,一曰喀齊。在前藏西南五百餘里地,曰札什倫布,即古之藏也。南界尼泊爾,東界衛地,西界阿里,北界新疆。乾隆十五年,設大臣鎮守。其城曰札什倫布城,有坐床,為班禪額爾德尼所駐,協理藏事。有汛三:在本城一;外二,曰江孜,曰定日。西境彭錯嶺。北境那木嶺。北有雅魯藏布江,出阿里西南界山,東流,有郭永河東北流注之。又受那烏克藏布河、薩布楚河、薩爾格藏布河,又近城北會多克楚河,至城西會南來之當出河,又逕布克什里山南,江至此已行二千五百餘里,又東入前藏界。北有打爾古藏布河,流入前藏,瀦為騰格里池,廣六百餘里。

江孜:在札什倫布城南二百里。駐守備一。南有帕裏邊寨,東連布魯克巴,西通哲孟雄,外接西洋部落噶里噶達。東有千壩,南有宗木小部落,西南有定結,北有拉孜,皆有官寨。東南:珠布拉大雪山。西南:喀木巴拉山及薩木嶺。定結之西有朋出藏布河,源有三,一西出書爾木藏拉山,一東出西爾中馬山,一東南出瓜查嶺。合而東南流,受西一小水。又南曰朋必拉,又南,有一水南自綽爾猛通那岡里山來會,折東流,受南北水各一。又東南流,受西南涿失岡千山、阿巴拉山之水二,又東流,受南一水,又東北至羅西喀爾城南,有一河,即西北拉喀拉布山二水,合東南流,逕城北。羅楚河自北納三水,南流合焉。又東北,羅藏布河自西北來注之。又東繞岡龍前山之北,折南流,受西來之牛藏布河。又東南,受帕裏藏布河。又西南,牛楚河西自年爾木城合數水來會。又東南流,出藏南境,過硃拉拉依部落,入厄訥特克國界,下流入雅魯藏布江。汛西有年楚河,源有二,一出硃母拉母山東北,一出其東順拉嶺下。泉池十數,匯為一水,北流,名章魯河,又東北,至娘娘廟東,有八水從東北喀魯嶺諸山,又南札木長山、社山來,合而西南流來會。轉西北流,過江孜城西,又西北過白滿城西,受四水來注。又北,始名年楚河。經日喀則城東南,過蘇木佳石橋,長七十丈,有十九洞,為藏地橋梁之冠。又北流,入雅魯藏布江,源長共八百餘里。西有帕裏藏布河,有一水西南流,匯為噶爾撮池,南流而西,又為查木蘇池,又西南流,折向東南,合東北來一水,又西南,會西北來之噶拉嶺水,又西逕帕裏城西,又西南受二水,土人名藏曲大河,西流入朋楚河。

定日:在札什倫布城西南七百餘里。駐守備一。有汛。城三面距邊,南有糸戎轄,西南有聶拉木,西有濟嚨,西北有宗喀。絨轄之東南有喀達,喀達之西南有陽布,俱接廓爾喀界。宗喀之南有布陵,南近廓爾喀,北接拉達克汗部落。其西北有薩喀,又西北極邊有阿里。以上各地俱有營官。東:崇烏拉山、甲錯山。西南:嘉汭大山。西:通拉山。喀達之西有霞烏拉山。宗喀之東有鞏塘拉山。布陵境內有岡底斯山,在阿里之達克喇城東北,直陜西西寧府西南五千五百九十餘里。其山高五百五十餘丈,周一百四十餘里,四面峰巒陡絕,高出眾山百餘丈,積雪如懸崖,浩然潔白。頂上有泉,流注至山麓,即伏流地下。前後環繞諸山,皆巉巖峭峻,奇峰拱列。按其地勢,出西南徼外,以漸而高,至此而極。山脈蜿蜒,分幹向西北者,為僧格喀巴布、岡裏木孫諸山,繞阿里而北,入西域之和闐南山及蔥嶺諸山。向東北者,為札布列斜而充、角烏爾充、年前唐拉、薩木坦岡、匝諾莫渾烏巴什、巴顏哈喇諸山。環衛地,竟青海,連延而下,六千餘里,至陜西西寧等處邊界。向西南者,為悶那克尼兒、薩木泰岡諸山,亙阿里之南,入厄訥特克國。向東南者,為達木楚克喀巴布岡、噶爾沙彌、弩金岡蒼諸山,歷藏、衛達喀木,至雲南、四川之境。康熙五十六年,遣喇嘛楚兒沁藏布蘭木占巴、理籓院主事勝住等,繪畫西海、西藏輿圖,測量地形,以此地為天下之脊,眾山之脈,皆由此起雲。水經注:「阿耨達山,西南有水名遙奴;山西南少東,有水名薩罕;少東,有水名恆伽。此三水同出一山,俱入恆水。」今阿里為藏中極西南地,近古天竺境。此山西出狼楚、拉楚、麻楚三大水皆西流,轉東而南,合為岡噶江,入南海。疑此即阿耨達山也。又有打母硃喀巴珀山,山形似馬。郎千喀巴珀山,山形似象。生格喀巴珀山,山形似獅。馬珀家喀巴珀山,山形似孔雀。皆與岡底斯山相連。岡噶江即出郎千喀巴珀山北麓,泉出匯為池,西北流,合東北來一水,又西而東北,公生池水伏而復出,合北來三水,西南流來會,為馬品木達賴池。自西流出為郎噶池,受東北來一水,從西流出,折向西南,曰狼楚河,曲曲二百餘里,有楚噶拉河自東北來注之。又西折北而東北,逕古格札什魯木布則城之西、則布龍城之東,折西北而西南流,逕則布龍城西南,又折而西北流,拉楚河自西北來會。三水既會,始名曰岡噶江。又東南流,出阿里界,逕馬木巴柞木郎部落,至厄訥特克入南海。朋出藏布河在定日北,東南流,納結楚河、隆崗河,入定結。有牛楚河,出喜拉岡參山,東南流,合東北岡布紇山水,又東南,逕濟嚨城南境,受北來查母硃山一水,始曰牛楚必拉。又東流,逕年爾母城北境,折而東南,又轉而東北流,會朋出藏布河。薩喀境內有鹽池。阿里東北九百餘里有達魯克池,隆布河與納鞠河皆入焉。


\end{pinyinscope}