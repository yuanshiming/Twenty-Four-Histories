\article{列傳五十二}

\begin{pinyinscope}
湯斌孫之旭陸隴其張伯行子師載

湯斌,字孔伯,河南睢州人。明末流賊陷睢州,母趙殉節死,事具明史列女傳。父契祖,挈斌避兵浙江衢州。順治二年,奉父還里。九年,成進士,選庶吉士,授國史院檢討。

方議修明史,斌應詔言:「宋史修於元至正,而不諱文天祥、謝枋得之忠;元史修於明洪武,而亦著丁好禮、巴顏布哈之義。順治元、二年間,前明諸臣有抗節不屈、臨危致命者,不可概以叛書。宜命纂修諸臣勿事瞻顧。」下所司。大學士馮銓、金之俊謂斌獎逆,擬旨嚴飭,世祖特召至南苑慰諭之。時府、道多缺員,上以用人方亟,當得文行兼優者,以學問為經濟,選翰林官,得陳爌、黃志遴、王無咎、楊思聖、藍潤、王舜年、範周、馬燁曾、沈荃及斌凡十人。

斌出為潼關道副使。時方用兵關中,徵發四至。總兵陳德調湖南,將二萬人至關欲留,斌以計出之,至洛陽譁潰。十六年,調江西嶺北道。明將李玉廷率所部萬人據雩都山寨,約降,未及期,而鄭成功犯江寧。斌策玉廷必變計,夜馳至南安設守。玉廷以兵至,見有備,卻走;遣將追擊,獲玉廷。

斌念父老,以病乞休,丁父憂。服闋,聞容城孫奇逢講學夏峰,負笈往從。康熙十七年,詔舉博學鴻儒,尚書魏象樞、副都御史金鋐以斌薦,試一等,授翰林院侍講,與修明史。二十年,充日講起居注官、浙江鄉試正考官,轉侍讀。二十一年,命為明史總裁官,遷左庶子。二十三年,擢內閣學士。江寧巡撫缺,方廷推,上曰:「今以道學名者,言行或相悖。朕聞湯斌從孫奇逢學,有操守,可補江寧巡撫。」瀕行,諭曰:「居官以正風俗為先。江蘇習尚華侈,其加意化導,非旦夕事,必從容漸摩,使之改心易慮。」賜鞍馬一、表裏十、銀五百。復賜御書三軸,曰:「今當遠離,展此如對朕也!」十月,上南巡,至蘇州,諭斌曰:「向聞吳閶繁盛,今觀其風土,尚虛華,安佚樂,逐末者多,力田者寡。爾當使之去奢返樸,事事務本,庶幾可挽頹風。」上還蹕,斌從至江寧,命還蘇州,賜御書及狐腋蟒服。

初,餘國柱為江寧巡撫,淮、揚二府被水,國柱疏言:「水退,田可耕,明年當徵賦。」斌遣覆勘,水未退即田,出水處猶未可耕,奏寢前議。二十四年,疏言:「江蘇賦稅甲天下,每歲本折五六百萬。上命分年帶徵漕欠,而地丁錢糧,自康熙十八年至二十二年,五年並徵。州縣比較,十日一限。使每日輪比,則十日中三日空閒,七日赴比。民知剜補無術,拌皮骨以捱徵比;官知催科計窮,拌降革以圖卸擔。懇將民欠地丁錢糧照漕項一例,於康熙二十四年起,分年帶徵。」又疏言:「蘇、松土隘人稠,而條銀漕白正耗以及白糧經費漕賸五米十銀,雜項差徭,不可勝計。區區兩府,田不加廣,而當大省百餘州縣之賦,民力日絀。順治初,錢糧起存相半,考成之例尚寬。後因兵餉急迫,起解數多,又定十分考成之例。一分不完,難逭部議。官吏顧惜功名,必多茍且。參罰期迫,則以欠作完;賠補維艱,又以完為欠。百姓脂膏已竭,有司智勇俱困。積欠年久,惟恃恩蠲。然與其赦免於追呼既窮之後,何若酌減於徵比未加之先。懇將蘇、松錢糧各照科則量減一二成,定適中可完之實數,再將科則稍加歸並,使簡易明白,便於稽覈。」又請蠲蘇、松等七府州十三年至十七年未完銀米,淮、揚二府十八九兩年災欠,及邳州版荒、宿遷九釐地畝款項,並失額丁糧,皆下部議行。九釐地畝款項,即明萬歷後暫加三餉,宿遷派銀四千三百有奇,至是始得蠲免。

淮、揚、徐三府復水,斌條列蠲賑事宜,請發帑五萬,糴米湖廣,下俟詔下,即行咨請漕運總督徐旭齡、河道總督靳輔分賑淮安。斌赴清河、桃源、宿遷、邳、豐諸州縣察賑,疏聞,上命侍郎素赫助之。先後奏劾知府趙祿星、張萬壽,知縣陳協濬、蔡司霑、盧綖、葛之英、劉濤、劉茂位等。常州知府祖進朝以失察屬吏降調,斌察其廉,奏留之。又疏薦吳縣知縣劉滋、吳江知縣郭琇廉能最著,而徵收錢糧,未能十分全完,請予行取。下部皆議駁,特旨允行。

斌令諸州縣立社學,講孝經、小學,修泰伯祠及宋範仲淹、明周順昌祠,禁婦女游觀,胥吏、倡優毋得衣裘帛,毀淫詞小說,革火葬。蘇州城西上方山有五通神祠,幾數百年,遠近奔走如騖。諺謂其山曰「肉山」,其下石湖曰「酒海」。少婦病,巫輒言五通將娶為婦,往往瘵死。斌收其偶像,木者焚之,土者沉之,並飭諸州縣有類此者悉毀之,撤其材修學宮。教化大行,民皆悅服。

方明珠用事,國柱附之。布政使龔其旋坐貪,為御史陸隴其所劾,因國柱賄明珠得緩;國柱更欲為斌言,以斌嚴正,不得發。及蠲江南賦,國柱使人語斌,謂皆明珠力,江南人宜有以報之,索賕,斌不應。比大計,外吏輦金於明珠門者不絕,而斌屬吏獨無。

二十五年,上為太子擇輔導臣,廷臣有舉斌者。詔曰:「自古帝王諭教太子,必簡和平謹恪之臣,統率宮僚,專資輔翼。湯斌在講筵時,素行謹慎,朕所稔知。及簡任巡撫,潔己率屬,實心任事。允宜拔擢,以風有位。」授禮部尚書,管詹事府事。將行,吳民泣留不得,罷市三日,遮道焚香送之。初,靳輔與按察使於成龍爭論下河事,久未決。廷臣阿明珠意,多右輔。命尚書薩穆哈、穆成額會斌勘議,斌主濬下河如成龍言。薩穆哈等還京師,不以斌語聞。斌至,上問斌,斌以實對。薩穆哈等坐罷去。

二十六年五月,不雨,靈臺郎董漢臣上書指斥時事,語侵執政,下廷議,明珠惶懼,將引罪。大學士王熙獨曰:「市兒妄語,立斬之,事畢矣。」斌後至,國柱以告。斌曰:「漢臣應詔言事無死法。大臣不言而小臣言之,吾輩當自省。」上卒免漢臣罪。明珠、國柱愈恚,摘其語上聞,並摭斌在蘇時文告語,曰「愛民有心,救民無術」,以為謗訕,傳旨詰問。斌惟自陳資性愚昧,愆過叢集,乞賜嚴加處分。左都御史璙丹、王鴻緒等又連疏劾斌。會斌先薦候補道耿介為少詹事,同輔太子,介以老疾乞休。詹事尹泰等劾介僥幸求去,且及斌妄薦,議奪斌官,上獨留斌任。國柱宣言上將隸斌旗籍,斌適扶病入朝,道路相傳,聞者皆泣下。江南人客都下者,將擊登聞鼓訟冤,繼知無其事,乃散。

九月,改工部尚書。未幾,疾作,遣太醫診視。十月,自通州勘貢木歸,一夕卒,年六十一。斌既卒,上嘗語廷臣曰:「朕遇湯斌不薄,而怨訕不休,何也?」明珠、國柱輩嫉斌甚,微上厚斌,斌禍且不測。

斌既師奇逢,習宋諸儒書。嘗訂:「滯事物以窮理,沉溺跡象,既支離而無本;離事物而致知,隳聰黜明,亦虛空而鮮實。」其教人,以為必先明義利之界,謹誠偽之關,為真經學、真道學;否則講論、踐履析為二事,世道何賴。斌篤守程、硃,亦不薄王守仁。身體力行,不尚講論,所詣深粹。著有洛學編、潛庵語錄。雍正中,入賢良祠。乾隆元年,謚文正。道光三年,從祀孔子廟。

孫之旭,字孟升。康熙四十五年進士,官編修,改御史。出為霸昌道,內遷左通政。所至皆有聲。

陸隴其,初名龍其,字稼書,浙江平湖人。康熙九年進士。十四年,授江南嘉定知縣。嘉定大縣,賦多俗侈。隴其守約持儉,務以德化民。或父訟子,泣而諭之,子掖父歸而善事焉;弟訟兄,察導訟者杖之,兄弟皆感悔。惡少以其徒為暴,校於衢,視其悔而釋之。豪家僕奪負薪者妻,發吏捕治之,豪折節為善人。訟不以吏胥逮民,有宗族爭者以族長,有鄉里爭者以里老;又或使兩造相要俱至,謂之自追。徵糧立掛比法,書其名以俟比,及數者自歸;立甘限法,令以今限所不足倍輸於後。

十五年,以軍興徵餉。隴其下令,謂「不戀一官,顧無益於爾民,而有害於急公」。戶予一名刺勸諭之,不匝月,輸至十萬。會行間架稅,隴其謂當止於市肆,令毋及村舍。江寧巡撫慕天顏請行州縣繁簡更調法,因言嘉定政繁多逋賦,隴其操守稱絕一塵,才幹乃非肆應,宜調簡縣。疏下部議,坐才力不及降調。縣民道為盜所殺而訟其仇,隴其獲盜定讞。部議初報不言盜,坐諱盜奪官。十七年,舉博學鴻儒,未及試,丁父憂歸。十八年,左都御史魏象樞應詔舉清廉官,疏薦隴其潔己愛民,去官日,惟圖書數卷及其妻織機一具,民愛之比於父母,命服闋以知縣用。

二十二年,授直隸靈壽知縣。靈壽土瘠民貧,役繁而俗薄。隴其請於上官,與鄰縣更迭應役,俾得番代。行鄉約,察保甲,多為文告,反覆曉譬,務去鬥很輕生之習。二十三年,直隸巡撫格爾古德以隴其與兗州知府張鵬翮同舉清廉官。二十九年,詔九卿舉學問優長、品行可用者,隴其復被薦,得旨行取。隴其在靈壽七年,去官日,民遮道號泣,如去嘉定時。授四川道監察御史。偏沅巡撫於養志有父喪,總督請在任守制。隴其言天下承平,湖廣非用兵地,宜以孝教。養志解任。

三十年,師征噶爾丹,行捐納事例。御史陳菁請罷捐免保舉,而增捐應升先用,部議未行。隴其疏言:「捐納非上所欲行,若許捐免保舉,則與正途無異,且是清廉可捐納而得也;至捐納先用,開奔競之途:皆不可行。更請捐納之員三年無保舉,即予休致,以清仕途。」九卿議,謂若行休致,則求保者奔競益甚。詔再與菁詳議,隴其又言:「捐納賢愚錯雜,惟恃保舉以防其弊。若並此而可捐納,此輩有不捐納者乎?議者或謂三年無保舉即令休致為太刻,此輩白丁得官,踞民上者三年,亦已甚矣;休致在家,儼然搢紳,為榮多矣。若云營求保舉,督撫而賢,何由奔競;即不賢,亦不能盡人而保舉之也。」詞益激切。菁與九卿復持異議。戶部以捐生觀望,遲誤軍需,請奪隴其官,發奉天安置。上曰:「隴其居官未久,不察事情,誠宜處分,但言官可貸。」會順天府尹衛既齊巡畿輔,還奏民心皇皇,恐隴其遠謫,遂得免。

尋命巡視北城。試俸滿,部議調外,因假歸。三十一年,卒。三十三年,江南學政缺,上欲用隴其,侍臣奏隴其已卒,乃用邵嗣堯,嗣堯故與隴其同以清廉行取者也。雍正二年,世宗臨雍,議增從祀諸儒,隴其與焉。乾隆元年,特謚清獻,加贈內閣學士兼禮部侍郎。

著有困勉錄、松陽講義、三魚堂文集。其為學專宗硃子,撰學術辨。大指謂王守仁以禪而託於儒,高攀龍、顧憲成知闢守仁,而以靜坐為主,本原之地不出守仁範圍,詆斥之甚力。為縣崇實政,嘉定民頌隴其,迄清季未已。靈壽鄰縣阜平為置塚,民陸氏世守焉,自號隴其子孫。

張伯行,字孝先,河南儀封人。康熙二十四年進士,考授內閣中書,改中書科中書。丁父憂歸,建請見書院,講明正學。儀封城北舊有堤,三十八年六月,大雨,潰,伯行募民囊土塞之。河道總督張鵬翮行河,疏薦堪理河務,命以原銜赴河工,督修黃河南岸堤二百餘里及馬家港、東壩、高家堰諸工。四十二年,授山東濟寧道。值歲饑,即家運錢米,並制棉衣,拯民饑寒。上命分道治賑,伯行賑汶上、陽穀二縣,發倉穀二萬二千六百石有奇。布政使責其專擅,即論劾,伯行曰:「有旨治賑,不得為專擅。上視民如傷,倉穀重乎?人命重乎?」乃得寢。四十五年,上南巡,賜「布澤安流」榜。

尋遷江蘇按察使。四十六年,復南巡,至蘇州,諭從臣曰:「朕聞張伯行居官甚清,最不易得。」時命所在督撫舉賢能官,伯行不與。上見伯行曰:「朕久識汝,朕自舉之。他日居官而善,天下以朕為知人。」擢福建巡撫,賜「廉惠宣猷」榜。伯行疏請免臺灣、鳳山、諸羅三縣荒賦。福建米貴,請發帑五萬巿湖廣、江西、廣東米平糶。建鼇峰書院,置學舍,出所藏書,搜先儒文集刊布為正誼堂叢書,以教諸生。福州民祀瘟神,命毀其偶像,改祠為義塾,祀硃子。俗多尼,鬻貧家女,髡之至千百,伯行命其家贖還擇偶,貧不能贖,官為出之。

四十八年,調江蘇巡撫,賑淮、揚、徐三府饑。會布政使宜思恭以司庫虧空為總督噶禮劾罷,上遣尚書張鵬翮按治。陳鵬年以蘇州知府署布政使,議司庫虧三十四萬,分扣官俸役食抵補,伯行咨噶禮會題,不應。伯行疏上聞,上命鵬翮並按。別疏陳噶禮異議狀,上諭廷臣曰:「覽伯行此疏,知與噶禮不和。為人臣者,當以國事為重。朕綜理機務垂五十年,未嘗令一人得逞其私。此疏宜置不問。」伯行尋乞病,上不許。鵬翮請責前任巡撫於準及思恭償十六萬,餘以官俸役食抵補。上曰:「江南虧空錢糧,非官吏侵蝕。朕南巡時,督撫肆意挪用而不敢言。若責新任官補償,朕心實有不忍。」命察明南巡時用款具奏。伯行又疏奏各府州縣無著錢糧十萬八千,上命並予豁免。

噶禮貪橫,伯行與之迕。五十年,江南鄉試副考官趙晉交通關節,榜發,士論譁然,輿財神入學宮。伯行疏上其事,正考官左必蕃亦以實聞,命尚書張鵬翮、侍郎赫壽按治,伯行與噶禮會鞫,得舉人吳泌、程光奎通賄狀,詞連噶禮。伯行請解噶禮任付嚴審,噶禮不自安,亦摭伯行七罪訐奏。上命俱解任,鵬翮等尋奏晉與泌、光奎通賄俱實,擬罪如律;噶禮交通事誣,伯行應奪官。上切責鵬翮等掩飾,更命尚書穆和倫、張廷樞覆按,仍如前議。上曰:「伯行居官清正,天下所知。噶禮才雖有餘而喜生事,無清正名。此議是非顛倒,命九卿、詹事、科道再議。」明日,召九卿等諭曰:「伯行居官清廉,噶禮操守朕不能信。若無伯行,則江南必受其朘削幾半矣。此互參一案,初遣官往審,為噶禮所制,致不能得其情;再遣官往審,與前無異。爾等能體朕保全清官之意,使正人無所疑懼,則海宇升平矣。」遂奪噶禮官,命伯行復任。

五十二年,江蘇布政使缺員,伯行疏薦福建布政使李發甲、臺灣道陳瑸、前祭酒餘正健,上已以湖北按察使牟欽元擢任。未幾,伯行劾欽元匿通海罪人張令濤署中,請逮治。令濤兄元隆居上海,造海船,出入海洋,擁厚貲,結納豪貴。會部檄搜緝海賊鄭盡心餘黨,崇明水師捕漁船,其舟人福建產,冒華亭籍,驗船照為元隆所代領,伯行欲窮治。是時令濤在噶禮幕,元隆稱病不就逮,獄未竟而死於家。噶禮前劾伯行,因摭其事為七罪之一。會上海縣民顧協一訴令濤據其房屋,別有水寨數處窩藏海賊,稱令濤今居欽元署中。上命總督赫壽察審,赫壽庇令濤,以通賊無證聞;復命鵬翮及副都御史阿錫鼐按其事,鵬翮等奏元隆、令濤皆良民,請奪伯行官。上命復審,且命伯行自陳,伯行疏言:「元隆通賊,雖報身故,而金多黨眾,人人可以冒名,處處可以領照。令濤乃顧協一首告,若其不實,例應坐誣;欽元庇匿,致案久懸。臣為地方大吏,杜漸防微,豈得不究?」既命解任,鵬翮等仍以伯行誣陷良民、挾詐欺公,論斬,法司議如所擬,上免其罪,命伯行來京。

旋入直南書房,署倉場侍郎,充順天鄉試正考官。授戶部侍郎,兼管錢法、倉場,再充會試副考官。雍正元年,擢禮部尚書,賜「禮樂名臣」榜。二年,命赴闕里祭崇聖祠。三年,卒,年七十五。遺疏請崇正學,勵直臣。上軫悼,贈太子太保,謚清恪。光緒初,從祀文廟。

伯行方成進士,歸構精舍於南郊,陳書數千卷縱觀之,及小學、近思錄,程、硃語類,曰:「入聖門庭在是矣。」盡發濂、洛、關、閩諸大儒之書,口誦手抄者七年。始赴官,嘗曰:「千聖之學,括於一敬,故學莫先於主敬。」因自號曰敬庵。又曰:「君子喻於義,小人喻於利。老氏貪生,佛者畏死,烈士徇名,皆利也。」在官所引,皆學問醇正,志操潔清,初不令知。平日齮齕之者,復與共事,推誠協恭,無絲毫芥蒂。曰:「已荷保全,敢以私廢公乎?」所著有困學錄、續錄、正誼堂文集、居濟一得諸書。

子師載,字又渠。舉人。以父廕補戶部員外郎。雍正初,授揚州知府。歲饑,高郵湖西民以縣吏報災輕,不得賑。師載行部,見饑民滿道,不待報而賑之。江都芒稻閘為淮、黃,高、寶諸河入江要津,夏潦盛漲。閘官利商人餌,謂非運使令不得啟。師載詢鹽艘須水六七尺,今過半,乃身往督役啟閘。其後芒稻閘屬府啟閉,遂以為例。累遷江蘇按察使,內擢右通政。再遷,授倉場侍郎,命協辦江南河務。授安徽巡撫,仍命赴南河協同防護。會河溢,奪官。上命誅疏防同知李焞、守備張賓,使師載視行刑,畢,釋之。再起為兵部侍郎,遷漕運總督。復授河東河道總督。師載長於治河。少讀父書,研性理之學,高宗稱其篤實。卒,贈太子太保,謚愨敬。

論曰:清世以名臣從祀孔子廟,斌、隴其、伯行三人而已,皆以外吏起家,蒙聖祖恩遇。隴其官止御史,而廉能清正,民愛之如父母,與斌、伯行如一,其不為時所容而為聖祖所愛護也亦如一。君明而臣良,漢、唐以後,蓋亦罕矣。斌不薄王守仁,隴其篤守程、硃,斥守仁甚峻,而伯行繼之。要其躬行實踐,施於政事,皆能無負其所學,雖趨鄉稍有廣隘,亦無所軒輊焉。


\end{pinyinscope}