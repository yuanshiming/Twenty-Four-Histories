\article{本紀九}

\begin{pinyinscope}
世宗本紀

世宗敬天昌運建中表正文武英明寬仁信毅睿聖大孝至誠憲皇帝諱胤禛,聖祖第四子也。母孝恭仁皇後烏雅氏。生有異徵,天表魁偉,舉止端凝。康熙三十七年封貝勒。四十八年封雍親王。

六十一年十一月,聖祖在申昜春園不豫,命代祀圜丘。甲午,聖祖大漸,召於齋宮,宣詔嗣位。聖祖崩。辛丑,上即位,以明年為雍正元年。命貝勒胤禩、皇十三弟胤祥、大學士馬齊、尚書隆科多總理事務。召撫遠大將軍胤禵來京。命兵部尚書白潢協理大學士。以楊宗仁為湖廣總督,年希堯署廣東巡撫。

十二月戊午,停止直省貢獻方物。壬戌,封貝勒胤禩為廉親王,胤祥為怡親王,胤祹為履郡王,廢太子胤礽之子弘晰為理郡王。更定歷代帝王廟祀典。癸亥,詔古今圖書集成一書尚未竣事,宜速舉淵通之士編輯成書。以輔國公延信為西安將軍,署撫遠大將軍事。甲子,詔直省倉庫虧空,限三年補足,逾限治罪。命富寧安為大學士,隆科多為吏部尚書,廉親王胤禩管理籓院尚書事。壬申,以張廷玉為禮部尚書。予大學士馬齊二等伯爵,賜名敦惠。

雍正元年癸卯春正月辛巳朔,頒詔訓飭督、撫、提、鎮,文吏至於守、令,武將至於參、游,凡十一道。丙戌,時享太廟。辛卯,祈穀於上帝。壬寅,頒賜提、鎮、副將大行遺念弓矢橐鞬。刑部尚書陶賴、張廷樞坐審訊陳夢雷一案釋其二子,降官。甲辰,封淳郡王子弘曙為長子、弘春為貝子。乙巳,大學士王掞乞休,允之。

二月辛亥朔,以佛格、勵廷儀為刑部尚書。壬子,以張鵬翮為大學士。乙卯,以皇十六弟胤祿出嗣莊親王博果鐸,襲其爵。以博果鐸之侄球琳為貝勒。庚申,訓飭貝子胤禟。乙丑,封輔國公延信為貝子。定部院書吏考滿回籍聽選例。敕科道官每日一人具摺奏事。辛未,以宜兆熊為福州將軍。趙之垣免,以李維鈞為直隸巡撫。己卯,副將軍阿喇衲奏羅卜腦兒回人投順。

三月甲申,罷西藏防兵戍察木多。加隆科多、馬齊、年羹堯太保。命督撫疏薦幕賓。封年羹堯三等公。壬辰,命故安和親王岳樂之孫吳爾占、色亨圖、經希及其子移居盛京,除屬籍。

夏四月辛亥,大行梓宮奉安饗殿,命貝子胤禵留護。丙辰,命怡親王胤祥總理戶部,封其子弘昌為貝子。設鄉、會試繙譯科。乙丑,復置起居注官。封皇十七弟胤禮為果郡王。丁卯,初禦乾清門聽政。制詔訓飭大學士、領侍衛內大臣、文武大臣凡三道。丙子,晉封淳郡王胤祐為親王。敕總兵官具摺言事。

五月庚辰,詔免雲南入藏兵丁應補倒斃馬匹。癸卯,御太和殿視朝。李維鈞請以州縣歲入彌補積虧。上曰:「州縣官令少從容,方可責之盡心興舉,豈可勒為他人補虧缺耶!」乙酉,敕理郡王弘晰移住鄭家莊。丁酉,命尚書徐元夢署大學士。辛丑,仁壽皇太后崩,帝之生母也,奉安梓宮於寧壽宮。封貝子胤禵為恂郡王。

六月丁巳,以左世永為漢軍都統。己未,加封孔子五世王爵。辛酉,命八旗無恆產者移居熱河墾田。壬戌,青海郡王額爾得尼為羅卜藏丹津所破,率屬來投,遣官撫之。其侄噶爾丹達錫續來歸附,命同居於蘇油。壬申,敕李維鈞:「畿甸之內,旗民雜處,旗人暴橫,頗苦小民。爾當整飭,不必避忌旗、漢形跡,畏懼王公勛戚,皆密奏以聞。」丙子,敕八旗人員有為本旗都統、本管王公刁難苛索者,許其控訴。

秋七月己卯,命侍郎常壽諭和羅卜藏丹津。乙酉,遣官赴盛京、江西、湖廣糶米運京。己丑,詔免江西漕糧腳耗運費誤追者。壬辰,改國語固山額真為固山昂邦,伊都額真為伊都章京。辛巳,停本年秋決。除紹興惰民丐籍。頒行孝經衍義。壬寅,命隆科多、王頊齡監修明史,徐元夢、張廷玉為總裁。

八月丁巳,以楊琳為廣東總督,孔毓珣為廣西總督。甲子,召王大臣九卿面諭之曰:「建儲一事,理宜夙定。去年十一月之事,倉卒之間,一言而定。聖祖神聖,非朕所及。今朕親寫密封,緘置錦匣,藏於正大光明匾額之後,諸卿其識之。」庚午,常壽疏報行抵青海,諭和羅卜藏丹津,不從。詔年羹堯備兵。辛未,上謁陵。

九月丁丑朔,葬聖祖仁皇帝於景陵,孝恭皇后祔焉。是日,五色雲見。己卯,上還京。辛巳,以郝玉麟為雲南提督。壬午,以張廷玉為戶部尚書,張伯行為禮部尚書。癸巳,以裕親王保泰管鑲黃旗事務。命纂修律例。丙申,以阿喇衲為蒙古都統。

冬十月戊申,敕授年羹堯撫遠大將軍,改延信為平逆將軍。癸亥,羅卜藏丹津執我使臣常壽,筆帖式多爾濟死之。癸酉,以阿爾松阿為禮部尚書,尹泰為左都御史。

十一月丁丑,賜於振等二百四十六人進士及第出身有差。戊寅,羅卜藏丹津入寇西寧,守備馬有仁、參將宋可進敗之於申中堡,賊遁。丙戌,年羹堯奏總兵楊盡信進剿番賊於莊浪椅子山,斬賊數百。得旨嘉獎。辛丑,冬至,祀天於圜丘,奉聖祖仁皇帝配享。

十二月丙午朔,以吳爾占等怨望,不準承襲安郡王,並撤所屬佐領。辛酉,年羹堯奏賊人來犯,參將孫繼宗擊敗之。安插洋人於澳門,改天主堂為公所,嚴禁入教。丁卯,冊嫡妃那拉氏為皇后,封年氏為貴妃,鈕祜祿氏為熹妃,耿氏為裕嬪。甲戌,祫祭太廟。

是歲,免直隸、江南等省四十九州縣災賦有差。朝鮮、琉球入貢。丁戶二千五百三十二萬六千二百七十,又永不加賦後滋生人丁四十八萬五百五十七。田賦徵銀三千二十二萬三千九百四十三兩有奇。鹽課銀四百二十六萬一千九百三十三兩有奇。鑄錢四十九萬九千二百有奇。

二年甲辰春正月辛巳,祈穀於上帝,奉聖祖仁皇帝配享。詔大學士圖海配享太廟。常壽自羅卜藏丹津處回,命監禁西安。丁亥,命岳鍾琪為奮威將軍,專征青海。丁酉,以高其佩為漢軍都統。庚子,建孔子廟於歸化城。

二月丙午,禦制聖諭廣訓,頒行天下。戊午,岳鍾琪兵至青海,擒阿爾布坦溫布等三虜,收撫逃散部落。詔以青海軍事將竣,策旺阿拉布坦恭順,罷阿爾泰及烏蘭古木兵。辛酉,詔臨雍大典,改幸學為詣學。癸亥,上耕耤田,三推畢,復加一推。甲子,敕州縣舉老農,予頂戴。年羹堯奏涼莊道蔣泂剿平阿岡部落,加按察使銜。丙寅,高其倬奏中甸番夷就撫。庚午,上祈雨於黑龍潭。

三月乙亥朔,上詣太學釋奠,御彞倫堂講尚書、大學,廣太學鄉試中額。丁丑,祭歷代帝王廟。庚辰,上謁陵。岳鍾琪師抵賊巢,羅卜藏丹津遁,獲其母阿爾泰喀屯,青海平。封年羹堯一等公,岳鍾琪三等公,發帑金二十萬犒軍。乙酉,清明節,上詣景陵行敷土禮。丁亥,還宮。

夏四月丁未,以孔毓珣為兩廣總督,李紱為廣西巡撫。庚戌,召王大臣訓飭廉親王胤禩,令其改行,並令王大臣察其善惡,據實奏聞。己巳,敦郡王胤蓪有罪,削爵拘禁。

閏四月丁丑,續修會典。丙戌,以嵇曾筠為河道副總督。丁酉,以蘇丹為蒙古都統。癸未,青海叛虜阿爾布坦溫布、吹拉克諾木齊、藏巴扎布械系至京,上御午門受俘。

五月癸卯朔,夏至,祭地於方澤,奉聖祖仁皇帝配享。貝勒阿布蘭復降為輔國公。丙辰,貝子蘇努坐廉親王黨削爵,並其子俱發右衛。辛酉,詔川、陜、湖廣、雲、貴督、撫、提、鎮:「朕聞各處土司,鮮知法紀,苛待屬人,生殺任性。方今海宇樂利,而土民獨切向隅,朕心不忍。宜嚴飭土司,勿得肆為殘暴,以副朕子惠元元至意。」壬戌,以那敏為滿洲都統。戊辰,貝子弘春削爵。

六月癸未,敕八旗勿擅毆死家人。乙酉,以青海平定,勒石太學。戊戌,上以闕里廟災,致祭先師,遣官監修。降貝子胤祹為鎮國公。李光復罷,以李永紹為工部尚書。

秋七月丁巳,禦制朋黨論,頒示諸臣。壬戌,以丁壽為阿爾泰駐防將軍。癸亥,副將軍阿喇衲卒於軍,上念其久勞於外,加予世職。

八月甲戌,命鄉、會試回避士子一體考試,別派大臣閱取。壬午,停本年秋決。庚寅,以田文鏡署河南巡撫。

九月辛丑朔,以阿爾泰軍功予丁壽世職。停戶部捐納事例。甲寅,命山西丁銀攤入地糧徵收,其後各省以漸行之。

冬十月乙亥,賜陳德華等二百九十九人進士及第出身有差。戊寅,封明裔硃之璉為一等侯,世奉明祀。癸未,詔京師建忠義祠。乙未,詔厄魯特郡王額駙阿寶賜往青海游牧。設寧夏駐防。丙申,刑部尚書阿爾松阿以無心效力,奪職削爵,發往盛京,以其伯音德襲果毅公。暹羅國貢稻種果樹。設直隸布政司、按察司,以巡撫李維鈞為總督。庚子,以音德、誇岱俱為領侍衛內大臣。丁未,以蘇丹為寧夏將軍。

十一月庚戌,弘晟有罪削爵。乙卯,以綽奇為蒙古都統,噶爾弼為漢軍都統。丁巳,高其倬奏官兵進剿仲苗,平之。辛酉,定稱孝莊文皇后山陵為昭西陵。

十二月癸酉,命太學立進士題名碑。癸未,廢太子胤礽薨,封理親王,謚曰密。以綽奇為奉天將軍。己丑,裕親王保泰有罪削爵,以其弟子廣寧襲封裕親王。設湖南學政。戊戌,祫祭太廟。

是歲,免江南、浙江等省五十七州縣衛災賦有差。朝鮮、安南、暹羅入貢。

三年乙巳春正月癸丑,詔以固安官地二百頃為井田,遣八旗閑散受耕。壬戌,以蔡珽為左都御史。癸亥,以阿齊圖為步軍統領。

二月庚午,日月合璧,五星聯珠。庚辰,上以三年服闋,行祫祭禮。丁亥,詔責年羹堯未能撫恤青海殘部,倘有一二人逃入準噶爾者,必重罪之。乙未,鄂倫岱坐廉親王黨奪職削爵,發往盛京,以其弟誇岱襲一等公。丁酉,召廷臣宣示胤禟罪狀,並及胤禩、胤蓪、胤禵。

三月丁未,以馬會伯為貴州提督。策旺阿拉布坦遣使入貢。設安徽學政。癸丑,大學士張鵬翮卒。禮部尚書張伯行卒。丁巳,蠲蘇、松浮糧四十五萬兩。滿保奏臺灣生番七十四社歸化。辛酉,年羹堯表賀日月合璧,五星聯珠,將「朝乾夕惕」寫作「夕惕朝乾」。詔切責之曰:「年羹堯非粗心者,是直不以朝乾夕惕許朕耳。則年羹堯青海之功,亦在朕許與不許之間,未可知也。顯系不敬,其明白回奏。」乙丑,敘總理王大臣、怡親王胤祥予一子郡王,隆科多、馬齊加予世職。廉親王胤禩不與,並嚴詔訓責之。

夏四月己卯,調年羹堯為杭州將軍。以岳鍾琪為川陜總督。遣學士眾佛保、副都統查史往準噶爾定界。以董吉那為江寧將軍。辛卯,以田從典為大學士。

五月癸亥,以左都御史尹泰為盛京禮部侍郎,兼理奉天府尹。

六月癸酉,詔年羹堯之子年富、年興,隆科多之子玉柱俱褫職。乙亥,命上三旗世職及登城巴圖魯之子,二十以下,十四以上,揀選引見錄用。削年羹堯太保,尋褫其一等公。

秋七月丁未,削隆科多太保。壬戌,大學士白潢罷,以高其位為大學士,張廷玉署大學士。命隆科多往阿蘭善山修城。壬戌,杭州將軍年羹堯黜為閑散旗員。癸亥,貝子胤禟有罪削爵。

八月辛未,李維鈞以黨年羹堯逮鞫,以李紱為直隸總督。壬辰,上駐圓明園。加怡親王胤祥俸,果郡王胤禮護衛。

九月甲寅,以硃軾為大學士,改蔡珽為吏部尚書,仍管兵部、都察院事。丙辰,逮系年羹堯下刑部。

冬十月戊辰,命巡撫不與總督同城者,參劾屬員,自行審結。丙子,封恆親王胤祺子弘晊輔國公。庚寅,以楊名時為雲貴總督,管巡撫事,鄂爾泰為雲南巡撫,管總督事。

十一月庚子,上謁陵。戊申,還宮。癸亥,以噶爾弼為奉天將軍。

十二月丁卯,降郡王胤禵為貝子。甲戌,廷臣議上年羹堯罪九十二款。得旨:「年羹堯賜死,其子年富立斬,餘子充軍,免其父兄緣坐。」辛巳,汪景祺以謗訕處斬。癸未,以覺羅巴延德為天津水師營都統。壬辰,祫祭太廟。

是歲,免直隸、江蘇、河南、浙江、廣東等省二十七州縣災賦有差。朝鮮、琉球、西洋國入貢。

四年丙午春正月甲午,上御太和殿受朝賀。朝正外籓,依先朝例,賚予銀幣。丁酉,宣詔罪狀皇九弟胤禟。戊戌,集廷臣宣詔罪狀皇八弟胤禩,易親王為民王,褫黃帶,絕屬籍,革其婦烏雅氏福晉,逐回母家,復革民王,拘禁宗人府,敕令易名名曰阿其那,名其子弘旺曰菩薩保。甲寅,削隆科多職,仍令赴鄂羅斯議界。乙卯,贈故尚書顧八代太傅,謚文端,上之授讀師也。

二月甲子,以孫柱為吏部尚書,兼管兵部。以法海為兵部尚書,福敏為左都御史。貝子魯賓、鎮國公永謙俱以議胤禩獄依違削爵,尋起魯賓為輔國公。大學士硃軾有母喪,賜白金四千庀葬事。乙酉,簡親王雅爾江阿削爵,以其弟神保住襲封。庚寅,以張廷玉為大學士,蔣廷錫為戶部尚書,以申穆德為右衛將軍。

三月丁丑,命丁壽屯兵特斯,備策旺阿拉布坦。壬戌,侍講錢名世投詩年羹堯事發,革去職銜,上親書「名教罪人」四字懸其門,並令文臣作為文詩刺惡之。

夏四月己卯,以範時繹為兩江總督。

五月癸巳,禁錮皇十四弟胤禵及其子白起於壽皇殿側,以子白敦為鎮國公。誅鄂倫岱、阿爾松阿於戍所。乙巳,改胤禟名為塞思黑,拘於保定。己酉,命順承郡王錫保食親王俸。封皇十五弟胤烜為貝勒,皇二十弟胤禕為貝子。

六月癸亥,以輔國公巴賽為振武將軍,備邊。乙丑,以查弼納為兵部尚書。

秋七月癸巳,釋回軍前御史陶彞等十三人。辛亥,命蔡珽專管都統。以查弼納、楊名時為吏部尚書。平郡王納爾素有罪削爵,以其子福彭襲封。

八月丙寅,停本年秋決。丁亥,李紱奏塞思黑卒於保定。

九月壬辰,以宜兆熊為湖廣總督,尋命福敏代之。以蔡良為福州將軍。貝子滿都護降為輔國公,撤出佐領。丁酉,輔國公阿布蘭以違例謝恩削爵,撤出佐領。戊戌,重九節,上禦乾清宮,賜宴廷臣,賦柏梁體詩。己亥,錫保奏阿其那卒於禁所。癸丑,起復大學士硃軾在內閣行走。乙卯,侍郎查嗣庭以謗訕下獄。

冬十月甲子,設浙江觀風整俗使。命鄉試五經取中之副榜及兩次取中副榜,準作舉人。戊辰,詔廷臣:「皇考臨御六十餘年,躬節行儉。宮廷地毯用至三四十年,猶然整潔。服御之物,一惟質樸,絕少珍奇。昨檢點舊器,及取回避暑山莊陳設,思慕盛德,實無終已。用特書此,以詔我子孫。」辛巳,裕親王廣寧削爵,永錮宗人府。甲申,以普雄苗地,界連川、滇,命川陜總督移駐成都。以鄂爾泰為雲貴總督,憲德為湖北巡撫。丙戌,琉球國謝賜匾額,貢方物。

十一月己亥,大學士高其位罷。壬子,敘富寧安久戍功,封一等侯。乙卯,詔浙江士習敝壞,工為懷挾,停其鄉會試。

十二月庚申,王大臣請將阿其那、塞思黑妻子正法。諭曰:「阿其那、塞思黑雖大逆不道,而反叛事跡未彰,免其緣坐。塞思黑之妻逐回母家禁錮。其餘眷屬,交內務府養贍。」辛酉,命河南、陜西、四川均攤丁銀入地並徵。乙丑,御史謝濟世疏劾田文鏡十罪,詔褫職遣戍。壬申,鄂爾泰奏剿辦仲苗就撫者二十一寨,查出熟地荒地三萬餘畝。壬午,以李紱為工部右侍郎,以宜兆熊為直隸總督,劉師恕協辦,以毛文銓為京口將軍。丙戌,祫祭太廟。

是歲,免直隸、山東、安徽、江西、湖廣等省六十三州縣衛災賦有差。朝鮮、琉球、蘇祿入貢。

五年丁未春正月戊子朔,時享太廟。壬寅,赦年羹堯之子之戍邊者。甲辰,王大臣奏黃河清,請朝賀,上不許。加文武官一級。敕八旗交納銅器,三年限滿,隱匿者罪之。乙巳,以孫柱署大學士。丙辰,以沈近思為左都御史兼吏部侍郎。

二月丁卯,上謁陵。甲辰,廣州駐防兵丁滋事,將軍李杕以徇庇論死。甲戌,上還京。甲申,上御經筵。丙戌,命李紱往廣西擒捕逸犯羅文綱。文綱自投來歸。

三月庚寅,敕本年會試於三月舉行,給與姜湯木炭。以廣祿襲裕親王。戊戌,上宣示蔡珽罪狀,下刑部拘訊。辛丑,開閩省洋禁。丙午,鄂羅斯察汗遣使臣薩瓦表賀登極,進貢方物,賞賚如例。內大臣馬武卒。大學士高其位卒。

閏三月乙丑,揀選下第舉人,分發直省,以州縣用。戊辰,以宜兆熊為吏部尚書,邁柱為湖廣總督。癸酉,烏蒙、鎮雄兩土府改設流官。己卯,以覺羅伊禮布為奉天將軍,常壽為江寧將軍。丙戌,弘升有罪削爵。

夏四月戊子,吐魯番回酋請進貢,不許,為已撤兵,又以其地許策旺阿拉布坦也。以福敏為吏部尚書,黃國材署兵部尚書。辛卯,賜彭啟豐等二百二十六人進士及第出身有差。癸巳,命州縣會學官舉優行生。乙巳,設宗室御史二員。

五月戊午,以拉錫為滿洲都統。查嗣庭死於獄,戮其尸。乙亥,敘烏蒙、鎮雄功,予鄂爾泰世職。

六月庚子,移盛京副都統一員駐錦州,設熊岳副都統。封誠親王胤祉子弘景為鎮國公。隆科多以罪削爵,以其弟慶復襲一等公。

秋七月乙卯,以富寧安為漢軍都統。己未,李永紹罷,以黃國材為工部尚書。加田文鏡尚書,為河南總督。己巳,以誇岱為工部尚書。丙子,晉封輔國公弘晊、鄂齊、熙良為鎮國公。已革貝勒蘇努塗抹聖祖硃諭,經王、大臣、刑部參奏。得旨:「蘇努怙惡不悛,竟令其子蘇爾金、庫爾陳、烏爾陳信從西洋之教。諭令悛改,伊竟抗稱:『原甘正法,不能改教。』今又查出昔年聖祖硃批奏摺,敢於狂書塗抹,見者發指。即應照大逆律概行正法。但伊子孫多至四十人,悉行正法,則有所不忍。倘分別去留,又何從分別。暫免其死,仍照前禁錮。」

八月己丑,上御經筵。庚寅,賴都罷,以常壽為禮部尚書。癸卯,追封故平南大將軍賴塔為一等公,其孫博爾屯襲。乙巳,喀爾喀郡王額駙策凌與鄂羅斯使臣薩瓦定界,以恰克圖為貿易之所,理籓院派員管理。

九月丙寅,定官員頂戴之制。以孫柱為大學士,查弼納為兵部尚書。己巳,鄂爾泰奏花苗內附,剿辦蒗蕖,平之,威遠惈苗內附。戊寅,刑部議上蔡珽獄,大罪十八,應立斬,妻子入辛者庫。得旨,改監候。

冬十月乙酉,命科道及吏部司官不必專用科目。丁亥,王大臣會審隆科多獄上,大罪五十,應斬立決,妻子入辛者庫,財產入官。得旨,隆科多著禁錮。以博爾屯為蒙古都統。

十一月癸丑,命查郎阿、邁祿備邊。丁巳,加浙江巡撫李衛為總督。丁卯,復鰲拜一等公,令其孫達福襲。敕修執中成憲。戊辰,鄂爾泰奏貴州長寨後路克猛等一百八十四寨生苗內附。乙亥,守護景陵大學士蕭永藻坐失察公銜廣善越分請安,褫職,仍依前守陵。庚辰,遣官清丈四川地畝。順承郡王錫保以徇庇延信奪親王俸,仍停郡王俸三年。

十二月壬午朔,以那蘇圖為黑龍江將軍。乙酉,命直省學政每六年拔取生員一次。王大臣審擬貝勒延信大罪二十,應斬決。得旨,延信免死,與隆科多一處監禁。辛丑,範時繹奏太倉州屬之七浦士民原自行修濬。上不許,曰:「民間之生計,即國計也。國用不敷之時,不得不藉資民力。方今國用充裕,仍發帑銀給之。」戊戌,左都御史沈近思卒。壬寅,以唐執玉為左都御史。庚戌,祫祭太廟。

是歲,免直隸、江蘇、江西、浙江、福建、湖廣、廣東等省三十四州縣災賦有差。朝鮮、鄂羅斯入貢。

六年戊申春正月己未,高其倬疏陳閩省械★情形。得旨:「此等處須鼓舞屬員實心盡力,方能有濟。設遇一二有為者,甫欲整理,輒目為多事。屬員窺見其隱,誰肯任怨向前。須知其難而終任之,二三年後始有成效也。」乙丑,晉封貝勒球琳為惠郡王,鎮國公弘春為貝子。己卯,命杭奕祿、任蘭枝使安南。

二月丙戌,晉封果郡王胤禮為親王。癸巳,上御經筵。庚子,以來文為江寧將軍。壬寅,賜歸流永順土司彭肇槐世職,並白金萬兩。庚戌,以嵇曾筠為兵部尚書,仍辦河工。

三月丁巳,大學士田從典罷,以蔣廷錫為大學士。庚午,以進藏官兵駐劄西寧,命巡撫杭奕祿督之。

夏四月甲申,以陳泰為滿洲都統。予告大學士田從典卒。癸卯,以查郎阿、稽曾筠為吏部尚書。壬寅,詔:「地方官私徵耗羨,難以裁革。惟在督撫審慎用之,不可以歸公。若歸公,則地方官又重衣復取民矣。」

五月癸丑,以郭沭為廣西巡撫。鄂爾泰奏剿辦東川逆苗祿天祐、祿世豪,平之,壬戌,詔:「八法內年老一條,義有未盡。凡年老而能辦事者,勿入八法。」丁卯,削富寧安侯爵,仍為大學士。命馬爾賽在大學士內辦事。乙亥,以田文鏡為河東總督,兼轄山東。以耿化祚為漢軍都統。

六月庚辰,詔六部員外郎、主事作為公缺,勿庸按旗升轉。癸未,置先賢仲弓後裔五經博士。丙戌,以蔡良為廣州將軍,石禮哈為福州將軍,尹繼善協辦江南河工。癸巳,以張廣泗為貴州巡撫,岳濬署山東巡撫。己亥,誠親王胤祉有罪降郡王,拘其子弘晟於宗人府。封理密親王子弘曣為輔國公。

秋七月辛亥,命李衛兼理江蘇緝捕。戊午,鄂爾泰奏遣兵剿平川境米貼逆苗。命以其事屬四川提督黃廷桂。辛酉,岳鍾琪奏頗羅鼐兵至西藏,喇嘛擒獻阿爾布巴、隆布奈、扎爾鼐等,西藏平。戊辰,以紀成斌為固原提督。壬申,大學士富寧安卒。賜故大學士寧完我三世孫寧蘭驍騎校,房一所,銀五百,四世孫寧邦璽拜唐阿。

八月甲申,上御經筵。以尹繼善署江蘇巡撫。乙酉,改湖廣桑植、保靖二土司為流官。以馬爾賽為大學士。甲午,以祖秉衡為京口將軍。丁未,詔復浙江鄉會試。

九月癸丑,命八旗勛舊子孫有犯法虧帑者,察實以聞。漢員中陣亡盡節及居官清正之子孫,同此察報。天津水師營都統公鄂齊以失察兵丁傷官削爵,降三等侍衛。丁卯,查郎阿奏領兵至藏,會同副都統馬喇、學士僧格訊明逆首阿爾布巴等,立時正法,餘眾處置訖。

冬十月丁亥,以鄂爾泰剿平廣西八達寨逆苗,兼督云、貴、廣西三省,發帑銀十萬犒滇、黔兵。辛卯,發內帑九十四萬代西征軍士賠償追款。以石文焯為禮部尚書,路振揚為兵部尚書。乙未,岳鍾琪奏建昌喇汝窩番賊作亂,討平之。詔:「湖廣土司甚多,供職輸將,與流官無異,該督撫勿得輕議改流。」以蔡仕舢為浙江觀風整俗使。癸巳,諭停諸王管理旗務。

十一月丙辰,設咸安宮官學,包衣子弟肄業。庚申,停本年決囚。戊辰,江西巡撫布蘭泰以不職免。添設欽天監西洋人監副一。

十二月甲午,免四川崇慶州七年額賦。丙申,大清律集解附例成。丁酉,以定藏功封頗羅鼐為貝子,理後藏事,揀選噶隆二人理前藏事,賞其兵丁銀三萬兩。庚子,命侍郎王璣、彭維新往江南清查逋賦。甲辰,祫祭太廟。

是歲,免直隸、江南、陜西、四川等省二十六州縣災賦有差。朝鮮入貢。

七年己酉春正月辛亥,鄂爾泰奏萬壽節日,雲南慶雲見。命宣付史館。丁巳,命陳元龍、尹泰為大學士。壬申,復蒙古恩格德爾侯爵為三等公,以其曾孫噶爾薩襲。蒙古二等伯明安晉封一等侯,令其孫馬蘭泰襲。都統伯四格有罪監禁,上念其祖莽固爾岱之功,釋之。癸酉,命侍郎法保等察修直隸至江南大道。

二月丁丑,命出征官兵行糧外仍給坐糧。以尹繼善為河道總督。戊寅,以多索禮為奉天將軍。甲申,上謁陵。庚寅,還京。設直隸巡農御史。己亥,命怡親王等查八旗世職有以絕嗣除爵者,許以族人紹封。乙未,上御經筵。以李杕為漢軍都統。蠲浙江本年額賦六十萬兩。

三月乙巳朔,以孔毓珂為江南河道總督,郝玉麟為廣東總督。岳鍾琪奏剿平雷波叛苗一百餘寨。戊申,鄂爾泰奏剿平丹江、九股等處生苗。蠲河南本年額賦四十萬兩。辛亥,以嵇曾筠為河南山東總督。丙申,上以準噶爾噶爾丹策零稔惡藏奸,終為邊患,命傅爾丹為靖邊大將軍,北路出師,岳鍾琪為寧遠大將軍,西路出師,征討準噶爾。甲子,以鄂善、莽鵠立俱為蒙古都統。辛酉,詔公巴賽為副將軍,順承郡王錫保為振武將軍,陳泰、袞泰、石禮哈、岱豪、達福、海蘭為參贊,旗兵六千,三省兵八千,蒙古兵八百,歸北路,駐扎阿爾泰;總兵官魏麟、閃文繡領車騎營兵八千,赴西路布爾庫。

夏四月甲午,以查郎阿署川陜總督,史貽直署福建總督。敕建雲、雨、風、雷壇廟。四川天全土司改流設州。高其倬劾海澄公黃應纘行賄承襲,應革職銜。詔寬免之。

五月戊午,湖南保靖、桑植、永順三土司改流設府縣。甲子,令漕船順帶商貨,於舊例六十石外,許至百石。乙丑,先是,岳鍾琪疏言有湖南人張熙投遞逆書,訊由其師曾靜所使。命提曾靜、張熙至京。九卿會訊,曾靜供因讀已故呂留良所著書,陷溺狂悖。至是,明詔斥責呂留良,並令中外臣工議罪。

六月己卯,以唐執玉署直隸總督。乙酉,以甘肅、四川、雲南、貴州、廣西轉輸勞費,免庚戌全年額賦,陜西免十分之三。

秋七月丙午,貴州都勻生苗及儂、仲生苗內附。甲寅,以果親王胤禮管工部,莊親王胤祿管滿洲都統。己巳,減暹羅國貢賦。

閏七月乙酉,以阿里袞為杭州將軍。

八月癸卯,以王釴為京口將軍。己酉,上御經筵。

九月戊子,改廣西鎮安為流。

冬十月庚戌,賜漢大臣子蔣溥等十三人舉人。甲子,詔曰:「江南清查逋賦一案,歷降諭旨甚明,重在分別官侵民欠。乃派往之員辦理不善,有以紳衿帶徵之項指為官侵者,有吏書侵蝕之項議令富戶攤賠者。又有將帶徵錢糧加增火耗者,甚且以停徵之項概令徵收者。惠民之政,轉而擾民,豈非司其事者之咎乎?其恪遵前旨妥辦。倘再犯諸弊,從重治罪。」戊辰,以內外諸臣勤慎奉職,加怡親王儀仗一倍,張廷玉少保,蔣廷錫太子太傅,勵廷儀太子少傅,傅爾丹、岳鍾琪、鄂爾泰俱少保,田文鏡太子太保,李衛、查郎阿、席伯俱太子少保。

十一月甲戌,發帑金百萬兩修高家堰石工。以馬會伯為兵部尚書,仍留軍前。戊寅,免功臣子孫施世驊等贓銀五十餘萬,以內庫銀撥補,其應得遣戍、監追、籍沒及妻子入官等罪,咸赦除之。戊子,停本年決囚。

十二月戊申,設廣東觀風整俗使及肇高學政。戊辰,祫祭太廟。

是歲,免江南、江西、浙江、福建、湖南、雲南、甘肅等省二十四州縣災賦有差。朝鮮、琉球入貢。

八年庚戌春正月丁丑,以總理陵寢事務領侍衛內大臣尚崇廙為盛京五部尚書。以那蘇圖為奉天將軍,常德為寧古塔將軍,卓爾海為黑龍江將軍。以慶復為漢軍都統。甲午,景陵瑞芝生。丁酉,唐執玉奏正月二十日鳳凰見於房山。得旨:「此事已據府尹孫家淦奏報。又據尚崇廙報稱天臺山中見一神鳥,高五六尺,毛羽如錦,群鳥環繞,向北飛去。朕躬德薄,未足致此上瑞。」發國子監膏火銀六千兩,歲以為常。

二月庚子朔,定外戚錫爵曰承恩公。甲辰,上御經筵。己酉,復賴士公爵。丁巳,復誠郡王胤祉為誠親王,貝勒胤烜為愉郡王,貝子胤禕為貝勒,皇二十一弟胤禧、皇二十二弟胤祜為貝子,皇二十三弟胤祁為鎮國公。戊辰,南掌國遣使來貢,請定貢期。上優詔答之,命五年一貢。

三月丁亥,命張廷玉、蔣廷錫管理三庫事務。甲午,以史貽直署兩江總督,頒行聖祖御纂書經傳說,上制序文。

夏四月,淳親王胤祐薨,謚曰度,以子弘暻襲郡王。癸卯,賜周澍等三百九十九人進士及第出身有差。丁未,定大學士為正一品,左都御史為從一品。癸亥,以嵇曾筠署江南河道總督,田文鏡兼理東河總督。

五月辛未,怡親王胤祥薨,上痛悼之,親臨其喪,謚曰賢,配享太廟。丁丑,噶爾丹策零遣使通問。命暫緩師期,召傅爾丹、岳鍾琪來京。移高其倬為兩江總督,劉世明為福建總督。壬午,上再臨怡賢親王喪。詔曰:「朕諸兄弟之名,皆皇考所賜。即位之初,胤祉援例陳請更改上一字,奏明母後,勉強行之。今怡親王薨逝,王名仍書原字,志朕思念。」辛卯,先是,誠親王胤祉會怡賢親王之喪,遲到早散,面無戚容,交宗人府議處。至是,議上,請削爵正法。得旨,削爵拘禁。癸巳,以岳超龍為湖廣提督。乙未,晉封貝子胤禧為貝勒,理郡王弘晰為親王,公弘景為貝子。復胤祹郡王。

六月戊戌朔,日有食之。壬寅,賜怡賢親王「忠敬誠直勤慎廉明」八字加於謚上。戊申,鄂爾泰奏黎平、都勻生苗內附。癸亥,馬會伯免,以唐執玉為兵部尚書,史貽直為左都御史。

秋七月戊寅,命建賢良祠。壬辰,遣官賑江南、湖南、直隸、山東等處被水災民。癸巳,命巡撫班次在副都統之上。

八月丙午,以山東被水較重,特免通省漕糧。辛亥,命怡賢親王子弘曉襲封親王,弘★別封郡王,均世襲。乙卯,京師地震。康親王崇安停管宗人府事,以裕親王廣祿管宗人府。

九月丁卯,以京師地震,賜百官半俸,賜八旗銀各三萬兩。乙酉,以高其倬相視太平峪吉地,予世職。辛卯,鄂爾泰奏猛弄白氏、孟連、怒子內附。

冬十月庚子,再定百官帽頂,一品官珊瑚頂,二品官起花珊瑚頂,三品官藍色明玻璃頂,四品官青金石頂,五品官水晶頂,六品官硨磲頂,七品官素金頂,八品官起花金頂,九品、未入流起花銀頂。辛亥,命查弼納為副將軍,往北路軍營。壬子,鄂爾泰奏恢復烏蒙府城,苗黨平。甲寅,以馬爾賽、張廷玉、蔣廷錫久參機務,各予伯爵世襲。闕里文廟成,命皇五子弘晝、淳郡王弘暻前往告祭。

十一月己巳,設孔廟執事官。乙亥,命各省落地稅、契稅勿苛索求盈。丙子,明詔申飭漢軍勛裔獲咎大員範時繹、尚崇廙、李永升等。戊子,敕各省解部銀兩,留其半以充公用。

十二月丁酉,命傅爾丹、岳鍾琪各回本軍。乙卯,紀成斌奏準噶爾賊眾犯闊舍圖卡倫,總兵樊廷擊敗之。予樊廷世職,銀一萬兩。其張朝佐等並予世職,賞銀有差。

是歲,免直隸、江南、山西、湖南、貴州等省十八州縣衛災賦。又免直隸、江南、山東、河南漕糧各有差。朝鮮、安南、南掌入貢。

九年辛亥春正月庚寅,詔撥揚州鹽義倉積穀二十萬石,加賑上年邳、宿被水災民。

二月乙未,愉郡王胤烜薨,謚曰恪,子弘慶襲郡王。撥通倉米十五萬石,奉天米二十萬石,採買米五萬石,運往山東備賑。戊戌,命常賚為鎮安將軍,率甘、涼兵駐安西。戊午,以田文鏡年老多病,命侍郎王國棟前往河南賑濟被水災民。壬戌,專設四川總督,以黃廷桂補授。

三月乙酉,以三泰為禮部尚書,鄂爾奇為左都御史。戊子,命揀選八旗家人二千,以伊禮布統之,為西路副將軍。

夏四月庚子,命史貽直、杭奕祿前往陜西宣諭化導。丙辰,鄂彌達奏瓊山、儋州生黎內附。

五月甲子,以石云倬為西路副將軍。命趙之垣、馬龍督運西路糧饟。

六月丙午,傅爾丹奏準噶爾入寇扎克賽河,率兵迎擊。辛亥,岳鍾琪奏準噶爾犯吐魯番,率兵赴援,賊遁,留兵屯戍。甲寅,上祈雨,是日,雨。

秋七月丁卯,召鄂爾泰來京。以高其倬為雲貴總督,尹繼善為兩江總督。己巳,黃廷桂奏瞻對番賊作亂,遣兵剿平之。癸酉,傅爾丹奏官兵進擊準噶爾不利,退至科布多。是役也,輕進中伏,傅爾丹棄大軍先退,至於大敗。副將軍查弼納、公巴賽、參贊公達福等均死之。甲戌,命馬爾賽為撫遠大將軍,敕錫保固守察罕瘦爾。岳鍾琪奏督兵進烏魯木齊。

八月己亥,以鄂彌達為青州將軍。丙午,移科布多兵駐察罕瘦爾。己酉,晉封錫保為順承親王。甲寅,岳鍾琪奏兵至納鄰河,距烏魯木齊二日程,探知賊遁,大兵即旋。命從優議敘。

九月乙亥,命康親王崇安前往軍營,給備裝銀萬兩。戊子,以劉於義為直隸總督,沈廷玉為直隸河道總督,硃藻為河東河道總督。己巳,皇後那拉氏崩,冊謚曰孝敬皇后。

冬十月丙午,錢以塏乞休,以魏廷珍為禮部尚書。準噶爾入寇克魯倫,侵掠游牧,親王丹津多爾濟、額駙郡王策凌合兵擊之,擒斬無算。上嘉之,各賜銀萬兩,晉策凌為親王。

十一月癸亥,命順承親王錫保為靖邊大將軍,降傅爾丹為振武將軍,降馬爾賽為綏遠將軍。命康親王崇安攝撫遠大將軍。乙丑,以史貽直為兵部尚書,彭維新為左都御史。

十二月庚寅朔,日有食之。己酉,聖祖實錄、聖訓告成。甲寅,以馬士傑署廣州將軍,準泰署福州將軍。丁巳,祫祭太廟。

是歲,免直隸、江南、河南、福建、陜西、湖南、廣西、甘肅等省九十三州縣衛災賦有差。朝鮮、琉球入貢。

十年壬寅春正月癸亥,孟春享太廟,皇四子弘歷行禮。壬午,命鄂爾泰為大學士。甲申,以軍前統領達爾濟為建勛將軍,駐兵白格爾。

二月,以王朝恩為直隸河道總督,魏廷珍為漕運總督。己亥,封鄂爾泰一等伯,世襲。庚子,岳鍾琪奏準噶爾犯哈密,遣總兵曹勷往援,敗之,賊由無克克嶺遁。副將軍石云倬坐不遮擊,逮問。癸丑,以張廣泗為西路副將軍,劉世明參軍事。

三月丁丑,大學士等疏劾岳鍾琪奏報不實,情詞互異。下部嚴議。

夏四月辛卯,置貴州古州鎮、清江鎮總兵各一員。乙巳,以海壽為戶部尚書,性桂為刑部尚書。降三等公嶽鍾琪為三等侯,仍護大將軍。丙午,以張大有為禮部尚書,範時繹為工部尚書。乙卯,詔修雲南嵩明州、尋甸州水利。

五月戊辰,以武格為揚武將軍,劉世明副之。

閏五月甲辰,恆親王胤祺薨,謚曰溫,子弘晊襲恆親王。原誠親王胤祉卒於景山禁所,賜銀五千兩,照郡王例殯葬。吏部尚書勵廷儀卒。庚戌,臺灣北路西番滋事,官兵討平之。癸丑,以李衛署刑部尚書。

六月丙辰,以莽鵠立為漢軍都統。壬申,高其倬奏雲南思茅土夷勾結元江夷人寇普洱郡城,遣總兵董芳率兵剿之。辛巳,辦理軍機大臣議奏恤贈戰歿喀爾喀臺吉策勒克輔國公,其子密什克襲。軍機大臣之設始於此。

秋七月丙戌,馬喇免,以武格為工部尚書。丁亥,山東鉅野牛產瑞麟。己丑,賜顧八代子孫銀一萬兩。丁酉,命鄂爾泰經略軍務。召嶽鍾琪來京。以劉於義為陜西總督,李衛為直隸總督。辛丑,準噶爾入犯烏孫珠爾,傅爾丹迎擊失利,下大將軍錫保覈敗狀以聞。乙巳,大學士蔣廷錫卒。己酉,以福敏協理大學士,唐執玉兼理刑部尚書。

八月丙辰,復恭親王之子海善貝勒原銜。庚午,西藏邊外巴爾布國雅木布、葉楞、庫庫穆三汗遣使進貢,優敕答之。壬申,北路副將軍親王丹津多爾濟、額駙親王策凌奏追擊準夷至額爾得尼招,殺賊萬餘,賊向推河遁去。甲申,撥帑銀二百萬兩解赴北路軍前備賞。

九月乙酉朔,論擊準夷功,加丹津多爾濟智勇名號,加策凌超勇名號,封其子車布登扎布為輔國公,餘升授有差。以馬爾賽縱賊失機,褫爵職處斬。己酉,削傅爾丹爵職。

冬十月壬戌,停本年決囚。削岳鍾琪爵職,逮京交兵部拘禁。

十一月丙戌,以常德為靖邊左副將軍。乙未,封吐魯番額敏和卓為輔國公。賜七世同居湖南沅江縣生員譙衿御書匾額。

十二月乙卯,賜恤北路陣亡諸臣查弼納、馬爾薩、海蘭、達福等有差。侍郎孫嘉淦有罪論死,命在銀庫處行走。乙丑,治呂留良罪,與呂葆中、嚴鴻逵俱戮尸,斬呂毅中、沈在寬,其孫發邊遠為奴,硃羽採等釋放。丙寅,武格以造言撤兵,逮問。辛巳,祫祭太廟。

是歲,免直隸、江南、山東、湖南等省七十五州縣災賦有差。丁戶二千五百四十一萬二千二百八十九,永不加賦後滋生人丁九十三萬六千四百八十六。田地八十九萬四百十六頃四十畝,徵銀二千九百八十七萬二千三百三十二兩六錢。茶三十四萬二千三百五十一引。鹽課銀三百九十八萬八千八百五十一兩。鑄錢六萬八千四百三十六萬二千有奇。朝鮮、巴爾布國入貢。

十一年癸丑春正月戊子,命海望、李衛察勘浙江海塘。修範公堤。壬辰,頒直省書院膏火銀各千兩。以高其倬為兩江總督,尹繼善為雲貴總督。庚子,命鄂爾泰巡閱北路軍務。丁未,上謁陵。

二月壬子,上見沿道安設水缸,蓄水灑道。上諭之曰:「蹕路所經,雖有微塵何礙。地方官當以牧養生民為重。若移奉上之心以撫百姓,豈不善乎?」癸丑,上還京。丙辰,以保明、查爾泰、伊勒慎俱為滿洲都統。己未,上御經筵。封皇二十四弟胤祕為諴親王,皇四子弘歷為寶親王,皇五子弘晝為和親王。貝勒弘春晉封泰郡王。壬戌,命彭維新協辦內閣。以吳士玉為禮部尚書,塗天相為左都御史。

夏四月壬子,特賜任啟運翰林,在阿哥書房行走。癸丑,賜陳倓等三百二十八人進士及第出身有差。乙卯,以嵇曾筠為大學士,仍管河督。以劉於義為吏部尚書,塗天相為刑部尚書,張照為左都御史。己未,徵舉博學鴻詞。

五月甲申,高其倬奏普思苗人刁興國叛,討平之。命編修張若靄,庶吉士鄂容安、鄂倫俱在辦理軍機處行走。乙未,命額駙策凌為靖邊左副將軍,常德副之,塔爾岱為靖邊右副將軍,永福副之,同戍科布多。續修會典成。壬寅,黑龍江將軍杜賚奏海島特門、奇圖山等處綽敏六姓內附,歲貢貂皮。己酉,誅前提督紀成斌。

六月戊午,蘇祿國王臣毋漢未毋拉律林奏伊遠祖東王於明永樂年間來朝,歸至山東德州病歿。長子歸國嗣王,次子安都祿,三子溫哈喇留守墳墓。其子孫分為安、溫二姓,歲領額設祭祀銀八兩,請以其後裔為奉祀生。從之。戊寅,哈元生奏討平九股逆苗。

秋七月乙酉,大學士陳元龍以年逾八旬乞休,加太子太傅致仕。李徽以越職言事褫職。裁湖南觀風整俗使。戊子,順承親王錫保削爵,子熙良仍襲郡王。以平郡王福彭為定邊大將軍。降親王丹津多爾濟為郡王,撤去勇號。

八月丁卯,以顧琮為直隸河道總督,趙弘恩為兩江總督,高其倬為江蘇巡撫。己巳,置順天府四路捕盜同知。

九月辛丑,鄂爾奇革職查訊。以慶復為戶部尚書,鄂長署步軍統領。

冬十月辛酉,以扣婁為蒙古都統,忠達公馬禮善為刑部尚書。

十一月甲辰,命果毅公訥親在辦理軍機處行走。

十二月戊午,詔曰:「前鄂彌達條奏臺灣建城。郝玉麟奏稱臺灣茨竹,栽植可以成城。臺灣變亂,率自內生。賊匪無城可踞,乃易蕩平。惟鹿耳門為臺郡門戶,於此建築砲臺,足資備御。栽植茨竹,相為籓籬。其淡水等處砲臺,並應建造,以時增修。」己未,以史貽直為戶部尚書,張照為刑部尚書,徐本為左都御史。壬戌,以高斌為江南河道總督。丙子,祫祭太廟。

是歲,免直隸、江蘇、安徽、江西、山東等省二十九州縣衛災賦,又免江蘇鹽場二十五引鹽課各有差。朝鮮、安南、蘇祿入貢。

十二年甲寅春正月辛丑,平郡王福彭進馬五百匹,解軍備用。壬寅,侍郎查克旦辦理車臣汗部落諸務得宜,加尚書銜,賜銀五千兩,入官房地人口給還。

二月癸丑,上御經筵。己未,晉封貝子胤祜為貝勒。乙丑,命侍讀春山、給事中李學裕冊封安南國王。壬申,命額駙策凌總理前敵軍務。癸酉,元展成奏坡東、坡西苗寨一百六十內附。旌廣東興寧縣老民幸登運年一百二歲,其子五人,各七八十歲,一門眉壽,加賜上用緞一匹。

三月丁丑,工部尚書範時繹免。戊戌,河南學政俞鴻圖以婪贓處斬,其父侍郎俞兆晟褫職。尹繼善奏剿平普思叛苗,招撫投誠人眾。得旨:「凡事懈於垂成,忽於既定。勉之。」

夏四月丁未,湖廣容美土司田民如有罪革退,改土歸流。康親王崇安薨,以伊叔巴爾圖襲爵,封其子永恩為貝勒。庚午,禁廣東象牙席,並禁民間購用。

五月己卯,施南宣撫司改設流官。癸巳,以李禧為漢軍都統。乙未,以準噶爾使來,停止進兵。己亥,命內務府總管來保前赴車臣汗部,協同查克旦辦事。

六月丁未,湖廣忠★等十五土司改設流官。

秋七月癸巳,命果親王胤禮經理達賴喇嘛駐藏,並至直隸、山西、陜西、四川閱兵。詔西北二路用兵年久,或乘此兵力直進賊境,或遣使往彼諭以利害,廷臣集議以聞。康親王巴爾圖等一議進兵,大學士張廷玉等一議遣使。上乃宣示用兵始末,從後議遣使。

八月丙午,遣傅鼐、阿克敦往準噶爾宣諭。壬戌,降貝子胤禕為公,泰郡王弘春降為貝子。

九月甲申,命侍郎呂耀曾、卿德福往貴州宣諭苗蠻。命雲南開爐鼓鑄。

冬十月丙午,果親王胤禮疏言:「臣工條奏,宜據實敷陳,不當摭拾塞責。」得旨:「所言甚是,曉諭輪班條奏官知之。」丁未,以鄂彌達署天津都統,阿里袞為青州將軍,傅森為杭州將軍。戊午,以郝玉麟為浙閩總督。以三泰、徐本俱協辦內閣事。己巳,景陵瑞芝生。

十一月壬申朔,前直郡王胤禔卒,命照貝子治喪,封其子弘昉為鎮國公。丙寅,敕續修皇清文穎。壬午,特詔福建漳、泉二府,化其強悍,勿再聚族械★C1。戊子,封理密親王子弘晁為輔國公。

十二月癸丑朔,敕廣西仍歸廣東總督兼轄。丁巳,以魏廷珍為兵部尚書,顧琮為漕運總督,硃藻為直隸河道總督,白鍾山為河東河道總督,高斌為江南河道總督。庚午,祫祭太廟。

是歲,免直隸、安徽等省十四州縣災賦,又直隸鹽場十四引鹽課各有差。朝鮮、琉球入貢。

十三年乙卯春正月己丑,以覺羅柏修為盛京將軍,那蘇圖為黑龍江將軍,赫星為寧夏將軍。

二月己酉,上御經筵。庚戌,以魏廷珍為禮部尚書。癸丑,上謁陵。己未,還京。甲子,以巴泰協辦大學士。

三月丁巳,上親耕耤田。戊子,詔曰:「地方編立保甲,必須俯順輿情,徐為勸導。若過於嚴急,則善良受累矣。為政以得人為要,不得其人,雖良法美意,徒美觀聽,於民無濟也。」

夏四月乙巳,聖祖文集刊成,頒賜廷臣。丁巳,停止廣東開採。

閏四月丁酉,準噶爾遣使臣納木喀賚表進貢。敕令定界。己亥,建先蠶壇於北郊。

五月戊申,給三姓八旗兵丁餉銀。丁巳,以貴州古州、臺拱逆苗滋事,命哈元生為揚威將軍,統領四省官兵討之。甲子,命果親王、皇四子、皇五子,大學士鄂爾泰、張廷玉等辦苗疆事務。工部尚書巴泰褫職。命刑部尚書張照、副都御史德希壽稽勘苗疆事務。丁卯,哈元生奏剿辦逆苗,黃平、施秉悉平。

六月乙亥,敕戶部清查各省耗羨。癸未,以查克旦為工部尚書。甲申,準土司由生員出身者一體應試。辛卯,減各省進獻方物。呂宋國饑,請糴。許之。丙申,命董芳為副將軍,協剿苗匪。

秋七月乙卯,鄂爾泰請辭伯爵、大學士。許之,給假養病,仍食俸。署甘州提督劉世明以失察兵丁搶劫論斬。丙辰,命硃軾往勘浙江海塘。辛酉,以邁柱、查郎阿為大學士,張廣泗為湖廣總督。

八月己巳,詔曰:「從前經理苗疆,本為乂安民生。乃經理不善,以致逆苗肆出,勾結熟苗,搶劫居民。是以安民之心,成虐民之政。返之初心,能勿愧乎?所有貴州本年錢糧,通行蠲免。其被賊州縣,蠲免三年,以示撫綏捄恤之意。」

丁亥,上不豫。戊子,上大漸,宣旨傳位皇四子寶親王弘歷。己丑,上崩,年五十八。是歲十一月丁未,恭上尊謚曰敬天昌運建中表正文武英明寬仁信毅睿聖大孝至誠憲皇帝,廟號世宗。乾隆二年三月,葬泰陵。

論曰:聖祖政尚寬仁,世宗以嚴明繼之。論者比於漢之文、景。獨孔懷之誼,疑於未篤。然淮南暴伉,有自取之咎,不盡出於文帝之寡恩也。帝研求治道,尤患下吏之疲困。有近臣言州縣所入多,宜釐剔。斥之曰:「爾未為州縣,惡知州縣之難?」至哉言乎,可謂知政要矣!


\end{pinyinscope}