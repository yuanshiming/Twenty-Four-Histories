\article{志九十二}

\begin{pinyinscope}
○職官四武職籓部土司各官

公侯伯子男額駙侍衛處鑾輿衛驍騎營八旗都統前鋒營護軍

營統領景運門直班八旗內務府三旗護軍營總統三旗包衣驍騎營三旗包衣護軍營

步軍統領火器健銳神機虎槍諸營鄉導處上虞備用處善撲營王公府屬各

官公主府長史陵寢駐防各官各省駐防將軍等官提督等官各處駐

劄大臣回部各官籓屬各官土司各官番部僧官

公、侯、伯、超品。子、正一品。男、正二品。輕車都尉、正三品。以上俱分三等。騎都尉、正四品。雲騎尉、正五品。恩騎尉,正七品。凡九等,以封功臣及外戚。

初,天命五年,論功序列五爵,分總兵為三等,副將、參將、游擊亦如之,牛錄額真稱備御。天聰八年,始設一等公,即五備御之總兵。及一、二、三等昂邦章京,即總兵。梅勒章京,即副將。扎蘭章京,一、二等即參將,三等游擊。牛錄章京。即備御。順治元年,加封功臣公、侯、伯世爵,錫之誥券。時公、侯、伯下無子、男,副、參即其爵也。四年,改昂邦章京為精奇尼哈番,梅勒章京為阿思哈尼哈番,扎蘭章京為阿達哈哈番,牛錄章京為拜他喇布勒哈番。授爵自拖沙喇哈番始,舊為半個前程,漢稱外所千總,正五品。遞上為拜他喇布勒哈番,漢稱外衛指揮副僉事,從四品。再一拖沙喇哈番,稱外衛指揮僉事,正四品。阿達哈哈番,三等稱外衛副同知,二等稱外衛指揮同知,俱從三品。一等稱外衛指揮副使,再一拖沙喇哈番,稱外衛指揮使,正三品。阿思哈尼哈番,三等稱外衛都指揮副同知,二等稱外衛都指揮同知,俱從二品。一等稱外衛都指揮副使,再一拖沙喇哈番,稱外衛都指揮使,俱正二品。精奇尼哈番。二等稱鑾儀衛都指揮同知,從一品。一等稱鑾儀衛都指揮使,正一品。積拖沙喇哈番二十六,為一等公。八年,定世襲罔替制。十八年,定合並承襲制。

康熙元年,以世爵合並至公、侯、伯者,仍與分襲。雍正二年,錫明裔硃之璉一等侯。乾隆十四年,錫名延恩。八年,嘉大學士張廷玉等輔弼勤勞,賜一等阿達哈哈番世襲,漢世職自此始。明年,錫公爵嘉名。如褒績、忠達類。外戚命為承恩公。往制為一等公。乾隆四十三年改三等。

乾隆元年,定精奇尼哈番漢字為子,阿思哈尼哈番為男,阿達哈哈番為輕車都尉,拜他喇布勒哈番為騎都尉,拖沙喇哈番為雲騎尉,滿文如故。十三年,定公、侯、伯以次封爵表。一等公襲二十六次,一等侯兼一雲騎尉襲二十三次,一等伯兼一雲騎尉十九次,一等男兼一雲騎尉十一次,自公至男,一、二、三等依次遞降。十四年,追錫侯、伯嘉名。如奉義侯、敦惠伯類。自是垂為永制。十六年,定世襲七品官為恩騎尉,是為九等。三十二年,嘉黃芳度功,予襲公爵十二世,並依八旗例,復給恩騎尉,優恤於無窮。時將軍張勇等,提督孫思克等,並緣此推恩,繇是漢官亦有世襲罔替例。同治中興,剖符析圭者,漢官為多,猶古武功爵也。光緒三十三年,制定創興大業者予子、男,號曰商爵,則頒爵之制少異已。

公主額駙,位在侯、伯上。尚固倫公主中宮所生女。曰固倫額駙,秩視固山貝子;尚和碩公主妃所生女及中宮撫養者。曰和碩額駙,秩視超品公。親王女曰郡主,額駙秩視武職一品。世子、郡王女曰縣主,額駙視二品。貝勒女曰郡君,額駙視三品。貝子女曰縣君,額駙視四品。入八分鎮國公、輔國公女曰鄉君,額駙視五品。近支格格予歲祿,遠支止予虛銜。下嫁蒙籓亦如之。所生之子,各予其父品級。

初,太祖時,額駙何和禮授都統,達爾漢繼之。太宗時,巴雅思祜朗授都統,拉哈繼之。自是御前侍衛大臣、護軍前鋒統領,皆為專職。亦有僅受歲祿,而護從隨征受命一充其任者。至出鎮西北,則自定邊左副將軍策凌始。踵其後者,世宗時,觀音保為領隊大臣,高宗時,色布騰巴勒珠爾為參贊大臣。其授文職者,止天命間蘇鼐、乾隆間福隆安二尚書而已。

侍衛處領侍衛內大臣,正一品。內大臣,初制正一品,後改從一品。各六人。鑲黃、正黃、正白旗各二人。散秩大臣、都統、護軍前鋒統領、滿大學士、尚書內特簡。散秩大臣無員限。從二品,食三品俸。主事一人。署主事三人。筆帖式二十有七人。內委署十五人。協理事務侍衛班領,正三品。侍衛班領,正四品。各十有二人。署班領二十有四人。侍衛什長七十有九人,宗室九人。侍衛一等正三品。六十人,旗各二十人。宗室九人。旗各三人。二等正四品。百五十人,旗各五十人。三等正五品。二百七十人,旗各九十人。宗室六十有三人。旗各二十一人。藍翎侍衛九十人。旗各三十人,三旗通為五百七十人。內隸黏竿處三十四人,上駟院二十四人,上虞備用處三十六人。善撲營、武備院無常額。四等侍衛、漢侍衛,分一、二、三等及藍翎。俱無員限。親軍校,正六品。署親軍校,初無品級。乾隆五十一年定從八品。各七十有七人。

領侍衛掌董帥侍衛親軍,偕內大臣、散秩大臣翊衛扈從。協理、主事、筆帖式,分掌章奏文移。侍衛掌營衛周廬,更番侍直。分兩翼宿衛。乾清門、內右門、神武門、寧壽門為內班,太和門為外班。行幸駐蹕如宮禁制。朝會、祭祀出入,則衛官填街,騎士塞路。領侍衛內大臣、侍衛班領,帥豹尾班侍衛。散秩大臣、侍衛什長,執纛親軍以供導從,大閱則按隊環衛。親軍校掌分轄營眾。其常日侍直者,御前大臣、王大臣兼任。御前侍衛、御前行走、乾清門行走,俱侍衛內特簡。無常員。故事,凡宿衛之臣,惟滿員授乾清門侍衛,其重以貴戚或異材,乃擢入御前。漢籍輒除大門上侍衛,領侍衛內大臣轄之。其以材勇擢侍乾清門者,班崇極矣。惟嘉慶問楊芳特授國什哈轄,漢國什哈內大臣,嘆為未有。其出入扈從者,後扈大臣二人,御前大臣、領侍衛內大臣兼任。前引大臣十人。內大臣、散秩大臣、前鋒統領、護軍統領、副都統兼任。所轄奏事處,御前大臣兼管。侍衛一人,御前侍衛、乾清門侍衛內特簡。章京六人,內府司員四人。各部、院司員二人。筆帖式二人,內府筆帖式兼充。奏蒙古事侍衛六人。乾清門或大門侍衛兼充。

初,太祖以八旗禁旅戡定區夏,鑲黃、正黃、正白三旗皆自將,爰遴其子弟,命曰侍衛,亦間及宗室秀彥、外籓侍子,統以勛戚,備環直焉。順治元年,定侍衛處員數。如前所列。時漢廕生亦與選,尋罷。康熙二十九年,擢武進士嫻騎射者為侍衛,附三旗。三十七年,增宗室侍衛,無常員。雍正七年定九十人。雍正三年,選藍翎侍衛材力魁健者置四等。後復如故。明年,定武進士一甲一名授一等侍衛,二、三名授二等,二甲選三等,三甲選藍翎,置滿洲主事一人。乾隆三十六年,以隨印協理事務侍衛班領為一等,侍衛班領為二等。凡十人置一長,三旗什長六十人,宗室九人。四十年,增委署親軍校七十有七人。嘉慶十九年,以散秩大臣無辦事責,諭凡擢都統者停兼職。

鑾輿衛掌衛事大臣一人。正一品。無專員,以滿、蒙王、公、大臣兼授。鑾輿使,初制正二品。康熙二年改正三品,七年復故。滿洲二人,凡滿缺並以蒙古人兼授。漢軍一人。凡漢軍缺並以漢人兼授。其屬:堂主事,滿洲一人。經歷經歷,漢一人。筆帖式,滿洲七人,漢軍三人。又六所、一衛:曰左所,曰右所,曰中所,曰前所,曰後所,日馴象所,曰旗手衛。冠軍使,初制正三品。康熙二年改正四品,七年復故。宗室一人,滿洲、漢軍七人。雲麾使,初制正四品。康熙二年改正五品,七年復故。宗室二人,滿洲、漢軍十有八人。治宜正,初制正五品。康熙二年改正六品,七年復故。宗室三人,滿洲、漢軍二十有九人。整宜尉,初制正六品。康熙二年改從,七年復故。雍正十年升正五品,後復改從六品。宗室三人,滿洲、漢軍二十有三人。鳴贊鞭官,由太常、鴻臚二寺贊禮郎、鳴贊官咨補。滿洲四人,學習二人。

鑾輿使掌供奉乘輿秩序鹵簿,辨其名物與其班列。凡祭祀、朝會、時巡、大閱,帥所司供厥事。左所掌輿乘輦路;右所掌傘蓋、刀戟、弓矢、殳槍;中所掌麾氅、幡幢、纛幟、節鉞、仗馬;前所掌扇壚、瓶盂、杌椅、星拂、御仗、椶薦、靜鞭、品級山;後所掌旗爪、吾仗;馴象所掌儀象、騎駕、鹵簿、前部大樂;旗手衛掌金鉦、鼓角、鐃歌大樂,兼午門司鐘,神武門鐘鼓樓直更。主事掌章奏。經歷掌文移。

其別設者:往制,步輦雲麾使一人,治宜正三人,駕庫管理整宜尉二人,俱漢軍為之。後分金、玉、象、革、木五輅,並拜褥、椶毯、蓖頭、亭座、駕衣諸管理,派冠軍使以次各官兼攝,則參用滿員。

順治元年,設錦衣衛,置指揮等官。明年,更名鑾儀衛,定各官品秩。時共五所,所止存一司。四年,省指揮使,置鑾儀使以次各官。明年,省副官及衛官百十有四人。六年,增攝政王下漢二品鑾儀使,三品冠軍使,四品雲麾使,五品治儀正,各二人;整儀尉三人。後俱省。九年,始以內大臣掌衛事。乾隆九年置兼理衛事一人。十四年省,二十六年復置總理衛事內大臣一人,三十年又省。十一年,定鑾儀使滿、漢各二人。康熙三十一年省漢一人。乾隆五十年分滿使為左、右。五十七年復舊制。陪祀冠軍使,漢二人。康熙二十三年,掌步輦事。三十七年,以一人掌庫事。四十八年俱停。設左、右、中、前、後五所,鑾輿、馴馬、擎蓋、弓矢、旌節、幡幢、扇手、斧鉞、戈戟、班劍十司。設馴象一所,分東、西二司。設旗手一衛,分左、右二司。定冠軍使十人,宗室一人,滿洲七人,漢軍二人。雲麾使二十有二人,宗室二人,滿洲十二人,漢軍八人。閒散六人。滿缺。治儀正二十有四人,宗室四人,漢軍二十人。閒散十有八人。滿缺。整儀尉二十有九人,宗室四人,滿洲十有五人,漢軍十人。十五年,省滿洲經歷一人。康熙十六年,改經歷為漢缺,增置滿洲主事一人。乾隆三十七年,增置鳴贊鞭官四人。嘉慶十三年增學習二人。四十八年,置總辦、協辦、堂務、冠軍使各一人。所、衛冠軍使兼充。嘉慶六年更名綜理七所事務冠軍使,派雲麾使二人協理。光緒三十三年,省冠軍使二人,雲麾使八人,治儀正十人,整儀尉四人。定宗室員限,如前所列。餘並滿、漢參用。宣統元年,避帝諱,改鑾儀使為鑾輿使,治儀正、整儀尉並易「儀」為「宜」。

驍騎營八旗都統,初制正一品,後改從一品。滿、蒙、漢軍旗各一人。副都統,正二品。旗各二人。參領,正三品。副參領,正四品。俱九十有六人。滿洲、漢軍各四十人,蒙古十有六人。佐領,正四品。驍騎校,正六品。俱千一百五十有一人。滿洲各六百八十有一人,蒙古各二百有四人,漢軍各二百六十有六人。協理事務參領四十人。滿洲、漢軍各十有六人,蒙古八人。本旗參領內選充。章京,筆帖式,俱百四十有四人,滿洲各六十有四人,蒙古各三十有二人,漢軍各四十有八人。隨印房行走散秩官無定員。

都統,副都統掌八旗政令,宣布教養,釐詰戎兵,以贊旗務。參領、副參領掌受事、付事以達佐領。佐領掌稽所治戶口田宅兵籍,歲時頒其教戒。協理各官掌章奏文移,計會出納。各營同。其特派者:直年旗大臣八人;其屬有參領,章京,筆帖式。旗員內派委。管理舊營房大臣,滿、蒙各一人;其屬有營總章京,驍騎校。新營房大臣,官房大臣,滿、蒙、漢軍各八人;其屬與舊營房同。左、右翼鐵匠局副都統,其屬有參領,散秩官,驍騎校。稽察寶坻等處駐防大臣,各二人;左、右翼世職官學總理大臣十人;其屬有參領章京,清語、騎射教習。十五善射處管理大臣,翼各一人;漢軍清文義學稽察學務參領八人。其分攝者:俸饟處、馬冊房、管理馬圈、藤牌營參領各官,漢軍鳥槍營領催各官,城門偏吉章京驍騎校,俱於旗員內選充。

初,太祖辛丑年,始編三百人為一牛錄,置一額真。先是出兵校獵,人取一矢,一長領之,稱牛錄,至是遂以名官。天命元年編制滿洲牛錄。八年增編蒙古牛錄。天聰四年,漢軍牛錄成。先分四旗,尋增為八旗。乙卯年,定五牛錄置一扎蘭額真,五扎蘭置一固山額真,左、右梅勒額真佐之。太宗御極,置總管旗務八大臣,主政事;即固山額真兼議政大臣。佐管十六大臣,主理事聽訟。即梅勒額真兼理事大臣。天聰八年,改額真為章京,固山額真如故。管梅勒曰梅勒章京,管扎蘭曰扎蘭章京,管牛錄曰牛錄章京。其隨營馬兵曰阿禮哈超哈。是為驍騎營之始,然猶統滿、蒙、漢軍為一也。九年,始分設蒙古八旗。崇德七年,復分設漢軍八旗。先是二年設二旗,四年分為四。二十四旗之制始備。順治八年,定扎蘭章京漢字稱參領。十七年,定固山額真漢字稱都統,雍正元年改滿文固山額真為固山昂邦。梅勒章京稱副都統,牛錄章京稱佐領,分得撥什庫稱驍騎校,並定都統、副都統員額。如前所列。參領,滿洲、漢軍旗各五人,蒙古各三人。尋各增一人。佐領隨事為員。分四等:部落長率屬歸誠,爰及苗裔,曰勛舊佐領;功在旗常,錫之戶口,曰優異世管佐領;止偕兄弟族眾來歸,授職相承,曰世管佐領;戶口寥落,合編數姓,迭為是官,曰互管佐領。康熙十三年復以各佐領餘夫增編公中佐領。驍騎校如參領數。康熙三十四年,增委署參領,視扎蘭為員限。雍正元年改副驍騎參領,定滿洲、漢軍旗各五人,蒙古各三人。雍正七年,增左、右司掌關防參領及司務等官。旗各二人。俱十三年省。明年,定漢軍上三旗為四十佐領,乾隆三十九年增鑲黃旗一人。四十年又增一人。五十五年又增一人。嘉慶九年省一人。下五旗為三十佐領,乾隆二十一年省正紅、鑲紅旗各二人,鑲藍旗一人。三十九年省正藍旗一人。及滿洲、鑲黃、正白、鑲紅旗各八十六人,鑲白旗、正藍旗各八十四人,正黃旗九十三人,正紅旗七十四人,鑲紅旗八十六人。蒙古正黃、鑲白旗各二十四人,正紅、鑲紅旗各二十二人,鑲黃旗二十八人,正白旗二十九人,正藍旗三十人,鑲藍旗二十五人。員數。乾隆元年,增置印務參領、章京。

前鋒營前鋒統領,正二品。王、公、大臣兼領。左、右翼各一人。自統領以下,俱滿、蒙人為之。護軍、火器、健銳各營同。參領,正三品。侍衛,初制正五品。乾隆元年升正四品。各八人。委署侍衛,給五品頂戴,仍食前鋒校月饟。各四人。前鋒校,正六品。各四十有四人。協理事務參領、侍衛,各一人。本翼參領、侍衛內充補。前鋒校各二人。本翼前鋒校內酌委。筆帖式四人。

統領掌前鋒政令,遴滿、蒙銳兵,以時訓練其藝。參領、侍衛掌督率前鋒,警蹕宿衛。

天聰八年,定巴牙喇營前哨兵為噶布什賢超哈。順治十七年,定噶布什賢噶喇衣昂邦漢字為前鋒統領,其章京為參領;置前鋒侍衛、前鋒校各官,並定員數。如前所列。雍正三年,置隨印協理事務參領、侍衛左、右翼各一人,前鋒校各二人。乾隆十七年,增委署前鋒侍衛,旗各一人。五十四年,置避暑山莊帶翎前鋒校十人。仍歸入前鋒校員數內。

護軍營護軍統領,正二品。八人。參領,正三品。副參領,初制正五品。雍正十二年升正四品。俱百十有二人。滿洲各八十人,蒙古各三十有二人。委署參領,給五品頂戴,護軍校內選委。五十有六人。護軍校,正六品。八百八十有五人。滿洲六百八十一人,蒙古二百有四人。委署護軍校給金頂虛銜,食護軍月饟。如署參領數。協理事務參領、副參領,各八人。各由本旗參領、副參領內選補。護軍校,本旗酌委。筆帖式,各十有六人。

統領掌護軍政令,遴滿、蒙精兵,以時訓練其藝。大閱為首隊,夾前鋒列陣。凡遇朝會,得舉非法。參領、副參領掌董率護軍。出則騎從夾乘輿車,居則宿衛直守門戶。

初,設巴牙喇營,統以巴牙喇纛章京,甲喇章京分領之。順治十七年,定巴牙喇纛章京漢字為護軍統領,旗各一人;甲喇章京為護軍參領,旗各十有四人。護軍校編制視佐領,乾隆三十三年增二百十四人。並置署護軍參領員額。雍正元年,改署參領為副參領,旗各十有四人。乾隆三十三年增十六人。三年,置隨印護軍參領、副參領、護軍校等官。乾隆十七年,增委署護軍參領,旗各七人。三十三年增三十有二人。四十一年,遴護軍材力優者七十有七人,為委署護軍校。

景運門直班大臣一人。前鋒統領、護軍統領番直。印務章京,前鋒、護軍印務參領十人番直。上三旗、下五旗各司鑰章京,本旗護軍參領番直奏充。俱一人。直班前鋒參領、護軍參領,二十有九人。前鋒二人,護軍二十七人。巴克什護軍如參領數。前鋒校,護軍校,九十有三人。前鋒二人,護軍九十一人。主事一人。上三旗主事、署主事,各一人番直。門筆帖式五人。上三旗十人,以五人番直。

圓明園八旗、內務府三旗護軍營掌印總統大臣一人。本營總統大臣內特簡。各營同。總統大臣無員限。王、公、大臣兼任。八旗營總護軍參領,各八人,俱正三品。副參領倍之,初制五品。雍正十年升正四品。署參領又倍之。初制六品。雍正十年升正五品。護軍校,正六品。副護軍校,從八品。各百二十有八人。協理事務營總護軍參領,各二人,護軍校四人。筆帖式三十有二人。三旗營總一人。初制四品。乾隆三十七年定三品銜食四品俸。護軍參領,三品銜食五品俸。副參領,四品銜食五品俸。委署參領,五品銜食護軍校俸。各三人。護軍校九人,副護軍校三人。筆帖式四人。

總統掌圓明園翊衛政令。駕出入則警蹕。環園門汛,督攝守衛。營總以下掌轄營眾警夜巡晝。雍正二年,設圓明園護軍營,置八旗營總八人,副護軍參領十有六人,署副參領三十有二人,護軍校八十人。十年增三十三人。乾隆十二年增十六人。並設內務府三旗護軍營,置參領、侍衛、委署參領、後改副參領。護軍校、委署參領,旗各一人,護軍校各三人,委署護軍校各一人,後改副護軍校。簡總統大臣領之。七年,八旗置護軍校七有二人。十年增四十人。乾隆十二年增十有六人。十年,三旗置營總一人,八旗護軍參領各一人。乾隆十六年,置隨印協理事務營總各官。

三旗包衣驍騎營參領,內務府郎中兼充。初制五品。乾隆三十六年定三品銜,仍食五品俸。副參領,初制六品。乾隆三十六年定四品銜,食俸如故。滿洲佐領,從四品。各十有五人。旗鼓佐領,漢軍十有八人,正黃旗世襲朝鮮佐領二人,正白旗回子佐領一人。三旗驍騎校三十有六人。正六品。內朝鮮二人,回子一人。校尉長驍騎校,二人。內管領,初制正五品。道光二十五年改從。副內管領,正六品。旗各十人。

三旗包衣護軍營統領三人。正三品。參領,初制五品。乾隆三十二年改四品銜。三十六年定三品銜,食俸如故。副參領,同驍騎校。委署參領,本旗護軍校內委署。各十有五人。護軍校,五品銜,雀翎。委署護軍校,金頂藍翎。各三十有三人。食護軍饟。護軍藍翎長十有五人。正九品。

三旗包衣前鋒營參領,護軍校、委署參領內簡選。雀翎。仍食護軍校俸。委署參領,護軍校內簡選。五品銜,雀翎。食俸如故。前鋒校,副護軍校內簡選。藍翎。仍食護軍饟。委署前鋒校,護軍內簡選。藍翎。各六人。藍翎長十有二人。金頂藍翎。

驍騎營參領、副參領掌備禁城宿衛,兼司襲職考射挑甲。佐領以下掌轄旗眾,稽覈戶口俸饟,籍達參領。護軍營掌守宮掖,典導引扈從。前鋒營掌習解馬、花馬箭。

初設內務府,置內管領四人。順治三年、六年俱增四人。十一年增八人。康熙二十四年又增四人。三十年增六人。順治元年,置內府三旗滿洲佐領九人,旗鼓佐領十有二人,康熙三十四年,旗各增二人。朝鮮佐領一人,康熙三十四年增一人。雍正十年改世管佐領。隸領侍衛內大臣。十八年,置滿洲佐領下護軍校各二人,旗鼓佐領內管領護軍校各一人。康熙二十三年省十二人。雍正九年增十五人。康熙十三年,改隸內務府。十六年,定三旗各編五參領,置護軍參領、驍騎參領,乾隆十六年遴府屬司官五人掌關防。舊置參領改為副參領。如其數。驍騎校編制視佐領。康熙三十四年增佐領三十三人,驍騎校亦如之。二十年,置委署護軍參領,雍正九年,旗各增五人。十二年各省五人。委署護軍校,雍正三年改副護軍校。九年,旗各增五人。十三年省。旗各五人。二十三年,增副內管領一人。二十四年增四人。三十年、三十四年俱增三人。三十四年,護軍仍隸侍衛處。三十六年,增侍衛、委署參領,旗各三人。雍正九年各增二人。乾隆三十年增一人,管前鋒營。四十三年,增驍騎營副參領如參領數。雍正十三年省。雍正元年,增護軍統領,旗各一人,復改隸內府。四年,置委署副驍騎校如佐領數。十三年省。乾隆十三年,始立前鋒營,置參領、委署參領、前鋒校各二人,以護軍統領轄之。十五年,增委署前鋒校二人。護軍內選用。二十五年,置回子佐領、驍騎校各一人。三十二年,增護軍藍翎長五人。四十七年,增校尉長驍騎校二人。嘉慶七年,增前鋒營藍翎長四人。宣統三年,改隸前鋒、護軍等營事務處。

步軍營提督九門步軍巡捕五營統領一人,親信大臣兼任。初制正二品。嘉慶四年升從一品。左、右翼總兵各一人。正二品。其屬:司務司務一人;筆帖式十有二人;左、右二司郎中各一人;員外郎、主事,各三人。司務以下俱滿缺。所轄:翼尉、正三品。副翼尉、從三品。協尉、正四品。副尉,正五品。滿、蒙、漢軍俱各八人。捕盜步軍校,正五品。滿洲二十有四人,蒙古、漢軍各八人。步軍校,滿洲百六十有八人,蒙古、漢軍各六十有四人。內職捕盜者四十人。委署步軍校,正六品。滿洲四十人,蒙古、漢軍各十有六人。城門領,初制正四品。乾隆十四年改從。城門吏,正七品。滿洲各十有八人,漢軍各七人。門千總,正六品。漢軍三十有二人。巡捕五營副將一人,中營置。參將四人。南、北、左、右營各一人。游擊、都司各五人,守備十有八人,千總四十有六人,把總九十有二人。將之下,品級詳見綠營。信砲總管,正四品。滿洲一人;監守信砲官,正五品。滿洲、漢軍各四人。

統領掌九門管鑰,統帥八旗步軍五營將弁,以周衛徼循,肅靖京邑。總兵佐之。郎中各官掌勾檢簿書,平決諍訟。司務掌典守檔冊,計會俸饟。翼尉各官掌分轄步軍,守衛循警。城門領掌司門禁,稽查出入。巡捕營各官掌分汛防守,巡邏糾察,以執御非違。信砲總管掌有警奉金牌聲眾。

初置步軍統領一人,左、右翼總尉各一人,乾隆十九年改翼尉。步軍校,八旗滿、蒙參領下各四人,漢軍各二人;乾隆十九年改步軍尉。三十六年復故。並定巡捕二營,置參將以次各官。以兵部職方司漢主事一人司政令。京城內九門、外七門,置指揮、千百戶隸之。順治四年改門千總。順治五年,置步軍副尉,滿、蒙、漢軍,旗各一人。乾隆十九年改協尉。十年,允尚書噶洪達請,設白塔山及內九門信砲各五,置漢軍信砲官左、右翼各二人。雍正二年更名,並定員限。乾隆八年始來隸。員數如前所列。十四年,置巡捕中營官。康熙十三年,始命步軍統領提督九門事務,並定城門尉、城門校,乾隆十九年改城門領、城門吏。內九門俱各二人,外七門俱各一人,千總門各二人,以統轄十六門門軍。二十四年,八旗滿、蒙各參領下增委署步軍校一人。三十四年定八旗滿洲各五人,蒙古、漢軍各二人。三十年,復命步軍統領兼管巡捕三營。三十四年,增捕盜步軍校四十人。步軍校內遴委。六十一年,置滿洲員外郎一人。雍正四年,置步軍參尉,乾隆十九年改副尉。滿、蒙、漢軍,旗各一人。七年,簡部臣一人協理刑名。乾隆四十三年省步軍統領,由都統、副都統授者仍置。明年,增滿洲員外郎一人,置主事二人。十三年,置滿洲司務一人。四十六年,以三營轄境遼廓,增設左、右二營,是為五營,並置副將各官。嘉慶四年,增左、右翼總兵各一人,郎中一人。九年,增副翼尉二人。

火器營掌印總統大臣一人。總統大臣無員限。王、公、侍衛內大臣、都統、前鋒護軍統領、副都統內特簡。內、外營翼長,正三品。署翼長營總,正三品。各一人;營總各三人。鳥槍護軍參領各四人,正三品。副參領倍之,正四品。署參領又倍之。從五品。鳥槍護軍校,正六品。藍翎長,俱各百十有二人。協理事務翼長、署翼長、營總各一人,鳥槍護軍參領四人,俱以內營人員兼充。委署參領上行走十人。以協理參領不敷督率,增內營三人、外營七人。筆帖式十有六人。

總統掌教演火器政令,遴滿、蒙兵習其藝者別為營,分內、外,以時較試。其御河旁一營,兼督水軍習楫棹,巡幸則備扈從。翼長各官掌分轄訓練。

康熙二十七年,設漢軍火器兼練大刀營,置總管、翼長各一人,副都統兼管。協領、參領,旗各一人,操練尉、驍騎校各五人。三十六年俱省。三十年,始設火器營,置鳥槍護軍參領十有六人,以旗員兼任。雍正三年省察哈爾八旗護軍參領,改入本營為專缺。乾隆二十七年省八人。鳥槍驍騎參領二十有四人,乾隆二十八年省。鳥槍驍騎校百十有二人,乾隆三十五年省入護軍校。簡王、公、大臣領之。乾隆二十八年,改置營總、鳥槍護軍參領,旗各一人,副護軍參領各二人,委署護軍參領各四人,護軍校藍翎長各二十有八人。三十五年,以副護軍參領八人兼司砲位。先是置管砲散秩官五十六人。乾隆二十八年省,至是來隸。並增正、副翼長各一人。三十八年,遴護軍校十人為委署參領上行走。

健銳營掌印總統大臣一人。總統大臣無員限。王大臣兼任。翼長、委署翼長、前鋒參領各一人,副前鋒參領八人,正三品。副參領倍之,正四品。署參領又倍之。從五品。前鋒校百人,正六品。副前鋒校四十人,前鋒內選用。藍翎長五十人。護軍內選用。番子佐領、防御各一人,驍騎校二人,前鋒軍水師教習、委署千總、把總各四人。筆帖式八人。協理事務章京無恆額。本營參領內委派。

總統掌左、右翼健銳營政令,遴前鋒、護軍習雲梯者別為營,以時訓練其藝。大閱為翼隊。會外火器營交沖,並督水軍習戰。翼長各官掌董率營卒。番子佐領掌督攝番兵。水師千、把掌教駕船駛風,演習水嬉。

乾隆十四年,設健銳營,駐香山,簡王大臣領之。分兩翼,置翼領各一人,八旗前鋒參領、副參領各一人,二十八年增前鋒參領二人,副參領八人。三十五年簡前鋒參領二人為委翼長。前鋒校各五人。十五年增十人,二十八年增二十四人,三十三年增二十六人。十五年,定昆明湖教水戰,置教習把總八人。內四人為委署千總,向天津、福建水師營調取。十八年,置委前鋒參領十有六人,二十八年、五十年俱增八人。副前鋒校四十人。三十九年,增藍翎長五十人。四十一年,金川番子徙京,置佐領、驍騎校各一人。五十三年,增番子驍騎校、防御各一人。

總理行營大臣六人,宗室、蒙古王大臣兼任。掌行營政令。巡幸前期,考其日月行程,以定翊衛扈從,並稽察各營翊衛官兵。所轄辦事章京十有六人。護軍參領兼充。

神機營掌印管理大臣一人。親王、郡王兼任。管理大臣無員限。王、公、領侍衛內大臣、都統、前鋒護軍各統領、副都統內特簡。掌本營政令,遴前鋒、護軍、步軍、火器、健銳諸營精捷者別為營,以時訓練其藝。大閱各備練式,分官兵以守衛。總理全營事務翼長三人,掌董帥隊伍。文案、營務、糧饟、覈對、稿案五處總理翼長七人。文案、營務各二人,餘一人。委翼長二人,文案、營務各一人。幫辦翼長二人。隸文案處。學習翼長三人,隸營務處。承辦章京一人,隸覈對處。差委侍衛章京七十有四人,隸營務處。委員九十有四人。文案三十九人,營務四十五人,糧饟六人,覈對七人,稿案五人。印務處委員二人。軍火局制造軍火器械。管帶官、營總各一人,辦事章京二人。軍器庫受付軍火器械。管帶官、委翼長、管庫章京各二人,委員四人。槍砲廠司訓練測量算學。總辦二人,委員二十有七人。機器局制造槍支、銅冒、火箭、鉛丸、火藥。總辦三人,提調二人,總監工一人,委員十人,辦事官二人。馬步隊兵二十五營,專操管帶二十有四人,幫操二十有五人,營總四十有一人,令官十有七人。

道光十九年,御前大臣奕紀請建神機營,鑄印信,未成軍。咸豐十一年,始練兵設營,置專操大臣十有六人,幫操侍衛章京二十有二人,帶隊章京百九十有六人。同治初,改訂官制,如前所列。簡親王領之。

虎槍營總統無員限。王、公、大臣兼任。總領六人。上三旗各二人,自一品至五品內特簡。虎槍校、委虎槍校,各二十有一人。旗各七人,俱虎槍營內選用。筆帖式六人。總統、總領掌轄本營官兵以備扈從,車駕蒐狩列前驅。

康熙二十三年,黑龍江將軍送滿兵善騎射者四十人,分隸上三旗,始設虎槍營,以總統一人領之,置總領虎槍校,旗各一人。雍正元年,增總領,旗各一人,虎槍校各六人,置委虎槍校各七人。乾隆三年,始鑄關防。

鄉導處掌印總統大臣一人。總統大臣內特簡。總統大臣無員限。前鋒統領、護軍統領、副都統兼任。章京三十有二人,旗各四人,護軍參領內選補。藍翎長四人,協理事務章京、章京內選充。筆帖式,各二人。本處掌度地建營。凡時巡省方,駕行佩櫜鞬前導。

上虞備用處亦曰黏竿處。管理大臣無員限。王、公、額駙、滿蒙大臣內特簡。黏竿長頭等侍衛一人,二等內揀補。二等三人,三等內揀補。三等二十有一人,藍翎內揀補。藍翎十有五人。拜唐阿內揀補。協理事務頭等侍衛一人,黏竿長頭等侍衛兼充。筆帖式三人。庫掌一人。庫拜唐阿內揀補。本處掌協衛扈從。

善撲營總統大臣無員限。都統、前鋒統領、護軍統領、副都統內特簡。左、右翼翼長各三人。本營侍衛教習、各營侍衛章京內揀補。協理事務翼長二人。翼長兼任。筆帖式六人。本營掌選八旗勇士習角牴技,扈從則備宿衛。

王公府屬各官長史,從三品。親王、世子、郡王、長子府各一人,司禮長,從四品。貝勒府一人,掌董帥府僚,紀綱眾務。散騎郎,世職領之。親王府四人,世子、郡王府三人,長子府二人,掌佐長史理府事。護衛,親王府二十人,一等六人,從三品;二等六人,從四品;三等八人,從五品。自三等以下,並戴藍翎。世子府十有七人,一等、二等各六人,三等五人。郡王府十有五人,一等六人,二等四人,三等五人。長子府十有二人,一等二人,二等四人,三等六人。貝勒府十人,二等六人,三等四人。貝子府六人,公府四人,俱三等。掌府衛陪從。典衛,親王府六人,四、五、六品各二人。世子府五人,四品一人,五、六品各二人。郡王府四人,五、六品各二人。長子府三人,五品二人,六品一人。貝勒、五品一人,六品二人。貝子、六品一人,七品二人。公七品一人。八品二人。府各三人,掌禮節導引。五旗參領各五人,從三品。佐領各七人,從四品。驍騎校如佐領數,從六品。掌王府所屬旗籍政令,稽田賦戶口。管領,從六品。親王府四人,郡王府三人,掌文移遣委事。典膳,從六品。親王、郡王府各一人,掌供食薦羞。司庫,從七品。親王、郡王府各二人,掌監守庫藏。司匠,從八品。親王、郡王府各四人,掌營繕修葺。牧長,從八品。親王府四人,郡王府三人,掌蕃育牛馬。

順治元年,定諸王、貝勒、貝子、公護衛員:攝政王三十人,一、二、三等各十人。輔政王二十有三人,一、二等各七人,三等九人。和碩親王二十人,一、二等各六人,三等八人。多羅郡王十有五人,一等六人,二等四人,三等八人。多羅貝勒十人,二等六人,三等四人。固山貝子六人,公四人。俱三等。八年,定王府武職官制,置長史、司儀長、散騎郎、護衛、典儀各官,並佐領下各置驍騎校有差。雍正四年,定王府散騎郎員數,貝子以下並省之。乾隆十九年,定王、公護衛、典儀等官,俱為從品。宣統元年,避帝諱,改司儀長為司禮長,典儀為典衛。公主府同。

先是怡賢親王贊襄世宗,莊恪親王輔翊高宗,俱封雙親王,護衛倍之。嘉慶初,儀、成二王並增置一、二、三等護衛各二人;定親王、慶郡王增置一等護衛一人,二、三等各二人。宣統嗣位,議定監國攝政王官員制度,較親王倍之,俱曠典也。

固倫公主府:長史,一等護衛,各一人,二、三等各二人;典衛二人。和碩公主府:司禮長一人;二等護衛二人,三等一人;六、七品典衛各一人。乾隆五十一年,始定公主府屬員數。

陵寢駐防各官興京副都統一人。轄永陵翼長各官及護守兵役。守陵總管各一人。正三品。翼長各二人。正三品。唯昭西陵、孝東陵、泰東陵、昌西陵、普祥峪定東陵、菩陀峪定東陵,專置防禦、驍騎校,額如下。司工匠各一人。初制五品。康熙八年升四品。永陵、福陵、昭陵置。防御各十有六人,正五品。驍騎校各二人。正六品。園寢守衛防御各八人,驍騎校各一人。

總管掌守衛陵寢,翼長以下悉隸之,受副都統節度。初,天聰八年,置永陵燒造磚瓦散秩五品官。順治五年增福陵、昭陵各一人。康熙八年改司工匠。順治二年,置福陵防御一人。明年增一人。十三年,福陵、昭陵置總管、翼領、乾隆五十九年改翼長。防御各官。乾隆二年,置各陵驍騎校二人,自是為定制。光緒元年,始置興京副都統。

各省駐防將軍等官將軍,初制正一品。乾隆三十三年改從。都統,從一品。專城副都統,正二品。同城者分守各地。掌鎮守險要,綏和軍民,均齊政刑,修舉武備。參贊大臣,掌佐畫機宜。領隊大臣掌分統游牧。品秩俱從原官。總管,正三品。副總管,正五品。掌分理營務。城守尉,正三品。防守尉,正四品。掌本城旗籍。參領、協領俱從三品。以次各官,分掌駐防戶籍,以時頒其教戒,仍隸京旗。亦有佐領或防禦分駐他所者,東三省、察哈爾所屬是也。初鑄大將軍、將軍諸印,庫藏經略、大將軍、將軍印凡百餘,乾隆十四年始毀。撫遠、寧遠、安東、征南、平西、平北大將軍印七,鎮海、揚威、靖逆、靖東、征南、定西、定北將軍印七,收藏皇史宬,命將出師,奏請頒給。康、雍間,有靖寇、安遠、奉命、平逆、平寇、建武、討逆、寧遠、靖邊、定邊、綏遠、振武、靖逆、蕩寇,乾隆間寧遠、靖邊、奮威、靖逆,嘉慶間定西,道光間揚威諸目,並頒印信。品秩俱從原官。

先是經略大臣、大將軍、將軍,簡王、貝勒、貝子、公或都統、親信大臣為之,大征伐則置,畢乃省。逮建八旗,駐防簡將軍、都統領之。將軍始專為滿官,西北邊陲大臣及城守尉各官,亦概定滿缺。自畿輔達各省,東則奉、吉、黑,西回、藏,北包內外蒙古,分列將軍、都統及大臣鎮撫之。摭其梗概,志之左方。

盛京駐防將軍一人。其屬有主事、筆帖式各官。吉林、黑龍江同。初以內大臣一人為留守。順治三年,改昂邦章京。康熙元年徙遼東,號遼東將軍。乾隆十二年,移駐盛京。光緒三十三年省,歸東三省總督兼攝。副都統四人。舊置梅勒章京二人。康熙元年更名。雍正五年徙一人駐錦州,復增置熊岳一人。道光二十三年徙熊岳一人駐金州。光緒元年增置興京一人。宣統元年省錦州一人。副都統銜總管一人。城守尉八人。盛京四人,興京、鳳凰、遼陽、開原城各一人。協領十有五人。內水師一人。防守尉二人。分駐牛莊、熊嶽。佐領百三十有一人。內宗室二人,水師二人。防禦百有二人。內水師四人。驍騎校二百有七人。內水師八人。

吉林駐防將軍一人。順治十年,置寧古塔昂邦章京二人。康熙元年更名。省一人。十五年,徙吉林。光緒三十三年省。副都統七人。順治間置二人。康熙十年徙一人來駐。十五年還駐寧吉塔。三十一年置伯都訥一人。五十三年置三姓一人。雍正三年置吉林一人。乾隆元年置阿勒楚喀一人。宣統元年俱省。協領二十有三人。參領一人。佐領百三十有七人。防禦八十有一人。驍騎校百四十有一人。舊置四、五、六品管水手官。咸豐二年置水師營總管一人。光緒十四年增置一人。宣統二年俱省。

黑龍江駐防將軍一人。康熙二十二年,嘉寧古塔副都統薩布素征俄有功,授將軍,駐璦琿。二十九年,徙墨爾根。三十八年,徙齊齊哈爾。光緒三十三年省。副都統七人。初置二人。康熙四十九年增置墨爾根一人。光緒五年改呼蘭城守尉為副都統。七年改呼倫貝爾總管為副都統。二十一年增置布特哈一人。二十五年增置通肯一人。三十一年省齊齊哈爾、呼蘭、布特哈、通肯副都統。三十三年省墨爾根、呼倫貝爾、黑龍江副都統。副都統銜總管一人。總管九人。內水師一人。協領二十人。參領一人。打牲處副總管二十有三人。佐領二百五十人。防禦二十有八人。驍騎校二百五十人。護軍校二人。水師營管水手四品官四人、五品官三人、六品官五人。

江南駐防將軍一人。順治二年,置昂邦章京。十七年,改總管。康熙二年,更名將軍,駐江寧。副都統二人。順治二年置,駐江寧。十六年增置京口二人。乾隆二十八年省京口一人。三十四年省江寧一人。協領十人。佐領四十有六人。防禦、驍騎校各五十有六人。舊置京口將軍。乾隆二十二年省。

福建駐防將軍一人。順治十三年,置固山額真。十七年,改都統。康熙二年省。十九年,置將軍,駐福州。副都統一人。康熙十九年置。雍正五年增一人。乾隆四十四年省一人。協領九人。內水師一人。佐領、防御各十人。內水師各二人。驍騎校二十有二人。內水師二人。

浙江駐防將軍一人。順治四年,置固山額真。十五年,改昂邦章京。十七年,改總管。康熙二年,更名將軍,駐杭州。副都統二人。順治十年置,分左、右翼,駐杭州。康熙十三年增漢軍二人。雍正七年徙杭州右翼一人,駐乍浦。乾隆十六年省漢軍一人。二十八年漢軍俱省。協領十有四人。內水師五人。佐領三十有四人。內水師十一人。防禦二十有八人。內水師八人。驍騎校四十有八人。內水師十六人。

湖北駐防將軍一人。康熙二十二年置,駐荊州。副都統二人。同時置,分左、右翼。協領十人。佐領四十有六人。防禦、驍騎校各五十有六人。

四川駐防將軍一人。乾隆四十一年置,駐成都。副都統一人。康熙六十年置。協領五人。佐領十有九人。防禦、驍騎校各二十有四人。

廣東駐防將軍一人。順治十八年置,康熙五年省,十九年復故,駐廣州。副都統二人。康熙二十年置漢軍二人。乾隆二十一年定滿洲、漢軍各一人。協領九人。佐領十人。防禦三十有四人。驍騎校三十有八人。康熙五年置廣西將軍、都統各一人。十三年省。

綏遠城駐防將軍一人。乾隆三年,置建威將軍,二十六年更名。二十八年,兼司土默特蒙古事務。初置都統一人,管土默特二旗。至是省入。副都統一人。康熙三十三年置歸化二人。乾隆二年置綏遠二人。十三年省二人。二十八年分駐二城。尋省綏遠一人。協領五人。佐領六十有四人。防禦二十人。驍騎校六十有九人。又歸化城初置都統二人,分左、右翼。康熙三十三年省右翼,四十四年復故。乾隆二十六年省左翼。二十八年俱省。

陜西駐防將軍一人。順治二年,置昂邦章京。康熙元年更名,駐西安。副都統二人。順治十八年置西安右翼二人。康熙二十八年增漢軍二人,徙一人駐江寧,以江寧左翼一人來駐。乾隆二十六年省左翼滿洲一人,右翼漢軍一人。二十八年定左、右翼各一人。三十七年徙一人駐涼州。四十九年復增一人。協領八人。佐領二十有三人。防禦、驍騎校各四十人。

甘肅駐防將軍一人。雍正三年置,駐寧夏。乾隆二年別置涼州一人。三十八年省。副都統二人。同時置,分左、右翼,駐寧夏。乾隆二年增涼州、莊浪各一人。二十八年省莊浪一人。三十四年省寧夏右翼一人。三十八年省涼州一人,徙西安一人駐涼州,曰涼莊副都統。城守尉一人。駐莊浪。協領七人。佐領三十有二人。防禦四十有一人。驍騎校三十有九人。

新疆駐防伊犁將軍一人。乾隆二十七年置。參贊大臣一人。副都統二人。光緒十年省參贊大臣,明年置副都統二人。十四年徙一人駐塔爾巴哈臺。領隊大臣四人。分駐索倫、額魯特、察哈爾、錫伯。總管六人。副總管七人。兼司駝場、馬場。協領十有二人。佐領、驍騎校各百有八人。防禦五十有六人。

熱河駐防都統一人。雍正二年置總管,嘉慶十五年改置。道光八年,命管承德刑名、度支。圍場總管一人。翼長二人。協領五人。佐領十有五人。防禦三十人。圍場八人。驍騎校二十有八人。圍場八人。前鋒校十人。

游牧察哈爾駐防都統一人。康熙十四年,置八旗總管各一人。乾隆二十六年,改置都統,駐張家口。副都統一人。初置二人。乾隆三十一年省一人。總管十人。副總管一人。參領、副參領各八人。佐領、驍騎校各百二十人。護軍校百十有五人。親軍、捕盜六品官各四人。

直隸駐防副都統二人。康熙二十七年,置山海關總管。乾隆七年,改置副都統。四十五年,增置密雲一人。城守尉二人。分駐保定、滄州,隸駐京稽察九處旗務大臣。協領四人。防守尉十有六人。駐東安、固安、採育里、雄縣、寶坻、霸州、良鄉者,所隸與城守尉同。駐古北、昌平州者,隸密雲副都統。駐永平、三河、喜峰口、玉田、順義、冷口者,隸山海關副都統。駐獨石口者,隸察哈爾都統。佐領二十有五人。防禦七十有三人。乾隆間,置天津水師營副都統、獨石口副都統各一人。後俱省。

山東駐防副都統一人。雍正十年置,駐青州。舊有將軍。乾隆二十六年省。城守尉一人。駐德州。協領四人。佐領、防禦、驍騎校各二十人。

山西駐防城守尉二人。順治六年置太原一人。康熙三十三年,右衛置將軍一人,護軍統領二人,副都統四人。三十七年省護軍統領、副都統各二人。乾隆二年省將軍、副都統。三十三年置右衛城守尉一人,隸巡撫。防禦、驍騎校各八人。

河南駐防城守尉一人。康熙五十七年置,駐開封,隸巡撫。佐領、防禦、驍騎校各十人。

提督等官提督軍務總兵官,從一品。掌鞏護疆陲,典領甲卒,節制鎮、協、營、汛,課第殿最,以聽於總督。鎮守總兵官,正二品。掌一鎮軍政,統轄本標官兵,分防將弁,以聽於提督。副將,從二品。為提、鎮分守險汛曰提標,為總督綜理軍務曰督標中軍,將軍標、河標、漕標亦如之。參將,正三品。游擊,初制正三品。順治十年改從。掌防汛軍政,充各鎮中軍官。都司,初制正三品。順治十年改從。十八年改正四品。康熙九年復故。二十四年定正四品。所掌視參、游,充副將中軍官。守備,初制正四品。康熙三十四年定正五品。掌營務糧饟,充參、游中軍官。千總,初制正六品。康熙三十四年,營千總改從六品。五十八年復故。把總,正七品。外委把總,正九品。額外外委,從九品。各掌營、哨汛地。

初制,提督、總兵無定品,系左右都督、都督、同知、僉事各銜。乾隆十八年停。始定品秩。提督典兵,自畿輔海甸迄雪山炎徼,星羅棋布。腹地兼以巡撫,承以總兵。副將以下,品目粲然,有事隨提、鎮為員,如隨征、營援、剿營之類。事畢乃省。自三籓之亂,提、鎮效用者眾。咸、同間,戡定發、捻,湘、淮、楚營士卒,徒步起家,多擢提、鎮,參、游以下官,益累累然,保舉冗濫,往往記名提、鎮,降充末弁,候補千、把,驟膺統將,官職懸殊,至斯已極。光緒間,創設海軍,亦置提、鎮,無績罷之。厥後更定陸軍官制,河、漕標營,以次並廢。綠營歲有汰革,厲行者浙江,次廣東、廣西、湖南、湖北,謹就可考者著於篇。

直隸提督一人。順治十八年置,駐大名。康熙二十七年省。三十年復故,徙古北口。總兵七人。天津、真定二鎮俱順治元年置。其真定,康熙二十七年省,雍正四年再置。宣化鎮,康熙七年改鎮朔將軍置。馬蘭鎮,雍正二年改副將置。泰寧鎮,乾隆元年置,兼內務府大臣。大名鎮,道光元年改副將置。通永鎮,二十三年改陜西西安鎮置。副將八人。山永協,順治六年置。通州協,八年改鎮置。河間協,康熙八年移真定協改置。開州協,雍正十年改參將置。督標中軍,十一年置。河屯協,乾隆元年改營置。大沽協,二十三年改營置。多倫諾爾協,光緒七年改都司置。參將八人。提標、紫荊關、務關路及保定城守、涿州、八溝、昌平、固關諸營。游擊二十有七人。都司五十有九人。河標一人。守備七十有二人。河營二人,河營協辦一人。千總百五十有七人。所千總二人。把總三百四十有六人。奉天捕盜營把總十有四人。

四川提督一人。初置剿撫提督。順治五年省。十七年復置,駐省。總兵四人。建昌鎮,順治四年置。川北鎮,十五年改保寧鎮置。重慶鎮,康熙八年移永寧鎮改置。松潘鎮,十年改副將置。副將八人。夔州協,康熙十年改鎮置。督標中軍,十九年置。維州協,乾隆十八年改威茂協置。阜和協,四十三年改都司僉書置。將軍標中軍,四十六年置。懋功協,四十七年改營置。綏寧協,嘉慶二年改營置。馬邊協,九年改綏定協置。參將七人。提標及瓘邊、普安、永寧、漳臘、越巂、會川諸營。游擊二十有三人。都司三十有二人。守備五十有一人。千總百十有四人。把總二百有七人。

廣東提督一人。順治八年置。十八年,徙惠州。康熙三年,置水師一人,駐順德。七年省。嘉慶十四年,改陸路提督,復置水師一人,駐虎門。光緒三十三年,並為一。尋以海盜警復故。宣統三年,仍省水師提督。總兵七人。潮州鎮、瓊州水師鎮,俱順治八年置。高州鎮,十二年置。碣石水師鎮,十一年置。康熙三年省,八年復故。南澳水師鎮,二十四年改海防參將置。南韶連鎮,嘉慶十五年改左翼鎮置。北海鎮,光緒十二年改平陽水師鎮置。其瓊州、南澳、碣石俱宣統三年省。副將十有三人。南雄協,順治八年置。龍門水師協、督標中軍,俱康熙四年置。中軍初分左、右翼,後並為一。廣州、惠州、黃崗、肇慶諸協,俱八年置。羅定協,十二年置。三江口協,四十一年置。順德水師協,四十三年改虎門協置。大鵬水師協,道光二十年改澄海協置。崖州水師協,二十二年改參將置。赤溪水師協,同治七年改廣海寨游擊置。宣統三年止留中軍及廣州協,餘俱省。參將十有二人。督標中軍左營、增城營。其督標右營、前營、提標中軍、肇慶海口水師、欽州、新會、平海、海門、澄海諸營,俱宣統三年省。游擊二十有七人。內、外海水師八人。內河水師三人。宣統三年止留瓊州鎮中軍、南韶連鎮中軍、靖遠營,各一人。都司三十有四人。外海水師二十人。內河水師八人。宣統三年止留廣州協左營兼中軍右營、佛山、饒平營、黃岡,各一人。守備八十有二人。外海水師二十人。內河水師八人。宣統三年止留增城營、從化、肇慶協、那扶,各一人。千總百六十有八人。宣統三年止留陸路提標中營北城一人。把總三百二十有七人。宣統三年止留廣州協右營纜路尾一人。

廣西提督一人。順治八年置,十七年省,尋復故,駐柳州。光緒十一年徙龍州。宣統三年徙南寧。總兵三人。左江鎮,康熙元年改右翼總兵置。右江鎮,雍正二年改泗城副將置。柳慶鎮,嘉慶十二年置。光緒三十年省,移右江鎮駐柳州,左江鎮駐百色。宣統三年復移百色駐龍州。副將七人。樂平協,順治十二年置。梧州協、潯州協,康熙二十一年改梧潯協分置。慶遠協,雍正七年置。新太協,八年置。鎮安協,十三年置。義寧協,乾隆六年置。宣統三年俱省。參將四人。宣統三年省融懷、全州二營,止留提標中軍左、增城二營。游擊十人。都司十有一人。守備二十有九人。千總六十有五人。把總百二十有一人。光緒二十九年後,止留撫標都、守各一人,提標守、千、把各一人,兩鎮游、千各一人。宣統三年俱議省。

雲南提督一人。順治十八年置,駐永昌。康熙元年徙大理。總兵六人。臨元鎮,順治十年置。開化鎮,康熙六年置。鶴麗鎮,七年置。昭通鎮,雍正九年改東蒙鎮置。普洱鎮,十年改元普鎮置。騰越鎮,乾隆四十一年改副將置。副將六人。督標中軍,順治十六年置。維西協,乾隆十二年置。曲尋協、楚雄協,俱三十五年改鎮置。永昌協,四十年改永順鎮置。順云協,道光二十九年改營置。參將十有一人。提標及尋霑、武定、元新、鎮雄、東川、永北、威遠、廣南、龍陵、鎮邊諸營。游擊二十有一人。都司十有六人。守備五十有一人。千總百有三人。把總二百十有四人。

貴州提督一人。順治十六年置,駐省。康熙六年徙安順。總兵四人。鎮遠鎮,康熙元年置,七年省。乾隆二年改臺拱鎮置。咸寧鎮,康熙三年置,六年省。乾隆元年復故。古州鎮,雍正七年置。安義鎮,嘉慶二年置。副將十人。銅仁協,順治十六年置。乾隆三年省,五年復故。定廣協,康熙三年置。平遠協,八年改鎮置。大定協,雍正三年改鎮置。遵義、清江、都勻三協,俱七年置。上江協,十三年置。松桃協,乾隆三年置。永安協,六年置。其都勻、上江,宣統三年俱省。參將七人。撫標、提標及羅斛、丹江、臺拱、黎平、朗洞諸營。游擊二十有五人。都司二十有三人。守備五十有二人。千總百二十有二人。把總二百有五人。

江南提督兼水師一人。順治二年,置江南提督,駐江寧。四年,置蘇松提督,駐松江,專轄蘇、松、常、鎮四府。康熙元年,省江寧一人,以蘇松一人轄全省。十四年,更名江寧提督,轄下江七府一州。增置安徽提督,分轄上江七府三州。十七年,省安徽一人,仍轄全省。總兵四人。蘇松鎮兼水師,順治二年置。狼山鎮,十八年改副將置。徐州鎮,嘉慶十四年改河標左營協置。崇明鎮兼水師,道光二十三年置。副將五人。督標中軍,順治五年置。江寧城守協,康熙七年改鎮置。太湖水師協兼轄浙江太湖游擊,乾隆十一年改參將置。里河淞北水師協,海門水師協,俱同治七年置。參將七人。撫標、提標、水師右營,又蘇州城守、鎮江、吳淞、川沙諸營。游擊二十有五人。水師十人。都司三十有四人。水師九人。守備五十有五人。水師十有五人。千總百十有六人。把總百八十有九人。衛守備一人。

安徽巡撫兼提督一人。康熙十四年置提督,十七年省。嘉慶八年,巡撫始兼銜。總兵二人。壽春鎮,乾隆二年改副將置。皖南鎮,咸豐五年置。副將一人。安慶協,順治四年改鎮置。參將五人。撫標及徽州、蕪採、寧國、六安諸營。游擊六人。都司八人。守備十有七人。千總二十有五人。把總五十有六人。衛守備九人。

江北提督一人。咸豐十年,置淮揚鎮總兵。光緒三十一年改置。副將一人。提標中軍左營。參將三人。提標右營,淮安城守、海州諸營。游擊五人。都司六人。守備十有二人。千總二十有八人。把總六十有一人。

長江水師提督一人。同治元年置。太平、岳州互駐,江南、湖廣兩總督轄之。總兵四人。江南瓜州鎮,江西湖口鎮,湖北漢陽鎮,湖南岳州鎮,俱同治五年置。副將五人。提標中軍,安慶營,江陰營,田鎮營,荊州營,俱同治五年置。參將六人。裕溪、金陵、吳城、饒州、猈州、沅州諸營。游擊十人。都司四十有二人。守備四十有三人。千總百五十有八人。把總百九十有五人。

山東巡撫兼提督一人。康熙元年置提督,駐青州。四年徙濟南,二十一年省。乾隆八年,巡撫始兼銜。總兵三人。登州鎮,順治十八年改臨清鎮置,轄陸路,康熙六年兼水師,道光三十年改轄水師兼陸路。兗州鎮,雍正三年改參將置。曹州鎮,嘉慶二十二年改參將置。副將三人。膠州協,順治十年置。沂州協,康熙二十二年改鎮置。臨清協,道光二十三年改文登協置。參將十人。撫標及萊州、即墨、青州、泰安、臺莊、德州、東昌、單縣、濟南城守諸營。游擊九人。水師二人。都司十有二人。守備二十有六人。水師三人。千總五十有六人。把總百十有二人。東河營副將、參將各一人。都司三人。守備十有一人。協辦五人。千總十有三人。把總二十人。衛守備三人。領運千總二十有四人。

山西巡撫兼提督一人。順治十八年置提督。康熙元年徙平陽,四年改徙太原,七年省,十三年復故,二十年又省。雍正十二年,巡撫始兼銜。總兵二人。大同鎮,順治元年置,六年省,十一年復故。太原鎮,康熙十一年改副將置。雍正六年升提督,九年復故。副將三人。殺虎口協,康熙三十年改寧武協置。蒲州協,雍正二年改游擊置。潞安協,咸豐十一年改潞澤營參將置。參將九人。撫標及太原城守、平陽、汾州、澤州、新平路、助馬路、東路諸營。游擊八人。都司十有七人。守備二十有九人。千總五十有一人。把總百十有二人。

河南巡撫兼提督一人。順治十八年置提督,駐河南府。康熙三年徙開封,七年省。乾隆五年,巡撫始兼銜。總兵三人。南陽鎮、河北鎮,俱順治元年置。歸德鎮,咸豐八年置,舊有參將隸之。副將二人。荊子關協,嘉慶六年置。信陽協,咸豐八年改營置。參將五人。撫標中軍及河南城守、汝寧、永城、彭德諸營。游擊七人。都司十人。守備二十有三人。千總四十有六人。把總八十有二人。領運千總四人。

陜西提督一人。順治二年置西安提督兼烏金超哈。康熙三年改固原提督。乾隆二十九年復故。嘉慶六年徙漢中,七年還駐固原。總兵三人。延綏鎮,順治元年置。漢中鎮,嘉慶三年改漢羌協置。陜安鎮,五年改興漢鎮置。副將五人。西安城守協、洮岷協、靖遠協,俱順治二年置。其洮岷,六年改參將,十四年復故。西安協,康熙四十年改參將,道光二十三年復移神木協改置。定邊協,順治六年移延綏鎮西協改置。潼關協,咸豐十年移靖寧協改置。參將十人。撫標、提標及西鳳、宜君、靜寧、神木、延安、寧陜、循化、蘭城城守諸營。游擊二十有七人。都司三十有八人。守備四十有四人。千總七十有二人。把總百七十有四人。

甘肅提督一人,舊為總鎮。康熙二年改置,二十二年省,三十年復故,駐甘州。二十四年徙涼州。二十九年徙張掖。總兵五人。寧夏鎮,順治元年置,康熙十五年升提督,二十年復故。西寧鎮,順治十五年置。涼州鎮,康熙二年改副將置,二十六年省,三十年復置,乾隆二十四年又省,越五年又置。肅州鎮,康熙三十年置。河州鎮,乾隆四十七年置。參將九人。督標左、右營,提標中營,及靜寧、甘州城守、靈州、花馬池、平羅、靈武諸營。游擊三十有六人。都司三十有七人。守備五十有六人。千總百有五人。把總二百四十有六人。

新疆提督一人。雍正十三年置哈密提督。乾隆二十四年省,移安西提督駐巴里坤,更名巴里坤提督。二十三年徙烏魯木齊。光緒十一年徙喀什噶爾,更名喀什噶爾提督。總兵三人。巴里坤鎮,乾隆二十九年移烏魯木齊鎮改置。伊犁鎮,四十四年置。阿克蘇鎮,光緒十年移喀什噶爾換防總兵置。副將七人。哈密協,乾隆二十四年置。瑪納斯協,四十二年置。烏什協,道光二十六年置。伊犁軍標塔城協,光緒九年置。烏魯木齊城守協,十三年置。回城協、莎車協,俱十四年置。參將八人。撫標、提標及濟木薩、精河、英吉沙爾、和闐、喀喇沙爾、霍爾果斯諸營。游擊二十人。都司十有七人。伊犁軍標四人。守備六十有一人。伊犁軍標六人。千總七十有五人。伊犁軍標八人。把總二百二十有八人。伊犁軍標二十人。

福建提督二人。轄陸路者,順治四年置,駐泉州。轄水師者,康熙元年置,駐海澄,七年省。十六年,以海澄公領之。十七年復故,駐廈門。總兵四人。汀州鎮,順治六年改左路總兵置,七年省,康熙三十六年改興化鎮復置。福寧鎮,順治十四年改參將置。漳州鎮,康熙二十七年改漳浦鎮置。建寧鎮,雍正十一年改副將置。副將八人。福州、興化、延平三城守協,俱順治七年置。督標中軍,十五年置。閩安水師協,康熙二十七年改鎮置。順昌協,咸豐八年置。金門水師協,同治五年改鎮置。海壇水師協,光緒十三年移澎湖協改置。參將九人。水師、陸路提標及督標左、右,泉州、邵武二城守、水師,閩安烽火門水師諸營。游擊三十人。都司二十有五人。內、外海水師八人。守備六十人。水師十有七人。千總八十有四人。把總百七十有九人。舊置臺灣總兵一人,副將三人,參將、游擊各四人,都司九人,守備十人,千總十有七人,把總十有一人。光緒二十一年棄省,革。

浙江提督兼水師一人。順治三年罝,駐寧波。康熙元年置水師提督,七年省,十四年復故,十八年又省。總兵五人。衢州鎮,順治四年置。溫州鎮兼水師,十二年置。處州鎮,康熙四十九年改平陽鎮置。定海鎮兼水師,雍正八年改左路總兵置。海門鎮兼水師,同治十一年置。副將十有一人。杭州城守兼水師,嘉興、湖州、紹興、金華、嚴州六協,俱順治五年置。樂清協,康熙元年置。象山協兼水師,八年改寧波協置。臺州協,九年置。瑞安水師協,雍正二年置。乍浦水師協,道光二十三年改參將置。參將六人。撫標、提標及鎮海水師、玉環兼水師,寧海、太平諸營。游擊二十人。外海水師十人。內河一人。都司二十有三人。外海水師三人。內河二人。守備五十有二人。外海水師十有七人。內河一人。千總百有九人。把總二百十有三人。自提督以次各官,俱宣統二年省。

江西巡撫兼提督一人。舊為總兵,駐南昌。順治三年改置提督。十八年徙贛州。康熙元年徙建昌,五年還駐南昌,七年省。十三年復故,徙九江,二十一年復省。乾隆十八年,巡撫始兼銜。總兵二人。九江鎮,順治二年置,康熙七年改南瑞鎮,十三年省,二十一年復置,嘉慶九年還駐九江。南贛鎮,順治三年省。副將二人。袁州協,順治三年置,康熙十三年升總兵,二十一年復故。南昌城守協,嘉慶五年改九江協置。參將、撫標及廣信、饒州、寧都、南安、吉安諸營。游擊各六人。都司二十有三人。水師二人。守備十有五人。水師一人。千總三十有一人。把總八十人。衛守備三人。領運千總二十有五人。

湖北提督一人。嘉慶六年置,駐襄陽。總兵二人。宜昌鎮,雍正十三年改彞陵鎮置。鄖陽鎮,嘉慶六年改襄陽鎮置。副將五人。黃州協,順治三年置,宣統元年省。施南協,乾隆元年置。督標中軍、竹山協,俱嘉慶六年置。漢陽協,同治四年置,宣統三年省。參將七人。提標,荊州、武昌二城守,均光、德安諸營。其興國營、撫標中軍,俱宣統三年省。游擊十有二人。都司八人。守備二十有九人。千總七十有二人。把總百四十有三人。衛守備十人。

湖南提督一人。舊為湖廣提督,駐辰州。嘉慶六年改置,徙常德。道光十八年還駐辰州。宣統三年省。總兵三人。永州鎮,康熙九年改副將置。鎮筸鎮,三十八年移沅州鎮改置。綏靖鎮,嘉慶二年置。副將九人。沅州協,順治元年置,八年改鎮,後復如故。寶慶協,十一年改都司置。靖州協,十五年置。長沙協、衡州協,俱康熙五年置。永順協,雍正七年置。永綏協,八年置。乾州協、常德協,俱嘉慶二年置。其寶慶、永順、常德,宣統元年俱省。參將七人。撫標及澧州、宜章、桂陽三營。其岳州城守、臨武二營,俱宣統元年省。提標中軍,三年省。游擊十有五人。都司十有七人。守備三十有四人。千總七十有七人。把總百五十有四人。屯守備、千總各六人。把總十人。衛守備一人。水師二人。

各處駐劄大臣烏里雅蘇臺定邊左副將軍一人。參贊大臣二人。雍正九年,設阿爾泰營置,轄唐努烏梁海五旗三佐領,兼轄土謝圖汗部汗阿林盟一部二十旗,賽音諾顏部齊齊爾里克盟一部二十四旗,並所附額魯特旗烏梁海十二佐領,車臣汗部喀魯倫巴爾和屯盟一部二十四旗,扎薩克圖汗部畢都裏淖爾盟一部十九旗,並所附輝特一旗,烏梁海五佐領。內參贊一人,以蒙古王、公、臺吉兼任。科布多參贊大臣,辦事大臣,各一人。乾隆二十六年置,轄札哈沁、明阿特、額魯特各一旗,阿爾泰烏梁海七旗又二旗,兼轄布爾干河新土爾扈特青色啟勒盟一部二旗,哈弼察克新和碩特部一旗、杜爾伯特烏蘭固木賽音濟雅哈圖盟左翼十一旗,右翼三旗,及所附輝特二旗。同治七年,增置布倫托海辦事大臣、幫辦大臣各一人,八年省,仍隸科布多。庫倫辦事大臣,幫辦大臣,各一人。雍正九年設互市處,駐司員經理。後改置辦事大臣,監督恰克圖俄羅斯通商事宜。乾隆四十九年增一人。尋定為額缺。內一人以蒙古王、公、臺吉兼任。所屬有印房章京,理刑司員,管理商民事務司員,筆帖式等官。分駐恰克圖辦事司員一人。塔爾巴哈臺副都統,乾隆二十九年置參贊大臣一人。光緒十四年省,移伊犁副都統來駐。領隊大臣,乾隆四十一年置,轄額魯特。所屬有印房章京,管理糧饟司員,筆帖式等官。西寧辦事大臣,乾隆元年置,轄青海三十六旗會盟。所屬有司員,筆帖式。各一人。西藏辦事大臣一人。雍正五年置。光緒三十四年增一人。宣統二年省一人。兼轄達木蒙古八旗。所屬有辦事司員,筆帖式。左、右參贊各一人。初置幫辦大臣,宣統二年改置。左參贊駐前藏,右參贊監督三埠通商事宜。所屬有繙譯、書記等官。川滇邊務大臣一人。光緒三十二年置,專司移殖。所屬有書記等官。總管十有六人,塔爾巴哈臺屬一人,科布多屬十人,唐努烏梁海五人,並歸定邊左副將軍兼轄。副總管一人,塔爾巴哈臺屬。參領三人。科布多屬。佐領、驍騎校各三十有三人。塔爾巴哈臺屬各三人,科布多屬各十七人,唐努烏梁海、蒙古達木俱各八人。守卡倫侍衛,自京調遣,三歲一更。邊鎮無額兵者,旗營、綠營官兵番戍,兼治屯焉。

烏魯木齊都統,副都統,各一人。初設安西提標綠旗五營。乾隆三十六年改滿兵駐防,置參贊大臣二人。三十八年復置領隊大臣二人,四十八年改置。協領六人。佐領、防禦、驍騎校各二十有四人。吐魯番領隊大臣一人。乾隆二十四年,建城闢展,置辦事大臣一人,以廣安城為回城。四十二年改置。協領二人。佐領、防禦、驍騎校各四人。所轄:回子四牛錄、佐領、驍騎校各四人。巴里坤、古城領隊大臣各一人。乾隆三十七年置參贊大臣、領隊大臣各一人。後俱改領隊大臣,徙一人駐古城。協領各二人。佐領、防禦、驍騎校各八人。庫爾喀喇烏蘇領隊大臣一人。初置侍衛,隸烏魯木齊。乾隆三十七年改置。所屬有管理糧饟官。又臺站、屯政文武各員,由陜甘、伊犁、烏魯木齊調充。哈密辦事大臣,幫辦大臣,各一人。乾隆二十九年置。所屬有印房章京,筆帖式。同治初,遭回亂,各地相繼淪陷,唯巴爾庫勒旗營僅留孑遺。光緒八年,議改新疆行省,烏魯木齊暨吐魯番各官並奏裁之。十年,省庫爾喀喇烏蘇各官,改直隸、州。明年,復省巴爾庫勒領隊大臣各官,遷旗營入古城,改置城守尉。

喀什噶爾參贊大臣,綜理八城事務。幫辦大臣,各一人。協理喀什噶爾、英吉沙爾事務。俱乾隆二十四年置……三十年徙參贊大臣駐烏什,改置辦事大臣,其幫辦大臣如故。五十三年復舊制。所屬有印房、回務處、經牧處、糧食襄局各司員,及筆帖式。英吉沙爾領隊大臣一人。兼管卡倫。乾隆二十四年置總兵,三十一年改置。所屬有筆帖式。葉爾羌辦事大臣,幫辦兼理糧饟事,各一人。乾隆二十四年置。二十六年置領隊大臣二人,後省。所屬有印房章京,回務章京,筆帖式。和闐辦事大臣兼領隊事一人。乾隆三十年置副都統一人。四十二年改置。所屬有章京,筆帖式。阿克蘇辦事大臣一人。乾隆二十四年置。三十二年並隸烏什。四十四年復移烏什領隊大臣駐。嘉慶二年,分為專城改置。所屬有章京,筆帖式。烏什辦事大臣一人。初置副都統。乾隆二十四年改置,三十年省,移喀什噶爾參贊、幫辦各大臣來駐,並置領隊大臣一人。四十四年移領隊大臣駐阿克蘇。五十二年,參贊、幫辦各大臣還駐喀什噶爾,復舊制。所屬有印房章京,管理糧饟官,筆帖式。庫車辦事大臣一人。乾隆二十四年置。所屬有印房章京,糧饟章京,筆帖式。喀喇沙爾辦事大臣一人。乾隆二十四年置。所屬有印房章京,糧饟章京,回務章京,筆帖式。高宗底定回疆,分建八城,置辦事、領隊各大臣。時英吉沙爾隸喀什噶爾,和闐隸葉爾羌,阿克蘇隸烏什。嘉慶二年始分立,以喀什噶爾參贊大臣綜之。光緒十年,新疆建行省,俱改直隸、州。

回部各官總理回務扎薩克郡王一人。協理圖撒拉克齊二人。駐哈密、闢展,歸誠著績,封爵世襲。阿奇木伯克。掌綜回務。伊犁,喀什噶爾,葉爾羌,和闐,伊里齊城,庫車及所屬沙雅爾,喀喇沙爾,庫爾勒及所屬布古爾,阿克蘇及所屬賽裏木,各一人,俱三品。喀什噶爾屬牌素巴特,英吉沙爾,和闐屬哈拉哈什城、玉隴哈什村、策勒村、克里雅城、塔克弩喇村,阿克蘇屬拜城,各一人,俱四品。喀什噶爾屬阿斯圖阿爾圖什、伯什克勒木、塔什密里克,葉爾羌屬英額齊盤、哈爾哈里克、和什喇普、托果斯鉛、牌斯鉛、桑珠、色勒庫爾、烏什,各一人,俱五品。喀什噶爾屬玉斯圖阿爾圖什三人,內兼管回兵藍翎玉資巴什二人,阿爾瑚、烏帕爾、葉爾羌屬巴爾楚克,阿克蘇屬柯爾坪,各一人,俱六品。伊什罕伯克。掌贊理回務。伊犁,喀什噶爾兼回兵總管,英吉沙爾,葉爾羌,和闐,伊里齊城,阿克蘇及所屬賽裏木,庫車及所屬沙雅爾,喀喇沙爾,庫爾勒,布古爾,各一人,俱四品。阿克蘇屬拜城一人,五品。葉爾羌屬色勒庫爾一人,六品。噶雜拉齊伯克。掌地畝糧賦。喀什噶爾兼回兵副總管,葉爾羌,各一人,俱四品。伊犁二人。和闐,阿克蘇及所屬賽裏木,庫車及所屬沙雅爾,各一人,俱五品。阿克蘇屬拜城一人,七品。商伯克。掌徵輸糧賦。喀什噶爾二人,內一人兼回兵副總管,葉爾羌一人,俱四品。和闐,伊里齊城二人。所屬哈拉哈什,阿克蘇及所屬沙雅爾,喀喇沙爾,庫爾勒,布古爾,各一人,俱五品。葉爾羌屬色勒庫爾一人,六品。哈資伯克。掌平決諍訟。喀什噶爾一人,五品。伊犁喀什噶爾一人,五品。伊犁喀什噶爾屬阿斯圖阿爾圖什、伯什克勒木、玉斯圖阿爾圖什、察拉根、阿爾瑚、罕愛里克,葉爾羌屬哈爾哈里克、托果斯鉛、坡斯坎木、和闐,伊里齊城及所屬哈拉哈什城、玉隴哈什村、策勒村、克里雅城、塔克弩喇村,阿克蘇及所屬賽裏木,烏什,庫車及所屬沙雅爾,喀喇沙爾,庫爾勒,布古爾,各一人,俱六品。葉爾羌屬色勒庫爾一人,七品。斯帕哈資伯克。掌理頭目諍訟。拉雅哈資伯克。掌理細民諍訟。以上二員俱五品,葉爾羌置。密喇布伯克。掌水利。喀什噶爾屬塔斯渾,葉爾羌及所屬牌斯鉛,各一人,俱五品。伊犁喀什噶爾屬伯什克勒木、罕愛里克、霍爾罕、和色爾布依、賽爾璊、托古薩克、阿爾巴特,英吉沙爾,葉爾羌屬英額齊盤、哈爾哈里克、喇普齊、鄂通、楚魯克,各一人,俱六品。喀什噶爾屬木什素魯克,英吉沙爾屬賽里克,和闐,伊里齊城及所屬圖薩拉莊、伯爾臧莊、哈拉哈什城、巴拉木斯雅莊、瑪庫雅莊、雜瓦莊、玉隴哈什村、三普拉莊、洛普莊、策勒村、克里雅城、哈爾魯克莊,各一人;阿克蘇六人,所屬賽裏木、拜城各一人;烏什,庫車各二人;庫車屬沙雅爾一人;喀喇沙爾,庫爾勒,布古爾,各一人,俱七品。訥克布伯克。掌匠役營建。喀什噶爾,葉爾羌,各一人,俱五品。和闐,伊里齊城,阿克蘇,庫車,喀喇沙爾及所屬布古爾,各一人,俱七品。帕提沙布伯克。掌巡緝獄囚。葉爾羌一人,五品。又葉爾羌,喀什噶爾,各一人,六品。和闐,伊里齊城及所屬哈拉哈什城,庫車,各一人,俱七品。莫提色布依伯克。掌回族教法。喀什噶爾一人,五品。葉爾羌一人,六品。和闐,伊里齊城,阿克蘇,庫車,各一人,俱七品。密圖瓦利伯克。掌田產稅務。喀什噶爾,葉爾羌,各一人,俱五品。和闐,伊里齊城,阿克蘇,各一人,俱七品。柯勒克牙拉克伯克。掌商賈貿易。葉爾羌一人,五品。巴濟吉爾伯克。掌理稅務。伊犁,喀什噶爾,阿克蘇,各一人,俱六品。烏什一人,七品。色迪爾伯克。掌襄理稅務。伊犁,喀什噶爾,葉爾羌,各一人,俱七品。阿爾巴布伯克。掌差役。喀什噶爾,葉爾羌,各一人,俱六品。葉爾羌屬色勒庫爾,阿克蘇,烏什,庫車,各一人,俱七品。巴克瑪塔爾伯克。掌果園。喀什噶爾,葉爾羌,各一人,俱六品。都管伯克。掌兵馬糧饟,官物文移。伊犁,喀什噶爾,葉爾羌,各一人,俱六品。和闐,伊里齊城,二人;所屬哈拉哈什城一人,阿克蘇,庫車,各三人;俱七品。哈喇都管伯克。掌臺站兵械。葉爾羌一人,五品。和闐,伊里齊城及所屬哈拉哈什城,各一人,俱七品。明伯克。掌千戶徵輸。喀什噶爾及所屬伯什克勒木、阿爾瑚、霍爾罕,葉爾羌及所屬英額齊盤、哈爾哈里克、鄂普爾,各一人,俱六品。又喀什噶爾三人,及所屬牌素巴特一人,阿斯圖阿爾圖什三人,塔斯渾二人,塔什密里克、玉斯圖阿爾圖什、烏帕爾、罕愛里克、和色爾布伊、賽爾璊、托古薩克、阿爾巴特、木什素魯克,英吉沙爾,葉爾羌屬巴爾楚克、密特西林,和闐,伊里齊城、圖薩拉莊、伯爾臧莊、素巴爾莊、哈拉哈什村、三普拉莊、濟普莊、克里雅城、哈爾魯克莊、策勒村,各一人,阿克蘇十六人,所屬賽裏木、拜城,各一人,烏什一人,庫車三人,所屬沙雅爾二人,喀喇沙爾屬布古爾一人,俱七品。玉資伯克。掌百戶徵輸。伊犁七十人,喀喇沙爾,庫爾勒四人,布古爾二人,俱七品。鄂爾沁伯克。掌數十人徵輸。葉爾羌屬鄂普爾一人,六品。雜布提墨克塔布伯克。掌教習經館。哲伯伯克。掌修造甲械。色依得爾伯克。掌巡察道路、園林果木。以上三員俱六品,葉爾羌置。什和勒伯克。掌驛館米芻。喀什噶爾,葉爾羌,各一人,俱六品。烏什,和闐,葉爾羌屬色勒庫爾,各一人,俱七品。六品伯克。掌修壩管臺。喀什噶爾二十一人,內兼管回兵藍翎玉資巴什三人。阿克蘇及所屬木蘇爾、達巴罕多蘭,葉爾羌屬喀爾楚、玉喇里克、塔爾塔克,各一人。七品伯克。掌司臺站。英吉沙爾,葉爾羌屬色勒庫爾、塔噶喇木,各一人。採鉛伯克。和闐屬克里雅城一人,五品。挖銅伯克。喀喇沙爾,庫爾勒及所屬布古爾,各一人。採銅伯克。阿克蘇三人。管銅伯克。庫車及所屬沙雅爾,各一人。自挖銅以下,俱七品。並隨事為員。由辦事大臣疏請。乾隆十九年,封吐魯番伯克莽里克扎薩克公,綜理回務,後獲罪,改封額敏和卓。置圖撒拉克齊佐之。三十四年,撫定西陲,因其舊名,置伯克等官。時隨征效力者,並封三品阿奇木,以葉爾羌授鄂對,喀什噶爾授色提巴爾第,庫車授鄂斯璊,和闐授漢咱爾巴,阿克蘇授達墨特,烏什授阿布都拉,是為六大城伯克,自三品至七品,各以授地為差。三品給二百帕籽特瑪帕地畝,種地人百名。四品百五十畝,人五十名。五品百畝,人三十名。六品五十畝,人十五名。七品三十畝,人八名。密喇布各員專司灌溉,例分地畝不再給,種地人各五名。徙阿克蘇回族駐伊犁,授茂薩額敏和卓次子。阿奇木。二十七年,伊犁建寧遠城,復移烏什、葉爾羌、和闐、哈密、吐魯番回族來駐,置大小各伯克。二十八年,定升補制。三十一年,移喀喇沙爾、庫爾勒回族駐庫轍瑪,省六品哈資一人,增四品阿奇木一人,與五品噶雜拉齊、七品玉資各伯克,並駐其地。三十八年,還駐庫爾勒,復舊制。嘉慶九年,依喀什噶爾、葉爾羌例,增伊犁六品巴濟吉爾、七品色迪爾各一人。道光八年,定三品至五品伯克由本城大臣填註履行,咨送喀什噶爾參贊大臣覆覈上聞,六、七品伯克咨送驗放。故事,大伯克回避本城,小伯克回避本莊,至申嚴禁令,葉爾羌屬色勒庫爾距卡倫遠,不在是例。並徙喀什噶爾五品訥克布、密圖瓦利、莫提色布依各三人駐罕愛里克,給五品阿奇木職銜,以六品哈資駐察拉根,主治農田,省阿斯圖阿爾圖什七品明伯克二人,徙一人佐之,別移一人駐阿爾瑚、抵補哈資。是歲以英吉沙爾事劇,賞六品哈資伊什罕銜,佐阿奇木治事。光緒十年,改建郡縣,俱省。以阿奇木、伊什罕職秩較峻,仍留原銜,俾別齊民。

籓屬各官外籓蒙古扎薩克,旗各一人,大漠內科爾沁等二十四部,旗四十有九。大漠外喀爾喀四部,旗八十有六。青海五部,旗二十有九。西套額魯特、額濟訥土爾扈特、杜爾伯特、土爾扈特、和碩特凡十部,旗三十有四。以王、貝勒、貝子、公、臺吉、塔布囊為之。不置扎薩克者,隸將軍、都統及大臣。掌一旗政令,協理臺吉二人或四人,唯土默特左翼旗、喀喇沁三旗稱塔布囊,與臺吉同。贊襄旗務。管旗章京各一人,副章京各二人,十佐領以下置一人。參領、六佐領置一人。佐領、百五十丁一人,或二百丁、或二百五十丁置一人。驍騎校,如佐領數。並佐扎薩克董理民事。回部哈密一旗扎薩克、協理臺吉、管旗章京、副章京各一人,參領二人,佐領十有三人。吐魯番一旗扎薩克一人,協理臺吉二人,管旗章京一人,副章京、參領各二人,佐領十有五人,伯克十人。所掌如蒙古制。

初定扎薩克綜理旗務,依內八旗編制,置管旗章京以次各官。順治十六年,置佐領、驍騎校百五十丁一人。嗣有所增益。十八年,定管旗章京、副章京員限。如前所列。雍正初,平青海,編旗置官如故事。

西藏達賴喇嘛一人,駐拉薩。掌全藏政令;班禪喇嘛一人,駐扎什倫布。掌後藏寺院與其教民:並受成於駐藏大臣。其屬:輔國公,一等臺吉,各一人。前藏唐古特三品噶布倫四人。掌綜理藏務。內一人喇嘛充補,不給頂戴。四品仔琫三人。掌稽商上事務。凡喇嘛庫藏出納之所曰商上。四品商卓特巴三人。掌庫務。五品葉爾倉巴、掌糧務。朗仔轄、掌治拉撒番民。協爾幫、掌刑名。碩第巴、掌治布達拉番民。六品達琫、掌馬廠。大中譯,各二人。六品卓尼爾、七品小中譯,各三人。以上三員,並掌噶廈事務。凡噶布倫議事之所曰噶廈。四品戴琫六人。五品如琫十有二人。六品甲琫二十有四人。七品定琫百二十人。第巴十有三人。管草一人,糌粑、柴、帳房各二人,門、牛羊廠各三人。五品邊營官二十有三人。江卡、喀喇烏蘇、官覺、補人、工布碩卡、絨轄爾營各一人。堆噶爾本、錯拉、拍克里、定結、聶拉木、濟嚨、博窩、達巴喀爾營各二人。喇嘛營一人,無頂戴。下同。大營官十有九人。桑昂曲宗、工布則崗、昔孜、協噶爾、納倉營各一人。乃東、瓊結、貢噶爾、侖孜、江孜營各二人。喇嘛營四人。六品中營官五十有九人。角木宗、打孜、作崗、江達、古浪、沃卡、曲水、突宗、僧宗、雜仁、鎖莊子、奪營,直谷、朗營,墨竹宮、卡爾孜、文扎卡、達爾瑪、聶母、拉噶孜、嶺營,嶺喀爾營各一人。洛隆宗、巴浪、仁本、仁孜、朗嶺、宗喀、撒噶、達爾宗、碩般多營各二人。桑葉、冷竹宗、茹拖、結登、拉里、沃隆、轄魯、策堆得、納布、錯朗、羊八井、麻爾江喇嘛營各一人,喇嘛營七人。七品小營官二十有五人。雅爾堆、拉歲、頗章、扎溪、色營,堆沖、汪墊、甲錯、瓊科爾結、蔡里、扎稱、折布嶺、扎什、洛美、嘉爾布營各一人。金東、撒拉、浪蕩、拉康、曲隆、朗茹、里烏、降、業黨、工布塘喇嘛營各一人。後藏唐古特三品大營官四人。拉孜喇嘛營二人。練金龍喇嘛營各一人。六品中營官十有七人。昂忍喇嘛營二人。仁侵孜、結侵孜寺、帕克仲、翁貢寺、千殿熱布結寺、托布甲、里卜、德慶熱布結寺、絨錯、央、蔥堆喇嘛營各一人。脅、千壩營各一人。喇嘛營二人。七品小營官十有六人。彭錯嶺喇嘛營二人。倫珠子、拉耳塘寺、達爾結、甲沖、哲宗、擦耳、晤欲、碌洞、科朗、扎喜孜、波多、達木牛廠喇嘛營各一人。凍噶爾、扎苦營各一人。僧官有國師、禪師、扎薩克大喇嘛、扎薩克喇嘛、大喇嘛、副喇嘛,並堪布監督之。藏地分衛、藏、喀木、阿里四部,各置噶布倫治其地,職任綦重。仔琫以降,為佐理國事官。戴琫以降,為各城典兵官。邊營官以降,為各城治民官。自國師至喇嘛,專司教事。置駐藏大臣轄之。昉自雍正三年,然猶未與達賴、班禪抗衡也。至乾隆五十七年,噶布倫以下始歸約束,大臣職權乃與埒。並增戴琫一人,原置五人,至是始定。如琫十有二人,定琫百二十人,升補各按其等差。其噶廈、小中譯、卓尼爾,擇東科譯言世家子弟。優秀者為之。

土司各官明代土司,淫昏暴戾,播州、水西、藺州、麓川,邊患如櫛。清鑒前轍,迭議歸流。曩昔土司隸外籓二,隸行省七。康、雍之盛,湖北散毛、舊為宣撫司,轄大旺安撫司,東流,臘壁二長官司。雍正十三年改來鳳縣。施南、舊為施州衛,轄忠建、忠孝二宣撫司,忠路、忠峒、東鄉五路、高羅、龍潭、金峒各安撫司,木冊、上愛茶峒、下愛茶峒、鎮南、搖把峒、鎮遠蠻彞、隆泰蠻彞、西萍蠻彞、劍南、思南、唐崖各長官司。雍正十三年改置恩施、宣恩、咸豐、利川四縣。容美,舊為宣慰司,轄盤順水、盡源、通塔坪各安撫司,椒山、瑪瑙、石梁、下峒、下岡、平茶、五峰、石寶各長官司。雍正十三年改置鶴峰州長樂縣。湖南永順、舊為宣慰司,轄施溶安撫司,下峒、田家峒、驢遲峒、臘惹峒、麥著黃峒、白崖峒各長官司,南渭、上溪二土官。雍正七年改置永順、龍山二縣。保靖、舊為宣慰司,轄五砦、筸子坪二長官司。雍正七年改縣。桑植舊為安撫司。轄美坪、朝南、那步、人士、黃河、魚龍、夾石、苦南、旱坪、蠶寮、金藏、拓山、爛洞、黃家、板山、龍潭、書洛十七峒,安福所上、下二峒。雍正七年改縣。及永綏、六里紅苗地。雍正八年改流官。乾州、鳳凰營,筸邊紅苗地。康熙四十三年改流官。並以生苗內附,列為郡縣。四川建昌、舊為指揮司。順治初改衛。雍正四年置寧遠府。松籓、舊為衛。雍正九年改流。天全、舊為六番招討司。雍正七年改流。打箭爐,舊為長河西魚通安遠宣撫司。雍正七年改流。廣西鎮安、舊為土府。康熙二年改流。泗城,舊為州。順治十五年升府。尋為土府。雍正五年改流。雲南開化、舊為教化、王弄、安南三長官司。康熙六年改流。昭通、舊為蒙地。雍正五年自四川來隸。明年改流。麗江、舊為土府。雍正初改流。鎮沅、舊為土州。雍正三年改流。四年自四川來隸。蒙化、舊為土府。康熙四年改流。威遠,舊為土州。雍正三年改流。明年自四川來隸。貴州威寧、舊為水西宣慰司。康熙元年置黔西府,改比喇塔為平遠府,大方城為大定府,四川馬撒為威寧。來隸後,改黔西諸府為州,並隸威寧。郎岱、雍正九年改流。歸化、康佐及仲苗地。雍正十二年改流。永豐,安籠長官司地。雍正五年改流。因時損益,遍置流官。乾隆以降,大小金川重煩兵力。酉陽、舊為宣慰司。乾隆元年改流。石砫,舊為宣撫司。二十七年改流。狉獉全革,猛緬炎荒,翕然內向。十三年改置緬寧。滇南邊徼,聞風震讋。三十一年討平莽匪,諸部內附,分置整賣、景線諸司。詳後。嘉、道之世,貴州守備、嘉慶二十五年省歸化屬一人。道光元年省安順府屬一人。四年省普安屬一人。十二年省普定縣屬一人。千總、道光元年省安順府屬二人。四年省歸化屬生苗枝、冊亨州同屬上分亭各一人。十年省普安縣屬上五苑枝一人。把總,道光元年省普定縣屬五人,郎岱屬六枝一人。四年省洛何枝、冊亨州同屬下分亭各一人。六年省平遠縣屬一人。八年省貞豐州羅浪亭一人。二十年省長塞一人。裁損尤多。光、宣之際,雲南富州、鎮康,四川巴塘、里塘、德爾格忒、高日、春科、瞻對、察木多,置吏一依古事。改巴塘曰巴安直隸經歷,駐鹽井。里塘曰順化縣巡檢,駐中渡河。鄉城曰定鄉縣縣丞,駐稻壩。並隸邊務大臣。兼轄明正、霍耳、五家、道塢、冷磧諸蠻部。廣西忠州、南丹、萬承、茗盈、全茗、結安、鎮遠、江州、下石西、上下凍、下雷、那地各州,羅白一縣,古零、定羅、安定、下旺諸巡司,永定長官司,永順副司,遷隆峒土官,停其襲職。向武、都康、安平、憑祥、思州諸州,上林、忻城、羅陽諸縣,東蘭、鳳山州同,上龍、白山、興隆諸巡司,代以漢官。覈衡厥實,隴沿舊制,湘、楚廓清,滇、蜀改流,十之三四。黔、桂長官州、縣,以今況往,弱半僅存,詳稽志乘,尚百數十。敘其世系,與其土地,凡武職非世襲,及番部僧官,附輯於後,庶有所考焉。

甘肅指揮使司:指揮使八人。正三品。平番縣屬三人:連城,順治元年魯宏襲;大營灣,九年魯之鼎襲;古城,十八年授魯大誥指揮同知,歲餘改襲。西寧府屬三人:南川,順治三年授納元按指揮僉事,雍正八年改襲;寄彥才溝,順治五年祁廷諫襲;北川,八年陳師文襲。河州屬一人:韓家集,舊為外委,乾隆六年韓世改襲。狄道州屬一人:臨洮衛,順治十六年趙樞勷襲。指揮同知七人。從三品。碾伯縣屬四人:趙家灣,順治元年趙瑜襲;上川口,五年李天俞襲;老鴉堡,六年阿世慈襲;勝番溝,祁國屏襲。平番縣屬一人:西大通峽口,魯培襲。俱九年授。西寧縣屬一人:起塔鎮,十年李珍品襲。河州衛沙馬族一人:順治二年何永吉襲。指揮僉事八人。正四品。洮州府屬一人:資卜,順治元年昝承福襲。平番縣屬一人:紅山堡,二年魯典襲。西寧縣屬二人:迭溝,十五年吉天錫襲;西川舊為外委,康熙四十年汪升龍改襲。碾伯縣屬三人:米拉溝,康熙十四年冶鼎襲;美都溝,三十七年甘廷建襲;硃家堡舊為外委,四十一年硃廷珍改襲。又洮州卓泥堡一人:舊為外委,四十五年楊朝樑改襲。千戶七人。正五品。河州保安撒喇四房、保安撒喇五族,平番,武威,永昌,古浪,碾伯各一人。副千戶二人。從五品。平番、洮州各一人。百戶九人。正六品。循化藏一人。平番、碾伯各二人。岷州四人。西寧千戶一人。巴彥南稱族。百戶二十有三人。蒙果爾津族、邕希葉布族、蘇魯克族、尼牙木錯族、庫固察族、稱多族、下扎武族、下阿拉克沙族、上隆壩族、下隆壩族、蘇爾莽族、多倫尼托克安都族各一人。阿里克族、扎武族各二人。格爾吉族三人。玉樹族四人。百長二十有六人。在黃河、大江、鴉礱江、瀾滄江、怒江各地。西藏百戶十有五人。納克書貢巴族、納克書色爾查族、納克書畢魯族、納克書奔頻族、納克書拉克什族、納克書達格魯克族、工⼙布納克魯族、依式夥爾族、勒納夥爾族、夥爾遜提麻爾族、上岡噶魯族各一人。工⼙布噶魯族、工⼙布色爾查族各二人。百長五十有二人。喀喇烏蘇河南岸各地。

四川宣慰使司:宣慰使七人。從三品。天全州屬一人:穆坪董卜韓胡,順治元年,堅參喃哈襲。茂州屬一人:瓦寺,九年授曲翊伸安撫司,康熙五十年論隨征西藏功,加桑朗溫愷宣慰司銜,嘉慶元年即真。雜穀屬一人:梭磨,雍正元年授長官司,乾隆十五年升安撫司,三十六年論隨征金川功,斯丹巴改襲。打箭爐屬四人:明正,康熙五年蛇蠟喳吧襲;布拉克底,四十年授綽布木凌安撫司,乾隆三十九年其孫阿多爾改襲;巴旺,乾隆二十九年綽布木凌長子囊索襲;德爾格忒,雍正六年授丹巴七立安撫司,十一年改襲:宣撫使司:宣撫使五人。從四品。越巂屬一人:工⼙部,康熙四十二年嶺南柱襲。西昌縣屬一人:沙麻,四十九年安鞏威襲。打箭爐屬三人:綽斯甲布,康熙四十一年授資立安撫司,乾隆四十年論隨征金川功改襲;里塘,康熙五十七年江擺襲;巴塘,五十八年羅布阿旺襲。安撫使司:安撫使十有六人。從五品。茂州屬一人:長寧,順治九年蘇廷輔襲。懋功屬一人:鄂克什,舊名沃日,十五年授巴碧太灌頂凈慈妙智國師,乾隆二十年色達拉改襲。鹽源縣屬二人:瓜別,康熙四十九年玉珠迫襲;木里,雍正八年六藏塗都襲。打箭爐屬十二人:單東革什咱,康熙三十九年魏珠布策凌襲;喇,四十九年阿倭塔爾襲;其雍正六年授者,霍爾竹緌,索諾木袞卜襲;霍耳章谷,羅卜策旺襲;瓦述餘科,沙克嘉諾爾布襲;霍耳甘孜孔撒,麻蘇爾特親襲;霍耳甘孜麻書,那木卡索諾木襲;霍耳咱,阿克旺錯爾恥木襲;春科,桑卜旺扎爾襲;林蔥,袞卜林親襲;上納奪,索諾木旺扎爾襲;下瞻對,策凌卜襲。副使二人。從六品。喇、春科各一人。長官司長官三十有七人。正六品。敘州府屬蠻夷、泥溪、平夷、沐川。龍安府屬陽地隘口。寧遠府屬威龍州、普濟州、河東、阿都、昌州、馬喇、工⼙部。雅州府屬沈邊、冷邊。瀘州屬九姓。打箭爐屬瓦述色地、上瞻對、茹,後隸西藏。瓦述毛丫、瓦述崇善、瓦述曲登、瓦述啯嚨、納林沖、瓦述更平、霍耳白利、霍爾東科、春科高日、蒙葛使結。理番屬從噶克、卓克採、丹壩各一人。副長官一人。正七品。阿部。千戶四十有一人。咱理松坪、雙則紅凹寨、班俗寨、川柘寨、佘灣寨、祈命寨、寒昐寨、商巴寨、穀爾壩、那浪寨、竹當寨、包子寺寨、甲多寨、墨蒼寨、阿強寨、呷竹寺、丟穀寨、雲昌寺、沙壩、阿裏洞寨、峨眉喜寨、七布寨、毛革阿按寨、麥雜蛇灣寨、酥州、黎溪州、迷易所、鹽井衛中所、左所、右所、古柏樹、瓦述寫達、瞻對峪納、上納奪、中郭羅克、押落寨、中阿樹、上瞻對、撒墩木期、古土拖車、阿朵阿與各一人。百戶百五十有九人。打箭爐屬八十有三人。松潘屬四十有一人。冕寧縣屬十有三人。馬邊屬六人。茂州屬四人。鹽源縣屬、會理州屬各二人。清溪縣屬、峨邊屬各一人。

廣西長官司:長官二人。慶遠府屬永定、永順各一人。副長官司:二人。永順。

雲南指揮使司:指揮使二人,普洱府屬孟艮,古孟掯,召丙襲;整欠,叭光捧襲。俱乾隆三十一年授。指揮同知一人。廣西州屬猛龍,乾隆三十一年叭護猛襲。宜慰使司:宣慰使一人。普洱府屬車里,古商產里,順治十八年刀穆禱襲。乾隆三十八年省,四十二年復故。土地十三版納:寧洱縣五,思茅八。宣撫使司:宣撫使七人,直隸耿馬一人:罕悶睆襲。騰越屬三人:南甸,古南宋,刁呈祥襲;隴川,古平緬,多安靖襲;乾崖亦曰平賴睒、渠瀾睒、刁建勛襲。俱平滇後授。永昌府屬一人:孟連亦曰哈瓦,舊為長官司,康熙四十八年刁派鼎改襲。普洱府屬二人:整賣,召納提襲;景線,吶賽襲。俱乾隆三十一年授。古八百媳婦國地。副使三人。騰越屬猛卯、盞達,龍陵屬遮放,各一人。安撫使司:安撫使二人。龍陵屬潞江,古怒江,甸線有功襲。芾市,唐書「芒施蠻」,放愛眾襲。俱平滇後授。長官司:長官三人,騰越屬戶撒臘撒,臨安府屬納樓、茶甸,各一人。副長官司二人。大理府屬十二關,臨安府屬虧容甸,各一人。土千戶一人。虧容甸。

貴州長官司:長官六十有五人。貴陽府屬中曹、養龍、白納、虎墜,定番州屬程番、小程番、上馬橋、盧番、方番、韋番、臥龍番、小龍番、金石番、大龍番、木瓜、麻鄉,開州屬乖西,龍里縣屬大谷龍、小谷龍、羊腸,貴定縣屬平伐、大平伐、小平伐、新添,修文縣屬底寨,永寧州屬頂營、募役、沙營,平越府屬楊義,黃平州屬巖門,都勻府屬都勻、邦水,麻哈州屬樂平、平定,獨山州屬豐寧上、豐寧下、爛土,鎮遠府屬偏橋,鎮遠縣屬邛水,思南府屬隨府辦事、蠻夷、沿河、祐溪、朗溪,思州府屬施溪,銅仁府屬省溪、提溪、烏蘿、平頭,黎平府屬潭溪、八舟、龍里、中林、古州、新化、歐陽、亮寨、湖耳、洪州,各一人。思州府屬都平、都素、黃道,各二人。副長官司:十有九人。白納、木瓜、乖西、底寨、都勻、蠻夷、都素、沿河、祐溪、朗溪、省溪、提溪、烏蘿、平頭、歐陽、湖耳、洪州、鎮寧縣屬康佐,石阡府屬石阡,各一人。偏橋左、偏橋右,各二人。邛水一人,後改七品土官。

四川土通判二人。石砫屬一人:順治元年授馬祥麟宣慰司,乾隆間,孔昭緣事降。雜穀屬一人:陽地隘口,順治六年王啟睿襲。土知事一人。龍安府屬龍溪堡,順治六年薛兆選襲。土巡檢二人。茂州屬牟托水、草坪,各一人。副巡檢一人。茂州竹木坎置。

廣西土知州二十有五人。歸順直隸州屬一人:上映,順治元年許國泰襲。慶遠府屬二人:南丹,是歲莫自乾襲;那地,九年羅德壽襲。並古蠻地。南寧府屬三人:歸德,莫道襲;果化,趙國鼎襲;忠州,黃光聖襲。鎮安府屬三人:下雷,許文明襲;向武,黃嘉正襲。俱元年授。都康,馮太乙襲,九年授。太平府屬十有六人:下石西,閉承恩襲;田州,岑廷鐸襲。俱元年授。萬承,許嘉鎮襲;思陵,韋懋遷襲;憑祥,李維籓襲;太平,唐波州地,李開錦襲;茗盈,李應芳襲,全茗,許家麟襲;結安,張邦興襲;佶倫,馮家猷襲;龍英,趙廷耀襲;都結,農廷封襲;江州,黃廷傑襲;上下凍,趙應琛襲;鎮遠,趙秉業襲。俱十六年授。其田州,光緒元年改流,置恩隆縣。土州同一人。東蘭州,順治九年韋光祚襲知州。雍正七年,朝輔緣事降普安州。康熙四十一年廢。土知縣四人。百色屬一人:上林,順治元年黃國安襲。慶遠府屬一人:忻城,九年莫猛襲。太平府屬二人:羅陽,黃啟祚襲;羅白,梁徵鼐襲。俱十六年授。土州判一人。舊土田州地。乾隆七年析置陽萬,一人。光緒五年改流。置恩陽分州。土巡檢九人。太平府屬上龍司,思恩府屬白山司、興隆司、定羅司、舊城司、安定司、都陽司、古零司,百色屬下旺司,各一人。從九品土官一人。思恩府轄。其不管理土峒者,正六品土官二人,從六品、正八品、正九品土官各一人,從九品土官一人,未入流土官二人。

雲南土知府二人。永昌府屬孟定,古景麻甸,罕宋襲;永寧,阿鎮麟襲。俱順治元年授。後永寧改隸永北。其景東、蒙化二人,俱康熙四年改流。土同知一人。隸廣南府,順治十六年儂鵬襲。土知州四人。永北屬一人:蒗蕖,康熙間改土舍,道光十七年阿為柱改襲。永昌府屬一人:灣甸,古細睒,景文智襲。明史誤「刀」姓。鎮康州一人:古石睒,刀悶達襲。明史誤「刀孟」。俱順治十六年授。土州同三人。永北屬順州,於祿祥襲。鎮南州,段光贊襲。姚州,高顯爵襲。俱順治十六年授。州同職銜一人。隸武定州。順治十六年授那天寵暮連鄉土目。雍正八年升那德洪千戶。同治元年那康保改襲。土州判二人。鎮南州,順治十六年陳昌虞襲。新興州,康熙二十二年王鳳襲。土知事一人。景東,順治十六年陶啟濱襲。土縣丞五人。楚雄、平彞、新平、蒙化、南澗各一人。土主簿二人。雲南、孟遠縣各一人。土典史一人。浪穹縣置。土巡檢十有九人。羅次縣練象關,祿豐縣南平關、湯郎馬,趙州定西嶺,浪穹縣蒲陀崆、鳳羽鄉、上江嘴、下江嘴,鄧川州青索鼻,雲龍州箭桿場,臨安府納更山,廣通縣回磴關、沙矣,舊景東保甸、三岔河,順寧府猛猛、大猛麻,鶴慶州觀音山,鎮南州阿雄關、鎮南關,各一人。土驛丞三人。鶴慶州在城驛、板橋驛、觀音山,各一人。其不管理苗裔村寨者,土通判二人,麗江府、鶴慶州,各一人。正八品土官一人。嘉慶三年省經歷置。

貴州土同知二人。鎮遠府屬一人:何大昆襲。獨山州屬一人:蒙一龍襲。俱順治十五年授。土通判、鎮遠府,順治十五年楊世基襲。土推官,鎮遠府,順治十五年楊秀瑋襲。各一人。土縣丞五人。安化、印江、餘慶縣,各一人。甕安縣屬甕水司、草塘司,各一人。土主簿二人。安化、餘慶縣,各一人。土吏目一人。黃平州重安司。土巡檢二人。永寧州盤江、安化,各一人。其不管理土峒者,正六品、正七品土官各一人,正八品土官三人,正九品、從九品土官各二人。右文秩凡七階。承襲、革除、升遷、降調隸吏部。

四川土游擊,駐越巂暖帶密。康熙四十九年授嶺安泰千戶。同治二年改襲。土都司,駐越巂松林地。康熙四十九年授王德洽千戶。同治二年改襲。各一人。屯守備十有二人。撫邊屯屬一人:攢拉別思滿阿忠本襲。章穀屯屬一人:攢拉宅龔阿安本襲。崇化屯屬一人,促浸河西固拉約爾瓦襲。懋功屯屬二人:攢拉八角碉木塔爾襲,攢拉漢牛工噶襲。松潘屬四人,雜穀腦沙加豆日襲,上孟董美諾更噶豆日襲,下孟董沙馬班馬襲,九子寨楊阿太襲。乾保寨二人:阿忠暨阿忠保襲。俱乾隆間授。土千總七人。西昌縣屬河西,雷波千萬貫,瓘邊屬瞻巴家、哈納家、蜚瓜家、魁西家,各一人。屯千總十有九人。促浸河西三人。雜穀腦、乾保寨、上下孟董、九子寨、促浸河東各二人。攢拉八角碉、攢拉漢牛、攢拉別思滿、攢拉宅龔,各一人。土把總七人。河西、千萬貫、膽巴、納哈、魁西,各一人。蜚瓜二人。屯把總三十有四人。促浸河西六人。雜穀腦、乾保寨、上下孟董、九子寨,各四人。攢拉漢牛、攢拉別思滿、促浸河東,各二人。攢拉八角碉、攢拉宅龔,各一人。

雲南土都司一人。駐鎮邊府大雅口。光緒十三年錄李芝龍隨征惈黑功授職。土守備五人。思茅屬二人:六本猛齋襲,景海猛彪襲。俱乾隆十三年授。騰越屬一人:茨竹寨,是歲授左正邦把總。道光二十一年,錄大雄隨征雲州烏土各寨功改襲,加明光宣慰司銜。鎮邊屬二人:蠻海,咸豐十年授石朝龍把總,光緒十三年,錄大餘隨征惈黑功改襲:大山,咸豐九年授石麟千總,光緒十三年,錄朝鳳平東王惈匪功改襲。土千總十有八人。雲龍州老窩六庫,維西奔子欄、阿墩子,思茅猛遮,寧洱府普藤、猛勇,威遠猛戛,騰越杉木籠隘,保山縣登梗、魯掌,永北羊坪,鎮邊猛角、猛董、圈糯、黃草嶺,新平縣斗門、磨沙補哈,順寧府猛撒,各一人。土把總三十有六人、雲龍州漕澗,臨安府稿吾卡,維西奔子欄,臨城瀾滄江、其宗喇普,思茅倚邦、猛遮、易武、猛臘、六順、猛阿、猛籠、橄欖壩,寧洱縣猛旺、整董,他郎儒林等裡、定南等裡,威遠猛戛、猛班,騰越大塘隘、明光隘、古勇隘,保山縣卯照,鎮邊下猛、引賢官寨、兼募乃寨、東河,元江州永豐里、茄革裏,新平縣喇博、他旦、老是達巖、旺瓦遮宗、哈正掌寨,各一人。又寧洱縣猛烏、烏得,各一人,光緒二十一年,割隸法蘭西。

甘肅土守備一人。洮州資卜族,世系無考。土千總十有六人。寄彥才溝、西川、起塔鎮、趙家灣、美都溝、米拉溝、西寧縣陳家臺、納家莊,各一人,資卜、勝番溝,各二人,上川口四人。土把總二十人。寄彥才溝、陳家臺、納家莊、起塔鎮、西川、趙家灣、美都溝、米拉溝,各一人,資卜二人,勝番溝四人,上川口六人。

貴州土千總十人。貴陽府屬青巖、吉羊枝,龍里縣屬大谷龍、羊腸,麻哈州屬養鵝,都江屬順德、歸仁,丹江屬雞講、黃茅、烏疊,各一人。土把總一人。小谷龍。其不管理村寨者,湖北世襲千總銜十人。江夏縣屬四人。漢陽縣屬、孝感縣屬各三人。把總銜五人。漢陽一人。孝感四人。湖南千總銜十有三人。石門縣屬、慈利縣屬各六人。永定縣屬一人。把總銜五十有二人。石門縣屬二十有二人。慈利縣屬二十有六人。桑植縣屬二人。龍山縣屬、永定縣屬各一人。貴州六品武土官二人。貴陽府屬、思南府屬各一人。七品武土官四人。鎮遠府屬三人。石阡府屬一人。右武秩凡五階。承襲、革除、升遷、降調,隸兵部。

武職非世襲者,雲南土守備三人。麗江府一人。中甸、迭巴二人。土千總七人。麗江府二人。大中甸神翁、小中甸神翁、中甸江邊神翁、中甸格咱神翁、中甸泥西神翁,各一人。土把總十有五人、中甸迭賓五人。小中甸迭賓、中甸江邊迭賓,各二人。中甸格咱迭賓、中甸泥西迭賓,各三人。土官二十有六人。中甸轄二十三人。麗江府木氏轄三人。初皆世襲。雍正二年改拔補。

番部僧官甘肅珍珠族國師、禪師,化族國師,靈藏族禪師,各一人。初隸河州。後珍珠、靈藏屬循化,餘雜處二十四關。禪定寺禪師,嘉慶十九年無人襲。由土司兼轄,隸洮州。番寺禪師,同治間回變後,不修職貢。各一人。垂巴寺、轄番人十族。著洛寺、轄番人二十三族。麻你寺轄番人二十一族。僧綱,圓成寺、轄番人四族。閻家寺後無人襲。僧正,各一人。


\end{pinyinscope}