\article{列傳一百七十}

\begin{pinyinscope}
張文浩嚴烺張井吳邦慶慄毓美

麟慶潘錫恩子駿文

張文浩,順天大興人。入貲為布政司經歷,投效東河,工竣,發南河。嘉慶十年,授山清外河同知,屢以河溢奪職,尋復之,補外河南岸同知。十九年,河督吳璥奏調赴睢工委用,擢署淮海道。二十四年,河溢儀封,復決武陟馬營壩,調辦馬營壩工,工竣,賜花翎。儀封決口猶未塞,仁宗以吳璥年老,命文浩署河東河道總督,專駐工次。疏陳築壩挑河估銀四百五十萬,報可。工竣,晉二品頂戴,兼兵部侍郎銜。道光元年春,欽天監奏彗星出東壁,分野在衛地,占主大水,敕文浩防範。侍郎吳烜請加高河堤,文浩疏言:「河灘高下不齊,長堤千餘里,未能一律增高,請加子堰二三尺。」從之。實授河道總督。三年,丁母憂,服未闋,以畿輔連年水患,召署工部侍郎,偕三品卿繼昌勘南北運河及永定河漫溢。詔繼昌還。文浩駐工會辦。工竣,與總督蔣攸銛合疏陳:「直隸河道漫水未涸,無從查勘,考詢各處堤墊,無不堙塞殘廢。每年二月方可動工,五月即須停止,工繁時促,斷難同時興作。請於來歲春融,周歷履勘,分別緩急估辦。」又言:「永定河為患,固由下口不能暢流,亦由上游無所宣洩。請修築重門閘,添設減水壩。又近年河流每多側注北岸,宜添築越堤以為重障。」

四年春,授江南河道總督。其秋,糧艘回空,黃河高於清水,停阻河北者數月,詔切責,降三品頂戴,命設法蓄清以資浮送。十一月,始全數渡黃。會洪澤湖漲水未消,高堰十三堡堤潰萬一千餘丈,山盱、周橋、悉浪菴亦過水八九尺,各壩漫溢。宣宗怒,褫文浩職,命尚書文孚、汪廷珍馳勘,劾文浩禦黃壩應閉不閉,五壩應開不開,蓄清過旺,以致潰決。命於工次枷號一月,遣戍新疆。回疆軍事起,隨營效力,事平,請釋回,不許。十六年,卒於戍所。

嚴烺,字小農,浙江仁和人。嘉慶中,入貲為通判,發南河,累擢徐州道,丁母憂。道光元年,服闋,授河南河北道。尋命以三品頂戴署河東河道總督,三汛安瀾,乃實授。汶水漫決既塞,疏言:「運河北路以蓄汶敵衛為最要機宜,必使汶水層層抬高,然後能敵衛水。請加高臨清口磚閘資收蓄。」從之。初,黎世序治南河多用碎石,乃奏請敕東河仿行,烺取其說,請於馬營北岸挑壩,仿南河拋護碎石,估工需銀十萬兩。布政使程含章、巡撫姚祖同先後言其不便,而馬營既放淤,壩前水勢已緩,烺仍請於壩尾沁水灌注之所拋護碎石,從之。

四年,南河高家堰潰決,調烺江南河道總督。五年,與尚書文孚、汪廷珍合疏陳:「蓄清敵黃為河務第一關鍵。蓄清全賴湖堤,堤潰則清水洩枯,重運經臨,無以資浮送。擬遵古人成法,借黃濟運。所慮運河窄小,黃流湍悍,多則不能容納,少則必致膠淺。議於御黃壩外建壩三道,鉗束黃流,俾有節制。又添築纖道,以資束水行纖。里、揚兩長河挑挖淤淺,幫培堤身,並豫儲料物,隨時築壩,逼溜刷淤。禦黃壩未啟,則先挑高堰引河,導清水入運;將啟,則嚴堵束清,杜黃水入湖。至修復湖堤,必乘天寒水涸,取土較易。擬就近採料,限大汛前砌高十層,備湖水漸長。共需帑銀三百萬。」又議覆侍郎硃士彥條上南河事宜,大要:「拆修高家堰壩工,先築越壩以便工作,並於石堤外拋碎石坦坡,可期永無塌卸。又於王家壩減壩內鹽河加築堤埽,及仁、義、禮舊壩處所添建石滾壩,以防異漲。」並如議行。於是偕孫玉庭等會辦重運。至五月禦黃壩啟放後,河道仍淺滯,漕船不能通行,就近盤壩,剝運難繼,玉庭被重譴,烺亦鐫級留任。

烺既因濟運事不敢擅離,不能巡河勘工,兩江總督琦善以為言,乃命烺周歷履勘,仍諭蓄足清水,為來年敵黃濟運之計。烺疏言:「從前黃河底深,湖水收至數尺,即可外注,堤身不甚吃重。今則湖水必蓄至二丈,始可建瓴而刷黃。以四百里浩瀚之湖水,恃一線單堤為之護,西風沖擊,勢必潰決。擬仿成法,於堤外築碎石坦坡,護堤既固,則湖水可蓄。」又偕琦善奏陳:「刷黃必須湖水收至二丈。上年楜水丈七寸餘,即致失事。刻下清水萬難蓄足,惟有蓄清減黃二法並行。碎石護堤,所以蓄清;改移海口,所以減黃。」詔妥籌具奏。尋又會陳:「由王營減壩至灌河口,可導黃入海。查灌河口外海灘高仰,轉無把握,惟拋碎石坦坡,可漸收蓄清刷黃之益,需費六百餘萬,應分年辦理。」

六年,洪湖石工既竣,烺知工未堅固,實不足恃,遂堅主碎石之工,每年拋石三十萬方,八年始能告成。宣宗怒斥:「烺調任以來,一籌莫展。禦黃壩至今不能啟放,辦理不善。念在東河修守尚無貽誤,降三品頂戴。」署河東河道總督;七年,實授,復二品頂戴。以蘭陽柴壩西北頂沖,前拋碎石已著成效,遇伏秋汛漲,仍形吃重,請加寬坦坡。八年,請續拋下北、蘭儀兩碎石,並於中河、祥河險工儲石備防。十一年,命侍郎鍾昌等抽查東河料垛,祥河、曹考兩料垛虛松殘朽,烺坐失察,降三品頂戴,鐫四級留任。尋以病請開缺。

十三年,病痊到京,疏陳浙江海塘事宜。十四年,命偕侍郎趙盛奎往勘,請分別緩急,改修柴埽,以護塘根,歲撥銀五萬備修費,從之。尋命毋庸在工督辦。復以病乞歸。十五年,河東河道總督吳邦慶劾烺虛拋碎石,並收受紅封盤費,以運同降補。二十年,卒。

張井,字芥航,陜西膚施人。嘉慶六年進士,以內閣中書用,改知縣,銓授廣東樂會。引見,特命改河南正陽,調祥符,遷許州直隸州知州。襄辦馬營壩大工,加知府銜,署汝寧知府。道光四年,擢開歸陳許道。尋以三品頂戴署河東河道總督。五年,秋汛安瀾,乃實授。增培黃河兩岸堤工,並修泉河堤,濬各湖斗門引渠,疏陳河工久遠大計,略曰:「今日之黃河,有防無治。每遇伏秋大汛,司河各官奔走搶救,竭蹶情形,惟日不足。及至水落霜清,則以目前可保無虞,不復求疏刷河身之策。漸致河底墊高,清水不能暢出,並誤漕運。又增盤壩起剝及海運等費,皆數十年來斤斤於築堤鑲埽,以防為治,而未深求治之之要有以致之也。當此河底未能疏濬之時,惟仍守舊規,以堤束水,而水不能攻沙,河身日形淤墊,必得有刷深之方,始可遂就下之性。」宣宗韙其言,命偕兩江總督琦善、南河總督嚴烺、河南巡撫程祖洛籌議,遂赴南河會勘。

六年,疏言:「黃河病在中滿,淤墊過甚,自應因勢利導。擬仿前大學士阿桂改河避險之法,導使繞越高淤,於安東東門之北別築新堤,以北堤改作南堤,中間抽挑引河,傍舊河而行。至絲網濱以下,仍歸海口,無淤灘阻隔,似可暢順東趨。去路既暢,上淤必掣深,得黃與清平,立啟禦黃壩,挑逼清水暢出刷黃,自有建瓴之勢。」詔嘉其有識,調江南河道總督,與總督琦善及副總河潘錫恩會議。以改河避淤,口門有碎石阻遏,諸多窒礙,請開放王營減壩,以期減落黃水,刷滌河身,從之。

既而給事中楊煊奏「啟放減壩,黃流湍急,鹽河勢難容納,恐滋流弊」,援嘉慶間減壩兩次漫口情形為證。復下詳議,井言:「煊稽考成案,於今昔情形似未周知。昔年開壩漫口時在五月,本年啟放定在霜後,來源無慮續漲。惟現據委員稟稱,去路未見通暢,是煊所奏不為無見。因思啟壩時水勢或可暢達,堵合後全河仍必抬高,恐徒深四邑之災,無補全河之病。請仍改河避淤。」上斥井持論游移,不許。是秋,開放減壩,如期堵合,被褒獎。七年,春汛,黃水倒漾,仍高於清水,御壩驟難啟放,漕船倒塘灌運,自請治罪,降三品頂戴。命大學士蔣攸銛、尚書穆彰阿往勘。會黃水低落,啟御壩,運船幸得全渡,詔斥井急於求功,泥於師古,革職留任,以觀後效。

八年,疏陳要工四事:黃河接築海口長堤,並於下游多築埽壩以資刷掣;洪澤湖添建滾壩,加寬湖堤;南運河移建昭關壩,加幫兩岸纖堤;北運河修復劉老澗石滾壩,補還南岸纖堤。命都統英和會同蔣攸銛查勘,以添築埽壩不能疏通積淤,海口築堤可從緩辦,餘如議行。九年,以兩屆安瀾,復二品頂戴,諭相機規復河湖舊制。疏言:「南河利害,全系清江,必清水暢出,助黃刷淤,則河與漕兩治。惟黃水積淤,必清高於黃數尺,又必啟壩時多、閉壩時少,乃能暢出滌刷。現在清水能出,僅免倒灌,不誤漕行,殊未易收刷滌之效。」十二年,桃源縣民聚眾私掘官堤掣溜,致成決口,革職,暫留任效力。御史鮑文淳、宗人府府丞潘錫恩並言黃水入湖,恐妨運道,命穆彰阿、陶澍會勘籌議。疏陳:「黃水入湖後,即由吳城七堡仍入黃河,僅淤沿堤,不及湖中,未入束清壩,不致病及運河。正河乾涸,正可將桃南、桃北兩間大加挑濬,除去中滿之患。」十三年,於家灣合龍,予四品頂戴。尋引疾歸。十五年,卒於家。

井任兩河凡十年。初治南河,銳意任事,洎興大工,糜帑三百餘萬而無成效,仍為補苴之計,用灌塘法,較勝借黃之險。勤於修守,世稱其亞於黎世序云。

吳邦慶,字霽峰,順天霸州人。以拔貢官昌黎訓導。嘉慶元年,成進士,選庶吉士,授編修,遷御史。巡視東漕,奏請重浚運河,並復山東春兌春開舊制。數論河漕事,多被採用。十九年,擢鴻臚寺少卿,命偕內閣學士穆彰阿督濬北運河。累遷內閣侍讀學士。二十年,出為山西布政使,調河南,護理巡撫。二十三年,擢湖南巡撫,調福建,未之任,湘潭土客民群鬥,死傷甚眾。侍郎周系英面陳與邦慶疏奏有異,命總督慶保往按。邦慶亦發系英私書,系英獲譴;邦慶鐫級,以三品京堂用,補通政使。二十五年,擢兵部侍郎,調刑部,尋授安徽巡撫。

黃水注淮,鳳、潁被災,而皖南苦旱,親赴災區賑撫。涇縣民徐飛瀧傷斃,邦慶誤聽承審官謂由於徐孝芳捏傷圖賴,奏捕之,激眾拒捕。命兩江總督孫玉庭鞫治,得其狀,詔斥邦慶幾釀冤獄,部議革職,予編修。累遷少詹事。道光十年,授貴州按察使,未之任,予三品卿銜,署漕運總督,尋實授。禁糧船裝載蘆鹽,請緝拿沿河窩頓。十一年,調江西巡撫。

十二年,授河東河道總督,以不諳河務辭,不許。初,嚴烺在東河,多用碎石拋護,歷年歲料未有節省,詔飭覈減。邦慶疏請:「酌改舊章,每年防料經費四成辦稭,六成辦石。蘭儀、商虞、下北三現工險要,仍專案請辦碎石。所議六成之石,積儲數年,使各皆存二千,方緩急可恃,則專案之石亦可逐年遞減。」從之。武陟攔黃堰民築民修,嗣歸管,工段歲增。十三年,奏定畫界立石,官民分守,如有新生埽工,先借帑辦理,按河北三府攤徵歸款。以山東運河全賴泉源灌注,請復設泉河通判,以專責成。壽東汛滾水壩外舊有土堰,為蓄汶敵衛,以利漕運,大水鄉民私開釀事,奏立志椿。濟運之水以七尺為度,重運過竣,啟堰以利農田,如議行。

初,邦慶著畿輔水利叢書,後在官,考河南通省志乘所載有水田處,臚列其水之衰旺,溉田多寡之數,為渠田說。修防之暇,率道捐貲造水車,就馬營壩北及蔡家樓大窪積水地七千餘畝試行墾治。先是,邦慶因碎石工劾嚴烺,罷之。既而給事中金應麟亦劾邦慶保舉過濫,動撥過多,十五年,命大學士文孚、山東巡撫鍾祥按之,坐違例調地方人員改歸河工,及以屬員為幕僚,員饋銀不奏參,褫職。詔復斥其參劾嚴烺遲至三年之久,亦屬取巧,念在任三屆安瀾,加恩復予編修。年已七十,遂告歸。二十八年,卒。

慄毓美,字樸園,山西渾源人。嘉慶中,以拔貢考授佑縣,發河南。歷署溫、孟、安陽、河內、西華,補寧陵,所至著績。父憂歸,道光初,服闋,補武陟。遷光州直隸州知州,擢汝寧知府,調開封。歷糧鹽道、開歸陳許道、湖北按察使、河南布政使,護理巡撫。十五年,擢河東河道總督。

毓美自為令時,於黃、沁堤工,馬營壩工皆親其事,勤求河務。時串溝久為河患,串溝者,在堤河之間,始僅斷港積水,久而溝首受河,又久而溝尾入河,於是串溝遂成支河,而遠堤十餘里之河變為切近堤身,往往潰堤。毓美蒞任,乘小舟周歷南北兩岸,時北岸原武汛串溝受水已三百丈,行四十餘里,至陽武,溝尾復灌入大河;又合沁河及武陟、滎澤諸灘水畢注堤下。兩汛素無工無稭,石堤南北皆水,不能取土築壩。毓美乃收買民磚,拋成磚壩數十所。工甫就而風雨大至,支河首尾皆決數十丈而堤不傷,於是始知磚之可用。疏陳辦理情形,以圖說進。

尋又疏言:「王屋莊進水之口,較前更寬百餘丈,由中泓大灘益向南淤,溜勢南緩而北緊。南股正河成為迂道,北股之溜勢轉建瓴。其故由廣武山前老灘坍千餘丈,溜趨山根,為山所遏,折回東北,中泓挺生淤灘。水口既日見刷寬,從省估計,約需銀十餘萬兩。至原陽兩岸堤根,因沿陂試拋磚塊,深資偎護。月石壩堵合,加高幫寬,迤下楊村、封丘二汛,灘水已停淤,壩下七十餘村莊居民安堵。惟串溝分溜,關系北岸全局,不能緩至來年興工,已借撥銀兩估辦。」允之。是役支河危險,賴磚工化險為平。

尋偕巡撫桂良勘奏:「老河分溜已有六分,王屋莊口寬勢順,磚土各壩未可深恃。原武十六堡當其頂沖,並有秦家廠、鹽店莊各灘水串溝分注,十七堡當支河尾閭皆險要,請購料豫防。」如議行。十六年,擇要挑濬修築魚臺汛堤岸,改民堰歸運河。十八年,旱,漕艘阻滯。濬泉源及各湖進水渠道,嚴諸閘啟閉。又濬曹州、濟寧河渠。十九年,奏定微山湖收納運水章程,但計水存丈三尺以內,即築壩蓄水,加高戴村壩以防旁洩。

初,毓美以磚工屢著成效,奏請許設窯燒造。御史李蓴疏言其不便,命尚書敬徵往勘,仍請改辦碎石,停止設窯。毓美上疏爭之曰:「豫省歷次失事,皆在無工處所。堤長千里,未能處處籌備。一旦河勢變遷,驟遇風雨,輒倉皇失措。幸而搶護平穩,埽工費已不貲。鑲埽引溜生工,久為河工所戒,昧者轉謂非此別無良策。查北岸為運道所關,往者原陽分溜,幾掣動全河,若非用磚拋護,費何可數計?今祥符下汛、陳留一汛灘水串注,堤根形勢,正與北岸同。濱河士民多有呈請用磚者,誠有見於磚工得力,為保田廬情至切也。夫事之有利於民者,斷無不利於國。特事近於創,難免浮言。前南河用石之始,眾議紛如,良由工程平穩,用料減少,販戶不能居奇。工簡務閒,游客幕友不能幫辦謀生,是以妄生浮議,賴聖明獨斷,敕下東河試辦,至今永慶平成。惟自用碎石,請銀幾七十餘萬,嗣改辦六成碎石,然因購石不易,埽段愈深愈多,經費仍未能節省。自試辦磚壩,三年未生一新工,較前三年節省銀三十六萬。蓋豫省情形與江南不同,產石祗濟源、鞏縣,採運維艱。磚則沿河民窯不下數十座,隨地隨時無誤事機。且石性滑,入水流轉,磚性澀,入土即黏,卸成坦坡,自能挑溜。每方磚塊直六兩,石價則五六兩至十餘兩不等。碎石大小不一,堆垛半屬空虛。尺磚千塊為一方,平鋪計數,堆垛均實。每方石重五六千斤,而磚重九十餘斤,是一方石價購磚兩方,而拋磚一方可當石兩方之用也。或謂磚塊入土易損裂,不知磚得水更堅,拋成磚壩,一經淤泥,即已凝結;或謂拋築磚壩,近於與水爭地,不知堤前之地,尺寸在所必爭。自來鑲埽之法,堤前必先築土壩數十丈,然後用埽鑲,設磚壩則無須乎埽。師土壩之意,不泥其法,拋作坦坡,大溜自然外移,未有可築土壩而不可築磚壩者。上年盛漲,較二年及十二年尤猛迅,磚壩均屹立不移。儀睢、中河兩,河水下卸,塌灘匯壩,搶鑲埽段,旋即走失,用磚拋護,均能穩定。是用磚搶辦險工,較鑲埽更為便捷。昔衡工失事,因灘陷不能鑲埽;馬工失事,因補堤不能得碎石。使知用埽不如用磚,運磚易於運石,則費省而工已固。現在各無工之處,串溝隱患,必應未雨綢繆。若於黃、沁下南豫儲磚塊,則可有備無患。應儲之磚,仍令向民間採買,不必員燒造,此外別無流弊。」卒如所議行。遂請以四成辦稭之款改辦磚塊。

又疏言:「從前治河用卷埽法,並有竹絡、木囷、磚石、柳葦。自用料鑲埽,以稭料為正宗,而險無定所,亦無一勞永逸之計。緣鑲埽陡立,易激水怒。其始水深不過數尺,鑲埽數段,引溜愈深,動輒數丈,無工變為險工。溜勢上提,必須添鑲;溜勢下坐,必須接鑲。片段愈長,防守愈難。新工既生,益形勞費。埽工無法減少,不得已而減土工,少購碎石,皆為茍且因循之計。自試拋磚壩,或用以杜新工,或用以護舊工,無不著有成效。且磚工不特資經久,而堆儲亦無風火堪虞。從此工固瀾安,益復培增土工,專用力於根本之地,既可免漫溢之患,亦保無沖決之虞。」宣宗深嘉納之。巡撫牛鑒入覲,諭以毓美治河得手,遇事毋掣其肘。二十年,京察,特予議敘。尋卒,優詔褒惜,贈太子太保,依總督例賜恤,賜其子燿進士,謚恭勤,祀名宦祠。

毓美治河,風雨危險必躬親,河道曲折高下鄉背,皆所隱度。每曰:「水將抵某所,急備之。」或以為迂且勞費,毓美曰:「能知費之為省,乃真能費者也。」水至,乃大服。在任五年,河不為患。歿後吏民思慕,廟祀以為神,數著靈應,加封號,列入祀典。

麟慶,字見亭,完顏氏,滿洲鑲黃旗人。嘉慶十四年進士,授內閣中書,遷兵部主事,改中允。道光三年,出為安徽徽州知府,調潁州,擢河南開歸陳許道。歷河南按察使、貴州布政使,護理巡撫。十三年,擢湖北巡撫。尋授江南河道總督,丁母憂,改署理,服闋,乃實授。疏陳籌辦南河情形,略曰:「近年河湖交敝,欲復舊制,不外蓄清刷黃。古人引導清水,三分濟運,七分刷黃,得力在磨盤埽。自廢棄後,河務漸壞,擬規復磨盤埽舊制。洪澤湖水甚寬,高家堰工絕險,各壩多封柴土蓄水,盛漲啟放,輒壞壩底,糜費不貲。應仿滾水壩成法,抬高石底,至蓄水尺寸為度。山圩五壩暨下游楊河境內車邏等壩,一遵奏定丈尺啟放,水定即行堵合。至黃河各工,當體察平險,節可緩之埽段,辦緊要之土工。一切疏浚器具,祗備運河挑挖。若黃河底淤,非人力所能強刷,惟儲備料工,遇險即搶,以防為治,而其要全在得人。又以蘆葦為工程必需,右營蕩地荒廢,產蘆不足,請築圩蓄水以資灌溉。」疏入,詔嘉其言正當,勖慎勉從事。

十四年,以洪澤湖老子山西北挑砌石壩,東西沙路加築碎石,高出湖面,以便水師巡哨及商民停泊,疏請淮海、常鎮等道另案用銀。詔以南河連歲安瀾,而工用日增,切責之。十九年,修惠濟正閘、福興越閘。會河湖並漲,險工疊生,請例外撥銀五十萬,詔允之,戒嗣後不得援例。署兩江總督。二十一年,河決祥符,黃水匯注洪澤湖,南河無事,詔嘉其化險為夷,予議敘。二十二年,英吉利兵艦入江,命籌淮、揚防務以保運道,請以鹽運使但明倫備防揚州,以清江為後路策應,捕內匪陳三虎等誅之。秋,河決桃北崔鎮汛,值漕船回空,改由中河灌塘,通行無誤,詔念防務及濟運勞,革職,免罪。二十三年,發東河中牟工效力,工竣,以四品京堂候補。尋予二等侍衛,充庫倫辦事大臣,乞病未行。病痊,仍改四品京堂。尋卒。著有黃運河口古今圖說、河工器具圖說。子崇實、崇厚,並自有傳。

潘錫恩,字蕓閣,安徽涇縣人。嘉慶十六年進士,選庶吉士,授編修。大考第一,超擢侍讀。道光四年,復大考一等,擢侍讀學士。時河患急,錫恩上疏條陳河務,略曰:「蓄清敵黃,為相傳成法。大汛將至,急堵禦黃壩,使黃水全力東趨。今年漕艘早渡,因禦黃壩遲堵,以致倒灌停淤,釀成大患。且欲籌減洩,當在下游,乃輒開祥符徬,減黃入湖。壩口已灌於下,徬口復灌於上,黃水俱無出路,湖底淤墊極高。若更引黃入運,河道淤滿,處處壅溢,恐有決口之患。」宣宗韙其議。五年,命以道員發往南河,補淮揚道。六年,加三品頂戴,授南河副總河。九年,母憂去官,服闋,授光祿寺卿。歷宗人府府丞、左副都御史,督順天學政。擢兵部侍郎,調吏部,仍留學政。十九年,內監狄文學以甥考試被黜,至錫恩私宅言所取錄多出請託,挾制訛詐,錫恩疏聞,特詔論文學大闢。二十二年,疏言:「黃河自桃北崔鎮汛、蕭家莊北決口穿運河,壞遙堤,歸入六塘河東注。正河自揚工以下斷流,去清口約有六七十里之遠,回空漕船,阻於宿遷以上。臣前任淮揚道時,詳辨戽水通船之法,行之十餘年,幸無貽誤。今若於中河西口外築箝口壩,添設草徬,以為黃水啟閉之用,即將楊家壩作攔清堰,以為清水啟閉之用。就中河運道為一大塘,道里長則容船眾,兩次啟閉,漕船可以全渡。惟黃水先已灌入運河,中泓淤墊,兩岸纖堤亦恐有沖缺,趕緊修濬,計需費亦不甚多。此時果可回空。來年即可出重,則蕭莊決口不妨從緩堵築。儻此法趕辦不及,祗有竟用引黃濟運之法。其臨黃箝口壩草徬照式築作,引黃水入壩送船,沿途多築對頭小壩,以偪溜刷深,庶免淤滯之患。迨出楊莊,匯入清河之水,即可牽挽南行。蓋南岸不可借黃者,恐其淤湖淤運。今所引黃水,一出楊莊口,仍歸舊河,自可用清口之水以刷滌之,應無流弊。」並以圖說進,下河督麟慶議行。麟慶亦主用灌塘法,與錫恩言合,尋代麟慶為江南河道總督。

時揚工漫溢,尚書敬徵等查勘,堵築決口,開挖引河,接挑長河淤墊,估銀五百七十萬兩有奇。御史雷以諴奏決口無庸堵合,祗須改舊河為支河,以通運道而節糜費,下錫恩會議。錫恩奏覆:「灌口非可行河之地,北岸無可改河之理,請仍堵築決口。漕船回空,仍由中河灌塘。」命侍郎成剛、府尹李僡赴工會同錫恩督辦。二十三年,夫工以下挑河四萬一百九十餘丈,工竣,啟除界壩,放水通暢。會河南中牟河決,黃水注湖,請放山盱各壩宣洩湖水,並將夫工導出湖水,引入中河,暫濟鹽柴轉運。復以上游河水陡落,間有淤墊,請改估蕭工以下未挑之工,並挑築大堤單薄卑矮處。是秋,湖水接長,掣卸高堰石工四千餘丈,搶護未決。二十四年,黃流未復故道,急籌濟運,並宣洩湖水,請啟放外南屬順清河,導引入河歸海。軍船抵壩,即由其處放渡,並於外南之北攔黃壩址築鉗口土壩,以資停蓄。尋奏:「黃河上游六月間陡長水丈餘,山盱林家西壩、舊義河直壩、及仁義河中間攔堰,間有掣塌,補修完密。里、河、揚三承受洪湖之水,兩岸纖堤舊有護埽者,致多刷蟄,亦擇要加鑲。」二十五年,中牟工始合龍,南河連年無險。

二十八年,以病乞歸。咸豐中,命在籍治捐輸團練。八年,前江西巡撫張芾劾其勸捐無狀,褫職。同治三年,捐京倉米折,復原銜,命赴安徽廬州會辦勸捐守御事。五年,鄉舉重逢,加太子少保。六年,卒。漕運總督張之萬疏陳錫恩治績,賜祭葬,謚文慎,入祀鄉賢祠。

子駿文,入貲為刑部郎中,改山東知府。咸豐末,捻匪犯省城,駿文率兵團迎擊於段家店,卻之。署青州,平淄川鳳皇山土匪,擢道員。同治中,巡撫閻敬銘、丁寶楨皆倚之。從寶楨會剿捻匪,塞河侯家林,功尤多,授兗沂曹道。光緒中,遷按察使。坐事降調,以諳習河事,仍留山東。歷治上下游要工,調河南鄭工,專任西壩,以合龍愆期,革職留工,工竣,復原官。授山西按察使,護理巡撫,遷福建布政使。十九年,卒於官。山東士民以其治河功,請建專祠。

論曰:河患至道光朝而愈亟,南河為漕運所累,愈治愈壞。自張文浩蓄清肇禍,高堰決而運道阻。嚴烺畏首畏尾,湖河並不能治。張井創議改河,而不敢執咎,迄於無成,灌塘濟運,賴以彌縫。麟慶、潘錫恩循其成法,幸無大敗而已。吳邦慶講求水利,而治河未有顯績。慄毓美實心實力,卓為當時河臣之冠,不獨磚工創法為可紀也。東河自毓美后,硃襄、鍾祥、文沖繼之,祥符、中牟迭決,東河遂益棘矣。


\end{pinyinscope}