\article{列傳七十六}

\begin{pinyinscope}
硃軾徐元夢蔣廷錫子溥邁柱白潢趙國麟

田從典子懋高其位遜柱尹泰陳元龍

硃軾,字若瞻,江西高安人。康熙三十二年,舉鄉試第一。三十三年,成進士,改庶吉士,散館授湖北潛江知縣。潛江俗敝賦繁,軾令免耗羨,用法必持平。有鬥毆殺人獄,上官改故殺,軾力爭之,卒莫能奪。四十四年,行取,授刑部主事,累遷郎中。四十八年,出督陜西學政。修橫渠張子之教,以知禮成性、變化氣質訓士。故事,試冊報部科,當有公使錢。軾獨無,坐遲誤被劾,士論為不平。會有以其事聞上者,上命軾畢試事。五十二年,擢光祿寺少卿。歷奉天府尹、通政使。

五十六年,授浙江巡撫。五十七年,疏請修築海塘:北岸海寧老鹽倉千三百四十丈,南岸上虞夏蓋山千七百九十丈;並議開中亹淤沙,復江海故道。又疏言:「海寧沿塘皆浮沙,雖長椿巨石,難期保固。當用水櫃法,以松、杉木為匱,實碎石,用為塘根,上施巨石為塘身。附塘為坦坡,亦用水櫃,外砌巨石二三重,高及塘之半,用護塘址。塘內為河,名曰備塘河。居民築壩積淤,應去壩濬河,即以其土培岸。」俱下部議行。杭州南、北兩關稅,例由巡撫監收。軾以稅口五十餘,稽察匪易,請委員兼理。部議以杭州捕盜同知監收,仍令巡撫統轄。五十八年,疏劾巡鹽御史哈爾金索商人賄,上命尚書張廷樞、學士德音按治,論如律。五十九年,擢左都御史。六十年,遭父喪,命在任守制,疏辭,上不許,請從軍自效。

上以山、陜旱災,發帑五十萬,命軾與光祿寺卿盧詢分往勸糶治賑。軾往山西,疏請令被劾司道以下出資贍饑民,富民與商人出資於南省糴米,暫停淮安、鳳陽等關米稅;饑民流徙,令所在地方官安置,能出資以贍者得題薦;饑民群聚,易生癘疫,設廠醫治。又疏言:「倉庾積貯,有司平日侵蝕,遇災復假平糶、借貸、煮粥為名,以少報多,有名無實。請敕詳察虧空,少則勒限補還,多則嚴究治罪。至因賑動倉穀,輒稱捐俸抵補,俸銀有限,倉穀甚多。借非實借,還非實還,宜並清覈。」皆從所議行。別疏請令山西各縣建社倉,引泉溉田。上謂:「社倉始於硃子,僅可行於小縣鄉村。若奏為定例,官吏奉行,久之,與民無益。山、陜山多水少,間有泉源,亦不能暢引溉田。軾既以為請,即令久駐山西,鼓勵試行。」軾自承冒昧,乞寢其議,上不許。未幾,川陜總督年羹堯劾西安知府徐容、鳳翔知府甘文煊虧帑,請特簡親信大臣會鞫。上命軾往勘,得實,論如律。六十一年,乞假葬父,歸。

世宗即位,召詣京師,充聖祖實錄總裁,賜第。雍正元年,命直南書房。予其母冷氏封。加吏部尚書銜,尋復加太子太保。充順天鄉試考官,嘉其公慎,進太子太傅。二年,兼吏部尚書。命勘江、浙海塘。三年,還,奏:「浙江餘姚滸山鎮西至臨山衛,舊土塘三道,本為民灶修築。今民灶無力,應動帑興修。自臨衛經上虞烏盆村至會稽瀝海所,土塘七千丈,應以石為基,就石累土。又海寧陳文港至尖山,土塘七百六十六丈,應就塘加寬,覆條石於巔,塘外以亂石為子塘,護塘址當修砌完固。至子塘處,依式興築。海鹽秦駐山至演武場石塘,圮八十丈,潰七十丈,均補築。都計工需十五萬有奇。江南金山衛城北至上海華家角,土塘六千二百餘丈,內三千八百丈當改為石塘。上海汛頭墩至嘉定二千四百丈,水勢稍緩,土塘加築高厚,足資捍禦。都計工需十九萬有奇。」下部議行。拜文華殿大學士,兼吏部尚書。

上命怡親王胤祥總理畿輔水利營田,以軾副之。四年,請分設四局,各以道員領其事。二月,軾遭母喪,命馳驛回籍,諭曰:「軾事母至孝,但母年八十餘,祿養顯揚,俱無餘憾。當節哀抑慟,護惜此身,為國家出力。」賜內帑治喪,敕江西巡撫俟軾至家賜祭。軾奏謝,乞終制,上允解任,仍領水利營田,期八月詣京師。九月,軾將至,遣學士何國宗、副都統永福迎勞,許素服終喪。上以浙江風俗澆漓,特設觀風整俗使,軾疏言:「風俗澆漓,莫甚於爭訟。臣巡撫浙江,知杭、嘉、湖、紹四府民最好訟。請增設杭嘉湖巡道,而以紹興屬寧臺道。民間詞訟冤抑,準巡道申理。」上從其請。六年,以病乞解任,上手詔留之。八年,怡親王薨,命軾總理水利營田。尋兼兵部尚書,署翰林院掌院學士。十三年,議築浙江海塘,軾請往董其役,上俞之,敕督撫及管理塘工諸大臣咸聽節制。

高宗即位,召還,命協同總理事務,予拜他喇布勒哈番世職。時治獄尚刻深,各省爭言開墾為民累,軾疏言:「四川丈量,多就熟田增加錢糧;廣西報部墾田數萬畝,其實多系虛無。因請通行丈量,冀求熟田弓口之餘,以補報墾無著之數。大行皇帝洞燭其弊,飭停止丈量;而前此虛報升科,入冊輸糧,小民不免苦累。河南報墾亦多不實。州縣田地間有未能耕種之處,或因山區磽確,旋墾旋荒;或因江岸河濱,東坍西漲。是以荒者未盡開墾,墾者未盡升科。至已熟之田,或糧額甚輕,亦由土壤磽瘠,數畝不敵腴田一畝,非欺隱者比。不但丈量不可行,即令據實首報,小民惟恐察出治罪,勉強報升,將來完納不前,仍歸荒廢。請停止丈量,飭禁首報,詳察現在報墾之田,有不實者,題請開除。」又疏言:「法吏以嚴刻為能,不問是非曲直,刻意株連,惟逞鍛鍊之長,希著明察之號。請敕督撫諭有司,讞獄務虛公詳慎,原情酌理,協於中正。刑具悉遵定制,不得擅用夾棍、大枷。」上深嘉納之。

乾隆元年,充世宗實錄總裁。九月,病篤,上親臨視疾。軾力疾服朝服,令其子扶掖,迎拜戶外。翌日,卒。遺疏略言:「萬事根本君心,用人理財,尤宜慎重。君子小人,公私邪正,判在幾微,當審察其心跡而進退之。至國家經費,本自有餘,異日倘有言利之臣,倡加賦之稅,伏祈聖心乾斷,永斥浮言,實四海蒼生之福。」上震悼輟朝,復親臨致奠,發帑治喪。贈太傅,賜祭葬,謚文端。

軾樸誠事主,純修清德,負一時重望。高宗初典學,世宗命為師傅,設席懋勤殿,行拜師禮。軾以經訓進講,亟稱賈、董、宋五子之學。高宗深重之,懷舊詩稱可亭硃先生,可亭,軾號也。子必堦,以廕生官至大理寺卿;璂,進士,官至左庶子;必坦,舉人,襲騎都尉。

徐元夢,字善長,舒穆祿氏,滿洲正白旗人。康熙十二年進士,改庶吉士,散館授戶部主事。二十二年,遷中允,充日講起居注官。尋復遷侍講。徐元夢以講學負聲譽,大學士明珠欲羅致之,其遷詞曹直講筵,明珠嘗薦於上。徐元夢以明珠方擅政,不一至其門,而掌院學士李光地亦好講學,賢徐元夢及侍講學士德格勒,亟稱於上前,二人者每於上前相推獎;明珠黨蜚語謂與光地為黨。二十六年夏,上禦乾清宮,召陳廷敬、湯斌、徐乾學、耿介、高士奇、孟亮揆、徐潮、徐嘉炎、熊賜瓚、勵杜訥及二人入試,題為理學真偽論。方屬草,有旨詰二人,德格勒於文後申辯,徐元夢卷未竟。上閱畢,於德格勒及賜瓚有所譙讓,命同試者互校,斌仍稱徐元夢文為是。

是時斌被命輔導皇太子,尋亦命徐元夢授諸皇子讀。秋,上御瀛臺,教諸皇子射,徐元夢不能挽強,上不懌,責徐元夢。徐元夢奏辯,上益怒,命撲之,創,遂籍其家,戍其父母。其夜,上意解,令醫為治創。翌日,命授諸皇子讀如故。徐元夢乞赦其父母,已就道,使追還。冬,掌院學士庫勒納奏劾德格勒私抹起居注,並言與徐元夢互相標榜,奪官逮下獄。二十七年春,獄上,當德格勒立斬,徐元夢絞。上命貸徐元夢死,荷校三月,鞭百,入辛者庫。上徐察徐元夢忠誠,三十二年,命直上書房,仍授諸皇子讀。尋授內務府會計司員外郎。四十一年,充順天鄉試考官。五十年,諭曰:「徐元夢繙譯,現今無能過之。」授額外內閣侍讀學士。五十一年,充會試考官。五十二年,擢內閣學士,歸原旗。

五十三年,授浙江巡撫,上諭之曰:「浙江駐防滿洲兵,爾當與將軍協同訓練。錢糧有虧空,爾宜清理,無累百姓。至於用人,當隨材器使,不可求全。」賜禦制詩文集及鞍馬以行。五十四年,疏言:「杭州、紹興等七府旱潦成災,已蒙蠲賑,並截漕平糶。未完額賦,尚有十三萬餘兩,請秋成後徵半,餘俟來歲。」上允之。又疏陳修復萬松嶺書院,上賜「浙水敷文」榜,因請以敷文名書院。

五十六年,左都御史及翰林院掌院學士缺員,吏部以請。上曰:「是當以不畏人兼學問優者任之。」以命徐元夢。上諭科場積習未除,命甄別任滿學政及考官不稱職者,皆劾罷之。五十七年,遷工部尚書,仍兼掌院學士。六十年,上賜以詩,謂:「徐元夢乃同學舊翰林,康熙十六年以前進士祗此一人。」

世宗即位,復命直上書房,授諸皇子讀。雍正元年,命與大學士張鵬翮等甄別翰詹各官不稱職者,勒令解退回籍。大學士富寧安出視師,命徐元夢署大學士。尋復命兼署左都御史,充明史總裁,調戶部尚書。四年,以繙譯本章錯誤奪官,命在內閣學士之列效力行走,仍司繙譯。八年,復坐前在浙江失察呂留良逆書,命同繙譯中書行走。十三年,充繙譯鄉試考官。

高宗即位,命直南書房,尋授內閣學士。擢刑部侍郎,以衰老不能理刑名,疏辭,調禮部。充世宗實錄副總裁。詔輯八旗滿洲氏族通譜,命與鄂爾泰、福敏董其事。復命直上書房,課皇子讀。乾隆元年,乞休,命解侍郎任,加尚書銜食俸,仍在內廷行走,領諸館事。二年,上臨雍,疏請以有子升堂配享,改宰我、冉求兩廡,而進南宮適、虙不齊升配。下大學士九卿議,以有子升祀位次子夏,餘寢未行。復乞休,上曰:「徐元夢年雖逾八十,未甚衰憊,可量力供職。」四年正月,召同諸大臣賦柏梁體詩。尋加太子少保。

六年秋,疾作,遣太醫診視,賜葠藥。冬十一月,疾劇,上諭曰:「徐元夢踐履篤實,言行相符。歷事三朝,出入禁近,小心謹慎,數十年如一日。壽逾大耋,洵屬完人。」命皇長子視疾。疾革,復遣使問所欲言。徐元夢伏枕流涕曰:「臣受恩重,心所欲言,口不能盡!」使出,呼曾孫取論語檢視良久。翌日遂卒,年八十七。上復命和親王及皇長子奠茶酒,發帑治喪。贈太傅,賜祭葬,謚文定。孫舒赫德,自有傳。

蔣廷錫,字揚孫,江南常熟人,雲貴總督陳錫弟。初以舉人供奉內廷。康熙四十二年,賜進士,改庶吉士。四十三年,未散館即授編修。屢遷轉至內閣學士。雍正元年,擢禮部侍郎,世宗賜詩賢之。廷錫疏言:「國家廣黌序,設廩膳,以興文教,乃生員經年未嘗一至學宮。請敕學臣通飭府、州、縣、衛教官,凡所管生員,務立程課,面加考校,講究經史。學臣於歲、科考時,以文藝優劣定教職賢否。會典載順治九年定鄉設社學,以冒濫停止。請敕督撫令所屬州、縣,鄉、堡立社學,擇生員學優行端者充社師,量給廩餼。鄉民子弟年十二以上、二十以下有志者得入學。」下部議,從之。二年,奏請續纂大清會典,即命為副總裁。調戶部。

三年,命與內務府總管來保察閱京倉。尋疏言:「漕運全資水利,宜通源節流,以濟運道。山東漕河,取資汶、濟、洸、泗四水,而四水又賴諸泉助成巨流。山東一省,得泉百有八十,其派有五,分水、天井、魯橋、新河、沂水是也。五派合為一水,是名泉河,舊設管泉通判。今雖裁汰,仍設泉夫。請飭有泉州縣,督率疏濬。濟南、兗州二府為濟水伏流之地,若廣為濬導,則散湮沙礫間者,隨地湧見。應立法泉夫濬出新泉,優賚銀米,歲終冊報,為州縣課最。諸泉所匯,為湖十五,各設斗門為減水閘,以時啟閉。漕溢則減漕入湖,漕涸則啟湖濟漕,號諸湖為水櫃。其後居民壅水占耕,壩圮閘塞,低處多生茭草,高處積沙與漕河堤並。請察勘未耕之地,就低處挑深,即以挑出之土築堤,復水櫃之制。諸湖開支河,以承諸泉之入,益漕河之流,建閘以時減放。舊制,運河於每歲十月築壩,分洩諸湖,來春三月冰泮,開壩受水。法久玩生,築壩每至十一月,則失之遲;開壩在正月初旬,又失之早。請飭所司築必十月望前,開必二月朔後,以循舊制。汶水分流南北,運道攸賴。明宣德間,築戴村壩於汶水南,以遏汶水入洸;建坎河壩於汶水北,以節汶水歸海。嘉靖時,復堆積石灘,水溢縱使歸海,水平留之入湖。歲久頹廢,萬一汶水北注,挾湖泉盡歸大清河,四百餘里運道所關非小。請飭總河相度形勢,修復舊石灘,改建滾水石壩,以為蓄洩。」上命內閣學士何國宗等攜儀器輿圖,會總河齊蘇勒、巡撫陳世倌履勘,請如廷錫奏。下九卿議行。

四年,遷戶部尚書,充順天鄉試考官。既入闈,諭曰:「廷錫佐怡親王董理戶部諸事,秉公執正,胥吏嫉妒懷怨。今廷錫典試,或乘此造作浮言,妄加謗議。令步軍統領、順天府尹、五城御史察訪捕治。」尋命兼領兵部尚書。遭母喪,遣大臣奠茶酒,予其母封誥,發帑治喪。命廷錫奉母喪還里,葬畢還京,在任守制。六年,拜文華殿大學士,仍兼領戶部,充聖祖實錄總裁。七年,加太子太傅。命與果親王允禮總理三庫,予世職一等阿達哈哈番。九年,廷錫病,上遣醫療治。十年夏,病復作,上命日二次以病狀奏。閏五月,卒,上為輟朝,遣大臣奠茶酒,賜祭葬,謚文肅。

廷錫工詩善畫,事聖祖內直二十餘年。世宗朝累遷擢,明練恪謹,被恩禮始終。

子溥,字質甫。雍正七年,賜舉人。八年,進士,改庶吉士,直南書房,襲世職。廷錫卒,溥奉喪歸,命葬畢即還京供職。十一年,授編修。四遷內閣學士。乾隆五年,授吏部侍郎。疏言:「凡條奏發九卿會議,主稿衙門酌定準駁。會議日,書吏誦稿以待商度,其中原委曲折,一時難盡。請於會議前二日將議稿傳鈔,俾得詳勘暢言。至命、盜案,刑部例不先定稿,俟議時平決;不關命、盜各案,亦宜先期傳知,庶為審慎。」下部議,如所請。

八年,授湖南巡撫。九年,疏言:「永順及永綏、乾州、鳳凰諸處苗民貪暴之習未除,城步、綏寧尤多狡惡。臣整飭武備,漸知守法。」諭曰:「馭苗以不擾為要,次則使知兵威不敢犯。此奏得之。」旋劾按察使明德不詳鞫盜案,奪官;驛鹽道謝濟世老病,休致。給事中胡定奏請湖南濱湖荒土,勸民修築開墾,令溥察議。溥奏言:「近年湖濱淤地,築墾殆遍。奔湍束為細流,洲渚悉加堵截,常有沖決之慮。沅江萬子湖、湘陰文洲圍,士民請修築開墾。臣親往履勘,文洲圍倚山面江,四圍俱有舊堤,已議舉行。萬子湖廣袤八十餘里,四面受水,費大難築,並於上下游水利有礙。臣以為湖地墾築已多,當防湖患,不可有意勸墾。」上韙之。

十年,授吏部侍郎、軍機處行走。十三年,擢戶部尚書,命專治部事。十五年,加太子少保。十八年,命協辦大學士,兼禮部尚書,掌翰林院事。二十年,兼署吏部尚書。二十四年,授東閣大學士,兼領戶部。二十六年,溥病,上親臨視。及卒,復親臨奠。贈太子太保,發帑治喪,賜祭葬,謚文恪。

子檙,進士。自編修累遷兵部侍郎;賜棨,初授雲南楚雄知府,再至戶部侍郎。並坐事奪官,左授光祿寺卿。復奪官,以世職守護裕陵。

邁柱,喜塔拉氏,滿洲鑲藍旗人。初授筆帖式,三遷戶部員外郎,授御史。康熙五十五年,巡視福建鹽課。雍正元年,巡視寧古塔。三年,命如荊州會將軍武納哈籍前任將軍阿魯家,償侵蝕兵餉。議荊州近縣民有原鬻地者,官購俾兵耕種,或招佃徵租,兵婚喪量佽之。下部議行。

擢工部侍郎,調吏部。命如江西按治德安知縣蕭彬、武寧知縣廖科齡虧帑,並命察通省錢糧積弊。尋命署巡撫。疏請以江西額徵丁銀攤入地糧,從之。五年,授湖廣總督,命俟江西事畢赴任。邁柱疏陳:「江西倉穀虧缺,弊在無穀無銀,虛報存貯,及至交代,又虛報民間借領,後任徵追,悉歸無著。又或出糶倉穀得價侵用,及至交代,以二錢一石折價,後任不敷糴補。又或因不敷之故,並此折價而亦侵用,及至交代,復稱民欠,多方掩飾。皆因前任巡撫裴幰度,布政使陳安策、張楷徇庇所誤。」上為奪幰度等官,察究追完。又言:「江西通省公用需款,請視河南、湖廣諸省例,提州、縣耗羨二分充用,另提充各員養廉,多至一分五釐,少至四釐,餘仍留州縣養贍。巡撫及司道,亦於所提一分五釐內量行支用。」又言:「江西被災州縣,設廠煮賑,米價石至一兩三四錢。請於未被災州縣發銀預購平糶。」又言:「南安、贛州,閩、廣交界,及鄱陽湖濱,最易藏奸。萬載、寧州等地,棚民聚集,素好多事。已飭嚴整塘汛,操練標兵,豫為之備。」得旨,嘉其條畫詳晰,令新任巡撫照行。尋讞定彬等俱論斬。並請令徇庇之上官分償虧帑,上命自雍正六年起著為例。獎邁柱秉公持正,下部議敘,乃赴湖廣任。

湖廣瀕江州縣頻年被水,邁柱令民間按糧派夫,修築江堤,議定確估土方夫數及加修尺寸,並歲修搶險諸例。疏聞,上發帑六萬,命視工多寡分給。鎮筸苗最悍,屢入內地剽掠。邁柱疏言:「臣聞雲南提督張國正先任鎮筸總兵,以雕剿法治苗。聞有警,詗為何種苗,所屬何寨,即攜兵馳往,圍寨搜擒。如雕之捕鳥,取其速而鳥可必得。臣今與總兵周一德循行此法,但期得罪人而止,不敢多為殺戮。」居數年,又疏言:「收繳六里鎮筸土司所藏鳥槍,完整者俾兵充用,餘改造農具,給土苗耕作。土苗所用環刀、標槍,亦令給價收繳。」上諭曰:「所奏深得賣刀買犢之意。環刀、標槍,自當收繳,可順其原,不宜強迫。」疏定苗與民為市,於分界地設市,一月以三日為期,不得越界出入。民以物往市,預報地方官,知會塘汛查驗。苗疆州縣立苗長,選良苗充民壯,備差遣訪緝。鄂爾泰督云、貴,建策改土歸流,邁柱亦行之湖廣,收永順、保靖、桑植三土司。永順設府縣,仍其名,又於府西北設縣曰龍山。保靖、桑植各設縣,仍其名。收容美土司設州,曰鶴峰,所屬五峰新設縣曰長樂。並改彞陵州為府,曰宜昌,領新設州縣。收第岡土司,改永定衛為縣,以其地屬焉。

上命通察湖廣積欠錢糧,都計銀三十餘萬,令與巡撫馬會伯、王國棟同董其事。逾年,報湖南已完六萬有奇,湖北已完八萬有奇。尋察出沔陽積欠內為官侵役蝕包攬未完者三萬有奇,其實欠在民者三萬二千有奇。上以沔陽常被水,民欠命予豁除。七年,邁柱疏請以湖廣額徵丁銀攤入地糧,從之。邁柱督湖廣數年,聲績顯著。他所區畫,如以漢陽通判移漢口,荊州通判移沙市。又裁施州、大田二衛所,合為縣曰恩施,復請改為府,曰施南,設縣四,曰宣恩、來鳳、咸豐、利川。宜昌既為府,設附郭縣曰東湖,又以歸州及所領長陽、興山、巴東諸縣隸焉。道州及寧遠、永明、江華諸縣鄰廣西,請以永州同知移江華,並分設游擊、守備,調駐兵千五百,與廣西桂臨營月三次會哨。永順、保靖、桑植三營新立,月餉給米折,永順石折一兩,保靖、桑植石折八錢,以苗疆米貴,不與他營同。上悉如所請。

十三年,召拜武英殿大學士,兼吏部尚書。乾隆元年,兼管工部。二年,以病乞解任。三年,卒,賜祭葬,謚文恭。

同時督撫入為大學士者,又有白潢、趙國麟。

潢,字近微,漢軍鑲白旗人。初授筆帖式,考授內閣中書,遷侍讀。授福建糧驛道僉事,以父憂去官。服闋,除山東登萊青道僉事,遷貴州貴東道參議。以巡撫劉廕樞薦,就遷按察使。潢操守廉潔,聞於聖祖,擢湖南布政使。未上官,會廕樞以請緩西師,命詣軍前察視,潢護貴州巡撫。貴州山多田少,諸鎮營兵餉米,於徵米諸州縣支發。以運道艱阻,改徵折色,遲至次年春夏,米值昂不足以糴。諸驛例設夫百、馬四十五,而巡撫以下私函付驛,謂之便牌,役夫至數百。潢奏請兵米於籓庫借支,州縣徵解歸項,並檄諸驛禁便牌。兵民困皆蘇。又以貴州僻遠,官於外,商於外,皆不肯歸,潢奏請勒限回籍。貴州民初以為不便,久之文物漸盛,乃思潢惠焉。

廕樞還貴州,調潢江西。入覲,至熱河謁上,即擢江西巡撫。潢革諸州縣漕節陋例,並令火耗限加一,舊加至三四者,悉罷除之,不率者奏劾。湖口關地險港窄,潢度關右武曲港山勢開闊,可容千艘,乃濬江口,建草壩,使估舟得聚泊。建亭頌潢德。會城西南有袁、贛二江,至臨江合流,舊有堤久圮,春夏水發,往往壞田廬。潢奏請重建,九閱月而成。民自是無水患,號為白公堤。五十九年,奏請補京職,授戶部侍郎。擢兵部尚書。六十一年,世宗即位,命協辦大學士。尋授文華殿大學士。疏辭,不許。充聖祖實錄總裁。雍正三年,以疾乞解任,許之。

潢撫江西時,南昌、吉安、撫州、饒州四府舊有落地稅千三百兩有奇,設大使徵收。潢以官役苛徵,令停收。巡撫、司道公捐代納,偽編納稅人名冊報部,王企崝、裴幰度代為巡撫,皆如潢例。及汪漋至,以其事聞,且請裁大使。上曰:「國家經制錢糧,豈可意為增減?若此稅不當收,潢當請豁免,何得以公捐代完,沽名邀譽?」下部議,奪潢官。漋亦坐左遷,稅如舊例徵收。乾隆二年,潢卒,命還大學士銜。

國麟,字仁圃,山東泰安人。祖瑗,手書春秋內外傳,史、漢蒙文授之。篤志於學,以程、硃為宗。康熙四十五年進士。五十八年,授直隸長垣知縣。當官清峻,以禮導民,民戴如父母。世宗聞其賢,雍正二年,擢永平知府。三遷福建布政使,調河南。擢福建巡撫,調安徽。御史蔣炳奏請州縣徵收錢糧,依部頒定額,刊印由單,申布政使覈發。國麟以安徽通省數百萬由單由司覈發,恐誤徵收,疏請停止。內閣學士方苞疏言:「常平倉穀原定每年存七糶三,南省地卑濕,應令因地制宜。」下督撫詳覈。國麟疏言:「安徽所屬州縣濱江湖者,當改糶半存半,他州縣仍循舊例。」並下部議行。乾隆三年,擢刑部尚書,調禮部,兼領國子監。四年,授文華殿大學士,兼禮部尚書。

六年,御史仲永檀疏劾內閣學士許王猷邀九卿至京師民俞長庚家吊喪,國麟亦親往,下王大臣勘不實。國麟乞引退,上留之。俄,給事中盧秉純復論國麟當上舉永檀疏面詰,陽若不知,出告其戚光祿寺卿劉籓長,籓長被命休致;國麟又告以為侍郎蔣炳所劾。上命大學士鄂爾泰、張廷玉召國麟及籓長相質,籓長力辯。上命毋深究,令鄂爾泰、張廷玉諭國麟引退。國麟疏未即上,上降詔詰責,左授禮部侍郎。七年,擢尚書。國麟乞引退,不許。逾數月,復以請,上不悅,命奪官,在咸安宮效力。八年,乃許其還里。十五年,詣京師祝上壽,賜禮部尚書銜。明年,卒。

田從典,字克五,山西陽城人。父雨時,明諸生。寇亂,挈子及兄之孤徙避,度不能兼顧,棄子負兄子以走。賊退,求得子草間,即從典也。

從典篤學,以宋五子為宗。康熙二十七年,成進士。旋居父喪,事必遵家禮。服終,就選。三十四年,授廣東英德知縣。縣地瘠,賦籍不可稽,詭寄逋逃,民重困。陋例兩加至八九錢,名曰「均平」。從典盡革之,清其籍。

四十二年,行取,四十三年,授雲南道御史。疏言:「督撫不拘成例,請調州縣,有秉公者,即有徇私者。州縣求調,其弊有三:圖優缺,避沖繁,預為卓薦地。督撫濫調,其弊亦有三:徇請託,得賄賂,引用其私人。名為整頓地方,簡拔賢良,實乃巧開捷徑。屢經敗露,有駭聽聞。嗣後請除江、浙等省一百一十餘縣錢糧難徵,及邊遠煙瘴地,仍舊例調補,其他不準濫調。」又疏言:「京官考選科道,令部院堂官保送,恐平日之交結,臨時之營謀,在所難免。請敕吏部,遇考選科道,凡正途部屬,及自知縣升任中、行、評、博,與翰林一體論俸開列,聽候考選。」均下部議行。巡視西城,罷鋪墊費。察通州倉儲,僦神祠以居,廟祝不受值,不入也。

四十九年,擢通政司參議。屢遷轉授光祿寺卿。寺故有買辦人,虧戶部帑至四十一萬餘,從典請限年帶銷。遷左副都御史,再遷兵部侍郎,並命兼領光祿寺。五十八年,遷左都御史。兩江總督常鼐疏言安徽布政使年希堯、鳳陽知府蔣國正婪取,為屬吏所訐。命從典與副都御史屠沂往按,國正坐斬,希堯奪官。五十九年,擢戶部尚書。雍正元年,調吏部。二年,協辦大學士。三年,授文華殿大學士,兼吏部尚書。六年三月,乞休,優詔褒許,加太子太師致仕。賜宴於居第,令部院堂官並集,發帑治裝,行日,百官祖餞,馳驛歸里,驛道二十里內有司送迎。入辭,賜御榜聯並冠服、朝珠。四月乃行,甫一舍,次良鄉,病大作,遂卒,年七十八。上聞,以從典子懋幼,遣內閣學士一、侍讀學士一為治喪,散秩大臣一、侍衛六奠茶酒,並命地方官送其喪歸里。賜祭葬,謚文端。

懋,自廕生授刑部員外郎,世宗命改吏部,遷郎中,授貴州道御史。乾隆初,遷禮科給事中。疏言河南秋審寬縱,巡撫尹會一、按察使隋人鵬下吏議。又劾工部尚書趙弘恩受賕,奪官,戍軍臺。遷鰿臚寺少卿。高宗獎懋敢言,超擢副都御史。遷刑部侍郎,調吏部。十一年,上責懋奏事每漏言,且嗜酒務博,命解任歸裏讀書。十四年,召授吏部侍郎。以僕從鬥毆傷人,責懋舊習未悛,仍命歸裏讀書。家居二十年,卒。

高其位,字宜之,漢軍鑲黃旗人。父天爵,語在忠義傳。其位初隸鑲白旗,自筆帖式管佐領。康熙間,以署參領從軍駐襄陽。叛將楊來嘉、王會等以二萬人出掠,將攻南漳,其位率二十騎覘敵,與遇,越敵隊入南漳,與共守,敵圍攻不能下。叛將譚弘以三萬人犯鄖陽,其位將百人扼楊谿鋪,與相持七十餘日。糧盡,煮馬韉以食。副都統李麟隆援至,合擊,大敗之。尋追論禦敵穀城失利,奪官。久之,授火器營操練校尉,襲其祖尚義二等阿達哈哈番。從大將軍裕親王福全討噶爾丹,戰於烏闌布通,破駱駝營,擢參領。授甘肅永昌副將。明法令,築堡塞,邊境肅清。遷湖廣襄陽總兵。擢提督,賜孔雀翎、櫜鞬、鞍馬。調江南。兩江總督常鼐有疾,上命其位署理。世宗即位,召入覲,旋命回提督任。奏請保護聖躬,上褒其有愛君之心,溫詔嘉許。雍正二年秋,奏飛鴉食蝗,秋禾豐茂。上以蝗不成災,傳示王大臣,賜詩褒之。冬,奏進黃浦漁人網得雙夔龍紐未刻玉印,上賜以四團龍補服。三年,授文淵閣大學士,兼禮部尚書,加太子少傅。以衰老辭,不許。改隸鑲黃旗。賜壽,賚榜聯及白金千。屢乞休,乃命以原官致仕。五年,卒,賜祭葬,謚文恪。

子高起,以廕生授四川茂州知州。累遷兵部尚書,坐事奪官逮治。乾隆初,戍軍臺,釋回。卒。

遜柱,棟鄂氏,滿洲鑲紅旗人。曾祖郎色,太祖時,從其兄郎格來歸。遜柱初授筆帖式,擢工部主事。再遷戶部郎中,授御史。歷翰林院侍讀學士、內閣學士、盛京工部侍郎。召改吏部,擢兵部尚書。雍正五年,署大學士,旋授文淵閣大學士,仍兼兵部尚書。遜柱長兵部十六年,屢陳奏部政,多所考覈釐正。十年,以老,命不必兼兵部。十一年,致仕,卒,年八十四,諭褒遜柱「醇厚和平」,賜祭葬。

尹泰,章佳氏,滿洲鑲黃旗人。初授翰林院筆帖式,再遷內閣侍讀。康熙二十七年,授翰林院侍講,充日講起居注官。三十四年,授國子監祭酒。三十七年,改錦州佐領。五十二年,以病罷,遂居錦州。世宗在籓邸,奉命詣奉天謁陵,過錦州宿焉,與語奇之,見其子尹繼善。雍正元年,召授內閣學士。遷工部侍郎,再遷左都御史。疏言:「六科書吏,賄通提塘,造為小鈔、晚帖,內開口傳諭旨,或誤繙清文,甚至偽造上有賜予及與諸臣問對,應請禁止。」二年,充會典總裁。三年,命以原品署盛京侍郎,兼領奉天府尹。疏言:「承德等九州縣原徵豆米,多貯無用。請自雍正四年始,停徵黑豆,按畝徵米,按丁徵銀,而以原貯米豆視時價出糶。」又言:「關東風高土燥,請掘地窖藏存穀,以節建倉工費。」

四年,山海關總管多索禮疏言應交莊頭餘地,尹泰不即派官丈收。命侍郎查郎阿往按,坐解府尹任,仍以左都御史協理奉天將軍。將軍噶爾議設外海水師,尹泰以為旅順、天津俱有水師,錦、復、蓋諸州亦可更番巡察,增設需費浩繁,於巡察無益。別疏以聞。下議政王大臣議,如尹泰言。六年,坐遺漏入官財產,奪官。尋命復官。七年正月,與尚書陳元龍同授額外大學士。尋授東閣大學士,兼兵部尚書。十三年,高宗即位,充世宗實錄總裁。乾隆元年,以老病乞罷,上留之。尹繼善自兩江總督入覲,授刑部尚書,俾使朝夕侍養。三年,復乞罷,命以原官致仕。尋卒,賜祭葬,謚文恪。尹繼善自有傳。

陳元龍,字廣陵,浙江海寧人。康熙二十四年一甲二名進士,授編修,直南書房。郭琇劾高士奇,辭連元龍,謂與士奇結為叔侄,招納賄賂,命與士奇等並休致。語互詳士奇傳。元龍奏辯,謂:「臣宗本出自高,譜牒炳然。若果臣交結士奇,何以士奇反稱臣為叔?」事得白,命復任。累遷侍讀學士。元龍工書,為聖祖所賞,嘗命就御前作書,深被獎許。上御便殿書賜內直翰林,諭曰:「爾等家中各有堂名,不妨自言,當書以賜。」元龍奏臣父之闓年逾八十,家有愛日堂,御書榜賜之。四十二年,再遷詹事。以父病乞養歸,賜葠。時正編賦匯,令攜歸校對增益。上南巡,元龍迎謁,御書榜賜之闓及元龍母陸。之闓卒,喪終,召元龍授翰林院掌院學士。

五十年,遷吏部侍郎。授廣西巡撫。值廣東歲歉,廣西米價高,元龍遣官詣湖南採米平糶。五十四年,修築興安陡河閘,護兩廣運道。並於省城擴養濟院,立義學,創育嬰堂,建倉貯穀。五十七年,擢工部尚書。六十年,調禮部。世宗即位,命守護景陵。七年,與左都御史尹泰同授額外大學士,尋授文淵閣大學士,兼禮部尚書。元龍在廣西,請開例民捐穀得入監。李紱為巡撫,請以捐穀為開墾費。上責其借名支銷,命元龍詣廣西清理。紱旋奏:「元龍分得羨餘十一萬有奇,除在廣西捐公費九萬,又助軍需十萬。今倉穀尚有虧空,應令分償。」及授大學士,命免之。十一年,以老乞休,加太子太傅致仕,令其子編修邦直歸侍養。行日,賜酒膳,令六部滿、漢堂官餞送,沿途將吏送迎。乾隆元年,命在籍食俸。尋卒,賜祭葬,謚文簡。

論曰:軾以德望尊,徐元夢以忠謇重。世宗譴允禩、允禟,徐元夢言:「二人罪當誅,原上念手足情緩其死。」二人者既死,吏議奴其子,軾言:「二人子實為聖祖孫,孰敢奴之?」世宗皆為動容。諒哉,古大臣不是過也。廷錫直內廷領戶部,邁柱等領疆節,卓然有績效。從典、尹泰皆以端謹奉職。古所謂大人長者,殆近之矣。


\end{pinyinscope}