\article{志三十四}

\begin{pinyinscope}
地理六

△安徽

安徽:禹貢揚及徐、豫三州之域。明屬南京。清順治二年,改江南省,置鳳陽巡撫及安廬池太巡撫,兼理操江軍務,並統於淮陽總督。六年,俱罷。十八年,設江南左、右布政使,以左布政轄安慶、徽州、寧國、池州、太平、廬州、鳳陽、淮安、揚州九府,暨徐、滁、和、廣德四直隸州,駐江寧。康熙元年,始分建安徽為省治,復置巡撫,駐安慶。三年,江南分一按察使來治。五年,割揚州、淮安、徐州還隸江寧右布政。六年,改左布政為安徽布政使司。雍正元年,以兩江總督統治安徽、江蘇、江西三省。二年,升鳳陽府屬之潁、亳、泗三州,廬州府屬之六安州,為直隸州。十三年,潁州升府,亳州復降屬潁。乾隆二十五年,安徽布政使亦自江寧來駐。東至江蘇溧水;西至湖北麻城;南至江西彭澤、浙江遂安;北至河南鹿邑。廣七百三十五里,袤六百六十六里。宣統三年,編戶三百一十四萬二千一百八十四,口一千六百二十二萬九百五十二。領府八,直隸州五,屬州四,縣五十一。其名山:霍、皖、黃、九華、陵陽、敬亭。其大川:大江、皖、涇、樅陽、巢湖、淮、潁、渦、滁、澮、西肥、北肥、洪澤湖。航路:東達江蘇,西達江西、湖北。驛路:自安慶北逾北峽關渡淮達江蘇徐州;自江心驛東南出清流關達江浦;自桐城西南達湖北黃梅。鐵路擬設蕪寧路。電線。

安慶府:沖,繁,難。安廬滁和道治所。巡撫,布政、提學、提法三使,巡警、勸業道,同駐。順治初,因明制,屬江南左布政使司。康熙六年,始分建安徽省。十四年,設提督,轄上江營汛。十八年,省提督,並入江南。乾隆二十五年,移左布政使來治。嘉慶八年,以巡撫兼提督,轄二鎮各標。西北距京師二千七百里。廣四百五十里,袤二百七十五里。北極高三十度三十七分,京師偏東三十四分。領縣六。懷寧沖,繁,難。倚。東北:大龍山。西:皖山,百子。西北:獨秀山。大江自望江入,逕城南而東北出趨池口,又東北入無為。皖水自潛山入,會長河,逕石牌港入江。北:黃麻河,一名黃馬河,自潛山入,會沙河、高河,達桐城為練潭河。西北:井田河,上達練潭。西:冶塘湖,由皖口入江。東北:長楓港,引蓮湖、槐湖水入江,即古之長風沙也。碎石嶺汛二,石牌市汛一。長楓、三橋二鎮,巡司各一。一驛:練潭。商埠濱江。桐城沖,繁,難。府東北百二十里。東北:浮度山,北峽山一名北峽關,與舒城界。西:掛車。北:龍眠山,有水流為龍眠河,入松山、鴨子諸湖。東南:大江自懷寧入,東流,逕縣西南練潭驛為練潭河。雙河出縣東,二派合流為孔城河,與東南長河、白兔河俱入練潭河,至樅陽入江。三道巖關,縣西,咸豐十年重築。六百丈、北峽關、練潭鎮、馬踏石巡司四。驛二:陶沖、呂亭。潛山沖,繁。府西北百二十里。北:灊山,一曰皖山,又名天柱。漢武帝登灊禮天柱,號為「南岳」,即此;道書所謂「第十四洞天」也。潛水今名前河,源出公蓋山,西流為開源澗。東南流,逕城北,東合皖水。出公蓋山,東逕烏石波,至城東崩河合潛水。南至石牌市,與太湖東諸水會,逕懷寧入江。東北:昆侖山,沙河出,會黃馬河入懷寧。吳塘堰,歷代開以灌田,康熙十一年修治。天堂砦,後部河所出。有巡司一。一驛:青口。太湖沖,難。府西北二百二十里。東:馬頭山。南:新寨,香茗。西北:龍山。北:珠子山。有關,西與英山界。太湖舊與小湖五,並堙。東北:銀河自潛山入,為後部河。右合羊角河,為龍灣河。匯南陽、青石、棠梨、羅溪諸河,為馬路河。環城而東,折東北仍之。後部、白沙巡司各一。一驛:小池。宿松沖,難。府西南百六十里。東北:嚴恭、烽火。東南:洿池。西南:得勝山。大江自湖北黃梅入,流逕小孤山。元立鐵柱於上,名「海門第一關」。分流東下入望江。二郎河一名揚溪河,承雷水,南流入望江。北:三溪河自湖北蘄州、黃梅分入,合於隘口,南流入江。東北:舊縣河出白崖諸山,合荊橋河,入望江之泊湖。東:張富池,會大小泊澇湖,龍南蓮若湖,白湖、棠梨、小黃三湖,趨於泊湖。又南,龍宮湖、大官湖,均東連泊湖,成巨浸。有便民倉鎮,南北糧倉貯此。有歸林灘鎮,舊置巡司,裁。其復興、涇江口二鎮有巡司二。一驛:楓香。望江簡。府西南百二十里。西北:大茗、小茗對峙。東:周河山。西:麒麟山。北:寶珠山。南:大江自宿松入,濱城緣娥眉洲東北流,至華陽口納泊湖。泊湖受宿松浮湖、茅湖諸水,合流為揚溪河,即雷水也。雷港,明時湮。今從華陽鎮入江。鎮四:蘇家、吉水、香新溝,又華陽。雷港,游擊駐。有巡司。雍正中自楊灣改。一驛:雷港。

廬州府:沖,難。隸安廬滁和道。明,廬州府,屬江南。順治初,因明制,改二州、六縣,屬江南左布政使司。康熙六年,分隸安徽省。雍正二年,升六安為直隸州,以英山、霍山二縣改屬,餘仍舊。南距省治四百六十里。廣二百二十里,袤二百一十里。北極高三十一度五十六分,京師偏東四十七分。領州一,縣四。合肥沖,繁,疲,難。倚。東:龍泉,青陽。東北:大小峴山。西南:紫蓬。東:浮槎。東南:四鼎山一名四頂山。東:巢湖一名焦湖,延袤四百餘里,中有三小山,曰鞋、曰姥、曰孤,港汊凡三百六十,納眾水而南注之江。東店阜河,南派河、三汊河,皆入焉。肥水逕雞鳴山,淮水來與之合,縣名昉此。東:逍遙津。梁園鎮。西:廬鎮關。梁園、青陽、官亭巡司三。督糧通判一。縣丞一。驛五:護城、金斗、店埠、派河、吳山廟。舒城沖,繁。府西南百二十里。南:春秋山、華蓋、鼓樂。西南:龍眠、七門山。東:巢湖,與合肥、廬江、巢分界,縣境諸水畢匯於此。龍舒河源出縣西孤井,東流會石塞河,流逕七門堰,又逕城南入巢湖。上七里河在縣西九里,西山諸水所匯,逕南溪入巢湖。其在縣七里者為下七里河,上接南溪,下達巢湖。七門堰在西南七門山下,有三堰:一烏羊,二千功,三槽櫝也。南北峽關、西陽山寨、上陽鎮有汛。曉天鎮巡司一。驛二:三溝、梅心。廬江簡。府南百八十里。東北:冶父山。西北:冷水關。兩山夾立如門。東:梅山,西:孺山,郎家。東南:礬山。東北:巢湖,西納三河,迤東金牛、清野諸水匯焉。其南白湖。南迤為後湖,西播為黃陂湖,匯縣河及作枋河。東出為青簾河,由無為入江。西南高子水,南羅昌河,並入桐城。冷水關有汛一,巡司一。驛一:廬江。巢簡。府東北百八十里。東:東山,濱江為險。東南:七寶山。西北:萬家山。西南:巢湖,舊居巢地,後陷為湖,因名。縣境諸川多自此導流。濡須水自湖東口流逕城南,一名天河水。東流,逕東北亞父山南。又東南,逕七寶、濡須兩山間,亦曰東關水,入無為。清溪河自巢湖導流,逕縣東,合芙蓉水,下流會濡須水。西柘皋、白露、巧溪、花塘諸河皆入巢湖。柘皋有汛。巡司、典史各一。二驛:高井、鎮巢。無為州繁,疲,難。府東南二百六十里。界城內紫芝山。東北:偃月,即濡須塢,東西有二關。西南:三公、九卿。西:孤避。北:青檀。南:大江自桐城入,為石炭河口。又東北,青簾水自廬江入為西河,合鵝毛、永安、直皁,是為泥汊河口。又北,神塘河口。又東逕北蟂蟣山,其西獺浦,入和。北有濡須水,自巢湖緣界,逕七寶山,又東為黃洛河,合州河、運河及馬腸、奧龍河,入含山為裕溪。有汛。黃洛、泥汊、奧龍、土橋巡司四。

鳳陽府:沖,繁,疲,難。分巡鳳潁六泗道治所。元,濠州。明初升府曰臨濠。洪武二年為中都。六年改中立府。七年更名鳳陽,屬江南。順治初,因明制,領五州、十三縣,屬江南左布政使司。康熙六年,分隸安徽省。雍正三年,升潁、亳、泗三州為直隸州,分潁上、霍丘屬潁,太和、蒙城屬亳,盱眙、天長、五河屬泗。十一年,分壽州置鳳臺縣。十三年,潁州府以亳州及所隸二縣屬之。乾隆二十年,省臨淮入鳳陽。四十二年,省虹縣入泗州。南距省治六百七十里。廣四百二十八里,袤四百八十里。北極高三十二度五十五分。京師偏東一度十二分。領州二,縣五。鳳陽沖,繁,疲。倚。明始析臨淮置。尋又割虹縣地益之,為府治。國初廢臨淮縣,省入。北:鳳凰山,府以此名。東北:烏雲山,出鍾乳。淮水自壽州入,逕城東北流入泗州。濠水出城南,有二源,至升高東有巨石絕水,即古濠梁,一名石梁河,東北入淮。渦水自蒙城入,逕城西北入懷遠。西:龍子河,源出南山,匯為湖,逕長淮關入淮。北:沫河,上承諸湖,逕城東北入淮,曰沫河口。東:溪河,一名大溪河,即古黃溪也。東:月明湖,北流入淮。東北:花園湖,東北,洪塘湖。東南:明孝陵,在縣西南,有城衛。順治七年,改設左衛,守備一。西北:長淮關。東北:臨淮關。鐵路所經:臨淮鄉、徐家橋、溪河集、蚌埠、小溪。有溪河集縣丞一。蚌埠鎮主簿一。臨淮鎮巡司一。驛三:王莊、濠梁、紅心。縣東南有鐵路。懷遠疲,難。府西北七十里。北:荊山。東南:塗山。南:平阿山。淮水自鳳臺入,逕縣東,過荊、塗兩山間,會渦、濠、沙、淝諸水,合流入泗州。北淝水自蒙城入,至縣正義村,匯為巨浸,下流入靈壁。清溝自渦陽龍山湖東南流,合十湖、天堰諸水,至縣北會淝水,而水始大。舊自靈壁南至沫河口入淮。渦水自鳳陽入,逕城北,東入淮,謂之渦口。南:洛水,北流入淮,亦名洛澗。沙水自潁州入,經荊山南入淮。上窯龍元集有主簿一。洛河巡司一。定遠沖,繁。府南九十里。西北:橫澗山。東:銀嶺。南:池河,自巢入,東北逕盱眙入淮,謂之池口。西:洛河,上承苑馬塘,即淝水支流。二河俱入於淮。芡河從西至,逕城南會淮水。岱山鋪有汛。瀘橋主簿一。池河巡檢一。驛三:定遠、張橋、永康鎮。縣東有鐵路。鳳臺繁,疲,難。府西南百八十里。明省入壽州治。雍正十一年,分壽州城東北隅增置。西北:八公山。東北:紫金山。南淝水自渦陽入,歷潁上,由峽口西入淮。西淝河一名夏肥水,自合肥入,至肥口入淮。白龍潭、顧家橋、石頭埠、劉家集、闞甿集有汛。闞甿集巡司一。驛二:太行、丁家集。壽州繁,疲,難。府西百八十里。壽春鎮總兵駐。城北:八公山,在淝北淮南,亦名北山。峽石山西北夾淮為險,在西岸為峽石,在東岸為壽陽山。西北:淮水自霍丘東逕正陽鎮,潁水流合焉,謂之潁口。又東至城北,淝水流合焉,謂之淝口,亦謂之淮口。又東北流入懷遠。淝水凡三。在州境者曰東淝河,在州東北,源出合肥雞鳴山。北流分為二,一東南入巢湖,一西北流至州入淮,乃淮南之淝水也。西北:潁水自潁州入,入淮處名潁尾。西:渒水自潁州入,北流達於淮,即沘水也。正陽關、瓦搗汛有汛,鳳陽通判駐。有稅關。正陽鎮巡司一。驛四:正陽關、安豐、姚皋店、瓦埠。宿州沖,繁,疲,難。府西北二百三十里。西北:相山、石山、土山。又諸陽山,一名睢陽山,在睢水之陽,睢水自河南永城入。南:澳水,一名濊水,今名澮河,亦自河南永城入,經靈壁東南入泗州五河。東南:沱水,出州東南紫蘆河,東流入靈壁,分二派,至泗州復合,由五河入淮,即洨水也。又北淝水,出州西龍山湖,本入渦,今入淮。西南:泡水,源出亳州舒安湖,流逕廢臨渙城,與澮水合。東南:澥河,亦東流入澮河,一名蟹河。睢水,州北,自河南入,逕相城故城,合瀆水及渒湖水,過陵子湖、崔家湖入泗州。宿州營原設都司一員,乾隆初改守備,嘉慶十一年又改都司。龍山、百善有分防營汛二。有衛。南平集,鳳潁捕盜同知一,州判一。時村集巡司一。驛四:大店、夾溝、睢陽、百善。城外有鐵路。靈壁沖,繁,疲,難。府西北百八十里。本虹縣靈壁鎮,宋始置縣。明屬宿州。清初降宿州,同隸鳳陽治。西南:齊眉。北:磬石。西:鳳凰山。北:黃河自江蘇徐州入,東南入睢寧,即古泗水。北淝水自懷遠入,逕城南,至鳳陽沫河口入睢。澮河、汴水、沱河皆自宿州入,逕縣境,下流入泗州,北小河上流即睢水,亦自宿州入,又東入江蘇睢寧。東有石湖,北有穆家湖、土山湖。雙興鎮州同一。固鎮有汛。巡司一。驛一:靈壁。

潁州府:繁,疲,難。隸鳳潁六泗道。明,潁州,屬鳳陽府。順治初,因明制,與潁上、太和二縣俱屬鳳陽。雍正二年,升直隸州,改隸安徽省,以潁上暨霍丘來屬,分太和屬亳州。十三年升府,增設阜陽縣,降亳州及所隸太和、蒙城二縣來屬隸。東南距省治八百四十里。廣二百一十里,袤二百二十里。北極高三十二度五十八分。京師偏西三十二分。領州一,縣六。阜陽繁,疲,難。倚。西:七旗嶺、金牛嶺。縣西南:仁勝崗。南:安舟崗。淮水自河南固始入,逕城南三河尖入鳳陽。潁水自河南登封入,逕城北東流,茨河、穀河來入之。北:沙河,承太和諸湖水亦來會。西:柳河,承小汝河、白洋湖諸水,並納於潁。東南流,至沫河口達於淮。西:舊黃河,原經城北合潁水。自河徙鹿邑,黃流遂絕。西北:沈丘鎮,即古寢丘。巡檢一。包家寨、永安鎮、西洋集、驛口橋有汛。王家集,通判一,縣丞一。一驛:橋口。潁上疲,難。府東南百二十里。西南:黃崗。東南:垂崗。北:管穀。西南:淮水自阜陽入,合清河、大潤河,至西正陽城,折東北八里垛。潁水自潁州入,逕漢慎縣,合烏江水,又東南合樊家湖,至城東。又東南,右合老梧岡湖來會,潁謂潁尾也,又東北入鳳臺。其北花水澗、袿溝、濟水入鳳臺。八里垛有汛。一驛:甘城。霍丘繁,疲,難。府東南二百九十里。明屬壽州。雍正初,改隸潁。南:九仙、九丈潭。西:長山,三山相連。西北:臨水山。淮水自河南固始入。西南:史家河自六安入,逕葉家集,錯固始復入,合曲河,至三河尖來會。又東合眾水,逕義城廢縣,分水戎河、渒河入鳳臺。水戎河亦入淮。葉家集有汛。洪家集、三河尖二巡司。開順集巡司、典史各一。亳州沖,繁,難。府北百八十里。明初降為縣,尋復故,屬鳳陽府。雍正十三年仍降屬州來隸。西:渦河,自河南鹿邑入,北馬尚河,合流入蒙城。馬尚河在城北,自河南商丘汴河分流,逕州境,包河來注之,下流入渦。其支流入河南永城,謂之澮水。南:淝河自河南鹿邑入,流至州境孟家橋,東流,逕城南入太和。又逕州之龍德寺入阜陽,即夏肥水也。西北聶家湖、花馬潭,東南百尺溝,均入渦。東:義門鎮。龍德寺集、翟家集有汛。州同一,駐丁園寺集。渦陽沖,繁,難。府東北二百七十里。同治三年,割阜陽、亳州、蒙城及鳳陽府之宿州地增置。南:雲夢山。東北:龍山。北:石弓山。北淝河自亳州入,瀦為白湖窪,又東入蒙城。渦河亦自亳入,會五毒溝、龍鳳溝、梭溝、銀溝、金溝諸水始大,東南流,逕蒙城,達懷遠,入淮。西南:蔡湖,亦入渦。東南西洋有汛。西北義門集巡司一。太和繁,疲,難。府西北八十里。明屬鳳陽。雍正間改隸潁。北:萬壽山。沙河自河南沈丘入,逕城南,達亳州,入潁,即潁水上流。東北:茨河,自河南鹿邑入,又東南入沙河,故沙河亦蒙茨河之名。其支流為宋塘河,流逕宋王城入穀河。穀河自西北臥龍岡分流入茨,銘河從之。南:柳河,舊黃河支流也,上通河南項城,下達潁州,合城西舒陽河入沙河。青泥淺有汛。洪山巡司及典史各一。蒙城繁,疲,難。府東北百八十里。順治初屬亳州,尋同太和改隸潁。西北:駝山、狼山。北:檀城山。渦水自渦陽入,逕城北,再折而東,南流,由懷遠渦口入淮。北淝河逕城北板橋集入鳳陽。雙澗集有汛。

徽州府:繁,疲,難。隸徽寧池太廣道。明,徽州府,屬江南。順治初因之,屬江南左布政使司。康熙六年,分隸安徽省。西北距省治五百七十里。廣三百九十里,袤二百二十里。北極高二十九度五十七分。京師偏東二度四分。領縣六。歙縣繁,疲。倚。南:紫陽山。東:問政山。西北:黃山,舊名黟山,盤亙三百餘里,浙、歙、饒、池諸山皆支脈也。豐樂水出黃山,流至城西合揚之水。揚之水自績溪入,達城西,名練溪,一名徽溪,南達歙浦,謂之浦口,為新安江上流,下至浙江建德,與東陽江合,為浙江上源。歙浦在縣南,練江、漸江合流於此。又南昌溪,北洪武水,皆足溉田。明初設課稅局,兼置巡司,今廢。阮溪司、黃山、街口渡巡司三。驛一:歙縣。休寧繁,疲。府西六十里。北:松蘿。東:萬安山。西:白嶽。西北:率山。率水出其陽,水南下而西流者會於彭蠡。其北水分二支:一出梅溪口入祁門,合孚溪水;一出彭沍阬口,會流至縣西江潭,合浙溪水,流逕南港、東港,會於率口,入歙浦,其下流為新安江。南:汊水出白際山,與佩瑯水、璜源水合流,繞縣南岐陽山下,因名汊水,又北流入浙溪。西:白鶴溪,源出黟縣吉陽山,合夾源、夾溪二水,逕縣南,與南港、東港合流入屯溪。屯溪,縣東南,為茶務都會,鹽捕同知駐此。太廈鎮巡司。一驛:休寧。婺源繁,疲。府西南二百四十里。北:浙源山,浙溪出,下流入休寧。梅源水出西梅源山,合武溪。婺水出西北大廣山,南會斜水入武溪。武溪水出北回嶺下,下流逕江西樂平入鄱陽湖。縣境之水,出自縣東及東北者,會流於汪口之西,為北港;出自縣北者,會流於清華之西,為西港。北至武口,二水合流,繞城而西,又西南流入江西德興,下流注鄱陽湖。項村巡司。一驛:婺源。祁門疲,難。府西百八十里。西:新安。東北:祁山。北:大共,亦大洪,巡司駐。大共水西流,合武亭及禾戍嶺水,至秀溪、霄溪下閭門灘,會大北港水,注倒湖,入江西浮梁。西武陵、騄溪二水,東南王公峰水,西南新安、盧溪二水,皆入大共。大洪巡司。一驛:祁門。黟縣簡。府西北百四十里。縣以黟山名,即今黃山也。西南:林歷。東北:吉陽,吉陽水出,一名黟水,西南流,北牛泉水東南來注之。又東南過噎澤,至白茅渡,會橫江水。橫江水南出武亭山,章水自東南流縣西來合之,至魚亭口,會魚亭水,復東流,合吉陽水,入休寧。西:武關,接祁門界。一驛:黟縣。績溪疲,難。府東北六十里。唐始分歙縣地置。東:大障山,一名玉山,山海經三天子鄣山即此。東北:巃鷁山,其山四合,中一徑通寧國。舊有叢山關,其下巧溪,亦名揚溪,流為揚之水,分二支,一北流入寧國,一南流至大屏山,乳溪水、徽水來注之。東北:大障水,會登水,合為臨溪。又西會上溪水,入練溪。東績溪源亦出揚溪,與徽水交流如績,縣名昉此。西北:太平鎮有徽嶺關。濠寨巡司。一驛:績溪。

寧國府:繁,難。隸徽寧池太廣道。明,寧國府,屬江南。順治初因之,屬江南左布政使司。康熙六年,分隸安徽省。西北距省治四百三十里。廣二百二十里,袤三百三十五里。北極高二十度二分。京師偏東二度十六分。領縣六。宣城繁,疲,難。倚。響山,縣南。城內:陵陽山。城北隅:敬亭山。南:響山。東南:華陽山,盤亙宣、涇、寧、旌四縣,華陽之水出焉。東流逕魯山為魯顯水。又東北流為魯溪,會句溪、宛溪、雙溪,北流入青草湖,復合南湖、慈溪,由蕪湖入江。東北有大南崎、小南崎湖。又綏溪一名白河,納廣德、建平諸水,入南湖。西:青弋江,漢志為青水,一名冷水,自涇縣入,匯西南境諸水,東北流,會太平黃池河,入蕪湖。北灣沚河有鎮,今為鹽埠,漕運並會此。其水出揚青口,亦會黃池河。西:青弋關。水陽鎮巡司。西河、楊柳鋪、沈村並有汛。一驛:宣城。寧國簡。府東北九十里。南:鳳山。東:銀山。南:巃鷁山。徽水自績溪入,合仙人洞、篁嶺、滑渡、葛村、龍潭諸水,是為西溪。又東北流合東溪。東溪出浙江天目山,入縣境,合湯公山、博里溪塘、千頃山、洋丁山諸水,流為杭水,北受宣城柏見溪水,是為句溪上源。岳山、湖樂二巡司。一驛:寧國。涇疲,難。府南百里。西南:石柱。東北:幙山。北:琴高。西南:藍山。南:涇水自旌德入,北流,一名藤溪,納楓村、小溪諸水,北入巖潭,與賞溪合。賞溪上源為舒溪、麻川,二水相合,出麻口,入縣境,會烏石水。藤溪,北流至城西南,納烏溪、西阬水、幙溪水,又北逕馬頭山蘆塘入青弋江。琴溪東北受曹溪、丁溪水,與賞溪合。南花林、方村二水,並入賞溪。東南有蘭石鎮、黃沙鎮。縣丞一,駐查村。茹麻嶺巡司一。一驛:涇縣。太平疲,難。府西南二百二十里。唐析涇縣地置。西:龍門。南:黃山,麻川出其麓,與舒溪合流,入涇之賞溪。梅溪水出縣北三門山,合麻川,為麻口。又有瀼、鐍二溪水,亦同注麻川。浮丘垣、譚家橋有汛。宏潭鎮巡司。一驛:太平。旌德繁,難。府南二百二十里。唐永泰中,始析太平縣置。東南:大鰲石島。北:石壁。西南:蛟山、天井。徽水自績溪入,南合清潭,霞溪水自東溪來注。又合績溪之龍頭水,北過石壁山,與抱麟溪、玉溪水合,是名三溪。北至龍首山入涇縣,為涇水上源。抱鱗溪源出黃華嶺,東流,與陶環溪、豐溪合,亦曰三溪。陶環溪即玉溪也。有分防營汛一。三溪鎮巡司。一驛:旌德。南陵繁,難。府西九十里。南:呂山,有泉湧出,即淮水之源也。南流至孔鎮浦,與漳水合,為澄清河。繞城東流為東溪,一名浣溪。縣南鵝嶺諸溪水皆匯焉。又北受籍山、後港、蒲橋諸水,為小淮河,並入蕪湖石硊渡入青弋江。西港源出玉山朗陵之南,合諸水北流,自西南水門入城,繞治前過東市,曰中港,其出城西北水門者曰後港。鵝嶺鎮巡司一。一驛:公館。

池州府:沖,疲。隸徽寧池太道。明,池州府,屬江南。順治初因之,屬江南左布政使司。康熙六年,分隸安徽省。西北距省治一百二十里。廣三百七十里,袤二百三十五里。北極高三十度四十五分。京師偏東五十九分。領縣六。貴池沖,繁。倚。西南:大雄山。東:碧山,濱湖。南:大棕。西:烏石。大江自東流緣界逕縣北至吉陽河,北折至大通河,入銅陵。西:貴池水,一名池口河,北達大江,古稱貴口。大通河東北與銅陵界。梅根河自青陽入,至縣東鬥龍山,沿五埠河口,合雙河,北注大江。一名梅根港,又曰錢溪,為歷代鑄錢之所。東北:清溪河,源出洘溪者為上清溪,出南太僕山者為下清溪,俱東北入江。西南:秋浦。西北:池口鎮。黃龍磯廢巡司一。殷家匯汛一。池口驛一。李陽河鎮巡司一。碧湖村縣丞一。青陽沖,難。府東八十里。北:青山。西南:九華,原名九子山,梅根水出,流入貴池。大江逕縣北百里,濱江有鎮曰大通,鹽茶所集。西:五溪俱出九華山,合流北匯為大通河。臨城河亦西流會於大通河。南:博山河、三溪河、七溪河,均下流入石埭。東南:陵陽鎮有廢司。五溪汛。一驛:青陽。銅陵沖,繁。府東北百二十里。東:銅井、杏山。北:鵲頭山,古名鵲頭戍。西:雲門。南:伏牛、石耳。西南:大江自貴池入,合大通河。大通河別派匯縣南之車橋湖,至大通鎮入江。北:天門水,出天門山,由縣東北至荻港達江,為境內眾水合流入江之口,匯而為河,縣東湖城所出之順安河來合焉。西接鳳心徬,北接黃滸。鳳心徬河會東湖、西湖水達荻港。黃滸河東北自南陵入,西流合荻港。棲鳳湖在縣東南,源出儀鳳嶺,下流通鳳心徬。西南和悅州,一名荷葉洲,汛一。並有大通營水師駐此。池太分防同知一。大通鎮巡檢一。驛一:銅陵。石埭疲,難。府東南百六十里。西:城子、雨臺。南:蓋山、慈云。北:陵陽。池口河源出櫟山,西流,經龍須河,會蒼隼潭,為秋浦,貢溪水西來入之。西:管溪,源亦出櫟山,至管口入石埭鄉,與大洪嶺水合。西南:鴻陵溪,西北流,合舒溪,自太平西北流入縣西舒泉鄉,合縣南之佘溪、前溪,縣北縣西之後溪、岳溪,西南之船溪,東入太平。縣西有巨石三,橫亙溪中,曰頭埭、中埭、下埭,縣名以此。有汛一。驛一:石埭。建德簡。府西南百八十里。治白象山麓。南:玉峰、南豐。西南:梅山。東南:艮木嶺,黃湓河出焉,東流入貴池。前河出東南石門嶺,匯為官池。後河出石門別嶺,亦名石門溪,一曰南河,流至雙河口,與貴池西溪水合,入東流。南:龍口河,縣南迤入江西饒州府之獨山湖。南:永豐鎮。有汛一。巡司一。東流沖,疲。府西百八十里。南:馬當山,橫枕江流,險。安慶、宿松、江西之彭澤,皆以此山為界。西南:大江,自馬當東北流,逕香口、青陽諸鎮,至黃湓河入貴池。城西江口河、南東流河、香口河,下流皆入江。南黃金、白洋二湖,東大清湖,亦皆入江。黃石磯,

東北濱江。香河鎮,明置巡司,今移駐青陽鎮。張家鎮舊有河泊所,雁汊鎮昔有巡司,今皆裁廢。有汛。驛一:東流。

太平府:沖,簡。隸徽寧池太廣道。長江水師提督駐。明,太平府,屬江南。順治初因之,屬江南左布政使司。康熙六年,分隸安徽省。西南距省治一百九十里。廣九十里,袤二百一十里。北極高三十一度三十八分。京師偏東二度三分。領縣三。當塗沖,繁。倚。南:凌家、甑山。南、東南:青山、龍山。北:採石山,一名牛渚。西:博望山,即天門山,又名東梁山,與和州西梁山夾岸對峙。大江自繁昌荻港入,過東西梁山,繞城北而東下採石入江南。東南:丹陽湖。東南再東則固城湖、石臼湖,總名三湖。徽、寧、池、廣及江寧之水畢匯,南流入蕪湖,北為姑熟溪上源。新壩,東南引姑熟水入城壕。中軍守備駐此。黃池河自宣城入,受丹陽南入之水,西北流,合夾河入江。烏溪、黃池鎮、金柱關有汛。池太分防捕盜同知一,管糧通判一,縣丞一。採石、大信巡司二。一驛:採石。蕪湖沖,繁。府西南六十里。東北:★山,山色純赤,古丹陽郡因此得名。西南:戰鳥山,一名孤圻山,對岸孤立為蟂磯。大江自繁昌入,逕城西,為中江故道。南:魯港,上承青弋江,下並高淳東灞之水入江。西南:蕪湖。自丹陽湖南支分流,合青弋江及五丈、路西諸湖之水,西流逕城南,為長河,北入江。東:扁擔河,即長河分流,入當塗,合大信河。東南:天成湖,亦丹陽湖下流所匯,流達長河。徽寧池太廣道、監督工關鈔關,駐江口。蕪湖、採石有汛。蕪湖關商埠,咸豐八年開。河口鎮巡司。一驛:魯港。繁昌簡。府西南百三十里。南:磕山,一名蜃居山。西北:鳳皇。東北:三山磯。大江自銅陵入,逕城北而東,過蕪湖、當塗入江南界,合黃滸河,匯於荻港入江。東:小淮水自南陵入,會城河入蕪湖。一驛:荻港。有汛。河口鎮、三山司、荻港巡司三。

廣德直隸州:繁,難。隸徽寧池太廣道。明初廣興府,置縣曰廣陽。尋降州,直隸江南。順治初因之,屬江南左布政使司。康熙六年,分隸安徽省。西距省治五百九十里。廣一百三十里,袤一百六十里。北極高三十度五十九分。京師偏東二度五十四分。領縣一。西:橫山。東南:桃花、乾溪。西北:白茅嶺。南:桐源山一名白石山。桐水出,南橫梗溪、東南鯉洪溪,皆合焉。北:九斗川,源出五花巖山,匯諸山澗水,西北流,逕建平,匯於郎溪。西:玉溪,繞城北,合碧溪、大源溪,同入建平之南綺湖。青洪山嶺,誓節渡有汛。州判一。杭村、廣安巡司二。建平繁,難。州西北九十里。西北:鳳棲山、五牙山。南:鎮山。西南:赤山。桐水自州入,逕城西入宣城,為白河川,匯於江南之丹陽湖,入大江,或謂之白石水。南綺湖受縣境諸水,北入丹陽湖。郎溪,三峽、蘇大二溪,逕城西南,匯諸山澗水,入南綺湖。白茅山有汛。梅渚巡司一。

滁州直隸州:沖,繁。隸安廬滁和道。明初以州治清流縣。省入,直隸江南。順治初因之,屬江南左布政使司。康熙六年,分隸安徽省。西南距省治五百五十里。廣一百四十里,袤三百一十里。北極高三十二度十七分。京師偏東一度五十三分。領縣二。州境皆山。西:瑯琊。東南:皇道。西北:清流河所出,一名北角河,繞城至烏衣,東合來安水入滁河。其別出者為白茅河,逕城西北入清流河。滁河東南自全椒入,合襄水、清流,曰三汊河口,下流入江蘇六合。大沙河自來安入,匯西北諸山溪水,至城東達清流河。小沙河源出西南菱山,逕城西,注石瀨澗以合清流。烏衣有汛。大槍鎮巡司一。有鐵路。全椒簡。州南五十里。北:覆釜山,城跨其上。西北:桑根山,有南隱、中隱、北隱。南:南崗。東南:九斗,一名徐陵山。滁河南源出廬,自合肥入,至石潭,與襄水合,入滁州。襄水源出西北石臼山,東南流,合澗穀諸水,亦至石潭達滁。西南:酂湖,居民引流資灌溉。南:六丈鎮。鳳皇橋有汛一。驛二:大柳、滁陽。來安簡,州東北四十里。西:嘉山。北:馬嶺山。東:五湖山。西北:北信山。來安水出五湖山,逕縣東,至水口鎮入滁州。西北:沛水,有二源,一出盱眙、招信界嶺下,一出練寺山,二水合而南流入州。獨山水、秋沛水皆由縣西北合流,至瓦店河,同入滁河。東北:白塔鎮。有汛。

和州直隸州:繁,疲。隸安廬滁和道。明初以州治歷陽縣,省入,尋復和州,直隸江南。順治初因之,屬江南左布政使司。康熙六年,分隸安徽省。西南距省治四百六十里。廣一百八十里,袤二百里。北極高三十一度四十四分。京師偏東一度五十一分。領縣一。西:歷陽。南:梁山。西北:烏石山。北:夾山。大江自無為州入,又東北入江蘇江浦。西南:柵山,與無為州分中流為界,即古濡須口也,白石水自含山西南來注之。東南:橫江,南直採石磯,亦名橫江浦,會開勝河,東流達江。南:裕溪河,源出巢湖,自無為入,上承牛屯河,入江。東北:石拔河、芝麻河、穴子河,皆入江。東南:當利浦,一名河口,大江之別浦也。州同一。牛屯河巡司一。裕溪、新河口、瓦蓬溝有汛。含山簡。州西六十里。北:大小峴山,一名赤焰山。西南:白石山,道書第二十一洞天也。濡須水出,是為東關口水,自巢湖東流,逕亞父山,出東關口,東南逕黃洛河,又南逕運漕河,至新浴口會西清溪河,至柵江口同入大江,一名天河。東南:銅城徬,受天河、黃洛河支流,東至徬口分流,一支為牛屯河,入州,一支南出,入三汊河。練固鎮、裕溪河鎮有汛。巡司二:運漕,裕溪。

六安直隸州:繁,疲,難。隸鳳潁六泗道。明初以州治六安縣,省入,屬鳳陽府,尋還屬廬州府。順治初因之。雍正二年,升直隸州,屬安徽省。東南距省治四百四十里。廣二百一十里,袤二百二十里。北極高三十一度五十分。京師偏東二分。領縣二。東:龍穴山一名龍池山,與合肥界。東南:洪家山,四圍壁立。南:大小同山。西南:團山,下臨渒水。渒水一名白沙河,源出霍山,逕城西,又北流入河南固始,即沘水也。西南青石河,西三元幢河、青龍河,皆入渒。東南:馬柵河,流逕舒城桃城鎮入巢湖。西:溶水河,源出齊雲山,西北流,入河南固始,合史河。西南:麻埠鎮。錢家集有汛。和尚司、馬頭汛二。巡司一。驛二:六安、椿樹崗。英山簡。州西南三百六十里。東:英山,縣以此名。北:雞鳴山。南:密峰尖、三吳山。西北:多雲山。西:岐嶺,通湖廣界。英山河出英山,有二源,東曰東矼,西曰西矼,南流至城南而合。會北澗水,流入湖北蘄水。南:雞兒河,亦由蘄水入江。北柳林關,西石門關,亦險要也。茅草畈有汛。七引店巡司一。霍山繁,難。州西南九十里。西北:霍山,又名天柱山,亦名南嶽。東:復覽山。西南:四十八盤。東南:鐵爐山。渒水即沘水,出沘山,俗名太陽河,北逕磨子潭,右合中埠及雙河,至天柱山西,左合漫水及陡山桃源河,又東北逕城西。有潛臺山,其西六安山。又北合新店河、楮皮嶺水,入州東梅子關。包家河有汛一。上土市鎮巡司。千羅畈鎮縣廢司。

泗州直隸州:繁,疲,難。隸潁六泗道。明屬鳳陽府。尋復升直隸州,以臨淮縣省入。順治初因之。康熙六年,分屬安徽省,隸鳳陽如故。十九年,州城圮,陷入洪澤湖,寄治盱眙。雍正二年,升直隸州,隸安徽省。乾隆四十二年,裁鳳陽府之虹縣,省入泗州為州治。泗州舊治在今州城東南百八十里。自明末清口久淤,舊黃河堤決,黃流奪淮,水倒灌入泗,州境時有水患。至清康熙十九年,城遂圮陷於湖。今州治即虹縣舊城。東北距省治七百六十里。廣二百九十里,袤二百里。北極高三十三度二十八分。京師偏東一度二十三分。領縣三。北:屏山,下有湖。南:鹿鳴山。東:秦橋山,有雙泉。東北:硃山,上有聖水井,下有峰山湖。南:淮水自鳳陽廢臨淮入,逕五河入洪澤湖。汴河自靈壁入,東南入淮,即莨蕩渠,一名浚儀渠,唐、宋通漕故道。睢河逕城北,東流,會安河窪,南注洪澤湖。潼河在故虹縣西,俗曰南潼河,自萬安湖流逕五河注淮。北潼水,在今州北,東流注駱馬湖。沱河在今州西南,源出宿州紫蘆湖,逕州東為南沱湖,州西為北沱河,二水合流入五湖。又石梁河、天井湖,西合漴水,過五河入淮。施家崗有汛。半城鎮,州判駐。雙溝鎮,同知駐。驛二:泗水、龍窩。盱眙疲,難。州南百里,濱湖倚山,無城郭。康熙間,泗州陷於湖,乃寄州治於此。後以虹縣省入泗州,乃復為屬如故。東:盱眙山,縣以此名。南:寶積山。北:陡山、龜山。東南:都梁。西北:浮山,濱淮水,故一名臨淮山。淮河逕城北,匯洪澤湖。與泗州中流分界。自五河流入,東北至清河口合黃河。東北:運河。池河自合肥入,北注于淮。洪澤湖舊名破釜塘,亦古洪澤鎮地,昔人開水門入以資灌田。自泗州陷入,湖界日巨,汪洋幾三百里,延袤皖、蘇二省。南以老子山、北以湖泊崗,與江蘇桃源縣分界。舊縣有汛一。驛二:淮原、都梁。天長疲,難。州東南百五十七里。南:橫山、冶山。西:望城崗。北:紅山。西北:石梁河,自滁州來安入,匯為五湖。北合德勝河,又東接高郵寘沙湖,其分流為樊梁溪。白塔河自來安入,合汊澗,逕石梁鎮,又東大河灣,至城西,右合白楊河,東北瀦為丁溪湖,播為感蕩、上泊、白馬、沂洋諸湖。其南秦蘭河,並入江蘇,注寘沙湖。東北:下河鎮。北:銅城鎮。汊澗有汛一。城門鄉巡司一。一驛:安淮。五河疲。州南百三十里。南:金崗。西南:翠柏。西:臥龍崗,下有龍潭。北:陡山崗。沱河水溢,淮水在城東一里。自故臨淮縣東北流逕此,又東入州境。澮河自靈壁入,舊逕城南一里,後水漲沙淤,徙於北滸,又逕城西北合沱河,又東入淮,或謂之澳水。東潼河自州入,逕天井湖,南至鐵鎖嶺入淮。漴河在城南二里。南湖在城南七里,匯眾流而成,流為此河,又東流入淮。以上所謂五河也。其交會處在城東二里,謂之五河口。西:臨淮關,有汛。驛一:五河。


\end{pinyinscope}