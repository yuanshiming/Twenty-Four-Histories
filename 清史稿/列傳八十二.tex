\article{列傳八十二}

\begin{pinyinscope}
隆科多年羹堯胡期恆

隆科多,佟佳氏,滿洲鑲黃旗人,一等公佟國維子,孝懿仁皇后弟也。康熙二十七年,授一等侍衛,擢鑾儀使,兼正藍旗蒙古副都統。四十四年,以所屬人違法,上責隆科多不實心任事,罷副都統、鑾儀使,在一等侍衛上行走。五十年,授步軍統領。五十九年,擢理籓院尚書,仍管步軍統領。六十一年十一月,聖祖大漸,召受顧命。世宗即位,命與大學士馬齊總理事務,襲一等公,授吏部尚書。旋以總理事務勞,加一等阿達哈哈番,以其長子嶽興阿襲。次子玉柱,自侍衛授鑾儀使。雍正元年,與川陜總督年羹堯同加太保。二年,兼領理籓院事。纂修聖祖實錄、大清會典並充總裁,監修明史。復與羹堯同賜雙眼花翎、四團龍補服、黃帶、紫轡。

三年,解步軍統領。玉柱以行止甚劣,奪官,交隆科多管束。羹堯得罪,上以都統範時捷疏劾欺罔貪婪諸狀,及妄劾道員金南瑛等,並下吏部議處。上諭曰:「前以隆科多、年羹堯頗著勤勞,予以異數,乃交結專擅,諸事欺隱。」命繳上所賜四團龍補服,並不得復用雙眼花翎、黃帶、紫轡。及議上,以時捷劾,請罷羹堯任;以妄劾南瑛,請嚴加治罪。上以前議徇庇,後議復過當,責隆科多有意擾亂,削太保及一等阿達哈哈番世職,命往阿蘭善等處修城墾地,諭曰:「朕御極之初,隆科多、年羹堯皆寄以心腹,毫無猜防。孰知朕視為一德,彼竟有二心,招權納賄,擅作威福,欺罔悖負,朕豈能姑息養奸耶?向日明珠、索額圖結黨行私,聖祖解其要職,置之閒散,何嘗更加信用?隆科多、年羹堯若不知恐懼,痛改前非,欲如明珠等,萬不能也!殊典不可再邀,覆轍不可屢蹈,各宜警懼,毋自干誅滅。」四年,隆科多家僕牛倫挾勢索賕,事發,逮下法司,鞫得隆科多受羹堯及總督趙世顯、滿保,巡撫甘國璧、蘇克濟賄。讞上,上命斬倫,罷隆科多尚書,令料理阿爾泰等路邊疆事務。尋命勘議俄羅斯邊界。

初,隆科多與阿靈阿、揆敘相黨附,既又與羹堯交結。至是,上盡發阿靈阿、揆敘及羹堯罪狀,宣示中外。又侍郎查嗣庭為隆科多所薦,坐悖逆誅死,上詰隆科多,隆科多不以實對。五年,宗人府復奏劾輔國公阿布蘭以玉牒畀隆科多藏於家,阿布蘭坐奪爵幽禁。上命奪隆科多爵,召還京,命王大臣會鞫。以聖祖升遐,隆科多未在上前,妄言身藏匕首以防不測;又自擬諸葛亮,奏稱「白帝城受命之日,即死期將至之時」;上躬祀壇廟,妄謂防刺客,令於案下搜查;上謁陵,妄奏「諸王心變」。具獄辭:大不敬之罪五,欺罔之罪四,紊亂朝政之罪三,黨奸之罪六,不法之罪七,貪婪之罪十六,凡四十一款,當斬,妻子入辛者庫,財產入官。上諭曰:「隆科多罪不容誅,但皇考升遐,大臣承旨者惟隆科多一人。今以罪誅,朕心有所不忍,可免其正法,於暢春園外築屋三楹,永遠禁錮;妻子免入辛者庫,嶽興阿奪官,玉柱發黑龍江。」六年六月,隆科多死於禁所,賜金治喪。

年羹堯,字亮工,漢軍鑲黃旗人。父遐齡,自筆帖式授兵部主事,再遷刑部郎中。康熙二十二年,授河南道御史。四遷工部侍郎,出為湖廣巡撫。湖北武昌等七府歲徵匠役班價銀千餘,戶絕額缺,為官民累。遐齡請歸地丁徵收,下部議,從之。疏劾黃梅知縣李錦虧賦,奪官。錦清廉得民,民爭完逋賦,諸生吳士光等聚眾閉城留錦。事聞,上命調錦直隸,士光等發奉天,遐齡與總督郭琇俱降級留任。四十三年,遐齡以病乞休。

羹堯,康熙三十九年進士,改庶吉士,授檢討。迭充四川、廣東鄉試考官,累遷內閣學士。四十八年,擢四川巡撫。四十九年,斡偉生番羅都等掠寧番衛,戕游擊周玉麟。上命羹堯與提督岳升龍剿撫。升龍率兵討之,擒羅都,羹堯至平番衛,聞羅都已擒,引還。川陜總督音泰疏劾,部議當奪官,上命留任。五十六年,越巂衛屬番與普雄土千戶那交等為亂,羹堯遣游擊張玉剿平之。

是歲,策妄阿喇布坦遣其將策凌敦多卜襲西藏,戕拉藏汗。四川提督康泰率兵出黃勝關,兵譁,引還。羹堯遣參將楊盡信撫諭之,密奏泰失兵心,不可用,請親赴松潘協理軍務。上嘉其實心任事,遣都統法喇率兵赴四川助剿。五十七年,羹堯令護軍統領溫普進駐里塘,增設打箭爐至里塘驛站,尋請增設四川駐防兵,皆允之。上嘉羹堯治事明敏,巡撫無督兵責,特授四川總督,兼管巡撫事。五十八年,羹堯以敵情叵測,請赴藏為備。廷議以松潘諸路軍事重要,令羹堯毋率兵出邊,檄法喇進師。法喇率副將岳鍾琪撫定里塘、巴塘。羹堯亦遣知府遲維德招降乍丫、察木多、察哇諸番目,因請召法喇師還,從之。

五十九年,上命平逆將軍延信率兵自青海入西藏,授羹堯定西將軍印,自拉裡會師,並諮羹堯孰可署總督者。羹堯言一時不得其人,請以將軍印畀護軍統領噶爾弼,而移法喇軍駐打箭爐,上用其議。巴塘、里塘本云南麗江土府屬地,既撫定,雲貴總督蔣陳錫請仍隸麗江土知府木興;羹堯言二地為入藏運糧要路,宜屬四川,從之。興率兵往收地,至喇皮,擊殺番酋巴桑,羹堯疏劾。上命逮興,囚雲南省城。八月,噶爾弼、延信兩軍先後入西藏,策凌敦多卜敗走,西藏平。上諭羹堯護凱旋諸軍入邊,召法喇還京師。

羹堯尋遣兵撫定里塘屬上下牙色、上下雅尼,巴塘屬桑阿壩、林卡石諸生番。六十年,入覲,命兼理四川陜西總督,辭,還鎮,賜弓矢。上命噶爾弼率兵駐守西藏,行次瀘定橋,噶爾弼病不能行,羹堯以聞。上命公策旺諾爾布署將軍,額駙阿寶、都統武格參贊軍務,駐西藏。青海索羅木之西有郭羅克上中下三部,為唐古特種人,屢出肆掠。阿寶以聞,上令羹堯與鍾琪度形勢,策進討。羹堯疏言:「郭羅克有隘口三,悉險峻,宜步不宜騎。若多調兵,塞上傳聞,使賊得為備,不如以番攻番。臣素知瓦斯、雜穀諸土司亦憾郭羅克肆惡,原出兵助剿。臣已移鍾琪令速赴松潘,出塞督土兵進剿。」尋,鍾琪督兵擊敗郭羅克,下番寨四十餘,獲其渠,餘眾悉降。

六十一年,羹堯密疏言:「西藏喇嘛楚爾齊木臧布及知府石如金呈策旺諾爾布委靡,副都統常齡、侍讀學士滿都、員外郎巴特瑪等任意生事,致在藏官兵不睦。」因請撤駐藏官兵。下廷臣議,以羹堯擅議撤兵,請下部嚴議,上原之,命召滿都、巴特瑪、石如金、楚爾齊木臧布等來京師,遣四川巡撫色爾圖、陜西布政使塔琳赴西藏,佐策旺諾爾布駐守。

自軍興,陜西州縣饋運供億,庫帑多虧缺。羹堯累疏論劾州縣吏,嚴督追償。陜西巡撫噶什圖密奏虧項不能速完,又與羹堯請加徵火耗墊補。上諭曰:「各省錢糧皆有虧空,陜西尤甚。蓋自用兵以來,師所經行,資助馬匹、盤費、衣服、食物,倉卒無可措辦,勢必挪用庫帑。及撤兵時亦然。即如自藏回京,將軍以至士卒,途中所得,反多於正項。各官費用,動至萬金,但知取用,不問其出自何項也。羹堯等欲追虧項以充兵餉,追比不得,又議加徵火耗。火耗止可議減,豈可加增?朕在位六十一年,從未加徵火耗。今若聽其加派,必致與正項一例催徵,肆無忌憚矣。著傳旨申飭。」命發帑銀五十萬送陜西資餉。

世宗即位,召撫遠大將軍允還京師,命羹堯管理大將軍印務。雍正元年,授羹堯二等阿達哈哈番世職,並加遐齡尚書銜。尋又加羹堯太保。詔撤西藏駐防官軍。羹堯疏陳邊防諸事,請於打箭爐邊外中渡河口築土城,移嵐州守備駐守;大河南保縣,移威茂營千總駐守;越巂衛地方寥闊,蠻、惈出沒,改設游擊,增兵駐守;松潘邊外諸番,阿樹為最要,給長官司職銜;大金川土目莎羅奔從征羊峒有功,給安撫司職銜;烏蒙蠻目達木等兇暴,土舍祿鼎坤等請擒獻,俟其至,給土職,分轄其地。下部議,從之。論平西藏功,以羹堯運糧守隘,封三等公,世襲。

青海臺吉羅卜藏丹津為顧實汗孫,糾諸臺吉吹拉克諾木齊、阿爾布坦溫布、藏巴札布等,劫親王察罕丹津叛,掠青海諸部。上命羹堯進討,諭撫遠大將軍延信及防邊理餉諸大臣,四川、陜西、雲南督、撫、提、鎮,軍事皆告羹堯。十月,羹堯率師自甘州至西寧,改延信平逆將軍,解撫遠大將軍印授羹堯,盡護諸軍。羹堯請以前鋒統領素丹、提督岳鍾琪為參贊大臣,從之。論平郭羅克功,進公爵二等。

羹堯初至西寧,師未集,羅卜藏丹津詗知之,乃入寇,悉破傍城諸堡,移兵向城。羹堯率左右數十人坐城樓不動,羅卜藏丹津稍引退,圍南堡。羹堯令兵斫賊壘,敵知兵少,不為備,驅桌子山土番當前隊;砲發,土番死者無算。鍾琪兵至,直攻敵營,羅卜藏丹津敗奔,師從之,大潰,僅率百人遁走。羹堯乃部署諸軍,令總兵官周瑛率兵截敵走西藏路,都統穆森駐吐魯番,副將軍阿喇納出噶斯,暫駐布隆吉爾,又遣參將孫繼宗將二千人與阿喇納師會。敵侵鎮海堡,都統武格赴援,敵圍堡,戰六晝夜,參將宋可進等赴援,敵敗走,斬六百餘級,獲多巴囊素阿旺丹津。羅卜藏丹津攻西寧南川口,師保申中堡。敵圍堡,堡內囊素與敵通,欲鑿墻而入。守備馬有仁等力禦,可進等赴援,夾擊,敵敗走,諸囊素助敵者皆殺之。羹堯先後疏聞,並請副都統花色等將鄂爾多斯兵,副都統查克丹等將歸化土默特兵,總兵馬覿伯將大同鎮兵,會甘州助戰,從之。

西寧北川、上下北塔蒙回諸眾將起應羅卜藏丹津,羹堯遣千總馬忠孝撫定下北塔三十餘莊。上北塔未服,忠孝率兵往剿,擒戮其渠,餘眾悉降。察罕丹津走河州,羅卜藏丹津欲劫以去。羹堯令移察罕丹津及其族屬入居蘭州。青海臺吉索諾木達什為羅卜藏丹津誘擒,脫出來歸,羹堯奏聞,命封貝子,令羹堯撫慰。敵掠新城堡,羹堯令西寧總兵黃喜林等往剿,斬千五百餘級,擒其渠七,得器械、駝馬、牛羊無算。以天寒,羹堯令引師還西寧。

尋策來歲進兵,疏:「請選陜西督標西安、固原、寧夏、四川、大同、榆林綠旗兵及蒙古兵萬九千人,令鍾琪等分將,出西寧、松潘、甘州、布隆吉爾四道進討,分兵留守西寧、甘州、布隆吉爾,並駐防永昌、巴塘、里塘、黃勝關、察木多諸隘。軍中馬不足,請發太僕寺上都打布孫腦兒孳生馬三千,巴爾庫爾駝一千,仍於甘、涼增買千五百。糧米,臣已在西安預買六萬石。軍中重火器,請發景山所制火藥一百駝,駝以一百八十斤計。」下廷議,悉如所請,馬加發千,火藥加發倍所請。

察罕丹津屬部殺羅卜藏丹津守者來歸,羹堯宣上指,安置四川邊外。墨爾根戴青拉查卜與羅卜藏丹津合力劫察罕丹津,其子察罕喇卜坦等來歸,羹堯令招拉查卜內附。又有堪布諾門汗,察罕丹津從子也,為塔兒寺喇嘛,叛從敵,糾眾拒戰,至是亦來歸。羹堯數其罪,斬之。羅卜藏丹津侵布隆吉爾,繼宗與副將潘之善擊敗之。西寧南川塞外郭密九部屢出為盜,羹堯招三部內附。餘部行掠如故,呈庫、沃爾賈二部尤暴戾。羹堯令鍾琪率瓦斯、雜穀二土司兵至歸德堡,撫定上下寺東策布,督兵進殲呈庫部眾,擒戮沃爾賈部酋,餘並乞降。

二年,上以羅卜藏丹津負國,叛不可宥,授鍾琪奮威將軍,趣羹堯進兵。西寧東北郭隆寺喇嘛應羅卜藏丹津為亂,羹堯令鍾琪及素丹等督兵討之,賊屯哈拉直溝以拒。師奮入,度嶺三,毀寨十。可進、喜林及總兵武正安皆有斬馘,復毀寨七,焚所居室。至寺外,賊伏山谷間,聚薪縱火,賊殲焉,殺賊六千餘,毀寺,誅其渠。青海貝勒羅卜藏察罕、貝子濟克濟札布、臺吉滾布色卜騰納漢將母妻詣羹堯請內屬,羹堯予以茶葉、大麥,令分居邊上。羹堯遣鍾琪、正安、喜林、可進及侍衛達鼐,副將王嵩、紀成斌將六千人深入,留素丹西寧佐治事。

二月,鍾琪師進次伊克喀爾吉,搜山,獲阿爾布坦溫布,喜林亦得其酋巴珠爾阿喇布坦等。師復進,羹堯詗知阿岡都番助敵,別遣涼莊道蔣泂等督兵攻之,戮其囊素。復擊破石門寺喇嘛,殺六百餘人,焚其寺。鍾琪師復進次席爾哈羅色,遣兵攻噶斯,逐吹拉克諾木齊。三月,鍾琪師復進次布爾哈屯。羅卜藏丹津所居地曰額母訥布隆吉,鍾琪督兵直入,分兵北防柴旦木,斷往噶斯道。羅卜藏丹津走烏蘭穆和兒,復走柴旦木,師從之,獲其母阿爾太哈屯及其戚屬等,並男婦、牛羊、駝馬無算。分兵攻烏蘭白克,獲吹拉克諾木齊及助亂八臺吉。時藏巴扎布已先就擒,羅卜藏丹津以二百餘人遁走。青海部落悉平。論功,進羹堯爵一等,別授精奇尼哈番,令其子斌襲,封遐齡如羹堯爵,加太傅;並授素丹、可進三等阿達哈哈番,喜林二等阿達哈哈番,按察使王景灝及達鼐、瑛、嵩、成斌拜他喇布勒哈番,提督郝玉麟及正安拖沙喇哈番。

阿拉布坦蘇巴泰等截路行劫,羹堯令繼宗往剿,逐至推墨爾,阿拉布坦蘇巴泰將妻子遁走。成斌等搜戮餘賊至梭羅木,擊斬堪布夾木燦垂扎木素。羹堯遣達鼐及成斌攻布哈色布蘇,獲臺吉阿布濟車陳;又遣副將岳超龍討平河州塞外鐵布等七十八寨,殺二千一百餘人,得人口、牲畜無算。羹堯執吹拉克諾木齊、阿爾布坦溫布、藏巴扎布檻送京師。上祭告廟、社、景陵、御午門受俘。羹堯策防邊諸事,以策妄阿喇布坦遣使乞降,請罷北征師,分駐巴里坤、吐魯番、哈密城、布隆吉爾駐兵守焉,轄以總兵,每營撥餘丁屯赤金衛、柳溝所田;設同知理民事,衛守備理屯糧,游牧蒙古令分居布隆吉爾迤南山中。寧夏邊外阿拉善以滿洲兵駐防。上悉從所請。

莊浪邊外謝爾蘇部土番據桌子、棋子二山為巢,皆穴地而居,官軍駐其地,奴使之;兵或縱掠,番御之,盡殲,置不問,番始橫。涼州南崇寺沙馬拉木扎木巴等掠新城張義諸堡。又有郭隆寺逸出喇嘛,與西寧納硃公寺、朝天堂、加爾多寺諸番相結,糾謝爾蘇部土番謀為亂。羹堯遣鍾琪等督兵討之,納硃公寺喇嘛降。師進次朝天堂,遣成斌、喜林及副將張玉等四道攻加爾多寺,殺數百人,餘眾多入水死,焚其寺。游擊馬忠孝、王大勛戰和石溝,王序吉、範世雄戰石門口,泂戰喜逢堡,蘇丹師次旁伯拉夏口,土番偽降,詗之,方置伏,縱兵擊之,所殺傷甚眾。泂搜剿棋子山,逐賊巴洞溝,土司魯華齡逐賊天王溝,先密寺喇嘛縛其渠阿旺策凌以獻。師入,轉戰五十餘日,殺土番殆盡。羹堯以先密寺喇嘛反覆不常,並焚其寺,徙其眾加爾多寺外桌子山;餘眾降,羹堯令隸華齡受約束。

條上青海善後諸事,請以青海諸部編置佐領。三年一入貢,開市那拉薩拉。陜西、雲南、四川三省邊外諸番,增設衛所撫治。諸廟不得過二百楹,喇嘛不得過三百。西寧北川邊外築邊墻,建城堡。大通河設總兵,鹽池、保安堡及打箭爐外木雅吉達、巴塘、里塘諸路皆設兵。發直隸、山西、河南、山東、陜西五省軍罪當遣者,往大通河、布隆吉爾屯田;而令鍾琪將四千人駐西寧,撫綏諸番。下王大臣議行。十月,羹堯入覲,賜雙眼花翎、四團龍補服、黃帶、紫轡、金幣。敘功,加一等阿思哈尼哈番世職,令其子富襲。

羹堯才氣凌厲,恃上眷遇,師出屢有功,驕縱。行文諸督撫,書官斥姓名。請發侍衛從軍,使為前後導引,執鞭墜鐙。入覲,令總督李維鈞、巡撫範時捷跪道送迎。至京師,行絕馳道。王大臣郊迎,不為禮。在邊,蒙古諸王公見必跪,額駙阿寶入謁亦如之。嘗薦陜西布政使胡期恆及景灝可大用,劾四川巡撫蔡珽逮治,上即以授景灝,又擢期恆甘肅巡撫。羹堯僕桑成鼎、魏之耀皆以從軍屢擢,成鼎布政使,之耀副將。羹堯請發將吏數十從軍,上許之。覲還,即劾罷驛道金南瑛等,而請以從軍主事丁松署糧道。上責羹堯題奏錯誤,命期恆率所劾官吏詣京師。三年正月,珽逮至,上召入見,備言羹堯暴貪誣陷狀,上特宥珽罪。

二月庚午,日月合璧,五星聯珠,羹堯疏賀,用「夕惕朝乾」語,上怒,責羹堯有意倒置,諭曰:「羹堯不以朝乾夕惕許朕,則羹堯青海之功,亦在朕許不許之間而未定也。」會期恆至,入見,上以奏對悖謬,奪官。上命更定打箭爐外增汰官兵諸事,不用羹堯議。四月,上諭曰:「羹堯舉劾失當,遣將士築城南坪,不惜番民,致驚惶生事,反以降番復叛具奏。青海蒙古饑饉,匿不上聞。怠玩昏憒,不可復任總督,改授杭州將軍。」而以鍾琪署總督,命上撫遠大將軍印。羹堯既受代,疏言:「臣不敢久居陜西,亦不敢遽赴浙江,今於儀徵水陸交通之處候旨。」上益怒,促羹堯赴任。山西巡撫伊都立、都統前山西巡撫範時捷、川陜總督岳鍾琪、河南巡撫田文鏡、侍郎黃炳、鴻臚少卿單疇書、原任直隸巡撫趙之垣交章發羹堯罪狀,侍郎史貽直、高其佩赴山西按時捷劾羹堯遣兵圍郃陽民堡殺戮無辜,亦以讞辭入奏,上命分案議罪。罷羹堯將軍,授閒散章京,自二等公遞降至拜他喇布勒哈番,乃盡削羹堯職。

十二月,逮至京師,下議政大臣、三法司、九卿會鞫。是月甲戌,具獄辭:羹堯大逆之罪五,欺罔之罪九,僭越之罪十六,狂悖之罪十三,專擅之罪六,忌刻之罪六,殘忍之罪四,貪黷之罪十八,侵蝕之罪十五,凡九十二款,當大闢,親屬緣坐。上諭曰:「羹堯謀逆雖實,而事跡未著,朕念青海之功,不忍加極刑。」遣領侍衛內大臣馬爾賽、步軍統領阿齊圖齎詔諭羹堯獄中令自裁。遐齡及羹堯兄希堯奪官,免其罪;斬其子富;諸子年十五以上皆戍極邊。羹堯幕客鄒魯、汪景祺先後皆坐斬,親屬給披甲為奴。又有靜一道人者,四川巡撫憲德捕送京師,亦誅死。五年,赦羹堯諸子,交遐齡管束。遐齡旋卒,還原職,賜祭。

希堯,初自筆帖式累擢工部侍郎。既,奪官,復起內務府總管,命榷稅淮安,加左都御史。十三年,為江蘇巡撫高其倬劾罷。乾隆三年,卒。

胡期恆,字元方,湖廣武陵人。祖統虞,明崇禎末進士。國初授檢討,官至秘書院學士。父獻徵,自廕生授都察院經歷,官至湖北布政使。期恆,康熙四十四年舉人。獻徵與遐齡友,歡若弟昆,期恆少從羹堯游。上南巡,獻詩,授翰林院典籍。出為夔州通判,有恩信,民為建生祠。羹堯為巡撫,薦期恆,遷夔州知府,再遷川東道。羹堯兼督陜西,復薦遷陜西布政使。期恆通曉朝章國故,才敏,善理繁劇,羹堯深倚之。羹堯挾貴而驕,惟期恆能以微言救其失。羹堯奴辱咸陽知縣,期恆執而杖之,自是諸奴稍斂戢。嘗諷羹堯善持盈,羹堯勿能用。及羹堯敗,諸為羹堯引進者,爭劾羹堯以自解;期恆惟引咎,終不言羹堯,乃下獄頌系。至高宗即位,始得釋。僑居江南,久之,卒。

論曰:雍正初,隆科多以貴戚,年羹堯以戰多,內外夾輔為重臣。乃不旋踵,幽囚誅夷,亡也忽諸。當其貴盛侈汰,隆科多恃元舅之親,受顧命之重;羹堯自代充為大將軍,師所向有功。方且憑藉權勢,無復顧忌,即於覆滅而不自怵。臣罔作威福,古聖所誡,可不謹歟!


\end{pinyinscope}