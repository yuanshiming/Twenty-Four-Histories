\article{列傳四十六}

\begin{pinyinscope}
宜里布哈克三阿爾護路什雅賚擴爾坤王承業王忠孝

宜里布,他塔喇氏,滿洲正白旗人,阿濟格尼堪子也。初授兵部副理事官。順治八年,襲三等伯爵,兼管牛錄。恩詔進一等伯。擢刑部侍郎,調吏部。鄭成功據臺灣為亂,議者謂當徙瀕海居民入內地,以避剽掠,絕接濟,命宜里布與尚書蘇納海歷江南、浙江、福建勘疆界。既定,還京師,擢正白旗蒙古都統。康熙七年,調本旗滿洲都統,列議政大臣。

吳三桂反,十三年,大將軍順承郡王勒爾錦率師討之,以宜里布參贊軍務。既至荊州,三桂自常德攻陷松滋,襄陽總兵楊來嘉、副將洪福叛附之,壁穀城、鄖陽間,窺覗郡邑,詔宜里布守宜昌。十四年,來嘉等犯南漳,順承郡王承制授宜里布討逆將軍印,與副都統根特往援。來嘉等引退,旋復犯均州,壘武當山下,宜里布督兵擊之,斬千餘級,來嘉等復引退。

時三桂屯松滋北山,緣江置戰艦,謀水陸並進。命宜里布與都統範達禮等守襄陽、均州諸處。三桂遣其將張以誠與來嘉等寇南漳,宜里布與總督蔡毓榮分率勁旅夾擊,斬三千餘級。十六年,命與將軍穆占率荊州滿洲兵自岳州下長沙,克茶陵。三桂兵奔攸縣,宜里布追擊之,斬四千餘級,俘百餘,克攸縣。

十七年,穆占師進克郴州、永興諸處,駐師郴州,而令宜里布守永興。三桂遣其將馬寶、胡國柱等來犯,與副都統哈克三督兵御之,力戰,殞於陣。喪還,遣內大臣奠茶酒,復遣侍衛諭其母曰:「宜里布侍朕久,深知其為人。出師有勞績,方謂功成奏凱,即可相見。忽聞陣沒,淒愴痛悼!爾家貧,予白金六百為治喪資。」賜祭葬,謚武壯。子阿什坦襲爵。

哈克三,佟佳氏,滿洲正藍旗人。父法薩里巴圖魯,以驍騎校從征戰沒。哈克三初授禮部筆帖式,累遷員外郎。順治十四年,改授巴牙喇甲喇章京。康熙二年,李自成餘孽李來亨等據茅麓山,剽掠為民害,從將軍穆里瑪等討之。賊入山,哈克三從巴牙喇纛章京堪泰自山後進,大破之;復與總兵於大海夾擊,多所斬馘,來亨自縊死:擢正藍旗蒙古副都統。十二年,調滿洲副都統,尋遷護軍統領。

十四年,察哈爾布爾尼叛,大將軍信郡王鄂扎率師討之,哈克三參贊軍務。師次達祿,布爾尼列陣以待,而隱兵山谷間以誘我師。土默特兵遇伏,哈克三力禦敗之。復督驍騎突賊陣,賊潰奔,斬馘甚眾,布爾尼以三十騎遁。敘功,授三等阿達哈哈番。

十六年,大將軍簡親王喇布討吳三桂,哈克三參贊軍務。三桂將韓大任據萬安,哈克三與副都統雅沁等分道進,大任渡河走。哈克三以山逕隘不容騎,請調綠旗兵守隘,斷賊餉道,上責其稽延,敕窮追毋縱入楚。賊竄興國山中,追擊之黃塘、新田鋪,師舍騎而徒,奮擊,賊大潰。復選輕騎夜逐賊至姜坑嶺,賊據險自保,哈克三分兵環攻,斬千餘級。大任等收餘眾走福建,屯老虎洞。十七年三月,與都統巴雅爾,副都統錫三、雅沁、布舒庫等分隊奪隘,斬六千餘級,獲所置總兵以下三百餘。大任窮蹙,率眾詣康親王軍降,哈克三還吉安。旋命赴湖南,與將軍穆占會師駐郴州。三桂將馬寶、胡國柱攻永興急,穆占令哈克三率師赴援,與都統宜里布力戰,同歿於陣。喪還,遣內大臣奠茶酒,予白金五百治喪,賜祭葬,謚武毅,進世職一等阿達哈哈番兼拖沙喇哈番,無子,以弟之子巴爾泰襲。

阿爾護,富察氏,滿洲鑲紅旗人,世居輝發。父鄂拜,國初來歸,管牛錄。從入關,擊走李自成。復從征福建有功,予世職一等阿達哈哈番。累遷鑲紅旗蒙古副都統。阿爾護初授王府長史。

康熙十三年,命署副都統,與將軍坤巴圖魯率師出駐汝寧。其冬,吳三桂將王屏籓等自四川窺陜西,提督王輔臣叛應之。與坤巴圖魯赴西安,十四年,將五百人守寶雞。賊來犯,擊之,敗去,追至天王村,撫定歸州十二堡,降其將七、兵千餘。八月,詔分將軍佛尼埒兵六百授阿爾護,守棧道諸險要。與三桂將彭時亨戰仰天池,大捷。出螞蟻河口,望見賊營分立九龍山,即以銳師宵加之,賊大潰。十月,三桂將石存禮等擁眾八千出棧道,據益門鎮口,分七營窺寶雞,兼為王輔臣聲援。阿爾護令軍中曰:「有能攻剋隘口者,賞與克州縣城同。」軍士氣奮,分三道上,直搗其巢,力戰,自巳至未,七營盡破,追奔十數里,射殪其將,獲旗幟、器械無算。迭破賊仰天池山下,及益門鎮東邵家山、黃兒子溝、沈家坡諸處,自是賊不敢出棧道。

十五年,授鑲紅旗蒙古副都統。從將軍穆占移師湖廣。十六年,三桂將吳國貴犯長沙,力戰,死之。事聞,謚敏壯,予三等阿達哈哈番。

路什,納喇氏,滿洲鑲黃旗人,世居章甲城。父克恩,歸太祖。路什以驍勇稱。崇德七年,師入兗州,以雲梯攻城,路什先登,克之,賜號「巴圖魯」,予世職游擊。

順治初,以甲喇額真從入關,與牛錄額真袞泰將步兵擊李自成;復與梅勒額真阿哈尼堪逐自成至慶都,敗之,自成潰而西。二年,從英親王阿濟格徇陜西,與梅勒額真阿喇善攻綏德,圍延安,七戰七克。時自成南走商州,奔湖廣,躡追至武昌,獲其孥。論功,進二等。

張獻忠據蜀,久不下。三年,從肅親王豪格西征,會叛將賀珍等犯漢中,分兵守雞頭關,路什與巴牙喇纛章京鰲拜擊卻之;追破珍於楚湖,入四川,所向皆捷。獻忠既滅,分兵剿餘賊,俘斬甚眾,進三等阿思哈尼哈番。

十五年,從信郡王多尼南征,師入貴州。明桂王將羅大順出拒戰於黔西州十萬溪箐,路什與噶布什賢噶喇依昂邦鄂訥、梅勒額真噶褚哈分兵擊之,連破數營,敵大潰。敘功,進二等。

吳三桂反,路什年已七十,請從征,遂從貝勒尚善徇湖南。康熙十七年秋,以偏師取湘陰,進洞庭湖,守九馬嘴。寇至,風大作,泊綠林灘,舟被擊,路什猶賈勇發矢石,擊殺十數人,力竭,死。時七月二十八日也。事聞,進一等兼拖沙喇哈番。

子布納海,襲。聖祖親征噶爾丹,布納海從內大臣費揚古出西路,戰於昭莫多。師有功,進三等精奇尼哈番。卒,子瑚什屯,降襲二等阿思哈尼哈番。

雅賚,納喇氏,滿洲正藍旗人。初任王府長史,兼佐領。康熙十三年,命署副都統,駐防江寧,未至,徙駐安慶。耿精忠遣其將擾江西,廣信、建昌、饒州並陷。大將軍安親王岳樂率禁旅南征,駐南昌,以雅賚與署領都統阿喀尼參贊軍務,移兵攻彭澤。既,詗知賊據小姑山,先遣兵擊之。賊結水寨拒戰,我軍出其後,陟山而下,斬其裨將,餘眾多被創赴水死。師進攻彭澤,城西臨江,南北皆倚山,路險峻,乃督兵略其東,陟山,樹雲梯以登。賊不能抗,縱火啟東門走,追擊敗之,遂進攻湖口。安親王軍至,賊棄城走都昌,雅賚追及之,敗竄鄱陽湖,所置吏以湖口降。

十四年,將水師逐賊鄱陽湖,趨五桂寨,賊棄寨走,其將黃浩浮舟來犯,擊卻之。追至梅溪、瑞洪、康山湖及壩口,先後得船數百,斬數千級,與陸軍會苜蓿灣,克餘干縣。復進徵建昌,精忠將邵連登據常興山,列營三十,雅賚攻其左,諸軍自右擊之,盡夷其巢,連登中流矢死。復與都統霍特徵廣信,次石峽,方暑,士馬疲渴,猝遇伏,師少卻,雅賚直前奮戰,中砲死,賜祭葬,謚襄壯,予世職拜他喇布勒哈番。

擴爾坤,薩克達氏,滿洲鑲紅旗人,世居那穆都魯。祖葉古德,歸太祖,編牛錄,俾統之。父喜福,任兵部理事官。崇德間,從征黑龍江,順治初,從征漢中,皆有功。復出討姜瓖,瓖將屯寧武關,分據左衛。喜福力戰,被巨創,卒於軍,世職累進二等阿達哈哈番。

擴爾坤初授牛錄額真。從徵貴州,戰屢捷。康熙初襲職,遷護軍參領,擢鑲紅旗蒙古副都統。吳三桂反,命率師駐防太原。十三年,徙駐西安。會四川告警,命進駐漢中。三桂將吳之茂犯廣元,遣兵敗之,復分水陸兼進,又擊之敗去。之茂遣裨將賀騰龍劫糧二郎關,擴爾坤馳擊,獲騰龍。之茂復遣裨將何德成犯廣元,分兵渡河擊卻之,逐北三十餘里。尋以七盤、朝天諸關復陷賊,詔還駐漢中。

十四年,漢中乏餉,將軍錫卜臣領兵還城固,擴爾坤率右翼兵殿後。三桂將彭時亨等擁眾八千據險邀阻,擴爾坤擊潰之,且戰且行三晝夜,次洋縣金水河,七戰皆捷。諸軍前行,仍令擴爾坤殿,俄賊環偪,力戰中創,殞於陣,賜祭葬,進世職三等阿思哈尼哈番。子遜扎齊,襲職,官至工部尚書。

王承業,字瓊山,江南廬江人。少入伍。康熙初,從軍福建,克金門、廈門。累擢游擊,遷廣西副將。十七年,將軍莽依圖徇廣西,以承業為新設援剿中營總兵,管副將事。十八年,吳世琮犯梧州,承業擊敗之。世琮棄營宵遁,克潯州。世琮以十萬人分屯貴州、廣西諸要隘,而自將兵圍南寧。承業赴援,設奇與城兵相犄角,戰新村西山之巔,斬六千餘級,世琮負重傷敗走,南寧圍解。遂自陶鄧山進剿柳州,叛將馬承廕以二萬人拒戰,擊敗之,乘勝定象州,承廕遂降。

其冬,將軍賚塔自南寧直進雲南,檄承業至西隆。吳世璠將何繼祖據石門坎,去安籠所三十里,地僻道險。十九年正月,承業奮勇入,連奪二隘口,復所城。繼祖退據黃草壩,列象拒戰,承業疾擊之,自卯至未,毀其營二十有二。克曲靖,取霑益,下馬龍、楊林,大小三十餘戰,無不披靡。既抵會城,壁城外歸化寺。世璠將胡國柄、劉起龍出拒,承業引守備林廷燏鏖戰,自卯至午,突入賊陣,砲中額,墜馬死。廷燏單騎馳救,賊矢雨集,亦殞於陣。事聞,詔贈承業右都督,廷燏贈都司僉事。

王忠孝,奉天人。以參將銜從軍屢有功,累擢署左翼總兵官。從將軍賚塔下雲南,為前鋒。克西隆,攻廣西縣,忠孝與所部游擊林桂選勇士數十人,越嶺疾馳下,大破賊。攻石門坎,師盛旗幟,鳴鼓角,趨上游,忠孝與桂督兵涉水,出間道繞其後,破敵砦。攻黃草壩,與桂引敵入谷,伏起,夾擊,斬世璠軍裨將。既破隘,師進薄雲南會城。國柄等出戰,忠孝與承業、廷燏同時戰死,贈都督同知。

廷燏,廣東南海人。桂,廣東番禺人。忠孝既戰沒,桂佐賚塔定雲南,代為左翼總兵官。

論曰:吳三桂白首舉事,號善用兵。屯松滋數年,不敢渡江攻荊州。晚欲通贛、粵道,宜理布、哈克三以死拒,然終不得達,安在其為善用兵也?阿爾護輩殺敵致果,授命疆埸。承業戰沒雲南城下,悍敵致死,誠有不易當者。故比而論之,亦以見戡定始末。他死事者,語別見忠義傳,不能遍著也。


\end{pinyinscope}