\article{列傳十六}

\begin{pinyinscope}
明安子昂洪多爾濟恩格類恩格類從子布當布顏代恩格德爾

子額爾克戴青古爾布什鄂齊爾桑布爾喀圖弼喇什色爾格克

阿濟拜恩格圖鄂本兌和濟格爾和濟格爾子拜音達里阿賴

布延阿爾沙瑚阿爾沙瑚兄子果爾沁額琳奇岱青德參濟旺

多爾濟達爾罕奇塔特徹爾貝洛哩弟沙哩岱奇塔特偉徵

奇塔特偉徵弟額爾格勒珠爾喀蘭圖扎克托會楚克圖英琿津

沙爾布

明安,博爾濟吉特氏。其先世元裔,為蒙古科爾沁兀魯特部貝勒。歲癸巳,葉赫貝勒布寨、納林布祿糾九國之師來侵,明安與焉,戰敗,明安乘驏馬獨身跳去,尋修好於太祖。上聞明安女賢,遣使往聘,歲壬子正月,明安送女至,上具車服以迎,與宴成禮。

天命二年正月,明安來朝,上出郊百里迎諸富爾簡岡,設宴慰勞。明安獻駝十、馬牛皆百,上優禮之,日設宴。留一月,明安辭,賜以四十戶,甲幣稱是,送之三十里。七年二月壬午,明安及同部貝勒兀爾宰圖、鎖諾木、綽乙喇札爾、達賴、密賽、拜音代、噶爾馬、昂坤、多爾濟、顧祿、綽爾齊、奇筆他爾、布顏代、伊林齊、特靈,喀爾喀部貝勒石里胡那克,並諸臺吉等三千餘戶,驅其牲畜來歸,授三等總兵官,別立兀魯特蒙古一旗。

天聰三年,與固山額真武訥格、額駙恩格德爾等伐察哈爾,降二千戶。五年,從上伐明,圍大凌河城。明總兵祖大壽出戰,明安與固山額真和碩圖等夾擊,大敗之。我師偽為明兵赴援狀,誘大壽復出戰,明安及兩翼固山齊進奮擊,大壽敗卻,尋率眾降,明安得優賚。六年,從上伐察哈爾。師還,以俘獲少,又違令不以隸戶籍,擅以官牛與所屬,復匿蒙古亡者,吏議當奪世職,上命罰鍰以贖。尋以內附諸蒙古所行多違令,罷蒙古旗,俱散隸諸貝勒所領牛錄,明安改隸滿洲正黃旗。八年,改三等昂邦章京。順治初,三遇恩詔,進二等伯。卒,謚忠順。雍正間,追進一等侯,加封號恭誠。子昂洪、多爾濟、綽爾濟、納穆生格、朗素。

昂洪初從父來歸,授游擊。天命十一年,從伐巴林、扎魯特諸部;天聰五年,從伐明,攻大凌河:俱有功,超進三等副將,賜號達爾漢和碩齊。七年,卒。子鄂齊爾,襲。八年,改三等梅勒章京。順治間,三以恩詔進,再以罪降,定為二等阿思哈尼哈番。洊擢內大臣,管鑾儀衛事。尋授領侍衛內大臣。十四年,卒,謚勤恪。乾隆初,定封三等男。

多爾濟亦從父來歸,授備御,尚主為額駙。天命十一年,從伐扎魯特,有盜馬遁者,多爾濟逐得之。尋又從伐棟奎、克什克騰諸部,又從伐朝鮮,皆有功。天聰五年,始設六部,以多爾濟為刑部承政,專理蒙古事。六年,以直上前失儀,又奉命選獵戶不當,吏議奪世職,上宥之,罰白金百。八年,從伐明,攻大同,上命多爾濟領中軍,圖魯什、烏拜分率左右軍,與明總兵曹文詔戰,大破之,逐至城下,獲馬百。崇德二年,授內大臣,預議政。四年,從鄭親王濟爾哈朗略錦州。六年,上伐明,駐軍松山、杏山間,命多爾濟與內大臣錫翰設伏高橋。明杏山兵千人,以糧不繼潛遁,伏發,敗之,逐至塔山,俘斬甚眾。尋以圍松山時,明總兵曹變蛟夜犯御營,多爾濟不能御,議罪,系三日,罰白金五百,仍敘高橋功,進一等梅勒章京。順治二年,以多爾濟夙荷太宗恩厚,進三等昂邦章京。四年,改三等精奇尼哈番。五年,卒。

弟綽爾濟,襲。坐事,削爵。弟納穆生格,襲。從征福建,沒於海,謚直勇。納穆生格既卒,復以綽爾濟襲。乾隆初,定三等子,多爾濟三世從孫博清額襲。三十四年,改襲一等恭誠侯,為其四世祖明安後。

朗素,明安幼子,襲明安世職。傳至孫馬蘭泰,雍正七年,上以明安舊勞,進一等恭誠侯,命署前鋒統領。九年,討準噶爾,授參贊大臣,疏言寇犯西爾哈昭,擊之敗退,擢領侍衛內大臣。召還,命在辦理軍機處行走。俄,察知在軍心匡怯,妄奏功,謫軍前自效,逮京論斬,系獄。乾隆初,復授副都統。又以扈從行圍後至,稱疾不治事,發拉林披甲。

初,明安所與同部諸貝勒入朝請內附,皆授世職有差,鎖諾木子穆赫林自有傳。又有恩格類、布當叔侄與明安同時來歸,布顏代歸稍後,皆從征伐有戰績。

恩格類、布當,博爾濟吉特氏。來歸,恩格類授游擊,布當授二等參將。天聰三年,太宗自將伐明,布當從攻遵化,與甲喇額真英俄爾岱合軍力戰,破明總兵趙率教,以功進三等梅勒章京。六年,散蒙古旗入滿洲,恩格類、布當皆隸正藍旗。崇德三年,授布當刑部右參政。四年,卒。布當弟色棱,襲恩格類世職。事太宗,伐明,克遵化,圍錦州。事世祖,從入關破賊,擊騰吉斯,並有功。進一等阿達哈哈番兼拖沙喇哈番。十二年,卒。

布顏代,博爾濟吉特氏。初為蒙古烏魯特貝子。天命七年,籍所轄戶口自西拉塔喇來歸,尚主為額駙,予二等參將世職,隸滿洲鑲紅旗。十一年,太祖自將伐明,攻寧遠,不克,偏師取覺華島,布顏代率蒙古兵從固山額真武訥格破敵壘,殲其眾,焚所積芻糧而還。

天聰元年,從伐朝鮮,師有功,分賜降戶及所獲馬。三年,從伐明,入龍井關,克大安口,下遵化,薄明都,四遇敵,戰皆勝。復擊明兵盧溝橋,以七人先入敵陣,遂破之。四年春,師還,駐遵化,明兵擊喀喇沁兵壘,布顏代趨援卻敵。尋與武訥格略地行山岡,遇敵,斬級四十餘。五年,授禮部承政,兼右翼蒙古梅勒額真。從圍大凌河,明兵出戰,布顏代傷於矛,仍力戰卻敵,斬一人。六年,從略宣府、大同邊外,收察哈爾部眾。師還,以匿俘獲,吏議削世職、罰鍰、奪俘獲及賜物,上命毋削世職。八年,上自將伐明,攻大同,布顏代與侍衛星訥等率蒙古巴牙喇兵八十人,經哈麻爾嶺,收察哈爾部眾。進次西拉木輪,降百餘戶;又進,遇察哈爾部俄爾塞圖等以所屬來降。還,與大軍會。以功,進三等梅勒章京。九年,蒙古旗制定,以布顏代為鑲紅旗固山額真。

崇德元年,從武英郡王阿濟格伐明,克昌平。師還出塞,明兵襲我後,布顏代為所敗,坐罷固山額真世職,降一等甲喇章京,罰鍰,奪俘獲。順治元年,以巴牙喇甲喇額真從入關,與梅勒額真和託等逐李自成至慶都。尋從豫親王多鐸定陜西。二年,加半個前程。復從下江南,渡黃河,與明兵戰,身被數傷,所乘馬亦創,猶力戰沖鋒殪敵,遂以創卒,年六十有一。子鄂穆布,襲職。

恩格德爾,博爾濟吉特氏。其先世元裔,為蒙古喀爾喀巴約特部長。當太祖初起兵時,喀爾喀裂為五部,巴約特其一也,恩格德爾父達爾漢巴圖魯,為其部貝勒,牧地曰西喇木倫。太祖起兵之十二年,歲甲午正月,喀爾喀部貝勒老薩、北科爾沁部貝勒明安始遣使來聘。又十一年,歲乙巳,恩格德爾來謁,獻馬二十,上優賚而遣之。明年,歲丙午冬十二月,恩格德爾率五部諸貝勒之使謁太祖,獻駝馬,奉表上尊號曰神武皇帝。自此蒙古諸部朝貢歲至。

天命元年,太祖初建國即皇帝位,距恩格德爾等初上尊號時十年矣。二年,恩格德爾來朝,上以貝勒舒爾哈齊女妻焉,號為「額駙」。三年夏四月,太祖始用兵於明,師次挖閧萼謨之野,恩格德爾與薩哈爾察國長薩哈連二額駙侍上,上與言金往事,因諭之曰:「朕觀古帝王轉戰勞苦,始致天位,亦未有能永享者。今朕此役,非欲覬天位而永享之也。但以明構怨於我,不得已而用兵耳。」

九年春正月,恩格德爾偕其妻郡主來朝,請率所部來歸,上嘉其誠,與之盟,賜以敕:「非叛逆,他罪皆得免。」命貝勒代善等帥師移所部至遼陽。既至,上郊勞,設宴章義站,賜恩格德爾及其弟莽果爾代雕鞍良馬一、貂裘一,恩格德爾子囊孥克、門都、答哈,莽果爾代子滿硃習禮猞貍猻裘一。既入城,賜田宅、金銀、貂、猞貍猻、段疋、器用及耕作之具,復分平定堡民屬焉。尋授恩格德爾、莽果爾代三等總兵官。旗制定,隸滿洲正黃旗。

天聰三年,與武訥格等帥師伐察哈爾,降二千戶。語詳武訥格傳。是年冬,從上伐明,入龍井關,克遵化,薄明都,上駐軍德勝門外。明督師袁崇煥率總兵祖大壽軍二萬人,自寧遠赴援,屯城東南。上令諸軍進戰,時恩格德爾與武訥格共將蒙古兵。恩格德爾率左翼,未成列,縱騎驟進,為所敗,卻走;武訥格以右翼突擊,乃敗敵。吏議恩格德爾當奪世職,上命貸之。四年春,克永平。恩格德爾行略地,遇明將將步卒三百,將戰;復有騎兵三千自玉田城突出,恩格德爾陽退誘敵,敵稍前,疑有伏,還走;因追躡其後,獲馬百。

五年,從圍大凌河城。明監軍道張春、總兵吳襄等軍四萬自錦州赴援,上親督諸軍擊破之。初戰,敵甚銳,蒙古兵右翼猛進,先入張春壘;左翼兵避矢石,進稍緩。吏議恩格德爾當奪世職,上復命貸之,罰鞍馬一、白金百。崇德元年五月,卒。順治十二年,追謚端順,立碑紀功。

子額爾克戴青,初任侍衛,授三等甲喇章京。恩格德爾既卒,以額爾克戴青襲父爵,而以所授世職予其弟索爾噶。順治二年,進二等昂邦章京。七年三月,遇恩詔,進三等侯。大學士剛林、祁充格等諷使附睿親王多爾袞,當改入正白旗,額爾克戴青不從,旋構吏議,降二等精奇尼哈番。世祖親政,嘉其持正無所阿,復進一等侯,列議政大臣,管鑾儀衛,擢領侍衛內大臣。再遇恩詔,又以索爾噶卒,仍兼三等甲喇章京,三進至一等公。十年,坐讞獄有所徇,降二等公。十四年,加少保,兼太子太保。十六年,額爾克戴青僕毆侍衛於市,先發誣侍衛;讞實,額爾克戴青坐徇縱,削爵奪官,留內大臣銜。十八年六月,卒,謚勤良。

恩格德爾初封,是時從例改三等昂邦章京,其長子囊弩克當襲。囊弩克先以從軍授二等甲喇章京,合為二等伯。康熙間,復為二等公,降襲一等侯。世宗時,特命襲三等公,加封號順義,旋改奉義。乾隆九年,定封一等奉義侯。

莽果爾代初與恩格德爾同授三等總兵官,改三等昂邦章京。順治初,從入關,破流賊。三遇恩詔,進一等伯。雍正間,降襲二等精奇尼哈番。乾隆初,定封一等子。

古爾布什,亦元裔,為喀爾喀臺吉,與恩格德爾同牧西喇木倫。天命六年十一月,偕臺吉莽果爾,率所屬六百戶,驅牲畜來歸。太祖御殿,入謁與宴,各賜裘:貂三,猞猁猻、虎、貉皆二,狐一;緣貂朝衣五,緣獺裘二,緣青鼠裘三,蟒衣九,蟒緞六,緞三十五,布五百,黃金十兩,白金五百兩,雕鞍一,鯊鞍七,玲瓏撒袋一,撒袋實弓矢八,甲胄十,僮僕、牛馬、田宅、雜具畢備。上以女妻古爾布什為額駙,賜名青卓禮克圖,畀滿洲、蒙古牛錄各一,授一等總兵世職,隸滿洲鑲黃旗。

天聰五年,太宗自將伐明,圍大凌河城。蒙古左翼兵戰不力,古爾布什當奪世職,上特貰之,罰鞍馬一、白金百兩。尋擢兵部承政。崇德三年,更定官制,改兵部右參政。六年,從伐明,圍錦州,敗敵於寧遠。七年,再圍錦州,敵兵出戰,古爾布什擊走之。

古爾布什屢坐事論罰,至是以元妃喪,輔國公扎喀納軍中歌舞,吏議古爾布什不呵禁,不舉劾,當奪世職、籍沒,上復特貰之。順治初,從入關,破流賊。復遇恩詔,累進一等精奇尼哈番。十八年正月,卒,謚敏襄。康熙間,降襲二等精奇尼哈番。乾隆八年,定封二等子。

莽果爾與古爾布什偕來,同被賞賚。太祖以族弟濟白裏女妻焉,亦授總兵。

鄂齊爾桑,博爾濟吉特氏,蒙古扎魯特部人。父巴克,為其部貝勒。天命四年,太祖既擊敗楊鎬,取開原;七月,復克鐵嶺,即夕,巴克與喀爾喀部貝勒介賽等將萬餘人赴援,翌旦遂戰,諸部師大敗,獲介賽等及巴克以歸。七年正月,鄂齊爾桑入質,請釋巴克,上許之。八年,巴克朝正旦,上悅,遣鄂齊爾桑與俱還。

太宗即位,以扎魯特部敗盟,貳於明,命貝勒代善、阿敏等將萬人討之,斬倡叛者貝勒吳爾寨圖,獲巴克及其二子,諸貝勒喇什希布、代青、桑噶爾寨等十四人以歸。上命隸滿洲鑲黃旗,賜衣服器用。尋授鄂齊爾桑牛錄額真。

天聰三年,從伐明,明步兵自薊州至,與揚古利共擊破之。五年,圍大凌河城,敗錦州援兵。八年,授世職三等甲喇章京。八月,復從伐明,攻大同。上命噶布什賢噶喇依昂邦圖魯什將左軍,甲喇額真吳拜將右軍,而以額駙多爾濟與鄂齊爾桑並將中軍,與明總兵曹文詔戰,大破之,追至城下,獲馬百。崇德二年,擢內大臣。六年,從上伐明,攻松山,明總兵曹變蛟夜犯御營,諸將未御戰者皆坐譴,上以鄂齊爾桑自蒙古來歸,特免之。順治二年正月,以其子喇瑪思尚主,授固倫額駙。二月,進鄂齊爾桑三等梅勒章京。五年,卒,以其子楚勒襲,恩詔進二等。乾隆初,定封二等男。

太宗時,諸博爾濟吉特之裔來歸,為將有戰功受封爵者,又有布爾喀圖、弼喇什、色爾格克。

布爾喀圖,初為喀喇沁部臺吉。天聰三年六月,使入貢,九月,來朝。十月,太宗自將伐明,以布爾喀圖嘗如明朝貢,習知關隘,使為導。師入邊,克龍井關,撫定羅文峪,分兵命布爾喀圖戍焉。四年正月,明將丁啟明等以三千人來攻,布爾喀圖與戰,明兵敗,入堡。翌日進兵,克其堡,獲啟明及裨將三,俘馘甚眾,賜號岱達爾漢。五年正月,以貝勒阿巴泰第四女妻焉。三月,從上伐察哈爾。察哈爾部眾有降而復叛者,劫軍中土默特部人畜,布爾喀圖追擊,斬逋者,足被創,尋挈所部來歸。蒙古旗制定,隸正藍旗。崇德元年六月,授一等昂邦章京。順治元年,卒。子班珠勒,襲。恩詔累進一等伯。乾隆初,定封一等子。

弼喇什,亦喀喇沁部臺吉。天聰二年二月,從其父貝勒布延謁太宗,請歸附。八月,上自將伐察哈爾,徵蒙古諸部兵,次綽洛郭爾。弼喇什從其汗拉斯喀布謁行在,獻財幣駝馬,上悉卻之,賜宴,與以甲胄,遂從上擊察哈爾,戰有功。旋又從貝勒岳託伐棟奎部,與甲喇額真薩木什喀、牛錄額真布顏、巴牙喇壯達博爾輝等同力戰破敵,斬百餘人。尋率所屬人戶來歸。蒙古旗制定,隸鑲紅旗。上妻以宗女,命貝勒代善贍焉。弼喇什自陳貧乏,上賜以金。崇德元年,授世職三等昂邦章京。三年,與明通市張家口,命弼喇什蒞焉。六年,復往蒞。時諸王大臣各遣其屬從,有盜禮親王代善金者,弼喇什坐失囚,論罰。順治三年,從豫親王多鐸逐騰機思,道卒。子多爾濟,襲。改三等精奇尼哈番,恩詔進二等。乾隆初,定封三等子。

色爾格克,先世居喀喇徹哩克部。父阿拜岱巴圖魯,天聰間率眾來歸,授世職三等甲喇章京,隸正白旗。卒,以色爾格克襲,授一等侍衛。崇德元年,從伐朝鮮。朝鮮國王李倧保南漢山城,師從之,色爾格克登山,身被創,賚馬三。五年,從鄭親王濟爾哈朗等伐明,圍錦州,色爾格克率侍衛二十人前搏戰。有僧格依者,自蒙古降明,為將,善戰,色爾格克擊斬之。鄭親王使啟心郎額爾赫圖還,上其功。六年,復圍錦州,令色爾格克選巴牙喇兵四十為伏以待敵,得明將一,奪甲與械,即以賜之。上自將禦洪承疇,命諸將設伏高橋,色爾格克斬明兵七,復賚馬二。又先眾破敵騎。師圍松山,為壕環其城,城兵出擊烏真超哈分守地,色爾格克以巴牙喇兵三十人赴援,城兵引退。七年冬,從貝勒阿巴泰伐明,越明都攻臨城;略山東,攻青州,皆力戰,被創。

世祖即位,錄阿拜岱巴圖魯舊勛及色爾格克戰功,復遇恩詔,授二等阿思哈尼哈番,擢內大臣。康熙十二年,聖祖加恩諸舊臣,色爾格克加太子少保。二十年,卒,謚勤敏。乾隆初,定封二等男。

阿濟拜,卓特氏,先世為蒙古巴林部人。旗制定,隸正藍旗。初事太祖,授牛錄額真。天命三年,太祖克撫順。師還,明總兵張承廕自廣寧襲師後,阿濟拜從貝勒阿巴泰還擊,破之。四年,破明總兵杜松於界凡。七年,敗明兵於沙嶺。阿濟拜皆在行間。

天聰三年,太宗伐明,阿濟拜以甲喇額真從,略通州,斬邏卒五,獲馬四;薄明都,與甲喇額真鄂羅塞臣等當袁崇煥,戰勝。九年,上命巴牙喇纛額真布哈將八十人略明邊,至寧遠,俘九人,獲馬四、牛百餘。還,出邊六十里,明兵八千追至,布哈殿,戰沒,阿濟拜與巴牙喇甲喇章京托克雅、哈談巴圖魯等還擊敗敵,護所俘獲以還。上命賚以牛馬,予牛錄章京世職。

順治初,從入關,擊李自成。阿濟拜署梅勒額真,為後隊。尋與固山額真伊拜逐寇山西,至澤州,數破賊壘,擢正藍旗蒙古梅勒額真。二年,加半個前程。三年,從肅親王豪格討張獻忠,道漢中,與固山額真巴哈納擊走叛將賀珍;徇秦州,與尚書星訥擊敗獻忠將高汝礪,獲馬騾百餘;進擊獻忠於西充,與巴牙喇纛額真阿爾津、蘇拜連戰皆捷。敘功,遇恩詔,累進一等阿達哈哈番兼拖沙喇哈番。九年八月,以老乞休,命解梅勒額真任。尋卒,謚忠勤。

恩格圖,失其氏,蒙古科爾沁部人。自哈達挈家來歸,授牛錄額真。與甲喇額真阿岱出駐伊蘭布裏庫,防蒙古游牧軼界,率十人巡徼,遇敵百人,追斬殆盡。聞明兵千餘將攻海州,率三百人馳擊,敗之。天聰間,屢從太宗伐明,薄明都,擊滿桂軍;攻遵化,破敵壘,入大安口:皆先眾奮擊。以功,予世職二等甲喇章京,擢兵部承政。蒙古旗制定,恩格圖隸正紅旗,即授本旗固山額真。

崇德元年,從伐明,與阿岱等為伏,殲明邏卒。復從伐朝鮮,薄其都城,與固山額真譚泰等樹雲梯以登。尋坐伐明時,戰松山,正藍、正白、鑲白三旗營汛錯亂,匿不劾;師還出塞,遇敵戰敗:罰鍰,奪俘獲。又坐伐朝鮮時,方食,上召不即赴,廝卒妄出,為朝鮮兵所殺,論罪,上命罰鍰以贖。三年,從貝勒岳託伐明,攻密雲,距墻子嶺五里,恩格圖率兵先諸軍越高峰,入邊破敵。五年,從鄭親王濟爾哈朗等伐明,圍松山,明兵夜出劫營,恩格圖率本旗兵擊敗之。六年,從上伐明,上命恩格圖與噶布什賢噶喇依昂邦吳拜擊明總督洪承疇,恩格圖違上方略,遇敵不前。師還,吏議當褫職,命罰鍰以贖。尋令與諸將更番戍松山。

順治元年,從入關,擊李自成,進一等甲喇章京,加半個前程。從豫親王多鐸西破賊,移師向江南,賊躡我師後,恩格圖殿,四戰皆勝。尋破明將鄭鴻逵於瓜洲,復自江南徇浙江,至杭州,破敵,獲舟三十五。克嘉興,下昆山。進三等梅勒章京。復自浙江徇福建,與固山額真漢岱共下府一、縣五;與梅勒額真鄂羅塞臣共下府一、縣八;戰於分水關、於南靖:皆有功。四年,進一等阿思哈尼哈番。五年,討江西叛將金聲桓,卒於軍。乾隆初,定封二等男。

鄂本兌,曼靖氏,其先為蒙古。入明為守備。天命六年,太祖取遼陽,鄂本兌以兵三十五、馬六十出降。其後蒙古旗制定,隸正黃旗。七年,從伐廣寧有功,授世職游擊。天聰元年,太宗伐明,屯錦州,命額駙蘇納選蒙古將士御敵塔山西,鄂本兌與焉,敵以二千人至,奮擊敗之。上移師寧遠,明總兵滿桂陣於城東,鄂本兌率五牛錄甲士破敵,進二等參將。二年,從上伐多羅特部,以二百人先驅,遇敵,敵稍北,復出精銳死戰,我師且卻,鄂本兌躍騎突前,敵敗遁,上督諸貝勒並進,殺其臺吉古魯,俘獲無算。進一等參將,擢右翼蒙古固山額真。

三年,從上伐明,明邊將五道迎戰,鄂本兌率所部兵擊敵,斬參將一,獲其纛,入大安口,遂進薄明都,克永平、灤州、遵化、遷安四城。上命鄂本兌與固山額真察哈喇等守遵化,貝勒阿敏駐永平,護諸將。明兵來攻,阿敏檄棄城引師退,敵已逼城下,鄂本兌以五十人出戰,斬邏卒七人,獲其馬,遂與察哈喇等全軍以還。鄂本兌為殿,明師追至,屢擊卻之,引出邊,師無所損,進三等副將。五年,從上伐明,圍大凌河,屯城西。敵出戰,爭已下諸臺堡,鄂本兌與固山額真和碩圖督兵並進,敵敗退入城,迫逐之及壕,敵死者甚眾。師還,得優賚。八年,改三等甲喇章京。九年正月,卒。康熙間,兄孫託克塔哈爾襲世職。從撫遠大將軍費揚古討噶爾丹有功,進三等精奇尼哈番。乾隆初,定封二等男。

和濟格爾,失其氏,蒙古烏魯特部人。入明為千總。太祖取廣寧,從石廷柱出降,授甲喇額真,隸烏真超哈。其後漢軍旗制定,隸正白旗,並從漢姓為何氏。和濟格爾事太祖,從伐巴林、棟奎諸部,有功。天聰三年,從伐明,詗敵薊州,斬邏卒三,敵三百來攻,和濟格爾沖鋒入,斬百總一。五年,復從伐明,圍大凌河,敗錦州援兵;城兵出樵採,爭臺堡,並擊敗之。與敵戰城下,我師執纛者墜壕,和濟格爾掖之出,復以鳥槍殪敵兵三。八年,授世職牛錄章京。崇德三年,復從貝勒岳託伐明。四年,烏真超哈析置四固山、八梅勒,以和濟格爾為鑲白旗梅勒額真。五年,從圍錦州,累敗敵。六年,復從圍錦州。敵自松山分踞高橋南三臺,和濟格爾以火器克之,殲敵百餘。七年,從克塔山、杏山二城,加半個前程,授正白旗梅勒額真。八年,從克中後所、前屯衛二城,進一等甲喇章京。順治三年二月,卒。

拜音達里,和濟格爾子,襲二等阿達哈哈番。事聖祖,自參領擢宣化總兵官。十三年,耿精忠反,移拜音達里為隨征福建總兵官,尚可喜請增兵戍廣東,上命與福建巡撫楊熙駐廣州。十五年,可喜子之信叛,拜音達里與熙督所部斬關突圍出,會大軍於贛州。上獎其忠勇,進一等。十九年,授駐防廣州副都統。二十七年,遷廣州將軍。三十七年,卒,以其子何天培襲。

天培時已官參領,累遷江南京口將軍。雍正初,命署江蘇巡撫。入為兵部尚書,出為江寧將軍;復入為正白旗漢軍都統,署兵部尚書。六年五月,上以天培阿附年羹堯、隆科多下刑部逮治,擬斬監候。乾隆元年,赦出獄。尋卒。天培既得罪,以拜音達裏曾孫何鈞降襲二等阿達哈哈番。乾隆間,更名立柱。官至貴州提督。

阿賴,莽努特氏,世居喀爾喀部。太宗時,挈其孥來歸,隸蒙古正黃旗。嘗奉使阿祿部,降其部長,上嘉其能,賜號「達爾漢」,免賦役。率兵五百逐逃人,窮追數月,斬倡叛者四人,盡俘以還。又率兵攻喀木尼喀部,俘其部長葉雷,獲戶口牲畜無算。崇德九年,授一等甲喇章京,又半個前程,加賜號庫魯克達爾漢。尋授禮部左參政、正黃旗蒙古固山額真。從攻錦州,設伏杏山邀擊,攻松山,敗敵。順治初,從固山額真葉臣徇山西,師還,賜白金三百;三年,從擊騰機思;六年,討姜瓖:皆有功,進二等阿思哈尼哈番。康熙十二年,加太子太保。十七年,卒,謚武壯。

布延,郭爾羅特氏,蒙古察哈爾部人。初在其部為塔布囊。天聰元年,偕昂坤杜棱來歸,隸滿洲正黃旗。從伐棟奎部,為導。從伐克什克騰部,首陷陣。再從甲喇額真圖魯什略明邊,俘其邏卒,斬百餘級,得樵車百餘、騾驢以百數。復略十三站,斬十級,得把總一、馬三。敘功,授世職牛錄章京。九年,偕布哈塔布囊略寧遠。既出邊,明兵千餘追至,布哈陷陣。哈談巴圖魯還戰,馬中矢僕,布延赴援,與之馬,力戰敗敵,賚俘一、馬二、牛三,進世職三等甲喇章京。

崇德元年二月,命齎書投明邊諸守將,歷松棚路、潘家口、董家口、喜峰口致責言焉。五月,從伐明,薄明都,敗明兵盧溝橋。三年二月,從伐喀爾喀部。七月,擢議政大臣,兼巴牙喇纛章京。九月,從伐明,自墻子嶺入,敗明兵,追擊,得馬八十七。四年,帥師戍烏欣河口。偕侍衛阿爾薩蘭攻松山,布延為伏,斬二十一級。詗敵錦州,斬邏卒八,得馬十二。五年,從睿親王多爾袞圍錦州,擊敗明步軍。六年,從鄭親王濟爾哈朗克錦州外城,與內大臣伊爾登戰最力,賚百金,進世職二等甲喇章京。八月,上自將擊洪承疇,其將曹變蛟夜襲上營,布延以內大臣不嚴守御,論罰。七年二月,師擊承疇,布延兵後至,當死,命論罰以贖。十一月,伐明,圍薊州。

順治二年,世祖以布延舊臣,進世職一等甲喇章京。其從子烏納海,先以戰死,恤贈世職牛錄章京,命布延並襲,進三等阿思哈尼哈番。八年,卒,次子茂奇塔特襲世職。

茂奇塔特,康熙三十五年,從征噶爾丹有功,加拖沙喇哈番,例進二等。乾隆初,定封二等男。

阿爾沙瑚,瓦三氏。初為察哈爾林丹汗護衛。林丹汗敗走唐古特,阿爾沙瑚帥所屬四十餘戶渡哈屯河來歸,隸蒙古鑲白旗,授世職游擊。崇德三年,從伐明,自墻子嶺入,屢敗明兵,行略地至濟南。四年三月,師還出塞,復擊敗太平寨明兵。五年,從伐索倫部,獲部長博穆博果爾及其孥。六年,從伐明,圍錦州。明以騎兵出松山,謀劫紅衣砲,阿爾沙瑚力戰卻之,又敗洪承疇所將步兵,以功進世職一等甲喇章京。八年,卒,以兄子果爾沁襲。

果爾沁時已為牛錄額真。從伐朝鮮,嘗以侍衛二十人敗敵。順治初,從入關,擊李自成,加半個前程。再遷鑲白旗蒙古梅勒章京,進世職二等阿思哈尼哈番。十七年,遷本旗固山額真。上命定西將軍愛星阿帥師與吳三桂合兵逐明桂王由榔,以果爾沁為副,敕愛星阿軍中機事皆俾果爾沁與議。十八年九月,師次大理,休馬力。逾月,出騰越,道南甸、隴川、猛卯。十一月,薄木邦,明將白文選方據錫箔江為守。果爾沁與固山額真遜塔,巴牙喇纛章京畢力克圖、費雅思哈,噶布什賢昂邦白爾赫圖等,簡精銳疾馳三百餘里,至江濱。文選毀橋走茶山,令總兵官馬寧以師從之,至猛養,文選降。師進次晚舊,得由榔以歸。康熙三年,進一等阿思哈尼哈番兼拖沙喇哈番。尋列議政大臣,調本旗滿洲都統。九年二月,卒,謚襄敏。乾隆初,定封二等男。

額琳奇岱青,博爾濟吉特氏。居翁牛特部,為察哈爾部宰桑。林丹汗敗走,所部皆潰,額琳奇岱青將來歸;會宰桑多爾濟塔蘇爾海率所屬游牧,與我師遇,倚山拒戰,敗遁,額琳奇岱青追及之,與謀偕降。天聰八年六月,上自將伐明,道塞外,師次波碩兌。額琳奇岱青、多爾濟塔蘇爾海及顧實、布顏代、塞冷等五宰桑率丁壯七百人及其孥二千口來歸,上遣將護詣沈陽,厚賚之,分隸蒙古正白旗。崇德元年,授世職二等昂邦章京。三年,從伐明,自青山口入,越明都,略地山東,累戰皆勝。六年,圍錦州,與阿爾沙瑚同功,進一等,世襲罔替。八年,卒。順治間,追謚勤良。子札木素,襲。聖祖即位,加恩諸大臣舊自察哈爾來者,札木素與內大臣噶爾瑪、散秩大臣沙哩岱等,並賜莊田、奴僕。康熙三年,授內大臣。六年,卒。乾隆初,定封一等子。

德參濟旺,博爾濟吉特氏,世居阿布罕。初為察哈爾部宰桑。林丹汗敗走,以所屬從。天聰八年,上自將伐明,略宣府,攻萬全左衛,遂出尚方堡二十里駐軍。時林丹汗走死大草灘,德參濟旺與噶爾瑪濟農、多尼庫魯克、多爾濟達爾漢諾顏號四大宰桑,挾林丹汗二福金,率丁壯二千人及其孥來歸,遣三十人先奏上。上進次克蚌,命所司運米三百石以待。二福金及德參濟旺等至,謁上行在,上與之宴,賜貂裘、鞍馬、牛羊。還師,復宴新附諸臣,德參濟旺等跪進酒,上曰:「朕本不飲酒,念爾曹誠意,當盡此一卮。」復酌酒遍賜之,並賚甲胄、衣裘,授世職一等昂邦章京,隸蒙古正黃旗。九年六月,察哈爾臺吉瑣諾木來降,上召宴,德參濟旺與焉。上因言:「察哈爾傾覆,爾諸臣來歸,朕皆預知。」德參濟旺奏曰:「聖諭及此,洵有如神之鑒也!」順治二年,坐事,降三等。三年,從豫親王多鐸北討騰機思,鄂特克山之役,及破土謝圖汗,德參濟旺皆與有功焉。語詳奇塔特徹爾貝傳。復進一等。是歲,改一等精奇尼哈番。五年八月,卒。乾隆初,定封一等子。

多爾濟達爾罕,博爾濟吉特氏,居翁牛特,為察哈爾部宰桑。與德參濟旺等同降,隸蒙古鑲黃旗。崇德元年,授世職一等梅勒章京,以為都察院承政。三年,更定官制,改參政。六年,上自將擊洪承疇,命多爾濟達爾罕偕承政阿什達爾罕度善地駐軍,並察諸軍斬級多寡,還報稱旨,擢內大臣,仍兼參政。七年七月,祖大壽來降,上幸牧馬所,命諸內大臣與較射,賞中的者,多爾濟達爾罕得駝一。十月,從饒餘貝勒阿巴泰伐明,行略地,自薊州至於兗州。師還,上言:「師自兗州還,右翼諸固山不遵貝勒期約,先左翼諸軍出塞。賴上威靈懾敵,我軍縱橫如行無人地,得全師以還。萬一有失,悔何及?」請論罰,上為停右翼諸軍賞。順治間,上推太宗舊恩,並考滿,進三等精奇尼哈番,復授都察院承政。七年,命以內大臣與議政,恩詔一等,兼拖沙喇哈番。十七年四月,卒,謚順僖。乾隆初,定封三等子。

奇塔特徹爾貝,哈爾圖特氏。初為察哈爾部宰桑。林丹汗敗,奇塔特徹爾貝以四百戶保哈屯河。天聰八年十一月,上使招焉,渡河次西拉木輪,旋從使者來歸,上厚賚之,隸蒙古正藍旗。林丹汗有八大福金,掌高爾土門固山事福金,其一也。林丹汗殂,所部宰桑袞出克僧格妻焉。上以袞出克僧格叛主,奪福金畀奇塔特徹爾貝。

崇德元年,授世職三等昂邦章京。三年九月,從伐明,自青山口入,越明都,略山東。明年,師還,以所部牛錄額真珠額文經三屯營,率兵役掠敵糧,戰死。奇塔特徹爾貝未及援,罰納馬。九月,從伐明,薄寧遠,以火攻擊卻明兵。六年,圍錦州,破洪承疇。既,復與阿爾沙瑚共擊敗明兵來劫砲者及承疇所將步兵,進世職二等昂邦章京。

順治初,從入關,逐李自成,至慶都。三年,從豫親王多鐸北討騰機思,師次英噶爾察克山,聞騰機思在滾葛魯臺,疾馳逐之,至鄂特克山,獲其孥。土謝圖汗以六萬人次扎濟布拉,為騰機思聲援,奇塔特徹爾貝等率所部擊之,敗走,逐北三十餘里。詰旦,碩類汗復以二萬人至,復擊之,亦敗走。以功進一等。康熙三年,卒。子鄂諾勒,襲。十八年,鄂諾勒以參領從護軍統領莽吉圖南討鄭錦,卒於軍。

洛哩,鄂爾沁氏。初為察哈爾林丹汗護衛。天聰六年,太宗自將伐察哈爾,林丹汗走死,洛哩持元初巴斯巴喇嘛所鑄嘛哈噶拉金佛,率百餘人來歸。隸蒙古正黃旗,授世職一等參將。崇德三年,從貝勒岳託伐明,自墻子嶺毀邊墻入,擊敗明總督吳阿衡。六年,從伐明,圍錦州,城兵出戰,左翼三旗巴牙喇兵擊之不利,退入壕,明師環之,逼洛哩分守地。洛哩力戰,沒於陣,恤贈三等梅勒章京。

洛哩兄沙濟,弟烏班和碩齊、沙哩岱。沙濟襲洛哩遺爵。烏班和碩齊當林丹汗走死,別率七十人來歸,授游擊。卒,以其弟沙哩岱襲。順治初,沙哩岱以牛錄額真從睿親王多爾袞入關擊李自成,復從豫親王多鐸討騰機思,擊敗土謝圖汗、碩類汗,進二等阿達哈哈番。尋沙濟亦卒,沙哩岱兼襲,合為二等精奇尼哈番,授散秩大臣。順治十八年,聖祖即位,加恩諸大臣舊自察哈爾來歸者,沙哩岱及內大臣噶爾瑪、散秩大臣札木素等,並賜莊田、奴僕。康熙元年,卒。乾隆初,定封二等子。

太宗時諸將自蒙古來歸以戰死者,又有奇塔特偉徵、巴賴都爾莽奈、巴賴都爾莽奈子阿南達、孫阿喇納,皆有聲績,自有傳。其事世祖戰死,則有袞楚克圖英、琿津、沙爾布。

奇塔特偉徵,博爾濟吉特氏,鄂爾多斯哈爾濟農族人也。世居克魯倫。太宗時,與其弟額爾格勒珠爾、喀蘭圖、扎克托會率所屬來歸,隸蒙古正黃旗。天聰八年正月,上遣蒙古軍略錫爾哈、錫伯圖,收察哈爾流散部眾,奇塔特偉徵與岱青塔布囊斬七十三人,降百餘人,獲馬駝數十。五月,上自將伐明,次古爾班圖勒噶,命蒙古軍別出間道,與大軍會錫喇烏蘇河,奇塔特偉徵行遇察哈爾五人將遁入阿祿部,擒以獻。九年五月,從貝勒多鐸伐明,次寧遠。奇塔特偉徵時為噶布什賢噶喇昂邦,率所部前驅,至大凌河西,明將劉應選、趙國志將七千人迎戰,我兵寡,奇塔特偉徵力戰,沒於陣,恤贈三等梅勒章京。

額爾格勒珠爾,崇德間,屢從伐明,徇山東,圍錦州,戰松山,皆有功。順治間,從入關擊李自成,予世職牛錄章京,加半個前程。卒,無子,以喀蘭圖子察琿襲。

喀蘭圖,崇德間為一等侍衛。順治初,世祖推太宗舊恩,復屢遇恩詔,世職累進二等阿達哈哈番。睿親王多爾袞攝政,請上幸其第,喀蘭圖方退直,聞上扈從無多人,即持弓矢趨詣左右防衛。及世祖親政,敕獎喀蘭圖忠篤,賜金帛、鞍馬、莊田,命以其族改隸滿洲正黃旗,進世職一等。尋以上行圍扈從愆遲,復為二等。事聖祖,累官理籓院尚書。乞老,授內大臣,加太子太保。卒,謚敏壯。子察琿兼襲,合為三等阿思哈尼哈番。康熙十三年,安親王岳樂討吳三桂,次袁州,與吳三桂將馬寶戰鈐岡山,死之。進二等。

扎克托會,事太宗,授正黃旗蒙古梅勒額真。從伐朝鮮,坐所部戰艦不時至,解官。尋以追敘來歸功,累遇恩詔,授世職一等阿達哈哈番兼拖沙喇哈番。卒,子錫喇布,襲。順治間,從靖南將軍珠瑪喇徇廣東,擊明將李定國,戰於新會,錫喇布力戰破敵,進三等阿思哈尼哈番。

兗楚克圖英,和勒依忒氏。初為察哈爾宰桑。林丹汗敗走,部眾皆潰散。天聰八年,太宗自大同還師,屯尚方堡,袞楚克圖英將二百餘人,與故宰桑德參濟旺等來歸。蒙古旗制定,隸正紅旗,授甲喇額真。崇德元年,授世職一等梅勒章京。二年,坐事,降一等甲喇章京。三年,從伐明,入墻子嶺,明兵自密雲至,袞楚克圖英引避,當譴,上以降將貸之,收其牲畜,分畀諸自察哈爾降者。六年,復從伐明,圍錦州,戰松山。八年,略寧遠,屢擊敗明兵。順治初,從入關,擊李自成,與固山額真恩格圖合軍力戰敗賊。二年,進三等梅勒章京。三年,從討張獻忠,屢戰皆勝。六年正月,從討姜瓖,攻大同,城兵出劫土默特營,袞楚克圖英赴援,中流矢,沒於陣,進二等阿思哈尼哈番。乾隆初,定封二等男,袞楚克圖英六世孫望吉爾襲。從討霍集占兄弟,戰死葉爾羌,贈一等男。

琿津,薩爾圖氏,世居敖漢部。太宗收敖漢,琿津從眾來歸,行失道,入明錦州。崇德六年,我師圍錦州,琿津與蒙古臺吉諾木齊、武巴什等縋城出降,授世職牛錄章京,隸蒙古鑲藍旗。旋授甲喇額真。順治初,從入關,擊李自成,署梅勒額真。督後隊,有功,加半個前程。六月,與固山額真覺羅巴哈納略山東。七月,移師徇山西,自成將陳永福據太原,琿津單騎行城下,城兵驟出,擊之,敗走,遂克太原,其屬州縣十有五皆下,賚白金。二年,與固山額真都雷逐自成至九江口,得其舟。三年,從肅親王豪格討張獻忠,時叛將賀珍據漢中,以二千人守雞頭關,拒我師。琿津率右翼兵從貝勒尼堪擊之,敗走,遂進兵入四川,與固山額真巴特瑪等擊獻忠,屢戰皆捷。

獻忠既誅,復與巴牙喇甲喇額真希爾根定涪州,以功進三等阿達哈哈番,真除鑲藍旗蒙古梅勒額真。復從鄭親王濟爾哈朗徇湖廣,時自成餘眾降於明,分屯寶慶、沅州諸郡縣。六年,琿津與噶布什賢噶喇依昂邦努山、梅勒額真拜音岱等攻克寶慶,徇沅州,破敵於洪江,斬所署總兵二、副將四、兵二千餘,得舟九。師還,進一等阿達哈哈番兼拖沙喇哈番。十五年,從信郡王多尼下雲南。十六年四月,克永昌。師渡潞江,明將李定國為伏磨盤山。師至,破其柵,琿津與固山額真沙爾布率眾深入,伏起,遂戰死,謚壯勤。

沙爾布,博爾濟吉特氏。崇德二年,自察哈爾率百餘丁來歸,授牛錄額真,即使轄其眾,隸蒙古鑲白旗。尋擢一等侍衛。至順治九年,三遷,授本旗固山額真。恩詔予世職拖沙喇哈番。十年十一月,命與寧南大將軍陳泰帥師守湖南。十二年,明將劉文秀、盧明臣、馮雙禮等將數萬人分道侵岳州、武昌,沙爾布與巴牙喇纛章京蘇克薩哈設伏邀擊,大敗之。敵復攻常德,舟千餘蔽江而下,沙爾布督軍截擊,六戰皆捷,縱火焚其舟。明臣赴水死,雙禮被創遁,文秀走桃源。沙爾布與巴牙喇纛章京都爾德等以師從之,文秀走貴州。十五年,從多尼下雲南。明年,與琿津同戰死,謚襄壯,進世職拜他喇布勒哈番。乾隆間,高宗命八旗世職先世以死事恤敘,襲次已滿者,皆予恩騎尉,世襲罔替,沙爾布等皆與焉。

論曰:蒙古喀爾喀、科爾沁諸部,東與扈倫四部接。太祖兵初起,一戰知不敵,率先歸附。明安、恩格德爾皆申以姻盟,賞延於世。鄂齊爾桑初為質子,恩禮與相亞。阿濟拜等於蒙古為庶姓,皆以功受賞。察哈爾林丹汗庭,西處宣、大邊外,太宗乘其衰,以兵收之。布延等有戰績,而洛哩諸人效命疆埸,尤有足多者。最初蒙古來附,即隸滿洲;有自明至者,又入漢軍。天聰九年,定蒙古旗制,先已籍滿洲、漢軍者,亦不復追改也。


\end{pinyinscope}