\article{志一百十}

\begin{pinyinscope}
○兵六

△水師

水師有內河、外海之分。初,沿海各省水師,僅為防守海口、緝捕海盜之用,轄境雖在海疆,官制同於內地。至光緒間,南北洋鐵艦制成,始別設專官以統率之。

其內河水師,天聰十年,自寧古塔征瓦爾喀,以地多島嶼,初造戰船。

天命元年,以水師循烏勒簡河征東海薩哈連部落。

順治初,以京口、杭州水師分防海口。八年,始於沿江沿海各省,循明代舊制,設提督、總兵、副將、游擊以下各武員,如陸營之制。各省設造船廠,定師船修造年限,三年小修,五年大修,十年拆造。十年,以水師克舟山,增造戰監,擴充兵額。十四年,增設崇明水師總兵官,調撥江寧、江蘇、安徽各省標兵萬人,分防吳淞江及崇明諸口。十六年,增設京口左右兩路綠旗水師總兵官。十八年,設吉林水師營造斛船及劃子船。

康熙八年,增設福建水師總兵官。十四年,改崇明總兵官為水師提督。十七年,設福建水師提督及參將以下各官。二十四年,裁京口右路水師,改左路水師為京口總兵官。二十六年,增設南臺水師營,置參將以下各官。二十九年,更定修造戰船之制,外海戰船哨船,自新造之年為始,三年後,以次小修大修,更閱三年,或大修,或改造。內江戰船哨船,則小修大修後,更閱三年,仍修治用之。三十四年,令督、撫、提、鎮,凡修理戰船銀兩,不得浮冒核減,致船料薄弱。五十二年,令趕繒等船,於船之首尾,刊捕盜各營鎮船名,以次編列。五十三年,增設金州水師營於海島內,選諳習水性者充之。五十六年,設松江水師營。

雍正二年,令沿海各督、撫出洋巡視。其戰船向由地方官修造者,改歸營員修造。是年,設乍浦水師營。三年,以滿洲兵丁未習水戰,增設天津水師營,以滿洲、蒙古兵二千人隸之。四年,以福建水師常駐內地,不耐風浪,浙江水師尤甚,乃更改舊制,於本省洋面巡哨外,每年選派船弁,在閩、浙外洋更番巡歷會哨,以靖海氛。五年,以杭州駐防旗兵抽練水師。江寧駐防旗兵,即以鎮江原有戰船,隸江寧將軍,督率旗兵習水戰。尋令旗兵四千人悉習水營事務。令江南、江西各水師營,於弓矢、鳥槍外,增練藤牌、大刀、鉤鐮槍、過船槍、鉞、斧、標彈等武器。戰船分大中小三等。增練排槍。湖廣水師,每兵千人,增鳥槍四百桿。六年,令水師船廠附近省城者,凡戰船造成,在城之督、撫、提、鎮會同驗看。是年,因浙江水師技藝生疏,乃於福建水師中,擇精練之兵,赴浙江教練。尋定浙江戰船用木之丈尺,及船身深廣之制。奉天水師亦如之。七年,以旅順水師不諳戰務,撥福建水師營精卒赴奉天教練。是年,增浙江乍浦水師營。八年,撥江寧駐防兵八百人隸乍浦營。旋因各省水師營承修造船之員,逐層需索,迨交收後,復盜賣損毀,各營皆然,京口標兵尤甚,令督、撫嚴懲之。九年,以文武各員承修戰船,每多貽誤,弊竇叢生,乃嚴治各員,限期修竣,以除巧脫中飽之弊。

十年,令天津水師大小趕繒船所用梗木舵牙及藤篾等具,收存備用。各省戰船設承修官,以董造船之役。由督、撫、提、鎮委副將、參將,會同文職道、府,領價督修,委都司會同文職府佐,辦料修造。隸將軍標者,委參領等官辦理。大修小修之年,各營呈報有司,題咨承修官,具冊領價。江南、江西、湖廣、福建、浙江、廣東等省,於屆修兩月前,領銀備料。臺灣、瓊州於四月前備料。天津、山東於八月前備料。各營駕船赴廠,承修官即於次月興工,如期修竣,違則懲之。其船名號各殊,大小異式,皆因地制宜。山東登州、膠州南北二汛海口趕繒船、雙篷船,福建大號趕繒船及二三號船、雙篷船,江西南湖營沙唬船,天津大小趕繒船,京口水師船,蘇州、狼山、川沙、吳淞水師船,湖北、湖南、廣東各水師船之船身大小,木板厚薄,咸遵定制,令道員會同副將等監視督造。廣東外海內河戰船亦如之。

十一年,定修造戰船限期,直隸限四月,福建、臺灣限十月,山東限六月,江西大修拆造限三月,小修限兩月,江南限四月,湖廣大修拆造限六月,小修限四月,浙江限四月,廣東瓊州限六月,其餘各廠均限四月。十二年,裁江蘇太湖營參將,改設太湖協副將,兼轄浙江太湖營游擊各官,定為內河水師營。十三年,議定天津、福建、浙江、廣東各戰船所需物料,或按年更新,或越年更新。

乾隆元年,議準江南各廠拆造及修理沙唬船、艍繒船,兩淮廠拆造沙唬船、修造趕繒船,於部價外,加津貼銀兩有差。各廠同之。二年,令山東登、膠南北二汛額設雙篷船、趕繒船,屆修之年,亦增津貼銀。三年,撥湖北武昌水師駐漢口,為漢陽水師營中軍。議準廣東各標營外海戰船拆造,視修工大小,加津貼有差。四年,因沿海各省戰船報部,有缺少至十之二三者,或侵蝕修船帑銀,或賃與商人謀利。令督撫嚴懲。又諭浙江艍繒船拆修視江蘇省之例,艄船視江蘇省沙唬船之例,量加津貼。五年,復申禁沿海戰船缺少賃用之弊。六年,以臺灣遠隔重洋,修造戰船,仍循舊制。其福建各船廠,興泉道之泉廠,與興、泉、永三府協辦,汀漳龍道之漳廠,與汀、漳、龍三府協辦,鹽法道承修之福建廠,與延、邵、建三府協辦。七年,裁江蘇黃浦營弁兵,改為提標水師右營。八年,加福建三船廠津貼銀。十二年,加臺灣船廠運費。十四年,令外海、內河水師戰船、哨船修竣後,承修官以船身丈尺及器具報有司毋損。

十五年,以閩、浙海洋綿亙數千里,遠達異域,所有外海商船,內洋賈舶,藉水師為巡護,尤恃兩省總巡大員,督飭弁兵,保商靖盜。而舊法未盡周詳,自二月出巡,至九月撤巡,為時太久。乃令各鎮總兵官每閱兩月會哨一次。其會哨之月,上汛則先巡北洋,後巡南洋。下汛則先巡南洋,後巡北洋。定海、崇明、黃巖、溫州、海壇、金門、南澳各水師總兵官,南北會巡,指定地方,蟬遞相聯,後先上下,由督撫派員稽察。至臺澎水師,仍循曩例。

十六年,令福建三江口營大小戰船,按季整洗。十七年,令各省水師,除江南省沙唬船、巡快船,福建省艍船,輕便易使,廣東虎門協營沙礁迂曲外,其沿海各省戰船,一律制備頭巾插花,借助風力,以資巡哨。巡船則仿民船,隨時修整。五十四年,以外海、內河戰船,舊例酌留一半為捕盜之用,其餘各船,次第屆期改造,咸令展期三月,福建、浙江、江南、山東各省,咸展期半年。五十五年,以搜捕海盜,戰船拙滯,允水師將弁之請,仿民船改制戰船,以期迅捷。五十八年,因廣東海盜充斥,自南澳至瓊、崖,千有餘里,水師戰船,雖有大小百數十號,僅能分防本營洋面,不敷追捕,致商船報劫頻聞。歷年捕盜,俱賃用東莞米艇,而船只不多,民間苦累。乃籌款十五萬兩,制造二千五百石大米艇四十七艘,二千石中米艇二十六艘,一千五百石小米艇二十艘,限三月造竣,按通省水師營,視海道遠近,分布上下洋面,配兵巡緝,以佐舊船所不及。五十九年,以浙江定海縣之舟山外有五奎山,外洋船隻,皆於此寄泊,實為海濱要區,於定海鎮標內,酌撥弁兵,更番戍守。六十年,以沿海戰船過於累重,不便捕盜,每屆修造,需費尤多,通飭各督撫,屆修造之年,俱仿商船之式改造,以所節浮費,為外洋緝捕之用。

嘉慶二年,浙江戰船俱仿民船改造。山東戰船亦仿浙省行之。其餘沿海戰船,於應行拆造之年,一律改小,仿民船改造,以利操防。五年,諭各省水師,向設統巡、總巡、分巡及專汛各員,出洋巡哨。奉行日久,有以千總等代巡之弊。嗣後令總兵官為統巡,副將、參將、游擊為總巡,都司、守備為分巡,遇有事故,以次代巡,不得以微員擅代。山東水師,向未有統巡等職名,亦一律行之。九年,因各省師船向遵部頒定式,僅能就近海巡查,不能放洋遠出,多改雇商船,出洋捕盜。廷臣建議,戰船改商船制度,以收實用。旋諭江蘇省濱海之區,屢有盜劫,所有舊式戰船,令疆臣仿廣東、福建、浙江之例,即行改制。十一年,諭沿海疆吏,當乾隆五十五年,曾嚴飭統兵官實力訓練舟師,乃日久玩生,弁兵於操駕事宜,全不練習,遇放洋之時,雇用舵工,名為舟師,不諳水務。嗣後通飭所轄各營,勒期訓練,一切帆舵各技,務皆嫺習。其最優者,不次擢用,惰者懲之。二十一年,規復天津水師營汛,以閩、浙、兩廣、兩江各省所裁水師,遵舊制募足額數,改隸天津水師,分營管轄。二十二年,增設天津水師總兵官,以專責成。

道光四年,諭福建疆臣,前以閩省戰船遲重,駕駛不便,曾裁汰十五船,其餘俟拆修之年,令承修官仿同安梭船式,一律改造。嗣後閩洋米艇,緝捕仍不得力,其已改造之勝字六號米艇八艘無須裁汰外,所有屆修之捷字六號十二艘,存營之勝字一號十號兩艘,修竣之勝字三號一艘,悉行裁撤。十年,令直隸、浙江、福建統兵官,增撥哨船,梭巡南北洋面。是年,定水師人員一年試驗之制,各統兵官隨帶出洋,親加考驗。又嚴定改用外海水師人員之制,其外省世職,及陸路呈改人員,有才具可用,或曾立功績者,由督撫保題。十三年,整頓浙江省水師,增造闊船、舢板船。十五年,以各省水師廢弛,憚於出巡,致盜案疊出,嚴飭水師提、鎮實力訓練緝捕。十八年,以各省戰船每屆修造之年,承辦各員,冒領中飽,不能如式制造,或以舊代新,或操駕不勤,馴至朽腐,令統兵大臣核實辦理。十九年,令督、撫、提、鎮禁將弁扣索之弊,並甄汰劣員,如有呈改召募,不得瞻徇。

二十年,以各省戰船修造草率,並有遲延積壓各弊。福建船廠所修成字四號大船,甫經拆造,即致破壞。自道光六年至二十年,積壓各船至三十艘之多。承修各員,悉予懲處。各廠應修之船,一律嚴催。其水師各船巡洋之餘,各提、鎮大員,飭將弁操練燂洗,毋任久泊海壖。又因廣東虎門海口為海防中路要區,以西境之香山,東境之大鵬,為左右兩翼,嘉慶十五年,設水師提督,節制各路。香山副將所轄水師,兵力稍厚。大鵬參將所轄弁兵,僅九百餘人。道光十年,又分為二營,其所轄大嶼山及尖沙嘴洋面,為夷船聚泊之所,乃擇要建砲臺二座,與水師相依護,以澄海副將改為大鵬協副將,移駐九龍山,增額設水師,兼守砲臺,增造大號中號米艇四艘,快船二艘,在水師各協營,抽配弁兵,巡緝洋面。

二十一年,以外夷船堅砲利,舊設外海水師,強弱不敵,等於虛設,擬改水師為陸師,專防內地。尋以海盜滋擾,全恃水師緝捕,廣東之虎門,為外海籓籬,尤藉舟師之力,乃定議緩裁。

二十二年,以海上用兵,專恃砲火,令各疆臣訓練弁兵,一律以施放砲位有準,為弁兵去取。又以海上用兵二載,閩、粵、江、浙水師,迭致挫敗,令四川、湖廣等省,採購巨木,速制堅船,駛往閩、浙等省,防守海疆。尋因各省戰船,如快蟹、拖風、撈繒、八槳等船,僅能用於江湖港汊,新造之船,亦止備內河巡緝,難於海上沖鋒。惟潘仕成捐資新制之船,堅固適用,砲亦得力,並仿美利堅國兵船制造船樣一艘,又仿英吉利國中等兵船之式,調取各省工匠,改造大船。其例修師船,一律停造,以資挹注。並以船砲圖說,飭江蘇、福建、浙江三省督撫詳勘,何者利用,由廣東省制成,分運各省。又因湖北省所轄長江千餘里,舊設宜昌鎮標,荊州、漢陽各水師營,戰船不能載砲。廣東匠役何禮貴曾為外洋造船,能造火輪及各式戰船,飭赴湖北,擇何項戰船利於長江駕駛,即就海船之式,量為變通。裕泰擬造之開浪船,於海戰未宜,罷之。

二十三年,飭沿海各提、鎮,於每歲出洋及巡洋事畢,所經歷情形,悉以上聞。三十年,因浙江省水師廢弛,飭有司整治船砲,嚴禁奸民接濟海盜,並令沿海將領,按時出洋會哨。又令山東疆臣,以三汛師船,四縣水勇,合而為一,統以專員,往來策應,並於扼要島嶼,設置大砲。

咸豐元年,以長江轄境綿長,令張亮基等購置船砲,擇要駐守。三年,調廣東外海水師拖罟戰船,及快蟹、大扒等船百艘,統以大員,由海道駛赴江寧,助剿粵寇。是年,江忠源疏請廣制戰船,以靖江面。旋令兩廣督臣,以廣東拖罟船式咨行四川、湖廣各督撫,或在本省,或在湖北宜昌一帶,迅簡工匠,造水師船百餘艘,每船載兵五十人,於三月內竣事。兼飭湖南、湖北二省,購船募兵,與長江下游艇船,協力防江。旋以所購民船不合用,乃收買江船之巨者,仿廣東船式,安置砲位,與廣東所募紅單船,及賃用拖罟船,駛赴江南剿寇。又以廣東內河及濱海各縣,均有捐造緝捕快蟹船,道光間,江海捕盜,悉藉其力。船頭藏巨砲,旁列子母砲,勇丁咸技藝精練,洵水戰最長。令各船由海道至長江會師。是年,曾國籓試造師船於湖南,以規模過小,乃就廣東之拖罟船、快蟹船二種,參酌其制,先造十艘,續增二、三十艘,以能載千斤之砲為度。至拖罟船,則由兩湖督撫如式制造。

四年,令廣東賃用之紅單船二十三艘,並修治十九艘,凡四十二艘,統一武員,駛入長江。是年,以粵寇竄擾東南,水師不敷剿堵,下游惟廣東紅單、拖罟等船漸集瓜洲,上游惟曾國籓新造戰船,自湖南順流而下,已達武昌。其九江、安慶等處,尚無戰艦,令張亮基、駱秉章購置江船及釣鉤等船,裕瑞、夏廷樾在四川採購材料,與駱秉章商辦。旋駱秉章以四川造船,江險而途遠,水程不便,仍在湖南購料製造。兩湖紳士丁善慶,遵曾國籓所定之式,已成大板艇五十號,長龍等船亦次第告成。長江剿寇,在江南取勝者,以紅單船、拖罟船二種為最,體勢雄壯,置砲最多,而能順風不能逆風,宜江面寬闊,不宜港汊。在湖南取勝者,以舢板船、長龍船、快蟹船三種為最,往來輕便,搜捕尤宜,而風急水溜,一下難於遽上,勢散而力單。令湖南水師沿江攻剿,與江南水師會合,各用其所長,以期制勝。

六年,以曾國籓在江西所造戰船,最為得力,令福濟選擇將弁,率工匠赴廬州仿造,所需洋砲,在上海撥款辦解。六年,胡林翼以長江水師,自五年春間回駐武、漢以後,戰艦無多,乃與駱秉章協商,督率船砲局各員,盡力籌謀,水師復振。湖南紳局所制船械,交至營中者,大小戰船凡三百餘艘,火藥四十餘萬斤,砲子一百四十萬斤,其餘各械咸備,請優詔獎之。水師重在砲位,廣東運到洋砲二百尊,續運六百尊,配置各師船,自武、漢至九江,所向克捷。惟長江水戰,上下游形勢不同,武、漢以上,利用輕便戰船,潯、皖以下,江面漸廣,利用巨艦,秋冬風勁宜巨艦,春夏宜小艇,船砲之大小,宜因時因地而損益之。請令兩廣督臣,續購大小洋砲,自四百斤至一千五百斤,凡八百尊,盡易舊式砲位,以利東征。八年,以天津原設水師,道光間,先後裁撤,乃籌復設,以重海防。令福建、廣東疆吏,各抽調大號戰艦,備齊砲械,由海道駛赴天津,設水師三千人。十年,令清淮籌防局籌款,為防湖水師常年經費,增設淮揚水師營,以保兩淮鹽場,兼佐陸軍。蕪湖孤懸水中,令曾國籓籌設寧國水師,以攻蕪湖,為克金陵之本。增設太湖水師,為克蘇州之本。

同治二年,諭沿海督臣舉水師將才。又令曾國籓所部內江水師,都興阿所部揚防水師,有勝外海水師之任者,各舉以聞。四年,在山東省仿長江戰船之式,造長龍、舢板船,於黃河水性駕駛合宜。以水師泊黃、運二河,防堵逸寇,必須分段扼守,而地勢綿長,不敷調遣。復由山東增造長龍船,並增舢板船十艘,以武職大員督領巡防。五年,改造江南海口之紅單廣艇三十艘,合原有廣艇凡四十艘,分防海口。六年,整頓福建臺灣海防,增置龍艚等船。

七年,曾國籓議改水師之制,以江南水師,向分外海、內河二支,外海水師六千七百七十六名,武員一百十八人,內河水師八千二十一名,武員一百三十三人。船數則近稽道光二十四年江南舊例,水師船二百七十五艘,朽壞居多,別造舢板船一百三十五艘,大船十二艘,約計各船不過載兵二千數百人,而額定之兵數,尚有萬餘人。徒費餉項,有水師之名,無舟楫之實,宜大為變通,講求實際。江蘇水師,其營制餉章,悉仿長江水師之例,外海之紅單廣艇,亦略增其餉,與李鴻章、丁日昌諸臣協力籌辦,期於外防與內盜並謀,舊制與新章兼顧。俟章程既定,沿海福建、廣東各省水師,均可酌改行之。

八年,部臣定議,從曾國籓所陳,改江蘇水師為內洋、外海、里河三大支,以資控御。里河水師,以原設提標右營,太湖左營、右營及增設淞北、淞南二營為五營,隸提督統轄。舢板船每營船數不等,一律興修,不得缺額。所有太湖七營,改為里河五營,其裁員歸並提標序補。馬新貽等續議,九江水師營改城守營,設陸汛四處。鄱陽營改陸汛二處。洞庭水師營改龍陽城守營。岳州水師營酌留水兵,隸陸路管轄。荊州水師營酌留弁兵有差。九年,諭定安等以寧靈各軍,運餉艱難,增造砲船,由黃河運送。所有大小戰船三十二艘,編為一營,設統帶等官。

十年,沈葆楨以外海兵船制成,應簡知兵大員督率操練。尋以福建水師提督李成謀為兵船統領。是年,曾國籓以江南水師章程初議十四條,嗣由馬新貽等增為二十五條,乃刪減歸並定為二十一條。外海六營,以次巡哨。內洋五營,分定汛界。里河五營,分定汛界。淞南營、淞北營、太湖左營、太湖右營酌增戰船。水師營所遺陸汛,移並留防。京口三營陸汛砲堤,分別管轄。改定轄營,及留設守備,要汛多留陸兵。定將弁處分則例。規定各營船數。酌定外委員數。酌定裁缺薪糧。營官建衙署地方,各營座船之數,官兵額定數目,書吏名數,雨蓬旗幟等費,官兵糧額,各船酌用槍砲數目,各船酌用火藥槍砲彈,綜計餉項之數,下所司核議行之。十一年,丁寶楨調撥福建省所制安瀾兵輪赴山東洋面巡緝。瑞麟調撥福建省小號兵輪赴奉天海口巡緝。

光緒四年,裁去廣東輪拖巡船水勇二千三百餘人。五年,以各省舉辦水師,奉天、直隸、山東、江蘇、浙江、福建、廣東次第駐泊兵輪,編制水師,而沿海各省形勢不同,操法未能一律,吳淞口為南北海疆適中之地,乃命江南提督李朝斌為外海兵輪統領,督率各省大小兵輪,定期在吳淞口會操。六年,以新置蚊砲船便利合用,續向外洋購置數十艘,募福建、廣東沿海精壯之民為水師,分屯北洋各海口。七年,以奉天旅順口原有旗營,艇船朽壞,弁兵疲弱,悉行裁汰,歸陸師巡防,別以快砲防海。時丁汝昌由英國率戰艦回國,為中國水軍航外海之始,乃擢丁汝昌為水師提督。

八年,以江南形勢,先海後江,朝議擬以長江水師提督駐吳淞口,狼山、福山、崇明三鎮標隸之,以江南提督移駐淮、徐,改福建水師提督為閩浙水師提督。尋左宗棠、彭玉麟議覆,以海防不外戰守二端,戰宜厚集兵力,守宜因勢設險,仍循舊制為宜。福建水師自裁兵加餉後,實存水師六千九百餘人,旗營水師三百餘人,各營拖罾、龍艚、快艇等大小戰船實存四十艘,臺灣、澎湖戰船六艘,大小兵輪十艘,宜聯合浙江省水師會操,官制則仍循其舊。

九年,以廣東內河之肇慶河面縣長六百餘里,僅有小巡船二十餘艘,不敷分布,九龍洋面水淺,大船難於行駛,乃於二處各增設淺水兵輪船。十年,試造尖底舢板船,分布海口。旋以船質弱小,罷之。十一年,彭玉麟以海防日亟,議設水師總統於吳淞,分設二鎮:一駐直隸大沽,凡盛京、直隸、山東、江南各海口戰船隸之;一駐福建廈門,凡浙江、福建、臺灣、廣東各海口戰船隸之。兩鎮每年周巡海口,會哨於吳淞。是為南北洋水師建議之始。十二年,議裁減浙江沿海水師。旋浙撫劉秉璋以舊額戰船二百五十餘艘,粵寇亂後,購造不及半數。光緒八年,裁水師船十三艘,停修舊船三艘,已符裁兵三成之數。惟巡洋之紅單船十一艘,不在額設裁減之例。十四年,因臺灣疆土日闢,改安平水師副將為臺東陸路副將,改鹿港游擊為安平水師游擊,任新設地方鎮守之職。十五年,以福建內河水師砲船舊額共九十八艘,頻年裁撤,實存三十艘。每船配置水師六人,專任巡緝內洋。十六年,調撥福建海壇水師駐防福清縣屬,以靖海盜。十七年,於湖南選鋒水師中、前、後、左、右營內,撥一千六百餘人,分防省城及岳州等處,撥長勝、毅安水師四百餘人,分防辰州、沅州、常德等處,撥澄湘水師三百餘人,分防衡州等處,以專責成。是年,因奉天遼河下游舊有巡船,上游則僅有陸隊兼巡,未有水師。乃增置長龍砲船一艘,舢板船八艘,於練軍內選撥弁勇,梭巡遼河上下及省南之洋河。

十九年,令提督黃翼升校閱長江提標五營,上江十三營,下江四營,定訓誡之規,禁陸居,戒嗜好,勤練藝,屏虛文,不得蹈綠營之習,日久玩生。閩、浙疆臣會議,以浙江省有元凱、超武二兵輪,福建省有伏波、琛航、靖遠三兵輪,與沿海水師協力緝捕。而浙省水師船自裁減後,僅存五十餘艘,閩省自馬江戰後,僅存艇船二十九艘,乃在寧波海口賃用紅單等船八艘,酌撥弁兵,以靖海盜。二十四年,令江蘇省之外海、內洋、里河、太湖四支水師,一律酌裁水勇。二十五年,以安徽省江防在下游者為東西梁山,建有砲堤砲臺,在上游者為闌江磯前江口及省城之江心洲,咸有砲臺,而缺乏水師,乃撥澄清營砲船二十五艘,及長江水師之蕪湖、裕溪、大通、安慶、華陽各營,聯絡防守。又令長江五省督撫,各派將領,不分畛域,嚴密設防。二十六年,以奉天鳳凰沿海一帶,素稱盜藪,曾由北洋撥兵輪巡洋,其支港各處,宜屯泊水師,乃於大孤山、太平溝、沙河三口岸,各造兵船三艘,酌配水師巡緝。三十四年,因浙江杭、嘉、湖三府捕匪兵單,於原有水師中,抽練游擊一隊,駐嘉興府,增練游擊三隊,分布嘉興、湖州各河港,以游擊小隊駐杭州省城,賃用上海商人之小輪船十艘,曳帶兵梭船巡水道,以期迅捷。在南洋船塢造淺水兵輪船四艘,配快砲八尊,江蘇亦制淺水兵輪船四艘,協同內河水師,仿歐西各國章程,編為聯隊,以資防剿。此整治水師之概略也。

其兵額之增減,船械之配置,各省隨時編定。外海水師,北自盛京,南訖閩、廣,凡拖繒、紅單等船隸焉。內河水師,各省巡哨舢板等船隸焉。奉天、直隸、山東、福建水師船均屬外海。江西、湖廣水師船均屬內河。江南、浙江、廣東水師船分屬外海、內河。其別練之師,有巡湖水師、巡鹽水師、親兵營、練軍營。同治以後,增定長江水師、太湖水師之制,視舊制加詳矣。

其巡防之規,外海水師巡防盛京,以協領為總巡,佐領、防禦、驍騎校為分巡。直隸等沿海各省,以總兵官為總巡,副將以下為分巡。各於所治界內,率水師沿海上下,更番往來,詰奸禁暴,兩界相交之處,戒期會哨,以巡緝情形,申報所屬將軍、總督、提督,委員稽察。若因風阻滯,各廠到界之日具報。其每歲定期,以二月、四月、五月為始,至九月事竣回營。有引避不巡,或巡而不周遍者,論如軍律。其內河水師巡防之制,長江自四川巫山而東,出三江口,至湖廣界,經嶽州、武昌、興國至江西界,經九江、江寧、京口等處,東至於海,各省將軍、總督、提、鎮分委旗標弁兵,沿江游巡,及界而還。

自康熙以後,以外海利用巨艦,內河利用輕舟,故船制屢改,而轄境遼闊,水師兵額,時有增加,遇有戰事,增艦尤多。征吳三桂之役,命尚善率舟師入洞庭湖取岳州。及鄂鼐統水師,增造鳥船百艘,沙船四百三十八艘,置水師三萬人。征臺灣之役,命萬正色督率湖南、浙江戰船二百艘,由海道赴福建。姚啟聖亦修戰船三百艘,水師二萬人。施瑯之克澎湖,用戰船三百艘,水師二萬人。施世驃之平硃一貴,用大小戰船六百餘艘。乾隆間,徵緬甸之役,命湖廣船匠造船於蠻暮,取道金沙江以攻緬甸,兼調福建、廣東水師助之。李長庚之剿海寇,在福建造大船三十艘,名曰霆船,配置大砲四百尊,合閩、浙水師全力,轉戰重洋,遂平蔡牽。

道光以後,海警狎至,木質舊船不敵外洋鐵艦之堅利。同治五年,始仿歐洲兵輪船式,於福建省開廠制造輪船。江蘇初設輪船四艘。十一年,廣東、山東各設輪船一艘,奉天設小輪船一艘,咸配置水師。

其後沿海各省購置兵輪,歲有增益,舊式水師戰船分別裁汰。至光緒中葉,綜各省外海、內河實存師船之數,奉天外海繒船十艘。直隸外海長龍船二艘,先鋒舢板船四十八艘。山東外海拖罾船十四艘,內河哨船六艘。江蘇外海輪船二艘,艇船八艘,內洋輪船二艘,舢板船六十艘,內河舢板船、艇船三百八十五艘,長江舢板船七十六艘,督陣舢板船七艘,長龍船十艘,巡哨舢板船一百二十八艘。安徽舢板船二百八十二艘,長龍船十五艘,八團船一艘,槍劃十艘,護卡巡船十五艘,督陣舢板船七艘,輪船二艘。江西長龍船十五艘,舢板船二百六十三艘,督陣舢板船六艘,輪船一艘。福建外海長龍船一艘,舢板船十九艘,小艇十四艘,哨船十四艘,龍艚船二艘,拕艚船一艘,內河砲船三十艘。浙江外海釣船二十七艘,艇船十二艘,龍艚船十七艘,哨船二艘,快船一艘,內河大舢板船五十八艘,中舢板船八十四艘,飛劃船四十九艘,長龍船座船二百十三艘,槍船八艘,砲船五艘。湖北督陣大舢板船八艘,長龍船十二艘,舢板船一百八十艘。湖南督陣大舢板船四艘,長龍船四艘,舢板船六十艘。廣東外海大小輪船二十二艘,巡船十四艘,拕船十艘,長龍船一艘,扒船一艘,內河兩櫓槳船一艘,櫓船一艘,槳船四十艘,巡船一百九十六艘,急跳船十五艘,平底槳船二艘,快哨船二艘,快船十四艘,快槳船七艘,艚船四艘,櫓船二艘。

各省戰船,咸分隸標營,擇地屯泊,以時會哨。外海師船,以海軍規制漸立,僅任沿海捕盜之責。各省內河師船,均仿長江水師舢板船之式。惟巡緝等船,分巡支河汊港,利用輕捷,船制少殊耳。

其漕、河水師營制,始於明代隆慶間。清代略更其制。以衛卒專司輓運漕糧,以營兵專任護漕,別設城守營守護城池。分漕院與巡撫為二,漕運總督標下,統轄左、右、中三營及城守四營,駐山陽境及漕運要地,分別置兵。淮郡舊為黃、淮二河交注之區,特建兩大閘,設河兵及堡夫守之。河營遂與漕營並重,各有副將、參將、游擊、守備等官。河營升遷之例,與軍功等,專司填築堤防之事,而緝捕之責不與焉。

清代水師武功之盛,守洞庭而平吳逆,戰重洋而殲蔡牽,下長江而制粵寇,東南數千里,威行桴鼓,勞臣健將,蹈厲功名,超逾曩代。及海禁宏開,鐵船橫駛,舟師舊制,弱不敵堅,遂盡失所恃。時會迫迕,非規畫之疏也。

凡直省舊額船數分防之制,分列於篇:曰東三省,曰直隸,曰山東,曰江南,附太湖湖標、漕標各水師,曰浙江,曰福建,曰廣東,曰廣西,曰湖南,曰湖北,曰安徽,曰長江水師。

東三省沿海各口岸,以金州、旅順口為尤要。清初即有水師之制。松花江、嫩江貫注吉林、黑龍江二省腹地。所設水師營汛,由吉林而北抵墨爾根、黑龍江一帶。至光緒間,旅順築海軍港,屯駐鐵艦,迥殊曩制。其東部之圖們、混同江上,時有俄羅斯戰艦侵軼,非舊制師船,械弱兵單,所能控制矣。列經制水師於後:

奉天旅順口,於順治初年設水師營,以山東趕繒船十艘隸之,始編營汛。康熙十五年,設水師協領二人,佐領二人,防禦四人,驍騎校八人,水兵五百人。五十三年,由浙江、福建二省船廠造大戰船六艘,由海道至奉省,駐防海口。

金州水師營隸城守尉,水兵百人。

吉林水師營,順治間,設四、五、六品官。光緒十四年,增設總管一人,六品官二人。

齊齊哈爾水師營,康熙二十三年,設總管一人,四品官二人,六品官二人,造船四、五、六品官各一人,領催八人,水兵二百六十八人,後增至五百六十八人,大戰船二艘,二號戰船十五艘。康熙四十年,撥歸黑龍江十艘。雍正間,撥歸墨爾根六艘,存大小戰船二十五艘,江船五艘,劃子船十艘。

墨爾根水師營,康熙二十三年,設四品官一人,領催一人,由本城協領兼轄,凡戰船六艘,水兵四十三人,雍正間增戰船六艘。

黑龍江水師營,康熙二十三年,設總管一人,五品官二人,六品官二人,領催八人,戰船三十艘,水兵四百十九人。四十年,自齊齊哈爾撥船增之,凡大戰船十艘,二號戰船四十艘,江船十艘,劃子船十艘。

直隸省水師,始於雍正四年,設天津水師營,都統一人,駐天津,專防海口,水師凡二千人,省內各河,咸歸陸汛,無內河水師。乾隆八年,增設副都統一人,水師千人,大小趕繒船二十四艘,仔船八艘。三十二年,以海口無事,徒費餉糈,全行裁汰。嘉慶二十一年,復設水師千人。旋設大名鎮,以水師總兵歸並大名,實存守備一人,參將一人,千總二人,把總三人,水師四百九十一人。嘉慶十九年,直隸督臣那彥成以官兵虛設,兵船多朽,疏請裁撤,仍並入大名鎮。咸豐八年,以海疆多警,增設海口六營,於大沽南北兩岸,修築砲臺,凡大砲臺五座,平砲臺十座,大砲九十九尊,水師三千人,以五百人為一營,分編左右六營。九年,改為一千八百人。同治八年,督臣李鴻章疏請酌定營制,設大沽協副將,駐新城海口,防守砲臺。光緒元年,李鴻章於大沽、北塘等處,增建砲臺,購置歐洲鐵甲快船、碰船、水雷船,以海軍將領統之,不隸舊制協標之內。

其內河水師船,始於同治間,仿長江水師之制,設督標水師中營,管帶官一人,哨官三十二人,水師四百七十六人,舢板戰船三十二艘,駐三岔河口,親兵總哨官一人,哨官十四人,水師二百二人,舢板戰船十四艘,駐西沽河口。

山東,順治元年,始於登州府設水師營,領以守備、千總等官,凡沙唬船、邊江船十三艘,水兵三百八十六人,駐扎水城,分防東西海口。十五年,移沂州鎮於膠州,改膠州水師為陸營。十八年,移臨清鎮於登州,以隸屬城守營之水師,改為前營水師。康熙四十三年,增設游擊二員及守備以下各官,增水師為千二百人,改沙唬船為趕繒船二十艘,分巡東西海口,東至寧海州,西至萊州府,分為前後二營,各專其職。四十五年,以前營水師移駐膠州,巡哨南海,後營水師駐水城,巡哨北海。五十三年,裁後營經制員弁,裁水師七百人,撥趕繒船十艘赴旅順口,僅存前營水師游擊等官,趕繒船十艘,分南北二汛,以游擊、守備分轄兵船之半。雍正七年,每船增兵十人,兩汛共增兵百人,增雙篷艍船七艘,每艘配兵三十人,南汛艍船三艘,北汛艍船四艘,北汛增將弁一人。九年,又增設艍船三艘,增兵一百九十人,每艍船共配兵四十人,南北汛各五艘。十二年,增將弁六人,又於成山頭增設東汛水師,抽撥南北汛趕繒船各一艘,雙篷艍船各一艘,分配戰守兵,撥南北汛將弁四人,配船巡哨成山、馬頭嘴一帶,與各汛會旗,總歸水師前營管轄,以本鎮統之。列定制於後:

前營水師,游擊、守備各一人,千總二人,把總四人,外委千總二人,外委把總四人,水戰兵八百人,守兵二百人,趕繒船十艘,雙篷艍船十艘,每船各帶腳船一艘。南汛駐膠州之頭營子,游擊一人,把總二人,外委千總、把總各一人,趕繒船四艘,雙篷船四艘,共配戰守兵四百人,南境巡哨至江南交界之鶯游山,東至榮成縣馬頭嘴,與東汛會旗。東汛駐養魚池,千總、把總各一人,外委千總、把總各一人,趕繒船四艘,雙篷艍船四艘,共配戰守兵四百人,南境巡哨至馬頭嘴,與南汛會旗,北境巡哨至成山頭,與北汛會旗。北汛駐登州府水城,中軍守備、千總、把總各一人,外委把總二人,趕繒船四艘,雙篷艍船四艘,共配戰守兵四百人,南境巡哨至成山頭,與東汛會旗,北境巡哨至隍城島,與直隸水師、盛京水師分界。

江南水師,順治初年,江蘇松江等營,各有捕盜小快船四十艘,常州、鎮江等營,各有一、二十艘不等。自康熙七年,查毀沿江海各營出海之船,其內河快船,亦從裁汰。嗣巡撫馬祜、提督楊捷疏請蘇、松、常、鎮四府,各塘汛設水師巡船三百二十五艘,以靖水盜。雍正元年,江蘇、浙江督撫會商,以太湖連跨二省,夙為盜藪,乃於湖濱各口,增設水師營汛巡船,分界巡防。其湖內各地,系二省交會者,令參將各率水師會同巡緝。五年,令京口八旗營仿天津水師之制,設京口水師營,分撥京口大小戰船二十艘至江寧練習。其駐寧水師,凡滿洲、蒙古兵千人,設協領四人,佐領、防禦、驍騎校各十二人。是為江南水師之始。

鎮守京口左右兩路水師,設統兵官二人,分統左右營,各置沙船二十八艘,水艍船八艘,犁繒船八艘,舵椗手二百二十人,水手、匠役四百九十二人。康熙二十一年,改隸江南提督標下,分為中、左、右三營。三十六年,裁總兵官,設副將以下各官,每營設沙船二十三艘,唬船七艘,小巴船四艘,水手四百六十八人。自雍正二年後,迭有改撥,每營存唬船二艘,小巴唬船七艘,渡馬淺船六艘。

江南提督水師標兵,順治四年,始設參將以下各官,分為中、左、右、前、後五營及城守六營。中營唬船一艘,巡船十五艘,中號四櫓哨船二艘,槳櫓快哨船二艘。左營唬船三艘,巡船十五艘,中號四櫓哨船二艘,槳櫓快哨船二艘。右營浦江游巡哨船四艘,改設哨船一艘,槳船一艘,二櫓哨船一艘,巡船七艘。前營中號四櫓哨船二艘,槳櫓快哨船二艘,巡船二十二艘。後營唬船一艘,中號四櫓哨船二艘,槳櫓快哨船二艘,巡船八艘。

松江城守營唬船一艘,中號四櫓哨船二艘,槳櫓快哨船二艘,巡船一艘。

金山營巡船十三艘。

柘林營巡船四艘。

青村營巡船二艘,小哨船一艘。

南匯營大罟船二艘,小哨船四艘。

川沙營捕匪大罟船三艘,放大罟船三艘,大罟船二艘,小哨船二艘,小號二櫓哨船二艘。

劉河營巡船八艘。

吳淞營沙船三艘,艍犁船四艘。

福山營沙船四艘,官渡船四艘,巡船十六艘,後改為福山鎮標,設總兵以下各官。

太湖營沙船、快船、巴唬船共三十二艘,後改為太湖協標,設副將以下各官。

常州營巡船二十九艘。

江陰營唬船二艘,巡船七艘。

靖江營唬船二艘。

楊舍營巡船二艘。

鎮江城守營,順治十五年,設鎮守蘇、松水師總兵官,分中、左、右水師三營,各設沙船九艘,趕繒船五艘,尋改為參將等官,設巡船二十三艘。

江南督標游兵營,順治初年,隸操江巡撫標下,設游擊以下各官,大唬船一艘,小唬船二十七艘。康熙元年,裁並入督標。

奇兵營,順治初年,隨操江巡撫赴安徽省駐防,改為太平右營,設游擊以下各官。康熙元年,裁並入安慶營。

瓜洲營,順治二年,設守備以下各官,專防江北水汛,唬船八艘。康熙元年,改為參將,並入江南督標。十一年,改為瓜洲城守備,唬船八艘。其各縣分防水師,寶應汛船十五艘,氾水汛船十四艘,永安汛船二十三艘,高郵汛船十六艘,江都汛船十四艘。

淮安廟灣營,順治初年,設游擊以下各官,沙船五艘,唬船四艘。

佃湖營,雍正九年,由廟灣營分防,設都司以下各官,沙船三艘,巡船四艘,唬船一艘,內河巡哨船一艘。

營城營,順治三年,設守備以下各官,唬船四艘,小巡船四艘。乾隆十一年,改為唬船一艘,巡海哨船二艘,三號四號沙船二艘,小巡船四艘。

小關營,雍正十一年,由鹽城營分防,設都司以下各官,沙船二艘,唬船一艘,快船一艘。

海州營,順治四年,設游擊以下各官,小巡船五艘。康熙三年,並入東海營,增設沙唬船十艘。

東海營,順治初年,設守備以下各官。十八年裁撤。康熙十八年復設。其分防汛地,鷹游內外洋汛船二艘,大浦汛商船二艘,海頭汛唬船一艘,臨洪口汛哨船二艘,高公島汛沙船一艘。

江蘇撫標左右營,順治四年,設參將以下各官,左營巡船十艘,右營巡船十艘。

蘇州城守營,順治四年,設參將以下各官,巡船五十八艘。

平望營,順治三年,設游擊以下各官,巡船十七艘。初隸提督標,乾隆以後,改隸巡撫標,巡船二十艘。

福山營,自提督標分防,設游擊以下各官,其沙船四艘,巡船六艘,官渡船四艘,船數仍如曩制。道光間,以海疆要地,改為鎮標,設總兵以下各官,分為中、左、右三營。

淞北營,原隸督標內河水師,同治十一年,改隸江南提標水師,增設副將以下各官,仿長江水師之制,設舢板船十六艘。

淞南營,同治十一年,改隸里河淞北協標,增設游擊以下各官,仿長江水師之制,設舢板船、座船凡三十七艘。

江北狼山鎮標,順治十八年,設總兵以下各官。分中、左、右三營。中營趕繒船一艘,沙船一艘,唬船三艘,渡船五艘。左營趕繒船一艘,沙船二艘,唬船三艘,渡船六艘。右營趕繒船六艘,沙船一艘,唬船四艘,渡船三艘。

泰州營,順治二年,設游擊以下各官,趕繒船二艘,沙船二艘。

掘港營,順治三年,設守備以下各官,唬船三艘。

康熙二十三年,以京口將軍標下沙船二十二艘,唬船十八艘,隸狼山鎮標,為海口巡防水師。二十八年,以戰船四艘,仍撥歸京口。四十八年,改為中、左、右三營。中營趕繒船三艘,沙船二艘。左營趕繒船三艘,沙船三艘。右營趕繒船四艘,沙船二艘,唬船四艘。雍正十年,實存大小水師船二十二艘。十三年,右營增小哨沙船一艘。同治五年,增設綏通、綏海二營,隸長江水師提督。

江南福山鎮標,道光二十三年,設總兵以下各官,分中、左、右三營,以舊有之福山營水師為福中營,蘇松奇兵營水師為福左營,楊厙水師為福右營。中營舊設巡船十五艘,日久朽壞無存,以沙船四艘增換闊頭舢板船五艘,左營設大小舢板船八艘,右營設大小舢板船五艘。同治九年,改定營制,以中營並入左右營,以左營原轄之海門屬西半洋沙等汛,隸通州營,以左營分中左右三哨,分駕巡船十二艘,出巡洋面,以右營駐防陸路各汛。

太湖水師,始於雍正間。太湖連跨蘇州、常州、湖州之境,為全吳巨浸。湖中風浪與江海異,故巡湖水師,船制亦殊。其衛所巡司則以巡船,水師則以哨船。雍正二年,設太湖營游擊、千總、把總各一人。五年,以大錢汛口為浙江省瀕湖要道,增守備、千總各一人,把總三人,戰守水兵原額千人,歷年裁並,實存水戰兵一百八十六人,守兵四百七十二人,分防各處:甪頭汛兵一百八十五人,沙快船五艘;西山汛兵六十九人,沙快船二艘;浙江烏程汛兵一百九十七人,沙快船九艘;伍浦汛兵六十九人,快巡船九艘;南浦汛兵一百七人,快巡船九艘。七年,以沙船六艘為湖中大汛巡防,其餘改小號巡船二十艘,巡緝支河小港。九年,分水師為左右二營,左營守備駐簡村,列汛凡六,當震澤縣界。千總一人,駐占魚口,列汛十有二,當吳縣、吳江、震澤界。把總二人:一駐東山,列汛凡八,當吳縣界;一駐吳江,列汛凡八,當吳江、震澤界。右營守備駐周鐵橋,列汛凡六,當宜興、陽湖界。千總一人,駐馬山,列汛十有四,當常州、無錫、陽湖界。把總二人:一駐黿山,列汛凡七,當吳縣界;一駐鳳川,列汛凡七,當宜興、荊溪界。乾隆間,設副將以下各官,水師戰船,凡巴唬船十六艘,沙船三艘,大快船七艘,小快船三十二艘。至道光間,存巴唬船十六艘,沙船二艘,大快船六艘,小快船二十艘,槳船十艘。迨咸豐年粵匪亂後,營伍船械全失。同治間,重整水師,盡易舊制,仿長江營制,設太湖協標二營,舢板戰船三十六艘。此江南水師之制也。

其長江水師之在江南省者,為瓜洲鎮標,轄瓜洲營、孟湖營、三江營、江陰營,戰船兵額,與各省長江水師同。

河道總督標營凡二十營,雍正七年,以漕標右營改隸河標設,巡船九艘。山清里河上營,康熙十七年設,船六十八艘。里河下營,雍正六年,由里河營分設,船十三艘。外河上營,船一百十四艘。山安海防河營,雍正七年,由外河營分設,船五十四艘。高堰上營,康熙三十八年,由盱眙營分設,船三十四艘。山盱下營,雍正七年,由高堰營分設,船十七艘。桃源安清營,康熙三十八年設,船二十三艘。揚河上營,康熙十七年設,船八十二艘。揚河下營,雍正七年設,船十四艘。徐河南北營,雍正六年設,船三十艘。邳睢河營,順治初年設,船七十五艘。宿虹南北營,順治初年設,船百艘。桃源南北營,順治初年設,船六十八艘。宿遷運河營,雍正六年設,船十九艘。凡河防各營,設守備以下各官,大小各船,分浚船、柳船二類,修防河工,以營制部勒之。

漕運總督水師標營,分中營、左營、右營、城守四營,以中、左、右三營任護漕之責,以城守四營任地方之責,駐山陽境及漕運所經之地。其運輓漕糧,則以衛卒任之。

浙江水師,杭州協錢塘水師營,順治初年,設守備各官,兵一百十五人,鱉子門汛兵七十九人,新城汛兵三十一人,塘棲汛兵九十三人,錢江汛兵七十七人,富陽汛兵一百五十人,防守河莊山唬船四艘,運河內河快唬船十一艘,錢塘江渡馬船六艘。

乍浦水師營,雍正二年,以定海鎮右營改歸乍浦,設參將各官,水戰兵二百四十人,守兵二百七十六人,戰船十艘,內洋岑港轄洋面汛三十三,內洋瀝港轄洋面汛十五,內洋岱山轄洋面汛十九。

嘉興協營,設副將各官,駐防府城,兵四百三十二人,快唬船五艘。海鹽汛兵一百七十五人,快唬船三艘。乍浦汛兵二百十三人,快唬船二艘。澉浦汛兵百人,快唬船一艘。石門汛兵一百十人,快唬船四艘。桐鄉汛兵七十六人,快唬船二艘。濮院汛兵六十一人,快唬船三艘。新城汛兵四十人,快唬船一艘。平湖汛兵九十九人,快唬船三艘。嘉善汛兵七十人,快唬船二艘。嘉興汛兵六十九人,快唬船二艘。王江涇汛兵五十六人,快唬船二艘。雍正十年,裁撤快唬船二十艘,改造大號巡船二十艘,小號巡船二十艘,分配各汛。

湖州協營,設副將各官,駐防府城,兵四百七十六人,快巡船十三艘。左營分防雙林汛兵五十人,快巡船三艘,德清汛兵三十四人,快巡船四艘,新市汛兵四十二人,快巡船四艘,含山汛兵四十二人,快巡船四艘,菱湖汛兵三十九人,快巡船五艘。右營分防泗安汛兵五十人,快巡船三艘。長興汛兵四十四人,快巡船二艘。武康汛兵二十人,快巡船一艘。馬要汛兵二十人,快巡船一艘。烏鎮汛兵二十四人,快巡船一艘。南潯汛兵五十八人,快巡船六艘。菁山汛兵十六人,快巡船一艘。梅溪汛兵八十人,快巡船二艘。

紹興協營,設副將各官,水師一千八百七十二人,用衛所之制,設臨海、觀海二衛,瀝海、三江二所。雍正十年,設周家路水師汛,置紹字一、二號巡船二艘。

寧波府,順治三年,設水師營參將二人,分左右二營,水戰兵四百人,守兵四百人。十四年,設寧臺溫水師總兵官及以下各官。康熙九年,設水師提督及左右二路總兵官,七年罷之。設總兵官一人,轄中左右水師三營,兵三千人。春秋二汛,率戰船出洋巡緝。其戰船之數,隨時增改。順治三年,水師左右二營,大小戰船五十二艘。九年,定海鎮左右二營,戰船四十九艘。十四年,水師左右前後四營,戰船二百二艘。康熙元年,水師前左右三營,戰船一百七十三艘。九年,定海鎮中左右三營,戰船八十艘,增設哨船二十艘。歷年裁汰,定為水艍船十二艘,犁繒船七號水艘中艍船一艘,中號犁繒船五艘,沙船七艘,雙篷艍船十三艘,唬船二艘,哨船二十艘。象山城守營,設副將各官,哨船四艘,海口汛兵一百五十人,哨船十艘。雍正四年,裁存四艘。昌石營,設都司等官,汛兵五百六十五人,戰船六艘。鎮海營,原設定海水師左右二營。雍正二年,改設鎮海營參將各官,汛兵二百三十五人,哨船八艘。

臺州府,順治十四年,設寧臺總鎮。十五年,改水師提督。尋改總兵。設黃巖鎮標三營,水師二千七百七十五人,戰哨船二十五艘。海門駐游擊等官。前所駐都司等官。右營分防海洋七汛:玉環山、乾江、雞齊山、標桃嶼、石塘、龍王堂、沙護。中營分防海洋六汛:郎幾山、黃礁門、深門、三山、老鼠嶼、川礁。左營分防海洋八汛:聖堂門、米篩門、白岱門、牛頭門、靖寇門、狗頭門山、茶盤山、迷江山。

溫州府,順治三年,設副將各官。十三年,改總兵官,設鎮標中左右水師三營,戰哨船二十二艘。中營水戰兵六十五人,守兵一百五十二人,戰船九艘,快哨船二艘,釣船三艘。分巡二處:一專防三盤口,水師百六十二人,戰船二艘;一專防長沙海洋,水師一百二十八人,沙戰船二艘。分防汛地凡七:曰霓嶴、黃大嶴、三盤、大門、長沙、鹿西、雙排。左營水戰兵六十八人,守兵一百七十三人,戰船九艘,快哨船二艘。分巡二處:一專防鳳山汛,一專防南龍海洋。分防汛地凡五:曰鳳皇山、銅盤山、南龍山、大瞿山、白腦門。右營轄陸地汛兵。瑞安水師營,設副將各官,水戰兵九十八人,守兵一百四十三人,內洋巡哨戰船四艘,外洋巡哨戰船五艘,快哨船四艘,釣船二艘。分巡二處:一專防北關洋,水師七十人,戰船一艘;一專防官山洋,水師五十人,戰船一艘。分防汛地凡六:曰北關、官山、金鄉嶴、琵琶山、南鹿山、四大嶼。玉環水師營,設參將等官,水戰兵一百四十五人,守兵二百五十四人,八槳船四艘,戰船四艘,快哨船四艘。左營轄陸地汛兵。右營水師一百八十四人,戰船四艘。分巡二處:一專防坎門,水師六十五人,戰船一艘;一專防長嶼,水師三十四人,戰船一艘。內洋凡三汛:曰烏洋、梁灣、黃門。外洋一汛,曰沙頭。左右營率水師一百八十四人,戰船一艘,輪巡洋面。又江口水師一百八十四人,戰船四艘。

雍正二年,額定四種戰船:曰水艍船,曰趕繒船,曰雙篷船,曰快哨船。其六槳船、八槳船,雍正七年後所增設也。

福建水師,順治十三年,始設福建水師三千人,唬船、哨船、趕繒船、雙篷船百餘艘。康熙二十四年,裁撤雙篷船八十艘,以二十艘分防臺灣及澎湖島。雍正三年,於福州、漳州、臺灣三處各設船廠,制造外海內河大小戰船。七年,設泉州船廠,修造各提、鎮、協標水師戰船。福州船廠承修四十六艘。泉州船廠承修四十八艘。漳州船廠承修五十二艘。臺灣船廠承修九十六艘。乾隆十六年,令三江口戰船按季燂洗。三十三年,裁撤哨船五十艘。嘉慶四年,令戰船悉改同安船式。五年,裁撤內地額設戰船三十艘,增造米艇船三十艘,編為勝字號。七年,以福寧府陸路鎮標左營改為水師左營,駐三沙海口,編新字號戰船十二艘。十年,增臺灣水師同安梭船三十艘,編為善字號,分設臺灣協標中左右三營。十一年,增米艇八艘,編為捷字號,又增大橫洋梭船二十艘,分編為集字號十艘,成字號十艘,分防內地。十三年,裁撤中號、小號梭船十七艘。十四年,增集字號、成字號大同安梭船二十艘,捷字號米艇八艘。十五年,裁撤臺灣港口善字號船二十一艘,於鹿耳門增守港師船十六艘,編為知字號,增八槳快船十六艘,編為方字號。十六年,裁撤各營中號、小號梭船三十七艘。道光二年,裁撤捷字號米艇、勝字號米艇共十五艘,餘改為一、二、三號同安梭船之式。七年,裁撤臺灣水師營知字號、方字號船共三十二艘,善字號船九艘,別造白底艍船三十二艘,編為順字號十六艘,濟字號十六艘,分撥臺灣協標中左右三營,澎湖協標艋舺營。

其外海戰船名號凡十類:曰趕繒船,曰雙篷艍船,曰雙篷船,曰平底哨船,曰圓底雙篷船,曰白艕船,曰哨船,曰平底船,曰雙篷哨船,曰平底讌船。內河戰船名號凡九類:曰八槳船,曰六槳平底小巡船,曰花駕座船,曰八槳哨船,曰小八槳船,曰中八槳船,曰大八槳船,曰花官座船,曰哨艍船。各船水師多寡之數,以船之大小為衡。

提督標分中、左、右、前、後五營,中營戰船九艘,左營八艘,右營八艘,前營十艘,後營十艘。總督標水師左營戰船二艘。金門協標後改鎮標,左營戰船九艘,右營九艘,改鎮標後,增戰船二艘。海壇協標後改鎮標,左營戰船十艘,右營八艘。閩安協標左營戰船七艘,右營七艘。福寧鎮標左營戰船十艘。烽火營戰船十一艘。南澳鎮標戰船十艘。銅山營戰船十一艘。臺灣協標中營戰船十九艘,左營十四艘,右營十六艘。澎湖協標左營戰十七艘,右營十六艘,艋舺營十四艘。

廣東水師,自順治九年設官弁千人,嗣設總督標水師,駐肇慶府,分為中、左、右、前、後五營。中營二櫓槳船一艘,急跳船一艘。左營槳船二艘,急跳船一艘,舢板船三艘。右營槳船二艘,急跳船二艘。前營急跳船二艘,舢板船四艘。後營槳船一艘,急跳船一艘,舢板船三艘。水師營二櫓槳船十四艘,四櫓槳船六艘,急跳船六艘。四會營四字號槳船三艘。新會營急跳船一艘,急跳槳船一艘,小舢板船二艘。後改肇慶城守協標,轄左右營、四會營、那扶營、永安營。以新會營改隸提標水師之順德協。

巡撫標轄水師左右營、廣州協左右營、三水營、前山營、順德協左右營、新會左右營、增城左右營、大鵬營、永靖營。光緒二十九年,裁廣東巡撫,以各營分隸提督標及廣州城守協。

水師提督標,康熙元年設,駐惠州府,轄四營。嘉慶後移駐虎門,分中、左、右、前、後五營,香山協左右營,順德協左右營,新會左右營,大鵬左右營,赤溪協左右營,清遠右營,廣海寨營,永靖營。凡六櫓船十一艘,八櫓船四艘,十櫓船二艘,十二櫓船二艘,米艇十一艘,撈繒船六艘,快槳船二十七艘,淺水槳船十二艘,巡船十四艘,二櫓船六艘,四櫓船十二艘,艍船四艘。嗣後裁廣海寨營,以清遠左右營隸三江口協標,以永靖營改隸撫標,又改隸城守協標,增設赤溪左右營。

南澳水師鎮標,左營戰船十艘,屬福建省,右營趕繒船九艘,仔船六艘,八槳船二艘。澄海協,左營艍船二艘,膨仔船二艘,烏船一艘,快槳船三艘;右營趕繒船一艘,艍船二艘,仔船一艘,烏船一艘,快槳船二艘。海門營趕繒船二艘,艍船二艘,仔船四艘,快槳船四艘。達濠營艍船二艘,仔船一艘,快槳船一艘。

碣石水師鎮標,康熙八年展界,分中左右三營,米艇十艘,哨船一艘。平海營,康熙元年,以惠州協右營駐平海所,雍正四年,設平海營,隸鎮標,一號趕繒船一艘,二、三、四號艍船三艘,五、六、七、八號拖風船四艘,一號快船一艘。歸善城守營,舢板哨船十三艘。惠來營,屬陸路。潮州鎮標,分中左右三營。城守營快船五艘。饒平營快船四艘。黃岡協左右營,左營哨船二艘,右營哨船二艘。

北海鎮標及城守營,康熙初年設。二十三年,改設龍門水師協標,分左右二營,左營水師八百二十三人,右營八百十一人,共大米艇三艘,中米艇四艘,小米艇一艘,撈繒船三艘,艍船一艘。乾隆二十年後,實存趕繒船二艘,艍船四艘,拖風船一艘,快馬船三艘。舊轄有硇州營,大小戰船二十七艘,後改隸高廉水師鎮標。

高廉鎮標陽江營,嘉慶十五年,以南韶連鎮標左翼兵移駐陽江,設陽江鎮標,左營大米艇五艘,撈繒船二艘,右營大米艇三艘,撈繒船一艘,後改隸高廉鎮標。電白營雙篷艍船七艘。吳川營外海雙篷艍船二艘,外海拖風船三艘,槳船二艘。硇州營舊為乾體營,大戰船十三艘,龍艇六艘,哨船五艘。康熙四十二年,改為硇州營,存趕繒船三艘,艍船六艘,拖風船十二艘,外海雙篷船四艘,快槳船七艘。東山營大米艇一艘,撈繒船二艘。

雷瓊鎮標,康熙二十七年設,分左右二營,趕繒船二艘,艍船六艘,快哨船六艘。雍正間,增快哨船十艘。嘉慶十五年,改稱水師營,左營水師八百七十六人,右營水師八百八十八人。海安營,康熙初年,設副將各官。八年,改設游擊,隸鎮標,大小哨船凡二十艘。白鴿寨營,順治初年,設參將各官,大小哨船九艘。康熙間裁撤,存哨船三艘。海口營,嘉慶十五年,設水師協標,左營水師四百九十二人,右營四百八十五人,後改參將,並左右營為一營。崖州水師協標,中營屬陸路,右營水師一、二、三號拖風哨船三艘,四、五、六號艍船三艘。

又廣東駐防八旗營水師,乾隆十年,設領催等三十人,水師四百七十人,分左右二營,匠役十二人,教習副工兵百人。

廣西水師,舊設駐柳州,後移駐龍州。康熙二十一年,以梧州地居兩廣之中,扼三江之要,分額設弁兵之半,於潯、南一帶,設哨船巡防。其後惟梧州、潯州、平樂、南寧、慶遠各府有經制水師,為數無多。

至光緒初年,以漓江、左江、右江水程綿亙,盜賊充斥,設水師五營。嗣因餉絀,並為三營。旋增募勇丁,凡巡哨船一百四十艘,兵丁一千三百餘人。仍苦不敷分布,乃復設水師五軍,以水程之長短,定師船之多少。自桂林府至平樂府,為中軍汛地,設將領四人,巡船四十艘,兵五百人。自梧州府至潯州府,為前軍汛地,設將領二人,巡船二十艘,兵三百五十二人。自太平府至南寧府,為左軍汛地,設將領三人,巡船三十艘,兵三百七十六人。自慶遠府至武宣,為右軍汛地,設將領四人,撥車扒船四艘,巡船三十六艘,兵五百三十六人。自南寧府至百色等河面,為後軍汛地,設將領三人,扒船八艘,巡船二十艘,兵四百二十四人。此光緒季年之制也。

其舊設水師弁兵船數列後:梧州府水師三營,設副將各官,水師千人,塘船十三艘,快船六艘,舢板船三十八艘。慶遠府協標左營,兼轄水師哨船二艘。平樂府水師哨船四十七艘。廣運營八槳哨船七艘,柳兵哨船七艘。大亮營八槳哨船一艘,柳兵哨船一艘。大定營八槳哨船一艘,柳兵哨船二艘。足灘營柳兵哨船十二艘。潯州府左營,兼轄來賓江口水師哨船,勒馬汛水師哨船。南寧府隆安縣水塘十八處,哨船十五艘,水師一百四十人,橫州水塘二十處,哨船三艘,水師三十四人。永淳縣水塘九處,哨船一艘,水師十人。

湖北水師,武昌府城守營,舊有水師營,設守備以下各官。乾隆二年,撥入漢陽營,任江、漢巡防之責。武昌省城,存城守營內河巡哨船五艘,下游道士洑營巡江船三艘。漢陽城守營兼轄水師營,戰船三艘,虎戰船一艘,漢川虎戰船二艘。黃州協營,巡江船三艘。蘄州城守營,巡江船二艘。荊州水師營,設守備以下各官,戰船二十五艘,巡江船二艘。宜昌府水師,順治十三年設彞陵鎮,轄水師前後二營。康熙十九年,改為彞陵水師協標。乾隆元年,改為宜昌鎮標,仍設水師前後二營,戰船三十艘,小船十一艘。經粵寇之亂,舊制無存。同治間,設長江水師。其屬湖北省者,為漢陽水師鎮標,轄漢陽營、田鎮營、簰洲營、巴河營。其戰船、兵額,與各省長江水師同制。

江西水師,清初設九江鎮標水師營,南湖水師營、鄱湖水師營,唬船二十艘,分防水巡,各營設塘船一艘。康熙元年,改九江鎮標為九江協標,水師七百七十三人,增設沙船三十艘,水汛巡哨船十七艘。乾隆間,實存沙船八艘,唬船二十三艘。後改為城守營。同治八年,裁撤城守營。其南湖水師營、鄱湖水師營,自設長江水師後,亦皆裁撤。長江水師之屬於江西省者,為湖口水師鎮標,轄湖口營、吳城營、饒州營、華陽營、安慶營,戰船、兵額,與各省長江水師同制。

安徽省水師,安慶鎮標、壽春鎮標及游兵營、泗州營,均有戰船。順治初年,安慶鎮標游兵營隸操江巡撫標。康熙元年,改隸江南總督標。泗州營舊隸江南提督標,後改隸安徽巡撫標。安慶鎮標,分防懷寧、桐城、望江、東流、貴池、銅陵及江西彭澤縣等處,大唬船一艘,小唬船二十二艘。游兵營,分防和州、無為、含山、銅城、繁昌、蕪湖、當塗等處及江蘇之江寧縣,大唬船一艘,小唬船二十七艘。壽春鎮標,潁州營哨船二艘,泗州營扒唬船四艘。經粵寇之亂,師船盡毀。同治間,設長江水師,屬安徽省者,為長江提督標中營,駐太平府,轄裕溪營、蕪湖營、大通營、金陵營,戰船、兵額,與各省長江水師同制。

湖南水師,清初設辰州、洞庭二營。康熙二十八年,裁辰州水師,改設嶽州水師營,歸岳州營參將兼轄,設守備各官,頭舵戰兵六十八人,水步戰兵六十五人,水守兵一百四十八人,分防岳州府城及東西湖、上下江二汛。自雍正至嘉慶,迭有增減,存頭舵戰兵三十四人,水步戰兵三十九人,水守兵一百四十二人,戰船十八艘。

洞庭水師營,原設洞庭協標。嘉慶二年,以洞庭副將、都司移駐常德,改常德為協。以常德游擊、守備移駐洞庭,改洞庭協為水師營,設游擊各官,戰兵一百九人,守兵四百三十六人,戰船十二艘,分防小船、游巡小船各十艘,分駐龍陽縣及東西湖各汛。承平日久,將弁兵丁,咸居陸地,船敝不修,舊制浸廢。

咸豐三年,曾國籓治水師於湖南,造船練兵,以長龍船、舢板船尤為便利。粵寇定後,至同治八年,裁撤水勇,設長江水師。在湖南境者,設嶽州鎮標四營,為岳州營、沅江營、荊州協標營、陸溪營。原設之岳州水師,歸並岳州城守營。原設洞庭水師,歸並龍陽城守營。

咸豐軍興以後,常於省城駐水師二營,湘潭駐水師一營,衡州駐水師一營,益陽縣則由省城撥師船駐防,常德駐水師一營,辰州駐一營,靖州之洪江駐一營,澧州則由常德撥師船駐防,又於岳州、安鄉合駐水師一營,不在經制水師之列,而分地駐巡,參錯布置,實與經制水師相輔云。

長江水師,道光季年,各省內河水師及沿江水師,船多朽敝,值操練之期,虛衍儀式。粵寇東犯,無以制之。咸豐三年,江忠源始建制艦練兵之議。四年,命侍郎曾國籓治水師於衡州,造拖罟、快解、長龍、舢板各船,惟舢板船尤為輕捷制勝,長龍船次之。大率水師一營,設長龍船一、二艘,舢板船或十餘艘,或二十餘艘,以拖罟船、快解船守營,不以出戰。其後水師日增,悉廢拖罟、快蟹舊式之船,專以舢板船摧敵。任彭玉麟、楊岳斌為水師統帥,循長江轉戰東下,克名城以百計,踣巨憝於金陵。

同治三年,東南底定,曾國籓、彭玉麟以江防重要,疏請設立長江經制水師。簡授長江水師提督一人,得專摺奏事,隸兩江、湖廣總督節制,率提標五營駐安徽太平府。每歲於所轄湖南、湖北、江西、安徽、江南五省江面巡閱。設嶽州、漢陽、瓜洲、湖口四總兵官。每鎮標各統水師四營,惟湖口鎮標五營,以狼山鎮標水師二營並隸之,凡二十四營。總兵及參將、游擊,於收泊戰艦處所立汛建署,為營汛治事之地。以船為家,不得在署常居。都司、守備各官以至兵丁,不得陸居。

總兵座船三艘,督陣舢板二艘,親兵十二人。副將座船二艘,督陣舢板一艘,長龍二艘,親兵十二人。游擊座船二艘,督陣舢板一艘,長龍一艘,親兵十二人。都司二人,各座船一艘,長龍一艘。守備二人,各座船一艘,舢板一艘,飛劃一艘。四哨千總八人,各座船一艘,舢板一艘,飛劃一艘。四哨把總九人,各座船一艘,舢板一艘,飛劃一艘。四哨外委十一人,各座船一艘,舢板一艘,飛劃一艘。又外委一人,管帶督陣舢板,有座船一艘,無舢板。戰船之大者,每艘或設兵二十人,為舵兵一人,頭兵一人,砲兵二人,槳兵十六人;或設兵二十五人,為舵兵一人,艙兵一人,頭兵一人,砲兵四人,槳兵十八人。舢板船每艘設兵十四人。

總兵以下各官,設稿書、書識,自七人至一人不等。以都司一人管駕長龍船為領哨,守備為副領哨。每哨戰船十艘。惟岳州、漢陽系游擊營制,而統戰船三十三艘,視參將例。左哨都司專任錢糧,右哨都司專任船砲軍械及巡查諸務。

大小戰船咸設砲位。長龍船千斤頭砲二位,七百斤邊砲四位,艘砲一位。舢板船八百斤頭砲一位,六、七百斤哨砲一位,船邊五十斤轉珠小砲二位。洋槍刀矛之屬,隨宜分配。旗幟以桅旗為主,懸方式長龍旗,凡長一丈二尺。舢板船旗長九尺,船艄懸尖式龍旗,書某標某營某哨。桅上小旗,或船首立旗,書駕船將弁之姓,以示區別。

凡駐師之處,漁船由水師編號稽查,以清盜源。其疏防之責,以哨官為專汛,營官為本轄,遇有盜劫,視汛地所轄題參。江、鄂各營,半年更調一次。副將與副將之營互調,參將、游擊與參將、游擊之營互調。每營調居客汛二次,又調回本汛一次,如承緝盜案未獲,則不得更調。

凡副將、參將以下,由本境巡撫節制,總兵由總督節制。土匪猝發,須用戰船,由督撫檄調境內水師往剿。總兵奉檄即發兵。督撫調水師操練,亦奉檄即行。其事涉重大者,督撫會同長江提督疏陳。其餘水營政務,由長江提督主持。

餉糈之制,將弁則視其職以定廉養公費。兵丁月餉,每名銀三兩有差。全軍餉糈,由沿江釐捐局指定支撥。

設火藥局於湖北、安徽,購硝斤於江蘇、江西、湖南。設子彈局於湖南之長沙。設造船廠於湖北之漢陽,江西之吳城,江南之草鞋夾。戰船均三年一修,十二年更換。

定水師事宜三十條,未盡者續定十條。銀米有稽,銓補有章,訓練有規。鄭重江防,嚴申禁約,有犯必懲。自荊州以達海門,沿江數千里,稱天塹雄師。至光緒季年,特命大臣查閱長江營伍,實存長龍、舢板戰船七百六十二艘,飛劃船六百四十二艘,水師弁兵一萬有七十九人。

其自荊州以上,溯江至宜昌、巴東,漢陽以上,溯江至襄陽、鄖陽,湖南之湘江、沅江,江西之吳城,以上諸河,各疆吏自設防營。其淮河一帶,自正陽關至洪澤湖,及江蘇境各支河水師,隸淮陽鎮標,光緒間,改設江北提督。凡清江營、洋河營、廟灣營、佃湖營、洪湖營、葦蕩營咸隸之。自鎮江以東,內河各汛及太湖水師五營,則統以江南提督。凡各省內河有水師者,悉改舊式,一準長江水師。其海口原有之狼山鎮、福山鎮,仍如前制,由鎮將督率大號戰船,巡防內海。惟狼山鎮兼隸長江水師提督,每營設大舢板船二十艘,並仿紅單、拖罟船式,設大號戰船數艘,多置砲位,為巡緝內洋之用。其長江水師營制防汛列後:

岳州設總兵官,置中軍中營游擊,戰船三十三艘,仿參將營之例,分防自城陵磯至鹿角、壘石、瀘陵潭、湘陰一帶。沅江設參將,屬岳州鎮左營,分防君山、西湖及常德、龍陽、華容等河通洞庭湖之處。其沅、湘等水汛,由湖南省別行設防。荊州設副將,屬岳州鎮後營,分防自荊州以下江面,石首、監利一帶,至荊河口止。陸溪口設游擊,屬岳州鎮前營,分防自荊州河以下江面,螺山、新堤及倒口內之黃蓋湖。

漢陽設總兵官,置中軍中營游擊,戰船三十三艘,仿參將營之例,分防自沌口以下江面,至團風等處,並防省城兩岸,後湖、青林湖。其漢水上通樊城千餘里及各河汊,由湖北省別行設防。簰洲設參將,屬漢陽鎮後營,分防自倒口以下江面至沌口,兼防金口以內之斧頭湖。巴河設游擊,屬漢陽鎮右營,分防自團風以下江面,黃州、蘭溪至道士洑,兼樊口以內之梁子湖。田家鎮設副將,屬漢陽鎮前營,分防自道士洑以下江面,湋源口、蘄州、武穴至陸家嘴,兼防湋源口及隆平以內之湖。

湖口設總兵官,置中軍中營游擊,分防自陸家嘴以下江面,至九江老洲頭。吳城設參將,屬湖口鎮左營,分防自湖口以內姑塘、南唐、渚磯一帶。饒州設參將,屬湖口鎮後營,分防都昌、鄱陽、康山一帶。其彭蠡湖東境各湖,南達省城贛江,由江西省別行設防。華陽鎮設游擊,屬湖口鎮右營,分防自老洲頭以下江面,彭澤縣、香口至東流等處,兼防吉水溝以內各湖。安慶府設副將,屬湖口鎮前營,分防自東流以下江面,黃石磯、李陽河至樅陽,兼防北岸鹽河及樅陽以下,南岸通殷家匯之河。

太平府設長江水師提督衙署,置中軍中營副將,分防金柱關以下江面至烏江。大通設參將,屬提標後營,分防自樅陽以下江面,池州土橋至荻港。蕪湖設游擊,屬提標右營,分防自荻港以下江面至裕溪口,並灣沚、青弋江等處。裕溪口設參將,屬提標左營,分防東西梁山江面至金柱關,兼防運漕、無為州各內河,及巢湖百餘里水汛。金陵草鞋夾設參將,屬提標前營,分防烏江以下江面至通江集,兼防江浦、六合內河。

瓜洲設總兵官,置中軍中營游擊,分防通江集以下江面至焦山,兼防內河至揚州。自揚州以上,高郵等湖,由淮揚鎮別行設防。孟河營設游擊,屬瓜洲鎮右營,分防南岸各夾江,自焦山至江陰口。其南岸內河,由松江提標別行設防。三江營設游擊,屬瓜洲鎮左營,分防北岸各夾江,自焦山至靖江口。其北岸內河,由淮揚鎮別行設防。江陰設副將,屬瓜洲鎮前營,分防自江陰以下江面,而至鹿苑港及壽興等河。其鹿苑港以下,由福山鎮標接防。

狼山鎮總兵,循舊日之制,增水師二營,兼隸長江提督。原統中左右三營,鹽捕、揚州、三江、泰州、泰興、掘港各營,悉仍其舊。惟通州設綏通營,置游擊各官,分防自靖江八團港以下江面至通州,凡長龍戰船二艘,督陣舢板一艘,舢板十艘,大舢板十艘,仍酌增紅單、拖罟等船。海門設綏海營,置副將各官,分防自狼山至海門北岸江口海汊,凡長龍戰船二艘,督陣舢板二艘,大舢板二十艘,仍酌增兵輪船,及紅單、拖罟等船。其崇明南岸海汊,由江南提督別行設防。

綜長江經制水師,副將六營,參將七營,游擊十一營,凡二十四營。


\end{pinyinscope}