\article{志八十八}

\begin{pinyinscope}
○選舉八

△新選舉

新選舉制,別於歷代取士官人之法。清季豫備憲政,仿各國代議制度,選舉議員,博採輿論。議員選舉有二:曰資政院議員選舉,曰各省諮議局議員選舉。自辛丑回鑾,朝廷銳意求治,派大臣赴各國考察政治,設考察政治館。命甄擇各國政法,斟酌損益,候旨裁定。光緒三十二年七月,詔曰:「考察政治大臣載澤等回國陳奏,國勢不振,由於上下相暌,內外隔閡;而各國所以富強,在實行憲法,取決公論。今日惟有仿行憲政,大權統於朝廷,庶政公諸輿論,廓清積弊,明定責成,以豫備立憲基礎。俟規模初具,妥議立憲實行期限。各省將軍、督、撫曉諭士庶人等,各明忠君愛國之義,合群進化之理,尊崇秩序,保守和平,豫備立憲國民之資格。」九月,慶親王奕劻等遵旨核議釐定官制,以「立憲國官制,立法、行政、司法三權並峙,各有專屬,相輔而行。立法當屬議院,今日尚難實行。請暫設資政院,以為豫備」。詔如所議。

三十三年,改考察政治館為憲政編查館。八月,諭曰:「立憲政體,取決公論,中國上、下議院未能成立,亟宜設資政院,以立議院基礎。派溥倫、孫家鼐為資政院總裁,妥擬院章,請旨施行。」尋諭:「各省應有採取輿論之所,俾指陳通省利病,籌計地方治安,並為資政院儲才之階。各省督、撫於省會速設諮議局,慎選公正明達官紳,創辦其事。由各屬合格紳民,公舉賢能為議員。斷不可使品行悖謬、營私武斷之人濫廁其間。凡地方應興應革事宜,議員公同集議,候本省大吏裁奪施行。將來資政院選舉議員,由該局公推遞升。」

三十四年六月,資政院奏言:「立憲國之有議院,所以代表民情,議員多由人民公舉。凡立法及豫算、決算,必經議院協贊,方足啟國人信服之心。大學云:『民之所好好之,民之所惡惡之。』孟子云:『所欲與聚,所惡勿施。』又云:『樂以天下,憂以天下。』皆此理也。昔先哲王致萬民於外朝,而詢國危國遷,實開各國議院之先聲。日本豫備立憲,於明治四年設左、右院,七年開地方會議,八年立元老院,二十三年遂頒憲法而開國會。所以籌立議院之基者至詳且備。謹旁考各國成規,揆以中國情勢,酌擬院章目次,凡十章。先擬就總綱、選舉二章呈覽。」報可。

是月憲政編查館會同資政院擬訂各省諮議局章程,並議員選舉章程。奏言:「立憲政體之要義,在予人民以與聞政事之權,而使為行政官吏之監察。東、西立憲各國,雖國體不同,法制各異,無不設立議院,使人民選舉議員,代表輿論。是以上下之情通,暌隔之弊少。中國向無議院之說,今議倡設,人多視為創舉。不知虞廷之明目達聰,大禹之建鞀設鐸,洪範之謀及庶人,周官之詢於外朝,古昔盛時,無不廣採與論,以為行政之準則,特未有議院之制度耳。今將創設議院,若不嚴定規則,事為之制,曲為之防,流弊不可勝言。中國地大民眾,分省而治。各省之政,主於督、撫,與各國地方之治直接國都者不同。而郡縣之制,異於封建,督、撫事事受命於朝廷,亦與各國聯邦之各為法制者不同。諮議局為地方自治與中央集權之樞紐,必使下足裒集一省之輿論,上仍無妨國家統一之大權。此日各省諮議局辦法,必須與異日京師議院辦法有相成而無相悖。謹仰體聖訓,博考各國立法之意,兼採外省所擬章程,參伍折衷,擬訂各省諮議局章程,別為選舉章程一百十五條,候欽定頒行。」詔飭各督、撫迅速舉辦,實力奉行,限一年內一律辦齊。並諭曰:「朝廷軫念民依,使國民與聞政事。先於各省設諮議局,以資歷練。凡我士庶,當共體時艱,同攄忠愛。於地方應興應革之利弊,切實指陳。於國民應盡之義務,應循之秩序,竭誠踐守。各督、撫當本集思廣益之懷,行好惡同民之政,虛衷審察,惟善是從。至選舉議員,尤宜督率有司,認真監督,精擇慎選。憲政編查館、資政院迅將君主立憲大綱,暨議院選舉各法,擇要編輯。並將議院未開以前應籌備各事,分期擬議具奏。俟親裁後,即將開設議院年限,欽定宣布。」

八月,憲政編查館、資政院會奏遵擬憲法議院選舉法綱要,暨議院未開以前逐年籌備事宜。自本年起,分九年籌備。其關於選舉議員者,第一年各省籌辦諮議局,第二年舉行諮議局選舉,各省一律成立,頒布資政院章程,舉行資政院選舉。第三年召集資政院議員舉行開院。第九年始宣布憲法,頒布議院法,暨上、下議院議員選舉法,舉行上、下議員議員選舉。諭令京、外各衙門依限舉辦。

先是資政院奏擬院章目次,第二章為選舉。宣統元年七月,資政院奏續擬院章,改訂第二章目次為議員,專詳議員資格、額數、分類、任期,而另定選舉詳細章程,以免混淆,從之。院章規定資政院議員資格,由下列各項人員年滿三十歲以上者選充。一,宗室王、公世爵;二,滿、漢世爵;三,外籓王、公世爵;四,宗室、覺羅;五,各部、院四品以下、七品以上官,惟審判、檢察、巡警官不與;六,碩學通儒;七,納稅多額人;八,各省諮議局議員。定額:宗室王、公世爵十六人,滿、漢世爵十二人,外籓王、公世爵十四人,宗室、覺羅六人,各部、院官三十二人,碩學通儒十人,納稅多額者十人。各省諮議局議員一百人。類別為欽選、互選。宗室王、公世爵,滿、漢世爵,外籓王、公世爵,宗室、覺羅,各部、院官,碩學通儒,納稅多額者,欽選。各省諮議局議員互選。任期三年,任滿一律改選。

九月,資政院會奏資政院議員選舉章程,疏言:「資政院議員選任之法,大別為欽選、互選二者,各有取義。而欽選議員名位有崇卑,人數有多寡,當因宜定制,取便推行。宗室王、公世爵,滿、漢世爵及外籓王、公世爵,階級既高,計數較少,應開列全單,恭候簡命。宗室、覺羅,各部、院官及納稅多額者,合格人數,與議員定額比例,多少懸殊。考外國上院制,敕任議員多經互選。擬略師其意,於欽選之前,舉行互選。各照定額,增列多名。好惡既卜諸輿情,用舍仍歸於宸斷。其碩學通儒,資格確定較難,人數調查不易,互選勢所難行。擬略仿從前保薦鴻博之例,寬取嚴用,以蒐訪之任,寄諸庶官。抉擇之權,授諸學部。仍寬定開列名數,冀不失欽選之本旨。以上各項,略採各國上院辦法,為建設上議院之基礎。而資政院兼有下院性質,不能無民選議員,與欽選相對待。特以諮議局為資政院半數議員之互選機關,諮議局議員本由各省合格紳民衣復選而來,則諮議局公推遞升之資政院議員,即不啻人民間接所選舉。公推遞升之標準,不能不以得票多寡為衡。但監督權屬於督、撫,非經覆定,不令遽膺是選。既與欽選大權示有區別,自與下院要義不相背馳。」詔如所議行。

資政院議員選舉章程之規定,宗室王、公世爵,列爵凡十二:一,和碩親王;二,多羅郡王;三,多羅貝勒;四,固山貝子;五,奉恩鎮國公;六,奉恩輔國公;七,不入八分鎮國公;八,不入八分輔國公;九,鎮國將軍;十輔國將軍;十一,奉國將軍;十二,奉恩將軍。按院章定額分配,自和碩親王至奉恩輔國公十人,自不入八分鎮國公至奉恩將軍六人。滿、漢世爵,以滿洲、蒙古、漢軍旗員及漢員三等男以上以之爵級為限,按定額分配。三等侯以上八人,一等伯至三等男四人。外籓王、公世爵,凡下列蒙古、回部、西藏各爵:一,汗;二,親王;三,郡王;四,貝勒;五,貝子;六,鎮國公;七,輔國公。按定額分配。內蒙古六盟,盟各一人;外蒙古四盟,盟各一人;科布多及新疆所屬蒙古各旗一人;青海所屬蒙古各旗一人;回部一人;西藏一人。凡各項世爵年滿三十歲以上,未奉特旨停止差俸,及因疾病或事故自請開去一切差使者,均得選充資政院議員。每屆選舉,資政院於前一年九月行知宗人府、各該管衙門、理籓部,分別查明合格者,造具清冊,於選舉年分二月以前,咨送資政院。由院分別開單,於三月以前,奏請按額欽選。其宗室王、公,滿、漢世爵,現任軍機大臣,參豫政務大臣,及資政院總裁、副總裁者,無庸選充。有缺額時,資政院隨時行知各該衙門,修正清冊。按爵級或部落應選充者,奏請欽選補足之。

宗室、覺羅,凡男子年滿三十歲以上,無下列情事者,得選充資政院議員:一,曾處圈禁或發遣者;二,失財產上信用被人控實未清結者;三,吸食鴉片者;四,有心疾者;五,不識文義者。其現任三品以上職官,審判、檢察、巡警官,及現充海、陸軍軍人者,無庸選充。按定額分配,宗室四人,覺羅二人,由各該合格人先行互選。於選舉年分二月初一日,在京師及奉天府行之。京師以宗人府堂官為監督,奉天以東三省總督為監督。每屆互選,資政院於前一年九月行知互選監督,照章舉行。設互選管理員,掌調查互選人,管理投票、開票、檢票等事宜。由互選管理員查明合格人員,造具互選人名冊,先期呈由互選監督宣示公眾。如本人認為錯誤遺漏,得於宣示期內,呈請互選監督更正補入。經批駁者,不得瀆請。互選選舉人及被選舉人,均以列名互選人名冊者為限。屆期互選監督應親蒞投票所,或派員監察之。互選人應親赴投票所自行投票,用記名單記法。互選人有因職務或因疾病、事故不能親赴投票者,得就互選人內委託一人代行投票,應由本人親書密封署名畫押,連同委託憑證,送致受託人。該受託人應將密封及委託憑證臨時向互選監督呈驗,方許代投。以得票較多數者為當選。互選當選人額數,各以議員定額之十倍為準。互選告竣,互選監督即日將當選人名榜示投票所。不原應選者,得於三日內呈明互選監督撤銷,將得票次多數者補入。互選管理員造具當選人名冊,連同票紙,呈由互選監督咨送資政院,由院將當選人名及得票數目,於選舉年分三月以前,奏請按額欽選。有缺額時,資政院隨時將本屆當選人開單奏請欽選補足之。本屆當選人數不足議員缺額之三倍時,應舉行臨時互選,一切照尋常互選辦理。

各部、院官,以下列各官為限:一,內閣侍讀學士以下,中書以上;二,翰林院侍讀學士以下,庶吉士以上;三,各部左、右參議以下,七品小京官以上;四,掌印給事中、給事中及監察御史。各官以年滿三十歲以上,具下列資格之一,得選充資政院議員:一,現任實缺者;二,曾任實缺未休致、革職者;三,奉特旨署理或奏署者;四,奉特旨候補、補用、選用或學習行走者;五,其餘候補滿三年以上者。由合格人先行互選,於選舉年分二月初一日在京師行之,以都察院堂官為監督。互選當選人額數,以議員定額之五倍為率,各部、院官選充資政院議員者,於院內職權,本衙門長官不得干涉。其因升轉降調致失原定資格者,即同時失資政院議員之資格。所有舉行互選、奏請欽選、補足缺額各辦法,與宗室、覺羅選舉同。

碩學通儒資格凡四:一,不由考試、特旨賞授清秩者;二,著書有裨政治或學術者;三,有入通儒院之資格者;四,充高等及專門學堂主要科目教習五年以上著有成績者。凡年滿三十歲以上,具前列資格之一,均得選充資政院議員。每屆選舉,資政院於前一年九月行知學部,由部通行京堂以上官、翰林、給事中、御史、各省督、撫、提學使、出使各國大臣,各蒐訪一人或二人,開具事實,保送學部審查。擇定合格得保多者三十人,作為碩學通儒議員之被選人。於選舉年分二月以前,咨送資政院。由院將被選人姓名及原保人姓名官職開單,於三月以前,奏請按額欽選。有缺額時,資政院隨時將本屆被選人照章奏請欽選補足之。本屆被選人數不足議員缺額之三倍時,應另行保送。

納稅多額人,以下列資格為限:一,男子照地方自治章程有選民權者;二,年納正稅或地方公益捐,在所居省分占額較多者。凡具此資格,年滿三十歲以上,得選充資政院議員。由合格人先行互選,於選舉年分二月初一日在各省城行之,以布政使或民政使為監督。每屆互選,資政院於前一年九月行知各省督、撫,照章舉行。互選監督會同商務總會總理、協理,遴派互選管理員。互選辦法與普通互選同。互選人額數。每省以二十人為限。投票用記名連記法,以得票過互選人數三分之一者為當選。互選當選人額數,以互選人額數十分之一為率。如當選人不足定額,就得票較多者,令互選人再行投票,以足額為止。其得票及格、額滿見遺者,作為候補當選人。當選人不原應選,得呈明互選監督撤銷,以候補當選人依次遞補。互選管理員造具當選人及候補當選人名冊,連同票紙,呈由互選督申送本省督、撫,各督、撫將當選人姓名及得票數目咨送資政院,由院開單,於三月以前,奏請按額欽選。有缺額時,資政院隨時將本屆當選人開單奏請欽選補足之。本屆當選人不足議員缺額之三倍時,以候補當選人遞補。候補當選人數不敷時,舉行臨時互選。

各省諮議局互選諮政院議員,按定額分配:奉天三人,吉林二人,黑龍江二人,順直九人,江蘇七人,安徽五人,江西六人,浙江七人,福建四人,湖北五人,湖南五人,山東六人,河南五人,山西五人,陜西四人,甘肅三人,新疆二人,四川六人,廣東五人,廣西三人,雲南四人,貴州二人。互選於選舉年分前一年十月十一日,在各省諮議局行之。以督、撫為監督。每屆互選,資政院於前一年九月行知各互選監督,照章舉行。屆期互選監督親蒞監察之。投票、開票、檢票等事,由諮議局辦事處管理。適用普通互選規則,互選選舉人及被選舉人均以該省諮議局議員為限。投票用記名連記法,以得票過互選人半數者為當選。互選當選人額數,以各該省議員額數之二倍為率。如當選人不足定額,就得票較多者,令互選人再行投票,以足額為止。其投票及格、額滿見遺者,作為候補當選人。諮議局辦事處造具當選人及候補當選人名冊,連同票紙,呈送互選監督,覆加選定,為資政院議員。不原應選者,得呈明互選監督辭退,依次將本屆當選人及候補當選人覆加選定補充。不敷選充者,舉行臨時互選。選定後,由互選監督造具名冊,連同當選人及候補當選人原冊,咨送資政院。凡選充資政院議員者,不得兼充本省諮議局議員,有缺額時,由院行知該省督、撫,覆加選定補充,或舉行臨時互選。此資政院議員欽選、互選辦法之概要也。

各省諮議局議員選舉章程之規定,議員之選任,用衣復選舉法。衣復選之別於單選者,單選逕由選舉人投票選出議員,衣復選則先由選舉人選出若干選舉議員人,更令選舉議員人投票選出議員是也。諮議局議員定額,因各省戶口尚無確實統計,參酌各省取進學額及漕糧多寡以定準則。奉天五十名,吉林三十名,黑龍江三十名,順直百四十名,江寧五十五名,江蘇六十六名,安徽八十三名,江西九十七名,浙江百十四名,福建七十二名,湖北八十名,湖南八十二名,山東百名,河南九十六名,山西八十六名,陜西六十三名,甘肅四十三名,新疆三十名,四川百零五名,廣東九十一名,廣西五十七名,雲南六十八名,貴州三十九名。京旗及各省駐防,以所住地方為本籍。但旗制未改以前,京旗得於順直議員定額外,暫設專額十名;各省駐防得於該省議員定額外,每省暫設專額一名至三名。選舉權之規定,用限制選舉法。凡屬本省籍貫之男子,年滿二十五歲以上,具下列資格之一者,有選舉諮議局議員之權:一,在本省地方辦理學務及公益事務滿三年以上著有成績者;二,在本國或外國中學堂及與中學同等或中學以上之學堂畢業者;三,有舉、貢、生員以上之出身者;四,曾任實缺職官文七品、武五品以上未被參革者;五,在本省地方有五千元以上之營業資本或不動產者。凡非本籍之男子,年滿二十五歲,寄居本省滿十年以上,有萬元以上之營業資本或不動產者,亦得有選舉權。被選舉權之規定及其限制:凡屬本省籍貫或寄居本省滿十年以上之男子,年滿三十歲以上者,得被選舉為諮議局議員。凡有下列情事之一者,不得有選舉權及被選舉權。一,品行悖謬、營私武斷者;二,曾處監禁以上之刑者;三,營業不正者;四,失財產上信用被人控實未清結者;五,吸食鴉片者;六,有心疾者;七,身家不清白者;八,不識文義者。其有所處地位不適於選舉議員及被選舉為議員者:一,本省官吏或幕友;二,軍人;三,巡警官、吏;四,僧、道及宗教師;五,學堂肄業生:均停其選舉權及被選舉權。其現充小學教員者,停其被選舉權。諮議局設議長一,副議長二,用單記投票法,分次互選。設常駐議員,以議員額數十分之二為額,用連記投票法,一次互選。凡議員三年一改選,議長、副議長任期同。常駐議員任期限一年。議長因事出缺,以副議長遞補。副議長出缺,由議員互選充補。議員出缺,以衣復選候補當選人依次遞補。議員改選,再被選者得連任,以一次為限。議員非因下列事由,不得辭職:一,確有疾病,不能擔任職務者;二,確有職業,不能常駐本省境內者;三,其餘事由,經諮議局允許者。

凡選舉區域,初選舉以、州、縣為選舉區,衣復選舉以府、直隸、州為選舉區。直隸、州及府之本管地方,均作為初選區。直隸無屬縣者,以附近之府為衣復選區。初選區,以同知、通判,州、縣以知州、知縣為初選監督。衣復選區,府以知府,直隸、州以同知、通判、知州為衣復選監督。府、直隸、州作為初選區者,得遴派教佐員為初選監督。初選、衣復選均設投票、開票、管理員、監察員若干名。管理員不拘官紳,監察員以本地紳士為限。初選區選舉人名冊及當選人姓名票數,由初選監督申報衣復監督;衣復選當選人姓名票數,由衣復選監督申報督、撫,分別咨報資政院、民政部立案。

選舉年限,三年一次,以正月十五日為初選日期,三月十五日為衣復選日期。凡初選舉,初選監督按地方廣狹、人口多寡、分劃本管區域為若干投票區,分設選舉調查員,按照選舉資格,詳細調查,將合格選舉人造具名冊,於選舉期六個月以前,呈由衣復選監督申報督、撫,並宣示公眾。如本人認為錯誤遺漏,得於宣示期內呈請初選監督更正。初選當選人額數,按照議員定額加多十倍。各初選區應出當選人若干名,由衣復選監督分配。投票用無名單記法,其有寫不依式者,夾寫他事者,字跡模糊者,不用頒發票紙者,選出之人不合被選資格者,作為廢票。以本區應出當選人額數除選舉人總數,所得半數,為當選票額。得票不滿當選票額以上者,不得為初選當選人。衣復選由初選當選人齊集衣復選監督所在地行之。衣復選當選人,即為諮議局議員。各衣復選區應得議員若干名,由督、撫按全省議員定額分配,投票當選,一切與初選同。

關於選舉之變更,如選舉人名冊有舞弊、作偽情事,或辦理不遵定章,被控判定確實者,初選、衣復選均無效。當選議員有辭任、或疾病不能應選,或身故,或被選資格不符,當選票數不實,被控判定確實者,其當選無效,各以候補當選人遞補。如選舉人確認辦理人員不遵定章,有舞弊、作偽證據,或當選人被選資格不符,當選票數不實,及落選人確信得票可當選而不與選,候補當選人名次錯誤、遺漏者,均得向該管衙門呈控。限自選舉日起三十日,凡選舉訴訟,初選向府、直隸、州衙門,衣復選向按察使衙門呈控。各省已設審判者,分別向地方高等審判呈控。不服判定者,初選得向按察使衙門,衣復選得向大理院上控。限判定日起三個月。已設審判者,照審判上控章程辦理。選舉人及辦理選舉人、選舉關系人,有違法行為,分別輕重,處以監禁、罰金有差;二年以上、十年以下,不得為選舉人及被選舉人。

專額議員選舉人及被選舉人,以京旗及駐防人員為限,選舉及被選舉資格,與諮議局普通議員資格同。各省駐防專額議員之數,視該省駐防取進學額全數在十名以內者設議員一名,二十名以內設二名,二十名以外設三名。初選當選人額數,以議員定額十倍之數為準。衣復選當選人額數,以議員定額為準。調查選舉人名冊,由督、撫會同將軍、都統,於京旗及駐防人員內,各酌派選舉調查員。當選、改選、補選及訴訟、罰則各事,均照諮議局選舉章程辦理。此各省諮議局議員初選、衣復選辦法之概略也。

各省諮議局選舉,宣統元年各督、撫次第奏報舉行。於九月初一日,召集開會,舉行互選資政、諮議員。二年四月,資政院奏請欽選各項議員,奉敕選定。以八月二十日為召集期,九月初一日,資政院舉行第一次開院禮。監國攝政王代行蒞選,頒諭嘉勉議員。三年九月,遵章第二次召集開會。

資政院、諮議局議員選舉外,尚有地方自治團體之選舉。地方自治為立憲基礎,列於籌備事宜清單。光緒三十四年、宣統元年,憲政編查館先後核議,民政部奏城、鎮、鄉、府、、州、縣及京師地方自治暨選舉各章程,各省次第籌辦。其選舉辦法,與諮議局議員選舉略有出入。以繁瑣,不備載。


\end{pinyinscope}