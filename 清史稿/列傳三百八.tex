\article{列傳三百八}

\begin{pinyinscope}
籓部四

○喀爾喀土謝圖汗部喀爾喀車臣汗部喀爾喀賽因諾顏部

喀爾喀扎薩克圖汗部

土謝圖汗部,稱喀爾喀後路,至京師二千八百餘里。東界肯特山,西界翁吉河,南界瀚海,北界楚庫河。

元太祖十一世孫達延車臣汗,游牧瀚海北杭愛山界。子十一,格哷森扎扎賚爾琿臺吉其季也。兄圖嚕博羅特、巴爾蘇博羅特、阿爾楚博羅特、鄂齊爾博羅特等,由瀚海南徙近邊,為內扎薩克敖漢、柰曼、巴林、扎嚕特、克什克騰、烏珠穆沁、浩齊特、蘇尼特、鄂爾多斯九部祖,詳各傳。獨所部號喀爾喀,留故土,析★萬餘為七旗,授子七人領之,分左、右翼。其掌左翼者,為第三子諾諾和及第五子阿敏都喇勒。諾諾和號偉徵諾顏,子五:長阿巴岱,號斡齊賴賽因汗;次阿布瑚,號墨爾根諾顏,徙牧圖拉河界,今土謝圖汗部二十扎薩克皆其裔。阿巴岱子二:長錫布固泰,號鄂爾齋圖琿臺吉,為扎薩克貝子錫布推哈坦巴圖爾、輔國公巴海、臺吉車凌扎布、青多爾濟四旗祖;次額列克,號墨爾根汗,為土謝圖汗察琿多爾濟、扎薩克郡王噶勒丹多爾濟、貝勒西第什哩、車木楚克納木扎勒、輔國公車凌巴勒、三達克多爾濟、臺吉巴朗、班珠爾多爾濟、辰丕勒多爾濟、朋素克喇布坦十旗祖。阿布瑚子三:長昂噶海,繼墨爾根諾顏號,為扎薩克郡王固嚕什喜,臺吉車凌、開木楚克、成袞扎布、遜篤布五旗祖;次喇瑚里,號達賴諾顏,為扎薩克臺吉禮塔爾一旗祖;次圖蒙肯,號昆都倫諾顏。初喀爾喀無汗號,自阿巴岱赴唐古特謁達賴喇嘛迎經典歸,為眾所服,以汗稱。子額列克繼之,號墨爾根汗。額列克子三:長袞布,始號土謝圖汗,與其族車臣汗碩壘、扎薩克圖汗素巴第同時稱「三汗」。

崇德二年,袞布偕碩壘上書通好。三年,遣使貢駝、馬、貂皮、雕翎及俄羅斯鳥槍,命喀爾喀三汗歲獻白駝一、白馬八,謂之「九白」之貢,以為常。

順治三年,蘇尼特部長騰機思叛逃,豫親王多鐸率師追剿,至扎濟布喇克,袞布遣喇瑚裡等以兵二萬援騰機思,為大軍所敗,棄駝馬千餘竄額爾克。楚琥爾者袞布族也,復私掠巴林部人畜,詔使責之。會所部額爾德尼陀音貢馬至,敕歸諭其汗等擒獻騰機思,並以所掠歸巴林。五年,騰機思降,袞布等表乞罪,詔各遣子弟來朝,不從。八年,以其部不歸巴林人畜,僅獻駝十、馬百入謝,嚴諭詰責。十年,命侍郎畢哩克圖往察巴林被掠人畜,袞布等匿不盡給。會喇瑚里之子臺吉木塔爾攜眾來歸,封扎薩克親王,駐牧張家口外塔嚕渾河,因詭言巴林人畜木塔爾盡攜往,應就彼取,並乞遣木塔爾等還。諭曰:「爾等不遵旨遣子弟來朝,不進本年九白常貢,不盡償巴林人畜。冒此三罪,反請遣還來歸之人,是何理耶?今即各遣子弟來朝,盡償巴林人畜,朕亦弗使木塔爾等還,爾自擇之!」是年秋,遣使補貢九白,至張家口,詔勿納。十二年夏,土謝圖汗察琿多爾濟繼其父袞布為左翼長,約同族墨爾根諾顏、達爾漢諾顏、丹津喇嘛等,表遣子弟來朝。諭曰:「爾等遵旨服罪,朕不咎既往,其應歸巴林人畜缺少之數,悉從寬免。嗣後逃人至此,當即遣還。」冬,復遣使乞盟,許之,賜盟於宗人府。是年,設喀爾喀八扎薩克,仍分左、右翼,命土謝圖汗及墨爾根諾顏各領左翼扎薩克之一。十五年,遣大臣齎服物賚之。

康熙二十三年,以其部與右翼扎薩克圖汗成袞構釁,命阿齊圖格隆偕達賴喇嘛使諭解之。二十六年,察琿多爾濟偕車臣汗諾爾布等疏上尊號,諭曰:「爾等恪恭敬順,具見悃忱,但宜仰體朕一視同仁、無分中外至意。自今以後,親睦雍和,毋相侵擾,永享安樂,庶慰朕懷,勝於受尊號也。」

二十七年,厄魯特噶爾丹掠喀爾喀,察琿多爾濟拒弗勝,偕族弟固嚕什喜等攜屬來歸,詔附牧蘇尼特諸部界,發歸化城倉米贍之。二十八年,復遣內大臣費揚古往賑,諭廷臣曰:「朕聞土謝圖汗屬眾有乏食致斃者,深為軫念。費揚古採買牲畜尚須時日,著速發張家口倉米運往散給,計支一月,牲畜繼之,則眾命可活矣。」二十九年,詔察所屬貧戶,遣就食張家口。

三十年春,上以察琿多爾濟來歸後,車臣汗烏默客、扎薩克圖汗成袞子策旺扎布踵至,喀爾喀全部內附,封爵官制宜更,且降眾數十萬錯處,應示法制俾遵守,將幸多倫諾爾行會閱禮,詔理籓院檄察琿多爾濟等隨四十九旗扎薩克先集以俟。尚書馬齊奉命往議禮,定賞格九等,坐次七行,以察琿多爾濟為之首。夏四月,駕至,喀爾喀汗、濟農、諾顏、臺吉等三十五人以次朝見,諭曰:「爾等以兄弟之親,自相侵奪,啟釁召侮,至全部潰散。其時若令四十九旗扎薩克將爾人眾收取,爾部早已散亡。朕好生之心出於天性,不忍視爾滅亡,給地安置,復屢予牲畜、糗糧以資贍養,用是親臨教誨,普加賞賚。會同之時,見爾等傾心感戴,特沛恩施,俾與四十九旗同列,以示一體撫育,罔分中外,爾等其知朕意。」尋命改所部濟農、諾顏舊號,封王、貝勒、臺吉有差,各授扎薩克,編佐領,仍留察琿多爾濟汗號統其眾,自是始稱土謝圖汗部。三十一年,改喀爾喀左右翼為三路,土謝圖汗稱北路。

三十五年四月,上親征噶爾丹,所部諸扎薩克奏:「臣等被噶爾丹掠,全部潰,賴聖主天威正其罪,請從徵效力。」諭毋庸盡行隨往。五月,大軍既破噶爾丹於昭莫多,凱旋,大賚之。明年,噶爾丹竄死,朔漠平,詔所部歸圖拉河游牧。四十年,賜牧產贍給。五十四年,以準噶爾策妄阿喇布坦煽眾喀爾喀,命散秩大臣祁里德率大軍赴推河偵御。廷議屯田鄂爾坤、圖拉裕軍食,詔詢土謝圖汗旺扎勒多爾濟勘奏所部可耕地,因言附近鄂爾坤、圖拉之蘇呼圖喀喇烏蘇、明愛察罕格爾、庫爾奇呼、扎布堪河、察罕廋爾、布拉罕口、烏蘭固木及額爾德尼昭十餘處俱可耕,命公傅爾丹選善耕人往屯種。是年,詔簡所部兵駐防阿爾泰。六十年,命土謝圖汗旺扎勒多爾濟督理俄羅斯邊境事。

雍正二年,北路軍營移駐察罕廋爾及扎克拜達哩克。三年,以增設賽因諾顏部,定所部為喀爾喀後路。四年,旺扎勒多爾濟等因額爾德尼昭乏相宜穀種,遣人購之俄羅斯,並請助屯田兵糧。諭廷臣曰:「前議屯田時,曾有奏言喀爾喀未必踴躍從事者。朕思此舉正為伊等計及久遠,豈有反不樂從之理?今果感恩抒誠,與朕意相符,殊可嘉尚,交理籓院議敘。」尋各予紀錄,並賚幣有差,詔如議。五年,以庫倫及恰克圖為所部與俄羅斯互市地,詔非市朝毋許俄羅斯逾楚庫河界。是年,賽音諾顏親王額駙與俄羅斯定界。九年,選兵隨大軍剿噶爾丹策凌。十三年,撤大軍還,詔所部兵留駐鄂爾坤及烏里雅蘇臺。

乾隆元年,復選兵赴鄂爾坤防秋。六年,命參贊大臣都統塔爾瑪善察閱防秋兵於烏克圖爾濟爾哈朗。以哲布尊丹巴呼圖克圖移居庫倫,命土謝圖汗敦丹多爾濟駐守其地護視之。十三年,選駝五百運歸化城米赴塔密爾軍營,命土謝圖汗延丕勒多爾濟督理俄羅斯邊境事。十七年,增防鄂爾坤兵。十九年,移駐鄂爾海喀喇烏蘇。是部扎薩克親王額琳沁多爾濟授西路參贊大臣。二十年,進剿達瓦齊於伊犁。時降酋阿睦爾撒納謀據伊犁,上燭其奸,詔入覲。定北將軍班第由尼楚袞軍營遣額琳沁多爾濟護之行。至烏隆古河,阿睦爾撒納以北路定邊左副將軍印授之,詭稱歸治裝,由額爾齊斯河馳遁。翌日,額琳沁多爾濟追之弗及,論罪削爵擬斬,諭賜自盡。多羅貝勒車布登亦以駐防庫克嶺,不力追叛遁之巴朗,降貝子。而扎薩克輔國公車登三丕勒以俘青袞咱卜功,扎薩克一等臺吉達什旺勒以擒叛遁之和碩特訥默庫功,扎薩克一等臺吉班珠爾多爾濟以獲阿睦爾撒納旗纛甲胄功,扎薩克一等臺吉三都布多爾濟以赴扎布堪獲阿睦納撒納之孥及班珠爾等,並誅叛賊固爾班和卓輩功,均進爵賚賞有差。

先是,土謝圖汗部編佐領,積三十七旗。以分置賽因諾顏部,析二十一旗,留十六旗,仍隸土謝圖汗部。尋增四旗。扎薩克凡二十,盟於汗阿林,設正副盟長及副將軍、參贊各一。爵二十有一:土謝圖汗一;扎薩克和碩親王一,由貝勒晉襲;附公品級一等臺吉一;扎薩克多羅郡王二,一由貝勒晉襲;扎薩克固山貝子二,一由郡王降襲,一由扎薩克臺吉晉襲;扎薩克輔國公六,三由扎薩克臺吉晉襲;扎薩克一等臺吉八,一由貝子降襲。

是部本為喀爾喀四部之首,內則哲布尊丹巴,住錫庫倫,外則鄰接俄羅斯,有恰克圖互市,形勢特重,號稱雄劇。乾隆二十七年,於是部中旗汗山北之庫倫置辦事大臣,以滿洲大員任之;別選蒙古汗、王、公、扎薩克一人為辦事大臣,同釐其務。和碩親王多羅額駙桑齋多爾濟以乾隆二十三年赴庫倫協理俄羅斯邊境事。二十七年,停互市。二十九年,桑齋多爾濟請增庫倫卡坐,派兵屯田依琫、布爾噶勒臺等處,不許。三十年六月,命阿里袞索琳查辦恰克圖潛通貿易一案,以桑齋多爾濟私聽蒙人仍與俄商貿易,論罪削爵;辦事大臣丑達以私市得賄正法。十月,以是盟扎薩克貝子伊達木什布管俄羅斯卡坐。三十三年,庫倫辦事大臣慶桂等奏俄羅斯遣使乞開關交易,允之。仍申內地商人圖增價值之禁。尋命桑齋多爾濟復任。

四十二年,定庫倫辦事大臣兼轄辦事章京,民、蒙交涉事件均具報辦理例。四十三年,桑齋多爾濟奏俄羅斯人私越邊口賣馬,俄員瑪玉爾不肯前來,暫停貿易,即咨示俄固畢納托爾,上是之。七月,諭桑齋多爾濟會同辦事大臣博清額,商辦內地商人給還俄羅斯欠貨。十一月,桑齋多爾濟卒,命土謝圖汗車登多爾濟往庫倫協同博清額辦事。四十五年,復開市。四十八年,以車登多爾濟私給乘騎烏拉黃緞照票,罷庫倫辦事大臣,命賽因諾顏親王拉旺多爾濟代之,仍命桑齋多爾濟之子郡王蘊端多爾濟隨同辦事,定喀爾喀四部烏拉章程。十二月,命蘊端多爾濟列名在辦事大臣勒保之前。四十九年,以俄羅斯屬布裏雅特人劫內地往烏梁海貿易商民,賠貨而不交犯,屢檄其國。五十年春,以俄羅斯覆文支吾推宕,復停恰克圖互市。辦事大臣松筠因定沿邊蒙古需用煙茶布疋章程。

五十一年九月,定土、車兩部及賽、扎兩部每年各帶一部人入圍場,土、車兩盟部落人交庫倫辦事王大臣帶領習圍,賽、扎兩盟部落人仍交烏里雅蘇臺將軍大臣帶領習圍,並令部落每年自汗、王至公各揀派一人,臺吉內各揀派四人,領職銜較大者二名,微末臺吉二名,仍作十名善射赴木蘭圍場例。五十四年,俄屬布裏雅特人傷我出卡巡兵,松筠檄俄固畢納托爾捕送置之法。適有自俄歸之土爾扈特喇嘛薩麻林言俄將興兵構釁。廷旨命松筠檄詢。五十五年,是部戈壁數旗災,扎薩克臺吉烏爾湛扎布報以應收賦及自畜牛羊賑給,並令有力臺吉官兵周恤貧者。事聞,上嘉之。五十六年,松筠奏俄守邊目力辨其誣,詔誅薩麻林,許俄復市。松筠與接任辦事大臣普福、協辦貝子遜都布多爾濟赴恰克圖,曉諭俄固畢納托爾,嗣後如遇會辦事件,應如例迅速完結,命盜案犯,應送恰克圖鞫實正法,彼此約束商販,毋有積欠,因與立約,永為遵守。

嘉慶七年三月,土謝圖汗車登多爾濟等備行圍進哨馬匹,上嘉之。八月,定土謝圖汗、車臣汗二部事務在庫倫會集,與辦事大臣一同辦理例。自是土、車二部重大事件,皆由庫倫辦事大臣專奏。允蘊端多爾濟請,每逾十年巡察俄羅斯交界卡倫一次。八年八月,允蘊端多爾濟請,土謝圖汗部扎薩克齊旺多爾濟、齊巴克扎布等旗,及哲布尊丹巴呼圖克圖徒眾所屬地方,免驅逐種地民人禁。嗣後另墾地畝,添建房屋,侵占游牧,並令從前租種者,按地納租。娶蒙女為妻者,身故之後,妻子給該處扎薩克為奴隸。呼圖克圖徒眾地方即為其所屬。並定該處居民按人給照,每年由蘊端多爾濟派員檢查,造冊報院;及再有無照之民任意棲止,盟長、扎薩克等治罪例。二十三年,庫倫遣蒙員同俄員勘明疆界。

道光四年三月,以庫倫章京尚安泰查驗伊琫等處種地民人不能核事,致民人等盤踞游牧,署車凌多爾濟扎薩克印務之臺吉貢蘇倫呈報驅逐,又誤毀領照人民房屋,命奪職,蘊端多爾濟等議處。仍申各旗容留無票民人之禁。七年,蘊端多爾濟卒,以綸布多爾濟代為庫倫辦事大臣。十二年,多爾濟拉布坦代之。十五年,多爾濟拉布坦奏喀爾喀招民墾復拋荒地畝章程,諭不許。十二月,命德勒克多爾濟為庫倫學習幫辦大臣。十八年,多爾濟拉布坦奏管卡倫扎薩克那木濟勒多爾濟擅以奇爾渾卡倫兵丁與明濟卡倫兵丁互相移駐,撤差,仍議處。十九年,允哲布尊丹巴往庫倫之北伊魯格河溫泉坐湯,命辦事大臣福英護視。四月,多爾濟拉布坦卒,以德勒克多爾濟代為庫倫辦事大臣。二十一年六月,俄羅斯薩納特衙門咨理籓院,聞中國嚴禁鴉片入界,已諄飭所屬不得在交界之處互相販帶偷運。諭庫倫辦事大臣嚴禁內地貿易人等在交界處所私行販運煙土,以綏外籓、除積弊。二十二年九月,德勒克多爾濟以庫倫地方商民盤踞一案,下部議處。

咸豐四年,土謝圖汗、車臣汗兩部汗、王、公、臺吉等請捐助軍需,溫旨卻之。八年,允俄羅斯使人由庫倫至張家口入京。十一年,德勒克多爾濟遷,以多爾濟那木凱代為庫倫辦事大臣,尋令車臣汗阿爾塔什達代之。以辦事大臣色克通額帶操演鳥槍兵丁赴恰克圖,命多爾濟那木凱妥辦庫倫事件。四月,色克通額奏俄商欲於庫倫貿易,行文阻止。六月,總理各國事務王大臣奏準俄人在庫倫修理公館。十一月,色克通額奏俄商擅往蒙古各旗貿易。諭守約開導,並交總理各國事務衙門照會俄使禁阻。十二月,撤恰克圖習槍官兵。

同治元年,定俄國陸路通商章程條款。三年,以新疆回亂,調土謝圖汗、車臣汗兩部蒙兵赴烏魯木齊等處助剿。四年三月,以土、車兩盟蒙兵潰散回旗,諭文盛等不必再令赴營。以圖盟援古城蒙兵逗留,扎薩克達爾瑪僧格嚴議。五年,命辦喀爾喀四盟捐輸。六年,調土、車兩盟兵一千五百名駐防卡倫。八年,改訂中俄陸路通商章程,兩國邊界貿易在百里內均不納稅;俄商許往中國所屬設官之蒙古各處,亦不納稅;其不設官之蒙古地方,該商欲前往貿易,亦不攔阻,惟該商應有邊界官執照。

九年二月,回匪東竄,自三音諾顏左翼右旗扎薩克阿巴爾米特游牧擾是部左翼後旗鎮國公巴勒達爾多爾濟游牧。辦事大臣張廷岳等奏:「蒙古地方幅員遼闊,蒙眾皆擇水草旺處游牧,相距數十里始有氈廬。且百餘年安享太平,久不知兵。賊知蒙古易欺,是以百數成群,縱橫肆擾。擬調駐卡倫蒙兵,檄兩部落盟長等帶往西南一帶,與各旗官兵協剿。庫倫地方塔廟甲於各旗,商賈輻輳,人煙稠密。現派桑卓特巴等調集喇嘛、鄂拓克防護廟宇。又令商民辦理保甲,以資守御。」六月,張廷岳等奏以土盟兵九百名交扎薩克公奈當等防守額爾德尼昭。七月,俄調馬隊在庫倫操演,諭張廷岳等查察。尋以烏里雅蘇臺危急,張廷岳等奏調土、車兩盟兵會剿。十二月,請以賽、扎兩盟協防庫倫官兵二百名歸並賽、扎兩盟,派兵分防要隘。

十年二月,回匪復竄額哲呢河一帶,圖犯庫倫。張廷岳等奏迅檄達爾濟等軍赴哈爾尼敦西北地方防剿。十一年,張廷嶽奏:「前調土、車兩盟官兵餉糈,上年由兩盟捐輸支給。烏城被陷,復奏調內地官兵來庫防剿,檄土、車兩盟及沙畢捐備馬三千匹,資漢兵騎乘,又借雇駝馬數千隻,分赴各臺。兩盟官兵自上年遣散,改征作防,應需駝馬三千餘隻,亦系各旗攤派。」四月,回匪竄是部左翼中旗郡王拉蘇倫巴咱爾游牧,焚掠府廟,東犯莫霍爾、嘎順等臺。張廷岳遣蒙員札齊魯克齊、伯克瓦齊爾等追敗之於烏拉特中旗沙巴克烏蘇地方。六月,副都統杜嘎爾奏回匪於四月由圖盟公巴勒達爾多爾濟游牧竄出順新地方。五月,竄郡王拉蘇倫巴咱爾游牧之巴爾圖叟吉地方。派吉爾洪額帶隊改道躡賊。時回匪復西竄左翼中左旗扎薩克達爾瑪僧格游牧,至烏拉特中公旗之布特拉地方。吉爾洪額會伯克瓦齊爾進擊,大勝之。

八月,回匪復竄是部左翼後旗公巴勒達爾多爾濟游牧,直趨翁吉河一帶。別股竄哲林等臺,賽爾烏蘇西北臺路斷。張廷岳等奏察哈爾所派達爾濟一軍抵翁吉河之烏勒幹呼秀地方,與是部左翼中左旗公齊莫特多爾濟及伯克瓦齊爾二營相犄角。是月二十一日,伯克瓦齊爾敗賊於察布察爾臺之察罕吉哩瑪地方。二十六日卯刻,伯克瓦齊爾星夜由間道窮追,繞出東犯庫倫匪前,敗之於阿達哈楚克山額里音華地方。午申刻連再捷,獲駝千餘、馬四百,圍賊於畢留廟,相持六晝夜。九月二日,達爾濟軍至畢留廟西北駐營,匪以投誠誑之,達爾濟遽阻伯克瓦齊爾軍巡邏,匪於是夜輕騎西遁。十二月,張廷岳等奏前竄烏、庫兩城回匪,現均返肅州老巢。宣化、古北口二軍於本年到庫,擇要設防,足資捍衛。土、車兩盟官兵擬裁半留半,每屆半年,輪換防護官署昭廟,撤沙畢兵。

十二年二月,回匪復擾左翼後旗公巴勒達爾多爾濟游牧,尋遁。三月,張廷岳等奏:「庫倫事務較繁,請土、車兩盟之協理將軍,飭令每年輪班在庫聽候差委,勿赴烏城。」下金順等會商覆奏。諭催山東於五月前解清庫倫餉銀十萬兩,賚庫倫商民團勇。定變通辦理庫倫軍需章程。十三年九月,庫倫辦事大臣阿爾塔什達卒,以那木濟勒端多布代之。

光緒元年,以庫倫解嚴,撤回直隸古北口練軍。四年十一月,以庫倫、哈拉河等處游匪尚多,仍撥直隸宣化練軍二百五十名駐之。五年二月,以穆圖善奏,諭飭土謝圖汗迅將撤回托里布拉克、圖固里克二臺幫臺官兵駝馬,催令仍回本臺。五月,予捐輸銀兩之土謝圖汗那遜綽克圖等獎。六年正月,以改議俄國歸還伊犁條約,籌備邊防,派土、車二盟兵二千蒙兵駐庫倫,撥軍火及備蒙古包銀。十二月,給庫倫防兵月餉。七年二月,撤駐庫倫蒙兵。四月,以庫倫為俄人來往沖途,調喜昌為庫倫辦事大臣,統新軍千人赴之。是年,中俄訂續改陸路通商章程,俄國商民往蒙古貿易者,祗能由章程附清單內。卡倫過界,應有本國官所發中、俄兩國文字,並譯出蒙文執照,註明姓名、貨色、包件、牲畜數目,於入中國邊界時,在卡倫呈驗。其無執照商民過界,任憑中國官扣留。

八年四月,喜昌奏考察庫倫時勢邊防情形,量議變通。一、庫倫與恰克圖屯軍分駐。一、恰克圖改設道員鎮守邊塞。一、庫倫選練土著學試屯墾。一、庫倫屬境暨接連鄰省地方酌量屯兵。下所司議,格。尋以喜昌奏劾土盟盟長車林多爾濟,罷之,並下理籓院,議注銷土、車兩盟王公等駐班烏里雅蘇臺會盟之案。八月,喜昌等奏庫倫近與俄鄰,為漠北第一咽喉。現駐兵設防,饋運轉輸,舊站繞遠,亟宜變通,改設捷徑。諭飭烏里雅蘇臺將軍、察哈爾都統迅速妥籌覆奏。

九年二月,喜昌奏臺站遲滯,擬飭運草養駝,以資供應,並陳報災不實等情。諭綏遠城將軍豐紳等按照原奏斟酌妥辦。三月,察哈爾都統吉和等奏穆霍爾、噶順等九臺之官兵潛逃,詔喜昌等飭各旗竭力供差,不準推卸,仍嚴禁兵丁騷擾臺站。八月,察哈爾都統吉和等奏撫恤災荒,安設臺站。喜昌又劾車林多爾濟權勢太重,把持公事,串通各旗虛報災荒,遣撤官兵需用駝只,復為掣肘,各旗派撥幫臺,延不到差。諭新任辦事大臣桂祥密查具覆。時俄勢日盛,諸部王公漸生攜貳。喜昌所議置官、駐軍、屯田、改臺諸大端,皆以消患未萌。中朝重更張,致所請無一行者,卒以病去,並撤其軍。辛亥之變,實釀於此,識者惜之。九月,喜昌奏飭圖什業圖汗部未被災各旗暫行幫臺。尋庫倫辦事大臣那木濟勒端多布免,以土謝圖汗那遜綽克圖代之。

十年正月,以土謝圖汗部左翼中郡王阿木噶巴扎爾等四旗被災特重,諭桂祥等妥籌減緩差徭,予勸捐賑災之哲布尊丹巴呼圖克圖扁額。十二年,桂祥劾哲布尊丹巴之商卓特巴索訥木多爾濟居心巧詐,意構邊釁,革之。十六年八月,庫倫辦事大臣安德等奏庫倫所屬恰克圖等處開辦金礦,華商既無可招,洋商則斷不可招集,陳窒礙難行情形,下所司知之。十二月,御史聯奏庫倫商卓特巴喇嘛達什多爾濟欺朦把持,擅權科斂,下理籓院。十八年七月,定聯接中俄陸路電線。哲布尊丹巴所住之廟被火,佛像經卷胥毀。土盟等四盟王公捐助重建,而商卓特巴以此假貸商人,攤派沙畢者遂重。二十年九月,安德奏日本變動,民情惶惑,請仍調官兵駐庫倫,諭李鴻章酌度。

二十二年六月,庫倫辦事大臣桂斌奏:「哲布尊丹巴呼圖克圖屬沙畢一項困苦特甚,流亡過多。呼圖克圖忠厚存心,用人失當,一任喇嘛等勾通內地商民以及在官人等百方詐取,若罔聞知。迨用度過窘,不得不加倍苛派,所由欠負累累,上下交困。體訪其屬堪布喇嘛諾們汗巴勒黨吹木巴勒為僧俗所仰慕,應責成清理已檄署商卓特巴巴特多爾濟等,凡一切商上應辦事宜,悉心諮商,妥為籌畫。先將沙畢等應派光緒二十二年分攤款,查照十年以前,各按牲畜多寡,秉公勻攤,不準加派,核實酌裁。近年增添浮費,務量所攤撙節動用,俾紓民力。並請將東營臺市甲首各商,每遇兩大臣節壽酬款項不減不增,按年代哲布尊丹巴歸商欠。」下所司知之。尋又奏定恰克圖規費,化私為公,提滿、蒙大臣經費。七月,奏請定庫倫大臣與哲布尊丹巴呼圖克圖往還體制是否平行,有曰:「公事之間,備極融洽;相見之際,多似參商。實則哲布尊丹巴已驕蹇跋扈,與辦事大臣積不相能。」十一月,桂斌奏:「土盟所屬西北旗界哈喇河一帶,向有開墾地畝,播種雜糧,曾經奏明不準續墾。每屆臺市章京更換實任,由庫倫大臣扎委會同扎薩克等前往清查有無續墾。茲屆應查之期,照章派委臺市章京理籓院員外郎奎顯往查,將所得陋規呈請核辦,約計二千數百兩。當將兩大臣此次款費全發商人收還,其餘各項,暫照成案分賚各員,俾資津貼。」

二十三年六月,辦事大臣連順奏哲布尊丹巴呼圖克圖與蒙古辦事大臣圖什業圖汗那遜綽克圖兩不相能,請革辦事大臣之任,諭從之,並飭嗣後遇有此等事件,務妥為斟酌,勿聽呼圖克圖一面之詞。以土盟中旗貝子朋楚克車林為庫倫辦事大臣。連順以:「桂斌所奏歸還哲布尊丹巴商欠辦法,四成實銀,分年帶銷,雖恤蒙情,未恤商情,致該商等虧累太多,不敢與沙畢內外兩倉及鄂拓克交易。而兩倉鄂拓克雖有牲畜,無處易換,市井蕭條,諸貨不能暢銷。現呼圖克圖之廟工久竣,應照桂斌所奏,不得苛派,休息蒙眾。兩倉所用貨物銀茶及鄂拓克息借之款,應循舊日章程,設法算撥。」又奏:「據土盟盟長密什克多爾濟轉據各旗呈報,現查各旗呈報,並無未領限票民人種地之事。其由庫倫臺市章京衙門請領限票來旗貿易者,均隨來隨往,或搭蓋土房存貨收賬,牛羊並不孳生。墾荒民人建房養畜,每年交地租茶數十箱或百箱不等。復據商民元順明等七家呈,認種荒地,每年有地租茶,牲畜存廠,每年有草廠茶。請將認交前大臣桂斌罰款原茶交還。」旨均如所請。並將查地陋規化私為公,裁臺市章京查地之差。

二十四年,勸辦昭信股票。連順奏圖什業圖汗、車臣汗兩步落王公及哲布尊丹巴呼圖克圖等,情原報效市平銀共二十萬兩。五月,土、車兩盟王公及哲布尊丹巴沙畢、喇嘛等陳請不原領昭信股票,溫諭嘉之,仍飭一並給獎。以設庫倫、恰克圖電線,理籓院奏採伐土盟各旗官山木植。

先是,庫倫西北各旗至恰克圖一帶內地人民,率以租地墾荒為名,偷挖金砂,俄人亦多越界潛採,查禁驅逐,具文而已。至是連順奏:「土、車兩盟各旗界內庫倫東北六臺地,約合三百四十餘里,西自鄂爾河、哈拉河至額能河,共有金礦三處。又西北九臺地,約合五百三十餘里,北自色埒河至伊魯河,共有金礦二處,周圍二百餘里,金苗甚旺,以伊魯河所產為最佳。惟產自河內,水勢頗深,人力掏取,所得有限。必用西法以機器汲水,雇工開挖,其利方厚。擬招集鉅款,延聘礦師,購運機器,相地開採。宜同時舉辦,於居中扼要之處,設一總廠。綜計成本約需銀三百萬兩。」復據天津稅務司俄人柯樂德利庫西稱蒙古金礦,中國集款興辦時,俄人亦原附股,仍可代為招集,嚴遵中國章程。如用俄人,應聽中國官員約束,通盤籌畫。鄂爾河等五處金礦,擬請招商集款,合力開採,由中國自行舉辦,並準附招俄股,請簡派大員專司督率。下總理各國事務衙門會同礦務大臣議行。尋命連順督辦蒙古鄂爾河等礦。

是年,李鴻章等奏中俄會訂條約。俄國準在中國蒙古地方貿易,其蒙古各處及各盟設官與未設官之處,均準貿易,照舊不納稅。其買賣貨物,或用現錢,或以貨易貨均可。並準俄民以各種貨物抵賬。在庫倫設領事,科布多、烏里雅蘇臺俟商務興旺添設。

二十五年十月,奏集股開採,以土、車兩盟同時共舉為宜,即集土、車兩盟長切實勸諭,俾知開礦之舉,不特保衛邊疆,且開蒙古生計,報聞。土盟盟長密什克多爾濟以連順等劾阻撓開礦,罷之。十一月,洛布桑達什面謾哲布尊丹巴,以玩褻黃教議處。理籓院奏蒙古王公等請停辦礦務,命昆岡、裕德往查辦,並諭連順緩辦庫倫礦務。十二月,庫倫、恰克圖電線工竣。二十六年,昆岡等奏停辦礦務,連順下部議處。拳匪事起,命辦事大臣豐升阿等備邊。

二十七年三月,豐升阿、朋楚克車林奏圖什業圖汗部落盟長貝子棟多布等呈,駕幸西安,請捐本年應得俸銀緞疋,並量力捐馬備用,哲布尊丹巴呼圖克圖等亦呈捐馬千匹,均允納之。六月,豐升阿等奏:「上年內地拳匪肇禍,猝啟兵端,庫倫、恰克圖等處中外各商,紛紛遷徙,互相疑懼。當與駐庫俄領事官施什瑪勒福等再三晤商,均能奉約惟謹,力顧邦交。彼時雖有俄兵防守,尤能實力保護中外商民、蒙眾等性命貲財,兩不相擾,請予寶星。」允之。

二十九年二月,以防守邊疆異常出力,予土盟盟長扎薩克敦都布多爾濟雙眼花翎,土盟參贊郡王阿囊達瓦齊爾紫韁,土盟副盟長扎薩克鎮國公察克都爾扎布、土盟副將軍親王杭達多爾濟、總管西卡倫額魯特扎薩克貝子達克丹多爾濟乾清門行走,餘給獎有差。閏五月,土盟王公及哲布尊丹巴等報效修正陽門工程銀,允核給獎敘。豐升阿等奏覆改設行省,以外蒙地方與內地邊疆情形不同,一例辦理,多有窒礙。得旨:「是。」下所司知之。九月,烏里雅蘇臺將軍連順等奏土、車二盟金礦續議開辦,參酌外蒙等情形,詳訂章程,妥籌布置。請準派稅務司洋員柯樂德為總辦,並簡派大員專司督率,下部議。十一月,以蒙古辦事大臣朋楚克車林自庚子以來,慎固邊圉,輯睦外人,恤商撫蒙,勤勞足錄,予紫韁。

三十年,辦事大臣德麟奏庫倫後地蒙民租佃,擬設清墾局,以杜與外人私墾,下戶部議。三月,德麟等奏辦庫倫統捐。達賴喇嘛以印藏啟釁,避之庫倫,詔延祉迎,令赴西寧。九月,予駐庫倫直隸練軍官弁獎,以保衛蒙商,防護外人。十月,德麟奏結圖盟左翼中旗扎薩克郡王阿囊塔瓦齊爾債案。

三十一年,辦事大臣樸壽奏創辦釐金,委差官賈得勝等分往頭臺暨恰克圖等處帶兵稽查偷漏,分段彈壓。七月,以理籓院奏,予哲布尊丹巴呼圖克圖女徒寮汗達拉額爾德尼車臣名號。十二月,設庫倫巡警兵丁,由蒙人揀選。三十二年六月,以土盟王公等承購練兵戰馬,依限選齊,予盟長公銜扎薩克一等臺吉敦都布多爾濟等獎有差。

三十三年四月,允開庫倫金礦,定權限章程。以庫倫蒙古辦事大臣朋楚克車林會同延祉督辦礦務。三十四年二月,辦事大臣延祉以派員勘丈各旗墾地,親王杭達多爾濟旗臺吉巴圖巴魯抗不備臺,請嚴加議處,允之。五月,增開依拉裕格倫南之克勒司。八月,試辦庫倫土藥統稅。設蒙養學堂,就選土、車兩盟及沙畢幼童,專習滿、蒙、漢語言文字,以興辦新政,蒙古通曉漢文漢語少,易致隔閡。

宣統元年閏二月,延祉等奏準設庫倫理刑司員。時哲布尊丹巴呼圖克圖之商卓特巴巴特瑪多爾濟捐學堂經費八千兩,延祉為請賞帶膆貂褂。得旨,下理籓院核給獎敘。十一月,以庫倫各廠所出金砂較往年暢旺,給監辦官等花紅。

二年五月,辦事大臣三多以土、車兩盟沙畢等三處屢報災祲,供億過繁,歷年息借華、俄債款,迭經報官索欠者,約計不下百餘萬兩,竟有估一旗之牲畜不足抵債者。而自供哲布尊丹巴外,光緒二十九年至宣統元年,庫倫大臣等修理衙署及器具鋪墊等項,已合銀十八萬餘兩,支應馬匹、食羊、柴炭等項尚不在內。因奏核定土、車兩盟沙畢供庫倫大小衙門柴炭、羊數目,及限制各官調任修署添物章程。其餘差使,統由各員自為籌備,並以物價昂貴,費用竭蹶,請加各員公費銀一萬二千兩。先侭庫倫外銷公款項下開支,倘有不敷,由庫倫金礦稅款暫撥,仍言金礦逐年漸有起色,蒙困一蘇,商務亦可興旺,稅額自必加增,解部之款,不至較往年為絀,下度支部議行。清中葉後,諸邊將軍、大臣以下俸給過薄,皆倚籓部供應為生計,三多此疏,可以例之。十月,三多奏喇嘛登曾奪犯拒捕一案,商卓特巴巴特瑪多爾濟迄不交出首要,歷次呈文,無理取鬧,要挾具奏,恐國家法令,官長政權,將難行於蒙地,請予斥革;哲布尊丹巴自二月奉嚴加約束電旨後,庫屬喇嘛安分守法,為近年所未有,請傳旨嘉獎:均允之。二年四月,是部親王朋楚克車林為資政院欽選議員。

三年,設庫倫審判各。軍諮府亦於庫倫設陸軍兵備處,派員統兵駐之。是年正月,三多奏宣統二年金礦應繳官稅計金砂易銀十九萬三千兩有奇,全數作為庫倫辦軍事的款。是月,開圖盟扎薩克那木薩賴旗奎騰河金礦。四月,開雅勒弼克金礦。閏六月,已革商卓特巴巴特瑪多爾濟報效辦理新政銀二萬兩,三多請賞還原銜,飭回庫倫署商卓特巴篆務,以是款作為修汽車路之需。八月,奏:「近來邊事日急,今沿途臺站,於來庫倫官員,則多方留難,於遞庫要件,則任意玩忽。請飭該管臺站認真整頓。」允之。九月,三多等以額爾德尼車臣報效銀一萬兩,奏準用杏黃圍車。時哲布尊丹巴與三多不協,是部親王杭達多爾濟等以債務素密結俄人,不悅新政。於是俄照會外務部,有不駐兵、不派官、不殖民之要求。

洎武昌事起,各省鼎沸,杭達多爾濟等遂於十月初九日擁哲布尊丹巴稱尊號,建元立國,置內閣。以喀爾喀八十六扎薩克名義通牒中外,指斥清廷,興復元業,驅逐在外蒙之滿清官兵。三多被迫去職,賽爾烏蘇管站站員亦於十二月去職。於是喀爾喀四部舉非清有。

是部地兼耕牧,礦產林木,均稱饒富。佐領共有四十九。

車臣汗部,稱喀爾喀東路,至京師三千五百里。東界額爾德尼陀羅海,西界察罕齊老圖,南界塔爾袞、柴達木,北界溫都爾罕。

元太祖十七世孫阿敏都喇勒有子謨囉貝瑪,駐牧克嚕倫河,生子碩壘,始號車臣汗。與其族土謝圖汗袞布、扎薩克圖汗素巴第同時稱三汗。子十一,今車臣汗部二十三扎薩克皆其裔。長嘛察哩,號伊勒登土謝圖,為扎薩克貝子達哩、臺吉旺扎勒扎布二旗祖;次察布哩,號額爾德尼臺吉,為扎薩克臺吉吹音珠爾一旗祖;次拉布哩,號額爾克臺吉,為扎薩克臺吉色棱達什一旗祖。次本巴,號巴圖爾達爾琿臺吉,為扎薩克鎮國公車布登一旗祖;次巴布,龔父汗號,為車臣汗烏默客,扎薩克郡王納木扎勒、朋素克,臺吉韜賚、羅卜藏、垂木扎素、額爾德尼、根敦八旗祖;次綽斯喜布,號額爾德尼琿臺吉,為扎薩克輔國公車凌達什,臺吉多爾濟達什、固嚕扎布三旗祖;次巴特瑪什,號達賚琿臺吉,為扎薩克貝勒車布登、輔國公車凌旺布、臺吉車凌多嶽特三旗祖;次車布登,號車臣濟農;次阿南達,號達賚濟農;次布達扎布,號額爾德尼濟農:均封扎薩克貝子。阿南達子貢楚克,授扎薩克臺吉,又自為一旗。

初,喀爾喀服屬於察哈爾。天聰九年,大軍平察哈爾,車臣汗碩壘偕烏珠穆沁、蘇尼特諸部長上書通好,貢駝馬。崇德元年春,以其部私與明市馬,諭責之曰:「明,朕仇也。前者察哈爾林丹汗貪明歲幣,沮朕伐明,且欲助之,朕故移師往征。天以察哈爾為非,故以其國予朕。今爾與明市馬,是助明也。爾當以察哈爾為戒,其改之!」碩壘遣偉徵喇嘛等來朝,請與明絕市,上嘉之,命察罕喇嘛往賚貂服、朝珠、弓、刀、金幣。二年,獻所產獸曰獺喜。三年,獻馬及甲胄、貂皮、雕翎,俄羅斯鳥槍,回部弓箙、鞍轡,阿爾瑪斯斧、白鼠裘,唐古特玄狐皮。詔歲貢九白,他物毋入獻。

順治三年,碩壘誘蘇尼特部長騰機思叛,遣子本巴等以兵三萬援,大軍敗之。師旋,詔責碩壘曰:「蘇尼特本察哈爾屬部,向化來歸,爾誘之使叛。朕遣兵追剿時,猶誡勿加兵於爾。詎意爾反稱兵抗拒,以致上蒼降譴,立見敗衄。儻非朕飭令班師,大兵既壓爾境,何難長驅直入耶?今爾若知自悔,欲贖前愆,其速擒騰機思來獻!」五年,騰機思乞降,碩壘遣使獻駝百、馬千入謝,詔遣子弟來朝。九年,以妄爭歲貢賞,諭責勿貢。十二年,巴布繼其父碩壘為車臣汗,遣子穆彰墨爾根楚琥爾來朝,詔宥前罪,貢九白如初。是年,喀爾喀左右翼設八扎薩克,命車臣汗領左翼扎薩克之一。十五年,遣大臣齎服物諭賚之。

康熙二十一年,以所屬巴爾呼人私掠烏珠穆沁部界,議增汛兵,嚴防禦。會貢使至,諭曰:「朕聞爾屬眾與界內蒙古互相竊奪,彼此效尤,恐乖生計。朕已飭界內人毋許出境滋擾,爾亦當約束所屬,守分安居。違者即拘治之,毋稍姑息。」二十二年,詔毋越噶爾拜瀚海游牧。巴布卒,子諾爾布嗣車臣汗。二十六年,偕土謝圖汗察琿多爾濟表上尊號,諭卻之。

二十七年,噶爾丹掠喀爾喀至克嚕倫河。時諾爾布及長子伊勒登阿喇布坦相繼卒,孫烏默客幼,臺吉納木扎勒等攜之來歸,從眾凡十萬餘戶,詔附牧烏珠穆沁諸部界,烏默客襲汗號如故。尋理籓院奏降眾日多,請授納木扎勒等為扎薩克轄之,報可。命科爾沁親王沙津等往示內地法度,諭曰:「朕因爾等為厄魯特所掠,憐而納之。今觀爾等並無法制約束部曲,恐劫奪不已,離析愈多。爰命增置扎薩克,分掌旗隊,禁止盜賊,各謀生業。爾等果能遵而行之,寇盜不興,禍亂不作,庶副朕撫育歸降、愛養群生之至意。」二十九年,選所部兵赴圖拉河,隨尚書阿喇尼偵御噶爾丹。三十年,駕幸多倫諾爾會閱,詔封王、貝勒、貝子、臺吉有差,各授扎薩克,編所部佐領,而以車臣汗烏默客統其眾。自是始稱車臣汗部。

三十一年,定所部為喀爾喀東路。三十四年,遣官往購駝馬。三十五年,上親征噶爾丹,師次克嚕倫河,烏默客等以兵從。凱旋,所部沿途慶獻,日億萬計。明年,詔歸克嚕倫河游牧。五十五年,諭所部選駝六千,以兵五千領之,由郭多里巴勒噶遜運軍糧赴推河。六十年,調兵防護烏梁海降眾於巴顏珠爾克。

雍正九年,選兵三千赴察罕廋爾軍營從剿噶爾丹策凌。十一年,復詔以所部兵千屯游牧西界,訓練防守,並追緝巴爾呼逃眾。十三年,撤還。

乾隆元年,選兵赴鄂爾坤防秋。六年,命參贊大臣都統塔爾瑪善察閱防秋兵於塞勒壁口。十三年,選駝五百運歸化城米赴塔密爾軍營。十七年,選兵四千駐防巴顏烏蘭。二十年,隨大軍剿達瓦齊於伊犁。二十一年,以所屬齊木齊格特人肆竊,命參贊大臣納穆扎爾等往緝,寘之法。諭扎薩克等曰:「朕因爾等不善經理游牧,以致盜賊肆行,特命大臣前往督緝。念皆起於饑寒,復令發帑賑給貧戶,以贍生業。爾等游牧,始皆寧謐。爾等習於玩愒,徒知盜賊已除,不復為貧者籌畫生計。又或目前尚知約束,日久漸至廢弛。當各統率所屬,詳察貧困之由,俾謀生有策,不至為非。即有頑悍不悛之徒,亦當嚴加約束,有犯必懲。務令上下安全,共享升平之福。」

蕩平準部之役,是部扎薩克郡王巴雅爾什第、扎薩克輔國公達爾濟雅均以俘叛賊包沁副總管阿克珠勒等功,巴雅爾什第晉親王,達爾濟雅晉貝子,扎薩克一等臺吉成袞扎布多爾濟以察逆賊青袞咱卜造偽符撤汛兵之詐,督兵嚴守各汛,予公品級,而貝勒旺沁扎布以死事伊犁,予優恤。

先是車臣汗部編佐領,置十一旗,後增十二旗。扎薩克二十有三,盟於克嚕倫巴爾河屯,設正副盟長各一,副將軍參贊各一。爵二十有六:車臣汗一;附輔國公一;扎薩克和碩親王一,由郡王晉襲;扎薩克多羅郡王一;附多羅貝勒一;扎薩克多羅貝勒一;扎薩克固山貝子二,一由貝勒降襲;扎薩克鎮國公一;扎薩克輔國公二,一由貝子降襲;公品級扎薩克一等臺吉一;扎薩克一等臺吉十三,一由貝子降襲,二由輔國公降襲;附鎮國公一,由貝子降襲。

二十五年八月,命車臣汗部落一體與土謝圖汗等三部落充派兵諸差。三十年,以是部扎薩克貝子旺沁扎布能約束屬下,捕獲私貿俄羅斯民人、蒙古等,上嘉之。四十七年,是部郡王桑齋多爾濟旗與黑龍江屬之呼倫貝爾巴爾虎處爭界,謂呼倫貝爾總管將音陳、阿魯布拉克等卡倫私自挪移。四十八年,呼倫貝爾總管三保會桑齋多爾濟及貝勒車凌多爾濟帶同耆老斟酌地圖,由界內挖出舊設卡倫所埋記木,貝勒車凌多爾濟將所屬人等全行收回,桑齋多爾濟仍稱阿魯布拉克一卡往外展占五十里。五十年,黑龍江將軍恆秀等查辦是部人等報稱阿魯布拉克卡並未外展占越,桑齋多爾濟坐罰俸。咸豐四年正月,是部車臣汗阿爾罕什達捐銀助軍,受之,卻王公等捐軍需之請。

同治二年,是部郡王等旗又與黑龍江巴爾虎爭界,尋命吉林將軍皁保勘之。三年。調是部兵援古城,潰歸。四年,扎薩克車林敦多布以逗留嚴議。六年,調車盟兵戍卡倫。九年,回匪東擾圖盟,是部供軍需,增戍役,應捐輸,勞費與圖盟等。九年十月,庫倫辦事大臣張廷岳以回匪東擾烏里雅蘇臺境,奏派是部貝勒幹丹準車林赴額爾德尼昭會剿。尋撤回。十年六月,以回匪踞圖盟左翼中旗郡王拉蘇倫巴咱爾游牧,圖犯庫倫,又派幹丹準車林統駐庫蒙兵赴噶爾沁圖裏克、托里布拉克二臺協剿。十一年十二月,以竄烏、庫兩城回匪均回肅州老巢,撤車盟官兵一半。十二年二月,張廷岳以烏里雅蘇臺將軍全順西征,庫倫籌備駝只,張廷岳派員赴圖、車二盟勸諭各王公等竭力捐助。

光緒七年,以改議俄約,調車盟兵駐庫倫。尋以約定撤之。二十二年,將軍崇歡以烏里雅蘇臺參贊大臣攤車盟規費特重,請禁之。庫倫辦事大臣桂斌以車臣汗阿爾塔什達任參贊大臣作俑,請追款,諭免之。是年,桂斌奏車盟報應襲臺吉已報未襲者有六百餘員,積壓未題者有三次之久。諭理籓院迅速核辦,不準積壓。二十五年九月,烏里雅蘇臺將軍連順奏車臣汗德木楚克多爾濟阻撓礦務,與俄人交密,形狀可疑,諭撤去差使。十一月,是部王公等又呈理籓院請停辦礦務,命昆岡等往勘緩之。二十六年,拳匪事起,庫倫辦事大臣豐升阿等調是部各旗官兵自備餉項,巡防邊卡。洎呼倫貝爾為俄兵所據,巴爾虎諸處避難官民均至是部界內,盟長等防守撫輯,均協所宜。二十八年,豐升阿以是部王公異常出力,請予獎勵。於是車盟盟長郡王多爾濟帕拉穆加親王銜,副盟長扎薩克鎮國公車林尼瑪挑御前行走,參贊扎薩克輔國公那爾莽達琥賞雙眼花翎,餘給獎有差。

宣統二年二月,內盟蒙匪托克托等竄擾是部貝子桑薩賴多爾濟旗,三多遣駐庫宣化練軍營官鄭春田等迎擊失利。電諭周樹模飭呼倫道汛派兵往接應,而蒙匪竄俄境。是年,是部郡王多爾濟帕拉穆為資政院欽選議員。三年閏六月,是部扎薩克貝子多爾濟車林等報效辦理新政銀兩,獎之。十一月,哲布尊丹巴稱尊號於庫倫,脅是部王、公、扎薩克等附之。

是部車臣汗阿爾塔什達、車林多爾濟父子皆為烏里雅蘇臺參贊大臣。有礦,有鹽池,有成吉思汗陵。佐領共有四十。

賽因諾顏部,稱喀爾喀中路,至京師三千餘里。東界博囉布爾哈蘇多歡,西界庫勒薩雅孛郭圖額金嶺,南界齊齊爾里克,北界齊老圖河。

元太祖十七世孫偉徵諾顏諾諾和有子五:長阿巴和,為土謝圖汗部祖;次塔爾呢,無嗣;次圖蒙肯;次巴賚。今賽因諾顏部二十四扎薩克,自厄魯特二旗外,皆其裔。圖蒙肯子十三:長卓特巴,號車臣諾顏,為扎薩克輔國公托多額爾德尼、諾爾布扎布、臺吉圖巴三旗祖;次丹津喇嘛,號諾捫汗,為扎薩克親王善巴、輔國公旺舒克、車凌達什、臺吉齊旺多爾濟、素達尼、多爾濟六旗祖;次車凌,次羅雅克,皆無嗣;次濟雅克,號偉徵諾顏,為扎薩克輔國公阿玉什一旗祖;次扎木本,其番不列扎薩克;次察斯喜布,號昆都棱,為扎薩克臺吉伊達木、納木扎勒二旗祖;次丹津,號班珠爾,為扎薩克超勇親王策棱子親王成袞扎布、郡王車布登扎布二旗祖;次畢瑪里吉哩諦,號巴圖爾額爾德尼諾顏,為扎薩克臺吉丹津額爾德尼一旗祖;次錫納喇克薩特,號琿臺吉,為扎薩克臺吉阿哩雅、薩木濟特二旗祖;次桑噶爾扎,號伊勒登和碩齊,為扎薩克臺吉沙嚕伊勒都齊一旗祖;次扣肯,號巴扎爾,為扎薩克臺吉濟納彌達一旗祖;次袞布,號昆都倫博碩克圖,授扎薩克郡王,今襲貝勒,其曾孫額墨

根,授扎薩克臺吉,又自為一旗。巴賚子一,曰噶爾瑪,為扎薩克鎮國公素泰伊勒登一旗祖。

初,喀爾喀有所謂紅教者,與黃教爭,圖蒙肯尊黃教,為之護持。唐古特達賴喇嘛賢之,授賽因諾顏號,令所部奉之視三汗。圖蒙肯卒,次子丹津喇嘛復受諾捫汗號於達賴喇嘛。

崇德三年,遣使通貢,優賚遣歸。五年,賜敕獎諭。順治四年,以偕其旗土謝圖汗袞布等合兵援蘇尼特部叛人騰機思,詰責之。七年,遣子額爾德尼諾木齊上書乞好,詔偕袞布約誓定議。十一年,額爾德尼諾木齊復奉表,諭曰:「爾奏言喀爾喀左翼四旗皆爾統攝,凡有敕諭,罔弗遵行。今即如所請,可速飭爾部長遣子來歸。有不遵者,即行奏聞。」十二年,偕袞布等各遣子弟來朝,詔宥前罪。尋設八扎薩克,命丹津喇嘛領左翼扎薩克之一,歲貢九白如三汗例。十八年,賜「遵文順義」號,給之印。

康熙三年,詔所屬毋越界游牧。丹津喇嘛卒,子塔斯希布襲。塔斯希布卒,子善巴襲,賜信順額爾克岱青號。二十七年,噶爾丹掠喀爾喀,善巴率屬來歸。詔附牧烏喇特諸部界。三十年,駕幸多倫諾爾會閱,詔封善巴等王、臺吉有差,各授扎薩克,編所屬佐領,隸土謝圖汗部。三十一年,善巴從弟策棱來歸。策棱者,圖蒙肯第八子丹津之孫,臺吉納木扎勒之子,後授固倫額駙和碩超勇親王、定邊左副將軍兼稱喀爾喀大扎薩克者也。三十六年,詔善巴等各歸舊牧。五十六年,選兵赴阿爾臺軍偵御策妄阿喇布坦。

雍正三年,上以所部系出賽因諾顏,較三汗裔繁衍,而額駙策棱自簡任副將軍,勞績懋著,命率近族親王達什敦多布,貝勒納木扎勒、齊素嚨,貝子策旺諾爾布,輔國公阿努哩敦多布、額琳沁、扎木禪旺扎勒,臺吉格木丕勒、齊旺、錫喇札布、達爾濟雅、根敦、車布登、巴朗、延達博第、呢瑪特、克什、諾爾布扎布,凡十九扎薩克,別為一部,以其祖賽因諾顏號冠之,稱喀爾喀中路,不復隸土謝圖汗部。喀爾喀有四部自此始。

九年,所部兵隨大軍剿噶爾丹策棱,擊其眾克爾森齊老及額爾德尼昭,大敗之。十三年,撤還。乾隆元年,選兵赴鄂爾坤防秋。六年,參贊大臣副都統慶泰察閱防秋兵於桑錦托羅海。十三年,選駝五百運歸化城米赴塔密爾軍營。尋調所部兵二千駐防錫喇烏蘇。十九年,移塔密爾軍營於是部中前旗之烏里雅蘇臺,以是部兵分駐扎布堪。二十五年,隨大軍剿達瓦齊,平之。二十六年,設烏里雅蘇臺至烏魯木齊臺站,留侍衛四員,餘撤之。

先是喀爾喀分設中路時,但以賽因諾顏名其部,以示別於三汗,未議襲號。三十一年,親王成袞扎布奏所部來歸。初,親王善巴為同族長,又世掌丹津喇嘛所遺印,請視三汗例,以善巴曾孫親王諾爾布扎布襲賽因諾顏號。詔允其請,俾與土謝圖汗、車臣汗、扎薩克圖汗均世襲罔替。蕩平準部之役,成袞扎布長子額爾克沙喇以剿叛賊巴雅爾功,封輔國公。策凌次子輔國公車布登扎布積俘準部宰桑庫克辛等、平達瓦齊、誅賊固爾班和卓、征哈薩克功,歷晉貝子、貝勒、郡王至親王品級。貝子車木楚克扎布積捕獲烏梁海宰桑、復設臺站及招降阿爾泰淖爾烏梁海功,歷晉封至郡王。扎薩克一等臺吉三都克扎布以協濟軍需,復予襲輔國公。扎薩克一等臺吉達什額以得叛賊布庫察罕功,予公品級。而貝子羅布藏車鄰以死事烏魯木齊,晉其子貝勒。

初,所部十九旗,後增三旗,附額魯特二旗。扎薩克二十有四,盟於齊齊爾里克,設正副盟長各一,副將軍、參贊各一。爵三十有三:扎薩克和碩親王二;附固山貝子一,由貝勒降襲;鎮國公一,由貝子降襲;輔國公二;公品級一等臺吉一;公品級三等臺吉一;扎薩克多羅郡王二,一由鎮國公晉封;扎薩克多羅貝勒二,一由郡王降襲,一由鎮國公晉襲;扎薩克鎮國公一,由扎薩克臺吉晉襲;附輔國公一;扎薩克輔國公五,一由扎薩克臺吉晉襲;公品級扎薩克一等臺吉一;扎薩克一等臺吉九;附輔國公一;公品級三等臺吉一;厄魯特扎薩克固山貝子二,一由郡王降襲,一由輔國公晉襲。

三十八年九月,以賽盟郡王車布登扎布為烏里雅蘇臺參贊大臣。四十二年十月,賽盟郡王車布登扎布率本部王、公、扎薩克、臺吉等進大行皇帝齋醮馬駝,溫諭卻之。四十五年六月,以賽音諾顏部落占據土謝圖汗游牧,諭博清額查明,毋使侵占。十月,定賽音諾顏、土謝圖汗兩部界址。

嘉慶四年,是部親王御前大臣拉旺多爾濟等請調集本盟兵馬助剿教匪,溫旨止之,並命理籓院傳知蒙古各盟,停其預備。七年八月,定喀爾喀賽因諾顏、扎薩克圖汗二部事務在烏里雅蘇臺會集,與定邊左副將軍一同辦理。八年,以是部齊巴克扎布旗容留種地民人,命交烏里雅蘇臺參贊大臣永保辦理。十二年五月,烏里雅蘇臺參贊大臣薩木丕勒多爾濟卒,以綸布多爾濟代之。

道光三年七月,以賽音諾顏盟長德木楚克扎布等於大路搶劫官人財物不能捕緝,詔嚴議。十月,烏里雅蘇臺將軍果勒豐阿奏:「烏里雅蘇臺地方,請準令商民等每年馱運茶七千餘箱赴古城兌換米面。如不敷,令湊買雜貨,仍照例給發印票,不準另往他處。」六年十一月,回疆軍興,賽音諾顏、扎薩克圖汗兩盟王、公、扎薩克等輸駝只助軍。七年十月,綸布多爾濟調庫倫辦事大臣。十二月,以車林多爾濟為烏里雅蘇臺參贊大臣。十八年,以哈薩克闌入卡倫,命車林多爾濟統賽、扎兩盟,杜爾伯特等蒙兵逐之。十九年正月,給驅逐哈薩克之賽、扎兩盟蒙古官兵俸賞行裝銀。四月,命車林多爾濟調兵驅逐復入烏梁海之哈薩克。八月,以驅逐哈薩克妥速,賚車林多爾濟親王俸一年。二十五年二月,賽盟郡王圖克濟扎布以不赴軍營,革副將軍,阿爾塔什達代之。

咸豐三年,賽、扎兩盟王、公、扎薩克等請捐助軍需,溫旨卻之。十一年,阿爾塔什達調庫倫,以車林敦多布代之。

同治三年,回匪陷烏魯木齊各城,調是部兵援古城,竟無功。五年七月,李云麟奏:「與明誼等會商,擬將扎薩克圖汗部、賽音諾顏兩部額兵全行派出,共一千八百名。其本愛曼操防之兵,徐為布置。旋因察漢烏蘇卡倫聞警,當與麟興等熟商。北路既有警報,擬每愛曼仍留五百兵備防本境。復商之車林敦多布,轉傳各盟長,將西兩盟額兵以外之壯丁,每盟再挑五百名,於八月派齊,隨後繼發。」並謂北路寇至不能御,差務不暇給,保貝勒晉丕勒多爾濟遇事勇敢,其才為喀爾喀四部王公之冠。適車林敦多布乞病,詔即以晉丕勒多爾濟代之。李云麟尋率賽、扎兩盟兵西進。十一月,至呼圖古蘭臺,扎盟兵變,賽盟兵亦潰,李云麟自奏回烏城,詔嚴責之。七年,晉丕勒多爾濟倡捐布倫托海新城經費,偕郡王桑噶西哩等捐銀二萬五千兩有奇。予晉丕勒多爾濟王銜,餘給獎有差。

九年二月,肅州回匪東竄,擾是部推河以西額爾德尼班第達呼圖克圖游牧,蒙兵潰於哈爾呢敦。閏十月己巳,庫倫辦事大臣張廷岳等奏:「回匪竄偪烏城,福濟、榮全督蒙兵二百在城防守,參贊大臣晉丕勒多爾濟督索倫、滿、漢兵五百迎擊,駐頭臺。竄匪三千現已抵二臺。」辛未,烏里雅蘇臺將軍福濟等奏:「回匪踞博克多山、推河口、額爾德尼昭等處。十月九日,竄至第十一烏特臺,文報不通,南臺蒙兵聞警先遁。」十一月戊申,福濟及參贊大臣榮全奏:「十月九日,賊千餘人由東南至西南山溝來撲東西南三門,東溝又來賊數千。初更,賊四面放火,毀柵而登,城池失陷。二十三日,賊由西南竄去。福濟遇救尚存,榮全奔向西北,於閏十月四日折回,定邊將軍印信遺失,榮全親兵護出伊犁將軍印信,暫時借用。」命福濟、榮全革職留任,諭杜嘎爾統察哈爾馬隊及已調吉林、黑龍江官兵赴烏城進剿。尋回匪西竄金山卡倫,晉丕勒多爾濟回烏里雅蘇臺。諭整飭臺站,疏通道路。十二月,諭晉丕勒多爾濟將張廷岳撤回官兵分布防守推河等處,福濟妥設霍呢齊及推河糧臺。癸酉,晉丕勒多爾濟奏飭賽、扎兩部落揀兵分扎烏城臺站,並防各旗游牧。乙酉,允福濟等請,設烏城駐班臺站扎薩克二員、管臺二員。諭福濟迅將哈爾呢敦等臺趕緊預備,催綏遠城所遣達爾濟一軍前進。是月,喇嘛棍噶扎拉參一軍自科布多援烏城。

十年正月,諭嚴催晉丕勒多爾濟設復烏城以南臺站。晉丕勒多爾濟劾福濟謬妄貽誤,自顧身命,將倉庫存項酬謝賊匪,眷屬皆系自盡,非為賊所害。福濟亦劾蒙古官員規避差使,請捏病告假規避,或飭傳故意遲行及始終不到者,均革職任,無職任者銷爵,仍令來營,從之。設霍呢齊臺轉運總局,福濟飭貢果爾帶察哈爾馬隊駐守之。榮全奏:「親往催辦烏城以南二十臺,行抵推河,見水臺氈房駝馬漸集。推河至哈爾呢敦五臺照舊布置,略有規模。請給自備駝馬幫臺之蒙古臺吉丁戶一半錢糧。」從之。以回匪復圖再擾烏城,諭福濟等整頓臺站,杜嘎爾軍毋得逗留。二月,諭福濟等妥為布置哈爾呢敦、額爾德尼昭、推河三處防守,並以達爾濟一軍行抵哈爾呢敦阻滯,飭督令各臺站妥為供支,毋誤戎機。三月,以烏屬各臺尚未備齊,致滯師行,諭切責福濟,並令傳知蒙古王公等率屬守御,予烏城殉難蒙兵恤。杜嘎爾奏派蘇彰阿帶黑龍江兵五百赴烏城,並調貢果爾一軍赴前敵各路。諭杜嘎爾赴察爾呢敦等處防剿。

四月,予賽盟臺吉車登丕勒吉雅捐銀面獎。杜嘎爾進駐貢鄂博地方。諭福濟等飭蒙古臺站應付駝馬等項。晉丕勒多爾濟以請歸游牧,罷烏里雅蘇臺參贊大臣,下院嚴議,以扎盟中左翼左旗貝勒多木沁扎木楚代之。福濟亦革任,以金順為烏里雅蘇臺將軍,奎昌署之。回匪復擾是部阿米爾密特游牧,焚掠固爾班賽汗等處。諭杜嘎爾會奎昌等迅速追剿。五月,回匪竄薩哈爾呢敦附近之薩巴爾圖河、推河一帶,杜嘎爾遣納魯肯一軍駐翁吉驛防之。六月,回匪竄擾霍爾哈順、霍呢齊二臺。諭慶春飭達爾濟於推河等處防守,杜嘎爾撥隊扼要駐扎,保護糧路。福濟等奏烏城調到吉林、黑龍江、察哈爾馬隊三千二百五十名,發圖、車、賽、扎四盟採買駝馬等銀各一萬兩。八月,回匪復竄入阿米爾密特旗,至巴彥罕山,逼近翁吉河。福濟等飭賽盟速派蒙兵五百名赴南臺哈爾呢敦堵截。九月,達爾濟一軍剿竄翁吉河之匪,殄之。杜嘎爾遣福珠哩率兵剿匪於阿米爾密特旗之那林渾第等處,殄之。是旗附近肅清。達爾濟亦敗賊於喀雅喀拉烏蘇地方。

十一年正月,肅州回匪復竄擾是部阿米爾密特旗游牧西南之濟爾哈朗圖地方。諭金順、奎昌等各設法保護所屬臺站。杜嘎爾奏派富珠哩一軍扼扎哈爾呢敦一帶。四月,回匪竄擾白託羅蓋及金山卡倫游牧,奎昌等遣馬隊追剿。九月,連敗之於沙爾魯爾頓及庫爾庫嚕地方,匪自阿育爾公旗竄扎哈沁。

十二年二月,烏里雅蘇臺將軍長順等以回匪屢擾賽、扎兩盟牧,暫令扎盟公車德恩敦多布多爾濟旗移於邊界相當之賽音諾顏部落右翼右後旗副將軍王格里克扎木楚、扎薩克瑪尼巴拉等旗游牧,賽盟扎薩克阿米爾密特旗移於本部賽音諾顏旗親王車林端多布等旗游牧。兩盟南界金山卡倫,亦令暫撤,俾作清野之計。奏入,得旨,下所司知之。十三年正月,烏城解嚴,長順等撥察哈爾新兵五百,令佐領依楞額統赴科布多,裁烏城賽、扎兩盟防兵五百,侍衛豐升阿統察哈爾馬隊仍駐扎巴罕河。

光緒六年,以改議俄約,調賽、扎兩盟蒙兵二千名駐烏里雅蘇臺。七月,以將軍春福等奏輔國公額爾奇博爾豁地方作為官屯。九月,予賽盟扎薩克濟爾哈朗報效屯地獎。七年六月,以俄約成,撤駐烏城之賽盟蒙兵。將軍杜嘎爾奏暫停辦博爾豁屯田。十一年九月,復設金山卡倫。十三年,署烏里雅蘇臺將軍祥麟等奏:「管理推河、扎克等臺吉巴扎爾等報所屬都特庫圖勒等三臺鼠災,請將都特庫圖勒臺暫移在諾們汗沙畢游牧內拜達里克河邊之敖爾楚克哈克圖地方,扎克、和博勒庫根兩臺向前移在賽盟右翼右後旗郡王吹蘇倫扎布旗屬之扎綏額奇叟吉、哈拉布拉克等地方。體察鼠災定息,青草暢茂,再飭各歸原臺當差。」允之。十九年,烏里雅蘇臺參贊大臣車林多爾濟病免,以那木濟勒端多布代之。二十一年十二月,修烏里雅蘇臺。二十三年,修烏里雅蘇臺河橋及河堤。二十五年九月,將軍崇歡奏查閱邊卡供給,每臺有加至百五十兩之事,此次免去。查閱南二十臺駝馬兩廠,專查五十五座臺卡供給應付,概從刪減。二十六年,崇歡奏以古城一帶蝗災,改採購戍守官兵日需米面於歸化城。是年以拳匪肇釁,邊防戒嚴,將軍連順等調賽、扎兩盟及烏梁海兵擇要防守,各王、公、扎薩克等挑選壯丁,籌幫軍食,均能嚴約屬下,勿欺凌俄商,保全大局。二十八年,請將奏入予賽盟盟長扎薩克郡王吹蘇倫扎布、親王那木囊蘇倫、副將軍扎薩克鎮國公剛珠爾扎布、副盟長扎薩克郡王固嚕固木扎布等獎有差,特予參贊大臣那木濟勒端多布黃馬褂。

二十九年,設烏城中、俄通商事務局。三十年八月,連順等以賽、扎兩盟呈報去冬今春雪災,牲畜倒斃。三十一年,是部中左末旗親王那彥圖請裁佐領所遺差戶。護將軍奎煥飭由本盟各旗分派,按旗接充。入夏亢旱,駝馬疲瘦,請緩查閱臺站,允之。三十二年,賽盟盟長吹蘇倫扎布卒,將軍奎煥等請於參贊大臣貝勒車登索諾木、親王那木囊蘇倫二員內簡一人為盟長。得旨,授那木囊蘇倫盟長。定例,盟長由理籓院請簡,此出將軍保奏,非恆格也。那木濟勒端多布之後,是部中左旗貝勒車登索諾木、中右旗郡王庫魯固木扎布相繼為烏里雅蘇臺參贊大臣。三十四年六月,御史常徽劾車登索諾木「捏報災情。本盟應派差使,不遵奏章赴邊。防守之差,以賄為定,蒙情不服,咸有戒心。如牧廠未報地界,任令開荒。駝馬捏報倒斃。孳生以多報少,弊混不可枚舉」。宣統元年,將車堃岫查覆,多為寬解,惟謂車登索諾木於本旗充當各差,或有互調他旗,以遠易近,避重就輕。管理旗務之扎薩克齊阿莫朦混自專,請革之,而為車登索諾木請免議。

二年,是部親王那木囊蘇倫、那彥圖為資政院欽選議員。三年,庫倫獨立,是部王公附之,將軍奎芳被迫去職。

是部額駙策凌之後,親王拉旺多爾濟、車登巴咱爾、達爾瑪、那彥圖多至御前大臣、領侍衛內大臣,為外扎薩克諸部所莫及。是部地兼耕牧,有礦,有鹽池,向稱饒富。共有佐領三十一。

扎薩克圖汗部,稱喀爾喀西路,至京師四千餘里。東界翁錦、西爾哈勒珠特,西界喀喇烏蘇、額埒克諾爾,南界阿爾察喀喇托輝,北界推河。

元太祖十六世孫格埒森扎扎賚爾琿臺吉有子七,分掌喀爾喀左、右翼。左翼牧圖拉河界,右翼仍留居杭愛山。其長子阿什海達爾漢琿臺吉、次子諾顏泰哈坦巴圖爾、第四子德勒登昆都倫、第七子鄂特歡諾顏同掌之。今扎薩克圖汗部十九扎薩克,自厄魯特一旗外,皆其裔。阿什海達爾漢琿臺吉子二:長巴延達喇,子賚瑚爾汗,為原封扎薩克圖汗策旺扎布及扎薩克貝勒卓特巴,臺吉喇布坦、額爾德尼袞布三旗祖;次圖捫達喇岱青,子碩壘烏巴什,號琿臺吉,為扎薩克貝勒根敦,輔國公沙克扎、齊巴克扎布,臺吉納瑪琳藏布、達什朋素克五旗祖。諾顏泰哈坦巴圖爾生土伯特哈坦巴圖爾,子二:長崆奎,號車臣濟農,為扎薩克郡王朋素克喇布坦、貝子博貝、輔國公索諾木伊斯札布,臺吉烏爾占、哈瑪爾岱青五旗祖;次賽因巴特瑪,號哈坦巴圖爾,為扎薩克輔國公袞占、臺吉伊達木扎布二旗祖。德勒登昆都倫生鍾圖岱,號巴圖爾,為扎薩克臺吉諾爾布一旗祖。鄂特歡諾顏生青達瑪尼默濟克,號車臣諾顏,為扎薩克輔國公通謨克、臺吉普爾普車凌二旗祖。

初,賚瑚爾為喀爾喀右翼長,所部以汗稱,傳子素巴第,始號扎薩克圖汗,與其族土謝圖汗袞布、車臣汗碩壘同時稱三汗。碩壘通好最先,袞布次之,素巴第最後。崇德三年,以其部謀掠歸化城,上統師征,所部遁,素巴第遣使謝罪,並貢馬及獨峰駝、無尾羊。諭曰:「朕以兵討有罪,以德撫無罪,惟行正義,故上天垂佑,蒙古、察哈爾諸部皆以畀朕。爾等皆其所屬,當即相率歸誠,不則亦惟謹守爾界。乃反興兵構怨,謀肆侵掠,豈以遠處西北,即為征討不及之區耶?今與爾約,嗣後慎勿復入歸化城界,重貽罪戾。」五年,復賜敕誡諭。

順治四年,素巴第聞詔責碩壘、袞布等納蘇尼特叛人騰機思及掠巴林罪,欲代為解,偕同族俄木布額爾德尼上書乞好。上因其書不稱名,詞近悖慢,切責之。七年,俄木布額爾德尼等詭稱行獵,私入歸化城界掠牧產,遣官飭歸所掠。會素巴第卒,子諾爾布嗣,稱畢錫哷勒圖汗,遣使入貢。諭曰:「朕本欲許爾等和好,故命察歸所掠以贖前罪。今爾等反以朕留爾逃人為詞,是何心耶?朕統一四海,爾等彈丸小國,勿恃荒遠,勿聽奸詞,致隕爾緒。」十二年,諾爾布偕俄木布額爾德尼各遣子來朝謝罪。十四年,復偕同族車臣濟農昆都倫陀音奉表乞好。詔宥前罪。十六年,遣大臣齎服物諭賚之。

先是喀爾喀左右翼設八扎薩克,諾爾布及俄木布額爾德尼、車臣濟農昆都倫陀音各領右翼扎薩克之一。諾爾布卒,子旺舒克襲,仍號扎薩克圖汗。俄木布額爾德尼卒,子額璘沁襲,號羅卜藏臺吉。康熙元年,額璘沁以私憾襲殺旺舒克,奔就厄魯特。其叔父袞布伊勒登避難來歸,封扎薩克貝勒,駐牧喜峰口外察罕和朔圖。詳喀爾喀左翼部總傳。九年,命旺舒克弟成袞襲扎薩克圖汗號,輯其眾。二十三年,成袞以額璘沁之亂,屬眾潰,多往依左翼土謝圖汗察琿多爾濟,屢索不獲,與構釁。命阿齊圖格隆等諭解之。會成袞卒,厄魯特噶爾丹謀掠喀爾喀,誘成袞子沙喇攻察琿多爾濟。沙喇因會噶爾丹於固爾班赫格爾,臺吉德克德赫等從往。察琿多爾濟惡之,追殺沙喇及德克德赫。二十七年,噶爾丹以兵三萬掠喀爾喀,至杭愛山,所部大潰。沙喇弟策旺札布偕同族色凌阿海等相繼來歸,詔附牧烏喇特諸部。三十年,駕幸多倫諾爾會閱,以所部屢經變亂被芟夷,詔封色凌阿海等王、貝子、臺吉有差,各授扎薩克,令集所屬編佐領撫輯之。而以成袞子策旺扎布為扎薩克圖汗,特封和碩親王,統其眾。自是始稱扎薩克圖汗部。三十一年,定所部為喀爾喀西路。三十六年,詔歸杭愛山游牧。四十年,賜牧產贍之。尋命策旺扎布仍襲扎薩克圖汗號。

雍正四年,遣額駙策凌等赴阿爾臺勘所部與準噶爾界。九年,大軍剿噶爾丹策凌,詔所部扎薩克等內徙游牧。十年,以準噶爾敗遁,諭曰:「去歲朕降旨令爾等徙居內地,並不感悅遵行,屢次催促,始勉強遷移。今幸大軍於蘇克阿勒達呼及額爾德尼昭兩敗賊眾,爾等始得安居,否則豈能保護牲畜乎?朕思爾等本屬一體,豈有甘居庸懦受人庇廕之理。嗣後各宜激烈奮發,不惟永享升平,亦且垂光史冊矣。」

先是扎薩克圖汗策旺扎布以從征退縮罪削爵,詔郡王朋素克喇布坦子格哷克延丕勒襲汗號。十二年,調兵駐防察罕廋爾。

乾隆元年,選兵赴鄂爾坤防秋。二年,定邊大將軍平郡王福彭奏:「喀爾喀四部防秋兵皆駐鄂爾坤,扎薩克圖汗部駐牧扎克拜達哩克西南,距鄂爾坤尤邇。請即令在彼駐防,徵調無難即至。」詔如所請。五年,諭曰:「前以軍務方興,恐爾部游牧被賊侵擾,悉令內徙。今噶爾丹策凌謹遵朕旨,奏稱不敢越阿爾臺游牧,甚屬恭順。朕亦降旨令爾部游牧毋逾扎布堪、齊克慎、哈薩克圖、庫克嶺等處。爾等當遍諭所屬,永遠遵行。儻有違令生事者,嚴行治罪。況今雖許準噶爾和好,罷息干戈,而平日不可不訓習武備,爾等其留意,毋忽!」六年,命參贊大臣副都統慶泰察閱防秋兵於哈里勒邁。十三年,選駝五百運歸化城米赴塔密爾軍營。十六年,敕禁所部越境與準噶爾及回眾私市。十七年,選兵千駐防錫喇烏蘇。二十年,隨大軍進剿達瓦齊。二十二年,以其部和托輝特郡王青袞咱卜叛,誅之。尋諭扎薩克圖汗部曰:「前因青袞咱卜負恩背叛,散布流言,眾喀爾喀間有煽動。經朕訓諭,爾等旋知悔悟,各奉職守。今逆賊就誅,黨附人等應分別治罪,以彰國憲。但爾等為國家臣僕百餘年,誤聽浮言,致干罪戾,並非有心附賊,免其查究。嗣後益宜仰體朕恩,湔滌前愆,約束所屬,各安本業,綏靜邊隅,長享太平之福。」

先是扎薩克圖汗部編佐領,蕩平準、回之役,是部扎薩克郡王品級貝勒青袞咱卜、貝勒連登扎布皆以叛誅,而輔國公旺布多爾濟積俘青袞咱卜及準部叛賊呢瑪功,晉襲貝勒,予郡王品級。一等臺吉扎薩克朗袞扎布積取庫車援賊及克庫車功,晉至鎮國公。二等臺吉諾爾布以不從叛賊策登扎布,授扎薩克一等臺吉。死事於阿裏固特之二等臺吉齊巴克扎布,追封輔國公,並授其子巴圖濟爾噶勒扎薩克。其扎薩克一等臺吉噶爾丹達爾扎,以率其屬戶口自準部特穆爾圖諾爾游牧復歸,授一等臺吉,其後授其子拉克沁噶喇扎薩克,編佐領隸是部。

先是扎薩克圖汗部編佐領分十旗,後增八旗,附厄魯特一旗。扎薩克十有九,盟於扎克畢賴色欽畢都哩雅諾爾,設正副盟長各一,副將軍、參贊各一。爵二十有二:扎薩克圖汗兼多羅郡王一;附公品級三等臺吉一,由輔國公降襲;郡王品級扎薩克多羅貝勒一;扎薩克鎮國公二,一由貝勒降襲,一由扎薩克臺吉晉襲;扎薩克輔國公六,一由貝子降襲;附輔國公一;扎薩克一等臺吉八;附輔國公一;厄魯特扎薩克一等臺吉一。

乾隆四十五年,以是部扎薩克巴哈圖爾侵占杜爾伯特游牧,嚴飭查辦,促令交還。嘉慶七年十月,收扎薩克圖汗布尼喇特納等進馬五百匹。道光六年,回疆軍興,是部捐駝馬助軍需。二十五年,定扎薩克圖汗盟支差章程,王、公、臺吉等將所屬喀木齊罕阿拉巴圖等牲畜分作二分,一分牲畜津貼佐領等出差;扎薩克臺吉喀木齊罕阿拉巴圖等所有牲畜,依佐領等一律按戶扣除大牲畜一雙,餘次牲畜,均與應派佐領下人等正項差務一律出派,其貧苦臺吉佐領下喀木齊罕阿拉巴圖等各均相監之。咸豐三年,是部汗、王、公、扎薩克等以軍興捐助軍需,溫旨卻之。

同治三年,回匪陷烏魯木齊等城,古城諸城被圍,調是部蒙兵援之。五年十一月,李云麟奏扎盟蒙兵抵呼圖古蘭臺,劫掠變亂。尋潰歸。九年六月,肅州回匪擾是部境。十月,竄聚博提哈拉烏蘇、庫努克等處殺掠。十一月,匪於陷烏城後,竄金山卡倫察罕博克多地方。十一年十月,奎昌等奏移鞥克巴雅爾所部察哈爾馬隊駐扎盟察罕淖爾地方防回匪犯烏城。九月,回匪竄是部輔國公車德恩敦多布多爾濟游牧。車德恩敦多布多爾濟自備軍裝軍火糧餉,督臺吉官兵,於十六、十七日再挫匪於景色圖及巴彥察汗地方,匪向西遁。事聞,予貝子銜。十二月,擾科城之回匪竄聚於扎部南境,奎昌派達爾濟帶隊攻剿。

十二年正月,奎昌等奏回匪於十一月竄扎盟所屬之那瑪勒吉幹昭地方,官軍於是月十一日進攻敗之,匪即北竄。追剿至十二日,匪又向察罕布爾噶奔竄,山勢險隘,負固相持。達爾濟趕帶馬隊前進,匪又越山遁聚巴里坤、扎盟交界地方。二月,烏里雅蘇臺將軍長順等以扎盟牧南各旗毗連肅州,屢被回匪擾害,奏暫移公棍楚克扎布、右翼前扎薩克桑青齊蘇隆、右翼後末瑪呢達拉等旗於本部扎薩克圖汗及右翼中參贊公密帕散布、中右翼末旗達什拉布坦、扎薩克車德恩多爾濟等旗游牧,扎薩克圖汗旗移本部落右翼左公銜扎薩克班扎班咱爾扎布、右翼末次扎薩克達散巴拉等旗游牧。俟賊匪肅清,即令各歸舊牧。下所司知之。十月,回匪竄擾圖謝公游牧,旋擾察幹河及莫爾根地方。長順等遣卓凌阿剿匪於圖謝公游牧之庫布奇爾果羅地方,勝之,救出蒙古男婦子女一百九十餘名。科布多所遣防禦喜莫得等率兵敗匪於阿育爾公旗庫倫喇嘛地方,救出被脅蒙民男婦三四百名。會棟呢特多爾濟軍敗之於烏蘭壩,匪向鞥克扎薩克旗以南逾山逃遁。十三年三月,予扎薩克圖汗等捐助烏里雅蘇臺城獎。

光緒初,烏魯木齊諸城克復,是部始解嚴。七年,徵是盟兵戍科布多。俄約成,撤去。二十一年,是部以甘肅回匪滋擾,文報改由臺路,撤回邊界游牧牲畜,為堅壁清野之計。二十三年,烏里雅蘇臺將軍崇歡等劾盟長扎薩克鎮國公阿育爾色德丹占扎木楚假公攤派,請革職,允之。二十四年,是部與賽音諾顏部王、公、扎薩克等輸昭信股票銀,並請報效,仍予獎。二十五年,是部扎薩克蘊多爾濟旗與科布多之扎哈沁爭界,志銳等奏所爭一為巴爾嚕克鄂博,一為鞥吉爾圖鄂博,一為田德克庫與喀拉占和碩界線,請飭理籓院秉公剖斷,允之。二十六年,拳匪肇釁,邊防戒嚴,是盟王、公、扎薩克等於徵兵籌餉均得出力。二十八年,予扎薩克圖汗索特那木拉布坦、副將軍扎薩克輔國公洛布桑端多布獎有差。宣統二年,索特那木拉布坦為資政院欽選議員。三年,庫倫獨立,脅是部汗、王等附之。

是部有礦,有鹽。佐領有二十一。


\end{pinyinscope}