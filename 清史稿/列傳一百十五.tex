\article{列傳一百十五}

\begin{pinyinscope}
常青藍元枚蔡攀龍梁朝桂普吉保丁朝雄鄂輝舒亮

常青,佟佳氏,滿洲正藍旗人。父安圖,官至江西巡撫。常青自寧郡王府長史累遷察哈爾都統,杭州、福州將軍。乾隆五十一年,署閩浙總督。諸羅縣民楊光勛與其弟爭家業,糾眾立會,縣吏捕治不服,常青令按察使李永祺往按。上以臺灣在海外,不可輕縱,諭勿使蔓延疏脫。尋實授閩浙總督。十二月,林爽文亂起,陷彰化,知縣俞峻死之。常青檄水師提督黃仕簡自鹿耳門進,副將丁朝雄從海壇鎮總兵郝壯猷自淡水進,都司馬元勛屯鹿仔港,分道部署;復如泉州會陸路提督任承恩調度,令金門鎮總兵羅英笈詣廈門彈壓。尋復令承恩自鹿耳門繼進。五十二年,奏賊陷諸羅。臺灣鎮總兵柴大紀堵剿,賊勢稍沮。爽文漳州人,其徒率漳籍。移會兩廣督臣防範,上責其張皇。授李侍堯閩浙總督,而移常青湖廣。

既又命常青渡臺視師,四月,至臺灣。劾仕簡、承恩遷延觀望,擁兵自衛;壯猷守鳳山,賊至,棄城走。諭逮承恩,罷仕簡候命,而誅壯猷,遂授常青為將軍。賊攻府城,常青督諸軍御戰,有所俘馘;賊攻桶盤棧,令游擊蔡攀龍等分駐力禦。奏入,上以常青年逾七十,能如此勇往督戰,手詔嘉獎,授其子刑部筆帖式喜明三等侍衛,馳驛往省,並賜御用搬指。旋奏爽文還大里杙舊巢,其徒莊大田等萬餘人分擾南路,擬先南剿大田,乃北取爽文。上韙之,下部優敘。旋奏剿賊南潭,殲賊六百餘;爽文之徒莊錫舍出降,擒偽軍師番婦金娘,請檻車送京師,上命授錫舍守備。又奏進剿鳳山,出城未十里,賊三面並進,官兵奮勇擊退;賊勢蔓延,請厚集兵力,遣大臣督戰。上命陜甘總督福康安往視師。旋奏:「賊犯府城,為丁朝雄擊退。官軍攻莊大田於南潭,殺賊二百餘。大營距府城未遠,勢相犄角,無後顧之虞。」得旨嘉獎,賜雙眼孔雀翎。旋迭奏鹽水港、笨港均為賊據,糧道既斷,諸羅勢甚危;令總兵魏大斌赴援,戰賊失利,又令游擊田藍玉援大斌。上以兵分力薄,飭常青調度失當。又諭:「常青駐軍桶盤棧,距南潭不過五里,不將賊目莊大田先行剿除,乃結營自守。肘腋之間,任其逼處。」

八月,命福康安為將軍,督諸將海蘭察、普爾普等大出師討爽文。諭常青,謂:「非責其師無功,特以年已七十,軍旅非所素習。福康安未至,仍當相機進剿。」旋奏:「賊自南潭來攻,侍衛烏什哈達等擊敗之。因雨後路滑,收兵;又進攻南潭,焚草藔數百間,以天晚,山徑偪仄,不便深入。」藔謂賊所居草屋也。上以其屢稱遇雨路仄收兵,傳旨嚴飭。上又聞賊詗知軍中暑濕多病,常青機事不密,又不督兵深入,屢詰責。旋奏總兵梁朝桂剿賊多斬獲,提督柴大紀報諸羅圍急,令副將蔡攀龍赴援。上諭令親援大紀,待福康安至,合軍進攻。旋奏同江寧將軍永慶等在竹篙厝等處殲賊甚眾;山豬毛社義民尤趫捷,獲砲一,生擒賊目張招。又奏總兵普吉保克月眉莊,距諸羅五里,令與大紀並力固守;又令諸生劉宗榮等給番社土目札諭防賊竄匿。屢得旨嘉許。

福康安渡臺灣。上授常青福州將軍,留辦善後,令從將軍職戴單眼孔雀翎。福康安劾大紀貪劣狀,上責常青徇隱,奪職,交福康安嚴鞫。福康安旋以常青自承徇隱,請交部治罪,上特宥之。召詣京師,署鑲紅旗蒙古都統。五十四年,授禮部尚書、鑲藍旗漢軍都統。五十八年,卒,謚恭簡。子喜明,官至徐州鎮總兵。

常青初視師,福州將軍恆瑞,水陸二提督任承恩、黃仕簡皆在行,戰無功。承恩、仕簡以誤軍機坐斬,臺灣平,赦出獄。仕簡至狼山鎮總兵,承恩亦至副將,恆瑞自有傳。

藍元枚,字簡侯,福建漳浦人,提督廷珍孫。父日寵,官福建銅山營水師參將。元枚襲三等輕車都尉世職。乾隆三十一年,命發廣東,以外海水師參將用,補海門營參將。累遷總兵,歷臺灣、金門、蘇松三鎮。四十九年,授江南提督。五十二年正月,臺灣民林爽文為亂,命元枚馳驛往泉州,署福建陸路提督,駐蚶江策應。至福州,奏言:「師渡臺灣,亂民潰散,慮入內山與生番勾結。」上諭令速捕治,俾盡根株。水師提督黃仕簡率兵討爽文,坐逗留奪官,以命元枚,並賜孔雀翎,授參贊,趣率兵渡鹿仔港,會總督常青進討。六月,元枚率兵次鹿仔港,與總兵普吉保師會,即夜,師分道自柴坑仔、大武隴入,殺賊甚眾。上嘉之,賜雙眼孔雀翎。

元枚所將止浙江兵二千,奏請益師,上命總督李侍堯發福建兵二千、廣東兵三千益元枚。時總兵柴大紀堅守諸羅,元枚使告大紀,期會兵攻斗六門。戰阿棟社,戰埤頭莊、大肚溪,屢殺賊。復進攻西螺,焚條圳塘、中浦厝諸地賊莊。元枚族人啟能等七十九人自賊中出,使為導。元枚奏聞,並言如察出啟能等已從賊,當立誅。上嘉其公當,賜緙絲蟒袍、上佩荷包,並諭:「啟能等既來歸,前此已否從賊,不須追詰。」諸羅被圍已兩月,大紀屢就告急,上屢趣元枚赴援,諭:「廷珍平硃一貴,七日而事定。元枚當效法其祖,毋負委任。」七月,元枚病作。八月,賊自竹子腳、大肚溪、柴坑仔三道來攻。元枚力疾出戰,病益劇,越十日,卒於軍,贈太子太保,發白金千兩治喪,賜祭葬,謚襄毅。元枚謚同廷珍,時稱小襄毅以別之。

蔡攀龍,福建同安人。自行伍屢遷至福建澎湖右營游擊。乾隆五十一年,林爽文為亂,巡撫徐嗣曾檄詣軍。五十二年,賊破鳳山,總兵柴大紀令督兵捕治。賊攻臺灣府城,攀龍出戰,屢破賊。賊屯西園莊,攀龍率諸將瑚圖裏、丁朝雄分道攻之,殺賊三百。賊復攻府城,總督常青令攀龍率諸將孫全謀、黃象新等御戰。賊乘東、南二門,攀龍等力戰,殺賊數百,奪九節砲。論功,擢北路協副將,賜孔雀翎。賊復至,攀龍督戰,復殺賊三百餘,予強勝巴圖魯名號。七月,常青令攀龍援柴大紀諸羅,上命授海壇鎮總兵。攀龍師至鹽水港,分八隊以進。雨大至,賊乘雨合圍,諸將貴林、楊起麟、杭富皆戰死。會大紀以師來迎,攀龍及全謀兵不及千人,偕運餉民三千人入諸羅,復出城殺賊。總督李侍堯聞攀龍兵達諸羅,未知貴林等戰死狀,謂諸羅圍已解,入告。上擢攀龍陸路提督,參贊軍務,貴林、起麟、全謀並遷官。俄,侍堯復疏陳,上命恤戰死諸將。

福康安既解嘉義圍,疏劾大紀,因言攀龍軍嘉義西門外,並無出城殺賊事,自請奪職。擬請令還海壇本任。上謂攀龍屢戰有功,其過尚可寬。五十三年,逮大紀治罪,移攀龍水師提督。師攻大武隴,令攀龍駐灣裏溪。爽文既擒,其弟勇及賊渠莊大田猶窺伺府城,攻灣裏溪,圖斷府城道。福康安遣攀龍分道進攻,頗有斬獲。事平,圖形紫光閣,列前二十功臣,上自為贊,許為臺灣戰將中巨擘。師還,諸將言攀龍平庸,福康安亦言未能勝任,左遷江南狼山鎮總兵。嘉慶三年,卒。

梁朝桂,甘肅中衛人。乾隆三十七年,以中衛營外委從征金川,先後攻克路頂宗、布朗郭宗及功噶爾拉、丫口、昔嶺、阿喀木雅。三十九年,克淜普,進攻喇穆喇穆山梁,奪日丫口。四十年,剿勒吉爾博寨,先登被創。四月,攻木思工噶克山,潛師入,盡克其城碉,據康薩爾至丫口山。十月,克西里山。錄功,賜孔雀翎。累遷陜西潼關協副將。金川平,列五十功臣,圖形紫光閣。累遷甘肅肅州鎮總兵。坐事罷。復起,自福建福寧鎮移廣東高廉鎮。

五十二年,臺灣林爽文為亂,莊大田應之,別為南路賊。朝桂率兵敗大田於蔦松,斬馘二百餘。賊眾數千犯大營,擊卻之,斃賊三百。將軍常青慮南路賊北擾諸羅,檄朝桂堵御,連敗之南潭、中洲、十三里莊,殲數百人。九月,常青移師北路剿爽文,以朝桂守臺灣府城,賊來犯,擊走之。其冬,援參贊恆瑞於鹽水港,毀賊藔,賜號奮勇巴圖魯;復同恆瑞自鹿仔草進剿鎮平莊,受創,力戰敗賊。時提督柴大紀被圍諸羅急,朝桂欲馳援,恆瑞不聽,大紀以聞,帝令將軍福康安察奏。會福康安抵鹿仔港,檄朝桂仍駐守鹽水港及鹿仔草。

五十三年春,就擢福建陸路提督。檄剿麻豆莊、大武隴屯賊,通郡城要道。大田時據大武隴拒守,朝桂自茅港尾繞至阿里港迎截;復赴打狗、竹仔各港口截其走路。大田力不支,自牛莊竄極南之郎嶠,負山阻海。福康安自風港進至柴城,分六隊直逼海岸,與朝桂環攻之,大田及他賊目四十餘悉就擒。臺灣平,再圖形紫光閣。金門巡洋艦被劫,以朝桂不能戢盜,移廣西。再移湖廣。卒。

普吉保,札庫塔氏,滿洲正黃旗人。乾隆三十年,以藍翎侍衛從軍征烏什,有功,補三等侍衛。三十七年,從參贊大臣舒常攻日旁,有功。三十九年,從副將軍豐升額攻凱立葉山,進抵迪噶拉穆札山。賊分三隊,普吉保偕侍衛瑪爾占等夾攻,斃賊無算,賜沖捷巴圖魯名號。四十年,攻噶爾丹寺諸地,連破木城、石碉。上獎普吉保勇往,累擢福建汀州鎮總兵。林爽文為亂,總督常青檄普吉保會剿,五十二年,率水師渡臺灣,迭破賊鹿仔港、八卦山,上嘉其奮勉。爽文見師至,退守斗六門、大里杙。普吉保以師進,爽文攻諸羅,赴援,抵笨港,率游擊海亮等殲賊數百,毀賊莊七,得旨嘉獎,賜玉搬指、荷包、蟒袍。笨港潰賊糾眾截我兵,普吉保擊斬甚眾。嗣以駐兵元長莊、月眉莊不進,旨嚴飭。尋攻大埔林,收復斗六門。爽文竄內山,普吉保從諸將徒步陟山搜捕。五十三年,以兵扼科仔坑口,合圍,俘爽文。南路莊大田亦就擒。臺灣平,圖形紫光閣。普吉保初克鹿仔港,以福康安疏薦,授臺灣總兵。明年,上念臺灣初定,慮普吉保不能勝,命解任。尋授廣西左江鎮,坐責把總黎振乾投水死,戍伊犁。卒。

丁朝雄,字伯宜,江蘇通州人。自行伍累擢福建臺灣水師副將。乾隆五十一年,以任滿赴部引見,至省城,聞林爽文亂起。朝雄策東港與鳳山犄角,爽文所必爭,白總督常青,請兵屯東港,斷其糧道。常青不能用,遣朝雄還臺灣,佐海壇鎮總兵郝壯猷討爽文。

五十二年春,壯猷偕朝雄率兵二千餘擊賊,馘三百,俘二十五。日將暮,賊復來攻,朝雄復殺賊百餘,賊始去。攻鳳山,朝雄乘東門,首諸軍入,鳳山遂復。黃仕簡檄朝雄守安平海口。賊攻府城,朝雄偕知府楊廷樺督兵民力禦。賊攻桶盤棧,朝雄為前鋒,出戰,臺灣道永福、同知楊廷理率兵民繼,復殺賊百餘,賊敗走。冬,朝雄偕游擊倪賓率兵千二百、義民二千餘攻東港。東港賊數萬,其渠吳豹以海岸淺,度舟不能至,不為備。朝雄遣諜以水注賊砲,乘雨至水漲,遣兵民分道登岸殺賊,俘豹。以兵寡不能克,報常青請益兵。常青令駐港口護餉道。既,令攻竹仔港,毀賊舟。

五十三年春,復攻東港,仍遣諜以水注賊砲,督兵攻渡口,賊驚竄,逐三十餘里,乃倚山而軍。賊夜來犯,朝雄戒勿動;及曉,賊倦,掩擊,大破之。爽文遣其徒來援,朝雄築壘困之。賊潰圍出,設伏斷其歸路,而自將追之,大破賊,遂復東港。福康安上其功,授海壇鎮總兵。既,福康安劾柴大紀受陋規,言朝雄為安平協副將時亦有此,當奪職戍軍臺,上以朝雄攻東港戰有功,命留任。林髟剌舵、林明灼者,海盜渠也,五十四年,朝雄巡洋至汜澳,破盜巢,得鬎舵等;而明灼拒殺參將張殿魁。上責總督伍拉納,伍拉納以屬朝雄,督舟師出海,遇諸大麥洋,俟其近,發大砲,斃數酋,明灼窮蹙,躍入海,官軍鉤致,俘以歸。

五十五年,追論朝雄在臺灣失察天地會邪教,當奪職;上諮伍拉納朝雄在官狀,伍拉納言朝雄督水師捕盜有勞,命還任。五十八年,攝水師提督。五十九年,入覲,至清江浦,病篤。乞罷歸,卒於上海舟中。

鄂輝,碧魯氏,滿洲正白旗人。自前鋒分發四川試用守備。七遷建昌鎮總兵。從大學士阿桂定蘭州回亂,予法什尚阿巴圖魯名號。再遷成都將軍。乾隆五十二年,署四川總督。將軍福康安討臺灣亂民林爽文,上命鄂輝率四川屯練降番濟師。尋授參贊,從渡海援嘉義。鄂輝屯東莊溪橋,攻克牛稠山竹柵,嘉義圍解。逐賊至大排竹,殲之。師攻斗六門,賊自山下撲,鄂輝督兵沖截,賊奔逸,攻克大埔林、大埔尾二莊,賊潰。爽文自所居大里杙奔內山番界,鄂輝逐之至集埔。五十三年春,詗知爽文所匿地曰東勢角,福康安督鄂輝及舒亮追捕,自歸仔頭至麻著社,分軍,鄂輝自撲仔離東山路進,舒亮直取東勢角。是役遂俘爽文,亂乃定。上命臺灣嘉義立諸將帥生祠,鄂輝與焉。師還,圖形紫光閣,賜雙眼孔雀翎、雲騎尉世職。鄂輝朝熱河行在。

廓爾喀侵西藏,據濟嚨、聶拉木諸地。上促鄂輝還四川,與提督成德帥師赴援,又命侍郎巴忠往按。巴忠先嘗為駐藏大臣,習藏事,示意噶布倫,令賂廓爾喀返侵地。鄂輝等遂與議和,疏陳善後事。尋授四川總督。五十六年,廓爾喀渝盟,復侵濟嚨、聶拉木諸地。上命將軍福康安督師討廓爾喀,責鄂輝誤用巴忠議致復生事,奪官,予副都統銜駐藏,聽福康安指揮,福康安令督餉。工部尚書和琳劾鄂輝得廓爾喀貢表不以上聞,命奪副都統銜,逮赴前藏荷校示罰。五十八年,命還京師,授拜唐阿。加員外郎銜,遷熱河總管。

嘉慶初,命以侍衛詣荊州從剿教匪,戰有功,以都統銜加太子少保,授湖南提督。屢破賊,與額勒登保等攻克石隆山,斬賊渠石柳鄧,封三等男。二年,擢雲貴總督。三年,卒,謚恪靖,祀賢良祠。四年,追論在湖北軍中受餽白金四千,罷祀。

舒亮,蘇佳氏,滿洲正白旗人。自前鋒累遷參領。師征金川,舒亮從副都統齊里克齊率健銳營為裨將。攻穆谷,舒亮伏山下待賊,殺賊甚眾。攻卡角,賊匿山溝,舒亮於密箐中望見火光,以火器就擊之,賊驚潰。以功,累遷鑲黃旗滿洲副都統。從克噶拉依,賜穆騰額巴圖魯名號。師還,圖形紫光閣。乾隆四十六年,大學士阿桂討撒拉爾亂回蘇四十三,舒亮從。初至,破賊華林山。賊掘濠設卡以自固。阿桂令海蘭察自山西攻賊卡,舒亮自南山進,當賊鋒,賊競出,射舒亮,傷左股,舒亮拔箭裹創,復戰,奪賊卡四,殺賊百餘。又與海蘭察詗賊不備,以土囊填濠渡軍,殲守濠賊,復奪十餘卡。蘇四十三既誅,復剿華林寺餘匪。事平,還京師。

林爽文之亂,福康安出視師,舒亮以正黃旗護軍統領為領隊大臣。至臺灣,福康安軍道笨港救嘉義,令舒亮出別道分賊勢。賊方據北大肚山拒我,舒亮迎擊,敗之,連破南大肚、王田、瀨湑、半山、坑子諸莊,遂克烏日莊。會福康安軍夾擊,解嘉義圍。五十三年,爽文竄匿東勢角。福康安督舒亮等追逮,令舒亮直取東勢角,山徑峻險,將卒皆步上,殺賊二千餘。爽文復走老衢峙,舒亮督諸軍急進,獲之,亂遂定。

上以臺灣遠在海外,主客民雜處,風俗素悍,命於府城及嘉義立諸將帥生祠,示威德。祠成,命並及在事疆吏,首福康安,次海蘭察、李侍堯、普爾普、鄂輝、徐嗣曾,而以舒亮殿焉。尋授鑲紅旗蒙古都統。師還,命監爽文及其徒賴大等生致京師。賴大道病,舒亮令誅之,不稱上意,命仍為護軍統領。敘功,予雲騎尉世職,再圖形紫光閣。出為荊州、黑龍江將軍。在黑龍江,坐私市貂皮,奪官,削世職。

川、陜、楚教匪起,命以三等侍衛從軍。嘉慶元年,戰襄陽,再戰劉家集,屢俘斬賊渠。攻當陽,先登,額中槍,奮進,殺賊千餘,獲其酋,遂克當陽,賜孔雀翎,授鑲藍旗漢軍副都統。賊自鍾祥分竄唐、鄧,設伏呂堰驛,西竄賊殲焉;乃合兵逐東竄賊,戰草店,復中槍,賚銀絲盒、荷包。旋以縱賊渡滾河,奪孔雀翎、巴圖魯。二年,坐賊渡漢江,降三品頂戴。三年,復以總督勒保劾剿賊不力,奪官,以兵丁留軍。尋卒。

論曰:林爽文亂起,常青及福州將軍恆瑞並水陸二提督,躬率師東渡,徘徊坐誤。高宗爵柴大紀,誅郝壯猷,欲以激勵諸將;繼以元枚代,功未竟而卒,終煩禁旅,始克底定。承平久,水陸諸軍不足用,不得獨為大紀罪也。鄂輝、舒亮從福康安出師,與攀龍、朝雄皆有戰績;然大紀力保危城,當時聲譽遠出諸將上。功名之際,有幸有不幸,固如是夫!


\end{pinyinscope}