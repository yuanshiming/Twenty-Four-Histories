\article{列傳一百二十九}

\begin{pinyinscope}
保寧松筠子熙昌富俊竇心傳博啟圖

保寧,圖伯特氏,蒙古正白旗人,靖逆將軍納穆札勒子。乾隆中,納穆札勒殉節回疆,錫封三等公。

保寧由親軍襲爵,授乾清門侍衛。從征金川,力戰,迭克要隘,將軍阿桂薦其才,擢陜西興漢鎮總兵。金川平,繪像紫光閣,禦制贊,褒其膽勇持重,少年如宿將。尋調河南南陽鎮、直隸馬蘭鎮,兼總管內務府大臣。擢江南提督。

四十九年,授成都將軍。甘肅石峰堡回叛,命選屯練番兵赴鞏昌、安定助剿,平之。五十一年,授四川總督。保寧謹慎有操守,盡心邊事。邊夷上下孟董、九子等寨生齒日繁,請增設營員,以屯練有勞績者拔補;改修打箭爐城,扼要築卡,駐兵捍衛;改黃梁、大定、白雞、白鹿等八寨熟苗編入民戶:並協機宜。

次年,調伊犁將軍,兼內大臣,籌備倉儲。疏言:「伊犁一年支糧十六萬六千餘石,不敷二萬三千石,歷就舊儲五十餘萬石內填補。現賸三十餘萬石,雖尚可敷十餘年之用,地處極邊,若不補籌餘糧,偶遇歉收,或有需糧之事,慮難接濟。請撥兵丁七百名,增開七屯,自來年耕種,歲可收糧一萬九千餘石,永遠備貯。」從之。又奏添設惠遠城鳥槍步甲四百名。五十五年,入覲,途次命赴四川暫署總督事。次年,回任,加太子少保,授御前大臣。惠遠城創立三十餘年,戶口日繁,於城東展築,擴舊城四分之一。伊犁無通曉俄羅斯語言者,請於京師俄館選派一人來教習官兵子弟,五年期滿,試最優者充筆帖式。俄屬烏梁海潛往哈屯河外汗山地方游牧,帝慮其滋事,命保寧察視,疏言:「烏梁海居住甚安戢,不必驅逐,飭邊卡防範,無庸添兵。」察哈爾兵丁及土爾扈特私竊哈薩克馬匹,緝獲,置之法。帝嘉保寧無偏袒,得外籓心,予議敘。

六十年,召授吏部尚書,兼鑲黃旗漢軍都統,甫數月,復出為伊犁將軍。嘉慶二年,協辦大學士,尋拜武英殿大學士,加太子太保,任邊事如故。土爾扈特家奴三吉污主母孀婦伯克木庫殞命,特詔予伯克木庫旌表。保寧疏陳駐防孀婦守節,未舉旌表之典,請照內地一體辦理。於是採訪各城,請旌者凡七十人,後著為令。七年,召還京,授領侍衛內大臣,管理兵部,兼管三庫。八年,因孝淑皇后山陵典禮會疏措詞不經,褫銜鐫級留任。

保寧兩鎮伊犁,歷十餘年,西陲無事,籓部悅服。既去任,朝廷遇邊疆興革,每諮決焉。十一年,以疾乞休,命在家食公爵全俸。逾兩年,卒,賜金優恤,謚文端,祠祀伊犁。

子慶祥嗣爵,殉回疆之難,自有傳。次子慶惠,由廕生授侍衛,歷官侍郎,三以罪黜復起。道光中,官至熱河都統,以疾歸,卒,謚勤僖。

松筠,字湘浦,瑪拉特氏,蒙古正藍旗人。繙譯生員,考授理籓院筆帖式,充軍機章京,能任事,為高宗所知。累遷銀庫員外郎。乾隆四十八年,超擢內閣學士,兼副都統。

五十年,命往庫倫治俄羅斯貿易事。先是,俄屬布哩雅特人劫掠庫倫商貨,俄官不依例交犯,僅罰償,流之遠地,檄問未聽命,詔停恰克圖貿易。松筠至,尋充辦事大臣。閉關後,邊禁嚴而不擾,遇俄人皆開誠待之。擢戶部侍郎。俄羅斯以貿易久停,有悔意,撤舊官,屢請開市,未許。卡倫兵出巡,復為布哩雅特人所殺。松筠曰:「舊事未了,又生旁支,然亦了事之機也。」檄俄官縛送三人,親訊於界上,斬其二,流其一,請兩案並結。詔斥專擅,褫職,仍留庫倫效力。會西路土爾扈特喇嘛薩邁林者,迷路入哈薩克,歸攜書信,訛言俄人誘致土爾扈特謀亂,下松筠察狀。疏言俄羅斯實恭順,無可疑。俄人亦自陳證薩邁林書信出偽造。詔置薩邁林於法,許復開市。五十七年,召俄官會議定約,親蒞俄帳宴飲,諭以恩信,大悅服。事歷八年然後定。召還京,授御前侍衛、內務府大臣、軍機大臣。命護送英吉利貢使回廣東,凡所要索皆嚴拒。

五十九年,署吉林將軍。尋命往荊州察稅務,道出衛輝,大水環城,率守令開倉賑恤。詔嘉獎,授工部尚書兼都統。充駐藏大臣,撫番多惠政。和珅用事,松筠不為屈,遂久留邊地。在藏凡五年。

嘉慶四年春,召為戶部尚書。尋授陜甘總督,加太子少保。時教匪張漢潮及藍號、白號諸黨擾陜、甘。松筠至,駐漢中,治糧餉給諸軍。自軍興,給陜西餉銀一千一百萬兩,至是續撥一百五十萬,設局清釐,按旬咨部。命陳諸將優劣,密疏言:「明亮知兵而罔實效;恆瑞前戰湖北功最,年近六旬,精力大減;慶成有勇無謀;永保無謀無勇,不能治兵,並不能治民;惟額勒登保、德楞泰能辦賊。」仁宗深嘉納之。明亮劾永保、慶成避賊,下松筠逮治。永保亦與荊州將軍興肇訐明亮誑報軍功,詔並褫職,遣尚書那彥成赴陜會鞫。會明亮已擊斃張漢潮,松筠請緩其獄,又請留撒拉爾回兵,令慶成率以協剿,帝不允。既而那彥成劾心互瑞棄藍號垂盡之賊,折回陜西,由松筠所誤。詔褫松筠宮銜、侍衛,仍留總督任。川匪犯南鄭,復分犯西鄉、沔縣、略陽。松筠素謂匪多脅從,可諭降,欲單騎赴之。副將韓嘉業固諫曰:「諭之不從而喪總督,大損國威,為天下笑。請先往。」嘉業果被害。賊竄徽縣、兩當。五年春,額勒登保、那彥成會剿,乃分路遁。於是命長麟代為陜甘總督,授松筠伊犁將軍,未之任,暫署湖廣總督。自請入覲面陳軍事,先在陜上疏言:「賊不患不平,而患在將平之時。既平之後,請弛私鹽、私鑄之禁,俾餘匪散勇有所謀生。」帝以其言迂闊,置之。至京,復以為請,忤旨,降副都統銜,充伊犁領隊大臣。

七年,擢伊犁將軍。乾隆中屢詔伊犁屯田,皆以灌溉乏水未大興,松筠力任其事,預計安插官兵。惠遠城需八萬畝,惠寧城需四萬畝,乃於伊犁河北引水開渠,逶迤數十里,又於城西北導水泉。凡兩城有水之地皆開渠,授田為世業,給穀種、田器、馬牛。然旗人多驕逸,或殺食所給牛,鬻田器棄不耕,反覆曉諭始聽命。比去任,凡墾田六萬四千畝。寧遠叛兵蒲大芳等譴戍塔爾巴哈臺,其黨馬友元等分戍南路諸城。十三年冬,大芳復謀逆,捕其黨五十餘人誅之。次年,檄調馬友元等百餘人赴伊犁種地,悉斬於途。詔斥未鞫而殺,失政體,降喀什噶爾參贊大臣。復授陜甘總督。

調兩江總督。南河自馬港口墊陷,黃水倒漾,淤運阻漕。偕河督吳璥察勘海口,請復故道。制疏沙器具,試之河口果驗;又造撥船千艘,改小運船,親駐河干督趲,渡黃回空皆迅速。迭疏論河務,宜引沁入衛,可利漕運。又謂吳璥於黃泥嘴、俞家灘逢灣取直,以致停淤,為璥等論駁。復密陳吳璥、徐端所論不實,工程虛捏,自請調任總河察其弊,又薦蔣攸銛、孫玉庭可任。帝以松筠忠實,治河非所長,用攸銛為河督,責令相助為理。尋兼署河督事。十六年,調兩廣總督,協辦大學士,兼內大臣。召為吏部尚書。

十七年,命往盛京會勘陵工,兼籌移駐宗室事,疏請小東門外建屋七十所,居閒散宗室七十戶,戶給田三十六畝。又言:「西廠大凌河東有可耕地三千頃,可移駐二千餘戶。東廠周數百里,地多積水,其水自北山柳條邊來,若相地開河,可涸出沃壤;又東柳河溝亦多積水,若自北山東橫開大渠,可得沃壤數千頃。」「續勘彰武臺邊門外迤西牧廠閒地,橫三四十里,縱六七十里,並可移駐。請於大凌河西廠東界先試墾種。」詔並允行。而試墾事為將軍晉昌奏罷,論者惜之。回京,授軍機大臣。未幾罷,改授御前大臣。

十八年,復出為伊犁將軍,拜東閣大學士,改武英殿大學士。以平定滑縣教匪,敘功,加太子太保。詔偕參贊長齡通籌新疆南北諸城出納,量減內地饋運。疏言:「北路塔爾巴哈臺歲需內地銀四萬數千兩,南路回疆八城歲需內地銀五萬數千兩,地方貢賦皆入經費之內,無庸議減。伊犁歲需內地經費銀六十萬兩,可撙節者無幾。惟烏魯木齊為新疆腹地,歲需銀一百一十餘萬兩,宜裁減。請復屯田,廣墾蘆灘荒地,開採銅鉛各礦,抽收迪化州、吐魯番木稅。」又議綠營糧餉,凡倉儲充裕處,改給銀米各半,並復乾隆四十六年以前捐監之例,使邊地就近納粟。所議或行或不行,於內地歲輸卒未大減。

喀什噶爾阿奇木伯克玉努斯聽其妻色奇納言,多不法,私與浩罕酋愛瑪爾交通。愛瑪爾欲使尊為汗,遣使請自設哈子伯克,用浩罕稅例徵安集延商。十九年,松筠巡視回疆,誅色奇納,械玉努斯,禁錮伊犁;拒浩罕之請,斥去其使。二十年,喀什噶爾回人仔牙敦作亂,親往治之。仔牙敦就獲,與布魯特比圖爾第邁莫特並置極刑。詔斥松筠不待命,削宮銜,召還京。松筠初任時,築四堡於伊犁河北,議移置八旗散丁,事未竟而去。再至,乃築室堡中,堡置百戶,戶授田三四十畝,三時務農,冬則肄武。規畫粗備,以屬代者,而代者不置意,田遂荒。

二十二年,詔來年幸盛京,抗疏諫阻,罷大學士,出為察哈爾都統,署綏遠城將軍。逾年,子熙昌歿,帝憐之,召還為正白旗漢軍都統。尋授禮部尚書,調兵部,復御前兼職。未幾,出為盛京將軍。松筠素以忠諒見重,在朝時,凡燕游執御之事,乘間直言無避。既屢忤旨,二十五年,以兵部遺失行印,追論,降山海關副都統。復以事,迭降為驍騎校。是年秋,仁宗崩於熱河,梓宮回京,宣宗步行於班僚中見之,扶而哭,翌日授左副都御史,擢左都御史。其復起也,甚負時望,然卒不安於位,未一月,出為熱河都統。

道光元年,召授兵部尚書,調吏部,復為軍機大臣。二年,暫署直隸總督。以代改理籓院奏稿,忤尚書禧恩,被劾,降六部員外郎。尋授光祿寺卿,遷左都御史。又出為盛京將軍,調吉林。數年之中,兩召還朝,為左都御史、禮部尚書;迭出署烏里雅蘇臺將軍、熱河都統、直隸總督。九年,調兵部尚書,往科布多鞫獄。十年,往山西按巡撫徐炘被控事。回疆方用兵,密疏有所論列,詔令陳善後方略,多被採納。是年秋,自以衰病請罷,數日復請任使,詔斥進退自由,負優禮大臣之意。又以前赴科布多囑道員徐寅代購什物,罷職,予三品頂戴休致。

至十二年,浩罕遣使進表,松筠曾言浩罕通商,邊境可靖,帝思其言,復頭品頂戴,署正黃旗漢軍副都統。命赴歸化城勘達爾漢、茂明安、土默特三部爭地,據乾隆朝圖記判定,三部皆悅服。還,授理籓院侍郎,調工部,進正藍旗蒙古都統。十四年,以都統銜休致。逾年,卒,年八十有二,贈太子太保,依尚書例賜恤,謚文清,祀伊犁名宦祠。

松筠廉直坦易,脫略文法,不隨時俯仰,屢起屢蹶。晚年益多挫折,剛果不克如前,實心為國,未嘗改也。服膺宋儒,亦喜談禪。尤施惠貧民,名滿海內,要以治邊功最多。

子熙昌,以廕生官至刑、工兩部侍郎,署熱河都統兼護軍統領。數奉使赴各省按事,亦被信用。嘉慶二十三年,卒於長沙,帝深惜之,贈都統,謚敬慎。

富俊,字松巖,卓特氏,蒙古正黃旗人。繙譯進士,授禮部主事,歷郎中。累遷內閣蒙古侍讀學士、內閣學士,兼副都統。嘉慶元年,擢兵部侍郎,充科布多參贊大臣。四年,授烏魯木齊都統,調喀什噶爾參贊大臣。歷葉爾羌辦事大臣、烏里雅蘇臺參贊大臣。召署鑲紅旗漢軍都統、兵部侍郎。

八年,出為吉林將軍,調盛京。清治民典旗地,限年首官,不首者治罪,追典價租息入官。富俊疏言:「一年之內,一千六百餘案,應追繳者不下萬人,年久轉典,株連繁多。旗、民多窮苦,既獲罪,又迫追呼,情實可憫,請悉寬免。」允之。十二年,考覈軍政,以潔己奉公,邊陲安輯,特詔褒美,予議敘。十五年,因採葠攙雜,受屬員蔽,褫職,遣往吉林效力。既而言官論關東三省賭博風熾,仁宗念富俊在官時曾嚴禁,即起授盛京工部侍郎,兼管奉天府尹及六邊邊門事務。十八年,授黑龍江將軍,疏請內外臣工三年更調,及禁奢、講武數事,詔以更調非可限年,餘並嘉納。又以東三省官兵技藝優嫺,每屆五年挑送京營,著為令。

十九年,調吉林將軍。先是,議籌八旗生計,詔勘吉林荒地開墾,移駐京旗,將軍賽沖阿言拉林近地閒荒可墾,未有規畫。富俊至,疏言:「乾隆中移駐京旗,建屋墾地,多藉吉林兵力,墾而不種,酌留數人教耕,一年後裁汰。京旗蘇拉不能耕作,始而雇覓流民,久之田為民有,殊失國家愛育旗人之意。今籌試墾,莫若先辦屯田。請發吉林閒散旗人一千名為屯丁,每丁給銀二十五兩、耔種二石,官置牛具,人給荒地三十晌。墾種二十晌,留荒十晌,四年徵糧,每晌一石。十年後移駐京旗,人給熟地十五晌,荒五晌,餘十晌荒、熟各半,給原駐屯丁為心互產,免徵其租。因利而利,糜帑無多,將來京旗移到,得種熟地,與本處旗屯犬牙相錯,學耕夥種,實為有益。」並詳列屯墾、出納、設官、經理事宜,詔如議行。

二十年,富俊親駐雙城子,地在拉林河西北,橫一百三十里,縱七十餘里,沃衍宜耕。遣員履丈,分撥伐木於拉林河上游,建立屯屋。分五屯,設協領一、佐領二,分左右翼統治之,即名屯地曰雙城堡,於二十一年一律開墾。是年霜早歉收,屯丁僅足餬口,又挈妻子者不敷居住,間有逃亡。乃展緩徵糧一年,添蓋窩棚,借給耔種,心始安。二十二年,調盛京。疏陳雙城堡餘荒尚多,續發盛京、吉林旗丁各千名往墾,分左、右二屯,舊屯名為中屯,遂復調富俊吉林,任其事。二十四年,先到屯丁千名,盛京旗人多有親族偕來,自原入屯,惟隸寧古塔者,因近地亦可耕荒,不原輕離鄉土,聽其還,以空額二百名改撥盛京。二十五年,復續到千名。富俊巡歷三屯,疏陳:「比屋環居,安土樂業,有井田遺風。中屯開墾在先,麥苗暢發,男耕婦饁,俱極勤勞。」仁宗大悅,報曰:「滿洲故里,佃田宅宅,洵善事也。」續議三屯應增事宜,詔嘉實心任事,予議敘。道光元年,疏言:「三屯開墾九萬數千晌,已著成效,可移駐京旗三千戶。請自道光四年始,每歲移駐二百戶,給資裝車馬,分起送屯,官給房屋牛具。」報可。二年,召授理籓院尚書,與玉瀾堂十五老臣宴,禦制詩有「勤勞三省,不凋松柏」之褒。

四年,復出為吉林將軍。方雙城堡之興屯也,富俊欲推其法於伯都訥圍場,以旗戶往往賴幫丁助耕,不如逕招民墾。前後疏六七上,為廷議所格。至是,復言伯都訥圍場荒地二十餘萬晌,募民屯墾,較雙城堡費半功倍,始允之。五年,丈地分屯,申畫經界,名曰新城屯。分八旗為兩翼,每翼初立二十五屯,後定為十五屯。每屯三十戶,以「治本於農務滋稼穡」八字為號。以次撥地,同時並墾。至七年,陸續認佃三千六百戶,總為一百二十屯,與雙城堡相為表裏。初議京旗每歲二百戶移駐雙城堡,至六年,僅陸續移到二百七十戶;七年,續移八十五戶:而地利頓興,自此雙城堡、伯都訥兩地號邊方繁庶之區焉。

墾事既定,復召為理籓院尚書,協辦大學士,兼鑲黃旗漢軍都統。次年,京察,以在吉林宣勞,予議敘。疏言:「京、外競尚浮奢,官民服飾及冠婚、喪祭,任意逾制,有關風俗人心。請依會典儀制,刊布規條,宣諭民間。」詔下有司議行。時富俊年逾八十,渥被優禮,遇常朝免其入直。迭讞獄盛京、吉林,俱稱旨。十年,調工部,拜東閣大學士,管理理籓院。十二年,復請禁僭用服色,犯者拿捕,詔斥徒滋擾累,寢其議。尋以天時亢旱,自稱奉職無狀,引年乞罷,不許。授內大臣。疏言:「科舉保薦,並認師生,餽遺關通,成為陋習。請嚴禁,以端仕進。」詔嘉納,申誡臣工務除積習。十四年,卒。帝悼惜,稱其「清慎公勤,克盡厥職」,贈太子太傅,親臨奠醊,謚文誠,入祀賢良祠。

富俊尚廉節,好禮賢士。在吉林時,請調黑龍江戍員馬瑞辰掌教白山書院,且被嚴斥。其治屯墾,專任竇心傳,卒以成功。

心傳,山西人。以進士官奉天寧海知縣,坐東巡治御道有誤,罷職。富俊知其才,闢佐墾務,規畫悉出手定,始終在事,以勞復官。世比諸陳潢之佐靳輔治河。

博啟圖,一等誠嘉毅勇公明瑞孫。嘉慶初襲爵,授頭等侍衛。歷兵部侍郎、察哈爾都統。道光七年,調吉林將軍,繼富俊之後,守其成規。治邊有法,富俊請以屯墾專任之。時京旗以邊地早寒,又助耕乏人,原往者少。博啟圖疏請減戶增田,許其買僕代耕,統居中屯,改建住屋,俾便禦寒;雖得請,尋召授工部尚書兼領侍衛內大臣,繼任者不果行其議,故移駐卒未如額。十四年,卒,贈太子太保,謚敬僖。

論曰:保寧、松筠、富俊並出自籓族,久膺邊寄,晉綸扉,稱名相,伊犁、吉林屯田,利在百世;然限於事勢,收效未盡如所規畫,甚矣締造之艱也!松筠在吉林,請開小綏芬屯墾,當時以不急之務沮之;至咸、同間,其地竟劃歸俄界。茍早經營,奚致輕棄?實邊之計,顧可忽哉!


\end{pinyinscope}