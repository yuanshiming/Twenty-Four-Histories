\article{列傳一百四十三}

\begin{pinyinscope}
洪亮吉管世銘穀際岐李仲昭石承藻

洪亮吉,字稚存,江蘇陽湖人。少孤貧,力學,孝事寡母。初佐安徽學政硃筠校文,繼入陜西巡撫畢沅幕,為校刊古書。詞章考據,著於一時,尤精揅輿地。乾隆五十五年,成一甲第二名進士,授翰林院編修,年已四十有五。長身火色,性豪邁,喜論當世事。未散館,分校順天鄉試。督貴州學政,以古學教士,地僻無書籍,購經、史、通典、文選置各府書院,黔士始治經史。為詩古文有法。任滿還京,入直上書房,授皇曾孫奕純讀。嘉慶三年,大考翰詹,試徵邪教疏,亮吉力陳內外弊政數千言,為時所忌。以弟喪陳情歸。

四年,高宗崩,仁宗始親政。大學士硃珪書起之,供職,與修高宗實錄,第一次稿本成,意有不樂。將告歸,上書軍機王大臣言事,略曰:「今天子求治之心急矣,天下望治之心孔迫矣,而機局未轉者,推原其故,蓋有數端。亮吉以為勵精圖治,當一法祖宗初政之勤,而尚未盡法也。用人行政,當一改權臣當國之時,而尚未盡改也。風俗則日趨卑下,賞罰則仍不嚴明,言路則似通而未通,吏治則欲肅而未肅。何以言勵精圖治尚未盡法也?自三四月以來,視朝稍晏,竊恐退朝之後,俳優近習之人,熒惑聖聽者不少。此親臣大臣啟沃君心者之過也。蓋犯顏極諫,雖非親臣大臣之事,然不可使國家無嚴憚之人。乾隆初年,純皇帝宵旰不遑,勤求至治,其時如鄂文端、硃文端、張文和、孫文定等,皆侃侃以老成師傅自居。亮吉恭修實錄,見一日中硃筆細書,折成方寸,或詢張、鄂,或詢孫、硃,曰某人賢否,某事當否,日或十餘次。諸臣亦皆隨時隨事奏片,質語直陳,是上下無隱情。純皇帝固聖不可及,而亦眾正盈朝,前後左右皆嚴憚之人故也。今一則處事太緩,自乾隆五十五年以後,權私蒙蔽,事事不得其平者,不知凡幾矣。千百中無有一二能上達者,即能上達,未必即能見之施行也。如江南洋盜一案,參將楊天相有功駢戮,洋盜某漏網安居,皆由署總督蘇凌阿昏憒糊塗,貪贓玩法,舉世知其冤,而洋盜公然上岸無所顧忌,皆此一事釀成。況蘇凌阿權相私人,朝廷必無所顧惜,而至今尚擁巨貲,厚自頤養。江南查辦此案,始則有心為承審官開釋,繼則並聞以不冤覆奏。夫以聖天子赫然獨斷,欲平反一事而尚如此,則此外沉冤何自而雪乎?一則集思廣益之法未備。堯、舜之主,亦必詢四岳,詢群牧。蓋恐一人之聰明有限,必博收眾採,庶無失事。請自今凡召見大小臣工,必詢問人材,詢問利弊。所言可採,則存檔冊以記之。倘所舉非人,所言失實,則治其失言之罪。然寄耳目於左右近習,不可也;詢人之功過於其黨類,亦不可也。蓋人材至今日,銷磨殆盡矣。以模棱為曉事,以軟弱為良圖,以鉆營為取進之階,以茍且為服官之計。由此道者,無不各得其所欲而去,衣缽相承,牢結而不可解。夫此模棱、軟弱、鉆營、茍且之人,國家無事,以之備班列可也;適有緩急,而欲望其奮身為國,不顧利害,不計夷險,不瞻徇情面,不顧惜身家,不可得也。至於利弊之不講,又非一日。在內部院諸臣,事本不多,而常若猝猝不暇,汲汲顧影,皆云多一事不如少一事。在外督撫諸臣,其賢者斤斤自守,不肖者亟亟營私。國計民生,非所計也,救目前而已;官方吏治,非所急也,保本任而已。慮久遠者,以為過憂;事興革者,以為生事。此又豈國家求治之本意乎?二則進賢退不肖似尚游移。夫邪教之起,由於激變。原任達州知州戴如煌,罪不容逭矣。幸有一眾口交譽之劉清,百姓服之,教匪亦服之。此時正當用明效大驗之人。聞劉清尚為州牧,僅從司道之後辦事,似不足盡其長矣。亮吉以為川省多事,經略縱極嚴明,剿賊匪用之,撫難民用之,整飭官方辦理地方之事又用之,此不能分身者也。何如擇此方賢吏如劉清者,崇其官爵,假以事權,使之一意招徠撫綏,以分督撫之權,以蕆國家之事。有明中葉以來,鄖陽多事,則別設鄖陽巡撫;偏沅多事,則別設偏沅巡撫。事竣則撤之,此不可拘拘於成例者也。夫設官以待賢能,人果賢能,似不必過循資格。如劉清者,進而尚未進也。戴如煌雖以別案解任,然尚安處川中。聞教匪甘心欲食其肉,知其所在,即極力焚劫。是以數月必移一處,教匪亦必隨而跡之。近在川東與一道員聯姻,恃以無恐。是救一有罪之人,反殺千百無罪之人,其理尚可恕乎?純皇帝大事之時,即明發諭旨數和珅之罪,並一一指其私人,天下快心。乃未幾而又起吳省蘭矣,召見之時,又聞其為吳省欽辨冤矣。夫二吳之為和珅私人,與之交通貨賄,人人所知。故曹錫寶之糾和珅家人劉全也,以同鄉素好,先以摺示二吳,二吳即袖其走權門,藉為進身之地。今二吳可雪,不幾與褒贈曹錫寶之明旨相戾乎?夫吳省欽之傾險,秉文衡,尹京兆,無不聲名狼藉,則革職不足蔽辜矣。吳省蘭先為和申教習師,後反稱和珅為老師,大考則第一矣,視學典試不絕矣,非和珅之力而誰力乎?則降官亦不足蔽辜矣。是退而尚未退也。何以言用人行政未盡改也?蓋其人雖已致法,而十餘年來,其更變祖宗成例,汲引一己私人,猶未嘗平心討論。內閣、六部各衙門,何為國家之成法,何為和珅所更張,誰為國家自用之人,誰為和珅所引進,以及隨同受賄舞弊之人,皇上縱極仁慈,縱欲寬脅從,又因人數甚廣,不能一切屏除。然竊以為實有真知灼見者,自不究其從前,亦當籍其姓名,於升遷調補之時,微示以善惡勸懲之法,使人人知聖天子雖不為已甚,而是非邪正之辨,未嘗不洞悉,未嘗不區別。如是而夙昔之為私人者,尚可革面革心而為國家之人。否則,朝廷常若今日清明可也,萬一他日復有效權臣所為者,而諸臣又群起而集其門矣。何以言風俗日趨卑下也?士大夫漸不顧廉恥,百姓則不顧綱常。然此不當責之百姓,仍當責之士大夫也。以亮吉所見,十餘年來,有尚書、侍郎甘為宰相屈膝者矣;有大學士、七卿之長,且年長以倍,而求拜門生,求為私人者矣;有交宰相之僮隸,並樂與抗禮者矣。太學三館,風氣之所由出也。今則有昏夜乞憐,以求署祭酒者矣;有人前長跪,以求講官者矣。翰林大考,國家所據以升黜詞臣者也。今則有先走軍機章京之門,求認師生,以探取禦制詩韻者矣;行賄於門闌侍衛,以求傳遞代倩,藏卷而去,制就而入者矣。及人人各得所欲,則居然自以為得計。夫大考如此,何以責鄉會試之懷挾替代?士大夫之行如此,何以責小民之言誇詐夤緣?輦轂之下如此,何以責四海九州之營私舞弊?純皇帝因內閣學士許玉猷為同姓石工護喪,諭廷臣曰:『諸臣縱不自愛,如國體何?』是知國體之尊,在諸臣各知廉恥。夫下之化上,猶影響也。士氣必待在上者振作之,風節必待在上者獎成之。舉一廉樸之吏,則貪欺者庶可自愧矣;進一恬退之流,則奔競者庶可稍改矣;拔一特立獨行、敦品勵節之士,則如脂如韋、依附朋比之風或可漸革矣。而亮吉更有所慮者,前之所言,皆士大夫之不務名節者耳。幸有矯矯自好者,類皆惑於因果,遁入虛無,以蔬食為家規,以談禪為國政。一二人倡於前,千百人和於後。甚有出則官服,入則僧衣。惑智驚愚,駭人觀聽。亮吉前在內廷,執事曾告之曰:『某等親王十人,施齋戒殺者已十居六七,羊豕鵝鴨皆不入門。』及此回入都,而士大夫持齋戒殺又十居六七矣。深恐西晉祖尚玄虛之習復見於今,則所關世道人心非小也。何以言賞罰仍不嚴明也?自征苗匪、教匪以來,福康安、和琳、孫士毅則蒙蔽欺妄於前,宜綿、惠齡、福寧則喪師失律於後,又益以景安、秦承恩之因循畏葸,而川、陜、楚、豫之民,遭劫者不知幾百萬矣。已死諸臣姑置勿論,其現在者未嘗不議罪也。然重者不過新疆換班,輕者不過大營轉餉;甚至拏解來京之秦承恩,則又給還家產,有意復用矣;屢奉嚴旨之惠齡,則又起補侍郎。夫蒙蔽欺妄之殺人,與喪師失律以及因循畏葸之殺人無異也,而猶邀寬典異數,亦從前所未有也。故近日經略以下、領隊以上,類皆不以賊匪之多寡、地方之蹂躪掛懷。彼其心未始不自計曰:『即使萬不可解,而新疆換班,大營轉餉,亦尚有成例可援,退步可守。』國法之寬,及諸臣之不畏國法,未有如今日之甚者。純皇帝之用兵金川、緬甸,訥親僨事,則殺訥親;額爾登額僨事,則殺額爾登額;將軍、提、鎮之類,伏失律之誅者,不知凡幾。是以萬里之外,得一廷寄,皆震懼失色,則馭軍之道得也。今自乙卯以迄己未,首尾五年,僨事者屢矣。提、鎮、副都統、偏裨之將,有一膺失律之誅者乎?而欲諸臣之不玩寇、不殃民得乎?夫以純皇帝之聖武,又豈見不及此?蓋以歸政在即,欲留待皇上蒞政之初,神武獨斷,一新天下之耳目耳。倘蕩平尚無期日,而國帑日見銷磨,萬一支絀偶形,司農告匱。言念及此,可為寒心,此尤宜急加之意者也。何以言言路似通而未通也?九卿臺諫之臣,類皆毛舉細故,不切政要。否則發人之陰私,快己之恩怨。十件之中,幸有一二可行者,發部議矣,而部臣與建言諸臣,又各存意見,無不議駁,並無不通駁,則又豈國家詢及芻蕘、詢及瞽史之初意乎?然或因其所言瑣碎,或輕重失倫,或虛實不審,而一概留中,則又不可。其法莫如隨閱隨發,面諭廷臣,或特頒諭旨,皆隨其事之可行不可行,明白曉示之。即或彈劾不避權貴,在諸臣一心為國,本不必避嫌怨。以近事論,錢灃、初彭齡皆常彈及大僚矣,未聞大僚敢與之為仇也。若其不知國體,不識政要,冒昧立言,或攻發人之陰私,則亦不妨使眾共知之,以著其非而懲其後。蓋諸臣既敢挾私而不為國,更可無煩君上之回護矣。何以言吏治欲肅而未肅也?未欲吏治之肅,則督、撫、籓、臬其標準矣。十餘年來,督、撫、籓、臬之貪欺害政,比比皆是。幸而皇上親政以來,李奉翰已自斃,鄭元鸘已被糾,富綱已遭憂,江蘭已內改。此外,官大省、據方面者如故也,出巡則有站規、有門包,常時則有節禮、生日禮,按年則又有幫費。升遷調補之私相餽謝者,尚未在此數也。以上諸項,無不取之於州縣,州縣則無不取之於民。錢糧漕米,前數年尚不過加倍,近則加倍不止。督、撫、籓、臬以及所屬之道、府,無不明知故縱,否則門包、站規、節禮、生日禮、幫費無所出也。州縣明言於人曰:『我之所以加倍加數倍者,實層層衙門用度,日甚一日,年甚一年。』究之州縣,亦恃督、撫、籓、臬、道、府之威勢以取於民,上司得其半,州縣之入己者亦半。初行尚有畏忌,至一年二年,則成為舊例,牢不可破矣。訴之督、撫、籓、臬、道、府,皆不問也。千萬人中,或有不甘冤抑,赴京控告者,不過發督撫審究而已,派欽差就訊而已。試思百姓告官之案,千百中有一二得直者乎?即欽差上司稍有良心者,不過設為調停之法,使兩無所大損而已。若欽差一出,則又必派及通省,派及百姓,必使之滿載而歸而心始安,而可以無後患。是以州縣亦熟知百姓之技倆不過如此,百姓亦習知上控必不能自直,是以往往至於激變。湖北之當陽,四川之達州,其明效大驗也。亮吉以為今日皇上當法憲皇帝之嚴明,使吏治肅而民樂生;然後法仁皇帝之寬仁,以轉移風俗,則文武一張一弛之道也。」

書達成親王,以上聞,上怒其語戇,落職下廷臣會鞫,面諭勿加刑,亮吉感泣引罪,擬大闢,免死遣戍伊犁。明年,京師旱,上禱雨未應,命清獄囚,釋久戍。未及期,詔曰:「罪亮吉後,言事者日少。即有,亦論官吏常事,於君德民隱休戚相關之實,絕無言者。豈非因亮吉獲罪,鉗口不復敢言?朕不聞過,下情復壅,為害甚鉅。亮吉所論,實足啟沃朕心,故銘諸座右,時常觀覽,勤政遠佞,警省朕躬。今特宣示亮吉原書,使內外諸臣,知朕非拒諫飾非之主,實為可與言之君。諸臣遇可與言之君而不與言,負朕求治苦心。」即傳諭伊犁將軍,釋亮吉回籍。詔下而雨,禦制詩紀事,注謂:「本日親書諭旨,夜子時甘霖大沛。天鑒捷於呼吸,益可感畏。」亮吉至戍甫百日而赦還,自號更生居士。後十年,卒於家。所著書多行世。

管世銘,字緘若,與亮吉同里。乾隆四十三年進士,授戶部主事。累遷郎中,充軍機章京。深通律令,凡讞牘多世銘主奏。屢從大臣赴浙江、湖北、吉林、山東按事,大學士阿桂尤善之,倚如左右手。時和珅用事,世銘憂憤,與同官論前代輔臣賢否,語譏切無所避。會遷御史,則大喜,夜起傍徨,草疏將劾之,詔仍留軍機處。故事,御史留直者,儀注仍視郎官,不得專達封事。世銘自言愧負此官,阿桂慰之曰:「報稱有日,何必急以言自見。」蓋留直阿桂所請,隱全之,使有待。嘉慶三年,卒。

穀際岐,字西阿,雲南趙州人。乾隆四十年進士,選庶吉士,授編修,與校四庫全書。充會試同考官,所拔多知名士。乞養歸,主講五華書院,教士有法。連丁父母憂,服闋,起原官。

嘉慶三年,遷御史。時教匪擾數省,師久無功,際岐遍訪人士來京者,具得其狀。四年春,上疏,略曰:「竊見三年以來,先帝頒師征討邪教,川、陜責之總督宜綿,巡撫惠齡、秦承恩;楚北責之總督畢沅、巡撫汪新。諸臣釀釁於先,藏身於後,止以重兵自衛,裨弁奮勇者,無調度接應,由是兵無鬥志。川、楚傳言云:『賊來不見官兵面,賊去官兵才出現。』又云:『賊去兵無影,兵來賊沒蹤。可憐兵與賊,何日得相逢?』前年總督勒保至川,大張告示,痛責前任之失,是其明證。畢沅、汪新相繼殂逝,景安繼為總督。今宜綿、惠齡、秦承恩縱慢於左,景安怯玩於右,勒保縱能實力剿捕,陜、楚賊多,起滅無時,則勒保終將掣肘。欽惟先帝昔征緬甸,見楊應琚挑撥掩覆之罪,立予拿問。今宜綿等曠玩三年之久,幸荷寬典,而轉益懷安,任賊越入河南盧氏、魯山等縣。景安雖無吞餉聲名,而罔昧自甘,近亦有賊焚掠襄、光各境,均為法所不容。況今軍營副封私札,商同軍機大臣改壓軍報。供據已破,雖由內臣聲勢,而彼等掩覆僨事,情更顯然。請旨懲究,另選能臣,與勒保會同各清本境,則軍令風行,賊必授首。比年發餉至數千萬,軍中子女玉帛奇寶錯陳,而兵食反致有虧。載贓而歸,風盈道路,嘲之者有『與其請餉,不如書會票』之語。先帝嚴究軍需局,察出四川漢州知州與德楞泰互爭報銷,及湖北道員胡齊侖侵餉數十萬,一則追賠,一則拿究。他屬類此者必多,尤宜急易新手清釐。則侵盜之跡,必能破露,不但兵餉與善後事宜均得充裕,銷算亦不敢牽混矣。」

間又上疏曰:「教匪滋擾,始於湖北宜都聶傑人,實自武昌府同知常丹葵苛虐逼迫而起。當教匪齊麟等正法於襄陽,匪徒各皆斂戢。常丹葵素以虐民喜事為能,乾隆六十年,委查宜都縣境,哧詐富家無算,赤貧者按名取結,納錢釋放。少得供據,立與慘刑,至以鐵釘釘人壁上,或鐵錘排擊多人。情介疑似,則解省城,每船載一二百人,饑寒就斃,浮尸於江。歿獄中者,亦無棺殮。聶傑人號首富,屢索不厭,村黨結連拒捕。宜昌鎮總兵突入遇害,由是宜都、枝江兩縣同變。襄陽之齊王氏、姚之富,長陽之覃加耀、張正謨等,聞風並起,遂延及河南、陜西。此臣所聞官逼民反之最先最甚者也。臣思教匪之在今日,自應盡黨梟磔。而其始猶是百數十年安居樂業人民,何求何憾,甘心棄身家、捐性命,鋌而走險耶?臣聞賊當流竄時,猶哭念皇帝天恩,殊無一言怨及朝廷。向使地方官仰體皇仁,察教於平日,撫弭於臨時,何至如此?臣為此奏,固為官吏指事聲罪,亦欲使萬禩子孫知我朝無叛民,而後見恩德入人,天道人心,協應長久,昭昭不爽也。常丹葵逞虐一時,上廑聖仁,下殃良善,罪豈容誅?應請飭經略勒保嚴察奏辦。又現奉恩旨,凡受撫來歸者,令勒保傳喚同知劉清,同川省素有清名之州縣,妥議安插。楚地曾經滋擾者,亦應安集。臣聞被擾州縣,逃散各戶之田廬婦女,多歸官吏壓賣分肥。是始不顧其反,終不原其歸。不知民何負於官,而效尤靦忍至於此極?若得懲一儆眾,自可群知洗濯。宣奉德意,所關於國家苞桑之計匪細也。」兩疏上,仁宗並嘉納施行。尋遷給事中,稽察南新倉,巡視中城。

雲南鹽法,官運官銷,日久因緣為奸,按口比銷,民不堪命;又威遠調取民夫,按名折銀,折後又徵實夫,迤西道屬數十州縣,同時閧變,解散後不以實聞,官吏骫法如故。際岐上疏痛陳其害,下雲南督撫察治。總督富綱請改鹽法以便民,巡撫江蘭方內召,欲沮其事,際岐復疏爭。初彭齡繼為巡撫,際岐門下士也,熟聞其事,始疏請鹽由灶煎灶賣,民運民銷,一祛積弊,民大便。語詳鹽法志。

蔡永清者,總督陳輝祖家奴,擁厚貲居京師,以助賑敘五品職銜,出入輿馬,揖讓公卿間。際岐疏劾,自大學士慶桂、硃珪以下,多所指斥,下刑部鞫訊,褫永清職銜,際岐坐論奏未盡實,降授刑部主事。累遷郎中。以老乞休,貧不能歸,主講揚州孝廉堂垂十年,卒。

自乾隆末,雲南之官於朝以直言著者,尹壯圖、錢灃,時以際岐並稱焉。

李仲昭,字次卿,廣東嘉應人。嘉慶七年進士,選庶吉士,授編修,遷御史。長蘆鹽商偽造加重法馬,每引浮百斤,損課滯銷。商人查有圻家鉅富,交通朝貴。自給事中花傑劾蘆鹽加價,連及大學士戴衢亨,不得直,且被譴,遂無敢言者。仲昭疏劾之,戶部猶袒商,或騰蜚語,謂仲昭索賄不遂。仁宗方幸熱河,命留京王大臣同鞫,得舞弊狀,有圻論如律,在事降革有差,人咸側目。仲昭又劾吏部京察不公,亦鞫實。既而赴戶部點卯,杖責書吏,戶部摭其事奏劾,下吏部議。群欲以傾仲昭,侍郎初彭齡號剛正,以妻喪在告,語人曰:「諸人欲報怨,加以莫須有之罪。李御史有言膽,臺中何可無此人?」部員聞彭齡言,遽議降四級,甫兩日而奏上,仲昭竟黜。

石承藻,字黼庭,湖南湘潭人。嘉慶十三年一甲三名進士,授編修。遷御史、給事中,敢言有聲。王樹勛者,江都人,乾隆末入京應試不售,乃於廣慧寺為僧,名曰明心。開堂說法,假扶乩卜筮,探刺士大夫陰私,揚言於外,人益崇信。達官顯宦,每有皈依受戒為弟子者。硃珪正人負重望,亦與交接。時和珅為步軍統領,訪捕治罪,以賄得末減,勒令還俗,遂游蕩江湖。值川、楚匪亂,投效松筠軍中,以談禪投所好,使易裝入賊寨說降,獎予七品官銜,洊擢襄陽知府。數年,入覲京師,不改故態。刑部尚書金光悌延醫子病,怵以禍福,光悌長跪請命,為時所嗤。嘉慶二十年,承藻疏請澄清流品,劾樹勛,下刑部鞫實,褫職,枷號兩月,發黑龍江充當苦差。仁宗獎承藻曰:「真御史也!」詔斥被惑諸臣,有玷官箴。其已故者免議,侍郎蔣予蒲、宋鎔以下,黜降有差。

二十四年,湘潭有土、客械斗之獄,侍郎周系英與巡撫吳邦慶互劾。承藻適在籍,系英子汝楨致書承藻詢其事,為邦慶所發,承藻牽連降秩。久之不復遷,終光祿寺署正。

論曰:仁宗詔求直言,下至末吏平民,皆得封章上達,言路大開。科道中竭誠獻納,如衛謀論福康安貪婪,不宜配享太廟。馬履泰論景安畏縮偷安,老師糜餉,及教匪宜除,難民宜撫;又論百齡舉劾失當。張鵬展論金光悌專擅刑部,戀司職不去。周栻論疆臣參劾屬員,不舉劣跡,恐悃愊無華者以失歡被劾;又論硃珪以肩輿擅入禁門,無無君之心,而有無君之跡。沈琨論宜興庇護屬員,致興株系諸生大獄;又諫阻東巡。蕭芝論端正風俗,宜崇醇樸。王寧煒論用人宜習其素,不可因保舉遽加升用;又論督撫壅蔽之習,及士民捐輸之累,州縣折收之患。游光繹論大臣未盡和衷,武備未盡整飭,原效魏元成十思疏以裨治化。諸人所言,雖有用有不用,當時皆推讜直。又龔鏜當松筠因諫東巡獲罪,密疏復陳,自庀身後事而後上,卒蒙寬宥。其章疏多不傳,稽之史牒,旁見紀載,謇諤盈廷,稱盛事焉。洪亮吉諸人身雖遭黜,言多見採,可以無憾。或猶以時方清明,目亮吉之效痛哭流涕者為多事,過矣。


\end{pinyinscope}