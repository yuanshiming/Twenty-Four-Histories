\article{列傳五}

\begin{pinyinscope}
諸王四

太祖諸子三

睿忠親王多爾袞豫通親王多鐸子信宣和郡王多尼信郡王董額

輔國恪僖公察尼多尼子信郡王鄂扎費揚果

睿忠親王多爾袞,太祖第十四子。初封貝勒。天聰二年,太宗伐察哈爾多羅特部,破敵於敖穆楞,多爾袞有功,賜號墨爾根代青。三年,從上自龍井關入明邊,與貝勒莽古爾泰等攻下漢兒莊,趨通州,薄明都,敗袁崇煥、祖大壽援兵於廣渠門外,又殲山海關援兵於薊州。四年,引還,多爾袞與莽古爾泰先行,復破敵。五年,初設六部,掌吏部事。從上圍大凌河,戰,多爾袞陷陣,明兵墮壕者百餘,城上?矢發,將士有死者。上切責諸將不之阻。祖大壽約以錦州獻,多爾袞與阿巴泰等以兵四千,僑裝從大壽作潰奔狀,襲錦州,錦州兵迎戰,擊敗之。事具阿巴泰傳。

六年五月,從征察哈爾。七年六月,詔問征明及朝鮮、察哈爾三者何先,多爾袞言:「宜整兵馬,乘穀熟時,入邊圍燕京,截其援兵,毀其屯堡,為久駐計,可坐待其敝。」八年五月,從上伐明,克保安,略朔州。九年,上命偕岳託等將萬人招察哈爾林丹汗子額哲,至朔州,毀寧武關,略代州、忻州、崞縣、黑峰口及應州,復自?師還渡河,多爾袞自平魯還。林丹汗得元玉璽曰「制誥之寶」,多爾袞使額哲進上,?臣因表上尊號。?歸化城攜降崇德元年,進封睿親王。武英郡王阿濟格等率師伐明,命王偕多鐸攻山海關綴明師,阿濟格捷至,乃還。從伐朝鮮,偕豪格別從寬甸入長山口,克昌州。進攻江華島,克之,獲朝鮮王妃及其二子,國王李倧請降。上還盛京,命約束後軍,攜朝鮮質子鷫、淏及大臣子以歸。

三年,上伐喀爾喀,王留守,築遼陽都爾弼城,城成,命曰屏城;復治盛京至遼河大道。八月,命為奉命大將軍,將左翼,岳託將右翼,伐明。自董家口毀邊墻入,約右翼兵會通州河西務。越明都至涿州,分兵八道,行略地至山西,南徇保定,擊破明總督盧象升。遂趨臨清,渡運河,破濟南。還略天津、遷安,出青山關。克四十餘城,降六城,俘戶口二十五萬有奇,賜馬五、銀二萬。五年,屯田義州,克錦州城西九臺,刈其禾。又克小凌河西二臺。迭敗明兵杏山、松山間。

圍錦州,王貝勒移營去城三十里,又令每旗一將校率每牛錄甲士五人先歸。上遣濟爾哈朗代將,傳諭詰責,對曰:「臣以敵兵在錦州、松山、杏山三城,皆就他處牧馬。若來犯,可更番抵御。是以遣人歸牧,治甲械。舊駐地草盡,臣倡議移營就牧,罪實在臣。」上復使諭曰:「朕愛爾過於?子弟,錫予獨厚。今違命若此,其自議之。」王自言罪當死,上命降郡王,罰銀萬,奪二牛錄。

,請濟師。上自將疾?六年,復圍錦州。洪承疇率十三萬人屯松山,王屢擊之,以敵馳六日,次戚家堡,將屯高橋。王請上駐松山、杏山間,分兵屯烏欣河南山,亙海為營。明兵屢?復前,上張黃?指揮,明兵引退。王偕洛託等趨塔山道橫擊之,明兵多死者;遂發?克塔山外四臺,擒王希賢等。尋以貝勒杜度等代將,王暫還。復出,七年,下松山,獲承疇,克錦州,大壽復降。進克塔山、杏山。乃隳三城,師還。?功,復親王。

八年,太宗崩,王與諸王、貝勒、大臣奉世祖即位。諸王、貝勒、大臣議以鄭親王濟爾哈朗與王同輔政,誓曰:「有不秉公輔理、妄自尊大者,天地譴之!」郡王阿達禮、貝子碩託勸王自立,王發其謀,誅阿達禮、碩託。尋與濟爾哈朗議罷諸王貝勒管六部事。順治元年正月,?朝鮮餽遺,告濟爾哈朗及諸大臣曰:「朝鮮國王因予取江華,全其妻子,常以私餽遺。先帝時必聞而受之,今輔政,誼無私交,不當受。」因並禁外國餽諸王貝勒者。濟爾哈朗諭諸大臣,凡事先白王,書名亦先之。王由是始專政。固山額真何洛會等訐肅親王豪格怨望,集議,削爵,大臣揚善等以諂附,坐死。

四月乙丑,上御篤恭殿,授王奉命大將軍印,並御用纛蓋,敕便宜行事,率武英郡王阿濟格、豫郡王多鐸及孔有德等伐明。丙寅,發盛京。壬申,次翁後。明平西伯吳三桂自山海關來書乞師,王得書,移師向之。癸酉,次西拉塔拉。答三桂書曰:「我國欲與明修好,屢致書不一答。是以整師三入,蓋示意於明,欲其熟籌通好。今則不復出此,惟底定中原,與民休息而已。聞流賊陷京都,崇禎帝慘亡,不勝發指,用率仁義之師,沈舟破釜,誓必滅賊,出民水火!伯思報主恩,與流賊不共戴天,誠忠臣之義,勿因向守遼東與我為敵,尚復來歸,必封以故土,晉為籓?懷疑。昔管仲射桓公中鉤,桓公用為仲父,以成霸業。伯若率王。國讎可報,身家可保,世世子孫,長享富貴。」

丁丑,次連山。三桂復遣使請速進,夜逾寧遠抵沙河。戊寅,距關十里,三桂報自成兵已出邊。王令諸王逆擊,敗李自成將唐通於一片石。己卯,至山海關,三桂出迎,王慰勞之。令所部以白布系肩為識,先驅入關。時自成將二十餘萬人,自北山列陣,橫亙至海。,不可輕敵。吾觀其陣大,首尾不相顧?我兵陣不及海岸,王令曰:「流賊橫行久,獷而。可集我軍鱗比,伺敵陣尾,待其衰擊之,必勝。努力破此,大業成矣。勿違節制!」既成良久,師譟。風止,自三桂陣右?列,令三桂居右翼後。搏戰,大風揚沙,咫尺不能辨。力突出,搗其中堅,馬迅矢激。自成登高望見,奪氣,策馬走。師無不一當百,追奔四十里,自成潰遁。王即軍前承制進三桂爵平西王。下令關內軍民皆薙發。以馬步兵各萬人屬三桂,追擊自成。乃誓諸將曰:「此行除暴救民,滅賊以安天下。勿殺無辜、掠財物、焚廬舍。不如約者,罪之。」自關以西,百姓有逃竄山谷者,皆還鄉里,薙發迎降。辛巳,次新河驛,使奏捷,師遂進。途中明將吏出降,命供職如故。

五月戊子朔,師次通州。自成先一日焚宮闕,載輜重而西。王令諸王偕三桂各率所部追之。己丑,王整軍入京師,明將吏軍民迎朝陽門外,設鹵簿,請乘輦,王曰:「予法周公以周公嘗負扆,固請,乃命以鹵簿列王儀仗前,奏樂,拜天,復拜闕?輔沖主,不當乘。」,乘輦,升武英殿。明將吏入謁,呼萬歲。下令將士皆乘城,毋入民舍,民安堵如故。為崇禎帝發喪三日,具帝禮葬之。諸臣降者,仍以明官治事。武英郡王阿濟格逐自成至慶都,大破之,獲其輜重。自成西奔,又令固山額真譚泰、準塔等率巴牙喇兵追至真定,自成敗走。巴泰齎敕慰勞。畿輔諸府縣先後請降,分遣固山額?王再遣使奏捷,上遣學士詹霸、侍真巴哈納、石廷柱略山東,葉臣定山西諸省,金礪等安撫天津。

王初令官民皆薙發,繼聞拂民原,諭緩之。令戒飭官吏,網羅賢才,收恤都市貧民。用湯若望議,釐正歷法,定名曰時憲歷。復令曰:「養民之道,莫大於省刑罰,薄稅斂。自明季禍亂,刁風日競,設機構訟,敗俗傷財,心竊痛之!自今咸與維新,凡五月初二日昧爽毆,田、婚細故,就有司告理。?以前,罪無大小,悉行宥免。違諭訐訟,以所告罪罪之。重大者經撫按結案,非機密要情,毋許入京越訴。訟師誣陷良民,加等反坐。前朝弊政,莫如加派,遼餉之外,復有剿餉、練餉,數倍正供,遠者二十年,近者十餘年,天下嗷嗷,朝不及夕。更有召買、糧料諸名目,巧取殃民。今與民約,額賦外,一切加派,盡予刪除。官吏不從,察實治罪。」六月,遣輔國公屯齊喀、和託,固山額真何洛會等迎上,定都燕京。

明福王由崧稱帝江寧,遣其大學士史可法督師揚州,設江北四鎮,沿淮、徐置戍。王致書可法曰:「予向在沈陽,即知燕京物望,咸推司馬。後入關破賊,得與都人士相接,識介弟於清班,曾託其手勒平安,拳致衷緒,未審以何時得達?比聞道路紛紛,多謂金陵有自立者。夫君父之仇,不共戴天。春秋之義,有賊不討,則故君不得書葬,新君不得書即位,所以防亂臣賊子,法至嚴也。闖賊李自成,稱兵犯闕,手毒君親,中國臣民,不聞加遺一矢。平西王吳三桂,介在東陲,獨效包胥之哭,朝廷感其忠義,念累世之宿好,棄近日之小嫌,爰整貔貅,驅除狗鼠。入京之日,首崇帝後謚號,卜葬山陵,悉如典禮。親郡王、將軍以下,一仍故封,不加改削。勛戚文武諸臣,咸在朝列,恩禮有加。耕市不驚,秋毫無擾。方擬秋高氣爽,遣將西征;傳檄江南,聯兵河朔,陳師鞠旅,戮力同心,報乃君國之仇,彰我朝廷之德。豈意南州諸君子,茍安旦夕,弗審事機,聊慕虛名,頓忘實害,予甚惑之!國家撫定燕都,得之於闖賊,非取之於明朝也。賊毀明朝之廟主,辱及先人,我國家不憚征繕之勞,悉索敝賦,代為雪恥,孝子仁人,當如何感恩圖報。茲乃乘逆寇稽誅,王師暫息,遂欲雄據江南,坐享漁人之利。揆諸情理,豈可謂平?將以為天塹不能飛渡,投鞭不能斷流耶?夫闖賊但為明朝祟耳,未嘗得罪於我國家也,徒以薄海同仇,特伸大義。今若擁號稱尊,便是天有二日,儼為勍敵。予將簡西行之銳,轉■H5東征,且擬釋彼重誅,命為前導。夫以中華全力,受制潢池,而欲以江左一隅,兼支大國,勝負之數,無待蓍龜矣。予聞君子之愛人也以德,細人則以姑息。諸君子果識時知命,篤念故主,厚愛賢王,宜勸令削號歸籓,永綏福祿。朝廷當待以虞賓,統承禮物,帶礪山河,位在諸王侯上,庶不負朝廷伸義討賊、興滅繼絕之初心。至南州?彥,翩然來儀,則爾公爾侯,列爵分土,有平西之典例在。惟執事實圖利之!輓近士大夫好高樹名義,而不顧國家之急,每有大事,輒同築舍。昔宋人議論未定,兵已渡河,可為殷鑒。先生領袖名流,主持至計,必能深惟終始,寧忍隨俗浮沉?取舍從違,應早審定。兵行在即,可西可東。南國安危,在此一舉。原諸君子同以討賊為心,毋貪一身瞬息之榮,而重故國無窮之禍,為亂臣賊子所竊笑,予實有厚望焉!記有之,惟善人能受盡言。敬布腹心,佇聞明教。江天在望,延跂為勞,書不宣意。」可法旋遣人報書,語多不屈。

京師民訛言秋七、八月將東遷,王宣諭當建都燕京,戒民毋信流言搖惑。又訛言八月屠民;未幾,又訛言上至京師,將縱東兵肆掠,盡殺老壯,止存孩赤。王復宣諭曰:「民乃國之本,爾曹既誠心歸服,復以何罪而戮之?爾曹試思,今上攜將士家屬不下億萬,與之俱來者何故?為安燕京軍民也。昨將東來各官內,命十餘員為督、撫、司、道等官者何故?為統一天下也。已將盛京帑銀取至百餘萬,後又轉運不絕者何故?為供爾京城內外兵民之用也。且予不忍山、陜百姓受害,發兵追剿,猶恨未能速定,豈能不愛京城軍民,反行殺戮?此所目擊,何故妄布流言?是必近京土寇,流賊間諜,有意煽惑搖動,已諭各部嚴捕。通?皆心。」?行曉諭,以安

九月,上入山海關,王率諸王?臣迎於通州。上至京師,封為叔父攝政王,賜貂蟒朝衣。十月乙卯朔,上即位,以王功高,命禮部尚書郎球、侍郎藍拜、啟心郎渥赫建碑紀績,加賜冊寶、黑狐冠一、上飾東珠十三、黑狐裘一,副以金、銀、馬、駝。二年,鄭親王等議上攝政王儀制,視諸王有加禮。王曰:「上前未敢違禮,他可如議。」翌日入朝,諸臣跪迎,命還輿,責大學士剛林等曰:「此上朝門,諸臣何故跪我?」御史趙開心疏言:「王以皇叔之親,兼攝政王之尊,臣民寧肯自外於拜舞?第王恩皆上恩,?臣謁王,正當限以禮數,與朝見不同。庶諸臣不失尊王之意,亦全王尊上之心。上稱叔父攝政王,王為上叔父,惟上得稱之。若臣庶宜於叔父上加『皇』字,庶辨上下,尊體制。」下禮部議行。其年六月,豫親王克揚州,可法死之,遂破明南都。閏六月,英親王逐李自成至武昌,東下九江,故明寧南侯降,江南底定。十月,上賜王馬,王入謝,詔曰:「遇朝賀大典,朕受王?左良玉子夢庚率禮。若小節,勿與諸王同。」王對曰:「上方幼沖,臣不敢違禮。待上親政,凡有寵恩,不敢辭。」王時攝政久,位崇功高,時誡諸臣尊事主上,曰:「俟上春秋鼎盛,將歸政焉。」

。英、豫二王與王同母,王視豫親王厚,每寬假之。豫?初,肅親王怨王不立己,有親王之徵蘇尼特也,王送之出安定門。及歸,迎之烏蘭諾爾。集諸大臣,語以豫親王功懋,宜封輔政叔王,因罷鄭親王輔政,以授豫親王。肅親王既平四川,王摘其微罪,置之死。四年十二月,王以風疾不勝跪拜,從諸王大臣議,獨賀正旦上前行禮,他悉免。五年十一月,南郊禮成,赦詔曰:「叔父攝政王治安天下,有大勛勞,宜加殊禮,以崇功德,尊為皇父攝政王。凡詔疏皆書之。」

六年二月,自將討大同叛將姜瓖,拔渾源。聞豫親王病痘,先歸。諭瓖降,未下。以師行在外,鑄行在印。禁諸王及內大臣干預部院政事及漢官升降,不論所言是非,皆治罪。七月,復徵大同,瓖將楊振威斬瓖降。十月,移師討喀爾喀二楚呼爾,徵敖漢、扎嚕特、察哈爾、烏喇特、土默特、四子部落以兵來會。至喀屯布拉克,不見敵,乃還。十二月,王妃博爾濟吉特氏薨,以冊寶追封為敬孝忠恭正宮元妃。

七年正月,王納肅王福金,福金,妃女弟也。復徵女朝鮮。令部事不須題奏者,付巽親王滿達海、端重親王博洛、敬謹親王尼堪料理。五月,率諸王貝勒獵於山海關,朝鮮送女至,王迎於連山,成婚。復獵於中後所,責隨獵王貝勒行列不整,罰鍰有差。七月,諭以京城當夏溽暑不可堪,擇地築城避暑。令戶部加派直隸、山西、浙江、山東、江南、河南、湖廣、江西、陜西九省地丁銀二百四十九萬兩有奇,輸京師備工用。八月,王尊所生母太祖妃烏喇納拉氏為孝烈恭敏獻哲仁和贊天儷聖武皇后,祔太廟。

尋有疾,語貝子錫翰、內大臣席訥布庫等曰:「予罹此大戚,體復不快。上雖人主,獨不能循家人禮一臨幸乎?謂上幼沖,爾等皆親近大臣也。」既又戒曰:「毋以予言請上臨幸。」錫翰等出,追止之,不及,上幸王第。王因責錫翰等,議罪當死,旋命貰之。十一月,復獵於邊外。十二月,薨於喀喇城,年三十九。上聞之,震悼。喪還,率王大臣縞服迎奠東直門外。詔追尊為懋德修道廣業定功安民立政誠敬義皇帝,廟號成宗。明年正月,尊妃為義皇后。祔太廟。

八十員。又以王?王無子,以豫親王子多爾博為後,襲親王,俸視諸王三倍,詔留護近侍蘇克薩哈、詹岱為議政大臣。二月,蘇克薩哈、詹岱訐告王薨時,其侍女吳爾庫尼將殉,請以王所制八補黃袍、大東珠素珠、黑貂褂置棺內。王在時,欲以兩固山駐永平,謀篡大位。固山額真譚泰亦言王納肅王福金,復令肅王子至第較射,何洛會以惡言詈之。於是鄭親王濟爾哈朗、巽親王滿達海、端重親王博洛、敬謹親王尼堪及內大臣等疏言:「昔太宗文皇帝龍馭上賓,諸王大臣共矢忠誠,翊戴皇上。方在沖年,令臣濟爾哈朗與睿親王多爾袞同輔政。逮後多爾袞獨擅威權,不令濟爾哈朗預政,遂以母弟多鐸為輔政叔王。背誓肆行,妄自尊大,自稱皇父攝政王。凡批票本章,一以皇父攝政王行之。儀仗、音樂、侍從、府第,僭擬至尊。擅稱太宗文皇帝序不當立,以挾制皇上。構陷威逼,使肅親王不得其死,遂納其妃,且收其財產。更悖理入生母於太廟。僭妄不可枚舉。臣等從前畏威吞聲,今冒死奏聞,伏原重加處治。」詔削爵,撤廟享,並罷孝烈武皇后謚號廟享,黜宗室,籍財產入官,多爾博歸宗。十二年,吏科副理事官彭長庚、一等精奇尼哈番許爾安各疏頌王功,請復爵號,下王大臣議,長庚、爾安坐論死,詔流寧古塔。

乾隆三十八年,高宗詔曰:「睿親王多爾袞攝政有年,威福自專,歿後其屬人首告,入關,肅清京輦,檄定中原,前勞未可盡泯。今其後嗣?定罪除封。第念定鼎之初,王實統廢絕,塋域榛蕪,殊堪憫惻。交內務府派員繕葺,並令近支王公以時祭掃。」四十三年正月,又詔曰:「睿親王多爾袞掃蕩賊氛,肅清宮禁。分遣諸王,追殲流寇,撫定疆陲。創制規模,皆所經畫。尋奉世祖車駕入都,成一統之業,厥功最著。歿後為蘇克薩哈所構,首告誣以謀逆。其時世祖尚在沖齡,未嘗親政,經諸王定罪除封。朕念王果萌異志,兵權在握,何事不可為?乃不於彼時因利乘便,直至身後始以斂服僭用龍袞,證為覬覦,有是理乎?實錄載:『王集諸王大臣,遣人傳語曰:「今觀諸王大臣但知媚予,鮮能尊上,予豈能容此?昔太宗升遐,嗣君未立,英王、豫王跪請予即尊,予曰:『若果如此言,予即當自刎。』誓死不從,遂奉今上即位。似此危疑之日,以予為君,予尚不可;今乃不敬上而媚予,予何能容?自今後有忠於上者,予用之愛之;其不忠於上者,雖媚予,予不爾宥。」且云:「太宗恩育予躬,所以特異於諸子弟者,蓋深信諸子弟之成立,惟予能成立之。」』朕每覽實錄至此,未嘗不為之墮淚。則王之立心行事,實為篤忠藎,感厚恩,明君臣大義。乃由宵小奸謀,構成?獄,豈可不為之昭雪?宜復還睿親王封號,追謚曰忠,配享太廟。依親王園寢制,修其塋墓,令太常寺春秋致祭。其爵世襲罔替。」

多爾博歸宗封貝勒,命仍還為王後,以其五世孫輔國公淳穎襲爵。四世祖鎮國公蘇爾發、曾祖輔國公塞勒、祖輔國恪勤公功宜布先已進封信郡王,至是與淳穎父信恪郡王如松並追封睿親王。嘉慶五年,淳穎薨。謚曰恭。子寶恩,襲。七年五月,薨,謚曰慎。弟瑞恩,襲。道光六年,薨,謚曰勤。子仁壽,襲。道光九年,上巡盛京謁陵,追念忠王,推恩賜三眼花翎。同治三年,薨,謚曰僖。子德長,襲。光緒二年,薨,謚曰?。子魁斌,襲。

豫通親王多鐸,太祖第十五子。初封貝勒。天聰二年,從太宗伐多羅特部有功,賜號額爾克楚呼爾。三年,從上伐明,自龍井關入,偕莽古爾泰、多爾袞以偏師降漢兒莊城。會大軍克遵化,薄明都。廣渠門之役,多鐸以幼留後,明潰兵來犯,擊??之。師還,次薊州,復擊破明援兵。五年,從圍大凌河城,為正白旗後應,克近城臺堡。明兵出錦州,屯小凌河岸,上率二百騎馳擊,明兵走。多鐸逐之,薄錦州,墜馬,馬逸入敵陣,乃奪軍校馬乘以還千餘。?。六年,從伐察哈爾,將右翼兵,俘其

,但止攻關外?七年,詔問征明及朝鮮、察哈爾三者何先,多鐸言:「我軍非怯於戰,豈可必得?夫攻山海關與攻燕京,等攻耳。臣以為宜直入關,庶饜士卒望,亦久遠計也。且相機審時,古今同然。我軍若弛而敵有備,何隙之可乘?吾何愛於明而必言和?亦念士卒勞苦,姑為委蛇。倘時可乘,何待再計。至察哈爾,且勿加兵;朝鮮已和,亦勿遽絕。當先圖其大者。」八年,從上略宣府,自巴顏珠爾克進。尋攻龍門,未下,趨保安,克之。謁上應州。復略朔州,經五臺山,還。敗明兵大同。九年,上遣諸貝勒伐明,徇山西,命多鐸率師入寧、錦綴明師。遂自廣寧入,遣固山額真阿山、石廷柱率兵四百前驅。祖大壽合錦州、松山兵三千五百屯大凌河西,多鐸率所部馳擊之,大壽兵潰。命分道追擊,一至錦州,一至松山,斬獲無算。翌日,克臺一,還駐廣寧。師還,上出懷遠門五里迎勞,賜良馬五、甲五。上嘉之曰:「朕幼弟初專閫,即能制勝,是可嘉也!」

崇德元年四月,封豫親王,掌禮部事。從伐朝鮮,自沙河堡領兵千人繼噶布什賢兵,至朝鮮都城。朝鮮全羅、忠清二道援兵至南漢山,多鐸擊敗之,收其馬千餘。揚古利為殘兵所賊,捕得其人,斬以祭。三年,伐錦州,自蒙古扎袞博倫界分率巴牙喇及土默特兵入明境,克大興堡,俘其居民,道遇明諜,擒之。詔與鄭親王濟爾哈朗軍會,經中後所,大壽以兵來襲,我軍傷九人,亡馬三十。多鐸且戰且走,夜達鄭親王所,合師薄中後所城。上統師至,敵不敢出。四年五月,上御崇政殿,召多鐸戒諭之,數其罪,下諸王、貝勒、大臣議,削爵,奪所屬入官。上命降貝勒,罰銀萬,奪其奴僕、牲畜三之一,予睿親王多爾袞。尋命掌兵部。十月,伐寧遠,擊斬明總兵金國鳳。

五年三月,命與鄭親王濟爾哈朗率師修義州城,駐兵屯田,並擾明山海關外,毋使得耕稼。五月,上臨視。附明蒙古多羅特部蘇班岱降,上命偕鄭親王以兵迎之,經錦州杏山,明兵來追,奮擊敗之,賜御?良馬一。圍錦州,夜伏兵桑阿爾齋堡,旦,敵至,敗之,追至塔山,斬八十餘級,獲馬二十。六年三月,復圍錦州,環城立八營,鑿壕以困之。大壽城守蒙古將諾木齊約降,師縋以入,擊大壽,挈降者出,置之義州。明援兵自杏山至松山,多鐸與鄭親王率兩翼兵伏錦州南山西岡及松山北嶺,縱噶布什賢兵誘敵,夾擊,大敗之。

洪承疇以十三萬援錦州,上自盛京馳六日抵松山,環城而營,明兵震怖,宵遁。多鐸伏兵道旁,明總兵吳三桂、王樸自杏山奔寧遠,我軍追及於高橋,伏發,三桂等僅以身免。嗣與諸王更番圍松山,屢破敵。七年二月,明松山副將夏承德遣人通款,以其子舒為質,約內應,夜半,我軍梯而登,獲承疇及巡撫邱民仰等。?功,進豫郡王。復布屯寧遠邊外綴明師,俘獲甚夥。

順治元年四月,從睿親王多爾袞入關,破李自成,進親王。命為定國大將軍,南征,定懷慶。進次孟津,遣巴牙喇纛章京圖賴率兵先渡,自成守將走,沿河十五寨堡皆降。再進次陜州,克靈寶。再進,距潼關二十里,自成兵據山列營,噶布什賢噶喇依昂邦努山及圖賴、鄂碩等擊破之。二年正月,自成親率步騎迎戰,師奮擊,殲其步卒,騎卒奔潰。及夜,屢犯屢北,鑿重壕,立堅壁。師進,發巨?迭戰,自成兵三百騎沖我師,貝勒尼堪、貝子尚善等躍馬夾擊,屢破敵壘,尸滿壕塹,械胄彌山野,自成精銳略盡,遁歸西安,其將馬世堯率七千人降。入潼關,獲世堯所遣致自成書,斬以徇。進次西安,自成先五日毀室廬,挈子女輜重,出藍田口,竄商州,南走湖廣。二月,詔以陜西賊付英親王阿濟格,趣多鐸自河南趨淮、揚。師退徇南陽、開封,趨歸德,諸州縣悉降。所至設官吏,安集流亡。詔褒多鐸功,賜嵌珠佩刀、■H7金鞓帶。四月,師進次泗州,渡淮趨揚州,遣兵部尚書漢岱等先驅,得舟三百餘,圍七日,克之,殺明大學士史可法。五月,師再進,次揚子江北岸,明將鄭鴻逵等以水師守瓜洲、儀真。師列營相持,造船二百餘,遣固山額真拜音圖將水師薄南岸,復遣梅勒額真李率泰護諸軍渡江。明福王由崧走太平。師再進,明忻城伯趙之龍等率文武將吏,籍馬步兵二十三萬有奇,使迎師。

多鐸至南京,承制受其降,撫輯遺民。遣貝勒尼堪、貝子屯齊徇太平,追擊明福王。福王復走蕪湖,圖賴等邀之江口,擊殺明將黃得功,獲福王。捷聞,上遣侍臣慰勞。明潞王常淓守杭州,遣貝勒博洛率師討之,潞王降。江、浙底定。多鐸承制改南京為江南省,疏請授江寧、安慶巡撫以下官。別遣精奇尼哈番吳兆勝徇廬江、和州,並下。詔遣貝勒勒克德渾代鎮江寧,召多鐸還京師。上幸南苑行郊勞禮,進封德豫親王,賜黑狐冠、紫貂朝服、金五千、銀五萬、馬十、鞍二。

三年,命為揚威大將軍,偕承澤郡王碩塞討蘇尼特部騰機思、騰機特等。師至盈阿爾察克山,聞騰機思方在袞噶嚕臺,疾行三晝夜,敗之於諤特克山,斬臺吉茂海。渡圖拉河,追至布爾哈圖山,斬騰機特子二、騰機思孫三,盡獲其孥。師次扎濟布喇克,喀爾喀土謝圖汗遣兵二萬,碩雷車臣汗遣兵三萬,迎戰。我師奮擊,逐北三十餘里,先後斬級數千,俘千餘,獲駝千九百、馬二萬一千一百、牛萬六千九百、羊十三萬五千三百有奇。師還,上出安定門迎勞,加賜王鞍馬一。

四年,進封為輔政叔德豫親王,賜金千、銀萬、鞍馬二,封冊增錄功勛。六年三月,以痘薨,年三十六。九年三月,睿親王既削爵,以同母弟追降郡王。康熙十年,追謚。乾隆四十三年正月,詔配享太廟。

多鐸子八,有爵者四:多尼、董額、察尼、多爾博、費揚古。費揚古自三等奉國將軍進封輔國公,坐事,奪爵。

信宣和郡王多尼,多鐸第一子。初封郡王。順治六年十月,襲豫親王。八年,改封信親王。九年,降郡王。十五年,命為安遠靖寇大將軍,偕平郡王羅科鐸等南征。師自湖南入,擊?。明將李定國焚盤江口鐵索橋走,師以浮橋濟,自交水進次松嶺?貴州,趨安莊走明將白文選。十六年正月,薄雲南會城,定國、文選挾桂王走永昌,遣貝勒尚善以師從之,克永昌及騰越。上使慰勞,賜御衣、蟒袍及鞍馬、弓矢。十七年五月,師還,遣內大臣迎勞。六月,追論云南誤坐噶布什賢昂邦瑚理布等磨盤山敗績罪,罰銀五千。十八年正月,薨,謚曰宣和。

子鄂扎,嗣。康熙十四年,命為撫遠大將軍,討察哈爾布爾尼。師次岐爾哈臺,詗知布爾尼屯達祿。鄂扎令留輜重,偕副將軍圖海及梅勒額真吳丹輕騎進。布爾尼設伏待,命分軍搜山澗,伏發,師與土默特兵合擊破之。布爾尼督兵列火器以拒,師奮擊,布爾尼大敗;復收潰卒再戰,又擊殲之,獲馬械無算。布爾尼以三十騎遁,中途為科爾沁部長沙津射死。察哈爾平,撫餘黨一千三百餘戶。師還,上迎勞南苑,詔褒功,賜金百、銀五千。尋掌宗人府事。二十九年,副恭親王常寧備噶爾丹。三十五年,從上北征,領正白旗營。三十八年,以惰,解宗人府。四十一年,薨,以多鐸子董額襲。

信郡王董額,多鐸第三子。初封貝勒。康熙十三年,命為定西大將軍,討叛將王輔臣。董額遣將梅勒額真赫業等守鳳翔,而率師駐西安。詔令進駐蘭州,董額未即行,上復命嚴守棧道。輔臣遣兵毀偏橋,斷棧道。詔責董額遷延,仍趣攻下平涼、秦州諸路。董額進克秦州禮縣,逐敵至西和,克清水、伏羌。復遣安西將軍穆占取鞏昌,蘭州亦下。尋與將軍畢力克圖、阿密達會師攻平涼,久未下。十五年,命大學士圖海視師,改授董額固山額真,聽圖海節制。十六年二月,削貝勒。三十一年,授正藍旗固山額真。四十二年,襲郡王。四十五年,薨。仍坐前罪,不賜恤。以鄂扎子德昭襲。雍正間,歷左、右宗正。乾隆二十七年,薨,謚曰?。以多鐸五世孫如松襲。

如松四世祖多爾博,多鐸第五子。初出為睿親王多爾袞後。多爾袞薨後,削爵。多爾博歸宗,封貝勒。多爾博生蘇爾發,襲貝子。蘇爾發生塞勒,塞勒生功宜布,皆襲輔國公。內大臣,綏遠城、西安將軍。襲爵,?功宜布生如松,歷都統、左宗人、署兵部尚書、領侍復授都統、右宗正。三十五年,薨,謚曰恪。尋以子淳穎襲睿親王,追進封。具睿親王多爾袞傳。

功宜布初薨,以德昭子修齡襲輔國公,授左宗正。四十三年,復襲豫親王。五十二年,薨,謚曰良。子裕豐,襲。嘉慶十八年,林清之變,所屬有從亂者,坐奪爵。弟裕興,襲。二十五年,奸婢,婢自殺。仁宗諭曰:「國家法令,王公與庶民共之。裕興不自愛惜,恣意乾紀,且親喪未滿,國服未除,罪孰大焉!」坐奪爵,幽禁。三年後釋之。弟豫全,襲。道光二十年,薨,謚曰厚。子義道,襲。歷內大臣、左宗正。同治七年,薨,謚曰慎。子本格,襲。亦歷內大臣、左宗正。德宗大婚,賜四團正龍補服。光緒二十四年,薨,謚曰誠。子懋林,襲。

輔國恪僖公察尼,多鐸第四子。順治十三年,封貝勒。康熙七年,授左宗正。十二年,吳三桂反,從順承郡王勒爾錦南征,參贊軍務。師次荊州,三桂已陷岳州。察尼偕將軍尼雅翰舟師進,三桂將吳應麒引七萬人自陸路來拒,擊?之。師次七里山,發?沈其舟十餘。方暑,還駐荊州。十四年,佩靖寇將軍印,援穀城。時南漳、興山已陷,敵逼彞陵,踞鎮荊山,掘壕為寨。察尼至彞陵,議增舟師,斷餉道。擊敵牛皮丫口,進攻黃連坪,焚其積聚,取興山。十五年,三桂移南漳、彞陵兵往長沙,勒爾錦令察尼還荊州,渡江趨石首,據虎渡口,擊敵太平街,斬三百餘級。翌日再出,遇伏,敗還荊州。詔責其無能。十七年八月,貝勒尚善薨於軍,命察尼代為安遠靖寇大將軍,規岳州。疏言:「舟師入湖,賊餉將絕。宜於湖水涸後,圍以木?,立椿列?,以小舟徼巡,為久困計。」上善其言,令副都統關保濟師。尋破敵南津港,斬千級。都統葉儲赫等進攻岳州,復破敵萬餘人。屢疏請增調水陸軍合圍,上皆許之。十八年正月,三桂將王度沖、陳珀等以舟師降,應麒棄城遁,遂復岳州。降官吏六百餘、兵五千餘,獲舟六十五、?六百四十有奇。二月,安親王岳樂自長沙進取衡州,察尼發綠旗兵濟師,尋復湘陰、安鄉。四月,命自常德征辰龍關,澧州以南諸軍聽調度。十九年三月,克辰龍關,復辰州。疏言:「途中霪雨泥濘,士馬須休養。」詔暫屯沅州。六月,詔以貝子彰泰率師下雲南,察尼勞苦久,率滿洲兵還京師。吏議退縮罪,削爵職、籍其家、幽禁,上念克岳州功,命但削爵。二十四年,授奉天將軍。二十七年,卒,賜祭葬視輔國公,謚恪僖。

費揚果,太祖第十六子。太宗時,坐罪賜死,削宗籍。康熙五十二年,聖祖命莽古爾尼雅罕呈宗人府請復宗籍,宗人府以聞,聖祖?泰、德格類子孫復宗籍。費揚果曾孫三等侍曰:「此事朕知之,但不詳耳。費揚果,太祖子,太宗時因獲大罪誅死者。」命復宗籍,賜紅帶。


\end{pinyinscope}