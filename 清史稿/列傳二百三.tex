\article{列傳二百三}

\begin{pinyinscope}
程學啟何安泰鄭國魁劉銘傳張樹珊弟樹屏周盛波

程學啟,字方忠,安徽桐城人。初陷賊中,陳玉成奇其勇,使佐葉蕓來守安慶。咸豐十一年,率三百人自拔來歸。曾國荃使領一營,戰輒請先。安慶北門石壘三最堅,學啟力攻拔之,絕賊糧道。未幾,遂克安慶,學啟功最,擢游擊,賜花翎。從國荃克無為、銅陵諸城,擢參將。

同治元年,李鴻章率淮軍規江蘇,請於曾國籓,以學啟隸麾下。瀕行,國籓勉之曰:「江南人譽張國樑不去口,汝好為之,亦一國樑也!」三月,抵上海,立開字營,凡千人,最為勁旅。屯虹橋,賊猝至,敗之。次日又至,擊退,追至七里堡,大破之,會諸軍克南鎮橋。五月,從鴻章援松江,軍於泗涇,賊酋陳炳文糾悍黨突營,分股繞攻上海,學啟營被圍,力禦,斃賊無算,仍不退。學啟開壁沖突,賊披靡,與諸軍夾攻,乃大潰。松江圍解,擢副將,賜號勃勇巴圖魯。進破賊於青浦東北,復其城。八月,賊酋譚紹光由蘇州來犯,敗之七寶鎮,進戰北新涇,平其壘數十,以總兵記名。

九月,紹光復大舉窺上海,圍水陸各營於四江口,學啟會諸軍進擊,賊扼橋布陣。學啟陷陣,截斷賊隊,胸受砲傷,裹創疾鬥,賊卻走,未渡河者悉殲之。三路圍擊,殲斃落水者數萬,盡毀賊營,以總兵記名加提督銜,授江西南贛鎮總兵。自虹橋、泗涇、四江口三捷,皆以少擊眾,於是增軍至三千人。

二年,進規蘇州,偕鴻章弟鶴章及英將戈登克太倉,賊酋蔡元隆詐降,擊殲之。鴻章令學啟總統諸軍,學啟曰:「昆山三面阻水,一面陸路達蘇州,先斷其陸,乃可克。」偕郭松林破蘇州援賊於正儀鎮,遂克昆山,以提督記名,予一品封典。連拔花涇、同里,克吳江。賊憑太湖結寨,學啟扼飛虹橋,殲其酋徐尚友,乘勝破湖賊,悉平洞庭東山諸壘。

七月,直抵蘇州婁門外永定橋駐軍。蘇州城大,四面阻水,寶帶橋為太湖鎖鑰,賊死力爭拒,合水陸軍大破之,平其壘,親督軍扼守。李秀成自江寧率眾來援,大戰竟日,擊走之。城賊數萬復來爭,亦擊退。進破五龍橋賊壘,留營駐守,分兵破嘉、湖援賊於百龍橋、八坼,逐北至平望。

十月,李秀成糾李侍賢同踞無錫以為援,為劉銘傳、李鶴章所綴,學啟督戰益急,連破賊於蠡口、黃埭,攻破滸墅關及十里亭、虎丘賊壘,於是蘇州之圍遂合。賊自盤門至婁門連壘十餘里,號曰「長城」,亦悉破。秀成知不可為,又江寧被圍急,遂以城守付其黨譚紹光,自出走。

賊酋郜雲官與副將鄭國魁舊識,密介通款,學啟與國魁及戈登單舸見雲官於洋澄湖,令斬紹光為信。秀成行三日,紹光會諸酋議事,雲官即座上殺之,開齊門降。明日,學啟入城,賊酋列名者八人,雲官外,曰伍貴文、汪安均、周文佳、範啟發、張大洲、汪懷武、汪有為,皆歃血為誓,然未薙發,乞總兵副將官職,署其眾為二十營,劃半城為屯。學啟佯許,密請李鴻章誅之。鴻章謂殺降不祥,且堅他賊死拒心,未決。學啟曰:「今賊眾尚不下二十萬,多吾軍數倍,徒以戰敗畏死乞降,心故未服。分城而處,變在肘腋,何以善其後?」鴻章乃許之。次日,諸酋出城謁鴻章,留宴軍中。酒半,健卒百餘挺矛入,刺八人皆死。學啟嚴陣入城,以雲官等首示眾眾曰:「八人反側,已伏誅矣!」賊黨驚擾,殺其悍者數百人,餘不問,分別遣留,皆帖服,蘇州平。乘勝偕李朝斌水師克平望,復嘉善。

三年春,進規嘉興,薄城下,破西門、北門賊壘七,分兵克秋涇、吳涇、合歡橋諸賊壘,逼賊築砲臺。賊自盛澤、新塍來援,皆擊走之,圍攻匝月,毀賊砲臺二十餘。發地雷,裂城百丈,揮軍肉薄而登,忽中槍貫腦,踣而復起,部將劉士奇繼之,遂克嘉興。捷聞,詔嘉其身受重傷,攻拔堅城,命安心醫治,頒賞珍品。尋以創重卒於軍。李鴻章疏陳其兩年之間,復江、浙名城十數,克蘇州為東南第一戰功。優詔賜恤,稱其謀勇兼優,贈太子太保,特遣員賜祭一壇,安慶、蘇州、嘉興建專祠,謚忠烈,予騎都尉兼雲騎尉世職,又加恩予三等輕車都尉世職,並為三等男爵。初學啟投誠時,妻子皆為賊殺,以弟子建勛嗣,襲爵。

何安泰,安徽舒城人。少為傭,陷賊,從學啟來歸,轉戰,無役不從。積功至記名總兵,加提督銜。從攻嘉興,履冰薄城,躍登中槍,死之,贈太子少保,予騎都尉世職。嘉興人哀之,為祠以祀。

鄭國魁,安徽合肥人。咸豐十年,兩江總督何桂清令募勇屯無錫高橋,桂清棄軍走,國魁從提督曾秉忠於上海。初李鴻章督師江蘇,檄領親兵水師後營,四江口、昆山、寶帶橋諸戰,功皆最,累擢至副將。蘇州既合圍,郜雲官與譚紹光不協,國魁遣人說之降,從程學啟會雲官,許雲官等二品武職,折箭誓不殺降,雲官如約獻城。國魁先往宣諭,次日,大軍始入。既而雲官等駢誅,國魁涕泣不食,自謂負約,辭不居功,仍以總兵記名,賜號勃勇巴圖魯。從克嘉興、江陰、常州,予一品封典。同治五年,從剿東捻,駐防山東嶧縣。捻平,以提督記名。光緒中,署天津鎮總兵。卒,附祀學啟專祠,蘇州士民思其功,建祠祀之。

劉銘傳,字省三,安徽合肥人。少有大志。咸豐四年,粵匪陷廬州,鄉團築堡自衛。其父惠世為他堡豪者所辱,銘傳年十八,追數里殺之,自是為諸團所推重。從官軍克六安,援壽州,獎敘千總。

同治元年,李鴻章募淮軍援江蘇,銘傳率練勇從至上海,號銘字營。招撫南匯降賊吳建瀛、劉玉林眾四千人,簡精銳隸其軍。賊由川沙來犯,擊敗之,連克奉賢、金山衛,累功擢參將,賜號驃勇巴圖魯。又破賊野雞墩、四江口,擢副將。常熟守賊以城降,被圍。二年春,銘傳會諸軍克福山,大破賊,解常熟圍,以總兵記名。進規江陰,楊厙為沿江要沖,悍賊堅守,銘傳會黃翼升水師進攻,賊由無錫、江陰兩路來援,迭受創退。李秀成糾眾十餘萬分水陸復來援,銘傳力戰敗之。七月,乘勝攻江陰,擒斬二萬,克其城,以提督記名。尋復無錫,加頭品頂戴。是年冬,進攻常州,敗賊於奔牛鎮。賊目邵小雙降,令扼丹陽。援賊以輪舟至,犯奔牛,以掣圍城之師,奮擊,破三十餘壘,毀其舟。三年春,合圍,破闉而入,擒斬賊首陳坤書,克常州,賜黃馬褂。進屯句容,江寧尋下,餘黨擁洪福瑱踞廣德,會諸軍擊走之。

四年,曾國籓督師剿捻匪,主用淮軍。淮軍自程學啟歿後,銘傳為諸將冠。調駐濟寧,尋分重兵為四鎮。銘傳移駐周家口,迭破賊瓦店、南頓、扶溝,改為移擊之師,擢直隸提督。援湖北,克黃陂,追賊至潁州,大敗之。銘傳建議平原追賊不能制其死命,乃築長堤,自河南至山東運河,驅賊沙河以南蹙之。工甫竣,豫軍防地為賊所破,乃分軍追剿,破之於鉅野。捻酋張總愚竄陜西,任柱、賴文光留山東,自此分為東西。

李鴻章代國籓督師,銘傳專剿東捻,東至鄆城,西至京山,大小數十戰。六年春,賊走尹隆河,與鮑超約期會擊。銘傳先期至,戰失利,部將唐殿魁死之。休屯信陽,整軍復進,追賊至山東。復議自運河至膠、萊,長圍困賊,杜其西趨。時兵、賊俱疲,朝命督戰益急,鴻章專倚銘傳。八月,解沭陽圍。戰贛榆,購降賊內應,槍斃任柱於陣,賊大潰。邀擊濰縣、壽光,薄之洋河、瀰河之間,殲賊幾盡。賴文光走揚州就擒,東捻遂平。國籓、鴻章奏捷,論銘傳為首功,予三等輕車都尉世職。以積勞致疾,乞假去軍。

七年春,張總愚突犯畿輔,急起銘傳赴援,以遲緩被譴責。及至東昌,會諸軍進剿鹽山、滄州、德平,仍用長圍策,蹙之運河東,縱橫合擊,殲賊殆盡,總愚走茌平,陷水死。西捻平,錫封一等男爵。詔屯張秋,九月,命督辦陜西軍務。率唐定奎、滕學義、黃桂蘭等搜剿北山回匪,疏陳大勢,引病乞罷,歸里。

光緒六年,俄羅斯議還伊犁,有違言,急備邊。召銘傳至京,疏陳兵事,略謂:「練兵造器,固宜次第舉行,其機括則在鐵路。鐵路之利,不可殫述,於用兵尤為急不可緩。中國幅員遼闊,防不勝防,鐵路一開,南北東西呼吸相通,無徵調倉皇之慮,無轉輸艱阻之虞,從此裁兵節餉,並成勁旅,一兵可得十兵之用。權操自上,不為疆臣所牽制,立自強之基礎,杜外人之覬覦,胥在於此。」疏上,雖格未行,中國鐵路之興,實自銘傳發之。

十一年,法蘭西兵擾粵、閩,詔起銘傳,加巡撫銜,督臺灣軍務。條上海防武備十事,多被採行。抵臺灣未一月,法兵至,毀基隆砲臺,銘傳以無兵艦不能海戰,伺登陸,戰於山後,殲敵百餘人,斃其三酋,復基隆,而終不能守。扼滬尾,調江南兵艦,阻不得達。敵三犯滬尾,又犯月眉山,皆擊退,殲敵千餘,相持八閱月。十一年,和議成,法兵始退。初授福建巡撫,尋改臺灣為行省,改臺灣巡撫。增改郡、、州、縣,改澎湖協為鎮,檄將吏入山剿撫南、中、北三路,前後山生番,薙發歸化。丈田清賦,溢舊額三十六萬兩有奇,增茶、鹽、金、煤、林木諸稅。始至,歲入九十餘萬,後增至三百萬。築砲臺,興造鐵路、電線,防務差具。加太子少保。十六年,加兵部尚書銜,命幫辦海軍事務。屢因病陳請乞罷,久始允之。

二十一年,朝鮮兵事起,屢召,以病未出。尋卒,詔念前功,贈太子太保,賜恤,建專祠,謚壯肅。

張樹珊,字海柯,安徽合肥人。咸豐三年,粵匪入安徽,樹珊與兄樹聲練鄉兵自衛,淮軍之興,自張氏始。五年,擊賊巢湖,率壯士數十人敗賊,擒斬賊目五人,進破巢縣賊營,敘外委。六年,復來安,隨官軍克無為州,擢千總。又克潛山,至太湖,遇賊數萬,樹珊僅五百人,軍糧火藥皆盡。賊屯堤上,樹珊選死士緣堤下蛇行入賊中,大呼擊殺,賊驚潰。七年,敗捻首張洛行於官亭。粵匪方與捻相勾結,皖北幾無完區,獨合肥西鄉以團練築堡差安,時出境從剿賊。九年,克霍山。十年,兩解六安圍。十一年,赴援壽州,克三河,擢都司,賜花翎。

同治元年,從李鴻章赴上海,名其軍曰樹字營。李秀成犯上海,會諸軍夾擊走之。七月,會克青浦。賊圍北新涇,樹珊偕程學啟力戰旬餘,賊始遁,擢游擊。進克嘉定,賊大舉圍四江口,樹珊偪賊而營,會諸軍奮擊,連破二十餘壘,遂解圍,擢參將,賜號悍勇巴圖魯。是年冬,常熟及福山賊以城降,而福山賊復叛,圍常熟。二年正月,樹珊率軍航海抵福山西洋港,風潮作,飄舟近賊巢,潮退不得行。樹珊曰:「兵法危地則戰。」登岸結壘未就,賊大至,樹珊疾搗中堅,槍傷左肘不少卻,拔出諸營之被圍者,進解常熟之圍,擢副將。會諸軍進攻江陰,樹珊扼南門,斷賊去路,城復,賊無得脫者,以總兵記名。進攻無錫,悍酋陳坤書、李世賢方以十萬眾圍大橋角,樹珊助剿,火賊輪船二、砲船十,殲斃甚眾,解其圍。李秀成復率眾數萬至,連營數十里,樹珊與諸軍夾擊,賊大潰。會蘇州已下,秀成率死黨入太湖,結常州賊,水陸分進,援無錫;時銘傳專擊外援賊,樹珊與諸軍合圍,十一月,拔之,以提督記名。偕兄樹聲及劉銘傳進攻常州,三年四月,克之,予一品封典,授廣西右江鎮總兵。

四年,曾國籓督師剿捻,駐徐州,以樹珊所部為親軍,令援山東,破賊於魚臺。議設四鎮,陳州之周家口為最要,初以劉銘傳駐之,既改銘傳為游擊之師,乃令樹珊移駐。五年三月,擊賊沙河,賊竄撲周家口,回軍夾擊敗之。五月,又敗賊於沙河東,樹珊以賊騎飄勿靡常,恥株守,請改為游擊之師。九月,馳解許州之圍。十月,逐賊山東境,連敗之豐南、定陶、曹縣。十一月,回軍周家口。賊竄湖北,偕總兵周盛波追剿。會郭松林敗績於臼口,賊焰愈熾,樹珊自黃岡追至棗陽,賊竄黃州、德安,樹珊馳援。諸將皆言賊悍且眾,宜持重,樹珊率親軍二百人窮追,抵新家徬。賊橫走抄官軍後,樹珊力戰陷陣,至夜半,馬立積尸中不能行,下馬鬥而死。後隊據鄉莊發槍砲拒賊,賊亦尋退,全軍未敗。事聞,詔惜其忠勇,從優議恤,予騎都尉兼一雲騎尉世職,建專祠,謚勇烈。七年,捻平,加贈太子少保。

弟樹屏,從諸兄治團練,積勞至千總。從樹珊至江蘇,轉戰松江、蘇州、常州,屢有功,累擢副將。從剿捻匪,迭破賊於豐縣、沛縣、魚臺。及樹珊戰歿德安,樹屏分領樹字三營駐周家口。東捻平,論功以提督記名,賜號額騰額巴圖魯。

同治六年,山西巡撫李宗羲奏調,令募新軍六營分駐大寧、吉州、壺口防回匪。十二年,兼統水陸駐河津,分防歸化、包頭。光緒二年,甘肅流賊犯河套後山,督軍追擊,連敗之,擒其渠曹洪照。事平,加頭品頂戴。四年,授太原鎮總兵,值旱災,樹屏捐運賑糧,出軍食之餘平糶濟饑民。六年,移防包頭。九年,調大同鎮。十三年,因傷病乞罷,十七年,卒,以前勞賜恤。

周盛波,字海舲,安徽合肥人。咸豐三年,粵匪陷安慶,皖北土匪紛起,盛波兄弟六人,團練鄉勇保衛鄉里,屢出殺賊。兄盛華及弟三人皆死事,惟存盛波與弟盛傳,以勇名。陳玉成、陳得才等屢擾境,盛波等以練丁二千隨方迎敵,相持數年,遂越境出剿近縣,餉械皆所自備,累獎守備。

同治元年,李鴻章募淮軍援江蘇,令盛波就所部選募成軍,曰盛字營。從至上海,破賊於北新涇,擢游擊。又大破賊於四江口,賜號卓勇巴圖魯。二年,克太倉,進昆山,扼雙鳳橋,復縣城。破麥市橋賊壘,擢副將。進攻江陰,擊敗援賊。會克縣城,以總兵記名。會攻無錫,毀賊船百餘,破惠山石卡,擒賊酋黃子隆,以提督記名,予一品封典。三年,合圍常州,盛波由小南門攻入。賊首就擒,以總兵侭先題奏。時江寧已復,餘黨黃文英走踞廣德,盛波追之至橫山,文英遁走。城賊拒戰,敗之,復廣德,進至寧國境而還,賜黃馬褂。

四年,從曾國籓剿捻匪。張總愚圍雉河集,盛波赴援,循渦河岸破賊。英翰軍突圍夾擊,圍始解。授甘肅涼州鎮總兵,敗捻匪於寧陵。五年,拔菏澤游莊寨、方埠賊巢,被珍賚。牛洛紅竄亳州,截擊於白龍王廟,大破之。是年冬,追賊雲夢,連敗之於兩河口、沙河、胡家店。六年,躡追任柱至信陽,與弟盛傳分路蹙之臺子畈山中,賊舍騎四竄,追及談家河,擒賊目汪老魁等。賴文光來援,復擊敗之。九月,破沭陽程寨賊,又敗之於石榴寨、高家寨,追至海州阿胡鎮,殲悍黨趙天福,東捻尋平。

七年,西捻張總愚竄畿輔,盛波追至陵縣土橋,馬步合擊,賊潰走。五月,盛波駐毛家莊,賊由吳橋來犯,設伏痛擊,斬級數千。襲賊於楊丁莊,陣斬總愚之侄張三彪。六月,會擊於茌平,總愚走死。西捻平,晉號福齡阿巴圖魯。

軍事定,以母老陳請回籍終養,尋以前年所部攻破河南唐縣民寨,慘斃多命,為巡撫李鶴年所劾,褫職,交李鴻章按治,以盛波身在前敵,免其科罪。九年,鴻章疏陳盛波功多,復原官。光緒十年,命在淮北選募精壯十營赴天津備防,責司訓練。丁母憂,奏,許弟盛傳回籍治喪,盛波仍留營。盛傳尋卒,所遺湖南提督即以盛波代署,疏辭,不允。服闋,實授。十四年,卒。

李鴻章疏陳戰績,謂其治軍嚴而不苛,人樂為用。善察地勢,審賊情,部曲經其指授,輒有家法。防海以來,所部為淮軍最大之軍,諸軍勛望無出其右。詔優恤,建專祠,謚剛敏。

周盛傳,字薪如,盛波之弟。盛傳偕諸兄集丁壯團練。咸豐三年,粵匪擾合肥,率百餘人擊敗之,擒賊目馬千祿。五年,兄盛華陣亡,盛傳與盛波分領團從,防戰數有功,獎敘把總。十一年,赴援壽州,擢千總。

同治元年,盛波從李鴻章援江蘇,盛傳充親兵營哨官,從克嘉定及戰四江口,累擢游擊。二年,回籍增募勇丁,會攻太倉,賊酋蔡元隆詐降,設伏狙擊官軍,盛傳獨嚴備,不為所挫。越數日,偕諸軍一鼓克之,駐軍雙鳳鎮,為賊所圍,連戰三晝夜,破之,克昆山,賜號勛勇巴圖魯。攻江陰,毀東門賊營,城復,擢參將。迭戰東亭鎮、興隆橋、鴨城橋、西倉,遂克無錫,功尤多,超擢以總兵記名。進攻常州,三年,進逼郡城南門,賊突出拒,盛傳且戰且築營,賊屢抄後路,皆擊退。登石橋督戰,橋斷墮水,又受砲傷,絕而復甦。越數日,裹創會攻,攀城先登。克常州,詔以總兵遇缺先行題奏,加提督銜。以撫標親兵三營改為傳字營,盛傳始獨領一軍,移防溧陽。尋會銘軍克廣德州。

四年,調剿捻匪,偕兄盛波援雉河集,自睢寧、宿州轉戰而前。將至,捻酋任柱以馬隊突犯,盛傳堅陣不動,出奇兵抄賊後,賊始卻,會諸軍夾擊,賊潰走,以提督記名。移防歸德。五年春,迭敗賊於考城、鉅野、城武、菏澤,詔嘉盛傳兄弟苦戰,同被珍賚。五月,偕盛波破牛洛紅於亳州,洛紅被創夜遁,道死。追賊扶溝、鄢陵、許州,扼防周家口。時以長圍困賊,盛傳築賈魯河長墻,檄調為游擊之師,解柘城、羅山圍。六年,授廣西右江鎮總兵,偕盛波蹙賊信陽譚家河,斬馘逾萬。追賊入山東,至江北海州,捻匪大衰。是年冬,任柱、賴文光均就殲。

七年春,偕盛波渡河會剿張總愚,敗賊於山東、直隸之間,守運河長墻。盛傳伏炸砲於吳橋毛家莊,合馬步逼賊入伏,砲發,賊尸蔽野。既而茌平合圍,總愚走死,賜黃馬褂。盛波乞假養親,盛傳代統全軍,從李鴻章移師湖北。

九年,從鴻章赴陜西剿回匪,賊踞宜川山中,督軍進剿,破之於河兒川、孔巖寨,分兵於宜、洛、鄜、延之間,以遠勢兜圍,先後擒賊酋馬志龍、戴得勝,北山悉平。

是年秋,鴻章移督直隸,疏調盛傳率所部屯衛畿輔。十年,移屯青縣馬廠。十二年,興修大沽北塘砲臺,築內外土城各一,大砲臺三,環置小砲七十有一。兵房、藥庫、倉廒、義塾及城外溝、河、橋、徬悉備,以所部任其役,捐盛軍欠餉以濟工費。十三年九月,工竣,詔遇提督缺出先行簡放。

時鴻章奉敕興復京畿水利,盛傳任津沽屯田事,履勘天津東南縱橫百餘里,沮洳蕪廢,議疏潦、濬河渠,引淡滌咸,以變斥鹵。光緒二年,調天津鎮,移屯興工,開南運減河,自靳官屯抵大沽海口,減河兩岸各開支河一、橫河六,溝水會河渠悉如法。建橋徬五十餘處,備蓄洩,使淡水咸水不相滲混,成稻田六萬餘畝。濱河斥鹵地沾水利,可墾以億計。至六年工竣。

八年,擢湖南提督,仍留鎮訓練士卒,悉用西法,著操槍章程十二篇,軍中以為法式。

十年,丁母憂,命改署理,予假回籍治喪。盛傳事親孝,未幾,以哀毀傷發卒,詔優恤,謚武壯,建專祠。

潘鼎新,字琴軒,安徽廬江人。道光二十九年舉人,議敘知縣。咸豐七年,投★安徽軍營,從克霍山,擢同知。十一年,父璞領鄉團助剿,被執不屈死。鼎新誓殺賊復仇,請分兵攻三河鎮,克之,負父骸歸。曾國籓聞而壯之,時方創淮軍,令募勇立鼎字營。

同治元年,從李鴻章援上海,連克奉賢、川沙、南匯,以知府用。克金山,又破賊虹橋,擢道員。二年,攻福山鎮,鼎新以開花砲炸賊壘,克之,解常熟圍,授江蘇常鎮通海道,以父喪未除,改署任。連破賊於楓涇及嘉善、西塘,加按察使銜。克平湖、乍浦、海鹽,獲賊銀三十餘萬兩充餉。破賊於興城、沈蕩、新豐。三年,會克嘉興,戰吳漊、南潯,會攻湖州,賊拒守晟舍,攻兩晝夜,傷脅,破升山九壘,奪三里橋,直抵城下,克湖州,加布政使銜,賜號敢勇巴圖魯。蘇、浙既定,賜黃馬褂,駐屯松江。

四年,僧格林沁戰歿,捻匪益熾,畿輔震動,詔徵勁旅入衛,李鴻章遣鼎新率砲隊航海赴天津。尋命所部十一營移駐濟寧,擢山東按察使。擊敗捻首賴文光於豐縣陳家莊,又追敗之於沛縣、魚臺、定陶。五年,敗賊於鉅野,解鄆城圍。築運河沿岸長墻,開黑風口淤河,引泗水灌之。賊屢敗於西華、太康,竄至油坊岡,鼎新夾擊,殪其酋。又追賊鄆城、菏澤、曹縣、東明,竄入河南境,追擊於杞縣柿園、嘉祥臥龍山。六年,遷山東布政使。築新河、濰河長墻,會諸軍守之。賊由東軍汛地偷渡濰河,沖出南竄,都司董金勝率馬隊尾追,敗之莒州、沭陽。鼎新追至海州石榴橋,據山下擊,時賊尚五六萬,連戰於馬陵山、臥龍寨,賊張兩翼來犯,鼎新為圓陣,賊不能撼,伺懈突擊,斬馘甚眾。追敗之剡城柴戶店、海州上莊,斬級千餘,殪賊目楊天燕、陳天福,其酋李宗世等乞降,加頭品頂戴。捻首任柱、賴文光先後就殲擒。

七年,馳援畿輔,鼎新至饒陽,賊趨保定,繞其前迎擊,敗之。尋破賊於滄州郭橋、柳橋,殪其酋羅六。又戰高唐、吳橋,於捷地開減河,築長墻,抵東昌。迭蹙賊於德平、陽信、商河,與諸軍合擊。西捻平,予雲騎尉世職,晉一等輕車都尉。

尋命從左宗棠剿回匪,鼎新請開缺省親。九年,丁母憂。服闋,李鴻章奏留辦天津海防。十三年,授雲南布政使。光緒二年,就擢巡撫,與總督劉長佑不合,三年,命來京另候簡用,乞假歸。五年,召天津隨辦防務,七年,回籍。

十年,法越兵事起,起署湖南巡撫,調授廣西巡撫。時徐延旭出關兵挫,故以鼎新代之,命按治提督黃桂蘭等失律罪,讞擬輕縱,嚴旨斥責。命督軍進諒山,扼屯梅谷、松堅牢諸隘,鼎新奏請諸軍歸雲貴總督岑毓英節制,自為之副,不允。又私謂終歸和局,以節餉為主,不得士心。初戰船頭、紙作社,奏捷。十二月,法兵大舉來犯,諒山陷,師退,自請治罪,詔帶罪立功。十一年正月,鎮南關失守,總兵楊玉科戰死,喪提督劉恩河以次十餘員。鼎新傷肘墜馬,倉皇失措,退至龍州,詔奪職。法兵由艽封窺龍州,賴馮子材、蘇元春、王德榜諸軍力戰,大破之,復鎮南關,追躡連捷,克諒山。和議旋成,鼎新乃解任回籍。十四年,卒於家。李鴻章疏陳前功,乞恩復原官。

吳長慶,字筱軒,安徽廬江人。父廷香,在籍治團練,咸豐四年,殉寇難,恤,予雲騎尉世職,見忠義傳。長慶襲世職,繼父領鄉團,先後從官軍克廬江、舒城,擢守備。十一年,會攻克三河。淮軍始創,領五百人,曰慶字營。

同治元年,從李鴻章至上海,破賊於虹橋,克奉賢、南匯、川沙,又破寶山竄賊,超擢游擊。二年,回籍募勇,會李秀成糾眾圍廬江,長慶登陴固守,出擊賊,走之。事定,率新募五營赴上海,進攻楓涇、西塘,克之,毀千窯賊巢,擢副將。規嘉善,破張涇匯賊壘。三年,會攻嘉興,左臂中槍,督士卒緣城上,克之,以總兵記名,賜號力勇巴圖魯。自是分兵援浙、閩,迭克郡縣。五年,追敘以提督總兵侭先題奏。

七年,從李鴻章剿捻匪,轉戰河南內黃、滑、濬,山東臨邑、德州,直隸寧津。捻平,賜黃馬褂,晉號瑚敦巴圖魯。調防江北,駐軍徐州。八年,鼎軍譁變,長慶扼截,斬其倡亂者,眾懼服,分別資遣數千人,旬日而定。事聞,予議敘。九年,移駐揚州,丁母憂,予百日假,仍留軍濬鹽河,興水利。尋復移屯江浦、江陰。十三年,增募四營築江陰、江寧砲臺。光緒元年,授直隸正定鎮總兵,仍留防江南。六合鄉民因漕重聚眾譁署,長慶馳至諭散,為請奏減漕額。寧國教民白會清不法,激變,毀教堂,構訟。建平人何渚被枉,長慶往按得實,為白於總督沈葆楨,平反之。率士卒濬江浦黑水河、四泉河、玉帶河,兩年始畢工。六年,擢浙江提督。尋調廣東水師提督,未之任,會法越軍事起,命幫辦山東軍務,四鎮皆歸節制,率所部屯登州。

八年,朝鮮內亂,禁軍犯王宮,殺大臣,王妃失蹤,燔日本使館,日本且發兵。命長慶率兵艦三往按治,先日兵至。廉知事由朝鮮王父大院君李昰應所主,至則昰應尚踞王宮,來謁,留語及暮,遣隊擁赴海口,命兵艦致之天津,次日擊散亂黨,迎復王妃。日本初欲藉故多所要挾,見事已定,氣為之沮。詔嘉其功,予三等輕車都尉世職,遂留鎮漢城。長慶在朝鮮兩年,修治道塗,救災恤民,示以恩信,國人感之。

十年,命移防金州,尋卒。詔優恤,建專祠,謚武壯,予其次子保初主事。保初後官刑部,上書言時政,辭職歸。

長慶好讀書,愛士,時稱儒將。保初亦文雅,有文風。

論曰:李鴻章創立淮軍,一時人材蔚起,程學啟實為之魁,功成身殞,開軍遂微。銘軍最稱勁旅,樹軍、盛軍、鼎軍亦各驂靳。粵寇平而捻匪熾,曾國籓欲全湘軍末路,主專用淮軍,平捻多賴其力。其後北洋籌防,全倚淮軍,而以盛軍為之中堅。劉銘傳才氣無雙,不居人下,故易退難進。守臺治臺,自有建樹。二張、二周,治軍皆有家法。潘鼎新防邊失律,不保令名。吳長慶戰績雖亞諸人,朝鮮定亂,能弭大變。及甲午邊釁起,宿將彫零,衛汝貴、葉志超等庸才僨事,為全軍之玷。後起僅一聶士成,庚子殉難,淮軍遂熸。四十年中,盛衰得失,於此見焉。


\end{pinyinscope}