\article{列傳一百四十八}

\begin{pinyinscope}
劉清傅鼐嚴如熤子正基

劉清,字天一,貴州廣順人。由拔貢議敘,授四川冕寧縣丞,擢南充知縣,政聲為一省之冠。

嘉慶元年,教匪起,清得民心,募鄉勇五百人擊賊,人樂為用。賊自為民時知其名,遇輒避之。繼從總督英善剿達州匪徐天德,數捷,率鄉勇羅思舉赴賊營諭降羅其清,未得要領;而徐天德與王三槐、冷天祿合陷東鄉,二年春,始復之,遂署東鄉。進克清谿場,擒賊黨王學禮,天德之舅也,言天德與王三槐皆有歸順意。總督宜綿令清往招三槐,遍歷諸賊壘,迎送奉酒食甚謹,宣示招撫,皆聽命,夜宿其帳中。三槐隨至大營,約期率所部出降,然實藉覘虛實,非真意。屆期,三槐詭稱於雙廟投降,伏匪為掩襲計,官軍預設備,擊敗之。時羅其清、冉文儔並聚方山坪,清偕總兵百祥奪多福山賊壘,會諸路兵攻方山坪,克之。賊竄通江、巴州,與徐天德、王三槐合,清所部鄉勇增至千餘人,桂涵、李子青等皆驍勇善戰,偕諸軍擊賊,疊有殲獲,羅、冉二匪漸蹙。

三年,署廣元縣事。總督勒保攻王三槐於安樂坪,未下,復令清往招撫。三槐恃前此出入大營無忌,留隨人劉星渠等為質,三槐遂詣軍門,勒保奏報大捷,俘三槐至京。廷訊時,言:「官逼民反。」仁宗詰之曰:「四川一省官皆不善耶?」對曰:「惟有劉青天一人。」劉青天者,川民以呼清也。帝深嘉之,特諭曰:「朕聞劉清官聲甚好,每率眾御敵,賊以其廉吏,往往退避引去。如果始終奮勇,民情愛戴,著勒保據實保奏。」尋以清治績戰功奏上,晉秩同知直隸州,賜花翎。於是劉青天之名聞天下。

四年,補忠州,加知府銜。參贊額勒登保破冉天元、張子聰於竹峪關,令清於通江、巴州招撫餘匪。自王三槐被誘,諸賊首皆疑憚不敢出;然感清無他,不忍加害,每至賊營,必留宿盡禮,其脅從者先後投出二萬餘人,遣散歸農,以功加道銜。命隨副都御史廣興駐達州治軍餉,擢建昌道。五年,冉天元等合諸路賊渡嘉陵江,總督魁倫退守鹽亭鳳凰山,令清集民團守潼河,上下三百餘里,多淺灘,盡撤防兵;清爭之,不可。賊果於太和鎮上游王家嘴偷渡,委罪於清,奪職,命以知縣用,留營效力。既而德楞泰破賊,天元伏誅,諸路竄賊旁皇通、巴之間,勒保以清去歲招降成效,責籌安撫。時川匪父子兄弟一家中不盡習教為賊,而奔竄往來,過鄉里輒歸視。清屯要隘,且剿且撫,遣人存問賊首家屬有歸誠之意者,潛令圖之,展轉相引,賊遂瓦解。藍號鮮大川,巴州人,號為狡悍。其族人文炳、路保及黨楊似山,清皆厚恤其家,感恩原效死,乃使文炳勸大川降,不可,且與似山謀殺文炳。似山乘間殺大川,與文炳、路保同降。巴州匪遂滅。六年,以功復原官,仍授建昌道。七年春,破賊於南江五方坪,擒賊首李彬及辛文等,加按察使銜,尋授四川按察使。敗藍號齊國典餘匪於兩河口,追擒其黨葛成勝。諸匪以次平,大功告蕆,下部議敘。

清在軍七年,先後招降三萬餘人。有業者歸鄉里,無業及有業原從者為鄉勇,後立戰功者三十餘人。其中茍崇勛、茍文耀、李彬、辛文、李世玉、趙文相,皆賊魁也。崇勛即茍文通,已奏報殲斃而改名。及軍事竣,當遣,清以諸人田廬焚蕩,驟散將復為賊,臨行重犒之。自向富室巨商貸金,人感其誠,多響應。事畢,積逋負至十萬。

八年,陜西餘匪自南山竄出棧道,清馳扼廣元,遣卒招撫被戕,詔斥輕信縱賊,以前功免罪,命理糧餉及搜捕餘匪、裁撤鄉勇。十年,事竣入覲,賜禦制詩,有曰:「循吏清名遐邇傳,蜀民何幸見青天!誠心到處能和眾,本性從來不愛錢。」時以異數榮之。丁繼母憂,去官,服闋,授山西按察使,遷布政使。忤巡撫初彭齡,劾其袒護屬吏,降四級,以從四品京堂用。清亦自陳不勝籓司之任,詔斥冒昧,降補刑部員外郎。熱河新設理刑司員,以清往,邊方草創,多持大體,斷獄平允,蒙民亦以青天呼之。

十七年,授山東鹽運使。十八年,河南教匪起,山東賊黨硃成良等應之,陷定陶、曹縣,巡撫同興恇懼,清自請將兵。承平久,兵習晏安,清躡草屩先之,以五百人敗賊於仿山,復定陶,又敗之於韓家廟,殪賊二千,進攻扈家集,縱火焚柵,賊突出皆死,誅賊首硃成良、王奇山,自滑縣奔至者並殲焉,兩閱月而事平。賊初起時,煽惑甚眾,清先解散其脅從,成良勢孤不得逞,故得速定。上嘉其以文職身先士卒,特詔褒獎,加布政使銜。尋授雲南布政使,仍留舊任。

清性坦率,厭苛禮,不合於上官,又不耐簿書錢穀,遂乞病,上亦知之,改授山東登州鎮總兵,調曹州鎮。道光二年,以老休致,命在籍食全俸。八年,卒,賜祭葬,祀山東名宦,官其孫熾昌為兵部主事;瑩,舉人。

傅鼐,字重庵,順天宛平人,原籍浙江山陰。由吏員入貲為府經歷,發雲南,擢寧洱知縣。乾隆末,福康安征苗疆,調赴湖南軍營司餉運,晉秩同知直隸州,賜花翎。

嘉慶元年,授鳳凰同知。治當苗沖,會大軍移征湖北教匪,降苗要求苗地歸苗,當事議允之。鼐知愈撫且愈驕,乃招流亡,團丁壯,於要害築碉堡,防苗出沒。苗以死力來攻,且戰且修,閱三年而碉堡成。有哨臺以守望,砲臺以禦敵,邊墻相接百餘里。每警,哨臺舉銃角,婦女、牲畜立歸堡,環數十里皆戒嚴。四年,擒苗酋吳陳受,加知府銜。巡撫姜晟疏薦鼐能勝艱鉅,方治鎮筸一帶荒田,均給丁壯,請俟事竣送部。時鎮筸左、右營黑苗最為邊患,五年,跴金塘苗出掠瀘溪,偕總兵富志那夜分三路搗其巢,伏兵隘路茍巖要擊,痛殲之,斃首逆吳尚保,苗始奪氣。詔嘉獎,命在任食知府俸。

六年,貴州苗復亂,湖南環苗地東、南、北三面七百餘里,其西二百餘里接貴州,未設備。石峴苗煽十四寨糾湖南苗叛,鼐率鄉勇千五百馳赴銅仁。貴州巡撫伊桑阿以招撫戡定上聞,各寨實尚沸然,槍械未繳。總督瑯玕至,急檄鼐會剿崖屯溝,黔兵攻其前,鼐夜由山徑入,連破五巢。上下湖山峽尤險,夜分兵圍攻,至次日克之,火其寨。三日中盡破諸寨,殲苗二千有奇。仿湖南法,建碉堡守之。伊桑阿因冒功誤邊伏法,錄鼐功,加道銜,總理邊務,並命以苗疆道員用。七年,丁父憂,詔鼐辦理邊防善後,民、苗悅服,難易生手,命留任。初,鼐建議遷永綏城於花園,副將營於茶洞,而貴州方藉永綏為聲援,尼其事。至是詔瑯玕察奏,乃赴銅仁面陳永綏孤懸苗中,形如釜底,有二難、三可慮;並請移湖南守備於貴州邊境螺螄堡,以為犄角,乃決議移之。既而群苗率眾來爭,鼐率鄉勇深入,苗大集,環之數重,以奇計突圍出。尋議勒繳槍械,苗酋石崇四等抗命,並阻丈田,十年,與其黨石貴銀糾眾數千來犯,敗之夯都河,追至孟陽岡,殲賊甚眾,生擒石崇四、石貴銀。是役因賊戕良苗,故得用苗兵深入,戰月餘,破寨十六,餘皆乞降,永綏苗遂平。屬高都、兩頭羊二寨皆震懾,無敢抗。事聞,予優敘,擢辰沅永靖道。

鼐治苗專用雕剿法,大小百戰,所用僅鄉勇數千。苗人於穹山峭壁驀越如平地,無部伍行列,伏箐中從暗擊明,銃銳且長,隨山起伏,多命中。鼐因苗地用苗技訓練士卒,囊沙輕走,習藤牌閃躍,狹路則用短兵。每戰後輒嚴汰,數年始得精卒千,號「飛隊」,風雨不亂行列,遺資道路無反顧,甘苦與共,是以能致死。

先是議興屯田,上書巡撫高杞曰:「防邊之道,兵民相輔。湖南苗疆,環以鳳凰、永綏、乾州、古丈坪、保靖五縣,犬牙相錯,營汛相距各數里。元年班師後苗擾如故,鼐竭心籌之,制勝無如碉堡。募丁壯數千,與苗從事。來則痛擊,去則修邊,前戈矛,後邪許。得險即守,寸步而前,然後苗銳挫望絕。湖南自乙卯二載用兵,耗帑七百餘萬。國家經費有常,頑苗叛服無定。募勇不得不散,則碉堡不得不虛;後患不得不慮,則自圖不得不亟。通力合作,且耕且戰,所以招亡拯患也。均田屯丁,自養自衛,所以一勞永逸也。相其距苗遠近、碉堡疏密,為屯田多少:鳳凰碉堡八百,需丁四千輪守,並留千人備戰,需田三萬餘畝;乾州碉堡九十餘,守丁八百,屯田三千餘畝;保靖縣碉堡四十餘,守丁三百,屯田千五百餘畝;古丈坪苗馴,止設碉堡十餘,守丁百,屯田五百餘畝;永綏新建碉堡百餘,留勇丁二千,亦屯田萬畝:而後邊無餘隙,環苗以成圈圍之勢,峻國防、省國計也。異族逼處,非碉堡無以固,碉堡非勇丁無以守,勇丁非屯田無以贍。邊民瀕近鋒鏑,固原割世業而保身家;後路同資屏蔽,亦樂捐有餘以補不足。所募土丁,非其子弟即其親族。距邊稍遠者,仍佃本戶輸租,視古來屯戍以客卒雜處,勢燕越矣。與其一旦散數千驍健無業子弟流為盜賊,何如收駕輕就熟之用而不費大帑一錢?惟執事圖之!」於是收叛產分給無業窮苗佃種。

自擒石崇四,餘匪原返侵地,永綏得萬餘畝,乾州、鳳凰二次之,乃續墾沿邊隙地二萬畝,曰「官墾田」,贖苗質民田萬餘畝,曰「官贖田」。以廩屯官授屯長,給老幼,籌補助,備犒賞,暨歲修城堡、神祠、學校、育嬰、養濟諸費。復以兵威勒交苗占民田三萬五千餘畝,苗自獻田七千餘畝。其經費田則佃租變價,屯丁田則附碉躬耕,訓練講武,設屯田守備掌之,轄於兵備道。屯政舉,使兵農為一以相衛,民、苗為二以相安。與官及兵民約曰:「毋擅入苗寨,毋稍役苗夫。」與苗約曰:「毋巫鬼椎牛群飲以糜財,毋挾槍矛尋睚眥釀釁。」請乾、鳳、永、保四編立邊字號,廣鄉試中額一名;苗生編立田字號,加中額一名,苗益感奮。十三年,屯務竣,入覲,詔曰:「傅鼐任苗疆十餘年,鋤莠安良,興利除弊,建碉堡千有餘所,屯田十二萬餘畝,收恤難民十餘萬戶,練兵八千人,收繳苗寨兵器四萬餘件;又多方化導,設書院六,義學百,近日苗民向學,革面革心。朕久聞其任勞任怨,不顧身家。今召見,果安詳諳練,明白誠實,洵為傑出之才,堪為巖疆保障。其加按察使銜,以風有位。」

十四年,擢湖南按察使。苗人籥留,命每年秋一赴苗疆撫慰邊人。鼐在苗疆,設木匭於門,訴者投牒其中,夜出閱之,黎明起視事,剖決立盡。兵民白事,直至榻前。及為按察使,一如同知時。下無壅情,事無不舉。十五年,兼署布政使。十六年,卒於官,仁宗深悼惜,詔謂:「倚畀方隆,正欲簡任疆寄。加恩贈巡撫銜,照贈官賜恤,賜祭一壇。」苗疆建專祠,祀湖南名宦。光緒中,追謚壯肅。

初,鼐排眾議以事攻剿,為大吏所惎,將中以開邊釁罪。監司阿意,旁掣其肘,鎮筸總兵富志那獨保全之。富志那從征金川,習知山碉設險之利,鼐實從受之,卒以成功。鼐歿後,二妾寡居,食於粥不給,其廉操尤著雲。

嚴如熤,字炳文,湖南漵浦人。年十三,補諸生,舉優貢。研究輿圖、兵法、星卜之書,尤留心兵事。

乾隆六十年,貴州苗亂,湖南巡撫姜晟闢佐幕,上平苗議十二事,言宜急復乾州,進永綏,與保靖、松桃、鎮筸聲勢可通。攻乾州道瀘溪,必先得大小章。大小章者,故土司遺民,名曰仡佬,驍健,與苗世仇。如熤募能仡佬語者往,開示利害,挾其酋六人出,推誠與同臥起,乃送質,率其屬陽投乾州為內應,約一舉破賊,因黔師牽掣未果。次年,卒賴其眾,救兩鎮兵於河溪。後復平隴,戰花園,皆為軍鋒。大小章於大府檄或不受,必得如熤手書始行雲。

嘉慶五年,舉孝廉方正。廷試平定川、楚、陜三省方略策,如熤對幾萬言,略謂:「軍興數載,師老財匱。以數萬罷憊之眾,與猾賊追逐數千里長林深谷中。投誠之賊,無地安置,則已降復亂;流離之民,生活無資,則良亦從亂。鄉勇戍卒,多游手募充。慮一旦兵撤餉停,則反思延亂。如此,則亂何由弭?臣愚以為莫若仿古屯田之法。三省自遭蹂躪,叛亡各產不下億萬畝,舉流民降賊之無歸、鄉勇戍卒之無業者,悉編入屯,團練捍衛,計可養勝兵數十萬。餉省而兵增,化盜為民,計無逾此。」仁宗親擢第一。次日,召詣軍機處詢屯政,復條上十二事。召見,以知縣發陜西。下其疏於三省大吏,令採行。

六年,補洵陽,縣在萬山中,與湖北邊界相錯,兵賊往來如織。時方厲行堅壁清野,如熤於築堡練團,措置尤力。賊至無可掠,去則抄其尾。又擇堅寨當沖者,儲糧供給官軍。徐天德、樊人傑敗於張家坪,因馬鞍寨阻其前,故不得竄。楊遇春破張天倫,亦賴太平寨夾擊之力。以功加知州銜,賜花翎。八年,擊湖北逸匪於蜀河口,斬王祥,擒方孝德,晉秩同知直隸州。新設定遠,即以如熤補授。九年,建新城,復於西南百餘里黎壩、漁渡壩築二石城為犄角。治團如洵陽,賊至輒殲,先後擒陳心元、馮世周。丁母憂,大吏議留任,辭不可,服闋,十三年,補潼關。尋擢漢中知府。兵燹後,民困兵驕,散勇逸匪,心猶未革。如熤聯營伍,立保甲,治堡寨,問民疾苦。興勸農事,行區田法,教紡織,使務本計。修復褒城山河堰及城固五門、楊填二堰,各灌田數萬畝,他小堰百餘,皆履勘濬治,水利普興。復漢中書院,親臨講授。於華州渭南開諭悍回,縛獻亡命數十人;於寧羌解散湖北流民;於城固擒教首陳恆義:皆治渠魁,寬脅從。令行禁止,人心帖服,南山遂大定。

道光元年,擢陜安道。會廷議川、楚、陜邊防建設事宜,下三省察勘,以如熤任其事,周歷相度,析官移治,增營改汛,建城口、白河、磚坪、太平、佛坪五,移駐文武。奏上,報可。如熤嘗言:「山內州縣距省遠,多推諉牽掣。宜仿古梁州自為一道及明鄖陽巡撫之制,專設大員鎮撫,割三省州縣以附益之,庶勢專權一,可百世無患。」以更張重大,未竟其議。三年,宣宗以如熤在陜年久,熟於南山情形,任事以來,地方安靖,特詔嘉獎,加按察使銜,以示旌異。巡撫盧坤尤重之,採其議增治於盩厔、洋縣界,增營汛於商州及略陽;檄勘全秦水利,於灃、涇、滻、渭諸川,鄭白、龍首諸渠,規畫俱備。社倉、義學,亦以次推行。五年,擢貴州按察使,未到官。六年,入覲,仍調陜西,抵任數日而卒,贈布政使。陜民請比硃邑桐鄉故事,留葬南山,勿得,乃請祀名宦。湖南亦祀鄉賢。

如熤自為縣令至臬司,皆出特擢。在漢中十餘年不調,得成其鎮撫南山之功。宣宗每論疆吏才,必首及之。將大用,已不及待。為人性豪邁,去邊幅,泊榮利,視之如田夫野老。於輿地險要,如聚米畫沙。所規畫常在數十年外,措施略見所著書。嘗佐那彥成籌海寇,有洋防備覽;佐姜晟籌苗疆,有苗防備覽;佐傅鼐籌屯田,有屯防書。又有三省邊防備覽,漢江南北、三省山內各圖,漢中府志及樂園詩文集。

子正基,原名芝,字山舫。副貢生。少隨父練習吏事。道光中,官河南知縣,有聲。擢鄭州知州。治賈魯河,息水患。河決開封,正基佐守護。治河兵獄,雪其冤,得河兵死力,城賴以完。母憂歸,服闋,補奉天復州。興屯練,捕盜有法,民殺盜者勿論。奉天治吏素弛,府尹下所屬,以正基為法,盜風為戢。引疾去。江南大吏疏調,擢授常州知府。二十九年,大水,勘災勤至,郡人感之,輸錢二十餘萬助賑,全活甚眾。累署淮揚道、按察使。咸豐初,侍郎曾國籓、呂賢基交章薦之,命赴廣西治軍需,授右江道。擢河南布政使,留廣西。時粵匪披猖,將帥齟,師久無功。正基曲為調和,疏論其事,謂:「師克在和,事期共濟。統兵大帥與地方大吏,宜定紛更不齊之勢,聯疏闊難合之情。布德信以服人心,明功罪以揚士氣。勿因賊盛而生推諉,勿因兵單而務自救,勿以小忿而不為應援,勿以偶挫而坐觀成敗。庶逆氛可殄,大功可成。」時以為讜言。二年,桂林圍解,賜花翎。尋隨大軍赴湖北,時武昌初復,命馳往撫恤難民,署湖北布政使。調廣東,復赴廣西清覈軍需。內召授通政副使,遷通政使。七年,引疾歸,卒。

論曰:亂之所由起與亂之所由平,亦在民之能治否耳。教匪起於官逼民叛,其間獨一得民心之劉清,卒賴以招撫,助誅剿之成功。征苗頻煩大兵,而未杜亂源,傅鼐乃以一一道之力,剿撫兼施,巖疆綏定。南山善後,嚴如熤始終其事,化榛莽為桑麻。此其功皆在一時節鉞之上,光於史策矣。


\end{pinyinscope}