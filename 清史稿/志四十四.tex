\article{志四十四}

\begin{pinyinscope}
地理十六

△四川

四川:禹貢梁州之域。明置四川等處承宣布政使司。清初因之。順治二年,置四川省,設巡撫,治成都。十四年,增設四川總督。康熙四年,改烏撒隸貴州。七年,改設川湖總督,駐湖北荊州。九年,移駐重慶。十九年,又改為川陜甘總督,駐陜西西安。雍正六年,改東川、烏蒙、鎮雄隸雲南,遵義隸貴州,省馬湖入敘州,改建寧衛為寧遠府,升錦、茂、達三州及資縣並為直隸州。七年,升雅州為府。十二年,升嘉定、潼川二州為府,升忠州為直隸州,置黔彭直隸。乾隆元年裁,改酉陽土司為酉陽直隸州,升敘永為直隸。十四年,復專設四川總督,裁巡撫,以總督兼理巡撫事,治成都。二十五年,改松潘衛為松潘直隸,改雜古腦為理番直隸。二十六年,改石砫土司為石砫直隸。嘉慶七年,升達州為綏定府。光緒三十年,升打箭爐為直隸。三十二年,設督辦川滇邊務大臣,駐巴塘。三十四年,改敘永為永寧直隸州,升打箭爐為康定府,升巴安縣為巴安府。宣統元年,改德爾格忒土司為登科府。東至湖北巴東縣;一千七百六十里。西至甘肅西寧番界;一千二百四十里。南至雲南元謀縣;二千三十里。北至陜西寧羌州。一千一百八十里。廣三千里,袤三千二百里。由康定府至前藏拉薩,駐藏辦事大臣駐。四千七百一十里。北極高二十七度五十四分至三十二度二十二分。京師偏西六度五十三分至十四度十二分。宣統三年,編戶五百四萬一千七百八十,口五千二百八十四萬四百四十六。都領府十五,直隸州九,直隸三,州十一,十一,縣百十八,土司二十九。其名山:東北有嶓塚。蜿蜒川、陜界者,巴山。西北自岷分二支:南迤於大金川東西者,青城、蒙、瓘眉,在西者,噶察克拉嶺、折多山;其在岷東南迤者,摩天嶺、劍門山。碩古里,自青海東巴顏喀喇分支。其大川:金沙、鴉龍、岷、嘉陵、渠、涪江,大渡河。航路:東境自夔至敘。驛路:自成都東北逾劍閣達陜西沔縣,西渡瀘定橋逾大雪山達西藏江卡。鐵路:川漢,未竣工。電線:自成都東達漢口,西達打箭爐。

成都府:沖,繁,難。明,府。成綿龍茂道治所。光緒三十四年裁總督。布政使、提學使、提法使、鹽運使,巡警道、勸業道,將軍、副都統、提督駐。舊領州六,縣二十五。順治十六年,省羅江入德陽,省彰明入綿。康熙元年,省崇寧入郫,省彭入新繁。九年,省華陽入成都。雍正六年,復設華陽,升綿、茂二州及資縣並為直隸州,以德陽、綿竹、安隸綿,汶川、保隸茂,資陽、仁壽、井研隸資,又省威入保。六年,復設崇寧、雙流、彭、彰明四縣屬府。七年,以彰明改屬龍安。東北距京師五千七百十里。廣二百四十里,袤二百七十里。北極高三十度四十二分。京師偏西十二度十六分。領州三,縣十三。成都沖,繁,難。倚。武擔山在城西北隅。西:龍華山。北:天回山。郫江自郫縣入,繞城東而南,入華陽,與錦江合,名二江,亦曰都江。沱江自新繁入,逕縣北,又東流入新都。金水河自城西穿城東出入江。摩柯池在城內。有天迥、沱江二鎮。一驛:錦官。華陽沖,繁,難。倚。康熙九年並入成都。雍正五年復置。東:大面山。西:西山,亦名雪嶺。南:六對、鐵爐。錦江一名汶江,自郫縣入,逕城南,折而東,會成都之郫江。又折而西,新開河自雙流來會,下流入彭山。浣花溪在城東南,一名百花潭。驛同成都。雙流沖。府西南四十里。康熙元年並入新津。雍正六年復置。南:應天、宜城。東南:普賢山。新開河自溫江入,逕城南,東流入華陽。石魚河、楊柳河亦自溫江入,逕城西南,合流入新津注大江。溫江繁。府西少南五十里。北:女郎、大墓二山。岷江俗名溫江,即金馬河,自灌縣入,西南入新津。石魚河在城西,自金馬河分流,楊柳河自石魚河分流。又酸棗河自郫縣入,東流逕城北,俱入雙流。新繁繁。府西北五十六里。西北:五龍、平陽。北:曲尺山。沱江即北江,自郫縣入,逕城南,入成都。北:清白江,即古湔水,自彭縣入,東入新都。錦水河亦自彭縣入,東流逕城南,都橋河自彭縣西南分清白江,東流逕城北,俱入新都。金堂繁,難。府東北七十里。西:金堂山,縣以此名。南:雲頂山,亦名百城山。東:三學山。綿陽河即綿水,自漢州合雒水入,右納馬木河。又南至焦山陂,西有清白江及其枝津督橋河自新都入,合於城東。其昆橋河即沱江,先後來會,是為中江,又南入簡州。有古城、下市、柏茂三鎮。新都沖,難。府北五十里。南:龍門、赤岸。北:麗元山。沱江即毗橋河,下流自成都入。督橋河、錦水河俱自新繁入。錦水又歧為利水河,並入金堂。其正流至城東南入湔水,在縣北,亦自新繁入,合彌牟水,東入金堂。有彌牟、軍屯二鎮。一驛:廣漢。郫沖。府西四十五里。西:平樂山。北:郫江自崇寧入,東流入新繁。郫江俗名油子河,自走馬河分流,逕城西,又東入成都。沱江自崇寧入,東流入新繁。西:九曲江分走馬河小支,繞城西北,下流入油子河。雙清河即走馬河,亦自崇寧入,東流入華陽,為錦江。有馬街一鎮。灌沖,繁。府西北百二十五里。西北:灌口、玉壘。南:趙公山。西南:青城山。縣西南一里離堆,秦李冰鑿江處。大江逕此分二大支,曰南江,曰北江。南江分三派:正派南流入崇慶為西河;東派為白馬河,又分為里石溪河,亦入崇寧;西派西南流,又分二支,俱入新津。北江分南北二大派。南派又分為三:曰龍安江,入崇慶;曰金馬河,入溫江;曰酸棗河,入郫縣。北大派之南派曰沱江,北派曰湔水,俱入崇寧。西南僚澤、西北玉壘、蠶崖三關。彭繁,疲,難。府北九十里。康熙元年,並入新繁。雍正六年復置。西北:彭門山,與牛心山隔江對峙。又有大隋、中隋二山。南:清白江自崇寧入,歧為督橋河,東入新繁。西北:王村河,源出五峰山,南流折東入漢川,為馬水河。錦水河亦自崇寧入,逕城南,東流入新繁。瀰濛水源出瑯邪山,即彌牟水上游,東流至新津入湔江。北:靜塞關。崇寧簡。府西北八十里。康熙元年省入郫。雍正七年復置。西:鐵砧山。北:金馬山。沱江自灌縣入,逕城南:東入郫縣。湔水自灌縣分沱江,東流四十里,逕城北,又東入彭縣,為清白江。郫江自灌縣入,逕城南,歧為走馬河。又一支為油子河,俱南,東入郫縣。徐偃河出郫江,亦自灌縣來會,東入彭縣。簡州沖,難。府東少南百二十里。東:李八百山。西:孝子山。東北:石鼓。西南:忠國。西北:丹景山。中江即沱江,或稱雁江,自金堂入,合絳水,南流入資陽。絳溪河發源西北月亮溝,東南流,逕城北,入江。西南:赤水,一名黃龍溪,西流入仁壽,即蘭溪上源也。有陽安關。巡司駐龍泉鎮。一驛:龍泉。崇慶州繁。府西南九十里。西:鶴鳴山。西北:龍華山。北:味江自灌縣入,逕州西,西南流,折東與白馬江合。白馬江由味江分流,逕城東,又東南會西河,入新津為白西河。黑石溪河自白馬江分流,至城東三江口,仍入白馬江。羊馬江在白馬江東,自灌縣分大江,東南流,逕州境,又南入新津。一驛:陽安。新津沖,難。府西南九十里。南:天柱山。北:平蓋山。東南:寶資山。岷江自溫江入,逕城東,又東南流,入彭山。北:白西河即味江,自崇慶入,東南流,合羊馬河,入江。汶井江即古僕乾水,今名南河,自邛州入,東北流,逕城南,又東入江。乾溪、溪水二河自灌縣分味江,西南流,折東南,逕城南,又東注汶井江。漢州沖,難。府北少東九十里。東:銅官、東覺二山。雁江自什邡入,至州東北合沈犀河,有白魚河亦自什邡來注之。又東南合雁江,入石亭江。石亭江即雒江,亦自什邡入,逕州北,東南入金堂。綿水自德陽入,逕州東,南流入雒江。一驛:廣漢。什邡繁。府北百三十里。南:雍齒山。西北:章山,即雒通山,雒水出焉,逕城北,東南流,入漢州。金雁河、沈犀河、白魚河三水並出縣境,亦入漢州。西:高鏡關。

重慶府:沖,繁,難。川東道治所。明,府順治初,因明制,領州三,縣十七。康熙元年,省銅梁、安居入合州,省璧山入永川,省武隆入涪州。八年,省定遠入合州。六十年,復置銅梁,以安居並入。雍正六年,復置大足、璧山、定遠三縣。十三年,升忠州為直隸州,酆都、墊江屬之。析黔江、彭水二縣置黔彭直隸。乾隆元年,改隸酉陽直隸州。二十九年,以巴縣江北鎮置江北。西北距省治九百六十里。廣五百六十里,袤五百九十里。北極高二十九度四十二分。京師偏西九度四十八分。領一,州二,縣十一。巴沖,繁,難。倚。城內巴山,縣以此名。東:塗山。又北:太華。西:逾越、縉雲。南:霖峰。縣東有明月峽者,大江逕此。大江自江津入,逕城東南,又東北入江北。嘉陵江即涪江,自合州入,南流至城東,與大江合。東:丹溪自綦江入,交龍溪自長壽入,俱入大江。巡司一,駐木洞鎮。西:佛圖關。驛二:朝天、白市。江津沖,繁,難。府南百二十里。南:鼎山。東:雲篆、珞黃。東:華蓋、女仙。東南:固城山。大江自合江入,東北流,逕縣西、北、東三面,亦名九字水,又東北入巴縣。南江即古僰溪,自綦江入,逕城東,又北入大江。筍溪源出南綦盤山,北流注南江。砦溪、樂城溪俱入大江。南:崖門關。一驛:茅壩。長壽沖。府東北百五十里。東:長壽山,縣以此名。北:銅鼓。西:牛心。東北:羅紋山。大江自江北入,逕城東入涪州。龍溪一名溶溪,即古容溪,自墊江入,南流入大江。海棠溪合桃花溪自鄰水入,逕城東北,一名梅溪,西南流入巴縣。一驛:龍溪。永川沖。府西北百八十里。西:英山。北:銅鼓。南:瀘龍。西北:溪山。侯溪上流曰車對河,西南流,至城南,會西來一水,南流為株溶溪,又南入大江。松子溉源出龍洞山,亦東入大江。一驛:東皋。榮昌沖。府西少南二百六十里。東:葛仙。南:寶蓋。北:駐蹕。東南:慶雲山。長橋河自大足入,逕城西思濟橋,為思濟河。西南流,至清江灘入瀘州。大鹿溪源出南山,南流,折東南入合江。一驛:峰高。綦江簡。府南三百里。西:扶歡。東:石筍。北:牛崗。南:祝融、蘿綠二山。僰溪亦名夜郎溪,自貴州桐梓來入,名綦江,逕城東,又西北流入江津。至南江口注大江。東骨溪、北金沙溪、西奉恩溪,並入綦江。南三舍溪、捍水二關。南川難。府東南二百五十里。東:九盤山、馬嘴山。西:永隆。南:方竹箐山,白水出,逕城南鎮江橋,名鎮江橋溪,屈流至城北水東橋,為大溪河,入涪州。四十八渡水源出馬嘴山,與流金水俱至水東橋合白水。水從溪源出水從山,西流入綦江,合南江,即南江別源也。南馬頭、北冷水二關。合州沖,繁,難。府北二百里。北:瑞應。西:牟山。南:銅梁。東:釣魚山。東北:書臺山。渠江即宕渠水,自廣安入,涪江自遂寧入,俱合嘉陵江。嘉陵江自定遠入,東北合渠江曰嘉渠口,又東南合涪江曰三江口,又南入江。北:跳石溪自銅梁入,東北流入涪江。二驛:劉家場、溫場。涪州沖,繁,難。府東少北三百五十里。東:龜山。西:五花、玉璧。北:鐵櫃、北巖。東南:武龍山。大江自長壽入,逕城北會涪陵江。涪陵江即古延江,自彭水入,北合大江。大溪河自南川入,東北流,逕州東南入涪陵江。巡司一,駐武隆鎮。一驛:涪陵。銅梁繁。府西北二百四十里。康熙元年並入合州。六十年復置。西:六瀛。東:新開山。南:雙山。西北有小銅梁山,縣以此名。涪江自遂寧入,逕城東北,又東南流入合州。安居溪一名關箭溪,又名瓊江,自遂寧入,逕城南,折東北流入涪江。馬灘河一名赤水溪,源出六瀛山,南流入大足,合沙溪河,入縣城。合巴川河,東南流,繞縣境如「巴」字,亦入城。與赤水溪合流,出城東流,合小安溪,東北入合州。有安居鎮巡司。大足繁。府西三百十里。康熙元年省入榮昌。雍正六年復置。南:雞棲。東:三華。西:龍。東南:玉城山。長橋河上流即岳陽溪,自安岳入,逕城西,又西南入榮昌。小安溪一名單石溪,東北流入永川。赤水溪自銅梁入,東北流,合沙河溪,仍入銅梁。東米糧、北化龍二關。壁山沖。府西少北百里。康熙元年省入永川。雍正六年復置。南:龍璫。北:縉雲。西南:垂壁。東南:王來山、來鳳。油溪二源,出湯口峽,一為來鳳橋溪,南流,一為馬坊橋溪,東南流,俱至斗牛石,合流入江津,注大江。有雙溪鎮。一驛:來鳳。定遠沖。府北少西二百九十五里。康熙八年並入合州。雍正六年復置。東:武勝山。北:焦石山。嘉陵江自南充入,環縣境北、東、南三面,南流入合州。花石溪源出岳池,西南流,鹽灘溪源出蓬溪,東南流,俱入嘉陵江。江北簡。府北一里。明為巴縣之江北鎮。乾隆十九年設。東:臥龍山。北:大華■D7山。東北:石城山。大江自巴入,逕東南,又東入長壽。涪江自合州入,逕南,又東南,與巴縣分水入大江。東:銅鑼峽關,為水路門戶。

保寧府:中,沖,繁。川北道治所。川北鎮總兵駐。明,府。順治初,因明制,領州二,縣八。雍正五年,改梓潼屬綿州直隸州。西南距省治六百二十里。廣七百一十里,袤六百里。北極高三十一度五十九分。京師偏西十度五十分。領州二,縣七。閬中沖,繁。倚。西:閬中山,縣以此得名。東:盤龍、文城。南:鐘山、玉立山。東北:大方山、靈山。嘉陵江即西漢水,亦曰閬水,自蒼溪入,南流逕城西,折東,又逕城南入南部。東河一名宋江,亦自蒼溪入,東南流,逕城東,與嘉陵江合。西水河自南部入,至梁家坡仍入之。西:鋸山關。一驛:錦屏。蒼溪府西北四十里。東:離堆、白鶴山。西:老池。南:小錦屏。東南:大獲山。西北:方山。嘉陵江自劍州入,逕縣東北,又南入閬中。東河自廣元入,逕大獲山麓,西南流,亦入閬中,塘溪河從之。曲肘川源出玉女山,東南流入江。南部繁。府東南七十里。東:龍奔山。西:蘭登山。南:南山,亦名跨鼇山。東南:離堆山。嘉陵江自閬中入,逕城東北,又東南流,入蓬州。西水河即小潼水,自劍州入,逕城南,又東南亦入蓬州。南溲水、西伏元溪、東安溪,皆嘉陵江之溢流也。縣丞、巡司駐富村驛。廣元沖,繁,難。府北三百里。潭毒山在北,下瞰大江。又七盤嶺為秦、蜀分界處。東:鳳凰山。西:烏奴、白馬。北:金城。東北:可沇山。嘉陵江自陜西寧羌州入,逕城西,又西南入昭化。宋江即東河,亦自寧羌入,逕城東,又南入蒼溪。北:潛水源出龍門山,逕龍洞口,至朝天驛入嘉陵江,漢壽水、滌溪從之。巡司二,駐神宣驛、百丈關。驛三:問津、神宣、望雲。昭化沖,繁。府北少西二百八十里。西:牛頭、人頭。南:仙人。北:大高、長寧。西北:木馬山。嘉陵江自廣元入,逕城東北,又南入劍州。白水江即羌水,自平武入,東南流入嘉陵江。清水江自劍州入,逕城西北,又東與白水江合。桔柏津在城東,即嘉陵、白水二江合流處也。西北:白水關。二驛:昭化、大木樹。巴州繁,疲,難。府東北三百五十里。東:東龕山。西:西龕山。在東又南:南龕、北龕。東南:石城。西北:義陽岳、木彊二山。巴江源出大巴山,自南江入,逕州東南入達縣。清水源出廣元東南境,逕恩陽廢縣西北,又東南流,逕州西南,宕水自通江入,注巴江。州判一,駐龍泉關。通江府東北五百五十里。東:大鐘。西:金童。南:秋錦。東北:龍山。宕水一名東河,源出陜西西鄉,西南流,逕城東會諾水。諾水源出陜西南鄭,亦名西河,逕城西與宕水合,入巴州。白石水一名清水,自西鄉入,西南入宕水,名洪口河。東白陽、北羊圈、東北濛壩三關。南江府東北四百七十里。東:望元山。西:龍耳山。南:公山。北:孤雲山。又大巴、小巴二山。巴江即宕渠水,源出大巴山,逕城東,又東南入巴州。東:難江,一名南屯河,上流曰三溪河,至兩河口入巴江。南平桑水,北明水、韓溪、蒼溪,俱從之。劍州沖,繁。府西北二百二十里。東:鶴鳴、浮滄。大劍山,亦曰梁山,相屬有小劍山,中為劍閣道。嘉陵江自昭化入,逕城東,又南入蒼溪。清水江即黃沙江,自平武流入,逕城北,又東入昭化。西小河即小潼水之下流也,又名武連河,源出五子山,東南流,入南都。北:劍門關。驛二:武連、劍門,驛丞駐。

順慶府:沖,繁,難。隸川北道。明,府。順治初,因明制,領州二,縣八。嘉慶十九年,改大竹、渠屬綏定府。西南距省治六百二十里。廣二百九十里,袤二百三十里。北極高三十度五十分。京師偏西十度十九分。領州二,縣六。南充沖,繁,難。倚。東:鶴鳴山。南:清居山。西:大小方山。嘉陵江自蓬州入,逕縣東,又南入定遠。西:西溪,源出西充,流溪,源出大耽山,俱東流至縣南,入嘉陵江,曲水、清溪水從之。鹽井在縣境者十有五。西充繁。府西北九十里。城西北隅西充山,縣以此名。東:亞夫、扶龍。西:瓊珠。南:南岷山。陵溪亦名小陵河,自縣西小陵鎮至三河口,與象溪、虹溪合流入南充。海棠川源出雙圖山,西流,折而南繞城,又南入南充,注嘉陵江。蓬州繁。府東北四十里。城北隅玉環山,嘉陵江水環之,故名。西:三合。南:永安。東:雲山。嘉陵江自南部入,南流,繞城三面如玦,折而南,入南充。清溪水源出營山之披衣山,南流入州,名清澹河,又四十里至州南清溪口入嘉陵江。鹽井一。營山繁。府東北百八十里。城西南營山,縣以此名。東:青羊。西:披衣。東北:大小蓬山。流江自儀隴入,七曲縈回,亦名七曲堰,逕城東,又東南入渠縣。瞰天溪源出西西巖,繞城東南流,至七曲堰入流江。清溪源出披衣山,西南流入蓬州。儀隴簡。府東北二百六十里。城內金城山。東:望龍山。南:南圖山。西:儀隴山,縣以此名,流江之水出焉。流江自儀隴山南流,折東逕城南,又東南入營山。平溪源出東允家山,南流入流江。廣安州繁。府東南百九十里。東:穀城。西:秀屏。南:猊峰。北:諫坡山。渠江自渠縣入,逕州北,謂之篆水。以江中有三十六灘,灘石縱橫,波紋如篆,故又名篆江。繞城而南,亦名洄水,又西南入合州。濃水即西溪水,源出北山,南流逕城西,折東至城南五里合渠江。清溪水自鄰水入,左會大池河,流至州南入渠江。鄰水繁,難。府東南二百七十里。南:皛然。東:寶穀山。北:銀華。西:少陵。東北:鄰山。鄰水上源即芭蕉河,自大竹入,西南流,逕城東,又西南與觀音河、寶石河合流入長壽。有鄰山、太平二鎮。岳池沖。府東南百二十里。東:岳安山、龍扶速山。北:龍穴。西:姜山,岳池水出焉。岳池水自姜山流至縣東,折南合靈溪、龍穴二水入定遠。

敘州府:要,沖,繁,難。隸永寧道。明,府。順治初,因明制,領縣十。旋改高州為高縣。雍正六年,改貴州永寧縣來屬,又裁馬湖府,以所轄屏山來隸。八年,復以永寧往屬敘永。乾隆二十六年,置雷波。二十九年,置馬邊。西北距省治七百九十里。廣五百九十里,袤三百七十五里。北極高二十八度三十八分。京師偏西十一度四十三分。領二,縣十一,土司四。宜賓沖,繁,難。倚。西:天倉、硃提。南:七星。西南:大小黎山。大江在縣東北,一名汶江,亦名都江,自犍為入,東南流,入南溪。馬湖江一名瀘水,即金沙江,自屏山入,逕縣南,又東與大江合。石門江,俗呼橫江,又名小江,自慶符入,至城西南,又東北合馬湖江。北:涪溪、蘇溪俱入大江。東:二郎關。慶符簡。府南少東百二十里。南:石門、興慶。東:迎祥山。石門江上流曰紋溪,源出雲南烏蒙,南廣水即古符黑水,自高縣入,俱東北流,逕城西,並入宜賓。富順沖,繁,難。府東北二百四十里。西:凌雲、瑪瑙。東:祿來、桂子。北:朝陽。西南:虎頭山。沱江一名金川,又名釜川,自內江入,逕城東,東南流入瀘州。榮溪自榮縣入,鼇溪源出縣東馬鞍山,俱入沱江。縣丞二,駐鄧井關、自流井。南溪沖。府東百十里。南:琴山、可廬。西:平蓋。北:瑞雲。東:龍騰山。大江自宜賓入,逕城南,又東入江安。西北:福溪亦名服溪,亦自宜賓入,南流入大江。僰溪與九盤溪合流至城東入江。一驛:龍騰。長寧簡。府東南百四十里。東:牛心。南:棫山、越王山。北:寶屏、龍瓘。東西二溪與冷水溪俱至縣東北淯井合流為淯溪,一名三江口。又東北至武寧砦,為武寧溪,又東北至安寧砦,為安寧溪,又東北至江安入大江。高簡。府西南百五十里。南:閣梯。北:連珠。東南:七寶。西南:騰山。宋江自雲南鎮雄入,北流,逕筠連東,分五道,北至平寨,逕城東而北。梅嶺溪自筠連入,至城北合宋江,又北入慶符。筠連簡。府西南二百五十里。南:暮春、黃牛。西:學士。東:景陽山。定川溪有二源,一出烏蒙黑桃灣,一出雲南鎮雄羊落溝,合流逕城西,又北入高縣,為梅嶺溪。珙簡。府南少東二百里。北:麒麟、芙蓉。西:虎牢。西北:梅得山。珙溪一名落浦河,逕縣西南,折而東北入長寧,合淯溪。興文簡。府東南百八十里。東:摩旗。東南:文印山。南:南壽山。水車河一名三渡河,源出故建武城山谷中,至縣東北,又西流,經梅嶺堡入長寧,注水育溪。隆昌沖,難。府東北二百七十里。北:道觀山。南:回龍山、玉蟾山。沱江自內江入,逕城西南入瀘州。小溪一名隆橋河,在縣東,自內江、榮昌二縣山溪水合流而成,東南流,亦入瀘州。屏山簡。府西南二百二十里。西:鏡山。東:書樓。東北:赤崖。西南:小悍山。馬湖江一名瀘水,即金沙江,自雲南昭通入。東北逕蠻夷、平夷二土司界,又東北逕城南,又東入宜賓,與大江合。泥溪、什噶溪、大鹿溪並入馬湖江。巡司駐石角營。馬邊沖,繁。府西六百里。本屏山地,初為馬邊營,乾隆二十九年改。東:煙遮山。南:大池山。北:龍泉山。東南:金鳳山。清水溪一名新鎮河,源出涼山蠻界,逕南,折北轉東,過沐川司入犍為。雷波繁。府西南五百七十里。本屏山地,名雷波鄉。康熙初置長官司。雍正六年改雷波衛。乾隆二十六年升。東:貝海。西:龍頭。北:雷番。西北:寶纛山。金沙江自雲南昭通入,逕南,

東北流,入屏山。南石城河,西南秦沙河,並源出蠻界,東流注金沙江。北馬湖,為黃種、芭蕉二溪上流。西南:神龍關。蠻夷長官司隸屏山。在縣西南,舊屬馬湖府。雍正五年改屬。東:大鹿山。西:什噶溪。沐川長官司隸屏山。在縣西北。東:青孤山。南:沐溪,東流入犍為界,泥溪長官司隸屏山。在縣西,元至元十三年,與馬湖路同置。明改縣,移司於此。仍明舊。平夷長官司隸屏山。在縣西。西北:隆馬崖山。馬湖江自雲南昭通入,又南有大紋溪。

夔州府:要,沖,繁,難。隸川東道。明,府。順治初,沿明制,領州一,縣十二。康熙六年,省大寧入奉節。七年,省新寧入梁山。九年,省大昌入巫山。雍正六年,升達州為直隸州,以東鄉、太平二縣往隸。七年,復置大寧、新寧二縣。旋改新寧隸達州,改梁山隸忠州。乾隆元年,改建始隸湖北施南府。西距省治一千七百四十里。廣四百十里,袤五百四十里。北極高三十一度十一分。京師偏西六度五十三分。領縣六。奉節沖,繁。倚。東:白帝山。赤甲與白鹽隔江,兩山對峙。西:官口。南:勝已、文山。北:天門山。東:瞿塘峽,峽口為灩澦堆,大江即岷江,自雲陽入,逕縣南,東流,出瞿塘峽,自峽以下謂之峽江,亦名鎖江,又東入巫山。東:大瀼水、清瀼水,並入大江。東瞿塘關。巫山沖,繁。府東百三十里。東:巫山,山有十二峰,亦曰巫峽。南:南陵山。北:磊頭。東北:金頭。西北:天縣山。南:大江自奉節入,東流逕巫峽,又東入湖北巴東。巫溪水一名昌江,自大寧入,東南流入大江。又烏飛水在縣西南,發源奉節山谷中,東北流,亦入大江。清溪、萬流溪從之。雲陽沖,繁。府西百四十里。東:石城。北:漢城、馬嶺。南:飛鳳。東南:新軍山。西北:大梁山。大江自萬入,逕城南,東流入奉節。彭溪一名開江,亦名臨江,自開入,東南流,逕城西入大江。湯溪水即東瀼河,東流逕五溪關,又東至城東入大江。鹽井十。鹽課大使駐雲安廠。萬沖,繁,難。府西少南二百八十里。東:黑象山。西:天城、魚存。南:南山。北:都歷、高梁。西南:羊尾山。西北:萬戶山。大江自忠州入,逕城南,又東入雲陽。苧溪即古池溪,自梁山入,至城西,復南流入江。開簡。府西少北二百三十里。北:盛山。西:大池。南:九龍。東南:瑞石。東北:熊耳山。開江亦曰臨江,即古彭溪,自新寧入,逕縣南,又東南會清江、墊江入雲陽。三潮溪、白水溪並東流入清江。大寧難。府北百八十里。東:鳳山。北:石柱、寶源山。東北:石鐘。巫溪一名昌江,源出縣境西北,逕城東,曰大寧河,又南入巫山。馬連溪即白楊河,逕城南,又東入大寧河。有鐵山關。

龍安府:繁。隸成綿龍茂道。明,府。順治初,因明制,領縣三。雍正九年,改綿州之彰明來隸。西南距省治六百五十里。廣七百七十里,袤五百二十里。北極高三十二度二十二分。京師偏西十一度四十九分。領縣四,土司一。平武繁。倚,東:左擔。西:太平。南:鎮南、羊角。北:火風。東南:箐青、石門山。涪江自松潘入,東流逕城南,青漪江一名小江河,即古廉瀼水,亦東南流,並入江油。白水江自甘肅文縣入,逕城西北,又東南流入昭化。石泉河自石泉入,逕縣東南入彰明。火溪河一名白馬河,有二源,流至陽地溢口而合,西南入涪江。又東青川溪,東流入劍州。縣丞駐青山鎮。東北:北雄關。江油簡。府東南二百六十里。東:竇圌山。西:玉枕、大匡。南:龍頭。北:白魚。西南:大小匡山。涪江自平武入,逕城東,與青漪江並東南流入彰明。龍潭溪源出竇圌山,流至石舍崖入涪江。東:涪水關。石泉簡。府西南三百二十里。南:石紐。東:金字山。西:千佛。東北:雞棲山。石泉河即湔水,自平武入,左合大魚口水,其西南源神泉河自茂入,西源壩底水自右來會,折東逕城南至素龍山,為石密溪,折南緣江油界入彰明。西石板、西北上雄二關。彰明簡。府東南三百二十里。東北:太華山。北:紫山、獸目山。涪江自江油入,分二派,夾城東西流,至縣南合,又南會石泉河入綿州。青漪江亦自江油入,南流入涪江。陽地隘口長官司隸平武。在縣北。宋為守御千戶。元至元時,授宣慰副使。明改置長官司。順治六年投誠,因之。

寧遠府:要,沖,繁,難。隸建昌道。建昌鎮總兵駐。明,建昌衛。順治初,因明制為衛。雍正六年改府,以會理州來屬,並置西昌、冕寧、鹽源三縣。越巂一隸之。宣統元年,增置鹽邊。二年,又置昭覺縣。東北距省治一千二百三十里。廣八百四十里,袤一千二百九十里。北極高二十七度五十四分,京師偏西十四度十二分。領二,州一,縣四,土司十一。西昌沖,繁。倚。舊建昌衛。雍正六年改縣。東:木托。西:天王山。南:巴洞。東北:涼山。東南:螺髻。西南:旄牛山。安寧河即孫水,自冕寧入,逕城北。熱水河自東來注之。又逕城西,西河自西來注之。北納東河、寧遠河,南納邛河,南流入會理。東西溪河、三岔河均入金沙江。石門、羅鎖、瀘沽、太平四關。巡司二,駐普威、德昌所。冕寧繁,難。府北少西百八十里。初仍明制為寧番衛。雍正六年改縣。東南:冕山,縣以此名。東:東山。南:南山。北:北山。孫水有三源,自縣北納瓦那河,逕城東南,西源三水合為小村河,又南至王家營,東源曰松溪河,合小相公嶺水,西北流曰瀘沽,來會,又南入西昌。若水即鴉龍江,自雅州入,西南入鹽源。沙沱、烏角、冕山、九盤四關。鹽源繁,難。府西南三百十里。明,鹽井衛。雍正六年改縣。南:柏林山。西:斛僰和。西北:刺紅瓦山。打沖河即鴉龍江下流,自冕寧入,逕城西北,納左所河。又南鹽井河,合雙橋、浪渠二水,與別列河、麥架河西北流來注。又東南納右所河。又南納椒崖、那噶諸河,入會理。雙橋、古得二關。阿所拉場巡司。鹽井二。昭覺繁,疲,難。府東北。舊為交腳汛地,在涼山夷巢中。宣統元年,剿辦涼山惈夷。二年,就汛地增設縣治,改今名,並移建昌中營守備駐之。會理州沖,繁。府南四百里。本會川衛。康熙二十九年分置會理州。雍正六年省會川衛,移州治衛城,隸寧遠。東:密勒山。西:斜山。南:白塔。西南:蘆那山。金沙江左瀆自鹽源入,右與雲南大姚分岸。安寧河自州北納公母河、一碗水,西南與打沖河合,並西流入之。又南納黎溪水,入雲南武定。東玉★河、玉虹河、會通河俱入金沙江。有瀘津、松坪、永昌、大龍、虎頭等關。巡司二,駐迷易所、窪鳥場。鹽邊府西南。鹽源縣屬阿所拉地。嘉慶二十二年增設巡司。宣統元年升。改今名。越巂沖,繁。府北少東二百八十里。初因明制為越巂衛。雍正六年廢衛設。南:大孤山、小相公嶺。西:小孤山、阿露山。又西南:巂山。大渡河自打箭爐入,納松林河、鹿子河,東北流,老鴉漩河自西來,合二小水注之,又東北入清溪。越巂河自西南,二水合流,逕東,惈儸河、臘梅營水東來注之,又東北納寧越營、桂賢村二水,入瓘邊,注大渡河。小相公嶺、青岡、海棠、曬經四關。經歷駐大樹堡。沙麻宣撫司隸西昌。在縣東北。康熙四十九年置。瓜別安撫司隸鹽源。在縣西北。康熙四十九年置。木里安撫司隸鹽源。在縣西北。雍正八年置。威龍州長官司隸西昌。在縣東南。元,威龍州地。明洪武間置司。仍明舊。普濟州長官司隸西昌。在縣西南。元,普濟州地。明洪武七年置土知州。康熙四十九年改置。昌州長官司隸西昌。在縣南。元,昌州地。明洪武九年以雲南大理府土職調守。仍明舊。河東長官司隸西昌。在縣東南。明為宣慰司。康熙四十九年改置。阿都長官司隸西昌。在縣東南。順治六年歸附。康熙四十九年授宣撫司。雍正六年改置。阿都副長官司隸西昌。雍正六年置。馬喇長官司隸鹽源。在縣西南。與雲南永北接界。康熙四十九年置。邛部長官司隸越巂。在北。康熙四十二年歸附,授宣撫司。五十二年改置。

雅州府:沖,繁,難。建昌道治所。明,雅州。順治初,因明制,為直隸州,領縣三。雍正七年升府,撫民同知駐靖西關地,在哲孟雄之北,為亞東出入要路。有商埠。以其地增置雅安縣。改天全土司為天全州,改長河西魚通安遠宣慰司為打箭爐。八年,改黎大所為清溪縣。均屬府。光緒三十年,升打箭爐為直隸。三十四年升康定府。東北距省治三百四十里。廣五百十里,袤三百八十里。北極高三十度四分。京師偏西十三度二十一分。領州一,縣五,土司一。雅安沖,繁,難。倚。西:雅安山,縣以此名。東:周公。南:嚴道山。北:七盤山。青衣江一名平羌江,俗稱雅河,即大渡水。自蘆山入,至縣北門外,東南流入洪雅。小溪河自名山入,邛水自榮經入,並入青衣江。北飛仙、金雞、南飛龍三關。名山沖,難。府東北四十里。城內月心山。西北:名山,縣以此名。西:蒙山。東:白馬。南:總岡。東北:百丈山。名山水在縣東二百步,東南流入雅安,為小溪河。百丈河源出蓮花山,東南流入蒲江,為鐵溪河。東:黑竹關。一驛:百丈。榮經沖,繁。府南九十里。北:銅山。東:孟山。西:中峻。南:邛崍、瓦屋、大關。榮、經二水為邛水上源。榮水出邛崍山,五派並發,流逕城西而合,又北流,繞城北,與經水合,曰榮經水。又北名邛水,入雅安。下改溪源出下改山,北流至城南,入經水。祭風溪在西,源出龍游山,入榮經水。西北紫眼、西邛崍、東北天險三關。一驛:箐口。蘆山簡。府西北百里。東:始陽山,即禹貢蒙山,相接為盧山。西北:通靈山,為外番要道。南:青衣水有二源:西源即天全州流入之沫水,東源出邛州伏牛山,即古青衣水,二水夾城東西流,會於城南,又西南流,折東入雅安。和川水自天全入,逕城南入青衣江,曰三江口。西北:靈關。東北:八步關。東南:飛仙關。天全州繁,難。府西少北百二十里。東:多功、臥龍。南:燕子。西:馬鞍。東北:金鳳山。沫水一名浮圖水,自羌界入,逕州北,東南流,入蘆山。南:和川水,一名始陽河,二源合而南流,折東亦入蘆山。碉門,吏目駐。西:禁門、仙人、紫石三關。清溪沖,繁。府西南百六十里。東:沖天。西:牛心。南:盤陀。東北:聖鐘山。又縣北五十里有大相公嶺,即榮經之邛崍山。大渡河一名瀘水,在縣南,自打箭爐入,與越巂分水,穿涼山夷界,入瓘邊為中鎮河。南:兩澗水,東源出邛崍山玉淵泉,逕城東,西源出邛崍山二源溪,流逕城東,西與漢水合,入大渡河。巡司一,駐黃木廠。南:黑崖、清溪二關。驛二:泥頭、沈村。董卜韓胡宣慰司隸天全。在州西北。仍明舊。有靈關河,逕司西北,與多功水合。又冷邊長官司,亦隸天全。沈邊長官司,隸清溪,均於宣統三年改流。

嘉定府;沖,繁。隸建昌道。明,嘉定州。順治初,因明制,為直隸州。領縣六。康熙十二年升府,以其地置樂山。嘉慶十三年,設瓘邊。北距省治三百九十里。廣六百餘里,袤二百九十里。北極高二十九度二十六分。京師偏西十二度三十一分。領一,縣七。樂山沖,繁,倚。城西隅高標山。東:凌雲、烏尤。北:白崖山。通江即岷江,自青神入,逕城東南,會陽江,入犍為。陽江即大渡河,自瓘眉入,逕城西南,與青衣江合。青衣江一名平羌江,自夾江入,逕城西,納泥溪、竹公溪二水,入岷江。西蘇溪,西南臨江溪,均自瓘眉入,蘇溪入青衣江,臨江溪入大渡河。東:安慶關。北:平羌、嘉禾二關。瓘眉繁。府西七十里。大峨、中峨、小峨三山俱在南。西南:綏山。西北:鏵山。大渡河亦名中鎮河,自峨邊入,徑城南,東北流,與羅目江合,入樂山為臨江溪。北:粗石河,發源大峨山麓,合符文水,東南流,逕城北,亦入樂山為蘇溪。西南:土地、大圍二關。洪雅繁。府西北百三十里。南:隱蒙、八面。東:烏尤、葛仙山。西:竹箐山。東北:金雞山。西南:遜周山。青衣江自雅安入,逕縣南,又東南入夾江,一名洪雅江。擁泔水出可慕山谷,逕縣入丹棱。龍門溪二源合流,東北入青衣江。花溪源出榮經,東北流,至城西入青衣江。西:竹箐關。夾江繁。府西北八十里。西:雲吟、平羌。東:虎履。南:鳳凰。北:大觀山一名觀斗山。青衣江自洪雅入,逕城西南,南流入樂山。西:飛水溪一名瀑布泉,與青衣江合。西南龍鼻溪,繞龍鼻山入江。西:鐵石關。犍為沖,難。府東南百二十里。南:子雲山。東:天馬。東:張綱山。北:舞鳳山。西南:沈犀山。岷江自樂山入,逕縣東,又東南入宜賓。沐溪、清水溪俱在南,並發源屏山,東流入江。東北四望溪,自榮入,逕三江鎮下與岷江合。鹽捕通判一,駐黃角井。大使一,駐牛華溪。榮繁,難。府東百五十里。北:鐵山、榮黎。東:梧桐。西:鳳西、白石、龍虎。南:龜泉山、五保山。榮溪自仁壽入,有二源,東西夾城流,至城南而合,東南流入富順。大牢溪源出鐵山,南流逕城西,至宜賓入岷江。縣丞一,駐貢井。威遠繁。府東二百六十里,西北:雲臺。西:龍泉、老君山。西北:龍泉。西:紫金山。西北:獻寶溪,一名硫黃溪,三源合流,至縣東,有龍會河自西北南流注之,即秦川溪也,南入富順。瓘邊要府西二百六十里。本瓘眉縣地。乾隆五十五年,設主簿分駐。嘉慶十三年裁主簿,置,設通判。九隘皆為地。南:龍山。東:藥子山,左界馬邊,右接夷境。西:橫木。北:馬湖山。中鎮水即大渡河,自清溪入,逕北,又東入瓘眉。屬有嶺夷十二姓地。

潼川府:中,繁,難。隸川北道。明,潼川州。順治初,因明制,為直隸州。領縣七。雍正十二年升府,以其地置三臺縣。西南距省治三百二十里。廣三百八十里,袤五百七十里。北極高三十一度六分。京師偏西十一度十六分。領縣八。三臺繁,難。倚。東:東山,在縣東四里。又黃龍、鼓樓。西:三臺山,縣以此名,南:印臺、金魚。西南:牛頭。東北:萬峰。中江即古五城水,自中江入,逕城西南入涪江。涪江自綿州入,逕縣東北入射洪。又東桃花溪,亦入射洪。縣產鹽,上井三,中井九,下井二百十六。縣丞駐葫蘆溪。射洪繁,難。府東南六十里。南:白巖。東:東武。北:金華。東南:通泉山。東北:公成山。涪江自三臺入,逕城東,又南流入蓬溪。梓潼水一名射江,亦曰瀰江,又曰白馬河,自鹽亭入,南流,逕東南獨坐山下入涪江。東:黃滸溪亦自鹽亭入,與梓潼水合。桃花水自三臺入,南流入涪江。通判一,駐太和鎮。鹽課大使駐青堤渡。鹽亭簡。府東少北百二十里。西:負戴山。東:光祿。南:寶蓮。北:金紫。鹽亭水亦名小沙河,發源縣東北境,下流入梓潼水。梓潼水自綿州梓潼入,逕城南,合鵝溪入射洪。有鹽井二十。中江難。府西百二十里。城內鬥山。東五城與西棲妙隔江對峙。西南:銅官山。中江水名凱江,自羅江入,逕城西南,又東北流入三臺。雙橋河源出縣西北白蓮洞,東南流,逕城西,轉南至銅魚山下入中江。巡司一,駐胖子店。遂寧繁,難。府東南二百十五里。東:銅盤、龍頭。西:箕山。北:廣山。西南:書臺,與寶嘉、金魚二山相連為三峰。涪江自蓬溪入,逕城東,又東南入合州。東北:郪江有三源,並東北流至蓬萊鎮,合入涪江。安居水一名關箭溪,自安岳入,逕城西南入銅梁。鹽井五十二。縣丞兼批驗大使駐梓潼鎮。蓬溪繁,難。府東南百九十里。東:蓬萊、赤城。西:龍門。南:銅缽。北:石龍。西北:龍馬山。涪江自射洪入,逕城西南流入遂寧。西北:郪江,東流至黃龍鋪入涪江。又北蓬溪,源出西充,西南流,逕城北,入遂寧。鹽井七百九十五。縣丞駐蓬萊鎮。鹽課大使駐康家渡。安岳繁,難。府南三百八十里。治後鐵峰山。東:紫薇、白雲。西:大雲。南:安泉。東南:雲居山。安居水自樂至入,逕城北,又東南入遂寧。魚海河有二源,一東流至城東,合入安居水。南:岳陽溪,東南流入大足。樂至簡。府南少西三百九十里。南:棋盤山。東:玉欄坡山、金雞山。西:周鼎。東南:乾瓘山。安居水源出縣東北,東流,玉帶溪源出縣西清水潭,東南流,並入安岳。又樂至池在縣東二里,縣以此名。

綏定府:繁,疲,難。隸川東道。明,達州。順治初,因明制,為夔州府屬之達州。雍正六年,升直隸州,以夔州之東鄉、太平、新寧三縣來屬。嘉慶七年升府,改名綏定,並於州地置達縣,升太平為直隸。十九年,以順慶府屬之大竹、渠二縣來隸。道光九年,移太平同知駐城口,改名城口,太平還為縣,均仍隸府。西距省治一千二百里。廣四百三十里,袤六百餘里。北極高三十一度十八分。京師偏西八度五十一分。領一,縣六。達繁,疲,難。倚。東:龍城山、大竹。南:火峰、南巖。西:石城、金華。東南:金匱、石門。東北:竹山。通川江即渠江,自東鄉入,逕城南,又西南入渠縣為宕渠江。南江自新寧入,東會瀘灘河,北流折西至城東入通川江。北水即巴江,自巴州入,並合通川江。西:鳳皇、鐵山、龍船三關。巡司駐麻柳場。東鄉簡。府東少北九十里。東:平樓、文字。西:印石。南:金榜。北:蟠龍。東南:瓘城山。西南:石人山。前、中、後三江為通川江上流,俱自太平入。至城東合流入達縣。長樂河上流為白龍、赤甲二泉,源出東長樂鎮,合西流,至城南入通川江。文字溪發源文字山,合前江。有高橋、馬渡二關。新寧繁,難。府東少南百一十里。西:屏山。東:雞山。南:冠子山。北:天馬。西南:鼓嘯山。東北:莪城山。南江自縣東北三角山發源,逕城南,折西北流,合聯珠峽水入達縣。瀘灘水源出大竹山,自達縣東南界北流,與南江合。開江在縣東北,東流入開縣。東:豆山關。渠簡。府西二百二十里。北:龍驤。西:玉蟾山。東北:八濛、大斌。渠江即宕渠水,自達縣入,逕城東,又西南入廣安。流江自營山入,東南流,與渠江合。白水溪源出東南白水洞,西流入渠江。北:衛渠關。縣丞駐三匯場。大竹繁。府西南百二十里。東:月城山。西:九盤、鄰山。東北:獅子山、金盤山亦名仙門山。仙門水自月城山發源,鄰水自鄰山發源,井西南流入鄰水。北:東流溪一名清溪河,西流入渠縣,注渠江。縣丞駐石橋鋪。太平要。府東北百四十里。南:翠屏。東:天池、板塞。北:大橫山。前、中、後三江俱自縣境發源,逕城東西,並入東鄉。白沙河源出板塞山,西南流,逕城南入後江。東:藍津關。城口繁,疲,難。府東北三百六十里。西:城口山,以此名。東南:金城。東北:黃礅山。北江自黃礅山發源,經大竹渡,折北入陜西紫陽為任河,注漢江。萬頃池在峽口山南,鄰境之水多源於此。東北:深溪關。

康定府:要。隸康安道。明,長河西魚通安遠宣慰司。康熙初,明宣慰司以地歸附。雍正七年,移雅州府同知來治,置打箭爐,仍隸雅州府。光緒三十年,升直隸。三十四年升府,改名康定,隸康安道,升里化縣為里化,並以河口、稻成二縣同隸府。宣統三年,舊隸打箭爐之宣慰、宣撫、安撫、長官各土司,全體改流,先後分別設治,並先各就其地置委員、理事等官。東北距省治九百六十里。廣六百四十里,袤八百三十里。北極高三十度九分。京師偏西十四度三十八分。領一,縣二。東:大山。南:無脊山。東南:大雪山。東北:郭達。西南:折多山,為入藏要道。鴉龍江即古若水,自青海境發源,南流,逕府西南入冕寧。大渡河即古涐水,自懋功入,逕府東,又南入清溪。瀘河源出折多山,東北流,至城西南,有木鴉河自番界東流來注,並入大渡河。有榷稅瀘關。巡司一,駐瀘定橋。一驛:烹壩。里化要。府西六百四十里。里塘宣慰、宣撫司地。舊設有糧務委員。光緒三十二年設里化縣。三十四年升。東:紫木喇山。東北:高日山。東:鴉龍江自喇滾入,有三渡水自鹽源之木里土司及雲南中甸來注之,會金沙江入馬湖。西南:色隆達河,源出額東額山,入金沙江。河口要。府西里塘、明正兩土司交界地,舊名中渡。光緒三十二年,里塘改流設縣。西有鴉龍江。稻成要。里塘土司地。舊名稻壩。光緒三十二年改流。三十四年設縣。縣丞一,駐貢噶嶺。

巴安府:要。康安道治所。督辦川滇邊務大臣、按察使銜爐安兵備兼分巡道駐。巴塘宣撫司地。光緒三十一年改流。三十三年置巴安縣。三十四年升府,並置三壩,鹽井、定鄉二縣隸之。東北距省治二千一百里。領一,縣二。東:龍新山、甲噶喇山。西南:寧靜山。巴沖楮河自瞻對入,與金沙江合。色楮河即金沙江,自三巖入,逕府西至得榮入雲南麗江。三壩要。府東二百三十里。巴塘、里塘兩土司交界地。三十三年改流。三十四年設,駐通判。鹽井要。巴塘土司地。光緒三十一年改流。三十四年設縣。瀾滄江自察木多入,繞由雲南入緬甸。定鄉要。里塘土司地。舊名鄉城。光緒三十二年改流。三十四年設縣。

登科府:要。德爾格忒宣慰司地。邊北道治所。宣統元年改流,析其地為五區。於北區設府,仍名登科,並置德化、白玉二州,石渠、同普二縣隸之。東北距省治三千三百五十里。領州二,縣二,土司十二。川、藏交隘,東連甘孜、瞻對,西鄰納奪、察木多,南與巴塘、乍丫接壤,北界西寧、俄落,乃金沙江之上游。德化州要。德爾格忒土司中區地,舊名更慶。宣統元年改流設州。鴉龍江自甘孜入,入瞻對。巴沖楮河自巴塘入,下流入金沙江。石渠要。府西北二百一十里。德爾格忒土司北區地。即雜渠卡,一名色許。宣統元年改流設縣。白玉州要。府南六百三十里。德爾格忒土司南區地。宣統元年改流設縣。北有海子山。同普要。德爾格忒土司西區第。宣統元年改流設縣。並分管察木多呼圖克圖及納奪土司之地。乍丫呼圖克圖地,入藏要路。宣統三年設理事官。察木多呼圖克圖地,亦名昌都。東接德格、納奪、貢覺,西與八宿、諾隆宗毗連。舊設有糧員,置兵戍之。宣統三年增設理事官。得榮巴塘土司地。與雲南接壤。宣統三年設委員。江卡舊為給藏地,置有兵戍。北接三巖、乍丫。西連波密、察木多。宣統二年收回。三年設委員。貢覺舊為給藏地。宣統二年收回。三年設委員。桑昂舊為給藏地。宣統二年收回。三年設委員。雜瑜舊為給藏地。宣統二年收回。三年設委員。三巖野番地。跨金沙江之上,有上巖、中巖、下巖之分。宣統二年歸附。三年設委員。甘孜麻書、孔撒兩土司地。宣統元年改流,設委員。兼管白利、東科、德格、倬倭、章穀之地。章穀土司地。與孔撒、麻書、德格、瞻對均接壤。改流後亦名爐霍屯。宣統三年設委員。道塢麻書、孔撒兩土司地。宣統三年改流設委員。瞻對舊為土司地,給與藏人。東連明正、單東、孔撒、麻書、章穀各土司界。南接里塘、毛丫、崇禧。西北與德格接壤。據鴉龍江之上游。有上瞻、中瞻、下瞻之分,亦名三瞻。宣統三年收回設委員。

邛州直隸州:中,沖,繁。隸建昌道。明,州。東北距省治百八十里。廣二百二十里,袤百五十里。北極高三十度十八分。京師偏西十二度五十三分。領縣二。東南:銅官山。南:文筆、古城。西:相臺、馬嵐、七盤。北:渠亭。西南:邛崍山。南:邛水,即古僕千水,亦名文井江,源出西北牛心山,東流入新津。牙江水、斜江水、水耤水俱自大邑入,東南流,與邛水合。西南:火井。南:夾門關。巡司駐火井漕。大邑繁,難。州北少東四十里。東:銀屏山。西:高唐山。北:霧中山。西北:鶴鳴山。牙江水源出縣境,水耤水源出鳳凰山,斜江水源出鶴鳴山,並東南流入州。東:乾溪鎮。蒲江簡。州東南六十里。南:金釜山、長秋山。北:白鶴山。南:蒲江自丹棱入,東北流入州,合邛水。北:鐵溪河自名山入,即百丈河,下流會蒲江入邛水。西南:黑竹關。

綿州直隸州:沖,繁,難。舊隸成綿龍茂道。光緒三十四年裁。明,成都府屬州。順治初,仍明制。雍正五年,升直隸州,以成都之綿竹、德陽、安及保寧之梓潼來隸,並設彰明、羅江二縣,尋改彰明屬龍安府。乾隆三十五年,移州治羅江,省羅江縣。嘉慶六年,還舊治,復設羅江。西南距省治二百七十里。廣三百里,袤百零五里。北極高三十度二十七分。京師偏西十一度三十五分。領縣五。東:金山。南:延賢。東北:天池。北:綿山,州以此名。涪江自彰明入,逕州北及東,又東南入三臺,亦謂之內水。龍安水、茶坪水俱自安縣入。並東南流,與涪江合。州產鹽,有中井十一,下井一。鹽捕州判駐豐穀井。縣丞駐魏城。驛二:魏城、金山。德陽沖,繁。州西南百五十里。北:鹿頭山、浮中山。綿水一名綿陽河,自綿竹入,東南流,逕城南入漢州。石亭水亦自綿竹入,逕城西南,入漢州。北:鹿頭關。一驛:旌陽。安繁。州西北百一十里。北:千佛。東:西昌山。南:浮山。東北:金山。黑水河一名寧口河,冷水河一曰乾河,並東南流入羅江。茶坪水源出千佛山,發源東南,逕城西會龍安水入州。西小壩、睢水,北曲山三關。綿竹繁。州西南百八十里。北:武都。南:文曲。西南:飛鳧。西北:紫巖山。綿水、石亭水俱自茂州入,左流為綿水,逕城北,東南入德陽。射水一名紫溪河,源出三溪山,逕城南,與石亭水合。白水河源出土司漆寨坪,東南流,逕城西南,馬尾河源出土司天池山,東南流,逕城西北,折而東,並入射水河。南:石碑鎮。梓潼沖,繁。州東北百二十里。東:兜率山。西:葛山。南:長卿山。北:五婦山。梓潼水一名歧江,源出龍安平武山谿,東南流,逕城西南,又南入鹽亭,即古馳水也。西北:九曲水,源出龍安洞子口,九轉入潼江。一驛:武連。羅江沖,繁。州西南九十里。北:潺山。南:天臺山。西南:龍池山。黑水、冷水俱自安縣入,東南流,至縣東北合,是為羅江。又折南,逕縣東入中江。南:芙蓉溪,源出白馬關下,東南流,至縣南,與羅江合,一名三紫水。西南:白馬關。一驛:羅江。

資州直隸州:,繁,難。隸川南永寧道。明,資縣。順治初,仍明制,為資縣,屬成都府。雍正五年,升直隸州,以成都之仁壽、井研、資陽、內江來屬。西北距省治三百四十里。廣四百三十里,袤五百里。北極高三十九度五十分。京師偏西十一度三十二分。領縣四。資山在西北,州以此名。南:銀山、鐵山。西南:玉京、金爐。西:盤石山。中江自資陽入,逕城西南為資江,亦曰中江。北納小溪,東納大濛溪,東南流入內江。珠溪源出井研北境,東北流,至州西北與中江合。大濛溪源出西龍家壩,又名都溪,東流逕城南,至唐明渡入資江。州判駐羅泉井。一驛:珠江。資陽繁,難。州西百三十里。東:寶臺、萬鐘。西:鳳臺。南:書臺。西南:獨秀,亦名資江。沱江亦名雁江,自簡州入,楊花溪自樂至西來注之。資溪、孔子溪均東來注之,南入州。一驛:南津。內江沖。州東南九十里。西:翔龍、華萼。東:降福。南:鏵影。西南:石城。東南:金紫山。沱江自州入,逕城南,清流河合高橋河入之,南入富順。西南:玉帶溪,流合中江。城內西北隅有桂湖,與中江通。一驛:安仁。仁壽繁,難。州西二百里。三隅山峙東、西、北三隅。南:覺山。西:天池。東:佛巖山。赤水一名黃龍溪,自簡州入,西流逕縣北,又西入彭山,合府河。魚蛇水發源縣西境,西南流入眉州。井研簡。州西南二百四十里。城內麟山。西:書臺、五星。北:瑞芝、九龍。東北:鐵山。西南:磨玉山。擁思茫水有二源,夾城西南流,合為泥溪,入樂山。縣產鹽,有上井四,中井七,下井二百二十六。

茂州直隸州:中。原隸成綿龍茂道。光緒三十四年裁。明,成都府屬州。順治初,仍明制。雍正六年,升直隸州,以成都之汶川及保縣來隸。嘉慶六年,省保縣入雜穀。東南距省治四百十里。廣百八十里,袤四百三十里。北極高三十度三十七分。京師偏西十二度三十一分。領縣一,土司六。東南:岷山,一名雪山,俗呼九嶺山,北自松茂,南接灌縣。東:五味山。南:巨人。北:茂濕山。岷江自松潘入,南流逕州西,亦曰汶江,黑水河即古翼水,東南來注,松溪自黑虎寨來注。又北,納三溪,南納南龍溪及白水河,西流入江。東桃坪、南七星、雁門、實大四關。一驛:來遠。汶川沖,繁。州西南百二十里。南:岷山,又南娘子嶺,為縣門戶。東:玉壘。西:河屏。北:壽山、七盤。東南:龍泉山。岷江自雜穀入,逕縣北,名汶江,亦名玉輪江。東納大溪口水,西納登溪溝水,逕城西南,桃川水自東來注,又草坡河、龍潭溝、天赦山水、臥龍關水,並東南來注,入灌縣。有桃關、徹底二關。驛二:寒水、太平。瓦寺宣慰司隸汶川。在縣西北。明為安撫司。嘉慶元年改置。司境有草坡河。沙壩安撫司隸州。在州北。仍明舊。靜州長官司隸州。在州東。仍明舊岳希長官司隸州。在州西。仍明舊。實大關長官司隸州。在州西。仍明舊。隴木長官司隸州。在州西。仍明舊。

忠州直隸州:繁,難。隸川東道。明,重慶府屬州。順治初,仍明制。雍正十二年,升直隸州,以重慶之酆都、墊江及夔州之梁山來隸。西距省治一千五百里。廣二百六十里,袤百八十里。北極高三十度十六分。京師偏西八度二十分。領縣三。東:毓秀。西:高盈山、屏風山。東南:塗山。東北:九亭山。大江自酆都入,逕城西,西溪來注。又逕州東,渰溪河來注。又東,塗井河自西來注。又北入萬縣。州產鹽,有上井三,中井八,下井二十四。州判駐石橋井,巡司駐敦里八甲。東南:塗井鎮。酆都簡。州西南百十里。東:青牛、大峰。西:石璧。南:金盤。東北:平都山。水經所謂「逕東望峽,東歷平都」者也。大江自涪州入,東北流,逕城南,又東北入州。渠溪自州西南流,葫蘆溪自石主西流,碧溪自金盤山東南流,並入大江。西:北涪鎮。墊江繁,難。州西北百三十里。東:佛轉山。西:白龍洞。南:望月。東南:將軍崖山。羅平水有三源,北源出石人山,西源出白龍洞,南源出將軍崖,會於三河口,又東與高灘溪合。高灘溪自梁山入,逕城東南,又西南入長壽,為龍溪。一驛:白渡。梁山繁,難。州西北百里。東:峰門。西:金鳳。南:石馬。北:高都。東:蟠龍山,下有溪東南流,入州,為塗溪。又桂溪,發源五斗山,北流逕城西,折西南流入墊江,為高灘溪。紵溪源出縣境,東南流入萬溪。虎溪鎮。一驛:太平。

酉陽直隸州:繁,難。隸川東道。明,酉陽宣慰司。屬重慶府。順治初,仍明制。雍正十二年,改重慶屬之黔江、彭水二縣置黔彭直隸。十三年,又改平茶長官司為秀山縣,屬。乾隆元年,廢,改為酉陽直隸州,以黔、彭、秀三縣來隸。西北距省治一千七百四十里。廣四百六十里,袤五百六十里。北極高二十八度五十一分。京師偏西七度三十八分。領縣三。北:酉陽山,州以此得名。東:龍山、荷敷。西:鬼巖。南:佛山。東南:三江山。黔江自貴州安化入,逕城西,納南溪河、洪渡河,入彭水。北河自湖北來鳳入,逕城東,南流,會邑梅河,折東入湖南保靖,為酉水。東南:疊溪,上承凱歌河,自貴州銅仁入,亦名買賽河,東北流,秀山之哨溪來會。又納後溪、容溪,東入酉水。州同駐龍潭鎮。巡司駐龔灘鎮。秀山繁,難。州東南二百六十里。西:高秀山,縣以此名。東:巴慣山。南:擎團、鼎桂。西南:白歲山,哨溪出焉,東與滿溪合,入州會買賽河。南:地澄溪,東合遵岫溪,入凱歌河。邑梅河在東南,有紅河溪會嘉塘河東北流注之,又與北河合。巡司駐石堤。黔江簡。州北二百八十里。東:酉陽山。北:黃連大堊山。西:金雞箐山。西南:梅子關山。唐崖河自湖北咸豐入,大木溪合七十八溪水來入之。阿蓬水亦名東小溪,逕城東南,又西南入州,為南溪河。有石勝、白崖、梅子三關。彭水難。州西二百里。西:壺頭山。東:甘山。南:丹陽。西南:盈川山。東北:伏牛山。涪陵江即黔江,自州入,西納長溪,北逕城西。龍嘴河自黔江來會,後江河、水洞河入之。又北納合溪河、射香溪,西入涪州合大江。東北:亭子關。東:鹽井、鬱山二鎮。巡司駐鬱山鎮。

眉州直隸州:沖,繁。隸建昌道。明,州。康熙初,彭山、青神二縣先後省入州。雍正六年復置,仍隸州。東北距省治百九十里。廣百六十里,袤百八十里。北極高三十度六分。京師偏西十二度三十一分。領縣三。西南:連鼇山。西:醴泉。北:盤龍。東:蟆頤山。下臨玻瓈江,一名蟆頤津,即岷江,自彭山入,逕武陽驛,分流復合,南入青神。醴泉江發源盤龍山,東西二源,出盤龍山,分流至州北,合為雙河口,繞州城與松江合,入岷江。思濛江在南,一名芙蓉溪,澭甘水在西南,一名金流江,俱自丹棱入,逕州東南流,並至青神與岷江合。有魚耶、東館二鎮。丹棱簡。州西九十里。南:長山。北:龍鵠山。東南:三峰、金釜二山。思濛江源出龍鵠山。夷郎川源出赤崖山,與思濛合,澭甘水自洪雅入,俱東南流入州。南:柵頭鎮。彭山繁。州北四十里。東:金華山。北:彭亡山,本名彭女,水名彭望。東北:崌崍、天社。西北:回龍山。大江一名汶江,又名武陽江,自新津入,逕城東北入州。府河即錦水,下流納赤水,俱自仁壽入,南流入大江。東北:雙江鎮。青神沖。州南八十里。西:熊耳。西:多棱山。東:上巖、中巖、下巖,即三巖。大江一名導江,自州入,南流入樂山。思濛江、澭甘水俱自州入,東北魚蛇水自仁壽入,西南流,並入大江。

瀘州直隸州:要,沖,繁,難。川南永寧道治所。明,州。光緒三十四年,析九姓鄉隸永寧州。西北距省治七百五十里。廣三百十里,袤二百二十里。北極高二十八度五十四分。京師偏西十度五十七分。領縣三。州治在忠山麓,即寶山,一名瀘峰。東:神臂巖。南:方山。北:玉蟾山。資江即沱江下流,自富順入,東流逕北門外,至城東北,與大江會。大江自納溪入,東北流,逕城南,折流合沱江,曰合江,又東入合江。悅江源出榮昌白馬洞,南流入大江。支江自富順椽子漕入,思晏江自榮昌入,並南入資江。九曲溪自隆昌入,南流至玉蟾山下合思晏江。南龍透、北玉蟾二關。巡司駐嘉明鎮。州判駐九姓鄉。納溪沖。州西南四十里。東:樓子、掇旗。西:冠山。南:馬鞍山。北:濱江。西:納溪,俗名清水河,即永寧河下流,源出阿永番部,東流入大江。南:倒馬、石虎二關。驛一:江門。合江沖,難。州東北百二十里。南:少岷,即安樂山。東南:榕山。西南:丁山。大江自州入,東流,逕北門東入江津。安樂溪一名小江,即古大涉水,亦曰習部水,自貴州仁懷入。之溪亦自仁懷入,合流至城東北入大江。南:符關。江安沖。州西南百十里。南:南照山。北:北照山。東:鳳凰山。大江在城北,自南溪入,東北流,入納溪。淯溪自長寧入,東北流,逕城西北,入大江。綿溪源出連天山,亦入大江,曰綿水口。

永寧直隸州:要,沖,繁,難。隸川南永寧道。明,敘州府。敘永同知及貴州都司永寧衛轄地。順治初,仍明制,置同知,隸敘州府。析永寧衛隸貴州威寧府。康熙二十六年,改衛為縣。雍正五年,地並入縣,改屬敘州府。八年,復設同知。乾隆元年,升為敘永直隸,以永寧縣來屬。光緒三十三年,以永寧移治古藺。三十四年,改曰永寧直隸州,改縣曰古藺,並析瀘州之瀘衛,分州地曰九姓鄉,置古宋縣屬焉。西北距省治九百九十里。廣四百餘里,袤三百九十里。北極高二十七度五十六分。京師偏西十一度十三分。領縣二。東:天馬山。西:寶真。南:青龍。東北:紅崖。東南:獅子山。永寧河亦曰界首河,一源自小井壩入,逕城西,一源自鐵矢坎入,合北流,通江溪自貴州入,納魚漕溪注之,入納溪,合大江。東:羅付大河,與貴州遵義接界,下流入烏江。東雪山、西北江門二關。驛一:永安。古藺繁,難。州東九十里。舊為巡檢司駐。光緒三十三年改永寧縣為今名,移治此。東:雪山。西:海漫山。赤水河自雲南鎮雄入,逕赤水衛東北,合永寧河入納溪。北:梯口關。縣丞一,駐赤水鎮。古宋沖,繁,難。州西。舊瀘衛。明設九姓長官司,屬永寧衛,後屬瀘州。順治四年歸附,仍明制。康熙二十四年並入瀘州。雍正四年設州同,後改州判。光緒三十四年裁,升縣改今名。西:中和山。南:古洞巖。魚漕溪東流入州,合通江溪。

松潘直隸:要,沖,繁,難。舊隸成綿龍茂道。明,松潘衛,隸四川都司。順治初,仍明制為衛,屬龍安府。雍正九年,裁衛置。乾隆二十五年,升直隸。舊隸成綿龍茂道。松潘鎮總兵駐。南距省治九百五十里。廣二百七十七里,袤二百二十里。北極高三十二度四十六分。京師偏西十二度五十一分。南:火焰山。北:大小分水嶺。西北:岷山,即瀆山,又謂之汶阜,一名沃焦山。禹導江處,其水曰瀆水,即岷江,一曰汶江。東:雪欄山,下有白水,為涪江之源。合三舍堡、羊峒口諸水,經小河營,曰小河,入平武。岷江自岷山之羊膊嶺南來,殺鹿洞一水東來注,經黃勝關弓槓口,一水西來注,逕東南,左納東勝河,右納窗河。又南,左納雲昌溝,右納山壩溪,經平定關入茂州。西:黑水河,有南北二源,合流亦入茂州。有望山、雪欄、風洞、紅崖、黃勝、平定、武都等關。巡司一,駐南坪。

石砫直隸:簡。隸川東道。明,宣慰司,屬夔州府。順治十六年歸附,仍明制,授宣慰司,屬夔州府。乾隆二十七年,升為直隸。西距省治一千二百里。廣二百三十里,袤二百四十里。北極高三十度十八分。京師偏西八度十五分。東:石砫山。南:大峰門山。北:方鬥山。大江自酆都入,右納神溪、鍾溪、沼溪,東北流入萬縣。東南:賓河有二源,俱自湖北利川入,曰龍嘴溪,曰冷箐溪,逕沙子關,合為三江溪。又西南流曰後河,逕北,大鳳溪來注。又西南,江池溪自龍潭來注。又西南為葫蘆溪,西北流入酆都,注大江。東沙子、南大風二關。巡司一,駐西界沱。

理番直隸:難。舊隸成綿龍茂道。明,雜穀安撫司,屬茂州。順治初,仍明制。乾隆十七年改,駐理番同知。二十五年,升直隸。嘉慶六年,以茂州屬之保縣入之。東南距省治三百八十里。廣九百六十五里,袤一百七十里。北極高三十一度四十分。京師偏西十三度十三分。領土司四。西:熊耳山。東:高碉。北:馬鞍、龍山。西北:姜維、花崖二山。大江自茂州入,逕東南,又南入汶川。沱江在城西北,有二源:南曰雜穀河,北孟董溝,並東南流,至城西北而合,折南入大江。西:大溪,源出梭磨土司東界大閉氻雪山,東南流,亦入大江。西南維關、鎮遠關,西北鎮安關。梭磨宣慰司西北。舊為長官司。乾隆四十年升置。大溪源出司境大雪山,東北流入。從噶克長官司西北。乾隆十八年置。卓克採長官司西。乾隆十四年置。丹壩長官司西。舊為土舍。乾隆二十四年改置。

懋功屯務:大小金川土司地。順治七年,小金川歸附。康熙六年,大金川歸附。雍正元年,授安撫司。乾隆四十一年,分置美諾、阿爾古兩。四十四年,並阿爾古入美諾。四十八年,改懋功,駐同知,理五屯事務。廣千四百五里,袤五百七十里。北極高三十度四十四分。京師偏西十三度六十分。領屯五,土司二。懋功屯治。東:巴郎山。南:漢牛雪山。北:日爾拉山。西南:喇嘛寺山。東北:商角山。小金川河自撫邊入,東南流,逕北,受南北兩山水,至章谷合金沙河。撫邊屯北百三十五里。北:孟拜山。西:空卡雪山。小金川河在屯南,合日爾拉、索烏、巴郎諸山水,西南入懋功。章穀屯西百八十里。東:墨爾多山、丹噶山。金川河自崇化入,逕屯東南,與小金川河合,折西南,流入打箭爐,為大渡河。崇化屯西二百五十里。東:刮耳崖。東南:丹噶山。東北:木果木山。金川河自綏靖入,逕屯西入章穀。小溪河發源空卡山,東流入小金川河。綏靖屯西二百七十里。東:索烏山。南:足古山。東南:功噶山。金川河自綽斯甲布土司入,逕屯西入崇化。鄂克什安撫司東。乾隆十五年置。綽斯甲布安撫司西。乾隆四十一年置。東:宜喜山。金川河自司境南流入綏靖。


\end{pinyinscope}