\article{列傳八十八}

\begin{pinyinscope}
訥親傅恆子福靈安福隆安福隆安子豐紳濟倫福長安

訥親,鈕祜祿氏,滿洲鑲黃旗人,額亦都曾孫。父尹德,附見其父遏必隆傳,訥親其次子。雍正五年,襲公爵,授散秩大臣。十年,授鑾儀使。十一年十二月,命在辦理軍機處行走。十三年,世宗疾大漸,訥親預顧命。高宗即位,莊親王允祿、果親王允禮、鄂爾泰、張廷玉輔政,號「總理王大臣」。授訥親鑲白旗滿洲都統、領侍衛內大臣,協辦總理事務。十二月,敕獎訥親勤慎,因推孝昭仁皇后外家恩,進一等公。乾隆元年,遷鑲黃旗滿洲都統。二年,遷兵部尚書。十一月,莊親王等請罷總理事務,訥親授軍機大臣。敘勞,加拖沙喇哈番世職。三年二月,領戶部三庫。九月,命協辦戶部。直隸總督李衛劾總河硃藻詐欺貪虐,命訥親與尚書孫嘉淦勘讞,藻坐流。訥親因與嘉淦條上永定河南北岸建築閘壩諸事。十二月,遷吏部尚書。四年五月,加太子太保。

訥親貴戚勛舊,少侍禁近,受世宗知,以為可大用。迨高宗,恩眷尤厚。訥親勤敏當上意,尤以廉介自敕,人不敢干以私。其居第巨獒縛扉側,絕無車馬跡。然以早貴,意氣驕溢,治事務刻深。左都御史劉統勛疏論訥親領事過多,任事過銳。上諭曰:「訥親為尚書,模棱推諉,固所不可,但治事未當,亦所不免,朕時時戒毋自滿。今見此奏,益當自勉。」語詳統勛傳。

九年正月,命訥親閱河南、江南、山東諸省營伍,並勘海塘、河工。時直隸天津、河間二府方以災治賑,令順道先往察覈。疏請展賑一月,從之。訥親使事既蕆,分疏上陳,其勘諸省營伍,言:「遍閱三省督撫、河漕、提鎮為標者十七,優絀互見。惟河南南陽、江南蘇松水師二鎮最劣。請下部覈賞罰。」其勘江、浙海塘,言:「舊日浙江潮自蜀山中小亹出入,近海寧為北大亹,近蕭山為南大亹,漲沙寬闊,為杭州、紹興二府保障。迨中小亹漸湮,潮趨蜀山北,震蕩為患。若濬中小亹故道,減大亹潮力,上下塘工悉可安堵;即中小亹未可遽復,則當擇險要多為坦坡,木石戧壩,俾撇水積淤資以御潮。至諸處柴塘,停沙阻水,無煩議改石工。入江南境,地平而潮緩,華亭舊塘堅緻,寶山新塘尺度參差,工作又不中程。金山、奉賢、南匯、上海皆土塘,距海稍遠,所司守護如法,當無他虞。」其勘洪澤湖,請濬鹽河俾通江,疏串場河俾達海,並停天然二壩、高堰下游二堤。其勘南旺湖,請以湖中涸地貸貧民耕稼。別疏言:「各直省政事,督撫下司道,司道下州縣,州縣官惟以簿書錢穀為事,戶口貧富、土地肥瘠、物產豐嗇、民情向背、風俗美惡、以及山川原隰、橋梁道路,皆漫置不省。官但有條教,民惟責納賦,浮文常多,實意殊少。請敕各直省督撫,令州縣官遍歷境內,何事當興舉,何事當整飭,行之有無治效,以實報長官,長官即是為殿最,以實達朝廷。似亦崇實效、去虛文、飭吏治、厚民生之一端也。」皆下部議行。

十年三月,協辦大學士。五月,授保和殿大學士,仍兼吏部尚書。十二年四月,命如山西會巡撫愛必達讞萬全民張世祿、安邑民張遠等挾眾抗官狀,論如律。愛必達及總兵羅俊、蒲州知府硃發等皆坐譴黜。十三年正月,命如浙江會大學士高斌覆勘巡撫常安貪婪狀,未至,高斌鞫得常安實受賕,訥親與共奏,論如律。三月,復命如山東會巡撫阿里袞治賑。

時大金川土司莎羅奔攻革布什咱土司犯邊,上命川陜總督張廣泗討之。大金川地絕險,阻山為石壘,名曰碉,師進攻弗克。四月,召訥親還京師,授經略大臣,率禁旅出視師。六月,訥親至軍,下令期三日克噶拉依,噶拉依者,莎羅奔結寨地也。師循色爾力石梁而下,攻碉未即克,署總兵任舉勇敢善戰,為諸軍先,沒於陣。訥親為氣奪,乃議督諸軍築碉,與敵共險,為持久。疏入,上重失任舉,又以築碉非計,手詔戒訥親,因時度勢,以為進止。訥親與廣泗合疏言:「天時地利皆賊得其長,我兵無機可乘。冬春間當減兵駐守,明歲加調精銳三萬,於四月進剿,足以成功,至遲亦不逾秋令。」訥親又別疏言:「來歲增兵,計需費數百萬。若俟二三年後有機可乘,亦未可定。」疏入,上諭曰:「卿等身在戎行,目擊情狀,不能確有成算,游移兩可。朕於數千里外,何從遙度?我師至四萬,彼止三千餘,何以彼應我則有餘,我攻彼則不足?卿等當審定應攻應罷,毋為兩歧語。」上知訥親不足辦敵,諭軍機大臣議召訥親還;又念大金川非大敵,重臣視師,無功而還,傷國體,為四夷姍笑。密以諭訥親,冀激奮克敵。居數月,師雖有小勝,卒未得尺寸地。訥親惟請還京面對,乃召訥親及廣泗詣京師,以岳鍾琪攝經略,傅爾丹攝川陜總督,復遣尚書班第同治軍事。尋奪訥親官,令自具鞍馬,從討噶爾丹贖罪,逮廣泗。

九月,命大學士傅恆代為經略,別遣侍衛富成逮訥親,責置對,並令富成錄訥親舉止言語以聞。上前後手詔罪訥親恆數千百言,略謂:「訥親受命總戎,乖張畏縮。疏言軍夜攻碉,自帳中望見火光,知未嘗臨敵。又言督軍攻阿利山,既回營,我軍數十人各鳥獸散。知偶臨敵,又先士卒退。富成疏訥親語『金川事大難,不可輕舉,此言不敢入奏』。訥親受恩久,何事不可言?如固不能克,當實陳請罷兵。乃事敗欲以不可輕舉歸過朝廷,狡詐出意外。又值續調兵過,輒言『此皆我罪,令如許滿洲兵受苦』。滿洲兵聞調,鼓舞振躍,志切同仇。訥親以為受苦,實嫉他人成功,搖眾心,不顧國事。孤恩藐法,罪不可逭。」

十月,諭「訥親先世以軍功封二等公,為孝昭仁皇后戚屬,供職勤慎,進一等公。獲罪,應仍以二等公俾其兄策楞襲爵」。訥親恃上恩,尚冀入見上自解,上復迭降手詔,謂:「軍旅事重,平日治事詳慎,操守潔清,舉不足言。」又謂:「訥親小心謹密,而方寸一壞,天奪其魄,雖欲幸免而不能。」十二月,廣泗既誅,上封遏必隆遺刀授侍衛鄂實,監訥親還軍,誅以警眾。十四年正月,上命傅恆班師,復諭鄂實即途中行法。是月戊寅,鄂實監訥親行至班攔山,聞後命,遂誅訥親。

傅恆,字春和,富察氏,滿洲鑲黃旗人,孝賢純皇后弟也。父李榮保,附見其父米思翰傳。傅恆自侍衛洊擢戶部侍郎。乾隆十年六月,命在軍機處行走。十二年,擢戶部尚書。十三年三月,孝賢純皇后從上南巡,還至德州崩,傅恆扈行,典喪儀。四月,敕獎其勤恪,加太子太保。時訥親視師金川,解尚書阿克敦協辦大學士以授傅恆,並兼領吏部。訥親既無功,九月,命傅恆暫管川陜總督,經略軍務。尋授保和殿大學士,發京師及諸行省滿、漢兵三萬五千,以部庫及諸行省銀四百萬供軍儲,又出內帑十萬備犒賞。十一月,師行,上詣堂子告察,遣皇子及大學士來保等送至良鄉。傅恆既行,上日降手詔褒勉。傅心互道陜西,言驛政不修誤軍興,上命協辦大學士尚書尹繼善攝陜西總督,主饋運。入四川境,馬不給,上又命尹繼善往來川、陜督察。旋以傅恆師行甚速,紀律嚴明,命議敘,部議加太子太傅,特命加太保。固辭,不允,發京師及山西、湖北馬七千佐軍。傅恆發成都,經天赦山,雪後道險,步行七十里至驛。上聞,賜雙眼孔雀翎,復固辭。

初,小金川土舍良爾吉間其兄澤旺於莎羅奔,奪其印,即烝於嫂阿扣。莎羅奔之犯邊也,良爾吉實從之,後詐降為賊諜。張廣泗入奸民王秋言,使領蠻兵,我師舉動,賊輒知之。傅恆途中疏請誅良爾吉等,將至軍,使副將馬良柱招良爾吉來迎,至邦噶山,正其罪,並阿扣、王秋悉誅之。事聞,上褒傅恆明斷,命拜前賜雙眼孔雀翎,毋更固辭。

十月,至卡撒,以屯軍地狹隘,與賊相望,且雜處番民巿肆中,乃相度移舊壘前,令總兵冶大雄監營壘。十四年正月,上疏言:「臣至軍,察用兵始末:當紀山進討之始,馬良柱轉戰而前,逾沃日收小金川直抵丹噶,其鋒甚銳。彼時張廣泗若速進師,賊備未嚴,殄滅尚易;乃坐失事機,宋宗璋宿留於雜穀,許應虎敗衄於的郊,賊得盡據險要,增碉備御。訥親初至,督戰甚急,任舉敗沒,銳挫氣索,軍無鬥志,一以軍事委張廣泗。廣泗又為奸人所愚,專主攻碉。先後殺傷數千人,匿不以聞。臣惟攻碉最為下策,槍砲不能洞堅壁,於賊無所傷。賊不過數人,自暗擊明,槍不虛發。是我惟攻石,而賊實攻人。賊於碉外為濠,兵不能越,賊伏其中,自下擊上。其碉銳立,高於浮屠,建作甚捷,數日可成,旋缺旋補。且眾心甚固,碉盡碎而不去,砲方過而復起。客主勞佚,形勢迥殊,攻一碉難於克一城。即臣所駐卡撒,左右山巔三百餘碉,計日以攻,非數年不能盡。且得一碉輒傷數十百人,得不償失。兵法,攻堅則瑕者堅,攻瑕則堅者瑕。惟使賊失所恃,我兵乃可用其所長。擬俟諸軍大集,分道而進。別選銳師,旁探間道,裹糧直入,逾碉勿攻,繞出其後。番眾不多,外備既密,內守必虛。我兵既自捷徑深入,守者各懷內顧,人無固志,均可不攻自潰。卡撒為進噶拉依正道,嶺高溝窄,臣當親任其難。黨壩隘險,亦幾同卡撒,酌益新軍。兩道並進,直搗巢穴,取其渠魁。期四月間奏捷。」上以金川非大敵,勞師兩載,誅大臣,失良將,內不懌。及是聞其地險難下,益不欲竟其事,遂以孝聖憲皇后諭命班師,而傅恆方督總兵哈攀龍、哈尚德等攻下數碉。上以金川水土惡,賜傅恆人蓡三斤,並及諸將有差,屢詔召傅恆還。又以孝聖憲皇后諭封一等忠勇公,賜寶石頂、四團龍補服。傅恆奏言:「金川事一誤,今復輕率蕆事,賊焰愈張。眾土司皆罹其毒,邊宇將無寧日。審度形勢,賊碉非盡當道,其巢皆老弱,我兵且戰且前,自昔嶺中峰直抵噶拉依,破竹建瓴,功在垂成,棄之可惜。且臣受詔出師,若不掃穴擒渠,何顏返命?」並力辭封賞,上不允,手詔謂:「匈奴未滅,無以家為,乃驃姚武人銳往之概。大學士抒誠贊化,豈與兜鍪閫帥爭一日之績?」反復累數千言,復賜詩喻指。

時傅恆及提督岳鍾琪決策深入,莎羅奔遣頭人乞降,傅恆令自縛詣軍門。莎羅奔復介綽斯甲等詣嶽鍾琪乞貸死,鍾琪親入勒烏圍,挈莎羅奔及其子郎卡詣軍門。語詳鍾琪傳。傅恆遂受莎羅奔父子降,莎羅奔等焚香作樂,誓六事:無犯鄰比諸番,反其侵地,供役視諸土司,執獻諸酋抗我師者,還所掠內地民馬,納軍械槍砲,乃承制赦其罪。莎羅奔獻佛像一、白金萬,傅恆卻其金,莎羅奔請以金為傅恆建祠。翌日,傅恆率師還。上優詔嘉獎,命用揚古利故事,賜豹尾槍二桿、親軍二名。三月,師至京師,命皇長子及裕親王等郊迎。上御殿受賀,行飲至禮。傅恆疏辭四團龍補服,上命服以入朝,復命用額亦都、佟國維故事,建宗祠,祀曾祖哈什屯以下,並追予李榮保謚,賜第東安門內,以詩落其成。

十九年,準噶爾內亂,諸部臺吉多內附。上將用兵,諮廷臣,惟傅恆贊其議。二十年,師克伊犁,俘達瓦齊以歸,諭再封一等公,傅恆固辭,至泣下,乃允之。尋圖功臣像紫光閣,上親制贊,仍以為冠,舉蕭何不戰居首功為比。二十一年四月,將軍策楞追捕阿睦爾撒納未獲,上命傅恆出視師,赴額林哈畢爾噶,集蒙古諸臺吉飭軍事。傅恆行日,策楞疏至,已率兵深入,復召傅恆還。

三十三年,將軍明瑞徵緬甸敗績,二月,授傅恆經略,出督師。時阿里袞以副將軍主軍事,上並授阿桂副將軍、舒赫德參贊大臣,命舒赫德先赴雲南,與阿里袞籌畫進軍。三十四年二月,傅恆師行,發京師及滿、蒙兵一萬三千六百人從征,上御太和殿賜敕,賚御用甲胄。四月,至騰越,傅恆決策,師循戛鳩江而進,大兵出江西,取道猛拱、猛養,直搗木梳,水師沿江順流下,水陸相應。偏師出江東取猛密,夾擊老官屯。往歲以避瘴,九月後進兵,緬甸得為備。傅心互議先數十日出不意,攻其未備,水師當具舟。上初命阿里袞造舟濟師,阿里袞等言崖險澗窄不宜舟,傍江亦無造舟所。上又命三泰、傅顯往視,言與阿里袞等同。及傅恆至軍,諮土司頭人,知蠻暮有山曰翁古多木,旁有地曰野牛壩,野人所居,涼爽無瘴。即地伐木造舟,野人樂受值,執役甚謹。傅恆即使傅顯佐蒞事。舟成,督滿、漢兵並從行奴僕,更番轉搬。又得茂隆廠附近砲工,令範銅為砲。狀聞,輒降旨嘉獎,為賦造舟行焉。

傅恆初議自將九千三百人渡戛鳩而西,師未集,七月,將四千人發騰越。上以經略自將師寡,促諸軍速集如初議。八月,傅恆自南蚌趨戛鳩。奏至,上方行圍木蘭,入圍獲包,畀福隆安以賜傅恆。傅恆道南底壩至允帽,臨戛鳩江,時猛拱大頭人脫猛烏猛、頭人賀丙等,詣傅恆請降。師至,脫猛烏猛將夾江諸夷寨頭人來迎,與賀丙具舟。傅恆命分兵徐濟,夾江為寨猛拱后土司渾覺亦請降,獻馴象四。上賚三眼孔雀翎,傅恆疏辭。師復進,取猛養,破寨四,誅頭人拉匿拉賽。設臺站,令瑚爾起以七百人駐守。遂至南董干,攻南準寨,獲頭人木波猛等三十五人。進次暮臘,再進次新街。

傅恆自渡戛鳩江,未嘗與緬甸兵戰,刈禾為糧,行二千里不血刃,而士馬觸暑雨多疾病。會阿桂將萬餘人自虎踞關出野牛壩,造舟畢成,徵廣東、福建水師亦至,乃合軍並進。哈國興將水師,阿桂、阿里袞將陸師,阿桂出江東,阿里袞出江西。緬兵壘金沙江兩岸,又以舟師扼江口。阿桂先與緬兵遇,麾步兵發銃矢,又以騎兵陷陣,緬兵潰。哈國興督舟師乘風蹴敵,緬兵舟相擊,死者數千。阿里袞亦破西岸緬兵,傅恆以所獲纛進。上復為賦詩,阿里袞感瘴而病,改將水師,旋卒。十一月,傅恆復進攻老官屯,老官屯在金沙江東,東猛密,西猛墅,北猛拱、猛養,南緬都阿瓦,為水陸通衢。緬兵伐木立寨甚固,哈國興督諸軍力攻,未即克。師破東南木寨,緬兵夜自水寨出,傅恆令海蘭察御之,又令伊勒圖督舟師掩擊,復獲船纛。緬兵潛至江岸築壘,又自林箐中出,海蘭察擊之,屢有斬馘。

師久攻堅,士卒染瘴多物故,水陸軍三萬一千,至是僅存一萬三千。傅恆以入告,上命罷兵,召傅恆還京。傅恆俄亦病,阿桂以聞。上令即馳驛還,而以軍事付阿桂。會緬甸酋懵駁遣頭人諾爾塔齎蒲葉書乞罷兵,傅恆奏入,上許其行成。傅恆附疏言:「用兵之始,眾以為難。臣執意請行,負委任,請從重治罪。」上手詔謂:「用兵非得已,如以為非是,朕當首任其過。皇祖時,吳三桂請撤籓,諮於群臣,議撤者惟米思翰、明珠數人。及三桂反,眾請誅議撤諸臣,皇祖深闢其非。朕仰紹祖訓,傅恆此事,可援以相比。傅恆收猛拱,當賜三眼孔雀翎,疏辭,俟功成拜賜。今既未克賊巢,當繳進賜翎,以稱其請罪之意。」懵駁遣頭人詣軍獻方物。十月,傅恆還駐虎踞關,上命傅恆會雲貴總督彰寶議減雲南總兵、知府員缺,釐正州縣舊制。三十四年二月,班師。三月,上幸天津,傅恆朝行在。既而緬甸酋謝罪表久不至,上謂傅恆方病,不忍治其罪。七月,卒,上親臨其第酹酒,命喪葬視宗室鎮國公,謚文忠。又命入祀前所建宗祠。其後上復幸天津,念傅恆於此復命,又經傅恆墓賜奠,皆紀以詩。及賦懷舊詩,許為「社稷臣」。嘉慶元年,以福康安平苗功,贈貝子。福康安卒,推恩贈郡王銜,旋並命配享太廟。

傅恆直軍機處二十三年,日侍左右,以勤慎得上眷。故事,軍機處諸臣不同入見,乾隆初,惟訥親承旨。迨傅恆自陳不能多識,乞諸大臣同入見。上晚膳後有所諮訪,又召傅恆獨對,時謂之「晚面」。又軍機處諸大臣既承旨,退自屬草,至傅恆始命章京具稿以進。上倚傅恆為重臣,然偶有小節疏失,即加以戒約。傅恆益謙下,治事不敢自擅。敬禮士大夫,翼後進使盡其才。行軍與士卒同甘苦。卒時未五十,上尤惜之。

子福靈安、福隆安、福康安、福長安。福康安自有傳。

福靈安,多羅額駙,授侍衛。準噶爾之役,從將軍兆惠戰於葉爾羌,有功,予雲騎尉世職。三十二年,授正白旗滿洲副都統。署云南永北鎮總兵。卒。

福隆安,尚高宗女和嘉公主,授和碩額駙、御前侍衛。三十三年,擢兵部尚書、軍機處行走,移工部尚書。三十五年,襲一等忠勇公。三十六年,用兵金川,總兵宋元俊劾四川總督桂林,命福隆安往讞。福隆安直桂林,抵元俊罪。四十一年,復授兵部尚書,仍領工部。金川平,畫像紫光閣。四十九年,卒,謚勤恪。

子豐紳濟倫,初以公主子,命視和碩額駙品秩,授鑲藍旗漢軍副都統、奉宸苑卿。四十九年,襲爵。累遷兵部尚書,領鑾儀衛。嘉慶間,再坐事,官終盛京兵部侍郎。十二年,卒。子富勒渾翁珠,襲爵。

福長安,自藍翎侍衛累遷至正紅旗滿洲副都統、武備院卿,領內務府。乾隆四十五年,命在軍機處學習行走。累遷戶部尚書。五十三年,臺灣平。五十七年,廓爾喀平。諸功臣畫像紫光閣,福長安皆與焉。嘉慶三年,俘王三槐,福長安以直軍機處得侯。四年,高宗崩,大學士和珅得罪,仁宗以福長安阿附,逮下獄,奪爵,籍其家。諸大臣議用朋黨律坐立斬,上命改監候,而賜和珅死,使監福長安詣和珅死所跪視。旋遣往裕陵充供茶拜唐阿,就遷員外郎。六年,以請還京,奪職,發盛京披甲。旋自驍騎校屢遷:再為圍場總管,一為馬蘭鎮總兵,再署古北口提督。屢坐事譴謫。二十一年,授正黃旗滿洲副都統。二十二年,卒。

論曰:高宗初政,寬大而清明,舉國熙熙,樂見太平。是時鄂爾泰、張廷玉負夾輔之重,然居中用事為天子喉舌,厥惟訥親,繼之者傅恆也。高宗手詔謂當鄂爾泰在朝,培養陶成,得一訥親;訥親在朝,培養陶成,得一傅恆。又謂訥親受恩第一,次則傅恆。訥親視師失上指,坐誅,終不沒其勤廉;傅恆再以受降還師,德心孚契,自以其謹慎,非徒藉貴戚功閥重也。


\end{pinyinscope}