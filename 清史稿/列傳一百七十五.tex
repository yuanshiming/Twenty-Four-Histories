\article{列傳一百七十五}

\begin{pinyinscope}
桂良瑞麟子懷塔布官文文煜

桂良,字燕山,瓜爾佳氏,滿洲正紅旗人,閩浙總督玉德子。入貲為禮部主事,晉員外郎。出為四川順慶知府,調成都。歷建昌道,河南按察使,四川、廣東、江西布政使。道光十四年,擢河南巡撫。嘉慶中,林清、李文成等以八卦教倡亂,既誅,而汲縣潞州屯墳塔猶祀其神曰「無生老母」,習教者猶眾。御史黃爵滋以為言,命桂良察治,毀其墳廟,廉得河南境內無生廟三十九所,並毀之;地方官失察,譴黜有差。十九年,擢湖廣總督,調閩浙,又調雲貴。二十年,兼署云南巡撫。滇省多盜,奏定緝捕章程;又請迤南、迤西、迤東各標營官兵責成巡道就近稽察。時貴州諸苗蠢動,鎮遠、黎平、都勻、古州苗尤悍,州縣不能制,疏請遴勁兵專主剿捕。二十五年,入覲,留京,署兵部尚書,兼正白旗漢軍都統。尋出為熱河都統。二十八年,召來京,以其女妻皇六子奕,授鑲紅旗漢軍都統。

咸豐元年,署吏部尚書,出為福州將軍。二年,召授兵部尚書。三年,粵匪陷江寧,京師戒嚴。桂良疏請各城門稽查增派八旗章京兵丁,補葺城上兵房,從之。未幾,粵匪竄河北,直隸總督訥爾經額出省防剿,命桂良駐保定為後路聲援,兼防西路要隘。望都、唐縣土匪起,捕誅之。是年秋,賊由山西犯畿南,訥爾經額師潰於臨洺關,隆平、柏鄉相繼陷。訥爾經額褫職逮治,授桂良直隸總督,詔責偕都統勝保速籌防剿。布政使張集馨出兵遷延,劾罷之。賊竄正定、定州、深州、河間、天津,勢剽甚,於是桂良率提督張殿元守保定,科爾沁郡王僧格林沁統大兵駐通州衛京師,勝保督師進剿。四年,大捷於獨流鎮,賊走踞阜城,又走連鎮,僧格林沁、勝保會攻,賊分竄山東,勝保追擊之。桂良遣張殿元赴武邑防堵,劾散秩大臣穆輅、健銳營翼長雙僖縱兵傷官擾民,議譴。

秋,英吉利、美利堅兩國兵船至大沽。時賊氛未靖,詔戒張皇,命桂良相機辦理。尋以前任鹽政崇綸歸調遣,令赴天津會議。英酋咆呤要索十六條,欲遣官駐京及踐廣州入城之約,中外官平禮接見,通商稅則變通舊約;美酋麥蓮則僅言通商一端。崇綸等嚴拒其駐京,餘事令赴廣東聽總督查辦。屢議無要領,咆呤等尋去。五年,僧格林沁連大破賊,賊首林鳳祥、李開芳先後就擒伏誅,畿輔肅清。七年,召拜東閣大學士,管理刑部,兼正藍旗蒙古都統。

八年春,英、法、俄、美四國聯軍北犯,毀大沽砲臺,泊天津城下,聲言將犯京師。倉猝援軍未集,命桂良偕尚書花沙納往議。敵情猖肆,要求益多:以遣官駐京、內江通商、內地游行、兵費賠償後,始交還廣東省城。四事廷議不允。復起故大學士耆英同與議,英人尤不悅,拒之,耆英以擅回京獲罪。桂良等議久不決,廷臣多主戰,實不足恃,而敵日以進兵為恐哧。俄、美兩國調停其間,卒徇所請定議,而通商稅則俟於上海詳定之。

五月,簽約退兵,遂命桂良偕花沙納赴上海,武備院卿明善、刑部員外郎段承實副之,會同兩江總督何桂清議稅則。文宗憤和約之成出於不得已,或獻策許全免入口稅以市惠,冀改易駐京諸條,密授桂良等機宜。八月,至上海,晉文華殿大學士,授內大臣。桂清力言免稅之不可,改約之難成,桂良亦贊其議,上甚怒,必責其補救一二端,而各國因廣東民團仍與為難,且出示偽載諭旨,堅欲罷兩廣總督黃宗漢,停撤民團。桂良等疏聞,乃解宗漢通商大臣,改授桂清。桂良等噤不敢言罷駐京諸事,先議稅則。

十二月,英使額羅金遽率兵船赴廣東,遂罷議。九年,回京,僅美利堅一國遵換通商之約,英軍復犯大沽,僧格林沁預設備,兵至,擊退之。十年,英法聯軍大舉來犯,我師失利。七月,復命桂良赴天津議和,要增兵費,入京換約,嚴詔拒絕。敵陷天津,進逼京師,上幸熱河,恭親王奕留守主撫議,桂良與焉。九月,於禮部換約,視八年原議益增條款,事具邦交志。尋命督辦各國通商事務。十一年,穆宗即位,回京,命在軍機大臣上行走。同治元年,卒,優恤,贈太傅,祀賢良祠,謚文端。

瑞麟,字澄泉,葉赫那喇氏,滿洲正藍旗人。由文生充太常寺讀祝官,補贊禮郎。道光二十七年,祫祭太廟,讀祝洪亮,宣宗嘉之,賜五品頂戴、花翎。二十八年,超擢太常寺少卿,又擢內閣學士,兼管太常寺。三十年,擢禮部侍郎。咸豐元年,兼鑲藍旗滿洲副都統、正黃旗護軍統領。三年,調戶部,命在軍機大臣上行走。時粵匪竄畿輔,踞靜海縣及獨流鎮,命瑞麟率兵從僧格林沁防剿,會攻獨流,克之。靜海賊竄陷阜城,又分竄連鎮及山東高唐州,瑞麟合擊,屢有擒斬。五年,克連鎮,賊首林鳳祥就擒,加都統銜,賜號巴達瑯阿巴圖魯,授西安將軍。未幾,擢禮部尚書,兼鑲白旗蒙古都統。

八年,英兵犯天津,命馳赴楊村籌防。洎撫議定,敵退。文宗知和不可恃,亟治海防,命瑞麟赴天津修築大沽砲臺。尋署直隸總督,增建雙港砲臺,調福建霆船戰船,增募水師。僧格林沁移師天津,分駐要隘。瑞麟回京,調戶部尚書緦拜文淵閣大學士,兼管禮部鴻臚寺、太常寺。九年,管理戶部。十年,充殿試讀卷官,授內大臣。六月,英法聯軍復犯天津,命率京兵萬人守通州。僧格林沁屢戰失利,敵軍進通州,瑞麟偕勝保御之八里橋,左右夾擊,勝保傷砲墜馬,軍潰,敵遂逼京師。瑞麟迎戰安定門外,敗績,褫職。車駕幸熱河,命扈從行在。是年冬,和議成,予侍郎銜,隨僧格林沁剿山東捻匪。攻鉅野羊山集賊巢,失利,馬蹶被傷,退軍濟寧,復褫職,召回京。十一年,授鑲黃旗漢軍都統,管神機營事。

同治元年,出為熱河都統,疏請招佃圍邊荒地八千頃充練餉,允之。二年,調廣州將軍。四年,兼署兩廣總督。信宜、化州土匪起,遣兵平之。粵匪汪海洋由福建竄廣東大埔,遣副將方耀擊走之。入閩會剿,復詔安、平和。賊復竄廣東境,連敗之於長樂、鎮平。時賊蹤往來於福建、廣東、江西界上,瑞麟偕左宗棠疏請三省會剿。詔提督鮑超由江西來援,四面環攻。十二月,殲偽偕王譚體元於黃沙壩,擒首逆汪海洋,誅之,餘賊肅清。捷聞,優詔嘉獎。

五年,實授兩廣總督。廣東素多盜,伏莽時起。時巡撫蔣益澧號知兵,瑞麟部將方耀、鄭紹忠皆能戰,先後破斬五坑客匪,曹沖、赤溪及新安、東莞諸匪,潮州、瓊州洋盜、土匪。九年,兼署巡撫。十年,復拜文淵閣大學士,仍留總督任。十三年,卒,詔嘉前勞,贈太保,祀賢良祠,謚文莊。

子懷塔布,由廕生授刑部主事,晉員外郎。以父恤典擢四品京堂,累遷禮部尚書,充內務府大臣。光緒二十四年,主事王照上書言事,久之始代奏,坐違旨抑格,褫職。未幾,皇太后訓政,起授左都御史,復充內務府大臣,遷理籓院尚書。二十六年,卒,贈太子少保,謚恪勤。

官文,字秀峰,王佳氏,滿洲正白旗人,先隸內務府正白旗漢軍。由拜唐阿補藍翎侍衛,累擢頭等侍衛。道光二十一年,出為廣州漢軍副都統,調荊州右翼副都統。粵匪既陷漢陽,將犯荊州。咸豐三年,將軍臺湧駐防德安,命官文專統荊州防兵。四年,擢荊州將軍。賊陷安陸、荊門、宜昌。時荊州兵多調赴武昌,分屯要隘,城中兵僅二千。監利又陷,官文遣軍復之;連復宜昌、石首、華容,於是荊州稍安,而武昌被圍急,官文遣將沿漢下援。

六月,武昌復失守,命官文統籌全局,規復武漢。因疏言:「賊情詭譎,軍情隨時變幻。武漢之賊一日不盡,荊州不得安枕。賊踞漢陽,倚江為險,絕我糧道,阻我援軍。今欲復武昌,必先攻漢陽,奪賊所恃之險,而後武昌可圖也。總兵雙保自潛江進剿,兵力過單。臣已令羅遵殿以戰船百艘自仙桃鎮、蔡店逕趨漢陽,與撫臣楊霈分道夾攻;又檄總兵福炘往助雙保,知縣吳振鏞進復沔陽以通餉道。惟賊踞岳州,南北援軍均受牽制,尤應先剿岳州之賊。曾國籓方統砲船駐湘陰,塔齊布之師已入岳州境,臣已促其速進,分兵阻江路。復派同知銜李光榮等率川勇防調弦口,張子銘防監利尺八口,都司宗維清沿江接應。荊州僅賸旗兵分守要隘,隨時接應,庶幾可進可退,不致有顧此失彼之虞。」疏入,報聞。尋曾國籓克岳州,賊艘悉出大江,官文遣涼州副都統魁玉、總兵楊昌泗赴螺山防江,殲賊甚多。八月,武昌、漢陽相繼復,論功被優敘。

五年,總督楊霈師潰德安,漢陽、漢口復陷,德安、隨州繼之,詔褫霈職,授官文湖廣總督。師次安陸,疏言:「賊自隨州退踞德安,兇鋒疊挫。惟天門、京山道路四通,儻竄襄河,勾連仙桃鎮以下股匪,不獨荊襄在在堪虞,上游各處均可北竄。現遣兵一由天門、皁市進剿,一往京山防守,臣駐安陸為兩路應援,咨固原提督孔廣順伺隙進取,署提臣訥欽為後應。俟欽差大臣西凌阿入楚,即統兵從襄河兩岸水陸並進,由漢川攻漢陽。」秋,西凌阿戰德安失利,乃命官文代為欽差大臣,馳援德安。賊棄城走,躡追之,直搗漢陽。十二月,督兵薄西門橋,迭敗賊於龜山、尾湖堤、五顯廟,破賊卡,毀東西土城。六年,賊造浮橋從西門分隊來犯,擊卻之。分兵河口斷其糧道,令副都統都興阿攻圍風焚積聚,賊勢漸蹙。秋,破漢陽城外賊營,連戰皆捷。巡撫胡林翼規復武昌。十一月,約同日水陸大舉,分攻武、漢,官文督軍分路進,水師擊漢陽東門,破五顯廟賊卡,李孟群又敗龜山援賊,王國才、楊昌泗由西門攻入,遂復漢陽,俘偽將軍等五百餘人。林翼亦復武昌,詔嘉獎,賜花翎。

七年,偕林翼疏言:「湖北為長江上游要害,武漢尤九省通衢,自來東南有事必爭之地。三次失陷,力攻兩載而後克之。目前相機防剿,不令賊乘間上竄,蹈從前覆轍。業派李續賓由南岸,都興阿、孔廣順、王國才由北岸,楊載福率水師由江路分道進剿。現北岸黃州至黃梅,南岸武昌至興國,均已肅清,崇、通一帶搜捕殆盡;李續賓抵九江,與曾國籓會合進攻;楊載福毀城外賊營;惟小池口賊壘未拔,派鮑超助攻。安徽之英山、太湖、宿松、望江接壤湖北,皆為賊藪,有窺伺上犯之心。飭王國才駐黃梅之大河鋪、界嶺巖,孔廣順駐蘄水之孔隴驛,巴揚阿率馬隊為各路應援,以固楚北門戶。道士洑水闊溜急,田家鎮兩山對峙,水師皆難久駐,酌留各營游巡江面,足備鎮馭。通籌大局,我軍已據水陸上游,實蓄破竹建瓴之勢。所慮江西七府未平,武昌尚有肘腋之患。賊若由通城、崇陽、興國竄逼武昌,反出江西各軍之上,自當固守武昌,以為後路根本。相機籌畫,節節進取,仍步步嚴防,庶軍情無返顧掣肘之虞,轉餉有源源不竭之利。」疏入,報聞。

初,官文由荊州將軍調總督,凡上游荊、市、襄、鄖諸郡兵事餉事悉主之。林翼以巡撫駐金口,凡下游武、漢、黃、德諸郡兵事餉事悉主之。南北軍各領分地,徵兵調餉,每有違言。武昌既復,林翼威望日起,官文自知不及,思假以為重,林翼益推誠相結納,於是吏治、財政、軍事悉聽林翼主持,官文畫諾而已。不數年,足食足兵,東南大局,隱然以湖北為之樞。

八年四月,復九江,論功,加太子少保。皖賊陷麻城、黃安,圍蘄州,先後破走之。七月,胡林翼丁母憂,官文疏請留林翼治軍,改為署理,從之。命官文暫行兼署巡撫,尋以湖廣總督協辦大學士。李續賓戰歿三河,皖、鄂震動。官文分兵扼蘄州、廣濟、麻城諸隘,固守九江、彭澤,水師嚴防江面,人心始定。九年,賊竄湖南,圍寶慶,檄荊宜施道李續宜赴援,大破之,寶慶圍解。十二月,復太湖,被優敘。十一年,拜文淵閣大學士,仍留總督任。時大軍圍安慶急,陳玉成、李秀成先後分兵犯湖北境,冀掣動局勢,遣將迭破之,所陷諸郡縣皆復。八月,克安慶,加太子太保。是年,胡林翼病歿,嚴樹森代之。

降捻苗沛霖踞安徽壽州,詔疆臣議剿撫之策。官文疏陳沛霖包藏禍心,罪大惡極,請伸天討。同治元年,遣副將周鳳山等剿捻於河南信陽、羅山,敗之;又破黃梅捻巢,收復十餘寨:晉文華殿大學士。發、捻合擾楚、豫之交,勢甚熾。荊州將軍多隆阿方督師赴陜西,官文以楚兵不敷分布,奏調回援。九月,多隆阿至,屢戰皆捷,襄河以北賊皆遠遁。三年,劾巡撫嚴樹森把持剛愎,黜之。六月,克復江寧,曾國籓奏捷,推官文列名疏首。詔嘉官文徵兵籌餉,推賢讓能,接濟東征,不分畛域,錫封一等伯爵,號果威,世襲罔替,升入正白旗滿洲,賜雙眼花翎。蓋褒其能與胡林翼和衷卒成大功也。

四年,僧格林沁剿捻戰歿於山東,詔追論前年發、捻擾湖北,官文不能就地殲除,僅驅出境,以致蔓延益熾,下嚴議,降三級調用,改革職留任,褫宮銜、花翎。五年,偕曾國籓奏設長江水師,如議行。湖北巡撫曾國荃劾官文貪庸驕蹇,命尚書綿森、侍郎譚廷襄往按,坐動用捐款,議革職,詔念前勞,原其尚非貪污欺罔,優與保全,解總督,仍留大學士、伯爵,罰伯俸十年。召還京,管理刑部,兼正白旗蒙古都統。尋出署直隸總督。

七年,捻匪張總愚由西路竄擾畿輔,下嚴議。尋李鴻章、左宗棠等入援,七月,捻匪平,復宮銜、花翎。八年,回京,管理戶部三庫,授內大臣。十年,卒,優詔賜恤,贈太保,賜金治喪,遣惠郡王奠醊,祀賢良祠,謚文恭。尋以疆臣請合祀湖北胡林翼專祠。

當官文之在湖北,事事聽林翼所為,惟馭下不嚴,用財不節,林翼憂之。閻敬銘方佐治餉,一日林翼與言,恐誤疆事。敬銘曰:「公誤矣!本朝不輕以漢大臣專兵柄。今滿、漢並用,而聲績炳著者多屬漢人,此聖明大公劃除畛域之效。然湖北居天下要沖,朝廷寧肯不以親信大臣臨之?夫督撫相劾,無論未必勝,即勝,能保後來者必賢耶?且繼者或厲清操,勤庶務,而不明遠略,未必不顓己自是,豈甘事事讓人?官文心無成見,兼隸旗籍,每有大事,正可借其言以伸所請。其失僅在私費奢豪,誠於事有濟,歲糜十萬金供之,未為失計。至一二私人,可容,容之;不可,則以事劾去之。彼意氣素平,必無忤也。」林翼大悟。及林翼歿,督撫不相能,官文劾嚴樹森去之;而曾國荃又劾官文去之。官文晚節建樹不能如曩時,然林翼非官文之虛己推誠,亦無以成大功,世故兩賢之。

孫興恩,襲伯爵。

文煜,字星巖,費莫氏,滿洲正藍旗人。由官學生授太常寺庫使,累遷刑部郎中。出為直隸霸昌道、四川按察使。咸豐三年,遷江寧布政使。時江寧已陷賊,文煜從琦善江北大營。四年,琦善歿於揚州,所部練勇及江北糧臺事宜,命文煜接辦。五年春,粵匪由瓜洲東竄沙頭港,文煜遣勇擊之,賊由對岸扎簰爭渡,偕水師以大砲合擊,賊退瓜洲。文煜以沙頭港為里下河門戶,賊所必爭,築土城砲臺,疏請添募練勇守御,從之。既而賊踞揚州,窺里下河,文煜擊之於萬安橋,大有斬獲,賊勢乃挫。七年,調江蘇布政使,治江南大營糧臺。以支給撙節,為軍中所不便,提督和春劾其拘泥,命來京候另簡用。尋授直隸布政使。

九年,英兵犯大沽,為僧格林沁擊退。戰後將議撫,命文煜從總督恆福赴北塘相機辦理。尋擢山東巡撫。捻匪圍曹縣,分黨擾安陵,檄曹州鎮總兵郝上庠合師內外夾擊,解曹州圍,安陵賊亦退。十年,捻匪又竄單縣,分擾嶧縣得勝徬,遣將擊走之。英法聯軍踞砲臺,文煜遣兵扼利津,自駐濰縣韓亭以防陸路北犯。尋敵船北駛犯北塘,文煜分軍入衛,駐通州,自率眾赴濟寧剿捻匪。

十一年,署直隸總督,尋實授。時和議既成,穆宗回鑾,畿輔馬賊四起,久未凈絕,屢詔責文煜搜捕。同治元年,坐山東降賊張錫珠等擾畿南督剿不力,褫職,戍軍臺。二年,僧格林沁奏調赴營差遣,尋授鑲黃旗蒙古副都統。三年,命赴甘肅慶陽督辦糧臺,以病請解職回旗。七年,起授正藍旗漢軍都統,尋出為福州將軍。十年,兼署閩浙總督。十三年,日本兵船窺伺臺灣,偕總督李鶴年、船政大臣疏陳防務。光緒三年,入覲,留京授內大臣、鑲白旗漢軍都統、左都御史,擢刑部尚書。七年,協辦大學士。九年,充總管內務府大臣。十年,拜武英殿大學士,以病乞罷。尋卒,贈太子少保,謚文達。兩江總督曾國荃等奏文煜咸豐中孤軍捍賊,保全里下河,請於揚州建專祠,允之。子志顏,理籓院侍郎。

論曰:桂良以帝室葭莩,與聞軍國,數膺議和之使,無所折沖。瑞麟從僧格林沁剿賊防夷,曾著勞勩。文煜亦處兵間,無功可錄。官文雖無過人之才,推賢讓能,奠安江漢,與曾國籓、胡林翼和衷規畫,竟完戡定之功。茅土同膺,旂常並煥,豈諸人所可並語哉?


\end{pinyinscope}