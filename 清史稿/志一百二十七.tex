\article{志一百二十七}

\begin{pinyinscope}
○交通四

△郵政

海國大通以來,異域僑民,恆自設信局。咸豐十一年訂約,駐京公使郵件,初與總理衙門交驛代寄。同治五年,改由總稅務司匯各駐京公使文件,遞天津寄上海。光緒五年,增設封河後由天津至牛莊、煙臺、鎮江三路郵差。迄十一年,郵務愈繁,總稅務司乃於天津、鎮江、上海各稅務司處專員理之。此總稅務兼理郵遞之權輿也。

初,光緒二年,總稅務司英人赫德建議創辦郵政。四年,始設送信官局於北京、天津、煙臺、牛莊,以赫德主其事,九江、鎮江亦繼設局。是為中國試辦郵政之始。十六年,命通商口岸推廣舉辦。十九年,北洋大臣李鴻章、南洋大臣劉坤一以各國增設各地信局,妨推廣之路,請速籌善策。總署付赫德議。

二十一年十二月,署南洋大臣張之洞疏請舉辦郵政。略言:「泰西各國視郵政重同鐵路,特設郵政大臣綜理。取資甚微,獲利甚鉅。即以英國而論,一歲所收之費,當中銀三四千萬兩。各國通行,莫不視為巨帑。且權操於上,有所統一,利商利民,而即以利國。近來英、法、美、德、日本先後在上海設立彼國郵局,其餘各口岸亦於領事署內兼設郵局,侵我大權,攘我大利,實背萬國通例。光緒十一年間,前浙江寧紹臺道薛福成據委員李奎條陳,請中國自行設局,以挽利權,並經稅務司葛顯禮前往香港、日本,向彼國商議,收回上海所設英、日兩國郵局,已有端倪。南洋大臣曾國荃曾據咨總理衙門,飭總稅務司赫德議復辦法。赫德亦謂此舉為裕國便民大政,陳有要端七事。並稱須有奏準飭辦之明文,使各國皆知系中國國家所設,即可商令各國將在中國所設之郵局撤回,並可商入萬國信會之舉。查各關試辦郵遞有年,未能推行及遠。外國所設信局,並未裁撤。良由稅關所辦郵遞,與國家所設,體制不同,故推廣每多窒礙。現復與葛顯禮面加籌議,知其情形熟悉,各關稅務司熟諳辦法者當不乏人。請飭總理衙門,轉飭赫德,妥議章程開辦。即推行沿江沿海各省,兼及內地水陸各路。務令各國將所設信局全撤,並與各國聯會,彼此傳遞文函,互相聯絡。如果認真舉行,各國在華所設信局必肯裁撤。此各國通行之辦法,有利無弊,誠理財之大端,便民之要政也。」

總理衙門疏言:「光緒二年間,赫德因議滇案,請設送信官局,為郵政發端之始。四年,擬開設京城、天津、煙臺、牛莊、上海五處,略仿泰西郵政辦法,交赫德管理。嗣因各國紛紛在上海暨各口設立郵局,慮占華民生計。九年,德國使臣巴蘭德來,請派員赴會。十一年,曾國荃咨稱州同李圭條陳郵政利益,並據寧海關稅務司葛顯禮申稱,香港英監督有原將上海英局改歸華關自辦之議。十六年三月,劄行赫德,以所擬辦法無損民局,即就通商各口推廣辦理。擬俟辦有規模,再行請旨定設。此各稅關試辦郵遞之始也。十八年冬,赫德以數年來創辦艱難,若再不奏請設立郵政局,恐將另生枝節。十九年五月,李鴻章、劉坤一稱江海關道聶緝椝稟稱,上海英、美工部局現議增設各口信局,異日中國再議推廣,必更維艱。考泰西郵政,自乾隆初年普國始議代民經理,統以大臣,位齊卿貳。各國以為上下交通,爭相仿效。葛顯禮呈送萬國郵政條例,聯約者六十餘國。大端以先購圖記紙,黏貼信面,送局以抵信資,其費每封口信重五錢者,取銀四分,道遠酌加。其取資既微,又有定期。百貨騰跌,萬里起居,隨時徑達。如有事時,並可查禁敵國私函。誠如張之洞所稱『權有統一,為利商利民即以利國』之要政也。溯自十八年以來,美國一國郵局清單一紙,所收銀圓至六十四兆二十萬九千四百九十圓之多。張之洞所舉英國收數當中銀三四千萬兩,尚系約略之辭。利侔鐵路,誠為不虛。且西國郵政與電局相輔,以火車輪船為遞送。近來法國設立公司輪船十艘,通名信船,遇口停泊,信包未到,不能開碇,其鄭重如此。中國工商旅居新舊金山、檀香山、新嘉坡、檳榔嶼、古巴、秘魯者,不下數百萬人,往往有一紙家書十年不達者,緣郵會有扣阻無約國文函之例也。中國郵政若行,即以獲資置備輪船出洋,藉遞信以流通商貨。其挽回利權,所關尤鉅。臣等博訪周諮,知為當務之急。爰於十九年劄飭赫德詳加討論。上年六月至十二月,復與總稅務司面商屢次,先後據其遞到四項章程,計四十四款。臣等詳加披閱,大致釐然,自應及時開辦。應請旨敕下臣衙門,轉飭總稅務司赫德專司其事,仍由臣衙門總其成,即照赫德所擬章程,定期開辦。應制單紙,亦由赫德一手經理。遇有應行酌改增添之處,隨時呈由臣衙門核定,務期有利無弊。至赫德呈內稱萬國聯約郵政公會,系在瑞士國,應備照會,寄由出使大臣轉交其國執政大臣,為入會之據。自可援萬國通例,轉告各國,將所設信局一律撤回。以上所議,如蒙俞允,即由臣衙門欽遵分別咨照劄飭辦理。俟辦有頭緒,即推行內地水陸各路,剋期興辦。並咨行沿江沿海及內地各直省將軍督撫知照,屆期即將簡要辦法,飭地方州縣曉諭商民,咸知利便。凡有民局,仍舊開設,不奪小民之利。並準赴官局報明領單,照章幫同遞送,期與各電局相為表裏。其江海輪船及將來鐵路所通處所,應如何交寄文信,由總稅務司與各局員會商辦理。官郵政局歲入暨開支款目,由總稅務司按結申報,臣衙門匯核奏報。」奉旨:如所議行。此開辦郵政之始末也。自是遍通全國,上下交受其利。

其郵政區域,北部東起朝鮮、渤海,西訖新疆、青海,北起西比利亞、蒙古,南訖江蘇、湖北、四川,而盛京、吉林、黑龍江、直隸、山東、山西、河南、陜西、甘肅括焉。中部東起浙江、福建,西訖西藏、雲南,北起安徽、陜西、河南、甘肅,南訖廣東、廣西、雲南,而江西、湖北、湖南、四川、貴州括焉。東部即長江下游,東起黃海,西訖湖北、江西,北起山東、河南,南訖福建,而江蘇、安徽、浙江括焉。南部東起臺灣,西訖緬甸,北起江西、貴州、湖南、四川,南訖越南,而福建、浙江、廣東、廣西、雲南括焉。

其郵局,則總局、副總局、分局、支局、代辦處,總計六千二百又一。其郵路里數,則郵差郵路、民船郵路、輪船郵路、火車郵路,總計三十八萬一千里。每面積百里,通郵線路七里又四九。其郵件,則通常、特種,總計三萬萬六千二百二十一萬六千二百三十九。其包裹,則通常、特種,總計件數三百零二萬二千八百七十二,重量一千零六萬零四百三十三啟羅。其匯兌,則旱匯局、火匯局,總計七百五十八,匯入銀數三百九十三萬六千兩,兌出銀數三百九十八萬四千二百兩,總計銀數七百九十二萬零二百兩。歲入經常二百五十二萬八千五百餘兩,臨時六百八十三萬五千八百餘兩。歲出經常二百八十二萬七千八百餘兩,臨時六百四十六萬六千五百餘兩。出入兩抵,實盈六萬九千九百餘兩。此據宣統三年統計也。

其各國郵局設於中國各口岸者,英國則上海、天津、漢口、煙臺、福州、廈門、廣州、汕頭、寧波九處。德國則上海、北京、天津、漢口、煙臺、福州、廈門、廣州、汕頭、南京、濟南、青島、宜昌、鎮江十四處。法國則上海、北京、天津、漢口、煙臺、福州、廈門、廣州、寧波、重慶、瓊州、北海、龍州、蒙自十四處。日本國則上海、北京、天津、漢口、煙臺、福州、廈門、廣州、汕頭、重慶、南京、牛莊、唐沽、沙市、蘇州、杭州十六處。美國則上海一處。俄國則上海、北京、天津、漢口、煙臺五處。此其大略也。


\end{pinyinscope}