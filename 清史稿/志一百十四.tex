\article{志一百十四}

\begin{pinyinscope}
○兵十

△訓練

清代訓練軍士,綜京、外水陸各營,咸有成規。而歷朝整軍經武之諭,則隨時訓練,因地制宜。茲分述之。

其定期訓練者,為領侍衛府三旗親軍訓練之制,鑲黃旗、正黃旗、正白旗每月分期習騎射二次,習步射四次。八旗驍騎營訓練之制,每月分期習射六次,都統以下各官親督之。春秋二季,擐甲習步射,由本旗定期。擐甲習騎射,由部臣定期。春月分操二次,合操一次,秋月合操二次,預奏操期。仲春孟秋,登城操習,兵部稽察之。歲以為常。八旗漢兵訓練之制,於春秋月試砲於盧溝穚,各旗咸出砲十位演放,五日而畢。越三年,鳥槍營兵與砲兵合演槍砲藤牌於盧溝橋。其春秋季常操,四旗合操四次,八旗合操二次,初冬則分遣各旗演習步圍。前鋒營訓練之制,月習步射六次,春秋擐甲習騎射二次,左右翼各分前鋒之半,兼習鳥槍,月習十次,均由統領督率。每年秋季,前鋒統領會同護軍統領奏聞,率所屬兵演習步圍二、三次。護軍營訓練之制,月習步射六次,春秋擐甲習騎射二次,與前鋒同。圓明園八旗護軍營訓練之制,月習步射六次,春秋習騎射,兼習鳥槍。步軍營及巡捕營訓練之制,八旗步軍習步射,城門驍騎習鳥槍,均以春秋操演。內九門,外七門,咸設砲位。每屆三年,隨同八旗兵運砲至盧溝橋演放。巡捕營參將、游擊,月考其屬之弓矢,守備等各練其汛兵。春秋兼習鳥槍,與城門驍騎同。內府三旗訓練之制,月習步射六次,春秋擐甲習射二次,立冬後,內府護軍及尚虞處執事等演習步圍,別選三旗護軍習馬射各技。火器營訓練之制,月習步射六次,騎射六次,馬上技藝六次。統轄鳥槍砲兵護軍驍騎各官,按日於本旗考驗。至合操之日,八旗分左右翼列陣,環施槍砲。秋季至盧溝橋演砲五日。健銳營訓練之制,月習雲梯鳥槍各藝六次,騎射步射鞭刀等藝六次,餘日於本期習槍箭。值駐蹕圓明園,左右翼各以舟演習水戰。旗營校閱之時,自七月開操至次年四月,設教場於九門外,將軍、都統、副都統掌校閱騎射槍砲之事,第其優劣,以為賞罰。春秋合操,與京營同。

陸路綠旗營訓練之制,總督所屬為督標兵,巡撫所屬為撫標兵,提督所屬為提標兵,總兵所屬為鎮標兵。每歲秋季霜降日,先期各營將弁肅伍赴教場,設軍幕。屆時軍士擐甲列陣,中軍建大纛於場中,統兵大臣於將臺上傳令合操,中軍揚旗麾眾,臺下舉砲三,軍中鳴角伐鼓,步騎甲士列隊行陣,施放火槍,連環無間,如京營之制。若長矛、藤牌、扁刀、短刀之屬,各因其地之宜,以教士卒,咸有成法。閱竟,試材官將士騎射技勇,申明賞罰,犒軍,釋甲歸伍。漕運總督標、河道總督標訓練之制,咸與京營同,各營將弁率其所屬,按日督練。八旗水師營訓練之制,每年春秋二季,將軍、都統、副都統督率官兵,分駕戰艦,奉天、福建、浙江、廣東水師,各赴海口,齊齊哈爾、墨爾根、江寧省水師,各赴江面,天津水師赴海口洋面。每年自四月至八月,於潮平風順時,張帆起碇,列陣出洋,以次鳴砲操演,餘日各率所屬講習水務。其綠旗水師營,有內河水師、江海水師,出洋會哨,信候各省不同。每歲春秋之季,乘艦列陣,揚帆駛風,鳴角發砲,操演咸如軍律。

其隨時訓練者,天聰七年,太宗始舉大閱之典。八旗護軍、漢軍馬步、滿洲步軍咸集。分八旗為左右翼,漢軍、滿洲步兵為二營,四方環立,前設紅衣砲三十位。上擐甲乘馬,諸貝勒率護軍如對嚴敵,親軍為後盾。傳令聞砲而進,聞蒙古角聲而退。次漢軍馬步,次滿洲步軍,進攻砲軍。大閱禮成。嚴申退後之令。崇德八年,大閱於沈陽北郊,前列漢軍砲手,次滿洲步兵、蒙古步兵,次騎兵,次守城應援兵,次守城砲兵,綿亙二十里,聞砲合戰。上親臨簡閱,步伐止齊,軍容整肅。

順治七年,誡各將領勿以太平而忘武備,弓馬務造精良。十一年,定每年閱操賞銀之制。定騎射各兵分期演習之制。定督、撫、提、鎮獎賞優等弁兵之制。

康熙十一年,令各省營伍,須武職大員巡察。嗣後各鎮臣以巡察之期上聞,不得擾累各營。十二年,以漢軍不能騎射者甚多,每旗宜增練火器。尋議八旗漢軍驍騎,每佐領下,增鳥槍兵十八名。十六年,令各營於安營駐宿之道,馳騁奔走之勞,皆須習練,不得僅拘操演成法,直省提、鎮,每歲督選標兵行圍,以習勞苦。十九年,定每年演放紅衣大砲之期。二十八年,定演砲之制。每年九月朔,八旗各運大砲十位至盧溝橋西,設槍營、砲營各一,都統率參領、佐領、散秩官、驍騎砲手咸往。工部修砲車,治火藥。日演百出,及進步連環槍砲。越十日開操。太常寺奏簡都統承祭,兵部奏簡兵部大臣驗操。各旗演砲十出,記中的之數。即於砲場合隊操演,嚴鼓而進,鳴金而止,槍砲均演九進十連環,鳴螺收陣還營。三十年,定春操之制。每旗出砲十位,火器營兵千五百名。漢軍每旗出砲十位,鳥槍兵千五百名。每佐領下之護軍鳥槍兵、護軍驍騎,每參領下之散秩官、驍騎校,及前鋒參領、護軍參領、侍衛等,更番以從。既成列,演放鳥槍,鳴螺進兵,至所指處,分兵殿後而歸。五十年,定火器營合操陣式。八旗砲兵、鳥槍兵,護軍驍騎,分立十六營。中列鑲黃、正黃二旗,次六旗,按左右翼列隊,將臺在中,兩翼各建令纛為表。每旗鳥槍護軍在前,次砲兵,次鳥槍兵,次驍騎。臺下鳴海螺者三,以次整械結隊出營。施號槍三,臺下及陣內海螺遞鳴,乃開陣演槍砲九次至十次,砲與鳥槍連環無間。

雍正四年,改定盧溝橋演槍砲為三年一次,均演一月。兵校等火藥器用,由工部預儲。五年,以滿洲夙重騎射,不可專習鳥槍而廢弓矢,有馬上槍箭熟習者,勉以優等。七年,以直隸營汛多演空槍,通飭直省將帥,令各營以鉛子演準。八年,劉汝麟建議,漢軍應習步圍。尋諭各旗兵於初冬行步圍,每旗行二、三次,統以各旗大臣,步行較獵,侍衛、打牲人等,一律學習。九年,以八旗官兵未能精整,統兵各官,擇不堪騎射者,立為一營,稍優者,別立一營。每營千人,勤加操練,化弱為強。又以兵丁重在步行,凡八旗兵給限一年習步,以日行百四十里為率,優者賞之。十年,以邊陲用兵,操演加勤,免各旗輪班值日,專習騎射長槍。十二年,定八旗漢軍驍騎演習鳥槍之制。春季二月為始,秋季八月為始,各習槍四十五日,本旗四翼仍合操二次。

乾隆四年,定旗兵合操之制。每年春季,本旗各營官兵,於本旗教場分操二次,八旗各營官兵,於鑲黃、正黃二旗教場合操一次。至秋季合各營大操,其隊伍號令,旗纛器械,均遵大閱之制。六年,議準八旗驍騎營步射由本旗定期,騎射由兵部定期。八年,令八旗漢軍至盧溝橋演放槍砲,於九月朔為始,演放一月,簡都統大臣監視,日演十出,兵部閱操之日,每旗各演百出,演畢,合操槍砲。其金鼓號令,悉如大閱之制。十年,以沿海水師,經大臣察閱,其操演多屬具文,未諳水務,通飭將軍、督、撫、提、鎮,實心訓練甄別。十四年,以旗兵習練雲梯,隨征金川有功,凱旋後,別立健銳營,雲梯兵千名為一營,統以大臣,專練雲梯、鳥槍、馬步射及鞭刀等藝,並隨侍行圍。又於昆明湖設趕繒船,以前鋒軍習水戰駕船駛風之技。是年,莽阿納上言,整頓邊省營伍章程:一、步弓均改五力以上,一、馬射與步射一式,一、馬兵騎射宜槍箭二技,一、鳥槍專練準頭,一、槍兵兼習弓矢,一、定優劣賞罰,一、預儲軍械,以固邊陲。十七年,定八旗漢軍藤牌兵之制,春季與旗兵一律操演,遇大閱及諸營合操,則守護砲位,入隊演習。三十六年,令鳥槍兵宜遵定例,於演槍時,檢回鉛子,以勵勤能。三十八年,定各營增演馬上四箭四槍之制。三十九年,以金川用兵,京城之健銳、火器二營,功績最多,令各省綠營習鳥槍兵弁,悉仿火器營進步連環之法操練,不得虛演陣式。尋定各營槍兵升補之序,以資鼓勵。四十年,令健銳營兵月習槍十二日,定三等為賞罰。四十三年,令各省習槍兵弁,仿京營火器操練之法,各總兵於巡閱時,有進步連環精熟者紀功。四十四年,令各省綠營兵習射,以五矢中三為一等。五十年,以綠營陣法,向習兩儀四象方圓等舊式,無裨實用,改仿京營陣式,由提督頒發各標鎮,如式教演。各營每月定期合操,並演九進十連環之陣。其堆撥應差兵丁,暇日一律練習。又以各省巡撫標兵,向供給使,訓練甚稀,飭各撫臣實力整理。其舊式之藤牌兵,均兼習鳥槍。五十五年,令軍機大臣會同兵部,審定演放砲位步數及懲勸之例。

嘉慶二年,罷水師冬令鳧水習藝,以恤兵艱。四年,令水營兵丁一律兼習陸戰。又令新疆屯田之兵,每營分半屯種,餘悉回營操練。令各省督撫,修理營汛墩臺。督操將、備,加力振奮。九年,令各統兵官習射以六力弓為度,習槍以迅速命中為度,申明教誡,力挽積習,不得養尊處優。十一年,令德楞泰等兵丁,以十成之一兼習長矛,其制不得逾丈。

道光元年,令各軍均習長矛步槍,不得專精馬槍。是年,楊芳上言:「兵丁於練騎射槍矛之外,加以車騎合步連環三項,融結參合,日操一隊,以五隊更番演習,六日合操為一陣。直隸額兵,抽練四成,得一萬五千三百餘人,成二百四十隊,按圖操演,以齊勇怯而節進退。」允之。二年,以廣東營伍廢弛,嚴飭撫臣,實力練習,不得多立章程。四年,罷撤梅花車砲陣式,專習部頒九進連環陣式。五年,允英和之請,以八旗圈馬四百匹,改撥巡捕營,令滿洲、蒙古馬兵演習騎射,春秋二季,步軍統領會同左右翼總兵簡閱,三年後親臨大閱,八年,令那彥成等回疆增設防兵,籌給餉糈,議定操兵章程,並於喀什噶爾防兵內,抽練二千名,伊犁滿兵亦勤習騎射,由參贊大臣及總兵督操。十五年,以山西滿、漢營伍廢弛,嚴飭閱兵大臣嚴明甄別。是年,常大淳上言,新疆、湖南、廣東、四川各營伍,日久生玩,滿營則奢靡自逸,漢營則糧額多虛。由於拔補之循私,操演之不實,以國家養兵之資,為眾人雇役之用。請飭將軍、督、撫,力除積習。遇剿匪保案,不得冒濫,以勵戎行。允之。並令各州縣額設民壯,一律充補訓練。十七年,令各省民壯,每月隨營操演,授以紀律,以輔兵力所不及。十八年,令盛京滿洲兵各勤操務,遇行圍之時,不得有雇役情弊。十九年,以四川各營,技疏膽怯,致夷匪日張,特簡大臣,督率鎮、道,親往校閱。二十二年,令天津增兵六千餘人,飭各將、備率新舊兵丁,悉加練習,首火砲,次鳥槍刀矛,輔以馬隊。遇警則各營聯合南北砲臺。命精能武員,專司稽察,講求方略。二十六年,令各州縣民壯,隨營調考刀矛雜技。三十年,令各督、撫、提、鎮,汰老弱冗濫之兵,抽練精壯,俾各營皆有選鋒勁旅。不得以工匠僕役,虛占兵糧。

咸豐元年,奕山等以伊犁及烏魯木齊二處滿洲營增練鳥槍,擬定考驗章程,並綠營一律辦理。三年,綜各省綠營額兵共六十餘萬人,除徵調之兵,所餘存營者,汰弱留強,定期分練。各省駐防旗兵亦如之。五年,令健銳、火器、圓明園八旗營,及前鋒、護軍、八旗漢軍營,飭閱兵大臣核實校令,分別勸懲。又令僧格林沁等增滿洲火器營操演陣式。十一年,以盛京、吉林、黑龍江馬隊官兵,日就疲弱,飭將軍、副都統,無論在城在屯,一體挑練,可造者多方鼓勵,貧苦者酌量周恤,遇行圍兵數不足,以餘丁隨同操演。

同治元年,以上海、寧波等海口官兵,延歐洲人訓練,令曾國籓、李鴻章、左宗棠等,酌選武員數十人,在上海、寧波習外國兵法,以副、參大員統之,學成之後,自行教練中國兵丁。又以廣東、福建營伍久弛,飭耆齡、劉長佑等於旗、綠營營內,擇驍勇員弁,習外國兵法。天津練軍亦如之。其內地營兵,仍遵舊章,隨時訓練。是年,令文煜等定京營綠旗兵槍隊砲車合陣之制。四年,醇郡王等訓練神機營兵及練兵三萬餘人,操演漸著成效,綠營亦就整肅。令仍隸醇郡王節制,督操閱兵大臣,一並閱看。是年,令崇厚率洋槍隊千五百人赴畿南,飭天津鎮、蘆臺鎮選擇標兵,增練新式洋槍。六年,以丁寶楨所擬訓練馬隊章程十四條,飭特普欽於黑龍江所屬、富明阿於吉林所屬打牲人內,招募壯丁三千人,遵章速練馬隊,以剿捻匪。曾經出師回旗之員,分起訓練,入關候調。十年,曾國籓建議,用兵十餘年,綠營幾同虛設。查閱江南營伍,約有四宗:曰經制綠營,曰新設水師,曰挑練新兵,曰留防勇營。凡陸兵四十一營,水師十一營,新兵十一營,防勇十二營,兵數實存二萬四千餘人。舊習宜改者,約有四端:一、兵丁應差與操演分為二事,應差以分塘分汛為額,操演以分營分哨為額。一、綠營餉薄兵疲,宜仿新軍練軍之制,裁兵加餉。一、舊用鳥槍土藥,不利戰陣,各營宜以次悉改洋槍。一、水師不得仍沿馬兵、戰兵、守兵之名,各省水師,皆應籌造船之費,以船為家,但兼陸操,不得居陸,外海、內洋、里河水師,器械船隻,力求精整。凡此皆事關全局,請特旨通行內外臣工,合議遵行。是年,令長江水師,及外海、內洋、里河水師,均應專習槍砲,不得藉口演習弓矢,致開陸居之漸。沿海兵輪水師,亦免習弓矢。十二年,沈葆楨以各兵輪雖分駐各省,而操演徵調必應聲勢聯絡,請飭兵輪統領,躬歷各海口,隨時調操。十三年,李鴻章以八旗、綠營兵,用弓矢刀矛抬槍鳥槍舊法訓練,固難制勝,即新練各軍,用洋槍者已少,用後膛槍及炸砲者更少,可靖內匪,而不可禦外侮。曾國籓曾擬以新械練兵,沿海七省,共練陸兵九萬人,沿江三省,共練三萬人,計年餉八百萬兩,總理衙門議以制勝之洋槍隊練習水戰,丁日昌議合各省練精兵十萬人,皆以費重未能遽行。陸軍與水師規制各殊,訓練亦異,水師猶可陸戰,陸軍不能操舟。請以現有陸營,一律選練洋槍,裁綠營疲弱之額,加新軍之餉,沿海防營,悉改後膛槍,於海岸要口,屯大支勁旅,專講操練及築壘諸事。各海口修洋式沙土砲臺,置十餘寸口大砲,擇良將勁兵練習,以命中及遠為度,以固海疆。

光緒五年,李鴻章以德國陸軍步隊尤精,得力在每日林操,熟演料敵應變之法,夏秋大操,熟演露宿野戰攻守之法。其法備於一哨,擴而充之,可營可軍。前於海防營內,選游擊等七員,赴德國學習林操及迎敵、設伏、布陣、繪圖各法三年餘,學成回國。乃於親軍營內,挑選哨隊,仿德國一哨之制,依法教練,漸次擴充。九年,李鴻章始創設水師學堂於天津,習駕駛等藝。十一年,張之洞酌定海防各營操練章程,舊式刀叉弓矢已無實用,改用新操,一練臥槍,一練過山砲隊,一練掘造地營,一練安放水雷,一練修築砲臺,一練臨敵散隊,一練洋式火箭,一練安設行軍電線,一練疾步逾濠越嶺,一練夜戰,一練堅守地營及濬濠築墻一切工程。是年,李鴻章以外洋留學生回華,於操法、陣法、電學、水雷、旱雷,均有心得,飭分赴各營教練弁兵,並設武備學堂。十二年,張之洞以廣東省駐防營,於光緒六年,選甲兵千五百人,改練洋槍洋砲及陣法,乃裁汰旗營水師,附入步軍,編為兩翼,合陣操演。飭制造局移解新式槍砲,增練砲隊。十三年,李鴻章以北洋武備學堂學生,於砲臺、營壘、馬隊、步隊、砲隊諸新法,咸有成就,飭令回營,轉相傳授。是年,張之洞始於廣東設水師、陸師學堂,水師分管輪及駕駛攻戰二種,陸師分馬步、槍砲、營造三種,兼採各國之長。二十年,張之洞之南洋水師學堂著有成效,加以獎勵。又於江寧省設陸軍學堂,講求地理、測量、營壘諸術,馬、步、砲隊諸法。

二十一年,張之洞建議,舊營積弊太深。人皆烏合,來去無恆,一弊也。兵皆缺額,且充雜差,二弊也。里居不確,良莠不分,三弊也。攤派刻扣,四弊也。新式槍砲,拋棄損壞,五弊也。營壘工程,不知講求,六弊也。營弁習尚奢華,七弊也。若以洋將統之,期其額必足,人必壯,餉必裕,軍火必精,技藝必嫺,勇丁不供雜差,將領不得濫充,此七者練兵之必要。所聘德國武將三十五人已來華,即仿德國營制,設步隊八營,二百五十人分為五哨,馬隊二營,一百八十騎分為三哨,砲隊二營,二百人分為四哨,工程隊一營百人,醫官、槍匠等咸備。凡勇丁二千八百六十人,餉四十四萬兩。俟操練有效,推廣加練,增至萬人。以此軍洋將移練第二軍,俾次第以成勁旅。是年,胡燏芬建議,新練各軍,宜用一律槍砲。北洋先練五萬人為大軍,南洋練三萬人,廣東、湖北練二萬人,餘省萬人,操法軍械,務歸一律,以便徵調。各省應一律設立武備學堂。

二十二年,始以新法訓練海陸各軍。各省設立學堂,同時舉辦。是年,張之洞始裁撤湖北武防等三旗,改練洋操二營,工程隊一營,仿直隸武毅軍新練洋操章程,參用德國軍制,聘德國武員為教習,以開風氣。是年,盛宣懷建議,全國綠營兵歲餉千餘萬,練勇歲餉亦千餘萬,凡八十餘萬人,徒耗財力,無裨實用,宜悉行裁撤。共練新軍三十萬人,就各省情形輕重,定兵數多寡,徵募訓練,悉仿西法。旋總理衙門以各省營伍,驟難盡裁,先就北洋新練兩軍,及江南自強軍、湖北洋操隊,切實教練。俟裁兵節餉,次第推廣。飭兩江、兩湖督臣,較準制造局槍砲畫一辦理。又於武昌城設武備學堂,聘洋員教習。

二十四年,令各省稽察缺額攤派之弊,嚴行革除。至操練之法,宜不拘成格,盡力變通,飭督辦軍務王大臣議之。尋以神機營、火器營、健銳營、武勝新隊,操演嫺熟,賞統兵大臣有差。令滿、蒙、漢各軍驍騎營、兩翼前鋒、護軍營,五成改習洋操,五成改用洋槍,八旗漢軍砲隊營、藤牌營,一並改練,神機營汰弱留強,共練馬步兵萬人。其陣法器械營制餉章,酌仿泰西兵制。是年秋,上親詣團河及天津大閱新操。又令各省增水師學堂學額,增造練船,習駕駛諸術。二十五年,以北洋各軍訓練三年,飭統兵大臣取各種操法,繪圖貼說以聞。步隊以起伏分合為主。砲隊以攻堅挫銳為期。馬隊以出奇馳驟為能。工程隊以擴地利、備軍資為事。以平時操練之法,備異日戰陣之需。二十六年,鄧華熙於安徽省城設立武備學堂,習槍砲戰陣諸學。

二十七年,以各省制兵防勇,積弊甚深,飭將軍、督、撫,就原有各營,嚴行裁汰,精選若干營,分為常備、續備、巡警等軍,更定餉章,一律操習新式槍砲。又令南北洋、湖北之武備學堂,山東之隨營學堂,酌量擴充,認真訓練。是年,劉坤一、張之洞等,以二十年來,各省練習洋操,屢經整頓,而舊日將領,於新操多未諳習。東西各國教將練兵要旨,約有十二:一曰教士以禮,使知有恥自重,一曰調護士卒起處飲食,一曰講明槍砲彈藥質性源流之法,一曰槍砲線路取準之法,一曰掘濠築壘避槍砲之法,一曰馬步砲各隊擇地借勢之法,一曰測量繪圖之法,一曰隊伍分合轉變之法,一曰守衛偵探之法,一曰行軍工程制造之法,一曰籌備行軍衣糧輜重之法,一曰行軍醫藥之法。各疆臣均應選擇統領、營、哨各官,均切實研究。練兵固亟,練將尤要。數年以後,非武備學堂出身者,不得充將弁。更請仿英、法之總營務處,日本之參謀部,於都城專設衙門,掌全國水陸兵制、餉章、地理繪圖、操練法式、儲備糧餉、轉運舟車、外交偵探等事。平日之預籌,臨時之調度,悉以此官掌之。兼採眾長,務求實用。令內外臣工合議。二十八年,設北洋行營將弁學堂,實演戰擊諸法。此歷朝訓練之規也。


\end{pinyinscope}