\article{志一百十五}

\begin{pinyinscope}
○兵十一

△制造

清代以弧矢定天下,而威遠攻堅,亦資火器。故京營有火器營鳥槍兵之制,屢命各省防軍參用槍砲。初皆前膛舊制,繼購歐洲新器。其後始命各省設局制造。制造之事,實始天津。當咸、同間,中原未靖,李鴻章疏請在天津設機器局,自造槍砲,以供北方軍隊之用。同時,江蘇亦創立機器局。

同治四年,江蘇巡撫李鴻章疏言,統軍在江南剿賊,習見西洋火器之精,乃棄習用之抬槍、鳥槍,而改為洋槍隊。留防各軍五萬餘人,約有洋槍四萬枝,銅帽月須千餘萬顆,粗細洋火藥十數萬斤,均在香港、上海購買。又開花砲四營,每砲一具,重者千餘斤,輕亦數百斤,砲具精堅,藥彈繁重。惟器械子彈皆系洋式,所用銅鐵木煤各項,均來自外洋。必須就近設局自造,以省繁費。江蘇先設三局。嗣因丁日昌在上海購得機器鐵廠一座,將丁日昌、韓殿甲二局移並上海鐵廠。以後能移設金陵附近,濱江僻地,最為久遠之謀。五年,閩浙總督左宗棠疏言,外洋開花砲,近日督飭工匠仿造,已成三十餘尊。用尺測量,施放與西洋同其功用。十三年,船政大臣沈葆楨疏請飭沿江海各省,仿津、滬二廠,自設槍砲子藥廠局。

光緒二年,李鴻章、沈葆楨、丁日昌疏請選派制造學生十四人,制造藝徒四人,由出洋監督帶赴法國學習制造。此項學生,既宜另延學堂教習課讀,以培植根本,又宜赴廠習藝,以明理法,俾兼程並進,以收速效,備他日監工之選。其藝徒學成後,可備分廠監工之選。凡所習之藝,均須新巧,勿循舊式。如有他廠新式機器,及砲臺、兵船、營壘、礦廠,應行考訂之處,由監督酌帶生徒前往學習。山東巡撫丁寶楨疏言:「今在山東省城創立機器制造局,不用外洋工匠一人,局基設在濼口,自春及秋,將機器廠、生鐵廠、熟鐵廠、木樣廠、繪圖房,及物料庫、工料庫大小十餘座,一律告成。其火藥各廠,如提硝房、蒸硫房、煏炭房、碾炭房、碾硫房、碾硝房、合藥房、碾藥房、碎藥房、壓藥房、成粒房、篩藥房、光藥房、烘藥房、裝箱房,亦次第告竣。其各廠煙筒,高自四十尺至九十尺不等,凡大小十餘座。所買外洋機器,次第運取。俟機件煤炭各種備全,廠局告成,不逾一年,即可開工。將來如格林砲、克魯伯砲、林明登槍、馬梯尼槍,均可自造,不至受制於人,並可接濟各省,由水路轉運。即使洋商閉關,不虞坐困也。」直隸總督李鴻章、兩江總督沈葆楨、江蘇巡撫吳元炳疏言:「上海制造局自同治四年開辦,閱七年,曾請獎一次。今又閱七年,先後增造機器二百三十三座,大小銅鐵砲三百四十八尊,砲架七百八十餘座,開花實心砲彈十萬一千餘顆,各式洋槍一萬八千六百餘枝,槍彈八十餘萬顆,火藥十七萬磅,其他零件關系軍事者甚多。在事諸人,寢饋於刀鋸湯火之側,出入於硝磺毒物之間,積數年之辛苦,乃克有此成績。請優獎以資鼓勵。」

三年,湖南巡撫王文韶疏言:「近年上海、天津、江寧均有制造局,濱海固宜籌備,而內地亦應講求。湘省一年以來,先建廠,次制器,仿造洋式,規模粗具。後膛槍及開花砲子,試演均能如法,與購自外洋者並無區別。以後隨時添造,自數千斤以至萬斤大砲,或鋼或銅,均可自造。湘省向產煤鐵,攸縣、安化各處所產之鐵,與洋鐵一律受鉆。火藥一項,督匠精造,與洋火藥不相上下。自光緒元年五月開辦,至二年十月,共用二萬二千餘兩。以後每月以三千兩為度。請援津、滬二局成案,專摺奏銷。」四川總督丁寶楨疏言:「川省已設機器局,今外洋機件運到,即行開局,自造洋槍子彈等項。」

四年,總理衙門王大臣疏言:「前陳海防事宜,有簡器一條,巨砲應如何購辦,各軍洋槍應如何一律,以後應如何自行仿造,請飭疆臣切實詳議以聞。」嗣據各將軍、督、撫覆陳:「有言前膛槍穩實者,有言後膛槍靈捷者,有言線槍勝於洋槍者,有言宜勤加操練磨洗者,有言不宜多購防新出更勝者,有言宜派人赴外洋學習者,有言宜內地設局以防後患者。臣等查外洋槍砲,近時皆用後膛,名目甚多,必須擇其至精之品,一律切實辦理,庶在彼不敢售其欺,在我得以適其用。外洋軍械價值,本無成案可考,故承辦之員,視為利藪。查上海為各洋商聚集之地,多在該處交易。請以精明廉正之員,總理其事。各省有委辦軍火者,責成該員核定。如有浮冒等事,嚴行治罪。至仿造外洋軍火,李鴻章先後奏在上海、天津設局制造。丁寶楨、王文韶亦在山東、湖南二省各設局廠,不用洋人,其費最省。丁寶楨復於四川設局。以上三局,均設在內地。滬局制造槍藥,歲用銀四、五十萬兩。津局歲用銀二十餘萬兩。近據李鴻章、沈葆楨奏報,津局造後膛砲,滬局則前膛、後膛洋槍並造,既非通力合作,未必易地皆宜。請飭兩局派得力人員,隨時酌核,畫一辦理。」時廷臣有議以上海機器局款,充固本餉及賑捐者。兩江總督沈葆楨疏陳,謂機器局締造十餘年,僅恃二成洋稅,入不敷出,而南北洋所用槍砲子藥,咸取給於此。海防重要,未可停工。

五年,丁寶楨疏言,四川機器局近以恩承、童華疏請停辦,奉諭令酌度辦理,仍請設法興辦,毋令廢墮,遂復開局制造。

七年,兩江總督劉坤一疏言:「金陵制造局,於光緒六年,即飭工匠加工制造。各軍撥用洋槍,先後已及萬枝。今軍械所尚存來福前膛槍一萬三千餘枝,馬梯尼後膛槍七千餘枝,林明登後膛槍八千餘枝,細洋火藥六十五萬餘磅,洋砲火藥四十餘萬磅,棉花火藥九萬九千餘磅,銅火一千萬磅,各項銅管火十七餘萬件,又水雷應用之電線七十五車,所儲尚不為少。而上海制造局現造之洋藥及林明登槍,可隨時接濟金陵。復定購機器,增設洋火藥局,並定購前後膛槍一萬五千枝,尚不在此數內。至各處明暗砲臺所用之砲位,有上海制造局現造之一百二十磅子之鋼砲,年內可成。金陵局中所造陸營之砲,亦多可用。」是年,督辦寧古塔等處事宜吳大澂疏請吉林創辦機器局。

十一年,直隸總督李鴻章疏言:「上海、江寧、天津、廣東各機器局,大都分造砲械子藥,以供各軍操練戰守之用,尚未能仿造後膛大砲。至若三、四寸口徑後膛小砲,後膛連珠砲,為水陸軍必需之利器,應就內地已開煤鐵礦近水之處,分設造槍、造砲專廠。至克魯伯鋼砲,近來德、奧、義各國,恐純鋼不盡合用,均改造硬銅後膛小砲,融煉別有新法。日本已聘洋匠仿造。中國亦宜踵行。各國後膛槍式樣不一,新式改用連珠,或六、七響,精利無匹。日本已設廠自造,中國亦宜專造,以應各省之用。約計造槍及小砲機器皆不過數十萬金,尚不甚鉅。水師所用之魚雷、伏雷,與砲並重。各種伏雷,中國機器局多能自造。至魚雷則理法精奧,別有不傳之秘,只可向西洋訂購。天津機器局已購備試雷修雷之具,仿造則未易言也。」兩廣總督張之洞疏言:「粵省請募款開設槍、雷各局,其大砲仍歸滬、閩二廠制造。」又疏言:「省城有機器局,城西增步地方有軍火局,以器具未備,僅能制小鋼砲開花子、尋常洋火藥、白藥、水雷殼、洋火箭、修理船砲尋常機器,除火藥、火箭尚可用,其餘能成而不能精。設局十餘年,用銀數十萬,迥非津、滬、閩各局之比。今重加整頓,以機器、軍火二局,並入城西增步一局,以就水運之便,名曰制造局,仍制槍砲彈火藥等物。其修理魚雷,歸黃埔雷局。就制械而言,以槍彈與行營砲為尤要。蓋購槍可用數年,購彈不能支三月,一舉而購槍數千則易,一舉而購砲數十則難。自宜分條並舉,循序圖功。期以一年半而鑄槍砲廠成,兩年而砲臺備,庶足以御強敵。」大學士左宗棠疏言:「各省制造局廠,宜合並籌辦,以專責成。前曾疏請開徐州、穆源各礦,為鐵甲鋼砲材料。茲奉諭飭議設廠處所,若論常格,自應由兩江、閩浙籌款試辦,或委公正富紳,集股創辦,並招通曉化學之人,研求煉法,俾速出鋼鐵應用。其實礦政船砲,相為表裏。應設海防全政大臣,所有制造船砲礦廠軍火事宜,皆宜一手經理,以歸畫一。」

十二年,兩江總督曾國荃疏報金陵洋火藥局竣工。四川總督丁寶楨疏言:「川省建設制造局,已及五年。仿造洋槍,為數不下一萬五千餘枝。除接濟廣西、雲南軍營外,局中尚存後膛洋槍三千五百枝,前膛洋槍四千枝。恐不敷用,向上海洋商訂購克虜伯開花砲、格林砲各十尊,另造得用之劈山砲七十餘尊,抬槍五百枝備用。其火器彈丸銅帽等,除撥用外,尚存九萬餘斤。今加工制造,每月可得火藥七千餘斤,以資接濟。」

十三年,四川總督劉秉璋疏言:「川省機器委員曾照吉等,能用巧思,不招洋匠,自教工徒,仿造外洋槍砲,創用水輪機器,以省煤力。又於省城外設局,以水機制造火藥。數年以來,成機三部,機器一千五百九十件,洋槍一萬四千九百枝,火藥二十八萬餘斤,銅火帽一千三百七十五顆,後膛藥彈六十八萬五千五百顆,鉛子六十萬五千顆,洋砲三具,成績甚優。」兩廣總督張之洞疏言:「前以籌辦海防,購運軍火,並濟雲南、廣西軍營,而後膛槍彈需用尤多,必須購置機器,自行仿制。乃在上海洋行購運制造槍彈機器來粵。正擬設廠開辦,適廣西撫臣李秉衡,以廣西所購槍彈機器一部,運解到粵,而廣西撤防,且無力設局,請留在廣東備用。當即在省城之北石井墟地方,創立制造槍彈廠一所。所有機器大廠一座,打鐵、烘銅殼、鍋爐、造木箱、裝子藥房共五處,儲料、發料庫各一處,又有裝蠟餅紙餅火藥及工匠等房,共安設機器二副,能造毛瑟、馬梯尼、士乃得、雲者士得四種槍彈。試辦之初,每日約造二千顆。熟習之後,每日可造八千顆。目前即可開造。尚有需用鎔銅、碾銅等機器,並增建廠屋,俟次第到齊,即可舉辦。」

十五年,張之洞疏言:「廣東籌建水師、陸師學堂,並於堂外建機器廠一座,鑄鐵廠一座,煙筒一座,及儲料所、打鐵廠、工匠房、操場、演武、石堤、馬頭等,約用銀六萬兩。機器廠內有十二匹馬力汽鍋機爐全座,大小旋鐵床、削鐵床、鉆鐵機、剪鐵機共一十七架,手用器具,銅鐵鋼料,約用英金二千五百鎊。其機器在英國廠訂購之。」又疏言:「前曾由文武官紳及鹽埠各商分年捐銀八十萬,造小兵輪十號。今接續捐募三年,專為購買制造機器並建築廠屋經費。乃電詢德國柏林地方力拂機器廠,訂購新式制造連珠毛瑟槍,及造克魯伯砲、過山砲各項機器全副,其汽機馬力加大,以便槍砲兼造,鍋爐並為一廠,較為節省。旋由出使德國大臣與該廠訂造槍機器一分,每日能造新式連珠十響槍五十枝,汽機馬力一百二十匹,又造砲機器一分,每年能成克魯伯砲口徑七生的半至十二生的之過山砲五十具,又購槍尾尖刀機器全分,價共一百八十一萬七千兩。今擇定省城西北石門地方,依山臨江,輸運便利,於建廠相宜,乃即日開工起築。其槍管鋼料及煉鋼罐等,均向德國名廠購備,以期精良。他日鐵礦各山開採得法,則鋼鐵材料取給內地,次第擴充,並可接濟各省軍營也。」

十六年,湖廣總督張之洞於湖北省城初建兵工廠。是年,總理海軍事務大臣與戶部會議,以廣東槍砲廠改移湖北省,開廠後,常年經費,由湖北籌辦。旋由湖廣總督張之洞覆陳:「鄂省開廠後,督飭洋匠,悉心考求。原定造槍機器一副,每年能造新式連珠十響毛瑟槍一萬五千枝,造砲機器每年能成克魯伯七生的半至十二生的行營砲及臺砲共一百具。又應添購造槍砲藥、造白藥、造彈、造砲車、造砲架各機器。每槍一枝,隨彈五百顆,每年須成槍彈七百五十萬顆。每砲一尊,外洋向例隨帶砲彈三百顆,茲就最少之數,亦須隨彈二百顆,每年須成實心彈、開花彈各種彈共三萬顆。統計一切經費,約需銀七十五萬兩。計一年所造槍砲全分,比外洋買價所省甚多。特是鉅款難籌,此次開廠試辦,所有槍砲藥彈,每年各造一半,約需銀四十萬兩。機器今已到鄂,置閒必至鏽壞,工匠亦必練習,方能精熟。就鄂省財力自行籌措,查四川機器制造局,系奏明支用土藥稅釐,今湖北槍砲廠乃奉旨特辦,較四川制造局大小懸殊,關系尤重。請將湖北省歲入土藥稅銀二十萬兩,川鹽加價銀十萬兩,共三十萬兩,撥充槍砲廠常年經費。將來各省需用,撥款由鄂廠代造,則隨時收回價本,即可推廣多造。此次鄂省新設槍砲廠所造各械,皆系南北洋、廣東、山東、四川等省制造局所無者。至鄂廠所造克魯伯各種車砲,尤為邊防海防及陸道戰守必不可少之利器。前大學士左宗棠曾言購械外洋,以銀易鐵,實為非計,一旦有警,敵船封口,受制於人,運購均無從下手。況陸續遠購之器,種式既殊,彈碼亦異,每至誤事。懲前毖後,則建廠自造,乃未雨綢繆之計也。」是年,兵工廠成。

十九年,直隸總督李鴻章、兩江總督劉坤一疏言:「上海機器局於光緒十五年,令道員劉麒祥辦理局務,專心創造新式槍砲,及自煉鋼料。外洋新出利器,不肯以秘法示人。其機括靈巧,猝難臆測。開辦之始,幾無端緒可尋。乃精選洋匠,博訪窮探,考索成式,參以心得,造成試驗之,有稍不如法者,拆改重造。於二年之內,盡群才之力,竟造成新式槍砲,並煉就鋼料,迭次考驗,與西洋所造一律精堅。」湖廣總督張之洞疏言:「湖北新建煉鐵廠告成,開煉生鐵爐一座,已煉成生熟鐵具銅碾鐵軌鐵條,均有成效。其煉西門士鋼廠,開煉時極險,北洋、上海各爐,迭有炸裂堵塞之患。鄂省此項鋼爐,飭洋匠詳考火候,向來至速須六點餘鐘出鋼,今止三點餘鐘已能煉就鋼料,成色無異洋制,足以為造砲之用。砲廠亦即開工,即以煉出之鋼,試造六生的半及七八生的克魯伯陸路車砲。若能鋼料精堅,演放有準,即可造十二生的大砲。以軍需孔急,飭工匠多煉西門士鋼,及貝色麻鋼,為制造槍砲之用。外洋陸戰,全恃連珠快砲,僅有後膛槍砲,不足以盡之,鄂廠添購制快砲機器,尤為利用也。煉鐵廠之鐵路運道,及洋匠華工,原為二爐之用。今止一爐,每年只能出鐵一萬五千餘頓,折虧甚鉅。馬鞍山煤井焦炭爐完工在即,擬以湖南省所出白煤和攙焦炭冶鍊,勉供二爐之用,始足以資周轉。」

二十年,總理衙門王大臣疏言:「軍務緊急,以趕造軍火為先務,而經費有限。以之購買外洋軍火則不足,且多須時日,以之就各省現有局廠加工制造,則軍火可倍而出之。前由戶部撥款,在吉林設立機器局,專供吉林、黑龍江二省常年操防之用。請飭吉林機器局加添工料,增造軍火,以應急需。」湖廣總督張之洞疏言:「湖北新設之漢陽鐵廠,先開生鐵大爐一座,日夜出鐵八次,共五十餘頓,以後日見進步,有每日出六七十頓者。其次乃煉熟鐵、煉貝色麻鋼、碾鐵條、制鋼軌以及錘煉烘壓各法,一時並舉。所出之鐵,雖系初煉,已與外洋相較,無甚軒輊。現在江夏馬鞍山煤井所出之煤,可作焦炭,合於煉鐵之用,已開橫穴煤巷,現擬進掘三層橫穴。外洋之大洗煤機及運煤之鐵掛線路,均已次第竣工。洋式焦炭爐十座,年內當可一律告成,足敷生鐵一爐及各廠煉鋼之用。參以湖南所產白煤油煤,即可二爐齊開。」此制造鋼鐵已有成效之情形也。

又疏言:「鐵廠之設,實兼採鐵、煉鋼、採煤三大端為一事。而開煤所費,幾與煉鐵相等,本難並入造廠煉鐵計算。開平煤礦,費至二百萬,始克成功。今鐵廠自經始至觀成,用款繁鉅,所有奏明撥用之款,早經用罄,雖以槍砲經費勻撥,不敷仍多。非原估續估之多疏漏,實因開煉以後經費,與造廠工程本系二事,必須先行籌墊一年。且事皆創舉,機局變更無常,隨時補救,增出用款,多在洋匠原擬之外,非預料所及。其增出之款,除零星雜費數十項不計外,舉其重大者數端:一、增購機爐工料,如增置十五頓大汽錘一具,增貝色麻大壓汽機一副,增造西門士爐底火泥管及造火磚機器,增改生鐵大爐架一座,爐內用磚,令與礦煤之性相合,增生鐵廠內之鐵瓦敞棚,增中西兩式洗煤機,增內地火磚焦炭爐,增鋪地鐵板,增廠內運物鐵路,增運礦煤鐵車,增爐上鐵蓋,爐外水池水溝,及四周之保險門,增銅鐵管及水箱,增化驗煤鐵大小各項器具材料,以及汽表風表水表,皆為精細貴重之件。一、增募開煉洋匠,原擬雇用八人,其餘雇用熟手之華匠百餘人應用。開煉之事,以生鐵大爐為重,中國向未煉過。若欲選用華匠,非有極聰明之人在廠精練多年,難與此選。即煉鋼各廠,亦非得專門名家之洋匠領首作工不可。若手法稍不中程度,即致變生意外,危險之至。現募到洋匠二十八人,均萬不可少,較原估八人多出二倍餘。一、添補不全機器,外洋運到之機件,沿途損缺頗多。其簡便者,由漢陽本廠自行修補二千餘件外,其重大精細機器,必須由外洋或上海洋行重行購補。或此種不甚靈動,則洋匠必另購一機以救之。或此式之爐,試煉焦炭不凈,或舊法所採之礦不多,則洋匠又思一法以損益之。曠日加工,致多糜費。一、外洋金鎊值價日昂,比初定機器時,價高過半。而改換機器,訪訂洋匠等事,日積月累,亦成鉅款。一、多用煤斤,凡鐵山煤礦,開採轉運,以及鐵廠起重運料、試鉆開井、抽水壓氣,無在不需機器,即無日不用煤斤,為數甚鉅。又生鐵大爐,購用外洋焦炭,試煉兩月,費亦不貲。各款皆原估所難周悉,加以煤井開至數十丈,已費盡人工機器之力,而煤層忽脫節中斷。外洋辦法,必仍就原處追尋,另行開井。而重開一井,非鉅款不辦。現實無此財力。若非馬鞍山煤井有成,則全恃湘煤,所費更鉅。此則時局變遷,多費用款,初非意料所及。前曾督飭局員及洋匠礦師,續估用款,以為能銷貨周轉,不致再有增加之款。乃移步換形,層折過多,加工遂致加料,費日因以費工,不特非局員所能限定,並非洋匠所能預知,多方補救,繁費滋多。今撥借各款,所餘無幾,若行銷挹注,必俟兩爐齊開,一年以後,始能流通周轉。尤須鋼鐵各料,胥臻精美,合於制造之用,方可期流通無滯。至暢銷後,尤防洋鐵有減價奪售之患。此開煉之初,必須寬籌經費,庶不致停爐待款。原擬就槍砲廠經費挹注,無如槍砲廠增設砲彈、槍彈、砲架三廠,計機器運費等,已需銀三十萬兩,建廠之費,尚不在內,勢不能全行撥用。值此廠工已竣,煉鐵已成之際,所欠者僅此籌墊之款。若鎔鋼煉鐵,因此停工,則制造槍砲,何所取資?當海防緊急之秋,而軍械缺乏,貽誤戎機,關系匪淺。今各省財力,自顧不遑,豈能協助。惟有就湖北本省各款,竭力勻撥周轉,機爐勿使停工,軍實得資接濟,庶不致功虧一簣也。」

又疏言:「前因開煉鋼鐵為造械之本,以槍砲廠經費勻撥濟用,而槍砲廠更形支絀。前辦海防所購軍械,每槍式參差,彈碼互異,及舊槍攙雜,藥彈潮濕,流弊滋多。故砲架、砲彈、槍彈三廠之設,萬不可緩。今竭力籌款,先將砲架、砲彈機器,於十八年夏間,在德國力拂廠購定制造水陸行營各種砲架機器全副,每年能成六七生的至十二生的砲架砲車一百副。購定制造克魯伯砲彈機器一副,每日能成六七生的至十二生的砲彈一百顆。其他開花彈、實心彈、群子彈、子母彈,均能自造。又購定小口徑槍彈機器一副,每日可成槍彈二萬五千顆,造銅板、造鉛條、裝藥入彈、修理器具俱全,共用銀三十萬兩有奇。又添廠屋、大小鐵梁、鐵地板、水泥、火磚各種建築工程,三廠合計共用銀十五萬八千兩。近日外洋快砲益精,即兵船八十磅至百磅之大砲,亦用機器造成。鄂廠本系制造新式連珠槍,若能兼造快砲,於軍事尤多裨益。已電詢洋廠,增購新式快砲機器及砲管各件,共價銀三萬兩有奇。其廠仍舊,俟機器到齊,即可改制,較之另起廠屋,所省經費實多。此種快砲六生的者,每分鐘可放三十出,九生的者,每分鐘可放二十餘出,洵為制勝之具也。」是年,陜西巡撫鹿傳霖疏請以甘肅省舊存制造軍火機器全具,運至陜西省城,試造槍砲子藥。

二十一年,奉天增練新軍,將軍依克唐阿遣員在山東、吉林、奉天、遼陽等處,制造銅鐵等各項砲位,華、洋各式步槍,以及砲車砲架,並購制造子彈、碾火藥、造地雷器具,暨刀矛等件,在正餉動支。山東巡撫李秉衡以山東省自設立槍砲機器局後,供給各路軍火,逐年增加制造,請增常年經費。兩江總督張之洞以前年任湖廣總督,創辦湖北漢陽煉鐵廠,及興國州、馬鞍山二處採煤,以供煉鐵之用,著有成效,請優獎在事人員。陜西巡撫張汝梅以陜西省各軍所用里明、毛瑟、中針、後膛各式洋槍,皆由他省協撥,不盡合用。咨商甘肅省撥舊存制造軍火之機器等件,運至陜西,即在省城設立機器局,試造槍砲子藥,隨時修理舊械。

兩江總督張之洞上言:「天津、江南、廣東、山東、四川原有制造局,所造軍需水陸應用各件頗多,而所成槍砲甚少。或止能造槍砲彈而不能造槍砲,或能造槍,而汽機局廠尚小,均宜量加擴充。福建船政局現有大鍋爐機器及打鐵各廠,並多諳悉機器員司工匠,若增置造槍砲機器,費省而工亦易集。如奉天為根本重地,而道遠難於接濟,宜專設一廠。陜西為中原奧區,且可以接濟西路,亦宜專設一廠。至各廠制造,大率皆宜以小口徑快槍及行營快砲為主,或槍砲並造,或槍砲分造,宜每項擇定一式,各廠統歸一律,以免參差。腹省各局,只須陸路過山小砲,即足供陸戰之用。若沿江沿海數局,並宜造船臺大快砲,每廠每年至少須出快槍五六千枝,陸路、過山二種小快砲一百餘尊,方能濟用。一面雇用洋匠,一面選派工匠赴外洋名廠學習,冀他日能擴充制造廠數處。惟各省局廠,上海、金陵二處雖各有制造局,而金陵局規模頗小,機器未備,所出槍砲無多。其設局之處,限於地勢,不能展拓,僅能擇行軍要需者酌增機器,究不能多。上海制造局雖較宏大,惟所造槍彈、砲彈、水雷、火藥及修理輪船等門類頗多,而不專一,並非專造快槍之機器,每月成槍不過百餘枝,亦無造陸路、過山二種快砲之機器。至大砲則一年或出一、二尊不等。且該局軍械,須運出吳淞江後,再轉入長江。若有兵事,敵人以戰船封口,一切轉運,立即束手。前此開局滬上,只圖取材便利,未能盡善。故沿江內地,必須添設局廠。湖北槍砲廠,因上年槍廠被火後,改造鐵料廠屋,修補機器,甚費經營。快砲所增新機,以工匠初試,未熟線路,猝難較準。今甫造快槍式樣數十枝,快砲式樣一尊,車砲二尊,均尚合用。以後所出,自可日多。惟槍機曾經火灼,敏速之力稍減。一年以內,人器相習,每年約計可造成快槍七八千枝,陸路、過山二種快砲百尊。局廠地踞上游,最為穩固。上可接濟川、湘、陜、豫,下可接濟江、皖,轉運甚便。若在江南另行擇地建造,所費至鉅。不如就湖北廠添購機器,廣為擴充,其鋼鐵即用鄂省鐵廠所煉。除鄂廠原造之數外,今每年能加出快槍一萬枝,無煙藥槍彈一千萬顆,陸路、過山二種快砲二百尊,砲彈二十萬顆。湖北向無新式藥廠,擬並造無煙藥、棕色藥、黑藥,令足敷各種槍砲之用。合計槍砲架藥彈各項機器,與外洋名廠考較,諸從節省,凡運費造廠,約需銀二百萬兩。又因湖北省鐵廠,開煤井,煉焦炭,煉各種精鋼、熟銅、熟鐵,正在緊要之際,槍砲廠則趕造五處廠屋,試造槍砲。此二廠皆經費支絀,所造軍械,非專供湖北之用,請就江南籌防局撥款協濟。」

又以「江南省制造局,自光緒十七八年,沿江各省,教案會匪紛起,深恐海上有警,當將制造局應行增制快槍快砲、新式火藥各件,籌議購機試造。迨光緒二十年,日本軍事起,各省徵調頻繁,處處調撥軍火,局中積年所造之槍砲藥彈,幾至撥發一空。自應及時擴充機器,加緊制造。近年軍械,以槍砲藥彈為先,而槍砲尤以新出快式為利。是以鄂省設廠自煉鋼料,為砲筒槍管之用。又因新式巨砲,皆用慄色火藥餅,快砲快槍皆用無煙火藥,局中自造者無多,應增置各項機器,擇要先辦。將煉鋼、制藥,及造快槍、快砲各機器數十座,向洋商定購。又購買基地,增建煉鋼廠、造慄色火藥餅廠、無煙火藥廠,及添購制造鋼料,與造火藥物料,合計用銀四十餘萬兩。其在外洋訂購之器件,與洋商籌議,令其暫行墊辦,不致稽延時日,先將各項機器運到,即可開廠制造。自光緒二十年海防戒嚴,各省防軍需用軍火甚急,而火藥子彈尤為大宗。外洋守局外之例,不肯代購。即使設法運購,而價值驟增數倍,遠涉重洋,敵船又不時邀截,至為困難。今江南制造局購機設廠,自能仿造,不待外求,自為當務之急。但局中常年經費,僅有二成洋稅數十萬兩,只能制造各項子藥,分濟南北兩洋操練防守之需。若加造新式槍砲接濟各軍,則機廠既增,工料自倍加於昔。擬於江海關常年洋稅,或洋藥稅釐,每年加撥銀二十萬兩,為擴充制造後常年工作之需」。

二十二年,成都將軍恭壽因四川省軍實不充,而防務重要,乃與駐防川省之八旗協領等量力捐廉,制造抬槍九十六枝,鳥槍四百八十枝,均用煆煉純鐵纏絲制造,堅實可恃。其舊存槍枝,一律修整,為操練之需。直隸總督王文韶以北洋機器局所造各種砲子,名目雖不同,而十生的半之子彈居多,皆系舊式,不盡合用。乃向洋商訂購洋式翻沙泥,及造彈各機器,自行仿歐西新式制造。兩江總督劉坤一考核機器局成績,於常年制造之外,煉鋼廠每年可出快砲快槍筒及槍砲槍件砲架器具等鋼料共二千二百餘頓,慄色火藥廠每年可出慄色火藥二十餘萬磅,無煙火藥廠每年可出無煙火藥六萬餘磅。所創立造槍砲新廠,購機已備,加工制造,每年可出快利新式槍一千五百枝,一百磅子之快砲六尊,四十磅子之快砲十二尊,快利新槍子一百三十餘萬顆,快砲子彈一千五百顆,大小鐵彈一萬餘顆,漸著成績。四川總督鹿傳霖以四川省機器局自光緒十二年至十七年,前督臣劉秉璋曾將在局出力人員獎勵。今又屆五年,所陸續造成機器藥彈等項,皆精良合用,增造後膛毛瑟抬槍亦頗快利。在局各員,仍行獎勵之。

直隸總督王文韶因京師練兵處王大臣以京營訓練,需用打帽抬槍一千五百枝,令北洋制造局如式制造,以應要需。乃造成邊機抬槍、中機抬槍各一枝,試放均屬靈捷合用。惟邊機抬槍分兩太重,不便施放。若用中機抬槍改造邊機,其尺寸斤兩,仍與中機抬槍一致。即令制造局按照此式,制造邊機前門大式抬槍五百枝,隨槍物件共五百分,以中機抬槍改造邊機前門小式抬槍一千枝,隨槍物件共一千分。其制造款項,由北洋作正開支。北洋制造局向有歲造荷砲子彈經費銀四萬兩,本年以此項荷彈歲費,改造後門抬槍。今練兵處需槍孔急,擬即以此款移用。

湖廣總督譚繼洵以「湖北省制造軍火,向年所造舊式抬槍、線槍、抬砲、劈山砲等項,均系前膛,不及後膛新槍砲之敏捷,擬向外洋購置機器,改造各項後膛槍砲,並制造砲彈槍彈銅殼等項。今因部臣允從奉天府丞李培元之議,令各省制造局兼造抬槍,並造內地火藥,籌度辦理。因抬槍、抬砲本中國向日制勝之具,將弁兵丁素所習練,今若改用後膛,操演易於精熟,用款不多,而日後可收大效。雖漢陽槍砲廠規模宏遠,而機器種類各有不同,若抬槍、抬砲等器,他日能制造精純,亦可為漢廠之助也」。山東巡撫李秉衡考核機器局成績,於光緒二十一年所造成各種火藥十五萬六千九百六十斤,大銅帽火七十二萬顆,開花炸子一千六百顆,炸子銅螺絲引門一千六百副,克雷力伯銅砲拉火銅管四萬四千枝,帶活架瓶砲九尊,大砲子一千四百九十顆,洋鉛彈丸一百三十九萬四百五十粒,添造各廠應用機器及熟鐵大鍋爐一具,修理各營損壞洋槍洋砲,制成各項軍火箱盒,修理槍子廠、軋銅廠房屋及大鍋爐,爐臺、烘銅爐、大煙筒、生鐵廠、保險爐、提硝房、工務廠之屋宇等,又採買硝磺銅鐵鋼鉛及華、洋各種物料,暨員匠工役薪工運腳雜費等,共支用銀六萬四千七百餘兩有奇。是年,戶部從吉林將軍長順之議,增吉林機器局制造軍火常年經費,除黑龍江軍隊領用外,其餘分給奉天防軍。

二十三年,大學士榮祿上言:「制造軍火,以煤鐵為根本。外洋購價日昂,中國各省煤鐵礦產,以山西、河南、四川、湖南為最,應令山西等疆吏籌款,從速開採,設立制造局廠,漸次擴充,以重軍需。」廷議允之。令督撫臣就地方情形認真籌辦,總期有備無患,庶足倉卒應變。是年,湖北巡撫譚繼洵以湖北省制造軍火,增置砲架、槍彈、砲彈三廠,所有機器工料之價,並改換新式快砲機器,尚需銀十四萬餘兩,即在籌捐項下撥給。

山東巡撫李秉衡上言:「山東機器局於光緒二十二年間所造軍火,共造成各種洋火藥十九萬六千餘斤,堅利遠後膛大抬槍二百十六枝,步槍六枝,大銅帽火四百四十二萬顆,粗細銅管拉火六萬二千枝,銅砲炸子二千一百顆,炸子引門二千一百副,砲子一千一百九十個,各種群子八萬四千八百個,各種後膛自來火帶藥槍子一百十六萬八千四百顆,洋鉛丸一百七十二萬一千五百粒,抬槍、抬砲、來福槍、鳥槍及裝配毛瑟槍、哈乞開司槍各種大小鉛丸一百五十九萬粒,卷筒鉛子二萬一千二百斤,並修成各營抬槍、抬砲、洋槍、洋砲,添買車床、鉆床及各項雜費,均歸戶部核銷。原有機器局,設法擴充制造,添造槍械,採購應用材料,增建廠屋,購買機器,乃於機器廠後建設洋式大槍廠一所。造槍需用銅鐵零件甚多,則熟鐵廠必須擴充,乃於舊鐵廠之後,另建洋式熟鐵大廠一所。造槍則用槍子倍多,乃於舊槍子廠之東,另建洋式槍子廠一所。槍子需銅最多,乃另建軋銅大廠一所。外洋制造廠,視鍋爐之大小,以定煙筒之高下。今造成九十五尺高之煙筒一座,七十五尺高之煙筒一座,五十五尺高之煙筒一座,鐵煙筒一座。廠基深掘五尺,煙筒基深掘八尺,均密釘排椿,上築三合土,蓋以大石板,再砌條石墻腳,則扁磚實砌,純灌灰漿,梁棟皆用外洋木之方而巨者,屋柱則生鐵鑄成,即機器常年震動,不致有鼓裂之虞。此外所增建者,軍械日富,則有存儲之區,工匠日多,則有休息之所,乃建軍火庫二十間,工匠房四十間。又建水龍房以備不虞,泥工廠以資修葺,皆不可少之工。共增廠四座,群屋八十餘間,較原廠擴充三分之二。至制造抬槍機器,外洋本無抬槍名目,故無此專用機器。嗣選通曉制造之員,與洋商參酌,定造抬槍機器,並可兼造毛瑟洋槍機器共六十餘種。此外地軸皮帶鎚鉗軸枕螺絲各種輪模刀鉆,共一百七十餘件,已陸續運解到省。俟機器及銅鐵鋼料運齊,工匠募足,即可開車制造。共用銀十二萬兩,先由籓庫及南運局籌給。」

大學士榮祿建議,通飭各省制造快槍、快砲、無煙火藥,並煉鋼鐵各項機器。海疆多事,武備為先,須通力合作,以備強敵。河南巡撫劉樹棠上言,河南機器局規模甚小,若遵榮祿所議,兼造各式軍械,財力實有未逮。豫省機器局建設於省城南門外卓屯地方。其造彈機器,已向上海信義洋行定購,在外洋加工造成,陸續運至河南,安置妥貼,開工制造槍彈火藥。其造抬槍車床,亦經運到,並訂購鋼筒五百枝。先造後膛抬槍五百桿,以資應用。本省新練之豫正全軍,一律改習洋操。又通令各州縣,籌款自練勇隊,所需槍械子藥,皆省局自造。

湖廣總督張之洞上言:「大學士榮祿議令產煤鐵各省,咸從速開採,已經設立有制造局廠省分,規模未備者,尤宜擴充,自煉鋼以迄造無煙藥彈各項機器,均須實力講求,以重軍需。所言切中機宜,亟應籌辦。湖北制造廠所造快槍、快砲,為新式最精之械。若有械無彈無藥,仍屬虛器。故既添設銅殼廠,又須添設無煙藥廠。因外洋裝配快槍、快砲,悉用無煙火藥,他項洋藥皆不合用。又槍管砲身,必須精煉之罐子鋼,方足以受無煙火藥之漲力。湖北鐵廠所煉之西門馬丁鋼,以之制他器,則已稱精良。以之制槍砲,則尚非極致。外洋罐子鋼之價值,數十倍於常鋼,非徒購運道遠也。故鋼藥二者,必須購機自造。雖物力困絀,終不敢畏難自沮,致已成之槍砲廠,有不全不備之弊。故於上年即飭局員在漢口禮和洋行議定向德國格魯森廠添購無煙火藥機,每十點鐘能出火藥三十三磅,每年約出火藥五十頓,共價德銀十三萬六千八百馬克。今機器已運至上海。上年又與禮和洋行訂購德國名廠煉罐子鋼機器全副,每日能煉罐子鋼二、三頓,鑄鋼機能鑄塊鋼,每塊重二頓,價值運保各費,共用德銀十三萬馬克,久已起運,即可到滬。至廠中侭制行營快砲,以備陸戰之用。因經費太絀,故於砲臺之大砲,未經議及。外洋新式十二生的長快砲,安置沿江砲臺,能施放有準,足禦敵艦。上年由出使德國大臣許景澄在力拂廠訂購十二生的快砲並架彈等機,共用德銀三十二萬五千馬克,機器月內可到。以上各機,皆屬無款可籌,不得不與洋商婉商墊欠,分期歸款,庶可及早舉辦。加以添購大小新式樣砲、碾銅板機、拉鋼機、壓鋼機、大汽錘以及添配最精之鋼模樣板等件,約須銀十數萬兩。再加增建廠屋,又需銀十餘萬兩。其增雇華、洋工匠常年制造工料之費,為數甚鉅,又需銀二十三萬兩。各款均無所出。如上海制造局年撥八十萬兩,嗣因添制快槍,並加撥常年工作之需,每年用款已逾百萬兩。現在湖北廠所造槍砲子彈,比津局既逾數倍,比滬局亦復加多,近又添造無煙火藥,添煉罐子鋼,添造砲臺所用十二生的大快砲,功用益廣,而常年經費僅土藥稅等三十六七萬兩,較滬局止及三分之一。惟有請加撥常年專款,符原估七十五六萬之數,庶可增料加工,使舊有各廠得盡機器之力,新增各廠早收美備之功。況近年武備最重,鄂廠調撥槍砲供給各處,為數甚多,造成槍砲,並非湖北一省之用。事關全局,滬廠、鄂廠,理無二致,軍實要需,必多為籌備也。」

二十四年,山西巡撫胡聘之以山西省向無機器制造局,亟宜籌辦。因派員赴天津向洋商定購制造槍砲各種機件,並酌建廠屋,雇集工匠,仿洋式自行制造。在省城北關外擇地建廠。因山西僻在內地,非通商口岸,凡辦料募匠等事,用費極昂,即以歸化城關稅盈餘之款撥用。各機器運到晉省,開廠興工。山東巡撫張汝梅以山東省機器局自創造至今,並未延聘西人,而內地風氣初開,其精於制造人員,實不多見。且所造全系銅鐵硝磺等火器,局員工匠,素鮮經驗,非洋匠專門之比,稍一不慎,即有損傷炸裂之虞,至難極險,與尋常差使不同,乃量予獎敘。

直隸總督裕祿以北洋之軍械共有二局,一為機器局,一為制造局。機器局所有制造火藥、毛瑟槍子銅帽、各式後膛砲彈及硝磺鏹水、雷電器具、卷銅煉鋼等機,每年能造黑色火藥七十餘萬磅,慄色火藥二十五萬餘磅,棉花火藥五萬餘磅,無煙火藥八千餘磅,毛瑟後膛槍子四百餘萬粒,銅帽火二千八百餘萬粒,鋼彈一千二百顆,大小砲子一萬四千餘顆。制造局每年能造七生的半開花砲子一萬二千顆,銅件一萬六千副,克魯伯鐵身砲車十具,銅管拉火二萬四千枝,哈乞砲子五萬餘顆,哈乞開司槍子二百十萬餘粒,云者士得槍子一百四十餘萬粒。而外洋所出軍械,日新月異。今各路軍營所用毛瑟快槍、小口毛瑟槍、格魯森五生的過山快砲、克魯伯七生的半陸路行營快砲、七生的過山快砲,頗為合用,宜次第仿造。

兩江總督劉坤一以江南省制造局之後膛抬槍,上海制造局之快利新槍,及大小砲位,均稱合用。金陵局機器無多,凡大宗軍火,胥由上海制造局供用。近年增設煉銅廠、慄色火藥廠、無煙火藥廠三處,其所制砲,有十二磅子六磅子二種快砲,與北洋所用快砲口徑相同。惟北洋之七生的快砲,湖北之三生的七快砲,南洋之六生的快砲,若購自外洋,終非久計。乃擬增換機爐,自行制備,專精仿造。所有槍砲子彈,與天津、湖北二廠咸歸一律。四川總督文光因前奉朝旨,令四川制造局漸次擴充。前督臣恭壽擬就川省原有機器局擴充制造,不必另設局廠。但機器局雖創設多年,而規制未宏,若欲廣制槍砲,殊不敷用。乃擬增置長刨床一部,小車床及壓銅機器、引長機器、齊口機器各四部,緊口機器二部,均已一律制全,靈動堅固,與購自外洋者不異。惟機器既已增加,則制造亦宜推廣,應加常年經費銀二萬兩,以備制造之需。

二十五年,湖廣總督張之洞上言:「軍實最為急需,利器必須完備,近日煉鋼造藥,尤為槍砲廠必不可少之需,無罐子鋼則槍砲不精,非無煙藥則槍砲無用。屢經奉旨,責令湖北與上海各局,趕造軍械,供京營之用。而籌款艱難,何從趕辦。前所請加撥宜昌關稅銀五萬兩,仍請照撥,俾購機建廠制造等事,徐底於成。上海制造局新增鋼藥三廠,每年加撥經費銀二十萬兩,鄂廠事同一律,舊設各廠,經費本屬不敷,新廠所需,更無從出。若從部議,不得動用關稅,則制造將無可措手。綜計新廠需款共二十餘萬兩,但能加撥宜昌稅銀五萬,當設法周轉,不使廠務停滯也。」吉林將軍延茂於吉林省機器局增置機器,並代造黑龍江鎮邊軍及靖邊新軍各營軍火。山東巡撫毓賢擴充東省機器局,增建制造新槍大廠、造槍子廠、熟鐵廠、軋銅廠、化銅廠、泥工廠、軍火庫房、水龍廠房、法藍爐房、儲器房。又造大小磚鐵煙筒鐵柵等件。黑龍江將軍恩澤上言:「黑龍江鎮邊軍,每年由練餉內提銀三萬兩充軍火經費,歸吉林機器局兼造。近年物料昂貴,實不敷用。以新編之師,操練宜勤,軍火尤為繁巨。應仿照奉天、吉林二省設局自造軍火成案,於黑龍江省城擇地設立專局,悉心制造。此項購買機器建築廠房各費,約用銀十萬餘兩,在鎮邊新軍歲需軍火經費內分年籌撥。」

是年,令各省疆臣,制造槍砲,為邊防第一要著。惟各省財力不齊,自應就原有局廠切實擴充,以備鄰近各省就近購用。又令各疆臣:「天津、上海、江寧、湖北等處,均有制造槍砲局廠,曾令督撫臣切實會商,務將所制槍砲膛口,子彈大小,各局統歸一律,以期通用。並將每年所造槍件子藥若干,據實上聞,並按季咨報戶部、神機營查核。乃為時已久,並未據報有案。槍砲為行軍要需,豈容因循延宕。」令裕祿、劉坤一、張之洞:「詳析查明各廠局所造槍砲,究系何項名目,是否業已會商,造成一律,迅即切實復陳。嗣後仍遵前旨,按年按季分別奏咨,毋得延緩。各督撫督率承辦各員,認真經理,精益求精。並將槍砲膛口子彈,彼此比較畫一,務令不差累黍,庶各省互相接濟,臨時不致缺誤。倘管理局員草率從事,虛糜經費,或演放時有炸裂等事,治以重罪。」旋經兩江總督劉坤一覆陳:「當飭滬、寧二處制造局員,將出入款項,核實勾稽,制造軍械,詳細考究。並令與天津機器局、湖北槍砲廠隨時知照,互相講求。復由上海制造局員馳赴湖北比較數次,兩局所制成槍砲子彈,格式分量,口徑大小,一律合膛,並無歧異。惟江寧制造局所造後膛抬槍,系出新創,各省槍械,均無此式。其兩磅子、一磅子後膛快砲,亦與上海局中所造一律。此外砲架、砲彈、各種槍子拉火等件,分解南北洋各軍應用。以經費有限,未能加撥擴充。該局在江寧城外,粗具規模。且居腹地形勝之區,一旦海上有事,在內地制造,接濟軍需,庶幾緩急足恃。至上海制造局,並能造各項快砲,除砲臺所用之大砲外,其所造四十磅一種,即北洋之十二生的快砲,其十二磅一種,即北洋之七生的半快砲,其六磅子一種,即北洋之五十七米裡快砲,其兩磅子一種,即湖北三生的七快砲。洋廠名稱雖殊,其尺寸大小,則不差累黍。今由上海制造局派員與天津、湖北二局逐一比試,均無參差。其快利新槍,系以舊機參用人工所造,亦頗便利。究嫌費用多而出槍少,去年飭各軍改用小口徑毛瑟快槍。本擬訂造此種槍枝及造槍彈機器,專一仿制,以歸一律。訪之上海各洋行,需款數十萬,為期且甚久,一時無此財力。遂仍用舊機,更易機簧,添配車座,訂購改造七米里之毛瑟槍枝槍彈等件,按照合同,每日可出槍十枝。俟安裝全備,即日開工,嚴定章程,按年按季上聞,以期核實。各局兼造各項快砲,均系新式,尚敷應用。至仿造小口徑毛瑟槍,僅有湖北、上海二廠,其機器一系新購專門,一系舊式更改,能力所限,每年造槍不多,各路軍營,恐難遍給。曾與直隸、湖廣督臣商酌添購造槍新機,無論在津、鄂、寧、滬何廠承造,均以款絀,未能即行擴充。南洋軍火經費,但期洋稅暢收,並竭力撙節,另款存儲,以備添置仿造小口徑毛瑟槍機器一部,能數年之內,機器購全,與湖北槍砲廠分途仿造,以期器械日精。又擬請設立工藝學堂,學習船械槍砲汽雷等各種制造,以廣人才。」是年,浙江巡撫劉樹堂向金陵軍械所撥用德國老毛瑟槍三千枝,子彈一百五十萬顆,供浙省防軍之用。

二十六年,直隸總督裕祿上言:「北洋機器局經費,每年用銀二十五萬餘兩,所造軍火,向供北洋海軍及淮、練各營操防之用。近年經費減收,而向例撥解軍火之外,又加以新練武衛等五大軍,而京師神機、虎槍等營,復時有調撥,每虞缺乏。況增募各軍,皆以快槍、快砲為利器,各項槍砲子彈,必須自行制造,始能不誤操防。因於光緒二十四年,始陸續購辦制造快砲快槍子彈及造無煙槍砲火藥等項機器,今始由外洋次第運至天津,安設入廠。並派員赴上海、江寧等處,將各局所造快槍、快砲格式,及槍子、砲彈分量,互相討論,取到江南、湖北二局所造槍砲各種子彈,詳加比較,以求畫一。所有北洋增造快槍子廠、無煙火藥廠、快砲子廠,並整頓煉鋼廠等項經費,每年至少須增用銀十五萬兩,應由部臣在各海關洋稅內加撥,以濟軍用。」

二十八年,兩江總督劉坤一以上海制造局自制之新式無煙快槍、車輪快砲協濟廣西軍營。四川總督奎俊以四川省機器局自光緒三年創建廠房,制造槍砲,五年停辦。六年奉旨復開局制造,並增修熟鐵鍋爐碾火藥各廠房,各洋火藥局,迄今二十餘年,所造軍械,成績頗多。而屋宇年久漸多朽壞,一律修造,以濟要工。上年因擴充制造,已增設繪圖委員,既經培修各廠,乃增繪圖房、白火藥房各一所。四川人心浮動,調撥威遠軍一營,常年駐守局旁,以資巡察。並建修表碼廠一所,為演試槍砲之地。閩浙總督許應騤以上年防務戒嚴,福建機器局制造槍子所需用魚子火藥,及海口砲臺所用砲位藥餅,因外洋禁售軍火,乃採購土硝硫磺,以備制造。復飭機器局,按照洋式,自造車輪快砲並快槍,共採買土硝七萬斤,硫磺一萬斤,自制成魚子洋式火藥五萬磅,各大砲藥餅六百九十三出,三磅子車輪快砲十二尊,十二磅子快砲二尊,後膛新式抬槍一百枝,修改後膛子輪快砲六尊,在海防經費內開支。

二十九年,兩江總督張之洞以滬上之制造局所有機器,七年以前所造,系林明登槍,乃外洋陳舊不用之式。兩年以前所造,系快利槍,乃制造局臆造之式,亦不甚合用。故槍械新舊湊配,出數無多,砲機亦未能完備,而歲費巨款,頗為可惜。當整頓武備之時,軍營所用槍械,宜歸一律。乃定議上海廠仿照湖北廠,改造小口徑新式毛瑟快槍。惟上海廠槍機不能全備,必須兼以人工,費工多而出槍少。近年雖增機整頓,每日止能出槍七枝,一年出二千餘枝,於武備大局無裨。其砲廠所造車輪砲,亦不甚合用,必須購新式造槍機器,每年能造五萬枝快槍者,添配新式造砲機器,每年能造大臺砲十尊,七生的半口徑快砲二百尊者,庶數年之後,足以應各省之求,而歸畫一。

江西巡撫夏★J9以江西省制造局規模狹小,擬先造快槍,向外洋定購小口徑毛瑟槍新式機器全副,每日約能出槍十五枝,彈殼機器全副,每日約能造槍彈三千顆,並向洋商酌配購機件,俾一機能造數器,以期價省而用宏。另備公用機器一副,為添配修理各廠機器之用。

閩浙總督崇善以福建省於光緒二十五年,將前所移附馬尾船廠之機器,仍移設省城水部門內,專制各砲臺砲子炸釘等項。旋於二十六年,在機器局旁擴充地基,增建槍子廠屋一座。又於二十八年,在省城西關外另設制造局,專造無煙快槍。其機器槍子二廠,自開辦至二十八年,止共用經費銀一十七萬八千餘兩,制成三磅子快砲二十四尊,與上海局所造砲同式,福字一號二號陸軍後膛快砲二尊,洋式十二磅半快砲二尊,而機簧標準,均不甚靈捷。尚有修改船廠舊式陸路快砲四尊,福強軍後膛車砲六尊,制造新式後膛抬槍一百枝,改造短柄洋槍一百枝,制造各項後膛槍子三百二十餘萬顆。其餘修理各項洋槍,制造前膛砲子彈等件,為費甚多。其機器槍子二廠,建設在水部門內人煙稠密之處,存儲軍火,大非所宜,不如西關外制造局地面寬大,不近民居。蓋制造槍砲,與制造子彈,本系一事,與其分廠而費大,不如合廠而費省。乃飭二廠一律暫行停造,歸並制造一局,將制成槍砲子彈及機件材料,妥為存儲。其員役工匠,大加裁減,每年只造各式抬槍,及各式子彈,以備操防所用。

山東巡撫周馥以山東省為海防要地,而軍隊器械不足。請向金陵制造局購新制三十七米里小快砲,湖北槍砲廠購格魯森五生的七過山快砲,並開花子彈。兩江總督張之洞以東西洋各國章程,於槍砲等件,每得新式,一律通行,其舊式軍械概行作廢。今湖北、上海二局,一律專造小口徑毛瑟快槍,乃將上海制造局所存快利槍枝悉行報廢,期軍火日精。河南巡撫張人駿以河南省機器局制造軍械,規模未備,亟應增購槍砲子彈需用銅鐵各料,並自造毛瑟快槍、無煙火藥。山東巡撫周馥以山東省機器局歷年造成各種西式槍枝火藥槍丸,今復採買外洋銅鐵各料,增造各廠機器爐房箱盒。是年,以湖北漢陽廠仿格魯森新式所造五生的三及五生的七之開花砲彈二種,又曼利亞槍彈、黎意槍彈各槍拉火,撥毅軍備用。福建機器局增造無煙火藥機器。

三十年,河南巡撫陳夔龍以河南省原有機器局,因陋就簡,未能講求新法,請增購機器十部,及一切應用物件,並購兩磅銅砲胚二十尊,四磅銅砲胚十尊,以備自行制造,逐漸開拓。兩江總督魏光燾擴充金陵機器局,仿照外洋,制造各式砲位架具、炸彈銅火,及砲臺需用各件,分設機器翻沙、鐵木、火箭各廠。

三十一年,兵部議江南、天津、山東各處機器局,並金陵洋火藥局,所有運送軍裝軍火等運費,一律報部。四川總督錫良因奉部臣議,自光緒三十年以後,所有修整廠房機器,並造成機器火藥洋槍等件,遵新章呈報戶部。山東巡撫楊士驤以山東省機器局自創設以來,所造西式各種火藥大銅帽火,各種後膛槍來福槍,各式洋鉛丸,並增各廠機器爐房,尚不敷用。又採買外洋銅鐵物料,擴充制造。河南巡撫陳夔龍擴充河南機器局,即開工制造槍砲子彈,以供軍實。是年,戶部定議,通飭各省所有機器制造局,以後如採購物料,必報部核銷。

三十二年,四川總督錫良綜核機器局成績,續造機器槍械、蜀利抬槍、利川手槍等一百有四起,火藥二萬餘斤,馬梯尼槍彈、毛瑟槍彈三十餘萬顆。湖廣總督張之洞以湖北省新增鋼藥各廠,所有經費,由兵工總局兌收。兩江總督周馥上言,上海制造局各項軍火,悉仿西式造成,分給各省,共經費二百三十八萬兩有奇,所用材料,多系洋產,工資物價,均無定例,難以常例相繩。陜西巡撫曹鴻勛以陜西省制造局陸續制給各營火藥三萬餘斤,鉛丸七千餘斤,為滿、綠各營操防之用。直隸總督袁世凱、兩江總督端方會議,令金陵機器局仿照外洋制造各式砲位車輛架具、炸彈銅火以及修配砲臺等處需用物件,分設機器翻沙、鐵木、槍子、卷銅、火藥各廠,雇募工匠,常川制造。四川總督錫良擴充川省制造槍砲所,造毛瑟槍彈,一切改良,仿造外洋九響毛瑟等槍子彈,亦能如式命中,修造機件,日益加多。是年,命政務處大臣會同部臣,嚴核各省機器槍砲局廠,五年保獎一次。

三十三年,陸軍部議建四大兵工廠,使所出軍械,日精日多,以備緩急之用。護理四川總督趙爾豐綜核機器局成績,於光緒三十二年內,共修理機器五十九起,舊式洋槍一千餘枝,新造法藍單響毛瑟槍一千四百餘枝,標刀帽火針簧一千四百二十餘起,洗把一百四十餘個,九響毛瑟槍藥彈一百零四萬二千餘顆,毛瑟槍藥殼三十三萬餘顆,單響毛瑟槍藥彈三十三萬六千顆,銅擊火八百顆,十三響馬槍彈一千二百顆,碰火二千顆,紅銅小火四十六萬顆,黃銅釘五十二萬顆,火槍八枝,洋鼓二百十二個,各項機件一萬五千十一起,已成洋火藥二萬八千一百八十五斤,均經試放合用,分別存儲。

湖廣總督張之洞創建湖北兵工廠,始於光緒十六年,經營籌度,至是年而規模始具。初辦時,每日所出七米里九口徑毛瑟快槍不過十餘枝,復經設法擴充,增置機器,以後每日可造成五十餘枝。槍彈一項,僅日造數千顆,逐漸加造至五萬餘顆。所造三生的七格魯森快砲,自開機至二十五年止,共造成六十餘尊。嗣於二十五年改造五生的七過山快砲,每年可造成六十餘尊至九十尊。開花砲彈,由五萬餘顆遞加至每年七萬餘顆。所造各項槍砲子彈,與來自外洋者無所區別。至鋼藥二種,逐年次第增設煉鋼、拉鋼各廠,所煉出鋼質,亦頗精良合用。火藥廠所造成無煙火藥,足能源源接濟,使兵工廠無誤制造子彈之用。所造軍械至三十二年年底止,共造成馬步快槍十萬一千六百九十枝,槍彈四千三百四十三萬七千九百三十一顆,各種快砲七百三十尊,前膛車砲一百三十五尊,各種開花砲彈六十三萬一千七百顆,前膛砲彈六萬零八百六十顆。辦事各員,不辭勞瘁,寒暑無間,乃能有此成績。光緒二十四年,曾加獎勵,今又及十年之久,仍匯案給獎。

安徽巡撫馮煦以安徽省所用槍彈,向年購自他省,乃以原有之造幣廠改為制造局,為自造子彈及修理槍械之用,遂購機募匠,開局興辦。四川總督錫良以上年曾派員出洋考察制造軍械事宜,即在德國名廠訂購制造小口徑毛瑟快槍及造子彈、造無煙火藥各種機器,分運到川。因舊日制造局無可展拓,乃另擇相宜之地,建築造槍廠、造槍彈廠、造無煙火藥廠,仿德國蜀赫廠新式自造。

三十四年,直隸總督楊士驤在保定省城內軍械局增建火藥庫及兵房。東三省總督徐世昌以近年東省新軍日增,乃於省城設立軍械總局,吉林、黑龍江二省各設分局,以修械司附屬之。

宣統元年,陸軍部建議,泰西各國軍械制造局廠內首領以次各官,多與我國副、協都統、參領、軍校諸秩相埒。我國制造軍械,設立學堂,將來制造人才造就日多,應仿各國成規,於各制造廠設工官以供驅使,湖廣總督陳夔龍以湖北省兵工鋼藥廠自成立以來,為軍械要需,每年經費,增銀至八十萬兩,以維局務。

二年,東三省總督錫良在奉天省垣設立軍裝制造局,選集木材鐵革各工師,分科制造,以供奉、吉、黑三省軍隊、巡警之用。

三年,吉林巡撫陳昭常以吉林省陸軍改編成鎮,設立軍械專局,附設修械司,備軍警之需。

綜舉各省制造軍械之事,同治元年,天津初造槍砲,二三年間,江蘇分設機器局於江寧、上海,共設三局。四年,並三局於上海,定名機器制造局。六年,天津擴充制造,設軍火機器局。九年,改名天津機器局。十三年,福建設機器局,自造開花砲。上海制造局仿造林明登槍。天津、上海二局,均仿造水雷。廣東設機器局、軍火局。上海、江寧二局,增槍砲子彈機。光緒二年,派學生藝徒出洋,分赴各國學習制造。湖南、山東二省,均設機器局,自造軍械,不用洋匠。三年,四川設局專造馬梯尼後膛槍。四年,津局造後膛砲。六年,江寧局造來福槍、馬梯尼槍、林明登槍。七年,上海局造砲臺鋼砲。吉林設機器局。江寧增設洋火藥局。十一年,廣東設制造局及水雷局。十三年,江寧局造田雞砲。廣東設槍彈廠。十六年,湖北設兵工廠,所造新式槍砲,為南北洋、川、廣各制造局所無,並籌備煉鐵廠及開煤礦,為制造之基。十八年,貴州設爐煉鐵。十九年,天津、上海二局,均設爐煉鐵。上海局增造新式槍砲。湖北設煉鐵廠。二十年,湖北增設砲架、砲彈、槍彈三廠。陜西運取甘肅舊存機器以備造械。二十一年,天津機器局改名總理北洋機器局。廣東造抬槍、線槍。湖北、江南二省,均增設煉鋼廠、慄色火藥廠、無煙火藥廠。陜西設機器局。二十二年,江南新廠造快利新槍。天津局購機造新式砲子。四川局造後膛毛瑟抬槍。天津局造中機、邊機前門抬槍。湖北廠以舊日之抬槍、線槍、抬砲、劈山砲,均改造後膛。山東增熟鐵廠、軋銅廠、槍子廠、大槍廠。河南局增造槍彈火藥及造抬槍機器。二十三年,湖北廠增造罐子鋼及造無煙火藥機器。二十四年,山西設制造槍砲廠。上海、天津二局,均增造快砲機器。二十五年,山東增建造槍、造彈、化銅、軋銅各廠。黑龍江設機器局。二十六年,福建增建槍子廠。天津增建快砲子廠、快槍子廠、無煙火藥廠。二十八年,江西局增造槍砲機器。二十九年,福建並造槍造藥二廠為一廠。三十年,河南局增造槍砲機器。三十三年,陸軍部議建四大兵工廠。四川設造槍廠、造彈廠、造無煙火藥廠。安徽建槍彈廠。宣統二年,奉天建軍裝制造局。三年,吉林設軍械局。各省機器局廠之設,歷時垂五十餘年,開局遍十七行省,幾經增改,漸就精良。此制造軍械之大概也。


\end{pinyinscope}