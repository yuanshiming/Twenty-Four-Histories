\article{志七十七}

\begin{pinyinscope}
○輿服一

皇帝五輅皇帝輦輿皇后輿車皇太后輿車附皇貴妃以下輿車

親王以下輿車親王福晉以下輿車京外職官輿車庶民附

命婦輿車

自虞廷[B173]繢,制創垂衣。車服之朌,式昭庸典。夏絻殷輅,文質異觀。迄乎有周,監於二代。巾車典路,司服司常。各隸專官,禮明物備。秦、漢以降,代有異同。品數彌繁,曩篇具載。明初木輅,乃用於郊,崇樸去雕,亦有足尚。清之太祖,肇起東陲,遠略是勤,戎衣在御。太宗纘服,遂定遼都。天聰六年,已命禮官考定儀衛,並因易服蔑祖之弊,鑒及金、元。國俗衣冠,一沿舊式。勿忘數典,昭示云礽。世祖入關,撫有中夏。武功耆定,文物浸昌。康、雍兩朝,續有制作。朝章國採,斯已粲然。亦越高宗,衣聞克紹。治承熙洽,鄉用儒臣。館闢三通,籀文緝𥮏。五輅之數,改符周官。參古準今,圖詳禮器。遂於乘御,增定已多。一代儀文,於斯為盛。自時厥後,上下相承,率蹈前規,尚無侈改。載稽諸制,爰志斯篇。鹵簿一門,迾從附著。光、宣之際,海、陸軍興。旗式服章,舊觀頓改。已見兵志,茲不衣復書。璽、寶、印、符,所以昭信。龍、龜、蟲、鳥,紐篆各殊。列代相沿,皆資法守。備詳定式,悉按等差。又自海通,國交最重。往來酬贈,仿制寶星。名級攸分,以榮佩戴。逮乎末季,新制漸繁,兼有爵章,行之未久。若斯之類,隱略云爾。

清初仍明舊,有玉輅、大輅、大馬輦、小馬輦之制,與香步輦並稱五輦。大朝日設於太和門東。又涼步輦、大儀轎、大轎、明轎、折合明轎,均左所掌之。冬至大祀、夏至祀方澤、並乘涼步輦,升殿日亦設於太和門東。乾隆七年,定大祀親詣行禮,均乘輿出宮,至太和門乘輦。祀畢還宮,仍備輿。八年,改大輅為金輅,大馬輦為象輅,小馬輦為革輅,香步輦為木輅,玉輅仍舊,是為五輅,鑾儀衛掌之。遇大朝會,則陳於午門外。十三年,諭定乘用五輅,自今歲南郊始。更造玉輦,改涼步輦為金輦,是為二輦。又定大儀轎為禮輿,改折合明轎為輕步輿,定大轎為步輿,是為三輿。南郊乘玉輦,北郊、太廟、社稷壇,乘金輦,朝日、夕月、耕耤以下諸祀,均乘禮輿。遇大朝會,則並陳於太和門外。行幸御輕步輿,駕出入則御步輿。皇子輿車,俟分封後始制。茲撮集禮器圖所載,其乾隆以前所定者為初制,依類附見,用備參稽。

皇帝玉輅,木質魨硃,圓蓋方軫,高一丈二尺一寸。蓋高三尺一分,青飾,銜玉圓版四。冠金圓頂一尺二寸九分,承以鏤金垂雲簷八尺一寸,貼鏤金雲版三層。青緞垂幨亦三層,繡金雲龍羽文相間。系帶四,繡金青緞為之,屬於軫。四柱高六尺七寸九分,相距各五尺六寸,繪金雲龍。門垂硃簾,四面各三。座縱八尺五寸,橫八尺四寸,環以硃闌,飾間金彩。闌內周布花毯。雲龍寶座在中,高一尺三寸,闊二尺九寸。兩輪各十有八輻,鏤花飾金。貫以軸轅二,長二丈二尺九寸五分,金龍首尾飾兩端。軫長一丈一尺一寸五分,徑八尺四寸。後建太常,十二斿,亦青緞為之,繰繡日月五星,斿繡二十八宿,裏俱繡金龍,下垂五彩流蘇。用攢竹魨硃竿,左加闟戟,右飾龍首,並綴硃旄五,垂青緌。升用納陛五級,左右闌皆魨硃金彩。駕象一,靷以硃絨紃。陳設時,行馬二承轅,亦魨硃直竿,兩端俱鉆銅。初制,玉輅尺寸與大輅同。輅上平盤、滴珠板、輪輻、輪輞、車心、軸首、及駕轅諸索制並同。惟無平盤下十有二鬲及左右八鬲之飾。輅亭前二柱,飾瀝粉貼金升龍,亭柱檻座尺寸,及門鬲明栨裝飾與亭內軟座下諸制,悉同大輅。惟屏風上雕沈香色描金雲龍五,屏後下三鬲雕木沈香色描金雲龍三,下雕雲板如其數,較大輅之制少異焉。輅頂、圓盤、天輪、輅亭前諸制,及太常旗、踏梯、行馬之類,皆與大輅同。

金輅,亦駕象一。圓蓋方軫,黃飾,銜金圓版四。黃緞垂幨三層。系帶四,亦黃緞為之,屬於軫。後建大旗,十有二斿,各繡金龍。餘如玉輅之制。按金輅之名,改由大輅。初制,大輅高一丈三尺九寸五分,廣八尺二寸五分。輅上平盤,前後車欞並雁翅及四垂如意。滴珠板下二轅,各長二丈二尺九寸有奇,俱硃魨鍍金銅龍首尾,鱗葉片裝釘。平盤下方箱,四周硃魨匡,前後十二鬲,內青地繪五彩雲鶴,左右八鬲,內上青下綠地,繪獸鳥各六。輪二,貫以軸,每輪十八輻輞,皆硃魨,抹金銅鈒花葉片裝釘。輪內車心各一,抹金銅鈒蓮花瓣輪盤裝釘。軸首左右鐵插貫之,抹金銅鈒龍頂管心裝釘。軸中纏紅絨駕轅諸索。輅亭高六尺七寸九分,四柱各長五尺八寸四分。檻座高九寸五分,前後柱戧金雲龍文,下山水。門高五尺一寸九分,廣二尺四寸九分。左右門各廣二尺二寸五分,上四周裝雕木沈香色描金香草板十二片。前左右各有鬲二扇,明栨全,皆硃魨,抹金銅鈒花葉片裝釘。鬲編黃線絳,後硃魨屏風,屏前上三鬲,雕沈香色描金雲龍五。上硃魨板,戧金雲龍一。中三鬲,沈香色描金雲龍三。下三鬲,描金雲板如其數。屏後上三鬲,硃魨戧金龍三。其次戧金雲板如其數。中三鬲,硃魨戧金龍四。其次沈香色描金雲板如其數。下三鬲,沈香色雲板亦如之。俱抹金銅鈒花葉片裝釘。亭內黃線絳編硃魨匡,軟座黃絨墜座大索四,下垂蓮花墜石,上施花毯草席,並大紅織金綺褥。硃魨坐椅一,上靠背雕沈香色描金雲龍一,下雕雲板一,硃魨福壽板一,並衣黃織金椅靠、坐褥、四圍椅裙全。周圍施黃綾帷幔,或用黃線羅。亭外青綺緣邊紅簾十扇,各用拽簾黃線絳二,黃銅圈全。輅頂並圓盤高三尺一分,鍍金銅蹲龍頂,帶仰覆蓮座,高一尺二寸九分垂攀頂黃絨索四。盤高二寸,上加硃魨。其下外四面沈香色地,描金雲青飾。輅蓋亭內貼金斗拱,承硃魨匡寶蓋,斗以八頂,黃綺冒之,名曰黃屋。中並四周繡五彩雲龍九。天輪三層,硃魨,上安雕木貼金龍耀葉板八十一片。三層間繪五彩雲,襯板數亦如之。盤下四周黃銅裝釘,上施金黃綺瀝水三層,每層摺片八十有一,間繡五彩雲龍文。四角垂青綺絡帶,各繡五彩雲升龍三。圓盤四角連輅座板用攀頂黃線圓絳,並貼金木魚。輅亭前有左右轉角闌幹二扇,後一字帶左右轉角闌幹一扇,皆硃魨。內嵌雕木貼金龍,間以五彩雲。三扇凡十二柱,各柱首雕木貼金蹲龍一,及描金五彩裝蓮花抱柱。闌內四周施花毯草席。其後樹太常旗二,黃雲緞為之,皆十有二斿。每斿內外各繡升龍一。硃魨攢竹竿二,左竿旗腰繡日月北斗,竿首用鍍金銅龍頭。右竿旗腰繡黻字,竿首用鍍金銅戟。各綴抹金銅鈴二,垂紅纓十有二,上施抹金銅寶蓋,下垂青線帉。踏梯一,木質硃魨,抹金銅鈒花葉片裝釘。行馬架二,木質硃魨,抹金銅葉片裝釘,上穿黃絨匾絳。黃布面絹里夾幰衣、油綢雨衣各一,紅油合扇梯、紅油拓叉各一。貼金銅寶瓶,並木雕貼金仰覆蓮座,雕花番草貼金象鞍、鞦轡、氈籠各二副。

象輅,服馬四,驂馬六,設游環和鈴,圓蓋方軫。高一丈一尺三寸,蓋高二尺六寸五分,紅飾銜象牙圓版四。紅緞垂幨三層,系帶四,亦紅緞為之,屬於軫。四柱高六尺四寸九分,相距各五尺八寸。座縱一丈五分,橫九尺一寸,環以硃闌。轅三,各長二丈二尺三寸,軫長一丈五分,徑九尺一寸。後建大赤,十有二斿,各繡金鳳。餘制與玉輅同。按象輅為大馬輦之改定。初制,大馬輦高一丈二尺五寸九分,廣八尺九寸五分。轅二,各長二丈五寸九分。輦上平盤、滴珠板、輪輻、輪輞、車心、軸首及駕轅諸索制,並如大輅。亦無平盤下前後十二鬲及左右八鬲之飾。輦亭高六尺四寸九分,硃魨。四柱長五尺五寸四分。檻座高如輦亭,上四周雕木沈香色描金雲板十二片,下亦如之。門高五尺九分,廣二尺四寸五分。左右門廣較減二寸。前及左右各有鬲二扇,後鬲三扇,明栨全,皆硃魨,抹金銅鈒花葉片裝釘。鬲心編黃線絳。亭內軟座,上施素毯。餘制與大輅同。輦頂並圓盤高二尺六寸五分,上下皆硃魨。輦蓋青飾,銅龍、蓮座、寶蓋、黃屋諸制悉如大輅。天輪三層亦如之。輦亭前一字闌幹一扇,後一字帶轉角闌幹一扇,左右闌幹二扇,內嵌絳環板,亦皆硃魨。四扇凡十有二柱,各柱首雕飾同大輅。闌內周布素毯草席,太常旗、踏梯、行馬、幰衣、雨衣之類亦如之。惟輅以象駕,輦以馬駕,故鞍韉、鞦轡、鈴纓之飾均備焉。

木輅,服馬二,驂馬四,設游環和鈴,圓蓋方軫。高一丈一尺六寸五分,蓋高三尺六寸一分,黑飾銜花梨圓版四。黑緞垂幨三層,系帶四,亦黑緞為之,屬於軫。四柱高六尺五分,相距各五尺一寸。座縱九尺,橫八尺八寸,環以硃闌。轅三,各長二丈一尺。軫長九尺,徑八尺八寸,後建大麾,十有二斿,各繡神武,餘俱如玉輅之制。按木輅為香步輦之改制。初制,香步輦高一丈二尺五寸,座高三尺,方廣八尺二寸五分。輦座硃魨,四周雕木五彩雲渾貼金龍板十二片,間以渾貼金仰覆蓮座,下雕木線金五彩雲板二十片。座下四轅,中二轅長三丈五尺九寸,左右轅長二丈九尺五寸有奇,皆硃魨,鍍金銅龍首尾裝釘,攀轅黃線圓絳八。輦亭高六尺五分,四柱各長五尺八寸。檻高二寸五分,亦皆硃魨,上四周雕木沈香色描金香草板十二片,抹金銅輅鈒花葉片裝釘。門較大馬輦高逾二寸,廣與之同。左右門廣二尺二寸。前左右各硃魨十字鬲二扇,雕沈香色描金雲龍板八片,下雲板如其數,俱抹金銅鈒花葉片裝釘。亭內布花毯草席,大紅織金綺褥,硃魨戧金龍坐椅一。靠背以下諸制與大小馬輦並同。輦頂並圓盤高二尺六寸有奇,鍍金銅蹲龍頂,餘制同大小馬輦。天輪制亦如之。輦亭前左右轉角闌幹二扇,後一字帶左右轉角闌幹一扇,皆硃魨,嵌雕木貼金龍,間以五彩雲。三扇凡十有二柱,各柱首雕飾與大輅同。闌內四周布花毯草席。亭內木雕渾貼金劍山一,硃魨腳踏一,黃緞衣全。踏梯一,木質硃魨,雕貼金行龍五彩雲絳環板六片,描金五彩水板十有二片,蹲龍四,皆抹金銅鈒花葉片裝釘。其幰衣、雨衣類悉同大小馬輦制。

革輅,服馬一,驂馬三,亦設游環和鈴,圓蓋方軫。高一丈一尺三寸,蓋高二尺五寸五分,泥銀飾銜圓黃革四。白緞垂幨三層,系帶四,亦白緞為之,屬於軫。四柱高五尺五寸九分,座縱一丈六尺,橫八尺三寸五分,環以硃闌。轅二,各長一丈九尺五分,軫長一丈六寸,徑八尺三寸五分。後建大白,十有二斿,各繡金虎,餘制均與玉輅同。按改小馬輦為革輅,始於乾隆八年。初制,小馬輦視大馬輦高廣皆減一尺。下二轅長一丈九尺五分。平盤、滴珠板、輪輻、輪輞諸制悉?大馬輦同。輦亭高五尺五寸九分,硃魨。四柱長五尺四寸五分。檻高一寸四分,上四周雕沈香色描金雲板十二片,下亦如之。門高五尺,廣二尺二寸五分。左右門廣較減一寸有奇。前左右各有鬲二扇,明栨全,皆硃魨,抹金銅鈒花葉片裝釘。鬲心編黃線絳。後硃魨屏風,雕沈香色描金雲龍五,及沈香色描金雲龍絳環板三,雲板數亦如之。周圍亦抹金銅鈒花葉片裝釘。亭座硃魨板上施素毯草席,紅織金綺褥。外紅簾四扇,其坐椅靠背以下諸制悉同大馬輦。輦頂並圓盤高視大馬輦減一寸。上飾鍍金銅寶珠頂。蓮座、寶蓋等飾,及天輪、輦亭前諸制,亦與大馬輦同焉。

玉輦,木質魨硃,圓蓋方座。高一丈一尺一寸,蓋高二尺,青飾、銜玉圓版四。冠金圓頂,承以鏤金垂雲。曲梁四垂,端為金雲葉。青緞垂幨二層,周為襞積。系紃四,黃絨為之,屬於座隅。四柱高五尺三寸,相距各五尺,繪雲龍。門高四尺八寸,冬施青氈門幃,夏易以硃簾,黑緞緣,四面各三。座高二尺四寸,上方七尺六寸,下方七尺七寸,綴版二層,上繪彩雲,下繪金雲,環以硃闌,高一尺六寸八分,飾間金彩。闌內周布花毯。雲龍寶座在中,高一尺三寸。左列銅鼎,右植服劍。轅四,內二轅長三丈八寸五分,外二轅長二丈九尺,金龍首尾銜兩端。升用納陛五級,左右闌皆魨硃,亦飾金彩,舁以三十六人。

金輦,圓蓋方軫。高一丈五尺,蓋高一尺九寸,飾蓋用泥金銜金圓版四。冠金圓頂。簷徑七尺一寸。黃緞垂幨二層。柱高五尺,相距各四尺九寸。門高四尺七寸五分。冬垂黃氈門幃,夏易以硃簾,黑緞緣,四面各三。座上方七尺三寸,下方七尺五寸,環以硃闌,高一尺三寸。轅四,內二轅長二丈八尺一寸,外二轅長二丈六尺一寸。舁以二十八人。餘如玉輦之制。按乾隆十三年,改涼步輦為金輦。初制,涼步輦高一丈一尺二寸,座高二尺五寸。輦座硃魨,座板並四面硃魨匡,雕木渾貼金雲板二十片,上貼金地雕五彩雲絳環板十二片,帶仰覆蓮座。下四轅,中二轅長二丈八尺五寸,左右二轅長較減二尺,皆硃魨,前後俱鍍金銅龍首尾裝釘,攀轅黃線圓絳八。輦亭高五尺五寸五分,方四尺八寸,硃魨。門高四尺七寸,廣二尺二寸。左右門廣亦如之。上四周沈香色描金香草板十二片,前左右各有鬲二扇,後鬲三扇,明栨全,皆硃魨,編以黃線絳。輦板上施花毯草席,並紅織金綺褥。硃魨戧金雲龍坐椅一,坐下四周雕木沈香色描金雲,其上靠背雕沈香色描金龍一,並五彩雲,下雕貼金雲板一片,硃魨福壽板一,並衣。亭內設雕木渾貼金劍山一,腳踏一,黃緞衣全。銅火爐及鍍金鑲嵌寶石銅爐各一,坐褥、椅裙、簾幔之類,悉與大小馬輦同。輦頂高二尺五寸,鍍金銅寶珠頂,帶仰覆蓮座,高一尺三寸二分,垂攀頂黃絨索四。頂硃魨,冒以黃氈,四角如意雲並緣絳,亦均黃氈為之。周圍施金黃綺瀝水二層,每層百二十四摺,繡雲龍,間以五彩雲文。腰繡行龍十六。或大紅羅冒頂,如意雲緣絳亦紅羅為之。四角鍍金銅云四朵。亭內寶蓋繡五龍,頂用硃魨木匡,冒以黃綺,謂之黃屋。頂心四周繡雲龍各一。輦亭四角至輦座攀頂黃線圓絳四,並貼金木魚。亭外圍紅氈面、金黃氈緣絳、絹里氈衣一副。輦亭前左右轉角闌幹二扇,後一字帶轉角闌幹二扇,皆硃魨,雕木玲瓏金地五彩妝雲板十六片。四扇凡十二柱,各柱首雕飾同大輅。闌內四周施花毯草席,踏梯一,木質硃魨,貼金五彩雲玲瓏板六片,描金水板十二片。蹲龍四,皆抹金銅鈒花葉片裝釘。又鍍金銅鉤四,金黃線圓絳,數亦如之。紅油高凳四,黃氈輓凳二,金黃布夾幰衣、金黃油綢雨衣各一。

禮輿,棻質。高六尺三寸。上為穹蓋二層,高一尺三寸。上層八角,飾金行龍。下四角,飾亦如之。冠金圓頂,承以鏤金垂雲,雜寶銜之。簷縱四尺七寸,橫三尺五寸。明黃緞垂幨二層,繡金雲龍。四柱高五尺,飾蟠龍,門端及左右闌飾雲龍,皆鏤金。內設金龍寶座,高一尺七寸,幃用明黃雲緞紗氈,各惟其時。左右啟櫺,夏用藍紗,冬用玻璃。直轅二,長一丈七尺六寸五分。大橫桿二,長九尺。小橫桿四,長二尺二寸五分。肩桿八,長五尺八寸。皆魨硃,繪金雲龍。橫鉆銅,縱加金龍首尾。舁以十六人。按禮輿為大儀轎之改定。初制,大儀轎高四尺八寸五分,頂高一尺三寸,廣二尺八寸。頂雙層,渾貼金雕九龍,雲花番草絳環,銷金龍瀝水二層,黃綾為之。魨金直竿二,前後橫竿如之。短扛四,肩扛倍之。撐竿二。轎頂蹲龍十二,金頂鈒龍文,嵌珊瑚青金松子等石。轎扛裝鍍金銅龍首尾。黃布幰衣、油綢雨衣、黃氈頂各一。

輕步輿,亦舁以十六人,木質魨硃,不施幰。蓋高三尺四寸。倚高一尺五寸八分,象牙為之。座高一尺八寸二分,縱一尺八寸三分,橫二尺二寸。踏幾高三寸,魨以金。直轅二,長一丈五尺四寸五分,加銅龍首尾。大橫桿二,長九尺一寸。小橫桿四,長二尺八寸四分。肩桿八,長五尺八寸五分,俱鉆銅。餘制與步輿同。按輕步輿之稱,改由折合明轎。初制,折合明轎,金漆雕花草獸面。廣二尺二寸,高三尺四寸。地平廣如轎身。直竿下數亦如大儀轎。裝飾、幰衣諸制並與明轎同。

步輿,亦舁以十六人,木質塗金,不施幰。蓋高三尺五寸。倚高一尺六寸五分,鏤花文。中為蟠龍座,座高一尺八寸五分,縱一尺八寸,橫二尺二寸。坐具冬施紫貂,夏以明黃妝緞。四足為虎爪螭首,圓珠承之,周繪雲龍。踏幾高三寸一分,籠以黃緞。直轅二,長一丈五尺五寸。大橫桿二,長七尺六寸,中為雙龍首相對。小橫桿四,長二尺八寸。肩桿八,長五尺六寸。餘同禮輿之制。按步輿為大轎之改稱。初制,大轎單頂硃魨,廣三尺,高五尺,貼金。頂廣視轎身較贏八寸,高八寸。銷金龍瀝水一層,黃綾為之。飾金蹲龍四。直竿下數亦如大儀轎。金頂以下諸制並同。

皇後鳳輿,木質,魨明黃,高七尺。穹蓋二重,高一尺五寸五分。上為八角,下方四隅,俱飾金鳳。冠金圓頂,鏤以雲文,雜寶銜之。簷縱五尺,橫三尺七寸六分,明黃緞垂幨,上下皆銷金鳳。四柱,高四尺七寸,皆繪金鳳。櫺四啟,網以青紃。前為雙扉,高二尺六寸,啟扉則舉櫺懸之,內魨淺紅。中設硃座,高一尺七寸。倚高一尺八寸,魨明黃,繪金鳳。坐具明黃緞繡彩鳳。前加撫式,明黃金鳳魨繪亦如之。直轅二,長一丈七尺二寸五分。大橫桿二,長八尺,中為鐵■J4金雙鳳相向。小橫桿四,長三尺。肩桿八,長五尺一寸。皆魨明黃,橫鉆以銅,縱加銅■J4金鳳首尾。舁以十六人。親蠶御之。按後妃輿車之制,改定於乾隆十四年。初制,鳳輿外並有鳳輦,柱高三尺六寸,廣五尺二寸。座高一尺八寸,周圍闌柱、絳環雕花卉,硃魨貼金飾。寶座在中,下有仙橋,座穿以藤。窗鬲編石青線,頂衣用黃結羅為之。銷金鳳瀝水二層。黃緞裏衣。外垂珠簾。直竿四,內扛倍之。短扛如內扛之數。俱硃魨。赤金頂鈒鳳文,嵌青金、珊瑚、松子等石。扛端裝金鳳首尾。紅油凳四。拓叉二。黃布幰衣、油綢雨衣各一。鳳輿制廣三尺一寸五分,柱高三尺三寸二分,門高二尺八寸,頂廣視面較贏八寸。頂樓六瓣,每瓣廣一尺五寸,共高一尺二寸。轅長一丈七尺五寸,輪高五尺,俱施黃油彩繪金鳳。赤金頂,鍍金葉片裝釘。黃素綾衣,上銷金鳳瀝水二層。

儀輿,木質,魨以明黃,高視鳳輿減一尺一寸。上為穹蓋,高六寸七分。冠金圓頂,塗金簷,縱四尺七寸。四隅系黃絨紃,屬於直轅。明黃緞垂幨。四柱,高四尺七寸。門幃紅裡,亦明黃緞為之。中設硃座,高一尺五寸,倚魨明黃,高一尺六寸,繪金鳳。坐具明黃緞,繡彩鳳。直轅二,長一丈五尺五寸。橫桿二,長七尺七寸,中為鐵■J4金雙鳳相向。肩桿四,長五尺二寸,兩端鉆銅■J4金。舁以八人。初制,儀輿廣三尺二寸,柱高三尺四寸,頂廣視面轅較贏三寸,高九寸,轅長與鳳輿同,輪較低二寸,俱施黃油。赤金頂,鍍金葉片裝釘。衣以黃雲緞為之。重簷瀝水,紅緞里。黃布幰衣、油綢雨衣、黃氈頂各一。

鳳車,木質,檗明黃,高九尺五寸。穹蓋二層,高一尺七寸,上繪八寶,八角飾以金鳳,下繪云文,四隅飾亦如之。冠金圓頂,鏤雲,雜寶銜之。簷縱四尺九寸,橫四尺。明黃緞垂幨,蓋明黃絡,四隅系紃,明黃絨為之,屬於軫。四柱,高三尺三寸,左右及後皆繪金鳳。中各啟櫺,網以青紃,門高三尺,上鏤金鳳相向。明黃緞幃,黃里。坐具亦明黃緞為之,上繡彩鳳。輪徑四尺九寸,各十有八輻。轅二,長一丈七尺五寸,兩端鉆以鐵■J4金。軫長六尺二寸。駕馬一。

儀車,木質,魨明黃,高九尺五寸。穹蓋,上圓下方,高九寸。冠銀圓頂,塗金。簷縱五尺五寸,橫四尺一寸。四隅系紃,明黃絨為之,屬於軫。明黃緞垂幨。四柱,高二尺八寸,不加繪飾,里魨淺紅。黃裏明黃緞幃。坐具亦明黃緞為之,上繡彩鳳。輪徑四尺,各十有八輻。轅二,長一丈五尺,鉆以鐵■J4銀,軫長五尺八寸,駕馬一。按初制無儀車,有大儀轎,廣二尺九寸,高四尺八寸。頂廣如儀輿。頂樓八瓣,俱施黃油。貼金雲鳳絳環,嵌五色寶石。黃綾為衣,上銷金鳳瀝水二層。直橫竿各二,短扛四,肩扛倍之,撐竿二,俱硃魨。轎頂飾金鳳十二,金頂鈒海馬文,嵌青、紅、藍三色寶石。轎扛裝鍍金銅鳳首尾。幰衣諸制與儀輿同。

皇太后輿車之制,與皇后同,惟繪繡加龍,故遂異其名曰龍輿、曰龍鳳車。乾隆十六年,皇太后六旬聖壽,皇上自暢春園躬奉慈駕入宮。皇太后御金輦,明黃緞繡壽字篆文。奉輦以二十八人。二十六年、三十六年,皇太后七旬、八旬聖壽,並御是輦,自暢春園入宮。定名曰萬壽輦。

皇貴妃翟輿,木質,魨明黃,繪繡皆金翟。橫桿中為鐵■J4銀雙翟相向,翟首■J4金。凡桿縱加銅■J4金翟首尾。肩桿四。舁以八人。餘同皇後鳳輿之制。按初制,皇貴妃輿車,有翟車、翟轎,無儀輿、儀車之稱。翟轎制廣二尺九寸,高四尺六寸。頂廣二尺五寸。頂樓六瓣。俱施金黃油,彩繪雲龍翟鳥,飾五色寶石。金黃綾衣,上銷金翟瀝水。直竿二,橫竿如之。肩扛四,撐竿二,俱硃魨。轎頂飾金翟十。純素金頂,銅事件全。黃布幰衣、油綢雨衣各一。

儀輿,木質,魨明黃。倚繪金翟。坐具繡彩翟。橫桿中為鐵■J4銀雙翟相向。翟首■J4金。餘與皇后儀輿制同。

翟車,木質,魨明黃。蓋飾金翟。左右及後均繪金翟。門亦鏤金翟相向。坐具繡彩翟。轅鉆以鐵■J4銀。餘如皇後鳳車之制。初制,翟車廣三尺一寸,柱高三尺三寸有奇,頂高一尺二寸,轅長一丈六尺六寸,輪高四尺八寸,俱施金黃油。金黃雲緞衣。重簷瀝水,紅絹里。純素金頂,鍍金銅事件全。幰衣、雨衣外,金黃氈頂一。

儀車,坐具繡彩翟。餘與皇后儀車制同。

貴妃翟輿、儀輿、儀車,皆木質,魨金黃。蓋、幨、坐具皆金黃緞,飾彩繡皆金翟。橫桿中為鐵■J4銀雙翟相向,翟首■J4金。凡桿皆縱加金翟首尾。餘俱同皇貴妃輿車之制。

妃嬪翟輿,木質,魨金黃。冠銅圓頂,塗金。直桿加銅魨金翟首尾。肩桿四,魨金。舁以八人。

儀輿,木質,魨金黃。冠銅圓頂,塗金。肩桿二。舁以四人。儀車,木質,魨金黃。冠,銅圓頂,塗金。餘如貴妃輿車制。初制,貴妃、妃、嬪車、轎,與皇貴妃同。惟車轎頂及事件俱銅質鍍金。

親王明轎一,木質,灑金,不施幰。蓋、轅、桿皆魨硃飾金。暖轎一,銀頂,金黃蓋幨,紅幃,緞、氈各惟其時。初制,親王明轎廣三尺三寸,地平廣與轎面同。俱施羊肝漆灑金,上下雕玲瓏花卉。直桿、橫桿、撐桿各二,肩桿四,俱硃魨貼金飾。紅布幰衣、油綢雨衣各一。

親王世子明轎一,制同前。暖轎一。紅蓋,金黃幨,紅幃。餘如親王。

郡王明轎一,暖轎一。紅蓋,紅幨,紅幃。餘同親王世子。初制,郡王以下、貝勒以上,俱坐明轎,八人舁之,如親王儀。輔國公以上,亦坐明轎,四人舁之。原乘馬者聽。郡王長子、貝勒明轎一,暖轎一。自貝勒以上,用輿夫八人。紅蓋,紅幨,紅幃。餘如郡王。

貝子明轎一,暖轎一。紅蓋,紅幨,青幃。餘如貝勒。

鎮國公明轎一,暖轎一。皁蓋,紅幨,皁幃。餘如貝子。

輔國公明轎一,暖轎一。青蓋,紅幨,青幃。餘如鎮國公。自輔國公以上,用輿夫四。

固倫公主暖轎一,金頂硃輪車一。皆金黃蓋,紅幃,紅緣,蓋角金黃幨。初制,固倫公主車轎蓋以金黃緞為之,蓋角垂簷皆紅緣。

和碩公主暖轎及硃輪車,紅蓋,紅幃,蓋角金黃緣。餘同固倫公主。和碩公主以下、縣主以上,輿用銀頂。並按初制,固倫公主車、轎皆紅緞為之,蓋角亦金黃緣。

郡主暖轎及硃輪車,紅蓋,紅幃,紅幨,蓋角皁緣。餘如和碩公主。初制,郡主蓋、幃與和碩公主同,惟蓋角青緣。

縣主暖轎及硃輪車,紅蓋,青幨,蓋角青緣。餘如郡主。初制,縣主蓋、幃俱同和碩公主,惟蓋角藍緣。

郡君車,紅蓋,紅幨,青幃,蓋角青緣。初制,郡君車蓋紅緞為之,藍幃,蓋角藍緣。

縣君車,皁蓋,紅幨,皁幃,蓋角紅緣。初制,縣君車蓋青緞為之,蓋角紅緣。

鎮國公女鄉君車,皁蓋,皁幃,紅幨,蓋角青緣。初制,鎮國公女鄉君車蓋、幃亦以青緞為之,蓋角藍緣。

輔國公女鄉君車,青幃,蓋去緣飾。餘如鎮國公女。郡君以下車皆硃輪。並按初制,輔國公女鄉君車青蓋、藍幃。

親王福晉暖轎及硃輪車,紅蓋,四角皁緣。金黃幨,紅幃,硃轅,輿用金頂。自親王以下、貝勒以上各側室,均降嫡一等。並按初制,親王妃車、轎紅蓋,紅幃,金黃垂幨,蓋角青緣。其側妃車、轎亦紅蓋,紅幃,蓋角青緣,紅垂幨。

親王世子福晉暖轎及硃輪車,紅幨。餘如親王福晉。初制,親王世子妃轎、車蓋、幃與親王側妃同。其側妃轎、車,紅蓋,紅幃,蓋角青緣,青垂幨。

郡王福晉暖轎及硃輪車,皁幨。餘如親王世子福晉。輿用銀頂。初制,郡王妃轎、車蓋、幃與親王世子側妃同。其側妃轎、車,紅蓋,紅幃,蓋角藍緣,藍垂幨。

郡王長子福晉暖轎及硃輪車,四角藍緣,藍幨。餘如郡王福晉。初制,郡王長子妃轎、車蓋、幃與郡王側妃同。其側妃轎、車,紅蓋,四角青緣,青幃,紅幨。

貝勒夫人暖轎及硃輪車,四角皁緣,皁幃。餘如郡王長子福晉。初制,貝勒夫人轎、車與郡王長子側妃同,其側夫人轎車,紅蓋,藍緣,藍幃,紅幨。

貝子夫人車,紅蓋,青緣,青幃,紅幨。初制,貝子夫人車與貝勒側夫人同。其側夫人車,青蓋,紅緣,青幃,紅幨。

鎮國公夫人車,硃輪,皁蓋,紅緣,皁幃。紅幨。自公夫人以上,蓋、幃均用雲緞,鎮國將軍夫人以下用素緞。並按初制,鎮國公夫人車蓋、幃與貝子側夫人同。其側夫人車,青蓋,藍緣,青幃,紅幨。

輔國公夫人車,硃輪,皁蓋,青緣,皁幃,紅幨。初制,輔國公夫人車蓋、幃與鎮國公側夫人同。其側夫車,青蓋,藍幃,紅幨。

鎮國將軍夫人車,硃輪,皁蓋,青緣,皁幃,紅幨。初制,鎮國將軍夫人車蓋、帷與輔國公側夫人同。

輔國將軍夫人車,硃輪,青蓋,紅幨,青幃。初制,輔國將軍夫人車蓋、帷皆以藍緞為之,紅垂幨。

奉國將軍淑人、奉恩將軍恭人車,均硃輪,皁蓋,皁幃,皁幨。初制,奉國將軍淑人及奉恩將軍恭人車,蓋、幃、幨皆以青緞為之。

民公夫人車,黑轅輪,綠蓋,皁緣,綠幨,皁幃。初制,公夫人車,皁蓋,青緣。

侯、伯夫人車,四角青緣。餘如民公夫人。初制,侯、伯夫人車,青幃,蓋角藍緣。

子夫人車,皁蓋。餘如侯、伯夫人。初制,子夫人車,青蓋,綠緣,綠幨,青幃。

男夫人車,皁蓋,不緣。餘如子夫人。初制,男夫人車,青蓋,青幃,綠幨。

滿洲官惟親王、郡王、大學士、尚書乘輿。貝勒、貝子、公、都統及二品文臣,非年老者不得乘輿。其餘文、武均乘馬。

漢官三品以上、京堂輿頂用銀,蓋幃用皁。在京輿夫四人,出京八人。四品以下文職,輿夫二人,輿頂用錫。直省督、撫,輿夫八人。司、道以下,教職以上,輿夫四人。雜職乘馬。

欽差官三品以上,輿夫八人。武職三品仍不得用。武職均乘馬。將軍、提督、總兵官,年逾七十不能乘馬者,奏聞請旨。初制,凡公、侯、伯以下職官,三品以上,坐四人暗轎,鍍金裝飾,銀螭,繡帶,青幔。四品以下,坐二人暗轎,或乘車,原乘馬者聽。其轎、車之制,四、五品素獅繡帶。六品以下,素雲頭素帶,青幔。漢武官有坐轎者,禁如例。

乾隆十五年諭:「本朝舊制,文、武滿、漢大臣,凡遇朝會皆乘馬,並不坐轎。從前滿洲大臣內有坐轎者,是以降旨禁止武大臣坐轎,未禁止文大臣。今聞文大臣內務求安逸,於京師至近之地,亦皆坐轎。若謂在部院行走應當坐轎,則國初部院大臣未嘗坐轎。此由平時不勤習技業,惟求安逸之所致也。滿洲大臣當思本朝舊制,遵照奉行。嗣後文大臣內年及六旬實不能乘馬者,著照常坐轎,其餘著禁止。」

庶民車,黑油,齊頭,平頂,皁幔。轎同車制。其用雲頭者禁之。

一品命婦車,黑輪、轅,皁蓋,青緣,綠幨,皁幃。

二品命婦車,皁蓋,不緣。餘同一品命婦。

三品命婦車,皁蓋,皁幃。餘同二品命婦。以上輿用銀頂。

四品命婦車,皁蓋,青幃,輿用錫頂。餘同三品命婦。

五品命婦以下車,青蓋,青幨,青幃。二品以上,蓋、幃、幨用繒,餘均用布。並按初制,內大臣、都統、大學士、尚書、左都御史命婦車,青蓋,綠緣,綠幨,青幃。散秩大臣、前鋒統領、步軍統領、副都統、侍郎、學士、副都御史、通政使司通政使、大理寺卿、詹事府詹事命婦車,青蓋,青幃,綠幨。頭等侍衛,參領,步軍總尉,王府長史,太常、太僕、光祿寺各正、少卿,通政司副使,大理寺少卿,國子監祭酒,內閣侍讀學士,翰林院讀講學士,侍讀,侍講,詹事府少詹事,庶子,諭德,洗馬,郎中,鴻臚寺卿,給事中,監察御史,輕車都尉命婦車,青蓋,青幃,青幨。閒散宗室、二等侍衛、佐領、貝勒長史、欽天監監正、內閣侍讀、國子監司業、鴻臚寺少卿、通政使司參議、詹事府中允、員外郎、步軍副尉、騎都尉命婦車,青蓋,藍幃,青幨。三等侍衛、雲騎尉、五品以下官命婦車,藍蓋,藍幃,青幨。


\end{pinyinscope}