\article{志八十六}

\begin{pinyinscope}
○選舉六

△考績

三載考績之法,昉自唐、虞。清沿明制,而品式略殊。京官曰京察,外官曰大計,吏部考功司掌之。京察以子卯午酉歲,部院司員由長官考覈,校以四格,懸「才、守、政、年」為鵠。分稱職、勤職、供職三等。列一等者,加級記名,則加考引見備外用。糾以六法,不謹、罷輭者革職,浮躁、才力不及者降調,年老、有疾者休致,註考送部。自翰、詹、科、道外,依次過堂。三品京堂由部開列事實,四、五品由王、大臣分別等第,具奏引見取上裁。大計以寅已申亥歲,先期籓、臬、道、府遞察其屬賢否,申之督、撫,督、撫覈其事狀,註考繕冊送部覆覈。才守俱優者,舉以卓異。劣者,劾以六法。不入舉劾者為平等。卓異官自知縣而上,皆引見候旨。六法處分如京察,貪酷者特參。

凡京察一等、大計卓異有定額,京官七而一,筆帖式八而一,道、府、、州、縣十五而一,佐雜、教官百三十而一,以是為率。非歷俸滿者,未及年限者,革職留任或錢糧未完者,滿官不射布靶、不諳清語者,均不得膺上考。其大較也。順治八年,京察始著為令,以六年為期。十三年,吏部奏定則例,三品以上自陳,四品等官吏部、都察院察考議奏,親定去留。筆帖式照有職官例一體考察。遇京察時,各官暫停升轉。尋復定考滿議敘例,三年考滿與六年察典並行。十七年,從左都御史魏裔介請,行糾拾之法,以補甄別所未及。唐熙元年罷京察,專用三年考滿例。三品以上仍自陳。餘官分五等:一等稱職者紀錄,二等稱職者賞賚,平常者留任,不及者降調,不稱職者革職。三年,御史季振宜請停考滿三疏,極言徇情鉆營,章奏繁擾,無裨勸懲。因停考滿自陳例。六年,復行京察。明年,甄別不及官三十七員。嗣以各部、院甄別司員,類多末職,二十三年,嚴諭指名題參,復甄汰王三省等三十六人。明年,京察又停。雍正元年復舉行,改為三年,自是為定制。

初,京察一等無定額,康熙三年,御史張沖翼疏請以部、院員數之多寡定一、二等名數,以息奔競,從之。乾隆間,部、院保送一等,或浮濫溢舊額,詔停兼部行走,仍歸本衙門另班聲敘,暨到任未滿半年,仍由原衙門註考等例。又罷未授職庶吉士保列一等之例,以示限制。四十二年,命部、院保送一等人數,毋庸過泥上屆成例,遞行裁減,以防溢額。應將上兩次數目比較,酌中定制。既無慮濫膺保薦,亦不至屈抑人才。五十年,定例保送一等人數,以不溢四十八年原額為準。後世踵行,間有增損,無甚懸殊也。向例部、院司官由吏部、都察院考覈,雍正四年,命內閣大學士同閱。乾隆九年,帝慮部、院堂官有瞻徇情面濫列一等者,敕大學士驗看,慎重甄別,不稱一等者裁去。十一年諭曰:「前命大學士分別去留,亦權宜辦理之道。察覈司員,惟堂官最為親切。要在平日留心體察,臨時舉措公平。如上次定一等者,三年中行走平常,當改為二、三等。上次原列二、三等者,三年中知所奮勉,即改為一等。庶察典肅而人知勸懲。」厥後考察權責,悉屬吏部,驗看特奉行故事而已。

大臣循例自陳求斥罷,候旨照舊供職,國初以來行之。乾隆八年,曾諭大臣自陳罷斥者舉賢自代。嗣以所舉不得其人,或樹黨營私,行不久即罷。十七年,帝以「內、外大臣親自簡擢,隨時黜陟,奚待三年?自陳繁文,相率為偽,甚無謂也」。詔罷其例。

先是京堂官無甄敘例,乾隆十五年,帝以三品以上堂官,具本自陳,部、院司員,皆令引見,而四、五品京堂不在自陳之列,亦無引見之例,吏部、都察院考語無實,龍鍾庸劣者得姑容,才具優長者無由見。特派王、大臣分別等第,奏聞引見。十八年,敕吏部開列三品京堂事實,親為裁奪。四十八年,以三品京堂不便派大臣驗看,令吏部帶額引見。嘉慶十二年,以三、四品京堂,向來京察但有降黜無甄敘,既與內、外大臣辦理兩歧,並不得與部、院司員同邀加級。於是予太常少卿色克精額等議敘,而予陳鍾琛等休致。自後三品以下京堂始有甄敘之例矣。

年老休致,例有明文。乾隆二十二年,定部、院屬官五十五歲以上,堂官詳加甄別。三十三年,改定京察二、三等留任各官,六十五歲以上引見。嘉慶三年,命京察二、三等官引見,以年逾七十為限。尋復舊例。六法處分綦嚴,長官往往博寬大之名,每屆京察,祗黜退數人,虛應故事,餘概優容,而被劾者又不免屈抑。雍正中,汪景祺、查嗣庭輩論列時政,以部員壅滯為言,有「十年不調、白首為郎」等語。帝責以怨望誹謗,而事實不得謂誣。蓋部員冗濫,康、雍時已然矣。

乾隆三年,鴻臚少卿查斯海疏言:「京官被劾,不無以嫌隙入吏議者。京察六法官,應援大計例送部引見。」從之。乾隆末,士夫習為諂諛,堂官拔識司員,率以逢迎巧捷為曉事,察典懈弛。仁宗初,銳意求治,頗思以崇實黜華,獎勵氣節,風示天下。嘉慶五年,詔部、院堂官慎重選舉,猷守兼優者膺首薦,餘寧取資格較久、謹願樸實之員,其少年浮薄、才華發越者,應令深其經練,下屆保列。尚書、侍郎各備冊密識賢否,公議同覽。十一年,大學士、尚書等議奏京察事宜:「捐納人員,限以年資,軍機處司員能兼部務者,方列上考,不許濫保充數。」報可。

道光四年,候際清贖罪舞弊一案,刑部司員恩德等朋謀撞騙堂官,以謬登薦牘,保列一等,下部議處。諭嗣後京察有冒濫徇私者連坐。七年,給事中吳傑奏:「大計、軍政,皆有舉有劾。近年六部辦理京察,除保舉一等外,不問賢否,概列二等。間有三等數人,仍予留任。六法不施,有勸無懲。應申明舊章,舉劾並用。」帝韙其言,降諭飭行。十五年,令於京察外隨時糾參,以為補救。咸豐十年,刑部堂官濫保不諳例案之員,朝廷務循寬大,輒以相習成風,不獨刑部為然,多為原宥。僅予大學士桂良等鐫級留任,出考堂官罰俸而已。穆宗即位,大難未平,厲精澄敘。同治五年,詔部、院堂官謹遵嘉慶五年備冊密識賢否、公議同覽之諭,並常川進署,與司員講求公事,藉覘其屬賢否。八年,又諭京察不得有舉無劾,冀湔滌舊習,一新庶政。然積重之勢,不能復返。光緒七年,禮部侍郎寶廷疏陳京察積弊,言之痛切,謂:「瞻徇情面之弊,不專在部、院堂官,當責樞臣考察,必公必嚴。樞臣果精白乃心,破除情面,不特能考察部、院司員之賢否,並能考察內、外大臣之賢否。而考察樞臣功過,在聖明獨斷。若朝廷先以京察為故事具文,何責乎樞臣,更何責乎部、院堂官!」論雖切中而難實行,徒託空言而已。宣統二年,吏部設立憲政籌備處,改考功司為考績科,主文職功過應行變通事宜。其時浮議紛紜,新舊雜糅,吏部等於贅疣矣。

大計始順治二年,御史張濩疏請有司殿最,宜以守己端潔、實心愛民為上考。部覆如議。明年,定朝覲考察,頒五花冊,令督、撫以四格註考。故事,計參外,臺、省例有拾遺。是歲計群吏,止據撫、按所揭為黜陟。臺、省擬循故事,內大臣不喜。大學士陳名夏力主之,給事中魏象樞亦以為請。得旨,糾拾官照大計處分挾私妄糾者論。自後臺、省意存瞻顧,糾拾者鮮。已,罷不行,而督、撫權乃日重矣。四年,定大計三年一舉,計處官不許還職。諭朝覲官曰:「貪酷重懲,闒茸罔貰。爾等姑許留任,當思祓濯前愆,勉圖後效。」嗣是每屆入覲之年,必嚴切誡飭以為常。舊例朝覲計典,籓、臬、府、州、縣正官皆入覲。順治九年,止令籓、臬各一員、各府佐一員代覲。十八年,給事中雷一龍疏言;「三年大計,勿得遺大吏而摘微員,懲去位而寬現在。請令籓、臬赴部,面同指實,按冊詳察。」下部議行。康熙元年,停籓、臬入覲,以參政、副使等官代。十二年,復令籓、臬入覲。二十五年,以朝覲藉端苛派,奸弊滋生,籓、臬、府佐入覲例悉罷。官吏賢否去留,憑督、撫文冊,布、按二司冊籍悉停止。國初大計與考滿並行,康熙元年,罷大計,止行考滿。司、道歷腹俸二年、邊俸一年半,有司歷邊俸二年、腹俸三年,錢糧全完者許考滿。分別地方荒殘、沖疲、充實、簡易四者開註,以政績多寡酌定等第。四年,考滿停,復行大計,為永制。大計舉劾註考,例由州、縣正官申送本府、道考覈;教官由學道,鹽政官由該正官考覈;轉呈布、按覆考,督、撫覈定,咨達部、院。河官兼有刑名、錢糧之責者,總河、督、撫各行考覈。專管河務者,總河自行考覈具題。

康熙二十三年,以籓、臬與督、撫親近,停其卓異。凡卓異官紀錄即升,不次擢用。歷朝最重其選,徇私濫保者罪之。康熙初,御史張沖翼請申嚴卓異定額,以詳覈事跡,使名實相副為言。下部議。六年,從御史田六善請,卓異官以清廉為本,司、道等官必註明不派節禮、索餽送,州、縣等官必註明不派雜差、重火耗、虧損行戶、強貸富民。以清吏之有無,定督、撫之賢否。其時廉吏輩出,靈壽令陸隴其等擢隸憲府,吏治蒸蒸,稱極盛焉。四十四年,詔舉卓異,務期無加派,無濫刑,無盜案,無錢糧拖久、倉庫虧空,民生得所,地方日有起色。其他虛文,不必開載。乾隆八年,命督、撫以務農本計察覈屬員,論者謂以勸農為勸吏之要,深得治本,與漢詔同風。先是雍正六年,定卓異薦舉失實處分,自行奏參者免。卓異官有貪酷不法,或錢糧、盜案未清,發覺者,原薦督、撫處分較司、道、府為輕。乾隆四十八年,改定卓異官犯贓,覈其年月在原薦上司離任前後,分別議處。臬司、道、府減督、撫一等,籓司照督、撫例,以道、府按例轉詳督、撫、籓司親為覈定也。五十年,帝以保薦卓異,向分正附,未明定限制,易開徼幸之漸。敕部詳覈各省大小、缺分多寡,酌中定制,裁去附薦名目。於是各省卓異官有定額,終清世無大變更也。

八法處分,行之既久,長吏或視為具文,每將微員細故,填註塞責。歷朝訓諭諄諄,力戒瞻徇,猶防冤抑。雍正元年,詔大計降級罰俸官,例不許卓異,果有居官廉幹因公詿誤者,淮與卓異。又以卓異八法舉劾不過數十人,其不列舉劾之平等官,自知縣以上,令督、撫註考,報部察核。四年,諭參劾人員或有冤抑及避重就輕等弊,除貪酷官無庸引見外,其不謹、浮躁、不及等被劾官,督、撫給咨送部引見。乾隆二十四年,帝以八法參本內不謹、浮躁官,未將何事不謹、何事浮躁、一一聲敘,或有公事無誤而節目闊疏,才具有為而氣質粗率,上司以意見不洽,概登白簡,不無可惜。其或敗檢逾閑,僅與避重就輕,均非整飭官方之意。命詳註實跡,不得籠統參劾。嘉慶八年,定督、撫隨時參劾闒冗平庸等事,未列敘寶跡,被劾官情原赴部引見者,得援大計六法例。此則考覈不厭詳密,冀搜求遺才,輔計典之不及也。嘉、道以後,計典一循舊例,督、撫奉行故事,鮮克振刷。道光八年,山東大計卓異,護撫賀長齡原註新城令容昺悃愊慈祥等語,詔以寬厚難膺上考,令各省薦舉體用兼備、熟明治理者。咸、同軍興,或地方甫收復,有待撫綏,或疆圉偪寇氛,亟籌保衛,敕各督、撫留心存記廉能之員,列上考,備擢用。時督、撫權宜行事,用人不拘資格,隨時舉措,固不能以大計常例繩其後也。

光緒間,言者每條奏計典積弊,請飭疆臣認真考察。屢詔戒飭。然人才既衰,吏治日壞,徒法終不能行。二十八年,詔各省設立課吏館,限半年具奏一次。三十一年,定考覈州、縣事實,分最優等、優等、平等、次等四級。顧課吏祗憑一日文字,考覈僅據一年事實,責以公當,蓋亦難矣。宣統二年,憲政編查館疏請考覈州、縣,分別學堂、巡警、工藝、種植、命盜、詞訟、監押、錢漕,以為殿最。由主管衙門另訂考覈章程。名目繁多,表冊虛偽,徒飾耳目,於勸懲無當也。至若舊例翰、詹大考,分別優劣,升調降革有差,為特別考績之法。外省司、道,年終有密考。州、縣一年期滿,教、佐六年俸滿,皆有甄別。則又隨時考核之法,不屬於察、計二典者。

武之軍政,猶文之考察,兵部職方司掌之。內、外衛、所,分屬於武選司。在京武職,由管旗及部、院覈奏;各省由統兵大員註考。京營千總以上,外省綠營守備以上,各由長官考覈,分操守、才能、騎射、年歲四格。舉劾與文職同。三品以上自陳,由部疏聞候旨。八旗世爵,則校其藝進退之。綠營舉劾,每於軍政後一年半舉行,題升一二人入薦舉班升用,劾者照軍政處分。此其大略也。

國初未立限制,順治九年,定六年一舉,是為軍政考覈之始。十一年,改定五年為期。十三年,從給事中張文光請,軍政卓異,照文官賜服旌勸,後改為加一級。康熙元年,停軍政,專行考滿。既而兵部疏請直省武職應依文官例,按年限由總督、提督會同舉劾。御史季振宜疏言:「武職考滿,營謀優等,剋扣軍饟,貽誤封疆。請按歷俸功次升轉。」於是六年定舉行軍政事宜,京、外武職長官,註以四格,並詳列履行、軍功,分別去留,咨部。必註明行止端方、弓馬嫻熟、管轄嚴肅、供職勤慎、不擾害地方等考語,方許薦舉。必有八法等款實跡,始行糾參。復令提督、總兵官自陳,提督由總督註考,總兵官由總督、提督註考。無總督省分,巡撫註考。嗣以滇省用兵,海內騷動,羽書倥傯,軍政曠不舉行者十年。至二十一年,滇逆蕩平,從給事中碩穆科請,舉行軍政大典,各官事實履行,自康熙十一年軍政後開起。九門千總等由九門提督註考。候補總兵官亦令自陳。副將以下候缺者,照舊例考察。六十一年,命在京武職領侍衛內大臣,八旗都統,前鋒、護軍、步軍統領,副都統等,毋庸自陳。考選軍政時,屬員註考,照外省舉劾例。各省駐防將軍、副都統等,照提、鎮例自陳。屬員照京城例。德州等處城守尉、協領,派大臣往考,會同察覈其屬,註考以聞。雍正元年,命平等官守備以上,督、撫、提、鎮註考。其冬,詔曰:「初次考選軍政,有出兵效力、年老俸深、尚能坐理者,留任。不宜留任者,另奏加恩。或雖未效力行間,而供職年久者,亦留心驗看。」此則垂念資勞,特頒寬典,非常例也。二年,諭各省所保副、參、游擊,輪流引見,察其人材弓馬,督、撫、提、鎮以其操守訓練,分別等第密陳。六年,山西太原總兵官袁立松疏陳平垣營守備梁玉廉潔敏練,以年老入參劾。帝以諳練才不可多得,命酌量以游擊題補,尤殊恩也。是年定卓異官原任有貪酷不法,或升調他省,別犯贓罪,原舉長官,分別處分。

乾隆二年,部議出兵效力人員,年老休致,令子弟一人入伍食糧,無子弟亦給守糧養贍。從之。時直省保題員弁,類以明白勤敏、才堪辦事列上選。十一年,諭嗣後保題,務重弓馬漢仗。十五年,以各省所保總兵官鮮當意,諭曰:「年滿千總一項,類多猥瑣。國家擢用武職,營伍為正途,拔補將弁,必選之若輩。緣次而升,皆自年滿千總始。折沖禦侮之用,豫籌於升平無事之日,不可視為緩圖。」二十四年,以大臣自陳例既罷,敕兵部於軍政年,將在京都統、副都統,在外駐防將軍、都統、副都統,各省提督、總兵官,分別三本,條舉事實候鑒裁,以重考績。四十二年,定衛、所綠營武職薦舉卓異尚未升轉,再遇軍政列平等者,將上次卓異註銷。嘉慶四年,定侍衛軍政考試,向例軍政年不許告病乞休,以杜規避。八年,申諭查閱營伍年分,事關考覈,照軍政例,不得告病、乞休。咸、同軍興,百度稍弛,軍政大典,相沿不廢。咸豐二年,黑龍江將軍英隆以俄兵窺伺,派將弁扼守要隘,疏請本年軍政展限舉行。不允。嗣湖廣總督程矞採等以軍務未竣,疏請展限,令凱撤後再行補考。並諭年老力衰者,隨時參辦。沿及德宗,雖加意振飭,勢成弩末,展限之舉,史不絕書。

光緒十四年,編定北洋海軍,由海軍衙門司黜陟。甲午以後,力鑒覆轍,裁綠營,練新軍,別訂考覈章程。三十二年,改兵部為陸軍部,其考覈隸軍衡司。宣統二年,設海軍部,其考覈隸軍制司。朝廷銳意革新,軍紀宜可少振。無如積習已深,時艱日棘,卒歸罔濟云。


\end{pinyinscope}