\article{列傳一百六十}

\begin{pinyinscope}
宗室奕山隆文宗室奕經文蔚特依順餘步雲

宗室奕山,恂郡王允四世孫,隸鑲藍旗。授乾清門侍衛。道光七年,從征喀什噶爾,擢頭等侍衛、御前行走。歷伊犁領隊大臣、參贊大臣。十八年,授伊犁將軍。二十年,偕副都統關福赴塔什圖畢治墾務,闢田十六萬四千餘畝,奏請置回千戶及五品伯克以下官。召授正白旗領侍衛內大臣、御前大臣。

二十一年,命為靖逆將軍,督師廣東,尚書隆文、提督楊芳為參贊副之。時英兵已陷虎門,楊芳先至,聽美利堅人居間,乞許通商,被嚴斥,促奕山速赴軍。三月,抵廣州。英艦橫亙省河,奕山問計於林則徐,則徐議先遣洋商設法羈縻,俾英艦暫退;塞河道,積沙囊於岸以御砲,然後以守為攻。奕山不能用,且自琦善撤防,舊儲木椿鉅石皆為敵移去,時以杉板小船游弋以誘我師。楊芳主持重,以募勇未集,不欲浪戰。奕山初亦然之,既而惑於左右言,欲僥幸一試,芳止之不可。夜進兵,乘風毀七艘,報捷,詰旦乃知誤焚民舟,而英兵大至,連舟抵城下;御於河南,互有殺傷,遂閉城。

敵以輪船襲泥城,副將岱昌等聞砲先遁,毀師船六十有奇,城外東西砲臺並陷。英兵進踞後山四方砲臺,奕山居貢院,砲火及焉,軍民惶懼,乃遣廣州知府餘保純出城見義律議息兵。義律索煙價千二百萬,美商居間減其半,並許給香港全島,英兵乃退。奕山偕隆文先退,屯距城六十里小金山,諱敗為勝。疏言:「義律窮蹙乞撫,照舊通商,改償費為追交商欠,由粵海關及籓運兩庫給之。」宣宗覽奏,以夷情恭順,詔允所請。閩浙總督顏伯燾迭疏劾其欺罔,下廣西巡撫梁章鉅察奏,乃得其狀,報聞。

英人既得賂於粵,移兵犯閩、浙。奕山等始收回大黃、獵德、虎門諸砲臺,填塞省河。鄉民於義律未退時,困之三元里,餘保純趨救始得出。於是團練日盛,中外皆言粵民可用,遂撤客軍,改募練勇。迭詔趣奕山等規復香港,實不能戰,惟屢疏陳颶風漂沒敵船,毀香港蓬藔,藉修砲臺未竣、造船未就為詞,以塞嚴詔。二十二年,英人撤義律回國,以濮鼎查代之,大舉犯浙江、江蘇。詔斥奕山陳奏欺詐,嚴議褫御前大臣、領侍衛內大臣、左都御史,仍留漢軍都統任。及和議定,追論援粵失機,褫職治罪,論大闢,圈禁宗人府空室。

二十三年,釋之,予二等侍衛,充和闐辦事大臣,調伊犁參贊大臣,署將軍。二十七年,調葉爾羌參贊大臣。安集延布魯特、回匪入邊,圍喀什噶爾、英吉沙爾,命陜甘總督布彥泰督師討之,奕山為副,連破賊於科科熱依瓦特及蘇噶特布拉克,賊遁走。論功,封二等鎮國將軍,賜雙眼花翎。尋授內閣學士,調伊犁參贊大臣,兼鑲黃旗蒙古都統。二十九年,授伊犁將軍。俄羅斯遣使至伊犁,請於伊犁、塔爾巴哈臺、喀什噶爾三處通商,詔允其二,惟喀什噶爾不許。咸豐元年,俄人復固請,仍拒之,偕參贊布彥泰與定伊塔通商章程十七條。祭酒勝保疏論當仿恰克圖通商舊例,限以時日、人數。奕山議:「撫馭外夷以信為主,既已議定章程,旋改必有藉口。」如所請行。累授內大臣、御前大臣,仍留將軍任。

五年,調黑龍江將軍。時俄羅斯以分界為名,欲得黑龍江、松花江左岸地,遣艦入精奇里江,建屋於霍爾托庫、圖勒密、布雅里。奕山疏陳陽撫陰防之策。七年,俄使請入京,拒不許。八年,俄人偕英、法、美三國合兵犯天津。三國窺商利,而俄志在邊地,於是俄使木裏裴岳幅至愛琿,堅請畫界,奕山允自額爾古納河口循黑龍江至松花江左岸之地盡屬之俄。俄使知奕山昧於地勢,駐兵黑龍江口,復索綏芬河、烏蘇里江地,奕山懾其兵威,勿能抗,疏稱未許,然已告俄使可比照海口等處辦理。逾年,與俄使會於愛琿,定約三條,鑱滿、蒙、漢三體字為界碑。大理寺少卿殷兆鏞劾奕山:「以邊地五千餘里,藉稱閒曠,不候諭旨,拱手授人,始既輕諾,繼復受人所制,無能轉圜。」詔切責之,革職留任;又以縱俄艦往黑龍江不之阻,褫御前大臣,召回京。

十一年,聯軍在京定約,因奕山前議,自烏蘇里江口而南逾興凱湖,至綏芬河、瑚布圖河口,復沿琿春河達圖們江口,以東盡與俄人,語具邦交志。尋復御前大臣,補正紅旗蒙古都統。同治中,封一等鎮國將軍,授內大臣。以疾罷。光緒四年,卒,謚莊簡。子載鷟,理籓院侍郎。載鷟子溥瀚,鑲黃旗蒙古副都統;孫毓照,一等奉國將軍。

隆文,伊爾根覺羅氏,滿洲正紅旗人。嘉慶十三年進士,選庶吉士,散館改刑部主事。坐事罷職,捐復,授翰林院侍講。累擢內閣學士。道光中,充駐藏大臣。歷吏部、戶部侍郎,左都御史,刑部、兵部尚書,軍機大臣。屢奉使出讞獄。偕奕山督師廣東,意不相合,甫至,病,憂憤而卒,謚端毅。

宗室奕經,成親王永瑆孫,貝勒綿懿子,承繼循郡王允璋後,隸鑲紅旗。授乾清門侍衛,歷奉宸院卿、內閣學士,兼副都統、護軍統領。道光三年,坐失察惇親王肩輿擅入神武中門,褫兼職,留內閣學士任。五年,遷兵部侍郎。十年,從征喀什噶爾回匪,事平回京,歷吏部、戶部侍郎。十四年,出為黑龍江將軍。十六年,召授吏部尚書,兼步軍統領。二十一年,協辦大學士。

英兵犯浙江,定海、鎮海及寧波府城相繼陷,裕謙死事,命為揚威將軍,督師往剿,都統哈哴阿、提督胡超為參贊,尋易侍郎文蔚、都統特依順副之。陛辭日,宣宗御勤政殿,訓示方略,特詔:「申明軍紀,凡失守各城逃將逃兵,軍法從事。」發交內庫花翎等件,有功者立予懋賞,勉以恩威並用,整飭戎行。大學士穆彰阿奏請釋琦善出獄,隨赴軍前效力,奕經卻之。

奕經分屬懿親,素謹厚,為上所倚重,奉命專征,頗欲有為而不更事,尤昧兵略。奏調陜甘、川、黔兵一萬人,請撥部餉一萬兩,倉猝未集,駐蘇州以待。上以諸將少可恃者,命凡文武員弁及士民商賈有奇材異能一藝可取者,許詣軍前投效。奕經渡江後,於營門設木匭,納名即延見,且許密陳得失。於是獻策者四百餘人,投效者一百四十餘人,而軍中所闢僚佐,多闒冗京員,投效者亦無異才。惟宿遷舉人臧紆青自負氣節,為言議撫徒損國威,始決主戰;又勸劾斬失律提督餘步雲以立威望,疏具而旋寢。以浙兵屢潰,不堪臨陣,召募山東、河南、安徽義勇。

浙事日亟,巡撫劉韻珂促援,遲不至,遂相惡。久駐江蘇,以供應之累,官吏亦厭之,餉需文報,皆延擱不時應。十二月,始抵杭州。前泗州知州張應云獻策規復寧波,奕經、文蔚皆然之,遂令總理前敵營務。應云以重貲購寧波府吏陸心蘭為內應,日報機密多虛誑。奕經禱於西湖關廟,占得「虎頭」之兆,乃議於二十二年正月寅日寅時進兵,屢遣諜,為敵所獲,漏師期。初,英兵踞府城僅二三百人,艦泊定海。至是,濮鼎查率十九艘兵二千散泊江岸,早為之備矣。奕經由紹興進曹娥江,而慈谿敵兵退。應雲請急進,遂駐慈谿東關,文蔚分屯長谿嶺,令提督段永福、餘步雲等趨寧波,游擊劉天保趨鎮海,副將硃貴駐大寶山,而應云率所募義勇駐駱駝橋,為諸軍策應,約於正月晦數路並舉。而敵已勾結應云部勇,勢且生變,不及待期,先二日輕軍分襲,不攜槍砲。永福等入寧波南門,中地雷,天保甫及鎮海城下,為敵砲擊退,皆大敗。越日,應云所具火攻船為敵所焚,軍中自驚,奔大寶山。硃貴收集潰兵圖進攻,敵兵已至,力戰竟日,殺傷相當,無援,貴死之。文蔚聞敗亦退,軍資器械棄失殆盡。奕經留軍紹興,回駐杭州,自請嚴議,詔原之。英艦乘勝由海窺錢塘江,以尖山海口淺阻,尋退去。

鄭鼎臣者,殉難總兵國鴻子,曾從父軍。奕經予二十四萬金,令募水勇規復定海,聞寧鎮之敗,逡巡海上。奕經督之嚴,乃報三月三日敗敵於定海十六門洋面,毀船數十,殲斃數百。劉韻珂以為欺罔,奕經遣侍衛容照等出洋查勘,得焚毀船木及壞械回報,乃疏聞,賜奕經雙眼花翎,鼎臣亦被獎。時寧波英兵忽退,留艦招寶山海口,改犯乍浦,陷之。奕經不能赴援,而以收復寧波奏,詔斥不先事預防,革職留任。既而英兵犯江南,陷鎮江,逼江寧,命奕經赴援,尋命駐王江涇防禦。奕經自寧波、慈谿之敗,軍心渙散,不能復用,益為劉韻珂所揶揄,議守議撫,一不使聞。及和議成,撤師,詔布奕經等勞師糜餉、誤國殃民罪狀,逮京論大闢。

圈禁逾年,與琦善同起用,予四等侍衛,充葉爾羌幫辦大臣。為御史陳慶鏞論劾,仍褫職。未幾,復予二等侍衛,充葉爾羌參贊大臣,調伊犁領隊大臣。坐審鞫英吉沙爾領隊大臣齋清額誣捕良回獄不當,褫職發黑龍江。三十年,釋回。咸豐初,歷伊犁、英吉沙爾領隊大臣。二年,召授工部侍郎,調刑部,兼副都統。三年,命率密雲駐防赴山東防粵匪,卒於徐州軍次,依侍郎例賜恤。

文蔚,費莫氏,滿洲正藍旗人。嘉慶二年進士,授翰林院檢討。累擢至兵部、工部侍郎,兼副都統、內務府大臣。方其駐長谿嶺也,聞諸路軍皆不利,欲移營走。敵雜難民潰兵猝至,焚毀營帳,乃奔曹娥江,收集潰兵,退保紹興。欲渡錢塘江,為劉韻珂所阻。尋以定海報捷,加頭品頂戴。軍事竣,追論失機,褫職下獄。逾年,釋出,予三等侍衛,充古城領隊大臣,復褫職。咸豐初,歷喀喇沙爾、哈密辦事大臣,駐藏大臣,奉天府尹。五年,卒。

特依順,他塔喇氏,滿洲正藍旗人,福州駐防。累遷協領。道光十三年,從平臺灣張丙亂,擢荊州副都統。歷騰越鎮總兵、密雲副都統、寧夏將軍。二十一年,予都統銜,授參贊大臣,督師廣東。尋命改赴浙江辦理軍務,駐守省城,署杭州將軍,遂實授。乍浦陷,坐革職留任。和議成,命籌辦浙江善後事宜。二十六年,調烏里雅蘇臺將軍。二十九年,卒。

餘步雲,四川廣安人。嘉慶中,以鄉勇從剿教匪,積功至游擊。平瞻對叛番,累擢重慶鎮總兵。道光七年,率本鎮兵從楊遇春徵回疆,破賊洋阿爾巴特莊;偕楊芳擊賊於毗拉滿,大敗之,復和闐,追擒賊酋玉努斯,授乾清門侍衛,擢貴州提督。調湖南。十二年,率貴州兵剿江華瑤趙金龍,偕提督羅思舉破賊巢,金龍就殲,加太子少保。復破粵瑤於永州藍山,擒其渠。從尚書禧恩赴廣東剿連州瑤,平之,賜雙眼花翎,予一等輕車都尉世職。歷四川、雲南提督,復調貴州。十八年,擒仁懷匪首謝法真,加太子太保,調福建提督。

二十年,英兵初陷定海,率師赴援,調浙江提督。二十一年,定海既收還,步雲駐防鎮海。裕謙來督師,疏言步雲不可恃,未及易而英兵猝至,復陷定海,三鎮戰歿。步雲屯招寶山,總兵謝朝恩分守金雞嶺。步雲號宿將,實巧猾無戰志,又嗛裕謙剛愎,將戰,裕謙召與盟神誓師,託疾不赴,且獻緩敵之策。敵攻其前,而以小舟載兵由石洞攀援登後山,步雲遽棄砲臺走,敵乃據招寶山俯擊鎮海城,金雞嶺及縣城先後陷。步雲退寧波,敵掩至,墜馬傷足,僅免,府城遂陷。步雲疏聞,委敗於裕謙。裕謙既歿,其妻赴京訟之。二十二年,從奕經規復寧波,不克,褫步雲職,逮京,命軍機大臣會刑部訊鞫。廷臣爭劾其罪,亦有原之者,獄久延,尚書李振祜堅持,讞乃定。詔曰:「餘步雲膺海疆重寄,未陣獲一賊,身受一傷,首先退縮,以致將士效尤,奔潰棄城,直同兒戲。儻不置之法,不惟無以肅軍政而振人心,且何以慰死節諸臣於地下?」步雲遂棄市。

論曰:奕山、奕經,天潢貴胄,不諳軍旅,先後棄師,如出一轍,事乃益不可為。其人皆庸闇不足責,當時廷臣不能預計,疆吏不能匡救,可謂國無人焉。奕山後復棄東北邊地,其貽患尤深。餘步雲庸懦巧猾,卒膺顯戮。宣宗於僨事諸人,皆從寬典,伸軍律者,僅步雲一人耳。


\end{pinyinscope}