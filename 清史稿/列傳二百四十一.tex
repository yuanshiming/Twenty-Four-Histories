\article{列傳二百四十一}

\begin{pinyinscope}
劉錦棠張曜劉典弟倬云

金順弟連順鄧增托雲布果權劉宏發曹正興

穆圖善杜嘎爾額爾慶額豐紳文麟明春富勒銘額徐學功

劉錦棠,字毅齋,湖南湘鄉人,松山從子也。從松山討捻,積勛至同知直隸州。從入陜,復同州、朝邑,釋省城圍,擢巡守道。同治七年,左宗棠西征,從克懷遠、鎮邊,還定綏德,賜號法福凌阿巴圖魯。進軍甘肅,攻金積堡,夷旁近七寨,破靈州。九年,擊馬五寨,松山戰死,詔加三品卿銜,接統其軍。軍新敗,偏裨自恃為宿將,滋驕,錦棠禮詘之。喪懸吳忠堡,或請徙它處,錦棠不可,曰:「櫬在軍,可系將士心。」宗棠貽書,為列堅守、退頓二策。錦棠謂:「不力戰,則靈州不保,必效力致死,而後軍可全。」於是一戰擒馬五,再戰破河、狄,軍復振。

是時馬化龍焰日熾,三決水困我軍,錦棠三拒之,不獲逞,糧且匱,率其子耀邦乞款附。錦棠曰:「諾。令若先繳馬械。」不應,再引馬連水入湖。會大風從西北起,濤齧堤岸,勢洶湧。錦棠囊土以御,化龍計益窘,哀詞乞耕墾。錦棠知其詐,隱卒下橋、永寧洞,又敗去,乘勢下蔡家橋,克東關。化龍度不得脫,於是三踵軍門乞撫矣。錦棠白宗棠請進止,乃徙陜回化平川,而分置甘回於靈州。論功,予雲騎尉世職,賞黃馬褂。十年,誅化龍父子,生致馬八條,置喪所,臠而祭之,遂輿喪歸。

明年,度隴攻西寧。次碾伯平戎驛,先破小峽,遣軍奪南北兩山,圍解,道員郭襄之率男婦二萬繦負來迎。是役也,提湘軍八十營,扼攻九十里,往往徹夜不休,露立冰天雪窖中,詔嘉之。十二年,克大通,斬叛官馬壽。遴陜回為旌善五旗,餘徙平涼、秦安、清水。白彥虎奔肅州。方湘軍之定西寧也,宗棠緣事責錦棠,盛氣辨,銜之,以故徇肅州未下,亦不召。及錦棠至,又大喜,為誇其軍以勵眾。錦棠計誅馬天祿,殺土回、客回立盡,關隴平。權西寧道。明年,破河州,獲閃殿臣,伏誅。乃合諸將蹙回於賈家集、郭家嘴,殲焉。

光緒元年,出關。時彥虎走依安集延,帕夏阿古柏助之,勢復熾。二年,至阜康,與金順計事,議先攻古牧。遣將分壁木壘河,而自領軍軍九營街。度戈壁乏水,佯掘井以懈敵,陰遣精騎襲奪黃田,通汲道,收古牧地。錦棠策烏城寇必駭奔,復自將精兵走之,遂復烏魯木齊、迪化,予騎都尉世職。

三年春,逾嶺西南攻達阪。寇引湖水衛城,泥深及馬腹。錦棠周城徼循,誡各營警備。列燧如白晝,轟擊之,彈落爆藥窖,聲砉然,人馬碎裂。乃下令軍中曰:「能縛獻服異服者賞。」於是愛伊德爾呼里以下皆就俘,愛伊德爾呼里,猶華言「大總管」也。且釋降回數千,給貲糧縱歸。或請其故,曰:「俾歸為我宣播朝威也,吾欲以不戰勝之。」自是破吐魯番、托克遜,南路門洞開,阿古柏如失左右手,亦被執,飲藥死。賞雙眼花翎。

已而彥虎據開都河西岸,覬入俄。師抵曲惠,與餘虎恩分擊,彥虎亦決水以阻。錦棠入喀喇沙爾城,廬舍漂沒,乃徙和碩特帳房河東數百戶,實後路,復庫爾勒。會軍中患饑乏,下令掘窖糧,獲數千石以濟。連下庫車、拜城。其南纏回苦安集延淫暴久,重以彥虎奔擾,益不堪命,旦夕望我軍如時雨。比至,各城阿奇木伯克、阿渾玉子巴什各攜潼酪,持牛羊來犒師。抵阿克蘇,錦棠先入城,受降畢,回皆伏服。聞彥虎奔烏什,亟遣旌善旗渡河復其城。於是東四城俱下,詔晉三品京卿。值喀城守備何步雲告亟,遂大舉出師,令虎恩、黃萬鵬分道進取,而自率師徑搗葉爾羌,並克之。彥虎遁入俄。錦棠進定英吉沙爾,遣董福祥收和闐,西四城亦下,錫二等男。

四年,錦棠既定喀城,以次巡歷葉爾羌、和闐。凡西人僑居其地者,英乳目阿喇伯十餘人,印度溫都斯坦五千餘人,咸服其勇略,稱為「飛將軍」云。方彥虎之入俄也,俄人處之阿爾瑪圖。錦棠猶致書圖爾齊斯坦總督,謂將入境搜捕,宗棠勸止之。俄復徙之托呼瑪克。其秋,彥虎又遣黨犯烏什邊,驟入格爾品。錦棠扼之瑪喇爾巴什,別遣將要其歸路,大敗之。未幾,安集延入,又破之玉都巴什。是歲補太常寺卿,轉通政使。五年,安夷復構布魯特內犯,戰烏帕爾,捕斬二千餘級。自是邊寇頗息警。

維時俄據伊犁,宗棠疏請崇其秩,資鎮撫,詔佐軍事。俄益增兵守納林河。已,宗棠入都,上以此專屬任錦棠關外事,命署欽差大臣。徙頓哈密,益治軍。逾歲除真。八年,和議成,錦棠策善後,請設新疆行省,省置巡撫、布政使,加鎮迪道按察使銜,道、府、州、縣視內地。立城垣、壇廟、學校、驛傳,又廣屯田,興水利。南疆歲徵賦至二十餘萬石。九年,擢兵部右侍郎,加尚書銜,旋除新疆巡撫,仍行欽差事。十一年,進駐烏魯木齊,奏省參贊大臣,改置都統,設喀什噶爾、阿克蘇、巴里坤提鎮各營。復增道、府、、縣,徙分防官駐要塞,南北郡縣之制始定。

先是,錦棠以祖母老病,累疏乞歸省,不許。十三年,申前請,始俞允。錦棠悉召諸部酋長大酺,遂發。所過,黃童白叟望風相攜負以迎,往往擁車數日不得走。十五年,加太子少保銜。明年,晉太子太保。二十年,晉錫一等男。會弟鼒以山西按察使入覲,垂詢近狀,欲強起之。適中日有違言,電旨趣召,未行而病作,朝廷書問日數至。疾革時,猶喃喃呼舊校指述邊事。未幾,卒,年五十一。事聞,震悼,謚襄勤,予建祠。

張曜,字朗齋,其先上虞人,改籍大興,既,復隸錢塘。生有神力,幼嘗持竿結陣,部勒群兒,無敢譁者。少長,依舊■L5蒯賀蓀。賀蓀宰固始,適豫捻起,集團勇三百屬之。捻★K4至,時已昏。曜獻策,謂:「伏軍城外,彼不知眾寡,可以計走也。」夜半,捻縱掠,轟擊退。僧格林沁追捻亟,遙見火光,詢知為曜部,召與語,其悅,命從軍。積勛為知縣,權知固始。皖捻來犯,嬰城守,寇駸駸西去。亡何,李秀成又構捻入,圍城三匝。捍禦七十餘日,城獲全。上嘉其功,賜號霍欽巴圖魯。

咸豐十年,擢知府。先後遭憂,仍留軍討皖捻。屢捷,晉道員。明年,除河南布政使。是時陳大喜、張鳳林各樹幟,延擾數千里。曜謂寇援斷,師未能驟克,寨中患饑乏,多猜貳,宜廣設購募間其黨。乃縱降者為內應,捻乃竄,諷諭各寨,皆款服。鳳林偽降,計擒之。

同治元年,御史劉毓楠劾其目不識丁,詔改總兵。二年,大喜走阜陽,戰秦宣寨。皖捻復入,曜慮華莊失,亟斂兵退,以銳師宵加之,殲渠率。時張總愚走鄢陵、臨潁欲西,曜拒之嵩山麓,令不得西。攻大金店,援寇四起,遣總兵保英略中路,為寇乘。擢手刃退縮者,士氣乃奮,卒敗之。攻太子望寨,久相持。曜曰:「捻詭悍,堅守山口,阻我進兵。坐為所綴,非策也。」間道出奇擊之。總愚西北走鎮平,追殺黑龍集。犯南陽,又戰卻之。三年,捻會宛南,總愚赴盧氏,嗾粵寇入豫。曜扼之,使不能合。次橋川,寇自西北至,狙伏以擊,寇奔楚。曜浮渡丹江,追越鄖西百四十里,會陜、楚軍至而還。四年,僧格林沁軍麻城,騶騎失利。曜赴難,七戰皆克。僧王既戰死,臺臣劾其養寇遺患。巡撫吳昌壽按覆,事白,曜假歸葬親。逾年,淮北捻益亟。朝旨趣復出,曜遂部合新舊選鋒號嵩武軍。厥後軍氣大振,論者謂為湘、淮軍後勁云。

六年,出頓許州八里橋,任柱等見曜大纛,駴走。梁山寇與合,五日至鉅野,渡運而東。曜與總兵宋慶往馳之。當是時,李鴻章議防運河北。首安山訖沈家口。曜等至,令慶築長墻。慶留副將蔣東才、參將李承先二軍屬曜。曜以沈家口黃、運交匯,調黃河水師入運助守。守河防運自此始。久之,總愚陷陜綏德,分擾米脂。朝命防河北。捻濟河入晉,犯吉州鄉寧,平、蒲告警。檄曜還豫,而捻已自絳州曲沃入偪豫疆,曜乃追敗之湯陰。

七年,捻東走,扼晉州西北路,折而南。諸將謀逐之,曜謂賊勢未蹙遽南奔,必有深謀。乃卷甲北趨,出其前二百里,至饒陽,果遇捻。捻不敢擊,錯愕去,潛渡滹沱。亟引兵至河干,未渡者殲焉。濟漳次清化,捻伐木為矛,又擊破之。長驅滄州,劉松山軍運東,曜自南夾搫,捻披靡,追至臨邑。初,李鴻章遣郭松林自臨邑築墻,屬之馬頰河,虛西南以餌敵。逮曜馳至,捻不肯深入,走濟陽。鴻章知計不售,使曜趨武定,遇捻於濱州,始敗退。會天大雨,河暴漲,諸將慮捻東逸,謀扼徒駭河。曜自博平守至東昌,誘捻入河套,與慶合擊之。捻眾陷泥淖中,死者枕藉。自是總愚不復能軍,逐北茌平,殺其黨且盡,騎能屬者十餘人耳。總愚自度不得脫,乃沉於河。論功,賞黃馬褂,予騎都尉世職。

八年,詔趣左宗棠赴涇州,責金順以邊外事,命曜自古城西進為後路,軍次蘭扇,破回於察漠綽爾,又敗之紅柳樹,阿拉善圍解。抵纏金,詗知寧夏西岸自石嘴山至中衛,陜回麕集。遣將要擊之,回遁走。金順赴沙金托海與議進兵事,方將鼓行而西,而寧夏降回復叛,圍郡城。遂倍道應赴,敗之城下。俄而陜回入賀蘭山。曜赴河北,南破漢渠集,圍納中閘,拔其壘,與金順夾渠而軍,殲守賊殆盡。會金積撫局成,通昌、通貴乞款附,獨王家甿不下。曜怒,破其堡,屠之。攻剋納家寨,河西無悍寇。詔屯之,兼顧阿拉善旗。是為寧郡河北之師。九年,授廣東提督,仍留軍。明年,加雲騎尉世職。

白彥虎據肅州,徐占彪攻弗克,請益師,宗棠檄曜頓金積助之。上以為勇,賞雙眼花翎。十三年,彥虎亡命出嘉峪關,窟烏魯木齊,哈密城南北附之。俄羅斯方擁伊犁,巴里坤且岌岌。朝命總防討,亟援哈密。曜剋日出關,師行乏水草,絕幕二千餘里,運餽艱阻,於是議立屯田。十三年,出屯,大興水利,墾荒地二萬畝,歲獲數萬石濟軍。光緒二年,師規南路,西取七克騰木、闢展及魯克沁臺、勝金臺、哈拉和卓城,降纏回萬餘,遂復吐魯番。明年,拔烏魯木齊,彥虎遁入俄。

俄歸伊犁,宗棠疏薦堪重任。六年,詔贊軍事,命移駐喀什噶爾,兼轄西四城,籌善後,所至創立義塾。回夙獷噬,至是頗聞弦誦聲。十年,入關防直北,賞巡撫銜,敘邊功,晉秩頭品。

明年,除廣西巡撫,未行,董所部治都城河,加尚書銜。旋命赴山東勘河,喻歲至壽張,調撫山東。東省河患日深,曜蒞任,首重河工,以黃、運並淤,非總濬通海不為功。時王家圈等處先後漫口,先議疏濬海口,挑淤培墊,並增築徒駭河兩岸堤工,以防氾濫,然後挑挖全河,參用西法,以機船疏運。凡南北兩岸堤墊口門,一律籌辦。疏上,皆從之。又先後築王家圈、姚家口、張村、殷河大寨、西紙坊、高家套各決口,復改濬韓家垣,以洩尾閭,莫不身親其事,計一歲中奔走河上幾三百日。有言河務者,雖布衣末僚,皆延致諮詢,唯恐失之。民或遇災,常籌粟賑濟。復建海岱書院於青州,葺洙泗書院於曲阜,士民德之。

十四年,被命襄辦海軍。明年,晉太子少保,命會閱南、北洋海軍。至煙臺,聞臺灣巡撫劉銘傳稱疾去,則抗章請行,優詔答之。十七年,方駐河干督工,疽發於背,回省就醫,遽卒。疾革時,猶貽書鴻章,首言山東為北洋門戶,亟宜治砲臺備不虞;次言新疆軍糈,部令裁營清釐,緩不濟急,恐失信外域,貽君父憂。遺疏入,上震悼,贈太子太保,謚勤果,予建祠。

曜魁梧倜儻,自少從戎,不廢書史,字法橅顏平原,書疏雅馴猶餘事。嘗鐫「目不識丁」四字印,佩以自勵。寧夏平,築樓面黃河,對賀蘭山,顏曰「河聲嶽色」,日嘯詠其中,人謂有羊叔子登峴風。居官垂四十年,不言治產事,性尚義,所得廉俸輒散盡。尤禮賢下士,士爭往歸之。其修道路,開廠局,精製造,凡有利於民者,靡不畢舉。死之日,百姓巷哭失聲,喪歸,且傾城以送。以兩世職並為男爵,子端本襲,官南韶連道。

劉典,字克盦,湖南寧鄉人。少伏溈山,與羅澤南友善,以學相期許。齋匪亂,集眾保鄉邑,敘訓導。參左宗棠戎幕,轉戰江西。善察形勢,嘗輕騎言冋敵營,夜率所部劫殺,數獲奇捷。李秀成欲以長圍困宗棠,斷曾軍糧運,典敗浮梁,又破之樂平,婺源餉道始達祁門。積勛至直隸州知州。宗棠撫浙,典以偏師討衢、嚴。同治改元,破馬金街,進克遂安,遷知府。擊花園港,李世賢遁,超授浙江按察使。世賢謀據金華,分黨擾龍游、湯溪、蘭谿,眾號數十萬。典還軍援衢,力據上游,悉夷東南北寇壘。明年,收蘭谿,諸軍亦下三城,浙東告寧。宗棠規杭州,策江、皖邊圉安,方可一意進取。乃令典將五千人,道嚴出皖南。當是時,新復郡縣糧饋不屬,典持印票空文,向民間貸糧,遇寇遮擊,而屯溪,而黟縣,所風靡。民望典軍如時雨,以故壺漿塞塗。沈葆楨謂其截擊寇眾,功不在克省城下。江、皖既平,賜號阿爾剛阿巴圖魯。其秋,父憂歸。

三年,詔起督師,典募新軍八千,次貴溪。世賢入閩陷漳州,汪海洋亦陷龍巖,勢復熾。典進汀、連,號西路軍。遇海洋,新軍輕進,敗績,還保連城。四年,再戰,斬寇萬餘,進復龍巖、南靖。世賢為高連升所蹙,奔粵,宗棠麾下壯士騎從者八百餘人馳之。典至南雄,語其將黃少春曰:「尾寇而追,非計也。寇返奔亟,必不久據嘉應,當走粵、閩邊。左軍孤,遇寇必不支。」乃持二旬糧,取道大嶺脊,晨夜應赴,抵大埔,先宗棠一日,遂會師復嘉應,晉二品服,予世職。事寧,乞歸省。

五年,宗棠徙督陜、甘,起典甘肅按察使,旋賜三品卿,佐軍事。典自紫荊關入,值捻竄渭北,乃駐潼關,偪渭而軍,扼其南渡。七年,詔署陜西巡撫。初,宗棠援晉,以徵回事屬典。典以關中戎備寡,調提督周達武壁隴、汧顧後路。至總前敵師干,則舉蔣益澧自代,朝旨弗許。尋復有是命。典遂進駐三原,調度諸軍,軍大振。明年,與宗棠定三路剿回策。已,復念民事,重入省,治善後,集流亡,舉屯牧,恤艱戹,革差徭。又以其時濬鄭白舊渠,關中漸喁喁望治矣。惟銳然以減餉自任,諸軍舊欠各餉,十給其七,士卒不無怨望雲。又明年,再乞歸省。

光緒元年,復命佐宗棠軍務,典辭以疾,詔罷其行。時譚鍾麟督西征餉事,力言司左軍後路非典莫屬。朝旨乃趣赴甘,於是典三起討賊矣。二年,至蘭州,宗棠以善後畀之。凡整軍節餉,以及生聚教誨,有裨於民生久遠者,咸殫心厥事。至關外平定,亦嘗指陳方略,贊畫功多。經營新疆凡三年,卒於軍次。詔視侍郎賜恤,謚果敏,予江、浙、陜、甘建祠。

典秉性清嚴,貴後自奉儉約。楊昌濬嘗詣典,環堵蕭然,一如寒素,寓書宗棠共稱之。

弟倬云,少隨典讀書長沙。典主鄉團,倬云以廩生治軍書。從援浙,領偏軍,戰常陷堅。李、汪二酋奔贛,扼臨江使不得西,敘知縣。龍巖既復,會糧罄,軍士乏食,為貸鄰邑以濟,民德之,建生祠。將軍庫克吉泰檄赴陜,值連升營哥匪謀變,戕主帥。倬雲馳入,殲其渠,餘眾悉定,再遷知府。時慶陽大饑,人相食。倬雲興屯政,立賑局,流民懷集。假歸,適會匪亂,連下龍陽、益陽,詔用道員。越法事起,赴閩綜營務,署按察使。以捕海盜名,晉二品秩,授汀漳龍道。興蠶桑,建書院,周恤堤防諸政,次第畢舉。尋謝病歸。二十九年,卒,恤如制。

金順,字和甫,伊爾根覺羅氏,滿洲鑲藍旗人,世居吉林。少孤貧,事繼母孝。初,從征山東,授驍騎校。嗣從多隆阿援湖北,復黃梅,賜號圖爾格齊巴圖魯。移師安徽,克太湖。歷遷協領。掛車之役,將騎旅直搗中堅,當者輒靡。

同治二年,從討陜回,連下羌白、王閣,賜頭品秩。復渡渭敗之零口。三年,漢南回奔鳳翔,趨灃峪,擊卻之,授鑲黃旗漢軍副都統。調西安,時粵寇集陜南,陜回導至灃河,金順御之,斬虜多。回入鄠,傍山西走,復率馬隊邀擊敗之。四年,攻寧夏南門,奪其砲臺。已,聞寇集黃河兩岸,率師分路進,陣斬其酋孫義保等,寇稍卻。明年,調寧夏副都統。七年,復寧條梁。聞榆林警,遂迎擊五龍山,大破之。追至邊外禿尾河,馬隊忍寒裸涉,要之金雞灘,回大潰。復遣將破之葭州。

八年,平綏德,朝旨以邊外事屬之。四月,回犯花馬池,遣部將富勒琿馳救。回自烏拉爭渡,奔札薩旗。金順自將出邊,回已遁。乃率師道札薩郡王答拉旗,自包頭迤西濟河而進。會天酷暑,暫頓什巴爾臺就水草,與張曜期會沙金托海。七月,自中灘鼓行而西,而寧夏回復叛,乃兼程赴援,敗之於城外。無何,甘回納萬元等迎戰漢渠,復與曜從東繞擊。回走納中閘,追至龍王廟南,悉拔其東南各壘,殲守賊殆盡。

九年,金積撫局成,獨王家甿未下,率其弟連順分兵迎擊,數獲勝。自是連順無役不從,積勛至金州副都統。金軍頗有聲,明年,克之,賞黃馬褂。又與曜破納家寨,河西悍黨殲焉。寧夏平,擢烏里雅蘇臺將軍。尋以赴鎮番未報,褫職,命即日赴肅州。既至,頓北崖頭,奏調曜軍助擊。時烏魯木齊提督成祿猶訴軍糧乏,難赴哈密,詔金順接統其軍。十二年,左宗棠至軍,約期並進。金順發地雷東北角,城潰,乘隙奪據其上,自是老弱伏服者相繼。城拔,復故官。

烏魯木齊都統景廉駐古城,與金順齟。宗棠奏言金順寬和,為群情所附。詔率所部二十營赴之,規烏城,於是遂發。出關數十里,至瀚海,吏士忽不行。詢之,則曰:「先鋒營駐,有所議。」金順知有變,疾馳視,手刃六人以徇,曰:「敢留者,視此!」軍以次行。瀚海既過,乃列六尸祝之曰:「雜賦不飽,佐以野蔬,天下無若西軍苦。此行度戈壁,乏水草,吾非不知。但不忍汝六人,如全軍何?如國家何?如關內生靈何?」聞其言者,無不激揚。道授正白旗漢軍都統。明年,至古城,與景廉會師。一日演砲,漢、回觀者數千百人。景軍指敗堵煙筒為的,擊之再,煙筒無恙。所部砲隊總兵鄧增、參將張玉林曰:「是何足擊?請卷旗卓之為的。」增先測視,請於金順再測視,既竟,砲響旗飛,若翦霞空際。已而玉林亦爾。觀者讙謼聲震遠近,回聞之氣奪。尋命佐新疆軍務。

光緒改元,代景廉為都統。二年,軍阜康。劉錦棠赴軍所商進止,議先攻古牧。乃輕騎襲黃田,通汲道,克之。連下烏魯木齊、迪化、昌吉、呼圖壁諸城,直偪瑪納斯,斬其偽帥馬興,南北二城以次皆下。賞雙眼花翎,予世職,調伊犁將軍。七年,詔接收伊犁,按圖劃界。十一年二月,軍標譁變,五月,再變,並譟餉戕官。伊地本極邊,協餉乖時,軍多疲饉。金順馭眾寬,將領營官相率蒙蔽,而總督譚鍾麟劾其上下縱恣,民怨沸騰,為陳籌餉易人之策。於是上召來京,以錫綸代之。道肅州,病卒。身後不名一錢,幾無以為斂。寮寀醵金,喪始歸。部伍縞素,步行五千里至京者,達二百人云。事聞,贈太子太保,謚忠介,予建祠。

妻託莫洛氏,婚甫逾月,囑事繼母,撫諸弟,遂出。轉戰二十餘年,至新疆,乃遣使往迓。謂使者曰:「太夫人老矣,寧能涉萬里?吾義不得獨行。且彼處有姬侍,宗祧不墜,吾又何求?」竟不往。時論賢之。

增,字錦亭,籍廣東新會。年十七從軍,積勛至游擊。西征之役,領開花砲隊,平金積,取河州,並以善用砲知名。方是時,錦棠治兵西寧,寇堅壁自守,而牧馬湟水北岸。增隔水轟擊,寇駭愕逾山遁。增馳之,先以輕騎當寇,乍戰佯北。寇易之,悉眾下山,我師以巨砲環擊,大潰。俄援寇至,壁平戎驛。錦棠不與戰,而使增據山上俯擊。寇懾砲威,退湟北,增復隔河擊之,皆走。錦棠攻高寨急,舁大砲列北山上,使增測準寇壘,發砲子六十餘,墻壁皆裂,賜號伊博德恩巴圖魯。規肅州,城高厚逾常制,增築砲臺臨城關,轟潰十餘丈。繼復築砲臺街口,裹創力戰,卒擊退之,晉總兵。從金順出關,以戰功擢提督。金順舉將才,稱增精究砲術。除伊犁鎮,調西寧。

光緒二十一年,解循化圍,回渡河趨巴燕戎格,增追至思觀。會札什巴陷,分三路擊之,城拔。六月,西寧回韓文秀等犯增營,增分軍迎擊,遇伏將卻,增手刃先退者以徇,眾皆躍馬陷陣,寇潰。時西寧南北西川、大通、碾伯、丹噶爾皆叛,增聞警,馳歸守郡城。八月,哆吧寇來襲城,薄小橋。增將出拒,或勸沮之,增曰:「寇氛甚惡,不力遏之,是示弱也。且主帥不出,將士孰肯用命?」遂往,短兵接,人百其氣,大敗之。自此寇望見鄧軍旗幟,輒不戰而遁。明年,克川北、營城,關內外平,授固原提督。既至,會甘軍搜治海城叛回。閱三年,海城回田百連復叛,遣將討平之。拳亂作,車駕西狩,召赴行在。回鑾,節度隨扈諸軍,晉頭品服。旋回任。三十一年,卒於官,詔附祀宗棠祠。

其時隨金順徵回著績者,又有托雲布、果權、劉宏發、曹正興。

托雲布,瓜爾佳氏,滿洲鑲藍旗人。初,從軍剿發、捻,賜號綽勒郭蘭闊巴圖魯。攻寧夏,釋平羅圍,襲擊黃河兩岸,數有功,累遷協領,坐事免。金順請留軍自贖,截擊竄寇於榆林,復官。進拔蘇家燒房、納中閘,晉副都統。時寇據金積,其旁堡砦並險固。托雲布充前鋒,大小數十戰,寇稍卻,克王家甿,賜頭品服;平通昌各寨,賞黃馬褂。自是從出關,迭克名城,即於軍前授青州副都統。瑪納斯之役,血戰六十餘日,天山以北告寧,予雲騎尉世職。光緒十一年,創發乞歸,賞食全俸。十八年,卒,予優恤。

果權,莫得里氏,滿洲正藍旗人,吉林駐防。驍騎校,從副都統福珠裡出師伊犁。以戰功,累遷協領。瑪納斯既復,金順薦署伊犁錫伯營領隊大臣,頓車排子屯田。詔念前勞,晉副都統,賜號志勇巴圖魯。光緒十七年,調充東三省練兵行營翼長,校閱吉林邊練各軍。二十七年,授呼蘭副都統。卒,恤如制。

宏發,黃陂人。正興,鄖西人。自同治初久從金順軍,復肅州,進新疆,屢有功,後皆官至提督。而宏發軍過玉門、安西,官民尤翕,頌贊不置雲。

穆圖善,字春巖,那拉搭氏,世居黑龍江齊齊哈爾,隸滿洲鑲黃旗。家貧,事親孝。初以驍騎校遷參領,從征直、魯、晉、豫,所向有功。援安徽,迭克城隘,賜號西林巴圖魯。同治元年,從多隆阿入陜,道鄧州,遇粵寇陳得才,敗之紫荊關,擢西安右翼副都統。時回氛熾,率步旅扼洛水北岸,半修營,半出擊寇,寇始奔。亡何,捻酋姜泰林犯武關,夜襲多軍。穆圖善設伏敗之,追群寇入鄂境,悉驅出西河口。二年,再入陜,攻高陵,先登,里創力戰,下之,賞黃馬褂。寇渡涇據南岸,穆圖善泅水而濟,寇大潰。三年,多隆阿圍盩厔,中砲,病篤,疏薦穆圖善賢,遂命署欽差大臣。其夏,擢荊州將軍,與劉蓉會辦陜事。

粵寇據樓觀、黑水、西駝峪,蓉遣蕭慶高趨鄠,穆圖善率師助擊,戰店子頭,敗績。七月,擊破大峪西堡,進攻樓觀。先是,得才入鄂,穆圖善遣二十八營赴援,無統帥。至是蓉奏請穆圖善往湖北,詔勿許,令移師赴甘。既至,與將軍都興阿議定先規寧夏。十一月,檄杜嘎爾、額爾慶額等攻破清水堡。逾歲,詗知群寇元日椎牛置酒,必不誡。穆圖善奪城南砲臺,連毀其寨。已,復慮寇乘春漲決渠下灌,分兵扼城東南。尋調寧夏將軍,主甘肅軍事。嗣以寧夏諸軍久不得要領,責之。五年,收靈州。初,回寇馬兆元攻陷州城,馬朝清計誅之,禁靈回無滋事。逮寧夏失,道且梗塞。朝清者化龍也。至是,親詣穆圖善哀詞乞款。會州紳亦請置官,乃使豐紳等往招撫,州事定。聞華亭回竄慶陽,復遣軍擊走之,城圍解。

明年,署陜甘總督,值歲大饑,人相食。乃馳書阿拉善王,令運蒙糧至河北,與軍民交易,食乃濟。是時米拉溝既下,河、洮、狄道、西寧回皆反正,而南八營李得昌各部,乞擇地安插。上慮回情叵測,敕穆圖善嚴備之。穆圖善令降回繳械,遣範銘赴洮,張瑞珍赴蕭何城,王得勝赴靜寧辦撫事,自是頗惑撫議。已,復使曹熙等赴河州,回羈之,遣黨潛襲省城,聲款附。穆圖善率輕騎往,中伏奔還,遂圍城五日,楚軍將彭楚漢等破之。而東鄉回嵎負如故,穆圖善親督諸軍敗之。十一月,攻州城,弗克,還蘭州。會傅先宗敗回禮縣,彭忠國敗回安定,穆圖善乃令進規渭源,而自從金縣進。七年正月,克之。乃使諸將會攻狄道,南北兩山相崟,中有平川,寨卡林立,先宗等一鼓破之,毀其寺,軍威大振。於是穆圖善赴前敵,北莊牟佛提率男婦三千人乞降,受之。師旋,復叛。穆圖善再渡河,直搗黑山頭、太子寺。寇斷我糧運,戰數失利,不獲已,退保狄道。明年,狄城糧盡,又退至秦州,寇乘之,師潰。朝旨令穆圖善甘軍受左宗棠節度。

初,穆圖善狃撫議,群回叛服靡恆,而所部百數十營皆徵糧民間。清水守將敖天印以橫暴激民變,逐防軍,殺縣役。提督黃金山率所部戰狄道康家巖,敗潰,北入皋蘭,四出淫掠。穆圖善乃遣潰勇屯寧夏,而訾報敖軍於朝,敖軍亦力詆之。於是宗棠調度諸軍,先秦州固餉源,遂赴涇州受總督印。

穆圖善既卸事,猶日歷四鄉,勸民修堡寨,置軍械,蘭人甚德之。詔仍駐蘭州,統西路軍。化龍黨崔三構河、狄回出擾,輒敗去。十年,河州賊襲陷皋蘭西古城,再敗之,長驅北山兔窩,寇大潰。其冬,會左軍渡河,連克要塞,寇退扼大東鄉,師聚而殲之。論功,予世職。

光緒元年,召署正白旗漢軍都統。會吉林馬賊竄巴彥蘇蘇,命權將軍,捕治之,漸散其黨與。明年,道員舒之翰獲譴,罪及舉主,褫職。又明年,起授青州副都統,擢察哈爾都統。五年,出為福州將軍。法人爭地越南,分兵艦窺閩疆,詔參宗棠軍事。出駐長門,誓師設伏,擊沉法艦一艘。既而防軍潰,法人登岸搦戰,伏起,轉敗為功。以故閩事壞,獨免議。十一年,詔授欽差大臣,會辦東三省練兵事。明年,以積勞卒於軍,謚果勇。予黑龍江、安徽、甘肅建祠,蘭民且樹碑志德焉。

杜嘎爾,哈勒斌氏,滿洲正藍旗人,黑龍江駐防。初從都興阿征粵寇,積勛至佐領,賜號莽賚巴圖魯。嗣以京口副都統從討甘回,規寧、靈,頗能以少擊眾。寇竄寶豐,克張家村、紅柳堡,深入沙磧,背水成軍,旬日間城復。攻寧郡,斬虜多,即於軍前調官寧夏。寧城回增建寨棚,首城南訖納家閘。與金順誘城東寇出,數敗之,乘勝破護城堤清水堡。尋隨都興阿赴奉天,調補正藍旗蒙古副都統,歷察哈爾,坐事免。光緒六年,起授烏里雅蘇臺將軍,撫士卒,恤蒙部。十四年,創發,乞休。明年,卒,謚武靖。

額爾慶額,字藹堂,格何恩氏,隸滿洲鑲白旗,墨爾根城駐防。以驍勇聞,歷遷至委參領。清水堡之役,賜號法福靈阿巴圖魯。會諸軍克狄道,授黑龍江副總管。剿金縣竄匪,擢涼州副都統。命佐關外軍事,統領吉、黑騎旅頓西湖,令寇不得西。烏城回自奎屯敗退安集海,擊卻之。光緒二年,聞白彥虎構瑪納斯南北二城回擾糧道,與總兵馮桂增、參將徐學功約期會師大河廠。額、馮二軍先行,徑薄北城,而南城回湧至,桂增負傷墜馬,寇舁入城。額爾慶額憤甚,先登陷陣,斬寇無算。因士卒傷亡多,止戰。屆期學功至,距城十餘里,見額爾慶額被創還,遂率所部救之。金順責其援不力,宗棠曰:「額爾慶額等輕進貪功,咎由自取。且先夕進攻,學功何能豫知耶?」

歷古城領隊大臣、科布多幫辦大臣。命偕參贊大臣升泰勘界,以奎峒山為科、塔兩城外蔽,哈巴河南流入中國,與俄官抗爭,始得展地定界。新疆底定,晉頭品秩。十二年,調伊犁。伊犁設副都統自此始。蒞任七年,興辦屯田,軍民輯睦。十九年,卒,恤如制。

豐紳,字漢文,吳扎拉氏,隸滿洲正白旗,吉林駐防。都興阿督江北軍,檄守揚州,以戰功歷遷至協領。克寶豐,取寧夏,數獲勝。穆圖善遣往靈州招撫,馬化龍就撫。穆圖善上其功,晉副都統。尋護寧夏將軍。時伏莽未靖,自寧城至靈州,隘口數十,為商旅來往孔道,豐紳詰奸禁暴,行旅便之。先後平陜匪西河、橫城堡,補官錦州,擢黑龍江將軍。坐事褫職。光緒間,起故官,歷綏遠城、江寧。中日事起,出駐通州,事寧回任。二十四年,卒。詔優恤,予建祠。御史彭述劾其侵冒,奪恤典。

文麟,字瑞圃,兀扎拉氏,滿洲正藍旗人。道光二十二年,考取內閣中書,遷侍讀。咸豐八年,出為甘肅蘭州道,調鎮迪。同治四年,回竄據古城,文麟上防守奇臺狀,上嘉其知大體。濟木薩者,回眾屯糧地也。文麟潛遣練勇攻克之,獲糧萬數千石。索煥章竄瑪納斯,分掠阜康、吐魯番、迪化。文麟分兵扼三臺要隘,上疏乞濟師。詔令嚴守濟木薩,援未至而哈密、奇臺相繼淪失。亟與巴里坤領隊大臣訥爾濟合兵進擊。聞寇集東路,使佐領恆昌先進,敗於奎蘇,而自請赴前敵。上怒,訶責之,下部議,坐擅離職守,降二級調用。

詔以藍翎侍衛充哈密辦事大臣。文麟遂率所部收復城垣。馬金貴、白彥虎先後圍攻,瀕危者數矣,文麟拊循士卒,卒能以饑軍驅強敵,俾纏回轉危為安。五年,遭母憂,改署任。明年,肅州回竄玉門,戰紅柳灣,敗之。回復大舉犯城關,文麟督軍嚴守,伺間出擊。圍解,乃為籌耕種,葺廬舍,訓練軍士,且戰且屯。服闋,以頭等侍衛補本官。益招哈密團首孔才至,以其練勇二百編入伍籍,遣往古城興屯修堡。後收徐學功散勇二千餘,任耕戰。於是古田、濟木薩屯政大舉。令充裨將,自是數與妥明、馬明、白彥虎相攻殺,所向皆捷。

十二年,肅城回數出關犯哈密東山,文麟令魏忠義出駐塔爾納沁堡,分扼各隘,剿撫馬賊,擒回馬五十九。旋魏軍大失利,文麟飛章自劾,被宥,益感奮,率所部進擊,力保危城。降敕褒嘉,加副都統銜。明年,彥虎援肅州,潰退安敦玉,文麟使騎旅追之,彥虎遁入山,肅州平。詔張曜等馳赴哈密,會文麟進剿。光緒二年,卒。

文麟治軍數載,囊無私蓄,與士卒同甘苦,故人皆原為盡命。及其沒也,闔營慟哭失聲。明春、富勒銘額先後狀其績以上,予褒恤,附祀新疆哈密專祠。

明春,巴羽特氏,隸蒙古正紅旗。初從勝保平河北,補前鋒校。征陜回,隸多隆阿麾下,以戰功數遷副都統。搗肅州,壓城為壘,與回相持者半載,追藍得全被重創。肅州回出掠安西、玉門、敦煌,明春馳逐三城間,三月,圍悉解,授哈密幫辦大臣。光緒二年,擢辦事大臣。時南疆平,肅纏民悉還故土。明春為晰地畝給貲糧,勸使復業。凡治道路,繕堤防,興水利,有裨民生久遠者,靡不具舉,民德之,至今猶虔祀云。十二年,卒,恤如制。

富勒銘額,佚其氏,隸滿洲鑲白旗,古城駐防。烏魯木齊陷,古城大恐。富勒銘額出與寇戰,數被創。事亟,如烏里雅蘇臺乞援,弗應,城破,全家殉焉。富勒銘額適在外,得免於難,憤詣文麟軍所,原從殺賊。紅柳灣之役,以功擢防禦。回擾安西,設計抗禦,斬虜多,並搜治西山逸匪,盡殲之,解敦煌圍,晉頭品秩,賜號堅勇巴圖魯,充古城協領。西陲告寧,置屯田,修兵房。以都統恭金堂薦,光緒十四年,授伊犁副都統。時游勇構哈薩克回寇邊,富勒銘額遣軍捕其酋,餘燼悉平。十六年,署將軍。增卡倫,整營制,索倫、錫伯、察哈爾、額魯特兵卒咸復游牧舊業。十九年,徙塔爾巴哈臺參贊大臣,練軍興屯,收還巴爾魯克山,清界置卡,其治復仿伊犁,屹然成重鎮。二十三年,乞歸。二十九年,卒,恤如制。

學功,烏魯木齊農家子。好技擊,值回亂,結健兒數十,掠回莊貲貨以自贍。遇漢民,力護之,雖邊外悍回皆已憚之矣。厥後附者益眾,集五千人,精練馬隊,每戰突陣,驟若風雨,回見之輒走。學功先後陣斬偽帥馬泰、阿奇木馬仲。仲子人得襲偽職,與妥明積不相能,妥明復以黨攻之。安酋帕夏乃約學功破吐魯番,進攻烏魯木齊,下之。妥明走綏來,數日死,帕夏遂據烏垣。

初,帕夏以學功善戰,故與交驩,冀藉其力,王哈密,以南八城,歸獻朝廷,已,知其無遠略,且百戰未得一階,益輕之,令還綏來南山。於是學功大恚,屢攻烏垣,其民人時降學功,時投人得,轉展屬役,迄不得息。同治七年,俄人構土回纏頭將襲烏垣,聲赴綏來易市,驅駝馬數千,載貨鈔至石河,去綏來八十里。學功以騎旅截之,僇數十人,餘縱還。自此俄人不敢東窺。十二年,彥虎率悍回數千分掠烏垣、綏來,學功復橫截之,殺數百人,奪橐駝五百。彥虎勢益孤。學功既任職,但承大將風指,異於初起血戰時矣。後與孔才並官至提督。孔才,哈密人。

論曰:從左宗棠立功西陲最名者,湘軍中稱二劉,豫軍中稱曜。之數人者,投袂攘難,不數月,廓清萬里,雖張騫、班超,奚多讓焉!金順、穆圖善提塞北健兒,橫行玉門、金嶺間,其志尤壯。文麟名出二人下,而招團興屯,兼任★戰,不煩國家一兵,遂定西邊,其功亦足並傳云。


\end{pinyinscope}