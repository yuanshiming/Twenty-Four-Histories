\article{志八十四}

\begin{pinyinscope}
○選舉四

△制科薦擢

制科者,天子親詔以待異等之才。唐、宋設科最多,視為優選。清代科目取士,垂為定制。其特詔舉行者,曰博學鴻詞科、經濟特科、孝廉方正科。若經學,若巡幸召試,雖未設科,可附見也。聖祖敦崇實學,康熙甲辰、丁未兩科,改試策論。既廷臣以古學不可猝辦,請仍舊制。

十七年,詔曰:「自古一代之興,必有博學鴻儒,備顧問著作之選。我朝定鼎以來,崇儒重道,培養人才。四海之廣,豈無奇才碩彥、學問淵通、文藻瑰麗、追蹤前哲者?凡有學行兼優、文詞卓越之人,不論已仕、未仕,在京三品以上及科、道官,在外督、撫、布、按,各舉所知,朕親試錄用。其內、外各官,果有真知灼見,在內開送吏部,在外開報督、撫,代為題薦。」嗣膺薦人員至京,詔戶部月給廩餼。明年三月,召試體仁閣。凡百四十三人,賜燕,試賦一、詩一,帝親覽試卷,取一等彭孫遹、倪燦、張烈、汪霦、喬萊、王頊齡、李因篤、秦松齡、周清原、陳維崧、徐嘉炎、陸葇、馮勖、錢中諧、汪楫、袁佑、硃彞尊、湯斌、汪琬、邱象隨等二十人。二等李來泰、潘耒、沈珩、施閏章、米漢雯、黃與堅、李鎧、徐釚、沈筠、周慶曾、尤侗、範必英、崔如嶽、張鴻烈、方象瑛、李澄中、吳元龍、龐塏、毛奇齡、錢金甫、吳任臣、陳鴻績、曹宜溥、毛升芳、曹禾、黎騫、高詠、龍燮、邵吳遠、嚴繩孫等三十人。三、四等俱報罷。命閣臣取前代制科舊事,查議授職。尋議:「兩漢授無常職。晉上第授尚書郎。唐制策高等特授尊官,次等予出身,因有及第、出身之目。宋分五等:一、二等皆不次擢用;三等為上等,恩數視廷試第一人;四等為中等,視廷試第三人;皆賜制科出身。五等為下等,賜進士出身。」得旨,俱授為翰林官。以光祿少卿邵吳遠為侍讀。道員、郎中湯斌等四人為侍講。進士出身之主事,中、行、評、博,內閣典籍,知縣及未仕之進士彭孫遹等十八人為編修。舉、貢出身之推、知,教職,革職之檢討、知縣及未仕之舉、貢、廕、監、布衣倪燦等二十七人為檢討。俱入史館,纂修明史。時富平李因篤、長洲馮勖、秀水硃彞尊、吳江潘耒、無錫嚴繩孫,皆以布衣入選,海內榮之。其年老未與試之杜越、傅山、王方穀等,文學素著,俱授內閣中書,許回籍。

雍正十一年,詔曰:「博學鴻詞之科,所以待卓越淹通之士。康熙十七年,特詔薦舉,召試授職,得人極盛。數十年來,未嘗廣為搜羅。朕延攬維殷,宜有枕經葄史、殫見洽聞、足稱鴻博之選者,當特修曠典,嘉予旁求。在京滿、漢三品以上,在外督、撫、學政,悉心體訪,保題送部。朕臨軒親試,優加錄用。」詔書初下,中外大吏,以事關曠典,相顧遲回。逾年,僅河東督臣舉一人,直隸督臣舉二人,他省未有應者。詔責諸臣觀望。高宗即位,再詔督促。期以一年內齊集闕下,先至者月給廩餼。

乾隆元年,御史吳元安言:「薦舉博學鴻詞,原期得湛深經術、敦崇實學之儒,詩賦雖取兼長,經史尤為根柢。若徒駢綴儷偶,推敲聲律,縱有文藻可觀,終覺名實未稱。」下吏部議,定為兩場,賦、詩外增試論、策。九月,召試百七十六人於保和殿,賜燕如例。試題首場賦、詩、論各一,二場制策二。取一等五人,劉綸、潘安禮、諸錦、於振、杭世駿等,授編修。二等十人,陳兆侖、劉藻、夏之蓉、周長發、程恂等,授檢討;楊度汪、沈廷芳、汪士鍠、陳士璠、齊召南等,授庶吉士。二年,補試體仁閣,首場制策二,二場賦、詩、論各一。取一等萬松齡,授檢討。二等張漢,授檢討;硃荃、洪世澤,授庶吉士。

自康、乾兩朝,再舉詞科,與其選者,山林隱逸之數,多於縉紳,右文之盛,前古罕聞。時承平累葉,海內士夫多致力根柢之學,天子又振拔淹滯,以示風勵,爰有保薦經學之制。乾隆十四年,詔曰:「崇尚經術,有關世道人心。今海宇升平,學士大夫精研本業,窮年矻矻,宗仰儒先者,當不乏人。大學士、九卿、督、撫,其公舉所知,不限進士、舉人、諸生及退休、閒廢人員,能潛心經學者,慎選毋濫。尋中外疏薦者四十餘人。帝為防幸進,下廷臣覆覈,得陳祖範、吳鼎、梁錫興、顧棟高四人。命呈覽著述,派翰林、中書官在武英殿各繕一部。尋授鼎、錫興國子監司業,召對勤政殿。祖範、棟高以年老不能供職,俱授司業銜。後不復舉行。

至屬車臨幸,宏獎士林,康熙四十二年、四十四年,聖祖巡幸江、浙,召試士子,中選者賜白金,赴京錄用有差。高宗六幸江、浙,三幸山東,四幸天津,凡士子進獻詩賦者,召試行在。優等予出身,授內閣中書;次者賜束帛。仁宗東巡津、澱,西幸五臺,召試之典,亦如前例。道光以後,科舉偏重時文。沿習既久,庸濫浮偽,浸失精意。三十年,候補京堂張錫庚請復開博學鴻詞科,以儲人才。禮部議以非當務之急,遂止。

洎光緒中葉,外侮孔棘,海內皇皇,昌言變法。二十四年,貴州學政嚴修請設經濟特科,下總理各國事務衙門會禮部覈議。八月,慈禧皇太后臨朝訓政,以經濟特科易滋流弊,罷之。庚子,京師構亂,乘輿播遷。兩宮怵於時局阽危,亟思破格求才,以資治理。

二十七年,皇太后詔舉經濟特科,命各部、院堂官及各省督、撫、學政保薦,有志慮忠純、規模閎遠、學問淹通、洞達中外時務者,悉心延攬。並下政務大臣擬定考試事宜。御史陳秉崧奏請力除夤緣積習,詔飭諸臣務矢至公。既三品以下京卿紛紛保送,帝覺其冗濫,適太僕少卿隆恩薦疏,上竟報寢,並命撤銷太常少卿李擢英前保諸人。二十九年,政務處議定考試之制,如廷試例,於保和殿天子親策之。凡試二日,首場入選者,始許應覆試,均試論一、策一。簡大臣考校,取一等袁家穀、張一麟、方履中、陶炯照、徐沅、胡玉縉、秦錫鎮、俞陛雲、袁勵準等九人,二等馮善徵、羅良鑒、秦樹聲、魏家驊、吳鍾善、錢鑅、蕭應椿、梁煥奎、蔡寶善、張孝謙、端緒、麥鴻鈞、許嶽鍾、張通謨、楊道霖、張祖廉、吳烈、陳曾壽等十八人。迨授官命下,京職、外任,僅就原階略予升敘,舉、貢用知縣、州佐,以視康、乾時詞科恩遇,浸不如矣。

三十四年,御史俾壽請特開制科,政務處大臣議以「孝廉方正、直言極諫兩科,皆無實際,惟博學鴻詞科,康熙、乾隆間兩次舉行,得人稱盛。際茲文學漸微,保存國粹,實為今日急務。應下學部籌議」。時方詔各省徵召耆儒碩彥。湖南舉人王闓運被薦,授翰林檢討。兩江、安徽相繼薦舉王耕心、孫葆田、程朝儀、吳傳綺、姚永樸、姚永概、馮澂等。部議以諸人覃研經史,合於詞科之選,俟章程議定,陳請舉行。未幾,德宗崩,遂寢。

孝廉方正科,始於康熙六十一年,世宗登極,詔直省府、州、縣、衛各舉孝廉方正,賜六品章服,備召用。雍正元年,詔曰:「國家敦勵風俗,首重賢良。前詔舉孝廉方正,距今數月,未有疏聞。恐有司怠於採訪,雖有端方之品,無由上達。各督、撫速遵前詔,確訪舉奏。」尋浙江、直隸、福建、廣西各薦舉二員,用知縣;年五十五以上者,用知州。其後歷朝御極,皆恩詔薦舉以為常。

乾隆元年,刑部侍郎勵宗萬言:「孝廉方正之舉,稍有冒濫,即有屈抑。從前選舉各官,鮮克公當。非鄉井有力之富豪,即宮墻有名之學霸。迨服官後,庸者或以劣黜,黠者或以贓敗。請慎選舉,以重名器。」吏部議準府、州、縣、衛保舉孝廉方正,應由地方紳士里黨合辭公舉,州、縣官採訪公評,詳稽事實。所舉或系生員,會學官考覈,申送大吏,覈實具題,給六品章服榮身。果有德行才識兼優者,督、撫逾格保薦赴部,九卿、翰、詹、科、道公同驗看,候旨擢用。濫舉者罪之。

五年,定考試例。除樸實拘謹、無他技能、不能應試者,例予頂戴,不送部外,其膺薦赴部者,驗看後,試以時務策、箋、奏各一於太和殿門內。道光間,改於保和殿,如考試御史例。

同治初元,明詔選舉,又以知縣黎庶昌條陳,諭令在京四品以上,在外督、撫、學政,各舉所知,不限紳士、布衣,以躬行實踐為先,毋得專取文詞藻麗者,濫膺盛典。其有年登耄耋,或誠樸無華,足為里閭矜式,不原來京者,州縣官歲時存問,賜以酒米。光緒六年,定自恩詔日起,予限八年,人文到部。每年二月、八月,各會驗奏考一次,逾限者止許章服榮身,不得與考。

初制授官用知州、知縣,厥後薦舉人眾,乃推廣用途,分別以知縣、直隸州州同、州判、佐雜等官及教職用。知縣得缺視拔貢,教職視大挑二等舉人,餘均分省試用序補。歷朝以來,有司奉行,第應故事。徇情冒濫之弊,臺諫屢以上聞。惟嘉慶朝湖南嚴如煜以對策第一,召見授知縣。咸豐朝湖南羅澤南以書生率湘勇越境剿賊,皆以勛績見稱於時。宣統初,各省所舉多至百數十人,少亦數十人,詔飭嚴行甄覈。選舉之風,於斯濫矣。

清代科目取士外,或徵之遺佚,或擢之廉能,或舉之文學,或拔之戎行,或闢之幕職,薦擢一途,得人稱盛,有足述焉。

太祖肇興東土,選拔英豪以輔大業,委輅杖策之士咸與擢用,或招直文館,或留預帷幄。乙卯十一月,諭群臣曰:「國務殷繁,必得賢才眾多,量能授職。勇能攻戰者,宜治軍;才優經濟者,宜理國;博通典故者,宜諮得失;嫺習儀文者,宜襄典禮。當隨地旁求,俾列庶位。」時削平諸國,設八旗制,需才亟。太宗即位,首任儒臣範文程領樞密重事。天聰八年,甲喇章京硃繼文子延慶上書,言:「我朝攻城破敵、斬將搴旗者不乏人,守境治民、安內攘外者未多見。」因疏舉漢人陳極新、刑部啟心郎申朝紀,足備任使。帝召延慶等御前,溫諭褒獎。命延慶、極新,文館錄用;朝紀仍任部事。九年,諭滿、漢、蒙古各官,薦舉人才,不限已仕、未仕,牒送吏、禮二部,具名以聞。直文館寧完我言:「古者薦舉之條,功罪連坐,所以杜弊端、防冒濫。請自後所舉之人,或功或罪,舉者同之。若其人砥行於厥初,改節於末路,許舉者隨時檢舉,免連坐。」帝嘉納焉。

世祖定鼎中原,順治初元,遣官徵訪遺賢,車軺絡繹。吏部詳察履歷,確覈才品,促令來京。並行撫、按,境內隱逸、賢良,逐一啟薦,以憑徵擢。順天巡撫宋權陳治平三策,首廣羅賢才以佐上理,並薦故明薊遼總督王永吉等。詔廷臣各舉所知。一時明季故臣如謝升、馮銓、黨崇雅等,紛紛擢用。中外臣工啟薦除授得官者,不可勝數。嗣以廷臣所舉,類多明季舊吏廢員,未有肥遯隱逸逃名之士。詔自今嚴責舉主,得人者優加進賢之賞,舛謬者嚴行連坐之罰。薦章止以履歷上聞,才品所宜,聽朝廷裁奪。儻以貲郎雜流及黜革青衿、投閒武弁,妄充隱逸,咎有所歸。若畏避連坐,緘默不舉,治以蔽賢罪。二年,陜西、江南平,詔徵山林隱逸,並故明文、武進士、舉人。山東巡撫李之奇以保薦濫及貲郎,詔旨切責。十三年,江南巡撫張中元薦故明進士陸貽吉、于沚,帝親試之。是年復詔各省舉奏地方人才,給事中梁鋐言:「皇上寤寐求才,詔舉山林隱逸,應聘之士,自不乏人。然採訪未確,有負盛舉。如江南舉呂陽,授監司,未幾以贓敗;山東舉王運熙,授科員,未有建明,以計典去。呂陽等豈真抱匡濟之才,不過為梯榮之藉耳。山林者何?謂遠於朝市也。隱逸者何?謂異於趨競也。必得其人,乃當其位。請飭詳加採訪。」疏入,報聞。

順、康間,海內大師宿儒,以名節相高。或廷臣交章論薦,疆吏備禮敦促,堅臥不起。如孫奇逢、李顒、黃宗羲輩,天子知不可致,為嘆息不置,僅命督、撫抄錄著書送京師。康熙九年,孝康皇后升祔禮成,頒詔天下,命有司舉才品優長、山林隱逸之士。自後歷朝推恩之典,雖如例行,實應者寡。

初制,督、撫升遷離任時,薦舉人才一次。嗣令歲一薦舉,部議大省限十人,小省限三四人,後復改二年薦舉一次。自順治十八年停差巡按,乃定各省巡撫應舉方面有司、佐貳、教官員額,總漕、總河應薦方面有司、佐貳額,亦著為例。康熙二年,御史張吉午奏:「三年考滿之法,一、二等稱職者,即系薦舉,請罷督、撫二年薦舉例。」從之。六年,停考滿。用給事中李宗孔言復薦舉,與卓異並行。先是漕、河薦舉例停。十二年,漕督帥顏保請復舊例,每年得舉劾屬吏示勸懲。部議行。因疏薦糧道範周、遲日巽、知縣吳興祚。詔擢興祚福建按察使。

聖祖親政,銳意整飭吏治,屢詔群臣薦舉天下廉能官。十八年,左都御史魏象樞疏薦清廉,原任侍郎高珩、達哈塔、雷虎、班迪,大理卿瑚密色,侍讀蕭維豫,郎中宋文運,布政使畢振姬,知縣張沐、陸隴其等十人。得旨分別錄用。並諭陸隴其廉能之員,宜任繁劇,如直隸清苑、江蘇無錫等縣,庶可表見其才。十九年,福建巡撫吳興祚薦按察使於成龍天下廉能第一,遷布政使,尋擢直隸巡撫。二十年入覲,帝溫諭褒美。問屬吏中亦有清廉者否?成龍以知縣謝錫袞、同知何如玉、羅京對。未幾,調成龍兩江總督。瀕行,疏薦直隸守道董秉忠、通州知州於成龍、南路通判陳大棟、柏鄉知縣邵嗣堯、阜城知縣王燮、高陽知縣孫宏業、霸州州判衛濟賢,並堪大用。會江寧知府缺,詔即以通州知州於成龍擢補。不數年,擢直隸巡撫。同時兩於成龍,先後汲引,並以清操特邀帝眷,時論稱之。二十三年,諭部臣保舉應補關差,僉以「有才及謹慎者不乏人,而操守實難知」對。帝曰:「清操如何可廢?如郝浴居官甚好,猶侵蝕錢糧,魏象樞曾薦郝浴,此事安能豫知!朕信部院堂官清操而委任之,堂官亦信司官而委任之。但將有守之人舉出,被舉者自能效力。」是年九卿、詹事、科、道遵旨疏舉清廉:直隸巡撫格爾古德,吏部郎中蘇赫、範承勛,江南學道趙侖,揚州知府崔華,兗州知府張鵬翮,靈壽知縣陸隴其等。二十六年,帝嘉直隸巡撫於成龍清廉,命九卿各舉廉吏如成龍者。大學士等薦雲貴總督範承勛、山西巡撫馬齊、四川巡撫姚締虞。帝謂承勛等居官皆優,但尚有勉強之意。成龍則出自誠心,毫無瞻顧。命加成龍太子少保銜,以勸廉能。四十年,敕總督郭琇、張鵬翮、桑額、華顯,巡撫彭鵬、李光地、徐潮薦道、府以下,知縣以上,清廉愛民者,勿計罣誤降罰,勿拘本省鄰屬,具以名聞。時天子廣厲風節,群士慕效,吏治丕變。循吏被薦膺顯擢者,先後踵相接。

先是廷臣會推廣西按察使缺,吏部侍郎胡簡敬,淮安人,以推舉淮揚道高成美違例獲譴,至是申禁九卿毋得保舉同鄉及本省官,復限每人歲舉毋逾十人。五十三年,尚書趙申喬舉潮州知府張應詔能耐清貧,可為兩淮運使。帝曰:「清官不系貧富,張伯行家道甚饒,任所日用皆取諸其家,以為不清可乎?一心為國即好官,或操守雖清,不能辦事,亦何裨於國?」

六十一年,世宗嗣位。諭曰:「知人則哲,自古為難。朕臨御之初,簡用人才,或品行端方,或操守清廉,或才具敏練,諸大臣密奏所知。勿避嫌徇私,沽名市恩,有負諮詢。」又以道、府、州、縣,親民要職,敕總督舉三員,巡撫舉二員,布、按各舉一員,將軍、提督亦得舉一員,密封奏聞。雍正四年,以各省所舉未能稱旨,詔切責之。令各明舉一人,不得雷同。時薦賢詔屢下,帝綜覈名實,賞罰必行。七年,以督、撫、布、按,為全省表率。命京官學士、侍郎以上,外官籓、臬以上,各密保一人,不拘滿、漢,不限資格,即府、縣中有信其可任封疆大僚,亦許列薦剡。

高宗重視親民之官,乾隆二年,諭仿雍正時例,督、撫、布、按,各密舉一、二人。次年,復命大學士、九卿舉堪任道、府人員,露章啟奏。八年,詔大學士舉編、檢能任知府者。十四年,命侍郎以上舉能任三品京堂者,尚書以上舉能任侍郎者。其時明揚、密保,並行不廢。科、道行取,自康熙七年復舊制。詔督、撫舉親民之官,賢能夙著者,親加選用。二十九年,詔九卿各舉所知。尚書王騭舉清苑知縣邵嗣堯,李天馥舉三河知縣彭鵬、靈壽知縣陸隴其,徐元文舉麻城知縣趙蒼璧。及廷推時,帝復問左都御史陳廷敬,廉者為誰?廷敬亦以隴其、嗣堯天下清官為言。時同舉十二人,俱用科、道。得人為最。乾隆四年,吏部奏請行取,高宗命尚書、都御史、侍郎於各部屬,州、縣內,秉公保舉,如康熙二十九年例。次年,諭「聖祖時如湯斌、陸隴其學問純正,言行相符,陳瑸、彭鵬操守清廉,治行卓越。天下之大,人材之眾,豈無與數人頡頏者?大學士、九卿其公舉備採擇」。

七年,帝思骨鯁質樸之士,如古馬周、陽城起布衣為御史者,詔大學士、九卿及督、撫,勿論資格,列名舉奏。嗣諸臣奏到,下吏部定期考試。明年二月,考選御史,試以時務策,帝親取中書胡寶瑔第一。引見,寶瑔、塗逢震等十人用御史,沈瀾發江南補用。既而從御史李清芳奏,選用御史,令吏部將合例人員奏請考試。於是保薦御史例罷。清代未設直言極諫之科,而選擇言官至為慎重,裨益政治,非淺鮮也。

自康、乾兩朝,敦尚實學,一時名儒碩彥,膺薦擢者,尤難悉數。康熙十七年,聖祖問閣臣,在廷中博學能詩文者孰為最?李霨、馮溥、陳廷敬、張英交口薦戶部郎中王士禎,召對懋勤殿,賦詩稱旨,授翰林院侍講。部曹改詞臣,自士禎始。三十三年,詔大學士舉長於文學者,王熙、張玉書疏薦在籍尚書徐乾學、左都御史王鴻緒、少詹事高士奇。召來京修書。乾學未聞命卒,詔進呈遺書,並召其弟秉義來京。四十五年,大學士李光地薦直隸生員王蘭生入直內廷,尋賜舉人、進士,授編修,洊躋卿貳。歷康、雍、乾三朝,凡天祿秘書,靡不與校勘之役。同時江南何焯,亦以寒儒賜舉人、進士,直南書房,授編修。被劾解官,仍直書局。亦光地薦也。雍正中,侍郎兼祭酒孫嘉淦薦舉人雷鋐學行,為國子監學正。乾隆初,尚書管監事楊名時薦進士莊亨陽、舉人潘永季、蔡德峻、秦蕙田、吳鼐,貢生官獻瑤、王文震,監生夏宗瀾等,潛心經學,並為國子監屬官。三十八年,詔開四庫館。延置儒臣,以翰林官纂輯不敷,大學士劉統勛薦進士邵晉涵、周永年,尚書裘曰修薦進士餘集、舉人戴震,尚書王際華薦舉人楊昌霖,同典秘籍。後皆改入翰林,時稱「五徵君」。此其著者也。

嘉慶初,和珅敗,仁宗下詔求賢。諭滿、漢大臣,密舉操守端潔、才猷幹濟、居官事跡可據者,降敕褎擢廉吏劉清,風厲天下。十九年,御史卓秉恬請嚴禁濫保,帝是之。宣宗即位,尚書劉鐶之薦起名儒唐鑒,授廣西知府。四川總督蔣攸銛薦川東道陶澍治行第一,擢按察使。澍好臧否人物,開籓皖中,入覲論奏,侃侃多所舉劾。宣宗疑之。密諭巡撫孫爾準察其為人,爾準條列善政,密疏保薦。遂獲大用,擢兩江總督。臨歿遺疏薦粵督林則徐繼己任。澍以知人稱,咸、同中興諸名臣,多為所識拔。

文宗嗣位,詔求直言。侍郎曾國籓疏陳:「本原至計,尤在用人。人材有轉移之道,培養之方,考察之法。」帝嘉納之。詔中外大臣薦舉人才。大學士穆彰阿奏保宗室文彩,聶澐。特旨用京堂。大學士潘世恩疏薦前總督林則徐、按察使姚瑩、員外郎邵懿辰、中允馮桂芬。尚書杜受田首薦則徐及前漕督周天爵。詔起則徐督師,天爵巡撫廣西。侍郎曾國籓薦太常少卿李棠階、郎中吳廷棟、通政副使王慶雲、江蘇淮揚道嚴正基、浙江知縣江忠源。尚書周祖培亦薦棠階、廷棟及郎中易棠等,多蒙擢用。雲貴總督吳文鎔、貴州巡撫喬用遷薦知府胡林翼,擢道員。

咸豐五年,以各省用兵,詔採訪才兼文武、膽識出眾之士。御史宗稷辰疏薦湖南左宗棠,浙江姚承輿,江蘇周騰虎、管晏,廣西唐啟華。命各督、撫訪察,送京引見。是時海內多故,粵寇縱橫。文慶以大學士直樞廷,屢密請破除滿、漢畛域,用人不拘資地。謂漢人來自田間,知民疾苦,熟諳情偽,辦賊當重用漢人。國籓起鄉兵擊賊,戰失利,謗議紛起。文慶獨謂國籓忠誠負時望,終當建非常功,宜專任討賊。又嘗奇林翼才略,林翼以貴州道員留楚帶勇剿賊,國籓薦其才堪大用,勝己十倍。一歲間擢湖北巡撫,文慶實中主之。袁甲三督師淮上,駱秉章巡撫湖南,文慶薦其才,請勿他調,以觀厥成。時論稱之。七年,林翼奏興國處士萬斛泉及其弟子宋鼎、鄒金粟,砥礪廉隅,不求仕進,請予獎勵。詔賞斛泉等七品冠服有差。時軍事方殷,迭飭疆吏及各路統兵大臣奏舉將才。林翼舉左宗棠,予四品京堂,襄辦國籓軍務。沈葆楨、劉蓉、張運蘭,命國籓、林翼調遣。他如塔齊布、羅澤南、李續賓、李續宜、彭玉麟、楊岳斌等,俱以末弁或諸生,拔自戎行,聲績爛然。曾、胡知人善任,薦賢滿天下,卒奏中興之功。

穆宗踐阼,以軍興後吏治廢弛,特擢天津知府石贊清為順天府尹,諭各省訪察循良,有伏處山林、德行醇備、學問淵通之士,督、撫、學政據實奏聞。尋國籓疏稱常州士民尚節義,城陷與賊相持。其士子多讀書稽古。如候選同知劉翰清,監生趙烈文、方駿謨、華蘅芳,從九品徐壽等,若使閱歷戎行,廓其聞見,有裨軍謀。詔譚廷襄、嚴樹森、左宗棠、薛煥訪求,遣送國籓軍營錄用。

同治元年,諭廷臣曰:「上年屢降旨令保舉人才,各督、撫已將政績卓著人員登諸薦牘。在京如大學士周祖培,大學士銜祁■A4藻、翁心存,協辦大學士倭仁,侍郎宋晉、王茂廕,科道高延祜、薛春藜、郭祥瑞等,各有薦舉。人臣以人事君,不必俟有明詔,始可敷陳。其各臚列事實,秉公保奏。」復屢諭國籓保薦督撫大員。國籓言:「封疆將帥,惟天子舉措之。四方多故,疆臣既有征伐之權,不當更分黜陟之柄,宜防外重內輕之漸,兼杜植私樹黨之端。」帝優詔褒答。

二年,河南學政景其濬奏副貢生蘇源生等學行,授本省訓導。命各學臣訪舉經明行修之士,酌保數人,不為恆制。九年,浙江學政徐樹銘,以採訪儒修,疏薦已革編修俞樾,請賞還原銜,送部引見;秀水教諭譚廷獻、舉人趙銘、江西拔貢楊希閔等,比照召試博學鴻詞例,予廷試。帝以樹銘私心自用,下部嚴議,鐫四級。此因薦舉獲譴,乃其變也。光緒七年,兩廣督臣張樹聲、撫臣裕寬,薦在籍紳士山西襄陵知縣南海進士硃次琦,國子監典籍銜番禺舉人陳澧篤行。詔予五品卿銜,以勵績學。

十年,以外釁迭啟,時事日艱。諭大學士、六部、九卿、直省將軍、督、撫:「無論文武兩途,有體用賅備,謀勇俱優,或諳習吏治兵事,熟悉中外交涉,或善制船械,精通算術,或饒有機智,饒勇善戰,或諳諫水師及沿海情形者,廣為訪求,具實陳奏。」二十一年,訪求奇才異能,精天文、地輿、算法、格致、制造學者。二十四年,翰林院侍讀學士徐致靖疏薦工部主事康有為、刑部主事張元濟、湖南鹽法長寶道黃遵憲、江蘇知府譚嗣同、廣東舉人梁啟超,特予召見。徵遵憲、嗣同至京,賞啟超六品銜,任譯書局事。時德宗親政,激於外勢,亟圖自彊。詔求通達時務人才,中外紛紛薦舉。而草茅新進之臣,刻勵求新,昌言變法矣。未幾黨禍起,慈禧皇太后訓政,有為竄海外,其弟廣仁及御史楊深秀、軍機章京譚嗣同、林旭、楊銳、劉光第棄市,致靖以黨附下獄禁錮,復追論原保諸臣罪。御史宋伯魯、湖南巡撫陳寶箴,開缺戶部尚書、協辦大學士翁同龢,俱削官永不敘用。禮部尚書李端棻謫戍邊,內閣學士張百熙下部議處。其他言新政者,斥逐殆盡。

迨庚子京師遘亂,越年和議成,兩宮西幸回鑾,時事日棘。三十三年,詔中外大臣訪求人才,不拘官階大小,有無官職,確知才堪大用,及擅專長者,切實薦舉。派王大臣察驗詢問,出具考語,召見。於時被薦人員,分起赴京,除官錄用者,至宣統間猶未已。然自光緒之季,改訂官制,增衙署,置官缺,破格錄用人員輒以千數,薦擢亦太濫矣。宣統元年,御史謝遠涵言:「變法至今,長官但舉故舊,士夫不諱鉆營。請嚴定章程,以貪劣聞者,反坐薦主,加以懲處。」疏下所司而已。

薦舉不拘流品。清代才臣,以佐雜洊躋開府者,如雍正間之李衛、田文鏡,乾隆間之楊景素、李世傑,政績最著。厥後捐納日廣,起家雜流,膺顯擢者無算,其人大都饒有幹局,以視科目循資遷轉,以資格坐致高位,蓋不侔也。薦舉之尤異者,康熙初,陜西提督王進寶,薦其子用予材武可勝副將,後以功擢總兵,父子同建節鉞。雍正間,雲南總兵趙坤擢貴州提督,請以其子秉鐸為貴州提標參將,帝允所請。孫嘉淦為祭酒,舉其弟揚淦為國子監學正,湖南衡永郴桂道汪榯,且薦其父原任刑部司官澐學問優裕,政事練達,授四川知府。此則舉不避親,其破除成例又如此。

徵闢幕僚,雍正元年詔吏部,嗣後督撫所延幕賓,將姓名具奏,稱職者題部議敘,授之職位,以示砥礪。乾隆元年,侍郎吳應棻以鼓勵賢才,請立勸懲之法。洎道光間,幕友濫邀甄敘,臺諫屢以為言,詔督、撫、鹽政,一切議敘,不許保列幕友,並嚴禁本省屬員濫充,違者吏部查參議處。然康熙時,布衣陳潢佐靳輔治河,特賜僉事道銜。雍正時,方觀承為定邊大將軍平郡王記室,以布衣召見,賜中書銜。乾、嘉間,名臣如王傑、嚴如煜、林則徐輩,皆先佐幕而後通籍。迨咸、同軍興,左宗棠、李鴻章、劉蓉等,多以幕僚佐績戎旃,成中興之業。曾國籓總制軍務,幕府號多才,賓從極一時人選,尤卓卓可紀者也。


\end{pinyinscope}