\article{列傳一百三十四}

\begin{pinyinscope}
楊遇春子國楨吳廷剛祝廷彪游棟雲羅思舉桂涵包相卿

楊遇春,字時齋,四川崇慶人。以武舉效用督標,為福康安所識拔。從征甘肅石峰堡、臺灣、廓爾喀,咸有功,累擢守備。

乾隆六十年,調赴苗疆,力戰解嗅腦圍,進援松桃,獨取道樟桂溪,山險寨密,率敢死四十人為前鋒,由間道縱馬入賊屯,呼曰:「大兵至矣!降者免死。」賊相顧錯愕;復呼曰:「降者跪!」於是跪者數千人,直抵城下,圍遂解,賜花翎。復解永綏圍,賜號勁勇巴圖魯。首逆吳半生就擒,擢游擊。額勒登保攻茶山,為賊所圍;遇春率壯士沖擊,奪據對山,縱橫決蕩,當者輒靡。福康安望之驚嘆,立擢參將。復乾州,擢廣東羅定協副將。

苗平而教匪起,嘉慶二年,從額勒登保赴湖北剿覃加耀、林之華,破芭葉山,連敗之長陽、宣恩、建始、恩施。加耀竄終報寨,峭巖陡絕,夜縋而登,擒加耀及其黨張正潮。三年,從額勒登保赴陜,敗李全於藍田,又敗高均德於紫溪嶺。五月,還湖北。張漢潮竄穀城,兜擊,大敗之,又敗之竹山菩提河,追躡入陜,敗之於平利孟石嶺。九月,敗高均德、李全於廣元吳家河。丁父憂,賜金治喪,命墨絰隨征。迭破羅其清於觀音坪、大鵬寨、青觀山,其清就擒,擢甘肅西寧鎮總兵。四年,從額勒登保斬蕭占國、張長庚,獲王光祖,斃冷天祿,功皆最,威震川、陜,婦孺皆知其名。追剿張子聰,自夏徂秋,迭敗之於梁山、雲陽、太平、開縣、通江間。子聰被追急,數與樊人傑、龔建、冉天元合,最後欲合王登廷。登廷踞馬鞍寨,進攻克之,躡追迭擊,擒其黨靳有年於土丫子,斬阮正漋於廣元雲霧山。

至冬,登廷由陜入川,與冉天元合。額勒登保率遇春與穆克登布會擊之於蒼溪貓兒埡。穆克登布違約,先期進,挫敗,遇春據廢壘力拒,燃草炬擲山下,戰徹夜,幸得全師,迭擊皆獲勝。登廷孑身至蒲江,為鄉團擒獻,斬之。五年,擢甘州提督,偕穆克登布破張天倫於兩當,又從額勒登保追楊開甲於商、雒,扼龍駒寨,殲張漢潮餘黨劉允恭、劉開玉,予雲騎尉世職。

遇春與穆克登布為經略左、右翼長,議每不合,自蒼溪戰後,益不相能。額勒登保等疏言:「諸將中惟遇春謀勇兼優,可當一面。請益所部兵,與經略、參贊分路剿賊。」遂以提督別領偏師,沿渭西上,剿汧、隴之賊。五月,擊伍金柱於漢陰手板巖及銅錢窖,戰方酣,楊開甲從間道突至,腹背受敵,自午至酉,圍愈急,有白袍賊手大旗,直犯遇春,相去咫尺,忽墜馬,則為後隊護槍所斃,乃金柱悍黨龐洪勝也。賊驚潰,額勒登保兵亦會,追賊至洋縣茅坪,斬開甲,又擒陳傑於大石阪。八月,斬金柱於成縣峽溝,斬宋麻子於鳳縣潘家溝。六年,破冉學勝於石泉石塔寺。高天德、馬學禮、王廷詔為大軍所驅,竄五郎壩。遇春方追學勝,偵知之,乘夜掩擊,天德等分竄,乃由斜峪關躡擊,阻其入甘肅之路,復破賊於鋼鋪廠,一晝夜馳四百里,追及廷詔於川、陜界鞍子溝擒之,天德、學禮竄禪家巖。遇春料賊由寧羌奔逸,急由斜谷趨二郎壩,設伏龍洞溪,賊果至,俘斬殆盡,二賊就擒,晉騎都尉世職。是役,釋降眾健者八百人,編為一隊,皆原效死。會經略檄合剿冉學勝,獲諜,得賊虛實,謂降眾曰:「汝等立功贖罪,此其時矣!」至紫陽天池山,賊於伏莽中突起,八百人力戰,沖賊為數段,遂大捷。張天倫糾五路賊聚洵陽,學勝復與合,大破之於孫家坡。追賊入川,擒冉天泗、王士虎於通江報曉埡。士虎故劇盜,專劫寨峒避大軍。遇春夜往捕,適賊由他路襲營,遇春不回救,伏巢外候賊歸,擒斬無遺。賊中有名號者剿除幾盡,餘匪以老林為藪。遇春專任搜剿,以遲緩,嚴詔切責。七年秋,殲茍文明,調固原提督。尋以大功戡定,詔遇春功尤著,殲首逆獨多,晉二等輕車都尉。

八年,丁母憂,賜金,給假四十日。茍文明餘黨茍文潤集千餘人,皆獷悍,蹂躪漢江左右,諸軍久役不振。遇春至,乃奮,連戰紅山寺、平溪河,殲之,賊氛漸清。十年,凱撤,詔許回籍補持母服百日,假滿入覲。會寧陜鎮兵變。鎮兵新設,入伍者多鄉勇、降賊,不易制。總兵楊芳赴固原攝提督,因停給鹽米銀,發包穀充糧,遂戕副將、游擊,劫庫獄以叛。遇春行至西安,聞變,偕巡撫方維甸馳往。詔德楞泰赴陜治其事,命遇春扼方柴關,賊銳甚,兵交數失利。賊首蒲大芳望見遇春,下馬遙跪,哭訴營官蝕餉狀,遇春曉以順逆,知可以義動,與楊芳謀,同主撫。諸帥尚猶豫,遇春按兵緩攻,令芳單騎入賊營諭之。越數日,大芳竟縛倡逆之陳達順、陳先倫詣遇春降。遂率大芳邀擊餘賊於江口,斬其渠硃先貴。德楞泰疏陳叛兵窮蹙乞命,請釋歸伍,詔斥縱叛廢法,降遇春寧陜鎮總兵,大芳等二百餘人皆戍新疆。十三年,入覲,命兼乾清門侍衛,仍授固原提督。

十八年,天理教匪李文成踞滑縣,命陜甘總督那彥成討之,以遇春為參贊。賊萃精銳道口鎮,遇春率親兵八十人,沿運河西進覘之,遇賊數千,即突擊,賊闢易,追渡河,擒斬二百;收隊少二人,復沖入賊陣,奪二尸還,賊為喪氣,遂斷浮橋,焚渡船,進攻,賊望見輒靡。尋克道口,復擊走桃源、輝縣援賊,合圍滑城,用地隧轟破之,文成自焚死。十二月,滑縣平,封二等男爵,賜黃馬褂。

陜西南山賊萬五倡亂,十九年正月,移師往討,斬萬五及其黨,凡兩越月蕆事,晉一等男。陛見,仁宗慰勞有加,命至膝前,執其手曰:「朕與卿同歲,年力尚強,將來如有軍務,卿須為朕獨當一面。」手賜珍物,見遇春長髯,稱美者再。時遇春弟逢春為曹州鎮總兵,命繞道視所練兵。宣宗即位,加太子少保,賜雙眼花翎。道光五年,署陜甘總督。

六年,回酋張格爾叛,詔遇春率陜、甘兵五千馳赴哈密。尋命大學士長齡為揚威將軍,遇春為參贊,會兵阿克蘇進剿。七年二月,連敗賊於洋阿爾巴特、沙布都爾、阿瓦巴特,擒斬數萬,追至渾河,距喀什噶爾十餘里,賊悉眾抗拒,列陣二十餘里。會大風霾,前隊迷道,未即至,將軍欲退屯十餘里,須霽而進,遇春不可,曰:「天贊我也,賊不知我兵多少,又虞我即渡,時不可失!且客軍利速戰,難持久。」乃遣千騎繞趨下游牽賊勢,自率大兵乘晦霧驟渡上游,砲聲與風沙相並,乘勢沖入賊陣,賊大奔。三月朔,遂復喀什噶爾,甫旬日,英吉沙爾、葉爾羌、和闐以次復,加太子太保。張格爾遠遁,詔遇春先入關。八年正月,楊芳擒張格爾於鐵蓋山,遇春入覲,捷音適至,帝大悅,賜紫韁,實授陜甘總督,圖形紫光閣。遇春坐鎮陜、甘凡十年,務持大體,不輕更張,討蒐軍實,鎮馭邊疆,皆有法。十五年,以老予告歸,召至京,陛辭,晉封一等昭勇侯,食全俸,禦制詩書扇賜之。十七年,卒於家,贈太子太傅、兵部尚書,賜金治喪,入祀賢良祠、鄉賢祠,謚忠武。

遇春結發從戎,大小數百戰,皆陷陣冒矢石,未嘗受毫發傷。仁宗詢及,嘆為「福將」。治軍善於訓練,疲卒歸部下即膽壯,或精銳改隸他人,仍不用命。將戰,步伐從容,雖猝遇伏,不至失措。俘虜必入賊三月以外始誅,老稚皆赦免。馭降眾有恩,尤得其死力。操守廉潔,治家嚴整,子弟皆謹守其家風。

弟逢春,久隨軍中,積功授重慶鎮標游擊。後從賽沖阿平陜西洋縣匪,累擢山東曹州鎮總兵,調兗州鎮。

子國佐,四川茂州營都司,加副將銜。

國楨,字海梁。以舉人入貲為戶部郎中,出任潁州知府,累擢河南布政使。洎回疆底定,宣宗推恩,就擢巡撫,疏請留其父部將訓練河南兵。武臣父子同時膺疆寄,與趙良棟、岳鍾琪兩家比盛焉。遇春歿,襲侯爵,服闋,授山西巡撫,歷官皆有聲。道光二十一年,擢閩浙總督。尋以腿疾乞歸,在籍食俸,數年卒。

遇春尤知人,獎拔如不及。識楊芳於卒伍中,力薦之,卒為大將,勛名與之埒,天下稱「二楊」,自有傳。部曲多洊至專閫,著者曰吳廷剛、祝廷彪、游棟云。

廷剛、四川成都人。由行伍征苗,擢守備。從遇春剿教匪,善偵敵。嘉慶四年,破王登廷於青龍坪,擢都司。五年,剿楊開甲、辛聰於龍駒寨,倍道掩襲,敗賊輝塔、洞寨。伍金柱踞手板巖,輕騎往探,獲賊諜,馳報,得大捷。追張天倫至馬桑壩,高天升、戴仕傑由箭桿山突出,迎擊,大敗之,擢游擊。六年,孫家坡之戰,分追餘賊至關埡,奪據山頂,賊多墜崖死,擢參將。追高見奇、姚馨佐至通江,山徑紆險,棄馬行,見賊數十人,奪路走,擒其酋,乃辛斗也。通江賊李彬夜竄熊家灣,廷剛先至,橫沖賊為二,後賊回竄,與大軍夾擊,大破之,擒魏中均、茍朝萬、王士元。七年,迭擊辛聰、劉永受於老君嶺、菜子坪、太平峒、燕子巖,賊四竄;偕祝廷彪徒步入山,追賊田峪,將歸隊,過桃川沙壩,見山樹紅旗,疑之,偵知賊首茍文明冒官軍,奮擊敗之,分路要截,擒斬數百。文明將入川,追至花石巖,見山上炊煙起,麾兵仰攻,文明知不能脫,擲跳巖下,就斬之;又擒殲茍七麻子、吳廷詔、張芳等。八年,搜剿南山餘匪,往來老林。九年,賊聚川、陜邊界,廷剛至桃木坪,賊乘霧沖撲,受矛傷,窮追越楚境,迭敗之石渣河、亢喜坡。進攻馬鞍山,賊伏陡崖,徑馳上,擒賈燦華、茍文華、王振、謝尚玉等。賊遁老山,偕祝廷彪選健卒持乾崿輕騎躡剿,遍歷險僻。至十年,擒斬殆盡,擢甘肅涼州鎮總兵,調漢中鎮。十八年,剿三才峽匪萬五,別賊起古子溝,分兵克之。萬五乘間連踞峒寨,敗之於袁家莊、平木山梁,分兵抄襲,設伏沙壩,擒其黨周在庭、周之順。萬五窮蹙,竄盩厔山中,為他軍所擒。進剿餘黨,擒尹朝貴、劉功。十九年,事平,詔廷剛首先進剿,功最,加提督銜。尋擢廣東陸路提督,未至,卒。詔念前勞,予優恤,謚壯勤。

廷彪,四川雙流人。由行伍征苗,擢守備。嘉慶五年,從遇春殲劉元恭、劉開玉,擢都司。六年,擒王廷詔,擢游擊。七年,剿賊平安寨,設伏長溝,乘夜掩擊,中矛傷,裹創力戰,斃茍文清於陣;偕吳廷剛殲茍文明於花石巖,擒茍文齊於鱉鍋山:擢參將。又破張世云於北溝口。八年,迭擊賊於老林、小岔溝、白果園,擒冉璠。九年,偕羅思舉追賊入界嶺老林,攻望都觀賊巢。從遇春擊賊鳳凰寨、壩口、馬鞍山,並多斬獲。十一年,擢漢中協副將。值寧陜兵變,赴南山截剿。甫定,瓦石坪周士貴復起,偕羅思舉合擊擒之,賜號迅勇巴圖魯。十四年,擢甘肅寧夏鎮總兵,調陜西西安鎮。十九年,剿三才峽匪萬五餘黨,偕吳廷剛擒尹朝貴於木瓜園。分路剿賊黃草坪,毀其巢,追入手板巖老林,賊詭降,設伏,擒其渠陳四,擢湖南提督。道光三年,內召,授頭等侍衛,仍兼提督銜。以熟悉南山情形,未幾,復授西安鎮總兵。在任凡十年,擢貴州提督,調浙江提督。二十年,英吉利兵陷定海,守招寶山,吏議褫職,詔留任。尋以年老休致,歸,卒於家。

廷彪果敢力戰,善撫士卒,當時降眾多生事,所部帖然,世稱之。

棟云,四川巫山人,寄籍華陽。以武舉補把總,從征廓爾喀、苗疆,積功累擢寧羌營游擊。從額勒登保剿教匪,與遇春偕,後乃為其部將。攻終報寨先登,功最。嘉慶三年,從遇春追張漢潮、詹世爵、李槐等,由漢中入川境。諸軍合剿於隘口,棟云據高俯擊,斷槐手,箭貫世爵胸,皆斃。漢潮竄梅子關,迎擊,敗之;又連敗之巴東及陜境兩河關。設伏王家河,賊至,痛殲之,窮追至河南盧氏,漢潮遁。四年春,敗賊涼沁河,兵僅五百,斬獲三百餘級。賊走龍駒寨,屯康家河,棟云躡之,忽山坳突出悍賊,中矛傷,戰愈力,射殪執旗者,賊乃卻。事聞,特詔嘉獎。四月,漢潮踞紅門寺,冒雨出間道擊走之,扼之黑龍口,與明亮、興肇為犄角。谿水漲,潛涉上游襲擊,賊大潰,又冒雨克欒家河。八月,敗賊犁澤坪,竄石峽子,棟云設伏野雞溝,與大兵夾擊,漢潮窮蹙入老林;分路追剿,擒李潮於張家坪,而漢潮已為明亮擊斃,至是獲其尸:擢甘肅提標參將。五年,擢安慶協副將。敗冉學勝於沔陽,連擊高天德、馬學禮於獅子梁、櫻桃埡;六年春,復破之於五郎坪、鳳凰山。天德、學禮為遇春所擒。餘黨踞八斗坪,棟雲分隊襲之,擒羅鳳友;又破伍金柱餘黨於三岔坪。至七年春,所部凱撤,擢狼山鎮總兵,父憂去官。十一年,授河州鎮。西寧番族出擾,棟云專剿貴德一路,破賊甘壩山,連敗之六哈圖河、什尖里、斡汪科合山,遂克沙卜浪賊巢,進至紅露井。番僧昂賢率十二族降,焚其巢,番境悉平。以母憂去,起補陜安鎮,調寧夏鎮。十八年,從遇春剿南山匪,數戰於隴州、沔陽,擒賊渠。二十三年,標弁江芝誣訐棟雲侵餉,下總督察治,得白,抵芝罪。棟雲坐私役兵丁,褫職,詔赴遇春軍委用。道光初,署鹽茶都司,乞病歸,卒。

羅思舉,字天鵬,四川東鄉人。少有膽略,蹻捷,逾屋如飛。貧困,為盜秦、豫、川、楚間。結客報仇,數殺不義者。遭厄,幸不死,久之自悔。教匪起,充鄉勇,誓殺賊立功名。

王三槐踞東鄉豐城為巢,眾數萬,官軍莫敢擊,出掠羅家壩,團勇不習戰。思舉見賊前鋒數百,詭呼曰:「數十人耳!」眾氣倍,擊走之。游擊羅定國使偵豐城,還報:「請率死士夜搗之,官兵外應,可一舉滅。」定國以為狂。思舉憤,獨攜火藥往,乘烈風燔之。賊黑夜相蹂殺,走巔巖,踣死無算,遂奔南壩場。是役,一夫走賊數萬,聲震川東,總督英善給七品軍功,隸副都統佛住。川賊以羅其清、冉文儔、徐天德、王三槐為最強,徐、王二賊合窺東鄉。思舉請佛住嚴備,勿聽。乃為知縣劉清說其清降,知其詐,馳歸,則賊已陷東鄉,戕佛住,清亦拔營去。時嘉慶二年正月也。調苗疆凱旋兵猶未至,總兵索費音阿率甘肅兵來援,用思舉策,扎營大團堡,開壕樹柵,埋火藥,誘賊入,轟之,遂奪金峨寺賊巢,復東鄉。賊竄重石子、香爐坪,德楞泰、明亮並以兵會,思舉請仍如破豐城事,德楞泰壯之。只身夜入賊營,會大雨,火藥不燃,賊覺,懼而遁。自是常將鄉勇,分路為奇兵,與官軍犄角,或為前鋒,殲孫士鳳於凈土庵,又敗賊於峨城山,皆以火攻劫營獲捷。

時川賊與襄陽賊齊王氏等合,雲陽教黨亦起應。獲諜,知王三槐將赴陳家山,即假所獲賊旗,夜馳往,聲言白號賊至,賊下山迎,悉誘殲之,擒賊首高名貴,其黨張長庚覺而奔,追斬甚眾,擢千總。三年,總督勒保誘擒三槐,其黨冷天祿踞安樂坪,環攻不下;召思舉往,夜率死士焚其巢。將明,殿旅出,大呼曰:「我豐城劫寨羅思舉也!」賊膽落,潰圍走。思舉戰績至是始上聞,擢守備。

德楞泰圍羅其清等於箕山,復召思舉問計。思舉相地勢,曰:「賊各隘皆壘石守,惟山後懸削數十丈,必恃險乏備。若官軍攻於前,使不暇他顧;我率勇敢者梯而上,可搗也。」如其言,夾擊,大破之,餘賊四逸。思舉料其必走方山坪,率鄉勇先往,伏坪後,越數日,賊為官軍追擊,果至,擒斬幾盡,遂獲其清。四年,其清餘黨踞東鄉四季坪,從提督七十五破之。秋,敗賊巴州豆真坡,又援田朝貴於鐵爐山。五年春,德楞泰剿冉天元於川西,檄思舉率鄉勇三千赴軍。戰青龍口,賊踞山險,選精銳九十人夜薄賊巢,破之。賊分趨農安,將入陜,思舉獻計,請致書額勒登保,約守陽平關,易裝潛入賊卡,殺二賊,眾追捕,乃棄所★K7書逸出。賊果不敢前,回竄江油。思舉先驅深入,伏起,奮斗,而賊以擋牌禦矢銃,困德楞泰於馬蹄岡;急趨救,使鄉勇人取石亂擊,毀擋牌。會冉天元馬蹶就擒,賊瓦解。假賊旗追逐餘匪,斬雷士玉。攻鮮大川於天寨子,山險不能上,德楞泰遣箭手五百助之,令伏巖下,先以鄉勇誘賊,俟擂石且盡,仰射,箭落如雨,賊退避,遂克之,思舉手擒賊六十餘人。德楞泰訶其輕生,聲色俱厲;思舉跪謝,良久出,則冠上已換花翎,由是深感德楞泰,樂為盡力。

尋從勒保防嘉陵江,七十五以桂涵新敗,調思舉代領所部鄉勇,擢都司。六年,殲張世龍於鐵溪河,擊援賊陳天奇,陣斬之,賜號蘇勒芳阿巴圖魯,擢游擊。自是轉戰老林,餉不時至,煮馬韉,啗賊肉以追賊。七十五卞急,屢為賊所窘,輒賴思舉援救得捷。既而七十五坐事逮,德楞泰攻茍文明於瓦山溪,賊踞楠木坪,三戰不克。召思舉率鄉勇至,皆衣狗皮,躡草履,人笑為匄兵,夜越後山伏,一戰破之,殲茍明獻、茍文舉。眾詫曰:「匄兵破賊矣!」始補給餉,制衣履,擢參將。七年,迭敗庹向瑤於風硐子、萬古樓,破齊國點於通江,殲張天倫、魏學盛於巴州。秋,擊劉朝選於仙女溪,遁鞋底山,擒之。又偕羅聲皋擒張簡、羅道榮於巴州。冬,唐明萬竄大寧,追至石柱坪,賊方食,奮擊,大潰,擒明萬。仁宗以明萬劇賊久稽誅,特詔嘉賚。諸賊漸就殲除,搜捕南山餘孽,兩年始清,擢太平協副將。十年,德楞泰剿寧陜叛兵,檄思舉赴軍,尋就撫,盡釋歸伍。思舉曰:「兵變,殺將陷城破官軍,亂無大於此者。反賞,是勸叛也!何以懲後?請誅首逆,以申國法。」諸將不可。後川、陜兵果數叛。十一年,思舉攻西鄉叛兵,斬首逆於陣,風稍息。署川北鎮,擢涼州鎮總兵,未之任,調重慶鎮。

二十年,中瞻對番酋洛布七力叛,夾河築碉。總兵羅聲皋不能克,許其降,以專擅遣戍。命思舉進剿,克四砦,洛布七力就殲,請分其地以賞上下瞻對諸出力頭目,事乃定。道光元年,擢貴州提督,歷四川、雲南、湖北提督。

十二年,湖南江華錦田寨瑤趙金龍為亂,與長寧趙福才糾合九沖瑤肆掠,提督海凌阿戰死,勢益熾。詔總督盧坤偕思舉討之,至永州,議遏賊南竄,斷其西道州、零陵、祁陽山徑,進兵兜擊。於是驅諸瑤出山,皆東竄常寧洋泉鎮,檄各路進逼合圍,四月,大破之,金龍中槍死,擒其妻子及死黨數十,賜雙眼花翎,予一等輕車都尉世職。時命尚書禧恩督師,未至軍,先三日奏捷。禧恩方貴寵用事,怒其不待,盛氣陵之。思舉曰:「諸公貴人多顧忌。思舉一無賴,受國厚恩至提督,惟以死報,不知其他!」禧恩無如何,則詰金龍死狀虛實,思舉獲其尸及所佩印、劍、木偶為證,乃止。二十年,卒於官,賜太子太保,謚壯勇。子本鎮,襲世職。

思舉既貴,嘗與人言少時事,不少諱。檄川、陜、湖北各州縣云:「所捕盜羅思舉,今為國宣勞,可銷案矣。」再入覲,仁宗問:「何省兵精?」曰:「將良兵自精。」宣宗問:「賞罰何由明?」曰:「進一步,賞;退一步,罰。」皆稱旨。晚年自述年譜。川中殄諸劇寇,多賴其力,功為人掩,軍中與二楊並稱。楊芳於諸將少許可,獨至思舉,以為「烈丈夫」。嘗酒酣袒身示人,戰創斑斑,為父母刲股痕凡七,其忠孝蓋出天性云。

同時起鄉勇者,桂涵名與之亞,包相卿較後出,亦至專閫。

涵,亦東鄉人。少恃勇,橫行鄉里,亡命出走。繼歸,與思舉同應募為鄉勇。父天聰,聚族黨屯罐子山。賊數為涵所窘,欲報之,萬眾來攻。涵率壯士伏隘,誘賊入空寨,痛殲之。嘉慶二年,從硃射鬥攻金峨寺,賊突出,圍涵於山峒,火熏水灌皆不傷,反多斃賊,賊乃走。尋戰凈土庵,偕思舉陷陣,大破之,徐天德黨眾幾盡殲。同里聞其屢捷,爭來投效,德楞泰、明亮特編涵字營,使涵領之,擢千總,由是知名。

三年,大軍圍安樂坪,冷天祿詐降出走,涵偵知之,伏兵於方家壩、魚鱗口,賊至伏發,擒斬甚眾,擢守備。四年,從德楞泰追賊入陜,每由間道出賊前,與官軍夾擊,數捷。又從硃射鬥殲包正洪於雲陽蘆花嶺。從七十五破龔建於開縣火峰寨,手擒建以獻,擢都司。五年,復從射鬥破賊雲陽,擒其渠李甲,縱歸,招出黨眾數百人,自是降者日至。

既而改隸勒保軍,始與思舉分路,轉戰川東西,所至有功,累擢游擊。六年,從阿哈保追湯思蛟於墊江,賊夜走,涵謂:「窮寇且死鬥,請先伏魏家溝。」俟其至,突擊,大破之。又從薛大烈追李彬、冉天士於通江,至小中河,大雪,賊不為備,涵率鄉勇夜半薄賊壘,與官軍四面乘之,賊奔曠野,勁騎沖踏,盡殲焉。彬遁,未幾,為劉清所獲。自七年後,復偕思舉遍歷老林,搜剿匿匪,累遷夔州協副將。九年秋,從經略、參贊圍餘匪於太平火燒梁山,峻無路。涵議:「守此相持,雖數月無如賊何。山下小溪通民峒,賊久困,必出劫峒糧,請以步卒伏山後。」賊果以驍銳千餘潛出,諸將皆死戰,半日殲之,前山自潰。遂殄滅凈盡,川、陜肅清。

十一年冬,綏定兵叛,涵在梁山聞變,慮本部兵與通,單騎馳入郡城,聲言越兩日出兵;密令弟吉出募鄉勇舊部為一隊,約期合攻。時賊踞景市廟,將往麻柳場。涵至,令急赴景市廟,中途改趨麻柳場,距賊數里止隊,入深箐,諜報賊逾千,且至,叱曰:「安得有此眾?」戒毋輕進,毋漏言涵至。既而賊自山沖下,三進三退,乃突起擊賊;而弟吉已率五百人據山頂,賊大潰,擒首逆王德先。叛兵起事甫五日,一鼓平之,賜號健勇巴圖魯。十三年,署重慶鎮,尋授川北鎮總兵。十九年,擊三才峽匪黨吳抓抓等於沔縣,走之。川北獲安。道光二年,擢四川提督。果洛克番匪劫西藏堪布貢物,命剿擒首逆曲俊父子,被優賚。在任十載,遇番、夷蠢動,兵至輒定。十三年,討越巂夷匪,連戰皆捷。忽遘疾,卒於軍。優恤,贈太子太保,謚壯勇。子三人,並晉官秩。

相卿,鄰水人。嘉慶六年,以鄉勇隸松潘鎮標。嘗從思舉擊陳朝觀於通江龍鳳埡,追賊受矛傷,裹創力戰。七年,破張天倫於巴州金子寺,相卿斬天倫轂子山下,給藍翎、八品頂戴。又殲張簡、唐明萬,功皆最。十年,思舉偵襄賊王世貴、謝應洪匿太平老林,檄相卿躡捕,殲之,授千總。十二年,剿瓦石坪叛兵,擢守備。累遷廣元營游擊。十三年,調征臺灣。會峨邊越巂惈夷叛,命回川從提督楊芳赴剿,攻克啯嚕崖。夷踞曲曲烏烏斯坡,相卿梯絕壁,牽挽負砲而上,破之,進毀巴姑賊寨,擢參將。十五年,惈夷復叛,攻克峨邊十三支夷巢,破越巂沈喳夷,抵濫田壩,兩叛夷悉降,累遷懋功協副將。剿馬邊夷,擒其渠,加總兵銜。再署建昌鎮總兵,總督鄂山、寶興皆以邊事倚之。十九年,病歸,卒。

論曰:川、楚之役,竭宇內之兵力而後定之。材武驍猛,萃於行間,然戰無不勝,攻無不取者,厥惟二楊及羅思舉為之冠。遇春謀勇俱絕,劇寇半為所殲。思舉習於賊情、地勢、險厄,強梁非其莫克。至於忠誠忘私,身名俱泰,遇春際遇之隆,固為稀覯;而思舉以藪澤梟傑,終保令名,煥於旂常矣。鄉兵出平鉅寇,亦自其為始云。


\end{pinyinscope}