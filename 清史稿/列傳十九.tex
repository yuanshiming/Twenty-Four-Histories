\article{列傳十九}

\begin{pinyinscope}
希福子帥顏保曾孫嵩壽範文程子承勛承斌孫時繹時捷時綬

時紀曾孫宜恆四世孫建中寧完我鮑承先

希福,赫舍里氏。世居都英額,再遷哈達。太祖既滅哈達,希福從其兄碩色率所部來歸。居有頃,以希福兼通滿、漢、蒙古文字,召直文館。屢奉使蒙古諸部,賜號「巴克什」。旗制定,隸滿洲正黃旗。

天聰二年,太宗伐察哈爾,以希福使科爾沁徵兵,土謝圖額駙奧巴止之曰:「寇騎塞路,行將安之?即有失,誰執其咎?」希福曰:「君命安得辭?死則死耳,事不可誤也。」遂行。再宿,達上所,復命曰:「科爾沁兵不赴徵,土謝圖額駙奧巴方率所部行掠,掠竟乃來耳。」上怒,使希福再往,以壯士八人從。行四晝夜,道遇敵,擊殺三十餘人,卒至科爾沁,以其兵來會。明年,奧巴來朝,上命希福與館臣庫爾纏輩責讓之,奧巴服罪,上駝馬以謝。敘功,授備御。從伐明,薄明都,敗明兵於城下。攻大凌河,援兵自錦州至,與譚泰爭先奮擊,破之。師還,又力戰敗追兵,進游擊。

崇德元年,改文館為內三院,希福為內國史院承政。尋授內弘文院大學士,進二等甲喇章京。二年,請禁造言惑眾,違者罪之,著為令。三年,偕大學士範文程建言定部院官制。希福雖以文學事上,官內院,筦機務,然常出使察哈爾、喀爾喀、科爾沁諸部,編戶口,置牛錄,頒法律,亭平獄訟;時或詣軍前宣示機宜,相度形勢,覈諸將戰閥,行賞,諭上德意於諸降人。每還奏,未嘗不稱旨也。順治元年,譯遼、金、元三史成,奏進,世祖恩賚有加。

希福故與譚泰有隙,屢誚其衰慵。譚泰暱附攝政睿親王多爾袞,因與其弟譚布構希福妄傳王語,謂堂餐過侈,詆謾諸大臣,構釁亂政,罪當死;王命罷官削世職,並籍其家。八年二月,世祖親政,雪其枉,仍授內弘文院大學士,復世職。九年,世祖以希福事太祖、太宗,銜命馳驅,殫心力。曩定鼎燕京,希福方削籍,功未賞,乃一歲三進為三等精奇尼哈番,世襲。是年十一月,卒,贈太保,謚文簡。長子奇塔特,襲職。乾隆初,定封三等子。

帥顏保,希福次子。康熙初,聖祖念希福事先朝久,躬預佐命,用大學士範文程、額色黑例,超授內國史院學士。八年六月,遷吏部侍郎。七月,授漕運總督。九月,疏言:「淮安水陸孔道,乃十五里中為關者三,板閘有鈔關,淮安有倉稅,隸戶部;清江有稅廠,隸工部。胥役繁冗,商民耗資失時,請減三為一,合並稅額,省胥役,便商民。」下部議,戶部言倉稅並鈔關便;工部言稅廠徵船料諸稅,葺治漕船,並鈔關不便。上心韙帥顏保言,下九卿科道再議,卒如所請。九年正月,疏言:「淮、揚被水,高郵、宿遷、桃源、鹽城、贛榆災尤重。舊逋漕米,例當補徵,民力不能勝。」下部議,請改折,仍補徵。上以諸縣頻歲被災,民重困,下部再議,免舊逋漕米三萬一千石有奇。十二年正月,偕河道總督王光裕疏請漕運畢事,當復舊例,舉劾所屬文武官吏。既得請,疏薦山東糧道遲日巽、河南糧道範周、無錫知縣吳興祚等,劾溧陽知縣王錫範等。十三年,吳三桂兵犯江西,十月,命帥顏保帥所部移鎮南昌。十二月,安親王岳樂師至,命罷還。十七年,岳樂進軍湖南,復命鎮南昌。九月,移吉安。十八年三月,招降吳三桂部將五十餘、兵萬餘。十九年八月,逮尚之信勘治,命帥顏保移鎮南雄、韶州。十月,命罷還。二十年五月,遷工部尚書。十二月,移禮部尚書。二十三年十二月,卒。子赫奕,自侍衛累遷工部尚書。

嵩壽,希福曾孫。雍正元年進士,選庶吉士,授編修。乾隆二年,冊封安南國王黎維禕,以侍讀充正使,賜一品服。累擢內閣學士。十四年,頒詔朝鮮,擢禮部侍郎。十九年,襲一等子爵。二十年,卒。

範文程,字憲斗,宋觀文殿大學士高平公純仁十七世孫也。其先世,明初自江西謫沈陽,遂為沈陽人,居撫順所。曾祖鏓,正德間進士,官至兵部尚書,明史有傳。

文程少好讀書,穎敏沉毅,與其兄文寀並為沈陽縣學生員。天命三年,太祖既下撫順,文寀、文程共謁太祖。太祖偉文程,與語,器之,知為鏓曾孫,顧謂諸貝勒曰:「此名臣後也,善遇之!」上伐明,取遼陽,度三岔攻西平,下廣寧,文程皆在行間。

太宗即位,召直左右。天聰三年,復從伐明,入薊門,克遵化。文程別將偏師徇潘家口、馬蘭峪、三屯營、馬欄關、大安口,凡五城皆下。既,明圍我師大安口,文程以火器進攻,圍解。太宗自將略永平,留文程守遵化,敵掩至,文程率先力戰,敵敗走。以功授世職游擊。五年,師圍大凌河,降其城,而蒙古降卒有陰戕其將叛去者,上怒甚,文程從容進說,貸死者五百餘人。時明別將壁西山之巔,獨負險堅守未下,文程單騎抵其壘,諭以利害,乃請降。上悅,以降人盡賜文程。

六年,從上略明邊,文程與同直文館寧完我、馬國柱上疏論兵事,以為入宣、大,不若攻山海。及師至歸化城,上策深入,召文程等與謀。文程等疏言:「察我軍情狀,志皆在深入。當直抵北京決和否,毀山海關水門而歸,以張軍威。若計所從入,惟雁門為便,道既無阻,道旁居民富庶,可資以為糧。上如慮師無名,當顯諭其民,言察哈爾汗遠遁,所部歸於我,道遠不可以徒行,來與爾國議和,假爾馬以濟我新附之眾。和議成,償馬值;不成,異日興師,荷天之寵,以版圖歸我,凡軍興而擾及者,當量免賦稅數年。此所謂堂堂正正之師也。否則,作書抵近邊諸將吏,使以議和請於其主,為期決進止。彼朝臣內撓,邊將外諉,遷延逾所期,我師即乘釁而入。我師進,利在深入;否,利在速歸;半途而返,無益也。」疏入,上深嘉納之。

七年,孔有德等使通款,而明兵迫之急,上命文程從諸貝勒帥師赴援;文程宣上德意,有德等遂以所部來歸。自是破旅順,收平島,討朝鮮,撫定蒙古,文程皆與謀。

崇德元年,改文館為內三院,以文程為內秘書院大學士,進世職二等甲喇章京。初,旗制既定,設固山額真。諸臣議首推文程,上曰:「範章京才誠勝此,然固山職一軍耳,朕方資為心膂。其別議之。」文程所典皆機密事,每入對,必漏下數十刻始出;或未及食息,復召入。上重文程,每議政,必曰:「範章京知否?」脫有未當,曰:「何不與範章京議之?」眾曰:「範亦云爾。」上輒署可。文程嘗以疾在告,庶務填委,命待範章京病已裁決。撫諭各國書敕,皆文程視草。初,上猶省覽,後乃不復詳審,曰:「汝當無謬也。」文程迎父楠侍養,嘗入侍上食,有珍味,文程私念父所未嘗,逡巡不下箸。上察其意,即命徹饌以賜楠,文程再拜謝。

世祖即位,命隸鑲黃旗。睿親王多爾袞帥師伐明,文程上書言:「中原百姓蹇離喪亂,備極荼毒,思擇令主,以圖樂業。曩者棄遵化,屠永平,兩次深入而復返。彼必以我為無大志,惟金帛子女是圖,因懷疑貳。今當申嚴紀律,秋毫勿犯,宣諭進取中原之意:官仍其職,民復其業,錄賢能,恤無告。大河以北,可傳檄定也。」及流賊李自成破明都,報至,文程方養痾蓋州湯泉,驛召決策,文程曰:「闖寇塗炭中原,戕厥君後,此必討之賊也。雖擁眾百萬,橫行無憚,其敗道有三:逼殞其主,天怒矣;刑辱搢紳,拷劫財貨,士忿矣;掠人貲,淫人婦,火人廬舍,民恨矣。備此三敗,行之以驕,可一戰破也。我國上下同心,兵甲選練,聲罪以臨之,恤其士夫,拯其黎庶。兵以義動,何功不成?」又曰:「好生者天之德也,古未有嗜殺而得天下者。國家止欲帝關東則已,若將統一區夏,非乂安百姓不可。」翌日,馳赴軍中草檄,諭明吏民言:「義師為爾復君父仇,非殺爾百姓,今所誅者惟闖賊。吏來歸,復其位;民來歸,復其業。師行以律,必不汝害。」檄皆署文程官階、姓氏。

既克明都,百度草創,用文程議,為明莊烈愍皇帝發喪,安撫孑遺,舉用廢官,蒐求隱逸,甄考文獻,更定律令,廣開言路,招集諸曹胥吏,徵求冊籍。明季賦額屢加,冊皆毀於寇,惟萬歷時故籍存,或欲下直省求新冊,文程曰:「即此為額,猶慮病民,其可更求乎?」於是議遂定。論功,並遇恩詔,進一等阿思哈尼哈番加拖沙喇哈番,賜號「巴克什」。復進二等精奇尼哈番。

順治二年,江南既定,文程上疏言:「治天下在得民心,士為秀民。士心得,則民心得矣。請再行鄉、會試,廣其登進。」從之。五年正月,定內三院為文臣班首,命文程及剛林、祁充格用珠頂、玉帶。七年,睿親王多爾袞卒。八年,大學士剛林、祁充格以附睿親王妄改太祖實錄,坐死。文程與同官當連坐,上以文程不附睿親王,命但奪官論贖。是歲即復官。九年,遇恩詔,復進世職一等精奇尼哈番,授議政大臣,監修太宗實錄。

時直省錢糧多不如額,一歲至缺四百餘萬,賦虧餉絀。文程疏言:「湖廣、江西、河南、山東、陜西五省亂久民稀,請興屯,設道二、同知四,令督撫選屬吏廉能敏幹者任之,以選吏當否為督撫功罪。官吏俸廩,初年出興屯母財,次年以所穫償。自後皆出所穫,官增而俸不費。屯用牛,若穀種,若農器,聽興屯道發州縣倉庫以具。屯始駐兵,地荒蕪多而水道便者,以次及其餘。地無主,若有主而棄不耕,皆為官屯。民原耕而財不足,官佐以牛若穀種,分所穫三之一,三年後為民業。編保甲,使助守望,絕奸宄。若無財,官畀以傭值。民將逭饑,流亡當大集。初年所穫糧草,聽屯吏儲留,出陳易新,為次年母財;有餘,畀近屯駐軍,勿為額以取盈。三年所穫浸多,僦舟車運以饋餉。毋煩屯吏,毋役屯民,毋用屯牛。屯所在州縣吏受興屯道指揮,屯吏稱其職,三歲進二秩,視邊俸;不職,責撫按糾舉;有所徇,則並坐:所謂信賞必罰也。」上深韙其議。

十年,復與同官疏:「請敕部院三品以上大臣,各舉所知,毋問滿、漢新舊,毋泥官秩高下,毋避親疏恩怨,舉惟其才,各具專疏,臚舉實跡,置御前以時召對。察其論議,覈其行事,並視其舉主為何如人,則其人堪任與否,上早所深鑒,待缺簡用。稱職,量效之大小,舉主同其賞;不稱職,量罪之大小,舉主同其罰。」上特允所請。

上勤於政治,屢幸內院,進諸臣從容諮訪。文程每以班首承旨,陳對稱上意。嘗值端陽,諸臣散直差早,上曰:「乘藉天休,猥圖安樂,人情盡然。特欲逸必先勞,俾國家大定,其樂方永。不然,樂亦暫耳。」復言:「人孰無過,能改之為美。成湯盛德,改過不吝。若明武宗嬉游無度,諉罪於其臣,豈修己治人之道耶?」文程因奏:「君明臣良,必交勉釋回,始克荷天休,濟國事。」上曰:「善。自今以往,朕有過即改。卿等亦宜黽勉,毋忘啟沃可也!」上嘗命遣官蒞各省恤刑,文程言:「前此遣滿、漢大臣巡方,慮擾民,故罷。今四方水旱災傷,民勞未息,宜罷遣使。現禁重囚,令各省巡撫詳勘,有可矜疑,奏聞裁定。」上從之。文程論政,務簡耍,持大體,多類是。

十一年八月,上加恩輔政諸臣,特加文程少保兼太子太保,文程疏謝,因自陳衰病,乞休。九月,上降溫諭,進太傅兼太子太師,致仕。上以文程祖宗朝舊臣,有大功於國家,禮遇甚厚:文程疾,嘗親調藥餌以賜;遣畫工就第圖其像,藏之內府;賚御用服物,多不勝紀;又以文程形貌頎偉,命特制衣冠,求其稱體。聖祖即位,特命祭告太宗山陵,伏地哀慟不能起。康熙五年八月庚戌,卒,年七十。上親為文,遣禮部侍郎黃機諭祭,賜葬懷柔紅螺山,立碑紀績,謚文肅,御書祠額曰「元輔高風」。文程子承廕、承謨、承勛、承斌、承烈、承祚,承謨自有傳。

承勛字蘇公,文程第三子也。以任子歷官御史、郎中。康熙十九年,譚弘叛,聖祖命承勛與郎中額爾赫圖如彞陵,趣將軍噶爾漢戰,並督湖廣轉粟運軍。二十年,師進攻雲南,命趣軍督餉如故。二十二年,還京,監崇文門稅。二十三年,上命九卿舉廉吏,承勛與焉,遷內閣學士。二十四年,授廣西巡撫,疏免容縣、鬱林州追徵陷賊後逋賦;定諸屬徵米,本折兼納。二十五年,擢雲貴總督,疏定雲南援剿兩協駐軍地,裁貴州衛十五、所十,改並州縣,並增設縣七。二十七年,湖廣兵亂,雲南時歲鑄錢,錢壅積,軍餉十之三皆予錢,軍勿便。會移左協赴尋甸,遂鼓譟為變,省城兵亦將起應,承勛誅其渠二十一人,亂乃弭。遂疏罷雲南鑄錢,以銀供餉。二十八年,番阿所殺土目魯姐走匿東川土婦安氏所,心互出掠為民害。事聞,上命郎中溫葆會承勛等如東川檄安氏獻阿所,斬之。

雲南自吳三桂亂後,康熙二十一年訖二十七年,逋屯賦當補徵,承勛疏請分年附徵,上命悉蠲之。二十九年,疏定雲南秋糧,本折兼納,貴州提督馬三奇請軍餉折銀,承勛疏言:「折賤困兵,折貴病民,宜以時損益。秋成,各府察巿值,本折兼納。」三十一年,疏設永北鎮,罷洱海營,增置大理府城守將吏。三十二年,入覲。

三十三年,遷都察院左都御史。六月,江南江西總督傅拉塔卒,上難其人,以授承勛。並諭:「承勛堅定平易,當勝此任。」承勛上官,琉移鳳陽關監督駐正陽關。江西民納糧,出貲俾吏輸省城,謂之腳價,尋以違例追入官,承勛疏請罷追,部議不可,上特允其請。江南地卑濕,倉穀易朽蠹,承勛疏請「江蘇、安徽諸州縣,歲春夏間,以倉穀十二三平糶,出陳易新」。又以江南賦重,疏請「州縣經徵分數,視續完多寡為輕重。康熙十八年後逋賦分年附徵,俾寬吏議,紓民力」。皆如議行。三十五年,淮、揚、徐諸府災,疏請發省倉米十萬石,續借京口留漕鳳倉存麥,治賑,民賴以全。三十八年,授兵部尚書。三十九年,命監修高家堰堤工。四十三年,工成,加太子太保。五十三年,卒。

承勛初授廣西巡撫,入辭,上誡之曰:「汝父兄皆為國宣力,汝當潔己愛民,毋信幕僚,沽名妄作。」及自雲貴總督入覲,上方謁孝陵,承勛迎謁米峪口,上曰:「汝父兄先朝舊臣,汝兄復盡節。朕見汝因思汝兄,心為軫戚。不見汝八九年,汝須發遂皓白如此。郊外苦寒,以朕所御貂冠、貂褂、狐白裘賜汝。汝且勿更衣,慮中風寒。明日可服以謝。」聖祖推文程、承謨舊恩,因厚遇承勛如是。

時繹,承勛子。雍正初,自佐領三遷為馬蘭鎮總兵。四年,命署兩江總督。是年,遷正藍旗漢軍都統。五年,移鑲白旗漢軍都統,並署總督如故。十二月,時繹疏:「請自雍正六年始,江蘇、安徽各州縣應徵丁銀,均入地畝內徵收。」地丁並徵始此。六年,授戶部尚書,仍署總督。時繹在官,嘗疏請就通州運河入海處,作涵洞以時蓄洩。規揚州水利,濬海口,疏車路、白塗、海溝諸水,泰州運鹽河為之堤。鹽城、如皋諸水入海處,為之閘若涵洞。釐兩淮鹽政,增漕標廟灣、鹽城二營兵吏。皆下部議行。上以蘇、松諸處多盜,時繹戢盜才絀,命以江蘇七府五州盜案屬浙江總督李衛。衛名捕江寧民張云如以符咒惑眾謀不軌,而時繹嘗與往還,衛因論劾。八年,命尚書李永升會鞫得實,誅雲如,解時繹任。召還京,命董理太平峪吉地。旋復命協理河東河務,河東總督田文鏡復以誤工論劾,諭曰:「朕以範時繹為勛臣後,加以擢用。硃鴻緒嘗奏時繹廉,至日用不能給,朕深為動念,優與養廉。後知時繹例所當得,未嘗不取。朕猶令增糈,蓋欲遂成其廉,使殫心力於封疆也。顧時繹袒私交,容奸宄,朕復密諭李衛善為保全。且範氏為大僚者,惟時繹及其從弟時捷,勛臣後裔,漸至零落,朕心不忍,所以委曲成全之者至矣。復命協理河務,豈意伏汛危急,時繹安坐於旁,置國事弁髦,視民命草芥。負恩職,他人尚不可,況時繹乎?」逮治,部議坐雲如獄論斬,上復特宥之。授鑲藍旗漢軍副都統。十年,授工部尚書,兼鑲黃旗漢軍都統。十二年,罷尚書。十三年,復以侍衛保柱劾行賄,下部議罪,尋遇赦。乾隆六年,卒。

承斌,文程第四子,襲一等精奇尼哈番。卒。

時捷,承斌子。自參領再遷為陜西、寧夏總兵。康熙五十七年,署陜甘提督。雍正元年,授陜西巡撫。三年,遷鑲白旗漢軍都統。五年,年羹堯得罪,世宗以羹堯嘗舉時捷,及羹堯敗,事連時捷,罷都統,授侍衛。八年,授散秩大臣,護陵寢。是時,時捷從兄時繹以協理河東河務誤工罷黜,世宗以文程諸孫無為大僚者,命時捷署古北口提督,直隸總兵官聽節制,詔勉以改過。旋移陜西固原提督。乾隆元年,例改一等子。二年,以病召還,授散秩大臣。三年,卒。

建中,時捷孫,襲一等男。自副參領再遷副都統、侍郎。嘉慶四年,授戶部尚書,署正黃旗漢軍都統。尋改都察院左都御史,出為杭州將軍。五年,卒,謚恪慎。

時綬,文程諸孫。雍正間,自筆帖式累遷至戶部郎中。乾隆初,復累遷至湖北布政使。十六年,署湖南巡撫,疏言:「湘陰、益陽諸縣,察有私墾千餘頃,皆瀕洞庭,歲旱方穫,請緩升科。洞庭諸私垸窒水道,勸禁增築。」報可。十八年,移江西巡撫,病免。二十一年,起授戶部侍郎,署都統,請赴西路屯田。二十四年,副都統定長劾時綬役兵漁利,遣使就讞,時綬未嘗役兵,特其僕從藉事求利,命奪官,交定長責自效。二十六年,授頭等侍衛,遷鑲藍旗漢軍副都統、吏部侍郎、哈爾沙爾辦事。三十一年,遷左都御史,仍留哈爾沙爾辦事。三十二年,授湖北巡撫。入對,上以時綬弱不能任封疆,三十三年,復授都統、左都御史。三十五年,遷工部尚書。明年,罷。四十七年,卒。

時紀,亦文程諸孫。乾隆初,以任子授工部員外郎。四遷,署廣東按察使。二十五年,俸滿入覲,諭範氏無大僚,授鑲紅旗漢軍副都統。二十六年,授工部侍郎。二十七年,疏請就京南諸州縣開田植稻,下直隸總督方觀承察土宜酌行。屢移倉場、戶部、禮部諸侍郎。四十二年,以年衰改副都統。尋卒。

宜恆,時綬子。乾隆中,自鑾儀衛、整儀衛,五遷,為福建福寧鎮總兵。四十七年,授正藍旗漢軍副都統。五十七年,授工部侍郎。嘉慶元年,遷戶部尚書。二年,卒。

文程曾孫行又有宜清,乾隆間官盛京工部侍郎;四世諸孫建豐,嘉慶時官吏部侍郎:皆以漢軍任滿缺,一時稱異數云。

寧完我,字公甫,遼陽人。天命間來歸,給事貝勒薩哈廉家,隸漢軍正紅旗。天聰三年,太宗聞完我通文史,召令直文館。完我入對,薦所知者與之同升,鮑承先其一也。尋授參將。四年,師克永平,命與達海宣諭安撫。又從攻大凌河及招撫察哈爾,皆有功,授世職備御。五年七月,初置六部,命儒臣賜號「榜式」得仍舊稱,餘稱「筆帖式」。

完我遇事敢言,嘗議定官制,辨服色。十二月,上疏言:「自古設官定職,非帝王好為鋪張。慮國事無綱紀也,置六部;慮六部有偏私也,置六科;慮君心宜啟沃也,置館臣;慮下情或壅蔽也,置通政。數事相因,缺一不可。上不立言官,不過謂我國人人得以進言,何必言官。臣請明辨之,我國六部既立,曾見有一人抗顏論劾者否?似此寂寂,豈國中真無事耶?舉國然諾浮沉,以狡滑為圓活,以容隱為公道,以優柔退縮為雅重,上皇皇圖治,亦何樂有此景象也?況今日秉政者,豈盡循理方正?屬僚既不敢非長官,局外又誰敢議權貴?臣知國中事,上亦時得聞知,然不過猶古之告密,孰若置言官,興利除害,皆公言之之為愈耶?言官既設,君身尚許指摘,他人更何忌諱?茍不至貪污欺誑,任其盡言,勿為禁制,此古帝王明目達聰之妙術也。若謂南朝言官敗壞,此自其君鑒別不明,非其初定制之不善也。我國『筆帖式』,漢言『書房』,朝廷安所用書房?官生雜處,名器弗定。不置通政,則下情上壅,勵精圖治之謂何也?至若服制,尤陶鎔滿、漢第一急事。上遇漢官,溫慰懇至,而國人反陵轢之。漢官不通滿語,每以此被辱,有至傷心墮淚者,將何以招徠遠人,使成一體?故臣謂分別服色,所系至大,原上勿再忽之也。臣等非才,惟耿介忠悃,至死不變。昨年副將高鴻中出領甲喇額真,臣具疏請留;今游擊範文程又補刑曹,諒臣亦不得久居文館。若臣等二三人皆去,豈復得慷慨為上盡言乎?」疏入,上頗韙之,命俟次第舉行。

六年正月,完我疏言:「昨年十一月初九日,自大凌河旋師,上豫議今年進取,至誠惻怛,推心置腹,藹然家人父子。臣敢不殫精畢思,用效駑鈍。臣聞千里而戰,雖勝亦敗。近年將士貪欺之習,大異於先帝時,更張而轉移之。上固切切在念,而曾未顯斡旋之術。人心不鍊,必不得指臂相使之用。分軍駐防,萬難調停,雖諸葛復生,無能為也。又況蜂蠆有毒,肘腋患生,疑貳之祖大壽,率寧、錦瘡痍之眾,坐伺於數百里間,杞人之見,不得不慮及也。」三月,上決策自將伐察哈爾,而完我以為大凌河降卒思遁,宜先圖山海,還取錦州,因上疏諫。四月,師西出,度興安嶺,次都勒河,偵言林丹汗西走。完我與同值文館範文程、馬國柱合疏申前議,略言:「師已度興安嶺,察哈爾望風遠遁,上威名顯襮。臣度上且罷西征,轉而南入。上憐士卒勞苦,不能長驅直入,徒攜子女、囊金帛而歸。茍若是,大事去矣!昔者遼左之誤,諉諸先帝;永平之失,諉諸二貝勒。今更將誰諉?信蓋天下,然後能服天下。臣等為上籌之,以為當令從軍蒙古,每人擇頭人三二輩,挾從者十餘人,從上南入,餘悉遣還部。然後嚴我法度,昭告有眾,師行所經,戒殺戒掠,務種德樹仁,宏我後來之路。今此出師,諸軍士賣牛買馬,典衣置裝,離家益遠,見財而不取,軍心怠矣,取則又蹈覆轍。上豈不曰『我厲禁取財,其孰敢違』?上耳目所及,或不敢犯;耳目所不及,孰能保者?無問蒙古部長,及諸貝勒,稍稍擾民,怨歸於上,此上所當深思者也。與其以長驅疲憊之兵入宣、大,孰若留精銳有餘之力取山海。臣等明知失上旨,但既見及此,不容箝口也。」是時上已決用兵於宣、大,五月,上駐歸化城,召完我等計事。完我等疏論機宜,語詳文程傳。翌日,上諭蒙古諸部及諸貝勒申軍律,蓋採完我等前疏所陳也。

七年正月,完我疏言:「近日朝鮮交益疏,南朝和未定,沈城不可以常都,兵事不可以久緩,機會不可以再失。漢高祖屢敗,何為而帝?項羽橫行天下,何為而亡?袁紹擁河北之眾,何為而敗?昭烈屢遘困難,何為而終霸?無他,能用謀不能用謀,能乘機不能乘機而已。夫天下大器也,可以智取,不可以力爭。臣請以棋喻,能者戰守攻取,素熟於胸中,百局而百不負。至於取天下,是何等事,而可以草草僥幸耶?自古君臣相需,先帝時,達拉哈轄五大臣,知有上不知有人,知有國不知有家,故先帝以數十人起,克成大業。上今環觀國中,如五大臣者有幾人耶?每侍上治事,不聞諫諍,但有唯阿;惟務茍且,不肯任勞怨。於國何利?於上何益?釣餌激勸,振刷轉移,臣望上於旦暮間也。古人有言:『騏驥之局促,不如駑馬之安步;孟賁之狐疑,不如庸夫之必至;雖有堯、舜之智,吟而不言,不如喑啞之指揮。』此言貴能行之。臣謹昧死上言,惟上裁擇。」

完我他所獻替,如論譯書,謂:「自金史外,當兼譯孝經、學、庸、論、孟、通鑒諸籍。」論試士,謂:「我國貪惰之俗,牢不可破,不當祗以筆舌取人,試前宜刷陋習,試後宜察素行。且六部中,滿、漢官吏及大凌河將備,當悉令入試,既可覘此等人才調,且令此等人皆自科目出,庶同貴此途不相冰炭也。」論六部治事,謂:「六部本循明制,漢承政皆墨守大明會典,宜參酌彼此,殫心竭思,就今日規模,別立會典。務去因循之習,漸就中國之制度,庶異日既得中原,不至於自擾。昔漢繼秦而王,蕭何任造律,叔孫通任制禮。彼猶是人也,前無所因,尚能造律制禮;今既有成法,乃不能通其變,則又何也?六部漢承政宜人置一通事,上亦宜以譯者侍左右,俾時召對,毋使以不通滿語自諉。」完我疏屢上,上每採其議。完我又嘗疏薦李率泰、陳錦,皆至大用。惟論用兵,力主自寧、錦直攻山海,不原出宣、大;孔有德、耿仲明降時,完我疏言當收其兵入烏真超哈,繼又言有德、仲明暴戾無才,其兵多礦徒,食盡且為盜:皆未當上旨。

九年二月,範文程上言薦舉太濫,舉主雖不連坐,亦當議罰。完我亦疏請功罪皆當並議,略言:「上令官民皆得薦舉,本欲得才以任事,乃無知者假此幸進,兩部已四五十人,其濫可見。當行連坐法,所舉得人,舉主同其賞;所舉失人,舉主同其罪;如有末路改節,許舉主自陳,貸其罪。如採此法,臣度不三日,請罷舉者十當八九;其有留者,不問皆真才矣。」上並嘉納。

完我久預機務,遇事敢言,累進世職二等甲喇章京,襲六次,賜莊田、奴僕,上駸駸倚任,顧喜酒縱博。初從上伐明,命助守永平,以博為禮部參政李伯龍及游擊佟整所劾,上為誡諭,宥之。十年二月,復坐與大凌河降將甲喇章京劉士英博,為士英奴所訐,削世職,盡奪所賜,仍令給事薩哈廉家。是年改元崇德,以文館為內三院,希福、文程、承先皆為大學士,完我以罪廢,不得與。

及世祖定鼎京師,起完我為學士。順治二年五月,授內弘文院大學士,充明史總裁。是年及三年、六年,並充會試總裁。又命監修太宗實錄,譯三國志、洪武寶訓諸書,復授二等阿達哈哈番。八年閏二月,大學士剛林、祁充格得罪,完我以知睿親王改太祖實錄未啟奏,當奪職,鄭親王濟爾哈朗等覆讞,以為無罪,得免。三月,調內國史院大學士,命班位祿秩從滿洲大學士例。尋授議政大臣。

十一年三月,疏劾大學士陳名夏結黨懷奸,臚舉名夏塗抹票擬稿簿,刪改諭旨,庇護同黨,縱子掖臣為害鄉里,凡七事;復言:「從古奸臣賊子,黨不成則計不行。何則?無真才,無實事,無顯功,故必結黨為之虛譽。欲黨之成,附己者雖惡必護,異己者雖善必仇,行之久而入黨者多。若非審察鄉評輿論,按其行事,則黨固莫可破矣。臣竊自念,壯年孟浪疏庸,辜負先帝,一廢十年。皇上定鼎,始得隨入禁地,謹守臣職,又復十年,忍性緘口。然愚直性生,遇事勃發,埋輪補牘,雖不敢行;若夫附黨營私以圖富貴,臣寧死不為也。皇上不以臣衰老,列諸滿大臣;聖壽召入深宮,親賜御酒。臣非土木,敢不盡心力圖報。名夏奸亂日甚,黨局日成。人鑒張煊而莫敢言,臣不憚舍殘軀以報聖主。」名夏坐是譴死。八月,加太子太保。十三年,加少傅兼太子太傅。

十五年九月,以老乞休,溫諭命致仕。康熙元年正月,聖祖念完我事太宗﹑世祖有勞,命官一子為學士。四年四月,卒,謚文毅。雍正六年七月,世宗命錄完我子孫,得曾孫蘭,以驍騎校待缺,賜宅,予白金五百。

鮑承先,山西應州人。明萬歷間,積官至參將。泰昌元年,從總兵賀世賢、李秉誠守沈陽城,遷開原東路統領新勇營副將,城守如故。經略熊廷弼疏請獎勵諸將,承先預焉,加都督僉事銜。是歲為天命五年。太祖已克開原,乃自懿路、蒲河二路進兵向沈陽。承先偕世賢、秉誠出城,分汛駐守,見太祖兵至,皆不戰退。上令左翼兵逐承先等,迫沈陽城北,斬百餘級而去。七年三月,上克水審陽、遼陽,世賢戰死,承先退保廣寧。八年正月,克西平堡,承先從秉誠及總兵劉渠、祁秉忠等自廣寧赴援,渠、秉忠戰死,承先與秉誠敗走,全軍盡殪。巡撫王化貞棄廣寧走入關,游擊孫得功等以廣寧降。承先竄匿數日,從眾出降,仍授副將。

天聰三年,太宗自將伐明,自龍井關入邊,承先從鄭親王濟爾哈朗略馬蘭峪,屢敗明兵,承先以書招其守將來降。師進薄明都,承先復招降牧馬廠太監,獲其馬騾以濟師。明經略袁崇煥以二萬人自寧遠入援,屯廣渠門外,憑險設伏。貝勒豪格督兵出其右,戰屢勝。是時承先以寧完我薦直文館,翌日,上誡諸軍勿進攻,召承先及副將高鴻中授以秘計,使近陣獲明內監系所並坐,故相耳語,云:「今日撤兵乃上計也。頃見上單騎向敵,有二人自敵中來,見上,語良久乃去。意袁經略有密約,此事可立就矣。」內監楊某佯臥竊聽,越日,縱之歸,以告明帝,遂殺崇煥。

四年,師克永平,承先從,以書諭遷安諸紳硃堅臺、卜文煥以城降,遂取灤州。上命承先與副將白格率鑲黃、鑲藍二旗兵守遷安,立臺堡五,明兵來攻,力戰卻之。明監軍道張春、總兵祖大壽等合諸軍攻灤州,貝勒阿敏令承先以守遷安兵守永平。及灤州破,阿敏棄永平,率諸將出冷口,東還沈陽。上命定諸將棄地罪,以承先、白格守遷安,完城退敵,釋弗問。五年,從攻大凌河,降翟家堡。

六年十一月,上詢文館諸臣,考各部啟心郎優絀以為黜陟。承先與寧完我、範文程疏言:「當察其建言,或實心為國,或巧言塞責,以為去留。」七年五月,孔有德、耿仲明來降,泊舟鎮江。承先疏言:「用舟師攻明宜急進,否則,明亦廣練舟師以御,即不能為功。」七月,既克旅順,承先復請移鎮江諸艦泊蓋州,收旁近諸島,以仁義撫其人。

八年五月,上伐明大同,明總督張宗衡、總兵曹文詔等遣承先子韜齎書請和。初,承先降,明人執韜系應州獄,至是出之,使以書來,山行,遇土謝圖濟農兵,奪其騎,斫韜及從者,皆死。兵去,韜復蘇。有馮國珍者,送韜至貝勒代善營,令與承先相見,遂使入謁上。上見韜創甚,留軍中,遣國珍齎書還。

九年正月,承先疏言:「臣竊見元帥孔有德、總兵耿仲明為其屬員請敕,上許其自行給劄。帝王開國,首重名器,上下之分,自有定禮。倘欲加意招徠遠人,可諭吏部奏請給劄,使恩出上裁。」上不謂然,諭曰:「元帥率眾航海遠來,厥功匪小。任賢勿貳,載在虞書。朕推誠待下,前旨已行,豈可食言?承先敗走乞降,今尚列諸功臣,給敕恩養。豈遠來歸順諸將吏反謂無功?朕此言亦非責承先也,彼以誠入告,朕亦以誠開示之耳。」

旋自察哈爾得元傳國璽,承先請命工部制璽函,卜吉日,躬率群臣郊迎入宮,仍以得璽敕示滿﹑漢、蒙古,上從之。既,承先與文館諸臣隨諸貝勒文武將吏請上尊號。崇德元年,改文館為內三院,承先授內秘書院大學士。三年,改吏部右參政。四年,漢軍八旗制定,承先隸正紅旗。五年,從鄭親王濟爾哈朗等圍明錦州,令防守袞塔。耕時明兵傷我農民,承先退避不及援,坐論死,上宥之。尋以病解任。順治元年,世祖定鼎燕京,承先從入關,賜銀幣、鞍馬。二年,卒,命大學士範文程視含斂。

子敬,授三等阿思哈尼哈番,官河北總兵。康熙四年,剿流賊郝搖旗,縱不追,坐降四級。復起為大同總兵。入為鑾儀衛鑾儀使。卒。

高鴻中與承先同直文館。克永平四城,承先助守遷安,而令鴻中助守灤州,蓋使文館諸臣習武事。旋以鴻中領甲喇額真。天聰五年,設六部,授刑部承政。六年,疏論刑部事當釐正者四,謂:「諸臣敕書賜免死,有罪宜先去『免死』字,更有罪乃追敕書,不當遽議削奪。諸臣坐罪輒罰鍰,非古制;且罰鍰視職崇庳,不問罪輕重,宜有定程。滿民有罪待讞,所屬牛錄若家主,輒與讞獄吏同坐,辨論紛擾,擬嚴定以罪,著為令。刑曹讞獄,滿、漢官會讞,民不便,宜令滿官主滿民獄訟,漢官主漢民獄訟。」旋復條奏時政,上諭文館諸臣曰:「上書建言,固不可禁遏。鴻中疏多言古人過失,昔元成吉思皇帝子察罕代以刀削檉柳為鞭,曰:『我國,父皇所定;此檉柳鞭,乃我所手創也。』其臣俄齊爾塞臣曰:『非先帝鳩工制此刀,則此檉柳豈能以指削,以齒齧耶?凡此土地人民一切諸政,皆先帝所創立。』今榜式等當以此等事相啟迪,毋妄議前人為也。」既又疏論兵,略謂:「上策宜薄明都,中策先取山海。當申軍令,毋辱婦女,毋妄殺人,毋貪財物。有以離家久得財多而勸還師者,上毋為所惑。」九年,以所屬戶口耗減,坐黜。

論曰:太祖時,儒臣未置官署。天聰三年,命諸儒臣分兩直,譯曰「文館」,亦曰「書房」;置官署矣,而尚未有專官,諸儒臣皆授參將、游擊,號榜式;未授官者曰「秀才」,亦曰「相公」。崇德改元,設內三院,希福、文程、承先及剛林授大學士,是為命相之始。希福屢奉使,履險效忱,撫輯屬部;文程定大計,左臺贊襄,佐命勛最高;完我忠讜耿耿,歷挫折而不撓,終蒙主契;承先以完我薦直文館,而先完我入相,參預軍畫。間除敵帥,皆有經綸。草昧之績,視蕭、曹、房、杜,殆無不及也。


\end{pinyinscope}