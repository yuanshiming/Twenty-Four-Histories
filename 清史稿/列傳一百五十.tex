\article{列傳一百五十}

\begin{pinyinscope}
曹振鏞文孚英和王鼎穆彰阿潘世恩

曹振鏞,字儷笙,安徽歙縣人,尚書文埴子。乾隆四十六年進士,選庶吉士,授編修。大考三等,高宗以振鏞大臣子,才可用,特擢侍講。累遷侍讀學士。嘉慶三年,大考二等,遷少詹事。父憂歸,服闋,授通政使。歷內閣學士,工部、吏部侍郎。十一年,擢工部尚書。高宗實錄成,加太子少保。調戶部,兼翰林院掌院學士。十八年,調吏部尚書、協辦大學士。尋拜體仁閣大學士,管理工部,晉太子太保。二十五年,仁宗崩,樞臣撰遺詔,稱高宗誕生於避暑山莊,編修劉鳳誥知其誤,告振鏞,振鏞召對陳之,宣宗怒,譴罷樞臣。尋命振鏞為軍機大臣。宣宗治尚恭儉,振鏞小心謹慎,一守文法,最被倚任。

道光元年,晉太子太傅、武英殿大學士。三年,萬壽節,幸萬壽山玉瀾堂,賜宴十五老臣,振鏞年齒居末,特命與宴繪像。四年,充上書房總師傅。六年,入直南書房。七年,回疆平,晉太子太師。八年,張格爾就擒,晉太傅,賜紫韁,圖形紫光閣,列功臣中。振鏞具疏固辭,詔凡軍機大臣別繪一圖,以遂讓功之心,而彰輔弼之效。禦制贊曰:「親政之始,先進正人。密勿之地,心腹之臣。問學淵博,獻替精醇。克勤克慎,首掌絲綸。」親書以賜之。十一年,以萬壽慶典賜雙眼花翎。

十五年,卒,年八十有一。自繕遺疏,附摺至十餘事。上震悼,詔曰:「大學士曹振鏞,人品端方。自授軍機大臣以來,靖恭正直,歷久不渝。凡所陳奏,務得大體。前大學士劉統勛、硃珪,於乾隆、嘉慶中蒙皇祖、皇考鑒其品節,賜謚文正。曹振鏞實心任事,外貌訥然,而獻替不避嫌怨,朕深倚賴而人不知。揆諸謚法,足以當『正』字而無媿。其予謚文正。」入祀賢良祠。擢次子恩★K9四品卿。

振鏞歷事三朝,凡為學政者三,典鄉會試者各四。衡文惟遵功令,不取淹博才華之士。殿廷御試,必預校閱,嚴於疵累忌諱,遂成風氣。凡纂修會典、兩朝實錄、河工方略、明鑒、皇朝文穎、全唐文,皆為總裁。駕謁諸陵及秋獮木蘭,每命留京辦事。臨雍視學,命充直講。恩眷之隆,時無與比。數請停罷不急工程,撙節糜費。世以鹽筴起家,及改行淮北票法,舊商受損,振鏞曰:「焉有餓死之宰相家?」卒贊成,世特以稱之。

文孚,字秋潭,博爾濟吉特氏,滿洲鑲黃旗人。由監生考授內閣中書,充軍機章京。嘉慶四年,從那彥成赴陜西治軍需。八年,隨扈秋獮,校射中四矢,賜花翎。十一年,以在直勤,擢四五品京堂,授內閣侍讀學士。歷鴻臚寺卿、通政司副使。命履勘綏遠城渾津、黑河鹼地改徵,及大青山牧廠餘地招墾事。十三年,予副都統銜,充西寧辦事大臣。疏言:「青海蒙、番,重利輕命。自來命盜諸案,一經罰服,怨仇消釋。若必按律懲辦,不第犯事之家仇隙相尋,被害者心反觖望,相習成風,不可化誨。溯蒙、番內附以來,雍正十一年大學士鄂爾泰等議纂番例頒行,聲明俟五年後始依內地律例辦理。乾隆年間疊經展限,茲復奉命詳議。臣以為番、民糾結滋擾,或情同叛逆,或關系邊陲大局,自應從嚴懲辦。若其自相殘殺及盜竊之案,向以罰服完結,相安巳久。必繩以內地法律,轉恐愚昧野番,群疑滋懼,非綏服邊氓之道。」疏入,下軍機大臣議行。

十六年,召回京,授鑲白旗滿洲副都統。偕內閣學士阮元勘議山西鹽務,疏請停止吉蘭泰鹽官運,改並潞商引額,以潞引之有餘,補吉課之不足,吉鹽許民撈販,限制水運至皇甫川而止,下部議行。尋授內閣學士,遷刑部侍郎。十八年,緣事降調,予二等侍衛,命赴山東治軍需。復授內閣學士,歷山海關副都統、馬蘭鎮總兵、錦州副都統。二十年,召授刑部侍郎。二十四年。命在軍機大臣上學習行走。偕侍郎帥承瀛赴山東鞫獄,並勘蘭儀決口,督濬引河。次年春,竣工,予議敘。調戶部,又調工部,擢左都御史。宣宗即位,以樞臣撰擬遺詔不慎,先後罷直,文孚獨留。道光二年,命往陜西按鞫渭南縣民柳全璧毆斃人命獄,論知縣徐潤受人囑託、疏脫正兇、事後得贓,枷號兩月,遣戍伊犁;升任西安知府鄧廷楨偏執枉縱,訊無貪酷,革職免發遣;巡撫硃勛失察,議革職,降四五品京堂。四年,仁宗實錄成,加太子太保。

南河阻運,詔責減黃蓄清;至十一月洪湖水多,啟壩而高堰、山盱石工潰決,命文孚偕尚書汪廷珍馳往按治,奏劾河督張文浩於御黃壩應閉不閉,五壩應開不開,湖水過多,致石工掣塌萬餘丈,請遣戍伊犁;兩江總督孫玉庭徇隱回護,交部嚴議。議於御黃壩外添建三壩,鉗束黃流。壩內外及束清、運口各壩兩岸築纖道,多作土壩,挑濬長河,幫培堤身,以利漕行。速挑引河,引清入運;堵閉束清壩,杜黃入湖;又議覆侍郎硃士彥條陳五事,由河臣勘辦。疏上,並依議行。命文孚等回京,責嚴烺、魏元煜辦理,而引黃濟運仍不得要領,河、漕交困。

八年,回疆底定,首逆就擒,晉太子太傅,賜紫韁,繪像紫光閣,禦制贊有「和而不同,公正以清」之褒。十一年,以吏部尚書協辦大學士。十四年,拜東閣大學士,管理吏部。十五年,轉文淵閣大學士。以疾請解職,優詔慰諭,許罷直軍機。十六年,致仕。二十一年,卒,贈太保,謚文敬。

英和,字煦齋,索綽絡氏,滿洲正白旗人,尚書德保子。少有俊才,和珅欲妻以女,德保不可。乾隆五十八年,成進士,選庶吉士,授編修,累遷侍讀。嘉慶三年,大考二等,擢侍讀學士。洎仁宗親政。知其拒婚事,嘉焉,遂鄉用,累遷內閣學士。五年,授禮部侍郎,兼副都統。六年,充內務府大臣,調戶部。以不到旗署為儀親王所糾,罷副都統。七年,直南書房。扈蹕木蘭,射鹿以獻,賜黃馬褂。授翰林院掌院學士。九年,帝幸翰林院,賜一品服,加太子少保,命在軍機大臣上學習行走。時詔稽巡幸五臺典禮,英和疏言教匪甫平,民未蘇息,請俟數年後再議,上嘉納之。尋自請獨對,論大學士劉權之徇情欲保薦軍機章京袁煦,上不悅,兩斥之。遂罷直書房、軍機,降太僕寺卿。歷內閣學士,理籓院、工部侍郎。

數奉使出按事,河東鹽課歸入地丁,而蒙古鹽侵越內地,命偕內閣學士初彭齡往會巡撫察議。疏言:「非禁水運不能限制蒙鹽,非設官商不能杜絕私販。請阿拉善鹽祗由陸路行銷,河東鹽仍改商運。吉蘭泰鹽池所產亦招商運辦。」事詳鹽法志。兼左翼總兵,復為內務府大臣。十二年,偕侍郎蔣予蒲查南河料物加價,議準增添,仍示限制,從之。復直南書房。十三年,命暫在軍機大臣上行走,調戶部、武英殿。進高宗聖訓廟號有誤,坐降調內閣學士。尋遷禮部侍郎。十八年,隨扈熱河,會林清逆黨為變,命先回京署步軍統領。擒林清於黃村西宋家莊,實授步軍統領、工部尚書。滑縣平,復太子少保。

十九年。將開捐例,廷議不一。偕大學士曹振鏞等覆議,獨上疏曰:「理財之道,不外開源節流。大捐為權宜之計,本朝屢經舉行。但觀前事,即知此次未必大效。竊以開捐不如節用,開捐暫時取給,節用歲有所餘。請嗣後謁陵,或三年五年一舉行,民力可紓。木蘭秋獮,為我朝家法,然蒙古迥迥昔比,亦請間歲一行,於外籓生計所全實大。各處工程奉旨停止,每歲可省數十萬至百餘萬不等。天下無名之費甚多,茍於國體無傷,不得任其糜費。即如裁撤武職名糧,未必能禁武官不役兵丁,而驟增養廉百餘萬,應請敕下部臣詳查正項經費外,歷年增出各款,可裁則裁,可減則減,積久行之,國計日裕。至開源之計,不得以事涉言利,概行斥駁。新疆歲支兵餉百數十萬,為內地之累,其地金銀礦久經封閉,開之而礦苗旺盛,足敷兵餉;各省礦廠,亦應詳查興辦。又戶部入官地畝,請嚴催升科,於國用亦有裨益。」疏入,詔以名糧巳飭覈辦,開礦流弊滋多,仍依眾議,豫工事例遂開。是歲調吏部,復命暫在軍機大臣上行走。

二十五年,宣宗即位,命為軍機大臣,調戶部。宣宗方銳意求治,英和竭誠獻替。面陳各省府、州、縣養廉不敷辦公,莫不取給陋規,請查明分別存革,示以限制。上採其言,下疆吏詳議,而中外臣工多言其不可,詔停其議,遂罷直軍機,專任部務。道光二年,以戶部尚書協辦大學士,兼翰林院掌院學士。四年,仁宗實錄成,加太子太保。五年,洪澤湖決,阻運道,河、漕交敝,詔籌海運,疆臣率拘牽成例,以為不可。英和奏陳海運、折漕二事為救時之計,越日復上疏,略謂:「河、漕不能兼顧,惟有暫停河運以治河,雇募海船以利運,而任事諸臣未敢議行者,一則慮商船到津,難以交卸;一則慮海運既行,漕運員弁、旗丁、水手難以安插。」因陳防弊處置之策甚悉。詔下各省妥議,仍多諉為未便,惟江蘇巡撫陶澍力行之,撥蘇、松、常、鎮、太五屬漕米,以河船分次海運。六年八月,悉數抵天津,上大悅,詔嘉英和創議,予議敘,特賜紫韁以旌異之。

張格爾犯回疆,英和疏陳進兵方略,籌備軍需,並舉長齡、武隆阿可任事,多被採用。七年,奏商人請於易州開採銀礦,詔斥其冒昧。調理籓院,罷南書房、內務府大臣。未幾,坐家人增租擾累,出為熱河都統。八年,命勘南河工程。回疆平,復太子少保。授寧夏將軍,以病請解職,允之。

初,營萬年吉地於寶華峪,命英和監修,嘗從容言漢文帝薄葬事,上稱善,議於舊制有所裁省,工竣,孝穆皇后奉安,優予獎敘。至是地宮浸水,譴責在事諸臣。詔以英和始終其事,責尤重,奪職,籍其家。逮訊,得開工時見有石母滴水,僅以土攔,議設龍須溝出水,英和未允狀,讞擬大闢,會太后為上言不欲以家事誅大臣,乃解發黑龍江充當苦差,子孫並褫職。十一年,釋回,復予子孫官。二十年。卒,贈三品卿銜。

英和通達政體,遇事有為,而數以罪黜。屢掌文衡,愛才好士。自其父及兩子一孫,皆以詞林起家,為八旗士族之冠。子奎照,嘉慶十九年進士,歷官至禮部尚書、軍機大臣,緣事奪職,復起為左都御史;奎耀,嘉慶十六年進士,官至通政使,後為南河同知。奎照子錫祉,道光十五年進士,歷翰林院侍講學士,後官長蘆鹽運使。

王鼎,字定九,陜西蒲城人。少貧,力學,尚氣節。赴禮部試至京,大學士王傑與同族,欲致之,不就。傑曰:「觀子品概,他日名位必繼吾後。」嘉慶元年,成進士,選庶吉士。丁母憂,服除,授編修。兩以大考升擢,累遷內閣學士。十九年,授工部侍郎。仁宗諭曰:「朕向不知汝,亦無人保薦。因閱大考考差文字,知汝學問。屢次召見奏對,知汝品行。汝是朕特達之知。」調吏部,兼署戶部、刑部。二十三年,兼管順天府尹事,復諭曰:「朕初意授汝督撫,今管順天府尹,猶外任也。且留汝在京,以備差往各省查辦事件。」自是數奉使出按事鞫獄。二十四年,調刑部,又調戶部。

道光二年,河南儀工奏銷不實,解巡撫姚祖同任,命鼎偕侍郎玉麟往按,暫署巡撫。疏陳:「儀工用款至辦奏銷,與部例成規不符。乃以歷辦物料、土方價值,合之豫省成規,互相增減,於稭料、引河等款增銷一百三十萬,夫工、麻斤各款減銷一百三十萬,雖有通融,銀數仍歸實用。惟八子錢一款,以銀易錢,多於舊價,每兩提八十文充入經費,而於各員應繳之銀,一並扣算,實違定制。」疏入,命覈實報銷,而薄譴祖同。是年,擢左都御史,父憂歸。五年,服闋,以一品銜署戶部侍郎,授軍機大臣。

浙江德清徐倪氏因奸謀斃徐蔡氏獄三年不決,按察使王維詢因自盡,巡撫程含章與按察使祁鞫之,甫得情而犯婦在監自縊。宣宗特命鼎典鄉試,就治其獄,廉得徐故富家,以獄破其產,官吏多受賕,勾結朦庇,致獄情譸幻。悉發其覆,置之法,浙人稱頌焉。六年,授戶部尚書。八年,回疆平,以贊畫功,加太子太保,繪像紫光閣。

蘆鹽積疲,商累日重,命鼎偕侍郎敬徵察辦。議以;「鹽務首重年清年款,先將節年帶徵釐剔,現年正款不難按數清完。道光二年以前未完銀九百餘萬為舊欠,三年以後未完銀為新欠,緩舊徵新。請以堰工加價二文,半解部充公,半抵完商欠。新欠抵完,續抵舊欠。蘆商生息帑本內,直隸水利、趙北口兩項非經費歲需,請停利三年。限滿加一倍利,本息同徵。舊有拔繳水利帑本一百十七萬兩,請停徵三年。自道光十一年起,歲徵十萬兩,五萬完舊本,五萬完新本,以恤商力。近年商力疲乏,不能預買生鹽,存坨新鹽多滷耗。請每包加鹽十三斤,俾資貼補,從此款目既清,庶經久可行。」又請免繳嘉慶十七年加價交官半文未完銀一百八十四萬餘兩。疏入,並允行。十年,蘆商呈請調劑,復命鼎及侍郎寶興往按。鼎以前次清查,傳集各商詳詢定議,皆稱可免虧累積壓,雖因銀價漸昂,尚不致遽形虧折,遂議駁。時淮鹽尤敝,兩江總督陶澍疏陳積弊情形,命鼎偕寶興會同籌議。中外論鹽事者,多主就場徵稅。疏言:「詳覈淮綱全局,若改課歸場灶,尚多窒兒。惟有就舊章大加釐剔,使射利者無可借端,欠課者無可藉口,似較有往轍可循。擬定章程十五條,曰:裁浮費,減窩價,刪繁文,慎出納,裁商總,覈滯銷,緩積欠,恤灶丁,給船價,究淹銷,疏運道,添岸店,散輪規,飭紀綱,收灶鹽。」又請裁撤兩淮鹽政,改歸總督辦理,以一事權。並詔允行。陶澍得銳意興革,淮綱自此漸振,鼎之力也。十一年,署直隸總督。十二年,管理刑部事務。十五年,協辦大學士,仍管刑部,直上書房。十八年,拜東閣大學士。二十年。加太子太保。

二十一年夏,河決祥符,命偕侍郎慧成往治之,尋署河督。議者以水勢方漲,不宜遽塞,請遷省城以避其沖,鼎持不可,疏言:「河灌歸德、陳州及安徽亳、潁,合淮東注洪澤湖,湖底日受淤。萬一宣洩不及,高堰危,淮、揚成巨浸,民其魚矣!無論舍舊址、築新堤數千里,工費不貲,且自古無任黃水橫流之理。請飭戶部速具帑,期以冬春之交集事。不效,原執其咎。」具陳民情安土重遷、省垣可守狀。初至汴城,四面皆水,旦夕且圮,躬率吏卒巡護,獲無恙。洎工興,親駐工次,倦則寢肩輿中。次年二月,工竣,用帑六百萬有奇。前此馬營工用一千二百餘萬,儀封工用四百七十五萬,原議以儀工為率。及蕆事,加增百餘萬,然事艱於前,微鼎用節工速,不能如是。敘功,晉太子太師。積勞成疾,命緩程回京。

自禁煙事起,英吉利兵犯沿海,鼎力主戰。至和議將成,林則徐以罪譴,鼎憤甚,還朝爭之力,宣宗慰勞之,命休沐養痾。越數日,自草遺疏,劾大學士穆彰阿誤國,閉戶自縊,冀以尸諫。軍機章京陳孚恩,穆彰阿黨也。滅其疏,別具以聞。上疑其卒暴,命取原不得,於是優詔憫惜,贈太保,謚文恪,祀賢良祠。後陜西巡撫請祀鄉賢,特詔允之。

鼎清操絕俗,生平不受請託,亦不請託於人。卒之日,家無餘貲。子沆,道光二十年進士,翰林院編修。

穆彰阿,字鶴舫,郭佳氏,滿洲鑲藍旗人。父廣泰,嘉慶中,官內閣學士,遷右翼總兵。坐自請兼兵部侍郎銜,奪職。

穆彰阿,嘉慶十年進士,選庶吉士,授檢討。大考,擢少詹事。累遷禮部侍郎。二十年,署刑部侍郎。因一日進立決本二十餘件,詔斥因循積壓,堂司各員並下嚴議,降光祿寺卿。歷兵部、刑部、工部、戶部侍郎。道光初,充內務府大臣,擢左都御史、理籓院尚書。以漕船滯運,兩次命署漕運總督。召授工部尚書,偕大學士蔣攸銛查勘南河。洎試行海運,命赴天津監收漕糧,予優敘。七年,命在軍機大臣上學習行走。逾年,張格爾就擒,加太子少保。授軍機大臣,罷內務府大臣,直南書房。尋兼翰林院掌院學士,歷兵部、戶部尚書。十四年,協辦大學士。承修龍泉峪萬年吉地,工竣,晉太子太保,賜紫韁。十六年,充上書房總師傅,拜武英殿大學士,管理工部。

十八年,晉文華殿大學士。時禁煙議起,宣宗意銳甚,特命林則徐為欽差大臣,赴廣東查辦。英吉利領事義律初不聽約束,繼因停止貿易,始繳煙,盡焚之,責永不販運入境,強令具結,不從,兵釁遂開。則徐防禦嚴,不得逞於廣東,改犯閩、浙,沿海騷然。英艦抵天津,投書總督琦善,言由則徐啟釁。穆彰阿窺帝意移,乃贊和議,罷則徐,以琦善代之。琦善一徇敵意,不設備,所要求者亦不盡得請,兵釁復起。先後命奕山、奕經督師,廣東、浙江皆挫敗。英兵且由海入江,林則徐及閩浙總督鄧廷楨、臺灣總兵達洪阿、臺灣道姚瑩以戰守為敵所忌,並被嚴譴,命伊里布、耆英、牛鑒議款。二十二年,和議成,償幣通商,各國相繼立約。國威既損,更喪國權,外患自此始。

穆彰阿當國,主和議,為海內所叢詬。上既厭兵,從其策,終道光朝,恩眷不衰。自嘉慶以來,典鄉試三,典會試五。凡覆試、殿試、朝考、教習庶吉士散館考差、大考翰詹,無歲不與衡文之役。國史、玉牒、實錄諸館,皆為總裁。門生故吏遍於中外,知名之士多被援引,一時號曰「穆黨」。文宗自在潛邸深惡之,既即位十閱月,特詔數其罪曰:「穆彰阿身任大學士,受累朝知遇之恩,保位貪榮,妨賢病國。小忠小信,陰柔以售其奸;偽學偽才,揣摩以逢主意。從前夷務之興,傾排異己,深堪痛恨!如達洪阿、姚瑩之盡忠盡力,有礙於己,必欲陷之;耆英之無恥喪良,同惡相濟,盡力全之。固寵竊權,不可枚舉。我皇考大公至正,惟以誠心待人,穆彰阿得肆行無忌。若使聖明早燭其奸,必置重典,斷不姑容。穆彰阿恃恩益縱,始終不悛。自朕親政之初,遇事模棱,緘口不言。迨數月後,漸施其伎倆。英船至天津,猶欲引耆英為腹心以遂其謀,欲使天下群黎復遭荼毒。其心陰險,實不可問!潘世恩等保林則徐,屢言其『柔弱病軀,不堪錄用』;及命林則徐赴粵西剿匪,又言『未知能去否』。偽言熒惑,使朕不知外事,罪實在此。若不立申國法,何以肅綱紀而正人心?又何以不負皇考付託之重?第念三朝舊臣,一旦置之重法,朕心實有不忍,從寬革職永不敘用。其罔上行私,天下共見,朕不為已甚,姑不深問。朕熟思審處,計之久矣,不得已之苦衷,諸臣其共諒之!」詔下,天下稱快。咸豐三年,捐軍餉,予五品頂戴。六年,卒。

子薩廉,光緒五年進士,由翰林官至禮部侍郎。

潘世恩,字芝軒,江蘇吳縣人。乾隆五十八年一甲一名進士,授修撰。嘉慶二年,大考一等,擢侍讀。和珅以其青年上第有才望,欲招致之,世恩謝不與通。以次當遷,和珅抑題本六閱月不上。仁宗親政,乃擢侍講學士。一歲三遷至內閣學士,歷禮部、兵部、戶部、吏部侍郎,督云南、浙江、江西學政。十七年,擢工部尚書,調戶部。母憂歸,服除,以父老乞養,會其子登鄉舉,具疏謝,坐未親詣京,降侍郎。帝鑒其孝思,仍允終養,居家十載。

道光七年,父喪服闋,補吏部侍郎,遷左都御史。再授工部尚書,調吏部。十三年,超拜體仁閣大學士,管理戶部。尋命為軍機大臣,兼翰林院掌院學士。晉東閣大學士,調管工部。充上書房總師傅,加太子太保。十八年,晉武英殿大學士。二十八年,以八十壽晉太傅,賜紫韁。其明年,引疾,迭疏乞休,溫詔慰留,僅解機務。三十年,文宗即位,復三疏,始得予告,食全俸,留其子京邸。咸豐二年,鄉舉重逢,詔就近與順天鹿鳴宴。次年,復與恩榮宴。四年,卒,遣親王奠醊,入祀賢良祠,謚文恭。

世恩歷事四朝,迭掌文衡,備叨恩遇。筦部務,安靜持大體。黑龍江將軍請增都爾特六屯,議地當游牧,開墾非計,不可許。言官奏山東鹽課請歸地丁,議山東場灶半毗連淮境,一歸地丁,聽民自運自銷,必為兩淮引課之累,不可行。

在樞廷凡十七年,益慎密,有所論列,終不告人。海疆事起,林則徐所論奏,廷議多贊之;及穆彰阿主撫,世恩心以為非,不能顯與立異。迨咸豐初詔舉人才,世恩已在告,疏言林則徐歷任封疆,有體有用,請徵召來京備用,並薦前任臺灣道姚瑩,文宗韙之,於罪穆彰阿時猶舉其言。次子曾瑩,道光二十一年進士,由編修官至吏部侍郎。孫祖廕,自有傳。

論曰:守成之世,治尚綜覈,而振敝舉衰,非拘守繩墨者所克任也。況運會平陂相乘,非常之變,往往當承平既久,萌蘗蠢兆於其間,馭之無術,措置張皇,而庸佞之輩,轉以彌縫迎合售其欺,其召亂可幸免哉?宣宗初政,一倚曹振鏞,兢兢文法;及穆彰阿柄用,和戰游移,遂成外患。一代安危,斯其關鍵已。英和才不竟用,王鼎忠貞致身,文孚、潘世恩皆恪恭保位者耳。


\end{pinyinscope}