\article{列傳一百八十二}

\begin{pinyinscope}
常大淳雙福王錦繡常祿王壽同蔣文慶

陶恩培多山吉爾杭阿劉存厚★闊周兆熊

羅遵殿王友端繆梓徐有壬王有齡

常大淳,字蘭陔,湖南衡陽人。道光三年進士,選庶吉士,授編修,遷御史。湖南鎮筸兵變,戕營官,鎮道莫敢誰何,大淳疏劾之。出為福建督糧道,署按察使。晉江縣獲洋盜三百八十餘人,總督欲駢誅之,大淳力爭,全活脅從者近三百人。司獄囚滿,大淳曰:「囚不皆死罪,獄無隙地,疫作且死。」乃分別定擬遣釋,囹圄一清。歷浙江鹽運使、安徽按察使。母憂歸,服闋,授湖北按察使,遷陜西、湖北布政使。三十年,擢浙江巡撫。

咸豐元年,海盜布興肆擾,疏劾黃巖、溫州、乍浦三鎮總兵應調遲延,親赴寧波,與提督會剿,降其渠,凡五月事定。二年,調湖北。粵匪犯長沙,土匪蜂起,或議停文武鄉試,大淳不可,終事無譁。尋調山西,未行,時總督程矞採駐防湖南,失機獲罪,徐廣縉代之,駐湖南督師,而賊勢益張。兩湖集兵長沙,防岳州者僅千人,大淳奏調陜甘兵未至,岳州土匪王萬里等踞桃林,檄防兵討之,萬里遁,而粵匪己走寧鄉,破益陽,出臨資口。

先是,大淳檄巴陵紳士吳士邁練漁勇防水路,扼土星港設柵,千人守之,商賈民船萬餘,皆阻柵不得行。及賊至,漁勇潰,船悉為賊有,水陸並下。提督博勒恭武守岳州,不戰而走,城遂陷。武漢大震,兵不滿五千,奏留江南提督雙福募勇繕城為守禦計,而兩司以下亦少應變才。大淳性仁柔,但以好語拊循士卒,莫能得其死力。賊至,先陷漢陽,作浮橋攻武昌。提督向榮自湖南來援,距城十餘里,阻賊不得前。十二月,賊由江岸穴地轟城,遂陷,大淳死之,妻劉、子集松、子婦馬、孫女淑英並殉。詔贈總督,謚文節,祀昭忠祠,並於湖北建立專祠。

同城文武被難者,提督雙福,學政、光祿寺卿馮培元,布政使梁星源,按察使瑞元,道員王壽同、王東槐、林恩熙,知府明善、董振鐸,同知周祖銜,知縣繡麟,而總兵王錦繡、常祿皆以援師入城助守,同殉焉。馮培元、王東槐自有傳。

雙福,他塔拉氏,滿洲正白旗人。由護軍從征喀什噶爾,洊升參領,出為湖北副將。剿崇陽匪鍾人傑,功最,賜號烏爾瑪斯巴圖魯,累擢河北、古州兩鎮總兵,江南提督。大淳疏請留防,改授湖北提督。城陷,死之。子德齡,同遇害。予騎都尉兼雲騎尉世職,謚武烈。

王錦繡,廣西馬平人。由行伍累擢雲南曲尋協副將。率滇兵赴廣西剿匪,擢鄖陽鎮總兵。常祿,富察氏,滿洲鑲白旗人。由護軍校洊擢雲南副將。剿廣西匪,擢河北鎮總兵,賜號強謙巴圖魯。錦繡、常祿轉戰廣西、湖南,皆有功績。及湖北告警,偕同赴援,戰於蒲圻,獲勝,遂入武昌嬰城固守。城陷,巷戰,同死之,並優恤,予騎都尉兼雲騎尉世職。錦繡謚壯節,常祿謚剛節。

王壽同,江蘇高郵人,尚書引之子。捐納刑部郎中。道光二十四年進士。用原官遷御史,出為貴州黎平知府,擢湖北漢黃德道。在黃州募勇,令子恩晉訓練,得精銳四百人。武昌被圍,壽同率以赴援。沖賊營縋城入,任戰守,屢擊斬攻城賊。以甕德法知賊由江岸穴地道,方鑿穴出擊,地雷發,壽同率恩晉巷戰,同遇害。予騎都尉世職,祀京師昭忠祠,與子恩晉同於本籍建忠孝祠,賜兩子恩錫、恩炳並為舉人。後左都御史單懋謙疏陳壽同治績,追謚忠介。

蔣文慶,字蔚亭,漢軍正白旗人。嘉慶十九年進士,授吏部主事,遷員外郎。出為雲南曲靖知府,調雲南府。道光十二年,擢甘肅寧夏道。在邊十年,濬渠,興水利。遷浙江按察使,護理巡撫,遷安徽布政使。文宗即位,下詔求賢,巡撫王植薦之,咸豐元年,就擢巡撫。奏請鳳、潁所屬宜練團,與保甲並行。

二年,粵匪犯長沙,命遣安徽兵一千赴援湖北。總督陸建瀛慮賊窺吉安,請所調兵改赴江西。文慶疏言:「安慶、潛山等營已起程者,毋庸北還;其未出境之徽、寧二營改赴江西;仍各募足千人,俾資援應。惟安徽兵僅六千,各有分防汛地,省垣單危。潁、鳳民團強勁,臣擬增募二千;如賊氛益熾,請調江蘇兵三千。統計庫帑撥解甘肅、河工及本省兵餉銀五十五萬兩,近又以十餘萬解楚,實已無餘。乞將續收地丁契雜及蕪、鳳兩關稅入截留備用。」建瀛以文慶張皇,漸生異議。及賊至岳州,復申募勇留餉前議,始奉總理安徽防剿之命,遣按察使張熙宇、游擊賡音布扼小孤山,自與壽春鎮總兵恩長籌守御。

三年正月,賊已陷武昌,陸建瀛督師迎剿,令福山鎮總兵王鵬飛以二千人防安慶,而調恩長為行營翼長。鵬飛駐兵北門外,以客將馭新兵,安慶勢益危。文慶母年八十餘,久病,送之登舟。建瀛方溯江而上,見之大怒,將具疏劾之,語頗聞。及至,文慶稱病不出,曰:「我旦夕且得罪去耳!」建瀛至黃州,賊連舟蔽江下,恩長戰歿,兵潰於武穴,建瀛遂返,過安慶,文慶要入城計事,已不及,熙宇、鵬飛皆棄防地走。漕督周天爵奉命助守安慶,方留剿鳳、潁土匪,書抵文慶畫退守廬州之策。文慶奏上其書,賊遽至,城北兵潰,而城中譁言將退廬州,紛紛縋城下,斬之不可止。文慶吞金不死,飲藥悶絕,家人輿之出,遇賊於門,遂被害。從僕以席覆尸,赴桐城呈報,漏言自裁事。賊既去,子長綬集僚屬耆老集視,然後殮。

詔詰遺疏與呈報不符,向榮疏陳本末,乃賜恤如例,予騎都尉世職,入祀昭忠祠,安慶建專祠,謚忠愨。

陶恩培,字益之,浙江山陰人。道光十五年進士,選庶吉士,授編修,遷御史。出為湖南衡州知府。咸豐元年,廣西賊起,衡州奸民左家發謀響應,捕誅之,晉秩道員。二年春,粵匪犯衡陽。總督程矞採方駐郡,聞警,遽欲退保省城。恩培曰:「衡州,楚之門戶,棄則全楚震矣!」勿聽。乃與約,毋撤糧臺,得便宜行事。恩培誅鋤內奸,撫循兵士。賊知有備,由他道竄陷道州,犯長沙,所至皆破,惟衡州獨完。御史黎吉雲以狀聞,文宗嘉之。三年,超擢湖南按察使。剿平衡山、安仁、瀏陽、醴陵土匪,遷山西布政使。巡撫駱秉章以恩培在湖南久,疏留襄辦防務,允之。尋調任江蘇。

四年,擢湖北巡撫。時武漢再復,城郭殘破,旁近皆賊蹤,總督楊霈擁兵廣濟,按察使胡林翼出省防剿。或說恩培曰:「省城不可守,宜遷治他郡。」恩培斥其非,兼程進,歲將盡蒞任,文武員弁不足三十,兵不盈千,餉不逾萬。恩培馳書曾國籓乞援,檄胡林翼回保省城。會楊霈敗走蘄州,次於德安。五月正月,漢陽、漢口並為賊踞,興國、通山、嘉魚土匪應之,武昌益孤。恩培盡焚沿江木植,盡驅諸船,故賊未得渡,而道員李孟群、知府彭玉麟以水師至,胡林翼以陸師至,聲勢稍壯。賊城沙坡堆,恩培欲先發制之,令林翼統諸軍冒雪出不意,三路攻賊。士卒畏寒不欲戰,渡江營沌口,師期頗洩,賊得為備。林翼慮兵力分,並為一路。舟師先薄小龜山,陸師繼進。賊出馬步數千,從漢口鈔我軍,復敗退大軍山。賊舟大集,晝夜攻城。楊霈約三路來援,以火為號。林翼、孟群整軍以待,屢見火起,為所紿,而霈軍不至。二月,賊由興國、通山來助攻。林翼兵隔江為賊所綴,不能渡。城中出兵連戰於青山、望江樓,皆挫。直逼大小東門,恩培自當之,令武昌知府多山守西北城。方戰,忽報漢陽門破,多山戰死。至暮賊麕集,士卒死傷略盡,恩培投蛇山紫陽塘殉焉。詔優恤,予騎都尉兼雲騎尉世職,謚文節,祀昭忠祠。後在湖北與吳文鎔合建一祠。

多山,赫舍里氏,滿洲鑲藍旗人。道光十四年舉人,刑部郎中。出為襄陽知府,舉行團練,剿賊有功,晉秩道員。調武昌府,署按察使。時司道多駐城外督戰,惟多山助城守,城陷,力戰死之,予騎都尉世職,謚忠節。

吉爾杭阿,字雨山,奇特拉氏,滿洲鑲黃旗人。由工部筆帖式洊遷郎中,充坐糧監督。咸豐三年,以孝和睿皇后奉安山陵,晉秩道員。揀發江蘇,補常鎮道,署按察使。粵匪已踞江寧、鎮江,會匪劉麗川陷上海。巡撫許乃釗檄吉爾杭阿偕總兵虎嵩林、參將秦如虎合師進剿。

劉麗川者,廣東香山人。貿易上海,習於洋商,與蘇松太道吳健彰有舊。素行不法,見粵匪勢盛,遂倡亂,糾客籍粵、閩、江右會黨二千人,於三年秋襲上海城,戕知縣袁祖德,劫道庫,吳健彰遁入領事署。鄰境亂民紛起應之,寶山、嘉定、青浦、南匯、川沙五城連陷。蘇紳捐募川勇千人,刑部主事劉存厚領之,隸於吉爾杭阿為軍鋒,連克青浦、嘉定。諸軍至,五城以次復。合圍上海,分南北兩營。

四年春,存厚穴地轟城,以援兵不繼退。賊由北門出犯,吉爾杭阿親燃砲擊卻之。賊又劫北營,虎嵩林兵挫。吉爾杭阿固守,得不潰,復擊退西門撲營之賊,超擢布政使,賜花翎,尋擢巡撫。復於南門掘地道,火發,副將清長先登,沒於陣,兵又退。地鄰租界,匪人暗濟餉械,久不下,乃於洋涇濱築墻塞濠,斷其糧道,賊始困。負嵎已經年,洋商貿易不便,吉爾杭阿開誠曉以利害,於是法國兵官請助剿,英、美領事允讓地設防。築土墻於陳家木橋,移營進逼,下令投誠免死,縋城出者日以千計。賊襲陳家木橋,擊敗之,擒斬悍黨偽將軍林阿朋。除夕,乘賊不備,地雷發,督兵躍城入,麗川縱火逸,追擒伏誅,餘賊盡殲。捷聞,文宗嘉其功,加頭品頂戴,賜號法施善巴圖魯。

五年,命率得勝之兵馳往向榮大營,幫辦軍務,專任鎮江一路。鎮江賊酋吳汝孝最桀黠,恃金山為犄角,銀山、寶蓋山並有伏賊。是年秋,迭攻鎮江西門、南門,堵截金山、瓜洲沿江援賊,累戰皆捷。虎嵩林克寶蓋山,吉爾杭阿駐營其上,乘黃山發巨砲轟城,賊卡盡毀。江寧賊集大股由北岸渡江來援,吉爾杭阿策高資鎮為賊糧道,遣兵截擊,賊退棲霞石埠橋。偕總兵德安扼剿,留劉存厚率三營守高資煙墩山。

六年春,賊糾悍黨陳玉成、李秀成等來援,提督張國樑御之於倉頭鎮。賊潛由小港出江順流下,城賊突出應,官軍為所乘,賊遂長驅進金雞嶺,逼寶蓋山大營。吉爾杭阿拒,賊未得逞,乃渡江犯儀徵、揚州。五月,賊數萬復犯高資,存厚告急。大營兵僅八千,或謂:「賊眾且銳,不可當,姑舍高資,徐圖大舉為便。」吉爾杭阿奮然曰:「一戰絕賊糧道,鎮江旦夕且下。吾寧以死報國耳!」遂馳抵煙墩,被圍,鏖戰五晝夜,親執旗指麾,猝中砲,殞。存厚護尸突圍出,為賊所要截,歿於陣,並遺骸失之。副都統繃闊投江死。鎮江軍亦潰,副將周兆熊死之。事聞,文宗震悼,追贈吉爾杭阿總督,予一等輕車都尉世職,謚勇烈。於殉難地方建專祠,上海亦建專祠。子文鈺襲世職,賜員外郎。

存厚,字仲山,四川榮縣人。捐納刑部主事。好談兵,侍郎王茂廕疏薦,命赴江南大營,向榮命率勇擊賊,輒勝。上海之役,始自領一軍,吉爾杭阿甚倚之。克青浦,冒矢石先登,洊保知府。及攻上海,誤殺洋婦,洋人憤,將發兵相攻。存厚單騎往曰:「此不足啟邊釁,請以一身償。若欲戰,雖死不相下也!」卒議償恤而定。圍攻凡數月,方略多出存厚。既克,以首功頒賞荷囊,授江寧知府,記名道員。從攻鎮江,奪銀山,破瓜洲援賊,爭金雞嶺,皆功最。吉爾杭阿以存厚有謀略,故令守高資,及赴援戰歿,存厚大慟,力戰突圍,欲返其尸,中道遇伏,殺賊數百人,馬陷淖,被戕。予騎都尉世職,謚剛愍。

繃闊,戴佳氏,滿洲正白旗人。官頭等侍衛。從僧格林沁剿林鳳祥,戰連鎮、高唐、馮官屯,積功授正紅旗蒙古副都統。調京口,偕吉爾杭阿援高資,軍潰,墮水中,從人拯之,曰:「吾與吉公偕!吉公死,吾不獨生。」復投江死,謚勇節。

兆熊,四川成都人。官副將。從攻鎮江,駐軍城南破子岡,當賊沖。吉爾杭阿既歿,破子岡為賊困,汲道斷,兆熊固守,時以計誘擊賊,殺傷甚多。乞援於張國樑,未至,圍益逼,素得士心,無一逃者。營破,燃火藥自焚,一軍同死,謚果愍。

羅遵殿,字澹村,安徽宿松人。道光十五年進士,直隸即用知縣,歷南樂、唐山、清苑諸縣,冀州直隸州,皆有聲績。擢浙江湖州知府,調杭州,擢湖北安襄鄖荊道。遵殿在浙,以捕盜名。至湖北,檄所屬治團練,楚北民團自此始。

咸豐二年,粵匪陷武昌,土匪郭大安謀應賊,捕斬之。三年,署按察使。會捻匪窺襄、樊,遵殿還襄陽籌防。總督張亮基疏陳遵殿得民心,請提標歸其調遣。四年,武昌再陷,皖賊竄德安、安陸、荊門,遵殿率五千人出屯王家河遏賊沖,克潛江,賜花翎。尋破賊於京山,復其城,屢遣襄勇助總督楊霈防剿。五年春,武昌復陷,襄陽有備,賊不犯境。六年,遷兩淮鹽運使,留湖北治糧臺。游勇煽饑民為亂,蔓延荊、襄、鄖、宜四郡,遵殿固守,待援兵至,大破之。是年秋,武漢克復,遵殿力固上游。以盜賊起於饑寒,勸置義倉七十餘所,以稅餘銀修老龍堤捍水患,就遷湖北按察使。八年,遷布政使。時胡林翼為巡撫,百廢具舉,重遵殿清德,吏事悉倚之。

九年,擢福建巡撫,未之任,調浙江。自賊踞江寧,皖南軍事餉事悉隸浙江。屯兵寧國,恃為屏蔽。及胡興仁為巡撫,不欲餉鄰軍,又劾統將鄭魁士他調去,賊窺浙益急。遵殿到官,痛吏習浮競,乃嚴舉劾,察營伍,或不便其所為,多毀之。省垣獨總兵李定太軍六千人,知不足恃,與胡林翼商調楚軍,倉猝難應。賊已由寧國竄入浙境。遣李定太出防湖州,而廣德已陷。

十年二月,賊由獨松關逼杭州,湖南遣蕭翰慶、李元度兩軍來援,翰慶戰死,元度道阻不得前。賊壁城南山上,下臨城中。乞師江南,未至,兵少,實不能戰。浙西初經寇亂,人不知兵,議戰議守,紛紜不定。會久雨,遵殿徒步泥淖中,守浹旬,城陷,仰藥死,妻女同殉,詔予優恤。尋以御史高延祜奏劾遵殿不能禦賊,罷其恤典。

遵殿任外吏二十年,廉介絕俗,家僅土屋數椽,胡林翼集賻,乃克歸喪。同治初,詔允曾國籓之請,念其歷官有聲,到浙未久,追贈右都御史,予騎都尉世職,謚壯節。

城陷時,署布政使王友端、署按察使繆梓、杭嘉湖道葉堃、寧紹臺道仲孫懋、署杭州知府馬昂霄、署仁和知縣李福謙同殉節。

友端,安徽婺源人。道光二十七進士,授戶部主事,遷郎中。出為浙江糧道,署布政使。當粵匪之窺浙也,言於遵殿曰:「皖邊軍弱,湖州空虛,請速備廣德。」遵殿至事急始遣軍,已無及。賊遂長驅至城下,友端復請列塹湧金、清波兩門為犄角,亦不用其言。賊穴道攻城,友端懸金三千募死士縋擊,遇雨,火器不燃而敗。臨死,自書「浙江布政使王友端」八字於衿上,予騎都尉世職,謚貞介。

梓,江蘇溧陽人。道光八年舉人,大挑知縣。歷署仙居、石門、奉化諸縣。罣誤去官。值清查倉庫、水災籌賑,奉檄佐理,皆得其力。準捐輸復官,晉同知。咸豐二年,河決阻漕,獻策行海運,即以任之。蕆事,擢知府。上海為賊陷,率兵助剿;復創議疏濬劉河海口以通漕運。歷寧波、杭州知府,署杭嘉湖道,兼鹽運使。六年,署按察使。粵匪由江西窺浙,梓統軍駐常山防之,授金衢嚴道。八年,粵匪陷江山,犯衢州,偕總兵李定太合擊走之,再署按察使。當賊圍杭州,梓署鹽運使兼按察使,管營務處,城守事專任之。臨時調集,兵不滿四千,城大,不敷守堞。人心惶懼,動輒譁譟。或以閉城為張皇,繼又謂戰緩為退縮。梓奔走籌守御,兩次縋城攻賊皆失利。城紳促戰急,而民與兵相仇。梓知不可為,以死自誓。守清波門雲居山,偵賊掘地道,急開內壕。未竣,地雷猝發,城圮軍潰。身被數十創,死之。事聞,賜恤。巡撫王有齡追論梓創議株守,奪恤典。及杭州再復,舉人趙之謙訴於京,下巡撫左宗棠確查。疏言:「梓居官廉幹,臨難慘烈,請還恤典。」後巡撫李瀚章、楊昌濬屢為疏請,贈太常寺卿,祀昭忠祠,並建專祠,予騎都尉世職,謚武烈。

徐有壬,字鈞卿,順天宛平人,原籍浙江烏程。道光九年進士,授戶部主事,洊升郎中。出為四川成綿龍道,署按察使。治啯匪,擒其魁,餘黨解散。遷廣東鹽運使,署按察使,清遠土匪戕官,馳剿平之。遷四川按察使。文宗即位,下詔求言,司道率引嫌,罕所陳奏。有壬獨密疏,論事切直。遷雲南布政使,調湖南。咸豐五年,以母憂回原籍。浙江巡撫何桂清奏起有壬治團防。粵匪由寧國窺湖州,有壬扼長興,設伏敗之,賊去。八年,服闋,命筦江蘇糧臺,擢江蘇巡撫。槍船匪首程鵬士擾嘉興、湖州,地方官不能制,潛至蘇州,偵獲之,置諸法。

有壬之起,由何桂清所薦。及同官江蘇,無所阿附。十年春,粵匪復犯湖州。有壬咨商桂清,遣游擊曾秉忠率舟師往援。水陸夾擊,賊被創退。尋復出東壩、溧陽,間道徑趨杭州。急請調提督張玉良馳援,杭州甫陷旋復。桂清奏捷,惟言籓司王有齡功,得擾擢,有壬僅予議敘。未幾,和春等師潰,退守丹陽,有壬急運糧械濟之,而張國樑、和春先後戰歿,何桂清棄常州不守。四月,賊遂長驅犯蘇州。有壬移檄責讓,桂清抗疏劾之。張玉良自請助守城,令屯葑門外,忽夜遁。明日,有壬巡城,廣勇通賊,開門納賊。短兵巷戰,賊矛刺有壬冠,抗聲罵賊,遇害。子震翼與妾、女同死。詔優恤,予騎都尉世職,謚莊愍,蘇州建專祠。

有壬幼時嘗覽族譜,得遠祖應鑣闔門殉節事,慨然曰:「吾他日當如此!」至是果驗。八歲解勾股術,父死,依叔父於京師,師事姚學塽。學必求有用,尤精歷算,著有務民義齋算學行世。

王有齡,字雪軒,福建侯官人。道光中,捐納浙江鹽大使,改知縣。歷慈谿、定海、鄞、仁和,皆有聲。以勞晉秩知府。咸豐五年,授杭州知府。巡撫何桂清器其幹略,迭署鹽運使、按察使,擢雲南糧儲道,仍留浙治防。桂清總督兩江,奏調赴上海議通商稅則。七年,擢江蘇按察使,遷布政使。有齡長於理財,桂清素信之深,一切倚畀,益得發舒,事皆專斷,巡撫受成而已。

十年,粵匪陷杭州,將以掣動江南全,局故援兵至,賊即不戰而走。桂清推功於有齡,遂擢浙江巡撫。詔趣率兵速赴,會辦軍務及善後事宜,而賊已回撲江南大營。和春等軍潰,常州、蘇州相繼陷,進逼嘉興,提督張玉良迎擊,敗績,杭州戒嚴。有齡率閩兵屯北新關外,遣撫標兵要賊於賣魚橋,夾擊敗之,賊乃卻。設捐輸局,奏請派在籍前左副都御史王履謙、前漕運總督邵燦督同辦理。賊眾十餘萬由徽州入浙,陷嚴州,合嘉興、廣德兩路分撲省城,有齡偕將軍瑞昌調兵迎擊走之,圍得解,復餘杭,加頭品頂戴。尋復嚴州。

十一年,復江山、常山、富陽、遂安、海寧、臨安等縣。賊擾太湖東山,總兵王之敬戰失利。至夏,賊復陷江山、常山、長興、金華、遂昌、松陽、處州、永康、義烏,革職留任。張玉良扼要隘為諸軍應援,兵先潰,賊勢益橫。檄諸將往援,無應者,處州鎮總兵文瑞率江西援兵三千,有齡待之素厚,乃自請行。進駐金華孝順街,聞蘭谿兵敗,遽潰;退守浦江,賊躡之,檄師往援,半途復潰:浦江、嚴州相繼陷。總兵劉季三、副將劉芳戰死於富陽。諸將見賊多走,不任戰,惟要索軍食。富民捐輸已倦,而有司持之急。於是團練大臣王履謙劾有齡虐捐,遇事多齟,上疏互訐。十月,蕭山、諸暨及紹興府皆陷,餉源遂絕。時援軍多不足恃,有齡復奏用李元度為按察使,募湘勇八千入浙,至龍游,阻不得前。賊酋李秀成悉眾圍杭州城,副將楊金榜敗死,張玉良攻克羅木營賊壘,亦中飛砲死:城中奪氣,且食盡,饑民死者枕藉。十二月,賊梯城入,兵潰,有齡服毒不死,縊於閣,秀成見之,為具棺殮焉。

事聞,言官顏宗儀、高延祜、硃潮先後疏劾勒捐斂怨,下曾國籓按,奏言:「有齡在浙,官紳不和,不能馭兵,以致僨事;仍以糧盡援絕,見危授命,大節無虧。」詔依例賜恤,謚壯愍。入祀昭忠祠,浙江,福建建專祠。同殉者,學政張錫庚、提督饒廷選、總兵文瑞、署布政使麟趾、按察使甯曾綸、督糧道暹福、仁和知縣吳保豐。錫庚、廷選、文瑞並自有傳。

論曰:粵匪自陷岳州,勢不可遏。及犯武昌,援兵雖至,無能為力。安慶倉猝籌防,益無措手矣。武昌凡三陷,湖北兵不可用,曾國籓言之痛切。杭州初陷,由於無兵,後則蘇、常已失,脣亡齒寒。蘇州素倚江南大軍為屏蔽,大軍潰,則勢難幸全。常大淳、蔣文慶、陶恩培、羅遵殿、徐有壬諸人,皆不失為承平良吏,短於應變,或因受事於已危,莫能挽救。王有齡素負才略,以掊克失人心,措施亦未盡當焉。吉爾杭阿治兵有法,克上海為全功,朝廷倚以規復鎮江,使非中道而殞,必有成效,其建樹非諸人所可同語也。


\end{pinyinscope}