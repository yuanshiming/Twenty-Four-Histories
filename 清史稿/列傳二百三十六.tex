\article{列傳二百三十六}

\begin{pinyinscope}
錫良周馥陸元鼎張曾易又楊士驤馮煦

錫良,字清弼,巴岳特氏,蒙古鑲藍旗人。同治十三年進士,用山西知縣,歷任州縣有惠政。光緒初,晉大旱,錫良歷辦賑務,戶必清查,款必實放,民皆德之。二十年,山東巡撫李秉衡奏調補沂州知府,擢兗沂曹濟道。抵任,值單縣大刀會滋事,亟率隊往,張示諭眾,祗擒首要,搜獲盟單,當眾焚之,匪黨感畏,皆散。調山西冀寧道,晉按察使。調湖南,擢布政使。

二十六年,拳亂召禍,京師危急。錫良以湖廣總督、湖北湖南巡撫會委,統率鄂、湘軍隊入衛,迎駕山西,立授巡撫。時和議未定,洋兵闌入晉邊。錫良念兩宮幸陜,和局固應兼顧,而保晉衛秦,亦不容忽。乃通令各軍嚴行防守,別遣委員出境犒師,相機因應,幸保無事。和約定,晉始弛防。

調湖北巡撫,復開缺。旋授河南河道總督。以事簡,奏請裁歸巡撫兼理,詔允行。調補河南巡撫,兼管河工。豫省吏治久隳,劾去道府以次數十人,政紀肅然。泌陽教案事起,立派兵馳捕首犯,被擾難民,無分民教,一律撫恤。調熱河都統。熱河本就蒙地設治,向沿舊習,不講吏事,尤患多盜。錫良首請改制,設立求治局,綜理吏治財政;開辦圍場荒地,以興墾務;整飭巡防,專意緝捕,匪風始戢。又以熱境地廣官少,奏請升朝陽縣為府,並增設阜新、建平、隆化三縣,熱河自此始有吏治。

二十九年,擢閩浙總督,調署四川。時方議借外款修川路,錫良力主自辦,集紳會議,奏設專局,招商股,籌公股,復就通省田租歲抽百分之三,名為租股,數年積至千萬以上,股款之多,為中國自辦鐵路最。三十年,廷議整飭藏事,藏人疑懼,駐藏幫辦鳳全被戕。錫良飛檄提督馬維騏督兵進剿,並令建昌道趙爾豐率師繼進,遂克巴塘,仍飭爾豐進討里塘。裏屬桑披寺築碉謀抗拒,爾豐以長圍困守六閱月,斷其汲道,始克攻破。桑寺既平,諸番心習服。於是自打箭爐以外,直至察木多、巴里、鄉城、德格等處,均改縣治,擴地至數千餘里;且興墾、開礦,設學廣教,番人漸知向化矣。

三十三年,調雲貴總督。滇省軍政久廢,器械尤缺,乃創練陸軍,設講武堂,添購槍砲,舊有防營一律改編,自是滇省始有新軍。滇多煙產,土稅為收入大宗,錫良毅然奏請禁種,各省煙禁之嚴,唯滇為最。滇南連越,越匪竄入河口,戕官擾境,立飭出隊分路截剿,數日而定。滇西土司以數十計,日漸恣橫。宣慰使刁安仁曾游東洋,外人稱以王爵,尤驕妄。聞有改土歸流之議,輒思蠢動。錫良先派員詢察,曉以利害,並令應襲各土司迅辦承襲,以安其心。刁安仁聞而畏感,遣其弟至,痛哭自陳改悔,邊境得以無事。

宣統元年,授欽差大臣,調東三省總督。東省自日俄戰罷,俄占北邊,日踞南境,局勢日危。錫良蒞任,即疏陳:「東三省逼邇京畿,關系大局。遼東租借之約,十三年即滿期,請朝廷主持,上下一心,以天下全力赴之,以贖回遼東半島為歸,否則枝枝節節為之,恐其不能及也。」疏入,不省。錫良又以東三省兩鄰分據,非修大支幹路,不足以貫串脈絡,因擬修錦州至璦琿鐵路。顧須橫貫南滿、東清,必非日、俄所原,尤非密借強國外款,不能取均勢而策進行。適美國財團代表游歷來奉,遂與密訂借款包修草約。三日議定,電奏請旨速正式簽定,即日、俄再爭,已落後著。乃部議梗緩,復機事不密,事竟報罷。及日俄協約,東事益急。錫良以救亡興政,均非款莫辦,再請商借二千萬兩,以千萬設銀行;其餘,半以移民興墾,半以開礦築路。仍不省。錫良慮東省危急情形,朝廷尚未深悉,乃請入覲面陳。

時醇親王監國攝攻,籌備立憲,廷議方注重集權。錫良先疏請實行憲法,歷陳:「立憲精神,在貴賤上下胥受治於法律,先革其自私自利之心。若敷衍掊克,似是而非,財力凋敝,人心渙漓,九年立憲,終恐為波斯之續。」又以近年重臣親貴出洋考察,徒飾觀聽,見輕外人,疏請停派,並慎選親貴實行留學。再疏諫中央集權,以為:「朝廷分寄事權於督撫,猶督撫分寄事權於州縣,無州縣即督撫不能治一省。如必欲以數部臣之心思才力,統治二十二行省,則疆吏咸為贅旒,風氣所趨,軍民解體。設有緩急,中央既耳目不及,外省則呼應不靈,為患實大。」均不報。至是,入都面陳監國,語尤切直,不省如故。告罷,又不允。

其時朝鮮為日並,錫良以事勢益迫,欲固民心,先厚民力,當以防匪為名,設立清鄉局,籌備預備巡警,部以兵法,實即民兵。奉人慮患思痛,爭先應募,期年得數萬人,全省皆兵。未幾,防疫事起,疫起俄境,沿東清鐵路,逐處傳染,未浹旬,蔓延奉、吉、黑三省。俄、日群思干涉,錫良以防疫純屬內政,嚴起防治,三月而疫絕。十一國醫士來奉考察,開萬國鼠疫研究會於省署,錫良主議,咸起頌之。

錫良督東,嚴吏治,肅軍制,清理財政,整頓鹽務,籌辦八旗生計,頗著成績。唯目睹內憂外患日危一日,顧所以為東邊計者,既多未如志,而朝政日非,民心日去,又無以挽救,屢稱病乞罷。三年,始允解任調理。

武昌變作,召入覲,廷議本以錫良赴山、陜督師,並請獨領一軍衛京畿。顧有人惎之,乃改授熱河都統,力疾赴任。遜位詔下,以病勢難支,乞罷,允之。臥病六年,堅拒醫藥,卒,年六十有六,謚文誠。

錫良性清剛,自官牧令,即挺立無所倚。嫉惡嚴,所蒞止,遇不職官吏,劾治不少恤;非義之財,一介不取;於權貴尤一無饋遺,故遇事動相牽制云。

周馥,字玉山,安徽建德人。初侍李鴻章司文牘,累保道員。光緒三年,署永定河道。初,天津頻患水,馥迭治津沽入海金鐘河、北運筐港減河及通州潮白河,設文武汛官資防守。並言天津為九河故道,不洩則水患莫瘳,請就上游闢減河而開屯田,南運下游分水勢。部議格不行。後提督周盛傳開興濟減河,屯田小站,實本馥議。丁艱,服除,署津海關道。朝鮮初通商,馥與美提督薛裴爾議草商約保衛之,首稱朝鮮為中國屬邦,固以防侵奪也,而樞府削之。馥私嘆曰:「分義不著,禍始此矣!」九年,兼署天津兵備道,俄真除津海關道。中法事起,鴻章命赴海口編民舶立團防。鴻章之督畿輔也,先後垂三十年,創立海軍,自東三省、山東諸要塞皆屬焉。用西法制造械器,輪電路礦,萬端並舉,尤加意海陸軍學校。北洋新政,稱盛一時,馥贊畫為多。醇親王校閱海軍,嘉其勞,擢按察使。再署布政使。築永定河北岸石堤衛京師,盧溝南減水石壩工尤鉅,自是河不溢。

中日開釁,馥任前敵營務處,跋涉安東、遼陽、摩天嶺之間,調護諸將,收集散亡,糧以不匱。和議成,乃自免歸。鴻章疏薦之,授四川布政使。至則課吏績,廣銀幣,積糧儲。慮教案易生釁,撰安輯民教示頒郡縣。未幾,拳亂作,八國聯兵內犯,鴻章為議和大臣,總督直隸,馥亦調直隸布政使。先隨鴻章入都,理京畿教案,數月事稍定,始赴保定受布政使印。先是法兵至保定,戕前布政使廷雍,遂踞司署。及聞馥來,列隊郊迎入署。久之,觀其設施,無間言,乃徐引去。鴻章卒,遂護直督。

俄擢山東巡撫,詔留議津榆路事。時和議雖成,外國兵壁天津,踞津榆鐵道,設都統,治民政,屢爭莫能得。至是,馥竟以片言解之。馥撫山東,值河決利津薄莊,議徙民居,不塞薄莊,俾河流直瀉抵海。沿河設電局,備石工,訖十餘年,河不為災。德踞膠州灣,築鐵道達省治,因占路側礦山。馥奏開濟南、周村商埠相箝制,德人意沮,自撤膠濟路兵,還五礦。

馥既膺疆寄,則益欲大有為,凡所以阜民財、瀹民智者,次第興舉,天子嘉之,擢署兩江總督,移督兩廣。三十三年,請告歸。越十四年,卒,謚愨慎。直隸、山東、江南士民皆祠祀之。

陸元鼎,字春江,浙江仁和人。同治十三年進士,以知縣即用,分山西,改江蘇。光緒二年,權知山陽。有奸豪民交通胥役,略人口行鬻,捕輒先遁。元鼎黎明起,盛儀從謁客,中道折至民家,破門入,縛治其豪,取出所略女婦數十人各放歸,驩聲雷動。補江寧,以憂歸。服除,坐補原缺,調上海。法蘭西人擊殺縣人沈兆龍,傷隱不見,法領事不承擊殺。元鼎曰:「時計表墜地,有鋼條內斷而磁面未損者,與此何以異?」領事語塞。如皋焚教堂,檄元鼎往視,教士聲言議不諧,當以兵戎見。元鼎曰:「如皋非軍艦所能至也。」不為動。抗議十餘日,乃定償銀四千,無他求。是時江南北焚教堂十餘所,次第定議,悉視如皋。

移知泰州。城河久淤墊,歲旱,民苦無水。元鼎濬治之,又移徙市廛迫河滸者,雖巨室無所徇。下河斜豐港故有堤,在泰州境者六十里,入東臺境,堤庳,水至勿能御。元鼎增高至十丈,廣如之,而豐其下以倍。工竟,按察使檄東臺治堤與泰州接,元鼎又助工十有一里,自是兩境無水患。尋調上元,援例以道員候選。

兩江總督劉坤一疏薦元鼎才任方面,二十一年,授惠潮嘉道,調江蘇糧道,遷按察使。陛見,溫語移時。論及前歲日本構戰,我軍槍彈多與口徑不合,以故敗。帝因諭樞臣戒督撫審軍實,且曰:「毋謂語由元鼎,使督撫生芥蒂也。」江陰焚教堂,縣吏捕首事者上之按察使。上海領事謂逮捕者非首犯,駐京公使言於總署,令領事往會鞫。元鼎曰:「會鞫有專官,按察使署非會鞫所。」領事言:「不會鞫,當觀讞也。」元鼎持不可,領事曰:「其如總署指揮何?」元鼎曰:「慎守國憲。官可辭,法不可撓!」領事怏怏去。樞臣聞而嘉之,曰:「不爾,又為故事矣。」尋署布政使,護巡撫。

二十九年,遷漕運總督,調湖南巡撫。時方在告,廣西匪起,窺湖南,貴州匪逼靖州。元鼎力疾赴官,籌邊防,與總督張之洞會奏以堵為防,不如以助剿為防。於是募勇,令提督劉光才防西路,令衡永道莊賡良入貴州,而道員黃忠浩佐之。賡良攻下龍貫峒,忠浩亦大敗悍賊於同樂。又令提督張慶雲助擊廣西四十八峒。亂徐定,朝命雲南布政使劉春霖移湖南,率所部滇軍助湘防。元鼎言滇軍不可用,已而後營果叛。醴陵會匪謀叛事洩,自承革命,語連日本留學生。元鼎誅二人,囚一人,他無所株連,人心大定。

徵兵之議起也,元鼎已調撫江蘇。上言:「南人柔脆,其應徵者多市井無藉,不勝兵。當專選江北淮、徐諸府,不當限區域。」部議格不行。其後逃亡相屬,如元鼎言。二十九年,京察開缺另簡。明年,召入京,奏對,語及江、浙爭滬杭鐵道事,元鼎力言士民忠愛無他心,上為動容。命以三品京堂候補,佐辦資政院事。俄,乞歸。宣統二年,卒於家。

張曾易又,字小帆,直隸南皮人。同治七年進士,以編修出知湖南永順府。地屬苗疆,號難治。斥貲募勇戢盜,悉置之法;吏之尤貪污者,彈劾之。徙知廣東肇慶府,有惠愛,督撫交章論薦。光緒二十年,除福建鹽法道。閩鹽踴貴,私運蜂起。為嚴立規約,奏免全釐以恤商,而正課亦饒。遷按察使,歲餘,病免。越三年,再起,召見,奏對稱旨,皇太后獎其明慎,即日授四川按察使,未到官,遷福建布政使。調廣西,桂故瘠區,又分任庚子賠款,益不支。曾易又改釐章,嚴比較,裁冗費,罷不急官吏,用以不絀。

二十九年,拜山西巡撫。日俄釁作,日軍進駐遼南。曾易又建議:「闢要地為商埠,別與日本密訂協守同盟之約,聲明不干內治。所慮者俄為日敗,必將取償於我;伊犁鄰近籓封,亦漸外鄉,故亟宜籌餉練兵,有備無患;而庫張鐵路可緩辦以伐其謀。」言頗扼要。馬賊劉天祐等擾後套,曾易又調集各軍討平之。

三十一年,調撫浙江。時浙西鹽梟煽熾,嘉湖統將吳家玉陰與梟通,都司範榮華尤不法。曾易又便道之官,或勸以兵從,曰:「是速之叛也!」遂輕騎逕嘉郡,召家玉入謁,諭以禍福,家玉不敢動,徐檄他將領其眾,而羈之甬東,僇榮華等,梟漸斂跡。浙路交涉久未決,草約逾定期,英領事猶堅執之。曾易又據約立爭,事乃定。

三十三年,頒下法律大臣沈家本試行訴訟法,曾易又言:「中國禮教功用遠在法律上,是以尊親之義,載於禮經。漢儒說論語,亦謂綱常為在所因,此各省所同,浙不能異者也。浙西梟匪出沒,浙東寇盜潛滋。治亂國用重典,猶懼不勝,驟改從輕,何以為治?此他省或可行,而浙獨難行者也。」於是逐條駁議之。

是年秋瑾案起。秋瑾者,浙江女生言革命者也,留學日本,歸為紹興大通學校教師,陰謀亂。曾易又遣兵至校捕之,得其左驗,論重闢,黨人大譁。調撫江蘇,俄調山西,稱疾歸。家居十四年,卒,年七十九。

楊士驤,字蓮府,安徽泗州人。光緒十二年進士,選庶吉士,授編修。保道員,補直隸通永道,擢按察使,遷江西布政使,復調直隸。三十一年,署山東巡撫。河貫東省千餘里,淤高而堤薄,歲漫決為巨害。士驤以為河所以歲決者,河工員吏利興修,又因以遷擢也。乃定章程:歲安瀾,官奏敘,弁兵支款如例;河決,官嚴參,不得留工效力,弁兵依律論斬。身巡河堤,厲賞罰,自是數年,山東無河患。曹州多盜,行清鄉法,嚴督捕。德兵違約,屯膠、高,久不撤。數月盜少戢,會各國撤京、津兵,士驤與德官議,遂盡撤駐路德兵。

三十三年,代袁世凱為直隸總督。世凱為政,首練軍籌款,尤多興革,務樹威信,北洋大臣遂為中外所屬目。士驤承其後,一切奉行罔有違,財政日竭,難乎為繼,而周旋因應,常若有餘,時頗稱之。明年,入覲。時議修永定河,士驤閱河工,疏言:「全河受病,一由下口高仰,宣洩不暢;一由減壩失修,分消無路。」盧溝橋以下舊有減壩,年久淤閉,宜折修,並挑減河,因請撥帑四十六萬餘兩。詔下部議。

宣統元年,德宗梓宮奉移西陵,詔所需不得攤派民間。士驤慨然思革百年之弊,疏曰:「國初因明季加派紛繁,民生彫敝,屢降旨申禁科累。近畿繁劇,供億多,不能盡革,故田賦較各省輕,而歲出差徭逾於糧銀之數。新政迭興,學堂、巡警諸費,無不取給於民,輸納之艱,日以加甚。擬官紳合查常年應官差徭,實系公用者,酌定數目,折交州縣自辦,不得濫派折錢;胥役書差,官給津貼。庶積弊一清,上下交益。」疏入,優詔答之。五月,卒,贈太子少保,謚文敬。

士驤少孤露,起家幕僚,至於專閫,與人無迕,眾皆稱其通敏云。

馮煦,字夢華,江蘇金壇人。光緒十二年一甲三名進士,授編修。疊上疏代奏,請圖自強,敦大本,行實政,德宗嘉納。典湖南鄉試,稱得士。二十一年,以京察一等授安徽鳳陽知府。鳳屬連年水澇,煦單騎按部,逐一履勘,以被災之重輕,定給賑之多寡,人霑實惠。並屢平反疑獄。總督劉坤一以心存利濟、政切先勞疏保,兩攝鳳潁六泗道。二十七年,遷山西按察使,調四川。廣安州有聚眾謀毀學堂者,獲四人,擬照土匪例正法。煦白大府,請按而後誅,以去就爭,至免冠抵幾,不得請不止。旋署布政使,復調安徽,兼署提學使。

三十三年,擢巡撫。時國是日非,海內外黨人昌言革命。巡撫恩銘被刺,眾情惶惑。煦繼任,處以鎮靜,治其獄,不株連一人,主散脅從,示寬大,人心始安。復疏言:「今者黨禍已亟,民生不聊。中外大臣不思引咎自責,合力圖強,乃粉飾因循,茍安旦夕,貽誤將來,大局阽危,日甚一日。挽救之方,唯以覈名實、明賞罰為第一義,而其要則在『民為邦本』一言。有尊主庇民之臣,用之勿疑;有誤國殃民之臣,刑之毋赦。政府能使天下自治,則天下莫能亂;政府能使天下舉安,則天下莫能危。根本大計,實系於此。」疏入,大臣權幸多忌嫉之。明年,遂罷。

宣統二年,江、皖大水,復起為查賑大臣,出入災區,規定辦法,施及豫東,未一年,凡賑三十九州縣,放款至三百餘萬。後復立義賑會。連年水旱,兼有兵災,遠而推至京、直、魯、豫、湘、浙,無歲不災,無災不賑,蓋自蒞官訖致仕,逮於耄老,與荒政相終始,眾稱善人。聞國變,痛哭失聲。越十有五年,卒,年八十五。

煦居官廉而好施。平素講學,以有恥為的,重躬行實踐。文章爾雅,晚境至鬻文自給云。

論曰:光緒初,督撫權重,及其末年,中央集權,復多設法令以牽制之,吏治不可言矣。錫良強直負重,安內攘外,頗有建樹。馥諳練,士驤通敏,元鼎辦交涉,曾易又論法律,並能持正。煦善治賑,與荒政相終始。「民為邦本」,善哉言乎!錫良初疏諫集權,樞廷轉相箝制。及事變起,大勢所趨,皆一如所言,世尤服其先見雲。


\end{pinyinscope}