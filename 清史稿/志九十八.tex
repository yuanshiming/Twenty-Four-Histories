\article{志九十八}

\begin{pinyinscope}
○食貨四

△鹽法

清之鹽法,大率因明制而損益之。蒙古、新疆多產鹽地,而內地十一區,尤有裨國計。十一區者:曰長蘆,曰奉天,曰山東,曰兩淮,曰浙江,曰福建,曰廣東,曰四川,曰雲南,曰河東,曰陜甘。

長蘆舊有二十場,後裁為八,行銷直隸、河南兩省。奉天舊有二十場,後分為九,及日本據金川灘地,乃存八場,行銷奉天、吉林、黑龍江三省。山東舊有十九場,後裁為八,行銷山東、河南、江蘇、安徽四省。兩淮舊有三十場,後裁為二十三,行銷江蘇、安徽、江西、湖北、湖南、河南六省。浙江三十二場,其地分隸浙江、江蘇,行銷浙江、江蘇、安徽、江西四省。福建十六場,行銷福建、浙江兩省。其在臺灣者,尚有五場,行銷本府,後入於日本。廣東二十七場,行銷廣東、廣西、福建、江西、湖南、雲南、貴州七省。四川鹽井產旺者,凡州縣二十四,行銷西藏及四川、湖南、湖北、貴州、雲南、甘肅六省。雲南鹽井最著者二十六,行銷本省。河東鹽池分東、中、西三場,行銷山西、河南、陜西三省。陜甘鹽池最著者,曰花馬大池,在甘肅靈州,行銷陜西、甘肅兩省。

長蘆、奉天、山東、兩淮、浙江、福建、廣東之鹽出於海,四川、雲南出於井,河東、陜甘出於池。其制法,海鹽有煎、有曬,池鹽皆曬,井鹽皆煎。論質味,則海鹽為佳,池鹽、井鹽次之。海鹽之中,灘曬為佳,板曬次之,煎又次之。論成本,則曬為輕,煎之用蕩草者次之,煤火又次之,木則工本愈重。此其大較也。

初,鹽政屬戶部山東司。宣統二年,乃命戶部尚書兼任督辦鹽政大臣,外遣御史巡視。後裁歸總督、巡撫管理。其專司曰都轉運使司。無運司各省,或以鹽法道、鹽糧道、驛鹽道、茶鹽道兼理。

其行鹽法有七:曰官督商銷,曰官運商銷,曰商運商銷,曰商運民銷,曰民運民銷,曰官督民銷,惟官督商銷行之為廣且久。凡商有二:曰場商,主收鹽;曰運商,主行鹽。其總攬之者曰總商,主散商納課。後多剝削侵蝕之弊,康熙、乾隆間,革之而未能去。惟兩淮以道光時陶澍變法,奏除引目,由戶部寶泉局鑄銅板印刷。順治三年,以淮、浙領引距京遠,設都理引務官駐揚州,至七年裁。十五年,發引於運司,尋命運司仍委員赴部關領,票亦領於部。

商人之購鹽也,必請運司支單,亦曰照單,曰限單,曰皮票,持此購於場。得鹽則貯之官地,奉天謂之倉,長蘆謂之坨。未檢查者曰生鹽,已檢查者為熟鹽,熟鹽乃可發售。兩淮總棧始由商主,後改官棧。四川以行銷黔、滇者為邊岸,本省及湖北為計岸,潼川州為潼岸。河東總岸立於咸豐初。其行陜西者,以三河口為之匯。行河南者,以會興鎮為之匯。山西則蒲、解,於安邑運城立岸,而澤、潞等處亦分立焉。

大抵暢岸外有滯地,或展限,或減引,或停運,或用並引附銷、統銷、融銷諸法。並引附銷者,將積鹽附入,三引銷一引。又納引半之課行一引之鹽,納三引之課行二引之鹽是也。統銷者,將積引統毀,其正雜錢糧令商人分年完繳。融銷者,以暢岸濟滯地是也。

凡引有大引,沿于明,多者二千數百斤。小引者,就明所行引剖一為二,或至十。有正引、改引、餘引、綱引、食引、陸引、水引。浙江於綱引外,又有肩引、住引。其引與票之分,引商有專賣域,謂之引地。當始認時費不貲,故承為世業,謂之引窩。後或售與承運者。買單謂之窩單,價謂之窩價。道光十年,陶澍在兩淮,以其抬價,奏請每引限給一錢二分,旋禁止。票無定域而亦有價。當道光、咸豐間,兩淮每張僅銀五百兩。後官商競買,逮光緒間,至萬金以上。又引因引地廣狹大小而定售額,票則同一行鹽地,售額亦同。嘉慶以前,引多票少,後乃引少票多,蓋法以時變如此。

若夫歲入,道光以前,惟有鹽課。及咸豐軍興,復創鹽釐。鹽課分二類:曰場課,曰引課。場課有灘課、灶課、鍋課、井課之分。長蘆有邊布,福建有坵折。邊布者,明時灶戶按丁徵鹽,商人納粟於邊,給銀報支,是謂邊鹽。其有場遠鹽無商支,令八百斤折交布三丈二尺。後改徵銀三錢,是謂布鹽。灶課向分地、丁為二。但丁不盡有地。雍正間,用長蘆巡鹽御史鄭禪寶言,將丁銀攤入於地徵收,由是各省如所奏行,然長蘆邊布之名猶仍舊。坵折者,鹽田所納錢糧,謂之折價。程所納錢糧,謂之鹽坵。其供應內府及京師、盛京各衙門之鹽,康熙中悉裁,祗供內府、光祿寺二十萬斤,折銀解部充納。引課有正課、包課、雜課。鹽釐分出境稅、入境稅、落地稅。逮乎末造,加價之法興,於是鹽稅所入與田賦國稅相埒。是以順治初行鹽百七十萬引,徵課銀五十六萬兩有奇。其後統一區夏,引日加而課亦日盛。乾隆十八年,計七百一萬四千九百四十一兩有奇。嘉慶五年,六百八萬一千五百一十七兩有奇。道光二十七年,七百五十萬二千五百七十九兩有奇。光緒末,合課釐計共二千四百萬有奇。宣統三年,度支部豫算,鹽課歲入約四千五百萬有奇。蓋稅以時增又如此。

順治二年,諭各運司,鹽自六月一日起,俱照前朝會計錄原額徵收。旋蠲免明末新餉、練餉及雜項加派等銀。十六年,戶部議準各商鹽船用火烙記船頭,不許濫行封捉,其過關祗納船料,如借端苛求,以枉法論。十七年,用兩淮巡鹽御史李贊元言,回空糧艘禁緝夾帶私鹽。康熙九年,兩淮巡鹽御史席特納、徐旭齡言:「兩淮積弊六大苦:一,輸納之苦;一,過橋之苦;一,過所之苦;一,開江之苦;一,關津之苦;一,口岸之苦。總計六者,歲費各數萬斤,應請革除。又掣摯三大弊:一,加鉈之弊;一,坐斤之弊;一,做斤改斤之弊。此三弊者,惟有嚴禁斤重一法,乞交部酌議。」定例,凡橋所掣摯,溢斤割沒,少者三四斤,多者七八斤,不得逾額。如夾帶過多,掣官虛填太重者,商則計引科罪,官則計斤坐贓,庶掣摯公而國法信。上命勒石嚴禁,立於橋所及經過關津口岸。席特納又陳:「自康熙七年,鹽臣差遣稍遲,前任鹽差於徵完本年課銀外,又重徵新鹽。鹽尚未賣一引,而課已徵至二十餘萬。此種金錢,追呼無措,非重利借債,即典鬻赴比,應請停止。」如所請行。十六年,用戶科給事中餘國柱言,命將商鹽掣驗每引加二十五斤,加課二錢五分,永遠革除,著為例。二十年,命革除三籓橫徵鹽課。

自滇、黔告變,所在揭竿蜂起,鹽無行銷地,商皆裹足不前,至亦榛墟彌望,無所得售。計臣以軍需所恃,督餉之檄,急如星火,商於是大困。時天下鹽課兩淮最多,困亦最甚,賴巡鹽御史劉錫、魏雙鳳多方撫恤,輸納忘疲。至是海內殷富,淮南寧國、太平、池州等府,及兩浙、山東、廣東、福建,先後增引,利獲三倍。不特額外照舊行銷,且原先呈課銀,請將以前停引補還。四川經明季之亂,江、楚人民遷移其地,食鹽日多,請引數倍於昔;所開之井,為滇、黔資,水陸無滯。而福建、廣東、兩浙招徠灶丁,墾復鹽地、鹽坵,報部升課者不絕。又兩浙各場漲墾蕩地二萬二千七百餘畝,廣東各埠每斤加七十斤,江西南、贛二府鹽引,至三十六年,加斤配課亦如之。上以寰宇升平,免浙江加斤銀之半,共三萬一千三百八十餘萬。三十八年南巡,復諭各鹽差:「向因軍需,於正額外更納所私得贏餘,著將此項停罷。其兩淮鹽課,前曾加四十萬,著減其半。」四十三年,用江南總督阿山言,革除兩淮浮費數十萬,勒石永禁。五十六年,長廬巡鹽御史田文鏡請將山東所裁鹽引補足辦課,經部議準。上以加引增課無益,不許。

先是順治二年,世祖定巡視長蘆、兩淮、兩浙、河東鹽政,差監察御史各一,歲一更代。其山東鹽務歸長蘆兼管,陜西歸河東兼管。十年停,鹽務專責成運司。尋因運司權輕,仍命御史巡察。康熙十一年,復停巡鹽。明年,巡撫金世德以直隸事繁,請仍差御史。於是兩淮、兩浙、河東皆復舊制。既而兩廣、福建並設巡鹽御史。五十九年,仍交督撫管理。

時鹽課惟廣東、雲南常缺額,因康熙初粵商由里下報充,三年一換,名為排商,故弊端百出。嗣將排商費萬餘兩入正課,舉報殷戶以充場埠各長商,而場商貲薄,不能盡數收買,致場多賣私。五十七年裁場商,由運庫籌帑本三十六萬,分交場員收買。且置艚船給水腳,運向東關潮橋,存倉候配。埠商配鹽,按包納價,獲有盈餘,名為場羨。其滷耗餘賸鹽斤,乃配引外多收餘鹽,發商行運。又有子鹽、京羨、餘鹽、羨銀等名。後餘鹽改引,將餘羨歸入正額,而粵鹽遂有辦羨之事。後粵商倒歇至五十餘埠,滇鹽由商認票辦運,而地無舟車,全恃人力,煎無煤草,全恃木柴,故運費工本皆重,而鹽課率以一分,又重於他省。富商棄之弗顧,強簽鄉人承充。及倒罷末由追繳,乃責里中按戶攤納。迨乾隆時,一蹶不振,遂令歷年督撫分償。

世宗初年,裁福建、浙江巡鹽御史。時上於鹽政頗加意。河東鹽池形低,屢為山水灌入,向例修墻築堰,皆派蒲、解十三州縣之民應役。從巡鹽御史碩色言,歲撥銀六千兩,以三千作歲修,三千貯運庫備大修,民累始紓。又以鹽法莫急於緝私,但有場私、有商私、有梟私,而鄰私、官私為害尤鉅。欲緝場私,必恤灶而嚴其禁。故於雍正二年兩淮範堤決,沿海二十九場為潮淹,特發帑金以賑。五年,以淮商捐銀建鹽義倉積穀,諭更立數倉於近灶地,以備灶戶緩急之需。此政之在於恤灶者。

六年,江南總督範時繹言:「兩淮灶戶燒鹽,應令商人舉幹練者數人,並設灶長巡役,查核鹽數,輸入商垣,以杜私賣。」兩淮巡鹽御史戴音保言:「場灶燒鹽之具,深者盤,淺者金敝,設有定數,而煎鹽以一晝夜為火伏,並巡查息火後私燒。近有灶戶私置鹽金敝,火伏又不稽查,故多溢出之數。請飭鹽官申嚴舊法。至淮南曬掃,惟有商人收買配運,酌加引課。」均命著為例。此所以嚴其禁也。

欲緝商私,必恤商而嚴其禁。故二年兩淮各場,因災灶鹽不繼,商本倍增,從巡鹽御史噶爾泰言,令將本年成本之輕重,合遠近腳價,酌量時值買賣。至食鹽難銷處,值有綱地行銷不敷,亦準改撥。兵部尚書盧詢請加引免課,以期減價敵私,命長蘆、兩淮每引加五十斤,免納課銀。此政之在於恤商者。十一年,從江南總督尹繼善言,改設淮南巡道,督理揚州、通州等處鹽務,並於儀徵之青山頭立專營緝私。

其稽官私也,自明以來,膺鹽差者,回京例有呈獻,及上嚴禁,始各將所得報繳。獨福建八萬餘兩為總督滿保查出,於是裁撤鹽官,鹽商命各場由州縣監管。嗣廣東總督楊琳言:「地方官辦課,必委之家丁衙役,非設鋪分賣中飽,即發地裡勒派。且恐貲本不足,挪動地丁錢糧。應將場商停設,發帑委官監收,埠商仍留運銷納課。」從之。

是時上於鹽官量重李衛。衛在浙江可稱者,莫如辦帑鹽。帑鹽者,由松江、臺州、溫州三府場鹽產旺,灶多漏私,衛請發帑銀八萬,交場員收買。復奏設玉環同知,使經理收鹽事,而舟山內港內洋、岱山附近之秀山長塗、平陽縣界之肥艚,均委官管理收發。崇明場鹽,令知縣主之。所收帑鹽,侭銷本處魚戶、蜇戶,漁鹽亦準引商、帑商運往他處銷售,各照科則納課外,輸經費銀一二三錢不等,除歸帑本經費,餘銀作為盈餘。由是私凈官暢,每年引不敷運,加領餘引十五萬。凡商運餘引,引輸租銀四分,所完課銀,與帑鹽盈餘,並案題報,年約銀十萬餘。

自上清釐鹽政,積弊如洗。然自裁革陋規,歸入正項,上又有「耗羨入正額,恐正額外復有耗羨,商何以堪」之諭,蓋已知其弊矣。十三年,署副都御史陳世倌言:「鹽課引有定額,斤有定數。按引辦課,未必果有奇贏,即獲微利,何妨留與商人,裕其貲本。乃近年多有以隨利歸公者,考其實乃陰勒商重出。故在官多一分之歸公,在商添一分之誅求,此商受其弊者也。又有以捐助題請者為急公,亦陰勒商總公派。及項無所出,非拖欠引綱,即暗增引斤,或高抬鹽價,此國與民並受其弊者也。請嗣後祗按引辦課,一切歸公捐助等名,應永遠停止。」上命莊親王議。尋覆如所請行。

時江西驛鹽道沈起元與江南總督趙宏恩書,亦言「昔年陋規,非皆收納,今以墨吏私贓作報部正款,在大員自無再收之理,而僚佐豈能別無交際?其為商累實甚」。後有聞於高宗者,乃將兩淮鹽政公費、運使薪水,及雲南黑、白、瑯井規體銀蠲除。

初,世宗從宏恩言,命給貧民循環號籌,聽於四十斤內負販度日。至乾隆初元,戶部題準六十歲以上、十五歲以下及少壯有殘疾、婦女老而無依者,許於本縣報明,給印烙腰牌木籌,日赴場買鹽一次。既兩淮巡鹽御史尹會一、兩廣總督鄂彌達先後奏言:「奸民藉口貧苦,結黨販私,兩查兵役,未便概撤。」後以貧民過多,停牌鹽,每名日給錢十文至二十四文。

尋改浙江巡撫為總督,兼管鹽政,諭酌定增斤改引法,將杭、嘉、紹三所引鹽,照兩淮舊額,每引加五十斤,松所照溫、臺例,改票引九萬餘道,引給四百斤,均不加課,以期復舊。又諭裁雲南贏餘,其價減至三兩以下,廣西仍減二釐,免徵兩廣鹽課每千斤餘平銀二十五兩。三年,改浙督仍為巡撫,兼管鹽政。六年,以淮南灶鹽暑月多耗,命五六月每引加耗十五斤,七八月遞減五斤。至十三年,淮北亦仿行。又命兩淮於定額外,每引加給十斤。

十六年,以省方所至,諭兩淮綱鹽食鹽於定額外每引加十斤。先是雍正初,因長蘆積欠甚多,每引加五十斤。嗣經部覆按所加斤折中核算,年應增課銀八萬六千餘兩。高宗念商力艱難,命減半納課。二十八年,裁運商支應。以雲南巡撫劉藻言,加給黑、白兩井薪本銀。四十二年,以河東鹽斤陸運虧折,命每斤加耗五斤。時價平銷速,兩淮請豫提下綱之引,歲入至五六百萬。惟乘輿屢次游巡,天津為首駐蹕地,蘆商供億浩繁,兩淮無論矣。

或遇軍需,各商報效之例,肇於雍正年,蘆商捐銀十萬兩。嗣乾隆中金川兩次用兵,西域蕩平,伊犁屯田,平定臺匪,後藏用兵,及嘉慶初川、楚之亂,淮、浙、蘆、東各商所捐,自數十萬、百萬以至八百萬,通計不下三千萬。其因他事捐輸,迄於光緒、宣統間,不可勝舉。鹽商時邀眷顧,或召對,或賜宴,賞賚渥厚,擬於大僚;而奢侈之習,亦由此而深。或有緩急,內府亦嘗貸出數百萬以資周轉。帑本外更取息銀,謂之帑利,年或百數十萬、數十萬、十數萬不等。商力因之疲乏,兩淮、河東尤甚。

五十一年,以兩淮歷四年未豫提,命江督查奏。尋請嗣後每間一綱豫提一次。上諭以正引暢銷為主,無庸拘定年限。厥後惟五十七年及嘉慶五年各行一次。且自三十三年因商人未繳提引餘息銀數逾十萬,命江蘇巡撫彰寶查辦,鹽政高恆、普福,運使盧見曾皆置重典,其款勒商追賠。至四十七、四十九兩年,乃先後豁免三百六十三萬二千七百兩有奇。後遇大經費,商人但藉輸將之數,分限完納,一二限後,率皆拖欠。

五十六年,江西巡撫姚棻奏:「建昌府界連閩省,路徑較多,必添設緝私卡巡,始收實效。」上曰:「行鹽分界,必使民食不至舍近求遠、去賤就貴乃善。建昌既距福建為近,其價必輕,何以不就近行銷?若酌改鹽徵、鹽課移彼地輸納,非惟便民,即私販亦將不禁自止。」旋兩江總督覺羅長麟、湖廣總督畢沅等奏稱:「小民惟利是圖,往往得寸思尺。如建昌劃歸閩省,則私販即可越至撫州,於全局所關不細。」乃命仍舊。既長麟奏請建昌設總店,屬縣設子店,分銷課引,依閩省時價斤減二文以敵私,更於各要隘分巡嚴緝。得旨速行。

河東自十年眾神保就現行賤價,定為長額,而商始困。後池鹽收歉,借配蘆、蒙、花馬池各鹽,又開運城西六十里之小池。時民食缺少,商倒無人承充,乃令退商舉報短商,五年更換,富戶因受累多規避。四十七年,巡撫農起奏準,仍定為長商,引地分三等配勻,復請加價二釐,試行三年再覈定。嗣經部議駁,得旨允行。久之,力仍竭蹶。五十六年,命馮光熊巡撫山西,調甘肅布政使蔣兆奎為山西布政使。初,兆奎以河東運使入覲,帝問辦潞鹽之策,以課歸地丁對。及光熊入京,命與軍機大臣議之。未定,而山西署巡撫布政使鄭源鸘疏至,力言不便。上曰:「課歸地丁,朕早慮及地方官曾受鹽規,必持異議。今鄭源鸘果然。伊調任河南,河南亦有行銷河東引地。倘從中阻撓,從重治罪。」八月,光熊言:「河東鹽務積疲,惟有課歸地丁,聽民自運。既無官課雜費,又無兵役盤詰及關津阻留,未有不前者。請自乾隆五十七年始,凡山西、陜西、河南課額,在於三省引地百七十二屬地丁項下攤徵。」於是山西攤二十八萬一千一百二兩、陜西攤十四萬六千三十七兩、河南攤八萬六千六百三十三兩各有奇,並議章程十:一,課銀各解本省籓庫,雖遇蠲免地丁之年,不得蠲免;一,部引停領,免納紙硃銀;一,無許地方官私收稅錢;一,鹽政運使以下各官俱裁汰;一,移河東道駐運城,總管三場;一,鹽池照舊歲修;一,三場仍立官秤牙行;一,課項內有並餘積餘等銀,應分別攤免;一,運阜運儲二倉穀石,應分別歸並存借;一,鹽政應支各款,各就近省籓庫動支。從之。五十七年,上幸五臺,光熊、兆奎奏言,自弛鹽禁,民無攤課之苦,有食賤之利。而陜西巡撫秦宗恩、河南巡撫穆和藺亦以鹽充價減聞。上甚悅。甘肅鹽課,雍正元年嘗攤入地丁,九年復招商,至是仍行前法。而陜西漢中、延安二府及鄜州各屬之食花馬池鹽者,亦一並攤入地丁焉。

嘉慶四年,命停各省鹽政中秋節貢物。五年,以雲南課額常虧,從巡撫初彭齡言,改為灶煎灶賣,民運民銷。其法無論商民,皆許領票。運鹽不拘何井,銷鹽不拘何地,完課後聽其所之。就諸井現煎實數,將定額勻算攤徵,有餘作為溢課,侭徵侭解。所有放票收課事宜,即歸井員經理。至八年,著為定章。十年,諭兩淮鹽每引加十斤,不入成本,以補虧折。先是蒙古阿拉善王有吉蘭泰鹽池,向聽民販於托克托城辦鹽,分銷山西食土鹽各地,不準運赴下游。其後稽察漸懈,竟順流而下,不獨池鹽為所占,且侵及長蘆、兩淮。十四年,陜甘總督那彥成奏辦奸民出販,請飭阿拉善王將所留漢、回奸民獻出。王懼,獻鹽池,命將其歲入銀八千兩如數賞給。尋戶部侍郎英和同山西、陜甘督撫會奏:「潞商賠累,緣以賤價定為常額。請照乾隆十年以前例,按本科價。其吉蘭泰池,潞商力難兼顧,請另招他商。」十五年,以新商虧課,改官運。工部侍郎阮元言:「官運不難,難於官銷。若虧課額,勢必委之州縣,非虧挪倉庫,即勒派閭閻,是能銷之弊更甚於不銷。」於是部議吉蘭泰引,請飭還阿拉善王,賞項停給。原定額引,改為潞鹽,餘引名吉蘭泰活引。

兩廣自康熙時發帑收鹽,運銷後乃收課。乾隆五十三年,總督孫士毅以商欠積至六十九萬八千餘兩,請停發帑本,令各出己貲,在省河設局經理。五十四年,新任總督福康安會同士毅籌定章程,並兩粵百五十埠為一局,舉十人為局商,外分子櫃六,責成局商按定額參以銷地難易,運配各櫃,所有原設埠地,悉募運商,聽各就近赴局及各櫃領銷,交課後發鹽二十九埠如舊。所謂改埠歸綱也。行之二十餘年,局商以無應銷之埠,歧視埠商。其始準局商捆運餘鹽,彌補帑息。嗣乃不問正引完否,貪銷餘鹽,反礙正引。疲埠欠餉,輒用鹽本墊解,久之虧益鉅,雖局商認完後,埠商仍按引捐輸,而此十人者已物故,家產蕩然矣。嘉慶十一年,總督蔣攸銛以聞,乃裁局商,改公局為公所。擇埠商六人經理六匱事,各有埠地,自顧己貲,不至濫用。且定三年更換,以免把持,謂之改綱歸所。二十五年,命停兩淮玉貢折價銀。

道光元年,兩江總督孫玉庭言,淮鹽至楚岸,本無封輪之例,鹽政全德始行之,請散賣為便。湖廣總督陳若霖奏稱積鹽尚多,若全開售,恐疏銷不及,鹽行水販壓價賒欠。諭俟積鹽售畢,再隨到隨賣。二年,兩淮巡鹽御史曾燠奏稱輪規散後,爭先跌價搶售,有虧商本。玉庭奏無其事。若霖言本年較前實溢銷二十六萬餘引。於是定議開輪。既,湖廣總督李鴻寶又言搶售難免,八年復封輪。

時兩淮私梟日眾,鹽務亦日壞。其在兩淮,歲應行綱鹽百六十餘萬引。及十年,淮南僅銷五十萬引,虧歷年課銀五千七百萬。淮北銷二萬引,虧銀六百萬。上召攸銛還京,以江蘇巡撫陶澍代之。尋遣戶部尚書王鼎、侍郎寶興往查。澍奏言:「其弊一由成本積漸成多,一由藉官行私過甚。惟有大減浮費,節止流攤,聽商散售,庶銷暢價平,私鹽自靖。」命裁巡鹽御史,歸總督管理。自九年後,御史王贈芳、侍講學士顧(𥫗純)、光祿卿梁中靖皆請就場定稅,太僕少卿卓秉恬又請仿王守仁贛關立廠抽稅法。下澍議。澍商於運使俞德淵,以為難行。遂覆稱:「課歸場灶有三難。一由灶丁起課。淮南煎鹽以金敝,淮北曬鹽以池,約徵銀百餘兩。灶皆貧民,若先課後鹽,則力未逮;先鹽後課,設遇產歉,必課宕丁逃。此灶丁起課之難行也。一由垣商納課。寓散於整,較為扼要。惟灶以己業而聽命商人,情必不原。況商惟利是視,秤收則勒以重斤,借貸則要以重息。灶不樂以鹽歸垣,商亦必無資完課。此垣商納課之亦難行也。一由場官收買。就各場產鹽引額攤定課額照納,似亦核實。無如淮課為數甚鉅,豈微員所能任?若聽其侭收侭解,難保不匿報侵欺。此場官收稅之亦難行也。」又言:「鹽在場灶,每斤僅值錢一二文,若就而收稅,則價隨課長,爭其利者必多。海濱民灶雜處,掃煎至易,將比戶皆私,課且更絀。至設場抽稅,或可試行一隅。若各省豈皆有隘可守?漏私必比場灶為甚。總之無官無私,必須無課無稅。業經有課有稅,即屬有官有私。如謂歸場灶或設鹽廠,即可化梟為良,恐未能也。」上韙之。

明年,澍周歷各場,擬行票鹽法於淮北,奏定章程十條。一,由運司刷印三聯票,一留為票根,一存分司,一給民販行運。立限到岸,不準票鹽相離及侵越到岸。二,每鹽四百斤為一引,合銀六錢四分,加以諸雜費,為一兩八錢八分。三,各州縣民販,由州縣給照赴場買鹽。其附近海州者,即在海州請領。四,於各場適中地立局廠,以便灶戶交鹽,民販納稅。五,民販買鹽出場,由卡員查驗,然後分赴指銷口岸。六,委員駐扎青口。七,嚴飭文武查拏匪棍。八,防河。九,定運商認銷法,以保暢岸。十,裁陋規。時窟穴鹽利之官胥吏舉囂然議其不便,澍不為動,委員領運倡導。既而人知其利,遠近輻輳,鹽船銜尾抵岸,為數十年中所未有。未及四月,請運之鹽,已逾三十萬引。是歲海州大災,饑民賴此轉移傭值,全活無算。是法成本既輕,鹽質純凈,而售價又賤,私販無利,皆改領票鹽。但所試行者,僅在湖運滯岸,皖之鳳陽、懷遠、鳳臺、靈壁、阜陽、潁上、亳州、太和、蒙城、英山、泗洲、盱眙、五河,豫之汝陽、正陽、上蔡、新蔡、西平、遂平、息縣、確山,與食岸在江蘇境之山陽、清河、桃源、邳州、睢寧、宿遷、贛榆、沭陽、安東、海州三十一州縣,而皖之壽州、定遠、霍山、霍丘、六安,豫之信陽、羅山、光州、光山、固始、商城十一州縣,皆昔所定為暢岸,尚仍舊法也。十三年,乃一律改票,惟前議科則較原額為減,復依原額引徵一兩五分一釐,益以各費,定銀二兩五分一釐,永不議加。於是所未改者,惟例由江運之桐城、舒城、無為、合肥、廬江、巢縣、滁州、來安,及由高郵湖運之天長九州縣,以地與淮南相錯,未宜招販,啟浸灌之端故也。

其立法在改道不改捆。蓋淮北舊額未嘗不輕,而由暢運至口岸,每引成本已達十餘兩,價不償本,故官不敵私。今票鹽不由槓壩淮所舊道,而改從王營減壩渡河入湖,且每包百斤,出場更不改捆,直抵口岸,除鹽價錢糧外,止加運費一兩,河湖船價一兩,每引五兩有奇,減於綱鹽大半。其江運數萬引亦仿此。自改章後,非特完課有贏無絀,兼疏場河、捐義廠、修考院,百廢俱興,蓋惟以輕課敵私,以暢銷溢額,故以一綱行兩綱之鹽,即以一綱收兩綱之課。時頗欲推行於淮南,不果。

及二十九年,湖北武昌塘角大火,燒鹽船四百餘號,損錢糧銀本五百餘萬,群商請退。於是總督陸建瀛從護理運使童濂言,請淮南改票法,較淮北為詳。如運司書吏積弊,則改為領引納課。設揚州總局辦理。漢口匣費雖裁,而應酬仍多,則改為票鹽運至九江,驗票發販,鹽船經過橋關,有掣驗規費,則改為壩掣後不過所掣,在龍江一關驗票截角,餘皆停免。鹽包出場至江口,其駁運船價及槓鹽各人工勒索,則改為商自雇覓。凡省陋規歲數百萬,又減去滯引三十萬,年祗行百零九萬引,每引正課一兩七錢五分,雜課一兩九錢二分,經費六錢五分八釐,食岸正課同,雜費減半。其要尤在以帶連之乙鹽為新引之加斤。乙鹽者,乙巳綱鹽船遭火,而商已納課,例得補運,故定為每運新鹽一引,帶乙鹽二百斤,每引六百斤,出場至儀徵,改為六十斤子包,一引十包。既裁浮費,又多運鹽二百斤,成本輕減過半。故開辦數月,即全運一綱之引,楚西各岸鹽價驟賤,農民歡聲雷動。是年兩淮實收銀五百萬兩,雖兩綱後復引滯課虧,則以起票自十引至千引不等,大販為小販跌價搶運所誤。始澍行於淮北,亦自十引起。然淮北地隘,淮南則廣,故利弊殊。又值粵亂起,鹺務全廢,非無補救之方也。

其在長蘆,乾隆以來,正雜課共徵七十餘萬。自嘉慶十四年南河大工,每斤加價二文,謂之河工加價。五年,又因高堰大工加價,三年後,半歸商,半歸公。八年,復將充公一文歸商,然歷年欠項已積至千數百萬矣。時銀價翔貴,商虧彌鉅,於是又加價以調劑之,或一文或二文。旋議行減引並包法,蓋蘆鹽三百斤成引,連加耗包索重三百四十斤,搬運築包等費,歷年加增,亦足病商。今以十引改築九包,減引一成。二十一年,再減引二成,照前改築。二十四年,又奏停額引十五萬,減去課銀六萬餘兩,而困仍莫蘇。蓋本因浮費重而欠課,因欠課多而增價,官鹽價貴,私鹽乘之,薊、遵六屬,梟販與官為敵,而永平七屬尤甚,不得已改為官辦。二十八年,商倒引懸,河南二十州縣、直隸二十四州縣,未運積引至百餘萬,未完積欠至二千餘萬。命定郡王載銓、倉場總督季芝昌,會同直隸總督訥爾經額查究。每引因費重需成本五兩有奇,乃就正課、帑利、雜款、積欠,釐為四類,其鹽價每斤減制錢二文以敵私,斤重則每引加百五十斤以恤商,州縣陋規則嚴行裁汰。引地懸岸,則直隸招商,河南改票,皆先課後鹽。至停引原限五年再酌展,約每引攤算僅二兩有奇。

其在山東,乾隆以來,引票正課徵銀十八萬九千八百八十餘兩,雜款共十萬一千八百餘兩。自嘉慶初帑息遞增至二十一萬餘兩,較正課增倍。十四年,南河大工加價二文,每年應欠二十九萬兩,較正雜課又增一倍。十七年,復議加價一文,以半歸商,半彌補商欠。而當年課項不能完,乃歸次年帶徵。帶徵又未完,乃按年分限,或十二限,或二十限,遞年推展。至道光元年,將河工加價停徵,而積欠已五百三十餘萬,然尚完課額。五年,因高堰大工,又議加價二文,奏明三年後半歸商、半歸公,然所完僅及半,正課反因之拖欠。至七年,全綱傾敗,於是設法調劑,以積欠款為一案,俟堰工加價歸商後,彌補帑本,酌留百二十九萬生息,餘銀二十七萬。至十二年起限,分二十限拔繳,南運每引加二十五斤,北運加二十斤,其歸補舊欠之半文加價,並歸商以輕成本,免徵南運十三州縣與票地臨朐等六縣堰工加價以敵私。而舊欠暨現年應交帑息猶不能完,於是將報撥之一文堰工加價悉數歸商,並將一分帑息減三釐,此道光十五年也。

時銀價日昂,虧折彌甚,迨臨朐等九州縣票商倒乏,因改官運。十七年,命鹽務歸巡撫管理,尋又議加二文。二十三年,停引票二成,以八成作總額,並停餘引。二十七年,又議引地加價二文,票地加一文。逾年,各岸竟倒懸二十餘處。時新舊積欠計八百餘萬,而十五年後所欠正雜課又九十餘萬,十九年後積欠八十餘萬,二十七八年皆未奏銷。於是定郡王等會同山東巡撫徐澤醇奏準將兩年奏銷免其造報,積引停運,積欠停徵。自二十九年始,改為先課後鹽,除有商運州縣外,皆改官運,無論官商,每引加七十斤,帑息每引減一錢,十八年二文加價亦減一文,以便民食。

其在浙江,自道光元年裁巡鹽御史,以巡撫帥承瀛兼管鹽政。承瀛疏言:「嘉慶十五年前,撫臣蔣攸銛清查浙江運庫墊缺銀數僅五十五萬餘兩,甫十載乃至百七十三萬三百兩。緣邇來引壅,舊綱未畢,新綱即開,套搭行銷,不能以一綱之課歸一綱之用。而每年奏銷有定限,但完正課,即報全完,其帶輸之款及外用銀,並未徵足,歷次河餉又須撥解,是以不得不於徵存銀內挪墊。而商捐用款,每遇交辦公事,奸商復借名浮支。臣今飭運司遇支解銀兩,如本款無銀即停給。或不得已,亦止以外款墊發內款,不準以內款墊給外款。」嗣後至六年,銷數皆及額運,庫存銀百二十八萬。自七年至十年復短銷,僅存十一萬。蓋因巡撫程含章請加增餘價,鹽貴引壅所致。迨十一年停止,銷數遂至九成。二十九年,命芝昌往查,時又短銷,僅至五六成。乃請將停歇各地招商承辦,並酌加鹽斤。

其在廣東,所辦羨銀頗多。蓋粵鹽至西省,每包申出鹽十餘斤,嗣又添買餘鹽萬包,發埠運銷,按九折較羨,是為秤頭鹽羨,約二萬七千餘兩。慶遠等五府苗疆食鹽無引額,皆捆運餘鹽,交近埠帶銷,為土司鹽羨,約五千餘兩。海船運鹽,灶戶補船戶耗,官為收買,發商運銷,是為花紅鹽羨,約四千餘兩。粵省鼓鑄,歲資滇銅十餘萬斤,滇省廣南府屬歲資粵鹽九萬餘包,每年兩省委員辦運,至百色交換,謂之銅鹽互易。又廣州駐防食鹽、育英堂鹽,各數十包,皆取之餘鹽,按包計羨,藉此充外支經費,故無雜課。正餉有部飯、平頭、紙硃等銀,又東省鹽船所過抽稅約四千餘,西省約四萬餘,其帑息則八萬餘。各項歷年拖欠,初省河因損款多,致奏銷遲緩。道光二十四年後,潮橋疲滯,甚於省河。然軍興糜爛,廣西淮鹽全棄於地,而粵課猶十得八九焉。

其在四川,始以潼川府之射洪、蓬溪產鹽為旺,嘉定府之犍為、樂山、榮縣,敘川府富順次之。不數年,射洪、蓬溪廠反不如犍、樂、富、榮。方乾隆四十九年,各處鹽井衰歇。有林俊者,官鹽茶道,聽民穿井不加課,蜀鹽始盛。惟潼川難如初。且產鹽花多巴少,又煎鹽用草工費,致欠課七萬,始議與犍商合行,以十二年為限,期滿歸清積欠,因請續合十二年,及期滿自辦。甫一載即欠二萬餘,於是復請續合。至道光八年,三次期滿,而其廠產鹽愈少,每年僅完正課,不完羨截。羨即羨餘。截者,於繳課截角時交納也。時漢州、茂州、巴州、劍州、蓬州、什邡、射洪、鹽亭、平武、江油、彰明、石泉、營山、儀隴、新寧、閬中、通江、安岳、羅江、安縣、綿竹、德陽、梓潼、南江、西充、井研、銅梁、大足、定遠、榮昌、隆昌三十一州縣,因滷衰銷滯,商倒岸懸,民在近廠買鹽以食,正雜課銀歸入地丁攤徵。蓋鹽商奢侈,家產日衰,乃覓殷戶出租於引商,名曰「號商」。所完課羨,須交引商封納,引商往往挪用,且官復有與為弊者。至三十年,全綱頹廢。會徐澤醇為總督,查積欠羨截銀共二十三萬七千餘兩,未繳殘引二十二萬八千五百八十一張。於是酌撥代銷,將號商姓名入冊,責其自行封。時惟犍、富邊商及成都、華陽計商稍殷實,銷岸亦暢,餘皆疲滯,而潼商尤甚。乃撤出黔邊所行水引,交犍、富兩商承辦。

其在雲南,自改章後,私鹽尤多,而諸井或常缺額,又在迤西、迤南。其東北隅食川鹽,東南隅食粵鹽,至難如期。道光六年,總督趙慎畛疏請就井稽鹽多寡,定地行銷。御史廖敦行又言分地行鹽,不若廣賝子井。上命新任總督阮元試行。其後諸大井淹廢,猶賴子井挹注,乃復振云。

長蘆於咸豐八年,經蒙古親王僧格林沁防津,奏準將道光二十八年減價二文起徵,名鹽斤復價,得銀十八萬餘。時粵匪北犯,運道多阻,鹽集濬縣之道口鎮,自道口南皆以販運。運商省岸費,有餘利,而坐地引商,借官行私,所獲尤厚。故同治五年,河南巡撫因河防,又議行銷河南引鹽,每斤再加二文,得八萬兩撤防。以七年滎陽大工耗帑百數十萬,改為滎工加價。於是較道光末增款二十六萬。山東因捻匪,不能南運。同治三年,積引百三十餘萬,分八年帶銷,雖部議提撥道光十八年一文加價解充京餉,每年約加銀七萬,而正課未能全完。

河東自嘉慶十四年南河大工,每斤加價一文,較乾隆課額已增至十六萬餘。十七年加入吉蘭泰活引,又六萬餘兩。河東鹽向侵淮岸,至道光十一年,淮北改票,反灌河東,而商力益困。乃將活引減半,河工加價減二成,既由招商變為舉報,又變為簽商,破產者眾。咸豐二年,命戶部侍郎王慶雲往查。尋奏定留商行票,分立總岸,商運鹽至,發販行銷,裁革州縣陋規銀二十七萬餘兩,運城商所攤公費七萬餘兩,並知池價踴貴,由坐商銷乏,將畦地出租,坐食銷價,夥租者按年輪曬,先曬者盜挖鹽根,囤私肥己,故每名價至百二三十兩。於是嚴禁,定白鹽不得過六十兩,青鹽不得過四十兩,澤、潞節省等銀攤入通省引內,每引九分,另籌經費辦公,每引七分,並酌加鹽斤,計成本引僅一兩六錢,商情悅服,原將活引之半及加價二成完納。未幾,殷商九十餘家,以急軍需,共捐銀三百萬,給永免充商執照,改為民運民銷。山西、陜西、河南為官運官銷,刪除河工活引節費名目,定每斤徵課銀三釐五毫,每名合銀百五兩,較前增七萬餘,此咸豐四年也。時長江梗阻,河東以侵淮綱大暢,先後加河南靈寶口岸引三百名。

山西岢嵐等食土鹽十三州縣,引二千四百九十四道,惟陜甘鹽池舊轄於河東。康熙二十八年,改令花馬小池歸甘肅疆臣管理,而大池如故。自咸豐五年,陜西巡撫王慶雲議改課歸地丁。慶雲旋調山西。吳振棫之奏言:「陜民貧乏,若徵鹽課,力實不遑,小民納無鹽之課,駔儈賣無課之鹽,事殊欠允。請飭豫省改招為便。」諭與慶雲會商。尋改為官民並運。時庫款支絀,部議令河東抽釐濟餉。巡撫以難行,第於額引加引,每名各取羨餘,約加銀五萬。直隸總督因海防亦請加斤加價,庚申綱遂加引六百名,辛酉綱加五百名,共加銀四十八萬,然惟辛酉綱全完。旋值陜回亂,捻匪竄河南、陜西,銷路驟塞,乃酌停加引。

兩淮於咸豐三年,以江路不通,南鹽無商收賣,私販肆行,部議令就場徵稅。四年,復令撥鹽引運赴琦善、向榮大營抵餉。怡良旋奏易引為斤,每百斤抽稅錢三百,以二百四十文報撥,以六十文作外銷經費。時湖廣總督、江西巡撫皆以淮引不至,請借運川、粵鹽分售於太湖南北,江西則食閩、浙、粵之鹽。部議由官借運,不若化私為官,奏準川、粵鹽入楚,商民均許販鬻,惟擇堵私隘口抽稅,一稅後給照放行。

北鹽自軍營提鹽抵餉,遂為武人壟斷。提督李世忠部下赴壩領鹽,棧鹽不足,輒下場自捆,夾私之弊,不可究詰。同治三年,御史劉毓槐疏請整頓。事下江督曾國籓。國籓疏論:「淮南鹽務,運道難通,籌辦有二難。一在鄰鹽侵灌太久。西岸食浙私、粵私而兼閩私,楚岸食川私而兼潞私,引地被占十年,民藉以濟食,官亦藉以抽釐,勢不能驟絕。一在釐卡設立太多。淮鹽出江,自儀徵以達楚西,層層設卡報稅,諸軍仰食,性命相依,不能概撤。臣思辦法不外疏銷、輕本、保價、杜私四者。自鄰鹽侵占淮界,本輕利厚,淮鹽難與之敵。查之既煩,堵且生變。計惟重稅鄰私,俾鄰本重而淮本輕,庶鄰鹽化私為官,淮鹽亦得進步。現已咨湖廣、江西各督撫,將鄰私釐金加抽,待至淮運日多,銷路日暢,然後逐之而申其禁,此疏銷之略也。近年楚西之鹽,每引完釐在十五兩以上。今改逢卡抽收為到岸銷售後匯總完釐。前收十五兩有奇,今楚岸祗十一兩九錢八分,西岸九兩四錢四分,皖省四兩四錢。既減釐以便商,人先售而後納,此輕本之略也。商販求利,皆原價昂,然往往跌價搶售。其始一二奸商零販,但求卸物先銷,不肯守日賠利。其後彼此爭先,愈跌愈賤,雖欲挽回以保成本,不可得也。現於楚西各岸設督銷局,鹽運到岸,令商販投局掛號,懸牌定價,挨次輪銷,時而鹽少,民無食貴之虞,時而銷滯,商無虧本之慮,此保價之略也。鹽法首重緝私。大夥私梟,不難捕拏,最易偷漏者,包內之重斤,船戶之夾帶。現改復道光三十年舊章,每引六百斤分八包,每包給滷耗七斤半,包索二斤半,共重八十六斤,刊發大票,隨時添給,並於大盛關、大通、安慶等處驗票截角,如有重斤夾帶,即提鹽充公。其各岸之兼行鄰鹽者,亦另給稅單,茍無單販私,即按律治罪,此杜私之略也。」

又論:「淮北鹽務,有必須停止者三,急宜整理者四。漕臣以清淮設防,令場商每包捐鹽五斤,每引共二十斤,旋因逐包捐繳不便,改每運鹽百包,帶繳五包,其應完鹽課及售出鹽價,雖經吳棠奏明作為清淮軍需,但錙銖而取之,瑣屑而派之,殊非政體所宜。此須停止者一也。徐州本山東引地,前因捻氛,引未到岸,經督辦徐宿軍務田在田奏準散運北鹽,畫收東課,日久弊多,採買則私自赴場,售銷則旁侵皖界。今東引業已通行,不能再託借運虛名,貽侵銷實患。此須停止者二也。北鹽已改捆為凈鹽,未改為毛鹽,皆須納課方準出湖。近來私梟句串營弁,朋販毛鹽,堵之嚴,則營員出而包庇,緝之疏,則官引盡被占銷。此須停止者三也。夫榷鹽之法,革其弊而利自興。臣所謂整理之方,蓋亦就諸弊既去,因勢利導耳。淮北綱引,前奏至戊午為止。今於五月接開己未新綱,惟兵燹後戶口大減,斷不能銷四十六萬引。請先辦正額二十九萬六千九百八十二引,引收正課一兩五分一釐,雜課二錢,又外辦經費四錢,倉穀河費鹽捕營各一分,他款一概刪除。此現籌整理者一也。近來軍餉賴鹽釐接濟,而處處設卡,商販視為畏途。從前每包約完釐錢二千餘。今擬自西壩出湖,先在五河設卡,每包收五百文,運赴上海,再於正陽關收五百文。他卡只準驗票,不準重收。蓋非減釐不足以輕本,非裁卡不足以恤商。此現籌整理者二也。淮北解餉,向以十成分攤。臨淮軍營四成,滁州四成,安徽撫營二成。今臨、滁兩營已裁,而漕臣應量予撥濟,嗣後仍應以十成分派,臣營五成,撫營四成,漕營一成。論兵數則小有裒益,論舊制則無甚更張。此現籌整理者三也。北鹽每引例定四百斤,捆四包,每包連滷耗重百十斤。近來棧鹽出湖,皆在西壩改捆,大包重百三十斤,鹽票不符。臣已嚴禁,並於例給大票外,將每船裝鹽包數亦填明艙口清單,庶可杜避重就輕,不致以多報少。此現籌整理者四也。」均如所請行。

國籓更張鹽法,與陶澍不同者,澍意在散輪,與玉庭、若霖同。國籓意在整輪,與全德、曾燠同。然玉庭、若霖籌辦散輪,必前兩月之輪賣畢,再開後兩月續到之輪,未嘗不以散寓整,澍實師其意。故國籓鑒於搶售之弊而主整輪,爰有總棧督銷之設,一以保場價,一以保岸價。總棧初以儀徵未易修復,設於瓜洲,後岸為水齧而圮,復移儀徵。督銷局鄂岸於漢口,湘岸於長沙,西岸於南昌,皖岸於大通。未幾,國籓移督直隸,李鴻章繼之。其所增捐,莫要於循環給運。其法以認引之事並歸督銷,俾商販售出前檔之鹽,即接請後檔之引。初行之淮南,後及於淮北。蓋參綱法於票法之中,以舊商為主而不易新商。商有世業,則官有責成,視以前驗貲掣簽流弊為少,自是歷任循之。

至光緒五年而增引之說起。增引者,部咨淮北增額八萬。時總督沈葆楨疏言:「近年鹽商以票價昂,覬覦增引。歷任鹽臣精鹽政者無過曾國籓,每審定一法,必舉數十年之利病,如身入其中,而通盤計之。然淮北引額,僅定為二十九萬有奇,豈置國計商情於不顧哉?鹽政之壞,首由額浮於銷,其始尚勉符奏銷之限,久乃不可收拾。於是新陳套搭,未幾而統銷融銷矣,又未幾而帶徵停運矣。惟額少則商少,商少則剔弊易,疏銷亦易也。」八年,左宗棠督兩江,乃請增引,淮北十六萬,淮南鄂岸十一萬、湘岸四萬、皖岸四萬二千餘。部議淮北照行,其鄂岸僅增三萬、湘岸一萬、皖岸一萬七千餘。

及曾國荃涖任,復將淮北加引奏免。蓋兩淮正課,初合織造、河工、銅斤等款,祗百八十餘萬,每引徵銀一兩餘。織造、河工、銅斤者,因鹽政運司養廉厚,陋規亦多,每年解送織造銀二十二萬,捐助河工五萬。三籓之變,滇銅阻隔,派各鹽差採買捐辦,水腳又五萬。及雍正中,裁減養廉規費以為正款,嗣復及他項。於是正雜內外支款遂鉅,每引增至六七兩,自改票後始輕。同治中,引地未復,而以釐補課實過之,正無庸增引也。

至南鹽銷數,向以鄂岸為多。及為川鹽所據,同治七年,國籓請規復引地,部議令川鹽停止行楚。湖廣總督李瀚章疏言未可停,惟於沙市設局,以川八成、淮二成配銷。後以包計,淮鹽較川鹽每包斤少,名二成實不及一成。十年,國籓復言:「川侵淮地,當使淮八成而川二成,或淮七、川三。今楚督以鄂餉數鉅,恐川鹽不暢,入款驟減。臣所求者,淮鹽得銷行楚岸,則商氣蘇,原將應得釐銀,多撥數成或全數歸鄂。」命川、楚督撫會議。國籓等疏言以「武昌、漢陽、黃州、德安四府還淮南,安陸、襄陽、鄖陽、荊州、宜昌五府,荊門州仍準川鹽借銷,湖南祗岳、常、澧三屬行銷川鹽,岳州、常德亦應歸淮,澧州暫銷川鹽」。經部議準。光緒二年,貴州肅清,御史周聲澍疏陳川鹽引地已復,請將湖南北各府州全歸淮南。部議如所請。於是葆楨奏稱湖北川釐,每年報部百五十餘萬串,計合銀不足九十萬,請令淮商包完。然湖廣督撫以川釐有定,慮包餉難憑,合辭袒川拒淮。至八年,宗棠復移文商榷,迄不果行。

長蘆自順治初祗徵課二十萬二千有奇。十二年,按明制查出寧餉酬商滴珠缺額等款,照舊徵解。康熙中,復增課增引,遂至四十二萬六千有奇。乾隆季年,以逐年誤課,參革者眾,於是眾商公議,完課外每引捐銀二錢,以備彌補,名為參課。迨道光末,課額愈重,岸懸愈多,於是又添懸岸課,每引交銀四分,而仍不足。至是國籓督直,疏言:「認商既交寄庫銀千餘兩,宜與保商以三年定限,凡欠在限內,於本商追繳二成,其一成綱總與出結之散商分賠,過限即無涉,以免畏避。」從之。

是時鹽臣自國籓、鴻章、葆楨外,惟宗棠及丁寶楨以能名。同治初,宗棠撫浙,疏言:「自金陵陷,淮鹽侵灌杭、嘉、松三所,惟紹所勉力搘柱。後行鹽地多不守,浙省亦陷。及浙東克復,始飭紹興暫辦票鹽,省城及嘉、湖繼定,而舊商力難運銷,請將四所通改票鹽,並設局稽查銷數。」經部議準。十年,御史奇臣奏言:「浙東府局,於商販鹽至,輒低其價,以便鹽行收買,旋復高其價,以便轉售,利歸中飽。應請裁撤。」部議敕下巡撫楊昌濬查覆。尋覆稱:「兩浙本先課後鹽。自改票運,因商力薄,僅完半課,其半課俟銷後補完。擬撤鹽行,仍留府局,督催後半課銀。」報可。

福建當乾隆時,西路延平、建寧、邵武三府屬十五州縣,東路福寧府屬五州縣,南路閩侯二縣,歸商辦,號「商幫」。南路福州、興化、漳州、泉州四府屬二十一州縣,由官辦,號「官幫」,亦謂之「縣澳官幫」,包與商辦,名「樸戶」。嗣後勻配西路各商代銷,於是有「代額」之名。商幫以課輕,樂於承運,而本課轉拖欠。嘉慶初,乃行帶徵與減引法。旋革除代額,久之倒罷相繼。道光元年,乃改簽商。時舊欠皆價新商,加以場務廢弛,官居省城,聽海船裝鹽,私相買賣,謂之「便海」,流弊滋多。至二十九年復倒罷,乃改官運,而承辦者以運本半入囊橐。蓋閩省行鹽,乾隆時用團秤,每百斤折申砝秤百六十斤,以三十斤抵償折耗。嘉慶中,改用部砝秤,又不給耗鹽,其擔引折篷引每百斤僅給四十二斤,令作百斤售賣,而完代額百斤之課,是以虧折日甚。其後法愈變愈壞。同治四年,宗棠為閩督,乃請改票運,飭各場官住場。西路以引商為票商,縣澳以樸戶為販戶,用鹽道票代引,名曰「販單」。西路以三十引起票,東南兩路及縣澳以百引起票,蓋西路每引六百七十五斤,東南路並縣澳每引百斤故也。裁雜課,令正課一兩加耗一錢,於領票時交納。外抽釐五錢,於行鹽各地設局抽收。計西路每引徵銀四兩五錢零,東南路及縣澳四錢四分零。後以西路課重,奏減每課一兩隨徵釐四錢。凡舊欠各款豁免。帑息既免,帑本則責令陸續歸還。是年徵課耗釐銀四十萬餘,帶收舊欠課十九萬餘,即以四十萬定為正額。行之數年,商情大歡,私販斂跡。

陜西花馬池鹽課,向由布政使收納。及同治十二年,宗棠為陜甘總督,因西陲用兵,改課為釐,在定邊設局抽收,名曰花定鹽釐。於是陜西鹽利歸於甘省。

初川鹽以滇、黔為邊岸。而黔岸又分四路,由永寧往曰永岸,由合江往抵黔之仁懷曰仁岸,由涪州往曰涪岸,由綦江往曰綦岸。至是運商困敝,所恃以暢銷者,惟濟楚一策。及淮南規復引地,滯引積至八萬有奇,積欠羨截百數十萬金。光緒初,寶楨督川,定官運商銷,先從事黔岸,籌章程十五條:曰裁減浮費,曰清釐積引,曰酌核代銷,曰局運商銷,曰兼辦計岸,曰引歸局配,曰展限奏銷,曰嚴定交盤,曰慎重出納,曰認真黔釐,曰實給船價,曰刪減引底引底者,運商向於坐商租引配鹽,引給銀二十餘兩,由商總租收,作為課稅羨截,領繳引費,及官吏委員提課規費,商局公費,餘數二兩,分交各坐商。至是歷年羨截,運商已繳,本應全革。惟因年久,姑準存一兩,曰添置辦票,曰酌留津貼,曰酌給獎敘。設總局於瀘州,四岸各設分局,檄道員唐炯為督辦。其後接辦滇岸,川鹽行滇,祗昭通、東川兩府有張窩、南廣兩局,謂之大滇邊、小滇邊。其辦理較黔岸為難者,滇自有鹽,侵越最易。寶楨籌堵遏法,至五年乃開運。

自官運商銷,計本年邊計各額引全數銷清外,復帶銷積引萬餘,所收稅羨截釐及各雜款又百餘萬,而奸民不便。會上遣恩承、童華查辦他岸,至川,富順富紳王餘照假灶戶具詞呈控,請改官督商銷。有旨垂詢。寶楨奏言:「官督商銷,利歸官與商,官運官銷,權全歸官,流弊皆大。惟官運商銷,官商可相箝制。」既而控案訊明,奏請拏辦。迨光緒末,各計岸亦多改官運焉。

此外如奉天由納稅改行引,自康熙中停止,無課者百七十餘年。同治六年,將軍都興阿奏準行榷釐法,每鹽一榷東錢千,為本地軍需。光緒三年,將軍崇厚請加作二千四百文。八年,將軍崇綺再請加二千四百文,名四八鹽釐,是為練兵之款。十七年,戶部籌餉加二千四百文,名二四鹽釐,是為解部之款。二十四年,將軍依克唐阿加千二百文,名一二鹽釐,是為興學之款。此三項總稱八四鹽釐。二十八年,將軍增祺又奏設督銷局,每斤加榷制錢四,謂之加價,以為官本。然原議由官設局收買,置倉運售,名為督銷,實則官運也。值日、俄戰起,亦未實行。三十二年,將軍趙爾巽請裁督銷之名,在奉天立官鹽總局,吉林、黑龍江立分局,聽商就灘納稅運銷。三十三年,東三省設行省,總督徐世昌又改官鹽總局為東三省鹽務總局,於是吉林、黑龍江始實行官運。初歲徵課銀二十四萬或四十萬,及爾巽至,滿百萬,其後至百四十萬。

蒙古鹽向歸籓部經理。其行銷陜、甘者,以阿拉善旗吉蘭泰池鹽為大宗,俗謂之紅鹽。道光以前,聽民運銷。咸豐八年,始招商承運,每百斤收銀八兩。同治間,遭回亂,商困課逋,經宗棠改課為釐,斤加制錢五。其在山西者,亦紅鹽最多。嘉慶初,阿拉善王獻吉蘭泰池,由官招商辦運,將口外各,大同、朔平二府,及太原、汾州等屬,向食土鹽州縣,劃為吉岸引地。至十七年廢除。凡入口者,由殺虎口徵稅,每斤一分五釐。其外尚有三種:曰鄂爾多斯旗鹽,曰蘇尼特旗鹽,俗謂之白鹽,曰烏珠穆沁旗鹽,謂之青鹽。初照老少鹽例,於口內行銷。嘉慶末納稅。至光緒時,皆改用抽釐法。

其在直隸者,則青鹽、白鹽,光緒二十八年察哈爾都統奏請抽釐,每斤制錢四,約年得銀十二萬有奇。明年,熱河都統亦照抽,每斤五文。是年直督又請在張家口設督銷局,在口外設廠收鹽,招商承辦,每千斤包納課銀二兩,約年得三萬有奇。三十三年,熱河亦設局,每百斤徵銀四錢。宣統元年,減為二錢五分,約年得六萬有奇。

新疆向聽民掣銷。光緒三十四年後,始於精河鹽池徵稅萬四千四百兩,迪化徵五千一百兩,鄯善徵二千四百兩,餘仍無稅。

初,鹽釐創於兩淮南北,數皆重。自國籓整頓,乃稍減。繼以規復淮綱,又議重抽川釐。咸豐五年,定花鹽每引萬斤抽釐八兩,嗣因商販私加至萬七千斤,川督駱秉章請就所加斤按引加抽十七兩,共正釐二十五兩。後各省皆加。及光緒時行銅圓,鹽價已暗增,而釐金外更議加價。

其事起雍正時。蓋長蘆鹽價,自康熙二十七年定每斤銀一分四毫至一分二釐六毫不等。雍正六年,巡鹽御史鄭禪寶疏稱「商課用銀,民間買鹽用錢。康熙時,銀一兩換制錢千四五百,每鹽一斤,錢十六文。今每兩合錢二千,而鹽價如故,亦有減至十三四文者,以錢易銀,不敷原數。應請部臣會同督臣詳議」。至十年,題準每斤加銀一釐。乾隆後推行他省,然其意在恤商而已。嘉慶五年,長蘆巡鹽御史觀豫因川、楚未靖,奏請加價濟用。仁宗諭曰:「以餉需擾及閭閻,朕不為也。今計食鹽者每日止一二文,若增價則人人受累。且私販必因鹽價過昂而起。」已而以河工需費,道光後猶多。至光緒二年,辦西徵糧臺,戶部侍郎袁保恆奏請各省一體加二文,以兩江總督沈葆楨力爭乃寢。

嗣是新政舉行,罔不取諸鹽利。如二十年因日本構釁設防,部咨各省每斤加收二文。二十七年因籌還賠款,加四文。三十四年,因抵補藥稅,又加四文,半抵補練兵經費,半歸產鹽省分撥用,其最著者也。時疆吏集商會議,僉以滯銷為憂,而勢不能已,自是所入較道光前又增數倍。然長蘆經拳匪之擾,商本損失,至借洋款。山東引票各地,自同治六年酌歸官辦,弊竇殊多。河東仍歸官民並運,而不能暢銷。福建之票運、四川之官運皆然。廣東潮橋,舊由官運,至時與六櫃統歸商辦,成效亦寡。雲南子井,存者寥寥。而淮、浙衰敝尤甚。

宣統元年,度支部尚書載澤疏言:「淮南因海勢東遷,滷氣漸淡,石港、劉莊等場產鹽既少,金沙場且不出鹽。若淮北三場,離海近,滷氣尚厚,惟麗鹽出於磚池,例須按池定引。近則磚池以外,廣開池基,甚至新基已增,舊灘未劃,致產額益無限制。而南商同德昌在淮北鋪池,北商尤以為不便。兩浙產鹽之旺,首推餘姚、岱山,次則松江之袁浦、青村、橫浦等場,皆板曬之鹽也。而杭、嘉、寧、紹所屬煎鹽各場,滷料亦購自餘姚。近年滷貴薪昂,成本加重,商家既舍煎而取曬,灶戶亦廢灶而停煎。煎數日微,故龍頭、長亭、長林等場久缺,而注重轉在餘、岱。餘姚海灘距場遠,岱山孤懸海外,向不設場,雖經立局建廒,而官收有限,私曬無窮。此產鹽各處之情形也。淮、浙行鹽,各有引地,而豫之西平、遂平,久成廢岸,湘之衡、永、寶三府及靖州,本淮界而銷粵鹽,鄂之安、襄、鄖、荊、宜五府及荊門州,本淮界而銷川鹽,浙之溫、臺、寧、處等處,祗抽釐尚未行引。就目前情形論之,淮北以三販轉運,於岸情每多隔膜,故票販不問關銷,豫販又多歸怨湖販,此其病在商情之不相聯,而各省抽稅,勢亦足以病商。淮南有四岸督銷,權等運司,故運司不能制督銷,分銷亦不盡受轄於督銷,此其病在官權之不相統,而商情渙散,勢亦足以自病。浙場距場近者,有肩引、住引之分。距場遠者,有綱地、引地之別。加以官辦商包,其法不一,紛紜破碎,節節補苴。至捆鹽出場,沿途局卡之留難,船戶之夾帶,則皆不免。此銷鹽各處之情形也。淮鹽行於蘇、皖,與浙鹽、東鹽引界鄰;行於豫岸,與東鹽、蘆鹽引界鄰;行於西岸,與浙、閩、粵鹽引界鄰;行於湘、鄂兩岸,與川鹽、鄂鹽引界鄰。而鄂之襄、樊,又為蘆私、潞私所灌,湘之衡、永、寶,又為粵私所占,兩浙引地,蘇、皖、西三岸皆與淮鄰,即本省之溫、臺等處,亦為閩私所侵,此皆犬牙相錯,時起爭端。近年京漢鐵路通車,貫豫省而下,淮、蘆之爭更烈。將來津浦、粵漢等路告成,淮界且四面皆敵,然此猶言鄰私也。尤甚者,皖、豫同為淮界,而皖之潁州與汝、光界壤,則以加價輕而及豫岸,臺、處同為浙境,而處之縉雲為臺商承辦,則又以包釐微而侵及處郡。江西建昌久為廢岸,近設官運局以圖規復,而貶價敵私。撫州已虞倒灌,上海租界向為私藪,近設事務所以籌官銷,而越界行運,蘇屬時有責言,是以淮侵淮、以浙侵浙也。大抵利之所在,人爭趨之,固未易遏,所恃惟緝私嚴耳。然弁勇窳敗,不能制梟販,而轉擾平民。地方官亦以綱法久廢,不負責成,意存膜視。此又引界毗連各處之情形也。近來籌款,以鹽為大宗,而淮、浙居天下中心,關於全局尤重。為整頓計,非事權統一不可。擬請將鹽務歸臣部總理,其產鹽省分,督撫作為會辦鹽政大臣,行鹽省分,均兼會辦鹽政大臣銜。」制曰可。其言南商鋪池者,蓋光緒三十三年,淮南因鹽不敷銷,於淮北埒子口葦蕩左營增鋪新池,謂之濟南鹽池。三十四年,北商稱有礙舊池銷路,經江督張人駿令按淮南缺額,以十萬引為率。三販轉運者,淮北票鹽,舊由票販自垣運至西壩,售於湖販,再由湖販運至正陽關,按輪售於岸販也。

載澤既受督辦鹽政大臣之命,乃設鹽政處,按各區分為八,先籌淮北。章程四:曰規復西遂廢岸,曰撤退淮邊蘆店,曰體恤路捐商累,曰包繳豫省釐價。咨商河南巡撫吳重熹,惟末條堅持仍舊。載澤又奏定於西壩設鹽釐總局,臨淮關設掣驗局,餘局卡悉裁,三販統改岸販,準自赴總局完納釐金加價,定每引為銀幣二元二角,折收庫平銀一兩六錢零,均一次收清。至土銷引地,酌減銀幣四角,折收一兩二錢,較原額少三成。此二年七月事也。

直隸張家口外收蒙鹽各場,向由商包辦,宣統元年,改為公司。至是復改設官棧,以各州縣為引岸,由商包引,每年二萬,徵銀十五萬七千。四川歸丁各地票運,咸豐後增至六十八州縣,官運常為所礙。至是奏查井灶就現有者為額,嚴禁偷賣,以杜票私。三年,以大清銀行款七百萬、直隸銀行款六十萬為蘆商償外債,收引地三十六歸官辦,設局天津。其永平七屬,道光間由州縣辦課。光緒二十九年,改設官運局。至是與新河、平鄉二縣無商認辦者,統歸津局經理。

初與各國通商,違禁貨物,不許出入口,鹽其一也。乃奉天之大連、旅順,吉林之長春,有日本鹽;吉林之琿春、延吉有朝鮮鹽;黑龍江之滿洲里、黑河,吉林之東寧,有俄羅斯鹽;廣西之鎮南關,雲南之蒙自,有法蘭西鹽;香港、澳門所在侵灌。至山東膠州灣租借於德,而侵即墨鹽場;奉天遼東半島租借於俄,又轉於日,而占金州鹽灘;與復州之交流、鳳鳴兩島,有包購餘鹽、派員緝私兩議。後緝私策行,購鹽不果。廣東廣州灣租借於法,吳川之茂琿場為所占,每運鹽至香港及越南銷售,以入內地,實皆敗亂鹽法。治鹺政者當有以善其後云。


\end{pinyinscope}