\article{志三十}

\begin{pinyinscope}
地理二

△奉天

奉天:禹貢青、冀二州之域。舜析其東北為幽、營。夏仍青、冀。商改營州。周,幽州。明,遼東都指揮使司。清天命十年三月,定都沈陽。天聰八年,尊為盛京。順治元年,悉裁明諸衛所,設內大臣、副都統,及八旗駐防。三年,改內大臣為昂邦章京,給鎮守總管印。康熙元年,改昂邦章京為鎮守遼東等處地方將軍。四年,改鎮守奉天等處地方將軍。光緒三十三年三月,罷將軍,置東三省總督、奉天巡撫,改為行省。北至洮南;與黑龍江界。南至旅順口;海界東南,以鴨綠江與朝鮮界。西至山海關;與直隸界。東至安圖。與吉林界。廣一千八百里,袤一千七百五十里。北極高三十九度四十分至四十四度十五分。京師偏東四度至十二度。宣統三年,編戶一百六十五萬五百七十三,口一千六十九萬六千零四。共領府八,直隸五,三,州六,縣三十三。案:盛京,天聰五年因明沈陽衛城增修。城周九里三百三十二步,高三丈五尺,厚一丈,女墻高七尺五寸,垛口凡六百五十一。門八:東之左曰撫近,右曰內治,南之左曰德盛,右曰天祐,西之左曰懷遠,右曰外攘,北之左曰地載,右曰福勝。門各有樓闉,加之角樓。四城之中為大政殿,太宗聽政之所也。殿西為大內。南向曰大清門,門內曰崇政殿,殿前東飛龍閣,西翔鳳閣。崇政殿直北為鳳凰樓,樓北清寧宮。宮之東曰衍慶宮、關睢宮,西曰永福宮、麟趾宮。鳳凰樓之前,東為師善齋,齋南日華樓,西協中齋,齋南霞綺樓。崇政殿東頦和殿,殿後介祉宮,宮後為敬典閣。崇政殿西為迪光殿,殿後保極宮,宮後繼思齋,齋後崇謨閣。大內之西文溯閣,藏書之所也。東南太廟。銀庫在大政殿南,織造庫在大內南。戶部、禮部、工部在銀庫東,刑部、兵部在織造庫西。御史公署在城東北隅。其外關城則康熙十九年建,高七尺五寸,周三十二里四十八步。門八:東之左曰小東關,右曰大東關,南之左曰大南關,右曰小南關,西之左曰大西關,右曰小西關,北之左曰小北關,右曰大北關。關城內南為天壇,東為地壇、為堂子,西南隅為社稷壇、為雷雨壇,東南隅為先農祠、為耤田。耤田西南隅設水門二,導小沈水自門出焉,下流注于渾河。其名山為醫巫閭、松嶺。其巨川為遼河、渾河。其重險:山海關、鳳凰城、威遠堡。其船路:自營口西南通天津,南通之罘,東南通朝鮮仁川。其鐵路:內屬者,營榆;屬日者,俄築東清枝路。其電線:西通天津,西南旅順,東南鳳凰、安東,東北吉林。

奉天府:沖,繁,疲,難。總督兼將軍,民政、提法、交涉、度支、鹽運司,勸業道,副都統駐。順治十四年四月,於盛京城內置府,設府尹。光緒三十一年八月,裁府尹設知府,為奉天省治。西南至京師一千四百七十里。廣八百七十里,袤九百九十里。北極高四十一度五十一分五十秒。京師偏東七度十五分。領一,州二,縣八。承德沖,繁,疲,難。倚。明,沈陽中衛。康熙三年置縣,附府。福陵在東二十里天柱山,昭陵在西北十里隆業山。有副都統兼二陵守衛大臣。渾河在南,即沈水,自撫順入,西南入遼中。左受高素屯、白塔鋪、於家臺河,右受馬官橋、萬泉河。萬泉亦稱小沈水。東北:大清山,蒲河所出,西南流,逕永安橋,入新民境。永安橋,崇德六年建。初,太祖定沈陽,以西路沮洳,命旗丁修疊道百二十里,直抵遼陽。太宗復建此橋,行旅便之。舊設驛四:西老邊,通新民;北懿路,通鐵嶺;東噶布拉村,通興京;南十里河,即明虎皮驛,通遼陽。鐵路三:京奉,東清,安奉。京奉鐵路行境內六十里,車站二,曰馬三家,曰沈陽,在小西關外,即京奉全路尾站。商埠,光緒二十九年八月中美約開。遼陽州繁,疲,難。府南百二十里。明,定遼中衛,兼置自在州。天命六年三月克遼陽,四月遷都於此。十年移沈陽。順治十年設遼陽府,遼陽縣附郭。十四年,府移縣來隸。康熙三年六月,縣升為州,仍隸府。有城守尉。南:千山山脈,東自懷仁老嶺入,為遼東半島之脊,山南之水,獨行入海,遼東山脈主峰也。北:太子河,自本溪入,西流至遼中境,迤南入海城。左受細河、藍河、湯河、沙河、鞍山河;右受十里河,渾河枝水,國語曰塔思哈河。舊設驛三:曰迎水寺、浪子山、甜水站。商埠,光緒三十一年中日約開。有東清鐵路。復州繁、疲。府南五百四十里。明為復州衛。天命七年三月復州降。康熙三年並入蓋平。雍正四年,分蓋平地置復州。十一年改為州,隸府。有城守尉。州境多山,西與西南皆海。其海曰復州灣。北:浮渡河。南:復州河,左受欒古河,皆西入海。東:沙河、清水、贊子、碧流,右受吊橋河,皆南入金州。長興島州判在西南百四十里海中,光緒三十四年置。其東北娘娘宮。港岸曰東崖、西崖,商船出入,海道咽喉也。水門子巡檢,光緒三十二年置。舊設鋪司四:北核桃哨、李官墳,通蓋平;南麻河鋪、欒古城,通金州。有東清鐵路。撫順沖,繁,疲,難。府東八十里。明,撫順千戶所。天命四年克撫順。光緒二十八年,分承德縣地設興仁縣,附府。三十三年移治撫順城,劃興京西北地入之,更名,仍隸府。東:薩爾滸山、鐵背山,皆天命四年破明兵處。南:渾河南北二源自興京入,合流西,左受章黨、馬郡丹、塔兒峪、拉古河,右受溫道、柳林、金花樓河,入承德。東有營盤市鎮,舊設驛一。薩爾滸南,奉撫運煤鐵路;西南姚千戶屯,安奉鐵路。開原繁、疲。府東北三百里。明洪武二十屴,置三萬衛於元開元路故城西,二十一年徙此。改開元為開原。永樂七年兼置安樂州。天命四年六月克開原。康熙三年六月置縣,隸府。有城守尉。東北:黃龍山。西北:遼河自康平入,左納馬鬃、亮子河。南:清河,右受碾盤河、扣河;又南沙河,皆西入遼河。東南:柴河,西入鐵嶺境。又東南英額河,西南入興京。邊門三:北馬千總臺,東北威遠堡,東南英額。舊設驛一。又有道,東南經石人溝至山城子,西經英城子達法庫門,東經威遠堡門至西豐,號四達通衢。有東清鐵路。鐵嶺沖,疲。府北一百三十里。明置鐵嶺衛。天命四年七月克鐵嶺。康熙三年六置縣,隸府。有防守尉。遼河在西,自開原東南流入,屈西南流入法庫境。其旁多水泊,曰蓮花泡、葦子、五角、蓮子、樂子諸湖,瀰漫十里,土人呼遼海,有遼海屯。北柴河,南範河,又南懿路河,皆西入遼河。舊設驛一。商埠,中日約開。有東清鐵路。海城繁,疲,難。府南二百四十里。明置海州衛。天命六年海州降。順治十年十一月置海城縣,隸遼陽府。十四年四月改隸。西六十里有牛莊防守尉。西南:唐王山。遼河在西。渾河自遼中入,曰蛤蜊河,左匯太子河,西流入之,名三岔河。北土河、鞍山河西入太子河,南入海州河,西入遼河。三岔巡司,康熙二十一年置,駐牛莊。西鄉、三家子、石佛寺等處舊有河道,繞流入遼,後淤塞。光緒三十四年開濬故河,涸出良田三十六七萬畝。東南有析木城市鎮。舊設鋪司四:西南營口,南大石橋,接蓋平;北鞍山站,接遼陽;東二道河,入岫巖。有東清鐵路。蓋平繁,疲。府西南三百六十里。明置蓋州衛。天命六年三月蓋州降。康熙三年六月置縣,隸府。有城守尉。又西南六十里有熊岳防守尉,故遼城也,舊駐副都統,後裁。東:棉羊山,縣東南諸山皆發脈於此。西瀕海曰蓋州灣。北:淤泥河。南:蓋州熊岳河、浮渡諸河,皆西流入海。東南:碧流河,即畢利河,出布霧山,南流入復州。舊設鋪司三:西北沒溝營,北大石橋,南熊嶽城。有紅旗廠、藍旗廠、吳家屯三鹽場。有東清鐵路。遼中繁,難。府西南一百四十里。明,定遼中衛、右衛地。光緒三十二年七月,分新民、遼陽、海城地,設治阿司牛錄鎮,尋劃承德西南境增入,置縣隸府。遼河在西,有冷家口。支流西南入盤山,曰分遼水,亦曰減河。正流南入縣境。又西柳河,南入分遼水。又西鷂鷹河支津,南入柳河。東:蒲河自新民入,南入渾河。又東南太子河支津二入之。西南:遼河。西:達都牛錄,縣丞駐,光緒三十三年置。本溪府東南一百二十里。明為清河城。光緒三十二年,分遼陽、興京、鳳凰地,設治本溪湖,置縣隸府。南:摩天嶺,一名太高嶺,山脈東連老嶺,西接千山。其北:細河,即萬流河,北流入遼陽。其南:草河、賽馬集河,南流入鳳凰。南:太子河南北二源,自懷仁、興京入,合流西入遼陽。東:清河,南入太子河。賽馬集巡檢,光緒三年置,屬鳳凰,三十二年來屬。舊設連山關驛。有安奉鐵路。金州沖,繁,疲、難。府南七百二十里。明置金州衛。雍正十二年置寧海縣,隸府。道光二十三年改金州,仍隸府。有副都統,寄治承德。境萬山環抱,東西南北皆海,惟東南一隅陸地,連復州成半島形。沙河、清水、贊子、碧流諸河在東北入海。有貔子窩市鎮。旅順口在西南。自旅順循半島以西,歷遼河口、大小凌河口至山海關,為渤海岸;以東歷碧流河口、莊河口、大洋河口至鴨綠江,為黃海岸。旅順鐵山角與山東登州頭對峙,為渤海口門。有舊水師營城。舊設鋪司一,石河驛。商埠:光緒二十三年中俄約開。海關在大連灣。有東清鐵路。

法庫直隸:沖,繁,難。省西北一百六十里。明,三萬衛地。康熙元年,設法庫邊門防禦。光緒三十二年,分新民府及開原、鐵嶺、康平三縣地,設治法庫門,置,直隸行省。法庫山在南。遼河自鐵嶺入,北流,屈西流,逕南入新民。其津渡處有三面船市鎮。西:沙河,南入遼河。又西秀水河,南入新民。有秀水河市鎮。城北門仍舊邊門。邊門外道路作三叉形。西行至彰武;北行由桃兒山、馬奇溝赴康平,可至吉林伯都訥;東北行由齊家店、公主屯赴昌圖,可至吉林長春。北邊沖要也。商埠,中日約開。

錦州府:繁,難。明置廣寧中、左、右屯三衛,隸遼東都指揮使司。崇德七年三月克錦州。康熙三年置廣寧府,並縣為治。四年改置,徙治錦。省西南四百九十里。廣五百三十里,袤百七十里。北極高四十度九分。京師偏東四度三十九分。領州二,二,縣三。錦沖,繁,疲,難。倚。明置廣寧中屯衛及左屯、右屯衛。康熙元年七月改錦州為錦縣,隸奉天府。三年六月改隸廣寧府。十二月罷廣寧,置錦州縣,附府。舊駐副都統。光緒三十四年裁。有協領。松山、杏山、塔山在南,皆崇德七年破明兵處。紫荊山在東,為縣境諸山冠。南瀕海。東大凌河,西小凌河,右受女兒河,皆南入海。西南:天橋廠巡檢,雍正元年置。又西南海濱有地伸出海中如三角形,曰葫蘆島,島勢向西環抱成一海灣。光緒三十四年,勘為通商港。舊設驛二:小凌河,十三山。京奉鐵路行境內一百一十里,車站四:錦州,雙陽甸,大凌河,石山站。鹽場八:上坎、天橋廠、大東山、白馬石、邰子屯、頭溝、四溝、沙溝。卡倫二:高家屯,天橋廠。錦西繁,難。府西九十五里。明,廣寧中屯衛地。光緒三十二年分錦縣西境置江家屯,尋更名。三十三年隸府。東:大虹螺、小虹螺山,山東七里河,南入海。女兒河導源直隸朝陽,東流入邊,逕北,迤東北流,又東流入錦縣。北:松嶺邊門。東北:虹螺峴市鎮。舊設高橋驛。京奉鐵路車站三:連山,高橋,女兒河。盤山沖,疲,難。府東一百七十里。明,廣寧盤山驛。光緒三十二年,分廣寧縣地及盤蛇驛牧廠地置,隸府。南瀕海。分遼水自遼中冷家口西南入,逕南入海。西南:沙河、東沙河、西沙河皆南入海。錦營鐵路自廣寧溝幫子站分支入境,東南入營口,長百二十餘里。車站三:胡家窩棚,雙臺子,大窪。鹽場五:藍石鰲、西夾信、南夾信、二道磧、二龍江。義州繁,疲,難。府北九十里。明,義州衛。天命七年正月克義州。崇德元年以封察哈爾。康熙十四年,察哈爾叛,討平之。六十一年設通判。雍正十一年,置州隸府。有城守尉。東北:英歌龍灣山。東南:望海山。西北:昆侖山。西南:大嶺、小嶺。大凌河導源直隸朝陽,東流入邊,逕州北,屈南流入錦縣。細河、清河導源直隸阜新,合流南入大凌河。小凌河亦導源朝陽,東流入邊,逕州西南,迤南流入錦縣。楊樹溝河南入小凌河。北有九官臺、清河、白土廠三邊門。舊設鋪司四:南大嶺關、隆祉、七里河,東大榆樹,皆通錦縣。寧遠州沖,繁,疲、難。府西南一百里。明置寧遠衛。順治元年克寧遠。康熙三年置州,隸廣寧府,尋改隸府。有城守尉。西北:青山。西南:望夫。東:首山。南瀕海。寧遠西河、寧遠東河,在城南合流,南入海。又西東沙、煙臺、東關站、六股諸河,皆南入海。有釣魚臺海口。海中島有桃花、菊花即覺華島,島西南小島二,曰小張山、大張山,相距間水勢深闊,足容大戰艦。島岸山可建砲臺。光緒三十四年勘為海軍港。西北:白石嘴、梨樹溝、新臺三邊門。市街四邑環錯。有山海關道稅局。舊設驛二:東關、寧遠。京奉鐵路車站三:東辛莊,沙後所,寧遠州。鹽場十:廠子溝、項家屯、蘇家屯、張莊、杜家臺、蜊蝗溝、五里橋、狐貍套、沙坨、大明山。廣寧沖,疲。府東北一百六十里。明,廣寧衛。天命七年克廣寧。康熙三年六月改廣寧為府,設廣寧縣。十二月府移錦州,縣隸府。有城守尉。醫無閭山在西,古幽州鎮,今有北鎮廟。東:沙河導源醫無閭山三道溝,東南流,逕城北而南,右受大石橋河,入盤山西南閭陽驛。河南流入盤山,曰西沙河。西北:馬市河,東南流入羊腸河。舊設廣寧驛。京奉鐵路行境內七十五里,車站三:羊圈子,溝幫子,青堆子。自溝幫子分支逕南歷盤山達營口,名錦營鐵路,計行境內三十里。有馬帳房、大臺、小臺、毛家屯、郭家屯、北井六鹽場。綏中沖,繁,疲,難。府西南一百九十里。明,廣寧前屯衛、中前所、中後所。順治元年,克廣寧前屯衛,中前、中後所。康熙三年,以其地並入寧遠州。光緒二十八年六月析出置縣隸府。北:大碏子山。西:松嶺,筆架山。南瀕海。東以六股河與寧遠界。六股河即古六州河,導源直隸建昌,從白石嘴邊門入。右受黑水、王寶河,迤南流入海。西:高兒、石子、涼水諸河,皆南入海。西:山海關。邊門十有七,在縣境者曰明水塘邊門。舊設驛二:山海關、涼水河。京奉鐵路行境內一百一十里,車站四:前所,前衛,荒地,綏中。

新民府:沖,繁,難。省西一百二十里。明,沈陽中衛與廣寧左衛地。嘉慶十八年六月,分承德、廣寧二縣地置新民,隸奉天府。光緒二十八年,升為府。廣五百三十里,袤百七十里。北極高四十一度五十六分。京師偏東七度三十三分。領縣二。無城。遼河自法庫入,屈西南逕古城。養息牧河自彰武入,左合秀水河,南入遼河。其東蒲河自承德入,逕黑魚泡,西有新開河自庫倫入,為柳河,並入遼中。又西鷂鷹河,南入鎮安。舊設驛二:白旗堡、巨流河。京奉鐵路車站四:白旗堡,新民府,巨流河,興隆店。商埠,中日約開。鎮安沖,難。府西一百五十里。明,廣寧衛之鎮安堡。光緒二十八年,分廣寧東境,設治小黑山,置縣隸府。西:羊腸河,導源直隸阜新,下流散漫。東沙河導源直隸綏東,南流,右受老河,入盤山曰南沙河,又東鷂鷹河,南溢為蓮花泡,入分遼水。小三家子,縣丞,光緒三十二年置。三十四年,其地設奉天官牧場。東北有半拉門市鎮。舊設驛二:小黑山、二道井。京奉鐵路行境內八十里,車站四:高山子,打虎山,勵家窩鋪,繞陽河。有卡:拉木屯、營城子二。彰武繁,疲,難。府北百十里。明初,置廣寧後屯衛,後徙。康熙三十一年設養息牧廠於此。光緒二十八年以養息牧墾地,設治橫道子,置縣隸府。縣境居彰武臺邊門外。東北:阿莫山。東:少陵哈達山。西北:杜爾筆山。西:柳河,又西鷂鷹河,皆導源直隸綏東,世所稱小庫倫也。東:養息牧河,導源科爾沁左翼前旗,皆南流入府境。西北:哈爾套街,縣丞,光緒二十九年置。有官商路三:一由縣治赴府,一由縣西北哈爾套街赴直隸綏東,一由縣西新立屯赴直隸阜新。

營口直隸:省西南三百六十里。明,蓋州衛之梁房口關。同治五年,設營口海防同知。宣統元年,分海城、蓋平兩縣地置,直隸行省。奉錦山海關道改為分巡錦新營口兵備道,駐。北:遼河自海城入,南迤東流,屈西流入海。納東南淤泥河,至蓋平遼河入海口。距治四十五里,輪舶交通之地也。初,境名沒溝營,為蒙古人窩棚。道光中辦海防,其地始重。通商後乃繁盛。錦營鐵路自盤山大窪車站入境,歷田莊臺至河北車站,長六十七里。又自河東牛家屯至大石橋,與東清鐵路接。有二道溝、三道溝等鹽場。漁業總局。商埠,咸豐十年天津約開。有海關。光緒三十二年設遼河巡船十艘。三十四年增安海、綏遼兩巡海兵艦。

興京府:繁,疲,難。省東南三百二十里。明,建州右衛。天聰八年,尊赫圖阿拉地曰興京。乾隆三十八年,設理事通判。光緒三年,改為興京撫民同知,移治新賓堡。宣統元年,升為府。廣六百六十里,袤三百一十里。北極高四十一度四十五分十五秒。京師偏東八度三十七分十六秒。領縣四。永陵在西四十里啟運山,駐副都統。西三十里興京城,駐協領。東納嚕窩集果爾敏珠敦,總謂之分水嶺山脈,上接庫哷納窩集,下連龍岡。山西之水皆入遼河,山東之水皆入松花江,為遼河、松花江之分水嶺,即漢志遼山也。渾河出其下。南源曰蘇子河,左合索爾科河,西北流,北源曰英額河,左合滾馬嶺河,西南流,俱入撫順。西南:平頂山,太子河北源所出,西入本溪。舊設驛一:穆喜。鋪司四:南老城、大呼倫、窪子嶺,入鳳凰境;東舊門,通懷仁。通化繁,難。府東南二百七十里。明,建州衛之額爾敏路。光緒三年置縣,隸興京同知。宣統元年改隸府。縣境居旺清邊門外。北:龍岡山脈,自興京、海龍間納嚕窩集入,迤邐而東,歷臨江直達長白山,亙二百餘里。山南之水皆入鴨綠江,山北之水皆入松花江,為鴨綠江、松花江之分水嶺,以其為永陵幹脈,故曰龍岡。南有渾江,自臨江入,西流,屈東流,復迤西南入懷仁。左受大羅圈溝河、小羅圈溝河,右受哈泥河、加爾圖庫河。舊設馬撥七:西哈馬河、快當帽子、英額布、歡喜嶺、半截拉子,入興京;又由快當帽子西南行,曰高麗墓、頭道溝等,達懷仁。懷仁疲,難。府南一百八十里。明,建州衛之棟鄂部。光緒三年置縣,隸興京同知。宣統元年改隸府。縣境居堿廠邊門外。老嶺在西南,太子河南源所出,西北入本溪。老嶺山脈自龍岡分入,迤西與摩天嶺接,山南之水皆入鴨綠江,山北之水皆入遼河,為遼河、鴨綠江之分水嶺,國語曰薩禪山。渾江自通化入,流經北、西、南三面,入輯安。富爾江合衣密蘇河自北,六道河、大雅河自西,流入渾江。富爾江口蓋古梁口也。古棟鄂河,南入大雅河。西:四平街巡檢,光緒四年置。渾江南流旋曲處有哈達山,乾隆十一年設莽牛哨於此,尋廢。舊設馬撥十:東北三層砬子、二棚甸子、硃胡溝、恆道川、長春溝,入通化境;西南大雅河、前牛毛、大青溝、砍椽溝、掛牌嶺,入寬甸。輯安疲,難。府東南四百二十里。明,建州衛之鴨綠江部。光緒二十八年,分通化、懷仁二縣地,設治通溝口,置縣隸興京同知。宣統元年改隸府。東北:老嶺岡。北:丸都山。鴨綠江在南,自臨江入,迤西南入寬甸。西:渾江自懷仁入,南入鴨綠江,曰渾江口。光緒三十四年,設鴨、渾兩江巡船。西岔溝門巡檢,光緒三年置,駐通溝口,二十八年移駐。舊設馬撥九:北同和嶺、梨樹溝、葦沙河、二道崴子、夾皮溝,入通化;西五道嶺、皮條溝、上漏河、二棚甸子,入懷仁。又光緒三十四年城東新闢一道,由錯草溝出臨江。臨江繁,難。府東南五百九十里。明,鴨綠江部。光緒二十八年,分通化縣地,設治帽兒山,置縣隸興京同知。宣統元年改隸府。北有龍岡。鴨綠江在南,自長白入,西北流,屈西南,入輯安。西:頭道溝,以次而東,而東北,沿鴨綠江有二十五道溝,皆岡前山水,南流注江,縣得其七,長白得其十八。北:三岔子,即長白山西南分水嶺,渾江所出,西南流,左受紅土崖河,入通化,舊所稱佟家江也,西北入道江。巡檢,光緒二十八年自帽兒山移駐,屬通化,宣統元年來屬。初,縣西北接通化,山路險絕。光緒三十四年改修,自林子頭越老爺嶺,歷三道陽岔達縣治,剷山梁谿,長百二十餘里,通車馬,名蕩平嶺道。

鳳凰直隸:沖,繁,難。省東南四百八十里。明置鳳凰城堡。天命六年降。乾隆四十一年,設鳳凰城巡司。光緒二年改置,直隸行省。廣六百六十五里,袤四百里。北極高四十度三十四分十六秒。京師偏東七度四十九分三十五秒。領州一,縣二。有城守尉。鳳凰山在南。四大嶺在西北。南瀕海。東:草河,右受通遠堡河,左合靉河,南入安東。東北:賽馬集河,南入靉河。西:大洋河,南入海。西北:哨子河,南入大洋河。東北靉陽、南鳳凰二邊門。舊設驛三:通遠堡、雪裡站、鳳凰城。有窟窿山至洋河口鹽場。商埠,中日約開。有安奉鐵路。岫巖州沖,繁,疲,難。西北一百八十里。明置岫巖堡。乾隆三十七年設岫巖城通判。光緒二年改為州,隸。有城守尉。南:羅圈背嶺。西北:分水嶺,大洋河出東南,流繞城東,右受雅河、大王攔溝河。又東南,哨子河自北來匯,屈南流,右受小洋河,入莊河。其左岸為境。舊設鋪司三:東哨子河,入境;北偏嶺、奔溝,接海城。安東繁,疲,難。東南一百五十里。明置鎮江城,天命六年降。光緒二年置縣,隸。分巡奉天東邊兵備道,宣統元年改為分巡興鳳兵備道,駐縣。縣境居鳳凰邊門外。北:元寶山。鴨綠江東自寬甸入,右受草河,迤南流入海。其海岸曰大東溝,即太平溝,木材輸出之地也。有巡司,光緒二十六年置。東有九連城鎮,對岸即朝鮮義州。舊設馬撥十一:東沙河鎮,北中江臺、大樓房、老邊墻,西北高麗店、營臺、湯山城、邊門口,西南白菜地、石橋崗、大東溝。有二道溝至窟窿山鹽場。大東溝商埠,中美約開。有海關。安奉鐵路。寬甸繁,疲,難。東北一百八十里。明,東寧衛之寬甸六堡。光緒三年置縣,隸。縣境居靉陽邊門外。東南:盤道嶺、望寶山。東北:掛牌嶺。鴨綠江南自輯安渾江口流入,西南入安東。右受小蒲石、永甸、長甸、大蒲石、安平諸河。東:渾江,右受小雅、北鼓、南鼓諸河。靉河導源西北牛毛嶺,西南入境。西南:長甸河縣丞,東北:二龍渡巡司,皆光緒三年置。東南有小蒲石河、東北有太平哨二市鎮。舊設馬撥十四:西大水溝、葡萄架、毛甸子、懸羊砬子、土門子、太平川、夾河口,入安東;東北馬牙河、曲柳川、頭青溝、寺院崴子、興隆峪、北土門子,入懷仁。

莊河直隸:沖,繁,難。省南六百里。明,鳳凰城、岫巖城、金州衛地。光緒三十二年,分鳳凰、岫巖州地置,隸東邊道。南瀕海。西以碧流河與復州、金州界。東以大洋河與鳳凰界。莊河導源西北雞冠山,南流,逕東入海。東:英阿、沙河,皆南入海。東:孤山、石城島二巡司。又東南百四十里,海中鹿島,宣統元年隸。大洋河亦稱大孤山港,港內商船通行,惟輪船不能進駛,寄泊鹿島。西花園口,東青堆子,皆臨海小商港。官商路三:東欒店,赴鳳凰;北八道嶺,赴岫巖;西北拉木屯,赴復州。

長白府:沖,繁。省東南九百八十里。明,建州衛之鴨綠江部。光緒三十三年,分臨江縣及吉林長白山北麓地,設治塔甸,置府。北極高四十二度。京師偏東十二度。領縣二。長白山在北。上有天池,舊曰闥門,形橢圓,斜長二十九里,周七十餘里。池深莫測,水鳴如鼓,七日一潮,土人謂池與海通。鴨綠江導源天池南曰靉江,南流至雙岔口,葡萄河自東北來匯,此下為中、韓界水,始名鴨綠江。屈西流,逕府南入臨江。西以八道溝與臨江界。東北至二十五道溝。府治居十八道、十九道溝間。唐滅高麗,用兵於此。府治對岸即朝鮮惠山鎮。初,府境僅治鴨綠江一小徑,倚巖臨澗,必乘木槽渡江,假道朝鮮。光緒三十四年新闢龍華岡道,自臨江新化街、史家蹚子以下入府西嘉魚河,至梨溝鎮達府治西,長約四百餘里,以避江道之險焉。安圖沖,繁。府東北四百里。明,建州左衛地。宣統元年,以府東圖們江源地,設治紅旗河口,置縣隸府。長白山在西。圖們江在南,導源紅土溝,即長白山東南分水嶺,東入吉林。東:紅旗河,導源荒溝,即長白山東北分水嶺,東南入圖們江。西北有二道江,自天池出,北流,曰二道白河。娘娘庫河導源荒溝,西北流,左合五道、四道、三道白河注之,屈西,富爾河自吉林南流注之,曰上兩江口,二道江之名始此。又西,左受頭道白河,入撫松。松花江正源也。西二百里布爾瑚里有天女浴池碑,土人呼圓池。東南七里湖,由府至縣之道,光緒三十四年勘定。自府東二十一道溝口入岡北行,出二十二道溝、十九道溝之間,至靉江源,經小白山後至新民屯,東行歷齊國屯、朝陽窩達縣治。由縣西北行至上兩江口,達吉林樺甸。東渡紅旗河,達吉林延吉。南渡圖們江,即朝鮮境。撫松沖,繁。府西北五百二十里。明,建州衛之訥音部。宣統元年,以府西北松花江上游地,設治雙甸,置縣隸府。長白山在東。頭道江在西,上源曰緊江、漫江。緊江導源長白山西坡,漫江導源章茂草頂山,即長白山西南分水嶺,合而西北流,湯河自吉林東北流注之,頭道江之名始此。又西北流,右受松香河。又西北,二道江自安圖西流來匯,曰下兩江口。此下統名松花江,入吉林。由府至縣之路,自府西梨溝鎮至十五道溝,西北行,逾嶺頂,經竹木里、漫江營、小谷山、石頭河、海青嶺、大營、湯河口,再北行達縣治。由縣西渡江,入吉林濛江。北循松花江,直抵吉林省城。

海龍府:沖,繁,難。省東北六百里。明,海西女真輝發、哈噠、葉赫三部。光緒五年,以流民墾鮮圍場地置海龍。二十八年,升府。領縣四。府境居英額邊門外。西:納嚕窩集果爾敏珠敦,與興京分山脈,唐謂之長嶺。輝發江在南,導源納嚕窩集東麓,北流屈東,左受橫道河、梅河、沙河、大沙河,右受押鹿、一統河,入輝南,國語曰遼吉善河,入松花江。英額河導源英額邊門東,當果爾敏珠敦西麓,西南入開原,即渾河北源。東:朝陽鎮。西:山城子鎮。舊設馬撥十:自城西沙河口、大黑嘴子、山城子、二龍山、郭家店、土口子、孤家子、李家店、八棵樹、貂皮屯,至尚陽堡入開原。又有道由城東奶子山至托佛入吉林城;東北馬家船戶至康大營入吉林伊通;牛心頂子至郭大橋入吉林磐石。東平繁,難。府西六十里。明,梅赫衛,後屬輝發部。光緒二十八年,分海龍屬之東圍場地,設治大度川,置縣隸府。東北:庫哷納窩集,山脈連綿,與果勒敏珠敦接。其南橫道河、梅河、沙河、大沙河,皆東南入府。其北小伊通河,西北入吉林。縣治居沙河北,西有鷂鷹河,東有柳樹河,南入沙河。官商路四:一,由縣南渡沙河、秀水河赴府;一,西渡鷂鷹河赴西豐;一,北赴西安;一,東北渡柳樹河,過黃泥河,赴吉林伊通。西豐繁,難。府西二百二十里。明,塔山左衛、罕達河衛,後屬葉赫部及哈達部。光緒二十八年,以大圍場西流水墾地之淗鹿,置縣隸府。縣境居威遠堡邊門外。達揚阿嶺在東南,清河所出,即哈達河,西入開原。南:扣河即瞻河,又南碾盤河,俱西入開原。東北:東遼河自西安入,北入吉林伊通,名赫爾蘇河。扣河上游有雙河鎮。官商路四:南由六馬架至老坡溝赴開原;西南由平嶺赴鐵嶺;由東南赴府及山城子;由東北赴吉林。西安繁,難。府西北百六十里。明,珠敦河衛、塔魯木衛,後屬葉赫部。光緒二十八年,分海龍屬之西圍場地,設治老虎嘴,置縣隸府。二十九年移治大興鎮。庫哷訥窩集在東,與東平分山脈。東:遼河導源窩集之轉心湖,西逕縣南,屈西北入西豐。左受渭津河、大小梨樹河,右受登杵、二道、頭道諸河,入遼河。北:楊樹河,西北入吉林。老虎嘴今名安吉鎮,在縣西北。官商路四:東由龍首山至東岡赴東平;南由梨樹社至望兒樓赴西豐;北由雙馬架至大臺房赴吉林伊通;又由仙人洞、溝嶺子至北廟子赴吉林。柳河沖,難。府西南一百二十里。明,建州衛地。光緒二十八年,分通化縣柳樹河縣丞地,置縣隸府。南:龍岡,與通化分山脈。一統河導源西南龍岡之金廠嶺,東北入府境。三統河導源西南龍岡之青溝子山,東流屈北入輝南境。縣治居一統河南。東:柳樹河,西流屈北入一統河。東北:窩集河,北入一統河。東:樣子哨,巡司,光緒三十二年置。官商路五:北渡一統河赴府;南由小堡赴通化;西由南山城子赴開原;西南由碗口溝赴興京;東由孟家店赴府。縣境東至吉林濛江。

輝南直隸:省東南六百八十里。明,輝發部。今北三十五里有輝發城。宣統元年,分海龍府東南八社,設治大肚川,置,直隸行省。移治謝家店。北:輝發城山,即聖音吉林峰。又北輝發江,自海龍合一統河入,東流,右受三統、黃泥、蛤螞、蛟河,入吉林。治居蛤螞河西,全境在輝發江南。西以窩集河、一統河與海龍界。東界吉林濛江。官商路四:西南由三間房場赴柳河;西北赴海龍府;東赴吉林濛江;東北由蛤螞河出海興社赴吉林磐石。

昌圖府:繁,疲,難。省東北二百四十里。明初置遼海衛於此,地名牛家莊,後屬福餘衛之科爾沁諸部。嘉慶十一年,以科爾沁左翼後博多勒噶臺王旗地,設昌圖額勒克理事通判。同治三年,改為昌圖遼海撫民同知。光緒三年,升府。廣二百八十里,袤二百九十里。北極高四十二度五十一分八秒。京師偏東七度四十二分三十五秒。領州一,縣三。府境居馬千總臺邊門外,無城。而遼河自遼源入,南入開原。南馬鬃河,北亮子河,俱西南入開原。又北昭蘇太河,左受條子河、蓮花泡河,西南入遼河。東北:八面城照磨,由梨樹城移駐。西南:同江口同知,宣統二年改經歷。同江口距遼河上游,商船薈萃。河流東徙,曲如懸瓠,光緒三十四年,挑河道取直,添築順水壩,逼河西行,以保商埠。舊設鋪司三:東北四面城、赬鷺樹入奉化;西北八棵樹,入康平。又道東南由永安堡至二道溝赴吉林;又由二道溝經伊通赴西豐。同江口商埠,中日約開。有東清鐵路。遼源州繁,難。府西北二百四十里。明屬福餘衛。光緒二十八年,分昌圖、康平、奉化地,設治鄭家屯,置州隸府。宣統元年三月,設分巡洮昌兵備道,駐州。東北有東西蛤拉巴山。內興安嶺山脈自烏珠穆沁旗東出,伏行蒙古平原中,至是特起二山。由是山脈行於東遼河外,至源為庫哷訥窩集,即長白山脈也。西遼河即西喇木倫河,導源克什克騰旗,新遼河即大布蘇圖河,導源札魯特旗,俱自科爾沁左翼中旗入,合流至三江口,東遼河自懷德入,西南流來匯,以下統名遼河,入昌圖。州治居西遼河西。有官商路六:西南張家窩鋪赴康平;北五道岡至新甸,赴吉林長春;東北閻陵窩鋪赴懷德;南白廟子赴府;西北下土臺赴洮南;西蒙古套力街赴博多勒噶臺王府。奉化繁,難。府東北一百四十里。明屬福餘衛。國初為科爾沁左翼中達爾罕王旗地,原名梨樹城。道光元年,設昌圖照磨。光緒三年,改置縣,隸府。東北:青石嶺、太平山。西北:二龍山。西:黑牛山、蘑菇山。南條子河,北昭蘇太河,俱西流入府。東遼河,自吉林伊通州赫爾蘇邊門入,北流,屈西南入遼河。環縣境東、北、西三面,稱遼河套。其右岸為懷德境。舊設鋪司二:東北小城子入懷德;東南四平街入府。又有道由縣東五里堡至翟家店,達赫爾蘇門,赴吉林伊通。有東清鐵路。懷德繁,難。府東北三百里。明屬福餘衛。國朝為科爾沁左翼中旗地。舊名八家鎮,初屬開原,同治五年劃歸昌圖,設分防經歷。光緒三年,改置縣,隸府。西以東遼河與奉化界。東界吉林。西北:哈拉巴、楊樹嶺、大青山。西南:團山。南:萬靈。東南:白龍駒、回龍山。夾城南北三道岡水,南香水河,西北朝陽山水,皆西入東遼河。東南:新開河,北入吉林長春。舊設鋪司三:西八屋、西南朝陽坡,皆入奉化;東南大嶺,接吉林長春。又有道由縣東南拉拉屯至鳳凰坡,赴吉林伊通;由縣西北小邊經八屋至邊壕赴遼源。有東清鐵路。康平繁,難。府西一百二十里。明屬福餘衛。國朝為科爾沁左翼後旗地。舊名康家屯,光緒三年移八家鎮經歷治此。六年,析科爾沁左翼中、後二旗南境,前賓圖王旗東境,改置縣,隸府。無城。南北巴虎山在西南。遼河自遼源入,其遼河岔入為老背河。右合公河,會牛莽牛河注之,入開原。西:秀水河自科爾沁左翼後旗入,南入法庫。西南:後新秋,主簿舊駐鄭家屯,二十八年移駐。舊設鋪司三:東南吳家店入開原;東小塔子入府;北太平街接科爾沁。又有道由縣西哈拉沁屯赴賓圖王府;迤西至青溝達熱河綏東;由縣北六家子赴達爾罕王府。

洮南府:繁,疲,難。省北九百里。明屬泰寧衛。光緒三十年,以科爾沁右翼前札薩克圖王旗墾地,設治雙流鎮,置府。領縣五。西北:敖牛山、野馬圖山,皆內興安嶺東南迤出支山,過此山脈伏行。洮爾河導源烏珠穆沁旗索嶽爾濟山,南流,逕本旗郡王府東流;交流河導源右翼中旗,左合那金河,自西來匯,東流入靖安。府治當匯口之南少西,地勢平原,河泡錯列。西北:乾安鎮,西與右翼中旗毗連,亦系烏珠穆沁往來大道。有照磨,光緒三十三年置。官商路七:一,府北八仙套海赴本旗郡王府;一,府北德勒順昭至高平鎮赴靖安;一,府西抱林昭至海廟西赴熱河綏東;一,府西五家子赴右翼中旗;一,府南叉乾他拉赴開通;一,府東英哥窩棚赴右翼後旗;一,府東金山堡至報馬吐岡赴安廣。舊有蒙古站曰奎遜布喇克,在府西。靖安繁,疲,難。府東北九十里。古東室韋地。明屬泰寧衛。光緒三十年,以右翼前旗墾地置縣隸府。西北:七十七嶺。南:洮爾河自府入,東屈,東北流,入鎮東。官商路三:一,南英哥套赴府;一,東北赴黑龍江;一,東南撮倫坡達右翼後旗赴吉林。舊有蒙古站諾木齊伯里額爾格,在縣西北。開通繁,疲,難。府南一百四十里。明屬泰寧衛。光緒三十年,以右翼前旗墾地,設治哈拉烏蘇,移治七井子,置縣隸府。地皆平原井泉,無山水。縣治當洮遼驛路之東,由巴彥昭北行六十里至縣治。又北行百里至叉乾他拉入府境。設有文報站四。又由巴彥昭南行,歷達爾罕王旗至遼源,為洮遼驛路,設站。惟中經達爾罕旗二百餘里荒地。宣統元年,始勘放旗界站荒,沿站兩旁各劃十里墾放,以利交通。又道由縣東南巷鷹溝出境,經郭爾羅斯前旗,直達吉林農安之新集廠。安廣沖,繁,疲,難。府東南百六十里。明屬泰寧衛。光緒三十一年,以科爾沁右翼後鎮國公旗墾地,設治解家窩堡,置縣隸府。北:太平嶺。南:長嶺。西:朝陽山。東北:沙坨子。東南:雙龍山、大黑山。山皆無木石。洮爾河自府入,受黃花碩泊水,東北流,屈東南,入黑龍江大賚,其北岸為鎮東境。官商路六:西包馬圖赴府;西南赴開通;西北六家子赴河北鎮國公本旗;東北托托寺赴黑龍江;東王賚屯赴黑龍江大賚;東南大榆樹入郭爾羅斯前旗赴吉林農安。醴泉沖,繁。府西北一百八十里。古鮮卑地。明屬泰寧衛。宣統元年,以科爾沁右翼中圖什業圖王旗墾地,設治醴泉鎮,置縣隸府。北:茂改吐山。南:霍勒河,導源札魯特旗,曰哈古勒河,曰阿嚕坤都倫河,合流入本旗境,東南至縣。有開化鎮城基,光緒三十二年,與醴泉鎮同時勘定。官商路四:縣東羅窩棚歷青陽鎮赴府;北渡交流河達黑龍江景星鎮;南赴本旗親王府;西赴烏珠穆沁旗。舊有蒙古站曰希嫩果爾,曰三音地哈希,在縣東,南達喜峰口,即蒙古草地也。鎮東府東北二百里。古東室韋地。明屬泰寧衛。宣統二年,以科爾沁右翼後鎮國公旗北段墾地,設治南叉干撓,置縣隸府。南:洮爾河自靖安入,東北流,屈東南,入黑龍江大賚會嫩江,所謂「與那河合」也。官商路四:西南薛家店赴府;南金圈窩鋪渡洮爾河赴安廣;西麻力洪茅頭赴靖安;東北利順昭赴黑龍江之大來氣鎮。縣西北舊有蒙古站哈沙圖。


\end{pinyinscope}