\article{列傳二百八十三}

\begin{pinyinscope}
忠義十

劉錫祺阮榮發程彬桂廕存厚榮濬錫楨等張景良倭和布

周飛鵬松興松俊等宗室德祜彭毓嵩楊調元楊宜瀚陳問紳

德銳皮潤璞榮麟等張毅喜明阿爾精額斌恆等譚振德熊國斌

陳政詩陸敘釗齊世名等羅長崟曹銘章慶徐昭益曹彬孫汪承第

吳以剛陶家琦等奎榮王毓江劉駿堂鍾麟何永清沈瀛申錫綬等

世增石家銘琦璘毛汝霖胡國瑞張舜琴鍾麟同範鍾岳等孔繁琴

王振畿張嘉鈺陳兆棠馮汝楨何承珍白如鏡何培清黃兆熊

張德潤張振德舒志來秀劉念慈李秉鈞王榮綬定煊長瑞

巴揚阿等王有宏何師程黃凱臣戚從雲盛成哈郎阿南山培秀等

桂城延浩文蔚佘世寬等高謙黃為熊文海趙翰階貴林量海等

額特精額文榮等玉潤勞謙光吉升張程九王文域譚鳳亭等

張傳楷孫文楷王乘龍趙彞鼎施偉李澤霖胡穆林更夫某

梁濟簡純澤王國維

劉錫祺,字佩之,直隸天津人。畢業將弁學堂。第八鎮成立,為正參謀官。光緒二十二年,南、北陸軍於河間會操,籌度有勞,加正參領銜。

宣統三年夏秋間,革命黨人之在武漢者數被破獲,總督瑞澂恣意捕殺,人人危懼。八月十九日,武昌變作,始僅工程營數十人,他軍無應者。瑞澂遽逃兵監,省垣無主。於是各營皆起,擁立都督黎元洪,稱軍政府,獨立。錫祺時方赴沙市,以二十六日回武昌,各營爭往迎謁,趣入見元洪,錫祺正色曰:「國家歲糜巨帑練兵,原期君等為國干城,以禦外侮。奈何一旦為人煽惑,遽爾發難乎?禍機一動,將無已時!吾不能為君等所為。」眾聞之譁怒,即於坐中擊殺之。事聞,照協都統例從優賜恤。

發難時,督隊官阮榮發出阻,眾遽擊斃之。榮發邑里未詳。

程彬,字筱竹,江西樂安人。時署占魚司巡檢。署在省城南,十九夜見城外火起,彬馳往救護,至望山門外正街,突遇陸軍砲隊入城,皆袖纏白布,彬大駭,厲聲問曰:「汝等反耶!此何為者?」眾舉槍擬之,彬益前致詰,遂遇害。以一巡檢犯難而死,人皆哀而壯之。

桂廕,字輯五,姓嵩佳氏,滿洲鑲藍旗文生。由刑部郎中、軍機章京外擢施南府知府,調安陸,以治堤盡力名。安陸為襄樊門戶,府城故無兵。武昌變聞,圖守計,並牒道請兵,已而旁郡德安、荊州皆陷。十月初五日,鄖陽兵驟變,圍府署,劫印信。桂廕攜妻富察氏趨入文廟,夫婦同縊崇聖殿中,死,衣帶中書有「虛生一世,不能報國安民」數語。臨殉難時,顧謂僕曰:「葬我必北面!」官紳流涕斂之,葬城內陽春臺側。

存厚,字寬甫,正白旗監生。由內務府郎中選宜昌府,調辦襄陽榷局。宣統三年十月,郡中黨人應武昌,存厚揮家人出避,曰:「吾嗣不絕,死無憾!」局丁旋縶存厚,擁至北門校場戕之,幼子被搜獲,驚死。

榮濬,字心川,荊州駐防,蒙古鑲藍旗人。光緒三十年進士,用知縣,發湖北,補天門縣。操行不茍。變作後,荊防旗人有自武昌脫歸者,道天門,語狀,且為榮濬危。榮濬以死自誓,集紳耆、練民團為保衛計。無何,黨眾來攻,遂被害。記名驍騎校炳安同死,僕成松亦殉焉。

同時殉國難者,為候補縣丞錫楨,姓汪氏,漢軍人。充沙市警察官,盡室被殲。簰洲司巡檢方祖楨,安徽桐城人。鄂軍頭目將入湘,道簰洲,土豪某夙銜祖楨,嗾人殺之江岸石花街。巡檢王萃奎,江西豐城人。佐穀城縣,治盜有聲。襄陽既變,屬邑響應,盜渠縶萃奎及一子、一孫殺之。蘄州吏目駱兆綸,字文卿,湖南江華人。亂作,知州亡去,州人以綸習吏事,遮留之。綸請送母至漢口乃還,至治所,以全省皆陷,事無可為,憤絕投河死。又襄陽府某縣典史,當變作時,晨起跽廨門外,過者叩頭要入,得負販者十八人。出銀幣二百枚分遺之,曰:「平生所積止此!城破義不得活,請助我殺敵。」眾感其義,各攜肩輿長木及負擔之具,噪而出。變軍方踞府署,出不意擊死者數人,俄而排槍起,某與十八人者皆死。候補知府張曾疇,字望屺,江蘇無錫人。以書跡似總督張之洞,為之洞所賞,充文案有年,榷漢陽車站貨捐。戰事起,避上海,仇者誣為挾貲遁,脅還漢口,會計出入悉符合,得還。黨人適同舟,面辱之,捽其冠,遽投江死。候補知縣聯森,字植三,蒙古鑲紅旗人,隸荊州駐防。光緒八年舉人,挑知縣,發廣東,改湖北。屢榷釐捐,能恤商。九月,道出漢陽,變兵爭索金,慷慨大罵,遇害。子寶焯、兄子寶明從死。

張景良,湖北人。將弁學堂畢業生。游學日本歸,充湖北新軍標統。武昌既擁立都督,景良慷慨說之曰:「朝廷已宣布立憲,不宜更言革命。公受知遇久,諸將惟公命是聽,盍三思之?」變軍怒,拘景良署中。時清兵攻漢陽,景良陽請赴前敵,以妻子為質,乃委充司令官。九月初六日之戰,清兵卻,景良率砲隊出,臨發,砲予彈一枚,槍予彈一排,甫戰彈罄,景良遽大聲促軍退,眾不知所為,遂大潰,死者枕藉,清兵得進屯大智門。後廉其故,殺景良,臨刑夷然,仰天大言曰:「某今日乃不負大清矣!」

倭和布,字清泉,滿洲正白旗人。家世以武功顯,獨兼肄文學。起家護軍藍翎長,歷二等侍衛。拳匪之變,歐人僑京者多被戕,倭和布護之甚至。或詰之,曰:「外人僑吾國,勝之不武。無故與八國構釁,敗將不國,吾敢重召亂乎?」旋扈駕西行,家人初以為戰死。出為湖北均光營參將,擢施南協副將。川寇陷黔江,率所部赴援,獲其渠,斬以徇。武昌變作,鄂將屯宜昌者應之,倭和布時以裁缺寓宜城,被執,勸降不應,以得死為幸,遂槍殺之。

周飛鵬,字翔千,江西新建人。由武舉人累官都司,充湖北襄防馬隊管帶,駐老河口。鄂軍變,縣無賴出獄囚,糾水師營謀變,飛鵬持不可,出佩刀與鬥。槍及馬腹,墜馬,槍繼至,洞胸死。裁缺荊州城守營參將玉萼,亦遇難死之。

松興,蒙古正白旗人,荊州駐防。以諸生改武職,累官協領,記名副都統,充常備軍統領。變兵入城,被縶入鄂,叱使跪,曰:「吾朝廷大吏,城不保,義當死。頭可斷,膝不可屈!」士紳三十餘人馳救之,已及於難。其戚善吉、庖人福全皆從死。

駐防之同時殉難者,在武昌有兵備處提調松俊,守楚望臺火藥庫,變兵攻庫,力戰死。三十標隊官重光,守籓庫,變兵掠取庫儲,重光大呼:「保全名譽!」被槍死。妻趙,子春年、長年、寶年同日殉。四十一標排長色德本,三十標副軍需官寶善,二十九標排長德齡、隊官東良、排長德培,均戰死。前泰寧鎮右營都司榮錦就養子書記官朗察所,拔所佩劍自裁,侄迎吉及朗察舉室自焚。驍騎校哲森以領軍械至省,自刺其腹死。陸軍小學教習舉人迎禧,平時於古人之當死不死者輒痛詆之。變作時,衣冠坐講堂,及難。副軍需官榮勛仰藥死,子額勒登額、穆貞額殉之。第八鎮執事官錦章謀召同志抵御,中途遇害,父榮喜即自盡。司書生恩特亨、雲騎尉榮清、排長倉生光均大罵不屈死。文生楚俊在督署,金培、榮森,司書生鈺壽、訥爾赫圖均在省與難。

在荊州者,聯長澤麟憤全省盡陷,發槍斃數人,被害。協領志寬,排長額哲蘇、依成額、關斌魁,恩騎尉扎勒杭阿,隊官王榮耀,均亡於陣。生員秋培城陷自盡。防禦多瑞仰藥死。記名驍騎校金霖嘗作萬言書,以旗制不良,力主變更,人多笑之。及變作,發槍自擊死。又知縣用模範講習所所員根壽於羊樓峒,文生陸營司書生定海於施南府,均死之。

其後死於江寧者,為生員占先、文志、恩昌,武生林福。死鎮江者,為生員榮有;副將赫成額則隨端方在資州,兵變遇害;軍諮府軍諮使良弼,自有傳。

宗室德祜,字受之,隸正藍旗,不詳其支派。宣統二年,由禮部儀制司郎中選授鳳翔府知府。三年九月,西安兵變,德祜聞警,即與知縣彭毓嵩籌備。有湘人劉瑞麟,以武職留陜,委令募團勇,與參將王某分任防守。初七夜,匪徒假民軍名號,驟集千餘人,攻府城。德祜與毓嵩登陴,激勵士卒拒守。至天明,匪氣奪,將引去,以有內應者,城遽陷。左右擁德祜走避,德祜曰:「此吾死所,尚何避為?」匪蜂至,呼曰:「知府滿人,且宗室,宜速殺之!」遂遇害。又殺其幼子二人。王參將,同州人。城破,與匪相搏,憤而自戕,舁至署乃死,名未詳。

毓嵩,字籛孫,四川宜賓人。由舉人官教諭。學政疏薦,用知縣,選陜西鳳翔,勤聽斷,時方興小學,必令讀經。城陷後,毓嵩解束帶自經,遇救未絕,乃從容出堂皇北向跪,起語眾曰:「吾有死耳,任爾等為之。」匪擁至署西北神祠,以白布纏其頸,毓嵩怒詈,遂被戕,梟其首去,年六十有二。子龢年,奔赴死所,為匪眾所逐,投井死。

楊調元,字龢甫,貴州貴築人。光緒二年進士,授戶部主事。丁父憂歸,服除,以母老不赴官。終母喪,乃入都,改知縣,選陜西紫陽縣。於秦境為極南,居萬山中,為楚、蜀會匪出沒地。以緝捕有名,遷長安,權華陰。疏濬河渠,復民田五萬畝。調華州,以獄事忤上官,解任。已,復補咸陽,擢華州,署富平、渭南等縣。

其署渭南,以宣統三年正月。先是,南方革命軍數起皆不得志,始改計結學生之隸新軍籍者,潛伏待應。陜軍勢弱,則又結會匪以厚其力。八月十九日,鄂變起,九月朔,陜變繼作。諸守令多委印去,調元獨謂守土吏當與城存亡,亟召紳民議守御。渭南北有號「刀客」者,殺人尋仇,數犯法,至是感調元義,爭效命,集者萬餘人,檄邑紳武進士韓有書統之。時鄰匪蜂起,渭南以守禦嚴,不能入。

臨潼武生張士原揚言受軍政府命,驟率眾徇城下,調元登陴語之曰:「吏所職,保民耳。無如所犯,則釋兵入見。必怙威圖一逞,則視力所極,當與決生死。」士原知不可侮,遂屏騎入廨,以議貸餉事,語侵調元。調元至是,躑躅廨後園中,仰天嘆曰:「吾誼應死,所以委曲遷就,欲脫吾民兵禍而後歸死耳。卼辱至此,尚可一息偷生乎?」遂投井死。民聞調元殉難,執士原磔之,並殺陜都督所派副統領及同黨數十人以徇。有書時出擊他盜,馳歸,葬調元畢家原。調元通古學,工詩文,有訓纂堂集、說文解字均譜等書。所作篆書,人尤寶之。

楊宜瀚,字吟海,四川成都人。兄宜治,官太常寺卿。宜瀚好學,嘗入烏魯木齊都統金順幕中,治軍書,知名,保知縣。中順天鄉試舉人,以知縣發陜西,補興平,調寶雞。以經術飾吏事,與調元齊名。署華州知州,民軍圍署索餉,以威劫入甘露寺中,有以事系獄賴宜瀚平反得出者,約護宜瀚出。入夜,宜瀚獨至神殿自經死。遺書親友,意思安閒,謂已得死所,無可哀者。

陳問紳,字子仲,湖北安陸人。入貲為縣令,發陜西,權甘泉,以能緝捕稱。調白水,邑刀匪素難治,武昌變起,乘間應之,糾眾攻城。時問紳初受任,一切無備,乃集紳民告之以不忍以一人故致全境糜爛,遽出城,大罵不屈死。妻吳,以護印不與,同被戕,並斃傭婦某。

德銳,滿洲人。官秦中久,歷長安、三原諸縣,有循聲。西安變作,八旗人多被禍,德銳時居會城,變兵突入,語德銳:「公得民心,我曹不忍死公,請速出城!」答曰:「感汝等意,然予滿人也,不忍獨生,刃加予頸可也。」遽起奪刀自刺死,妻、子均自裁。

皮潤璞,湖北大冶人。官榆林縣典史,有強項稱。變作,匪徒縛榆林鎮總兵張某、中營游擊瑞某送獄,潤璞斥之。群怒,以利刃相擬,不為動,紛加以刃,分股體為數段。妻聞訊,即以身殉。榆林守備穆克精額同時死,闔門自盡。

時殉難者,候補道榮麟,字仲文,滿洲人。變作,方榷白河釐金,全家投井死。候補知州張存善,字次章。榷鳳翔鹽釐,死事所。候補直隸州知州寶坪,字子鈞,西安駐防。一門殉難者七人。候補同知廣啟,字少漁;候補通判嚴濟,字寬甫:均滿洲人,與於難。

張毅,字仁府,直隸天津人。父夢元,官福建布政使,護臺灣巡撫,以清廉著稱,卒,贈太子少保。毅由廕生官部曹,改道員,分山西,奏調陜西,授甘涼道。宣統三年六月,擢安徽提法使。八月,自隴入秦,將入覲,九月,抵乾州,變作,道梗。變軍偵知之,請為參謀官,斥之,攖眾怒,羈留不得脫。會疾作,州人知毅賢,言於變軍,乃出就醫。毅念惟一死可自完,十一月初十日夜加丑,乘間投井死。毅無官守,中道遘變,卒完大節,世尤多之。

喜明,字哲臣,西安駐防。舉人。宣統三年九月,民軍猝起,攻旗營,將軍文瑞督戰,喜明領兵百餘人,獨樹一幟,誓以書生效死。戰不利,歸告母曰:「吾屬死不免。」母曰:「婦女以潔身為重,可受辱乎?」帥子婦二、幼孫一,投井死。喜明有三女匿鄰廟中,走入手刃之,蘸血書壁曰:「喜哲臣三女死於此。」還至家,縱火自焚死。

附生春祥,素端謹。聞變後,語兄若弟曰:「城破家必亡,自古全家盡節,有光史冊,吾原死矣。」則皆應曰:「諾!」城陷,聞砲聲近,曰:「可矣!」遂偕兄、弟、妻、子輩十餘口焚死,無一免者。

直隸州州判阿爾精額,榷釐金於方計堡,受代還,道咸陽,變軍將劫之,為之語曰:「吾當未亂時,志欲以忠報國,敢偷活耶?」義之,不加害。乃入邸舍,肅衣冠,北向自刎死。妻張氏,即吞金以殉。

城破時陣亡者,為協領斌恆、恩瑞、存福、培基,佐領貴升、特克什肯、慶喜、巴克三圖、恆秀、瑞明、額哲本、達朗阿、興智、恩壽、玉祥、西拉本、奇徹亨、恩撤亨,防禦存喜、存升、恩成、林福、色清額、平升、胡圖靈額、惠文、鶴齡、奇巽、蘇克敦、訥拉春、惠源、呢克通阿、哲爾精額、惠祥,驍騎校奎亮、林啟、啟弟正目林璋、都倫太、景文太、薩立善、文昭、伊吉斯琿、智厚、惠慶、惠啟,副官惠璋、鹽大使文煥,舉人惠斌,生員金常,武舉人德森布,騎都尉昌廣、益光,雲騎尉俊亮、和瑞、松善、特伸布、富海、勝春、海亮、多鑾太、達林、和順、忠云、玉恆、培文、存祿、倭什琿、鳳玉、惠撤亨,恩騎尉培緒、鳳山、恩瑞、奎德、貴成、錫齡、崇喜、倭仁額。徇難者,為佐領圖切琿,候補直隸州知州寶坪,直隸州州同俊興。候補知縣德銳自刺死,妻、子同殉。防禦多英,與長子舉人奎成率妻、女等投井死,次子生員奎章,伏井慟哭從死,族弟奎斌、奎莊皆死之。巡官惠祥率警生守城,城陷,投井死,家屬從死者六人。從子廣興既殉,母趙氏,年六十餘,執短刀闖入民軍,欲殺敵,尋自刎死。生員音德本走多公祠自經死,弟領催額哲亨城陷死。傷亡者,佐領圖們布、善印、全瑞。

旗兵之死於此役有名冊可稽者,凡千餘人,官弁兵丁之家屬遇害及自盡者尤眾。論者謂各省駐防,於辛亥國變,以西安死難為最烈且最多云。

譚振德,字子明,直隸天津人。始入武備學堂,調新建陸軍,派充山西四十三協協統。時山西僅陸軍一協,振德寬而有制,兵士親之。巡撫陸鍾琦履任之三月,武昌變作,陜西響應,召軍官議省防。振德與參議官姚鴻法建議接濟河南軍火,而以重兵助守潼關,鍾琦從之。遂於九月初七日發新軍一、二營子彈,令於翌日出蒲州,屯潼關,又令熊國斌帶第三營繼之。有構於一、二營者,謂熊營將於中途襲擊,適第二營管帶姚維籓以請棉衣未得為憾,聞之,憤激,謀變。明日,擁眾入省城,振德聞警,不及集兵衛,馳出遮道,對眾有所宣喻,維籓恐其撓眾心,舉槍斃之。遂趨撫署,鍾琦父子殉難,國斌以不肯附和亦被戕。鍾琦自有傳。

陳政詩,字詠笙,浙江仁和人。年十九,從湘軍西征,將軍穆圖善器之。從至奉天,充防營統領。光緒初,以知縣發山西,歷署州縣,以廉惠稱。以剿套匪功擢知府,以道員用。調浙,統嘉、湖水陸防軍,中讒罷。宣統元年,浙撫增韞奏言政詩軍紀嚴,有廉將風,詔復原官,再發山西。三年,統帶南路巡防隊,駐澤州,兼署澤州府。武昌變作,陜西響應,晉新軍亦變,戕巡撫。時政詩駐聞喜隘口鎮,遏變兵南趨。敵千人,脅土匪亦千人,以三百人屢敗之。方乘勝進擊,清廷詔命停戰,乃駐師絳州。敵勾結旁近土匪,勢復張。政詩以去絳則南路即與秦軍接,全晉將不保,誓死守。十一月二十日,敵攻城,城紳迎以入,政詩巷戰,力不支,被執,罵不絕口,剖心臠割死。弟敷詩,山西候補同知,隊官陳順興、劉占魁,均同時被難。

陸敘釗,字磐芝,順天大興人,原籍蕭山。少勵志節,從軍甘肅,保知縣。曾國荃撫山西,招入幕。擢直隸州,發山西,歷官州、縣凡十二,皆有聲。宰靈丘十年,尤得民。拳匪逼晉邊,大治鄉團,縣境晏然。宣統初,薦卓異,補河東監掣同知。太原變作,河東戒嚴。敘釗先以盛暑督濬鹽池致疾,至是疾甚,強起治防守事。秦軍來襲,晉軍應之,城陷,預服阿夫容膏,衣冠出堂皇,厲聲訶之,刃交下,無完膚,殞於座。子文治,聞變以毀卒,幼子亦為變兵所戕。時論謂與巡撫陸鍾琦父死忠、子死孝、鄉里同、氏族同、死難情事略同,推為奇烈。

時署陶林同知齊世名,天津人;岢嵐州知州奎彰,天鎮縣知縣世泰,均京旗人:先後均以兵變被戕。

羅長崟,字申田,湖南湘鄉人。光緒二十一年進士,改庶吉士,授編修。捐升道員,發江蘇,改四川。趙爾豐督川邊軍事,長崟在幕府多贊畫。宣統二年,簡駐藏左參贊,駐藏大臣聯豫以兵備任之。會閱新調川軍,以譁譟故,與協統鍾穎有隙,且覈鍾穎入藏軍資用浮冒,汰二十餘萬,鍾穎益嫉。三年五月,鍾穎率師征波密,戰屢挫。長崟馳往,奪其軍,得鍾穎失機狀。方激勵軍士規進取,而軍多會黨,氣囂甚,長崟馭將又嚴。及秋,內地變作,軍在藏者遽變,掠長崟私宅,波密軍繼之。縶長崟,屈辱之。偶得脫,自投崖下,未死,復曳之起,卒被戕。長崟之死,鍾穎實陰嗾之,後家人愬得實,置鍾穎於法。

曹銘,浙江上虞人。由諸生參四川總督劉秉璋幕,保知縣。歷治西藏夷務,著績,擢道員。巴塘邊亂番聚族十餘,陰為犄角。銘往解散,趙爾豐軍得深入勘定,功尤偉。署嘉定府,旋委石堤釐局。局介黔、楚間,往者皆中飽,銘絲毫不染。成都變作,匪眾入局,露刃逼索釐款,拒不應,中十餘創,垂絕乃委去。縣紳來視,以先事窖藏金指視之,點驗畢,遽卒。

章慶,字勤生,浙江會稽人。以通法家言游蜀,就幕職。為總督錫良等所器,保知縣,所至有聲。署劍州,倡捐萬金修文廟,擒巨逆王文朗,殲其黨九十餘人。調南部,河徙齧城,築長堤御之,城以完。調冕寧縣,有橋綰轂川南,毀於水,渡者以水駛多溺。慶制鐵梁數十丈,行旅稱便。普支夷擾境,慶廉威所被,濟以兵力,夷歸誠,出被掠者多人。補射洪,擢道員,在任候補。其任西昌也,值川省爭路事起,哥匪張國怔與裁缺千總黃義庫,偵知寧遠軍隊出防,城中無備,聯內匪襲城,慶督團眾御之,力竭死之。妻顏、猶子鏞及胥役、僕從同死者二十餘人。

徐昭益,字謙侯,浙江烏程人,咸豐季年殉難江蘇巡撫有壬從孫。隨父游蜀,以通法家言,佐治有聲,官知縣。宣統三年四月,攝威遠。同志會起,土匪附會名義,乘機報怨,四出剽掠。匪過境,昭益率團丁數百人出城解散,不從。匪以全力進偪,昭益念母老,居危城,命親丁護送還省。母臨行勉以大義,昭益泣涕受命,謂必不負母訓以辱先人,聞者皆為感動。九月十三日,匪薄城下,奸民為內應,團丁未訓練,猝戰遽潰。昭益乘騎亦受創,退而守城。其酋七八人入事,昭益厲聲問:「何不殺我?」其一酋突出利刃剚昭益腹,死之。

曹彬孫,字藹臣,順天武清人。以舉人勞績保知縣,發四川,權奉節,補開縣,未赴。七月,省城之爭路構釁,匪徒欲附同志軍起事,彬孫隨方禁阻,未敢逞。武昌發難,夔府響應。十月初十日,彬孫率團勇出巡,行至協臺壩,眾暴起,團勇先受煽,不戰而散。彬孫被執,割其首,置縣公案。警察長徐某,失其名,安慶人,同時被戕。

汪承第,字棣圃,江蘇太倉州人。由州學生佐學幕,以知縣發四川。寧遠夷亂,檄運兵械,至則知府黃承麟留辦剿撫事,充營務處,攝大足。川漢鐵路擬派租股,請歲減萬餘金,民困以紓。攝永川,解散公口秘密會,編練保甲,群盜屏跡。既受代,大吏仍以營務屬之。同志軍起,雙流境尤囂張,檄攝縣事,捕誅其尤者,人心少定。未幾,省城變作,土寇四起,以事至簇橋,被困,中槍死,十月二十日也。

吳以剛,字克潛,江蘇陽湖人。以知縣發四川,嘗權彭縣,縣銅廠通松潘、茂州夷地,素為盜藪,胥吏與通,十餘年不獲一犯。以剛乘冬至朝賀禮畢,馳馬自率隊擒之,未午,獲著名巨盜數人歸。宣統三年,以父憂,充重慶屬水路巡警提調。武昌變作,黨人謂以剛藏軍器,執而戕之。

時候補縣丞陶家琦在重慶,誣與以剛通謀,並遇害。候補知縣湖南文某,字晉巖,省城兵變,亦與於難。

奎榮,字聚五,滿洲正紅旗人,成都駐防。同治十三年繙譯進士,用知縣,發四川。奎榮篤嗜程、硃書,務躬行。性溫厚,與人語,惟恐傷之。始權南充,偶誤決一獄,屈者恚而得狂疾,聞之大戚,曰:「是予之罪也!」亟集兩曹,自引咎,平反之,自是聽斷益平。尤留意風化,在峨眉任,捐俸購儒先書,集書院諸生定課程,親為講授。歷犍為、彭水、慶符諸縣,所至勸學,一如在峨眉時。庚子前,以老告休,捐居宅為學校用。鐵路爭事起,總督趙爾豐持之急,奎榮太息,謂「損下益上失民心,蜀禍將自此始」,遂避地郊居。同志軍起,復遷入城。十月初四日,紳民迫總督交政權,又訛傳北京失守,遂託疾不食。或謂年已篤老,毋過自苦,奎榮慨然曰:「國事如此,吾輩尚偷生耶?」至十四日餓死,年八十。奎榮德望為蜀士推重,皆稱聚五先生。既殉節,益崇敬之。

王毓江,字襟山,安徽宿州人。父心忠,官江南總兵。毓江將家子,有材略,以知縣官江蘇,復以道員改發陜西,充兵備處總辦。餘誠格擢湘撫,檄調湖南,仍管兵備處事。長沙變,被執,罵不絕口,被亂兵所戕,到湘才九日。

同時死難者,候補游擊劉駿堂,湖南益陽人。光緒庚子,自立軍謀起漢上,事敗。駿堂時管帶院署衛隊,捕黨人最力,黨中尤恨之。至是自益陽拘至省城,徇於市,駿堂罵不絕聲,眾憤怒,叢擊斃之,並籍其家。

鍾麟,字書春,蒙古正白旗人。光緒二十九年進士,用知縣,發湖南,補瀏陽。攝永順,宣統二年,調嘉禾。省城難作,衡永郴桂道通令輸款,麟聞大慟。即集士紳謂曰:「麟蒞縣經歲,無德於民。今國亡城危,請諸君先殺麟以謝百姓。幸縣城不罹兵禍,死無所恨!」皆相顧錯愕,為好語慰之。九月二十一日,民軍圍縣署,鍾麟坐堂皇,屑金自盡。預伏火內室,妻邱氏熸焉。兩子及次子婦均遇難。

典史何永清,字澤溥,四川新津人。捐典史,發湖南,歷權州同、州吏目,屏絕規費,胥役畏之。嘗於除夕,有富商以金為壽,請系一負債者,永清曰:「除夕人皆歡聚,我拘之,非人情。我受金而使人一家皇皇,尤非此心所安。」峻拒之,其廉介類此。變作,誓與鍾麟死守。或有諗永清者,謂:「邑侯旗籍,民軍恐不相容,公幸自愛。有變,當奉公主縣事。」永清謝之,不為動。道令至,永清痛哭,懸印於肘,自經死。

沈瀛,字士登,江蘇吳縣人。嘗刲臂療母疾。以勞保知縣。嘗從湘撫吳大澂出關,事轉運,絲毫不自潤。累署武陵、長沙,奏擢知府。宣統二年春,長沙以米貴肇事,尾撫署,以瀛前任長沙得民心,復令攝任,緝匪賑貧,省城復安。三年八月,充營務處提調。新軍既變,黃忠浩被戕。瀛方出巡,新軍遮入諮議局,請為長沙守,不可;請仍宰長沙,又不可;錮諸室,令所親勸之,至泣下,瀛曰:「官大清州縣二十年,一朝背之,異日將何面目見人乎?」言已大哭。與前湘鄉知縣城固申錫綬同忍饑,以死節相勉。黨人知不可屈,擁二人出,罵不絕口,同死之。時長沙協都司熊得壽為人狙擊死。忠浩自有傳。

世增,字益之,為祖大壽後,隸正白旗漢軍。由生員入同文館,通法文。隨使英、俄諸國,歷保道員,加布政使銜。嘗譯西藏全圖、西伯利亞鐵路圖進呈。光緒三十二年,授寧紹臺道,外務部調丞參上行走。三十三年,授兗沂曹道,擢雲南按察使,調交涉使。宣統二年,擢布政使。三年七月,調甘肅,未行,而革命難作。時新簡滇籓未至,或諷世增速交替,可脫險,以「義不當茍免」辭之;事亟,法領事韋禮敦勸入領事館,又謝之。有懟世增者,則曰:「人孰無恥,安有一省大吏求庇外人者?得死,命也!」揮眷屬出,獨抱印不去。

九月十三日,兵變,世增夕懷印步謁總督李經羲,僕紀祥從,總督拒不見,乃歸。出手槍自擊,紀祥遽奪之,恚曰:「汝誤我!」軍隊突入,擁至講武堂,索金助餉,斥之。韋禮敦聞訊來視,且允代任餉銀二萬,變兵略無圖害意。夜半,槍聲作,楊某紿守兵,謂電請大兵且至,眾遂叩寢門,迫世增為都督,且以槍擬之,卒不應,排槍起,中五彈死。紀祥圖殉,眾義之,獲免。乃市薄槥斂。事上聞,贈巡撫,謚忠愍。

石家銘,字訂西,湖南湘潭人。治刑名,游滇,佐大府幕,凡邊防扼塞及通商各國科條章約靡不諳究。雲南自界連英、法領土,交涉尤繁,文書往復,惟家銘隨方應付,動中翾要,歷任總督皆倚重之,以縣丞累擢知府。宣統元年,補昭通,三年,調澂江,尋改開化。視事數月,審結滯獄數百起,多所平反。九月十五日,巡道所募新兵驟變,署中僅哨弁李世清率衛兵二十人守御,相持竟夜,子彈盡,仰藥不死;和金屑服之,又不死;乃令世清燃火油,以身投入,世清哭隨之,遂共焚死。世清,雲南人。

琦璘,滿洲鑲紅旗人。由部曹選授雲南澂江府知府,調補順寧,嚴正廉潔,對屬吏不少假借。省城兵變,正籌議集兵往剿。先是順寧縣令蕭貴祥疏脫要犯,援例上劾,貴祥銜之。至是結巡防營乘不備入城,貴祥假他事請琦璘至文昌廟會議,突起圍之。琦璘理喻不退,遂大罵,眾怒,遽開槍擊殺之。城中大亂,貴祥遁去。

毛汝霖,字澤卿,四川成都人。雲南候補知州。宣統三年,榷永昌府釐金,代行知府事。九月初六日,騰越兵變,永昌民大震,集民團守御。十二日,電傳省城變作,知事不可為,仰藥死。營官羅某,民軍入城,不屈被害,碎其尸。

胡國瑞,字瓊笙,湖南攸縣人。舉人。光緒二十九年,挑知縣,發雲南。始攝霑益知州,清積訟逾百。三十三年,署彌勒,縣多盜,易八令不能治,告戍將:「我行,君繼之,出不意,可擒也。」如其策,破賊巢,擒其渠斬之。明年大潦,蠲賑並舉,以循績上聞,被旨嘉獎。旋補江川,擢大關同知,皆未之任。時請修墓歸里,既受代矣,變作,遣家屬作,寓子書曰:「省垣不守,布政使被戕,餘無殉節者。臣子之義,萬古為昭。予雖無守土責,然實官也。俟北信,當死即死。」旬日後,訛傳京師破,明日有汲於署東井者,井上有雙履,往視之,則屹立井中死矣,背有遺書,曰:「自經不死,又復投井。」又書曰:「京師淪陷,用以身殉。達人不取,愚者終不失為愚。」於是縣吏棺斂之,邑人請封其井,題曰胡公井。

張舜琴,字竹軒,雲南石屏州人。舉人,選昆明縣訓導。講正學,尚名節,士皆敬之,擢順寧府教授。事繼母孝,迎養學舍,顏其堂曰「不冷」。監師範學校,人疑舜琴改平時宗旨,及觀其學規嚴肅,壹準禮法,皆翕服。外國教習亦僉曰:「張先生正人。」學使葉爾愷調充學務議紳。變作,有令剪發,即夕闔戶仰藥死。

鍾麟同,字建堂,山東濟寧州人。威海武備學堂畢業。治軍嚴整,累保道員。以嘗從軍龍州,調入滇,充陸軍第十九鎮統制官。宣統三年九月初九日,七十三標兵變,夜半,自北校場入城。麟同率衛隊扼五華山,手發機關砲,斃者數百,而七十四標駐巫家壩者應之,更迭戰山下。軍械局員陰與之合,移巨砲城上,攻五華,蟻附上,衛隊傷亡多,子彈亦盡,突圍轉戰,慨然曰:「身為統將,乃破壞至此,何面目生存耶?」以手槍自擊而僕,變軍碎其尸,剖心啖之。上聞,有「忠骸支解,慘不忍聞」之諭,謚忠壯。

同時死難者:輜重營管帶範鍾岳,字靜甫,直隸鹽山人,力戰死;七十二標標統羅鴻奎,直隸天津人,被執不屈死;七十四標副官張之泮,直隸河間人,遇毒死;七十二標第三營管帶張恩福,直隸靜海人,大罵被害。

孔繁琴,字韻笙,安徽合肥人。以文童投武衛軍,入武備學堂,畢業,充哨官。庚子拳亂,扈兩宮西狩,與兄繁錦殿後,奪回龍泉關,名以起。嘗調廣西幫辦紹字營,駐柳州。營本降匪改編,將調入城,疑而譁變,戕統軍,繁琴奮擊之,殲甚眾。又調廣東管帶巡防隊。惠州匪聲言欲投誠,脅紳求一見,繁琴盛服單騎往,覺有異,出匕首刺之,立斃。匪黨將致死,援者至,乃免。地方亦以匪首死,始不復擾。歷保知縣,宣統元年,調雲南,充蒙個防軍分統。以勞補靖邊同知,又以賑獎知府。民軍之變,獨率一營扼普雄。軍至,急與戰,死甚眾。已而左膝中彈傷,弁兵請退,怒,以槍擊之,所部遂潰,僅七人死守不去。民軍中有素重繁琴者說之,又以槍斃數人。乃大憤,發一槍,問:「降否?」曰:「不降。」累問之,答如故。至十三槍,乃中要害死。管帶張榮魁與繁琴本同學,是日亦戰死。榮魁亦安徽人。

王振畿,字化東,山東滕縣人。天津武備學堂畢業,充哨長,累擢至統領,改道員,入滇,總辦兵備處,治軍有節制。變作,欲墜城死,僅傷左股,遂被執。勸降不從,見害。

張嘉鈺,字武平,湖南鳳凰人。起世職,累官至總兵。宣統三年,署騰越鎮。武昌變

起,有自省遺嘉銍書諷其達時變者,嘉鈺謂:「我所知者,與城存亡而已,其他非我所能行,亦非所忍聞也。」未幾,騰越防軍起應民軍,九月初六日圍鎮署,出堂皇彈壓,兵猝入,被戕。

陳兆棠,字樹甘,湖南桂陽州人。父士傑,山東巡撫,自有傳。宣統三年,兆棠官惠州府知府。九月,粵中黨人起應武昌,總督張鳴岐遁香港,民軍遂踞省城,設軍政府。潮州鎮趙國賢自盡死,所統防軍擾亂,守、道、知縣皆逃。士民懼,堅留兆棠收撫防軍,部署未定,二十八日,民黨糾眾攻府署,火及宅門,左右挾兆棠出。民軍懸賞購執,令輸餉十萬貸死,兆棠曰:「死則死耳,安有鉅金助爾謀反?」眾怒,縛之柱,中十三槍乃絕。國賢自有傳。

馮汝楨,字萊云,浙江桐鄉人。以諸生捐知縣,發廣東。榷商讞獄,咸舉其職。宣統三年七月,攝西寧。廣州變起,黨軍闖縣署,脅汝楨懸白旗示歸順,持不可。俄而槍聲作,乃朝衣冠出大堂,眾爭前,槍刃交集,洞胸穿肋,斷右臂,死之。

何承珍,字性存,湖南湘潭人。少治說文學。光緒六年,學政陶方琦按臨長沙,以漯字為題,承珍徵引詳贍,文譽以起。光緒季年,廣東陸路提督秦炳直招入幕,於軍事多所贊畫。時提督駐惠州,以總稽查任之。宣統三年八月,革命軍起,惠及鄰境匪皆蠢動。聞營官有通敵者,密告炳直,而營務處劉殿元以全力讓主帥自任,否則偕死。承珍感其意,以首觸地謝之。亡何,餉匱薪米竭,援師不至,承珍以死自誓。城陷,夕歸私室,自書絕命時日,置衣帶中,並遺書誡子,自經死。炳直上聞,以「忠義可嘉」褒之。

白如鏡,字顯齋,隸鑲黃旗漢軍。由筆帖式補鑾儀衛官,出為興寧營都司。宣統元年,署潮州左營游擊,兵變不屈死。

何培清,字鏡亭,廣東歸善人。入提標,補千總。光緒三十四年,領連和防營,提督秦炳直才之。調博羅,剿羅桂幫匪,盡殲之。會鄂變,粵應之,民軍猝集,攻博羅。培清以三百人登陴守兩晝夜,敵不得逞。奸民開門迎民軍,執培清,不欲死之。甫出,猝遇羅桂餘黨,出不意,狙擊死之。

時又有黃兆熊者,名家玟,以字行,湖南湘潭人。久從秦炳直為惠安水師營哨官。博羅既失,民軍薄惠州,兆熊被調入城守,三日目不交睫。城陷,傳提督被害,悲愴不欲生。時全城搶攘,獨攜槍至城堞間,以足趾觸槍機,洞貫胸腹死。

張德潤,南雄人。以千總充香山巡防營管帶官。革軍入縣城,守南門力戰,援絕被執,殺之,投尸江中。嘉應州游擊柏某,時亦以兵變遇難。

張振德,並失其籍。廣西候補知府,充巡防隊統領。十月,潯州亂,率師至黃茅規進剿,眾寡不敵,中槍死。時南寧府知府攝思恩府舒志,亦以兵變死之。

來秀,字樂三,姓聶格里氏,滿洲鑲藍旗人。由繙譯生考取筆帖式,歷官刑部,屢決疑獄。充軍機章京。光緒三十三年,出知汀州府。大吏議加汀鹽價,力爭罷。武昌事起,福建響應,總督松壽殉難,全省無主。來秀在官多惠政,士紳憂來秀滿洲,為人指目,謁請護避汕頭,來秀以大義自矢,不之允。九月三十日,郡城驟懸白旗。來秀知事不可回,朝服坐大堂,北向叩頭,仰藥死。松壽自有傳。

劉念慈,字晉芝,湖北鍾祥人。由廩生選教諭,俸滿,以知縣發福建,補永安。福州既亂,土匪倚山險,聚眾數百人,念慈募勇督剿。匪負嵎抵抗,勇被槍死,念慈亦重傷,為匪擁去,索銀幣取贖。念慈即間遣人持絕命書歸,且曰:「慎毋來贖,以增羞貽累!」卒絕粒不食死。

李秉鈞,漢軍正白旗人。由謄錄敘知縣,選泰寧,有治聲。革命變作,慨然曰:「國亡與亡,義也!第縣治無官,民將失所。」召紳士議保衛,法既定,仰藥死。繼妻烏蘇氏亦抑藥殉之。

王榮綬,字笛青,湖南善化人。以軍功起家,官甘肅。光緒二十八年,改選連江縣知縣,嚴於捕緝,黨人莫敢留縣境。受代寓省城,被拘至軍政府,責以前事,抗辭不屈,被害。

定煊,福州駐防。諸生。有幹略,官佐領。武昌變起,將軍樸壽日料軍實、簡卒伍。旗民能勝兵者,皆授以兵,而任定煊為捷勝營管帶,日夕操練。防軍圖變,於九月十八日,揚言旗營將開砲洗城以懼眾。四鼓,砲聲隆起,分撲軍、督兩署。樸壽親督所部血戰兩晝夜,防禦長瑞、驍騎校巴揚阿主軍書,發憤從戰,相繼殞於陣。前者殭,後者繼,變軍不支,漸引卻。偵利槍巨業皆在於山,定煊從樸壽於二十日夕,短衣草屨,督死士襲山壘,深入,中砲死。

長瑞、巴揚阿均繙譯舉人,同隸駐防之前鋒森俊、蘇都里、達哈使、尚阿里,領催桂斌、慶銘,舉人松音,均陣亡。教員麟瑞,舉人裕彤與兄筆帖式裕豐,族兄哨官鑠欽額,均殉難。樸壽自有傳。

王有宏,字金波,直隸天津人。同治五年,投效銘軍,充兵目。自平定發、捻餘孽,與剿臺灣番法人攻臺灣諸役,均隨軍有功,擢至游擊。日本渝盟,奏調山海關辦防務。和議成,入江南防營,以緝梟匪勞,記名總兵。江蘇巡撫鹿傳霖器之,從入秦,扈從兩宮回鑾。尋為河南巡撫張人駿奏留,倚以練軍。人駿督兩廣,移督兩江,皆從。管江南緝捕營,兼統總督衛隊。宣統三年八月,湖北告變,檄統選鋒十營會提督張勛江防軍守江寧,嘗請率三千人赴滬守制造局,斷蘇、杭鐵道,未果。無何,江蘇巡撫程德全宣布獨立,率兵攻江寧,提督張勛與戰,頗勝,而變軍別出一支攻督署,有宏以機關砲擊卻之。十月初旬,德全以江浙聯軍至,麕集薄城,有宏馳出通濟門,以三百人戰。民軍以遠鏡測知有宏所在,發槍,子中左腹,猶植立,督軍士進擊,左右舁至醫院,乃絕。電聞,贈太子少保,謚壯武。

何師程,字雲門。由襲騎都尉擢副將,保總兵,補江南督標中軍。十月十二日,寧垣陷,自戕。

黃凱臣,本名彩,以字行,江蘇江都人。入徐寶山虎字營為哨官,敘功至游擊,以事去職,至賣茶自給。武昌變起,江寧將軍鐵良添募十營助城守,凱臣領其一。省城既陷,各營相約懸白旗,凱臣語所親曰:「城不守,而相率降附,吾實恥之!」聯軍至,橫刀大呼殺敵,馳入陣,被戕。

戚從雲,徐州人。由行伍官千總,隸江蘇巡防營,以能緝捕名。蘇、滬獨立時,從云率巡防一營駐黃渡,抵抗不從,遂為民軍所戕。

盛成,字挹軒,本荊州駐防。同治初,金陵克復,調江寧,由驍騎校累擢鑲黃旗佐領。民軍攻江寧,知城不可守,約知交城破各挈孥就火藥庫,謀同死。十月十一日,城破,有言繳械免死者,眾要盛成往,不應,率子婦趙,孫國瑞,女三,赴藥庫,攜酒痛飲,炷香以待炸發。

哈郎阿,字叔芬。素與盛成善,聞之,亦挈妻張,子成仁、成義,女一,往,同時熸焉。旁近旗民無老幼男婦,巨響一震,死不知數。

南山,字壽民。充貼寫,累擢防禦。初從將軍鐵良駐軍北極閣,城破,知同僚集都統署,馳入,言曰:「吾輩受國厚恩,今宜發天良,背城一戰。不濟,則以死繼之!」無應者。出召軍士語如前,亦無應者。恚甚,發槍自擊死。妻某,聞南山殉節,抱其子縱火自焚死。

培秀,字希賢。先以襁褓子授其戚,以阿芙蓉膏飲一女、一侄女,夫婦自焚死。

防禦松柏與妻、子、女八人,驍騎校恩鈞夫婦,副前鋒寶林全家,防禦長年,均自焚死。隸某旗洪某,聞變,先以妻女投官井,與同居劉永祥闔室舉火自焚。洪失其名,永祥,微者也。中學教習興發,約同營前鋒錦秀同投塘水死。小學校長富勒渾布,嘗以世濁獨清,誓與屈靈均為伍,有欲縛獻民軍者,躍入水,猶抗聲語曰:「吾今日遂吾志矣!」不受援,死。防禦嚴德海,驍騎校愛仁阿、榮生,均率妻、女、子、婦,千總色勒善夫婦,佐領廣照,世職關秀昆,相率投水死。防禦果仁布,城破自盡。世職鹿鳴,自經死。隊官汝霖、彭興,教練官恩錫,執事官魁穠,均以不屈被害。

陣亡者,為驍騎校趙金泉,教練官鵬興,排長海祥,砲隊官趙壽昌。被戕者,千總富有,世職金珍、祥泰、韓萬興、鴻錫、侯恩、俊卜、金海、永潮、韓萬富,文生衣吉斯渾。

凡旗兵戰死及眷屬與難見姓名者數百人。事定,掩埋叢塚凡十三處,其數不可稽。生員長明,以在杭州武備學堂肄業,為同學斫之死。

桂城,字仲籓,姓伊布杼克氏,蒙古鑲紅旗人,世京口駐防。由生員入武備學堂,考送日本振武、士官諸學校。入聯隊實習,調江寧為憲兵協軍校,管陸軍警察營。宣統三年九月,變作,遺妻、子槍令自裁;族人在軍者,咸勖以大義。時第九鎮統制徐紹楨駐秣陵關,往謁,知桂城不與同志也,拘荒祠中。新軍敗雨花臺,遷怒桂城,擁之出,中數槍死。後二年,補謚剛愍。

延浩,字子餘,蒙古鄂依羅特氏,漢姓文。既老,赤面白須,善騎射,如少年。官協領,以原品食俸。載穆殉節,默不語,具衣冠北面再拜,殭臥不食卒。

文蔚,字子貞,蒙古人。同治初,從將軍都興阿軍,累擢佐領。變作,家人勸出避,誓死不應。一夕,痛飲,哭不止,家人謂其醉也,中夜遽卒,蓋陰以毒物自戕矣,年八十。

協領佘世寬,驍騎校恩厚、同源,佐領春濤、延熙,防禦貴慶、延福,前鋒錦章、炳炎,領催東皋、德慶、延昌、松廷、三元、錫昌,雲騎尉良,師範學校校長崇樸,生員崇椿,同以絕食死。防禦吉瑞嘔血死。領催德霈自經死。前鋒鍾祥、達邦,領催慶耀、升奎、國能、殿倫、發昆,五品頂戴發元,生員穆都哩,同自經死。前鋒德尚,領催清泰,投江死。舉人恩沛,吞玻璃死。佐領榮康、德興、普亮,前鋒國棟、和庸及弟啟瑞,領催文光、延熙及弟延本、海春,恩騎尉延章、西登布,武舉人炳南,生員喜德,師範畢業生錫蕃,均受傷死。安徽縣丞壽餘及二子德興、德祚,同日遇害。其被調江寧者,排長國權、海靖、文馨、啟貞,與桂城同日死。排長炳升,守北城戰死。馬兵那康元,遇敵軍南門,搜軍械,不服,縛於樹,支解死。

高謙,字敬亭,湖南沅江人。同治季年,從左宗棠度隴司書記,以勞保縣丞,發安徽。光緒八年,宗棠督兩江,委謙淮北督銷分局,連任十有七年,鹽商饋遺皆不受;受代,典衣裘而行:商民頌之。三十三年,補安徽阜陽縣丞,清嚴不妄入民間一錢,知縣有過舉,輒陰為規正,民尤愛戴之。宣統三年九月二十五日,安慶變作,變兵旋入阜陽,左右勸謙引避,厲聲斥曰:「名位雖卑,大節不易,吾豈茍活者耶?」即夕飲鴆自盡。凌晨家人入視,則衣冠端坐,氣絕,面如生,年七十有四。民聞之,皆走哭,議立祠祀之,因亂未果。

黃為熊,字子祥,江西德化人。由舉人挑知縣,發浙江,署於潛,再署東陽。民好訟,積案千百,排日決事,民畏而感之。署蘭溪,除盜匪殆盡,益興學重農。治行上聞,被獎。省城變作,聞之欲自裁,翌日,聞訛言謂京師陷,大慟曰:「主憂臣辱,主辱臣死,何顏見地方人民耶?」亂民來奪縣印,正色諭之,不許,抱印自經。僚友趨救,氣已絕,面如生。

文海,字雲舫,漢軍鑲藍旗人。由拔貢生用知縣,發浙江,一攝長興,充勸業道科長。新軍變,入寓搜軍械,得洋槍,將縶之,文海發槍,斃一人,傷二人,出報其黨,被收,慷慨不屈,引頸受刃死。

趙翰階,字春亭,山西祁縣人。父受璧,奉天昌圖知府,有惠政。翰階隨侍邊塞,習騎射,以任俠重鄉里。拳匪之變,嘗乘垣斃其酋。增韞素與習,官浙江巡撫,令充衛隊管帶。杭垣變作,撫署被圍,率猶子趙錦標等突圍入護巡撫家屬,穴墻匿民舍。明日,聞巡撫為新軍所拘,往救之,挈錦標持手槍出,為變兵所執,曰:「我北方男子,豈畏死者!」遂與錦標同被害。

貴林,字翰香,滿洲正紅旗人,杭州駐防。官協領,與浙人士游,有賢名。浙兵變,駐防營猶抗拒,相持二日。浙人勸罷戰,招貴林出營議事垂定,有陷之者,謂旗營反覆不可信,且誣貴林署毒各坊巷井中,變軍誘之出,槍斃之。同出者,子量海,舉人存炳,佐領哈楚顯,同被戕。

額特精額,字蔚如,杭營正紅旗防禦,駐守武林門。辛亥九月十四夜,變兵強令開城,額特精額喝問:「何人?」以「革命黨」對,遂斥曰:「汝等狗也!我不死,城不能開。」獨持槍擊眾,眾環攻,慘剁死,暴尸數日,居近商民始殮之。

文榮,字如山,蒙古巴岳特氏。世襲雲騎尉。變兵攻旗營三日,堅不下,使來議和,合營官兵原效死力爭,將軍德濟遽遣貴林出許之,官兵皆擲槍軍署,痛哭散去。文榮憤不欲生,手書十六字曰:「杭營失守,忠義掃地。清流北向,是吾死所!」遂投河死。

迎喜,號壽芝,滿洲鑲白旗人。年八十餘矣,當議和時,詣軍署以死爭,大呼曰:「八旗受國恩三百年,今事至此,若輩猶欲靦顏偷生乎?」遂歸,閉戶自經死。

金海,正藍旗前鋒校。變兵架巨砲吳山,遙轟旗營,眾議啟城馳奪之,金海原從戰,聞議和,遂棄械於河,亦自經死。

希曾,正藍旗監生,前南昌知府盛元孫。變兵入營多劫殺,希曾斥之曰:「既議和矣,奈何猶為盜賊行?」眾怒,擊,竟剁尸如泥。時旗人皆自危,頗有無故被殺者,其姓名不能盡詳矣。

玉潤,漢軍鑲紅旗人。光緒季年,以鑾儀衛治儀正出補秦州營游擊。武昌事起,甘肅僻遠,總督長庚素持鎮靜,聞陜西擾亂,乃戒嚴。時有道員黃越者,宿與南方黨人通,充軍事參議,欲通陜中民軍謀獨立。以陜中民軍屢敗,乃陰引川軍入甘為援。玉潤偵知,日與守備習斌籌守御,以限於兵額,末由增募。是時南北款議成,甘、陜電斷不相聞。越於秦州各官獨憚玉潤忠鯁,壬子正月二十三日,遂率眾入城據各署局,而以兵圍游擊署。玉潤列隊出拒,身自督戰,終以兵少不敵,玉潤中槍,殞於陣。

勞謙光,字佩蘭,山東陽信人。少讀書,有用世志。入北洋武備學堂,畢業,山西設武備學堂,聘為教習,管帶馬隊營,捐知縣,遂官於晉。新政創始,若督練處、警察學堂並充提調官。數歲,移充北洋常備軍第三鎮參謀官,兼軍需官,擢第六鎮工程管帶官。武漢變起,率工程營赴前敵,築橋漢上,將以濟師,敵爭之力,砲子雨下,躬督視不卻,猝中砲死,時十月初六日。死而橋卒成,清師得渡,得漢陽,清廷主兵者遂有停戰之議。

吉升,字允中,滿洲鑲黃旗人。以學生官本旗前鋒,入海軍學習,積資充海籌兵艦幫帶官。湖北告警,海軍奉調赴援,至者兵艦十五艘、魚雷艇二艘。清軍攻漢陽,海軍助勢,而砲發多不命中。未幾,言煤罄,相率下駛。九月二十一日,海籌與海容、海琛三巡洋艦奉令離漢口,二十三日抵九江。時江西九江已響應武昌,海容、海琛遂相約懸白旗,停泊。海籌管帶喜昌不欲,邀吉升同遁,吉升★M4然涕下,曰:「國家經營海軍四十年,結果乃如是耶?」發憤投長江死。

張程九,字子澐,奉天臺安人。由歲貢考充盛京宗室學教習,任滿,以知縣用。宣統元年,選為奉天諮議局議員。三年九月,鄂變起,地方不逞之徒,假改革名義,狡然思逞,臺安齊某糾眾將起事,憚程九持正不敢發。程九聞警,至省謁總督趙爾巽,請派隊剿辦,免塗炭地方,爾巽允其請,並令回縣辦鄉團以資捍衛。程九歸,經縣西佛牛錄,為群賊所伺,設伏遇害。恤贈知府,賞世職。

王文域,字伯若,四川人。知山東樂安縣,辛亥冬,為變兵所戕;黑龍江海倫府巡防馬隊管帶官譚鳳亭,於十月陣亡:有旨優恤。伊犁將軍志銳被戕,僕呂順以樸誠著,臨難護主,同死之。從死者,武巡捕官劉從德,四川人;教練官春勛,京旗人。志銳自有傳。

張傳楷,字睿斌,直隸青縣諸生。充宗人府供事,敘勞得知州。革命軍起,舉朝震恐,自親貴達官而下,惟日以徙家入外人居留地為事。傳楷憤甚,詣都察院上說帖,請代奏,院官無在者,止院門,哭三日,無一官至。遜位詔下,拔所佩刀自戕死。自銘十六字曰:「成仁取義,孔、孟所垂。讀書明理,舍此何為!」

孫文楷,字模卿,山東益都人。同治癸酉舉人。潛心著述,尤精金石之學,以收藏貧其家,力耕自養,恆屢歲不入城市。有適野集、一笑集,皆詠田事詩也。遜位詔下,家人秘不以聞。經月,忽入城訪友歸,即仰藥自盡。將死,囑其子曰:「吾行吾所安耳,毋謂我死節也!」著有老學齋文集二卷,今吾吟草四卷,稽庵古印箋四卷,古錢譜等書。

王乘龍,字少枚,福建龍溪人。安貧好學,以歲貢生授經裏中。閩軍應武昌,乘龍感愴,彌日不食。翦發令下,長至謁宗祠,宗人勸之,乘龍不一語。入夕,乃潛設香案自經死,案上遺詩曰:「膚發千鈞重,綱常萬古新,毀形圖茍活,何以見君親!」年六十有一。

趙彞鼎,字煥文,江蘇江陰縣諸生。好程、硃之學。武昌變起,蘇撫程德全應之,憤痛絕食。十月初九日,出而不返,明日,家人跡至三賢祠樓,則衣冠北面懸梁間,氣絕矣。檢篋得遺筆千餘言,有曰:「我死合君臣之義,冀斯人不以我君為滿洲而漠視之!原國家大兵早至,反正者免,脅從者赦。」又曰:「我為國故不死於家,會文講學地,正欲以明人倫也。」

施偉,字卓齋,江蘇高淳縣諸生。傲岸絕俗,以兄喜譚新學,心非之。遜位詔下,大慟。壬子元日,具衣冠拜家祠,自書輓句祠壁,投塘水死。

李澤霖,字郇雨,廣東香山縣諸生。教授生徒,以小學、近思錄為日課。聞變,絕粒五日死。先手書「清處士李郇雨墓」七字授其子,俾刊墓道。且命二子毋入學校,毋出仕。

胡穆林,失其名,湖北江陵縣諸生。變作,上書荊州將軍議戰守事,將軍壯之。時電報被毀,具乞援牘,令賚以北行。至資福寺,為通敵之警察所偵,縶沙市敵營,訶之曰:「汝漢人,奚助滿人為?」穆林叱之,遇害。

杭州望江門有更夫某者,夜鳴鉦巡於市,變軍自城外入,方昧爽,猝見之,急鳴鉦大呼兵反,狂走向官署,冀警備。軍訶之不止,追及,槍擊之,立斃。

梁濟,字巨川,廣西臨桂人。父承光,卒官山西,貧不能歸,寓京師,喜讀戚繼光論兵書暨名臣奏議。光緒十一年,舉順天鄉試,時父執吳潘祖廕、濟寧孫毓汶皆貴,濟不求通。迨毓汶罷政,始一謁之。大挑二等,得教諭,改內閣中書,十餘年不遷。舉經濟特科,亦未赴。三十三年,京師巡警招理教養局,濟以總局處罪人,而收貧民於分局,更立小學課幼兒,俾分科習藝,設專所售之,費省而事集。由內閣侍讀署民政部主事,升員外郎。在部五年,未補缺。遜位詔下,辭職家居。明年,內務部總長一再邀之,卒不出。歲戊午,年六十,諸子謀為壽,止之,不可,避居城北隅彭氏宅。先期三日,昧爽,投凈業湖死,時十月初七日也。遺書萬餘言,惓惓者五事:曰民、曰官、曰兵、曰財、曰皇室,區畫甚備。予謚貞端。

有吳寶訓者,字梓箴,蒙古人。嘗為理籓院員外郎。素與濟游,聞濟死,痛哭。越日,亦投凈業湖死。

簡純澤,字廉靜,湖南長沙人。父桂馥。純澤生七歲,即出嗣伯父敬臨。敬臨以總兵從左宗棠軍攻金積堡叛回戰死,謚勇節,賜騎都尉世職。純澤自幼吐棄俗學,嘗入粵從西人習軍械制造法。桂馥客游新疆,久不歸,迄二十餘年無耗,純澤乃以襲職從度隴軍,欲遂出嘉峪關覓之。隴督以荒遠堅阻,而行文地方官搜訪,卒不能得,則大痛,謂他日不求死鄉里也。入陜西,為布政使升允所重。庚子,升允率師勤王,純澤與營官歐丙森從。遇夷兵正定,斬數百人。疾作,聞丙森戰死,力疾請戰,升允尼之,上書責升允,詞甚直。正定令將迎夷師入,下令軍中嚴陣待,夷懾之,解去。升允擢巡撫,檄管武備學堂,兼領新軍,後復檄充新軍教練官。會后撫以貪黷聞,非門金不得通,積二歲不往。又與道員王毓江議軍事不協,謝歸里。國變後,居數年,悲吒不解。丙辰夏,北行之京師,旋客天津。後一年至煙臺,游煙霞洞,去之威海,投海死。獲其尸,有自書絕命詞,以樹墓碣鐫「大清遺民」四大字為獲尸者告,感其義,斂而葬諸海濱,且立碣焉。

王國維,字靜安,浙江海寧州諸生。少以文名。年弱冠,適時論謀變法自強,即習東文,兼歐洲英、德各國文,並至日本求學。通農學及哲學、心理、論理等學。調學部,充圖書館、編譯名詞館協修。辛亥後,攜家東渡,乃專研國學。謂:「尼山之學在信古,今人則信今而疑古,變本加厲,橫流不返。」遂專以反經信古為己任。著述甚多,擷其精粹為觀堂集林二十卷。返國十年,以教授自給。壬戌冬,前陜甘總督升允薦入南書房,食五品俸,屢言事,皆褒許。甲子冬,遇變,國維誓死殉。駕移天津,丁卯春夏間,時局益危,國維悲憤不自制,於五月初三日,自沉於頤和園之昆明湖。家人於衣帶中得遺墨,自明死志,曰「五十之年,祗欠一死!經此世變,義無再辱」雲云。謚忠愨。海內外人士,知與不知,莫不重之。


\end{pinyinscope}