\article{列傳一百八十四}

\begin{pinyinscope}
陸建瀛楊文定青麟崇綸何桂清

陸建瀛,字立夫,湖北沔陽人。道光二年進士,選庶吉士,授編修,直上書房,洊遷中允。大考擢侍講,轉侍讀。二十年,出為直隸天津道,累擢布政使。時英吉利擾浙江,沿海戒嚴,徵西北兵聚畿輔,建瀛供防軍,處善後,皆應機宜。所歷有名績。

二十六年,擢雲南巡撫,俄調江蘇。先是,南漕缺額,部議設局江蘇,官民捐米運京以裕倉儲。當陶澍撫蘇,即以漕河費鉅病國,議行海運,官吏爭撓之,暫行輒罷。至是建瀛與兩江總督壁昌主海運甚力,合言其便,議蘇州、松江、太倉白糧改由海運,從之。後復推至常、鎮諸府。二十九年,廷臣會議南漕改折,建瀛與總督李星沅極言其窒礙,事遂不行。

擢兩江總督。值大水,民饑,招徠米商,籌議撫恤,並疏消積水,請籌撥帑一百五十萬備賑。吳城六堡河決阻運,命偕侍郎福濟往勘,疏陳通籌湖、河大勢,添塘避徬,對壩逼溜,攻刷海口各事宜,並如議行。淮鹽積敝,自陶澍創改淮北為票鹽,稍稍蘇息;而淮南擅鹽利久,官吏衣食於鹽商,無肯議改者,建瀛悉其弊。會淮南鹽大火於武昌,官商折閱數百萬,課大虧,引滯庫絀。三十年,乃疏請立限清查運庫,並統籌淮南大局,改訂新章十條,務在以輕本敵私,力裁繁文浮費。鴻臚寺少卿劉良駒亦請變通淮南舊章,仿淮北行票法,與建瀛所議同。方施行矣,而給事中曹履泰奏請復根窩舊制,御史周炳鑒言淮南改票不便,並下建瀛議。覆疏辨駁詳至,文宗韙之,詔綜斡全局,除弊興利,以裨國計。建瀛議於揚州設局收納,以清運署需索之源;於九江等處驗發,以清楚西岸費之源。正雜錢糧並納,則課額不虧;新舊商販一體,則引額無缺。灶私場私,專責江南;江私鄰私,兼責各省;而以徠商販,積帑賦,自總其成。由是奪官吏中飽歲百餘萬,惎謗叢作,建瀛銳自發舒,不之恤。朝廷信任益專,命有掣肘撓法者罪之。湖北鹽道鄒之玉沿用整輪,江西鹽道慶云強索月給,湖北同知勞光泰作移岸三論,刊板傳播,並劾罷之。

咸豐元年,河決豐北,命建瀛往勘,奏請以工代賑,偕南河總督楊以增督工。二年,以盛漲停工,降四品頂戴。

是年秋,粵匪洪秀全犯湖南,越洞庭而北,勢張甚。建瀛猶在豐工,疏上戰守事宜,文宗嘉之,諭以審度軍情,如須親往,可速籌方略,不遙制。既而漢陽、武昌相繼陷。十二月,復建瀛頭品頂戴,授欽差大臣,督師赴九江上游扼守。建瀛由工次還江寧,徵調倉猝。三年正月,賊棄武昌,蔽江東下,建瀛欲行,或謂賊鋒銳難驟當,建瀛尚輕之,檄壽春鎮總兵恩長為翼長,領標兵二千當前鋒,自率兵千餘進次九江。恩長猝與賊𥫗,戰死江中,師大潰。建瀛途逢潰卒白敗狀,從兵盡駭。江西巡撫張芾壁九江,亦引軍退走,賊遂陷九江。建瀛駕小舟經小孤山不敢留,過安慶,巡撫蔣文慶邀之,不入;徑回江寧,收蕪湖、太平兵屯東西梁山,閉城為守禦計。布政使祁宿藻故不滿建瀛,面責之。將軍祥厚兵防內城,無任戰守者。建瀛大窘,稱疾謝客者三日。於是祥厚、宿藻等疏劾建瀛棄險失機,進退無據,並及江蘇巡撫楊文定違旨去江寧,上大怒,諭曰:「陸建瀛一戰兵潰,不知收合餘燼,與向榮大軍協力攻擊;並不力守小孤山,扼賊入皖之路;又不親督兵據守東西梁山,以障金陵。倉皇遁歸,一籌莫展,以致會垣驚擾,士民播遷。楊文定藉詞出省,張皇自全,罪均難逭。建瀛已革職,交祥厚拿問,解刑部治罪。」尋籍其家,革其子刑部員外郎鍾漢職。時建瀛收兵乘城,閱十三日,城破遇害。事聞,詔建瀛尚不失城亡與亡之義,復總督銜,如例議恤,並還其家產。御史方俊論之,乃撤恤典。

建瀛才敏任事,喜賓禮名流,又善事要津,多為延譽,由是聞望猋起,朝寄日隆。乃昧於軍旅,略無宿備,一敗失措,名城陷為賊窟,糜爛東南,遂獨攖天下之重咎云。子鍾漢,後官江蘇知府,咸豐十年,在軍治糧餉,遇賊江陰,死之,贈太僕寺卿。

楊文定,安徽定遠人。道光十三年進士。由刑部主事洊升郎中,出為廣東惠潮嘉道,累擢江蘇巡撫。咸豐三年,文定奏江南兵力柔脆,節經徵調,城內兵單,請濟師,命山東兵二千赴援。未至,奉命守江寧,聞建瀛兵敗,退守鎮江。江寧陷,賊分黨犯鎮江,副都統文藝集兵七百守陸路,文定自率艇船八、舢板十二泊江中,賊至不能御,鎮江復陷,退江陰,詔革職逮治,論大闢。六年,減死遣戍軍臺,尋歿。

青麟,字墨卿,圖們氏,滿洲正白旗人。道光二十一年進士,選庶吉士,授編修,遷中允。大考二等,擢侍講。五遷至內閣學士。督江蘇學政有聲。咸豐二年,擢戶部侍郎。學政任滿,命督催豐北塞決工程。三年,回京,復出督湖北學政,調禮部侍郎。

時粵匪由江西回竄湖北,青麟按試德安,聞警停試,督率知府易容之募鄉勇籌防守,府城獲全。疏陳軍事,請湖北、江西、安徽三省合剿,以期得力。四年,授湖北巡撫。城中兵僅千人,荊州將軍臺湧署總督,未至;而賊由黃州進至漢陽、漢口,渡江欲撲武昌。青麟督總兵楊昌泗、游擊侯鳳岐與副都統魁玉水陸合擊,卻之;復敗之豹子海、魯家港,毀賊壘五。已而賊撲塘角、占魚套,逼攻省城,青麟武勝門督戰,城中忽火起,土匪內應,兵盡潰,遂失守。青麟將自經,眾擁之趨長沙,折赴荊州。

初,文宗聞其出家貲犒軍,甚嘉之,至是憤武昌屢失,棄城越境,罪尤重,詔曰:「青麟簡任封圻,正當賊匪充斥,武昌兵單餉匱。朕以其任學政時保守德安,念其勤勞,畀以重任。省垣布置,屢次擊賊獲勝。八十餘日之中,困苦艱難,所奏原無虛假,朕方嚴催援兵接應。六月初間,魁玉、楊昌泗等連破賊營,但能激厲力戰,何致遽陷?嬰城固守,解圍有日,猶將宥過論功。縱力盡捐軀,褒忠有典,豈不心跡光明?乃倉皇遠避,徑赴長沙,直是棄城而逃。長沙非所轄之地,越境偷生,何詞以解?若再加寬典,是疆臣守土之責,幾成具文,何以對死事諸臣耶!朕賞罰一秉大公,豈能以前此微勞,稍從末減?俟到荊州時,交官文傳旨正法。」遂棄市。

逾數月,曾國籓復武昌,奉命查歷任督撫功罪,疏言:「武昌再陷,實因崇綸、臺湧多方貽誤,百姓恨之,極稱吳文鎔忠勤愛國,於青麟亦多恕辭。查文鎔既沒,青麟幫辦軍務,崇綸百端齟:求弁兵以護衛,不與;請銀兩以制械,不與;或軍務不使聞知,或經旬不得相見。自賊踞漢陽、漢口,縱橫蹂躪,廬舍蕩然。百姓尚恃有青麟督兵驅逐,出示憐民。崇綸則並此無之矣。」疏入,乃斥罷臺湧,論崇綸罪。

崇綸,喜塔臘氏,滿洲正黃旗人。由內閣貼寫中書充軍機章京,洊升侍讀。出為陜西鳳邠道,調直隸永定河道,歷云南按察使、廣東布政使。

咸豐二年,擢湖北巡撫,時武昌方為賊踞,次年春,賊棄武漢東下,分擾江南、江西,崇綸始抵任。既而賊復上竄,陷興國州田家鎮,進黃州。崇綸疏言:「武漢民遷市絕,餉乏兵單。請移內就外,以剿為先。」未幾,賊犯漢陽,窺武昌。總督吳文鎔初至,與崇綸意相迕。及賊退,崇綸遂以閉城株守劾之。文宗慮兩人不能和衷,且僨事,命文鎔出剿,而責崇綸防守。文鎔率師薄黃州,崇綸運輸餉械不以時,惟促速戰。四年正月,文鎔兵敗,死之。崇綸自請出剿,謀脫身走避,文宗燭其隱,不許。會丁憂,青麟代之,仍命崇綸留湖北協防。又以病乞罷,上怒,褫其職。六月,武昌陷,崇綸先一日出走,徑往陜西。及曾國籓論劾,命逮治。服毒自盡,以病故聞。

何桂清,字根雲,雲南昆明人。道光十五年進士,選庶吉士,授編修。遷贊善,直南書房。五遷至內閣學士。二十八年,擢兵部侍郎,以憂去,服闋,補原官,調戶部。咸豐二年,督江蘇學政。粵匪擾江南,桂清疏陳兵事,劾疆吏巽耎僨事,侃侃無所避,文宗奇之。四年,調倉場侍郎,旋授浙江巡撫。

自賊踞江寧,東南震動。安徽徽州、寧國二府為浙江屏蔽,桂清嚴防要隘,別遣一軍屯守黃池,扼蘇、浙之沖,賊來犯,會提督鄧紹良擊卻之。五年,檄道員徐榮剿賊黟縣、石埭,戰頗利,賊眾大至,徽勇潰走,榮眾寡不敵,遂戰歿。桂清因言徽、浙脣齒,宜主客一心,事乃濟。疏入,諭戒地方官吏不分畛域。時賊陷徽州各屬,桂清檄知府石景芬、副將魁齡等,攻復徽州府城及休寧,分布所部於昌化、於潛、淳安,杜賊來路。安徽巡撫時移駐廬州,徽、寧二郡懸絕江南,不能遙制,命桂清兼轄之。江西賊侵入浙境,陷開化,犯遂安,桂清檄鄧紹良等合擊之,賊退徽境。周天受、石景芬等連復黟縣、石埭。桂清疏請添改鎮道員缺,俾專責成,以石景芬為徽寧池太道;豫祺為總兵,不得力,復以江長貴易之。又用桂清議,命前侍郎張芾駐皖南治團練,督辦徽、寧防務,尋命兼顧浙江衢、嚴兩郡,與桂清協力制賊。六年,檄鄧紹良、秦如虎、都興阿等合攻寧國,別遣江長貴擊敗贛賊之襲太平者,連捷,克寧國府城。朝廷益嘉桂清,思大用之。

杭州知府王有齡最為桂清倚用,擢權運、臬兩篆,為通判徐徵訐控。桂清覆奏,辭悻悻,被詰責。遂以病乞罷,詔慰留之。會兩江總督怡良解職,文宗以籌餉事重,難其人,大學士彭蘊章薦桂清餉徽軍無缺,可勝任。七年春,命以二品頂戴署兩江總督,尋實授。力薦王有齡,擢任江蘇布政使,專倚餉事。江寧久為賊窟,總督駐常州,軍事由將軍和春主之,而提督張國樑為幫辦,前督怡良但任運饋而已。桂清屢疏陳方略稱旨,諭飭和春和衷商酌。是年冬,克鎮江,以濟餉功,加太子少保。十年春,又因克九洑洲,晉太子太保。桂清意氣發舒,倚畀益重,甚負時望。

大軍屢捷,合圍江寧,賊勢窘蹙,四出求援。偽忠王李秀成乃謀竄浙,分大軍之勢,由安徽廣德徑趣杭州。倉猝城陷,惟將軍瑞昌守駐防內城未下,詔促桂清、和春遣軍速援。於是檄提督張玉良率兵馳赴,至則內外夾擊,賊遽走。臨安、孝豐、安吉諸城相繼復。詔嘉桂清功,予優敘。時賊已圍金壇,陷江陰,遣總兵馬得昭、熊天喜、曾秉忠,副將劉成元水陸分路御賊,兵分益單。賊乃合眾十餘萬出建平、東壩,一由東壩趨江寧,一由溧陽窺常州,桂清聞之,幾失所措。會馬得昭、周天孚分援蘇、常,賊已趨金壇,陷句容。句容為大營後路,自此隔絕。張玉良回軍抵常州,和春飛檄調援大營,桂清留勿遣,復調馬得昭,亦莫之應。王有齡已擢浙江巡撫,貽書桂清戒勿離常州一步,且曰:「事棘時危,身為大臣,萬目睽睽,視以動止。一舉足則人心瓦解矣。」蓋規之也。

會大雨雪,大營兵凍餒,索餉不得,乃譟亂,相率盡潰。和春、張國樑退守丹陽。桂清疏陳:「丹陽以上軍務,和春、張國樑主之;常州軍務,臣與張玉良主之。」部署稍定,即進規溧陽,而賊已逕犯丹陽,國樑死之,和春奔常州,桂清大驚。總理糧臺查文經等希其意,請退保蘇州。桂清即疏陳軍事付和春,自駐蘇州籌餉。將行,常州紳民塞道請留,從者槍擊,死十餘人,始得脫。張玉良留守,尋亦走。士民登陴,數日城陷,屠焉。桂清至蘇州,巡撫徐有壬拒勿納,疏劾其棄城喪師狀。和春退至無錫,傷殞。桂清託言借外兵,遂之上海。蘇州亦陷,有壬殉之,遺疏再劾桂清,詔褫職逮京治罪。

會各國聯軍犯京師,車駕幸熱河,遷延兩年。王有齡及江蘇巡撫薛煥皆其故吏,疊疏為乞恩,不許。言官數劾奏,同治元年,始就逮下獄,讞擬斬監候。大學士祁俊藻等十七人上疏論救,尚書李棠階力爭,讞乃定。桂清援司道稟牘為詞,下曾國籓察奏。國籓疏言:「疆吏以城守為大節,不宜以僚屬一言為進止。大臣以心跡定罪,不必以公稟有無為權衡。」是冬,遂棄市。

桂清由侍從出任疆事,才識明敏。在兩江值英吉利構釁,迭陳應付之策。偕大學士桂良等議稅則,多中肯綮,亦不能盡用其言。晚節敗裂,誤國殄民,雖廷議多有袒之者,卒難撓公論云。

論曰:陸建瀛、何桂清皆以才敏負一時之望,膺江表重寄。建瀛當軍事初起,不能預有規畫,臨事倉皇。桂清無料敵之明,又失效死之節。二人者身名俱隕,罪實難辭。青麟受事於危急之秋,艱難支拄,終以越境被誅,論者猶有恕詞焉。


\end{pinyinscope}