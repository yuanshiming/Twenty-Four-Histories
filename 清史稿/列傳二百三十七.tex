\article{列傳二百三十七}

\begin{pinyinscope}
李鶴年文彬任道鎔許振禕吳大澂

李鶴年,字子和,奉天義州人。道光二十五年進士,由編修改御史,轉給事中。父憂歸,服除,命赴河南襄辦軍務。同治元年,授常鎮通海道,署河南按察使,調直隸,授布政使。四年,擢湖北巡撫,調河南。

時捻匪由山東南竄,鶴年以為十餘年來賊屢擾歸、陳、南、汝間,即去而他竄,必假道於豫。乃增募兩軍各萬餘人,一曰毅軍,宋慶統之;一曰嵩武軍,張曜統之;更以馬隊屬善慶,與兩軍為犄角。於是宋慶等軍大破張總愚睢州,鶴年親赴陳、留、杞督戰。任、賴各逆復乘虛北擾,鶴年以賊蹤無定,防河尤急。賊果犯中牟,以有備不得逞,乃於省治西決堤引水南流,擾及長垣。鶴年飛檄水陸各軍沿堤剿堵。賊西走湖北麻城、黃岡,詔飭宋慶一軍越境會剿,殲賊無算。鶴年自駐許州策應,賊竄裕州,慶擊敗之。善慶及淮軍劉銘傳大敗賊贛榆,任柱被戕死。賜鶴年頭品頂戴。七年,奉命督師出境,馳抵磁州。捻犯近畿,更由滑、濬等處沿河東趨。坐防堵不力,再議處。豫軍告捷,賞還頂戴。總愚溺死,捻匪平,照一等軍功議敘。

十年,擢閩浙總督。明年,陛見,賜紫禁城騎馬。旋署福州將軍,兼署巡撫。詔詢海防事宜,覆奏言:「海防之策,莫重於練兵、籌餉、制器、用人四端。四者之中,以用人為急務,而尤在專其責成。沿海疆臣固責無旁貸,第無統率大員,仍恐意見紛歧,臨事推諉。」上韙其議。

光緒元年,調河東河道總督,兼署河南巡撫。七年,授河南巡撫,仍兼河督。十年,坐審辦盜犯胡體安連疏抗辨,部議革職,以祝嘏恩賞降二級職銜。十三年,署河督,疏言:「黃河分流,自宋時河決澶州,分為二派。明築黃陵岡,始合為一。河性上漫則下淤,今兩路皆淤,急宜疏支河以預籌宣洩。」報可。逾年,鄭工復決,發軍臺效力。未幾釋歸,並賞三品銜。十六年,卒。宣統元年,開復原官。

鶴年有知人鑒,少與文祥同學相淬厲。及居言職,嚴疏劾肅順跋扈,而奏起曾國籓於家,謂必能辦賊。拔宋慶、張曜統豫軍,後皆為名將。治豫久,多善政,豫人刻石頌之。始任河督,黑岡堤潰,不絕如縷。鶴年親督工二十餘晝夜,險工克濟。德宗嘗詢李鴻藻以善治河者,鴻藻舉鶴年,上亦識前事之枉也,故再任河督。其卒也,豫民有流涕者。三子葆恂博學多文,尤知名。

文彬,字質夫,納喇氏,內務府滿洲正白旗人。咸豐二年進士,授戶部主事。十年,以員外郎隨扈幸熱河。明年,遷郎中,出知山東沂州府。捻匪逼府城,會師攻拔賊巢,擒匪首孫化詳等。敘功,以道員用。同治四年,隨布政使丁寶楨敗賊滕縣臨城驛,更繞赴東平防賊北竄。補兗沂曹濟道,擢按察使。收復海豐,擢布政使。十年,署巡撫,補漕運總督。再署巡撫,旋還任。

光緒五年,督漕北上,因請陛見,並與河督李鶴年、巡撫周恆祺會商運河事宜,通籌河道寬深,改設運口,導引衛河,設立堤壩,繪具圖說以進。略謂:「現時北運口在張秋南八里廟,與南運口斜對,相距二十餘里。黃流至此雖收束,而溜勢散漫,歧汊甚多。大抵溜勢近南則北口淤墊,近北則南口淺阻。故漕船出南運口入黃後,必東北行二十里,至黃溜匯一之史家橋,再南行二十里,至八里廟北運口,汛水大漲,方能入運。今擬移北運口於史家橋北六里。黃河西岸,由阿城徬東堤開河一道至陶長堡,為出黃入運口門,築壩灌塘,則黃水不至奪溜,可免牽挽之難。黃、運之間,自賈工合龍後,每伏秋大雨,水無所洩,民間低地有積水數年不得耕種者,若將陂水引歸一塘,不惟蓄水濟運,又可涸復民田。運口既定,即可導引衛河。自直隸元城集東三里衛河曲處鑿新河一道,經直隸之南樂、山東之朝城,至張秋南之蕭口涵洞入運。計衛高於運九丈餘,長百五十餘里,導以濟運,勢如建瓴。更有大小二丹水,亦可由衛濟運。凡建四徬二壩及挑河築堤,估銀七十六萬。較之借黃濟運旋挑旋淤者,相去遠矣。」

又嘗偕兩江總督吳元炳奏復淮流故道,略謂:「淮水匯四十餘河瀦於洪澤湖,楊莊以下雲梯關為入海故道,餘波入運濟漕。遇旱,復蓄淮流由運河分入淮揚各徬洞,以溉民田。自洪澤湖不能瀦水,張、福引河又不通暢,每遇盛漲,運河一線東堤,其勢岌岌。儻竟沖溢,不至以里下河為壑不止。論者謂必設法束水,然與其上游議堵,何如下游深通。」因條上疏濬楊莊以下舊河入海故道。

未幾,卒,有詔褒錫。兩江總督劉坤一以文彬遺愛在民,請建專祠清江浦,允之。子延煜,舉人,四川鹽茶道;延熙,舉人,九江知府;延燮,進士,武昌知縣;延照,舉人,禮部員外郎。

任道鎔,字筱沅,江蘇宜興人。拔貢,考授教職。咸豐中,在籍襄辦團練,除奉賢訓導。以籌餉勞,晉秩知縣,銓當陽,多善政,調江夏。同治二年,擢知順德府。畿南匪起,行堅壁清野法,修治城堡,屢擊賊於沙河、平鄉間。會捻眾北犯,道鎔率練勇守沙河。夜與賊遇,揮眾奮擊,矛傷及身,不退,賊徐引去,晉秩道員。洺河自廣平入,久淤塞。道鎔與鄰郡合濬,又濬郡北響水河,復民田萬餘頃。總督曾國籓、李鴻章迭薦之。十一年,調保定,尋擢開歸陳許道。剔河工積弊,驗工料必以實。嘗冒風雨搶護中河險工,四晝夜始定。

光緒元年,署按察使。授江西按察使,省獄羈囚四百餘人,道鎔便宜訊決,三月而清。四年,遷浙江布政使,調直隸。直隸自軍興,州縣報銷未清,又數值謁陵大差,交代糾葛。道鎔分別新舊案,定限清結。裁革州縣攤捐,實發養廉銀以恤吏,勸屬縣積穀備荒。七年,擢山東巡撫,疏陳營務廢弛,易置統將,以綠營額餉練新軍,責郡縣勤緝捕。泰山、沂水之間,驛路崎嶇,發卒開治平坦,行旅便之。旋以保獎已革知府潘駿群被議,又以失察編修林國柱預報起復,被劾褫職,降道員。家居久之。

二十一年,起河道總督。故事,河督,開封、濟寧並設行署。自咸豐時,常駐開封,山東河事由巡撫專治。至是復改議河督駐濟寧,而河南巡撫兼治河。道鎔言:「官吏不相屬,則令難行,不如仍舊便。」報可。時河患多在下游,河督專司上游,事簡。道鎔務節費,歲以餘帑還司庫。二十六年,拳匪起,河南奸民乘機煽亂。道鎔處以鎮靜,練河標三營助省防。次年,調浙江巡撫。承國威新挫後,民教相閧,案多未結,持平訊決之。籌集償款,衡其緩急,民不重困。二十八年,乞病歸。逾三年,卒於家,年八十三。

許振禕,字仙屏,江西奉新人。咸豐初,以拔貢生參曾國籓戎幕。迨楚軍困於江西,都邑相繼陷,振禕偕內閣中書鄧輔綸募鄉兵擊賊進賢、東鄉,旋復吉安。敘功,以同知銓選。同治二年,成進士,授職編修,出督陜甘學政。時河州降回復叛,而西寧諸郡回、漢民亦日相仇殺,試事久停不舉。振禕始按試各郡,多錄降人子弟,補行八次歲科試,入學者數千人,回民大服。建味經書院於涇陽,廣置書籍,以化其獷俗。又請陜、甘分闈鄉試,各設學政,允之。總督左宗棠以謂邊氓長治久安之效,胥基於此。父憂歸。

光緒二年,起故官。八年,授彰衛懷道,減屬縣差徭費歲二十餘萬。豫修里河堤防,淮海各鹽區得免水患。十六年,擢河東河道總督,築滎澤大壩,胡家屯、米童寨各石壩,河賴以無患。其要尤在嚴稽察,不私財權,令七徑赴司庫支領,故積弊徐而工堅。二十一年,遷廣東巡撫,禁賭闈姓,粵民利賴之。二十四年,裁廣東、雲南、湖北三巡撫缺,振禕調內用。乞假歸,逾年卒。附祀江蘇、河南曾國籓祠。

吳大澂,字清卿,江蘇吳縣人。同治元年秋,彗星見西北,詔求直言。大澂方為諸生,入都應京兆試。上書言:「致治之本,在興儉舉廉,不言理財而財自裕。若專務掊克,罔恤民艱,其國必敝。」後六年成進士,授編修。穆宗大婚典禮隆縟,疏請裁減繁費,直聲震朝右。出為陜甘學政,奏以倉頡列祀典,允之。又薦諸生賀瑞麟、楊樹椿篤志正學,給瑞麟國子監學正銜,樹椿翰林院待詔銜,士風為之一變。時詔修頤和園,大澂復言時事艱難,請停止工作。疏入,留中。

光緒三年,山、陜大饑,奉命襄辦賑務。躬履災區查勘,全活甚眾。左宗棠、曾國荃、李鴻章等交章論薦。四年,授河北道。時比歲薦饑,貧民減價鬻田,十不得一。巡撫塗宗瀛飭荒歲賤價之田準取贖,然往往為勢家所持,以故失業者眾。惟大澂能判決如巡撫恉。

六年,詔給三品卿銜,隨吉林將軍銘安辦理西北邊防。大澂周歷要隘,始知琿春黑頂子地久為俄人侵占。因請頒舊界圖,將定期與俄官抗議,未得旨。時有韓效忠者,登州人,傭於復州侯氏。負博進,遁往吉林夾皮溝。地產金,在寧古塔、三姓東,萬山環繞,廣袤七八百里。流冗嘯聚其中,亡慮四五萬,咸受效忠約束。效忠嚴而不擾,眾服其公允,屢抗大軍不出。大澂單騎抵其巢,留宿三日,勸效忠出,效忠猶豫,意難之。大澂曰:「我不疑若,若乃疑我耶?」對曰:「非敢疑公。某負罪久,萬一主兵者執前事為罪。某死不恨,辜公意奈何?」大澂挺以自任,遂與效忠出,奏給五品頂戴,子七品,孫登舉有平寇功,授參將。七年,授太僕寺卿。法越事起,會辦北洋軍務,駐防樂亭、昌黎。

十年,遷左副都御史。俄,命使朝鮮,定其內亂,鹽運使續昌副之。至則日本使臣井上馨避不肯見,而挾朝鮮左議政金宏集於議政院,索償兵費三十萬。大澂謂續昌曰:「是蔑我也!」立率兵至議政院,排闥入,責數宏集:「柄國敗壞國事。今定約稍不慎,便滋異日紛,非所以靖國也。」宏集唯唯,井上馨亦氣懾,減索兵費十一萬而去。

十一年,詔赴吉林,會同副都統伊克唐阿與俄使勘侵界,即所侵琿春黑頂子地也。遂援咸豐十一年舊界圖立碑五座,建銅柱,自篆銘曰:「疆域有表國有維,此柱可立不可移。」於是侵界復歸中國,而船之出入圖們江者亦卒以通航無阻。十二年,擢廣東巡撫。葡萄牙侵界至澳門香山。總署與立約通商,畫澳門歸葡轄。大澂持不可,條上駁議,不報。

十四年,鄭州河再決,上震怒,褫河督李鶴年職,以大澂代之。是年冬,河工合龍,大澂力居多。大澂盛負時譽,會海軍議起,以醇親王奕枻為總理。大澂素與王善,治河功成,實授河道總督,加頭品頂戴。大澂遂疏請尊崇醇親王稱號禮節。疏入,孝欽顯皇后震怒,出醇親王元年所上預杜妄論疏頒示天下。大澂幾得嚴譴,以母喪歸,乃已。

十八年,授湖南巡撫。朝鮮東學黨之亂也,日本與中國開釁,朝議皆主戰。大澂因自請率湘軍赴前敵,優詔允之。二十一年,出關會諸軍規復海城,而日本由間道取牛莊。魏光燾往御,戰不利。李光久馳救之,亦敗,僅以數騎免。大澂憤湘軍盡覆,拔劍欲自裁,王同愈在側,格阻之,同愈以編修參大澂軍事也。光燾請申軍法,大澂嘆曰:「余實不能軍,當自請嚴議。」退入關,奉革職留任之旨。乃還湖南,尋命開缺。二十四年,復降旨革職永不敘用。二十八年,卒,年六十八。

大澂善篆籀,罷官後,貧甚,售書畫、古銅器自給。著有古籀補、古玉圖考、權衡度量考、恆軒古金錄、䦛齋詩文集。

論曰:河患日棘,而河臣但歲慶安瀾,即為奇績,久未聞統全局而防永患,求治難矣。鶴年以善治河稱,文彬論治河改運口,復淮流,亦頗有識。道鎔剔河工積弊,務節減,振禕督工嚴,盡革中飽,尤以勤廉者,皆足收一時之效,然徒治標,非治本計也。大澂治河有名,而好言兵,才氣自喜,卒以虛憍敗,惜哉!


\end{pinyinscope}