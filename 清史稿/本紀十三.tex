\article{本紀十三}

\begin{pinyinscope}
高宗本紀四

三十一年春正月壬申朔,詔以御宇三十年,函夏謐寧,寰宇式闢,自本年始,普免各省漕糧一次。甲戌,免甘肅靖遠等十四州縣、陜西延安等三府州屬積年逋賦。丙戌,雲南官軍剿莽匪於猛住,失利。調楊應琚為雲貴總督,吳達善為陜甘總督,以和其衷護之。調劉藻為湖廣總督,湯聘署陜西巡撫。癸巳,刑部尚書莊有恭以讞段成功劾案不實,褫職下獄,籍產。調李侍堯為刑部尚書,以張泰開為禮部尚書,範時綬為左都御史。

二月壬寅,劉藻降湖北巡撫,仍與雲南提督達啟下部嚴議。以定長為湖廣總督,調李因培為福建巡撫,常鈞為湖南巡撫,湯聘為雲南巡撫。庚戌,上謁東陵。辛亥,和其衷以彌補段成功虧空,褫職逮問。以舒赫德署陜甘總督。命四達赴陜西會彰寶審辦段成功虧空一案。調明山為陜西巡撫,以吳紹詩為江西巡撫。庚申,上還京師。辛酉,莊有恭論斬。壬戌,上謁泰陵。癸亥,劉藻褫職,留滇效力。甲子,以鄂寧為湖北巡撫。戊辰,上還京師。

三月丁亥,劉藻畏罪自殺。己丑,楊應琚以復猛籠等土司內附奏聞。

夏四月辛丑,楊應琚奏大猛養頭人內附,官軍進取整欠、孟艮。壬寅,以莽匪整欠平,宣諭中外。丙午,和其衷論斬,段成功處斬。丁未,免雲南普藤等十三土司本年額賦及猛籠逋賦。甲子,賜張書勛等二百一十三人進士及第出身有差。

五月甲戌,上詣黑龍潭祈雨。戊寅,命正一真人視三品秩。丙戌,上詣黑龍潭祈雨。

六月丙午,楊應琚奏猛勇頭目召齋及猛龍沙頭目叭護猛等內附。戊申,予故三品銜西洋人郎世寧侍郎銜。

秋七月丙子,上奉皇太后秋獮木蘭。己卯,以阿里袞、于敏中扈從,命舒赫德兼署戶部尚書。壬午,上奉皇太后駐蹕避暑山莊。是日,皇后崩。癸未,諭以皇后上年從幸江、浙,不能恪盡孝道,喪儀照皇貴妃例。癸巳,御史李玉鳴奏皇后喪儀未能如例,忤旨,戍伊犁。丁酉,楊應琚奏補哈大頭目噶第牙翁、猛撒頭目喇鮓細利內附。

八月己亥,賑湖南湘陰等十三縣衛水災。癸丑,上幸木蘭行圍。宥莊有恭罪,起為福建巡撫。甲寅,伊犁蝗。乙卯,江蘇銅山縣韓家堂河決。癸亥,裁察哈爾副都統,留一員駐張家口。

九月壬申,免甘肅靖遠等九縣,紅水、東樂二縣被旱額賦。己卯,賑山東歷城等五十五縣、東昌等五衛所水災,並蠲新舊額賦。乙未,楊應琚赴永昌受木邦降。

冬十月己亥,上奉皇太后還京師。戊申,楊應琚奏整賣、景線、景海各部頭人內附。辛亥,韓家堂決口合龍。兵部尚書彭啟豐降補侍郎。甲寅,以陸宗楷為兵部尚書。壬戌,增設雲南迤南道。

十一月乙亥,楊應琚奏,緬甸大山、猛育、猛答各部頭人內附。戊寅,以楊應琚病,命楊廷璋赴永昌接辦緬匪。癸巳,命侍衛福靈安帶御醫往視楊應琚病。

十二月乙巳,調鄂寧為湖南巡撫,以鄂寶為湖北巡撫。癸丑,以巴祿為綏遠城將軍。

是歲,朝鮮、琉球入貢。

三十二年春正月乙亥,雲南官軍剿緬匪於新街,失利,諭楊廷璋回廣東。

二月乙未,以楊應琚病,命其子江蘇按察使楊重英赴永昌襄理軍務。丙午,雲南官軍與緬匪戰於底麻江,失利,逮提督李時升下獄。戊申,調鄂寧為雲南巡撫。甲寅,莊親王允祿卒。丙辰,上臨奠。己未,上巡幸天津。癸亥,賑奉天承德等五州縣及興京鳳凰城災民。

三月乙丑朔,上閱子牙河堤。召楊應琚入閣辦事,以明瑞為雲貴總督。丙寅,調託庸為工部尚書,以明瑞為兵部尚書。己巳,免直隸全省逋賦。庚午,上閱天津駐防滿洲兵。以阿桂為伊犁將軍。壬申,上閱綠營兵。庚辰,上還京師。辛巳,大學士楊應琚褫職。壬午,以緬匪入寇盞達、隴川,宣示楊應琚貽誤罪狀。癸未,命鄂寧赴普洱辦軍務。庚寅,以李侍堯為兩廣總督,召楊廷璋為刑部尚書。癸巳,以鄂寧署云貴總督。

夏四月己酉,上詣黑龍潭祈雨。庚戌,以雲南邊境瘴盛,命暫停進兵。庚申,命張泰開以禮部尚書管左都御史事,嵇璜署禮部尚書。

五月己巳,以鄂寶為貴州巡撫,定長兼署湖北巡撫。庚午,以範時綬為湖北巡撫。調張泰開為左都御史,嵇璜為禮部尚書。壬申,命陳宏謀管工部。丙子,雲南官軍失利於木邦,楊寧等退師龍陵。庚寅,李時升、硃侖處斬。

六月辛酉,以額爾景額為參贊大臣,遣赴雲南。

秋七月,福建巡撫莊有恭卒,調崔應階代之。以李清時為山東巡撫,裘曰修為禮部尚書。壬午,上奉皇太后秋獮木蘭。戊子,上奉皇太后駐避暑山莊。己丑,盛京將軍舍圖肯免,以新柱代之。

閏七月甲寅,賜楊應琚自盡。丙辰,緬匪渡小猛侖江入寇雲南茨通。

八月癸酉,調裘曰修為工部尚書,董邦達為禮部尚書。丁丑,上幸木蘭。乙酉,以鍾音為廣東巡撫。乙丑,諭明瑞以額勒登額代譚五格分路進兵。

九月庚子,賑湖北江夏等二十七縣、武昌等七衛水災。甲寅,命託恩多署兵部尚書。

冬十月壬戌,賜李因培自盡。己卯,諭明瑞以將軍管總督。

十一月壬寅,賑甘肅平涼等三十四州縣被雹災民。壬子,調鄂寶為湖北巡撫。丁巳,密諭明瑞,以阿瓦不能遽下,退師木邦。

十二月甲戌,楊寧褫職戍伊犁。戊寅,明瑞奏渡大疊江進軍錫箔,波龍等處土司頭人羅外耀特等內附。

三十三年春正月辛卯,明瑞奏克蠻結。壬辰,封明瑞一等誠毅嘉勇公,賜黃帶、紅寶石頂、四團龍補服。丁酉,明瑞進軍宋賽。庚子,調彰寶為山東巡撫,以蘇爾德為山西巡撫。丙午,盛京將軍新柱卒,調明福代之。閩浙總督蘇昌卒。丁未,命阿里袞為參贊大臣,往雲南軍營。以崔應階為閩浙總督,富尼漢為福建巡撫。甲寅,緬人圍木邦。

二月丙寅,諭用兵緬甸,輕敵致衄,引為己過,令明瑞等班師。額勒登額、譚五格褫職逮問。命鄂寧回雲南,阿里袞署云貴總督,駐永昌。緬人陷木邦,珠魯訥死之。戊寅,上還圓明園。丙戌,明瑞等敗績於猛育,死之。召阿桂來京,以伊勒圖署伊犁將軍。命傅恆為經略,阿里袞、阿桂為副將軍,舒赫德為參贊大臣,赴雲南。以鄂寧為雲貴總督,調明德為雲南巡撫。以福隆安為兵部尚書,命在軍機處學習行走。以永德為浙江巡撫,調彰寶為江蘇巡撫,富尼漢為山東巡撫,鄂寶為福建巡撫,程燾為湖北巡撫。

三月癸巳,免山東高苑等三縣三十二年被水額賦。乙巳,調鄂寶為廣西巡撫,鍾音為福建巡撫,良卿為廣東巡撫,錢度為貴州巡撫,巴祿為察哈爾都統,傅良為綏遠城將軍。癸丑,免江西南昌等十三縣三十二年被水額賦。

夏四月丁卯,調錢度為廣東巡撫。己巳,免安徽安慶等七府州屬三十二年被水額賦。壬申,御試翰林、詹事等官,擢吳省欽等三員為一等,餘升黜有差。試由部院入翰林等官,擢覺羅巴彥學為一等,餘升擢有差。甲申,磔額勒登額於市,譚五格處斬。乙酉,上臨奠明瑞、扎拉豐阿、觀音保。

五月庚申,命明德赴永昌。乙丑,色布騰巴勒珠爾病免,以伊勒圖為理籓院尚書。庚午,改命官保署理籓院尚書。辛巳,以範時綬為左都御史。壬午,以阿桂為雲貴總督。尹繼善、高晉以兩淮鹽務積弊匿不以聞,均下部嚴議。

秋七月癸巳,上奉皇太后秋獮木蘭。甲午,調託庸為兵部尚書。以官保為刑部尚書,仍兼署理籓院尚書。己亥,上奉皇太后駐避暑山莊。辛丑,以尹勒圖為伊犁將軍,仍兼理籓院尚書。壬子,紀昀以漏洩籍沒前運使盧見曾諭旨,褫職,戍烏魯木齊。

八月丁卯,允俄羅斯於恰克圖通商。辛未,上幸木蘭行圍。壬申,直隸總督方觀承卒,以楊廷璋代之。調裘曰修為刑部尚書,以蔡新為工部尚書。甲戌,李侍堯奏,暹羅為緬人所破,其國王之孫詔萃奔安南河仙鎮,土官莫士麟留養之,內地人甘恩敕據暹羅,乞封敕。嘉獎莫士麟,命甘恩敕求其主近支立之,不得自王乞封號。己卯,加託恩多、于敏中、崔應階太子太保,託庸、楊廷璋太子少保。

九月戊子,以嵩椿署伊犁將軍。乙未,上回駐避暑山莊。戊戌,高恆、普福論斬。丁未,上奉皇太后還京師。以鄂寶為山西巡撫。黑龍江將軍富僧阿改西安將軍,以傅玉代之。

冬十月己未,免甘肅平涼等十二州縣三十二年被災額賦。辛未,以宮兆麟為廣西巡撫。辛巳,高恆、普福、達色處斬,改海明等緩決。

十一月戊戌,以緬人來書不遜,諭阿里袞籌進剿。

十二月己未,以富明安為山東巡撫,揆義署湖北巡撫。漕運總督楊錫紱卒,以梁翥鴻署之。乙丑,湖廣總督定長卒,調吳達善代之,彰寶兼署兩江總督,明山為陜甘總督。調阿思哈為陜西巡撫,以文綬為河南巡撫。丁卯,召明福來京,以額爾德蒙額署盛京將軍。甲戌,賑奉天承德等四州縣水災。壬午,留阿思哈為河南巡撫,改文綬為陜西巡撫。

三十四年春正月丙戌,免雲南官兵所過地方及永昌等三府州本年額賦。其非經過地方,免十分之五,並免湖北、湖南、貴州三省官兵經過地方本年額賦十分之三。庚寅,以緬人書詞桀驁,命副將軍阿桂與副將軍阿里袞協助傅恆征剿。辛卯,命明德為雲貴總督,駐永昌,喀寧阿為雲南巡撫。壬辰,阿里袞等敗緬人於南底壩。撥運通倉米二十萬石賑霸州等十二州縣災。甲午,右部哈薩克阿勒比斯子卓勒齊等來朝。乙未,調恆祿為盛京將軍,傅良為吉林將軍,常在為綏遠城將軍。辛丑,傅恆赴雲南。命官保署戶部尚書。裁寧夏右翼副都統、吉林拉林副都統。命常青署綏遠城將軍。癸卯,賜傅恆御用盔甲。戊申,命官保協辦大學士,以福隆安署刑部尚書。癸丑,以南掌國王之弟召翁遣使請兵復仇,諭阿桂等預備由南掌分路進兵。

二月甲寅朔,嵇璜緣事降調,以程景伊為工部尚書。乙丑,以富尼漢為安徽巡撫。癸未,命傅恆整飭云南馬政。以諾倫為綏遠城將軍。

三月乙酉,命伊犁將軍伊勒圖往雲南軍營。己丑,命伊爾圖為烏里雅蘇臺參贊大臣。辛丑,正白旗領侍衛內大臣福祿罷,以阿桂代之。丙午,命阿桂署云貴總督。丁未,右部哈薩克斡里蘇勒統等入覲,命坐賜茶,賚冠服有差。戊申,賑甘肅皋蘭等二十九州縣上年災民。蠲安徽合肥等十六州縣及廬州等五衛上年額賦。

夏四月己未,以溫福為福建巡撫。壬申,傅恆進兵老官屯,阿桂進兵猛密。丁丑,賜陳初哲等一百五十一人進士及第出身有差。

五月己丑,裁江寧副都統一。

六月丙辰,以阿思哈為雲貴總督,喀寧阿為河南巡撫。丁巳,傅恆奏猛拱土司內附。戊寅,湖北黃梅江堤決,命湖廣總督吳達善、湖北巡撫揆義勘之。

秋七月丁亥,以明德署云貴總督,移駐騰越、經理軍務。辛卯,設伊犁巴彥岱城領隊大臣一。傅恆奏猛密土司內附。甲午,李侍堯奏暹羅仍為甘恩敕所踞。丁酉,禮部尚書董邦達卒。己亥,調陸宗楷為禮部尚書,蔡新為兵部尚書。以吳紹詩為刑部尚書,海明為江西巡撫,梁國治為湖北巡撫。己酉,李侍堯檄莫士麟會暹羅土目討甘恩敕。

八月乙丑,上幸木蘭行圍。己巳,以蔡琛自縊獄中,褫福建按察使孫孝愉職,發軍臺。

九月丙戌,阿桂進抵蠻暮。己丑,上回駐避暑山莊。乙未,上奉皇太后回鑾。己亥,命阿桂、伊勒圖自蠻暮迓傅恆會師。壬寅,命劉統勛會勘山東運河。癸卯,傅恆奏猛拱土司渾覺率眾來降。上嘉之。特賞三眼孔雀翎。戊申,傅恆進抵猛養。阿桂奏克哈坎,渡江。命阿桂據新街剿賊。

冬十月乙卯,命彰寶署云貴總督,明德署云南巡撫。調永德為江蘇巡撫。起熊學鵬署浙江巡撫。以增海署伊犁將軍。丁巳,傅恆奏攻克猛養。癸亥,梁國治兼署湖廣總督。甲子,以阿桂不能克老官屯,奪副將軍,為參贊大臣。命伊勒圖為副將軍。調喀寧阿為貴州巡撫,富尼漢為河南巡撫。以胡文伯為安徽巡撫。乙丑,傅恆奏進抵新街。命彰寶駐老官屯。壬申,調永貴為禮部尚書,託庸為吏部尚書,伊勒圖為兵部尚書,以託庸兼署。調吳紹詩為禮部尚書。以裘曰修為刑部尚書。

十一月乙酉,副將軍、戶部尚書阿里袞卒於軍。命阿桂仍在副將軍上行走,並以尹勒圖為副將軍,烏三泰、長青為參贊大臣。調官保為戶部尚書。以素爾納為刑部尚書,託恩多署左都御史。戊子,傅恆等進攻老官屯。癸巳,以黃登賢為漕運總督。丙申,以緬地煙瘴,官軍損失大半,命班師屯野牛壩,召經略傅恆還,阿桂留辦善後。己亥,起觀保署左都御史。丁未,傅恆等攻老官屯不克。其土官以緬酋猛駮蒲葉書詣軍營乞降。上命班師。

十二月辛亥,免雲南辦理軍需地方及永昌等三府州明年錢糧十分之五。其直隸、河南、湖北、湖南、貴州等省官兵經過州縣並免十分之三。調宮兆麟為湖南巡撫,以德保為廣東巡撫,陳輝祖為廣西巡撫。乙卯,傅恆等奏緬酋猛駮稱臣納貢。諭俟來京時降旨。己巳,上以來年奉皇太后謁東陵,巡幸天津,免經過地方及天津府屬乾隆三十五年錢糧十分之三。以阿桂為禮部尚書。

三十五年春正月己卯朔,以上六十壽辰,明歲皇太后八十萬壽,詔普蠲各省額徵地丁錢糧一次。辛卯,以增海為理籓院尚書。丁未,授喀爾喀和碩親王成袞扎布世子拉旺多爾濟為固倫額駙。

二月乙丑,上奉皇太后謁東陵。庚午,上奉皇太后回鑾,駐盤山。壬申,以緬酋猛駮貢表不至,諭彰寶備之,並嚴禁通市。

三月己卯,上奉皇太后還京師。起吳紹詩為刑部郎中。辛巳,調宮兆麟為貴州巡撫,吳達善以湖廣總督兼署湖南巡撫。壬午,上奉皇太后謁泰陵,巡幸天津。丙戌,上謁泰陵。己丑,免經過州縣及天津府屬乾隆三十一年至三十三年積欠地糧銀及常借災借穀石,直隸乾隆三十一年至三十三年積欠地糧銀及折色銀兩。減直隸軍流以下罪。免直隸乾隆三十一年至三十三年因災緩徵銀穀。甲午,上奉皇太后駐蹕天津府。丙申,上閱駐防兵。經略大學士傅恆還京師,命與福隆安俱仍為總管內務府大臣。戊戌,調永德為河南巡撫,薩載署江蘇巡撫。癸卯,上奉皇太后還京師。己酉,以緬酋索木邦土司線甕團等,諭責哈國興粉飾遷就,召來京,以長青代為雲南提督。己未,召傅良來京,命富椿為吉林將軍。丙寅,天津蝗,命楊廷璋督捕。庚午,上詣黑龍潭祈雨。是月,蠲浙江仁和等八州縣,杭嚴、嘉湖二衛,陜西定遠縣三十四年被水被雹額賦。

五月丁丑朔,日食。壬午,以皇八子擅自進城,褫上書房行走觀保、湯先甲職,並戒諭之。乙未,以祈雨命刑部清理庶獄,減軍流以下罪。

閏五月丙午朔,命裘曰修赴薊州、寶坻一帶捕蝗。戊申,京師大雨。己未,命溫福為吏部侍郎,在軍機處行走。甲子,裘曰修以捕蝗不力免,調程景伊為刑部尚書。以範時綬為工部尚書,張若溎為左都御史。

六月甲申,諭阿桂等調海蘭察、哈國興進兵。丙戌,河南永城、江蘇碭山、安徽宿州等州縣蝗。丁亥,調官保為刑部尚書,素爾納為戶部尚書。壬辰,命豐升額署兵部尚書。甲午,貴州古州苗香要等伏誅。命侍郎伍納璽往古北口會同提督王進泰查勘水災,發帑銀二萬兩恤之,並開倉賑糶。

秋七月乙巳朔,李侍堯奏,河仙鎮土官莫士麟請宣諭緬番恢復暹羅,不許。丙午,以增海為黑龍江將軍,溫福為理籓院尚書。命和爾精額、伍納璽往古北口籌辦河工。壬子,以小金川與沃克什土司構釁,命四川總督阿爾泰傳集小金川土司勸諭之。癸丑,上臨和親王弘晝第視疾。丁巳,和親王弘晝卒。太保大學士傅恆卒。戊午,賞來京祝嘏之百十二歲原任浙江遂昌縣學訓導王世芳國子監司業銜,並在籍食俸。辛酉,以裴宗錫為安徽巡撫。甲子,截漕糧二十萬石賑武清等六縣水災。以諾穆親為雲南巡撫。

八月戊寅,以副將軍阿桂辦事取巧,褫領侍衛內大臣、禮部尚書、鑲紅旗漢軍都統,以內大臣革職留任辦副將軍事。己卯,以永貴為禮部尚書,觀保為左都御史。阿爾泰奏僧格桑伏罪,交出達木巴宗地方及所掠番民。辛巳,命劉統勛兼管吏部。丙戌,萬壽節,上詣皇太后宮行禮。御太和殿,王以下文武各官進表,行慶賀禮,奉旨停止筵宴。命豐升額在軍機處行走。己丑,上奉皇太后幸熱河。乙未,上奉皇太后駐蹕避暑山莊。己亥,上幸木蘭。

九月丙午,命阿爾泰為武英殿大學士,仍留辦四川總督事。戊午,上回駐避暑山莊。甲子,命高晉兼署漕運總督。

冬十月癸酉朔,上奉皇太后回鑾。辛巳,召崔應階來京,命鍾音署閩浙總督。壬午,召阿爾泰來京,以德福署四川總督,吳達善兼署湖南巡撫。召薩載來京,命李湖署江蘇巡撫。甲午,阿桂等奏老官屯緬目遣使致書,請停今歲進兵,允之。丁酉,大學士陳宏謀以衰病乞休,溫旨慰留。

十二月甲戌,免新疆本年額糧十分之三。丙子,以崔應階為漕運總督。丙戌,諭阿桂、彰寶密議進剿緬匪。庚寅,以李湖為貴州巡撫。

三十六年春正月甲辰,免福建臺灣府屬本年額徵粟米。乙巳,免廣東廣州、韶州等府州屬本年官租十分之一,廣西桂林七府州屬本年官租及桂林平樂等府州學租十分之三。丁未,免四川寧遠等四府州屬、建昌鎮標各營、雷波等民番本年額糧。己未,調德福署云貴總督,命阿爾泰回四川總督任。

二月甲戌,上奉皇太后東巡。庚辰,命內大臣巴圖濟爾噶勒會同集福讞烏梁海副都統莫尼扎布等互控之案。辛巳,大學士陳宏謀以病乞休,允之,加太子太傅。免直隸滄州等十五州縣民欠借穀,並武清縣本年錢糧十分之五。癸未,命侍郎裘曰修會同楊廷璋、周元理籌辦直隸河工。丙戌,免山東經過州縣本年額賦十分之三、災地十分之五。免山東泰安等二縣本年地丁錢糧。庚寅,免山東濟南各屬民欠借穀及東平州、東平所逋賦。以阿桂請大舉征緬,申飭之。辛卯,免山東濟南等六府屬民欠麥本銀兩。命劉綸為大學士,兼管工部,於敏中協辦大學士。調程景伊為吏部尚書,範時綬為刑部尚書,以裘曰修為工部尚書。丙申,上奉皇太后褐岱嶽廟,上登泰山。乙巳,上至曲阜謁先師孔子廟。丙午,上釋奠先師孔子。丁未,上謁孔林。祭少昊陵、元聖周公廟。賜衍聖公孔昭煥族人銀幣有差。戊申,上奉皇太后回鑾。乙卯,予大學士尹繼善等、尚書官保等、總督楊廷璋等、巡撫鍾音等議敘。內閣學士陸宗楷等原品休致。戊午,以富明安為閩浙總督、周元理為山東巡撫。庚申,以甘肅比歲偏災,免通省民欠籽種口糧倉穀。甲子,上至捷地閱堤。乙丑,納遜特古斯處斬。己巳,以阿桂奏辨非於本年大舉征緬,下部嚴議。

夏四月辛未朔,以李侍堯為內大臣。甲戌,命戶部侍郎桂林在軍機處行走。丁丑,上奉皇太后還京師。乙酉,以旱命刑部清理庶獄,減軍流以下罪,直隸亦如之。丙戌,上詣黑龍潭祈雨。壬辰,大學士尹繼善卒。乙未,賜黃軒等一百六十一人進士及第出身有差。

五月辛丑朔,調吳達善為陜甘總督,文綬署之,勒爾謹護陜西巡撫。調富明安為湖廣總督,永德為湖南巡撫。以何煟為河南巡撫,兼管河務,鍾音為閩浙總督,餘文儀為福建巡撫。癸卯,命減秋審緩決三次人犯罪。甲辰,諭立決人犯當省刑之際,暫緩行刑,著為令。乙巳,阿桂以畏葸褫職,降兵丁效力。命溫福馳赴雲南署副將軍事。壬戌,以高晉為文華殿大學士,兼禮部尚書,仍留兩江總督任。召阿爾泰入閣辦事,以德福為四川總督。

六月辛未,直隸北運河決。甲戌,以努三為正黃旗領侍衛內大臣。戊寅,命巴圖濟爾噶勒赴伊犁辦土爾扈特投誠事宜。己卯,諭土爾扈特投誠大臺吉均令來避暑山莊朝覲,命額駙色布騰巴勒珠爾馳驛迎之。壬午,致仕大學士陳宏謀卒。癸巳,命土爾扈特部眾暫駐博羅博拉。以金川土舍索諾木請賞給革布什咱土司人民,命阿爾泰詳酌機宜,毋姑息。

秋七月壬寅,阿爾泰等奏小金川土舍圍攻沃克什,命剿之。乙巳,命侍郎桂林帶銀一萬兩赴古北口會同提督王進泰賑水災。丙午,永定河決。丁未,命舒赫德署伊犁將軍。戊申,上秋獮木蘭。以小金川復侵明正土司,諭阿爾泰等進剿。丁巳,上奉皇太后啟鑾。癸亥,上奉皇太后駐避暑山莊。丙寅,以此次巡幸木蘭。沿途武職懈忽,楊廷璋、王進泰等均下部嚴議。

八月己丑,定邊左副將軍、喀爾喀扎薩克和碩親王成袞扎布卒,以車布登扎布為定邊左副將軍,額駙拉旺多爾濟襲扎薩克和碩親王。罷德福軍機處行走。庚寅,召大學士兩江總督高晉來京,查勘永定河工。命薩載兼署兩江總督。壬辰,永定河決口合龍。癸巳,上幸木蘭行圍。丁酉,命阿爾泰仍管四川總督事,召德福回京。

九月戊戌朔,停本年勾決。癸卯,命理籓院侍郎慶桂在軍機處行走。乙巳,土爾扈特臺吉渥巴錫等入覲,賞頂戴冠服有差。命副將軍溫福、參贊大臣伍岱赴四川軍營,會商進剿。辛亥,封渥巴錫為烏納恩素珠克圖舊土爾扈特部卓哩克圖汗,策伯克多爾濟為烏納恩素珠克圖舊土爾扈特部布延圖親王,舍楞為青塞特奇勒圖新土爾扈特部弼哩克圖郡王,巴木巴爾為畢錫哷勒圖郡王,餘各錫爵有差。甲寅,上回駐避暑山莊。丁卯,以文綬為四川總督,勒爾謹為陜西巡撫。調永德為廣西巡撫,梁國治為湖南巡撫,陳輝祖為湖北巡撫。

冬十月戊辰朔,以三寶為山西巡撫。己巳,上奉皇太后回鑾。以舒赫德為總統伊犁等處將軍,伊勒圖為塔爾巴哈臺參贊大臣,安泰為烏什參贊大臣。甲戌,宥紀昀,賞翰林院編修。乙亥,上奉皇太后還京師。己卯,高晉等奏桃源陳家道口河工合龍,上嘉之。命高晉、裘曰修、楊廷璋查勘南運河。丁亥,召楊廷璋為刑部尚書,以周元理為直隸總督,徐績為山東巡撫。甲午,陜甘總督吳達善卒,調文綬代之。

十一月己酉,董天弼奏攻取小金川牛廠。丙辰,上奉皇太后御慈寧宮,恭上徽號曰崇慶慈宣康惠敦和裕壽純禧恭懿安祺皇太后,頒詔覃恩有差。以溫福為武英殿大學士,兼兵部尚書,桂林為四川總督。丁巳,調素爾納為理籓院尚書,以舒赫德為戶部尚書。辛酉,皇太后萬壽聖節,上詣壽康宮,率王大臣行慶賀禮。壬戌,董天弼進攻達木巴宗,失利。甲子,小金川番復陷牛廠。

十二月庚午,溫福奏進駐向陽坪,攻小金川巴朗拉山碉卡,不克。桂林奏克小金川約咱寨。褫四川提督董天弼職,以阿桂署之。乙亥,蠲甘肅隴西等三十三州縣三十三年被水旱雹霜等災額賦。丙戌,以大金川酋僧格桑遣土目赴桂林軍營獻物,命給賞遣歸。己丑,溫福奏克巴朗拉碉卡。癸巳,溫福奏進駐日隆宗地方,董天弼收復沃克什土司各寨。

三十七年春正月辛丑,免奉天錦州二府額徵米豆。免浙江玉環、海寧兩縣額徵銀穀。免山西大同等二府額徵兵餉米豆穀麥,並太原等十四府州及歸化城各屬十分之三。壬寅,免和林格爾等處及太僕寺牧廠地畝額徵銀,並清水河額徵銀及太僕寺牧廠地畝額徵米豆十分之三。癸卯,刑部尚書楊廷璋卒,以崔應階為刑部尚書,嘉謨署漕運總督。乙巳,溫福奏攻克小金川曾頭溝、卡丫碉卡。丁未,桂林奏克郭松、甲木各碉卡。庚戌,以恆祿為內大臣。癸丑,建烏魯木齊城,駐兵屯田。癸亥,命尚書裘曰修協同直隸總督周元理濬永定河、北運河。

二月丁卯,以阿桂為四川軍營參贊大臣。甲戌,上幸盤山。丙戌,上回鑾,幸圓明園。丁亥,以色布騰巴勒珠爾為四川軍營參贊大臣。乙未,免陜西西安等十二府州上年額徵本色租糧。

三月丙申朔,免江蘇金壇等十一州縣六年至十年逋賦。戊戌,以索諾木策凌為烏魯木齊參贊大臣,德云為領隊大臣,命俱受伊犁將軍節制。乙巳,以豐升額為四川軍營參贊大臣。己酉,河南羅山縣在籍知縣查世柱,以藏匿明史輯要,論斬。壬子,桂林奏攻克大金川所據革布什咱土司之木巴拉等處。乙卯,溫福奏攻克小金川資哩碉寨。丁巳,桂林奏攻克吉地官寨。溫福奏攻克小金川阿克木雅寨。桂林奏攻克革布什咱土司之黨哩等寨,及小金川扎哇窠崖下碉卡。

夏四月丙寅朔,桂林奏攻克小金川阿仰東山梁等寨。豁甘肅節年民欠倉糧三百七十六萬石有奇。壬申,桂林奏盡復革布什咱土司之地,及攻克小金川格烏等處。諭溫福、桂林進剿索諾木。乙亥,授李湖雲南巡撫,圖思德貴州巡撫。壬午,改安西道為巴里坤屯田糧務兵備道,甘肅道為安肅兵備道,涼莊道為甘涼兵備道。裁烏魯木齊糧道。庚寅,賜金榜等一百六十二人進士及第出身有差。甲午,桂林攻小金川達烏東岸山梁,失利。

五月乙未朔,以溫福劾色布騰巴勒珠爾貽誤軍務,褫爵職。丙申,免直隸滄州等十五州縣積年逋賦。丁酉,以舒赫德為領侍衛內大臣。命福隆安赴四川查辦阿爾泰劾桂林乖張捏飾一案。命託庸暫兼管兵部尚書,索爾訥署工部尚書。壬寅,命戶部侍郎福康安在軍機處行走。癸卯,命海蘭察等赴四川西路軍營,鄂蘭等赴四川南路軍營。調容保為綏遠城將軍。桂林以隱匿挫衄,褫職逮問。以阿爾泰署四川總督。己未,上奉皇太后幸避暑山莊。甲子,湖廣總督富明安卒,以海明為湖廣總督,海成為江西巡撫。免直隸大興等十五州縣額賦有差。

六月乙丑朔,上奉皇太后駐避暑山莊。溫福等攻克小金川東瑪寨。諭阿桂督上中下雜穀及綽斯甲布各土司進剿金川。丁丑,蠲甘肅皋蘭等二十五縣旱災額賦。辛巳,盛京將軍恆祿卒,調增海代之。以傅玉為黑龍江將軍。甲申,調文綬為四川總督,海明為陜甘總督,以勒爾謹署之。命阿爾泰署湖廣總督。丙戌,阿爾泰罷,調海明為湖廣總督。以勒爾謹署陜甘總督,調富勒渾為陜西巡撫。命倉場侍郎劉秉恬赴四川西路軍營督餉。辛卯,湖廣總督海明卒,以富勒渾代之,陳輝祖署。命巴延三為陜西巡撫。

秋七月乙未,命刑部侍郎鄂寶赴四川南路軍營督餉,授勒爾謹陜甘總督。

八月己巳,阿桂奏攻克小金川甲爾木山梁碉卡。以阿桂為內大臣。賞布拉克底土司安多爾「恭順」名號,巴旺土婦伽讓「恭懿」名號。壬申,溫福等奏小金川賊襲瑪爾迪克運路,海蘭察等敗之。己丑,小金川犯黨壩官寨,阿桂遣董天弼援之。

九月壬寅,溫福奏進至木蘭壩,賊毀南北兩山碉卡,聚守路頂宗山梁。諭嚴防後路。阿桂奏綽斯甲布土司分兵進攻勒烏圍。上送皇太后回鑾。戊申,上自避暑山莊回鑾。甲寅,上奏皇太后還京師。

冬十月壬申,董天弼奏攻克穆陽岡等卡。壬午,阿桂奏攻克小金川甲爾木山梁。

十一月乙未,溫福等奏攻克路頂宗及喀木色爾碉寨。丙申,除四川樂山等九州縣三十五年坍廢鹽井額賦。辛丑,廣州將軍秦璜以納僕婦為妾,褫職逮訊。設涼州副都統。裁西安副都統一。丙午,溫福等奏克博爾根山等碉寨。戊申,阿桂奏攻克翁古爾壟等城寨。己酉,命富勒渾赴四川,以陳輝祖兼署湖廣總督。癸丑,阿桂奏攻克得裡等碉寨。丁巳,阿桂奏攻克邦甲、拉宗等處,拉約各寨番人降。

十二月癸亥,阿桂奏攻克僧格宗碉寨。癸酉,以溫福為定邊將軍,阿桂、豐升額俱為副將軍,舒常、海蘭察、哈國興俱為參贊大臣,福康安為領隊大臣,復興等為溫福一路領隊大臣,興兆等為阿桂一路領隊大臣,董天弼等為豐升額一路領隊大臣。賞給綽斯甲布土司工噶諾爾布「尊追歸丹」名號。丙子,溫福奏攻克明郭宗等碉卡。丁丑,阿桂奏攻克美諾碉寨。庚辰,溫福奏彭魯爾等寨番人就撫。辛巳,溫福等奏克布朗郭宗、底木達碉寨,澤旺降,僧格桑逃往金川。乙酉,秦璜以婪贓論斬。丙戌,授薩載江蘇巡撫。丁亥,文綬以袒徇褫職,命劉秉恬為四川總督,仍督餉,以富勒渾署之。

三十八年春正月壬辰,召永德來京,調熊學鵬為廣西巡撫,三寶為浙江巡撫。鄂寶仍授山西巡撫。以小金川平,緩四川官兵經過之成都等五十一州縣三十八年額賦及分辦夫糧之溫江等九十州縣三十七年蠲剩額賦。番民賦貢,一體緩之。溫福等進剿金川,分由喀爾薩爾、喀拉依、綽斯甲布三路進兵。甲辰,哈薩克博羅特使臣入覲。以阿爾泰婪贓,賜自盡。戊午,調永貴署戶部尚書,以阿桂為禮部尚書。

二月庚申朔,諭溫福等檄索諾木擒獻僧格桑。

三月庚寅朔,日食。壬辰,上詣泰陵。奉皇太后巡幸天津,免所過地方及天津府屬本年錢糧十分之三。癸巳,上閱永定河堤。丁酉,上謁泰陵。戊戌,上命簡親王豐訥亨奉皇太后自申昜春園啟鑾,免蹕路所經之宛平等二十州縣及天津府屬各州縣三十三年至三十六年逋賦。己亥,免直隸三十三年至三十五年逋賦。庚子,上閱澱河。乙巳,上奉皇太后駐蹕天津。己酉,上奉皇太后回鑾。免通州、寶坻等九州縣三十六年逋賦。壬子,上閱永定河。丙辰,上奉皇太后還京師。

閏三月己巳,以扎拉豐阿為御前大臣。命劉統勛等充辦理四庫全書總裁。乙酉,以素爾訥署工部尚書。

夏四月戊戌,以綽克托為烏什參贊大臣。庚戌,命索琳以署禮部侍郎在軍機處行走。辛亥,命慶桂以理籓院侍郎、副都統為伊犁參贊大臣。丙辰,諭高晉賑清河等州縣及大河、長淮二衛被水災民。戊午,加大學士溫福、戶部尚書舒赫德、工部尚書福隆安太子太保,禮部尚書王際華、工部尚書裘曰修太子少傅,禮部尚書阿桂、署兵部尚書豐升額、直隸總督周元理、閩浙總督鍾音、四川總督劉秉恬太子少保。

五月辛酉,工部尚書裘曰修卒,以嵇璜代之。丙寅,上奉皇太后啟鑾,免經過地方本年錢糧十分之三。壬申,上奉皇太后駐蹕避暑山莊。乙亥,盛京將軍增海卒,調弘晌代之。丁丑,改烏魯木齊參贊大臣為都統,以索諾木策凌為之,仍聽伊犁將軍節制。己卯,猛遮土目叭立齋等內附。癸未,召車布登扎布來京,命拉旺多爾濟署烏里雅蘇臺將軍。乙巳,阿桂等奏金川番賊陷喇嘛寺糧臺,襲據底木達、布朗郭宗。己酉,鄂寶奏金川番賊襲據大板昭。壬子,定邊將軍溫福、四川提督馬全、署貴州提督牛天畀敗績於木果木,俱死之。癸丑,以阿桂為定邊將軍,贈溫福一等伯。小金川酋僧格桑父澤旺伏誅。大學士劉綸卒。甲寅,以富勒渾為四川總督,起文綬為湖廣總督。丙辰,阿桂奏剿洗小金川番賊,盡毀碉寨,諭嘉之。

秋七月戊午朔,召舒赫德來京,以伊勒圖為伊犁將軍,慶桂為塔爾巴哈臺參贊大臣。己未,金川番賊陷美諾、明郭宗,海蘭察退師日隆。諭阿桂由章谷退師,豐升額退駐巴拉朗等處。癸亥,命富德為參贊大臣赴軍營,命阿桂撤噶爾拉之師。甲子,命舒赫德為武英殿大學士。調阿桂為戶部尚書,永貴為禮部尚書。丙寅,齊齊哈爾蝗。丁卯,以溫福乖方僨事,革一等伯爵,仍予恤典。褫劉秉恬職。命議恤木果木陣亡提督馬全、牛天畀,副都統巴朗、阿爾素納,總兵張大經及各文武員弁。丙戌,諭阿桂先復小金川,分三路進剿。

八月戊子,以阿桂為定西將軍。命於敏中為文華殿大學士,舒赫德管刑部,劉統勛專管吏部。己丑,命程景伊協辦大學士。調王際華為戶部尚書,蔡新為禮部尚書,嵇璜為兵部尚書。以閻循琦為工部尚書。戊戌,以明亮為定邊右副將軍,富德為參贊大臣。壬寅,上幸木蘭行圍。

九月壬戌,降海蘭察為領隊大臣。甲子,上回駐避暑山莊。戊辰,上送皇太后回鑾。己巳,索諾木挾僧格桑歸大金川,以其兄岡達克往美諾。諭阿桂乘機收復。允戶部請開金川軍需捐例。壬申,上自避暑山莊回鑾。甲戌,以多敏為科布多參贊大臣,車木楚克扎布為烏里雅蘇臺參贊大臣。戊寅,上奉皇太后還京。庚辰,吏部尚書託庸致仕,調官保為吏部尚書。以英廉為刑部尚書,仍兼管戶部侍郎事。

冬十月乙巳,和碩諴親王允祕卒。己酉,褫車布登扎布定邊左副將軍職,仍留親王銜,以瑚圖靈阿代之。

十一月丁卯,阿桂等奏進剿小金川,攻克資哩山梁等處,收復沃克什官寨。戊辰,命福祿往西寧辦事。召伍彌泰回京。己巳,阿桂等奏克復美諾,命進剿金川。辛未,軍機大臣、大學士劉統勛卒,上親臨賜奠,贈太傅。壬申,召梁國治來京,在軍機處行走。調巴延三為湖南巡撫。以畢沅為陜西巡撫。癸酉,明亮等奏克復僧格宗等碉寨。

十二月癸巳,以彰寶為雲貴總督。辛丑,命李侍堯為武英殿大學士,仍管兩廣總督事。

是歲,朝鮮、安南來貢。

三十九年春正月丙子,以姚立德為河東河道總督。丁丑,阿桂等克贊巴拉克等山梁。

二月甲申朔,命豐升額等助阿桂進攻勒烏圍。丁亥,明亮等奏克木谿等山梁。戊戌,豐升額等克莫爾敏山梁。乙巳,蠲江蘇山陽等十州縣衛三十八年水災額賦有差。丁未,上詣東陵,並巡幸盤山。庚戌,謁昭西陵、孝陵、孝東陵、景陵,至孝賢皇后陵奠酒。臨故大學士公傅恆塋賜奠。辛亥,上駐蹕盤山。

三月庚申,阿桂等克羅博瓦山梁,加阿桂太子太保,以海蘭察為內大臣,額森特為散秩大臣。甲子,上幸南苑行圍。辛未,阿桂等克得斯東寨。庚辰,明亮等克喀咱普等處,上嘉賚之。

夏四月乙酉,順天大興等州縣蝗。辛亥,以京師及近畿地方旱,命刑部清理庶獄,減軍流以下罪,直隸如之。戊戌,以御史李漱芳劾福隆安家人滋事,上嘉之,予敘。

五月癸丑朔,命刑部減秋審、朝審緩決一二次以上罪。丙寅,彰寶以病解任,以圖思德署云貴總督。戊辰,上奉皇太后秋獮木蘭。甲戌,上奉皇太后駐蹕避暑山莊。

六月癸卯,阿桂等奏克穆爾渾圖碉卡。

秋七月甲寅,阿桂等克色淜普山碉卡。己未,阿桂等克喇穆喇穆山等碉卡。壬戌,阿桂等克日則雅口等處寺碉。乙丑,烏魯木齊額魯特部蝗。庚午,明亮等克達爾圖山梁碉卡。甲戌,以於敏中未奏太監高雲從囑託公事,下部嚴議。以阿思哈為左都御史。乙亥,命阿思哈在軍機處行走。太監高雲從處斬。辛巳,阿桂等克格魯瓦覺等處碉寨。

八月壬午朔,日食。壬辰,富德等克穆當噶爾、羊圈等處碉卡。丁酉,上幸木蘭行圍。癸卯,金川頭人綽窩斯甲降,獻賊目僧格桑尸。

九月乙卯,山東壽張縣奸民王倫等謀逆,命山東巡撫徐績剿捕之。丁巳,命大學士舒赫德赴江南,同高晉塞決口。戊午,上回駐避暑山莊。命舒赫德先赴山東剿捕王倫。庚申,命額駙拉旺多爾濟、左都御史阿思哈帶侍衛章京及健銳、火器二營兵,往山東會剿王倫。辛酉,王倫圍臨清,屯閘口。壬戌,上送皇太后回鑾。癸亥,以天津府七縣旱,命撥通倉米十萬石備賑。丙寅,上自避暑山莊回鑾。丁卯,山東兗州鎮總兵惟一、德州城守尉格圖肯以臨陣退避,處斬。庚午,以江蘇山陽等四縣水災,命免明年額賦。壬申,上奉皇太后還京師。丙子,山東臨清賊平,王倫自焚死。

冬十月辛巳朔,以楊景素為山東巡撫。壬辰,免臨清新城本年未完額賦,並舊城未完額賦十分之五。丙午,以徐績為河南巡撫。

十一月癸丑,明亮等克日旁等碉寨。甲寅,以舒赫德為御前大臣。阿桂等克日爾巴當噶碉寨。以阿桂為御前大臣,海蘭察為御前侍衛。丙辰,以四川成都等一百四十府州縣行軍運糧,免歷年額賦有差。戊辰,阿桂克格魯古丫口等處碉寨。

是歲,朝鮮、琉球來貢。

四十年春正月甲戌,阿桂等克康爾薩山梁。

二月己卯,阿桂等克甲爾納等處碉寨。丙戌,阿桂克斯莫思達碉寨。癸巳,以李瀚為雲南巡撫。

三月辛亥,上幸盤山。甲寅,上駐蹕盤山。蠲江南句容等十九州縣,淮安、大河二衛三十九年水旱災額賦。壬申,蠲長蘆屬滄州等六州縣、嚴鎮等六場,河南信陽等五州縣三十五年旱災額賦。

夏四月戊寅朔,蠲安徽合肥等十四州縣、廬州等四衛三十九年旱災額賦。丙戌,四川軍營參贊大臣、領侍衛內大臣、和碩親王、固倫額駙色布騰巴勒珠爾卒。己丑,命明山為烏里雅蘇臺參贊大臣。壬寅,賜吳錫齡等一百五十八人進士及第出身有差。癸卯,阿桂等克木思工噶克丫口等處城碉。明亮等克甲索、宜喜。乙巳,明亮等克達爾圖等處碉寨。以明亮、福康安為內大臣。

五月己酉,蠲直隸霸州、保定等三十九州縣三十九年旱災額賦。甲寅,阿桂等奏克巴木通等處碉卡。丁巳,明亮奏克茹寨、甲索等處碉卡。戊辰,阿桂等奏克噶爾丹等碉寨。壬申,上幸木蘭,奉皇太后駐湯山行宮。明亮等奏克巴舍什等處碉寨。乙亥,阿桂等奏克遜克爾宗等處碉寨。加封定邊右副將軍、果毅公豐升額為果毅繼勇公。

六月丁丑朔,蠲湖北漢陽等十五州縣、武昌等六衛一所三十九年旱災額賦。戊寅,上駐避暑山莊。癸未,上詣廣仁嶺萬壽亭迎皇太后駐蹕避暑山莊。壬辰,以豐升額為兵部尚書。丙申,領隊大臣額爾特褫職逮治。庚子,設管理烏魯木齊額魯特部落領隊大臣,以全簡為之。

秋七月壬戌,阿桂等奏攻克昆色爾等處山梁碉寨。丁卯,阿桂等克章噶等碉寨。額洛木寨頭人革什甲木參等率眾來降。庚午,蠲甘肅皋蘭等七州縣三十九年被水被旱額賦。阿桂等克直古腦一帶碉寨。

八月丙子朔,日食。丁丑,阿桂等克隆斯得寨。明亮等克扎烏古山梁。己卯,以霸州等三十餘州縣被水,撥直隸籓庫銀五十萬兩賑之。辛卯,上幸木蘭行圍。己亥,阿桂等奏克勒烏圍之捷,進剿噶喇依賊寨。上命優敘將軍阿桂,副將軍豐升額,參贊大臣海蘭察、額森特等功。辛丑,召舒赫德赴熱河行在。癸卯,封羅卜藏錫喇布為貝子。乙巳,命侍郎袁守侗等赴貴州,讞知府蘇墧稟揭總督、籓、臬袒護同知席纘一案。

九月庚戌,蠲湖北鍾祥等十二州縣並武昌等七衛三十九年旱災額賦。癸丑,上回駐避暑山莊。丁巳,上送皇太后回鑾。辛酉,以圖思德劾蘇墧浮收勒索,命袁守侗等嚴鞫之。丙寅,以明亮請赴西路失機,嚴斥之,仍奪廣州將軍。丁卯,上奉皇太后還京師。阿桂等克當噶克底等處碉寨。

冬十月己卯,召駐藏辦事伍彌泰,以留保住代之。己丑,以霸州等六州縣被災較重,命即於閏十月放賑。庚寅,蠲甘肅皋蘭等十七州縣水雹霜災額賦。壬辰,上還宮。丙申,調裴宗錫為貴州巡撫,命袁守侗暫署,圖思德署云南巡撫,李質穎為安徽巡撫。

閏十月壬子,蘇墧以侵稅誣訐,處斬。壬戌,明亮等奏克扎烏古山梁。甲子,阿桂等奏克西里山黃草坪等處碉卡,總兵曹順死之。命袁守侗赴四川,同阿揚阿讞冀國勛一案。復封慶恆為克勤郡王。壬申,明亮等克耳得穀寨。

十一月,明亮等克甲索諸處碉卡。乙酉,福祿以立塔爾一案未能鞫實,革,戍伊犁。己丑,阿桂克西里第二山峰,並進圍鴉瑪朋寨落。壬辰,明亮等奏攻得克爾甲爾古等處碉卡。壬寅,阿桂等奏克舍勒固租魯、科思果木、阿爾古等處碉寨。

十二月甲辰朔,日食。丁未,工部尚書閻循琦卒,調嵇璜為工部尚書,蔡新為兵部尚書,以曹秀先為禮部尚書。阿桂等克薩爾歪等寨落。丙辰,以阿桂為鑲黃旗領侍衛內大臣。調熊學鵬為廣東巡撫,以吳虎炳為廣西巡撫。甲子,明亮等由達撒穀進兵,連克險要山梁及沿河格爾則寨落。丙寅,阿桂等克格隆古等處寨落。庚午,阿桂等由索隆古進據噶占山梁,直搗噶喇依。其頭人色木里雍中及布籠普阿納木來降。壬申,明亮等克甲雜等隘口,並後路巴布里、日蓋古洛,進抵獨松隘口,剋日會搗噶喇依。其頭人達固拉得爾瓦等來降。


\end{pinyinscope}