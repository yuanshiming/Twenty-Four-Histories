\article{列傳二百九十八}

\begin{pinyinscope}
列女四

長山鋪烈婦胡二妻唐之坦妻曹李岸妻焦方引妻毛

林其標妻韓馮云勷妻李曹邦傑妻張林守仁妻王

張四維妻劉李長華妾吳周兆農妻王陳國材妻周

吳廷望聘妻池李正榮聘妻霍項起鵠妻程於某妻蔡

張義妻李黃敬升妻王伊嵩阿妻鈕祜祿氏張廷桂妻章

郝某妻單陳廣美妻李賀邦達妻陸鄭宗墩妻陳任有成妻陳

丁三郎妻丁採芹妻孫王如義妻向狄聽妻王林邦基妻曾

錢瀞甫妻汪謝作棟妻王繆文鬱妻邱黃壽椿妻管

馮桂增妾李黃翥先妾彭方恮妻趙姚森桂妻宋

惲毓華妻莊弟毓德妻許侄寶元妻袁曲承麟妻袁尹春妻張

李氏陳三義妻王游開科妻趙孫崇業妻金張某妻田

張氏女湯氏女滄州女張氏孫大成妻裔楊某聘妻章

孟黑子妻苑北塘女藍某妻芮氏女樂某妻左蕭氏

黃氏女吳氏女顧氏張氏許會妻張趙海玉妻任殷氏

嘉興女王某妻李何先佑妻孫邢氏遷安婦白鎔妻尹

林氏洪某妻徐敖氏塗氏吳氏楊氏趙氏王氏許氏

梅氏張氏秦某妻崔李某妻管王某妻徐陳潛聘妻崔

硃承宇妻曹陳有量妻海樊廷桂妻張

李有恆聘妻楊陳某妻劉埜妻李曲氏女宋氏五烈女

龔行妻謝女巧楊文龍聘妻孫梁至良妻鄭郭進昌妻李

龔良翰妻陳王均妻湯李氏女翠金張元尹妻李張檢妻顏

萬某妻曾李繼先妻侯田氏女馬某聘妻苗

高日勇妻楊羅季兒妻劉氏女鍾某妻蔡段舉妻盧

王某妻劉張良善妻王李青照妻張姚際春女

王敦義妻張陳維章妻陸何氏女謝亞煥妻王

張樹功妻吳郭某妻李趙謙妻王

郭氏女何氏女沈鼎猷妻嚴鐵山婦

汪氏女賀氏女馮光琦女郭君甫妻吳

黃聲諧妻王徐惟原妻許柯叔明妻鞏胡某妻裘陳儒先妻李白洋女

高氏婦段吳考女曹氏女劉廷斌女張氏女孫嫗

陳氏婢邱氏婢董氏任氏盧尚義妻梁白氏王氏

秦士楚妻洪張氏婢楊氏婢江貴壽妻王張祿妻徐

任氏婢鄭氏女王氏婢徐氏女丁香江金姑羅氏

隴聯嵩妻祿者架聘妻直額羅廷勝妻馬羅朝彥妻劉

安於磐妻硃後妻田田養民妻楊李任妻矣鄂對妻熱依木

索諾木榮宗母麥麥吉堅參達結妻喇章次妻夭夭沙氏女

嘉義番婦施世燿妻苗

長山鋪烈婦,無姓氏,不知何許人。李自成南奔,驅荊、襄之民以從,婦與其夫俱被掠。行至江夏長山鋪,其夫道殕,婦僅餘一珥,出以乞人求瘞其夫,有少年應焉。瘞既,竟欲強其婦從去,婦入穴枕其夫慟哭,觸顙流血,以土自掩,曰:「乞並瘞我!」眾挽之不起,日暮,風雨至,乃委去。平明往視,則血被面死矣,眾因並瘞之。

胡二妻,失其氏,吳洞庭人。婦父,舟人;胡二,農也,有母,兄若弟皆別居。婦與二曰:「吾夫婦各減數口食,猶足以飽母,有如母但一子,不獨養,又誰養乎?」夫婦忍饑養母,時時具甘脆。母喪,求地以葬。夫婦勤,歲倍收,始有居室,而二病瘵。鄉好鬼,婦獨不信,奔走醫藥。二病甚,婦曰:「我聞糞苦者生,甘者死。」嘗之而甘,二竟死,無子。婦計兄公一子,叔二子,詣叔,匄其次為後,姒不可。居數月,兄公舉次子,又詣兄公曰:「吾女三歲,乳未盡,今兄公舉次子,天其欲使吾夫得有後乎?」兄公頷之。婦歸語父,貸百錢,將祀其夫告立後。其父欲嫁婦,不許,且罵之,兄公亦中悔,婦乃自經夫柩側,時康熙五年十二月。明年,縣人黃中堅等為斂錢,與其夫合葬。

唐之坦妻曹,海寧人。康熙十五年秋,之坦卒,曹矢死,治衣衾必有副。食砒,不死;屑錢吞之,又不死。既斂,復飲滷,吐下而解;乃不食二十二日,夜投舍傍池,家人出之,死矣,頃復蘇。曹謂其舅、姑及母曰:「大人愛我,乃苦我也!」於是復飲食,操作如常,織自制衣一稱,婢乞餘布,不與。家人竊議曰;「數尺布,尚惜之,宜不死矣!」及冬,黃梅方花,曹視而嘆,為賦詩,美其不落,復不食。至歲除,出餘布縊之坦柩旁,乃死。

李岸妻焦,睢州人。姑嚴,織紉炊舂皆焦任之。岸卒,方斂,焦縊,遇救;比葬,再縊,再遇救,乃操作如平時。卒哭,拜墓歸,復縊,乃死。

方引妻毛,遂安人。父際可,為祥符知縣,而引父象瑛官編修。引病瘵,自京師詣河南,既婚,未三日卒。家人聞毛許引死也,閑之密。一日,登樓自擲墜地,嘔血,絕復蘇,遂歸於方氏,促為引營葬。久之,地始定,葬有日,於是謂其人曰:「吾葬當同是日也!」遂不食,家人喻之百端,起辭祖姑及舅及母皆四拜,終不食,十九日乃卒。時康熙二十九年二月癸亥朔,距引喪十年。

林其標妻韓,福清人。其標貧,依姊居,鬻餈自給。鄰媼乞之粟,韓曰:「是必償!」其標病,韓代鬻餈,垂蘆簾自蔽。少間,析麻苧為布,以易米若藥。其標語韓曰:「吾以貧累汝,終且以死累汝!吾死,汝自為計。」韓痛絕不能語。其標死,韓告其姊曰:「乞辦兩棺,並覓一抔土,俾夫婦相依!」盡散器物償鄰媼,遂自經。

馮云勷妻李,武定人,大學士之芳女。李年十五,適云勷。事舅姑謹,立侍竟日,無怠,命坐則坐,命退則退。之芳督浙江,當耿精忠叛,駐軍衢州,傳語洶洶,李獨謂賊不足平,坦然無懼。雲勷卒,無子,李方舉次女,矢死,遂不飲食。其兄延醫,手為調藥,拒不納。越數日,令侍者扶行,傍柩側,遽絕。

曹邦傑妻張,鎮寧州人。邦傑早卒,張為文以祭,曰:「嗚呼!痛妾命之不辰也。幼失嚴慈,撫育無人,形影伶仃,莫可言狀!幸得於歸夫子,庶幾夙夜事之,百年守之。憶吾父擇婿時,亦曰:『吾女幸矣,終身之仰望者非婿耶?如賓如友,同心而同德者非婿耶?』私心自慶,在妾尤深。孰意甫歸故里,遽嬰疢疾,妾向之喜者,化而為憂,忘餐廢寢,祈以身代。而天不假年,黃粱一覺,羽化升矣,傷心哉!夫子之人,如金如玉,夫子之文,如海如潮。而今巳矣,不可復見矣!天耶人耶?孰為之耶?禮稱未亡人,妾不忍未亡也。詩云:『之死靡佗。』妾惟知之死也。九原匪遠,妾必從之。嗚呼!淒淒惻惻,踽踽涼涼。拊膺呼號,瞻望無將。臨風灑涕,對景悲傷。削骨代筆,曷罄衷腸!夫子乎,其知之乎?何不飆輪少待,使妾欲追而難跡乎?靈其不寐,庶鑒妾心。」邦傑死三日,張遂殉,康熙三十七年事也。

林守仁妻王,侯官人。守仁以優貢生客死京師,無子,女汀哥,前室出也。王矢死。逾年,守仁喪還,王治喪竟,一日,為汀哥制履成,嘆曰:「生一日,當作一日事。」因語汀哥曰:「母去,兒無恐,但歲時具杯酒,一脡肉,母當歸,不相哧也。」頃之,午食竟,入室自經,藏香屑袖中,解尸氣也。

張四維妻劉,四維,錢塘人;劉,漢軍,失其所隸旗。四維父商於廣東,挈四維以行。劉父官潮州知府,見四維幼慧,因與論婚。四維父喪其資斧,而四維長多病,遂跛,劉父母欲別擇婿,劉矢死,父母莫能奪,乃召四維就婚。劉既失父母歡,姊婿達官子,相侮,劉勸四維挈以歸。劉辭父母,奩具一不取,勤苦作畫刺繡易薪米,四維亦力學,舉於鄉。康熙五十九年,四維試禮部,不第,卒於京師,劉聞,遂殉焉。

李長華妾吳,長華,鄆城人;吳,封丘人。幼孤,為人賣入娼家,矢死不從,其兄贖以歸,為長華妾。長華以選人客京師,居八年,貧病死,其友檢討孫勷為具斂,吳飲鴆,勷往救,誡毋死,待長華子迎喪。後十餘日。長華子迎喪至,知其事,亦勸毋死,且將以其子為之孫,吳即夕自經死。勷葬長華廣寧門外真空寺側,以吳祔。

周兆農妻王,長沙人。兆農樵於山,大風拔木,被創死。遺腹生子,母家憫其貧,勸改適。王拜姑,泣而言曰:「兒不孝,敢以呱呱者累老人!」語未竟,大慟。姑知其且死,夜與俱寢,稍寐,聞有異,呼家人蹋戶入,火之,見王頭系於床,右手握拳,爪陷掌,左手指床上兒。死時年十九。

陳國材妻周,江寧人,居揚州。歸國材逾月,遽卒,周日夕居喪次,誓從死。籍遺財授其族子曰:「明年寒食,以一卮酹我夫婦。」其父往慰喻之,周曰:「兒無舅姑,無子,客居無所依,義當死,父勿誤兒!然兒死不忍傷肢體。」遂吞金環二,不死;時周羸甚,餌大黃,冀暴下死,反下所吞金環。乃不飲食,七日,猶坐語;又數日,眸陷欲枯,目光注國材棺不轉,兩手據席爬搔,席草寸寸碎裂。不飲食二十日,雍正九年三月癸未卒,距國材死五十有一日。縣人為葬孫大成妻裔墓側。其先又有烈女池、霍,四塚比立如鱗次。

吳廷望聘妻池,江都人。廷望從軍戰死,廷望父欲以妻其幼子,使其從母喻意,池不可,自經。

李正榮聘妻霍,甘泉人。生十九年,事父母孝。許字正榮,才十日,而正榮卒。霍號慟自殺。二女之葬,提督學政、右中允楊中訥為之銘曰:「蜀岡之巔,平山之側,鬱乎蒼蒼,憑高西望而嘆息。曰有同縣二烈女,此其幽宅。」裔自有傳,葬在池、霍后。

又有項起鵠妻程,亦揚州人。程嫁三月,起鵠行賈,死廣西,訃聞,程自經。州人葬其側,合為祠,號「五烈」。

於某妻蔡,名貞仙,金壇人。年十九,將嫁而婿病,卜者言:「迎婦吉。」貞仙母難之。貞仙請於母曰:「彼欲已病而違之,非義。」乃行,而婿病不起。及斂,納釵一、釧一於棺,自經棺旁,救不死;諷姑為翁置媵,姑從之,且使主家事。忌者譖之,因辭於姑,忌者遂言是且有他志,乃矢死。取所讀書、所為詩詞盡焚之。釵於髻,釧於腕,旦起襲故衣,問安於姑所,辭色如常時。午侍食,既撤,入室縊。時乾隆二年六月壬戌,年二十五。貞仙有從父嘗過視貞仙,問曰:「聞舅姑以譖常挫汝,有之乎?」對曰:「否,古賢婦未有訟其舅姑者,即死,毋有他言。」

張義妻李,交城人。義坐罪當斬,免死,遣廣西義寧,李與偕。義死遣所,李具棺以斂,以遺金上縣。至夕,呼鄰媼共宿,俟其熟寐,赴水死,時乾隆五年九月辛未。縣具其事上巡撫,巡撫以聞,下禮部,禮部議:「殉夫者令甲有明禁,惟李以從夫罪遣,孤殉節,非激烈輕生比,請旌表。」得旨:「依議。」

黃敬升妻王,昆山人。敬升貧,客授,王佐以績,食不足,制闢蚊藥,鬻諸市。敬升病疫,一日門不啟,鄰人壞垣入視,敬升死於床,王死床下,兒臥地號,胸系王書,略言:「貧不能斂其夫,食制藥紅砒以殉,冀有惻隱者,斂夫育兒,身填溝壑不恨!」有士人為斂其夫婦,將兒去,育以長。

伊嵩阿,拜都氏,滿洲鑲黃旗人;妻希光,鈕祜祿氏,正白旗人,總督愛必達女也。伊嵩阿為大學士永貴從子,早卒。方病時,希光割股進,終不起,許以死。愛必達、永貴共喻之,誓畢婚嫁乃殉。為伊嵩阿弟娶,嫁女妹及二女,次女行之明日,自縊死。張遺詩於壁,略謂:「十載要盟,此日當報命。」乾隆四十六年三月事也。永貴疏聞,高宗為賦詩,旌其節。

張廷桂妻章,名孔榮,廷桂,常熟人;章,秦安人。廷桂父為吏陜西,初娶魏,其父宰秦安,廷桂從焉。既歸,避事,復游秦安,因贅於章。居八年,事解,乃以孥還。廷桂貧,恆出客游,卒於撫寧。喪歸,章為營葬。既窆,將自投穴中,為家人所持。章一女字催鳳,廷桂從弟廷梅,許生子為立後,乃依廷梅居。廷桂嘗入貲牒授主簿,或購其牒,章曰:「吾以貧鬻牒,罔國家,罪也,況夫名,其可二耶?」遂焚其牒。既終喪,復自經。家人覺,解之。次日既夕,赴水死。

死時為絕命辭數章,詞旨哀惻,其卒章曰:「憶往事兮,雙淚沾巾。想當年兮,妾病沉昏。感君愛兮,信誓殷勤。云妾歿兮,君必亡身。嗟今日兮,命不由人。君先亡兮,妾豈偷存!痛萬里兮,生會無因。輕一命兮,地下從君。求神明兮,引我孤魂。覓天涯兮,不惜艱辛。得伴君兮,死亦歡欣。十七年兮,夫婦深恩。食糟糠兮,敢怨君貧!中路訣兮,命蹇時屯。喪葬畢兮,不死何云?傷幼女兮,失母誰親!死為君兮,此外奚論?」

又留書與廷梅曰:「初聞訃,即欲死,念無後,無人主喪葬。今服除,死更無餘事。前議叔生子為立後,毋誑我!家貧,止田十四畝,當以十畝與所後子,四畝與催鳳,遺十金為我埋先夫塋次。」催鳳旋殤,廷梅亦不為立後。後二十年,縣諸生柏渭、吳慶長等始為合葬。

郝某妻單,永寧人。郝奇醜,眇小,且跛,一目,口不能言。御小車,遂呼曰小車,而單美,鄰婦恆訕焉。單曰:「夫可憎乎?吾命也,請勿再言!」單躬紡績,養舅姑,育子。舅姑死,鬻所居破屋以葬。嘗數日不舉火,族人憫之,予蕎麥數斗制餅以鬻,分其餘以飽。乾隆五十年,歲饑,單為鄰婦佐女紅,貸餘食食夫及子。逾年,夫疫死,子亦殤,單裂席裹尸,以木杴掘坎瘞焉。杴折,手捧土,瘞畢,血殷地。乃號曰:「天乎!單氏事畢矣,而猶生乎?」坐破窯中,餓數日死,年二十六。族人瘞之夫側,里稱賢婦墓云。

陳廣美妻李,河內農家女也。生二十四年而歸廣美,廣美已病,李與異室居,侍疾甚謹。事舅,日具饍甚恭。閱三月,廣美死。母往視之,且語之曰:「兒雖嫁,猶處子也,何患無佳婿?」李誓不更適。葬之明日,出廚刀,囑舅礪焉。曰:「為翁作面,虞其鈍也。」其舅竟礪以授李,李闔戶。其舅知李且死,排戶入,見李猶立,右手握刀,首墮負於背,幾不屬,血從鬢間溢,殷地。其舅疾呼,族鄰畢至,其母亦至,乃僕。李死嘉慶五年四月丁未。

賀邦達妻陸,震澤人。待年於賀氏。邦達病,舅姑用卜人言,使成婚,逾月而邦達死。或語其舅姑:「婦雖婚,猶處子也,盍為擇婿?」陸聞,集族姻出拜,誓毋貳。居三年,語姑曰;「我夜數夢吾夫,豈魂魄常從我耶?」遂入室自縊死。時嘉慶十六年四月辛酉,陸年十九。

鄭宗墩妻陳,名淑定,長樂人。宗墩客他縣,舁病歸,卒,無子。陳求死,父喻止之。陳力織,葬姑及宗墩。舅以居隘,命歸依父。嘉慶二十五年,父卒,還省舅。退告叔弟曰:「兄歿十二年矣,未亡人懼傷吾父心,久而不死。今已矣,舅老,有叔在。叔能以子為兄後,兄其瞑乎!」遂縊。

任有成妻陳,蕭山人。有成無昆弟,賈諸暨,卒,亦無子。舅姑命歸母家,將徐奪其志,陳矢死不可。力積貲為舅卜妾吳,逾年而有子。舅姑卒,陳與吳居,育夫弟。

錢儀吉為作二陳傳,謂:「當死生危苦之際,進退合度,得禮意云。」

丁三郎妻,失其氏,宜興人。嫁逾年,夫死,不哭亦不拜,家人莫測也。後四十九日,既奠,婦出就案前立,視其主,久之,拜,拜時若呼三郎,遂伏地不能起,掖之,則已死。

丁採芹妻孫,震澤人。嫁半年,採芹病瘵,舅姑謂婦命兇,詬罵之。孫飲泣,脫簪珥,具湯藥。採芹病日篤,謂孫曰:「我且死,所不能瞑目者為汝耳。汝無子,家貧,母家亦無可依,當奈何?」孫泣曰:「我念之熟,恐戚君,故不敢言。人孰不死,死貴得所,當先待君地下耳!」採芹垂泣不答,孫乘間自縊,道光六年四月也。採芹乃扶病而拜曰:「從我於既死,不若殉我於將死,烈哉!」三日採芹亦死。

王如義妻向,涪州人。幼能為詩文。如義,農家子,向恆勸之讀。道光十六年,如義暴卒,姑喻之嫁,矢以死。舅病,為刲股。家益貧,將強遣之,二十三年三月戊申,自沉荷花灘死。將死,為絕命詩十首,其序曰:「妾涪陵向氏女,適王氏,未一年,而夫即世。昨歲翁又不幸。孤苦煢獨,人勸以非禮,衣食事小,名節事大,惟一死以明志。夜題詩十首,藏笥中,他日閱妾詩,毋累阿姑也!」及入水,粘一紙橋柱,書五字,曰「名節江中見」,死時年二十五。

狄聽妻王,名甥。聽,溧陽人,道光九年進士,官至廣西道監察御史;王,江陰人。十九年七月,聽卒官,八月,子驄殤,九月丁巳,王縊。王幼承父蘇教,通經史大義,能詩。將殉,作書告聽諸同歲,略言:「夫亡當即死,諸君俱言撫孤重,故未敢爾。孤又夭,復何言?念兩世單傳,不可無後,今已立後,可報舅、姑、夫子地下!」王嘗撫從女,年十七,已許字,留金囑遣嫁。又諭所後子,期明年以喪還葬,與前母三棺同穴,以殤祔。並令斥貲佽祖祠,成父志。書末題曰:「我自歸家去,人休作烈看。」康熙間錢塘林邦基妻曾所為絕命詩也。

曾,名如蘭,邦基卒,曾立其兄子為後,葬舅姑畢,具牒上縣請死,知縣慰止之。後十日,題辭,吞金殉。

錢瀞甫妻汪,武進人。善女工,所入足自給。而瀞甫博,傾其貲。其姑嚴,雖寒餓不敢告也。夜風雪,家人皆臥,薄絮衣篝燈守後戶,待瀞甫。嘗以除夕跪而諫:「無更博。」瀞甫為少止。後客死餘干,汪請立後,所當立者不可汪意,乃勿復言。葬畢,自經死。未死前一日,以十碗致某醫,曰:「我為人無所受恩,惟是人嘗診我,以是償也。」

謝作棟妻王,孟津人,王家白鶴鎮,作棟家南硃村。作棟卒,王將殉,祖姑及舅姑勉以撫孤。王朝夕奠,必抱其孤拜,哭涕如雨。祖姑聞之,為輟食,王乃飲泣,不敢聲。喪終,其孤殤,祖姑亦歿,王歸訣父母,父母慰喻之。道光二十二年四月辛巳,作棟死三期,先日王哭於墓,誓死。晡,盡以衣物與二女妹,夜中縊。晨,眾蹋戶入,一鐙置高處,照屋梁,板障其外,王內衣皆密紉,貌如生。

繆文鬱妻邱,吳江同里人。同里有敝俗,歲二、三月祠劉猛,將輿以出,少年傅粉墨為婦人,參錯儀衛。聞文鬱故磨豆家傭,與其役。日昳過門,女伴呼邱出觀,邱以為恥,恚,闔戶。文鬱歸,戒毋更出。越宿,文鬱病,或恫以「神怒,且死」。邱曰:「聰明正直為神,豈以茫昧致人死者?吾夫未即死,即死,吾與俱死耳!」數日,文鬱竟死,邱迎母與居。三日,語母入市市楮,邱自縊柩側。

黃壽椿妻管,壽椿,江蘇華亭人;管,陽湖人,父光烈,母林,皆死寇。壽椿官江西德安典史,光緒二年,卒。時壽椿父如琳官浙江上虞梁湖巡檢,管將壽椿喪挈子女以歸。至曹娥江,距梁湖一日程,遣子女先行謁祖父母,管飲藥死。

馮桂增妾李,桂增,臨朐人;李,肅州農家女。桂增從左宗棠討叛回馬四,軍其地,納焉。桂增會師新疆,李留肅州,與部曲諸婦居。李御諸婦有法度,諸婦憚之,若部曲之事其帥。光緒二年正月,桂增克瑪納斯城,軍寡,為賊所乘,戰死。李方有身,日夜哭。既生子,逾年殤。桂增喪還,李迎奠喪甚慟,須臾僕,不語。視之,死,蓋先時已仰藥也。

黃翥先妾彭,翥先,鍾祥人;彭,貴築人,先為田興恕婢。興恕戍新疆,寄家秦州,翥先方知秦州,得彭以為妾。光緒二年,宗棠駐軍秦州,翥先為主計,四年,卒。彭悉發篋,以衣物屬翥先子,吞金死。

方恮妻趙,陽湖人。祖母方,節婦。父烈文,嘗知易州,有文行。歸恮,食貧,持門戶。光緒四年,恮客游,遽卒。趙方有身,烈文迎以歸,徐告之,慟絕,首觸牖,將死,家人共寬喻之。既免身,生女,趙曰:「生女亦善,使我無系戀也。」後八日,自經死。

姚森桂妻宋,秦安人。森桂卒,宋入廚下自剄,血自咽出汨汨。姑入視,右手握刀,猶力作再割狀。母至,束以帛,乃能語,曰:「死已決,毋緩我!」引母手掩口鼻,又解帶使縊,母手顫不可任。睹宋狀至慘怛,乃飲以毒,毒自創溢。但聞宋咽中若曰:「斫我,斫我!」久之,乃無聲,遂死。

惲毓華妻莊,陽湖人。毓華死,莊飲藥殉。毓華弟毓德妻許,毓德死,許絕食殉。毓華侄寶元妻袁,寶元死,袁先服毒,急救之,復絕食三日以殉。世稱「惲氏三烈婦」。

曲承麟妻袁,承麟,沈陽人;袁,名桂珵,遼陽人。嫁未百日,承麟卒,袁仰藥殉。

尹春妻張,歙人。初為黃氏婢,名桂喜。主婦程,知書,嘗與諸娣姒說古列女事,桂喜竊聽,輒稱羨。既嫁而孀,遂矢死。詣肆求毒藥,肆以他藥予之,飲不死。市櫬,臥其中,主婦泣喻之,對曰:「桂喜聞主母講列女時,意已決,不可回也!」卒不食死。

李氏,高密人。夫嘉猷,失其氏。嘉猷惑於讒,娶不與同室。及病,李奉事甚謹,禱於神請代。嘉猷聞而悔,遂死,李自經以殉。

陳三義妻王,掖縣人。王未行,病而瞽,其父辭於三義,三義曰:「吾聘時未瞽也,聘而瞽,猶娶而瞽,其可棄乎?」娶三年,王目良愈,三義尋卒。王曰:「夫不負吾,吾豈負夫?」遂縊。

游開科妻趙,馬邊人。開科貧,贅於趙。趙有母及兄,皆厭之。趙脫簪珥別賃屋以居,食盡,不貸於母家。一日,趙還省母,方食,開科至,趙推食與之,母及兄逐開科,禁趙毋歸,且言:「此餓莩死,何患無家?」趙縊死。

孫崇業妻金,赤城人。崇業嗜酒,不治生,金勸之不聽。順治中,歲祲,崇業計鬻金,陽語當偕詣戚屬。金察其詐,曰:「汝乃忍嫁我,我嫁必且死。然至汝家二十餘年,詎忍恝然行?盍沽酒為別!」崇業出沽酒,金抽刃斷喉死。

張某妻田,萬全人。夫游蕩,田屢諫。一日嘆曰;「我生不能勸,死或憶我言。」因仰藥死。死時猶呼其夫,勸改過。

張氏女,婁縣人。農家女。嫁魯氏子,姑與夫迫使為汙行,不從,箠楚凍餒,凡三四年,志不變。康熙二十六年三月,其夫將劫以他往,夜入萬安橋下水中死。

又有湯氏女,奉天人。有娼家為客娶之,使為娼,箠楚困辱,卒自殺。

滄州女,不知其姓,名黛城。年十五,鬻入娼家,使應客,不從,撻辱之,大罵。娼家支解之,棄尸於河。

張氏,都昌人。康熙十三年,耿精忠為亂,張之夫熊應鼎將從賊,張諫,勿聽;質裙沽酒,以飲且勸,終不可。乃告於其族,矢死。應鼎入於賊,張自殺。

孫大成妻裔,江都人。大成母姣,二女嫁而歸,皆與縣吏通。迫欲汙裔,裔告大成,俱縊,救不死。裔歸省母,告母狀,持母袂哭。臨去,檢母奩,得青白線各一束,因曰:「兒必不辱母!」俄縣吏宿姑室,復呼裔,不應;姑詈,亦不應。縣吏醉,裸而譟窗下。裔以青白線綴上下衣,復合為絙,縊。姑覺,不救,遂死。鄰知其事,感泣拜裔尸。或語侵姑,姑反脣,眾譁以告官。官庇吏,旌裔,葬平山堂右岡,而不竟其獄。後數十年,縣隸以事辱裔兄子,死於水。裔兄痛子,亦死。

楊某聘妻章,字原姑,秀水人。年十九,縣隸請婚,父不許,許楊氏。縣隸與其徒譟於門,誣原姑與有私,原姑夜縊死。縣吏欲寬隸,獄上,巡撫持不可,乃絞隸,旌原姑。

裔死康熙六十年四月戊申,原姑死嘉慶六年九月甲午。

孟黑子妻苑,黑子,大城人;苑,東安人。其姑素無行,會永定河決,工役大集,賣酒堤上。強苑與偕,苑不從,窘辱之。姑與惡少入婦室飲,婦終不可犯,姑益怒。婦度終不免,自沉死。夫行求其尸,四日,得之武清境。又四日乃斂。方盛暑,尸未朽也。

北塘女子,業磨豆為腐,母迫為娼;新河藍某妻,失其姓,姑迫為娼:皆自殺。

武清芮氏女秉貞,寧河樂某妻左,並以姑迫與惡少暱,自殺。

蕭氏,靈州人,為張文彩妻。文彩有友悅蕭美,欲污之,蕭力拒。友懟蕭,譖諸文彩,謂蕭不潔。文彩信之,紿蕭歸寧,與其友共殺諸途。後事雪,雍正十二年旌。

黃氏女,昭文人。嫁張氏子,為縣小吏。其母有所私,迫女從之,日箠楚。或謂女:「盍歸?」女曰:「女既嫁,安歸?待死而已!」乾隆十六年夏,方暑,姑與所私裸而飲,女避,所私起持之。女大號曰:「奴敢污我!」持案上酒器提之。姑怒,批其頰,復榜掠之。夜半,女入井死。

吳氏女,震澤人。喪父母,方六歲,字李氏而待年焉。稍長,美,李氏子行賈,久未歸。姑悍,私於里豪。里豪啗姑金,欲得女。女勿從,姑撻之極楚。鄰嫗問其故,女不肯言。當暑,浴,姑納里豪於室,鍵其戶。女呼,不應,挾剪拒,創裏豪,里豪持女褻衣去。女求死,姑操巨箠撻之,女引剪自,未殊。鄰人戒其姑,毋急女。女與鄰女款曲如平時,晡啜粥盡一甌,鄰女謂不死矣。夜漏二刻,自溺門外溪水死。時乾隆三十七年七月丁未。

顧氏,泰州人。夫張世英,日誨顧淫,顧不可。或貸世英錢,世英陰欲顧與私,沽酒飲貸錢者,嗾其母呼顧出,不應;與之酒,覆杯,慟。貸錢者亟去,其母搤顧吭,幾絕。鄰里咸憤,訴於州,世英乞悔過,以顧歸。與其母益日夜迫之,顧飲滷,不得死。乾隆十六年十月戊戍,世英語顧:「冬無衣,盍如吾言?即得錢衣汝。」顧曰:「我寧死不辱。」世英恚,夜扼殺之,年十七。

張氏,丹陽人。夫陳彭年,嫁十年矣。彭年貧,欲嫁張,張涕泣不應;紿使出,而密使媒從,張覺之,號慟求死。邏卒以告官,官笞彭年,令張還母家。張曰:「我適陳矣,死生以之。」彭年益迫張,張度終不免,從容言曰:「我無如何,今當聽爾!」起隨彭年走出村。塘水方盛,張躍入水死。死之日,為乾隆十九年六月戊辰。

許會妻張,潁州人。姑姣而虐,惡張端謹不類,日詬且撻,張事姑益恭。姑病,刲股以療,姑虐如故。姑與鄰寺僧通,欲亂張。姑匿僧室,召張入,而出鍵其戶,張大號,僧遁去。翌日,自沉於井。有司捕得僧,論如律。鄉人裂僧尸以祭張。

趙海玉妻任,名環,汝州人。姑故與鄰人通,夜半,挾刃入任室,詬而免。亦井死,年十九。

殷氏,天津人,為同縣邢文貴妻。文貴故無行,其母趙,姣。文貴初娶于,以貞慎不相入,出之。復娶殷,殷貞慎尤逾於,趙惡之,與文貴日捶楚,沃以沸湯,施燔灼焉,體盡潰。有司聞,使吏就視,殷拒不可。旋卒。有司收趙及文貴,論如法。

嘉興女,失其氏,嫁賣酒家王氏子。姑當壚,習與酒人姣,惎女不應,乃裁抑不使飽。縣中李氏母,故大家女,聞賣花媼言女事,愍女有志,輒令媼市胡餅畀女。一日見女餓,憊甚,而幾上置餈果,媼怪女何棄不食,女曰:「李夫人飽我,哀我志也!此物西家以餌我,我有餓死耳,豈可食乎?」李母病,且死,遺錢十餘緡周女。女感泣,語媼:「我終不負李夫人望!」惡少艷女久,嗾姑將脅以威。女漸聞之。乃請於姑,代當壚。姑喜,授女戶鑰。數日,女夜啟後戶投水死。乾隆二十年六月事也。

王某妻李,字黑姑,天津人。姑不貞,與鹽運使隸有私,計欲並污李。隸與姑飲,役李,李恥之,恆不如姑指。姑以他故詈且撻,待隸為之解,復示意李,終不可,而隸意未已。李枕側置刀以自衛,姑逐其子出,夜持被就李共寢。夜半,啟戶納隸,隸迫李,李呼,姑掩其口。取刀自剄,未殊,母來視之,復甦,語其故。並言:「方自剄,血溢,不知人。漸聞隸語姑,當言夫婦相爭詬自戕,宜無知者。」越三日乃死,其兄告官,笞隸,不竟其獄,道光六年七月事也。

何先佑妻孫,桂陽人。先佑父在時,為先佑求塾師,授之讀。未幾喪父,其母以家政屬塾師,因私焉。孫既歸,嘗晨謁姑,塾師在其室,孫趨而避。塾師與姑謀並亂之。塾師出,孫入,諫姑曰:「家雖貧,粗有門閥,翁勤苦終身,不得意,所屬望者先佑。姑念翁與先佑,勿復近塾師。」姑慚,戒毋洩。孫曰:「婦所言為門戶耳,雖先佑不敢告,第原姑終念婦言。」塾師既與姑謀,遂屢挑孫,孫以告姑,又諫,姑終毋納。塾師入孫室,孫大詬,塾師陽避。孫欲還告其祖,忍未發。姑陽出,塾師復入孫室,潛抱持之。孫號,奮擊。先佑入,塾師乃走。孫傷於脅,遂自經死。時乾隆二十九年三月。明年,獄上,斬塾師,徙其姑新疆。

邢氏,字福,濬縣人。農家女也,而有容色。嫁袁顯旺,姑姣,群奸聚其室,驚邢美,挑之,不從。其姑誘且詆,邢若為勿喻也者。謀益急,夜出,將赴水,風失道,遇同村人送還父家。父願,與復至袁氏。群奸迫其父使具狀,曰:「女再逃,杖死。」夜二鼓,群奸縛邢裸撻數百,邢有娠,不勝楚,求滅燈,死不恨。群奸縋邢於梁,而撻之益毒。五鼓燈盡,邢死。使顯旺劙其頸,若自戕。官捕群奸,論如法。

遷安婦,不知其姓。夫行賈,翁耄,姑私於傭。傭計並污婦,稍近婦,婦色甚厲。乃與其姑謀,嗾翁污婦,婦不可,遂嗾翁殺婦。絮塞口,杙椓下體死。

白鎔妻尹,亦遷安人。鎔出為優,姑有外遇,迫婦,絕飲食,日啜米沈。逾月,姑縛尹,以熾鐵烙下體。尹號,擊其首,發皆燃,一目裂,遂死。

林氏,平湖人。嫁顧大,家乍浦湯山麓。顧大母故娼也,惡少往來其室,強林具茗,不可。母惎林,與諸惡少謀,必欲並污之,林竊出赴海。未至,值鄰女,送之還;母益仇林,與大日共笞之,靳其食,不令飽。居年餘,為嘉慶九年正月,方改歲,惡少至,群飲,林復竊出赴海;既日受笞,且久饑,行不前。大追至,執以歸,母遂欲殺林。撞以重器,腰肋俱折,復砲烙其下體。是月丙戌晦,林死。事發,論大如律。

洪某妻徐,金谿農家女也。姑與兄公有盜行,徐至未逾月,察得之,大戚。脫簪珥畀洪,囑遠行賈以避,屢諫姑,姑不納,乃自經。

敖氏,涼州人,嫁駐防涼州旗人四十九。四十九有友相狎,丐與敖通,四十九許之,假以衣,夜入室,敖聞語,辨非夫也,奪戶出,友遁。敖詈四十九,俟其出,自溺水■L7中死。

塗氏,梁山人,嫁甘克桂。克桂游蕩,破其家,塗以女紅供日食。克桂負賈錢,將以塗償。一日,克桂從塗取故衣易錢以飲,醉歸,塗泣,克桂摑其頰,曰:「行且鬻爾!」塗曰:「吾矢死不往。」克桂撻之,兩晝夜不已,塗自經死。

吳氏,彰化人,嫁康氏子。姑不貞,欲並亂之,吳不從;乃效治囚法,榜掠之無算,卒不為屈,剚刃其腹死。道光七年事也。

楊氏,江都木工女,嫁曹氏子。姑迫使為汙行,楊不從,乃絕其食,鞭之至累千。造諸酷刑,榜掠無完膚,創重死。鄰以告縣吏,笞其舅及夫,葬諸梅花嶺下。

趙氏,桐城人,夫同縣孫某。洪秀全兵將至,其夫降,受署置。咸豐十一年,秀全兵破桐城,其夫戴黃巾,被黃袍,乘馬迎趙。趙望見,大慟曰:「汝非我夫也!父母遣我嫁乃諸生孫某,非作賊孫某也!且汝既讀書為士人,豈不知孫氏望族,文武仕宦不絕,而失身降賊,意氣揚揚自得,我不忍見也!」起,投塘死。子數歲,從之下。

同時又有王氏,合肥人。夫繆錫疇,將降秀全,王力諫不聽,自經死。

許氏,名領姑,歙人,夫亦縣諸生。咸豐十年,賊至,其舅將降,許泣諫,勿納,亦自經死。其舅後忤賊,舉家皆為戮。

梅氏,名蘭姑,不知何縣人。嫁夫不肖,欲攜以為豪家奴,梅不可;又使出乳人子為傭,亦不可。夫引僧入其室,梅力拒。鄰以告官,官笞僧及其夫。夫怒梅甚,窘辱捶楚無不至;又徙居木工家,夜,諸惡少入室,將強汙之。鄰復以告官,官未即聽其獄,梅自經死。

張氏,武進人,字沈盤德。父母卒,大母老,待年於沈。盤德父故無賴,屢挑女,女謹避之,又不令歸省。張之戚有與沈鄰者,女大母偶過之,女聞,得間問安否,因密訴其事。嗚咽曰:「兒命苦,惟有死耳!」又嗚咽久之。囑大母曰:「勿揚於人也!」未幾,里中為優,舉家往觀,女獨在,盤德父驟逼之,力拒得脫。度終不免,自經死。

秦某妻崔,陽高人。夫惡,崔諫勿聽,撻辱之。逾年,坐罪流徙,懼見侮,先殺其子而自殺。

李某妻管,南平人。夫不肖,管數諫,累被撻辱,逼之嫁,奔還母氏。卒鬻於富家,乃自殺。

王某妻徐,東鄉人。姑夏,早寡,而子無藉,夏戒勿聽,徐規之,輒鞭撻欲死。夏謂徐:「夫無恩,可嫁。」徐不去。

陳潛聘妻崔,名秋,宣德人。秋大父與潛父希孔同官於肇慶,秋大父卒官,因迎秋至官廨,而潛在裏,阻亂,未婚。順治十年。希孔罷官,還道高明,遇仇家,熸焉。縶秋及希孔二妾,將汙之,秋罵甚厲。仇生瘞秋,以蜜傅其面,引蟻嘬之,秋至死,罵不絕。二妾亦生瘞死。

硃承宇妻曹,承宇,無錫人;曹,武進人:皆農家也。生二子、一女,而承宇死。承宇弟迫之嫁,曹以死拒。遍告鄰里戚族,乞言於叔,得毋嫁,承宇弟不許;請終喪,不許;請及大祥,不許;乃請得見其姊,許之。曹夜挈兒女詣姊家,曰:「我初不欲嫁,今已矣!特不能累累然抱兒女作新婦,暫累姊,三日後,當相取,慎勿告吾叔!」姊謾許之,兒啼索乳,曹泣曰:「癡兒!母豈能長乳爾耶?」辭姊出,復還視兒女,再三囑姊。姊曰:「三日耳,何言之數?」乃去,哭於承宇墓,還,遂縊。姊往哭之,目猶視,許育其兒女以長,乃瞑。及斂,左臂創未合,蓋承宇病時嘗割臂也。父為訟於縣,罪迫嫁者。

陳有量妻海,銅山人。有量,儒家子。貧無食,轉徙常州。居逆旅,貲盡,惡少矙海年少,與有量游,且周之;時其亡,挑海,海詈之,走。是時漕粟至京師,其舟謂之糧船,主者皆豪猾。惡少繩海於主者,亦引與有量游,招使佐會計。且謂:「舟行當經徐州,盍以孥歸?」有量以告海,海問孰為引致,則惡少嘗為所挑詈而走者也,謝毋往。惡少使其曹訟有量逃人,有量懼,乃以海入其舟。海入舟,日獨處,主者使有量有事於近縣,而夜就海,強抱持之。海號,撾其面,猶不釋,大呼殺人。舟人盡驚起,始得免。即夕,自經。主者藏其尸積粟中,賄舟人。有篙師藍九廷者,愍海死,卻主者賄,告官,乃按誅主者及惡少。常州人葬海於南郊,會者殆千人。

樊廷柱妻張,襄城人。廷柱早卒,張奉姑撫二子。縣中有無賴子二,倚兵籍為暴,艷張欲汙之。康熙五十五年四月戊申,日方午,姑與其幼子出郭穫麥,二子就塾。二無賴詗張獨居,共入室,張走避。一直前持之,一扼其吭,哧以死,張不為屈。取菜刀揕其面,為所奪。入室就床側解佩刀,刀長操其室,方出,又為無賴奪,遂共曳張使伏,張輒躍而起,屢僕屢立。捽其發,縷縷脫,呼益急。二無賴度終不可犯,一拾所解刀斫張額,張僕,一取菜刀斷其喉,遂死。鄰見二無賴出自張室,衣漬血,告官。縣吏憚兵家子,欲坐廷柱弟宣,民大譁,乃以疑獄上。後四年,河道周銓元署按察使,察獄辭,詫曰:「此何名疑獄?城中殺人,非荒野;日午,非昏夜。且殺人者有主名,此何名疑獄?」下縣逮二無賴,一前數月發狂死,將死,自承殺張;一戮於市。

李有恆聘妻楊,偃師人。少喪母,十七未嫁。父為隸,歲暮,猶行役。一夕大雪,同村有屠者,持刀入女室,女堅拒,被殺。質明,其父歸,見女死,咽斷,左手數創,右手持衣帶不釋。出戶外,逐雪上血跡至屠者家,得刀於床下。屠者死獄中。

陳某妻,不知其姓,吳人。夫圬者,出就傭。鄰有酒人過,調婦,婦語夫,夫漫授以刃曰:「彼來,汝殺之!」復出就傭。酒人夜排戶入,婦擲刃,酒人拾刃刃婦,洞胸死。兒號,鄰婦入視,一村皆集,獨酒人者不至,求之,方避入鄰村。告於官,誅之。里有老塾師曹叔素,盡出所蓄金為建祠,圖像以祭。

劉埜妻李,太康人。姑令採菽,鄰村子持鐮過,調婦,婦力拒,舉鐮剚胸死。越數日,鄰村子疾作,持鐮趨採菽所,自言殺婦狀,乃執以告官。兩家故有連,賄罷訟。逾年,疾復作,持鐮趨採菽所,抉胸斷喉死。

曲氏女,字登,永寧人。年十三,父守瓜,母呼女饁之,父令女代守。鄰園叟五十餘,望見女獨坐柿樹下,前調之。女怒罵,叟執其臂,女躍上樹,叟攀樹,曳以下,女號益厲,乃走。女歸訴父母曰:「兒臂為人執,不為急湔洗,何能立天地間乎?」明日,持刀奔至鄰園叟門外,自剄死,目瞠視,立不僕,血湧出不止。叟出戶見之,反走,提廚刀至女門外,跽,亦自剄死。

宋氏五烈女,肅寧農家女也。父佃於勢家,為莊頭,其主視若奴僕。生女四、女孫一,長,並有容色。其主將迫使為媵,五女一夕自經死。以白縣,縣憚勢家,不敢上聞,葬而為之碣,曰「宋氏五烈女之墓」,康熙三十四年事也。

東安陶子明妻張,解萬有妻劉,清苑戴國妻鄭,為營兵所挑,不從,見殺。

通州邢德重妻王,為營兵所挑,入井死。

龔行妻謝,興化人。縣被水,行挈妻女至鎮江,屑豆為腐以活。鎮江故屯軍,有江寧無賴子入軍籍,窺謝及女有容。一日行出,挾群少過之,遂挑謝。謝倉皇號呼,無賴擊謝僕,女奔救,又犯女,急走避。無賴偽為行券索償,因毆行。行愬縣官,官笞行,且逮謝。謝持女泣曰:「以吾故,陷汝父,吾死不足恤,獨憐汝耳!」女亦泣曰:「母死,女何能生?即生,且蒙不潔。原相從,得仍為母子。」相持而慟。雞初鳴,投水死。女名巧。

楊文龍聘妻孫,字秀,錢塘人。秀年十五,待年於夫氏。文龍從父行販,秀依姑共處。鄰家子無賴入室,牽其衣,秀嚙其指,乃去。方暑,秀晚浴,鄰家子穴壁,持其足。秀驚起白姑,姑告諸鄰。或引無賴謝,秀提以茶碗,中他人,其人亦無賴,相與噪於門,言終當致之。秀慮不免,密紉上下衣,出視姑膳,膳畢,復瀹茗進,乃入室,飲滷死。巡撫聞,按誅無賴,為文以祭。

梁至良妻鄭,至良,海陽人;鄭,澄海人。至良卒,其兄為諸生,迫鄭嫁。鄭遺腹生子,家有田八畝,鄭悉推與至良兄,自分圃畝許。力種溉,傭於群從娣姒間,縫紉舂磨,得米奉姑食子女。歲大無,至良兄憾其不嫁,夫婦眾撻辱之。鄭念不可留,夜檢故衣,付其女,曰:「明晨母當去,若善視幼弟!」明晨,跪姑前泣告當還母家,遍辭群從諸娣姒,遂行。至廣濟橋,仰天呼夫名三,投韓江死。雍正六年六月庚辰朔也。

郭進昌妻李,永寧人。進昌卒,矢不嫁,與女若婿居。進昌弟貪而狡,計嫂年三十許,尚艾,嫁可得錢,乃詣李,微諷之。李怒,叱使去,進昌弟與族子謀,鬻女為富家妾,約以騎迎。至日,進昌弟入李室,將強扶李出,婿與女詬斗。李忽改容,戒勿譁,入室作妝,以小刀薙鬢,遂上馬去。至王範鎮,李大呼,袖中出薙鬢小刀刺喉,喉斷,血噴十餘丈,墜馬死。鎮人大驚,共執進昌弟,問狀,呼婿與女訴官,論如律。

龔良翰妻陳,葉縣人。良翰卒,孤女才三歲,後母欲嫁之。陳依叔父居,叔母有弟窺陳美,夜持刀入自牖,陳與鄰女宿,盜至,推鄰女床下,徒手捍盜,指斷目傷,身數創,卒不得亂。叔父聞,撞扉,盜牖出,陳息僅屬。鄰女出床下,血淋漓被體。叔父心知盜其婦弟也,告官,置諸獄,陳遂不食。叔母勖以育女,乃復食。既女殤,而縣吏鞫盜獄未定,若有疑於陳,召庭質,雍正七年五月辛亥,陳自經死。後五年,縣吏坐罪去,事乃白。

王均妻湯,均,吳人;湯,寶山人。湯故富,均贅於湯,湯父母遇之薄。均客授,湯治針黹以養父母。稍久,有田十二畝。雍正十年秋七月,海潮大至,均夫婦倉卒緣樹,均攀枯枝折,溺焉,湯父母憖不問。湯使僮午求均尸,三日始得之,被發徒跣赴尸所,哭幾絕。既斂,湯父母欲火之,湯不許,瘞均田中。湯遺腹生女,名之曰潮音。湯父母迫使嫁,輿至,湯麻衣腰絰,抱潮音繞場號。眾劫納輿中,湯父母奪潮音,將抵諸石,午自旁篡得之,歸諸王氏。眾卒舁湯去,湯哭數夕不絕聲。守者稍怠,自經死。湯父母以疫死訃於王,棄湯柩所死家。居數年,慮事洩,惎其人焚柩。午自詭湯氏使往視,既焚骨入罌,午易以空罌,得湯骨瘞均側。潮音亦前殤,祔焉。

李氏女,名蘭香,長安李氏婢也。李氏有僕,私欲妻蘭香,未敢言。會有客至,治具,主母命蘭香取具樓上,僕從登,扃門,就擁之。蘭香號,持之堅,卒不從。僕慮事敗,以麻稭剚其腹,深數寸,遂死。

翠金,不知其氏,平湖施氏婢也。主客授於外,翠金侍主婦,不茍言笑。鄰有無賴夜持刃逾垣入,翠金呼,無賴懾以刃,翠金曰:「我不畏死!」罵愈厲,遂見殺。

張元尹妻李,永寧人。生女而元尹卒,李以己有色,自晦,不逾閾。居十餘年,其家僕夜持刀逾墻,拔戶樞,入其室,李聞其聲,僕也,罵:「萬逆!」僕出刀曰:「不從,截汝脰!」李奮頸呼曰:「截,截!」聲未斷,已殊。手足擊床震,女驚呼,家人縛僕送官,自言殺李狀,論如律。所居村曰太原村。

張檢妻顏,其同縣人。幼聞人言太原村張烈婦,輒嗚咽流涕。長有色,歸檢,出應試。客作伺顏夜省姑,懷刃潛入室,匿桁下。人定,出,登床,顏驚。脅以刃,罵。起奪刃劙掌,罵益急。迭刺胸臂肋腋十餘創,死。客作夜走,還其家,捕得,坐誅。

萬某妻曾,南城人。萬愚甚,有父不能養。曾力女紅食其舅,且自食。萬嘗忤其父,告官,縣隸至,見婦美,乃為計出萬,且引使為隸,假以錢,招共居。曾謂夫曰:「汝與彼不相識,何以能得此?此其意,蓋在我也!」辭毋往,隸怒,索錢。曾有女才四五歲,隸曰:「汝無錢,當鬻此女以償。」萬乃鬻女,曾至所鬻家抱以歸,且罵隸。隸益怒,告官謂曾忤其姑。官令逮至,撻其面數十。是夜曾抱其女赴水死。曾嫁時,姑死久矣。

李繼先妻侯,忻州人。奸民謀汙之,不遂,誣以不潔,訟之官。官不能白,侯自裁訟庭。

田氏女,巴縣人。幼喪父母,依兄嫂以居。年十五,美,有無行生欲挑之。鄰有優人妻與謀,要女過其家,強以酒,欲汙之。怒詈,脫歸告兄,兄訟於縣。生丐縣中有力者語縣吏,誣女有汙行,縣吏撻其兄而釋生,女忿自殺。

馬某聘妻苗,肅寧人。早喪母,將嫁,謁外祖母,止宿。鄰僕瞷其美,夜持刃排闥入,女驚呼。傭婦起沮,僕殺之。外祖母奔救,又殺之。客作聞聲持械入,與鬥,刃頓,取莝刀支解之。因持女,女呼益急,莝刀擊之,創遍體死。時乾隆三年六月己亥。

高日勇妻楊,鎮番人。日勇傭於馮氏,與楊俱。馮挑焉,楊不從,因辭去。馮從子尤艷楊,乾隆十六年七月甲申,馮氏子詗楊獨處,逾垣入。楊方炊,力拒,馮氏子擲塊中楊,楊僕,遽死。馮氏子懸其尸,若自罄然,扃戶走。日勇訴縣,窮治馮氏子,伏法。

羅季兒妻秋蟬,不知其氏,武昌人。為攸人傭,欲逼汙之,不勝辱,季兒、秋蟬皆自殺。

劉氏女,小字惠,舞陽人。年十六,美而端。父母出力田,女獨居治枲。鄰子入其室,女詬,鄰子出,復還掩其口。女怒,嚙鄰子,傷手。稍解,女搏膺號。鄰媼入視,鄰子乃去。晡,父母還,女言其事,大慟,謂為無賴辱,當死。父母慰喻之百端,卒自縊。告官,鄰子詭言故與女有私。按女尸,處子,乃論殺鄰子。

鍾某妻蔡,嘉定人。生農家,年二十一而嫁,嫁三月夫死。力作,日斷布三疋,易粟養姑。姑憐之,勸使更嫁,蔡泣誓以死。有女妹嫁無賴子,欲得蔡,語姑偽為其弟娶者。姑察蔡志堅,弗許,因構蜚語衊蔡。姑審其誣,將率蔡愬諸縣,無賴子陽使其妻歸謝,而陰告母,將結惡少夜劫之。姑惶遽無所出,縊焉。蔡覺,趨救得甦,姑哽咽語曰:「吾女遇不淑,重為新婦累,吾不忍見新婦之受其累也!」蔡曰:「母無慮!婦留,母不得安;婦去,母不得食。雖然,叔幼,非母焉依?請得卒哭焉以往。」乃奠夫,慟,入戶,解絰自經死。

段舉妻盧,延津人。盧有色,一夕,與其子女為賊縊殺室中。知縣詣視,盧帛系頸,爪殷血,子女縊床上。知縣求賊,村人集視,一人手屈匿袖中。令出手,絮裹指端,發視有齧跡,視胸及股皆爪傷。問之,乃自言:「艷盧色,夜穴墻入,盧驚呼,掩其口,齧我指。捽而逼之,屢僕屢起,爪傷我身,乃出腰間帛縊殺之。子女號,因並縊焉。」獄上,盧得旌。乾隆十八年事也。

王某妻劉,懷仁人。歲大無,豪族結奸儈貨沒饑人子女。劉度不免,從容語其夫曰:「姑老子幼,不耐饑,旦暮俱死,無益,計不若鬻我。誠得多金,姑與子可無死。汝第送我於郊,我得以身完!」夫忍而許之。儈至,遂鬻婦,夫送之行。四日,儈屏其夫,夫未去。劉語儈曰:「我夫不能庇我,以至此,戀戀何為者?是非痛詈之,弗肯去也。」儈以為誠然,縱飲且醉。劉出,呼其夫,拔簪刺喉死。儈皆驚,散去。

張良善妻王,鞏縣雙槐村人。事舅姑孝。父為傭,母呼王還。家故貧,穴土為室,母出,與幼弟二禮居。有族子故無賴,夜以刀劙戶側土,土落,王驚問,族子已入室。王怒叱曰:「我而姑也,而禽獸,速出!」族子出刀。曰:「刀何為者?任爾殺不懼。」族子刺王中左肋,血溢自襦濺數步,益怒詈,復刺左右肋及乳。王奪刀,刃裂掌僕。二禮亦呼,族子斫其臂,亦僕。王復自地上躍起,疾出戶,呼殺人,族子從之。王創甚,躓於石顛樹下,族子劙其口,王口齧刀齒有聲。族子抽刀,破其頤,王不能言,聲猶厲,身霍霍不已。斷其喉,乃死。乾隆三十五年十月事也。質明,裏見王死,呼其父歸,二禮言姊死狀。眾聞王死烈,吊者日千餘,上於官,誅族子。

李青照妻張,興國人。鄉人赴官云南,青照將妻、子以從,鄉人艷張,屢挑之,張以語青照。過長沙,青照與妻、子夜脫走,青照復還取行橐,張抱子以待。長沙縣役與相值,詰得其情,引以行。稍遠,乃逆青照脅以逃人,詐得金,並解所佩象齒蝦蟆去,至張所示之,詭言青照招使往。張從登舟,役迫之,抱子入江死。青照聞告官,論役如律,乃自經死。

姚際春女,浮梁人。際春方遠行,女侍母居。有母之族為傭者,佻而獷,女惡之。告母,母謂彼於汝尊行也,宜無他。居稍久,傭益恣,女復告母:「不逐傭,且殺兒。」母遣傭,傭不行,挾刃入女室。女躍且呼,傭剚其腹,腸出。母入視,傭自剄。女目未瞑,移時甦,猶語其母曰:「兒惜此身以報父母,獨憾父出不一訣也!」語竟,血飛濺,承塵盡赤,乃絕。

王敦義妻張,新陽人。敦義早卒,而家富,其弟覬得之。有無賴為之計,夜使年少僕匿張床下,而偽為捕賊者。僕自承與張私,因呼里長縛僕並及張。天初明,偽為縣役持牒逮張,又偽為居間者,使張予金緩其事。張歸,心知為叔賣,有女字俞氏,遂出橐中裝為一囊,攜女之俞氏,以女託翁、媼,歸自經死。

陳維章妻陸,名趙鳳,諸暨人。父效忠。初有黠者聞陸美,欲娶之,以齒非偶,偽為其弟聘,而陰為弟別娶於李。效忠聞,絕黠者,歸女於維章。黠者易婚書,賄媒妁,以訟於縣,縣判歸黠者。黠者以輿俟,得判則劫持陸置輿中,疾舁去。陸方持祖姑服,黠者迫更衣,不可,手裂其衰。陸詣縣,袖剪以往,計不直,則自殊。倉卒被劫持行,不得出。及拒黠者,裂衰,剡觸手,乃不敢迫,使弟婦李守之。李憐陸,又自念處亂家,時時與陸屏語,或相持泣,數日乃共縊,繩不足,續以帶。時道光四年二月,陸與李皆十七。

何氏女,山陰人,居通州。鄰有黠者聘為其弄兒婦,冀並亂之,女截發自誓。鄰里以告官,官判歸父母家別嫁。女減食六閱月,垂死,告父母曰:「兒失身於匪人,重見逼迫,不幸告官,又不幸判別嫁,此子誠不肖,兒則夫也。兒欲為之死,又不敢傷父母意,乃減食以求死。初減十五,逾二月減七,又二月減九,今不食已三日,兒死非病,原父母勿悲。」遂卒。

劉宏芳聘妻周,霍州人,未行而宏芳卒,周亦減食,數月乃死。

謝亞煥妻王,名杏芳,東莞寶潭村人。年二十一,歸亞煥,未期而寡,從姑居。有諸生奸暴為縣豪,瞷王美,使告其姑,欲為從子娶。姑辭焉,則宣言將毀其居。一日,將數十人至,大譟升屋,撤椽發瓦,姑走匿。王出語眾曰:「若曹欲何為?我在也,勿驚我姑!」豪呼眾篡之歸。王故惎豪,採毒草自備,輿中食之盡。至豪家,登堂,毒發死。豪夜還其尸,瘞於亞煥側。姑與其母家愬縣,獄成,豪瘐死,道光十一年事也。

張樹功妻吳,常熟人。樹功卒,吳遺腹生男,矢不嫁,事姑撫孤子。樹功有弟共居,姒賢,與吳相得。死,而再娶得悍婦,奴婢視吳母子,吳安之。歲饑,悍婦凌吳,樹功弟用婦言欲嫁之。吳痛哭告其子曰:「汝今九歲,饑寒可自知,我將舍汝從汝父去矣!」其子魯,不知母將死也,吳遂自經。

郭某妻李,仁和人。早寡。杭州初定,防軍守諸門,勢張甚。車過,男子下,婦人必卷幔。李從家人避兵郊外,歸入錢塘門,方小病,門卒遙見之,為嫚語,李坐車中微聞之。至家,慟曰:「我不幸為門卒語所辱,我不可以生!」晨夕涕泣,不食二十餘日,卒。

趙謙妻王,威縣人。當暑,謙出,王獨寢,風入牖簾開,若有窺者,王忿不欲生。舅姑及謙曲喻之,終不釋。曰:「與其疑而生,不若疑而死。」遂自經。

郭氏女,鳳陽人。順治十一年,女年十四。樓居,鄰火,女披衣下樓,見救火者眾,不欲前,躍入火中死。

何氏女,汜水人。侍祖母同寢,夜火,其兄援祖母出,復入救女,女以衣履不具,終不出,與妹二、表妹一同死。

沈鼎猷妻嚴,浙江山陰人。寡,遇火,倉卒不得衣。救者至,出其子門外,復閉門焚死。

鐵山婦,德化人。火至傍舍,鐵山塹高,迫不得上,或援以手,婦不肯上,及於火死。

汪氏女,與賀氏女,皆歙人,家縣之東門,相鄰也。父母俱歿,各居小樓中,汪長賀一歲,賀時從刺繡,相親若姊妹。縣大火,初發,汪未寢,驚走出,呼家人救賀。往叩門,賀自樓上問曰:「姊出乎?」曰:「已出,故使來相迎。」少頃,賀復曰:「吾求外襦不得,不可以出,幸謝姊!」既而火及,汪氏之人欲排戶入救之,賀怒詈,乃不敢前,竟焚死。還報汪,汪曰:「妹死,吾何忍獨生?」趨賀死所,躍入火,亦死。

馮光琦女,郭君甫妻吳,皆盱眙人。光琦恆為客,女母死,囑吳為侶。遇火,女扶母棺號,火益烈,救不至,吳引女出,女堅不肯起,俱焚死。

黃聲諧妻王,婺源人。寇至,扶姑行避寇,道失姑,跡之至渡口。水方盛,行度橋,橋欲圮,有男子援以手,卻之。橋圮,墮水,據木浮中流。男子以雨蓋授,復卻之,遂溺。

徐惟原妻許,南陵人。康熙間盜起,許行當涉水,從者請負以行,許曰:「僕焉可負我?」寇大至,入水死。

柯叔明妻鞏,貴池人。大水,叔明及其子已出,使僕負鞏,鞏以僕裸,不肯出,死於水。

胡某妻裘,新城人。大水,比戶皆乘屋。鄰有裸而登者,裘恥之,不上,溺死。

陳儒先妻李,不知何許人。夜半水至,鄰人呼升屋避,李衣逐水去,死不出。

白洋女,不知何許人。康熙四十七年,大水,從流至白洋。有拯之者,女以無衣,不就拯,死。

高氏婦,六安人。避水鄰樓,惡男女雜處,挈幼女下,立曠地。水大至,其夫垂綆使援以上,終不上,竟死。

段吳考女,稷山人。雍正七年六月,山水夜發,壞廬舍,女從水浮沉葦間。鄰人赴援,女以無衣,不肯出,入水死,年十五。

曹氏女,無為人。州有寺僧與婦人私,鄰童入寺見之,僧殺而埋焉。童父訟於州,僧辭服。僧念罪當死,不如多所連染,得稽刑。乃妄言良家子女與通者三十餘人,女家故近寺,亦在誣中。州吏盡逮諸婦,女白父,當詣庭自列,父不可,旦入城,謀諸吏。忽女自至,意色自如,詣庭。州吏出僧質,僧曰:「汝非曹氏女耶?」女曰:「然。」僧曰:「吾所交惟汝最久且密。」女曰:「果爾,吾身有異人處,汝當知。」僧辭遁。女固請入室使婦驗,則下體有疣贅,州吏始知僧言妄,慰遣女歸。女既歸,嘆曰:「吾所以蒙恥詣庭者,非為自表暴,蓋欲全此三十餘人而救其死耳。今事既白,吾廢人也,安用生為?且可使昏暴之吏,有所愧懼也。」遂自經死。

劉廷斌女,四川溫江人。廷斌道光七年官臺灣鎮總兵,八年,卒官。喪還,渡海,遇盜。盜殺其家十七人盡,女以美獨不殺。有客附舟哀,盜擲岸上,盜以女還。居十餘年,生四子。一日,女入寺禮佛,見僧似若相識。既歸,省僧即附舟客也。乃為牒具遇盜始末,復入寺,密以畀僧。僧告官,官取盜及其徒悉誅之。縶四子,以問女,女曰:「我所以受汙不即死者,仇未報耳!仇報矣,此曹豈我子哉?」手刃四子,自縊死。

張氏女,山東人。貧為婢,其主明魯王近屬也。明亡,張挈硃氏子流離旁郡,行傭不給,得巨室子之。硃氏子稍長,為諸生。聖祖即位,詔先朝諸宗人得以本姓歸田廬,張乃為硃氏子泣言其故。硃氏子復姓,召諸長老,原為張加冠,事之如母。張艴曰:「吾硃氏不成妾也,今主君主婦何在?吾何敢竊位!吾以姐始,亦以姐終,原勿復言!」俗謂婢曰「姐」,故張言如是。

崇德五年,師伐明,下河間,河間知府曲阜顏賡明自焚。有孫嫗者,傭於顏,挾其幼孫光敏,從師出關,間道徒步還曲阜,歸顏氏。孫與張同以義行稱。

陳氏婢金蓮,梁縣人,縣諸生陳其珍家婢也。流賊破縣,金蓮負其珍幼子以逃。賊追及,令棄陳氏子,與俱去,金蓮不可。賊斫陳氏子,金蓮身覆翼之,被數創,終不舍。賊去,金蓮死,陳氏子得全。

邱氏婢新喜,瀘江人。邱氏富,寇至,舉室走匿。執新喜,問其主安在,榜之垂斃,終不言。寇退,創重死,邱氏世祠焉。

董氏,江都人,傭於韓氏。順治二年,師下揚州,韓氏夫婦及其長子皆死難。主婦蕭將死,以其幼子魏託於董,方三歲。即夕,董懷幼子匍匐亂軍中,出自竇,匿江濱,拾麥穗啖之,得不死。亂定,魏育於故人家,將婚,迎董。董疾甚,輿以來,語新婦曰:「媼病且死,不復見爾夫婦!爾夫昔抱持從萬死中活,有今日。其人賢,雖貧勿憂,後且大,毋效世俗兒女子,易爾夫也!」

任氏,西充人,夫曰楊汝學。傭縣中龐可還家,為其子憼乳母。流寇亂四川,可還且死,以憼囑任。俄而寇萬騎猝至,任負憼走,間道得脫。歲大饑,從汝學流轉陜西,嘗棄兄弟之子而全龐氏子。四川定,任曰:「龐故儒也,子今且九歲,弗使就學,吾何以對龐君?」攜以歸,使就學,夫婦力耕以給。憼中康熙二年舉人,任曰:「吾乃今無媿於龐君!」尋卒。

同時又有袁氏,明侍郎李兆家婢。李氏,兆子映庚乳母也。流寇亂,兆兄完謀舉義兵,不克,其族熸焉。袁以計脫映庚,李行求映庚,得之僧寺,藏其家衣復壁。範士龍者,兆僕也,自兆所至,因送映庚還兆。士龍歸西充,歲饑,妻子五人皆餓死,蓋亦義者云。

盧尚義妻梁,文安人。織席以養姑,得遺金,告於姑,求主者還之。主者食鬼以布,告於姑,堅辭不受。世宗時,命御史鄂昌等巡察直隸,以其事聞,特敕嘉獎,賜米十石、布十疋,並命有司扁其門,以旌良淑。

白氏,秦安人,為張翠侍女。翠妻先卒,而病且死,目其子女泣。白曰:「君逝矣,此呱呱者,婢責也!」翠頷之,而泣不止。白挽髻拜床下,曰:「婢今為君婦,豈以死生異其志也!」翠乃瞑。白撫其子女至老。

王氏,名秋波,為晉江蔡氏婢。主將以為妾,而卒,無子。秋波長,家人遣之,秋波泣曰:「郎君將以為妾,郎君死,不可以貳。有為郎君後者,婢請得撫之。不然,當殉。」族人義焉,以從子六韜為其主後。娶於吳,生子,而六韜又卒。秋波與吳同處撫孤。

秦士楚妻洪,晉江人。早寡,事姑撫子,不憚艱苦。父家覆於仇,中危法當收孥,侄走匿秦氏。收者至,秦氏之人皆走避,洪獨不走。收者詰之,對曰:「無也。」斫以刃,被數創,終不言洪氏孤匿處。

張氏婢,海寧人。主母寡而貧,其兄割屋與其婢居,紡績以食。婢事主母謹,主母病將殆,無收恤之者。婢度事亟,招媒氏,原自鬻,以其值治喪,曰:「無多求,得七十緡,以為主母斂。事畢,吾來為之婦。」以告主母,主母感其義。主母死,婢以七十緡為之斂。事畢,要夫家以輿迎,婢撫棺痛幾絕,既蘇,再拜乘輿去。

楊氏婢,不知何許人,亦不詳其氏與名,主江西清江楊氏。楊氏之妾寡,將嫁,前一夕,呼婢,不應。怒曰:「汝,我婢也,何敢爾!」婢曰:「我楊氏婢耳,汝今誰家婦者?曰我婢我婢!」妾方持剪,墜,起,環走至曙。呼其婢曰:「我復為爾主,汝當何如?」婢叩頭泣,妾亦泣,遂謝媒妁不行。後將嫁其婢,婢曰:「人以我一言故,忍死至今,我亦終不去楊氏門。」

江貴壽妻王,名保姑,歙人。貴壽樵也,年倍王,王事之無怨語。既嫠,入縣曹氏為其女保母。曹氏女嫁,從之往。咸豐十一年,出避賊,曹氏女方娠,不能行,乃匿諸深草中,而立以護之。賊至,創喉,猶求糠和水食曹氏女,凍餒數日死。曹氏女卒得免。

張祿妻徐,深州人。同治七年,張總愚之徒破州,賊掠二女至其家,叱祿使飼馬,而令徐監二女炊。徐詰二女皆世族,炊竟,賊皆據案食,徐導二女潛出巷,指歸路。二女請徐偕,徐曰:「我去,賊且殺我夫。」歸就祿,謀偕走,賊見,問二女,徐忿罵賊,賊殺之。

任氏婢祥,不知何許人,亦不知其氏。任氏子,僕也,故家京師東郭門外,徙保定。囑其母於祥曰:「余將之廣平,餘妻不足恃,而善事餘母。」祥與其母居三年,母病,促任氏子歸,歸則母已死。任氏子慟絕而甦,夜半,猶哽咽,翌晨視之,則亦死。既斂,其妻將挾幼女嫁,祥爭之,乃留女。女方四歲,乞食以為養,鄰里義焉,共周之。持二棺還葬,祥終不嫁。

又有通州鄭氏女,婢於馬氏。馬氏中落,他奴僕皆去,而鄭獨留,侍疾,育幼主,以浣衣得值贍其主。歷七十餘年,終不去,以處子終。

王氏婢,不知其氏,石屏人。王氏夫婦皆死,其子元勛生七月,婢已嫁生子,乃撫而乳之。稍長,賣餈餌,供饘粥,令入塾,使其子事之甚謹。元勛卒舉於鄉。

徐氏女,平湖人,為曹氏婢,名曰春梅。其主死,遺子女各一。春梅年二十餘,不嫁,撫其子女。其子女有過,涕泣勸導,勤苦,畢婚嫁。其主有兄迫欲嫁之,終不行。

丁香,不知其氏,雲南南寧人。為程氏婢,程氏女嫁於吳,丁香從。吳中落,程氏女以女紅自給,丁香執役不稍怠。程氏女謂曰:「有富家以數十金聘汝,我受金,汝亦得所,盍行乎?」丁香跪,誓死相從,程氏女知其意堅,乃不復言。後益貧,丁香出為傭,得貲以養,數十年卒不嫁。

江金姑,金谿人,為硃氏婢。硃氏女歸江,媵焉。江夫婦皆卒,金姑矢不嫁,育其孤,娶婦,未有子,其孤又夭。金姑告於江氏之族為立後,佐婦撫所後子,至成立。

羅氏,荔浦僮婦也。夫死,不更嫁。僮俗善歌,或以歌誘婦,必正色不為動,以節顯於僮。

隴聯嵩妻祿,鎮雄人也。鎮雄故土司,聯嵩世領其地為土知府。卒,子慶侯嗣。雍正五年,坐事奪職,收其地,設流官。所部欲為變,祿喻之曰:「我家以忠著,今日宜安義命,毋妄動。」所部乃解。八年,烏蒙土民叛,祿親至舊所部各寨,申喻利害,至欲自殺,所部僉讋服。祿躬率眾衛官廨,佐軍食,城恃以全。總督鄂爾泰建坊表其忠,請於朝,封安人,予田二十畝,使供隴氏祀。

者架聘妻直額,貴州大定仲民。既許嫁,者架貧,不能娶。直額父母欲女別嫁,不可;強之,自殺。

羅廷勝妻馬,名阿透,寧各司羊海寨仲民女也。廷勝死,阿透年二十六,父欲為別嫁,阿透哭於廷勝墓,自經死。

羅朝彥妻劉,名阿全。朝彥,仲民;劉,甕安人。朝彥死,其弟欲妻嫂,引強暴迫劉,自殺。

安於磐妻硃、後妻田,於磐,貴州蠻夷司長官。初娶硃,事姑孝,姑病,刲股,卒。復娶田,於磐病,刲股。於磐卒,撫諸子成立。

田養民妻楊,養民,朗溪司長官;楊,邑梅司人也。年十二,母病,刲股。

李任妻矣,習瓘人,夷羅厄女也。羅厄為李氏佃,李氏欲汙之,不從。縛置積薪上,曰:「不從,將焚!」矣大罵,遂焚死。事聞,罪李氏。

鄂對妻熱依木,鄂對,庫車回頭人,與其酋霍集占有隙。霍集占以葉爾羌叛,鄂對與其子鄂斯滿棄家走,迎師於伊犁。霍集占破庫車,憾鄂對不附,執熱依木欲納之,不可;殺其子女三,而囚之,熱依木脫走。師克霍集占,授鄂對貝勒、葉爾羌阿奇木伯克,鄂斯滿二等臺吉、庫車阿奇木伯克。居數年,烏什回叛,熱依木在庫車,請於辦事大臣曰:「回性喜效尤,今烏什叛,葉爾羌戶眾,伯克、阿渾輩不知順逆,鄂對懦無斷,請得往助之。」熱依木行五日至葉爾羌,伯克、阿渾輩入見,言烏什,熱依木漫應之,期明日會飲。明日,眾集,熱依木曰:「汝等皆無藉,蒙大皇帝恩為太平民,今烏什叛,即日夷滅,乃欲效尤,為不忠不義鬼耶?吾力尚能殺爾曹,爾曹今日毋思出此門!」眾愕顧,門守甚嚴,皆跪白無反狀。熱依木乃具筵,曉以利害,眾皆泣。則出歌姬勸飲盡醉,陰使人遍收諸家戰具,驅其馬,令遠牧。鄂對日率諸伯克集辦事大臣庭,夜分散,眾大定。及烏什破,多所誅戮,葉爾羌獨全。

瓦寺土司索諾木榮宗母麥麥吉,早寡,撫索諾木榮宗成立。綏輯番落,有功於邊,被詔旌表。

明正土司堅參達結妻喇章,無子,次妻夭夭生二子。堅參達結死,喇章、夭夭同護土司印,撫二子成立。乾隆間,從征金川有功,亦被詔旌表。

沙氏女,會理州人。父為土千戶,所屬土百戶自氏富,妻以女。嫁,弟送之往。將入自氏所轄境,女語其弟曰:「自氏,奴也;汝,主也。我受父命不敢違,汝不當入。」涕泣而別。女至自氏,自氏子求合,女堅拒之,不食七日死。

嘉義番婦,加溜灣社番大治妻也。大治死,原變故俗,不更嫁,引刀誓曰:「婦發可封,婦臂可斷,婦節不可移!」力耕育其子,居三十七年乃卒。

施世燿妻苗,世燿,龍溪人;苗,傌辰港夷女。世燿死,苗自經殉焉。


\end{pinyinscope}