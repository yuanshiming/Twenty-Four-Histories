\article{志九十五}

\begin{pinyinscope}
○食貨一

明末,苛政紛起,籌捐增餉,民窮財困。有清入主中國,概予蠲除,與民更始。逮康、乾之世,國富民殷。凡滋生人丁,永不加賦,又普免天下租稅,至再至三。嗚呼,古未有也。道、咸以降,海禁大開,國家多故。耗財之途廣,而生財之道滯。當軸者昧於中外大勢,召禍興戎,天府太倉之蓄,一旦蕩然,賠償兵費至四百餘兆。以中國所有財產抵借外債,積數十年不能清償。攤派加捐,上下交困。乃改海運以節漕費,變圜法以行國幣,講鹽政以增歲入,開礦產以擴財源。以及創鐵路,改郵傳,設電局,通海舶。新政繁興,孳孳謀利,而於古先聖王生眾食寡、為疾用舒之道,昧焉不講。夫以唐、虞治平之世,而其告舜、禹也,諄諄以「四海困窮,天祿永終」為戒。有國者其可忽哉!茲取清代理財始末,條著於篇。

戶口田制

戶口清之民數,惟外籓扎薩克所屬編審丁檔掌於理籓院。其各省諸色人戶,由其地長官以十月造冊,限次年八月咨送戶部,浙江清吏司司之。而滿洲、蒙古、漢軍丁檔則司於戶部八旗俸餉處。年終,將民數匯繕黃冊以聞。

其戶之別,曰軍,曰民,曰匠,曰灶。此外若回、番、羌、苗、瑤、黎、夷等戶,皆隸於所在府、、州、縣。凡民,男曰丁,女曰口。男年十六為成丁,未成丁亦曰口。丁口系於戶。凡腹民計以丁口,邊民計以戶。蓋番、回、黎、苗、瑤、夷人等,久經向化,皆按丁口編入民數。其以戶計者,如三姓所屬赫哲、費雅喀、奇勒爾、庫葉、鄂倫春、哈克拉五十六姓,甘肅各土司,及莊浪所屬番子,西藏各土司所屬三十九族,烏里雅蘇臺所屬唐努烏梁海貢貂戶,科布多所屬阿爾泰烏梁海貢貂戶、貢狐皮戶,阿爾泰諾爾烏梁海貢貂戶、貢灰鼠皮戶,皆是。至土司所屬番、夷人等,但報明寨數、族數,不計戶者不與其數。

凡民之著籍,其別有四:曰民籍;曰軍籍,亦稱衛籍;曰商籍;曰灶籍。其經理之也,必察其祖籍。如人戶於寄居之地置有墳廬逾二十年者,準入籍出仕,令聲明祖籍回避。倘本身已故,子孫於他省有田土丁糧,原附入籍者,聽。軍流人等子孫隨配入籍者,準其考試之類是也。又必辨其宗系。如民人無子,許立同宗昭穆相當者為後。其有女婿、義男及收養三歲以下小兒,酌給財產,不得遂以為嗣之類是也。且必區其良賤。如四民為良,奴僕及倡優為賤。凡衙署應役之皁隸、馬快、步快、小馬、禁卒、門子、弓兵、仵作、糧差及巡捕營番役,皆為賤役,長隨與奴僕等。其有冒籍、跨籍、跨邊、僑籍皆禁之。

世祖入關,有編置戶口牌甲之令。其法,州縣城鄉十戶立一牌長,十牌立一甲長,十甲立一保長。戶給印牌,書其姓名丁口。出則注所往,入則稽所來。其寺觀亦一律頒給,以稽僧道之出入。其客店令各立一簿,書寓客姓名行李,以便稽察。及乾隆二十二年,更定十五條:一,直省所屬每戶歲給門牌,牌長、甲長三年更代,保長一年更代。凡甲內有盜竊、邪教、賭博、賭具、窩逃、奸拐、私鑄、私銷、私鹽、跴曲、販賣硝磺,並私立名色斂財聚會等事,及面生可疑之徒,責令專司查報。戶口遷移登耗,隨時報明,門牌內改換填給。一,紳衿之家,與齊民一體編列。一,旗民雜處村莊,一體編列。旗人、民人有犯,地方官會同理事同知辦理,至各省駐防營內商民貿易居住,及官兵雇用人役,均另編牌冊,報明理事查核。一,邊外蒙古地方種地民人,設立牌頭總甲及十家長等。如有偷竊為匪,及隱匿逃人者,責令查報。一,凡客民在內地貿易,或置有產業者,與土著一律順編。一,鹽場井灶,另編排甲,所雇工人,隨灶戶填注。一,礦廠丁戶,廠員督率廠商、課長及峒長、爐頭等編查。各處煤窯雇主,將傭工人等冊報地方查核。一,各省山居棚民,按戶編冊,地主並保甲結報。廣東寮民,每寮給牌,互相保結。一,沿海等省商漁船隻,取具澳甲族鄰保結,報官給照。商船將船主、舵工、水手年貌籍貫並填照內,出洋時,取具各船互結,至汛口照驗放行。漁船止填船主年貌籍貫。其內洋採捕小艇,責令澳甲稽查。至內河船隻,於船尾設立粉牌,責令埠頭查察。其漁船網戶、水次搭棚趁食之民,均歸就近保甲管束。一,苗人寄籍內地,久經編入民甲者,照民人一例編查。其餘各處苗、瑤,千百戶及頭人、峒長等稽查約束。一,雲南有夷、民錯處者,一體編入保甲。其依山傍水自成村落者,令管事頭目造冊稽查。一,川省客民,同土著一例編查。一,甘肅番子土民,責成土司查察。系地方官管轄者,令所管頭目編查,地方官給牌冊報。其四川改土歸流各番寨,令鄉約甲長等稽查,均聽撫夷掌堡管束。一,寺觀僧道,令僧綱、道紀按季冊報。其各省回民,令禮拜寺掌教稽查。一,外來流丐,保正督率丐頭稽查,少壯者遞回原籍安插,其餘歸入棲流等所管束。自是立法益密。

時各省番、苗與內地民人言語不通,常有肇釁之事。二十四年,定番界、苗疆禁例。凡臺灣民、番不許結親,違者離異。各省民人無故擅入苗地,及苗人無故擅入民地,均照例治罪。若往來貿易,必取具行戶鄰右保結,報官給照,令塘汛驗放始往。

棚民之稱,起於江西、浙江、福建三省。各山縣內,向有民人搭棚居住,藝麻種箐,開爐煽鐵,造紙制菇為業。而廣東窮民入山搭寮,取香木舂粉、析薪燒炭為業者,謂之寮民。雍正四年,定例照保甲法一體編查。乾隆二十八年,定各省棚民單身賃墾者,令於原籍州縣領給印票,並有親族保領,方準租種安插。倘有來歷不明,責重保人糾察報究。五十五年,諭:「廣東總督奏稱,撤毀雷、廉交界海面之潿洲及迤東之斜陽地方寮房,遞回原籍,免與洋盜串通滋事,並毀校椅灣等三十二處寮房共百六十二戶,另行撫恤安插。沿海各省所屬島嶼,多有內地民人安居樂業。若遽飭令遷移,使數十萬生民流離失所,於心何忍。且恐辦理不善,轉使良民變而為匪。所有各省海島,除例應封禁者外,餘均仍舊居住。至零星散處,皆系貧民,尤不可獨令向隅。而漁戶出洋探捕,暫在海島搭寮棲止,亦不可概行禁絕。且人民既少,稽察無難,惟在各督撫嚴飭文武員弁編立保甲。如有盜匪混入,及窩藏為匪者,一經查出,將所居寮房概行燒毀,俾知儆懼。其漁船出入口岸,務期取結給照,登記姓名。倘進口時藏有貨物,形跡可疑,嚴行盤詰,自不難立時拏獲也。」五十七年,諭:「據福寧所奏,山東一省海島居民二萬餘名口,各省海島想亦不少。當遵照前言,不準添建房屋,以至日聚日眾。仍應留心訪察,勿任勾結匪徒,滋生事端。」咸豐元年,浙江巡撫常大淳奏言:「浙江棚民開山過多,以致沙淤土壅,有礙水道田廬。請設法編查安插,分別去留。」如所議行。

四川經張獻忠之亂,孑遺者百無一二,耕種皆三江、湖廣流寓之人。雍正五年,因逃荒而至者益眾。諭令四川州縣將人戶逐一稽查姓名籍貫,果系無力窮民,即量人力多寡,給荒地五六十畝或三四十畝,令其開墾。

其吉林寧古塔、伯都訥、阿勒楚喀、拉林等地方,乾隆二十七年定例不準無籍流民居住。及三十四年,吉林將軍傅良奏:「阿勒楚喀、拉林地方流民二百四十二戶,請限一年盡行驅逐。」上曰:「流寓既在定例之前,應準入籍墾種,一例安插,俾無失所。」嘉慶中,郭爾羅斯復有內地新來流民二千三百三十戶,吉林有千四百五十九戶,長春有六千九百五十三戶,均經將軍奏令入冊安置。其山東民人徙居口外者,在康熙五十一年已有十萬餘人。聖祖諭:「嗣後山東民人有到口外及由口外回山東者,應查明年貌籍貫,造冊稽查,互相對覈。」其後直隸、山西民人亦多有出口者。

雍正初,因陸續設古北口、張家口、歸化城三同知管理,旋移萬全縣縣丞於張家口,其古北口增設巡檢一,歸化城增設通判四、巡檢一,各按所屬民人,照保甲法,將姓名籍貫注冊,逐年咨部查覈。凡民人出入關口,由原籍州縣給印票驗明放行。所有放過票張,造冊報部。

其福建、廣東民人徙居臺灣者尤眾。嘉慶十五年,浙閩總督方維甸奏:「噶瑪蘭田土膏腴,內地民人流寓者多。現檢查戶口,漳州人四萬二千五百餘丁,泉州人二百五十餘丁,粵東人百四十餘丁,與生熟各番雜處,必須有所鈐制。」於是議增噶瑪蘭通判一。此外如江蘇銅、沛兩縣,自黃河退涸,變為荒田,山東曹、濟等屬民人陸續前往,創立湖團,相率墾種。銅、沛土民因客民占墾,日相控斗。同治五年,戶部奏:「查明容留捻匪之刁、王兩團,驅回原籍。安分良團,即令各安生業。」凡此夷、漢之雜處,土、客之相猜,慮其滋事,則嚴為之防,憫其無歸,則寬為之所,要皆以保甲為要圖。

顧保甲行於平時,而編審則丁賦之所由出也。編審之制,州縣官造冊上之府,府別造一總冊上之布政司。凡軍、民、匠、灶四籍,各分上中下三等。丁有民丁、站丁、土軍丁、衛丁、屯丁。總其丁之數而登黃冊。督撫據布政司冊報達之戶部,匯疏以聞。順治十四年,命州縣官編審戶口,增丁至二千名以上,各予紀錄。康熙五十一年,有「新增人丁永不加賦」之諭,自是聖祖仁政,遂與一代相終始。顧丁有開除,即不能不有抵補。故康熙五十五年,戶部請以編審新增人丁補足舊缺額數,如有餘丁,歸入滋生冊內造報,從之。高宗諭內閣曰:「朕查上年各省奏報民數,較之康熙年間,計增十餘倍。承平日久,生齒日繁,蓋藏自不能如前充裕。且廬舍所占田土,亦不啻倍蓰。生之者寡,食之者眾,朕甚憂之。猶幸朕臨御以來,闢土開疆,幅員日廓,小民皆得開墾邊外地土,藉以暫謀衣食。然為之計及久遠,非野無曠土,家有贏糧,未易享升平之福。各省督撫及有牧民之責者,務當隨時勸諭,俾皆儉樸成風,惜物力而盡地利,慎勿以奢靡相競,習於怠惰也。是時編審之制已停,直省所報民數,大率以歲造之煙戶冊為據。行之日久,有司視為具文,所報多不詳覈,其何以體朕欲周知天下民數之心乎?」又諭:「據鄭輝祖稱,從前所辦民數冊,歲歲滋生之數,一律雷同。似此簡率相沿,成何事體!所有各省本年應進民冊,均展至明年年底。倘再疏舛,定當予以處分。」當時民冊恐不免任意填造之弊,然自聖祖以來,休養生息百有餘年,民生其間,自少至老,不知有兵革之患,而又年豐人樂,無有夭札疵癘,轉徙顛踣以至於凋耗者,其戶口繁庶,究不可謂盡出子虛也。

至編審之停,始於雍正四年。直隸總督李紱改編審行保甲一疏略云:「編審五年一舉,雖意在清戶口,不如保甲更為詳密,既可稽察游民,且不必另查戶口。請自後嚴飭編排人丁,自十六歲以上,無許一名遺漏。歲底造冊,布政司匯齊,另造總冊進呈。冊內止開裏戶人丁實數,免列花戶,則簿籍不煩而丁數大備矣。」乾隆五年,戶部又請令各督撫於每年十一月,將戶口數與穀數一並造報;番疆、苗界不入編審者,不在此例。從之。三十七年,從李瀚請,永停編審。自是惟有運漕軍丁四年一編審而已。

蓋清承明季喪亂,戶口凋殘。經累朝休養生息,故戶口之數,歲有加增。約而舉之:順治十八年,會計天下民數,千有九百二十萬三千二百三十三口。康熙五十年,二千四百六十二萬一千三百二十四口。六十年,二千九百一十四萬八千三百五十九口,又滋生丁四十六萬七千八百五十口。雍正十二年,二千六百四十一萬七千九百三十二口,又滋生丁九十三萬七千五百三十口。乾隆二十九年,二萬五百五十九萬一千一十七口。六十年,二萬九千六百九十六萬五百四十五口。嘉慶二十四年,三萬一百二十六萬五百四十五口。道光二十九年,四萬一千二百九十八萬六千六百四十九口。咸、同之際,兵革四起,冊報每缺數省,其可稽者,只二萬數千萬口不等。光緒元年,三萬二千二百六十五萬五千七百八十一口。

三十二年,釐定官制,以戶部為度支部,而改前所設之巡警部為民政部,調查戶口,歸其職掌,各省則以巡警道專司其事。明年,諭直省造報民數,務須確查實數,以為庶政根本。民政部奏稱:「伏查三十二年黑龍江、安徽、江蘇、福建、甘肅、廣西、雲南丁冊,並三十一年丁冊,均未補造。在各督撫明知逾限,例當查參,而積習挽回不易。臣部於接收伊始,籌一切實辦法,擬請敕下各督撫,責成府、、州、縣,分鄉分區,自行調查丁口確數,統以每年十二月底截算,以清界限。仍限次年十月送部匯奏。」制可。

宣統元年,復頒行填造戶口格式,令先查戶口數,限明年十月報齊,續查口數,限宣統四年十月報齊。至三年十月,據京師內外城、順天府、各直省、各旗營、各駐防、各蒙旗所報,除新疆、湖北、廣東、廣西各省,江寧、青州、西安、涼州、伊犁、貴州、西寧各駐防,泰寧鎮、熱河各蒙旗,川、滇邊務,均未冊報到部外,凡正戶五千四百六十六萬八千有四,附戶千四百五十七萬八千三百七十,共六千九百二十四萬六千三百七十四戶;凡口數男一萬三千九百六十六萬二千四百一十,女九千九百九十三萬二千二百有八,共二萬三千九百五十九萬四千六百六十八口。

自雍正十三年戶部題準,福建臺灣府生番百九十九名,匯入彰化籍,廣西慶遠府歸流土民百七十九名,匯入宜山籍,嗣後臺灣生番、四川生番、嶺夷歸化者甚眾,定例令專管官編立保甲,查緝匪類,逢望日宣講上諭,以興教化,自是番民衣冠言語悉與其地民人無異,亦有讀書應考者。

及同治、光緒間,交通日廣,我國之民耕種貿遷,遍於重瀛,亦有改入他國版籍之事。宣統元年,外務部會同修訂法律大臣擬定國籍條例。因各國國籍法有地脈系、血脈系,即屬地、屬人兩義,兩義相持,必生牴觸,於是採折衷制,分為固有籍、入籍、出籍、復籍四章,注重血脈系辦法。憲政編查館就所定四章釐為二十四條。

其固有籍章,第一,凡不論是否生於中國,均屬中國國籍者,其疑有三:一,生而父為中國人者;二,生於父死以後而父死時為中國人者;三,母為中國人而父無可考,或無國籍者。第二,若父母均無所考,或均無國籍,而生於中國地方者,亦屬中國國籍。其生地並無可考而在中國地方發見之棄兒,同。

其入籍章,第三,凡外國人原入中國國籍者,準其呈請入籍。其必具備之款五:一,寄居中國接續至十年以上者;二,年滿二十歲以上,照其國法律為有能力者;三,品行端正者;四,有相當之貲財或藝能,足以自立者;五,照其國法律,於入籍後即應消除本國國籍者。其本無國籍人原入中國國籍者,以年滿二十歲以上,並具備前項第一、第三、第四款者為合格。第四,凡外國人或無國籍人有殊勛於中國者,雖不備一至四各款,得由外務部、民政部會奏請旨,特準入籍。第五,凡外國人或無國籍人婦人嫁與中國人者;以中國人為繼父而同居者;私生子,父為中國人,經其父認領者;私生子,母為中國人,父不原認領,經其母認領者。如有此等情事之一,均作為入籍。惟婦女嫁與中國人,須以正式結婚呈報有案者為限。餘款以照其國法律尚未成年及未為人妻者為限。第六,凡男子入籍者,其妻及未成年之子應隨同入籍。其照其國法律並不隨同銷除本國國籍者,不在此限。若其妻自原入籍,或入籍人自原使未成年之子入籍者,雖不備第三條一至四各款,準其呈請入籍。第七,入籍人成年之子現住中國者,唯不備第三條一至四各款,亦準呈請入籍。第八,凡入籍人不得就之官職:一,軍機處、內務府各官及京、外四品以上文官;二,各項武官及軍人;三,上下議院及各省諮議局議員。此等限制,特準入籍人十年以後、餘入籍人二十年以後,得由民政部請旨豁免。第九,凡呈請入籍者,應聲明入籍後遵守中國法律,及棄其本國權利,出具甘結,並由寄居地方公正紳士二人各出具保結。第十,凡呈請入籍者,應具呈所在地方官,詳請所管長官咨請民政部批準牌示,給予執照為憑。其在外國者,應具呈領事,申由出使大臣,或徑呈出使大臣咨部存案。

其出籍章,第十一,凡中國人原入外國國籍者,應先呈請出籍。第十二,凡中國人準出籍,其款有四:一,無未結之刑、民訴訟案件;二,無兵役之義務;三,無應納未繳之租稅;四,無官階及出身。第十三,凡中國人婦女嫁與外國人者;以外國人為繼父而同居者;私生子,父為外國人,其父認領者;私生子,母為外國人,其父不原認領,經其母認領者。如有此等事情之一,均作為出籍。惟婦女嫁與外國人,以正式結婚呈報有案者為限。餘款以照中國法律尚未成年及未為人妻者為限。第十四,凡男子出籍者,其妻及未成年之子一並作為出籍。若妻自原留籍,或出籍人原使其未成年之子留籍,準其呈明,仍屬中國國籍。第十五,凡婦女有夫者,不得獨自呈請出籍。其照中國法律尚未成年及無能力者,亦不準自行呈請出籍。第十六,凡中國人出籍者,所有在內地特有之利益,一律不得享受。第十七,凡呈請出籍者,應自行出具甘結,聲明並無第十二條所列各款及犯罪未經發覺情事。第十八,凡呈請出籍者,應具呈本籍地方官,詳請該管長官咨請民政部批準牌示。其在外國者,應具呈領事,申由出使大臣,或徑呈出使大臣咨部。其未經呈請批準,不問情形如何,仍屬中國國籍。

其復籍章,第十九,凡因嫁外國人而出籍者,若離婚或夫死後,準其呈請復籍。第二十,凡出籍人之妻,於離婚或夫死後,及未成丁之子已達成年後,均準呈請復籍。第二十一,凡呈準出籍後,如仍寄居中國接續至三年以上,合第三條三、四款者,準其呈請復籍。其外國人入籍後又出籍者,不在此限。第二十二,凡呈請復籍,應由原籍同省公正紳商二人出具保結,並具呈所在地方官,詳請所管長官咨請民政部批準牌示。第二十三,凡復籍者,非經過五年後,不得就第八條所列各款之官職。第二十四,本條例自奏準奉旨後,即時施行。

此外改籍為良,亦有清善政。山西等省有樂戶,先世因明建文末不附燕兵,編為樂籍。雍正元年,令各屬禁革,改業為良。並諭浙江之惰民,蘇州之丐戶,操業與樂籍無異,亦削除其籍。五年,以江南徽州有伴儅,寧國有世僕,本地呼為「細民」;甚有兩姓丁口村莊相等,而此姓為彼姓執役,有如奴隸,亦諭開除。七年,以廣東蜑戶以船捕魚,粵民不容登岸,特諭禁止。準於近水村莊居住,與齊民一體編入保甲。乾隆三十六年,陜西學政劉嶟奏請山、陜樂戶、丐戶應定禁例。部議凡報官改業後,必及四世,本族親支皆清白自守,方準報捐應試。廣東之蜑戶,浙江之九姓漁船,諸似此者,均照此辦理。嘉慶十四年,又以徽州、寧國、池州三府世僕捐監應考,常為地方所訐控,上諭:「此等名分,總以現在是否服役為斷。如年遠文契無考,著即開豁。」

八旗人丁,定例三年編審一次,令各佐領稽查已成丁者,增入丁冊。有隱匿壯丁入官,伊主及佐領、領催各罰責有差。凡壯丁三百名為一佐領,後改定為二百名。康熙四年,令滿洲、蒙古佐領內餘丁多至百名以上,原分兩佐領者,聽。雍正四年,諭八旗都統及直省駐防都統、將軍等,交與佐領、驍騎校、領催,將新舊壯丁逐戶開明,並編審各官姓名,保結送部。其未成丁,及非正身良家子弟,並應除人丁,驗實開除。五年,令凡編審丁冊,每戶書另戶某人某官,無官則曰閒散某,上書父兄官職名氏,傍書子弟及兄弟之子,及戶下若干人。或在籍,或他往,皆備書之。其各省駐防旗員兵丁,及外任文武各官子弟家屬,令各將軍、督撫造冊咨送該旗。乾隆六年,令八旗編審各佐領下已成丁及未成丁已食餉之人,皆造入丁冊,分別正身開戶,戶下於各名下開寫三代履歷。其戶下人祖父或系契買,或系盛京帶來,或系帶地投充,分別注明。正戶之子弟,均作正身分造。

七年,諭:「八旗漢軍,其初本系漢人。有從龍入關者,有定鼎後投誠者,有緣罪入旗與夫三籓戶下歸入者,有內務府、王公包衣撥出者,以及招募之砲手,過繼之異姓,並隨母因親等類,先後歸旗,情節不一。中惟從龍人員子孫,皆系舊有功勛,無庸另議更張。其餘各項人民等,朕欲廣其謀生之路。倘原改歸原籍,準其一例編入保甲。有原外省居住者,亦準前往。此內如有世職,仍許承襲。不原出旗者,聽。」八年,又諭:「前降諭旨,原指未經出仕及微末之員而言。至於服官既久,世受國恩之人,其本身及子弟,均不得呈請出旗。」十二年,又諭:「八旗別載冊籍之人,原系開戶家奴冒入正戶,後經自行首明,及旗人抱養民人為子,有原出旗為民者,其入籍何處,均聽其便。本身田產,並許帶往。」二十六年,定漢軍凡現任外省自同知、守備以上,京員自主事以上,旗員自五品以上,俱不許改歸民籍。其餘在京報明該旗咨部轉行各省,在外呈明督撫咨報部旗,編入民籍,並準一體考試。

大抵清於八旗皆以國力豢養之。及後孳生籓衍,雖歲糜數百萬金,猶苦不給,而逃人之禁復嚴,旗民坐是日形困敝。及乾隆初,御史舒赫德、範咸、赫泰,戶部侍郎梁詩正等,先後奏請清查東三省曠地,俾移住開墾,以圖自養。雖疊奉諭旨議行,然終未能切實舉辦。至八旗戶下人開戶,必有軍功勞績,或藝能出眾,亦有本主念其服勤數世,準其另戶,或放出為民者,亦有不準放出為民,但準開戶者,其例又各不同雲。

田制曰官田。初設官莊,以近畿民來歸者為莊頭,給繩地,一繩四十二畝。其後編第各莊頭田土分四等,十年一編定。設糧莊,莊給地三百晌,晌約地六畝。莊地坐落順、保、永、宣各屬,奉天、山海關、古北口、喜峰口亦立之,皆領於內務府。此外有部、寺官莊,分隸禮部、光祿寺。又設園地,植瓜果蔬菜,選壯丁為園頭。世宗初,設總理專官,司口外報糧編審。南苑本肄武地,例禁開田。宣宗嘗諭前已開者並須荒棄。而咸、同間,嵩齡、德奎、劉有銘、鐵祺先後疏陳開放,均嚴旨詰斥。然至光緒季年,仍賦予民。自後承地者乃接踵矣。

考各旗王、公、宗室莊田,都萬三千三百餘頃。分撥各旗官兵,都十四萬九百餘頃。凡王公近屬,分別畀地,大莊給地畝四百二十至七百二十,半莊二百四十至三百六十,園給地畝六十至百二十或百八十,王府管領及官屬壯丁人三十六畝,不支糧。凡撥地以現在為程,嗣雖丁增不加,丁減不退。

順治元年,定近京荒地及前明莊田無主者,撥給東來官兵。圈地議自此始。於是巡按御史柳寅東上滿、漢分居五便。部議施行。二年,令民地被指圈者,速籌補給,美惡維均。四年,圈順直各州縣地百萬九千餘晌,給滿洲為莊屯。八年,帝以圈地妨民,諭令前圈占者悉數退還。十年,又令停圈撥。然旗退荒地,與游牧投來人丁,仍復圈補。又有因圈補而並圈接壤民地者。康熙初,鼇拜專柄,欲以正白旗屯莊予鑲黃旗,而別圈民地圈補。戶部尚書蘇納海、總督硃昌祚、巡撫王登聯咸以不如指,罪至死。聖祖親政,諭停止圈地。本年所圈房地俱退還。又以張家口、山海關等處曠土換撥各地,並令新滿洲以官莊餘地撥給,其指圈之地歸民。是為旗退地畝。

凡官地,例禁與民交易。然旗人不習耕種,生齒日繁,不免私有質鬻。雍正初,清理旗地,令頒帑贖回。凡不自首與私授受者,胥入官為公產。旗地,令宗人府、內務府八旗具各種地畝坐落四至,編制清冊,是為紅冊,以備審勘旗民田土之爭。乾隆初,定回贖旗地仍歸原佃承種,莊頭勢豪爭奪者罪之。凡贖入官地並抵帑、籍沒等田,皆徵租,曰旗租。舊查交入官地定租,由旗員主之。三十四年,以直督楊廷璋言,停其例。民租旗地,本限三年。或私行長租,業戶、租戶科以違禁律。八旗地主,久禁奪佃增租。自和珅筦大農,奏改前章,於是旗人及府莊頭率多撤地別佃,貧民始多失業。嘉慶五年,部臣請復申前禁。詔纂入定例通行。咸豐初元,又申令如額徵租,主佃皆不得以意贏縮。若典鬻旗地,從盜賣官地律,授受同懲。顧日久法疏,或指地稱貸,或支用長租,陽奉陰違,胥役訛索句結,弊遂叢生。雖屢申明誡,往往因他故,禁弛靡常。洎光緒中,乃定此業無論舊圈自置,概不準售與民人。惟從前民購升科者,仍予執業。

盛京官莊,於順治初即定八旗屯界。旋令沙河以外、錦州以內,旗員家丁給地,人三十六畝。康熙中,定以奉天所屬地畀新滿洲遷來者,凡丈出地為頃三十二萬九千餘,以二十七萬六千三百餘頃為旗地,按旗分界。又設各旗官員莊屯,各城兵丁,均酌給隨缺地畝。旋令索倫、達呼爾官兵耕種墨爾根地,奉天官兵耕種黑龍江地。乾隆初,設黑龍江屯莊,呼蘭立莊四十所,選盛京旗丁攜家往,官為資裝築屋庀具,丁給地畝六十,十丁一莊,每六畝給籽種二斗,莊給牛六頭,口糧並給。溫德亨、都爾圖亦如之。凡隨缺官地歸旗入冊,禁職官侵占。嘉慶間,令盛京入官地畝,應招無地貧民領租,職官子弟不得承種。管界各官,並不得於所管區以子弟之名置房地。道光中,寧古塔、伯都訥、三姓、阿勒楚喀、拉林各官莊,共原額地萬二百晌,吉林八旗與各處旗地暨烏拉旗地,共三十六萬五千九十二晌。而光緒初,撥三姓荒為官兵隨缺地,計晌二萬九千餘。宣統時,以奉省各旗地多盜典隱占之弊,令通稽確覈,毋與清賦溷淆,先城旗,後外城,依次釐定。此官莊之屬東三省者。

直省各置駐防旗兵,立莊田於所駐地,給田人各三畝。其全眷挈赴者,前在京所得圈地撤還。旗員分畀園地,多則二百四十畝,少則六十畝,各省不盡同。惟浙江駐防無田,仍支俸饟。乾隆時,弛防兵置產之禁,惟八旗官仍禁如故。光緒之季,諭:「所在檢旗丁名數,侭舊有馬廠莊田,畫地口分,責以農作。其本無廠田,或有而弗備者,所司於鄰近分購民地配發,以為世業。由漸推廣,俾旗丁歸農,受治州縣,與齊民不異。」未及實施。蒙古初分五等。一、二等備與莊屯、園地。三等以下,祗與莊屯。各守土疆,毋得越境。後漸有民人賤收蒙地者。乾隆中定「有質鬻者峻罰之,著為永令」。分撥外籓官地,其略如此。故明內監莊田,總領於戶部。其宗室祿田散在各省者,胥視民田起科。先是以新城、固安官地二百田十頃制井田,選旗民百戶,戶授百畝,公百畝,共力養公田。嗣更於霸州、永清仿行,然成效卒鮮。乾隆初,改屯莊。擇勤敏者充屯戶,按畝科糧。是為井田改屯地。

凡京師壇壝官地,暨天下社稷、山川、厲壇、文廟、祠墓、寺觀、祭田公地,一切免徵。建國初,賜聖賢裔祭田。其孔林地、四氏學學田、墓田地、墳地,咸除租賦。學田,專資建學及贍恤貧士,佃耕租而租率不齊,舊無常額。乾隆中,都天下學田萬一千五百八十餘頃。光緒變法,直省遍興學堂,需費無藝,則又撥所在荒地,劃留學田以補劑之。耤田行於首都先農壇。壇地凡千七百畝。雍正間,令疆吏飭所屬置耤田。東西陵地,紅椿以內例絕耕樵。東陵白椿界外初聽民耕。道光朝乃嚴其禁,青椿以外,遵、薊、密、承諸界內兵民私墾,至地萬餘區,久益增廓。光緒末,定為計區勘丈,將熟地分則升科,儲學堂之用焉。牧馬草場在畿輔者,順治二年,以近畿墾荒餘地斥為牧場,於順天、津、保各屬分旗置之。自御馬廠以下,各按其旗地牧養。親王方二里,郡王一里,亦圈地也。

曰屯墾。康熙中,招墾天津兩翼牧地,計畝二萬一千五百餘。乾隆時,丈直隸馬廠地振業貧民,命曰恩賞官地。在盛京者,奉天屯衛各地,八旗分作牧廠,自東迤西,本禁民墾,於定界所築封堆制限之。然大凌河東廠、西廠荒地三十一萬八百餘畝,養息牧餘地萬四千六百晌,乾、嘉中陸續放墾。後又綜各城旗馬廠可墾地三十八萬九千餘畝,悉歸城旗承種,並令八旗王公及閒散宗室,於所分牧地原墾者,得自呈報。惟松筠請於養息閒壤移駐旗人,以費絀而罷。咸豐中,以大凌西岸墾妨馬政,申禁如前。而同治二年,變通錦州、廣寧、義州廠荒,西廠留牧,東廠招佃;其東北隅之高山子地數萬畝,義州教場閒地萬餘畝,並行租佃,以為城兵伍田。然是時西廠有旗領舊地,久而越墾妨牧。八年,命劃棄之。於是大凌河墾議遂沮。而吉、黑山荒多牧獵場,益嚴杜奸民攬售矣。養息牧地,初放時判東西界,置專官掌其租入。彰武本官牧,旋亦勸墾議科。於是養息牧生熟地共放六十一萬八千八百餘畝,其餘荒八萬九千六百餘畝,餘地三萬五千三百餘畝,即以為蒙、漢雜居牧佃,兼拊畜窮黎。吉林之烏拉,康熙時,於五屯分莊丁地,遂為五官牧場,頗富零荒。宣統時,撥充學田,放墾實地二千三百餘晌。

凡駐防營皆置馬廠,其牧莊旁餘,靡不放墾。至荊防馬廠墾熟之地,久畀諸民,而石首、監利,光緒末釐出廠地二萬餘畝,俱令招墾,以租息濟警政小學。宣統初,寧夏滿營牧地餘界,開渠墾地,畝可二十一萬,旗、民各半之。民領則納價為旗兵墾本。三年,安徽萬頃湖牧場,改墾放田八萬二千七百餘畝,其流民占耕及民間認荒者,皆名曰佃民,其留旗丁田二萬畝,亦招民佃,歲輸穀麥,是為官佃。至是以抗租膠葛,定議民租田,令公司補價承業,資八旗生計焉。

口外牧場,隸獨石者為御馬廠。此外禮部、太僕寺、左右翼及八旗,均有牧場在張家口外。而殺虎口之議畝租,察哈爾屬之戢私墾,大青山之寬免民占,奕興地之招商領耕,列朝因時制宜,不拘成例。其後密雲、熱河同時放荒。熱河寬曠,於留牧外得地千四五百頃,更以三一留牧,餘咸招墾。地利闢而耕牧不相妨,甚善政也。

明之設衛也,以屯養軍,以軍隸衛。洎軍政廢而募民兵,屯軍始專職漕運,無漕者受役不息,屯戶大困。清因明之舊,衛屯給軍分佃,罷其雜徭。順治元年,遣御史巡視屯田。三年,定屯田官制。衛設守備一,兼管屯田。又千總、百總,分理衛事。改衛軍為屯丁。六年,定直隸屯地輸租例。其時裁屯田御史,繼裁巡按,由巡撫主之。十三年,定屯軍貼運例。浙江各衛有屯無運與無屯有運者,均徵撥帖,屯戶困始少蘇。康熙十五年,以各衛荒田在州縣轄境,軍地民田多影射,令檄所司清釐。雍正二年,從廷臣請,並內地屯衛於州縣,裁都司以下官。惟帶運之屯,與邊衛無州縣可歸者,如故。九年,令屯衛田畝可典與軍戶,不得私典與民。

乾隆元年,豁免廣東屯田羨餘,因除各省軍田額外加徵例。先是屯丁鬻產,官利其稅入,給契允行。至此又令運田歸船者,並禁軍民復典。實則各省典屯於民,所在而有。六年,定屯田限一年。無論在軍在民,並清出歸丁贍運。十二年,漕督顧琮請田已典與民者,令旗丁購贖。然民執業久,丁貧無以贖,從阿思哈言,釐江西丁田,在軍歸軍,在民增租給丁,永為定制。三十七年,又以漕督嘉謨奏,命清理湖廣、江、浙、山東等省屯田。明年,裴宗錫因陳兩江向不歸運之裁衛屯田,加徵津費。帝以累民,不允。四十年,鄂撫陳輝祖奏:「武昌諸衛清出典鬻屯田,請加津贍運。」部議:「如此則私相授受者知誡,而仍不病失業,庶典鬻之弊漸除。」五十年,以長沙、澧州原有弁田,轉售紛紜,令除弁田名,準民產授受。五十四年,畢沅等奏,各省屯丁四年一編審,止稽戶口之數,其田產或有漏匿,以時覈之。百餘年來,屯田利病與漕運終始。及南漕改海運,屯衛隱蔽難稽,至是而一大變。

光緒二十四年,太常卿袁昶奏理屯田,因有改衛為屯之諭,令天下覈衛田畝數,詳定租章。而江西以租悉充餉,與他省贍運者不同,籥仍舊貫。二十七年,劉坤一、張之洞條議屯衛宜裁。略稱:「運軍久虛,衛官復無事,一衛所屬屯田,或隔府,或跨省,一切操諸胥吏之手,田餉弊竇,不可勝窮。」明年,諭各省勘實屯地,檄屯戶稅契執業,改屯餉為丁糧,歸州縣徵解。除屯丁、運軍名目,裁衛官。是時綜計各省屯田約二十五萬餘頃,顧多與民田殽雜。又各丁私相質售,久失其舊。重以兵後冊籍蕩然,糧產無從鉤金。漕督陳夔龍陳大要三端:一,分丁業民業;一,現徵毋追原額;一,補繳田價宜輕。而江、皖、兩浙俱折衷定規,分別交價輸稅。如淮、揚、徐四衛,定有上則三兩、中二兩、下一兩,屯稅每兩納三分,餘互有同異。惟山東以艱歉請免徵納。鄂督張之洞則謂湖北衛田,軍戶仰贍,即民人冒替,率非素封,均難責其呈價,僅有徵契稅而已。其稅價視民田率。洎三十一年,宜城屯口構釁,以衛田例不便也。之洞更籌簡易八法,大旨刪除原則,分年減稅豁派,累免雜課。但學堂捐與民田同,以備改屯為民。如式者官予文證。嗣湘省亦仿此行焉。宣統元年,浙撫增韞更請令承田者但刻期報明,統不納價。部議即允占業,屯價不妨量收。蓋屯衛嬗變,時勢然也。

清自開創初,撥壯丁於曠土屯田。又近邊屯處,築城設兵以衛農人。世祖始入關,定墾荒興屯之令。凡州、縣、衛無主荒地,分給流民及官兵屯種。如力不能墾,官給牛具、籽種,或量假屯資。次年納半,三年全納。大學士範文程上屯田四事:一,選舉得人;一,收穫適宜;一,轉運有方;一,賞罰必信。上是之。令凡自首投誠者,授荒田為永業。魏裔介亦請饑民轉徙,得入籍占田。罪徒當遣者,限年屯墾,已事釋還。其原留占業者,聽。定直省屯田,官助牛種者,所收籽粒三分取一;民自備者,當年十分取一,二年、三年三分取一。初定勸懲例,限年之法甚嚴。康熙初,慮官吏虛報攤派,停限年令。尋御史徐旭林論墾荒三弊,言甚切至,然限年卒不可行。旋令士民墾地二十頃,試其文理優者,以縣丞用;百頃以知縣用。凡新墾地,初定三年起科。嗣又寬至六年後。尋令通計十年。既仍用六年例,亦有循三年舊制者。

雍正初元,諭升科之限。水田六年,旱田十年,著為例。當順、康間,直省大吏以開拓為功,其報墾田總額,多者如河南,至萬九千三百六十一頃,少者如山東,百二十頃有奇。世宗末年,以數多不實,嚴誡審覈。其有浮飾,論如律。定議敘法。凡官吏召佃資墾者,按戶數多寡,軍民自措工本者,按畝數多寡行之。乾隆時,令官山、官地,無論土著、流人,以呈報之先後予墾。民地由業主先報。或實力絀,他人始得承之。凡屯戶加墾者,俱令改屯升科。又令已墾之地,宜慎防護。凡官民地,於水道蓄洩相關,毋擅行墾。儻帖己業,私墾塘堰陂澤為田,立予懲艾。

今考歷朝屯墾之政,首直省屯田,次新疆屯田,次東三省開墾,次蒙古開墾,及青海、熱河等處墾務,悉具於篇。

當順治初元,令山西新墾田免租稅一歲;而河南北荒地九萬四千五百餘頃,允巡撫羅繡錦言,俾兵課墾。二年,順天行計兵授田法,每守兵予可耕田十畝,牛具、籽種官資之。又直隸、山東、江北、山西,凡駐滿兵,給無主地令種。四年,給事中梁維本請開秦、豫及廬、鳳荒田。六年,令各省兼募流民,編甲給照,墾荒為業,毋豫徵私派,六年後按熟地徵糧。十年,定四川荒地聽民開墾。陜荒則酌調步兵,官給牛、糧。

康熙六年,定江、浙等省分駐投誠官兵屯田,人給荒田五十畝,得支餉本。其眷屬眾者,畝數量口遞加。福建無荒,則分駐有屯諸省。七年,御史蕭震疏言:「國家歲費,兵餉居其八,而綠旗兵餉又居其八。誠屯田黔、蜀,以駐郡縣之兵,耕郡縣之地,則費省而荒漸闢。」下部議行。時直隸、陜西、粵、閩先後定墾荒例,而四川更立特例,官吏準立功論。於是湘、鄂、閩、魯、晉、豫等省空荒任民播種,限年墾齊。

雍正四年,甘肅、寧夏之插漢、托輝地平衍,可墾田六十萬餘畝,招戶認領,戶授百畝。五年,粵督阿克敦陳近年粵東墾弊四:一,豪強占奪;一,胥吏婪索;一,資本不充;一,土瘠懼為課累。勸導法五:定疆界,杜苛取,貸籽種,輕科額,廣招徠。其後惠、潮貧民墾肇慶屬地,高、廉、雷屬山荒墝埆,皆給資招墾,並免升科。嗣瓊州亦如之。又擴滇、黔墾計,烏蒙兵民並承,戶勿逾二頃。其各省入蜀民人,戶給水田畝三十,旱田畝五十。甘肅安西久行兵墾,移眷駐防,以與涼、肅二鎮。屯兵多貧,墾貲悉出官貸,並令邊省、內地零星可墾者,聽民、夷墾種;及山西新墾瘠地,自十畝以下,陜西畸零在五畝以下,俱免升科。凡隙地及水沖沙雜,與田不及畝者,及邊省山麓河壖曠土,均永遠免科。浙江新漲沙塗,民、灶皆承領,百畝為號,十號為甲,十甲老農導耕。後值漲地,人咸利之。嗣有侵墾西湖之禁。乾隆五十九年,巡撫吉慶言,沿海沙地灘漲靡常,約十三萬三千餘畝,悉令入官,交原佃耕作納租,永著為例。凡各省州縣每歲新墾荒田荒地,以及蕩地湖淤,督撫隨時疏報升科。蓋雍、乾以來,各省軍屯民墾,稱極盛焉。

福建各番鹿場曠土,例許租與民耕。然臺灣自歷任鎮臣創莊招佃,往往侵據民、番地。乾隆時,諭禁武弁墾荒。旋禁土民私購番田。五十三年,福康安請撥餘地畀番、民自種,遴壯健作屯丁。內山未墾及入官荒廢埔地八千八百餘甲,每甲準民田十一畝零,共屯丁四千,分地任耕,免賦而不給餉,從之。嘉慶中,噶瑪蘭開闢田園七千五十甲有奇。道光初,定番社未墾荒埔分給民人徵租。粵西設土兵、俍兵,均給軍田。粵東有俍田、瑤田,仍按田充兵,其田均禁民典。臺灣番地亦然。顧雲南永北、大姚等處,漢典夷地,積隙數十年。道光建元,措理稍定。十三年,四川復有漢耕夷地之釁,乃析界址,令漢、夷不得互占。又用滇督阮元議,禁流民私佃苗田,並近苗客戶典售苗產。十六年,以開化、廣南、普洱地多曠閒,流民覆棚啟種,因議論入戶甲。御史陶士霖論其病農藏奸,禁之。

先是江蘇漲灘,冒墾日甚,迨道光八年,始定歸公。而官產民業,糾互繳繞。於是江督陶澍建言聽民承售。部議江河不以墾殖為利,則沙洲不得以占鬻徇民。仍一律入官處置。尋耆英謂「民間價購興築,一旦奪還,跡類爭利。請寬其既往而閼其將來」。從之。二十三年,祁言修復虎門等砲臺,須屯田防護。明年,程矞採募丁二千試行。上曰:「以本地之民種本地之田,守要隘即捍身家,允為長算。」

同治初元,以軍儲亟,檄鳳、潁等屬戍兵墾鄰近廢田,以漸推行諸郡。山東遭教匪之亂,鄒、滕諸縣田里為墟。三年,決用移民策,而東昌、臨清、兗、曹各屬逆產及絕戶地,盡沒入官。五年,乃有辦理湖團之諭。湖團者,曹、濟客民種蘇、齊界銅、沛湖地,聚族立團。既而土著歸鄉,控鬩無已。然客墾由官招集,不乏官荒,所占土田不甚廣,且訟者非實田戶也。於是曾國籓研燭其情,為之驅逐莠戶,留其良團,各安所業。陜西叛絕荒產,前一歲諭令籌設屯田。巡撫劉蓉言軍事方殷,不如招墾便。部從其議。乃定募墾新章四:曰正經界,立制限,緩錢糧,定租穀。廣東沿海沙地,定例水涸報勘,承墾者人勿過一頃,三年成熟,照水田起科。至後搢紳壟斷侵漁,因命查文禁止。

當是時,值東南兵火之餘,農久失業。光祿少卿鄭錫瀛言國家歲入金約四千數百萬,餉糈支耗半之,宜廣屯田養兵以節費。尋御史汪朝棨稱各省新復土疆,宜急墾闢。徐景軾亦以修農利、安流徙為言。由是曾國籓於皖,楊昌濬於浙,皆分別土、客,部署開荒。而馬新貽於蘇,劉典於陜,亦汲汲督勸。曾璧光、黎培敬前後於黔興屯田之政。八月,用蘇廷魁言,籌墾蘭儀以下乾河灘地。十一年,諭陜西延、榆各屬,地瘠民貧,宜亟墾闢,嚴州縣考成。時回眾初就撫也。

先是御史黃錫彤請設蘇、皖屯營,選湘、淮散勇墾沿江地。光緒二年,硃以增亦言:「或謂屯政宜邊陲不宜腹地,不知有荒可墾,何兵不可農,何地不可屯?但抽調數營,陸續興舉,將來化兵為農,裨國非細。」時津海防兵營墾有效,故云然。曾國籓嘗言:「必得千畝無主之田,不與民田雜,方可資兵立屯。」李鴻章亦謂兵民雜處,不宜於內地。議遂寢。

初貴州屯軍於古州、八寨、臺拱、丹江、清江五,分設百二十堡,為屯八千九百三十九戶。戶給上田六畝,中八畝,下十畝,附近山地不限。逮乾隆中,禁止承佃屯軍私鬻。嘉慶初,銅仁、石峴苗地建碉卡,置屯軍,每軍百名,設百戶一,總旗二。每軍一名予水田四畝,百戶六畝,總旗五畝,皆免租。洎同治初,更定黎平屯章。及是,羅應旒言:「黔苗建屯已久,虛名鮮實,不如去兵之名,收農之實。」時屯軍凡十衛,尋奏定分為兩番,與守兵同,操防徵調各額,屯設之百戶、總旗等。有不力者,立時革替。先是沈桂芬有疏陳安置旗人聽往各省之議。御史黃元善亦稱山西暨江蘇等省開荒,當仿雙城堡舊章,令旗民移墾。顧以事體艱鉅,未盡舉也。十二年,臺灣巡撫劉銘傳籌墾內山番荒,伐木變價,以資撫恤。十六年,湖南洞庭新漲淤洲,建南洲治,入官佃租,共勘實民田十三萬餘畝,官田八萬九千二百餘畝。二十二年,桂撫史念祖言,墾西各屬官民荒田可墾,令官力為倡,酌簡屯兵,督令開熟,任民領耕,量地厚薄定科,計各屬總墾荒田萬四千三百餘畝。

時陜西清荒甚力,巡撫張汝梅言:「陜地兵祲交乘,百姓流散,北山氣候,夏寒霜早,穡事無憑,又人工少而穀價廉,得不償失。匪惟客民去留無定,即土民亦作輟靡常。欲求地不復荒,惟紓首墾期限,寬牧令責成,則民少逃亡,官不顧慮,而公私兩益矣。」二十五年,定新陽荒蕪額田約十萬畝,無主者作官田招領,分田、地、場三等繳價,名曰系腳錢,有主限期報墾,逾限入官。從江督劉坤一請也。二十八年,陜撫升允言:「西安馬廠各荒地,試開水旱田,行屯墾。營哨官賦地畝自六十以下,屯勇人十畝。每百畝貸官牛兩頭,籽種三石,官備農器,一年還牛,二年全交。並擬令分年節餉。開屯之初,歲發全餉,二歲裁半,三歲盡裁。」嗣後地為水沖雹壞,穡入弗豐,因復上言:「驅無餉之兵,使自食其力,勢且壯志銷於畎畝,精銳蝕於農作,有屯而實無兵,有兵而實無用,轉非創屯本意,不如不裁其餉,而悉以屯利歸公,再頒歲穫之二三行賞,此所謂兩利者也。」

江西義寧、新昌之交,有黃岡山,自明以還,恆為盜藪。二十九年,從巡撫柯逢時請,開地以益民。直隸安州白洋澱淤地肥沃,是歲弛禁,招民佃作,分四等收預租。三十一年,海洲、贛榆間有雞心、燕尾二灘,利墾牧。又徐州微山湖淤灘地,均召民墾升科。三十二年,議定廣西墾荒丁壯既稀,資本又絀,乃仿外洋法,招商領墾。南寧則招商本立公司,募裁兵充墾丁。至宣統初,共放山荒十六萬六千五百餘畝。三十三年,江督端方上言蘇屬兵後荒田不下二百餘萬畝,請令歷年報荒者定為板荒,餘新荒,許各戶指報豁糧,俱由局招墾,則虛荒易查。又定墾章,區別官荒民荒,分三等輸價,受荒無問土客,皆得領種。三十四年,清丈安徽沿江洲地,計懷寧等州縣官荒應繳價者共三十萬餘畝。廣東瓊崖從未開殖,至是集商本創公司,官行清丈,分官荒民荒,先正其經界。宣統三年,雲南清出荒地五十六萬畝,安徽官民荒地四萬一千餘頃,河南沙荒地三萬三千餘頃。可墾者分三等,曰輕沙,曰平沙,曰重沙,各州縣試行招墾,多則四百數十頃,少亦二三十頃。浙江仁和等屬,墾熟甲地山隴百八十餘頃,各府紳商領墾荒地萬五千餘畝。甘肅自光緒季年設局墾荒,達二十餘萬畝。

新疆屯田,始康熙之季,察罕諾爾地駐兵,因於蘇勒厄圖、喀喇烏蘇諸處創屯種,令土默特兵千,每旗一臺吉,遣監視大臣一人。而哈密、巴里坤、都爾博勒及西吉木、布隆吉爾等,咸議立屯。命傅爾丹、蘇爾德、梁世勛分職其事。吐魯番亦駐屯兵。雍正三年,命喀爾喀駐兵墾鄂爾昆田。

乾隆初,定一兵墾二十五畝,凡兵二千五百,種地三之,駐守二之。時回部如闢展各要沖,多設屯,厚兵力。逮準噶爾平,版圖益廓,邊防與屯政相維。七年,川陜總督尹繼善請以蔡把什湖地租與回民,假貲耕種,事得允行。二十年,以伊犁西境喀爾喀東陲多閒壤,悉遣滿、漢、蒙兵數千開屯,視蒙古授田例。又設額爾齊斯屯田,巴里坤亦置屯,遣甘、涼、肅屯地兵五百往種,秋收後入城,三年更迭,塔勒納沁開田三千餘畝。

二十三年,用雅爾哈善、永貴等言,於闢展、魯克察克、吐魯番、烏魯木齊,托克遜、哈喇沙爾規度官墾。是時饋饟猶亟,誡巴里坤至伊犁循序增屯,其原挾家者,俾安業如內地村莊。初人種十五畝,令益五畝。置新舊屯兵萬七千,出帑三百萬備籽種諸用。而特納格、昌吉、羅克倫均益兵廣屯。大率烏魯木齊增墾以來,歲穫悉供伊犁餉需。伊犁墾成,又資接續,更移喀什噶爾等回眾二千五百戶屯阿克蘇。其事則黃廷桂、楊應琚、兆惠等主之。定章百兵一屯,地畝人二十,分小麥十一、穀七、青稞豌豆各一。然吐魯番、闢展、魯克察克兵屯外皆兼回屯,而庫車東、哈喇沙爾西,或分布多倫回人溉種。

二十五年,伊犁屯議起,於河南之海努克立回屯,察罕烏蘇立兵屯。翌年,又於葉爾羌、喀什噶爾、阿克蘇、烏什等城增回屯,減兵額。時戎事方息,惟厄塞留兵,餘齊赴伊犁屯殖,穫粟贏裕,即益屯兵。兵不供屯,則招集流人,分土任業。巴里坤饒賸壤,穆壘土沃泉滋,俱募人大開阡陌。蓋舒赫德、阿桂、明瑞等所建為多。三十七年,陜督文綬以新疆餘地宜推廣募墾,條列五事以聞。

四十一年,令葉爾羌成丁餘回,特畀耕地編戶,凡千五百戶為一所,三千戶為一衛。初,烏魯木齊屯地,共綠旗兵三千,二千操練,一千屯耕,番休,三歲後令移眷,官予資裝。及地日廓而兵不贍,率遷甘肅貧民,不靳煩費,赤貧全給,小康半之,歲穰自原挈家則不給。四十五年,定眷兵分編戶籍,其牛籽、農具、屋價、口糧,皆官措貸,約升科時,分三年繳納。凡承種新疆熟地,本年升科,新墾三年後升科,而商民承墾新地,戶三十畝,六年升科。蓋自此楚呼楚、穆壘、瑪納斯、庫爾哈喇烏蘇,屯務駸駸日近矣。

新疆軍屯分數,人穫細糧十五石至十八石,官議敘,兵丁賞一月鹽菜銀,二十五石倍之,十二石以上,功過半,不及,官議處。兵重責留屯,次年收足予復。烏魯木齊但穫糧十一石以上即敘賞。塔勒納沁尤磽瘠,賞罰遞降殺之。無鹽菜則給口糧,其阿奇木伯克等則賞緞匹。顧伊犁額多苦累,福康安嘗以為言。最後將軍長庚請仿烏魯木齊例行,然部議仍未及減也。向例遣犯得留種新地,哈密各屬截留伊、烏遣犯墾耕,年滿乃各致其所,罪重勿留。又以不敷農作,僅限斷洋盜而已。後令情輕者改防為眷,用羈縻之。遣犯穫額兵丁,其敘賞諸事從原例。

嘉慶十三年,撥塔爾巴哈臺兵赴伊犁殖田,以農隙簡練,置武員領之,三年一更迭。而伊犁原定屯兵三千,每歲耕種,於中抽調如乾,藉習戎備,其數歲有增減,各視其時,已耕之十八屯,番休輪種,以息地力。尋定自二十年始,每年加種兩屯雲。初,伊犁多可耕田,令惠遠、惠寧兩滿城派閒散旗人分地試種,借給牛具,成效昭然。九年,松筠因言照錫伯營屯種例,分畀旗兵地畝,各使自耕,永為世產。以有妨操務,祗令轉交閒散代耕。二十五年,令滿營兼種雜糧,先後分田四萬四千餘畝,授八旗閒散自耕,但不得違禁佃租,私相典賣。

道光初,既勘定張格爾,令回兵試墾大河拐,增額則募貧回。於是烏什、阿克蘇、和闐每散布回戶行墾,烏魯木齊屬阜康、奇臺暨吐魯番,均募民戶,伊犁惠遠城迤東,亦選土著,阿卜勒斯荒,俱撥回戶,設五莊,莊百戶,戶得地畝二百,喀喇沙爾則裁屯安戶,庫車荒地,亦予無業回人,葉爾羌屬巴爾楚喀多曠土,則廣招眷民。其霍爾罕新田,散與回戶,喀什噶爾初開地,分處河東西,東畀回人,西招民戶,或專屬,或兼募,冀相安而已。凡民人赴回疆領地,皆官給印券,自齎以行,其徵糧多至畝二斗四升,次小麥八升,次六升五合,最少三升,大率視壤肥瘠為斷。阿卜勒斯入三色糧十六石,滿營馬兵練餉於茲取贍。自嘉、道以來,數十年中,伊犁屯墾,後先其事者,將軍松筠、那彥成、布彥泰等,而林則徐遣戍日,履勘諸地,又興水利於伊拉里克,厥績尤偉焉。

同治二年,都統平瑞上言,烏魯木齊閒曠孳生馬廠,招商戶移墾,並請於伊犁各城,一律經畫分屯地畀屯兵。命次第興舉。三年,飭哈密推廣原屯。

光緒三年,侍讀張佩綸請抽旗丁屯新疆。陜甘總督左宗棠謂有所窒礙疑阻凡六事,議遂寢。是時南路纏、民富庶,荒曠尚稀,北路鎮、迪各屬,墾熟地不過十二三,賦納既虧,閭里窳敝。已而建置新疆省治。十三年,巡撫劉錦棠更酌定新章,戶給地六十畝,官借籽糧二石,農具銀六兩,葺屋銀八兩,牛兩頭,二人即當一戶,月給鹽菜口糧,立限初年還半,次年全繳,繳訖,按畝起徵,第三年半徵,次年足全額。仍仿營田制,十戶一屯長,五十戶一屯正,每屯正五,設一管領專員,正、長領地貸本,悉如戶民,總計安納土、客千九十戶,以次推行。而南路各屬新墾地萬九千餘畝,分年起徵,均不領墾費。丈清南北兩路各則荒熟地千一百四十八萬畝有奇。各城伯克向有養廉地,自改郡縣,裁伯克廉地一律入官佃租。十七年,魏光燾分劃伊犁各地歸旗屯、民屯各六萬餘畝,使各自力耕。其後土、客生息蕃庶,歲屢有秋,關內漢回挾眷承墾,絡繹相屬。

宣統三年,巡撫袁大化言:「新疆夙號農牧國,今日貧瘠,由地曠人疏。自迪化以西,精河以東,遍地官荒,草湖葦灘,無慮千萬頃,而南疆東路蕭曠亦同。擬集華僑立公司,速效非易。今令在新各員,有獨力或合貲開荒灼著明效者,分別奏獎,以示鼓勵。」事得允行。

金川在乾隆四十年以武功底定,初從定西將軍阿桂言,於西川之攢拉就近屯田,其美諾、底木達等處,令駐兵受地習耕,別斯璊以次改土為屯,各置屯弁處理。又帛噶爾、角堯諸降番,悉視屯兵例,概畀以牛具籽糧。其番戶多者三四十,少者一二十,初墾免賦,三年後輸糧,旋令駐兵挈眷前赴,而丁口日增,又撥地戶三十畝,俾加墾自給,地利浸闢矣。於是四川之懋功五屯,安置降番,亦戶給地畝三十,選精壯千人,半為屯練給餉,半為餘丁無餉。厥後釐出荒壤,亦分等加賚,巴塘、里塘沃區亦不乏。至光緒三十三年,川督趙爾豐疏籌墾計,招內地農戶而官資遣之焉。

關外土曠人稀,蒙古地尤廣袤,利於屯墾。清初分旗有定界,繼因邊內壤瘠糧虧,拓邊移墾。天聰中,令各牛錄就各屯近地,擇種所宜。以沈佩瑞言,於廣寧東西、閭陽驛,選壯農充步卒屯田,分八固山,釐牛錄為二等,備牛種農具,令材敏者率屯兵往耕。崇德五年,官兵於義州築城開屯。康熙二十五年,以錦州、鳳凰城等八處荒地分給旗民營墾,又遣徒人屯種盛京閒壤。二十八年,定奉天等處旗、民各守田界,不得互相侵越。乾隆五年,侍郎梁詩正請置八旗閒散屯邊,以廣生計,命阿里袞往奉天相度地宜。於時吉林寧古塔、伯都訥、阿勒楚喀、三姓、琿春及長春,俱事墾殖,貧無力者,發官帑相貸。四十年,流人偷墾岫巖牧場地畝,遂定例使入官納租。四十二年,以大凌河西北杏山、松山地豐美,徙閒散宗室,資地三頃,半官墾,半自墾,築屋編屯,助其籽具。五十五年,令奉天自英額至靉陽邊止,丈荒分畀城旗之無田者,除留圍場葠山,餘均量肥瘠配給,禁流民出口私墾,而積久仍予編戶。嘉慶十六年,令各關隘詰禁之。

初以八旗口眾,撥拉林地俾開田墾種。十七年,賽沖阿言「拉林近地閒荒可墾者二萬五千餘晌,而三道卡、薩里諸處地多未墾,請移駐旗人」。尋富俊請揀屯丁千人,撥荒三十晌,給銀二十五兩,籽糧二石,墾二十晌,留十晌,試種三年後,第四年起交糧。俟移駐京旗分給以熟十五晌、荒五晌,餘荒熟各五晌,即與原種屯丁為業而免兵糧。已,富俊建議更於拉林之西北雙城堡開屯,移駐京旗閒散,為地九萬數千晌,移戶三千,年移二百戶,依戶劃地,一切費悉領於官,區中、左、右三屯,屯鑿井二,選丁給地,例同拉林,京旗領地五年後,徵糧二十石,每大屯容四十屯,每旗五屯,置總、副屯達各八人,每屯屯丁京旗各三十戶,二三人以上即準戶論,三屯各建義塾課幼丁。

道光五年,移駐戶七十七,墾熟地三萬三千一百餘晌,蓋富俊、松筠始終其事,故其效甚著。自後當事浸懈,又其地早霜氣寒,原徙者少,於是博啟圖改移駐戶為千,因以所餘地,戶益十五晌,閒散不任耕,得買僕或賃傭以助。英和嘗上言宜推廣成功,而緒卒弗竟。伯都訥空曠圍場二十餘萬晌,荒久壤腴,視雙城堡事半功倍。富俊請令分屯畫界,略仿前規,命其地曰新成,綴列戶號,前後凡百二十屯。章凡六七上,廷議旋以雙城堡事未遑他及,且用弗充,事竟已。二十八年,令鳳凰城邊私墾地,已熟及中墾者,招佃徵租。無幾,旗、民報墾至二十四萬畝。

咸豐四年,開吉林五常堡荒田。先是齊齊哈爾設官屯,令罪徒及旗奴承種。尋以游惰遣退,選壯丁補之。嗣御史吳焯謂呼蘭蒙古爾山荒宜墾,尋以葠珠禁域,兼妨邊務,竟不行。

同治時,廣寧南之盤蛇驛,擬放地百萬畝,民領及半。厥後水患頻仍,迄光緒末,開放始竣。是時金場流民失業,用富明阿言,以藏沙諸河暨樺皮甸子諸處官荒畀墾,免交押價,而法庫門、靉江往往有游民偷墾。迨都興阿履查,靉江西岸密邇朝鮮,安置匪易,惟嚴禁越渡,以謹其防。有沿江陰墾騷擾沿邊者,立予拘罰。九年,乃就靉陽門至鳳凰門邊荒九十一所,分勘展界,綏奠窮黎,而私墾充塞邊境如故。

光緒七年,吳大澂上言:「寧古塔之三岔口壤沃宜耕,可募齊、魯願農,編屯一營,以實邊塞。」十四年,將軍希元始設局立制,以邊瘠收薄,限十年後升科。尋設五社,墾地萬三千四百晌有奇。二十二年,延茂覆陳吉林開墾,始誤於旗、民之不和,繼誤於委員之自利,開局十六年,得不償失。部議因定分別裁留。於是方正泡、藲梨場、二道漂河、頭二道江、螞蜒河、大沙吉洞等河,亟亟以拓地殖民為務。初,吉林放有攬頭包領,雖荒甿綿袤,輒刻期集事,而弊溢於利,至是始懲革焉。又腹地加荒附著各屯,多寡不等,皆甚饒沃,領者麕至,則探籌決之。先是十二年,黑龍江將軍恭鏜請開呼蘭屬通肯荒地,疏陳十利。已而決議實行。至二十四年,營通肯克音荒務,畫屯安井,招民代佃,民納課糧,旗供正賦,官為之契,不奪佃益租。二十五年,墾布特哈之納謨爾河閒荒約四十萬晌,旗民領佃,入費免租,從恩澤請也。越八年,訥河以南放墾三十七萬五千一百餘晌。

二十八年,吉林設局清賦,兼放零荒,各屬旗戶原無糧額,各地查報科徵。顧其時經界既淆,包套詭寄,棼如亂絲。旋日、俄變生,事益棘手。將軍達桂、巡撫陳昭常先後清覈,至宣統初元,都吉林大租原地為晌百一十八萬三千一百有奇,浮多二十八萬四千八百餘晌。其明年,通吉省民田、旗地及夾段零荒勘放訖事,又清出七十九萬三千三百餘晌。浮多地者,如地形方及東西長,均以西為浮多,南北長則以北,西北有廬墓則以東南。或一地兼二則,次則即浮多也。

奉天大圍場分東西流二圍,自國初撥留是荒,有鮮圍十五以捕鮮,大圍九十以講武。日久防弛,流人私墾歷年。光緒初,將軍岐元奏以二十圍增海龍治,就地升科。至三十年,海龍兩翼升科者,已達百二十九萬八百餘畝。餘八十五圍。西四十五圍,於二十二年議墾,至三十年放訖,其正零山荒樹川草甸三百二萬二千餘畝。其荒價畝納銀一兩二錢,山場熟地六錢,生三錢,城鎮基地畝二十兩。其久年私墾土地則倍納二兩四錢,中下差減,原戶領回,不原則撤放。東四十圍,以安置金州遷戶,開禁撥荒,迄三十一年,共放百十二萬七千二百餘畝。城地上者畝二十五兩,中二十兩,下十五兩。荒地畝收正課二分,耗十分。其始兩流圍荒地聽民擇,所餘夾荒,往往侵墾,吏緣為奸。自廷傑重勘,一清積弊。東流圍即東平全境,隱並殆過西流,訟鬩滋繁。三十二年,覆丈兩流山荒,俱十畝作七畝。至浮多地已先納價,未及折合,則限八年升科,以平劑之。大率熟地當年起科,荒地四年為限。時日、俄構兵,奉省稅滯帑虛,復查東邊海龍各屬私墾餘荒,收價集資,藉維新政。又丈放錦州屬海退河淤及各滋生地畝共三十二處云。

黑龍江地,當光緒十八年,於綏化之北團林子設屯田旗戶千二百,巴蘭蘇蘇之山林設戶六百有餘,計戶授田,戶四十五晌,中以十五晌歸屯丁永業,三十晌起科。拳匪亂作,流徙頻年,續於鐵山包招戶,又招撫璦琿各屯,久乃稍還其舊。然是時江省以東,民戶日蕃,污萊攘剔,十才二三,富豪包攬居奇,零戶無力分領,放荒速而收價遲,領地多而開地少。三十三年,乃議變通,令閒退兵原農者,分年給墾,寓殖於屯。宣統元年,又令廣招徠,定獎章,杜包承,賞經費。戶仍領地十五晌,晌收公費四錢,大都荒價量地為等差。木蘭、綏化晌收銀七錢,通肯二兩一錢,呼蘭、墨爾根押租則一兩四錢,贏朒不齊,均加徵一五經費,其大較也。時又酌留嫩江迤西未放各荒為無地官兵生業。撥兵助屯之策,始自哈拉火燒試行,而地鮮上腴,兵惰不耐耕,畝僅穫斗糧,甚且無顆粒收入,口食仍仰給於官,因復議緩。二年,仍改招民佃。

初,奉省厲行清賦,凡浮多地限令民戶首實,納價起科,歷三歲餘,僅得荒熟地八十餘萬畝。已而議局建,用分年免價法。東督錫良上言:「清賦重升科不重收價,其利久暫懸殊。又東省為八旗根本,旗、民雜居皆土著,異於各省駐防,內外城旗隨缺伍田,向有定額,即計口授田遺意。數百年來,戶口增而地不給,口分體大難舉,墾種事便易行。今長白新設治,移殖最宜,如以實邊之策,資厚生之利,所謂兩益者也。夫必先去其待食於人之習,然後漸為人自為養之謀,給田則奮於力農,徙地則除其依賴,為八旗計,無要於此。」三年,奉天各屬大放民荒,共得十二萬畝。

自順治時,令各邊口內曠地聽兵治田,不得往墾口外牧地。顧其地豐博宜農,雍正初,遣京兵八百赴熱河之哈喇河屯三處創墾,設總管各官。旋置張家口同知,十分其地,歲人耕逾分予敘,不及五分處罰。洎乾隆初,熱河東西共畫旗地約二萬頃。古北口至圍場舊無民地,歷年民墾滋紛,乃令分撥旗戶。未幾,高斌請還其舊,從之。熱河自改州縣後,山場平原,講求開殖,悉向蒙古輸租,沿襲已久。其圍場周千餘里,為圍七十二,置總管一,駐防旗兵千。

同治中,用都統瑞麟言,展墾閒荒,以濟兵食,令招富戶承領,禁占毗連民地,於紅椿外定界立卡倫。尋翼長貴山等以阻撓得罪。時全圍已放其半,領荒者漸侵正圍,於是諭河東西佃墾及偷墾地一律封禁,斥遣私墾諸戶。其侵入山坡溝岔,乃報領匿多為少者,重按之。其後庫克吉泰部署茲事,將旗佃圍外隱地,撥補圍內民佃,俾得移徙安業,以清圍界。然委員措置失宜,奸佃抗聚生釁,經崇實再舉勘量,更定照冊永禁已騰之正圍,瑞麟繼之,仍無要領。

光緒初,御史鄧慶麟臚列積弊,已而定議舉辦京旗徙戶開屯,其後確勘熱河五川荒地頃數,都二千三百有奇,平川地僅及其半,旋即招墾,以押荒抵餉。季年,都統錫良論開放圍荒十事,大要留圍座,編號目,增荒價,杜攬售,事皆允行。

蒙古當康熙時,喀喇沁等旗地,以民種而利其息入,輒廉募之,致妨游牧。乾隆初,亦令察哈爾蒙、民易居,但雜處積年,戶眾墾蕃,難歸徙而輕生釁,議者數稱驅斥之便。至嘉慶初,土謝圖汗各旗地,常有游民棲息。蒙人負民債不能償,而貧民復苦無歸,則為之明界設限,不咎前失,儻將來私開一壟,增遷一人,坐所管盟長等罪,其租課官不之問,各扎薩克自徵之。時郭爾羅斯熟地畝二十六萬五千餘,糧畝四升為定率。至十一年,墾者踵相屬,因伸關禁,並諭禁私與民授受,違者臺吉連坐之。然流人私種成習,莫能格也。初令歸化種地人按編甲例,歲上其籍,而口外綏遠等地,僅容孑身商販往來,挈室者有禁。其後科爾沁屬達爾汗、賓圖二王旗,卓哩克圖、冰圖二旗所招墾戶,亦均編甲社,置鄉長焉。

道光十二年,盛京將軍裕泰上科爾沁墾章八事:凡寫地必以自名,毋過五頃;一地衣復寫者,後戶與前戶相均;村屯或典於民,追契折償;地主無力回贖,任民再種,限年抵還;年滿第允自種,或租與原佃,不得復典及招人;民戶交地後,得自踏閒荒,白局承種;其蒙種熟地,毋許租人;界外民開者亦毋許影射。咸如擬行。土默特牧場,舊惟任意墾治,嗣分餘地畀蒙人,口率一頃,而佃與民種者多。至十七年,令入蒙押租,以其四佐官用,其租息無業蒙人四之,公家及本旗貝勒各三之。同治七年,徙喀喇沁越墾諸戶分歸各旗。

光緒七年,創烏里雅蘇臺墾田十頃六十畝為一屯,凡為屯七,濬渠、建居、牛、籽諸費,亦官為補助。八年,選庫倫土著於圖什、車臣西部落學試屯墾,從喜昌請也。當蒙古生息浸盛時,於地之不妨牧者墾之,曰牧地,又有租地、養贍地、香火地,皆自種自租。九年,山西巡撫張之洞言「豐、寧二、歸綏五,自招墾蒙荒而戶日蕃,所在餘荒,時亦畀無業佃民租種,其租所入,除例與蒙旗外,凡開地基本薪公歲耗彌補一切,皆取給其間,為益匪細」。

二十一年,奉天將軍增祺請丈放各蒙荒,副都統壽山亦以為言,而國子司業黃思永請墾內蒙伊克昭、烏蘭察布二盟牧地,盟長有謂妨其生業者,未克實施。是時晉邊之豐鎮、寧遠墾民積數萬戶,而扎賚特、杜爾伯特、郭爾羅斯陸續報墾,人爭趣之。察哈爾旗牧及草地雖禁私開,然自咸豐中馬廠弛禁,至近歲越占紛紜,客戶旗丁,訟不勝詰。二十四年,都統祥麟因言「欲蒙地無私墾,必嚴科罪,欲蒙員無私放,必懲奸商」。

二十八年,命侍郎貽穀督墾務,籌察哈爾事,陳擴充變通數端,大旨主「清舊墾,招新墾。蒙旗生計在耕不在牧。蒙古於地租,或抵償,或私肥,或一地數主,抑且數租,黠商乘間包攬。宜由各旗總管詳晰呈明,交地開放,悉汰從前地戶商總等名,設墾務公司於兩翼,各旗先後試辦,各盟旗順令即獎,抗延即罰」。於是伊克昭盟郡王等旗,及準噶爾,以次報地。杭錦、烏審頗反覆,烏蘭察布亦懷疑,已皆赴議。綏遠已墾未墾地畝,在乾隆初即無確數,迄今八旗牧廠,地雜沙石,中墾者希,民情觀望。乃建議自將軍以下俱指認地畝,為商民導。旋以財用不足,創牛捐,並推廣屯捐繼之。凡丈蒙地,五尺為弓,二百四十弓為畝,百畝為頃,頃編為號。察哈爾兩翼,則畝以三百六十弓,編號以五頃。札薩克圖畝則二百八十八弓,十畝為晌,四十五晌為方。凡蒙旗荒價,半歸國家,半歸蒙旗。其歸蒙者,自王、公、臺吉至於壯丁、喇嘛,釐其等差,各有當得之數。凡地額設者為排地,向免押租。生地畝收押租三錢三分,滋生地倍之。貽穀以恤蒙艱,故畝收押荒二錢外,僅加一錢,局用取其六,本旗取其四。杭錦在後套近渠水地,押荒上地畝八錢,中七錢,下六錢。又言租數多則累民,少則累蒙,此旗與彼旗難強同,外蒙與內蒙不一例,因定烏審、札薩克、郡王三旗荒價,上則三錢,中二錢,下一錢。鄂托克、準噶爾兩旗地區四等,別立中下一則,鄂旗上則四錢,準旗上則六錢,中四錢,以下均差減。烏蘭盟四子王、達爾罕、茂明安及烏拉特後旗皆旱地,悉如向章。

三十四年,文哲琿訐貽穀敗壞邊局,查辦大臣鹿傳霖論其辦墾有二誤四罪,因策善後四事,謂「荒價及繩丈從寬,則丈放易,欲多收地價,則應先侭原佃承耕,減歲租而加渠租,以其租充渠費,渠增即地增,地增即租增,久之斥鹵皆腴壤矣」。貽穀既逮系,信勤繼之。減杭錦荒價,上地頃九十兩,其次遞減以五,最下七十兩。分烏拉特地為東、西、中三公。旱地押荒分六等,上地頃百四十兩,次百,中七十,中次四十,下二十,下下十兩。先提公費三成,其餘半蒙半公,胥如例。其歸蒙地租亦四等,渠地畝歲徵渠租四分五釐。

科布多及烏蘭古木試行屯墾,肇自康熙末年。時參贊連魁陳辦科屬新政,謂「烏蘭古木、巴雅特均科屬杜爾伯特牧地,宜廣營墾。科布多屬雖積沙漠,而札哈沁旗、明阿特左右翼各旗及厄魯特旗,各臨其所屬河泊,沿河田陌可耕者多,興墾實邊,於是乎在」。廷議允行。若烏梁海屬布倫托海蒙地,自同治時開屯,頒帑金十萬。嗣李雲霖以操切激兵變,墾事中停。至是修渠告成,以上渠屯兵並合下渠,從其便也。阿爾泰旗高寒稀雨澤,僅成官屯四、民屯一云。札薩克圖王公旗荒,每晌上等四兩四錢,中二兩四,下一兩四,均收一五經費。凡依次領地,熟地百晌,須兼生荒二百晌。王旗至十一年放竣,都六十二萬五千餘晌。其明年,續放旗界山餘各荒,設洮南屬縣二。公旗自招之戶曰紅戶,臺吉壯丁等私招者曰黑戶。洮南沿荒段放齊後,河北荒段,至宣統元年,共丈十九萬四千餘晌。圖什業圖蒙荒,亦仿札薩克圖成案。

熱河蒙荒,喀喇沁東旗已成良沃,敖漢半磽確,巴林較富。都統廷傑建言八事,以漸興舉。其蒙旗荒之隸奉屬者,約放八萬九千餘晌,而昭烏達盟阿魯科爾沁、東西扎魯特三旗可耕地,共八千頃,上則頃收價七十兩,中五十,下三十。扎賚特蒙旗新舊放荒綜六七萬晌,置大賚,捆出本旗蒙屯四十七所,外旗五十九所,近地餘荒,晌收押租一兩四錢。時復丈科爾沁公旗地二十四萬一千四百餘晌,郭爾羅斯後旗沿江地荒而實腴,晌加收公費三十兩,蒙地及學務各半之。及是開放無餘。翌年,城甸餘荒亦畢放。長春本前旗蒙地,凡四十一萬九千餘晌。宣統二年,復放新荒,以公費資辦府屬審判,拓荒務以裨新政。更定巴林荒價,上則頃七十兩,中五十,下三十。達爾汗王旗採哈新甸荒地分三則,上則晌六兩,中四兩,下二兩。二共放實荒六萬二百餘晌。三年,復放達爾罕洮、遼站荒,備置驛通道焉。

青海向為蒙、番牧藪,久禁漢、回墾田,而壤沃宜耕者不少。曩年羹堯定議開屯,發北五省徒人能種地往布隆吉爾興墾。最後慶恕主其事,以番族雜居,與純全蒙地殊異,極陳可慮者五端。嗣又勸導蒙、番各族交地,以資拓殖,無論遠近漢民皆得領,惟杜絕回族,以遏亂萌。於是開局放荒,黃河以南出荒萬餘畝,迤北至五萬餘畝。又慮其反覆也,募實兵額,分留以鎮讋之。番地僻,山峻且寒,僅燕麥菜籽,雖歲穰,畝收不過升四五,課務取輕,以次推行。近地始自光、宣之際,議墾荒尤亟,以物力之不易,而大舉之無時,冀其地無棄利,人靡餘力,蓋猶有待焉。

清丈蘆洲田畝,前允行之九江濱江蘆地,原定下則起科,是後蘆洲徵糧,普令以一分以下為率。奉天廣寧一帶蕩田墾殖舊矣,嗣以將軍弘晌言,開鷂、鷹二河蕩田三十八萬二千餘頃,令三年後升科,五年後丈量。而牛莊等處葦塘,近年河徙荒出,葦商大半匿墾,往往召爭,先後訂變通章程,迥別於故荒舊例。尋又丈放鳳凰、岫巖、安東葦塘約十餘萬畝,按地編號,具魚鱗圖冊,事在光緒末年。江南葦營草地,向由大河衛子領墾納租,而江北則置樵兵備河務,左右兩營,當海州、阜寧間,共地八千五百餘頃,而續涸新漲不與焉。自河道改而樵兵虛設。宣統時部議裁汰,改為放荒,任人入貲承業雲。

自光緒中葉,御史曾忠彥疏請振興農學,特立農工商部,專其職司。數詔天下長吏,講求釐剔荒產,以為振興之資。宣統初,部上農林推廣二十二事,始於籌款辦荒,而坦區宜闢田,山隴畸零邊地宜林木,責所司各於其境測驗氣候土性,表之圖之,荒價之免否,升科之緩急,分等釐別,而以考覈官吏編報成績,以行其懲勸。復訂種樹行水獎掖專例。洎乎革命勢成,事之未畢舉者,正復不少也。

曰營田水利。聖祖時,墾天津荒地萬畝為水田。世宗於灤、薊創營田,設營田水利府,命怡親王董其事。王與大學士硃軾匯上事例四端。尋於天津等屬分立營田四局,領以專官。因地勢濬流築圩,建閘開渠,民人原耕者,官給工本,募江、浙老農,予月餼,教耕穫,翌年,得熟田百五十餘頃。至雍正七年,營成水田六千頃餘,雖糜帑不貲,而行之有驗,惜功未竟,後漸廢弛。獨磁州溝洫如故,歲常豐稔。

高宗飭直督李衛修治水田,復遣大理卿汪漋總江南水利工務,南北並營。已而高斌言桑乾河兩岸可開大渠,引水治稻田,從之。嘉慶之季,命方受疇經畫直省水利,兼戒魯、晉、豫亦於其境各籌所施。顧猶有言直隸難舉水田者。百年以來,李光地、陸隴其、硃軾等皆詳言直隸水田利益,林則徐擬開近畿水田疏尤切至。財絀議沮,迄未暢行。自後僧格林沁在大沽口屬捐興水利,得稻田四千二百餘畝,崇厚繼之,頻年勸墾鹽水沽亦頗效。其後周盛傳鎮天津,修水利,成稻田六萬餘頃,土潤穫饒,至今利之。

同治時,陜西西安、同州等屬設局釐荒產,興營田。洎光緒中,次第招墾至三萬四千餘畝,改局為所,州縣領理之。時直屬營田半荒棄,三晉洊災,臺臣夏獻馨、唐樹楠、彭世昌、劉瑞祺等先後疏言水利,華煇亦陳八事。直督王文韶謂「輕租價以恤民艱,疏溝渠以利水道,則樂墾者多」,因是天津營田徵租至四萬九百餘畝。山東巡撫張汝梅亦請疏河道,濬溝渠,以興水利為農政本源;陜甘總督升允則請於陜西募水利新軍左右兩旗,將來撥歸屯所,授地使耕,藉廣屯政。其後奉天以東西遼河、大凌河諸川無涓滴水利,亦奏定採內地引渠灌地諸法,先就小河枝水鑿渠試辦焉。


\end{pinyinscope}