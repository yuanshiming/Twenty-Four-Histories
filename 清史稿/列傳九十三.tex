\article{列傳九十三}

\begin{pinyinscope}
曹一士李慎修李元直陳法胡定仲永檀

柴潮生儲麟趾

曹一士,字諤廷,江蘇上海人。雍正七年進士,改庶吉士,散館授編修。十三年,考選云南道監察御史。高宗即位,諭群臣更番入對。一士疏言:「敬讀諭旨,曰『百姓安則朕躬安』,大哉王言,聞者皆感涕。臣愚以為欲百姓之安,其要莫先於慎擇督撫。督撫者守令之倡。顧其中皆有賢者、有能者,賢能兼者上也,賢而不足於能者次之,能有餘而賢不足者又其次也。督撫之為賢為能,視其所舉而了如。今督撫舉守令,約有數端:曰年力富強,曰治事勤慎,曰不避嫌怨。徵其實跡,則錢糧無欠,開墾多方,善捕盜賊。果如所言,洵所謂能吏也。乃未幾而或以贓汙著,或以殘刻聞,舉所謂貪吏、酷吏者,無一不出於能吏之中,彼誠有才以濟其惡耳。夫吏之賢者,悃愊無華,惻怛愛人,事上不為詭隨,吏民同聲謂之不煩。度今世亦不少其人,而督撫薦剡曾未及此,毋亦輕視賢而重視能之故耶?抑以能吏即賢吏耶?臣恐所謂能者非真能也,以趨走便利而謂之能,則老成者為遲鈍矣;以應對捷給而謂之能,則木訥者為迂疏矣;以逞才喜事而謂之能,則鎮靜者為怠緩矣;以武健嚴酷、不恤人言而謂之能,則勞於撫字、拙於鍛鍊者謂之沽名釣譽、才力不及,而摭拾細故以罷黜之矣。至於所取者潰敗決裂,則曰臣不合誤舉,聽部議而已。夫有誤舉必有誤劾,誤舉如此,則誤劾者何如?誤舉者猶可議其罪,誤劾者將何從問乎?臣以為今之督撫,明作有功之意多,而惇大成裕之道少;損下益上之事多,而損上益下之義少;此治體所關也。皇上於凡丈量開墾、割裂州縣、改調牧令,一切紛更煩擾,皆行罷革。為督撫者,度無不承流宣化,所慮者,彼或執其成心,飾非自護;意為迎合,姑息偷安。臣敢請皇上特頒諭旨,剖析開導,俾於精明嚴肅之中,布優游寬大之政。所屬守令,敕於保題薦舉時,分列賢員、能員,然後條疏實事於下。能員有敗行,許自行檢舉;賢員著劣跡,則從重處分。倘所舉皆能而無賢,則非大吏乏正己率屬之方,即賢者有壅於上聞之患。督撫之賢否,視所舉而了如矣。」疏入,上為通諭諸督撫。

一士又請寬比附妖言之獄,並禁挾仇誣告,疏言:「古者太史採詩以觀民風,藉以知列邦政治之得失、風俗之美惡,即虞書在治忽以出納五言之意,使下情之上達也。降及周季,子產猶不禁鄉校之議。惟是行僻而堅,言偽而辨,雖屬聞人,聖人亦必有兩觀之誅,誠恐其惑眾也。往者造作語言,顯有悖逆之跡,如罪人戴名世、汪景祺等,聖祖、世宗因其自蹈大逆而誅之,非得已也。若夫賦詩作文,語涉疑似,如陳鵬年任蘇州知府,游虎丘作詩,有密奏其大逆不道者,聖祖明示九卿,以為『古來誣陷善類,大率如此』。如神之哲,洞察隱微,可為萬世法。比年以來,小人不識兩朝所以誅殛大憝之故,往往挾睚眥之怨,借影響之詞,攻訐詩文,指摘字句。有司見事風生,多方窮鞫,或致波累師生,株連親故,破家亡命,甚可憫也。臣愚以為井田封建,不過迂儒之常談,不可以為生今反古;述懷詠史,不過詞人之習態,不可以為援古刺今。即有序跋偶遺紀年,亦或草茅一時失檢,非必果懷悖逆,敢於明布篇章。使以此類悉皆比附妖言,罪當不赦,將使天下告訐不休,士子以文為戒,殊非國家義以正法、仁以包蒙之意。伏讀皇上諭旨,凡奏疏中從前避忌,一概掃除。仰見聖明廓然大度,即古敷奏採風之盛。臣竊謂大廷之章奏尚捐忌諱,則在野之筆札焉用吹求?請敕下直省大吏,察從前有無此等獄案、現在不準援赦者,條列上請,以俟明旨欽定。嗣後凡有舉首文字者,茍無的確蹤跡,以所告之罪依律反坐,以為挾仇誣告者戒。庶文字之累可蠲,告訐之風可息矣。」上亦如其議。

雍正間督各省開墾,督撫以是為州縣課最,頗用以厲民。一士疏言:「開墾者所以慎重曠土,勸相農夫,本非為國家益賦起見也。臣聞各省開墾,奉行未善,其流弊有二:一曰以熟作荒。州縣承上司意旨,並未勘實荒地若干,預報畝數,邀急公之名。逮明知荒地不足,即責之現在熟田,以符報額。小民畏官,俯首而從之,咸曰:此即新墾之荒地而已。一曰以荒作熟。荒地在河壖者,地低水溢,即成沮洳;在山麓者,上土下石,堅不可掘;州縣悉入報墾之數。民貧乏食,止貪官給牛種草舍,餬旦夕之口,不顧地之不可墾也。十年之後,民不得不報熟,官不得不升科。幸而薄收,完官不足。稍遇歲歉,卒歲無資,逃亡失業之患從此起矣。然且賦額一定,州縣不敢懸欠,督撫不敢開除,飛灑均攤諸弊,又將以熟田當之。是名為開墾,有墾之名無墾之實也。茲二弊者,緣有司但求地利,罔惜貽害;大吏惟知慮始,不暇圖終;是以仁民之政,反啟累民之階。臣請敕下直省督撫,凡開墾地畝,無論已未升科,俱令州縣官覆勘,內有熟田混報開墾,舉首除額,免其處分;如實為新墾,具印結存案,少有虛偽,發覺從重治罪:則以熟作荒之弊可免矣。新墾應升科,督撫遴員覆勘,磽確瘠薄,即與免賦;倘因報墾在先,必令起賦,以貽民累,發覺從重治罪;則以荒作熟之弊亦可免矣。」

乾隆元年,遷工科給事中。故事,御史遷給事中,較資俸深淺。一士入臺僅六月,出上特擢。尋疏劾原任河東河道總督王士俊,疏未下,語聞於外。上疑一士自洩之,召對詰責,下吏議,當左遷,仍命寬之。一士復疏請復六科舊職,專司封駁,巡視城倉、漕鹽等差,皆不當與。又疏論各省工程報銷諸弊,請敕凡有營造開濬,以所須物料工匠遵例估定,榜示工作地方。又疏論州縣官讞獄,胥吏上下其手,竄改獄詞,請飭申禁。又疏論鹽政諸弊,請毋令商人公捐,禁司鹽官吏與商人交結;小民肩挑背負,戒毋苛捕;大商以便鹽船阻通行水道,戒毋堵截。皆下部議行。一士病哽噎,即以是年卒。

一士晚達,在言官未一歲,而所建白皆有益於民生世道,朝野傳誦。聞其卒,皆重惜之。

李慎修,字思永,山東章丘人。康熙五十一年進士,授內閣中書。遷主事,出為浙江杭州知府。雍正五年,入為刑部郎中,歷十餘年,治獄多所平反。有侵帑獄,初議以挪移從末減,慎修執不可;或諷以上意,亦不為動。乾隆初,出為河南南汝光道,移湖北武漢黃德道,以憂去。服除,授江南驛鹽道。引見,高宗曰:「李慎修老成直爽,宜言官。」特除江西道監察御史。疏論戶部變亂錢法,苛急煩碎。歷舉前代利害,並言錢值將騰貴,窮極其弊。上元夜,賜諸王大臣觀煙火,慎修上疏諫,以為玩物喪志。上喜為詩,嘗召對,問能詩否,因進言:「皇上一日萬幾,恐以文翰妨政治,祈不以此勞聖慮。」上韙之,載其言於詩。嘗謂慎修曰:「是何眇小丈夫,乃能直言若此?」慎修對曰:「臣面陋而心善。」上為大笑。復出為湖南衡郴永道。十二年,乞病歸,卒。

高密李元直為御史在其前,以剛直著。慎修與齊名,為「山東二李」。京師稱元直「戇李」,慎修「短李」。

元直,字象山。康熙五十二年進士,改庶吉士,散館授編修。雍正七年,考選四川道監察御史,八閱月,章數十上。嘗歷詆用事諸大臣,謂:「朝廷都俞多,籲咈少,有堯、舜,無皋、夔。」上不懌,召所論列諸大臣大學士硃軾、張廷玉輩並及元直,詰之曰:「有是君必有是臣。果如汝所言無皋、夔,朕又安得為堯、舜乎?」元直抗論不撓,上謂諸大臣曰:「彼言雖野,心乃無他。」次日,復召入,獎其敢言。會廣東貢荔枝至,以數枚賜之。未幾,命巡視臺灣,疏請增養廉、絕饋遺,並條上番民利病數十事。臺灣居海外,巡視御史至,每自視如客,事一聽於道府。元直悉反所為,時下所屬問民疾苦。欲有所施措,督撫劾其侵官,遂鐫級去。家居二十餘年,卒。世宗嘗曰:「元直可保其不愛錢,但慮任事過急。」又嘗諭諸大臣曰:「甚矣才之難得!元直豈非真任事人?乃剛氣逼人太甚。」元直晚年言及知遇,輒泣下。初在翰林,與孫嘉淦、謝濟世、陳法交,以古義相勖,時稱四君子。及嘉淦總督湖廣,治濟世獄,徇巡撫許容意,為時論所不直,元直遂與疏焉。

法,字定齋,貴州安平人。康熙五十二年進士,自檢討官至直隸大名道。講學宗硃子,著明辨錄,辨陸、王之失。蒞政以教養為先,手治文告,辭意懇摯。既久,人猶誦之。

胡定,字登賢,廣東保昌人。雍正十一年進士,改庶吉士,授檢討。乾隆五年,考選陜西道監察御史。七年,湖南巡撫許容劾糧道謝濟世,下湖廣總督孫嘉淦按治,將坐濟世罪,八年二月,定疏陳容陷濟世、嘉淦袒容狀,錄湖南民揭帖,謂布政使張璨、按察使王玠、長沙知府張琳、衡州通判方國寶、善化知縣樊德貽承容指,朋謀傾陷;並述京師民諺,目容為媼,謂其妒賢嫉能如婦人之陰毒。疏入,上命戶部侍郎阿里袞如湖南會嘉淦覆勘,並令定從往。會湖南嶽常道倉德密揭都察院,發璨請託私改文牘狀,阿里袞至湖南,雪濟世枉。上奪嘉淦、容等職,諭謂:「定為言官,言事不實,自有應得之罪譴。今既實矣,若止為濟世白冤抑,其事尚小;因此察出督撫等挾私誣陷,徇隱扶同,使人人知所儆戒,此則有裨於政治,為益良多。至諸行省督撫舉劾必悉秉公心,方為不負委任,若以愛憎為舉劾,如嘉淦、容居心行事,豈不抱媿大廷,負慚夙夜?諸督撫當深自儆省,以嘉淦、容為戒。」定於是負敢言名。

轉兵科給事中,巡視西城。求居民善惡著稱者,皆榜姓名於衢。民有訟者,即時傳訊判結。西山臥佛寺被竊,同官誤以僧自盜奏,定廉得真盜,僧得雪。旋以母老乞歸養。服除,復授福建道御史。疏論內務府郎中某朘民為私利,按治事不實,奪職下刑部,久之讞定,罷歸。二十二年,上南巡,定迎駕杭州,復原銜。卒,年七十九。著有雙柏廬文集。

仲永檀,字襄西,山東濟寧人。乾隆元年進士,改庶吉士,授檢討。五年,考選陜西道監察御史。疏請酌減上元燈火聲樂,略言:「人君一日萬幾,一有暇逸之心,即啟怠荒之漸。每歲上元前後,燈火聲樂,日有進御。原酌量裁減,豫養清明之體。」上降旨,謂:「書云『不役耳目』,詩云『好樂無荒』,古聖賢垂訓,朕所夙夜兢兢而不敢忽者。惟是歲時宴賞,慶典自古有之,況元正獻歲,外籓蒙古朝覲有不可缺之典禮。朕踵舊制而行之,未嘗有所增益。至於國家政事,朕仍如常綜理,並未略有稽遲。永檀胸有所見,直陳無隱,是其可嘉處,朕亦知之。」

京師民俞君弼者,為工部鑿匠,富無子。既死,其戚許秉義謀爭產。內閣學士許王猷與同族,囑招九卿會其喪,示聲氣,且首君弼有藏鏹。步軍統領鄂善以聞,詔嚴鞫,秉義論罪如律,並奪王猷職,旨戒飭九卿。六年,永檀奏:「風聞鄂善受俞氏賄萬金,禮部侍郎吳家駒赴吊得其貲;又聞赴吊不僅九卿,大學士張廷玉以柬往,徐本、趙國麟俱親會,詹事陳浩為奔走,謹據實密奏,備訪查。」又言:「密奏留中事,外間旋得消息,此必有私通左右暗為宣洩者。權要有耳目,朝廷將不復有耳目矣。」疏入,上疑永檀妄言,命怡親王,和親王,大學士鄂爾泰、張廷玉、徐本,尚書訥親、來保按治,摘永檀奏宣洩密奏留中果何事,又謂權要私通左右,此時無可私通之左右,亦無能私通左右之權要,詰何所見,命直陳。鄂善僕及居間納賕者,皆承鄂善得俞氏賄,和親王等以聞。上召和親王、鄂爾泰、訥親、來保同鄂善入見,上溫諭導其言,鄂善乃承得白金千。上諭鄂善曰:「汝罪於律當絞。汝嘗為大臣,不忍棄諸市。然汝亦何顏復立於人世乎?汝宜有以自處。」既又下和親王等會大學士張廷玉、福敏、徐本,尚書海望,侍郎舒赫德詳議,如上諭。乃命訥親、來保持王大臣奏示鄂善,鄂善乃言未嘗受賕。上因怒責鄂善欺罔,奪職下刑部,又命福敏、海望、舒赫德會鞫,論絞,上仍令賜死。家駒、浩並奪職。永檀答上詢宣洩留中事,舉吳士功密劾史貽直以對。和親王等諮察大學士趙國麟等赴俞氏會喪雖無其事,然語有所自來。上乃獎永檀摘奸發伏,直陳無隱,擢僉都御史。

國麟獨奏辨,言:「永檀風聞言事,以蒙恩坐論之崇班,而被以跪拜細人之醜行。事有流弊,宜防其漸。數有往復,當保其終。明季言路與政府各分門戶,互相擠排,綱紀浸以大壞。在今日權無旁撓,言無偏聽,寧為未然之慮,不弛將至之防。乞特降諭旨,明示天下,以超擢永檀為獎其果敢,宥其冒昧。嗣後凡詆斥大臣按之無實者,別有處分。則功過不相掩,而賞罰無偏曲。如以臣言過戇,乞賜罷斥,或容解退,以全初心。」上手詔謂:「超擢永檀,亦善善欲長、惡惡欲短之意,大學士所云,老成遠慮,朕甚嘉納。其入閣視事,毋違朕意。」而國麟求去益力,給事中盧秉純劾國麟,謂:「上詢國麟嘗會俞氏喪否,出以告其戚休致光祿寺卿劉籓長,語無狀。」上召籓長,令鄂爾泰、張廷玉、徐本、訥親、來保按其事,因謂籓長市井小人,國麟與論姻,又嘗奏薦,事非是。遣鄂爾泰等諭意,令請退。居數日,國麟疏不至,乃特詔左遷,留京師待缺。秉純語過當,籓長刺探何緣被譴,不謹,皆奪職。

又擢永檀左副都御史。貴州甕安民羅尚珍詣都察院訴家居原任四川巡撫王士俊侵其墓地,命永檀如貴州會總督張廣泗按治,士俊論罪如律。河南巡撫雅爾圖劾永檀自貴州還京師,道南陽,縱其僕撻村民,下部議罰俸。七年十二月,命如江南會巡撫周學健治賑,未行,永檀以密奏留中事告大學士鄂爾泰子鄂容安。上命奪職,下內務府慎刑司,令莊親王,履親王,和親王,平郡王,大學士張廷玉、徐本,尚書訥親、來保、哈達哈按其事。鄂容安、永檀自承未奏前商謀,既奏後照會。王大臣等用洩漏機密事務律論罪,上責其結黨營私,用律不合,令會三法司覆讞。王大臣等因請刑訊,並奪大學士鄂爾泰職逮問,上謂鄂爾泰受遺大臣,不忍深究,下吏議,示薄罰。永檀、鄂容安亦不必刑訊,永檀受恩特擢,乃依附師門,有所論劾,無不豫先商酌,暗結黨援,排擠異己,罪重大;鄂容安罪亦無可逭,但較永檀當末減。命定擬具奏,奏未上,永檀卒於獄。鄂容安論戍,上寬之,語在鄂容安傳。

柴潮生,字禹門,浙江仁和人。雍正二年舉人,授內閣中書,充軍機處章京。累遷工部主事。乾隆七年,考選山西道監察御史。是歲旱,上降詔求言。潮生疏言:「君咨臣儆,治世之休風;益謙虧盈,檢身之至理。臣伏讀上諭有云:『爾九卿中能責難於君者何人?陳善閉邪者何事?』此誠我皇上虛懷若谷、從諫弗弗之盛心也。今歲入春以來,近京雨澤未經霑足,宵旰焦勞,無時或釋。惟是天時雨暘,難以窺測;而人事修省,不妨過為責難。修省於事為者,一動一言,純雜易見;修省於隱微者,不聞不見,朕兆難窺。君心為萬化之源,普天率土,百司萬姓,皆於此託命焉。皇上萬幾餘暇,豈無陶情適興之時?但恐一念偶動,其端甚微,而自便自恕之機,或乘於不及覺,遂致潛滋暗長而莫可遏。則俄頃間之出入,即為皇功疏密所關。伏乞皇上於百爾臣工所不及見,左右近習所不及窺,朝夕愈加劼毖,豈特隨時修省致感召之休徵已哉?」

八年,天津、河間二府大旱。九年,潮生復疏言:「河間、天津二府經流之大河三:曰衛河,曰滹沱河,曰漳河。其餘河間分水之支河十有一,瀦水之澱泊十有七,蓄水之渠三;天津分水之支河十有三,瀦水之澱泊十有四,受水之沽六:水道至多。向若河渠深廣,蓄洩有方,旱歲不能全收灌溉之功,亦可得半。即不然,而平日之蓄積,亦可支持數月,以需大澤之至。何至拋田棄宅,挈子攜妻,流離道路哉?水利之廢,即此可知矣。甘霖一日不足,則賑費固不可已。臣竊以為徒費之於賑恤,不如大發帑金,遴遣大臣經理畿輔水利,俾以濟饑民、消旱潦,且轉貧乏之區為富饒。救時之急務,籌國之遠謨,莫以易此。臣考漢張堪為漁陽太守,於狐奴開稻田八千頃,狐奴今"昌平也。北齊裴延俊為幽州刺史,修古督亢坡,溉田萬餘畝,督亢今涿州也。宋何承矩為河北制置使,於雄、鄚、霸州興堰六百里灌田。明汪應蛟為天津巡撫,捐俸開二千畝,畝收四五石。今東西二澱,即承矩之塘濼,天津十字圍,即應蛟水田之遺址。國朝李光地為巡撫,請興河間水田,言涿州水占之地,每畝售錢二百,開成水田畝易銀十兩。上年總督高斌請開永定河灌田,亦云查勘所至,眾情欣悅。臣聞石景山有莊頭修姓,自引渾河灌田,比常農畝收數倍。蠡縣亦有富戶自行鑿井,旱歲能收其利。霸州知州硃一蜚勸民開井二十餘口,民頗賴之。證之近事,復確有據,則水利之可興也決矣。今請特遣大臣齎帑金數十萬兩,往河間、天津二府,督同道府牧令,分委佐貳雜職,除運道所關,及滹沱正流水性暴急,慎勿輕動,其餘河渠澱泊,凡有故跡可尋者,皆重加疏浚。又於河渠澱泊之旁,各開小河;小河之旁,各開大溝:皆務深廣,度水力不及則止。節次建立水門,遞相灌注。旱則引水入溝以溉田,潦則放閘歸河以洩水。其離水遼遠之處,每田一頃,掘井一口,十頃掘大塘一口,亦足供用。其中有侵及民田,並古陂廢堰為民業已久者,皆計畝均分撥還,即將現在受賑饑民及外來流民,停其賑給,按地分段,就工給值,酌予口糧,寧厚無減。一人在役,停其家賑糧二口;二人在役,停其家賑糧四口。其餘口及一戶皆不能執役者,仍如例給賑。其疏浚之處,有可耕種,即借予工本,分年徵還。更請別簡大臣,齎帑金分巡直隸各府,一如河間、天津二府,次第舉行。或曰:『北土高燥,不宜稻種,土性沙鹼,水入即滲,挖掘民地,易起怨聲。前朝徐貞明行之而立敗,怡賢親王與大學士硃軾之經理亦垂成而坐廢,可為明鑒。』臣按九土之種異宜,未聞稻非冀州之產,玉田、豐潤秔稻油油。且今第為之興水利耳,固不必強之為水田也。或疏或浚,則用官資,可稻可禾,聽從民便。此不疑者一也。土性沙鹼,是誠有之,不過數處耳,豈遍地皆沙鹼乎?且即使沙鹼,而多一行水之道,比聽其沖溢者不猶愈於已乎?此不疑者二也。若以溝渠為捐地,尤非知農事者。凡力田者,務盡力而不貴多墾。今使十畝之地,捐一畝以蓄水,而九畝倍收,較十畝皆薄入孰利?況捐者又予撥還。此不疑者三也。至前人屢行屢罷,此亦有由,貞明所言百世之利,其時御史王之棟參劾,出於奄人勛戚之意。其疏亦第言滹沱不可開,未嘗言水田不可行也。但其募南人開墾,即以地予之,又許占籍。左光鬥之屯學亦然。是奪北人之田,又塞其功名之路,其致人言也宜矣。至營田四局,成績具在。當日效力差員,不無舉行未善,所以賢王一沒,遂過而廢之,非深識長算者之所出也。非常之原,黎民所懼,所貴持久,乃可有功。秦開鄭、白之渠,利及百世,而當時至欲殺水工鄭國。漢河東太守番系引汾水灌田,河渠數徙,田者不能償種。至唐長孫恕復鑿之,畝收十石。凡始事難,成事易。賡續以終之則是,中道而棄之則非。此不疑者四也。至水利既興,招募農師,造作水器,逐年作何經理,俾永無湮塞,應聽在事大臣詳加籌畫。皇上視民如子,凡有賑恤,千萬帑金亦無可惜。即如開通京師溝道,估費二十餘萬,以視興修一省水利,輕重較然。況此舉乃以阜財,非以費財。天災國家代有,荒政未有百全,何如擲百萬於水濱,而立收國富民安之效?縱有堯災湯旱,亦可挹彼注茲,是謂無弊之賑恤。連年米價屢廑聖懷,盡停採買,豈可久行?捐監輸倉,亦非上策。若小民收穫素裕,自然二鬴有資。臣訪問直隸士民,皆云:『有水之田較無水之田,相去不啻再倍。』是謂不竭之常平。近畿多八旗莊地,直隸亦京兆股肱,皆宜致之富饒,始可居重馭輕。漢武帝徙豪民於關中,明成祖遷富家於帝裏,固非王政,不失深謀。若水利既興,自然軍民兩利,是謂無形之帑藏。且雨者水土之氣所上騰而下澤也,土氣太甚,則水氣受制。直隸近年以來,閔雨者屢矣。但使水土均調,自可雨暘時若,是謂有驗之調燮。且水性分之則利,合之則害;用之則利,棄之則害。故周用有言:『人人皆治田之人,即人人皆治水之人。』張伯行亦主此論。陸隴其為靈壽令,督民濬衛河。其始頗有怨言,謂開無水之河以病民;既而水潦大至,獨靈壽有宣導,歲竟有秋。貨殖者旱則資舟,為國者備斯無患,是謂隱寓之河防。今生齒日繁,民食漸絀。臣愚以為盡興西北之水田,闢東南之荒地,則米價自然平減。但事體至大,請先以直隸為端,行之有效,次第舉行。樂利萬年,庶其在此!」

十年,疏陳理財三策,言:「治天下要務,惟用人、理財兩大事。承平日久,供億浩繁,損上益下,日廑宸衷;而量入為出,似尚未籌至計。禮曰:『財用足故百志成。』若少有窘乏,則蠲徵平賦、恤災厚下之大政俱不得施。遲之又久,則一切茍且之法隨之以起。此非天下之小故也。頃見臺臣請定會計疏,言每年所入三千六百萬,出亦三千六百萬。就今日計之,所入僅供所出。就異日計之,所入殆不足供所出。以皇上之仁明,國家之閒暇,而不籌一開源節流之法,為萬世無弊之方,是為失時。臣等荷恩,備官臺省,不能少竭涓埃,協贊遠謨,是為負國。以臣之計,一曰開邊外之屯田以養閒散,一曰給數年之俸餉散遣漢軍,一曰改捐監之款項以充公費,三者行而後良法美意可得而舉也。滿洲、蒙古、漢軍各有八旗,丁口蕃昌,視順治時蓋一衍為十,而生計艱難,視康熙時已十不及五,而且仰給於官而不已。局於五百里之內而不使出,則將來上之弊必如北宋之養兵,下之弊亦必如有明之宗室,此不可不籌通變者也。臣聞奉天沿邊諸地,水泉肥美,請遣幹略大臣,分道經理。視可屯之處,發帑建堡墩,起屋廬,置耕牛農具,令各旗滿洲除正身披甲在京當差,其次丁、餘丁力能耕者前往居住。所耕之田,即付為永業,分年扣完工本,更不升科。惟令農隙操演,數年之後皆成勁卒。逐年發往軍臺之人,令其分地捐貲效力,此後有原往者,令其陸續前往。此安頓滿洲閒散之法也。漢軍八旗已奉聽其出旗之旨,以定例太拘,故散遣寥寥。今請不論出仕與否,概許出旗。其家現任居官者給三年俸餉,無居官者給六年俸餉。其家產許之隨帶,任其自便。則貧富各不失所,而五年以後國帑節省無窮。即一時不能盡給,分作數年以次散遣,都統以下、章京以上各官,改補綠旗提鎮將弁。此安頓漢軍之法也。臣又按耗羨歸公,天下之大利,亦天下之大弊也。康熙間,法制寬略,州縣於地丁外私徵火耗,其陋規匿稅亦未盡釐剔。自耗羨歸公,一切弊竇悉滌而清之,是為大利。然向者本出私徵,非同經費,其端介有司,不肯妄取,上司亦不敢強,賢且能者則以地方之財治地方之事,故康熙間循吏多實績可紀,而財用亦得流通。自耗羨歸公,輸納比於正供,出入操於內部,地丁公費,除官吏養廉無餘剩;官吏養廉,除分給幕客家丁修脯工資,及事上接下之應酬,輿馬蔬薪之繁費,亦無餘剩。地方有應行之事、應興之役,一絲一忽取公帑,有司上畏戶、工二部之駁詰,下畏身家之賠累,但取其事之美觀而無實濟者,日奔走之以為勤。故曰天下之大弊也。夫生民之利有窮,故聖人之法必改。今耗羨歸公之法勢無可改,惟有為地方別立一公項,俾任事者無財用窘乏之患,而後可課以治效之成。臣請將常平倉儲仍照舊例辦理,捐監一項留充各省公用,除官俸兵餉動用正項,餘若災傷當拯恤,孤貧當養贍,河渠水利當興修,貧民開墾當借給工本,壇廟、祠宇、橋梁、公廨當修治,採買倉穀價值不敷,皆於此動給,以地方之財,治地方之事。如有大役大費,則督撫合全省而通融之;又有不足,則移鄰省而協濟之。稽察屬司道,核減屬督撫,內部不必重加切核,則經費充裕,節目疏闊,而地方之實政皆可舉行。設官分職,付以人民,只可立法以懲貪,不可因噎而廢食。唐人減劉晏之船料,而漕運不繼;明人以周忱之耗米歸為正項,致逋負百出,路多饑殍。大國不可以小道治,善理財者,固不如此。此捐監之宜充公費也。三法既行,則度支有定,經費有資,當今要務,無急於此者。伏乞皇上深留睿慮,敕公忠有識大臣,詳議施行。」

尋遷兵科給事中,巡視北城。乞歸侍母,孝養肫至。貧,以醫自給。久之,卒。

儲麟趾,字履醇,江南荊溪人。乾隆四年進士,改庶吉士,授編修。進諸經講義,援據儒先,責難陳善,辭旨醇美。十四年,考選貴州道監察御史。編修硃荃與大學士張廷玉有連,督四川學政,母死發喪緩。麟趾疏劾,語不避廷玉,高宗以是知其伉直。

嘗大旱,麟趾應詔上疏,略言:「臣聞天道若持衡然。故雨暘寒燠,無時不得其平;而氣化偶偏,必於亢陽伏陰示其象。然往來推行,久而必復其常者,天道之無私也。君道法天,亦若持衡然。故喜怒刑賞,無事不得其平;而意見偶偏,必於用人行政露其機。然斟酌損益,終必歸於大中至正者,君德之極盛也。漢臣董仲舒曰:『善言天者,必有驗於人,天人相應,捷於桴鼓。春秋所以詳書災異也。』皇上至聖極明,豈復有纖芥之事足以召祲而致災者?但愚臣蠡測管窺,以為自古人主患不明,惟皇上患明之太過;自古人主患不斷,惟皇上患斷之太速。即如擢一官、點一差,往往出人意表,為擬議所不及。此則皇上意見之稍偏,而愚臣所謂聖明英斷之太過者也。史臣之贊堯曰:『乃聖乃神。』宋儒硃子曰:『聖人,神明不測之號。』夫所貴乎不測者,錯綜參伍,與時偕行,而非於彼於此不可思議之謂也。此雖不足上累聖德萬分之一,然臣尤原皇上開誠布公,太和翔洽,要使天下服皇上用人之至當,不必徒使天下驚皇上用人之甚奇。若雲防微杜漸,不得不爾,則國法具在,試問諸臣行事邪正,又誰能欺皇上之洞鑒者?抑臣又聞之,唐臣韓愈曰:『獨陽為旱,獨陰為水。君陽臣陰,有君無臣,是以久旱。』今皇上宵衣旰食,焦勞於法宮之中,而王公大臣拱手備位,不聞出其謀畫,上贊主德,輔宣聖化。是君勞於上,臣逸於下,天道下濟而地道不能上行。其於致旱,理或宜然。臣區區之忱,原皇上虛中無我,一切用人行政,不改鑒空衡平之體。又於一二純誠憂國之大臣,時賜召對,清宴之餘,資其輔益。必能時雨時風,消殄旱災矣。」

麟趾累遷太僕寺卿,移宗人府府丞。引疾歸,家居十餘年。卒,年八十二。

論曰:諫臣之益人國,最上匡君德,次則綢繆軍國,洞百年之利害。若夫擊邪毖患,岳岳不避權要,固亦有不易言者。高宗嗣服,虛己納諫。一士、慎修、潮生、麟趾,其所獻替,合陳善責難之誼。潮生所論理財三策尤閎遠,惜不能用也。定劾許容,永檀彈鄂善,皆能舉其職者。永檀乃以漏言敗,異哉!


\end{pinyinscope}