\article{表七}

\begin{pinyinscope}
外戚表

班書始立外戚恩澤侯表,遼、明二史因之。遼外戚不皆有封爵,然世選北府宰相預政事。明則揚、徐二王僅假虛號,自後皆封侯伯。嘉靖間,詔不得與汗馬餘勛並列。惟分封大邑,帶礪相承,未嘗區以別也。清初,太祖娶於葉赫,草昧干戈,制度未備。太宗、世祖娶於蒙古,追進崇封,外戚恩澤自此始。雍正八年,世宗詔定外戚為承恩公。乾隆四十三年,高宗又詔後族承恩,與佐命功臣櫛風沐雨、拓土開疆者實難並論,俱改為三等公。名既專屬,等復攸殊,裁抑制防,視明尤肅。用是終清世外家皆謹守法度,無預政事者,不可謂非詒謀之善也。明史用班氏例,兼及宦官、恩幸之得封者,尤清所未有。茲次第諸後族為外戚表。凡以外戚封,及其家初有爵以外戚進者皆入焉。後族別以功封,仍列功臣世爵表。


\end{pinyinscope}