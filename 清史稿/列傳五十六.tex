\article{列傳五十六}

\begin{pinyinscope}
索額圖明珠餘國柱佛倫

索額圖,赫舍里氏,滿洲正黃旗人,索尼第二子。初授侍衛,自三等洊升一等。康熙七年,授吏部侍郎。八年五月,自請解任效力左右,復為一等侍衛。及鼇拜獲罪,大學士班布爾善坐黨誅,授索額圖國史院大學士,兼佐領。九年,改保和殿大學士。十一年,世祖實錄成,加太子太傅。十五年,大學士熊賜履票本有誤,改寫草簽,既又毀去。索額圖與大學士巴泰、杜立德等疏劾,賜履坐罷歸。十八年,京察,侍講學士顧八代隨征稱職,翰林院以「政勤才長」注考,索額圖改注「浮躁」,竟坐降調。語詳顧八代傳。

索額圖權勢日盛。會地震,左都御史魏象樞入對,陳索額圖怙權貪縱狀,請嚴譴。上曰:「修省當自朕始!」翌日,召索額圖及諸大臣諭曰:「茲遘地震,朕反躬修省。爾等亦宜洗滌肺腸,公忠自矢。自任用後,諸臣家計頗皆饒裕,乃朋比徇私,益加貪黷。若事情發覺,國法具在,決不爾貸!」是時索額圖、明珠同柄朝政,互植私黨,貪侈傾朝右,故諭及之。上並書「節制謹度」榜賜焉。

十九年八月,以病乞解任,上優旨獎其「勤敏練達,用兵以來,贊畫機宜」,改命為內大臣。尋授議政大臣。先是索額圖兄噶布拉,以冊謚孝誠仁皇后推恩所生,封一等公;弟心裕,襲索尼初封一等伯;法保,襲索尼加封一等公。二十三年三月,以心裕等嬾惰驕縱,責索額圖弗能教,奪內大臣、議政大臣、太子太傅,但任佐領,並奪法保一等公。二十五年,授領侍衛內大臣。

時俄羅斯屢侵黑龍江邊境,據雅克薩,其眾去復來,上發兵圍之。察罕汗謝罪,使費耀多囉等來議界。二十八年,上命索額圖與都統佟國綱往議。索額圖奏謂:「尼布楚、雅克薩兩地當歸我。」上曰:「尼布楚歸我,則俄羅斯貿易無所棲止,可以額爾固納河為界。」索額圖等與議,費耀多囉果執尼布楚、雅克薩為請。索額圖等力斥之,仍宣上意,以額爾固納河及格爾必齊河為界,立碑而還。

二十九年,上以裕親王福全為大將軍,擊噶爾丹,命索額圖將盛京、吉林、科爾沁兵會於巴林,敗噶爾丹於烏闌布通。以不窮追,鐫四級。三十五年,從上親征,率八旗前鋒、察哈爾四旗及漢軍綠旗兵前行,並命督火器營。大將軍費揚古自西路抵圖拉。上駐克魯倫河,噶爾丹遁走。費揚古截擊之於昭莫多,大敗其眾。三十六年,上還幸寧夏,命索額圖督水驛,會噶爾丹死。敘功,復前所鐫級。四十年九月,以老乞休,心裕代為領侍衛內大臣。

索額圖事皇太子謹,皇太子漸失上意。四十一年,上閱河至德州,皇太子有疾,召索額圖自京師至德州侍疾。居月餘,皇太子疾愈,還京師。是歲,心裕以虐斃家人奪官。四十二年五月,上命執索額圖,交宗人府拘禁,諭曰:「爾為大學士,以貪惡革退,後復起用,罔知愧悔。爾家人訐爾,留內三年,朕意欲寬爾。爾乃怙過不悛,結黨妄行,議論國事。皇太子在德州,爾乘馬至中門始下,即此爾已應死。爾所行事,任舉一端,無不當誅。朕念爾原系大臣,心有不忍,姑貸爾死。」又命執索額圖諸子交心裕、法保拘禁,諭:「若別生事端,心裕、法保當族誅!」諸臣黨附索額圖者,麻爾圖、額庫禮、溫代、邵甘、佟寶並命嚴錮,阿米達以老貸之。又命諸臣同祖子孫在部院者,皆奪官。江潢以家有索額圖私書,下刑部論死。仍諭滿洲人與偶有來往者,漢官與交結者,皆貸不問。尋索額圖死於幽所。

後數年,皇太子以狂疾廢,上宣諭罪狀,謂:「索額圖助允礽潛謀大事,朕知其情,將索額圖處死。今允礽欲為索額圖報仇,令朕戒慎不寧。」並按誅索額圖二子格爾芬、阿爾吉善。他日,上謂廷臣曰:「昔索額圖懷私,倡議皇太子服御俱用黃色,一切儀制幾與朕相似。驕縱之漸,實由於此。索額圖誠本朝第一罪人也!」

明珠,字端範,納喇氏,滿洲正黃旗人,葉赫貝勒金臺石孫。父尼雅哈,當太祖滅葉赫,來降,授佐領。明珠自侍衛授鑾儀衛治儀正,遷內務府郎中。康熙三年,擢總管。五年,授弘文院學士。七年,命閱淮、揚河工,議復興化白駒場舊閘,鑿黃河北岸引河。旋授刑部尚書。改都察院左都御史,充經筵講官。十一年,遷兵部尚書。十二年,上幸南苑,閱八旗甲兵於晾鷹臺。明珠先布條教使練習之,及期,軍容整肅,上嘉其能,因著為令。

康熙初,南疆大定,留重兵鎮之:吳三桂雲南,尚可喜廣東,耿精忠福建。十餘年,漸跋扈,三桂尤驕縱。可喜亦憂之,疏請撤籓,歸老海城。精忠、三桂繼請。上召諸大臣詢方略,戶部尚書米思翰、刑部尚書莫洛等主撤,明珠和之。諸大臣皆默然。上曰:「三桂等蓄謀久,不早除之,將養癰成患。今日撤亦反,不撤亦反,不若先發。」因下詔許之。三桂遂反,精忠及可喜子之信皆叛應之。時爭咎建議者,索額圖請誅之。上曰:「此出自朕意,他人何罪?」明珠由是稱上旨。十四年,調吏部尚書。十六年,授武英殿大學士,屢充實錄、方略、一統志、明史諸書總裁,累加太子太師。迨三叛既平,上諭廷臣以前議撤籓,惟明珠等能稱旨,且曰:「當時有請誅建議者,朕若從之,皆含冤泉壤矣!」

明珠既擅政,簠簋不飭,貨賄山積。佛倫、餘國柱其黨也,援引致高位。靳輔督南河,主築堤束水,下游不濬自通。於成龍等議濬下游,與異議。輔興屯田,議者謂不便於民,多不右輔,明珠獨是其議。蔡毓榮、張汧皆明珠所薦引者也,迨得罪按治,恐累舉者,傅輕比,上諭斥,始定。與索額圖互植黨相傾軋。索額圖生而貴盛,性倨肆,有不附己者顯斥之,於朝士獨親李光地。明珠則務謙和,輕財好施,以招來新進,異己者以陰謀陷之,與徐乾學等相結。索額圖善事皇太子,而明珠反之,朝士有侍皇太子者,皆陰斥去。薦湯斌傅皇太子,即以傾斌。會天久不雨,光地所薦講官德格勒明易,上命筮,得夬,因陳小人居鼎鉉,天屯其膏,語斥明珠。事具德格勒傳。

二十七年,御史郭琇疏劾:「明珠、國柱背公營私,閣中票擬皆出明珠指麾,輕重任意。國柱承其風旨,即有舛錯,同官莫敢駁正。聖明時有詰責,漫無省改。凡奉諭旨或稱善,明珠則曰『由我力薦』;或稱不善,明珠則曰『上意不喜,我從容挽救』;且任意附益,市恩立威,因而要結群心,挾取貨賄。日奏事畢,出中左門,滿、漢部院諸臣拱立以待,密語移時,上意罔不宣露。部院事稍有關系者,必請命而行。明珠廣結黨羽,滿洲則佛倫、格斯特及其族侄富拉塔、錫珠等,凡會議會推,力為把持;漢人則國柱為之囊橐,督撫籓臬員缺,國柱等展轉徵賄,必滿欲而後止。康熙二十三年學道報滿應升者,率往論價,缺皆預定。靳輔與明珠交結,初議開下河,以為當任輔,欣然欲行。及上欲別任,則以於成龍方沐上眷,舉以應命,而成龍官止按察使,題奏權仍屬輔,此時未有阻撓意也。及輔張大其事,與成龍議不合,乃始一力阻撓。明珠自知罪戾,對人柔顏甘語,百計款曲,而陰行鷙害,意毒謀險。最忌者言官,惟恐發其奸狀,考選科道,輒與訂約,章奏必使先聞。當佛倫為左都御史,見御史李興謙屢疏稱旨,吳震方頗有彈劾,即令借事排陷。明珠智術足以彌縫罪惡,又有國柱奸謀附和,負恩亂政。伏冀立加嚴譴。」

疏入,上諭吏部曰:「國家建官分職,必矢志精白,大法小廉。今在廷諸臣,自大學士以下,惟知互相結引,徇私傾陷。凡遇會議,一二倡率於前,眾附和於後,一意詭隨。廷議如此,國是何憑?至於緊要員缺,特令會同推舉,原期得人,亦欲令被舉者警心滌慮,恐致累及舉者,而貪黷匪類,往往敗露。此皆植黨納賄所致。朕不忍加罪大臣,且用兵時有曾著勞績者,免其發覺。罷明珠大學士,交領侍衛內大臣酌用。」未幾,授內大臣。後從上征噶爾丹,督西路軍餉,敘功復原級。

明珠自罷政後,雖權勢未替,然為內大臣者二十年,竟不復柄用。四十七年,卒。子性德、揆敘自有傳。

餘國柱,字兩石,湖廣大冶人。順治九年進士,授兗州推官。遷行人司行人,轉戶部主事。康熙十五年,考授戶科給事中。時方用兵,國柱屢疏言籌餉事,語多精覈。二十年,擢左副都御史。旋授江寧巡撫,請設機制寬大緞疋。得旨:「非常用之物,何為勞費?」當明珠用事,國柱務罔利以迎合之,及內轉左都御史,遷戶部尚書,湯斌繼國柱撫江蘇;國柱索斌獻明珠金,斌不能應,由是傾之。二十六年,授武英殿大學士,益與明珠結,一時稱為「余秦檜」。會上謁陵,中途召於成龍入對,成龍盡發明珠、國柱等貪私。上歸詢高士奇,士奇亦以狀聞。及郭琇疏論劾,言者蜂起,國柱門人陳世安亦具疏糾之,頗中要害,國柱遂奪官。既出都,於江寧治第宅,營生計,復為給事中何金蘭所劾,命逐之回籍。卒於家。

佛倫,舒穆祿氏,滿洲正白旗人。自筆帖式累遷內閣學士。吳三桂既死,其孫世璠猶據滇、黔,命佛倫總理糧餉,通鎮遠運道,旋兼理四川糧餉。事平,遷刑部侍郎。尋遷左都御史,擢工部尚書,轉刑、戶兩部。先是下河工程,靳輔與按察使於成龍議不協,命佛倫偕侍郎熊一瀟等勘議。佛倫受明珠指,議如輔言,為總漕慕天顏所劾。御史陸祖修亦劾佛倫袒輔,且言:「九卿會議時,尚書科爾坤等阿佛倫意,尚書張玉書、左都御史徐乾學言興屯所占民田應還之民,科爾坤置不聞。他九卿或不得見只字。」上怒,下部嚴議。及郭琇劾明珠,指佛倫為明珠黨,因解佛倫任。召輔等廷對,佛倫乃奏停屯田,並汰前所設官。部議奪佛倫官,上命留佐領。旋授內務府總管。

出為山東巡撫,疏請均賦役,令紳民一體應役,詔嘉其實心任事。初,濰縣知縣硃敦厚以贓私為巡撫錢鎯所發,乞徐乾學請於鎯,獲免,且內擢主事。至是事發,下佛倫鞫實,乾學坐奪官。佛倫又劾琇知吳江縣時,嘗侵公帑,其父景昌故名爾標,乃明御史黃宗昌奴,坐賊黨誅,琇改父名冒封典,當追奪。乾學故附明珠,後相失,或傳琇疏乾學實主之,故佛倫以是報。尋擢川陜總督,入為禮部尚書。三十八年,授文淵閣大學士。三十九年,琇入覲,訟父受誣。上詰佛倫,自承不實,當奪官,援赦得免。未幾,以原品休致。旋卒。

論曰:康熙中,滿洲大臣以權位相尚者,惟索額圖、明珠,一時氣勢熏灼,然不能終保令名,卒以貪侈敗。索額圖以附皇太子得罪,禍延於後嗣。明珠與索額圖競權,不附皇太子,雖被彈事罷相,聖祖猶念其贊撤籓,力全之,以視索額圖,豈不幸哉?若國柱、佛倫,則權門之疏附矣。


\end{pinyinscope}