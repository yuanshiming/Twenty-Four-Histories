\article{列傳一百六十一}

\begin{pinyinscope}
姚文田戴敦元硃士彥何凌漢李振祜宗室恩桂

姚文田,字秋農,浙江歸安人。乾隆五十九年,高宗幸天津,召試第一,授內閣中書,充軍機章京。嘉慶四年一甲一名進士,授修撰。迭典廣東、福建鄉試,督廣東、河南學政,累遷祭酒。

十八年,入直南書房。會因林清之變,下詔求言,文田疏陳,略謂:「堯、舜、三代之治,不越教養兩端:為民正趨向之路,知有長上,自不干左道之誅;為民廣衣食之源,各保身家,自不致有為惡之意。近日南方患賦重,北方患徭多,民困官貧,急宜省事。久督撫任期,則州縣供億少,寬州縣例議,則人才保全多。」次年復上疏,言:「上之於下,不患其不畏,而患其不愛。漢文吏治蒸蒸,不至於奸,愛故也。秦顓法律,衡石程書,一夫夜呼,亂者四起,畏故也。自數年來,開上控之端,刁民得逞其奸;大吏畏其京控,遇案親提,訐訴不過一人,牽涉常至數十,農商廢業,中道奔波,受胥吏折辱,甚至瘐死道斃。國家慎刑之意,亦曰有冤抑耳。從前馬譚氏一案,至今未有正兇,無辜致斃者累累。是一冤未雪,而含冤者且數十人。承審官刑撻橫加,以期得實,其中冤抑,正復不少。欲召天和,其可得乎?頃者林清構逆,搜捕四出,至今未已。小人意圖見長,不能無殃及無辜,奉旨嚴禁,仰見皇上如天之仁。臣以為事愈多則擾愈眾,莠民易逞機謀,良善惟增苦累。應令大小官吏,可結速結,無多株引,庶上下相愛,暴亂不作矣。至所謂養民之政,不外於農桑本務。大江以南,地不如中原之廣,每歲漕儲正供,為京畿所仰給者,無他,人力盡也。兗州以北,古稱沃衍;河南一省,皆殷、周畿內;燕、趙之間,亦夙稱富國。今則地成曠土,人盡惰民,安得不窮困而為盜賊?歲一歉收,先請緩徵,稍甚則加蠲貸,又其甚則截漕發粟以賑之,所以耗國帑者何可算也。運河屢淤,東南漕未可恃,設有意外,何以處此?臣見歷來保薦州縣,必首列勸課農桑,其實盡屬虛談,從無過問。大吏奏報糧價,有市價至四五千錢,僅報二兩內外,其於收成,又虛加分數,相習成風。但使董勸有方,行之一方而收利,自然爭起相效,田野皆闢,水旱有資,豈必盡資官帑,善政乃行哉?民之犯刑,由於不率教;其不率教,由於衣食缺乏而廉恥不興。其次第如此,故養民為首務也。」奏入,仁宗嘉納之,特詔飭各省以勸課農桑為亟,速清訟獄,嚴懲誣枉。

二十年,擢兵部侍郎,歷戶部、禮部。二十二年,典會試。二十四年,督江蘇學政。道光元年,江、浙督撫孫玉庭等議禁漕務浮收,明定八折,實許其加二。文田疏陳積弊曰:「乾隆三十年以前,並無所謂浮收。厥後生齒日繁,物價踴貴,官民交困,然猶止就斛面浮取而已。未幾而有折扣之舉,始每石不過折耗數升,繼乃至五折、六折不等。小民終歲勤動,事畜不贍,勢必與官抗。官即從而制之,所舉以為民罪者三:曰抗糧,曰包完,曰掗交丑米。民間零星小戶、貧苦之家,拖欠勢所必有。若家有數十百畝之產,竟置官賦於不問,實事所絕無。今之所謂抗糧者,如業戶應完若干石,多齎一二成以備折收,書吏等先以淋尖、踢腳、灑散多方糜耗,是已不敷;再以折扣計算,如準作七折,便須再加三四成,業戶必至爭執。間有原米運回,州縣即指為抗欠,此其由也。包完者,寡弱之戶,轉交有力者代為輸納。然官吏果甚公正,何庸託人?可不煩言而自破。民間運米進倉,男婦老幼進城守待,陰雨濕露,猶百計保護,恐米色變傷。謂其特以醜米掗交,殆非人情。惟年歲不齊,米色不能畫一,亦間有之。然官吏非執此三者,不能相制,生監暫革,齊民拘禁,俟其補交,然後請釋。不知此皆良民,非莠民也。此小民不能上達之實情也。然州縣亦有不能不爾者,自開倉訖兌運,修整倉廒蘆席、竹木、繩索、油燭百需,幕丁胥役脩飯工食,加以運丁需索津貼滋甚,至其平日廉俸公項不能敷用。無論大小公事,一到即須出錢料理。即如辦一徒罪之犯,自初詳至結案,約須百數十金。案愈巨則費愈多。遞解人犯,運送糧鞘,事事皆需費用。若不取之於民,謹厚者奉身而退,貪婪者非向詞訟生發不可,吏治更不可問。彼思他弊獲咎愈重,不若浮收為上下咸知,故甘受民怨而不惜。其藉以自肥者固多,而迫於不獲已者蓋亦不少。言事者動稱『不肖州縣』,州縣亦人耳,何至一行作吏,便行同茍賤?此又州縣不能上達之實情也。州縣受掊克之名,而運丁陰受其益,然亦有不能不然者。昔時運道深通,運丁或藉來往攜貨售賣以贍用;後因黃河屢經倒灌,運道受害,慮其船重難行,嚴禁多帶貨物。又從前回空帶鹽,不甚搜查;近因鹽商力絀,未免算及瑣屑,而各丁出息遂盡。加以運道日淺,反多添夫撥淺之費。此費不出之州縣,更無所出。此又運丁不能上達之實情也。數年前因津貼日增,於是定例只準給三百兩。運丁實不濟用,則重船不能開,州縣必獲咎戾,不免私自增給,是所謂三百兩者虛名耳。頃又以浮收過甚,嚴禁收漕不得過八折。州縣入不敷出,則強者不敢與較,弱者仍肆朘削,是所謂八折者亦虛名耳。然民間執詞抗官,官必設法箝制,而事端因以滋生,皆出於民心之不服。若將此不靖之民盡法懲處,則既困浮收,復陷法網,民心恐愈不平。若一味姑容隱忍,則小民開犯上之風,將致不必收漕,而亦目無官長。其於紀綱法度,所關實為匪細。」疏入,下部議。時在廷諸臣多以為言,文田持議切中時弊,最得其平。詔禁浮收,裁革運丁陋規,八折之議遂寢。

四年,擢左都御史。七年,遷禮部尚書。尋卒,依尚書例賜恤,謚文僖。

文田持己方嚴,數督學政,革除陋例,斥偽體,拔真才,典試號得士。論學尊宋儒,所著書則宗漢學。博綜群籍,兼諳天文占驗。林清之變未起,彗入紫微垣;道光初,彗見南斗下,主外夷兵事:文田皆先事言之。

戴敦元,字金溪,浙江開化人。幼有異稟,過外家,一月盡讀其室中書。十歲舉神童,學政彭元瑞試以文,如老宿;面問經義,答如流。嘆曰:「子異日必為國器!」年十五,舉鄉試。乾隆五十五年,成進士,選庶吉士,散館改禮部主事,銓授刑部主事,典山西鄉試。累遷郎中。嘉慶二十四年,出為廣東高廉道。道光元年,擢江西按察使。

敦元初外任,以情形非素習,蘇州多粵商,過訪風土利弊,久之始去,盡得要領。至江西,無幕客,延屬吏諳刑名者以助,數月清積牘四千餘事。二年,遷山西布政使,單車之任,輿夫館人莫知為達官。籓署有陋規曰釐頭銀,上下取給,敦元革之,曰:「官有養廉,僕御官所豢,何贏餘之有?」調湖南,護理巡撫。三年,召授刑部侍郎,自此歷十年,未遷他部,專治刑獄,剖析律意,於條例有罅漏,及因時制宜者,數奏請更定。每日部事畢,歸坐一室,謝絕賓客。十二年,擢刑部尚書,典會試。十四年,卒,優詔賜恤,稱其清介自持,克盡職守,贈太子太保,謚簡恪。

敦元博聞強識,目近視,觀書與面相磨,過輒不忘。每至一官,積牘覽一過,他日吏偶誤,輒摘正之,無敢欺者。奏對有所諮詢,援引律例,誦故牘一字無舛誤,宣宗深重之。至老,或問僻事;指某書某卷,百不一爽。嘗曰:「書籍浩如煙海,人生豈能盡閱?天下惟此義理,古今人所談,往往雷同。當世以為獨得者,大抵昔人唾餘。」罕自為文,僅傳詩數卷。喜天文、律算,討論有年,亦未自立一說。卒之日,笥無餘衣,囷無餘粟,庀其貲不及百金,廉潔蓋性成云。

硃士彥,字修承,江蘇寶應人。父彬,績學通經,見儒林傳。士彥承家學。成嘉慶七年一甲三名進士,授編修。纂國史河渠志,諳習河事。大考擢贊善,督湖北學政。累遷侍讀學士,入直上書房。歷少詹事、內閣學士。道光二年,擢兵部侍郎。四年,以南河高堰壞,疏陳河工事宜,論:「高堰石工宜切實估修;堰內二堤宜培補;黃河盛漲,宜兩岸分洩;山盱五壩宜相機開放;黃河下游無堤之處宜接築。」下勘河大臣文孚籌議酌行。尋督浙江學政。奏禁諸生包漕鬧漕,以端士習。御史錢儀吉劾士彥任性,詔嘉士彥能任勞怨;惟斥其父彬就養閱卷,及命題割裂,薄譴之。九年,典會試,督安徽學政,尋擢左都御史,召還京。

十一年,遷工部尚書。是秋,江蘇大水,河、淮、湖同時漲溢,命偕尚書穆彰阿往勘。穆彰阿先回京,遂偕左都御史白鎔察視江蘇、安徽水災賑務。疏言:揚河掣卸石工,及纖堤耳閘,應令工員賠修;又以淮、揚地方官多調署,情形未熟,請飭江寧布政使林則徐、常鎮通海道張岳崧總司江北賑務,從之。尋奏:「續查下河積潦之區,被災尤重,浮開戶口,為辦賑積弊。應令委員查明後,即於本鄉榜示,放賑時,州縣官據委員原查總發一榜,總查抽查,憑以核辦。」又奏:「山盱屬添建滾水石壩,本年啟放過水,現已無從查驗。工員面稱啟放時石底間有沖裂,壩下灰土亦損,請俟水落責修完固。堰、盱兩淮、湖石工掣卸二百餘丈,固限未滿,應令賠修。其石後磚工灰工間有殘缺,應令補築。又盱堰大堤,加幫土工間有蟄低浮松之處,應培補,責成河兵種柳護堤。其已估未辦之高堰頭、二兩堡,未估之智、信兩壩,應即興辦。此項與黃河險要不同,向來保固一年。請嗣後各土堤及運河堤岸,均改保固三年。運河埽工於經歷一年後,再加保固二年,驗明堅整,始準埽汛修防。」「安徽無為州江壩及銅陵縣壩工程緊要,均應借款興修。」並下所司議行。又劾鹽城、宿松、青陽等縣報災遲延遺漏,請懲處;捐賑紳民應給議敘;禁胥吏婪索挑剔:並從之。

十二年,事竣回京。南河於家灣奸民陳端等盜挖官堤,掣動河流,復偕穆彰阿往勘。疏言:「九月初旬,清口出水二尺有餘,高堰長水二丈一尺,勢至危險。其時吳城七堡未開,洪湖吃重。此時既開放,湖水分減。現交冬令,一月後即難興工,湖多積水,風烈堪虞,請加緊趕辦。」尋命復偕侍郎敬徵往勘。十三年,奏於家灣正壩雖合龍,請飭加鑲追壓,以免出險。覆訊挖堤諸犯,治如律。又偕敬徵覆勘河、湖各工,請分別緩急,以次辦理。父憂歸。

十六年,服闋,署吏部尚書,偕尚書耆英赴廣東、江西鞫獄。十七年,授兵部尚書。查勘浙江海塘,遂赴南河驗料垛工程,盤查倉庫。以庫存與卷冊不符,劾河庫道李湘茝,褫職。又赴安徽、河南按事,疏陳常平倉糶買章程,「請各省囚糧遞糧作正開銷,毋動倉穀;平糶必市價在八錢以上始準出糶;採買須俟年豐穀賤,且必在出糶二三年後,以紓民力而袪宿弊」。如議行。十八年,兼管順天府尹事,典會試。調吏部尚書。士彥以綜覈為宣宗所知,奉使按事皆稱旨。尋卒,詔嘉其性情直爽,辦事公正,贈太子太保,賜其四子舉人、副榜貢生有差,謚文定。

何凌漢,字仙槎,湖南道州人。拔貢,考授吏部七品小京官。嘉慶十年一甲三名進士,授編修。大考二等,擢司業。累遷右庶子。典廣東、福建鄉試,留福建學政。令諸生自注誦習何經,據以考校,所取拔貢多樸學。道光六年,授順天府尹。京畿獄訟繁多,自立簿籍,每月按簿催結,無留獄。遷大理寺卿,仍署府尹。在任凡五年,歷左副都御史、工部侍郎。典浙江鄉試,留學政。命偕總督程祖洛按訊山陰、會稽紳幕書役句結舞弊,鞫實,請褫在籍按察使李澐職,餘犯軍流有差。任未滿,調吏部侍郎,召回京,兼管順天府尹事。調戶部,復調吏部,仍兼署戶部侍郎。

御史那斯洪阿條陳地方官有錢糧處分,不準升調,及變通雜稅,下部議。凌漢兼吏、戶兩部,駁之,謂:「理煩治劇,每難其人,若格以因公處分,必至以中平無過者遷就升調。且吏治與催科本非兩事,未有因循良而帑藏空虛者,亦未有因貪濁而倉庫充盈者,是在督撫為缺擇人,不為人擇缺,正不必徒事更張,轉滋窒礙。」又謂:「地方各稅,有落地雜稅,及房屋典當等稅,已極周密;至京師九門外有鋪稅,天津、新疆沿壕鋪面有房租,因系官地、官房也。今欲盡天下之府、、州、縣仿照定稅,則布帛菽粟民生日用所需,市儈將加價而取諸民以輸官,水腳火耗,官又將取之於民;且閉歇無常,稅額難定,有斂怨之名,無裕國之實。」前議遂寢。

十四年,擢左都御史,遷工部尚書,仍兼管府尹如故。累署吏部尚書。十七年,吏部因京察一等人員有先由御史改官者議駁。凌漢以不勝御史,非不勝外任者比,如此苛繩,有妨言路。御史改部之員,例準截取。至京察雖無明文,從前有御史降調保送員外郎者,援以請旨。因面奏現任大員花傑、吳榮光,皆曾由御史改降,遂奏俞允。

十九年,調戶部尚書。四川總督寶興請按糧津貼防邊經費,議駁之,略謂:「川省地丁額徵六十六萬,田賦之輕,甲於天下。現議按糧一兩加津貼二兩,百畝之家,不過出銀三兩,即得百萬兩,小民未必即苦輸將。然較原課幾增兩倍,非藏富於民之義,軍需藉資民力,尤不可率以為常。請於各省秋撥項下借撥百萬兩,以三十萬為初設邊防經費,餘或發商,或置田,所獲息以四萬為常年經費,二萬提還借款,於防邊恤民兩有裨益。」詔允行。是年,典順天鄉試。子紹基亦典試福建,父子同持文柄,時人榮之。二十年,卒,贈太子太保,謚文安。紹基官編修,見文苑傳。

李振祜,字錫名,安徽太湖人。嘉慶六年進士,授內閣中書。典廣西、雲南鄉試,遷宗人府主事。調兵部,遷員外郎,典陜甘鄉試,改御史、給事中。巡視淮安漕務,劾戶部郎中錢學彬系不勝外任之員,違例截取知府,詔譴吏、戶二部堂官,予振祜議敘;又劾都察院京察給事中色成額先經列入六法,自赴公堂辯論,干求改列三等,反覆視若兒戲,都御史被嚴議,色成額仍列有疾。

累遷內閣侍讀學士,督山東學政。應詔密陳山東積弊四事,略曰:「吏事叢脞,莫甚於官民不相安也。詞訟之繁,始由於官吏不辦,今又變而不敢辦。欲結一案,輒慮翻控;欲用一刑,輒慮反噬。鞫案之時,有倚老逞刁者,有恃婦女肆潑者,有當堂憤起者,有抗不畫供者,總由官吏恩信不結於平時,明決不著於臨事,以畏葸之才識,治刁悍之民風,殆於鑿枘不相入矣。案牘壅滯,半由外府不辦事也。各府州案件,動輒提省,委交首府,其中有不必提而輕提者,亦有各府州畏難而稟請提省者。濟南府統轄十六州縣,自治不暇,而舍己耘人,勢必兩廢。各府州畏難之事,輒以一稟提省卸責,轉得遂其取巧偷安之計。且疑難案件,本地聞見較真,遠提至省,則茫無頭緒,必致訟師盤踞省城,遇事挑唆,一事株連數十人,一案壓擱一二載,是欲辦案而轉以延案,欲弭訟而適以滋訟矣。緝捕無策,則盜賊充斥也。東省盜賊,結黨剽掠,處處有V酳;齁吱星浚湫∏遠澦櫫蟺粒環址拭⽁ǎ涫坷嘁喔首魑鴨搖=偃ヂ砼#劾帳輳髂空諾ǎ斂晃飯佟W茉擋兌巰び牘戳餃輾衷擼偈彼托擰I踔潦掄咭運嚦夜儼段郟員訃鬯絞晡恪V菹⼂攘匡噸剩植喚擦凡噸ǎ患任槁竦林停植謊賢ǖ林鎩<嬉宰怨舜Ψ鄭薊涫危笪。趟幻狻<┎噸茲鞜恕G覆磺澹蚩骺漳訊乓病6≈菹卣憂福戮膳慚塚拔食!F潯子伸督淮磺澹勻娜我災潦湃危B葛不清者,比比皆是。官虧而外,更有書虧。查書虧情弊,或串通幕丁,朦混本官;私雕假印,偽造串票。有滿其私橐而遠颺者,有挾制本官而自供不諱者。州縣回護處分,隱忍代認,而奸書遂益以侵蝕為得計。錢糧之弊如此。」疏入,上嘉納之。又劾泰安知府延璐、東昌知府熊方受請,飭交撫臣查察嚴參;又劾東昌知府王果陵辱生員,褫王果職;又察出假印試卷、勾結舞弊之人,奏請懲辦。

道光二年,遷太僕寺少卿。父憂去官,服闋,補順天府丞。歷通政司副使、光祿寺卿、太常寺卿、宗人府丞。十五年,署順天府尹。累遷內閣學士。十八年,授工部侍郎,調吏部,兼署倉場侍郎。二十一年,擢刑部尚書。浙江提督餘步雲海疆僨事,逮問治罪。廷臣猶有為議輕比者,振祜堅持,得伸法。二十八年元旦,加恩年老諸臣,加太子太保。二十九年,因病乞休,許之。三十年,卒,年七十四,謚莊肅。

宗室恩桂,字小山,隸鑲藍旗。道光二年進士,選庶吉士,授編修。九遷至內閣學士,兼副都統。十五年,授盛京工部侍郎,尋召為兵部侍郎,調吏部。因曠文職六班,降內閣學士。歷工部、吏部侍郎,管理國子監事,兼護軍統領、左右翼總兵。十九年,典順天鄉試,偕大理寺卿何汝霖往浙江按學政李國杞被劾事,遂查勘南河、東河料垛,奏劾虛缺浮用者,議譴有差。二十年,充內務府大臣,管理上駟院。議增圓明園丁四百名,命偕尚書賽尚阿督率訓練。

二十一年,授理籓院尚書,兼署左都御史。劾太常寺丞豐伸及查倉御史廣祜不職,並罷之。署步軍統領。奏言:「京城巡捕五營槍兵一千名,不足以資捍衛,增設一千。裁撤藤牌弓箭等兵,改為槍兵;不敷者,於各營兵丁內揀選足額。輪派二百名打靶,操演陣式。」詔議行。二十二年,調禮部尚書,又調吏部,實授步軍統領。上御閱武樓,親閱圓明園兵丁槍操,步式整齊,施放有準,嘉恩桂督率有方,賜花翎。時議節冗費,恩桂先已奏裁上駟院馬六百餘匹。又奏言南苑六圈,請裁其二,並裁各圈及京圈馬二百餘匹。上駟院、司鞍、司轡、蒙古醫生舊支馬乾銀,均減半給,如議行。以兼攝事繁,罷管內務府,二十五年,復之。

恩桂在吏部,嚴杜冒濫。兼步軍統領衙門最久,先後逾十年,綜覈整頓,釐定章程,訓練兵卒,皆有實效,宣宗甚倚之。二十六年,京察,特予議敘。又幸南苑,見草木牲畜蕃盛,嘉恩桂經理得宜,加一秩。迭奉命治倉胥舞弊,及戶部捐納房書吏賄充司員、收受陋規諸獄,並持正不撓法。二十八年,卒於官,上深悼惜,稱其任勞任怨,殫竭血誠,贈太保,賜金治喪,謚文肅。

論曰:姚文田建言切中時弊,戴敦元清介幹事,其風概越流俗矣。硃士彥之治河,何凌漢之掌計,李振祜之執法,並號稱職。恩桂奏績金吾,肅清輦轂,一時稱矯矯焉。


\end{pinyinscope}