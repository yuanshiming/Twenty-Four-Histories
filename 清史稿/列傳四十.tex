\article{列傳四十}

\begin{pinyinscope}
莫洛陳福王之鼎費雅達李興元陳啟泰吳萬福

陳丹赤馬閟葉映榴

莫洛,伊爾根覺羅氏,滿洲正紅旗人,世居呼納赫魯。祖溫察,太祖時來歸。莫洛初授刑部理事官,累遷工部郎中。康熙六年,擢左副都御史。七年,出為山西陜西總督。陜西饑,平涼、臨洮、鞏昌、西安、延安、鳳翔、漢中、興安諸府州多逋賦,有司令現戶均輸,民苦之,奏請蠲免。迭疏清釐加派、火耗諸弊。八年,輔政大臣鼇拜獲譴,法司以莫洛附鼇拜,請逮問,詔以能任事,貸勿治,仍留任。九年,計典,仍以前罪奪職。陜西民籥留,甘肅巡撫劉斗同,提督張勇、柏永馥等疏言莫洛清正,在官有善政,乞留以慰民望。上諭曰:「簡用督撫,原以綏輯地方,愛養百姓。莫洛既能得民,其免處分,供職如故。」俄擢刑部尚書。

十三年,吳三桂等奏請撤籓,上敕廷臣議,皆主勿徙,惟莫洛與米思翰、明珠議撤。三桂反,四川提督鄭蛟麟等叛應之。二月,命莫洛經略陜西,拜武英殿大學士,仍管兵部,賜以敕印,既至,策遣諸軍征四川。時蛟麟兵據廣元百丈關,莫洛遣都統馬一寶、將軍席卜臣赴漢中,副都統科爾寬赴廣元,擊賊。十月,蛟麟將何德成犯寧羌,為官軍所敗,還奔四川,莫洛因遣提督王輔臣駐其地。逾月,蛟麟將彭時亨復據七盤、百丈諸關,劫略陽糧艘,截陸運棧道。

廣元軍缺餉兩月矣,總兵王懷忠所部潰散,而輔臣亦陰懷異志。輔臣故與莫洛有卻,奉檄使隨征,益怏怏,藉口戎備寡,莫洛益以騎兵二千,少之;又以馬疲瘠不任用搖軍心,軍無鬥志。十二月,莫洛至寧羌,兩營相去二里許。先是,上命莫洛統綠營步旅下四川,嗣慮巴蜀道險,令貝勒洞鄂率滿洲騎兵兼程繼進。軍未至,是月庚寅朔,癸巳,輔臣煽所部噪餉,襲劫莫洛。莫洛督兵擊卻之。甫定,輔臣復率悍黨至,砲矢雨坌,莫洛被創,卒於軍。恤典久未行,二十二年,命予祭葬,謚忠愍,授世職拜他喇布勒哈番兼拖沙喇哈番。子常安,襲。

陳福,字箕演,陜西榆林人。國初師定陜西,福以武舉應募,從寧夏總兵劉芳名剿寇。敘功,授守備。又從都統李國翰下四川,遷遵義游擊。康熙初,從總督李國英討李自成遺黨郝搖旗、李來亨等。敘功,加右都督銜,擢成都副將,遷重慶總兵。十二年,入覲,授寧夏總兵。

吳三桂反,鄭蛟麟以四川叛應之,遣使誘福。福家留重慶,弟奇官守備,妻子亦在賊中,賊以是劫福。福執其使,具疏入告,遣其弟諸生壽齎詣京師。上嘉福忠,授拜他喇布勒哈番,並官壽主事。輔臣據平涼,福上戰守方略。十四年,擢陜西提督,進三等阿思哈尼哈番,又官奇參將。進規花馬池,惠安、安定、定邊諸城堡,以次皆下。上擢壽鴻臚寺少卿。福率副將泰必圖乘勝薄固原,圍之匝月。輔臣遣其將來援,城賊亦突出,泰必圖戰死。福引兵還靈州,斬逃將賈從哲、張元經以徇。

上命福佐貝勒洞鄂攻平涼。福疏言固原有賊萬餘,若我兵徑趨平涼,慮賊斷我餉道,當先取固原,上韙之。十二月,福督兵取固原。天寒大雪,士卒苦遠役,且懲前敗,有戒心。是月甲寅朔,乙亥,師次惠安,下令:「五鼓會食,集城下,後者斬。」夜半時,參將熊虎等鼓譟入,刺福死。上以趙良棟代,收虎及首謀把總劉德及營兵戕福者悉誅之,贈福三等公,以三等精奇尼哈番世襲,謚忠愍。建祠寧夏。擢奇天津總兵。時福子世琳、世勛並陷賊,命以壽子世怡襲爵。

事定,壽棄官入四川求福妻子,得之遵義山中,將入都,上召世琳入見,問母子流離狀,深愍之,命襲爵,改籍寧夏。旋授直隸三屯協副將。累遷古北鎮總兵、鑾儀使。世琳子益,益子大用,相繼襲爵。益官至楚姚鎮總兵。大用乾隆間官江南提督,所屬游擊楊天相,獲海盜,總督蘇凌阿讞以為誣,誅天相,大用亦被譴。嘉慶初,予守備銜,休致。

王之鼎,字公定,漢軍正紅旗人。父世選,仕明為參將。歸太宗,授三等昂邦章京。從世祖入關,征江南有功,進二等。卒,之鼎襲,進一等精奇尼哈番,署參領。從貝勒屯齊征湖南,擊走李定國、孫可望。授正紅旗漢軍副都統。駐防貴州。康熙元年,授福建中路總兵,討鄭錦,克廈門、金門、古浪諸島。三年,敗錦將黃盛、林茂、裴德等,拔銅山衛,進三等伯。八年,召還,仍授本旗副都統。十年,授江南提督。

十二年,授鎮海將軍,駐守京口。吳三桂、耿精忠相繼反。上命之鼎分兵防安慶,而以安南將軍華善帥師佐之。之鼎調崇明沙船,江陰、瓜洲戰艦,扼津要,令綠旗水師駐黃浦操防,兼備水陸。上命簡親王喇布為揚威大將軍,駐江寧,之鼎贊軍事。十七年,改福建水師提督,加定海將軍。閩寇日蹙,而楚、蜀間軍事方亟,請移鎮要地自效。

十八年,調四川提督。十九年,到官,會寇犯永寧,遂率總兵李芳述等討之,戰屢勝。六月,勇略將軍趙良棟將進剿雲、貴,調芳述守敘州,之鼎留鎮永寧。九月,吳世璠遣將尤廷玉、胡國柱攻永寧,圍之數匝。時城中糧盡已兩閱月,之鼎猶率兵挑戰,士氣倍奮。嗣為賊詗知,築長圍以守。至是月甲子,寇穴城入,總兵費雅達,副將楊三虎,游擊周尚功,守備李逢春、魯明芝、席豹督兵巷戰,皆死。之鼎解印付家人,令間道走成都,率總兵楊魁、何成德、王永世、傅汝友,游擊陳先鳳、陳田、劉應科等出禦賊,身受重創。賊湧至,之鼎自度不能免,拔劍自刎,未殊,與魁等俱被執,傳送貴陽。賊黨夏國相等百計誘降,之鼎厲聲叱曰:「死則死耳,肯向鼠輩乞活耶?」久之,賊知不可奪,遂遇害。魁等皆不屈死。事聞,贈之鼎太子少保,謚忠毅。子毓賢,官至貴州布政使,毓秀襲爵。

費雅達,漢軍正白旗人。自整儀尉累遷潼關副將。王輔臣叛,廷議設漢中總兵討賊,以授費雅達,署都督僉事。進取漢中,破賊彞門鎮,抵秦嶺,拔北木城,與王進寶會師奪武關。敘功,加都督同知。永寧之役,城陷身死,贈左都督、太子少傅,謚忠勇。魁等皆予恤。

從三虎等戰死者,又有千總蔣得福、趙鳴鳳、王英傑;從之鼎死者,又有從軍廕生潘濟世:並恤如例。

李興元,字若始,漢軍鑲黃旗人。以拔貢授直隸沙河知縣,報最,遷祁州。歷江西吉安、直隸永平知府,晉陜西隴右道。康熙十一年,授雲南按察使。其明年,詔敕有司審理平西籓下逃兵。時平西勛莊棋布,管莊者殺人奪貨,滋為民患。訟牒命、盜兩案,兵居半。又勒平民為餘丁;不從,則曰:「是我逃兵也。」稱貸重息,人或絲毫負,亦以「逃兵」誣之,有司亡誰何。興元素持風力,諗知劉昆強項,令為審事官。有犯者論如法,部民德之,而大忤三桂意。

三桂將叛,使冶者鑄印,昆詗知,白興元,興元啟巡撫硃國治,趣入告。國治遲數日始發,為三桂邏卒所得,遂作亂。召各官集議,以國治苛虐失民心,殺之;迫授興元偽職,興元叱之曰:「汝內為國戚,外封親王,受恩重矣,何叛為?我為丈夫,義可殺不可辱,惟一死以報朝廷。」三桂怒,杖而下之獄。雲南知府高顯辰及昆皆不屈,旋以興元及昆戍騰越衛。十八年,師克湖南,時三桂已死,其子世璠使刺殺興元。師困滇城,興元二子廕秀、奇秀亦被殺。

事定,其三子萃秀詣軍所申訴,巡撫王繼文上其狀,贈太常寺卿。萃秀官至安陸府知府。昆當興元未死,出避民間。事定,復補登州同知,遷常德知府。

陳啟泰,字大來,漢軍鑲紅旗人。順治四年,自貢生知直隸滑縣,有聲。行取,擢御史。奏言:「滿洲部院官凡遇親喪,宜離任守制,以廣孝治。」從之。十一年,出為蘇松糧道。康熙三年,調福建漳南道。八年,轉巡海道。時山寇遍受耿精忠劄,勢洶洶。啟泰嚴保甲,立團長,親督所司捕賊。有干禁令者,輒痛繩以法,奸宄屏息。

十三年,精忠叛,偽檄至漳州。啟泰密與海澄公黃梧議拒守,會梧病,精忠復招鄭錦為助。啟泰自度不能守,語妻劉曰:「義不偷生,忠不附賊,死吾事也。然死而妻子為僇,吾何以瞑?」劉請殉,家人皆原從死。乃以巨盎置酒下藥,劉及侍妾婢僕飲者二十一人。幼子方六歲,持觴拜而飲。啟泰朝服坐堂皇,召僚屬與訣,引弓弦自絞死,僚屬為殯。錦兵入,見置棺縱橫,皆垂淚。事聞,贈通政使,賜葬祭。

啟泰子汝器,聞變,赴漳州迎喪,為鄭錦兵掠去。逾二年乃脫還,詣京師,上念其父子忠義,加贈工部侍郎,授汝器右通政。三十三年,復予啟泰謚忠毅。建祠福州,御書「忠義流芳」為祠額。汝器官至安徽巡撫。方精忠叛時,諸郡望風納叛;所不肯以城降者,啟泰死漳州,總兵吳萬福死福寧。

萬福,漢軍鑲紅旗人。初仕明為守備。崇德七年,師圍松山,從副將夏承德來歸,授牛錄額真。入關,從征李自成有功,累敘二等阿達哈哈番。出為福寧總兵。張煌言兵屢入,與總兵李長榮分路擊卻之,累進右都督,精忠叛,萬福嬰城固守,城破,死之,闔家被害。幕客孫墢、百總潘騰鳳並殉。事聞,贈萬福左都督、太子少保,謚忠愍。

陳丹赤,字獻之,福建侯官人。順治十七年舉人,選授重慶推官,攝府事,兼署夔州府。時張獻忠初滅,蜀東尚淪於賊,徵師四集,丹赤給餉不乏。墾荒萊,緩刑禁,報最,遷刑部主事,再晉兵部郎中。出為浙江按察司僉事、分巡溫處道,署按察使。

康熙十三年,入覲,道山東。會吳三桂反,詔入覲官悉還治所。丹赤歸至東昌,聞耿精忠亦叛,亟間道還。適平陽叛將司定猷構精忠兵偪瑞安,丹赤獨居城上,泣諭父老,誓與城存亡。海寇硃飛熊乘間肆掠,鄉民爭入城,總兵祖弘勛欲不納。丹赤曰:「城以人為固,人以食為命。民輦粟入城,民即兵,食即餉。亟宜納之,與共守。」於是來者數萬。寇湧至,攻南門甚亟,副將楊春芳忽撤兵去,人心洶懼。丹赤日馳牒乞援,晨夜徼循,以忠義厲士卒,皆感泣,願死守。

弘勛將以溫州叛,陽遣游擊馬文始助守,實以詗丹赤,丹赤誓以身殉。六月甲午朔,弘勛陳甲仗華蓋山,集文武官計事,欲以脅丹赤。千總姚紹英知其謀,勸勿往,丹赤不顧,策馬去。至則兵露刃夾階立,坐定,弘勛曰:「彼眾我寡,將若何?」丹赤曰:「提標前鋒五千人已集,且民心效死,戰即不足,守自有餘。吾此來商以舟濟師,顧乃計多寡邪?」弘勛曰:「舟安在?」丹赤語通判白鼇宸曰:「河干泊舟不少,皆鄉民所棄。以濟援師,何患無舟?」弘勛語塞,春芳厲聲言曰:「城中糧盡,縱有兵有舟,誰為我用?」丹赤曰:「若言誤矣。吾軍糧餉足供六閱月,且遠近鄉民輸粟入。若乃為此言惑軍心邪?」有自懷中出帛書者,精忠招弘勛獻城檄也,丹赤怒,碎而擲之地,曰:「此豈可污吾目?吾頭可斷,城不可得也!」弘勛執其手,好語慰之,丹赤曰:「封疆之臣死封疆,不知其他。」弘勛知不可奪,目千總高魁持斧擁丹赤出,罵益厲,執斧者斷其臂,大呼曰:「臣事畢矣!」兵刃交下,遂遇害。十六年,浙江巡撫陳秉直疏請恤,贈通政使,謚忠毅。三十八年,上南巡,丹赤子一夔時為湖州知府,迎謁,上書額賜之。

馬閟,字奉璋,陜西武功人。順治十一年舉人,授山東昌樂知縣,有惠政。康熙十三年,補永嘉。明決有才,清覈圖籍,不數月而政成。華蓋山集議,弘勛戕丹赤,閟躍而起曰:「國家豢若輩,反黨賊殺封疆大吏,吾恥與若輩俱生!」遂罵不絕口,同時遇害。事聞,贈布政司參政。三十五年,敕建祠溫州,祀丹赤及閟,亦曰「雙忠」。四十二年,上南巡,閟子逸姿官江南布政司參議督糧道,迎謁,疏引丹赤例求賜謚,上允之,謚忠勤,亦賜御書額如一夔。丹赤役林莪、僕張亦寶,閟從子穎姿,皆從死。

葉映榴,字炳霞,江南上海人。順治十八年進士,選庶吉士。時方嚴治江南逋賦士紳,映榴在籍中,降國子監博士。累遷禮部郎中。出榷贛關,會吳三桂叛,贛南北路絕。映榴與同官守險要,撫流民,境獲寧。提學陜西巡撫鄂愷薦其才,康熙二十四年,授湖廣糧儲道。清積逋,減耗羨,事有不便於民者,輒與大吏力爭。

二十七年五月,廷議省湖廣總督,並裁督標兵。楚兵素剽悍,有夏逢龍者,尤桀黠,能以小信義結其伍,隱附之。檄既下,裁兵洶洶亡所歸。總督徐國相還朝已登舟,眾圍訴索餉,不得,遂大譁。時巡撫柯永升初上官,映榴攝布政使才三日。事急,映榴白永升,請予兩月糧遣散,不許。眾入巡撫署,露刃呼譟。映榴復白永升,請好言慰遣之。永升出,眾語不遜,永升曰:「若輩欲反邪?」眾曰:「反也奈何?」刃傷永升臂,奪其印,復刃傷足,僕,遂擁映榴至閱馬場。永升得間自經死。逢龍自號「總統兵馬大元帥」,幟以白,迫布政使以下官受偽職,映榴紿以無殺掠,三日後徐議之。乃令其妻陳奉母吳自水溝出,解印付其僕,乃手具遺疏。是月丁酉,朝服升公座,罵賊,拔佩刀自刎死。

疏略曰:「臣一介豎儒,叨沐皇上高厚深恩,歷擢今職。嘗以潔己奉公,自矢夙夜,但媿才具庸劣,未效寸長。茲值裁兵夏逢龍倡亂,劫奪撫臣敕印,分兵圍臣衙門,露刃逼脅。臣幼讀詩書,粗知節義,雖斧鑕在前,豈肯喪恥偷生?臣母年七十有六,在臣任所;臣長子旉,遠在原籍;其餘二子尚未成童,煢煢孤嫠,死將安歸?因遣妻女奉母潛逃。臣如微服匿影,或可幸免以圖後效。伏念臣守土之官也,城存與存,城亡與亡,義所當然。今勉盡一死,以報國恩。所恨事起倉猝,既不能先事綢繆,默消反側;復不能臨期捍禦,獨守孤城。上辜三十載之皇恩,下棄七旬餘之老母,君親兩負,死有餘慚。」上覽疏,深愍傷之,召廷臣展讀,聞者皆感泣。下部議恤,部議援陳丹赤例,贈通政使,特旨贈工部侍郎。次年上南巡,旉迎謁,手書「忠節」二字賜之,遂以為謚。立祠武昌,書「丹心炳冊」扁以賜。

雍正八年,錄忠臣後,授其子旉鳳陽知府;芳蔚州知州,尋改員外郎;孫鳳毛內閣中書。與映榴同時死者,都司宣德仁,贈副將。

論曰:功令褒死事,倉卒遇變與艱難效死者同,所以獎忠義也。莫洛與福,先事宜知有變,師行有進無退,雖死不撓。之鼎效忠於孤城,興元抗節於大憝。若啟泰、丹赤、映榴,皆能死其官者。啟泰以其家殉,與馬雄鎮比烈;映榴遺疏款款,則又範承謨蒙穀自序之亞也。


\end{pinyinscope}