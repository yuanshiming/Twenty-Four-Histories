\article{列傳一百八十三}

\begin{pinyinscope}
吳文鎔潘鐸鄧爾恆

吳文鎔,字甄甫,江蘇儀徵人。嘉慶二十四年進士,選庶吉士,授編修。屢膺文衡,稱得士。六遷為翰林院侍讀學士。督順天學政,剔弊清嚴,在任累擢詹事、內閣學士。召回京,署禮部侍郎,尋實授。調刑部,兼署戶部侍郎。迭命偕大學士湯金釗赴安徽、浙江、江蘇及南河按事。道光十九年,出為福建巡撫,時方嚴煙禁,英吉利窺伺沿海,偕總督鄧廷楨籌防,敵兵至,不得逞,二十年,調湖北巡撫,未行,暫護閩浙總督。明年,入覲,改江西巡撫。值歲祲,力籌撫血⼙,裁減漕丁陋規。在江西數年,舉廉懲貪,吏治清明。捕教匪戴理劍等,及南安、贛州會匪,並置諸法。

二十八年,調浙江巡撫。入境過衢州,廉得游擊薛思齊貪劣,劾戍新疆;又劾不職縣令五人。因官多調攝,徒煩交代,政無考成,奏革其弊,風氣為之一變。以覈辦清查,本省官吏不可信,請簡派戶部司員來佐理,詔不許。未幾,命偕侍郎季芝昌清查浙江鹽務,奏籌變通章程以專責成,除浮費為要務,鹽課日有起色。浙東漁山島為盜藪,檄水師捕獲百餘人,毀其巢。二十九年,大水,文鎔以遇災恐懼,上疏自劾請罷,詔以其言近迂,嚴斥之。文鎔親赴嘉、湖諸屬察災輕重,力行賑撫。秀水令江忠源勤廉稱最,治賑治盜及塘工皆倚辦,以憂去。文鎔嘆曰:「賢如江令,可令其無以歸葬乎?」自支養廉五百兩畀之,奏辦賑功,以忠源首列。三十年,海塘連決,文鎔馳勘,落水幾殆,自劾疏防,革職留任。塘工竣,復職。

擢雲貴總督。咸豐元年,入覲,文宗甚重之,嘉其忠誠勇於任事,勖以察情偽,惜身體,文鎔益感奮。永昌邊外夷匪肆掠,久不靖,文鎔至,檄土守備左大雄深入搜捕,擒斬數百,匪遁雪山外。粵匪日熾,文鎔疏論提督向榮冒功託病,恐誤軍事,詔選將才,奏保游擊巴揚阿等九人。貴州黎平知府胡林翼治團練剿土匪,令得便宜從事,疏薦之。江忠源在廣西軍中,文鎔致書曰:「永安賊不滅,若竄湖南,不可制矣!」二年,調閩浙總督,未行,而粵匪果由湖南北竄,破武昌。三年春,遂踞江寧,東南大震。雲南永昌回匪亦蠢動,文鎔調兵扼險,親駐尋甸督剿。

尋調湖廣總督。粵匪方自下游上竄,連陷黃州、漢陽。文鎔九月抵任,是日田家鎮諸軍失利,武昌戒嚴,城晝閉,居民一夕數驚。巡撫崇綸欲移營城外為自脫計,文鎔誓與城存亡,約死守待援,議不合。賊已逼城,文鎔坐城上激厲將士,守數旬,圍解。崇綸轉以閉城坐守奏劾,詔促進復黃州。文鎔方調胡林翼率黔勇來會剿,又約曾國籓水師夾攻,擬俟兩軍至大舉滅賊。崇綸屢齕之,趣戰益急。文鎔憤甚,曰:「吾受國恩厚,豈惜死?以將卒宜選練,且冀黔、湘軍至,收夾擊之效。今不及待矣!」四年正月,督師進薄黃州,屯堵城。大雪,日行泥淖,拊循士卒,而輜糧不時至。賊分路來犯,都司劉富成擊卻之。賊復大至,文鎔揮軍力戰,後營火起,眾潰,投塘水死之。崇綸奏稱失蹤,署總督臺湧至,乃得實以聞。詔依總督陣亡例賜恤,予騎都尉兼雲騎尉世職,謚文節,祀京師昭忠祠。

逾數月,曾國籓進兵黃州,訪詢居民,備言戰歿狀,皆流涕。於是疏陳當時無水師,不能制賊。文鎔籌置之難,為崇綸傾陷牽掣,以至於敗;且諱死狀,欲以誣之。文宗震怒,逮崇綸治罪,文鎔志節乃大白。同治中,湖北請建專祠。

潘鐸,字木君,江蘇江寧人。道光十二年進士,選庶吉士,散館改兵部主事,充軍機章京。洊升郎中,遷御史。二十年,出為湖北荊州知府,擢江西督糧道。歷廣東鹽運使、四川按察使、山西布政使,署巡撫。

二十八年,擢河南巡撫。時議漕糧酌改折色,鐸疏言:「戶部有南漕折價交河南等省採買之議,是他省且須在河南採買。若將本省額徵之米分別改徵折色,於政體兩歧,於倉儲有損無益。河南歷年辦運踴躍,一經改徵,轉滋流弊,循舊章為便。」議遂寢。賈魯河經祥符硃仙鎮,為商賈舟楫所集。自黃河決於中牟,賈魯河淤塞,責工員賠濬,久未復。鐸勘鎮街南北淤最甚,議大濬,請率屬捐銀五萬兩興辦;又奏擇要增培沁河民堤以資捍禦:並如所請行。咸豐元年,坐所薦陳州知府黃慶安犯贓,降二級調用,授山西按察使。二年,遷湖南布政使。粵匪方由湖南北竄,漢陽、武昌相繼陷,巡撫張亮基擢署總督,以鐸暫代之,命赴岳州督防。三年,巡撫駱秉章至,乃以病乞罷,許之。直隸總督訥爾經額疏薦,詔赴山西會辦防剿事宜。尋因前在湖南布政使任內岳州等城失守,下部議,俟補官日降二級調用。復以病乞退,居山西久之。

十一年,予二品頂戴,起署云貴總督。雲南回、漢相仇,稔亂已久。巡撫徐之銘傾險,挾回自重,總督張亮基為所齮齕去。布政使鄧爾恆擢陜西巡撫,行至曲靖,之銘嗾副將何有保遣黨戕害,以盜殺聞,命鐸往治之。亮基亦被命赴滇督辦軍務。時之銘已為回眾所挾持,所陳奏多誇誕,莫可究詰。鐸、亮基先後取道四川,與駱秉章籌商,冀資其兵力以規進取。四川亂亦未平,遽不得要領。滇將林自清為亮基舊部,與回眾不協,率所部入川。之銘慮亮基至於己不利,嗾回眾揚言拒之,亮基益觀望。鐸秉性忠正,詔屢敦促,命赴貴州按事,遂由黔入滇,僅從僕數人。在途或以危詞相怵,不之顧。

同治元年九月,抵任,治鄧爾恆被戕之獄。何有保已前死,捕兇犯誅之。見撫局初定,省城稍安,屢密疏陳:「徐之銘尚能撫回,被劾各款,請俟張亮基到後會同查辦。」又云:「馬如龍求撫出於誠心,岑毓英鯁直有戰功,加以閱歷,乃有用之材。」鐸意欲因勢利導,徐圖補救。於是詔亮基移署貴州巡撫,滇事專責鐸與之銘,蓋羈縻之也。回人掌教馬德新,之銘所諂事。初見鐸貌為恭順,後漸跋扈。武職多越級僭用翎頂,之銘所擅賞,鐸面斥之。元新營參將梁士美乃臨安土豪,不與回教聯和。馬如龍誓欲剿滅,鐸不可,強出師,與岑毓英同敗歸,欲添調兵練,鐸復阻之。回紳田慶餘議設公局,通省糧賦稅釐悉歸之,文武職官亦由公舉,鐸以非政體斥止,由是馬如龍等皆不悅。

馬榮者,迤西回酋杜文秀之黨,之銘檄署武定營參將。二年正月,榮忽率二千人至省城,踞五華書院,鐸令出,遷延三日,乃親往諭遣,榮抗恣不聽,其所部回練遽攢刺,鐸臨殞罵不絕口。雲南知府黃培林、昆明知縣翟怡曾同被害。榮遂縱兵大掠,官衙民居悉遍。惟岑毓英勒兵守籓署,之銘遁往潛匿。越兩日,毓英始殮鐸尸。回眾擁馬德新為總督。馬如龍在臨安,聞警馳至,馬榮已率眾攜所掠散去。如龍殺餘匪數十人及附亂者百餘,謂馬德新不當為總督,取關防授之銘兼署。之銘以巡撫讓如龍,如龍不受,遂令署提督,一切拱手聽之。事聞,詔嘉鐸「萬里赴滇,不避艱險,見危授命,大節懍然」。依總督陣亡例賜恤,贈太子太保,予騎都尉兼雲騎尉世職,入祀云南昭忠祠,謚忠毅。子四人,並錄授京職。

當鐸之親諭馬榮也,約之銘同往,竟不至。事定,疏奏諉為杜文秀勾結武定匪犯省城,又諱匿馬榮委署參將事。論者謂榮之為亂,之銘實與知之。於是褫之銘職,聽候治罪。授勞崇光總督,賈洪詔巡撫,皆不能至。雲南軍事分隸於馬如龍、岑毓英,崇光駐貴陽遙制之,至五年,始入滇履任。馬榮已先為如龍等剿除,之銘亦死,迄未就逮雲。

鄧爾恆,字子久,江蘇江寧人,總督廷楨子。道光十三年進士,選庶吉士,授編修。出為湖南辰州府知府。父憂,服闋,補雲南曲靖府。平尋甸叛回馬二花,彌勒土匪吳美、硃順,招撫昆陽回匪,甚有聲績,擢鹽法道,累遷按察使、布政使。咸豐十一年,擢貴州巡撫,未行,調陜西。徐之銘袒回,營將多與通。副將何有保者,之銘私人,尤不法。慮爾恆入覲發其罪,諷有保害之以滅口。爾心互行次曲靖,宿於知府署。有保使其黨史榮、戴玉棠偽為盜,戕之,掠其行橐。有保索所劫物不得,執拷二人。玉棠潛逸,糾黨攻殺有保。鐸至,擒二人誅之。詔爾恆依陣亡例賜恤,予騎都尉世職,謚文愨。

論曰:吳文鎔由卿貳出膺疆寄,凡十餘年,風採嚴峻,時推其治行亞於林則徐。潘鐸亦負端人之望。二人者晚任艱危,並受事於岌岌之日,守正不阿,盡瘁完節,不可復以成敗苛論矣。其死也,皆由同官所構陷。國家於巖疆要地,督撫同駐,豈非以資鈐制,備不虞哉!然推諉牽掣,因之而生;甚且傾軋成釁,貽禍封疆。楚、滇覆轍,蓋其昭著者也。至光緒中,其制始改焉。


\end{pinyinscope}