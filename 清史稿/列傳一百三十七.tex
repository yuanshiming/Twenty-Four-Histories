\article{列傳一百三十七}

\begin{pinyinscope}
李長庚子廷鈺胡振聲王得祿邱良功陳步雲

許松年黃標林國良許廷桂

李長庚,字西巖,福建同安人。乾隆三十六年武進士,授藍翎侍衛。出為浙江衢州營都司,累遷樂清協副將。五十二年,署福建海壇鎮總兵。鄰海有盜,誤指所轄界,坐褫職。罄家財募鄉勇,捕獲巨盜,起用,補海壇游擊,遷銅山參將。自乾隆季年,安南內亂,招瀕海亡命劫內洋,以濟餉為患,粵東土盜鳳尾、水澳兩幫附之,遂益肆擾。五十九年,夷艇始犯福建三澎,長庚擊走之。

嘉慶二年,遷澎湖協副將,擢浙江定海鎮總兵。三年,迭擊洋匪於衢港及普陀。四年,鳳尾幫引夷艇入溫州洋,敗之,賜花翎。五年夏,夷艇合水澳、鳳尾百餘艘萃於浙洋,逼臺州。巡撫阮元奏以長庚總統三鎮水師擊之,會師海門。賊泊松門山下相持,颶風大作,覆溺幾盡,其泊岸及附敗舟者皆就俘,獲安南偽侯倫貴利等四總兵,磔之,以敕印擲還其國。是年,擢福建水師提督,尋調浙江。安南烏艚船百餘號,總兵十二人,分前中後三隊,所獲四總兵,其後隊也。

未幾,安南新阮內附,受封守約束,艇匪無所巢穴。其在閩者,皆為漳盜蔡牽所並,有艇百餘,粵盜硃濆亦得數十艘。牽,同安人,奸猾善用眾,既得夷艇,凡水澳、鳳尾諸黨悉歸之,遂猖獗。阮元與長庚議夷艇高大,水師戰艦不能制,乃集捐十餘萬金付長庚,赴閩造大艦三十,名曰霆船,鑄大砲四百餘配之。連敗牽等於海上,軍威大振。

八年,牽竄定海,進香普陀山,長庚掩至,牽僅以身免,窮追至閩洋,賊船糧盡帆壞,偽乞降於總督玉德,遣興泉永道慶徠赴三沙招撫,玉德遽檄浙師收港,牽得以其間修船揚帆去。浙師追擊於三沙及溫州,毀其船六。牽畏霆船,賄閩商造大艇,高於霆船,出洋以被劫報,牽得之,渡橫洋,劫臺灣米以餉硃濆,遂與之合。

九年夏,連宗八十餘入閩,戕總兵胡振聲,詔治閩將不援罪,長庚總統兩省水師。秋,牽、濆共犯浙,長庚合諸鎮兵擊之於定海北洋,沖賊為二,自當牽,急擊,逐至盡山。牽以大艇得遁,委敗硃濆,濆怒,於是復分。十年夏,調福建提督。牽聞長庚至,遂竄浙,追敗之青龍港,又敗之於臺州斗米洋。復調浙江提督。

十一年正月,牽合百餘艘犯臺灣,結土匪萬餘攻府城,自號鎮海王,沉舟鹿耳門阻援兵。長庚至,不得入,諜知南汕、北汕、大港門可通小舟,遣總兵許松年、副將王得祿繞道入,攻洲仔尾,連敗之。二月,松年登洲仔尾,焚其藔,牽反救,長庚遣兵出南汕,與松年夾擊,大敗之。牽無去路,困守北汕。會風潮驟漲,沉舟漂起,乃奪鹿耳門逸去,詔奪花翎、頂戴。四月,蔡牽、硃濆同犯福寧外洋,擊敗之,追至臺州斗米洋,擒其黨李按等。

長庚疏言:「蔡逆未能殲擒者,實由兵船不得力,接濟未斷絕所致。臣所乘之船,較各鎮為最大,及逼近牽船,尚低五六尺。曾與三鎮總兵原預支養廉,捐造大船十五號,而督臣以造船需數月之久,借帑四五萬之多,不肯具奏。且海賊無兩年不修之船,亦無一年不壞之槓料。桅柁折則船為虛器,風篷爛則寸步難行。乃逆賊在鹿耳門竄出,僅餘船三十,篷朽硝缺;一回閩地,裝篷燂洗,煥然一新,糧藥充足,賊何日可滅?」詔逮治玉德,以阿林保代。既至福建,諸文武吏以未協剿、未斷岸奸接濟、懼得罪,交譖長庚。阿林保密劾其逗留,章三上,詔密詢浙江巡撫清安泰。清安泰疏言:「長庚熟海島形勢、風雲沙線,每戰自持柁,老於操舟者不及。兩年在軍,過門不入。以捐造船械,傾其家貲。所俘獲盡以賞功,士爭效死。八月中戰漁山,圍攻蔡逆,火器瓦石雨下,身受多創,將士傷百四十人,鏖戰不退。賊中語:『不畏千萬兵,只畏李長庚。』實水師諸將之冠。」且備陳海戰之難,非兩省合力不能成功狀。時同戰諸鎮,亦交章言長庚實非逗留。仁宗震怒,切責阿林保,謂:「朕若輕信其言,豈不自失良將?嗣後剿賊專倚長庚,儻阿林保從中掣肘,玉德即前車之鑒!」並飭造大同安梭船三十,未成以前,先雇商船備剿。長庚聞之,益感奮。是年秋,擊賊於漁山,受傷,事聞,復還翎頂。

十二年春,擊敗牽於粵洋大星嶼。十一月,又擊敗於閩洋浮鷹山。十二月,遂偕福建提督張見升追牽入澳,窮其所向,至黑水洋。牽僅存三艇,皆百戰之寇,以死拒。長庚自以火攻船掛其艇尾,欲躍登,忽砲中喉,移時而殞。時戰艦數十倍於賊,見升庸懦,遙見總統船亂,遽退,牽乃遁入安南外洋。上震悼,褒血⼙,初擬俟寇平錫以伯爵,乃追封三等壯烈伯,謚忠毅,於原籍建專祠。

長庚治軍嚴,信賞必罰,自偏裨下至隊長水手,耳目心志如一,人人皆可用。與阮元同心整厲水師,數建功,為玉德所忌。及阿林保之至閩也,置酒款長庚,謂曰:「大海捕魚,何時入網?海外事無左證,公但斬一酋,以牽首報,我飛章告捷,以餘賊歸善後辦理。公受上賞,我亦邀次功,孰與窮年冒風濤僥幸萬一哉?」長庚謝曰:「吾何能為此?久視海船如廬舍,誓與賊同死,不與同生!」阿林保不懌。既屢劾不得逞,則飛檄趣戰。長庚緘所落齒寄其妻,志以身殉國。既歿,詔部將王得祿、邱良功嗣任,勉以同心敵愾,為長庚雪仇。二人遵其部勒,卒滅蔡牽,竟全功焉。

長庚無子,養同姓子廷鈺為嗣,襲伯爵,授二等侍衛。道光中,出為南昌副將,累擢浙江提督。因病不能巡洋,奪職家居。咸豐初,治本籍團練,迭克廈門、金島、仙游,授福建提督。尋以誤報軍情解任,仍會辦團練。十一年,卒,孫經寶襲爵。

胡振聲,亦同安人,提督貴子。起行伍,累擢至溫州鎮總兵。從長庚大破夷艇於臺州松門洋,自是屢從長庚擊賊海上。嘉慶九年六月,率二十六艘運舟材赴福建,至浮鷹洋,遇賊,與總兵孫大剛夾攻,殲賊甚眾,而舟為砲焚,閩師不能救,遂被害。優恤,謚武壯,予騎都尉兼雲騎尉世職。

王得祿,字玉峰,福建嘉義人。林爽文倡亂,陷縣城。得祿家素豐,捐貲募鄉勇,助官軍復之,授把總。明年,賊復圍城,從總兵柴大紀固守。及圍解,率鄉勇搜捕大坪頂等處餘匪,焚瑯嶠賊巢,賊渠莊大田就擒。臺灣平,賜花翎、五品頂戴,遷千總。嘉慶元年,巡洋至獺窟,遇賊,得祿先登,擒吳興信等。歷年出洋捕海盜,號勇敢,累擢金門營游擊。七年,從李長庚擊蔡牽於東滬洋,擒賊目徐業等百餘人,又擒呂送於崇武洋,被獎敘。九年,從總兵羅仁太擊賊於虎頭山洋面,獲船械甚多。十年,擊蔡牽於虎井洋,敗之,署澎湖協副將。九月,遇牽於水澳,焚其舟,擒殲硃列等百餘人。十一年春,牽入臺灣,圍府城。李長庚令得祿與許松年駕小舟自安平港入偵之,帆檣彌望,夜縱火焚賊舟,遂入屯柴頭港。明日,賊自洲仔尾攻府城北門,得祿率兵躡其後,大呼以前,賊驚卻。城內軍出夾攻,大敗之,乘勝至洲仔尾,破其營,賊乃遁。五月,牽復竄鹿耳門,得祿首先沖擊,獲船十,沈船十一。敘功,加總兵銜。尋擢福寧鎮總兵。

十二年,調南澳鎮。七月,敗硃濆於雞籠洋,獲船十四。十一月,又敗其黨於古雷洋,射殪賊目硃金,擒張祈,被獎敘。未幾,李長庚戰歿,命得祿與邱良功繼任軍事。十三年,擢浙江提督。既而調福建,邱良功代之。時阮元再任浙江巡撫,張師誠為福建巡撫,兩省合力,得祿與良功同心滅賊。十四年八月,同擊蔡牽於定海漁山,敗之。牽東南走,追至黑水洋,合擊累日,良功以浙舟駢列賊舟東,得祿率閩舟列浙舟東,戰酣,良功舟傷暫退,得祿舟進,附牽舟,諸賊黨隔不得援。牽鉛丸盡,以番銀代,得祿額腕皆傷,擲火焚牽舟尾樓,復沖斷其柁。牽知不免,舉砲自裂其舟沉於海。詔以牽肆逆十有四年,渠魁就殲,厥功甚偉,錫封得祿二等子爵,賜雙眼花翎。餘黨千二百人,後皆降,海盜遂息。

得祿為福建提督歷十載,屢疏陳緝捕事宜,改定水師船制,皆如議行。二十五年,調浙江提督。道光元年,乞病歸。十二年,臺灣張丙作亂,得祿率家屬擒賊目張紅頭等,加太子少保。十八年,臺匪沈和肆掠,輸糧助守,晉太子太保。二十一年,英吉利犯廈門,命駐守澎湖。次年,卒,贈伯爵,謚果毅。次子朝綸襲子爵,官戶部員外郎。

邱良功,福建同安人。起行伍,屢以獲盜功,洊擢閩安協副將。嘉慶十年,偕許松年會剿蔡牽,追至小琉球,見臺灣師船二為賊圍,赴援,松年舉旗招之,未至。以違調遣被劾,褫職逮訊。得白,復原官,署臺灣副將。十一年春,從李長庚擊蔡牽,破洲仔尾賊巢,牽乘間逸,奪頂戴。五月,破牽於鹿耳門,賜花翎。十二年,硃濆犯淡水,偕王得祿追至雞籠洋,連敗之,擒殲甚眾,被優敘。十三年,擢浙江定海鎮總兵。十四年,擢浙江提督。偕王得祿合擊蔡牽於漁山外洋,乘上風逼之,夜半浪急,不得進。明日,復要截環攻,牽且戰且走,傍午逾黑水洋,見綠水。良功恐日暮賊遁,大呼突進,以己舟逼牽舟,兩篷相結。賊以椗沖船,陷入死鬥。良功腓被矛傷,毀賊椗,得脫出。閩師繼之,牽遂裂舟自沉。論功,錫封三等男爵,次於王得祿。或為之不平,良功曰:「海疆肅清,已為快事,名位軒輊何足計?」二十二年,入覲,卒於途,賜恤,謚剛勇。子聯恩襲男爵,官直隸河間協副將。

陳步雲,浙江瑞安人。入伍隸水師,數獲盜,以勇力稱,授溫州營把總。從良功追蔡牽,步雲以四十人駕舟徑逼牽艦鏖斗,舟小不相當,見兩提督至,亟投火罐焚賊艦,以長戟鉤舷,率數卒躍登,短兵相搏,殲牽妻及其黨。賊艦已壞,牽猶持利刃踞柁樓,顧欲取之。良功隔船疾呼,船與水平,速去,放長繩水中援之起,而牽船沒矣。步雲身被十數創,兩提督皆臨慰視。事聞,賜獎武銀牌,擢千總。累遷閩安副將。總督孫爾準欲裁減師船,步雲言李提督所造船高大堅緻,其利遠勝同安夾板、快駒諸船、裁之緝匪無具,有事不能制敵,議乃寢。爾準薦其才可勝專閫,入覲,宣宗曰:「汝即隨邱、王兩提督攻沉蔡牽之陳步雲耶?」詢戰功甚悉。遂擢定海鎮總兵,歷瓊州、福寧、金門、海壇諸鎮。道光十九年,以傷發,乞解職。三十年,卒。

許松年,字蓉俊,浙江瑞安人。以武舉效力水師,從李長庚積功至提標參將。嘉慶十年,護理金門鎮總兵。擊蔡牽於小琉球;又擊硃濆、烏石二於宮仔洋,從李長庚追敗之於閩、粵交界甲子洋。又迭擊牽於青龍港、斗米洋。十一年,偕王得祿敗牽於臺灣洲仔尾,跐海水而登,焚溺無算。是年夏,李長庚攻牽於鹿耳門,松年扼張坑、返埕洋面,獲賊船一,沉船三,又於水澳擒蔡三來等。李長庚論水師將材,舉松年可獨當一面,總督阿林保以疏聞。十二年,從長庚擊蔡牽於大星嶼、浮鷹洋,松年躍入賊船獲之,被優敘。十三年,硃濆潛匿東湧外洋,命松年躡剿,遂移師入粵。追至長山尾,了見賊船四十餘,知其最巨者為濆所乘,並力圍攻,濆受砲傷,未幾斃。詔嘉松年奮勇,克殲渠魁,賜花翎,予雲騎尉世職。粵匪張保仔竄閩洋金門、廈門,松年遣漁船誘之,以舟師圍擊,獲船七,沉船六,被優敘。十五年,傷發回籍,尋丁母憂。十九年,授甘肅西寧鎮總兵,歷延綏、漳州、天津、碣石諸鎮。道光元年,擢廣東陸路提督,調福建水師提督。六年,臺灣械斗,松年方閱兵,彈壓解散,總督孫爾準與之不協,尋以治理輕縱,被議褫職,留臺效力。乞病歸,卒於家。子錫麟,襲世職。

黃標,字殿豪,廣東潮州人。由行伍拔補千總,擢守備。乾隆五十五年,艇匪肆掠,總督福康安議練水師,募奇才異能者領之。標技勇過人,生長海壖,習知水道險易,能久伏水底,視物歷歷可數,特被識拔。以捕獲龍門洋盜及狗頭山匪功,擢都司,署游擊。

嘉慶元年,剿匪於南澎外洋,獲李超勝等三十餘名。仁宗素知其名,詔嘉緝捕勤能,擢參將。二年,俘洋盜胡三勝等,復擊斃安南匪首,盡獲其眾,被優敘。三年,遷澄海副將。未幾,擢廣東左翼鎮總兵,命總統巡洋水師,責以肅清海盜。四年,剿匪大放雞山及雙魚桅、夾門外洋,殲獲甚眾,賜花翎,命繪像以進。尋以盜劫鹽艘被劾,詔原之。六年,復擊賊於南澎外洋,獲田亞猛等。七年,偕提督孫全謀剿博羅會匪,連破羊矢坑、羅溪營要隘,搗其巢。事平優敘,並被珍賚。自將水師,飲食寢處與士卒共,先後獲匪六百餘名,粵海倚為保障。八年,偕孫全謀出海捕賊,賊遁廣州灣。標議合兵守隘,俟賊糧盡可盡殲。全謀慮持久有風濤患,乃分兵,賊得突圍逸出。標嘆曰:「此機一失,海警未已!」憤懣成疾。尋坐師久無功,吏議奪職留任。未幾,卒。

自安南夷艇散後,餘黨留粵者分五幫:曰林阿發、曰總兵保、曰郭學顯、曰烏石二、曰鄭乙。提督錢夢虎、孫全謀皆庸材,不能辦賊。標歿後,益無良將,惟林國良、許廷桂以死事聞。

國良,福建海澄人。世襲騎都尉,授廣東碣石鎮標游擊,累遷海澄副將,繼標為左翼鎮總兵。十三年,追剿烏石二於丫洲洋,擊沉數艘,賊艦續至益多。國良以傷殞,優恤,謚果壯。

廷桂,廣東歸善人。由行伍擢千總。乾隆中,從征臺灣,累遷海門營參將。國良歿,護理左翼鎮總兵。十四年,擊殲匪首總兵保於外洋,圍其餘黨。張保仔率大隊來援。眾寡不敵,廷桂死之。賜恤,予雲騎尉世職。

洎蔡牽既滅,惟粵匪存,於是百齡為兩廣總督,乃斷接濟,整軍紀,越一年,剿撫以次定。東南海氛始靖。

論曰:東南海寇之擾,始末十有餘年。惟浙師李長庚一人能辦賊,以閩帥牽掣而阻成功,然長庚忠誠勇略聞於海內,上結主知,廟算既孚,乃專倚畀。洎閩、浙合力,賊勢浸衰,不幸長庚中殞,而王得祿、邱良功等以部將承其遺志,卒殲渠魁。粵將惟黃標可用,而未盡其才。百齡乘閩、浙殄賊之後,剿撫兼施,遂如摧枯拉朽。要之海戰惟恃船堅砲利,與斷接濟而已,循之則勝,違之則敗。得失之林,故無幸哉!


\end{pinyinscope}