\article{列傳九十九}

\begin{pinyinscope}
傅清拉布敦班第子巴祿鄂容安納穆札爾三泰

傅清,富察氏,滿洲鑲黃旗人,李榮保次子,傅恆弟也。雍正間,授侍衛。乾隆初,累遷至直隸天津鎮總兵。康熙中定西藏,留兵鎮撫,以大臣駐藏辦事,為員二,嗣省其一。是時駐藏副都統索拜當代,命傅清以副都統往。十一年,疏言:「西藏處徼外,西北界準噶爾,北通青海,為四川西南外郛。自雍正十二年設塘汛,不特傳送官文書,且以聯絡聲氣。上年索拜以節費議撤汛,使藏人任郵遞,謂之番塘。未幾輒被盜。今準噶爾當入藏熬茶,番塘恐滋誤。請自打箭爐至藏復置塘汛,酌沖僻遠近,當得兵千人以內。」議如所請。

十二年,西藏郡王頗羅鼐卒。頗羅鼐愛其次子珠爾默特那木札勒,請以為嗣,遂襲爵為郡王。上諭傅清曰:「頗羅鼐更事多,黽勉事中國。珠爾默特那木札勒幼,傅清宜留意。如珠爾默特那木札勒思慮所未至,當為指示。」傅清疏言:「頗羅鼐在時,長子公珠爾默特策布登出駐阿里克夏,當令珠爾默特那木札勒帥師出駐騰格里諾爾、喀喇烏蘇諸處。今仍遣珠爾默特策布登駐阿里克夏,令別遣宰桑駐騰格里諾爾、喀喇烏蘇諸處。」又以準噶爾入藏熬茶,請增兵分路防護。上命與珠爾默特那木札勒商榷,毋涉張皇。十三年,命以提督拉布敦代,傅清還。復授天津鎮總兵,遷古北口、固原提督。珠爾默特那木札勒請撤留藏兵,上從之。旋以副都統紀山代拉布敦。

十四年,紀山疏言珠爾默特那木札勒與達賴喇嘛有隙,請移達賴喇嘛置泰寧。上知珠爾默特那木札勒乖戾且為亂,命駐藏大臣復舊置二員,予傅清都統銜,自固原復往。紀山復疏謂珠爾默特那木札勒言其兄珠爾默特策布登將舉兵相攻,上命傅清途中詗虛實。傅清疏言:「珠爾默特策布登未嘗構兵,特珠爾默特那木札勒妄言,藉以奪其兄分地。臣至藏,即將珠爾默特那木札勒懲治。」是時上已遣侍郎拉布敦代紀山,因諭傅清,珠爾默特那木札勒乖戾且為亂,令熟計密奏。

十五年,傅清與拉布敦先後至藏,珠爾默特那木札勒迫其兄珠爾默特策布登至死,遂逐其子,遣使通準噶爾,叛益有跡。上命副都統班第赴西藏,與傅清、拉布敦密謀取進止,仍詔傅清、拉布敦毋輕發,並密諭四川總督策楞勒兵為備。珠爾默特那木札勒謀愈急,絕塘汛,軍書不得達。傅清與拉布敦未得上詔,計以為:「珠爾默特那木札勒且叛,徒為所屠。亂既成,吾軍不得即進,是棄兩藏也。不如先發,雖亦死,亂乃易定。」

十月壬午,召珠爾默特那木札勒至通司岡駐藏大臣署,言有詔,使登樓,預去其梯,若將宣詔。珠爾默特那木札勒方拜跪,傅清自後揮刀斷其首。於是其黨羅卜藏札什始率眾圍樓數重,發槍砲,縱火,傅清中三創,度不免,自剄死。拉布敦死樓下。主事策塔爾、參將黃元龍皆自殺。通判常明中矢石死。從死者千總二、兵四十九、商民七十七。事聞,上軫悼,宣示始末,謂其「揆幾審勢,決計定謀,心苦而功大」。傅清追封一等伯,謚襄烈,旋命立祠通司岡。喪還,上臨奠。其子孫以一等子世襲,賜白金萬。

班第至藏,戮羅卜藏札什等,疏陳珠爾默特那木札勒自立名號,通款準噶爾,稱策旺多爾濟那木札勒為汗,請其發兵至拉達克為聲援。上復降詔褒傅清、拉布敦,建祠京師,命曰雙忠。子明仁,以侍衛襲子爵。從征金川,卒於軍。

拉布敦,棟鄂氏,滿洲鑲紅旗人。其先對齊巴顏,於太祖時率所部來歸,語見阿蘭珠、朗格諸傳。父錫勒達事聖祖,自贊禮郎累遷吏部尚書。出署川陜總督,還京師。以鎮筸苗為亂,命偕副都統圖斯海、徐九如帥師討之,降三百一寨,剿十五寨。錫勒達與荊州副都統珠滿、湖廣提督俞益謨所戡定者,天星寨、龍椒洞、排六梁等三寨。亂定,與總督於成龍、巡撫趙申喬議立營汛,增設官吏為撫綏,復還京師。卒。

拉布敦,其第六子也。生有力,能彎十力弓,左右射。工詩文,習外國語言。康熙間,襲叔祖勒爾圖三等阿達哈哈番世職。雍正朝,從傅爾丹討準噶爾,戰於和通呼爾哈諾爾;又從策凌討準噶爾,戰於額爾德尼昭:皆有所斬馘,授世管佐領。上命軍中舉驍勇之士,拉布敦與焉,賜孔雀翎。乾隆初,累遷正紅旗滿洲副都統。八年,復討準噶爾,授參贊大臣,出北路。九年,授定邊左副將軍。其冬,疏言:「厄魯特宰桑額勒慎等內牧布爾吉推河,烏梁海得木齊札木禪內牧布延圖河源。布爾吉推河在阿爾臺山梁外,布延圖河源在阿爾臺山梁內,距卡倫不遠,已闉坐卡侍衛等嚴防。」十年冬,疏言:「烏梁海得木齊烏爾巴齊等避雪,內牧黃加書魯克,距卡倫不遠。託爾和烏蘭、布延圖、哈瑪爾沙海諸卡倫外,皆有準噶爾人蹤跡,仍闉坐卡侍衛等嚴防。」尋召還京師,授正白旗滿洲副都統。復出署古北口提督。

十三年,駐藏副都統傅清當代,命拉布敦往。十四年,召還,以紀山代,授工部侍郎。未終歲,上徵紀山還,復命赴藏。十五年,授左都御史。尋與傅清謀誅珠爾默特那木札勒,其黨羅卜藏札什圍樓,拉布敦挾刃躍下樓,擊殺數十人,自剖其腹死。上聞,贈爵、賜金、立祠如傅清。命以拉布敦之族升隸正黃旗,謚壯果。子隆保,以侍衛襲子爵。誤班奪官,爵除。

班第,博爾濟吉特氏,蒙古鑲黃旗人。康熙間,自官學生授內閣中書。五遷,雍正初至內閣學士。四川、雲南徼外與西藏定界,命偕副都統鄂齊如西藏宣諭。遷理籓院侍郎。坐事左遷,在內閣學士上行走。十一年,命在軍機處行走。乾隆三年,授兵部侍郎。外擢湖廣總督。剿鎮筸、永綏亂苗,兩閱月而畢,上嘉焉。五年,以憂還京師。六年,命仍在軍機處行走,授兵部尚書。

十三年,師征金川,授內大臣,出督軍餉,加太子少保。尋按四川巡撫紀山加徵累民狀,命即署巡撫。時訥親、張廣泗師久無功,上諮班第,但言廣泗罪狀,語不及訥親。上諭曰:「班第雖職餉,然為本兵軍機大臣,軍事及將弁功罪,皆職掌所在,不得以督餉,一切置不問。」左遷兵部侍郎。

十四年,予副都統銜赴青海辦事。西藏郡王珠爾默特那木札勒有叛跡,駐藏辦事大臣傅清、拉布敦疏聞。上移班第代拉布敦,未至,珠爾默特那木札勒謀益急,傅清、拉布敦召至廨,誅之。其徒卓呢、羅卜藏札什等遂叛,傅清、拉布敦死之。公班第達執卓呢、羅卜藏札什等,班第至,按訊,又得其黨德什奈等凡二十七人,悉誅之。上以藏酋授王爵名位過重,命班第達以公爵管格隆事,令班第宣諭。班第又疏陳珠爾默特那木札勒與準噶爾通書謀叛狀,上命誅珠爾默特那木札勒妻子。四川總督策楞等以師至,會議西藏善後諸事。西藏大定。十六年,授都統銜。十七年,還京師,仍在軍機處行走,授正紅旗漢軍都統。出署兩廣總督。

十九年,師征準噶爾,復授兵部尚書,署定邊左副將軍,出北路。準噶爾內亂,輝特臺吉阿睦爾撒納來降。詔以明歲進兵,諭班第籌畫。班第以軍中駝馬牛羊宜牧地,得扎布堪、呢圭諸處,冬令暖,富水草,令喀爾喀親王額琳沁多爾濟等往督牧。遣兵擒烏梁海宰桑東根、赤倫等,收其眾數千戶。復令參贊大臣薩喇爾將兵擒準噶爾宰桑庫克新瑪木特、通瑪木特,收其眾,得牲畜無算。上獎班第奮勇果斷,予子爵,世授正黃旗領侍衛內大臣,賜白金千。十二月,授定北將軍,召來京示方略。

二十年正月,大舉討準噶爾,班第出北路,阿睦爾撒納授定邊左副將軍為副;永常以定西將軍出西路,薩喇爾授定邊右副將軍為副。班第與阿睦爾撒納等議以二月出師。阿睦爾撒納將六千人先行,班第將二千人繼其後。班第至齊齊克淖爾,以馬不給,令千五百人先,留五百人待馬再進。至喇托輝,與阿睦爾撒納軍合。上以阿睦爾撒納為準噶爾人所知,令其前行易招撫,戒班第仍令阿睦爾撒納先行毋合軍。班第至額爾得裏克,復令阿睦爾撒納先行。四月,師至博羅塔拉,得達瓦齊所遣徵兵使者,知伊犁無備。班第謀約西路軍銳進。五月,遂克伊犁。達瓦齊以萬人保格登山,侍衛阿玉錫以二十餘騎擊之,驚走。上獎班第功,封一等誠勇公,賜寶石頂、四團龍補服、金黃絳朝珠。班第以伊犁厄魯特生計甚艱,不足供大兵,六月,疏請留察哈爾兵三百、喀爾喀兵二百移駐伊犁河北尼楚袞治事。諸軍次第遣還。是月,獲達瓦齊,獻俘京師。

軍初出,上察阿睦爾撒納有異志,令班第嚴約束。及伊犁既定,上令和碩特四部部置汗,將以阿睦爾撒納為輝特汗。阿睦爾撒納覬總統四部,意不慊,置副將軍印不用,用故準噶爾臺吉噶爾丹策凌菊形小印檄諸部,諱其降,言以中國兵定亂,叛跡漸著。上召阿睦爾撒納,以九月至熱河行在,行飲至禮,與他部汗同受封。參贊大臣色布騰巴爾珠爾率遣還諸軍以歸。阿睦爾撒納乞代奏,冀總統四部,期七月俟命。色布騰巴爾珠爾歸,不敢聞。以班第趣阿睦爾撒納詣熱河,令參贊大臣額林沁多爾濟與俱。阿睦爾撒納怏怏就道,而上念阿睦爾撒納終且叛,諭班第宜乘其未發討之,毋濡忍貽後患。諭至,阿睦爾撒納已行。上又命鄂容安等擒治。

八月,阿睦爾撒納行至烏隴古,解副將軍印還額林沁多爾濟,走額爾齊斯,遂叛。伊犁道梗。阿睦爾撒納之黨克什木、巴朗、敦克多曼集、烏克圖等作亂,班第與鄂容安以五百人拒戰,自固勒札赴空格斯,轉戰至烏蘭庫圖勒,賊大至,圍合。班第拔劍自剄,鄂容安同殉。上初聞班第等陷賊,令參贊大臣策楞自巴里坤間使傳諭毋以身殉。策楞聞訛傳班第等自賊中出,以聞,上解所佩荷包為賜。既聞班第等死事狀,降詔謂:「班第、鄂容安見危授命,固為可憫;然於事無補,非傅清、拉布敦為國除兇者比。」二十一年,師復定伊犁。喪還,上親臨奠,並令執克什木、巴朗等,馘耳以祭。又以薩喇爾同陷賊不能死,令監往旁視。尋以班第義烈,仍如傅清、拉布敦故事,京師建祠,亦曰雙忠。旋復命圖形紫光閣。

子巴祿,初以察哈爾總管從軍,襲一等誠勇公,授鑲紅旗蒙古都統,從定伊犁。師討霍集占,授參贊大臣,授將軍兆惠有功,命駐軍和闐。戰伊西洱庫爾淖爾,屢敗霍集占。師還,加雲騎尉世職,圖形紫光閣,為後五十功臣首。出為涼州、綏遠城將軍、察哈爾都統。卒。

鄂容安,字休如,西林覺羅氏,滿洲鑲藍旗人,大學士鄂爾泰長子。雍正十一年進士,改庶吉士。世宗命充軍機處章京。乾隆元年,授編修,南書房行走。再遷,五年,授詹事府詹事。鄂爾泰承旨固辭,上曰:「鄂容安與張廷玉子若靄,皇考命在軍機處行走,本欲造就成材。朕茲擢用,鄂爾泰毋以己意辭。」是時直軍機處大臣與章京皆曰行走,無異辭也。尋又命上書房行走。七年,以與聞左副都御史仲永檀密奏留中事,奪職,語在永檀傳。八年,命仍在上書房行走,授國子監祭酒。十年,襲三等伯爵,後五年加號襄勤。十二年,授兵部侍郎。

十三年,出為河南巡撫,賜孔雀翎。河南境伏牛山界陜西、湖北二省,袤延八百餘里,鄂容安行部入山親勘。又以界上諸關通大道,易藏奸宄,飭行保甲,入奏,上嘉焉。衛輝參將阮玉堂督操,鞭所部兵,兵譁。鄂容安疏請先治譁兵罪,然後罷玉堂,毋令兵驕,亦當上指。鄂容安又令糴補諸府、州、縣常平倉榖都二十九萬石有奇,浚治開封、歸德、陳州三府幹枝諸水,以慎蓄洩、廣灌溉。上獎其留心本務。

十五年,上巡幸河南,鄂容安疏言河南士民樂輸銀五十八萬七千有奇,上曰:「朕巡幸方岳,從不以絲毫累民,曾何藉於輸將?且省方問俗,勤恤民隱,尚慮助之弗周,豈容供用轉資於下?鄂容安此奏失政體。其以輸銀還之士民。」鄂容安疏請罪,又言:「士民輸銀出本原,還之恐不免胥吏中飽,仍請允其奏。」上意終不懌。還幸保定,鄂容安入見,不引謝,上詰責,令痛自改悔,不得有絲毫糜費粉飾,為補過之地。

十六年,移山東巡撫。濟南被水,米貴。鄂容安請用乾隆十三年例,暫弛海禁,招商往奉天糴運。旋與東河總督顧琮規塞張秋掛劍臺河決,培築運河堤,自臺兒莊至德州千有餘里,循堤建堡房。塞太行堤涵洞,以紓寧陽等縣水患。十七年,疏陳山東州縣吏交代庫銀倉榖多有虧缺,下各府考覈。又移江西巡撫。

十八年,授兩江總督。十九年,疏言:「江南地廣事繁,胥役弊滋甚。淮安等府藉賑為弊,蘇州等府藉漕為弊,徐州府藉應徭為弊,當嚴覈懲治。令各屬胥吏遵經制原額,禁偽冒及額外無名白役。」是年考績,加太子少傅。

上將用兵準噶爾取達瓦齊,以鄂容安年力方盛,勇壯曉暢,召授參贊大臣。二十年,永常以定西將軍出西路,薩喇爾以定邊右副將軍為副,鄂容安實從。諭曰:「漢西域塞外地甚廣,唐初都護開府擴地及西北,今遺阯久湮。鄂容安在軍,凡準噶爾所屬及回部諸地,有與漢、唐史傳相合可援據者,並漢、唐所未至處,當一一諮詢記載。」旋偕薩喇爾入告,途中撫降諸部落,並檄諭達瓦齊,賚荷包、鼻煙壺。

及師定伊犁,值胡中藻以賦詩誹上誅。中藻為鄂爾泰門生,鄂爾泰從子鄂昌與唱和,連坐。上責鄂容安不為陳奏,行賞獨不及。命與班第駐守伊犁。

阿睦爾撒納叛跡漸著,鄂容安入告。上令與薩喇爾率師至塔爾巴哈臺相機捕治。阿睦爾撒納入覲,中途遂叛,伊犁諸宰桑應之。鄂容安與班第力戰不支,相顧曰:「今日徒死,於事無濟,負上付託矣!」班第自剄。鄂容安腕弱不能下,命其僕剚刃於腹,乃死。故事,大臣予謚者,內閣擬二謚請上裁,以翰林起家者例謚「文」,至是擬「文剛」、「文烈」,上抹二「文」字,謚剛烈。圖形紫光閣,上親為贊,有曰:「用違其才,實予之失。」蓋重惜之也。以次子鄂津襲爵,官至伊犁領隊大臣,坐事奪官;以鄂容安長子鄂岳襲爵。

納穆札爾,圖伯特氏,蒙古正白旗人,都統拉錫子。納穆札爾自閒散授藍翎侍衛。累遷工部侍郎、鑲藍旗滿洲副都統。乾隆十五年,西藏珠爾默特那木札勒之亂既定,命偕班第駐西藏。議增設噶卜倫,皆予扎薩克銜。自喀喇烏蘇至庫車增臺八,設兵。準噶爾通藏,凡阿里、那克桑、騰格里淖爾、阿哈雅克四路,各於隘口設卡倫。又有勒底雅路,為準噶爾犯藏時間道,亦駐兵防守。迭疏陳請,皆如議行。

十九年,杜爾伯特諸部來降,命赴北路料理游牧。偕喀爾喀親王得親扎布規畫安置輝特、和碩特十三旗於固爾班舒魯克,杜爾伯特十旗於鄂爾海西喇烏蘇,分界駐牧,設卡倫防範。納穆札爾撫降人頗至,當夏,慮赴京領餉不耐炎暑,請遣使轉餉至張家口散給;及秋,杜爾伯特諸旗遇霜雪損畜,入告,予米五百石賑撫。輝特、和碩特諸旗生計絀,奏濟以糧畜。

阿睦爾撒納叛,命駐烏里雅蘇臺。旋移戶部侍郎。二十一年,和托輝特臺吉青滾雜卜亦叛,納穆札爾慮喀爾喀諸部為所動,傳檄諭以利害。上嘉之,授參贊大臣,從將軍成袞扎布率索倫兵追捕青滾雜卜。十一月,師至杭哈獎噶斯,已近俄羅斯境,捕得青滾雜卜,檻送京師。上獎納穆札爾勇往,封一等伯,世襲,號曰勤襄。二十二年,授工部尚書、正紅旗滿洲都統,命駐科布多。旋又命移駐布延圖。十月,署定邊左副將軍。二十三年,議烏梁海降人酋曰察達克所屬鄂拓克置得木齊、收楞額,治庶事。請以得木齊改佐領,收楞額改驍騎校,歲貢貂皮送烏里雅蘇臺,賚以緞布。疏入,如所議。

師討霍集占,復授參贊大臣,出西路。尋授靖逆將軍,會雅爾哈善攻庫車。及兆惠代雅爾哈善,將師自阿克蘇進逼葉爾羌,至喀喇烏蘇,為霍集占所圍。納穆札爾及參贊大臣三泰先奉命帥師濟兆惠軍,兆惠遣副都統愛隆阿、侍衛奎瑪岱來迎。納穆札爾道遣愛隆阿先還,而與三泰、奎瑪岱將二百騎夜進,遇賊三千餘,圍數重,力戰矢盡,遂沒於陣。上聞,追封三等義烈公,謚武毅。祀昭忠祠。回部平,圖形紫光閣。

子保寧,自有傳。保泰,自拜唐阿累遷察哈爾都統,與雅滿泰同為駐藏大臣,廓爾喀侵藏,保泰坐請達賴喇嘛、班禪額爾德尼避兵,又匿廓爾喀未構兵前表貢方物,及遣使有所請不以入奏,上改其名曰俘習渾,與雅滿泰同奪職荷校,先後予杖者四。藏事定,戍俘習渾黑龍江。赦還。雅滿泰復授侍衛。

三泰,石氏,漢軍正白旗人,都統石文炳孫也。父觀音保,官至都統。三泰,自藍翎侍衛累遷正紅旗漢軍副都統、吏部侍郎。乾隆二十三年,命軍機處行走,調戶部侍郎。命以參贊大臣行走從納穆札爾出西路。七月,命納穆札爾、三泰率健銳營及索倫、察哈爾兵應兆惠。夜進,期以黎明至兆惠軍。遇賊,眾寡勢不敵,力戰,三泰墜馬,徒步擊賊,中創死。三等侍衛彰武、藍翎侍衛班泰、管站四品花翎西拉布、護軍校委署章京齊旺扎布及兆惠所遣迎師三等侍衛奎瑪岱,皆死。上聞,追封三等子,謚果勇。

石廷柱之裔,本以散秩大臣世襲,至是,別授其兄祥泰散秩大臣。回部平,圖形紫光閣。上追悼納穆札爾、三泰死事,為賦雙義詩,以傅清、拉布敦殉西藏,班第、鄂容安死伊犁相擬。謂「此六人者,事異心同,皆與國休戚之藎臣也」。子佛柱,襲子爵、散秩大臣,官阿克蘇領隊大臣。

論曰:高宗朝徼外諸叛,霍集占最桀驁耐戰,方其困兆惠保葉爾羌,非師武臣力,幾不能克。阿睦爾撒納既叛,師未接,輒遠竄,非霍集占比也。珠爾默特那木札爾欲背中國,乃汗準噶爾,尤愚妄,殆不足數。六臣所遇異,故其效亦殊。大誅既加,罪人斯得,咸廩廩稱義烈矣。


\end{pinyinscope}