\article{列傳一百十六}

\begin{pinyinscope}
宋元俊薛琮張芝元董天弼柴大紀

宋元俊,字甸芳,江南懷遠人。以武進士授四川成都營守備,遷懷遠營都司。乾隆二十年,孔撒、麻書兩土司構釁,金川、綽斯甲布兩土司乘隙為亂,元俊為撫定,集孔撒、麻書、金川、綽斯甲布、革布什咱、綽沃、白立、章谷、瞻對諸土司斷曲直,使頂經立誓。累遷阜和營游擊。

二十九年,金川土司郎卡侵丹壩、綽斯甲布兩土司,諸土司請兵,署總督阿桂、提督岳鍾琪奏令元俊偕署副將長清諭各土司合兵進剿。移漳臘營參將,坐事左遷。三十五年,小金川土司澤旺之子僧格桑掠鄂克什,阿桂檄元俊宣諭僧格桑還侵地及所掠番民。復補阜和營游擊。三十六年,革布什咱頭人結郎卡子索諾木據革布什咱官寨,戕土司策楞多布丹,總督阿爾泰復令元俊往宣諭。小金川圍鄂克什、達木巴宗,侵明正土司,據納頂寨,元俊與參將薛琮、都司李天佑率兵討之,收納頂寨,進攻索布大寨。琮率兵自山梁潛度,元俊與天佑渡河夾擊,獲石卡十八,屢戰皆捷,明正土司碉寨七百餘盡復。

師入小金川境,取噶中拉、莫如納、扎功拉等地,進克納咱。阿爾泰及侍郎桂林以聞,擢松潘鎮總兵。師攻甲木,賊據喇嘛寺為固。元俊及守備陳定國攻破之,盡收所屬城、卡、碉、寨,據墨爾多山梁。師復進,天佑、定國攻西山梁,元俊同侍衛六十一、參將巴克坦布等自喇嘛寺繞攻郭松,參領普寧自西山麓沿河攻甲木,侍衛哈青阿及琮自東山麓攻卡丫。師行以夜半,戰自卯至巳,卡丫、郭松、甲木皆克。賜元俊孔雀翎。

三十七年,師攻革布什咱,元俊請於桂林,分兵為五道:一自郭宗濟野宗攻木巴拉博租;一自章穀渡河夾攻,俾賊前後受敵,兩軍既合,先據默資溝,截金川來路,進取吉地官寨;一自巴旺之高石、嘉舉諸山,分道攻薩瑪多監藏布覺,取吉地;一自茂紐攻沙沖;一自喀勒塔爾攻黨哩,會兵取丹東。策定,元俊及游擊吳錦江等自章穀渡河據格藏橋,哈青阿、天佑出郭宗濟野宗,兩隊軍夾攻,賊驚潰,遂克木巴拉博租、薩瑪多監藏布覺諸地。進克吉地官寨及默資溝。參將常泰等克黨哩,都司李天貴等克沙沖,元俊復克丹東。復革布什咱地三百餘里,民戶二千餘。

桂林遣陳定國調綽斯甲布兵駐軍界上,備調遣。上責桂林不令元俊乘勝取金川。元俊旋與散秩大臣阿爾泰劾桂林欺誑及諸罪狀,上為奪桂林職,令阿爾泰署四川總督,命額駙、尚書、公福隆安按治。未至,詔元俊督兵赴綽斯甲布率土兵進攻金川。元俊奏:「自戰失利,士氣消沮,現在兵力不足並按兩金川。請敕調湖南、湖北、山西、甘肅兵二萬,分三道進軍,計兩月可竟事。」上以元俊請益師,未免張皇,令福隆安會阿爾泰、阿桂與元俊詳悉覈計。上諭軍機大臣,謂:「元俊能治事,熟番情;但其人似狡猾好事,當留意駕馭。」

尋,福隆安疏陳所劾桂林狀不實,上以方進兵,元俊熟番情,諸事不必窮究;惟言:「桂林以白金畀金川贖被掠官兵罪最重,今汪承霈自承出其意。承霈以曹司從軍,不當與其事。當詰汪騰龍,成信讞。」福隆安復疏言:「騰龍以金囑王萬邦待巴旺、布拉底克歸迷道官兵予金為賞,元俊誘萬邦令具札言桂林使贖被掠官兵。事為元俊陷。」上乃怒,責元俊奸狡負恩,命奪職逮問,籍其家。參贊阿桂疏言:「元俊在川日久,熟番情,為近邊土司所信服。諸將能馭番無出其右。臣遇事多與詢商,冀收指臂之效。乞恩仍留軍中,倘奮勉出力,使詐使貪,原所不廢;如剛愎逞私,即據實嚴劾。」上命留總兵,還所籍財產。元俊同副都統永平、博靈阿等潛赴墨壟溝,進至郡崢。乘月督軍登山薄賊卡,正大霧,我師騰躍入卡,克山梁三道、碉卡二十有四,進克格魯克石。金川酋圖占丹壩官寨,綽斯甲布土司發兵往助,阿桂奏令元俊增兵往剿,未行,卒於軍。

元俊在邊久,善馭諸土司。往時賚諸土司繒帛輒窳敝,元俊必以善者,諸土司皆喜。元俊出行邊,諸土司率妻子出謁,畀以茶、菸、簪珥,視若家人。稍不循法度,即訶譴,皆悚息聽命。打箭爐徼外夾壩出沒,元俊至,無敢犯行李者。諸番小有動靜,爭來告,以故元俊諸所措置皆中窾要。其得罪,上亦知其枉。既卒,其子猶戍邊。四十一年,金川平。元俊部將張芝元請於阿桂,謂元俊有功無罪,徒以忤專閫被羅織,語甚切。阿桂為疏請,赦其子還。

薛琮,陜西咸寧人。父翼鳳,河南南陽鎮總兵。琮以廕生入巡捕營。累遷四川漳臘營參將。阿爾泰討金川,以琮從。克納頂、邊穀諸碉寨。溫福代阿爾泰視師,攻巴朗拉,琮戰最力。又克卡丫,取通甲木。攻阿仰東山,總督桂林與都統鐵保、提督汪騰龍將兵取墨壟溝,令琮將三千人自甲木、噶爾金後繞山道應大軍夾擊。桂林中道引還卡丫,又檄鐵保、騰龍令退。琮深入,糧盡,待桂林不至。桂林令都司廣著赴援。賊據高峰曰博六古通,險阻,廣著師不得度。琮督兵直進,毀柵十餘,奪碉七十餘。賊力拒,琮督兵仰攻,中槍,沒於陣,軍盡覆,同死者都司張清士、陳定國等二十五人。阿桂破翁古爾壟,立祠戰地祀琮等。

琮在諸將中號能戰,元俊與最厚。嘗與期旦日會師,孰後至當斬。琮至後二刻,元俊遣騎持刀呼取薛參將頭。琮望見笑曰:「琮頭當與賊,不與公也!」奮前奪數碉反。元俊猶為琮請罪,以功論贖乃已。及桂林誤琮戰沒,元俊憤激論劾,卒以是得罪。

張芝元,四川清溪人。以千總從副將軍明亮征金川有功,積官至越巂營參將。金川酋以番僧詗軍事,芝元言於明亮曰:「軍事每為賊知,非去其諜,滅賊無日矣。」會大風雪,明亮命芝元率數十人偽若以他事出者,宿番僧寺中。芝元故通番語,與僧飲甚歡,僧醉眠,芝元出寺聚柴焚之,僧皆死。賊諜斷,因招降其眾。尋從成都將軍特成額駐兵江卡,捕夾壩,圍本肯賊寨,焚其碉,斃賊甚眾,擢懋功協副將。臺灣林爽文為亂,芝元率屯練降番佐軍。參贊海蘭察等分攻大埔林、中林、大埔尾三莊,芝元為策應。賊據小半天山,將軍福康安等自前山進,芝元與領隊大臣普爾普領兵別為一隊,夜半先發,繞大山夾攻賊後。黎明,諸軍同抵山麓,攀援上,賊力拒,芝元先登,拔其柵,斬獲無算,並堵賊去路。未幾,爽文就擒。臺灣平,擢建昌鎮總兵,圖形紫光閣,列前二十功臣。尋調松潘鎮總兵。廓爾喀掠西藏濟嚨、聶拉木,上命芝元率屯練降番往討之。芝元至,值大雪,山谷皆滿。芝元手大刀指揮,士卒皆感激用命,賊敗走。廓爾喀再叛,芝元偕提督成德督兵攻聶拉木,守拍甲嶺隘口斷賊援,聶拉木遂下;乘勝攻濟嚨,復克之,賊懼,乞降。未幾,卒。五十八年,論平定廓爾喀功,再圖形紫光閣,列後十五功臣。

芝元少以小校事元俊,後乃雪元俊枉。人以是多芝元,亦益賢元俊能知人也。

董天弼,字霖蒼,順天大興人。自武進士授四川提標前營守備。乾隆初,師征金川,天弼在軍有功。累遷維州協副將。金川酋郎卡攻丹壩土司,天弼偕游擊宋元俊諭郎卡歸所掠,毀所築碉,兵罷,遷松潘鎮總兵。旋擢四川提督。郭羅克部劫西藏入貢喇嘛,上命天弼按治,未得其渠,詔責其茍且。三十五年,小金川土司澤旺子僧格桑為亂,攻鄂克什土司色達克拉,圍其寨。天弼督兵駐達木巴宗,檄僧格桑斂兵退色達克拉,以其寨糧盡,乞徙達木巴宗。天弼與總督阿爾泰議留兵戍焉。

三十六年,僧格桑復圍達木巴宗,並略木耳宗、巴朗拉諸地。天弼自打箭爐出邊,徵省標及松潘、維州諸鎮協兵,行至眠龍岡,賊已得巴朗拉,築碉卡為久守計,且斷我兵路。天弼議襲山神溝以解達木巴宗圍,尋將四百人自山神溝至德爾密,克碉七,賊竄走;再進取畢旺拉,賊乘霧來犯,土兵驚潰,德爾密、畢旺拉皆陷。天弼疏請罪,上以天弼所將兵本少,總督阿爾泰不預策應援,宥其罪,諭以「當奪勉。再不努力,獲罪滋重矣」。天弼復將五百人自木坪陟堯磧,順山攻甲金達對面山梁,取碉二。天弼以鄂克什牛廠當要道,分兵殲守廠賊,駐軍其地;乘勝上下截擊,木坪、鄂克什諸土司錯壤,要隘皆為我軍有。未幾,賊復襲據牛廠。上以阿爾泰師久無功,奪官。因責:「天弼始終貽誤,與阿爾泰同罪,奪官,留軍中充伍。如更退縮,正軍法。」尋命下成都獄。詔未至,天弼以甲金達山峻不可上,求間道,得溝在兩崖間。會大風雪,天弼率兵自溝中潛度,遂至達木巴宗,擊僧格桑色達克拉;潰圍出,並克木耳宗,迎溫福師與會。上聞,命貸死,留軍中。阿桂令天弼監火藥軍械。三十七年,師克資哩,阿桂令天弼將五百人駐焉。尋予副將銜,授重慶鎮總兵。命督兵赴曾頭溝,進至梭磨,梭磨土婦請以千人從。事聞,賜花翎。天弼督兵攻堪卓溝,自間道出納雲達,深入賊境五十餘里,克山梁三,破碉卡三十餘、木城三。迎溫福師會於布朗郭宗,克大板昭、木丫寨,得碉三十六、卡十六。上以溫福已得布朗郭宗進克底木達,天弼所克不過空寨,疏語頗鋪張,手敕戒之。尋授領隊大臣。

三十八年,復為四川提督。時小金川已定,溫福督師進討大金川,令天弼以五百人守底木達。溫福進駐木果木,號大營;底木達當賊來路,為要隘。溫福檄三百人益大營,又去其後援。時溫福以軍屢勝,不以賊為意。金川頭人七圖葛拉爾思甲布等以千餘人詐降,溫福使與廝養雜處,因誘諸降人為變,諗底木達兵弱無後援,六月乙丑朔,潛自山後擁眾攻底木達,天弼率所部二百人抽刀力戰,至夜半,賊以鳥槍數百環擊,殺之。越九日,劫大營,溫福亦死焉。上先命天弼駐丹壩,旋命移駐布朗郭宗,軍中傳賊來犯。時天弼方屯美諾,上命奪官逮治。總督劉秉恬疏言:「天弼自美諾馳赴底木達,途遇賊,右脅中槍死。」仍以貽誤軍事籍其家,戍其子舉人聯伊犁。

金川既平,獲七圖葛拉爾思甲布,傳送熱河行在,廷訊,具言天弼死事時力戰狀,乃赦聯還,授內閣中書。

柴大紀,浙江江山人。自武進士授福建守備。累擢至海壇鎮總兵,移臺灣鎮。乾隆五十一年十一月,林爽文亂起。爽文漳州人,徙彰化,所居村曰大里杙。時奸民相聚,號天地會,漳州人莊煙為之魁,爽文與相結,謀為變。臺灣知府孫景燧馳詣彰化,督知縣俞峻、副將赫生額、游擊耿世文捕治,焚數小村以怵之。爽文因民怨,夜糾其徒來襲,赫生額等皆戰死。明日,遂破彰化,景燧亦殉焉。傍攻諸羅、鳳山,皆陷。大紀時以總兵守府城,賊分道來攻,大紀出駐鹽埕橋御之,擊沉賊舟數十,馘千餘。

五十二年春,水師提督黃仕簡、陸路提督任承恩先後赴援。大紀出攻諸羅,克之,即移軍守諸羅。旋以守府城功,賜花翎。上以仕簡、承恩師久無功,授總督常青將軍,渡臺灣視師。爽文攻諸羅,自二月至四月凡十至,大紀督游擊楊起麟、守備邱能成等出戰,殺賊數千。爽文之徒張慎徽偽降,大紀察其詐,置諸法。臺灣諸府縣皆編竹為城,不耐攻,大紀以忠義率兵民誓堅守。上嘉大紀勞,賜荷包、奶餅,下部議敘。六月,授福建陸路提督,仍兼領臺灣總兵。鹽水港者,諸羅通府城糧道也,賊來攻,大紀力禦之。上促常青赴援,予大紀壯健巴圖魯名號,參贊軍務。八月,上以常青衰老不能辦賊,命福康安為將軍,仍令大紀參贊;而常青令總兵魏大斌援諸羅,賊邀諸途,退駐鹿仔草;復令總兵蔡攀龍援諸羅,大紀出戰,迎入城共守。上移大紀水師提督,而以陸路提督授攀龍。十一月,加大紀太子少保。上以諸羅被圍久,縣民困守,奮力向義,更縣名為嘉義。賊攻城益急,上密諭大紀:「不必堅執與城存亡,如遇事急,可率兵力戰,出城再圖進取。」大紀疏言:「諸羅居臺灣南北之中,縣城四周積土植竹,環以深壕,壕上為短垣,置砲,防衛堅固。一旦棄之而去,為賊所得,慮賊勢益張,鹽水港運道亦不能守。且城廂內外居民及各莊避難入城者共四萬餘人,助餉協守,以至於今。不忍將此數萬生靈付逆賊毒手!惟有竭力保守,以待援兵。」上手詔謂:「所奏忠肝義,披覽為之墮淚!大紀被圍日久,心志益堅,勉勵兵民,忍饑固守,惟知以國事民生為重。古之名將,何以加之?」因封為一等義勇伯,世襲罔替,並命浙江巡撫瑯玕予其家白金萬,促福康安赴援。

十二月,福康安師至,嘉義圍解,大紀出迎,自以功高拜爵賞,又在圍城中,倥傯不具櫜鞬禮,福康安銜之,遂劾大紀詭詐,深染綠營習氣,不可倚任。上諭謂:「大紀駐守嘉義,賊百計攻圍,督率兵民,力為捍衛。朕諭以力不能支,不妨全師而出。大紀堅持定見,竭力固守,不忍以數萬生靈委之於賊。朕閱其疏,為之墮淚。福康安乃不能以朕之心為心乎?大紀嘗奏賊以車載槍砲攻城,今福康安言得賊攻城大車,又委棄槍砲,為我軍所得,足見大紀前奏不虛。大紀又奏縣城食盡,地瓜、花生俱罄,以油■K5充食。當時義民助餉,未必遽至於此。但大紀望援心急,以食油■K5為詞。普吉保、恆瑞兩軍尚復觀望不進,若雲猶有餘粟,則兩路赴援更緩。此時縣城存亡未可知,安怪大紀過甚其詞耶?大紀屢荷褒嘉,在福康安前禮節或有不謹,致為所憎,直揭其短。福康安當體朕心,略短取長,方得公忠體國之道。」侍郎德成自浙江奉使還,受福康安指,訐大紀。上命福康安、李侍堯、徐嗣曾、瑯玕按治,福康安臨致書軍機大臣,言:「大紀縱兵激民為變,其守嘉義,皆義民之力。大紀聞命,欲引兵以退,義民不令出城,乃罷。」事聞,上諭謂:「守諸羅一事,朕不忍以為大紀罪,至其他聲名狼藉、縱兵激變諸狀,自當按治。」命奪大紀職,逮問。福康安尋以大紀縱弛貪黷、貽誤軍機,議斬,送京師。上命軍機大臣覆讞,大紀訴冤苦,並言德成有意周內,迫嘉義民證其罪,下廷訊,大紀猶力辯。五十三年七月辛巳,命如福康安議棄市,其子發伊犁為奴。

論曰:元俊、天弼在邊久,熟情偽,習形勢,諸番仰其威惠。元俊戹於桂林,激而欲自白,不得直;微阿桂右之,罪且不測。天弼又見嫉於溫福,驅至寡之兵以投方張之寇,既死猶尚以為罪。若大紀有功無罪,為福康安所不容。高宗手詔,可謂曲折而詳盡矣,乃終不能貸其死。軍旅之際,捐肝腦,冒鋒刃,求尺寸之效,困於媢嫉,功不成而死於敵,若功成矣,而又死於法。嗚呼,可哀也已!


\end{pinyinscope}