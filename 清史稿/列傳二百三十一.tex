\article{列傳二百三十一}

\begin{pinyinscope}
黃體芳子紹箕宗室寶廷宗室盛昱張佩綸何如璋

鄧承修徐致祥

黃體芳,字漱蘭,浙江瑞安人。同治二年進士,選庶吉士,授編修。日探討掌故,慨然有經世志。累遷侍讀學士,頻上書言時政得失。晉、豫饑,請籌急賑,整吏治,清庶獄,稱旨。時議禁燒鍋裕民食,戶部覈駁,體芳謂燒鍋領帖,部獲歲銀三萬,因上董恂奸邪狀,坐鐫級。

光緒五年三月,惠陵禮成,主事吳可讀為定大統以尸諫。詔言:「同治十三年十二月初五日降旨,嗣後皇帝生有皇子,即承繼大行皇帝為嗣。吳可讀所奏,前旨即是此意。」於是下群臣議,體芳略言:「『即是此意』一語,止有恪遵,更有何議?乃激烈者盛氣力爭,巽畏者囁嚅不吐,或忠或謹,皆人臣盛節,而惜其未明今日事勢也。譬諸士民之家,長子次子各有孫,而自祖父母視之則無異。然襲爵職必歸之長房者,嫡長與嫡次之別也。又如大宗無子,次宗止一嫡子,然小宗以嫡子繼大宗,不聞有所吝者,以仍得兼承本宗故也。唯君與民微有不同。民間以嫡子繼大宗,則大宗為主,本宗為兼。天潢以嫡子繼帝系,則帝系為主,本宗可得而兼,親不可得而兼。若人君以嫡子繼長支,則固以繼長支為主,而本宗亦不能不兼。蓋人君無小宗,即稱謂加以區別,亦於本宗恩義無傷。此兩宮意在嗣子承統,慈愛穆宗,亦即所以慈愛皇上之說也。今非合兩統為一統,以不定為豫定,就將來承繼者以為承嗣,似亦無策以處之矣。試思此時即不專為穆宗計,既正名為先帝嗣子,豈有僅封一王貝勒者乎?即不專為皇上計,古來天子之嗣子,豈有以不主神器之諸皇子當之者乎?即僅為穆宗計,皇上可如民間出繼之子乎?即僅為皇上計,穆宗可如前明稱為皇伯考乎?夫奉祖訓,稟懿旨,體聖意,非僭。先帝今上皆無不宜,非悖。明其統而非其人,非擅。論統系,辨宗法,正足見國家億萬年無疆之庥,非干犯忌諱。此固無意氣可逞,亦無功罪可言也。」疏入,詔存毓慶宮。自是劾尚書賀壽慈飾奏,俄使崇厚誤國,洪鈞譯地圖舛謬,美使崔國英赴賽會失體,皆人所難言,直聲震中外。

七年,遷內閣學士,督江蘇學政。明年,授兵部左侍郎。中法事起,建索還琉球、經畫越南議。十一年,還京,劾李鴻章治兵無效,請敕曾紀澤遄歸練師,忤旨,左遷通政使。兩署左副都御史,奏言自強之本在內治,又歷陳中外交涉得失,後卒如所言。十七年,乞休。二十五年,卒。子紹箕、紹第,並能承家學,而紹箕尤贍雅。

紹箕,字仲弢。光緒六年進士,以編修典試湖北。晉侍講,擢庶子。京師立大學堂,充總辦。究心東西邦學制,手訂章條。遷侍讀學士。歷充編書局、譯書局監督。出為湖北提學使。東渡日本,與其邦人士論孔教、輒心折。歸,未幾,卒。

宗室寶廷,字竹坡,隸滿洲鑲藍旗,鄭獻親王濟爾哈朗八世孫。同治七年進士,選庶吉士,授編修。累遷侍讀。光緒改元,疏請選師保以崇聖德,嚴宦寺以杜干預,覈實內務府以節糜費,訓練神機營以備緩急,懿旨嘉納。大考三等,降中允,尋授司業。是時朝廷方銳意求治,詔詢吏治民生用人行政,寶廷力抉其弊,諤諤數百言,至切直。晉、豫饑,應詔陳言,請罪己,並責臣工。條上救荒四事,曰:察釐稅,開糧捐,購洋米,增糶局。復以災廣賑劇,請行分貸法。畿輔旱,日色赤,市言訛駴,建議內嚴防範,外示鎮定,以安人心。歷遷侍講學士,以六事進,曰:明黜陟,專責任,詳考詢,嚴程限,去欺蒙,慎赦宥,稱旨。五年,轉侍讀學士。

初,德宗繼統嗣文宗,懿旨謂將來生有皇子,即繼穆宗為嗣。內閣侍讀學士廣安請頒鐵券,被訶責。至是,穆宗奉安惠陵,主事吳可讀堅請為其立後,以尸諫,下廷臣議。寶廷謂:「恭繹懿旨之意,蓋言穆宗未有儲貳,即以皇上所生之子為嗣,非言生皇子即時承繼也,言嗣而統賅焉矣。引伸之,蓋言將來即以皇上傳統之皇子繼穆宗為嗣也。因皇上甫承大統,故渾涵其詞,留待親政日自下明詔,此皇太后不忍歧視之慈心,欲以孝弟仁讓之休歸之皇上也。廣安不能喻,故生爭於前;吳可讀不能喻,故死爭於後。竊痛可讀殉死之忠,而又惜其遺摺之言不盡意也。可讀未喻懿旨言外之意,而其遺摺未達之意,皇太后早鑒及之,故曰『前降旨時即是此意』也。而可讀猶以忠佞不齊為慮,誠過慮也。宋太宗背杜太后,明景帝廢太子見深,雖因佞臣妄進邪說,究由二君有自私之心。乃者兩宮懿旨懸於上,孤臣遺疏存於下,傳之九州,載之國史,皇上天生聖人,必能以皇太后之心為心。請將前後懿旨恭呈御覽,明降諭旨,宣示中外,俾天下後世咸知我皇太后至慈,皇上至孝至弟至仁至讓,且以見穆宗至聖至明,付託得人也。如是,則綱紀正,名分定,天理順,人情安矣。因赴內閣集議,意微不合,謹以上聞。」

又奏:「廷臣謂穆宗繼統之議,已賅於皇太后前降懿旨之中,將來神器所歸,皇上自能斟酌盡善,固也。然懿旨意深詞簡,不及此引伸明晰,異日皇上生有皇子,將繼穆宗為嗣乎,抑不即繼乎?不即繼似違懿旨,即繼又嫌跡近建儲。就令僅言繼嗣,不標繼統之名,而臣民亦隱以儲貳視之,是不建之建也。而此皇子賢也,固宗社福;如其不賢,將來仍傳繼統乎,抑舍而別傳乎?別傳之皇子,仍繼穆宗為嗣乎,抑不繼乎?即使仍繼穆宗,是亦不廢立之廢立也,豈太平盛事乎?至此時即欲皇上斟酌盡善,不亦難乎?廷議之意,或以皇上親政,皇子應尚未生,不難豫酌一盡善之規。然國君十五而生子,皇子誕育如在徹簾之前,又何以處之乎?與其留此兩難之局以待皇上,何如及今斟酌盡善乎?且懿旨非皇上可改,此時不引伸明晰,將來皇上雖斟酌盡善,何敢自為變通乎?此未妥者一也。廷議又謂繼統與建儲,文義似殊,而事體則一,似也。然列聖垂訓,原言嗣統之常,今則事屬創局,可讀意在存穆宗之統,與無故擅請建儲者有間,文義之殊,不待言矣。今廷議不分別詞意,漫謂我朝家法未能深知,則日前懿旨『即是此意』之謂何,臣民不更滋疑乎?此未妥者又一也。」疏入,詔藏毓慶宮。其他,俄使來議約,朝鮮請通商,均有所獻納。

七年,授內閣學士,出典福建鄉試。既蕆事,還朝,以在途納妾自劾罷,築室西山,往居之。是冬,皇太后萬壽祝嘏,賞三品秩。十六年,卒。

子壽富,庶吉士。庚子,拳匪亂,殉難,自有傳。

宗室盛昱,字伯熙,隸滿洲鑲白旗,肅武親王豪格七世孫。祖敬徵,協辦大學士。父恆恩,左副都御史。盛昱少慧,十歲時作詩用「特勤」字,據唐闕特勤碑證新唐書突厥「純特勒」為「特勤」之誤,繇是顯名。光緒二年進士,既,授編修,益厲學,討測經史、輿地及本朝掌故,皆能詳其沿革。累遷右庶子,充日講起居注官。

閩浙總督何璟、巡撫劉秉璋收降臺匪黃金滿,盛昱劾璟等長惡養奸,請下吏嚴議,發金滿黑龍江、新疆安置。尚書彭玉麟數辭官不受職,劾其自便身圖,啟功臣驕蹇之漸。浙江按察使陳寶箴陛見未行,追論官河南聽獄不慎,罷免;張佩綸劾其留京幹進,寶箴疏辯,盛昱言其嘵嘵失大臣體,請再下吏議。朝鮮之亂也,提督吳長慶奉北洋大臣張樹聲檄,率師入朝,執大院君李罡應以歸,時詫為奇勛。盛昱言:「出自誘劫,不足言功,徒令屬國寒心,友邦騰笑。宜嚴予處分,俾中外知非朝廷本意。」為講官未半載,數言事,士論推為謇諤。

十年,遷祭酒。法越構釁,徐延旭』唐炯坐失地逮問,盛昱言:「逮問疆臣而不明降諭旨,二百年來無此政體。」並劾樞臣怠職。太后怒,罷恭親王奕等,而詔醇親王奕枻入樞府,盛昱復言:「醇親王分地綦崇,不宜嬰以政務。」其夏,命廷臣會議和戰大局,盛昱主速戰,力陳七利,謂:「再失事機,噬臍無及。」

盛昱為祭酒,與司業治麟究心教士之法,大治學舍,加膏火,定積分日程,懲游惰,獎樸學,士習為之一變。十四年,典試山東。明年,引疾歸。盛昱家居有清譽,承學之士以得接言論風採為幸。二十五年,卒。

張佩綸,字幼樵,直隸豐潤人。父印塘,官安徽按察使,卒於軍。佩綸,成同治十年進士,以編修大考擢侍講,充日講起居注官。時外侮亟,累疏陳經國大政,請敕新疆、東三省、臺灣嚴戒備,杜日、俄窺伺。晉、豫饑,畿輔旱,乃引祖宗成訓,請上下交儆,條四目以進:曰誠祈,曰集議,曰恤民,曰省刑。恭親王奕遭讒構,復請責王竭誠負重,上嘉納之。通政使黃體芳繼陳災狀,語稍激,絓吏議,佩綸力爭,被宥。尋丁憂,服竟,起故官。時琉球已亡,法圖越南亟,佩綸曰:「亡琉球則朝鮮可危,棄越南則緬甸必失。」因請建置南北海防,設水師四大鎮;又薦道員徐延旭、唐炯知兵堪任邊事,其招致劉永福黑旗兵為己用。是時吳大澂、陳寶琛好論時政,與寶廷、鄧承修輩號「清流黨」,而佩綸尤以糾彈大臣著一時。如侍郎賀壽慈,尚書萬青藜、董恂,皆被劾去。

光緒八年,雲南報銷案起,王文韶以樞臣掌戶部,臺諫爭上其受賕狀,上方意任隆密,乃援乾隆朝梁詩正還家侍父事,請令引嫌乞養,不報;又兩疏劾之,遂罷文韶,而擢佩緰署左副都御史,晉侍講學士。明年,法越構釁,佩綸章十數上,朝廷始遣兵征土寇、綴敵勢,法人不便其所,佯議和,而陰使人攻陷南定。佩綸請乘法兵未集,敕粵督遣水越都,而樞臣狃和局,慮佩綸梗議,令往陜西按事。已而法果襲順化,脅越與盟,越事益壞。使歸,命在總理各國事務衙門行走。

十年,法人聲內犯,佩綸謂越難未已,黑旗猶存,萬無分兵東來理,請毋罷戍啟戎心,上韙之。詔就李鴻章議,遂決戰,令以三品卿銜會辦福建海疆事。佩綸至船廠,環十一艘自衛,各管帶白非計,斥之。法艦集,戰書至,眾聞警,謁佩綸亟請備,仍叱出。比見法艦升火,始大怖,遣學生魏瀚往乞緩,未至而砲聲作,所部五營潰,其三營殲焉。佩綸遁鼓山麓,鄉人拒之,曰:「我會辦大臣也!」拒如初。翼日,逃至彭田鄉,猶飾詞入告,朝旨發帑犒之,命兼船政。嗣聞馬尾敗,止奪卿銜,下吏議。閩人憤甚,於是編修潘炳年、給事中萬培因等先後上其罪狀。時已坐薦唐尚、徐延旭褫職,至是再論戍。

居邊釋還,鴻章再延入幕,以女妻之。甲什戰事起,御史端良劾其干預公事,命遂回籍。庚子議和,鴻章薦其諳交涉,詔以編佐辦和約。既成,擢四五品京堂,稱疾不出。三十四年,卒。

何如璋,字子瓘,籍廣東大埔。同治七年進士,選庶吉士,授編修。以侍讀出使日本。歸,授少詹事,出督船政。承鴻章旨,狃和議,敵至,猶嚴諭各艦毋妄動。及敗,藉口押銀出奔,所如勿納,不得已,往就佩綸彭田鄉。佩綸慮敵蹤跡及之,紿如璋出。士論謂閩事之壞,佩綸為罪魁,如璋次之。如璋亦遣戍。後卒於家。

鄧承修,字鐵香,廣東歸善人。舉咸豐十一年鄉試,入貲為郎,分刑部。轉御史,遭憂歸。光緒初,服闋,起故官。與張佩綸等主持清議,多彈擊,號曰「鐵漢」。先後疏論闈姓賭捐,大乖政體;關稅侵蝕,嬰害庫帑;以考場積弊,陳七事糾正之;吏治積弊,陳八事肅澄之。又劾總督李瀚章失政,左副都御史崇勛無行,侍郎長敘等違制,學政吳寶恕、葉大焯,布政使方大湜、龔易圖,鹽運使周星譽諸不職狀。會邊警,糾彈舉朝慢弛,請召還左宗棠柄國政。逾歲,彗星見,則又言宗棠蒞事數月,未見設施,而因推及寶鋆、王文韶之昏眊,請罷斥,回天意。是時文韶方鄉用,權任轉重,會雲南報銷案起,又嚴劾之,仍不允。久之,遷給事中。

時朝鮮亂平,琉球案未結,上言簡知兵大臣駐煙臺,厚集南北洋戰艦番巡,留吳長慶軍戍朝互犄角。越南亂作,法人襲順化,復請詔百官廷議定國是,皆不報。十年,越事益壞,首劾徐延旭、唐炯失地喪師,趙沃、黃桂蘭擁兵僨事,宜肅國憲。其夏,法人原媾和,承修聯合臺諫上書,極言和議難恃。旋與司業潘衍桐密上間敵五策,並劾李鴻章定和之疏,嫉劉永福敢戰,言之憤絕。亡何,法果敗盟,侵臺灣雞籠,樞臣議和戰未決。於是承修再陳三策:「法所恃為援者西貢、東京。我若師分三路,亟攻越南,彼將自救不暇,策之上也。分兵為守,敵至則戰,敵退不追,老師糜餉,利害共之,策之中也。若慮餉詘運阻,不敢言戰,則其禍不勝言矣,是謂無策。」補鴻臚寺卿,充總理各國事務大臣。自此陳說兵事,章凡十三上,多見採納。嗣以中允樊恭煦獲譴,上疏營救,坐鐫秩。明年,赴天津佐鴻章與法使巴特納商和約,定新約十款。還,乞歸省。

未出都,命赴廣西與法使會勘中、越分界,至則單騎出關會法使浦理燮。浦理燮欲先勘原界,承修據約先欲改正界限,不相下,乃陽以文淵、保樂、海寧歸我,而陰電其駐京使臣,詆承修違約爭執,謂非先勘原界,勢將罷議。朝廷不獲已,許之。承修遂有三難二害之電奏,略言:「附界居民,不願隸法,先勘原界,慮滋事變,難一。保樂牧馬,游勇獷盛,道路梗阻,難二。原界碑折,十不存五,巉崗聳巘,瘴雨炎翳,人馬不前,難三。且原界既勘,彼必颺去,新界奚論?駈驢、文淵俱不可得,關門失險,戰守兩難,害一。文淵既失,北無寸地,關內通商,勢將迫脅,越既不存,粵將焉保?害二。」疏入,不省。

十二年,法人別遣狄隆、狄塞爾來會。適法官達魯倪思海至者蘭,為越人擊殺。狄使懼,又恥而諱其事,堅請按圖畫界,朝旨報可。於是首議江平、黃竹、白龍尾各地割隸越。承修指圖籍抗爭,狄使不能屈,欲分白龍尾半之左歸我而右歸越。承修以其地為欽海外戶,法得之則內偪防城,外斷東興、思勒,是無欽、廉也。議久之,暫與定約三條,猶未決,而狄使竟以兵力驅江平、黃竹居民內徙。朝廷慮啟邊釁,命先勘欽西至桂省全界,承修遂與訂定清約,語詳邦交志。十三年,具約本末以上,復官。十四年,謝病歸,主講豐湖書院,讀書養母。十七年,卒於惠州。

徐致祥,字季和,江蘇嘉定人。咸豐十年進士,選庶吉士,授編修。晉中允,典試山東。累遷內閣學士,督順天學政。遭憂去,服闋,起故官。光緒十年,法越構兵,德璀琳以和議進,朝旨未決。致祥上三策,謂:決戰宜速,任將宜專,軍勢宜聯。閩事棘,言何璟、張兆棟無幹濟才,而薦楊岳斌、張佩綸堪重任,頗嘉納。時議築鐵路,致祥聞而惡之,痛陳八害,並請力闢邪說,亟修河工,上責其誕妄,鐫三級。越二年,鐵路議再起,又再阻止之。先後封事十數上,而惓惓於抑奄寺,治河工,為時論所美。歷典福建、廣東鄉試。十八年,授大理寺卿,連劾樞臣禮親王世鐸、山西巡撫阿克達春,而糾彈張之洞尤不遺餘力。尋命視學浙江,有嚴名。

中日之役,我師敗績,上奕劻、李鴻章誤國狀,請逮葉志超、衛汝貴等寘之法,而畀馮子材、劉永福以征討名號,庶可振國威、作士氣。會山東教案起,德使海靖勒罷李秉衡職。致祥曰:「昔歲罷劉秉璋,今茲罷李秉衡,是朝廷黜陟之大權操之敵人也。為請顧全國體,毋懾敵。」私念國是不振,亂未有已,乃援引聖祖篤信硃子垂為家法往事,請舉行經筵以輔聖德,皆不報。秩滿,還朝,遷兵部右侍郎。二十四年,上違豫,眾情驚疑,復以輔導君德之說進。

是時國家多故,聖嗣尚虛,致祥為重國本計,略言:「昔宋真宗取宗室子養之宮中,逮仁宗既生,即遣歸邸;厥後仁宗、高宗、理宗皆踵行之。有子而遣養子歸邸者,真宗是也。無子而即以養子傳授神器者,仁宗之於英宗,高宗之於孝宗,理宗之於度宗是也。今以宗社系託之重,臣民屬望之切,深維至計,取則前朝,慎選近支宗室兄弟之子數人,擇親擇賢,入侍禁中,止以為子,不以為儲,恪遵家法,既可默察其賢否,徐以俟皇子之生。則皇上未有子而有子,皇太后未有孫而有孫,而穆宗付託之大業,亦繼承有屬矣。」乃未幾,果有立溥俊為大阿哥事。二十五年,卒。

論曰:體芳、寶廷、佩綸與張之洞,時稱翰林四諫,有大政事,必具疏論是非,與同時好言事者,又號「清流黨」。然體芳、寶廷議承大統,惓惓忠愛,非佩綸等所能及也。承修以搏擊為能,致祥以誕妄受責,君子譏之。唯盛昱言不妄發,潔身早退,庶超然無負清譽歟?


\end{pinyinscope}