\article{列傳三十九}

\begin{pinyinscope}
甘文焜子國璧範承謨子時崇馬雄鎮傅弘烈

甘文焜,字炳如,漢軍正藍旗人,其先自豐城徙沈陽。父應魁,從入關,官至石匣副將。

文焜善騎射,喜讀書,尤慕古忠孝事。以官學生授兵部筆帖式,累遷禮部啟心郎,屢奉使稱旨。康熙初,授大理寺少卿,遷順天府府尹。崇文門榷稅不平,疏劾之。廷議令兼攝,文焜曰:「言之而居之,是利之也。」固辭。六年,授直隸巡撫,奏復巡歷舊制。單車按部,適保定、真定所屬諸縣患水災,疏請蠲歲賦。總督白秉真以賑費浩繁,請聽官民輸銀米。文焜斥廉俸以助。議敘,加工部侍郎。

七年,遷雲貴總督,駐貴陽。時吳三桂鎮雲南,欲藉邊釁固兵權,詭報土番康東入寇,紿文焜移師,又陰嗾凱里諸苗乘其後。文焜策康東無能為,凱里近肘腋,不制將滋蔓,先督兵搗其巢,斬苗酋阿戎。既平,約云南會剿康東。三桂慮詐洩,謂康東已遠遁,繇是益憚之。文焜巡歷云、貴各府州皆遍。十年,遭母憂,上命在任守制。文焜又遣兵擊殺臻剖苗酋阿福。疏乞歸葬,許給假治喪。三桂請以雲南巡撫兼督篆,令督標兵悉詣雲南受節度,而以利啗之,冀為己用。

十二年,文焜還本官,適撤籓議起。三桂反,殺巡撫硃國治,遣其黨偪貴陽。文焜聞變,使族弟文炯齎奏入告,牒貴州提督李本深率兵扼盤江。本深已懷貳,先以書覘文焜意。文焜手書報之,期效張巡、南霽雲誓死守,而本深不之顧。本標兵已受三桂餌,紛潰弗聽調。文焜度貴陽不可守,令妾盛率婦女七人自經死,獨攜第四子國城赴鎮遠,思召湖北兵扼險隘,使賊不北出。十二月丙申朔,癸卯至鎮遠,守將江義已受偽命,拒弗納。文焜渡河至吉祥寺,義遣兵圍之。文焜望闕再拜,拔佩刀將自殺,國城大呼請先死,奪其刀以刎而還之,尸乃踣,血濺文焜衣。文焜曰:「是兒勇過我!」遂自殺,年四十有二。從者筆帖式和善雅圖殉。

亂平,貴州巡撫楊雍建以文焜治績及死事狀上聞,予優恤。遣其長子宣化同知國均迎喪還京師,使內大臣佟國維迎奠盧溝橋,贈兵部尚書,謚忠果。建祠貴陽,上賜「勁節」二字顏其額。子七,國璧尤知名。

國璧,字東屏,以任子授陜州知州,改蘇州同知,擢山西平陽、浙江寧波知府,名循吏。聖祖南巡,幸杭州,御書硃子詩及「永貞」額以賜。諭曰:「汝父盡節,朕未嘗忘,此為汝母書也。」累遷雲南巡撫。坐事罷。雍正間,起為正黃旗漢軍都統。乾隆三年,授綏遠城右翼副都統。復罷。十二年,卒。

範承謨,字覲公,漢軍鑲黃旗人,文程次子。順治九年進士,選庶吉士,授弘文院編修。累遷秘書院學士。康熙七年,授浙江巡撫。時去開國未久,民流亡未復業,浙東寧波、金華等六府荒田尤多。總督趙廷臣請除賦額,上命承謨履勘。承謨遍歷諸府,請免荒田及水沖田地賦凡三十一萬五千五百餘畝。杭州、嘉興、湖州、紹興四府被水,民饑,承謨出布政使庫銀八萬,糴米湖廣平糶,最貧者得附老弱例,肩鹽給朝夕,全活甚眾。並疏請「漕米改折,石銀一兩。明年麥熟,補徵白糧,以三年帶徵。災重者如例蠲免」。得旨允行。十年,以疾請解職,召還。總督劉兆麒、提督塞白理疏言浙民請留承謨一百五十餘牒,給事中姜希轍、柯聳,御史何元英等亦言:「承謨受事三載,愛民如子,不通請謁饋遺。劾罷貪墨,廉治巨猾,剔除加耗、陋規、私派諸弊。浙民愛戴,深於饑渴。」上命承謨留任。十一年,承謨復疏言湖、嘉兩府白糧加耗,多寡不一,請每石加四斗五升為限;又奏蠲溫、臺二衛康熙九年以前逋賦及石門、平陽未完輕齎月糧:皆下戶部議行。

十月,擢福建總督,疏辭未允,請入覲。十二年七月,至京師,入對。承謨疾未愈,命御醫診視,賜藥餌。疾稍差,趣赴官,賜冠服、鞍馬。福建總督初駐漳州,至是以將撤籓,命移駐福州。吳三桂反,承謨察精忠有異志,時方議裁兵,承謨疏請緩行。又報巡歷邊海,欲置身外郡,便徵調防禦。事未行而精忠叛,陽言海寇至,約承謨計事。巡撫劉秉政附精忠,趣同行。承謨知有變,左右請擐甲從,承謨曰:「眾寡不敵,備無益也。」遂往。精忠之徒露刃相脅,承謨挺身前,罵不絕口。精忠拘之土室,加以桎梏,絕粒十日,不得死。精忠遣秉政說降,承謨奮足蹴之僕,叱左右掖之出,曰:「賊就僇當不遠,我先褫其魄!」為賊困逾二年,日冠賜冠,衣辭母時衣,遇朔望,奉時憲書一帙懸之,北鄉再拜。所居室迫隘,號曰蒙穀。為詩文,以桴炭畫壁上。

時有部曲張福建者,手刃奪門入,連斬數賊,力竭死。蒙古人嘛尼為偽散騎郎,精忠遣守承謨,感承謨忠義,謀令出走。事洩,精忠將磔之,大言曰:「吾寧與忠臣同死,不原與逆賊同生!」

十五年,師克仙霞關,精忠將降,冀飾詞免死,懼承謨暴其罪。九月己酉朔,甲子夜半,精忠遣黨偪承謨就縊。幕客嵇永仁、王龍光、沈天成,從弟承譜,下至隸卒,同死者五十三人。語互詳忠義傳。舊役王道隆奉遣他出,還至延平,聞變,自刎死。賊焚承謨尸,棄之野,泰寧騎兵許鼎夜負遺骸藏之。十六年,喪還京師。上遣內大臣侍衛迎奠,贈兵部尚書、太子少保,謚忠貞,御書碑文賜其家。十九年,精忠伏誅。赴市曹日,承謨子時崇臠其肉祭墓。福建民請建祠祀之,御書「忠貞炳日」扁於楣。承謨所為畫壁集,上親制序。

時崇,字自牧。以難廕出知遼陽州,遷直隸順德知府,有惠政。累遷福建按察使。陛辭日,上顧謂群臣曰:「此開國名臣孫,殉難忠臣子也!」四十七年,擢廣東巡撫,兼鹽政。越二年,擢福建浙江總督。五十四年,入為左都御史。明年,授兵部尚書。命出塞築莫代察罕廋爾、鄂爾齋圖杲爾臺站凡四十有七所。又明年,還朝。尋卒。閩人思其德,附祀承謨祠。

馬雄鎮,字錫蕃,漢軍鑲紅旗人,鳴佩子。以廕補工部副理事官,歷遷左僉都御史、國史院學士。康熙八年,授山西巡撫。未上,改廣西。時群盜蝟起,構瑤、僮掠梧州、平樂二府,不數月討平之。累疏請平鹺價,建學宮,定有司邊俸,省軍糧運費,並罷諸採買累民者,皆得旨允行。

十二年,吳三桂反。十三年,孫延齡以廣西叛應之,圍雄鎮廨,脅降。時巡撫無標兵,雄鎮督家人拒守。密令守備易友亮赴柳州趣提督馬雄來援,弗應。雄鎮自經,為家人救免,以蠟丸馳疏請兵。延齡詗知之,幽雄鎮,置家人別室。三桂使招降,雄鎮不為屈。會傅弘烈勸延齡反正,延齡躊躇未決,雄鎮得以間遣長子世濟齎疏詣京師,友亮導之出,客楊啟祥護行,至贛州,江西巡撫董衛國以聞。上遣使護入京,至,授世濟四品京卿。居數月,雄鎮又具疏陳粵西可復狀,付長孫國楨,俾與客硃昉鑿垣出。既,又遣州人唐守道、唐正發潛負次子世永出,次第詣京師。又為延齡知,系其孥於獄。雄鎮憤自剄,復為賊所奪,幽之別室。

十六年十月,三桂遣其從孫世琮殺延齡,擁雄鎮至賊壘,迫使降,雄鎮大呼曰:「吾義守封疆,不能寸斬汝以報國,死吾分也!」賊戕其幼子世洪、世泰怵之,罵益厲,賊殺之,時年四十有四。從者馬云皋、唐進寶、諸兆元等九人同時死,妻李,妾顧、劉,女二人,世濟妻董、妾苗,並殉。語互詳列女傳。雄鎮尸暴四十餘日,友亮收其骸骨,槁葬焉。

雄鎮被縶三年,日著書賦詩。既死,客孫成、陳文煥乘間脫走,抵蒼梧,以所著擊笏樓遺稿及匯草辨疑歸世濟。十七年,弘烈以雄鎮死狀入告,命議恤。擢世濟大理寺少卿。成以舉人授同知,文煥授知縣。旋又授友亮、守道、正發、啟祥游擊、守備有差。十八年,世濟如廣西迎雄鎮喪至京師,贈太子太傅、兵部尚書,謚文毅。三桂既平,歲正,上宴群臣,特命世濟及陳啟泰子汝器至御座前賜酒。世濟官至遭運總督,世永歷運使;國楨官江南常鎮道,督餉入藏,卒於軍。

傅弘烈,字仲謀,江西進賢人。明末,流寓廣西。順治時,以總督王國光薦,授韶州同知,遷甘肅慶陽知府。

吳三桂蓄逆謀久,康熙七年,弘烈密以告,逮治,坐誣,論斬。九年,上特命減死戍梧州。及三桂反,將軍孫延齡、提督馬雄以廣西叛應之。弘烈欲集兵圖恢復,陽受三桂偽職,入思州、泗城、廣南、富川諸土司,歷交阯界,募義軍得五千人,遂移檄討賊,從尚可喜軍規肇慶。三桂甚惎之,使馬雄如柳州害其家百口。弘烈說延齡反正。鎮南將軍覺羅舒恕軍贛州,弘烈密致書言延齡妻孔四貞,定南王有德女,未忘國恩,延齡可招撫。又致書奉詔招撫督捕理事官麻勒吉,言王師速進南安,弘烈自韶州策應,則兩粵可定。舒恕、麻勒吉先後以聞,上嘉其忠誠,授廣西巡撫、征蠻滅寇將軍,俾增募義兵,便宜行事。

弘烈克梧州,下昭平、賀、鬱林、博白、北流、陸川、興業諸州縣,進復潯州,遣平樂知府劉曉齎疏上方略。論功,加太子少保,並加曉參議道。當是時,馬雄據柳州,三桂諸將分據平樂、南寧、橫州,勢洶洶。弘烈雖屢捷,惟新軍缺砲馬,假於尚之信,弗應。吳世琮既殺延齡,陷平樂,襲弘烈梧州,弘烈擊敗之。十七年,與將軍莽依圖圍平樂,戰失利,弘烈與互訐。詔謂弘烈兵未支俸餉,奮勇收復諸路。莽依圖自平樂退賀縣,又言糧乏,再退梧州,使弘烈所復郡縣盡棄於賊,因飭莽依圖圖效。弘烈督兵進,賊數萬渡左江,弘烈戰敗。賊陷藤縣,逼梧州。十八年,之信軍至,弘烈分兵水陸,乘賊攻城時三面夾擊,賊潰走,遂下藤縣,克平樂,進復桂林。

弘烈密疏言延齡舊部宜善為解散,又言之信怙惡反覆,當早為之所。馬雄死,子承廕仍附三桂,受偽封懷寧公,詭言乞款附,弘烈許之,為疏聞。詔授承廕昭義將軍統其眾。弘烈規取雲、貴。十九年二月,次柳州,承廕期弘烈會議,弘烈至,承廕以其眾叛,襲破其營,執送貴陽。世璠誘以偽職,弘烈曰:「爾祖未反時,吾已劾奏,料汝家必為叛逆。汝敢以此言污我邪?」世琮百計說之,罵益厲。十月辛丑,遇害。十一月,征南將軍穆占復貴陽,收遺骸,以死狀聞,贈太子太師、兵部尚書,謚忠毅。二十二年,允廣西巡撫郝浴請,建雙忠祠於桂林,祀弘烈及馬雄鎮。

論曰:方諸籓盛強,朝廷所置督撫,勢不足以相抗。文焜雖與三桂分疆而治,所部貳於三桂久矣。若承謨之於精忠,雄鎮之於延齡,皆同城逼處,惟以身殉,無他術也。弘烈異軍特起,又與莽依圖相失,勢孤,遂困於承廕。要其忠義激烈,作士氣,怵寇心,皆不為徒死者。嗚呼,烈已!


\end{pinyinscope}