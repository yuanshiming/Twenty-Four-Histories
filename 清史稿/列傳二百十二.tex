\article{列傳二百十二}

\begin{pinyinscope}
李僡吳棠英翰劉蓉喬松年錢鼎銘吳元炳

李僡,字惠人,陜西華陰人。道光二年進士,直隸即用知縣,補撫寧,調青縣。舉卓異,歷滄州、深州,擢大名知府。調保定,擢大順廣道,遷按察使。二十一年,擢順天府尹。二十三年,南河決,命偕侍郎成剛馳往督工。二十六年,出為江蘇布政使,以病歸。三十年,起授甘肅布政使。咸豐元年,擢河南巡撫。長蘆鹽政疲敝,言官請變通懸岸,僡偕直隸總督訥爾經額議改直、豫懸岸,分別官辦、商販。二年,調山東。粵匪由武昌東下趨江寧,僡遣精兵二千馳援,親赴兗、沂、曹諸府察形勢,分兵扼隘防守。履行河堤,令黃河渡船悉歸曹縣劉家口、單縣董家口,斷他口私渡。檄候補道慶凱等駐兵要隘,搜捕捻匪。

三年,江寧陷,徐州捻、梟諸匪蜂起,僡再赴兗、沂、曹諸府督防。未幾,揚州陷,僡令防軍分三路:游擊王鳳祥等駐郯縣紅花埠為東南路,總兵百勝等駐嶧縣韓莊閘及陰平為中路,總兵三星保巡劉家、董家二口,遏賊北竄,為西南路。傳駐宿遷迤北,與百勝等犄角。四月,賊自浦口北竄安徽,陷滁州,逼鳳陽臨淮關。僡進駐宿遷,慮徐州守兵弱,請移山西、陜西、綏遠諸路援兵策應。五月,賊自亳州經米家集竄河南,陷歸德、擾劉家口。僡命防軍進擊,民團繼之,毀北岸船,賊不得渡。有由曹河駛入者,乘半渡擊沉之,賊敗退。尋自河南汜水北渡溫縣,西路告警,僡自曹州分兵馳援,督師繼之。比賊圍懷慶,僡會諸軍力戰,解其圍。捻匪擾歸德境,毗連曹、單、僡留陜、甘兵九百會剿,自引師回防東路。

自粵匪起,所至各行省皆瓦解,疆吏能禦賊不使入境且出境剿賊者,惟僡一人。文宗深嘉之,屢欲擢任總督,以山東為畿輔屏蔽,倚僡為重,故未果。尋卒於官,優詔悼惜,贈總督、太子少保,謚恭毅。子啟詔,署湖南桂陽州,殉難,贈道銜。

吳棠,字仲宣,安徽盱眙人。道光十五年舉人,大挑知縣,分南河,補桃源。調清河,署邳州。山東捻匪入境,率團勇擊走之,還清河。咸豐三年,粵匪陷揚州,時圖北竄,棠招集鄉勇,分設七十二局,合數萬人,聯絡鄰近十餘縣,合力防禦,有聲江、淮間。丁母憂,士民攀留,河道總督楊以增疏請令治喪百日後,仍署清河。太常寺少卿王茂廕疏薦,詔詢以增,亦以治績上,特命以同知直隸州即補,賜花翎。六年,丁父憂,仍留江蘇,以剿匪功,累擢以道員即補。十年,補淮徐道,命幫辦江北團練。皖北捻匪出入,以徐、宿為孔道,山東土匪時相勾結,一歲數擾,棠督軍屢擊走之。

十一年,擢江寧布政使,署漕運總督,督辦江北糧臺,轄江北鎮、道以下,令總兵龔耀倫等破賊於阜寧、山陽,解安東圍。漕督舊駐淮安府城,棠以清江浦地當沖要,築土城駐之。捻匪大舉來撲,督軍力戰擊退,賊踞眾興集相持,令驍將陳國瑞進攻,戰十日,大破之,賊遁泗州。督屬縣築圩寨,堅壁清野,收撫海州,贛榆土匪,先後遣將擊捻匪,擒李麻子於曹八集,斬何申元於洞裏莊,殲卜里於半截樓,又破山東幅匪於郯城徐家圩、鎰陽集、長城等處。

同治二年,實授漕運總督。令陳國瑞進剿沂州,迭殲渠魁,國瑞遂隸僧格林沁軍。苗沛霖叛陷壽州,棠令總兵姚慶武、黃開榜水陸赴援。疏言:「欲拯臨淮之急,必須一軍由宿、蒙直搗懷遠,使苗逆急於回顧,臨淮始可保全。削平之策,尤須數道進兵,方能制其死命。」又密陳:「皖北隱患,淮北鹽務疲敝,悉由李世忠盤剝把持,其勇隊在懷、壽一方盤踞六年,焚掠甚於盜賊。苗平而淮北粗安,李存而淮南仍困,請早為之計。」詔下僧格林沁等籌辦。

三年,加頭品頂戴,署江蘇巡撫。四年,調署兩廣總督。棠疏陳:「江境尚未全平,請收回成命,專辦清淮防剿。」詔嘉其不避難就易,仍留漕督任。軍事初定,即籌復河運。署兩江總督,未幾,回任。五年,調閩浙總督。

六年,調四川總督。時蜀中軍事久定,養兵尚多,而協濟秦、隴、滇、黔,歲餉不貲。棠令道員唐炯剿貴州龍井苗匪,復麻哈州。道員張文玉等克黃平州,疏請遣周達武一軍入黔助剿,即調達武貴州提督,餉仍由四川任之。平苗之役,賴其力焉。

八年,雲貴總督劉岳昭劾棠赴川時僕從需索屬員饋送,言官亦劾道員鍾峻等包攬招搖,命湖廣總督李鴻章往按。鴻章覆奏:「川省習尚鉆營,棠遇事整頓,猾吏造言騰謗。」詔責棠力加整飭,勿稍瞻顧,斥岳昭率奏失實,惟坐失察鍾峻等薄譴。十年,署成都將軍,奏撥捐輸銀二十萬兩賑饑民。十三年,雲南、貴州軍事先後肅清,以協餉功被優敘。灌縣山匪作亂,令提督李有恆剿平之,斬其渠餘其隆。疏言:「部章新班遇缺先人員補官較易,服官川省者,報捐不惜重利借貸,其中即有可用之才,夙累既重,心有所分,官債雖清,民生必困。請敕部另議變通,俾試用甄別年滿、歷練較久諸員,得有敘補之期,實於吏治有益。」

光緒元年,剿敘永匪及雷波叛蠻,平之。以病乞罷。二年,卒,詔優恤,謚勤惠。

英翰,字西林,薩爾圖氏,滿洲正紅旗人。道光二十九年舉人。咸豐四年,揀發安徽,以知縣用。九年,署合肥。粵匪擾皖北,督鄉團擊敗之。又破賊華子岡、小河灣,擢同知。十一年,署宿州。同治元年,捻匪來犯,英翰偕總兵田在田克高黃山寨,進破湖溝,擢知府,賜花翎。二年,捻首張洛行為僧格林沁大軍所敗,回老巢,英翰擊敗之於青甿。會攻克雉河集,英翰授策降人,擒洛行送僧格林沁大營誅之,授潁州知府。巡撫唐訓方及袁甲三交章薦英翰沉勇有謀,剿水會北匪圩功多。苗沛霖復叛,攻蒙城、壽州,英翰督兵攻克蒙城附近賊圩,又擊敗沛霖所遣攻壽州兵。會總兵姚廣武破韓村賊寨,攻狼山,賊棄壘遁,蒙城糧道始通。署廬鳳道,擢按察使。復督兵援蒙城,攻蔡家圩,斷賊糧道,遣參將程文炳等四出截擊,夷賊壘數十。僧格林沁、富明阿諸軍先後至,大破賊,沛霖就殲,附沛霖諸圩盡克,賜號格洪額巴圖魯。

三年,粵匪合捻匪由陜南竄湖北,將遙為江寧聲援,其鋒甚銳。僧格林沁調英翰赴援,賊方圍麻城,襲破柏子塔賊屯,賊渠陳得才等自白臬走閻家河,英翰督軍迎戰,破之。尋以請獎冒濫,奪勇號,降五級留任。賊自松子關竄皖境,巡撫喬松年奏調英翰回援,克金家寨。賊竄六安、青山,會諸軍擊走之。群賊麕聚英山、霍山,連破之於樂兒嶺、土門、黑石渡。時江寧已下,僧格林沁大軍進逼,賊皆攜貳,陳得才仰藥死。馬融和有眾數萬,英翰令郭寶昌招致之。賊首藍成春亦降,餘小頭目紛紛求撫。僧格林沁以成春乃粵中老賊,斬之以徇。未至者遂散走,而張總愚、牛洛紅、任柱、賴文光等勾結復熾。論功,英翰復賜號鏗僧額巴圖魯,擢安徽布政使。

四年,捻匪自河南竄山東,僧格林沁戰歿,遂大舉犯安徽,覬復踞蒙、宿舊巢,英翰屯雉河集,為賊所圍。道員史念祖佐英翰且戰且守,凡四十五日,援軍至,突圍夾擊,大破之,賊乃解圍引去,晉號達春巴圖魯。五年,就擢巡撫。前撫喬松年調陜西,剿西捻張總愚,以皖軍郭寶昌從行,其餉仍由英翰籌供。東捻由固始犯皖境,皖軍扼之,復竄麻城,英翰率軍防六安。六年,賊復由楚、豫入山東,方議就運河築長圍圈賊,英翰分撥皖軍,令黃秉鈞扼宿遷,張得勝扼貓兒窩灘,程文炳以騎兵備游擊,餘承先率水師由洪湖入運河,捻勢漸蹙。英翰丁父憂,予假一月治喪,改署任。是年冬,捻首任柱為淮軍所殲,餘黨散撲運河,皖軍截擊,收降數千人,賴文光走揚州就擒。東捻平,論功,予三等輕車都尉世職。

再疏請終制,報可,而西捻渡河北犯。七年春,畿輔戒嚴,英翰率軍馳援,命駐河南。英翰奏以所部交河南巡撫李鶴年調遣,請回旗守制,詔慰留之。遂會諸軍圍賊於運河東,捻眾聚殲,加太子少保,辭,不許。八年,回旗營葬,請留京,予假兩月,期滿仍回任。十年,於亳州捕叛捻宋景詩,誅之。

十三年,擢兩廣總督。粵匪悍酋楊輔清敗逸後,猶潛匿福建晉江,令降將馬融和等往捕,至是始就擒,奏請誅之。光緒元年,入覲,晉二等輕車都尉世職。廣東闈姓捐奉旨嚴禁,英翰奏請弛禁助餉,又因隨員招搖,為廣州將軍長善等所劾,召還京,被議,褫職。未幾,命還世職,以二品頂戴署烏魯木齊都統。二年,實授。尋卒,贈太子太保,復勇號,賜恤,謚果敏。安徽省城及鳳陽、壽州、宿州、阜陽、蒙城、渦陽並立專祠,賜其母銀二千兩,人葠六兩。無嗣,弟英壽襲世職。

劉蓉,字霞仙,湖南湘鄉人。諸生。少有志節,與曾國籓、羅澤南講學。軍事起,佐澤南治團練。咸豐四年,從國籓軍中,既克武昌,轉戰江西。五年,澤南由江西回援湖北,蓉從之,領左營。弟蕃,戰歿於蒲圻,蓉送其喪歸,遂辭軍事。尋丁父憂,胡林翼奏徵之,不出。十一年,駱秉章督師四川,聘參軍事,疏薦其才,詔以知府加三品頂戴,署四川布政使,尋實授。秉章於軍事吏治,悉倚蓉贊助,亦時出視師,藍、李諸匪以次削平。事詳秉章傳。

同治元年,石達開由滇、黔邊境入四川。預調諸軍羅布以待,秉章令蓉赴前敵督戰,達開不得逞,徘徊於土司地,窮蹙就擒。蓉親往受俘,檻送成都誅之,被旨嘉獎。時粵、捻諸匪藍成春、陳得才等竄擾陜南,踞漢中、城固等城,川匪餘孽亦入陜蔓延,勢方熾。多隆阿督師關中,注重北路回匪,於南路未能兼顧。官文疏薦蓉堪當一面,於是命蓉督辦陜南軍務,擢陜西巡撫。秉章分兵四千授蓉,總兵蕭慶高、何勝必兩軍先赴援,亦隸之。又遣將赴湖南增募萬人,蓉於十月進屯廣元。三年春,漢中粵、捻諸匪因江寧被圍急,促其回援,遂自退,趨湖北。蓉入漢中部署屯防,清餘匪。

多隆阿圍盩厔久未下,聞蓉將至,督攻益急,克之,而多隆阿受重傷。三月,蓉抵省城,多隆阿尋卒於軍,其所部雷正綰、陶茂林諸軍剿西路回匪,入甘肅;穆圖善一軍議令赴援湖北。五月,川匪合粵、捻由鎮安、孝義突犯省城,蓉集諸軍擊之於鄠、盩厔之間,尋偕穆圖善會擊於郿縣,賊西走略陽,入甘肅,陷階州,令何勝必等會川軍周達武攻之。四年,克階州,川匪餘孽悉平。雷正綰軍譁變,其部將胡士貴率叛兵回擾涇州,蓉遣軍扼隘,散其脅從,誅士貴。

會編修蔡壽祺疏劾恭親王奕,牽及蓉,指為夤緣,詔詰蓉令自陳。蓉奏辦,自言薦舉本末,並訐壽祺前在四川招搖,擅募兵勇,為蓉所阻,挾嫌構陷。復為內閣侍讀學士陳廷經所劾,命大學士瑞常、尚書羅惇衍按究,坐漏洩密摺,降調革任。陜甘總督楊岳斌疏言陜西士民為訴枉乞留,詔蓉仍署巡撫。

五年,奏薦賢能牧令龔衡齡等,請予升階,下部議駮。蓉疏言:「近來登進之途,多出於從軍,而究心民瘼者,仍潦倒於下吏。陜西瘡痍未起,急應旌舉賢能以為之勸。」上特允之。先是,蓉任鳳邠道黃輔辰經理回民叛產,設法墾治,歲獲穀數百萬斛,成效甚著,因奏:「陜西兵後荒蕪,以招徠開墾為急務。應視兵災輕重,荒地多少,以招墾成數為州官吏勸懲。」報可。尋以病乞開缺,上允其請,以喬松年代之,仍留陜西治軍。捻匪張總愚入陜,逼省城,蓉與松年議不合,所部楚軍三十營,統將無專主,士無戰心,屯灞橋,為賊所乘,大潰。詔斥蓉貽誤,奪職回籍。十二年,卒。湖南巡撫王文韶疏聞,命復官,陜西請祀名宦祠。

喬松年,字鶴儕,山西徐溝人。道光十五年進士,授工部主事,再遷郎中。咸豐三年,以知府發江蘇,除松江,調蘇州。會匪劉麗川據上海,省城潮勇潛與通,松年偵知之,白上官誅其為首者。丁父憂,總督怡良奏留,從克上海,擢道員,賜花翎,授常鎮通海道。六年,從怡良駐常州,署兩淮鹽運使。八年,丁本生父憂,總督何桂清復奏留。

九年,授兩淮鹽運使,兼辦江北糧臺。十年,奏劾南河河道總督庚長擅提淮北存鹽變價充餉,又截留山西解江北糧臺餉銀;復劾庚長在清江聞警猶演劇設宴,迨寇急,倉皇退守。命侍郎文俊往按得實,庚長褫職逮問。又疏論用勇不如用兵,請發京師護軍營暨北五省綠營赴江北防剿。英吉利、法蘭西兵入犯,京師戒嚴,松年請赴畿輔督兵禦敵,諭止之。十一年,設江南北兩糧臺,仍命松年辦理。敘勞,以按察使記名。

同治二年,擢江寧布政使,仍留辦糧臺,擢安徽巡撫。三年,抵任,駐防臨淮。時苗匪已平,李世忠亦解兵柄,捻匪竄河南、湖北。松年增募勇千人,就潁、宿間設防,奏請雉河集地處交沖,當建縣設官,從之。又奏苗沛霖餘黨自非積惡,請予寬貸;李世忠散遣勇丁,恐流為盜,飭州縣整頓捕務。粵、捻諸匪自湖北麻城、羅田東竄入皖境,松年移軍壽州,急調英翰自湖北回援,令硃淮森屯正陽關,蔣凝學迎擊於英山,克金家寨。英翰等敗賊於陶家河、黑石渡,僧格林沁大軍追至合擊,諸賊窮蹙,紛紛乞降,先後凡十餘萬。賊首陳得才後至,為蔣凝學擊敗,服毒死,獲其尸。上飭英翰等移軍進剿,松年請留英翰防皖境,郭寶昌援河南,蔣凝學赴湖北。

四年,僧格林沁戰歿,上命曾國籓督師山東。松年奏:「國籓久治軍務,氣體較遜於前。李鴻章才識亞於國籓,而年力正強,如以代國籓督師山東,必能迅奏蕩平。」疏上,報聞。時捻匪大舉犯皖北,圍英翰於雉河集,國籓遣援軍至,乃擊走之。

五年,調陜西巡撫,前任巡撫劉蓉奉命留陜辦理軍務。時捻匪張總愚竄入陜境,松年初至,與蓉意見不合,奏劾蓉軍政隳壞,留陜無益,蓉亦劾松年掣肘,貪利徇私。十二月,賊逼省城,蓉軍潰於灞橋。六年正月,提督劉松山援軍至,破賊雨花寨,連戰皆捷,省城始安。迭奏請師,鮑超軍援陜迄不至,皖軍郭寶昌應調來援,偕劉松山轉戰涇、渭之間,屢捷。總愚窺同州,欲渡河,未得逞,趨陜北。六月,總督左宗棠至陜,軍事始有統轄。松山、寶昌等連破賊於北路,至冬,總愚由垣曲渡河,循太行東趨,松山、寶昌尾追。七年春,宗棠率師入衛畿輔,陜西自捻匪出境,西路回氛仍未靖,松年以病乞假歸。九年,病痊,授倉場侍郎。

十年,授河東河道總督。奏言:「今日言治河,不外兩策:一則堵銅瓦廂決口,復歸清江浦故道;一則就黃水現到處築堤束之,俾不至橫流,至利津入海。權衡輕重,以就東境築堤束黃為順水之性,事半功倍。前數年大溜全趨張秋,後又決胡堰、洪川口、霍家橋、新興屯諸地,黃流穿運,節節梗阻。惟有盡堵旁洩之路,自張秋西南,沙河迤北,就舊堤修補,為黃河北堤;又自張志門起,至沈家口、馬山頭,築新堤一百八十餘里,為黃河南堤:俾仍全趨張秋,借以濟運。」下廷臣議行。十三年,奏請裁東河總督,以巡撫兼領河工,下部議,格不行。光緒元年,卒,謚勤恪。

錢鼎銘,字調甫,江蘇太倉人。父寶琛,湖北巡撫。鼎銘,道光二十六年舉人,從寶琛治團練。會匪劉麗川據上海,青浦周立春起應之,陷嘉定,鼎銘與嘉定舉人吳林募勇從官軍復其城,授贛榆訓導。入貲為戶部主事,丁父憂歸。江南大營再潰,諸郡縣淪陷,巡撫薛煥退保上海一隅。曾國籓既克安慶,團練大臣龐鍾璐等議乞援,道路梗阻。鼎銘奮然請行,乘洋商輪船溯江上,至安慶謁國籓,陳吳中百姓阽危,上海中外互市,榷稅所入,足運兵數萬,不宜棄之資賊。策畫數千言,繼以痛哭,國籓遂決策濟師。時薛煥遣將至湖南募勇萬二千,國籓知所募皆各軍汰遺,不可用,令鼎銘往解散。遇諸漢口,鼎銘簡留精壯九百人,餘悉罷歸,無譁者。還上海,籌餉十八萬,租船五,復率赴安慶迎師。於是國籓奏令延建邵道李鴻章率淮勇五千人赴之。同治元年三月,至上海,鴻章尋署江蘇巡撫,奏請以鼎銘參軍事,多所贊畫。積功,擢道員,賜花翎,加布政使銜。

五年,鴻章代國籓督師剿捻匪,令鼎銘駐清江浦,主轉運糧餉軍仗,迄捻匪滅,始終無絀誤。鴻章與漕運總督張之萬累疏薦。國籓移督直隸,奏調以從。八年,授大順廣道,就遷按察使,又遷布政使。十年,擢河南巡撫。十一年,捻匪餘孽蠢動,鼎銘令總兵崔廷桂剿平之。用直隸練軍制,就河南三鎮額兵,簡其精壯,抽練馬步各三營,重其額餉,擇駐沖要地訓練,期年成軍。修水利,鑿賈魯河故道,南自周家口,北至硃仙鎮,又西北至鄭州京水寨,疏積沙,補殘堤,俾上游無水澇,下游通舟楫。復濬勺金河、丈八溝、餘濟河、永豐渠以資灌溉。令諸州縣勸民按畝出穀,就鄉分倉,擇公正紳耆董其事,毋假手胥吏,通省積穀九十餘萬石。提督張曜一軍出關剿回,全軍餉由河南供給無缺。光緒元年,卒,賜恤,謚敏肅。

吳元柄,字子建,河南固始人。咸豐十年進士,選庶吉士。從團練大臣毛昶熙回籍治團練,從解固始圍,擊退息縣竄匪,擒捻首陳得一。十一年,汝寧捻首陳大喜竄居霍莊寨,元炳偕道員張曜攻克之。同治元年,巡撫嚴樹森奏:「元柄驍捷善戰,所向有功,軍中最得力,請散館後仍令回河南。」命免散館授檢討,仍留河南委用。大喜負固平輿,其黨踞李旗屯,元炳偕張曜先平伊莊、陳莊、劉樓賊壘,乘勝下李旗屯,進攻楊樓,破之。旋克平輿,殲捻首張鳳林。二年,克張岡賊巢,汝南肅清,擢侍講。尋攻息縣鮑家寨,克之。三年,拔譚家圩,附近賊寨,次第削平。

丁母憂,回籍,巡撫張之萬奏起赴軍。四年,以汝、光諸地稍定,請終制,允之。六年,補原官。九年,超擢侍講學士。十年,命署湖南布政使。十二年,擢湖北巡撫,調安徽,再調江蘇。光緒二年,疏陳:「銀捐新例,新班遇缺先及遇缺兩項,得缺最速,流弊亦多,於政體大有關系,不可不嚴防其弊。請明定章程,變通辦理。」下部議行。山東、安徽比歲饑民流及淮、揚,元炳截漕撫恤,並疏高寶河、鹽運河,以工代賑。署兩江總督者三,兼署江蘇學政者一。七年,丁本生母憂,去官。十年,入覲,命察山東河工、海防,授漕運總督。十一年,調安徽巡撫。十二年,卒,賜恤。河南巡撫倪文蔚疏陳元炳戰功,遺愛在民,請於汝寧建專祠。

論曰:李僡守山東,吳棠保江淮,當時皆負時望。英翰剿捻,戰績最多,及任皖疆,甚得民心。劉蓉抱負非常,佐駱秉章平蜀,優於謀略而短於專將,治陜不竟其功。喬松年在皖倚用英翰而奏績,在陜不能與劉蓉和衷,徒促僨事。錢鼎銘慷慨乞師,為平吳之引導,治豫亦有聲。吳元炳以詞林事軍旅,其際遇特異焉。


\end{pinyinscope}