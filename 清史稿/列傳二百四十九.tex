\article{列傳二百四十九}

\begin{pinyinscope}
丁汝昌衛汝貴弟汝成葉志超

丁汝昌,字禹廷,安徽廬江人。初隸長江水師,從劉銘傳征捻,積勛至參將。捻平,賜號協勇巴圖魯,晉提督。光緒初,留北洋差序,赴英國購兵艦,歷法、德各營壘廠局,還綜水師。八年,朝鮮與美議互市,請蒞盟,汝昌與道員馬建忠東渡監約。既而朝軍譁變,焚日使署,遂率濟遠、揚威二艦赴仁川、漢城護商,而日軍已先至,汝昌還請益師。隨統七艦以濟,薄王京,與吳長慶及建忠謁李應罡,執以歸。九年,授天津鎮總兵。會越南南定陷,乘兵艦往江平及欽州白龍尾,徼循海口,賞黃馬褂。十四年,定海軍經制,命為海軍提督。軍故多閩人,汝昌以淮軍寄其上,恆為所制。總兵以下多陸居,軍士亦去船以嬉,又值部議停購船械,數請不獲,蓋海軍廢弛久矣。二十年,賞加尚書銜。

朝亂再起,汝昌欲至濟物浦先攻日艦,將啟行,總署電柅之。逮日艦縱橫海上,海軍始集大東溝、鴨綠江口。定遠為汝昌座船,戰既酣,擊沉其西京丸一艘。已,致遠彈藥盡,被擊,總兵鄧世昌戰死。自是連喪五艦,不復能軍。汝昌猶立望樓督戰,忽座船砲震,暈而僕,舁以下。汝昌鑒世昌之死,慮諸將以輕生為烈,因定海軍懲勸章程,李鴻章上之,著為令。旅順陷,汝昌渡威海,是時兩軍相去二百二十餘里,朝士爭彈之,褫職逮問。鴻章請立功自贖,然兵艦既弱,坐守而已。

逾歲,日軍陷榮城,分道入衛。汝昌亟以木簰塞東西兩口,復慮南岸三臺不守、砲資敵,欲毀龍廟嘴臺砲,陸軍統將戴宗騫電告鴻章,責其通敵誤國,不果毀。待援師不至,乃召各統領力戰解圍。會日暮大風雪,汝昌盡毀緣岸民船,而南北岸已失,日艦入東口猛攻,定遠受重傷,汝昌命駛東岸,俄沉焉,軍大震,競向統帥乞生路,汝昌弗顧,自登靖遠巡海口。日艦宵入口門,擊沉來遠、威遠,眾益恐。道員牛昶炳等相鄉泣,集西員計議。馬格祿欲以眾挾汝昌,德人瑞乃爾潛告曰:「眾心已變,不如沉船夷砲臺,徒手降,計較得。」汝昌從之,令諸將同時沉船,不應,遂以船降,而自飲藥死,於是威海師熸焉。事聞,諸將皆被恤,汝昌以獲譴,典弗及。宣統二年,海軍部立,舊將請賜恤,始復官。

衛汝貴,字達三,安徽合肥人。從劉銘傳征捻,累遷至副將,晉總兵。事平,授河州鎮,李鴻章薦其樸誠忠勇,留統北洋防軍。歷授大同、寧夏諸鎮,均未之官,統防軍如故。

光緒二十年,日朝戰起,率馬步六千餘人進平壤,臨行,鴻章誡以屏私見,嚴軍紀。至牙山,退成歡,與日軍相見,尋復趨平壤合大軍,與副都統豐紳阿頓守城南江岸。平壤,朝舊京也,聞我軍至,爭攜酒漿以獻;而軍士多殘暴,掠財物,役丁壯,淫婦女,汝貴軍尤甚,殺義定朝民,眾滋忿。復蝕軍糈八萬運家,軍大譁,連夕自亂,互相蹈藉。時馬玉昆血戰大同江,浮舟往援,敵稍卻。玄武門嶺失,即竄走。鴻章方據葉志超牒奏捷,俄而安東、鳳凰陷,踉蹌走岫巖,岫巖陷,走奉天。朝士交章糾其罪,詔褫職逮問。汝貴治淮軍久,援朝時年已六十矣。其妻貽以書,戒勿當前敵,汝貴遇敵輒避走。敗遁後,日人獲其牘,嘗引以戒國人。明年,金巢送京師,按實,論死。

其弟汝成官至總兵。援旅順,六統帥不相轄,汝成與趙懷益爭毆,鴻章函責之。逮日軍至,姜桂題等猶力禦,而汝成已先遁。詔逮治,未蹤獲,乃籍其家。後不知所終。

葉志超,字曙青,安徽合肥人。以淮軍末弁從劉銘傳討捻,積功至總兵。戰淮城被創,仍奮擊卻之,逐北天長,又敗之汊河,賜號額圖渾巴圖魯。規南樂,戰德、平間,頻有功。捻平,留北洋。光緒初,署正定鎮總兵,率練軍守新城,為大沽後路。後徙防山海關,李鴻章薦其優智略,予實授。十五年,擢直隸提督。越二年,熱河教匪亂,志超率師討之。平建昌,連克榆林、沈家窩館、貝子廟,釋下長皋圍,進攻烏丹城,擒其渠李國珍磔之,賞黃馬褂、世職。

二十年,朝鮮乞師,鴻章令選練軍千五百,率太原總兵聶士成頓牙山。志超遲留不進,鴻章責之,不得已啟行。而日軍已據王京要隘,牙山兵甚單,駐朝商務委員袁世凱數約志超電請北洋發戰艦赴仁川,增陸軍駐馬坡。鴻章始終欲據條約,恐增兵為彼藉口,勿許,並戒志超毋啟釁。亡何,高升商輪運兵近豐島,被擊沉。士成謂志超曰:「海道既梗,牙山絕地,不可守。公州背山面江,勢便利,戰而勝,可據以待援;不勝,猶得繞道出也。」志超從之。日軍偪成歡,士成以無援敗,趨公州就志超。而志超已棄公州,間道出漢陽東,士成追及之。當是時,大軍集平壤,乃卷甲而趨之,二日始至。志超以成歡一役殺傷相當,鋪張電鴻章,鴻章以聞,獲嘉獎,賞銀二萬犒軍,拜總統諸軍之命。

志超意甚滿,日置酒高會,徒築壘環砲為守。日軍詗至大同江,為我軍逐去,遂以屢捷入告。時統帥居城中,日軍夾江而陣,兩岸相轟擊。東南二路戰少利,志超莫敢縱兵,趣回城。日軍乘間以濟,據山阜,左寶貴出御之,被巨創。志超將私逸,寶貴不從,以兵監之。寶貴自守玄武門嶺,矢必死,登城指麾,為砲所中而殞。志超亟樹白幟乞罷戰,日人議受降,請帥兵歸,弗許,乃潛向北走。朝兵銜之刺骨,於其出城時槍擊之,死者不可稱計。日軍復要之山隘,兵潰,回旋不得出,擠而死者相枕藉。諸將盡委械而去,於是朝境內無我軍矣。

志超奔安州,士成謂安地備險奧,可固守,弗聽。逕定州,亦棄不守,趨五百餘里,渡鴨綠江,入邊始止焉。事聞,奪志超職,鴻章請留營效力,弗許。次年,械送京師,下刑部鞫實,定斬監候。二十六年,赦歸,歲餘卒。

論曰:甲午之役,海陸軍盡覆,辱莫大焉。汝昌雖有罪,而能以一死報國,尚知畏法。汝貴、志超喪師失地,遺臭鄰邦,靦然求活,終不免於國典,何其不知恥哉?


\end{pinyinscope}