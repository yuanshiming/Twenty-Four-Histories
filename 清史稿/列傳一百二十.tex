\article{列傳一百二十}

\begin{pinyinscope}
五岱五福海祿成德馬彪常青

官達色烏什哈達瑚尼勒圖敖成圖欽保木塔爾

岱森保翁果爾海珠爾杭阿哲森保

五岱,瓜爾佳氏,黑龍江人。乾隆十八年,命隸滿洲正黃旗。初以前鋒從征準噶爾,授三等侍衛,賜墨爾根巴圖魯名號。戰葉爾羌,復遷二等侍衛。霍罕使者至,命往宣諭,授正黃旗漢軍副都統,賜騎都尉世職。三十六年,從將軍溫福討金川,授參贊大臣。攻巴朗拉,克之,授正黃旗蒙古都統。

京旗目吉林、黑龍江諸部人為烏拉齊,鄙之不與為伍,溫福以是輕五岱。五岱密疏言:「溫福在軍好安逸,不親督戰,自以為是,寒將士之心。」溫福亦劾:「五岱剛愎自用,自成都至軍,途中奪驛馬騷擾;方攻巴朗拉,綠營兵驚退,五岱不能禁,詐言被創昏暈。」上命豐升額、色布騰巴勒珠爾詣軍中按治。色布騰巴勒珠爾等疏言鞫五岱俱不承,請奪其職,留軍前自效,上責色布騰巴勒珠爾等所論列不得要領;復疏言溫福輕五岱,致起釁。溫福疏辨,謂五岱與色布騰巴勒珠爾朋比謀傾陷,上命色布騰巴勒珠爾等逮五岱詣熱河行在。是時尚書福隆安奉使如四川,疏言五岱無奪驛馬及攻巴朗拉詐言被創事,色布騰巴勒珠爾亦未嘗袒五岱。五岱至熱河,軍機大臣廷鞫,戍伊犁。居數月,授藍翎侍衛,命從阿桂出南路聽差遣。阿桂令率土兵赴美諾、明郭宗諸地,相機夾擊。尋授頭等侍衛。

木果木師潰,阿桂駐宜喜。命五岱為領隊侍衛,率貴州兵防後路。阿桂為定西將軍,授五岱正藍旗蒙古副都統,復為參贊大臣。從副將軍豐升額自丹壩進攻凱立葉,山峻,未深入。上命豐升額佐阿桂合軍進,而以五岱駐凱立葉牽賊勢,賊屢來攻,屢擊敗之。五岱疏言軍中護軍校等缺,當擇應升人員,請上命。上以參贊佐將軍治軍事,不得自專,責五岱非是。阿桂、豐升額自日爾巴當噶進攻,五岱自凱立葉督兵夾擊,進逼勒烏圍。阿桂令五岱移駐日則丫口。尋率兵協攻珠寨及噶朗噶各寨。師攻勒烏圍,五岱率所部自東北入,合攻克之。金川平,圖形紫光閣,列後五十功臣。

出為塔爾巴哈臺領隊大臣。四十九年,自塔爾巴哈臺詣京師,至蘭州,聞石峰堡回為亂,請從軍。上諭陜甘總督李侍堯,以五岱嘗從征金川,知軍事,令率兵進攻。侍堯令偕副都統永安、提督剛塔討賊,自馬家堡逐賊至鹿鹿山,大霧,駐軍數日,詗賊出後山,分軍捕治,命署固原提督。戰伏羌城外,殺賊三百餘,賊遁入山,遣兵搜捕,俘二百三十餘。復逐賊至秦安縣,擬進攻底店。上令尚書福康安視師,五岱從,克底店;進攻石峰堡,率兵搜捕黑矻塔、白楊嶺餘匪,毀床子灘禮拜寺,回亂平。上以五岱自塔爾巴哈臺班滿還京,道聞回亂,自請從軍;福康安未至,轉戰擊賊,奮勉,予騎都尉世職。尋擢鑲藍旗蒙古都統,充上書房總諳達,授領侍衛內大臣。卒。

五福,富察氏,滿洲鑲白旗人。自世襲佐領累遷四川維州協副將,乾隆三十五年,小金川土司澤旺與鄂克什土司色達克拉構兵,五福請於總督阿爾泰,檄澤旺責使服罪。澤旺子僧格桑尤桀驁,漸侵明正土司,乃令五福將五百人屯梭磨界樸頭,擢松潘鎮總兵,如美諾護糧道。小金川平,偕松茂道查禮按行邊徼屯練,及新附汗牛十四寨。

時僧格桑竄大金川,大金川土司索諾木與同為亂。上慮兩酋逃往鄂羅克,命五福駐丹壩。丹壩,往鄂羅克道所必經也。賊襲攻底木達及大板昭。師自登春入,五福自後路會攻。尋請以副將西德布率兵還丹壩,而躬巡梭磨,土婦卓爾瑪初附,加以駕馭。上命五福事畢仍還屯丹壩。五福旋自丹壩進攻穆爾津山,再戰陟其岡,毀賊碉,敗援賊。師進攻,五福以三百六十人為應,令官兵作攻撲狀綴賊,土兵伏作固頂水卡旁。賊至,伏發,殪其頭人,遂進攻山半賊碉,五福督兵斫碉門殺賊。將軍阿桂等師克格魯克古丫口,將達丹壩,五福隔山見師至,即督兵攻普籠、瑪讓諸碉,同時盡毀,於作固頂以下傍水設營卡。

師進攻勒烏圍,五福自陡烏當噶夾攻,斃賊甚眾,進攻榮噶爾博,毀賊碉一。師屯巴克圖仰木山巔,五福克薩木卡爾山下諸碉卡,與大軍會。自達烏達圍進攻,五福同總兵常祿保等為應。既克黃草坪,賊自山後出,五福夾擊敗之。師自奔布魯木進攻,為三隊,五福與副都統烏什哈達率第三隊,圍賊碉。賊越碉竄,與第一、二隊合,至西里正寨,賊潰遁。分攻瓦爾占、舍勒固租魯,夜移砲轟毀之。進攻薩爾歪賊寨,復為三隊,五福與都統海蘭察自中路進,賊棄寨竄;復繞出寨後,殲賊甚眾,賊寨皆下。金川平,圖形紫光閣,列後五十功臣。師既還,以兩金川地勢寥闊,命五福將三千人屯美諾。尋擢廣西提督。卒。

海祿,齊普齊特氏,蒙古正藍旗人。以前鋒從征伊犁,定邊右副將軍兆惠屯濟爾哈朗,副將軍富德攻葉爾羌,攻伊西洱庫爾淖爾,海祿皆在軍中,賜花翎,並號噶卜什海巴圖魯。又以邊功,擢二等侍衛。溫福討金川,海祿將四百人攻斑斕山及斯當安,攻日耳、東瑪、美美諸寨,及固卜濟山梁,又克路頂宗、喀木色爾諸寨,破明郭宗溝內碉卡。自前鋒參領攝陜西固原鎮總兵。溫福師敗績,海祿自美諾退巴朗拉,定西將軍阿桂論劾,當奪職,命寬之。師自資哩南山入,得阿喀木雅山上碉一。至路頂宗,山陡峻,夜半潛入賊壘,殲賊三十餘,墜崖死者相枕藉,遂拔路頂宗,即督兵進攻明郭宗,克之。直抵美諾,賊驚潰,獲大砲十餘、米糧百餘石,擢固原鎮總兵。

從阿桂自薩爾赤鄂羅山攻克登古碉卡。復自喇穆喇穆迤西進,得石卡一。攻得斯東寨、色淜普、喇穆喇穆山梁,屯日則丫口要路。又攻該布達什諾木城,連克碉寨。攻遜克爾宗,賊出伏兵,擊之潰。旋偕副都統富興進至達爾沙朗,克大碉五,並克伊格爾瑪迪等碉卡。再進,偕副都統烏什哈達奪羅卜克鄂博溝內碉寨,攻克格魯克古山梁。再進攻康薩爾,督兵躍壕入,賊竄。再進,攻克勒吉爾博山梁,乘勝沿河擊賊,大破之。師攻木思工噶克丫口,海祿以兵應,殲賊甚眾。攻克邁過爾山梁,復偕烏什哈達攻丫口左木城、石碉,拔之。又自舍圖枉卡分攻巴占,攀藤扶石,自山腰斜上,遂奪據毗色爾,進攻章噶大碉,克之,並奪木城一。偕襄陽鎮總兵官達色攻黃草坪,占其地。移直隸天津鎮總兵。旋率土兵奪兜窩碉卡,復奪取莎羅奔甲爾瓦沃雜爾所居之拉布咱占。又偕副都統書麟等攻則朗噶克,焚噶爾噶木、勒烏、果木得克、聶烏諸賊寨。金川平,圖形紫光閣,賜騎都尉世職,擢雲南提督。

四十六年,入覲。至湖南,聞薩拉爾回蘇四十三叛,請從軍。賊占華林山,海祿從海蘭察攻之,多所斬獲。旋進至華林寺,毀賊巢,殲焉。授烏魯木齊都統。

海祿刻覈吏事。在邊,禁古城迤北瑚圖斯金廠。重定新疆屯田徵租功過,視舊例為苛。追論文武吏士剝下營私狀,領隊大臣圖思義、提督彭廷棟以下皆坐譴。又請裁汰經費,視內地編保甲;臺灣民坐械斗戍邊,入烏魯木齊鐵廠輸作,予巴里坤諸地戍兵為奴;皆議行。復疏請自哈密至精河設臺車三百五十,烏魯木齊設臺車一百五十,定值視雇商車減三之二。烏什辦事大臣綽克托、塔爾巴哈臺辦事大臣惠齡、陜甘總督福康安皆言車值過薄,福康安並力陳設臺車不若雇商車便。上為罷海祿議,造臺車糜帑,令責償。伊犁將軍伊勒圖又疏請罷海祿所議屯田徵租功過及戍邊入鐵廠例,左授伊犁額魯特領大臣。

五十三年,劾將軍奎林毀佛像,辱職官,折罪人手足擲水中,得遣戍罪人贓,又於哈薩克以羊易布,私其羨金。上奪奎林職,令海祿並詣京師,命諸皇子、軍機大臣會刑部廷鞫。奎林承毀佛像、殺罪人,餘事皆無據。上命並奪海祿職,在上虞備用處拜唐阿上效力行走。尋授藍翎侍衛,累遷至福建陸路提督。卒。

成德,鈕祜祿氏,滿洲正紅旗人。初入健銳營充前鋒。從征準噶爾、葉爾羌,俱有功。征緬甸,從將軍明瑞自錫箔進兵,攻賊舊小蒲坡,中槍傷,戰猛拜、天生橋、猛城諸地。從副將軍阿里袞攻頓拐,毀其寨。從經略大學士傅恆渡戛鳩江,自猛拱、猛養進兵,敗賊於新街。定邊右副將軍溫福征小金川,成德從攻斯當安,裹創力戰,進攻巴朗拉。再進,克資哩、古布濟、八角寨諸地,復被創;自空卡、昔嶺進兵,屢捷,累遷四川川北鎮總兵。木果木大營陷,溫福死之,成德時將別軍駐美諾,亦陷於賊,命奪官,仍留任。將軍阿桂令自南山攻取阿喀木雅,會領隊大臣額森特、總兵海祿三道並進,擊東溝賊碉,殲賊甚眾。路頂宗、明郭宗諸營卡皆下,復美諾,賜黑狐冠。小金川平,復官。

師自穀噶入大金川,抵羅博瓦山,成德偕總兵特成額等分兵綴賊。復會克色淜普山,奪堅碉數十。進攻喇穆喇穆東面山碉,賊分兩路襲師後,擊敗之。偕散秩大臣普爾普等奪石碉四,又偕總兵官達色攻克該布達什諾木城,會內大臣海蘭察進圍遜克爾宗,賜號賽尚阿巴圖魯。進攻甲爾納寨,圍急,賊潛以皮船渡,成德擊破之。賊據赤布寨,其北為得思古寨,循溝下有噶朗噶、噶爾噶諸寺,碉寨繁密。師循溝進,破最東水碉。成德乘勝奪大碉五、木城二,直抵瀕河噶爾丹寺,賊奔潰,師克舍圖枉卡。成德潛師至日則丫口,與游擊普吉保上下合擊,破石碉八、木城四,遂克遜克爾宗,賊退勒烏圍,復進,會師破之。進克甘都瓦爾、黃草坪等處,遂克噶拉依。金川平,圖形紫光閣,列前五十功臣。署四川提督。三暗巴番渠安錯煽亂,督兵捕治,命真除。

五十三年,廓爾喀侵後藏,命成德為參贊大臣,督兵偕總督鄂輝、駐藏大臣侍郎巴忠會剿。巴忠授意噶布倫丹津旺珠爾與廓爾喀議歲費、還侵地,成德爭不獲,即以此議入奏。師還,授成都將軍。後藏不如約,靳歲費不與,廓爾喀復來犯,巴忠自經死。上命鄂輝、成德督兵定藏自贖;復以濡滯失機,奪將軍,予副都統銜,以領隊大臣屬將軍福康安調遣。攻聶拉木,與穆克登阿夜督兵進。成德攻寨西北,穆克登阿出西南,擲火彈殺賊,破寨,盡殲守寨賊,無一得脫者。福康安自濟嚨進兵,令成德等分道進屯德親鼎山,克敵卡,自俄瑪措山進,迭克果果薩喇嘛寺,乘夜取札木鐵索橋。又自江各波邁山梁趨隴岡,與彥吉保會;逐賊至利底,與福康安師會,所向克捷。廓爾喀乞降,師還,命成德以副都統銜充駐藏幫辦大臣。圖形紫光閣,前十五功臣,以成德為殿。尋命署杭州將軍。

仁宗即位,移署荊州將軍。教匪起,成德偕總督惠齡攻賊宜都灌灣腦山,擒賊首張正謨。尋以縱賊竄逸,奪勇號。四年,致仕,卒。以曾孫女配宣宗為孝全皇后,追封三等承恩公,謚威恪。子穆克登布,自有傳。

馬彪,甘肅西寧人。以行伍從軍,累遷至四川川北鎮總兵。高臺縣丞邱天寵私伐巴彥濟魯薩林木,貝勒羅卜藏達爾札訴於上,詞連彪,奪職。尋賜游擊銜,駐雅爾。復起,除云南昭通鎮總兵。

乾隆三十六年,師征金川,將軍溫福以彪屢出師勇往,令將貴州兵三千以從,克巴朗拉碉卡,賜花翎。師自達木巴宗分三道趨資哩,彪偕侍衛額森特等自北山進,奪賊碉卡,斬馘百餘,與師會。彪以貴州兵二千駐資哩北山梁,東西距三十餘里。賊夜犯都司黃壯略、守備王廷玉營,彪與侍衛巴三泰馳援,敗賊,失砲三。上以彪戰甚力,不之罪。嗣都司徐大勇等守色布色爾,賊屯十里外高峰。參贊五岱檄彪赴援,未至,副將色倫泰戰沒。五岱劾彪逗遛,當奪職,上命留任。尋自碩藏噶爾進駐色布色爾,阿桂軍次喇卜楚克山麓,偪木闌壩;令彪伏兵東崖下,克其水碉。進攻色爾渠,彪從參贊豐升額等擊東瑪砦,克之。乘勝攻哲木克郭羅郭羅美羅喇嘛寺諸寨,皆下,奪碉五,俘馘數十。攻美美卡,彪率二百人自山梁小徑入。賊來援,力戰破之。美美卡至日喀爾橋,有小徑曰兜烏。賊毀橋築卡以拒,彪伐木為橋濟兵,賊棄卡走。又與提督哈國興合克喀木色爾穆拉斯郭寨,遂據兜烏。尋自達克蘇山後攻明郭宗,彪將千人自格實迪下攻,賊棄碉竄,授西安提督。復偕侍衛烏爾圖納遜攻達爾圖大碉,斃竄賊甚多。遂偕領隊大臣華善等以六千人駐宜喜,賊來犯,擊之,斬賊三十餘人。以三千五百人攻達爾圖碉,未下。賊自沙壩三道襲宜喜軍,又別遣賊夜撲達爾圖軍,擊走之。師克乃當,至獨松,彪與賊戰中巴布里、下巴布裏及瑪雅岡角木,賊皆棄寨遁。旋與副將欽保克爾瑪及札烏古山梁,與總兵敖成克甲索。

金川平,赴西安任。圖形紫光閣,列前五十功臣。移湖廣提督。卒,贈太子太保,謚勤襄,予雲騎尉世職。

常青,蘇木克氏,滿洲鑲白旗人。自前鋒累遷護軍參領。外擢雲南曲尋鎮總兵。從將軍明瑞討緬甸,戰於蠻結。明瑞將中軍,常青與領隊大臣觀音保踞西山梁。賊突至,常青等奮擊,馘二百餘;賊敗竄,又馘二千餘,俘三十四。再戰天生橋、宋寨、黃土岡諸地,屢敗賊。明瑞軍敗績,上召常青入對,命仍還雲南,從副將軍阿里袞出萬仞關。經略大學士傅恆令詣野牛壩督造戰船,率兵赴新街,殺賊奪寨,獲敵舟及糧械。旋自新街進攻老官屯,克毛西寨。師還,授雲南提督。

乾隆三十八年,師征金川,令率雲南兵二千赴打箭爐佐將軍阿桂出西路。偕都統海蘭察攻斯達克拉、阿噶爾布里、碩藏噶爾諸山梁,克之,留屯美諾。師攻布朗郭宗,阿桂奏請常青策應。常青遣游擊福敏泰駐木波,游擊保寧駐噶魯什呢,守備張啟貴駐美臥溝,而與副都統富興率兵為布朗郭宗聲援。西藏語謂為盜曰「放夾壩」,常青與富興督綠營兵捕盜,焚其林。阿桂師進攻勒烏圍,常青與富勒渾護餉道,自明郭宗至大板昭,兵卒巡視,分守小沙壩、沙壩、三松坪諸地,自間道出功噶爾拉擊賊。上嘉之,諭以此路官軍久未進攻,今自間道出奇,足以綴賊;惟地勢險峻,仍戒其輕舉。金川平,圖形紫光閣,列後五十功臣。

移古北口提督,而以海祿代之。疏言緬甸方議撫,請暫留張鳳街,與海祿相機籌辦。上以夷性多疑,文檄仍用常青舊銜,俟事定赴新任。歷浙江、江南、直隸、福建陸路提督,又繼海祿為烏魯木齊都統,移西安將軍。卒,謚莊毅。

官達色,瓜爾佳氏,滿洲正黃旗人。以前鋒從征準噶爾。將軍兆惠自鄂壘扎拉圖轉戰至特訥格爾,上方南巡,遣官達色及副護軍校兆坦齎疏詣行在,召對,授藍翎侍衛。準噶爾平,予雲騎尉世職。迭遷副參領,外擢雲南順云營參將。自陳不通漢文,乞還京師,經略大學士傅恆討緬甸,以官達色監鑄砲,令從軍。旋授健銳營前鋒參領。

乾隆三十六年,將軍溫福征金川,令將成都駐防兵四百人從攻巴朗拉山梁,與烏什哈達督兵自山右登,奪卡六。再戰,官達色發砲毀賊碉,戰三晝夜,克之,賜號巴爾丹巴圖魯,畀白金百。師逾達木巴宗至斯底葉安,賊力拒,官達色發砲隳其碉樓,命署四川松潘鎮總兵。師乘雪擊賊,賊引退。官達色逐賊,賊亂流渡,竄河喀木雅。移軍逼賊寨,官達色發砲擊之,寨垂破,賊夜遁。溫福督師攻南山,官達色與總兵牛天畀合軍,天畀取第二碉,官達色取第三碉,復命署湖北襄陽鎮總兵。

師攻達爾圖,賊蔽碉為固,官達色發砲擊之,日斃賊數十。師進,破碉二,拔柵,殲賊甚眾。副將軍豐升額攻穀噶,官達色與侍衛普濟保等以四千人往會。旋以將軍阿桂檄,從參贊海蘭察攻喇穆喇穆,奪卡三,逼碉下擲火彈,以雨不燃,暫引退。復以六百人直陟高峰,峰有大碉二,夜半,援石壁蟻附登,伏碉旁,黎明突起,遂破二碉。進攻該布達什諾,賊為大碉倚壕,輔以木城。官達色督兵冒槍石躍壕以度,劃碉址成,遂援以上。賊退保木城,阿桂令海蘭察出城後,官達色當其前,力戰克之。再進,攻默格爾山梁,官達色與額森特等合軍取碉三。旋與海蘭察、額森特分道裹糧深入,攻格魯克古丫口,克當噶海寨及陡烏當噶大碉,焚沙木拉渠寨。循格魯克古山梁以下,賊傍箐置卡,督兵攻之下。真除襄陽鎮總兵。

再進,攻勒吉爾博,戰於山麓,破賊碉;再進,攻榮克爾博,克其麓木城。督兵陟山巔,與普爾普逾溝拔木柵二十六。自舍圖柱卡循昆色爾山梁,攻據雅木則碉,取果克山諸碉寨,圍拉枯喇嘛寺,盡殲之。再進,與海蘭察等同攻章噶,賊綠碉鑿深溝,設柵其上,官達色督兵拔柵以覆溝,援附至碉巔下攻,賊驚竄,遂克之。與海蘭察合軍向勒烏圍,分攻隆斯得,其地有三寨,克其二;遂潛破後寨,寨內蓄鉛子,積地二尺許,火藥百餘簍,悉收以佐軍,設砲臺,偪轉經樓,與保寧、彰靄合軍克之,勒烏圍亦下。與海蘭察等攻達烏,連破諸碉寨。進攻西里,賊四出力禦,官達色逾溝與戰,賊穿林逃。攻黃草坪,海蘭察當其前,官達色與海祿拔溝北柵為應。攻奔布魯木峰木城,亦與海蘭察偕。攻瓦喇占,發砲破其碉。循瓦喇占而下曰薩爾歪,有寨三,海蘭察當其前,官達色與烏什哈達左右合擊,賊棄寨走,邀殪之。攻科布曲木城,又與海蘭察偕,官達色冒槍石先登。攻朗阿古,海蘭察自山腰險徑度兵,官達色與烏什哈達出其左。攻雍中喇嘛寺,官達色與普爾普等自右入,皆力戰殺賊,遂破噶拉依。金川平,圖形紫光閣,列前五十功臣,予一等輕車都尉世職。移山西大同鎮總兵,再移直隸宣化總鎮兵。卒。

烏什哈達,吉林滿洲正黃旗人。師征緬甸,以前鋒校從,有功,賜號法福哩巴圖魯。師征金川,以三等侍衛從,其與官達色同克巴朗拉也,賊攻據所駐山,復力戰破賊,奪其山還。事聞,上以功過足相當,宥之。戰屢有功,累擢正藍旗蒙古副都統。師還,圖形紫光閣,列前五十功臣,予騎都尉兼雲騎尉世職。外授和闐領隊大臣,訐辦事大臣德風受賂,按治不盡實,奪職。師征臺灣,以頭等侍衛從,與普爾普自茅港轉戰,通嘉義道。尋將水師至瑯嶠,獲莊大田,還前所賜勇號。再圖形紫光閣,列後三十功臣。師征廓爾喀,以鑲紅旗蒙古副都統從,先行治道,躓而傷。師還,賞不及,入見,以為言。上責其巧佞,奪職,戍伊犁。嘉慶初,赦還。師征川、楚教匪,以頭等侍衛從,賊渠王三槐擁眾渡江,烏什哈達與戰,死之,予輕車都尉世職。

瑚尼勒圖,鄂訥氏,黑龍江人。以護軍入滿洲鑲黃旗。累遷護軍參領。從征金川,亦與巴朗拉之役,賜號多卜丹巴圖魯。攻資哩南山,戰自喇卜楚克山梁,繞登高峰,奪賊卡二,遂陟其巔,又奪賊卡二。復從海蘭察等攻羅博瓦前山,賊二百餘自其右緣山梁斜上,瑚尼勒圖擊殺十餘人,賊遁走,進攻該布達什諾,克之,加副都統銜。復進攻遜克爾宗,焚賊寨十餘,賊來援,卻之。師攻勒烏圍,遣瑚尼勒圖奪據默格爾山,進占日爾巴當噶爾之西。危峰突起,海蘭察等更出其西,自密拉噶拉木山巔下擊,遂克凱立葉,諭嘉獎。乘勝攻克日爾巴當噶山陽左右五碉。又從海蘭察等攻取桑噶斯瑪特山寨。與福康安督兵將出箐,見賊碉二,奮勇躍入殺賊,賊潰,擢鑲藍旗蒙古副都統。師攻達佳布、安吉諸碉,督兵自山腰賊碉間攀越而過,先入碉,皆克之。進攻木思工噶克,令瑚尼勒圖攻丫口。潛師而入,游擊梁朝桂等為繼,丫口峰左右碉十有四,同時皆破。師次榮噶爾博,有山梁曰巴占,為勒烏圍門戶,賊守禦甚力。諸將議自舍圖柱卡間道入,而使瑚尼勒圖屯巴占分賊勢。師克章噶,瑚尼勒圖亦取巴占。分攻隆斯得寨,以斧破寨門,獲所儲鉛藥,遂攻下勒烏圍。復攻西里山梁,瑚尼勒圖與烏什哈達督兵徑陟,克大碉三、木城四。師攻西里正寨,與福康安以火攻破寨;又與海蘭察取朗阿古,攻克得拉古碉卡;復自巴薩沙進,取奇什磯官寨,與福康安等克雍中喇嘛寺。金川平,圖形紫光閣,列前五十功臣,轉鑲紅旗蒙副都統。尋授散秩大臣,管理健銳營。卒。

敖成,字丹九,陜西長安人。入伍,從征瞻對、金川、庫車,戰喀喇烏蘇河,攻葉爾羌,俱有功。乾隆三十八年,師再徵金川,成以廣西右江鎮總兵入覲,上詢知成嘗出師瞻對、金川,賜花翎,並畀白金百,給驛詣軍前。旋移甘肅寧夏鎮,以將軍阿桂請,復移貴州鎮遠鎮。師三道進,副將軍明亮出南路,請以成駐僧格宗防後路。上慮成未足當一面,命從明亮軍進討。桂林疏言:「南路當自塔克撒至宜喜諸地設防。成自薩穆果穆渡河,經美諾至塔克撒駐軍。」明亮移軍宜喜,攻達爾圖山梁,使成偕副都統舒景安率師攻日旁,奪賊卡二,破碉寨四百餘,殲賊甚眾。諸軍攻宜喜,圍合,詗甲索守賊皆老弱,當攻其瑕。成偕副將常泰等率土、漢兵二千五百分三道進,破其要隘,先後奪碉十一。上嘉其勇,賜號僧格巴圖魯。復自達爾圖山梁進攻噶爾丹,直薄巴布裏山脊。值夜大雪,潛師出碉後奮擊,連克防隘賊卡四。守碉賊驚潰,追斬無算。復偕常泰攻克碾占,偕提督馬彪率師至甲雜官砦,賊棄寨潰竄。師三路畢會,遂克噶喇依。金川平,圖形紫光閣,列前五十功臣。禦制贊,以乘雪取巴布裏比諸李愬之入蔡州。擢貴州提督,入覲,賜黃馬褂。卒,贈太子太保,謚勇愨,予雲騎尉世職。

圖欽保,瓜勒佳氏,滿洲鑲黃旗人。以前鋒校從將軍明瑞徵緬甸,有功,授三等侍衛,賜號法福禮巴圖魯。遷健銳營副前鋒參領。乾隆三十七年,從將軍阿桂征金川,以皮船濟師,襲達烏西山碉卡。圖欽保與總兵王萬邦自其左進,攻克其碉。復與侍衛三寶等合兵,至邦甲山梁,緣溝以登,盡取諸碉卡,自山下夾攻,賊潰。師至納圍納札木,副將軍明亮等分兵三道並進,圖欽保與游擊穀生炎攻山坡碉卡,賊力拒。復與侍衛德赫布三面合圍,壘石卡逼賊,賊棄碉夜遁。師進至僧格宗,圖欽保自河西科多渡橋攻河東,至喀咱木籠山梁,抵奢壟,賊奔美諾。復與參領拉布棟阿以五百人取馬奈。擢湖南長沙協副將。師復進,抵薩克薩谷,其北曰茹寨,麥方熟,賊設調以衛,圖欽保力攻克之,焚沿河各寨,賊竄出,中矢被槍及墜河死者無算,麥田十餘里,皆為我兵所據。事聞,上手詔獎勉。復攻石真噶山下木城,毀賊寨,再進,攻扎烏古山梁,功最,擢陜西固原鎮總兵。事定,圖形紫光閣,與德赫布並列前五十功臣。四十六年,撒拉爾回叛,圖欽保將五百人助戰。賊退踞八蠟廟、水磨溝諸地,圖欽保從都統海蘭察率兵越水磨溝自山梁進逼賊巢。賊自山坡逆上,圖欽保持刀奮戰,馬蹶,墜山下,被創,卒,賜白金七百。

木塔爾,小金川人。乾隆三十七年,小金川頭人僧格桑為亂,拒我師,木塔爾率親屬及所部降。將軍溫福令從軍,即率土兵奪八角碉,降千餘人,復官寨。攻木果木,面中石傷。克達響谷山梁,槍傷額。累擢三等侍衛,賜孔雀翎。僧格桑竄大金川,大金川頭人索諾木匿之,與同亂。將軍阿桂令木塔爾偵路,約內應,遂克阿不里,招其叔朗納降。金川山徑歧互,阿桂令木塔爾指畫,繪圖呈覽;又以功噶爾拉賊守堅,諮木塔爾。木塔爾言:「穀噶山路崎嶇,樹木深密。若密遣精兵畫伏夜行,出賊不意,亦一策也。」從之。戰有功。官兵護臺站,遇賊稍卻。阿桂令木塔爾偕降人賡噶率土兵截擊,擒頭人穆工阿魯庫。攻噶魯什尼後山及登春諸地,擒頭人拉爾甲,創僧格爾結,以功賜緞。賊遣別斯滿尼僧布薄偽降,私詢木塔爾軍事,木塔爾密以聞。上嘉其誠,果擢頭等侍衛。師攻喝拉依,索諾木等出降,賜號贊巴巴圖魯。圖形紫光閣,列後五十功臣。授八角碉屯守備,督帛噶爾角克及薩納木雅諸地降人屯田。

四十六年,甘肅撒拉爾回蘇四十三攻陷蘭州,上命領侍衛內大臣海蘭察軍討之,木塔爾從,中槍傷,賜銀緞。復攻華林寺,再受傷,賜二品銜,以四川管理降番副將題補。四十九年,甘肅固原回田五等餘黨踞石峰堡,上命成都將軍保寧討之,木塔爾從,力疾赴調,賜散秩大臣銜。至石峰堡,屢有斬獲,被石傷。

五十三年,從征臺灣,偕侍衛博斌等生擒首逆莊大田於瑯嶠。臺灣平,復圖形紫光閣,列前二十功臣。

五十六年,廓爾喀為亂,攻陷聶拉木。木塔爾從成德守木薩橋,獲頭人格枌達喀嘰哈等,加副都統銜。師攻濟嚨,木塔爾偕侍衛哲森保先攻克東南山梁,移兵逐賊,復濟嚨,殲賊數百,殪賊目七。師攻雅爾賽拉、博爾東拉,木塔爾率兵自噶多普紆道渡河,奪石卡、木城。廓爾喀平,再圖形紫光閣,列後十五功臣。上特召慰勞,賜酒,賚銀緞。

六十年,從征苗匪。賊居下石花、土空等處,循沿河山坡築城卡,阻我師。總督福康安遣木塔爾於下游河岸設伏,賊出卡搶掠,突出擊之,奪其渡船。師進迫之,賊不能御,連克城卡。進攻土空,偕總兵花連布等連戰三畫夜,破之,賜荷包。以病還師,至資陽,道卒,賜白金百。

岱森保,庫雅拉闊綽里氏,滿洲正紅旗人。以黏竿處拜唐阿從征緬甸。移師征金川,與攻路頂宗、喀木色爾,授藍翎侍衛。戰於昔嶺,賊乘高而下,以火器奮擊,賊潰,授三等侍衛。戰於羅博瓦,殲賊數十,復奪取喀木喇瑪山碉,擢二等侍衛,賜號布隆巴圖魯。攻勒吉爾博山梁,拔鹿角,躍壕,以火彈擲碉巔,破之。從將軍阿桂攻勒烏圍,發砲斷其橋,隧以入柵,克木城,與諸軍合攻,勒烏圍遂下,授頭等侍衛。師還,圖形紫光閣,列後五十功臣。

乾隆四十四年,以護軍參領從征臺灣。與侍衛烏什哈達等擊賊沙嵌,進至蔦松,殲賊二百餘。擊賊中洲,發巨砲殺賊,進擊賊南潭,賊潰,焚賊藔數百。再進,擊賊三坎店,奪賊中砲械。尋從閩浙總督常青等援諸羅,出鹽水港,戰賊屢勝,賜副都統銜。福康安視師,岱森保攻賊牛莊,賊阻溪為固,督兵逾溪擊之,俘斬甚眾,乘銳抵南潭,遂俘莊大田等。師旋,再圖形紫光閣,列後三十功臣。擢正黃旗蒙古副都統。出為伊犁領隊大臣。

廓爾喀為亂,上命岱森保將索倫、達呼爾兵千人,偕參贊大臣海蘭察自京師道青海入西藏,佐福康安等討之。既至,福康安令偕成都將軍成德將三千人向聶拉木綴賊。分兵自措克沙木間道入,自率兵趨親鼎山,破賊卡,賊敗竄。旋偕侍衛永德道哈那滾木山,克扎木。復偕成德敗賊多洛卡,追躡至俄賴巴,分兵兩路深入,廓爾喀酋降。復圖形紫光閣,列後十五功臣。

嘉慶初,教匪起,命岱森保討賊陜、甘。張漢潮侵五郎,自盩厔出大建溝擾洵陽,偕總兵長春、副都統綸布春隨所在御之。上責肅清甘肅境,與西安巡撫臺布選能戰兵四千有奇,逐賊轉戰,屢有克捷。五年秋,擊賊沔縣,以兵寡未獲窮追,還軍駐長寨。疾作,行至漢中,卒。

翁果爾海,噶巴喀氏,滿洲鑲黃旗人。初充親軍,遷藍翎侍衛。乾隆五十二年,從福康安征臺灣,擊賊八卦山,斬馘無算,賜號額騰額巴圖魯。累遷二等侍衛。林爽文遁老衢峙,義民高振以告。翁果爾海與追擊,獲之。臺灣平,予騎都尉世職。

五十六年,廓爾喀侵後藏,從將軍福康安、參贊海蘭察往討之。賊據擦木,其地兩山夾峙,惟一徑可通。夜雨,翁果爾海分兵潛進,越山直上山梁,與師會,薄賊寨,逾墻入,殲賊數百,克其碉。賊奪據濟嚨官寨,師圍之。翁果爾海直攻東南山梁,賊恃碉拒師;督兵緣碉上,殲賊六百餘,擢頭等侍衛。賊據熱索橋,師自擺馬奈撒入,與夾河相持。翁果爾海自峨綠山紆道出上游,斫木編筏潛濟,自間道疾馳攻賊寨,師悉渡,賜副都統銜。賊竄協布魯,負水築卡為守,師不得即渡,暮雨,伏兵林中,夜將半,援木涉水進擊。師繞出對山,並力下攻,賊潰走,追斬三百餘,焚寨五;遂進攻東覺,道噶多。翁果爾海從海蘭察為前鋒,紆道出雅爾賽拉、博爾東拉,穿林越箐,潛師步行。賊為木城三、石卡七,守甚堅。翁果爾海督兵逾險攻之,右臂創甚劇,援兵至,奮勇轉戰,殪頭人二、餘賊二百有奇,賊乃遁,悉隳其城卡,賜白金五十。廓爾喀平,圖形紫光閣,列後十五功臣。授鑲黃旗蒙古副都統。嘉慶初,卒。

珠爾杭阿,顏扎氏,滿洲正黃旗人。自前鋒累擢二等侍衛。從征甘肅石峰堡亂回,賜號錫利巴圖魯。乾隆五十六年,廓爾喀侵後藏。上命鄂輝、成德討之,命珠爾杭阿佐軍,鄂輝以第理浪古、窩浪卡兩地當沖要,令珠爾杭阿察形勢,督兵屯守。尋偕侍衛永德攻克聶拉木寨,賜大緞。復偕將軍福康安自宗喀攻擦木,與參贊大臣海蘭察合軍,自正路攻賊寨,克之,賜大小荷包。復同頭等侍衛阿滿泰等克濟嚨,遷頭等侍衛。復從海蘭察攻雅爾賽拉、博爾東拉,毀木城、石卡,殲賊甚眾。又破賊於瑪木拉,加副都統銜。進攻噶勒拉堆補木大山,分兵三路,珠爾杭阿偕三等侍衛阿哈保等自右路夾擊,焚賊卡。復自橫河上游修橋渡,攻集木集,克之,尋命為領隊。廓爾喀頭人拉特納巴都爾降。福康安令珠爾杭阿護貢使詣京師。圖形紫光閣,列後十五功臣。累遷御前侍衛、正白旗護軍統領。神武門獲為逆者陳德,賜騎都尉世職,授鑲藍旗滿洲副都統。卒。

哲森保,薩克達氏,滿洲鑲藍旗人。初充吉林烏拉馬甲。征緬甸,偕侍衛阿爾蘇拉擊賊新街,從副都統明亮擊賊老官屯。從討王倫,侍衛音濟圖擒賊,將就縛,突有賊持械出拒,哲森保射殺之。從討蘇四十三,攻華林山,槍殪賊渠,哲森保亦被創,賜號法福里巴圖魯。累擢二等侍衛、乾清門行走。再出討石峰亂回,中石傷,擢頭等侍衛,授公中佐領。從征廓爾喀,攻擦木。哲森保與翁果爾海各將一隊,自東、西兩山分進,克之。攻濟嚨,首奪東南山梁;師繼進,遂克濟嚨官寨。賊斷熱索橋,哲森保與阿滿泰出間道,越峨綠山,自上游砍樹結筏潛渡,驟攻賊卡,賊駭愕奔竄,師得濟,賜副都統銜。至博爾東拉,與賊力戰,左膝中槍,賜白金百,令還濟嚨休養。至協布魯,創發,卒。廓爾喀平,圖形紫光閣,列後十五功臣,祀昭忠祠,賜騎都尉世職。

子富永,亦在軍,以戰功累擢三等侍衛,襲職。官至鑲黃旗蒙古副都統。卒。

論曰:金川地小而險,懸崖絕壁,壘石為碉,師至不能下。高宗讀太宗實錄,知其時攻城用雲梯,命斅其制,督八旗子弟習焉。師再出攻碉,賴是以濟。諸將有勞者,五福將四川兵,彪將貴州兵,常青將雲南兵,成將綠營,木塔爾將土兵,餘皆率禁旅;而官達色督砲兵,圖欽保佐健銳營,尤專主攻碉,摧堅決險,非豫不為功。成德、岱森保及木塔爾復從征廓爾喀有功。翁果爾海等未與金川之役,而屢從征伐,轉戰立勛名,亦裨佐之良也。


\end{pinyinscope}