\article{列傳十}

\begin{pinyinscope}
萬子扈爾干孟格布祿扈爾干子岱善孟格布祿子吳爾古代

楊吉砮兄清佳砮楊吉砮子納林布祿金臺石清佳砮子布寨布寨子布揚古

布占泰拜音達里

萬,哈達部長也。萬自稱汗,故謂之萬汗。明譯為王臺,「臺」「萬」音近。明於東邊酋長稱汗者,皆譯為「王」某,若以王為姓,萬亦其例也。哈達為扈倫四部之一,明通稱海西。哈達貢於明,入廣順關,地近南,故謂之南關。

萬姓納喇氏,其始祖納齊卜祿。納齊卜祿生尚延多爾和齊,尚延多爾和齊生嘉瑪喀碩珠古,嘉瑪喀碩珠古生綏屯,綏屯生都勒喜。都勒喜子二:克什納、古對硃顏。古對硃顏之,於諸部中最強,修貢謹,又捕叛者猛克有勞?後別為烏喇部。克什納,嘉靖初掌塔山左,明授左都督,賜金頂大帽;既,為族人巴代達爾漢所殺。克什納子二:長徹徹穆,次旺濟外蘭。克什納死時,徹徹穆子萬奔席北部境綏哈城,而旺濟外蘭奔哈達,遂為其部長。明以其偵寇功,授都督僉事。葉赫部長褚孔格數為亂,旺濟外蘭執而僇之,奪其貢敕七百道,及叛,旺濟外蘭為所殺。其子博爾坤舍進殺父仇,迎從兄萬於綏哈城,?所部十三寨。後其部部,遠交而近攻,勢益盛,遂以哈達為國,稱汗。興祖諸子環?,略?還長其部。萬能用其居赫圖阿喇,號「寧古塔貝勒」,與董鄂部構釁。興祖第三子索長阿為其子吳泰娶萬女,蓋嘗乞兵於萬以御董鄂部。

萬居靜安堡外,室廬、耕植與他部落異,事明謹。是時王杲領建州,與韃靼東西遙應,窺遼塞,萬支拄其間不令合。明使繼其大父克什納為都督。王杲盜邊,開原兵備副使王之弼檄萬,令王杲還所掠。萬入建州寨,要王杲盟於撫順關下,復通市如故。土默特徙帳遼東,萬入貢,多奪其馬。已而,土默特弟韋徵與萬為婚,其從子小黃臺吉擁五萬騎,介葉赫復請婚於萬,萬懼而許之。小黃臺吉以馬牛羊、甲胄、貂豹之裘遺萬,築壇刑白馬為盟,約毋犯塞。居無何,小黃臺吉要萬犯塞,萬不可,乃罷,時為萬歷元年。明年,王杲亂,遼東巡撫張學顏檄萬捕王杲。萬令海西、建州諸酋款塞,乞先開市,游擊丁仿語之曰:「必得王杲都督大疼克等叩關,督撫以聞,許開市,遂縛獻王杲所?而後市可圖也。」萬復率建州掠遼軍八十四人,及種人兀黑,以兀黑嘗殺漢官也。又明年,捕得王杲,檻致京師。明進萬右柱國、龍虎將軍,官二子都督僉事,賜黃金二十兩、大紅師子紵衣一襲。

是時萬所領地,東則輝發、烏喇,南則建州,北則葉赫,延袤千里,保塞甚盛。萬暴而黷貨,以事赴訴,視賂有無為曲直。部下皆效之,使於諸部,驕恣無所忌,求賄鷹、犬、雞、豚惟所欲。使還,意為毀譽,萬輒信之。以是諸部皆貳。而葉赫部長清佳砮、楊吉砮兄弟,以父褚孔格見僇,心怨萬。萬納其女弟溫姐,又以女妻楊吉砮,卵翼之。萬老而衰,楊吉砮復婚於哈屯恍惚太,勢漸張。萬子扈爾干尤暴,所部或去從楊吉砮。楊吉砮構烏喇與扈爾干為仇,遂收故所部諸寨為旺濟外蘭所侵者,取其八寨,惟把太等五寨尚屬萬。自是輝發、烏喇諸部皆不受約束,萬地日蹙,憂憤不自憀。萬歷十年七月,萬卒。葉赫聞萬死,使求故貢敕,扈爾干曰:「我父以汝兄弟故,卒用憂憤死,今尚問敕書乎?」勿與,告哀於明。明以萬忠,賜祭,予採幣、四表裏。

萬有子五:扈爾干為長;仲、叔皆前死;季孟格布祿,溫姐子也;又有康古魯,為萬外婦子。萬卒,康古魯與扈爾干爭父業。扈爾干怒曰:「汝,我父外婦子也,寧得爭父業乎?不避我,我且殺汝!」康古魯因亡抵清佳砮,清佳砮妻以女。是時太祖初起兵。八月,扈爾干以兵從兆佳城長李岱劫太祖所屬瑚濟寨,太祖部將安費揚古、巴遜以十二人追擊,殺哈達未附。?兵四十人,還所掠。扈爾幹旋卒。孟格布祿年十九,襲父職龍虎將軍、左都督,康古魯聞扈爾干死,遂還,烝溫姐。

扈爾干有子曰岱善,與唐古魯、孟格布祿析萬遺業為三。康古魯報扈爾干之怨,釋憾於其子;孟格布祿亦以母溫姐故,助康古魯,共攻岱善;而清佳砮、楊吉砮兄弟謀攻萬子孫報仇,十一年七月,挾暖兔、恍惚太等萬騎來攻。明總督侍郎周詠念岱善弱,孟格布祿少,請加敕部諸酋,神宗許之。十二月,楊吉砮等復挾蒙古科爾沁貝勒甕阿岱等萬騎來攻,孟格布祿及岱善以二千騎迎戰而敗。自是兵屢至,恣焚掠不已。十二年,明總兵李成梁誘殺清佳砮、楊吉砮兄弟,所部讋服,誓受孟格布祿約束。

葉赫難始紓,而內訌復急。清佳砮子布寨、楊吉砮子納林布祿乘隙圖報怨。十五年四月,納林布祿以恍惚太萬騎攻把泰寨,明兵來援,圍解;乃陰結其姑溫姐,嗾孟格布祿佐康古魯圖岱善。先是扈爾幹許以女歸太祖,十六年,岱善親送以往,太祖為設宴成禮。是年納林布祿復以恍惚太五千騎圍岱善。孟格布祿將其孥從納林布祿往葉赫,居十八里寨,於是圖岱善益急,而康古魯誘岱善所部叛岱善,略其貲畜,納林布祿並掠岱善妻哈爾屯以去。明邊吏議絕孟格布祿市,以所部及土田、牲畜盡歸於岱善。孟格布祿不聽,復與布寨、納林布祿、康古魯入開原,溫姐偕。開原兵備副使王緘令裨將襲其營,執溫姐、康古魯以歸。巡撫顧養謙諭孟格布祿:「和岱善,還所掠,否則斷若母頭矣!」王緘以為戮溫姐則孟格布祿益攜,不如釋之,而囚康古魯,待朝命。溫姐既得脫,遁還。孟格布祿自葉赫攻岱善,自焚其所居,劫溫姐去。王緘坐是奪職。

十六年二月,河西大饑,岱善乞糴於明,明予粟百斛。李成梁出師討布寨、孟格布祿,圍其城,布寨、孟格布祿請降,成梁振旅還。開原兵備副使成遜議釋康古魯,和諸部;總。不如釋康古魯,使和岱?督侍郎顧養謙亦謂:「岱善弱而多疑,即殲諸酋立之,不能有其善,則萬子孫皆全。岱善內倚中國,外結建州,陰折北關謀,實制東陲勝策也。」夏四月,遂釋康古魯而諭之曰:「中國立岱善,以萬故;囚汝,以助北關侵岱善也。汝亦萬子,不忍殺。今釋汝,和諸酋,修父業。岱善安危,汝則任之。」康古魯聽命,因令岱善以叔父事康古魯,以祖母事溫姐,刑牲盟;且進布寨、納林布祿使者誡諭之,為均兩部,敕孟格布祿出岱善妻子五人,及所部種人三百二十三、婦稚五百四十三、馬牛羊數百,歸岱善。康古魯偕溫姐歸故寨,居月餘,康古魯病且死,語溫姐及孟格布祿,戒部曲毋盜邊負明恩。康古魯死,孟格布祿謀盡室徙依葉赫,度溫姐不從,微告布寨、納林布祿以兵至。孟格布祿縱火燔其居,趣溫姐行,溫姐不可,強扶持上馬,鬱鬱不自得,七月亦死。

布寨、納林布祿誘孟格布祿圖岱善如故。成遜令諸酋面相要釋憾,並入貢,而太祖日強盛,布寨、納林布祿與有隙。二十一年夏六月,糾孟格布祿及烏喇、輝發四部合兵攻太祖,略戶布察寨。太祖率兵追之,設伏於中途,引兵略哈達富兒家齊寨。哈達兵至,太祖欲使退,以單騎殿。孟格布祿以三騎自後相迫,一騎出於前,太祖引弓射?引敵至設伏所,揮前騎,前騎在右,回身自馬項上發矢,矢著於馬腹,遂逸去。三騎驟至,太祖馬驚幾墜,右足絓於鞍,復乘,遂射孟格布祿馬踣地,其從者秦穆布祿授以己馬,挾以馳。太祖率所部兵騎者三、步者二十,逐而擊之,斬十二人,獲甲六、馬十八,以還。九月,復從布寨、納林布祿以九部之兵三萬人攻太祖,戰於黑濟格城下,九部之兵熸,布寨殲焉。

二十五年,葉赫諸部請成於太祖,盟定輒背之。二十六年,孟格布祿所居城北溪流血。二十七年秋,納林布祿攻孟格布祿,孟格布祿不能支,以其三子質於太祖,乞師。太祖使費英東、噶蓋以兵二千戍哈達。納林布祿恐,乃構明開原譯者為書,誘孟格布祿使貳於明,將襲擊費英東等。費英東等詗得之,以告太祖。九月丁未朔,太祖帥師攻哈達。貝勒舒爾哈齊請為前鋒,薄孟格布祿所居城。兵出,舒爾哈齊使告太祖曰:「彼城兵出矣!」太祖曰:「豈為此城無兵而來耶?」躬督兵進。舒爾哈齊兵塞道,太祖軍循城行,城上發矢多傷者,遂攻城,癸丑,克之。揚古利生得孟格布祿,太祖命勿殺,召入謁,賜以所御貂帽、豹裘,置帳中。既,孟格布祿與噶蓋謀為亂,事洩,乃殺之。

二十九年春正月,太祖以女妻孟格布祿子吳爾古代,明使來讓,太祖遣吳爾古代還所部。納林布祿歸所掠敕六十道,請於明,補雙貢如故事。已而,納林布祿復糾蒙古掠哈達。哈達饑,乞糴於明,明不與,至鬻妻子、奴僕以食。太祖周恤之,遂以吳爾古代歸。哈達亡。

楊吉砮,葉赫部長,孝慈高皇后父也。其先出自蒙古,姓土默特氏,滅納喇部據其地,遂以地為姓;後遷葉赫河岸,因號葉赫。其貢於明,取道鎮北關,地近北,故明謂之北關。

始祖星根達爾漢生席爾克明噶圖,席爾克明噶圖生齊爾噶尼。正德初,齊爾噶尼數盜都督僉事。褚?邊,斬開原市。八年,其子褚孔格糾他酋加哈復為亂,旋就撫,授達喜木魯孔格阻兵數反覆,為哈達部長旺濟外蘭所殺,明賜敕書及所屬諸寨,皆為所奪。

褚孔格子太杵。太杵子二:長,清佳砮;次即楊吉砮。能撫諸部,依險築二城,相距可數里,清佳砮居西城,楊吉砮居東城,皆稱貝勒。明人以譯音,謂之「二奴」。是時哈達萬汗方強,楊吉砮弟兄事萬謹,萬納其女弟溫姐,藉勢浸驕,數糾建州王杲侵明邊。明討王杲,而清佳砮,楊吉砮不與,蓋萬實庇之,既又以女妻楊吉砮。然楊吉砮兄弟日夜思復先世褚孔格之仇,怨萬。會萬老,勢衰,楊吉砮復婚於哈屯恍惚太,以隙復故地季勒諸寨。萬子扈爾干所屬白虎赤等先後叛歸楊吉砮,楊吉砮勢日盛,萬遂以憂憤死。死而諸子內爭,其庶孽康古魯亡抵清佳砮,清佳砮妻以女,益間萬子孫使自相圖。

既而太祖兵起,嘗如葉赫,楊吉砮顧知為非常人,謂太祖曰:「我有幼女,俟其長,當使事君。」太祖曰:「君欲結姻盟,盍以年已長者妻我?」楊吉砮對曰:「我雖有長女,恐未為嘉偶。幼女端重,始足為君配耳。」太祖遂納聘焉。

萬歷十一年,楊吉砮弟兄率白虎赤,益以暖兔、恍惚太所部萬騎,襲敗孟格布祿,斬三百級,掠甲胄一百五十;益借猛骨太、那木塞兵,焚躪孟格布祿所部室廬、田稼殆盡。明分巡副使任天祚使齎布帛及鐵釜,犒楊吉砮兄弟,諭罷兵。楊吉砮兄弟言:「必得敕書盡轄孟格布祿等然後已。」既,復焚孟格布祿及其仲兄所分莊各十,岱善莊一,脅所屬百餘人去。既,又以恍惚太二千騎馳廣順關,攻下沙大高寨,俘三百人,挾兵邀貢敕。

十二年,巡撫李松與總兵李成梁謀誅楊吉砮兄弟,哈達亦以請。明制,凡諸部互市,築墻規市場,謂之「市圈」。成梁使召楊吉砮弟兄,當賜敕賞賚,乃伏兵中固城,距開原可四十里,待其至。已而楊吉砮弟兄挾恍惚太二千騎擐甲叩鎮北關,守備霍九皋遣使讓之曰:「若來就撫,甲騎數千何為者?」楊吉砮兄弟乃請以三百騎入圈。李松令參將宿振武、李寧等夾城四隅為伏,戒軍中曰:「虜入圈,聽撫則張幟,按甲毋動;不則鳴?,皆鼓行而前,急擊之勿失。」松與任天祚坐南樓,使九皋諭楊吉砮兄弟。楊吉砮兄弟則益兵,以精騎三千屯鎮北關,而以三百騎入圈。楊吉砮兄弟請敕書部勒孟格布祿等,九皋譙讓之,漸急,楊吉砮兄弟瞋目,語不馴,李松奮髯抵幾叱之。九皋麾楊吉砮等下馬,楊吉砮目從者白虎赤,白虎赤拔刀擊九皋,微中右臂。九皋還擊楊吉砮從者一騎踣,餘騎?噪擊明兵。軍中?如雷,伏盡起,遂殺清佳砮、楊吉砮、白虎赤、清佳砮子兀孫孛羅、楊吉砮子哈兒哈麻,及諸從者,斬三百十有一級。勒兵馳出關,成梁先自中固城至,圍擊葉赫軍,斬千五百二十一級,奪馬千七百有三,遂深入楊吉砮弟兄所居寨。師合圍,旦日,諸酋出寨門蒲伏,請受孟格布祿約束,刑白馬攢刀為誓,成梁引師還。自是葉赫不敢出兵窺塞擾哈達為亂。明總督張佳胤等以陣斬「二奴」聞,成梁、松、天祚、九皋、振武、寧予廕進秩有差。

居數年,清佳砮子布寨、楊吉砮子納林布祿繼為貝勒,收餘燼,謀傾哈達報世仇,挾以兒鄧數侵掠,闌入威遠堡。納林布祿尤狂悖,要貢敕如其諸父,頻歲糾恍惚太攻岱善不已;且因其姑溫姐煽孟格布祿、唐古魯圖岱善,俾哈達內訌。會明助岱善,襲執康古魯。

十六年二月,巡撫顧養謙決策討布寨、納林布祿。成梁帥師至海州,雪初消,人馬行淖中,馬足膠不可拔。成梁計擊虜利月明,軍抵開原已下弦,不如三月往,遂壁海州,養謙壁遼陽。是歲,河西大饑,斗米錢三千,菽二千,發海州、遼陽穀贍軍。月將晦,成梁自海州乘傳出,三月十有三日,至開原。令岱善軍以白布綴肩際為幟,雞鳴,發威遠堡,行三十里,至葉赫屬酋落羅寨。成梁使召落羅,落羅駭兵至,迎謁,命以一幟樹寨門,材官十人守之,戒諸軍毋犯;挾落羅及其從者三騎俱,又行三十里,至葉赫城下。布塞棄西城,奔納林布祿與明軍夾道馳,明軍不敢先發。二酋麾其騎突明軍,殺三人,成梁乃縱兵?,並兵以拒,其擊之。游擊將軍吳希漢先驅,流矢集於面,創甚,弟希周奮起,斬虜騎射希漢者,亦被創。明軍如墻進,葉赫兵退入城守。城以石為郭,郭內外重?障,以巨桁為柵。城中有山,鑿山周遭為?,絕峻,為羅城其上,外以石,內以木,又二重,中構八角樓,置妻孥、財貨。明師攻二日,破郭外柵二重。城上木石雜下,先登者輒死,城堅不可拔。成梁乃斂兵,發巨?擊城,城壞,穿樓斷桁,葉赫兵死者無算,殲其酋把當亥,斬級五百五十四,城中皆號泣。明軍車載雲梯至,直立,齊其內城,將置巨?其上。二酋始大懼,出城乞降,請與南關分敕入貢。成梁令毋攻,燔雲梯,戒諸軍毋發其窖粟,遂引師還。四月朔,釋康古魯遣還,因進葉赫使者諭曰:「往若?順,朝廷賞不薄。江上遠夷以貂皮、人參至,必藉若以通。若布帛、米鹽、農器仰給於我,耕稼圍獵,坐收木枲、松實、山澤之利,為惠大矣。今貢事絕,江上夷道梗,皆怨若。我第傳檄諸部,斬二酋頭來,俾為長,可無煩兵誅也。今貸若,若何以報?」遂與哈達均敕。永樂初,賜海西諸部敕,自都督至百戶,凡九百九十九道。至是,畀哈達、葉赫分領之,以哈達?順,使贏其一。

秋九月,納林布祿送其女弟歸太祖,太祖率諸貝勒迎之,大宴成禮,是為孝慈高皇后。

十九年,納林布祿令宜爾當、阿擺斯漢使於太祖,且曰:「扈倫諸部與滿洲語言相通,宜合五為一。今屬地爾多我寡,額爾敏、扎庫木二地,盍以一與我!」太祖曰:「我為滿洲,爾為扈倫,各有分地。我毋爾取,爾毋我爭。地非牛馬比,豈可分遺?爾等皆知政,不能諫爾主,奈何強顏來相瀆耶!」遣其使還。既而納林布祿又令尼喀里、圖爾德偕哈達、輝發二部使者復至,太祖與之宴。圖爾德起而請曰:「我主有傳語,恐為貝勒怒。」太祖問:「爾主何語?我不爾責。」圖爾德曰:「我主言曩欲分爾地,爾靳不與。儻兩國舉兵相攻,我能入爾境,爾安能蹈我地乎?」太祖大怒,引佩刀斷案曰:「爾葉赫諸舅,盍嘗躬在行間,馬首相交,裂甲毀胄,堪一劇戰耶?哈達惟內訌,故爾等得乘隙掩襲,何視我若彼易與也!吾視蹈爾地,如入無人境,晝即不來,夜亦可往,爾其若我何!」因詆布寨、納林布祿父見殺於明,至不得收其骨,奈何出大言,以其語為書,遣巴克什阿林察報之。布寨要至其寨,不令見納林布祿,遣還。

未幾,長白山所屬硃舍裏、訥殷二路引葉赫兵劫太祖所屬東界洞寨。二十一年夏六月,扈倫四部合兵攻太祖,布寨、納林布祿為戎首,劫戶布察寨。太祖以師御之,遂侵哈達。秋九月,復益以蒙古科爾沁、席北、卦爾察三部,硃舍裏、訥殷二路,攻太祖,謂之「九姓之師」。太祖將出師,祀於堂子,祝曰:「我初與葉赫無釁,葉赫橫來相攻,糾集諸部,為暴於無辜,天其鑒之!」又祝曰:「原敵盡垂首,我軍奮揚,人不遺鞭,馬無顛躓,惟天其助我!」是時,葉赫兵萬人,哈達、烏喇、輝發三部合兵萬人,蒙古科爾沁三貝勒及席北、皆懼,太祖戒勉之。朝發虎闌哈達,夕宿扎喀?卦爾察三部又萬人,凡三萬人。太祖兵少,城。葉赫兵方攻黑濟格城,未下。旦日,太祖師至,面城而陣,使額亦都以百人先。葉赫兵罷攻城來戰,太祖軍迎擊,斬九級,葉赫兵小卻。布寨、金臺石及蒙古科爾沁三貝勒復並力合攻,金臺石者,納林布祿弟也。布寨將突陣,馬觸木,踣,太祖部卒吳談趨而前,伏其身刺殺之。葉赫兵見布寨死,皆痛哭,陣遂亂。九姓之師以此敗。布寨死,子布揚古嗣為貝勒。

二十五年春正月,扈倫諸部同遣使行成於太祖曰:「吾等兵敗名辱,繼自今原締舊好,申之以婚媾。」布揚古請以女弟歸太祖,金臺石請以女妻太祖次子臺吉代善,上許之,具禮以聘。宰牛馬告天,設?酒、塊土及肉、血、骨各一器,四國使者誓曰:「既盟之後,茍棄婚媾,背盟約,如此土,如此骨,如此血,永墜厥命!若始終不渝,飲此酒,食此肉,福祿永昌。」太祖誓曰:「彼等踐盟則已,有或渝者,待三年不悛,吾乃討之。」布揚古女弟,高皇后侄也,是時年十四。未幾,太祖遣將穆哈連侵蒙古,獲馬四十。納林布祿邀奪其馬,執穆哈連歸於蒙古。烏喇貝勒布占泰亦背盟結納林布祿。二十七年,太祖克哈達。以明有責言,使哈達故貝勒孟格布祿子吳爾古代還所部。二十九年,納林布祿以兵侵之,太祖遂以吳爾古代歸。三十一年秋九月,高皇后疾篤,思見母,太祖使迎焉。納林布祿不許,令其僕南太來視疾,太祖數之曰:「汝葉赫諸舅無故掠我戶布察寨,又合九姓之師而來攻我,既乃自服其辜,歃血誓天為盟誓,而又背之,許我國之女皆嫁蒙古。今我國妃病篤,欲與母訣,而又不許,是終絕我也!」既而,高皇后崩。三十二年春正月,太祖帥師攻葉赫,克二城,曰張,曰阿氣蘭;取七寨,俘二千餘人而還。

三十五年,納林布祿聞輝發貝勒拜音達裡使貳於太祖,太祖以是取輝發,納林布祿不能救;而布揚古女弟受太祖聘,十六年不遣,年三十,烏喇貝勒布占泰將強委禽焉。四十年,太祖討布占泰。四十一年,師再舉,遂克烏喇,布占泰亡奔葉赫。布揚古欲遂以女弟嫁之,布占泰遜謝不敢娶,為別婚。是時納林布祿已死,其弟金臺石嗣為貝勒,與布揚古分居東、西城如故。秋九月,太祖使告葉赫執布占泰以獻,使三往,不聽。太祖謀伐之,先期遣第七子巴布泰率所屬阿都、乾骨裏等三十餘人質於明。至廣寧,謁巡撫都御史張濤,請敕葉赫遣布占泰,濤以聞,神宗下部議,以為質子真偽莫可辨,拒勿納。太祖乃以四萬人會蒙古喀爾喀貝勒介賽伐葉赫。會有逋卒洩師期,葉赫收張、吉當阿二路民堡。太祖圍兀蘇城,城長山談、扈石木降,太祖飲以金?,賜冠服;遂略張、吉當阿、呀哈、黑兒蘇、何敦、克布齊賚、俄吉岱七城,下十九寨,盡焚其廬舍儲峙,以兀蘇城降民三百戶還。

葉赫愬於明,以兵援,遇介賽,戰勝,遂遣使讓太祖,令游擊馬時柟、周大岐率兵千,挾火器,戍葉赫。太祖至撫順,投書游擊李永芳,申言:「侵葉赫,以葉赫背盟,女已字,悔不遣,又匿布占泰;故與明無怨,何遽欲相侵?」遂引師還。

金臺石有女,育於其兄納林布祿,嫁介賽。金臺石既為貝勒,殺納林布祿妻,介賽假辭為外姑復仇,覬得布揚古女弟以解。布揚古女弟誓死不原行。介賽治兵攻葉赫。既而喀爾喀貝勒巴哈達爾漢為其子莽古爾代請婚,布揚古將許之。明邊吏諭布揚古,姑留此女,毋使葉赫。四十三年夏?縻;而以兵分屯開原、撫順及鎮北堡為犄角,?太祖及介賽望絕,冀相五月,布揚古遂以其女弟許莽古爾代,秋七月婚焉。太祖聞,諸貝勒皆怒,請討葉赫,不許。請侵明,又不許,且曰:「此女生不祥,哈達、輝發、烏喇三部以此女構怨,相繼覆亡。今明助葉赫,不與我而與蒙古,殆天欲亡葉赫,以激其怒也。我知此女流禍將盡,死不遠矣。」布揚古女弟嫁莽古爾代未一年而死,死時年蓋三十四,明所謂「北關老女」者也。是歲為太祖天命元年。

太祖既稱帝建國,始用兵於明。三年,攻撫順、清河。明經略侍郎楊鎬使諭葉赫發兵撓太祖。秋九月,金臺石子德爾格勒侵太祖,克一寨,俘四百七人,斬八十四級。明賜以白金二千兩、採緞表裏二十。四年春正月,太祖謀報之,使大貝勒代善以兵五千戍札喀關阻明師,而躬督兵伐葉赫。辛卯,入其境,經克亦特城、粘罕寨,至葉赫城東十里,克大小屯寨二十餘。葉赫乞援於明,明開原總兵馬林以師至,合城兵而出,見太祖兵盛,不敢擊。太祖亦引還。二月,楊鎬大舉伐太祖,使都司竇永澄徵兵於葉赫,葉赫以二千人應。至三岔北,明師覆,永澄死之。太祖謀使所屬詐降於金臺石,金臺石不應。六月,太祖攻開原,葉赫復以二千人援,至則開原已下。秋八月,經略侍郎熊廷弼初視事,葉赫使期復開原,廷弼厚賚之。

太祖惎葉赫,八月,大舉伐之。己巳,師出,聲言向沈陽,以綴明師。壬申,至葉赫城下,太祖攻金臺石東城,而命諸貝勒馳向西城取布揚古。布揚古與其弟布爾杭古以城兵出西郭,陟岡,鳴角而噪,望太祖軍盛,斂兵入。諸貝勒遂督軍合圍。太祖圍東城,入其郛,布攻具,呼金臺石降,不聽,曰:「吾非明兵比,等丈夫也,肯束手降乎?寧戰而死耳。」太祖麾兵攻城,兩軍矢交發,太祖軍擁楯陟山麓,將穴城,城上下木石,擲火器。太祖軍冒進,穴城,城圮,師入,城兵迎戰,敗潰,皆散走。太祖使執幟約軍士毋妄殺,執黃蓋,令降者免死,城民皆請降。金臺石以其孥登臺,太祖軍就圍之,命之下。金臺石求見四貝勒盟而後下,四貝勒為太宗,高皇后所出,金臺石甥也。四貝勒方攻西城,太祖召之至,使見金臺石。金臺石曰:「我未嘗見我甥,真偽烏能辨?」費英東、達爾哈在側,曰:「汝視常人中有奇偉如四貝勒者乎?且曩與汝通好時,嘗以媼往乳汝子德爾格勒,盍使媼辨之!」金臺石曰:「何用媼為也!觀汝輩辭色,特誘我下殺我耳。我石城鐵門既為汝破,縱再戰,安能勝?特我祖父世分土於斯,我生於斯,長於斯,則死於斯可已。」四貝勒勸之力,金臺石使阿爾塔石先見太祖,太祖復令諭降。金臺石又求見其子德爾格勒,德爾格勒至,金臺石終不下。四貝勒將縛德爾格勒,德爾格勒曰:「我年三十六,乃今日死耶!殺可也,何縛焉?」四貝勒以德爾格勒見太祖,太祖撤所食食之,命四貝勒與共食。且曰:「爾兄也,善遇之!」金臺石妻將其幼子下,金臺石引弓,其從者復甲。太祖軍進毀臺,金臺石縱火,屋宇皆燼。太祖諸將謂金臺石且死,軍退。火燼,金臺石潛下,為太祖軍所獲,縊殺之。

諸貝勒圍西城,布揚古聞東城破,與布爾杭古使請降,並請盟無死。大貝勒曰:「汝輩畏死,盍以汝母先,汝母我外姑也,我寧能殺之?」布揚古母至軍,大貝勒以刀劃酒,誓,飲其半,使送布揚古、布爾杭古飲其半,乃降。大貝勒以布揚古見太祖,布揚古行復勒馬,大貝勒挽其轡,命毋沮。見太祖,布揚古以一膝跪,不拜而起。太祖取金?授之,布揚古復以一膝跪,酒不竟飲,不拜而起。太祖命大貝勒引去,以其懟也,即夕亦縊殺之。貸布爾杭古。攻殺明游擊馬時楠戍兵,殲焉。楊鎬聞警,使總兵李如楨自撫順出,張疑兵為葉赫聲援,得十餘級而退。

神宗命給事中姚宗文行邊,求葉赫子孫,德爾格勒有女子子二,嫁蒙古,各賜白金二千。明臣請為金臺石、布揚古立廟,又以哈達餘裔王世忠為金臺石妻至,授游擊,將以風諸部,然葉赫遂亡。

太祖以德爾格勒歸,旗制定,隸滿洲正黃旗,授三等副將。太宗天聰三年,改三等梅勒章京,卒,八年,子南楮嗣。十年,察哈爾林丹汗殂,所部內亂,太宗遣貝勒多爾袞帥師略地。林丹汗福金號蘇泰太后,南楮女兄也,因使南楮諭降。南楮至其帳,呼其人出,語之曰:「爾福金蘇泰太后之弟南楮至矣!」其人入告,蘇泰太后大驚,使故葉赫部來媵者視之,信。蘇泰太后號而出,與南楮相抱持,遂使其子額哲出降。南楮旋以罪奪爵,復以南楮弟索爾和嗣。乾隆初,改二等男。

布爾杭古分隸正紅旗,亦授三等副將。再傳,坐事,奪世職。

布占泰,烏喇部長,太祖婿也。烏喇亦扈倫四部之一,與哈達同祖納齊卜祿。納齊卜祿五傳至克什納、古對硃顏兄弟。克什納之後為哈達部。古對硃顏生太蘭,太蘭生布顏。布顏收附近諸部,築城洪尼,濱烏喇河,因號烏喇,為貝勒。

布顏子二:布干、博克多。布顏死,布乾嗣為部長。布干子二:滿泰、布占泰。布干死,滿泰嗣為部長。萬歷二十一年夏六月,葉赫糾扈倫諸部侵太祖,滿泰以所部從。秋九月,葉赫再糾扈倫諸部,及蒙古科爾沁所部,及滿洲長白山所屬,大舉分道侵太祖。滿泰使布占泰以所部從,與哈達貝勒孟格布祿、輝發貝勒拜音達里合軍萬人。戰敗,葉赫貝勒布寨死於陣,科爾沁貝勒明安單騎走。戰之明日,卒有得布占泰者,縛以見太祖,曰:「我獲俘,將殺之。俘大呼勿殺,原自贖。因縛以來見。」跽太祖前,太祖問誰何,對曰:「烏喇貝勒滿泰弟布占泰也,生死惟貝勒命。」叩首不已。太祖曰:「汝輩合九部兵為暴於無辜,天實厭之。昨陣斬布寨,彼時獲汝,汝死決矣!今見汝,何忍殺?語有之曰:『生人勝殺人,與人勝取人。』」遂解其縛,與以猞猁猻裘,撫育之。

居三年,二十四年秋七月,遣還所部,使圖爾坤黃占、博爾焜蜚揚占護行。未至,滿泰及其子淫於所部,皆見殺。布占泰至,滿泰有叔興尼牙,將殺而奪其地,二使者嚴護之,興尼牙謀不行,乃出奔葉赫,卒定布占泰而還。冬十二月,布占泰以女弟妻貝勒舒爾哈齊。二十五年春正月,與葉赫諸部同遣使請盟,盟甫罷,布占泰旋執太祖所屬瓦爾喀部安褚拉庫所推者羅屯、噶石屯、汪吉努三人送葉赫,使招所部貳於太祖;又以滿?、內河二路頭人為泰妻都都祜所寶銅錘?納林布祿。二十六年春正月,太祖命臺吉褚英等伐安褚拉庫路。冬十二月,布占泰來謁,以三百人俱,太祖以舒爾哈齊女妻之,賜甲胄五十,敕書十道,禮而遣之。二十九年冬十一月乙未朔,布占泰以其兄滿泰女歸太祖。布占泰初聘布寨女,既又聘明安女,以鎧胄、貂、猞猁猻裘、金銀、駝馬為聘,明安受之而不予女。三十一年春正月,布占泰使告太祖曰:「我昔被擒,待以不死,俾我主烏喇,又妻我以公主,恩我甚深。我孤恩,嘗聘葉赫、蒙古女,未敢以告。今蒙古受聘而復悔,我甚恥之!乞再降以女,當歲歲從兩公主來朝。」太祖允其請,又以舒爾哈齊女妻焉。

三十五年春正月,東海瓦爾喀部蜚悠城長策穆特黑謁太祖,自陳屬烏喇,為布占泰所虐,乞移家來附。太祖命貝勒舒爾哈齊、褚英、代善率諸將費英東、扈爾漢、揚古利等以兵三千至蜚悠城,收環城屯寨五百戶,分兵三百授扈爾漢、揚古利護之先行。布占泰使其叔博克多將萬人要諸途。日暮,扈爾漢依山結寨以相持。翌日,烏喇兵來攻,揚古利率兵擊敗之,烏喇兵引退,渡河陟山為固。褚英、代善等率後軍至,緣山奮擊,烏喇兵大敗,代善陣斬,俘其將常住、胡里布等,斬三千級,獲馬?博克多。是日晝晦,雪,甚寒,烏喇兵死者甚五千、甲三千以還。

三十六年春正月,太祖復命褚英及臺吉阿敏將五千人伐烏喇,克宜罕阿麟城,斬千人。布占泰糾蒙古科爾沁貝勒甕阿代,合軍屯所居城外二十里,畏褚?,獲甲三百,俘其餘英等軍強,不敢進,引還。秋九月,遣使復請修好,太祖使報問。布占泰執納林布祿所部種人五十輩,?太祖使者盡殺之。又遣使來請曰:「我數背盟,獲罪於君父,若更以女子子妻我,撫我如子,我永賴以生矣。」太祖復允其請,又以女子子妻之。

四十年,布占泰復背盟,秋九月,侵太祖所屬虎爾哈路,復欲娶太祖所聘葉赫貝勒布寨女,又以鳴鏑射所娶太祖女。太祖聞之怒,癸丑,親率兵伐之。庚申,兵臨烏喇河,布占泰以所部迎戰,夾河見太祖軍甲胄甚具,士馬盛強,烏喇兵人人惴恐,不敢渡。太祖循河行,下河濱屬城五,又取金州城,遂駐軍焉。冬十二月辛酉朔,太祖以太牢告天祭纛,青白氣見東方,指烏喇城北。太祖屯其地三日,盡焚其儲峙。布占泰晝引兵出城,暮入城休。太祖率兵毀所下六城,廬舍、糗糧皆燼,移軍駐伏爾哈河渡口。布占泰使使者三輩以舟出見太祖,布占泰率其弟喀爾喀瑪及所部拉布泰等繼以舟出,?舟中而言曰:「烏喇國即父國也,幸毋盡焚我廬舍、糗糧。」叩首請甚哀。太祖立馬河中,數其罪。布占泰對曰:「此特讒者離間,使我父子不睦。我今在舟中,若果有此,惟天惟河神其共鑒之!」拉布泰自旁儳曰:「貝勒既以此怒,曷不以使者來詰?」太祖責之曰:「我部下豈少汝輩人耶?事實矣,又何詰?河冰無時,我兵來亦無時。汝口雖利,能齒我刃乎?」布占泰大懼,止拉布泰毋言。喀爾喀瑪為乞宥,太祖乃命質其子及所部大酋子,遂還營。五日引還,度烏喇河濱邑麻虎山巔,以木為城,留千人戍焉。

十二月,有白氣起烏喇,經太祖所居南屬虎攔哈達山。布占泰旋復背盟,幽太祖及舒爾哈齊女,將以其女薩哈廉子綽啟鼐及所部大酋子十七人質於葉赫,娶太祖所聘貝勒布寨女。四十一年春正月,太祖聞,復率兵伐之。布占泰期以是月丙子送其子出質,而太祖軍以乙亥至,攻下孫扎泰及郭多、俄謨三城。丙子,布占泰以兵三萬越伏爾哈城而軍,太祖猶欲諭之降。諸貝勒代善、阿敏,諸將費英東、何和里、扈爾漢、額亦都、安費揚古皆請戰,曰:「我利速戰,但慮彼不出耳。今既出,平原廣野,可一鼓擒也!舍此不戰,厲兵秣馬,何為乎來?且使布占泰娶葉赫女,辱莫甚焉!雖後討之,何益?」太祖曰:「我荷天寵,自少在兵間,遇勁敵,無不單騎突陣者!今日何難率汝輩身先搏戰。但慮諸貝勒、諸將或一二夷傷,我所深惜,故欲出萬全,非有所懼也。今汝輩志一,即可決戰。」因命被甲,諸貝勒、諸將則大歡,一軍盡甲,令曰:「勝即奪門,毋使復入!」乃率兵進。布占泰自伏爾哈城率兵還,令其軍皆步為陣,兩軍距百步。太祖軍亦皆舍馬步戰,矢交如雨,呼聲震天。太祖躬入陣,諸貝勒、諸將從之縱擊,烏喇兵大敗,死者十六七。師入,太祖坐西門樓,命樹幟。布占泰餘兵不滿百,還至城下,見幟則大奔。遇代善,布占泰兵皆潰,僅以身免,奔葉赫。太祖使請於葉赫,葉赫不聽。後七年,太祖克葉赫,布占泰蓋已前死。

拜音達里,輝發部長也。輝發亦扈倫四部之一,其先姓益克得里氏,居黑龍江岸。尼馬察部有昂古里星古力者,自黑龍江載木主遷於渣魯,居焉。時扈倫部噶揚噶、圖墨土二人居張城,二人者姓納喇氏,昂古里星古力因附其族,宰七牛祭天,改姓納喇,是為輝發始祖。

昂古里星古力子二:留臣、備臣。備臣子二:納領噶、耐寬。納領噶生拉哈都督,拉哈都督生噶哈禪都督,噶哈禪都督生齊訥根達爾漢,齊訥根達爾漢生王機褚。王機褚收鄰近諸部,度輝發河濱扈爾奇山,築城以居,因號輝發。城負險堅峻,蒙古察哈爾部扎薩克圖土門汗嘗自將攻之,不能克。王機褚死時,其長子前死,孫拜音達里,殺其叔七人,自立為貝勒。

萬歷二十一年夏六月,葉赫糾哈達、烏喇諸部侵太祖,拜音達里以所部從。秋九月,復舉兵,拜音達里與哈達貝勒孟格布祿、烏喇貝勒布占泰合兵萬人,兵敗,還。二十三年夏六月,太祖攻輝發,取所屬多璧城,輝發將克充格、蘇猛格二人戍,殲焉。二十五年春正月有攜心。拜音?,與葉赫諸部同遣使行成於太祖。居數年,拜音達里之族有叛附葉赫者,部達里懼,以所屬七人之子質於太祖,太祖發兵千人助之鎮撫。葉赫貝勒納林布祿使告拜音達里曰:「爾以質子歸我,亦歸爾叛族。」拜音達里信之,乃曰:「吾其中立於滿洲、葉赫二國之間乎!」遂取質子還,以其子質於納林布祿。納林布祿殊無意歸叛族,拜音達里以告太祖,且曰:「吾前者為納林布祿所誑,怙舊恩,敢請婚。」太祖許之。既而拜音達里背約不娶,太祖使詰之曰:「汝昔助葉赫,再舉兵侵我。我既宥爾罪,復許爾婚。今背約不娶,何也?」拜音達里詭對曰:「吾子質葉赫,須其歸,娶爾女,與爾合謀。」因築城三重自固。及其子自葉赫歸,太祖復遣使問,拜音達里倚城堅,度兵即至,足以守,遂負盟。三十五年秋九月丙申,長星出東方指輝發,八夕乃滅。乙亥,太祖率師討之。甲辰,合圍,遂克之,殺拜音達里及其子,安集其民,帥師還。輝發亡。

論曰:扈倫四部,哈達最強,葉赫稍後起,與相埒,烏喇、輝發差弱。其通於明,皆,令於所部則曰「國」。太祖漸強盛,四部合攻之,兵敗縱散,以次覆滅。太祖與?以所領四部皆有連,奪其地,殲其酋,顯庸其族裔。疆埸之事不以婚媾逭,有時乃藉口以啟戎,自古則然,不足異也。


\end{pinyinscope}