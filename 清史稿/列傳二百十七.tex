\article{列傳二百十七}

\begin{pinyinscope}
雷正綰陶茂林曹克忠胡中和何勝必蕭慶高楊復東

周達武李輝武唐友耕

雷正綰,字偉堂,四川中江人。由把總從軍湖北,積功至游擊,賜號直勇巴圖魯。咸豐八年,從多隆阿援安徽石牌、潛山、太湖、桐城,諸戰皆功最,累擢副將,以總兵記名。十一年,敗黃文金於蔣家山、項家河、江家河、麻子嶺,一月五捷,授陜安鎮總兵。同治元年,克廬州,以提督記名。

從多隆阿援陜西,詔正綰先赴本任,未至,擢陜西提督,幫辦軍務,駐西安。二年,多隆阿既克東路,令正綰規三原,屢破賊。會解鳳翔圍,進援甘肅,連戰靈臺、鎮原,皆捷。三年,破賊崇仁、新城,進逼平原。會多隆阿卒於軍,都興阿繼督師甘肅,正綰仍奉命為副。克平涼,斬賊首鐵酉、羽輕林,賜黃馬褂。於是固原踞賊悉竄龍山鎮,追擊敗之。賊回竄,又陷固原。正綰疾趨蓮花城,欲襲其巢穴,遇伏,受矛傷,部下亡千餘人,裹創攻蓮花城,克之,詔嘉其勇。四年春,克固原,進攻黑城子,斬賊首黑虎。克官橋、李旺二堡,擒賊首木棍等。乘勝薄預望城,破下馬關、半角城賊壘,進規靈州,分兵解安定圍。

七月,偕曹克忠攻金積堡,軍餉不繼,為賊所圍,饑潰。正綰自劾,褫勇號、黃馬褂,黜幫辦,歸總督楊岳斌節制。正綰弟總兵雷恆及副將李高啟等以主將失職,煽亂,犯涇州,正綰不能制止,憤欲自裁。詔念前功,不加罪,責令整軍剿賊圖自贖。命巡撫趙長齡會楊岳斌按訊,正綰縛送雷恆等置之法。當事變初起,謠諑紛淆,詔斥劉蓉張皇妄奏,許正綰專摺奏事以慰之。所部招集增募僅三千人。

五年,蘭州兵變,回匪窺伺,正綰支拄於平涼、固原之間,破賊於橫河川,克平涼,復黃馬褂、勇號。六年,左宗棠入陜督師,正綰率軍助剿,援慶陽。七年,兩破賊於長武,克黃家堡。八年,會攻董志原,克之,晉號達春巴圖魯。又破白彥虎於李旺堡。會攻金積堡,當西路,屢克要隘,合圍。及馬化隆伏誅,被優敘。

光緒十年,法越兵事起,命率甘軍駐鳳凰城,固邊防,事定回任。兩遇萬壽慶典,加太子少保、尚書銜。二十一年,循化撒回倡亂,督剿無功,革職留任。二十三年,罷,卒於家,仍以前勞賜恤。

陶茂林,湖南長沙人。以武童入湘軍,轉戰湖北、江西,積功至游擊。咸豐八年,胡林翼調為楚軍營官,扼黃州,破賊霍山、舒城,克建德,擢參將。十年,從多隆阿破賊於桐城掛車河,擢副將。十一年,破賊施家山,擒其渠,及克安慶,賜號鍾勇巴圖魯。同治元年,克廬州,先登,以總兵記名。

遂從多隆阿西征,破賊於武關。從剿回匪,解同州圍。克羌白鎮、王閣村賊巢,功皆最,授漢中鎮總兵。鳳翔被圍久,茂林率三千人往援,連戰解圍,擢甘肅提督。粵匪出寶雞山口,擾郿縣、盩厔,茂林要擊雨門鎮、二嶺關,迭敗之。進克汧陽、隴州。遂會雷正綰分道規平涼,陣斬賊首木仲沅訥三等,克之,賜黃馬褂。進拔張家川賊巢,破龍山鎮、蓮花城援賊,解安定圍。克金縣,破賊惠城,擒其渠黑牙古。四年,克黑城賊巢,解靖遠圍。進攻會寧,所部索餉譁潰五營,賊乘之,六營皆陷。茂林調後路四營來援,突圍出,退駐安定。巡撫劉蓉疏陳甘軍積弊,論茂林不職,茂林亦以兵潰自劾。詔斥廢法營私,以致兵潰而叛,遂罷職,歸。

十年,貴州巡撫曾璧光調茂林赴黔協剿。復新城,克安順賊巢,平古州、丹江苗,復原官。光緒二年,收復下江、永從各城,破六峒賊巢,加頭品頂戴,晉號愛星阿巴圖魯。十六年,署古州鎮總兵,卒於官。

曹克忠,直隸天津人。初投效湘軍,嗣從多隆阿,積功至都司。咸豐十年,令募五百人為忠字營,大破援賊於潛山、太湖,洊擢參將,賜號悍勇巴圖魯。掛車河之捷,擢副將。克桐城、宿松諸城,以總兵記名。同治元年,克廬州。

後從多隆阿西征,武關、同州諸戰皆從。二年,攻羌白鎮,克忠單騎往諭賊,賊請降,察其詐,潛師會攻,下之,乘勝奪王閣村,予一品封典。尋率烏拉馬隊及楚勇七營屯長安、鄠縣之間。光泰廟為入省要沖,賊踞之以扼糧路,克忠擊走之。分隊清西路餘匪,省城始安。以提督記名,授河州鎮總兵。渡渭連破賊於白起營、馬家埠、白吉原,邠州平,陜回西趨。三年,平麟游諸匪。會援甘肅,連破賊於西河口、黑水峪,赴河州本任。克秦安,解秦州圍,賜黃馬褂。

四年,攻蕭何城及馬定嘴,將臺、隆德諸堡,悉平。克海城,回匪並竄李旺堡、同心城,攻下之。偕雷正綰規取金積堡,屯強家沙窩,數有斬獲。輕進,為賊所包鈔,正綰軍先潰,克忠亦退。因前功免罪,授甘肅提督。時陶茂林、雷正綰軍相繼譁變,回氛益熾,自楊岳斌楚軍外,僅克忠一軍與之相持。克忠援鞏昌,賊敗走,又毀董家堡賊巢。五年,援洮州,次李岐山,回目馬芳乞降,誅其酋丁重選等而還。

蘭州標兵變,楊岳斌令克忠移軍鎮懾。克忠至,人心稍定,然糧餉俱竭,乞病回籍。十年,詔起赴陜接統淮軍,專防肅州。十一年,所部有結會匪者,甘軍馬世俊騎兵亦變,降捻多叛應,克忠遣兵平之。復乞病解軍事。十一年,署甘肅提督,尋解職歸。

光緒九年,命募六營防山海關。十年,授廣東水師提督。十一年,病罷,食全俸。二十年,命治天津團練,統津勝軍。二十二年,卒,賜恤。

胡中和,字元廷,湖南湘鄉人。咸豐初,從湘軍剿粵匪,積功擢把總。六年,從蕭啟江援江西,復袁州,超擢都司,賜花翎。七年,從克臨安,中砲傷,以游擊留湖南補用。八年,破賊上屯渡,乘勝復撫州,擢參將。九年,復南安,擢副將。石達開由寶慶竄廣西,陷興安,遣黨攻桂林,自率悍賊屯大溶江。中和從蕭啟江往援,大破賊於大溶江,賊竄貴州境,加總兵銜,賜號伊德克勒巴圖魯。十年,蕭啟江率軍援四川,中和從之。啟江卒於軍,中和偕何勝必、蕭慶高等分領其眾。

剿滇匪李永和於井研,連戰皆捷,賊解圍遁,以總兵記名。尋授四川建昌鎮總兵。十一年,永和竄踞富順牛腹渡,兩岸築堅壘,背水而陣。中和選銳卒沿河設伏,自率羸師誘之,賊大出,伏發,截其歸路,俘斬無算,賊壘盡夷,進解大邑之圍,予二品封典。

駱秉章督師蒞蜀,檄中和偕緒軍援綿州。滇匪藍朝柱在諸賊中最狡悍,圍綿州日久。軍至,連破之,圍始解,又敗之西山觀。朝柱竄丹棱,與李永和合攻眉州。中和馳援,賊分路來撲,中和突陣,矛傷腮,血殷衣,不顧,奮擊破之,解眉州圍。進攻丹棱,朝柱遁走,復其城,以提督記名。同治元年,擢雲南提督。李永和自眉州敗後,竄踞青神,諸軍進剿,數敗之,永和遁犍為龍場,負嵎死抗。中和圍之,壘石墻,編木柵,外浚深壕,密布梅花椿。賊知必死,突攻蕭慶高營,中和截擊,敗退,連戰七日。賊伏不出,乃使降賊譚仁曲持書約降,期會於豬市坡,預伏兵賊巢旁。永和與其黨卯得興數十騎來會,伏起分攻,焚其巢。永和、得興駭奔,追擒之,降其眾五千。詔嘉中和運籌決勝,生擒渠魁,賜黃馬褂。

石達開擾蜀邊,中和偕蕭慶高、何勝必合擊於橫江,走之。二年春,達開復分路犯蜀,自率大隊數萬由米糧壩渡金沙江。中和督軍扼化林坪、瀘定橋,擊破之,賊走工⼙部土司山中,達開旋就擒。調四川提督。三年,破滇匪於敘永。初,李永和既誅,餘黨竄陜西,至是入甘肅,陷階州。四年,中和偕總兵周達武往剿,毀龍王廟、三官殿賊壘,逼階州城下,掘地道轟城,克之,斬賊酋蔡昌齡,盡殲其黨。階州平,被珍賚。

冬,剿苗匪於建武,腰中彈傷,力戰敗之。五年,剿屏山賊,解馬邊圍,誅賊酋宋任傑等,餘匪悉平。十三年,調雲南提督。光緒二年,抵任。三年,平騰越夷匪。七年,丁母憂歸里。九年,卒,賜恤。

何勝必,湖南湘鄉人。咸豐中,勝必應募入湘軍,從蕭啟江轉戰江西、廣西,積功至副將。從入蜀,分統湘果右軍,破李永和於井研,又破之於資州,陣斬賊酋王二官,賜號御勇巴圖魯。十一年,會破滇匪藍朝柱於西山觀,又敗諸青衣壩,解眉州圍,追至青神,擒斬甚眾,授甘肅肅州鎮總兵。同治元年,會諸軍克青神,追賊宜賓,擒賊目周廷光。偕胡中和誘擒李永和於犍為龍場,二年,偕蕭慶高援漢中,戰油坊街,不利,漢中、城固相繼陷,革職留軍。三年,會攻法慈院賊壘,再敗之牟家壩,乘勝薄漢中城下,捻渠陳得才遁走,克漢中,復原官。又破陳得才於上元觀,克城固,進規階州。四年,卒於軍,賜恤,謚威愨。

蕭慶高,湖南湘鄉人。隸楚軍,積功至副將。蕭啟江援蜀,調從軍,以井研之捷,賜號果勇巴圖魯。破李永和於資州,以總兵記名。會剿藍朝柱,解綿州圍。同治二年,偕何勝必援漢中。油坊街之戰,勝必先敗,慶高赴援不及,同革職留軍。三年,克漢中,同復官。追賊至城固,梯城而入,賊潰走。四年,進攻洋縣,遣死士入城為內應,克之。賊酋曹燦章走踞八里坪,夾攻破之,燦章就擒,授漢中鎮總兵。五年,卒,謚武毅。

楊復東,湖南瀏陽人。咸豐十年,從胡中和援蜀。十一年,戰富順牛腹渡,解大邑圍,擢守備。敗藍朝柱於綿州,擢都司。又破朝柱於崇慶,毀石羊場,焚賊巢,擢游擊。同治元年,復丹棱,擢參將。克青神,平鐵山賊壘,擒李永和。擢副將。五年,總督駱秉章疏陳復東歷年防剿滇、黔諸賊功多,以總兵記名。七年,授四川川北鎮總兵。光緒二年,調雲南開化鎮。六年,卒。

周達武,字夢熊,湖南寧鄉人。咸豐四年,應募入李續賓營,從克岳州、武昌,累功擢守備。戰湖口,晉都司。達武每戰陷陣,手大旗蕩決,續賓異之,使領百人曰信字營,常為軍鋒。八年,克黃安,擢游擊,賜花翎。從續賓攻舒城,達武率死士先登,左耳受槍傷,克城後,留守。俄續賓覆軍三河,舒城守軍亦潰,達武以創重回湖南。九年,石達開圍寶慶,巡撫駱秉章令達武募五百人號曰章武軍,從知府劉岳昭援寶慶,守東關,屢拒戰破賊。圍解,擢參將。十年,援廣西,克富川平古城、連塘賊壘,復賀縣,擢副將,加總兵銜。石達開分黨犯永明、柘牌,連戰破之,擢總兵。十一年,會諸軍克會同,賊走湖北,陷來鳳。同治元年春,從劉岳昭攻克之,予二品封典。

駱秉章督師四川,調達武從剿。抵涪州,會賊酋周紹勇由大寧竄陜西,達武扼之窄子口,地當兩山間,令部將李輝武逾險而入,賊潰走,追至大竹安吉場,擒紹勇及其黨吳崇禮等,檻送成都斬之,賜號質勇巴圖魯。又破郭刀刀於儀隴大儀寨,陣斬其弟占彪及悍黨馬玉音,追奔至巴州鼎山鋪,擒刀刀,餘黨皆降。紹勇與刀刀並為蜀中劇賊,至是悉平,授四川建昌鎮總兵,加提督銜。二年,護理提督。

粵匪陳得才圍漢中,眾號十萬,石達開亦由高縣走寧遠,全蜀大震。達武增募軍四千人,往來游擊。三年,得才之黨梁福成合川匪蔡昌齡由漢中竄甘肅階州,達武議以剿為防,率師越境,攻克江東水、嚴家灣賊壘。進攻階州,自將臺山穴地達城根,地雷發,城崩,選鋒四百人先入,大軍繼之,遂克階州,斬福成、昌齡。以提督記名,並頒珍賚。尋平松潘叛番,授貴州提督,仍留防重慶,備滇邊。五年,剿平馬邊教匪,斬匪首宋仕傑、熊文才。

六年,捻匪竄陜西,左宗棠咨調會剿,令部將李輝武率三千赴陜。七年,破越巂惈夷於普雄,進克西昌交腳夷巢,斬級數千,諸夷悉降,賜黃馬褂,晉號博奇巴圖魯。九年,詔赴貴州提督任,率所部六千人行,沿途平苗砦。先是貴州剿寇仰客軍,出省城百里即莫能制馭。達武與巡撫議增募至三萬人,分任戰守,由龍里進凱渡,截上下游賊為二,復都勻,分軍破賊永寧、威寧。十年,遣鍾開蘭攻克麻哈州之高水塘等地數十砦;遣何世華破粵賊李文彩、苗酋李高腳於都勻、獨山,收復八寨、三腳諸城,並克鎮寧、歸化及吳秀河、斑竹園諸苗砦,復清平、黃平二城。始與楚軍席寶田合。十一年,會席軍敗苗酋張臭迷之黨於茶牛坡,斬馘甚眾,降者數萬。追至冷水溝,生擒賊酋,餘黨李高腳、李文彩竄荊蓬坎,分三路追擊,盡殄之。旋破群苗於清平香爐山,寶田擒張臭迷。苗疆平,予騎都尉世職。

光緒元年,乞病歸。三年,授甘肅提督。十年,肅州妖民王林倡亂高臺,討平之,斬王林。十九年,萬壽慶典,加尚書銜。二十年,卒官,賜恤,建專祠。

弟康祿,從達武剿賊廣西、湖南,歷保知縣。同治元年,從赴蜀,破周紹勇,擢知州。四年,從克階州,擢知府。從至貴州,總理營務。十一年,下游肅清,擢道員。駐軍普安新城,招撫流亡。十二年,會匪煽亂,康祿督親軍百人往討,眾寡不敵,死之。贈內閣學士,予騎都尉世職,謚壯節。

李輝武,湖南衡山人。周達武部將。咸豐中,從剿粵匪,洊擢游擊。十一年,從入四川,剿涪川鶴游坪踞賊,擒賊酋周紹勇、郭刀刀。輝武功為多,擢副將,賜號武勇巴圖魯。同治三年,從援階州,輝武由伍家坪進軍,扼州城外北山條竹埡。四年,攻破橋頭裡賊壘,又破賊於孟家莊,殲城外賊殆盡。穴地破城,輝武先登,擒賊目蔡四。巡部,以總兵記名。從討松潘叛番,拔其巢。尋攻黑河番,焚芝麻第五寨,餘寨皆降。乘勝連破大松樹及竹自三寨,以提督記名。

六年,捻匪竄陜西,輝武率步隊五營赴援,剿破汧陽、隴州、寶雞諸賊,西路肅清。八年,剿董志原竄匪,斃賊目王明章,晉號福凌阿巴圖魯,授漢中鎮總兵。九年,偕提督劉端冕分擊北山回匪,破翟三、禹得彥於縣頭鎮、陳村。十一年,擢甘肅提督,仍留防漢中。光緒四年,卒,賜恤。

輝武在漢中久,軍民相安。疏濬府城東河道達漢川,旁引溝渠以資灌溉,民食其利;又修復褎斜棧道,商旅便焉。沒後,士民籥請建祠,從之。

唐友耕,雲南大關人。咸豐中,滇匪起,陷賊,至四川敘州,自拔來歸。從戰有功,授千總,署通江營守備。賊擾鹽井,屢從戰擊走之,擢守備。十年,戰峨眉索橋,受傷,破賊雙福場,進平天全茅山賊壘,擢都司,賜號額勒莫克依巴圖魯。十一年,援潼川,破賊解圍,擢副將。駱秉章督師至蜀,檄友耕會諸軍援綿州,令自石橋鋪進攻,友耕觀望不前,被劾,褫職留營。既而會援眉州,友耕軍先至,戰比有功,圍解,復原官。戰青神,陣斬賊目張興,身被二傷,裹創力戰,賊大敗。

同治元年,破石達開黨賴裕新於工⼙州。三月,達開圍涪州,友耕馳援,解其圍,授四川重慶鎮總兵。會諸軍復長寧,賊引去。是年冬,達開屯敘州雙龍場,分黨屯橫江,友耕攻破江岸賊壘。二年春,賊由橫江竄新灘溪,與屏山隔一水,友耕慮賊乘間偷渡,乃濟江設伏,誘賊深入,敗之。六月,達開謀渡金沙江,官軍扼之不得進,改趨天全土司地,友耕擊沉賊筏;達開奔老鴉漩,復為土兵所遏,遂就擒。友耕擢雲南提督,留屯川南。四年,丁母憂,詔改署提督,友耕請終制,許之。七年服闋,署四川總督崇實奏緩陛見,令募勇防川北。八年,調赴雲南,招降回寇李本忠等,賜黃馬褂。光緒六年,署四川提督,八年,卒。

論曰:雷正綰、陶茂林、曹克忠皆多隆阿部下戰將。多隆阿歿後,甘肅軍事實倚三人,以餉匱兵變,遂難成功。克忠較有謀略,其軍獨全,終以病引退,後猶稱為宿將。胡中和、周達武等皆以楚軍平蜀寇。唐友耕以蜀軍頡頏其間,並躋專閫。達武晚任貴州軍事,與席寶田同定苗疆,建樹較閎達焉。


\end{pinyinscope}