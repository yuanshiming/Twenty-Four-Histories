\article{列傳一百四十二}

\begin{pinyinscope}
魁倫廣興初彭齡

魁倫,完顏氏,滿洲正黃旗人,副將軍查弼納孫也。襲世管佐領,兼輕車都尉,授四川漳臘營參將,累擢建昌鎮總兵。嘗入覲,高宗詢家世,魁倫陳戰功甚悉。乾隆五十三年,擢福州將軍。喜聲伎,制行不謹,總督伍拉納欲劾之。伍拉納故貪,逼勒屬吏財賄,復縱洋盜,盜艇集五虎門外不問。魁倫遂疊疏劾閩省吏治廢弛,伍拉納及巡撫浦霖溺職,按察使錢受椿等迎合助虐。上怒,褫伍拉納等職逮問,命長麟署總督,偕魁倫鞫訊,得伍拉納等貪婪及庫藏虧絀狀,俱伏法。伍拉納為和珅姻戚,當按治時,上切責長麟瞻徇,罷去,以事由魁倫舉發,特寬之,代署總督,嚴捕海盜,屢獲其魁。

嘉慶元年,實授總督。三年,巨盜林發枝投首,海患稍戢。以母憂歸。自治閩獄。以伉直聞於時,仁宗尤眷之。四年,起署吏部尚書。魁倫屢於上前自稱昔治四川啯匪功,謂賊不難辦,請赴軍前,時上督責諸將平賊甚急,經略勒保未稱帝意,命魁倫赴四川,逮勒保治罪,即代署總督,駐達州治軍餉。勒保獲譴由蜚語,既就逮,所部訴其冤,乞代奏,魁倫稍稍為置辯,終以玩誤軍務讞擬重闢,軍心因之渙散,不為用。額勒登保繼為經略,與德楞泰先後赴甘肅剿竄匪,魁倫專任四川軍事。

五年春,冉天元糾數路殘匪潛匿大竹,魁倫逡巡未發,賊脅眾數萬由定遠渡嘉陵江,圖擾川西,魁倫繞道鄰水,自順慶追剿,檄總兵七十五還守重慶。上以數年來賊氛皆在川東北,惟川西完善,地為軍餉所出,斥魁倫疏防,革職留任。賊尋渡江掠蓬溪,諸將獨總兵硃射鬥力戰而兵少,魁倫約為接應復不至,射鬥戰死。魁倫退屯潼川,降三品頂戴,詔責嚴守潼河,曰:「此爾生死關頭也!」復起勒保為四川提督,偕德楞泰進剿川西、川北。四月,賊伺川西備嚴,乘間竄渡潼河,焚太和,逼成都,上怒魁倫屢失機縱賊,褫職逮問,命勒保代署總督。侍郎周興岱往會鞫,尋逮克賜死,子扎拉芬戍伊犁。

魁倫居官廉,自為尚書時,詔寬減閩關賠繳銀六千兩,至是罄家產不足償,上益憐之,給還宅一區,俾其妻有所棲止;又因其孫幼稚,命扎拉芬到戍三年釋歸,宣諭廷臣,使知法戒焉。

廣興,字賡虞,滿洲鑲黃旗人,大學士高晉第十二子。入貲為主事,補官禮部。敏於任事,背誦案牘如瀉水,大學士王傑器其才。累遷給事中。嘉慶四年,首劾和珅罪狀,擢副都御史。命赴四川治軍需,綜覈精嚴,月節糜費數十萬金,為時所忌,以騷擾驛傳被劾,上優容之。復屢與總督魁倫互劾,召還,左遷通政副使。九年,擢兵部侍郎,兼副都統、總管內務府大臣,署刑部侍郎。同僚輕其於刑名非素習,廣興引證律例,屢正誤讞,眾乃服。十一年,奏劾御前大臣定親王綿恩揀選官缺專擅違例,廷臣察詢,不直所言,降三品京堂,罷兼職。尋補奉宸苑卿,擢刑部侍郎,復兼內務府大臣。上方倚任,廣興亦慷慨直言,召對每逾晷刻。上曰:「汝與初彭齡皆朕信任之人,何外廷怨恨乃爾?」廣興頫首謝。數奉使赴山東、河南按事,益作威福,中外側目。

內監鄂羅哩者,自乾隆中充近侍,年七十餘,嘗至朝廊與廣興坐語,以長者自居。廣興艴然曰:「汝輩閹人,當敬謹侍立,安得與大臣論世誼乎?」鄂羅哩恨次骨,思以中之。十三年冬,內庫給宮中紬段不如數,且窳敗,鄂羅哩言由廣興剋減,上即命傳諭,出而漫言之,廣興不知為上旨,坐而與辯。鄂羅哩入奏其坐聽諭旨,上怒,一日,面詰廣興,廣興言總管太監孫進忠與庫官勾通,欲交外省織造,藉遂需索規費之計。上以其不能指實庫官何人,挾詐面欺,下廷臣議罪,尋寬之。罷職家居,於是與廣興不協者,蜂起媒孽其短。上密諭山東、河南兩省巡撫察奏,遂交章劾其奉使時任意作威,苛求供頓,收納餽遺諸罪狀,下獄議絞。上親廷訊,尚欲緩其獄,廣興未省上意,抗辯無引罪語,而贓私有實據,上益怒,遂置之法,籍其家,子蘊秀戍吉林,並罪兩省官吏及山東言官各有差。

廣興伉爽無城府,疾惡嚴,喜詆人陰私。既得志,驕奢日甚,縱情聲色,不能約束奴僕,終及於禍。

初彭齡,字頤園,山東萊陽人。乾隆三十六年,巡幸山東,召試,賜舉人。四十五年,成進士,選庶吉士,授編修。五十四年,遷江南道御史。劾協辦大學士彭元瑞徇私為婿侄營事,元瑞被黜;又江西巡撫陳淮以貪著,劾罷之,風採振一時。累遷兵部侍郎。

嘉慶四年,出為雲南巡撫。時總督富綱請罷官鹽,改歸民運民銷,詔下彭齡議。疏上,略曰:「滇鹽向例官督灶煎,分井定額,按月完納省倉。行銷之法,按州縣戶口多寡定額,地方官備價運銷交課。其始灶戶所領官給薪本敷裕,交足額鹽之外,尚有餘鹽;官售額鹽,扣還腳價之外,尚有餘課。行之日久,不肖州縣勾通井官,私買額外餘鹽,行銷肥己。灶戶利於賣私,益滋偷漏。前巡撫劉秉恬遂令州縣額銷十萬斤者加銷一二萬,以資辦公。灶戶薪本不敷,無力加煎,手毚和灰土,州縣滯銷,因有派累之事。乾隆五十六年,鹽道蔣繼勛以官銀盡買安寧等井私煎之鹽,並發州縣銷售,欲以彌縫虧空。額鹽積壓愈多,於是州縣又有計口授鹽、短秤加課之弊。煙戶無論男女老幼,皆應交課,窮困已極。迤西一帶,遂至聚眾抗官,斃差焚屋。前年威遠惈夷滋擾,即有此等奸民。祿豐一案,亦由鹽務起釁,江蘭並匿情不奏。富綱到滇,實見有不得不改章以甦民困者。竊思滇鹽官運官銷,積弊難返,應如督臣所奏,改為就井收課,聽民自便。」於是損益原奏,令灶戶自煎自賣,商販領照,聽其所之,試行二三年,再定各井歲額,下部議行。又籌置堡田,免徭役加派,滇民感之。劾前撫江蘭匿抱母、恩耕二井水災不奏,蘭因黜罷。

六年,自陳親老,乞改京職,允之。以貴州巡撫伊桑阿代。途次劾伊桑阿驕奢乖戾,苛派屬員,剿石峴苗飾詞冒功。遣使勘實,置伊桑阿於法。回京,授刑部侍郎。七年,偕副都統富尼善往貴州按事,劾巡撫常明鉛廠之弊,褫職治罪,即代署巡撫。尋調署云南巡撫,劾布政使陳孝升、迤西道薩榮安以維西軍務冒帑,治如律。八年,偕侍郎額勒布清查陜西軍需,自巡撫秦承恩以下,黜罰有差。調工部侍郎,又調戶部。

九年,誤聽湖北巡撫高杞言,劾湖廣總督吳熊光受賄,不得實,後復以獨對時密諭私告杞,事覺,下廷臣議罪,以大闢上。仁宗知彭齡無他,不欲因言事加重譴,詔斥諸臣所擬過當,有意杜言事者之口;又念彭齡親老,免遠謫,罷職家居。逾年,起授右庶子,驟遷內閣學士。

十一年,偕侍郎英和往陜西讞獄,途經山西,命察議河東鹽務。尋授安徽巡撫。壽州武舉張大有因妒奸毒斃族侄張倫及雇工人,總督鐵保徇蘇州知府周鍔以自中蛇毒定讞,彭齡推鞫得實,詔嘉之,特予議敘,鐵保等降黜有差。父憂歸。

十四年,奪情授貴州巡撫,固辭不起。服闋,署山西巡撫,遂實授。劾前巡撫成齡需索供應,又劾布政使劉清、署按察使張曾獻及府州縣多人,尋調陜西。河東道劉大觀揭劾初彭齡任性乖張,命回山西聽勘,以怒斥前撫金應琦及瞻徇知府硃錫庚,部議革職,詔寬之,降補鴻臚寺卿。遷順天府尹。

十六年,偕尚書托津清查南河工帑,劾罷營四十八員,復偕尚書崇祿往福建讞獄。遷工部侍郎,署浙江巡撫。尋命往兩湖按訊湖北按察使周季堂及湖南學政徐松,季堂無貪跡,惟袒庇屬員,褫職,免治罪;松需索陋規,出題割裂聖經,褫職遣戍。

十七年,調戶部侍郎。時兩江總督百齡劾南河總督陳鳳翔誤啟智、禮兩壩,鳳翔已被譴,自訴辯,又訐百齡信任鹽巡道硃爾賡額督辦葦蕩失當,命彭齡、松筠往按。百齡於啟壩時實同畫諾,遂請薄懲百齡,而硃爾賡額被重譴,語詳百齡等傳。署南河總督,尋調倉場侍郎。

十九年,命往廣西按訊巡撫成林,以恣意聲色,用度侈靡,褫成林職,籍其家。擢兵部尚書,特命署江蘇巡撫,清查虧空,疏言:「虧空應立時懲辦,而各省督撫往往密奏,僅使分限完繳。始則屬官玩法,任意侵欺;繼則上司市恩,設法掩蓋。是以清查為續虧出路,密奏為緩辦良圖,請飭禁。」帝韙之。劾江寧布政使陳桂生、江蘇布政使常格催徵不力,並褫職。尋巡撫張師誠回任,仍命彭齡會同清查。彭齡與百齡、師誠意不合,各擬章程,上詔斥其不能和衷。既而疏劾百齡、師誠受關道鹽員饋銀,又劾陳桂生弊混,命大學士托津、尚書景安往按,至則百齡、師誠嗾屬員多方沮格,所劾並不得實。上以彭齡性褊急,嫉惡過嚴,斥其輕躁,降內閣學士,召回京。茅豫者,以部員隨赴廣西,因留江蘇佐理,改知府。至是彭齡疏陳豫兩耳重聽,代為乞假。詔斥越職專擅,再降,以翰林院侍讀、侍講候補。百齡復劾彭齡沉湎於酒,事一委茅豫,文致陳桂生之罪,私拆批摺,挾怨誣參;且豫實非耳聾,亦徇欺。上怒,褫彭齡職,停其母九旬恩賚,令閉門思過。

二十一年,起為工部主事。丁母憂,未歸,請改注籍順天,服闋,以員外郎用。道光元年,授禮部侍郎,尋擢兵部尚書。三年,萬壽節,與十五老臣宴,繪圖於萬壽山玉瀾堂,禦制詩稱其耿介,優賚珍物。四年,以年老休致,食半俸。五年,卒,詔優恤。

論曰:甚矣直臣之不易為也!赤心為國,犯顏批鱗,而人主諒之。茍有排異己市盛名之心,借徑梯榮,眾矢集焉;況身罹負乘,或加之貪婪乎?魁倫、廣興之所以不得其死也。初彭齡雖亦褊躁,然實政清操,蹶而復起,克保令名,宜哉!


\end{pinyinscope}