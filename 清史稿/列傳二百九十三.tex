\article{列傳二百九十三}

\begin{pinyinscope}
疇人一

薛鳳祚杜知耕龔士燕王錫闡潘檉樟方中通揭暄梅文鼎子以燕孫成曾孫鈁弟文鼐文鼏明安圖子新陳際新張肱劉湘煃王元啟硃鴻博啟許如蘭

推步之學,由疏漸密。泰西新法,晚明始入中國,至清而中、西薈萃,遂集大成。聖祖聰明天亶,研究歷算,妙契精微。一時承學之士,蒸蒸鄉化,肩背相望。二百年來,推步之學,日臻邃密,匪特闢古學之榛蕪,抑且補西人之罅漏。嘉慶初,阮元撰疇人傳,後學一再續之,唐、宋以來,於斯為盛。今甄其卓然名家者著於篇,其政事、文學登於列傳及儒林、文苑者;西人官欽天監,廁於卿貳,各自有傳者:不具列焉。

薛鳳祚,字儀甫,淄川人。少習算,從魏文魁游,主持舊法。順治中,與法人穆尼閣談算,始改從西學,盡傳其術,因著算學會通正集十二卷,考驗二十八卷,致用十六卷。其曰對數比例者,乃西算以假數求真數之便法也;曰中法四線,以西法六十分為度,不便以十進位,改從古法,以百分為度,所列止正弦、餘弦、正切、餘切,故曰四線。其推步諸書:曰太陽太陰諸行法原,曰木火土三星經行法原,曰交食法原,曰歷年甲子,曰求歲實,曰五星高行,曰交食表,曰經星中星,曰西域回回術,曰西域表,曰今西法選要,曰今法表,皆會中、西以立法。以順治十二年乙未天正冬至為元,諸應皆從以起算。以三百六十五日二十三刻三分五十七秒五微為歲實,黃、赤道交度有加減,恆星歲行五十二秒,與天步真原法同。梅文鼎謂其書詳於法,而無快論以發其趣,蓋其時新法初行,中、西文字展轉相通,故詞旨未能盡暢。然貫通其中、西,要不愧為一代疇人之功首云。

鳳祚定歲實秒數為五十七,與奈端合,與穆尼閣以為四十五秒者不同,則其學非墨守穆氏可知。或譏其謹守穆尼閣成法,依數推衍,非篤論也。

杜知耕,字端甫,號伯瞿,柘城舉人。精研幾何,以利瑪竇、徐光啟所譯幾何原本復加刪削,作幾何論約七卷,後附十條,則知耕所作也。言其法似為本書所無,其理實涵各題之內,非能於本書之外別生新義也。稱後附者,以別於丁氏、利氏之增題也。又雜取諸家算學,參以西人之說,依古九章為目,作數學鑰六卷。言數非圖不明,圖非指不明,圖中用甲乙等字作志者,代指也,故其書於圖解尤詳。梅文鼎稱其圖註九章,頗中肯綮云。

龔士燕,字武任,武進人。少穎異能文,講求性理,旁通算術,發明蔡氏律呂新書,推演黃鍾圜徑、開方密率諸法,而於元太史郭守敬授時術尤得其秘。如求冬至時刻,上推百年加一算,以為歲周三百六十五日二十四刻二十五分之內,滿百年消長一分。核之春秋日食三十七事,多與符合。又如推晦、朔、弦、望,以太陽之盈與太陰之遲,以太陰之疾與太陽之縮皆相並,為同名相從;以太陽之盈與太陰之疾,以太陰之遲與太陽之縮皆相減,為異名相消:乃得盈縮遲疾化為加減時刻之差。以此加減朔望之大、小餘分,得定朔弦望諸時刻。至盈、縮、遲、疾,郭守敬創平、立、定三差,理隱數繁,能審其機括,繪圖以明之。

又如赤道變黃道之法,謂在二至後者,以度率一零八六五除赤道積度變為黃道宿度;在二分後者,以度率一零八六五乘赤道積度變為黃道宿度。凡此授時之術,引伸益明。其餘月離五星等法,與回回、西洋諸算,遇有疑難,無不洞悉。至日、月體徑有大小,交食限數有淺深,具見其奧。且悟唐順之弧容直闊之法,以推求太陰出入黃道,在內在外,不離乎六度。自是一應七政、氣朔、交食諸端,按法而推,百不失一。

康熙六年,詔募天下知算之士,於是入都。其時欽天監用大統算七政多不合天,奉旨在觀象臺每日測驗,而金星比算差至十度。因修改古法,乃據七年所測表景推測太盈縮,又據日測五星行度,考其遲疾。彼此推求加減,氣、閏、轉、交諸應,測驗皆與天合。蓋其法亦本郭守敬,太陽為氣應,推冬至日躔用之;太陰周天為轉應,朔望用之;日月地球之運,同在一直線,視點上為交應,推日月食用之;合氣盈、朔虛之奇零為閏應,推閏月用之;此外又有金、木、水、火、土同聚一宿為合應,推五星用之。

修改諸應,取順治元年甲申為元,以應世祖章皇帝撫有中夏之祥,欽天監名為「改應法」。既改氣、閏、轉、交諸應,復改遲、疾限及求差諸法,又改冬至黃道日出分依步中星內法。又盈縮遲疾無積度,日食無時差,皆與天合。臺官交章保薦。八年,歷書告成,奏對武英殿,授歷科博士。時有薦西人南懷仁等於朝,及其實測諸術,驗且捷,遂定用西法,而古歷卒不行。

十年,以疾歸,著有象緯考一卷、歷言大略一卷。其天體論一卷及闇虛、中星、交食、定朔、五星諸論俱佚。

王錫闡,字曉菴,吳江人。兼通中、西之學,自立新法,用以測日、月食不爽秒忽。每遇天晴霽,輒登屋臥鴟吻察星象,竟夕不寐。著曉庵新法六卷,序曰;「炎帝八節,歷之始也,而其書不傳。黃帝、虞、夏、殷、周、魯七歷,先儒謂系偽作。今七歷俱存,大指與漢歷相似,而章蔀氣朔,未睹其真,為漢人所託無疑。太初、三統,法雖疏遠,而創始之功,不可泯也。劉洪、姜岌,次第闡明,何、祖專力表、圭,益稱精切。自此南、北歷象,率能好學深思,多所推論,皆非淺近所及。唐歷大衍稍密,然開元甲子當食不食,一行乃為諛詞以自解,何如因差以求合乎?」

又曰:「明初元統造大統歷,因郭守敬遺法,增損不及百一,豈以守敬之術果能度越前人乎?守敬治歷,首重測日,餘嘗取其表景,反覆布算,前後牴牾。餘所創改,多非密率。在當日已有失食失推之咎,況乎遺籍散亡,法意無徵。兼之年遠數盈,違天漸遠,安可因循不變耶?元氏藝不逮郭,在廷諸臣,又不逮元,卒使昭代大典,踵陋襲偽。雖有李德芳苦爭之,然德芳不能推理,而株守陳言,無以相勝,誠可嘆也!」

又曰:「萬歷季年,西人利氏來華,頗工歷算。崇禎初,命禮臣徐光啟譯其書,有歷指為法原,歷表為法數,書百餘卷,數年而成,遂盛行於世。言歷者莫不奉為俎豆。吾謂西歷善矣,然以為測候精詳可也,以為深知法意未可也。循其理而求通,可也;安其誤而不辨,不可也。姑舉其概:二分者,春、秋平氣之中;二至者,日道南、北之中也。大統以平氣授人時,以盈縮定日躔。西人既用定氣,則分、正為一,因譏中歷節氣差至二日。夫中歷歲差數強,盈縮過多,惡得無差?然二日之異,乃分、正殊科,非不知日行之朓朒而致誤也。歷指直以怫己而譏之,不知法意一也。諸家造歷,必有積年日法,多寡任意,牽合由人。守敬去積年而起自辛巳,屏日法而斷以萬分,識誠卓也。西歷命日之時以二十四,命時之分以六十,通計一日為分一千四百四十,是復用日法矣。至於刻法,彼所無也。近始每時四分之,為一日之刻九十六。彼先求度而後日,尚未覺其繁,施之中歷則窒矣。乃執西法反謂中歷百刻不適於用,何也?且日食時差法之九十有六,與日刻之九十六何與乎?而援以為據,不知法意二也。天體渾淪,初無度分可指,昔人因一日日躔命為一度,日有疾徐,斷以平行,數本順天,不可損益。西人去周天五度有奇,斂為三百六十,不過取便割圜,豈真天道固然?而黨同伐異,必曰日度為非,詎知三百六十尚非天真有此度數乎?不知法意三也。上古置閏,心互於歲終,蓋歷術疏闊,計歲以置閏也。中古法日趨密,始計月以置閏,而閏於積終,故舉中氣以定月,而月無中氣者即為閏。大統專用平氣,置閏必得其月,新法改用定氣,致一月有兩中氣之時,一歲有兩可閏之月,若辛丑西歷者,不亦盭乎!夫月無平中氣者,乃為積餘之終,無定中氣者,非其月也。不能虛衷深考,而以鹵莽之習,侈支離之學,是以歸餘之後,氣尚在晦;季冬中氣,已入仲冬;首春中氣,將歸臘杪。不得已而退朔一日以塞人望,亦見其技之窮矣,不知法意四也。天正日躔,本起子半,後因歲差,自醜及寅。若夫合神之說,乃星命家猥言,明理者所不道。西人自命歷宗,何至反為所惑,謂天正日躔定起丑初乎?況十二次命名,悉依星象,如隨節氣遞遷,雖子午不妨異地,豈玄枵、鳥咮亦無定位耶?不知法意五也。歲實消長,昉於統天,郭氏用之,而未知所以當用;元氏去之,而未知所以當去。西人知以日行最高求之。而未知以二道遠近求之,得其一而遺其一。當辨者一也。歲差不齊,必緣天運緩疾,今欲歸之偶差,豈前此諸家皆妄作乎?黃、白異距,生交行之進退;黃、赤異距,生歲差之屈伸;其理一也。歷指已明於月,何蔽於日?當辨者二也。日躔盈縮最高,斡運古今不同,揆之臆見,必有定數。不惟日月星應同,但行遲差微,非畢生歲月所可測度耳。西人每詡數千年傳人不乏,何以亦無定論?當辨者三也。日月去人時分遠近,兒徑因分大小,則遠近大小,宜為相似之比例。西法日則遠近差多,而兒徑差少;月則遠近差少,而兒徑差多。因數求理,難會其通。當辨者四也。日食變差,機在交分,日軌交分,與月高交分不同;月高交於本道,與交於黃道者又不同。歷指不詳其理,歷表不著其數,豈黃道一術足窮日食之變乎?當辨者五也。中限左右,日月兒差,時或一東一西。交、廣以南,日月兒差,時或一南一北。此為兒差異向與兒差同向者加減迥別,歷指豈以非所常遇,故置不講耶?萬一遇之,則學者何以立算?當辨者六也。日光射物,必有虛景,虛景者,光徑與實徑之所生也。闇虛恆縮,理不出此。西人不知日有光徑,僅以實徑求闇虛。及至推步不符,復酌損徑分以希偶合。當辨者七也。月食定望,惟食甚為然,虧復四限,距望有差。日食稍離中限,即食甚已非定朔。至於虧復,相去尤遠。西歷乃言交食必在朔、望,不用朓朒次差。當辨者八也。」

又曰:「語云:『步歷甚難,辨歷甚易。』蓋言象緯森羅,得失無所遁也。據彼所說,亦未嘗自信無差。五星經度,或失二十餘分,躔離表驗,或失數分,交食值此,所失當以刻計;凌犯值此,所失當以日計矣。故立法不久,違錯頗多,餘於歷說已辨一二。乃癸卯七月望食當既不既,與夫失食失推者何異乎?且譯書之初,本言取西歷之材質,歸大統之型範,不謂盡隳成憲,而專用西法,如今日者也。餘故兼採中、西,去其疵類,參以己意,著歷法六篇,會通若干事,改正若干事,表明若干事,增輯若干事,立法若干事。舊法雖舛,而未遽廢者,兩存之;理雖可知,而上下千年不得其數者,缺之;雖得其數,而遠引古測,未經目信者,別見補遺,而正文仍襲其故。為日一百幾十有幾,為文萬有千言,非敢妄云窺其堂奧,庶幾初學之津梁也。」

其法:度法百分,日法百刻,周天三百六十五度二十五分六十五秒五十九微三十二纖,內外準分三十九分九十一秒四十九微,次準九十一分六十八秒八十六微,黃道歲差一分四十三秒七十三微二十六纖。列宿經緯:角一十度七十三分七十九秒,南二度一分二十三秒,亢一十度八十二分二十四秒,北三度一分一秒,氐一十八度一十六分一十四秒,北四十三分九十六秒,房四度八十三分六十三秒,南五度四十六分一十九秒,心七度六十六分二秒,南三度九十七分三十八秒,尾一十五度八十二分七十八秒,南一十五度二十一分九十秒,箕九度四十六分九十六秒,南六度五十九分四十九秒,南斗二十四度一十九分八十二秒,南三度八十八分九十三秒,牽牛七度七十九分五十五秒,北四度七十五分一十七秒,婺女一十一度八十二分二秒,北八度二十分五十九秒,虛一十度一十二分九十一秒,北八度八十二分七十秒,危二十度四十一分四秒,北一十度八十五分六十二秒,營室一十五度九十二分二十秒,北一十度七十一分七十一秒。

先是曉菴新法未成,作歷說六篇,歷策一篇,其說精核,與新法互有詳略。又隱括中、西步術,作大統西歷啟蒙。丁未歲,因推步大統法作丁未歷稿。辛酉八月朔日食,以中、西法及己法豫定時刻分秒,至期,與徐發等以五家法同測,己法獨合,作推步交朔測小記。又以治歷首重割圜,作圜解。測天當據儀晷,造三晷,兼測日、月、星,因作三辰晷志。俱能究術數之微奧,補西人所不逮。與同時青州薛鳳祚齊名,稱「南王北薛」云。歷策有云:「每遇交會,必以所步、所測課較疏密,疾病寒暑無間,變周、改應、增損、經緯、遲疾諸率,於茲三十年所。」亦可以想見作者實測之詣力矣。

潘檉樟,字力田。與王錫闡同邑友善。錫闡嘗館其家,講論算法,常窮日夜。檉樟著辛丑歷辨曰:「昔堯命羲和,曰以閏月定四時成歲,蓋歷法首重置閏。而春秋傳曰:『先王之正時也,履端於始,舉正於中,歸餘於終。』所謂始者,取氣朔分齊為歷元也;所謂中者,月以中氣為定,無中氣者則為閏也;所謂終者,積氣盈、朔虛之數而閏生焉也。自漢以降,歷術雖屢變,未有能易此者。唯西域諸歷則不然,其法有閏年、有閏日,而無閏月。蓋中歷主日,而西歷主度,不可強同也。今之為西歷者,乃以日躔求定氣、求閏月,不惟盡廢中國之成憲,而亦自悖西域之本法矣。故十餘年來,宮度既紊,氣序亦訛。如戊子之閏三月也,而置在四月;庚寅之閏十一月也,而置在明年之二月;癸巳之閏七月也,而置在六月;己亥之閏正月也,而置在三月。其為舛誤,何可勝言!然非深於歷者,未易指摘。至於辛丑之閏月,則其失顯然無以自解矣。何也?閏法論平氣而不當論定氣,若以平氣,則是年小雪在十月晦,冬至在十一月朔,而閏在兩月之間。所謂閏前之月中氣在晦,閏後之月中氣在朔者也。今以定氣,則秋分居九月朔,故預於七月朔置閏,然後秋分仍在八月,而霜降、小雪各歸其月。無如大寒定氣乃在十一月朔,而十二月又無中氣,既不可再置一閏,則是同一無中氣之月,而或閏或否。彼所云太陽不及交宮即置為閏者,何獨於此而自背其法乎?蓋孟秋非歸餘之終,故天正不能履端於始,地正不能舉正於中也。如此,則四時不定,歲功不成,而閏法又安用之?且壬寅正月,定朔舊法在丙子丑初,即彼法亦在丙子子正,則辛丑之季冬當為大盡,而明年正月中氣復移於今歲之秒。彼亦自覺其未安,故進歲朔於乙亥,而季冬為小盡之月,皆所謂欲蓋彌彰者耳。即辛丑歲朔,以彼法推,當會於亥正,而今在戌正,差至六刻,其他牴牾,更難枚舉。噫!作法如是,而猶自以為盡善,可乎?蓋其說以日行盈縮為節氣短長,每遇日行最盈,則一月可置一氣,是古有氣盈、朔虛,而今更有氣虛、朔盈矣。然或晦朔兩節而中氣介其間。如丙戌仲冬,去閏稍遠,猶可不論;獨辛丑仲冬,冬至、大寒俱在晦朔,去閏最近,進退無據。茍且遷就,有不勝其弊者。夫閏法之主平氣,行之已數千年矣,今一變其術,未久而輒窮,至於無可如何,則又安取紛更為也!」檉樟後坐法死。弟耒,亦學歷算,見文苑傳。

方中通,字位伯,桐城人。集諸家之說,著數度衍二十四卷,附錄一卷。言:「九章皆出於句股,環矩以為圓,合矩以為方,方數為典。以方出圓,句股之所生也;少廣,方圓所出也。方田、商功,皆少廣所出。一方一圓,其間不齊,始出差分,而均輸對差分之數,盈朒借差求均。又差分、均輸所出,而以方程濟其窮。度量衡原出黃鍾,粟布出焉,黃鍾出於方圓者也。」又言:「古法用竹徑一寸長六分二百七十一而成六觚為一握,後世有珠算而古法亡矣。泰西之筆算、籌算,皆出九九。尺算即比例規,出三角。乘莫善於籌,除莫善於筆,加減莫善於珠,比例莫善於尺。」其珠算歸法,三一三十一,四一二十二之類,「十」字俱作「餘」字。其尺算以三尺交加,取數祗用平分一線。時廣昌揭暄亦明算術,與中通論難日輪大小,得光肥影瘦之故,及古今歲差之不同,須測算消長以齊之。一晝夜人一萬三千五百息,每息宗動天行十萬里有奇。別錄為一書,曰揭方問答。

揭暄,字子宣,廣昌人。著璇璣遺述七卷,一名寫天新語。論日月東行如槽之滾丸,而月質不變。又謂七政之小輪。皆出自然,如盤水之運旋而周遭,以行疾而成旋渦,遂成留逆。於五星西行,日月盈縮,皆設譬多方,言之近理。康熙己巳,以草稿寄梅文鼎,抄其精語為一卷,稱其「深明西術,而又別有悟入,其言多古今所未發」。卒年逾八十。

梅文鼎,字定九,號勿庵,宣城人。兒時侍父士昌及塾師羅王賓仰觀星象,輒了然於次舍運轉大意。年二十七,師事竹冠道士倪觀湖,受麻孟旋所藏臺官交食法,與弟文鼐、文鼏共習之。稍稍發明其立法之故,補其遺缺,著歷學駢枝二卷,後增為四卷,倪為首肯。

值書之難讀者,必欲求得其說,往往廢寢忘食。殘編散帖,手自抄集,一字異同,不敢忽過。疇人子弟及西域官生,皆折節造訪,有問者,亦詳告之無隱,期與斯世共明之。所著歷算之書凡八十餘種。

讀元史授時歷經,嘆其法之善,作元史歷經補註二卷。又以授時集古法大成,因參校古術七十餘家,著古今歷法通考七十餘卷。授時以六術考古今冬至,取魯獻公冬至證統天術之疏,然依其本法步算,與授時所得正同,作春秋以來冬至考一卷。元史西征庚午元術,西征者,謂太祖庚辰;庚午元者,上元起算之端也。歷志訛太祖庚辰為太宗,不知太宗無庚辰也。又訛上元為庚子,則於積年不合。考而正之,作庚午元算考一卷。授時非諸古術所能方,郭守敬所著歷草,乃歷經立法之根,拈其義之精微者,為郭太史歷草補註二卷。立成傳寫魯魚,不得其說,不敢妄用,作大統立成註二卷。授時術於日躔盈縮、月離遲疾,並以垛積招差立算,而九章諸書無此術,從未有能言其故者,作平立定三差詳說一卷,此發明古法者也。唐九執術為西法之權輿,其後有婆羅門十一曜經及都聿利斯經,皆九執之屬。在元則有札馬魯丁西域萬年術,在明則馬沙亦黑、馬哈麻之回回術、西域天文書,天順時具琳所刻天文實用,即本此書,作回回歷補註三卷,西域天文書補註二卷,三十雜星考一卷。表景生於日軌之高下,日軌又因裡差變移,作四省表景立成一卷。周髀所言裏差之法,即西人之說所自出,作周髀算經補言主一卷。渾蓋之器,最便行測,作渾蓋通測憲圖說訂補一卷。西國以太陽行黃道三十度為一月,作西國日月考一卷。西術中有細草,猶授時之有通軌也,以歷指大意隱括而注之,作七政細草補注三卷。新法有交食蒙求、七政蒙引二書,並逸,作交食蒙求訂補二卷、附說二卷。監正楊光先不得已日食圖,以金環食與食甚分為二圖,而各有時刻,其誤非小,作交食作圖法訂誤一卷。新法以黃道求赤道交食,細草用儀象志表,不如弧三角之親切,作求赤道宿度法一卷。謂中、西兩家之法,求交食起復方位,皆以東西南北為言。然東西南北惟日月行至午規而又近天頂,則四方各正其位。非然,則黃道有斜正之殊,而自虧至復,經歷時刻,展轉遷移,弧度之勢,頃刻易向。且北極有高下,而隨處所見必皆不同,勢難施諸測驗。今別立新法,不用東西南北之號,惟人所見日月員體,分為八向,以正對天頂處為上,對地平處為下,上下聯為直線,作十字橫線,命之曰左、曰右,此四正向也;曰上左、上右,曰下左、下右,則四隅向也。乃以定其受蝕之所在,則舉目可見,作交食管見一卷。太陽之有日差,猶月離交食之有加減時,因表說含糊有誤,作日差原理一卷。火星最為難算,至地谷而始密,解其立法之根,作火緯圖法一卷。訂火緯表記,因及七政,作七政前均簡法一卷。天問略取緯不真,而列表從之誤,作黃赤距緯圖辨一卷。新法帝星、句陳經緯刊本互異,作帝星句陳經緯考異一卷。測帝星、句陳二星為定夜時之簡法,作星軌真度一卷。以上皆以發明新法算書,或正其誤,或補其缺也。

康熙己未,明史開局,歷志為錢塘吳任臣分修,經嘉禾徐善、北平劉獻廷、田比陵楊文言,各有增定,最後以屬黃宗羲,又以屬文鼎,摘其訛誤五十餘處,以算草、通軌補之,作明史歷志擬稿一卷。雖為大統而作,實以闡明授時之奧,補元史之缺略也。其總目凡三:曰法原,曰立成,曰推步。而法原之目七:曰句股測望,曰弧天割圜,曰黃赤道差,曰黃赤道內外度,曰白道交周,曰日月五星平立定三差,曰里差刻漏。立成之目凡四:曰太陽盈縮,曰太陰遲疾,曰晝夜刻,曰五星盈縮。推步之目凡六:曰氣朔,曰日躔,曰月離,曰中星,曰交食,曰五星。

又作歷志贅言一卷,大意言:「明用大統,實即授時,宜詳元史缺載之事,以補其未備。又回回歷承用三百年,法宜備書。又鄭世子歷學已經進呈,宜詳述。他如袁黃之歷法新書,唐順之、周學述之會通回歷,以庚午元歷之例例之,皆得附錄。其西洋歷方今現行,然崇禎朝徐、李諸公測驗改憲之功,不可沒也,亦宜備載緣起。」

己巳,至京師,謁李光地於邸第,謂曰;「歷法至本朝大備矣,而經生家猶若望洋者,無快論以發其趣也。宜略仿元趙友欽革象新書體例,作簡要之書,俾人人得其門戶,則從事者多,此學庶將大顯。」因作歷學疑問三卷。

光地扈駕南巡,駐蹕德州,有旨取所刻書籍回奏,光地匆遽未及攜帶,遂以所辢刻歷學疑問謹呈。奉旨:「朕留心歷算多年,此事朕能決其是非,將書留覽再發。」二日後,召見光地,上云:「昨所呈書甚細心,且議論亦公平,此人用力深矣,朕帶回宮中仔細看閱。」光地因求皇上親加御筆,批駁改定,上肯之。

明年癸未春,駕復南巡,於行在發回原書,面諭光地:「朕已細細看過。」中間圈點塗抹及簽貼批語,皆上手筆也。光地復請此書疵繆所在,上云:「無疵繆,但算法未備。」蓋其書本未完成,故聖諭及之。

未幾,聖祖西巡,問隱淪之士,光地以關中李顒、河南張沐及文鼎三人對。上亦夙知顒及文鼎,乙酉二月,南巡狩,光地以撫臣扈從,上問:「宣城處士梅文鼎焉在?」光地以「尚在臣署」對。上曰:「朕歸時,汝與偕來,朕將面見。」四月十九日,光地與文鼎伏迎河干,越晨,俱召對御舟中,從容垂問,至於移時,如是者三日。上謂光地曰:「歷象算法,朕最留心,此學今鮮知者,如文鼎,真僅見也。其人亦雅士,惜乎老矣!」連日賜御書扇幅,頒賜珍饌。臨辭,特賜「績學參微」四大字。越明年,又命其孫成內廷學習。

五十三年,成奉上諭:「汝祖留心律歷多年,可將律呂正義寄一部去,令看,或有錯處,指出甚好。夫古帝王有『都俞籲咈』四字,後來遂止有『都俞』,即朋友之間,亦不喜人規勸,此皆是私意。汝等須竭力克去,則學問長進。可並將此意寫與汝祖知之。」恩寵為古所未有。

文鼎圖注各直省及蒙古各地南北東西之差,為書一卷,名分天度里。地既渾員,則所云二百五十里一度,緯度則然,若經度離赤道遠,則里數漸狹。故惟路正東西行,自有一定算法;路或斜行,則其法不可用為立法。若兩地各有北極高度,又有相距之經度,而無相距里數,是有兩邊一角,而求餘一邊,即可以知斜距之里。若先有斜距之里數而求經度,是為三邊求角,亦可以知相距之經度。其法並用斜弧三角形立算,可與月食求經度之法相參,而且簡易的確。

文鼎於測算之圖與器,一見即得要領,古六合、三辰、四游之儀,以意約為小制,皆合。又自制為月道儀,揆日測高諸器,皆自出新意。嘗登觀象臺,流覽新制六儀,及元郭守敬簡儀、明初渾球,指數其中利病,皆如素習。其書有測器考二卷,又自鳴鐘說一卷,壺漏考一卷,日晷備考一卷,赤道提晷一卷,勿菴揆日器一卷,加時日軌高度表一卷,揆日測說一卷,璇璣尺解一卷,測量定時簡法一卷,勿庵測望儀式一卷,勿庵仰觀儀式一卷,月道儀式一卷。

其說曰:「月道出入於黃道,猶黃道之出入於赤道也。自古及今,未有為之儀器者。今依渾蓋北密南疏之度,以黃極為樞,而月道半在其內,半出其外,則月緯大小之理,及正交、中交、交前、交後之法,可以眾著。儀以銅為之,略如渾蓋,其上盤為月道,亦如渾蓋天盤之黃道圈;其下盤黃道經緯,分宮分度,並以黃極為心,而侭邊以黃緯九十五度少半為限。出黃道南五度少半,月道所到也。」

禮部郎中李煥斗嘗從文鼎問歷法,作答李祠部問歷一卷。滄州老儒劉介錫同客天津,問歷法,作答劉文學問天象一卷。又言生平於難讀之書,每手疏而攜諸篋,以待明者問之,於歷學尤多,作思問編一卷。緯度以測日高,因知北極為用甚博,古用二至二分,今則逐日可測,承友人之問,作七十二候太陽緯度一卷。潘天成從文鼎學歷,而苦於布算,作寫歷步歷法一卷授之。又授時步交食式一卷,文鼎季弟文鼏之稿也。步五星式六卷,文鼎與其仲弟文鼐共成之者也。

文鼎每得一書,皆為正其訛闕,指其得失,又古歷列星距度考一卷,從殘壞之本,尋其普天星宿,入宿去極度分,中缺二星,又從閩中林侗寫本補完之,而斷以為授時之法。萬歷中利瑪竇入中國,始倡幾何之學,以點線面體為測量之資,制器作圖,頗為精密。學者張皇過甚,未暇深考,輒薄古法為不足觀;而株守舊法者,又斥西人為異學:兩家之說,遂成隔礙。文鼎集其書而為之說,用籌、用尺、用筆,稍稍變從我法。若三角、比例等,原非中法可賅,特為表出。古法方程,亦非西法所有,則專著論,以明古人之精意不可湮沒。又為九數存古,以著其概。總為中西算學通例一卷。

餘分九種:一,勿庵籌算七卷。二,筆算五卷。皆易橫為直,以便中文。三,度算一卷,原無算例,其弟文鼏補之,而參以嘉禾陳藎謨尺算用法。又有矩算,用一尺一方板,則文鼎所創。四,比例數解四卷。釋穆尼閣所譯之對數。五,三角法舉要五卷。其目有五:曰測量名義,曰算例,曰內容外切,曰或問,曰測量。六,方程論六卷,安溪李鼎徵為刻於泉州。七,幾何摘要三卷,就原本刪繁補遺。八,句股測量二卷,就周髀、海島諸術,錄要以存古意。九,九九數存古十卷,九數即九章隸首之法,僅存者九章之目耳。後有作者,莫能出其範圍。

外有書一十七種為續編:一,少廣拾遺一卷。古有一乘方至九乘方相生之圖,而莫詳所用。後或增之至十乘,惟四乘方與十乘方不可借用他法,因為推演至十二乘方,有條不紊。二,方田通法一卷,算家有捷田二十三法,廣之為百二十有四。三,幾何補編四卷。幾何原本六卷,止於測面,七卷以後,未經譯出,取測量全義量體諸率,實考其作法根源,以補原書之未備。而原書二十等面體之說,向固疑其有誤者,今乃得其實數。又原本理分中末線,但有求作之法,而莫知所用。今依法求得十二等面及二十等面之體積,因得其各體中棱線及輳心對角諸線之比例。又兩體互相容及兩體與立方、立員諸體相容各比例,並以理分中末線為法,乃知此線不為徒設。四,西鏡錄訂註一卷。五,權度通幾一卷。重學為西術一種,載於比例規解者多譌誤,今以南勛卿儀象志互相訂補,其數始真。六,奇器補註二卷。關中王公徵奇器圖說所述引重轉木諸制,並有裨於民生日用,而又本於西人重學,以明其意。嘗以書史所傳,如漢杜詩作水以便民,及王氏農書諸水器之類,睹記所及,如劉繼莊詩集載筒車灌田法,稍為輯錄,以補其所遺,而圖與說不相應者正之,以西字為識者易之。七,正弦簡法補一卷。大測諸書,言作八線表之法詳矣,薛鳳祚書有用矢線求度法,為之作圖,以明其意。因得兩法,在六宗、三要之外,而為用加捷。兩法者,一曰正弦方冪倍而退位得倍弧之矢,一曰正矢進位折半得半弧正弦上方冪。八,弧三角舉要五卷。歷書皆三角法也,內分二支:一曰平三角,一曰弧三角。凡歷法所測,皆弧度也,弧線與直線不能為比例,則剖析渾員之體,而各於弧線中得其相當直線。即於無句股中尋出句股,此法之最奇而確者。弧三角之用法雖多,而其最著明者,為黃赤交變一圖。反覆推論,了如列眉,熟此一端,則其餘不難推及矣。測量全義第七、第八、第九卷專明此理,而舉例不全,且多錯謬。其散見諸歷指者,僅存用數,無從得其端倪。天學會通圈線三角法,作圖草率,往往不與法相應。一以正弧三角為綱,仍用渾儀解之。正弧三角之理,盡歸句股。參伍其變,斜弧三角之理,亦歸句股矣。其目:曰弧三角體式,曰正弧句股,曰求餘角法,曰弧角比例,曰垂線,曰次形,曰垂弧捷法,曰八線相當。九,環中黍尺五卷。舉要中弧度之法已詳,然更有簡妙之用宜知。測量全義原有斜弧兩矢較之例,所立圖姑為斜望之形,而無實度可言。今一以平儀正形為主,凡可以算得者,即可以器量。渾儀真象,呈諸片楮,而經緯歷然,無絲毫隱伏假借。至於加減代乘除之用,歷書舉其名不詳其說,疑之數十年,而後得其條貫,即初數次數甲數乙數諸法。其目:曰總論,曰先數後數,曰平儀論,曰三極通幾,曰初數次數,曰加減法,曰甲數乙數,曰加減捷法,曰加減又法,曰加減通法。十,巉堵測量二卷。古法斜剖立方,成兩巉堵形,巉堵又剖為二,成立三角,立三角為量體所必需,然此義皆未發。今以渾儀黃赤道之割切二線成立三角形,立三角本實形,今諸線相遇成虛形,與實形等,而四面皆句股,西法通於古法矣。又於餘弧取赤道及大距弧之割切線,成句股方錐形,亦四面皆句股,即弧度可相求,亦不言角,古法通於西法矣。二者並可以堅楮為儀象之,則八線相為比例之理,了如掌紋。而郭守敬員容方直矢接句股之法,不煩言說而解。其目:曰總論,曰立三角摘要,曰渾員內容立三角,曰句股錐,曰句股方錐,曰方巉堵容員巉堵,曰員容方直儀簡法,曰郭太史本法,曰角即弧解。十一,用句股解幾何原本之根一卷。幾何不言句股,而其理莫能外。故其最難通者,以句股釋之則明。惟理分中末線似與句股異源,今為游心於立法之初,仍不外乎句股,益信古句股義包舉無遺。徐光啟譯大測表,名之曰割圜句股八線表,其知之矣。十二,幾何增解數則。其目有四:曰以方斜較求斜方,曰切線角與員內角交互相應,曰量無法四邊形捷法,曰取平行線簡法。並就幾何各題而增,不入補編,附前條共卷。十三,仰觀覆矩二卷。一查地平經度為日出入方位,一查赤道經度為日出入時刻,並依裏差,用弧三角立算,與歷書法微別。十四,方員冪積二卷。歷書周徑率至二十位,然其入算,仍用古率十一與十四之比例,豈非以乘除之際難用多位歟?今以表列之,取數殊易,乃為之約法,則徑與周之比例即方、員二冪之比例,亦即為立方、立員之比例,殊為簡易直捷。十五,麗澤珠璣一卷。友朋之益,取其有關算學者。十六,算器考一卷。十七,數學星槎一卷。

文鼎歷學疑問,曾呈御覽,後又引申其說,作歷學疑問補二卷,皆平正通達,可為步算家準則。

文鼎為學甚勤,劉輝祖同舍館,告桐城方苞曰:「吾每寐覺,漏鼓四五下,梅君猶構燈夜誦,乃今知吾之玩日而愒時也。」居京師時,裕親王以禮延致硃邸,稱梅先生而不名。李文貞公命子鍾倫從學,介弟鼎徵及群從皆執弟子之禮。宿遷徐用錫,晉江陳萬策,景州魏廷珍,河間王之銳,交河王蘭生,皆以得與參校為榮。家多藏書,頻年游歷,手抄雜帙不下數萬卷。歲在辛丑,卒,年八十有九。上聞,特命有地治者經紀其喪,士論榮之。

子以燕,字正謀。康熙癸酉舉人。於算學頗有悟入,有法與加減同理,而取徑特殊,能於恆星歷指中摘出致問,文鼎所謂「能助餘之思」也。早卒。

成,字玉汝,以燕子。文鼎疑日差既有二根,即宜列二表,成以為:「定朔時既有高卑盈縮之加減矣,復用於此,豈非衣復乎?」文鼎因其說,然後悟交食之非缺,比之童烏九歲能與太玄。康熙乙未進士,改編修,與修國史。成肄業蒙養齋,以故數學日進。禦制數理精蘊、歷象考成諸書,皆與分纂。所著增刪算法統宗十一卷,赤水遺珍一卷,操縵卮言一卷。

明代算家,不解立天元術,成謂立天元一即西法之借根方,其說曰;「嘗讀授時歷草求弦矢之法,先立天元一為矢,而元學士李冶所著測圜海鏡,亦用天元一立算。傳寫魯魚,算式訛舛,殊不易讀。明唐荊川、顧箬溪兩公互相推重,自謂得此中三昧。荊川之說曰:『藝士著書,往往以秘其機為奇,所謂天元一云爾,如積求之云爾,漫不省其為何語。』而箬溪則言:『細考測圜海鏡,如求城徑,即以二百四十為天元,半徑即以一百二十為天元,即知其數,何用算為?似不必立可也。』二公之言如此,餘於顧說頗不謂然,而無以解也。後供奉內廷,蒙聖祖仁皇帝授以借根之法,且諭曰:『西人名此書為阿爾熱八達,譯言東來法也。』敬受而讀之,其法神妙,誠算法之指南,而竊疑天元一之術頗與相似。復取授時歷草觀之,乃煥然冰釋,殆名異而實同,非徒似之而已。夫元時學士著書,臺官治歷,莫非此物。乃歷久失傳,猶幸遠人慕化,復得故物。東來之名,彼尚不忘所自,而明人視若贅疣而欲棄之。噫!好學深思如唐、顧二公,尚不能知其意,而淺見寡聞者,又何足道哉?」

明史館開,成與修天文、歷志,呈總裁書曰:「一歷志半系先祖之槁,但屢經改竄,非復原本,其中訛舛甚多。凡有增刪改正之處,皆逐條簽出。一,天文志不宜入歷志,擬仍另編。蓋歷以欽若授時,置閏成歲,其術委曲繁重,其理精微,為說深長。且有明二百七十餘年沿革非一事,造歷者非一家,皆須入志。雖盡力刪削,卷帙猶繁。若加入天文志之說,則恐冗雜不合史法。自司馬氏分歷與天官為二書,歷代因之,似不可易。一,天文志例載天體、星座、次舍、儀器、分野等事,遼史謂天象千古不變,歷代之志天文者近於衍,其說似是而非。蓋天象雖無古今之異,而古今之言天者,則有疏密之殊。況恆星去極,交宮中星,晨昏隱現,歲歲有差,安得謂千古不易?今擬取天文家精妙之說著於篇;其不足信者,擬削之。」

又時憲志用圖論曰:「客問於梅子曰;『史以紀事,因而不創。聞子之志時憲也用圖,此固廿一史所無,而子創為之,宜執事以為非體而欲去之也。而子固執己見,復呶呶上言,獨不記昌黎之自訟乎?吾竊為子危之!』梅子曰:『吾聞史之道貴信而直,餘本不原為史官,總裁謂時憲、天文兩志非專家不能辦,不以為固陋而委任之。餘既不獲辭,不得不盡其職。今客謂舊史無圖而疑餘之創,竊謂史之記事,亦視其信否耳,因、創非所計也。夫後史之增於前者多矣,漢書十志,已不侔於八書,而後漢皇后本紀,與魏書之志釋老,唐書之傳公主,宋史之傳道學,皆前史所無,又何疑於國史用圖之為創哉?且客未讀明史耶?明史於割員弧矢、月道距差諸圖,備載歷志,何明史不疑為創,而顧疑餘乎?』客曰:『後史增於前者,必非無因,若明史之用圖,亦有說歟?』梅子曰:『疑以傳疑,信以傳信,春秋法也,作史者誰能易之?古之治歷者數十家,大率不過增損日法,益天周,減歲餘,以求合一時而已。即太初之起數鍾律,大衍之造端蓍策,亦皆牽合,並未能深探天行之故,而發明其所以然之理。本未嘗有圖,史臣何從取而載之?至元郭太史修授時,不用積年日法,全憑實測,用句股割員以求弦矢,於是有割圜諸圖載於歷草。作元史時,不知採摭,則宋、王諸公之疏也。明之大統,實即授時。本朝纂修明史諸公,以義非圖不明,遂採歷草入志,其識極超。復經聖君賢相鑒定,不以為非體而去之,俾精義傳於無窮,洵足開萬古作史者之心胸矣。至於時憲立法之妙,義蘊之奧,悉具於圖,更不可去。如必以去圖為合體,豈以明史為非體,

而本朝之制不足法歟?且客亦知時憲之圖所自來乎?我聖祖仁皇帝憫絕學之失傳,留心探索四十餘年,見透底蘊,始親授儒臣,作圖立說,以闡明千古不傳之秘,即御制歷象考成是也。餘親承聖訓,實與匯編之列。彼前輩纂修明史,尚不忍沒古人之善,創例以傳之。而餘以承學之臣,恭紀禦制,顧恐失執事之意,而遷就迎合,以致聖學不彰,貽誤後學,尚得謂之信史乎?不信之史,人可塞責,而何用餘越俎而代之?余之呶呶,非沽直,不得已也。然則韓子之自訟,亦謂其言之可以已者耳。使韓子果務為容悅以求幸免,則諍臣之論,佛骨之表,又何為若是其侃侃哉?』客唯唯而退。」

又儀象論曰:「齊政授時,儀象與算術並重。蓋非算術,無以預推節候以前民用;非儀象,無以測現在之行度,以驗推步之疏密,而為修改之端也。虞書『璇璣玉衡』,為儀象之權輿,其制不傳。漢人創造渾天儀,即璣衡遺制,唐、宋皆仿為之。至元始有簡儀、仰儀、闚幾、景符等器,視古加詳矣。明於齊化門南倚城築觀象臺,仿元制作渾儀、簡儀、天體三儀,置於臺上,臺下有晷影堂,圭表壺漏,國初因之。康熙八年,命造新儀,十一年,告成,安置臺上,其舊儀移藏他室。五十四年,西人紀理安欲炫其能而滅棄古法,復奏制象限儀,遂將所遺舊器用作廢銅,僅存明仿元渾儀、簡儀、天體三儀而已。所制象限儀成,亦置臺上。按明史云:『嘉靖間修相風桿及簡、渾二儀,立四大表以測晷影,而立運儀、正方案、懸晷、偏晷,具備於觀象臺,一以元法為斷。』余於康熙五十二三年間,充蒙養齋匯編官,屢赴觀象臺測驗。見臺下所遺舊器甚多,而元制簡儀、仰儀諸器,俱有王珣、郭守敬監造姓名。雖不無殘缺,然睹其遺制,想見創造苦心,不覺肅然起敬也。乾隆年間,監臣受西人之愚,屢欲廢臺下餘器作銅送制造局,賴廷臣奏請存留,禮部奉旨查檢,始知僅存三儀,殆紀理安之燼餘也。夫西人欲藉技術以行其教,故將盡滅古法,使後世無所考,彼益得以居奇,其心叵測。乃監臣無識,不思存什一於千百,而反助其為虐,何哉?乾隆九年冬,有旨移置三儀於紫微殿前,古人法物,庶幾可以永存矣。」

又論句股曰:「句股和較相求,言算學者莫不留心,其法可謂詳且備矣,未有以句股積與句弦和較為問者。元學士李冶著測圜海鏡,用餘句、餘股立算,神明變化,幾如五花八門,亦未及此。豈俱未計及耶?抑有其法而遺之耶?統宗少廣章內,雖有句股積及句弦較兩題,乃偶合於句三股四之數,非通法。昔待罪蒙養齋,匯編數理精蘊,意欲立法以補其缺。先用平方展轉推求,皆不能御,思之累日,而後得用帶縱立方求句股二法。」

卒,年八十有三,謚文穆。

鈁,字導和,成第四子也。成纂叢書輯要六十餘卷,圖皆所繪。刪訂統宗圖,十之七八,皆出其手。年二十六,卒。

文鼐,字和仲,文鼎從弟也。初學歷時,未有五星通軌,無從入算。與兄文鼎取元史歷經,以三差法布為五星盈縮立成,然後算之,共成步五星式六卷。早卒。

文鼏,字爾素,文鼎季弟也。著中西經星同異考一卷。以三垣二十八宿星名,依步天歌次第,臚列其目,而以中、西有無多寡分注其下,載古歌、西歌於後。古歌即步天歌,西歌則利瑪竇所撰經天該也。其南極諸星,則據湯若望算書及南懷仁儀象志,為考證補歌,附之於末。其發凡略言:「齊七政,非先定恆星,則無從著手。故曰『七政如乘傳,恆星其地志也;七政如行棋,恆星其楸局也。』曰『恆』者,謂其終古不易;曰『經』者,謂其不同緯星南北行,『經』亦有『恆』之義焉。是編專以中、西兩家所傳之星歌星名考其多寡同異,故曰經星同異考。星官之書,自黃帝始,重黎、羲和,志天文者,紛糅不一。漢張衡云:『中外之官常明者百有二十四,可名者三百二十,為星二千五百,微星之數蓋萬一千五百二十。』至三國時,太史令陳卓始列甘、石、巫咸三家所著星,總二百八十三官,一千四百八十四星。自唐以來,以儀考測,迨宋兩朝志,始能言某星去極若干度,入某星若干度,為說較詳。此中國之言星學者。西儒星學遠有端緒,據其書所譯,周赧王丙寅古地末一測,漢永和戊寅多祿某一測,明嘉靖乙酉尼穀老一測,萬歷乙酉第穀一測,崇禎戊辰湯若望一測。國朝康熙壬子,南懷仁著儀象志,又依歲差改定黃經及赤經。今依南公志表,稽其大小,分為六等。一等大星一十有六,二等星六十有八,三等星二百有八,四等星五百一十有二,五等星三百四十有二,六等星七百三十有二,總計一千八百七十八。其微茫小星,則不能以數計。此泰西之學也。」

文鼏又有累年算稿,文鼎為錄存,名曰授時步交食式一卷。又有幾何類求新法,算書中比例規解,本無算例,文鼎作度算,用文鼏所補,而參之以陳藎謨尺算用法。

明安圖,字靜庵,蒙古正白旗人。官欽天監監正。受數學於聖祖,預修御定歷象考成後編、御定儀象考成。因西士杜德美用連比例演周徑密率及求正弦、正矢之法,知其理深奧,索解未易,因積思三十餘年,著割圜密率捷法四卷。一曰步法,於杜氏三法外,補創弧背求通弦、求矢法,仍杜氏原法,但通加一四除耳。又弦、矢求弧背,並通弦、矢求弧背,凡六法,合杜氏共成九法。其弦求弧背法,以弦為連比例二率,半徑為一率,求得二、四、六、八、十諸率,以一、三、五、七、九之五數各自乘,為累次乘數。二、三、四、五、六、七、八、九相挨,兩兩相乘,為累次除數,即用二率為第一得數。復置四率,以第一乘數乘之,第一除數除之,為第二得數。又置六率,以第一、第二乘數乘之,第一、第二除數除之,為第三得數。又置八率,以第一、第二、第三乘數乘之,第一、第二、第三除數除之,為第四得數。如是累求,至所得數祗一位止,乃★之,即所求之弧背也。矢求弧背法,倍正矢為連比例三率,亦以半徑為一率,求得五、七、九、十一諸率。以一、二、三、四、五之五數各自乘,為屢次乘數,三、四、五、六、七、八、九、十相挨,兩兩相乘,為屢次除數,即用三率為第一得數。復置五率,以第一乘數乘之,第一除數除之,為第二得數。又置七率,以第一、第二乘數乘之,第一、第二除數除之,為第三得數。又置九率,以第一、第二、第三乘數乘之,第一、第二、第三除數除之,為第四得數。如是累求,至所得數祗一位而止。開平方,即所求之弧背也,通弦求弧背,亦各加一四除。矢求弧背,則三率又多加一四。因更創餘弧求弦矢,餘弦矢求本弧,及借弧與正、餘弦互求四術。二曰用法,以角度求八線,及直線、弧線、三角形邊角相求,共設七題。謂今法所以密於古者,以用三角形也。然三角形非用八線表不能相求,惟用此法,以之立表則甚易,以之推三角形,則不用表而得數同。三、四兩卷曰法解,皆闡明弦、矢與弧背相求之根。其法先以一分弧通弦求二分弧通弧弦之數,次以一分、二分弧通弦求三分、四分全弧通弦之數,以一分三分弧通弦求五分全弧通弦之數。又因二分、五分相乘得十分,十分自乘得百分,十分、百分相乘得千分,十分、千分相乘得萬分。遂以半徑為一率,一分弧通弦為二率,各如相乘之率數,求得十、百、千、萬諸分弧率數。比例得弧背求通弦,應減四率二十四分之一,加六率八十分之一,減八率一百六十八分之一,加十率二百八十八分之一,減十二率四百四十分之一,加十四率六百二十四分之一,減十六率八百四十分之一。各四歸之,則二十四得六,為二三相乘數;八十得二十,為四五相乘數;一百六十八得四十二,為六七相乘數;二百八十八得七十二,為八九相乘數;四百四十得一百一十,為十與十一相乘數;六百二十四得一百五十六,為十二與十三相乘數;八百四十得二百一十,為十四與十五相乘數。故以二、三、四、五、六、七、八、九等數兩兩相乘,為屢次除數。又以通弦求得二率一分多,四率一分,六率九分,八率二百二十五分,十率一萬一千二十五分,十二率八十九萬三千二十五分,十四率一億八百五萬六千二十五分,得後率分數為實。各遞降二等,使二率降為四率,四率降為六率,得前率分數為法。以法除實,得四率一分,為一自乘數;六率九分,為三自乘數;八率二十五分,為五自乘數;十率四十九分,為七自乘數;十二率八十一分,為九自乘數;十四率一百二十一分,為十一自乘數;十六率一百六十九分,為十三自乘數:故以一、三、五、七、九等數各自乘為屢次乘數。次求通弦法,求得十、百、千、萬諸分弧正矢率數,比例得弧背求正矢,應減五率十二分之一,加七率三十分之一,減九率五十六分之一,加十一率九十分之一,減十三率一百三十二分之一,加十五率一百八十二分之一,減十七率二百四十分之一;而十二為三四相乘數,三十為五六相乘數,五十六為七八相乘數,九十為九與十相乘數,一百三十二為十一與十二相乘數,一百八十二為十三與十四相乘數,二百四十為十五與十六相乘數,故以三、四、五、六、七、八、九等數兩兩相乘,為屢次除數。又以正矢求得五率一分多,七率四分,九率三十六分,十一率五百七十六分,十三率一萬四千四百分,十五率五十一萬八千四百分,十七率二千五百四十萬一千六百分,為後率分數,各遞降二等為前率分數。如前通弦法,除得五率一分為一自乘數,七率四分為二自乘數,九率九分為三自乘數,十一率十六分為四自乘數,十三率二十五分為五自乘數,十五率三十六分為六自乘數,十七率四十九為七自乘數,故以一、二、三、四、五等數各自乘,為屢次乘數。書未成而卒,子新續之。

新,字景臻,安圖季子。充食俸生。安圖病且革,以所著捷法授之,新遵父命,與門下士陳際新、張肱共續成之。

陳際新,字舜五,宛平諸生。官靈臺郎,為監正。續明安圖割圜密率捷法,尋緒推究,質以生前面授之言。至乾隆甲午,始克成書。

劉湘煃,字允恭,江夏人。聞梅文鼎以歷算名當世,鬻產走千餘里,受業其門,湛思積悟,多所創獲。文鼎得之甚喜,曰:「劉生好學精進,啟予不逮!」其與人書曰:「金、水二星,歷指所說未徹,得劉生說,而後二星之有歲輪,其理確不可易。」因以所著歷學疑問囑之討論,湘煃為著訂補三卷。又謂歷法自漢、唐以來,五星最疏,故其遲、留、伏、逆皆入於占,至元郭守敬出,而五星始有推步經度之法,而緯則猶未備。西法舊亦未有緯度,至地谷而後有五星緯度,已在守敬後矣。歷書有法原、法數,並為歷法統宗。法原者,七政與交食之歷指也;法數者,七政與交食經緯之表也,故歷指實為造表之根本。今歷所載金、水,歷指如其法以造表,則與所步之表不合,如其表以推算測天,則又密合,是歷雖有表數,而猶未知立表之根也。」乃作五星法象五卷,文鼎深契其說,摘其要目為五星紀要。

湘煃又欲為渾蓋通憲天盤安星之用,以戊辰歷元加歲差,用弧三角法,作恆星經緯表根一卷,及月離交均表根、黃白距度表根各一卷,皆補新法所未及也。所著又有論日、月食算稿各一卷,各省北極出地圖說一卷,答全椒吳荀淑歷算十問書一卷。

王元啟,字宋賢,號惺齋,嘉興人。乾隆辛未進士,授將樂縣知縣。究心律歷句股之學,著書已刻者為惺齋雜著。內有史記、漢書正譌兩種,其正史記之譌者,為律書、歷書、天官書各一卷;正漢書之譌者,為律歷志上下二卷。未刻者為歷法記疑、句股衍、角度衍、九章雜論。而句股衍一書,因繁求簡,最為精晰。分甲、乙、丙三集,甲集術原三卷,乙集綱要二卷,丙集晰義四卷。甲集首卷通論術原,為句股因積求邊張本。二卷專論立方,因及平方法。三卷專論和數開立方,所以盡立方諸數之變。乙集兩卷,為相求法百二十三則之綱要。丙集四卷,即相求法,逐則分晰其義,專取發明立法之意。

其總序曰:「句股弦相求法,參以和較,凡得七十八則,求句股中函數。又有冪積求容員、容方、容縱方,及依弦作底求容方,與句股求外方、外員之數。又有積數與句股和較相求容方,與句股餘數相求之法。綜而計之,凡得二十九則。立表測量,得求高、求遠、求深三則,重表亦然。舊算書多簡略,詳者又苦錯出無緒。間嘗力為區別,使各以類從,先定相求法百十三則。甲申仲秋,復理前緒,逐一布算,捷於舊法,而舊法仍附見,以資參考。至以中函積與弦之所和、所較相求而得句、股、弦之正數,舊法罕見,今亦竊擬一法,以附於後。又別創截弦分兩,及補句求股、補股求句之法,分為六則,使不成句股之形,亦化為句股。並載不成句股求中函積二則,容方、容員四則,外切員徑一則,員內累求句股六則,凡又一十九則。以該西術三角之算,兼備割員之用。使學者知周髀一經,於術無所不該。後人不能觸類旁通,以盡其變,故使西術得出而爭勝,其實西術亦出周髀,不能出折句為股之外也。」

又略例引言曰:「算家句股一門,為術最繁,非鑿指一數以為布算之準,難以虛領其義。然如廣三修四見於經者,特其正例,正例外變例尤多。必欲正變兼呈,則一卷中彼此錯出,使閱者耳目數易,轉增煩憒。茲特標舉略例,人並不成句股之形亦附見焉,以盡句股之變,而該三角之法。」

又答友問句股書曰:「欲求句股,先學開方,方有正方、縱方之異。縱方則以修廣之和、較數開之,其次則求四率比例,有三率求四率之法,有二率求三率之法,又有一率求三率之法。知此即可以知求句、股、弦各無零數法。以三率之中率為主,倍中率為股,首末二率相減為句,相加為弦。依此衍之,得句股略例十數則,然後以句、股、弦為正數,兩數相加為和,相減為較。又有句股三數相加減之和較數,弦與和,和弦與較和三數相加之和數也;弦與較,較弦與和較三數相減之較數也。三數相加減,今名之為兼三和較。凡正數和較之數各三,兼三和較各二,共十三數。十三數中,隨舉兩數,即可求句股弦全數。凡得相求法九十四則,而容方、容員、截股分兩、立表測量單表、重表之法,猶不與焉。其次則求截弦分兩之法,是為一句股分兩句股,即可以知不成句股亦可以分兩句股。不成句股分兩句股,即西法三角算之所由名,今則總以句股概之。其法取大小兩句股形,小股與大句同數者合為一形,即為不成句股之形。分之為兩,則所謂中垂線者,即小矩之股,大矩之句。以此衍之,又得不成句股略例二十餘則。依類推之,又得合形分兩、削形求全二法。合形分兩,則有正合形截偶分兩、反合形截中分兩、偏合形截邊分兩之法。削形求全,則有削去正矩、偏矩之殊,偏矩中又有淺削、深削之分。知此則句股之學盡矣。」元啟嘗曰:「我無他長,惟好學深思,心知其意而已。」然其句股術一書,幾欲駕梅文鼎而上之,為算術中不可少之書云。

硃鴻,字云陸,秀水人。嘉慶七年進士,改翰林院庶吉士,散館授編修。擢御史,歷給事中,出官督理湖南糧儲道。研精算學。同郡錢儀吉譔三國會要,集乾象、景初二術成,嘗為作注。烏程陳傑時為臺官博士,陽湖董祐誠亦客京邸,皆日從講數,各出所得相質問。舊無橢圓求周術,為祐誠言,圜柱斜剖,則成橢員,可以句股形求之。祐誠既發明其說,系以圖釋。初得杜德美割圜九術寫本,以示祐誠,創圖解三卷。既成,復得密率捷法於李潢家,則蒙古監正明安圖師弟續繹之書也,與傳寫本互異。鴻曾依杜法步算,徑一者,周三一四一五九二六五三五八九七九三二三八四六二六四三一八六三六七四七二二七九五一四,周十者,徑三一八三零九八八六一八三七九零六七一五三七七六七五四六六九六三八九零五六六六一。徐有玉採入務民義齋算學中。道光十年後,辭官仍居京師,譔考工記車制參解。又評程氏易疇考工創物小記,多所糾正云。

博啟,字繪亭,滿洲正白旗人。乾隆中,官欽天監監副。嘗因句股和較之術,前人論之極詳,獨句股形中所容之方邊、員徑、垂線三事,尚缺而未備。爰以三事分配和較,創法六十。惜其書未刊,法不傳。今所傳者,惟有方邊及垂線求句、股、弦一題。法用平行線剖容方冪為四小句股形,借垂線為小句股和,借方邊為小弦,求小句小股。以小股與垂線比,若方邊與句比;以小句與垂線比,若方邊與弦比。道光初,方履亨官監正,每舉此題課士。其後得甘泉羅士琳力為表章,博術乃復明於世。

羅論云:「曩者聞方慎菴監正言繪亭監副有是法,失傳。因仿監副遺法,用平行線剖半員冪為四小句股形,以半圓徑減垂線餘,借為小句股和,借半員為小弦,求得小句、小股。以小股比垂線,若半員徑比股;以小股比股,若半員徑比弦。又以半員徑減方邊,得較。用平行線剖較冪為四小句股形,借半員徑為小句股和,借較為小弦,求得小句、小股。以小股比半員徑,若方邊比句;以小句比半員徑,若方邊比股,以小股比股,若較比弦。用補副監之遺。復用天元術演得三事和較六十題,更立天、地兩元為廣例二十五術,撰句股容三事拾遺四卷。更試變通其術,御以八線,取方邊用方斜率,得容方中之斜線。以垂線為一率,半徑為二率,斜線為三率,求得四率為正割。檢八線表得度用,與四十五度相加減,得垂線所分之大小兩弧,副以半徑為一率,垂線為二率,小弧正割為三率,求得四率為句。如以大弧正割為三率,求得四率為股,又如以大小兩弧之兩正切為三率,求得四率,為大小兩弧之兩分弦,相人並得弦餘。二題仿此,其得數同,而尾數有奇零。以八線表所列之數至單位止,單位以下,棄其餘分,故不能如句股與天元所得之密合。或有妄詆天元術不能馭三角和較者,抑知天元創於宋、明之間,安能逆知西法之有三角而豫為立法?要在學者善為會通耳。試設平三角形,有一角而角在兩邊之中,有大邊與對邊和,有小邊與對邊和,求三道及垂線,此西人常法所不能御者。若立天元一術,則任求何邊或和數或較數,皆一平方即得。然則天元之與西法,其優劣可見矣。」

許如蘭,字芳谷,全椒人。乾隆三十年舉人,大挑知縣,分發福建。因親老改江西,歷任浮梁、新建等縣事。丁憂服闋,赴福建,題補侯官,未履任,會瘴氣發,病卒。

如蘭性敏,所讀書皆究心精妙,於歷算始習西法,通薛鳳祚所譯天步真原、天學會通。時同縣山西寧武同知吳烺受梅文鼎學於劉湘煃,如蘭因並習梅氏歷算。又於乾隆四十年夏,謁戴震於京都,受句股割圜記。四十四年,謁董化星於常州。戴傳緝古算經十書,而董則專業薛氏者也。由是兼通中、西之學。

嘗謂其弟子胡早春曰:「古人以句股方程列於小學,童而習之,人人能曉,今則老宿不能通其義。一則時尚帖括,視句股為不急之務;再則習為風雅,不屑持籌握算,效疇人子弟所為。噫,過矣!」又謂:「士大夫不精弧矢之術,雖識天文,無益也。疇人算工不明象數之理,雖能步算,無益也。」著有乾象拾遺、春暉樓集諸書,今多散佚。

其存者,有書梅氏月建非專言斗柄論後,略曰:「天氣渾淪,無可識認,古人不得已,即以恆星為天以識日躔。恆星積久而差,冬至日躔不在原宿,始立歲差之法。古謂恆星不動,而黃道西移。今測普天星座皆動,其經緯之度,不隨赤道運轉,而順黃道東移。故謂黃道不動,而恆星東行,與七政同一法。」又謂:「古人以中數為歲,朔數為年。上古氣朔同日,故月建起於節氣,而不起於中氣;日躔過宮,起於中氣,而不起於節氣。起於節氣,故曰冬至子之半;起於中氣,故曰冬至日躔星紀之次也。然則一歲十二建,乃天道經歷十二辰,故謂之月建,此萬古不易者也。斗柄所指分位不真,且恆星東移,積久有差,辨之誠是也。但古人云:『斗為帝車,斟酌元氣而布之四方』。又曰:『招搖柬指。』不過言天道無跡。可見順時布化,斗柄有象可徵耳。拘泥其詞,則惑矣。」其歲差說略曰:「恆星一年東行五十餘秒,又黃、赤二道斜交,並非平行,於左旋至速之中,微斜牽向右。日之於天,猶經緯之於日也。日行至黃道分至節氣之限,則春秋寒暑皆隨之而應。七政躔於各宮,遇各宮燥濕寒溫風雨,則隨恆星之性而應。然則冬、夏二至,乃黃道上子、午之位也。春、秋二分,乃黃道上卯、酉之位也。惟唐、虞時冬至日躔虛中,恆星之子中,正逢黃道之子中。嗣是漸差,而東周在女,漢在斗,今在箕。黃道之子,非恆星之子也。以醜宮初度為冬至者,因周時冬至恆星已差至丑,周人即以恆星為黃道之十二次,故命丑為星紀,言諸星以此紀也。其實丑乃周時恆星之宿度,並非恆星之子中。今並不在丑,又移至寅十餘度矣。由今箕一以上溯古虛五,歷年四千有餘,已差至五十八度,此恆星東行之明驗也。」其他著論無關歷算者不錄。


\end{pinyinscope}