\article{志四十}

\begin{pinyinscope}
地理十二

△浙江

浙江:禹貢揚州之域。明設布政使司。清初為浙江省,置巡撫,福建置總督。兼轄之。駐福州。順治十五年,置浙江總督。駐溫州。康熙元年移駐杭州。八年裁,尋復。二十五年復裁,兼轄如故。雍正五年,改巡撫為總督。十二年,仍為巡撫。乾隆元年,復置浙江總督。三年,改閩浙總督,自是為定制。順治五年,遣固山額真金礪來杭駐防,掌平南將軍印。康熙初年改將軍,總督駐福州,將軍、巡撫駐杭州。三十六年,舟山置定海縣,以舊縣改置鎮海。雍正六年,增置溫臺玉環。道光二十一年,升定海為直隸。乾隆三十八年,升海寧縣為州,降安吉州為縣。領府十一,直隸一,州一,一,縣七十五。東至海中普陀山;四百九十里。西至安徽歙縣界;三百七十里。南至福建壽寧界;七百八十四里。北至江蘇吳縣界。二百里。廣八百八十里,袤一千二百八十里。北極高二十七度三十五分至三十度五十八分。京師偏東一度五十五分至五度四十分。宣統三年,編戶三百八十八萬八千三百一十一,口一千六百一十四萬九千四百五。其名山:會稽、天目、四明、天臺、括蒼、金華。其大川:浙江、浦陽江、苕溪。天目自餘杭飛翥而入,為黃山三天都之一。

杭州府:沖,繁,難。杭嘉湖道治所。初治嘉興府,今改駐。巡撫,布政、交涉、提學、提法、鹽運各司,糧儲、巡警、勸業各道,及將軍、副都統,織造,同駐。明為浙江布政使司,領縣九。順治初,因明制。乾隆三十八年,升海寧縣為州。東北距京師四千二百里。廣百九十五里,袤百三十里。北極高三十度十七分。京師偏東三度三十九分。領州一,縣八。錢塘沖,繁,難。倚。西:靈隱山,古武林山,西湖源此。北為南北二高峰。西南:天竺山,其東丁家山。瀕湖周三十里。唐刺史白居易、宋守蘇軾導。厥後水淺葑橫,縱成蘇堤,橫成白是。迤西為孤嶼,有行宮,與城內吳山為二。其北聖塘澗水,石函三閘,以時防洩。其東湧金閘,導之入城,曰城河。浙江,古淛河,東南自富陽入。城河出武林門,會西谿入下塘河,一名宦塘河,逕江漲橋。有鹽場司,兼管吳山驛。出北新關,有橋曰拱宸。光緒二十一年與日本約,定為通商埠。抵奉口陡門。左會苕溪。有西谿、瓶窯二鎮。宣統元年移府同知駐瓶窯。有武林驛。城南鹽場司。浙江驛。仁和沖,繁,難。倚。南:鳳凰山。西北:皋亭山。浙江,西南自蕭山、錢塘入,東北流入海。捍海石塘,自錢塘烏龍廟一堡至戚井村十二堡,西防同知治。又東至翁家埠十七堡,中防同知治。城河出候潮門入上塘河,舊名運河,一曰夾官河,北流,右出枝津為備塘河,入海寧。下塘河西自錢塘入,西北流者宦塘河,與苕溪會。其北流者為新開運河,逕塘棲,歧為二,一入德清,一入海寧。苕溪自錢塘緣西北界入德清、武康為界水。有鹽場司。湯鎮、塘棲鎮巡司二。又德勝、臨平二鎮。海寧州疲,繁,難。府東北百七十里。東:黃灣山,臨黃灣浦入石墩山,迤東鳳凰山,並建砲臺。浙江,西南自仁和入,出鱉子亹為大海。自海鹽至此,潮流倒灌,與江水相薄,此為浙西第一門戶,南北二大亹扼其中。潮昔趨南,後改徙北,一線危堤,屢受沖激。自仁和十七堡至南門外三十三堡,東防同知治。又自一堡至十八堡界海鹽,累朝修築。下塘河西北自石門、德清界入,逕永安橋,歧為二。北支為運河,入石門,為長安塘。東支復歧為二,一周王廟塘河,一許公塘河,入海鹽。左出枝津為硤石河,入海鹽、桐鄉為界水。上塘河自仁和入,流為二十五里塘河,合備塘河,會袁花塘河,入海鹽,為招寶塘。有袁花、郭店、硤石、石墩、長安、馬牧港六鎮。長安,州判駐。戴家橋有行宮,有汛。有巡司。許村、西路鹽場二。富陽沖,繁。府西南九十里。東:五洩水。下二洩屬諸暨。洩或作「洩」,洩溪源此。西:貝山。北:桐嶺。富春江,浙江上流,西南自桐廬入,納浦江。湖洑水即壺源江,右合剡浦,左納莧浦,自天目山伏流,入縣西北始出山。錯出復入,流為白洋溪,逕城南,合安吳川,抵漁山埠,北入錢塘,南入蕭山。城河即慶春河,起觀山訖莧浦。陡門二。有漁山、靈橋、場口、湯家、洋波場五鎮。會江驛。餘杭繁,難。府西北七十里。南:由拳山。西北:禹航山。北:獨松嶺,並百丈、幽嶺為三關。南苕溪、中苕溪西自臨安分入而合,會北苕溪,為瓶窯大河。又一支出武康界孔井山入焉。南苕性悍,逼臨城東,廝二瀽以引之,自滾壩洩流為餘杭塘河。其南南湖,北流為黃母港,會苕溪,分上湖、下湖。鎮三:雙溪、石瀨、閑林。臨安簡。府西北百里。西南:臨安山,縣以此名。西:楓樹嶺。天目山,即山海經浮玉之山,苕水出其陰,合董、平、鵠三溪。又東南,右合潘溪,左合馬跑泉,側城北而東,錦溪合南溪入焉。又東為南苕溪,西北為中苕溪。松溪出南黃嶺,其西婁塘,當苕之沖,乾隆五年圮。有青山、亭川、板橋、化龍、橫阪、三口、鶴山七鎮。青山近城。公、姥二山夾鎖苕源,最險要。於潛簡。府西北百七十里。東:石柱山。西北:龍翔嶺。西:天目,上有兩池,若左右目。左屬臨安曰東溪,右曰西溪,出尖頂,合流,逕白鶴橋雙溪口,合虞溪為浮溪。至寮車橋,左合藻溪,右合交溪。紫溪西南自昌化入合之。其上流柳溪。有千秋、白沙、桐嶺、豪千、孔夫諸關。新城簡。府西南百二十里。西:大雷山。西北:青牛嶺。並天目支阜。南:百丈嶺,界餘杭。葛溪出,合武源、里仁二水,出大源橋,右合菖溪,左納槎溪,上承分水廣陵溪為三溪口,逕練頭莊為練頭溪,合松溪為雙港口,曰鼉江。北有塔山堰。有東安鎮。昌化簡。府西二百十里。西南:福泉山,其東蘆嶺。南:楊嶺。西:昱嶺。北:嶠嶺、黃花嶺。並置關。又馬頭嶺,上溪出,合高溪、仁里溪,東流為無他溪。合雲溪,右納頰口溪,逕晚山下為西晚溪,逕城南為雙溪。又南為下阮溪、三溪。伽溪南峽川,上博溪東南納分水青坑溪、覽溪,西南承蕭、浦二水,以達柳溪。其中柯相公潭,與於潛為界水。有手穴手、頰口、柯橋三鎮。

嘉興府:沖,繁,疲,難。隸杭嘉湖道。副將駐。乾隆十五年裁所,並海寧衛為嘉興府。西南距省治百八十里。廣百五十里,袤百里。北極高三十度五十二分。京師偏東四度三分。領縣七。嘉興沖,繁,疲,難。倚。府境之水二派,曰武林、天目,而天目派由石門、秀水入運,則合武林為一。長水塘南自桐鄉、海鹽界入,合練浦塘。海鹽塘東南自其縣入,並匯於南湖,一名鴛鴦湖,東南接滮湖。六里涇承南湖水,歧為二,一魏塘,一漢塘。合王廟、空廟、眾歡諸塘,左出枝津為伍子塘。有王店、新豐、鍾埭、新禮四鎮。王店、新豐有汛。西水驛有丞。有鐵路。秀水沖,繁,難。明宣德四年析嘉興置,同附郭。西南:運河自桐鄉入,合石人涇,左出枝津為新塍南塘,側城西南注南湖。新塍塘西北自江蘇震澤入,納新塍北塘,與南塘合,逕北麗橋。長水、海鹽二塘東南自嘉興注之,是為秀水,縣以是名。東北流,右出枝津北流,瀦為姚涇、楊舍、上馬諸港,分趨南官、北官、連四、梅家、陸家諸蕩,入江蘇吳江。魏塘,東自嘉興入,入嘉善。爛溪,西北自桐鄉入,入震澤、吳江為界水。王江涇,通判駐。舊設同知及東西兩塘協辦同知,並裁。濮院鎮、新塍、九里匯有汛。新城、陸門二鎮。嘉善繁,疲,難。府東北三十里。南:瓶山。魏塘自秀水入,會東郭湖塘,貫西城壕,出東門流為楓涇塘,入江蘇婁縣。伍子塘南自嘉興入,貫南城壕,出北門入祥符蕩。其北沈家、白魚、上白諸蕩,西北烏盆潭、木斜湖、吳家漾,並入江蘇青浦。西北:汾湖潀流匯處,播為南北許蕩、南北夏墓蕩,入吳江。斜塘鎮,縣丞駐。楓涇鎮,主簿駐。天寧莊鎮,有汛。魏塘、陶莊、乾家窯三鎮。海鹽繁,難。府東南八十里。南:秦駐。東南:白塔。西北:獨山。海,東北自平湖入,逕縣城,又南至澉浦。道光二十四年設水師都司。其西長墻山,橫截海灣,建砲臺。捍海石塘,西南接海寧,東北亙平湖。秦駐塢水出秦駐山,歧為三,通曰秦溪,縱橫數十里,貫以招寶、烏坵兩塘。招寶塘西南自海寧入,烏坵塘出長生橋合之,是為嘉興塘。又自興城東海貫城壕出北門流為平湖塘。長水塘亦自海寧入,緣西北界,錯出復入,入嘉興、桐鄉為界水。有鮑郎、海沙二鹽場司。海口、枕蕩二鎮。有汛。石門沖,繁,難。府西南八十里。明為崇德。康熙元年改名。西北:含山。運河,西南自德清入,納海寧下塘枝水。左枝為南界涇,入歸安,右納下塘河、長安塘,並自海寧南注之。左枝為南沙渚塘,入海寧、桐鄉為界水,側城南而北,右出二枝津為中北沙渚塘,又北襟塘,左石人、瓜塔、沙木諸涇,折東環灣如帶,是為王灣。其塘右諸涇、半截運河注塘,左半由含山入歸安。有玉溪鎮。皁林驛。平湖繁,疲,難。府東南五十四里。東南:雅山,又苦竹山,水師戰艦泊焉。迤東羊、許二山,峭立海中,為江、浙分疆處,浙西第三門戶也。海,東南自江蘇金山衛入,又西逕乍浦。雍正二年設水師營,七年,移杭州副都統來駐。道光三年移府海防同知並駐。東西兩海口,北接廣陳汛。自此入澉浦達杭州,為錢塘江口北岸,西人名乍浦灣。漢塘西自嘉興入,分流注當湖。右得平湖塘,西南自海鹽入合之。左得乍浦塘,出東南前黃山,合何陳塘注之。東北流,歧為二,分流復合,入泖湖,其口曰硃洞港。有汛一。廣陳塘右出枝津為鹽船河,出放港為秦河,入泖湖,正渠並入之。鎮五:白門、廣陳、戶浦、新埭、青蓮寺。有白沙灣巡司。蘆瀝、橫浦二鹽場。天後宮、觀山麓、陳山嘴砲臺。桐鄉繁,難。府西南五十里。東:殳山。南:王家山。運河西南自石門入,枝津入震澤界為爛溪,正渠逕永新橋歧為三,南流注永新港,達石人涇,北流注五往涇入爛溪,東流入秀水。石人涇亦自石門入,合瓜塔涇及北沙渚塘,逕屠甸,復合沙木涇。南沙渚塘亦自石門入,合中沙渚塘,入海寧。長水塘自海寧、海鹽緣東南界入海鹽、嘉興為界水。鎮五:濮院、爐鎮、皁林、陳莊,又青墩巡司。

湖州府:繁,疲,難。隸杭嘉湖道。明,領州一,縣五。副將、所千總駐。乾隆三十八年,改安吉為縣。東南距省治百八十里。廣百八十二里,袤百三十八里。北極高三十度五十二分。京師偏東三度二十七分。領縣七。烏程繁,疲,難。倚。南:衡山、金蓋山。西北:弁山。太湖,東北八十里,古震澤,周五百里,匯上游諸水。大小雷山扼其東,西亙長興,北至小雷界吳江。浙源為東西二苕溪。東苕溪東南自歸安入,合西塘河注碧浪湖。山塘溪亦自其縣注之。合妙喜港,左得呂山塘,西北自長興入合之。左出枝津為北塘河,分瀦二十五漊港。西苕溪亦自長興入,合四安溪。左出枝津小梅港及橫港,分瀦十一漊港。正渠與東苕合,是為江渚,匯合瀦大錢港入太湖,三十六漊港並入之。其東,運河自歸安入,合潯溪入震澤。爛溪自歸安、桐鄉緣界,左出枝津為白米塘河,納歸安中塘河並入之。太湖營守備駐,同知駐烏鎮,通判駐南潯溪,並晟舍、大錢、馬要、圓通橋、小梅、青山、伍浦有汛。南潯、大錢湖口二巡司。苕溪驛。歸安繁,疲,難。倚。東南:長超山。西南:梅峰山。東北:太湖。東苕溪,東南自德清入,左出枝津為吳興塘,納石門含山塘注錢山漾。西塘河,南自武康入,洛舍漾逾埭溪注之,與東苕合。逕城南,呂山塘西自烏程注之,右出枝津為菜花涇,播為運河,歷月河為霅溪,抵臨湖水門。自錢山漾至此,與烏程為界水。雙菱鎮,守備駐。縣丞駐射村港鎮;主薄駐菱湖鎮。並涵山、善連有汛。璉巿、埭溪二巡司。長興沖,繁。府西北六十里。西:白石山。西北:碣石山。北:啄木嶺,界江蘇荊溪。東北:太湖。大雷、小雷西南自安吉入,其西四安溪,出硃灣嶺,合罨畫溪。西北箬溪二源合於長安渡,故曰合溪。罨畫溪右出枝津為呂山塘,歧為二:一中橫塘,入烏程;一南橫塘,入北橫塘。正渠逕橫石橋,與北橫塘會。其北顧渚溪,出懸臼嶺,流為紫花澗,瀦於包洋湖,分二十八漊港,南至蔡浦,接烏程小梅港,西至夾浦,為顧渚溪來源。橫溪,出東北橫玉山,分瀦長大、上周、蔣家、金村四港。香山嶺水瀦雙橋港,浮渚嶺水瀦斯圻港,並入太湖。鎮六。夾浦,縣丞駐。有四安、合溪二巡司。並新塘有汛。德清繁,疲,難。府南九十里。東:德清山,本烏山,縣以此名。東北:澉山。西北:白峴山。苕溪南自仁和入,納武康南塘河,逕南水門,曰龜溪。左出枝津入洛舍漾,為歸安、武康界水。正渠貫城壕西北流,東苕溪分運河入之。東塘河,其枝津運河亦自仁和入,緣東南界錯出復入。有錢巿鎮巡司。武康疲,難。府南百二十里。東:封山。西北:草乾。又銅峴山為餘英溪北源,南源出西上郎山,匯於牌頭,逕新塘灘為前溪,會湘溪、後溪。其枝津側城南而東,左得封溪故道,又東北合阜溪,左出枝津注洛舍漾。苕溪自錢塘入,緣東南界合官塘河,北流為餘不溪。鎮二:牌頭、上柏。安吉疲,難。府西南百三十里。明為州。乾隆三十八年降。東南:白楊山。北:金烏山,界長興。東南:獨松嶺,界餘杭。東溪出大溪,即苕溪,西南自孝豐入,逕塔潭。東溪合梅園溪,復納孝豐豐食、吳渚二溪入焉。側城東合丁埠港水,北流,左合理溪,右合魯家溪,逕梅溪。渾水瀆亦自孝豐注之,又北合四公溪。小溪巿、梅溪鎮、遞鋪鎮有汛。乾隆十七年移州判駐南溪。孝豐簡。府西南九十里。西南:天目山,界臨安、於潛。又桃花山。南:廣苕山,苕溪出,合深溪、橫溪。其東大海嶺,東濱溪出,下流為吳渚溪,又東巿嶺,大溪出,下流為豐食溪。梅家山溪出北梅家山,下流為渾水瀆。有天目山巡司。沿乾鎮。

寧波府:沖,繁。寧紹臺道治所。提督駐。康熙二十六年,改定海曰鎮海,移置定海於舟山。宣統三年,增置南田。西北距省治四百四十里。廣二百二十四里,袤二百八里。北極高二十九度五十五分。京師偏東四度五十七分。領縣六。鄞沖,繁,難。倚。西南:四明山。東傍海為鄮山。西南:灌頂、梅園山、海浦、羊求山。海,東南自象山入,逕大嵩水口。順治十七年裁所置游擊。雍正七年設同知。東接瞻崎,南毗鹽場,有司。又北通東錢湖,匯東境諸水,有南北二塘、梅墟石塘。奉化江南自其縣入,鄞江出四明山,合而北流,為甬江。又與慈谿江合,河流縱貫。道光二十三年開租界,與英立約,為五口通商之一。逕白沙巿。左出一枝津,首白沙,訖張家堰,與鎮海為界水。西南:南塘河出四明,歧為二:前港貫城壕,注日月二湖;後港即里弄港,會中塘、西塘及中南二河入江。其東前塘河,三源匯於橫溪,出橫石橋,會中塘、北塘河,逕和安橋,為三河總渠,注大石碶入江。有浙海關。四明水驛。鐵路。橫山嶴、猛港等砲臺。慈谿繁,疲,難。府西北五十里。西南:大寶、句餘。東南:石柱山。海,西北自餘姚入,北抵海鹽。迤東有海王山,又東為松浦港口。港分杜湖水,出三眼橋,界鎮海。慈谿江上流即姚江,自餘姚入,逕丈亭渡,歧為二:前江歷車廄嶺,抵大浹江口,會甬江;後江貫城壕,出東郭,曰管山江,南抵西渡會前江,西抵化紙閘會橫溪。西南:藍溪自龔村匯二十六隩水,出玉女山。西南諸水出四明,入蛟門,北資杜、白二湖。海壖設塘置閘,曰松浦、淹浦、古窯、津浦、洋浦。鎮五:丈亭、洪塘、東埠、松浦有巡司、向頭廢司。鶴鳴鹽場司。車廄驛。有瓜蒂山、東山砲臺。奉化疲,難。府西南八十里。東南:奉化山,縣以此名。又鮚埼山。光緒十年法兵艦來犯砲臺,斃其將孤拔,遂遁。海,東北自鄞入,逕湖頭渡關,又西逕塔山城、應家棚,東接楊村汛,又西為河泊所。其口有懸山。又天門山,下即漢志天門水,南為江彭山,界象山、寧海。縣溪出西南大公隩,七十二曲,硃、白二溪逾趙河注之。抵璡琳碶,歧流為長它江,抵三江口。金溪出東金瓘山,逕白杜河來會。其西剡溪,出六詔嶺,合左溪,納西晦溪,是為奉化江。又東合長它江,迤東北會甬江。塔山城巡司。應家棚守備。有鮚埼鎮。連山驛。祥嶺、董公、桐照等砲臺。鎮海沖,繁。府東北六十里。海,西北自慈谿入,東至澥浦,水師參將駐,為郡北要害。又東逕招寶山抵鉗口門。道光二十一年英兵艦由此登岸。其東蛟門,西虎蹲,並稱天險。又東穿山所,臨黃歧洋。又東崎頭角,臨崎洋。頭長、跳嘴山扼其口,並為郡東要害。轉南至霩[B232]所,南接昆亭汛。迤南撲蛇山,臨雙嶼港。又南至荒嶼,界鄞。外洋各島,其著者,東北七姊妹山,東西霍山,迤東搗杵山。東距金塘水道為大隩子港。轉南有天隍山,東西二嶼,界象山。湧江西南自鄞入,入海口為大浹江口,即古甬句,東自張家堰至此,與鄞為界水。西北諸水瀦為鳳浦、沈窖、靈緒、白沙四湖,播為巨河。夾江河西自鄞分甬江水,逾白沙,歷鷺林,入前大河。中大河上流後江自慈谿入,北流為西河。大閘河上流松浦亦自其縣入,歧為三:一抵澥浦入海,一流為西大河入浹江河,一逕箭港為後大河。其中港貫前、後兩河,並入城河,出頭二閘入海。其東南上河注逕大碶巿,中河逕穿山碶,會蘆江河海。有莊巿、柴埠二鎮。定海關有管界、長山、穿山三巡司。龍頭、穿山、清泉三場。有北城角、威遠、定遠、宏遠、平遠、綏遠、靖遠、鎮遠各砲臺。象山簡。府東南二百七十里。海,西自奉化入,逕西周渡,虎山扼其口,南接泗州頭汛。逕東西塔嘴入,為陳山渡,接海口汛。又東逕前倉所,西接珠溪汛,東對牛鼻山。其東北獵戶角,為南岸盡處。迤南逕爵溪城,青門、羊背諸山扼之,並為郡南要害。其南天目山,東即韭山列島。又南至昌國,順治中裁衛,置水師營都司。又南至石浦,明為所,道光三年移府海防同知駐。南出為東門,與小銅礁對峙。中為銅瓦門,道光二十二年英兵艦來犯,由此門入。過此曰下灣門、金齒門,西為林門、珠門。又南至大田島。光緒初,派開墾委員駐此。宣統三年改為撫民,移府通判、左營游擊駐樊嶴,守備、千總駐龍泉、鶴浦兩塘。島北為石浦港,西即三門灣。轉西至臺寧嶼,界寧海。東大河出王家嶴及旋井、飛鳳諸山,注會源碶。南大河出鳳躍山,自西水門接諸河,注朝宗碶。西大河出郭家諸澗,注靈長碶,並入海。上洋三碶蓄三河水,防洩下洋,下洋永豐諸碶防洩入海。有南田、竹山二巡司。前嶴嶺、高塘山等砲臺。南田簡。舊隸象山。宣統三年新置,治大佛頭山麓。孤峙海中,東、南、西大洋,惟北距石浦較近,水程十有餘里。海中十洲,此為第一。明湯和懼趙宋遺族苞,擬廢象山、棄滃洲,遂徙南田居民。後復有群入墾煎者。道光三年,巡撫帥承瀛奏謂「明與定海、玉環並封禁。嗣定、玉展闢而南田獨否,以彼泥潮而此沙壚,匪船易留,故復徙之」。大小共一百八嶴,南路四十九,北路五十九。

定海直隸:簡。隸寧紹臺道。總兵及同知駐。古句章地。明為衛。康熙二十七年改縣。道光二十一年升直隸。西距省治七百六十里。廣百四十里,袤八十三里。北極高二十九度五十九分。京師偏東五度五十八分。舟山,古翁洲山,即定海山。康熙、道光間陷於英。咸豐間復陷英、法。澳外島嶼屹齒。西洋螺角,東竟留角,對峙若門。洋螺南為螺頭山,西即大榭山,接象山港。口北各島為南險汛。又六橫山西對前倉嘴,牛鼻山扼其中。其東南為桃花山、登步山。桃花東北、登步東南為硃家島,中有烏沙山,曰烏沙門。東岸狼灣,其東普陀山西北嘴對舟山東嘴,中為蓮花洋,西即沈家門,商舶鱗萃。北達蘭秀灣,西北距千覽角,曰龜水道,青山屹立,中為灌門,航路最險。蘭秀以北為官山,中為乩山門。官山以北為岱山,中為高定洋,利停泊。其西北大沙澳,北距長白山,中為長白水道。其西為岑港。西北接大小沙汛。又西即金塘水道。其東北為冊子山,中為西堠門。岱山以西為兩頭洞山。又大漁山、嶼心腦山與乍浦為犄角。以東為竹嶼港。又東曰大小長塗、東西福山,並為北險汛。其東北大衢山,四圍多澳。有岑港、道頭二巡司。瀝港、沈家門二鎮。定遠、振威、永清等砲臺。

紹興府:沖,繁,難。隸寧紹臺道。副將、衛守備駐。西北距省治百四十里。廣三百二十里,袤二百九十里。北極高三十度五分。京師偏東四度四分。領縣八。山陰沖,繁,難。倚。西北:興龍山,南麓本臥龍山,康熙二十七年駐蹕,改。南:龜山、陽臺、蘭渚山、秦望山。西北:塗山、梅山。東北:蕺山。海,自蕭山入,逕三江口,為杭州灣南岸水口,對岸為海寧。南大亹、中小亹扼其中。潮昔趨南,暴岸沖擊,其後海塘東接會稽,西亙蕭山。浦陽江西南自諸暨入。運河西北自蕭山入,合鑒湖枝津北注瓜渚湖。湖分青電湖水入西水門,復合入銅盤湖港,抵港口與西小江會。江分為二,自蕭山古萬安橋入,緣北界,西溪出雞頭山注之。逕錢清鎮,錯出復入,抵三江閘。湘湖自蕭山貫運河來會,又東入海。鑒湖,古鏡湖,周三百五十里,今衹存西溪及會稽,若耶溪為其別源,湘湖為其正源,僅十五里矣。三江城,通判駐,有鹽場司,與錢清為二。有柯橋巡司,蓬萊驛。會稽沖,繁。倚。南:會稽山,有禹陵,縣以此名。其宛委、秦望、天柱,並為支阜。海,東北自山陰入,逕瀝海城,南接蟶浦。西曰西會渚,北與澉浦遙對,為險汛,有防海塘。曹娥江上流剡溪,東南自上虞入,納嵊小舜江,錯出復入,歷曹娥壩,抵宣港入海。運河自曹娥壩分諸溪河水,逕通陵橋,會欑宮河,宋六陵在焉。出五雲門西,有若耶溪出化山注之,入山陰運河。有三江、東江、曹娥鹽場。曹娥巡司。東關驛。纂風鎮。平水關、宣港、臨山砲臺。蕭山沖,繁,難。府西北百十里。東南:大羅山。東北:龕、赭二山。浙江西北自富陽入,浦陽江西南自諸暨入,合於漁浦街。古時浦陽與浙江閡,後開磧堰始通。抵中小亹,出南大亹入海。海潮自鱉子亹入,為龕、赭所束,洪濤奔突,捍以危堤二十餘里。西小江,古潘水,出臨浦巿山,歷麻溪壩,貫運河,入山陰,下至三江口入海。運河自西興渡引浙江水,逕望湖橋,湘湖匯西南諸山水貫之,又東南入山陰。臨浦鎮,縣丞駐。有漁浦、河莊山二巡司,義橋鎮汛。西興水驛有丞。錢清課場。有西陵、漁臨兩關。北祇菴砲臺。諸暨簡。府西南百十里。東:紫薇、鐵崖山。西:洞巖、雞冠、五洩山。北:銀冶、杭烏山。浦陽江南自浦江入,一名上西江,合酥溪,東北流,合上瀨溪,與上東江會。江出東陽界東白山,曰孝義溪,合開化溪,流為洪浦江,合下瀨溪注之,是為浣江。逕城東,歧為二,東曰下東江,合楓橋港諸溪,西曰下西江,合五洩諸溪,分而復合,亦曰大江,並入蕭山。有楓橋鎮。長清關。餘姚疲,繁,難。府東北百十里。南:大吳山。西:龍泉山,古緒山。北:歷山。東北:四明、石匱山。海,北自上虞入,逕臨山衛。康熙八年移廟山巡司駐。四十七年移瀝海守備並駐。北臨山港,東泗門港,為濱海要口。逕破山浦,有防海塘、利濟塘。外砲臺七所。其西南姚江,出太平山及菁山,古句章渠水,錯出復入,納上虞馬渚橫河,貫兩城間,抵竹山潭,合蘭塑港,逾姜家渡,納慈谿官船浦,是為丈亭江。鎮四:梁壽、眉山、廟山,其三山有巡司。石堰、鳴鶴二鹽場。中村、北溪、梁衕、周巷、周家路有汛。姚江驛,康熙九年並入縣。上虞繁,疲。府東百二十里。南:覆卮山。西南:象田山。西北:夏蓋山,南臨夏蓋湖,匯白馬、上妃二湖水,周百五里。北枕海,西北自會稽入,逕瀝海所,有四衛、施湖二隘。其塘外為沙塗。上虞江即曹娥江,古柯水,亦曰東小江,上流剡溪,西南自嵊入,納會稽小舜江,逕梁湖堰,其東為運河。外有通水河,逕百官渡,其東為馬渚橫河,抵備塘。自梁湖堰至此,與會稽為界水,有曹娥驛,康熙元年裁丞。其側鹽場二。梁湖鎮巡司。嵊沖,繁。府東南百八十里。東:金庭山。西南:五龍、真如二山。分水岡,剡溪出,合大小白山水,東南流,右合珠溪,左合籮松溪,逕白楊村,納富潤、江田二溪,側城東南,有潭遏溪、寶溪注之,是為剡溪。黃澤溪亦自其縣入,合北莊溪注之。又西北,合丫溪、強口溪、嵊溪,入上虞為曹娥江,即古浦陽江也。東北西梅溪,出大屏山,入奉化。鎮三:浦口、長樂、三界。有汛。新昌簡。府東南二百十里。東:天姥山:東南:關嶺。東北:蘇木嶺。東港溪自天臺入,合洩上山溪、潛溪,下流為潭遏溪。西港溪上流夾溪,西南自東陽入,合三洲譚溪,下流為寶溪。北港溪出奉化界蔡嶴山,歷巖頭嶺,別源自寧海緣界合為黃澤溪。彩煙鎮。黃渡有汛。

臺州府:疲,難。隸寧紹臺道。海門鎮總兵駐。原名黃巖鎮,總兵駐黃巖。光緒二年移此。西北距省治五百九十里。廣三百七十里,袤二百七十里。北極高二十八度五十三分。京師偏東四度三十九分。領縣六。臨海繁,疲,難。倚。西南:括蒼山。東:了倭山。南:蓋竹山,道書「十九洞天第二福地」也。東蔡嶺、西石松山,並築石城。海,東自寧海入,逕坡壩江,白岱山扼其口。納上流花橋港,南逕泗淋汛,納上流洞港,又南逕有殿角,南對白梳角,曰清塘門。桃渚港出東大羅山,合矩溪入之。其★C9順治十八年廢,康熙十一年復,守備駐。自白梳角迤南至白沙山,為臺州灣口北岸,迤南至前所城,游擊及巡司駐。南對海門,順治十七年裁衛,總兵駐。其西家子鎮,同知駐,是為椒江口。口外群島聯群,迤南為竇門山、麂青山。臨海江二源,北為始豐溪,自天臺入,合大石溪、歸溪為百步溪,出三江村,西南永安溪,合黃沙溪、芳溪來會,是為靈江。逕雙港口,合大田港,逕三江口,會永寧江,是為椒江。又東,合章安、東邏二浦入海。鎮二:蛟湖,其花橋縣丞駐。有杜瀆鹽場司。赤城驛。牛頭頸、外沙、小圓山砲臺。黃巖疲,繁,難。府東南六十里。東:永寧山。南:委羽山,道書「第二洞天」。西:黃巖山,縣以此名。海,東自臨海入,逕浪磯山,為臺州灣口南岸,有丁進、洪輔兩塘,長六十餘里,內為鹽地,北接臨海,南亙太平。西北:永寧江出西塵山,別源出黃巖山,合流為大橫溪,逕大砩頭為寧溪。逕烏巖為烏巖溪,合柔極、小坑二港為長潭。又東南合官嶴水、茅畬溪,逕山頭洲為斷江。逕後垟,合西江,是為澄江。逕東浦,外東浦即東官河,側城東北,合里東浦,為黃林港,下流為永寧江。南官河匯沙埠、九峰諸水,南支接太平金清港,北支貫城壕,左出枝津,分流復合,入西江。又北流為里東浦。烏巖鎮,縣丞駐。有長浦巡司。黃巖鹽場司。丹霞驛。烏巖三港口、沙埠、寧溪、洋嶼、白湖塘有汛。天臺簡。府西北九十里。北:天臺山,周八百里,支阜赤城,有玉京洞,道書「第六洞天」。始豐溪西南自東陽入,合寒、明二巖,鷓鴣諸山水,側城西,左得青溪,合桃源瀑布、關嶺溪壑注之,東抵鳳凰山下,合寶華及螺歡、倒靈諸溪,折南又合大小淡溪。西北:福溪出天臺山西麓,混水溪出南麓,其東泳溪出蒼山東北麓。又界溪出龍鳴山。有清溪鎮。桑洲驛。仙居疲,難。府西九十里。西南:韋羌山、景星巖。北:羅城巖。西北:蒼嶺,一名風門。西南:大溪,南源出永嘉界坑山,曰永安溪,左出枝津抵安仁嶺下,曰安仁溪,入縉雲。西源自縉雲入,曰金坑水,合仙人溪,逕四都,與南源會。逕洋山潭,合里溪。又東北,右得馬嶺溪、大陳山水,左得珠母、韋羌、南溪諸水。又東北,納萍溪。上流櫸溪,西北自永康入合之,逕城東,合白水溪、彭溪,抵塔山西,合硃溪,入臨海。有皤灘鎮汛。寧海簡。府東北百有八里。水師參將駐。西北:龍須山。北:天門山。海,東北自奉化入,逕浮溪口,納上流鐵江,轉東逕黃墩港,納上流白渚溪,錯出為象山港南岸。又自石浦入,逕臺寧嶼,又西逕茶院港,分出東北許家、雙坑二山,合流為柞浦溪,逕龍口塘。又西逕白嶠港,上流白溪,西南自天臺入,別源出西桃花山,合逕亭頭渡來會,西南逕清溪口,上承天臺泳溪,逕旂門渡。又南為亭旁、海游二溪口,一承天臺界溪,一出西南分水嶺,合逕連蛇渡。又西南逕健跳所,守備駐,臨健跳江。上流橫渡溪,合小白溪來會。外有健陽塘,東北對石浦城,是為寧海灣口門。群島錯峙,其著者為田灣島,島東為青門山,臨牛頭洋,北為五嶼門,外硃門洋,內蛇蟠洋,並險汛。鎮四:海嶴、越溪、亭旁,其海游,縣丞駐。有長亭鹽場司。硃嶴驛。太平簡。府東南百四十里。水師參將駐。南:石盤山。西南:靈山。西:溫嶺。海,東南自臺州灣入,逕金星門,又南道士冠山、盤馬山。其東白巖山,中為搗臼門。又東沙鑊山,東南積穀山,東北即臺州群島。其著者上下大陳山,轉南逕松門城,置守備,臨松門港,松門山扼其口,中有窄水道。其東牛山島,又東蘇丹島,東南三蒜島。轉西逕隘頑寨,群峰刺天,慢游嶺尤嶮仄。中有大海灣,錯出溫州府境,逕天澳、木杓諸山,轉西北逕楚門城。西北:白漿渚溪,正源大溪與別源小溪合流為雙溪,北流折東為新建河,合桃溪、溫嶺溪,逕大溪口。西南西溪出梅嶺來會,是為金清港,北通黃巖官河。又東逕新河城為迂江,縣丞駐,又東入海。閭溪一名練溪,並入之。有蒲岐、溫嶺二鎮。松門巡司。鳳尾、盤馬、沙角、寺前鎮、石塘、金清、箬裏有汛。

金華府:沖,繁,難。隸金衢嚴道。副將駐。明初為寧越府,後復改。東北距省治四百五十里。廣三百四十里,袤二百四十四里。北極高二十九度十分。京師偏東二度二十一分。領縣八。金華沖,繁。倚。北:金華山,縣以此名,古曰常山。東南:至道山,康熙二年,耿逆遣兵踞此。東陽江自義烏入,曰東港,合航慈溪,東南流,納孝順溪、薌溪、赤松溪,逕城南,合城中七寶渠。南港自武義入,西北流,抵燕尾洲,與東港會,是為婺港,一名雙溪。又西北,桐溪、白沙溪並自湯溪入,抵柵頭,有盤溪承徐公湖、九龍山水,流為黃煙溪注之,此北渠也。古時南渠與衢港會,今淤狹。有孝順鎮。雙溪驛。蘭谿沖,繁,難。府西北五十里。東:銅山。西:硯山,界龍游、壽昌。婺港東南自金華入,合黃煙溪,逕城西南。衢港自龍游入,納壽昌游埠溪,錯出復入,左得永昌溪,逕蘭陰山下,會楊子港,是為蘭港。又北合虎溪、乾溪、香溪、抵施家灘,納浦江大梅溪。有黃湓堤,康熙、雍正間屢修築。鎮三:平渡、香溪,又女埠有廢司。瀔水驛。東陽繁,難。府東北百里。東南:大盆山,界天臺。東:玉山,一名封山。東北有東西白山,接太白山。東陽江二源:南源定安溪,即歌溪,出大盆墨嶺,合金蒙坑、茅洋諸水,逕雙溪口;北源上白溪,出東白山,會西白山水,南流,合白峰溪、渼沙溪來會,西流,右得筧竹溪、蟠溪,左得龍化溪、泗渡溪,又西合雅溪、郎坑溪。畫溪出大盆西麓,始豐溪出南麓,其東上夾溪出尖山巿,下夾溪出天笠山。有白坦、永寧二鎮。白峰、夾溪汛。義烏疲,難。府東北百十里。南:淡雲、八保。北:黃山。東江,古烏傷溪,自東陽入,合廿三里諸溪,折西南會瑞雲溪、麟溪。又西南,右合繡湖,左合占溪及善溪,逕江灣巿,會畫溪,又南納吳溪,入金華。其北航慈溪,出覆釜山,會仙洪巖水,緣西北界入。素溪出西南古寺坑。北酥溪出清潭山。又洪巡溪出西北綢巖。鎮四:龍祈、酥溪、佛堂、廿三里溪。有汛。永康疲,難。府東南百十里。東:方巖。東南:靈巖。南:絕塵山。永康港二源:北源華溪,出密浦山,逕社山下為鶴鳴溪,合酥溪,出仁政橋;南源南溪,即建陽溪,自縉雲入,右合盧溪,左合橫坑溪,逕水崢巖,右合李溪,逕雙溪口,兩源相合,是為永康港。又西,合西門、烈橋、高坑諸溪,入武義。東北:雙牌溪,出八盆嶺,下流為靈溪,入縉雲。又東櫸溪,出大嶺,下流為萍溪,入仙居。櫸溪村,府都司及縣丞駐。有孝義、里溪汛。華溪驛。武義疲,難。府東南七十五里。東:百義山,又烏牛山,界永康。西:銅釜山。武義港上承永康港,東自其縣入,合清溪、郭衕溪,側城東北,左得熟溪,西南自宣平入,合諸溪水匯焉。西北流,右合東溪、硃吳溪,左合桃溪,入金華。其南日溪自麗水入,合洩溪,入宣平。西:梅溪自宣平入,入金華。東北:素溪,出大撈箕山,自金華錯入,仍入之。浦江簡。府東北百十里。東:半壁山、五路嶺。西北:深褭山,湧泉為深褭溪,浦陽江源此。別源出西並硎嶺,東流為吳溪來會。又合諸溪水,側城南,有東、西二溪夾流注之,是為南江。合澄、左二溪,又東北,右得大溪,即演溪,東南自義烏入,合流逕康侯山下,為潮溪。又東,右合深溪、白麟諸溪入諸暨。南:梅溪,出雷公、城竇諸山,西流入蘭溪。西北:湖源溪出石楂嶺,逕五洩山,錯出復入浦江,下流為湖洑水。潢溪、胡公、斤竹有汛。湯溪簡。府西南五十里。西北:湯塘山,縣以此名。南:銀嶺。東南:輔倉山,白沙溪出。瀔江即衢港,西自龍游入。古無「瀔」字,當即漢志穀水。東北流,合莘版溪。又東北,左得雙溪,上流永安溪枝津,北自蘭谿入合之。右得潦溪,上流游埠溪,亦自其縣入合之。是為三港口。又東北合羅埠溪,入遂昌。白沙溪,南自遂昌入,合諸溪水。堰三十六,而金華得其十。

衢州府:沖,繁,難。金衢嚴道治所,總兵駐。明洪武初改龍游府,明年復改,屬浙江布政使司。順治八年,浙閩總督移此。康熙二十三年裁。東北距省治五百四十里。廣二百二十五里,袤二百二十里。北極高二十九度二分。京師偏東二度三十五分。領縣五。西安沖,繁。倚。南:爵豆山。北:銅錢嶺。西北:銅山。衢港二源:南源文溪,即江山港,自其縣入;北源信安溪,即常山港,亦自其縣入,會於雙港口,亦曰西溪。側城西北,合柘溪、青岡溪,東抵雞鳴山下。右得東溪,南自遂昌入,出石室堰來會,古曰定陽溪。又東北,合銀坑、羅張、勝塘諸溪,逕屏風灘,合芝溪。又東逕馬葉埠,入龍游。樟樹鎮,縣丞駐。有金旺巡司,巖剝、柏固二廢司。上航驛。上方、新橋街、杜澤、朝京埠有汛。龍游沖,難。府東北七十里。西:龍山,又岑山。北:烏石山、大乘山,八十里梅嶺。北有龍游港,即衢港,西自西安入,合金村源水,逕下溪灘,右得靈山港,南自遂昌入,合桐溪、小蓮嶺水來會。又東,右得鬥潭溪,北自壽昌入合之。又東合築溪,錯出復入者再,又東北入湯溪。有湖頭鎮巡司。亭步驛。溪口前巿汛。江山沖,疲。府西南七十五里。仙霞嶺,南百里,上置五關,其楓嶺為浙、閩分疆處,順治十一年置游擊駐二十八都,縣丞並駐。康熙九年並入福建。十三年仍隸兩省。又江郎山,即隋書江山。大溪一名鹿溪,出仙霞諸嶺,匯東角、箬坑、白石諸水,逕城東,合三橋溪、逸溪,入西安。西有文溪,分出,復匯於禮賢鎮東北,與大溪會,是為江山港。其北石崆溪,出斜馱山。又峴山溪出大寨山東峰。峽石鎮,同知駐。有清湖鎮巡司,兼管廣濟水驛。順治十年自常山來隸。並靈谷山、官溪、外村有汛。常山沖,繁。府西八十里。東有常山,縣以此名。南;峴山、巖嶺。北:三衢山。馬金溪北自開化入,合馬尪溪,逕源口,合謝源水,逕疊石,為金川。逕儻溪橋,合諸山水,逕清水潭,廝以官壩,外紫港,內廣濟港。昔時文溪自江山入,達金川,為三合水,注內港。後湖涸,水道徙南。又東石硿溪、峴山溪並自江山注之。又東合虹橋溪、芳枝溪,入西安。有草坪鎮巡司。球川鎮。馬車曹、會關有汛。有鎮平、甘露二鎮。開化簡。府西北百六十里。東:鴉金嶺,界常山。北:礦山,又馬金、金竹二嶺。馬金溪二源:一出汪公嶺,即馬金支阜;一出西北際嶺,會於辛田渡,東南流,合金竹嶺水,側城東南,左得汪邊溪,出北謳歌嶺,貫城壕,出南門合之,西會白沙溪達華埠。左得池淮溪,逕藤巖下,曰池淮阪。逕星口巿,為星口溪,合流入常山。是為常山港。西北:洪源溪,入江西德興。鎮二:馬金、華埠。金竹嶺巡司。

嚴州府:簡。隸金衢嚴道。副將駐。乾隆二十五年裁衛,並入杭州。東北距省治二百九十里。廣三百七十里,袤百七十五里。北極高二十九度三十七分。京師偏東三度三分。領縣六。建德簡。倚。東:高峰山。西:銅官山。北:烏龍山。新安江西自淳安入,右納艾溪,東北流,合洋溪、下涯溪、西溪,側城南。蘭港東南自蘭谿入,合三合溪、大小洋水來會,是為浙江南源,一名丁字水。又東北合佘浦、苔溪,逕七里瀧,左合胥溪,又東北合岔柏溪。清渚港東北自桐廬錯入,會杜息溪,並入桐廬注之。東湖出烏龍山,合建安山水,由佘浦出口。康熙十一年築壩,水漲,繞江家塘注西湖入江。有安仁、乾潭、三都、洋溪、大溪五鎮。東烏石關,東南三河關。有富春驛。淳安簡。府西北六十五里。東:龜鶴山。南:雲濛山。西南:雉山。前溪出西北塘塢山。新安江自歙縣入,一名徽港,左得蜀口溪,東南流,合富至源、雲源溪,又東合桐梓溪,折南合景溪。逕南山東麓,左得東溪上流進賢溪,匯諸山水注之,逕城南,合雲濛溪。又東南,右得武強溪。又東南,合商家源、洋溪、錦溪,入建德。羅伍溪出東北白坑嶺,羅溪出東塢山,龍溪出西北官山尖。鎮四:威坪、茶園、街口、港口有汛。桐廬簡。府東北九十五里。東北:桐廬山,縣以此名。西北:雞籠山。西:富春山。新安江西自建德入,為七里瀨,即富春渚,合蘆茨溪,逕麻車山麓,左得清水港,西北自分水入,合琴溪,錯出復入,逕桐君山下,西會分水港,是為桐江。又東曰下淮,江流扼要處。又東北合窄溪,東梓溪。湖源溪東南自浦江入,仍入之。有芝廈、舊縣、柴埠、窄溪、翽岡五鎮。桐江驛。遂安簡。府西南百八十里。西南:洪洞山。西:白石山,又百漈嶺,界安徽休寧。武強溪出,合雙溪、仙溪、華溪,東南流,左合大連嶺水,右合前後溪,側城南,合連溪、靈巖溪。折東,右得龍溪,北自淳安注之。逕寺前村,有鳳林港合東西港注之。又東北,合罟網、東亭二溪,入淳安。有鳳林、橫沿、郭村、安陽、東亭五鎮。壽昌簡。府西南九十里。南:硯山,亙金、衢二郡。西:萬松山。壽昌溪出鵝籠山,合大小源、松坑二溪,東北流,合交溪,為大同溪。又東北,合梅溪、曹溪,自城西而東,曰艾溪,東北逕城南,至淤堨,為淤堨溪。又東北至湖岑阪,為湖岑溪,北抵羅桐埠,入建德。有大同、新巿二鎮。分水簡。府西北百二十三里。東:獅、象二山。南:胥嶺、設峰。西南:雲梯嶺、銅橋山,最險要。天目溪一名分水港,上流虞溪,東北自於潛入,逕印渚渡,為印渚溪。右得前溪,西南自淳安入,合羅伍溪及羅溪,匯於雛溪。東南流,合塘源水、夏塘溪,抵畢浦,左合文嶺、良梅諸山水,右合斜尖山水。其南歌舞溪,出歌舞嶺,會直塢、海高塢諸水,下流入建德,為清渚港。有畢浦、百江二鎮。

溫州府:沖,難。溫處道治所。總兵駐。明領縣五。雍正六年,增置玉環。西北距省治八百九十里。廣百六十里,袤五百里。北極高二十八度。京師偏東四度二十一分。領一,縣五。永嘉沖,繁。倚。城有九斗山,內華蓋,道書「第十八洞天」。西北:大若巖,即赤水山,「第十二福地」。東南:大羅山。南:吹臺山。西:甌浦山。北:孤嶼山,橫亙江中,英領事署在焉。海,東自樂清入,為甌江口,南逕龍灣陡門,又南逕寧村所。康熙九年改寨置游擊。海口曰溫州灣,靈昆島扼之。甌江上流大溪,西南自青田入,東流,合菰溪及韓埠、上戍二港。側城西北。右得會昌湖,分出郭、瞿、雄三溪,合流逕望江門外。光緒二年煙臺之約,立租界。逕陡門橋北,右合塘河,抵永樂界,為館頭江。其右合雙井、茶山二河,又東南合瑤溪、白水溪入海。枬溪鎮,縣丞駐。有西溪巡司,永嘉鹽場司,窯嶴鎮兼驛。沙頭、碧蓮、韓埠、楓林、雙溪有汛。有龍灣山、茅竹嶺、狀元橋砲臺。瑞安沖,繁。府南八十里。水師副將駐。東:龍山。北:集雲山、大小二洋山。海,東北自永嘉入,逕梅頭城,又南逕海安所,又南逕飛雲江口,有關。其外洋鳳凰山與西江橫山對峙,曰鳳凰門。迤北大小丁山。迤東南齒頭山。東:長帶山。迤北南策山,與東策、北策相望。北策以西,與永嘉大瞿以東,稱佳澳焉。飛雲江上流大溪西自泰順入,合桂溪,逕山口村,右合洄溪。又東合九溪、方坑溪。又東北,左得漈門溪,匯諸溪水,折東南,合半溪,左納南岸塘河入海。有大山、江岸二巡司,雙穗鹽場。黑城、宋埠有汛。樂清沖,繁,難。府東北八十里。水師副將駐。北:雁蕩山。東:窯嶴山。西:章嶴山,與沙角、黃華並置寨。黃華有關,追臨海口,為第一門戶。海,東北自太平入,逕大荊城,游擊駐。轉西南逕鏵鍬埠,北接大嵩汛。又西南逕蒲岐,至城南,為甌江口。自木杓山至此,曰樂清灣。東北:新巿河,東源出白龍山,西源出玳球、赤巖、硐坪諸山,合流為黃雙塘溪。北,梅溪出左原諸山,流為石埭河,並匯萬橋港入海。東:芙蓉川,分出長蛇嶺及西中奧四十九盤嶺,合流為清江,北接蔡嶴汛,南接光巖汛。又白溪出雁蕩東麓,逕靈巖,流為凈名溪。東北:蒲溪,二源,一出石門潭,合南閤、北閤諸水,一出荊潭,合桐垟隘、門嶺諸水,匯於水漲,並入海。城河即東溪,出縣治東北諸山。左出諸枝河,並通西城河。河即西溪,逕下馬橋,與東溪合,是為運河,西南入館頭江。東逕磐石,都司及巡司駐。南接天妃汛,與龍灣對峙,為第二門戶。又東至白沙嶺入海。自此逕曹田汛,抵歧頭山,即海口也。有館頭鎮,嶺店驛。縣丞駐大荊城。有長林鹽場。東門、西山嶺、鎮甌、歧頭砲臺。平陽沖,繁。府南百三十里。水師副將駐。西:雁蕩山,對樂清曰南雁蕩。其東焦溪、天井洋、赤巖諸山。西南:分水嶺,泉出瀧上,東西分流,以限閩、浙。海,東北自瑞安入,逕沙園城,南逕鰲江口,又南逕金鄉營,東北接舥艚汛。又南為大濩海口,官山島扼之,分南北水道。西南:鰲江,古曰始陽江。南港二源,一平水,一燥溪,歧為東西溪。北港二源,一順溪,一梅溪,兩港會於蕭家渡西,合逕羅源山下為橫陽江。逕錢倉鎮為錢倉江。又東合東塘河,抵墨城汛入海。城河分出西南毗巖嶺諸山,入城為腰帶水,匯於抗雲橋,一出東郭入海,一出北郭為城北運河。其夾嶼橋河在南夾嶼下,下匯城南諸水,歧為二,一西塘河,一東塘河,分趨入海。南運河出東南金獅山,合直浹河,赤溪出西南礬山,並入之。二鎮:仙口、錢倉。蒲門巡司。天富鹽場。下垟諸砲臺。泰順簡。府西南百三十里。東:飛龍山。南:石嶺。西:雙港嶺。仙居溪出西北諸山,逕洪口渡,洪溪會葛溪注之,古曰漁溪。東北流,左得三插溪,東北自景寧入,會左溪。又東北,右合莒岡溪,左納青田下窄口溪,古曰龍溪。其北太平溪,出上莊,貫城壕,出南門,合白溪,錯出復入,緣界抵赤水坑口,會雙港溪。溪自壽寧入,逕五步,合棠坪水,逕石竹洲,合周邊諸水,抵交溪村,會四溪及仕陽、龜伏二溪,與福建霞浦、福安為界水。有甌西第一、分水、桂峰、武嶺頭鎮南諸關,排嶺、牛頭上下排、龍巖嶺、分水排諸隘。池村、三魁二鎮。有巡司。墩頭隘、吳家墩、洋岡、後街有汛。玉環簡。府東北二百里。參將及同知駐。坎門、釣艚隩勢險要。釣艚東即鷹捕隩。北車首頭與東北木杓山斜對,中為棧頭港,東通靈門港,外列虎叉、雞冠、羊嶼諸山。東南至鹿門。外洋虎叉以東為披山。外洋以北為白馬嘴,嘴東有沙角、鐙臺、茅草諸山。西有花巖浦。西北接後交汛。進此曰漩門,兩山崟束,一水中流,航路最險。其西為分水山,北為苔山。分水以東為楚門港,以南為烏洋港,西接浦歧港。又南為西青嶼、烏巖。北為大青、小青。迤西為茅峴山。又西江廷山。其南大烏、小烏。又南為蓮嶼。西南為大門、小門。迤東南為黃大嶴,中有重山。黃大嶴之西,重山之北,中曰天門。又東南為狀元隩,為三盤山。東北為鹿棲山。西北至大巖頭。又北接梁灣汛,東南即黃門。門東為南排山。北教場隩、里隩。巾嶴寨鎮,玉環巡司,蛇嶼砲臺。

處州府:簡。隸溫處道。總兵及衛守備駐。北距省治一千九十八里。廣四百九十里,袤四百十里。北極高二十八度二十五分。京師偏東三度二十五分。領縣十。麗水簡。倚。都司駐。道光二十八年改守備。東:銀場山、楊梅岡。北:麗陽山,縣以此名。大溪,西南自雲和入,左得松陰溪,西自松陽來會,為大港頭。又東合松阮水,為郭溪。又東合通濟渠,折北,左得畎溪。西北自宣平入,合西岸溪來會。西北稽勾溪,納宣平小溪,是為三港口。逕溪口,合麗陽水,環城南為洄溪。又東,左得好溪,東北自縉雲入,合嚴溪注之。鎮二:寶定,其碧湖,縣丞駐。十八都、蓬蒿嶺、皁阮、庫頭、卻金館、沙溪有汛。有保定鎮。括蒼驛。青田簡。府東南百五十里。北:青田山,縣以此名,一曰大鶴山,道書「第三十六洞天」。西:石門山,「第三十七洞天」。南:方山。大溪出西南龍須山,上承洄溪,西北自麗水入,合海溪、芝溪、中坑、石藤諸水。又東南,右納小溪,上承山溪,自景寧入,流為浣紗溪,復流為雙溪來會。左合石溪,側城西南。折東合顧溪,入永嘉為甌江。西南:浯溪出蒲斜嶺,下流為下窄口溪。又南田坑水出天馬山,入瑞安,下流為泗溪。大溪七十二灘,在青田者都三十有三。黃亶鎮,縣丞駐。有芝田驛,黃亶、淡洋二廢巡司。縉云簡。府東北九十里。東:括蒼山。西南:馮公嶺,古桃枝嶺,上有桃花隘。好溪出東大盤山,逕大皿為九曲溪,合黃檀、潤川二水,逕冷水三港口,合虯里溪,右得靈溪,北自永康入合之。又西南,合棠、赤二溪,出賢母橋,合管溪,逕岱石,會訪溪,逕城南。左合荊坑水,右合貞溪,為南港溪。又西入麗水,下達溫州入海。其北南港溪,出雪峰山,合建洋溪,逕縣北,匯梅、龜二溪。又北為黃碧溪,入永康,下會蘭溪入浙江。西南:巖溪,出紗帽嶺,合芳溪,亦入麗水。又龍溪,出分水,仰納仙居安仙溪,下流為金坑水。有丹峰驛。松陽簡。府西北百二十里。西南:箬簝山。南:白峰、尖山。北:竹客嶺,勢險仄。松陰溪西北自遂昌入,合東湖山水,逕卯酉山下,合霏溪、侖溪,逕青龍堰,右得大竹溪,分出西南香乳、玉巖二山,會為夏川,合南岱、亞岱水為中隩川,又合小竹溪,三臺水來會。又東南,右得竹客溪,北支入宣平,南支與松陰合,逕城西合循居溪。又東南,合蛤湖、石倉源,會裕溪、小槎溪入麗水。東白岸溪,出桐鄉山。有舊治鎮。乾隆二十八年移汛於此。遂昌簡。府西北百八十里。西南:君子山。東:尹公山。西:奕山、湖山。北:兌谷山、金石巖。南:貴義嶺,前溪出,南支入龍泉,北支逕城南,出東關橋,會後溪上流柘溪,東流為好川,匯梅山二水,是為雙溪。又東為航川。其西蔡溪上流住溪,西南自龍泉入,合碧隴源,出宏濟橋,合關川為鐘溪。逕周公村,左得東川,上承福建浦城罟網水入之。抵龍鼻頭隘,右出枝津為梭溪,即柘溪上流。正渠入西安,為烏溪港。北官溪出侵雲嶺,右得馬戍源,出湯溪界銀嶺,入為桃溪,合白水源,下流為靈山港。有高平、關堂二隘。龍泉疲,難。府西南二百四十里。南:豫章山。又琉華山,下有琉田,土宜陶,有烏赩窯、青赩窯,今曰龍泉窯。北:黃鶴嶺。大溪西南自慶元入,曰秦溪,合小梅溪,錯慶元復入,會山溪。逕查田巿,又東北至獨田灘,蔣溪合桑溪來會。合豫章川,逕南大垟村,瀦為劍池湖。出披雲橋,合錦川。逕城南,中阻槎洲,分流復合。右合大貴溪,左合鐵杓溪,東北流。右得白雁溪,上流前溪,西北自遂昌入合之。又東北合道太、安仁二溪,錯出復入者再。其西南住溪,自福建浦城入,東北入遂昌。下流為蔡溪,碧隴源亦自其縣入。一溪出南九漈山,東南入景寧。安仁莊,縣丞駐。查田廢司。五都、洋村有汛。慶元簡。府西南四百里。南:赤摶嶺。東:九臺山。西北:蠻頭山,秦溪出,下流為大溪。山溪出東源頭山,小梅溪出北大拗嶺,並入龍泉。東北鈐高山,南洋溪出。東南雞冠山,魚頭溪出,並入景寧。其東光石山,蓋竹溪出,合濛洲溪、交劍水,逕城北為大溪。左合竹坑溪,右合焦坑溪,逕八都鎮。右得蕓洲溪,西南自福建政和入合之,是為槎溪,循棘蘭西入福建松溪。其西北竹口溪,出雷風山,合下水祭、新窯二溪並入之。舉溪出東南棠廕山,入政和。有汛。伏石、大澤二關。雲和簡。府西南百二十里。東南:白龍山。北:牛頭山。南:前溪山,兩山竦峙,勢險要。西:巖山。北:石鏡巖,大溪逕其南,自龍泉入,錯出復入,右合烏椹源,左合麻、梅二洋,復錯出,自洽川口入,合洽川及硃坑、烏龍坑水,折東南,會浮雲溪。溪出西南黃棧坑,合硃源水,出利濟橋,有霧溪合新溪注之。環城而東,右合黃溪,左合雙溪,為戈溪。又合諸坑水,為規溪,入麗水。豐源水出西南豐源山、入景寧。宣平簡。府西北百二十里。東:岱石山。南:俞高山。西:竹喀嶺。雙溪二源:東源分出龍樊嶺,上坦、小妃匯岡山下,曰東溪,亦曰午溪;西源出礱坑山,合新錦溪,曰西溪,亦曰申溪。兩源匯於綠巖潭。東南流,納松陽竹客溪,又東南,右納歐澗水,上流日溪,東自武義入合之。左合石浦水,又東南,右納松陽白石溪,入麗水為畎溪。北:梅溪出黃塘山,東流入武義。汛五:曰竹客、玉巖山、陶村、和尚田、式河頭。景寧簡。府南百四十里。東:羅岱山。西南:豸山。北:莘田嶺。山溪上流南洋溪,西自慶元入,左得英川,即定度溪,上流一溪西自龍泉來會,東南流,合標溪,逕新亭村,右得豐源水,西北自雲和來會,折東北,右得鶴沐溪,會塵溪入焉。又合大小順坑水,入青田為小溪。南白鶴溪,出梨樹嶺,下流為三插溪。有龍首、龍匯、白鹿諸關。


\end{pinyinscope}