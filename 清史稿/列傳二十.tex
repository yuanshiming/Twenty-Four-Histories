\article{列傳二十}

\begin{pinyinscope}
圖爾格兄徹爾格伊爾登弟超哈爾超哈爾子額赫里巴奇蘭岱松阿

岱松阿子阿納海巴漢齊爾格申巴都里巴都裏從弟子海都托克雅

葉臣子車爾布蘇魯邁蘇魯邁子蘇爾濟鄂洛順翁鄂洛

珠瑪喇瓦爾喀珠瑪喇瓦爾喀珠瑪喇弟伊瑪喇

圖爾格,滿洲鑲白旗人,額亦都第八子也。少從太祖征伐,積功授世職參將。尚和碩公主。太宗即位,八旗各設大臣二,備調遣,亦號「十六大臣」,以圖爾格佐鑲白旗。尋遷本旗固山額真,列八大臣。天聰元年,上伐明,圖爾格率所部從攻錦州,不克,隳大小凌河二城而還。二年,追錄其父額亦都功,進世職總兵官。

三年,從伐明,克遵化。四年,上還師,命貝勒阿敏護諸將屯永平,而圖爾格與正黃旗固山額真納穆泰,正紅旗調遣大臣湯古岱,榜式庫爾纏、高鴻中守水欒州。明監軍道張春,總兵官祖大壽、馬世龍、楊紹基等,合軍來攻,圖爾格與納穆泰、湯古岱分地設汛以守。明兵攻納穆泰急,圖爾格分兵授裨將阿玉什使赴援。明兵舉火,火將及城樓,有執纛者乘雲梯以登,阿玉什揮刀斬之,奪其纛,明兵稍卻。阿敏聞明兵攻灤州,遣巴都禮以數百人赴之,夜三鼓,突圍入,明兵發巨砲,城圮,城樓焚。圖爾格等守四日,度不能御,率所部夜棄城,為散隊潰圍出。會雨,明兵截擊,死者四百餘人。至永平,阿敏遂盡棄諸城,引師出塞,令圖爾格為殿。師還,命收諸將議罪,上詰責圖爾格、納穆泰等,湯古岱因引罪請死。上曰:「汝等不能全師歸,陷於彼為敵所殺,歸至此朕又殺之,於朕復何益?且汝等既攜俘獲人畜而還,何不收我士卒與之俱來?彼等何辜,忍令其呼天搶地以死也!」圖爾格坐削總兵官,解固山額真。

五年,初設六部,起圖爾格為吏部承政。上自將伐明,攻大凌河,督諸軍合圍,令圖爾格從正白旗固山額真喀克篤禮當城東迤北。城兵出攻城南砲臺,圖爾格不及騎,徒步擊走之。略松山,大凌河旋下。八年,與固山額真譚泰帥師略錦州。上自將伐明,命貝勒濟爾哈朗留守,使圖爾格帥師屯張古臺河口,防敵自沿海至。既,又使與梅勒額真勞薩帥師出邊,渡遼河,循張古臺河駐軍,衛蒙古諸部。

是時察哈爾部林丹汗死,其子額哲不能馭其眾,諸宰桑皆來降。九年,命貝勒多爾袞等為帥,納穆泰將右翼,圖爾格將左翼,徇察哈爾,至其庭,額哲遂降。師還,略明邊,自平魯衛入塞,躪代州,乘勝至忻口,遇伏,敗之,逐北至崞縣,殲明兵。還過平魯衛,明兵邀於途,圖爾格戰,陷陣,得數百級,明兵引入城,不敢出。圖爾格度追師且至,設伏以待,與納穆泰將千人為殿。明將祖大壽等以三千人赴戰,圖爾格返兵步戰,力沖其中堅,伏起夾擊,明兵大奔,乃徐引兵出塞。十年,敘功,授世職一等梅勒章京。

崇德元年,復授鑲白旗固山額真。從武英郡王阿濟格伐明,圖爾格率所部自坤都入邊,會於延慶,遂深入,克十六城。攻昌平,下雄縣,圖爾格皆先登。旋坐女為貝勒尼堪福晉詐取僕女為女,事發,貸死奪官。八月,復命攝固山額真。四年,上命睿親王多爾袞為奉命大將軍,率師伐明,圖爾格從,擊破明太監馮永盛、總兵侯世祿軍。復與固山額真拜音圖敗明兵於董家口,毀邊墻,奪青山關入,下四城。

五年,從多爾袞帥師攻錦州,取其禾,屢擊敗明兵。又與固山額真葉克書將三百人伏烏忻河口,伺城兵出牧,驅牲畜以歸。明兵千餘人逐戰,葉克書馬中矢蹶,敵將兵焉,圖爾格射敵殪,翼葉克書上馬,並力擊敵,敵敗去復至,凡六合,圖爾格身中二十餘創,猶殿後力戰,護所俘還。敘功,復進世職三等昂邦章京。尋授內大臣。六年,太宗自將伐明,困洪承疇松山,圖爾格從。明總兵曹變蛟夜犯御營,兵至倉卒,守營大臣侍衛皆未集,圖爾格首發矢殪二人,與弟伊爾登、宗室錫翰督親軍攢射,變蛟中創敗去。復從諸貝勒邀擊明敗兵,戰於塔山,為伏於高橋,殺敵無算。

七年十月,上命饒餘貝勒阿巴泰為奉命大將軍,以圖爾格副之,帥師伐明。左翼道界嶺口,右翼破石城、雁門二關,並深入,越明都,自畿南徇山東,南極兗州,克府三、州十八、縣六十七,獲明魯王以派及樂陵、陽信、東原、安丘、滋陽五郡王,他宗室官屬千餘人。遇敵三十九戰皆勝,俘三十六萬九千、駝馬騾驢牛羊五十五萬一千三百有奇,得黃金萬二千、白金二百二十萬有奇,珠緞衣裘稱是。八年六月,師還,賜白金千五百。世祖即位,敘功,進三等公。順治二年二月,卒。九年,謚忠義。配享太廟,立碑墓道。雍正九年,定封三等果毅公,世襲。

子武爾格,從征皮島,戰死;科布梭,襲三等昂邦章京。貝子屯齊等訐鄭親王濟爾哈朗諸罪狀,因及太宗崩時圖爾格等共謁肅親王豪格,將奉以嗣位,而以上為太子。王大臣議追奪圖爾格公爵,命但削科布梭世職。科布梭亦訐其父當太宗崩時,以與白旗諸王有隙,命三牛錄護軍具甲胄弓矢衛其門,其祖母,其父,及其從父遏必隆;又嘗叱辱格格,格格,遏必隆妻也。語詳遏必隆傳。順治八年,上親政,命科布梭襲三等公,恩詔進二等。九年,追論科布梭妄訐其父,削爵。遏必隆兼襲進一等公,自有傳。

兄徹爾格,隸滿洲鑲黃旗。幼事太祖,從征伐有勞,授備御,進游擊。天命十年四月,上命王善、達珠瑚及徹爾格率千五百人伐瓦爾喀部,王善,上族弟也。師大捷,多所俘獲。及還,上先五日出郊獵於避廕,四日乃罷獵,至木戶角洛,與師會。王善等入謁,行抱見禮,以酒二百甕並出獵所獲獸百餘饗從征士卒,並及降人。還至沈陽北岡,復以酒四百甕、牛羊四十,列四百筵為大宴。既入城,又賜從征者人白金五兩。尋進徹爾格三等總兵官。

太宗即位,設八大臣,徹爾格領鑲白旗。天聰元年正月,從貝勒阿敏等伐朝鮮,師還。尋解固山額真授其弟圖爾格。二年五月,從貝勒阿巴泰等伐明,隳錦州、杏山、高橋三城。五年七月,初設六部,授刑部承政。尋遷兵部承政。七年八月,命與刑部承政索海偵明邊,至錦州,斬七級,獲把總一、兵九。十月,明副將尚可喜來降,上命徹爾格偵其蹤跡。八年二月,奏言:「可喜行且至,道遠馬不給,請諸牛錄凡有馬四者,借二以給用。」崇德二年四月,從武英郡王阿濟格等攻皮島,師還,以屢違軍令,削爵罷官。三年七月,更定部院官制,起授工部左參政。五年二月,擢戶部承政。八年,考滿,復授牛錄章京世職。世祖定鼎燕京,加半個前程。順治二年二月,卒。

子陳泰、法固達、拉哈達。陳泰、拉哈達自有傳。法固達襲世職,進三等阿達哈哈番,尋卒。

伊爾登,額亦都第十子,與圖爾格同旗。幼,太祖育之宮中,長授侍衛。屢從征伐,城界凡、薩爾滸,皆有勞,賚蟒服,授世職游擊。累進三等副將。太宗即位,各旗置大臣二備調遣,伊爾登與其兄圖爾格同佐鑲白旗。尋命帥師戍國南界。天聰三年九月,攻麞子島,島故明將毛文龍所轄,文龍為袁崇煥所殺。伊爾登帥師行略地,得舟四,沉之,俘其人以歸。十月,從伐明,攻龍井關,隳其水門入,斬明將易愛、王遵臣,盡殲其眾。攻遵化,敗明山海關援兵,斬其將趙率教,薄明都。四年,克永平、灤州、遵化諸城。師還,進一等副將。圖爾格罷固山額真,以授伊爾登。五年八月,攻大凌河城,伊爾登當城東迤南,深溝堅壘,環而守之,卒以破敵。六年,上自將伐察哈爾,命與貝勒阿巴泰等留守。

七年六月,上以伐明若朝鮮若察哈爾三者何先,諭諸貝勒大臣各陳所見。時上留諸軍駐山海關外屯田,諸貝勒大臣皆請先用兵於明。伊爾登亦言:「與其頓兵關外,不若徑入內地。察諸城孰可攻者,多率步兵具梯牌,乘機摧陷,何堅不克?況蓄銳已久,人有戰心,及時而用之,所謂事半而功倍也。」七月,上命從貝勒岳託、德格類等取旅順,與固山額真葉臣將二千五百人戍焉。八年,上自將伐明,自上方堡入,命伊爾登從貝勒阿濟格、多爾袞、多鐸等帥師自巴顏硃爾格入龍門,與上軍會宣府,擊敗明兵,得馬百餘。攻保安,克之,進拔靈丘。伊爾登忤諸貝勒,又與固山額真貝子篇古等相詆諆,下法司集讞,坐奪世職,★罷固山額真,復授圖爾格,仍罰鍰。尋從豫親王多鐸伐朝鮮,師還,復從武英郡王阿濟格攻皮島,坐先軍纛渡江,復罰鍰。

崇德三年,起授巴牙喇纛章京。四年春,從武英郡王阿濟格伐明,伊爾登以三十人行略地,敗明兵千人,掠其馬。上自將大軍駐錦州。四月,阿濟格以其師會攻松山、杏山,詗知明總兵祖大壽、太監高起潛將二千人出戰,我師為伏以待,敵逡巡不前。伊爾登以四十人紆道致敵,且戰且卻,伏發合擊,大敗明兵。六月,命充議政大臣,兼內大臣。

六年六月,從鄭親王濟爾哈朗圍錦州,明總兵洪承疇以師赴援,屯松山西北。鄭親王令右翼軍擊之,戰不利,退保乳峰山。敵入兩紅旗、兩藍旗駐軍地,固山額真葉臣等斂兵不與爭。伊爾登將多爾機轄與恭順王孔有德及蒙古敖漢、柰曼、察哈爾諸部兵禦敵,躍馬突陣,縱橫馳擊,身被數創不少卻,馬踣,易之,三戰益奮,明兵凡四合圍,卒潰圍出。上嘉其勇,復世職三等梅勒章京,賜白金四百。

八月,上自將禦洪承疇,陳師松山、杏山間,命諸貝勒大臣分道截擊明兵。伊爾登與公塔瞻率巴牙喇兵為伏於高橋,甫出營,遇明兵千人自杏山潛出,擊斬之,遂至高橋;又遇明兵六百餘人自杏山南奔塔山,伏起,明兵熸焉。上移營逼松山,明將曹變蛟夜犯御營,圖爾格率先射敵,伊爾登與內大臣宗室錫翰整兵拒戰,變蛟敗遁。上命侍衛大臣疏防及戰不利者皆罰鍰,賞禦敵將士,伊爾登得優賚。

世祖定燕京,論功,遇恩詔,累進二等伯。順治十三年,以老致仕。上旌伊爾登功,命得乘馬入朝,輒召對賜食。圖其像,一藏內庫,一畀其家。康熙二年,卒,謚忠直。

伊爾登勇冠諸軍,尤長於應變,潛機制敵,諸宿將皆弗能及。子前卒,孫噶都襲,官至鑲黃旗蒙古副都統、領侍衛內大臣。乾隆初,定封一等男。

超哈爾,徹爾格弟,與同旗。幼事太祖,授牛錄額真。天聰八年,予牛錄章京世職。九年,與牛錄額真納海、巴雅、彰屯等齎書詣喜峰口、潘家口、董家口諸處諭明守邊將吏,還遇戍卒邀戰,斬獲百餘人,擢巴牙喇甲喇章京。崇德元年,從武英郡王阿濟格伐明,將入邊,遇邏卒,迎戰,俘二人,獲馬四。薄明都,奪砲以擊敵,殺百餘人。轉戰至盧溝橋,再遇敵,戰皆勝。二年,列議政大臣。三年七月,更定部院官制,授禮部左參政。

九月,從睿親王多爾袞伐明,自青石口入邊,會師涿州。超哈爾率所部攻任丘,穴地隳其城,趨趙北口,明兵毀橋,師不得渡,乃騎出水西襲明兵後,明兵大敗。南略山東,從克濟南。四年春,師還,出邊,超哈爾殿,敗明兵於太平寨。五年,轉兵部右參政。六年,從鄭親王濟爾哈朗伐明,圍錦州,城兵出戰,超哈爾率所部奮擊,逐入郭,力戰沒於陣。太宗深惜之,賜白金六百一十兩,進世職二等甲喇章京。順治間,追謚果壯,立碑紀績。子格黑禮、額赫里。格黑禮襲世職,凡四年而卒。

額赫里以牛錄額真襲世職,尋遷甲喇額真。從鄭親王濟爾哈朗徇湖廣,屢敗明兵。師還,授京城中城理事官。遷都察院理事官。累進二等阿思哈尼哈番。順治九年,命帥師戍江寧。鄭成功侵福建,駐軍海澄。平南將軍金礪請益師進剿,上命額赫裏將千五百人以往,與金礪會師擊成功,大破之,遂攻海澄,復敗成功兵。十二年,擢兵部侍郎。以功進一等阿思哈尼哈番。十六年,成功兵逼江寧,給事中楊雍建劾樞臣失職。明年,甄別部院諸臣,上以額赫里弗任勞怨,解任,降三等阿思哈尼哈番。康熙初,復為兵部侍郎。擢工部尚書。卒。

子英素襲。從征準噶爾有功,進二等阿思哈尼哈番。卒,子郎保仍襲三等阿思哈尼哈番。從大將軍傅爾丹征準噶爾,和通呼爾之敗,郎保殉焉,恤進二等阿思哈尼哈番。

巴奇蘭,納喇氏,世居伊巴丹。旗制定,隸滿洲鑲紅旗。太祖兵初起,巴奇蘭率眾來歸。屢從征伐,沙嶺之役,率五牛錄兵當前鋒,敗敵。天命十一年,從攻寧遠,克覺華島,授游擊。太宗即位,各旗設調遣大臣二,巴奇蘭佐正黃旗。

天聰三年,從伐明,薄明都,駐軍城北,擊敗明總兵滿桂等。七年,從伐明,攻旅順口。巴奇蘭率白奇超哈兵與鑲白旗固山額真薩穆什喀方舟而前,敵負崖,戰甚力,巴奇蘭被數創,冒矢石奮擊,且號於眾曰:「孰能先登,吾昪其功於上前!」於是牛錄額真雍舜、珠瑪喇超距登崖,巴奇蘭督眾兵從之上,敵殊死戰,我軍少卻。巴奇蘭疾呼曰:「敵兵敗矣!」士卒皆踴躍騰藉入,遂克之,進三等副將。

八年五月,太宗自將伐明,貝勒濟爾哈朗留守,巴奇蘭副之。十二月,命偕薩穆什喀分將左右翼兵伐虎爾哈諸部,師行,上諭之曰:「此行道殊遠,慎毋憚勞。得俘,撫以善言,與共甘苦。攜以還,皆可為我用。汝曹當善體朕意。」九年五月,師還,上御殿設宴,親酌金卮勞之,分賚所獲牲畜,命籍降人二千餘戶俾安業,進一等梅勒章京。十年二月,病創潰,卒,贈三等昂邦章京。乾隆初,定封三等子。巴奇蘭伐虎爾哈諸部,牛錄額真岱松阿實從。

岱松阿,佟佳氏,世居雅爾湖。旗制定,隸滿洲正紅旗。初亦逮事太祖。天聰二年,從伐明錦州,下十三站以東二十餘臺。七年,命與甲喇額真英俄爾岱使朝鮮。語見英俄爾岱傳。八年,予牛錄章京世職。及巴奇蘭等師還,有功,加半個前程。崇德元年,從英親王阿濟格伐明,徇昌平。二年,戍海州,擊明兵旅順口,得舟二,俘七人,斬二人,命賚銀布。六年,卒。

阿納海,岱松阿子,襲職,授牛錄額真。順治二年,從擊李自成,逐至富池口,掠其舟。三年,從擊張獻忠。師至西安,叛將賀珍以馬步二千人拒守雞頭關,阿納海與巴牙喇纛章京鰲拜等擊破賊壘,遂徇四川,屢破獻忠兵。五年,授工部理事官,尋兼任甲喇額真。六年,從討叛將姜瓖,攻大同,掘塹環城,城兵出戰,阿納海及固山額真噶達渾與戰屢勝。敘功,並遇恩詔,累進一等阿達哈哈番。十八年,從靖東將軍濟席哈討山東土寇於七,敗其黨喬玉季於連山,賊夜出,阿納海與戰,中創卒,進三等阿思哈尼哈番。

巴漢,亦岱松阿子,襲職。康熙十三年,以參領從副都統碩塔、穆森等討耿精忠。次安慶,聞建德陷,巴漢率兵詗之。至赤頭關,精忠兵出戰,擊之潰,遂導諸軍攻克之。十一月,精忠兵四千餘攻南康,巴漢從碩塔、穆森等擊敗之,斬千餘,盡收其械。十六年,從鎮南將軍莽依圖、江寧將軍額楚等討吳三桂,自廣東徇廣西,破三桂將蔣雄於樹梓墟。十八年,三桂將吳世琮攻南寧,巴漢從莽依圖等赴援,世琮屯新寧州西山下,列鹿角為陣。巴漢與戰,多所俘馘,世琮負傷引去。南寧圍解。二十年,從征南大將軍賴塔進兵,敗三桂將何繼祖等於西隆州,奪石門坎、黃草壩諸隘,遂趨曲靖。會湖南、四川兩路兵,進克雲南。復從都統希福擊三桂將馬寶、巴養元等於楚雄烏木山。二十五年,論功,進一等阿思哈尼哈番。二十九年三月,卒。

齊爾格申,世居寧古塔,以地為氏。兄納林率百餘人歸太祖,太祖命籍其眾為牛錄,以為牛錄額真。旗制定,隸滿洲鑲白旗。納林卒,齊爾格申代為牛錄額真,率所部屯達卜遜木城。明兵攻耀州,齊爾格申赴援,敗之淤泥河,還駐平山。海濱鬻鹽者千人,具舟將出海,齊爾格申夜襲之,千人皆殪。明錦州守者以兵至,齊爾格申與戰,面中槍,戰愈力,明兵敗去。

天聰六年,修蓋州城,移民以實之,命齊爾格申與梅勒額真石國柱、甲喇額真雅什塔等帥師戍焉。八年,授世職牛錄章京。蓋州與明為界,諸新附多亡去歸於明。齊爾格申將兵行海濱,值明兵以舟迎逃人,已入海。齊爾格申涉水追射,殪舟中執槍者及逃人一,遂躍入其舟,獲明備御一、邏卒十有三。又將兵視北新渡口,諜言明兵以舟五十餘泊島中,命為伏以待,明兵二十餘入島伐木,伏發,盡獲之。明兵以舟泛於海,有二人遙呼曰:「我逃人也,誰敢逐我者?」齊爾格申乘小舟逐之,斬一人,俘一人,餘舟明兵皆驚潰。

崇德元年,以齊爾格申出戍能稱職,賜良馬。五月,從武英郡王阿濟格伐明,薄大同,徇延慶,有所俘獲。世祖朝為福陵總管。順治七年,授世職拖沙喇哈番。齊爾格申從弟多尼喀,以攻萊陽先登,賜號「巴圖魯」,授世職牛錄章京,加半個前程,至是卒,以齊爾格申兼襲為一等阿達哈哈番,復以恩詔進三等阿思哈尼哈番。康熙十二年,卒。

巴都里,性佳氏,滿洲鑲藍旗人。父剛格,當太祖時率其族來歸。巴都裏屢從戰伐,授牛錄額真,兼甲喇額真。天聰八年,從伐明,攻大同,與宗室拜音圖為導,未入邊,得察哈爾宰桑四,擢巴牙喇纛章京。崇德元年,從伐朝鮮,巴都里與巴牙喇纛章京鞏阿岱圍南漢山城,屢敗敵。二年,從上獵葉赫,巴都里及哈寧阿所部行列不整,上嚴詰責之。三年,從伐明。明年,從濟南還,師出青山口,明師追至,巴都里率所部還戰,巴牙喇兵有被創墜馬者,令他兵護以歸,棄於道,坐罰鍰。六年,授兵部參政,兼任鑲藍旗滿洲梅勒額真。八年,與梅勒額真鄂羅塞臣伐黑龍江,降圖瑚勒禪諸城。師還,予世職半個前程。遷鑲藍旗滿洲固山額真。卒。

海都,其從弟杭嘉子也,襲職。恩詔,進拜他喇布勒哈番兼拖沙喇哈番。順治間,從擊明將孫可望、李定國、白文選,皆有功。康熙中,署護軍統領。從討吳三桂,卒於軍。敘功,進三等阿達哈哈番。

托克雅,先世居瑚爾哈,以地為氏。兄納罕泰,為瑚爾哈部屯長,天命四年,將其戚屬及所部百餘戶來歸,太祖使迎勞賜宴,賚裘服、奴僕、田宅、器用、牛馬。旗制定,隸滿洲正紅旗。尋授納罕泰扎爾固齊,托克雅牛錄額真。天聰三年,遷巴牙喇甲喇章京。從伐明,入自龍井關,遇明三屯營邏卒,斬五人,獲馬七。護糧以行,明兵來劫,復斬數人,獲纛一。遂與大軍會,從克遵化。五年三月,與甲喇額真榜素等將百人略錦州。八月,圍大凌河城,移屯斷錦州、松山道。明兵自錦州至,擊卻之,逐至城下,俘馘甚眾。八年,從伐明,攻大同,歸還出尚方堡,察哈爾諸宰桑來歸,上命托克雅率師護降人以還。敘功,授甲喇章京世職。九年,戰於寧遠,與阿濟拜等敗敵。語詳阿濟拜傳。崇德三年八月,從貝勒岳託伐明,越明都,趨山東,圍臨邑,托克雅以雲梯攻克其城,賚馬及白金。四年六月,擢正紅旗蒙古梅勒額真。六年,從圍錦州,與明總督洪承疇戰,當敵砲,被數創。七年,解梅勒額真任。順治元年,起為陵寢總管。二年九月,卒,年六十有三。

葉臣,完顏氏,世居兆佳。歸太祖。旗制定,隸滿洲鑲紅旗。天命四年,從伐明,攻鐵嶺,蒙古兵助守拒戰,奮擊破之。六年,復從伐明,克遼陽,以功授游擊。太宗即位,各旗置調遣大臣二,葉臣佐鑲紅旗。

天聰元年,從貝勒阿敏伐朝鮮,以六十人闌入明邊,俘邏卒六。攻義州,與牛錄額真艾博先登,以功授二等參將。率兵戍蒙古,捕斬逋逃,進三等副將。四年,從太宗伐明,攻永平,上命葉臣與副將阿山選部下壯士二十四人,樹雲梯先登。語詳阿山傳。城既克,上嘉嘆,且諭諸將曰:「他日復攻城,毋令先登。驍將,當共惜之!」進三等總兵官,授議政大臣。諭以政有闕失,當盡言,葉臣對曰:「臣受恩重,原罄所知入告,但恐臣識未逮耳。」五年,授鑲紅旗固山額真。從伐明,圍大凌河城,葉臣以所部當城西迤南。城兵出劘我壘,葉臣與額駙和碩圖等督兵夾擊,殲敵過半。

七年六月,上命諸貝勒大臣陳時政,時有議直擊山海關者,葉臣疏言:「今我師方聚,宜先往大同、宣府覘察哈爾蹤跡,近則攻之;若遠,即入明邊,進逼明都。伐木為梯,晝夜環攻,即不遽克,亦足以威敵。」上韙其言。是月,從貝勒岳託、德格類等攻明旅順口,斬獲無算。八年,從貝勒代善自喀拉鄂博入得勝堡,略大同,西至黃河,擊敗明朔州騎兵。崇德元年五月,從武英郡王阿濟格等伐明,既入邊,分兵下安州;又合攻寶坻,穴其城,克之。十二月,從上伐朝鮮,與諸固山額真率阿禮哈超哈兵入其王都。二年四月,從阿濟格攻明皮島,與阿山督白奇超哈兵乘小舟攻島西北隅,麾兵先進,斬明總兵沈世奎,島下。師還,進一等總兵官。四年,從貝勒岳託等伐明,入青山口,略太平寨。岳託令每旗遣梅勒章京一,每牛錄簡甲士三,使葉臣與固山額真譚泰為將,攻克其關,遇敵十三戰皆勝,得馬六十。七年,命代貝勒阿巴泰戍錦州。

順治元年,從入關,率師徇山西。師所經行,自直隸饒陽至河南懷慶,傍近諸府縣悉下,進克太原。先後定府九、州二十七、縣一百四十一,署置官吏,安輯居民。明將李際遇屯河南境,依山為寨。唐通、董學禮降李自成,擁眾擾邊。葉臣皆招使來降,山西底定。師還,至定州,土寇有自號掃地王者,糾徒黨剽掠,葉臣遣兵討平之。比至京,坐擅毀禁垣,屏其功不錄,但賜白金六百。二年,豫親王多鐸定江南。七月,命貝勒勒克德渾為平南大將軍,以葉臣佐之,代多鐸鎮撫;並命大學士洪承疇招撫南方諸行省,敕滿洲諸軍會葉臣調遣,有不順命者,葉臣發兵搜捕,輒奏績。十一月,以自成餘黨一隻虎等出沒武昌、襄陽、荊州諸府,命葉臣從勒克德渾移師剿除。三年十月,師還,賜黃金三十、白金五百。四年,改一等精奇尼哈番。五年,卒。是年七月,定封二等精奇尼哈番,以長子車爾布襲;復兼一拖沙喇哈番,以第五子車赫圖襲。

車爾布初官甲喇額真。崇德六年,從攻錦州,與諸將共為伏,破明兵,擢巴牙喇纛章京。從入關,擊李自成,追及於安肅,復追及於慶都,殲賊甚眾,授世職拜他喇布勒哈番。既,復從英親王阿濟格西討自成,師出塞,道土默特、鄂爾多斯;入塞渡黃河,鑿冰以濟。順治二年春,師至榆林,賊夜襲蒙古軍,車爾布與牛錄額真蘇拜往援,破敵,還軍遇伏,復縱擊卻之,與固山額真伊拜拊循旁近諸府縣。師進圍延安,與梅勒額真羅璧戰敗城兵。自成走湖廣,車爾布與巴牙喇纛章京鰲拜以師從之,攻安陸,得舟八十;復與巴牙喇甲喇章京噶達渾逐賊九宮山,敗其騎兵,自成死。師還,授議政大臣,加一拖沙喇哈番。

三年,從肅親王豪格討張獻忠,屢戰皆捷,與貝勒尼堪等徇遵義、夔州諸府縣。尋以巴牙喇纛章京哈寧阿被圍,車爾布未及援,降拖沙喇哈番,輟其賞及既襲父爵。六年,姜瓖以大同叛,車爾布從英親王阿濟格帥師討之。瓖出攻鑲紅旗營,車爾布率巴牙喇兵御之,瓖敗走。瓖黨自阻馬、得勝二路分兵循北山逼我軍,瓖復以城兵出戰,鰲拜率先當賊,車爾布與梅勒額真譚布合兵繼進,遂殲瓖兵。兩遇恩詔,累進三等伯。十二年十二月,命與寧海大將軍伊爾德率師徇浙江,擊斬明魯王將王長樹、王光祚、沈爾序等。與伊爾德自寧波航定海,分三路進攻,敵萬餘,列舟二百,戰敗;逐之,至衡水洋,斬思、六禦,獲其將林德等百餘人,遂克舟山。語互見伊爾德傳。以功進一等伯,兼拖沙喇哈番。十五年十二月,命與安南將軍明安達理戍貴州。十六年二月,復命移駐荊州。八月,鄭成功入攻江寧,車爾布與明安達理自荊州赴援,循江而下,擊敗成功將楊文英,斬其裨將,獲舟及諸攻具。十七年十一月,師還。十八年,改鑲紅旗蒙古都統。康熙三年,以久疾解都統,降三等伯。七年三月,卒。乾隆十四年,定封號曰威靖。

初,從葉臣攻永平,先登凡二十四人,蘇魯邁其一也。

蘇魯邁,嵩佳氏,滿洲正藍旗人,世居棟鄂部。父遜札哩,歸太祖,太祖錄其長子蘇巴海,授牛錄章京。天命三年,蘇魯邁從伐明,攻撫順,樹雲梯先登。六年,授牛錄額真。復從伐明,取沈陽、遼陽。天聰元年,從阿敏伐朝鮮,攻義州,蘇魯邁以二十人先諸軍登城。三年,從太宗伐明,攻克洪山口城。予世職備御。其從葉臣攻永平也,城上火器發,蘇魯邁面中槍,不退;敵砲裂自焚,冒火援雲梯上,城遂下。上遣醫視創,賜號「巴圖魯」,賚牲畜、布帛,進世職游擊。復從伐明,取旅順,略寧遠,戰必先眾,恆以被創受賞。崇德元年,從武英郡王阿濟格伐明,將入邊,攻雕鶚城,砲傷口,因以殘疾家居。順治間,恩詔,累進三等阿思哈尼哈番。康熙元年十一月,卒,謚勤勇。蘇魯邁子蘇爾濟、遜哈、三塔哈、鄂洛順、翁鄂洛。

蘇爾濟,順治初以噶布什賢轄從入關,與噶布什賢噶喇依昂邦錫特庫擊敗李自成將唐通於一片石。三年,從端重親王博洛徇福建,敗明將姜正希於汀州,予世職拜他喇布勒哈番。七年,卒。

鄂洛順,事聖祖。以二等護衛從建威將軍佛尼埒討吳三桂,敗其將高定;以前鋒統領從裕親王福全擊噶爾丹。有功,累遷江寧將軍。卒。

翁鄂洛,事聖祖。從征南大將軍賚塔討吳世璠,師自廣西入,戰石門坎,敗其將何繼祖;再戰黃草壩,復敗繼祖,獲詹養、王有功。薄雲南,殲胡國柄,逐捕馬寶、巴養元等。以功進三等阿達哈哈番。卒。

珠瑪喇,碧魯氏,世居葉赫。太祖時,率所部虎爾哈人來歸。旗制定,隸滿洲鑲白旗,授牛錄額真。天聰三年,從伐明,次遵化,擊敗明兵。後三日,太宗臨視遵化,明兵自山海關至,將入城,珠瑪喇以邏卒十人御之,所擊殺甚眾。薄明都,遇明總兵滿桂、黑雲龍、麻登雲、孫祖壽諸軍入大紅門,與額駙揚古利、甲喇額真音達戶齊擊之明兵左次,旋克永平。復攻昌黎,先登,被六創。以功授備御。尋坐事奪世職。五年,從圍大凌河城,明監軍道張春赴援,珠瑪喇與甲喇額真鄂諾迭戰,破其前鋒。

六年,從伐察哈爾,次穆魯哈岱,獲布延圖臺吉,殲其從者百餘,俘其孥。七年,從貝勒德格類、岳託攻旅順口,將巴牙喇兵十人,以舟登擊甕城。巴奇蘭既令於眾,珠瑪喇與牛錄額真雍舜超躍而上,大聲自名曰:「珠瑪喇登城矣!」被三創,不少卻,卒拔其城。上聞,嘉嘆,親酌金卮以賜,復世職。九年,從貝勒多鐸伐明,圍錦州,夜設雲梯以攻,被創甚。

崇德元年,從伐朝鮮,力戰,克山寨;從伐明,敗明總兵,取四縣。三年,授兵部理事官。從伐明,圍錦州。明兵屯廣寧北苕峙山,珠瑪喇別將四十人破其寨;又招降別軍屯駱駝山及大凌河北山諸寨。六年,命與甲喇額真僖福監張家口互市。事畢,所司劾珠瑪喇以私財為市,且索馬蒙古,論死,上命寬之,復奪世職,輸所市物入官。尋從鄭親王濟爾哈朗復圍錦州,敵將奪我軍砲,珠瑪喇擊之退;既,復至,珠瑪喇射殪敵,敵乃潰。七年,與噶布什賢噶喇依昂邦沙爾虎達伐虎爾哈部,降喀爾喀木等十屯,俘壯丁千餘及牲畜、輜重以歸,上命迎勞。

順治初,珠瑪喇以甲喇額真從入關,擊李自成。尋授正藍旗滿洲梅勒額真、兵部侍郎,復世職。二年十一月,與梅勒額真和託等帥師駐防杭州,珠瑪喇將左翼。馬士英、方國安自嚴州侵餘杭,珠瑪喇擊之走。還,未至杭州三十里,遇土寇,復擊破之。國安等仍以數萬人分屯江東諸山及杭州郊外硃橋、範村諸地,珠瑪喇與總兵田雄、副將張傑等分兵逐捕。三年,率師徇福建,與巴牙喇纛章京敦拜擊破明兵。五年,從征南大將軍譚泰討叛將金聲桓於江西,與固山額真何洛會及沙爾虎達等屢敗聲桓兵,焚其舟千三百有奇,下九江及其屬縣凡六。遷正白旗蒙古固山額真、吏部尚書。世職累進三等阿思哈尼哈番。

十年冬,坐選授山東驛道房之麒嘗占籍青州不詳勘,罷尚書。十一年,明將李定國等寇廣東,命珠瑪喇為靖南將軍,副以敦拜,率師討之。方攻新會,尚可喜、耿繼茂等軍於三水,扼隘列屯。珠瑪喇至,與合軍擊敵,戰於珊洲,斬副將一,獲裨將十餘,馘一百五十餘級。師至新會,定國所將步騎卒四萬,分據山險列砲,以象為陣。珠瑪喇督將士力戰,定國兵既卻,復出兵四千餘人自山上馳下,我師力禦敗之,奪其山,定國兵乃遁。十二年二月,定國走高州,珠瑪喇遣梅勒額真畢力克圖、鄂拜等以師從之,戰於興業,再戰於橫州,定國渡江焚其橋,我師躡其後,三戰皆勝。定國走入安隆,珠瑪喇與尚可喜等復高州、雷州、廉州三府及所屬州三、縣八;又克廣西境州二、縣四:凡二十二城。得象十六、馬二百有奇,他器械稱是。上賜敕獎勵。九月,師還,入見,上諭大學士馮銓等曰:「珠瑪喇率師征廣東捷歸,年方五十。壯年能立功,為有福也!」賜茶慰勞。部議進一等阿思哈尼哈番兼一拖沙喇哈番,上以珠瑪喇等擊破定國,雪衡州、桂林之憤,功高不當循常格,命再議,進三等精奇尼哈番。十五年,致仕。康熙元年,卒,謚襄敏。

瓦爾喀珠瑪喇,那木都魯氏,居瓦爾喀部渾春地。祖察禮,率族歸太祖。旗制定,隸滿洲正白旗。珠瑪喇方少時,即從太祖征伐,授牛錄額真。以同時有碧魯珠瑪喇,命綴地於名以為別。

天聰八年,授世職牛錄章京。嘗率噶布什賢兵十人,逐得蒙古亡者四十三人,上特予優賚。崇德二年,與牛錄額真喀凱等分道伐瓦爾喀部,徇額勒約索、額黑庫倫、僧庫勒諸路,俘獲甚眾。以功加半個前程。三年,授吏部理事官。四年三月,從貝勒岳託伐明,攻故城,夜以雲梯登,克之。明總兵侯世祿師赴援,珠瑪喇徒步突敵軍,力戰,世祿敗去;珠瑪喇創甚,明太監高起潛師復至,負創戰尤力,起潛亦敗去。十月,從略錦州,敗明兵,入邊至太平寨,明兵嚴陣以待,珠瑪喇徒步大呼,入陣斫鹿角,中創不稍卻,明兵大潰。十一月,從承政索海、薩穆什喀伐索倫部,珠瑪喇俘十有九人。道攻虎爾哈部雅克薩,焚其郛,牛錄額真和託先登,珠瑪喇繼之,克其城。師還,次黑龍江之濱,虎爾哈潰兵復合,烏魯蘇屯酋博穆博果爾以六千人擊正藍旗後隊,珠瑪喇與索海設伏掩擊,殲敵略盡。以功進三等甲喇章京。

六年,從伐明,圍錦州,擊敗松山騎卒。明總督洪承疇赴援,營松山西北,我師與戰,右翼敗;敵萃於左翼,珠瑪喇力戰,砲傷頷,踣且絕,上深悼之,賜襚以斂。後三日復蘇,上聞喜甚,令加意休養,毋即從軍,命監造盛京塔,塔成,厚賚之。旋令率師戍錦州,明兵來攻,戰竟夜,敵敗去,斬四十餘級,得雲梯及軍械。累進一等甲喇章京。

順治初,從入關,擊李自成,平馬山口土寇,以功加半個前程。二年十月,調戶部理事官。十一月,與固山額真巴顏等帥師會定西大將軍何洛會西討張獻忠。三年,肅親王豪格代何洛會督諸軍向階州,聞獻忠兵屯禮縣南,遣珠瑪喇分兵擊之,獻忠兵驚竄;復與巴牙喇纛章京鰲拜進兵西充,獻忠死,乃還師。六年,從討叛將姜瓖,次左衛。瓖兵屯城外迎戰,珠瑪喇擊之走,城遂下。逐賊寧武關,瓖兵置砲山岡以拒,珠瑪喇與甲喇額真烏庫禮疾馳據岡脊,破其壘,瓖所置總兵劉偉以關降。師還,擢正白旗梅勒額真。世職累進一等阿思哈尼哈番。十年三月,卒,祀四川名宦。

伊瑪喇,其弟也。肅親王定四川,伊瑪喇以巴牙喇侍衛從。師次保寧,獻忠將趙雲桂來攻。伊瑪喇登城射中雲桂目,賊駭走,師從之,大捷,即襲其兄世職,授甲喇額真。康熙十三年,從揚威將軍阿密達討叛將王輔臣。十四年五月,克寧州。九月,進攻平涼,未至八里,輔臣率萬餘人出拒,伊瑪喇從貝勒洞鄂與戰,輔臣敗入城。十五年,從撫遠大將軍圖海復攻平涼,至城北虎山墩詗賊,賊合步騎猝至,伊瑪喇奮戰,賊敗去。事平,師還。二十七年,乞休。三十四年五月,卒,亦祀四川名宦。世宗即位,命錄戰功未受賞者,加伊瑪喇拖沙喇哈番。

論曰:太宗與明戰,下大凌河,克錦州,皆以全力爭。壬午之師,間道深入數千里,如行無人之境,為前此所未有,則圖爾格之績也。以是戰多踵為功宗。伊爾登、巴奇蘭、齊爾格申輩皆驍武,從太宗征伐,搴旗陷陣;而葉臣、珠瑪喇入關後,又以夙將力戰策勛。大業將成,群才翊運,盛矣!


\end{pinyinscope}