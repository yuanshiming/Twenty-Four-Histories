\article{列傳九十四}

\begin{pinyinscope}
尹繼善劉於義陳大受張允隨陳宏謀

尹繼善,字元長,章佳氏,滿洲鑲黃旗人,大學士尹泰子。雍正元年進士,改庶吉士,授編修。五年,遷侍講,尋署戶部郎中。上遣通政使留保等如廣東按布政使官達、按察使方原瑛受賕狀,以尹繼善偕。鞫實,即以尹繼善署按察使。六年,授內閣侍讀學士,協理江南河務。是秋,署江蘇巡撫,七年,真除。疏禁收漕規費,定石米費六分,半給旗丁,半給州縣,使無不足,然後裁以法。平糶盈餘,非公家之利,應存縣庫,常平倉捐穀聽民樂輸,不得隨漕勒徵。命如議行。又疏請崇明增設巡道,兼轄太倉、通州。並釐定永興、牛羊、大安諸沙分防將吏。福山增隸沙船,與京口、狼山諸汛會哨。又請移按察使駐蘇州,蘇松道駐上海。皆從之。旋署河道總督。九年,署兩江總督。十年,協辦江寧將軍,兼理兩淮鹽政。疏言:「鎮江水兵駐高資港,江寧水兵駐省會,各增置將吏。狼山復設趕糸曾大船,與鎮江、江寧水兵每月出巡察,庶長江數千里聲勢聯絡。」上嘉之。尹繼善請清察江蘇積欠田賦,上遣侍郎彭維新等助為料理,又命浙江總督李衛與其事。察出康熙五十一年至雍正四年都計積虧一千十一萬,上命分別吏蝕、民欠,逐年帶徵。尹繼善等並議敘。又請改三江營同知為鹽務道,並增設緝私將吏。

十一年,調雲貴廣西總督。思茅土酋刁興國為亂,總督高其倬發兵討之,擒興國,餘黨未解。尹繼善至,諮於其倬,得翾要,檄總兵楊國華、董芳督兵深入,斬其酋三,及從亂者百餘。元江、臨安悉定。分兵進攻攸樂、思茅,東道撫定攸樂三十六寨,西道攻六囤,破十五寨,降八十餘寨。疏聞,上諭曰:「剿撫名雖二事,恩威用豈兩端?當撫者不妨明示優容,當剿者亦宜顯施斬馘,俾知順則利,逆則害。今此攻心之師,即寓將來善後之舉,是乃仁術也。識之!」十二年,奏定新闢苗疆諸事,請移清江鎮總兵於臺拱,並移設同知以下官,增兵設汛,從之。又奏雲南濬土黃河,自土黃至百色,袤七百四十餘里。得旨嘉獎。尋詔廣西仍隸廣東總督。十三年,奏定貴州安籠等營制。貴州苗復亂,尹繼善發雲南兵,並徵湖廣、廣西兵策應。遣副將紀龍剿清平,參將哈尚德收新舊黃平二城,合兵徇重安。副將周儀等復餘慶,獲苗酋羅萬象等。總兵王無黨、韓勛剿八寨,總兵譚行義剿鎮遠。又令無黨合廣西、湖南兵與行義會,破苗寨,斬千餘級,獲苗酋阿九清等,苗亂乃定。乾隆元年,貴州別設總督,命尹繼善專督云南。二年,奏豁雲南軍丁銀萬二千二百有奇。入覲,以父尹泰老,乞留京侍養。授刑部尚書,兼管兵部。三年,丁父憂。四年,加太子少保。五年,授川陜總督。郭羅克部番復為亂,尹繼善檄諭番酋執為盜者以獻,事旋定。六年。奏陳郭羅克善後諸事,請設土目,打牲予號片,寬積案,撤戍兵,上皆許之。七年,丁母憂。

八年,署兩江總督,協理河務。疏言:「毛城鋪天然壩,高郵三壩,皆宜仍舊。」上諭令斟酌,因時制宜。九年,衛入覲,還,上命傳旨開天然壩,且曰:「衛奏河水小,壩宜開。」尹繼善覆奏,略言:「衛不問河身深淺,但問河水大小,非知河者也。河淺壩開,宣流太過。湖弱不敵黃強,為害滋甚。」上卒用尹繼善議。十年,實授兩江總督。十二年,疏言:「阜寧、高、寶諸地圩岸分年修治,務令圩外取土,挑濬成溝,量留涵洞,使旱澇有備。鳳、潁、泗三屬頻遭水患,河渠次第開濬,而田間圩塍實與為表裏,亦陸續興修。俟有成效,推行遠近。」上諭曰:「此誠務本之圖,實力為之。」

十三年,入覲,調兩廣,未行,授戶部尚書、協辦大學士、軍機處行走,兼正藍旗滿洲都統。未幾,復出署川陜總督。嗣以四川別設總督,命專督陜、甘。大學士傅恆經略金川,師經陜西,上獎尹繼善料理臺站、馬匹諸事,調度得宜。十四年,命參贊軍務,加太子太保。十五年,西藏不靖,四川總督策楞統兵入藏,命兼管川陜總督。

十六年,復調兩江。十七年,尹繼善以上江頻被水,疏請濬宿州睢河、彭家溝,泗州謝家溝,虹縣汴河上游,築宿州符離橋,靈壁新馬橋,砂礓河尾黃甿橋、翟家橋,詔如所請。羅田民馬朝柱為亂,檄總兵牧光宗捕治,並親赴天堂寨,獲朝柱家屬、徒黨,得旨嘉獎,召詣京師。十八年,復調署陜甘總督。雍正間,開哈密蔡伯什湖屯田,乾隆初,以畀回民。貝子玉素富以屢歉收請罷。尹繼善奏言:「從前開渠引水,幾費經營。回民不諳耕作,頻歲歉收。萬畝屯田,棄之可惜。請選西安兵丁子弟,或招各衛民承種。」上韙其言。

調江南河道總督。十九年,疏言:「河水挾沙而行,停滯成灘。有灘則水射對岸,即成險工。銅、沛、邳、睢、宿、虹諸地河道多灘,宜遵聖祖諭,於曲處取直,開引河,導溜歸中央,借水刷沙。河堤歲令加高,務使穩固,而青黃不接,亦寓賑於工。」詔如議行。命署兩江總督,兼江蘇巡撫。二十一年,疏請濬洪澤湖入江道,開石羊溝,引東西灣兩壩所減之水,疏芒稻閘達董家溝引河,引金灣閘壩所減之水,加寬廖家溝河口,引璧虎、鳳凰兩橋所減之水,並濬各河道上游,修天妃、青龍、白駒諸閘,從之。實授兩江總督。二十二年,疏言:「沛縣地最卑,昭陽、微山諸湖環之,濟、泗、汶、滕諸水奔注。請於荊山橋外增建閘壩,使湖水暢流入運。又沂水自山東南入駱馬湖,出盧口入運,阻荊山橋出水。當相度堵修。」上以所言中形勢,嘉之。旋與侍郎夢麟等會督疏治淮、揚、徐、海支幹各河暨高、寶各工,是冬事竟,議敘。二十五年,上命增設布政使,尹繼善請分設江寧、蘇州二布政使,而移安徽布政使駐安慶。二十七年,上南巡,命為御前大臣。二十九年,授文華殿大學士,仍留總督任。三十年,上南巡,尹繼善年七十,御書榜以賜。召入閣,兼領兵部事,充上書房總師傅。三十四年,兼翰林院掌院學士。三十六年,上東巡,命留京治事。四月,卒,贈太保,發帑五千治喪。令皇八子永璇奠醊,永璇,尹繼善壻也。賜祭葬,謚文端。

尹繼善釋褐五年,即任封疆,年才三十餘。蒞政明敏,遇糾紛盤錯,紆徐料量,靡不妥貼。一督云、貴,三督川、陜,四督兩江。在江南前後三十年,最久,民德之亦最深。世宗最賞李衛、鄂爾泰、田文鏡,嘗諭尹繼善,謂當學此三人。尹繼善奏曰:「李衛,臣學其勇,不學其粗。田文鏡,臣學其勤,不學其刻。鄂爾泰,宜學處多,然臣亦不學其愎。」世宗不以為忤。高宗嘗謂:「我朝百餘年來,滿洲科目中惟鄂爾泰與尹繼善為真知學者。」御制懷舊詩復及之。子慶桂,自有傳。

劉於義,字喻旃,江蘇武進人。康熙五十一年進士,改庶吉士,授編修。在翰林文譽甚著,凡有撰擬,輒稱旨。雍正元年,命直南書房,遷中允。再遷侍講,督山西學政。三年,遷庶子,上諭以留心民事。歲饑,無積貯,奏請歲以耗羨四萬於太原、平陽、潞安、大同買米貯倉,春糶秋補,上命巡撫伊都立酌量舉行。四年,一歲四遷,擢倉場侍郎。倉吏積習,鬻正米以購篩颺耗米抵額。於義嚴出入,稽餘米定數,宿弊一清。七年,命察覈西寧軍需。八年,遷吏部侍郎。命與侍郎牧可登如山東察賑,並按按察史唐綏祖劾濟南知府金允彞袒鄒平知縣袁舜裔虧空,論如律。

九年,授直隸河道總督。奏天津截留漕糧,省津貼諸費,但給地方官耗米百之一。又奏青龍灣諸地,侍郎何國宗議建雞心閘十四阻水,當停。並請展壩面,使無礙水道。均如議行。擢刑部尚書,仍理河務。尋署直隸總督。直隸盜犯,依律不分首從皆斬。大名劫盜十餘案,每案數十人。於義以兇器祗田具,贓物僅米穀,乃饑民借糧爭奪,非盜,奏請得末減。直隸盜案視各省分首從自此始。

十年,署陜西總督。十一年,授吏部尚書,仍署總督。累疏言甘、涼為軍需總匯,糧草價昂,兵餉不敷養贍。請酌借耔糧農器,於瓜州諸地開墾屯種,耕犁以馬代牛,並募耕夫二百,教回民農事。又於赤金、靖逆之北湃帶湖及塔兒灣築臺堡為保障,安家窩鋪口別開渠供灌溉。又疏請甘、涼設馬廠,牧長、牧副,視太僕寺條例,歲十一月,察馬匹孳生多寡,為弁兵升降賞罰。均如所請行。十三年,命大學士查郎阿代於義領陜西總督,予於義欽差大臣關防,留肅州專筦軍儲。乾隆元年,奏言:「蘭州浮橋始於前明,用二十四艘,兩埠鐵纜百二十丈。自有司遞減四舟,纜僅七十丈,於是埠基砌入河心,水益湍急,沖潰屢見。請動用公帑改復原式。庶河寬水緩,以便行旅。」得旨允行。

查郎阿入覲,於義仍署陜西總督。二年,召還京。三年,查郎阿劾承辦軍需道沈青崖等私運侵帑,辭連於義。上遣侍郎馬爾泰會查郎阿按治,於義坐奪官,並責償麥稞價銀三萬餘兩。甘肅自康熙末至雍正初,虧帑金一百六十餘萬,文書散缺。於義奉命察覈,逮任總督,部署西師往返,凡四年,屯田築堡,安集流移,輸送軍糧戰馬,其勞最多。以簿領過繁,得過亦由此。

五年,起署直隸布政使。七年,授福建巡撫,疏請裁減閩鹽課外加派。漳州民陳作謀、臺灣民王永興等謀為亂,遣將吏捕治。八年,調山西,召補戶部尚書。九年,調吏部尚書、協辦大學士。御史柴潮生請修治直隸水利,命同直隸總督高斌勘察。議濬檿牛河;開白溝河支流,西澱亦開支河,東澱河道裁灣取直,子牙河疏河口,築堤界,別清渾;疏鳳河;濬塌河澱;引唐河入保定河;濬正定諸泉,引以溉田;並修復營田舊渠閘。是為初次應舉各工。十年,署直隸總督,加太子太保。是冬,報初次工竟。復議還鄉河裁灣取直,築運薊河西堤;挑張青口支河、新安新河;拓廣利渠,望都至安肅開溝;並裁永定河兜灣。是為二次應舉各工。引塌河澱漲水入薊運河;疏天津賈家口、靜海蘆北口諸河;及慶雲馬頰河、鹽山宣惠河。是為三次應舉各工。又令署直隸河道總督,疏請減慶雲賦額。上命減地丁十之三,著為令。十二年夏,報二、三次工竟。召還。

十三年二月,奏事養心殿,跪久致僕,遽卒。賜祭葬,謚文恪。

陳大受,字占咸,湖南祁陽人。幼沉敏,初授內則,即退習其儀。既長,家貧,躬耕山麓。同舍漁者夜出捕魚,為候門,讀書不輟。雍正十一年,成進士,選庶吉士。乾隆元年,授編修。二年,大考翰詹諸臣,日午,上御座以待。大受卷先奏,列第一,超擢侍讀。五遷吏部侍郎。四年,授安徽巡撫。初視事,決疑獄,老吏駭其精敏。廬、鳳、潁諸府時多盜,有司多諱匿,大受定限嚴緝,月獲盜五十輩,得旨褒美。淮南、北洊饑,發倉穀賑之。穀且盡,繼以麥。又告糶江南、廣東,且發且儲。時頻歲饑民掠米麥以食,有司以盜論。哀其情,奏原六十餘人。麥熟,禁鵕麴造酒及大商囤積。又以高阜斜陂不宜稻麥。福建安溪有旱稻名畬粟,不須溉灌,前總督郝玉麟得其種,教民試藝有獲。因令有司多購,分給各州縣,俾民因地種植。事聞,上諭曰:「諸凡如此留心,甚慰朕懷。」

是年,調江蘇,疏請飭糧道較定各州縣漕斛,及先冬令民搜蝻子。屢諭嘉獎,並以搜蝻子法令直隸總督高斌仿行。常州、鎮江、太倉三府州被水災,發倉治賑。江南舊多借堰圩塘,或有久廢者,被水後尤多潰敗,工鉅費重,民力不能勝。大受出官粟借之,召民興築,計時而成。於江浦繕三合、永豐、北城諸圩,於句容復郭西塘黃堰,蘇州、太倉疏劉家河,灌溉瀦洩,諸工畢舉。七年秋,黃河決古溝、石林,高、寶、興、泰、徐諸州縣罹其患,大受馳視以聞。上命截漕米協濟,大受乃命多具舟,候水至分載四出,舳艫數百里,一日而遍。丹陽運河口藉湖水灌輸,淤沙需疏濬,大受奏定六年大修,每年小修。後高宗南巡,禦制反李白丁都護歌曰:「豈無疏濬方,天工在人補。輪年大小修,往來通商賈。」蓋嘉其奏定歲修法利於漕運也。

十年,有旨蠲明年天下錢糧,大受疏請核準漕項科則,曉諭周知;匯覈地丁耗羨,同漕項並完;酌定業戶減租分數,通飭遵行。得旨嘉獎。戶部議禁商人貯米,大受謂:「商人貯米,得少利即散,貯不過一歲,民且利焉。請弛禁便。」又言:「城工核減,意在節用。用省而工惡,再修且倍之。」上皆韙其言。常州俗好佛,家設靜堂,自立名教。江寧、松江、太倉漸染其習。大受疏請飭有司防禁,移佛入廟;堂內人田屋產,量為處置。上諭曰:「此等事須實力,不可欲速。不然,則所謂好事不如無也。」

十一年,加太子少保,調福建。十二年,疏言:「近海商民,例許往暹羅造船販米。內渡時若有船無米,應倍稅示罰。」部議從之。疏言:「巡臺御史巡南北二路,臺灣、鳳山、諸羅、彰化四縣具廚傳犒賞,往往濫準詞訟。又於額設胥役外,俾奸民注籍,恃符生事。」上命自乾隆五年起,巡臺御史均下部嚴議。又疏言:「臺灣番民生業艱難,向漢民重息稱貸。子女田產,每被盤折。請撥臺穀二萬石分貯諸羅、彰化、淡水諸縣,視鳳山例接濟。其不原借者聽。」報可。臺灣民、番雜處,土音非譯不通。有奸民殺人賄通事,移坐番罪,疑之,再鞫,竟得白。或言海上有島十四,為田萬餘畝,可開墾,前政以入告。大受以島地久在禁令,一旦開禁,聚人既多,生奸尤易。設兵彈壓,為費彌甚,利不敵害,輒奏罷之。召授兵部尚書。十三年,調吏部,協辦大學士、軍機處行走。十四年,金川平,晉太子太傅。秋,署直隸總督。十五年,授兩廣總督。陛辭請訓,上曰:「汝直軍機處兩年,萬幾之事,皆所目擊,即朕訓也。何贅辭?惟中外一心足矣。」尋命協理粵海關。兩粵去京師遠,吏媮民哤,大受以猛治之,舉劾不法吏,政令大行。十六年,以病乞解任,溫詔慰留。未幾,卒,賜祭葬,謚文肅,祀賢良祠。

大受眉目皆上起,豐髯有威。清節推海內。以微時極貧,祿不逮親養,自奉如布衣時。子輝祖,自有傳。

張允隨,字覲臣,漢軍鑲黃旗人。祖一魁,福建邵武知府,有政績,祀名宦。允隨入貲為光祿寺典簿,遷江南寧國同知,擢雲南楚雄知府。雍正元年,調廣南。丁母憂,總督鄂爾泰等請留司銅廠。二年,授曲靖知府,擢糧儲道。鄂爾泰復薦可大任,上召入見。五年,擢按察使。未幾,遷布政使。雲南產銅供鑄錢,寶源、寶泉二局需銅急,責委員領帑採洋銅,洋銅不時至。允隨綜銅廠事,察知舊廠產尚富,增其值。民樂於開採,舊廠復盛。又開大龍、湯丹諸新廠,歲得銅八九百萬斤供用。乃停採洋銅,國帑省,官累亦除。八年,調貴州。未幾,授雲南巡撫。允隨官云南久,熟知郡國利病,山川險要,苗、夷情狀。十一年,思茅土酋刁興國糾徼外苦蔥蠻等為亂,蔓延數州縣。允隨與總督高其倬遣兵討之,思茅圍解。亂苗遁攸樂,知縣章綸以事詣會城,至蟃蜯村,遇寇死。允隨趣兵進,擒興國。餘眾走臨安,復擊破之。允隨疏以鎮沅、思樂府縣皆新改土為流,請立學,設教職,定學額。又疏以雲南各府州或兵少米多,請以額徵秋米石折銀一兩;或兵多米少,請以額徵條銀兩收米一石。十二年,疏請於廣西府開爐鼓鑄。皆下部議行。十三年,疏報蒙化墾田二十六頃有奇。

乾隆二年,署云南總督。疏言:「雲南水利與他省不同,水自山出,勢若建瓴。大率水高田低,自上而下,當濬溝渠,使盤旋曲折,承以木見、石槽,引使溉田。偶有田高水低,則宜車戽。又或雨後水急,則宜塘蓄。低道小港水阻恐傍溢,則宜疏水口使得暢流。山多沙磧,水發嫌迅激,則宜築堤墊,俾護田畝。臣令有司勘修,工小,令於農隙按田出夫,督率興作;工稍大者,出夫外,應需工料,令集士民公議需費多寡。有田用水者,按田定銀數,借庫帑興工。工畢,分年還款。工大非民力能勝,詳情覆勘,以官莊變價,留充工費。」報聞。

三年,請停鑄錢運京。是冬,入覲。四年,正歲,上宴廷臣,賦柏梁體詩,允隨與焉。五年,疏言:「雲南鹽不敷民食,安寧得洪源井,試煎,年獲二十一萬餘斤。麗江得老姆井,試煎,年獲十八萬餘斤。分地行銷,定為年額。」上獎為有益之事。署貴州總督。六年,廣東妖民黃順等遁匿貴州境,有司捕得奏聞。上諭曰:「汝不以五日京兆自居,盡心治事可嘉。」

復署云南總督。兵部議各省有增設兵額,量加裁減。允隨奏:「雲南昭通、普洱二鎮有增設兵額,地處邊要,未可裁減。惟有通覈合省標、鎮、營、協,按額均減,分計則兵裁無幾,合計則餉省已多。標、鎮、營、協應裁兵一千一百六十,先裁餘丁四百四十八。餘俟缺出停補。」從之。允隨請濬金沙江,上命都統新柱、四川總督尹繼善會勘。疏言:「金沙江發源西域,入雲南,經麗江、鶴慶、永北、姚安、武定、東川、昭通七府,至敘州入川江。東川府以下,南岸隸雲南,北岸隸四川。營汛分布,田廬相望。至大井壩以上,南岸尚有田廬,北岸皆高山。山後沙馬、阿都兩土司地,從前舟楫所不至。自烏蒙改流設鎮,雲南兵米,每歲糴自四川,皆自敘州新開灘至永嘉黃草坪五百八十里,溯流而上。更上自黃草坪至金沙廠六十里,商舶往來。臣等相度,內有大漢漕、凹崖、三腔、鑼鍋耳諸灘險惡,應行修理。更上自金沙廠至濫田壩二百二十七里,十二灘,濫田壩最險,次則小溜筒。臣等相度開鑿子河。更上自雙佛灘至蜈蚣嶺,十五灘相接,石巨工艱。臣等令改修陸路,以避其險。雲南地處極邊,民無蓋藏,設遇水旱,米價增昂。今開通川道,有備無患。」上諭曰:「既可開通,妥協為之,以成此善舉。」允隨主辦其役,計程千三百餘里,費帑十餘萬,經年而工成。

八年,疏言:「大理洱海發源鶴慶濔沮河,至大理,合蒼山十八溪,匯而成海。下自波羅甸出天生橋,趨瀾滄江。海袤百二十里,廣二十餘里;而天生橋海口寬不及丈,每致倒流,淹浸濱海民田。臣飭將海口疏治寬深,自波羅甸下達天生橋,分段開濬,疊石為堤,外栽茨柳,為近水州縣袪漫溢之患。海口涸出田萬餘畝,令附近居民承墾,即責墾戶五年一大修,按田出夫,合力疏濬。」授雲南總督,兼管巡撫。九年,疏報東川阿壩租得銅礦,試煎,月得銅四萬餘斤。十年,加太子少保。

十二年,授雲貴總督。疏言:「苗、惈種類雖殊,皆具人心。如果撫馭得宜,自不至激成事變。臣嚴飭苗疆文武,毋許私收濫派,並禁胥役滋擾。至苗民為亂,往往由漢奸勾結。臣飭有司稽察捕治。」又疏言:「貴州思州諸府與湖南相接,今有辰、沅饑民百餘入貴州境採蕨而食。臣已飭貴州布政使、糧驛道以公使銀賑濟。如有續至,一體散給安置。」諸疏上,並嘉獎。十五年,入覲,授東閣大學士,兼禮部尚書,加太子太保。十六年,卒,賜祭葬,謚文和。

陳宏謀,字汝咨,廣西臨桂人。為諸生,即留心時事,聞有邸報至,必借觀之。自題座右,謂「必為世上不可少之人,為世人不能作之事。」雍正元年恩科,世所謂春鄉秋會。宏謀舉鄉試第一,成進士,改庶吉士,授檢討。四年,授吏部郎中。七年,考選浙江道御史,仍兼郎中。監生舊有考職,多以人代。世宗知其弊,令自首,而州縣吏藉察訪為民擾。宏謀疏請禁將來,寬既往。召見,徵詰再三,申論甚晰,乃允其奏,以是知其能。授揚州知府,仍帶御史銜,得便宜奏事。丁父憂,上官留之,辭,不許。遷江南驛鹽道,仍帶御史銜,攝安徽布政使。又丁母憂,命留任,因乞假歸葬。

十一年,擢雲南布政使。初,廣西巡撫金鉷奏令廢員墾田報部,以額稅抵銀得復官,報墾三十餘萬畝。宏謀奏言:「此曹急於復官,止就各州縣求有餘熟田,量給工本,即作新墾。田不增而賦日重,民甚病之,請罷前例。」上命雲南廣西總督尹繼善察實,尹繼善請將虛墾地畝冒領工本覈實追繳。乾隆元年,部議再敕兩廣總督鄂彌達會鉷詳勘。宏謀劾鉷欺公累民,開捐報墾不下二十餘萬畝,實未墾成一畝,請盡數豁除。時鉷內遷刑部侍郎,具疏辨。上命鄂彌達會巡撫楊超曾確勘。二年,宏謀復密疏極論其事。高宗責「宏謀不待議覆,又為是瀆奏。粵人屢陳粵事,恐啟鄉紳挾持朝議之漸」。交部議,降調。尋鄂彌達等會奏,報墾田畝多不實,請分別減豁。鉷下下降黜有差。

三年,授宏謀直隸天津道。五年,遷江蘇按察使。六年,遷江寧布政使,甫到官,擢甘肅巡撫,未行,調江西。九年,調陜西。十一年,復調回江西。尋又調湖北。十二年,川陜總督慶復劾宏謀在陜西愛憎任情,好自作聰明,不持政體。部議奪官,上命留任。未幾,復調陜西。上諭曰:「此汝駕輕就熟之地,當秉公持重,毋立異,毋沽名。能去此結習,尚可造就也。」署陜甘總督。十五年,加兵部侍郎。其冬,河決陽武。調河南巡撫。十七年,調福建。十九年,復調陜西。二十年,調甘肅。再調湖南,疏劾布政使楊灝侵扣穀價。上嘉其不瞻徇,論水顥罪如律。二十一年,又調陜西。

二十二年,調江蘇。入覲,上詢及各省水災,奏言皆因上游為眾水所匯,而下游無所歸宿,當通局籌辦。上以所言中肯綮,命自河南赴江蘇循途察勘。十二月,遷兩廣總督,諭曰:「宏謀籍廣西,但久任封疆,朕所深信。且總督節制兩省,專駐廣東,不必回避。」二十三年,命以總督銜仍管江蘇巡撫,加太子少傅。二十四年,坐督兩廣時請增撥鹽商帑本,上責「宏謀巿恩沽名,痼習未改」。下部議奪官,命仍留任。又以督屬捕蝗不力,奪總督銜,仍留巡撫任。二十六年,又以失察滸墅關侵漁舞弊,議罷任,詔原之,諭責「宏謀模棱之習,一成不變」。調撫湖南。二十八年,遷兵部尚書,署湖廣總督,仍兼巡撫。召入京,授吏部尚書,加太子太保。

宏謀外任三十餘年,歷行省十有二,歷任二十有一。蒞官無久暫,必究人心風俗之得失,及民間利病當興革者,分條鉤考,次第舉行。諸州縣村莊河道,繪圖懸於壁,環復審視,興作皆就理。察吏甚嚴,然所劾必擇其尤不肖者一二人,使足怵眾而止。學以不欺為本,與人言政,輒引之於學,謂:「仕即學也,盡吾心焉而已。」故所施各當,人咸安之。

在揚州值水災,奏請遣送饑民回籍,官給口糧,得補入賑冊,報可。鹽政令淮商於稅額外歲輸銀助國用,自雍正元年始,積數千萬,率以空數報部。及部檄移取,始追徵,實陰虧正課,宏謀奏停之。

在雲南,方用兵惈夷,運糧苦道遠,改轉搬遞運,民便之。增銅廠工本,聽民得鬻餘銅,民爭趨之。更鑿新礦,銅日盛,遂罷購洋銅。立義學七百餘所,令苗民得就學,教之書。刻孝經、小學及所輯綱鑒、大學衍義,分布各屬。其後邊人及苗民多能讀書取科第,宏謀之教也。

在天津,屢乘小舟咨訪水利,得放淤法,水漲挾沙行,導之從堤左入、是右出。如是者數四,沙沉土高,滄、景諸州悉成沃壤。按察江蘇,設弭盜之法,重誣良之令,嚴禁淹親柩及火葬者。

在江西,歲饑,告糴於湖廣。發帑繕城垣,築堰埭,修圩堤閘壩,以工代賑。南昌城南羅絲港為贛水所趨,善沖突,建石堤捍之。左蠡硃磯當眾水之沖,亦築堤百丈,水患以平。又以錢貴,奏請俟雲南銅解京過九江,留五十五萬五千斤,開爐鼓鑄;並以舊設爐六,請增爐四:詔並許之。又以倉儲多虧缺,請令民捐監,於本省收穀,以一年為限。限滿,上命再收一年。又以民俗尚氣好訐訟,請令各道按行所屬州縣,察有司,自理詞訟,毋使延閣滋累。上命實力督率,毋徒為具文。

在陜西,募江、浙善育蠶者導民蠶,久之利漸著。高原恆苦旱,勸民種山薯及雜樹,鑿井二萬八千有奇,造水車,教民用以灌溉。陜西無水道,惟商州龍駒寨通漢江,灘險僅行小舟。宏謀令疏鑿,行旅便之。又以陜西各屬常平倉多空廒,亦令以捐監納穀。並請開爐鑄錢,如江西例。戶部撥運洋銅,鑄罄,採雲南銅應用,錢價以平。請修文、武、成、康四王及周公、太公陵墓,即以陵墓外餘地召租得息,歲葺治。皆下部議行。

在河南,請修太行堤。又以歸德地窪下,議疏商丘豐樂河、古宋河,夏邑響河,永城巴溝河,民力不勝,請發帑濬治。

既至福建,歲歉米貴,內地仰食臺灣,而商舶載米有定額,奏弛其禁以便民。又疏言福建民囂競多訟,立限月為稽覈,以已未結案件多寡,課州縣吏勤惰。又言福建地狹民稠,多出海為商,年久例不準回籍。請令察實內地良民或已死而妻妾子女原還里者,不論年例,許其回籍,從之。

在湖南,禁洞庭濱湖民壅水為田,以寬湖流,使水不為患,歲大熟。江南災,奏運倉穀二十萬石濟之,仍買民穀還倉。

再至陜西,聞甘肅軍需缺錢,撥局錢二百萬貫濟餉,上嘉其得大臣任事體。疏請興關外水利,濬赤金、靖逆、柳溝、安西、沙州諸地泉源,上命後政議行。又以準噶爾既內附,請定互巿地,以茶易馬充軍用,詔從之。

其治南河,大要因其故道,開通淤淺,俾暢流入海。督民治溝洫,引水由支達幹,時其蓄洩。徐、海諸州多棄地,遇雨輒淫溢,課民開溝,即以土築圩,多設涵洞為旱潦備;低地則令種蘆葦,薄其賦。其在江蘇,尤專意水利,疏丁家溝,展金灣壩,濬徐六涇白茆口,洩太湖水,築崇明土塘御海潮,開各屬城河。又疏言:「蘇州向設普濟、育嬰、廣仁、錫類諸堂,收養煢獨老病,並及棄嬰。請將通州、崇明濱海淤灘,除附近民業著聽升科,餘撥入堂。又通州、崇明界新漲玉心洲,兩地民互爭,請並撥入,以息爭競。」上諭曰:「不但一舉而數善備,汝亦因此得名也。」

及督湖廣,疏言:「洞庭湖濱居民多築圍墾田,與水爭地,請多掘水口,使私圍盡成廢壤,自不敢再築。」上諭曰:「宏謀此舉,不為煦嫗小惠,得封疆之體。」

逮入長吏部,疏言:「文武官弁,均有捕盜之責。乃州縣捕役,平時豢盜,營兵捕得,就讞時任其狡展,或且為之開脫。嗣後應令原獲營員會訊。」上嘉其所見切中事理。又疏言:「河工辦料,應令管河各道親驗加結。失事例應文武分償,而參游例不及,應酌改畫一。」下河督議行。又言:「匿名揭帖,循例當抵罪,所告款內有無虛實,仍應按治。則宵小不得逞奸,有司亦知所警。」上亦韙之。

二十九年,命協辦大學士。三十二年,授東閣大學士,兼工部尚書。三十四年,以病請告,迭諭慰留。三十六年春,病甚,允致仕,加太子太傅,食俸如故。賜御用冠服,命其孫刑部主事蘭森侍歸。詔所經處有司在二十里內料理護行。上東巡,覲天津行在,賜詩寵其行。六月,行至兗州韓莊,卒於舟次,年七十六。命祀賢良祠,賜祭葬,謚文恭。

宏謀早歲刻苦自勵,治宋五子之學,宗薛瑄、高攀龍,內行修飭。及入仕,本所學以為設施。蒞政必計久遠,規模宏大,措置審詳。嘗言:「是非度之於己,毀譽聽之於人,得失安之於數。」輯古今嘉言懿行,為五種遺規,尚名教,厚風俗,親切而詳備。奏疏文檄,亦多為世所誦。曾孫繼昌,字蓮史。嘉慶二十四年鄉試,二十五年會試、廷試,俱第一,授修撰。歷官至江西布政使。

論曰:乾隆間論疆吏之賢者,尹繼善與陳宏謀其最也。尹繼善寬和敏達,臨事恆若有餘;宏謀勞心焦思,不遑夙夜,而民感之則同。宏謀學尤醇,所至惓惓民生風俗,古所謂大儒之效也。於義督軍儲、策水利,皆秩秩有條理。大受剛正,屬吏憚之若神明,然論政重大體,非茍為苛察者比。允隨鎮南疆久,澤民之尤大者,航金沙江障洱海,去後民思,與江南之懷尹繼善、陳宏謀略相等,懿哉!


\end{pinyinscope}