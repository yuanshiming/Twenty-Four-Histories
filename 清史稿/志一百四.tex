\article{志一百四}

\begin{pinyinscope}
○河渠四

△直省水利

清代軫恤民艱,亟修水政,黃、淮、運、永定諸河、海塘而外,舉凡直省水利,亦皆經營不遺餘力,其事可備列焉。

順治四年,給事中梁維請開荒田、興水利,章下所司。十一年,詔曰:「東南財賦之地,素稱沃壤。近年水旱為災,民生重困,皆因水利失修,致誤農工。該督撫責成地方官悉心講求,疏通水道,修築堤防,以時蓄洩,俾水旱無虞,民安樂利。」

康熙元年,重修夾江龍興堰,又鑿大渠以廣灌溉。二年,修和州銅成堰,龍首、通濟二渠。交城磁瓦河漲,水侵城,築堤障之。三年,修嘉定楠木堰。九年,修郿縣金渠、寧曲水利。十二年,重修城固五門堰。十九年,濬常熟白茆港、武進孟瀆河。二十三年,修五河南湖堤壩。二十七年,修徽州魚梁壩。三十七年,命河督王新命修畿輔水利。

三十八年,聖祖南巡,至東光,命直隸巡撫李光地察勘漳河、滹沱河故道。覆疏言:「大名、廣平、真定、河間所屬,凡兩河經行之處,宜開濬疏通,由館陶入運。老漳河與單家橋支流合,至鮑家嘴歸運,可分子牙河之勢。」三十九年,帝巡視子牙河堤,命於閻、留二莊間建石閘,隨時啟閉。御史劉珩言,永平、真定近河地,應令引水入田耕種。諭曰:「水田之利,不可太驟。若剋期齊舉,必致難行。惟於興作之後,百姓知其有益,自然鼓勵效法,事必有成。」四十年,李光地言:「漳河分四支,三支歸運皆弱,一支歸澱獨強。遇水大時,當用挑水壩等法,使水分流,北不至挾滹沱以浸田,南不至合衛河以害運。」如所請行。

四十三年,挑楊村舊引河。先是子牙河廣福樓開引河時,文安、大城民謂有益,青縣民謂不便,各集河干互控。至是河成,三縣民皆稱便。天津總兵官藍理請於豐潤、寶坻、天津開墾水田,下部議。旋諭曰:「昔李光地有此請,朕以為不可輕舉者,蓋北方水土之性迥異南方。當時水大,以為可種水田,不知驟漲之水,其涸甚易。觀琉璃河、莽牛河、易河之水,入夏皆涸可知。」次年部臣仍以開墾為請,諭以此事暫宜存置,可令藍理於天津試開水田,俟冬後踏勘。

四十八年,濬鄭州賈魯河故道,自東趙訖黃河涯口新莊。於東趙建閘一,黃河涯口築草壩石閘各一。甘肅巡撫舒圖言:「唐渠口高於身,水勢不暢,應引黃河之水匯入宋澄堡。如水不足用,更於上游近黃處開河引水,酌建閘壩,以資蓄洩。」從之。江蘇巡撫於準言:「丹陽練湖,冬春洩水濟運,夏秋分灌民田。自奸民圖利,將下湖之地佃種升科,民田悉成荒瘠。請復令蓄水為湖,得資灌溉。」從之。五十七年,以沛縣連年被水,命河督趙世顯察勘。世顯言:「金鄉、魚臺之水,由沛之昭陽湖歷微山湖,從荊山口出貓兒窩入運。近因荊山口十字河淤墊,致低田被淹。應將沙淤濬通,再於十字河上築草壩。若遇運河水淺,即令堵塞,俾水全歸微山湖,出湖口閘以濟運,則民田漕運兩有裨益。」從之。

世宗時,於畿輔水利尤多區畫。雍正三年,直隸大水,命怡親王允祥、大學士硃軾相度修治。因疏請濬治衛河、澱池、子牙、永定諸河,更於京東之灤、薊,京南之文、霸,設營田專官,經畫疆理。召募老農,謀導耕種。四年,定營田四局,設水利營田府,命怡親王總理其事,置觀察使一。自五年分局至七年,營成水田六千頃有奇。後因水力贏縮靡常,半就湮廢。是年命侍郎通智、單疇書,會同川督岳鍾琪,開惠農渠於查漢托護,以益屯守,復建昌潤渠於惠農渠東北。六年,浚文水近汾河渠,引灌民田,開嵩明州楊林海以洩水成田。八年,帝以寧夏水利在大清、漢、唐三渠,日久頹壞,命通智同光祿卿史在甲勘修。是年修廣西興安靈渠,以利農田,通行舟。濬陳、許二州溝洫。

十年,雲貴總督鄂爾泰言:「滇省水利全在昆明海口,現經修濬,膏腴田地漸次涸出。惟盤龍江、金棱、銀棱、寶象等河俱與海口近,亟宜建築壩臺。」又言:「楊林海水勢暢流,周圍草塘均可招民開墾。宜良江頭村舊河地形稍高,宜另開河道以資灌溉。尋甸河整石難鑿,宜另濬沙河,俾得暢流。東川城北漫海,水消田出,亦可招墾。」均從之。十二年,營田觀察使陳時夏言:「文安、大城界內修橫堤千五百餘丈,營田四十八頃俱獲豐收。但恐水涸即成旱田,請於大堤東南開建石閘,北岸多設涵洞,以資宣洩。」從之。

乾隆元年,大學士嵇曾筠請疏濬杭、湖水利。兩廣總督鄂彌達言:「廣、肇二屬沿江一帶基圍,關系民田廬舍,常致沖坍,請於險要處改土為石,陸續興建。」下部議行。江南大雨水,淮陽被淹,命濬宿遷、桃源、清河、安東及高郵、寶應各水道。二年,命總督尹繼善籌畫雲南水利,無論通粵通川及本省河海,凡有關民食者,及時興修。陜西巡撫崔紀陳鑿井灌田以佐水利之議。諭令詳籌,勿擾閭閻。

三年,大學士管川陜總督事查郎阿言:「瓜州地多水少,民田資以灌溉者,惟疏勒河之水,河流微細。查靖逆衛北有川北、鞏昌兩湖,西流合一,名蘑菇溝。其西有三道柳條溝,北流歸擺帶湖。請從中腰建閘,下濬一渠,截兩溝之水盡入渠中,為回民灌田之利。」貴州總督張廣泗請開鑿黔省河道,自都勻經舊施秉通清水江至湖南黔陽,直達常德,又由獨山三腳坉達古州,抵廣西懷遠,直達廣東,興天地自然之利。均下部議行。四年,安徽布政使晏斯盛言,江北鳳、潁以睢水為經,廬州以巢湖為緯,他如六安舊有堤堰,滁、泗亦多溪壑,概應動帑及時修濬,從之。川陜總督鄂彌達等言:「寧夏新渠、寶豐,前因地震水湧,二縣治俱沉沒。請裁其可耕之田,將漢渠尾展長以資灌溉。惟查漢渠百九十餘里,渠尾餘水無多,若將惠農廢渠口修整引水,使漢渠尾接長,可灌新、寶良田數千頃。」上嘉勉之。

五年,河督顧琮言:「前經總河白鍾山奏稱『漳河復歸故道,則衛河不致泛溢,為一勞永逸之計』。臣等確勘,自和兒寨東起,至青縣鮑家嘴入運之處止,計程六百餘里,河身淤淺,兩岸居民稠密。若益以全漳之水,勢難容納,則改由故道,於直隸不能無患,然不由故道,又於山東不能無患。惟有分洩防禦,使兩省均無所害,庶為經久之圖。」總辦江南水利大理卿汪漋言:「鹽城東塘河及阜寧、山陽各河道,高郵、寶應下游,及串場河、溱潼河,俱淤淺,應挑濬。其串場河之範堤,及拼茶角二場堤工,俱逼海濱,應加寬厚。揚州各閘壩應疏築,限三年告成。」均如所請行。安徽巡撫陳大受言:「江北水利關系田功。原任籓司晏斯盛奏定興修,估銀四十餘萬。竊思水利固為旱澇有備,而緩急輕重,必須熟籌。各州縣所報,如河圩湖澤,及大溝長渠,工程浩繁,民力不能獨舉,自應官為經理。其餘零星塘荅,現有管業之人,原皆自行疏濬,朝廷豈能以有限錢糧,為小民代謀畚鍤?」上韙之。河南巡撫雅爾圖言:「豫省水利工程,惟上蔡估建堤壩,系防蔡河異漲之水。其餘汝河、滍河堤堰,應令地主自行修補。至開濬汝河、潁河等工,請停罷以節糜費。」報聞。

六年春,雅爾圖言:「永城地窪積潦,城南舊有渠身長三萬一千餘丈,通澮河,年久淤淺。現乘農隙,勸諭紳民挑濬,俾水有歸。」又言:「前奉諭旨,開濬省城乾涯河,復於中牟創開新河一,分賈魯河水勢,由沙河會乾涯河,以達江南之渦河而匯於淮,長六萬五千餘丈,今已竣工。」賜名惠濟。

九年,御史柴潮生言:「北方地勢平衍,原有河渠澱泊水道可尋。如聽其自盈自涸,則有水無利而獨受其害。請遣大臣齎帑興修。」命吏部尚書劉於義往保定,會同總督高斌,督率辦理。尋請將宛平、良鄉、涿州、新城、雄縣、大城舊有澱渠,與擬開河道,並堤墊涵洞橋閘,次第興工。下廷議,如所請行。先是御史張漢疏陳湖廣水利,命總督鄂彌達查勘。至是疏言:「治水之法,有不可與水爭地者,有不能棄地就水者。三楚之水,百派千條,其江邊湖岸未開之隙地,須嚴禁私築小垸,俾水有所匯,以緩其流,所謂不可爭者也。其倚江傍湖已闢之沃壤,須加謹防護堤塍,俾民有所依以資其生,所謂不能棄者也。其各屬迎溜頂沖處,長堤連接,責令每歲增高培厚,寓疏濬於壅築之中。」報聞。

十一年,大學士署河督劉於義等疏陳慶雲、鹽山續勘疏濬事宜,下部議行。青州瀰河水漲,沖開百餘丈決口,旋堵。博興、樂安積水,挑引河導入溜河。十二年夏,宿遷、桃源、清河、安東之六塘河,及沭陽、海州之沭河,山水漲發,地方被淹,命大學士高斌、總督尹繼善,會同河臣周學健往勘。議於險處加寬挑直,建石橋,開引河,官民協力防護,從之。十三年,湖北巡撫彭樹葵言:「荊襄一帶,江湖袤延千餘里,一遇異漲,必借餘地容納。宋孟珙知江陵時,曾修三海八櫃以瀦水。無如水濁易淤,小民趨利者,因於岸腳湖心,多方截流以成淤,隨借水糧魚課,四圍築堤以成垸,人與水爭地為利,以致水與人爭地為殃。惟有杜其將來,將現垸若干,著為定數,此外不許私自增加。」報聞。十四年,雲南巡撫圖爾炳阿以疏鑿金沙江底績,纂進金沙江志。

十七年,江蘇巡撫莊有恭言:「蘇州之福山塘河,太倉之劉河,乃常熟等八州縣水利攸關,歲久不修,旱澇無備。請於附河兩岸霑及水利各區,按畝酌捐,興工修建。」得旨嘉獎。十八年,陜甘總督黃廷桂言:「巴里坤之尖山子至奎素,百餘里內地畝皆取用南山之水,自山口以外,多滲入沙磧,必用木槽接引,方可暢流。請於甘、涼、肅三處撥種地官兵千名,前往疏濬。」如所請行。以江南、山東、河南積年被水,而山東之水匯於淮、徐,河南之水達於鳳、潁,須會三省全局以治之,命侍郎裘曰修、夢麟往來察閱,會江蘇、安徽、河南各巡撫計議。尋曰修言:「包、澮二河在宿、永連界處,為洩水通商之要道。入安徽境內有石橋六,應加寬展。洪河、睢河與虹縣之柏家河、下江之林子河、羅家河,應補修子堰。鳳臺之裔溝、黑濠、涇泥三河應挑深,使暢達入淮。」夢麟言:「碭山、蕭縣、宿遷、桃源、山陽、阜寧、沭陽共有支河二十餘,應分晰疏濬。」均從之。

二十三年,豫省開濬河道工竣,允紳民請,於永城建萬歲亭,並御制文志之。山東巡撫阿爾泰言:「濟寧、汶上、嘉祥毗連蜀山湖,地畝湮沒約千餘頃,擬將金線、利運二閘啟閉,使湖水濟運,坡水歸湖,可以盡數涸出。」得旨嘉獎。二十四年,濬京師護城河及圓明園一帶河。御史李宜青請疏濬畿輔水源,命直隸總督方觀承條議以聞。觀承言:「東西二澱千里長堤,即宋臣何承矩興堰遺跡。今昔情形有異。倘泥往跡,害將莫救。如就澱言利,則三百餘里中水村物產,視昔加饒,惟遇旱而求通雨澤於水土之氣,則人事有當盡者耳。」四川總督開泰言:「灌縣都江大堰引灌成都各屬及眉、邛二州田畝,寧遠南有大渡河,自冕寧抵會理三口,與金沙江合,支河雜出,堰壩最多,俱應相機修濬。」部議從之。

初,御史吳鵬南請責成興修水土之政,命各督撫經畫。浙江巡撫莊有恭言水之大利五,江、湖、海、渠、泉。他省得其二三,而浙實兼數利。金、衢、嚴三郡,各有山泉溪澗,灌注成渠,堰壩塘蕩,無不具備。惟仁和、錢塘之上中市、三河垸、區塘、苕溪塘,海鹽之白洋河、湯家鋪廟、涇河,長興之東西南漊港,永嘉之七都新洲陡門、九都水湫、三十四都黃田浦陡門,實應修舉,以收已然之利。至杭州臨平湖、紹興夏蓋湖,有關田疇大利,應設法疏挑,或召佃墾種,再體勘辦理。」允之。

二十五年,阿爾泰疏言:「東省水利,以濟運為關鍵,以入海為歸宿。濟、東、泰、武之老黃河、馬頰、徒駭等河,兗、沂、曹之洸、涑等河,共六十餘道,皆挑濬通暢。運河民墊計長七百餘里,亦修整完固。青、萊所屬樂安、平度、昌邑、濰縣、高密等州縣,應挑支河三十餘,俱節次挑竣。萊州之膠萊河,納上游諸水,高密有膠河,亦趨膠萊,易致漫溢,應導入百脈湖,以分水勢。沂州屬蘭、郯境內應開之武城等溝河二十五道,又續挑之響水等溝河二十五道,引窪地之水由江南邳州入運,並已工竣。」帝嘉之。

二十六年,河東鹽政薩哈岱言:「鹽池地窪,全恃姚暹渠為宣洩。近因渠身日高,漲漫南北堤堰禁墻內。黑河實產鹽之本,年久淺溢。涑水河西地勢北高南下,倘汛漲南趨,則鹽池益難保護。五姓湖為眾水所匯,恐下游阻滯,逆行為患。均應及時疏通。」從之。明年,帝南巡,諭曰:「江南濱河阻洳之區,霖潦堪虞,而下游蓄洩機宜,尤以洪澤湖為關鍵。自邵伯以下,金灣及東西灣滾壩,節節措置,特為三湖旁疏曲引起見。若溯源絜要,莫如廣疏清口,乃及今第一義。至六塘河尾閭橫經鹽河以達於海,所有修防事宜,該督、撫、河臣會同鹽政,悉心覈議以聞。」

二十八年,帝以天津、文安、大城屢被霪潦,積水未消,命大學士兆惠督率經理。又以曰修前辦豫省水利有效,命馳往會勘,復命阿桂會同總督方觀承酌辦。阿桂等以「子牙河自大城張家莊以下,分為正、支二河,支河之尾歸入正河,形勢不順。請於子牙河村南斜向東北挑河二十餘里;安州依城河為入澱尾閭,應挑長二千二百餘丈;安、肅之漕河,應挑長三千七百餘丈。其上游之姜女廟,應建滾水石壩,使水由正河歸澱。新安韓家墊一帶為西北諸水匯歸之所,應挑引河十三里有奇」。如所議行。

二十九年,改建惠濟河石閘。修湖北溪鎮十里長堤,及廣濟、黃梅江堤。濬江都堰,開支河一,使漲水徑達外江。三十二年,修築澱河堤岸,自文安三灘里至大城莊兒頭,長二千七百餘丈。山東巡撫崔應階言:「武定近海地窪,每遇汛漲,全恃徒駭、馬頰二河分流入海。徒駭下游至霑化入海處,地形轉高,難議興挑。勘有壩上莊舊漫口河形地勢順利,應開支河,俾兩道分洩。」江蘇巡撫明德言:「蘇州南受浙江諸山經由太湖之水,北受揚子江由鎮江入運之水,伏秋汛發,多致漫溢。請修吳江、震澤等十縣塘路。」均從之。

三十三年,滹沱水漲,逼臨正定城根,添築城西南新堤五百七十餘丈,回水堤迤東築挑水壩五。河神祠前築魚鱗壩八十丈。槁城東北兩面,滹水繞流,順岸築埽三百六十丈,埽後加築土墊。三十五年,挑濬蘇郡入海河道,白茆河自支塘鎮至滾水壩,長六千五百三十餘丈;徐六涇河自陳蕩橋至田家壩,長五千九百九十餘丈。三十六年,濬海州之薔薇、王家口、下坊口、王家溝四河。以直隸被水,命侍郎袁守侗、德成分往各處督率疏消。尚書裘曰修往來調度,總司其事。山東巡撫徐績查勘小清河情形,請自萬丈口挑至還河口,計四十里,使正、引兩河分流,由河入泊,由泊達溝歸海。詔如所議行。廣西巡撫陳輝祖言:「興安陡河源出海陽山,至分水潭,舊築鏵嘴以分水勢,七分入湘江為北陡,三分入漓江為南陡,於進水陡口內南北建大小天坪,以資蓄洩,復建梅陽坪,以遏旁行故道,並以引灌糧田。近因連雨沖陷,請修復土石各工。」下部知之。

三十八年,挑濬禹城漯河、高密百脈湖引河。四十年,修築武昌省城金河洲、太乙宮濱江石岸。江南旱,高、寶皆歉收。總督高晉,河督吳嗣爵、薩載合疏言:「嗣後洪湖水勢,應以高堰志椿為準,各閘壩涵洞相機啟放,總使運河存水五尺以濟漕,餘水侭歸下河以資灌溉。」從之。四十一年,修西安四十七州縣渠堰共千一百餘處。總督高晉言:「瓜洲城外查子港工接連回瀾壩,江岸忽於六月裂縫,坍塌入江約百餘丈,西南城墻塌四十餘丈。現在水勢已平,擬將瓜洲量為收進,讓地於江,並沿岸築土壩以通纖路。」諭令妥善經理。

四十二年,山西巡撫覺羅巴延三言:「太原西有風峪口,旁俱大山,大雨後山水下注縣城,猝難捍禦。請自峪口起,開河溝一,直達汾水,所占民田止四十餘畝,而太原一城可期永無水患。」四十三年,疏濬湖州漊港七十二。修昌邑海堤,居民認墾堤內鹼廢地千二百餘頃。濬鎮洋劉河,自西陳門涇上頭起,至王家港止。四十四年,改建宣化城外柳川河石壩,並添築石坦坡。漳河下游沙莊壩漫口,淹及成安、廣平,水無歸宿。於成安柏寺營至杜木營,繞築土墊千一百餘丈。

四十七年,雲南巡撫劉秉恬言:「鄧川之瀰苴河,上通浪穹,下注洱海,中分東西兩湖。東湖由河入海,河高湖低,每遇夏秋漲發,回流入湖,淹沒附近糧田。紳民倡捐,將湖尾入海處堵塞,另開子河,引東湖水直趨洱海,又自青石澗至天洞山,築長堤、建石閘,使河歸堤內,水由閘出,歷年所淹田萬一千二百餘畝,全行涸出。」得旨嘉獎。又言:「楚雄龍川江自鎮南發源,入金沙江。近年河溜逼城,請於相近鎮水塔挑濬深通,導引河溜復舊。又澂江之撫仙湖下游,有清水、渾水河各一,渾水之牛舌石壩被沖,匯流入清,以致為害。請於牛舌壩東另開子河,以洩渾水,並將河身改直,使清水暢達。」上獎勉之。

五十年,河南巡撫何裕城言:「衛河歷汲、淇、滑、濬四縣,濱河田畝,農民築堤以防淹浸,不能導河灌田。輝縣百泉地勢卑下,而獲嘉等縣較高,難以紆回導引。其餘汲縣、新鄉並無泉源,祗有鑿井一法,既可灌田,亦藉以通地氣,已派員試開。」濬賈魯、惠濟兩河。修寧夏漢延、唐來、大清、惠農四渠。五十一年,山東商人捐資挑濬鹽河,並於東阿、長清、齊河、歷城建閘八。

五十三年,荊州萬城堤潰,水從西北兩門入,命大學士阿桂往勘。尋疏言:「此次被水較重,土人多以下游之窖金洲沙漲逼溜所致,恐開挑引河,江水平漾無勢,仍至淤閉。請於對岸楊林洲靠堤先築土壩,再接築雞嘴石壩,逐步前進,激溜向南,俟洲坳刷成兜灣,再趁勢酌挑引河,較為得力。」報聞。五十四年,濬通惠河、朝陽門外護城河及溫榆河。五十五年,培修千里長堤,瀦龍河、大清河、盧僧河等堤,鳳河東堤,及西沽、南倉、海河等疊道,改建豐城東西是石工。築潛江仙人舊堤千二百八十餘丈。挑濬永城洪河。

五十七年,兩江總督書麟等言:「瓜洲均系柴壩,江流溜急,接築石磯,不能鞏固。請於回瀾舊壩外,拋砌碎石,護住埽根,自裹頭坍卸舊城處所靠岸,亦用碎石拋砌,上面鑲埽。嗣後每年挑溜,可期溜勢漸遠。」得旨允行。又言:「無為州河形兜灣,應將永成圩壩加築寬厚。擬於馬頭埂開挖河口三十丈,曾家腦至東圩壩舊河亦展寬三十丈,俾河流順暢。」上韙之。改蕭山荷花池堤為石工,堵河內民堰漫口五十餘丈,修復豐城江岸石堤。五十九年,荊州沙市大壩,因江流激射,勢露頂沖,添建草壩。

嘉慶五年,挑濬檿牛河、黃家河,及新安、安、雄、任丘、霸、高陽、正定、新樂八州縣河道。六年,京師連日大雨,撥內帑挑濬紫禁城內外大城以內各河道,及圓明園一帶引河。文安被水,命直督陳大文詳議。疏言:「文地極窪,受水淺,地與河平,自建治以來,別無疏濬章程。惟查大城河之廣安橫堤,為文邑保障,迤南有河間千里長堤,可資外衛。兩堤之中,有新建閘座,以洩河間漫水。再於地勢稍下之龍潭灣,開溝疏濬,或不致久淹。」從之。

八年,伊犁將軍松筠言:「伊犁土田肥潤,可耕之地甚多,向因乏水,今擬設法疏渠引泉,以資汲灌。應請廣益耕屯,以裕滿兵生計,並借官款備辦耕種器物。」如所請行。十一年,疏築直隸千里長堤,及新舊格澱堤。十二年,湖廣總督汪志伊言:「堤垸保衛田廬,關系緊要。漢陽等州縣均有未涸田畝,未築堤塍。應亟籌勘辦,以興水利而衛民田。」從之。十六年,以畿輔災歉,命修築任丘等州縣長堤,並雄縣疊道,以工代賑。十七年,濬武進孟瀆河。挑阜寧救生河,太倉劉河。修天津、靜海兩縣河道。濬東平小清河,及安流、龍拱二河,民便河。十八年。江南河道總督初彭齡疏陳江省下河水利,宜加修理。得旨允行。十九年,大名、清豐、南樂三縣七十餘莊地畝,久為衛水淹沒,村民自原出夫挑挖,請官為彈壓。御史王嘉棟疏言:「杭、嘉、湖被旱歉收,請開濬西湖,以工代賑。」皆允之。二十一年,疏濬吳淞江。二十二年,章丘民言,長白、東嶺二山之水,向歸小清河入海。自灰壩被沖,水歸引河,章丘等縣屢被水災。命禮部侍郎李鴻賓往勘。次年,巡撫陳預疏言:「小清河以章丘、鄒平、長山、新城為上游,高苑、博興、樂安為下游,正河及支派溝多有淤墊。請先疏濬上游,並將滸山等二泊一湖挑挖寬深,則水勢不至建瓴直注,下游亦不驟虞漫溢。」得旨允行。建沔陽石閘,挑引渠,以時啟閉。

二十五年,修都江堰。御史陳鴻條陳興修水利營田事宜,命直隸、山東、山西、河南各督撫一體籌畫興舉。修襄陽老龍石堤。庫車辦事大臣嵩安疏報別什托固喇克等處挑渠引水,墾田五萬三千餘畝。有詔褒勉。

道光元年,修湖州黑窯廠江堤,濬涇陽龍洞渠、鳳陽新橋河。二年,加築襄陽老龍石堤。濬正定柏棠、護城、洩水、東大道等河,並修斜角、回水等堤。興修杭州北新關外官河纖道。直隸總督顏檢請築滄州捷地減河閘壩,濬青縣、興濟兩減河,修通州果渠村壩墊。皆如議行。疏濬銅山荊山橋河道,及南鄉奎河。挑江都三汊河子、鹽河五閘淤淺,及沙漫州江口沙埂。修豐城及新建惠民橋堤。三年,修汾河堤堰,並移築李綽堰,改挖河身。修天門、京山、鍾祥堤垸,及監利櫻桃堰、荊門沙洋堤。挑挖熱河旱河,並添修荊條單壩。堵文安崔家窯、崔家房漫口。修河東鹽池馬道護堤,並濬姚暹渠、李綽堰、涑水河。刑部尚書蔣攸銛言:「上年漳河漫水下流,由大名、元城直達紅花堤,潰決堤墊,由館陶入衛,應亟籌議。」命大學士戴均元馳勘。尋奏言:「元城引河穿堤入衛,河身窄狹,應挑直展寬,以暢其流。紅花堤以下新刷水溝五百餘丈,應挑成河道,以期分洩。」又:「漳自南徙合洹以來,衛水為其頂阻,每遇異漲,民墊不能捍禦,以致安陽、內黃頻年沖決。今漳北趨,業已分殺水勢。擬於樊馬坊、陳家村河幹北岸築壩堵截,使分流歸並一處。自柴村橋起,接連洹河北岸,建築土壩,樊馬坊以下王家口添築土格土壩,以免串流南趨,使漳、洹不致再合。」詔皆從之。

四年,築德化、建昌、南昌、新建四縣圩堤。修培荊州萬城大堤橫塘以下各工,及監利任家口、吳謝垸漫決堤塍。給事中硃為弼請疏浚劉河、吳淞,及附近太湖各河。御史郎葆辰請修太湖七十二漊港,引苕、霅諸水入湖以達於海。御史程邦憲請擇太湖洩水最要處所,如吳江堤之垂虹橋、遺愛亭、龐山湖,疏剔沙淤,剷除蕩田,令東注之水源流無滯。先後疏入,命兩江總督孫玉庭、江蘇巡撫韓文綺、浙江巡撫帥承瀛會勘。玉庭等言:「江南之蘇、松、常、太,浙江之杭、嘉、湖等屬,河道淤墊,遇漲輒溢。現勘水道形勢,疆域雖分兩省,源委實共一流。請專任大員統治全局。」命江蘇按察使林則徐綜辦江、浙水利。

御史陳澐疏陳畿輔水利,請分別緩急修理。給事中張元模請於趙北口連橋以南開橋一座,以古趙河為引河,並挑北盧僧河,以分減白溝之獨流。帝命江西巡撫程含章署工部侍郎,辦理直隸水利,會同蔣攸銛履勘。含章請先理大綱,興辦大工九。如疏天津海口,濬東西澱、大清河,及相度永定河下口,疏子牙河積水,復南運河舊制,估修北運河,培築千里長堤,先行擇辦。此外如三支、黑龍港、宣惠、滹沱各舊河,沙、洋、洺、滋、洨、唐、龍鳳、龍泉、瀦龍、檿牛等河,及文安、大城、安州、新安等堤工,分年次第辦理。又言勘定應濬各河道,塌河澱承六減河,下達七里海,應挑寬罾口河以洩北運、大清、永定、子牙四河之水入澱。再挑西堤引河,添建草壩,洩澱水入七里海,挑邢家坨,洩七里海水入薊運河,達北塘入海。至東澱、西澱為全省瀦水要區,十二連橋為南北通途,亦應擇要修治。均如所請行。濬虞城惠民溝,夏邑巴清河、永城減水溝。玉庭言:「三江水利,如青浦、婁縣、吳江、震澤、華亭承太湖水,下注黃浦,各支河淺滯淤阻,亟應修砌。吳淞江為太湖下注幹河,由上海出閘,與黃浦合流入海。因去路阻塞,流行不暢,應於受淤最厚處大加挑浚。」得旨允行。

五年,陜西巡撫盧坤疏報咸寧之龍首渠,長安之蒼龍河,涇陽之清、冶二河,盩厔之澇、峪等河,郿縣之井田等渠,岐山之石頭河,寶雞之利民等渠,華州之方山等河,榆林之榆溪河、芹河,均挑濬工竣,開復水田百餘頃至數百頃不等。修監利江堤,襄陽老龍石堤。已革御史蔣時進畿輔水利志百卷。直隸總督蔣攸銛疏陳防守千里長堤善後事宜,報聞。安陽、湯陰廣潤陂,屢因漳河決口淤墊,命巡撫程祖洛委員確勘挑渠,將積水引入衛河,使及早涸復。築荊州得勝臺民堤。

七年,閩浙總督孫爾準言:「莆田木蘭陂上受諸渠之水,下截海潮,灌溉南北洋平田二十餘萬畝。近因屢經暴漲,泥沙淤積,陡門石堤損壞,以致頻歲歉收。現經率同士民捐資修培南北兩岸石工告竣。」得旨嘉獎。濬漢川草橋口、消渦湖口水道。御史程德潤言荊山王家營屢決,下游各州縣連年被災。請飭相度修築。命湖廣總督嵩孚籌議,因請仿黃河工程切灘法,平其直射之溜勢,再將下游沙洲開挑引河,破其環抱,以順正流。帝恐與水爭地,虛糜無益,命刑部尚書陳若霖等往勘。覆言:「京山決口三百二十餘丈,鍾祥潰口百七十餘丈,正河經行二百餘年,不應舍此別尋故道。惟有挑除胡李灣沙塊,先暢下游去路,將京山口門挽築月堤,展寬水道,鍾祥口門于堵閉後,添築石壩二,護堤攻沙。」帝韙之,命嵩孚駐工督辦。

八年,河南巡撫楊國楨言:「湯河、伏道河並廣潤陂上游之羑河、新惠等河,向皆朝宗於衛,因故道久湮,頻年漫溢。現為一勞永逸之計,因勢利導,悉令暢流。又南陽白河、淅川、丹江水勢浩瀚,俱切近城根,亟應築碎石、磨盤等壩二十餘道,分別挑溜抵御。」均如所請行。挑濬冀州東海子淤塞溝身,以工代賑。

九年,修宿遷各河堤岸,丹陽下練湖閘壩。濬宿州奎河。築喀什噶爾新城沿河堤岸。兩江總督蔣攸銛言:「徐州河道,如蕭縣龍山河,邳州睢寧界之白塘河,邳州舊城民便河,碭山利民、永定二河,又沛縣堤工,邳州沂河民墊,豐縣太行堤,皆最要之工,請次第估辦興挑。」從之。十年,修湖北省會江岸,並添建石壩。挑濬漳河故道。修保定南關外河道,及徐河石橋、河間陳家門堤。濬東平小清河,及安流、龍拱二河。修公安、監利堤。

十一年,修南昌、新建、進賢圩堤,及河間、獻縣河堤,天門漢水南岸堤工。桐梓被水,開濬戴家溝河道。命工部尚書硃士彥察勘江南水患,疏請修築無為及銅陵江壩。給事中邵正笏言江湖漲灘占墾日甚,諭兩江總督陶澍、湖廣總督盧坤等飭屬詳勘,其沙洲地畝無礙水道者,聽民認墾,否則設法嚴禁。十二年,挑除星子蓼花池淤沙,疏通溝道,並築避沙塹壩。修築南昌、新建圩堤,又改豐城土堤為石。

十三年,湖廣總督訥爾經額請修襄陽老龍及漢陽護城石堤,武昌、荊州沿江堤岸。兩江總督陶澍請修六合雙城、果盒二圩堤埂,濬孟瀆、得勝、灣港三河,並建閘座。均如議行。戶部請興修直隸水利城工,命總督琦善確察附近民田之溝渠陂塘,擇要興修,以工代賑。御史硃逵吉言,湖北連年被水,請疏江水支河,使南匯洞庭湖,疏漢水支河,使北匯三臺等湖,並疏江、漢支河,使分匯雲夢,七澤間堤防可固,水患可息。御史陳誼言,安陸濱江堤塍沖決為害,請建五閘壩,挑濬河道,以洩水勢。疏入,先後命訥爾經額、尹濟源、吳榮光等遴員詳勘。

十四年,修良鄉河道橋座。濬沔陽天門、牛氾支河,漢陽通順支河,並修築濱臨江、漢各堤。濬石首、潛江、漢川支河,修荊州萬城大堤,華容等縣水沖官民各垸。濬碭山利民、永定二河。築南昌、新建、進賢、建昌、鄱陽、德安、星子、德化八縣水淹圩堤。修潛江、鍾祥、京山、天門、沔陽、漢陽六州縣臨江潰堤,以工代賑。修邳、宿二州縣沂河堤墊、及王翻湖等工。濬太倉、七浦及太湖以下泖澱,並修元和南塘寶帶橋。

十六年,濬河東姚暹渠。修庫車沿河堤壩。濬海鹽河道。又貸江蘇司庫銀濬鹽城皮大河、豐縣順堤河,並修築堤工,從兩江總督林則徐等請也。命大學士穆彰阿、步軍統領耆英、工部尚書載銓,勘估京城內外應修河道溝渠。十七年,修武昌沿江石岸,鍾祥劉公菴、何家潭老堤,潛江城外土堤,及豐城土石堤工,並建小港口石閘石埽。十八年,修黃梅堤。濬豐潤、玉田黑龍河。

十九年,修武昌保安門外江堤,蘄州衛軍堤,漢陽臨江石堤。葉爾羌參贊大臣恩特亨額覆陳巴爾楚克開墾屯田情形。先是,帝允伊犁將軍特依順保之請,命於巴爾楚克開墾屯田。嗣署參贊大臣金和疏陳不便,復命恩特亨額詳籌。至是,疏言:「該處渠身僅三百二十八里有奇,沿堤兩岸培修,水勢甚旺,足資灌溉。並派屯丁分段看守,遇水漲時,有渠旁草湖可洩,不致淹漫要路。」諭:「照舊妥辦,務於屯務邊防實有裨益。」伊犁將軍關福疏報,額魯特愛曼所屬界內塔什畢圖,開正渠二萬五千七百餘丈,計百四十餘里,得地十六萬四千餘畝,實屬肥腴,引水足資灌溉。詔褒勉之。

是歲漢水盛漲,漢川、沔陽、天門、京山堤垸潰決。二十年,總督周天爵疏報江、漢情形,擬疏堵章程六:一,沙灘上游作一引壩,攔入湖口,再作沙是障其外面,以堵旁洩;一,江之南岸改虎渡口東支堤為西堤,別添新東堤,留寬水路四里餘,下達黃金口,歸於洞庭,再於石首調弦口留三四十里沮洳之地,瀉入洞庭;一,江之北岸舊有閘門,應改為滾壩,冬啟夏閉;一,襄陽上游多作挑壩,撐水外出,再於險要處所,加築護堤護灘;一,襄陽河四面堤畔,應用磚石多砌陡門,夏令相機啟閉;一,襄河水勢浩大,應添造滾壩,冬啟夏閉,於兩岸低窪處所,引渠納水。下所司議行。是年修華容、武陵、龍陽、沅江四縣官民堤垸,又修荊州大堤,及公安、監利、江陵、潛江四縣堤工。

二十二年,堵鹿邑渦河決口。先是,黃水決口,大溜直趨渦河,將南岸觀武集、鄭橋、劉窪莊、古家橋及淮寧之閻家口、吳家橋、徐家灘、婁家林、季家樓堤頂漫塌,太和民田悉成巨浸,阜陽以次州縣亦被漫淹。至是,安徽巡撫程楙採言:「豫工將次合龍,渦河決口若不及時興修,下游受害益深。請敕河南撫臣迅籌堵築。」從之。湖廣總督裕泰等疏報江水盛漲,沖陷萬城堤以上之吳家橋水閘,並決下游上漁埠頭大堤,直灌荊州郡城,倉庫監獄均被淹漫。水消退後,而埠頭漫口較寬,勢難對口接築。擬修挽月堤一,並先於上下游各築橫堤一。如所請行。修築庫倫堤壩,及鄒縣橫河口、李家河口民堰。

二十三年,直隸總督訥爾經額疏陳直隸難以興舉屯政水利,略云:「天津至山海關,戶口殷繁,地無遺利。其無人開墾之處,乃沿海鹼灘,潮水咸濇,不足以資灌溉。至全省水利,歷經試墾水田,屢興屢廢,總由南北水土異宜,民多未便。而開源、疏泊、建閘、修塘,皆需重帑,未敢輕議試行。但宜於各境溝洫及時疏通,以期旱澇有備,或開鑿井泉,以車戽水,亦足裨益田功。」如所議行。修海陽寮哥宮、涸溪、竹崎頭堤工。

二十四年,修江夏江堤。濬海州沭河。七月,荊州江勢汎漲,李家埠內是決口,水灌城內。江陵虎渡口汛江支各堤亦多漫溢。諭總督裕泰籌款修築。九月,萬城大堤合龍。伊犁將軍布彥泰等言:「惠遠城東阿齊烏蘇廢地可墾復良田十餘萬畝,擬引哈什河水以資灌注,將塔什鄂斯坦田莊舊有渠道展寬,接開新渠,引入阿齊烏蘇東界,並間段酌挑支河。」又言:「伊拉里克地畝與喀喇沙爾屬蒙古游牧地以山為界,該處河水一道,由山之東面流出,距游牧地尚隔一山,於蒙古生計無礙,堪以開墾。請濬大渠支渠並洩水渠,引用伊拉里克河水。」又言:「奎屯地方寬廣,有河一道,系由庫爾喀喇烏蘇南山積雪融化匯流成河,近水地畝早有營屯戶民承種。又蘇沁荒地有萬餘畝,土脈肥潤,祗須挑渠引水,可以俱成沃壤。」均如所請行。

二十五年,濬賈魯河,修汶上馬踏湖民堰。命喀喇沙爾辦事大臣全慶查勘和爾罕水利,疏言:「和爾罕地本膏腴,宜將西北哈拉木扎什水渠並東南和色熱瓦特大渠接引,可資耕種。中隔大小沙梁,業已挑通,宜於沖要處砌石釘椿,使沙土不致坍卸,渠道日深,足以灌溉良田。」又言:「伊拉里克地居吐魯番所轄托克遜軍臺之西,土脈腴潤,謂之板土戈壁,其西為沙石戈壁。二百餘里,至山口出泉處,有大阿拉渾、小阿拉渾兩水,匯成一河。從前渠道未開,水無收束,一至沙石戈壁,散漫沙中,而板土戈壁水流不到,轉成荒灘。今將極西之水導引而東,在沙石戈壁鑿成大渠三段,復於板土戈壁多開支渠,即遇大汛,水有所歸。又吐魯番地畝多系掘井取泉,名曰卡井,連環導引,其利甚溥。惟高埠難引水逆流而上,應聽戶民自行挖井,冬春水微時,可補不足。」下廷臣議行。

二十六年,烏魯木齊都統惟勤請修理喀喇沙爾渠道壩堤,並陳章程四,命伊犁將軍薩迎阿覆覈,尚無流弊,詔如所請行。六塘河堤沖潰,各州縣連年被水,命兩江總督璧昌等覈辦。覆言,海州境內六塘河及薔薇河淤墊沖決,田廬受淹,於運道宣防,大有關系,應從速借款挑築,允之。修溫榆河果渠村壩埽。二十七年,扎薩克郡王伯錫爾呈獻私墾地畝,內有生地四千八百三十餘畝,接濬新渠二,添開支渠二,以資分灌。

二十八年,兩江總督李星沅請修沛縣民墊埽壩,裕泰請修江夏堤工、鍾祥廖家店外灘岸,直隸總督訥爾經額請修築萬全護城石壩,均如所請。御史楊彤如劾河南撫臣三次挑挖賈魯河決口,費幾百萬,迄無成功,請敕查辦。詔褫鄂順安以下職。新任巡撫潘鐸疏言:「賈魯河工程應以復硃仙鎮為修河關鍵。惟硃仙鎮內及街南北河道淤墊最甚,今議添辦柴稭埽工,以防兩岸淤沙。其淤沙最深處,挑濬較難,另擇乾土十數里,改道以通舊河,責成各員賠修,限四十五日工竣。」從之。

二十九年,江蘇巡撫傅繩勛言:「陰雨連綿,積水無從宣洩,以致江、淮、揚等屬堤圩多被沖破。請仿農政全書櫃田之法,以土護田,堅築高峻,內水易於車涸,勸民舉行,以工代賑,並查勘海口,開挖閘洞洩水。」帝嘉勉之。三十年,修襄陽老龍石堤,及漢陽堤壩,武昌沿江石岸,潛江土堤、鍾祥高家堤。御史汪元方以浙江水災,多由棚民開山,水道淤阻所致,疏請禁止。諭巡撫吳文鎔嚴查,並命江蘇、安徽、江西、湖廣各督撫一體稽查妥辦。

咸豐元年,浙江巡撫常大淳疏陳清理種山棚民情形,略言:「浙西水利,餘杭、南湖驟難濬復,應先開支河、修石閘,以資蓄洩。上游治而下游之患亦可稍平。浙東則紹興之三閘口外,鄞縣、象山等河溪,現經籌挑。」報聞。三年,太常卿唐鑒進畿輔水利備覽,命給直隸總督桂良閱看,並著於軍務告竣時,酌度情形妥辦。

同治元年,御史硃潮請開畿輔水利,並以田地之治否,定府縣考績之殿最。命直隸總督文煜等將所轄境內山泉河梁澱湖及可開渠引水地方詳查,並妥議章程。尋覆疏言:「有可舉行之處,或礙於地界,或限於力量,或當掘井制車,或須抽溝築圩,均設法催勸,推行盡利。」三年,江蘇士民殷自芳等以「山陽、鹽城境內市河、十字河、小市河蜿蜒百里,東注馬家蕩,沿河民田數千頃,旱則資其灌溉,潦則資其宣洩。自乾隆六年大挑以後,迄今百餘年,河淤田廢,水旱均易成災。墾請挑濬築墟,引運河水入市河,以蘇民困」。命兩江總督、江蘇巡撫覈辦。

五年,御史王書瑞言,浙江水利,海塘而外,又有漊港。烏程有三十九水婁,長興有三十四漊。自逆匪竄擾後,泥沙堆積,漊口淤阻,請設法開濬。又言蘇、松諸郡與杭、嘉、湖異派同歸,湖州處上游之最要,蘇、松等郡處下游之最要。上游阻塞,則害在湖州,下游阻塞,則害在蘇、松,並害及杭、嘉、湖。請飭江蘇一並勘治。從之。六年,濬清河張福口引河。八年,安徽巡撫吳坤修言,永城與宿州接壤之南股河,久經淤塞,下接靈壁,低窪如釜,早成巨浸,水無出路,擬查勘籌辦。從之。

九年,濬白茆河道,改建近海石閘。江蘇紳民請濬復淮水故道,命兩江總督、江蘇巡撫、漕運總督會籌。覆疏言:「挽淮歸故,必先大濬淤黃河,以暢其入海之路,繼開清口,以導其入黃,繼堵成子河、張福口、高良澗三河,以杜旁洩。應分別緩急興工,期以數年有效。」下部議,從之。是年內閣侍讀學士鍾佩賢亦以疏濬海港為請。於是浙撫楊昌濬言:「漊港年久淤塞,查明最要次要各工,分別估修,擬趁冬隙時,先將寺橋等九港及諸、沈二漊趕辦,其餘各工及碧浪湖工程,次第籌畫,應與吳江長橋及太湖出水各口同時修濬。」得旨允行。

十年,修龍洞舊渠,並開新渠以引涇水。江蘇巡撫張之萬請設水利局,興修三吳水利。於是重修元和、吳縣、吳江、震澤橋竇各工。最大者為吳淞江下游至新閘百四十丈,別以機器船疏之。凡太倉七浦河,昭文徐六涇河,常熟福山港河、常州河,武進孟瀆、超瓢港,江陰黃田港、河道塘閘、徒陽河、丹徒口支河,丹陽小城河,鎮江京口河,均以次分年疏導,幾及十年,始克竣事。先是侯家林決口,河督喬松年以為時較晚,請來年冬舉辦。至是,巡撫丁寶楨言,此處決口不堵,必致浸淹曹、兗、濟十餘州縣,若再向東南奔注,則清津、里下河一帶更形吃重,請親往督工堵築。詔獎勉之。

十二年,以直隸河患頻仍,命總督李鴻章仿雍正間成法,籌修畿輔水利。旋議定直隸諸河,皆以澱池為宣蓄。西澱數百里河道,為民生一大關鍵,先堵趙村決口,築磁河、瀦龍河南堤,以禦外水,挑濬盧僧、中亭兩河,分減大清河水勢,以免倒灌。並疏通趙王河道,將茍各莊以上巨堤及下口鷹嘴壩各建閘座。是年秋,直隸運河堤決,內閣學士宋晉請擇修各河渠,以工代賑,從之。十三年,挑濬天津陳家溝至塌河澱邊減河三千七百餘丈,又自塌河澱循金鐘河故道斜趨入薊運河,開新河萬四千一百餘丈,俾通省河流分溜由北塘歸海。石莊戶決口,奪溜南趨,命寶楨速籌堵築。旋以決口驟難施工,請在迤下之賈莊建壩堵合,即於南北岸普築長堤。而北岸濮州之上游為開州,並飭直督合力籌辦。

光緒元年,濬文安勝芳河,修菏澤賈莊南岸長堤及北岸金堤。二年,濬張家橋新舊泗河。三年,濬濟寧夏鎮迤南十字河。給事中夏獻馨請修水利以裕民食,諭各督撫酌奪情形,悉心區畫。四年,修補濱江黃柏山至樊口四十里老堤,並於樊口內建石閘。五年,修都江堰堤,灌縣、溫江、崇慶舊淹田地涸復八萬二千餘畝。

七年,挑濬大清河下游,使水暢入東澱,並於獻縣硃家口古羊河東岸另闢滹沱減河,使水歸子牙河故道,達津入海。濬寶坻、武清境內北運減河。大學士左宗棠請興辦順直水利,以陜甘應餉之軍助直隸治河之役。總督李鴻章言:「近畿水利,受病過深,凡永定、大清、滹沱、北運、南運五大河,及附麗之六十餘支河,原有閘壩堤墊,無一不壞,減河引河,無一不塞,而節宣諸水之南泊、北泊、東澱、西澱,早被濁流填淤,僅恃天津三岔口一線海河,迤邐出口。平時既不能暢消,秋冬海潮頂托倒灌,節節皆病。修治之法,須先從此入手。五大河中,以永定之害為最深。其大清、北運、南運,須分別挑濬築堤,修復減河。滹沱趨向無定,自來未設堤防。同治七年,由槁城北徙,以文安大窪為壑,其故道之難復,上游之難分,下游之難洩,曾國籓與臣詳陳有案。東西澱寬廣數百里,淤泥厚積,人力難施。頻年以來,修復永定河金門閘壩,裁灣切灘,加築是段。大清河則於新、雄境內開盧僧減河,霸州、文安境內接開中亭、勝芳等河,分洩上游盛漲;於任丘開趙王減河,分洩西澱盛漲;又於文安左各莊至臺頭挑河身二十餘里,以暢下游去路。滹沱河則於河間及文安挖開引河二,又於獻縣硃家口另闢減河三十餘里,均歸子牙河達津。北運河則於通州築壩,挽潮白河歸槽,於香河王家務、武清筐兒港修復石壩,以洩漲水,於天津霍家嘴疏濬引河,以通下口。又於武清、寶坻挑挖王家務、筐兒港兩減河,以資暢洩。南運河則於青、滄、靜海修堤二百餘里,於靜海新官屯另闢減河六十餘里,使別途出海。又於天津城東永定、大清、滹沱、北運交會之陳家灣,開河百餘里,分洩四大河之水,逕達北塘入海。其無極、蠡、博、高陽一帶,則堅築珠龍河是,以防滹沱北越。任丘至天津一帶,則加築千里堤、格澱堤,使河自河而澱自澱。又於廣平開洺河,順德挑澧河,趙州濬、槐、午諸河。此外河道受害較深者,均酌量疏築。今宗棠請以隨帶各營移治上游,正可輔直隸之不逮。此後應修何處,當隨時會商,實力襄助。」疏入,命恭親王奕、醇親王奕枻會同辦理。是年加修子牙河堤萬七千四百餘丈,文安西堤二千九百餘丈,展寬靜海東堤二千四百餘丈。

九年,安徽學政徐郙言:「江、皖兩省水患頻仍,亟須挑泗、沂為導淮先路,仿抽溝法,循序疏治,由大通口引河入海,洩水較易。」命宗棠、昌濬會商籌辦。尋疏覆言:「天下無有利無害之水,疏舊黃河,分減泗、沂,近年已著成效,自當加挑寬深,兼疏大通口以暢出海之途,設復淮局於清江,派員提調。估計分年分段興辦,去其太甚之害,留其本然之利。江北於皖省為下游,下游利,上游自無不利矣。」報聞。

十年,河南巡撫鹿傳霖言:「豫省地勢平衍,衛、淇、沁、潭襟帶西北,淮、汝、渦、潁交匯東南,如果一律疏通,加以溝渠引灌,農田大可受益。今河道半皆壅滯,溝渠亦多荒廢,擬借人力以補天災,派員分赴各州縣履勘籌畫,或疏或濬,志在必成,使民間曉然於有利農田,自能踴躍用命。」詔如所請行。宗棠言:「興修江南水利各工,最大者為硃家山、赤山湖。硃家山自浦口至張家堡,接通滁河,綿亙百二十餘里。赤山湖自道士壩、蟹子壩至三汊河下游,亦綿亙百二十里。兩年工竣,不惟沿江圩田均受其利,而糧艘貨船亦可由內河行,尤屬農商兩便。」下部知之。十一年七月,以張曜所部十營、馮南斌二營、蔣東才四營,濬京師內外護城河,十一月竣工。十三年,河決鄭州,全溜注淮,因濬張福口引河,及興化之大周閘河、丁溪場之古河口、小海三河,俾由新陽、射陽等河入海。十四年,鑿廣西江面險灘,由蒼梧迄陽朔七百餘里,共開險灘三十五。

十六年,江蘇巡撫剛毅以寶山蘊藻河道失修,迤西大壩壅遏水脈,請興工挑築。給事中金壽松言利少害多,命總督曾國荃妥籌。覆疏言,擬拆去同治間所築土壩,以通嘉定、寶山之水道,仍規復咸豐間所建舊閘,以還嘉定之水利。另開引河以通河流,俾得隨時宣洩。下部知之。挑濬餘杭南湖,並疏濬苕溪。華州羅紋河下游各村連年遭水,沿河數百頃良田盡成澤國。巡撫鹿傳霖請由吳家橋北大荔之胡村,開渠引水注渭,則其流舒暢,被淹民田,即可涸復耕作,從之。

給事中洪良品以直隸頻年水災,請籌疏濬以興水利。事下總督籌議。鴻章言:「原奏大致以開溝渠、營稻田為急,大都沿襲舊聞,信為確論,而於古今地勢之異致,南北天時之異宜,尚未深考。夫以太行左轉,西北萬峰矗天,伏秋大雨,口外數千里千溪萬派之水,奔騰而下,畿南一帶地平土疏,頃刻輒漲數尺或一二丈,沖蕩泛溢,勢所必然。聖祖慮清濁河流之不可制也,乃築千里堤、格澱是,使澱與子牙河各行一路。世宗慮永定河南行之淤澱也,令引渾河別由一道,改移下口。其餘官堤民堤,今昔增築,綜計不下三四千里,沙土雜半,險工林立,每當伏秋盛漲,兵民日夜防守,甚於防寇,豈有放水灌入平地之理?今若語沿河居民開渠引水,鮮不錯愕駭怪者。且水田之利,不獨地勢難行,即天時亦南北迥異。春夏之交,布秧宜雨,而直隸彼時則苦雨少泉涸。今釜陽各河出山處,土人頗知鑿渠藝稻。節屆芒種,上游水入渠,則下游舟行苦淺,屢起訟端。東西澱左近窪地,鄉民亦散布稻種,私冀旱年一穫,每當伏秋漲發,輒遭漂沒。此實限於天時,斷非人力所能補救者也。以近代事考之,明徐貞明僅營田三百九十餘頃,汪應蛟僅營田五十頃,董應舉營田最多,亦僅千八百餘頃,然皆黍粟兼收,非皆水稻。且其志在墾荒殖穀,並非藉減水患。今訪其遺跡,所營之田,非導山泉,即傍海潮,絕不引大河無節制之水以資灌溉,安能藉減河水之患,又安能廣營多穫以抵恃大之入?雍正間,怡賢親王等興修直隸水利,四年之間,營治稻田六千餘頃,然不旋踵而其利頓減。九年,大學士硃軾、河道總督劉於義,即將距水較遠、地勢稍高之田,聽民隨便種植。可見直隸水田之不能盡營,而踵行擴充之不易也。恭讀乾隆二十七年上諭『物土宜者,南北燥濕,不能不從其性。儻將窪地盡改作秧田,雨水多時,自可藉以儲用,雨澤一歉,又將何以救旱?從前近京議修水利營田,始終未收實濟,可見地利不能強同』。謨訓昭垂,永宜遵守。即如天津地方,康熙間總兵藍理在城南墾水田二百餘頃,未久淤廢。咸豐九年,親王僧格林沁督師海口,墾水田四十餘頃,嗣以旱潦不時,迄未能一律種稻,而所費已屬不貲。光緒初,臣以海防緊要,不可不講求屯政,曾飭提督周盛傳在天津東南開挖引河,墾水田千三百餘頃,用淮勇民夫數萬人,經營六七年之久,始獲成熟。此在潮汐可恃之地,役南方習農之人,尚且勞費若此。若於五大河經流多分支派,穿穴堤防濬溝,遂於平原易黍粟以秔稻,水不應時,土非澤埴,竊恐欲富民而適以擾民,欲減水患而適以增水患也。」

十七年,剛毅言:「吳淞江為農田水利所資,自道光六年浚治後,又經六十餘年,淤墊日甚。前年秋雨連旬,河湖汎濫,積澇竟無消路。去年十月,派員開辦,並調營勇協同民夫,分段合作,約三月內可告竣。」報聞。鴻章又言:「寶坻青龍灣減河,自香河之王家務經寶坻至寧河入海。去歲霪雨兼旬,河流狂漲,橫堤決岸,寶坻受害獨深。廣安橋以下,河身淺窄,大寶莊以上,並無河槽,應與昔年所開之普濟河、黃莊新河一律挑深,添建石閘。」沈秉成、松椿言:「淮南堰圩所管之洪澤湖,關系水道利病鹽漕諸務。今全湖之水下趨,毫無節制。現勘得應行先辦之工,曰修復三壩,曰修整束水堤,曰展挑三福口,計三項工程,不過數萬兩可以集事。或有議於禮河迤西蔡家莊建滾水石壩,使水可蓄洩,較有把握。惟巨款難籌,應暫緩辦。」均詔如所請。堵築吳橋宣惠河缺口二。河陜汝道鐵珊,以閿鄉北濱黃河,城垣屢被沖坍,因於城外築大石壩,挑溜護城。

十八年,疏鑿福山港、徐六涇二河,及高浦、耿涇、海洋塘、西洋港四河。山東巡撫福潤言:「小清河為民田水利所關,年久淤塞。前撫臣張曜籌議疏通,因工漲款絀,僅修下游博興之金家橋至壽光海道,長百餘里。其上游工程,應接續興挑,庶使歷城等縣所受各水,悉可入海。今擬規復小清河正軌,而不拘牽故道,由金家橋而西取直,擇窪區接開正河,歷博興、高苑、新城、長山、鄒平至齊東曹家坡,長九十七里,又於金家橋迤下開支河二十四里,至柳橋,以承濟麻大湖上游各河之水,引入新河,計長四千二百餘丈。」詔從之。

二十年,崇明海岸被潮沖嚙,逼近城墻。於青龍港東西兩面設立敵水壩四,加建木橋,疊砌石塊,以御風潮。二十一年,署兩江總督張之洞言:「黃河支流之減水河洪河,自虞城、夏邑、永城經碭山、蕭縣,達宿州、靈壁、泗州之睢河,而注于洪湖。其間湖港紛歧,皆下注睢河。乾隆年間,以睢河不能容,導水為三,曰北股、中股、南股。中股為睢河正流。咸豐初,黃河日益淤墊,漸及改徙,豫、江、皖各河亦逐段淤阻,水潦泛溢為害,尤以永、蕭、碭為甚。同治間建議疏河,恆以工程過大,屢議屢輟。今擬改道辦法,導北股河之水以達靈壁岳河,導中股、南股河之水合流入宿州運糧溝,以達澮河,而運糧一溝恐不能容納,應治沱河梁溝以復其舊,使各河之水皆順軌下注洪湖,不致橫溢,則各屬水患永息矣。」詔如所請行。

二十二年,御史華煇疏陳興修水利八事:曰引泉,曰築塘,曰開渠,曰通湖,曰開井,曰蓄水,曰用車,曰填石。下所司議。二十四年,濬太倉劉河,自殷港門至浦家港口四千一百餘丈。二十八年,江西巡撫李興銳言:「近年水患頻仍,皆由鄱陽湖日見淤淺,而長江昔寬今狹,驟遭大雨,疏洩不及,遂至四溢為災。請於冬晴水淺時,購制挖泥機器輪船數艘,將全湖分別挑挖。其上游河道亦一律擇要疏治。既為防水患起見,亦為興商務張本。」從之。修湖北省城北路堤紅關至春山八段,南路堤白沙洲至金口十段,以禦外江之汎漲。建石閘數座,以備內湖之宣洩。又於附郭沿江十餘里,一律增修石剝岸。濬小清河,開徒陽河百二十餘里。

宣統元年,署直隸總督那桐言:「通州占魚溝堤岸,自光緒九年決口,流入港溝而歸鳳河。嗣後屢堵屢潰。至二十四年大汛復決,迄今未能堵閉,以致武清百數十村頻年潰沒。今擬於占魚溝暫建滾水壩,俾全溜不致旁趨。倘遇盛漲,即將土墊挑除,俾資分洩。一面將上游堤壩挑補整齊,疏濬青龍灣等處引河,以減盛漲,築攔水墊以御渾流,修估龍鳳河以疏積潦。滾水壩工程應即興辦。其修堤及疏引河,應於本年秋後部署,來年二月興工。攔水墊及龍鳳河,應於來年秋後部署,次年二月興工。均限伏汛前報竣。」下部議行。湖廣總督陳夔龍請修復江、襄潰口,略謂:「江、襄各堤,以潛江之袁家月堤為最要。此次潰口,堤身沖刷,頓落四百餘丈,回流湍急,附近悉成澤國,應及時築合。此外郭家嘴、禹王廟潰堤,及天門黑牛渡、沔陽呂蒙營、公安高李公、松滋楊家腦、監利河龍廟各堤工,均擬派員督辦籌修,以期鞏固。」從之。

寧夏滿營開墾馬廠荒地,先治唐渠,以裕瀦停之地。挑濬百二十餘里,曰正渠;自靖益堡開支口,引水西北行四十餘里而入之溝,曰新渠;沿渠列小口四十,挾水以歸諸田,曰支渠。唐渠以西,淪為澤國,非溝以宣之不為功。自杏子湖起,穿溝二百八十餘里,建大小石閘、木閘四十二,石橋、木橋三十三,經始上年九月,至本年八月告成,名曰湛恩渠,約成腴田二十萬畝。是年,東三省總督、奉天巡撫合詞請修遼河,先從雙臺子河堤入手,次年續修鴨島、冷家口工程,並挑挖海口攔江沙,與遼河工程同時舉辦。下部知之。


\end{pinyinscope}