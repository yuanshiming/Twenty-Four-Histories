\article{志一百三}

\begin{pinyinscope}
○河渠三

△淮河永定河海塘

淮水源出桐柏山,東南經隨州,復北折過桐柏東,歷信陽、確山、羅山、正陽、息、光山、固始、阜陽、霍丘、潁上,所挾支水合而東注,達正陽關。其下有沙河、東西淝河、洛河、洱河、芡河、天河,俱入於淮。過鳳陽,又有渦河、澥河、東西濠及漴、澮、沱、潼諸水,俱匯淮而注洪澤湖。又東北,逕清河、山陽、安東,由雲梯關入海。逕行湖北、河南、安徽、江蘇四省,千有七百餘里,淮固不為害也。自北宋黃河南徙,奪淮瀆下游而入海,於是淮受其病。淮病而入淮諸水泛溢四出,江、安兩省無不病。夫下壅則上潰,水性實然,故治河即所以治淮,而治淮莫先於治河。有清一代,經營於淮、黃交匯之區,致力綦勤,糜帑尤鉅。迨咸豐中,銅瓦廂決,黃流北徙,宋、元來河道為之一變。然河徙淤留,導淮之舉又烏容已。今於淮流之源委分合,及清口之蓄洩,洪澤湖之堰壩工築,皆備列焉。

順治六年夏,淮溢息縣,壞民田舍。康熙元年,盱、泗民由古溝鎮南及穀家橋北盜決小渠八,淮水強半分洩高、寶諸湖,而清口淮弱,無力敵黃。六七年間,淮大漲,沖潰古溝、翟家墩,由高、寶諸湖直射運河,決清水潭,又溢武家墩、高良澗,清口湮而黃流上潰。十五年,淮又大漲,合睢湖諸水並力東激,高良澗板工決口二十六,高堰石工決口七,涓滴不出清口。黃又乘高四潰,一入洪澤湖,由高堰決口會淮,並歸清水潭,下流益淤墊。

總河靳輔言:「洪澤下流,自高堰西至清口約二十里,原系汪洋巨浸,為全淮會黃之所。自淮東決、黃內灌,一帶湖身漸成平陸,止存寬十餘丈、深五六尺至一二尺之小河,淤沙萬頃,挑濬甚難。惟有於兩旁離水二十丈許,各挑引河一,俾分頭沖刷,庶淮河下注,可以沖闢淤泥,徑奔清口,會黃刷沙,而無阻滯散漫之虞。」輔又言:「下流既治,淮可直行會黃刷沙,但臨湖一帶堤岸,除決口外,無不殘缺單薄,危險堪虞。板工固易壞,即石工之傾圮亦不可勝數。惟堤下系土坦坡,雖遇大水不易沖,今求費省工堅,惟有於堤外近湖處挑土幫築坦坡。每堤一丈,築坦坡寬五尺,密布草根草子其上,俟其長茂,則土益堅。至高堰石工,亦宜幫築坦坡,埋石工於內,更為堅穩,較之用板用石用埽,可省二十一萬有奇,且免沖激頹卸之患。」又言:「自周家閘歷古溝、唐埂至翟家壩南,估計築三十二里之堤,並堵此原沖成之九河,及高良澗、高家堰、武家墩大小決口三十四,需費七十萬五千有奇,皆系用埽,不過三年,悉皆朽壞。臣斟酌變通,除鑲邊裹頭必須用埽,餘俱宜密下排椿,多加板纜,用蒲包裹土,繩扎而填之,費可省半,而堅久過之。今擬改下埽為包土,仍築坦坡。」制可。十八年,大濬清口、爛泥淺、裴家場、帥家莊引河,使淮水全出清口,會黃東下。

三十五年,總河董安國因泗州知州莫之翰議,請開盱眙聖人山禹王河,導淮注江,略言:「禹王古河,自盱眙聖人山歷黑林橋、桐城鎮、楊村、天長縣迄六合之八里橋,各有河形溪澗崗不等。若開引入江,則天長、楊村、桐城各汊澗,大水時可不入高郵湖,湖水不致泛溢,而下河之水可減。至古河之口,現與淮不通流,必立閘座,水小閉閘以濟漕,漲則開閘以洩水,庶淮水洶湧之勢可減。」格廷議不行。明年,上有宜堵塞高堰壩之諭。逾二年,總河於成龍申塞六壩之請。會病卒,未底厥績。其年水復大至,已堵三壩,旋委洪流。三十九年,張鵬翮為總河,盡塞之,使淮無所漏,悉歸清口;又開張福、裴家場、張家莊、爛泥淺、三岔及天然、天賜引河七,導淮以刷清口;又以清口引河寬僅三十餘丈,不足暢洩全湖之水,加開寬闊。於是十餘年斷絕之清流,一旦奮湧而出,淮高於黃者尺餘。四十年,築高堰大堤。

四十四年,聖祖南巡,閱高堰堤工,詔於三壩下濬河築堤,束水入高郵、邵伯諸湖。又洪湖水漲,泗、盱均被水災,應於受水處酌量築堤束水。四十五年,兩江總督阿山等請於泗州溜淮套別開河道,直達張福口,以分淮勢,計費三百十餘萬。部議靳之。廷臣亦以河工重大,請上親臨指示。逾年,上南巡閱河,諭曰:「詳勘溜淮套地勢甚高,雖開鑿成河,亦不能直達清口。且所立標桿多在墳上,若依此開河,不獨壞田廬,甚至毀墳塚,何必多此一事。今欲開溜淮套,必鑿山穿嶺,不獨斷難成功,且恐汛水泛溢,不浸入洪湖,必沖決運河。」命撤去標桿,並譴阿山、鵬翮等有差。上又謂:「明代淮、黃與今迥異。明代淮弱,故有倒灌之虞。今則淮強黃弱。與其開溜淮套無益之河,不若於洪湖出水處再行挑濬寬深,使清水愈加暢流,為利不淺。」四十九年,加長御黃西壩工程,從河督趙世顯請也。

雍正元年,重建清口東西束水壩於風神廟前以蓄清,各長二十餘丈。三年,總河齊蘇勒因硃家海沖決,湖底沙淤,恐高堰難保,改低三壩門檻一尺五寸以水曳湖水,而救一時之急。不知水愈落,淮愈不得出,致力微不能敵黃,連年倒灌,分溜直趨。李衛頗非之。先是高堰石工未能一律堅厚。至七年冬,發帑百萬,命總河孔繼珣、總督尹繼善將堤身卑薄傾圮處拆砌,務令一律堅實。十年秋,高堰石工成。

乾隆二年,用總河高斌言,飭疏濬毛城鋪迤下河道,經徐、蕭、睢、宿、靈、虹各州縣,至泗州之安門陡河,紆曲六百餘里,以達洪湖,出清口,而淮揚京員夏之芳等言其不便。下各督撫及河、漕督會議,並召詢斌。斌至,進圖陳說,乃知芳等所言非現在情形,卒從斌議。明年,毛城鋪河道工竣。四年,高宗以高堰三壩既改低,過岸之水足洩,用大學士鄂爾泰言,永禁開放天然二壩。五年秋,西風大暴,湖浪洶湧,高堰汛第八堡舊堤撞擊,倒卸十四段,旋修補之。六年,斌言:「江都三汊河乃瓜、儀二河口門,瓜河地勢低,淮水入瓜河分數少,故溜緩不能刷深,河道致日漸淤墊。應築壩堵閉瓜河舊口門,於洋子橋營房迤下別挑越河,減淮水入瓜河之分數,則儀河可分流刷淤,並堵閉瓜洲廣惠閘之舊越河,於閘下別開越河,使閘越二河水勢均平,既緩淮水直下入江之勢,於運道更為便利。」七年,河湖並漲,議者又謂淮河上游諸水俱匯入洪湖,邵伯以下宜多開入江之路。斌亦以為言。於是開濬石羊溝舊河直達於江,築滾壩四十丈,並開通芒稻閘下之董家油房、白塔河之孔家涵三處河流,增建滾壩,使淮水暢流無阻。八年,淮暴漲丈餘,逼臨淮城,改治於周樑橋。

十六年,上以天然壩乃高堰尾閭,盛漲輒開,下游州縣悉被其患,命立石永禁開放。並用斌言,於三壩外增建智、信二壩,以資宣洩。十八年七月,淮溢高郵,壞車邏壩、邵伯二閘,下河田廬多沒。二十二年,以湖水出清口,賴東西二壩堵束,並力刷黃,湖水過大,奔溢五壩,亦恐為下河患。因定制五壩過水一寸,東壩開寬二丈,以此遞增,泐石東壩。嗣是遇湖水增長,即展寬東壩以水曳盛漲,有展寬至六七十丈者。二十七年,上言:「江南濱湖之區,每遇大汛,霖潦堪虞,洪澤一湖,尤為橐籥關鍵。為澤國計安全,莫如廣疏清口,為及今第一要義。現在高堰五壩高於水面七尺有奇,清口口門見寬三十丈,當即依此酌定成算。將來兩壩水增長至一尺,拆寬清口十丈,水遞長,口遞寬,以此為率。」是年六月,五壩水志逾一尺。河督高晉遵旨拆寬清口十丈,宣洩甚暢。三十二年,南河總督李宏言:「正陽關三官廟舊立水志,考驗水痕,本年所報消長,與下游不符。請於荊山、塗山間及臨淮鎮,各增設水志一,以驗諸水消長。」允之。三十四年,上恐高堰五壩頂封土障水,不足當風浪,命酌加石工。高晉等言其不便,乃增用柴柳。四十年,大修堰、盱各壩及臨河磚石工。

先是上以清口倒灌,詔循康熙中張鵬翮所開陶莊引河舊跡挑挖,導黃使北,遣鄂爾泰偕斌往勘,以汛水驟至而止。旋完顏偉繼斌為河督,慮引河不易就,乃用斌議,自清口迤西,設木龍挑溜北趨,而陶莊終不敢議。次年,南河督吳嗣爵內召,極言倒灌為害。薩載繼任,亦主改口議。上乃決意開之。於是清口東西壩基移下百六十丈之平成臺,築攔黃壩百三十丈,並於陶莊迤北開引河,使黃離清口較遠,清水暢流,有力攻刷淤沙。明年二月,引河成,黃流直注周家莊,會清東下,清口免倒灌之患者近十年。

五十年,洪湖旱涸,黃流淤及清口,命河南巡撫畢沅祭淮瀆,疏賈魯、惠濟諸河流以助清,湖水仍不出,黃復內灌。上欲開毛城鋪、王家營減壩,下大學士阿桂等議。阿桂言:「欲治清口之病,必去老壩工以下之淤,尤當掣低黃水,使清水暢出攻沙,不勞自治。」於是閉張福口四引河,浚通湖支河,蓄清水至七尺以上,治開王營壩減洩黃水,盡啟諸河,出清口滌沙,修清口兜水壩,易名束清壩。復移下惠濟祠前之東西束水壩三百丈於福神巷前,加長東壩以禦黃,縮短西壩以出清,易名禦黃壩。

嘉慶元年,湖水弱,清低於黃者丈餘,淮遏不出。淮漲則開山盱五壩、吳城七堡,黃漲或減水入湖,以救清口之倒灌。五年,用江督費淳、河督吳璥言,開吳城七堡引渠,使洩湖水入黃,以減盛漲。八年,黃流入海不暢,直注洪澤湖。璥赴海口相度,請力收運口各壩,止留口門,清雖力弱難出,黃亦不能再入。七月,淮漲,高堰危甚,開信、義兩壩洩水。西風大作,壞仁、智兩壩,淮南奔清口。上責璥,遂罷免。九年春,湖水稍發,伏汛黃仍倒灌。河督徐端以束清壩在運口北,分溜入運,致不敵黃,請移建湖口迤南。從之。十一年,江督鐵保言:「潘季馴、靳輔治河,專力清口,誠以清口暢出,則河腹刷深,海口亦順,洪澤亦不致泛濫。為今之計,大修閘壩,借清刷沙,不能不多蓄湖水。即不能不保護石堤,尤不能不急籌去路。」又偕徐端陳河工數事:一,外河之方家馬頭及三老壩為淮、揚保障,宜填護碎石;一,義壩宜堵築;一,仁、智、禮、信四壩殘損宜拆修。廷議如所請。上恐四壩同修,清水過洩,命次第舉行。

十五年十月,大風激浪,義壩決,堰、盱兩工掣坍千餘丈。議者謂宜築碎石坦坡,以費鉅不果。璥與端請加培大堤外靳輔所築二堤,以為重門保障,亦為廷議所駁。及陳鳳翔督南河,復申二堤之請。下江督百齡議。百齡言不若培修大堤。十七年,遣協辦大學士松筠履勘,亦主百齡議。於是築大堤子堰,自束清壩尾至信壩迤南止。鳳翔以不知蓄清於湖未漲之先,即啟智、禮兩壩,致禮壩潰,下游淹,清水消耗,貽誤全河,為百齡所劾,奪職遣戍。十八年,百齡及南河督黎世序以仁、義、禮三壩屢經開放,壞基跌塘,請移建三壩於蔣家壩南近山岡處,各挑引河,先建仁、義壩,因禮壩基改築草壩,備本年宣洩。上命先建義壩,如節宣得宜,再分年遞修。二十三年,增建束清二壩於束清壩北,收蓄湖水。

道光二年,增修高堰石工。四年冬,河漲,洪澤湖蓄水至丈七尺,尚低於黃尺許,高堰十三堡堤頂被大水掣動,山盱周橋之息浪菴亦過水八九尺,各壩均有坍損。上遣尚書文孚、汪廷珍履勘,而褫河督張文浩職。十三堡缺口旋塞。侍郎硃士彥言:「高堰石工在事諸臣,惟務節省,辦理草率。又因搶築大堤,就近二堤取土,事後亦不培補。至山盱五壩,宣洩洪湖盛漲,未能謹守舊章,相機開放,致石工掣卸。」並下文孚等勘覈。明年春,從文孚等議,改湖堤土坦坡為碎石,於仁、義、禮舊壩處所各增建石滾壩,以防異漲。

八年,上以禦黃壩上下積淤丈餘,清水不能多蓄,禦黃壩終不可開,下南總河張井等籌議。井等言:「乾隆間湖高於河七八尺或丈餘,入夏拆展御黃壩,水曳清刷淤,至冬始閉。嘉慶間,因河淤,改夏閉秋啟。而黃水偶漲,即行倒灌。今積淤日久,縱清水能出,止高於黃數寸及尺餘,暫開即閉,僅免倒灌,未能收刷淤之效。」上不懌,曰:「以昔證今,已成不可救藥之勢。為河督者,祗知水曳清水以保堰,閉禦壩以免倒灌,增工請帑,但顧目前,不思經久,如國計何?如民生何?如後日何?」

十年,井言:「淮水歸海之路不暢,請於揚糧之八塔鋪、商家溝各斜挑一河,匯流入江,分減漲水,並拆除芒稻河東西閘,挑挖淤灘,可抵新闢一河之用。」從之。十二年,移建信壩於夏家橋。十四年,以義字引河跌深三四丈,堵閉不易,允河督麟慶請,改挑義字河頭。二十一年,河決祥符,奪溜注洪澤湖,而江潮盛漲,又復頂托,因拆展御黃、束清及禮、智、仁各壩,並啟放車邏等壩,以洩湖水。二十三年,河決中牟,全溜下注洪澤湖,高堰石工掣卸四千餘丈,先後拆展束清、御黃、智、信各壩,並啟放順清、禮、義等河,金灣舊壩及東西灣壩同時並啟,減水入江。

咸豐五年,河復決銅瓦廂,東注大清河入海。黃河自北宋時一決滑州,再決澶州,分趨東南,合泗入淮。蓋淮下游為河所奪者七百七十餘年,河病而淮亦病。至是北徙,江南之患息。士民請復淮水故道者,歲有所聞。

同治八年,江督馬新貽濬張福口引河,淮遂由清口達運。嗣又挑楊莊以下之淤黃河,以洩中運河盛漲。九年,新貽等言:「測量雲梯關以下河身,及成子河、張福口、高良澗一帶湖心,始知黃河底高於洪湖底一丈至丈五六尺不等,必先大濬淤黃,使淮得暢流入海,繼闢清口,導之入舊黃河,再堵三河,以杜旁水曳而資抬蓄。然非修復堰、盱石工,堅築運河兩堤,不敢遽堵三河、闢清口。統籌各工,非數百萬金不能集事。擬分別緩急,次第籌辦,不求利多,但求患減,為得寸得尺之計,收循序漸進之功。」

光緒七年,江督劉坤一言:「臣此次周歷河湖,知淮揚水利有關國計民生。前議導淮,未可中輟。自楊莊以下,舊黃河淤平,則山東昭陽、微山等湖之水,由中運河直趨南運河,夏秋之間,三閘甚形吃重。自洪澤湖淤淺,淮水不能合溜,北高於南,水之分入張福引河者無多,大溜由禮河徑趨高、寶等湖。上年挑濬舊黃河後,山東蛟水屢次暴發,由此分瀉入海。築禮壩後,湖水瀦深,且由張福河入運口者頗旺。此挑舊河、築禮壩之不無微效也。惟是張福河淺,湖水仍趨重禮河越壩,終為可慮。倘遇湖水汎濫,禮河即無越壩,亦難分消,必開信、智兩壩,由高寶湖入南運河,亦必開車邏、南關等壩,由里下河入海,沿途淹沒田廬,所損匪細。今擬就張福河開挖寬深,以引洪澤湖之水,復挖碎石河,以分張福河之水,由吳城七堡匯順清河。水小則由順清入運,途紆而勢稍舒,水大則由舊黃河入海,途直而勢自順。約三四年間,便可告竣,所費尚不過鉅。議者或謂導淮入海,當盡瀉洪湖之水,有妨官運民田。臣以為別開引河,或不免有此患。今循張福河、碎石河故道以歸順清河,自非淮漲一二丈,則順清河之水何能高過中運河,溢出舊黃河?如使淮水暴漲,方有潰決之虞,惟恐水無去路,此正導淮之本意也。議者或謂多引湖水入運,恐三閘不能支持。不思洪湖未淤以前,湖水四平,蓄水深廣,張福以外,有四引河以濟漕運。維時黃未北徙,每遇漕船過閘,方且蓄清敵黃,以五引河全注運口,而三閘屹然,今特張福一河,決無致損三閘之理。且上年挑通舊黃河,已分減中運河水,其入南運河者不過三四成。湖水雖增,與前略等,即遇大水,有舊黃河可以分減,亦不至專出三閘也。議者又謂如此,導淮無弊,亦屬無利,何必虛費帑藏。其說亦不盡然。夫治水之道,必須通盤規畫,並須預防變遷。洪湖南有禮河,北有張福河,均為分洩淮水。而水勢就下,禮河常苦水大,築禮河壩所以蓄張福之水,濬張福口所以顧禮河之堤,彼此互相維系。如使禮河受全湖之沖,新壩恐不能保,續修則所費彌鉅,不修則為害滋深,下者益下,高者益高,張福河漸形壅塞矣。且導淮之舉,原防盛漲肆虐。如引湖由張福出順清,以舊黃河為出海之路,偶有泛溢,該處土曠人稀,趨避尚易。若張福不暢,全湖之水折而南趨,則淮揚繁盛之區,億萬生靈將有其魚之嘆。導淮之利,見於目前者猶小,見於日後者乃大也。」疏入,下部知之。

八年,江督左宗棠言:「濬沂、泗為導淮先路,洵為確論。惟雲梯關以下二百餘里,河身高仰,且有遠年沙灘。昔以全黃之力所不能通者,今欲以沂、泗分流通之,其勢良難。大通在雲梯關下十餘里,舊黃河北岸,系嘉慶中漫口,東北流四十餘里,至響水口,接連潮河,至灌河口入海。就此加挑寬深,出海較便。沂、泗來源,當大為分減,淮未復而運道亦可稍安,淮既復而歸海無虞阻滯。此疏濬下游,宣洩沂、泗,實導淮先路,不可不亟籌者也。淮挾眾流,匯為洪澤,本江、皖巨浸。自道光間為黃所淤,北高南下,由禮河趨高寶湖以入運者垂三十年。今欲導之復故,不啻挽之逆流。自張福口過大通、響水口入海,三百五十餘里,節節窒兒,非下游暢其去路,上游塞其漏卮,其不能舍下就高入黃歸海也明甚。查張福口及天然引河,皆北趨陳家集之大沖,至碎石河以達吳城七堡,又北至順清河口,接楊莊舊黃河。張福河面六十餘丈,宜加寬深,天然河更須疏瀹,吳城七堡一帶高於張福河底丈六七尺,尤必大加挑濬,使湖水果能入黃,然後可堵禮河,以截旁趨之出路,堵順清河,以杜運河之奪河。此引淮入海工程,當以次接辦者也。湖水不高,不能入黃。太高,不特堰、盱石工可慮,運口閘壩難支,且於盱眙、五河近湖民田有礙。擬修復智、信等壩以洩湖漲,更建閘大沖,俾湖水操縱由人,多入淮而少入運。此又預籌以善其後者也。」

三十四年,江督端方會勘淮河故道,力陳導淮四難,因於清江浦設局,遴紳籌議。久之無端緒,乃撤局。宣統元年,江蘇諮議局開,總督張人駿以導淮事列案交議,決定設江淮水利公司,先行測量,務使導淮復故,專趨入海。二年,侍讀學士惲毓鼎以濱淮水患日深,上言:「自魏、晉以降,瀕淮田畝,類皆引水開渠,灌溉悉成膏腴。近則沿淮州縣,年報水災,浸灌城邑,漂沒田廬,自正陽至高、寶,盡為澤國,實緣近百年間,河身淤塞,下游不通,水無所歸,浸成汎濫。是則高堰壩之為害也。異時黃、淮合流,有南下之勢,治河者欲束淮以敵黃,故特堅築高堰壩頭,逼淮由天妃閘以濟運。今黃久北徙,堰壩無所用之,當別籌入海之途。其道有二,以由清口西壩、鹽河至北潮河為便。尾閭既暢,水有所歸,不獨潁、壽、鳳、泗永澹沈災,即高、寶、興、泰亦百年高枕矣。」事下江督張人駿、蘇撫程德全、皖撫硃家寶勘議。人駿等言:「正事測量,俟測勘竣,即遴員開辦。」報聞。三年,御史石長信言:「導淮一舉,詢謀僉同。美國紅十字會亦擬遣工程師來華查勘。則我之思患預防,尤不可緩。江蘇水利公司既允部撥費用,安徽亦應設局測量,以為消弭巨災之圖。」下部議允之。

導淮之舉,經始於同治六年。時曾國籓督兩江,嘗謂「復瀆之大利,不敢謂其遽興,淮揚之大害,不可不思稍減」。迨黃流北徙,言者益多,大要不出兩策。一謂宜堵三河,闢清口,濬舊河,排雲梯關,使由故道入海。一謂導淮當自上流始,洪澤湖乃淮之委,非淮之源,宜於上游闢新道,循睢、汴北行,使淮未注湖,中途已洩其半,再由桃源之成子河穿舊黃河,經中河雙金閘入鹽河,至安東入海,使全淮分南北二道,納少瀉多,淮患從此可減。二說所持各異。然同、光以來,濬成子、碎石、沂、泗等河,疏楊莊以下至雲梯關故道,固已小試其端。卒之淮為黃淤,積數百年,已無經行之渠,由運入江,勢難盡挽,迄於國變,終鮮成功。

永定河亦名無定河,即桑乾下游。源出山西太原之天池,伏流至朔州、馬邑復出,匯眾流,經直隸宣化之西寧、懷來,東南入順天宛平界,逕盧師臺下,始名盧溝河,下匯鳳河入海。以其經大同合渾水東北流,故又名渾河,元史名曰小黃河。從古未曾設官營治。其曰永定,則康熙間所錫名也。永定河匯邊外諸水,挾泥沙建瓴而下,重巒夾峙,故鮮潰決。至京西四十里石景山而南,逕盧溝橋,地勢陡而土性疏,縱橫蕩漾,遷徙弗常,為害頗鉅。於是建堤壩,疏引河,宣防之工亟焉。

順治八年,河由永清徙固安,與白溝合。明年,決口始塞。十一年,由固安西宮村與清水合,經霸州東,出清河;又決九花臺、南里諸口,霸州西南遂成巨浸。康熙七年,決盧溝橋堤,命侍郎羅多等築之。三十一年,以河道漸次北移,永清、霸州、固安、文安時被水災,用直隸巡撫郭世隆議,疏永清東北故道,使順流歸澱。

三十七年,以保定以南諸水與渾水匯流,勢不能容,時有汎濫,聖祖臨視。巡撫於成龍疏築兼施,自良鄉老君堂舊河口起,逕固安北十里鋪、永清東南硃家莊,會東安狼城河,出霸州柳岔口三角澱,達西沽入海,濬河百四十五里,築南北堤百八十餘里,賜名永定。自是渾流改注東北,無遷徙者垂四十年。三十九年,郎城澱河淤且平,上游壅塞,命河督王新命開新河,改南岸為北岸,南岸接築西堤,自郭家務起,北岸接築東堤,自何麻子營起,均至柳岔口止。四十年,加築南岸排椿遙堤,修金門閘。四十八年,決永清王虎莊,旋塞。五十六年,修兩岸沙堤大堤,決賀堯營。六十一年,復決賀堯營,隨塞。

雍正二年,修郭家務大堤,築清涼寺月堤,修金門閘,築霸州堂二鋪南堤決口。三年,因郭家務以下兩岸頓狹,永清受害特重,命怡親王允祥、大學士硃軾,引渾水別由一道入海,毋使入澱,遂於柳岔口少北改為下口,開新河自郭家務至長河,凡七十里,經三角澱達津歸海,築三角澱圍堤,以防北軼。又築南堤自武家莊至王慶坨,北堤自何麻子營至範甕口,其冰窖至柳岔口堤工遂廢。十二年,決梁各莊、四聖口等處三百餘丈,黃家灣河溜全奪,水穿永清縣郭下注霸州之津水窪歸澱。總河顧琮督兵夫塞之。十三年,決南岸硃家莊、北岸趙家樓,水由六道口小堤仍歸三角澱。

乾隆二年,總河劉勷勘修南北堤,開黃家灣、求賢莊、曹家新莊各引河,濬雙口、下口、黃花套。六月,漲漫南岸鐵狗、北岸張客等村四十餘處,奪溜由張客決口下歸鳳河。命吏部尚書顧琮察勘,請仿黃河築遙堤之法。大學士鄂爾泰持不可,議「於北截河堤北改挑新河,以北堤為南堤,沿之東下,下游作洩潮墊數段,復於南北岸分建滾水石壩四,各開引河:一於北岸張家水口建壩,即以所沖水道為引河,東匯鳳河;一於南岸寺臺建壩,以民間洩水舊渠入小清河者為引河;一於南岸金門閘建壩,以渾河故道接檿牛河者為引河;一於南岸郭家務建壩,即以舊河身為引河。合清隔濁,條理自明」。詔從其請。

四年,直督孫嘉淦請移寺臺壩於曹家務,張客壩於求賢莊。又於金門閘、長安城添築草壩,定以四分過水。顧琮言,金門閘、長安城兩壩水勢僅一河宣洩,恐汛發難容,擬分引河為兩股,一由南窪入中亭河,一由楊青口入津水窪。又言郭家務、小梁村等處舊有遙河千七百丈,年久淤塞,請發帑興修。均從之。五年,孫嘉淦請開金門閘重堤,濬西引河,開南堤,放水復行故道。六年,凌汛漫溢,固、良、新、涿、雄、霸各境多淹。從鄂爾泰議,堵閉新引河,展寬雙口等河,挑葛漁城河槽,築張客、曹家務月堤,改築郭家務等壩。八年,濬新河下口,及董家河、三道河口,修新河南岸及鳳河以東堤墊。又疏穆家口以下至東蕭莊、鳳河邊二十里有奇。九年,以範甕口下統以沙、葉兩澱為歸宿,兩汛水多歸葉澱,遂疏注沙澱路,並將南北舊減河濬歸鳳河。

十五年五月,河水驟漲,由南岸第四溝奪溜出,逕固安城下至牛坨,循黃家河入津水窪,一由檿牛河入中亭河。命侍郎三和同直督堵御,於口門下另挑引河,截溜築壩,遏水南溢,使歸故道。十六年,凌汛水發,全河奔注冰窖堤口,即於王慶坨南開引河,導經流入葉澱,以順水性。十九年,南墊水漫堤頂,決下口東西老堤,奪溜南行,漫勝芳舊澱,逕永清之武家廠、三聖口,霸州之信安入口。明年,高宗臨視,改下游由調河頭入海,挑引河二十餘里,加培墊身二千二百餘丈。二十一年,直督方觀承請於北墊外更作遙堤,預為行水地,鳳河東堤亦接築至遙墊尾。從之。二十四年,大雨,直隸各河並漲,下游悉歸澱內,大清河不能宣洩,轉由鳳河倒漾,阻遏渾流,南岸四工堤決。命御前侍衛赫爾景額協同直督剋日堵築。

三十五年、三十六年,兩岸屢決。三十七年,命尚書高晉、裘曰修偕直督周元理履勘,疏言:「永定河自康熙間築堤以來,凡六改道。救弊之法,惟有疏中洪、挑下口,以暢奔流,築岸堤以防沖突,濬減河以分盛漲。」遂興大工,用帑十四萬有奇。自是水由調河頭逕毛家窪、沙家澱達津入海。三十八年,調河頭受淤,其澄清之水散漫而下,別由東安響水村直趨沙家澱。四十年,堵北三工、南頭工漫口。四十四年,展築新北堤,加培舊越堤,廢去瀕河舊堤,使河身寬展。四十五年,盧溝橋西岸漫溢,北頭工沖決,由良鄉之前官營散溢求賢村減河歸黃花店,爰開引溝八百丈,引溜歸河。五十九年,決北二工堤,溜注求賢村引河,至永定河下游入海。旋即斷流,又漫南頭工堤,水由老君堂、莊馬頭入大清河,凡築南堤百餘丈。又於玉皇廟前築挑水壩。

嘉慶六年,決盧溝橋東西岸石堤四、土堤十八,命侍郎那彥寶、高杞分駐堵築,並疏濬下游,集民夫五萬餘治之。禦制河決嘆,頒示群臣。兩月餘工竣。十五年,永定河兩岸同時漫口,直督溫承惠駐工堵合之。十七年,河勢北趨,葛漁城淤塞,水由黃花店下注。乃於舊淤河內挑挖引河,並於上游築草壩,挑溜東行,另建圈堤以防泛衍。二十年,拆鳳河東堤民墊以去下壅。六月大雨,北岸七工漫塌,開引河,由舊河身稍南,直至黃花店,東抵西洲,長五千六百九丈。九月,水復故道。二十四年,北岸二工漫溢,頭工繼溢,側注口門三百餘丈,大興、宛平所屬各村被淹。九月塞決口,並重濬北上引河。

道光三年,河由南八工堤盡處決而南,直趨汪兒澱。四年,侍郎程含章勘議濬復,未果。十年,直督那彥成請於大範甕口挑引河,並將新堤南遙墊加高培厚。報可。十一年春,河溜改向東北,逕竇澱,歷六道口,注大清河,汪兒澱口始塞,水由範甕口新槽復歸王慶坨故道。十四年,宛平界北中、北下汛決口,水由龐各莊循舊減河至武清之黃花店,仍歸正河尾閭入海。良鄉界南二工決口,水由金門閘減河入清河,經白溝河歸大清河。爰挑引河,自漫口迤下至單家溝,間段修築二萬七千四百餘丈。二十四年,南七工漫口,就迤北三里許之河西營為河頭,挑引河七十餘里,直達鳳河。三十年五月,上游山水下注,河驟漲,北七工漫三十餘丈,由舊減河逕母豬泊注鳳河。勘於馮家場北河灣開引河,十月竣工。

咸豐間,南北堤潰決四次。時軍務方棘,工費減發,補苴罅漏而已。

同治三年,因河日北徙,去路淤淺,於柳坨築壩,堵截北流,引歸舊河,展寬挑深張坨、胡家房河身,經東安、武清、天津入海。六年以後,時有潰決。八年,直督曾國籓請於南七工築截水大壩,兩旁修築圈墊,並挑濬中洪,疏通下口,以免壅潰。從之。十年,南岸石堤漫口,奪溜逕良鄉、涿州注大清河入海。明年,允直督李鴻章請,修金門閘壩,疏濬引河,由童村入小清河。石堤決口塞。十二年,南四工漫口,由霸州檿牛河東流。爰將引河增長,復築挑水壩一。

光緒元年,南二汛漫口,隨塞。四年,北六汛決口,築合後,復於坦坡墊尾接築民墊至青光以下。十年,以鳳河當永定河之沖,年久淤墊,以工代賑,起南苑五空閘,訖武清緱上村,間段挑濬,並培築堤壩決口。十六年,大水,畿輔各河並漲,永定北上汛、南三汛同時漫決。命直督迅籌堵築,添修挑壩岸堤,又疏引河六十餘里。十八年夏,大雨,河水陡漲,南上汛灰壩漫口四十餘丈。給事中洪良品言北岸頭工關系最重,請接連石景山以下添砌石堤,以資捍衛。下所司籌議。因工艱費鉅,擇要接築石堤八里,並添修石格。十九年冬,因頻年潰決為患,命河督許振禕偕直督會勘籌辦。振禕陳疏下游、保近險、濬中洪、建減壩、治上游五事。直隸按察使周馥並建議於盧溝南岸築減水大石壩,以水抵涵洞上楣為準,逾則瀉去。詔如所請。二十二年,北六工、北中汛先後漫溢,由韓家樹匯大清河,遂挑濬大清河積淤二十餘里。

二十五年,詔直督裕祿詳勘全河形勢,以紓水患。裕祿言:「畿輔緯川百道,總匯於南北運、大清、永定、子牙五經河,由海河達海,惟永定水渾善淤,變遷無定。從前下口遙堤寬四十餘里,分南、北、中三洪。嗣因南、中兩洪淤墊,全由北洪穿鳳入運。」因陳統籌疏築之策七:一,先治海河,俾暢尾閭,然後施工上游;一,宜以鳳河東堤外大窪為永定下口;一,修築北運河西堤;一,規復大清河下口故道於西沽;一,修築格澱;一,修築韓家樹橫直各堤;一,疏濬中亭河,以期一勞永逸。需費七十七萬有奇。帝命分年籌辦。適有拳匪之亂,不果行。

三十年後,南北岸屢見潰決,均隨時堵合。論者以為若將險工全作石堤,灣狹處改從寬直,並於南七工放水東行,傍澱達津,再加以石壩分洩盛漲,庶幾永保安瀾云。

海塘惟江、浙有之。於海濱衛以塘,所以捍禦咸潮,奠民居而便耕稼也。在江南者,自松江之金山至寶山,長三萬六千四百餘丈。在浙江者,自仁和之烏龍廟至江南金山界,長三萬七千二百餘丈。江南地方平洋暗潮,水勢尚緩。浙則江水順流而下,海潮逆江而上,其沖突激湧,勢尤猛險。唐、宋以來,屢有修建,其制未備。清代易土塘為石塘,更民修為官修,鉅工累作,力求鞏固,濱海生靈,始獲樂利矣。

順治十六年,禮科給事中張惟赤言:「江、浙二省,杭、嘉、湖、寧、紹、蘇、松七郡皆濱海,賴有塘以捍其外,至海鹽兩山夾峙,潮勢尤猛。故明代特編海塘夫銀,以事歲修。近此款不知銷歸何地,塘基盡圮。儻風濤大作,徑從坍口深入,恐為害七郡匪淺。請嚴飭撫、按勒限報竣,仍定限歲修,以防患未然。」下部議行。康熙三年,浙江海寧海溢,潰塘二千三百餘丈。總督趙廷臣、巡撫硃昌祚請發帑修築,並修尖山石堤五千餘丈。二十七年,修海鹽石塘千丈。三十七年,颶風大作,海潮越堤入,沖決海寧塘千六百餘丈,海鹽塘三百餘丈,築之。五十七年,巡撫硃軾請修海寧石塘,下用木櫃,外築坦水,再開濬備塘河以防泛溢。五十九年,總督滿保及軾疏言:「上虞夏蓋山迤西沿海土塘沖坍無存,其南大亹沙淤成陸,江水海潮直沖北大亹而東,並海寧老鹽倉皆坍沒。」因陳辦法五:一,築老鹽倉北岸石塘千三百餘丈,保護杭、嘉、湖三府民田水利;一,築新式石塘,使之穩固;一,開中小亹淤沙,使江海盡歸赭山、河莊山中間故道,可免潮勢北沖;一,築夏蓋山石塘千七百餘丈,以禦南岸潮患;一,專員歲修,以保永固。下部議,如所請行。

雍正二年,帝以塘工緊要,命吏部尚書硃軾會同浙撫法海、蘇撫何天培勘估杭、嘉、湖等府塘工,需銀十萬五千兩有奇,松江府華、婁、上海等縣塘工,需銀十九萬兩有奇,部議允之。六年,巡撫李衛請將驟決不可緩待之工,先行搶修,隨後奏聞。「搶修」之名自此始。十一年,命內大臣海望、直督李衛赴浙查勘海塘,諭曰:「如果工程永固,可保民生,即費帑千萬不必惜。」尋請於尖、塔兩山間建石壩堵水,並改建草塘及條石塊石各塘為大石塘,更於舊塘內添築上備塘。十二年,因堵尖山水口、開中小亹引河久未施工,責浙督程元章等督辦不力,命杭州副都統隆升總理,御史偏武佐之。五月工竣。十三年,命南河督嵇曾筠總理塘工。曾筠言:「海寧南門外俯臨江海,請先築魚鱗石塘五百餘丈,保衛城池。」下廷臣議行。

乾隆元年,署蘇撫顧琮請設海防道,專司海塘歲修事。曾筠請於仁、寧等處酌建魚鱗大石塘六千餘丈,均從之。明年,建海寧浦兒兜至尖山頭魚鱗大石塘五千九百餘丈。四年,允浙撫盧焯請,築尖山大壩,次年秋工竣,禦制文記之。六年,左都御史劉統勛言:「前據閩浙總督德沛請改老鹽倉至章家菴柴塘為石塘,廷議準行。臣意以為草塘改建不必過急,南北岸塘工實不宜緩。蓋通塘形勢,海寧之潮猶屬往來滌蕩,而海鹽之潮,則對面直沖,其大石塘歲久罅漏,尤宜及早補苴。臣以大概計之,動發七十萬金,而通塘可有苞桑之固。」疏入,命統勛會同浙督德沛、浙撫常安察勘。尋覆稱:「改建石工,誠經久之圖,但須寬以時日,年以三百丈為率。」七年,總督那蘇圖請先於最險處間段排築石簍,俟根腳堅實,再建石塘。越二年,遣尚書訥親勘視。疏言:「仁、寧二邑柴塘穩固,若慮護沙坍漲無常,第將中小亹故道開濬,俾潮水循規出入,上下塘俱可安堵。」於是改建石工之議遂寢。七月,蘇撫陳大受言:「寶山地濱大海,月浦土塘被潮沖刷,請建單石壩,外加椿石坦坡各百七十丈,並接築沙塘,使與土塘聯屬,中設涵洞宣洩。」下部議行。

十一年,常安言:「蜀山迤北有積沙四五百丈,橫亙中間。先就沙嘴開溝四,以引潮水攻刷。今伏汛已過,南沙坍卸殆盡,蜀山已在水中,潮汐漸向南趨。倘秋汛不復湧沙,則大溜竟行中小亹矣。」報聞。十二年,常安委員疏濬蜀山一帶,用切沙法疏刷。十一月朔,中小亹引河一夕沖開,大溜經由故道,南北岸水遠沙長,皆成坦途。十三年,大學士高斌、訥親先後奉命查勘塘工。斌請於東西柴石各塘後身加築土堰,手黨護潮頭。四月,訥親疏陳善後事宜,命巡撫方觀承酌議。觀承請於北塘北大亹故道,及三里橋、掇轉廟等處,設竹簍滾壩,堵御潮溝,大小山圩改建塊石塘,南塘各工,預籌防護,並將右營員弁兵丁調派,分汛防駐。下廷議允行。十六年,允巡撫永貴請,改建山陰宋家漊土塘為石塘,加築坦水。

十七年,巡撫雅爾哈善言:「中亹山勢僅寬六里,浮沙易淤,且南岸文堂山腳有沙嘴百三十餘丈,挑溜北趨,北岸河莊山外亦有沙嘴五十餘丈,頗礙中亹大溜。現將兩處漲沙挑切疏通,俾免阻滯。」得旨嘉勉。十九年,因浙省塘工無險,省海防道。二十一年,喀爾吉善言:「水勢南趨,北塘穩固,而險工在紹興一帶。擬於宋家漊、楊柳港,照海寧魚鱗大條石塘式,建四百丈。」從之。二十三年,增築鎮海縣海塘。二十六年,蘇撫陳宏謀言,常熟、昭文濱海地方,從太倉州境接築土塘。嗣開白茆河、徐六涇二口,建閘啟閉。本年潮漲,石墻傾圮,請改為滾壩。得旨允行。

二十七年,帝南巡,閱海寧海塘工。諭曰:「朕念海塘為越中第一保障。比歲潮勢漸趨北大亹,實關海寧、錢塘諸邑利害。計改老鹽倉一帶柴塘為石,而議者紛歧。及昨臨勘,則柴塘沙性澀汕,石工斷難措手,惟有力繕柴塘,得補偏救弊之一策。其悉心經理,定歲修以固塘根,增坦水石簍以資擁護。」又諭曰:「尖山、塔山之間,舊有石塘。朕今見其橫截海中,直逼大溜,實海塘扼要關鍵。就目下形勢論,或多用竹簍加鑲,或改用木櫃排砌。如將來沙漲漸遠,宜即改築條石壩工,俾屹然如砥柱,庶北岸海塘永資保障。該督撫等其善體朕意,動帑儹辦,並勒石塔山,以志永久。」二十八年,蘇撫莊有恭言:「江南松、太海壖土性善坍,華亭、寶山向築坦坡,皆不足恃。應仿浙江老鹽倉改建塊石簍塘。」詔如所請。三十年春,帝南巡,閱視海寧海塘。諭曰:「繞城石塘,實為全城保障。塘下坦水,祗建兩層,潮勢似覺頂沖。若補築三層,尤資裨益。著將應建之四百六十餘丈一律添建。」三月工竣。

三十五年,巡撫熊學鵬請於蕭山、山陰、會稽改建魚鱗大石塘。帝以潮勢正趨北亹,與南岸渺不相涉,斥之。三十七年,巡撫富勒渾疏報中亹引河情形,略言:「潮頭大溜,一由蜀山直趨引河,一由巖峰山西斜入引河,至河莊山中段會合,互相撞擊,仍分兩路西行,隨令員弁於引河中段挑堰溝二十餘道,導引潮溜,俾復中亹故道。」諭曰:「潮汛遷移,乃噓吸自然之勢,若開挖引河,恐徒勞無益。止宜實力保衛堤塘,以待其自循舊軌,不必執意開溝引溜,欲以人力勝海潮也。」

四十三年,浙撫王亶望疏陳海塘情形,命江督高晉會同相度。尋疏言:「章家菴一帶柴工五百丈,潮神廟前柴塘三百丈,應添建竹簍,並排列兩層椿木以防動搖。」從之。四十五年,帝南巡,幸海寧尖山閱海塘。十二月,命大學士阿桂、南河督陳輝祖赴浙履勘。疏言:「海塘工程,應建石塘二千二百丈,若改為條石,施工易而成事速,約計三年可以蕆工。」又言:「辦理魚鱗石塘,仿東塘之例,量地勢高下,用十六層至十八層,約需三十萬。」帝命工部侍郎楊魁駐工協辦,次年八月竣工。四十九年,帝幸杭州,閱視海塘,諭曰:「老鹽倉舊有柴塘,一律添建石塘四千二百餘丈,於上年告竣,自應砌築坦水保護。乃該督撫並未慮及,設遇異漲,豈能抵禦?著將柴塘後之土順坡斜做,並於其上種柳,俾根株盤結,則石柴連為一勢,即以柴塘為石塘之坦水。至範公塘一帶,亦必接建石工,方於省城足資鞏護。著撥帑五百萬,交該督撫覈算,分限分年修築。」五十二年工竣。

嘉慶四年,浙撫玉德請改山陰土塘為柴塘。十三年,浙撫阮元請改蕭山土岸為柴塘。十六年,浙撫蔣攸銛請將山陰各土塘堤一律建築柴塘;蘇撫章煦請將華亭土塘加築單壩二層。均從之。

道光十三年五月,巡撫富呢揚阿疏言「東西兩防塘工,先擇尤險者修築,需銀五十一萬二千餘兩」。十一月,又言「限內限外各工俱掣坍,需銀十九萬四千餘兩」。十二月,又言「東塘界內,應於前後兩塘中間,另建鱗塘二千六百餘丈,需銀九十二萬二千兩」。均下部議行。十四年,命刑部侍郎趙盛奎、前東河督嚴烺,會同富呢揚阿查勘應修各工。尋疏言:「外護塘根,無如坦水,擬自念里亭汛至鎮海汛,添建盤頭三座,改建柴塘三千三百餘丈;其西塘烏龍廟以東,應接築魚鱗石塊;海寧繞城石塘,應加高條石兩層。俟明年大汛時續辦。」遣左都御史吳椿往勘,留浙會辦。十六年三月工竣,計修築各工萬七千餘丈,用銀一百五十七萬有奇。三十年,巡撫吳文鎔疊陳海塘石工沖缺,令速搶辦。十月工竣。

咸豐七年八月,海塘埽各工猝被風潮沖坍。十二月,次第堵合。同治三年,御史洪燕昌言浙江海塘潰決,請速籌款修理。部議將浙海關等稅撥用。五年,內閣侍讀學士鍾佩賢疏陳海塘關系東南大局,有四害三可慮。命巡撫馬新貽詳勘,修海寧魚鱗石工二百六十餘丈。六年,以浙江海塘工鉅費多,議分最要次要修築,期以十年告竣。七年,兩江總督曾國籓等請修華亭石塘護壩,嗣是塘工歲有修築。

光緒三年,修寶山北石塘護土,建護塘攔水各壩,及仁和、海寧魚鱗石塘千三百餘丈。十年,修昭文、華亭、寶山等處塘壩及石坦坡。十二年,浙江巡撫劉秉璋言,海鹽原建石塘四千六百餘丈,積年坍損過半,擬擇要興辦,埋砌者五百丈,建復者四百六十丈,需銀二十萬。允之。十八年,浙撫劉樹棠疏言,海寧繞城石塘坍塌日甚,請添築坦水,以塘工加抽絲捐積存餘款先行開辦,隨籌款次第興修。從之。十九年,修太倉茜涇口椿石坦坡百五十一丈,鎮洋楊林口椿石二百丈,昭文施家橋至老人濱雙椿夾石護壩二百丈,華亭外塘純石斜壩四十六丈。

綜計兩省塘工,自道光中葉大修後,疊經兵燹,半就頹圮,迄同治初,興辦大工,庫款支絀,遂開辦海塘捐輸,並勸令兩省絲商,於正捐外,加抽塘工絲捐,給票請獎。旋即停止。光緒三十年,浙江巡撫聶緝椝請復捐輸舊章,以濟要工。因二十七年以後,潮汐猛烈,次險者變為極險,擬將柴埽各工清底拆築,非籌集鉅款,不能歷久鞏固云。


\end{pinyinscope}