\article{列傳四十四}

\begin{pinyinscope}
趙國祚許貞周球徐治都胡世英唐希順李麟

趙應奎趙賴李芳述陳世凱許占魁

趙國祚,漢軍鑲紅旗人。父一鶴,太祖時來歸。天聰間,授三等甲喇章京。國祚其次子也。初授牛錄額真,屯田義州。從征黑龍江。取前屯衛、中後所。順治初,從征江南,克揚州、嘉興、江陰,皆有功。世職自半個前程累進二等阿達哈哈番。歷官自甲喇額真累遷鑲白旗漢軍固山額真。

十三年,加平南將軍,駐師溫州。十五年,授浙江總督。鄭成功犯溫州,國祚督兵擊卻之,得舟九十餘。成功又犯寧波,副都統夏景梅、總兵常進功等督兵擊卻之,奏捷,上以成功自引退,疏語鋪張,飭毋蹈明末行間陋習,罔上冒功。成功旋大舉犯江寧,督兵防禦,事定,部議國祚等玩寇,當奪官,詔改俸。國祚督浙江四年,頗盡心民事。歲饑,米值昂,發帑平糶,並移檄鄰省毋遏糶,民以是德之。十八年,調山東,復調山西。康熙元年,甄別各直省督撫,國祚以功不掩過,解任。

吳三桂反,十三年,起國祚江西提督,駐九江。三桂兵入江西境,命移駐南昌。耿精忠應三桂,亦遣兵犯江西,陷廣信、建昌。國祚與將軍希爾根、哈爾哈齊督兵赴援,精忠將易明自建昌以萬餘人迎戰。師分道縱擊,破賊,逐北七十餘里,克撫州。明復以萬餘人來攻,國祚與前鋒統領沙納哈、署護軍統領瓦岱等奮擊破之,斬四千餘級。十四年,大將軍安親王岳樂請以國祚隨征,報可。十五年,師進攻長沙,三桂兵來犯,國祚擊之敗走。尋命移駐茶陵。十八年,長沙下,從安親王攻寶慶。世璠將吳國貴據武岡,國祚與建義將軍林興珠督兵力戰,砲殪國貴,克武岡。國祚以創發乞休。二十七年,卒,年八十,賜祭葬,謚敏壯。子月襲職,自廣東駐防協領累遷至正紅旗漢軍都統。

許貞,字藎臣,福建海澄人。初為鄭氏將。康熙三年,率所部至漳州降,授左都督,駐九江。尋移贛縣,以荒地畀降兵屯墾,號「屯墾都督」。

十三年,耿精忠反,遣其將賈振魯、曾若千犯贛州,陷石城,圍寧都。廣信、建昌諸山寇應之,州縣多殘破。貞選所部得健卒四百,會游擊周球赴援,敗賊於黃地,斬級千餘,獲甲幟、砲械無算,遂解寧都圍,復石城。未幾,賊犯興國,貞馳剿,多斬獲。進攻雩都、瑞金,戰天華山、李芬江、長樂里,屢破賊,克橋頭、五仙、白奇、田產、江頭、上龍、寶石諸寨,降賊萬餘,出難民三萬有奇。巡撫白色純上其功,詔嘉許,加太子少保。總督董衛國請增置撫建廣總兵駐建昌,即以命貞。貞督兵復宜黃、崇仁、樂安諸縣。精忠使誘貞,貞不發書,械其使以聞,予世職拖沙喇哈番。

時大將軍安親王岳樂駐建昌,精忠遣其將耿繼善、楊玉太、李懋珠等分屯城外麻姑、二聖諸山,岳樂憂之,貞曰:「賊雖多,易與,請先破一砦。」即夕馳攻蕭家坪,破一砦。岳樂乃督兵自吉安進攻長沙,留滿洲兵五百俾貞守建昌。貞所部僅二千,賊詗守兵寡,攻城,分屯城東南從姑山,貞自將銳卒攻之,直上破壘,賊引卻。麻姑山最峻,賊數萬人屯其上,環山立寨。貞休兵數月,時就山麓操演,賊易之,不為備。十五年,春水發,溪澗皆可舟,賊寨隔水為浮橋相屬。貞復引兵操山麓,出不意,督兵直上突賊壘,別將舟載薪焚浮橋,一日破六十餘砦,斬其將揭重信等,其眾殲焉。

繼善屯二聖山,餘眾分屯沙坪、紅門、梓木嶺。貞復休兵數月,當暑,督兵出攻,肉薄,陟崖,大破賊。繼善棄砦走入杉關,師從之,進克金谿、南豐。復進克廣昌,攻瀘溪。瀘溪在萬山中,精忠將楊益茂、林鎮等以四萬人守隘,為之柵。貞督兵陟嶺,援柵以上,焚其壘,遂克瀘溪。懋珠寇南豐,貞赴援,擊賊楊梅巖,斬其將王大耀等,進克新城。十六年,懋珠、玉太走入樂安,副都統尼滿、提督趙賴與貞會師進攻。貞出西路,擊賊白石嶺,復樂安。玉太以六千人來降。韓大任自吉安走入樂安,貞督兵擊之,遇於跌水嶺。一日與八戰,走寧都,立木城都湖塹而守。圍之兩月,大任出走,敗之永豐,又敗之黃塘老虎峒,眾死亡略盡;走福建,詣康親王軍前降。

十七年,逐賊廣昌,破藤吊、楓樹二寨。二寨地絕峻,貞駐師永安山與相對,發火器遙擊,焚其寨,乃破之。敘功,進世職拜他喇布勒哈番,擢撫建廣饒吉南六府提督。廣信土賊江機、楊一豹以數萬人屯江滸山,倚險立木城,四出剽掠。貞與總督董衛國分兵進攻,迭克要隘。賊退入雞公山、猴子嶺,復襲破之,斬萬六千餘級。一豹走洪山。十八年,貞復督兵自弋陽雙港進攻,屢挫賊,斬一豹弟一虎及其眾二千四百有奇。機、一豹俱竄走。命貞提督江西全省軍務。十九年,逐賊入江滸山,貞詗賊謀夜劫營,令築壘,兵露刃立垣下,別將伏林中。賊至,見垣內刃如林,驚走,伏發,大破之。一豹、機亦走福建降。

二十一年,自陳乞罷,詔慰留。尋調廣東提督,朝京師,上褒勞備至,加拖沙喇哈番。貞蒞粵十四年,造哨船,設塘汛,晝夜巡邏,盜賊屏跡。三十四年,卒,贈少傅,賜祭葬。

貞和易,謙抑不伐。馭軍嚴整,戒所部毋淫掠。收城邑,他將議攻山砦,貞曰:「寇亂方烈,民結寨自保,非盡盜也。」止勿攻,全活甚眾。江西民甚德之。

周球,字季珍,江南來安人。順治十二年武進士,授廣州衛守備,署南贛營都司,管游擊事。石城陷,總兵劉進寶遣球赴援,與貞合兵擊賊寧都。民避亂紅石崖洞,賊積薪洞口,將舉火焚之。球兵至,賊走,民以得全。既克石城,復與貞援興國,球破南安土寇,克崇義、上猶境中諸寨,除游擊。復與游擊李天柱援會昌,破賊。康熙十四年,叛將陳升引精忠將郭應輔等陷龍泉,球與天柱破黃土關,克龍泉。逐賊,升自林中出誘戰,伏起,球督兵奮擊,大破之。攻左安口,陟自險徑,砲殪升。十五年,贛州增城守兵,球授參將,管副將事。被巡撫佟國正檄援信豐,破黃士標、王割耳等。十六年,援會昌,戰五里排。語詳國正傳。敘功,加都督僉事。復從貞擊破韓大任。大任既降,球與游擊唐光耀督護降卒至福州。復被大將軍簡親王喇布檄,以二千人從征湖南,守安仁,援永興,立營雞公山,屢破賊,加右都督。十八年,擢太原總兵,進左都督。調漢中,再調真定。二十二年,卒,贈太子少保,賜祭葬。

徐治都,漢軍正白旗人。父大貴,事太宗,援牛錄額真,兼工部理事官。師攻錦州,戰松山、杏山,克塔山,取中後所、前屯衛,皆在行間。順治間,從征太原,自河南徇江南。累遷刑部侍郎,兼梅勒額真。駐防杭州,領左翼。徇福建,攻海澄,還定舟山。累功,授世職三等阿思哈尼哈番,加太子少保。卒,謚勤果,賜祭葬。

治都,初授佐領,兼參領。康熙七年,擢直隸天津總兵。八年,調湖廣夷陵。吳三桂反,十三年,陷沅州,治都率師赴援。時四川文武吏附三桂,叛將楊來嘉、劉之復應之。治都妻許聞鄰境兵民皆從逆,權以治都令約束將弁,撫慰士卒,並脫簪珥勞軍。會上命治都還守夷陵,來嘉、之復以舟師來攻。治都督兵水陸防禦,擊卻之。來嘉據南漳,分路出犯,治都與襄陽總兵劉成龍會師合擊,所斬殺過半。敘功,加左都督。十五年,來嘉復以舟師來攻,治都循江堵截。總兵廨瀕江,寇舟逼廨,妻許督兵與戰,中砲死。總督蔡毓榮、提督桑額疏聞,具述治都忠奮不顧家狀。十八年,擢提督,以胡世英代為總兵。

賊將王鳳岐據巫山,上命治都戒備。治都練水師,修五板船百,令世英領之;而與成龍督兵出歸州、興山、巴東,扼形勢,相機進剿。十九年,師次巫山,來嘉、鳳岐以萬餘人拒守。師奪隘,賊突出,治都揮刃力戰,來嘉棄馬越山走,擒鳳岐,斬三千餘級,克巫山。進向夔州,夔州賊將劉之衛、瞿洪升以城降。叛將譚弘遣其子天秘、族人地晉、地升詣軍前請降,繳敕印。上命治都還守夷陵。弘復叛,陷瀘、敘二州。治都與鎮安將軍噶爾漢督兵溯江上,分軍為三隊擊賊,進克下關城。二十年,進向雲陽,屢敗賊。時弘已死,天秘走萬縣。治都復進復梁山、忠州。敘功,進四級。

二十七年,湖廣督標裁兵夏逢龍作亂,據武昌。治都督兵赴剿,至應城,與賊遇,力戰卻之。遂駐師應城。賊萬餘環攻,治都分兵內外夾擊,賊大潰,奔德安。逢龍乘北風聯巨艦二十順流下,見治都水師嚴整,不敢攻,乃登龍川磯攻陸師。治都督兵迎擊,晝夜鏖戰,斬殺殆盡。逢龍合餘眾泊鯉魚潦,治都令諸將鄭興、楊明錦防賊登陸,而自將水師循江發火器焚賊舟。逢龍再攻陸師,復戰卻之,斬七百餘級,餘多赴水死。其將胡耀乾等以武昌降,逢龍走黃州。振武將軍瓦岱督八旗兵至,黃岡諸生宜畏生擒逢龍以獻,磔於市,亂乃定。捷聞,賜孔雀翎,予世職拖沙喇哈番。

治都師未還,桃源土寇萬人傑為亂,治都妻孔督兵剿平之。三十二年,朝京師,賚御用冠服。三十三年,詔嘉治都功,用孫思克、施瑯例授鎮平將軍,仍領提督事。三十六年,卒,贈太子少保,謚襄毅,賜祭葬。

治都在湖廣十八年,整飭軍紀,民感其惠,為立祠以祀。

胡世英,字汝迪,安徽歙縣人。初從福建總督李率泰軍。累功至參將。康熙元年,遷湖廣督標中軍副將。十二年,擢副總兵,守荊州。吳三桂反,總督蔡毓榮檄為中軍。十四年,大將軍順承郡王勒爾錦自荊州渡江擊三桂,世英以四百人為前鋒。師集圍合,賊援至,沖我師,斷為二。世英張左右翼略陣,度師已畢濟,乃分騎隊逆戰,人持二炬,賊驚不敢逼,徐引還。十六年,常、澧諸郡饑,三桂將吳應麒屯岳州,糶倉穀以為利。世英密令人市焉,白勒爾錦乘賊饑督兵水陸並進。世英為前鋒,棹小舟直抵巴陵,溯風而戰,偪岸且近。世英呼而登曰:「得城陵磯矣!」師畢登,破賊壘。十八年,應麒走,城民迎師入。勒爾錦請設隨征四鎮,世英授後路總兵,尋調夷陵。十九年,從治都克巫山,擒鳳岐,進取重慶。以病還夷陵,未幾卒。

唐希順,甘肅武威人。自行伍補涼州鎮標把總。康熙十三年,王輔臣叛,希順從總兵孫思克進剿河東,轉戰有功。十五年,從圍平涼,破賊虎山墩,希順奮勇爭先,手足被傷。敘功,予參將銜,管提標千總。尋遷守備,偕參將康調元攻復階州、文縣。

十九年,勇略將軍趙良棟征四川,調希順從軍,遷四川川北鎮標游擊。時吳世璠將胡國柱等踞關山大象嶺,良棟軍由雅州進剿,復榮經。賊退入箐口驛,分兵扼周公橋、黃泥鋪諸隘,結五營守險。希順從總兵李芳述及調元等進攻土地橋,連破其壘。抵橋口,選步兵千,由間道穿山箐,自山頂下攻。會橋口兵夾擊,賊潰遁。乘夜追襲,次日,復敗賊於黎州,克其城,追至大渡河,奪渡口三,遂復建昌。其冬,從良棟自金沙江下雲南,敗賊於玉皇閣、三市街。二十二年,敘功,予左都督銜。累遷臺灣水師副將。三十二年,擢貴州威寧鎮總兵。

三十五年,聖祖親征噶爾丹,命希順隸西路進剿。自貴州率親丁百,馳抵寧夏。大軍已出塞,希順兼程進,與孫思克軍會,破噶爾丹於昭莫多。敘功,予世職拖沙喇哈番,擢四川提督。疏言:「川省幅員遼闊,蠻、苗雜處,水陸交錯。提標三營,請視各省提標例,營設兵八百。川省額兵三萬六千,臣清釐積弊,兵額充足。即於原額內酌量營汛緩急抽調。提標兵雖他移,餉仍其舊。標下將備等官,材技優長,弓馬嫺熟,又諳蜀中地利。請如松潘、疊溪等營保題事例,擇員題補。」允之。

打箭爐舊屬內地,上以西藏番部嗜茶,許西藏營官在打箭爐管理土伯特貿易事。三十九年,營官喋巴昌側集烈為亂,侵據河東烏泥、若泥、嵐州、善慶、擦道諸處,戕明正、長河西土司蛇蠟喳吧。總督錫勒達奏請移化林營參將李麟督兵捕治。賊復攻圍烹壩、冷竹關,希順檄各路兵赴化林,密疏聞。上命侍郎滿丕統荊州滿洲兵進剿,並詔希順相機行事。蠻兵五千餘,立營十四,在磨西面及磨岡等處。希順雪夜渡瀘水,分兵三路進攻:一自子牛攻哪吒頂,一自烹壩攻大岡,一自督兵出咱威攻磨西面及磨岡。別遣兵自頭道水登山,馳下夾攻。戰五日,各路俱捷,殲蠻兵五千餘,斬喋巴昌側集烈,遂復打箭爐,喇嘛、番民俱降。尋抵木鴉,番目錯王端柱等繳敕印,歸附喇嘛、番民萬二千餘戶。捷聞,詔嘉獎。尋疏陳善後事,並允行。未幾,以病乞休,命解任調理。四十七年,卒,予祭葬如制。子際盛,襲職,入籍四川。

李麟,陜西咸陽人。自行伍從勇略將軍趙良棟下雲南。敘功,以都司僉書用。康熙三十五年,從振武將軍孫思克擊噶爾丹於昭莫多,大敗之。累遷四川化林營參將。三十九年,昌側集烈作亂,麟奉檄移兵渡瀘,招安咱威、子牛、烹壩、魁梧四處。尋提督唐希順令麟順瀘水至哦可,出磨西面後,夾攻磨岡。麟軍夜迷失道,比明,反出磨西面前,遂攻蠻營,奪磨西面。打箭爐平。希順追劾麟避險就易駐咱威,致失烹壩;又進兵迷道,誤軍機。詔總督錫勒達及滿丕等訊鞫,以有功免治罪。累擢登州總兵。

五十七年,策妄阿喇布坦擾西藏,命麟選精兵百,自寧夏赴軍前。五十九年,詔都統延信為平逆將軍,率兵進藏,以麟參贊軍務。尋令護送第六世達賴喇嘛進藏,至沙克河,賊乘夜襲營,擊敗之,連敗賊於齊諾郭勒、綽瑪喇等處。西藏平,麟率兵自拉里凱旋。六十年,授陜西固原提督。雍正元年,遷鑾儀使。追敘平藏功,加右都督,予世職拖沙喇哈番。以老致仕。尋卒。

趙應奎,河南商丘人。少入伍,從恭順王孔有德征湖南、廣西,俱有功。累遷至湖廣施南副將。

康熙十三年,吳三桂陷長沙,調應奎為江西袁州副將。袁州地逼長沙,又有棚寇,與三桂兵句連。應奎以所部兵力弱,斥貲增募,並家丁助戰,擒斬賊渠硃益吾等。尋自慈化進剿黃塘、楚山、上慄市,屢敗賊。總督董衛國請設袁臨鎮,即以應奎為總兵官。三桂遣賊犯袁州,應奎力守。未幾,其將硃君聘等以數萬人自萍鄉來犯,應奎敗之西村,斬萬五千餘級。分兵趨萬載,斬其將邱以祥等,復其城。三桂使誘降,應奎令子衍慶呈部,部議加應奎左都督,衍慶署都司僉書。尋降敕嘉其忠藎,予世職拜他喇布勒哈番。十四年,遣游擊楊正元剿棚賊於分宜、新喻,擒斬甚眾,盡毀其巢。三桂將揭玉卿犯萬載,遣游擊陳素綸等敗之,斬級千餘;又敗之於白良。三桂將黃立卿復以三桂書誘降,應奎令子衍祥呈部,部議加應奎軍功一等,衍祥授鴻臚寺少卿。十五年,遣游擊李顯宗等逐三桂兵至仙居橋、沙溪、湖塘,皆敗之。三桂兵復結瀏陽諸寇陷萬載,應奎進剿,賊截龍河渡口,夾岸迎拒。應奎督兵渡河,先斬守口賊,直入其壘,賊大潰,追斬無算,復萬載,詔嘉獎。尋授三等阿達哈哈番。

十七年,上以江西已定,命應奎統本標官兵移鎮茶陵、攸縣。疏言:「自三桂反,袁州密邇湖南,臣率孤軍征剿,上游幸獲安全。但彼時兵力苦單,漕運亦匱,臣捐貲贍養親兵,或自備馬匹,或獎以虛銜。嗣戶部侍郎溫岱奏見臣督親丁防禦,蒙恩給臣所養健丁千人步戰兵餉,令臣量為設官管轄。惟兵丁既叨餉餼,而所設管轄官未議實授。今臣移駐茶、攸,僅率標兵二千六百,現奉征南將軍穆占、定南將軍華善調往酃縣千四百人。健丁一營,隨臣左右。仰冀天恩,各予實銜,開支實俸。」詔從之。未幾,賊犯永興,敗之。十八年,從大將軍簡親王喇布復祁陽、新寧。大將軍安親王岳樂檄剿賊武岡州楓木嶺,敗三桂將胡國柱等。尋偕貴州提督趙賴攻克龍頭山、泡洞口、瓦屋塘、雲霧嶺、五子坡諸寨。三桂將馬寶敗遁,追擊之,復會同、黔陽等縣。未幾,建義將軍馬承廕以柳州叛,從簡親王率兵討之,承廕降。

二十一年,命以提督充廣西左江鎮總兵。敘功,進二等阿達哈哈番。疏言:「臣昔任思南副將,深知左江為滇、黔門戶,接壤交南,環以僮、瑤,土司不時反覆。鎮標額設四營,共兵三千有餘,多從逆歸命者,習成驕悍。臣標健丁一營,半系親屬,久經訓練,請率赴新任,以資鈐壓。」從之。未幾,以病累疏乞休,詔輒慰留,命衍祥馳驛歸省。應奎卒,贈太子少保,謚襄壯。

趙賴,漢軍正藍旗人。父夢彩,事太宗,以監修福陵,授世職二等阿達哈哈番。賴襲職。從謙郡王瓦克達征叛將姜瓖,以功進一等,並兼拖沙喇哈番。擢正藍旗漢軍副都統。康熙十三年,從大將軍順承郡王勒爾錦討吳三桂,擢貴州提督,統兵駐九江,調江西。韓大任陷吉安,賴率兵擊敗之。復調湖南,從簡親王喇布剿賊衡山,復衡州府。迭克耒陽、祁陽等縣。敗三桂將吳國貴等,復武岡。十九年,從大將軍貝子彰泰、將軍蔡毓榮進攻貴州,迭克賊寨。敗馬寶於洪江,復黔陽,旋自沅州趨鎮遠,復黎平、銅仁、思州、思南等府。偕將軍穆占敗三桂將高啟隆、夏國相等,復平遠府。大軍進征雲南,詔賴留鎮貴州,擢正藍旗漢軍都統。以老乞休。三十一年,卒。

李芳述,四川合州人。初入伍,隸貴州大定總兵劉之復標下。剿水西土司安坤有功,授千總。

康熙十三年,吳三桂反,之復從逆,脅芳述往湖北,據夷陵、巴東關隘。芳述脫走,留四川,其妻子在大定。越五年,乃得取妻子至敘州。吳世璠加芳述偽總兵,令自巫山襲鄖、襄。芳述留重慶。十九年,勇略將軍趙良棟進取成都,芳述遣人赴軍前呈繳偽劄,率重慶、瀘州、敘州所屬州縣文武吏降。良棟令芳述撫永寧,即移軍駐守,修繕城垣。甫竣事,世璠將毛友貴等以數萬人來犯,芳述迎擊,賊卻走。尋以悍卒數千偪城,夜樹雲梯攀堞,芳述督兵鏖戰,斃賊千餘,斬友貴於陣。世璠將胡國柱、王邦圖等以顯武將軍印招芳述,芳述封送良棟。良棟以聞,詔授隨征總兵。

未幾,賊陷仁懷、合江。芳述移兵守敘州,擒賊諜,斬以徇。賊來犯,芳述壁城外真武、翠屏諸山,賊不得逞,潛退馬湖,謀出木川、犍為襲成都。芳述言冋知之,先率兵至犍為扼其沖,大破賊,躡擊至新增黃茅岡,斬殺過半。降其將夏升、羅應甲等,拔被掠民二千有奇。擢西寧總兵官,仍從征雲南。二十年,良棟令為前鋒,自洪雅、榮經二縣出大象嶺之左,敗賊關山。時國柱踞建昌,聞關山、大象嶺俱失,棄建昌走雲南。芳述渡金沙江,會良棟軍取雲南,奪得勝橋,拔其東西二營,遂克雲南。

三十一年,遷貴州提督。四十年,雲南總督巴錫疏劾游擊高鑒,語連芳述徇隱,芳述亦疏訐巴錫,上遣侍郎溫達往讞。芳述應奪俸,免之。四十二年,湖南鎮筸紅苗作亂,芳述移兵會剿,深入苗地,平塘山及葫蘆、天星諸寨。疏言:「貴州苗、民雜處,控制尤在得人。保題武職,請以久任苗地、熟悉風土者揀選題補。」詔允行。四十五年,詔獎「芳述久鎮邊境,馭軍有法。現今舊將,罕與比倫」。特加太子少保,授鎮遠將軍。四十七年,卒,贈太子少傅,謚壯敏,賜祭葬。

陳世凱,字贊伯,湖廣恩施人。初附明桂王,為忠州副總兵。順治十六年,川陜總督李國英駐師重慶,世凱來降,授副將銜。康熙十年,李自成餘黨劉一虎等以數萬人犯巫山,世凱擊卻之。尋從國英進剿,以功加總兵銜。十一年,授杭州副將。

十三年,耿精忠反,浙江總督李之芳駐師衢州,令世凱援金華。甫渡江,聞寇犯龍游,即遣兵馳擊,通衢州餉道。既至金華,精忠將閻標自永康、武義來犯,世凱與副都統沃申御之,發砲擊賊。既,復與總兵李榮逐賊湯溪,分兵出賊後,而自當其前,獲所置監軍道徐福龍等。精忠將陳重自東陽、葉鍾自浦江先後來犯,與副都統瑪哈達、石調聲督兵擊之敗走。援義烏,破精忠將周彪。敘功,授溫州總兵,加都督僉事。精忠將徐尚朝以數萬人逼金華,世凱出城南十二里與戰,寇甫集,大呼陷陣,寇潰奔,逐北十餘里,殺傷過半。尚朝與精忠將馮公輔合,得五萬人,據積道山,立木城石壘。世凱乘大霧進兵,破木城,斬級萬餘,尚朝敗走。

大將軍康親王傑書師進次金華,令世凱及瑪哈達、榮規處州。十四年,世凱復永康,進攻縉雲,擊破尚朝兵,克之。精忠將沙有祥守處州,壘桃花嶺拒守。世凱等師三道入,奪嶺,有祥走,克處州。尚朝來犯,三戰破賊,獲其裨將,斬八百餘級。移師徇松陽,從貝子傅喇塔攻溫州。十五年,精忠將曾養性及叛將祖弘勛以四萬餘人拒我師,世凱與提督段應舉奮擊,獲其裨將。詔傅喇塔進徵福建,世凱以所部從。擊養性得勝山,破其壘。寇舟屯江山,督兵擊之,師行乃無阻,復云和、泰順諸縣。精忠降,世凱還鎮溫州。十六年,加左都督,予世職拖沙喇哈番。屢招降鄭錦將陳彬、劉天福等。二十二年,進拜他喇布勒哈番。朝京師,上獎其績,諭「輯兵愛民,毋以功大生驕傲」,賜鞍馬、裘服。

二十三年,擢浙江提督。上制聖訓十六條,宣諭士民。世凱請令將卒一體講讀,並援引經史依類附注,為書三卷,奏進頒行。又奏春秋祭文廟,宜令武職行禮。下九卿議行。二十八年,復朝京師,命還任,以疾未行,卒。遣內大臣佟國維、侍衛馬武奠茶酒,賜祭葬,謚襄敏。字天培,授都司。累遷至浙江提督。世凱勇敢善戰,所向有功,軍中呼為陳鐵頭。

浙中諸將,佐之芳戡亂者,又有李榮、王廷梅、牟大寅、鮑虎、蔣懋勛。榮,字華庵,廣寧人。黃巖總兵。廷梅,順天人。武進士。自督標中營副將遷平陽總兵。大寅,湖廣人。鎮海總兵。虎,字雲樓,山西應州人。初授南贛鎮標前營游擊。擊李成棟有功,累遷浙江嚴州城守副將。從之芳御精忠,克壽昌。破土寇黃應茂。尋代榮為黃巖總兵。懋勛,浙江臨海人。溫州總兵。謚襄僖。

許占魁,字文元,陜西蒲城人,流寓遼東。順治初,從豫親王多鐸定江南,授陜西陽平關參將。六年,土寇趙榮貴擁明宗人森滏號秦王,聚數萬人犯階州。占魁從間道出碧魚口襲其後,先與定西將軍李國翰、臨鞏總兵王允久期夾擊,大破之。遷山西平陽副將。土寇張武挾硃秀唐號魏王,掠聞喜。占魁與游擊苗成龍等分道搜剿,戰紫家峪,擒秀唐等,斬級百餘。累調直隸紫荊關副將。康熙九年,擢延綏總兵,駐榆林。

十三年,提督王輔臣、副將硃龍俱叛應吳三桂,占魁舉首龍所與逆書,上嘉之,下部議敘,加都督同知。延綏標兵多應調徵四川,龍等窺榆林防守單弱,屢糾眾來犯。占魁遣副將張國彥、孫維統,游擊謝鴻儒、錢應龍等分道堵剿,自督兵擊賊綏德。賊踞城以拒,發砲,斃賊數百。占魁慮賊襲榆林,率維統等還守榆林,令國彥守波羅堡。龍誘波羅營千總劉尚勇等叛,逼國彥,劫奪敕印。國彥闔門自焚死。叛將孫崇雅戕神木道楊三知、知縣孫世譽、守備張光鬥等,遂踞神木,勢張甚。占魁遣子登隆詣闕告急,詔授登隆鴻臚寺少卿,趣將軍畢力克圖、都統覺和托自大同移師赴援。占魁遣維統、應龍等從覺和托擊賊,擒斬無算。復魚河、響水、波羅諸堡,進克神木。畢力克圖復綏德、延安,擒崇雅、尚勇等,悉誅之。國彥、三知等並賜恤,從征將弁敕議敘。

占魁疏言:「王輔臣嗾硃龍竊踞定邊,遂陷綏德、米脂、葭州、神木,賊騎至歸德堡,北距榆林僅二十里。臣集闔城官民誓死守城。嗣因臨洮、鞏昌、延安、慶陽、平涼、漢中、興安、固原相率從逆,榆林一城獨存,餉道隔絕,百姓日食糠秕。臣斥貲購米,計口授食。及大兵既至,道臣高光祉籌措糧需,將士奮勇擊賊,剋期奏凱,危城得安。皆由文武同心,兵民合力。其在城各官勞績,祈敕部覈議,為固守孤城者勸。」上俞之,俱命優敘。占魁進左都督,予世職拜他喇布勒哈番。尋以疾乞罷,溫旨慰留。十六年,擢鑾儀使。占魁復以病辭,允馳驛回籍,仍食俸。卒,贈太子少保,賜祭葬視一品,謚恪敏。子登隆,官至雲南臨安知府。

論曰:順治初,漢兵降,猶分隸漢軍;其後撫定諸行省,設提鎮,置營汛,於是有綠營。以綠營當大敵,建愬定之績,自三籓之役始。蔡毓榮、趙良棟將綠營直下雲南諸行省,以戰伐顯者,如國祚輩,皆彰彰有名氏,而治都、芳述功尤著。貞治屯墾,奮起效績,不煩餉運,蓋更有難能者。腹心爪牙,由此其選矣。


\end{pinyinscope}