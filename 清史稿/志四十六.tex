\article{志四十六}

\begin{pinyinscope}
地理十八

△臺灣

臺灣:古荒服之地,不通中國,名曰東番。隋開皇中,遣虎賁陳棱略澎湖三十六島。明嘉靖四十二年,海寇林道乾掠近海郡縣,都督俞大猷征之,追至澎湖,道乾遁入臺灣。天啟元年,閩人顏思齊引日本國人據其地。久之,為荷蘭所奪。清順治十八年,海寇鄭成功逐荷蘭人據之,偽置承天府,名曰東都,設二縣,曰天興,曰萬年。其子鄭經改東都為東寧省,升二縣為州。康熙二十二年討平之,改置臺灣府,屬福建省,領縣三。雍正元年,增置彰化縣,領縣四。光緒十三年,改建行省。光緒十三年九月庚午,閩浙總督楊昌濬、臺灣巡撫劉銘傳會奏,略云:「臺灣疆域,南北相距七百餘里,東西近者二百餘里,遠或三四百里,崇山大溪,鉤連高下。從前所治,不過山前迤南一線,故僅設三縣而有餘。自後榛莽日開,故屢增治而猶不足。光緒元年,沈葆楨請設臺北府、縣以固北路,又將同知移治卑南以顧後山,全臺官制,粗有規模。然彼時局勢,未聞擇要修舉,非一勞永逸之計也。臣等公同商酌,竊謂建置之法,恃險與勢,分治之道,貴持其平。臺省治理,視內地為難,而各縣幅員,反較多於內地。如彰化、嘉義、鳳山、新竹、淡水等縣,縱橫二百餘里、三百里不等,倉卒有事,鞭長莫及。且防務為治臺要領,轄境太廣,則耳目難周,控制太寬,則聲氣多阻。至山後中、北兩路,延袤三四百里,僅區段所設碉堡,並無專駐治理之員,前寄清虛,亦難遙制。現當改設伊始,百廢俱興,若不量予變通,何以定責成而垂久遠?臣銘傳於上年九月親赴中路督剿叛番,沿途察看地勢,並據各地方官,將境內扼塞道里、田園山溪,繪圖貼說,呈送前來。又據撫番清賦各員弁將撫墾地所陸續稟報,謹就山前後通局籌畫,有應添設者,有應改設者,有應裁撤者。查彰化橋孜圖地方,山環水衣復,中開平原,氣象宏敞,又當全臺適中之地,擬照前撫臣岑毓英議,就該處建立省城,分彰化東北之境設首府曰臺灣府,附郭首縣曰臺灣縣,將原有之臺灣府、縣改為臺南府安平縣。嘉義之東,彰化之南,自濁水溪至姑石圭溪止,截長補短,方長四百餘里,擬添設一縣曰雲林縣。新竹苗慄街一帶,扼內山之沖,東連大湖,沿山新墾荒地甚多,擬於新竹西南各境添設一縣曰苗慄縣,合原有之彰化,及埔里社通判,一、四縣,均隸臺灣府屬。其鹿港同知一缺,應即裁撤。淡水之北,東控三貂嶺,番社歧出,距縣太遠。基隆為臺北第一門戶,通商建埠,交涉紛繁,現值開採煤礦,修造鐵路,商民麕集,尤賴撫綏。擬分淡水東北四堡之地,撤歸基隆管轄,將原設通判改為撫民理番同知,以重事權。此前路添改之大略也。後山形勢,北以蘇溪為總隘,南以卑南為要區,控扼中權,厥惟水尾。其地與擬設之雲林縣東西相直,現開路百九十餘里,由丹社嶺集集街經達彰化,將去省城,建立中路,前後脈絡,呼吸相通,實為臺東鎖鑰。擬添設直隸州知州一員曰臺東直隸州,左界宜蘭,右界恆春,計長五百餘里,寬約四十里、十餘里不等,統歸該州管轄,仍隸於臺灣兵備道。其卑南舊治,擬請改設直隸州同一員。水尾迤南,改為花蓮港。其內已墾熟田約數千畝。其外海口水深數丈,稽查商舶,彈壓民番,擬請添設直隸州判一員,常川駐扎,均隸臺東直隸州屬。此後路添設之大略也。謹按臺灣疆土賦役,日增月廣,與舊時羈縻僑置情形迥不相同,因地制宜,似難再緩。況年來生番歸化,狉榛之性初就範圍,尤須分道拊循,藉收實效。輯遐牖邇,在在需員,臣等身在局中,既不敢遇事紛更,以紊典章之舊,亦不敢因陋就簡,以失富庶之基,損益酌中,期歸妥協。」二十一年,割隸日本。省在福建東南五百四十里。西北距京師七千二百五十里。東界海;西界澎湖島;南界磯頭海;北界基隆城海。廣五百里,袤一千八百里。一統志載戶口原額人丁一萬八千八百二十七,滋生男婦大小口共一百七十八萬六千八百八十三,戶二十二萬四千六百四十六。領府三,州一,三,縣十一。臺灣屹峙海中,為東南屏障,四面環海,崇山峻嶺,橫截其中,背負崇岡,襟帶列島。浪嶠南屏,雞籠北衛,澎湖為門戶,鹿耳為咽喉。七鯤身毗連環護,三茅港匯聚澄泓。畜牧之饒,無異中土。誠東南一大都會也。

臺灣府:沖,繁,疲,難。為臺灣省治。巡撫、布政使、分巡兵備道兼按察使銜,共駐。其地東及東南界臺東州;西及北界海;南及西南界臺南府;東北界臺北府。廣袤里數闕。北極高二十四度三十三分。京師偏東四度二十分。領縣四,一。臺灣沖,繁,疲,難。倚。分彰化縣治。葫蘆墩,巡司駐。彰化繁,難。府北百里。鹿港,縣丞駐。雲林難。林圮埔,縣丞駐。苗慄沖。大甲,巡司駐。埔里社調。府東南。其山在府境者,西北:五鶴、牛困山。西:史老榻山。南:蘆芝、芎根、郡坑、松柏山、土山。東:內山。濁水出埔里社東南山,西南流,左合二水,經雲林縣東北,一水自南來注之。曲北,右納一水,經縣北。又西北,一水自嘉義縣來,北流注之。又西經牛埔厝,歧為三支:一支曰石龜溪,西為牛椆溪;一支曰虎尾溪,經汕頭厝為麥藔港,並經縣西入於海;一支為東螺溪,又歧為三,曰刺桐港即番挖港,曰鹿港,曰二林港,並經彰化縣西入於海。大肚溪上源曰合水溪,出埔里社東南魚池仔,西北流,合南硿溪,經西北,北港溪、北硿溪並西流注之。又西,珠子山二水合西北流注之,經府治南,左右各納一水,經大肚街為大肚溪,又西北入於海。大甲溪出苗慄縣東南,合數小水,西南流,右出支津注於吞霄溪。正渠南流,左納一水,折西北,經鐵砧山南,又西北入於海。吞霄溪出苗慄縣東南,合大甲溪支津,經縣南,西北流入於海。後壟溪出苗慄縣東南山,合一水,西北流,經五鶴山,南至鋼羅灣,夾二洲,又西北,經縣治北,右通中港溪,左納一水,入於海。中港溪出縣東山,緣界西北流,經縣治,左出支津合後壟港支水為烏眉溪,與正渠並西北入於海。

臺南府:沖,繁,難。舊臺灣府改設。東北距省治二百里。東及東南界臺東州;西及南界海;北及東北界臺灣府。廣袤里數闕。北極高二十三度。京師偏東三度三十一分。領縣四,一。噶瑪蘭頭圍,巡司駐。安平沖,繁,難。倚。大武壟、斗六門二巡司。鳳山繁,難。府南八十里。下淡水,縣丞駐。枋寮,巡司駐。嘉義繁,難。府北一百十七里。笨港,縣丞駐。佳里興,巡司駐。恆春疲,難。澎湖簡。府西水程二百四十里。澎湖,總兵駐。澎湖八罩巡司。其山在府境者,北:太湖、白水、木岡山。東北:大福興、大利山。東:觀音、枕頭山。北:華山。東南:武吉、草山。南:虎頭、龜山。西:鳳山,鳳山縣以此得名。鳳山北大岡滾水、大武壟、大木岡山。縣東傀儡山,俗曰加禮山。澎湖懸居海中。牛椆溪出嘉義縣東,西北流,經治北,與布袋嘴港並西流入於海。八掌溪出雲林縣界,西北流,經平鼻山北半月山,南合瀵箕湖及一小水,西流至鹽水港,入於海。急水溪二源,並出雲林縣西界,經嘉義縣東南,合西流,又經急水鋪南,左納十八重溪,又經鐵線橋街北,又西入於海。曾文溪出府治東北,西北流,經大武壟北,右納茄拔溪,左納一小水,經府治北,又西經倒風港,入於海。柴頭港出府治東北山,西北流,經治北,又西合德慶港為安平港,入於海。二層行溪出府治東,茄定港出雁門關嶺,阿公店溪出鳳山縣東北,並西流入於海。淡水溪出府治東六張犁,西南流,右納一水,左納二水,經下淡水西,鳳山縣治東,至潮州厝汛北,西冷水溝水出縣東芋匏山,合二水西南流注之。又西南匯為東港,入於港海。茄藤港在鳳山縣南,西流入於海。率芒溪出恆春縣北武吉山,合一水,西流入於海。刺桐港、楓港、五重溪、三重港、射寮溪並在恆春縣北,西流入於海。龍鑾潭在恆春縣南,西北流入於海。豬犬勞束港在恆春縣東,東流入於海。

臺北府:沖,繁。西南距省治三百五十里。東、北、西界海;南界臺東州;西南界臺灣府。廣袤里數闕。北極高二十五度十七分。京師偏東五度十五分。領縣三,一。淡水沖。倚。新竹疲,難。府西南。宜蘭疲,難。府東南。頭圍,縣丞駐。基隆沖,繁。府東北二百七十五里。其山在府境者,北:大屯、沙帽、大武壟山。東北:雞籠山,在基隆東。府城東:攀山。南:瓦窯山、大羈尖山、五指山。西南:橫山、金面山、虎頭山。西南:嵌山。海環府東、北、西三面。基隆口在基隆東北。扈尾口在府治西北。磺溪出府治南山,合石頭溪,東北流,左右各納一小水,至枋橋街,紅仙水合擺接溪諸水西流注之。又北經府治西,艋舺、十八重溪水北流折東注之。至大稻埕。大隆洞溪出基隆東雞籠山,合一水西流注之。又西北,分流復合,經扈尾港入於海。南崁港上流為大過溪,在府治西北。中瀝溪、土牛溝、紅毛港、鳳山崎溪、舊港、油車港、香山港並在新竹縣西北入於海。三貂溪在基隆東南,草嶺大溪、加禮遠港、蘇澳門並在宜蘭縣南,俱入於海。

臺東直隸州:沖,繁,疲,難。卑南改設。西北距省治五百里。東及南界海;西及西北界臺灣府;北界臺北府;西南界臺南府。廣袤里數闕。北極高二十二度二十五分。京師偏東四度。卑南,州同駐。花蓮港,州判駐。其山在州境者,北:岐來山、能高山。西:秀姑巒山。東:丁象山。西出八同關,為秀姑巒山一帶番社,系屬巒番所居。西南一帶高山番社,系屬昆番所居。大港上源曰打馬溪,出秀姑巒山,東流經治北,右合網網溪,左合一水,經奇密社北,入於海。卑南大溪出州西南新武洛社,合三水東南流入於海。花蓮港二源,並出州西北,合數小水,經太平廠南入於海。東澳、南澳、大濁水溪、大清水溪、小清水溪、得其黎溪、三棧溪、尤丹溪、米侖港並在州東北,入於海。紅蝦港、黎仔阬溪、郎阿郎溪、馬武窟溪、八里芒溪、呂家望溪、知本溪、大苗里溪、虷子侖溪、大足高溪、乾子壁溪、大烏萬溪、巴塱衛溪、魯木鹿溪、牡丹灣、八磘灣並在州東南,入於海。


\end{pinyinscope}