\article{列傳七}

\begin{pinyinscope}
諸王六

聖祖諸子

貝子品級允禔理密親王允礽誠隱郡王允祉

恆溫親王允祺淳度親王允祐允禩允禟

輔國公允示我履懿親王允祹怡賢親王允祥

恂勤郡王允愉恪郡王允潖果毅親王允禮

果恭郡王弘適簡靖貝勒允禕慎靖郡王允禧

質莊親王永瑢恭勤貝勒允祜郡王品級誠貝勒允祁

諴恪親王允祕

世宗諸子

端親王弘暉和恭親王弘晝懷親王福惠

聖祖三十五子:孝誠仁皇后生承祜、理密親王允礽,孝恭仁皇后生第六子允祚、世宗、恂勤郡王允,敬敏皇貴妃章佳氏生怡賢親王允祥,溫僖貴妃鈕祜祿氏生貝子品級允示我,順懿密妃王氏生愉恪郡王允潖、莊恪親王允祿、第十八子允祄,純裕勤妃陳氏生果毅親王允禮,惠妃納喇氏生承慶、貝子品級允禔,宜妃郭絡羅氏生恆溫親王允祺、第九子允禟、第十一子允禌,榮妃馬佳氏生承瑞、賽音察渾、長華、長生、誠隱郡王允祉,成妃戴佳氏生淳度氏生第八子允禩,定妃萬琉哈氏生履懿親王允祹,平妃赫舍里氏生允禨,通嬪納喇氏生萬黼、允禶,襄嬪高氏生第十九子允禝、簡靖貝勒允禕,謹嬪色赫圖氏生恭勤貝勒允祜,靜嬪石氏生郡王品級誠貝勒允祁,熙嬪陳氏生慎靖郡王允禧,穆嬪陳氏生諴恪親王允祕,貴人郭絡羅氏生允示禹,貴人陳氏生允示爰。允祿出為承澤裕親王碩塞後,允祚、允禌、允祄、允禝皆殤,無封。承瑞、承祜、承慶、賽音察渾、長華、長生、萬黼、允禶,允騕、允禨、允示爰皆殤,不齒序。

固山貝子品級允禔,聖祖第一子。上有巡幸,輒從。康熙二十九年,命副裕親王福全御噶爾丹。上以允禔聽讒,與福全不協,私自陳奏,慮在軍中僨事,召還京師。未幾,福全師還,命諸王大臣勘鞫。福全初欲發允禔在軍中過失,會有嚴旨戒允禔不得與福全異同,福全乃引罪。語在福全傳。三十五年,從上征噶爾丹,命與內大臣索額圖統先發八旗前鋒、漢軍火器營與四旗察哈爾及綠旗諸軍駐拖陵布喇克待上。西路大將軍費揚古軍後期,下軍中大臣議,亦遣官諮允禔。上遂進軍昭莫多。既捷,允禔留中拖陵犒軍,尋召還。三十七年三月,封直郡王。三十九年四月,上巡視永定河堤,鳩工疏濬,命允禔總之。

四十七年九月,皇太子允礽既廢,允禔奏曰:「允礽所行卑?,失人心。術士張明德嘗相允禩必大貴。如誅允礽,不必出皇父手。」上怒,詔斥允禔兇頑愚昧,並戒諸皇子勿縱屬下人生事。允禔用喇嘛巴漢格隆魘術魘廢太子,事發,上命監守。尋奪爵,幽於第。四月,上將巡塞外,諭:「允禔鎮魘皇太子及諸皇子,不念父母兄弟,事無顧忌。萬一禍發,朕在塞外,三日後始聞,何由制止?」下諸王大臣議,於八旗遣護軍參領八、護軍校八、護軍八十,仍於允禔府中監守。上復遣貝勒延壽,貝子蘇努,公鄂飛,都統辛泰,護軍統領圖爾海、陳泰,並八旗章京十七人,更番監守,仍嚴諭疏忽當族誅。

雍正十二年,卒,世宗命以固山貝子禮殯葬。子弘昉,襲鎮國公。卒。子永揚,襲輔國公。坐事,奪爵。高宗以允禔第十三子弘晌封奉恩將軍,世襲。

理密親王允礽,聖祖第二子。康熙十四年十二月乙丑,聖祖以太皇太后、皇太后命立為皇太子。太子方幼,上親教之讀書。六歲就傅,令大學士張英、李光地為之師,又命大學士熊賜履授以性理諸書。二十五年,上召江寧巡撫湯斌,以禮部尚書領詹事。斌薦起原任直隸大名道耿介為少詹事,輔導太子。介旋以疾辭。逾年,斌亦卒。太子通滿、漢文字,嫺騎射,從上行幸,賡詠斐然。

二十九年七月,上親征噶爾丹,駐蹕古魯富爾堅嘉渾噶山,遘疾,召太子及皇三子允祉至行宮。太子侍疾無憂色,上不懌,遣太子先還。三十三年,禮部奏祭奉先殿儀注,太子拜褥置檻內,上諭尚書沙穆哈移設檻外,沙穆哈請旨記檔,上命奪沙穆哈官。三十四年,冊石氏為太子妃。

三十五年二月,上再親征噶爾丹,命太子代行郊祀禮;各部院奏章,聽太子處理;事重要,諸大臣議定,啟太子。六月,上破噶爾丹,還,太子迎於諾海河朔,命太子先還。上至京師,太子率?臣郊迎。明年,上行兵寧夏,仍命太子居守。有為蜚語聞上者,謂太子暱比匪人,素行遂變。上還京師,錄太子左右用事者置於法。自此眷愛漸替。

四十七年八月,上行圍。皇十八子允祄疾作,留永安拜昂阿。上回鑾臨視,允祄病篤。上諭曰:「允祄病無濟,區區稚子,有何關系?至於朕躬,上恐貽高年皇太后之憂,下則系天下臣民之望,宜割愛就道。」因啟蹕。

九月乙亥,次布爾哈蘇臺,召太子,集諸王大臣諭曰:「允礽不法祖德,不遵朕訓,暴戾淫亂,朕包容二十年矣。乃其惡愈張,僇辱廷臣,專擅威權,鳩聚黨與,窺?肆惡虐伺朕躬起居動作。平郡王訥爾素、貝勒海善、公普奇遭其毆撻,大臣官員亦罹其毒。朕巡幸陜西、江南、浙江,未嘗一事擾民。允礽與所屬恣行乖戾,無所不至,遣使邀截蒙古貢使,攘進御之馬,致蒙古俱不心服。朕以其賦性奢侈,用凌普為內務府總管,以為允礽乳母之夫,便其徵索。凌普更為貪婪,包衣下人無不怨憾。皇十八子抱病,諸臣以朕年高,無不為朕憂,允礽乃親兄,絕無友愛之意。朕加以責讓,忿然發怒,每夜偪近布城,裂縫竊視。從前索額圖欲謀大事,朕知而誅之,今允礽欲為復仇。朕不卜今日被鴆、明日遇害,晝夜戒慎不寧。似此不孝不仁,太祖、太宗、世祖所締造,朕所治平之天下,斷不可付此人!」上且諭且泣,至於僕地,即日執允礽,命直郡王允禔監之,誅索額圖二子格爾芬、阿爾吉善,及允礽左右二格、蘇爾特、哈什太、薩爾邦阿;其罪稍減者,遣戍盛京。次日,上命宣諭諸臣及?侍敢不從,即其中豈無奔走逢迎之人?今事內幹?官兵,略謂:「允礽為太子,有所使令,連應誅者已誅,應遣者已遣,餘不更推求,毋危懼。」

上既廢太子,憤懣不已,六夕不安寢,召扈從諸臣涕泣言之,諸臣皆嗚咽。既又諭諸臣,謂:「觀允礽行事,與人大不同,類狂易之疾,似有鬼物憑之者。」及還京,設氈帷上駟院側,令允礽居焉,更命皇四子與允禔同守之。尋以廢太子詔宣示天下,上並親撰文告天地、太廟、社稷曰:「臣祗承丕緒,四十七年餘矣,於國計民生,夙夜兢業,無事不可質諸心者未有不亡。臣以是為鑒,?心者未有不興,失?天地。稽古史冊,興亡雖非一轍,而得深懼祖宗垂貽之大業自臣而隳,故身雖不德,而親握朝綱,一切政務,不徇偏私,不謀?小,事無久稽,悉由獨斷,亦惟鞠躬盡瘁,死而後已,在位一日,勤求治理,斷不敢少懈。不知臣有何辜,生子如允礽者,不孝不義,暴虐慆淫,若非鬼物憑附,狂易成疾,有血氣者豈忍為之?允礽口不道忠信之言,身不履德義之行,咎戾多端,難以承祀,用是昭告昊天上帝,特行廢斥,勿致貽憂邦國,痛毒蒼生。抑臣更有哀籥者,臣自幼而孤,未得親承父母之訓子,遠不及臣,如大清歷數綿長,延臣壽命?,惟此心此念,對越上帝,不敢少懈。臣雖有,臣當益加勤勉,謹保終始;如我國家無福,即殃及臣躬,以全臣令名。臣不勝痛切,謹告。」

太子既廢,上諭:「諸皇子中如有謀為皇太子者,即國之賊,法所不宥。」諸皇子中皇八子允禩謀最力,上知之,命執付議政大臣議罪,削貝勒。十月,皇三子允祉發喇嘛巴漢格發允礽所居室,得厭勝物十餘事。上幸南苑行圍,遘疾?隆為皇長子允禔魘允礽事,上令侍,還宮,召允礽入見,使居咸安宮。上逾諸近臣曰:「朕召見允礽,詢問前事,竟有全不知者,是其諸惡皆被魘魅而然。果蒙天佑,狂疾頓除,改而為善,朕自有裁奪。」廷臣希旨有請復立允礽為太子者,上不許。左副都御史勞之辨奏上,上斥其奸詭,奪官,予杖。

既,上召諸大臣,命於諸皇子中舉孰可繼立為太子者,諸大臣舉允禩。明日,上召諸大臣入見,諭以太子因魘魅失本性狀。諸大臣奏:「上既灼知太子病源,治療就痊,請上頒旨宣示。」又明日,召允礽及諸大臣同入見,命釋之,且曰:「覽古史冊,太子既廢,常不得其死,人君靡不悔者。前執允礽,朕日日不釋於懷。自頃召見一次,胸中乃疏快一次。今事已明白,明日為始,朕當霍然矣。」又明日,諸大臣奏請復立允礽為太子,疏留中未下。上疾漸愈,四十八年正月,諸大臣復疏請,上許之。

三月辛巳,復立允礽為皇太子,妃復為皇太子妃。五十年十月,上察諸大臣為太子結黨會飲,譴責步軍統領託合齊,尚書耿額、齊世武,都統鄂繕、迓圖。託合齊兼坐受戶部缺主沈天生賄罪,絞;又以鎮國公景熙首告貪婪不法諸事,未決,死於獄,命剉尸焚之。齊世武、耿額亦以得沈天生賄,絞死。鄂繕奪官,幽禁。迓圖入辛者庫,守安親王墓。上諭謂:「諸事皆因允礽。允礽不仁不孝,徒以言語貨財囑此輩貪得諂媚之人,潛通消息,尤無恥之甚。」

五十一年十月,復廢太子,禁錮咸安宮。五十二年,趙申喬疏請立太子,上諭曰:「建儲大事,未可輕言。允礽為太子時,服御俱用黃色,儀注上幾於朕,實開驕縱之門。宋仁宗三十年未立太子,我太祖、太宗亦未豫立。漢、唐已事,太子幼沖,尚保無事;若太子年長,左右?小結黨營私,鮮有能無過者。太子為國本,朕豈不知?立非其人,關系匪輕。允礽儀表、學問、才技俱有可觀,而行事乖謬,不仁不孝,非狂易而何?凡人幼時猶可教訓,及長而誘於黨類,便各有所為,不復能拘制矣。立皇太子事,未可輕定。」自是上意不欲更立太子,雖諭大學士、九卿等裁定太子儀仗,卒未用。終清世不復立太子。

五十四年十一月,有醫賀孟頫者,為允礽福金治疾,允礽以礬水作書相往來,復囑普奇舉為大將軍,事發,普奇等皆得罪。五十六年,大學士王掞疏請建儲,越數日,御史陳嘉猷等八人疏繼上,上疑其結黨,疏留中不下。五十七年二月,翰林院檢討硃天保請復立允礽為太子,上親召詰責,辭連其父侍郎硃都納,及都統銜齊世,副都統戴保、常賚,內閣學士金寶。硃天保、戴保誅死,硃都納及常賚、金寶交步軍統領枷示,齊世交宗人府幽禁。七月,允礽福金石氏卒。上稱其淑孝寬和,作配允礽,辛勤歷有年所,諭大學士等同翰林院撰文致祭。六十年三月,上萬壽節,掞復申前請建儲。越數日,御史陶彞等十二人疏繼上。上乃嚴旨斥掞為奸,並以諸大臣請逮掞等治罪,上令掞及彞等發軍前委署額外章京。掞年老,其子奕清代行。

六十一年,世宗即位,封允礽子弘?為理郡王。雍正元年,詔於祁縣鄭家莊修蓋房屋,駐劄兵丁,將移允礽往居之。二年十二月,允礽病薨,追封謚。六年,弘?進封親王。乾隆四年十月,高宗諭責弘?自視為東宮嫡子,居心叵測,削爵。以允礽第十子弘勩襲郡王。四十五年,薨,謚曰恪。子永曖,襲貝勒。子孫循例遞降,以輔國公世襲。允礽第三子弘晉、第六子弘曣、第七子弘晁、第十二子弘晥皆封輔國公。弘曣卒,謚恪僖。子永瑋,襲。事高宗,歷官左宗正,廣州、黑龍江、盛京將軍。卒,謚恪勤。永曖四世孫福錕,事德宗,官至體仁閣大學士。卒,謚文慎。

誠隱郡王允祉,聖祖第三子。康熙二十九年七月,偕皇太子詣古魯富爾堅嘉渾噶山行宮,上命先還。三十二年,闕里孔廟成,命偕皇四子往祭。凡行圍、謁陵,皆從。三十五年,上親征,允祉領鑲紅旗大營。三十七年三月,封誠郡王。三十八年,敏妃之喪未百日,允祉薙發,坐降貝勒,王府長史以下譴黜有差。四十三年,命勘三門底柱。四十六年三月,迎上幸其邸園,侍宴。嗣是,歲以為常,或一歲再幸。

四十七年,太子既廢,上以允祉與太子索親睦,召問太子情狀,且曰:「允祉與允礽雖暱,然未慫恿其為惡,故不罪也。」蒙古喇嘛巴漢格隆為允禔厭勝廢太子,允祉偵得之,發其事。明年,太子復立,允祉進封誠親王。五十一年,賜銀五千。

聖祖邃律歷之學,命允祉率庶吉士何國宗等輯律呂、算法諸書,諭曰:「古歷規模甚好,但其數目歲久不合。今修歷書,規模宜存古,數目宜準今。」五十三年十一月,書成,奏進。上命以律呂、歷法、算法三者合為一書,名曰律歷淵源。

五十八年,上有事於圜丘,拜畢,命允祉行禮。五十九年,封子弘晟為世子,班俸視貝子。六十年,上命弘晟偕皇四子、皇十二子祭盛京三陵。世宗即位,命允祉守護景陵。雍正二年,弘晟得罪,削世子,為閒散宗室。

六年六月,允祉索蘇克濟賕,事發,在上前詰王大臣,上責其無臣禮,議奪爵,錮私第。上曰:「朕止此一兄。朕兄弟如允祉者何限?皆欲激朕治其罪,其心誠不可喻。良亦朕不能感化所致,未可謂盡若輩之罪也。」命降郡王,而歸其罪於弘晟,交宗人府禁錮。八年二月,復進封親王。五月,怡親王之喪,允祉後至,無戚容。莊親王允祿等劾,下宗人府議,奏稱:「允祉乖張不孝,暱近陳夢雷、周昌言,祈禳鎮魘,與阿其那、塞思黑、允交相黨附。其子弘晟兇頑狂縱,助父為惡,僅予禁錮,而允祉銜恨怨懟。怡親王忠孝性成,允祉心懷嫉忌,並不懇請持服,王府齊集,遲至早散,背理蔑倫,當削爵。」與其子弘晟皆論死。上命奪爵,禁景山永安亭,聽家屬與偕,弘晟仍禁宗人府。十年閏五月,薨,視郡王例殯葬。乾隆二年,追謚。

子弘暻,封貝子。子孫遞降,以不入八分輔國公世襲。五世孫載齡,襲爵。事德宗,官至體仁閣大學士。卒,謚文恪。

恆溫親王允祺,聖祖第五子。康熙三十五年,上征噶爾丹,命允祺領正黃旗大營。四十八年十月,封恆親王。五十一年,賜銀五千。五十八年,封子弘升為世子,班祿視貝子。雍正五年,坐事,削世子。十年閏五月,允祺薨,予謚。子弘晊,襲。乾隆四十年,薨,謚曰恪。子永皓,襲郡王。五十三年,薨,謚曰敬。弘升子永澤,襲貝子。子孫循例遞降,以鎮國公世襲。弘升既削世子,乾隆十九年卒,予貝勒品級,謚恭恪。

淳度親王允祐,聖祖第七子。康熙三十五年,上征噶爾丹,命允祐領鑲黃旗大營。三十七年三月,封貝勒。四十八年十月,封淳郡王。五十一年,賜銀五千。五十七年十月,正藍旗滿洲都統延信征西陲,命允祐管正藍三旗事務。雍正元年,進封親王,詔褒其安分守己,敬順小心。復命與誠親王允祉並書景陵碑額,以兩王皆工書故。八年四月,薨,予謚。

子弘曙。聖祖命皇十四子允為撫遠大將軍,駐甘州,令弘曙從。聖祖崩,世宗召還京,封世子。雍正五年,坐事削,改封弘暻為世子。允祐薨,弘暻襲。乾隆四十二年,薨,謚曰慎。子永鋆,襲貝勒。子孫遞降,以鎮國公世襲。永鋆子綿洵,事穆宗,官涼州副都統。轉戰河南、直隸、山東、湖北,克臨清,破連鎮、馮官屯,皆有功。遷荊州將軍。卒,謚莊武。

允禩,聖祖第八子。康熙三十七年三月,封貝勒。四十七年九月,署內務府總管事。

太子允礽既廢,允禩謀代立。諸皇子允禟、允示我、允,諸大臣阿靈阿、鄂倫岱、揆敘、王鴻緒等,皆附允禩。允禔言於上,謂相士張明德言允禩後必大貴,上大怒,會內務府總管凌普以附太子得罪,籍其家,允禩頗庇之,上以責允禩。諭曰:「凌普貪婪巨富,所籍未盡,允禩每妄博虛名,凡朕所施恩澤,俱歸功於己,是又一太子矣!如有人譽允禩,必殺無赦。」翌日,召諸皇子入,諭曰:「當廢允礽時,朕即諭諸皇子有鉆營為皇太子者,即國之賊,法所不容。允禩柔奸性成,妄蓄大志,黨羽相結,謀害允礽。今其事皆敗露,即鎖系,交議政處審理。」允禟語允,入為允禩營救,上怒,出佩刀將誅允;允祺跪抱勸止,上怒少解,仍諭諸皇子、議政大臣等毋寬允禩罪。

逮相士張明德會鞫,詞連順承郡王布穆巴,公賴士、普奇,順承郡王長史阿祿。張明德坐凌遲處死,普奇奪公爵,允禩亦奪貝勒,為閒散宗室。上復諭諸皇子曰:「允禩庇其乳母夫雅齊布,雅齊布之叔廄長吳達理與御史雍泰同榷關稅,不相能,訴之允禩,允禩借事痛責雍泰。朕聞之,以雅齊布發翁牛特公主處。允禩因怨朕,與褚英孫蘇努相結,敗壞國事。允禩又受制於妻,妻為安郡王岳樂甥,嫉妒行惡,是以允禩尚未生子。此皆爾曹所知,爾曹當遵朕旨,方是為臣子之理;若不如此存心,日後朕考終,必至將朕躬置乾清宮內,束甲相爭耳。」上幸南苑,遘疾,還宮,召允禩入見,並召太子使居咸安宮。

未幾,上命諸大臣於諸皇子中舉可為太子者,阿靈阿等私示意諸大臣舉允禩。上曰:「允禩未更事,且罹罪,其母亦微賤,宜別舉。」上釋允礽,亦復允禩貝勒。四十八年正月,上召諸大臣,問倡舉允禩為太子者,諸臣不敢質言。上以大學士馬齊先言眾欲舉允禩,因譴馬齊,不復深詰。尋復立允礽為太子。五十一年十一月,復廢允礽。

六十一年十一月,上疾大漸,召允禩及諸皇子允祉、允祐、允禟、允示我、允祹、允祥同受末命。世宗即位,命允禩總理事務,進封廉親王,授理籓院尚書。雍正元年,命辦理工部事務。皇太子允礽之廢也,允禩謀繼立,世宗深憾之。允禩亦知世宗憾之深也,居常怏怏。封親王命下,其福晉烏雅氏對賀者曰:「何賀為?慮不免首領耳!」語聞,世宗憾滋甚。會副都統祁爾薩條奏:「滿洲俗遇喪,親友饋粥吊慰。後風俗漸弛,大設奢饌,過事奢靡。」上用其議申禁,因諭斥:「允禩居母妃喪,沽孝名,百日後猶扶掖匍匐而行;而允示我,允禟、允指稱饋食,大肆筵席,皇考諭責者屢矣。」二年,上諭曰:「允禩素行陰狡,皇考所深知,降旨不可悉數。自朕即位,優封親王,任以總理事務。乃不能輸其誠悃以輔朕躬,懷挾私心,至今未已。凡事欲激朕怒以治其罪,加朕以不令之名。允禩在諸弟中頗有治事材,朕甚愛惜之,非允禟、允示我等可比,是以屢加教誨,令其改過,不但成朕友于之誼,亦全前三復教誨之理?朕一身上關宗廟社稷,不得不為?皇考慈愛之衷。朕果欲治其罪,豈有於防範。允禩在皇考時,恣意妄行,匪伊朝夕,朕可不念祖宗肇造鴻圖,以永貽子孫之安乎?」

三年二月,三年服滿。以允禩任總理事務,挾私懷詐,有罪無功,不予議敘。尋因工部制祈榖壇祖宗神牌草率,阿爾泰駐兵軍器粗窳,屢下詔詰責允禩;允禩議減內務府披甲,上令覆奏,又請一佐領增甲九十餘副。上以允禩前後異議,諭謂:「陰邪叵測,莫此為甚!」因命一佐領留甲五十副不即裁,待缺出不補。隸內務府披甲諸人集允禩邸囂閧,翌日,又集副都統李延禧家,且縱掠。上命捕治,諸人自列允禩使閧延禧家,允禩不置辯。上命允禩鞫定為首者立斬,允禩以五人姓名上,上察其一乃自首,其一堅稱病未往,責允禩所讞不實。宗人府議奪允禩爵,上命寬之。允禩杖殺護軍九十六,命太監閻倫隱其事,厚賜之。宗人府復議奪允禩爵,上復寬之。

四年正月,上御西暖閣,召諸王大臣暴允禩罪狀,略曰:「當時允禩希冀非望,欲沽忠孝之名,而事事傷聖祖之心。二阿哥坐廢,聖祖命朕與允禩在京辦事,凡有啟奏,皆蒙御批,由允禩藏貯。嗣問允禩,則曰:『前值皇考怒,恐不測,故焚毀筆札,御批亦納其中。』此允禩親向朕言者。聖祖升遐,朕念允禩夙有才幹,冀其痛改其非,為國家出力,令其總理事務,加封親王,推心置腹。三年以來,宗人府及諸大臣劾議,什伯累積,朕百端容忍,乃允禩詭譎陰邪,狂妄悖亂,包藏禍心,日益加甚。朕令宗人府訊問何得將皇考御批焚毀,允禩改言:『抱病昏昧,誤行燒毀。』及朕面質之,公然設誓,詛及一家。允禩自絕於天,自絕於祖宗,自絕於朕,斷不可留於宗姓之內,為我朝之玷!謹述皇考諭,遵先朝削籍離宗之典,革去允禩黃帶子,以儆兇邪,為萬世子孫鑒戒。」並命逐其福晉還外家。

二月,授允禩為民王,不留所屬佐領人員,凡朝會,視民公、侯、伯例,稱親王允禩。諸王大臣請誅允禩,上不許。尋命削王爵,交宗人府圈禁高墻。宗人府請更名編入佐領:允禩改名阿其那,子弘旺改菩薩保。六月,諸王大臣復臚允禩罪狀四十事,請與允禟、允並正典刑,上暴其罪於中外。九月,允禩患嘔噦,命給與調養,未幾卒於幽所。諸王大臣仍請戮尸,不許。

乾隆四十三年正月,高宗諭曰:「聖祖第八子允禩,第九子允禟結黨妄行,罪皆自取。皇考僅令削籍更名,以示愧辱。就兩人心術而論,覬覦窺竊,誠所不免,及皇考紹登大寶。皇考晚年屢向朕諭及,愀然不樂,意頗悔?,怨尤誹謗,亦情事所有,特未有顯然悖逆之之,若將有待。朕今臨御四十三年矣,此事重大,朕若不言,後世子孫無敢言者。允禩、允禟仍復原名,收入玉牒,子孫一並?入。此實仰體皇考仁心,申未竟之緒,想在天之靈亦當愉慰也。」

允禟,聖祖第九子。康熙四十七年,上責允禩,允禟語允,入為保奏,上怒。是時,上每巡幸,輒隨。四十八年三月,封貝子。十月,命往翁牛特送和碩慍恪公主之喪。五十一年,賜銀四千。

雍正元年,世宗召允回京,以諸王大臣議,命允禟出駐西寧。允禟屢請緩行,上譴責所屬太監,允禟行至軍。二年四月,宗人府劾允禟擅遣人至河州買草、勘牧地,違法肆行,請奪爵,上命寬之。三年,上聞允禟縱容家下人在西寧生事,遣都統楚宗往約束,楚宗至,允禟不出迎,傳旨詰責,曰:「上責我皆是,我復何言?我行將出家離世!」楚宗以聞,上以允禟傲慢無人臣禮,手詔深責之,並牽連及允禩、允、允示我私結黨援諸事。七月,山烏雅圖等經平定毆諸生,請按律治罪,陜西人稱允禟九王,為上?西巡撫伊都立奏劾允禟護所聞,手詔斥為無恥,遂奪允禟爵,撤所屬佐領,即西寧幽禁,並錄允禟左右用事者毛太、佟保等,撤還京師,授以官。

四年正月,九門捕役得毛太、佟保等寄允禟私書,以聞,上見書跡類西洋字,遣持問允禟子弘暘,弘暘言允禟所造字也。諭曰:「從來造作隱語,防人察覺,惟敵國為然。允禟在西寧,未嘗禁其書札往來,何至別造字體,暗藏密遞,不可令人以共見耶?允禟與弘暘書用硃筆,弘暘復書稱其父言為『旨』,皆僭妄非禮。允禟寄允示我書言『事機已失』,其言尤駭人。」命嚴鞫毛太、佟保等。諸王大臣請治允禟罪,命革去黃帶子,削宗籍,逮還京,令胡什禮監以行。五月,令允禟改名,又以所擬字樣奸巧,下諸王大臣議,改為塞?楚宗及侍思黑。

六月,諸王大臣復劾允禟罪狀二十八事,請誅之。胡什禮監允禟至保定,命直隸總督李紱暫禁,觀其行止。紱語胡什禮「當便宜行事」,胡什禮以聞,上命馳諭止之,紱奏無此語。八月,紱奏允禟以腹疾卒於幽所。上聞胡什禮與楚宗中途械系允禟,旋釋去,胡什禮又,乃起流言也。乾?妄述紱語,命並逮治。其後紱得罪,上猶責紱不以允禟死狀明白於隆間,復原名,還宗籍。子弘晸,封不入八分輔國公,坐事奪爵。

輔國公允示我,聖祖第十子。康熙四十八年十月,封敦郡王。五十七年,命辦理正黃旗滿洲、蒙古、漢軍三旗事。允示我與允禟、允皆黨附允禩,為世宗所惡。雍正元年,澤卜尊丹巴胡土克圖詣京師,謁聖祖梓宮,俄病卒,上遣送靈龕還喀爾喀,命允示我齎印冊賜奠。允示我託疾不行,旋稱有旨召還,居張家口。復私行禳禱,疏文內連書「雍正新君」,為上所知,斥為不敬。兵部劾奏,命允禩議其罪。四月,奪爵,逮京師拘禁。乾隆二年,高宗命釋之,封輔國公。六年,卒,詔用貝子品級祭葬。

履懿親王允祹,聖祖第十二子。康熙四十八年十月,封貝子。自是有巡幸,輒從。五十六年,孝惠章皇后崩,署內務府總管事務,大事將畢,乃罷。五十七年,辦理正白旗滿洲、蒙古、漢軍三旗事。六十年,上以御極六十年,遣允祹祭盛京三陵。六十一年,授鑲黃旗滿洲都統。世宗即位,進封履郡王。雍正二年,宗人府劾允祹治事不能敬謹,請奪爵,命在固山貝子上行走。二月,因聖祖配享儀注及封妃金冊遺漏舛錯,降鎮國公。八年五月,復封郡王。高宗即位,進封履親王。乾隆二十八年七月,薨,予謚。

子弘昆,先卒,用世子例殯葬,餘子皆未封。高宗命以皇四子永為允祹後,襲郡王。四十二年,薨,謚曰端。嘉慶四年,追封親王。子綿惠,襲貝勒。嘉慶元年,薨,追封郡王。以成郡王綿懃子奕綸為後,襲貝子,進貝勒。子孫循例遞降,以鎮國公世襲。

乾隆四十二年,高宗南巡,還蹕次涿州,有僧攜童子迎駕,自言永庶子,為側室福晉王氏所棄,僧育以長。上問永嫡福晉伊爾根覺羅氏,言永子以痘殤。乃令入都,命軍機大臣詰之。童子端坐名諸大臣,諸大臣不敢決。軍機章京保成直前批其頰,叱之,童子乃自承劉氏子,僧教為妄語。斬僧,戍童子伊犁,仍自稱皇孫,所為多不法。上命改戍黑龍江,道庫倫,庫倫辦事大臣松筠責其不法,縛出,絞殺之,高宗嘉其明決。

怡賢親王允祥,聖祖第十三子。康熙三十七年,從上謁陵。自是有巡幸,輒從。六十一年,世宗即位,封為怡親王。尋命總理戶部三庫。雍正元年,命總理戶部。十一月,諭:「怡親王於皇考時敬謹廉潔,家計空乏,舉國皆知。朕御極以來,一心翊戴,克盡臣弟之道。從前兄弟分封,各得錢糧二十三萬兩,朕援此例賜之,奏辭不已,宣諭再四,僅受十三萬;復援裕親王例,令支官物六年,王又固辭。今不允所請,既不可;允其請,而實心為國之懿親,轉不得與諸弟兄比,朕心不安。」下諸王大臣議。既,仍允王請,命王所兼管佐領俱為王一等一員、二等四員、三等十二員,豹尾槍二、長桿刀二,每佐領增親軍二名?屬,加護。二年,允祥請除加色、加平諸弊,並增設三庫主事、庫大使,從之。

三年二月,三年服滿。以王總理事務謹慎忠誠,從優議敘,復加封郡王,任王於諸子?中指封。八月,加俸銀萬。京畿被水,命往勘。十二月,令總理京畿水利。疏言:「直隸河與汶河合流東下。滄、景以下,春?河、澱河、子牙河、永定河皆匯於天津大直沽入海,河?多淺阻,伏秋暴漲,不免潰溢。請將滄州磚河、青縣興濟河故道疏濬,築減水壩,以洩之漲;並於白塘口入海處開直河,使磚河、興濟河同歸白塘出海;又濬東、西二澱,多開引河,使脈絡相通,溝澮四達;仍疏趙北、苑家二口以防沖決。子牙河為滹沱及漳水下流,其下有清河、夾河、月河同趨於澱,宜開決分注,緩其奔放之勢。永定河故道已湮,應自柳義水所歸,應逐年疏濬,使濁水不能為患。?口引之稍北,繞王慶坨東北入澱,至三角澱,為又請於京東灤、薊、天津,京南文、霸、任丘、新、雄諸州縣設營田專官,募農耕種。」四年二月,疏言直隸興修水利,請分諸河為四局,下吏、工諸部議,議以南運河與臧家橋以下之子牙河、苑家口以東之澱河為一局,令天津道領之;苑家口以西各澱池及畿南諸河為一局,以大名道改清河道領之;永定河為一局,以永定分司改道領之;北運河為一局,撤分司以通永道領之:分隸專官管轄。尋又命分設京東、京西水利營田使各一。三月,疏陳京東水利諸事。五月,疏陳畿輔西南水利諸事。皆下部議行。

七月,賜御書「忠敬誠直勤慎廉明」榜,諭曰:「怡親王事朕,克殫忠誠,職掌有九,而公爾忘私,視國如家,朕深知王德,覺此八字無一毫過量之詞。在朝諸臣,於『忠勤慎明』尚多有之,若『敬誠直廉』,則未能輕許。期咸砥礪,以副朕望。」七年六月,命辦理西北兩路軍機。十月,命增儀仗一倍。十一月,王有疾。八年五月,疾篤,上親臨視,及至,王已薨,上悲慟,輟朝三日。翌日,上親臨奠,諭:「怡親王薨逝,中心悲慟,飲食無味,寢臥不安。王事朕八年如一日,自古無此公忠體國之賢王,朕待王亦宜在常例之外。今朕素服一月,諸臣常服,宴會俱不必行。」越日,復諭舉怡親王功德,命復其名上一字為「胤」,配享太廟,謚曰賢,並以「忠敬誠直勤慎廉明」八字加於謚上。白家甿等十三村民請建祠,允之。撥官地三十餘頃為祭田,免租賦。命更定園寢之制,視常例有加。又命未殯,月賜祭;小祥及殯,視大祭禮賜祭;三年後,歲賜祭。皆特恩,不為例。乾隆中,祀盛京賢王祠。命王爵世襲。

子弘曉,襲。乾隆四十三年,薨,謚曰僖。子永瑯,襲。嘉慶四年,薨,謚曰恭。孫奕勛,襲。二十三年,薨,謚曰恪。子載坊,襲。明年,薨。弟載垣,襲。事宣宗,命在御大臣。咸豐八年,賜紫禁城內肩?前大臣行走,受顧命。文宗即位,歷左宗正、宗令、領侍輿。

載垣與鄭王端華及端華弟肅順皆為上所倚,相結,權勢日張。九年,命赴天津察視海防。十年正月,萬壽節,賜杏黃色端罩。七月,英吉利、法蘭西兩國兵至天津,命與兵部尚書穆廕以欽差大臣赴通州與英人議和。時大學士桂良已於天津定議,上許英使額爾金至通州簽約,英使額爾金請入京師親遞國書,不許。兵復進,上以和議未成,罷載垣欽差大臣。未幾,扈上幸熱河。及和議定,?臣請還京師,上猶豫未決。十一年七月,文宗崩,穆宗即位,載垣等受遺詔輔政,與端華、景壽、肅順及軍機大臣穆廕、匡源、杜翰、焦祐瀛稱「贊襄政務王大臣」,擅政。九月,上奉文宗喪還京師,詔罪狀載垣等,奪爵職,下王大臣按治,議殊死,賜自盡。事詳肅順傳。爵降為不入八分輔國公,並命不得以其子孫及親兄弟子承襲。同治元年,以莊親王允祿四世孫載泰襲輔國公,收府第敕書。三年七月,師克江寧,推恩還王爵。九月,以寧郡王弘?四世孫鎮國公載敦襲怡親王,還敕書。光緒十六年,薨,謚,奪爵,以先薨免罪。弟之?曰端。子溥靜,嗣。二十六年八月,薨。九月,坐縱芘拳匪啟子毓麒,襲。

寧良郡王弘?,允祥第四子。世宗褒允祥功,加封郡王,任王於諸子中指封,允祥固辭不敢承。及允祥薨,世宗乃封弘?寧郡王,世襲。乾隆二十九年八月,薨,謚曰良。子永福,仍循例襲貝勒。四十七年九月,薨,謚恭恪。子綿譽,仍襲貝勒。子孫遞降,以鎮國公世襲。載敦紹封怡親王,即以載泰襲鎮國公。

允祥諸子:弘昌,初封貝子,進貝勒,坐事奪爵;弘暾,未封早世,聘於富察氏,未婚守志,世宗愍之,命視貝勒例殯葬;弘昑,亦用其例。

恂勤郡王允,聖祖第十四子。康熙四十八年,封貝子。五十年,從上幸塞外。自是輒從。五十一年,賜銀四千兩。五十七年,命為撫遠大將軍,討策妄阿喇布坦。十二月,師行,上御太和殿授印,命用正黃旗纛。五十八年四月,劾吏部侍郎色爾圖督兵餉失職,都統胡錫圖索詐騷擾,治其罪。都統延信疏稱:「準噶爾與青海聯姻婭,大將軍領兵出口,必有諜告準酋者,不若暫緩前進。」上命駐西寧。五十九年正月,允移軍穆魯斯烏蘇,遣平逆將軍延信率師入西藏,令宗查布防西寧,訥爾素防古木。時別立新胡必爾汗,遣兵送之入藏。十月,延信擊敗準噶爾將策零敦多卜等於卜克河諸地。六十年五月,允率師駐甘州,進萬?次吐魯番。旋請於明年進兵。閏六月,和爾博斯厄穆齊寨桑以厄魯特兵五百圍回民,回餘人乞援。允以糧運艱阻,兵難久駐,若徙入內地,亦苦糧少地狹,哈密扎薩克額敏皆不能容,布隆吉爾、達里圖諸地又阻瀚海,請諭靖逆將軍富寧安相機援撫,從之。十月,召來京,面授方略。六十一年三月,還軍。

世宗即位,諭總理王大臣曰:「西路軍務,大將軍職任重大,但於皇考大事若不來京,恐於心不安,速行文大將軍王馳驛來京。」允至,命留景陵待大祭。雍正元年五月,諭曰:「允無知狂悖,氣傲心高,朕望其改悔,以便加恩。今又恐其不能改,不及恩施,特進為郡王,慰我皇妣皇太后之心。」三年三月,宗人府劾允前為大將軍,苦累兵丁,侵擾地方,糜費軍帑,請降授鎮國公,上命仍降貝子。四年,諸王大臣劾,請正國法。諭:「允止於糊塗狂妄,其奸詐陰險與允禩、允禟相去甚遠。朕於諸人行事,知之甚悉,非獨於允有所偏徇。今允居馬蘭峪,欲其瞻仰景陵,痛滌前非。允不能悔悟,奸民蔡懷璽又造為聽,宜加禁錮,即與其子白起並錮於壽皇殿左右,寬以歲月,待其改悔。?大逆之言,搖惑」高宗即位,命釋之。乾隆二年,封輔國公。十二年六月,進貝勒。十三年正月,進封恂郡王。二十年六月,薨,予謚。

第一子弘春,雍正元年,封貝子。二年,坐允禩黨,革爵。四年,封鎮國公。六年,進貝子。九年,進貝勒。十一年,封泰郡王。十二年八月,諭責弘春輕佻,復降貝子。高宗即位,奪爵。別封允第二子弘明為貝勒。乾隆三十二年,卒,謚恭勤。子孫循例遞降,以不入八分鎮國公世襲。弘春曾孫奕山,自有傳。

愉恪郡王允潖,聖祖第十五子。康熙三十九年,從幸塞外。自是輒從。雍正四年,封貝勒。命守景陵。八年,封愉郡王。九年二月,薨,予謚。子弘慶,襲。乾隆三十四年,薨,謚曰恭。子永珔,襲貝勒。子孫循例遞降,以輔國公世襲。

果毅親王允禮,聖祖第十七子。康熙四十四年,從幸塞外。自是輒從。雍正元年,封亦如?果郡王,管理籓院事。三年,諭曰:「果郡王實心為國,操守清廉,宜給親王俸,護之,班在順承郡王上。」六年,進親王。七年,命管工部事。八年,命總理戶部三庫。十一年,授宗令,管戶部。十二年,命赴泰寧,送達賴喇嘛還西藏,循途巡閱諸省駐防及綠營兵。十三年,還京師,命辦理苗疆事務。世宗疾大漸,受遺詔輔政。

高宗即位,命總理事務,解宗令,管刑部。尋賜親王雙俸,免宴見叩拜。密疏請蠲江南諸省民欠漕項、蘆課、學租、雜稅,允之。諭曰:「果親王秉性忠直,皇考所信任。外間頗疑其嚴厲,令觀密奏,足見其存心寬厚,特以宣示九卿。」允禮體弱,上命在邸治事,越數日一入直。乾隆元年,坐事,罷雙俸。三年正月,病篤,遣和親王弘晝往視。二月,薨,上震悼,即日親臨其喪。予謚。無子,莊親王允祿等請以世宗第六子弘適為之後。

弘適善詩詞,雅好藏書,與怡府明善堂埒。御下嚴,晨起披衣巡視,遇不法者立杖之,故無敢為非者。節儉善居積,嘗以開煤?奪民產。從上南巡,囑兩淮鹽政高恆鬻人葠牟利,又令織造關差致繡段、玩器,予賤值。二十八年,圓明園九州清宴災,弘適後至,與諸皇子談笑露齒,上不懌。又嘗以門下私人囑阿里袞。上發其罪,並責其奉母妃儉薄,降貝勒,罷一切差使。自是家居閉門,意抑鬱不自聊。三十年三月,病篤,上往撫視。弘適於臥榻間叩首引咎,上執其手,痛曰:「以汝年少,故稍加拂拭,何愧恧若此?」因復封郡王。旋薨,予謚。

子永,襲。五十四年,薨,謚曰簡。子綿從,襲貝勒。孫奕湘,襲鎮國公。歷官副都統,廣州、盛京將軍,兵部尚書。加貝子銜。卒,謚恪慎。子孫遞降,以輔國公世襲。

簡靖貝勒允禕,聖祖第二十子。康熙五十五年,始從幸塞外,自是輒從。雍正四年,封貝子。八年二月,進貝勒。十二年八月,命祭陵。稱病不行,降輔國公。十三年九月,高宗即位,復封貝勒,守護泰陵。乾隆二十年,卒,予謚。子弘閏,襲貝子。子孫循例遞降,以不入八分鎮國公世襲。

慎靖郡王允禧,聖祖第二十一子。康熙五十九年,始從幸塞外。雍正八年二月,封貝子。五月,諭以允禧立志向上,進貝勒。十三年十一月,高宗即位,進慎郡王。允禧詩清秀,尤工畫,遠希董源,近接文徵明,自署紫瓊道人。乾隆二十三年五月,薨,予謚。

二十四年十二月,以皇六子永瑢為之後,封貝勒。三十七年,進封質郡王。五十四年,再進親王。永瑢亦工畫,濟美紫瓊,兼通天算。五十五年,薨,謚曰莊。子綿慶,襲郡王。綿慶幼聰穎,年十三,侍高宗避暑山莊校射,中三矢,賜黃馬褂、三眼孔雀翎。通音律。體孱弱。嘉慶九年,薨,年僅二十六。仁宗深惜之,賜銀五千,謚曰恪。子奕綺,襲貝勒。道光五年,坐事,罰俸。十九年,奪爵。二十二年,卒,復其封。子孫循例遞降,以鎮國公世襲。

恭勤貝勒允祜,聖祖第二十二子。康熙五十九年,始從幸塞外。雍正八年二月,封貝子。十二年二月,進貝勒。乾隆八年,卒,予謚。子弘曨,襲貝子。卒。子永芝,襲鎮國公。坐事,奪爵,爵除。

郡王品級誠貝勒允祁,聖祖第二十三子。雍正八年二月,封鎮國公。十三年十月,高宗即位,進貝勒。屢坐事,降鎮國公。四十五年,復封貝子。四十七年,進貝勒。四十九年,加郡王銜。五十年,卒,予謚。子弘謙,襲貝子,嘉慶十四年,加貝勒品級。卒,子永康,襲鎮國公。卒,子綿英,襲不入八分鎮國公。卒,無子,爵除。

諴恪親王允祕,聖祖第二十四子。雍正十一年正月,諭曰:「朕幼弟允示必,秉心忠厚,賦性和平,素為皇考所鍾愛。數年以來,在宮中讀書,學識亦漸增長,朕心嘉悅,封為諴親王。」乾隆三十八年,薨,予謚。第一子弘暢,襲郡王。六十年,薨,謚曰密。子永珠,襲貝勒。道光中,坐事,奪爵。弘?,允祕第二子,字仲升。乾隆二十八年,封二等鎮國將軍。三十九年,進封貝子。屢坐事,奪爵。嘉慶間,授奉恩將軍。卒。弘?工畫,師董邦達,自署瑤華道人,名與紫瓊並。永珠既奪爵,以弘?孫綿勛襲貝子。子孫遞降,以鎮國公世襲。

世宗十子:孝敬憲皇后生端親王弘暉,孝聖憲皇后生高宗,純懿皇貴妃耿佳氏生和恭親王弘晝,敦肅皇貴妃年佳氏生福宜、懷親王福惠、福沛,謙妃劉氏生果恭郡王弘適,齊妃李氏生弘昀、弘時、弘昐。弘適出為果毅親王允禮後。弘昀、弘昐、福宜、福沛皆殤,無封。弘時雍正五年以放縱不謹,削宗籍,無封。

端親王弘暉,世宗第一子。八歲殤。高宗即位,追封親王,謚曰端。

和恭親王弘晝,世宗第五子。雍正十一年,封和親王。十三年,設辦理苗疆事務處,命高宗與弘晝領其事。乾隆間,預議政。弘晝少驕抗,上每優容之。嘗監試八旗子弟於正大光明殿,日晡,弘晝請上退食,上未許。弘晝遽曰:「上疑吾買囑士子耶?」明日,弘晝入謝,上曰:「使昨答一語,汝齏粉矣!」待之如初。性復奢侈,世宗雍邸舊貲,上悉以賜之,故富於他王。好言喪禮,言:「人無百年不死者,奚諱為?」嘗手訂喪儀,坐庭際,使家人祭奠哀泣,岸然飲啖以為樂。作明器象鼎彞盤盂,置幾榻側。三十年,薨,予謚。子永璧,襲。三十七年,薨,謚曰勤。子綿倫,襲郡王。三十九年,薨,謚曰謹。弟綿循,襲。嘉慶二十二年,薨,謚曰恪。子奕亨,襲貝勒。卒,子載容,襲貝子。同治中,加貝勒銜。卒,謚敏恪。子溥廉,襲鎮國公。

懷親王福惠,世宗第七子。八歲殤。高宗即位,追封親王,謚曰懷。


\end{pinyinscope}