\article{志八十五}

\begin{pinyinscope}
○選舉五

△封廕推選

封贈之制,文職隸吏部,八旗、綠營武職隸兵部。順治間,覃恩及三年考滿,均給封贈。康熙初,廢文、武職考滿封贈。

文職封贈之階,初正一品、特進、光祿大夫,尋改光祿大夫。從一品光祿大夫,後改榮祿大夫。正二品資政大夫。從二品通奉大夫。正三品通議大夫。從三品中議大夫。正四品中憲大夫。從四品朝議大夫。正五品奉政大夫。從五品奉直大夫。正六品承德郎。從六品儒林郎,吏員出身者宣德郎。正七品文林郎,吏員出身者宣議郎。從七品徵仕郎。正八品修職郎。從八品修職佐郎。正九品登仕郎。從九品登仕佐郎。

武職封贈之階,初分三系。一曰滿、漢公、侯、伯封光祿大夫,後改建威將軍。二曰八旗。一品光祿大夫。二品資政大夫。三品通議大夫。四品中憲大夫。五品奉政大夫。六品承德郎,後改武信郎。七品文林郎,後改奮武郎。八品修職郎。九品登仕郎。乾隆三十二年,改同綠旗。三曰綠旗營。封贈官階屢變。初制正、從一品榮祿大夫。正二品驃騎將軍。從二品驍騎將軍。正三品昭勇將軍。從三品懷遠將軍。正四品明威將軍。從四品宣武將軍。正五品武德將軍。從五品武略將軍。正六品昭信校尉。從六品忠顯校尉。後增正七品奮勇校尉。乾隆二十年,改正二品武顯大夫。從二品武功大夫。正三品武義大夫。從三品武翼大夫。正四品昭武大夫。從四品宣武大夫。正五品武德郎。從五品武略郎。正六品武信郎。從六品武信佐郎。正七品奮武郎。三十二年,改正一品建威大夫。從一品振威大夫。增從七品奮武佐郎。正八品修武郎。從八品修武佐郎。八旗與綠營制度始畫一。五十一年,改正一品建威將軍。從一品振威將軍。正二品武顯將軍。從二品武功將軍。正三品武義都尉。從三品武翼都尉。正四品昭武都尉。從四品宣武都尉。正五品武德騎尉。從五品武德佐騎尉。正六品武略騎尉。從六品武略佐騎尉。正七品武信騎尉。從七品武信佐騎尉。正八品奮武校尉。從八品奮武佐校尉。增正九品修武校尉。從九品修武佐校尉。於是文、武官階等級相侔矣。

文、武正、從一品妻封一品夫人。滿、漢公妻為公妻一品夫人。侯妻為侯妻一品夫人。伯妻為伯妻一品夫人。正、從二品夫人。正、從三品淑人。正、從四品恭人。正、從五品宜人。正、從六品安人。正從七品孺人。正、從八品八品孺人。正、從九品九品孺人。武職八旗八品以下、綠旗營七品以下妻無封。後改綠營正七品妻封孺人。

順治五年,定制,凡遇恩詔,一品封贈三代,誥命四軸。二、三品封贈二代,誥命三軸。四、五品封贈一代,誥命二軸。六、七品封贈一代,敕命二軸。八、九品止封本身,敕命一軸。凡軸端一品用玉,二品用犀,三、四品用裹金,五品以下用角。

凡推封之例,順治初制,父祖現任者,不得受子孫封。致仕及已故者許給,原棄職就封者聽。兩子均仕,其父母受封,從其品大者。婦人因子封贈,而夫與子兩有官,亦從其品大者。父官高於子者,嫡母從父官,生母從子官。為人後者,已封贈祖父母、父母,請以本身妻室封典貤封本生祖父母、父母者,許貤封。康熙五年,定父職高於子者,依父原品封贈。官卑於子者,從子官封贈。武職子現任文職,封贈依文官例。雍正三年,定四品至七品官原將本身妻室封典貤封祖父、母者,八、九品官原貤封父、母者,皆許貤封。三品以上貤封曾祖父、母者,請旨定奪。乾隆間,折中禮制,頗有更定。二十七年諭曰:「子孫官品不及祖、父之崇,則父為大夫子為士,記有明文。舊例依祖、父原階封贈,殊未允協,其議改之。」吏部議定文、武官子孫職大,祖、父職小,依子孫官階封贈。祖、父職大,子孫職小,不得依祖、父原品封贈。父官高於子者,生母從子官封贈,嫡母、繼母不得依父官請封,原依子官受封者聽。武職子任文職者亦如之。五十年,定一品至三品官不得貤封高祖父、母,四品至七品官不得貤封曾祖父、母,八品官以下不得貤封祖父、母。

道光以後,捐封例開。二十三年,許三品以上官欲捐請本生曾祖父、母封贈者,得依貤封曾祖父、母例報捐。二十八年,許四品至七品官捐請貤封曾祖父、母,八品官以下捐請貤封祖父、母,均依常例加倍報捐。而限制始廢矣。舊例八、九品官許封父、母,不封本身妻室。應封妻者,止封正妻一人。正妻未封已歿,繼室當封者,正妻亦得追贈。其再繼者不得給封。道光二十三年,許八品以下捐封人員欲捐請及妻室者,加倍報捐。咸豐二年,許京、外文職及捐職人員得先封本身及原配、繼配妻室,再依本身品級為第三繼妻捐封。四年,並從部議,第三繼妻以後,誼同敵體,亦許依次遞捐矣。舊例仕宦至三品,幼為外祖父、母撫養,其外祖父、母歿無嗣者,許依其官階貤贈,其餘外姻不許貤封。道光二十三年,許捐封人員為其受恩撫養之母舅、舅母、姑夫、姑母、姨夫、姨母、妻父、妻母依貤封外祖父、母例,捐請貤封。咸豐三年,並許貤封曾祖父、母,伯叔祖父、母,伯叔父母,庶母,兄、嫂並嫡堂伯叔祖父、母,嫡堂伯叔父、母,嫡堂兄、嫂,從堂、再從堂尊長及外曾祖父、母,外祖父、母,妻祖父、母。按例定品級,一體捐請。又許為人婦者,為其已故夫之祖若父捐職請封。為人後者,為祖若父貤封其先人,展轉推衍,而經制蕩然矣。

加級請封之制,其初限制亦嚴。順治初,凡恩詔加級者,以新加之級給封。康熙五十二年,定例七品以下加級請封,不得逾五品,五、六品不得逾四品,三、四品不得逾二品,捐級不得計算。乾隆間,外官加級不論新舊,不得依加級請封。五十年,部議京官八品以下,得依加級請五品封,不惟逾分,亦覺太優。嗣後八品以下不得逾七品,在外未入流不得給封,原捐納榮親者,許其捐封。從之。嘉慶後,限制漸寬。京、外官恭遇覃恩,許報捐新級請封。議敘三、四品職銜人員,加級捐請二品封典,許加倍納銀,按現任及候補、候選例給封。咸豐初,撫廣捐例,京、外各官及捐職人員,由加級及捐加之級捐封者,現任及候補、候選三、四品官,許捐至二品。其五、六品加等捐請三品封者,依常例加倍報捐。加等捐請至二品者,依四品職銜得捐二品封例,加倍半報捐。其七品加等捐請三、四品封,八品以下加等捐請五、六品封,均依常例,分別加倍報捐。十年,定三品人員加級捐封,按一品人員銀數加倍,許給從一品封。二、三品虛銜人員捐從一品封,應按二、三品實職銀數加成或加倍報捐。其有為外姻捐從一品封者,許各按二、三品實職虛銜銀數,再行分別加成報捐。

陵夷至光緒中,御史李慈銘疏曰:「治國之要,惟賞與罰。罰固不可稍逾,賞亦豈可或濫!康熙、乾隆兩朝,享國久長,慶典武功,僂指難盡。其時內外臣工,屢逢恩詔,論功行賞,班序秩然,未有越等者。今則外官道員多至二品,其封皆至一品矣。知府、同知多加三品,其封皆至二品矣。牧、令大半四品,簿、尉大半五、六品,其封率至三、四品矣。夫爵賞者,人君所以進退賢愚,中臣所以奔走吏士。得之太易,則人不知恩,予之太驟,則士無由勸。尊卑不別,等級不明,長偽士之浮囂,惑小民之觀聽,非所以尊朝廷、清流品也。」奏上,亦未殺減。

厥後外患頻仍,人才缺乏。二十六年,詔停報捐實官,而虛銜封典報捐如故。宣統元年,吏部議定條例,京官依加級、外官依本任請封,頗欲規復舊制,格不得行。明年,改定京官依加級,外官依加銜,五品人員許請至三品封贈,八品人員許請至六品封贈。欲稍事補救,而積重難返矣。

廕敘之制,曰恩廕,曰難廕,曰特廕。恩廕始順治十八年,恩詔滿、漢文官在京四品、在外三品以上,武官在京、在外二品以上,各送一子入監。護軍統領、副都統、阿思哈尼哈番、侍郎、學士以上之子為廕生,餘為監生。初制,公、侯、伯予一品廕,子、男分別授廕。雍正二年改世職俱依三品予廕。乾隆三十四年,定公、侯、伯依一品,子依二品,男依三品予廕。雍正初,定例廕生、廕監生通達文義者,交吏部分各部、院試驗行走。其十五歲以上送監讀書者,年滿學成,咨部奏聞,分部、院學習。又令文、武廕、監生通達文理者,遵例考試,以文職錄用。其幼習武藝,人材壯健,原改武職者,呈明吏部,移兵部改廕。

考試之法,雍正三年,令廕生到部年二十以上者,奏請考試引見。乾隆十一年,定考試以古論及時務策,欽派大臣閱卷,評定甲乙,進呈御覽。文理優通者,交部引見。荒謬者,發回原籍讀書,三年再試。歷代遵例行。光緒三十一年,免漢廕生考試如滿員例。

錄用之法,漢廕生有內用、外用、改武職用三途。內用者,雍正元年定制,尚書一品用員外郎,侍郎二品用主事,總督同尚書,巡撫同侍郎。七年,改定正一品用員外郎、治中,從一品用主事,正二品用主事、都察院經歷、京府通判,從二品用光祿寺署正、大理寺寺副,正三品用通政使司經歷、太常寺典簿、中、行、評、博,從三品用光祿寺典簿、鑾儀衛經歷、詹事府主簿、京府經歷,四品廕生與捐納貢監考職者一例,輪班選用。乾隆七年,定左都御史廕同尚書。同治十年,定河道總督廕用員外郎、主事。宣統間,改革官制,裁撤各官,以相當品級改用。外用者,乾隆間定制,正一品用府同知,從一品用知州,二品用通判,三品用知縣,漢世職子爵用知縣,終清世無變更。改武職用者,雍正間定制,在京一品尚書等官,在外總督、將軍,廕生用都司銜管都司。二品侍郎等官,巡撫、提督,用署都司銜管都司。三品副都御史等官,布政使、總兵官,用守備銜管守備。按察使、加一品銜副將,用署守備銜管守備。二品銜副將,用守御所千總。乾隆間定漢子爵三品用千總,男爵四品用把總。

漢軍錄用,康熙十二年原定一品用員外郎,二品用大理寺寺正、知州。雍正七年,用知州者以主事改補。乾隆五年,定三品用七品筆帖式,四品用八品筆帖式。宣統元年,吏部奏言:「漢文、武官廕生,按品級正、從授職,滿廕生不分正、從。漢廕生引見,以內用、外用擬旨,滿廕生以文職侍衛旗員擬旨。惟光緒三十二年以後,漢員一體簡授,旗職若現任都統、副都統,廕生依滿例給廕,不無窒礙。擬請原系尚書、侍郎改授升授者,都統依漢尚書例,副都統依漢侍郎例,三品以下京堂、監司升授之副都統,依漢正二品例,仍以內用、外用擬旨。」允之。

初制,非現任官不得廕,內務府佐領以下官不給廕。康熙六年,定各官不論級銜,均依實俸廕子,是年始許內務府佐領以下官子弟給廕。十二年,並許原品解任食俸者給廕。

先是康熙三年定廕、監生已得官及科目中式者,不得補廕。乾隆四十五年改定嫡長子孫有科名尚未選用,及有職銜原承廕者,許補廕。道光以後,捐例宏開,既得官職,仍許補廕。銓選混淆,幸進滋多。

光緒二十二年,御史熙麟奏言:「吏部銓選,以奉特旨人員統壓各班,然如廕生暨及歲引見之員,曾捐道府,引見奉諭仍以道府選用者,本系捐班,部章竟歸特旨班銓選。比年以來,率皆營私取巧,預捐道府,為他日例邀特旨統壓各班之地。致使同一廕生暨及歲人員,而廉吏兒孫,興嗟力薄,紈褲子弟,逞志夤緣,於世道人心,大有關系。請以此等人員加捐道、府者,與捐納人員同班銓選。」下部議行。

難廕,順治三年定制,官員歿於王事者,依應升品級贈銜,並廕一子入監讀書,期滿候銓。康熙十八年定殉難官依本銜廕子,不依贈銜。雍正十二年,奏定官員因公差委,在大洋、大江、黃河,洞庭、洪澤等湖,遭風漂歿者,依應升品級廕、贈,在內洋、內河漂歿者,減等廕、贈,八品以下,贈銜不給廕。乾隆六十年定官員隨營任事,催餉盡力,因病身故者,依內洋、內河漂歿例廕、贈。道光二十三年,許八品以下官因公漂歿及軍營病故者,贈銜,廕一子監生,許應試,不得銓選。光緒二年奏定現任官遇賊殉難及軍營病故,如系以何種官階升用、補用、即用並捐保升銜者,依升階、升銜、贈銜,依實官給廕。候補、候選者,依現任官廕、贈。休致、告病者,依原官廕、贈。降調者,依所降官廕、贈。已揀選之舉人,就職、就教之恩、拔、副、歲、優貢生,並考有職銜之捐納貢監生,各按品級、依現任官廕、贈。未經揀選舉人,依七品例。恩、拔、副、歲、優貢生依八品例。廩、增、附文生員依九品例廕、贈。虛銜頂戴人員,止予贈銜,不給廕。

乾隆以前,旗員效力行間,懋著勞績,及臨陣捐軀者,其子孫例得世職。年未及歲,已承襲未任職者,給半俸。綠營員弁陣亡議恤,僅得難廕而已。乾隆四十九年詔曰:「旗員及綠營人員,效命疆埸,同一抒忠死事,何忍稍存歧視。嗣後綠營員弁軍功議敘恤賞,仍依舊例。陣亡人員,無論漢人及旗人,用於綠營者,一體給予世職。襲次完時,依例酌給恩騎尉,俾賞延於世。」自是漢員死難者,亦多得世職矣。

凡殉難贈銜,總督加尚書銜者,贈太子少保銜。巡撫加副都御史銜者,贈左都御史銜。布政使贈內閣學士銜。按察、鹽運使贈太常寺卿銜。道員贈光祿寺卿銜。知府贈太僕寺卿銜。同知、知州、通判贈道銜。知縣贈知府銜。教諭、訓導贈國子監助教、學錄銜。其餘各官,按品級比例加贈。光緒二年,定內洋、內河漂歿及軍營病故者,減等贈銜。惟總督、巡撫、布政使,無庸議減,仍減等給廕。

凡給廕,康熙間定制,三品以上廕知州,四品以下至通判廕知縣,布政、按察、都轉鹽運三司首領官及州、縣佐貳六品、七品官廕縣丞,八品、九品官廕縣主簿,未入流廕州吏目。光緒二年,定遇賊殉難官給廕如康熙舊制。惟知縣廕州判,軍營病故及因公漂歿者,減等廕子。武職難廕,有都司、守備、千總、把總,與恩廕改用武職同。凡給世職,陣亡提督,依參贊、都統例,給騎都尉兼一雲騎尉。總兵官依副都統例,給騎都尉。副將以下,把總、經制、外委以上,依參領以下及有頂戴官以上例,俱給雲騎尉。應襲人員年十八歲者,送部引見,發標學習。未及歲者給半俸,及歲補送引見。光緒間,部章恩廕許分發,難廕不得援例。二十二年,熙麟奏言:「恩廕既分部並外用,待之已優,又予分發,難廕專外用,待之已絀,又不予分發,殊失其平。今時事多艱,需人孔亟。正賴鼓天下忠義之氣,俾臨難毋茍。顧於恩廕則為顯宦兒孫擴功名之路,於難廕不為忠臣後裔開一線生機,是使國殤飲恨於重泉,忠義灰心於臨事。請飭部臣援恩廕外用例,一體分發補用。」下部議行。

特廕,乾隆三年詔曰:「皇考酬庸念舊,立賢良祠於京師。凡我朝宣勞輔治完全名節之臣,永享禋祀,垂譽無窮。其子孫登仕籍者固多,或有不能自振、漸就零落者,朕甚憫焉。其旁求賢良子孫無仕宦者,或品級卑微者,各都統、督、撫,擇其嫡裔,品行材質可造就者,送部引見加恩。」四十七年,原任廣西巡撫、滅寇將軍傅弘烈曾孫世海等,降旨錄用。嘉慶四年,追贈已故御史曹錫寶副都御史,依贈銜給予其子廕生。歷代眷念功臣後嗣,恩旨屢頒。光緒季年,海內多故,因思將帥有功之臣,詔曰:「咸、同以來,發、捻、回匪,次第戡定。文武大員,勛績卓著。懋賞酬庸,閱時五十餘年。各勛臣子孫,名位顯達者,固不乏人;而浮沉下位,伏處鄉里者,亦復不少。」令各督、撫、都統詳察勛臣後裔,有無官職,匯列上聞。軍機大臣繕單呈覽。前西安將軍多隆阿次孫壽慶、曾孫奎弼,湖北提督向榮曾孫楷、乃全,安徽巡撫江忠源孫慎勛、曾孫勤培,布政使銜、浙江寧紹臺道羅澤南孫長耿、曾孫延祚,協辦大學士、四川總督駱秉章孫懋勛、曾孫毓樞,江南提督張國樑孫繩祖、繼祖,巡撫銜、浙江布政使李續賓孫前普、曾孫正繩,兵部尚書彭玉麟次孫見綏、曾孫萬澂,陜甘總督楊岳斌子正儀、孫道澂,四川提督鮑超次子祖恩、孫世爵,署安徽巡撫、布政使李孟群孫興仁、興孝,江西南贛鎮總兵程學啟嗣子建勛,廣東提督劉松山孫國安、曾孫家琨,貴州提督馮子材次子相華、孫承鳳等,命各按官級升用。湖南提督塔齊布,令訪明立嗣,奏請施恩。其明年,又詔開列勛績最著之臣,前雲貴總督劉長佑,臺灣巡撫、一等男劉銘傳,贈布政使、道員王★A5,綏遠城將軍福興,福建陸路提督、一等男蕭孚泗,記名提督、一等子、河南歸德鎮總兵李臣典,浙江提督鄧紹良,都統銜、廣東副都統烏蘭泰,署廣西提督、甘肅肅州鎮總兵張玉良,工部左侍郎呂賢基,漕運總督袁甲三,都察院副都御史、江西巡撫張芾,署貴州巡撫韓超,布政使銜,福建督糧道趙景賢,雲南鶴麗鎮總兵硃洪章,直隸總督郭松林,廣東等省巡撫蔣益澧,布政使銜、江南道員溫紹原,署安徽廬鳳潁道金光箸,護軍統領恆齡,新疆巡撫、一等男劉錦棠,記名提督、廣西右江鎮總兵張樹珊,贈布政使銜、升用知府、天津知縣謝子澄,令各都統、督、撫訪明有無後嗣,有何官職,請旨施恩。若夫乾隆四十八年錄用明臣經略熊廷弼五世孫世先,督師袁崇煥五世孫炳,則推恩特廕勝代忠臣後裔,尤曠典也。

任官之法,文選吏部主之,武選兵部主之。吏部四司,選司掌推選,職尤要。凡滿、漢入仕,有科甲、貢生、監生、廕生、議敘、雜流、捐納、官學生、俊秀。定制由科甲及恩、拔、副、歲、優貢生、廕生出身者為正途,餘為異途。異途經保舉,亦同正途,但不得考選科、道。非科甲正途,不為翰、詹及吏、禮二部官。惟旗員不拘此例。官吏俱限身家清白,八旗戶下人,漢人家奴、長隨,不得濫入仕籍。其由各途入官者,內則修撰、編、檢、庶吉士、主事、中書、行人、評事、博士,外則知州、推官、州縣教授,由進士除授。內閣中書、國子監學正、學錄、知縣、學正,由舉人考授及大挑揀選。小京官、知縣、教職、州判,由優、拔貢生錄用。員外郎、主事、治中、知州、通判,由一、二品廕生考用。此外貢監生考職,用州判、州同、縣丞、主簿、吏目、京通倉書、內閣六部等衙門書吏、供事,五年役滿,用從九品未入流。禮部儒士食糧三年,用府檢校、典史。吏員考職,一等用正八品經歷,二等用正九品主簿,三、四等用從九品未入流。官學生考試,用從九品筆帖式、庫使、外郎。俊秀識滿、漢字者考繙譯,優者用八品筆帖式。厥後官制變更,略有出入。其由異途出身者,漢人非經保舉、漢軍非經考試,不授京官及正印官,所以別流品,嚴登進也。

凡內、外官分滿洲缺、蒙古缺、漢軍缺、漢缺。滿洲又有宗室、內務府包衣缺。其專屬者,奉天府府尹、奉錦、山海、吉林、熱河、口北、山西、歸綏等道缺。各直省駐防官、理事、同知、通判為滿洲缺。唐古特司業、助教、中書、游牧員外郎、主事為蒙古缺。欽天監從六品秋官正為漢軍缺。宗人府官為宗室缺。內務府官為內務府包衣缺。此外京師各衙門、陵寢衙門、盛京五部、各直省地方俱設額缺。滿洲京堂以上缺,宗室漢軍得互補。漢司官以上缺,漢軍得互補。外官蒙古得補滿缺,滿、蒙包衣皆得補漢缺。惟順天府府尹、府丞,奉天府府丞,京府、京縣官,司、坊官不授滿洲。刑部司官不授漢軍。外官從六品首領,佐貳以下官不授滿洲、蒙古。道員以下不授宗室。其大凡也。

官吏論俸序遷曰推升,不俟俸滿遷秩曰即升。內而大學士至京堂,外而督、撫、籓、臬,初因明制由廷臣會推。嗣停會推,開列題請。太常、鴻臚、滿洲少卿,開列引見。不開列,以應升員擬正、陪引見授官曰揀授,論俸推取二十人引見授官曰推授。京司官、小京官、筆帖式,分留授、調授、揀授、考授,皆引見候旨,餘則選。外官布政使、按察使開列,運使請旨。道府缺有請旨、揀授、題授、調授、留授,餘則選。、州、縣缺同道、府,無請旨者。佐雜、教職、鹽官,要缺則留,餘或咨或選。初京司官缺,題、選無定例,長官以意為進退。久之,員缺率由題補,而應升、應補、應選者多致沈滯。乾隆九年,詔以各司題缺咨部註冊,餘缺則選,不得混淆。於是定各部各司漢郎中、員外郎、主事各幾缺題授,餘若干缺則選。道光間,更定題補缺額,嗣各部時有增益。順治十二年,詔吏部詳察舊例,參酌時宜,析州、縣缺為三等,選人考其身、言、書、判,亦分三等,授缺以是為差。厥後以沖、繁、疲、難四者定員缺緊要與否。四項兼者為最要,三項次之,二項、一項又次之。於是知府、同、通、州、縣等缺,有請旨調補、部選之不同。

凡選缺分即選、正選、插選、並選、抵選、坐選,各辨其積缺不積缺,到班者選之。選班有服滿、假滿、俸滿、開復、應補、降補、散館庶吉士、進士、舉、貢、廕生、議敘、捐納、推升。大選雙月,急選單月。滿、蒙、漢軍上旬,漢官下旬,筆帖式中旬。初制,選人均到部投供點卯,已而例停,令各回籍,部查年月先後掣選,寄憑赴任。康熙二年,給事中於可託言:「寄憑既慮頂冒,遠省選人往返輒經年。遇有事故,繳憑更選,亦復需時。懸缺遲久,劾署員肆貪,催新任速赴者,連章見告。宜仍令人文到部,按次銓選。」八年,御史戈英復以為言。議行。自是應選者悉赴部投供點卯,為永制。聖祖念選人一時不能得官,往往饑寒旅邸,令吏部截留一年選人留京,餘聽回籍。御史田六善言:「半載以來,截留推官八十選一人,知縣三百選三十一人,餘須守候三、四年。陪掣空簽,選期難料。當按名挨掣實簽,臨選前兩月投供。」下部議,罷按月點卯及掣空簽,詔減半截留人數。選人投供,初於應選前月十五日,距選期近,出缺美惡易滋弊。後改每月初一日投供,間一選期銓補,著為令。選人得缺,初試以八股時文,尋罷。改書履歷三百字,條列治民厚俗、催科撫字、讞獄聽訟諸方法,謂之條陳。補任、升任,並須敷陳舊任地方利弊。然條陳多倩作,或但作頌聖語,其制未久亦廢。選人例由吏部會九卿驗看,後增科、道、詹事。康熙二十七年,從御史荊元實言,令州、縣、同、通等官掣缺後,俱隨本引見,後世踵行焉。故事,大臣驗看月官,查有行止不端、出身不正、祖父有錢糧虧空或人缺不相當者以聞。乾隆時,月官有人缺不稱,引見時帝輒為移易,頗足劑銓法之窮。十年,引見月官,帝以知縣周仲等四人衰頹,特降教職。十二年,復親汰衰庸不勝知縣四人,而切責驗看諸臣之不糾舉。厥後分發、候補、試用之州、縣、同、通,且一體引見,不限實官。久之,州、縣、同、通在外補官,及雜職分發,並得援例捐免引見,驗看益視為具文,無足輕重矣。

內、外官互用,本有成例。初行內升、外轉制。在內翰、詹、科、道四衙門品望最清,升轉特異他官。編、檢遷中允、贊善曰開坊,他若翰、詹、坊、局、國子監堂官、京堂,俱得升調,大考上第,擢尤不次。外轉例始順治十年,詔定少詹事以下二十一員用司、道,治行優者,內擢京堂。尋更定正、少詹事用布政,侍讀學士用按察,中允用參政,編、檢用副使。十八年,復定侍讀以下每年春秋外轉各一員,讀、講用參政,修撰用副使,編、檢用參議。未幾例停。康熙二十五年,甄別翰林官平常者,外用同知、運副、提舉通判。二十八年,編修李濤外簡知府,翰林官授知府自濤始。三十七年,左都御史吳涵言編、檢升轉遲滯,請破格外用,照編修李濤、檢討汪楫例,補知府一、二人。若破格改授,請照少詹王士禎、徐潮,侍讀顧藻,編修王九齡例,用副都御史、通政使。帝納其言,為授檢討劉涵知府。雍正初,以編、檢、庶吉士人多,內用科、道、吏部,外用道、府、州、縣,以疏通之。嗣是編、檢率內升坊缺,用科、道,外授道、府,以為常。吏部六官之長,初定司官內升、外轉歲各一人。已,罷其制。康熙八年,用御史餘縉言復之。四十年,例復停,與他部司員一體較俸。給事中升轉歲一次,御史倍之,外簡道、府,內擢京堂。五十九年,詔定歷俸制,由編、檢、郎中授者限二年,員外郎或主事授者遞增一年。乾隆十六年,定科、道三年升轉一次,五十五年停其例。內官外用,京察外有截取保送,皆俟俸滿保送。分發截取,則選繁簡,由長官定之。府、牧、令、丞、倅皆得以其班次改外。外官內升,初定司、道歲三人,漢人以科目出身,且膺卓異、俸薦俱優者為限。

知縣行取,蓋仿明制,初有薦推、知皆得考選科、道。康熙間屢詔部臣行取賢能,內用科、道。吳江知縣郭琇、清苑知縣邵嗣堯、三河知縣彭鵬、靈壽知縣陸隴其、麻城知縣趙蒼璧,皆以大臣薦舉,行取授御史,得人稱最。四十三年,川撫能泰請罷督、撫保題例,帝韙之。詔嗣後知縣無錢糧盜案者,省行取三、四員。明年,御史黃秉中言知縣考選科、道,殊覺太驟。廷議停止。尋定行取三年一次,直隸、江南、湖廣、陜西各五員,餘省三員、一員不等,以主事補用。雍正間,刑部尚書徐本請復行取御史舊制,格於部議。行取官用主事者,初選補猶易,後與捐納間補,遂病壅滯。乾隆元年,令視武官保舉註冊例,仍留本任。已赴京者,許外補同知。時各省視行取為具文,例以無參罰之次等州、縣應選,十六年罷之。洎光緒季年,令州、縣以上實官及曾署缺者,一體考試御史。非復行取遺意,亦行之未久而罷。

銓選按格擬註,憑簽掣缺,拘於成例,歷代間行保薦制,以補銓法之不逮。順治初,定保舉連坐之法。十二年,以直隸保定、河間,江南江寧、淮、揚、蘇、松、常、鎮,浙江杭、嘉、湖、紹等三十府,地方緊要,詔京、外堂官、督、撫各舉一人備簡,不次擢用。已,有以貪庸敗者,給事中任克溥言:「皇上對天下知府中權其繁劇難治者三十,許二品以上官薦舉,破格任用。為時未久,以貪劣劾罷者數人。諸臣不能仰承聖意,秉公慎選,乞下吏議。」從之。康熙七年,詔部、院滿、漢官才能出眾者,許不計資補用。明年,吏部請罷保薦,仍循俸次升轉,以杜鉆營賄賂。報可。四十年,令總督郭琇、張鵬翮,巡撫彭鵬、李光地等,各舉道、府、州、縣惠愛清廉者以聞。世宗御極,屢詔京、外大臣薦舉道、府、同、通、州、縣,所舉非人,輒遭嚴譴。戶部尚書史貽直言:「遷擢宜循資格,資格雖不足以致奇士,而可以造中材。捐棄階資,幸進者不以為獎勵之公,而陰喜進取之獨巧;沈滯者不自咎才智之拙,而徒怨進身之無階。請照舊例,循階按級,以次銓除。果才猷出眾,治行卓越,仍許破格薦擢。」從之。乾隆間,厲行保薦之法,司、道、郡守,多由此選。宣宗初元,郎中鄭裕、知府阿麟、唐仲冕,皆以大臣推舉,陟方面、擢疆圻。歷代相沿,率以薦賢舉能責諸臣工,間亦破格任用。初京職簡道、府,疆吏察其才不勝任,疏請調京任用,多邀俞允。乾隆初,廷臣有以衰廢之人不宜復玷曹司為言者,詔切止之。嗣是外官才力不及者,但有休致、降補,無內用矣。

官吏升轉論俸,惟外官視年勞為差,異於京秩。在外有邊俸,有腹俸。腹俸之道、府、州、縣佐貳、首領官,五年無過失,例得遷擢。邊俸異是。廣東崖州、感恩、昌化、陵水等縣,廣西百色、太平、寧明、明江、鎮安、泗城、凌雲、西隆、西林等府、、州、縣及忠州、河池等數十雜職,為煙瘴缺。雲南元江、鶴慶、廣南、普洱、昭通、鎮邊等府通判、同知,鎮雄、恩樂、恩安、永善、寧洱、寶寧等州、縣,貴州古州兵備道,黎平、鎮遠、都勻、銅仁等府同知,清江、都江、丹江通判,永豐知州,荔波知縣,四川馬邊、越巂同知,為苗疆缺。俱三年俸滿,有政績、無差忒者,例即升用。江蘇太倉、上海等十縣,浙江仁和、海寧等十七縣,山東諸城、膠州等七州、縣,廣東東莞、香山等十三縣,福建閩侯等九縣,為沿海缺。直隸良鄉、通州等十二州、縣,河南祥符、鄭州等十一州、縣,山東德州、東平等十三州、縣,江南山陽、邳州等十三州、縣,為沿河缺。歷俸升擢,與邊俸同。邊疆水土惡毒,或不俟三年即升。其水土非甚惡劣,苗疆非甚緊要者,升遷或同腹俸。乾隆間,定邊缺、夷疆、海疆久任之制,升用有須滿八年或六年者,則為地擇人,不拘牽常例也。

選班首重科目正途。初制,進士知縣惟雙月銓五人,選官有遲至十餘年者。雍正二年,侍郎沈近思請單月復銓用四人。於是需次二、三年即可得官。舉、貢與進士雖並稱正途,而軒輊殊甚。順治間,貢生考取通判,終身無望得官。乾隆間,舉人知縣銓補,有遲至三十年者。廷臣屢言舉班壅滯,然每科中額千二百餘人,綜十年計之,且五千餘人,銓官不過十之一。謀疏通之法,始定大挑制。大挑六年一舉行,三科以上舉人與焉。欽派王大臣司其事,十取其五。一等二人用知縣,二等三人用學正、教諭。用知縣者,得借補府逕歷、直隸州州同、州判、縣丞、鹽庫大使。用學正、教諭者,得借補訓導。視前為疏通矣。異途人員,初與正途不相妨。康熙初,生員、例監、吏員出身官,須經堂官、督、撫保舉,始升京官及正印官。無保舉者,郎中、員外郎、主事以運同、府同知分別補用。漢軍捐納官,朝考後方得授官。十八年,復令捐納官蒞任三年稱職者,題請升轉,否則參劾,以示限制。自二十六年,以宣大運輸,許貢監指捐京官正印官者,捐免保舉。尋復許道、府以下納貲者,三年後免其具題,一例升轉。於是正途、異途始無差異。乾、嘉以後,納貲之例大開,洎咸、同而冗濫始甚。捐納外復有勞績一途,捐納有遇缺侭先花樣,勞績有無論題選咨留遇缺即補花樣,而正途轉相形見絀。甲榜到部,往往十餘年不能補官,知縣遲滯尤甚。光緒二年,御史張觀準條上疏通部員之法:一,捐納部員勿庸減成;一,主事俸滿即準截取;一,散館主事侭數先選;一,進士主事準以知縣改歸原班銓選。報可。順天府府尹蔣琦齡亦言各省即用知縣,不但無補缺之望,幾無委署之期,至有以得科名為悔者。廷臣多以進士知縣壅滯,紛請變更成例,帝輒下所司覈議。十六年,御史劉綸襄言:「近日諸臣條奏選補章程,吏部議覈,日不暇給。朝廷設官,惟期任用得人,以資治理,非能胥天下仕者使盡償所原也。國家缺額有定,士子登進無窮。安得如許美官,以待縈情膴仕之人?徒滋紛擾,無濟於事。」帝為下詔切止之。是時異途競進,疆吏多請停分發。吏部以仕途幸濫,申多用科甲之請。勢已積重,不能返也。

滿人入官,或以科目,或以任子,或以捐納、議敘,亦同漢人。其獨異者,惟筆帖式。京師各部、院,盛京五部,外省將軍、都統、副都統各署,俱設筆帖式額缺。其名目有繙譯、繕本、貼寫。其階級自七品至九品。其出身有任子、捐納、議敘、考試。凡文、武繙譯舉人、貢監生,文、武繙譯生員,官、義學生、驍騎閑散,親軍領催,庫使,皆得與試。入選者,舉、貢用七品,生、監用八品,官、義學生、驍騎閒散等用九品。六部主事,額設百四十缺,滿、蒙缺八十五,補官較易。筆帖式擢補主事,或不數年,輒致通顯。其由科甲進者,編、檢科僅數人,有甫釋褐即遷擢者。翰林坊缺,編、檢不敷補用,得以部院科甲司員充之,謂之外班翰林。外官東三省、新疆各城,各省駐防文、武大員,俱用滿人。甘肅、新疆等邊地道、府、同、通、州、縣,各省理事、同知、通判,皆設滿洲專缺。滿缺外,漢缺亦得補用。其有終養回旗,得授京秩。內、外文職選補,一時不能得官,及降調、咨回各員,許改授武職,尤特例也。

保舉為國家酬庸之典,所以勵勞勩、待有功也。歷朝纂辦實錄,各館奉敕修書,及各省軍營、河工、徵賦、緝盜有功者獎敘。康熙十一年世祖實錄成,四十九平定朔漠方略成,副總裁以下官但獎加級。六十一年算法成書,始議以三等敘功,獎應升、加等、即用有差。康、雍兩朝實錄成,從總裁請,無議敘。嘉慶間,修書館臣請超一、二等優獎,帝不許。尋定非特旨專設之官,不得議敘、升用,歷代踵行。其軍營、河工等獎案,始不過加級,或不俟俸滿即升,名器非可幸邀。迨季世以保舉為捷徑,京、外獎案,率冒濫不遵成例。光緒元年,御史王榮琯請下越階保升之禁。帝韙之。三年,以纂修穆宗實錄過半,與事諸臣俱保升並加銜,備極優異。十年,部議限制保舉,五、六品京堂、翰、詹坊缺,及遇缺題奏,俱不得擅保。未幾,仍有以候補郎中保京堂,編、檢保四、五品坊缺,及應升缺並開列在前者。咸、同軍興,保案踵起。吏部於文選司設專處司稽核,事之繁重,與一司埒。同治十二年,閩撫王凱泰言:「軍興以來,保案層迭,開捐以後,花樣紛繁。軍營保案,藉花樣以爭先恐後,各項保舉,又襲軍營名目以紛至沓來。名器之濫,至今已極。盈千累百,徒形冗雜。請敕部察核京、外各班人員,酌留二、三成,餘令回籍候咨取。」下所司覈議。軍功外,號稱冗濫者,為河工保。光緒二十年,御史張仲炘言:「山東河工保案,近年多至五、六百人。部定決口一處,獎異常、尋常者六人。該省所報決口多寡,輒以所保人數為衡。圖保者以山東為捷徑,捐一縣丞、佐雜,不數月即正印矣。請飭所司嚴定章程。」帝俞其請。

三十二年,御史劉汝驥復言:「史治之蠹,莫如保舉一途。其罔上營私者,一曰河工。國家歲糜數十萬帑金以慎重河防,封疆大吏乃以此為調劑屬員之舉。幸而無事,丞、倅保州、縣矣,同、通保府、道矣。一曰軍功。工廠之鼓噪,饑民之嘯聚,輒浮誇其詞曰大張撻伐。耳未聞鼙鼓,足未履沙場,而謬稱殺敵致果、身經百戰者,比比然也。一曰勸捐。順天賑捐一案,保至千三百餘人,山東工賑,保至五百餘人,他省歲計亦不下千人。請嚴禁徇情濫保,以杜幸進。」下所司核議限制之法。其時吏部投供月多至四、五百人,分發亦三、四百人,選司原設派辦處,司其事者十餘人,猶虞不給。季年乃毅然廢捐納,停部選,為疏通仕途,慎選州、縣之計。然捐例雖停,而舊捐移獎,層出不窮。加以科舉罷後,學堂卒業,立獎實官。舉、貢生員考職,大逾常額。且勛臣後裔,悉予官階,新署人員,虛銜奏調。紛然錯雜,益難紀極。宣統三年,裁吏部,設銓敘局,雖有刷新政治之機,而一代銓政,終不復能廓清也。

武職隸兵部,八旗及營、衛官之選授,武選司掌之。內而驍騎、前鋒、護軍、步軍、火器、健銳、虎槍各營,外而陵寢、圍場、熱河、烏里雅蘇臺、科布多、阿爾泰、烏梁海、西寧、西藏、塔爾巴哈臺游牧、察哈爾、綏遠城、各省駐防,皆旗缺,屬八旗。門千總為門缺,屬漢軍。河營、陸路、水師皆營缺,滿、漢分焉。漕運為衛缺,漢軍、漢人得兼補。旗缺副都統以上開列,餘則揀選。五品以上題補,六品以下咨補。綠旗總兵以上,初用會推,嗣罷其例,開列具題。副將投供引見,亦有開列者。其次要缺則題,簡則推,把總拔補。其大略也。

凡滿、漢入仕,有世職、廕生、武科。八旗世職,公、侯、伯、子、男補副都統,輕車都尉、騎都尉補佐領,雲騎尉補防禦,恩騎尉補驍騎校。漢伯、子、男用副將,輕車都尉用參將,騎都尉用游擊或都司,雲騎尉用守備。尚書至副都御史等官,總督、將軍至二品銜副將廕生改武者,用都司、守備、守御所千總、衛千總。武科進士一甲一名授頭等侍衛,二、三名授二等侍衛,二、三甲揀選十名授三等侍衛,十六名授藍翎侍衛,餘以營、衛守備補用。漢軍、漢人武舉揀選一、二等用門千總及營千總,三等用衛千總。其以資勞進用者,營伍差官,提塘,隨幫,隨營差操,經制及外委,千、把總、無責任效用官,因功加都督至副將等銜者用游擊。加參將、游擊銜者用都司。加都司、守備銜者用守備。加千總銜者拔補把總。武進士、武舉充提塘差官滿三年,由部考驗弓馬,優者用營、衛守備,次者武舉用防禦所千總。武舉隨營差操滿三年,以營千總拔補。隨幫三運報滿,用衛千總。凡部推之缺,歲二月,參將、游擊缺,用漢一、二等侍衛一人。四、六、八月游擊、都司缺,用漢三等侍衛三人。正、三、五、七、九月都司缺,用藍翎侍衛五人。正月、七月營守衛缺,以門、衛千總升用。其餘單月缺輪補之班七,雙月缺輪補之班十二,衛守備單月缺輪補之班十一,雙月缺輪補之班六,守御所千總、衛千總缺,俱不論雙、單月推選,惟門千總專於雙月銓補焉。

滿人入官,以門閥進者,多自侍衛、拜唐阿始。故事,內、外滿大臣子弟,五年一次挑取侍衛、拜唐阿,以是閒散人員,勛舊世族,一經揀選,入侍宿衛,外膺簡擢,不數年輒致顯職者,比比也。綠旗武職,占缺尤多。向例山海關至殺虎口、保德州副、參、游、都、守缺,綠旗補十之三,滿洲補十之七。馬蘭、泰寧二鎮,直隸、山西沿邊副、參、游、都、守缺,滿、漢各補其一。雍正六年,副都統宗室滿珠錫禮言京營參將以下、千總以上,不宜專用漢人。得旨:「滿洲人數本少,補用中、外要缺已足,若京營參將以下悉用滿洲,則人數不敷,勢必有員缺而無補授之人。」乾隆間,揀發各省武職,率以滿人應選。帝曰:「綠營將領,滿、漢參用,必須員缺多寡適均,方合體制。若概將滿員揀發,行之日久,綠營盡成滿缺,非所以廣掄選而勵人材。」飭所司議滿、漢間用之法。兵部議上,凡行走滿二年之漢侍衛,與巡捕營八旗滿、蒙人員,由該管大臣保送記名。揀發時,與在部候補、候推者,按滿、漢分派引見。如所議行。三十八年,兵部復疏言:「直隸、山西、陜西、甘肅、四川五省,自副將至守備,滿缺六百四十七,各省自副將至守備,千一百七十九缺,向以綠營人員選補。現滿、蒙在綠營者逾原額兩倍,實緣各省請員時,多用滿員揀選。請嗣後除原用滿員省分外,其河南、山東、江南、江西、湖廣腹地及閩、浙、兩廣海濱煙瘴等省,需員請揀,應於綠營候補候選,及保卓薦人員,並行走年滿之頭、二、三等侍衛、藍翎侍衛,一並揀選。」從之。自是綠營滿、漢員缺始稍劑其平,非復從前漫無限制矣。

武職以行伍出身為正途,科目次之。故事,考驗部推人員衰老病廢者,勒令休致。惟軍功帶傷者,雖年老仍行推用。副、參例以俸深參、游題補。若有軍功保舉,雖俸淺亦得與焉。科目自康熙初即病壅滯。御史硃斐疏請定科目、行伍分缺選用之制,外委、效力等與武進士、武舉較人數多寡,仿二八分缺之例,先選科目人員。其外委各弁,須有戰功及捕盜實績,不得止憑咨送選補。下所司議行。雍正初,廷臣有請改並衛、所各州、縣者,部議:「科甲人員,專選衛、所守備、千總,若盡裁衛、所,必致選法壅滯,事不可行。」帝不許。為定榜下進士增用營守備以調劑之。乾隆十五年,給事中楊二酉言:「各省、衛守備歸部選者三十九缺,現武進士以衛用者積至數百人,提塘差官、效力報滿歸班選用者亦數十人,加以新例飛班壓銓,缺少班多,選用無期。請照乾隆元年例,將三等武進士再行揀選,一、二等以營用,三等仍以衛用。」報可。向例揀選武進士以營用者,選缺猶易,衛用往往濡滯不能得官。洎道光間,衛用武進士得捐改營用,而裁缺衛守備、衛千總、守御所千總,均準改歸綠營。營守備以上官,並得報捐分發。由是部推、外補,同一沈滯,不僅科目為然矣。

凡不屬於部推之缺,皆題補豫保註冊者最先授。定例邊疆、內河、外海水師員缺及陸路緊要者得豫保。康熙九年,兵部疏言:「總督、提、鎮遇標、營員缺,不論地方緩急,銜缺相當,輒將標員坐名題補,使俸深應補人員致多壅滯。請定副將以下、守備以上缺出,實系近海、沿邊、巖疆人地相宜者,酌量題補,餘不得率行題請。」從之。雍正五年,詔部推缺由各督、撫、提、鎮保題備用。乾隆初,罷陸路近省豫保例。十年,江督尹繼善言:「武職豫保,咨部註冊,遇缺掣補,誠慎重要缺之良法。乃或豫保之初,年力本強,數年後漸已衰老,騎射生疏,營伍廢弛。請將豫保滿三年未得缺者,各提督再行甄別,果堪升用,出具考語咨部,否則註銷。」報可。

其時保薦別以三等,限以五年,於副將堪勝總兵、參將堪勝副將者,尤慎選。一經保薦,輒予升擢。洎咸、同軍興,十餘年保題舊例不復行,所恃以鼓勵人材者,惟軍功保舉。獎敘之案,層出不窮。以兵丁積功保至提、鎮記名者,殆難數計。同治五年,詔以記名提、鎮無標、營可歸者,發往各省各營差遣。各省投標候補者,提、鎮多至數十,副、參以下數百,本職補官,終身無望,於是定借補之法。提、鎮準借補副、參、游缺,副、參、游準借補都、守缺,都、守準借補千、把總缺。雖內停部推,外停侭先,仍不足疏通冗滯。

光緒季年,詔裁綠營,練新軍,罷武科,設武備學校。一時新軍將、弁,與學成授官者,特為優異。歷朝武職尊重行伍之意,蕩無復存。雖綠營武職未盡廢除,然無銓法可言云。


\end{pinyinscope}