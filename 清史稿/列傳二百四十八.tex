\article{列傳二百四十八}

\begin{pinyinscope}
宋慶呂本元徐邦道馬玉昆依克唐阿榮和長順

宋慶,字祝三,山東萊州人。家貧落魄,聞同裏宮國勛知亳州,往依為奴。亳捻孫之友偽就撫,慶察其意叵測,請擊之。國勛壯其志,署為州練長。之友降,遂接統其眾,號奇勝營,薦授千總。自是守宿州,剿豫匪,釋鳳陽圍,保徐、泗後路。逾三歲,擢至總兵,賜號毅勇巴圖魯。既貴,過亳,謁所主,仍易僕廝服,執事上禮益恭,人傳為美談。

同治改元,唐訓方撫安徽,裁臨淮軍,而以三營屬慶,毅軍自此始。三年,苗沛霖圍蒙城,慶絕其餉道。會僧格林沁軍至,轟擊之,寇宵遁。苗酋死,慶為安撫餘眾,自壽州正陽關所蒞皆下。兩淮告寧,調赴豫。時張曜為翼長,慶往訪,詳詢地勢寇情。曜喜曰:「諸將無問及此者,君來,豫之福也!」遂與交驩。明年,授南陽鎮總兵。無何,曹州賊勢熾,慶被困鄧州刁河店,會糧罄,勢且不敵。乃令部將馬玉昆率壯士三百,潛出立營通饋運,軍氣復振,寇乃解去。已而張總愚決河圖北犯,慶據堤迎擊,敗之,西走;而任柱、賴文光復竄豫,湘軍將劉松山助慶軍盡驅入楚疆,豫略平。巡撫李鶴年因增練兩大軍,令曜領嵩武軍,而以毅軍專屬慶。六年,與曜扼黃河,蹙捻至山東,聚而殲之。論功,賞黃馬褂,更勇號格洪額。時總愚南擾河津,偪解州,詔慶與曜分守河北。逾歲,捻竄畿疆,慶率師入衛,轉戰雄、任、祁、高間,與諸軍大破之,總愚赴水死。予二等輕車都尉,授湖南提督。

八年,左宗棠西征,慶引兵從,抵神木,再戰再捷。明年秋,命參哈、寧剿匪事,旋移督西川,皆在軍遙領。十三年,河、狄撫回閃殿臣叛,楚軍戰失利。時慶駐涼州,奉檄往援。三日馳五百餘里,抵沙泥站,眾縛其渠以獻,誅之,事遂定。光緒元年,師還。六年,徙防旅順,十餘年,軍容稱盛。醇賢親王奕枻被命巡閱,嘆為諸軍冠,親解袍服贈之。兩宮眷遇優渥,加太子少保、尚書銜。

二十年,中日失和,慶統毅軍發於旅順,與諸軍期會東邊九連城。軍未集而平壤已失,廷旨罷總統葉志超,以慶代之。慶與諸將行輩相若,驟稟節度,多不懌,以故諸軍七十餘營散無有紀。又坐守江北一月,以待日軍過義州,慶頓中路九連城,嚴戒備。日軍渡鴨綠江,戰失利,直趨鳳凰城,退扼大高嶺。旅順圍亟,朝命聶士成守之,敕慶往援。頓蓋平,屢搗金州不得進,而旅順已失。慶退守熊岳,自請治罪,被宥。未幾,復州又失。日軍西陷海城,慶亟赴之,擊敵感王寨。前軍方勝,後隊訛傳敵拊背,駭潰,復退守田莊臺,遼陽益危。慶凡五攻城弗能拔,朝廷思倚湘軍,命慶與吳大澂佐劉坤一軍。慶率徐邦道、馬玉昆兵萬二千人頓太平山,戰卻之,大澂敗入關。慶方以三萬人駐營口,聞警,還扼遼河北岸;而日軍盡以所獲砲列南岸猛攻,慶軍潰而西,於是遼河以東盡為日有矣。詔褫職留任。

二十四年,徙守山海關,入覲,釋處分。和議成,留豫軍三十營屬之,賜名武衛左軍,駐錦州。二十八年,卒,晉封三等男,予建祠,謚忠勤。子天傑,五品京堂,襲爵。

慶從戎久,年幾八十,短衣帕首,躞蹀冰雪中,與士卒同甘苦,人以為難云。

呂本元,安徽滁州人。初隸李鴻章軍,隨剿粵匪、捻匪,轉戰蘇、皖、魯、豫各省。援鄂、援陜屢立功,歷保總兵,賞強勇巴圖魯勇號。鴻章總督直隸,調入直。光緒初,授四川重慶鎮,仍留統盛軍馬步各營。中日戰起,檄本元統隊出關,兼程至安州。平壤失,從宋軍退守大高嶺。本元令各軍夜樹旗各要隘,廣設疑兵,亙二百餘里。敵至,疑頓不前,乘其疲襲擊之,復與聶軍敗之分水嶺。議成,還直。二十六年,拳禍起,署天津鎮,擢直隸提督,統淮、練各軍。剿匪受彈傷,事平,賞黃馬褂。調浙江,勤訓練,尤嚴治盜,常親督隊入山搜剿,連斃匪首。浙省議裁綠營,本元贊畫始就緒。宣統二年,病,乞罷。尋卒。

徐邦道,四川涪州人。初從楚軍討粵寇,積勛至參將。還本籍籌防,解城圍,遷副將。越境援陜西漢中,賜號冠勇巴圖魯。旋坐漢中失守,褫職。嗣從副將楊鼎勛援蘇,再援浙、閩,以戰功釋處分。同治六年,從劉銘傳剿平東捻,復官。明年,張總愚犯減河,邦道嚴扼橋口,大敗之。更勇號鏗僧額,遷總兵,署江蘇徐州鎮。光緒四年,擢提督,調駐天津軍糧城,授正定鎮。

東事起,慶以旅順守將赴防九連城,李鴻章別令姜桂題等守旅順,邦道助之。日軍入貔子窩,邦道語諸將曰:「金州若失,則旅順不可守,請分兵御之。」諸將各不相統,莫之應。邦道自率所部趨大連灣。是時銘軍分統趙懷益守其地,邦道至,固請兵,乃分步旅隨邦道行。日軍大集,遂占金州,進偪大連,懷益奔旅順。越十日,日軍來爭旅順,諸將相顧無措,邦道率殘卒至,憤甚,思自效,請增兵,不許;請械,許之,乃率眾拒戰土城子,挫之。日軍大至,乃退。道員龔照嶼先一日遁,諸將亦奪民船以濟,蓋日軍未至而旅順已墟矣。邦道奔復州依慶,詔褫職。慶令守蓋州,邦道自牛莊移師還,而蓋平亦已失,合章高元擊之,弗勝。桂題往援,邦道請夜搗蓋平,桂題辭,諸軍皆退營口。邦道乃從慶擊敵太平山,與玉昆力戰卻之,俄仍敗潰。復與湘軍將李光久攻海城,亦弗克,遂退。逾歲,卒,復官,予優恤。

馬玉昆,字景山,安徽蒙城人。以武童從宋慶攻捻,積功至都司,賜號振勇巴圖魯。任柱等困慶登州,玉昆銳身馳救,圍立解,繇是以驍果名。捻平,擢總兵。剿秦、隴回,數獲勝,更勇號曰博奇。既克肅州,賜頭品服。嗣從金順出嘉峪關,連下烏魯木齊、昌吉、瑪納斯,擒其渠黑瞎子。天山南北告寧,賞黃馬褂,予世職。玉昆居西域先後十餘年,收復名城以十數,暇輒使部下屯墾闢地利。李鴻章疏薦將才,謂可繼宋慶。光緒間,調赴直隸。

二十年,補授山西太原鎮。會日朝構釁,玉昆統毅軍赴援,次平壤,壁南門外大同江。日軍來攻,玉昆守東岸,血戰久,援至,敵敗去。已而玄武門失,葉志超令其速撤軍,乃歸平壤。日軍占蓋平,諸將皆退營口。玉昆從慶頓太平山,日軍猛攻之,玉昆戰最力,擊退其眾。無何,日軍大集,慶陷重圍,墜馬負創,玉昆抉圍入,翼之出,傷亡殊多。轉戰田莊臺、感王寨,以千餘人抗強敵,屹然自全。

二十五年,擢浙江提督。明年,調還直隸。適拳匪肇亂,聯軍入寇,玉昆統武衛左軍御之。初戰天津,繼戰北倉,相持月餘,卒以無援退。車駕西幸,命隨扈。又明年,還京,加太子少保。二十八年,朝陽土寇竊發,玉昆倍道應赴,破其卡,生擒首惡鄧萊峰誅之。三十四年,病卒,贈太子太保,予二等輕車都尉,謚忠武。

依克唐阿,字堯山,扎拉里氏,滿洲鑲黃旗人,吉林駐防。以馬甲從征江南。移師討捻,敗張洛行於大回村、濉溪口,屢著戰績,積勛至佐領。同治初,馬賊陷伊通,依克唐阿以少襲眾,斬其酋劉果發等,又破之昌圖,攻克劉家店,復長春,遷協領,賜號法什尚阿巴圖魯。搜捕殘匪,獲白凌阿、焦西平,晉副都統。十一年,補官黑龍江。光緒五年,移呼蘭,呼蘭設副都統自此始。明年,母憂歸。

時俄人以議改伊犁條約有違言,烏里雅蘇臺參贊喜昌夙諗依克唐阿諳戰術,請敕就近募獵戶守琿春。會吉林戒嚴,依克唐阿遂募兵五千擇隘分守,而自率師駐其地。琿春故重鎮,其東南海參崴,俄尤數窺伺,廷議設副都統鎮之,於是又改調琿春。十年,被命佐吉林軍事。十五年,擢黑龍江將軍。

二十年,日朝戰起,依克唐阿請率軍自效,乃進咸鏡道,繞赴漢城迎擊,上嘉之。左寶貴軍失利平壤,日軍西進,命移駐九連城。尋以日軍渡江來攻,復令徙上游御之。依克唐阿與戰於蒲石河,連克蒲石河口、古樓子。宋慶退駐大高嶺,依克唐阿孤軍不能獨守,遂退寬甸。宋軍南援旅、大,聶士成軍接防,乃定夾攻之約。依軍由寬甸繞進賽馬集迎擊日軍,先戰懸羊砬子,連勝之草嶺河、通遠堡、草河口。日軍大集,橫斷聶、依兩軍,士成亟趨分水嶺拊其背,依軍還擊之,陣斬一中尉。又西而東,大戰於金家河,軍稍挫。日軍先已占鳳凰城,依克唐阿謀襲之,分左右翼以進,戰一面山,敵來爭,左翼潰,右翼統領永山遇伏死,依克唐阿保餘軍退,詔革職圖後效。

逾歲,海城陷,遼西危棘,詔責長順守遼陽,依克唐阿助之,發帑金五十萬濟依軍。既至,議以攻為守。乃集諸將置酒,取刀刺臂血,攪而飲之,相矢以死。依軍遂進取海城,軍騰鰲堡、耿莊,數戰弗勝。會榮和至軍,亟趣之出。榮和先進北路,奪三卡,其左樹木幽深,令隱兵備抄襲,而自列陣曠野,伏槍以待。日軍據山巔轟擊我師,彈落積雪中,漬不發。我師還擊,僕者眾,再發再僕。眾爭傍山出,伏槍具舉,死以百數。榮和所部募自塞邊外,善避擊,傷者恆少,所謂「東山獵戶」也。是役以千人抗日軍數千,故依軍聲譽遠出諸軍上。

罷戰詔下,日人將歸我遼東,依克唐阿力請三路分兵鎮懾,稱旨。又條上練兵隊、築砲臺、造鐵路、制槍械、開礦產、治團練六事,朝旨以礦政尤要,敕妥籌開採。又明年,晉頭品秩,授鑲黃旗漢軍都統。其秋,出為盛京將軍。既蒞事,糾貪墨,整營制,晰分釐稅,歲增餉銀數十萬。復撤還金州奉軍,杜俄人藉口,境內稱治。二十五年,卒,謚誠勇,予建祠。

依克唐阿勇而有謀,性仁厚,不嗜殺,每有俘獲,不妄戮一人。轉戰吳、皖、魯、豫,先後救出難民以十數萬計,至今人尸祝之。初與長順訂兄弟交,長順兄事之。及議遼陽戰守,語不協。依克唐阿毅然獨任其難,曰:「孰使我為兄也者?」其雅量如此。

榮和,字育堂。二等侍衛,官至副都統。戰後所部育字營多驕縱,命李秉衡查辦,革職治罪。

長順,字鶴汀,達呼裏郭貝爾氏,隸滿洲正白旗,世居布特哈。起家藍翎侍衛,隨文宗車駕狩熱河。會馬賊陷朝陽,從大學士文祥討平之。嗣復從侍郎勝保征捻,轉戰直、魯、皖、豫,以驍勇稱。同治元年,解潁州圍,以功遷二等。

多隆阿主陜西軍事,調赴軍,至潼關,大敗寇眾,賜號恩特赫恩巴圖魯。進攻咸陽馬家堡,被巨創,援至,又大破之,咸陽復,晉頭等。三年,悍回馬化隆據寧夏,分其黨駐清水堡成犄角,師久攻不下。長順曰:「不先翦其羽翼,城未可克也!」乃自靈州襲清水堡,乘勝取寧夏,拔之,晉副都統,賜頭品服。時長順年未四十,而戰常陷堅。每當兵潰時,或抄襲其後,或橫阻其前,俾潰者得整列,以是常轉敗為勝。其旗幟尚白,寇望見之,輙呼曰:「小長將軍至矣!」相與戒勿犯,其為寇所憚若此。

六年,移師蘭州。時省城戎備寡,回眾數千突來犯,長順率百人隱小溝,出不意疾擊之,寇愕走,又敗之平番、皋蘭、狄道,既復規取河州,連破太子寺、高家集,被賞賚。八年,授鑲紅旗漢軍副都統。越二年,出署烏里雅蘇臺將軍,坐事免。

光緒二年,復官,左宗棠調赴甘肅,歷署巴里坤領隊大臣、哈密幫辦大臣。初,新疆南路勘界議起,當事者與俄使相持久不決。至是,長順陟巉巖,披蒙茸,獲見高宗御書界碑,俄使始無異辭,乃定。明年,假歸,歷授正白旗漢軍都統、內大臣。十四年,出為吉林將軍。既蒞事,賑菑荒,維圜法,均釐榷,澄吏治,清盜源,整旗務,境內一切皆治辦。又創修吉林通志,書成上之。

二十年,日軍陷海城,遼陽危。朝命長順往援,節制奉天各軍,並嚴詔:「遼陽有失,唯長順是問。」時潰軍紛集遼城下,署知州徐慶璋方閉城不令入,軍大閧。會長順領百騎至,斬閧者一,餘令還駐沙河。先是長順被命以軍五千分隊應赴,先至者令壁本溪湖,自輕騎入遼陽。亂既定,日軍諜者亦不知其止百騎也,第歸言某將軍至。日軍遂止弗前,遼陽乃保。已而進攻海城,戰數日弗勝,長順奏趣宋慶會師,詔不許。湘軍將陳湜至,又請劉坤一令合攻,亦未果。及日軍繞道復攻遼陽,適慶璋守靦峒峪,長順與依克唐阿回援,得無恙。和議成,請疾歸。

二十五年,復起吉林將軍。拳亂作,俄羅斯內犯,奉天、黑龍江皆主戰,長順獨持不可。又上言拳匪不可恃,東省鐵路隨地皆駐俄兵,宜善為羈縻,寧嚴守以待戰,毋先戰以啟釁。上嘉其老成持重,奉、吉軍事悉屬之。戰釁既開,奉、黑皆罹災,而吉林安堵,人服其先見。日俄之戰,守中立,獨無所犯。三十年,卒,贈太子少保,予一等輕車都尉,謚忠靖,入祀賢良祠。

長順聳幹赬面,須眉灑然。富膽略,恆持短矛單騎穿賊陣,為士卒先。往往以少制眾,以奇制勝,兼謀勇,一時稱良將雲。

論曰:中日之戰,淮軍既覆,湘軍隨之,唯豫軍強起支搘。慶與玉昆先後失利,亦不復能自振焉。東三省練軍自成軍後,終未當大敵,而依克唐阿、長順一奮其氣,遂保遼陽而無失,中外稱之。喪師辱國者數矣,此固差強人意者哉。


\end{pinyinscope}