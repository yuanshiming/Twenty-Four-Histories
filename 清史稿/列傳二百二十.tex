\article{列傳二百二十}

\begin{pinyinscope}
金國琛黃淳熙吳坤修康國器李鶴章弟昭慶吳毓蘭

金國琛,字逸亭,江蘇江陰人。咸豐中,以諸生謁羅澤南於江西,使參軍事。每出戰,部伍嚴整,倉猝犯之,屹然不亂。轉戰弋陽、廣信、武昌、黃州,累功擢知縣。七年,李續賓代澤南,使總理營務。率師會襲湖口,克之。進復彭澤小姑洑、泰坪關,擊退援賊,晉秩同知直隸州。八年,從克九江,窺安徽,下太湖、潛山、桐城。續賓戰歿三河,國琛與其弟續宜招集散亡,勞徠撫慰,重申紀律,軍勢復振。

九年秋,石達開犯湖南,圍寶慶。國琛從續宜赴援,毀田家渡賊壘,又敗賊賀家坳,斬悍賊胡德孝,賊走廣西,擢知府。其冬,胡林翼、曾國籓規皖,精兵猛將萃於潛山、太湖。陳玉成糾眾數十萬,結捻匪龔瞎子圍鮑超於小池驛,救兵迭失利。先是林翼以國琛行軍善規地勢,令率十四營冒雪趨天堂備援。至事急,乃出高橫嶺,屯仰天庵,俯視賊營,皆在目中。賊驟見旗幟,大驚。十年正月,賊乘霧登山來犯,國琛揮軍突起躪之,合山下軍奮擊,斬馘逾萬,乘勝克潛山、太湖。林翼疏陳:「非鮑軍之堅忍,不能久持;非國琛之出奇制勝,不能轉危為安。」論功,擢道員。

十一年,粵匪復犯湖北,國琛馳援武昌,連復孝感、雲夢,進攻德安。賊酋馬融和死鬥,卒以長圍克之,加布政使銜。尋授湖北安襄鄖荊道,仍兼治軍。樊城地沖要,商賈所集,督軍士築土城,不煩民力,賴為保障。時捻匪西擾關中,命國琛率師赴援,以鄖西戒嚴,留未行。

同治元年,馬融和以眾六萬圍南陽,國琛越境往援,力戰解城圍,拔出難民數萬。巡撫嚴樹森忌之,劾其不遵調度,以同知降補。二年,曾國籓調統義從營。擊賊徽州,屢捷於豹嶺、佛嶺、黃傋口、小溪。皖南肅清,復原官,補甘肅鞏秦階道。以母老假歸。光緒元年,起復廣東督糧道,擢按察使。五年,卒於官。

國琛以儒生治軍十餘年,堅苦踔厲,號為名將。居官亦有政聲。

黃淳熙,字子春,江西鄱陽人。道光二十七年進士,湖南即用知縣,歷署綏寧、會同。剛直為時所忌,引疾閒居。咸豐三年,巡撫駱秉章廉知其賢,使強起之。七年,署湘鄉,有異政。尋丁父憂。鄱陽方陷賊,移家於湘鄉。曾國籓方起督浙江軍,闢參軍事,不就。九年,石達開犯湖南,秉章檄淳熙募勇千六百人防省城,時出剿賊。達開由寶慶竄踞嶺東,分黨犯江華,淳熙破之於掛勾嶺,遂夜襲嶺東賊營,躡至江、藍,殲殪甚眾。進剿賊黨賴裕新,乘霧敗之,破杉木根、黃馬寨而還。十年,達開黨眾四出,淳熙轉戰於永、道、綏、靖諸州,復宜章、桂陽。前後三十餘戰,皆捷,累擢知府,以道員記名。所部曰果毅營,增至三千人。

駱秉章奉命赴四川督師,湘軍名將勁兵多從曾國籓、胡林翼,劉蓉薦淳熙兵精善戰,秉章遂以淳熙與劉岳昭從行。至荊州,岳昭復留,獨淳熙以所部當軍鋒。分道溯峽上,次萬縣,聞順慶被圍,率師赴援。五月,至,賊走定遠,追之,距定遠二十里,望賊屯城西南,連十餘里,城東北江水環繞,賊方造浮橋渡水。淳熙分三路進,遇賊即前搏之,擲火焚其屯,賊大亂爭走,二十餘壘悉潰,擒斬數千。殲首賊何國樑,解散脅從萬餘人。賊黨彭紹福率眾千餘屯東岸,收集潰賊,竄二郎場,淳熙銳進,欲一戰平之。二郎場在山中,羊腸曲逕,通遂寧兩路,均為涪江阻。別賊硃甲眾數千由青岡壩至,四路設伏。淳熙遣偵不見賊,五鼓師行,𥫗賊燕子窠,擊走之,逼二郎場。賊分兩路繞山麓上,淳熙知有伏,令諸營左右搜捕,自率中軍策應。兵分,伏賊起,遍布山岡。官軍走田塍,泥深輒陷。淳熙率親卒拒戰,被圍,策馬突陣,陷淖中,棄馬,手刃十餘賊,中矛僕,擁至場,支解燔之。賊懾其軍勇猛,不復追,餘部整隊還,賊亦遁走。淳熙雖戰歿,湘軍之威因之頓振。詔贈布政使,賜恤,加贈內閣學士,謚忠壯。

吳坤修,字竹莊,江西新建人。捐納從九,分發湖南。道光二十九年,賑湘陰水災,勤於事。從剿李沅發,以府經歷、縣丞補用。咸豐二年,粵匪犯長沙,以守城功擢知縣。曾國籓創立水師,坤修司軍械。四年,水師攻九江,入鄱陽湖,為賊所阻不得出,令坤修單騎往南康,導往吳城、南昌。五年,率舟師防瑞豐。以父憂歸。既而武昌復陷,坤修從羅澤南援湖北,復咸寧、蒲圻、崇陽、通城,累擢同知,賜花翎。進規武昌。

六年,江西軍事不利,胡林翼令坤修領新募軍曰彪字營,會湘軍援江西。復新昌、上高。由新昌取道羅坊攻奉新,梯城而登,賊死守不能拔,乃先下安義、靖安,後萃軍奉新。時江西餉絀,坤修傾家貲,並勸族里富人出銀米餉軍;又籌銀四萬兩解省垣,收集平江潰勇。七年春,克奉新,累擢道員。尋授廣東南韶連道,仍留軍,克瑞州。是年冬,東鄉師潰,被劾褫職。九年,駐師撫州。江西巡撫耆齡檄督辦撫、建、寧三屬團練,始立團防營,駐貴溪。移德興,出援徽州。十年,克建德。秋,徽防軍潰,坤修方假歸,其弟修敳攝軍事,守嶺外郭村。調回江西,曾國籓令守湖口,而巡撫毓科檄援建昌。賊由金谿竄東鄉,坤修自撫州迎擊於鄧家埠,大破之。賊復出貴溪竄安仁,遏之不得渡河,乃竄德興、萬年,將擾景德鎮。坤修由饒州馳援景德,以固祁門大軍後路。會賊由建德上犯,國籓令援湖口。坤修且戰且進,先賊至,城恃以完,加鹽運使銜。

同治元年,李秀成自蘇州援江寧,分犯蕪湖,會軍擊卻之,又會克金保圩、高淳、溧水及溧陽、東壩各要隘,遣散降眾數萬。三年,加布政使銜。江寧克復,以按察使記名。四年,署徽寧池太廣道,授安徽按察使。五年,署布政使。六年,巡撫英翰駐潁州,出境剿捻,坤修轉輸餉運,未嘗遲乏。七年,署巡撫,實授布政使。東捻平,請假回籍補終父母喪。九年,回任。十一年,卒。巡撫英翰疏陳其戰功政績,賜恤,贈內閣學士。

康國器,初名以泰,字交修,廣東南海人。少為吏員。道光末,從軍,以勞授江西贛縣桂源司巡檢。咸豐初,粵匪犯江西,土寇蜂起,國器募死士三百,贛南道周玉衡檄擊賊烏兜、良口,克萬安。造船三十艘,習水戰。六年,從克饒州,累擢知縣,署南城。石達開陷瑞、撫、臨、吉四郡,國器從克樟樹鎮,連戰瑞州、臨江、鉛山、安仁,擢同知。十一年,廣東巡撫耆齡檄剿陽山賊。賊踞藍山,地阻絕,負隅十餘年。國器緣崖歷磵出賊後,破石柵九,奪砲臺,毀其老巢。遣子熊飛單騎說降劇賊練四虎,其魁梁柱走豬頭寨,穴山攻獲之。進軍赫巖,擒賊渠周裕等。藍山平,擢知府。同治元年,援浙,從蔣益澧圍湯溪,明年春,克之,擢道員。三年,克餘杭,功最多,授福建延建邵道,始專統一軍。

粵匪汪海洋犯閩,陷武平、永平,李世賢踞漳州、龍巖與之合,旁郡縣多沒於賊。左宗棠議三路進兵,國器自請當龍巖。進軍雁石,令熊飛壁鐵石洋,三戰薄城下,破其眾數萬,並敗古田援賊。四年正月,遂克龍巖。賊走永定,分踞苦竹、奎洋,勢猶熾。國器進擊苦竹,乘夜大霧,火賊營,破二十餘壘。海洋以悍黨來援,敗之於東阬,又敗之大溪,乃竄廣東大浦境。未幾,海洋復犯永定,國器馳毀羅灘橋;賊分七路來撲,海洋自陣獅龍嶺,所部皆死黨,旗幟遍巖谷。國器曰:「賊精銳盡萃於此,若摧之,餘眾必奔。」乃堅壁深溝,伺怠出擊。先破其伏,分道猛進,斬馘數千,盡獲其軍實,海洋跳而免。時漳州亦下,李世賢西遁,遇國器於塔下,縱兵擊之,降其眾二萬人。海洋走廣東,踞鎮平。國器進壁鎮平東南高思塘,分軍扼程官埠,賊數來犯,卻之。國器知海洋將襲高思而虛攻程官埠,乃戒程官軍勿為動,設伏兩山間。海洋果率悍黨來撲,誘入,伏突起,槍斃其梟汪大力、黃十四,海洋傷腕,陣斃及墮巖磵死者無數。胡瞎子攻程官,亦敗走。尋克鎮平。十二月,會諸軍擊賊嘉應,海洋伏誅,餘孽悉平。

五年,擢按察使。七年,遷廣西布政使。十年,護理巡撫。十一年,內召,以疾歸。光緒十年,卒。左宗棠疏陳戰績,請恤,格於吏議,特詔允之。

國器治軍能以少擊眾,常傷足而跛,軍中號康拐子,悍賊皆畏之。子熊飛,積功至浙江候補道,勇而有謀,常為軍鋒。國器數獲奇捷,實資其力雲。

李鶴章,字季荃,安徽合肥人,大學士鴻章弟。諸生。從父兄治本籍團練,屢出戰有功,以州同用。咸豐十一年,從克菱湖賊壘,復安慶,擢知縣,賜花翎。同治元年,從鴻章援江蘇,常率親兵佐督戰。北新涇、四江口諸役,功皆最。又攻枝福山、許浦海口賊壘,招降常熟踞賊錢森仁。鴻章引嫌,奏捷不敘其勞,特旨詢問,命一體議敘,以知州用,加四品銜。二年,會克太倉,規蘇州。分諸軍為兩路,其進昆山一路,以程學啟為總統;由常熟進江陰者,鶴章督之。迭戰於常熟之王莊,江陰之南漍、北漍、顧山,毀賊壘,破援賊,會克江陰,擢知府。進攻無錫,踞賊黃子隆死守,李秀成屢來援;及蘇州既克,潰賊亦麕聚,鶴章督水陸諸軍力戰克之,以道員記名簡放。詔嘉鶴章:「能與兄同心戮力,為國宣勤。此次未行破格之獎,為鴻章功不自私,俾得報勞將士,鼓舞眾心。指日常州、金陵次第奏捷,克成全功,更當與郭松林、劉銘傳等同膺懋賞。」鶴章進趨常州,與劉銘傳會攻,破援賊,解奔牛之圍。三年四月,克常州,賜黃馬褂,授甘肅甘涼道。是年冬,曾國籓調其軍赴湖北。

四年,以甘肅回亂棘,命赴本任,鶴章以傷發未行。尋疾甚,國籓為奏請開缺,留襄營務。未久,乞病歸,遂不出。以捐助山西賑金,加二品銜。光緒六年,卒於家。曾國荃疏陳:「李鴻章平江蘇,鶴章與程學啟各分統一路。請將戰績宣付史館,於立功地建專祠。」允之。子經羲,官至雲貴總督。

弟昭慶,初從曾國籓軍,淮軍既立,國籓留五營,令昭慶領之,駐防無為、廬江。同治元年,從鴻章至上海,解常熟圍,克嘉興、常州,皆在事有功。四年,國籓督師剿捻匪,昭慶總理營務,統武毅、忠樸等軍。及鴻章代國籓,令赴前敵擊賊,馳逐鄂、皖、東、豫之間,累擢至記名鹽運使。捻匪平,留防江、淮。十二年,卒,贈太常寺卿。

吳毓蘭,安徽合肥人。咸豐十年,粵、捻合擾皖北,毓蘭以從九品偕兄毓芬集團練助剿鳳、潁間,從解壽州圍,擢縣丞。同治元年,李鴻章率師援上海,毓蘭從軍東下,克柘林、奉賢、南匯、川沙、青浦、金山,皆與有功,擢知縣。二年,克嘉定,解北新涇、四江口之圍,加同知銜,領華字副營。擊賊吳江八斥、牛尾墩、同里等處,進克平望、黎里,調守嘉善。三年,率所部從總兵程學啟攻嘉興,戰於合歡橋。毓蘭率槍船冒險渡河,先破賊卡,繞出賊營後,立拔之。進抵城下,賊以巨砲拒河口,學啟被傷,毓蘭率先鋒攻益厲,掘河口架橋濟師,晝夜環攻,轟陷城垣百餘丈。賊死抗不下,賊酋黃文金自湖州來援,力擊走之,遂克嘉興。毓蘭緣梯先登,擢直隸州知州,賜花翎。

調守溧陽,降賊屯城中,勢岌岌,突有金壇賊至,毓蘭與兄毓芬議乘賊初至破之,設伏以誘。賊敗走烏鴉嶺,毓蘭與毓芬兩路夾擊,擒斬無算。窮追至建平境,陣斬賊目林得英、黃有才,擒黃金龍。溧陽既定,調守長興。時大軍已破湖州,毓蘭偵賊將竄泗安鎮,與毓芬夜率健卒八百冒雨疾走,潛渡觀音橋,賊不意兵至,棄糧械而走。追至泗安,降者數千,敘功擢知府。四年,調守揚州,移廬州。五年,回屯揚州。追論平浙西功,以道員選用。

六年,捻匪賴文光敗竄至揚州,為毓蘭所獲,以道員記名簡放。七年,尋加布政使銜。十年,李鴻章調充海防營務處,筦天津機器局。光緒六年,授天津河間兵備道。濱海多盜,毓蘭按名捕置諸法。修南運河、子牙河堤,及千里堤灣,靜海、軍糧城河道,興水利。八年,卒,優恤,附祀曾國籓天津專祠,揚州建專祠。

論曰:金國琛為羅、李舊部。黃淳熙後起,獨立一幟,雖非楚籍,並為湘軍名將。淳熙戰勝殞身,國琛遭忌鎩羽,皆未盡其才。吳坤修、康國器起於令尉,功施爛然。李鶴章才績出眾,堪膺大用,後竟不出。吳毓蘭以擒獲巨憝顯名。功名之際,遭際固難測哉!


\end{pinyinscope}