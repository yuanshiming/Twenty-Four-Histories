\article{列傳十三}

\begin{pinyinscope}
揚古利勞薩子程尼圖魯什子巴什泰覺羅拜山子顧納岱顧納岱子莫洛渾

西喇布子馬喇希阿蘭珠阿蘭珠弟布爾堪納爾察納爾察子瑚沙

達音布朗格朗格子和託從弟雍舜瑪爾當圖瑪爾當圖子烏庫理

喀喇喀喇孫舒里渾洛多歡崆古圖巴篤理穆克譚穆克譚子愛音塔穆

達珠瑚達珠瑚子翁阿岱

揚古利,舒穆祿氏,世居渾春。父郎柱,為庫爾喀部長,率先附太祖,時通往來,太祖遇之厚,命揚古利入侍。郎柱為部人所戕,其妻襁負幼子納穆泰於背,屬鞬佩刀,左右射,奪門出,以其族來歸。部人尋亦附太祖。揚古利手刃殺父者,割耳鼻生啖之,時年甫十四,太祖深異焉。日見信任,妻以女,號為「額駙」。旗制定,隸滿洲正黃旗。

太祖令揚古利守汛鴨綠江,警備嚴密,無敢犯者。伐輝發多璧城,阻水不得進,揚古從之,遂薄城,多所俘馘。歲癸巳,略硃舍裏路、訥殷路,戊戌,略安褚拉?利亂流而濟,庫路,皆有功。歲己亥,從克哈達,揚古利先登,擒貝勒孟格布祿。歲丁未正月,徙蜚悠城曰:「吾儕平居相謂死?,揚古利與扈爾漢率兵三百護行,烏喇以萬人要諸路。揚古利勵?於疾寧死於敵,此非臨敵時乎?」持矛突陣,殺烏喇兵七,敵稍卻。夾河相持,諸貝勒軍總至,大破之。五月,從貝勒巴雅喇等伐渥集部,取赫席黑路,為前鋒。馬兒古里村人驚兵至,走負山,因攻據其山巔,馳下擊之,盡殲丁壯,俘子女以歸。九月,伐輝發,越柵二重,先入,奪其城。歲庚戌七月,從臺吉阿巴泰等伐渥集部,略木倫路,克吳兒瑚麻村,望林中。歲壬子九月,從討烏喇,攻金州城,城中迎射拒,??起,即馳赴之,往復者三,俘獲甚進戰。攻青河,烏喇貝勒布占泰兵?揚古利冒矢攻克之。歲癸丑正月,再討烏喇,揚古利先迫城,聚一隅疾攻,遂拔之。?甚銳,太祖傳矢命諸將退,揚古利持不可,麾

天命四年三月,明經略楊鎬大舉來侵,總兵杜松等攻界凡,大貝勒代善等帥師御之。我軍屯吉林崖,明軍屯薩爾滸山,兩軍相薄,揚古利與貝勒阿巴泰等爭先赴敵,破其軍,松等皆戰死。是夕,明總兵馬林以兵至,營於尚間崖。翌旦,移兵往攻,太祖命被創者勿往,揚古利裹創系腕,率十牛錄兵,憑高馳擊,林兵大潰。七月,攻鐵嶺,遇蒙古貝勒介賽兵,擊破難之;揚古利拔刀揮本旗兵先登?之,遂獲介賽。六年三月,從太祖攻沈陽,壕深塹堅,奪敵所植竹簽以阻軍者,遂克之。進攻遼陽,復先登陷陣,破其步卒,奪河橋,與明兵戰於沙嶺,大敗之。遼陽既拔,太祖嘉其戰多屢受創,命位亞八貝勒,統左翼兵,授一等總兵官,誡勿更臨陣。

十年,揚古利守耀州,明將毛文龍遣兵三百來攻,略城南蕎麥沖,揚古利率兵追擊,盡殲之。旋進三等公。天聰三年九月,同阿山等捕逃人,至雅爾古,遇文龍所部越塞採葠者,擊殺九十六人,獲千總三及其從者十六人以還。十月,從伐明,薄明都,擊敗滿桂兵於城北,砲兵陷敵伏中,揚古利率親軍十餘人奪圍入,悉出之。軍還,從貝勒阿巴泰等略通州,焚舟千餘。攻薊州,明軍來援,太宗督右翼三旗攻其西,貝勒代善等督左翼四旗攻其東,右翼兩紅旗兵少卻,揚古利率正黃旗兵直前突陣,敵敗走。太宗命兩紅旗將佐納鍰自贖,以賜揚古利,揚古利分畀將士,不自私。六年,太宗伐察哈爾,命貝勒阿巴泰等及揚古利居守,明兵來侵,諸貝勒御之。錦州戰,明兵銳甚,六旗俱卻,揚古利大怒,獨率本旗兵奮擊破之。旋復從太宗入明邊,攻大同、宣府,與貝勒阿巴泰等拔靈丘,隳王家莊,取之。

七年六月,太宗諮諸將兵事,揚古利言:「用兵不可曠隔,若逾年不用,敵以其時乘間修備,慮誤我再舉。我暇,一年再徵;不暇,亦一年一徵:乃為善策。我今當深入敵境克城堡,貝勒諸將已痘者駐守,未痘者從上還都。不克,則縱兵焚其村聚;民降者編伍,拒者俘以還。各旗獻俘,視牛錄為多寡,兵士所獲聽自取。若此,則人人貪得,不待驅迫,爭出私財買馬,兵氣益揚矣。戍邊,貝勒許番代,他將卒不許番代。不耐勞苦,豈有能拓地成功業者乎?或謂用兵數,且妨農。婦子相隨,且行且穫,何妨農之有?朝鮮、察哈爾宜且置之,山海關外寧遠、錦州亦當緩圖,但深入腹地。腹地既得,朝鮮、察哈爾自來附矣。」時諸大臣所見亦略同,太宗遂定策伐明。八年五月,復錄揚古利前後戰功,進超品公,位亞貝勒,帽頂嵌珠。

崇德元年五月,命武英郡王阿濟格,饒餘貝勒阿巴泰及揚古利帥師伐明,入邊,克畿內諸州縣凡十二城,五十八戰皆捷,獲總兵巢丕昌等,俘十餘萬。出邊,擊敗三屯營、山海關援兵。九月,師還,太宗出都十里迎勞。獻捷,設宴,親酌卮酒賜三帥。十一月,論伐明諸將違律,阿濟格出邊不親為殿,揚古利坐不諍,罰土黑勒威勒。

十二月,太宗親伐朝鮮,揚古利從。二年正月,師濟漢江,屯江岸,朝鮮全羅、忠清二道兵來援,營漢城南。是月丁未,太宗命豫親王多鐸及揚古利擊之,值雪,陰晦,敵陣於山下,縱兵進擊,自麓至其巔,多鐸鳴角,招揚古利登山督戰。揚古利將馳赴,朝鮮敗卒伏崖側,竊發鳥槍,中揚古利,創重,遂卒,時年六十六。明日,多鐸率兵逼敵營,朝鮮兵已夜遁,得揚古利尸以歸。太宗親臨哭奠,賜御用冠服以殯。喪還,太宗迎於郊,命陪葬福陵。葬日,太宗復親奠。

揚古利初事太祖,凡在行間,率先破敵,沖鋒挫銳,所向披靡。太宗誡不令臨陣,而遇敵忘軀,奮發不自已。行軍四十餘年,大小百餘戰,功業絕特,而持身尤敬慎。太宗嘗命。其葬,以本牛錄八戶守?。是年十?本牛錄護軍為之守門,賜豹尾槍二,以親軍二十人為一月,追封武勛王,立碑墓道。順治中,世祖命配享太廟。康熙三十七年,聖祖巡盛京謁陵,親奠其墓。三十九年,復建碑為文表績。雍正九年,定世爵為一等英誠公。

揚古利子二:長,阿哈旦,以軍功授拖沙喇哈番;次,塔瞻,襲超品公,擢內大臣。崇德六年八月,太宗親將御明洪承疇,戰於錦州,敵遁,命塔瞻設伏追擊,斬獲甚營松山,明總兵曹變蛟夜率兵突近御營,塔瞻不能御,降一等公。順治四年,卒,以其子愛星阿襲,愛星阿自有傳。

勞薩,瓜爾佳氏,世居安褚拉庫。太祖伐瓦爾喀部,取安褚拉庫,勞薩來歸。旗制定,隸滿洲鑲紅旗。天命六年,從伐明,克遼東,授游擊。天聰二年,從伐蒙古多特羅部,進二等參將。三年,從伐明,薄明都,與圖魯什等敗敵德勝門外,斬五十餘級,獲馬數十,進一等參將。八旗選精銳為前鋒,號「噶布什賢」。勞薩驍勇善戰,使為將,號「噶喇依章京」。每出師,前行偵敵,所向有功。五年八月,從伐明,圍大凌河城。上聞明援兵自錦州至,遣勞薩與圖魯什以兵二百偵敵,上與貝勒多鐸以兵二百繼其後。明兵至,逐勞薩等至小凌河,突近上前,上渡河躬陷陣,後軍亦至,共擊敗之。時我將覺善被圍,又有裨將與敵戰,敵揮刃將及,勞薩直前奮擊,悉拯之出,還,白上,上親酌金?以勞。明監軍道張春等合馬步兵四萬,渡小凌河,嚴屯拒戰,勞薩受上指,領纛而前,力戰破敵壘。十月,復與圖魯什往錦州松山偵敵,遇明兵,奔寧遠,斬其執纛者十餘人。

十一月,聞察哈爾兵至,勞薩率兵百偵敵。會察哈爾兵引去,追擊之,逾興安嶺,勿及,甲仗、駝馬委於道者,悉收以還。六年四月,從伐察哈爾,師次博羅額爾吉,勞薩率兵前行,收蒙古流散者二百餘人。五月,與阿山率兵百至喀喇莽柰偵敵,遇察哈爾邏卒,逐而斬之。我國諜者劉哈為敵困,敵兵殆百人,勞薩以七騎大呼破圍入,挾之出,敵披靡敗走。尋偵察哈爾汗棄地,遁已遠,還白上,上乃自布龍圖班師,至枯?,勞薩還與大軍會。

七年,上命勞薩與圖魯什等將三百人略寧遠,分其兵兩翼突入沙河所,斬三百人,獲裨將一、牲畜二百七十。八年二月,復略錦州松山邊境,往錦州投書明總兵祖大壽。五月,與圖爾格率兵出邊,渡遼河,沿張古臺河屯戍,衛蒙古,扼明兵。勞薩屢以寡勝眾,功多,進三等副將,賜號碩翁科洛巴圖魯。十二月,察哈爾部眾來歸,命勞薩將百人迎護。九年四月,從貝勒多爾袞收降察哈爾部眾,師還,略明邊,勞薩夜率兵進敗寧武關兵,遂毀關入,進略代州;復進略忻州,度黑峰口,遇明邏卒四十人,悉擊斬之,獲其馬。

崇德元年,偕吳松等齎書諭明松棚路潘家口諸戍將,因偵敵邊隘,多所俘馘。上伐朝鮮,命勞薩與戶部承政馬福塔以兵三百先為賈人裝,晝夜行,將至朝鮮,其戍將出御,力戰,盡殪其眾。朝鮮國王李倧使勞師郊外,以其間走南漢山城。師還,吏議勞薩備不嚴,使倧得走,當奪世職論罰,上命毋奪職。二年,授議政大臣。三年二月,從伐喀爾喀,上使勞薩齎書諭明宣府將吏歸歲幣、開巿。勞薩獲喀爾喀四十餘人,收其財物、牲畜,縱使去。師還,吏議勞薩罪當死,上特命宥之。八月,從貝勒岳託、杜度伐明,自密雲墻子嶺口入。岳託奏言:「噶布什賢將領勞薩等逐潰兵,得明邏卒,詗知墻子嶺堅不易拔,嶺東西高處可越。」分軍四路深入,明兵合馬步八千人拒戰,阿蘭泰所將蒙古兵稍卻,勞薩與圖賴等奮戰陷陣,明兵敗去,其夜復至,勞薩擊卻之,遂入其壘;又率所部逐敵,斬百七十餘級,俘九十,獲馬百三十有奇,進二等梅勒章京。

五年五月,與吳拜偵敵廣寧邊境,自中後所入,循海而南,斬二百級。上自將攻錦州,勞薩伏兵高橋,縱敵弗擊,論罪,降世職,奪賜號。六年四月,從鄭親王濟爾哈朗伐明,圍錦州,設伏,擊明兵松山,獲馬百九十。勞薩逐明兵,見敵援至,使騎馳問濟爾哈朗曰:「敵援至,若之何?」濟爾哈朗以為怯,聞於上,上曰:「勞薩素勇敢,且身被重創,不當議小過。」五月,明總督洪承疇以兵六萬援錦州,屯松山北,我師未集,勞薩力戰,敗其前鋒。會上命睿郡王多爾袞等濟師,復與戰,大敗之。勞薩行塔山東偵敵,獲敵騎,克錦州外城。九月,命復勞薩世職、賜號。旋代洪尼喀為梅勒章京。是月,上自將督多爾袞等與承疇決戰,勞薩從多爾袞陷陣,力戰,死之。既克敵,上遣內大臣攜酒臨奠,恤贈三等昂邦章京,以其子程尼襲。

程尼既襲職,三遇恩詔,進一等伯,任議政大臣。順治九年,從敬謹親王尼堪征湖南。十一月,及明將李定國戰於衡州,我師敗績,沒於陣,恤贈拖沙喇哈番。十二年,追謚國初以來有功諸將,勞薩謚忠毅,程尼謚誠介,並立石紀績。

勞薩弟羅壁,初以軍功授阿達哈哈番,至是並襲程尼世職,進為二等公。卒,其子降襲一等伯。再傳,無嗣,乾隆間,續封二等子。

圖魯什,伊爾根覺羅氏,世居葉赫。歸太祖。旗制定,隸滿洲鑲黃旗。天命九年,為。擢甲喇額真,授?牛錄額真。蒙古有亡者,逐得之。十年,命率兵至旅順口捕盜,俘獲甚游擊世職。

天聰三年,從太宗伐明,圖魯什先驅偵敵,至大安口,城下兵出戰,圖魯什單騎奮擊,師繼至,克之。自遵化向明都,明兵自薊州踵師後,圖魯什設伏擊卻之。十二月,上軍明都西南,令圖魯什與梅勒章京阿山循城覘敵多寡。獲諜,言明總兵滿桂、黑雲龍、麻登雲、孫祖壽合兵四萬,屯永定門南二里許。還白上,且曰:「敵盛,宜及其不虞,乘夜擊之。」夜三鼓,秣馬蓐食,八旗及蒙古左、右翼兵俱進。圖魯什率所部先馳入敵壘,敵陣亂,師從之,明師遂敗,斬桂、祖壽,獲雲龍、登雲。與勞薩、席爾納等往來游擊,屢有斬馘。四年正月,從貝勒阿巴泰、濟爾哈朗逐斬叛將劉興祚,進二等參將。既,復從貝勒阿敏守永平,諜告明兵且至,圖魯什以四十人偵之,巴篤理、屯布祿等以百人策應,共擊敗明別將張弘謨兵。語詳巴篤理傳。已而,明兵大至,阿敏棄永平引師還。命往視邊墻,率兵五十為三隊,麾使後,獨與四騎先至塞下,蒙古數十人猝起,相薄兩垣間,環而射之,圖魯什突圍出,顧所將騎卒皆陷圍中,一騎中矢且僕,復大呼馳入,援三騎挾傷者俱歸。

五年八月,從伐明,攻大凌河,明援兵二千自松山至,圖魯什與阿山、勞薩等以兵二百迎擊,敗之,斬百餘級,獲三纛。還,上酌金?勞之。九月,攻錦州,明援兵自錦州至,與勞薩從上破敵。語詳勞薩傳。復遵上指,令軍中張旗幟,舉?,偽若明兵來援,致城兵出戰,伏起,敵敗走。明監軍道張春等集諸路軍來援,渡大凌河,屯長山。圖魯什先以偏師邀擊,小勝。戊戌之夕,上親督騎兵襲敵壘,圖魯什先進,兩軍力戰,卒破明師,獲張春。十月,偵錦州松山,斬明兵執纛者。十一月,逐察哈爾兵,逾興安嶺。

六年,從伐察哈爾,次博羅額爾吉,招流亡,皆與勞薩偕。上令哈爾占具糧糗儲烏蘭哈達,而以甲喇額真顏布祿、牛錄額真董山司轉運,愆期,糧糗不時至,吏議當死。上命覆皆言法不當宥,圖魯什言:「曩者上申諭『臨陣而退當斬』,然亦嘗恩宥;今罪顏布?讞,祿、董山而貸其死,實惟上恩。」上從之。

八年二月,略錦州。五月,擢噶布什賢噶喇依昂邦,進三等副將。六月,復從伐察哈爾。七月,至歸化城,遇察哈爾諸宰桑以千二百戶來降,率以謁上。是月,毀明邊墻入大同城,擊敗明總兵曹文詔?,與瑚什布等擊敗明總兵祖大弼軍,略地至宣化,攻懷遠,設伏左城西,使圖魯什如宣府偵敵。閏八月乙酉,遇大弼偵卒十五人,圖魯什單騎馳?軍。上駐左擊,矢中其腹,猶力戰不已,斬二人,俘十三人。圖魯什創甚,上親迎視之。丁亥,卒於軍,賜號碩翁科羅巴圖魯,進三等總兵官。順治間,追謚忠宣。

?子巴什泰,襲爵。事世祖。三遇恩詔,進一等伯。順治九年三月,在上前為蒙古侍瑣尼所戕,進三等侯。子珠拉岱,襲。康熙間,定封一等精奇尼哈番。乾隆元年,改一等子。

覺羅拜山,景祖弟包朗阿曾孫也。景祖兄弟凡六,分城而居,包朗阿次第五,居尼麻喇城。太祖既起兵,族人惎太祖英武,謀欲害太祖,包朗阿子孫獨不與,率先事太祖。太祖起兵之三年,攻哲陳部托漠河等五城,合兵戰於界凡,包朗阿諸孫札親、桑古里皆從。

拜山事太祖差後。旗制定,隸滿洲鑲黃旗。天命六年,從太祖伐明,攻沈陽。明將有?自城南來,拜山迎戰,斬副將一,遂降其?號禿尾狼者,驍悍善戰,拜山殪諸陣。明兵悉。既克遼東,授游擊。天聰元年,從太宗伐明,攻錦州未下,移師攻寧遠。錦州兵潛出躡師後,拜山與牛錄額真巴希競起還擊,戰死。太宗親臨其喪,酹酒哭之,賜人戶、牲畜,贈三等副將。子顧納岱,襲。

顧納岱既襲職,天聰八年,改三等梅勒章京。崇德三年,從伐明,戰於山海關,敗明兵。逐敵至豐潤,師或出採薪,明兵起乘之,顧納岱馳赴奮擊,援以歸。徇山東,擊敗明內監馮永盛、總兵侯應祿,克博平,進一等梅勒章京。

順治元年,顧納岱以擺牙喇纛章京從睿親王多爾袞入關擊李自成。十月,從豫親王多鐸逐自成至陜州,賊依山為陣,顧納岱與圖賴率擺牙喇兵馳擊,斬獲大半。二年二月,自成將劉元亮以千餘人夜覘我師,顧納岱出擊敗之。鑲黃、正藍、正白三旗兵繼進,賊大奔,遂克潼關,逼西安,加半個前程。三月,從豫親王徇河南,渡淮。四月,至揚州,與伊爾都齊等率擺牙喇兵軍於城南,獲舟二百餘。翌日,合師薄城下,七日而拔。進克明南都,溯江至蕪湖,擊明將黃得功,敗其舟師。移師從貝勒博洛徇蘇州,克昆山,攻江陰,發?破城,顧納岱先登。復移師趨浙江,略平湖,水陸並勝,收其戰艦。攻嘉興,明兵出御,背城為陣,顧納岱與固山額真恩格圖、漢岱等合擊之,三戰三勝。七月,師還,進三等昂邦章京。

四年,從豫親王征蘇尼特部,討騰機思,騰機思走喀爾喀,分遣蒙古兵追擊,敗之於歐特克山;復自土喇河西行,敗喀爾喀兵於查濟布喇克。尋以恩詔進二等精奇尼哈番。五年,從征南將軍譚泰下江西,討金聲桓,至九江,擊破聲桓兵;進攻南昌,中砲,沒於陣。贈一等精奇尼哈番,以其子莫洛渾襲。

莫洛渾授參領。順治十七年,從安南將軍達素徇福建,討鄭成功,攻?門,死之。聖祖以拜山、顧納岱、莫洛渾三世死王事,贈莫洛渾三等伯,謚剛勇。

太祖始起,諸族人未附,有龍敦者,為景祖第三兄索長阿子,於太祖為從叔,撓太祖尤力。太祖討尼堪外蘭、討李岱,漏師期,又構太祖異母弟薩木占殺噶哈善哈思虎,皆龍敦所為也。然其從子旺善事太祖。太祖再攻兆佳城,取寧古親,旺善為敵踣,敵俯撲,出刃將刺;太祖未及甲,直入發矢,中敵額,殪,援旺善起。其後屢從征伐。天命十年,偕達珠瑚、。上出郊迎之,行抱見禮,慰諭甚至。?車爾格,以千五百人伐瓦爾喀部,俘獲甚

太祖既盛強,龍敦子鐸弼、托博輝皆從。天命七年,太祖伐明,使鐸弼與貝和齊、蘇把海留守遼陽。太宗初即位,設八大臣,以托博輝領正藍旗。

又有土穆布祿,為景祖幼弟寶實諸孫。十年,命與阿爾代、毛海、光石等屯耀州。太宗設十六大臣,使與薩璧翰為托博輝佐。

又有郎球、巴哈納,皆索長阿之裔,俱致通顯,自有傳。

太祖,授扎爾固齊?西喇布,世居完顏,以地為氏。太祖初起兵,率所部來歸,常翼。歲癸巳,略富爾佳齊,哈達人西忒庫抽矢射貝勒巴雅拉,西喇布以身當之,中二矢,遂卒,恤贈游擊。子二:噶祿、馬喇希。旗制定,隸滿洲鑲紅旗。噶祿襲職,從攻沙嶺有功,進二等參將。卒,無子。

馬喇希,天聰九年,授佐領。尋襲其兄噶祿世職。崇德二年,從都統葉克舒等伐卦勒察。三年七月,授刑部理事官。八月,遷蒙古梅勒額真。四年,再遷固山額真。從睿親王多爾袞圍錦州,坐徇王貝勒等私遣兵歸,離城遠駐,罰如律。復從貝勒阿巴泰等入黃崖口,所至克捷。順治元年四月,從睿親王多爾袞入關破李自成,追擊至慶都。十二月,與都統阿山征陜西,自蒲州渡河擊賊。論功,進一等甲喇章京兼半個前程。尋命移師從豫親王多鐸下江南。二年五月,自歸德渡河至泗北淮河橋,明守將焚橋走,師夜濟,與都統宗室拜音圖以紅衣砲攻克武岡寨,引兵而東。至常州,明將黃蜚以步兵數萬御戰,擊破之,遂下宜興,道破明水軍。至昆山,都統恩格圖等方攻城,馬喇希率所部兵趨頹堞,先登,遂克之,復拔常熟。師還,進三等梅勒章京。

四年八月,從肅親王豪格徇陜西,至漢中。叛將賀珍走西鄉,馬喇希與都統鰲拜分兵,進二等阿思哈尼哈番。五年,睿親王多爾袞出獵,馬喇希坐?馳擊,及於楚、湖,斬馘甚與都統噶達渾等私獵,貶秩。八年,世祖親政,詔復職。再遇恩詔,進三等精奇尼哈番。九年九月,命與定南將軍、護軍統領阿爾津帥師定廣東。十月,命移軍鎮漢中。十二月,復命移軍定湖廣辰州、常德諸路。十一年,卒。

十二年,世祖命追錄國初以來有功諸將,皆視一品大臣,予謚,立碑墓道,於是西喇布謚順壯,馬喇希謚忠僖。

太祖諸將,當帝業未成,?死行間,與西喇布同時易名紀績者,又有扎爾固齊阿蘭珠、梅勒額真納爾察。

阿蘭珠,棟鄂氏,世居瓦爾喀什。父阿格巴顏,與其兄對齊巴顏並為屯長。太祖攻杭佳城,守城者為阿格巴顏妻父,令助守,阿格巴顏不可,曰:「以德誅亂,宜也。吾安能助亂而拒有德乎?」尋與對齊巴顏各率所屬歸太祖。旗制定,隸滿洲鑲紅旗。對齊巴顏子噶爾瑚濟、阿蘭珠皆授牛錄額真,分轄所屬。阿蘭珠旋擢扎爾固齊。從伐烏喇,直前沖擊,人馬皆被創,下馬步戰,遂沒於陣。恤贈三等甲喇章京,以其弟布爾堪襲。順治間,追謚順毅。

布爾堪襲職,授甲喇額真。天聰四年,與武賴、哈寧阿等率精兵百略明邊,獲明諜三,遂渡大凌河,斬四十餘級,俘百六十。八年,重定各牛錄所屬人戶,以新附瑚爾哈百人增隸布爾堪。尋戍牛莊,獲蒙古逃人,進二等甲喇章京。崇德元年,卒。

納爾察,鈕祜祿氏,世居安圖,隸蘇克蘇滸河部。國初來歸,授備御,隸滿洲鑲黃旗。歲戊申,從太祖討烏喇,攻伊罕阿林城,先登克之,擢梅勒額真。後攻沙嶺,不待大軍至,獨進,沒於陣,以長子佛索里襲世職。順治間,追謚端壯。

瑚沙,納爾察次子。初授牛錄額真。天聰六年,從太宗伐明,入大同。與圖魯什等行偵敵,遇明兵四百,瑚沙彎弓躍馬,疾馳入陣,敵皆披靡。略地至崞縣,屢擊敗明兵。崇德三年,從貝勒岳託伐明,與鰲拜先驅,遇明騎兵三百,突出搏戰,瑚沙以八騎擊卻之。遂率左翼擺牙喇兵越燕京,徇山東。明太監高起潛等率兵出御,瑚沙與羅什等連戰皆捷,逐北數十里。上以佛索里不勝任,畀瑚沙襲世職,為噶布什賢章京。六年,從伐明,攻錦州,轉戰松山、杏山間,屢有斬獲。七年,加半個前程。十月,貝勒阿巴泰等帥師伐明,上命瑚沙從,俟師入邊,以軍事還報。八年春,師還,使瑚沙從噶布什賢噶喇昂邦努山等,以兵九。?十人詣界嶺口迎師,俘敵甚

順治初,從入關擊李自成,戰於一片石,瑚沙率本旗噶布什賢超哈當自成將唐通,逐自成至慶都;復從噶布什賢噶喇昂邦席特庫設謀誘敵,夾擊破之。六月,從固山額真葉臣征山西,至汾州,偕甲喇額真道喇、圖爾賽等,擊破自成將白輝。二年,從英親王阿濟格征陜西,克綏德、延安。牛錄額真哈爾漢俄班駐軍南山,為賊所乘,戰死,瑚沙率數騎突入,得其尸以還。自成奔湖廣,追剿至安陸,擊敗自成將邵章,掠其舟以東。至九江口,與席特庫率前鋒二十人破賊壘,逐自成至於九宮山。自成既殕,瑚沙復與甲喇額真蘇拜、希爾根等逐捕餘賊,斬二千餘級,進三等甲喇章京。三年,從肅親王豪格討張獻忠於漢中,擊敗叛將賀迎戰,瑚沙奮擊敗之,肅親王遂殪獻忠。五年,進二等阿達?珍,逐獻忠至於西充,獻忠引哈哈番。

六年,從鄭親王濟爾哈朗略湖南。時明桂王由榔猶駐廣西,其總督何騰蛟守湘潭。師既克長沙,渡湘水攻之,前鋒兵薄城,敵分三門出戰,瑚沙與席特庫力戰,破城西兵,生致騰蛟。明兵潰,遂克湘潭,於是衡州、寶慶、永州、辰州諸郡縣次第皆下。進二等阿思哈尼哈番。九年,擢鑲黃旗蒙古副都統,命與學士蘇納海使朝鮮鞫獄。十一年,兼任工部侍郎。十二年,擢本旗蒙古都統,授議政大臣。十五年,從信郡王多尼下雲南。十六年,從克永昌。十七年,師還。以永昌初下,縱兵入城擾民,降三等阿思哈尼哈番。康熙三年,卒。分世職為二,第五子瑚弼圖襲一等阿達哈哈番,第二子碩伯海襲拜他喇布勒哈番。

達音布,他塔喇氏,世居札庫木。天命三年來歸,隸滿洲正白旗,任牛錄章京。從太或為蒙古誘遁,達音布與楞額禮率兵逐之,?祖征伐,輒為軍鋒,積戰閥授備御。來歸諸部及於達岱塔,擊敗蒙古兵,得逃人以歸。六年,太祖伐明,略奉集堡,達音布先驅,斬諜克敵,進游擊。蒙古扎魯特貝勒昂安嘗執我使畀葉赫,又屢遣兵要我使,攘牲畜。八年,太祖命臺吉阿巴泰等將三千人討之。達音布時為噶布什賢噶喇昂邦,與雅希禪、博爾晉率五十騎先大軍行,乘夜渡遼河,略昂安所轄厄爾格勒,復馳百餘里,逼昂安所居寨,昂安以牛車載妻子率從者二十餘騎出寨。雅希禪、博爾晉麾三十餘騎下馬將搏戰,達音布引十餘騎勒馬立,昂安謀遁,不欲戰,直前沖騎兵,冀突圍出,達音布拒戰,方彎弓注矢,昂安所部乘隙挾短矛?達音布,中其口,墮馬。我兵沖擊,昂安父子及從者盡殪,俘其孥。達音布遂以創卒。師還,予恤,進世職為游擊。

。崇?子阿濟格尼堪,阿濟格尼堪子宜理布,並有傳。第三子岱袞,屢從征伐,授侍德二年,圍錦州,戰死,贈備御。達音布死最烈,子孫貴列爵,順治間賜謚乃不及。

太祖諸偏裨死事者,牛錄額真喀喇,以御劉綎戰死。又有牛錄額真額爾納、額黑乙,將五百人屯深河,與綎戰林中,死之。甲喇額真布哈、石爾泰,牛錄額真朗格,從太祖攻沈陽,既下,明總兵陳策等來援,與戰,陷陣死。瑪爾當圖從太祖圍錦州,戰死。喀喇、額爾納、額黑乙死時,太祖方草創,未有恤贈。布哈贈參將,石爾泰、朗格贈游擊,而瑪爾當圖死時已授游擊。朗格子和託、瑪爾當圖子烏庫理事太宗,喀喇孫舒里渾、洛多歡、崆古圖事世祖,皆有戰功,賞延於世。

朗格,棟鄂氏,對齊巴顏子,阿蘭珠弟也。對齊巴顏來歸,語見阿蘭珠傳。戰死,得世職,以長子棟世祿襲。旗制定,隸滿洲鑲紅旗。

和託,其次子也。崇德七年正月,授本旗梅勒額真。從鄭親王濟爾哈朗等伐明,圍錦州。明總兵祖大壽以其城降,遂進克塔山。鄭親王籍所俘獲,令和託還奏。上命分賚軍中死傷將士,並令齎敕撫明杏山守將,曰:「汝以善言招之,降則已;否則以砲攻,砲發而彼降,亦可許也。」和託至軍,如上指宣示,砲發,明將降。師還,得優賚。旋追議諸將徇部卒失律,和託當罰鍰,以前勞得免。十月,從貝勒阿巴泰伐明,自界嶺口毀邊墻入,至黃崖口。軍中議分兩翼夾攻,輔國公斐洋古令和託督左翼,建雲梯攻城。和託周視畢,復曰:「城可登,無以梯為也。」乃率巴牙喇兵四十人毀城入,斬守備一,餘悉潰。復合右翼圍薊州,擊敗明總兵白騰蛟、白廣恩,遂徇山東,克兗、萊、青諸府。明年,師還,授吏部參政、兼梅勒額真。

順治元年,從入關擊李自成,予世職牛錄章京。上遣侍郎王鰲永招撫山東,明副總兵楊威據登州。鰲永請兵,上命和託與梅勒額真李率泰、額孟格帥師討之。鰲永至青州,為降將趙應元所戕。和託等師至,牒巡撫陳錦、總兵柯永盛會師逼青州。應元復請降,和託與李率泰計許之降,遣兵夜捕斬應元及其黨數十人,宥脅從勿誅,青州遂定。錦亦下登州。上命和託與李率泰移軍河南,會豫親王多鐸下江南,賚黃金、紫貂,進世職三等甲喇章京。二年,從貝勒勒克德渾徇浙江,定杭州。明將方國安以兵至,和託將左翼御之富陽,斬副將二、參將二、游擊五,國安兵大敗。復破敵下關直溝,毀其木城。上命和託與梅勒額真珠瑪喇率所部滿洲、蒙古兵駐防杭州。三年四月,卒。

雍舜,對齊巴顏從子,授牛錄額真。英果,戰輒當前鋒。累擢鑲紅旗固山額真。天聰三年,從上伐明,圍遵化,率本旗兵攻城西南,克之。四年,從取永平,授二等參將世職。貝勒阿敏棄永平還師,雍舜獨贊其議,坐罷官,奪世職,籍沒。七年,從貝勒岳託舟師攻旅順,明將黃龍城守,師克之。論功,先登崖者,巴奇蘭、薩穆什喀;先登城者,雍舜、珠瑪喇:復世職。崇德二年,從克皮島,擢梅勒額真。四年,從征索倫,設伏敗敵,進一等參將。六年,從攻錦州,戰墜馬,得他騎,引本旗兵趨左翼;及右翼勝,乃馳擊,爭赴敵。坐欺謾論罪,命寬之,解梅勒額真。順治初,遇恩詔,進二等阿思哈尼哈番,復官固山額真。卒。子庚圖,先以功授拜他喇布勒哈番,同為一等阿思哈尼哈番。

瑪爾當圖,扎庫塔氏,先世居和克通吉。太祖時,率百餘人,授游擊。從攻錦州,戰死。

子烏庫理,年十六,即從征伐。太宗命領甲喇額真,襲瑪爾當圖世職。崇德三年,從貝勒岳託伐明,略山東,明太監馮永盛以兵至,擊敗之;攻濟南,雲梯兵未至,烏庫理攀雉堞先登,麾所部兵畢上,克其城。師還,將出塞,與白奇超哈統將薩穆什喀殿,敵不敢逼,道經太平寨,復步戰敗敵。七年,從伐明,復攻錦州,戰於松山,敵敗走,旋合潰兵屯北山,壘甚固,烏庫理直前擊之,三戰皆捷。

順治初,入關,從固山額真葉臣攻太原,率十騎繞城周視,城兵驟出搏戰,烏庫理與。尋從英親王阿濟格定陜西、湖廣、江西諸省。師還,至池?甲喇額真薩璧圖奮擊,俘馘甚州,偵明將黃斐,擊之,得舟十二。還京師,授兵部理事官,加半個前程。三年,從肅親王豪格下四川,討張獻忠,敗其將高汝礪;逐獻忠,再破之。五年,從討叛將姜瓖,攻寧武關,所署巡撫姜輝、總兵劉惟思以三千人赴援,內外兵夾擊。烏庫理率三旗巴牙喇兵轉戰關下,瓖兵萬餘陣以待,烏庫理擊破之,復發砲克其城?及偏關西河營,七戰皆勝。師將至左,殪瓖兵。八年,進一等阿達哈哈番。十年,鄭成功寇福建,命與理事官額赫理率禁旅及江?寧、杭州駐防兵濟師,至海澄,敵以火器守隘,烏庫理連破其壘。敵毀橋,烏庫理躍馬先涉,敵驚潰,師乃畢渡;敵又以三千餘人屯海岸,烏庫理步戰敗之。先後與固山額真金礪等,劘敵寨數十,降其兵數千人,復加拖沙喇哈番。

十二年,授大理寺卿,疏言:「滿洲士卒歲從征討,市馬制械皆自具,其孥留京師,請恩賚。行軍所至,民多失所,雖被旨賑貸,當安輯,俾自為生計,請?部議便宜。綠旗死事將卒,請下所司贍其妻子。江、廣、閩、浙濱江、海,盜賊出沒,請敕諸省督撫,要隘設重兵。西北厄魯特、俄羅斯諸部尚阻聲教,請敕理籓院議互市條例,通貿易。」所陳凡五事,皆下部議行。

尋命視黃河決口。十三年,授漕運總督。十七年,授盛京總管。康熙元年,改總管為將軍,仍以命烏庫理。是時,盛京置戶、禮、工三部,烏庫理請增設刑部,廷議如所請。四年,卒。析世職為二,長子俄謨克圖,襲三等阿達哈哈番;次子佛保,襲拜他喇布勒哈番兼拖沙喇哈番。

喀喇,棟鄂氏,先世居瓦爾喀。當太祖時,以其族來歸。從征伐有功,授牛錄額真,賜號「巴圖魯」。天命四年,御明總兵劉綎,力戰,被七創,以傷卒。

子扎福尼。天聰四年,從伐明,攻灤州,有三卒為敵所得,扎福尼陷陣援之出。以功,予世職備御。八年十二月,從白奇超哈統將巴奇蘭等伐黑龍江,加半個前程。

舒里渾,扎福尼子也。初以巴牙喇壯達從軍。從攻大凌河城,敗蒙古軍。及扎福尼卒,襲世職。順治二年,從英親王阿濟格西逐李自成至延安,七捷。自成走湖廣,以師從之,次安陸,得舟十四。三年,從豫親王多鐸北討騰吉斯,力戰,多俘馘,擊敗喀爾喀土謝圖汗、碩類汗。師還,授牛錄額真。六年,從端重親王博洛西徇大同,擊敗姜瓖所署巡撫姜建勛等。十一年,擢巴牙喇纛章京。十五年,授正黃旗滿洲梅勒額真。從信郡王多尼南征雲南,戰涼水井,敗明將李成蛟;戰雙河口,敗明將李定國。師還,進三等阿思哈尼哈番。十八年八月,卒。

洛多歡,舒里渾弟。從軍,取旅順,圍錦州,皆有功。崇德七年,從貝勒阿巴泰伐明,克順德府,先登,賜號「巴圖魯」。累進世職至一等阿達哈哈番兼拖沙喇哈番。

崆古圖,亦舒里渾弟也。順治間,從靖南將軍陳泰徵福建,克興化府,先登。自巴牙喇壯達擢甲喇額真。十七年,洛多歡卒,襲世職。康熙十三年,從副都統雅賚、阿喀尼等討耿進忠,自安慶向江西,敗賊小孤山,復彭澤、宜黃、崇仁、樂安諸縣。十五年,移師討吳三桂,攻萍鄉,敗其將夏國相,師下湖南。十八年,戰楓木嶺,敗其將吳國貴,復武岡。二十四年,卒。子多博海,襲。

特爾勒,舒里渾孫也。康熙間,從征南大將軍賚塔討吳世璠,敗其將何繼祖,奪石門坎、黃草壩;乘夜拔嵩明、丹城,遂克雲南。又從都統希福逐馬寶,破胡國柱。以功,予世職拜他勒布喇哈番。卒。

太祖嘗為故勛臣雅巴海祈天:「乞轉生朕家!」又為布哈孫、朗格等八人祈曰:「宥其微失!」太祖未舉兵以前,有族難,侍者帕海死之,似即雅巴海。布哈孫等事不著。

巴篤理,世居佟佳,以地為氏。天命初,與其弟蒙阿圖來歸。太祖命編所屬為二牛錄,隸滿洲正白旗。太祖察巴篤理才,使為扎爾固齊。積戰功,授游擊。十?,使兄弟分領其年,明發兵航海至旅順,繕完故城,駐軍以守。巴篤理從貝勒莽爾古泰攻之,城下,盡殲明兵。十一年,明將毛文龍遣兵夜襲薩爾滸城,城兵砲矢交發,明兵退,結營。巴篤理率兵自山而下,大呼乘敵,敵潰走,追斬二百餘級。

天聰三年,從伐明,克遵化有功。太宗親酌金?勞之,進二等參將。四年正月,從貝勒濟爾哈朗守永平。三月,明將張弘謨率兵來侵,甲喇額真圖魯什以四十人先,巴篤理與噶布什賢噶喇昂邦屯布祿以百人繼。伏起,屯布祿敗走,巴篤理與圖魯什殿,力戰,其弟課約馬著矢且踣,巴篤理斬敵兵,奪馬授其弟,殪三十餘人,敵乃退。五月,明兵圍灤州,貝勒阿敏守永平,不即赴援,城垂破,乃遣巴篤理率兵赴之,乘夜突圍入城。方議並力堅守,敵發巨砲焚城樓,守將納穆泰等度力不能支,棄城依阿敏,阿敏亦棄永平東還。廷議諸將罪,以巴篤理突圍赴援,釋勿論。

五年,授禮部承政。六年,使朝鮮,定職貢額數。八年八月,太宗自將伐明,巴篤理從,至應州,命與貝勒阿巴泰等取靈丘縣王家莊。巴篤理督軍攻堡,既被創,猶奮擊,中流矢,卒。太宗聞之泣下,曰:「此朕舊臣,轉戰數十年,?命疆場,深可惜也!」恤贈三等副將。順治十三年,追謚敏壯。子卓羅,自有傳。

蒙阿圖,自牛錄額真累擢梅勒額真,坐私立屯莊,罷。天聰三年,從伐明,敗敵於遵三千。逾年師還,上自出郊宴勞。授游擊世職,擢工部承政?化。尋命帥師伐瓦爾喀,俘其。崇德三年,以老解職,召見,諭之曰:「爾等舊臣,朕見之輒心喜,可不時來見也!」未幾,卒。

國初諸將,事太祖創業復佐太宗從征伐而戰死者,勞薩、圖魯什功最高,巴篤理、穆克譚、納爾特與相亞,達珠瑚為俘所賊。順治中,皆追謚。納爾特事具其父雅希禪傳中。

來歸,授牛錄額真。從太?穆克譚,戴佳氏,世居杭澗,隸哈達。穆克譚從其父兄率祖征伐,戰必陷陣,攻則先登,賜號「巴圖魯」。有查海胡色者,叛太祖歸哈達,穆克譚從其父兄追之,戰,其父兄皆死。從子厄爾諾亦叛歸哈達,穆克譚單騎逐斬之。旗制定,隸滿洲鑲藍旗。天命元年,從伐瓦爾喀,戰敗,諸將孟庫噶哈皆走,舒賽、阿爾虎達將為敵得,穆克譚與燕布裏等八人沖敵陣,援之出。師還,太祖譴孟庫噶哈,奪所獲畀穆克譚。六年,叛去,我師追之,戰不利,穆克?從伐明,攻耀州,先登,克之,命戍焉。蒙古人海色與其譚策馬大呼,直前刺殺海色,餘悉潰。以功授二等副將。太宗即位,各旗設調遣大臣,以穆克譚佐本旗。天聰元年四月,從伐朝鮮。六月,阿山、阿達海兄弟叛,將歸明,貝勒阿敏夜帥師追之,穆克譚從,射阿達海,阿達海力戰,抽刀斫穆克譚墜馬,幾殆,卒挾以俱還。五年,從伐明,圍大凌河,穆克譚以本旗兵從固山額真宗室篇古當城西南。城兵出挑戰,圖賴先進,穆克譚從之,薄壕,舍騎步戰,將迫敵入壕。城上?矢競發,城兵續出,奮拒力戰,歿於陣。太宗惜之,曰:「穆克譚我舊臣,不值於此畢命也!」贈一等副將,世襲。順治間,追謚忠勇,立碑墓道。子愛音塔穆。

愛音塔穆襲父爵,兼領穆克譚舊轄牛錄,益壯丁五十。順治初,從入關破李自成。旋從豫親王多鐸徇河南,與梅勒額真沙爾瑚達屢敗賊,逐賊至潼關,為殿,賊自後來襲,三至三卻,愛音塔穆功也。二年,河南既平,從定江南。六年八月,從鄭親王濟爾哈朗下湖廣。時明桂王由榔駐武岡,湖南諸郡縣半為明守。愛音塔穆帥師自長沙而南,克寶慶,擊馬進忠、王進才皆有功。自成將劉體純與其黨袁宗第等屯洪江為十寨,緣沅江拒守。愛音塔穆與尚書阿哈尼堪督軍渡江,連破賊寨,賊潰,遂與阿哈尼堪駐守沅州。十二月,賊將王強來犯,與阿哈尼堪共擊卻之。九年,遇恩詔,累進二等精奇尼哈番。十一月,從靖南將軍珠瑪喇四萬,列象?,據山峪,方相?略廣東,時明將李定國攻新會,平南王尚可喜赴援,定國有持。愛音塔穆等師至,合擊大破之,逐北二十餘里,定國遁去。十二年閏五月,論功,進一等精奇尼哈番。康熙十九年,卒。

子公圖,襲。三十五年,從撫遠大將軍費揚古征噶爾丹昭莫多,戰勝,進三等伯。子永泰,降襲二等精奇尼哈番。乾隆元年,改一等子,世襲。

達珠瑚,兆佳氏,先世居訥殷。祖達爾楚,國初來歸。旗制定,隸滿洲正藍旗。達珠瑚初任牛錄額真。從太祖伐烏喇,斬級四千。從克西林屯,俘其人以歸;追者至,還擊敗之,斬級五千。從伐葉赫,斬級三百,俘五十人。遇明人越境採參,斬三十人,俘六人。敵侵寧古塔,出戰,斬其將及兵百,獲甲百副、馬三百匹。授三等副將。天命十一年,伐東海瓦爾喀部,又伐卦爾察部,皆有功。太宗即位,設十六大臣,伊遜及達珠瑚佐鑲黃旗。天聰元年,太宗伐朝鮮,克義州,留兵駐守,命達珠瑚分將之。旋復帥師伐瓦爾喀。師還,為俘卒所賊。八年,以其子翁阿岱襲三等梅勒章京。太宗復遣將伐瓦爾喀,因誡之曰:「前遣達珠瑚,以?見害。念其從事久,有勞,方令襲世職。汝曹未能如達珠瑚之功,儻不自慎,欲覬例外恩,不可得也。」順治間,追謚襄敏。

翁阿岱襲職為甲喇章京。從伐虎爾哈,加半個前程。累遷都察院參政、正藍旗梅勒額真。時方攻錦州急,命與梅勒額真多積禮帥師屯戍,譏逋逃。崇德六年,從圍錦州,與明總督洪承疇戰,屢勝。尋進攻松山,力戰,沒於陣。賚白金千兩,進一等梅勒章京。無子,以弟之子濟木布襲。康熙間,降襲一等阿思哈尼哈番。乾隆元年,改一等男,世襲。

論曰:國之將興,必有熊羆之士,不二心之臣,致身事主,蹈死不反顧,乃能拓土破敵,弼成大業。揚古利負大將才略,功視額亦都、費英東伯仲間;勞薩、圖魯什驍勇冠軍,戰必將選鋒陷陣;若拜山三世?忠,西喇布、達音布、巴篤理等以死勤事,亦其亞也。觀太祖祈天之語,惓惓於舊將;太宗以達珠瑚為戒,又以恭袞不從令,雖陣亡,猶付吏議。其申軍律,惜將材,恩威兼盡,開國基於是矣。


\end{pinyinscope}