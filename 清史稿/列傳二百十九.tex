\article{列傳二百十九}

\begin{pinyinscope}
蕭啟江張運蘭唐訓方蔣凝學陳湜李元度

蕭啟江,字濬川,湖南湘鄉人。少賈於蜀,後始折節讀書。咸豐三年,入塔齊布軍。四年,從平岳州,克武昌、漢陽、興國、大冶、蘄州,敘縣丞,晉秩州同。五年,廣東賊犯湖南,湘撫駱秉章檄啟江募兵協剿,曰果字營,自是獨將一軍。攻茶陵踞匪,率壯士數十人薄南門,賊自民廛躍出,攢矛環刺,啟江手擒數賊,賊莫敢逼。尋會克其城,賊走江西,陷弋陽、興安。啟江偕羅澤南復兩城,進收廣信,賜花翎,擢同知。

六年,劉長佑援江西,總統諸軍,啟江屬焉,駐師瀏陽。賊陷萬載,啟江大破之櫧樹潭、大橋、竹阜,遂復萬載;而崇通賊復犯瀏陽,援賊大至,撲營,啟江鏖戰敗之,躡至八角亭,毀其壘。會曾國華馳至,偕由洪塘、新昌、上高搗瑞州。前軍至登龍橋,擊退袁州賊,進攻新昌、上高,拔之,擢知府。進攻袁州,啟江與長佑分地扼賊。長佑攻西南,啟江攻東北,盡平城外賊屯。城賊惶懼,啟江策臨吉賊必來援,設伏敗之,盡奪其輜重。尋破賊合山,克分宜,加道銜。進攻臨江,七年正月,大捷陰岡嶺,斬其酋。賊勢以孤,乃潛約撫、建、新淦援賊趨太平墟,犯長佑營。長佑戰失利,營陷,賊乘勢回犯陰岡嶺。部將田興恕、楊恆升突陣,斬悍酋數人,師從之,賊崩潰,夷其壘四十七。城賊窮蹙乞降,而賊首仍負嵎死拒,乃誘其出戰,啟江揮軍疾進,薄城而登,遂克之,擢道員,加按察使銜。長佑尋以病歸,劉坤一代之。啟江與進攻撫州,連下宜黃、崇仁。撫州賊屯樟樹鎮,將伺官軍渡贛襲臨江,啟江與坤一回擊,大破之。進次上頓,距撫州十五里,築壘甫畢,賊至,迎擊敗之。進攻高橋,賊棄城遁,追斬千七百有奇。撫州復,加布政使銜。

九年,賊陷南安,糾眾數萬犯贛州,踞新城墟及池江諸地。時曾國籓督援浙軍,啟江率所部從,檄援贛州。啟江遣田勇三千誘賊,賊爭出赴利,啟江摧鋒直進,斬級數千。田勇者,江西募農夫防賊,貪鹵獲,倚湘軍無所畏,集者四萬。啟江曰:「眾而不整必敗。」禁之不可,遇伏果潰。湘軍為少卻,部將胡中和力戰斷後,復進敗之,平新城墟、池江、小溪、鳳凰城諸賊壘,賊退入南安。南安故有二城夾水,賊分屯相犄角,軍至皆棄而走。啟江進屯城外青隴、黃隴,結壘自固,令曰:「入城者斬。」有頃,賊果還南城,攻之,敗走。啟江曰:「賊狡而弱,吾直驅之耳!」攀堞以登,賊奪西門走,追殺數十里,賜號額埒斯圖巴圖魯。進信豐,會總兵遮克敦布攻吳家嶺,啟江率中營進。賊萬餘來撲,擊敗之,破先溪橋賊壘,城兵出而夾擊,立解其圍。時江西郡縣皆復。

石達開由崇義竄湖南,郴、桂所屬皆告警,啟江馳防。賊已由永州竄圍寶慶,啟江自臨、藍趨永州,扼東安,屯白牙市。劉長佑、李續宜解寶慶圍,追至白牙,啟江會軍夾擊,擒其酋楊家廷、馬繼昌於陣。賊竄入廣西,陷興安,盡集悍黨大溶江遏追師,遣別賊直犯桂林。啟江由全州趨興安,復其城;攻大溶江,大捷,解桂林圍,以按察使記名。移軍回湖南。

四川軍事急,命啟江率師往援。十年春,甫至,以疾卒於軍。詔贈巡撫,從優賜恤,謚壯果,湖南、江西並建專祠。其所部留四川,駱秉章用以平賊焉。

張運蘭,字凱章,湖南湘鄉人。咸豐初,從王珍轉戰衡、永、郴、桂,積功擢同知。六年,戰通城,運蘭設三伏,營前斬賊酋張庸忠,擒魯三元,克通城,又大破賊於崇陽白蜺橋,賜花翎。七年,從王珍援江西,迭捷於臨江、吉安、樂安、新城、廣昌,功皆最。王珍卒於軍,運蘭與王開化分領其眾。吉安賊窺永豐,運蘭屢敗之,擢知府。又破賊於峽江橋阜灘、獅子山。移軍吉水,扼賊三曲灘,相持數日,血戰十數次,斬賊渠黃錫昆。渡贛江,破石達開於硃山橋,達開焚屯而遁,遂解永豐圍,擢道員。八年,略定樂安、宜黃,逼建昌,敗賊於厚坪。破水南賊巢,分剿南源、里塔墟、劉家坑,直搗謝坑,毀賊壘,斬其酋廖雄篙等,復南豐。建昌之圍始合,五月,克之,加按察使銜。賊復犯南豐,擊走之,追及新城杭山,降賊眾數千。

時詔起曾國籓督師規浙江,國籓行次江西,賊已入閩,疏調運蘭及蕭啟江率所部從。會賊陷安仁,別將失利,運蘭進擊,大破之,殲賊數千,克安仁,賜號克圖格爾依巴圖魯。由杉關進剿破賊順昌,回援景德鎮,戰於李村,斬馘二千餘,解散千計。九年,援饒州,敗賊於慄樹山,克浮梁,加布政使銜。

是年秋,粵匪犯湖南寶慶,運蘭回援,疊破賊於宜章、星子、市禾洞,追至廣東連州,破九陂、石塘、白虎墟賊巢,殄賊逾萬,授開歸陳許道。十年,曾國籓軍祁門,運蘭偕鮑超破賊黟、歙。十一年,克休寧,擢福建按察使。再復黟縣,盡夷賊壘。時運蘭統五千人防徽州,尋移防寧國,值大疫,悍賊麕集,與霆軍力拒之。同治元年,拔旌德。二年,命援廣東,搗陽山石塋賊巢,降其眾三千,擒巨酋李復猷於連州。

三年,赴福建按察使任。時江、浙逸賊眾猶十餘萬,由江西入閩,蹂汀、漳二郡。運蘭率五百人趨武平,遇賊,眾寡不敵,總兵賀世楨、王明高,副將雷照雄皆戰歿;運蘭被執,罵賊,支解之。事聞,贈巡撫,予騎都尉世職,謚忠毅。武平及湖南、廣東建專祠。

唐訓方,字義渠,湖南常寧人。道光二十年舉人,大挑教諭。咸豐三年,曾國籓創水師,訓方領副右營,嗣改入陸軍。從羅澤南克蒲圻,復武昌,又從攻興國金牛堡。國籓命募常寧勇五百人統之,曰訓字營。從克田家鎮、蘄州、廣濟,拔黃梅,進軍濯港,敗悍酋羅大綱。是夕,賊謀襲大營,訓方巡營驚覺,賊退走。明日,攻孔壟街口,訓方率壯士踏肩陟高墉,諸軍乘之,遂破孔壟。

五年,從澤南援江西,克弋陽、興安、廣信、德興、浮梁。援義寧賊屯城外雞鳴、鳳凰二山,與城犄角。訓方逼雞鳴山下,督隊先登,賊驚潰,乘勝拔其城。從澤南援武漢,克蒲圻,進攻武昌。累擢知府,賜花翎。六年正月,率三百人夜由占魚套至藕塘,奪二壘,又破援賊於豹子海。會襄陽土匪高二倡亂,圍府城,巡撫胡林翼令訓方偕舒保馬隊往剿。破賊於峪山,援賊至,又敗之。進克樊城,追至呂堰驛,斬女賊宋氏。援宜昌,破賊於南漳,權襄陽知府。七年二月,川匪劉尚義犯宜城,揚言趨荊門,而使南漳賊襲府城,訓方備之,急扼武安堰,賊奔據武安城,進攻之。會都統巴揚阿來招降,訓方進剿高二於璩灣,乘雪夜進攻,擒之;而巴揚阿所撫賊復叛,掠鄖、房、保山、竹山、竹谿、保康、興山。訓方會陜西軍連破之武當山金頂,斬其渠,餘賊降。襄郡悉定。先以克武漢論功以道員記名,至是加按察使銜,授湖北督糧道。

陳玉成合捻匪犯蘄、黃,訓方自襄陽赴援,連戰敗賊,進屯張家塝。胡林翼令於蘄州境內建碉卡,訓方以二千人守之,賊迭來攻,皆擊退,賜號奇齊葉勒特依巴圖魯。調援臨淮。尋以李續賓軍覆三河,回防湖北,屯陳德園。九年,會攻太湖,賊圍鮑超於小池驛,多隆阿不能救,令訓方移軍近鮑營為接應。甫至,築壘未就,為賊所乘,乃退屯新倉。十年,解軍事,赴糧道任。未幾,連擢湖北布政使。十一年,胡林翼駐軍英山,病甚。賊上犯黃州,抵灄口,武昌震動,訛言繁興。訓方處以鎮靜,誅亂民數人,人心始定。灄口賊亦擊退。

同治元年,安徽巡撫李續宜因母喪奪情,請假回籍,舉訓方自代,命暫行署理。苗沛霖反側久,遂叛,安徽諸軍皆不能制。二年,僧格林沁大軍至,始平之。撫循降圩,收其兵械,奏移鳳臺,治下蔡雉河集,增立渦陽縣。都統富明阿奏劾訓方,降調。三年,署湖北按察使,尋署巡撫,授直隸布政使,兼統練軍出省防剿。七年,西捻平。請開缺省墓。光緒三年,卒於家。湖北請祀名宦祠。

蔣凝學,字之純,湖南湘鄉人。咸豐初,在籍治鄉團。五年,從羅澤南克武昌,獎國子監典簿。六年,率湘左兩營從巡撫胡林翼攻武昌。屯賽湖堤,引江水入湖,合長圍,進薄城下,平賊壘十餘。武昌復,論功擢知縣。從克黃州、大冶、興國,逼九江。七年,分統三營屯北岸陸家嘴,攻小池口,屢戰皆捷。都興阿檄攻童司簰。童司簰背江據湖,通黃梅要隘,賊五六萬踞之。至則賊數搏戰,凝學堅持不退。尋陳玉成來援,眾議退兵,凝學曰:「童司簰不克,水師往來失所據,九江之師亦掣肘,勢所必爭。」請增兵千人,宵濟合水師,連日鏖戰,破之,平賊壘數十,進克黃梅,擢同知。八年,會攻九江府城。凝學穴地道迤東而南,地雷發,壞城垣百餘丈,從缺口入,殲賊甚眾,擢知府,賜花翎。連復麻城、黃安,擢道員。

十月,李續賓三河軍覆,官文檄凝學間道遏剿。會多隆阿、鮑超擊賊於宿松花亭子,破之。賊退太湖、潛山,凝學駐防荊橋。九年,移屯黃州羅田,會攻太湖。十二月,陳玉成大舉來援,凝學移軍龍家涼亭,與鮑超小池驛之軍為犄角,留四營遏太湖東門,城賊出,擊退之。十年正月,鮑營被圍急,凝學進援,甫拔營,賊大隊來抄,凝學揮軍截擊,多隆阿率馬隊應之,戰竟日,擒斬二千餘。乘勝攻羅山,沖賊壘,諸軍合擊,賊大潰,加鹽運使銜。十一年,陳玉成復犯湖北,凝學回援武昌縣,敗賊赤壁山下,復其城。會總兵成大吉等攻黃州數月不下,詔降賊目劉維楨,復蘄州,選出眾五百人為忠義營,使維楨詐稱援軍,誘城賊出,擊之,遂克黃州,以道員記名,加布政使銜。

苗沛霖叛,陷壽州,凝學進屯六安,克霍丘,增募水陸軍。苗黨姚有志、潘塏等乞降,各圩多反正,授甘肅安肅道。同治元年,移屯潁州。二年,粵匪李世賢北竄,凝學移軍舒城,擊敗之,又追敗之六安,賊引去。苗沛霖復圍壽州,凝學回援,破賊於牛尾岡。壽州尋陷,凝學坐救援不力,褫布政使銜,仍駐防潁州。會僧格林沁督師剿沛霖,凝學克霍丘各圩,水師分駐三河尖、臨淮關,進破黃梁集,克潁上,收附近城各圩,斬賊黨苗呆和、苗呆花,復懷遠。沛霖勢日蹙,遂走死。

三年,粵匪陳得才等糾眾三十萬自陜西回竄,圖救江寧。凝學屯英山,遏賊金家鋪,敗之。賊復自麻城犯霍山,凝學退石家嘴,與按察使英翰相犄角,伺賊過狙擊,殪千餘人,拔出難民數千。英山解嚴,復布政使銜。進援湖北,收復羅田、蘄水、麻城三縣,解蔡家河圍。賊復竄安徽,凝學躡追,沿途襲擊,繞出賊前,遏之霍山長嶺庵。路險,賊不虞兵猝至,多墜澗死,降者三四萬,賊首陳得才仰藥死。簡降眾為步隊五營、馬隊三營,餘悉遣散。

是年冬,陜甘總督楊岳斌奏調凝學赴甘肅,行次樊城,會霆軍譁變。凝學所部亦以欠餉不靖,請於巡撫鄭敦謹,借款資遣湘左八營,留忠義營於湖北,自請回籍養病。命兩月假滿仍赴甘肅。五年,募湘勇二千,號安字營。至西安,巡撫劉蓉奏請凝學屯涇州,兼顧關隴。六月,敗回匪於華亭,與提督雷正綰、總兵張在山等約夾擊,深入被圍,士卒死傷七百餘人,總兵周太和、周清貴,副將黃德太等均歿於陣。凝學潰圍出,屯平涼,轉戰而前,至省城,署蘭州道。六年八月,回匪犯蘭州,守城兵僅凝學所部千餘人,登陴固守,屢出奇兵焚賊壘,賊尋退,以按察使記名。八年,署按察使。九年,復署蘭州道,擢山西按察使。光緒元年,遷陜西布政使。四年,以病解官,未行而卒。賜恤,贈內閣學士。

陳湜,字舫仙,湖南湘鄉人。咸豐六年,曾國荃赴援江西,招湜襄軍事,從克安福、萬安。七年,進圍吉安。國荃奔喪去軍,湜代領其眾。尋以父憂歸。八年,從蔣益澧援廣西,克平樂。賊趨桂林,湜率四營遏之於大灣車埠,敗之,乘勝劃蘇橋壘。從攻柳州,克潯州。九年,石達開圍寶慶,湜募千人出祁陽赴援,與李續宜夾擊破之。十年,曾國荃圍安慶,使湜總軍事。湜規地形,請堨樅陽口蓄水阻援賊,力扼集賢關,從之。賊酋陳玉成來援,阻水,趨集賢關,擊破之。十一年,克安慶,自是獨領一軍。循江而東,會克諸城隘,累擢至道員。

同治元年,從國荃攻江寧,建議先並力九洑洲,斷江北接濟,先後會諸軍擊走李秀成、李世賢援眾。二年,城圍合,湜當西路,克江東橋、七甕橋、紫金山諸隘,賜號著勇巴圖魯。三年六月,克江寧,湜入旱西門,遇李秀成率死黨出走,逆擊反奔,尋為他軍所擒,以按察使記名。

四年,授陜西按察使,調山西。捻匪方熾,陳籌防五策,建水師於龍門、砥柱間。五年,捻酋張總愚謀渡渭,湜令水師焚三河口浮橋,督民團備渭北,賊不得逞。六年,命湜駐汾州,節制文武。冬,總愚乘河冰合,竄入山西,七年春,犯畿輔。湜以疏防褫職,譴戍新疆,巡撫鄭敦謹疏請留防。冬,陜回將乘隙渡河,屢擊走之,詔免發遣。

左宗棠西征,檄湜率五營出固原,斷漢伯堡賊南趨河州之道,殲餘彥祿餘黨於羅家嶮。九年,金積堡平,復原官。十年,進規河州,宗棠令湜盡護諸將渡洮進攻。連克陳家山、楊家山、董家山諸回堡,逼攻太子寺老巢,破其外壕。十一年,提督傅先宗等戰歿,賊乘勝來攻。湜陽置酒高會,密令總兵沈玉遂急搗之,馬占鼇窮蹙乞降,縛悍酋狗齒牙子等以獻。河州平。十二年,叛酋馬桂源、馬本源踞巴燕戎格,湜率軍進討,二酋敗遁。湜善視其孥,遂因占鼇來降,數其罪誅之,並斬馬五麻諸悍目,賜號奇車伯巴圖魯。四月,逾河收循化。循化撒拉回素獷悍,恃險擾邊。湜深入其阻,群回縛悍目馬十八、沈五十七等二十餘人獻軍前,繳械受約束。湜規地勢,修城設官,分營扼駐,與西寧、碾伯、河州聲息相通焉。尋謝軍事回籍。

光緒八年,兩江總督曾國荃奏調統水陸諸軍,兼治海防,駐軍吳淞。以私行游宴被劾歸。十二年,復出統南洋兵輪,總湘、淮諸軍營務,授江蘇按察使。二十年,遼東兵事起,詔集舊部防山海關,移屯關外鞍山站。二十一年春,進駐大高嶺,遣將援遼陽。和議成,擢江西布政使。命剿甘肅叛回,未行,復駐山海關。二十二年,卒,贈太子少保。

湜從曾國荃最久,後屢蹶,仕久不進。世稱為宿將,光緒中,命繪中興功臣於紫光閣,徵集諸將之像,湜與焉。

李元度,字次青,湖南平江人。以舉人官黔陽教諭。曾國籓在籍治團練,元度上書數千言言兵事,國籓壯之,招入幕。咸豐五年,國籓移軍江西,令元度募勇三千屯湖口。六年,移屯撫州,偕江軍林源恩合防。與賊相持久之,餉絀,分軍克宜黃、崇仁;而賊自景德來援,撫州賊出攻江軍營,林源恩死之。元度突圍出,移屯貴溪,防廣信。七年,賊二萬來襲玉山,守卒僅七百人,元度迎戰,斷賊浮梁,賊以步隊綴軍,騎賊趨上游跐水渡。乃回城拒守,被攻兩晝夜,元度立埤兒間,彈中左頰。賊忽罷攻,鉦鐃雜作,知其穴地道,乃掘壕以防,伺其穿隧及壕殪之。賊技窮引去,伏兵邀擊,安仁、弋陽、廣信皆平。元度先已累擢知府,以道員記名,至是加按察使銜,賜號色爾固楞巴圖魯。八年,率所部平江軍援浙江,敗賊玉山子午口。會克常山、江山,授浙江溫處道。

十年,曾國籓督師皖南,調元度安徽寧池太道,防徽州。至甫三日,賊由旌德糾合土匪散軍入績溪叢山關。遣同知童梅華、都司單綏福率千人往援,敗挫。賊趨郡城,元度退走。國籓奏劾,褫職逮治。會浙江巡撫王有齡奏調援浙,元度不待命,回籍募勇八千,號安越軍。將行,粵匪犯湖南,巡撫文格留其軍守瀏陽,偕諸軍破賊,詔賞還按察使銜,並加布政使銜。

會杭州陷,王有齡死,詔左宗棠代之。元度率軍入浙,與李定太守衢州,授浙江鹽運使,署布政使。國籓以元度罪未定,不聽勘遽回籍,復劾革職,交左宗棠差遣。言官再論劾,命國籓、宗棠按治。國籓奏:「徽州之失,元度甫至,情有可原。」宗棠疏言:「杭州失陷,非因其逗留所致。惟落職後求去索餉,不顧大局。」論遣戍。沈葆楨、李鴻章、彭玉麟、鮑超等交章薦其才,代繳臺費,免罪歸。同治初,貴州巡撫張亮基奏起剿教匪,以功復原官,擢雲南按察使。光緒八年,丁母憂。服闋,補貴州按察使,遷布政使。十三年,卒於官。

元度擅文章,好言兵,然自將屢僨事。所著先正事略、天岳山館文集,並行世。

論曰:蕭啟江、張運蘭功在江西,在湘軍中資勞最深,中道而殞,故恤典特隆。唐訓方、蔣凝學轉戰功多,舊部散亡,再出遂不競。陳湜、李元度皆躓而復起。元度文學之士,所行不逮其言,軍中猶以宿望推之爾。


\end{pinyinscope}