\article{志一百三十四}

\begin{pinyinscope}
邦交七

瑞典那威丹墨和蘭日斯巴尼亞比利時義大利

瑞典即瑞丁,在歐羅巴西北境,與那威同一區。雍正十年始來華互市。道光二十七年春二月,與瑞典及那威國訂通商約。時法、美諸國通商,俱仿英和約條款。瑞本小國,亦求照英、法、美三國成案議通商條約。時瑞鋼鐵等項價甚賤,並求酌減稅則。兩廣總督兼五口通商善後事宜耆英以各項稅鈔甫經議定通行,未便因瑞鋼鐵率議輕減,不許;惟通商條約奏請許之。遂與瑞公使李利華訂約三十三條。同治六年,政府派出使大臣志剛等游歷各國,至瑞遞國書。光緒三年八月,瑞典開整理萬國刑罰監牢會,使臣愛達華達擺柏照會駐英使臣郭嵩燾,請中國派員入會。嵩燾以聞,許之。

十八年五月,瑞典國教士梅寶善、樂傳道二人往麻城縣宋埠傳教,被毆致斃,上海瑞典總領事柏固聞,赴鄂見張之洞,要求四事:一,辦犯;一,撫恤;一,參麻城縣知縣;一,宋埠設教堂。時犯已緝獲,張之洞允辦犯、撫恤,而參麻城縣則不許,謂麻城縣事前力阻,事後即獲正犯,未便參劾。至開教堂,宋埠民情正憤,改在漢口武穴覓一地建堂,柏固亦不允。久之,始議定絞犯二名,給兩教士各一萬五千元,失物諸項一萬五千元,期二十月後再往傳教。

三十四年六月,與瑞修改通商條約。先是瑞使倭倫白來京,請覲見呈遞國書,並照稱奉本國君主諭,請修改通商條約,並錄其君主所給議約全權文憑送外務部。外務部以道光二十七年所訂瑞典那威條約系兩國聯合所立,近兩國已各獨立,前訂之約距今六十年,通商情形今昔不同,當重訂約,以資遵守,許之。於是瑞使擬具約稿三十九款,大致多採各國與中國所訂約款。外務部以所擬款目繁多,另擬約稿,並為十七款。研商久之始定議。外務部因上奏,言:「臣部另擬約稿,歸並為十七款。查向來與各國所訂條約,我多允許與各國利益,而各國鮮允許與我利益,按諸彼此優待之例,實非平允。惟光緒七年巴西條約暨二十五年墨西哥條約,多持平之處。此次擬議約稿,注重此意,不使各項利益偏歸一面,更於各約中採用較為優勝之條,取益防損。如第三款領事官應照公例發給認許文憑,第十款訂明俟各國允棄其治外法權,瑞典亦必照辦,第十三款聲明給與他國利益,立有專條者,須一體遵守,方準同霑,俱系參照巴西、墨西哥二約。第十二款入教者犯法不得免究,捐稅不得免納,教士不得干預華官治理華民之權,俱系參照中美商約。又瑞使原擬約稿有數款照錄英、美、日各商約,今皆刪去。如商標、礦務之類,則以第十三款內載所有商業、工藝應享各利益均一體享受等語括之,如加稅、免釐之類,則以第十四款內載中國與各國商允通行照辦遵守等語括之,以免掛一漏萬。於第五款內又載進出口稅悉照中國與各國現在及將來所訂之各稅則辦理等語,亦可為將來加稅不得異議之根據。此外各款,如派駐使、設領事,及通商、行船一切事宜,始終不離彼此均照最優待國相待之意,以扼要領而示持平。雖瑞典遠在歐洲北境,現尚無前往貿易之華商,其所許我利益,未能遽霑實惠,然際此中外交通,風氣日開,不可不預為地步。數旬以來,與瑞使往返磋磨,間有字句刪改無關出入之處,亦輒允其請,而大旨已臻妥協。謹錄全約款文,恭呈御覽。如蒙俞允,應請簡派全權大臣一員,會同瑞使署名畫押,仍候批準互換。」疏入,報可。宣統元年四月,在北京互換。

丹墨即嗹馬,在歐羅巴洲西北。其來市粵東也,以雍正時,粵人稱為「黃旗國」。同治二年三月,丹馬遣其使臣拉斯勒福來華,抵天津,徑赴京師。署三口通商大臣董恂以丹使並未知照,無故來京,亟函知總署,飭城門阻之。而英使言:「丹國來人乃本館賓客,請勿阻。」總署遂置不問。英威妥瑪復代請立約,恭親王告以丹使擅越天津來京議約,萬難允其立約。威妥瑪乃言丹與英為姻婭之國,並援法使為布路斯、葡萄牙代請換約之例固請。王大臣等因語以丹使如欲中國允行,宜循中國定章,仍回天津照會三口通商大臣,方可立約。威妥瑪乃請嗣後外國使臣到津,應令天津領事告知中國常例,又為函致三口大臣代為之謝。大臣等以聞,朝旨交總署核議。旋派工部左侍郎恆祺會同三口通商大臣、兵部左侍郎崇厚辦理。

五月,約成,大致以英約為本。初,恆祺等議約擬仿照大西洋成案,威妥瑪謂丹系英國姻婭,應從英文義。辯論久之,各有增減,定和約五十五款,通商條約九款,稅則一冊。明年五月,丹遣水師副提督璧勒來滬,派提督銜李恆嵩及江蘇布政使劉郇膏與換約。屆時李恆嵩等向璧勒索觀應換條約,而原定印約未攜,只另書英字條約。璧勒謂此約系照英文原定條約繕寫工整,以示尊崇中國之意,並無別故;又以本國軍務方殷,不能久待。遂將條約核對,與英文相符,允互換。屬將原定用印和約補訂照繕和約之內,補鈐丹副提督印信,並簽押,遂互換收執。九年十月,丹遣使來華呈遞國書,報中國簡派使臣蒲安臣、志剛、孫家穀使丹之聘也。十年,復呈遞國書。

光緒七年十月,督辦中國電報事宜盛宣懷與丹總辦大北電報公司恆寧生會訂收遞電報合同。先是同治十年,丹國大北公司海線,由香港、廈門迤邐至上海,一通新嘉坡、檳榔嶼以達歐洲,名為南線,一通海參崴,由俄國亞洲旱線以達歐洲,名為北線,此皆水線也。至同治十二年,又擅在上海至吳淞設有旱線。至是中國甫設電局,因先與訂合同十四條:一,中國電報寄往外國之線路;二,電局與大北互定通電之價;三、四,由中國寄外國、外國寄中國內地之報,其報價應先行收清,後再劃還,並在上海立冊,每月互對;六,電價概由自定,惟寄外國報須按照萬國電報定章,又傳報可自編新碼;七,電局與大北往來用英文,惟合同以華文為主;八,大北原竭力幫助中國設電,惟中國自主之事不得干預;十,大北海線、中國旱線如有斷絕停滯,互相通知;十一,中國電政歸北洋大臣主持,有向大北購料者,應稟明北洋核奪;十二,大北應繳回中國電報之費,每三月一結。時法、英、美、德四國以大北公司僅有單股海線,又沿途祗通廈門口岸,其餘如汕頭、福州、溫州、寧波各口皆距較遠,請添設海線,就便通至各口。拒之,仍專與大北公司合辦。方議立合同,大北公司恆寧生欲載明中國不再租陸線與他人,且須永租大北,議遂中止。

九年,李鴻章致總署及盛宣懷,擬中、英、丹三公司合約,英、丹海線均至吳淞為止,將丹自淞至滬旱線購回,由我代遞。議久之始收回。初,大北公司原稟六條內,有「不準他國及他處公司於中國地界另立海線,又中國欲造海線、旱線與大北有礙者,不便設立」二條,為大北公司獨得之利益。因之中國亦取得總署、南北洋及出使大臣往來電報,「凡從大北電線寄發者,不取報費」,為中國獨得之利益。當時鴻章已批準咨行。英、美、法、德各使聞之,合詞照會總署。威妥瑪復援同治九年允英人設海線之案,必欲大東公司添設,政府不能阻。因之大北公司恆寧生請將中國官報照常給費。旋復來電,謂「自十月初三日為始,所有中國頭等官報由大北電線寄發者,須照章付足電資,方為發報」等語。

十六年,薛福成議與大北及大東公司訂立合同。初,大北與大東慮我與俄接陸線奪其水線之利,故原訂明滬、福、廈有水線處,貼中國十分之一,其餘各口出洋報費,悉歸華局續議,並允報效海線官電之費。嗣因各國並俄使牽制,以致久擱。至是,由福成另議,祗讓官電費,不要貼價,歲銀十萬圓。

和蘭,明史作「荷蘭」,歐羅巴濱海之國。清順治十年,因廣東巡撫請於朝,原備外籓、修職貢。十三年,齎表請朝貢,部議五年一貢,詔改八年一貢,以示柔遠。十八年,鄭成功攻臺灣,逐和蘭而取其地,詔徙沿海居民,嚴海禁。康熙二年夏六月,和人始由廣東入貢:刀劍八,皆可屈伸;馬四,鳳膺鶴脛,能迅走。二十二年,和蘭以助剿鄭氏功,首請開海禁通市,許之。乾隆元年冬十月,裁減和蘭稅額。初,和蘭通商粵省,納稅甚輕,後另抽加一稅。至是,諭曰:「朕聞外洋紅毛夾板船到廣,泊於黃埔,起所帶砲位,然後交易,俟交易事竣,再行給還。至輸稅之法,每船按樑頭徵銀二千兩左右,再照則抽貨物之稅,此向例也。近來砲位聽其安置船中,而於額稅之外,將伊所帶置貨現銀另抽加一之稅,名曰繳送,殊與舊例不符。朕思從前洋船到廣,既有起砲之例,仍當遵守。至加添繳送銀兩,尤非嘉惠遠人之意。」命照舊例裁減,並諭各洋人知之。

同治二年秋八月,與和蘭立約。和蘭與中國通商最早,至是見西洋諸國踵至,亦來天津援請立約。三口通商大臣崇厚以聞,朝議許之,即命崇厚在津與其使臣訂和約十六款。初和蘭使送來約稿,皆照英、法各國及參用續立之布、西、丹國等條約、章程,分別各款請議。三口通商大臣崇厚答以現在各口通商,均有定章,不必多列條款。和使亦允刪減,惟前往京師、南京通商,並內地傳教、減稅,暨在京互換條約各節,以和文為正義。爭論久之,始允刪去。而於稅則一層,許另立一款,議明各國稅則屆重修之年,和國亦許重修。並與照會,言將來重修稅則時,亦應按照價值秉公增減。遂定議:一,通使;二,海舶通商;三,游歷;四,傳教;六、八至十二,關稅;六、七,交涉案件;十三,交際議文;十四,行移文書各用本國文字;十五,利益均霑;十六,批準一年內換約。此與和蘭立約之始。三年五月,和公使礬大何文以換約期將屆,遣員伯飛鯉詣天津三口通商大臣,請在廣東省城換約。崇厚以所請符原議,奏請簡員往。朝廷命廣東巡撫郭嵩燾為換約大臣。屆期,和使僅以鈔錄副本上。嵩燾駮令取原本再定換約期。逾年始換。

十年四月,出使各國大臣志剛、孫家穀詣和蘭呈遞國書。十二年四月,和蘭公使費果蓀來華呈遞國書,總署允與各國使臣同覲見,禮節亦如之。光緒七年,和使牒中國,稱本國將於光緒九年夏在都城亞摩斯德爾登等處設立衒奇公會,請中國與各國同入會,許之。是年,以候補道三品卿銜李鳳苞充德義和奧四國出使大臣,此為和蘭遣使之始。八年二月,和使費果蓀復將衒奇會章程,及增擬華商赴會章程,並開中國物產及工藝奇巧制造等件,請其會集運往。總署飭各海關照辦。十一年,出使大臣許景澄如和蘭遞國書。十三年,許景澄出使期滿,以內閣學士洪鈞代之。

是年,兩廣總督張之洞特派副將王榮和、知府餘瓗先往和蘭所屬南洋各島調查,和蘭不允。前出使大臣許景澄與和外部辯論,以游歷為名,和始允行。既返,張之洞上疏請設領事,略謂:「日里有華工萬餘眾,噶羅巴華民七萬餘眾,其附近之波哥內埠、文丁內埠、以及三寶壟、與疏羅、及麥裏芬、及泗裏末、及惹加,皆和屬地,華人二十餘萬眾,宜設總副領事以資保護。」旋議從緩。

二十年,出使大臣許景澄請禁機器進口,牒和蘭外部,略謂:「外洋各項機器,除中國自購並託洋商代購外,其洋商自行販運機器,查系無兒華民生計性命之物,酌照稅則不載之貨估價值百抽五,準其進口。若洋商販運機器有礙華民生計性命者,皆不準進口。」二十一年,命許景澄遞萬壽致謝國書。二十四年,以候補四品京堂呂海寰充出使德國大臣,兼充和、奧兩國公使。二十五年,各國在和都海牙設保和公會,和使牒中國請入會,許之。旋派前駐俄使臣楊儒赴會。又推廣紅十字會、水戰條約,請用御寶,由駐俄使臣胡惟德轉送和政府。

二十七年,呂海寰以和屬南洋各島虐待華民,乃上言:「和屬南洋各島開埠最早,華民往彼謀生者亦最多。噶羅巴一島尤為薈萃之區,寄居華民不下六十萬人。初尚優待,後因迫令入籍,率多殘虐,其故以中國未經設立領事保衛之也。各島有所謂瑪腰、甲必丹、雷珍蘭者,管理華人,以生長其島者充之,擅作威福。華人初到,概入供堂問供註冊;赴各鄉營生,須經批準,方許前往。嗣下不準華民居鄉之例,限二十四點鐘立將生意產業賤售而去,逾限罰銀逐出,產業消歸無有。此其一。又華人到和屬地,向須憑照方準登岸。嗣又變立新例,無論有無憑照,登岸後帶至官衙,繩圈一處,俟查老客有原日出口憑照放行,新客則馳入繩圈之內,候帶入瑪腰公館照像,俟有人擔保始放,否則輒上鐐杻刑具,遇有輪船,驅逐出境。此其二。又華人來往本島貿易,必領路票,使費外仍繳印花銀若干,到一處又須掛號,再繳銀若干。如一日到三五處,則兩處繳費亦須三五次。掛漏查出重罰。此其三。又華人詞訟,審費照西人最多之例,科罰則照土番最重之例。縱令理直,追回銀數,已不敷狀師之費,以至沉冤莫訴。此其四。再如華人家資產業,身故後權歸和官。雖妻子兒女執遺囑照章領取,亦必多方挑剔,反復延宕;若無遺囑,則產業概沒入官。此其五。華人在日裏承種菸葉者,往往系由奸販誘惑拐騙出洋,身價五六十元、八九十元、三四十元不等。立據三年為期,入園後不準自由出入,雖父兄子弟不能晤面。加以剋扣工資,盤剝重利,華人吞聲忍氣,呼籥無門。且各國人民皆得購地自業種菸,華人獨否。此其六。以上苛虐各節,慘不忍聞。正擬設法向和廷理論,忽英文報紙載有班喀地方,華人在錫礦各廠作工,突遇水患,饑寒潮濕,病死相仍。又經廠主勒購廠物,物劣價昂,支借工資,則一兩納息五錢,以致積憤肇事,為廠主槍擊,死傷無算。和官拿獲逃散華民,窮詰再三,始知為廠主苛刻所致。按華工素循規矩,若非相待太苛,必不至於啟釁等語。竊思華民作工各島,受此任意凌虐,與古巴之夏灣拿同一殘忍。領事之設,斷難再緩。迭與和外部大臣樸福爾再三爭論,並譯錄商稟及報紙所載苛待情形,詳為申述。復備文照會,請其允設領事,保我僑民生計。彼外部以事屬籓部為詞,支梧未決。臣復照會彼外部,以新嘉坡、小呂宋等處,中國早設有領事。即以荷屬之噶羅巴而論,歐、美各國無不設有領事,何獨於中國而靳之?反覆辯論,稍有轉機。查和屬島嶼林立,應設領事之處有七:即如噶羅巴、三寶壟、泗里歪、望加錫、勿裏洞、日里、文島等處,均關緊要。今一時萬難遍設,惟噶羅巴一島,設立總領事一員,萬不可緩。」奏入,交外務部議。二十八年,外部議準在噶羅巴等處設立領事,未實行。

三十年,各國議免紅十字會施醫船稅鈔,請中國派員赴和蘭會議,許之。是年,熱河都統松壽奏稱:「蒙古喀喇沁王貢桑諾爾布擬與和商白克耳合辦本旗右翼地方巴達爾胡川金礦,作為華洋合辦,股本各居其半,一切遵章辦理。」外務部以「喀喇沁王原將右翼全旗指給逸信公司開辦五金各礦業,經飭令畫清界限,不得包占全旗。若今又遽允和蘭商人,難保不滋轇轕,應請暫緩。」報可。三十一年,和使照稱本國南洋屬地蘇門答臘以北名撒般者,遇有外國兵船進口,施放敬砲,請外務部知照南北洋大臣。三月,外務部奏:「萬國保和會和解公斷條約業經批準。各國欲在和蘭都城設立萬國公所,作為公斷衙門,請中國派員入會作為議員。」許之,尋以伍廷芳充選。保和會即弭兵會也。是月和使照稱本國屬地茫咖、薩巴東二處,遇有外國兵船進口,不再施放敬砲,仍請外部知照南北洋大臣。八月,萬國弭兵會舉和人男爵米何離斯為判斷公堂總辦。十月,簡知府陸徵祥充出使荷國大臣,並兼辦保和公會事宜。三十二年,派駐美使署顧問洋員福士達充和蘭保和會公斷議員。

宣統二年,和京設萬國禁煙會,請中國派員入會。尋遣外務部右丞劉玉麟往。嗣因禁煙會展期,劉玉麟簡充英使,別遣出使德國大臣梁誠赴會。三年四月,與和定設立領事約。初,和送交領約全稿十七條,政府命陸徵祥與議。顧約文外另有附則一條,謂施行本約,不得以所稱和蘭臣民之人視為中國臣民,徵祥議加以「亦不得以中國臣民視為和蘭臣民」一句,和外部不允,乃命徵祥回京,由外務部照請和使來署接議。和使初仍持前議,繼允將附則改為公文,不入約。又久之,始允將「生長和屬之人,遇有國籍紛爭,在彼屬地可照和律解決」等語,備文互換。又一面將「此項人民回至中國,如歸中國籍,亦無不可」等語,由彼備文敘明存案。議遂定。外務部於是上言:「臣部查和屬設領,系積年懸案,屢議屢擱,垂二十年。此次重提前議以來,一年有奇,始克開議。旋因附則一條,致生枝節,彼此研商,又更兩稔。蓋近世各國國籍法,多偏重出生地主義。生長其地之人,大率隸屬其籍。而我國新定之國籍法,則採用血脈主義。根本解釋,迥然不同。彼之欲加附則者以此,我之堅持刪去亦以此。至回國僑民沿用外籍,誠多流弊。茲定明和屬人民回至中國可歸華籍,藉資補救。其非出生於和屬之僑民,仍可認為華籍,與我國國籍法亦不致相背。就此結束,俾可迅派領事,以慰僑民喁喁之望。」又奏和屬苛例修改情形,略謂:「華人流寓和屬所最難堪者,如種種苛例,臣部迭據華商來稟,電駐和使陸徵祥向和政府交涉。彼初以為治理屬地數百年,成例未易更張,強詞拒駁。經我大臣極力磋商,據稱警察裁判,祗允將改良之法從事調查,未能即時遽改。其入境、居留、旅行三項,允先修改。現入境新章雖尚未見頒布,而居留及旅行二者,先已從爪哇、馬渡拉兩島改有新章,較之舊例已多寬大。」奏入,派陸徵祥為全權大臣,與和使貝拉斯署名畫押。條約用法文。

日斯巴尼亞,一名西班牙,即大呂宋也。明嘉靖初,據南洋之蠻里喇,是為小呂宋,檣帆遂達粵東。及清咸豐八年,見英、法、俄、美立五口通商之約,遂與葡萄牙同請立約,不許。同治三年五月,西班牙使臣瑪斯復來請,並呈所奉全權憑據。三口通商大臣崇厚令瑪斯在天津候旨。朝廷復命候補京堂薛煥蒞津,會同崇厚與瑪斯議約。瑪斯援丹馬、西洋各國進京議約之案。薛煥等以丹馬等國雖在京議約,仍赴天津填寫定約日期,不得謂之在京立約。瑪使始允在津商辦。久之,出所擬條款,有為各國條約所無者,而於駐京一節,立意尤堅。久之始議定,共定條約五十二款,專條一款。六年四月,崇厚與瑪斯始公立文憑互換。十年,穆宗親政,各國請覲見呈遞國書,日使與焉。自是歲沿為例。

光緒三年,日國因索伯拉那船遭風案,聲稱欲派兵船來臺灣。福建巡撫丁日昌上奏,言:「西班牙屬島小呂宋之北,即連臺灣之南,海中山勢,斷續相接,較之日本尤為迫近。本年五六月間,用兵蘇祿,攻破其城,故有狡焉思逞之意。非亟加整備,速辦礦務、墾務、水雷、鐵甲船、輪路、電線諸舉,無以圖自強。」已而兵船不果來。

是年日屬地古巴因招華工,請訂專約。時日使為伊巴理,政府派總理各國事務大臣沈桂芬、毛昶熙、董恂、夏家鎬、成林為全權大臣,與議約。先是光緒元年,總署奏派陳蘭彬出使美國及日斯巴尼亞、祕魯三國,辦理交涉事件。日與祕魯均有應議華人出洋承工事宜。祕魯已經李鴻章議有條款。日則自陳蘭彬查復後,復由總署議定保護華工條款,與各國使臣定期晤論。日使丁美霞及各國使臣亦議具條款,復將此條款參酌合而為一。正在會議,適滇省有戕斃英國繙譯官馬加理事,英使威妥瑪來言,事遂中止。自滇案議結,伊巴理時已來京晤議,訖未就緒。至是始議出章程凡十六款:一,維持同治三年天津條約,不得收留中國逃人;二,既除去前約承工出洋未能盡善之情,所有賠償一層作罷論;三,華人出洋須出情原,不得勉強及施詭譎之計;四,聽華民前往,不得禁阻;五,出洋報名領蓋印執照;六,派遣領事;七,予華人隨便往來準單,須與各國人一律;八,訴訟事件;九,查驗華民多寡之數;十,載華民出洋應守之船規;十一、十二,資送華工回國事件;十三,限制華人前往居住事件;十四,執照準單一切事宜,新到之華人與期滿之華人享同等利益;十五,此次條約未載之利益,中國若與他國,則日國應一體均霑;十六,換約事件及期限。是為重訂華工條款,畫押蓋印,明年換約,復公立文憑。六年,小呂宋華民請設領事,不果。

十三年四月,張廕桓由美赴日都馬得利呈遞國書,屆期君後臨朝,張廕桓恭捧國書敬遞,君後親接後,即付外部謨烈,起立與廕桓為英語,繙繹代答。禮成,君後回宮,廕桓立送,君後回顧,三曲膝為禮。時中國議在小呂宋設領事,日外部已允發準照,而商務總辦米阿斯以條約未載為言。張廕桓商之律師科士達,謂若必挾條約為言,約內第四十七款「中國商民至小呂宋貿易,應與最優之國一律相待」,此明文也。而日官所收身稅、路稅,自丁卯換約起,至甲申,共十八年,小呂宋刊發新例止,共徵華人銀七百七萬八千一百六十一元二角四仙。專徵華人每人歲納九元六仙,甲申後乃兼徵西人,每人一元五角,華人則四元五角。計至丁亥共四年,又長徵銀五十二萬八百三十六元。又路照一項,西人每徵四角五,華人每徵則一元二五,又須預納一年身路稅,無理之甚。即與西人比較,將四角五除去,實長徵華人八角。自丁卯至丁亥,廿一年,共銀七十二萬九千一百七十元四角,預納之身路稅猶在外也。又每華人歲徵醫院費二角五仙,甚微,自丁卯換約至本年,廿一年,共徵銀二十二萬七千八百六十五元七角五仙。此項與甲申以前之身路稅,均系獨徵華商,甚違一律優待之約。此中人數,就去年正月至九月數目,共計華人四萬三千四百零三人,逐年清計,尚不止此數也。苛待華人如此,應索償已往,禁遏將來,方合辦法。旋得外部文,言日後將議新例,為各領事而設,而於小呂宋設官一事仍不能決。尋見日後,並見兩公主及君姊,問答如禮。廕桓旋去日赴美,議久無效。

十五年,張廕桓受代,以崔國因出使美日祕大臣,駐美,別遣楊慕璿為駐日參贊。十六年四月,崔國因自美赴日遞國書。屆期,日接引大臣以宮車來迎。是日大君主未御殿,後著公服南面坐,國因奉國書,入門行三鞠躬禮,各問君主起居,退。十九年,崔國因受代,以四品京堂楊儒為出使美日祕大臣。

二十六年七月,聯軍入京。八月二日,日使葛絡幹函留京辦事大臣,稱各國統兵各員及公使人等,定於四日辰刻入大內瞻仰,許之。二十七年,各國要求使臣會同覲見必在太和殿,一國使臣單行入覲必在乾清宮,及遞國書用御輿入中門,皇帝親陪宴等。以日使葛絡幹領銜,政府準駁有差。明年,日君主阿肅豐第十三行加冕禮,駐京日使賈思理照會總署,欲中國遣專使往賀。出使美日祕大臣伍廷芳亦以為請。政府乃以張德彞為賀日加冕專使。

比利時舊名彌爾尼壬。清初,其國商船曾來粵東。道光季年,法人復為請通市,而貨舟不至。及五口通商,比遣使臣包禮士赴上海,呈請照各國立約通商。時薛煥撫江蘇,答以應與無約諸邦同一通市,無須另立條約。包禮士謂須入都定議,阻之,允暫留上海。先是咸豐九年,比遣使臣怡性要求蘇撫何桂清三條:一,比官商眷屬、船隻、貨物,與中國相待最優國同視;二,定約後以十二年為度;三,和約議定,須請用寶。至是復以為請。薛煥亦開三條:一,各口均設領事;二,禁商民赴內地游歷、通商;三,使臣不得赴京。比使堅不允更易。辯論久之,始議定。初,比使稱本國主為「大皇帝」,煥援英稱君主例稱「君主」,遂定約四條。時同治元年六月也。

四年七月,比遣使臣金德來華,牒三口通商大臣、兵部左侍郎崇厚,謂前包禮士與薛煥所訂約,未將兩國通商章程並各等事宜詳敘,請再議,不允。迭牒要求,於是派董恂、崇厚為全權大臣,辦理比通商事務。金德旋擬約五十款,大致皆採各國條約。董恂並去三款,共存四十七款。旋畫押鈐印。五年九月,在水扈與蘇撫郭柏廕換約,並致君主第二禮波勒德國書。郭柏廕以西洋通商各國從無恭進國書之事,金德稱系新君嗣立,應當入告,乃許呈進。九年六月,比復進國書,請使臣駐京,許之。

十一年冬,使臣許景澄如比都伯魯色遞國書,君主及其妃並邀宴宮內,參贊隨員均預焉。又是年剛果國立為自主之邦,奉比國君主為君,比侍從大臣伯施葛辣照會中國,比主復致國書,自稱「大比利時國主留波德第二謹上書大清國仁聖威武大皇帝陛下:竊查剛果地方設有商會,開闢疆土,曾與各國訂約,立為自主之邦,又推不佞為該處之主。現經議院核準,自應統馭此邦,理合報明大皇帝陛下。惟此新國,乃專歸不佞兼轄,並非比國統屬。闢地之始,允宜宣教布化,治政養民,聯與國之誼以敦和睦,興通商之利以固邦基,盡心圖維,升平同慶,仰副各國期望之意。尚祈大皇帝眷顧優隆,俾免隕越」雲云。十三年正月,比使以本國匯印各國稅則,請中國入會,許景澄以聞。政府旋致比外部,謂:「中國現行稅則即各國議定通行稅則,各國條約均經載明,此外別無通商稅則,與西洋諸國各約各訂者情形不同,未便入會。」五月,比遣謝惠施為駐京公使,呈遞國書,並覲見。

光緒十五年三月,以江蘇按察使陳欽明為英法義比大臣。十七年八月,請中國派員入第四次鐵路公會,考求鐵路新法,許之。十八年正月,湖廣總督張之洞遣繙譯俞忠沅,帶工匠十人,赴比國工廠學煉鋼鐵。二十三年,議借外債修盧漢鐵路,比領事法蘭吉詣張之洞言其國家原借,比他國尤為公道。尋與比商定議,共十七款:二,借四百五十萬金鎊,九扣,實付銀四百零五萬鎊,分四期交到;三,按周年四釐起息;四,前十年還利不還本,十年後,分二十年還清;五,以路業作保;六,五年工竣;八,由比派工程師,名曰監察,但督辦大臣一人節制;九,外國路員由監察遴薦,督辦定派,公司所用工路人員,除監察外,均歸督辦所派之大員節制,中西員如有意見,聽督辦核定,但準監察在旁聽斷;十,比員如有不職,由督辦勒退;十一,材料侭中國本有者購買,如購外料,將一半投標,其餘由比公司照辦;十二,所購外料,比公司應扣五釐之用;十四,此合同期內,比公司無論何事,均不得託他國商民管理,並不能將此合同轉與他國及他國之人;十五,如中國未到合同之限,原將此款一概還清,利息即以清還之日停止。已又增訂合同,又續訂詳細合同,於原利四釐之外,加收四毫。又辦事銀行按所付照酬以二毫半,各股票提前還本者,亦酬以二毫半。

二十五年,比使請增漢口租界,謂沿江日本租界旁地,除設鐵路站外,中間尚餘一萬尺,本國請用一千尺,不允。已而駐漢比國總領事復見張之洞,援同治四年中比條約第十二款,仍請在漢口日本界下給比租界百丈。張之洞告以各國專界皆須有專約。同治四年之約,祗言比人在通商各口宜居住、宜建造之處,可聽其租地建造,並無圈畫租界歸比國管轄之語。因與約三條:比人在漢口如欲租地居住,上有英、俄、法、德、日本各界,下有自日本界至鐵路中國之地,均宜居住,可聽各與業主議租。在他國租界,則遵守各界巡捕納捐各章程。在租界外中國之地,則遵守中國巡捕納捐章程,不準自修道路、自設巡捕,亦不準抗違拏犯。一也。比商欲買何處,可向業主商議,彼此情原,公平議價,照條約不得強壓迫受租值。二也。有比商一家,即議地一段,不能預圈空地一片歸比管轄,以致暗中作成租界。三也。

二十八年四月,比使詣外務部,謂漢口租界早經購妥地畝,即將圈築圍墻。外務部命張之洞查復。之洞致外務部,謂:「比人在漢口鐵路總站附近夾鐵路兩旁,購地一大片,請劃為租界。當告以鐵路為中國之路,總站處不能為他國所占,萬難照辦。囑其沿江一段,後至距鐵路三十丈,左至距鐵路總站六十丈止,作為租界,其餘路線以後沿路之三十丈、六十丈各地段,必須全數讓還中國。此系格外通融辦法。比使來鄂時,亦已當面切實辯論。迨飭關道備文照會比領事,比領事照復,將給與租界照收,而未提及其餘應還中國地段。務望囑其早日照鄂定界址定界,將餘地歸還。若再延宕,即已準之界亦不能作為租界。請堅持駮之。」久不決。是年八月,比商赴信陽辦貨,運至漢口,並未請領聯單,又抗不完釐。張之洞飭關道暨稅司詰之。

二十九年八月,與比公司訂汴洛鐵路借款合同暨行車合同,附鐵路管理材料廠章程、土木合同、購地章程。先是光緒二十五年,鐵路大臣盛宣懷奏請將開封、河南兩府枝路統歸總公司籌款接造,奉旨報可。旋因拳匪事起,停議。至是,比公司代理人盧法爾重申前議,於是盛宣懷乃與盧法爾商議借款。因上奏言:「盧漢幹路在滎澤左近渡河,東至開封,約一百七十里,西至河南府,約二百五十里,現由盧法爾估計,應借工款一百萬鎊,約合法金二千五百萬佛郎克,議明利息期限悉照盧漢章程,俟合同簽定後九個月內開辦。所有議訂合同各條,飭由總公司法文參贊候選道柯鴻年等與盧法爾數月研商,並經臣盛宣懷與河南巡撫陳夔龍逐條斟審,刪汰商榷,並經外部增改,定細目二十九條,又行車合同十條。」奉旨:交外務部覈議具奏。外務部奏言:「臣等查盧漢分枝開封、河南兩府,既經奏蒙俞允,自應準其展造。本年六月,盛宣懷函造合同到部。臣詳加復核,其還本、付息、用人、購器一切辦法,均與盧漢合同相符,而意義較為周密。惟合同第二十三款內載『倘日後中國國家準由河南府接長至西安府,督辦大臣可以應允先侭比公司按照本合同章程妥商議辦』等語。查二十五年十月盛宣懷原奏,雖經申明自洛以通秦隴,應歸總公司籌款接造,而此段枝路地勢綿長,將來如議用華款自辦,亦不可不預留地步。當令添敘『倘中國國家自行籌款,或招集華商股本,接展此路,比國公司不能爭執』。又令於行車合同第九款內添敘『中國郵政局由此鐵路寄送各郵件,應特備專車;沿途各站,皆須備給房屋,以設郵局,均照中國各鐵路通行章程辦理。沿途並不得由承辦之國另設郵局』等語,以保權利。」硃批:依議。宣懷遂與盧法爾定議,借金款二千五百萬佛郎克,合英金一百萬鎊,年息五釐;歸還之期,由賣票之第十年起,分二十年均還。

三十年二月,張之洞聞比國欲在湘造湘陰過常德至辰州一路,特電致湘撫趙爾巽,以紳商稟請承辦拒之。

三十四年,始議收回漢口比國租界。張之洞上奏,言:「比國乘鐵路購地之際,在漢口私購民地三萬六千餘方,以預備鐵路比國工人賃住為辭。自光緒二十四年向總署索訂比國路界,經臣力拒,自光緒二十四年起議,相持至二十八年。比使復迭向外務部催咨。臣思比國原購地段,緊倚京漢鐵路南端江邊馬頭之劉家廟火車站,包過鐵路,實扼南北鐵路咽喉,於中國管理鐵路主權,及京漢、粵漢兩路交接之馬頭,大有妨礙,堅不允許。僅就濱江一邊劃地一萬六千餘方,擬作比界,東北兩面,皆與鐵路相離數十丈。比使復求加寬,駮以查明窒礙,咨復外務部酌復。自是又相持數年。比駐漢領事將所買地契送交關道稅印,要挾甚力。臣思此地跨越鐵路,橫當要沖,雖一再駮令減讓,究於附近鐵路地權地利有損,不如議價收回,留作擴充華商貿易,以永保權利。惟自鐵路告成後,地價數十倍於前。經臣磋議經年,始將全數基地議定價銀八十一萬八千餘兩,暫行息借華洋商款墊付。」奏入,報可。

義大利即意大利亞,後漢書所稱大秦國也,在歐羅巴洲南境。康熙九年夏六月,義國王遣使奉表,貢金剛石、飾金劍、金珀書箱、珊瑚樹、琥珀珠、伽南香、哆囉絨、象牙、犀角、乳香、蘇合香、丁香、金銀花露、花幔、花氈、大玻★鏡等物。使臣留京九年,始遣歸國。召見於太和殿,賜宴。聖祖以其遠泛重洋,傾誠慕義,錫賚之典,視他國有加。

同治五年秋八月,義國使臣阿爾明雍介駐京法國領事德微亞詣三口通商大臣、兵部左侍郎崇厚請立通商條約,許之。旋派戶部左侍郎譚廷襄為全權大臣,會同崇厚辦理通商條約。九月,阿爾明雍偕法國繙譯官李梅親齎所擬條約五十五款請核,並遞國書。其約大致本丹國和約而參用法、布等國條約,獨禁用「夷」字一條,本之英約。而中國於義向未稱「夷」,與英事實不同,政府以無關緊要,亦不予駮。遂定議。其目之要者為二,附稅則一,與法、布二國同,與英、美、丹、奧、日本各國權度名略異。通商章程善後九款,與丹、奧、比等國大致同。約定後,阿使回國。旋由法使伯洛內致送我國訂約大臣圓形金牌,上印本國君主容儀,以為紀念,受之。

六年九月,義使駱通恩抵滬請換約,朝命江蘇布政使丁日昌與互換。法領事狄隆赴日昌行館,聲稱此次義國換約,派伊為繙譯官,請日昌先往駱通恩處致候。日昌告以義國公使奉其國差遣出使中華,應先見中國使臣,致其君命,方為盡禮。狄隆又言前在天津照會,聲明於九月在滬換約。今已十月。日昌告以上年比利時國訂於九月換約,先於五月通知。今義國訂於九月換約,遲至九月中旬始行通知。由三口通商大臣咨呈總署王大臣,奏請派使用寶,委員齎送來蘇。現於十月換約,已極迅速。其遲延不在中國也。屆期,駱通恩偕法總領事白來尼、副領事狄隆等齊集日昌行館,公服帶劍,恭請聖安。日昌偕蘇松太道應寶時等按章禮待。駱通恩索觀憑據,日昌恭捧諭旨,給與開讀,並將條約公同展對。駱通恩出視條約一匣,綴有義國君主用印之銀盒蠟餅,裝飾整齊,惟系用洋字另書,並無上年在京所定原本。日昌不允互換。駱通恩免冠懇求,自認錯誤,謂值新舊使臣交換之際,誤以為有其國君主用印之條約即可為憑,致將原約漏未攜帶。此次蒙恩準予換約,各國皆知。今屆期不換,實覺無顏對人等語。白來尼等亦為之代求,原代為繙譯,並謂現帶用洋字條約,儻與漢文原約文義不符,惟法國領事是問,懇為通融辦理。日昌與應寶時商明,先飭洋務委員督同熟諳意大里亞國文義之監生沈鼎鐘,並白來尼等,將駱通恩所齎洋字條約與奉頒條約詳校無訛,仍不允與換。駱通恩一再情懇,日昌乃與變通,告以貴使祗齎有君主用印之洋字條約一分,則中國使臣亦祗能先將我皇上用寶之漢文條約一分與之互換,所附洋文條約,暫為拆下,留在上海道署,限駱使於四個月內取上年原定條約來換此約,並聲明彼時祗能由蘇松太道就近與換,不再遣使。駱使允照辦,惟四個月限期改為六個月。十年三月,義遣使臣費三多來華,並遞國書,兼考求浙江養蠶事。

光緒十一年夏,義國擬開養生會,請中國入會。十五年,命江蘇按察使陳欽銘為出使英法義比大臣,旋代以大理寺卿薛福成。十七年春二月,薛福成呈遞國書,義王出見,慰勞備至,立談甚久,大旨謂「義與中國數百年來交誼最先,極為企慕。我觀地圖,始知中國之大,義國之地不及中國十分之一」雲云。旋辭退,禮三鞠躬,復握手。次日謁見王後,亦鞠躬,遵西例也。二十二年,以四品卿銜羅豐祿為出使英法義比國欽差大臣。二十五年,義國索三門灣,不許。先是各國皆於中國索有海軍根據地,至是義命駐京公使瑪爾七諾向總署要求租借三門灣,向總署發最後通牒,要求四日內答復。未幾,義政府命取消最後通牒,調馬爾七諾回國。

二十八年,義請派專使駐京,許之。政府亦以許鎯為出使義國專使。十一月,呈遞國書,義主躬親接授。向例公使見義主無座,至是賜坐。逾月,又見義後及義太后。義主設宴宮中,請各國公使,義主義後均入座。席散,義後詳詢中華文字書籍。二十九年三月,義國開農學會,請中國入會,鎯派員往。四月,許鎯譯送義國財政考於外務部,謂義國幅員廣袤不及中國十分之一,而歲入之款較中國多至五倍,歲出之款較中國亦多四倍有餘。十月,又譯送義國關卡稅則於外務部,謂徵稅章程二十條,應稅之物分十七類,共三百六十八種,又包皮稅及去包皮章程十六條,註冊費章程十一條,其中綜核至悉,分析至精,較之中國通商稅則,疏密懸殊,冀中國取則。是月許鎯請商部派員赴義考察商務,謂「義國在華商務無多,間有他國商人運華貨來義者,除蠶繭、茶葉二宗外,他物絕尟。至華商從未到義國及其屬地貿易,應即派員考察」雲云。二十九年,日、俄開戰。十二月,義與英、美、德、法公同照會俄、日,云:「除滿洲外,不得在北洋水陸境內開戰。」三十年,許鎯又譯義國榷煙志及銀行章程。三十一年,許鎯譯送義國國債冊律章程匯編及官售煙價表。

三十二年夏,駐滬義領事面遞約稿十一條於商約大臣呂海寰、盛宣懷,海寰等即將歷次與外務部電商之加稅、傳教、嗎啡鴉片、國幣、治外法權等五款照交,因致外務部及鄂督張之洞、直督袁世凱,謂:「查義約前四條系新款:一,欲絲貨出口興旺,索開紹興、無錫兩處口岸;一,原襄助中國詳細考求養蠶學堂,及設立局所,代為經理;一,於未加稅以前,改訂蘇杭鐵路運貨釐金,推廣義商辦繭稅單期限。後七條為英、美各約所有,均略變其詞:一,內地行輪;一,治外法權;一,華洋合股;一,礦務;一,國幣;一,優待利益;一,條約期限及以義文為正義等。」外部得電,即逐款指駮。海寰等因告義領,義領一再爭辨。遂議口岸援日約長沙例照辦,蠶學用兼聘教員字樣。大致已就,已忽翻異,欲廢議約。海寰等恐於加稅有礙,欲照所擬允準,令稅司為轉圜焉。

是年,義國密拉諾賽會,牒請中國派員入會,並送到章程各冊及會場總圖。許鎯得牒,當將總章全譯,分章九門,祗譯子目,因致外務部,謂:「此會原起,系為慶賀義大利、瑞士兩國交界地方所鑿新潑龍山洞鐵道告成而設。歐洲山洞鐵道,向以法、義交界之蒙斯尼山洞工程為最鉅,計長一萬二千二百三十三邁當。現開之新潑龍山洞,計長一萬八千七百四十三邁當,實為歐洲山洞第一深長鐵道。從前輪船商貨運往北歐者,必由法國馬賽起岸陸運。今此路告成,以後可改由義境之折努阿起程陸運。此為義國新得商利之大端,故會中章程以陸運、海運、河運三項居首。中國各省現議開鐵道,如派員前來考察,似於講求路政有裨。」政府得電,許之。義又設農業會,意在聯絡地球諸國崇本勸農,請中國入會。計此次入會者四十國,會員共一百十人,前後會議者十,分議者五。許鎯僅於開會及簽押日一到而已。


\end{pinyinscope}