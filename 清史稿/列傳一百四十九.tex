\article{列傳一百四十九}

\begin{pinyinscope}
方積硃爾漢楊頀廖寅陳昌齊硃爾賡額查崇華

方積,字有堂,安徽定遠人。拔貢生。以州判發四川,補閬中知縣,署梁山。達州東鄉賊起,梁山當其沖,賊犯縣境,營白兔山守兵潰。積以一百人據小山為疑兵,賊不敢進。築砦二百餘所,令人自為守。他縣流民依集者三十餘萬人,賊至無所掠食,屢出奇兵擊走之。堅壁清野之法,蓋自梁山始。既而萬縣寶靈寺賊起,越境剿平之,又助大兵殲伍文相於石壩山,卻林亮功於望牛埡,斃亮功弟廷相,賜花翎。擢寧遠知府,仍留駐梁山,凡四年。至嘉慶六年,諸路賊漸平,調夔州,繼劉清為建昌道。涼山生番叛,率師討平之。未幾,里塘正土司索諾木根登殺副土司,奪其印,副將德寧兵為所困。積單騎往,密授舊頭目希拉工布方略,以其眾破之。歷川北道、鹽茶道,擢按察使。馬邊、峨眉嶺諸夷結梁山生番盜邊,積偕提督豐紳由馬邊三河口鑿山深入,克六拔夷巢,遂出赤夷間道,進攻嶺夷十二地。浹旬之間,每戰皆捷。曲曲烏助逆死拒,潛師出其後,殄之。遷布政使。

積官四川二十餘年,馳驅殆遍,山川風土,了然於胸,用兵輒獨當一面。及任籓司,僚屬多故交,一無瞻徇。清節自勵,尤為時稱。卒於官,祀名宦。

硃爾漢,字麗江,順天大興人。少為戶部吏。乾隆中,官甘肅靖遠典史,母憂去官。服闋待次,時平涼回酋田五作亂,爾漢與通判吳廷芳、知縣黃家駒守靖遠城,賊來攻。靖遠回豪哈得城等期夜半為內應,爾漢得其情,令守者悉登城不得下,至哈得成家,陽科其穀餉軍,因拘之;分遣人誘擒城下賊,賊之雜守者在城上已數十人,縣役鐵光保最為劇賊,猝擒之。角聲起,扼城上賊無脫者,外賊覺,遂引去。由是以知兵聞,擢隆德知縣。徙底店砦降回,擢涇州直隸州知州。擒教匪劉松,擢鞏昌知府。

嘉慶元年,教匪起,蔓延三省。二年,四川賊尤熾,總統宜綿駐達州,檄爾漢參軍事。是時王三槐踞方山坪,白巖山者,地險固,賊渠林亮功、樊人傑屯山上,與方山坪為聲援。將軍舒亮、提督穆克登布屯山前之韓彭坳,爾漢兵三百、鄉勇三千屯山後之排亞口。排亞口之上曰金鳳觀,曰草店,曰鴨坪,一日盡攻克之。復進,有木柵當隘,不見賊,惟以犬守。兵躍攀柵,賊自崖旁斫傷之,鳴鑼掣旗,左右賊大至,爾漢慮斷後路,退師。先是與韓彭坳諸師為期,中道而止,賊得專力山後,故不克。既而奉節賊千餘來援,敗之,擒賊渠邱廣福。巖賊久困欲走,傾巢來犯,戰一晝夜不得路,仍退。爾漢攻之三閱月,博戰被創,乃回鞏昌。

三年,運麥十萬石餉軍,行至成縣,賊渠高均德來奪,敗之於格樓壩,擒其黨李德勝。四年,張漢潮犯秦州,爾漢赴成縣會剿。鞏昌警至,馳還,賊已據城東鴛鴦河,夜掠賊卡而入,城守始固,以功擢鞏秦階道。生番鐵布者,居西傾山中,眾十餘萬,乘教匪猖獗,時出盜內地。爾漢以鐵布未叛亂,且地險,一構兵非數年不能平。鐵布奉回教,乃召其阿渾諭之,於是來首者踵至。一日書姓名一紙,曰:此鐵布黨也。又出一圖,曰:盜巢及要隘盡於此。分遣百餘人捕之,悉就擒,鐵布遂定。六年,川、楚、陜賊漸蹙,餘賊多竄甘肅,率兵扼剿,凡數十戰皆捷。八年,甘肅匪平,上功最,賜花翎。

爾漢有識斷,能得人死力,奴客悉以兵法部之。自出仕即在行間,後遂與教匪相終始。用兵有法,所用鄉勇侯達海,侍衛李榮華,武舉劉養鵬,千總鄒坤、桂攀桂皆操刺勇健善戰,故所至有功。尋調廣東肇羅道,擢廣西按察使,署布政使。十二年,卒於官。

楊頀,字邁功,江西金谿人。乾隆四十九年進士,授刑部主事。總辦秋審,執法平。內監訟其弟妻,頀按律杖贖守夫墓。和珅方總刑部,意有所徇,駁詰之,頀面爭。和珅叱曰:「司員敢爾!」頀厲聲曰:「司員主稿,知為刑獄得其平耳!何叱為?」和珅不能奪。及珅敗,擢員外郎。仁宗召見,嘉其有守,命解餉四十萬兩赴四川濟軍。川、陜大吏交章論薦,授陜西延榆綏道。時三省清釐叛產,撫恤難民,事方殷,詔責疆吏慎選公正大員如及劉清者任其事。頀周歷田野,綜覈不茍,民漸復業。巡撫秦承恩檄府縣募民補伍,頀曰:「農工商賈各有其業,若預選送營,曠日失業,與抽丁何殊?」議乃寢。調甘肅平慶涇固鹽法道。

嘉慶九年,擢安徽按察使,捕六安州匪劉成巨置諸法。十三年,遷江寧布政使。淮、揚大水,乘舴艋歷災區訪問疾苦,渡湖幾覆,災黎感之。尋以失察山陽知縣王仲漢冒賑,坐褫職。詔頀查賑認真,平日實心辦事,留河工效力。復起用,歷淮海道、浙江按察使、江蘇布政使。二十二年,擢浙江巡撫。未幾,坐臨海民毆差釀大獄,降四品京堂;復不俟代去任,降禮部郎中。引疾歸。道光五年,重宴鹿鳴,加四品卿銜。卒,年八十五。

廖寅,字亮工,四川鄰水人。乾隆四十四年舉人。家貧,不能常試禮部,十二年中,僅再至都。以大挑知縣官河南,署葉縣。時教匪方熾,葉當沖,寅撫民不擾。民有從逆者,捕其魁乃定。長子思芳有武略,省父至葉,任以守衛事。詔捕教首劉之協,久不獲。一日,思芳巡歷近郊,見二人縶馬坐樹下語,異之,歸戒門者伺狀。俄二人入城飲肆中,有識之者,其一即之協。寅趣思芳往與雜坐,出不意縛之,鞫得實,械至都伏法。特擢江蘇鎮江知府。濬丹陽九曲河,築徬,以時啟閉,民便之。擢江西吉南贛道,兼筦關榷,正稅外無多取,吏胥奉法。會南昌煽亂,捕首惡置法。安遠復亂,單騎往諭,解散黨與,耆民等縛其魁以獻,事遂平。歷署布政使、按察使。嘉慶十六年,遷兩淮鹽運使。恤灶丁,治私梟,鹽課漸增。河北滑縣教匪起,總督百齡檄寅往徐州協守御。會捕逆匪劉第五,誤系同姓名者,坐失察降調,上念其擒劉之協功,許捐復原職。以老病歸,遂卒。

思芳少時居鄉治團練,從軍數有功,官至江蘇候補道。在葉手擒劉之協,名聞天下。後以捕劉第五獲罪下獄,尋赦之。

陳昌齊,字賓臣,廣東海康人。乾隆三十六年進士,選庶吉士,授編修,累遷中允。大學士和珅欲羅致之,昌齊以非掌院,無晉謁禮,不往。大考,左遷編修。尋授御史,遷給事中。

昌齊生海邦,習洋盜情狀。上疏論剿捕事,略曰:「洋匪上岸,率不過一二百人,陸居會匪助兇行劫。沿海居民皆採捕為生,習拳勇,諳水勢,匪以利誘,往往從匪。可以為盜,即可用以捕盜。宜令地方官明示,有能出洋剿捕,或遇匪上岸,殲擒送官驗實者,船物一概充賞。被誘從匪者,能擒盜連船投首,免罪。則兵力所未及,丁壯亦必圖賞力捕。仍令地方各官稽戶口,編保甲,以清其源。於各埠訪拏濟匪糧物,各市鎮嚴緝代匪銷贓,俾絕水陸勾通之路。庶幾洋面肅清,地方寧謐。」

嘉慶九年,出為浙江溫處道。時海寇蔡牽肆擾,昌齊修戰艦,簡軍伍,募人出海繪浙、閩海洋全圖,纖悉備具。每牒報賊情及道里遠近稍有虛妄,必指斥之。與提督李長庚深相結納,俾無掣肘,鞫海盜必詳盡得其情。德楞泰奉命按閱閩、浙,議申海禁,謂不數月盜可盡斃。昌齊曰:「環海居民耕而食者十之五,餘皆捕魚為業。若禁其下海,數萬漁戶無以為生,激變之咎誰任之?」德楞泰改容稱善。在任五年,以鞫獄遲延,部議鐫級。江南、福建大吏闢調,皆不往。歸里,主雷陽粵秀講席。修通志。考據詳覈,著書終老焉。

硃爾賡額,原名友桂,字白泉,漢軍正紅旗人,裔出明代。王父孝純,工詩古文,有異才,由四川知縣歷官至兩淮鹽運使。

硃爾賡額納貲為兵部主事,充軍機章京,累遷郎中,出為江安糧道。兩江總督蘇凌阿閽人為和珅舊奴,恣睢用事,廉得其狀,白而逐之。從總督赴安徽察治劉之協逆黨,株連數百人,多所省釋。署安徽布政使,引疾歸。以母老乞改京秩,授戶部郎中。和珅奴劉全之婿號檳榔蔣者,倚勢奪民產,訟於部,刑責不稍貸。西賈利旗產,嗾言官疏陳,使得與漢民通售買,下部議,啖以重賄,卻之,持不可。大學士硃珪管部,聞而重之。故事,自告改京官,不外用。珪薦其才守可大受,復出為廣東潮州知府。海盜方張,硃濆尤黠悍,乃親歷海壖,治鄉團,調鎮兵千守沿海,斷內奸接濟。濆糧絕,屢敗走臺灣,潮盜膽落,因其窮蹙解散之。盜魁黃茂高、許雲湘、王騰魁、楊勝廣、黃德東、關兆奎受撫,選其強幹者編入練勇。會匪李崇玉踞惠、潮山谷中,時游弋海上,使降人招之自首,硃濆部眾亦有來投者。會以母憂去,未竟其事,服闋,補雲南曲靖。

嘉慶十四年,百齡為兩廣總督,疏請調硃爾賡額廣東,擢高廉道,署督糧道,剿匪事一以倚之。勘海口砲臺舊在山上,發砲輒從桅頂過,悉改建於山麓,屢碎盜艦,挫其鋒。暫改運鹽由陸,撤紅單船入內港,以杜接濟。戒並海郡縣嚴斷水米,如在潮州時。匪勢漸蹙,用舊降人招郭學顯就撫。未幾,鄭一妻與張保仔率眾逾萬泊虎門,要總督親至海口面議,文武噤莫敢決,硃爾賡額獨進曰:「保仔自知罪大,眾多無糧,拂其請,將死鬥。請撤兵衛,單舟逕詣,諭以恩威,必可集事。」先遣南海、番禺兩令往傳命,使熟籌而志堅。翌晨,從百齡登舟,行四十里,見列艦數百,夾水如衢,舉砲迎,聲震城中。請總督過舟,叱之曰:「保仔當泥首乞命,如仍驕肆遲疑,無死所矣!」迨晡,保仔登舟,請留三千人招西路賊烏石二,不聽則擒之以自贖,許之,給米千石慰遣。保仔乃使餘眾登岸受撫,自起椗出洋。群謂其所散皆罷弱,自留精銳,得米將不可制,笑應之曰:「此不必以口舌爭。」至期,保仔果誘烏石二至高州,誅之。海盜悉平,以功獲優敘,賜花翎。尋調署南韶道。

十六年,河決李家樓,特命百齡為兩江總督治河事,調硃爾賡額為江南鹽巡道。至則佐百齡定計,接築洪澤湖口束清壩,逼溜刷深太平河,使水有所歸。次年,李家樓決口合龍,新築格堤遏水與大堤平。初,當事主守格堤,奉嚴旨,失守者從軍法。至是見事危急,請改守大堤,聽河溜穿格堤而下,免旁洩之險。又新築減壩受水攻,展側上游築斜壩挑水,數日壩根掛淤乃穩固。所籌措工事悉合機宜。葦蕩營久為弊藪,樵兵空額無人,營員領帑,臨時雇募,弁目專其利。又為灘棍所持,蕩料歸灘棍者十五六。歸弁目者十二三,歸工用者十一二,歲僅得葦十數萬束。百齡檄硃爾賡額督治其事,乃請以蕩地不產柴者給樵兵,人四十畝,給牛具籽種,建棚廠以居,蕩始有兵。濬溝渠便筏出入,採運始及遠,建衙署俾營員常年駐蕩,民挾制偷竊者有禁,蕩始有官。受事之年,採足正額二百四十萬束。於是灘棍之利盡失,員得料抵價,少所沾潤,皆不便之。適有船兵中途改束,斤重不敷,八藉欲撼搖全局。百齡悉其奸,偕河督察訊,硃爾賡額往勘定十七年新葦,每束箍口以二尺八寸為率,增舊三寸,估右營得葦八百萬束;會署江寧布政使,未及估左營。時河督陳鳳翔為百齡所劾,自訴於朝,命尚書松筠、侍郎初彭齡按訊,牽及葦蕩事。員熒說,嗾驗尾幫,舟載餘葦九百束,據其重率,以衡已收三百萬束之數,斥為不足,遂被劾虛糜錢糧,苦累樵兵,遣戍伊犁。時冤之。

硃爾賡額因百齡前劾鳳翔詞不盡實,獄無結正,原以身任,遂不辯。在戍六年,放還,久之,卒。

查崇華,字九峰,安徽涇縣人。少孤,游福建傭書。久之,福州將軍魁倫闢佐幕,甚見信任。魁倫劾總督伍拉納、巡撫浦霖,即命署總督,治其獄。閩地瘠苦,歷任大吏責供張無藝,所屬羅織大戶勒賄,民不堪命,至是貪酷之吏悉伏辜。崇華名聞於時。納貲為通判,留福建。

嘉慶十四年,海盜蔡牽平,以功賜花翎。硃渥欲歸誠,未決,崇華只身至海舶,諭以禍福,遂受降。十七年,署臺灣淡水同知。高媽達妖言惑眾,捕獲,訊得劉林、祝現謀以次年閏八月望在京師舉事,四方起應之。崇華牒請奏聞,大吏以其語不經,置之,僅以傳教罪誅高媽達。至十八年九月十五日,果有林清、祝現之變,劉林者即林清別名也。自高媽達伏法,福建匪黨巳解散,得無事。尋以道員謁選,授河南南汝光道。教匪鉅魁劉松久在逃,懸緝十餘載,偵知潛匿安徽宿州傳教,捕獲之。母憂去官。

道光二年,補陜西鳳邠道。值大軍征張格爾,調駐嘉峪關治軍需。自川、楚軍興,將吏習於糜費,崇華一主覈實,以內地馬駝出關不耐寒苦,關外有臺站應付,長雇徒糜芻秣,悉罷之,節帑甚鉅。凡三署按察使,治獄明慎。以老乞歸,卒。

論曰:剿平教匪,不獨賴將帥戮力,一時守土之吏,與有勞焉。最顯者為四川劉清,而方積亦倡行堅壁清野,保障一方,後復屢定番亂,蜀人與清並稱。他如硃爾漢之保鞏昌,楊頀之清叛產、撫難民,廖寅之擒劉之協,皆卓有建樹。陳昌齊、硃爾賡額於治海寇並具謀略,而硃爾賡額功尤顯矣。查崇華預發林清逆謀,為疆臣所格;及筦西征軍需,以撙節稱,故同著於篇。


\end{pinyinscope}