\article{列傳六十八}

\begin{pinyinscope}
費揚古滿丕碩岱素丹馬斯喀佟國綱邁圖格斯泰

阿南達子阿喇納吉勒塔布殷化行

潘育龍孫紹周從孫之善額倫特康泰泰弟海

費揚古,棟鄂氏,滿洲正白旗人,內大臣三等伯鄂碩子。狀貌魁異。年十四,襲爵。

康熙十三年,從安親王岳樂率兵徇江西討吳三桂。三桂將黃乃忠糾眾萬餘自長沙犯袁州,費揚古與副都統沃赫、總兵趙應奎擊敗之,克萬載。十五年,擊走夏國相於萍鄉,進圍長沙,累戰皆捷。十八年,復敗吳國貴於武岡。師還,擢領侍衛內大臣,列議政大臣。

噶爾丹劫掠喀爾喀,遣使諭罷兵,不從,數擾邊境。二十九年,授裕親王福全為撫遠大將軍,率師討之,命費揚古往科爾沁徵兵,參贊軍事。秋,擊敗噶爾丹於烏闌布通。三十二年,歸化城增戍兵,以費揚古為安北將軍駐焉。三十三年,噶爾丹遣使至,請入貢。費揚古發兵迎護,偵其眾男婦千五百有奇,留之歸化城。疏聞,上察噶爾丹意叵測,陽為修好,潛遣入內地窺探,命侍郎滿丕諭責其使,遣之還。七月,聞噶爾丹將窺圖拉,詔費揚古偕右衛將軍希福率軍往御。希福請益兵,上責其疑沮,令勿偕往。尋以圖拉無警,慮噶爾丹將趨歸化城,詔費揚古旋師。三十四年,噶爾丹至哈密,費揚古往御,乃自圖拉河西竄。尋授右衛將軍,仍兼攝歸化城將軍事。疏言:「聞噶爾丹據巴顏烏闌,距歸化城約二千里,宜集兵運糧,於來年二月進剿。」詔授費揚古撫遠大將軍,以都統伊勒慎,護軍統領宗室費揚固、瓦爾達,副都統碩岱,將軍舒恕參贊軍事。尋召入覲,授以方略。

三十五年二月,詔親征,三路出師,以黑龍江將軍薩布素出東路,費揚古出西路,振武將軍孫思克、西安將軍博霽自陜西出鎮彞並進,上親督諸軍自獨石口出中路。上與費揚古期四月會師圖拉。費揚古師自翁金口進次烏闌厄爾幾,再進次察罕河朔,與孫思克師會,而上已循克魯倫河深入。五月,費揚古師至圖拉,疏言:「西路有草之地為賊所焚,我軍每迂道秣馬,又遇雨,糧運遲滯,師行七十餘日,人馬疲困,乞上緩軍以待。」上進次西巴爾臺,再進次額爾德尼拖洛海。噶爾丹屯克魯倫河,聞上親督師至,升孟納爾山遙望,見御營,大驚,盡棄其廬帳、器械遁去。上命馬思喀為平北大將軍,逐噶爾丹,並密諭費揚古要擊,親督大軍躡其後。次中拖陵,費揚古偵知噶爾丹走特勒爾濟,遣前鋒統領碩岱、副都統阿南達、阿迪等率兵先往挑戰,且戰且卻,誘至昭莫多。昭莫多者,蒙古語「大林」也,在肯特嶺之南、土臘河之北。費揚古分兵三隊,東則京城、西安諸軍及察哈爾蒙古兵,屯山上;西則右衛、大同諸軍及喀爾喀蒙古兵,沿河列陣;孫思克率綠旗兵居其中。並遵上方略,令官兵皆步戰,俟敵卻,乃上馬沖擊。噶爾丹眾猶有萬餘人,冒死鏖斗,自未至酉,戰甚力。費揚古遙望噶爾丹後陣不動,知為婦女、駝畜所在,麾精騎襲其輜重,敵大亂,乘夜逐北三十餘里,至特勒爾濟口,斬級三千餘,俘數百人,獲駝馬、牛羊、廬帳、器械無算。噶爾丹妻阿奴喀屯素悍,能戰,亦殪於陣。噶爾丹引數騎遠竄,費揚古令阿南達詣御營奏捷。上乃班師,令費揚古駐守科圖。

尋命移駐喀爾喀郡王善巴游牧地,詗噶爾丹所往。甫至,噶爾丹潛使臺吉丹濟拉率千五百人入掠喀爾喀牲畜、糗糧,遣副都統祖良璧御卻之,追至翁金河,丹濟拉敗遁。尋以馬疲,請移軍駐喀喇穆倫。會噶爾丹使其宰桑格壘沽英等來請納款,上再幸塞外,駐蹕東斯垓。召費揚古至行在入對,上褒其功,奏曰:「軍中機務,皆遵皇上指授,並未有所效力。況西路糧匱馬乏,不能前進。及聞駕至克魯倫,官兵無不奮發,不俟督責,力戰破敵。奈臣庸劣,皇上窮追困蹙之寇,臣不能生擒以獻,實臣罪也。」上曰:「噶爾丹窮蹙,朕不忍悉加誅戮,不如撫而活之。」對曰:「此天地好生之仁,非臣等所能測也。」賜御佩櫜鞬、弓矢,命還軍。

三十六年春正月,阿南達自肅州奏哈密回人擒獻噶爾丹子塞卜騰巴爾珠爾等,上以其疏錄示費揚古,並賜胙肉、鹿尾、關東魚,諭曰:「時當上元令節,眾蒙古及投誠厄魯特等齊集暢春園,適阿南達疏至,眾皆喜悅。爾獨居邊塞,不得在朕左右,故以疏示,並問爾無恙,即如與爾相見也。」

二月,上復親征,自榆林出塞,詔費揚古密籌進剿。費揚古以去歲未生擒噶爾丹,請解大將軍任,上不允,令便宜調遣軍馬。費揚古進次薩奇爾巴爾哈孫,丹濟拉使來,言噶爾丹至阿察阿穆塔臺飲藥自殺,欲攜其尸及其女鍾齊海率三百戶來歸。費揚古以聞,上乃班師,令費揚古駐察罕諾爾以待。六月,丹濟拉至哈密。費揚古有疾,詔昭武將軍馬思喀代領其軍。還京師,仍領侍衛內大臣,進一等公,仍以未生擒噶爾丹疏辭,不允,因諭曰:「昔朕欲親征噶爾丹,眾皆諫止,惟費揚古與朕意合,遂統兵西進。道路遼遠,兼乏水草,乃全無顧慮,直抵昭莫多,俾奸狡積寇挫衄大敗。累年統兵諸將,未有能過之者。」又曰:「屢出征,知為將甚難。費揚古相機調遣,緩急得宜,是以濟事。」

四十年,從幸索約勒濟,中途疾作,上駐蹕一日,親臨視疾,賜御帳、蟒緞、鞍馬、帑銀五千,遣大臣護之還京師。尋卒,賜祭葬,謚襄壯。以子辰泰襲一等侯、兼拖沙喇哈番。

費揚古樸直有遠慮。昭莫多破賊,費揚古令幕府具疏減斬馘之數,備言「師行迷道絕糧,皆臣失算,賴聖主威福,徼幸成功,非意料所及」。幕府或咎其失體,費揚古曰:「今天子親御六師,如見策勛,易啟窮兵黷武之漸,非國家福也。」及還京師,上嘗命大臣校射,費揚古以臂痛辭。出語人云:「我嘗為大將事,一矢不中,為外籓笑,損國家威重,故不敢與角耳。」

滿丕,伊爾根覺羅氏,滿洲正藍旗人。世管佐領,自贊禮郎累遷御史,兼管佐領。以事奪官。從都統郎坦赴尼布楚與俄羅斯使臣議界,還授理籓院郎中。

二十九年,偕員外郎鄂齊爾賚敕宣示噶爾丹。時大將軍裕親王福全統師往烏闌布通,上親臨邊指授方略,滿丕以噶爾丹奏書至,因言賊距大軍僅百里,請往擊之。上許之,遂赴烏闌布通督火器營,擊敗噶爾丹,得頭等功牌。累擢理籓院侍郎。三十三年,費揚古進軍圖拉,尚書阿喇尼率蒙古兵為前哨,命滿丕協同經理驛站。三十四年,命往歸化城協理軍務。三十五年,上親征,命將兩藍旗兵赴費揚古軍,自翁金趨圖拉,破賊昭莫多。奉詔還歸化城,察視凱旋官兵行糧,及撫輯降人。旋仍赴費揚古軍,移駐喀爾喀游牧界外塔拉布拉克,偵防噶爾丹,收降其部人札木素等。未幾,噶爾丹竄死,召還京,列議政大臣,予拖沙喇哈番世職。

三十九年,命往四川勘撫番、蠻,同提督唐希順攻復打箭爐。於是雅隴江濱瞻對、喇袞、革布什咱、綽斯甲布諸土目各率所屬戶口投誠。奏請授五品安撫司,其副為六品土百戶,從之。擢正藍旗蒙古都統,以疾乞罷,尋卒。

碩岱,喜塔喇氏,滿洲正白旗人。先世居尼雅滿山,有昂郭都哩巴顏者,歸太祖,碩岱其五世孫也。初授二等侍衛,兼甲喇額真。世祖幸南苑,碩岱與一等伯巴什泰及蒙古侍衛素尼並從。素尼猝拔刀殺巴什泰,碩岱即舉所執長槍擊素尼,立僕,擒之,置諸法。上嘉其勇敢,予世職拜他喇布勒哈番兼拖沙喇哈番。授巴牙喇甲喇章京。

從將軍卓布泰南征,渡盤江,擊敗李成蛟。復進攻李定國,度磨盤山遇伏,力戰破之。又從將軍濟席哈討定山東土寇於七。康熙初,擢前鋒統領。吳三桂反,命率兵先諸軍發,駐守荊州。尋命參贊順承郡王勒爾錦軍務。未幾,罷參贊,從將軍穆占等攻長沙。三桂將馬寶、胡國柱等犯永興,碩岱往援失利,棄營入城。穆占劾之,還京師,罷官,奪世職。

二十九年,起為正白旗滿洲副都統,從定北將軍瓦岱征噶爾丹,至克魯倫河,偵賊遠遁,遂還。尋偕都統噶爾瑪率兵駐大同。三十五年,大將軍費揚古出師西路,命碩岱署前鋒統領,率大同護軍二百八十人為前鋒。噶爾丹遁往西路,命費揚古要擊,偵賊至特勒爾濟口,令碩岱率前鋒挑戰,誘至昭莫多,合圍奮擊,斬獲無算。師還,擢內大臣,復世職,進三等阿達哈哈番。五十一年,卒。子海綬,於雍正七年以護軍校隨大將軍傅爾丹征準噶爾,擊賊和通呼爾哈諾爾,陣沒,議恤,予世職拖沙喇哈番。

素丹,富察氏,滿洲正黃旗人,費雅思哈子。襲世職,授護軍參領。從裕親王擊噶爾丹,戰烏闌布通,中箭傷。擢護軍統領,命帥師駐大同。康熙三十五年,上親征噶爾丹,命素丹發兵與費揚古刻期並進。尋召赴行在,統前鋒兵為導。上次克魯倫河,素丹請俟費揚古軍至夾擊。師還,賜內廝馬,改授前鋒統領。以疾解任。

雍正初,命大將軍年羹堯征青海,起素丹參贊軍務。西寧郭隆寺喇嘛助亂,素丹與提督岳鍾琪討平之。授正黃旗蒙古都統,署固原提督。尋改正紅旗滿洲都統,列議政大臣,仍駐守陜西。七年,師征準噶爾,命素丹將西安滿洲兵出涼州,卒於軍,賜祭葬,謚勤僖。

馬斯喀,富察氏,滿洲鑲黃旗人,米思翰長子。初授侍衛兼佐領。康熙二十七年,自護軍參領授武備院卿。二十八年,遷鑲黃旗滿洲副都統。尋擢內務府總管、領侍衛內大臣,兼管火器營。

三十五年,上親征噶爾丹,馬斯喀率鑲黃旗鳥槍兵以從,先期命與諸大臣議定出征營陣隊伍序次。上駐郭和蘇臺,命閱留牧馬群,議分馬群為七,擇水草佳處為牧地。上進駐西巴爾臺,距克魯倫河已近,而費揚古軍未至圖拉,諭王大臣集行營議。信郡王鄂扎請駐師以待,馬斯喀與內大臣蘇勒達、明珠請進薄敵營,上從之。復進次克魯倫河,噶爾丹望見御營嚴整,遂驚遁。上親統師逐之,至拖諾山。授馬斯喀平北大將軍,率師進至巴顏烏闌。噶爾丹敗於昭莫多,北走,所部丹巴哈什哈等詣馬斯喀軍降。馬斯喀與費揚古師會,收集降人,遣兵★送至張家口外,乃還師。列議政大臣。復從上出塞,率師駐大同。

三十六年春,授昭武將軍,移師駐寧夏,都統巴渾德、齊世,將軍薩布素,都統兼前鋒統領碩鼐,護軍統領嵩祝,總兵王化行並參贊軍務。尋命與費揚古會師,馬斯喀以將軍參贊費揚古軍務。初,伊拉古克三胡圖克圖盜馬歸噶爾丹,及噶爾丹死,復投策妄阿拉布坦。費揚古令馬斯喀率師追之,次摩該圖,不能及,引師還。上遣侍郎常綬等諭策妄阿拉布坦,得伊拉古克三胡圖克圖以歸,誅之。馬斯喀坐追剿遲緩,當奪官,上命留內務府總管及佐領。

四十一年,授鑲白旗蒙古都統。四十三年,卒,賜白金千,遣內大臣奠茶酒;發引,命皇子往送。賜祭葬,謚襄貞。

佟國綱,佟佳氏,滿洲鑲黃旗人,佟圖賴子。初隸漢軍,領牛錄額真,授侍衛。康熙元年,襲三等精奇尼哈番,授內大臣。十四年,察哈爾布爾尼為亂,授安北將軍,率師駐宣府。布爾尼亂定,引還。十六年,推孝康章皇后外家恩,贈佟圖賴一等公,仍以國綱襲。二十年,授鑲黃旗漢軍都統。疏陳世系,請改入滿洲,下部議,許以本支改入滿洲。二十八年,命與內大臣索額圖等如尼布楚,與俄羅斯使臣費耀多羅等議立約定界。

二十九年,大將軍裕親王福全率師討噶爾丹,以國綱參贊軍務。八月己未朔,師次烏闌布通,噶爾丹屯林中,臥駝於前,而兵伏其後。國綱奮勇督兵進擊,中鳥槍,沒於陣。喪還,命皇子迎奠。將葬,上欲親臨,國綱弟國維及諸大臣力阻,乃命諸皇子及諸大臣皆會,賜祭四壇,謚忠勇。上以翰林院撰進碑文不當意,乃自為制文,有曰:「爾以肺腑之親,心膂之寄,乃義存奮激,甘蹈艱危。人盡如斯,寇奚足殄?惟忠生勇,爾實兼之!」雍正初,加贈太傅。

邁圖,亦佟佳氏,滿洲正白旗人。父烏進,國初自哈達來歸。邁圖初授侍衛,從信郡王多尼下貴州,破明桂王將李成蛟於涼水井,李定國於雙河口、於魯噶。從康親王傑書徇福建,討耿精忠,授行營總兵,戰黃巖,克建陽。從將軍拉哈達破鄭錦將何祐於太平山,復興化,拔泉州。從將軍賚塔破錦將劉國軒、吳淑於蜈蚣山,復長泰。皆有功。康熙二十五年,授正白旗蒙古副都統兼佐領。尋署前鋒統領,從征厄魯特,戰烏闌布通,陣沒,謚忠毅,進世職三等阿達哈哈番。

格斯泰,瓜爾佳氏,滿洲鑲白旗人,先世居瓦爾喀。父赫勒,歸太祖。從伐明,攻獻縣,先登。入關,西討李自成,破潼關。下江南,徇浙江,破明兵嘉興城下。以牛錄額真授拜他喇布勒哈番。

格斯泰初為睿親王護衛,從大將軍伊爾德克舟山;從都統瑪奇下雲南,破賊石門坎、黃草壩,克雲南會城:皆有功。累擢前鋒參領兼管佐領。從國綱戰烏闌布通,國綱戰沒,格斯泰直入賊營,左右沖擊,出而復入者再。乘勝追賊至河岸,阻於淖,賊麕集,格斯泰力戰,與邁圖等皆歿於陣。師將發,上賜之馬,格斯泰請自選,得白鼻。或言白鼻古所忌,格斯泰曰:「效命疆場,吾夙原也!何忌?」師還,裕親王奏:「方戰時,親見一將乘白鼻馬三入敵陣,眾皆識為格斯泰也。」賜祭葬,視副都統,予世職拜他喇布勒哈番。

阿南達,烏彌氏,蒙古正黃旗人。祖巴賴都爾莽柰,初事察哈爾林丹汗。林丹汗敗走,率所部二百三十餘戶保哈屯河。逾歲,歸太宗,授一等梅勒章京。從攻寧遠,敗明兵。復從攻錦州,戰死,贈三等昂邦章京。

父哈岱,年十七,從父攻寧遠,敵矢殪父馬且踣,哈岱不遑甲,馳入陣,下馬掖其父超乘,步從擊敵,與俱還。太宗嘉其勇,厚賚之。父死,襲世職。屢從伐明,敗明兵。入關定江南,徇浙江,擊騰機思,討姜瓖,取舟山,皆在行間。康熙間,授內大臣。討吳三桂,命與侍衛阿喇尼徵喀喇沁、翁牛特、蘇尼特諸部兵,分駐大同、河南、兗州,備調發。卒,謚勤壯。

阿南達,哈岱次子也,以一等侍衛兼佐領。康熙八年,鰲拜敗,坐黨附罪斬,聖祖特寬之。

二十七年,噶爾丹侵掠喀爾喀諸部,命偕喇嘛商南多爾濟齎敕諭罷兵。噶爾丹遣使入朝,而侵掠如故。二十九年,命往會喀爾喀諸部兵討噶爾丹,以尚書阿喇尼、都統額赫訥等先後率師出塞。阿南達還奏,言:「噶爾丹為拖多額爾德尼擊敗,偵卒還報,有二人共一騎者,有削木為兵者,狀至窮蹙。請發兵討之。」上命選察哈爾兵六百,率以赴圖拉,益額赫訥軍。尋阿喇尼請移西路軍會剿,阿南達率兵渡瀚海,會大將軍裕親王福全,敗賊於烏闌布通。三十一年,命赴寧夏招和碩特部臺吉巴圖爾額爾克濟農來降,擢正黃旗蒙古都統。三十二年,聞噶爾丹將取糧哈密,授郎坦為昭武將軍,召阿南達還。

三十五年,上親征噶爾丹,命阿南達如喀爾喀諸部求習塞外途逕者二十人為導。上次克魯倫河,噶爾丹將走還特勒爾濟,阿南達方從費揚古自圖拉向昭莫多。費揚古令阿南達等先擊噶爾丹,偽敗以致敵,至昭莫多,縱擊敗敵,事具費揚古傳。阿南達赴行在奏捷,上召詢戰狀,對曰:「噶爾丹聞上親征,惶駭竄走。不虞我兵絕其歸路,突然交戰,擒斬過半,死傷枕藉。屬下人多怨懟,降者甚眾,噶爾丹深以為悔。費揚古慮涉矜張,疏報捷,特約略言之。」上乃班師,命阿南達駐守肅州。尋移軍邊境,詗噶爾丹蹤跡。阿南達遣兵分駐昆都倫、額濟內諸處。復與提督李林隆移砲赴布隆吉爾,度要隘留軍策應,乃還肅州。上以其章示議政諸臣,獎阿南達防邊能稱職也。

噶爾丹自昭莫多敗後,部眾多離散。噶爾丹多爾濟者,其妻弟也,陰持兩端。阿南達至布隆吉爾,獲其邏卒,縱歸招之降,遂遣使通款。阿南達因其使檄哈密回部:「噶爾丹且至,當擒獻。」即傳語噶爾丹多爾濟:「噶爾丹至哈密,哈密且擒獻,當為哈密助。」未幾,噶爾丹遣族子顧孟多爾濟等與達賴喇嘛、青海諸臺吉通聲聞。阿南達復至布隆吉爾偵知之,率兵追及於素爾河,擒其使人,以其書十四函馳奏。

三十六年,哈密回部擒噶爾丹子色卜騰巴爾珠爾及其從者徽特和碩齊等,送阿南達。繼又獲厄魯特土克齊哈什哈。土克齊哈什哈實戕我使臣馬迪,至是始就擒。先後檻送京師。尋復疏言厄魯特晉巴徹爾貝來降,詢知噶爾丹窮促狀。是歲上復親征,命與林隆率甘州兵二千出布隆吉爾。次塔爾河,聞噶爾丹已死,所部臺吉丹濟拉將竄巴里坤依噶爾丹從子策妄阿喇布坦,因往追之,未及,上命還駐布隆吉爾。丹濟拉詣哈密乞降,阿南達護使謁上行在。敘昭莫多功,予拖沙喇哈番世職。尋奉命率兵駐西寧。四十年,卒,賜祭葬。雍正二年,追謚恪敏。

阿喇納,阿南達長子。少襲其祖哈岱世職,授三等侍衛,累進散秩大臣。策妄阿喇布坦繼噶爾丹為寇,侵哈密。康熙五十四年,上命尚書富寧安視師,屯巴爾庫爾。五十五年,授阿喇納參贊大臣,選八旗察哈爾勁卒及嘗從阿南達出塞者,得四百人,率之以行。五十六年,授富寧安靖逆大將軍,令阿喇納將一千三百人,自烏闌烏蘇深入烏魯木齊。至通俄巴錫搜山,俘一百數十人,收駝馬牛羊,躪其稼乃還。五十九年,師入西藏,富寧安復令率四千人自吐魯番出邊,至齊克塔木,破賊敵壘。進至皮禪,回民三百餘以城降,師遂會富寧安於烏闌烏蘇,引還。

六十年,上命率師進取吐魯番,因留駐其地。策妄阿喇布坦來犯,阿喇納行與遇。令分兵為三,突入陣,策妄阿喇布坦敗入林中,棄馬步戰,我師發槍擊殺準噶爾兵百餘,乃敗走,逐北數十里,俘獲甚眾。授協理將軍,築城屯墾,為持久計。阿喇納久居邊塞,悉敵情,疏請進兵伊犁。下議政大臣議,以賊已遠竄,暫緩進兵。雍正元年,擢鑲紅旗蒙古副都統。師征青海,命率兵二千駐布隆吉爾。賊酋阿喇布坦蘇巴泰來襲,遣師追至推默爾,大敗之。未幾,卒於軍。遺疏為父請謚,上特許之。賜白金千,遣官護喪歸,謚僖恪,加拜他喇布勒哈番,以其子伍彌泰兼襲,合為三等伯。乾隆間,定封號曰誠毅。伍彌泰自有傳。

吉勒塔布,李佳氏,滿洲正紅旗人,覺善第三子。初授侍衛兼前鋒參領。康熙十一年,授正紅旗蒙古副都統。

十三年,耿精忠叛,命偕副都統拉哈率師駐江寧。尋令援浙江。從將軍貝子傅喇塔攻嵊縣,與精忠將曾養性等戰於黃瑞山,督兵乘夜分兩翼沖擊;又遣兵循山麓疾上,以鳥槍旁擊之,養性敗潰,克仙居。十四年,養性與叛將祖弘勛犯臺州,吉勒塔布與都統沃申赴援,戰於平山嶺,殪賊四千餘;奪梁蓬隘道,遇賊伏,盡殲之。直趨黃巖,副都統穆赫林督兵夾擊,養性夜走溫州。克黃巖,復戰於上塘嶺。攻溫州,久未下。十五年,養性復以四萬餘人來犯,吉勒塔布遣兵分道逆擊。進剿處州,過三角嶺,循江度師。養性以百餘舟屯江上,陸兵屯得勝山下,據險拒我師。吉勒塔布與總兵陳世凱分道拔賊壘,又以砲擊賊舟,沉諸江。師次溫溪渡口,擊敗精忠將馬成龍等,斬千餘級,遂與康親王師會衢州。偕都統賚塔等擊精忠將馬九玉,戰於大溪灘。吉勒塔布督兵逾三濠,進焚木城,克江山,九玉敗遁。遂度仙霞嶺,進克浦城、建陽諸縣。從康親王進次福州,精忠降。

十六年,擊鄭錦同安。十八年,與錦將劉國軒戰於下坑、於歐溪頭、於郭坑,皆勝,斬二千餘級,收海澄。與沃申駐師漳州。二十一年,師還,累擢護軍統領、正紅旗蒙古都統。二十七年,授兵部尚書,列議政大臣。

噶爾丹侵喀爾喀,上命吉勒塔布與都統巴海等徵科爾沁諸部兵備邊。尋命往蘇尼特,度水草佳處為喀爾喀牧地。二十九年,命與尚書阿喇尼出塞,自歸化至圖拉置臺站,率師會喀爾喀諸部,自洮瀨河進攻噶爾丹。噶爾丹掠烏珠穆秦部,至烏勒輝河,我師與遇,分兵乘夜挑戰。喀爾喀兵違節度,亂陣,戰失利。吉勒塔布當奪官,命留佐領,率兵駐呼魯固爾河。旋命與內大臣阿密達同駐克勒,待裕親王師至,分三隊以進。吉勒塔布為第一隊,大敗噶爾丹於烏闌布通。三十年,詔移喀爾喀土謝圖、車臣兩部歸附人牧近邊。上出塞撫綏,令吉勒塔布與尚書馬齊、班第等,先期集歸附人於上都河、額爾屯河以待。上慮巴圖爾額爾克濟農掠喀爾喀,命吉勒塔布督喀爾喀諸部兵為備。三十一年,巴圖爾額爾克濟農降,罷兵歸。三十五年,擢都統。三十六年,卒,賜祭葬。

殷化行,字熙如,陜西咸陽人。初以王姓成康熙九年武進士。十三年,從經略莫洛討吳三桂,授守備。會王輔臣叛,莫洛遇害,化行被脅羈秦州,稱病不為賊用。逾年,自拔歸,總督哈占奏復原職,補火器營守備。從振武將軍佛尼勒戰牛頭山,攻克上、下嶺。三桂將王屏籓據漢中,以二萬人犯寶雞。大將軍圖海檄化行赴援,破敵,解西山堡圍。復自大泥峪取兩河關,復興安州城。十九年,佛尼勒援永寧,化行為前鋒,敗敵托川,擊走三桂將胡國柱於安寧橋。調援敘州,與西寧總兵李芳述守城,賊分三路來攻,擊卻之。圖海、哈占合疏陳化行奮戰狀,特擢漢中城守營副將。二十年,逐國柱,迭戰安邊、敘馬、連峰、石盤關諸處,屢克要隘,復馬湖府城。

二十二年,追議輔臣叛時被脅,坐奪官。哈占以化行未為輔臣用,從征有勞,奏復原職,授直隸三屯營副將。二十三年,敘功加一等,授都司僉書,兼管副將事。二十五年,上幸畿東,化行扈從行圍,賜上用佩刀。二十六年,擢福建臺灣總兵,賜貂裘、白金。時議城臺灣,化行言地皆浮沙,難以鞏固,令部下人致樹一,植為城,數日而成。諸部亦各植木城,繕治甲兵,防禦以固。三十年,移襄陽。陜西旱,米價騰貴,民多流移。詔發襄陽米二萬石水運至商州,改陸運至西安。命內閣學士德珠與化行及總督丁思孔往督水陸輓運,並護流民還里。三十二年,移登州。復移寧夏。

三十五年,上親征噶爾丹,三路出師,發陜西兵當西路,遣刑部尚書圖納會將軍、督、撫及河西提、鎮議進兵事。化行陳方略,詔報可。時綠旗兵統於振武將軍孫思克,率涼州總兵董大成、肅州總兵潘育龍及化行自寧夏出塞,會大將軍費揚古進剿。化行領所部兵三千至翁金河,簡精卒前進,遇敵昭莫多。山崖峻削,其南漸紘,有小山橫亙,化行急據其巔,麾軍士畢登。敵猝至山腹,發砲擊之,噶爾丹率眾死鬥,鋒甚銳。化行使告費揚古曰:「賊陣堅,宜遣一軍沖其脅,賊婦女輜重俱在後陣,劫之必亂。」費揚古從之。化行望山下兩軍將薄陣,鼓行而下,敵披靡,死傷枕藉。噶爾丹敗遁,詔班師。是役化行功最。

三十六年,疏請率兵二千至郭多里巴爾哈孫偵擒噶爾丹。會上西巡,將幸寧夏,化行迎謁,奏請行圍花馬池觀軍容。上曰:「師行賴馬力。今噶爾丹未滅,寧夏兵至花馬池,往來七八日,馬必疲。獵細事耳,罷獵而休馬,以獵噶爾丹何如?」乃令化行率所部兵五百人從昭武將軍馬思喀復出塞。尋命化行參贊軍務,諭謂綠旗總兵官未有授參贊者,並賜孔雀翎。師次郭多里巴爾哈孫,會大將軍費揚古兵。進至洪郭羅阿濟爾罕,噶爾丹死,詔班師。化行還寧夏。

三十七年,請復本姓。敘昭莫多功,予拖沙喇哈番世職。擢廣東提督。三十九年,瓊州營游擊詹伯豸等擾黎人,黎人王鎮邦為亂,以化行約束不嚴,降級留任。四十年,連、陽瑤為亂,里入峒、油嶺二排尤兇橫。化行率總兵劉虎駐師裏入峒,遣副將林芳入排,使執為亂者以獻。瑤人戕芳及所從兵役。上命尚書嵩祝為將軍,令化行及廣西、湖南提督各發兵討之。四十一年夏,會師連州,分扼要隘,瑤人懼,縛獻為亂者李貴、鄧二等,置諸法,餘悉就撫。尋追按芳被戕,化行、虎不能救,虎奪官,化行休致。四十二年,上幸西安,化行迎謁,授其子純四等侍衛。四十九年,卒。

潘育龍,字飛天,甘肅靖遠人。初入伍,從征李來亨等於茅麓山,有功。康熙十四年,王輔臣叛,育龍從副將偏圖攻三水、淳化,復從揚威將軍阿密達戰涇州。寧夏道梗,大將軍董額使育龍赴提督陳福軍,自紅河川、白馬城諸要隘轉戰七晝夜,達寧夏。駐靈州,招撫散卒。總督哈占調援山陽,敗賊於甘溝口。十五年,從撫遠大將軍圖海奪平涼城北虎山墩。累擢守備。十七年,吳三桂兵犯牛頭山、香泉,育龍從總兵王好問等出間道擊破之。十八年,克梁河關,斬三桂將李景才、景文略等;薄興安,三桂將謝泗、王永世以城降。敘功,擢都司僉書。叛將譚弘據川東,育龍從哈占進剿,復大竹、渠縣。遷游擊。

二十七年,以總督噶思泰薦,擢甘州副將。學士達瑚等自西藏使旋,至嘉峪關外,為西海阿奇羅卜藏所掠。將軍孫思克使育龍偕游擊韓成等搗其巢,斬級四百有奇,阿奇羅卜藏遁。事聞,詔嘉獎。三十年,赴寧夏防剿噶爾丹。時改肅州協為鎮,即以育龍為總兵。三十一年,降番罕篤與羅卜藏額林臣、奇齊克等復叛,育龍追至庫列圖嶺,斬四十餘級,獲百二十人。三十四年,噶爾丹屬回塔什蘭和卓等五百餘人入犯,渡三岔河,育龍擊擒之。三十五年,從征噶爾丹,遇賊昭莫多,飛砲中育龍右頤,益力戰,賊敗遁。師還,召至京師,上撫視其創,命御醫診視,賜衣一襲。移鎮天津。敘功,予拖沙喇哈番世職。

四十年,擢陜西提督,賜孔雀翎。四十二年,上西巡,育龍迎謁山西,賜御書榜。駐蹕渭南,閱固原將卒校射,顧大學士馬齊等曰:「朕巡歷諸省,綠旗無如潘育龍兵者。」命加秩。尋特授鎮綏將軍,領提督如故。四十九年,上幸五臺,育龍迎謁,賞賚優渥,親制詩章寵之。時有陳四等率妻子游行鬻技,走馬上竿,鵕索算卦,俗名曰卦子。人既眾,遂為盜。育龍捕得五百九十餘人。有司讞鞫,因疏請飭各省督撫責所屬鄉村堡寨,遇令改業,編戶為民,給荒地開墾,馬騾牲畜變為牛種,載入賦役全書。下部議行。尋以病累疏乞休,詔輒慰留。五十八年,卒,贈太子少保,賜祭葬,謚襄勇。

孫紹周,改籍陜西西安。襲世職,授二等侍衛。累遷廣西慶遠協副將。雍正初,總督鄂爾泰奏開古州、都江河道,以定旦、來牛二寨苗梗路,檄紹周統廣西兵赴古州諸葛營,與貴州副將趙文英會剿,盡平賊寨。擢雲南提督,賜花翎。調古北口,以病解任。乾隆十八年,卒。高宗追念育龍軍功,特予恩騎尉世職,以紹周子忱嗣。

之善,育龍從孫,仍籍甘肅靖遠。初從育龍征噶爾丹。昭莫多之役,力戰中槍,詔來京師醫治。四十二年,上幸西安,之善迎謁臨潼,授藍翎侍衛,賜孔雀翎。補肅州鎮標游擊。策妄阿喇布坦以二千人侵哈密,之善率兵二百擊敗之。上嘉其勇,超擢陜西潼關副將。從靖逆將軍富寧安擊準噶爾於烏魯木齊,多俘獲。雍正初,青海臺吉羅卜藏丹津叛,侵布隆吉爾,與參將孫繼宗引兵夾擊,斬獲無算。擢四川川北總兵,移鎮陜西西安。之善以邊外遼闊,當設卡路杜窺伺,乃遣兵於沙州西路伊遜察罕齊老圖及察罕烏蘇諾爾分路偵御。並以住牧熟夷數百戶,分置諸要隘,詗敵情,督修西安城及沙州五堡,以二千四百戶屯田沙州,籌牛種,建房舍。疏聞,上深嘉之,命署固原提督。諭曰:「此軍乃汝叔祖潘育龍所整理,為天下第一營伍,流風餘韻,至今可觀。若不能企及,何顏以對朕耶?」尋以目眚解任。十一年,卒。

額倫特,科奇哩氏,滿洲鑲紅旗人,佛尼埒子也。佛尼埒卒官,家貧不能還京。四川總督哈占請留額倫特西安效力,部議不許,上特允之。康熙二十三年,授西安駐防佐領。三十年,從將軍尼雅翰逐厄魯特巴圖爾額爾克濟農,又從將軍郎坦赴克錫圖額,皆有勞。三十五年,上親征噶爾丹,從大將軍費揚古出西路,破敵昭莫多。以功授世職拖沙拉哈番,擢協領。四十三年,上幸西安閱武,設宴,特命額倫特近御座,親賜之飲。諭曰:「爾父宣力行間,爾亦入伍能效力,故賜爾飲。」尋遷西安副都統。調荊州副都統。四十九年,擢湖廣提督。五十二年,授湖廣總督。尋命履勘湖南諸州縣荒壤,得四萬六千餘頃。疏請聽民開墾,六年後以下則起科。五十四年,命往按太原知府趙鳳詔貪墨狀,論罪如律。

厄魯特策妄阿拉布坦犯哈密,上遣尚書富寧安等率師討之。五十五年,命額倫特署西安將軍,主軍餉。策妄阿喇布坦自噶順汛山後道沙拉侵青海,執臺吉羅卜藏丹濟布以去,命額倫特率師駐西寧,為青海諸部應援。五十六年,策妄阿拉布坦遣其將策凌敦多布侵西藏。命額倫特移軍青海,與青海王臺吉等議屯軍形勝地。額倫特疏言西寧入藏道有三,庫庫賽爾嶺、拜都嶺道皆寬廣,請與侍衛色楞分道進兵。五十七年,策凌敦多布入西藏,破布達拉城,戕拉藏汗,執其子蘇爾咱,遂據有其地。六月,額倫特與色楞分道進兵,額倫特出庫庫賽爾嶺。七月,至齊諾郭勒,策凌敦多布遣兵夜來侵,擊之退。次日復至,額倫特親督兵緣山接戰,賊潰遁,追擊十餘里,多所斬獲。疏入,上深嘉其勇。俄,策凌敦多布遣兵潛出喀喇烏蘇,額倫特率所部疾趨渡河,扼狼拉嶺,據險禦敵。比至喀喇烏蘇,色楞以兵來會,合力擊賊。賊數萬環攻,額倫特督兵與戰,被重創,戰益力。相持者數月。九月,復厲兵進戰,射殺賊甚眾。矢盡,持刀麾兵斫賊,賊益兵合圍,額倫特中傷,猶力戰,遂沒於陣。五十八年,喪還,上命諸王以下迎城外,內大臣、侍衛至其家奠茶酒。世宗即位,進世職三等阿達哈哈番,賜祭葬,謚忠勇。

額倫特與川陜總督音泰皆自行伍中為上所識拔。額倫特以廉潔著,上嘗與張伯行並稱,謂在督撫中操守最優也。

康泰,甘肅張掖人。初入伍,累擢至游擊。從將軍孫思克擊噶爾丹,以功授世職拖沙喇哈番。四遷四川提督。額倫特駐西寧,泰率松潘兵千餘出黃勝關為應援。兵譟,奪官,命自具鞍馬從軍。從額倫特入藏,戰喀喇烏蘇,躍馬殺賊,矢集於臂,叱其子拔矢,裹臂復戰,陣沒。贈都督同知,謚壯勇。

弟海,陜西涼州總兵。將所部從額倫特,同時戰死。贈世職拖沙喇哈番。

論曰:厄魯特亦出於蒙古,析為四衛拉特,其一曰綽羅斯,牧伊犁。噶爾丹戕兄子自立,乃號準噶爾,移帳阿爾泰山,兼有四衛拉特。北侵喀爾喀,南侵衛藏。聖祖再親征,乃摧敗以死。烏闌布通之役,噶爾丹敗遁,我軍亦重衄。佟國綱以元舅死綏。及戰昭莫多,費揚古麾饑疲之眾,當困鬥之寇,蹈瑕以破堅,則謀勇勝也。馬斯喀、阿南達、吉勒塔布、化行、育龍先後在事有勞。額倫特孤軍殉寇,青海之師,準部之滅,皆於是乎起。謹書之以著其本末。


\end{pinyinscope}