\article{志九十七}

\begin{pinyinscope}
○食貨三

△漕運

清初,漕政仍明制,用屯丁長運。長運者,令瓜、淮兌運軍船往各州縣水次領兌民,加過江腳耗,視遠近為差;而淮、徐、臨、德四倉仍系民運交倉者,並兌運軍船,所謂改兌者也。逮至中葉,會通河塞,而膠萊故道又難猝復,借黃轉般諸法行之又不能無弊,於是宣宗採英和、陶澍、賀長齡諸臣議復海運,遴員集粟,由上海雇商轉船漕京師,民咸稱便。河運自此遂廢。夫河運剝淺有費,過閘過淮有費,催趲通倉又有費。上既出百餘萬漕項,下復出百餘萬幫費,民生日蹙,國計益貧。海運則不由內地,不歸眾飽,無造船之煩,無募丁之擾,利國便民,計無逾此。洎乎海禁大開,輪舶通行,東南之粟源源而至,不待官運,於是漕運悉廢,而改徵折漕,遂為不易之經。今敘次漕運,首漕糧,次白糧,次督運,次漕船,次錢糧,次考成,次賞恤,而以海運終焉。

漕運初悉仍明舊,有正兌、改兌、改徵、折徵。此四者,漕運本折之大綱也。順治二年,戶部奏定每歲額徵漕糧四百萬石。其運京倉者為正兌米,原額三百三十萬石:江南百五十萬,浙江六十萬,江西四十萬,湖廣二十五萬,山東二十萬,河南二十七萬。其運通漕者為改兌米,原額七十萬石:江南二十九萬四千四百,浙江三萬,江西十七萬,山東九萬五千六百,河南十一萬。其後頗有折改。至乾隆十八年,實徵正兌米二百七十五萬餘石,改兌米五十萬石有奇,其隨時截留蠲緩者不在其例。山東、河南漕糧外有小麥、黑豆,兩省通徵正兌。改耗麥六萬九千五百六十一石八斗四升有奇,豆二十萬八千一百九十九石三斗一升有奇,皆運京倉。黑豆系粟米改徵,無定額。凡改徵出特旨,無常例。

折徵之目有四:曰永折,曰灰石米折,曰減徵,曰民折官辦。永折漕糧,山東、河南各七萬石,石折銀六錢、八錢不等;江蘇十萬六千四百九十二石有奇,石折銀六錢不等;安徽七萬五千九百六十一石有奇,石折銀五錢至七錢不等;湖北三萬二千五百二十石,湖南五千二百十有二石各有奇,石均折銀七錢。其價銀統歸地丁報部。灰石改折,江蘇二萬九千四百二十四石,浙江萬八千六百五十三石,遇閏加折四千十有五石,石折銀一兩六錢,以供工部備置灰石之用,自順治十七年始也。

次年,飭江南、浙江、江西三省大吏,凡改折止許照價徵收,如藉兌漕為名,濫行科索者,即行參勘。又以蘇、松、常、鎮四府差(每系)賦重,漕米每石折銀一兩,其隨漕輕齎席木贈截等銀,仍徵之耗米,及給軍行月贈耗等米,亦按時價折徵。康熙八年,定河南漕糧石折銀八錢。九年,浙江嘉、湖二府被災,每石折徵一兩。五十八年,覆準河南附近水次之州縣,額徵漕糧每石八錢內,節省銀一錢五分,仍令民間上納,餘六錢五分,令徵本色起運。至距水次較遠及不近水次之州縣,額徵米石,仍依舊例徵銀八錢,以一錢五分解部,餘交糧道採辦米石。雍正元年,以嘉、湖二屬州縣災,諭令收徵漕米本折各半,其折價依康熙九年例。六年,議定河南去水次稍遠州縣,均徵本色,惟南陽、汝寧二府屬,河南府之盧氏、嵩、永寧三縣及光、汝二州並屬縣,又離水次最遠之靈寶、閿鄉,路遠運艱,共酌減米萬五千六十二石有奇,免其辦解,分撥內黃、濬、滑、儀封、考城等五縣協辦,於五縣地丁銀內扣除完漕,照部價每石八錢,以六錢五分辦運,節省之一錢五分,徵解糧道補項。其南、汝等府屬,每石折銀八錢解司,以抵濬、滑等五縣地丁銀數,所謂減徵是也。

乾隆二年,以大濬運河,江蘇淮安之山陽、鹽城、阜寧,揚州之江都、甘泉、高郵、寶應各

縣漕糧,每石徵折銀一兩。其後海州、贛榆兩邑亦然。山東、河南向所改徵黑豆,不敷支給,河南再改徵二萬石,山東四萬石。三年,湖廣總督德霈言湖南平江距水次五百餘里,請改折色,分撥衡陽、湘潭代買兌運,從之。七年,江西瀘溪以折價八錢不敷採買,定嗣後每年八月借司庫銀撥縣採買,照買價徵銀歸還。其後江蘇之嘉定、寶山、海州、贛榆,安徽之寧國、旌德、太平、英山,湖北之通山、當陽諸州縣,悉遵此例。十一年,定河南祥符等四十州縣額徵粟米內,每年改小麥萬石,與漕米黑豆並徵運通。

十六年,以京師官兵向養馬駝,需用黑豆,豫、東二省自雍正十年以來,於漕糧粟米內節次改徵,每年額解黑豆二十萬九千餘石,每省酌量再改徵黑豆一二萬石。尋定山東三萬石,河南二萬石,額徵粟米,照數除抵,其節省銀一錢五分為運腳之用者並徵之。十八年,倉場侍郎鶴年言:「現在京倉黑豆六十萬餘石,足供三年支放,請自明年始,豫、東二省應運黑豆,酌半改徵粟米,分貯京、通各倉,則豆無潮黰之虞,粟價亦平。」從之。

二十六年,以江蘇之清河、桃源、宿遷、沭陽不產米粟,命嗣後先動司庫銀兩,按照時價採辦,令民輸銀還欸,是謂民折官辦。其後阜寧、旌德、泰興、寧國、太平、英山諸縣皆仿行之。

二十一年諭曰:「漕糧歲輸天庾,例徵本色。勒收折色,向干嚴禁。現值年豐穀賤,若令小民以賤價糶穀,交納折色,是閭閻終歲勤劬,所得升斗,大半糶以輸官,以有限之蓋藏,供無窮之朘削,病民實甚。著通諭有漕省分大吏,飭所屬徵收糧米,概以本色交納,無許勒折滋弊。如有專利虐民者,據實嚴參。」然州縣往往仍藉改折浮收,雖有明令,莫能禁也。

正兌、改兌、改折之外,復有截漕及撥運。各省截留漕船,介於起運停運之間,行月二糧,應給應追,向無定例。自乾隆元年,議定江蘇、安徽、浙江截留漕船應支本折月糧三修銀,照數全給。至行糧盤耗贈銀負重等項,按站發給。若幫船截留本次,或旋兌旋卸,或數月後清,贈米亦按月計算。江西船大載重,每年三修銀不敷,則取辦於行月二糧。遇有截留,將原領折耗行月贈銀贈米斛面米均免扣追。嗣以運軍掛欠之項,諭將雍正十二年以前各省截留漕船應追等項悉免之。七年,以各省截留漕船已兌開行,例須扣追,酌定加給,視程途遠近、船糧多寡為衡。山東、河南每船給銀五十兩。江南、浙江六十兩,湖廣七十兩,江西九十兩,以充各軍在次修船置備器具,及雇募舵工水手安家養贍之用。其應給之銀,即於行月折色銀內扣給。十八年,諭曰:「前命截留南漕二十萬分貯天津水次各倉備用,但恐旗丁等於米色斛面任意攙和短少,而州縣胥役又往往藉端勒索,令方觀承飭天津道親往監看。嗣後截漕之省,俱派就近道員稽查,不得委州縣。著為令。」

撥運者,截留山東、河南所運薊州糧,撥充陵糈及駐防兵米者也。康熙三十四年,議定年需粟米三萬六百餘石,將山東漕糧粟米照數截留,以原船自天津運至新河口,撥天津紅剝船百五十艘,運至薊州五里橋,船載百石,每百里給腳價一兩三錢二分,所需之銀,於過閘入倉腳價內撥給。四十五年,定密雲駐防兵米,在豫、東二省每年徵存薊糧項下撥運,令該縣於春夏之交,赴通領運收倉。平時由水運,有故則陸運。腳價由地糧銀內給發。次年,令豫、東各添撥米百石,備支銷折耗。又撥運保定、雄縣兩處駐防兵米,截至西沽就船受兌,以節耗費。嘉慶初,因東省輪免漕糧,先令豫省兌運,不敷之數,許動支節年倉存薊米,並動碾公穀。其後河南被災,亦準在薊倉存米存穀內碾動。其各州縣派撥之數,薊州五萬八千六百石、易州三萬八千六百石各有奇,密雲一萬一千五百餘石,保定、雄縣共三千一百餘石,良鄉暨大興之採育三百餘石,順義、昌平二百餘石,霸州、東安、固安、寶坻三百餘石,玉田及遷安之冷口各五百餘石,滄州二千七百餘石。又青州駐防兵米二千一百餘石,亦於薊糧內截留運供,德州駐防兵米不敷,亦得動支。此撥運之大略也。

各省之徵收漕糧也,向系軍民交兌,運軍往往勒索擾民。順治九年,始改為官收官兌,酌定贈貼銀米,隨漕徵收,官為支給。雍正六年,以江、浙應納漕糧為額甚巨,若必拘定粳米,恐價昂難於輸將,以後但擇乾圓潔凈,準紅白兼收,秈稉並納,著為令。乾隆初,奏定民納漕米,隨到隨收,嚴禁蠹書留難。四年,諭曰:「朕聞湖北糧米,以十五萬一千餘石運赴通倉,名曰北漕,十二萬六千餘石為荊州官米,名曰南漕,二項原可合收分解。乃有不肖州縣,分設倉口,令糧戶依兩處完納,以圖多得贏餘,重累吾民。著行文該省,將二項漕糧合收,永遠遵行。」七年,定直省有漕各屬,於隔歲年終,刊易知由單,條悉開載,按戶分給,以杜濫科。十年,工部侍郎範燦奏:「江南下江徵收漕米,向借漕費之名,或九折,或八折,自巡撫尹繼善定每石收費六分,諸弊盡革。久之,吏胥復乘緊兌之際,多方刁難,小民勢難久待,不得不議扣折。」諭飭有漕省分大小官吏,嚴行釐剔積弊。嘉慶八年,禁止各州縣漕糧私收折色,及刁生劣監收攬包交。

凡漕糧皆隨以耗費,耗皆以米,正兌一石耗二斗五升至四斗,改兌一石耗一斗七升至四斗,皆隨正入倉,以供京、通各倉並漕運折耗之用。其南糧又有隨船作耗米,自五升至二升三升不等,以途之遠近為差。嘉慶間,定江蘇漕糧耗米原備篩颺,耗米四升有奇。嗣後以二升餘劃付旗丁,二升隨糧交倉。浙江、江西、兩湖悉依此例。逮漕務改章,凡改徵折色各省,耗米亦折價與正米並徵,自是漕耗之名遂廢。

初,各省漕糧改為官收官兌,贈貼名稱,山東、河南謂之潤耗,江蘇、安徽謂之漕貼,浙江謂之漕截,江西、兩湖謂之貼運,其數多寡不一,隨糧徵給,均刊列易知由單,私派挪移者罪之。其後江南每糧百石,竟私截至百餘兩,浙江至三十餘兩。糧道劉朝俊以貪婪漕貼萬二千餘兩被劾,給事中徐旭齡亦疏陳贈耗之弊。然貪官污吏,積習相沿,莫能禁也。康熙十年,議定江寧等府起運耗米及正糧一體貼贈,蘇、松、常三府改折灰石,幫貼漕折等銀悉免之。二十四年,令各省隨漕截銀免解道庫,徑令州縣給發。乾隆七年,定江南漕米贈耗永免停支例。各省收漕州縣,除隨正耗米及運軍行月糧本折漕贈等項外,別收漕耗銀米,其數亦多寡不一,此項耗外之米,皆供官軍兌漕雜費及州縣辦公之用者也。

輕齎銀者,始於有明中葉。以諸倉兌運,須給路費,徵耗米,兌運米一平一銳,其銳米量取隨船作耗,餘皆折銀,名曰輕齎。清因之。每年正兌米一石,江西、兩湖諸省加耗四斗六升或六斗六升,銳米皆一斗。加耗四斗六升者,則以三斗隨船作耗,而以連銳二斗六升折銀一錢三分;加耗六斗六升者,則以四斗隨船作耗,而以連銳三斗六升折銀一錢八分,謂之三六輕齎。江蘇、安徽每石加耗五斗六升,銳米一斗,除四斗隨船作耗,而以餘米二斗六升折銀一錢三分,謂之二六輕齎。山東、河南每石加耗三升,銳米一斗,除二斗五升隨船作耗,餘米一斗六升折銀八分,謂之一六輕齎。其改兌止有耗米,或三斗二升至一斗七升不等,止給本色隨船作耗,而以存米二升易銀一分,謂之折易輕齎。均每升折徵銀五釐,解倉場通濟庫。康熙四十七年,令每年江南等省額解輕齎銀三十八萬四千兩,內除山東、河南、湖廣、江西、浙江、江南等省額解銀二十四萬六千九百餘兩,仍留通濟庫應用,其蘇松糧道所屬額解銀十三萬七千餘兩,徑解戶部。如倉場不敷,得咨行戶部支發。尋分撥蘇松糧道所屬額解輕齎銀五萬分解通濟庫備用。用此項輕齎銀,例應兌漕通以濟運務,外此有席木竹板等存,皆隨漕交納,其尺寸長短廣狹,均有定制。

道光二十九年,兩江總督李星沅奏南漕改折,戶部定價太輕,開不肖州縣浮勒之端。江蘇巡撫陸建瀛亦言其不便。遂罷改徵折色。同治四年,曾國籓、李鴻章請將江蘇鎮洋、太倉二州縣漕糧改徵折色,不許。光緒十年,翰林院侍讀王邦璽疏陳丁漕有五弊、三難、五宜、三不可。是時直省丁漕積欠頻仍,故邦璽以為言。二十三年,侍講學士瑞洵言南漕改折,有益無損。先是江、浙漕米,除河運十二三萬石外,歲約海運百二十餘萬。二十年,辦理海防,江、浙各省各折十之五六。翌年,兩江總督張之洞擬令蘇省州縣收折收本仍其舊,而由官全行折解。部令仍運本色。張之洞復奏,蘇漕全折,歲可省運費八十萬,浙江全折,兩湖採買全停,剝船挑河各費、漕職衛官各項,均可酌減,歲可省百五十萬。嗣戶部以庫儲支絀,請將江蘇海運漕糧暫減運三十萬石,得銀九十八萬餘兩。奕劻等奏言:「南漕歲有定額,兵民生計攸關,京師根本重地,尤須寬為儲備。言者動稱折漕歲五六百萬,實則不過百餘萬有奇,似不宜輕議更張。」從之。

漕糧之外,江蘇蘇、松、常三府,太倉一州,浙江嘉、湖兩府,歲輸糯米於內務府,以供上用及百官廩祿之需,謂之白糧。原額正米二十一萬七千四百七十二石有奇。耗米,蘇、松、常三府,太倉一州每石加耗三斗,以五升或三升隨正米起交,餘隨船作耗,共二萬七百七石有奇;嘉、湖二府每石加耗四斗,以五升或三升隨正米起交,餘隨船作耗,共萬三千四百八十八石有奇。康熙初,定白糧概徵本色,惟光祿寺改折三萬石,石徵銀一兩五錢。十四年,議定江南白糧仿浙省例,抽選漕船裝運,每船給行月糧米六十九石三斗,銀五十六兩七錢六分。經費銀,浙江舊例四百五十七兩一錢一釐,議減去銀百二十六兩二錢四分、米二十八石。嗣以運漕、運白事同一體,裁江、浙白糧經費,仿漕糧之例,支給行贈銀兩。至白糧悉系包米運送,並無折耗,俟抵通照例交收。

先是江、浙輸將白糧二十二萬餘石,太常寺、光祿寺各賓館需用二千餘石,王公官員俸約需十五六萬石,內務府、紫禁城兵卒及內監食用需一萬石,尚餘五萬石。乾隆二年,高宗謂:「光祿寺等處收支,原以供祭祀及賓館之用,在所必需。其王公百官俸米,應用白糧酌減其半,以粳米抵充。至賚賞禁城兵卒及內監米石,應將白糧易以粳米,以紓民力。」自是實徵白糯不過十萬石有奇矣。又準松江、太倉額徵白糯,改徵漕糧,即在派運白米十萬石內通融盈縮,以均應減應運之數。浙江向不產糯,白糧中糯米一項,隨漕統徵糙粳,官為易糯兌運。兩省白糧經費前已議裁,至是復照舊例徵收。江蘇徵銀十八萬六千九百八十五兩有奇,米萬八千八百八十九石有奇,舂辦米二萬一千三百九十九石有奇,浙江徵銀四萬五千七十五兩有奇,米三千九百六十九石,舂辦米萬三千二百九十石有奇,共實徵銀二十三萬二千六十一兩,米五萬五千七百四十八石有奇。除給運弁運軍、並解通濟庫為運送京、通各倉腳價之用,餘銀及米折,均造冊送部酌撥。逮嘉慶中,白糧經費,江蘇徵銀六萬餘兩,米及舂辦米各萬餘石,浙江徵銀五萬餘兩,米三千餘石,舂辦米萬餘石,共實徵銀十一萬四千五百十八兩有奇,米五萬三千七百二十九石有奇,較之乾隆時經費銀所減又逾半矣。

江、浙之運白糧也,初沿明代民運之制。嗣以臨期雇募民船,時日稽遲,改行官運;仍不便民,乃令漕船分帶,以省官民之累。康熙三年,定浙江行漕帶法,需船百二十六艘,於漕幫內抽出六十二艘裝運,增造六十四艘並入僉運,後江蘇亦踵行之。每船裝運五百石,擇軍船殷實堅固者裝運,五年一易。制定每年未兌之前,責令糧道赴次查驗,如運軍力疲、船不堅固者,別選殷軍補運。十六年,漕運總督瑚寶奏:「江蘇運白糧船向例五年更調,但為時過久,請依漕船三年抽調例,定運白三年即行另選。」從之。江、浙兩省運白糧船,原定蘇州、太倉為一幫,松江、常州各為一幫,嘉興、湖州各一幫,領運千總每幫二,隨幫武舉一。改行官運後,以府通判為總部,縣丞、典史為協部,吏典為押運。旋裁押運。後白糧改令漕船帶運,復裁總、協二部。蘇、松、常每府增設千總二,更番領運,每幫設隨幫百總一,押趲回空。浙江增設千總四、隨幫二,蘇州、太倉倉運白糧船,原定百十八艘,船多軍眾,分為前後兩幫,增設千總二、隨幫一。白糧減徵後,並兩幫為一,其千總隨幫悉予裁減。

清初,都運漕糧官吏,參酌明制。總理漕事者為漕運總督。分轄則有糧儲道。監兌押運則有同知、通判。趲運則有沿河鎮道將領等官。漕運總督駐淮南,掌僉選運弁、修造漕船、派撥全單、兌運開幫、過淮盤掣、催趲重運、查驗回空、覈勘漂流、督催漕欠諸務,其直隸、山東、河南、江西、江南、浙江、湖廣七省文武官吏經理漕務者皆屬焉。糧道,山東、江安、蘇松、江西、浙江、湖北、湖南各一。河南以開歸鹽驛道兼理。糧道掌通省糧儲,統轄有司軍衛,遴委領運隨幫各官,責令各府清軍官會同運弁、僉選運軍。兌竣,親督到淮,不得委丞倅代押。如有軍需緊要事件,須詳明督撫、漕臣,方許委員代行其職務。

監兌,舊以推官任之。推官裁,改委同知、通判。山東以武定同知,東昌清軍同知,濟南、兗州、泰安、曹州四通判,濟寧、臨清兩直隸州同;河南以歸德、衛輝、懷慶三通判;江南以江寧、蘇州督糧同知,松江董漕同知,鳳陽同知,蘇州、揚州、廬州、太平、池州、寧國、安慶、常州八管糧通判,太倉州臨時添委丞倅一;浙江以湖州同知,杭州局糧通判,嘉興通判;江西以南昌、吉安、臨江三通判;淮北、湘南每年於通省同知、通判內詳委三員,監兌。江西、湖廣、安徽監兌押淮之員尋裁。

凡開兌,監兌官須坐守水次,將正耗行月搭運等米,逐船兌足,驗明米色純潔,面交押運官。糧船開行,仍親督到淮,聽總漕盤驗。糧數不足、米色不純者,罪之。道、府、不揭報,照失察例議處。意存袒護,照徇庇例議處。

押運本糧道之職,但糧道在南董理運務,無暇兼顧。江、浙各糧道,止令督押到淮盤驗,即回任所。總漕會同巡撫遴委管糧通判一,專司督押,約束運軍,防範侵盜攙和等弊。山東、河南通判各一,江南七,浙江三,江西二,湖北、湖南各一。後因通判官卑職微,復令糧道押運。其漕船回空,仍令通判管押。過淮必依定限,如有遲誤,照重運違限例議處。江南、浙江、江西尋復通判押運之制。

押運同知、通判抵通後,出具糧米無虧印結,由倉場侍郎送部引見。糧道押運三次,亦準督撫咨倉場侍郎送部引見。其員弁紳董隨同押運到通,並準擇尤保獎,以昭激勸。其後各省大吏往往藉漕運保舉私人,朝廷亦無由究詰也。

淮北、淮南沿河鎮道將領,遇漕船入境,各按汛地驅行,如催趲不力,聽所在督撫糾彈。江南京口、瓜洲渡江相對處,令鎮江道督率文武官吏催促,並令總兵官巡視河干,協催過江。總兵裁,改由副將管理。雍正三年,巡漕御史張坦麟條上北漕事宜:一,自通抵津,沿河舊汛窵遠,請照旱汛五里之例,漕船到汛,催漕官弁坐視阻抵不行申報者,依催趲不力例參處;一,沿途疏淺約十三四處,坐糧難以兼顧,請交各汛弁率役疏通,應銷錢糧,仍令坐糧管理。從之。巡漕御史伊喇齊疏劾河南糧道提催之弊,巡撫尹繼善亦疏請革除各州縣呈送監兌押運官役陋規。凡漕船回空到省,未開兌之前,責成本省巡撫及糧道,既開兌出境,則責成漕督及沿途文武官吏,抵津後,責成倉場侍郎、坐糧及天津總兵、通州副將,嚴行稽查。有違犯者,捕獲懲治。

四十八年,漕督毓奇言:「各省督押,惟山東糧道抵通,餘祗押抵淮安。嗣後各省重運,俱令糧道督押本幫至臨清,出具糧米無虧印結,即行回任。其自臨清抵通,概令山東糧道往來催趲。山東運河,每年十一月朔煞壩挑淺。開壩之日,以南省漕船行抵臺莊為準。微山等湖收蓄眾泉,為東省濟運水櫝,不許民間私截水源。徬河遇春夏水微,務遵漕規啟閉。漕船到徬,須上下會牌俱到,始行啟板。如河水充足,相機啟閉,以速漕運,不得兩徬齊啟,過洩水勢。其在江中偶遇大風,原可停泊守候,而催漕官吏惟知促迫,軍船冒險進行,恆有漂沒之虞。回空之船,管運員及運丁等恆意存怠玩,或吝惜雇價,將熟習舟子遣散,留不諳駕馭之人,而押運員弁每先行回署,並不在船督率,往往有運船失風之事。」上諭飭「沿途各員催趲,應察風色水勢,毋得過於急迫,至涉險失事,亦不得因此旨遂任意逗留,致逾定限」。初,運河中銅鉛船及木排,往往肆意橫行,民船多畏而讓之。糧船北上,亦為所阻。至是令巡漕御史轉飭沿途文武員弁,將運漕船催趲先行,餘船尾隨,循次前進,恃強爭先、不遵約束者,罪之。

領運員弁,各省糧船分幫,每幫以衛所千總一人或二人領運,武舉一人隨幫效力。順治六年,奏定就漕運各衛中擇其才幹優長者授職千總,責其押運,量功升轉,掛欠者治罪追償。其後裁衛所外委百總,改為隨幫官。康熙五十一年,揀候選千總三十員,發南漕標效力,如有領運千總員缺,聽總漕委署押運,果能抵通全完,倉場總督咨送兵部,準其即用。揀選武舉,候推守衛所千總有原補隨幫者,可在總署處呈明,遇缺準其頂補,三年無誤,以衛千總推用。雍正二年,漕運總督張大有奏稱山東、河南輪運薊州、遵化、豐潤官兵米石,沿途管押及回空催趲,例責成押運通判,請添設薊糧千總二,更番領運,從之。各衛既有千總領運,而漕臣每歲另委押運幫官,分為押重押空,一重運費二三千金,一空運費浮於千金,幫丁之脂膏竭,而浮收之弊日滋矣。嘉慶十二年,諭漕督不得多派委員,並禁止運弁等收受餽贈。十四年,巡漕御史又請大加減省。自咸豐三年河運停歇,船只無存,領運之名亦廢。

巡漕御史本明官,順治初省。雍正七年,以糧船過淮陋規甚多,並夾帶禁物,遣御史二,赴淮安專司稽察。糧船抵通,亦御史二稽察之。乾隆二年,設巡漕御史四:一駐淮安,巡察江南江口至山東交境;一駐濟寧,巡察山東臺莊至北直交境;一駐天津,巡察至山東交境;一駐通州,巡察至天津。凡徵收漕糧,定限十月開倉,十二月兌畢。惟山東臨清徬內之船,改於次年二月兌開,依限抵通,徬外之船,仍冬兌冬開。乾隆間,令徬內徬外一律春兌春開,從漕督楊錫紱請也。嘉慶四年,諭曰:「冬兌冬開,時期促迫。嗣後東省漕糧,仍照舊例起徵,運赴水次,立春後兌竣開幫,翌年改為冬兌春開。」十五年,令徬河內外幫船,照春兌春開例辦理。江北冬漕,定於十二月朔開兌,限次年二月兌竣開行。

凡漕兌,首重米色。如有倉蠹作奸,攙和滋弊,及潮濕霉變,未受兌前,責成州縣,既受兌後,責在弁軍,覈驗之責,監兌官任之。如縣衛因米色爭持,即將現兌米面同封固,送總漕巡撫查驗,果系潮濕攙雜,都令賠換篩颺,乃將米樣封送總漕,俟過淮後,盤查比較,分別糾劾。然運軍勒索州縣,即借米色為由。州縣開倉旬日,米多廒少,勢須先兌。運軍逐船挑剔,不肯受兌,致糧戶無廒輸納,因之滋事。運軍乘機恣索,或所索未遂,船竟開行,累州縣以隨幫交兌之苦。及漕米兌竣,運弁應給通關。通關出自尖丁。尖丁者,積年辦事運丁也,他運丁及運弁皆聽其指揮。尖丁索費州縣,不遂其欲,則靳通關不與,使州縣枉罹遲延處分。運軍運弁沆瀣一氣,州縣惟恐誤兌,勢不得不浮收勒折以供其求。上官雖明知其弊,而憚於改作。且慮運軍裁革,遺誤漕運,於是含容隱忍,莫之禁詰。州縣既多浮收,則米色難於精擇。運軍既有貼費,受兌亦不復深求。及至通州,賄賣倉書經紀,通挪交卸,米色潮濕不純之弊,率由於此。積重難返,而漕政日壞矣。乾隆間,漕運總督顧琮條上籌辦漕運七事:一,州縣親收漕糧,以免役胥藉端累民;一,杜匿富僉貧包丁代運之弊;一,受未開之幫船催令速行;一,糧船過淮後,分員催趲,以速運漕;一,河道舊有橫淺,豫為疏濬,以免阻滯;一,各閘俱照漕規,隨時啟閉,江、廣漕船攜帶竹木,限地解卸;一,回空三升五合餘米,速給副丁,以濟回時食用。詔從其議。

各省漕糧過淮,順治初,定限江北各府州縣十二月以內,江南江寧、蘇、松等處限正月以內,江西、浙江限二月以內,山東、河南限正月侭數開行。如過淮違誤,以違限時日之多寡,定督撫糧道監兌推官降罰處分。領運等官,捆打革職,帶罪督押。其到通例限,山東、河南限三月朔,江北四月朔,江南五月朔,江西、浙江、湖廣六月朔。各省糧船抵通,均限三月內完糧,十日內回空。倉場定立限單,責成押幫官依限到淮,逾限不能到次,照章糾劾。

承平日久,漕弊日滋。東南辦漕之民,苦於運弁旗丁,肌髓已盡,控告無門,而運弁旗丁亦有所迫而然。如漕船到通,倉院、糧、戶部云南司等處投文,每船需費十金,由保家包送,保家另索三金。又有走部,代之聚斂。至於過壩,則有委員舊規,伍長常例,上斛下蕩等費,每船又須十餘金。交倉,則有倉官常例,並收糧衙署官辦書吏種種需索,又費數十金。此抵通之苦也。逮漕船過淮,又有積歇攤派吏書陋規,投文過堂種種費用。總計每幫漕須費五六百金或千金不等。此過淮之苦也。從前運道深通,督漕諸臣只求重運如期抵通,一切不加苛察。各丁於開運時多帶南物,至通售賣,藉博微利。乾隆五十年後,黃河屢經開灌,運道日淤,漕臣慮船重難行,嚴禁運丁多帶貨物,於是各丁謀生之計絀矣。運道既淺,反增添夫撥淺之費,每過緊要閘壩,牽挽動須數百人,道路既長,限期復迫,丁力之敝,實由於此。雖經督撫大吏悉心調劑,無如積弊已深,迄未能收實效也。

各省漕船,原數萬四百五十五號。嘉慶十四年,除改折分帶、坍荒裁減,實存六千二百四十二艘。每屆修造十一,謂之歲造,其升科積缺漂沒者,謂之補修改造,限以十年。至給價之多寡,視時之久暫、地之遠近為等差。造船之費,初於民地徵十之七,軍地徵十之三,備給料價。不足,則徵軍衛丁田以貼造漕船。十年限滿,由總漕親驗,實系不堪出運,方得改造,有可加修再運者,量給加修銀,仍令再運。按年計算,舊船可用,不驗明駕運,督撫查實糾劾。司修造漕船各官,或詐朽壞,或修造未竣詐稱已完,或將朽壞船冊報掩飾,或承造推諉不依限竣工,或該管官督催不力,及朽壞船不估價申報,均降罰有差。

直隸、山東、鳳陽地不產木,於清江關設廠,由船政同知督造。江寧各幫共船千二百餘,亦於清江成造。自儀徵逆流抵淮,四百餘里,沿途需用人夫挽曳,船成後復渡大江,道經千里,到次遲延,縣官急於考成,旗丁利於詐索,船未到即行交兌,名曰轉廒,於是贈耗、使費、賠補、苛索諸弊日滋,運軍苦之。嗣裁船政同知,統歸糧道管理,令運軍支領料價赴廠成造,不敷,即於道庫減存漕項銀內動支。徐州衛、河南後幫漕船,向亦在清江船廠成造,駕赴河南水次兌糧,程途遼遠,易誤兌限。尋改在山東臨清設廠成造。遇滿號之年,令各軍於江、安道庫銀內領價成造。其濟南前幫,則在江南夏成鎮成造,嗣又改於臨清胡家灣設廠。

船成查驗之法九:一驗木,二驗板,三驗底,四驗樑,五驗棧,六驗釘,七驗縫,八驗艙,九艙頭梢。山東各幫於額運漕船外,向設量存船三十。江蘇揚州亦有量存船二十四。先後議裁,並將揚州衛應裁之船,抵補江、興二衛貧疲軍船。乾隆八年,漕運總督顧琮上漕船變通事宜:一,漕船當大造之年,遇有減歇,即停造一年,與先運之船年限參差,將來無須同時配造;一,賠造之船已出運多次,恆欠堅固,嗣後將賠造接算原船,已滿十年尚能出運者,準其將船在通售賣;一,滿號之船,向俱分年抽造,其中堅固者,交總漕擇令加修,出運一次,許其流通變賣。從之。二十九年,漕督楊錫紱言:「各省漕船當十運屆滿應行成造之年,如運糧抵通,準在通變價。再買補之船未經滿運,或中途猝遇風火,請準就地折變。」詔從其議。大河、淮安等幫漕船,恆有遭風沈溺之事。阿桂奏稱,因船過高大,掉挽維艱所致,請較原定尺寸酌量減小。嘉慶十五年,復酌減江、廣兩省漕船尺寸。運丁利於攬載客貨,船身務為廣大,不知載重則行遲,行遲則壅塞,民船被阻,甚有相去數丈守候經旬者;兼之強拏剝運,捶撻交加,怨聲載道,不僅失風之虞也。十七年,以浙省成造漕船賠累日甚,每船除例給二百八兩外,復給銀五百九十餘兩,以紓丁力。漕船建造修葺,其費有經常,有額外,年糜國帑數十百萬。及其出運,勒索於州縣者又數十百萬。催趲迎提,終歲勞攘,夾帶愈多,雖蘇、松內河,亦無歲不剝運。剝運仍責舟於沿途,甚至攔江索費,奪船毀器,患苦商民,抗違官長,以天庾為口實,援漕督為護符,文武吏士,畏其勢焰,莫或究詰。

凡漕船載米,毋得過五百石。正耗米外,例帶土宜六十石,雍正七年,加增四十,共為百石,永著為例。旋準各船頭工舵工人帶土宜三石,水手每船帶土宜二十石。嘉慶四年,定每船多帶土宜二十四石。屯軍領運漕糧,冬出冬歸,備極勞苦,日用亦倍蓰家居,於是有夾帶私貨之弊。漕船到水次,即有牙儈關說,引載客貨,又於城市貨物輻輳之處,逗留遲延,冀多攬載,以博微利。運官利其餽獻,奸商竄入糧船,藉免國課。其始運道通順,督漕諸臣不事苛察。逮黃屢倒灌,運道淤淺,漕臣嚴申夾帶之禁,丁力益困。

當商力充裕時,軍船回空過淮,往往私帶鹽斤。漕運總督張大有條上六事:一,長蘆、兩淮產鹽之處,奸民勾串灶丁,私賣私販,伺回空糧船經過,即運載船中,請嚴行禁止,違者俱依私鹽例治罪;一,糧船回空時,請於瓜洲、江口派瓜洲營協同員搜查;一,運司等官拏獲私鹽,請依專管兼轄官例議敘;一,隨幫官專司回空,有能拏獲私鹽三次及幫船三次回空無私鹽事者,以千總推用;一,每船量帶食鹽四十斤,多帶者以私鹽例治罪;一,例帶土宜之外,包攬商船木筏者,照漏稅例治罪,貨物入官。自是禁網益密矣。幫丁困苦,爰有津貼之議。江蘇漕船,以松江幫丁力為最疲。定例松、太等屬每船津貼銀三百兩,旋加為五百兩。幫丁視為額給之項,仍欲另議津貼,開船遲延,州縣恐貽誤獲譴,恆私食鬼之,以致津貼日增,流弊無已。

漕運抵通及遇淺,皆須用剝船。清初設紅剝船六百艘,每船給田四十頃,收租贍船,免其徵科。近畿州縣距河甚遠,恆雇覓民船,河干游民藉之邀利,及接運漕糧,往往有盜賣攙和之弊,甚有盜賣將盡,故傾覆其船,逮運官查明,仍責地戶賠償,傾家蕩業。又領船船戶例受天津鈔關部差管轄,每歲河冰未泮之日,部差催促過堂守候,莫不有費,苦累實甚。三十九年,裁紅剝船,依原收租數分派各省,於漕糧項下編徵,解糧道庫支發。乾隆二年,定每船給紅剝銀二兩,由隨幫千總領發,漕船遇淺,由運軍自雇民船,坐糧酌定雇價。十三年,增設★船六十艘,造船及用具夫役工食,均於紅剝銀內支用,餘仍分給運軍。南糧入北河後,官為雇船剝運,糧艘未到,剝船先期預備,守候累日,且有妨商鹽挽運。五十年,諭令另造剝船,南糧抵北河,即剝運赴通,嗣後毋得封固民船,致滋擾累,違者罪之。尋議定官備剝船千二百艘,發交附近沿河天津等十八州縣收管,如有商貨鹽斤,許其攬載,四月以後,調赴水次,毋得遠離。翌年復添造三百隻,交江西、湖廣成造,運送天津,與原設剝船在楊村更番備剝。豫、東二省,因水淺阻滯,定造剝船三百艘,交德州、恩、武城、夏津、臨清五州縣分管。

清初沿明衛所之制,以屯田給軍分佃,罷其雜徭。尋改衛軍為屯丁,毋得竄入民籍,五年一編審,糧道掌之。康熙初,定各省衛所額設運丁十名。三十五年,定漕船出運,每船僉丁一名,餘九名以諳練駕馭之水手充之。凡僉選運丁,僉責在糧道,舉報責衛守備,用舍責運弁,保結責通幫各丁。尋僉本軍子弟一人為副軍。雍正初,免文學生員僉運。先是江蘇按察使胡文伯以江、安十衛去蘇、松水次遙遠,遇有應更換之丁,運官赴衛查僉,往返須時,請預僉備丁,造冊送糧道,轉送總漕備案。經戶部議準。漕督楊錫紱上疏爭之,略言:「預僉閒丁,其不必者有二,不便者有二。各省衛幫,貧富不等。殷富之幫,本無俟閒丁預備;貧乏之幫,遇有應換之丁,百計搜查,求一二殷丁且不可得,安有數十閒丁可以預備?其不必一也。又殷實軍丁,生計粗裕,猝遇收成歉薄,一二年或即轉為貧乏,今既僉選註冊矣,設需用之時,已經貧乏,是仍以疲丁應選,其不必二也。至送糧道點驗,僕僕道途,廢時失業,不便一也。衛所州縣書吏,喜於有事,富者賄脫,貧者受僉,不便二也。請停止預選閒丁註冊。」從之。

舊制漕船旗丁十名,丁地五頃。其後丁地半歸民戶,運丁生計貧乏,經戶部行文清查,不許民間侵占。乾隆初,巡漕御史王興吾奏:「屯田籍冊年久散失,無可稽考。亦有冊籍僅存而界址難於徵實,或軍丁典佃於民,而展轉相售、屢易其主者。清田歸運,徒滋擾累。蓋津貼之舉已成通例,民出費以贍丁,丁得項以承運,相沿既久,無礙於漕。況丁得田不能自耕,勢必召佃收租,是與未贖時之津貼同一得項承運,未見有益也。」二十五年,錫紱奏:「漕運之有疲幫,實緣運丁債負為累。浙江之金、衢、嚴、溫、處、紹、臺、嘉等幫,江南之江、淮、興、武、鳳陽、大河等幫,債欠尤多,幫疲益甚。欲除私負之累,莫若出借官帑。請於浙江江、安道庫各提銀六萬兩,專備疲幫領借。每歲督運道員,查按沿途及抵通需用銀數,提交押運,至期散給,於次年新運應領項下扣還,俟疲幫漸起,奏明停止。

各省州縣衛幫承僉運丁,均以奉文派僉日起,限兩月僉解,並查明田地房產,造冊送總漕存案。設有虧短掛欠,令其賠補。若僉派後實系賣富差貧,或棄船脫逃,或重僉已革之丁,以及徇情出結、將軍丁改入民籍者,承僉之員降二級調用,不準抵銷。其上司照失察例議處。從漕督毓奇請也。道光十三年,給事中金應麟奏:「江、浙內河一帶漕船,訛詐商民,有買渡、排幫等名目。州縣以兌米畏其挑剔,置若罔聞,滯運擾民,為害甚大。」詔林則徐、富呢揚阿嚴行查禁。

運軍往來淮、通,終歲勤苦,屯田所入有限,於是別給行月錢糧資用,其數各省不一。江南運軍每名支行糧二石四斗至二石八斗,月糧八石至十二石。浙江、江西、湖廣行糧三石,月糧九石六斗。山東行糧二石四斗,月糧九石六斗。其通、津等衛協運河南漕船運丁行月之數,與山東同。各省領運千總等官,於廩俸外多有兼支行糧者。行月二糧,舊時本少折多,且折價每石不過三四五錢,各處官丁常有偏枯之控。詔令漕督議定查照歲支行月舊額本折各半,折色照漕欠每石銀一兩四錢,永著為令。康熙二十九年,行月錢糧設立易知由單,列明應給各項錢糧,丁各一紙,照款支給。如官役剋扣婪索,許本丁將事由載單內,於過淮時陳控。

雍正元年,覆準運船到次,先將本色行月錢糧於三日內給發折色銀,由衛守備出具印領受,領運千總鈐章,解道驗明,以半給軍,半封固,糧道賚淮,由總漕監發,愆期遲延者罪之。乾隆五年,議定運丁於解淮驗給一半錢糧內,酌留回空費用,數多者扣留三之一,少者酌扣八兩,令糧道另行封兌,於過淮時交隨運官弁收領,俟抵通交糧後,給發各丁。緣各省漕船回空,每因資斧缺乏,不能及時抵次也。十年,漕督顧琮上言:「糧道所押幫船,多少不同,兌開復有遲早,必俟最後之幫開竣,方得赴幫督察,而首進之幫,又不免守候領銀之累。請仍令糧道兌準封給領運千總,解淮呈驗散給。」從之。

凡漕船停歇,月糧減半給發,民船停運,給月糧原額四之一。三十年,車駕南巡,截留江、浙二省冬兌漕糧各十萬石,減歇之船,於應給月糧外,加恩再給十之二,以示體恤。運軍月糧,遇閏按月本折均平支給,尋罷。嗣以閏月錢糧乃計日授食,各軍春出冬歸,停支一月,不免枵腹。山東、河南、浙江、江寧、鳳陽等衛閏月有糧,仍照原額支給。山東、浙江及蘇、太等衛,遇閏各有額編加徵銀,江、興等衛無之,遇閏於道庫減存銀內支用。江西、湖北、湖南系按出運船米之數支給。河南遇閏亦無加徵銀,向準山東等省一例支給,經部駁追,尋準其照支。

各省運軍名數參差不齊。江、浙每船十一二名不等。嗣議定每船概以十軍配運,按名支給行月。安慶衛舊系按漕用軍按名派行月二糧。自畫一裁減後,每船祗用十軍,而所載漕糧則倍於他船,應仍按糧支給行月。山東德州等衛有自雇民船裝運漕糧者,一體支給行月錢糧。江寧省衛無贍運屯田,遇有減存,同出運之船支給安家月糧。江淮、興武二衛,原

減駕軍二名,準其復設,派給行月二糧,例由布政司行文各府州縣支領,每船餽遺書吏六七金不等,否則派撥遠年難支錢糧及極遠州縣,而州縣糧書又有需索,每船約二三金不等。十金之糧,運丁所得實不及半也。

漕糧為天庾正供,司運官吏考成綦嚴。順治十二年,定漕、糧二道考成則例。經徵州縣衛所各官,漕糧逾期未完,分別罰俸、住俸、降級、革職,責令戴罪督催,完日開復。康熙二年,議定隨漕行月、輕齎各項錢糧,總作十分計算,原參各官限一年接徵,而接徵之員止限半年,殊未平允。嗣後接徵官限一年,糧道、知府、直隸州一年半,巡撫二年。如仍不完,照原參分數議處。其經徵督催白糧各官考成條例,悉與漕糧同。白糧項下減存經費銀不得擅用,違者題參,並勒令賠繳。糧道完儲錢糧,春秋造冊達部,候撥解京餉。年終及離任日,籓司盤查出如有侵虧,揭報巡撫題參。

凡漕欠,無論多寡,均發各糧道嚴追,承追官吏嚴查本弁本軍產業,估計變售償補。如運軍侵糧逃逸,報明戶部,行文總督提究。掛欠米石,追完補運,與本幫原欠米不符者,將過淮不駁換之總漕及督漕、承運各官並採買搭運之員,一並糾劾。其運到之米,按數收用,以免累及運軍。承平日久,法令日弛,糧道及監兌、押運官既不親臨水次,糧船抵淮,漕總復不嚴行稽查,於是弁軍任意折銀,沿途盜賣,抵關時遂多掛欠矣。

四十五年,令嗣後耗贈漕截等銀米,暫存糧道倉庫,俟回空時,倉場查明,按其掛欠數扣抵。不足,以行糧抵補。旋議定掛欠漕糧不及一分至六分之弁軍治罪,總漕、糧道按所欠分數議處,並將所欠漕糧,由總漕、糧道及監兌、押運、僉丁、衛所各官至運丁,分別擔任,均限定期內償還。不完,總漕、糧道交部議,運官、運軍分別治罪,仍責成總漕、糧道賠償。全完者,優敘。

糧船抵通起卸漕米,例買別幫餘米抵補。雍正三年,奏準嗣後漕米如有不足,即分別參處償還,不得以別幫餘米買補。其運軍日用餘米,許其售賣,餘並禁阻。

漕船經涉江湖,偶遇風濤漂沒,沿途催趲各官,及汛地文武官,親臨勘驗出結,總漕及巡撫覆勘奏免。若軍弁詐報漂沒,及漂沒而損失不多,乘機侵盜至六百石者,擬斬;不及六百石,充發極邊,漕米按數賠繳。文武官遇漕船沈溺,不將情由申報,押運官弁巡查不謹,致失火焚毀者,俱降一級調用。地方官不協救,延燒他船者,罰俸一年。雍正初,奏準漕船在內河失風漂沈者,不許豁免,押運官弁照失於防範例,罰俸一年。如有假捏,嚴加治罪,出結官弁,從重議處。凡海洋江河遭風漂沒,領運弁軍幸獲生全者,照軍功保守在事有功例,晉級賜金。其漂沒身故者,官弁照軍功陣亡例,分別準廕加贈,運軍給予祭葬銀。

乾隆七年,議定漕船失風火災,船未沉沒,無論已未過淮,即令修固復載抵通。如已被沈難戽者,雇民船載運,隨幫過淮盤驗抵通。如失事在過淮以後,黃河中流,民船難募,令先分通幫帶運開行,沿途仍雇覓民船裝載。通幫各丁,出具互結,稍有虧欠,責令償補。江、廣漕船失風沈溺,如果不堪戽修,無論已未滿號,地方官驗明,申報總漕,就近變價,令運弁賚交糧道發給。回空漕船失事亦如之。嗣議準江蘇、浙江、山東、河南等省買補船艘,如已滿號,遇失風事故,就近折變,價銀封交員弁攜回,由糧道驗給各軍,以補新漕。漕船遇冰凌迅下,致被損壞,及雷火焚毀,沈失米糧,免其償補。

各省漕糧,歲有定額,凡荒地無徵者,督撫勘實報免,隨漕銀米,一例蠲免。災傷之區,應徵漕糧,及折改漕價,酌量各被災輕重,分別緩徵、帶徵。遇帶徵之年,復又被災傷,分年壓徵帶補。沿江沿海田地坍沒水中者,保題豁免。水旱偏災民地,例得蠲免,惟應船役,即被災甚重,仍須供修船雇募等事,不得同邀寬典。康熙三十七年,議定京畿通州、武清、寶坻、香河、東安、永清六州縣紅剝船戶所領地,水旱一體蠲免。水淹田畝,例於歲終確勘,涸前起徵,淹則停免。雍正十年,定淹田漕米照壓徵例,俟冬勘後,涸則帶徵,淹則豁免。

蘇、松、太三屬為東南財賦之區,賦額最重。世宗以來,屢議蠲緩,然較之同省諸府縣,尚多四五倍或十數倍。道光時,兩遭大水,各州縣每歲歉蠲減,遂成年例。嗣是徵收之數,除官墊民欠,每年僅得正額之七八或五六而已。軍興以後,兩府一州,受害尤酷。同治二年,諭江督、蘇撫查明,折衷議減,期與舊額本輕之常、鎮二府,通融覈計,著為定額。其紳戶把持、州縣浮收諸弊,永遠禁革。四年,戶部遵議:「江蘇常、鎮、太五屬編徵米,系會同漕贈行月南恤局糧等款徵收。應如李鴻章等所奏,無分起運留支,一體並減,酌科則之重輕,視減成之多寡,計原額編徵米豆二百二萬餘石,減五十四萬餘石。」民困稍舒。曾國籓又請將蘇、松地漕錢糧一體酌減。部覆漕項為辦運要需,若議覈減,費必不敷,勢須另加津貼,於民生仍無裨益。詔令國籓、鴻章仿浙省成例,覈實刪減浮收,並嚴禁大戶包攬短交等弊。是年減浙江杭、嘉、湖三屬米二十六萬餘石。

海運始於元代,至明永樂間,會通河成,乃罷之。清沿明代長運之制。嘉慶中,洪澤湖洩水過多,運河淺涸,令江、浙大吏兼籌海運。兩江總督勒保等會奏不可行者十二事,略謂,「海運既興,河運仍不能廢,徒增海運之費。且大洋中沙礁叢雜,險阻難行,天庾正供,非可嘗試於不測之地。旗丁不諳海道,船戶又皆散漫無稽,設有延誤,關系匪細」。上謂「海運既多窒礙,惟有謹守前人成法,將河道盡心修治,萬一嬴絀不齊,惟有起剝盤壩,或酌量截留,為暫時權宜之計,斷不可輕議更張,所謂利不百不變法也」。自是終仁宗之世,無敢言海運者。

道光四年,南河黃水驟漲,高堰漫口,自高郵、寶應至清江浦,河道淺阻,輸輓維艱。吏部尚書文孚等請引黃河入運,添築閘壩,鉗束盛漲,可無泛溢。然黃水挾沙,日久淤墊,為患滋深。上亦知借黃濟運非計,於是海運之議復興。詔魏元煜、顏檢、張師誠、黃鴻傑各就轄境情形籌議。諸臣憚於更張,以窒礙難行入奏。會孫玉庭因渡黃艱滯,軍船四十幫,須盤壩接運,請帑至百二十萬金。未幾,因水勢短絀,難於挽運,復請截留米一百萬石。上令琦善往查,覆稱玉庭所奏渡黃之船,有一月後尚未開行者,有淤阻禦黃各壩之間者,其應行剝運軍船,皆膠柱不能移動。上震怒,元煜、玉庭、檢均得罪。

協辦大學士、戶部尚書英和建言:「治道久則窮,窮則必變。河道既阻,重運中停,河漕不能兼顧,惟有暫停河運以治河,雇募海船以利運,雖一時之權宜,實目前之急務。蓋滯漕全行盤壩剝運,則民力勞而帑費不省,暫雇海船分運,則民力逸而生氣益舒。國家承平日久,航東吳至遼海者,往來無異內地。今以商運決海運,則風颶不足疑,盜賊不足慮,霉濕侵耗不足患。以商運代官運,則舟不待造,丁不待募,價不待籌。至於屯軍之安置,倉胥之稽察,河務之張弛,胥存乎人。矧借黃既病,盤壩亦病,不變通將何策之從?臣以為無如海運便。」詔仍下有漕各省大吏議。時琦善督兩江,陶澍撫安徽,咸請以蘇、松、常、鎮、太倉四府一州之粟全由海運。乃使布政使賀長齡親赴海口,督同地方官吏,招徠商船,並籌議剝運兌裝等事。嗣澍言:「現雇沙船千艘,三不像船數十,分兩次裝載,計可運米百五六十萬石。其安徽、江西、湖廣離海口較遠,浙江乍浦、寧波海口或不能停泊,或盤剝費鉅,仍由河運。」上乃命設海運總局於上海,並設局天津。復命理籓院尚書穆彰阿,會同倉場侍郎,駐津驗收監兌,以杜經紀人需索留難諸弊。

六年正月,各州縣剝運之米,以次抵上海受兌,分批開行。計海運水程四千餘里,逾旬而至。米石抵通後,轉運京倉,派步軍統領衙門文武員弁沿途稽查。沙船耗米,於例給旗丁十八萬餘石內動放,所節省耗米六萬石,仍隨同起運。承運漕糧每石給耗米八升,白糧耗米一斗,以補正米之不足。仍將漕運商耗覈出二成,白糧覈出三成,由津局給價收買,隨正交運。漕糧無故短少霉變,於備帶耗米內補足;不敷,勒令買補。如有斫桅松艙傷人等事則免之。船戶腳價飯米折色並津貼等銀,先於受兌後發七成,餘三成交押運員弁,到壩後查無弊端,始行全發。沙船餘米不下十萬石,初照南糧例,聽天津人照市價收買。嗣以商人希圖賤價售買,改由官為收買,其價銀由江南委員轉發船戶,後仍令商船自行售賣。

每屆海運期,沿海水師提鎮,各按汛地,派撥哨船兵丁,巡防護送,並派武職大員二,隨船赴津。上海交兌時,先期咨照浙江提鎮水師營出哨招寶、陳錢一帶地方,江南提鎮水師營出哨大小洋山,會於馬跡山,山東總鎮出哨成山、石島,會於鷹游門,以資彈壓。山東洋面,責成游擊、守備,搜查島嶼,防護迎送。後以邵燦言,停派護送武職大員,責成沿海水師逐程遞護。嗣寧、滬商人各置火輪船一,遇新漕兌開行時,分別扼要巡防。

剝船,直隸舊設二千五百艘,二百艘分撥故城等處,八百艘留楊村,餘千五百艘集天津備用。後雇賝堪裝漕糧二百五十石民船五百艘,以備裝載。商船首次抵津,先僅府縣倉腳廟宇撥卸三十萬石,餘令剝船徑運通倉。隨將天津倉腳廟宇所儲漕米運通,無庸轉卸北倉,致多周折。至商船二次抵津,如剝船不敷裝載,即將米先儲府縣倉腳廟宇;不敷,再剝儲北倉。隨令原剝將所儲米石侭數運通。剝船足敷裝載,即按首次商船辦法,不必分儲北倉,以歸簡便。剝船百六十只為一起,由經紀自派人分起押運交倉,押運員役稟報倉場,復馳回續押後起米船。經紀等止須帶領斛手到船起卸,如有藉端刁難需索,交地方官從嚴治罪。

各州縣經管剝船,每年例給修艌銀五兩,三年小修一次,給費二十兩,歲終漕竣,逐一挑驗,船身堅固者,酌量修艌,如損壞較甚,即核賞估價,所需經費,於道庫油艌銀項下動撥。封河守凍期內,每船工食銀十五兩,運米百石,給腳價八兩四錢,食米一石一斗五升。嗣每百石加腳費五兩。李鴻章因官剝船戶貧困滋弊,例定工食銀十五兩,僅領一半,不敷贍家,請每船由蘇、浙漕項內酌貼五兩,部格不行。鴻章上疏爭之,詔從其議。商船領運漕糧,迅速無誤,萬石以下給匾額,五萬石獎職銜,每次奏保以百二三十人為限。

七年,蔣攸銛請新漕仍行海運。上以近年河湖漸臻順軌,軍船可以暢行,不許。其後各省歲運額漕,逐漸短少,太倉積粟,動放無存。二十六年,詔復行海運。二十七年,議準蘇、松、太二府一州漕白糧米,自明歲始,改由海運。三十年,復令蘇、松、太二府一州白糧正耗米,援照成案,由海運津。咸豐元年,戶部尚書孫瑞珍請河海並運。御史張祥晉請將江蘇新漕,援案推廣常、鎮各屬及浙江,一體海運。下江督陸建瀛、蘇撫楊文定、浙撫常大淳妥議。覆稱明年蘇、松、常、鎮、太四府一州漕白糧米,請一律改由海運。浙漕礙難海運,請仍循舊章,從之。二年,建瀛上籌辦海運十事,下部議行。是年以浙江漕船開兌過遲,回空不能依期歸次,詔來歲新漕改為海運,從巡撫黃宗漢請也。五年,河決銅瓦廂,由張秋入大清河,挾汶東趨,運道益梗。六年,截留江蘇應運漕糧二十萬石供支兵餉,實運漕白正耗及支賸給丁餘耗米七十五萬五千餘石,其歉緩南漕,令各州縣依限催徵運通。

同治七年,議試用夾板船裝運採買米石,水腳銀數悉仍沙船例,給銀五錢五分,挽至天津紫竹林,由商董就近寄棧,聽驗米大臣會同通商大臣驗收過剝,所需小船剝價、棧租、挑力,每石給銀七分,由商董承領經理。又每石給保險銀三分,設有遭風拋失,責令貼補。至每米千石,隨耗八十石,備帶餘米二十石,剝船食耗米十一石五斗。又每百石給津、通剝價銀八兩一錢四釐,通倉個兒錢折銀二兩,均照海運正漕採買各案辦理。是年以津沽河面狹隘,常有沈船失米之虞,於大沽增設海運外局。

九年,浙江巡撫楊昌濬奏:「浙省來歲新漕,酌擬海運章程十四條:一,委員分辦,以專責成;一,新漕仍由上海受兌放洋,白糧仍循案裝盛麻袋,首先運滬;一,寬備海運商船,並由蘇省多撥沙船,移浙濟用;一,經耗等米,仍照數支給,商耗飭帶本色並餘耗申糙等米搭交倉;一,增給天津剝船耗米,以彌虧欠;一,津、通經費,照案備帶,羨等款,仍按數抵解;一,商船準帶砲械,並由商捐輪船護送,仍責成沿海水師實力巡防;一,天津交米後,循舊責成經紀,續到之船,仍由天津道驗收;一,循案加增海運經費;一,米船到津,應多添排數,寬備剝船;一,商船水腳等項,照案核給,並二成免稅,酌定賞罰;一,商船二成免私之貨,仍以米石計斤,所帶竹木,照案免稅;一,商船回空載貨,照向章免稅;一,米船抵津交卸,嚴禁經紀斗斛剝船需索浮費。」下部議行。十年,鴻章言:「剝船守候苦累,每載米百石,請加給腳價銀五兩,並另籌運白糧民船守候口糧銀萬二千兩,由蘇、浙糧道庫漕項內撥解;不敷,則由司庫通融借撥。」

十一年,昌濬請以輪船運漕,從之。輪船招商,由商人借領二十萬串為設局資本,盈虧悉由商任之。購堅捷輪船三艘,每年撥海運漕米二十萬石,由招商輪船運津,其水腳耗米等項,仍照向章辦理。輪船到津,命直督籌備剝船轉運,並會同倉場侍郎臨棧查驗,仍仿照白糧例,由江、浙撫道運通交納,以杜折耗偷漏。輪船協運江、浙漕糧,簽明某省漕白糧米字樣於米袋之上。糧米上棧時,由滬局派員監兌;兌竣,即由輪船商局給收米回文,以後裝船起運,俱由商局覈辦,滬局不再與聞。其棧費夫力,亦由商局任之。凡漕糧派裝輪船,輪船商局酌委執事,會同滬局詳驗,米色乾潔,方行收兌,交輪局押赴浦江東棧斛收。抵津,飭津局各員董提前驗收,以免壅滯。輪船每艘載米三千石,填發連單,由津局稽核,一切領銀領米等結悉罷之。輪船運米,由上海道填給免稅執照,並援例得酌帶二成貨物。其洋藥及二成之外另帶貨物,仍須納稅。

喬松年奏山東境內黃水日益汎濫,運河淤塞,擬因勢利導,俾黃水先驅張秋。其張秋南北,普行挑濬,修建徬壩以利漕。丁寶楨、文彬奏請挽復淮、徐故道。事下廷臣會議。復稱銅瓦廂決後,舊河身淤墊過高,勢不能挽復淮、徐故道。至借黃濟運,築堤束水,與導衛濟運之法同一難行。鴻章奏請仍由海道轉運,令各省酌提本色若干運滬,由海船解津,餘照章折解,以節運費。並隨時指撥漕折銀兩採買接濟,並請停止河運採買糧石,推廣海運。仍下部議。先是江北漕糧,由河運通,至是亦試辦海運。十三年,奏準江西在滬採買漕糧八萬石,交招商局由海運津,每石腳價銀二兩七錢。光緒元年,湖南漕糧採辦正耗米二萬三百四十五石,湖北採辦三萬石,均交招商局由海運津。江西、湖南尋停。

寶楨奏運河廢壞,莫非黃水之害,治運必先治黃。應先將微山湖之湖口雙閘及各減閘,迅速修砌,及時收蓄,以保湖瀦;運河正身亦須量為疏濬。嗣桂清、畢道遠、廣壽、賀壽慈等亦以籌款修復運河為請。黃元善復稱:「自黃河北徙,運河阻滯,改由海運,原屬權宜之計。當時奏定江蘇漕額,以河運經費作為海運支銷,每石不得過七錢。嗣以經費不敷,迭次請增。江蘇所加,距一兩不遠,浙江已加至一兩,較道光二十八年、咸豐二年海運經費尚有節省歸公者,大相逕庭。且海運歷涉重洋,風波靡定,萬有不測,所關匪細。河運雖迂滯,而沿途安定,經費維均。自各省以達京倉,民之食其力者,不可數計。裕國利民,計無善於此者。現停運未久,及時修復,尚屬未晚。再遲數年,河道日淤,需費更鉅。臣以為河運迂而安,海運便而險,計出萬全,非復河運不可。」上命河督、漕督及沿河各督撫籌畫具奏。沈葆楨疏駁桂清、畢道遠等請將有漕省分酌提漕項及將海運糧石分出十數萬石改辦河運之議,並力言「河運決不能復。運河旋濬旋淤,運方定章,河忽改道,河流不時遷徙,漕路亦隨為轉移。而借黃濟運,為害尤烈。前淤未盡,下屆之運已連檣接尾而至,高下懸殊,勢難飛渡。於是百計逆水之性,強令就我範圍,致前修之款皆空,本屆之淤復積。設令因濟運而奪溜,北趨則畿輔受其害,南趨則淮、徐受其害,億萬生靈,將有其魚之嘆,又不僅徒糜巨帑無裨漕運已也」。七年,令直督飭招商局有協運漕糧時,酌分道員駐津驗兌,並責成糧道嚴督治漕事人員,兌米時加意查察。因招商局協運江、浙漕糧,有攙雜破碎諸弊故也。

十年,法人構釁,海運梗阻。太常卿徐樹銘言:「漕糧宜全歸河運,請於運道經行處疏濬河流,修治閘壩,並選雇民船以濟運。」明年,曾國荃言:「來年河運酌添江蘇漕糧五萬石,並將邳、宿河道淤淺處,酌估挑濬。」從之。盧士傑言:「鄭州黃河漫口奪溜,山東運河十里堡門外積淤日寬,回空漕船,不能挽抵口門。現寧、蘇新漕待船裝載,邳、宿挑淤築壩,必待空船過竣,方可興工。」上命迅飭疏濬積淤,俾漕船早日南下。十五年,從山東巡撫張曜請,改撥海運漕米二十萬石仍歸河運。曾國荃、黃彭年奏:「江、安河運米石,業經截留充賑。蘇屬河運漕米十萬,前已改歸海運,各州縣起運,均已抵滬,驟改河運,窒礙難行。且雇船將近千艘,亦非旦夕可致。請俟本年冬漕,再行遵旨提前河運,以期規復舊章。」制可。

十九年,北運河上游潮、白等河狂漲,水勢高於堤顛數尺,原築上堰,俱沒水中,運河水旱大小決口七十餘處,由津運京米麥雜糧千數百艘,在楊村阻淺,命鴻章將各口門堵合,並疏濬河身,停蓄水勢,以利舟行。二十二年,王文韶奏:「南漕改行海運,惟江北漕糧仍由河運,復於蘇、松項下提撥米十萬石並入河運。船多道遠,自黃入運,自運入衛,節節阻滯,船戶窮無復之,竊米攙水,諸弊叢生。本年漕船到津,較昔已遲二三月,誠恐有誤回空。已飭並程催趲,剋日兌收。但此次截留江北漕米五萬石,米色尚佳。江蘇五萬石,米色參差,甚或蒸變,剔除晾曬,幾費周章,蓋運受黃病,已非人力所能挽救。擬請自本年始,改撥蘇漕之十萬石統歸海運。其江蘇冬漕仍辦河運,以保運道。」下部議行。御史秦夔揚以江北河運勞費太甚,疏請停辦,改折解部。部議漕糧關系京倉儲積,未便遽更舊制。

二十六年,以戰端既開,從陳璧請,於清江浦設漕運總局。車駕西幸,轉運局移漢口,清江改設分局。是年南漕改用火車由津運京。二十七年,以財用匱乏,諭:「自本年始,直省河運海運,一律改徵折色,責成各省大吏清釐整頓,節省局費運費,並查明各州縣徵收浮費,勒令繳出歸公,以期匯成巨款。」奕劻請於應辦白糧外,每年採辦漕糧百萬石,純用粳米,並不得率請截留,從之。二十八年,部議本年江、浙漕糧,純歸招商局輪船承運,費應力從減省。盛宣懷奏:「近年滬局輪船,因事起運太遲,棧耗既鉅,及運至塘沽,又值聯軍未退,費用倍於常時。二十六、二十七兩年,招商局所領水腳,實不敷所出。本年太古洋行原減價攬載,英、日議定商約,均欲漕運列入約章,臣等力拒之。蓋招商局為中國公司,前李鴻章奏準漕米、軍米悉歸招商局承運,實寓有深意也。此次詳察中外情形,擬請自二十八年冬漕始,於向章每石輪船水腳保險等項漕米銀三錢八分八釐一毫內減去五分,永為定制。」從之。

江、浙漕糧由海運津,向用剝船運至通倉,每石支耗米一升一合五勺,名曰「津剝食耗」。自南漕改用火車運京,此項耗米,改令隨正交倉。嗣因運米事竣,每有虧耗,許仍舊支給,以抵車運虧耗云。


\end{pinyinscope}