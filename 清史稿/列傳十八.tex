\article{列傳十八}

\begin{pinyinscope}
佟養性孫國瑤李永芳石廷柱馬光遠弟光輝李思忠子廕祖廕祖子鈵金玉和子維城王一屏一屏子國光國光子永譽孫得功張士彥士彥子朝璘金礪

佟養性,遼東人。先世本滿洲,居佟佳,以地為氏。有達爾哈齊者,入明邊為商,自開原徙撫順,遂家焉。天命建元,太祖日益盛強。養性潛輸款,為明邊吏所察,置之獄,脫出,歸太祖。太祖妻以宗女,號「施吾理額駙」,授三等副將。從克遼東,進二等總兵官。

太祖用兵於明,明邊吏民歸者,籍丁壯為兵。至太宗天聰間,始別置一軍,國語號「烏真超哈」。五年正月,命養性為昂邦章京,諭曰:「漢人軍民諸政,付爾總理,各官受節制。爾其殫厥忠,簡善黜惡,恤兵撫民,毋徇親故,毋蔑疏遠。昔廉頗、藺相如共為將相,以爭班秩,幾至嫌釁。賴相如舍私奉國,能使令名焜耀於今日。爾尚克效之!」又諭諸漢官曰:「漢人軍民諸政,命額駙佟養性總理,各官受節制。其有勢豪嫉妒不從命者,非特藐養性,是輕國體、褻法令也,必譴毋赦!如能謹守約束,先公後私,壹意為國,則爾曹令名亦永垂後世矣。」

是歲,初鑄砲,使養性為監。砲成,銘其上曰「天祐助威大將軍」,凡四十具。師行則車載以從,養性掌焉。八月,上伐明,圍大凌河城。養性率所部載砲越走錦州道為營,擊城西臺,臺兵降;又擊城南,壞睥睨;翌日,擊城東臺,臺圮,臺兵夜遁,盡殲之。九月,明兵出關援錦州,上遣親軍迎擊,養性以所部兵五百從,敵潰遁。明監軍道張春合諸路兵援大凌河,夜戰,上督騎兵擊破之。方追奔,明潰兵復陣,上命養性屯敵壘東,發砲毀敵壘。十月,攻於子章臺,發砲擊臺上堞,臺兵多死者。十一月,祖大壽以大凌河降,上命盡籍城中所儲槍砲彈藥付養性。尋率兵隳明所置臺壕,自大凌河至於廣寧。

六年春正月,上幸城北演武場閱兵,養性率所部烏真超哈試砲,擐甲列陣,軍容甚肅。上嘉養性能治軍,因追獎大凌河戰功,賜雕鞍良馬一、白金百,遂遍及諸將,自石廷柱以下皆有賜,設宴以勞之。養性疏言:「新編漢兵,馬步僅三千有奇,宜盡籍漢民為兵,有事持火器而戰,無事則為農。火器攻城,非砲不克,三眼槍、佛朗機鳥槍特城守器耳,宜增鑄大砲。兵食未足,宜令民廣開墾,無力者官畀牛若種,穫則以十一償。」四月,上自將伐察哈爾,命與貝勒阿巴泰、杜度,大臣揚古利、伊爾登留守。七月,卒。順治間,追謚勤惠。

子普漢,改襲二等昂邦章京。卒,以弟六十襲。崇德四年,漢軍旗制定,隸漢軍正藍旗。順治四年,改二等精奇尼哈番。遇恩詔,累進三等伯。

國瑤,六十子也,襲爵。康熙九年,授本旗副都統。十二年,吳三桂反,特命國瑤為鄖陽提督,帥師鎮撫。十三年,襄陽總兵楊來嘉以穀城叛附三桂,鄖陽副將洪福應來嘉,劫部兵千餘攻國瑤。國瑤率游擊杜英、佟大年以健丁三百拒戰,福退,復至,苦戰數日,斬二百餘級,福敗遁。事聞,加左都督。十四年,福挾諸叛將分五道來犯,復擊敗之,逐戰泥河口、板橋河,斬其將林躍等七輩、兵數百人。

十五年,四川叛將譚弘與福等復分道來犯,弘屯鄖江北,福掠鄖江南,相聲援;國瑤分兵御之,戰坪溝、戰黃畈、戰九里岡,又渡鄖江戰江南岸,皆勝,焚其舟及械,斬獲無算。福復遣眾伏鄖江兩岸,以三十餘舟順江下,泊琵琶灘,偪鄖陽運道為寨。國瑤及將軍噶爾漢、撫治楊茂勛等率水陸兵大破之,戰陡嶺,福敗走,運道復通。敘功,加一等。

十六年,以捐俸賚軍賑難民,加太子少保。十七年,進討來嘉、福,戰於房縣,獲其將五十二輩、印十二、劄牌二十四,遂克其城,進復保康。十八年,與噶爾漢等攻興安,久而不下,命削戰陡嶺所敘加一等。六月,授福建將軍。二十八年,卒,謚忠愨。乾隆初,定封二等子。達爾哈齊子養真,自有傳。

李永芳,遼東鐵嶺人。在明官撫順所游擊。太祖克烏喇,烏喇貝勒布占泰走葉赫。太祖伐葉赫,葉赫愬於明。明使告太祖,誡毋侵葉赫。太祖以書與明,言葉赫渝盟悔婚,復匿布占泰,不得已而用兵,躬詣撫順所,永芳迎三里外,導入教場,太祖出書畀永芳,乃引師還。

後三歲為天命元年,又三歲,始用兵於明。四月甲辰昧爽,師至撫順所,遂合圍,執明兵一,使持書諭永芳曰:「明發兵疆外衛葉赫,我乃以師至。汝一游擊耳,戰亦豈能勝?今諭汝降者:汝降,則我即日深入;汝不降,是誤我深入期也。汝多才智,識時務,我國方求才,稍足備任使,猶將舉而用之,與為婚媾;況如汝者有不加以寵榮與我一等大臣同列者乎?汝若欲戰,我矢豈能識汝?既不能勝,死復何益?且汝出城降,我兵不復入,汝士卒皆安堵。若我師入城,男婦老弱必且驚潰,亦大不利於汝民矣。勿謂我恫喝,不可信也。汝思區區一城且不能下,安用興師?失此弗圖,悔無及已。降不降,汝熟計之。毋不忍一時之忿,違我言而僨事也!」永芳得書,立城南門上請降,而仍令軍士備守具。上命樹雲梯以攻,不移時,師登陴,斬守備王命印等。永芳冠帶乘馬出降,固山額真阿敦引永芳下馬,匍匐謁上,上於馬上以禮答之,傳諭勿殺城中人。東州、馬根單二城及沿邊諸臺堡五百餘,悉下。是日,上駐撫順。明日,命隳其城,乃還。編降民千戶,遷之赫圖阿喇。命依明制設大小官屬,授永芳三等副將,轄其眾,以上第七子貝勒阿巴泰女妻焉。太祖伐明取邊城,自撫順始;明邊將降太祖,亦自永芳始。

是年七月,上復伐明,拔清河。四年,克鐵嶺。六年,下遼、沈。永芳皆從,以功授三等總兵官。明巡撫王化貞及諸邊將屢遣諜招永芳,永芳輒執奏,上嘉獎,敕免死三次。

太宗即位,以朝鮮與明將毛文龍相應援,納逋逃,命貝勒阿敏等帥師討之,永芳從。上諭阿敏等曰:「朝鮮理當討,然非必欲取之。凡事相機度義而行。」克義州,分兵攻鐵山,擊走文龍;進下定州、安州,次平壤,其官民皆遁,遂渡大同江。朝鮮王李倧使齎書迎師,諸貝勒答書歷數其罪,許以遣大臣蒞盟,當班師。使既行,師復進,次黃州,倧使馳告已遣大臣蒞盟。阿敏欲遂攻其都城,諸貝勒謂宜待所遣大臣至,永芳進曰:「我等奉上命,仗義而行。前與朝鮮書,許以遣大臣蒞盟當班師,今食言不義。盍暫駐待之?」諸貝勒皆是其言,阿敏怒,叱永芳曰:「爾蠻奴,何多言!我豈不能殺爾耶?」師再進,次平山,倧所遣大臣至師,卒如永芳議,遣劉興祚、庫爾纏如倧所,蒞盟而還。

八年,永芳卒,有子九人。漢軍旗制定,隸正藍旗漢軍。次子李率泰,自有傳。

第三子剛阿泰,順治初,官宣府總兵。時姜瓖為亂,山西北境諸州縣土寇蜂起,瓖既平,所部竄匿代州、定襄、繁峙五臺山中。剛阿泰先後逐捕,諸山砦悉平。旋以屬吏侵餉劾罷。

第五子巴顏,天聰八年襲父爵,例改三等昂邦章京。崇德間,以參領從太宗徵科爾沁;圍錦州,與洪承疇戰松山城下:皆有功。七年,定漢軍八旗,以巴顏為正藍旗固山額真。八年九月,從鄭親王濟爾哈朗等徵寧遠,拔中後所、前屯衛。順治元年,進二等昂邦章京。旋與固山額真石廷柱剿寇昌平,與固山額真葉臣徇直隸饒陽,河南懷慶,山西澤州、潞安諸府縣,師還,賜白金五百。二年,從定西大將軍和洛輝自陜西徇四川,流寇孫守法、賀珍犯西安,再戰大破之,逐至黑水峪,斬守法;又破流寇一隻虎於商州,克延安諸路山寨。四年,例改二等精奇尼哈番。五年,進一等精奇尼哈番。討叛將姜瓖,從睿親王多爾袞復渾源州;從英親王阿濟格復左衛;從巽親王滿達海復朔州、汾州及太谷諸縣。巴顏在軍將左翼,挾火器以攻,所向皆克。八年,敘平姜瓖功,復遇恩詔,進一等伯。九年,卒。乾隆間,定封號曰昭信。四十年正月,命以其族改隸鑲黃旗。

石廷柱,遼東人。先世居蘇完,姓瓜爾佳氏。明成化間,有布哈者,為建州左衛指揮。布哈生阿爾松阿,嘉靖中襲職。阿爾松阿生石翰,移家遼東,遂以「石」為氏。石翰子三:國柱、天柱、廷柱。萬歷之季,廷柱為廣寧守備,天柱為千總。太祖師至,巡撫王化貞走入關,天柱先與諸生郭肇基出謁,且曰:「吾曹已守城門矣。」翌日入城,廷柱從眾降,授世職游擊,俾轄降眾。

蒙古巴林部貝勒囊努克背盟劫掠,廷柱從上討之,取其寨,收牲畜以還,進三等副將。天聰三年,太宗命率兵搜剿明故毛文龍所轄諸島,敵自石城島來犯,擊斬二百人,俘十九人。尋從上伐明,薄明都。四年,師還,至沙河驛,廷柱與達海諭城中軍民出降;又與達海以千人詗漢兒莊,漢兒莊與三屯營、喜峰口諸堡先已降而復叛,至是復降。

五年,明總兵祖大壽築城大凌河,上自將圍之。大壽窮蹙,使從子澤潤射書請降,並乞上令廷柱往議。廷柱與達海至城南,先使姜桂詗大壽,桂故明千總,為我軍所俘。大壽使游擊韓棟從桂出迎廷柱,並以其義子可法為質。廷柱乃逾壕與語,大壽言決降,惟乞速取錦州,俾妻子得相見。廷柱以告,上復遣廷柱諭指,大壽乃降。是時佟養性為烏真超哈昂邦章京,廷柱為副。六年,養性卒,廷柱代為昂邦章京。從伐察哈爾,多斬獲。七年,從貝勒岳託伐明,攻旅順,師還,上酌金卮以勞,進三等總兵官。八年,從伐明,攻應州,克石家村堡。九年,復從伐明,與明兵戰大凌河西,斬明副將劉應選,獲游擊曹得功等。

崇德元年,上自將伐朝鮮,命廷柱帥所部整兵械,儲糗崿,挾火器以從。二年,既定朝鮮,還攻皮島,廷柱與戶部承政馬福塔攻其北隅。尋追論朝鮮王李倧謁上時廷柱亂班釋甲,及縱士卒違法妄行罪,解任,罰鍰,奪賞賚。是年分烏真超哈為左、右翼,以廷柱為左翼固山額真。三年,上與諸臣論兵事,舉呂尚相勖,廷柱言:「呂尚制閫外專生殺,故所向有功。今臣等若有過下部逮訊,雖牛錄以下亦當比肩對簿,其何以堪?」諸臣以其言戇,請下刑部議罪論死,上命宥之。是年十月,從伐明,攻錦州,克城外諸屯堡,進破城旁臺。臺上餘敵兵潛自間道走,廷柱弗追擊,部議降爵罰鍰,上復命宥之。

四年二月,上自將伐明,烏真超哈諸將孔有德、耿仲明、尚可喜、馬光遠及廷柱皆率所部從。上駐軍松山,命廷柱攻城南臺,毀其堞,臺兵不能御,守將王昌功等四十餘人出降。上登松山南岡度地形,命廷柱從。可喜以砲攻城南門之左,廷柱與光遠先取城西南隅臺,諸將繼進,攻城,城堞皆盡,會以日暮罷。明日攻益急,城兵守禦甚固,我兵緣雲梯上不能入,死者二十餘人,廷柱兄子達爾漢亦被創。上召詢諸將,皆謂攻必克。翌日復集議,有德、仲明、可喜、光遠欲鑿地道以攻,廷柱持不可。上責廷柱曰:「爾為主將,恇怯無鬥志,與諸將異議。爾豈因兄子被創,故驚怖不欲戰耶?」廷柱惶恐,對曰:「臣昔嘗巡邏至此,知地中有水石不可穿,且亦不能越壕而過,故不敢不言。今眾皆謂可攻,臣焉敢獨異?」乃與有德等鳩役於城南鑿地道。初,祖大壽既降,請得入錦州,乃復叛,為明守;至是,聞松山急,遣蒙古兵三百乘夜入城,詗得我軍謀,多為備,地道不能達,乃罷攻。師還,部議廷柱攻城不盡力,當罷任罰鍰,上仍命宥之。時烏真超哈復析為八旗,合二旗為一固山,於是漢軍旗制始定。廷柱隸正白旗。

六年七月,廷柱上言:「錦州為遼左首鎮,我師築壘浚壕,誓必翦滅,以策進取,誠至計也。第明恃大壽為保障,我師圍之急,彼必益發援兵,並力一戰。宜及此時簡精銳,分布各旗屯田所,遇警即並進。如敵已立營,以砲環擊,伺其稍動,我師即突起乘之,轉戰過錦州,至松山、杏山間,敵必敗走,則錦州破矣。錦州既破,關外八城聞而震動。昔年克沈陽,遼陽從之下;克沙嶺,廣寧亦從之下。此其明徵也。近聞喀爾喀扎薩克圖揚言取歸化,恐陰欲取鄂爾多斯。臣擬令鄂爾多斯移牧黃河南,使與歸化相接,彼此策應。仍選才勇將士挾火器戍焉,而令王貝勒帥師道宣、大,略應州、雁門。歸化有警,輕騎倍道赴援。明所恃為遼東援者,不過宣大、陜西榆林、甘肅寧夏諸路。我師西入,諸路自顧不遑,豈能復出援遼哉?此一舉而兩得也。明援兵自寧遠至松山,所齎行糧不過六七日,其鋒少挫,勢必速退;即宿留數日,終且託糧盡而返。宜設伏於高橋險★C7處,鑿壕截擊,仍發勁兵綴其後,使進退無路,則彼援兵皆折而降我矣。我師遇敵步兵,每奮勇陷陣,彼軍多火器,恐致傷夷。宜詗敵遠離城郭,或憑據高阜,水竭糧匱,乃環而攻之。夜則鑿壕以守,晝則發砲以擊。不一二日,勢且生變,其斃可坐俟也。洪承疇書生耳,所統援遼諸鎮,皆烏合亡命,外張聲勢,內實恇懾。如大壽為我師所破,承疇與諸將縱得脫去,亦東市就僇而已。彼聞上恩豢降將,或慕義納款,亦未可料。今明災異迭見,流寇方熾,乘時應運,定鼎中原,機不可失。」疏入,上深嘉之。九月,師圍松山,敵夜犯廷柱營,廷柱力禦,斬十餘級,獲刀甲、槍砲無算,進二等昂邦章京。七年,定漢軍八旗,置八固山,以廷柱為鑲紅旗固山額真。

順治元年四月,從師入關,破李自成。五月,與固山額真巴顏等平昌平土寇。六月,與固山額真巴哈納帥師撫定山東諸郡縣。七月,移師會固山額真葉臣共克太原。山西、河南悉平。師還,賜白金五百兩,進一等昂邦章京。四年,改一等精奇尼哈番。六年,從討叛將姜瓖,復渾源、太谷、朔州、汾州。十二年五月,授鎮海將軍,駐防京口。十四年二月,以老乞休,加少保兼太子太保,致仕,進三等伯,世襲。十八年二月,卒,贈少傅兼太子太傅,謚忠勇,立碑紀績。

廷柱兄國柱,亦自廣寧降,與天柱先後授三等副將。廷柱六子,三子華善,四子石琳,自有傳。

馬光遠,順天大興人。明建昌參將。天聰四年,我師克永平,光遠以所部降,命隸正藍旗,授梅勒額真,賜冠服、鞍馬。五年,上復伐明,圍大凌河,光遠從,招城南臺降,得百總一、男婦五十餘,即畀光遠育焉。

六年十一月,光遠疏言:「六部既設,當建內閣,選清正練達二三臣為總裁,日黎明入閣。八家固山、六部承政,有事詣閣集議,請上指揮。」並議置六科,立八道言官。翌日再疏,申言六科職掌。七年正月,烏真超哈昂邦章京佟養性及光遠合疏言:「上及諸貝勒豢漢官恩厚,臣等叨冒首領。上有命,敢不竭心力。臣等有罪,聽諸臣彈劾。諸漢官如或抗令欺公,誑言誤事,諉避偷安,玩法科斂,臣等當彈劾,不敢避忌。惟慮諸漢官茹怨,以蜚語中臣等,臣等得罪,雖死不知其故。乞上及諸貝勒鑒臣等意,今後有過失,即時處分;有讒言,即時質問:俾僉邪不得行其險慝。」三月,光遠疏陳整飭軍政:省戎器,視牧馬,習砲,治砲車,節火藥,謹城守,制火箭,建藏砲儲藥之局,贍鑄砲造藥之役,厚養砲兵,凡十事。七月,上命舊隸滿洲戶下漢人十丁授棉甲一,得千五百八十人,命光遠等統之,分補舊甲喇缺額。

時孔有德、耿仲明來降,克旅順。光遠言:「有德等初來,登萊、旅順並各島兵艦隨至江口不敢歸,畏明法也。今旅順既失,江口兵艦必退保登萊。宜急遣水師逐彼舟後,乘風而西。上親帥師取山海,進攻北京,不半載大事可定。」十月,授一等總兵官。八年三月,疏請出師:「一自薊東入,一自八里鋪趨山海關,內外夾攻,先取其水關,則山海關易下也。既克山海關,還取祖大壽,整旅而西,進攻北京,塞沖要,阻運道,不數月必有內變。但乞上於出師之日,戒諭將士,毋殺,毋淫,毋掠貨財,毋焚廬舍。四方聞之,皆引領而歸上矣。」四月,改一等昂邦章京。九年七月,甄別轄治漢人各官,以各堡戶丁增減行賞罰,丁減初額三之一者削世職為民。光遠疏言:「各官功次不等,皆蒙敕賜世襲,得之至艱。今以養人不如法,皆罷為民,眾情驚懼。乞恩從重議罰,而毋遽奪世職;令戴罪視事,使功不如使過。臣為王法持平,敢昧死以請。」梅勒額真張存仁亦以為言,上從之。十年四月,諸臣勸進,漢將列孔有德、耿仲明、尚可喜、石廷柱及光遠,凡五人。

崇德元年十二月,從伐朝鮮,克平壤、江華島。二年八月,分烏真超哈為兩翼,置固山額真二。以廷柱轄左翼,光遠轄右翼。三年,上伐明,攻錦州,烏真超哈運火器為前驅。尋與有德以火器克臺五,復與廷柱克李雲屯、柏士屯、郭家堡、開州、井家堡,俘七百有三,得牲畜稱是。光遠率甲喇額真郎紹貞圍攻錦州城旁臺,敵遁,不追擊,上詰之,光遠妄辯,當奪職,上命罰鍰。四年,上復伐明,光遠以所部克松山西南隅臺,降其將楊文顯,攻城不克。語詳石廷柱傳。師還,數其罪而罷之;又以庇所部參將季世昌鑄砲子不中程,論死,上特宥之。六月,析烏真超哈為八旗,置固山額真四,復起光遠為正黃、鑲黃兩旗固山額真。漢軍旗制定,光遠隸鑲黃旗。順治四年,以老病乞休。康熙二年,卒,謚誠順。以弟之子思文襲爵。恩詔進三等伯。乾隆初,定封一等子。

光輝,光遠弟。明武舉。與其兄光先從光遠來降。天聰七年,授光先二等參將,光輝游擊。崇德三年,任戶部理事官。以貸官商物不償,罷官,奪世職。四年六月,漢軍旗制定,授鑲黃旗梅勒額真。六年,兼任吏部。七年,以從克杏山城,復世職。師已克錦州,命光輝從固山額真孟喬芳詣錦州監鑄砲。八年,以從克中後所、前屯衛二城,進一等甲喇章京。

世祖入關定鼎,參政改侍郎,光輝仍貳吏部。順治四年,考滿,加拖沙喇哈番。五年,從征南大將軍譚泰討江西叛將金聲桓,聲桓既誅,譚泰將承制授光輝江西提督,光輝辭。既,譚泰欲以都察院理事官紀國先為都司,國先亦辭。譚泰劾國先,辭連光輝,吏議從重比,上命罷光輝梅勒額真、侍郎,降世職為拜他喇布勒哈番。七年,復任梅勒額真。八年,上命追錄光輝軍功,屢遇恩詔,累進三等阿思哈尼哈番。五月,授戶部侍郎。十月,命以兵部尚書、右副都御史總督直隸、山東、河南三行省。

十年九月,膠州總兵海時行叛,為暴萊、沂間,光輝帥師討之。時行走宿遷,師從之,復走永城。光輝會漕運總督沈文奎帥師自靈壁向永城,戰洪河集西,大破之,縛時行以歸。以功加級,任子。十一年,甄別諸督撫,加太子少保,以老病乞休。十二年七月,卒,謚忠靖。

光先,順治間遇恩詔,亦授三等阿思哈尼哈番,官山西左布政使。

李思忠,字葵陽,鐵嶺人。父如梴,明遼東總兵官寧遠伯成梁族子也,仕明為太原同知,罷歸居撫順。太祖天命三年,始用兵於明,克撫順,得思忠,如梴徙還鐵嶺。明年,師下鐵嶺,如梴及弟如梓子一忠、存忠死之。六年,定遼陽,敕思忠收其族人,俾復故業,即授牛錄額真,予世職備御。尋以獲諜,進游擊。

天聰三年,太宗自將伐明,取永平等四城。師還,貝勒阿敏護諸將分守,察哈喇以蒙古兵守遵化,思忠及甲喇額真英固勒岱等為之佐。既而明將謝尚忠等來攻,思忠與戰,敵三進三卻。阿敏議棄四城東還,檄察哈喇合軍出塞。時尚忠攻遵化正急,發火箭焚我軍火器,我軍方恇擾,思忠戒無輕動,徐結陣出城,挾降吏四人以俱,身為殿,出塞無一亡失。師還,上譴阿敏等,以思忠力戰,貸勿罪。五年,從固山額真楞格裡等伐明,攻南海島,未至,遇明兵茨榆坨,俘十一人,得舟五。明兵爭舟,思忠與戰,砲傷額,勿卻,卒敗明兵,進二等參將。九年,察漢官所領城堡戶口盈耗,思忠轄沙河堡郎寨,增丁百十有三,上嘉賞,賜狐裘一襲,進三等梅勒章京。尋命駐蓋州。崇德二年,命修遼陽諸城,思忠疏言:「蓋州處邊,士卒任防守,餘丁僅足以耕。今棄農就役,工窳而農亦廢。請俟諸城工竟,庀役造磚從事。」上允其請。七年,漢軍旗制定,隸正黃旗。

順治元年,從豫親王多鐸徇陜西,破潼關;下江南,克揚州,撫定江北州縣凡十。三年二月,命以梅勒額真戍西安。三月,擢陜西提督。恩詔,累進一等阿思哈尼哈番兼拖沙喇哈番。十一年,致仕。十四年七月,卒。

思忠子五,第三子顯祖襲爵,世祖賜名塞白理,授二等侍衛、甲喇額真。康熙初,授隨征江南左路總兵官,遷廣東水師提督,改浙江提督。耿精忠叛,自福建侵浙江,塞白理疏請分兵援臺州,防寧波。尋從貝子傅喇塔擊走精忠將曾養性。十四年九月,卒於軍。乾隆初,定封三等男。

廕祖,思忠次子。事世祖,自廕生授戶部員外郎,三遷兵部侍郎。順治十一年,直隸災,命與尚書巴哈納等治賑。尋授兵部尚書,右副都御史,總督直隸、山東、河南三行省。疏請蠲被災諸州縣秋糧,招流民還故里,當隨地安集,以時予賑,毋使道殕。又疏言:「直隸濱海北塘、澗河、黑洋諸地,宜分兵駐守。」時議禁海船,魚鹽米麥不能轉輸,請官為編號,譏其出入,則商民皆便。並下部議行。

十四年四月,疏發河南管河道方大猷貪惏狀,上切責河道總督楊方興失劾,奪大猷官,鞫治論死。有高鼎者,據五臺山為亂,出三岔口擾真定,廕祖遣井陘道陳安國諭降,悉散其黨。疏言:「太行天下險,三岔居其沖,林密山深,藏奸甚易。自鼎降,其黨散在民間,雖戍以兵,視營壘為傳舍。當置游擊一,定額兵六百,耑司守御。」上從之。

是歲廕祖年才二十有九,會湖南北用兵,上察廕祖才,加太子太保,移督湖廣。師方徇貴州,故李自成諸將郝永忠、袁宗第、劉體純、李來亨輩挾十餘萬人降於明,踞鄖、襄間,擾餉道。廕祖請選襄陽水師及均、黃、漢陽諸營兵二千人戍穀城,地扼上游;選武昌洞庭營兵千人戍九谿,斷通蜀道。十五年,漢陽、天門、潛江、沔陽諸郡縣水災,上命廕祖治賑,民賴以得拯。

十六年,經略大學士洪承疇疏請發湖北提鎮標兵六千人戍雲南,廕祖以承疇已發湖廣兵萬三千五百有奇,湖南新收降人數萬,鄖、襄間流賊未殄,留兵不宜復發,請敕承疇就滇中召募,下部議行。復疏議:「討永忠等,請敕四川總督李國英帥師駐重慶,扼巫峽,阻達州;西安將軍富喀禪帥師趨興安;河南協剿兵詣襄陽合軍。臣督諸軍分出彞陵、襄陽、鄖陽,三道深入,期一舉滅賊。」疏既上,會鄭成功犯江南,詔將軍明安達哩將荊州駐防兵赴援,部議緩師期。十七年,以疾乞罷。康熙三年,卒。祀直隸、山東、河南、湖廣名宦。

鈵,廕祖子。事聖祖,自佐領授兵部員外郎。十三年,以參將從征吳三桂,再遷御史。二十七年,湖廣夏逢龍為亂,上授鈵湖北按察使。累擢兵部侍郎。三十五年,上親征噶爾丹,命與左都御史於成龍等督餉。三十七年,授山東巡撫,以疾辭,改授安徽巡撫。三十九年,疾未瘳,被彈事罷。四十二年,山東饑,鈵請往助賑,卒於賑所。

金玉和,遼東人。仕明為開原千總。太祖克開原,玉和降,授甲喇額真,予世職三等副將。漢軍旗制定,隸正黃旗。天聰五年,擢禮部承政。六年,上閱兵,玉和與額駙佟養性等率所轄烏真超哈擐甲列陣試砲,上賚以鞍馬。八年,考績,進二等副將。崇德元年,坐與吏部參政李延庚互舉子弟,罷官,降世職三等甲喇章京。二年,從武英郡王阿濟格伐明皮島,以水師戰不利,玉和不赴援,論死,上特宥之,但削世職。四年,復授甲喇額真。六年,從圍明錦州,屢敗敵。敵夜攻壕塹,擊卻之,斬級五十。七年,錦州下,並克塔山,予世職牛錄章京。八年,從鄭親王濟爾哈朗伐明寧遠,與王國光同克前屯衛、中後所二城。順治元年,擢工部參政。敘寧遠功,進三等甲喇章京。既入關,遷梅勒額真。從軍河南,署懷慶總兵官。時李自成竄陜西,餘黨掠河南,犯濟源縣城,玉和帥師往援,至則城已陷,夜半遇賊,力戰,中流矢,沒於陣。河南巡撫羅繡錦疏報得玉和遺骸於柏鄉西,請賜恤,進二等梅勒章京。乾隆初,定封二等男。

金維城,玉和子也。崇德初,師攻錦州,維城以甲喇額真奉命與梅勒額真金礪督餉,屢從伐明有功。克中後所、前屯衛二城,維城亦在行間。累官正白旗漢軍梅勒額真,兼兵部參政,世職至牛錄章京。從入關,改兵部侍郎,兼梅勒額真如故。順治四年,改拜他喇布勒哈番。考績,加拖沙喇哈番。復從定湖廣,同克武岡、沅州、靖州,進一等阿達哈哈番。調正黃旗漢軍梅勒額真。十年,坐總兵任珍行賕罷官,降世職為三等。十五年,卒。

子世礪,康熙間,以佐領從平南大將軍賚塔征福建,敗敵江東橋。鄭成功將劉國軒攻漳州,世礪戰死,予世職拖沙喇哈番。

太祖克開原,玉和與同官王一屏、戴集賓、白奇策,守堡百總戴一位降;下廣寧,游擊孫得功,守備張士彥、黃進、石廷柱,千總郎紹貞、陸國志、石天柱降;收遼河諸城堡,參將劉世勛,游擊羅萬言、何世延、閻印,都司金礪、劉式章、李維龍、王有功、陳尚智,備御硃世勛、黃宗魯,中軍王志高,守堡閔雲龍、俞鴻漸、鄭登、崔進忠、李詩、徐鎮靜、鄭維翰、臧國祚、周元勛、王國泰,各以所守城堡來降。玉和、一屏、得功、士彥、廷柱、礪皆以有功授世職。廷柱自有傳。

王一屏,先世本滿洲,姓完顏氏。初降,授牛錄額真。漢軍旗制定,隸正紅旗。天聰八年,授世職三等甲喇章京。旋卒。

子國光,以牛錄額真兼戶部理事官,襲職。擢正紅旗漢軍梅勒額真,兼戶部參政。八年,從鄭親王濟爾哈朗伐明,克前屯衛、中後所二城,進二等甲喇章京。順治元年,改戶部侍郎,兼梅勒額真如故。從定西大將軍和洛輝禦寇西安。考滿,進一等阿達哈哈番。遷本旗固山額真。六年,從英親王阿濟格討叛將姜瓖,克左衛、朔州、汾州、太谷四城。敘功,遇恩詔,累進一等阿思哈尼哈番。十年,從定遠大將軍、貝勒屯齊征湖廣,擊敗明將李定國、孫可望。十二年,從寧海大將軍伊勒德援浙江,擊敗明將鄭成功、張名振。十三年二月,授兩廣總督,諭獎其才品,賜蟒服、鞍馬,加太子太保。十五年,以疾解任。十八年,聖祖即位,授鎮海將軍,帥師鎮潮州。康熙三年,與平南王尚可喜會師討碣石叛將蘇利,師至海豐,偵破敵伏,逕燈籠山。蘇利乘我軍未成列,以萬餘人搏戰,我軍左右夾擊,賊潰遁。薄碣石衛城,環攻拔之,斬蘇利及所部陳英、李慧等,遂殲其餘黨。五年,還京,仍任本旗都統。九年,卒,謚襄壯。

子永譽,字孝揚,襲爵。十二年,授河南提督。河北總兵蔡祿叛應吳三桂,內大臣阿密達帥師討之。上命永譽如懷慶,拊循士卒,因請留駐鎮撫。旋設安慶提督,以授永譽。耿精忠將宋標方自饒州犯徽州,十四年,永譽督兵駐建德,令參將傅爾學破標於餘干,俘標,磔於市。尋移駐徽州。十七年,江西平。改永譽江南提督,駐松江。十九年,遷廣東將軍。二十年,疏言:「廣東瀕海,陸路兩鎮,請各以一營改練水師。」二十二年,復請留滿洲兵四千駐防廣東省城。皆如所議行。二十七年,授本旗都統。二十三年三月,命定北將軍瓦岱帥師屯張家口,詗噶爾丹,以永譽與都統喀岱等參贊軍務。三十五年,上親征噶爾丹,分漢軍為四營,永譽帥正黃、正紅二旗出中路,噶爾丹不戰遁。永譽與平北將軍馬斯喀督兵追躡,偵噶爾丹行遠,乃還。三十六年,從上至寧夏,命督餉運,貯黃河西岸,聞噶爾丹竄死,罷,還。四十三年,卒。乾隆十八年,命其族改隸滿洲正紅、鑲白二旗。

孫得功,在明為廣寧巡撫王化貞中軍游擊,化貞倚得功為心膂。太祖圍西平堡,劉渠等赴援,令得功從。渠等戰死,得功潛納款於太祖,還言師已薄城,城人驚潰。化貞走入關,得功與進、紹貞、國志等,率士民出城東三里望昌岡,具乘輿,設鼓樂,執旗張蓋,迎太祖入駐巡撫署,士民皆夾道俯伏呼萬歲。時天命七年正月庚申,月之二十四日也。上授得功游擊,隸鑲白旗,轄降眾,移駐義州。天聰六年十月,得功疏言:「上命修城,天寒土凍,徒勞民力而不能堅固,請俟春融。又上發帑畀官兵市布制冬衣,官已足用,兵人給銀五錢六分,得布不足以為衣,乞恩使人得市布一二疋,官兵均霑上澤。」七年四月,又疏言:「禁淡巴菰,令未能行。步兵皆用火器,尤宜申諭戒革。上令民輸糧,因禁百穀不得入市,貧民無所得食,則宜任民便。」八年,追敘得功廣寧功,授三等梅勒章京。旋卒,以其子孫有光襲。漢軍旗制定,改隸正白旗。以從克前屯衛、中後所及順治間討姜瓖有功,並遇恩詔,進三等精奇尼哈番。卒。乾隆初,定封一等男。得功次子思克,自有傳。

張士彥,化貞中軍守備。太祖兵至,化貞走入關,士彥降。漢軍旗制定,隸正藍旗。天聰八年,與一屏同授三等甲喇章京。旋乞休。

子朝璘,襲職。崇德七年,授牛錄額真。從貝勒阿巴泰伐明,敗敵於膠州。八年,與國光同功,進二等甲喇章京。遷兵部理事官。順治二年,從豫親王多鐸下江南,克揚州、江陰,率兵戍蘇州,擊敗明將黃斐。四年,從恭順王孔有德等平湖南,破明將劉承胤於夕陽橋,克武岡;復破明將張先璧於黔陽,克沅州。六年,從討姜瓖,復與國光同功。考滿,遇恩詔,進一等阿思哈尼哈番,授正藍旗漢軍梅勒額真。十年,授都察院左副都御史。十三年,遷戶部侍郎。尋出為江西巡撫。江西當金聲桓亂後,民少田蕪,御史笪重光請蠲賦,下朝璘議。朝璘疏言:「田畝荒蕪,惟從容勸墾,則熟者恆熟,荒者不終荒。若急於徵賦,則始以荒為熟,漸至熟者仍荒,非足國恤民計也。南昌、瑞州二府新墾田四十餘頃,請三年後起科;未墾二千餘頃,請與豁除。」上允其請。十五年,加兵部尚書銜。十八年,擢江西總督。康熙二年,右布政使王庭疏請減南昌府屬浮糧,下朝璘議。朝璘疏言:「江西重賦,自陳友諒始,明世因之。前巡撫蔡士英請減袁、瑞二府賦額,未及南昌。南昌諸州縣,惟武寧為友諒鄉里,賦額循元、宋之舊。他六縣一州,請敕部核減。」戶部覈上南昌府屬浮糧銀十二萬五千有奇、米十四萬九千有奇,上命悉蠲之。三年,朝璘疏言:「吉安舊食粵鹽,遠且阻,請改食淮鹽。」下所司從之。四年,以江西總督省入江南,解任。五年,授福建總督。六年,以老疾乞休。越十餘年,卒。

金礪,遼東人。明武進士,為鎮武堡都司。初降,授甲喇額真,予世職三等副將。天聰五年,始設六部,以礪為兵部承政。六年,上閱兵,與玉和等並賜鞍馬。調戶部承政。八年,考績,進二等梅勒章京。崇德二年,從伐明,攻皮島,甲喇額真巴雅爾圖等先入敵陣,礪與副將高鴻中所將水師不進,前軍以是敗,坐論死,上以礪與鴻中來歸有功,特宥之。四年,漢軍旗制定,礪隸鑲紅旗,復為甲喇額真。五年,授吏部參政。六年,擢固山額真。迭克松山、塔山、前屯衛、中後所,授世職三等甲喇章京。順治元年,從入關,五月,與梅勒額真李率泰安集天津亂民;六月,復與固山額真葉臣宣撫山西。時李自成西遁,其將陳永福猶據太原,礪與葉臣潛往覘焉,城兵驟出,礪擊敗之,督本旗兵發砲克其城。師還,賜白金四百兩,進世職二等甲喇章京。二年,從順承郡王勒克德渾征湖廣,明將馬進忠降復叛,礪與固山額真劉之源擊進忠武昌,奪戰艦六十餘,遂下湖南,戰衡州,斬明將黃朝宣;復戰長沙,斬明將楊國棟。師還,賜黃金二十兩、白金四百兩,進世職一等阿達哈哈番。

六月,授平南將軍,鎮浙江。遇恩詔,加拖沙喇哈番。明魯王以海及其臣阮進、張名振屯舟山,礪與梅勒額真吳汝玠等率兵自寧波出定海,會總督陳錦破獲進於橫洋,遂克舟山,名振擁以海出走。九年,鄭成功攻漳州,命礪帥師赴援,至泉州,成功退屯江東橋。礪自長泰進屯漳州城北,分兵萬松關為犄角,七戰皆勝,漳州圍解,海澄、南靖、漳浦諸縣悉定。敘功,遇恩詔,進一等阿思哈尼哈番兼拖沙喇哈番。十一年,授陜西四川總督。十三年,引年乞休,加太子太保致仕。康熙元年,卒。

論曰:養性、廷柱先世本滿洲,懷舊來歸,申以婚媾。永芳歸附最先,思忠為遼左右族,皆蒙寵遇,各有賢子,振其家聲。光遠初佐養性,後與廷柱分將漢軍,罷而復起。玉和戰死。同時諸降將有績效,賞延於世,或其子顯者,得以類從。後先奔走,才亦盛矣。


\end{pinyinscope}