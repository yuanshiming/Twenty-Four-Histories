\article{志八十三}

\begin{pinyinscope}
○選舉三

△文科武科

有清科目取士,承明制用八股文。取四子書及易、書、詩、春秋、禮記五經命題,謂之制義。三年大比,試諸生於直省,曰鄉試,中式者為舉人。次年試舉人於京師,曰會試,中式者為貢士。天子親策於廷,曰殿試,名第分一、二、三甲。一甲三人,曰狀元、榜眼、探花,賜進士及第。二甲若干人,賜進士出身。三甲若干人,賜同進士出身。鄉試第一曰解元,會試第一曰會元,二甲第一曰傳臚。悉仍明舊稱也。世祖統一區夏,順治元年,定以子午卯酉年鄉試,辰戌丑未年會試。鄉試以八月,會試以二月。均初九日首場,十二日二場,十五日三場。殿試以三月。

二年,頒科場條例。禮部議覆,給事中龔鼎孳疏言:「故明舊制,首場試時文七篇,二場論、表各一篇,判五條,三場策五道。應如各科臣請,減時文二篇,於論、表、判外增詩,去策改奏疏。」帝不允。命仍舊例。首場四書三題,五經各四題,士子各占一經。四書主硃子集註,易主程傳、硃子本義,書主蔡傳,詩主硃子集傳,春秋主胡安國傳,禮記主陳澔集說。其後春秋不用胡傳,以左傳本事為文,參用公羊、穀梁。二場論一道,判五道,詔、誥、表內科一道,三場經史時務策五道。鄉、會試同。乾隆間,改會試三月,殿試四月,遂為永制。

鄉試,先期提學考試精通三場生儒錄送,禁冒濫。在監肄業貢、監生,本監官考送。倡優、隸、皁之家,與居父母喪者,不得與試。卷首書姓名、籍貫、年貌、出身、三代、所習本經。試卷題字錯落,真草不全,越幅、曳白,塗抹、污染太甚,及首場七藝起訖虛字相同,二場表失年號,三場策題訛寫,暨行文不避廟諱、御名、至聖諱,以違式論,貼出。士子用墨,曰墨卷。謄錄用硃,曰硃卷。主考墨筆,同考藍筆。乾隆間,同考改用紫筆。未幾,仍用藍。試士之所曰貢院,士子席舍曰號房,撥軍守之曰號軍。試官入闈封鑰,內外門隔以簾。在外提調、監試等曰外簾官,在內主考、同考曰內簾官。亦有內監試,司糾察,不與衡文事。以大員總攝場務,鄉試曰監臨。順天以府尹,各省初以巡按御史,巡按裁,巡撫為之。會試曰知貢舉,禮部侍郎為之。順天提調以府丞,監試以御史。初,各省提調以布政使,監試以按察使,各副以道員。雍正間,以籓、臬兩司為一省錢穀、刑名之總匯,入闈月餘,恐致曠滯,提調監試,專責二道員。會試監試以御史。殿試臨軒發策,以朝臣進士出身者為讀卷官,擬名第進呈,或如所擬,或有更定。一甲狀元授修撰,榜眼、探花授編修,二、三甲進士授庶吉士、主事、中書、行人、評事、博士、推官、知州、知縣等官有差。

有清以科舉為掄才大典,雖初制多沿明舊,而慎重科名,嚴防弊竇,立法之周,得人之盛,遠軼前代。其間條例之損益,風會之變遷,系乎人才之盛衰,朝政之得失。述其大者,不可闕也。

鄉、會試首場試八股文,康熙二年,廢制義,以三場策五道移第一場,二場增論一篇,表、判如故。行止兩科而罷。四年,禮部侍郎黃機言:「制科向系三場,先用經書,使闡發聖賢之微旨,以觀其心術。次用策論,使通達古今之事變,以察其才猷。今止用策論,減去一場,似太簡易。且不用經書為文,人將置聖賢之學於不講,請復三場舊制。」報可。七年,復初制,仍用八股文。二十四年,用給事中楊爾淑請,禮闈及順天試四書題俱欽命。時詔、誥題士子例不作,文、論、表、判、策率多雷同剿襲,名為三場並試,實則首場為重。首場又四書藝為重。二十六年廢詔、誥,既而令五經卷兼作。論題舊出孝經,康熙二十九年,兼用性理、太極圖說、通書、西銘、正蒙。五十七年,論題專用性理。世宗初元,詔孝經與五經並重,為化民成俗之本。宋儒書雖足羽翼經傳,未若聖言之廣大,論題仍用孝經。

乾隆三年,兵部侍郎舒赫德言:「科舉之制,憑文而取,按格而官,已非良法。況積弊日深,僥幸日眾。古人詢事考言,其所言者,即其居官所當為之職事也。時文徒空言,不適於用,墨卷房行,展轉抄襲,膚詞詭說,蔓衍支離,茍可以取科第而止,士子各占一經,每經擬題,多者百餘,少者數十。古人畢生治之而不足,今則數月為之而有餘。表、判可預擬而得,答策隨題敷衍,無所發明。實不足以得人。應將考試條款改移更張,別思所以遴拔真才實學之道。」章下禮部,覆奏:「取士之法,三代以上出於學,漢以後出於郡縣吏,魏、晉以後出於九品中正,隋、唐至今,出於科舉。科舉之法不同,自明至今,皆出於時藝。科舉之弊,詩、賦祗尚浮華,而全無實用。明經徒事記誦,而文義不通。唐趙匡所謂『習非所用,用非所習』是也。時藝之弊,今該侍郎所陳奏是也。聖人不能使立法之無弊,在因時而補救之。蘇軾有言:『得人之道,在於知人。知人之道,在於責實。』能責實,雖由今之道,而振作鼓舞,人才自可奮興。若惟務徇名,雖高言復古,法立弊生,於造士終無所益。今謂時文、經義及表、判、策論皆空言剿襲而無用者,此正不責實之過。凡宣之於口,筆之於書,皆空言也,何獨今之時藝為然?時藝所論,皆孔、孟之緒言,精微之奧旨。參之經史子集,以發其光華;範之規矩準繩,以密其法律。雖曰小技,而文武幹濟、英偉特達之才,未嘗不出乎其中。不思力挽末流之失,而轉咎作法之涼,不已過乎?即經義、表、判、論、策,茍求其實,亦豈易副?經文雖與四書並重,積習相沿,士子不專心學習。若著為令甲,非工不錄。表、判、論、策,皆加覆覈。必淹洽詞章、通曉律令,而後可為表、判。有論古之識,斷制之才,通達古今,明習時務,而後可為論、策。何一不可見之施為,切於實用?必變今之法,行古之制,將治宮室、養游士,百里之內,置官立師,訟獄聽於是,軍旅謀於是。又將簡不率教者,屏之遠方,終身不齒。毋乃紛擾而不可行?況人心不古,上以實求,下以名應。興孝則有割股、廬墓以邀名者矣,興廉則有惡衣菲食、敝車羸馬以飾節者矣。相率為偽,借虛名以干進取。及蒞官後,盡反所為,至庸人之不若。此尤近日所舉孝廉方正中所可指數,又何益乎?司文衡職課士者,誠能仰體諭旨,循名責實,力除積習,杜絕僥幸,文風日盛,真才自出,無事更張定制為也。」遂寢其議。時大學士鄂爾泰當國,力持議駁,科舉制義得以不廢。

二十二年,詔剔舊習、求實效,移經文於二場,罷論、表、判,增五言八韻律詩。明年,首場復增性理論。御史楊方立疏請鄉、會試增周禮、儀禮二經命題。帝以二禮義蘊已具於戴記,不從。四十七年,移置律詩於首場試藝後,性理論於二場經文後。五十二年,高宗以分經閱卷,易滋弊竇。且士子專治一經,於他經不旁通博涉,非敦崇實學之道。命自明歲戊申鄉試始,鄉、會五科內,分年輪試一經。畢,再於鄉、會二場廢論題,以五經出題並試。永著為令。

科場擬題最重。康熙五十二年,以主司擬題,多取四書、五經冠冕吉祥語,致多宿構幸獲。詔此後不拘忌諱。向例禁考官擬出本身中式題,至是弛其禁。歷科試官,多有以出題錯誤獲譴者。先是康熙五十六年,從詹事王奕清言,場中七藝,破、承、開講,虛字概不謄寫,以防關節。乾隆四十七年,令考官預擬破、承、開講虛字,隨題紙發給士子遵用。嘉慶四年,以無關弊竇,廢止。制藝篇末用大結,有明中葉,每以此為關節。康熙末年,懸之禁令。乾隆十二年,編修楊述曾有復用大結之請,大學士張廷玉等以為無益而弊竇愈起,奏駁之。初場文原定每篇限五百五十字,康熙二十年增百字。五十四年,會元尚居易以首藝字逾千二百,黜革。乾隆四十三年,始定鄉、會試每篇以七百字為率,違者不錄。自是遵行不易。三場策題,原定不得逾三百字。乾隆元年,禁士子空舉名目,草率塞責。其後考官擬題,每問或多至五六百字,空疏者輒就題移易,點竄成篇。三十六年,左都御史張若溎以為言,詔申明定例。五十一年,定答策不滿三百字,照紕繆例罰停科。然考官士子重首場,輕三場,相沿積習難移。制義體裁,以詞達理醇為尚。順治九年壬辰,會試第一程可則以悖戾經旨除名。考官學士胡統虞等並治罪。

世宗屢以清真雅正誥誡試官。乾隆元年,高宗詔曰:「國家以經義取士,將以覘士子學力之淺深,器識之淳薄。風會所趨,有關氣運。人心士習之端倪,呈露者甚微,而徵應者甚鉅。當明示以準的,使士子曉然知所別擇。」於是學士方苞奉敕選錄明、清諸大家時文四十一卷,曰欽定四書文,頒為程式。行之既久,攻制義者,或剽竊浮詞,罔知根柢,楊述曾至請廢制義以救其弊。四十五年,會試三名鄧朝縉首藝語意粗雜,江南解元顧問四書文全用排偶,考官並獲譴。嘉慶中,士子撏撦僻書字句,為文競炫新奇,御史辛從益論其失。詔曰:「近日士子獵取詭異之詞,以艱深文其淺陋,大乖文體。考官務各別裁偽體。支離怪誕之文,不得錄取。」歷代輒以釐正文體責考官,而迄無實效。議者謂文風關乎氣運。清代名臣多由科目出身,無不工制義者。開國之初,若熊伯龍、劉子壯、張玉書,為文雄渾博大,起衰式靡。康熙後益軌於正,李光地、韓菼為之宗。桐城方苞以古文為時文,允稱極則。雍、乾間,作者輩出,律日精而法益備。陵夷至嘉、道而後,國運漸替,士習日漓,而文體亦益衰薄。至末世而剿襲庸濫,制義遂為人詬病矣。

光緒二十四年,湖廣總督張之洞有變通科舉之奏。二十七年,鄉、會試首場改試中國政治史事論五篇,二場各國政治藝學策五道,三場四書義二篇、五經義一篇,其他考試例此。用之洞議也。行之至廢科舉止。

鄉、會考官,初制,順天、江南正、副主考,浙江、江西、湖廣、福建正主考,差翰林官八員。他省用給事中、光祿寺少卿、六部司官、行人、中書、評事。某官差往某省,皆有一定。康熙三年除其例。順天初同各省,簡正、副二人。乾隆中葉增為三,用協辦大學士、尚書以下,副都御史以上官,編、檢不復與矣。道光中,簡三四人。同治後,額簡四人。初,考官不限出身,康熙初,主事蔡騶、曹首望俱以拔貢典試。十年,從御史何元英請,考官專用進士出身人員。然舉人出身者間亦與焉。雍正三年,頒考試令,始限翰林及進士出身部、院官,仍參用保舉例。乾隆九年,御史李清芳言:「大臣保舉應差主考四十九人,滿洲四,各直省十六,餘均江、浙人。保薦者大都平日往來相知,饒於財而憑於勢。至守正不阿者,不肯伺候公卿之門,邊隅之士,聲氣不通,交游不廣,無人薦舉。請將合例人員通行考試。」帝疑清芳未列保薦,激為是語,不允所請,仍考試、保舉並行。三十六年後,考試遂著為令。初御試錄取名單皆發出,其後密定名次,不復揭曉。嘉慶以後,更別試侍郎、閣學及三品京堂等官,曰大考差。會試總裁,初用閣、部大員四人或六人,多至七人。嗣簡二三人或四五人。咸豐後,簡四人,以為常。

同考官,初,順天試京員,推、知並用。各省用甲科屬官及鄰省甲科推、知,或鄉科教官,房數無定。會試初用二十人,翰林官十二,六科四,吏、禮、兵部官各一,戶、刑、工部官每科輪用一。嗣額定十八人,順天試同。康熙五十四年,令不同省房官二人同閱,互相覺察,用三十六人。未幾即罷。康、雍間,順天房考停用京員,止用直隸科甲知縣。各省停用本省現任知縣,專調用鄰省在籍候選進士、舉人。大省十八,中省十四,小省十二至十,均分經校閱。厥後增減不一,小省減至八人。乾隆間,禮闈及順天同考,始欽簡京員,各省復用本省科甲屬官。四十二年,停五經分房之例。至順天房考,南、北省人回避南、北皿卷,邊省人回避中皿卷,會房則同省相回避雲。

考官綜司衡之責,房考膺分校之任,歷代極重其選。康熙間,順天同考官庶吉士鄭江以校閱允當,授職檢討。雍正元年,會試總裁硃軾、張廷玉持擇公允,帝嘉之,加太傅、太保有差。其衡鑒不公、草率將事者,罰不貸。而交通關節賄賂,厥辜尤重。順治十四年丁酉,順天同考官李振鄴、張我樸受科臣陸貽吉、博士蔡元禧、進士項紹芳賄,中田耜、鄔作霖舉人。給事中任克溥奏劾,鞫實。詔駢戮七人於市,家產籍沒,戍其父母兄弟妻子於邊。考官庶子曹本榮、中允宋之繩失察降官。江南主考侍講方猶、檢討錢開宗,賄通關節,江寧書肆刊萬金傳奇記詆之。言官交章論劾,刑部審實。世祖大怒,猶、開宗及同考葉楚槐等十七人俱棄市,妻子家產籍沒。一時人心大震,科場弊端為之廓清者數十年。康熙五十年辛卯,江南士子吳泌、程光奎賂副考官編修趙晉獲中。二人素不能文,輿論譁然。事聞,命尚書張鵬翮會江南督、撫嚴鞫。蘇撫張伯行劾總督噶禮賄賣徇庇,噶禮亦劾伯行他罪,詔俱解任。令鵬翮會總漕赫壽確訊,覆奏請鐫噶禮級,罷伯行職。帝怒二人掩飾和解,復遣尚書穆和倫、張廷樞往鞫,奏略如鵬翮等指。部議,互訐乖大臣體,應並褫職。帝卒奪噶禮職。以伯行清名素著,褫職仍留任。處晉及同考王曰俞、方名大闢,以失察奪正考官左必蕃官。是年福建房考吳肇中亦以賄伏法,考官檢討介孝、主事劉儼失察削職。咸豐八年戊午,順天舉人平齡硃、墨卷不符,物議沸騰,御史孟傳金揭之。王大臣載垣等訊得正考官大學士柏葰徇家人靳祥請,中同考編修浦安房羅鴻繹卷。比照交通囑託、賄買關節例,柏葰、浦安棄市,餘軍、流、降、革至數十人。副考官左副都御史程庭桂子郎中炳採,坐接收關節伏法,庭桂遣戍。蓋載垣、端華及會審尚書肅順素惡科目,與柏葰有隙。因構興大獄,擬柏葰極刑。論者謂靳祥已死,未為信讞也。然自嘉、道以來,公卿子弟視巍科為故物。斯獄起,北闈積習為之一變。光緒十九年,編修丁維禔典陜試,同年友饒士騰先期為之展轉囑託。事覺,俱逮問。士騰自殺,尋並削職。有無與關節賄賂而獲咎者,康熙三十八年己卯,御史鹿佑劾順天闈考試不公,正考官修撰李蟠遣戍,副主考編修姜宸英牽連下吏,未置對,死獄中。宸英浙江名士,善屬古文,舉朝知其無罪,莫不嘆惜。四十四年乙酉,順天主考侍郎汪霦、贊善姚士藟校閱草率,落卷多不加圈點。下第者束草如人,至其門戮之。事聞,奪職。六十年辛丑,會試副總裁左副都御史李紱用唐人通榜法,拔取知名之士。下第者喧閧於其門,被劾落職,發永定河效力。然是闈一時名宿,網羅殆盡,頗為時論所許。其他賄通關節,未經敗露,與因微眚獲譴者,例尤不一。

鄉試解額,順治初定額從寬,順天、江南皆百六十餘名,浙江、江西、湖廣、福建皆逾百名,河南、山東、廣東、四川、山西、陜西、廣西、雲南自九十餘名遞殺,至貴州四十名為最少。俱分經取中。順天試直隸生員貝字號約占額十之七,北監生皿字號十之三,宣化旦字、奉天夾字僅二三名。江南試南監生皿字號約十之二,餘為江、安並闈生員額。南雍罷,南監中額並入北監。十四年,監生分南、北卷,直隸八府,延慶、保安二州,遼東、宣府、山東、山西、河南、陜西、四川、廣西為北皿,江南、浙江、江西、福建、湖廣、廣東為南皿,視人數多寡定中額。十七年,減各直省中額之半。康熙間,先後廣直省中額。五十年,又各增五之一。雍正元年,湖南北分闈,照舊額分中。各省略有增減。乾隆元年,順天皿字分南、北、中卷,奉天、直隸、山東、河南、山西、陜西為北皿,江南、江西、福建、浙江、湖廣、廣東為南皿,各中額三十九。四川、廣西、雲南、貴州另編中皿,十五取一。江南分上下江,取中下江江蘇十之六,上江安徽十之四。九年,嚴定搜檢之法。北闈以夾帶敗露者四十餘人,臨時散去者三千八百數十人,曳白與不終篇、文不切題者又數百人。帝既治學政、祭酒濫送之罪,詔減各直省中額十之一。於是定順天南、北皿各三十六,中皿改二十取一,貝字百二,夾、旦各四,江南上江四十五,下江六十九,浙江、江西皆九十四,福建八十五,廣東七十二,河南七十一,山東六十九,陜西六十一,山西、四川皆六十,雲南五十四,湖北四十八,湖南、廣西皆四十五,貴州三十六。自是率行罔越。光緒元年,陜、甘分闈,取中陜西四十一,甘肅三十。咸、同間,各省輸餉輒數百萬,先後廣中額。四川二十,江蘇十八,廣東十四,福建及臺灣十三,浙江、湖南、湖北、江西、山西、安徽、甘肅、雲南、貴州各十,陜西九,河南、廣西各八,直隸、山東各二。視初定中額尚或過之。

會試無定額,順治三年、九年俱四百名,分南、北、中卷。浙江、江西、福建、湖廣、廣東五省,江寧、蘇、松、常、鎮、淮、揚、徽、寧、池、太十一府,廣德一州為南卷,中二百三十三名。山東、山西、河南、陜西四省,順天、永平、保定、河間、真定、順德、廣平、大名八府,延慶、保安二州,奉天、遼東、大寧、萬全諸處為北卷,中百五十三名。四川、廣西、雲南、貴州四省,安、廬、鳳、滁、徐、和等府、州為中卷,中十四名。十二年,中卷並入南、北卷。厥後中卷屢分屢並,或更於南、北、中卷分為左、右。或專取川、廣、雲、貴四省,各編字號,分別中一、二、三名。五十一年,以各省取中人數多少不均,邊省或致遺漏,因廢南、北官、民等字號,分省取中。按應試人數多寡,欽定中額。歷科大率三百數十名,少或百數十名,而以雍正庚戌四百六名為最多,乾隆己酉九十六名為最少。

五經中式,仿自明代。以初場試書藝三篇,經義四篇,其合作五經卷見長者,因有「二十三篇」之目。順治乙酉,山東鄉試,法若真以全作五經文賜內閣中書,一體會試。康熙丁卯順天鄉試,浙江監生查士韓、福建貢生林文英,壬午順天南皿監生莊令輿、俞長策,皆以兼作四書、五經文二十三篇違式,奏聞,俱授舉人。詔嗣後不必禁止,旋著為令。鄉、會試五經卷,於額外取中三名。二場添詔、誥各一,於是習者益眾。直隸、陜西等省,至有以五經卷掄元者。五十年,增各省鄉試一名,順天二名,會試三名。五十六年,停五經應試。雍正初,復其制。順天皿字號中四名,各省每額九名加中一名。大省人多文佳,額外量取副榜三四名。四年丙午,詔是科以五經中副榜者,準作舉人,一體會試,尤為特異。乾隆十六年,始停五經中式之例。

至歷代臨雍,增北闈監生中額,恩詔廣鄉、會試中額,均屬於常額外也。鄉、會試正榜外取中副榜,會試副榜免廷試,咨吏部授職。康熙三年罷之。鄉試副榜原定順天二十名,江南十二,江西十一,浙江、福建、湖廣各十,山東、河南各九,山西、陜西、四川、廣東各八,廣西六。取文理優者,不拘經房。康熙元年停取。十一年,取中如舊例。增云南五,貴州四。嗣是各直省率正榜五名中一名,惟恩科廣額不與焉。雍正四年,準是科由副榜復中副榜者作舉人,非常例也。

雍正五年,命各省督、撫、學政甄別衰老教職休致之缺,以是年會試落卷文理明順之舉人補授。乾隆間,屢行選取如例,大、中、小省各數十名。明通別為一榜。二十六年,廷議於明通榜外選取中書四十名,其餘年力老成、宜課士者,另選用學正、學錄數名。報可。五十五年悉罷。此後下第者,於正榜外挑取謄錄,北闈數百名或百數十名。會試額定四十名,備各館繕寫,積資得邀議敘。此則旁搜博採、俾寒畯多獲進身之階也。

八旗以騎射為本,右武左文。世祖御極,詔開科舉,八旗人士不與。順治八年,吏部疏言:「八旗子弟多英才,可備循良之選,宜遵成例開科,於鄉、會試拔其優者除官。」報可。八旗鄉、會試自是年始。其時八旗子弟,每牛錄下讀滿、漢書者有定額,應試及各衙門任用,悉於此取給,額外者不得習。往往不敷取中。故自十四年至康熙十五年,八旗考試,時舉時停。先是鄉、會試,殿試,均滿洲、蒙古為一榜,漢軍、漢人為一榜。康熙二十六年,詔同漢人一體應試。尋定制,鄉、會場先試馬步箭,騎射合格,乃應制舉。庶文事不妨武備,遂為永制。初八旗鄉試,僅試清文或蒙古文一篇,會試倍之。漢軍試書藝二篇、經藝一篇,不通經者,增書藝一篇。二、三闈試論、策各一。逐科遞加,自與漢人合試,非復前之簡易矣。

鄉試中額,順治八年,定滿洲、漢軍各五十,蒙古二十,嗣減滿洲、漢軍各五之一,蒙古四之一。康熙八年,編滿、蒙為滿字號,漢軍為合字號,各取十名。二十六年,再減漢軍五名。後復遞增。乾隆九年,詔各減十之一,定為滿、蒙二十七,漢軍十二。同治間,以輸餉增滿、蒙六名,漢軍四名。各省駐防,初亦應順天試,嘉慶十八年,始於駐防省分試之。十人中一,多不逾三名,副榜如例。會試初制,滿洲、漢軍進士各二十五,蒙古十。康熙九年,編滿、合字號,如鄉試例,各中四名。嗣亦臨時請旨,無定額。

宗室不應鄉、會試,聖祖、世宗降有明諭。乾隆八年,宗人府試宗學,拔其尤者玉鼎柱等為進士,一體殿試,是為宗室會試之始。未久即停。嘉慶六年,宗室應鄉、會試始著為令。先期宗人府或奉天宗學考試騎射如例,試期於文闈鄉、會試場前,或場後,或同日,試制藝、律詩各一,一日而畢。鄉試九人中一人。會試,考官酌取數卷候親裁,別為一榜。殿試、朝考,滿、漢一體,除庶吉士等官有差。

順治十五年,帝以順天、江南考官俱以賄敗,親覆試兩闈舉人,是為鄉試覆試之始。取順天米漢雯等百八十二名,準會試。江南汪溥勛等九十八名,準作舉人。罰停會試、除名者二十二名。惟吳珂鳴以三次試卷文理獨優,特許一體殿試,異數也。康熙三十八年,帝以北闈取士不公,命集內廷覆試。列三等以上者許會試,四等黜之。五十一年壬辰,順天解元查為仁以傳遞事覺而逸,帝疑新進士有代倩中式者,親覆試暢春園,黜五人。會試覆試自是始。乾隆間,或命各省督、撫、學政於鄉試榜後覆試,或專覆試江蘇、安徽、江西、浙江、廣東、山西六省丙午前三科俊秀貢監中式者,或止覆試中式進士,或北闈舉人,臨期降旨,無定例。五十四年,貢士單可虹覆試詩失調訛舛,不符中卷,除名。詔旨嚴切,謂「禮闈非嚴行覆試,不足拔真才、懲幸進」。至嘉慶初,遂著為令。道光二十三年,定制,各省舉人,一體至京覆試,非經覆試,不許會試。以事延誤,於下三科補行。除丁憂展限外,託故不到,以規避論,永停會試與赴部銓選。覆試期以會試年二月。咸、同間,因軍興道路梗阻,光緒季年,以辛丑條約,京師停試,假闈河南,俱得先會試後覆試,非恆制也。覆試詩文疵謬,詩失粘,抬寫錯誤,不避御名、廟諱、至聖諱,罰停會試、殿試一科或一科以上。文理不通,或文理筆跡不符中卷者黜。乾隆五十八年,中式舉人鄧棻春等八名補覆試,停科者五,斥革者二,監臨俱獲譴。歷科因是黜罰者有之。洎末造益趨寬大,光緒十九年,北闈倩作、頂替中式者至數十人,言官劾舉人周學熙、湯寶霖、蔡學淵、陳步鑾、黃樹聲、萬航六人,下所司舉出錄科中卷不符者,學淵、樹聲、航三人俱斥革,餘覆試無一黜者,監臨各官均免議,而僥幸者接跡矣。

定例各省鄉試揭曉後,依程限解卷至部磨勘,遲延者罪之。蓋防考官闈後修改試卷避吏議也。磨勘首嚴弊幸,次檢瑕疵。字句偶疵者貸之。字句可疑,文體不正,舉人除名。若干卷以上,考官及同考革職或逮問。不及若干卷,奪俸或降調。其校閱草率,雷同濫惡,雜然並登,及試卷不諳禁例,字句疵蒙謬纇,題字錯落,真草不全,謄錄錯誤,內、外簾官、舉子議罰有差。禁令之密,前所未有也。磨勘官初禮部及禮科主之,康熙間,始欽派大臣專司其事。解額漸廣,試卷日多,於是令九卿公同磨勘。六部官牽於職事,以其餘暇勘校,往往虛應故事。乾隆初,改任都察院科、道五品以上,科甲京堂、中、贊以上翰、詹官,集朝房磨勘。嗣復增編、檢。額定四十人,以專責成。先是磨勘試卷不署名,亦無功過之條。與斯役者,每託名寬厚,不欲窮究。乾隆二十一年,始令磨勘官填註銜名。二十五年,復增大臣覆勘例,分別議敘、議處,功令始嚴。是年特派秦蕙田、觀保、錢汝誠為覆勘大臣。事竟,原勘官御史硃丕烈劾其瞻徇,下軍機大臣覈覆。蕙田等實有誤駁及疏漏之處,丕烈亦以彈劾不實,俱下部議。其時磨勘諸臣慎重將事,不稍假借,一變因循敷衍之習。太僕寺卿宮煥文、御史閻循琦、硃稽、硃丕烈,嘉慶初御史辛從益,俱以抉摘精審聞於時。

歷科考官舉子因是譴黜者不乏人,而藉端報復,蓋亦有之。乾隆六十年乙卯,會元為浙江王以鋙,第二名即其弟以銜,帝心異之。正總裁侍郎竇光鼐素與和珅不協,且以詆訶後進忤同列,均欲藉以傾之。因摘兩人闈墨中並有「王道本乎人情」語,以為關節。抑寘以鋙榜末,停其殿試,降光鼐四品休致,鐫副總裁侍郎劉躍雲、祭酒瑚圖禮四級。及廷試傳唱,以銜第一,上意釋然。諭廷臣曰:「此亦豈朕之關節耶?」以鋙後亦入詞館。嘉慶五年,磨勘官辛從益、戴璐於北闈策題、試卷指摘不遺餘力。從益江西籍,向以嚴於磨勘稱。是科江西僅中一人,璐子下第,人謂因是多所吹求。上聞,命二人退出磨勘班。同治間,鴻臚寺少卿梁僧寶復以磨勘過嚴為人所憚。蓋自磨勘例行,足以糾正文體,抉剔弊竇,裨益科目,非淺鮮也。

庶吉士之選無定額。順治三年,世祖始策貢士於廷,賜一甲三人傅以漸等及第,簡梁清寬等四十六人為庶吉士。四年、六年復選用。九年,以給事中高辛允言,按直省大小選庶吉士。直隸、江南、浙江各五人,江西、福建、湖廣、山東、河南各四人,山西、陜西各二人,廣東一人,漢軍四人。另榜授滿洲、蒙古修撰、編修、庶吉士九人。自是考選如例。惟滿、蒙、漢軍選否無常。康熙間,新進士得奏請讀書中秘。輒以家世多任館閣,或邊隅素少詞臣為言。間邀俞允。故自四十五年至六十七年科中,各省皆有館選。世宗令大臣舉所知參用,廷對後,親試文藝。雍正元、二年間,漢軍、蒙古、山西、河南、陜西、湖南及諸邊省每不入選。三年,太常寺少卿李鍾峨疏請分省簡選,廣儲材之路。廷議駁之。五年,詔內閣會議簡選庶常之法,尋議照雍正癸卯科例,殿試後,集諸進士保和殿考試,仍令九卿確行保舉。考試用論、詔、奏議、詩四題。是為朝考之始。乾隆元年,御史程盛修言:「翰林地居清要,欲得通材,務端始進。自保舉例行,而呈身識面,廣開請託之門;額手彈冠,最便空疏之輩。宜亟停止。」報可。高宗諭禁向來新進士請託奔競、呈送四六頌聯之陋習,既慎校文藝,復令大臣察其儀止、年歲,分為三等,欽加簡選。三年,罷大臣揀選例,依省分甲第引見,臨時甄別錄用。後世踵行其制。嘉慶以來,每科庶常率倍舊額,各省無不入選者矣。

凡用庶吉士曰館選。初制,分習清、漢書,隸內院,以學士或侍讀教習之。自康熙九年專設翰林院,歷科皆以掌院學士領其事,內閣學士間亦參用。三十三年,命選講、讀以下官資深學優者數人,分司訓課,曰小教習。六十年,以禮部尚書陳元龍領教習事。厥後尚書、侍郎、閣學之不兼掌院事者,並得為教習大臣,滿、漢各一。雍正十一年,特設教習館,頒內府經、史、詩、文,戶部月給廩餼,工部供張什物,俾庶吉士肄業其中,尤為優異。三年考試散館,優者留翰林為編修、檢討,次者改給事中、御史、主事、中書、推官、知縣、教職。其例先後不一,間有未散館而授職編、檢者。或供奉內廷,或宣諭外省,或校書議敘,或召試詞科,皆得免其考試。凡留館者,遷調異他官。有清一代宰輔多由此選,其餘列卿尹膺疆寄者,不可勝數。士子咸以預選為榮,而鼎甲尤所企望。康熙間,庶吉士張逸少散館改知縣,遷秦州知州,其父大學士玉書奏乞內用,復得授編修。三十年辛未,上以鼎甲久無北人,親擢黃叔琳一甲三名。叔琳,大興人。雍正間,大學士張廷玉子若靄,廷對列一甲第三,廷玉執不可,上為抑寘二甲第一,誠重之也。

先是,順治九年,選庶常四十人,擇年青貌秀者二十人習清書,嗣每科派習十數人不等,散館試之。乾隆十三年,修撰錢維城考列清書三等,命再試漢書,始留館。其專精國書者,漢文或日就荒落。十六年,高宗以清書應用殊少,而邊省館選無多,命雲南、貴州、四川、廣東、廣西等省庶吉士不必派習清書,他省視人數酌派年力少壯者一二員或二三員,但循舉舊章,備國朝典制已足。其因告假、丁憂、年齒已長者,例準改習漢書。於是習者日少。道光間例停。穆宗初元,令以治經、治史、治事及濂、洛、關、閩諸儒之書課諸庶常。光緒季年,設進士館,課鼎甲庶吉士及閣部官以法政諸科學,或貲遣游學異國。業成而試,優者授職獎擢。俱未久即罷。

達官世族子弟,初制一體應試,而中式獨多。其以交通關節敗者,順治十四年,少詹事方拱乾子章鉞應江南試,以與正主考方猶聯族獲中,事覺遣戍。康熙二十三年,都御史徐元文子樹聲、侍講學士徐乾學子樹屏同中順天試,上以是科南皿悉中江、浙籍,命嚴勘。斥革五人,樹聲、樹屏俱黜。三十九年,帝以搢紳之家多占中額,有妨寒畯進身之路。殿試時,諭讀卷諸臣,是科大臣子弟置三甲,以裁抑之。尋詔定官、民分卷之法,鄉試滿、合字號二十卷中一,直省視舉額十分中一,副榜如之。會試除云南、貴州、四川、廣西四省外,編官卷二十人中一。未幾罷會試官卷。乾隆十五年,廷臣有以官生過優為言者,部議仍舊,詔責其回護,並及吏、禮二部司官編官卷之不當,令再議。始議中額二十五中官卷一,吏、禮部司員及內閣侍讀子弟停編官卷。明年再議,以京官文四品、外官文三品、武二品以上及翰、詹、科、道等官為限。並減中額,順天十四,浙江六,餘省五至一名。二十三年,大學士蔣溥、學士莊存與復以為言。令官生大省二十卷中一,中省十五卷,小省十卷中一,滿、蒙、漢軍如小省例,南、北皿如中省例,中皿額中一名,不足一名入民卷。永以為例。鄉、會試考官、房考、監臨、知貢舉、監試、提調之子孫及宗族,例應回避。雍、乾間,或另試,或題由欽命,另簡大臣校閱。乾隆九年停其例,並受卷、彌封、謄錄、對讀等官子弟、戚族亦一體回避矣。

有清重科目,不容幸獲。惟恩遇大臣,嘉惠儒臣耆年,邊方士子,不惜逾格。歷代優禮予告或在職大臣,與夫獎敘飾終之典,賜其子孫舉人、進士,有成例者無論已。至如雍正七年,廷臣遵旨舉出入闈未中式之大學士蔣廷錫子溥、尚書嵇曾筠子璜等十二人,俱賜舉人。侍郎劉聲芳子俊邦以疾未與試,賜舉人,尤為特典。康熙間,浙江舉人查慎行,江蘇舉人錢名世、監生何焯,安徽監生汪水顥,以能文受上知。召試南書房,賜焯、灝舉人。四十二年,賜焯、灝、蔣廷錫進士。六十年,以內廷行走舉人王蘭生、留保學問素優,禮闈不第,俱賜進士。雍正八年,賜江南舉人顧天成、廣東舉人盧伯蕃殿試。乾隆十八年,賜內廷行走監生徐揚、楊瑞蓮舉人。四十三年,助教吳省蘭、助教銜張羲年以校四庫書賜殿試,俱非常例。乾隆以來,凡年七十以上會試落第者,予司業、編、檢、學正等銜。鄉試年老諸生,賜舉人副榜。雍正十一年,詔於雲、貴、廣東西、四川、福建會試落卷,擇文理可觀、人材可用者,拔取時餘等十人,一體殿試,趙繩其等四十人,揀選錄用。乾隆初,揀選如例,則邊省士子猶沐殊恩也。

歷科情形略異者,順治三年,從大學士剛林請,以天下初定,廣收人才,再舉鄉、會試。十六年,以雲、貴新附,綏輯需人,再舉禮部試,均不循子丑之舊。康熙十六年,鄉試順天專遣官,山東、山西、陜西並河南省,湖廣、江西並江南省,福建並浙江省考試。試期九月,十五人中一,不取副榜,亦無會試。江南榜江西無中式者。咸、同間軍興,各直省或數科不試。或數科並試,倍額取中。或一省止試數府、州、縣,減額取中。試期或遲至十月、十一月,不拘成例。順天正主考,初制均差翰林官。康熙初,沿明制,以前一科一甲一名為之。士子希詭遇者,得預通聲氣。二十年,修撰歸允肅主順天闈,撰文自誓力除積弊,不通關節,榜後下第者譁然,冀興大獄。刑部尚書魏象樞暴其事,浮議始息。制亦尋廢。二年,順天春秋題「邾子」訛「邾人」,罷考官白乃貞等職。士子因書子字貼出者,弘文院官覆試,優者準作舉人,無中式者。雍正元年,順天榜後,命大學士王頊齡等同南書房翰林檢閱落卷,中二人。是年會試覆檢如前,中落卷七十八人。二年,中七十七人。乾隆元年,中三十八人。後不復行。雍正四年,以浙人查嗣庭、汪景祺著書悖逆,既按治,因停浙江鄉、會試。未幾,以李衛等請,弛其禁。七年,廣東連州知州硃振基私祀呂留良,生員陳錫首告,上嘉之。令是科連州應試完場舉子,由學政遴取優通者四人賞舉人。乾隆四十六年辛丑會試,江南解元錢棨領是科會、狀。嘉慶二十五年庚辰會試,廣西解元陳繼昌亦領是科會、狀,士子艷稱「三元」。有清一代,二人而已。八旗與漢人一體考試,康、乾以來,無用鼎甲者。同治四年,蒙古崇綺以一甲一名及第,光緒九年,宗室壽耆以一甲二名及第,漢軍鼎甲尤多。至歷代捐輸軍饟、賑款、園庭工程賞舉人,拏獲叛匪及殺賊立功,有貢監給舉人、舉人給進士之例,則又一時權宜之制也。

初,太宗於蒙古文字外,制為清書。天聰八年,命禮部試士,取中剛林等二人,習蒙古書者俄博特等三人,俱賜舉人。嗣再試之。順治八年,舉行八旗鄉試,不能漢文者試清文一篇,再舉而罷。康熙初,復行繙譯鄉試,自滿、漢合試制舉文,罷繙譯科。雍正元年,詔八旗滿洲於考試漢字生員、舉人、進士外,另試繙譯。廷議三場並試,滿、漢正、副考官各二,滿同考官四。詔鄉試止試一場,或章奏一道,或四書、五經量出一題,省漢考官,增謄錄,餘如文場例。嗣後繙譯諭旨,或於性理精義及小學,限三百字命題。乾隆三年,令於糸番譯題外作清文一篇。七年,定會試首場試清字四書文,孝經、性理論各一篇。二場試繙譯。凡滿洲、漢軍滿、漢字貢、監生員、筆帖式,皆與鄉試。文舉人及武職能繙譯者,準與會試。先試騎射如例。蒙古繙譯科,雍正九年,詔試蒙古主考官一,同考倍之。初令鄉、會試題,俱以蒙字譯清字四書、章奏各一道。乾隆元年,改譯清文性理小學,與滿洲繙譯同場試,別為一榜。時應清文鄉試者,率五六百人額中三十三名,應蒙文鄉試者,率五六十人額中六名。原定繙譯鄉、會試三年一次,然會試訖未舉行。乾隆四年,以鄉試已歷六科,八月始行會試。中滿洲二十名,蒙古二名。因人數無多,詔免殿試,俱賜進士出身,優者用六部主事。二十二年,以繙譯科大率尋章摘句,無關繙譯本義,詔停。四十三年,復行鄉試,罷謄錄對讀。明年會試,向例須滿六十人,是科僅四十七人,特準會議,免廷試,如四年例。自是每屆三年,試否請旨定奪。五十二年,更定鄉、會試五年一次,然會闈自五十三年訖嘉慶八年,僅一行之,猶不足定例六十名之數。且槍冒頂替,弊端不可究詰。蒙文嘗以不足七八人停試。雖詔旨諄諄勉以國語騎射為旗人根本,而應試者終屬寥寥。八年,從侍郎賡音請,復舊制三年一舉以為常。二十四年,定鄉、會覆試如文闈例。道光八年,罷繙譯同考官,末年始有用庶吉士者。各省八旗駐防,初但應漢文鄉、會試,道光二十三年,改試繙譯,十人中一,三名為額。宗室應繙譯試,自乾隆時始。別為一題,中額欽定。

武科,自世祖初元下詔舉行,子午卯酉年鄉試,辰戌丑未年會試,如文科制。鄉試以十月,直隸、奉天於順天府,各省於布政司,中式者曰武舉人。次年九月會試於京師,中式者曰武進士。凡鄉、會試俱分試內、外三場。首場馬射,二場步射、技勇,為外場。三場策二問、論一篇,為內場。外場考官,順天及會闈以內大臣、大學士、都統四人為之。內場考官,順天以翰林官二人,會闈以閣部、都察院、翰、詹堂官二人為之。同考官順天以科甲出身京員四人,會闈以科甲出身閣、科、部員四人為之。會試知武舉,兵部侍郎為之。各直省以總督、巡撫為監臨、主考官,科甲出身同知、知縣四人為同考官。外場佐以提、鎮大員。其餘提調、監射、監試、受卷、彌封、監門、巡綽、搜檢、供給俱有定員,大率視文闈減殺。殿試簡朝臣四人為讀卷官,欽閱騎射技勇,乃試策文。臨軒傳唱狀元、榜眼、探花之名,一如文科。

初制,一甲進士或授副將、參將、游擊、都司,二、三甲進士授守備、署守備。其後一甲一名授一等侍衛,二、三名授二等侍衛。二、三甲進士授三等及藍翎侍衛,營、衛守備有差。凡各省武生、綠營兵丁皆得應鄉試,武舉及現任營千、把總,門、衛、所千總,年滿千總,通曉文義者,皆得應會試。惟年逾六十者,不許應試。其後武職會試,以武舉出身者為限。康熙間,欲收文武兼備之材,嘗許文生員應武鄉試,文舉人應武會試,頗滋場屋之弊。乾隆七年,以御史陳大玠言,停文武互試例。

考試初制,首場馬箭射氈球,二場步箭射布侯,均發九矢。馬射中二,步射中三為合式,再開弓、舞刀、掇石試技勇。順治十七年,停試技勇,康熙十三年復之。更定馬射樹的距三十五步,中三矢為合式,不合式不得試二場。步射距八十步,中二矢為合式。再試以八力、十力、十二力之弓,八十斤、百斤、百二十斤之刀,二百斤、二百五十斤、三百斤之石。弓開滿,刀舞花,掇石去地尺,三項能一、二者為合式,不合式不得試三場。合式者印記於頰,嗣改印小臂,以杜頂冒。三十二年,步射改樹的距五十步中二矢為合式。乾隆間,復改三十步射六矢中二為合式。馬射增地球,而弓、刀、石三項技勇,必有一項系頭號、二號者,方準合式,遂為永制。

內場論題,向用武經七書。聖祖以其文義駁雜,詔增論語、孟子。於是改論題二,首題用論語、孟子,次題用孫子、吳子、司馬法。

鄉試中額,康熙二十六年制定,略視各省文闈之半。雍正間小有增減,惟陜、甘以人材壯健,弓馬嫻熟,自康熙訖乾隆,先後各增中額三十名。咸、同間,各省輸餉廣額如文闈例。綜計順天中額百十,漢軍四十,奉、錦三,江南八十一,福建六十三,浙江、四川各六十,陜西五十九,河南五十五,江西、廣東、甘肅各五十四,山西五十,山東四十八,雲南四十二,廣西三十六,湖北三十五,湖南三十四,貴州二十五。會試中額多或三百名,少亦百名。康熙間,內場分南、北卷,各中五十名。五十二年,始分省取中,臨期以外場合式人數請旨裁定。

嘉慶六年,仁宗以科目文武並重,文闈條例綦嚴,防弊周密,武闈考官面定去取,尤易滋弊,命比照文闈磨勘例,鄉試題名錄將中式武生馬步射、技勇一一詳註進呈。各省交兵部,順天另簡磨勘官覈對。濫中及浮報者懲不貸。覆試始乾隆時。初制從嚴,僅會闈行之。不符者罰停科,考官議處。三次覆試不合式,除名。道光十五年,始覆試順天武舉如會試例。咸豐七年,覆試各省武舉如順天例,然稍從寬典矣。

初制,外場但有合式一格,其中弓馬優劣,技勇強弱,無所軒輊。內場但憑文取中,致嫺騎射、習場藝者或遭遺棄。康熙五十二年,令會試外場擇馬步射、技勇人材可觀者,編「好」字號,密送內簾。內場試官先於好字卷內,擇文理通曉者取中。不足,始於合式卷內選取。雍正二年,從侍郎史貽直言,各省鄉試外場一體別編好字號,嗣於好字號再分雙好、單好。內場先中雙好,次中單好。而合式卷往往千餘人,僅中數人,因之內場槍冒頂替諸弊並作。乾隆二十四年,御史戈濤奏革其弊,於是外場嚴合式之格,內場罷四書論,文理但取粗通者,而文字漸輕。嘉慶十二年,鄉、會試內場策論改默寫武經百餘字,無錯誤者為合式。罷同考官,遂專重騎射、技勇,內場為虛設矣。歷代踵行,莫之或易。光緒二十四年,內外臣工請變更武科舊制,廢弓、矢、刀、石,試槍砲,未許。二十七年,卒以武科所習硬弓、刀、石、馬步射無與兵事,廢之。

滿洲應武科始雍正元年,鄉試中二十名,會試中四名。十二年,詔停,數十年無復行者。嘉慶十八年,復舊制。滿、蒙鄉試中十三名,各省駐防就該省應試,率十人中一,多者十名,少或一名。會試無定額。凡驍騎校,城門吏,藍翎長,拜唐阿,恩騎尉,親軍前鋒,護軍,領催,馬甲,巡捕營千總、把總及文員中書,七、八品筆帖式,廕生,俱準與武生同應鄉試。鄉、會試內、外場與漢軍、漢人一例考試。


\end{pinyinscope}