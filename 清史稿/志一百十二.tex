\article{志一百十二}

\begin{pinyinscope}
○兵八

△邊防

中國邊防,東則三省,北則蒙邊,西則新、甘、川、藏,南則粵、湘、滇、黔,而沿邊臺卡,亦內外兼顧,蓋邊防與國防並重焉。茲分述之:曰東三省,曰甘肅,曰四川,曰雲南,曰廣東,曰廣西,曰蒙古,附直隸、山西,蒙邊防務,曰新疆,曰西藏,曰苗疆,曰沿邊墩臺、卡倫、鄂博、碉堡。

東三省為陪都重地,曰奉天,曰吉林,曰黑龍江,東連日、韓,北連俄羅斯,邊防尤要。

奉天當康熙元年,廷臣建議,自興京至山海關,東西千餘里,自開原至金州,南北千餘里,有河東西之分:河東自開原至牛莊,河西自山海關歷松杏山、大凌河,為明季邊防之地,戶口寥落,請預籌實邊。嗣後休養生聚,城鎮日繁。凡大城十四,邊門二十餘。至同治間,邊界漸廣。將軍都興阿以鳳、靉二邊門外之地,自靉陽門外八里甸子東至兩江匯口,轉西南至海沿而下,直至貢道北老邊墻,南路經孤頂子等岡,由西南至舊邊小黑山,均展拓為邊界。此外若大東溝江海相連之處,一律查勘,以綏籓服。尋以地方遼闊,增調防軍。其防軍之外,尤以練軍為重。光緒二年,崇實以金州、大東溝等處,旗兵不足,增練步隊分防。十一年,王大臣等會議,奉天界接朝鮮,舊以遼陽迤東鳳凰城等四城為要地。今則水路趨重大連灣、旅順口,陸路自同治間開墾荒地以後,耕廛比櫛,直抵鴨綠江西岸。額設防兵二萬二千餘人,新設練軍及緝捕勇丁一萬三千餘人,而練習新式槍砲者不及半數,宜加練大支勁旅,扼要屯駐。宣統元年,以延吉一帶為交涉要地,令奉省疆吏調遣軍隊,分配憲兵,建築營房。新設之長白,開山通道,駐兵建署。鴨綠江上游之防務,亦次第籌備。蓋自日據朝鮮,與奉、吉接壤,東邊防務,日益亟矣。

吉林凡大城八,邊門四。其防務至重者,一為琿春,與俄羅斯偪壤,兼接朝鮮,曠無障阻。一為三姓,乃松花江之上游,伯都訥腹地之屏蔽。其三岔口,可由蒙古草地達奉天法庫邊門。光緒初年,就未練之兵及八旗臺站西丹內,選精壯者,練馬步四營。七年,吳大澂始於吉林創設機器局,制造軍械,並於扼要處建築砲臺。以陸路轉運維艱,協商直隸督臣李鴻章,派員及熟手工匠至吉林開廠。俟廠局告成,再於寧古塔、琿春等處增築砲臺。十一年,增練馬隊步隊共六營,足四千五百人之數,隸左右翼統率訓練。吉林額設防兵及烏拉牲丁,凡一萬五千餘人,內靖萑苻,外支強敵,時虞不給云。

黑龍江凡大城六,新舊卡倫七十一。中、俄接界,向以尼布楚與恰克圖為重地,故斥堠之設,多在北徼。舊制於歲之五、六月間,齊齊哈爾、墨爾根、黑龍江三處疆吏,各遣協領、佐領等官,率兵分三路,至格爾畢齊、額爾古訥、墨里勒克、楚爾海圖等處巡視,歲終具疏以聞。康熙二十三年,始設將軍以下各官以鎮守之。凡前鋒、領催、馬甲、匠役、養育兵,咸歸統率,額設之兵,一萬三千餘人。光緒元年,以正兵六千人,西丹四千人,合練步隊萬人。時俄騎東略,沿邊自北而東,列戍防秋,遂無寧歲。六年,加練西丹五千人,分布愛琿、呼倫貝爾、布特哈、墨爾根、呼蘭、齊齊哈爾等處。原有馬隊二千人,加練千人,秋冬之際,招集打牲人等,加以訓練,以佐兵力。八年,籌備黑龍江邊防,在奉天調教習,在天津運砲械,共練馬隊五千人,分駐各城。裁舊設卡倫二十六處,以新練之隊伍巡防。十一年,命奉天、吉林、黑龍江三省疆吏各練勁兵,為東西策應之師,並墾闢荒地,開採礦山,為實邊之計。黑龍江復增練馬步各營。蓋自俄人侵食黑龍江以北,及烏蘇里江、興凱湖以東各地,處處與我連界,邊防日重。及俄築東清鐵道,日占南滿,於是防不勝防云。

甘肅北達蒙部,南雜番、回,西接新疆、寧夏,以河套為屏籓,西寧與撒喇相錯處,為西陲奧區。

康熙三十三年,增戍兵五百於大馬營,控扼雪山要路,增馬步兵三千人於定羌廟,以守硤口,咸隸於肅州總兵官。

雍正二年,青海蕩平,於西寧之北,川邊之外,自巴爾託海至扁都口一帶,創築邊墻城堡,於青海、巴爾虎、鹽池等處,設副將以下各官,於大通河南北,設參將以下各官。以西陲重要,全省馬步戰守兵凡五萬七千餘人,關外換防兵凡九千餘人,兵額獨多於他省。三年,以布隆吉爾為安西鎮,設總兵等官,額兵五千人。因莊浪西之仙米寺地方,山深林密,設守備等官,移涼州高古城額兵駐守。五年,於大通鎮設馬步兵二千人,以白塔川、側爾吐二處逼近邊境,各設兵八百人。以插漢地方遼闊,設寶豐、新渠二縣,設文武各官,並增戍兵,控制賀蘭山一帶。八年,岳鍾琪於吐魯番通伊犁之要路,嚴設卡倫,巴爾庫庫等處,多駐防兵,闊舍圖地方,為南北二山鎖合處,屯駐重兵,分防南北山口。十一年,因西路之布隆吉爾,北連哈密,西接沙州,為關外重地,乃建築城垣,屯兵防守。

乾隆四十九年,福康安、阿桂籌備邊防,自蘭州迤東至涇州一千餘里,北達邊城,外則番族環居,內則回民錯處,墩戍寥落,乃擇要增設營戍,凡將弁二十三人,兵丁二千人。嗣又增兵三千人,北路靖遠,南路秦、階,大河東西各處,互為捍衛。

道光二年,以察罕諾門汗投誠,其所轄二十族,分為左右二翼,視蒙古例,每翼統以專員,嚴稽關卡,以孤河北野番之勢。三年,因青海蒙古向未有受事盟長,乃就青海二十旗內,設正副盟長各一人,隨同官兵習武,以防番眾渡河。十一年,楊遇春於察罕託洛地方,增設蒙古兵,分作二班,布守各卡,以佐官兵。二十三年,富尼揚阿於將軍臺、會亭子二處,各建城垣,防禦西番。二十六年,布彥泰以番賊擾邊,規復防河舊制,增兵千人,分布沿河渡口。又哈喇庫圖爾營所屬之南山根,及南川營所屬之青石坡二處,為野番出入總路,各以汛兵駐守。永安營、紅崖營、永昌協所屬之扁都口、石灰關各要口三十八處,均撥兵巡守,自數十人至百餘人不等。沿邊小口,各備坑塹,以遏賊騎。時番賊恃其槍馬便利,頻年竄擾,亦斯門沁地方,為番騎來往要區,募獵戶千人編為一軍,供遠探近防之用。旋以亦斯門沁設兵,僅可防甘、涼二州之扁都口等隘二十七處,兵力尚嫌不足,復於沙金城設兵千人,以防涼州所屬之一顆樹等三十處隘口,於野牛溝設兵千人,以防甘州所屬之大磁窯等十八處隘口。提鎮大員,復督率沿邊將弁,先事預防。

自粵寇披猖,回匪乘之,玉關、雪嶺間,騷然不靖。咸豐元年,以番賊復出,令琦善等撥兵設卡,嚴密巡防。二年,令舒興阿等督率邊卡文武,修濠壘,增馬探,各營定期會哨,分途堵截。四年,因西寧一帶,番族窺伺,增募獵戶三千人,分防隘口。八年,以青海迤西戈壁,給番民暫居,令西寧總兵、道員,定立界址。九年,令甘肅省疆臣督辦團練事宜。

同治十年,豫師等於甘、涼各處邊隘,自平番至威遠各口,及巴燕戎格山後與西寧番地通連者,一律加意嚴防。張曜因甘肅之金塔一帶,邊墻損壞,平番之裴家營,古浪之大靖、土門,甘、涼之南山各口,時有土番竄擾,分遣員弁偵探防堵。十一年,左宗棠於河州迤西之西南北三面,毗連番界,及槐樹關、老鴉關、土門關三隘口,與抱罕羌人接境之處,以歸化之番眾僧俗四千人,馬四千餘匹,防守各關。是時,平定關、隴,皆客軍之力,數萬額兵,幾同虛設。左宗棠懲前毖後,乃減兵加餉,繕器械,簡軍實,以重邊防。惟新設之靈武、化平、硝海三營兵數無多,逼近蒙、番之永昌、莊浪、松山三營,仍循舊額云。

四川西連衛、藏,北接青海,南盡蠻夷。自雍正、乾隆間,青海、大小金川次第綏定,沿邊之防,以打箭爐為尤重。

康熙三十九年,移化林營於打箭爐,以防爐番。

雍正元年,年羹堯於川、陜各處邊隘,擇要增兵。一為中渡河口,乃通西藏要路,修築土城,以守備移駐。一為保縣,在大河之南,乃土番出沒之所,一為越巂,地多蠻惈,一為松潘外之阿樹,及黃勝岡、察木多,均撥兵駐守,設游擊、千總等官。二年,青海蕩平,於邊外單葛耳斯地方,設參將等官。暗門、拉科、恆鈴子三處,設守備等官。河州保安堡,設游擊等官。打箭爐外之木雅吉達、鴉龍江中渡、里塘、巴塘、鄂洛五處,設總兵、副將等官,率兵駐守。六年,岳鍾琪因河東西苗民改土歸流以後,建昌遂為沿邊重鎮,乃於柏香坪、冕山、寧番、寧越、鹽井、波沙、托木、熱水等處,增設將備營汛,合原有之弁兵,咸隸於建昌鎮標。十年,黃廷桂以建昌之竹核,及涼山西南之魚紅地方,當諸蠻出入門戶,穀堆、格落二處,大赤口、小河壩、勒必鐵、阿都四處,皆系邊要,乃於竹核設將備兵丁共三千人,阿都設兵千人。

乾隆十七年,岳鍾琪因番眾投誠,以威茂副將移駐雜穀腦,設兵千二百人,西南境與梭卓接壤之處,均設汛駐兵。四十一年,金川平定,於雅州建城,命提督移駐,增兵六千五百人,分守沿邊。四十四年,設懋功、綏靖、崇化、撫邊、慶寧五營,制同內地,隸松潘鎮總兵,以控番徼。四十五年,特成額因川邊外之察木多,曾設游擊等弁兵,控制西藏。今藏事敉平,乃抽撥營兵,移防江卡,增築碉房,並於三暗巴一帶,設守備等官。

道光十三年,以副將率兵二千人,駐大樹堡,濬濠建碉,兼防河道。以松潘屯千九百人,歸並峨邊。十九年,因川邊防兵僅四千餘人,不敷防守,於馬邊、雷波、越■、峨邊、屏山各縣增兵二千人,增練兵千六百人,改營制,修碉堡,★飭鎮道各員,於秋冬分巡邊界。尋以馬邊等,夷匪不靖,命大臣齊慎親往籌防。

同治十二年,因峨邊蠻族投誠,擇充千、百戶等職,編制夷兵,建修碉堡。

光緒二十三年,鹿傳霖以三瞻地接里塘,為入藏北界,擬設定瞻直隸,而移建昌道於打箭爐,仿金川五屯之制,設立屯官及將弁汛兵,並接展電線至前藏。其事議而未行。三十三年,部臣議裁★邊防軍,趙爾豐以川邊原有巡防五營,已屬不敷調遣,遂定議緩裁。

宣統初年,趙爾巽以打箭爐外所有改土歸流屬地,擬悉隸於邊務大臣,★增設官吏,寬籌經費,協濟兵食,以固邊圉。三年,趙爾豐收回三瞻,土司之梗化者,遂自請歸流雲。

雲南沿邊,環接外夷,南境之蒙自,當越南國,西南境之騰越,當緬甸國,尤為南維鎖鑰。騰越界連野番,舊設八關九隘,以土練駐防。緬甸國入貢之道,向由虎踞關入,經孟卯、隴川等處,以達南甸,設南營都司以備之。自外海輪舶南來,直抵新街,商賈咸趨北道,由騰城西南行,經南甸、千崖、盞達三宣撫司,歷四程而至蠻允,過此即野人境。其間有三路。下為河邊路,中為石梯路,上為炎山路。下路較近,上路則柴草咸便,行四日至蠻暮,入緬甸界。舟行一日,可達新街。又印度東境之野山,系珞瑜番族,英吉利人由印度侵入,闢地種茶桑,其地可通孟養而達騰越,邊外強鄰野俗,錯處可虞。明代舊置銅壁、巨石、萬仞諸關,以固邊圉。水道則海珀江自千崖以下,水勢漸寬,與大金沙江合流,元代征緬甸,以舟師制勝,取建瓴之勢也。其永昌、順寧、大理三府,及蒙化一,楚雄府之姚州,皆迤西邊界,山深箐密,漢、夷雜處。清初原設永順鎮總兵,迨改鎮為協,僅於永昌城駐兵,沿邊一帶,有鞭長莫及之虞。

雍正二年,青海平定,於鴉龍江各處,設副將等官,宗鄱地方,當云南孔道,設參將等官,以靖邊服。三年,因威遠大山為苗、惈盤踞之所,乃於普茶山各處,設參將等官,兵丁千二百人,並於九龍江口設立防汛。四年,以四川阿墩子地方當中甸門戶,移其防汛歸雲南省管轄,與里塘、打箭爐相為犄角。五年,以中甸延袤千里,為滇省西南籓籬,維西乃通西藏要隘,增設參將營於中甸,守備營於維西。六年,因烏蒙、鎮雄二處地方遼闊,於烏蒙設總兵等官,鎮雄設參將等官,分隘駐防。所有舊設之貴州威寧營,雲南鎮雄營、東川營咸隸烏蒙鎮總兵調遣,建築城垣。旋增兵千五百人,設尋甸州參將等官。七年,設普洱府及普洱鎮將,標兵三千二百人,分防各路。

乾隆三十二年,以木邦為通緬甸要路,並九龍江、隴川、黑山門各隘,咸以兵駐守。四十三年,李侍堯因永昌、普洱等府,向以鎮、協標千五百人,在三臺山、龍江一帶駐營防緬,冬去春回,頗形煩累。雲南省控制全邊,重在騰越。其南甸之東南為杉木籠,距虎踞關百餘里,當騰越左臂。南甸之西南為千崖,距銅壁、萬仞、神護、巨石諸關,均一二百里,實為各路咽喉。乃於杉木籠、千崖二處各增將弁營汛。龍陵地方,道通木邦,原駐兵千五百人,其南三臺山尤為扼要,亦增設弁兵。以順寧一路舊有之額兵,分駐緬寧,與永順右營協同防守。總督、提、鎮大員,每年酌赴騰越邊外巡閱一周,以期嚴密。

嘉慶十七年,以雲南邊外野夷惈匪肆擾,而緬寧、騰越各隘,皆瘴癘之地,難駐官兵,復設土練兵一千六百人,以八百人駐守緬寧之丙野山梁等處,八百人駐守騰越之蠻章山等處,省官兵徵調之勞。

道光間,林則徐於迤西移改協、營,增設弁兵。其扼要之處,為永平縣、永昌府龍街汛、永定汛、漾濞汛、姚關汛等,凡二十一汛,咸增兵駐防,而瀾滄江橋尤為扼險。順寧府毗連夷地,以龍陵協與順寧參將對調。緬寧、錫臘、右甸、阿魯、史塘等處防軍,或分汛多而存城少,或分汛少而存城多,地之夷險與兵之多少不均,咸酌量增調。大理府原駐提督,而上下二關,及太和縣城、彌渡、紅巖、趙州等處,尚屬空虛,均增兵填防。姚州、蒙化二處,亦改汛增兵。

同治間,雲南平定,岑毓英因迤西麗江府城地處極邊,界連西藏,麗江、劍川交界之喇雞鳴地方,系江邊要隘,江外即野人境,向未設兵。乃以麗鶴鎮都司移駐麗江府,劍川營都司移駐喇雞鳴。此外楚雄府屬之八哨地方三、四百里間,惈夷雜居,亦系要地,令楚雄協副將設汛駐兵。十三年,以昭通標兵之半,赴金沙江外駐守。

光緒七年,劉長佑因劍川城地當孔道,為迤西要區,以喇井營移駐劍川。喇井亦瀾滄江要地,以吉尾汛移駐,互相會哨。十一年,岑毓英因滇省入越南之路,以白馬關為要,法蘭西人通商之路,以蒙自縣為沖,沿邊千里,處處錯壤,留防之兵一萬六千人,編為三十營,以白馬關隸開化鎮總兵,蒙自隸臨元鎮總兵,每年瘴消之際,親赴邊陲,簡閱營伍。個舊錫廠,規制宏大,廠丁數萬人,漢、夷雜處,且通三猛、蠻耗各路,乃增設同知一員,移臨元之都司營兵駐防個舊,調原駐開化游擊移守白馬關,以右營都司分防古林,移右營守備駐長岡嶺,以臨元游擊駐蒙自縣,右營都司分防水田,右營守備分防嵩田,為因地制宜之計。自雲南入緬甸,共有六途,以蠻允一途為捷徑,沿邊由西而南而東,皆野人山寨,布列於九隘之外,兵團守望,時虞不足。乃調關外勁旅二千餘人,與原有防軍及鄉團、土司,協力警備。十四年,岑毓英以邊境惈黑夷匪,頻年滋事,分別剿撫。惈黑所屯踞之地,分上下改心,在瀾滄江畔,界接土司,其東西大路,與緬甸逼處,為順寧、普洱兩府屏蔽,其下改心地方,尤為扼要。乃增設鎮邊撫夷,擇地建築城垣,並設參將等官,駐防兵丁一千一百五十八人。二十二年,鹿傳霖以維西協所屬阿墩子汛地,界接川邊之巴塘,左臨瀾滄江,右挹金沙江,地勢至要,英緬鐵路所經,相距漸近,僅四、五日程。乃協商四川疆臣,酌設重鎮,並於川、滇交界處,兩省各設文武員弁,協力防邊。雲南自英據緬甸,法奪越南,防守兩難。光緒之季,西南騰越、臨安兩路,創設團練,稍資捍衛。而餉絀兵單,邊防漸弛矣。

廣東邊防,海重於陸。同治十三年,越南不靖,督臣瑞麟慮其越界,以防勇二千人扼守欽州。光緒八年,曾國荃因欽州之東興街,地接越南,撥勁兵二營駐守,續撥老勇三營助之。十年,法蘭西侵奪越南,彭玉麟督辦粵防,以欽州與廉州並重,增調營勇赴欽、廉,恐地廣兵單,以團練協守。至省內防務,則有三江口之排瑤,瓊、崖之黎匪,時或出巢滋事。排瑤山境四百餘里,康熙四十一年,於瑤境適中之三江口,設立寨城,置副將等官,兵丁千餘人。道光十二年,增三江口戍兵二千人,建築碉臺,以控制悍瑤。光緒十三年,張之洞剿平瓊州黎匪,山路開通,收撫黎眾十萬人,定撫黎章程十二條。粵省負山帶海,西來歐舶,首及粵洋,陸路僅欽、廉一路當敵,防戍較易於海疆也。

廣西南邊,綿亙千餘里,原設隘所百有九處,分卡六十六處,與越南之諒山、高平、宣光等處接壤。叢山密箐中,小徑咸通。鎮南關至龍州一路,地較寬平,為中越商旅通衢,東出太平、南寧,西出歸順、鎮安之總匯。自龍州以東,下水直達梧、潯,有建瓴之勢。歷朝南籓向化,自清初至道光、咸豐間,惟於龍憑營所轄水陸各隘口,以戍兵及沿邊土司協力防守。

同治十一年,令馮子材等就戍所之鎮揀選各營,分布各隘,是為防軍守邊之始。迨法、越戰事起,邊氛日亟,徵調頻煩,兵無久駐之地。

光緒十二年,中、法款議既成,兩廣總督張之洞以鎮南一關,鈐轄中外,固屬極沖之地,即鎮南關之中後左右各路,亦須分兵設防。關以內之關南隘及憑祥土州為中路。自關以東,明江轄之由隘,寧明州轄之羅隘,思陵土州轄之愛店隘,上思州轄之百侖隘、剝機隘為東路。自關以西,龍州轄之平西關、水口關,下凍土州轄之布局隘、梗花隘,歸順州轄之頻峒隘、龍邦隘,鎮安轄之猛峒隘、剝淰隘、百懷大隘等為西路。以上各隘,咸增兵屯守。以十二營防鎮南關中路,以四營防東路,六營防西路。其道路寬者,築臺置砲;路窄者,設卡浚濠;甚僻者,則掘斷徑路,禁阻往來。豫造地營。無事則操練,有警則徵調赴援。廣西提督由柳州移駐龍州,以控制邊夷。而邊境過長,貴能扼要。關前隘為諒山來路,羅隘為間道所通,歸順之龍邦隘,鎮安之那坡隘,分扼牧馬、保樂夷寇來路,由隘當文淵之沖,即龍州後路,下凍土州通鎮邊聲息,令駐邊各將領,宜加嚴防。旋督臣張之洞以沿邊之新太協、上思營、鎮安協各營兵,或改勇補兵,或裁兵留勇,各就所宜,即分防之舉,為並省之謀。十三年,復移駐鎮、道各員,以資分任。

二十三年,譚鍾麟因邊境迤長凡千七百里,僅恃營汛,終嫌單薄,乃扼要建築砲臺。原有防軍二十營,以分防見少,每營止能抽撥二棚駐守砲臺。二十六年,蘇元春因南、太、泗、鎮及上思、歸順四府二州,皆為邊地,勇丁數僅萬人,凡三關百隘,沿邊砲臺,皆須防守。乃以新募調赴江南之五營,並抽調邊軍五營,合成十營,為剿辦沿邊游勇土匪之用。三十年,柯逢時令各州縣增募勇丁八千餘人,給以毛瑟後膛槍,以佐防軍,並令各屬勸民間多築碉堡,藉禦外侮。

三十一年,李經羲以廣西沿邊,全恃防軍,近年邊防大軍,專駐龍州訓練,而南、太、鎮等郡,以迄滇邊,無復邊營蹤跡,客軍又撤回過半,乃酌增防營,募土著親兵,就地防禦。蓋廣西制兵,舊額六萬二千餘人,自同治四年以後,屢加裁汰,由制兵而趨重防軍。法、越事起,於邊地防軍,尤為注重。至光緒季年,改練新軍,非復防營規制矣。

蒙古以瀚海為界畫,其部落之大類有四:曰漠南內蒙古,曰漠北外蒙古,曰漠西厄魯特蒙古,曰青海蒙古。清初,漠南蒙古臣服最先。至康熙初年,而漠北喀爾喀三部內款。及親征準噶爾,而青海諸部來庭。惟漠西厄魯特部,至乾隆間始征定焉。漠北外四盟蒙古,康熙間初定,增為五十五旗。雍正間,增三音諾顏部,共前三部為四部。乾隆中,增至八十二旗。其會盟分四路:土謝圖汗為中路,車臣汗為東路,札薩克圖汗為西路,三音諾顏為北路。乾隆間,築城於烏里雅蘇臺及科布多二處以鎮撫之。其統率蒙兵之制,內札薩克之兵,統於盟長。外札薩克之兵,統於定邊左副將軍。杜爾伯特及新土耳扈特、和碩特之兵,統於科布多辦事大臣。土耳扈特之兵統於伊犁將軍。青海各部落之兵,統於西寧辦事大臣。雍正間,西陲未靖,阿爾泰及河套以北,迤西直達巴里坤,平原沙磧,數千里間,無險可扼。乃於四臺至三十五臺,每臺選精兵駐守,互為聲援。於烏里雅蘇臺城外山顛扼要處,復各建砲臺,屯重兵於特斯臺錫里。旋增設卡路八處於鹽口、戈壁二口,遣兵更番巡探,以期嚴密。其時防在西徼,而北鄙無驚。自乾隆間蕩平準部,而衛拉特來歸,內外各盟長,從征回、準,屢奏邊勛,新舊土耳扈特,同膺茅土,北境俄羅斯亦方輯睦,陰山、瀚海間,百有餘年無事矣。

迨咸豐、同治間,中原多故,蒙邊亦多不靖。同治四年,增熱河馬隊三百人。五年,以包頭鎮為綏遠要區,原有防兵,積年疲乏,調吉林馬隊協同駐守。六年,李云麟以三音諾顏蒙兵專防烏城,而招募奇古民勇駐八里岡,與科布多、塔爾巴哈臺二城蒙兵為犄角。八年,以布倫託海各領隊大臣所有旗兵,改隸科布多大臣,分防熱河等處。令烏梁海總管,自卜果蘇克霸至沙賓達巴哈與俄羅斯接界處,新立鄂博界牌八處,嚴密巡察。徙厄魯特僧眾於阿爾泰山,徙俗眾於青格里河。九年,調大同、宣化練軍二千人駐防庫倫,修復推河以北至烏城十五臺站,並牧馬三千匹,熱河增練洋槍隊三百人,以固庫倫西路之防。十年,以濟斯洪果爾臺站為察哈爾及歸化、綏遠運糧要區,撥兵駐守。令蒙古各臺,自張家口至八臺,以察哈爾都統管理。自九臺至科布多,及庫倫、歸化二路,以各盟長管理。每臺增設駝馬百五十匹,凡軍械糧食,接護轉運,以利軍行。十一年,改建烏里雅蘇臺石城,並整頓沿邊臺務。庫倫西接俄疆,向未設兵,乃於圖、車兩盟蒙兵內,輪派四百人,分駐庫倫四境。十二年,調察哈爾馬隊協防烏里雅蘇臺。旋以軍臺四十四站,地勢綿長,分防散漫。乃分為四路,於中二路擇要駐營,調綏遠城馬隊移防哈爾尼敦,以原有之兵守塞爾烏蘇。

至光緒間,新疆大定,西顧無虞,而北境俄患漸偪。光緒六年,調宣化練軍、直隸步隊赴庫倫防俄。七年,因烏城三面鄰俄,邊防重要,而原有防軍,技藝生疏,烏城共駐蒙古練軍及黑龍江、察哈爾馬隊二千五百人,由京營派教習前往教練,俾成勁旅。十八年,李鴻章以熱河東境山谷叢雜,毗連奉天,撥直隸練軍馬步隊各一營擇要駐防。二十四年,以熱河、察哈爾為蒙邊要地,令各都統等選練兵丁,整備軍實。三十二年,以熱河馬步隊三營改編為常備軍,其兵額均次第補足。時內外蒙古兵日益孱弱,俄人遂駸駸闌入,烏梁海以南受其牢籠,喀魯倫河以東恣其墾牧,鄂博、卡倫遂同虛設矣。

直隸沿蒙邊防務,雍正九年,令直隸疆臣修治邊墻,其古北、宣化、大同三處,咸募兵增防。自獨石口以西,至殺虎口一帶要隘,亦酌增弁兵。十年,於獨石口改設副將以下各官,增額兵八百人,邊墻沖要處,增設鹿柵木柵,以備堵御。自清初至乾隆、嘉慶朝,蒙邊綏輯。咸、同之間,西陲用兵,蒙匪亦漸滋事。同治四年,以直隸北境沿邊關口五十餘處,兵數甚單,調撥京師火器營、威遠隊、提標馬步隊,分駐喜峰口、鐵門關、灤陽、灑河橋、遵化、羅文峪迤北迤西等處。光緒七年,李鴻章以北邊多倫地兼蒙旗,僅有新舊防軍七百餘人,不敷分布,增調宣化練軍馬隊一營分段梭巡。十八年,以直隸防軍五營駐古北口。十九年,李鴻章因古北口防營調回內地,而熱河地廣兵單,乃別練馬隊三哨,與原有之朝陽馬隊一營、圍場馬隊百人,互為援應。直隸邊務,重在海疆,東之山海關,為遼、沈門戶,南之天津、大沽,為京師屏蔽。其北境惟緝捕蒙匪,無事重兵屯戍也。

山西邊界之歸化、綏遠、包頭鎮,控扼草地,毗連大青山,南抵殺虎口,西逾纏金,東接得勝口,與蒙古、回部錯壤。咸豐軍興以後,官兵四出征討,邊備空虛,寇盜乘機竊發。同治六年,左宗棠督師秦、晉,以山西省弁兵團勇均不可恃,乃分撥營勇,駐守黃河西南兩岸,別募三千人,赴禹門、保德間防守,並造砲船四十艘,酌配水師,駐垣曲、三門一帶。軍事定後,防軍旋撤。光緒間,曾國荃調撥湘軍,擇要屯守,而兵數僅一千二百人。九年,張之洞以雁門關為晉邊要口,止有練軍千人,令各營以次抽練,以固邊防。十年,增練大同、太原二鎮馬步營。衛榮光增練馬隊五旗,以三旗駐口外,二旗駐口內,以佐湘軍之不逮。由山西省迤西,為陜西之北境,惟榆林、神木一隅,地接蒙疆,而障以長城,環以河套,民情馴樸,防務更簡於燕、晉也。

新疆為西域三十六國故壤,歷代籌邊列戍,近在玉門,遠亦僅龍堆而外。自乾隆年準部平,道光朝回疆定,至光緒間,再定天山,開省治於迪化城,設五府三十六縣。而俄羅斯邊境由北而西,綿延錯亙。自奪取霍罕三部後,伊犁及南路喀什噶爾皆與俄屬相接。全境中界天山,分南北二路。北路為準噶爾部落,西北以伊犁為重鎮,烏魯木齊當往來孔道,塔爾巴哈臺為北境屏籓。南路悉回族所居,烏什當適中之地。葉爾羌、喀什噶爾雄冠諸城。英吉薩爾西達外籓。

乾隆十八年,以準噶爾逼處邊境,哈密及西藏北路雖已設防,而選將備,具駝馬,簡軍實,勘水草,儲糧餉,修城垣,諸端待理。命疆吏先事籌備,次第施行。哈密已駐重兵,而防所全恃卡倫。天山冰雪嚴寒,加意撫循士卒。南路各城,以滿洲營、綠旗營協同防守。和闐、庫車、闢展諸城,則但設綠旗營兵。其卡倫臺站,自哈密西至闢展,北至巴里坤,自闢展西至庫車,北至烏魯木齊,自庫車西至烏什,又西至葉爾羌,又西至喀什噶爾,其南至和闐,視卡倫之大小,定戍兵之多寡。各臺站設駝馬車輛毋缺,前行阻水,則造舟以濟之。二十四年,戡定準部,北路重地,咸分兵設防,山川隘口,悉置卡倫臺站。各卡倫設索倫、錫伯、厄魯特兵丁自十名至三十餘名有差。各臺站設滿洲、綠旗、察哈爾兵丁各十五名。南路各城設辦事大臣。其理事回官阿奇木伯克以下,各有所司,分統回兵,隸駐防大臣調遣。二十六年,設伊犁馬步兵二千五百人。二十七年,設伊犁將軍及參贊領隊大臣。三十一年,設烏魯木齊辦事大臣。

嘉慶二年,於惠遠城之北關,增調戍兵。

道光六年,以新疆防軍已增至萬餘人,令疆吏調兵四千人赴回疆,二千人赴阿克蘇,協力防堵。又因喀什噶爾防兵較少,於城北要隘增兵三營,城南增兵二營。八年,分遣喀什噶爾防兵四千三百人防守各路,選精壯二千人分十班教練。那彥成因阿克蘇為南路要地,增兵千人,合原有防兵凡二千餘人,以控制南北二路。其冰嶺一路,北通伊犁西南卡倫,外通烏什之捷徑,一律封禁。喀什噶爾、葉爾羌、英吉薩爾各卡倫,向僅駐兵十餘名,乃於各卡倫適中處,凡通霍罕、巴達克山、克什米爾外夷之路,增築土堡,以都司等官率兵駐守,兵數自數十人至二百人不等。九年,於喀什噶爾邊界增卡倫八處。十一年,回疆大定,命參贊大臣駐葉爾羌,總理八城回務,節制巴里坤、伊犁兩路滿、漢兵一萬四千餘人,分防各路。喀什噶爾之八卡倫,道通霍罕,築土堡三座,增建兵房。葉爾羌所屬卡倫,通克什米爾外夷要隘,英吉薩爾通布魯特要隘,各修土堡駐兵。於烏克蘇、烏什二處,各駐八旗兵一千三百人。於喀什噶爾駐綠營兵三千人,為前鋒,兼守邊卡。英吉薩爾駐馬隊五百人,綠營兵千人,為前後二城中權接應之師。巴爾楚克綠營兵三千人,築堡屯守。和闐增足防兵五百人。所餘滿、漢兵六千餘人,悉數駐葉爾羌,隸參贊大臣統轄,遇警援剿。其喀什噶爾、葉爾羌舊額回兵,仍挑補訓練,以替防兵。十四年,以索倫、錫伯、察哈爾、額魯特四處營兵,守衛伊犁沿邊大小卡倫七十餘座,按期會哨,統兵將領,不得輕出邀功。

咸豐二年,廷臣會議,以新疆南北路駐兵益多,數逾三萬,頻年由內地換防,殊苦煩費,乃於伊犁等處綠營兵內調撥換班,其不足者,就地募之。

咸、同間,中原用兵,關外南北各城,邊氛四起。同治二年,調察哈爾蒙兵,悉數由科布多赴烏魯木齊屯守。五年,調烏里雅蘇臺蒙兵六千人赴伊犁。九年,調黑龍江兵二千人,察哈爾兵千人,馬隊二百餘人,馳赴烏城,並令喀爾喀各盟長,隨時整頓蒙兵。十年,在烏梁海一帶,安設臺站,迤西亦一律設臺,直抵塔爾巴哈臺。十一年,因庫爾喀喇烏蘇等處,為晶河要地,招募勇丁,協同馬隊防守。調宣化、古北口營兵,分赴烏城。十二年,調大同、宣化兵千人,赴防塔爾巴哈臺。十三年,以塔城為西路防務扼要之區,調伊犁迤北之察哈爾兵二千人,及蒙古兵益之。尋命左宗棠由關、隴西征,天山內外,次第戡平,而俄羅斯亦歸我伊犁。

光緒三年,左宗棠於伊犁增築砲臺,多駐勁旅。劉錦棠就關外營勇之精壯者,編為制兵,改行餉為坐糧,參用屯田之法,以足軍實。張曜更定新疆營制三事:一、增騎兵,佐步兵之不逮;一、重火器,減養兵之費,為購器之資;一、設游擊之師,駐南北路之間,預防俄患。六年,恭鏜因烏魯木齊之鞏寧城,接壤精河,旁達烏城間道,而舊城已圮,乃於迪化城外高原,別建新城,以駐防兵,而資控扼。十二年,劉錦棠以巴里坤滿營歸並古城,伊犁共駐馬步防軍二十八營,酌裁新募之勇,編留精壯,為馬隊九旗,步隊十三旗,自伊犁至大河沿及精河以東,分路駐防。十四年,額爾慶額因塔爾巴哈臺駐防漢隊,久役思歸,就甘肅額兵,及察哈爾部內,選二千六百人調防。十五年,復於塔城增募防兵,凡步隊三營,馬隊四旗,弁勇二千人。十六年,以伊犁滿洲營,經兵亂後,額數久虛,酌量挑補,定為二千人,再挑留錫伯、新滿洲千人,以備不足。伊犁漢隊改立標營,凡步隊一營,馬隊二營,格林砲隊一哨。惠遠城北關設砲隊一哨,定遠城設馬隊三旗。十七年,楊昌濬因塔城境內,漢、蒙、回、哈雜居,東接烏梁海,西接伊犁,地既險要,路復分歧,共增將弁三十一員,步隊三旗,馬隊四旗,以備巡防彈壓。十九年,以總兵官駐防綏定,統漢隊三千人,策應四境,若廣仁城、果子溝、三臺、瞻德城、三道河、霍爾果斯、拱宸城、寧遠城,以馬步砲隊分防。三十一年,潘效蘇因新疆兵費過重,改練土著,遣散客軍。回纏民性各殊,以二三成攙入漢軍訓練,漢軍則各營旗皆減為哨,節餉防邊,始能兼顧。

宣統二年,札拉豐阿因塔爾巴哈臺屏蔽西北,以原有馬步砲隊,及左右旗蒙、滿隊,悉改新式操法。時中朝方議減餉裁兵,未遑遠略。俄羅斯正經營東陲,遂暫安無事云。

西藏初設駐藏大臣,而番眾仍統屬於喇嘛。當崇德七年,達賴、班禪與厄魯特同時入貢。順治、康熙間,朝請不絕。康熙之季,準噶爾侵藏,由西寧進兵平之。

雍正五年,弭噶隆之爭,以頗羅鼐有定亂功,進封郡王。十年,留雲南兵於察木多,以防番眾。

乾隆十五年,除頗羅鼐王爵,始設駐藏大臣,與達賴、班禪參互制之。其西南之廓爾喀,時窺藏境,中朝以兵力佐之,收復巴勒布所侵占藏地,增設塘汛守兵十三處,以寨落之多寡為衡,前藏增唐古特兵八百人,後藏增四百人。五十四年,始於前後藏各設番兵千人。其通內地之定日、江孜二處要隘,各設番兵五百人,就近選補。設戴琫三人,以二人駐後藏,一人駐定日。增江孜戴琫一人。前藏番兵隸駐防游擊,後藏番兵隸駐防都司。令四川督臣以頭等將備為駐藏之選,統以大臣。其駐藏之兵,令駐藏大臣親為校閱。嗣因定日、江孜為各部落來藏必經之路,各增防汛,設守備等官。打箭爐之外,擇地設游擊等官。五十八年,和琳等會勘後藏邊界及鄂博情形,江孜番、漢兵已敷防守,惟定日地方遼闊,為聶拉木、宗喀、絨轄三處總匯之區,其捷徑如轄爾多、古利噶等處,均為要隘,增設番兵,統以戴琫,修寨落以備棲止,立鄂博以守界畫。

道光二年,懲治聶拉木、絨轄各營官私釋喇嘛之罪,別遣番兵補營兵之額。二十一年,令番兵習弓矢者,改習鳥槍。二十二年,令後藏大臣督率將弁教練堆葛爾本挖金番民武技。

咸豐五年,以廓爾喀不靖,駐防兵單,令喇嘛等聯絡防範,調前藏僧俗土兵二千人赴策墊地方防範。

同治四年,駐藏大臣滿慶等,調派土兵及統兵番員防備披楞。八年,因披楞侵占哲孟雄,與唐古特相持,令恩麟等整頓後藏番、漢營伍。十一年,命德泰赴藏,校閱江孜、定日後藏三汛防營,以固哲孟雄及聶拉木門戶。

光緒二十四年,駐藏大臣文海因後藏定日地方營伍及靖西設防,駐藏大臣久未巡視,乃率兵親往各處校閱。光緒季年,駐藏大臣聯豫仿內地制,設武備學堂,擇營弁衛隊及達木三十九族中之優秀者,習速成科,俟畢業後,先練一營,以開風氣。

宣統二年,聯豫因工布平定,以馬步砲隊工程隊分地駐守。旋疏請裁去幫辦大臣,設左右參贊,分駐前後藏。三年,波密野番滋事,即以工布之兵剿辦,並以步隊擇地駐防,為各營後援。

至川軍入藏之舉,始於雍正初年,準噶爾窺邊,詔以川、陜兵二千人駐防,設正副大臣,分駐前後藏。其時雲南省軍隊亦分途入藏。事定,仍撤歸原省。歷朝鎮撫藏地,多用漢軍、番卒。至光緒三十一年,四川督臣錫良奏調川軍出打箭爐,並招募土勇為向導,以剿竄回。是年八月,巴塘喇嘛戕害大臣,全藏震動。四川提督馬維祺、建昌道趙爾豐合兵進克巴塘、里塘,勘平邊亂。三十二年,里塘逆番桑披復率眾倡亂,錫良命趙爾豐等以川軍討平之。其時番僧與北部回民日就衰弱,全藏邊境,為英吉利、俄羅斯遠勢所包,藏事遂不可問云。

苗疆當貴州、湖南之境,叛服靡常,歷朝皆剿撫兼施。康熙三十八年,以鎮筸居苗疆沖要,改沅州鎮為鎮筸鎮,設總兵以下各官,增額兵千人,合原有之兵凡二千一百人,以防紅苗。雍正九年,復增兵二千人。是年,鄂爾泰因都江與清水江形勢劃分,增設清江鎮標,以新設之丹江、臺拱等營,及原有之銅仁、鎮遠等營,咸隸清江鎮。而以都勻、黎平,並上江、下江各協、營,隸於古州鎮總兵。乾隆元年,楊名時銳意治苗,以貴州省苗眾分生熟二苗,生苗在南,熟苗在北,乃屯駐重兵於內地,而擇鄰苗之要道,增修壁壘,使民有所歸,兵有可守,遇苗眾出巢滋事,則互相援剿,戰勝勿事窮追,兼撫熟苗,俾漸知向化。五年,那蘇圖因永順所屬,緊接苗疆,且與湖北省之容美土司、四川省之酉陽土司連界,乃以永順協標兵改隸鎮筸鎮總兵,聯絡楚南聲勢,合力防苗。

嘉慶初年,戡定苗疆以後,於鳳凰、乾州、永綏、古丈坪、保靖各縣,沿邊次第建修屯堡碉臺,築邊墻以嚴界畫,築土堡以資守御,築哨臺以憑了望,碉卡則戰守咸資,砲臺則堵截尤利。設練勇千餘人,屯丁七千人,墾闢屯防田十三萬一千餘畝,悉以屯兵耕種。其地皆附近碉堡,以便駐守,且節餉糈。歷嘉、道兩朝,沿邊寧謐。

咸豐軍興以後,苗眾乘機肆擾。至同治年,席寶田等大舉平苗,雖間有剽掠之事,以防勇隨時剿撫。光緒十二年,譚鈞培因苗民馴擾無常,乃仿傅鼐防苗之法,增修石碉土堡,由附郭而漸及山林險阻之處,互為守望,以備苗民出入,於舊日之苗疆營制,無所變更也。

沿邊墩臺、卡倫、鄂博、碉堡,清初於各省邊境扼要處,設立墩臺營房,有警則守兵舉煙為號。寇至百人者,掛一席,鳴一砲;至三百人者,掛二席,鳴二砲;至五百人者,掛三席,鳴三砲;至千人者,掛五席,鳴五砲;至萬人者,掛七席,連砲傳遞。康熙七年,諭各省將領,凡水陸孔道之旁,均設墩臺營房,駐宿兵丁,傳報緊急軍機,稽察匪類,護衛行人。乾隆三年,兵部議定汛兵缺少處,按地方★僻情形,酌量撥補器械,務令整備,隨時察驗。有離汛誤防者革責,官吏嚴懲之。

其軍臺之制,始於順治四年,自張家口迤西,黃河迤東,設臺三百四十四座,臺軍七百三十二名。自張家口迄山海關迤西,設臺四百十七座,臺車一千二百五十一名。

蒙古各旗臺、卡、鄂博之制,以大漠一望無垠,凡內外札薩克之游牧,各限以界,或以鄂博,或以卡倫。盛京、吉林則以柳條邊為界,依內興安嶺而設。其內蒙古通驛要口凡五道,曰喜峰口、古北口、獨石口、張家口、殺虎口,以達於各旗。內蒙路近,商旅通行,水草無艱。其外蒙古之驛,則由阿爾泰軍臺以達於邊境各卡倫。康熙朝征準噶爾時,設定邊左副將軍,而外蒙古軍臺之設,由內而外,其制始密。自察哈爾而北,而西北,而又西,迄烏里雅蘇臺,共置四十八臺。康熙三十一年,自古北口至烏珠木秦,置臺九。自獨石口至浩齊忒,置臺六。自張家口至四子部落,置臺五。自張家口至歸化城,置臺六。自殺虎口至吳喇忒,置臺九。自歸化城至鄂爾多斯,置臺八。自喜峰口至扎賴特,置臺十六。乾隆三十四年,自喜峰口路扎賴特盡處起,置臺十四。自古北口路烏珠木秦盡處起,置臺六。自殺虎口路吳喇忒大路外起,置臺七。自張家口路四子部落盡處起,置臺十六。喀爾喀則自備郵站。其東路首站曰尼爾得尼拖羅海,西路首站曰哈拉尼敦,後路首站曰肯特山。迤邐而北,直抵三音諾顏境,其首站曰博羅布爾哈蘇。凡汗、王、貝勒過境,警晨夜,飼牲畜。商旅出其途,亦資捍衛焉。

圍場卡倫之制,規取高地為之,或于岡,或於阪,或於山川之隙,隨宜設置。其柳條邊境之設立卡倫者,東為崖口,西為濟爾哈朗圖,北為色堪達巴漢色欽等處,又西為庫爾圖羅海等處,又南為木壘喀喇沁等處,又南而西為珠爾噶岱等處,又南為海拉蘇臺等處,又南而東為巴倫克得依等處。老柳邊在外,卡倫在內。其故地在周阹之中者,為翁牛特,為哈喇沁,為敖漢,為奈曼,為喀爾喀,左翼等故地咸在焉。

其恰克圖及沿邊鄂博、卡倫之制,因山河以表鄂博,無山河則表以卡倫。鄂博者,華言石堆也。其制有二:以壘為鄂博,以山河為鄂博。蒙古二十五部落,察哈爾牧廠八旗各如其境,以鄂博為防。其與俄羅斯接界,中間隙地,蒙古語曰薩布。凡薩布皆立鄂博以申畫之。恰克圖之中、俄邊界,凡俄國卡倫、房屋,在鄂爾懷圖山頂,中國鄂博、卡倫,適中而平分之。如有山河,即橫斷山河為界。由沙畢納依嶺至額爾古訥河岸,向陽為中國,背陰為俄國。蓋沿邊之地,自黑龍江、庫倫、烏里雅蘇臺、科布多四屬迤邐而西,凡八十二卡倫。科布多所屬極西之卡倫,曰和尼邁拉呼。由此渡額爾齊斯河至輝邁拉呼一帶卡倫,均與俄羅斯接壤。

其在黑龍江境內之卡倫,以將軍轄之。在蒙古喀爾喀等部落之卡倫,按其游牧遠近,每卡倫設章京一員,率兵攜眷戍守。遇森林叢雜,難立鄂博、卡倫之處,則削大樹而刊識之。

自同治七年裁撤科布多境內卡倫以後,各項哈薩克人赴界強據。光緒初年,乃於烏克克等處,由沁達蓋圖烏爾魯向西南至馬尼嘎圖勒幹止,與塔爾巴哈臺卡倫相接,一千數百里之要隘,與俄羅斯接壤者,均設卡倫。所有協理臺吉等員,咸復舊制。

其新疆全境之卡倫,分南北二路。北路之塔爾巴哈臺,與科布多毗連,以額爾齊斯河為界,河東卡倫隸科布多,河西卡倫隸塔爾巴哈臺。自輝邁拉呼至塔城,夏季設大小卡倫十三處,冬季設卡倫八處。此外皆哈薩克游牧之地。塔城西南一帶卡倫八處,界連伊犁。卡倫以外,為哈薩克游牧。伊犁東北七百餘里,與塔城接界之處,由哈布塔海達闌一帶而南,設大小卡倫二十三處。此外亦哈薩克游牧。又西而南,至伊犁河北岸,設大小卡倫八處,乃索倫領隊大臣專轄。自伊犁河南而西,設大小卡倫十六處,乃錫伯領隊大臣專轄。卡倫之外,與哈薩克接壤。其錫伯屯牧西南,因有回子屯所,每年夏秋設卡倫於達耳達木圖,以資巡察。由錫伯卡倫迤西轉南而東,設大小卡倫十七處,乃厄魯特領隊大臣專轄。西南為布魯特游牧,西北為哈薩克游牧。又厄魯特游牧東南,界連喀喇沙爾之土爾扈特、和碩特游牧,設大小卡倫八處,亦厄魯特領隊大臣專轄。其伊犁城北塔耳奇一帶,及伊犁河渡口,設卡倫七處,專為哈薩克貿易交通,並稽察逃人而設,乃惠寧領隊大臣專轄。此伊犁及塔爾巴哈臺大小卡倫之方向也。

其南路自伊犁南經木蘇耳達巴罕至回疆烏什城西北一帶,設卡倫六處,外通布魯特,乃烏什辦事大臣專轄。自烏什而西,經草地及布魯特游牧地樹窩子等處七百餘里,直達喀什噶爾城,由城東北而西轉南,設卡倫十七處,外通布魯特,西達霍罕安集延,乃喀什噶爾領隊大臣專轄。自喀什噶爾東南行二百餘里,至英吉沙爾城,由城西北而南,設卡倫十二處,外通布魯特,西南行千數百里,至巴達克山,乃英吉沙爾領隊大臣專轄。自英吉沙爾東行三百餘里,至葉爾羌城,由城西南轉而東北,設卡倫七處,西南一帶,外通布魯特,東北一帶,專為稽查逃人,乃葉爾羌辦事大臣專轄。又東南行七百餘里,至和闐城,城外之東西河,共設卡倫十二處,為稽查採玉回民,又札馬耳路通阿克蘇,專設卡倫一處,均和闐領隊大臣專轄。自葉爾羌東北行一千四百里,至阿克蘇城,其東北路通著勒士斯,專設卡倫一處,稽查喀喇沙爾所屬之土耳扈特游牧,乃阿克蘇辦事大臣專轄。又東北行七百餘里,至庫車城,由城西北而南,設卡倫五處,又東北行八百餘里,至喀喇沙爾城,城之東北設卡倫二處,又東北行九百餘里,至吐魯番城,由城西南而東,設卡倫六處,又東北行一千七百餘里,至哈密城,城東北設卡倫四處,均由駐扎各城大臣專轄。此回疆各城所屬大小卡倫之方向也。

自咸、同朝回逆鴟張,俄羅斯復乘機蠶食,邊堠盡廢。迨新疆定後,至光緒五年,收回伊犁,與俄羅斯畫定邊界,規復舊日卡倫之制。卡倫之例有三:其在內者曰常設卡倫,在外者曰移設卡倫,最在外者曰添設卡倫。三者惟常設卡倫為永遠駐守之地。餘皆值氣候和暖則外展,寒則內遷,進退盈縮,或千里,或數百里不等,沙漠浩蕩,漫無定準,皆在常設卡倫之外。自西域亂作,凡移設、添設之卡倫,悉為俄人所攘奪。左宗棠平定新疆,乃與俄羅斯重定界約,凡常設卡倫以外,均作為甌脫之地,中、俄邊境之民,彼此不居,以免逼處。其常設卡倫,嚴申舊制,邊烽少息矣。

其黔、楚苗疆碉堡之制,始於嘉慶朝征苗之役,傅鼐精練鄉兵,遍設碉堡,師苗技以制苗,遂平邊患。自湖南乾州界之木林坪起,至中營之四路口,築圍墻百數十里,以杜竄擾。其險隘處增設屯堡,聯以碉卡。鳳凰境內,設堡卡碉臺八百八十七座。永綏境內,設堡卡碉臺一百三十二座。乾州境內,設汛碉一百二十一處。古丈坪及保縣境內,設汛碉六十九處。環苗疆數百里,烽燧相望,聲息相聞。關墻則沿山澗建之。砲臺則擇沖要處築之,哨臺則於關墻之隙修之。卡碉屯堡,則因地制宜,或品字式,或一字式,或梅花式。其修建之制,關墻則土石兼施,砲臺則以石砌,而築土以實中心,哨臺亦石砌,環鑿槍孔,高峻堅實。碉樓之制亦然。關墻以嚴邊界,砲臺以備堵截戰守,哨臺為巡邏了望之用,屯堡為邊民聚衛之所,卡碉則戰守兼資。其防守兵丁,有警則荷戈,無事則秉耒,進攻退守,為持久計,以待敵之可勝,遂以底定蠻荒雲。


\end{pinyinscope}