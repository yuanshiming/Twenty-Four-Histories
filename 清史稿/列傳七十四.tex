\article{列傳七十四}

\begin{pinyinscope}
佟國維馬齊子富良馬齊弟馬武馬武子保祝

阿靈阿子阿爾松阿揆★鄂倫岱

佟國維,滿洲鑲黃旗人,佟國賴次子,孝康章皇后幼弟,孝懿仁皇后父也。順治間,授一等侍衛。康熙九年,授內大臣。吳三桂反,子應熊以額駙居京師,謀為亂,以紅帽為號。國維發其事,命率侍衛三十人捕治,獲十餘人,械送刑部誅之。二十一年,授領侍衛內大臣、議政大臣。二十八年,推孝懿仁皇后恩,封一等公。

二十九年,師征噶爾丹,命參贊大將軍裕親王軍務,次烏闌布通,與兄都統國綱並率左翼兵進戰。國綱戰沒,國維自山腰出賊後擊之,潰遁。師還,以未窮追,部議當奪官,命罷議政大臣,鐫四級留任。三十五年,從上征噶爾丹,出獨石口,以駝運稽遲請罪,上貰之。三十六年,復從上征噶爾丹,噶爾丹竄死。敘功,還所鐫級。四十三年,以老解任。

四十七年,皇太子允礽以病廢幽禁,上鬱怒成疾。國維奏:「皇上治事精明,斷無錯誤。此事於聖躬關系甚大,請度日後若易於措置,祈速賜睿斷;若難於措置,亦祈速賜睿斷。總之,將原定意指熟慮施行為是。」上命諸大臣保奏諸皇子中孰可為皇太子者,諸大臣舉皇子允禩,上愈不懌。旋以皇太子病愈,命釋之。四十八年正月,召諸大臣詰孰先舉允禩,實出大學士馬齊。上召國維,舉國維前奏語,問:「爾既解任,事與爾無與。乃先眾陳奏,何意?」國維對:「臣雖解任,蒙皇上命為國舅,冀聖躬速愈,故請速定其事。」上曰:「將來措置難易,至時自知之。人其可懷私而妄言乎?」次日,復諭曰:「爾每言祝天求佛,原皇上萬歲。嗣後惟深念朕躬,謂諸皇子皆吾君之子,不有所依附而陷害其餘,是即俾朕易於措置也。」閱月,上已定復立允礽為皇太子,又諭曰:「爾乃國舅,又為大臣。皇太子前染瘋疾,朕為國家計,安可不行拘執?後知為人鎮魘,調治全愈,又安可不行釋放?朕拘執皇太子時,並無他意。不知爾肆出大言,激烈陳奏,果何心也?諸大臣聞爾言,眾皆恐懼,遂欲立允禩為皇太子,列名保奏。朕臨御已久,安享太平,並無所謂難措置者,臣庶亦各安逸得所。今因爾言,群小復肆為妄語,諸臣俱終日憂慮,若無生路。此事關系甚重,爾既有此奏,必有確見,其何以令朕及皇太子、諸皇子不致殷憂,眾心亦可定?其明白陳奏。」國維引罪請誅戮。上復諭曰:「朕特為安撫群臣,非欲有所誅戮。爾初陳奏,眾方贊爾,謂如此方可謂國家大臣。今爾情狀畢露,人將謂爾為何如人?朕斷不加爾誅戮,爾其無懼,但不可卸責於朕。觀爾言迷妄,其亦為人鎮魘歟?」

五十八年,卒,賜祭葬。雍正元年,贈太傅,謚端純。世宗手書「仁孝勤恪」榜,命表於墓道。子隆科多,自有傳。

馬齊,富察氏,滿洲鑲黃旗人,米斯翰子。由廕生授工部員外郎。歷郎中,遷內閣侍讀學士。康熙二十四年,出為山西布政使,擢巡撫。馬齊入覲,上褒其居官勤慎,勉以始終如一。久之,上命九卿舉督撫清廉如於成龍者,以馬齊及範成勛、姚締虞對。尋命偕成龍、開音布往按湖廣巡撫張汧貪黷狀。初命侍郎色楞額往按上荊南道祖澤深,並令察汧,色楞額曲庇,不以實陳。馬齊與成龍覆按,具得汧、澤深貪墨狀,並色楞額論罪如律。

二十七年,遷左都御史。時俄羅斯遣使請定界,詔遣大臣往議。馬齊疏言:「俄羅斯侵據疆土,我師困之於雅克薩城,本可立時剿滅,皇上寬容,不忍加誅。今悔罪求和,特遣大臣往議,垂之史冊,關系甚鉅。其檔案宜兼書漢字,使臣並參用漢員。」詔如議行。尋命偕尚書張玉書等勘閱河工。二十九年,列議政大臣。都御史與議政,自馬齊始。尋遷兵部尚書。時喀爾喀諸部避噶爾丹侵掠,舉族內鄉。詔沿邊安插,命馬齊偕侍郎布圖等先期檄左右翼部長至上都河、額爾屯河兩界以待。上出塞,喀爾喀諸部朝行在,定諸王、貝子、公等爵秩牧地。烏珠穆沁臺吉車根等叛附噶爾丹,命馬齊往按,寘諸法。調戶部尚書。三十五年,上親征噶爾丹,命馬齊檄喀喇沁、翁牛特兵備戰。還京師,兼理籓院尚書。噶爾丹旋敗遁,詔來春復親出塞,命先期往寧夏安置驛站。三十八年,授武英殿大學士,賜御書「永世翼戴」榜。

四十七年冬,皇太子允礽既廢,儲位未定,佟國維奏請速斷。上召滿、漢文武諸大臣集暢春園議諸皇子中孰可為皇太子者。上意在復立皇太子,而諸皇子中貝勒允禩覬為皇太子最力,諸大臣揆敘、王鴻緒及佟國綱子鄂倫岱等為之羽翼。集議日,馬齊先至,張玉書後入,問:「眾意誰屬?」馬齊言眾有欲舉八阿哥者。俄,上命馬齊毋預議,馬齊避去。阿靈阿等書「八」字密示諸大臣,諸大臣遂以允禩名上,上不懌。明年正月,召諸大臣問其日先舉允禩者為誰,群臣莫敢對。上嚴詰,群指都統巴琿岱。上曰:「是必佟國維、馬齊意也。」馬齊奏辯。巴琿岱言漢大臣先舉。上以問大學士張玉書,玉書乃直舉馬齊語以對。上曰:「馬齊素謬亂。如此大事,尚懷私意,謀立允禩,豈非為異日恣肆專行計耶?」馬齊復力辯,辭窮,先出。翌日,上諭廷臣曰:「馬齊效用久,朕意欲保全之。昨乃拂袖而出,人臣作威福如此,罪不可赦!」遂執馬齊及其弟馬武、李榮保下獄。王大臣議馬齊斬,馬武、李榮保坐罪有差,盡奪其族人官,上不忍誅,命以馬齊付允禩嚴錮,李榮保、馬武並奪官。

四十九年,俄羅斯來互市,上念馬齊習邊事,令董其事,李榮保、馬武皆復起。尋命馬齊署內務府總管。五十五年,復授武英殿大學士。

世宗即位,降敕褒諭,予一等阿達哈哈番,尋命襲其祖哈什屯一等阿思哈尼哈番,進二等伯,加太子太保。雍正元年,改保和殿,進太保。三年,復降詔褒其忠誠,加拜他喇布勒哈番,以其子富良襲。十三年,引疾乞罷,許致仕。乾隆四年,病篤,高宗諭謂馬齊歷相三朝,年逾大耋,舉朝大臣未有及者,命和親王及皇長子視疾。尋卒,年八十八,贈太傅,謚文穆。子富興,襲爵,坐事黜,以富良襲,進一等伯。十五年,加封號曰敦惠。

富良,自散秩大臣授鑾儀衛鑾儀使,累遷西安將軍,兼領侍衛內大臣。卒,謚恭勤。

馬武,馬齊弟。初授侍衛,兼管佐領。累擢鑲白旗漢軍副都統。因馬齊得罪奪官。旋起內務府總管,遷鑲白旗蒙古都統。世宗即位,授領侍衛內大臣。雍正四年,卒,命視伯爵賜恤,授三等阿達哈哈番,賜祭葬,謚勤恪。

馬武子保祝,初授侍衛。累遷直隸提督,以病解任,起正紅旗蒙古都統。卒,謚恭簡。

阿靈阿,鈕祜祿氏,滿洲鑲黃旗人,遏必隆第五子。初任侍衛,兼佐領。康熙二十五年,襲一等公,授散秩大臣,擢鑲黃旗滿洲都統。阿靈阿女兄,上冊為貴妃。貴妃薨,殯朝陽門外,阿靈阿舉家在殯所持喪。與兄法喀素不睦,欲致之死,乃播蜚語誣法喀。法喀以聞,上震怒,奪阿靈阿職,仍留公爵。尋授一等侍衛,累遷正藍旗蒙古都統,擢領侍衛內大臣、理籓院尚書。四十七年,與揆敘、王鴻緒等密議舉允禩為皇太子。上以馬齊示意諸大臣,予嚴譴,不復窮治興大獄。五十五年,卒。

子阿爾松阿,降襲二等公,擢領侍衛內大臣、刑部尚書。雍正二年,世宗召諸大臣諭曰:「本朝大臣中,居心奸險,結黨營私,惟阿靈阿、揆敘為甚。當年二阿哥之廢,斷自聖衷。豈因臣下蜚語遂行廢立?乃阿靈阿、揆敘攘為己力,要結允禩等,造作無稽之言,轉相傳播,致皇考憤懣,莫可究詰。阿靈阿子阿爾松阿柔奸狡猾,甚於其父。令奪官,遣往奉天守其祖墓;並將阿靈阿墓碑改鐫『不臣不弟暴悍貪庸阿靈阿之墓』,以正其罪。」四年,命誅阿爾松阿,妻子沒入官。乾隆元年,以阿靈阿墓碑立祖塋前,墓已遷而碑尚存,命去之。妻子釋令歸旗。

揆敘,字凱功,納喇氏,滿洲正黃旗人,大學士明珠子。康熙三十五年,自二等侍衛授翰林院侍讀,充日講起居注官。累擢翰林院掌院學士,兼禮部侍郎。奉使冊封朝鮮王妃。尋充經筵講官,教習庶吉士。遷工部侍郎。

初,明珠柄政,勢焰薰灼。大治園亭,賓客滿門下。揆敘交游既廣,尤工結納,素與允禩相結。皇太子既廢,揆敘與阿靈阿等播蜚語,言皇太子諸失德狀,杜其復立。四十七年冬,上召滿、漢大臣問諸皇子中孰可為皇太子者,揆敘及阿靈阿、鄂倫岱、王鴻緒等私與諸大臣通消息,諸大臣遂舉允禩。事具馬齊傳。

五十一年,遷左都御史,仍掌翰林院事。疏言:「近聞外省塘報,故摭拾大小事件,名曰『小報』,駭人耳目。請飭嚴禁,庶好事不端之人,知所儆懼。」詔允行。五十六年,卒,謚文端。雍正二年,發揆敘及阿靈阿罪狀,追奪揆敘官,削謚。墓碑改鐫「不忠不孝陰險柔佞揆敘之墓」。

鄂倫岱,滿洲鑲黃旗人,佟國綱長子。初任一等侍衛。出為廣州駐防副都統。康熙二十九年,擢鑲黃旗漢軍都統,襲一等公。三十五年,上親征噶爾丹,鄂倫岱領漢軍兩旗火器營,出古北口。扈蹕北巡塞外。三十六年,擢領侍衛內大臣。坐事降一等侍衛。尋授散秩大臣。四十六年,復授領侍衛內大臣。五十九年,命出邊管蒙古驛站。世宗立,召還,授正藍旗漢軍都統。

雍正三年,諭曰:「鄂倫岱與阿靈阿皆黨於允禩。當日允禩得罪,皇考時方駐蹕遙亭,命執允禩門下宦者刑訊,具言鄂倫岱等黨附狀。鄂倫岱等色變,不敢置辯。四十九年春,皇考自霸州回鑾,途中責鄂倫岱等結黨,鄂倫岱悍然不顧。又從幸熱河,皇考不豫,鄂倫岱日率乾清門侍衛較射游戲。皇考於行圍時數其罪,命侍衛鞭撻之。鄂倫岱頑悍怨望,雖置極典,不足蔽辜。朕念為皇祖妣、皇妣之戚,父又陣亡,不忍加誅。令往奉天與阿爾松阿同居。」四年,與阿爾松阿並誅,仍諭不籍其家,不沒其妻子。

子補熙,自廕生授理籓院員外郎,襲國綱拜他喇布勒哈番世職,官至綏遠城將軍。卒。謚溫僖。

論曰:理密親王既廢,自諸皇子允禟、允示我輩及諸大臣多謀擁允禩,聖祖終不許。誠以儲位至重,非可以覬覦攘奪而致也。佟國維陳奏激切,意若不利於故皇太子,語不及允禩,而意有所在,馬齊遂示意諸大臣。然二人者,皆非出本心,聖祖諒之,世宗亦諒之,故能恩禮勿替,賞延於後嗣。若阿靈阿父子、揆敘、鄂倫岱、王鴻緒固擁允禩最力者,世宗既譴允禩,諸臣生者被重誅,死者蒙惡名,將安所逃罪?鴻緒又坐與徐乾學等比,被論。事別見,故不著於此篇。


\end{pinyinscope}