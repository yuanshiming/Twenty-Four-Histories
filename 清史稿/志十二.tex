\article{志十二}

\begin{pinyinscope}
天文十二

日食月五星凌犯掩距太白晝見日變月變

日食三統、四分,皆有推月食術,而無推日食術。由日食或見或否,或淺或深,隨地而變。不詳其數,立術綦難。故自古以為尤異,每食,史冊必書。後人推日食之術密矣,猶必書者,從其朔也。其見於本紀,無食分及所次宿,備以入志,言推步者考焉。

順治元年八月丙辰朔午時,日食二分太,次於張。五年五月乙丑朔卯時,日食九分強,次於觜觿。七年十月辛巳朔巳時,日食七分太,次於亢。十四年五月癸卯朔寅時,日食六分半強,次於觜觿。十五年五月丁酉朔辰時,日食四分少,次於畢。

康熙三年十二月戊午朔申時,日食九分弱,次於南斗。五年六月庚戌朔申時,日食九分太強,次於東井。八年四月癸亥朔未時,日食五分半,次於胃。十年八月己卯朔申時,日食二分,次於翼。二十年八月辛巳朔辰時,日食三分太強,次於翼。二十四年十一月丁巳朔申時,日食二分少強,次於心。二十七年四月癸卯朔辰時,日食九分太強,次於胃。二十九年八月己未朔卯時,日食二分太,次於翼。三十年二月丁巳朔午時,日食三分少強,次於危。三十一年正月辛亥朔午時,日食五分少強,次於危。三十四年十一月己未朔申時,日食八分半強,次於尾。三十六年閏三月辛巳朔辰時,日食既,次於婁。四十三年十一月丁酉朔午時,日食四分半強,次於心。四十五年四月戊子朔酉時,日食六分半弱,次於胃。四十七年八月甲辰朔申時,日食五分少強,次於翼。四十八年八月己亥朔卯時,日食五分弱,次於翼。五十一年六月癸丑朔寅時,日食五分太弱,次於東井。五十四年四月丙寅朔酉時,日食六分少弱,次於胃。五十八年正月甲戌朔申時,日食七分,次於危。五十九年七月丙寅朔巳時,日食七分,次於柳。六十年閏六月庚申朔酉時,日食四分,次於東井。

雍正八年六月戊戌朔巳時,日食九分,次於東井。九年十二月庚寅朔卯時,日食八分太,次於南斗。十三年九月丁酉朔辰時,日食八分少強,次於角。

乾隆七年五月己未朔辰時,日食七分強,次於畢。十年三月癸酉朔巳時,日食一分少弱,次於東壁。十一年三月丁卯朔午時,日食七分弱,次於營室。十二年七月己丑朔申時,日食二分少強,次於柳。十六年五月丁酉朔辰時,日食四分太弱,次於昴。二十三年十二月癸丑朔申時,日食八分太強,次於南斗。二十五年五月甲辰朔酉時,日食九分太弱,次於參。二十七年九月庚申朔酉時,日食五分太弱,次於角。二十八年九月乙卯朔辰時,日食七分,次於軫。三十四年五月壬午朔酉時,日食三分半強,次於畢。三十五年五月丁丑朔辰時,日食四分弱,次於昴。三十八年三月庚寅朔未時,日食四分少弱,次於營室。三十九年八月壬午朔辰時,日食三分太強,次於張。四十年八月丙子朔午時,日食四分半強,次於七星。十二月甲辰朔巳時,日食一分太強,次於南斗。四十九年七月甲寅朔卯時,日食二分弱,次於柳。五十年七月戊申朔辰時,日食四分少強,次於柳。五十一年正月丙午朔巳時,日食七分少弱,次於婺女。五十三年五月壬戌朔酉時,日食三分半弱,次於畢。五十四年十月癸丑朔巳時,日食五分太弱,次於氐。六十年正月甲申朔卯時,日食九分弱,次於南斗。

月五星凌犯掩距天官書言「相凌為★」,又云「七寸以內必之」,謂緯度相迫如交食也。今法,兩星相距三分以內為凌,月與星相距十七分以內為凌,俱以相距一度以內為犯,相襲為掩。欽天監每年預推月五星入此限者,繕冊進呈,本名凌犯書,雍正初年,改名相距書。既憑占候,即課推步,各循本稱,★志所在之宿。

順治元年七月庚寅,熒惑犯歲星於昴。三年十一月辛酉,月犯歲星於柳。四年二月壬午,掩歲星於觜。六年六月癸巳,熒惑犯歲星於翼。七年五月甲子,月犯歲星於亢。十年十月庚午,熒惑犯歲星於女;十一月己亥,太白犯歲星於女;十二月甲申,熒惑犯歲星於壁。十七年五月壬申,熒惑犯太白於柳。十月甲午,太白犯歲星於軫。

康熙四年六月丁卯,月犯填星於箕。八年十二月丁卯,太白犯填星於虛。十二年三月甲戌,月犯太白於壁。十三年六月戊申,熒惑犯填星於奎。十八年五月己未,月犯歲星於婁。二十一年八月丙申,熒惑犯填星於柳;九月己巳,歲星掩填星於柳。三十年六月戊寅,熒惑犯歲星於婁。三十一年五月丁卯,太白犯熒惑於星。三十三年十一月癸未,犯填星於斗。三十八年十二月丁丑,犯填星於危。四十年二月庚申,犯歲星於虛。五十五年十月戊戌,犯填星於軫。五十七年九月丙申,辰星犯填星於亢。

雍正二年十二月丙申,太白距填星於斗。三年二月壬申,距歲星於危。九年十月癸丑,熒惑距歲星於翼。十二年正月甲申,距歲星於心。

乾隆元年三月乙巳,距歲星於女;壬戌,月距太白於奎;五月己亥,太白距填星於昴;八月戊辰,辰星距太白於翼;十二月乙丑,太白距歲星於女;丁亥,辰星距歲星於虛。

二年正月癸巳,月距太白於室;二月乙酉,辰星距歲星於危;五月己丑,距太白於觜;八月癸亥,距熒惑於張。閏九月乙丑,月距歲星於危;十月癸巳,如之。

三年二月壬辰,太白距歲星於壁;辛丑,辰星距歲星於壁;三月乙卯,距太白於婁;四月乙未,太白距辰星於觜;五月丙寅,熒惑距歲星於奎;丁丑,辰星距填星於井;七月壬子,月距辰星於張。

四年五月庚申,太白距歲星於胃;六月壬午,辰星距填星於井;七月乙巳朔,太白距填星於井;丁未,月距辰星於虛;壬戌,辰星距熒惑於翼;十月癸未;太白距熒惑於氐;乙酉,辰星距熒惑於氐;癸巳,月距填星於井;丙戌,辰星距太白於氐;十一月庚申,月距填星於井;十二月戊子,五年正月乙卯、四月丁丑,如之;五月丙寅,辰星距歲星於觜;六月壬申,月距填星於井;閏六月甲子,距歲星於井;七月壬辰,距熒惑於井;八月壬寅,熒惑距歲星於井;庚申,月掩歲星於井;距熒惑於井;九月戊子,復距;十月甲寅,距歲星於井;丙辰,距熒惑於鬼;十一月辛巳,距歲星於井;甲午,距太白於危;十二月戊申,距歲星於觜;六年正月乙亥,如之;三月庚午,掩歲星於井;壬申,距熒惑於井;五月甲子朔,辰星距太白於觜;丁亥,太白距歲星於井;己丑,辰星距填星於柳;六月甲辰,太白距填星於柳;八月辛亥,距熒惑於亢;九月壬午,辰星距熒惑於氐;十月丙申,月距太白於箕;壬子,熒惑距辰星於尾;十一月癸亥,月距熒惑於尾;十二月甲寅,辰星距熒惑於牛。

七年五月乙酉,月距熒惑於昴;八月乙丑,辰星距歲星填星於星,歲星距填星於星;庚戌,太白距填星於星;壬子,距歲星於星;十月乙卯,距辰星於氐;十二月癸卯;月距熒惑於張;乙巳,辰星距太白於斗。

八年正月庚午,月距熒惑於星;四月戊申,熒惑距填星於星;閏四月壬戌,距歲星於張;五月丙午,太白距填星於張;六月甲寅,距歲星於張;七月癸巳,辰星距填星於張;庚子,距歲星於翼;八月乙卯,月距熒惑於氐;十一月辛丑,距歲星於軫;十二月壬子,辰星距熒惑於女;己巳,月距歲星於軫。

九年二月辛酉,距填星於張;癸亥,掩歲星於軫;三月甲申,辰星距熒惑於奎;庚寅,月距歲星於翼;四月戊午,如之;六月乙亥,辰星距太白於柳;七月戊寅,月距填星於張;丙申,辰星距填星於張;八月戊申,月距歲星於軫,太白距填星於翼;壬申,距歲星於角;十月庚戌,辰星距歲星於角;戊辰,月距熒惑於張;己巳,距填星於翼;十一月辛丑,熒惑距填星於翼。

十年二月丁未,月距太白於婁;三月甲戌,熒惑距填星於翼;四月壬子,月距熒惑於翼;五月戊戌,辰星距太白於畢;癸巳,距填星於軫;十月壬子,太白距填星於軫;甲子,辰星距歲星於氐;十一月甲午,太白距歲星於房。

十一年二月己亥,月距熒惑於室;癸亥,辰星距太白於室;三月壬午,距熒惑於奎;閏三月癸卯,太白距熒惑於婁;五月乙巳,辰星距熒惑於觜;六月戊辰,月距太白於星;七月壬戌,距熒惑於柳;八月庚寅,距熒惑於張;九月庚子,辰星距填星於軫;十二月癸亥,熒惑距填星於角。

十二年七月辛卯,月距辰星於張;十月庚寅,辰星距太白於翼;十一月辛卯,月距熒惑於女;甲午,太白距辰星於尾;十二月丁巳朔,距歲星於斗;戊午,月距太白於斗。

十三年正月己丑,辰星距歲星於牛;癸丑,月距歲星於女;二月辛未,太白距熒惑於奎;五月壬寅,辰星距熒惑於觜;六月乙卯,月距辰星於井;八月壬辰,距歲星於女;九月己卯,太白距熒惑於軫;十月庚戌;距填星於氐;十二月癸卯,辰星距歲星於危。

十四年二月甲申,太白距歲星於危;三月戊辰,月距熒惑於斗;乙亥,辰星距太白於胃;六月乙巳,月距辰星於井;十月己卯,距太白於箕;癸卯,辰星距填星於氐;十一月己巳,熒惑距歲星於室。

十五年正月癸酉,月距辰星於危;三月丁巳,辰星距歲星於壁;四月乙亥,月距熒惑於畢;庚子,太白距歲星於奎;五月壬寅朔,辰星距熒惑於井;己未,熒惑距辰星於井;七月己未,辰星距熒惑於星;九月己未,太白距熒惑於翼;十月戊寅,距辰星於角;十一月戊申,辰星距太白、填星於心。

十六年正月癸亥,月距熒惑於尾;三月癸丑,太白距歲星於婁;四月己巳,月距太白於昴;閏五月壬辰,距歲星於昴;七月丁亥,距歲星於畢;八月丁酉,距辰星於角;九月壬辰,距太白於角;十月乙未,距填星於尾;戊申,距歲星於畢;十一月丙子,距歲星於昴;庚寅,距填星於尾;十二月己未,距太白於箕。

十七年正月庚午,距歲星於昴;二月戊戌,距歲星於畢;四月庚申,距太白於昴;五月甲戌,距填星於尾;辛巳,太白距歲星於觜;壬午,辰星距歲星於觜;癸未,距太白於觜;六月辛丑,月距填星於尾;甲寅,辰星距熒惑於星;七月丁亥,太白距熒惑於張;八月庚寅,月距辰星於翼;十月乙卯,距熒惑於氐。

十八年五月辛未,辰星距太白於畢;十一月甲戌,填星距辰星於斗;十二月丙午,太白距填星於斗。

十九年正月壬子,辰星距填星於斗;五月己丑,熒惑距歲星於柳;乙巳,太白距歲星於柳;丁未,辰星距歲星於柳;六月丙辰,太白距熒惑於張;七月甲辰,辰星距歲星於張。

二十年正月丁丑,熒惑距填星於斗;辛丑,月距太白於牛;三月辛巳,太白距熒惑於危;庚子,月距熒惑於室;七月庚寅,辰星距太白於柳;九月己亥,月距歲星於軫;二十一年正月己丑,如之;八月辛酉,熒惑距歲星於角;十一月壬戌,辰星距填星於女。

二十二年正月乙巳,太白距填星於女;辛酉,月距辰星於虛;三月乙卯,距填星於虛;六月丁丑,距填星於女;九月戊戌,如之;十一月壬寅,太白距填星於女。

二十四年十二月乙未,月距熒惑於翼;二十五年正月辛酉,如之;甲戌,太白距歲星於女;二月乙巳,距填星於室;三月甲戌,距辰星於奎;四月癸巳,辰星距太白於胃;十一月辛亥,太白距熒惑於斗;十二月戊寅,距歲星於虛;己亥,熒惑掩歲星於危。

二十六年正月庚午,辰星距熒惑於室;十月癸巳,月距太白於亢。

二十七年五月戊午,辰星距太白於井;己未,月距歲星於婁;七月丁卯,距熒惑於氐;十月辛丑,距填星於壁;十二月丁酉,距歲星於婁。

二十八年四月丁未,太白距填星於婁;五月壬申,距歲星於昴;六月甲寅,辰星距熒惑於井;七月丁卯,太白距熒惑於柳。

三十年二月庚寅,辰星距太白於虛;八月己巳,月距歲星於柳;十月甲子、三十一年三月己卯,如之;六月辛亥,太白距填星於畢;辛酉,辰星距歲星於星。

三十二年五月甲子朔,太白距熒惑於井;丙寅,月距熒惑於井;六月丁未,辰星距熒惑於鬼;七月丁卯,太白距歲星於翼;八月乙丑,月距太白於亢。

三十三年九月丙午,距填星於井。

三十四年七月丁未,距太白於井;十一月乙未,辰星距歲星於心;乙巳,太白距歲星於心,熒惑距歲星於心;丁未,月距熒惑於心。

三十五年二月庚午,距歲星於尾;四月丁卯,辰星距太白於畢;五月辛卯,月距歲星於尾;丙申,太白距辰星於井;閏五月戊午,月距歲星於尾;六月丙戌、八月庚辰,如之。

三十六年四月丙申,熒惑距填星於柳;七月己未,太白距填星於柳;癸亥,辰星距熒惑於翼;十一月丙寅,距歲星於牛;十二月戊辰,太白距歲星於牛。

三十七年正月己未,辰星距熒惑於女;辛酉,距歲星於女;甲子,熒惑距歲星於女;二月己巳,月距太白於壁;八月乙丑,距辰星於翼;庚寅,距太白於柳。

三十八年五月己巳,熒惑距填星於張;庚辰,月距歲星於壁;八月戊戌,辰星距填星於張;九月戊辰,月距歲星於室;十月乙未、十二月庚寅,如之。

三十九年正月戊午,距歲星於壁;九月壬戌,太白距填星於翼;十月丙午,月距填星於翼;十二月辛丑,距填星於軫;四十年正月戊辰,如之;三月乙丑,太白距歲星於胃;四月己丑,月距填星於翼;十二月丁巳,辰星距熒惑於虛。

四十一年六月壬戌,熒惑距歲星於井。

四十二年五月戊辰,月距歲星於井;八月甲寅,太白距歲星於柳;十月戊申,辰星距填星於亢;十一月癸亥朔,太白距填星於亢;辛卯,月距太白於尾;十二月甲午,距辰星於斗。

四十三年三月丁丑,距填星於氐;四月乙巳如之;六月辛卯,距太白於氐;丙辰,太白距歲星於星;閏六月丙寅,月距填星於亢;八月辛酉,距填星於氐;九月庚子,熒惑距歲星於張;十月甲申,月距填星於氐;十二月戊寅,距熒惑於氐。

四十四年二月庚申,熒惑距填星於氐;甲戌,月距熒惑於氐;三月辛丑,如之;四月壬午,距太白於胃;九月戊子,辰星距歲星於軫;十一月癸巳,距太白於斗。

四十五年正月己亥,月掩歲星於角;二月壬戌,太白距熒惑於婁;三月癸巳,月距歲星於角;五月癸卯,辰星距熒惑於井;七月壬午,月距歲星於角;九月己亥,辰星距歲星於亢;十月庚午,太白距熒惑於軫;壬申,月距熒惑太白於軫;十二月辛酉,太白距填星於尾;壬申,熒惑距歲星於氐。

四十六年七月辛亥,月距熒惑於斗。

四十七年五月辛丑,辰星距熒惑於井;九月戊申,熒惑距太白於翼;十月丁卯,歲星距填星於箕;丙子,辰星距太白於氐。

四十八年三月乙未,熒惑距歲星於斗;六月甲子,月距太白於柳;十二月戊寅,辰星距歲星於女。

四十九年二月丙子,距歲星於虛;三月丙戌朔,太白距歲星於虛;五月甲戌,月距歲星於危;八月乙未,距歲星於虛;十二月庚寅,太白距歲星於危。

五十年二月癸未,熒惑距填星於牛;壬辰,辰星距歲星於室;三月癸丑,月距太白於畢;五月甲寅,熒惑距歲星於壁;戊午,辰星距太白於昴;十月癸未,月距填星於斗;十二月戊寅,距填星於牛。

五十一年三月庚戌,距熒惑於參;七月丙午,太白距熒惑於張;九月乙亥,月距太白於尾;丁丑,辰星距熒惑於亢。

五十二年二月庚子,熒惑距填星於虛;三月丁丑,太白距熒惑於室;五月乙酉,距歲星於畢;七月丙戌,熒惑距歲星於參;庚寅,月距熒惑於參;十月辛亥,距歲星於參;十一月己卯,距歲星於畢;十二月辛亥,太白距填星於虛。

五十四年閏五月庚戌,距歲星於鬼。

五十五年三月癸卯,熒惑距歲星於柳;八月壬子,月距熒惑於亢;甲戌,太白距歲星於張。九月戊戌,辰星距太白於軫。十一月壬寅,距熒惑於斗。

五十六年十月壬寅朔,距歲星於軫;己巳,月距太白於角;辛未,太白距歲星於角。十一月丁酉,月距歲星於角;戊戌,距太白於氐。

五十七年正月辛卯,距歲星於亢。二月己未,距歲星於角。三月丁酉,距太白於室。五月丙午,距熒惑於翼。十月丁丑,距填星於奎。十一月癸卯,太白距熒惑於牛;甲辰,月距填星於奎。

五十八年正月丙申,距熒惑於危。二月甲戌,辰星距熒惑於壁。九月戊午,月距太白於翼。十月乙酉,距熒惑於翼。十一月丙辰,太白距歲星於尾。

六十年正月辛亥,月距歲星於斗。五月壬戌,太白距填星於畢。七月乙卯,辰星距太白熒惑於柳,太白距熒惑於柳;壬戌,月距歲星於斗。十二月庚辰,太白距歲星於女。

太白晝見太白見於午位者,康熙元年四月庚午,四年六月甲戌,俱不著時。七年六月癸酉至丁丑,俱未時。九年五月戊午、乙丑,十年六月甲午,十二年六月庚申,十三年十一月丁卯,十五年五月甲申,俱不著時,九月丙戌巳時。十六年十二月辛酉,不著時。十七年五月庚申,巳時。乾隆八年七月庚寅、壬辰,俱未正三刻。十年七月丙子、丁丑,俱辰時。十三年八月丙午至辛亥,九月癸丑、丙辰、丁巳、己未、辛酉、壬戌、丙寅至辛未、甲戌、己卯,十月丙戌,俱巳時。十四年十二月丙子、丁丑、己卯、辛巳、丙戌至己丑、辛卯、乙未、戊戌、壬寅、癸卯,十五年正月己酉,俱未時。四月庚子,五月壬寅、乙巳、丁未、己酉、壬子、癸丑、丁巳至己未,六月乙亥至丁丑,十八年六月辛亥,俱巳時。二十四年閏六月丁亥、戊子、壬辰,二十九年六月甲申,俱未時。十月庚辰、甲申、辛卯,俱巳時。三十年十一月癸酉、庚辰至甲申、丙戌、己丑、癸巳、乙未至戊戌、庚子、辛丑,十二月乙卯、丁巳、辛酉、戊辰,三十一年正月丙子,俱申時。三十二年閏七月癸卯、丙午、丁未、庚戌、壬子至乙卯,俱未時。丁巳,申時。戊午、庚申,八月壬戌朔,俱未時。十月戊辰至庚辰、壬午至甲申,俱巳時。丁亥至庚寅,俱辰時。十一月辛卯朔、壬辰,俱巳時。癸巳、丁酉、己亥、俱辰時。五十四年十二月戊午、己未、癸亥、丙寅,俱未時。

太白見於巳位者,順治十一年五月辛亥,與日爭明。十七年九月庚辰,康熙四年三月辛卯,俱不著時。七年六月癸酉至丁丑,俱午時。十二年六月辛酉,十八年十一月丙辰,俱不著時。乾隆八年十月辛酉、甲子、丙子至十一月壬午、乙酉,十三年八月乙未,九月壬申、丁丑,十月癸未、甲申、丙戌至己丑、壬辰,乙未、丙申、戊戌至辛丑、甲辰、丁未、戊申,十一月辛亥朔、壬子、甲寅、乙卯、丁巳、己未、乙丑、丁卯、癸酉、乙亥,十五年五月壬戌,十八年六月戊戌,三十二年十月戊辰至庚辰、壬午至甲申,十一月辛卯朔、壬辰,俱辰時。

太白見於未位者,順治九年九月乙未,康熙八年十二月丁卯,十二年正月丁亥,俱不著時。十五年九月丙戌,午時。乾隆元年十二月庚午、癸酉、甲戌、己丑,二年正月庚寅朔、壬辰、癸巳、丁巳,二月辛酉、乙丑、庚辰,三年十月甲申至丙戌、戊戌、己亥、乙巳、丙午、戊申,六年十月己未至辛酉,十一月甲子、戊辰、庚午至壬申,丙子至戊寅、甲申、丙戌、戊子,十二月壬辰、癸巳、丁酉、戊戌、戊申,十年正月辛巳、癸未至丁亥、辛卯至丁酉、壬寅,二月癸卯朔、戊申,俱申時。丙辰,酉時。丁巳、癸亥、己巳、壬申,三月庚辰、癸未,十一年十月丙寅、丁卯、己巳、甲戌至丁丑、壬午、乙酉、己丑至辛卯,十一月癸巳、甲午、戊戌至辛丑,十四年十月乙巳,十一月壬子、癸丑、乙卯、丙寅、丁卯、辛未至甲戌,十二月丁丑、己卯、辛巳、丙戌至己丑、辛卯、乙未、戊戌、癸卯,三十年十月丁巳、戊午、辛酉、丙寅至戊辰,十二月癸卯、甲子,俱申時。戊辰,酉時。三十一年正月癸酉、辛巳、壬午,俱申時。三十二年閏七月丁巳,酉時。八月甲子,五十四年十二月戊午、己未、癸亥、丙寅、辛未、乙亥,俱申時。

太白見於辰位者,乾隆七年六月癸巳、甲午、丁酉、辛丑至癸卯,俱寅卯二時。丁未、戊申、庚戌、壬子、乙卯、丁巳,七月癸亥、戊寅至庚辰、壬午、甲申、乙酉,八月丁亥朔、戊子、庚寅,十年六月丁卯,七月辛巳,十三年九月甲寅,十五年六月戊寅,五十五年七月壬辰、庚子、甲辰至戊申,八月丙辰、己巳,俱卯時。

太白見於申位者,康熙二年七月丙申,連日如之,不著時。乾隆三年九月丙寅、丁卯,二十四年六月丙辰、戊午,閏六月乙酉,三十二年閏七月辛丑,俱酉時。

太白見於卯位者,康熙四年六月丙辰,不著時。

太白見於酉位者,乾隆八年五月辛卯、壬寅、甲辰至丙午,俱戌初。

太白見於辰、巳二位者,乾隆二年七月己亥、癸丑,八月甲子、癸酉、乙亥至己卯,癸未至乙酉,九月丁亥至辛卯、乙未、丁酉、庚子,甲辰至丙午,閏九月丙辰朔、辛酉、癸亥,十年七月壬申、戊寅至庚辰、壬午,十三年八月戊子、丙申至庚戌,九月乙卯氾丙辰、戊午、辛酉、丙寅至戊辰,十五年五月丁未、己酉、壬子、癸丑、戊午、己未,六月丙子、丁丑,俱卯、辰二時。

太白見於未、申二位者,乾隆二年正月丙申至戊戌、癸卯,戊午至二月庚申、丙寅、戊辰、庚午,三年九月戊辰、己巳、辛未,五年五月辛亥,八年七月己丑、庚寅、壬辰、甲午、甲辰至丙午、戊申、己酉,十五年正月己酉,三十二年閏七月壬寅、癸卯、丙午至戊申、庚戌,壬子至乙卯、戊午、庚申,八月壬戌朔、丙寅、丁卯,俱申、酉二時。

太白見於巳、未二位者,乾隆十年七月丙子、丁丑,俱卯巳二時。十三年八月辛亥,九月己未、辛未、甲戌、己卯,十五年四月庚子,五月乙巳,俱辰、午二時。

太白見於卯、辰、巳三位者,乾隆十年六月庚午,十五年五月丁巳,十八年六月癸巳、甲午,俱寅、卯、辰三時。

太白見於辰、巳、未三位者,乾隆十三年九月癸丑、丁巳、壬戌、己巳、庚午,十五年五月壬寅,六月乙亥,十八年六月辛亥,俱卯、辰、午三時。

太白晝見不著位者,順治元年六月庚午,九月己酉,三年正月己未,六年八月甲午,七年十二月辛丑,康熙七年九月戊戌、己亥,二十一年十月乙未至戊戌,二十三年五月己卯至庚寅,俱不著時。

日變月變崇德七年四月庚戌,二日並出,上大下小,須臾大日散沒。順治元年二月癸亥,月中有黑子。七年三月己未,日赤如血。十一年四月庚申朔,日出時色變赤;戊子,日色變白。十四年二月乙酉,日赤如血。康熙元年二月丁卯,日赤如血;戊辰,日出色如血,無光。十三年六月丙午,月生光一道,色蒼白。十九年四月己巳,日赤無光。二十一年六月乙巳戌時,日射青氣二道。乾隆八年三月辛巳,日赤無光。二十九年六月甲申,月見正午。十一月壬子,如之。四十八年六月戊辰,日心中出白圈,向東成圍。五十八年正月壬子,日生赤黃色大半環及大圍圈各一。二月戊子,日生赤黃色大半環。


\end{pinyinscope}