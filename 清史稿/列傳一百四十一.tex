\article{列傳一百四十一}

\begin{pinyinscope}
萬承風周系英錢樾秦瀛李宗瀚韓鼎晉硃方增

萬承風,字和圃,江西義寧人。乾隆四十六年進士,選庶吉士,授檢討。直上書房,侍宣宗讀。六十年,典試雲南。時仁宗在潛邸,賜詩寵行。累遷翰林院侍讀。嘉慶三年,大考,降檢討。四年,督廣東學政。瓊州海寇猝發,承風以聞,命總督吉慶按治,總兵西密揚阿等以恇怯置吏議。累遷侍講學士,任滿還京,直上書房,擢詹事。督山東學政,整厲士習,扶持善類。洊擢禮部侍郎,命還京。

十二年,督學江蘇。以清江浦、荷花塘河工取勢太直,屢築屢圮,奏請復舊,詔如議行。調兵部。十四年,上五旬萬壽,陳請解任還京祝嘏,詔嚴斥,左遷內閣學士。調安徽學政。定遠士子與鳳陽胥役有隙,至試期輒修怨,當事者庇胥役,士益憤,承風疏請下巡撫嚴治胥役,置諸法。擢兵部侍郎,還京,仍直上書房,充經筵講官。十七年,引疾歸,尋卒,入祀鄉賢祠。宣宗即位,追念舊學,贈禮部尚書銜,謚文恪。道光十二年,晉贈太傅,子方楙等加恩有差。

周系英,字孟才,湖南湘潭人。乾隆五十八年進士,選庶吉士,授編修,累遷侍講。嘉慶十年,督四川學政。十四年,入直南書房,擢太常寺卿。尋改直上書房,授三阿哥讀。上諭:「不但授讀講習詩文,當教阿哥為人居心以忠厚為本。」系英請加授資治通鑒,以知古今治亂興衰之故,悉民間疾苦,上韙之。轉光祿寺卿,督山西學政。任滿回京,仍直上書房。十九年,擢兵部右侍郎,母憂去,服闋,補吏部侍郎。

二十四年,湘潭民與江西客民閧,相殺傷,巡撫吳邦慶亦籍江西,陳奏偏袒。系英詢齎奏人,得事始末,於召對時面陳,乃調邦慶福建,詔以獄事畀總督察治。系英素以樸直被眷遇,邦慶初與善,約地方事有見聞必告,至是手書言其曲直;系英子汝楨亦致書在籍給事中石承藻詢獄事:書並為邦慶得,先後以兩書上聞。上怒系英庇鄉人,部議革職,猶命以編修用。繼以汝楨致書事,褫職回籍。

道光初,以四品京堂召用,歷翰林院侍讀學士、內閣學士。二年,遷工部侍郎,督江西學政,尋調江蘇,許密摺言地方利病,人才臧否。會瀕江大水,學政駐江陰,系英目擊災狀,貽書督撫,留官吏素得民者治賑務,假庫帑三萬兩購米平糶,民感之。四年,調戶部左侍郎,卒於任。

錢樾,字黼棠,浙江嘉善人。乾隆三十七年進士,選庶吉士,授編修。典陜西鄉試,督四川學政。直上書房。兩典江西鄉試,督廣西學政,累擢少詹事。嘉慶四年,還京,仍入直。驟遷內閣學士、禮部侍郎,督江蘇學政。時吳縣令甄輔廷治諸生糾控罪過當,學政平恕曲徇所請,斥革生員二十五人。上聞之,解平恕任,以樾代,至則先復諸生名,僅坐首事者三人,士民稱慶。方其赴任,途中見行船有大書「內廷南府」者,因上疏劾奸吏詭託,上累聖明,詔飭關津禁絕,嚴罪所司。

時南河邵壩決口,瓜、儀私梟充斥,為閭閻害,命樾密訪以聞。疏陳:「黃河自豫東界至桃、宿以上,水緩沙停,致河高堤淺,所在防潰。請於霜降後鳩工疏正河,並增築堤防,先務所急。又以私梟為患,皆由官鹽價貴,民利食私,若稍平鹽價,則私梟自絕。」疏入,俱報可。尋調吏部,任滿回京,調戶部,兼管錢法堂事務。奏請申禁改漕折色,以清弊端。復調吏部,九年,坐失察書吏舞弊,以告病治中趙曰濂虛選運同,降內閣學士,樾上疏置辯,議革職,加恩賜編修。十年,擢鴻臚寺少卿,督山東學政。累遷大理寺少卿、內閣學士。母憂歸,服闋,引疾不出。二十年,卒。

秦瀛,字凌滄,江蘇無錫人,諭德松齡玄孫也。乾隆四十一年,以舉人召試山東行在,授內閣中書,充軍機章京,洊遷郎中。五十八年,出為浙江溫處道,有惠政。嘉慶五年,擢按察使。寧、紹、臺三府水災,有司匿不報,瀛力言於巡撫,乃得賑。調湖南,衡州歲歉,有司匿不報,方議派濟陜西兵米,瀛復力言於巡撫,留米平糶。七年,以病歸。逾兩年,起授廣東按察使,督郡縣治盜,擒著盜梁修平、吳叚喜置諸法。撫瓊州黎匪,嚴禁賭博白鴿票。

十年,遷浙江布政使,入覲,乞內用,授光祿寺卿,轉太常寺卿。疏陳廣東治盜事宜,略曰:「海盜始在高、廉,近則闌入廣州。大股如鄭一、烏石二、總兵寶、硃濆等,聲勢甚張。內地順德、香山、新會三縣,連有肆劫,以馬觀、李英芳為之魁,與海盜勾結,捕急則遁入海中。統將出海,藉詞遷延,不能盡力。黜提督孫全謀,而魏大斌即為之續。臣愚以為剿捕之法:一曰討軍實。水師廢弛,則帑餉虛糜。洋商、鹽商捐輸寬裕,經手之員尚有侵漁,遣委之將仍復驕惰,非立法痛懲,徒資耗費。一曰樹聲威。盜善偵探,非先聲讋人,盜已輕我。兵行之日,督撫宜舉觴歡飲;有功而歸,開轅行賞,不用命者,殺無赦。一曰戒虛飾。擒盜豈能皆真,一念邀功,讞多失實,偶有平反,不復深咎。嗣後總期弋獲真盜,毋縱毋枉。至守禦之法,尤宜急講。砲臺防守口岸,口岸多而汛兵少,盜船乘間直入;巡船復少,不能御盜,且為盜資。保甲僅屬虛名,縱役訛索,反成厲政。欲行保甲團練,先須百姓服從。臣以為嚴防守必先澄清吏治,澄吏治必先固民心。一曰清獄訟。粵民好訟,大小案件,諭旨嚴飭,尚多沉擱。殆由案之初起,遲延不辦,土棍訟師,從而把持,遂至供情屢易,莫可窮究。惟有督飭州縣,有一案即清一案,務洗慵惰偏私之習。一曰抑冗濫。六計尚廉,近海州縣有緝捕解犯之責,尤宜撙節,庶不虧倉庫而累閭閻。一令到任,幕友長隨,多人坐食,勢不能復為廉吏。雜職武弁,惟利是圖,稍授以權,即挾制文吏。雜職差委過多,亦滋擾累。一日懲蠹役。胥役熟習地方情形,串同官親家屬,肆為民害。廣東胥役,每有暗通盜匪,收受陋規,此尤不可不嚴行懲創也。三者既舉而吏治澄,吏治澄而民心固,於以舉行保甲團練,無不可使之民,即無不可行之法矣。」疏上,詔下疆吏採行。遷順天府尹。

十二年,擢刑部侍郎。以宗室敏學獄會擬輕縱,議褫職,詔原之,左遷光祿寺卿。歷左副都御史、倉場侍郎。詔整頓倉場,慮瀛齒衰,以二品頂戴調左副都御史。尋授兵部侍郎,復調刑部。瀛治獄平慎,在浙辨定海難民十二人非盜。及海盜誣攀族人,已入告,卒更正省釋。在部治運丁盜米,訐者謂以藥置米中立溢,試之不驗,仁宗親試明其枉,尤為時稱。十五年,以病解任。道光元年,卒。

瀛工文章,與姚鼐相推重,體亦相近雲。

李宗瀚,字春湖,江西臨川人。乾隆五十八年進士,選庶吉士,授編修。嘉慶三年,大考二等,擢左贊善。累遷侍講學士,充日講起居注官。五年,典福建鄉試,母憂歸,服闋,補原官,轉侍讀學士。九年,督湖南學政,歷太僕寺卿、宗人府丞、左副都御史。二十年,丁本生母憂,服闋,在籍奏請終生祖母養,允之。道光三年,遭祖母喪。先是禮臣建議,為父後者為生祖母終三年喪,宗瀚幸奉功令,既而部議仍改期服,宗瀚本生父秉禮已老,而有子四人,以出繼不得終養。五年,入都,召見,詢家世官資甚悉。宗瀚具陳終養始末,宣宗為之嗟嘆,遂補原官。八年,擢工部侍郎,典浙江鄉試,留學政。十一年,丁本生父憂,哀毀,扶病奔喪,卒於衢州,以衰服殮,年六十三。

宗瀚孝謹恬退,中歲以養親居林下十年,書法尤為世重。

韓鼎晉,字樹屏,四川長壽人。乾隆六十年進士,選庶吉士,授檢討。嘉慶九年,改御史。疏言天主教流傳之害,請申禁以絕根株,從之。以母老請終養,十六年,服闋,補原官。疏陳四川積弊六事,曰:禁科派以安閭閻,除啯匪以防積漸,查卡房以全民命,禁拐騙以警貪頑,嚴攤捐以養廉潔,覈戎政以歸實效。又言京師賭風大熾,多屬王公大臣輿夫設局,倚勢骫法,帝命指實,下詔嚴治。逾日,獲賭案三,大學士、步軍統領祿康輿夫為之魁。親貴近臣,莫不悚息。

巡視山東漕務,轉工科給事中、光祿寺少卿,督陜甘學政。疏言:「榆、綏諸州縣倉貯空虛,宜設法籌補,其地資蒙古糧食接濟。今腹裏邊外俱荒,當分別安置撫恤。」又言:「南山善後事宜,宜行堅壁清野之法。山內流民雜處,最為奸藪,當嚴行保甲,使奸宄無所匿。軍中擄脅難民子女,請嚴禁。南山附近及豫東並經兵燹,宜慎選牧令,以蘇民氣。川北荒歉,與陜、甘毗連,鹽梟啯匪多出其中,請先事豫防。」並下疆吏如所請行。歷鴻臚寺卿、通政司副使、太常寺卿、左副都御史。

二十四年,命察視近畿水災,督黃村賑務。督福建學政,疏言:「閩中吏治久窳,請不限資格,用廉幹吏補汀、漳、泉三郡望緊要缺,久其任以專責成。漳、泉營伍通盜,請責提鎮立予重典,勿稍袒庇。」道光六年,遷倉場侍郎,以病罷。起補工部侍郎,京察,原品休致。卒於家,祀鄉賢祠。

硃方增,字虹舫,浙江海鹽人。嘉慶六年進士,選庶吉士,授編修。典云南鄉試,遷國子監司業。十八年,教匪之變,方增劾直隸總督溫承惠貽誤地方,黜之。

應詔陳言,論用人理財,略曰:「近今大臣中,罕有以進賢為務者。蓋薦舉之事,易於徇私,黨援交結,不得不防,而大臣亦遂引嫌自避。夫大臣避徇私之名,而忘以人事君之責,所謂因噎廢食,非公忠體國者所宜有也。至於任用之方,則無過於考言詢事。皇上博訪周諮,徐為印證。於召對時,各就所長,諭使面陳,果能洞悉原委,又當試之以事,以觀其能踐與否。如或敷奏並無條理,則其人固不足用,而大臣之識見優絀,心地公私,亦可見矣。抑臣思臣工居職,茍非闒茸齷齪者流,孰不思自效?況蒙皇上訓飭至再至三,而猶故習相仍,驟難振拔者,良有數端:條例過繁,文案蒨屑,雖有強敏之吏,而精神疲於具文,其實關於政治民生,轉致不能詳覈。一也。差務絡繹,公私賠累,身家之恤不遑,民物之懷漸恝。二也。訐告之風,至今益甚。嘗有以田土、鬥毆細故而叩閽京控者,有司畏其挾制,不得不姑息委蛇。雖有急公自好者,其尋常蒨屑之事,豈皆一一可達聖聰?甚至匿名揭帖,無主名之可指。蠹吏猾胥,奸民惡僕,求謀不遂,懲治過嚴,皆可造作飛語,訐及陰私。足使任事之心,不寒而慄,委曲隱忍。奸宄橫行,大都由此。三也。今皇上欲整飭因循積習,臣愚以為必先除此三者之弊,庶廓然無所疑畏,而得專精實政矣。經國之方,理財尤要。古者以三十年之通制國用,斟酌盈虛,量入為出,用能經常不匱。今戶部歲入歲出,年一匯奏。惟中外未合為一,條緒繁賾,極難釐剔。且凡撥解即謂之出,並未實計所用。新舊牽溷,凌雜益甚,而出納諸款,又因有無定之款,盈朒參差。以故一歲之中,所出幾何,覈之所入,贏餘若干,不能得其實數。請旨敕下戶部,歲入歲出,宜合中外為一。核計贏餘總數,仍取前一二歲所贏餘,確實比較,然後審其輕重緩急,舉一切例內例外諸用款,有可裁省停緩者,酌加撙節。庶合於古人通年制用之法,而度支充裕矣。」

二十年,入直懋勤殿,編纂石渠寶笈、秘殿珠林。尋督廣西學政,累遷翰林院侍讀學士。道光四年,大考第一,擢內閣學士。典山東鄉試。七年,督江蘇學政。十年,卒。

方增熟諳朝章典故,輯國史名臣事跡,為從政觀法錄,行於世。

論曰:萬承風、周系英、錢樾以侍從之臣,軺車所至,建白卓然。秦瀛之治績,李宗瀚之孝行,非僅以文藻稱。韓鼎晉、硃方增侃侃獻納,言有體要,皆風採著於朝列矣。


\end{pinyinscope}