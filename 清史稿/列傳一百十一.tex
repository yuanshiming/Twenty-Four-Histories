\article{列傳一百十一}

\begin{pinyinscope}
方觀承富明安周元理李湖李瀚李世傑

袁守侗鄭大進劉峨陸燿管幹貞蔣兆奎胡季堂

方觀承,字遐穀,安徽桐城人。祖登嶧,官工部主事。父式濟,康熙四十八年進士,官內閣中書。僑居江寧,坐戴名世南山集獄,並戍黑龍江。觀承尚少,寄食清涼山寺。歲與兄觀永徒步至塞外營養,往來南北,枵腹重趼。數年,祖與父皆沒,益困。然因是具知南北厄塞及民情土俗所宜,厲志勤學,為平郡王福彭所知。雍正十年,福彭以定邊大將軍率師討準噶爾,奏為記室。世宗召入對,賜中書銜。師還,授內閣中書。乾隆二年,充軍機處章京。累遷吏部郎中。七年,授直隸清河道。署總督史貽直奏勘永定河工,上諭之曰:「方觀承不穿鑿而有條理,可與詳酌。」八年,遷按察使。九年,命大學士訥親勘浙江海塘及山東、江南河道,以觀承從。尋擢布政使。十一年,署山東巡撫。十二年,回布政使任。十三年,遷浙江巡撫。十四年,擢直隸總督,兼理河道。十五年,加太子少保。二十年,加太子太保,署陜甘總督。二十一年,回直隸任。

觀承撫山東時,議以安山湖畀民承墾升科,奏言:「湖中尚有積水,但二麥布種於水已涸之後,收穫於水未發之先。故雖有水患,民原承墾升科。升科後,官徵民納,例重秋收。秋禾被水,請蠲、請賑、請豁,徒致紛繁。即如南旺湖,亦經臺臣條奏畀民承墾。臣從訥親履勘,見卑處水涸,高處如屋如巖,意謂水不能及。臣至山東,方知夏秋間運河及汶水暴漲,賴以分減,運道得保無虞。凡大川所經,眾水所注,其宣洩瀦蓄之區,恆閱數年、數十年,有若閒置,一旦實得其用,未可以目前忘久遠。安山湖亦運河洩水地,應視南旺湖例,夏麥秋禾,分季收租。除去升科名目,應徵、應免,悉從其宜。國利而民亦不病。」又奏:「義倉與社倉同為積貯,但社倉例惟借種,義倉則借與賑兼行,而尤重在賑。設倉宜在鄉不宜在城,積穀宜在民不宜在官。秋穫告豐,勸導輸納,歲終將穀數奏明,不必開具管收除在。則其數不在官,法可行久。」

撫浙江,海塘引河出中小亹安流,北大亹沙漲成陸。觀承履勘,丈出地三十五萬餘畝,畀民承墾。又以引河既出中小亹,民間失地,以附近村地二萬餘畝撥補。復察各地咸氣未除,民不能即耕,令灶戶以未種地交民承佃,使灶戶得租,貧民得地。分疏以聞,上嘉之。

督直隸二十年,治績彰顯。以兼理河道,治水尤著勞勩。直隸五大河,永定河渾流最難治。觀承初上官,即疏言:「永定河自六工以下,河形高仰,請就舊有北大堤改移下口,庶水行地中,暢下無阻。」上諭以「改移下口不可輕言」。明年春,上臨視永定河堤,禦制詩示觀承,大指謂河堤但可培厚,不可加高;略移下口,取易於趨下,亦補偏救弊之策。是夏,永定河南岸三工汙溝奪溜。上以江南河道總督高斌豆瓣集漫口圖示觀承,觀承奏:「豆瓣集為中河餘水漫溢,故可於水緩處施工。永定河若但堵月堤,溢水無歸路。仍塞漫口,偪溜入引河,復故道。」上韙之。又明年春,疏言:「永定河下口掣溜出冰窖壩口。請即於坦坡墊尾東北斜穿三角澱,開引河入葉澱,自鳳河轉入大清河。」廷議以時初過凌汛,慮盛漲浹沙淤澱,令觀承覆奏。奏言:「冰窖壩口掣溜,在上七工尾,低於正河丈二三尺。南距南坦坡,北距北大堤,有漫衍而無沖溢,此地勢之順也。水由壩出,非沖決亦非開放,民情不怨,此人事之順也。凌汛改移,經理有暇,此天時之順也。今日必應改移,不復稍存歧見。至慮盛漲梜沙淤澱,渾水至三十里外,水渙沙停,當無此慮。且臣亦嘗計及,故不使東循龍尾直入鳳河,而引入葉澱,迂其途而廣其地,更可經久無患。」上命尚書舒赫德、河東總督顧琮會勘,如觀承議。自是永定河下口出冰窖。

居二年,復疏言:「永定河下口漸淤。請於北岸六工尾開堤放水,至五道口,導歸沙家澱,仍自鳳河入大清河。」廷議以甫改冰窖下口,何以又請於北岸六工開堤放水,令觀承覆奏。奏言:「冰窖改口後,水勢暢順。上年盛漲,下口十里內淤阻。今請於北岸六工放水,循南墊而行,仍以鳳河為尾閭,實於現在情形為便。」自是永定河下口又改自北岸六工入鳳河。旋請以鳳河東堤及韓家墊隸永定河道,又請於下口北墊外更作遙墊,為勻沙散水之用,並加築鳳河東是,與遙墊相接。觀承治永定河凡再改下口,相時決機,從之輒利。

河決長垣、東明,命觀承往勘。疏言:「二縣以太行堤為衛,其地南高北下。河南陽武諸縣水北注,賴此堤捍之。康熙六十年後,屢被沖決。請於堤西開新引河,導水入舊引河東注,即以所起土別築新堤。」命如所議。觀承疏請治子牙河,自楊家口至閻兒莊,改支河為正河。復於閻兒莊北循堤濬新引河接黑港舊引河,俱於子牙橋北入正河。疏請治滹沱河,自晉州張岔山口改流,南出寧晉入滏陽河,當順新道。疏請治漳河,自臨漳東南改流趨大名,分支:一出城北,一流入河間。當於河口築壩,斷水南流。疏淤濬河,引水歸故道。皆如議行。又疏濬易州安國河,開渠灌田,賜名曰安河。上以河南巡撫胡寶瑔督民間繕治道路溝洫,令觀承仿行。觀承方令諸州縣以工代賑,修堤墊,濬減河,築疊道,凡三十二州縣。既奉命,奏言:「正定、順德、廣平、大名等地民力易集,近年漳、漆、滏、洺諸水疏通。他處亦先後開工。要使瀝水有歸,農田杜患。」逾年,疏報自大興、宛平東至撫寧,西至易、涿,西南至望都,東南至阜城;復循運河自武清至吳橋,凡二十二州縣,築疊道,開溝渠,諸工皆竟。

直隸北境東自熱河,西至宣化,皆接蒙古界,流民出塞耕蒙古地。永定河改道冰窖之歲,土默特貝子哈木噶巴牙斯呼郎圖議驅民收地。觀承疏言:「貧民無家可歸,即甘受驅逐,而數萬男婦,內地亦難於安置,請簡大臣按治。」上遣侍郎劉綸等往勘,議仍用原定年限,語詳綸傳。是歲,理籓院尚書納延泰議撤多倫諾爾鋪司,毋占蒙古游牧。觀承奏:「多倫諾爾自設鋪司,文移資送郵,解餉得棲止,行旅亦堪投宿,並無礙於游牧。今於南茶棚、上渡、轉山子、水泉子諸地量留屋宇,如或藏匿匪類,責所司究治。」

觀承復請熱河編立煙戶,令有司稽察。附近敖漢、柰曼、翁牛特、土默特諸部,副都統歲周巡。理籓院議商人領票赴恰克圖、庫倫貿易,不得往喀爾喀各旗私與為市,並禁張家口設肆。觀承疏言:「禁張家口設肆,商人赴恰克圖、庫倫者日少。內地資蒙古馬羊皮革,蒙古亦需內地茶布,有無不能相通,未見其益。請令商人領票赴恰克圖、庫倫,仍許經過喀爾喀各旗相為交易,但不得久居放債,礙蒙古生計。」御史七十五請於多倫諾爾收稅,觀承奏:「內地茶布自張家口往,毋庸重徵。惟恰克圖、庫倫等地互市,及克什克騰木植,當於多倫諾爾徵稅。」

右衛兵移駐張家口,觀承疏言:「歲支米粟不敷一萬四千餘石。請以宣化、懷來、懷安、蔚、西寧五州縣徵豆改粟米出糴,至張家口糴米,可得八千餘石。又以領催、前鋒、馬兵歲米五之一改折加給,俾兵食有資,而轉輸可省。」兵部議以張家口副將隸察哈爾都統,觀承疏請將邊外七汛隸都統,左衛、懷安仍隸宣化鎮。

漕船自清江至通州,天津為南北運河樞鍵。二十二年,漕船遲至,上命觀承督民船起剝。觀承於北倉設席囤貯米,令交兌船泊北倉南,起剝船泊北倉北,皆傍東岸。一幫限二里,同時起米不相妨。西岸行空船,計日畢事。疏請發庫帑給腳價,明歲新漕歸款。二十四年,上以北運河水淺,截先到漕艘留米四十萬石貯北倉。觀承疏言:「前幫截留,後幫繼進,為日無多。請以剝為截,令先到各幫每船剝若干,使得輕便,餘米仍抵通州交兌。應截五六百船全米,勻為千船半米。俟河水漲發,繼進之船,浮送無阻。」諭獎其妥協。上以各省錢貴,用山東布政使李渭議,禁富民積錢,家限五十串。觀承奏:「富民積錢,勢不能按戶而察之。與其限所積不能稽所入,請令交易在三十兩以下者許用錢,過是即用銀,違者收以官價。富民積錢,諭令易銀,違者以十之二入官。至尋常出入,應各從其便。」上問:「成效若何?」觀承言:「富戶錢漸出,市值亦平減。」廷議各省糶米,商人往往藉口昂值,下觀承覈議。觀承疏:「請需米省分具款交產米省分,令有司代購。則牙儈不敢抗地方官教令,操縱自如。」疏並下部議行。

觀承督陜、甘,董理儲糈,送駝馬,運糧茶,上敕以妥速為要。方冬,疏言哈密至巴里坤大阪積雪,遣兵剷除,請日加面四兩。在陜、甘四閱月,即返直隸。觀承蒞政精密,畿輔事繁重,乘輿歲臨幸,往來供張。值西征師行,具營幕芻糧,未嘗少乏,軍興而於民無擾。尤勤於民事,嘗請以永定河淤灘,堤內外留十丈,備栽柳取土,餘畀守堤貧民領耕輸租。又請以永定河葦地改藝秋禾,又以麥田牧羊,奏請申禁。又舉木棉事十六則,為圖說以進,上為題詩。溝渠疊道工竟,又請將欒城、柏鄉、內丘、定興、安肅、望都諸縣改築磚城。涿州拒馬河橋圮,令改建石橋。又重建衡水縣西橋,請賜名安濟。政無鉅細,皆殫心力赴之。

二十八年,上命勘天津等處積水,責觀承玩誤,下部議奪官,命寬之。御史吉夢熊、硃續經交章劾觀承,上諭曰:「觀承在直久,存息事寧人之見。前以天津等處積水未消,予以懲儆,而言者動以為歸過之地。直隸事務殷繁,又值災歉,措置不無竭蹶。言易行難,持論者易地以處,恐未必能如觀承之勉力支持也。」三十年,上南巡,賜詩。三十三年,病瘧,遣醫診視。八月,卒,賜祭葬,謚恪敏。禦制懷舊詩,入五督臣中。子維甸,自有傳。

富明安,富察氏,滿洲鑲紅旗人。初授筆帖式。累遷戶部郎中。乾隆十一年,授廣東惠潮嘉道,歷廣東高廉、糧驛,廣西蒼梧諸道,福建、廣西按察使。二十六年,遷江西布政使。請以南昌同知、通判二員定一員為滿缺,專司繙譯清文。上以江西無駐防滿洲兵,不允。二十八年,命往巴里坤辦事。三十二年,廣東巡撫明山劾富明安官糧驛道浮收倉米,奪官,逮京師鞫治。事白,復官,命署山西布政使。三十三年,護巡撫。劾雁平道時廷靄縱僕擾民,坐奪官。

擢山東巡撫。疏言:「高密百脈湖受五龍河、膠河諸水,夏秋常苦泛溢。請濬引河,引膠河北入膠、萊運河,涸出新地得四百餘頃。」上嘉之。太僕寺少卿範宜賓奏請裁減東省閉壩後驛夫工食,富明安疏言:「水驛夫役終歲在驛,閉壩多在十一月,開壩有早至正月者,中間相距兩月餘,而銅、鉛諸船守凍,尚須守護。節省無多,窒礙轉甚,非政體所宜。」從之。

三十五年,疏言:「小清河行章丘、鄒平、長山、新城、高苑、博興、樂安七縣六百餘里。源出章丘,東至新城、高苑間分支,北為支脈溝;又東至博興分支,南為豫備河。至樂安入淄水歸海。比年湖泊淤塞,春夏水漲,民田常被其害。現就樂安境內挑淤培堤,並疏濬南、北支渠,使支幹通流,建瓴而下。博興、樂安可復膏腴。章丘、鄒平、長山、新城、高苑諸縣附近湖泊涸出,有益於民。民咸原出力興工,毋庸動帑。」諭曰:「有利於民,事在應為,但不可滋弊耳。」

三十六年,又奏:「濟寧西北當運河西岸,受上游曹州境內諸水。以運河勢高,不能洩水入運,遂至間段停積。飭濬舊有五渠,使南匯昭陽湖,並同時修治沂水、涑水、墨河、響水諸渠二十餘處,及運河東岸徒駭、馬頰諸河,洩漲水入海。」上以「知勤民之本」嘉之。三十八年,授閩浙總督,調湖廣。三十九年,京山民嚴金龍父子為亂,捕得置諸法。卒,贈太子太保,謚恭恪。

周元理,字秉中,浙江仁和人。乾隆三年舉人。十一年,以知縣揀發直隸,補蠡縣。調清苑。以總督方觀承薦,擢廣東萬州知州,改霸州。以修城未竣,留清苑。會有部胥持偽劄馳傳者,察其奸,詰問具服,事上聞,上才之。調易州,擢宣化知府。母憂歸。上屢出巡幸,畿輔當其沖,宮館、驛傳、車馬、芻牧諸役,主辦非其人,往往為民厲,奏起元理董其事。服闋,補廣平,調天津,又調保定。擢清河道,遷按察使,再遷布政使。三十六年,命從尚書裘曰修、總督楊廷璋勘青縣、滄州減河。用元理議,請撤閘改用滾水壩,並定每歲測量疏濬,從之。旋授山東巡撫。奏:「小清河發源章丘長白山,至樂安溜河門入海。章丘至博興,有滸山、清河諸泊為納水之區。請先將二泊濬深開廣,遇水發時,有所停蓄,然後聽其入河分注歸海。並於每年農隙,疏濬下游各河。」未半載,擢直隸總督。

三十七年,疏言:「直隸雨多河漲,行潦無歸,行旅多滯。民間堤墊沖決,田廬受患。請用以工作賑例,勘修沖途諸州縣疊道,並濬良鄉茨尾雅河,新城、雄縣盧僧河;修新城、清河、雄、任丘、獻諸縣堤墊。」上遣尚書裘曰修按行直隸河工,元理與合疏言:「直隸諸水,千支萬派。總由三汊河為入海之道,全資西岸疊道,置橋穿運,而東匯入海河。出口西岸舊有橋十一,今擬添建橋九,俾無壅遏,上游不至受害。格澱堤自當城以下改為疊道,酌添涵洞,使行水暢順。子牙河下游澄清,不使清河受淤。」詔如所請。雄縣民訴知縣胡錫瑛私鬻倉穀,上遣曰修及侍郎英廉按治得實,論罪。上諭曰:「直隸治賑,周元理奏言有司料理妥實。今有雄縣事,所稱妥實者安在?」下吏議,奪官,命留任。三十八年,加太子少保。

三十九年八月,山東壽張民王倫為亂,破壽張、堂邑、陽穀,犯東昌及臨清,奪糧艘為浮橋,欲渡運河。上以畿南地相接,敕守要害。元理馳至故城,令布政使楊景素、總兵萬朝興、副將瑪爾清阿以兵千二百駐臨清西岸遏其沖。大學士舒赫德率禁旅討賊,賊渡西岸犯我師,瑪爾清阿擊敗之。賊潰復合,又為我師所敗,進奪浮橋。賊退保臨清舊城,元理令朝興督兵助攻,倫自焚死,亂旋定。尋與侍郎兼順天府尹蔣賜棨勘八旗在官荒地,請招佃承墾,八年後起租;沮洳庳下之區,並為開溝洩水;下部議行。四十年,元理年七十,召至京,御書榜賜之。四十一年,與學政羅源漢請熱河增建學校。四十三年,上命改熱河為承德府,令元理壽畫。疏請改設州一縣五,增置官吏如制。並請開附近潘家口汛煤窯。四十四年,坐井陘知縣周尚親勒派累民,民上訴,元理請罪民。上命尚書福隆安按治,責元理袒護,奪官,予三品銜,令修正定隆興寺自贖。尋授左副都御史,仍署直隸總督。四十五年,遷兵部左侍郎,擢工部尚書。四十六年,引疾歸。四十七年,卒。令江蘇布政使致祭。

元理為治舉大體,泛愛兼容。時以有長者行重之,為方觀承所識拔。時同入薦剡者曰李湖,亦有名。

湖,字又川,江西南昌人。乾隆四年進士。初授山東武城知縣,調郯城。累遷直隸通永道,調清河道。遷直隸按察使,再遷江蘇布政使。三十六年,擢貴州巡撫,三十七年,調雲南。四十年,總督彰寶以貪婪得罪,責湖隱忍緘默不先劾奏,奪官,予布政使銜,往四川軍營會辦軍需奏銷。四十三年,授湖南巡撫,四十五年,調廣東。湖敏於當官,在貴州規畫鉛運,在雲南釐剔銅政,均如議行。所至以清嚴為政。其蒞廣東,以廣東夙多盜,番禺沙灣、茭塘近海為盜藪,密言冋姓名、居址及出入徑途,知群盜以七月望歸設祀,飭文武吏圍捕。旬日間誅為首者二百有奇,而釋其脅從,盜風以息。旋條奏申明員弁,責成編船移汛,設施甚備,令行法立,民咸頌之。卒,贈尚書銜,謚恭毅,祀賢良祠。

李瀚,字文瀾,漢軍鑲黃旗人。少孤,母苦節食貧,撫以成立。瀚選入咸安宮肄業。雍正十年舉人,充景山官學教習。乾隆十三年,授山東榮城知縣。二十三年,遷膠州知州。在官八年,民頌其惠,築堤曰李堤,立石紀焉。三十一年,擢武定知府。大水,乘小舟勘賑,幾溺,卒竟其事。徒駭河久塞,請發帑濬治,自是連歲無水患。三十四年,擢袞沂曹道。覈防河諸費,歲節以萬計,而是益堅。三十六年,擢江西布政使。奏請停編審,上諭曰:「丁銀既攤入地糧,滋生人丁,遵康熙五十二年聖祖恩旨,永不加賦。各省民穀細數,督撫年終奏報。五年編審,不過沿襲虛文,應永行停止。」護巡撫。戶部用湖南布政使吳虎炳議,禁小錢,並及古錢。瀚奏:「收買小錢二千四百餘斤,古錢僅四十餘斤,前代流傳,銷磨殆盡。應援兩江總督高晉奏準例,聽民間行使。如有私鑄古錢,仍與小錢一例查禁。」從之。又奏言:「時憲書按省刊載太陽出入、晝夜、節氣時刻。今江南分江蘇、安徽,湖廣分湖北、湖南,陜西分甘肅,請添註省名,分晰開載。」如所請行。四十年,授雲南巡撫。行至貴州,道卒。

李世傑,字漢三,貴州黔西人。少倜儻,喜騎射。年二十餘,折節改行。乾隆九年,入貲為江蘇常熟黃泗浦巡檢。知縣李永書引與同堂聽訟,縣人稱其平。總督尹繼善、巡撫莊有恭薦卓異,遷金匱主簿。有恭檄充巡捕官,為入貲以知縣留江蘇。二十二年,除泰州知州。始至,訟未結者四百餘案,晝夜據案視事,不五月報結。巡撫陳宏謀薦堪勝知府。二十七年,擢鎮江知府。上命裁京口駐防漢軍,世傑捐廉集貲,人予餉三月、衣一襲,裁者三千人,皆分畀職役。三十年,擢安徽寧池太廣道。丁父憂,服闋,三十六年,授四川鹽驛道。未幾,擢按察使。

師征金川,總督桂林檄世傑駐打箭爐,督約咱路軍需。木果木之敗,副將軍阿桂全師暫退,軍中餉銀數萬巨錠,募運還,無應者。世傑令曰:「委於賊,寧散於民!」從軍貿易者數萬人,爭取立盡。世傑督隊護其後,密檄關吏,見持餉銀入口者皆令還官,鋌酬以給銀五兩,帑獲全。師復進,鑄砲缺炭,檄世傑營辦。世傑令伐樹劄木城卡衛,掘地為大窯數十,復伐樹而薪焉。不旬月,炭足供鑄。守禦僧格宗發敵伏,俘十六人以還。阿桂以聞,賜孔雀翎。四十年,擢湖北布政使,乃留軍督餉。四十二年,金川平,乃上官。四十四年,擢廣西巡撫。丁母憂。四十六年,命署湖南巡撫,服闋真除。四十七年,調河南。大學士阿桂督塞青龍岡決口,疏引河,上命占用民田當安頓調濟。世傑尋奏請以北岸涸出地畝,劃給南岸占用民田。四十八年,奏引河新築南堤,捐廉種柳,別疏釐定防護新河將吏官制。

遷四川總督。四川自軍興後,徵調賦斂無藝,倉庫如洗。世傑潔己率屬,休養生息,俾漸復舊觀,上嘗舉世傑功風厲諸省。世傑疏劾酉陽知州吳申,州民入湖廣界為盜,不即捕治。上諭曰:「四川盜匪,前此大加懲創,地方安靜,乃復有焚殺搶劫之事,皆世傑因循玩愒所釀成。」傳旨申飭。甘肅回復亂,世傑奏遣川北總兵富祿率兵赴援,建昌總兵魁麟防昭化、廣元。上以回亂漸定,諭世傑鎮靜。

五十年,世傑年七十,入覲,與千叟宴。州縣捕金川逃兵不力,例奪官,仍留任,準調不準升。世傑奏請準令捐復,上嚴斥之,下吏議。旋又允陜西巡撫何裕城請,命世傑免議。湖廣饑,告糴於四川,世傑請以近水次諸州縣常平倉穀碾米三十萬石。既,浙江亦告糶,世傑以浙江視湖廣遠,運米濟賑,緩且不及;又請以備應湖廣糴米,撥十萬石先濟浙江。上嘉世傑得封疆大臣體,命議敘。

五十一年,調江南總督。世傑遘疾,乞解任,上不許。秋大雨,河決司家莊。偕安徽巡撫書麟、河道總督李奉翰籌工費,請開捐例。上諭之曰:「戶部庫銀尚存七千餘萬,帑藏充盈,足敷供億。世傑何必為此鰓鰓言利之舉?捐納未嘗無人才,而庸流因之並進博膴仕。一二年後,得廉俸過於所出,國家並無實際,銓政官方,兩無裨益。此奏不可行。」尋復命大學士阿桂蒞工,及冬,工乃竟。五十二年,狼山鎮陳傑疏言各營火藥短少,上命察覈。世傑奏:「鎮屬鹽城等五營硝磺缺額,磺產山西,例二年一次採運。近因運使歲需煙盒,磺銀催解不前,不能如例,以致支絀。」上諭曰:「硝磺軍火要需,向俱採辦足額。以兩江而論,安徽據奏足額,何獨江蘇短缺?兩淮年例,歲不過煙盒七架、大小爆竹一萬,所需能幾?有司採運遲延,以此卸罪。世傑以此率涉支飾,令兩淮鹽政徵瑞會同料理。」世傑尋劾江寧布政使袁鑒於各屬磺價尚未解齊,誤將運使煙盒價牽敘,下吏議。又以河督題報葦蕩營新淤灘地產柴數與案不符,責世傑未察覈;世傑復偕徵瑞奏言硝磺缺額,由採運稽遲,請將歷任布政使議處。上諭曰:「世傑等本當治罪,但以事涉上供,從寬降鑒江寧知府,停世傑養廉三年。」並罷兩淮例進煙盒、爆竹。

復調四川總督。五十三年,巴勒布夷為亂,據西藏屬聶拉木、濟嚨。上命世傑撥駐防綠營及明正、巴塘、里塘、德爾革爾諸土司兵赴西藏;而世傑得駐藏大臣慶林牒,已發駐防綠營兵及屯練降番合三千人,令提督成德等率以行。奏入,上命毋發明正、巴塘、里塘、德爾革爾諸土司兵。世傑奏:「奏諭已令諸土司發兵,諸土司近尚安靜。既調復停,恐番性生疑,仍令備調。」上嘉世傑相機妥辦,不拘泥遵旨,解御佩大小荷包賜之。世傑又奏發米萬三千三百石運西藏,足敷兵食。上褒世傑盡心,命移駐打箭爐。迭疏報成都將軍鄂輝率兵千二百入藏,副將那蘇圖率屯練五百駐打箭爐。尋以巴勒布夷遠遁,諭世傑還成都。五十四年,秋審,四川原定緩決、刑部改情實者凡七案。上責世傑寬縱,以其老,且平日治事覈實,免議。世傑薦川北道明安,引見,上以其年衰,改主事,世傑下吏議。世傑以病請解任,上令侍衛慶成偕醫診視,賜人葠,並令自審病輕則來京,重則回籍。五十五年三月,入覲,授兵部尚書,賜紫禁城乘肩輿。江蘇句容吏侵蝕錢糧漕米,上責世傑在兩江未覺察,命以原品休致回籍。五十九年,卒,年七十九,賜祭葬,謚恭勤。

世傑仕而後學,摘發鉤距,必得要領。上每言其不通文理,然屢褒其能事,禮遇優厚。世傑長子漳州知府華國早卒,上降詔慰勉。其孫舉人再瀛,會試未中式,令一體殿試,授禮部主事。及世傑入為尚書,再瀛病卒,召其次子知州華封授員外郎,俾奉侍。華封官至兩廣鹽運使。

袁守侗,字執沖,山東長山人。乾隆九年舉人,入貲授內閣中書,充軍機處章京。遷侍讀。再遷吏部郎中。考選江西道御史,授浙江鹽驛道。二十八年,遷廣西按察使。奏言:「煙瘴充軍人皆兇悍,請分撥泗城、鎮安、寧明、東蘭諸地;解役疏脫斬絞重囚,短解問徒,長解問流;各署書役貼寫幫差,濫收滋弊,請量定多寡,分別汰留。」又言:「卓異官,籓、臬、道、府甫到任未三月,停止出結。」部議均從之。三十四年,丁父憂,服闋,命以三品京堂仍充軍機章京,補太僕寺卿。遷吏部侍郎,調刑部。命如雲南按布政使錢度貪婪狀,論如律。三十八年,兼署禮部,命在軍機大臣上學習行走,兼管順天府尹。復命如雲南按保山知縣王錫供給總督彰寶虧空兵糧,論如律。調吏部。又命如貴州按總督圖思德劾鎮遠知府蘇墧貪婪狀,罪至死。暫署貴州巡撫。又如四川按松岡站員冀谷勛侵蝕軍米,論如律。四十一年,遷戶部尚書。復命如四川按富德濫用犒軍銀,即監詣京師,賜黑狐端罩。

四十二年,調刑部。命如甘肅勘驗捐收監糧。復命偕兩江總督高晉籌堵儀封漫口。四十四年,奏言遵兜袖法築兩壩,以期回溜分入引河。又與高晉會奏引河頭去口門稍遠,開引溝三百餘丈,直達引河,繪圖奏聞。上以所擬引河向南,恐紆回不能得勢,於圖內硃筆標識,令向北改直。尋奏壩工蟄陷,兩壩鑲築兜收。遵諭將引河頭西首淤灘切去,俾溝口向西北,開寬,引溜下注。是年四月,授河東河道總督。調直隸總督。四十五年,疏請修築北運河筐兒港減水石壩。四十六年,甘肅監糧舞弊成大獄,上以守侗勘驗不實,下吏議,奪官,命留任。丁母憂,去官。

四十七年,諭勘浚伊家河,疏山東積水。守侗詣勘,奏請自善橋以北抵楊家樓,長七千餘丈,展寬浚深,堵築缺口,拆改礙水橋座,諭速行辦理。尋復授直隸總督。四十八年,卒,贈太子太保,賜祭葬,謚清愨。

鄭大進,字退谷,廣東揭陽人。乾隆元年進士。授直隸肥鄉知縣。累遷山東濟東道。二十九年,山東淫雨,高唐、茌平諸縣水漲阻道。大進相度宣洩,水不為患。巡撫崔應階薦其能,遷兩淮鹽運使。三十六年,丁父憂,去官。服除,上召至熱河,命署浙江按察使。尋授湖南按察使。四十年,遷貴州布政使。四十三年,授河南巡撫。四十四年,調湖北。旋署湖廣總督。奏:「安陸、荊州二府濱臨江、漢,以堤為衛。今夏漲發,鍾祥、潛江、荊門、江陵堤決,已一律修復,惟潛江長一垸地窪沙積,築堤難固,應擇地勢較高處築月堤。鍾祥、永興、保安諸垸地當沖,亦應築月堤,俾水發江寬,不致出險。又有劉家巷是應並修築。」四十五年,奏:「武昌濱江上游,諸水匯流,繞城而東。江漲沖刷,堤根虛懸。現修武昌城畢,請並修堤,毋使水齧城。」均從之。又奏言:「湖廣邪教為害,總督班第奏請枷責發落,俾免株連。牧令遂視為自理詞訟,率不通詳。請自今以後,據實呈院司覈辦,諱匿徇縱者劾之。」上韙其言。

四十六年,授直隸總督。命勘永定河工。奏言:「六工以下河身內舊有民居,乾隆十五年給價遷移。又以下口改流,奏令暫回繳原給房價,減糧田畝,依舊徵收。今勘南、北兩岸,自頭工至六工,村落已盡遷移。六工以下,水勢遷徙靡常,累將北墊改築展寬。南、北兩堤遙隔五十餘里,其中居民五十餘村,水漲以船為家,應令遷移。永清柳坨諸村、東安孫家坨諸村旗、民二百八戶,已勘定地址,令陸續移居。河身較遠之村,仍準暫住。禁築壩修房,以杜占居。」報聞。四十七年二月,賜孔雀翎、黃馬褂。五月,奏保定九龍河經清苑、安州至任丘入澱,年久積淤。請舊有望都鄉閘、殷家營、高嶺村三閘外,於望都樊村建石閘一,清苑冉村、鄧村、營頭建石閘三。並修整諸舊閘,開濬安州、新安、任丘諸縣河。皆稱旨,加太子少傅。卒,賜祭葬,謚勤恪。

劉峨,字先資,山東單縣人。入貲授知縣。乾隆二十三年,選直隸曲陽知縣。調宛平。盧溝橋有逆旅,多陰戕過客沒其財,峨發其奸。西山煤礦多藏匿亡命,峨散其黨與,先後捕治置諸法。三遷通永道,以母憂歸。起天津道,仍調通永道,以父憂歸。未一年,上命署清河道,服闋真除。四十五年,遷湖北按察使。石首有寡婦,兄公謀其產,誣之,死於獄。峨治官書發其枉,逮其兄公至,親鞫,論如律。四十六年,遷安徽布政使,調山西。四十八年,擢廣西巡撫。甫兩月,遷直隸總督。輔國公弘晸遣僕至靜海冒占入官地,事聞,上諭峨:「遇王公以下私遣人乾有司,無問是非曲直,即據實奏聞。」長蘆鹽政徵瑞奏漕艘至楊村,以民船剝運,鹽運遲誤。上謂非特鹽運遲誤,且恐商貨壅滯,令峨赴天津與徵瑞議民船編號輪雇,照例發價,並定赴通回空限期,下部議行。分疏劾中倉監督趙元搢嗾毆民至死,三河知縣王治岐挪用旗租,並論如律。謁避暑山莊祝嘏,賜孔雀翎、黃馬褂。南宮民魏玉凱訴縣人李存仁習邪教,上遣侍郎姜晟會鞫。存仁坐誅,玉凱妄及無辜,論戍。四十九年,上遣尚書金簡會勘盧溝橋下游沙淤,請於中泓五孔抽溝三道。上以抽溝水緩,命中泓五孔全行疏濬。徵瑞請捐銀三十萬造剝船濟運,上以直隸木材少,命湖廣、江西二省分造。峨奏言:「北倉存漕四十餘萬,俟新造剝船刑齊,先行運通。」上許之。

五十一年七月,廣平民段文經、元城民徐克展為亂,夜入大名,戕大名道熊恩紱。峨奏聞,即督兵馳往捕治,得從亂者王國桂等,自列向習八卦教,及文經、克展蓄謀為亂狀。上令峨捕文經、克展,久之未獲,累降旨詰責。十月,河南巡撫畢沅奏於亳州獲克展,檻送京師,而文經終未能得。五十二年,命停峨本年廉俸。山東學政劉權之迎眷屬赴官,途遇盜,峨坐奪官,命留任。

五十三年,命偕山東巡撫長麟等勘議糧艘在德州剝運。五十五年,巡城御史穆克登額等獲建昌盜,自列嘗劫建昌錢鋪,有同為盜者,系清苑獄二年未決。上責峨廢弛,遣侍衛慶成逮清苑知縣米復松詣京師,下刑部論罪;奪峨孔雀翎、黃馬褂,降調兵部侍郎。未幾,擢尚書。五十六年,命如河南按虞城民訴縣役事,又如江西按廣豐武弁包漕、崇義民發塚棄骸事,並訊明,論如律。峨至崇義,入深山中勘塚地,江西民稱之。五十七年,從上幸熱河,賜還孔雀翎、黃馬褂。六十年,以疾乞解任,加太子少保,原品休致。卒,賜祭葬,謚恪簡。

陸耀,字青來,江南吳江人。乾隆十七年舉人。十九年,考授內閣中書,充軍機處章京。奉職勤慎,有急務立辦,大學士傅恆深器之。上出巡幸,俱令扈從。累遷戶部郎中。三十五年,出為雲南大理知府,以親老請改補近省,調山東登州府。三十六年,調濟南府。上書巡撫徐績,請留南漕廣積貯。三十七年,授甘肅西寧道。燿乞績代奏,乞假送母居京師,上命改授運河道。上書河道總督姚立德,言:「兗州、泰安二府泉四百七十八,當濬渠導泉,俾由高趨下,其流不絕。」又言:「運河例歲冬閉壩,春挑濬,天寒晷短,民役俱憊。宜修復南旺、濟寧、臨清月河,並於彭口南岸亦開月河。歲九、十月漕艘商舶皆從此行,以其時疏濬運河。」皆用其議。又請修河渠志,成運河備考。

三十九年,壽張民王倫為亂,去濟寧二百里,有欲閉城者,燿不可,曰:「寇未至閉城,示之怯也。且何忍拒吾民使散逸被賊害且脅誘耶?」乃募鄉兵助守,坐城闉任稽察,事旋定。四十年,擢按察使,燿議以流犯罪輕,請免其解司;四十三年,擢布政使,燿議流外壅積,請停分發:皆從之。燿母老,病狂疾,奏乞解任終養,上許之。四十六年,丁母憂。運河築堤,上以燿習河務,命往山東會運河道沈啟震董其役。四十八年,命署布政使,服闋真除。

四十九年,擢湖南巡撫。湖南鹽商例有餽,峻卻之,命平鹽價如其數。疏請增嶽麓、城南二書院膏火,又疏請申親老告養例,請敕各督撫不論現任、試用,通飭呈明終養。又奏:「湖南社倉前巡撫劉墉令湘陰等四十五州縣勸捐,得穀十二萬;勒限嚴催,僅耒陽等十五州縣交齊,餘未足數者十七縣,全未交者十三縣。如湘陰、巴陵、武陵諸縣濱臨江湖,地多磽瘠;桂陽、瀘溪、辰谿諸縣介在山僻,民鮮蓋藏;若執前捐數目,責令全完,民間未霑借貸之益,轉受追呼之擾。請凡現在未收者停止催繳。」上允其奏。燿以病請解任。旋卒。

燿自幼立志以古人自期,學兼體用。居官廉儉。入覲,門吏留裝物索貲;燿乃置衣被城外而假於友,覲已還之。初至長沙,總督特升額以閱兵至,見翟方午食,惟菽乳蔬蓏,訝之。燿曰:「天不雨,方齋,故所食止此。」特升額怒其奴曰:「吾館舍酒肉臭,何不以祈雨告?」還館舍,命悉撤去。

管幹貞,字松崖,江南陽湖人。乾隆三十一年進士,改庶吉士,授編修。考選貴州道御史。巡視西城,訟牒皆親判;周行郊內外,捕治諸不法者。先後命巡漕天津、瓜、儀,凡十二年。累遷至光祿寺卿。幹貞以漕船回空,多守凍打冰,令先通下游,免上游冰下注,益增堅厚,後遂守其法。疏言:「運河以諸湖為水櫃,誠使節節疏通,雖遇旱澇,可以節宣。否則雨少無籌濟之方,雨多無容水之地。至引黃入運,系一時權宜。茍疏濬得宜,黃河全力下注,運河自不致停沙。」又奏請治駱馬湖,使運河水有所蓄洩,並得旨議行。遷內閣學士。五十三年,擢工部侍郎。

五十四年,授漕運總督。糧艘至天津楊村,每以水淺須起撥,運丁不能給舟值,例由長蘆鹽運使以鬻鹽錢貸運丁,借直隸籓庫銀歸款,運丁分年繳納。其後議停,運丁多不便,幹貞請如舊例。又疏陳江西軍丁疲敝,請籌款增補,行、月二糧折價;借官銀代償積逋,令分年輸納;寬限清釐屯田,俾藉以調劑。並從之。五十五年,賜孔雀翎、黃馬褂。疏言;「漕艘百餘幫,役夫數萬人,最易藏奸生事。上年新漕,飭嚴立規條,行必按伍,止則支更。親行督察,乃知別有奸人隨運潛行。督飭捕治數十人,交州縣確擬嚴懲。」得旨嘉獎。五十八年,疏言:「蘇州太倉押運官,例抵淮後改委赴通。中途分更,互相推諉。請自水次抵通,始終其事,庶官有專司。」又請河南豁免緩徵,停運減存船隻,就近赴山東受雇撥運。又請各幫水手短纖,責成頭舵工丁以素識誠實之人充補,免聚眾竊盜諸累。皆報可。各省開兌,多至春初,又在在逗遛,遇水淺或河溢,有在河北度歲者。幹貞嚴飭弁丁修艌受兌,復冬兌春開舊制。糧艘起運,每策馬督催,風雨不避。或不歸所乘舟,支帳露宿。微弁出力,必親慰勞。運丁舟人不用命,立予懲罰。當時或苦其苛急,及回空省費,無絲毫派累,咸大悅服。高宗嘗召見褒其能,謂可亞楊錫紱。五十九年,以疾乞假,命兩江總督書麟攝其事。疾愈,任事如故。

幹貞成進士時,禮部改「貞」為「珍」,六十年,命仍原名。嘉慶元年,戶部議江、浙白糧全運京倉,以羨米為耗,浙江運丁如議交運。幹貞以江南餘米較少,執議不行,交部嚴議,奪官。三年,卒。子遹群,官浙江巡撫。

蔣兆奎,字聚五,陜西渭南人。自副貢生補甘肅張掖縣教諭。乾隆三十一年,成進士。三十三年,教諭俸滿,授四川合江知縣。調灌縣,丁憂。師征小金川,攻熱耳,總督富勒渾奏留兆奎從軍,駐達烏圍治餉。既破熱耳,移餉往。俄,大金川助亂,兆奎知熱耳不足守,復移糧達烏圍。已而,他所糧悉被焚。將軍阿桂才兆奎,使駐日隆治餉,兼司令砲局。旋調署華陽,加知州銜。四川盜號啯嚕子,擾尤溪。兆奎捕得盜渠,獲首犯。服闋,遷山西澤州同知。擢太原知府。以巡撫農起薦,擢河東鹽運使。五十四年,遷按察使,仍兼理鹽務。尋遷甘肅布政使。五十六年,高宗八旬萬壽,兆奎入祝嘏。時河東商困,兆奎議改鹽課歸地丁,上命如山西同巡撫馮光熊勘議。旋議山西、陜西、河南三省應納正雜課四十八萬餘兩,均入三省行鹽完課納稅百七十二州縣地丁,兩加九分有奇,下部議行。五十七年,上以河東鹽價減,銷暢,兩三月內,發販鹽數倍於往年,商民交便。褒兆奎始終承辦,收效甚速,賜孔雀翎。

旋授山西巡撫。五十九年,迎蹕,賜黃馬褂。六十年,以山西錢賤,請停寶晉局鑄錢,從之。嘉慶元年,詔與千叟宴。尋命毋詣京師,仍加恩賚。奏劾汾州知府張力行挾訟事婪索,冀寧道鄧希曾等回護同官。奪力行官,命兆奎授鞫。又發力行侵帑狀,坐斬。二年,以病乞解任,歸。

四年,高宗崩,兆奎入臨,即授漕運總督。固辭,不許。旋奏言:「整頓漕運,要在恤丁。今陋規盡革,旗丁自可節費;而生齒日繁,諸物昂貴,旗丁應得之項,實不敷用,急須調劑。前讀上諭:『有漕州縣,無不浮收,江、浙尤甚,每石加至七八斗。』歷來交納,視為固然。今若劃出一斗津貼旗丁,餘悉革除。所出有限,所省已多。不特千萬旗丁藉資濟運,即交糧億萬花戶皆沾恩無窮。」疏入,上嫌事近加賦,飭與有漕省分各督撫另議調劑。兆奎疏言:「各督撫所議調劑,有名無實。兩江費淳所奏,不敷運費;江蘇擬四升七合,安徽擬二升,焉能有濟?」因力請罷斥。上責兆奎粗率,並諭:「加賦斷不可行。此外如何設策善後,令再覈議。」兆奎奏請:「每船借給銀百兩,於各糧道庫支領,分三年,以旗丁應領之項扣還。山東、河南兩省路途較近,減借五十兩;有漕各省本有輕齎,原應徵米,斗折銀五分。請仍徵本色,按照旗丁米數,分給白糧。無輕齎,請通融勻給。」上以「所擬損民益丁,巧避加賦之名,仍存加賦之實」,遣侍郎鐵保會淳詳察。兆奎又奏:「旗丁運費本有應得之項,惟定在數十百年之前。今物價數倍,費用不敷。近年旗丁尚可支持者,以州縣浮收,向索兌費,並折收行月等米,以之貼補一切經費。今革除漕弊,浮費可省,兌費不能減。臣才識短淺,惟恐貽誤,求上別簡賢員,原從小心敬畏而來,不敵氣質用事。」上即命鐵保代兆奎,召授工部侍郎。

尋授山東巡撫。御前侍衛明安泰山進香,還京師,奏山東有司私餽銀八百,並及途中營汛墩房坍塌。上以詰兆奎,兆奎復奏辯,且稱老病,求去。上怒其忿激,念廉名素著,降三品卿銜休致。七年,卒。

胡季堂,河南光山人,侍郎煦子。初以廕生授順天府通判,改刑部員外郎,遷郎中。出為甘肅慶陽知府,再遷甘肅按察使,調江蘇。江蘇按察使移駐蘇州,而獄猶在江寧,季堂請更置,報可。乾隆三十九年,擢刑部侍郎,四十四年,遷尚書。季堂屢奉使諸省讞獄,直隸、吉林、江蘇皆一至,山東四至,河南再至。察得唆訟者嚴治之;有誣訴,論如律,不稍貸。初使河南按商丘獄,上諭之曰:「季堂河南人,按本省事尤當秉公持正。勿以事涉大吏,慮將來報復,稍為瞻顧。」商丘民湯秉五迫孀婦劉為妻,劉絕食死。其獄已題旌,劉父猶陳訴,並及順刀神拳會民事,察得唆訟者罪之。使山東按平度獄,州民羅有良與人鬥,誤蹴其母死。萊州知府徐大榕原勘無誤,乃坐是奪官,當平反,得旨嘉獎。再使山東,暫署巡撫。山東災,請截本省漕米治賑。還京師,加太子少保,再兼署兵部尚書。

嘉慶三年,授直隸總督,賜孔雀翎。四年,仁宗親政,季堂疏發和珅罪狀。尋請以籍沒其僕呼什圖米麥萬餘石,分借文安、大城被水村民。長新店盜發,上責季堂廢弛,削太子太保,奪孔雀翎。下吏部議,奪官,去頂帶留任。河南內黃知縣陶象柄獲長新店首盜,季堂奏聞。上嘉季堂不邀功,還頂帶;又獲從犯,還孔雀翎。是時川、楚、陜教匪為亂,五年,季堂奏:「教匪稽誅,臣聞經略額勒登保、參贊德楞泰等由川而楚而陜而甘,數千百里窮追,接戰輒勝。是教匪所恃,不在勢眾而在得間能逃也。川、楚、陜連界,崇山峻嶺,斷澗深溝,在在險阻。教匪竄匿其間,劫掠而食,不煩裹糧;迫民前驅,不煩招集。官兵至,輒翻山越澗而逃。官兵必先運糧,又須探路,諸費周章。即道路可通,餱糧可繼,而日夜追躡奔走,其勢必疲。是教匪逸而兵勞也。臣愚以為當先嚴守要隘,俾教匪無路可奔;乃宣上德意,散其脅從,然後臨之以兵,分道進剿。教匪途窮食盡,計日可平。聞陜省有團練鄉勇,或一二村,或數村,聯合築堡為聲援。川、楚可推而行之,令各守本境,俾自護其田廬婦子。則教匪雖多,驟難肆擾。官兵剿撫兼施,無顧此失彼之慮。」上諭曰:「所論極是。總之能堵方能剿,能剿方能撫,大端不外乎此。」

尋以病乞解任,還太子太保。卒,贈太子太傅,遣御前侍衛豐伸濟倫奠醊,謚莊敏。子鈺,進士,直隸清河道;鏻,湖南鹽法道。

論曰:牧民於平世,自庶而求富,修水利,飭農功,其先務也。觀承殫心力於是,政行畿甸。富明安、元理、瀚皆以此為急,各著績效。幹貞籌運道,尤重行水。世傑起下僚,介而能恕。燿以學為政,所施未盡其蘊。季堂論治教匪,後來堅壁清野之議,已發其端。我有先正,言明且清,諸臣所論列,足當之矣。


\end{pinyinscope}