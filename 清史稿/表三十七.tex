\article{表三十七}

\begin{pinyinscope}
疆臣年表一(各省總督河督漕督附)

一國治亂,君相尸之。一方治亂,岳伯尸之。清制:疆帥之重,幾埒宰輔。選材特慎,部院莫儗,蓋以此也。開國而後,戡籓拓邊,率資其用。同治中興,光緒還都,皆非疆帥無與成功。宣統改元,始削其權,則不國矣。唐之方鎮,元之行省,史不表人,識者病之。今表疆臣,先列督、撫,附以河、漕。東三省外,北盡蒙、疆,西極回、藏,將軍、都統,參贊、辦事大臣有專地者,皆如疆帥,今並著焉。


\end{pinyinscope}