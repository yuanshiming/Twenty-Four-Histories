\article{列傳二百六十}

\begin{pinyinscope}
張勛康有為

張勛,字少軒,江西奉新人。少孤貧。投效廣西軍,預法越之戰,累保至參將。日韓釁啟,隨毅軍防守奉天。袁世凱練兵小站,充管帶。拳匪亂作,統巡防營防剿,敘功擢副將,賞壯勇巴圖魯。兩宮回鑾,隨扈至京,諭留宿衛,授建昌鎮總兵,擢雲南提督,改甘肅,皆不赴。日俄戰後,調奉天,充行營翼長,節制三省防軍,賞黃馬褂。旋命總統江防各軍,駐浦口,調江南提督。

武昌變起,蘇州獨立,總督張人駿、將軍鐵良方與眾籌戰守,有持異議者,勛直斥之。翌日,新軍變,勛與戰於雨花臺,大破之。江、浙軍合來攻,糧援胥絕,乃轉戰,退而屯徐州,完所部。人駿、鐵良走上海。命勛為江蘇巡撫,攝兩江總督,賞輕車都尉。遜位詔下,世凱遣使勞問,勛答曰:「袁公之知不可負,君臣之義不能忘。袁公不負朝廷,勛安敢負袁公?」世凱歷假勛定武上將軍、江北鎮撫使、長江巡閱使、江蘇都督、安徽督軍。及建號,勛首起抗阻,並請優待皇室,保衛宮廷。

世凱卒,各省有所謀,群集徐州,推勛主盟。勛於是提兵北上,叩謁宮門,遂復闢。連下詔令,首頒復政諭云:「朕不幸以沖齡繼承大業,煢煢在疚,未堪多難。辛亥變起,我孝定景皇后至德深仁,不忍生民塗炭,毅然以祖宗創垂之重,億兆生靈之命,付託前閣臣袁世凱,設臨時政府,推讓政權,公諸天下,冀以息爭弭亂,民得安居。乃國體自改共和以來,紛爭無已,迭起干戈,強劫暴斂,賄賂公行,歲入增至四萬萬而仍患不足,外債增出十餘萬萬而有加無已。海內囂然,喪其樂生之氣,使我孝定景皇后不得已遜政恤民之舉,轉以重苦吾民。此誠我孝定景皇后初衷所不及料,在天之靈,惻痛難安;而朕深居宮禁,日夜禱天,徬徨飲泣,不知所出者也。今者復以黨爭激成兵禍,天下洶洶,久莫能定,共和解體,補救已窮。據張勛等以『國本動搖,人心思舊,合詞奏請復闢,以拯生靈』,各等語。覽奏,情詞懇切,實深痛懼。既不敢以天下存亡之大責,遂輕任於眇躬;又不忍以一姓禍福之■L8言,置兆民於不顧。權衡重輕,天人交迫,不得已允如所奏,於宣統九年五月十三日臨朝聽政,收回大權,與民更始。自今以往,以綱常名教為精神之憲法,以禮義廉恥收潰決之人心。上下以至誠相感,不徒恃法守為維系之資;政令以懲毖為心,不得以國本為嘗試之具。況當此萬象虛耗,元氣垂竭,存亡絕續之交,朕臨深履薄,固不敢有樂為君,稍有縱逸。爾大小臣工,尤當精白乃心,滌除舊染,息息以民瘼為念。為民生留一分元氣,即為國家延一息命脈,庶幾危亡可救,感召天庥。所有興復初政亟應興革諸大端,條舉如下:一,欽遵德宗景皇帝諭旨,大權統於朝廷,庶政公諸輿論,定為大清帝國君主立憲政體。一,皇室經費,仍照所定每年四百萬元數目,按年撥用,不得絲毫增加。一,懍遵本朝祖制,親貴不得干預政事。一,實行融化滿、漢畛域,所有以前一切滿、蒙官缺已經裁撤者,概不復設。至通婚易俗等事,並著所司條議具奏。一,自宣統九年五月本日以前,凡與東西各國正式簽定條約,及已付債款合同,一律繼續有效。一,民國所行印花稅一項,應即廢止,以紓民困;其餘苛細雜捐,並著各省督撫查明,奏請分別裁撤。一,民國刑律不適國情,應即廢除,暫以宣統初年頒定現行刑律為準。一,禁除黨派惡習,其從前政治罪犯,概予赦免;儻有自棄於民而擾亂治安者,朕不敢赦。一,凡我臣民,無論已否剪發,應遵照宣統三年九月諭旨,悉聽其便。凡此九條,誓共遵守。皇天后土,實鑒臨之!」次諭議立憲,設內閣;京、外官暫照宣統初年官制辦理;其現任文武大小官員,均照常供職。

先後命官以勛及陳寶琛、劉廷琛等為內閣議政大臣,次則內閣閣丞萬繩栻、胡嗣瑗,大學士為瞿鴻禨、升允,顧問大臣趙爾巽、陳夔龍、張英麟、馮煦等,各部尚書梁敦彥、張鎮芳、雷震春、沈曾植、勞乃宣等,侍郎李經邁、李瑞清、陳曾壽、王乃徵、陳毅、顧瑗等,丞參辜湯生、章■J7、黎湛枝、梁用弧等,都御史張曾易又,副都御史胡思敬、溫肅;並召鄭孝胥、吳慶坻、趙啟霖及陳邦瑞、硃益籓等均來京。又以勛兼直隸總督、北洋大臣,仍留京。各省督、撫、提、鎮,皆就現任者改之。命下,各省多不應,而馬廠師起,稱討逆軍,傳檄討勛,勛自請罷斥。及攻都城,勛與戰,以兵寡不支,荷蘭公使以車迎入使館。旋赴津,居久之,卒,年七十,謚忠武。勛亢爽好客,待士卒有恩,所部數萬人,無一斷發者,世指為「辮子軍」。臨戰,盡納家屬妻妾子女別室,不聽避,蓋自懟負國,誓骨肉俱殉。及事亟,外人破戶劫之始脫云。

康有為,字廣廈,號更生,原名祖詒,廣東南海人。光緒二十一年進士,用工部主事。少從硃次琦游,博通經史,好公羊家言,言孔子改制,倡以孔子紀年,尊孔保教,先聚徒講學。入都上萬言書,議變法,給事中餘聯沅劾以惑世誣民,非聖無法,請焚所著書。中日議款,有為集各省公車上書,請拒和、遷都、變法,格不達。復獨上書,由都察院代遞,上覽而善之,命錄存備省覽。再請誓群臣以定國是,開制度局以議新制,別設法律等局以行新政,均下總署議。

二十四年,有為立保國會於京師,尚書李端棻,學士徐致靖、張百熙,給事中高燮曾等,先後疏薦有為才,至是始召對。有為極陳:「四夷交侵,覆亡無日,非維新變舊,不能自強。變法須統籌全局而行之,遍及用人行政。」上嘆曰:「奈掣肘何?」有為曰:「就皇上現有之權,行可變之事,扼要以圖,亦足救國。唯大臣守舊,當廣召小臣,破格擢用;並請下哀痛之詔,收拾人心。」上皆韙之。自辰入,至日昃始退,命在總理衙門章京上行走,特許專擢言事。旋召侍讀楊銳、中書林旭、主事劉光第、知府譚嗣同參預新政。有為連條議以進,於是詔定科舉新章,罷四書文,改試策論,立京師大學堂、譯書局,興農學,獎新書新器,改各省書院為學校,許士民上書言事,諭變法。裁詹事府、通政司,大理、光祿、太僕、鴻臚諸寺,及各省與總督同城之巡撫,河道總督,糧道、鹽道,並議開懋勤殿,定制度,改元易服,南巡遷都。未及行,以抑格言路,首違詔旨,盡奪禮部尚書、侍郎職。舊臣疑懼,群起指責有為,御史文悌復痛劾之。上先命有為督辦官報,復促出京。

上雖親政,遇事仍承太后意旨,久感外侮,思變法圖強,用有為言,三月維新,中外震仰。唯新進驟起,機事不密,遂致害成。時傳將以兵圍頤和園劫太后,人心惶惑。上硃諭銳等籌議調和,有「朕位且不能保」之語,語具銳傳。於是太后復垂簾,盡罷新政。以有為結黨營私,莠言亂政,褫職逮捕。有為先走免,逮其弟廣仁及楊銳等下獄,並處斬。復以有為大逆不道,構煽陰謀,頒硃諭宣示,並籍其家,懸賞購捕。有為已星夜出都航海南下,英國兵艦迎至吳淞。時傳上已幽廢,且被弒,有為草遺言,誓以身殉,將蹈海。英人告以訛傳,有為始脫走,亡命日本,流轉南洋,遍游歐、美各國。所至以尊皇保國相號召,設會辦報,集貲謀再舉,屢遇艱險不少阻。嘗結富有會,起事江漢,皆為官兵破獲,誅其黨。連詔大索,毀所著書,閱其報章者並罪之。初,太后議廢帝,稱病徵醫,久閉瀛臺,旦夕不測。有為聞之,首發其謀,清議爭阻,外人亦起責言,兩江總督劉坤一言「君臣之分已定,中外之口難防」,始罷廢立。拳匪起,以滅洋人、殺新黨為號,太后思用以立威,遂肇大亂,凡與有為往還者,輒以康黨得奇禍。

宣統三年,鄂變作,始開黨禁,戊戌政變獲咎者悉原之,於是有為出亡十餘年矣,始謀歸國。時民軍決行共和,廷議主立憲,而有為創虛君共和之議,以「中國帝制行已數千年,不可驟變,而大清得國最正,歷朝德澤淪浹人心,存帝號以統五族,弭亂息爭,莫順於此」。內閣總理大臣袁世凱徇民軍請,決改共和,遂下遜位之詔。有為知空言不足挽阻,思結握兵柄者以自重,頗游說當局,數年無所就。丁巳,張勛復闢,以有為為弼德院副院長。勛議行君主立憲,有為仍主虛君共和。事變,有為避美國使館,旋脫歸上海。

甲子,移宮事起,修改優待條件,有為馳電以爭,略曰:「優待條件,系大清皇帝與民國臨時政府議定,永久有效,由英使保證,並用正式公文通告各國,以昭大信,無異國際條約。今政府擅改條文,強令簽認,復敢挾兵搜宮,侵犯皇帝,侮逐後妃,抄沒寶器,不顧國信,倉卒要盟,則內而憲法,外而條約,皆可立廢,尚能立國乎?皇上天下為公,中外共仰,豈屑與爭,實為民國羞也!」明年,移蹕天津,有為來覲謁,以進德、修業、親賢、遠佞為言。丁卯,有為年七十,賜「壽」,手疏泣謝,歷敘恩遇及一生艱險狀,悲憤動人。時有為懷今感舊,傷痛已甚,哭笑無端。自知將不起,遂草遺書,病卒於青島。

有為天資瑰異,古今學術無所不通,堅於自信,每有創論,常開風氣之先。初言改制,次論大同,謂太平世必可坐致,終悟天人一體之理。述作甚多,其著者有孔子改制考、新學偽經考、春秋董氏學、春秋筆削大義微言考、大同書、物質救國論、電通,及康子內外篇,長興學舍、萬木草堂、天游廬講學記,各國游記,暨文詩集。

論曰:光、宣兩朝,世變迭起,中國可謂多故矣。其事皆分見於紀、傳。斷代為史,辛亥以後,例不能詳。唯丁巳復闢,甲子移官,實為遜位後兩大案,而勛與有為又與清室相終始,亦不可遂沒其人。明末三王及諸遺臣,史皆勿諱,今仿其體,並詳著於篇,庶幾考有清一代之本末者,有所鑒焉。[一][一]按:關內本此卷是上卷移來的勞乃宣傳、沈曾植傳,無張勛傳、康有為傳。傳後有論,其文是:「論曰:乃宣、曾植皆學有遠識,本其所學,使獲竟其所施,其治績當更有遠到者。乃朝局遷移,掛冠神武,雖皆僑居海濱,而平居故國之思,無時敢或忘者。卒至憔悴憂傷,賚志以沒。悲夫!」關外一次本於張勛傳後附有張彪傳,全文如下:

張彪,字虎臣,山西榆次人。以武生歸撫標,巡撫張之洞器賞之,擢外委,隨調粵、鄂至兩江。時新練陸軍,充管帶,監修江陰江防砲臺。復還湖北,充護軍管帶。光緒二十三年,奏派赴日本考查軍政,歸,督修漢口後湖堤工,創漢陽兵工廠。累保副將,賞壯勇巴圖魯勇號,兼常備軍鎮統,授松潘總兵,留充陸軍第八軍鎮統制官。南北新軍會操於彰德,賞花翎;再會操太湖,更勇號曰奇穆欽。宣統二年,擢湖北提督,加陸軍副都統。三年,新軍變,總督瑞澂棄城走,彪率衛隊巷戰,自夜至日午,不能支,退召水師。瑞澂劾以構變潛逃,詔革職,圖後效。復充湘豫鄂援軍總司令,率殘軍保漢口。禁衛軍及北洋軍南下,督隊先驅,屢有克捷。既復漢陽,還原官。官軍請改共和,要彪署名,力欲之,遂稱病去。東渡日本,歸寓津,築張園自隱。乙丑,迎蹕駐園,供張服用,夙夜唯勤。丁卯秋,病篤,見駕臨視,已不能起,強啟目含淚而逝,年六十八。


\end{pinyinscope}