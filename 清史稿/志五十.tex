\article{志五十}

\begin{pinyinscope}
地理二十二

△貴州

貴州:禹貢荊、梁二州徼外之域。清初沿明制,設貴州布政使司,為貴州省。順治十六年,設巡撫,治貴陽,並設雲貴總督,分駐兩省。康熙元年,改貴州總督。四年,仍為雲貴總督,駐貴州。二十一年,移駐雲南。舊領府十。康熙三年,增置黔西、平遠、大定、威寧四府。二十二年,大定、平遠、黔西降州,隸威寧府。雍正五年,增置南籠府。六年,割四川遵義來屬。七年,復升大定、降威寧。乾隆四十一年,升仁懷,嘉慶二年,升松桃,均為直隸,改南籠為興義府。三年,降平越府為直隸州。十四年,升普安為直隸州。十六年,改。東至湖南晃州;五百四十里。西至雲南霑益;五百五十里。南至廣西南丹;二百二十里。北至四川綦江。五百五十里。東北距京師七千六百四十里。廣一千九十里,袤七百七十里。北極高二十五度四分至二十八度三十三分。京師偏西七度三十三分至十度五十五分。宣統三年,編戶一百七十七萬一千五百三十三,口八百五十萬三千九百五十四。共領府十二,直隸三,直隸州一,十一,州十三,縣三十四,土司五十三。驛道:一東出鎮雄關達湖南晃州;一西逾關索嶺達雲南平彞;一西北渡六廣河達四川永寧。電線:北通重慶、畢節,又分達威寧至雲南。

貴陽府:沖,繁,難。巡撫、布政使、提學使、按察使、糧儲道同駐。光緒三十四年裁糧儲道,設巡警道、勸業道。宣統元年改按察使為提法使。順治初,因明為軍民府,領州三,縣一。康熙十一年,增置龍里縣。二十六年,裁「軍民」字,增置貴築、修文二縣,又改平越府之貴定來隸。三十四年,省新貴入貴築。雍正四年,置長寨。光緒七年,以羅斛州判地置,移長寨同知駐,降長寨為鎮,並入定番。廣一百五十里,袤三百七十里。北極高九度五十二分。京師偏西九度五十二分。領一,州三,縣四。南:青巖土千總一。東:虎墜司長官一,雍正八年裁。貴築沖,繁,難。倚。明,貴州、貴前二衛。康熙二十六年改置,與新貴同城。三十四年省新貴入之。城內:翠屏山。東:銅鼓、棲霞、石門。北:貴山,府以此名。南:斗巖,板橋最高。西北:黔靈山,又木閣山,延袤百里,亙修文境內,通黔西。南明河自廣順入,合濟番河、四方河、阿江河,折東,龍洞河北流注之,又北入開州。雞公河自清鎮北流入境,又北仍入清鎮。貫城河出崆巃山,合城北擇溪水入城中,南流注南明河。東南:圖安關。東北:鴉關。驛一:皇華。南:白納司正副長官一,中曹司土千總一。西北:養龍司長官一。順治初,承明屬府,康熙間改屬縣。順治十五年,設中曹司正副長官一,雍正七年裁。又喇平司,康熙二十三年裁。貴定沖,繁。府東百十里。順治初,因明隸平越。康熙二十六年改隸。南:文筆、天馬、松牌、連珠山。西:金星、銀盤。北:陽寶、西華。東北:蔡苗山。甕首河出縣西平伐土司,東北錯入都勻,復逕縣南,加牙河自龍裡來注之。又北,八字河注之,北流,與博奇河會,折西北流,至巴香汛,合南明河。十萬溪,在縣北,苗眾每恃險為亂。東:玉杵關、谷滿關。西:馬桑關、甕城關。驛一:新添。有汛。南:新添司長官一。又平伐、大平伐、小平伐司長官一。西牌土舍一。東丹平、北把平二司,均裁。龍里沖,繁。府東五十里。明,龍里衛。康熙十一年改置。南:龍駕。西:長沖。北:雲臺。西南:回龍山。東門水出縣東南,老羅水、新安水西南分流,逕城北,合為博奇河。東龍洞河西北來注之,入貴定。加牙河出穀者巖,東流入甕首河。東:隴聳關。西:黎兒關。驛一:龍里。北:大谷龍土千總一。小谷龍土把總一。南:羊腸土千總一。又西北:龍裡司,裁。修文沖。府北五十里。明,敷勇衛。康熙二十六年改置。城內:屏山。西:寶峰。北:鳳凰、將軍。東:西望山,綿亙百餘里。東南:龍岡。烏江自黔西入,即黔江,逕城西北,合雞公河,北流為六廣河,入開州。雞公河自清鎮入,石洞水合孟沖水西注之,又北注烏江。東北:底寨司正副長官一。開州難。府東百二十里。東:魯郎。西南:南望山、陰陽山。南明河自貴築入,逕城東,洗泥河東北注之,又北流,落旺河東北注之,又東為清水江,合烏江。烏江自修文入,為六廣河,逕城西北,納沙溪水、養龍水,逕城北,洋水河、橫水河合流注之,東南會清水江,緣遵義境入甕安。可渡河出城東南,伏流復出,為落旺河,東注清水江。東北:西司正副長官一,裁。定番州難。府南百里。定廣協副將駐。東:琴山。南:三寶、筆架。西:旗山。東南:松岐。西北:屏風山,濛江出,即連江,一曰牂牁江,一曰都泥江,出城西北山中,逕廣順再入境,崇水、潮井水注之,又西南入羅斛。豐寧河自都勻入,注巴盤江,錯入羅斛,合北盤江,東流,入廣西那地土州。上馬橋河出西北廢上馬橋司東,東北流,入貴築,注南明河。南:石門關、克度關。東北:程番關。大塘、長寨州判二。附郭程番司長官一。東南:大龍番、小龍番司長官一。南:韋番、羅番司長官一。西南:木瓜司正副長官一,麻鄉司長官一。東北:盧番司長官一。西:牛路、木官土舍一。又東:金石番司。南方番、盧山、洪番、臥龍番,西大華,西北上馬橋、小程番七司,裁。廣順州難。府西南百十里。西:真武。東:螺擁、白雲。南:天臺山。南明河出城東北,折東入貴築。雞公河自普定入,麻★河注之,折北入安平。尤愛河在城東從仁里,東流注濟番河。東:白崖關、翁桂關。西:文馬關。北:燕溪關。長寨州判一。有宗角、長寨、同筍三汛。有金築司,裁。羅斛府西南四百二十里。順治初,因明隸廣西泗城土州,尋改隸泗城府。雍正三年割置永豐州,設州判,隸南籠府。乾隆十四年改隸定番。光緒七年置。東南:老人峰。西南:六合山。濛江自定番入,剋孟河自普定、猛渡河自歸化合流注之,又南流,注北盤江。北盤江合南盤江自貞豐東流入,受濛江水,入那地土州。又巴盤江在城東北,上流曰豐寧河,自都勻入,合藤茶河,東南入廣西泗城。扎佐巡司一。有羅斛汛。羅斛打拱土千戶一。何往土外委一。

安順府:沖,繁,難。舊隸貴西道。提督駐。順治初,沿明制,為軍民府。康熙二十六年,裁「軍民」字。東北距省治百八十里。廣三百十里,袤百六十里。北極高三十六度十二分。京師偏西二十度二十四分。領二,州二,縣三。西北:西堡司副長官一。西南:安谷、西堡二司,裁。普定沖,繁,難。倚。明,普定衛。康熙十一年改置,省定南所入之。城內:塔山。東:飛虹、巖孔。南:屏風。東南:旗山。西北:舊坡、新坡山。寧穀河出東山,合數水,西南流入鎮寧。簸渡河自鎮寧入,東北流入安平。剋孟河出縣東南,南流入羅斛。猛渡河出縣西南,南流入歸化。雞公河上源為大水河,出縣東北,東南流入廣順。東:羅仙關、楊家關。南:半天關。西:牛氾關、大屯關、老虎關、打鐵關。驛一:普利。有寧谷廢司。上五苑枝土千總,裁。鎮寧州沖,繁。府西五十里。康熙二十六年,省安莊衛入之。南:玉京、青龍。東:東坡山。西:白巖、慈母山。北:九十九隴,周百餘里。南:烏泥江,源出山箐中,匯諸溪澗水,東北定番寧穀河自普定入,合州西諸水,南流入貞豐,注北盤江。簸渡河自郎岱入,墮極河南流,谷龍河合三岔河北流,並注之。東北流,緣普定界入平遠。東:猴兒關。西南:土地關、鳳凰關、石龍關。驛二:安莊、坡貢。有坡貢汛。東康佐、北十二營二司,裁。永寧州沖,繁。府西百四十里。城內:頂箐山。東:二龍。南:箭眉。西:普肇、安籠箐山。西北:紅崖山。北盤江自郎岱入,拖長江自普安合庚、戌二河,東北流注之,逕城西,納西坡河、馬涼河,又屈西南,馬畢河自安南東北流注之,折東入貞豐。西:梅子關。慕役巡檢一。有關嶺、慕役、上卦三汛。西:沙營頂營長官一。盤江土巡檢一。清鎮沖,繁。府東北百二十里。明,威清衛。康熙二十六年改置,省鎮西衛,赫聲、威武二所入之。東:獅子山。南:馬鞍。西:銅鼓。北:羊耳山。雞公河自安平入,北流,逕城西,曲循城北,錯入貴築,又北入修文。三岔河自安平入,折西北流,牛場河西南來注之,亦入修文。西有滴澄關。安平沖,繁。府東六十里。明,平壩衛。康熙二十六年改置,省柔遠所入之。東:金鼇、高峰。南:圓帽、天臺。東南:馬頭山。簸渡河自普定入,逕天馬山,北流入平遠。雞公河自廣順入,羊腸河東流注之。羊腸河雙源夾城流,至縣南十里而合,又屈東北,與麻線河會,

折北入清鎮。東:銅鼓關。南:沙子關、楊家關。東南:平壩關。郎岱簡。府西百八十五里。明,土司隴氏地。康熙五年平之。雍正九年置。永安協副將駐。北盤江自普安入,逕西,又東南流,入永寧。簸渡河自水城入,合北諸水,折東流,入鎮寧。東:石龍關。西:打鐵關。驛一:毛口。有羊腸巡司一。歸化要。府南百六十里。明,康佐長官司及鎮寧、定番、廣順三州交錯之地。雍正八年置。巖下河出西,南流,錯入貞豐,復入境。烏泥河西南流來會,復入貞豐。猛渡河自普定入,復東南入羅斛。東:擺浪關。北:銀子關。南:紅沙關。有大營、壩陽、白巖、猴場、鼠場、牛場六汛。

都勻府:要。隸貴東道。副將駐。順治初,因明制。領州二,縣一。康熙中,置都勻。雍正中,闢八寨、都江、丹江,置同知一,通判二。十一年,廣西荔波割隸。西北距省治二百四十里。廣三百二十里,袤四百五十里。北極高二十六度十三分。京師偏西九度三分。領三,州二,縣三。西南:六硐司長官一。南:王司、吳司長官司一。又東天壩、西南平州、西丹行三司,裁。都勻繁。倚。明,都勻衛。康熙十一年置。城內:東山。西:龍山。北:夢遇。西南:凱陽山。馬尾河為清水江南源,出縣西南,合一小水,又北納邦水河、龍潭河,東流入麻哈。麥沖河出縣南,合四小水,西南流為豐安河,入獨山。西:石屏關、威鎮關。北:平定關。南:都勻司,西:邦水司長官一,明屬府。順治初改隸。平浪司長官一,明屬府。順治初改隸。雍正五年裁。麻哈州繁,難。府北六十里。東:皮隴、天臺。南:天馬。西:玉屏、銅鼓山。南:麻哈河,有二源,經城西合為一水,又名兩岔江,北流入平越。馬尾河自府東流入境,逕吳家司,北流入清平。穀硐、卡烏二汛。南:樂平司長官一。落戶土舍一。東:平定司長官一。宣威土舍一。北:養鵝土千總一。西:舊司土舍一。獨山州耍。府西南百二十里。南有獨山,州以此名。東:文漢山。南:鎮靈。西:行郎山。南:獨山江,即都江上源,古牂柯江也,出水巖梅花峒,東北流,經爛土司,馬場河分流注之,折東入都江。西:鳳飲河,出飛鳳井,環城流,入獨山江。豐安河自都勻入,逕城北,深河、平舟河來注之,再西入長寨。南:雞公關。北:阿坑關。三角州同一。巴開、打略二汛。附郭獨山司長官一。南:豐寧上長官司一。東南:豐寧下長官司一。三捧土舍一。東:爛土司長官一。東北:普安土舍一。清平衡,難。府東北百二十里。明,縣。康熙七年省入麻哈州,十一年復置,裁清平衛入之。南:水箐。東:棋盤。北:侍講山。東南:香爐。東北:天榜山。豬梁江為清水江北源,自平越入,會麻哈河,東流入黃平。東南:馬尾河,即劍江,自都勻入,北流入清水。南:雞場關。凱里縣丞一。排養、爐山二汛。東:凱裡司安撫使,裁。荔波要。府東南二百里。順治初,承明隸廣西慶遠府。雍正十年改隸。東:水排山。北:分水嶺。荔泉在城北,縣以此名。勞村江出縣東北,西南流,與峨江會。峨江河,南北二源,合於水董,再西南,永長溪自古州逕都江南,合數小水注之,入廣西南丹土州。南:黎明關。西:馬甲關。方村縣丞一。有三洞、方村二汛。八寨要。府東九十里。明,天壩土司地。雍正六年,平苗疆置。西:得鹿山、大登高山,均險要。西:馬尾河,自都勻入,東北流,入麻哈。龍泉自龍井、南泉自丹江,均入馬尾河。都江自獨山逕都江南,一水出北坡腳寨,南流入都江境來會。南:羊勇關。北:五里關。有九門汛。東南:楊武排調司長官一。東:永安司長官一。丹江要。府東北百四十里。明,生苗地。雍正六年,平苗疆置。西南:九門山。東南:牛皮箐,迤邐數百里,亙八寨、都江、古州界。大丹江源出西南,小丹江自東南來會,曰九股河,東北流,入臺拱。東:防里河,西流入丹江。雞講、黃茅、烏疊、頂冠、空稗、松林六汛。東北雞講、北黃茅、西南烏疊土千總一。都江要。府東南二百二十二里。明,來牛大寨地。雍正六年,平苗疆置。西:柳疊山。東北:大坪山。都江上流曰獨山江,自獨山東流入,羊烏河合烏溝河來會,又東入古州。北:排常關。有順德、歸仁土千總一。

鎮遠府:衡,繁,難。隸貴東道。總兵駐。順治初,因明制。西南距省治四百五十二里。廣一百七十五里,袤二百五里。北極高二十七度二分。京師偏西八度十三分。領二,州一,縣三。治後,石屏山。山半有穴,久雨水注則江溢。東南:思邛山,都波、都來二山。邛水司南:馬首山。偏橋司南:石柱山。偏橋司長官一。左副、右副長官一,嗣改左副、右副為七品土官。鎮遠沖,繁。倚。康熙二十二年以湖廣鎮遠衛來屬,省入縣。東:鐵山、中河山、馬場山、觀音巖。南:五老山。北:大小石崖山。東北:打杵巖。西:鼓樓坡山。清水江自施秉入,逕鎮遠土司東入臺拱。邛水有二源,合流逕邛水司南,入清江。德明河源出德明洞,東南入臺拱,注清水江。潕水自施秉入,白水溪、小由溪諸水注之,逕城西南為鎮陽江,又東納焦溪,東北流,入青溪。西北:金石關。北:文德關、鎮雄關。東:雞鳴關。邛水,縣丞一。四十八溪,主簿一。東南:邛水司正副長官一,嗣改為七品土官。施秉沖,難。府西南七十里。康熙二十二年以湖廣偏橋衛來屬。二十六年省入縣。城內:飛鳳山。東:金鐘、玉屏。北:三臺山、岑山。清水江自黃平東流入,納一小水,又東流入臺拱。潕水自黃平東北流入,受瓦窯河、杉木河諸水,小江南自黃平來會,謂之兩江河,東流入鎮遠。西:欄橋關。勝秉,縣丞一。偏橋廢驛。天柱繁,疲,難。府東南二百十里。順治初,因明隸湖廣靖州。雍正五年改隸黎平府,十一年來隸。東:高雲山、茨嶺山。南:春花、黃少。西:蓮花。北:柱石山,縣以此名。清水江自開泰入,逕城南,直銀水、等溪東南流注之,入湖南會同。西江一曰等溪,東南流,至城北,入鑒水江。東:老黃田關。南:王橋關。西:西安哨關。北:渡頭關。柳霽,縣丞一。遠口巡司一。岔處、革溪二汛。黃平州沖,繁,難。府西南一百三十里。順治初,因明屬平越。康熙二十六年徙州治於舊興隆衛。嘉慶三年來隸。東:飛雲巖。南:鼓臺山。西:斗巖山。北:北辰、岑舟、石林山。清水江上源二,並自清平入,逕城南,合東流,入施秉。潕水出州南金鳳山,北流,合西來二小水,東北入施秉。東:冷水河、秀水溪、高溪,下流合秀水入重安江。東:馬鬃嶺關、大石關。舊州城巡檢一。驛一:重安江。黃平汛。東巖關司、東南重安司長官一。又有朗城司土吏目,裁。臺拱要。府東南一百三里。明,九股生苗地。雍正十一年,平苗疆置,移清江同知駐之。北:貓坡山。東:蓮花。西南:臺雄山。清水江即施洞河,自鎮遠入,在城北,自黃平流入,折東南,逕革東汛,入丹江。九股河一名巴拉河,自丹江北流入境,至城西,斬水西北流來注之,折東北,入清水江。番招、臺雄、革東、稿貢四汛。清江要。府東南一百九十里。明,清水江苗地。雍正八年,平苗疆置,設同知。十一年,移同知於臺拱,改通判。清江協副將駐。南:白索。西:公鵝、三臺。北:柳羅山、白濟關山。清水江自臺拱東南流入,邛水自左來注之。烏蔑河、烏擁河、烏拉河自丹江入,匯為南哨河,自右來注之。再東納德河,入開泰。東:東鎮關。北:白濟關。

思南府:繁。隸貴東道。順治初,沿明制。西南距省治六百四十五里。廣四百里,袤五百六十里。北極高二十七度五十六分。京師偏西八度五分。領縣三。城內:中和山。東:東勝、思唐。西:巖門、白鹿。北:雙峰、象山。烏江自石阡入,鸚鵡溪、板坪河會清江溪注之,折東錯入安化。北行至齊灘場,復入府境,曹溪東流注之,小郎壩水北流注之,再北復入安化。東:石峽關、武勝關、永勝關。南:芙蓉關。西北:鸚鵡關。東朗溪司、北沿河司長官一。西:西山陽洞蠻夷司,裁。安化繁。舊附郭。光緒八年移治大堡。府北百三十里。東:鳳凰、蓮花。南:文中。北:柱巖、椅子山。西南:倉廩山,下俯煎茶溪,有泉名第一。烏江自府東北流入,思邛江自印江西北流注之。三岔小河自四川酉陽西流注之,東北流,入酉陽。洪渡河自龍泉東北流入,經簡家溝,下流曰豐樂河,一水西北來注之,錯入婺川,復東北入縣境,北流入四川酉陽。西北:覃韓偏刀水廢土巡司一。婺川繁,難。府西北二百四十里。東:大巖。南:泥塘。西:華蓋。東北:長錢山。北:臥龍山。豐河自安化入,合龍登河,曉洋江合白皎溪東北來注之,又東北,復入安化。芙蓉江出縣西,西北流,錯入正安,復逕縣西北,北流入四川涪州。東:焦巖關、水雲關。西:石將關。北:九杵關、烏金關、石板關、青巖關。印江簡。府南四十里。東:文筆、峨嶺、大聖、登山。西:河縫山。北:石筍山。思邛江自松桃入,折北流,合一小水入安化,注烏江。東:峨嶺關、仡楠關。南:秀寶關。

思州府:沖。隸貴東道。順治初,因明制。領長官司四,不領縣。雍正五年,割湖廣平溪、清浪二衛來屬。尋改玉屏、青溪二縣。西南距省治五百四十里。廣一百九十里,袤二百六十里。北極高二十七度十一分。京師偏西七度五十五分。領縣二。東:巖前、龍塘。南:聖德。西:盤山、岑鞏。北:紅崖、六農山。鎮陽江自青溪入,逕城東南入玉屏。潞瀨河出府西北,合洪寨河,東南流,又納施溪、灑溪、架溪諸水,東南入鎮陽江。易家河出府東北,合文水河,南流亦入鎮陽江。東:都哨關。南:清平關、黃土關。東北:占魚關。西:盤山關。附郭都坪司,西南都素司、東北黃道溪司正副長官一,嗣裁副長官。北:施溪司長官一。玉屏沖,繁。府東一百里。順治初,因明湖廣平溪衛。雍正五年改置來隸。北:玉屏山,縣以此名。城內:回龍山。東:三臺、月屏山。南:道定山,與雙薦峰對峙。界牌山為諸蠻出入要路。鎮陽江自府東北入,流逕城北,名曰平江,北流入湖南晃州。西:野雞河,匯西溪、梭溪諸水,逕飛鳳山、野雞坪入平江。太平河從之。青溪沖,繁。府南九十里。順治初,因明湖廣清浪衛。雍正五年改置來隸。縣治後北障山。東:竺云。西:靈寶山。北:觀音山。鎮陽江即青溪江,自鎮遠入,鐵廠河合竹坪河、描龍河注之,東北流入府。東:清浪關、雞鳴關。西:粟子關。

銅仁府:中,繁,難。隸貴東道。副將駐。順治初,因明制。康熙四十三年,平紅苗,設正大營,以同知駐其地。雍正八年,平松桃紅苗,移同知駐,以正大營地割隸銅仁縣。嘉慶三年,升松桃為直隸,以烏羅、平溪二司地撥歸轄。光緒六年,剿平梵凈山匪,移銅仁縣治江口,即提溪吏目駐地,分府屬五硐歸縣,分縣屬六鄉及壩盤等三鄉之半歸府親轄,移吏目大萬山。西南距省治六百六里。廣一百七十里,袤二百七十里。北極高二十七度三十八分。京師偏西七度三十分。領縣一。南:銅崖,府以此名。東:石笏、天臺。南:天馬、六龍。西:諸葛山。北:翀鳳山。大江即辰水,自縣東流入府,合甕怕洞水,又東與小江合。小江發源梵凈山,合茶山塘水,南流與辰水會,東入湖南麻陽,謂之麻陽江。東:龍勢、石榴、漾頭等關。北:倒馬、芭龍、甕梅、倒水等關。大萬山吏目一。正大、施溪二汛。東南省溪司、西提溪司正副長官一。銅仁繁。府西北九十里。月波山在縣治右,形如半月,斜對三巖,高十餘仞。西北有梵凈山,周五六百里,跨思南、鎮遠、松桃、印江界。南:五雲山。西南:百丈山。辰水出梵凈山,有二源,右源納標桿河、羊溪數小水,東南逕提溪司,左源經哨上渡,納一小水,至提溪司與右源會,省溪、凱洪溪注之,東流入府。正大營縣丞一。滑石汛。

遵義府:中,沖,繁。舊隸貴西道。副將駐。順治初,因明制,為軍民府,隸四川。康熙二十六年,裁「軍民」字。雍正五年改隸。西南距省治二百八十里。廣七百九十里,袤三百六十里。北極高二十七度三十七分。京師偏西九度二十九分。領一,州一,縣四。遵義沖,繁,難。倚。順治初,因明隸四川。雍正七年,同府改隸。東:香風、三臺。西:洪關、元寶、大水田山、婁山。北:大樓、龍巖、定軍山。西北:永安山。烏江緣開州入,中渡河、樂閩河及二小水南流注之,又東南會清水江,入甕安。湘江出縣西北龍巖山,二源合南流,洪江合鳳凰溪來會,南逕湄潭,至甕安注烏江。赤水河自仁懷入,沙壩河合數小水北流注之,又納鹽井河,錯入桐梓。東:三渡關。西:烏江關、落濛關。北:太平關。驛四:烏江、播川、松坎、湘川。桐梓繁,疲,難。府北百二十里。順治初,因明隸四川。甕正七年,同府改隸。東:石女、九龍山。北:扶歡。南:金馬。西:金鵝山。赤水河自仁懷東流入遵義,復錯入縣境,齋郎河合溱溪水西流注之,復入仁懷。松坎河,即綦江上源,自正安入,出縣東北,二源合,西北流,坡頭河自正安西流注之,又北入四川綦江。石嘴河,即溫水上源,出縣西北,入仁懷界。北:張九關。東北:石壺關。綏陽簡。府東一百里。順治初,因明隸四川。雍正七年,同府改隸。東:綏陽山,縣以此名。南:鼓山、冠子。北:波利山、仙人山。西:金子山。樂安河一曰鹿塘河,二源出縣北,合南流入遵義,注湘江。湄潭河自湄潭南流,逕城東南,仍入湄潭。小烏江一曰渡頭河,出縣北,合桑木塘水、關渡河,北流入正安。東:九杼關、石卯關、苦竹關。西:郎山關。南:板閣關。東有桑木關、龍洞關。正安州難。府東北三百四十里。順治初,因明為真安,隸四川。康熙中,遷治古鳳。雍正二年改正安。七年,同府改隸。南:羅蒙山、石場清凈。西:紬子、峻嶺。北:豹子山。小烏江自綏陽入,右納牛渡河,左納清溪河,又東北流,注芙蓉江。三江河自四川綦江入,納安四溪水,又東北入婺川,亦曰芙蓉江。坡頭河自綦江西南流,逕縣境,又西入桐梓,注松坎河。北:老鷹關、青巖關。西:白巖關。仁懷沖。府西北百八十里。順治初,因明隸四川。雍正七年,同府改隸。八年,移治亭子壩。東:翠濤。西:夕陽。北:牛心山。西北:老色山。赤水河自四川永寧入,逕瑒子關,合二小水,錯入遵義、桐梓。折西北,復入縣,右納楓香壩河,左納九溪河,古藺河北流注之。又西北,入赤水南,曲折西流,復東北,再入縣境,納高洞河,入四川合江。溫水自桐梓入,合三岔溝水,入四川綦江。溫水場府經歷一。有汛。赤水要。府西北二百四十里。雍正八年,以通判分駐之,留元壩改置仁懷。乾隆四十一年升直隸。光緒三十四年改名,降。東:天臺。南:三臺、五老。西:官山,綿長三百餘里。赤水河自仁懷入,永思河亦自仁懷來注之,南納儒溪、泥溪、猿猴溪,北納葫蘆溪、堯壩溪、沙壩溪,經南,後溪注之。又北流,風水溪並二小水注之,東北流,仍入仁懷。南:葫蘆關。西:中箐關。猿猴汛。

石阡府:簡。舊隸糧儲道。順治初,沿明制。康熙中,省葛彰、苗民。雍正中,省石阡副司。西南距省治五百七十四里。廣六十五里,袤四百四十里。北極高二十七度二十九分。京師偏西八度十九分。領縣一。阡山,自平越入境,蜿蜒數百里,府以此名。東:九龍、鎮東。南:松明、十萬山。西:萬壽山。北:香爐山。烏江自餘慶入,落花屯水東南流注之。龍底河有二源,經府治西,合一小水,東北流注之,入思南。龍底河一曰白巖河,上源為包溪,北流逕黃芧囤,納大溪、凱科溪,再北入烏江。南樂回溪,西北深溪、北洋溪,皆入龍底河。東:松明關。東南:大定關。西南:鎮安關、錫樂平關。北:鎮夷關。龍泉繁,難。府西北二百四十里。城內:鳳凰山。南:將軍山。西:綏陽。北:雞翁山。龍泉出鳳凰山麓,縣以此名。羊子河、貫石河並出縣西,合東流,逕義陽山南,為義陽江。右合一水,東流為清江溪,入思南。洪渡河出縣西北山,東北流,入安化。大水河亦出縣西北,合小水河東流從之。東:張教壩關。西:平水口關、虎踞關。偏刀水汛。土縣丞、土主簿一,均裁。

黎平府:繁,疲,難。隸貴東道。順治初,因明制,領縣一:永從。雍正三年,以湖南五開、銅鼓二衛來屬。五年,改二衛為開泰、錦屏二縣,又以湖南靖州之天柱縣來屬。七年,增設古州。十二年,改天柱屬鎮遠府。乾隆三十五年,增設下江。道光十二年,降錦屏為鄉,以其地屬開泰。西北距省治八百八十里。廣四百七十里,袤四百三十里。北極高二十六度九分。京師偏西七度三十一分。領二,縣二。城內:五龍山,中黃龍。東:太平。南:醜家。西北:寶唐,山勢重疊。自北而南,亙百餘里。洪州吏目一。有黎平汛。東南洪州、北潭溪、歐陽、湖耳司正副長官一。東北新化,西古州,北龍里、中林、八舟、亮寨司長官一。又三郎司、赤溪湳洞司,裁。同知及理苗照磨駐古州。通判駐下江。吏目駐洪州。泊里長官司。開泰繁,難。倚。東:龍見、大巖。東北:掛榜。北:龍標、楚營、八舟山、茶山。西南:銅關鐵寨山。清水江自清江入,烏下江合二水東北流注之。新化江出天甫山,亦東北流注之,入天柱。永從溪自永從入,東北流曰潘老河,東入湖南靖州。東:寧溪關、黃泥關。東南:燕窩關。錦屏縣丞一。有汛。朗洞縣丞一。永從簡。府南六十里。縣治後:飛鳳山。南:上下皮林山。東南:鹿背山。西南:標瑞、龍圖山。福祿江上流即古州江,自下江東南流入境,經丙妹南,錯入廣西懷遠。曹平江亦自下江東南流入境,經丙妹北,東入懷遠。永從溪二源出縣南,合流,北入開泰。丙妹,縣丞一。有永從、丙妹二汛。古州要。府西一百八十里。古州總兵、貴東道駐。東:雙鳳。西:俾飛、擺喇山。西南:獅子山。都江自都江入,名古州江,左納彩江,入下江。榕江、車江並出北,合流注之,折東南入下江。朗洞江出北,東北流入開泰,注烏下江。東:永鎮關。西:歸化關。有王嶺、寨蒿、小都江三汛。下江要。府西南一百八十里。南:朋論山。西南:崖雞、烏地、霧惈、九千里山,亙數百里。都江自古州東南流入,逕南入永從。東江、溶江自古州合流入境,下游曰曹平江,東南流入永從。弱女江源出南,東北流至雙江口,小溪東北流來會,再東北入古州江。

大定府:要。舊隸貴西道。明,貴州宣慰司及烏撒軍民府地。副將駐。康熙三年,平水西、烏撒,以大方城置。二十六年,降州,隸威寧府。雍正七年,復升府。東南距省治三百三十里。廣五百八十五里,袤六百六十里。北極高二十七度四分。京師偏西十度五十五分。領一,州三,縣一。東:萬松、火焰、鳳山、凰山。西:五老山。北:大雞。東北:九龍。西北:雙山。烏江自畢節入,暑仲河、通德河皆北流注之,又東,落折河合打雞關諸水,折南來注之。烏西河合石溪河自北來,惈龍河自南來,皆注之,又東分入平遠。赤水河自畢節入,經府北,納永岸小河,臥牛河合油杉河諸水,東北入黔西。東:老蒙關。南:那集關。西:奢東關、樂聚關。北:大弄關、柯家關。倉上、烏西二汛。平遠州繁,難。府東南八十里。康熙三年平水西、烏撒,以比喇壩置府。二十二年降州,隸大定。二十六年改隸威寧。雍正七年仍來隸。平遠協副將駐。東:懸霧、東山。南:獅子、鳳凰。西:白巖山。北:墨續山。烏江自府南入,高家河、卜牛河東北流注之。又東,納以麥河水,入黔西。西:木底河,即鴨池河,自水城入,受武著河,錯入安順,北古河,合墮極河南流注之,復逕城東,名簸渡。會牛塘河諸水,北流入黔西。東:織金關。南:鳳凰關、望城關。黔西州繁,難。府東二百二十里。康熙三年,以水西置府。二十二年,降州來隸。二十六年改隸威寧。雍正七年仍來隸。城內:獅子山、牛飲山。南:石虎。北:分水嶺。東:金雞山。又十萬溪箐,懸崖絕壁,四面皆砦。西北:白塔山、杓里箐、比喇大箐。儸革河即六歸河,自府入,平溪南流注之。又東,鴨池河自平遠入,又東會簸渡河,東入修文,為烏江南源。以濟河,源出州西北,西南流,合打鼓寨水,東北流,渭河合烏箐河來會,沙河合鼓樓水、三現身水,東南來注之,入修文。赤水河,自府東北流,逕州境,入四川永寧。西:化榨關。沙溪、沙土、右革闌、鴨池、西溪、六廣、黃沙諸汛。威寧州要。府西三百八十三里。康熙三年以烏撒置府。雍正七年,降州來隸。威寧鎮總兵駐。東:飛鳳山。東北:翠屏。西:火龍、麻窩。北:三臺、烏門。南:石龍、千丈崖。七星河為烏江上源,出州南,合八仙海、泚處海諸水,東北流,過清水塘,入畢節,再入州境,菩薩海南注之,黑章河北注之,又東,復入畢節。北盤江,出州西山,二源合南流,經瓦渣汛,西為瓦渣河,又南,錯入雲南宣威,為可渡河。牛欄江自雲南會澤入,合膩書河,又北流,入雲南恩安。洛澤河出州西北,合數小水東北流,亦入恩安。東:石駝關、梅子關。南:雲關。北:可渡關。西北:分水嶺關。得勝坡巡司一,有汛,與江半坡二。水西宣慰使一,裁。畢節沖,繁,難。府西北一百里。明,畢節赤水衛地。康熙二十六年置,隸威寧府。雍正七年改隸。貴西道駐。光緒三十四年裁,改巡警道,移駐貴陽。東:木稀山。南:脫穎。西:七星。北:石筍山。東北:東陵山、雪山、層臺山。烏江自威寧入,亦名七星河,過瓦甸汛,再入州境,又東復逕縣境,則底河自雲南鎮雄入,合後所河,南流注之。又東南,合二小水入府。赤水河即赤虺河,自雲南鎮雄入,納杉木河,入府。東:木稀關。南:落淅關。西:老鴉關。畢赤汛。水城要。府西南二百九十里。明,水西地。雍正十年置。東:將軍、玉筍山。南:馬龍。北:麒麟、文筆山。簸渡河一曰鴉池河,出西以且海,合一水,東北流,經城北,折東南,水城河東北來會。又納扒瓦河、以固汛水、武著河諸水,錯入郎岱。北盤江自雲南宣威入,喇雍河合桃花溪水自威寧來注之,北納結裏山東西二水及黑勝汛水,南納木冬河,入盤州。州東:猴兒關。西:卡子鬥關。普擦、豬場二汛。

興義府:要。舊隸貴西道。安義鎮總兵駐。順治初,因明為安籠所。康熙二十五年,置南籠,移貴陽通判駐之,仍隸府。雍正五年升府。嘉慶二年,改興義。東北距省治五百八十里。廣七百四十里,袤五百五里。北極高二十五度四分。京師偏西十度五十五分。領一,州一,縣三。東:龍井山,珍珠泉出焉。將軍山。西:九峰山。北:玉屏、萬壽山。南:紅江即南盤江,自興義入,都威河西南流注之,又東入貞豐。北盤江自貞豐南流,錯入府境,仍入州。魯溝河,源出府北,左納阿棒河,又東入貞豐,注北盤江。綠海,出府城東北,眾水所匯。南:梅子關。馬鞭田、哈馬隘、狗場、卡子、額老諸汛。貞豐州要。府東北一百二十里。雍正五年,析廣西西隆州紅水江以北地設永豐州,隸南籠府。嘉慶二年改貞豐。署後枕峻山。東:六合山。北:九盤、花江、巖山。西南:籠鶴山,綿亙數十里。北盤江自永寧入,寧穀河亦自州來注之,又南會巖下河,錯入府,仍逕州境,左納魯溝河、綠海,南流與南盤江會。南盤江自府入,八臥溪北來注之,又東合北盤江,東北入羅斛。東:坡呈箐關。西南:者黨關。北:石關。冊亨,州同一。定頭、高坎、王母、渡邑四汛。普安沖,繁。府西北二百四十里。明,新城、新興二千戶所。順治十八年置,隸安順府。康熙二十二年移治新興。雍正五年改隸。東:烏龍、直武。南:九峰山。西:八納山。北:落馬、大小尖山、羅摩塔山。拖長江自盤州入,有三小水合流注之,又東北入永寧。深溪河源出縣南,右合阿希河,左合木郎河,東南流入興義,注馬別河。抹角河自盤州入,合一小水,西南流,入雲南平彞。西北:堅固關。北:芭蕉關。驛二:罐子窯、楊松。舊設驛丞,裁。新城,縣丞一。土州同裁。安南沖,繁。府北二百四十里。明,安南衛。康熙二十六年置,隸安順府。雍正五年改隸。城內:天馬山。東:盤江。西:晴龍、白基山。西北:尾灑山。西北:毛口河,即北盤江上源,自盤州入,東南入郎岱。西西寧河、西坡河,北甲猛河,下流皆入盤江。南:巴林河,北流逕普安,至城東,又為大章河,下流合阿里河,注北盤江。東:盤江關、海馬關。西:烏鳴關。南:老鴉關。盤江十一城,明天啟間築。驛一:列當。舊設驛丞,裁。阿都、廖箕二汛。興義要。府西北八十里。雍正五年於黃草壩設州判,隸普安州。嘉慶三年裁,改置縣,隸府。十四年改隸普安直隸州。十六年仍來隸。南:筆架。東:白馬。北:獅子、馬鞍山。南:盤江上源曰八達河,自雲南羅平入,逕城西南,沿界東北流,九龍河亦自羅平入,合上江水注之,又東納中江、下江二水,逕城南,馬別河自普安南流注之,又東入府。棒鮓巡檢一。亦資孔驛丞一。盤州要。府西三百里。順治初,因明普安州,隸安順府。康熙二十六年省普安衛入州。雍正五年改隸。嘉慶十四年升直隸州。十六年改直隸。光緒三十四年改名,降,仍隸府。南:猗蘭山,為滇、黔分界處。西:黑山,上有潭。北:廣武山,絕頂有泉九,匯為大池。西南:黨壁山。盤江自水城入,納羅摩塔河,東南流,入郎岱。拖長江出西南平彞所,北流,有一水自海子鋪來會,至輭橋驛,合數小水入普安。豬場河出北,折東合二水,又東入普安,注拖長江。抹角河出西南,亦入普安。南:倒木關。西:分水嶺關。東南:安籠箐關。驛一:山門。上舍、白沙、劉官三汛。

松桃直隸:要,繁,疲,難。隸貴東道。副將駐。明,紅苗地。康熙四十三年,討平紅苗,設正大營,置同知,隸銅仁府。雍正八年,平松桃,置,移同知駐。嘉慶二年,升直隸,益以銅仁府屬平頭、烏羅二土司地。西南距省治八百四十五里。廣二百八十里,袤二百二十里。北極高二十八度八分。京師偏西七度三十三分。城內:蓼皋山。東:七星山。北:秋螺。南:獅子。西北:龍頂山。武溪出西,為酉水西南源,合二水東流,北入四川秀山。沱江出南,東流入湖南鳳凰。思邛江出西,二水合西流,入印江。西:平頭關、野貓關。有盤石、護國、木樹、芭茅、石峴諸汛。西:烏羅、平頭司長官一。

平越直隸州:沖,繁,難。舊隸糧儲道。順治初,因明為軍民府。康熙十一年,改平越衛為縣,附郭。二十六年,省「軍民」字。嘉慶三年,降直隸州,省平越縣。西南距省治一百九十里。廣一百八十里,袤三百三十里。北極高二十六度三十八分。京師偏西九度五分。領縣三。城內:福泉山。東:黎峨山。東南:疊翠山,群峰插天,中為老人峰。西:滃霾、楊山、杉木箐山,峰巒高峻。豬梁江為清水江北源,出州西北,合數水,逕黃絲驛,西北府城水、卡龍河、西南羊場河均注之,東流入清平。白水河一曰嶰河,源出州西北,南流逕牛場,有二水來合,入豬梁江。東:羊場關。南:武勝關。北:七星關。驛三:酉陽、黃絲、楊老。汛三:酉陽、楊老,打鐵關。西楊義司,西北高坪、中坪司長官一。甕安難。州北六十里。東:筆架山、都凹山。西:仙橋、白樂。北:九峰、玉華峰。烏江自開州入,逕城北,湘江自遵義南來注之,又東,甕安河、坪橋河、紅頭鋪河、草塘司河東北流注之,湄潭河自遵義南來注之,東入餘慶。東南:藍家關。西:黃灘關。西北甕水,東北草塘,土縣丞一。湄潭繁,疲,難。州北三百三十里。城內:玉屏山。西:瑪瑙。北:覺仙。南:象山、牛星山。湄潭河二源,自大小板角關入,合南流,至城北,匯數小水,西南流,逕遵義入甕安,注烏江。北:土溪河自正安入,至老木凹,合青龍水,入婺州,注豐樂河。東:錫洛關。西北:板角關。北:青龍關。餘慶簡。州東北一百四十里。南:中華、拱辰。西:九龍山。北:夢瓘山、牛塘山。烏江自甕安入,餘慶司水南流注之。河自甕安納小江、豬場河,東北流,牛場河即白泥江,納新村水,亦東北流注之,又東北入石阡。南:頭關。西:中關。西北:餘慶土縣丞一。東北:白泥土主簿一。


\end{pinyinscope}