\article{列傳二百一}

\begin{pinyinscope}
李臣典蕭孚泗硃洪章劉連捷彭毓橘

張詩日伍維壽硃南桂羅逢元李祥和蕭慶衍吳宗國

李臣典,字祥雲,湖南邵陽人。年十八從軍,初隸王珍部下,後從曾國荃援江西,隸吉字營。咸豐八年,戰吉安南門外,國荃受重創,臣典大呼挺矛進,追殺至永豐、新淦。國荃奇其勇,超擢寶慶營守備。克景德鎮,復浮梁,皆為軍鋒。十年,從戰小池驛,晉都司,賜花翎。進規安慶,戰菱湖,逼賊屯,扼其北,國荃傷股墜馬,臣典馳救以歸。偕張勝祿、張詩日戰樅陽,破援賊,水師得以進屯。十一年,攻安慶西門賊壘,陳玉成糾楊輔清數萬人圍官軍數重,戰至日中未決,馳告諸將曰:「事急矣,成敗在此舉!」臣典橫槊前驅,與諸營合力決蕩,賊大奔,斬首數千級,遂拔安慶,擢參將,賜號剛勇巴圖魯。

同治元年,從國荃乘勝下沿江各城隘,進軍江寧,臣典會取丹陽鎮,奪秣陵關,以總兵記名。軍中疫作,李秀成大舉來援,逼壘鏖戰,國荃督陣,砲傷頰,臣典與副將倪桂節力衛之,桂節陣亡。賊方攻西路急,臣典曰:「此虛聲也,請備東路。」既而賊果萃東路,參將劉玉春死之。砲彈穿壁墻如雨注,臣典死守,卒不能入。圍解,加提督銜。二年,偕趙三元夜襲雨花臺石城,束草填壕,緣梯將上,賊驚覺,燃砲轟擊,軍少卻。臣典搴旗大呼躍而上,諸軍繼之,擲火彈毀敵樓,城立拔,以提督記名。尋授河南歸德鎮總兵。偕蕭孚泗、張詩日等攻奪紫金山,又敗諸校場,連克近城諸壘。三年,克天保城,江寧之圍始合。五月,克地保城。

六月,諸軍番休進攻,賊死拒,殺傷相當。臣典偵知賊糧未盡,諸軍苦戰力漸疲,謂國荃曰:「師老矣!不急克,日久且生變。請於龍膊子重掘地道,原獨任之。」遂率副將吳宗國等日夜穴城,十五日地道成,臣典與九將同列誓狀。翼日,地雷發,臣典等蟻附入城,諸軍畢入。下令見長發者、新薙發者皆殺,於是殺賊十餘萬人。臣典遽病,恃壯不休息,未幾,卒於軍,年二十七。

捷上,列臣典功第一,錫封一等子爵,賜黃馬褂、雙眼花翎。命未至而臣典已歿,詔加贈太子少保,謚忠壯,吉安、安慶、江寧各建專祠。

蕭孚泗,湖南湘鄉人。咸豐三年,入湘軍,從羅澤南轉戰江西、湖北,洊擢守備。六年,從曾國荃援江西,克安福、吉水、萬安諸縣。七年,克峽江,擢游擊,賜花翎。八年,從攻吉安,賊出撲孚泗營,開壁奮擊,斃悍賊多名。旋克吉安,擢參將。九年,江西肅清,擢副將。會攻太湖,十年春,大戰小池驛,復太湖,孚泗功多,賜號勷勇巴圖魯。進攻安慶,戰菱湖,孚泗於東路橫壕倚水築新營,會擊屢破賊。分道攻安慶城外諸壘,賊援迭至,與城賊相應,更番撲官軍營壘,孚泗等且戰且築壘,賊不得逞;又偕水師副將蔡國祥截獲賊糧。八月,以地雷壞城,復安慶,以總兵記名。加提督銜,授河南歸德鎮總兵。

同治元年,國荃循江東下,孚泗為前鋒,攻拔西梁山。會水師克太平、蕪湖,破金柱關、東梁山,進克秣陵關、江心洲,乘勝逼江寧,以提督記名。李秀成來援,分黨趨江心洲截運道,孚泗等逆擊敗之。賊攻孚泗後營砲臺,相持十餘日,賊以地雷毀營墻,孚泗以火藥數十桶擲轟,賊不得入。伺賊疲,孚泗與彭毓橘突出夾擊,踏平賊壘數十,賜黃馬褂。二年,偕總兵李臣典襲克雨花臺石城,追至上方橋,斬馘數千,又破秣陵關賊卡。夜襲上方橋,結筏渡河,扼雙橋門,連破賊隘。偕彭毓橘縱火焚其橋,襲賊屯,擢福建陸軍提督。三年,既克天保城,孚泗出鍾山北,於太平門築三壘守之,絕賊糧道。

六月,進占龍膊子山石城,孚泗與李臣典築砲臺山上,距城僅十餘丈,積沙草高與城齊,作偽攻狀,潛於其下鑿地道。賊宵攻毀砲臺,副將陳萬勝戰死,明日,會師逼城下,總兵郭鵬程、王紹羲復中砲死。及地道成,火發城圮,將士爭登,賊擲火藥抵拒,死僕相繼。孚泗手刃退者數人,士氣乃奮,盡從缺口入。李秀成匿民舍,孚泗索獲之,並擒洪仁達。論功,賜封一等男爵,賜雙眼花翎。尋丁父憂歸。光緒十年,卒於家,優恤,謚壯肅。

硃洪章,字煥文,貴州黎平人。咸豐初,應募為鄉勇,從黎平知府胡林翼剿新寧竄匪,又剿黃平榔匪,擒匪首劉瞎麼,以功獎外委。四年,從林翼援湖北,會克岳州。從塔齊布攻武昌,破賊洪山,遂隸塔齊布軍。戰大冶、半壁山、田家鎮、孔壟、小池口,攻九江,無役不從,以勇名。塔齊布卒,從周鳳山。鳳山敗,隸畢金科。六年,克饒州,擢千總。金科戰歿,代領其軍。江西不給餉,張芾倚蔽皖南,資之,軍始不散。又以會攻四十里街,他將敗績,被劾,降把總。九年,從曾國荃復景德鎮,復官,以守備補用。遂從曾國荃部下,戰績始著。

十年,從攻太湖,解小池驛之圍,晉都司。進攻安慶,爭壕奪壘,斬刈甚多。十一年,克安慶,超擢參將,賜號勤勇巴圖魯。從國荃由皖東下,連奪沿江要隘,擢副將。進屯雨花臺,江寧城賊出撲,屢擊破之。及援賊至,大營被圍,迭以地雷毀營墻,悍賊擁入,口銜利刃,匍匐而進。洪章督隊發槍砲,擲火焚燒,斃賊無算,傷亡士卒甚多,久之始解。洪章先以迭克城隘,以總兵記名,至是加提督銜。

同治三年,攻江寧久不拔,及開地道於龍膊子山麓告成,議推前鋒。國荃召諸將署名具軍令狀,洪章署第一,武明良第二,劉連捷第三,其他以次署畢,共得九人。發火城崩,洪章率所部長、勝、煥字三營千五百人,從倒口首先沖入,賊倉猝從城頭擲火藥傾盆下,士卒死四百餘人。洪章入城後,結圜陣與賊排擊。諸將畢入,乃分軍為三,洪章趨中路,直攻天王府之北,短兵巷戰一日夜,搜斬逆酋尤眾,賜黃馬褂,予騎都尉世職,無論提鎮缺出,侭先題奏。初敘入城功,李臣典以決策居第一,洪章列第三,眾為不平。洪章曰:「吾一介武夫,由行伍擢至總鎮。今幸東南底定,百戰餘生,荷天寵錫,已叨非分,又何求焉?」

四年,授湖南永州鎮總兵。光緒二年,調雲南鶴麗鎮,署昭通、臨安、騰越諸鎮。鶴麗地卑下,水潦常沒民田,有新河洩水,通塞無常。洪章在鎮,躬率士卒開濬數次,水患為紓,民感之。十四年,因病乞開缺,病痊,曾國荃調留兩江,洪章憑吊龍膊子山,祭死士瘞所,國荃為樹碑紀事。十五年,署狼山鎮總兵。二十年,張之洞檄募十營防金沙衛。二十一年,卒於軍。之洞疏陳戰績,稱其收復江寧,功實第一。詔宣付史館,從優議恤,謚武慎,附祀曾國籓、國荃、胡林翼專祠。

劉連捷,字南雲,湖南湘鄉人。以外委隸同族劉騰鴻湘後營,轉戰湖北。羅澤南薦諸巡撫胡林翼,檄領副後營,擢千總。咸豐六年,從騰鴻援江西,戰瑞州,騰鴻中砲殞,連捷率所部攻城,拔之。為曾國籓所重,薦改文職,以知縣留江西補用。從曾國荃克吉安,擢同知,赴安徽助剿,十年,大捷於小池驛,擢知府。由集賢關攻安慶,破援賊。十一年,復破援賊於集賢關,克安慶,擢道員,賜號果勇巴圖魯。

同治元年,攻巢縣東關,賊立石墻於羅星山,連捷率死士夜渡河縱火燒賊營,進克西梁山、濡須口,渡江克太平府、金柱關、蕪湖,乘勝進軍江寧。連捷軍屯顏行,李秀成、李世賢糾大眾來攻,以炸砲破營壁,連捷築橫墻拒之,常乘賊懈夜出破賊壘。及賊退,以按察使記名,加布政使銜。大營恃無為州通餉道,連捷率三千人往守,營城外石澗阜。二年,李秀成困以長圍,軍糧垂盡,彭玉麟勸突圍出,連捷誓死守。彭毓橘來援,合擊賊,走之,再復巢縣、含山、和州,賜黃馬褂。偕水師進攻九洑洲、下關。

三年,龍膊子山地道成,偕諸軍沖入城。江寧平,以布政使記名,加頭品頂戴,予騎都尉世職。湘軍凱撤,曾國籓留連捷軍三千人駐守舒城、桐城防捻匪。會霆營叛卒擾江西,連捷督軍追剿,駐防吉安、贛州,會剿粵匪餘黨於廣東嘉應州,盡殲之。連捷以傷病歸,家居十載。光緒中,曾國荃撫山西,奏起連捷練軍包頭,從國荃移屯山海關,又從至江南治江防。十三年,卒,賜恤,贈內閣學士,建專祠,謚勇介。

彭毓橘,字杏南,湖南湘鄉人。從曾國荃援江西,積功敘縣丞。及進安徽,小池驛、菱湖諸戰皆有功,又屢破援賊,累功擢知府。會諸軍下沿江諸要隘,渡江克太平府、金柱關、蕪湖,擢道員,賜號毅勇巴圖魯。

大軍逼江寧,毓橘與諸將分路取丹陽鎮、秣陵關諸要隘,夷賊壘數十,進攻雨花臺石城,賊死拒未下。李秀成率眾來援,大營被圍。毓橘方染疫,力疾御戰,伺懈出擊,破賊壘。解圍後,毓橘與劉連捷援江北,合水師連復江浦、和州、含山、巢縣四城,江北大定。削平江寧附近諸賊壘,毓橘功為多。龍膊子地道火發,督軍沖入,手刃退者。論功最,以布政使記名,予一等輕車都尉世職。

尋授福建汀漳龍兵備道,未之任,曾國荃疏調毓橘統湘軍赴湖北,捻匪竄擾黃州、安陸,毓橘進剿,戰比有功。同治六年,師次蘄水,毓橘率小隊數百,周覽地勢,至麒麟凹,賊大至,被圍,搏戰,死傷略盡。毓橘馬陷泥淖,被執,罵賊被害。事聞,詔視布政使陣亡例議恤,建專祠,贈內閣學士,謚忠壯,加騎都尉世職,並為三等男爵。

張詩日,湖南湘鄉人。咸豐五年,以外委隨羅澤南戰江西,克義寧。六年,改隸曾國荃軍,克安福,戰吉安。八年,復萬安、吉水,超擢守備。九年,以克吉安及景德鎮、浮梁,累擢游擊。十年,援小池驛,復太湖、潛山,晉參將。

從攻安慶,率三營破援賊於樅陽。十一年,克安慶,擢副將,加總兵銜,賜號幹勇巴圖魯。同治元年,從克沿江要隘。及抵江寧,力守大營,破援賊。累擢,以提督記名。二年,屢破江寧城外賊壘,賜黃馬褂。

三年,克天保、地保兩城。方開掘龍膊子地道,李秀成夜自太平門突出來犯,又詐為官軍,別從朝陽門東隅出,偪營縱火,詩日偕諸將力戰卻之。地道火發,城崩,詩日率士卒登龍廣山,奪太平門;復循神策門轉戰至獅子山,奪儀鳳門。論功最,予一等輕車都尉世職。

四年,授直隸宣化鎮總兵。五年,從曾國籓剿捻,破張總愚、牛洛紅於西平,又敗之萬金寨,進攻雙廟賊巢。賊以馬隊襲官軍後,詩日分軍回擊,追敗之洪河,又敗之郾城、召陵。因傷發回籍,六年,卒。曾國籓疏陳詩日克復江寧,當西北一路,論功在李臣典、劉連捷、蕭孚泗之次,優恤,謚勤武。

伍維壽,湖南長沙人。從曾國荃援江西,攻安慶,克沿江要隘,擢副將。奪雨花臺、聚寶門外石壘,累擢記名總兵,賜號毅勇巴圖魯。偕硃南桂破神策門,入城,率馬隊追逸賊至湖熟鎮,擒斬賊酋李萬材等,以提督記名,賜黃馬褂,予騎都尉世職。六年,授陜西漢中鎮總兵,調甘肅寧夏鎮,光緒元年,卒。

硃南桂,湖南長沙人。羅澤南舊部,轉戰兩湖,積功至副將,賜號勖勇巴圖魯。同治元年,解金柱關圍。二年,克薛鎮、博望鎮,以總兵記名。及克江寧,南桂先破神策門月城,梯而入,賜黃馬褂,予雲騎尉世職。尋授河南歸德鎮總兵。五年卒,賜恤,謚勤勇。

羅逢元,湖南湘潭人。由武生入伍,從剿廣西。曾國籓治水師,充營官,轉戰湖北、江西,累擢副將。繼從曾國荃克安慶,以總兵記名,賜號展勇巴圖魯。進克沿江要隘,抵江寧,還守太平,屯金柱關。賊酋陳坤書大舉來犯,逢元堅守,屢戰,以少擊眾,擒斬逾萬,以提督記名。及克江寧,由南門舊陷口梯登,賜黃馬褂,予雲騎尉世職。以傷發假歸,光緒四年,卒,賜恤。

李祥和,湖南湘鄉人。初從羅澤南,積功至游擊。嗣從曾國荃,克吉安,復安慶,累擢副將,賜號著勇巴圖魯。同治元年,從大軍進江寧,力守大營,破援賊,以提督記名。三年,攻克地保城,先登,築砲臺俯擊城中,地道始成。論功,賜黃馬褂,予雲騎尉世職。四年,授安徽壽春鎮總兵,從劉松山赴陜西剿捻匪。六年,戰洛川大賢村,中砲陣亡,賜恤,謚武壯。

蕭慶衍,湖南湘鄉人。應募入湘軍右營,轉戰江西、湖北,積功至副將。克太湖、潛山,以總兵記名,賜號剛勇巴圖魯。同治二年,援江浦,復含山、巢縣、和州,加頭品頂戴。三年,渡江會攻江寧,克上方橋,進鍾山,築三壘太平門外。城破,於缺口沖入,奪朝陽、洪武二門,賜黃馬褂,予雲騎尉世職。

吳宗國,湖南長沙人。以勇目從剿湖北,累擢守備。同治元年,從曾國荃沿江東下,迭克要隘,功多,累擢參將,賜號資勇巴圖魯。二年,破聚寶門外上方橋、江東橋各賊壘,會水師破九洑洲、印子山賊巢,擢副將。從李臣典重開地道,城賊防益嚴,砲彈雨下。宗國手藤牌,持長繩,冒砲彈,狙行而前,抵城下測丈而返,始興工開掘。大功克成,以提督記名,予一品封典。五年,偕提督郭松林剿捻德安,戰羅家集,中伏,歿於陣,依提督例賜恤,予騎都尉世職。

論曰:洪秀全踞江寧十有餘年,曾國荃於蘇、浙未定之先,孤軍直搗,城大援眾,事勢綦難。及援賊既破,困獸之斗,人人致死,歷兩年之久,竟蕆大功。固由指揮素定,而在事諸將,同心效力,奮不顧身,其勇毅之烈,紀太常而光紫閣,無愧色焉。掇其尤者著於篇。


\end{pinyinscope}