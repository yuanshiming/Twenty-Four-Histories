\article{志十四}

\begin{pinyinscope}
天文十四

△客星流隕雲氣

客星太祖丁未年九月丙申,彗星見東方。

天命三年十月丙寅,彗星見東方,尾長五丈,每夜漸移向北斗,十九日而沒。

順治九年十一月庚寅,異星蒼白氣見於參,西北行入畢。

康熙三年十月己未朔,有星茀於軫,見東方;丁卯,尾長七八寸,蒼色,指西南;丁亥,尾長三尺餘,指西北,逆行至翼;十一月戊戌,尾長五尺餘,指北方,至張;庚子,至井;癸卯,往西北行至昴;乙巳,尾指東北,至胃;庚戌,至婁,尾指東,青色;十二月壬戌,至奎,體小,尾長二尺餘。四年二月己巳,東南方有異星見於女;甲戌,尾長七寸,指西南,蒼白色;丁丑,尾長尺餘,往東北順行至虛;辛巳,至室,體漸大,尾長八尺餘;乙酉,至壁,尾長五尺餘。七年正月甲子,西南白光,長六尺餘,尾指東南,占曰天槍;二月乙亥,漸長至四丈餘,尾掃天苑、九斿、軍井;丁亥,沒。十二年二月癸巳,異星見於婁,大如核桃,色白,尾長尺餘,指東方;甲午,仍見。十五年正月戊子,異星見於天苑東北,色白。十六年三月癸卯,東北方有異星見於婁,體色光明潤澤,尾長尺餘,指西南,占曰含譽。十九年十月戊子,彗星見右執法,色白,尾長尺餘,指西方,東行甚速;壬寅,近太陽不見。十一月丙辰朔,尾跡夕見西方;壬戌見星體,色蒼白,尾長六丈餘,寬二尺餘,指東北。二十一年七月己巳,彗星見北河之北,色白,尾長二尺餘,指西南,往東北行甚速;壬申,入午宮,尾長六尺餘。二十二年閏六月庚戌,異星見於五車北八穀東,色白,往西南逆行;戊辰,入五車。二十三年五月甲申,異星見太微垣,東屬軫,色白明大,往東北順行;乙酉,行四度餘,至右攝提下。二十五年七月庚寅,異星見東方,近地平,色白,東行不急;丁酉,凡行十六度,至柳,微有尾跡;壬寅,至星,漸沒。二十七年十月己酉,異星見奎,色白,凡三夜。二十九年八月己酉,異星見箕,色黃,凡二夜。

雍正元年九月己丑,異星見弧矢下,色白,體微,芒長尺餘,指西北,逆行至井。

乾隆二年六月丁卯,異星出右更東,色白,屬婁,向西南行;丙子,仍見。七年正月丁亥,異星見東南方;戊子,出地二十七度餘,大如彈丸,色黃,尾長四尺餘,指西南,屬醜宮,在天市垣徐星外,逆行四旬餘不見。八年十一月己亥,彗星見奎、壁之間,大如彈丸,色黃白,尾長尺餘,向東指,屬戌宮,逆行至九年正月辛卯,凡五十三日,行二十九度餘。十三年三月癸丑,異星見東方,大如榛子,色黃,尾長二尺餘,向西南指,在離宮第三星南,順行至四月甲寅朔,行三度,尾長尺餘,體小光微;壬戌,至螣蛇;乙丑,至王良;丙寅,不見。十四年五月甲寅,瑞星見東方,大如雞卵,形長圓,色黃白,光瑩潤澤,行不急,出天津,入芻★,占曰含譽。二十四年三月壬辰,彗星見東南方;甲午,出虛第一星下,大如榛子,色蒼白,尾長尺餘,指西南,順行;癸卯,體小光微,尾餘三四寸;戊申,全消。四月戊辰,彗星見西南方,在張第二星上;己巳,離張六度,大如榛子,色蒼黃,尾光散漫,長二尺餘,指東南,順行;壬申,形跡微小;丁丑,更微;己卯,漸散;五月壬午,全消。十一月戊辰,異星見東南方,在井第四星下,大如榛子,色蒼黃,向西北行;癸酉,行四度,在胃,微有尾跡;十二月丁丑朔,全消。三十四年七月甲辰,彗星見東南方,在昴下;丁未,大如彈丸,色蒼白,尾長三尺,指西南,順行甚速;八月丁卯,與太陽同宮不見;十月辛亥,見西方,在列肆第二星下,體勢微小,尾長一尺;丙子,全消。三十五年閏五月己酉,異星見東南方,在天弁第一星西,大如彈丸,色蒼黃;癸丑,向北行三十二度;乙丑,不見。十一月乙丑,彗星見東南方,長尺餘;丙寅,在柳第二星下;戊辰,色蒼白,尾指東南,每日向西北行十餘度;庚午,微暗;辛未,全消。

流隕隕星如斗者,太祖戊子年九月辛亥朔夜,時徵王甲城,士馬皆驚。

流星如盆者,乾隆十四年九月壬申,出婁宿,色赤,入天苑,有光,有尾跡。

流星如★者,順治四年十一月庚辰,自天中西北行入蜀,有聲,色赤,光燭地,★犬皆驚。五年九月辛巳如之,聲如雷。十五年六月辛未,自西北至東南,有聲,色赤,不著光、尾跡。

流星如盞者,順治四年五月戊午寅時,自西北至西南,色青白,有光。七年八月甲午,自東南至東北,色赤黃,入斗,不著光、尾跡。八年四月己酉,自氐宿南行,色青白;五月戊寅,自亢宿西南行,色白,★小星隨之,入翼;九年三月丙子,自中天西南行,色赤,入畢;俱有光,有尾跡。九月丙申,自中天入紫微垣,色赤;十年八月丙寅,自中天入天市垣,色青赤;俱有光。十二年四月甲子,自亢宿入危,色赤黃,有光,有尾跡。癸酉,出房宿,色青黃,入尾;十六年七月甲申,出牛宿,東北行,色赤黃,至蜀沒;俱不著光跡。康熙二年八月丁巳,自虛宿入紫微垣,小星隨之;三年九月戊申,自中天入奎;俱色赤;六年正月戊寅,出鬼宿,色青,隨後有聲,入土司空;七年二月戊子,出大角,色赤黃,入箕;十二年九月甲午,出勾陳,色青白,至蜀沒;俱有光、有尾跡。十六年九月己亥酉時,自正北下行,色赤白,尾跡如蛇,有光。十九年六月癸酉酉時,自西南向東北,聲如雷,尾跡如匹布。二十五年十一月壬午,出左樞,色白,至蜀沒,尾長竟天。十二月戊寅,出軫宿,色青黃,入騎官。二十九年二月丁亥,出河鼓,色黃,入尾。乾隆十九年正月丁巳一更,出奎宿,西北行;二十二年三月戊申一更,出西北方,下行;俱色青。二十四年閏六月甲申五更,出土司空,下行;三十五年九月戊辰三更,出室宿,西北行;俱色赤,俱入雲。三十六年二月己丑昏,出上弼,下行,色黃,不著入。三十九年十月丙戌二更,出天廩,西行;四十年九月丁未一更,出壘壁陣西,下行;四十一年三月丁丑昏,出翼宿,西南行;十月丁卯曉,出平道,下行,俱色赤。十二月癸丑,出天苑,下行;四十二年八月壬戌一更,出右旗,西行;俱色白。五十四年十二月己卯昏,出參宿,下行;五十五年九月壬寅一更,出五車,西行;五十六年三月庚寅五更,出天津,下行;五十七年五月丙辰一更,出天棓,東北行;五十八年九月己亥五更,出婁宿,西南行。五十九年十月丙寅曉,出張宿,下行;六十年閏二月戊戌三更,出大角,西北行;俱色赤,俱入雲,俱有光、有尾跡。

流星如飲鐘者,康熙八年九月乙卯巳時,出午位,色赤黃,入巳位,不著光、尾跡。

流星如杯者,乾隆十八年七月甲戌三更,出奎宿,東南行,色赤,入雲,有光,有尾跡。

流星如桃者,順治五年八月癸巳朔,自中天東北行,不著色,入天關。十三年正月癸卯,自奎宿入天中,色黃白,俱有光、有尾跡。康熙二年八月乙巳,自中天至心,不著色、光、尾跡。四年六月壬申,出建星,入南斗;辛巳,出天咅,入河鼓;又出閣道,入離宮,俱色赤。壬午,出庶子,入開陽,色赤黃。九月甲申朔,出女宿,入羽林軍,小星隨之,不著色。十二月壬申,出南河,入柳,小星隨之;五年正月己酉,出參旗,入天苑,俱色青白。二月戊午,出五車,至蜀沒;五月乙酉,出勾陳,入大陵,俱色赤;十月戊午,出少宰,入天棓,色黃;俱有光、有尾跡。六年二月庚戌,出氐宿,入大角,色黃,有尾跡。八年四月癸亥,出天弁,入氐,色青白,有光,尾跡先直後曲,留東,咸結為雲氣,如魚形,向東散。十年正月己未,出勾陳,入華蓋,色黃白,有光。十一年七月辛未,出東井,入畢,色青黃,有光,有尾跡。十三年三月甲申辰時,自西北至西南,色白,有光。十五年九月丁未,出外屏,入墳墓,色青黃,有光,有尾跡。十六年四月丁未朔,出紫微垣,在雲中,往北行,不著色,映地有光。十七年九月辛丑,出昴宿,入閣道,色青白,有光,有尾跡。十八年七月己未,出勾陳,入文昌,色青黃,有光。十九年五月壬辰,出攝提,入房,色青黃;閏八月己酉,出外屏,入建星,前小後大,色赤黃;十二月甲午,出勾陳,入大陵,色青;俱有光、有尾跡。二十一年正月戊辰,出大陵,入壁,色青白,有光。二十二年二月丁丑,出明堂,入軫,色青白,有光,有尾跡。二十三年二月己丑,出七星,入地,小星隨之,色青,有光。二十九年二月丁亥,出郎位,入軒轅;八月乙亥,出參宿,入弧矢;三十二年二月癸卯,出房宿,入尾;俱色青。三十三年三月壬戌,出女宿,入危,色赤;三十六年十月丙辰,出五車,入弧矢,色白;俱有光、有尾跡。

流星如★子者,乾隆十二年十月戊辰二更,出閣道,東北行,色青;十四年二月乙酉昏,出王良,下行,色赤;五月庚申曉,出織女,西北行,色青;俱入雲,有光,有尾跡。

十五年正月壬申二更,出天槍,西北行,色青,入雲,有光。五月戊午曉,出天船,下行,色青,入雲;八月戊子曉,出天狼,東行,色赤,入柳;十六年八月丙申二更,出鬥宿,下行,色青,入雲;十七年六月丁巳昏,出女★,西北行,色赤,入天理;戊午昏,出織女,東南行,色青,入河鼓;十八年六月乙酉曉,出河鼓,南行,色白,入雲;俱有光、有尾跡。己酉昏,出東南雲中,下行,色赤,入斗,有光。

十九年正月癸亥曉,出南河,下行,色赤;二十年五月甲午昏,出亢宿,東南行,色青;俱入雲。

二十一年六月甲子一更,出河鼓,西北行,色赤,入貫索。七月辛未三更,出宗正,西行,色青。十一月丙申四更,出文昌,西北行;二十三年七月戊子二更,出王良,下行;十一月壬辰一更,出左樞,西北行;十二月辛酉五更,出南河,下行;俱色赤。

二十四年正月癸未朔二更,出弧矢,西南行,色青;二月庚辰一更,出柳宿,西南行,色赤;俱入雲,俱有光、有尾跡。閏六月乙酉五更,出天倉,下行,色赤,入雲,有光。七月丙寅二更,出奎宿,下行;己巳二更,出勾陳,下行;二十五年六月辛丑昏,出王良,南行;俱色赤;七月己酉一更,出危宿,下行,色青;俱入雲,有光,有尾跡。

二十六年二月己卯昏,出外屏,下行,色黃,入雲,有光。辛卯二更,出五帝座,東南行,色青,入雲,有光,有尾跡。九月丁巳二更,出虛宿,下行,色赤,入雲,有光。

二十七年正月乙未朔二更,出中臺,東北行,不著色。癸丑曉,出天棓,下行,色赤。

二十八年二月庚戌一更,出西方雲中,下行,色黃;六月壬子一更,出天廚,西南行;九月戊寅四更,出天市垣市樓,東行;二十九年四月庚寅昏,出四輔,西北行;七月辛酉曉,出閣道,南行;俱色赤。八月庚辰朔一更,出天錢,下行,色青;三十年閏二月庚午二更,出軒轅,東北行,色赤;俱入雲,有光,有尾跡。六月丁卯昏,出東北雲中,東南行,色黃;一更出天津,東行,色赤;俱入雲,有光。七月壬午曉,出王良,西行,色青。九月庚子五更,出王良,下行;十月戊辰五更,出中臺,東南行;俱色赤,俱入雲。

三十一年六月甲子曉,出壁宿,西南行,色赤,入羽林軍。十月庚子二更,出天津,下行,色青。丙午五更,出壁宿,西行;曉出南河,下行;己未曉,出軍市,東行;十一月甲戌五更,出文昌,下行;三十二年二月甲寅五更,出角宿,東南行;俱色赤,俱入雲,俱有光、有尾跡。六月庚申昏,出西南雲中,下行,色赤,入雲,有光。閏七月癸巳曉,出八穀,東北行;九月庚申二更,出瓠瓜,西南行;十月癸亥二更,出天津,下行;己巳一更,出昴宿,東南行;庚午五更,出五車,北行;俱色赤。

三十三年七月己丑曉,出河鼓,西北行,色黃。乙卯一更,出鬥宿,下行;八月辛酉一更,出左樞,下行;俱色青。乙亥一更,出天槍,下行;九月丙午五更,出壁宿,下行;俱色赤。丁未二更,出天苑,下行,色黃。庚戌曉,出五車,西南行;十月壬戌二更,出五車,下行;十一月乙酉朔曉,出天狼,下行;俱色赤。

三十四年三月戊子曉,出庫樓,下行;五月丁亥二更,出天廚,下行;俱色青。七月辛卯三更,出開陽,下行,色赤。辛丑昏,出左旗,西南行,色黃。八月乙卯曉,出天苑,下行;辛未昏,出鬥宿,下行;俱色青。十二月癸酉五更,出井宿,下行;三十五年正月壬寅曉,出螣蛇,東行;癸卯五更,出帝座,西南行;二月辛亥一更,出北河,東南行;丁卯五更,出大角,西北行;三月丙申二更,出大角,東北行;七月丁未二更,出天市垣梁星,西北行;九月乙丑曉,出五車,下行;辛未二更,出天棓,下行;十月丙子曉,出軒轅,東南行;三十六年正月庚午二更,出南河,下行;俱色赤。十月戊辰朔五更,出畢宿,南行,色黃。十一月壬戌五更,出鬼宿,西北行;三十七年七月丙辰昏,出天弁,下行;俱色赤。十一月甲辰曉,出柳宿,東南行,色蒼白。十二月庚辰五更,出貫索,下行,色黃。三十八年正月庚子曉,出氐宿,西行,色赤。九月乙丑昏,出天桴,南行,色蒼白。丁丑二更,出參宿,下行;十月乙巳一更,出女宿,下行,俱色黃。戊申一更,出壘壁陣,西行,色赤。

三十九年三月乙丑曉,出角宿,下行,色黃。七月戊寅昏,出勾陳,西行;九月庚午二更,出八穀,下行;癸酉二更,出天囷,下行;俱色赤。十月乙酉二更,出右樞,下行,色白。丙戌五更,出奎宿,下行,色赤。丁亥五更,出天廩,南行;十二月辛巳昏,出漸臺,下行;俱色黃。四十年四月乙巳昏,出勾陳,西行;五月甲戌昏,出上臺,下行;六月戊寅五更,出虛宿,下行;甲辰五更,出瓠瓜,下行;七月丙辰五更,出王良,下行;丁巳昏,出勾陳,南行;戊午二更,出奎宿,西行;俱色赤。八月丙子朔四更,出奎宿,西行,色蒼白。丁丑三更,出昴宿,下行,色黃。九月丙午朔三更,出天廩,下行,色赤。己巳三更,出羽林軍,下行,色蒼白。十月癸巳一更,出昴宿,下行,色白。甲午三更,出五車,南行;十二月丙辰一更,出卷舌,北行;俱色黃。辛酉昏,出北河,西南行;四十一年三月癸酉一更,出五車,東南行;戊子二更,出帝座,下行;俱色赤。四月己巳二更,出尾宿,西行,色白。五月甲戌曉,出離宮,南行,色赤。戊戌昏,出女宿,下行;曉出閣道,下行,俱色白。六月壬寅曉,出天津,西南行;戊申一更,出室宿,下行;俱色赤。乙丑二更,出天津,西南行,色青。七月甲申四更,出畢宿,下行;壬辰四更,出霹靂,下行;癸巳二更,出奎宿,南行;甲午昏,出閣道,下行;俱色赤。一更出昴宿,下行,色白。己亥一更、五更俱出奎宿,南行;九月乙未昏,出左旗,西北行;俱色赤。十月乙巳曉,出屏星,下行,色白。甲子三更,出庶子,下行,色黃。乙丑三更,出左樞,西北行,色白。丁卯五更,出右執法,下行;四十二年三月己巳昏,出北斗天樞,西北行;俱色赤。甲午昏,出軒轅,西北行,色白。四月戊申曉,出左旗,東北行。五月甲戌一更,出天市垣鄭星,東行;六月癸卯四更,出離宮,西南行;俱色赤。七月己巳二更,出女宿,下行,色黃;三更出天紀,南行,色赤;曉出天船,南行,色黃;出參宿,下行,色白。庚午二更,出貫索,下行,色赤。丙戌二更,出天市垣蜀星,西北行,色白。癸巳一更,出左樞,西行。八月庚戌四更,出天囷,北行;癸丑二更,出天溷,下行;俱色赤。十二月戊午二更,出天倉,下行,色黃。

四十三年二月丙辰五更,出七公,東北行,色赤。丁巳一更,出奎宿,下行,色黃。四月丙辰二更,出右旗,下行;五月丙寅曉,出奎宿,西南行;俱色赤。六月戊寅二更,出閣道,西行,色黃。八月乙丑五更,出井宿,東南行;十月乙亥三更,出卷舌,北行;十一月戊子曉,出右攝提,西南行;俱色赤。壬辰一更,出上★,西行,色白。癸丑曉,出翼宿,南行,色黃。丙辰二更,出昴宿,西北行;十二月戊辰二更,出文昌,北行;癸酉一更,出天樞,西行;戊寅一更,出天權,下行;二更出勾陳,下行;四十四年正月辛丑一更,出天權,西北行,俱色白。甲辰四更,出南河,下行,色赤。戊申一更,出上★,下行,色黃;四月辛巳五更,出天市垣趙星,西行,色赤;俱入雲。六月丙子二更,出西北雲中,東南行,色白,入室宿。八月戊寅一更,出羽林軍,下行,色白。九月庚戌曉,出軍市,下行,色赤。

四十五年二月庚申一更,出參宿,西行,色黃。丁丑三更,出軫宿,西行;七月乙未三更,出天津,北行;八月壬子曉,出上★,下行;俱色赤。庚午二更,出王良,西南行,色黃。九月辛卯曉,出畢宿,西南行;十月甲寅昏,出織女,下行;俱色赤。己未二更,出天樞,西行,色白。十一月辛巳二更,出王良,東南行,色黃。四十六年五月庚寅一更,出河鼓,下行,色赤。六月辛卯二更,出勾陳,西南行;乙未昏,出房宿,西行;俱色黃。九月丙寅二更,出文昌,下行;丁卯二更,出婁宿,西南行;十月辛未二更,出玉井,下行;俱色赤。丙申五更,出天樞,下行,色黃。戊戌曉,出玉井,下行;十一月癸卯三更,出大角,南行;甲辰曉,出畢宿,北行;俱色赤。

四十七年六月己巳昏,出貫索,西南行,色白。乙未四更,出奎宿,西南行,色赤。七月戊戌一更,出壁宿,南行,色黃。九月壬寅曉,出郎位,下行;癸卯昏,出貫索,下行;十月己巳三更,出昴宿,東行;俱色赤。

四十八年四月壬戌三更,出瓠瓜,下行,色白。乙丑一更,出五帝座,西北行;五月庚子四更,出天棓,東北行;七月丙午昏,出文昌,下行;九月庚寅一更,出天船,下行;十月甲戌一更,出土司空,下行;四十九年正月丁亥朔曉,出天槍,東北行;甲寅一更,出天槍,下行;閏三月壬午二更,出天津,下行;四月丁酉一更,出開陽,西北行;六月丁亥五更,出壘壁陣東井,西行;壬辰二更,出危宿,下行;甲辰昏,出天棓,下行;七月丁巳昏,出開陽,下行;俱色赤。十二月壬午二更,出織女,下行,色白。

五十年三月己卯五更,出左執法,下行;五月甲戌二更,出天津,下行;八月壬午曉,出瓠瓜,下行;庚寅四更,出牛宿,下行;戊戌五更,出勾陳,西行;十月乙巳四更,出五車,東南行;五十一年閏七月丙申一更,出天廚,西南行;十月辛丑朔昏,出危宿,下行;己未一更,出王良,東南行;丙寅四更,出大陵,下行;五十二年五月戊子五更,出螣蛇,南行;八月己亥三更,出五車,下行;辛亥五更,出壁宿,西南行;辛酉二更,出天倉,下行;五十三年四月乙巳二更,出文昌,下行;六月辛丑二更,出箕宿,下行;己未三更,出奎宿,下行;七月壬戌五更,出織女,下行;己巳二更,出壁宿,下行;丁亥三更,出貫索,下行;十月乙未四更,出井宿,下行;五十四年五月己未昏,出天棓,下行;閏五月戊戌曉,出尾宿,下行;辛丑昏,出天津,下行;五十五年四月癸丑四更,出文昌,下行;五月癸未昏,出天棓,下行;俱色赤。六月癸亥五更,出尾宿,下行,色白。七月壬午二更,出帝座,西北行;丙戌四更,出五車,下行;八月丁巳一更,出王良,下行;戊辰曉,出文昌,下行;十一月甲申昏,出墳墓,下行;十二月丁卯一更,五十六年四月丙午曉,俱出文昌,下行;己酉曉,出郎位,下行;五月丁酉五更,出貫索,下行;七月丁丑三更,出室宿,下行;壬午曉,出奎宿,西南行;癸未昏,出文昌,下行;二更出天津,南行;三更出壁宿,下行;五更出危宿,下行;乙未昏,出文昌,下行;八月壬申曉,九月戊寅四更,俱出天倉,下行;丁亥昏,出文昌,下行;十月壬寅朔五更,出軒轅,東南行;癸丑四更,出畢宿,下行;俱色赤。十一月己卯昏,出墳墓,下行,色黃。十二月辛酉昏,出文昌,下行;五十七年三月丁酉五更,出貫索,東北行;六月戊寅昏,出織女,西北行;丁亥昏,出亢宿,下行;庚寅曉,出室宿,下行;壬辰五更,出五車,下行;甲午曉,出昴宿,下行;九月辛丑五更,出天倉,下行;辛亥昏,出文昌,下行;乙丑曉,出軒轅,東南行;十月丙子四更,出畢宿,下行;五十八年三月戊申曉,出室宿,下行;六月甲申二更,出大陵,下行;戊子五更,出天紀,下行;七月甲午四更,出七公,下行;丙申二更,出貫索,下行;戊申昏,出織女,西南行;甲寅昏,出貫索,下行;丁巳一更,出危宿,下行;八月戊辰三更,出室宿,東南行;九月甲午一更,出閣道,下行;己酉昏,出壁宿,下行;辛亥昏,出鬥宿,下行;丁巳一更,出室宿,東南行;十月壬午二更,出危宿,下行;五十九年六月乙酉一更,出王良,下行;十月乙卯朔一更,出天棓,下行;己巳四更,出五車,東北行;六十年二月丁巳昏,出王良,下行;七月庚午二更,出昴宿,下行;俱色赤,俱入雲,俱有光、有尾跡。

流星如李者,康熙七年四月乙亥,出右執法,入翼;十二月癸酉,出參宿,入軍市;俱不著色。八年十月己丑,出伐星,色青白,入天狼。九年六月庚戌,出離宮,入虛;十年九月戊寅,出室宿,入羽林軍;俱不著色,俱有光、有尾跡。十二年八月丙午,出螣蛇,色青白,入心,微有尾跡。十四年十一月戊子,出張宿,色青白,入天廟,有光。十六年八月甲寅,出常陳,色青赤,入氐。十七年正月丙子,出參宿,色青白,入九斿。十八年十月庚午,出右旗後,小星隨之,色青赤,入候星。二十一年六月乙巳,出天市垣,色青白,入心、尾之間。十一月戊申,出東井,色赤黃,入上臺。二十四年三月戊辰,出建星,色白,入壘壁陣。二十五年九月壬午朔,出胃宿,色赤,入東壁。二十六年七月癸未,出壘壁陣,色青白,入天紀,自東南至西北竟天;二十八年二月乙卯,出東次將,色白,入氐;三十年十月丁未,出胃宿,色白,入天倉;三十一年正月己卯,出貫索,色青,入亢;九月癸丑,出東井,色白;入天苑,三十二年七月辛亥,出王良,色黃,入五車;三十五年十月甲辰,出少微,色青,入庶子;三十八年十一月乙未,出勾陳,色赤,入王良;四十七年九月戊戌,出內屏,色青,入文昌;五十三年八月壬申,出蜀星,色赤,入尾;己卯,出牛宿,色青,入南海;五十六年十二月丁未,出畢宿,色青,入天倉;俱有光、有尾跡。六十年十一月丙午未時,自西北至東南,色赤,有尾跡。雍正元年三月壬午,出左樞,色青,入天津;二年四月庚戌,出左執法,色赤,入角;俱有光、有尾跡。

流星如核桃者,乾隆八年八月乙卯未正,出東北雲中,下行,色黃,入雲,微有尾跡,以晝見。其餘乾隆年間一千五百有餘,皆以昏、曉及夜見。

流星如慄者,康熙十一年五月壬子,出天★,入奎,有光,有尾跡。

流星如彈丸者,康熙十七年五月庚申辰時,出西南,色赤,有光。七月戊午酉時,出西北,色青。乾隆元年五月壬戌午正,自西南方下行,色黃。七月癸卯戌初初刻,出東北,高五十餘度,下行至二十餘度沒,色白,有光,有尾跡,皆以晝見。其餘順治年間五,康熙年間六十二,雍正年間一十三,乾隆年間三千一百有餘,皆以昏、曉及夜見。

流星如榛子者,乾隆年間一十四,皆以夜見。

雲氣太祖壬子年九月癸丑,東方有藍、白二氣。癸丑年九月庚辰,日傍有青、紅二氣,對照如門,祥光四★。乙卯年三月甲戌,有黃氣■天,人面★之皆黃。十月戊申,有紅、綠祥光二道夾日,又有藍白光一道,掩★日上如門。天命三年正月丙子,有黃氣貫月中,其光寬二尺許,約長三四丈。四月壬子,有藍、黑氣二道,自西而東,橫■於天。五月乙卯,有紅、綠、白三氣,自天下垂,覆營左右,上圓如門。九月甲寅,東南有白氣,自地沖天,末偏銳如刀,約長十五丈,凡十六日而滅。五年三月癸丑夜,有藍、白二氣,自西向東,繞月之北,至南而止。

天聰五年八月丁卯,明兵來攻阿濟格貝勒,大霧不見人,忽有青藍氣自天沖入敵營,霧忽中開如門,我兵遂克。崇德六年九月辛巳黎明,東方有金光大如斗,內復有金光一道直如椽,沖天而起。

順治元年六月庚午酉時,有白氣自西南至東北。十月壬辰,五色雲見日上。三年正月壬戌,北方雲中有赤光如火影。四年五月庚戌,有白氣自西南至東北。十月壬辰,五色雲見日上。十二月壬辰如之。八年十二月壬子夜,有白氣從艮至乾。十年六月丙申,青赤氣生日上下。十二年六月庚午,北方有青黑雲氣,變幻如龍。

康熙三年十二月甲戌,金星生白氣一道,長三丈餘。五年二月庚申亥時,中天蒼白氣四五道。三月庚寅酉時,東南黑氣一道。六年八月己亥,有白光一道,自東至西。七年正月甲子酉時,西南白光一道,尾至東南入地,約長六尺餘,十餘日漸長至四丈餘,掃天苑、天斿、軍井。八年六月甲申,西北直氣一道。十一年二月甲午,五色雲見中天,歷巳至申。乙未如之。六月戊子,五色雲見日上。十二月癸卯,五色雲見日旁。十二年正月庚辰,西北至東南,蒼白氣經天如匹布。十三年六月己巳夜,東北蒼白雲一道。七月甲戌,白氣一道貫日,自南至北,長六丈餘。十五年三月乙酉,五色雲見中天。七月戊戌、庚戌皆如之。十六年三月甲辰,五色雲見日傍。七月癸未、十七年二月戊辰皆如之。六月辛巳,青氣一道,寬五尺餘。壬午,蒼白氣一道,青氣三道,寬尺餘。癸未,青氣一道,寬六尺餘,俱自西北至東南。十八年八月乙丑,正北黃黑雲一道,變赤黃色,寬四尺餘,長數丈。十九年十一月戊午至辛酉,西南蒼白氣一道,寬尺餘,銳指東北,長三丈餘,漸長至四丈餘。二十年六月辛卯,東北青氣六道。十月癸未,正北黑雲一道,穿北斗,約長三丈餘。二十四年十月丙午,日上蒼白雲★出五色鮮明。三十五年五月戊辰,五色雲見中天。四十一年二月甲寅酉時,西南白雲一道,長二丈餘,寬尺餘,穿天倉、天苑,入地平,至丁巳,長三丈餘。六十一年十一月癸卯,五色雲見日上。

雍正元年九月丁丑朔,五色雲見中天。十月辛未、二年正月辛巳皆如之。五年八月辛亥醜時,正北黑雲一道,東西俱至地平,寬尺餘。七年三月戊辰,五色雲見日旁。十一月丙申,慶雲見於曲阜,環捧日輪,歷午、未、申三時,於時上發帑金修建闕里文廟。八年正月辛巳,五色雲見日下。六月辛丑子時,正北至東南,黑雲一道,寬尺餘。九年九月乙酉丑時,西北至東西,白雲二道,寬尺餘。十三年正月,雲南奏報,年前十月二十九日,大理等府五色雲見;廣東高州府如之。

乾隆元年十月壬戌未時,五色雲見日上及旁。癸亥未時,乙丑辰、巳二時皆如之。二年正月辛卯子時,西南至東北,黑雲一道,寬一尺。三年七月己巳卯時,西北白雲一道,寬三寸,長一丈餘,往西南行。四年三月乙丑寅時,東南雲一道,寬尺餘,長數丈。丙寅巳時,北方白雲一道,寬七八寸,長三丈餘。八月乙未,北方白雲一道,寬尺餘,自東至西。五年三月丁巳亥時,東南白雲一道,寬尺餘,長三丈餘。五月辛酉亥時,東南黑雲一道,寬尺餘,長二丈餘。七年正月戊寅子時,月下白雲一道,寬尺餘,長三丈餘。二月丁酉午時,五色雲繞日。戊申巳時,見日旁;亥時,北方白雲一道,寬二尺餘,自東至西。八月己酉子時,東方白雲一道,寬尺餘,長五尺。甲寅巳時,五色雲捧日。九月甲子午時繞日。十月庚寅辰時、丁酉辰時、八年三月丙辰辰時皆如之。己巳巳時見日上,丁丑酉時如之。閏四月辛酉夜子時,月上白雲一道,寬尺餘,自西北至東南。戊寅辰時,五色雲見日旁。六月甲子酉時,見日上。戊辰未時,繞日。甲戌巳、午二時如之。七月丁酉子時,繞月。戊戌子時,北方白雲一道,寬尺餘,自東至西。乙巳午、申二時,五色雲繞日。八月丁卯,見月上。戊辰亥時,繞月。十月丙辰巳時,見日下。丁巳申時,繞日。壬戌巳時,見日下。九年正月乙巳辰時至午時,見日上。五月癸未戌時,繞月。十一年七月乙巳亥時,中天白雲一道,長丈餘。十二年六月辛酉丑時,西南至正東,黑雲一道,寬三尺餘,俱至地平。丁亥,五色雲繞日。十四年二月庚辰子時,東南黑雲一道,寬二尺餘,長十丈餘。十一月戊申卯時,東方白雲一道,寬尺餘,長丈餘。十八年二月丁亥朔申時,五色雲見日上。十九年四月丙申子時,中天白雲一道,自東南向西,寬尺餘,長二丈餘。二十一年五月辛巳亥時,東南白雲一道,寬尺餘,長數丈。閏九月乙卯醜時,東北至西北如之。


\end{pinyinscope}