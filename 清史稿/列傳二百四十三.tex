\article{列傳二百四十三}

\begin{pinyinscope}
馬如龍和耀曾楊玉科李惟述蔡標段瑞梅夏毓秀

何秀林楊國發張保和

馬如龍,雲南建水人,本名現,回中世族。以勇聞。咸豐間,滇回俶擾,如龍以武生起澂江,自立為偽帥。時杜文秀僭號大理,如龍遣使與通,授以偽職,不受,始有卻。遂據有新興、昆陽、晉寧、呈貢、嵩明、羅次、易門、富民,入寇省城,勢駸盛。同治元年,巡撫徐之銘復主撫議,提督林自清臨陣宣播朝威,招之歸款,如龍自稱三世效忠,願反正。岑毓英單騎往諭,如龍益心折,與盟南門外,悉反侵地。朝旨破格授如龍總兵,楊振鵬等分署六營武職。

是時,臨安獨撓撫局,如龍怒,率師鼓行而南,戰失利,署臨元參將梁士美奪其旗鼓,如龍被創,僕,左右負以奔。總督潘鐸嚴檄其撤兵,如龍阻於士美軍,弗能達。明年,授鶴麗鎮。會回弁馬榮據省城,鐸被害。如龍聞警,即致書士美,約共釋私仇,雪公憤,士美許諾,期相見臨安城下。如龍貽士美洋槍,士美亦選勁勇助如龍。如龍乃星夜旋軍,與毓英共擊之,斬馬士麟、馬有才於陣,榮宵遁,遂代自清署提督。武定陷,如龍遣參將馬青雲等馳援,守備夏毓秀先登,克之,連復十餘城。文秀聞而忌之,致書馬德新,痛斥如龍自殊同教。如龍亦遍馳書迤西回民,歷數文秀狂悖及德新不諳大義,勸勿為所惑。德新入省,申割地媾和議,如龍力止之,事遂寢。其秋,攻克尋甸,擒馬榮,解省伏誅。毓英攻曲靖,回懼,願執馬聯升以獻,乞貸死,如龍馳至軍前,力為請命,許之,剖榮尸祭鐸。迤東平,詔加提督銜,賜號效勇巴圖魯。

五年,命主迤西軍事,圖大理。以振鵬攻賓川,副將李惟述攻鎮南,昭通鎮總兵楊盛宗取道四川攻永北,署騰越鎮田仲興攻蒙化,護普洱鎮李錦文攻威遠,並受如龍節度。六年,如龍軍次祿豐,適大理回入前場關,遣總兵哈國安、副將楊先芝大破之。振鵬性陰鷙,不甘為如龍下,至是聞勞崇光卒,叛志遂決;而國安、先芝亦懷二心,日與寇使往還,軍心乃解體。無何,楚雄、大姚相繼告警。時如龍駐定遠,軍數夜驚。群目或擁兵觀望,或臨陣先奔,或竟為寇充鄉導。如龍知勢已去,乃稱疾還省,自是文秀遂輕視如龍矣。

七年,大舉犯省城,如龍以回練不足恃,乃專倚漢兵守城,斥私財三萬金、米三千石濟軍。晨夜登陴守,擊寇梁家河,破之,寇稍卻。初,振鵬之叛也,約國安等為內應,至是國安謀刺如龍,事覺,誘誅之。先芝等頗自危,會如龍出大西門擊寇,戰方酣,先芝等遂倒戈回刃,如龍幾不免,亟馳入益兵禦守。於是馬世德、馬文照、馬葵等相率叛歸文秀,偪南城,據江右館,人心大震。適惟述、馬忠援師至,勸其與毓英協力,如龍然之,踵軍門上謁。毓英推誠慰勞,勖以報國,如龍益用命。俄而文秀遣悍黨數萬出賓川,如龍分部兵二千御之。武定附省,回久閉門不戰,突出奪大虹山二壘,如龍親擊之,拔其一。毓英攻澂江,馬自新率眾往援,未至,如龍詗知之,遣馬興勤馳入鐍兮,計斬自新,外援頓絕。澂江既下,又分兵攻城外賀家村、小魚村、下普坪,並克之。

八年,與毓英攻江右館,寇轟拒,洞穿如龍甲,卒大破之。先後連克武定、羅次,更勇號法什尚阿。已而澂江再陷,城外寇勢復熾。毓英攻城南巨壘,如龍方臥病,聞槍砲聲,力疾赴前敵,攻克五花寺、羊神廟,乘勝偪江右館,如龍先登,彈中腹,踣地,舁之歸。毓英上聞,賜內府丹藥,予實授。復與毓英分軍攻安寧各隘,扼寇歸路。群回益蹙,其酋段成功、蔡廷棟先獻款。如龍扶病出城,與毓英嚴兵以待,成功等率五千人伏地請罪,南關告寧,遣兵攻克西壩。時毓英克江右館,俘虜多,如龍躬詣寇營,勒回自相斬獻,省城圍始解。餘匪並入土堆。師攻昆陽亟,回酋赴省乞撫。振鵬畏誅,猶嵎負。如龍渡滇池至,遣將執悍目馬似良,陰散其枝黨。聲某日還,振鵬出送,捕治之,昆陽平。毓英攻土堆,如龍率師來會,縱火攻之,省城外遂無遺寇。

九年,如龍出督新興軍,田仲興戰死,如龍亦被創,斷東溝困之,拔其城,遂統全軍赴河西擊東溝。溝分大小二寨,哈國治、馬成林分居之,並背山面田,勢險奧。逾歲,先攻小東溝,盡選河西壯勇助擊。回懼,縛國治乞降,受而誅之。進取大東溝,如龍陷陣,為槍所中,創甚,越三月小差。直抵龍門村,奮擊破之。全滇底定,賞黃馬褂。十三年,調湖南。光緒四年,創發,乞歸。

如龍性豪縱,筦雲南提篆日,惟娛聲色。巡撫賈洪詔彈之,置勿問。既閒廢,徙居四川重慶,益不自檢。每宴客,招妓侑酒,琵琶聲中輒慷慨道少年時事云。十七年,卒,恤如制。

和耀曾,雲南麗江人。父鑒,大理城守營都司。咸豐二年,太和回謀亂,往覘之,被殺。詔贈雲騎尉世職,耀曾襲,矢復仇,毀家募士。與賓川廩生董文蘭會師洱河,兩克大理及鄧川、上關,以義勇著,遠近爭歸附。楊玉科、張潤並隸麾下,後皆為名將。總督吳振棫薦其才,署中營守備。

十年,權維西協左營都司。明年,大理回來犯,敗之於橋頭。已而祿豐、昆陽陷,復率把總高聯甲戰石鼓,大破之。乘勝攻克麗江,留土弁王天爵駐守,而自引兵規鶴慶。寇乘隙再陷麗城,耀曾軍失利,退守石鼓。同治元年,再克之,遷參將。徙頓曲靖,夷卡郎寇巢,略昭通,戰公雞山、龍洞,師弗勝。與護昭通鎮楊盛宗往援,斬其酋所朝升,遷副將,徙守富平。八年,城陷,褫職逮問。尋以克楚雄、祿豐,貸勿治,留軍自贖。十年,克澂江,復官。明年,攻迤西,連破蒙化、趙州、上下關,於是大理籓籬盡失。是冬,穴地道轟其城,拔之。又明年,取大小圍埂。積勛晉記名總兵,賜號達春巴圖魯。自是與玉科定錫臘,下順寧,破雲州,擢提督。進克小猛統,大吏以叛產予其殘廢部伍,固辭弗獲,乃斥家財遣之歸,而以其地佐書院餐錢及賓興費,並選開敏子弟集廨宇,延師課讀。又與李惟述克騰越。全滇平,賞黃馬褂,檄署永昌協。

永昌自遭喪亂,比戶凋殘。耀曾至,撫流亡,除苛擾,革奸暴,教之治生,民漸復業。時烏索賊柳映蒼復叛,奉檄與總兵徐聯魁等會擊。十三年,克之,以次削平土司諸地。光緒二年,參將蘇開先誘練軍譁變,據騰越。王道士與合,順、雲豪奸悍卒乘機竊發,永昌練目李朝應之,掠施甸,迤西大擾。岑毓英以耀曾諳究邊情,奏署騰越總兵。耀曾為固本計,先赴永昌,調團守隘,率師追討,擊潰李朝,餘黨悉平。總督劉長佑謂其不即至,劾之,鐫二級;論克順、雲功,免議,權漢中鎮總兵。

六年,詔各省督撫舉將才,毓英以耀曾應,擢授鎮遠鎮。居鎮十六年,節虛糜,贍儲積,為置營田,建兵房,制器械,軍政大治。復以其餘設義塾,平道路,勸農桑,士議謂有儒將風。二十三年,卒。民感其德,請附祀毓英祠,麗江亦建祠致祭焉。

楊玉科,字雲階,寄籍麗江。其先居湖南善化,既貴,還本籍。同治初,從和耀曾討回。岑毓英征曲靖,識拔之,命領百人為前鋒,積功至守備。四年,署維西協。李祖裕叛,殺把總陳聰。毓英慮生變,檄玉科代之。玉科至,刺殺祖裕,宣諭部眾,皆伏服,遂克麗江、鶴慶,繇是顯名。

俄而杜文秀來援,擁眾可二十萬。玉科所部止數千人,屢戰弗勝。二城復陷,玉科潰圍出走永北。六年,從克鎮雄,長驅豬供箐、海馬姑,與有功,敘游擊。七年,西寇環偪省城,玉科繞四川會理,間道襲元謀、馬街,規武祿,抄其後,克之,進平羅次。八年,平柯渡、可郎,遷副將,賜號勵勇巴圖魯。乘勝規嵩明,下尋甸。毓英奏令主三姚軍事,連復大姚、浪鄧。省城圍解,擢總兵。明年,破姚州土城,被巨創。益開地道三十餘穴,雷發,北城陷,遂拔,擒偽將馬金保、藍平貴。三姚平,擢提督,易勇號瑚松額。無何,州西警,復令主大理、麗江軍事,發全師速援賓、鄧,遂敗寇雲南驛。其冬,克長邑村,進規鍊鐵,擒偽都督楊占鵬。於是大理北路定,權開化鎮總兵。

十年春,克賓川。初,永昌之陷也,玉科為偽將馬雙元所得,見其人可用,勸歸命,與訂交,囑異時得志相援手,縱之歸。至是約為內應,克之,署提督。攻大理小關,邑寇詐降,設座禮拜寺,約玉科往。比入,心動,命移座;動如故,命再移,有間,地雷發,得不死。玉科怒,手刃四人,雙元銳身護之出,竟復其地。

逾歲,連下漾濞、趙州,進規大理。其地東臨洱海,西倚蒼山,自文秀竊據,內築土垣,包偽禁城其中。玉科掘隧以攻,轟潰東南城,諸軍蹈隙入。寇死拒,復窖地雷破之。頓蓮花池,益師五千環攻城。文秀開壁出蕩,親擊之。敗退,飲毒不即死,其黨蔡廷棟舁以獻,氣息僅屬,割其首解送省城。毓英至,廷棟佯乞款,陰埋地雷於行館,迎玉科。玉科諾之,潛至偽府,據砲樓大呼,兵士爭血戰。毓英隱卒城外,度玉科已達,乘夜梯登。兩軍既合,巷戰竟日,寇披卻,越數日,奪門走。克偽都,獲文秀家屬及廷棟等百三十人。捷入,賞黃馬褂,予騎都尉世職。十二年,克錫臘、順寧,移師協取雲州,再予一騎都尉。全滇告寧,改一等輕車都尉。明年,入覲,垂詢滇池戰狀,視傷痕惻然。光緒改元,還署任,賜頭品服,晉錫二等男。

是時,滇邊野夷殺英官馬嘉理,英公使訴於朝,朝旨趣捕急。玉科搜獲而通凹、臘都等十五人,金巢送省城伏誅。讞定,會鄧川羅洪昌謀亂,襲州城,遂移師馬甲邑,克東山,擒渠率。二年,移廣西右江鎮。創發,乞解職,疏甫上,適蘇開先陷騰越,勢岌岌。玉科力疾視師,不百日悉平之,被賞賚。三年,徙廣東高州鎮。六年,署陸路提督,坐其侄汝楫仇殺知府孔昭鈖,鐫三級。尋復。

十年,法越事起,率師出關,扼觀音橋,法軍至,設三伏敗之。聞穀松警,亟往援,而敵已乘虛入,數戰皆利。明年,法以重兵入關,教民應於內。玉科曰:「吾百戰餘生,今得死所矣!」開關搦戰,中砲亡,諸軍皆潰,至無人收其尸。李秉衡蒞關,乃歸其喪,妻牛氏殉焉。追贈太子少保,謚武愍,予大理、鎮南關建祠。毓英所部多驍將,玉科外,首推李惟述。

惟述起錦江紳團,嘗與和耀曾施方略,謀所以綴寇,故省城獲保無事。逮馬榮敗,回眾走城外,猶留頓弗去,毓英患之,召惟述計誅其悍將。悍將故暱惟述,一日,天鄉明,惟述率千人入其壁,悍將方沐,詰所來。惟述曰:「奉上官檄討野夷,不識路徑,故來問。」悍將指畫以示,惟述從其背擊殺之,大呼曰:「為兵者出前門,從逆者出後門!」回眾驚散,省城遂無寇蹤。累勛至都司,補鶴麗鎮游擊。克楚雄,遷參將,署維西協。與經歷鍾念祖分攻廣通、南安,下之,補順云協,署開化鎮總兵,仍留駐其地。無何,寇湧至,城再陷。惟述慮殘民,佯議和,卒以計脫歸,坐免。

是時,省城復震,馬如龍專倚漢兵守城。惟述分領其眾,內詰奸宄,外禦強敵,省城復安。論功,復故官。從毓英攻楊林,寇敗潰,然猶堅守小偏橋、十里鋪,冀斷我糧餽。惟述乘勝克一撮纓、蕭家山,又與岑毓寶攻克石虎岡,運道始達。進平羅次,復楚雄,軍勢大振。已而州西又告急。毓英謂西軍弛律,咎在諸將不和,乃以大理、麗江軍事屬玉科,而屬惟述以雲、蒙、趙。惟述遂攻克鎮南,鎮南為大理屏蔽,寇以全力死守,至是拔之,寇益蹙。上念其苦戰,賜珍物。

迤西用兵,頻歲饑嗛。先是,惟述遣軍攻雲南,久弗克,彌渡亦旋得旋失。嗣與玉科謀,乃檄諸軍毋浪戰,期秋穫整軍。屆期果大破雲南驛,分兵略彌渡,並克之。又與玉科會軍蒙、趙。杜文秀者,故永昌累,初匿大小圍埂。其據大理也,圍埂回實助之。玉科圖取大理,惟述亦統兵克大圍埂,而小圍埂猶據壁自保。逾歲,轟克之,檄署騰越鎮,收其地。進攻烏索,未下,遭憂歸,不復出。久之,卒於家。

初,玉科嘗殺仇,持其首謁毓英,意詰責即為變。毓英笑勿問,且善撫之。惟述性戇直,業騾馬,初不知希榮貴。及奉上賞白玉搬指,適與指合,乃驚嘆天子聖神,益效忠無貳志。所設市肆,悉以「巴圖魯」號名之,其榮幸朝命如此。平滇,楊、李功為多,而玉科用兵,則尤神於出沒雲。

蔡標,字錦堂,貴州威寧人。家貧,落魄無以自資,入滇,設湯餅肆宜良。以膽略稱。久之,充練目。從岑毓英軍克宜良、路南,補把總。同治二年,馬榮據省城,毓英堅守籓署,誓與城存亡。標領死士數十人潛至,叩門入,毓英驚喜。標問:「有軍械否?」曰:「有。」標曰:「寡不敵眾,奈何?當為公募兵!」遂往宜良、路南鳩集舊部,得千人,毓英賴以成軍。籓署獲全,標之力也。及馬如龍至,標率效力戰,榮敗走。從征迤東西,連下十餘城,進規曲靖。寇襲潘文元營,標率三百人頓陶家屯扼後路。張保和蹙寇至海壩,標分兵要之,寇潰入城。克曲靖,遷守備,賞花翎。

五年,毓英西征,標引兵從。時鎮雄降寇復叛,漆維新據子山,李開甲據磺山。毓英策先攻角奎,令標為前驅。抵雄塊,寇出拒,大破之,連拔二山,斬二逆。明年,從攻豬供箐,與諸將直搗中堅,下之。移師海馬姑,奪紅巖、尖山,乘勝薄其柵。標賈勇先登,諸軍鼓譟繼進,擒渠率。凱旋,擢游擊。七年,署鎮雄營參將。會杜文秀偪省垣,標出宜良、湯池,略七甸。未幾,武定、祿勸連告警,復與楊國發攻富民,綴寇勢。寇攻楊林亟,標往援,連破小街、白龍橋巨壘。壘甫得,旋復失,勢益熾。標入自長坡,寇殊死戰,不可敗。翼日,自石子河逾文筆山而下,佯北,誘寇入,攻克東山寺,盡夷楊林寇壘。八年春,援師宗,攻破洛紅甸、豆溫鄉,拔其城。於是嵩明、富民相繼收復,省城始安。

明年,威寧陳大桿據紅崖,楊紹貴等據香爐山,四出剽奪。標越境助擊,誘執陳酋,金巢送州城;吳奇忠亦破香爐山。事寧,擢標總兵。十一年,諸軍環攻大理偽城,標略其南,力戰一晝夜,克之。南門寇欲竄下關,標復自城追出截擊之,無幸免者,晉提督。十二年,移攻雲州,抵猛朗,望見寇壁堅緻,標曰:「此宜先絕外援也!」乃遣陸純綱等扼邦蓋、丙弄,而自率師克猛朗,殲其酋丁雁甲。論功,賞黃馬褂,檄署鶴麗鎮總兵。全師抵城下,標攻北門,段瑞梅等自東南梯而下,轟擊之,盡殪。又先後平永北、賓川妖匪,騰越、烏索降匪,開化、大窩子竄匪,更勇號額爾克。光緒二年,入覲,道貴州,毓英留統威寧練軍,扼守要隘。已,復平梵凈山餘匪、桐梓會匪、湖南董倒寨回匪。

七年,毓英移撫福建,標率滇軍渡臺,詔補雲南開化鎮,仍駐臺北。逾歲,赴本官。十年,法越事起,標募舊部出關,宣光、臨洮數戰皆利。其守富良江,遍掘地營,法砲不能中,岑軍駐河內者遂不為所窺。著有地營圖說,甚明晰。十三年,署云南提督。毓英檄治惈黑山軍事,標率師前進,效力合攻,夷緣江百數十壘,誅其魁張春發,拓地千五百里。二十年,錄平永北夷匪、廣南游匪功,賞雙眼花翎、頭品服。越六年,再入覲。會兩宮西幸,即赴行在,隨扈入陜。抵西安,廷旨命招舊部。尋坐約束不嚴褫職,詔念前功,予留任。明年,還滇,以所部罷弱,解遣之,釋處分。三十一年,徙廣東瓊州鎮。次年,卒。附祀毓英祠,予威寧建祠。

瑞梅,字春堂,籍劍川。有勇略,年十六從軍,隸毓英麾下,戰常陷堅。攻豬供箐、柯渡、大理,並冒險進。歷龍陵營參將,維西、永昌協副將。同治十三年,入覲,賞黃馬褂,予雲騎尉世職,擢記名提督。尋署騰越鎮總兵。光緒間,以邊兵亂,城陷,隨復之。後卒於官。

夏毓秀,字瑯溪,雲南昆明人。少以義勇著。滇回亂,以堡長從軍,充選鋒。昆明被圍久,糧餽阻絕,道殣相望。毓秀率團勇助擊,運道始通,補千總。師克路南、祿豐,積勛至守備。

同治二年,岑毓英引兵西,遣毓秀略富民,擒其酋馬富,富,馬榮弟也。乘勝克嵩明、陸涼、武定,署參將。毓英慮元謀回撓後路,使毓秀要之。攻克附城巨壘,偪攻縣城,截其糧路,寇患饑乏,棄城走,進復馬街。三年,回酋李芳園陷白井,擊卻之。規曲靖,師屢失利。毓秀至,寇狃數勝,易視之,且登城作謾語。毓秀憤甚,率死士先登,疾擊之,寇大潰,合兵下霑益、馬龍。明年,補提標右營游擊,統領四十八堡民兵。七年,西寇陷祿豐,毓秀敗績,退安寧,分兵扼腰站、祿。逾歲,寇湧至,再敗,毓秀退入省城,坐免官。已而寇大舉分道入,馬如龍出大西門御之,參將楊先芝等倒戈相鄉,毓秀被重創。又明年,攻楊林,擊破十里鋪,復官。毓英規安寧,毓秀自蓖郎繞出碧雞關下,潛師襲大小普坪,克之。進取獨樹鋪,會岑毓寶復其城。九年,論克廣通、南安功,遷副將。

十年,攻東溝,寇出拒,敗之,師深入,毓秀陷重圍,逕路危■C7,棄馬步戰,身受十數刃。如龍馳救,舁歸壁,暈眩死,有間蘇,將校環泣,毓秀慨然曰:「丈夫以身許國,馬革裹尸,固大快事!奚悲為?」聞者莫不感奮。創小差,整軍復進,卒夷寇壘,擢總兵,賜號利勇巴圖魯。移攻雲州,寇築碉環城誓死守,師久攻不下。毓秀先分兵奪碉,孤城危棘,寇無固志,遂拔之,以次復騰越及大小猛統。十三年,入覲,上垂視傷痕,慰勞備至,益感激願用命。會創發,乞歸。

光緒二年,赴四川,統領省標十營。七年,松潘番蠢動,數擾邊,命署總兵治之。既至,擒首惡,撫良懦,番民以安。其地固荒服,設學額百餘年,多為他邑人所占,謳誦益寂寥。毓秀方夷大難,即選聰穎子弟入署讀書,斥私財建書院,廣延名宿,崇化勵賢,至是始聞弦歌聲。九年,實授。蒞鎮十載,培堤岸,濬溝洫,儲倉廩,士民德之,至建生祠以祀。

二十年,朝鮮亂起,日軍侵奉天。毓秀自請赴前敵,比入京而和議成。會鹿傳霖出督四川,奏毓秀自隨,於是再蒞松潘鎮任。初,甘肅循化番族拉布浪寺夙強悍,數越界侵掠。毓秀初蒞鎮,遣兵防守,安撫餘眾,而拉部擅命如故。既復任,遂率將士出關,克碉十餘,擒渠率,斬以徇諸夷。諸夷皆伏服,莫敢惕息。蜀邊寧靜,擢提督。巴塘西三巖野番數入邊,商旅苦之,號稱「夾壩」。毓秀率眾入其部落,招誘首領,宣播朝威,動以禍福利害,諸番皆束首歸命,晉頭品秩。

二十六年,授貴州提督。會拳亂作,亟統兵入衛。抵蒲州,車駕西幸,命率師駐韓侯嶺,許專摺奏事。明年,調湖北,命分所部留守太原。毓秀以三子瑞符領六營詣防,而自率全軍隨扈北上。尋移廣西。逾歲,行次廣東,總督陶模奏署陸路提督。九月,還湖北。宣統二年,創發,卒於官,謚勇恪。

毓秀性忠樸,不治家人生產。治軍數十年,布衣蔬食,見者不知其為專閫云。

何秀林,雲南宜良人。少從岑毓英軍,攻羅川,襲定遠,略曲靖,每戰必克,累功至守備。討豬供箐,寇悉銳出,圍攻姜飛龍前營,毓英往援,令秀林策應,於是夾擊,大破虜,復進搗中堅,擒其酋陶新春,合師剿克海馬姑,遷游擊。同治七年,西寇圍省城,從毓英自宜良七甸破大小石壟、麻苴、新村,進取大樹營。運道達,移師呈貢,敗晉寧、昆陽援賊,拔其城,遷副將,賜號效勇巴圖魯。

攻澂江,迭克要隘,直薄城下,城寇遁,毓英攻西北二門,秀林助之。張元林敗入城,官軍梯而登,馬忠入西門,秀林入東門,元林懼,仰藥死。澂江平,與李廷標協守楊林。八年,寇犯邑市舊縣,防軍告亟,秀林赴之,連破馬家沖、前街、邑市。會廷標亦往援馬龍,兩軍以無主將失和,寇蹈瑕入楊林。秀林聞警馳還,勵眾堅守,而都司楊桐等先潰,秀林遂陷重圍。李惟述援軍弗能至,秀林力盡,潰圍出,被巨創,退保宜良、北屯。楊林陷,坐免。秀林營員何裔韓傷重幾死,猶攜文卷以行,與秀林收集潰兵,赴省助戰,大板橋之役,與有力焉。

其秋,攻易門,與署知縣周廷獻克西門、大小龍口及黃泥堆,斷樵汲,分兵佯攻西北,誘寇出,而遣將潛襲西南。秀林督軍沖入,寇惶恐,偽乞撫,秀林弗許,卒大破之,復故官。無何,粵寇陷祿豐,秀林約練目丁同義反正。同義倒戈以應,秀林分軍奪門入,擒渠率,城遂復,晉總兵。九年,師攻澂江,秀林破城外五山巨壘,寇掘地為營,師久無功。秀林詐退,隱卒誘之,回酋馬敏功等墮其計,並殞於陣,館驛遂無遺寇,進克鐍兮,擢提督。明年,補普洱鎮總兵。

光緒十年,法越事起,從毓英出關,統三千六百人駐興化。法軍退宣光,勒兵而進,丁槐軍西南,秀林軍東南,攻大寺、大寨,破之。城內法軍開壁出蕩,秀林所部中彈數十人,戰益力。法軍馳入壁,城外壘柵林立,砲臺棋布。秀林數攻城,為所絓,乃開地道轟潰之,於是攻城無所阻,遂偪城而軍。十一年,法軍數萬來援,劉永福軍潰。秀林遺馬維騏往救,堅守地營,敵不得逞。已而維騏亦被困,秀林至,法軍乃解去。周視各營,傷亡既眾,不獲已,退頓城下。策敵必猛攻,豫窖地雷以待。敵果至,雷發,法軍死傷枕藉。秀林乃從容集殘軍,退保同安,圖再舉。和議成,罷戍,移臨元鎮。十六年,卒。

楊國發,雲南建水人。討雲南、貴州匪,以戰功數遷至守備。咸豐十年冬,署提督申有謀攻富民,國發長左翼,諸生張執中導之出麥廠間道,克黃土坡、永安莊。入城,圍攻之,寇棄城走,遷都司,賜號果勇巴圖魯。明年,進剿祿豐及廣通各井,皆下之。

同治二年,從岑毓英西征,連下十餘城,直趨楚雄。國發先克古山寺、雙橋巨壘,飛炬焚之,奪東門入,城克。會大姚告警,國發領兵赴救,破援賊桃花村。合城圍,知縣硃士逵舉火應,約期啟關,大姚平。移攻鎮南,以寇援大至,檄還省。三年,權元新營參將,與諸軍拔曲靖,並復霑益、馬龍,再遷至副將。四年,廣西州土寇嘯亂,殺游擊陳萃、知縣李瑞枝,國發率師討之,斬其酋張顯,境賴以安。越三年,西寇圍楚雄亟,國發從間道入,與守將李惟述日夜鏖戰,經月餘,攻不剋,糧盡援絕,城陷。國發冒圍出,仍繞道還省。

七年,寇分路大舉,一自富民據城西北,一自安寧據城西南。毓英入援,遣國發扼楊林。俄而李芳園等悉眾來犯,勢張甚。國發告亟,毓英使蔡標赴之,與國發破小街、白龍橋。旋為寇所乘,地復失,乘勢偪城下,縛草束薪,累積如堵墻,列槍砲其上,俯擊城中,謂之「柴碼兵」,將士損折過半。國發不獲已,再告亟,請援師,毓英自將兵破之,檄國發署普洱鎮總兵,頓師桃園,接應諸路。

八年,尋甸回圍馬龍,國發至,會諸軍戰卻之。夜將半,進掩賊營,乘風縱火,熛煙張天,盡焚其壘,圍解。轉鬥逐北,連破十里鋪、小偏橋、長坡六十餘里,遷總兵。規彌勒竹園村,馬世德構開、廣回來援,國發破之趙林寨。十年,攻雲南縣,與惟述會軍普淜,分道入,國發迭克要害。寇竄觀音寺,國發麾兵擊之,又毀七碉,直薄城下,與惟述軍合。十一年,轟裂城垣三十餘丈,相繼而登,巷戰一晝夜,拔之,留所部守其地。秋,徙鎮下關,進圍榆城,先後克大小圍埂,擢提督,賞黃馬褂,更勇號綽勒歡。

十三年,再權普洱鎮。光緒七年,毓英撫福建,國發率師駐臺北。明年,還雲南。十年,從毓英援越南宣光、臨洮,每戰皆克,予優敘。二十六年,卒,附祀毓英祠,予本籍建祠。

張保和,雲南師宗人。初從岑毓英討回寇,積功至守備。同治六年,越境討豬供箐,屢獲勝。寇竄海馬姑,復與蔡標等合擊之。攻大寨,悍酋張項七死拒,保和執予以刺,墮馬,梟其首,寇氣懾,諸軍乘之,大捷,遷游擊。明年,西寇圍省城,毓英入自曲靖,遣保和為前鋒,攻克石虎岡,馳救邵甸,破之。移師楊林,迭克要害,皆揮矛沖陣,所向輒靡。寇見保和旗幟即反奔,無敢與抗者。數負重傷,裹創力戰,氣不少沮。先後攻克呈貢、晉寧、富民、嵩明,大小百餘戰,未嘗一挫。省圍解,遷副將,賜號揚勇巴圖魯,署楚雄協。

規昆陽,遣都司陳貴等自津徑取河西鄉,而自率師攻仁和街,越墻而入,手刃悍寇數人,一鼓克之,河西亦平,於是城圍合。保和揆形勢,謂宜先克海口,因勒兵以進,村民爭迎附,二十餘寨皆下,遂復州城,署開化鎮總兵。九年春,攻彌勒竹園,馬世德赴救,保和迎擊之,身先陷陣,彈貫鼻及眼,血盈面,士卒憤懣,卒大破之。連克上下壩,竹園平。赴本官,更勇號曰剛安。進取茂克,戰數捷,阿迷、大莊望風讋憚。奪後山,增築砲臺,俯瞰其寨,日夜轟擊之,汲路絕。寇駭乞降,保和許之,收器械,捕惡黨,徙降回大莊。十一年,以次復田心、日者鄉。時趙發攻鐍兮未下,保和自開化馳剿。直抵馬街,破上下兩寨,徙降回新興,擢提督。十二年,論克騰越功,賞黃馬褂,授鶴麗鎮總兵。

十三年,開化大窩子土夷復叛,毓英收撫之,檄保和再蒞開化鎮任,發兵二千,責千總李瑤等戍其地。瑤等縱兵凌虐,於是土夷大憤。逾歲,光緒改元,回酋馬河圖嗾與漢民閧,保和欲樹功,與署知府姚嘉驥侈張其事,請調兵數千,發餉巨萬,剋期大舉。毓英廉得實,斥之。保和怒,乃罷戍,以失守聞。毓英遣何秀林進擊,保和詗將至,宵入燔數寨,稱克復。毓英乃罷保和。明年,調湖南永州鎮。三年,卒。

保和在滇將中以智勇著,功亦盛。其卒也,年未四十,時人惜之。

論曰:滇回初起,勢頗盛,自如龍反正,其氣始衰。然非有以善馭之,剿撫兼施,滇事亦未易定也。耀曾善於結士,玉科神於用兵,標等皆善攻堅;而毓秀忠樸,兵後能崇儒興學,尤稱知本,民建生祠以祀之,宜哉!


\end{pinyinscope}