\article{本紀十一}

\begin{pinyinscope}
高宗本紀二

十一年春正月庚午,以紀年開帙,命減刑。癸未,命慶復進剿瞻對,為李質粹聲援。辛卯,賑江蘇銅山、安徽宿州等州縣饑。甲午,朝鮮入貢。李質粹進攻靈達,班滾之母赴營乞命,仍縱歸。上飭其失機。諭慶復督兵前進。

二月戊戌,賑山西大同等十二州縣饑。辛丑,召北路軍營參贊大臣拉布敦、烏勒來京,以塔爾瑪善、努登代之。癸卯,上幸南苑行圍。丁未,免廣東新寧等州縣、雲南鶴慶府水災額賦。辛亥,以三月朔日食,詔修省以實。定皇后不行親蠶禮之年遣妃代行。丙辰,免河南永城等五縣水災額賦。庚申,西藏臺吉冷宗鼐以攻瞻對擅徹兵,論斬。諭宥其死。

三月己巳,免直隸鹽山等八州縣水災額賦。甲戌,賑雲南白鹽井水災。乙亥,準噶爾臺吉策旺多爾濟那木札勒以新立,遣使哈柳貢方物,請派人往藏熬茶。戊寅,慶復至打箭爐,劾李質粹等老師玩寇,請續調官兵進剿,允之。辛巳,遣內大臣班第等赴瞻對軍營。壬午,賜哈柳等宴。召見,允其往藏熬茶,頒如意賚之。甲申,賜準噶爾臺吉策旺多爾濟那木札勒敕。予故臺吉噶爾丹策零布施。丙申,免湖北潛江等州縣上年水災額賦。慶復奏進駐靈雀。

閏三月丁酉朔,飭陜西修列代陵墓。庚子,召白鍾山來京,命顧琮署江南河道總督,高斌暫管之,以劉統勛署漕運總督。賑直隸宣化府饑。賑甘肅隴西等十二州縣水旱雹霜災。丙午,命汪由敦署左都御史。癸丑,左都御史杭奕祿休致,以阿克敦代之。

夏四月丁丑,白鍾山褫職,發南河效力。戒軍機處漏洩機密。以鄂昌署廣西巡撫。丁亥,免湖南湘陰等五縣水災額賦。己丑,免廣東新寧等四州縣水災額賦。

五月丙申朔,以盛安為左都御史,阿克敦為刑部尚書。丁酉,諭顧琮查明南河虛糜之款,令白鍾山賠補。壬寅,免山西大同等十八州縣上年旱霜各災額賦。丙午,慶復奏進攻瞻對,番酋班滾計日授首。加慶復太子太保。戊申,免甘肅靖遠等三縣上年旱災額賦。己酉,永除直隸慶雲縣每年額賦十分之三。乙卯,達賴喇嘛等請宥班滾,不許。以傅清代奏,嚴飭之。

六月丙寅,慶復、班第等會攻丫魯尼日寨,克之。班滾自焚死。丁卯,以打箭爐口內外番從徵效力,再免貢賦二年。丙子,京城地震。壬辰,命送還俄羅斯逃人於恰克圖。

秋七月丙申,加那蘇圖、策楞太子少傅銜,周學健太子少保銜。丁酉,命高斌赴江蘇察看黃、運工程,劉於義署直隸河道總督。壬寅,四川大乘教首劉奇以造作逆書,磔於市。庚戌,周學健奏捕天主教二千餘人。上以失綏遠之意,宥之。壬戌,賑湖北漢川等七縣水災。癸亥,以雲南張保太傳邪教,蔓延數省,諭限被誘之人自首,其仍立教堂者捕治之。丁卯,召吉林將軍巴靈阿來京,命阿蘭泰代之。賑直隸慶雲等七縣場旱災。己巳,以四川提督李質粹進剿瞻對欺飾,罷之。免廣寧等處旗地水災賦。辛未,賑湖南益陽等四州縣水災。癸酉,加賞江蘇、安徽被水災民修葺房屋銀。乙酉,賑山東金鄉等十一州縣衛水災。庚寅,上御瀛臺,賜宗室王公等宴。改崇雅殿為敦敘殿。辛卯,上御瀛臺,賜大學士、九卿、翰林、科道等宴,宣示七言律詩四章。壬辰,福建上杭縣民羅日光等糾眾請均佃租滋事,捕治之。癸巳,允朝鮮國王請,停奉天設檿牛哨汛兵。

九月甲午朔,除浙江歸安等三縣沙積坍卸地賦。戊戌,訓督撫實心行政。賑山東滕縣等三州縣、兩淮板浦等六場水災。己亥,命高斌往奉天疏濬河道。辛丑,停今年秋決。以周學健為江南河道總督。調陳大受為福建巡撫,以安寧署江蘇巡撫。定欽差大臣巡閱各省營伍例。賑河南鄭州等三州縣水災。壬寅,命訥親兼管戶部。免甘肅隴西等九州縣水災額賦。癸卯,上奉皇太后啟蹕詣泰陵,並巡幸五臺山。丁未,上謁泰陵。己酉,阿里袞患病,以班第署山西巡撫。庚戌,賚經過直隸州縣耆民。甲寅,賑江蘇豐縣等三州縣雹災。乙卯,上駐蹕五臺山射虎。以山西風俗醇樸,諭疆吏教養兼施,小民崇習禮讓。丙辰,免山西五臺縣明年額賦十分之三。丁巳,召馬爾泰來京,以喀爾吉善為閩浙總督。調塞楞額為山東巡撫,陳宏謀為江西巡撫,以徐杞為陜西巡撫。庚申,上奉皇太后回蹕。壬戌,召鄂彌達來京,以塞楞額為湖廣總督。調阿里袞為山東巡撫,愛必達為山西巡撫。賑河南鄢陵等二十六州縣水災。

冬十月甲子,賑山西陽曲等二十二州縣水雹各災。丁卯,上閱滹沱河堤。賑湖北漢川等九州縣衛水災。庚午,上奉皇太后駐蹕保定府。壬申,上閱兵,賜銀幣有差。甲戌,以張廣泗發摘逆犯魏王氏、劉奇等,予敘。定加山西歸綏道兵備銜,稽查靖遠營。戊寅,上奉皇太后還京師。調開泰為江西巡撫,陳宏謀為湖北巡撫。庚辰,免張廷玉帶領引見,並諭不必向早入朝及勉強進內。壬午,命汪由敦軍機處行走。癸未,御史萬年茂以劾學士陳邦彥等獻媚傅恆不實,褫職。戊子,免安徽壽州等二十三州縣水災額賦。辛卯,撥賑江蘇淮、揚、徐、海各屬災民銀糧二百二十萬兩石有奇。

十一月癸巳,寢甄別科道之命。御史李兆鈺下部議處。乙未,以河南學政汪士鍠考試瞻徇,褫職。免江蘇山陽等二十四州縣衛水災額賦,並分別蠲緩漕糧有差。乙巳,除奉天錦縣等二縣沖壓地賦。己酉,予故內閣學士張若靄治喪銀,並諭張廷玉節哀自愛。辛亥,李質粹發軍前效力。戊午,慶復奏大金川土司莎羅奔擾小金川,倘不遵剖斷,惟有用番力以收功。上是之。

十二月癸亥,召班第來京,以陶正中護山西巡撫。甲子,賑湖北潛江等七州縣衛水災。乙丑,以傅清奏達賴喇嘛看茶之綏繃喇嘛鎮壓郡王頗羅鼐,賜手敕慰解之,並諭以與達賴喇嘛同心協力,保安地方。戊辰,以瑚寶為駐防哈密總兵。甲戌,免直隸靜海蟲災額賦,並賑之。丁丑,以張廷玉年老,命其子庶吉士張若澄在南書房行走,俾資扶掖。戊寅,賑甘肅安定等州縣旱災。免山東金鄉等八州縣水災額賦。庚辰,除廣西永福水沖地賦。癸未,準噶爾臺吉策旺多爾濟那木札勒遣使瑪木特等入覲,召見於太和齋。己丑,賑蘇尼特、阿巴噶等旗災。陳大受奏,蘇祿國遣番官齎謝恩表番字、漢字二道,與例不符,卻之,仍優給番官令回國。上嘉為得體。

十二年春正月壬辰,命玉保辦理準噶爾使赴藏事務。甲午,免山西太原等六府八州及歸化城額征本色十分之三,大同、朔平二府全蠲之。乙未,賜瑪木特宴於豐澤園。戊戌,免江蘇海州等三州縣及板浦等六場民灶舊欠。丁未,賑山東壽光等十三州縣饑。乙卯,賜準噶爾臺吉策旺多爾濟那木札勒敕,允所遣西藏念經人在哈集爾得卜特爾過冬及貿易。

二月辛酉朔,免吉林上年旱災應交租穀。壬申,上謁昭西陵、孝陵、孝東陵、景陵。紀山奏大金川土司侵革布什咱土司,誘奪小金川土司澤旺印信。諭飭修守御,毋輕舉動。甲戌,上幸盤山。庚辰,賑山東蘭山饑。壬午,除河南孟縣沖坍衛地額賦。癸未,上還京師。戊子,原任內務府大臣丁皁保年屆百齡,賜御書扁額朝服糸採幣。免湖北棗陽上年水災額賦。

三月,免山西陽曲等二縣上年水災額賦。辛丑,召慶復入閣辦事,調張廣泗為川陜總督。復設雲貴總督,以張允隨為之。命圖爾炳阿為雲南巡撫,孫紹武為貴州巡撫。賑河南水災。以大金川土司掠革布什咱、明正各土司,擾及汛地,命慶復留四川,同張廣泗商進剿,並飭張廣泗撫馭郭羅克、曲曲烏、瞻對、巴塘諸番。免江蘇淮安等四府州屬上年水災額賦。大學士查郎阿乞休,允之。乙巳,西藏郡王頗羅鼐卒,以珠爾默特那木札勒襲封郡王。丙午,以高斌為文淵閣大學士,來保為吏部尚書。調海望為禮部尚書,傅恆為戶部尚書。命索拜駐藏,協同傅清辦事。免安徽壽州等二十三州縣衛上年水災額賦。丁未,命副都統羅山以原銜管阿爾泰軍臺,並商都達布遜諾爾馬廠事務。己酉,命張廣泗進剿大金川土司莎羅奔。西路軍營參贊大臣保德期滿,以那蘭泰代之。庚戌,免直隸薊州等十四州縣上年水災額賦。戊辰,命高斌往江南會同周學健查勘河工,並清理錢糧積弊。己巳,以那蘇圖署直隸河道總督。壬午,給訥親欽差大臣關防,命往山西會同愛必達讞安邑等二縣聚眾之獄。甲申,召雅爾圖回京。

五月辛卯,召準泰來京,以策楞兼管廣東巡撫。丙申,賑山東安丘等二縣饑。甲辰,祭地於方澤,以旱屏鹵簿。乙巳,命刑部清理庶獄,減徒以下罪。己酉,上詣黑龍潭祈雨。辛亥,愛必達免,調準泰為山西巡撫。壬子,以福建、山東、江南、廣東、山西迭出挾制官長之獄,諭:「頑民聚眾,干犯刑章,不得不引為己過。各督撫其諄切化導,使愚民知敬畏官長,服從教令。」

六月庚申朔,諭來春奉慈輿東巡,親奠孔林,命各衙門豫備事宜。辛未,命貴州巡撫節制通省軍務。霍備以不查劾州縣虧空褫職,發軍臺效力。壬申,賑山東益都等七州縣饑。丙子,小金川土司澤旺率眾降,並歸沃日三寨。官兵進剿大金川,攻毛牛及馬桑等寨,克之。召慶復回京。

秋七月己丑朔,撫恤山東歷城等二十州縣衛水雹各災。命高斌等疏濬江蘇六塘等河。丙申,命納延泰賑蘇尼特等六旗旱災。癸卯,停劉於義兼管戶部,以訥親代之。丙午,賑順直固安等七十五州縣水旱雹災。戊申,上奉皇太后幸避暑山莊。癸丑,張廣泗進駐小金川美諾寨,分路攻剿,受小金川降。乙卯,上奉皇太后駐避暑山莊。戊午,賑長蘆永利等三場旱災灶戶。

八月辛酉,上奉皇太后幸木蘭行圍。丙寅,賑長蘆、海豐等二縣灶戶。戊辰,上行圍溫都爾華。賜蒙古王、公、臺吉等宴。辛未,採買熱河八溝等處米,賑蘇尼特六旗旱災。癸酉,賑江蘇蘇、松等屬潮災。丙子,命賑蘇尼特六旗銀,均用庫帑,免扣王貝勒等俸。辛巳,慶復奏進攻刮耳崖,連戰克捷。諭:「小小破碉克寨,何以慰朕。」壬午,賑浙江壽昌等三縣水災。乙酉,賑順直霸州等十五州縣水災。賑湖南耒陽等九縣、陜西朝邑、廣東順德等三縣水災。

九月戊子朔,免經過地方額賦十分之三。賑甘肅伏羌等十縣、雲南安寧等三州縣旱災。上奉皇太后回駐避暑山莊。癸巳,以江蘇崇明潮災,淹斃人民一萬二千餘口,免明年額賦,仍賑之。乙巳,賑安徽歙縣等八州縣衛、河南通許等二十七州縣、山東齊河等八十七州縣水災。丁酉,上奉皇太后回蹕。乙巳,撥奉天糧十萬石賑山東。丁未,致仕大學士查郎阿卒。戊申,諭江蘇清查積欠,以陳維新與侍郎陳德華規避,均褫職。壬子,賑河南許州水災。甲寅,以顧琮為浙江巡撫,蘊著為漕運總督。乙卯,賑兩淮呂田等二十場水災。丁巳,以陳大受為兵部尚書,調潘思矩為福建巡撫,以納敏為安徽巡撫。

冬十月辛酉,以蘇祿復遣番人至福建申理呂宋番目劫奪貢使事,諭:「島夷互爭,可聽其自辦,不必有所袒護。」乙丑,上以皇太后疾,詣慈寧宮問安視藥。是日,宿慈寧宮。每日視藥三次,至辛未皆如之。庚午,賑江蘇阜寧等二十州縣衛水災。丁丑,免吉林被水地方額賦。戊寅,賑浙江海寧等十一縣水災。己卯,以準噶爾赴藏熬茶,宰桑巴雅斯瑚朗等至得卜特爾交易,召慶復回京。壬午,賑江蘇常熟等十九州縣衛潮災,上元等十五州縣衛旱災,命江蘇復截明歲漕糧四十萬石備賑。癸未,諭張廣泗勿受莎羅奔降。

十一月丁亥朔,上詣皇太后視藥,日三次,至己丑皆如之。召阿里袞來京,以赫赫護山東巡撫。癸巳,賑浙江壽昌等三縣饑,補豁被災額賦。己酉,額駙策凌陛見,以塔爾瑪善暫署定邊副將軍。庚戌,賑江蘇崇明等縣災民有差。癸丑,賑山東東平等州縣衛災民。辛酉,賑安徽歙縣等州縣衛水災。己巳,召徐杞來京,調陳宏謀為陜西巡撫,以彭樹葵署湖北巡撫。賑山東齊河等八十五州縣水災。辛未,予告大學士徐本卒。乙亥,以張廣泗進剿大金川,命黃廷桂署陜甘總督。賑直隸天津等六州縣水災。張廣泗奏莎羅奔請降,告以此次用兵,不滅不已。上以「用卿得人」勉之。己卯,以大學士慶復進剿瞻對,奏報班滾自焚不實,命褫職待罪。以班第、努三均奏班滾自焚,罷御前行走。庚辰,以來保為武英殿大學士。

十三年春正月壬辰,賑江蘇阜寧等縣、安徽宿州等五州縣水災。庚子,命傅恆兼管兵部尚書事。辛丑,命訥親赴浙江同高斌會鞫巡撫常安。乙巳,命阿克敦協辦大學士,傅恆協辦巡幸內閣事務。戊申,上至曹八屯。甲寅,大學士張廷玉乞休,溫諭慰留之,停兼理吏部,以來保代之。

二月戊午,上東巡,奉皇太后率皇后啟鑾。癸亥,上駐蹕趙北口,奉皇太后閱水圍。朝鮮、琉球入貢。甲子,賑直隸天津等十五州縣水災。丙寅,常安坐婪收褫職。壬申,福建甌寧會匪作亂,總兵劉啟宗捕剿之。癸酉,加經過山東被災州縣賑一月。罷奇通阿領侍衛內大臣,以阿里袞代之。乙亥,免直隸、山東經過州縣額賦十分之三。戊寅,上駐蹕曲阜縣,免駐蹕之山東曲阜、泰安、歷城三縣己巳年額賦。己卯,上釋奠禮成,謁孔林。詣少昊陵、周公廟致祭。命留曲柄黃傘供大成殿,賜衍聖公孔昭煥及博士等宴。壬午,上駐蹕泰安府。癸未,上祭岱嶽廟,奉皇太后登岱。

三月乙酉,減直隸、山東監候、緩決及軍流以下罪。丁亥,命班第赴金川軍營協商軍務。諭張廣泗、班第調岳鍾琪赴軍營,以總兵用。戊子,上至濟南府,幸趵突泉。己丑,上奉皇太后閱兵,謁帝舜廟。庚寅,上閱城,幸歷下亭。免浙江餘姚等五縣潮災本年漕糧。壬辰,上奉皇太后率皇後回蹕。癸巳,免安徽歙縣等七州縣衛上年被水額賦。乙未,上至德州登舟,皇后崩,命莊親王允祿、和親王弘晝奉皇太后回京,上駐蹕德州。召完顏偉回京,以顧琮為河東河道總督,愛必達為浙江巡撫。協辦大學士、吏部尚書劉於義卒。辛丑,還京師。大行皇后梓宮至京,奉安於長春宮。上輟朝九日。壬寅,四川成都等二十三州縣地震。甲辰,皇太后至京師,上迎還壽康宮。乙巳,上至長春宮大行皇后梓宮前致奠。丙午,上親定大行皇后謚曰孝賢皇后。以皇長子屆喪未能盡禮,罰師傅、諳達等俸有差。丁未,上至長春宮大行皇后梓宮前行殷奠禮。命高斌、劉統勛查辦山東賑務。己酉,大行皇后梓宮移觀德殿。頒大行皇后敕諭於各省。遣官賚敕諭於朝鮮及內札薩克、喀爾喀、哈密、青海等處。辛亥,調愛必達為貴州巡撫,以方觀承為浙江巡撫。丁巳,加傅恆、那蘇圖、張廣泗、班第太子太保,喀爾吉善太子少保。庚申,召駐藏副都統傅清來京,以拉布敦代之。正白旗領侍衛內大臣伊勒慎卒,以那蘇圖、旺札勒署。來保免兼領侍衛內大臣,以豐安代之。壬戌,上至觀德殿祭大行皇后。甲子,命訥親經略四川軍務。協辦大學士阿克敦免,以傅心互代之,並兼管吏部尚書。哈達哈署兵部尚書。免上年江蘇常熟等十六州縣衛潮災、上元等十四州縣衛旱災額賦。乙丑,調梁詩正為兵部尚書,以蔣溥為戶部尚書。免江蘇山陽等十八州縣衛上年被災額賦。丁卯,軍機大臣蔣溥免,以陳大受代之。癸酉,以陳大受協辦大學士,達勒當阿為刑部尚書。乙亥,起原任川陜總督岳鍾琪赴金川軍營,賞提督銜。調阿蘭泰為盛京將軍,以索拜為寧古塔將軍。丙子,起傅爾丹為內大臣,赴金川軍營。加賑福建臺灣等二縣旱災。戊寅,晉一等侯富文為一等公。庚辰,裁都察院僉都御史、通政司右通政、大理寺少卿、詹事府少詹事、太僕寺少卿、國子監司業漢缺各一。改通政司滿參議一缺為右,滿、漢左通政為通政副使。

五月甲申朔,賜梁國治等二百六十四人進士及第出身有差。乙酉,免直隸文安等三十二州縣上年被水額賦。丙戌,命傅恆署戶部三庫事。庚寅,阿克敦論斬。辛卯,張廣泗奏克戎布寨之捷。丁酉,免河南通許等二十八州縣水災額賦。壬寅,免安徽旌德等七州縣衛上年旱災額賦。甲辰,上至觀德殿冊謚大行皇后曰孝賢皇后,頒詔。丙午,釋阿克敦於獄,命署工部侍郎。戊申,免山東永利等八場上年水災額賦。壬子,免山西永濟等十二州縣上年水雹災額賦。

六月丙辰,李坦以祭祀久不到班,奪伯爵。申誡旗員。庚申,御試翰林、詹事等官,擢齊召南等三員為一等,餘升黜有差。御試由部院入翰林、詹事等官,擢少詹事世貴記名升川。癸亥,賑陜西耀州等二十二州縣旱災。戊辰,四川汶川縣典史謝應龍駐沃日土司,阻鎮將移營。上嘉之,予州同銜。己巳,命兆惠兼管戶部事。庚午,裁歸化城土默特左右翼副都統。甲戌,諭禁廷臣請立皇太子,並責皇長子於皇后大事無哀慕之誠。上至觀德殿孝賢皇后梓宮前奠酒,行百日致祭禮。

秋七月癸未朔,皇太后懿旨:「嫻貴妃那拉氏繼體坤寧,先冊立為皇貴妃,攝行六宮事。」丁亥,免福建長樂等二縣上年旱災額賦。戊子,諭訥親等速奏進兵方略。壬辰,貸山東農民籽種銀。免江蘇宿遷上年水災額賦。甲午,命高斌會周學健勘河、湖疏洩事宜。乙未,以山西永濟等五縣歉收,撫恤之。戊戌,德沛免,調達勒黨阿為吏部尚書,以盛安為刑部尚書。辛丑,賑直隸青縣等二十九州縣旱災。癸卯,阿里袞請減饑民掠奪罪,諭斥為寬縱養奸,不許。賑山東歷城等二十九州縣水雹等災。丙午,常安論絞。

閏七月癸丑朔,以阿克敦署刑部尚書,德通為左都御史。丙辰,免直隸霸州、固安水災額賦。賑湖南益陽等八州縣水災。戊午,以彭樹葵為湖北巡撫。戊辰,周學健以違制薙發,逮下獄。命高斌管南河總督。尹繼善以瞻徇,褫職留任。己巳,上幸盤山,以新柱署湖廣總督。召安寧來京,以尹繼善兼理江蘇巡撫。寧古塔將軍索拜遷古北口提督,以永興代之。辛未,以訥親奏金川進剿持兩議,諭斥之,並申飭傅爾丹、岳鍾琪、班第等。壬申,上駐蹕盤山。癸酉,調準泰為山西巡撫,阿里袞為山東巡撫,鄂昌為江蘇巡撫,舒輅為廣西巡撫。塞楞額以違制薙發,逮下獄。丁丑,賑雲南昆陽等州縣水災。戊寅,召阿里袞來京,以唐綏祖護山東巡撫。己卯,免江蘇元和等十縣本年雹災額賦。庚辰,上還宮。

八月甲申,以班第署四川巡撫。乙酉,以謁泰陵,命莊親王允祿等總理在京事務。癸巳,追議征瞻對言狂奏罪,下慶復於獄,許應虎論斬。庚子,諭撫恤四川打箭爐地震災民。命來保兼管工部尚書。辛丑,上詣泰陵。甲辰,召安寧來京。乙巳,上謁泰陵。丙午,免直隸慶雲等二縣九年逋賦。丁未,命戶部侍郎兆惠赴四川軍營督運。訥親請調兵三萬進剿,不許。戊申,命倉場侍郎張師載往江南隨高斌學習河務。己酉,上還京師。

九月壬子朔,調鄂昌為四川巡撫。命策楞、高斌會鞫周學健。戊午,賜塞楞額自裁。己未,召北路參贊大臣塔爾瑪善、努三來京,以穆克登額、薩布哈善代之。訥親等奏克申札、申達諸城。調策楞為兩江總督,尹繼善為兩廣總督。辛酉,召訥親、張廣泗來京。命傅爾丹護四川總督,與岳鍾琪相機進討。甲子,起董邦達在內廷行走。命尚書班第赴軍營,同傅爾丹、岳鍾琪辦理軍務。命軍營內大臣以下聽傅爾丹節制。丁卯,召黃廷桂來京,以瑚寶署甘肅巡撫,兼辦陜甘總督事。己巳,上幸靜宜園閱兵。壬申,簡親王神保住以凌虐兄女,奪爵。癸酉,命德沛襲簡親王。丁丑,諭責訥親、張廣泗老師糜餉,飭訥親繳經略印。己卯,命傅恆暫管川陜總督事,赴軍營。命侍郎舒赫德軍機處行走。庚辰,訥親、張廣泗以貽誤軍機,褫職逮問。召張廣泗來京,發訥親北路軍營效力。以傅恆為經略,統金川軍務。辛巳,命來保暫管戶部。

冬十月壬午朔,調滿洲兵五千名赴金川軍營。諸王大臣請治訥親罪。諭責訥親負國負恩,候回奏再行降旨。乙酉,召尹繼善來京,以碩色為兩廣總督,鄂容安署河南巡撫。賑湖南新寧縣水災。丙戌,班第以不劾訥親罪,降調。以舒赫德為兵部尚書。丁亥,命傅恆為保和殿大學士,兼管戶部。戊子,移孝賢皇后梓宮於靜安莊,上如靜安莊奠酒。乙丑,賑山東鄒平等三十州縣衛水災。以尹繼善為戶部尚書。辛卯,上幸豐澤園,賜經略傅恆並從征將士宴。岳鍾琪奏克跟雜之捷。壬辰,調開泰為湖南巡撫,以唐綏祖為江西巡撫。甲午,賑山西陽曲等十五州縣旱災。戊戌,上幸寶諦寺,閱八旗演習雲梯兵。丁未,賑安徽阜陽等州縣衛災。己酉,命尹繼善協辦大學士。壬子,幸重華宮,賜經略傅恆宴。癸丑,上詣堂子行祭告禮,及祭吉爾丹纛。甲寅,賑江蘇銅山縣、湖北漢川等八州縣衛水災。丙辰,命各省巡撫皆兼右副都御史銜。丁巳,上幸南苑行圍。戊午,上閱兵。戊辰,賜周學健自裁。平郡王福彭卒,輟朝二日。己巳,命尹繼善在軍機處行走。賑福建晉江等十四縣旱潮等災。庚午,免直隸文安等三縣水災地租。癸酉,上幸豐澤園,賜東三省兵隊宴,並賞賚有差。以策楞為川陜總督,雅爾哈善署兩江總督。以傅恆日馳二百餘里,嘉勞之。甲戌,給尹繼善欽差大臣關防,署川陜總督。丁丑,以訥親請命張廣泗、岳鍾琪分路進兵,責以前後矛盾,逮治之。己卯,以用兵金川勞費,密諭傅恆息事寧人。庚辰,分設四川、陜甘總督,以尹繼善為陜甘總督,策楞為四川總督,管巡撫事,鄂昌為甘肅巡撫。調舒赫德為戶部尚書,瑚寶為兵部尚書。

十二月甲申,定內閣大學士滿、漢各二員,協辦大學士滿、漢一員或二員,改所兼四殿二閣為三殿三閣。乙酉,加傅恆太保。命阿克敦協辦大學士。丁亥,以黃廷桂為兩江總督。上御瀛臺,親鞫張廣泗。戊子,遣舒赫德逮訥親赴軍營,會傅恆嚴鞫之。以海望署戶部尚書,哈達哈署兵部尚書、步軍統領。辛卯,慶復、李質粹論斬。大學士陳世倌罷。壬辰,張廣泗處斬。丙寅,密諭傅恆,明年三月不能奏功,應受降撤兵。丁酉,命川、陜督撫皆聽傅恆節制,班第專辦巡撫事務,兆惠專辦糧運。免高斌大學士,仍留南河總督任。癸卯,命傅恆等訊明訥親,以其祖遏必隆刀於軍前斬之。甲辰,賑陜西耀州等二十五州縣旱災。

十四年春正月辛亥,諭傅恆、岳鍾琪由黨壩進剿,傅爾丹辦理卡撒一路。癸丑,以大學士張廷玉年老,命五日一進內備顧問。諭傅恆以四月為期,納降班師。乙卯,賑山東金鄉等州縣災。丁巳,命傅爾丹、達勒黨阿、舒赫德、尹繼善、策楞參贊大金川軍務。戊午,命瑚寶署陜甘總督,侍郎班第褫職,仍署四川巡撫。甲子,召傅恆還京。命尚書達勒黨阿、舒赫德、尹繼善均回任,策楞、岳鍾琪辦理大金川軍務。丙寅,以傅爾丹請深入,嚴飭之。丁卯,以大金川莎羅奔、郎卡乞降,命傅恆班師,特封忠勇公。丙子,諭傅恆受莎羅奔等降。丁丑,南掌國王島孫進牙象。

二月乙酉,唐綏祖請率屬捐廉助餉。上以不知政體,嚴飭之。丙戌,加來保太子太傅,陳大受、舒赫德、策楞、尹繼善太子太保,汪由敦、梁詩正太子太師,達勒黨阿、納延泰、阿克敦、哈達哈太子少師。壬辰,傅恆奏,於二月初五日設壇除道,宣詔受大金川土司莎羅奔、土舍郎卡降。賜傅恆四團龍補服,加賜豹尾槍二、親軍二,岳鍾琪加太子少保。癸巳,以岳鍾琪親至勒烏圍招莎羅奔等來降,諭特嘉之。丙申,召拉布敦、眾佛保來京。庚子,命舒赫德查閱云南等省營伍,會同新柱勘金沙江工程,以瑚寶署湖廣總督。乙巳,上幸豐澤園演耕。莎羅奔進番童番女各十人,詔卻之。

三月癸丑,命皇長子及裕親王等郊迎傅恆。乙卯,上奉皇太后至靜安莊孝賢皇后梓宮前臨奠。丁巳,上率經略、大學士、公傅恆詣皇太后宮問安。封嶽鍾琪為三等公,加兵部尚書銜。己未,命傅恆兼管理籓院,來保兼管兵部。命那木札勒、德保仍為總管內務府大臣。辛酉,上詣東陵。甲子,上謁昭西陵、孝陵、孝東陵、景陵。丁卯,上至南苑行圍。癸酉,上謁泰陵。甲戌,賑湖北漢川等六州縣水災。乙亥,免直隸保安等十州縣旱災額賦。丁丑,裁直隸河道總督,兼理加入關防敕書。富森改西安將軍。以傅爾丹為黑龍江將軍。

四月壬午,上御太和殿,奉皇太后命,冊封嫺貴妃那拉氏為皇貴妃,攝六宮事。甲申,改來保兼管刑部。召蘊著來京,以顧琮署漕運總督。命納延泰等勘察哈爾災。乙酉,加上皇太后徽號曰崇慶慈宣康惠皇太后,次日頒詔覃恩有差。辛卯,免山東鄒平等二十州縣水災、甘肅皋蘭等十二州縣雹災額賦。召彭樹葵來京,調唐綏祖為湖北巡撫,以阿思哈為江西巡撫。命倉場侍郎張師載以原銜協辦江南河務。戊戌,以瑚寶為漕運總督,命唐綏祖署湖廣總督。調哈達哈為兵部尚書,以三和為工部尚書。免山東王家岡等四場額賦。己亥,命江西巡撫兼提督銜。庚子,召納敏來京,以衛哲治為安徽巡撫。乙巳,賑福建臺灣等三縣災。免湖南新寧上年水災額賦。

五月乙卯,免甘肅皋蘭等十三州縣旱災額賦。丙辰,免安徽阜陽等十三州縣衛上年旱災額賦。辛酉,上至黑龍潭祈雨。

六月丙申,賑甘肅渭源等州縣旱災。己亥,廣西學政胡中藻以裁缺怨望,命來京候補,仍下部嚴議。

秋七月戊申,賑福建光澤等二縣水災。庚戌,免湖北漢川等六州縣上年水災額賦。辛亥,直隸總督那蘇圖卒。免福建晉江等九縣潮災額賦。壬子,以方觀承為直隸總督,陳大受署之,永貴署山東巡撫。命來保兼管吏戶二部,阿克敦兼署步軍統領。庚申,上奉皇太后駐避暑山莊。辛酉,命傅恆、陳大受譯西洋等國番書。丁卯,上奉皇太后木蘭行圍。乙亥,補蠲山西永濟等六州縣被災額賦。

八月庚辰,上行圍巴顏溝,蒙古諸王等進筵宴。壬午,賑湖北羅田等二縣水災。癸卯,賑河南延津等七縣水災。甲辰,賑湖北潛江等十三州縣水災。

九月乙卯,上奉皇太后回蹕。乙丑,授鄂容安河南巡撫。丙寅,瞻對番目班滾降。賜慶復自裁。

冬十月甲午,賑浙江錢塘等二十二州縣、鮑郎等十八場水災。賞傅清都統銜,同紀山駐藏,掌欽差大臣關防。丁酉,召八十五來京,以卓鼐為歸化城都統。戊戌,飭四川嚴緝啯匪。以珠爾默特那木札勒縱恣,諭策楞、岳鍾琪、傅清、紀山防之。喀爾喀臺吉額林沁之子旺布多爾濟獲額魯特逃人,上嘉賚之。免江蘇阜寧等二十三州縣漕糧有差。己亥,免直隸薊州等十八州縣水災額賦,並賑之。甲辰,召原任左副都御史孫嘉淦來京。

十一月丁未,命梁詩正兼管吏部尚書。癸亥,命刑部尚書汪由敦署協辦大學士。戊辰,大學士張廷玉乞休,允之。庚辰,以劉統勛為工部尚書。辛巳,起彭維新為左都御史。癸未,賜張廷玉詩,申配饗之命。丁亥,汪由敦以漏洩諭旨,免協辦大學士,留尚書任。以梁詩正協辦大學士。辛卯,削致仕大學士張廷玉宣勤伯爵,以大學士原銜休致,仍準配享太廟。調哈達哈為工部尚書,舒赫德為兵部尚書,海望為戶部尚書。以木和蘭為禮部尚書,新柱為吉林將軍,永興為湖廣總督。乙未,召衛哲治來京,調圖爾炳阿為安徽巡撫,岳濬為雲南巡撫。以蘇昌為廣東巡撫。

十五年春正月丙午,免直隸、山西、河南、浙江未完耗羨。免江蘇、安徽、山東耗羨十分之六。丁未,命張允隨為東閣大學士,碩色為雲貴總督,陳大受為兩廣總督,梁詩正為吏部尚書,李元亮為兵部尚書。甲寅,上幸瀛臺紫光閣,賜準噶爾使尼瑪宴。乙卯,召紀山回京,命拉布敦同傅清駐藏辦事。壬戌,命工部侍郎劉綸在軍機處行走。李質粹處斬,王世泰、羅於朝論斬。

二月乙亥,上奉皇太后西巡五臺,免經過地方額賦三分之一。庚辰,朝鮮入貢。丙戌,上奉皇太后駐蹕五臺山菩薩頂。己丑,定邊左副將軍喀爾喀超勇親王策凌卒,命貝勒羅布藏署定邊左副將軍。丁酉,再免山西蒲縣等二縣上年被災額賦十分之三。戊戌,上駐趙北口行圍。辛丑,採訪經學遺書。癸卯,上閱永定河堤工。

三月丙午,加張允隨太子太保,蔣溥、方觀承、黃廷桂太子少保。再免直隸薊州等十七州縣額賦十分之三。己酉,上奉皇太后還京師。甲寅,孝賢皇后二周年,上詣靜安莊致奠。乙卯,致仕大學士張廷玉回籍,優賚有加,令散秩大臣領侍衛十員護送之。戊午,免安徽貴池等三十州縣十四年水災額賦,並賑之。乙丑,免湖北潛江等四州縣十四年水災額賦。庚午,免山東鄒平等二十七州縣衛十四年水災額賦。

夏四月丙子,雲南省城火藥局災。壬辰,起阿桂在吏部員外郎上行走。乙未,罷致仕大學士張廷玉配享。免安徽貴池等三十州縣衛十四年水災額賦。戊戌,召拉布敦來京,命班第駐西藏,紀山駐青海。

五月庚戌,上詣黑龍潭祈雨。辛亥,命刑部清理庶獄,減徒杖以下罪,直隸亦如之。癸丑,諭九卿科道直陳闕失。甲寅,召新柱來京,以卓鼐為吉林將軍,眾佛保為歸化城都統。庚午,上詣黑龍潭祈雨。

六月丙子,以喀爾喀親王成袞札布為定邊左副將軍。丙申,賑直隸樂亭水災。以保德為北路軍營參贊大臣。

秋七月丙午,廣東巡撫岳濬褫職。命圖爾炳阿、衛哲治仍留雲南、安徽巡撫任。己酉,命劉統勛赴廣東查折米收倉積弊。庚申,汪由敦降兵部侍郎。以劉統勛為兵部尚書,孫嘉淦為工部尚書。乙丑,緬甸入貢。

八月壬申,上御太和殿,奉皇太后懿旨,冊立皇貴妃那拉氏為皇后。癸酉,以冊立皇后,上率王大臣奉皇太后御慈寧宮行慶賀禮,加上皇太后徽號曰崇慶慈宣康惠敦和皇太后。丁亥,上奉皇太后率皇后謁陵,並巡幸嵩、洛。戊子,命紀山赴西寧辦事,班第赴藏辦事,代拉布敦回京。庚寅,上奉皇太后謁昭西陵、孝陵、孝東陵、景陵。甲午,左都御史德通、彭維新,左副都御史馬靈阿以瞻徇傅恆議處,降革有差。丁酉,賑山東嶧縣等七州縣水災。

九月庚子朔,以梅★成為左都御史。壬寅,上奉皇太后率皇后謁泰陵。癸卯,御史索祿等以劾蔣炳矯飾,諭斥其有心亂政,褫職。丙午,吏部奏原任大學士張廷玉黨援門生,又與硃荃聯姻,應革職治罪。上特免之。己酉,上駐正定府閱兵。辛亥,以拉布敦為左都御史。丙辰,免河南經過地方額賦十分之三。丁巳,上駐蹕彰德府,幸精忠廟。辛酉,上駐蹕百泉,奉皇太后幸白露園。準噶爾臺吉策旺多爾濟那木札勒為部人所弒,立其兄喇嘛達爾札。癸卯,再免河南歉收地方額賦十分之五。乙丑,賑福建閩縣等九縣水災。己巳,免河南祥符等縣明年額賦。雲南河陽地震。

冬十月辛未,幸嵩山。丙子,上奉皇太后駐蹕開封府。戊寅,上幸古吹臺。加鄂容安為內大臣。賑浙江淳安水災。甲申,調愛必達為雲南巡撫、開泰為貴州巡撫,以楊錫紱為湖南巡撫。乙酉,免江蘇清河等九州縣水災額賦。戊子,免山西應州等三州縣水災額賦。甲午,免順直固安等四十六州縣水雹各災額賦,仍賑貸有差。戊戌,賑江蘇溧陽等州縣水災。

十一月辛丑,上奉皇太后率皇后還京師。己酉,賑甘肅平涼二十八州雹旱災。壬子,免山東蘭山等縣旱災額賦,並賑之。癸丑,珠爾默特那木扎勒謀作亂,駐藏都統傅清、左都御史拉布敦誘誅之。其黨卓呢羅卜藏扎什等率眾叛,傅清、拉布敦遇害。甲寅,命策楞、岳鍾琪率兵赴藏,調尹繼善赴四川經理糧餉,命侍郎那木扎勒同班第駐藏。逮紀山來京,命舒明駐青海,眾佛保署之。乙卯,宣諭珠爾默特那木扎勒戕其兄車布登及悖逆諸狀。追贈傅清、拉布敦為一等伯,封傅清子明仁、拉布敦子根敦為一等子,世襲。命侍郎兆惠赴藏,同策楞辦善後事宜。丙辰,命舒赫德仍在軍機處行走。調穆和藺為左都御史,以伍齡安為禮部尚書。召雅爾哈善來京,以王師為江蘇巡撫。丁巳,命策楞擇藏番目與班第協辦噶布倫事務。乙丑,以阿里袞為湖廣總督,調阿思哈為山西巡撫,衛哲治為廣西巡撫,以定長為安徽巡撫。戊辰,以捕獲卓呢羅布藏扎什等,亂已定,止岳鍾琪進藏,命駐打箭爐。

十二月庚午朔,賑盛京高麗堡等六站水災。壬申,始命漢大臣梁詩正等恩廕分部學習。戊寅,賑兩淮莞瀆等三場水災。庚辰,命舒赫德勘浙江海塘。壬午,烏里雅蘇臺參贊大臣薩布哈沙褫職,以寶德代之。戊子,賑盛京遼陽等七城、承德等六州縣水災,並蠲緩額賦有差。癸巳,唐綏祖被劾免,以嚴瑞龍護湖北巡撫。

十六年春正月庚子,以初次南巡,免江蘇、安徽元年至十三年逋賦,浙江本年額賦,減直省緩決三次以上人犯罪。以上年巡幸嵩、洛,免河南十四年以前逋賦。辛丑,賑安徽宿州等州縣上年水災。癸卯,以江蘇逋賦積至二百二十餘萬,諭釐革催徵積弊。丙午,免甘肅元年至十年逋賦。以嚴瑞龍署湖北巡撫。辛亥,上奉皇太后南巡。癸丑,免經過直隸、山東地方本年額賦十分之三。自是南巡皆如之。壬戌,卓呢羅布藏札什等伏誅。癸亥,賑安徽歙縣等十五州縣旱災。甲子,免山東鄒平等縣逋賦及倉穀。

二月辛未,賑山東蘭山等七州縣旱災。癸酉,免兩淮灶戶逋賦。乙亥,命喀爾喀親王德沁扎布為喀爾喀副將軍,公車布登扎布為參贊大臣。丙子,上奉皇太后渡河,閱天妃閘。丁丑,閱高家堰。辛巳,免山東嶧縣等七州縣水災額賦有差。乙酉,上幸焦山。丙戌,調定長為廣西巡撫。己丑,上駐蹕蘇州,諭三吳士庶,各敦本業,力屏浮華。辛卯,宣布珠爾默特那木扎勒叛逆罪狀,懲辦如律。嚴瑞龍褫職,命阿里袞兼湖北巡撫。壬辰,免江蘇武進等縣新舊田租,免興化縣元年至八年逋賦。癸巳,準噶爾使額爾欽等覲於蘇州行宮。

三月戊戌朔,上奉皇太后幸杭州府。貸黑龍江呼蘭地方水災旗民,免官莊本年額賦。免浙江淳安縣水災本年漕糧。己亥,以張師載為安徽巡撫。庚子,上幸敷文書院,幸觀潮樓閱兵。甲辰,裁杭州漢軍副都統。乙巳,上祭禹陵。丙午,上奉皇太后還駐杭州府。丁未,閱兵。戊申,命高斌仍以大學士銜管河道總督事。庚戌,諭浙江士庶崇實敦讓,子弟力田。命班第掌駐藏欽差大臣關防。辛亥,東閣大學士張允隨卒。癸丑,上奉皇太后駐蹕蘇州府。甲寅,賑廣東海康等縣水災。乙卯,幸宋臣範仲淹祠,賜園名曰高義,賞後裔範宏興等貂幣。辛酉,上奉皇太后幸江寧府。壬戌,上祭明太祖陵。乙丑,賜紀山自裁。丁卯,起陳世倌為文淵閣大學士。免江蘇江浦等十五州縣被災額賦有差。

夏四月辛未,吉林將軍卓鼐改杭州將軍,以永興代之。免甘肅皋蘭等九州縣十三年被災額賦。癸酉,上閱蔣家壩。免江南沛縣九年以前逋賦。甲戌,賑浙江永嘉等十州縣場衛水災。賑廣東龍川等十二州縣十五年水災。丙子,賑江蘇山陽等二十四州縣衛十五年水災。己卯,免甘肅狄道等二十州縣十四年被水旱雹霜災額賦有差。以恆文為湖北巡撫。癸未,免河南鄢陵等十六州縣十四年水災額賦。乙酉,永興褫職逮問,吉林將軍卓鼐降調,以傅森代之。丙戌,上駐蹕泰安府,祀東嶽。戊子,詔以五月朔日食,行在徹懸、齋戒。己丑,遣履親王允祹代行常雩禮。

五月丁酉朔,日食。丁未,上臨奠都統傅清、左都御史拉布敦。戊申,以永興等誣劾唐綏祖,給還籍產,召來京。辛亥,賜吳鴻等二百四十三人進士及第出身有差。丁巳,免廣東海康等十一州縣十五年風災額賦。己未,嚴瑞龍以誣告唐綏祖,論斬。癸亥,賑山東掖縣等六州縣潮災。

閏五月戊寅,調黃廷桂為陜甘總督,尹繼善為兩江總督。戊子,以永貴為浙江巡撫。壬辰,命保舉經學之陳祖範、吳鼎、梁錫興、顧棟高進呈著述,原赴部引見者聽。癸巳,直隸河間等州縣蝗。是月,免山西太原等十九州縣上年水雹等災額賦有差。賑山東壽光等六縣、官臺等三場,福建寧化等二縣水災,雲南劍川等七州縣地震災。

六月己亥,起唐綏祖為山西按察使。壬子,賑江蘇靖江縣雹災。賑廣東英德等四州縣水災。賑山西鳳臺、高平水災。甲寅,免江蘇沛縣上年水災額賦。丙辰,免浙江永嘉等七州衛上年旱災額賦。賑福建寧化等縣水災。庚申,緬甸入貢。辛酉,免安徽壽州等二十五州縣水災額賦。甲子,準噶爾部人布圖遜林特古斯來降。

秋七月庚午,賑福建歸化等縣水災。壬申,上奉皇太后秋獮木蘭。戊寅,上奉皇太后駐蹕避暑山莊。己卯,河南陽武十三堡河決。庚辰,上奉皇太后巡幸木蘭,行圍。乙卯,免山西清水河雹災額賦。丙戌,賑陜西朝邑縣水災。己丑,賑山東平度等州縣水災。壬辰,賑山西鳳臺等九縣水災。

八月乙未,賑浙江海寧等六十五州縣衛所及大嵩等場旱災。賑江西上饒等七縣被旱災。賑湖北天門旱災。丙申,賜陳祖範、顧棟高國子監司業銜。戊戌,以碩色舉發偽撰孫嘉淦奏稿,假造硃批,諭方觀承等密緝之。己酉,上奉皇太后回駐避暑山莊。辛亥,命修房山縣金太祖陵、世宗陵。丁巳,上奉皇太后還京師。己未,賑河南商丘等十四縣水災。庚寅,準泰以徇隱偽奏,褫職逮問。調鄂容安為山東巡撫、舒輅為河南巡撫、鄂昌為江西巡撫,以楊應琚為甘肅巡撫。命高斌赴河南辦陽武河工。辛酉,以莊有恭為江蘇巡撫。癸亥,免甘肅平涼等五州縣雹災碩賦。乙丑,定明年二月各省舉行恩科鄉試。詔停本年秋決。癸酉,賑山東鄒平等五十三州縣水災。丙子,上奉皇太后詣泰陵。丁丑,賑福建福安等二縣水災。庚辰,上奉皇太后謁泰陵。是日,回蹕。甲申,命舒赫德赴江南查辦偽撰孫嘉淦奏稿事。庚寅,命陳世倌兼管禮部。兩廣總督陳大受卒,調阿里袞代,以永常為湖廣總督。辛卯,賑河南上蔡等州縣水災。癸巳,賑福建霞浦等四縣潮災。

冬十月戊戌,以範時綬署湖南巡撫。壬寅,賑長蘆屬富國等七場、山東王家岡等三場水災。甲寅,賑安徽歙縣等十八州衛旱災。丙辰,賑江蘇銅山等八州縣水災。調陳宏謀為河南巡撫,舒輅為陜西巡撫。賑山東齊東等七州縣本年水災、榮成縣雹災。戊午,賑直隸武清等二十六州縣水雹災。癸亥,賑山東官臺二場灶潮災。

十一月甲戌,賑河南祥符等五縣水災。乙亥,賑直隸東明等三州縣本年水災。庚辰,陽武決口合龍。乙酉,以皇太后六旬萬壽,上徽號曰崇慶慈宣康惠敦和裕壽皇太后,頒詔覃恩有差。丙戌,命高斌、汪由敦會勘天津河工。戊子,皇太后聖壽節,上奉皇太后御慈寧宮,率王公大臣行慶賀禮。

十二月癸巳朔,以烏爾登為北路軍營參贊大臣。丁酉,濬永定河引河。戊戌,賑吉林琿春地方本年水災。庚子,賑山東鄒平等五十五州縣水災。壬寅,以雅爾哈善為浙江巡撫。甲辰,濬直隸南北兩運減河。命多爾濟代班第駐藏辦事。辛亥,賑浙江鄞縣等六十州縣衛所、大嵩等八場旱蟲災。

十七年春正月乙亥,賜準噶爾使圖卜齊爾哈朗等宴。庚戌,設盛京總管內務府大臣,以將軍兼管。甲申,以準噶爾達瓦齊、阿睦爾撒納內訌,增兵阿爾泰邊隘。命舒赫德、玉保查閱北路軍營。丙戌,以阿巴齊、達清阿為北路參贊大臣。丁亥,賑江蘇銅山等六州縣、安徽歙縣等九州縣被災貧民。辛卯,修直隸永定河下口及鳳堤。

二月乙未,以鍾音為陜西巡撫。己亥,釋準泰。甲寅,上詣東陵。丙辰,布魯克巴之額爾德尼第巴貢方物。丁巳,上謁昭西陵、孝陵、孝東陵、景陵。戊午,上駐蹕盤山。己未,賑山西山陰、虞鄉被災貧民。辛酉,修房山縣金太祖、世宗陵。

三月戊辰,以浙東災重,諭雅爾哈善加賑,毋令流移。庚午,上還宮。壬申,以莫爾歡為歸化城都統。戊寅,福建巡撫潘思矩卒,調陳宏謀為福建巡撫,以蔣炳為河南巡撫。

夏四月甲午,免山東齊東等十二州縣衛上年水災額賦。乙巳,免直隸武清等二十三州縣上年水災額賦。庚戌,免浙江海寧等七十三州縣衛及大嵩等十三場上年水災額賦。丁巳,免直隸永利等四場、山西山陰等縣上年水災額賦。

五月辛未,直隸東光、武清等四十三州縣蝗。庚辰,賑河南祥符等十四縣水災。己丑,賑甘肅狄道等十四州縣上年水災。山東濟南等八府蝗,江南上元等十二州縣生蝻。

六月甲午,準噶爾部人呢雅斯來降。丁未,御試翰林、詹事等官,擢汪廷興等三員為一等,餘升黜有差。試滿洲由部院改入翰林、詹事等官,擢德爾泰為一等,餘降用有差。丙辰,以鄂樂舜為甘肅巡撫。

秋七月丁丑,上奉皇太后秋獮木蘭。己卯,免所過州縣錢糧十分之三。癸未,上奉皇太后駐避暑山莊。丁亥,賑江蘇銅山等縣水災。

八月丙申,順天鄉試內簾御史蔡時田、舉人曹詠祖坐交通關節,處斬。壬寅,撫賑福建晉江等縣風災。甲辰,上奉皇太后巡幸木蘭,行圍。丙午,命黃廷桂查辦陜西賑恤。乙卯,賑陜西咸寧等二十一州縣旱災。

九月辛酉,西洋波爾都噶爾亞國遣使入貢。四川雜穀土司蒼旺作亂,命岳鍾琪率兵剿之。庚午,蘇祿番目所齎入貢國書不合,飭喀爾吉善等遣回國。甲戌,四川官軍克雜穀腦,降番寨一百有六。予策楞、岳鍾琪優敘。戊寅,減甘肅張掖等五縣偏重額賦。賑河南被災饑民。己卯,上奉皇太后還京師。庚辰,協辦大學士、吏部尚書梁詩正請終養,許之。以孫嘉淦為吏部尚書、協辦大學士,汪由敦為工部尚書。辛巳,準噶爾喇嘛根敦林沁等來降。丁亥,召尹繼善來京,以莊有恭署兩江總督。蒼旺伏誅。

冬十月戊子朔,賜秦大士等一百四十一人進士及第出身有差。召鄂昌來京,以鄂容安署江西巡撫,楊應琚署山東巡撫。壬寅,阿思哈奏平陽紳民捐賑災銀。諭不忍令災地富民出貲,飭還之。調定長為山西巡撫,以李錫泰為廣西巡撫。己酉,上詣東陵,並送孝賢皇后安地宮。壬子,上謁昭西陵、孝陵、孝東陵、景陵。丁巳,賑江蘇上元等十九州縣、山西臨晉等十州縣、湖北鍾祥等二十五州縣衛旱災。四川雜穀、黑水後番上下寨來降。

十一月庚申,上還京師。甲子,命刑部尚書劉統勛在軍機處行走。戊辰,賑山西聞喜等五州縣旱災。庚辰,以鄂容安為江西巡撫。

十二月戊子,賑甘肅皋蘭二十一州縣水災雹災。己丑,修陜西永壽等九縣城,以工代賑。賑河南武陟縣水災。黑龍江將軍富爾丹卒,以綽爾多代之。乙巳,御史書成請釋傳鈔偽奏稿人犯忤旨,褫職。諭陳宏謀毋究捕天主教民。

十八年春正月戊午,賑陜西耀州等三十七州縣、山西永濟等十一州縣旱災。丙寅,廣東東莞縣匪莫信豐等、福建平和縣匪蔡榮祖等作亂,捕治之。戊寅,調黃廷桂署四川總督,尹繼善署陜甘總督,以鄂容安兼署兩江總督,班第署兩廣總督。辛巳,鄂昌等褫職逮問。乙酉,免山東章丘等三十一州縣衛積年逋賦。

二月丁亥朔,以岳鍾琪請用兵郭羅克,諭黃廷桂議奏。丙申,上謁泰陵。丁酉,上祭金太祖、世宗陵。江南千總盧魯生坐偽撰孫嘉淦奏稿,磔於市。己亥,皇太后自申昜春園啟蹕至涿州,上詣行宮請安。壬寅,上奉皇太后御舟至蓮花澱閱水圍。丙午,免河南夏邑等五縣十六年被水額賦。丁未,命兆惠赴藏辦事。戊申,上閱永定河工。庚戌,上幸南苑行圍。辛亥,免江蘇上元等十州縣十七年水災額賦。

三月癸亥,以雅爾哈善於查辦偽奏稿不加詳鞫,下部嚴議。戊寅,賑安徽壽州等十一州縣衛上年旱災饑民。己卯,以開泰署湖廣總督,定常署貴州巡撫。辛巳,賑湖北十九州縣衛上年旱災。

夏四月丁亥,錢陳群諫查辦偽奏稿,上斥以沽名,並飭勿存稿,以「爾子孫將不保首領」諭之。己丑,西洋博爾都噶里雅遣使貢方物,優詔答之。以恆文署湖廣總督。甲午,賜西洋博爾噶都裏雅貢使宴。乙未,免雲南劍川州十六、七年地震水災額賦有差,並賑恤之。辛丑,賜西洋博爾都噶里雅國王敕,加賚文綺珍物。丙午,以旱命刑部清理庶獄,減徒以下罪,直隸亦如之。丁未,上詣黑龍潭祈雨。壬子,命永常、努三往安西,給欽差大臣關防。

五月癸亥,減秋審、朝審緩決三次以上罪。丁卯,山東濟寧、汶上等州縣蝻。免廣東豐順等三縣上年水災額賦。辛未,免浙江仁和等六縣、仁和場上年水災額賦,並賑恤之。辛未,準噶爾臺吉喇嘛達爾札與達瓦齊相攻被執,達瓦齊自為臺吉。

六月癸巳,以策楞署兵部尚書。乙未,浙江上虞人丁文彬以衍聖公孔昭煥發其造作逆書,鞫實,磔之。丙申,天津等州縣蝗。

秋七月甲子,順天宛平等三十二州縣衛蝗。壬申,江南邵伯湖減水二閘及高郵車邏壩同時並決,命策楞、劉統勛會同高斌查辦水災。賑安徽歙、太湖等縣水災。庚辰,命莊有恭賑高郵、寶應水災。壬午,停各省分巡道兼布政使司參政、參議,按察使司副使、僉事等銜,及升用鴻臚寺少卿。

八月戊子,命履親王允祹代祭大社、大稷。賑兩淮板浦等場水災。戊戌,上奉皇太后秋獮木蘭。庚子,高斌免,以策楞署南河河道總督,同劉統勛查辦河工侵虧諸弊。辛丑,命永常、開泰各回本任。甲辰,上奉皇太后駐蹕避暑山莊。乙巳,撥江西、湖北米各十萬石賑江南災。丁未,上奉皇太后巡幸木蘭,行圍。庚戌,高斌、張師載褫職,留河工效力,以衛哲治為安徽巡撫。辛亥,賑江蘇銅山十二州縣水災、山東蘭山等縣水災。

九月庚申,賑湖北潛江等三縣水災。壬戌,河南陽武十三堡河決。丁卯,以扈從行圍畏葸不前,褫豐安公爵、田國恩侯爵,阿里袞罷領侍衛內大臣。以弘升為正白旗領侍衛內大臣。庚午,以皇后至盤山,命舒赫德為領侍衛內大臣管理內務府大臣隨往。江蘇銅山河決。壬申,命舒赫德協辦江南河工,以阿里袞署領侍衛內大臣,隨扈盤山。以尹繼善為江南河道總督,鄂容安為兩江總督,調永常為陜甘總督,開泰為湖廣總督,黃廷桂為四川總督,以定常為貴州巡撫,胡寶瑔為山西巡撫,範時綬為江西巡撫,楊錫紱為湖南巡撫。召班第來京,以策楞為兩廣總督。癸酉,上奉皇太后駐避暑山莊。甲戌,左都御史梅成休致。丙子,諭將貽誤河工之同知李焞、守備張賓斬於銅山工次。命策楞等縛高斌、張師載令目睹行刑訖釋放。丁丑,賑山東利津等縣水災。

冬十月庚寅,蘇祿國王遣使勞獨萬查剌請內附,下部議。辛卯,召劉統勛來京。乙未,賑山東海豐等六縣本年潮災。命鍾音署陜甘總督。辛丑,以楊錫紱為左都御史,調胡寶瑔為湖南巡撫,恆文為山西巡撫,以張若震為湖北巡撫。癸卯,免江蘇阜寧等二十六州縣衛新舊額賦有差。乙巳,賑安徽太湖等三十州縣衛水災。庚戌,免浙江錢塘等二十八州縣衛所旱災額賦有差。

十一月己未,召蘇昌來京,以鶴年為廣東巡撫。癸亥,江西生員劉震宇以所著治平新策有「更易衣服制度」等語,處斬。甲子,賑甘肅皋蘭等二十九州縣衛所水雹災,並免額賦有差。甲戌,以楊應琚為山東巡撫。準噶爾杜爾伯特臺吉車凌烏巴什等率所部來降。丙子,賑浙江玉環旱災。庚辰,安徽池州府知府王岱因虧空褫職,潛逃拒捕,處斬。

十二月丙戌,賑兩淮富安等場旱災。命歸降杜爾伯特臺吉車凌等移居呼倫貝爾。丁亥,協辦大學士、吏部尚書孫嘉淦卒。命玉保、努三、薩喇勒為北路參贊大臣。命舒赫德赴鄂爾坤軍營。庚寅,命戶部尚書蔣溥協辦大學士,以黃廷桂為吏部尚書,仍管四川總督,鄂爾達署之。丙申,江南張家馬路及邵伯湖二閘決口同日合龍。庚子,以準噶爾臺吉達瓦齊未遣使來京,諭永常暫停貿易。

十九年春正月壬子,賑安徽宿州等十五州縣衛、江蘇阜寧等十五州縣衛上年水災。壬戌,命薩喇勒等討入卡之準噶爾烏梁海。乙亥,命楊錫紱署吏部尚書,罷鄂彌達兼管。丁丑,琉球入貢。己卯,準噶爾臺吉車凌等入覲。

二月丙申,賑山東蘭山十八年水災。戊戌,蘇祿入貢,命廣東督、撫檄國王毋以內地商人充使。賑山東昌邑等四縣、永豐等五場潮災。癸卯,召策楞來京。乙巳,準噶爾烏梁海庫本來降。己酉,命策楞赴北路軍營。

三月辛亥朔,以白鍾山為河東河道總督,楊應琚署之。準噶爾臺吉阿睦爾撒納等與達瓦齊內閧。戊午,命舒赫德、成袞札布、薩喇勒來京。喀爾喀親王額琳沁多爾濟管理喀爾喀兵事。庚申,四川提督岳鍾琪卒。賑湖北潛江等四州縣衛水災,並蠲賦有差。癸亥,免直隸大城等十州縣十八年水雹旱災額賦。庚午,免安徽太平等二十五州縣衛十八年水災額賦,並賑之。乙亥,賑兩淮富安等十二場灶戶。

夏四月庚辰朔,加劉統勛、汪由敦太子太傅,方觀承、喀爾吉善、黃廷桂太子太保,鄂容安、開泰太子少傅,永常、碩色太子少保。命準噶爾臺吉車凌等入覲。庚寅,成袞札布降喀爾喀副將軍,以策楞為定邊左副將軍。辛卯,召班第回京。以楊應琚署兩廣總督。丙午,命都統德寧、準噶爾臺吉色布騰為北路軍營參贊大臣。是月,免長蘆滄州等二場上年旱災灶戶、直隸滄州等二州上年水災灶戶額賦。賑甘肅皋蘭等十五州縣上年旱災。賑安徽宿州等十二州縣、江蘇阜寧等二十三州縣上年水災。

閏四月庚戌朔,賜莊培因等二百三十三人進士及第出身有差。己未,免湖北潛江等四州縣衛上年水災額賦。辛未,色布騰入覲,命大學士傅恆至張家口傳旨迎勞,封貝勒。壬申,京師雨。

五月辛巳,命清保為黑龍江將軍。以準噶爾內亂,諭兩路進兵取伊犁。召永常、策楞來京,面授機宜。甲申,上奉皇太后巡幸盛京。戊子,免安徽太平等二十五州縣衛上年水災額賦。庚寅,上奉皇太后駐蹕避暑山莊。封準噶爾臺吉車水夌為親王,車凌烏巴什為郡王,車凌孟克為貝勒,孟克特穆爾、班珠爾、根敦為貝子。癸巳,免浙江廟灣等十一場十八年被水灶戶額賦,災重者賑之。丁酉,免長蘆屬永阜等三場上年水災灶戶額賦。戊戌,召陳宏謀來京。命劉統勛協同永常辦理陜甘總督事務。調陳宏謀為陜西巡撫,鍾音為福建巡撫。己亥,召雅爾哈善來京,調鄂樂舜為浙江巡撫,以鄂昌為甘肅巡撫。

六月壬子,賑福建龍溪等州縣水災。庚申,賑甘肅皋蘭等五州縣旱災。壬戌,阿睦爾撒納等為達瓦齊所敗,奔額爾齊斯夔博和碩之地。諭策楞等接應歸附。壬申,命雅爾哈善署戶部侍郎,在軍機處行走。

秋七月辛巳,賑直隸薊州等州縣水災。壬午,上奉皇太后詣盛京。癸未,命護軍統領塔勒瑪善、副都統扎勒杭阿為北路軍營參贊大臣。丙戌,以烏梁海人巴朗逃,降車布登為貝子,參贊大臣安崇阿、德寧論斬。丁酉,阿睦爾撒納率部眾來降,命薩喇勒迎勞。己亥,上駐蹕彰武臺河東大營,奉皇太后御行幄。庚子,以喀爾喀臺吉丹巴札布失機,命處斬。召策楞、舒赫德、色布騰、薩喇勒來京,以額琳沁多爾濟署將軍,兆惠為參贊大臣。壬寅,命阿睦爾撒納入覲。丙午,以班第為兵部尚書,署定邊左副將軍。以阿里袞為步軍統領。賑江蘇興化等州縣水災。

八月辛亥,授楊應琚兩廣總督。癸丑,命達勒黨阿為黑龍江將軍。甲寅,上駐蹕吉林。乙卯,上詣溫德亨山望祭長白山、松花江。丁巳,召鄂容安赴行在,以尹繼善署兩江總督。己未,賑齊齊哈爾等三城水災。庚申,賑甘肅皋蘭等五州縣旱災。丙寅,上閱輝發城。丁卯,命阿睦爾撒納游牧移鄂爾坤、塔密爾。癸酉,以車凌孟克及車凌烏巴什、訥默庫為西路參贊大臣。乙亥,北路以達勒黨阿、烏勒登、努三、兆惠為參贊大臣,西路以薩喇勒、阿蘭泰、玉保為參贊大臣。

九月丁丑朔,賑兩淮角斜等場灶潮災。辛巳,上奉皇太后率皇后謁永陵。薩喇勒等征烏梁海。甲申,免甘肅皋蘭等十五州縣被水被雹額賦。丙戌,謁昭陵、福陵。丁亥,上奉皇太后駐蹕盛京。戊午,上率群臣詣皇太后行慶賀禮。御崇政殿受賀。免奉天府所屬本年丁賦。自山海關外及寧古塔等處,已結、未結死罪均減等,軍流以下悉免之。朝鮮國王李昑遣使詣盛京貢獻。己丑,停本年秋決。辛卯,上謁文廟。癸巳,上御大政殿,盛京宗室、覺羅、將軍等進御膳。甲午,上奉皇太后率皇后自盛京回蹕。己亥,減直隸武清等四縣額賦。辛丑,以班第為定邊左副將軍,鄂容安為參贊大臣。癸卯,命車凌烏巴什、訥默庫、車凌孟克等赴西路,在參贊大臣上行走,喀爾喀王巴雅爾什第等在北路軍營領隊上行走。

冬十月癸丑,賑山東惠民等十六州縣衛、永和等三場水災。甲寅,調衛哲治為廣西巡撫,鄂樂舜為安徽巡撫,以周人驥為浙江巡撫。乙卯,賑安徽壽州等十九州縣衛本年水災、山西馬邑雹災。丙辰,上奉皇太后還宮。戊午,上御太和殿,受王以下文武百官進表朝賀。己未,以工部尚書汪由敦管刑部尚書。辛酉,賑江蘇阜寧等十六州縣衛水災,並蠲賦有差。辛未,移京城滿洲兵三千駐阿勒楚喀等處屯墾,增副都統一、協領一。庚午,以鄂彌達署吏部尚書。

十一月戊寅,賑福建諸羅等二縣風災。上幸南苑。蘇祿國王蘇老丹嘛喊奡麻安柔律噒遣使貢方物。準噶爾克爾帑特臺吉阿布達什來降。庚辰,賑順天直隸武清等十五州縣被水被雹饑民,並免額賦有差。乙酉,上幸避暑山莊。丁亥,輝特臺吉阿睦爾撒納、杜爾伯特臺吉訥默庫等率降眾於廣仁嶺迎駕。是日,上召見阿睦爾撒納等賜宴,賞賚有差。戊子,封阿睦爾撒納為親王,訥默庫、班珠爾為郡王;杜爾伯特臺吉剛多爾濟、巴圖博羅特,輝特臺吉札木參、齊木庫爾為貝勒;杜爾伯特臺吉布圖克森、額爾德尼、羅壘雲端,輝特臺吉德濟特、普爾普、克什克為貝子;輝特臺吉根敦札布等,杜爾伯特臺吉布顏特古斯等為公;杜爾伯特臺吉烏巴什等,輝特臺吉伊什等為一等臺吉。以輝特親王阿睦爾撒納為北路參贊大臣,郡王訥默庫為西路參贊大臣。命額琳沁多爾濟為西路參贊大臣,召班第來京。命阿睦爾撒納署將軍,額駙色布騰巴勒珠爾協辦。命車凌同車凌烏巴什往西路軍營,訥默庫同阿睦爾撒納、班珠爾往北路軍營。戊戌,上還京師。

十二月戊申,以班第為定北將軍,阿睦爾撒納為定邊左副將軍,永常為定西將軍,薩喇勒為定邊右副將軍。辛亥,上幸大學士來保、予告大學士福敏第視疾。以親王固倫額駙色布騰巴勒珠爾、親王銜琳沁、郡王訥默庫、班珠爾、郡王銜青滾雜卜、尚書公達勒黨阿、總督伯鄂容安、護軍統領烏勒登為北路參贊大臣,親王額琳沁多爾濟、車凌、郡王車凌烏巴什、貝勒車凌孟克、色布騰、貝子扎拉豐阿、公巴圖孟克、瑪什巴圖、將軍阿蘭泰為西路參贊大臣。癸亥,安南國王黎維禕進方物。賑甘肅河州等十五州縣衛水災。丙寅,調鄂容安為西路參贊大臣,命阿蘭泰、庫克新瑪木特為北路參贊大臣。

二十年春正月丁丑,命定邊左副將軍阿睦爾撒納率參贊大臣額駙色布騰巴勒珠爾、郡王品級青滾雜卜、內大臣瑪木特、奉天將軍阿蘭泰由北路進徵,定邊右副將軍薩喇勒率參贊大臣郡王班珠爾、貝勒品級札拉豐阿、內大臣鄂容安由西路進徵。癸未,以阿里袞署刑部尚書。癸卯,免烏梁海、札哈沁、包沁等貢賦一年。

二月乙巳朔,日食。命兆惠留烏里雅蘇臺協辦軍務,在領隊大臣上行走。丙午,朝鮮貢方物。乙卯,上謁東陵。戊午,上謁昭西陵、孝陵、孝東陵、景陵,至孝賢皇后陵奠酒。己未,召範時綬來京,調胡寶瑔為江西巡撫,以楊錫紱署湖南巡撫,蔣溥署吏部尚書。賑山東惠民等十二州縣衛水災。庚申,準噶爾噶勒雜特部人齊倫來降。丁卯,賑雲南易門、石屏地震災民。己巳,賑江蘇高郵等州縣衛上年災民。

三月丙子,永常等奏額魯特業克明安巴雅爾來降。戊寅,免江蘇江浦等二十二州縣衛十九年水災額賦。己卯,上詣泰陵。召鄂昌來京,調陳宏謀為甘肅巡撫,以臺柱署陜西巡撫。壬午,上謁泰陵。乙酉,上駐蹕吳家莊,閱永定河堤。丙戌,上幸晾鷹臺行圍,殪熊一虎二。召大學士、九卿、翰詹、科道,諭胡中藻詩悖逆,張泰開刊刻、鄂昌唱和諸罪,命嚴鞫定擬。庚寅,上還京師。鄂昌褫職逮問。壬辰,高斌卒。釋張師載回籍。乙未,扎哈沁得木齊巴哈曼集、宰桑敦多克等來降。庚子,免直隸霸州等六州縣本年旱災額賦。壬寅,準噶爾臺吉噶勒藏多爾濟等來降。

夏四月丙午,額林哈畢爾噶宰桑阿巴噶斯等來降。壬子,致仕太保、大學士張廷玉卒,命遵世宗遺詔,配饗太廟。甲寅,胡中藻處斬。乙丑,吐魯番伯克莽噶里克來降。免長蘆永利等三場、海豐一縣水災額賦。丙寅,免山東惠民等十六州縣水災額賦。丁卯,綽羅斯臺吉袞布扎布等並葉爾羌等回部和卓木來降。戊辰,琉球國世子尚穆遣使入貢請封,允之。壬申,集賽宰桑齊巴汗來降。

五月甲戌朔,免安徽壽州等十九州縣衛水災額賦。喀爾喀車臣汗副將軍公格勒巴木丕勒褫爵,留營效力,以扎薩克郡王得木楚克代之。戊寅,賑奉天承德等七州縣水災。庚辰,命翰林院侍講全魁、編修周煌往琉球冊封。辛巳,和通額默根宰桑鄂哲特等來降。壬午,庫圖齊納爾宰桑薩賚來降。甲申,準噶爾宰桑烏魯木來降。戊子,阿勒闥沁鄂拓克宰桑塔爾巴來降。己丑,達瓦齊遁特克斯。庚寅,史貽直原品休致。賜鄂昌自盡。辛卯,命黃廷桂為武英殿大學士,仍留四川總督任。調王安國為吏部尚書,以楊錫紱為禮部尚書,何國宗為左都御史。調陳宏謀為湖南巡撫,以吳達善為甘肅巡撫,圖爾炳阿為河南巡撫。壬辰,阿睦爾撒納奏克定伊犁,賞阿睦爾撒納親王雙俸,封其子為世子。晉封班第、薩喇勒為一等公,瑪木特為三等公。賞色布騰巴勒珠爾親王雙俸。封扎拉豐阿為郡王,車布登扎布、普爾普為貝勒。賞車凌親王雙俸。封車凌烏巴什、班珠爾、訥默庫為親王,策楞孟克為郡王。再授傅恆一等公爵。軍機大臣等俱優敘有差。賑江蘇清河、銅山等州縣水災。癸巳,召達勒黨阿來京協辦大學士,以綽勒多署黑龍江將軍。大學士傅恆辭公爵,允之。封班第為誠勇公,薩喇勒為超勇公,瑪木特為信勇公。

六月癸卯朔,以平定準部告祭太廟,遣官告祭天、地、社、稷、先師孔子。命四衛喇特如喀爾喀例,每部落設盟長及副將軍各一人。丙午,阿睦爾撒納奏兵至格登山,大敗達瓦齊之兵。封喀喇巴圖魯阿玉錫、巴圖濟爾噶勒、察哈什等男爵,並授散秩大臣,餘賞賚有差。己酉,加上皇太后徽號曰崇慶慈宣康惠敦和裕壽崇禧皇太后,頒詔覃恩有差。癸丑,阿克敦免,以鄂彌達為刑部尚書,仍署吏部尚書,阿里袞署兵部尚書,降永常為侍郎。命大學士黃廷桂為陜甘總督,調開泰為四川總督。召劉統勛來京,以碩色署湖廣總督,愛必達署云貴總督。己未,羅卜藏丹津等解送京師,遣官告祭太廟,行獻俘禮。庚申,上御午門受俘,宥羅卜藏丹津罪,巴朗、孟克特穆爾伏誅。甲子,以班第等奏阿睦爾撒納與各頭目往來詭秘,擅殺達瓦齊眾宰桑,圖據伊犁。溫旨令即行入覲。戊辰,獲達瓦齊,準部平。

秋七月戊寅,杜爾伯特臺吉伯什阿噶什等來降。丁亥,烏蘭泰以獲達瓦齊封男爵。黑龍江將軍綽勒多改荊州將軍,以達色代之。

八月丙午,賑江蘇海州等七州縣水災雹災。丁未,上奉皇太后巡幸木蘭。壬子,上奉皇太后駐蹕避暑山莊。甲寅,賑山東金鄉等二十二州縣衛水災。封準噶爾臺吉伯什阿噶什為親王。丁巳,上奉皇太后至木蘭行圍。庚申,召尹繼善來熱河。

九月壬申朔,免福建臺灣等三縣上年被水額賦。甲戌,上御行殿,綽羅斯噶勒臧多爾濟等入覲,賜宴。阿睦爾撒納入覲,至烏隴古,叛,掠額爾齊斯臺站。丙子,準噶爾頭目阿巴噶斯等叛。起永常為內大臣,仍辦定西將軍事,策楞、玉保、扎拉豐阿為參贊大臣。命哈達哈留烏里雅蘇臺,會同阿蘭泰辦事。丁丑,阿睦爾撒納犯伊犁。庚辰,頒招撫阿睦爾撒納諭。壬午,上奉皇太后回駐避暑山莊。癸未,賜噶勒臧多爾濟等冠服,封噶勒臧多爾濟為綽羅斯汗,車凌為杜爾伯特汗,沙克都爾曼濟為和碩特汗,巴雅爾為輝特汗。晉封喀爾喀郡王桑齋多爾濟為親王。命哈達哈等討阿睦爾撒納。丁亥,命策楞為定西將軍。以喀爾喀郡王巴雅爾什第等捕誅包沁叛賊臺拉克等,晉封巴雅爾什第為親王,沙克都爾扎布為貝勒,達爾扎諾爾布扎布為貝子。賑浙江山陰等十五州縣、曹娥等五場、湖州一所,雲南劍川一州本年被水災民。賑湖北江陵等八州縣衛本年被水災民。庚寅,逮永常來京,降策楞為參贊大臣,以扎拉豐阿為定西將軍。劉統勛舍巴里坤退駐哈密,切責之。丙申,逮劉統勛來京,命方觀承往軍營辦理糧餉,以鄂彌達署直隸總督。噶勒臧多爾濟之子諾爾布琳沁討阿巴噶斯,敗之,獲得木齊班咱,加封郡王。封貝勒齊木庫爾為郡王。以阿里袞署刑部尚書,調汪由敦為刑部尚書。戊戌,戶部尚書海望卒。

冬十月辛丑朔,策楞褫職逮問,命副都統莽阿納、喀寧阿為西路領隊大臣。甲辰,以衛哲治為工部尚書,鄂寶署廣西巡撫。戊申,賑浙江會稽等州縣場所水災。命富德為參贊大臣。壬子,宥劉統勛、策楞發軍營,以司員效力。癸丑,賑山東鄒縣等十九州縣衛、官臺等四場水災。丁巳,達瓦齊等解至京,遣官告祭太廟社稷,行獻俘禮。戊午,上御門樓受俘,釋達瓦齊等。賑安徽無為等三十二州縣被水饑民。命李元亮署工部尚書。辛酉,起策楞為參贊大臣,署定西將軍,命進剿阿睦爾撒納。甲子,將軍班第、尚書鄂容安敗績於烏蘭庫圖勒,死之。副將軍薩喇勒被執。丙寅,命哈達哈為定邊左副將軍,雅爾哈善為參贊大臣,達勒黨阿為定邊右副將軍,阿蘭泰為烏里雅蘇臺參贊大臣。

十一月辛未,以杜爾伯特貝勒色布騰為北路參贊大臣。癸酉,以策楞為內大臣兼定西將軍,扎拉豐阿為定邊右副將軍,達勒黨阿為參贊大臣。宥青滾雜卜罪。甲戌,以鄂勒哲依、哈薩克錫喇為參贊大臣,尼瑪為內大臣兼參贊大臣。雲南劍川州地震。壬午,調鄂樂舜為山東巡撫,高晉為安徽巡撫,錫特庫為巴里坤都統。癸未,宥達瓦齊罪,封親王,賜第京師。甲午,噶勒雜特得木齊丹畢來降。

十二月癸卯,起烏勒登為領隊大臣。以盧焯署陜西巡撫。丙午,命侍郎劉綸往浙江查辦前巡撫鄂樂舜,並查江南、浙江賑務。戊申,免伊犁本年貢賦。以吉林將軍傅森為兵部尚書,額勒登代之。己未,賑索倫、達呼爾水災霜災。賑湖北潛江等六州縣衛水災。賑兩淮徐瀆等十二場、山西岢嵐州本年水災各有差。


\end{pinyinscope}