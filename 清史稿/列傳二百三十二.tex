\article{列傳二百三十二}

\begin{pinyinscope}
吳可讀潘敦儼硃一新屠仁守吳兆泰何金壽安維峻

文悌江春霖

吳可讀,字柳堂,甘肅皋蘭人。初以舉人官伏羌訓導。道光三十年,成進士,授刑部主事。晉員外郎,遭憂去,主講蘭山書院。會撒拉番蠢動,被命佐團練。服闋,起故官。遷吏部郎中,轉御史。各國使臣請覲,議禮久未決,可讀請免拜跪,時論韙之。烏魯木齊提督成祿誣民為逆,擊殺多人,虛飾勝狀,為左宗棠所劾。可讀繼陳其罪有可斬者十,不可緩者五,尋逮問,讞上論斬,廷臣請改監候。可讀憤甚,復疏爭:「請斬成祿以謝甘民,再斬臣以謝成祿。」語過戇直,被訶責,鐫三級。歸,復掌教蘭山。逾年,穆宗崩,德宗纘業,起吏部主事。

光緒五年,穆宗奉安惠陵,自請隨赴襄禮。還次薊州,宿廢寺,自縊,未絕,仰藥死,於懷中得遺疏,則請為穆宗立嗣也。其言曰:「罪臣聞治不諱亂,安不忘危。危亂而可諱忘,則進苦口於堯舜,為無疾呻吟,陳隱患於聖明,為不祥舉動。罪臣前因言事獲譴,蒙我先皇帝曲賜矜全,免臣以斬而死,以囚而死,以傳訊觸忌而死。犯三死而未死,不求生而再生,則今日罪臣未盡之餘年,皆我先皇帝數年前所賜也。欽奉兩宮皇太后懿旨,以醇親王之子承繼文宗顯皇帝為子,入承大統為嗣皇帝,俟嗣皇帝生有皇子,即承繼大行皇帝為嗣。我皇上仁孝性成,承我兩宮皇太後授以寶位,將來千秋萬歲時,必能以我兩宮皇太后今日之心為心。而在廷之忠佞不齊,即眾論之異同不一。以宋初宰相趙普之賢,而猶首背杜太后;以明大學士王直之為舊臣,而猶以黃請立景帝太子一疏不出我輩為愧。賢者如此,遑問不肖?舊人如此,奚責新進?名位已定者如此,況在未定。惟有仰求我兩宮皇太后再降諭旨,將來大統,仍歸大行皇帝嗣子,嗣皇帝雖百斯男,中外臣工均不得以異言進。如此,則猶是本朝子以傳子之家法,而我大行皇帝未有子而有子,即我兩宮皇太后未有孫而有孫,異日繩繩揖揖相引於萬代者,皆我兩宮皇太后所自出而不可移易者也。彼時罪臣即欲有言,繼思降調不得越職言事。今逢我大行皇帝奉安山陵,恐積久漸忘,則罪臣昔日所留以有待者,今則迫不及待矣。謹以我先皇帝所賜餘年,為我先皇帝上乞數行懿旨,惟望我兩宮皇太后、我皇上憐其哀鳴,勿以為無疾呻吟、不祥舉動,則罪臣雖死無憾。尤原我兩宮皇太后、我皇上體聖祖、世宗之心,調劑寬猛,養忠厚和平之福,任用老成;毋爭外國之所獨爭,為中華留不盡;毋創祖宗之所未創,為子孫留有餘。罪臣言畢於斯,命畢於斯,謹以大統所系上聞。」吏部奏諸朝,詔憫其忠,予優恤。下群臣議,遂定以繼德宗之統為穆宗之子,無異論。

可讀臨歿遺書與其子之桓,謂出薊州一步即非死所。之桓遂成其遺志,葬薊州。都人即所居城南舊宅祠祀之。

有潘敦儼者,字清畏,籍江寧,總督鐸子。以任子官工部郎中,遷御史。默念穆宗嗣統未有定議,孝哲毅皇后又仰藥殉,遂疏請表揚穆後潛德,更謚號,並解醇親王奕枻職任,詔嚴斥奪職。歸隱於酒,閱二十餘年,卒。

硃一新,字蓉生,浙江義烏人。鄉舉對策語觸時忌,主司李文田特拔之。入貲為內閣中書。光緒二年,成進士,選庶吉士,授編修。法越事起,數上書主戰,又嘗畫海防策,語至切要。典湖北鄉試,稱得士。十一年,轉御史,連上封事,言論侃侃,不避貴戚。

內侍李蓮英漸著聲勢。逾歲,醇親王奕枻閱海軍,蓮英從,一新憂之。而適值山東患河,燕、晉、蜀、閩患水,遂以遇災修省為言,略曰:「我朝家法,嚴馭宦寺。世祖宮中立鐵牌,更億萬年,昭為法守。聖母垂簾,安得海假採辦出京,立寘重典。皇上登極,張得喜等情罪尤重,謫配為奴。是以綱紀肅然,罔敢恣肆。乃今夏巡閱海軍之役,太監李蓮英隨至天津,道路譁傳,士庶駴愕,意深宮或別有不得已苦衷,匪外廷所能喻。然宗籓至戚,閱軍大典,而令刑餘之輩乎其間,其將何以詰戎兵崇體制?況作法於涼,其弊猶貪。唐之監軍,豈其本意,積漸者然也。聖朝法制修明,萬無慮此。而涓涓弗塞,流弊難言,杜漸防微,亦宜垂意。從古閹宦,巧於逢迎而昧於大義,引援黨類,播弄語言,使宮闈之內,疑貳漸生,而彼得售其小忠小信之為,以陰竊夫作福作威之柄。我皇太后、皇上明目達聰,豈有跬步之地而或敢售其欺?顧事每忽於細微,情易溺於近習,侍御僕從,罔非正人,辨之宜早辨也。」疏上,太后怒,詰責疏言「苦衷」何指?一新曰:「臣所謂『不得已苦衷』者,意以親籓遠涉,內侍隨行,藉以示體恤、昭慎重也。顧在朝廷為曲體,在臣庶則為創見。風聞北洋大臣以座船迎醇親王,王弗受,而太監隨乘之,至駴人觀聽。一不謹慎,流弊遂已至斯,臣所為不能已於言也。」詔切責,降主事。乞終養歸。

張之洞督粵,建廣雅書院,延為主講。一新博極群書,洞知兩漢及宋、明諸儒家法,務通經以致用。諸生有聰穎尚新奇者,必導而返諸篤實正大,語具所箸無邪堂答問中。卒,年四十有九。

屠仁守,字梅君,湖北孝感人。同治十三年進士,選庶吉士,授編修。光緒中,轉御史。時政出多門,仁守因天變請修政治,條上六事,曰:杜諉卸,開壅蔽,慎動作,抑近習,軫民瘼,重國計,而歸本於大公至正、敬天勤民,疏上不省。又以海軍報效,雜進無次,僥幸日多。仁守痛陳五弊:資敘不計,弊一;名器冒濫,弊二;勸懲倒置,弊三;求益得損,財計轉虧,弊四;駔儈朋侵,莫可究詰,弊五。五弊既滋,乃生三患:患病民,患妨賢,患隳紀綱法度。「特以自海軍衙門達之,奉懿旨行之,毋或敢貿然入告,遂使謗騰衢路,而朝廷不聞,患伏隱微,而朝廷不知,群小得志,寵賂滋張。若不停止,即承平無事,猶或召亂,況時局孔艱乎?」疏入,詔從之,權貴益側目。

十五年,太后歸政,仁守慮僉人讒構兩宮,易生嫌隙,疏請依高宗訓政往事:「凡部院題本、尋常奏事,如常例;外省密摺、廷臣封奏,仍書皇太后、皇上聖鑒,俟慈覽後施行。」並請太后居慈寧宮,節游觀。詔嚴責,革職永不敘用。既歸,主講山西令德堂。二十六年,兩宮西狩,起用五品京堂,授光祿寺少卿。尋卒。

吳兆泰,字星階,籍麻城。與仁守友善,互相厲以道義。光緒二年進士,閱十年,以編修考授御史。時國防廢弛,海軍尤不振,朝廷乃移其費修頤和園。兆泰上疏力爭,略謂:「畿輔奇災,嗷鴻遍野,殭僕載塗,此正朝廷減膳徹樂之時,非土木興作之日。乞罷園工,以慰民望,以光繼列祖列宗儉德。」太后怒,罷其官。歸裏後,歷主龍泉、經心書院講席,充學務公所議長。宣統二年,卒。

其先有何金壽者,字鐵生,籍江夏。同治元年一甲二名進士,授編修。出督河南學政,還充日講起居注官。光緒二年,晉饑,上儲糧平糶策。越二年,畿輔旱,金壽曰:「此樞臣可盡彈也!」乃援漢代天災策免三公為言,請罷樞臣、回天意。越日,命下,恭親王奕等五人並褫職留任,直聲震一時。五年,復瀝陳時弊,斥言中外臣工皆瞻徇,侃侃不撓。上以所奏為祛積習,特宣示。忤當軸意,出知江蘇揚州府。未出都,會崇厚與俄定約,敕下廷臣議。金壽引西國上下議院例,請資眾論,折強敵。逾歲到官,錄築堤功,賜三品服。八年秋,禱雨中暍,病卒,貧不能歸葬。總督左宗棠等上其事於朝,謂有古循吏風雲。

安維峻,字曉峰,甘肅秦安人。初以拔貢朝考,用七品小京官。光緒六年,成進士,改庶吉士,授編修。十九年,轉御史。未一年,先後上六十餘疏。日韓釁起,時上雖親政,遇事必請太后意旨,和戰不能獨決,及戰屢敗,世皆歸咎李鴻章主款。於是維峻上言:「李鴻章平日挾外洋以自重,固不欲戰,有言戰者,動遭呵斥。淮軍將領望風希旨,未見賊先退避,偶見賊即驚潰。我不能激勵將士,決計一戰,乃俯首聽命於賊。然則此舉非議和也,直納款耳,不但誤國,而且賣國。中外臣民,無不切齒痛恨。而又謂和議出自皇太后,太監李蓮英實左右之,臣未敢深信。何者?皇太后既歸政,若仍遇事牽制,將何以上對祖宗,下對天下臣民?至李蓮英是何人斯,敢干政事乎?如果屬實,律以祖宗法制,豈復可容?唯是朝廷受李鴻章哃喝,不及詳審,而樞臣中或系私黨,甘心左袒,或恐決裂,姑事調停。李鴻章事事挾制朝廷,抗違諭旨。唯冀皇上赫然震怒,明正其罪,布告天下,如是而將士有不奮興、賊人有不破滅者,即請斬臣以正妄言之罪。」疏入,上諭:「軍國要事,仰承懿訓遵行,天下共諒。乃安維峻封奏,託諸傳聞,竟有『皇太後遇事牽制』之語,妄言無忌,恐開離間之端。」命革職發軍臺。維峻以言獲罪,直聲震中外,人多榮之。訪問者萃於門,餞送者塞於道,或贈以言,或資以贐,車馬飲食,眾皆為供應。抵戍所,都統以下皆敬以客禮,聘主講掄才書院。二十五年,釋還,遂歸里。三十四年,起授內閣侍讀,充京師大學總教習。宣統三年,復辭歸。越十有五年,卒。

維峻崇樸實,尚踐履,不喜為博辨,尤嚴義利之分。歸後退隱柏崖,杜門著書,隱然以名教綱常為己任。每談及世變,輒憂形於色,卒抑鬱以終。著有四書講義、詩文集。

文悌,字仲恭,瓜爾佳氏,滿洲正黃旗人。以筆帖式歷戶部郎中,出為河南知府,改御史。光緒二十四年,變法詔下,禮部主事王照應詔上言,尚書許應癸不為代奏。御史宋伯魯、楊深秀聯名劾以守舊迂謬,阻撓新政,諭應騤明白回奏,覆奏稱珍惜名器,物色通才,並辭連工部主事康有為,請罷斥驅逐。奏上,以抑格言路,首違詔旨,禮部尚書、侍郎皆革職,賞照四品京堂。

文悌以言官為人指使,黨庇報復,紊亂臺諫,遂上疏言:「康有為向不相識,忽踵門求謁,送以所著書籍,閱其著作,以變法為宗。而尤堪駭詫者,託辭孔子改制,謂孔子作春秋西狩獲麟為受命之符,以春秋變周為孔子當一代王者。明似推崇孔子,實則自申其改制之義。乃知康有為之學術,正如漢書嚴助所謂以春秋為蘇秦縱橫者耳。及聆其談治術,則專主西學,以師法日本為良策。如近來時務、知新等報所論,尊俠力,伸民權,興黨會,改制度,甚則欲去拜跪之禮儀,廢滿、漢之文字,平君臣之尊卑,改男女之外內。直似只須中國一變而為外洋政教風俗,即可立致富強,而不知其勢小則群起鬥爭,立可召亂;大則各便私利,賣國何難?曾以此言戒勸康有為,乃不思省改,且更私聚數百人,在輦轂之下,立為保國會,日執途人而號之曰:『中國必亡,必亡!』以致士夫惶駭,庶眾搖惑。設使四民解體,大盜生心,藉此以集聚匪徒,招誘黨羽,因而犯上作亂,未知康有為又何以善其後?曾令其將忠君愛國合為一事,勿徒欲保中國而置我大清於度外,康有為亦似悔之。又曾手書御史名單一紙,欲臣倡首鼓動眾人伏闕痛哭,力請變法。當告以言官結黨為國朝大禁,此事萬不可為。以康有為一人在京城任意妄為,遍結言官,把持國事,已足駭人聽聞;而宋伯魯、楊深秀身為臺諫,公然聯名庇黨,誣參朝廷大臣,此風何可長也!伏思國家變法,原為整頓國事,非欲敗壞國事。譬如屋宇年久失修,自應招工依法改造,若任三五喜事之徒曳之傾倒,而曰非此不能從速,恐梁棟毀折,且將傷人。康有為之變法,何以異是?此所以不敢已於言也。」疏上,斥回原衙門行走。

太后復訓政,賞文悌知府,旋授河南知府。二十六年,兩宮西狩,文悌迎駕,擢貴西道。乞病歸,卒。

江春霖,字杏村,福建莆田人。光緒二十年進士,選庶吉士,授檢討。二十九年,轉御史,首論都御史陸寶忠幹煙禁,不宜為臺長,劾親貴及樞臣疆臣,章凡數十上。德宗季葉,袁世凱出督畿輔,入贊樞廷,權勢傾一時。春霖獨論列十二事,謂:「洪範有言:『臣之有作威作福,其害於爾家,兇於爾國。』左氏傳云:『受君之祿,是以聚黨,有黨而爭命,罪孰大焉?』今世凱所為,其心即使無他,其跡要難共諒。歷考史冊所載權臣,大者貽憂君國,小者禍及身家。窺竊神器之徒,姑置勿論,即功在社稷,如霍光、李德裕、張居正,亦以權寵太盛,傾覆相尋。今不獨為國家計,宜加裁抑,即欲使世凱子孫長守富貴,亦不可無善處之法。」嗣是糾彈世凱及慶親王奕劻父子,連上八疏,皆不報,然朝貴頗嚴憚之。

宣統改元,醇親王載灃既攝政,其弟載洵、載濤分長軍諮、海軍,頗用事。春霖謂:「古者鄭寵共叔,失教旋譏,漢驕厲王,不容終病,載在史冊,為萬世戒。二王性成英敏,休戚相關,料不至蹈覆轍,而慎終於始,要宜杜漸防微。」又謂:「景皇帝以神器付之皇上,沖齡踐阼,軍國重事,監國攝政王主之。治同其樂,亂同其憂,國之不保,家於何寄?」篇末又言:「監國歲未及周,物議沸騰,至於此極。臣不禁為祖宗三百年國祚效賈生痛哭流涕長太息矣!」明年,又劾江西巡撫馮汝騤謾欺狀,效宋臣包拯七上彈章,末復言:「是非不明,請將前後章奏明詔宣示,敕部平議。」語至戇直,被訶責。復劾奕劻老奸竊位,多引匪人;非特簡忠良,不足以贊大猷、挽危局。詞連尚書徐世昌,侍郎楊士琦、沈云沛,總督陳夔龍、張人駿,巡撫寶棻、恩壽等十數人。朝旨再責之,令回原衙門行走。春霖遂稱疾歸。越八年,卒。

論曰:有清列帝,家法最嚴,迨至季世,創制垂簾,於是閹寺漸肆,而親貴權要亦聲勢日著,雖有直言敢諫之士,無補危亡,亦盡其心焉而已。可讀尸諫,幸鑒孤忠。一新、仁守、維峻先後直言,皆以語侵太后獲罪。文悌言攻結黨,實啟黨爭,而春霖連劾權貴,言尤痛切,當國者終於不悟。又有太監寇連才,上書泣諫,請太后歸政,廢頤和園,且言:「不為祖宗天下計,獨不自為計?」終以違制被刑以死。建言又何得以閹官少之?類無可歸,故附見於此。


\end{pinyinscope}