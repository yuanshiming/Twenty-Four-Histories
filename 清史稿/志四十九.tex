\article{志四十九}

\begin{pinyinscope}
地理二十一

△雲南

雲南:禹貢梁州徼外地。清初沿明制,置承宣布政使司,為雲南省,設巡撫,治雲南府,並設雲貴總督,兩省互駐。康熙元年,改雲南總督,駐曲靖。三年,裁貴州總督並云南,駐貴陽。二十二年,移駐雲南。雍正五年,定雲貴總督兼轄廣西。十二年,停兼轄廣西。乾隆元年,設雲南總督。十二年,改雲貴總督。光緒中,裁巡撫。領府二十,直隸州一。康熙五年,降北勝直隸州為州,隸大理。八年,降尋甸府為州,隸曲靖。三十七年,升北勝州為永北府,省永寧。雍正三年,改威遠土州為直隸。四年,割四川之東川府來隸。五年,以四川烏蒙、鎮雄二府來隸。六年,降鎮雄為州,屬烏蒙。東川、鎮雄,元屬雲南,明屬四川。烏蒙,元屬四川,明初屬雲南,後改屬四川。七年置普洱、八年置開化二府。九年,改烏蒙為昭通府。乾隆三十一年,永北降直隸。三十五年,廣西、武定、元江、鎮沅四府降直隸州,景東、蒙化二府皆降直隸,省姚安屬楚雄,改鶴慶府為州,屬麗江。嘉慶二十四年,升騰越州為直隸。道光二年,改分防同知,又改鎮沅直隸州為直隸。光緒十三年,置鎮邊撫夷直隸。二十四年,升鎮雄州為直隸。東至廣西泗城;七百五十里。南至交阯界;七百五十里。北至四川會理;四百里。西至天馬關,接緬甸界。二千三百一十里。西南:英領緬甸。光緒中,曾紀澤謀與英勘界,索八募,復我太平江以南漢龍、天馬、虎踞、鐵壁四關侵地,議未決。薛福成繼之,力持前畫。騰越西以伊洛瓦諦江源流為界,江東野人山地概歸中國,尚可由大盈江之新街入伊洛瓦諦,經阿瓦至仰光海口行輪,又索還故壤二千餘里。及中東事起,俄、法、德居間,後贈法以紅江甌脫及孟俄地,英藉口改薛約,割科干,復許以滇緬鐵路,而邊事日棘,片馬不守矣。廣二千五百一十里,袤一千一百五十里。北極高二十九度三十分至二十一度四十分。京師偏西十度二十九分至十九度十分。宣統三年,編戶一百五十四萬八千一十四,口六百四十萬三千九百三。共領府十四,直隸六,直隸州三,十二,州二十六,縣四十一;又土府一,土州三,土司十八。驛路:東達貴州普安,東南達廣西百色,西達緬甸八募,西南達緬甸景東。鐵路:滇越。電線:東北通重慶,西通八募,東南通南寧。

雲南府:沖,繁,疲,難。云武分巡、糧儲道治所。總督、巡撫,布政、提學、提法三使,鹽法、巡警、勸業各道駐。東北距京師八千二百里,廣三百七十里,袤二百九十八里。北極高二十五度六分。京師偏西十三度三十七分。領州四,縣七。昆明沖,繁,疲,難。倚。城內:五華山、螺山。山有潮音洞,山側有翠湖。東:鸚鵡山。西:太華、聚仙。南:萬德。北:商山。東北:龍泉山。西南:碧雞山。盤龍江自嵩明入,西南流,逕城東,合銀棱河,至縣南,匯為滇池。滇池一名昆明池,長百二十餘里,縣東諸水入焉,下流折入昆陽州曰海口,即螳螂川上游。金棱河自城東北松華壩分盤龍口水入滇池。寶象河自嵩明入,西南流,逕城南,亦入滇池。西:碧雞關、高嶢關。東:金馬關。同知一,駐南關。驛二:板橋、滇陽。汛二:昆明、板橋。富民簡。府西北七十里。東:天馬山。西:臥雲、玉屏。南:靈芝。北:法華山。螳螂川自安寧入,納城西北農納水,入武定州祿勸,為普渡河。大營河出昆明西北山,西流入境,洞溪水來會,西至城北入螳螂川,清水河從之。宜良沖,繁。府東百二十里。北:萬壽。南:雉山。東:客爭容山,縣鎮山也。西:石燕。東南:駱駝山。西南:鳳凰山。西:大城江,自河陽之楊宗海流入,逕城西北,折東南,分二支,同入大池江。大池江即八達河,為南盤江上流。西北有湯池。嵩明州沖,難。府東北百三十里。城內:黃龍山。東:馬頭。西:靈雲、登花。西北:東葛勒山,元梁王結寨地。南:鳳谿、石華山。龍巨河一曰龍濟溪,自尋甸入,南流匯為嘉利澤,一名楊林海,逕城東南,納楊梅河、對龍河諸水,匯為澤,周百餘里。東南出河口,折北流入尋甸,為車洪河、寬郎河、邵甸河,合九十九泉,西南流,會牧養河。又西南,入昆明,為盤龍江,即滇池上源也。西南:兔兒關。驛一:楊林。晉寧州繁,難。府南九十里。城內:螺髻山。東:梅溪、五龍。西:石美山,與百花山相望。南:石壁。東南:玉案。西南:石魚山。西有天女城。滇池在州西北,大堡河自新興來會,又西北逕城西,分數道流入滇池。盤龍河源出五龍山,分二派,一西北流入大壩河,一東北流分為二,一入澂江撫仙湖,一入昆池。呈貢沖,繁。府南四十里。明與歸化同隸晉寧。康熙八年省歸化入焉。北:三臺山。東:軍營。南:龍翔。東南:象兔、羅藏山。滇池在縣西南,東撈魚河、南淤泥河、東南梁王河皆匯焉。洛龍河源出城東白龍潭,西流,會黑龍潭,貫城注滇池,南沖河偕清水河從之。南:太平關,臨、澂孔道。安寧州沖,繁。府西七十里。康熙六年省三泊入昆陽。雍正三年又改其地來隸。城內:太極、白華。西北:蔥山。東:印山、龍馬。西:羅青。南:天馬山。螳螂川一名安寧河,自昆陽入,北流入富民。鳴蟻河源出龍洞,北流,望洋河,又北資利河,同來會。折東北,至州東南,入螳螂川。有煎鹽水,出岈峻山。有大井、石井、河中、大界、連然等鹽井。驛二:祿表、安寧。羅次簡。府西北百三十里。西:金鳳。北:百花。南:崛嵕山、九戌山,易江出焉。東北有苴麼崱裒山,綿亙縣西,兩峰相望。易江北流入祿豐。金水河東北流,納青龍山南北二溪水,又折西北,匯碧城河水、東渠河水,折西亦入祿豐,名星宿江。北:煉象關。祿豐簡。府西北二百十里。西:三次和山,舊名蒙荅縛山。北:象頭、馬頭山。東:姚陵山。星宿江自羅次入,納南河、九渡河諸水,南入易門。易江亦自羅次入縣東境,繞安寧,復折入境,東南入易門。東:老鴉關。驛一:祿豐。昆陽州疲,難。府南百二十里。東:龍泉山。西:月山、珊蒙果山。南:金龜山。北:望州。東南:御屏。西北:怱蒙山。滇池在城東北隅。螳螂川自滇池分三支,西北入安寧。渠濫川逕城東南入滇池。南:鐵爐關。易門簡。府西南二百五十里。城內:龜山。東:屏山、左右旗山、鼓山。西:象山。東南;虎頭山。星宿江自祿豐入,南流,納太和川水,又南匯大小綠汁河,入丁癸江。南流,易江亦自祿豐入,南流匯上下渠江水,廟兒山水自東來合焉。折西,納獅山水、速末水,合星宿江為丁癸江,南流入習峨。

武定直隸州:隸云武分巡、糧儲道。明,武定府。領州二,縣一。乾隆三十五年,降為直隸州,裁府治和曲,降祿勸州為縣。東南距省治二百四十里。廣三百六十里,袤三百三十九里。北極高二十五度三十二分三十秒。京師偏西十三度五十七分,領縣二。北:甲甸背。西北:猗朵。西:獅子山。北:金沙江自元謀入,左有會川,衛水自四川會理合東安河南流來注。又東流,合大環川,入祿勸。盤龍河源出羅次白花山,為鳩水河,東北流,至城東,左會鷂鷹河,為盤龍河,東入祿勸。南:小營關。東南:小甸關。西北:油榨關、龍街關。明只舊、草起二鹽井,今廢。巡司一,駐金沙江岸。元謀難。州西北九十里。東:定見山。西:翠峰。南:馬頭。北:溫泉、蓮花山。北:金沙江自大姚入,合西溪河,即龍川江下流,自廣通北流入境。又北會南號河、黑占乾河、元馬河、羅又乾河、午茂乾河、爐頭河,自大姚東流,合為苴寧河,又北入金沙江。元馬塚,州北二十里。華陽國志謂縣有元馬,日行千里,元馬河以此得名。土人呼馬為「謀」,縣以此氏焉。東:望城關。祿勸難。州東北二十里。明,州,乾隆二十五年降。東北:烏蒙山,一名雪山。北:法塊、幸丘。東南:普照山。北:金沙江自州入,東流,勒洟溪、東洟溪自幸丘山合北流入焉,又東納普渡河水,烏龍河自烏蒙山北流注之,又東入東川巧家。普渡河即螳螂川下流,自富民北流入境,納掌鳩河水,北逕雪山入金沙江。西南:雄關。

大理府:沖,繁。迤西道治所。提督駐。順治初,因明制。康熙五年,降北勝直隸州為州來屬。三十一年,仍直隸。東南距省治八百九十里。廣九百六十里,袤二百二十里。北極高二十五度四十四分。京師偏西十六度十一分。領州四,縣三,長官司一。太和沖,繁。倚。西:點蒼山,高六十里,山椒懸瀑,注為十八溪,綿亙百餘里,府之鎮山也,西拱縣城如抱弓然。西洱河,亦名洱海,形如月抱珥,亦曰珥河。縣東五里,即古葉榆澤,源出浪穹北,境內諸水入焉。長百三十里,闊三十九里,下流會樣備江,逕趙州入蒙化。北:上關,亦曰龍首關,又曰石門關。南:下關,亦曰龍尾關。諺曰:「蒼山雪,洱海月,上關花,下關風。」下關貿易極盛,南陬名鎮。趙州沖,繁。府南六十里。東:九龍山,與州西鳳儀山對峙。西:三臺山。南:昆彌。東南:龍母。西南:華藏山。大江一名波羅江,有二源,合流而北,逕城南,折東會玉閬泉、烏龍、雙塔諸水,北入洱海。白崖江即禮社江,上流自雲南縣入,流經白崖,有鼻窗廠水及赤水江來會,入蒙化。東南:蒿菁關、松花關。南:彌渡市鎮,通判駐。驛二:西嶺、德勝。雲南沖,難。府東南百三十里。明屬趙州。順治初,改隸府。東:飛鳳。西:金龍。南:青華山。北:梁王山,禮社江與一泡江同源於此。一支南流至團山壩,分為三。其一南流為溪溝,逕青華山南,入趙州,為禮社江東源。其二東流,一逕縣南,匯為青龍海,一逕縣北,匯為品甸王海,仍歸青龍海,海周四十餘里,灌田利溥。一支北出為周官此夕海,合流而南,至雲南縣,折而東北,納你甸河諸水,為一泡江,入姚州。東北:楚場關。東南:安南關。土主簿駐白巖川。鄧川州疲。府北九十里。東:玉案、雞足。西:象山、彌勒山。東南:鼎勝。南:伏虎。北:天馬山。西北:覆鐘山。羅時江源出鐘山下綠玉池,亦曰西湖,南流逕象山下,又東南至上關。悶地江源出州東北焦石洞,亦曰東湖,南流逕城東,又南至上關,彌苴佉江自浪穹來注之,入洱海。高澗河源出雞足山,北流,羅陋河自鶴慶來會,合為枯木河,入賓川。東:大把關。驛一:鄧川。浪穹簡。府北百十里。明屬鄧川州。順治初,改屬府。西:鐵甲場山,悶江所出。西南:鳳羽山。黑惠江自劍川入,亦曰白石江,南流逕城西,納諸山溪水,入太和為樣備江。大營河源出劍川,南流,鳳羽河源出清源洞,北流,並會寧河。寧河源出罷谷山,匯為茈碧湖,南流,逕城東北,南會大營河,折西,納悶江、鳳羽河二水,曰三江口。又南,逕城東蒲陀崆,為彌苴佉江,歷鄧川入太和,即洱海上源也。西:羅坪關。西北:大樹關。東北:一女關。有蒲陀崆、鳳羽鄉、上江嘴、下江嘴巡司四。賓川州難。府東百二十里。西:雞足山。東:鍾英山。北:華蓋、摩尼。東北:赤石巖山。西北:翠屏山。東北:金沙江自鶴慶入,東流,納答旦河、一泡江諸水,入姚州。答旦河一曰六溪河,其源有六,曰鍾良溪、銀溪、石寶溪、寒玉溪、通洱溪、赤龍溪,並自城西東流,又北逕城西北,豐樂溪自盒子孔山來會,亦曰七溪,北流入金沙江。西南:畢羅關。雲龍州繁,難。府西五百里。東:大羅山,明設大羅衛,今廢。西:三崇山。北:清水朗。東北:大雒馬山,與西小雒馬夾河相望。西:瀾滄江自麗江入,納沘江、表村河、松牧溪諸水,南入永昌。怒江、俅江自俅夷境入,逕三崇山,南入永昌。北:太平關。東:新關。有大井鹽課大使。鹽井六:順蕩、諾鄧、石門、天耳、山井、師井。十二關長官司府東三百里。本云南縣楚場地。元置十二關防送千戶所。明置長官司,隸大理,徙一泡江之西。清因之。土官李姓,世襲。東:白沙坡。西:觀音箐。

麗江府:要。隸迤西道。明為軍民府,領州四,縣一。順治十六年,改土府,省所屬州縣並入。雍正元年設流官。乾隆二十一年,置中甸。三十五年,置麗江縣為府治,改鶴慶府為州,並所屬劍川州、維西來隸。東南距省治一千二百四十里。廣六百七十里,袤九百五十九里。北極高二十六度五十二分。京師偏西十六度二分。領二,州二,縣一。麗江疲,難。倚。明,通安州。乾隆三十六年改今名。西南:老君山,南幹諸山之祖。西北:雪山,一名玉龍。西:花馬:漢藪山,高百仞,上有三湖。西:怒江即潞江,源出西藏布喀池,自夷境入,南流入雲龍。瀾滄江自維西入,分二,正支西納白水,南流入雲龍,分支為漾備江,東流納老君山下分江諸水,入劍川。金沙江即麗水,亦自維西入;納漢藪山橋頭、巨甸諸水,入鶴慶。東:雪山門關。西:石門關。有麗江井鹽課大使。鶴慶州繁,難。府東南三百五十五里。明,軍民府,領劍川、順州。康熙中,順州省入。乾隆三十六年降州來隸。西南:方丈山,為南詔十七名山之一。南:半子。北:湯乾。東北:三臺山。東:金沙江自麗江入,東南流,合漾共江,一名鶴川,亦自麗江入,納境內諸水,瀦為湖,伏流石穴中三里,南出為腰江,折東流注金沙江。西南:觀音山河,南流入大理浪穹。南:宣化關。北:印塘關。西南:觀音山關,鶴麗鎮總兵駐。劍川州沖。府南九十里。明屬鶴慶,今改隸。東:青崖山。南:夜合。西:石鍾山。西北:老君山,與麗江分界。白石江自麗江合分江水緣界入,合磨刀去石河。又東南,大橋頭河亦曰黑惠江,出西北老君山,東南會千木河、螳螂河,至城南為劍湖,廣六十里,合桃羌河諸水,西南出為劍川,曲流三折入浪穹。南:大理國望德故城。鹽井二:彌沙、橋後。中甸要。府北二百三十里。明,麗江府地。康熙時,吳三桂以其地畀達賴喇嘛。雍正五年,來隸鶴慶府,移劍川州州判駐之。乾隆十一年設治,隸府。東南:雪山,與麗江雪山接,兩崖壁立,金沙江貫其中,流逕城東南,與維西以江為界,左合碩多岡河,入麗江。多克楚河、里楚河,並自四川里塘入,為無量河,入永北。維西簡。府西北七十里。明末拓元臨西西北吐蕃地為土府。雍正五年設治,隸鶴慶府,通判駐之。乾隆十一年隨鶴慶來隸。雪山東金沙江自四川巴塘入,總文河自巴塘東來注之,折東南,納所楚河水入麗江。瀾滄江亦自巴塘納徐那山水,又南流,永青河水自城東北來注之,入麗江。

楚雄府:沖。隸迤西道。明領州二,縣五。康熙八年,省咢嘉入南安。雍正七年,省定邊改隸蒙化府。乾隆三十五年,裁姚安府,以所轄姚州及大姚縣來隸。東距省治四百二十里。廣三百七十五里,袤五百八十里。北極高二十五度四分。京師偏西十四度四十五分。領州三,縣四。楚雄沖,繁。倚。城內:雁塔山,即古金礦山。西:峨𡸮山。西南:九臺、碧藏山。龍川江自鎮南入,納大石河、青龍河諸水,折東北,合方家河,緣定遠界入廣通。東:平山關。南:雪裏關。呂合一驛。土縣丞駐縣西南。廣通沖。府東七十里。東:高登山。西:鳳山。南:臥象山,與伏獅山對峙。東北:阿陋雄山,有阿陋井、猴井,俱產鹽。龍川江自定遠入,東北流,納立龍、清風、羅申諸水,北流入元謀,注金沙江。立龍河自北,清風河自東,並入龍川江。有阿陋井鹽課大使。回磴關土巡司。驛二:路田、翀資。定遠簡。府北百二十里。東:寶華。西:烏龍、雲龍山。東北:諸葛鼇峰、寶應山,俱在舊瑯鹽井司境。絕頂峰在黑鹽井司境。龍川江自楚雄入,納瑯溪、零川、龍溝河、紫甸河諸水,入廣通。縣境產鹽,舊設瑯鹽井提舉司,後裁。黑鹽井提舉司駐寶泉鄉。土主簿駐縣西。驛一:新田。南安州難。府東南五十里。康熙八年省咢嘉縣入。雍正九年設州判駐焉。西南:表羅山。東:健林蒼山。南:茶山。青龍河源出州北,入楚雄。馬龍河源出鎮南,南流,大廠河東南流,二水相合為禮社江。妥稍關、鵝毛關、會稽關,俱在州南。鎮南州沖,疲。府西北七十里。東:石鼓、五樓。南:石吠。西:苴力鋪山,白龍河出其下,納清水河、平夷川諸水,與龍川江合流入楚雄。西:白崖江,自姚州緣界入,入南安。北:十八盤山,連廠河出,入姚州。其東紫甸河,入定遠。東南:阿雄關,土巡司駐。西:鎮南關、鸚鵡關,土州同駐。永寧鄉,土州判駐。驛一:沙橋。姚州繁。府西北二百一十里。明,姚安府。乾隆三十五年裁府,以附郭之姚州改隸。東:白馬山、燕子山。西:赤石、龜祥。東北:妙峰。西北:回龍、象嶺山。一字水源出黎武山,北流,逕白鹽境,又西北入一泡江。香水河出黎武南麓,西南流,入大姚。蜻蛉河出三窩山,西北流,瀦為大石硼,北流,納回龍廠河,折東入大姚。陽派河源出金秀山,北流匯為陽片湖,又北流,會連場河,同入蜻蛉河。北:白鹽井有提舉司。驛一:普淜,州判駐。土州同駐州西南六十里。大姚簡。府西北三百二里。南:幾山。北:方山、龍山。西北:玉屏山。羊氾江源出城北麼(此夕)村,東北流入金沙江。香水河自姚州入,南流入大姚河。大姚河源出鎮南北十八盤山,納蛟龍江、苴郤河、紫丘、濫泥箐諸水,入金沙江。白馬河、臥馬刺河、矣資河從之。東:黎石關。西:龍門關。有苴郤巡司。

永昌府:要。隸迤西道。明為軍民府。領州一,縣二,土府一,土州二。順治十六年,鳳溪、施甸二長司省入。乾隆三十年,削「軍民」字。三十五年,置龍陵。嘉慶中,騰越升直隸。道光二年降。東距省治一千三百四十五里。廣六百九十里,袤一千一百二十里。北極高二十五度六分。京師偏西十七度四分。領二,縣二,土府一,土州二,宣撫司五,安撫司三,長官司二。保山繁,難。倚。城內:太保山,縣以此名。東:哀牢山。西:九隆。南:法寶。西北:怒江,自雲龍入,納西溪、雪山、蒲縹、坪市、八灣諸水,東南入潞江。東北:瀾滄江,自雲龍來,與永平分水,納羅岷北山水、沙木河水,東南入順寧。南甸河,上流為清水河,有二源,合流而南,郎義河自城北來會,至城東,匯為青華海。折東南,穿峽口洞出,為枯柯河,南入灣甸土州。南:蒲關、水眼關。北:甸頭關。東南:老姚關。東北:山塔關。西北:馬面關。施甸、沙木和巡司二。永平簡。府東北百七十里。東:天馬、羅武。西:和丘。北:羅木。西南:博南山、花橋山。銀龍江出東北阿荒山,南流至城東南,納羅木場、曲洞河、花橋河諸水,入順寧,入瀾滄江。東:勝備江,源出羅武山,東南納九渡、雙橋諸水,至蒙化入碧溪江。西南:花橋關。東北:上甸關。龍陵要。府西南二百九十里。明,猛弄司。乾隆三十五年置同知,徙駐。東:怒江,自潞江土司東南流入境,納野豬河、施甸河、邦買、回環諸水,南流折西,歷孟定土府入緬甸。龍川江緣西界,納香柏河、芒市河,西南流,合南歌郎水,逕遮放南入瑞麗江。東:象達關。南:遮放副宣撫司,本隴川宣撫司地,明萬歷十二年以多恭為副宣撫使,管遮放。今因之。騰越要。迤西道駐。府西三百六十里。騰越鎮總兵駐。明屬永昌府。嘉慶二十五年升直隸。道光二年降。光緒間,開埠通商。東:高黎貢山,一名昆侖崗,山頂有泉,東入保山,西入騰越,又名分水嶺。北:明光。西:雅烏猛弄。西北:姊妹山。龍川江源出西藏桑楚河,亦曰麓川江,至城東,納曲石江水,折而西,至天馬關入緬甸。大盈江亦曰大車江,源出赤土山,曰馬邑河,西流至城東北,納馬場河、黃坡、緬箐、橋頭、曩拱諸水,南與檳榔江會,有盞達河北流來注,西南逕銅壁關東、鐵壁關北,入蠻募土司,入大金沙江。西:檳榔江,東南流,入乾崖土司,會大盈江。東:龍川江關。南:鎮夷關。西:滇灘關。西北:神護關。孟定土府府東南八百七十里。明,土府。順治初因之。土官罕氏世襲,隸府。北:無量山,跨鎮康、耿馬兩土司界。南丁河,自緬寧入,納無量山水,西南流,納南卡、南路、南們、南底、南滾諸水,西逕府北,折南入阿瓦。怒江自龍陵入,俗名喳哩江,逕府北入緬甸。為府境之險要。灣甸土州府東南二百二十里。土官景姓世襲,隸府。西北:高黎貢山。東:孟通山。枯柯河自保山入,南流,姚關水來會,又南至城西北,會鎮康河。鎮康河自鎮康入,左納響水河,右納杜偉山水,北與枯柯河會,合為南甸河。折西,流入龍陵,注怒江。有黑泉,毒不可涉。北:姚關。鎮康土州府南三百八十里。古石睒黑僰所。土官刁姓世襲,隸府。東南:烏木龍山。西:無量山,即蒙樂山。鎮康河有二源,一出烏木龍山北麓,西北流,一出無量山北麓,東北流,合為烏木龍河,逕城西南,怕紅河來會,為鎮康河,折北逕城西,入灣甸。南:昔剌寨。西南:控尾寨。潞江安撫司隸府。府西南百三十五里。明,柔遠府,旋改潞江長官司。永樂九年升安撫司。順治初因之。土官線氏世襲。東:雷弄山。南:掌元山、高侖山。潞江自保山入,南流入龍陵。南:何坡寨。西南:景罕寨。東南:細甸。皆蠻酋結寨處。南:全勝關。孟連長官司隸府。在南。古名哈瓦。明永樂四年置長官司,直隸雲南都司。嘉靖中裁。萬歷十三年復置。順治初因之,屬永昌。乾隆二十九年改屬順寧。光緒二十年還屬。東北:孟連河,東南流入阿瓦。南甸宣撫司隸騰越。南七十里。明置南甸府,屬騰沖,旋改州。正統八年升宣撫司,直隸布政司。順治初因之,改隸騰越。土司刁氏世襲。東:丙弄蠻乾山,土酋世居其上。南:沙木籠山。西南:牙山,延袤百餘里,山泉流入南牙江。南牙江一名小梁河,即大盈江上流,納猛送水,西入乾崖。乾崖宣撫司隸騰越。西南百二十里。明置府,屬麓川平緬司。永樂元年析置長官司。正統九年升宣撫司,直隸布政司。順治初因之,改隸騰越。土官刁氏世襲。東:雲籠山,雲籠河出焉。南:云晃山。西:刺朋山布嶺。北:白蓮山,土官居之。大盈江自南甸入,名安樂河,西逕司北,與檳榔江會,又西南入盞達。盞達副宣撫司隸騰越。西南百四十里。本干崖地。明正統中置。萬歷中為緬據。順治中復置。嘉慶二十四年隸騰越。土官刁氏世襲。北:盞達山,盞達河出焉,西南會曩送河入檳榔江。檳榔江自乾崖入,逕司東南境,西南流入臘撒。隴川宣撫司隸騰越。西南百四十里。明置麓川平緬軍民宣撫司。正統十一年改置,治隴把,與乾崖、南甸稱為三宣撫,後入於緬。順治初復置,隸騰越。土官多氏世襲。有摩犁、孔明、寄箭、羅木諸山。東:龍川江,亦曰麓川江,自芒市入,西南流入遮放。西北為大金沙江。芒市安撫司隸騰越。東南四十里。古為怒謀、大枯睒、小枯睒之地。明,芒市府。正統九年改置長官司,直隸布政司,後升安撫。順治初因之,改隸騰越。土司放氏。西南:青石山,峭拔萬仞,夷砦居之。芒市河源出司西北境,西南流入遮放。猛卯安撫司隸騰越。西南百四十里。本木邦地。明析置蠻莫宣撫司。萬歷三十年,改土酋長。順治初復置。十六年改今名。土司思姓。司治後蠻哈山,山如象鼻。北:等練山,山有等練城,又有雷哈、打線諸地,皆司境險要。東:龍川江自遮放入,納碗頂河、蠻膽河諸水,又西南出漢龍、天馬關間,又西入緬甸。又西南,那莫江,下流入大金沙江。戶撒長官司隸騰越。西南百九十里。本莪昌夷地。明置土司。雍正二年裁。乾隆三十一年復置。臘撒長官司隸騰越。西南二百二十里。與戶撒同時置。西北:檳榔江自盞達入,西南流入緬甸。

順寧府:繁,難。隸迤西道。明,順寧府,領州一。順治初,沿明制。乾隆十二年,升猛緬長官司為緬寧。三十五年,置順寧縣為府治。東距省治一千二百里。廣三百四十里,袤六百九十里。北極高二十四度三十六分。京師偏西十六度二十二分。領一,州一,縣一,宣撫司一。順寧要。倚。東:東山、九龍。西:旗山。南:曇花、把邊、瓊岳。北:★山、偰山、墨玉、阿魯司泥、赤龜。東南:猛盬者石山。西南:西粵山,山下有瓊英洞。北:黑惠江,一名碧雞江,即樣濞江,自蒙化入,南流,繞津山東麓,合瀾滄江。瀾滄江自保山入,東南流,合高見槽河、三苔菁水,會黑惠江,入雲州。順甸河、順寧河合流從之。阿鐸河源出阿鐸山,南流入緬寧,注猛緬河。南:把邊關。西南:等臘關。縣西北:望城關、金馬關。府經歷駐縣西北右甸。緬寧要。府南三百里。明,猛緬長官司,隸雲州。乾隆十二年,置隸府,兼大猛撒之地,亦稱三猛。西南:梳頭山。東:銀錠、翠屏、天喜、接天。西:高嵐。南:鳳凰山、烏龍山,北對松猢猻山。瀾滄江自景東入,逕東南入鎮邊。猛緬河,即南丁河上游,源出南猛準之分水嶺,折東北,納雲州小河水及四十八道水,又西至猛賴南,為猛賴河,入孟定。南:分水嶺關。西:箐口關。北:錫蒲關。南:猛猛土巡司。雲州要。府東三十里。東北:無量山,即蒙樂山,東:阿輪山,層峰疊巘,四時蒼翠。西:蠻賴山,多竹。北:八刺、天馬。南:猛卯、蠻彌山。瀾滄江自順寧入,合順寧河,東逕州南,猛郎河、猛麻河注焉。又東入景東。南有永鎮關。小河水細流支分,凡四十八道,西南猛賴、西溪水,俱流入緬寧,注猛緬河。南:永鎮關,大猛麻土巡司駐。耿馬宣撫司府西南二百五十三里。古蠻地。本屬孟定土府。明萬歷十三年,析孟定地置安撫司,旋升宣撫司,以喳哩江為界,北距孟定百里。順治中,罕悶睆投誠,仍授宣撫司,世襲,隸永昌。乾隆二十九年改隸順寧。西:三尖山、養馬山。西南:們河源山。西北:南路河源山。北:耿馬河源山。南:們河西流,南路河北流,並入孟定。耿馬河南流,合南別河入鎮邊,即辣蒜江上源也。

永北直隸:繁,疲,難。隸迤西道。明,北勝州,隸鶴慶府,興瀾滄衛同治。康熙五年,降為屬州,隸大理。二十六年,省衛入州。三十一年,復為直隸州。三十七年,升永北府,以永寧土府隸之。三十八年,又以鶴慶府屬故順州地入焉。乾隆三十五年,改直隸。光緒三十四年,以屬之華榮莊經歷改設知縣,仍隸。東南距省治一千四里。廣四百七十五里,袤八百二十里。北極高二十六度四十三分。京師偏西十五度三十一分。領縣一,土府一,土州一。東:壺山、阿剌山。東南:大坡難嶺,高二萬餘丈,巔有龍湫。西:三刀山、伏虎山。西南:瀾滄山,衛、驛皆以此得名。西北:太保山,一曰近屯東山,下有九龍潭。其西為近屯西山,下有草海。西:金沙江自鶴慶入,緣西南入大姚。無量河自中甸入,納走馬河、觀音河、他留河、沘那河、三渡河諸水,南入金沙江。經歷司二,一駐舊衙坪,一駐華榮莊。今改縣知事一,駐金沙。順州土州同在西百二十里。西:西山關。南:南山關。北:北山關。華坪縣□□里,本名華榮莊,舊設經歷於此。光緒三十四年,雲貴總督錫良奏改縣,即以莊為縣治。永寧土府北四百五十里。明屬鶴慶,尋升為府。土官阿姓。領長官司四,今屬。北:卜兀山、剌不。東南:甲母。東北:六捏山。打沖河源出府南;北流為三岔河,又北至府東南,為勒基河,又北至府東南,納瀘沽湖水,東入四川,注鴉蹐江。瀘沽湖在府東三十里,中有三島,周二十五里,東北流,入打沖河。蒗蕖土州北百八十里。明屬鶴慶,尋廢。順治初,土官阿化投誠,未授職。康熙三十一年改土官為土舍。道光十九年復設土州,仍以阿氏襲。西南:綿綿山,麥架河出,亦曰蒗蕖水,折東北為挖開河,納別別河、鹽井河入鴉蹐江。走馬河源出東南惈儸關,西南流,入永北。羅易江自州北流,入永寧瀘沽湖。

蒙化直隸:要。隸迤西道。明,蒙化府。康熙四年,置流官,設掌印同知。雍正七年,省楚雄府之定邊入之。乾隆三十五年,改直隸。東距省治八百二十里。廣二百里,袤二百九十五里。北極高二十五度十九分。京師偏西十五度五十七分。明,蒙化故衛。康熙六年裁。西:文華、屯庫、交椅、金牛。南:甸尾。北:蒙舍山、天耳山一名甸頭山、石母山。東南:玉屏山、螺盤山、月牙山。西南:五印山。西北:巃鸃圖山。西南:瀾滄江自永昌入,南入順寧。西北:漾濞江自太和入,緣西流入順寧。禮社江有二源:東源曰白崖瞼江,東自趙州入,納毗雌江水,東南流;西源曰陽江,西北自花判山南流,納盟石河、教場河、錦溪、五道河、定邊河、窩接河諸水,東南與白崖瞼江會,曰禮社江,東南流,入南安。阿集左河,即把邊江上流,東南流,納虎街、牛街、安定河諸水,南入景東。諸始河納七溪諸水,西南流,入順寧。東:隆慶關。東南:白普關。巡司三:一駐南澗,即廢定邊城;一駐瀾滄江;一駐漾濞江。鎮一:迷渡。

景東直隸:繁,疲,難。隸迤西道。明,景東府。康熙四年,置流官,設掌印同知。乾隆三十五年,改直隸。東北距省治一千一百七十五里。廣三百四十里,袤四百二十里。北極高二十四度二十九分三十秒。京師偏西十五度三十一分。治後玉屏山。東:鳳山,舊土官陶姓世居。西:無量山,即蒙樂山,連亙三百餘里,與蒙化、雲州、緬寧、鎮邊接界,即禹貢梁州蒙山也。南:錦屏、孔雀、南鯨。北:鶴籠山。東南:瑞霞。西北:景董山,明建景東衛城於上。西南:瀾滄江,自蒙化入,緣西界入鎮邊。江上漢永平中建蘭津橋,兩岸峭壁,鎔鐵系南北,古稱巨險。把邊江一名中川河,東南流入鎮沅。又猛統河、者乾河,均南流入鎮沅。景谷河流入威遠。鹽井四:在南者曰磨臘、磨外,在西者曰大井、小井。南:景蘭關、母瓜關。北:安定關。西北:保甸土司,明宣德中建,土官陶姓,世襲巡司。北:三岔河土司,明弘治中建,土官楊姓,世襲巡司。東北:板橋驛,土官阿姓,世襲驛丞。有猛統巡司一。

曲靖府:沖,繁,疲,難。迤東道治所。明,曲靖府,領州四,縣二。康熙八年,省亦佐入羅平,又降尋甸府為州來隸。三十四年,改舊平彞衛為平彞縣來隸。雍正五年,析霑益州地置宣威州。西南距省治三百里。廣三百九十里,袤六百二十里。北極高二十五度三十三分。京師偏西十二度三十九分。領州六,縣二。南寧沖,難。倚。東:青龍、白水、關山。西:勝峰。南:石寶、觀音。北:龍華山。東南:湯池、蓮花、楊梅、瀟湘。交河自霑益入,納南、北河水,逕縣北,合白石江,折南,瀟湘江自馬龍入,西南入陸涼。東:東海子、黑龍潭,均資灌溉。白水關驛丞兼巡司,裁,移白崖巡司駐。南寧一驛。霑益州沖,難。府北三十里。康熙二十六年裁平彞衛,分境屯賦並州。三十五年仍改歸平彞。雍正五年分置宣威州。北:花山洞,交河出,即水經溫水,南盤江上源也,東南流,逕州東北,納玉光溪、沙河、阿幢河諸水,入南寧。別有盤江,自貴州畢節入,繞州北境,仍入貴州南安。南:松韶關、阿幢橋關。有炎松巡司一。驛二:松林、炎方。陸涼州疲,難。府南百二十里。明置陸涼衛。康熙六年裁衛入州。東:丘雄山、平山。西:老鴉、月砑、鐵山、桃花山。南:終南山、天馬山。交河即南盤上流,自南寧入,納板橋河、關上河、乾沖河、匯為中埏澤,折西流,納大龍潭水,又西合西山大河、鋪上河,入宜良,為大池江。東北:陸涼湖,與中埏澤相連,周百餘里。南:大生關。西:木容關。北:石嘴頭關。驛一:普陀。羅平州難。府東南二百七十里。東:金雞、雲峰、淑龍。西:天目、月濤。南:五臺、碧泉。北:安樂山、祿南山。黃泥河自貴州普安入,緣平彞界注塊澤河。復入,右合恩勤河,逕州東南。西:樓革江自師寶入,右會魯沂河,逕城北注之,至江底。八達河會西源交河入貴州興義,九龍河從之。板橋、偏山、大水井、恩勤諸汛。馬龍州沖,難。府西南五十里。西:楊唐山,一名關索嶺,上有夷關。又木容、華蓋、龍鼎、羅仵侯、中和諸山。瀟湘江源出木容山,東北合流為龍潭河,又西南為白蟒河,折西入尋甸。響水河出州東北,東流會札海子水,東入南寧為白石江。東:三叉口關。西南:分水嶺關。驛一。尋甸州沖,繁。府西百三十里。明,尋甸府。康熙八年降州來隸。東:哇山、中和山、小關索嶺。西:三棱山,山有九十九泉。南:石龍、梁王。北:珀璫山。車湖源出花箐哨山,會北山諸水瀦為湖,一名清水海,周數十里,北入會澤界為小江。龍洞,州北,三龍泉,州西,咸利灌溉。車洪河自嵩明入,亦曰尋川河,納歸龍河、玉帶河、螳螂河諸水,為阿交合溪,又東北入會澤。果馬溪源出果馬山,南流合花箐哨水,入嵩明為龍巨河。東南:木密關。北:八叉關。有易古巡司。驛一:易龍。平彞沖,繁,難。府東北九十里。明,平夷衛。康熙二十六年省衛入霑益州。三十四年改平彞。東:蠻崗山、旱感山。南:宗孟山。北:蒙洞山。塊澤河自霑益入,東流為響水河,又東至城西為十里河,又南納貴州普安明月所水,南入羅平。東:豫順關、宣威關。北:分山關。驛一:多羅。宣威州疲,難。府北二百三十里。明,霑益州。順治十六年移州治於交水。雍正五年析霑益州新化里至高坡頂置。東:宣威嶺。北:獅山、斗山、光山、馬鞍、鷓鴣。東南:木宗山。車洪江自尋甸入,納赤水河、西澤河水,北入會澤。可渡河自貴州威寧入,有二源,合為瓦岔河,會得吉河、皁衛河諸水,東北流入貴州,即北盤江上流。宛溫水源出州南東屯,北流,納州西境諸水,入可渡河。可渡關在焉,巡司駐此。驛一:儻塘。

東川府要。隸迤東道。明,東川府,尋改隸四川。康熙三十八年,設流官。雍正四年,改隸雲南。五年,置會澤縣,治巧家汛。六年,移縣附郭。嘉慶十九年,設分防巧家同知。南距省治五百九十五里。廣五百里,袤四百二十里。北極高二十六度二十一分四十一秒。京師偏西十三度一分。領一,縣一。會澤要。倚。西:天馬、雲弄、納雄。北:青龍山,山有青龍洞。西南:絳雲露山,盤亙七十餘里,接祿勸界。車洪江一名牛欄江,自宣威緣界入,納沙河、小河,流逕貴州威寧,折西北入巧家。小江自尋甸入,為阿汪河,納花溝、普翅諸水,逕碧穀壩為碧谷江,北流入巧家。以禮河源出縣南野馬川,東北納麥則、夷溪諸水,環府治,歧數支,仍同流入巧家。頭道河源出縣東犀牛塘,西北流入巧家。西南:者海一巡司。巧家要。府北二百四十里。雍正四年置會澤縣,治此。六年移縣附郭。嘉慶十九年析會澤縣地置。東:堂瑯山,水經注所謂「羊腸繩屈,八十餘里」,即此。西:鬯拙。北:大樂。東北:大涼山。西北:歸化山。西:金沙江自祿勸入,納四川會通河水,又東流,納會澤以禮河、牛欄江及境內木期古水、木期古北水,東北入魯甸。牛欄江西流,與魯甸分水,納頭道河水,並入金沙江。木期古土千戶,乾隆三十一年設,祿氏世襲。

眧通府:最要。明,烏蒙府。尋改隸四川。雍正五年,改隸雲南。六年,設流官,置恩安、永善兩縣,降鎮雄府為州,並屬府。九年,改今名。光緒三十四年,析永善之副官村置靖江縣,仍升鎮雄為直隸州。東南距省治九百二十里。廣五百五十里,袤六百三十里。北極高二十七度二十分。京師偏西十二度三十六分三十秒。領二,縣二。恩安繁,難。倚。明屬烏蒙府。雍正六年置。東:寶山、我未山。東南:樸窩。西南:博特。東北:撒途。西北:九龍山。金沙江自魯甸入,北流入永善。擦拉河自魯甸入,東北流,會普五寨水、淄泥河、八仙海水,瀦為湖。又東流入大關。大關最要。府北百八十里。雍正六年設大關通判。九年設府同知,駐此,移通判駐魯甸。西:犄角山。北:雞爪山、梨山。東南:雪山。南:龍聚山。灑魚河自恩安入,會大關河,北流,逕鹽井渡,會永善河,又北流為大紋溪,入四川慶符。東北:角魁河自鎮雄入,西北流,入大紋溪。西南:豆沙關。北:鹽井渡巡司。魯甸簡。府西南四十里。雍正九年置,移大關通判駐此。北:魯甸山,以此名。南:樂馬廠山、大黑山。北:大小涼山,山峰危聳。金沙江自巧家入,北流,逕西南入恩安。牛欄江自貴州威寧入,西北流,至南入金沙江。擦拉河源出大黑山,東北流,會馬鹿溝水,入恩安。灑魚河源出大涼山,東流,納居樂河水,入恩安。靖江舊為永善縣境副官村,縣丞駐此。光緒三十四年改縣隸府。北:巴布梁山,蠻酋居之,廣千里,袤二百餘里。東北:龍頭山,森林繁茂,礦產極盛。

鎮雄直隸州:最要。隸迤東道。明,鎮雄府,隸四川。雍正五年,改隸雲南。六年,降為州,屬眧通府。光緒三十四年,升直隸州。廣、袤、北極偏度,闕。東:鳳翅、黃甲。西:九龍、沙吶。南:竹雞山、硌砌雄山。北:烏通山。白水江自貴州威寧入,名八匡河,會九股水、黃水河、小溪河,逕牛街西北,入四川筠連,為定川溪。角魁河亦自威寧入,為洛澤河,又西北,納龍塘、威洛河諸水,西北入大關。黑墩河西北流入四川筠連。洛甸河東流入四川永寧。苴虯河,東南流入貴州威寧。彞良,州同;威信,州判、知事駐。西北:牛街。母亨巡司一。鹽井二。

澂江府:繁,難。隸迤東道。明,澂江府,領州二,縣三。康熙八年,省陽宗入河陽。西北距省治百二十里。廣二百三十六里,袤百七十五里。北極高二十四度四十二分。京師偏西十三度二十七分。領州二,縣二。河陽沖,繁。倚。康熙八年,省陽宗縣入焉。東:雲龍山。西:虎山。北:羅藏。東南:赦人、天馬。東北:碌碌山。明湖一名陽宗湖,周七十餘里,合錦溪、日角溪、七古泉諸水瀦為湖,北入宜良,為大成江。南:撫仙湖,一名羅伽湖,周三百餘里,東入鐵池河,東流入路南。東北:玗扎溪,一名東大河,合鏡莊、北坡二泉,西南入撫仙湖。羅藏溪、立馬溪、石澗溪、西浦泉諸水並從之。東北有東關、中關、西關。江川沖,繁。府東南九十里。東:海瀛山,一名孤山,特立撫仙湖中。北:屈顙顛山,上有泉,三派分流,西入滇池,東入撫仙湖,南入星雲湖。星雲湖納上河、中河、下河諸水,周八十餘里,東由海門入河陽,匯為撫仙湖。兩湖相通,中有界魚石。北:關索嶺關。驛一:江川。新興州繁。府南百二十里。東:連珠。西:馬拖羅山。南:玉乞山、研和東山。北:金蓮、落伽、臥牛。大溪自江川入,會香柏河、撒喇河,又西納羅麼溪、羅木箐二水,至州西北為玉溪。玉溪河自江川入,納西河、窯溝水、牟溪、黑龍潭,又西會甸苴河、良江河、清水河諸水,南入習峨,即曲江上流也。北;刺桐關。路南州沖,繁。城內:鹿阜山。東南:遮口山。南:紫玉、香花。西南:竹子山,峰高千仞。大池江,即鐵池河上流,自陸涼西流入,逕州北境,納小河水入宜良,復自河陽流入州西南境,繞竹子山三面,納巴盤江水為鐵池河,又南納撫仙湖諸水入寧州。東南:革泥關。驛一:和摩。

廣西直隸州:沖,繁,難。隸迤東道明,廣西府,領州三。康熙八年,省維摩州,改置三鄉縣。九年,省入師宗。雍正九年,設師宗州,州同駐舊維摩州之丘北。乾隆三十五年,降府為直隸州,降師宗、彌勒為縣,降丘北同知為縣丞。道光二十年,升丘北縣丞為縣。西北距省治四百里。廣六百三十里,袤三百一十里。北極高二十四度三十九分。京師偏西十二度三十八分。領縣三。東:靈龜山,下有矣邦池。南:文筆。北:騎鶴。西:阿盧山,山洞深邃,洞泉流入西溪,逕城西與東溪合,入矣邦池。池一名龍甸海,中有島,周三十餘里,又東南匯為支酺,又南,伏流入盤江。盤江一名南盤江,自彌勒入,東北流,逕五嶆,入丘北。巴甸河,一名巴盤江,一名潘江,南流入彌勒。五嶆,州判駐白馬嶆。師宗難。州北八十里。明,師宗州。乾隆三十五年改縣。東:恩容山。西:通元洞。南:塊卯。北:鎖北門山。盤江自丘北入,流逕縣西,與廣西西林縣分水,五羅河水南來注之,東北流入羅平。師宗水北流至縣東南,有水自落龍洞北流來會,又北至大河口,通元洞水南流折東來會,又北入羅平,注蛇場河。彌勒沖,繁。州西九十里。明,彌勒州。乾隆三十五年改縣。東:盤江山。西:阿欲部山。南:部籠山。北:陀峨。西南:十八寨山,山箐連屬。盤江自阿迷入,逕盤江山南,納石穴中濁水,名混水江,又東北入州界。巴甸河自州南入,為瀑布河,納赤甸泉、白馬河、山金河、阿欲泉、竹園村、龍潭諸水,西南入盤江。北:革泥關。西南:涅沼關。有竹園村一巡司。丘北要。州東南二百九十里。明,維摩州地。康熙八年改置三鄉縣。九年省,設州同駐此。乾隆三十五年改州同為縣丞。道光二十年改縣。北:革龍山。西:盤籠。南:石龍。盤江自州境入,納清水河,東北流入師宗。驛一:任城。

臨安府:繁,疲,難。隸臨安開廣道。明,臨安府,領州五,縣五。康熙五年,省新化入新平。雍正十年,改新平屬元江。乾隆三十五年,降建水為縣。北距省治四百三十里。廣五百七十里,袤四百八十里。北極高二十三度四十分。京師偏西十三度二十三分。領州三,縣五。建水疲,難。倚。明,建水州。乾隆三十五年改縣。東:石巖山,一名蒙山,山有水雲、南明、萬象三洞。西:馬鞍山。南:煥文山、五老峰。北:回龍山、晴山。東南:矣和波山。西南有猛屏、曲通山。瀘江自石屏入,納黃龍潭、白沙江、象沖河、塌沖河水,伏流閻洞中,東出為樂蒙河,入阿迷。禮社江自石屏入,逕虧容土司境,東南入蒙自。曲江自通海入,納狗街汛、羚羊河水,西入蒙自。黑江自思茅緣界納茨通壩、猛蚌諸水,南流入交趾。臨元鎮總兵駐此。猛丁縣,西南百六十里。光緒十六年,改土歸流,設府經歷。北:曲江巡司一。南:納更土巡司一。西南:納樓有中場、鵝黃、摩訶三礦。長官司一,光緒九年裁。西南:虧容長官司一,阿氏世襲。西北:大關。東北:箐口關。驛一:曲江。石屏州難。府西八十里。南:石屏山,州以此得名。又南:鍾秀。東:迥龍山。北:集英、乾陽。西南:左能、思陀。東南:五爪山。瀘江源出州西寶秀湖,周三十里,夾城東流,匯為異龍湖,周百五十里,中有三島。東流入建水為瀘江,即盤江最遠之一源也。北河納白花霙、昌明諸水,西流過龜樞,奔洪為龜樞河,折南流,為三百八渡河,有州南南河納五塘、彌勒溝諸水,西流來會,又南入禮社江。禮社江自元江流入西南土司境。清水河、南鼎河諸水東南流入建水。西:寶秀關,巡司一,乾隆二十年裁。西南:落恐長官司一,土官陳姓世襲。西南:左能長官司一,土官吳氏世襲。思陀長官司一,土司李氏世襲。南:瓦渣、溪處土官各一,康熙四年省,尋復置。驛一:寶秀。阿迷州沖,繁。府東南百二十里。東:東山、水城山,周圍渚澤。西:日沖、漾田。南:南洞山。東南:雷公。西南:萬象洞山。北:火山,東北有火井。樂榮河即瀘江,自萬象洞伏流,東出,繞漾田山麓,至燕子洞又伏流,東出,納東山水,折東北入盤江。盤江自寧州入,南流,至州東北會瀘江水,入彌勒。清水河自蒙自入,至水泉山入樂榮河。白期河出祿豐鄉,東南流,入文山。東:東山關。西:阿寶關。寧州沖,繁。府東北二百五十里。東:陽暮山。西:丹鳳山。南:雙獅山。北:華蓋山。東南:登樓山,山頂有池,方百步。婆兮江,即鐵池河,自建水入,會於婆兮甸,又東南會曲江。曲江自通海入,納瓜水,東流入阿迷,為盤江。撫仙湖、星雲湖俱北與河陽分界。杞麓湖西南與通海分界。西北:甸苴關。通海難。府東北百五十里。東:東華。西:西華。南:秀山,一名螺峰。北:梅山。西南:黃龍。東北:靈寶。曲江自河西入,納東山、龍泉、六村河諸水,東入寧州。杞麓湖一名通海,周百五十里,白馬溝、秀山溝、黃龍山諸水皆入焉,與河西湖中分界,與寧州湖邊分界。東:寧海關。南:建通關。驛一:通海。河西簡。府西北百八十里。東:碌溪山。西:普應、佛光、仙人洞山。南:茶山、九街子。北:琉璃山、夾雄山、碧山、黃草壩山。曲江上流為合流江,自習峨入,亦曰碌碌河,逕縣西,納舍郎河水,東入通海,為曲江。杞麓湖源出碌溪山,凡跨三邑,周百五十里,北:曲陀關。習峨難。府西北二百六十里。東:登雲山。西:老魯關、五鳳。西北:勝郎。東北:習山,其後峨山,縣以此得名。曲江自新興入,亦曰猊江,逕縣北會練江。練江源出勝郎山,流逕石屏,名龍車河,東北會於猊江,為合流江,入河西。丁癸江自易門入,西南至新平入禮社江,即元江上流也。西北:伽羅關。西:老魯關、興衣關。蒙自繁,難。府東南百五十里。東:大小雲龍山。西:目則山,即蒙自山,縣以是名。南:天馬山。東南:屏風。西南:麒麟。禮社江自建水入,為梨花江,納蠻迷渡、蠻提渡、個舊廠諸水,又東至蠻板渡,納稿吾卡水,又東南至蠻耗汛,入文山。東北:長橋海,源出縣西大屯壩,曰矣波海,南流逕新安所,有法果泉、學海逕縣南來會,下流合白期河,為三岔河,又南流,與紅河會於河口,為中、法通商要口。新安所在城西南十五里。南:蓮花灘,入越南大道。光緒間開埠通商,設臨安開廣道,有稅關,移臨元鎮總兵同駐此。東南:石馬腳關。西:菁江關。西南:楊柳口關、大窩關。南有打巫白箐,又南至江滸,地名矣吝母,渡江為勒古簿地,路通交趾。光緒間設府同知,駐個舊。

廣南府:要。隸臨安開廣道。明,廣南府。順治十八年,改流官。康熙八年,省廣西府之維摩州,以其地來隸。乾隆元年,設寶寧縣為府治。西北距省治八百五十里。廣七百二十里,袤四百三十里。北極高二十四度十四分。京師偏西十一度二十二分。領縣一,州一。寶寧要。倚。乾隆元年置。東:零雨山。南:麻卯、僻令。東南:寶月關山。西北:速部、板郎、木主三山,山各一泉,為西洋江源,東南流入富州。馬別河自文山入,納者種河諸水,北入師宗。普梅河自文山入,為藤條江,東南入交趾。西北有寶寧溪,縣以此得名。東:寶月關。南:普塘,府經歷駐。土富州府東南二百六十里。土同知儂氏世襲。光緒間設通判。城內:翠嶺。西:袪丕山。西北:花架、玉泉。西北:西安山,山洞深邃。西洋江自寶寧入,折東北,錯入廣西西林界,右合剝江,左郎河水,仍入廣西百色。西南:普梅河,自文山入,為木奔江,入越南,左賴河從之。東:剝隘鎮。

開化府:最要。隸臨安開廣道。總兵駐。明,教化、王弄、安南三長官司,屬臨安府。康熙六年,改流設府。八年,省廣西府維摩州,分其地來隸。雍正六年,命侍郎杭奕祿、學士任蘭枝賜交阯鉛廠河內地四十里,以馬白賭咒河下流為界。八年,置文山縣為府治。嘉慶二十五年,改馬白關同知為安平,仍屬府。西北距省治七百五十里。廣一千一百四十五里,袤四百二十五里。北極高二十三度二十一分。京師偏西十二度九分。領一,縣一。文山要。倚。雍正八年省通判經歷置。東:東文山,縣以此得名。西:秀石、蓑衣。北:鳳虎山。西南:西華山,層巒疊嶂,連絡如屏,橫列三十六峰。教化廢長官司治在焉。西南:紅河,即禮社江下流,自蒙自入,左新現河、右龍膊河注之,東南流入安平。白期河,一名三岔河,自蒙自流入,有那木果河注之,南流入安平界。開化大河源出縣西白龍潭,北流,匯六十五潭水,至烏期石洞出,為烏期河,折東南流,為盤龍河,伏流,至府東北復出,經府東,折而南,至天生橋汛,伏流出安平。北馬別河,東普梅河,並入寶寧。南:洪衣關、大窩關。縣丞駐江那。安平要。府南百三十里。明,安南長官司地,屬臨安府。康熙四年,長官司王朔作亂,討平之。六年,改屬府。嘉慶二十五年改,並析文山縣之東安、逢春、永平三里地屬之,仍附郭。道光三年移今治。西:天洞山,頂有石洞,瀑布飛流。西南:阿得山,綿亙無際。紅河自文山入,西南至河口汛,與白期水會。白期河自文山入,納吉林箐諸水,與紅河會,入交阯。盤龍河自文山入,南流至交阯城汛,有牛羊河來會,又東南,納左右數小水,入交阯。普梅河自寶寧入,一名那樓江,仍南流入寶寧。攀枝花河,西,下流為壩不河。南:馬白河、歸仁里二小水,均西南流入盤龍河。南:馬白關。

鎮沅直隸:最要。隸迤南道。明,鎮沅府。雍正五年,設流官,並改者樂甸長官司為恩樂縣來隸。乾隆三十五年,降直隸州。道光二十年升,省恩樂入焉。東北距省治九百一十里。廣三百四十里,袤二百九十里。北極高二十三度四十九分。京師偏西十五度二十一分。東:雲龍、石花。西:案板。南:馬容。東南:波弄。東北:哀牢山。東:魯馬河,自景東入,逕新平,復流入境,又南流人他郎,為阿墨江。東北:景來河自景東入,納蠻崗、阿薩、大弄、凹必諸水,東南入他郎,為把邊江。樹根河,亦名蠻況河,南流折西,猛統河自景東來會,為杉木江,又西南入威遠。東南:猛賴河,合欄馬河,南流入威遠。東北:恩樂故城,府經歷駐。新撫巡司,雍正十三年設,駐新撫。鹽井二:東南曰波弄,東北曰案板。雍正三年設鹽大使駐此。東北:舊祿谷寨長官司。

鎮邊直隸:最要。隸迤南道。明始置猛甸長官司。乾隆十二年,設緬寧,今境隸之。光緒十三年,析惈黑土司地上改心為猛猛土巡檢轄境,下改心為孟連宣撫司轄境。以小黑江為界。置,以猛朗壩為治。西南距省治一千八百二十里。廣四百九十里,袤一千零四十里。北極偏度闕。南:東崗。北;仙人、習遠。東南:儒岡。西南:西監、佧佤。西北:多衣嶺、老炭山。西北:小黑江,即辣蒜江,源出耿馬、孟定兩土司境。納仙人山水、南猛河水,東流入瀾滄江。瀾滄江自緬寧入,合蠻怕河、南底河,東南流入思茅。黑河,一名札糯江,自北流,經大雅口東入瀾滄江。乾河自西磨刀廠東流,經小寨,納南木河水,入思茅。南:西河,一名金河。西南有南康河,合落水洞、合英河、龍塘諸水,南流來會,入蟒冷。上改心東,下改心北,光緒十三年設二巡司分駐之。西境有佧佤、蟒冷諸夷。

元江直隸州:最要。隸迤南道。明,元江府。領州二。順治六年,設流官。十八年,省恭順、奉化二州入之。雍正十年,以臨安府新平縣來隸。乾隆三十五年,降直隸州。東北距省治五百二十里。廣三百里,袤二千一百里。北極高二十三度三十六分。京師偏西十四度十九分。領縣一。轄土職五。儒林裏轅門,復設轅門千總三。永豐里、茄革把總二。東:玉臺山,一名羅槃山,凡二十五峰。西:瓦納。西北:九龍。西南:寶山,一名銀礦山。元江即禮社江,自新平入,納漫★河、甘莊河、南淇河諸水,逕城東,南流,會清水河、南河、矣落河諸水,入石屏。李仙江自他郎入,納布固江、薩普江,名三江口,入建水,為藤條江。龜樞河自新平入,納廠溝、大小哨諸水,東南入石屏,名三百八渡,入禮社江。南:猛甸關。北:青龍關。西南:界牌關。西北:瓦厄關、定南關、杉木關。巡司一,駐因遠。新平難。州北二百里。明屬臨安府。雍正十年來隸。東:馬鹿塘山。西:哀牢山;高百數十里,廣八百里,滇南最高山也。北:硿硿山。硿硿山北有諸龍山與馬籠,皆蠻酋結寨處。南:南峒山,山七十二峒,巡司駐。西北:元江,有二源,一曰禮社江,一曰麻哈江,自習峨入,其上流為星宿江,名三岔河。逕哀牢山麓,納化龍河、賓橘河、了味河、馬龍河諸水,南入州界為元江。龜樞河即習峨,流入之。臘猛,納縣東北境羊毛沖、牛毛沖諸水,南逕魯魁山北,納亞泥河、清水河、三他拉河、窯房、得勒諸箐水,南流經大開門,為大開河,又東南流,納石屏之北河水,折西,經魯魁山南,納藤子箐諸水,入州東界。巡司一,駐楊武壩。

普洱府:最要。迤南道治所。普洱總兵駐。明,車裏宣慰司,屬元江府。土官那氏世襲。雍正七年,置普洱府。東北距省治一千二百三十里。廣一千七百九十里,袤一千二百四十里。北極高二十三度一分。京師偏西十五度十二分。領三,縣一,宣慰司一。寧洱要。倚。明,車裏宣慰司地。順治十六年編隸元江府。康熙三年調元江府通判分防普洱。其車里十二版仍屬司。雍正七年裁通判,以所屬普洱等處六大茶山及橄欖壩江內六版地置府。乾隆元年裁攸樂通判,置縣附郭。東:錦袍山,一名光山。西:太乙。南:雙星。北:觀音、玉屏。東南:班鳩坡,高出群峰,行途艱危。把邊江自他郎入,納磨黑、慢岡二河水,東南仍入他郎。猛賴河自威遠入,西南流入思茅。普洱河一名三岔河,合金龍河水,南流至縣南,合東河水,又南會南蘊河,入思茅。補遠江,源出縣東南,納整董河水,會大開河,東南入思茅。府經歷駐通關哨。東:磨黑井,設鹽大使。猛烏、整董井二鹽大使,今裁。同治十三年設石膏井提舉。光緒間,割猛烏、烏得與法。威遠最要。府西三百四十里。明,威遠直隸州。雍正三年改,屬鎮沅。又設猛班巡司。乾隆三十五年改隸府,並以猛戛、扛哄、猛班三土弁隸焉。東南:集翠山、鐵廠山。西南:仙人腳山。西:波麻。北:雷貫。瀾滄江自鎮邊入,杉木江納景谷江、寶谷江水來會,入思茅。猛撒江一名猛賴河,自鎮沅入,納暖里河、鐵廠河水,入寧洱。經歷駐猛戛。西南有戛關。西香、抱母二鹽井,雍正三年設鹽大使,駐抱母。八年移駐香鹽井,名抱香井,今改隸石膏井。思茅最要。府南百二十里。明,車里地,名思茅寨。雍正十三年設治,分車里九土司及攸樂土目地隸焉。東:倚象、鐵山。西:玉屏、六困。東南:六茶山:曰攸樂、曰蟒支、曰革登、曰蠻磚、曰倚邦、曰漫撒。易武山亦產茶。瀾滄江自威遠入,納猛撒江水,又東南,納南鐘、南勻諸水,繞九龍山麓,名九龍江,至車里北。南哈河自遮放入,又東會羅梭江,東南入交阯。羅梭江上源為清水河,南流逕寧洱為大開河,仍流入境,納龍谷、猛臘諸水,又西南入九龍江。南:永靖關。東南:倚象關。他郎要。府東北百六十里。明,恭順土州。順治十八年省入元江府。雍正十年設。乾隆三十五年改屬府。東:球香、水癸。西:紅巖、猛連、遮蔽、靈山。東南:太極山。西南:班了、法山。把邊江自鎮沅入,逕寧洱,仍南流,至南入元江。阿墨江自鎮沅入,納慢會河水、他郎河水,為布固江。寧洱南平湖,匯流灌田。車裏宣慰司轄江外六版地。乾隆三十八年宣慰司刁維屏潛逃,裁革。四十二年,以刁土宛復襲。


\end{pinyinscope}