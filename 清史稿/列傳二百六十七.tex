\article{列傳二百六十七}

\begin{pinyinscope}
儒林一

孫奇逢耿介黃宗羲弟宗炎宗會子百家王夫之兄介之

李顒李因篤李柏王心敬沈國模史孝咸韓當邵曾可

曾可孫廷採王朝式謝文洊甘京黃熙曾曰都危龍光

湯其仁宋之盛鄧元昌高愈顧培彭定求湯之錡施璜

張夏吳曰慎陸世儀陳瑚盛敬江士韶張履祥錢寅

何汝霖凌克貞屠安世鄭宏祝洤沈昀姚宏任葉敦艮

劉汋應撝謙硃鶴齡陳啟源範鎬鼎黨成李生光

白奐彩黨湛王化泰孫景烈胡承諾曹本榮張貞生

劉原淥姜國霖劉以貴韓夢周梁鴻翥法坤宏閻循觀任瑗

顏元王源程廷祚惲鶴生李塨刁包王餘佑李來章冉覲祖

竇克勤李光坡從子鍾倫莊亨陽官獻瑤王懋竑硃澤澐喬僅

李夢箕子圖南張鵬翼童能靈胡方馮成修勞潼勞史桑調元

汪鑒顧棟高陳祖範吳鼎梁錫璵孟超然汪紱余元遴

姚學塽潘諮唐鑒吳嘉賓劉傳瑩劉熙載硃次琦成孺

邵懿辰高均儒伊樂堯

昔周公制禮,太宰九兩系邦國,三曰師,四曰儒;復於司徒本俗聯以師儒。師以德行教民,儒以六藝教民。分合同異,周初已然矣。數百年後,周禮在魯,儒術為盛。孔子以王法作述,道與藝合,兼備師儒。顏、曾所傳,以道兼藝;游、夏之徒,以藝兼道。定、哀之間,儒術極醇,無少差繆者此也。荀卿著論,儒術已乖。然六經傳說,各有師授。秦棄儒籍,入漢復興。雖黃老、刑名猶復淆雜,迨孝武盡黜百家,公、卿、大夫、士、吏,彬彬多文學矣。東漢以後,學徒數萬,章句漸疏。高名善士,半入黨流。迄乎魏、晉,儒風蓋已衰矣。司馬、班、範,皆以儒林立傳,敘述經師家法,授受秩然。雖於周禮師教未盡克兼,然名儒大臣,匡時植教,祖述經說,文飾章疏,皆與儒林傳相出入。是以朝秉綱常,士敦名節,拯衰銷逆,多歷年所,則周、魯儒學之效也。兩晉玄學盛興,儒道衰弱,南北割據,傳授漸殊。北魏、蕭梁,義疏甚密。北學守舊而疑新,南學喜新而得偽。至隋、唐五經正義成,而儒者鮮以專家古學相授受焉。宋初名臣,皆敦道誼。濂、洛以後,遂啟紫陽。闡發心性,分析道理,孔、孟學行不明著於天下哉!宋史以道學、儒林分為二傳,不知此即周禮師、儒之異,後人創分,而闇合周道也。元、明之間,守先啟後,在於金華。洎乎河東、姚江,門戶分歧,遞興遞滅,然終不出硃、陸而已。終明之世,學案百出,而經訓家法,寂然無聞。揆之周禮,有師無儒,空疏甚矣。然其間臺閣風厲,持正扶危,學士名流,知能激發。雖多私議,或傷國體,然其正道,實拯世心。是故兩漢名教,得儒經之功;宋、明講學,得師道之益:皆於周、孔之道,得其分合,未可偏譏而互誚也。

清興,崇宋學之性道,而以漢儒經義實之。御纂諸經,兼收歷代之說;四庫館開,風氣益精博矣。國初講學,如孫奇逢、李顒等,沿前明王、薛之派,陸隴其、王懋竑等,始專守硃子,辨偽得真。高愈、應手為謙等,堅苦自持,不愧實踐。閻若璩、胡渭等,卓然不惑,求是辨誣。惠棟、戴震等,精發古義,詁釋聖言。後如孔廣森之於公羊春秋,張惠言之於孟、虞易說,凌廷堪、胡培翬之於儀禮,孫詒讓之於周禮,陳奐之於毛詩,皆專家孤學也。且諸儒好古敏求,各造其域,不立門戶,不相黨伐,束身踐行,闇然自修。周、魯師儒之道,可謂兼古昔所不能兼者矣。

綜而論之,聖人之道,譬若宮墻,文字訓詁,其門徑也。門徑茍誤,跬步皆歧,安能升堂入室?學人求道太高,卑視章句,譬猶天際之翔,出於豐屋之上,高則高矣,戶奧之間,未實窺也。或者但求名物,不論聖道,又若終年寢饋於門廡之間,無復知有堂室矣。是故但立宗旨,即居大名,此一蔽也。經義確然,雖不逾閑,德便出入,此又一蔽也。今為儒林傳,未敢區分門徑,惟期記述學行;若有事可見,已列於正傳者,茲不復載焉。

孫奇逢,字啟泰,又字鍾元,容城人。少倜儻,好奇節,而內行篤修。負經世之學,欲以功業自著。年十七,舉明萬歷二十八年順天鄉試。連丁父母憂,廬墓六年,旌表孝行。與定興鹿善繼講學,一室默對,以聖賢相期。

天啟時,逆閹魏忠賢竊朝柄,左光斗、魏大中、周順昌以黨禍被逮。奇逢、善繼故與三人友善。是時善繼以主事贊大學士孫承宗軍事。奇逢上書承宗,責以大義,請急疏救。承宗欲假入覲面陳,謀未就而光鬥等已死廠獄。逆閹誣坐光鬥等贓鉅萬,嚴追家屬。奇逢與善繼之父鹿正、新城張果中集士民醵金代輸。光鬥等卒賴以歸骨,世所傳範陽三烈士也。臺垣及巡撫交章論薦,不起。孫承宗欲疏請以職方起贊軍事,其後尚書範景文聘為贊畫,俱辭不就。時畿內賊盜縱橫,奇逢攜家入易州五峰山,門生親故從而相保者數百家。奇逢為部署守御,弦歌不輟。順治二年,祭酒薛所蘊以奇逢學行可比元許衡、吳澄,薦長成均,奇逢以病辭。七年,南徙輝縣之蘇門。九年,工部郎馬光裕奉以夏峰田廬,遂率子弟躬耕,四方來學者亦授田使耕,所居成聚。居夏峰二十有五年,屢徵不起。

奇逢之學,原本象山、陽明,以慎獨為宗,以體認天理為要,以日用倫常為實際。其治身務自刻厲。人無賢愚,茍問學,必開以性之所近,使自力於庸行。其與人無町畦,雖武夫捍卒、野夫牧豎,必以誠意接之。用此名在天下而人無忌嫉。著讀易大旨五卷。奇逢學易於雄縣李崶,至年老,乃撮其體要以示門人。發明義理,切近人事。以象、傳通一卦之旨,由一卦通六十四卦之義。其生平之學,主於實用,故所言皆關法戒。又著理學傳心纂要八卷,錄周子、二程子、張子、邵子、硃子、陸九淵、薛瑄、王守仁、羅洪先、顧憲成十一人,以為直接道統之傳。

康熙十四年,卒,年九十二。河南北學者祀之百泉書院。道光八年,從祀文廟。奇逢弟子甚眾,而新安魏一鼇、清苑高鐈、範陽耿極等從游最早。及門問答,一鼇為多。睢州湯斌、登封耿介皆仕至監司後往受業,斌自有傳。

介,字介石,登封人。順治九年進士,翰林院檢討。出為福建巡海道,築石城以防盜。康熙元年,轉江西湖東道,因改官制,除直隸大名道。丁母憂,服除不出。篤志躬行,興復嵩陽書院。二十五年,尚書湯斌疏薦介踐履篤實,冰★自矢,召為少詹事。會斌被劾,介引疾乞休。詹事尹泰等劾介詐疾,並劾斌不當薦介。尋予假歸,卒。所著有中州道學編、性學要旨、孝經易知、理學正宗,大旨以硃子為宗。

中州講學者,有儀封張伯行、柘城竇克勤、上蔡張沐等,皆與斌、介同時。伯行自有傳,沐見循吏傳,克勤附李來章傳。

黃宗羲,字太沖,餘姚人,明御史黃尊素長子。尊素為楊、左同志,以劾魏閹死詔獄,事具明史。思宗即位,宗羲入都訟冤。至則逆閹已磔,即具疏請誅曹欽程、李實。會廷鞫許顯純、崔應元,宗羲對簿,出所袖錐錐顯純,流血被體;又毆應元,拔其須歸祭尊素神主前;又追殺牢卒葉咨、顏文仲,蓋尊素絕命於二卒手也。時欽程已入逆案,實疏辨原疏非己出,陰致金三千求宗羲弗質,宗羲立奏之,謂:「實今日猶能賄賂公行,其所辨豈足信?」於對簿時復以錐錐之。獄竟,偕諸家子弟設祭獄門,哭聲達禁中。思宗聞之,嘆曰:「忠臣孤子,甚惻朕懷。」歸,益肆力於學。憤科舉之學錮人,思所以變之。既,盡發家藏書讀之,不足,則鈔之同里世學樓鈕氏、澹生堂祁氏,南中則千頃堂黃氏、絳雲樓錢氏,且建續鈔堂於南雷,以承東發之緒。山陰劉宗周倡道蕺山,以忠端遺命從之游。而越中承海門周氏之緒,授儒入釋,姚江之緒幾壞。宗羲獨約同學六十餘人力排其說。故蕺山弟子如祁、章諸子皆以名德重,而禦侮之功莫如宗羲。弟宗炎、宗會,並負異才,自教之,有「東浙三黃」之目。

戊寅,南都作防亂揭攻阮大鋮。東林子弟推無錫顧杲居首,天啟被難諸家推宗羲居首。大鋮恨之刺骨,驟起,遂按揭中一百四十人姓氏,欲盡殺之。時宗羲方上書闕下而禍作,遂與杲並逮。母氏姚嘆曰:「章妻、滂母乃萃吾一身耶?」駕帖未行,南都已破,宗羲踉蹌歸。會孫嘉績、熊汝霖奉魯王監國,畫江而守。宗羲糾里中子弟數百人從之,號世忠營。授職方郎,尋改御史,作監國魯元年大統歷頒之浙東。馬士英奔方國安營,眾言其當誅,熊汝霖恐其挾國安為患也,好言慰之。宗羲曰:「諸臣力不能殺耳!春秋之孔子,豈能加於陳恆,但不謂其不當誅也。」汝霖謝焉。又遺書王之仁曰:「諸公不沉舟決戰,蓋意在自守也。蕞爾三府,以供十萬之眾,必不久支,何守之能為?」聞者皆韙其言而不能用。

至是孫嘉績以營卒付宗羲,與王正中合軍得三千人。正中者,之仁從子也,以忠義自奮。宗羲深結之,使之仁不得撓軍事。遂渡海屯潭山,由海道入太湖,招吳中豪傑,直抵乍浦,約崇德義士孫奭等內應。會清師纂嚴不得前,而江上已潰。宗羲入四明山結寨自固,餘兵尚五百人,駐兵杖錫寺。微服出訪監國,戒部下善與山民結。部下不盡遵節制,山民畏禍,潛爇其寨,部將茅翰、汪涵死之。宗羲無所歸,捕檄累下,攜子弟入剡中。聞魯王在海上,仍赴之,授左副都御史。日與吳鍾巒坐舟中,正襟講學,暇則注授時、泰西、回回三歷而已。

宗羲之從亡也,母氏尚居故里。清廷以勝國遺臣不順命者,錄其家口以聞。宗羲聞之,亟陳情監國,得請,遂變姓名間行歸家。是年監國由健跳至滃洲,復召之,副馮京第乞師日本。抵長崎,不得請,為賦式微之章以感將士。自是東西遷徙無寧居。弟宗炎坐與馮京第交通,刑有日矣,宗羲以計脫之。甲午,張名振間使至,被執,又名捕宗羲。丙申,慈水寨主沈爾緒禍作,亦以宗羲為首。其得不死,皆有天幸,而宗羲不懾也。其後海上傾覆,宗羲無復望,乃奉母返里門,畢力著述,而四方請業之士漸至矣。

戊午,詔徵博學鴻儒。掌院學士葉方藹寓以詩,敦促就道,再辭以免。未幾,方藹奉詔同掌院學士徐元文監修明史,將徵之備顧問,督撫以禮來聘,又辭之。朝論必不可致,請敕下浙撫鈔其所著書關史事者送入京,其子百家得預參史局事。徐乾學侍直,上訪及遺獻,復以宗羲對,且言:「曾經臣弟元文疏薦,惜老不能來。」上曰:「可召至京,朕不授以事。即欲歸,當遣官送之。」乾學對以篤老無來意,上嘆息不置,以為人材之難。宗羲雖不赴徵車,而史局大議必咨之。歷志出吳任臣之手,總裁千里遺書,乞審正而後定。嘗論宋史別立道學傳,為元儒之陋,明史不當仍其例。硃彞尊適有此議,得宗羲書示眾,遂去之。卒,年八十六。

宗羲之學,出於蕺山,聞誠意慎獨之說,縝密平實。嘗謂明人講學,襲語錄之糟粕,不以六經為根柢,束書而從事於游談。故問學者必先窮經,經術所以經世。不為迂儒,必兼讀史。讀史不多,無以證理之變化;多而不求於心,則為俗學。故上下古今,穿穴群言,自天官、地志、九流百家之教,無不精研。所著易學象數論六卷,授書隨筆一卷,律呂新義二卷,孟子師說二卷。文集則有南雷文案、詩案。今共存南雷文定十一卷,文約四卷。又著明儒學案六十二卷,敘述明代講學諸儒流派分合得失頗詳,明文海四百八十二卷,閱明人文集二千餘家,自言與十朝國史相首尾。又深衣考一卷,今水經一卷,四明山志九卷,歷代甲子考一卷,二程學案二卷,輯明史案二百四十四卷,又明夷待訪錄一卷,皆經世大政。顧炎武見而嘆曰:「三代之治可復也!」天文則有大統法辨四卷,時憲書法解新推交食法一卷,圜解一卷,割圜八線解一卷,授時法假如一卷,西洋法假如一卷,回回法假如一卷。其後梅文鼎本周髀言天文,世驚為不傳之秘,而不知宗羲實開之。晚年又輯宋元學案,合之明儒學案,以志七百年儒苑門戶。宣統元年,從祀文廟。

宗炎,字晦木。與兄宗羲、弟宗會俱從宗周游。其學術大略與宗羲等。著有周易象辭三十一卷,尋門餘論二卷,圖書辨惑一卷,力闢陳摶之學。謂周易未經秦火,不應獨禁其圖,至為道家藏匿二千年始出。又著六書會通,以正小學。謂揚雄但知識奇字,不知識常字,不知常字乃奇字所自出也。又有二晦、山棲諸集,以故居被火俱亡。康熙二十五年,卒,年七十一。

宗會,字澤望。明拔貢生。讀書一再過不忘。有縮齋文集十卷。

百家,字主一。國子監生。傳宗羲學,又從梅文鼎問推步法。著句股矩測解原二卷。康熙中,明史館開,宗羲以老病不能行,徐乾學延百家入史館,成史志數種。

王夫之,字而農,衡陽人。與兄介之同舉明崇禎壬午鄉試。張獻忠陷衡州,夫之匿南嶽,賊執其父以為質。夫之自引刀遍刺肢體,舁往易父。賊見其重創,免之,與父俱歸。明王駐桂林,大學士瞿式耜薦之,授行人。時國勢阽危,諸臣仍日相水火。夫之說嚴起恆救金堡等,又三劾王化澄,化澄欲殺之。聞母病,間道歸。明亡,益自韜晦。歸衡陽之石船山,築土室曰觀生居,晨夕杜門,學者稱船山先生。

所著書三百二十卷,其著錄於四庫者,曰周易稗疏、考異,尚書稗疏,詩稗疏、考異,春秋稗疏。存目者,曰尚書引義、春秋家說。夫之論學,以漢儒為門戶,以宋五子為堂奧。其所作大學衍、中庸衍,皆力闢致良知之說,以羽翼硃子。於張子正蒙一書,尤有神契,謂張子之學,上承孔、孟,而以布衣貞隱,無鉅公資其羽翼;其道之行,曾不逮邵康節,是以不百年而異說興。夫之乃究觀天人之故,推本陰陽法象之原,就正蒙精繹而暢衍之,與自著思問錄二篇,皆本隱之顯,原始要終,炳然如揭日月。至其扶樹道教,辨上蔡、象山、姚江之誤,或疑其言稍過,然議論精嚴,粹然皆軌於正也。康熙十八年,吳三桂僭號於衡州,有以勸進表相屬者,夫之曰:「亡國遺臣,所欠一死耳,今安用此不祥之人哉!」遂逃入深山,作祓禊賦以示意。三桂平,大吏聞而嘉之,囑郡守餽粟帛,請見,夫之以疾辭。未幾,卒,葬大樂山之高節里,自題墓碣曰「明遺臣王某之墓」。

當是時,海內碩儒,推容城、盩厔、餘姚、昆山。夫之刻苦似二曲,貞晦過夏峰,多聞博學,志節皎然,不愧黃、顧兩君子。然諸人肥遯自甘,聲望益炳,雖薦闢皆以死拒,而公卿交口,天子動容,其著述易行於世。惟夫之竄身瑤峒,聲影不出林莽,遂得完發以歿身。後四十年,其子敔抱遺書上之督學宜興潘宗洛,因緣得入四庫,上史館,立傳儒林,而其書仍不傳。同治二年,曾國荃刻於江南,海內學者始得見其全書焉。

兄介之,字石子。國變,隱不出。先夫之卒。

李顒,字中孚,盩厔人。又字二曲,二曲者,水曲曰盩,山曲曰厔也。布衣安貧,以理學倡導關中,關中士子多宗之。父可從,為明材官。崇禎十五年,張獻忠寇鄖西,巡撫汪喬年總督軍務,可從隨征討賊。臨行,抉一齒與顒母曰:「如不捷,吾當委骨沙場。子善教吾兒矣。」遂行。兵敗,死之。顒母葬其齒,曰「齒塚」。時顒年十六,母彭氏,日言忠孝節義以督之,顒亦事母孝。饑寒清苦,無所憑藉,而自拔流俗,以昌明關學為己任。有餽遺者,雖十反不受。或曰:「交道接禮,孟子不卻。」顒曰:「我輩百不能學孟子,即此一事不守孟子家法,正自無害。」

先是顒聞父喪,欲之襄城求遺骸,以母老不可一日離,乃止。既丁母憂,廬墓三年,乃徒步之襄城,覓遺骸,不得,服斬衰晝夜哭。知縣張允中為其父立祠,且造塚於戰場,名之曰「義林」。常州知府駱鍾麟嘗師事顒,謂祠未能旦夕竣,請南下謁道南書院,且講學以慰學者之望,顒赴之,凡講於無錫,於江陰,於靖江、宜興,所至學者雲集。既而幡悔曰:「不孝!汝此行何事,而喋喋於此?」即戒行赴襄城。常州人士思慕之,為肖像於延陵書院。顒既至襄城,適祠成,乃哭祭招魂,取塚土西歸附諸墓,持服如初喪。

康熙十八年,薦舉博學鴻儒,稱疾篤,舁床至省,水漿不入口,乃得予假。自是閉關,晏息土室,惟昆山顧炎武至則款之。四十二年,聖祖西巡,召顒見,時顒已衰老,遣子慎言詣行在陳情,以所著四書反身錄、二曲集奏進。上特賜御書「操志高潔」以獎之。顒謂:「孔、曾、思、孟,立言垂訓,以成四書,蓋欲學者體諸身,見諸行。充之為天德,達之為王道,有體有用,有補於世。否則假途干進,於世無補,夫豈聖賢立言之初心,國家期望之本意耶?」居恆教人,一以反身實踐為事,門人錄之,為七卷。是時容城孫奇逢之學盛於北,餘姚黃宗羲之學盛於南,與顒鼎足稱三大儒。晚年寓富平,關中儒者咸稱「三李」。三李者,顒及富平李因篤、郿李柏也。

李因篤,字天生,富平人。明庠生。博學強記,貫串注疏。舉博學鴻儒,試授檢討。未逾月,以母老乞養,詔許之。母歿,仍不出。因篤深於經學,著詩說,顧炎武稱之曰:「毛、鄭有嗣音矣!」又著春秋說,汪琬亦折服焉。

李柏,字雪木,郿縣人。九歲失怙,事母至孝。稍長,讀小學,曰:「道在是矣!」遂盡焚帖括,而日誦古書。避荒居洋縣,入山屏跡讀書者數十年。嘗一日兩粥,或半月食無鹽。時時忍饑默坐,間臨水把釣,夷然不屑也。昕夕謳吟,拾山中樹葉書之。門人都其集曰槲葉集。年六十六,卒。

王心敬,字爾緝,鄠縣人。乾隆元年,舉孝廉方正。心敬論學,以明、新、止至善為歸。謹嚴不逮其師,注經好為異論,而易說為篤實。其言曰:「學易可以無大過矣,是孔子論易,切於人身,即可知四聖之本旨。」著有豐川集、關學編、豐川易說。

沈國模,字求如,餘姚人。明諸生。餘姚自王守仁講致良知之學,弟子遍天下。同邑傳其學者,推徐愛、錢德洪、胡瀚、聞人詮,再傳而得國模。少以明道為己任。嘗預劉宗周證人講會,歸而闢姚江書院,與同里管宗聖、史孝咸輩,講明良知之說。其所學或以為近禪,而言行敦潔,較然不欺其志,故推純儒。山陰祁彪佳以御史按江東,一日,杖殺大憝數人,適國模至,欣然述之。國模瞠目字祁曰:「世培,爾亦曾聞曾子曰『如得其情,則哀矜而勿喜』乎?」後彪佳嘗語人曰:「吾每慮囚,必念求如言。恐倉卒喜怒過差,負此良友也。」明亡,聞宗周死節,為位哭之痛,已而講學益勤。順治十三年,卒,年八十有二。

孝咸,字子虛。繼國模主姚江書院。嘗曰:「良知非致不真。」又曰:「空談易,對境難。於『居處恭,執事敬,與人忠』三語,精察而力行之,其庶幾乎!」家貧,日食一粥,泊如也。順治十六年,卒,年七十有八。

韓當,字仁父。國模弟子。自沈、史歿後,書院輟講垂十年,而當繼之。其學兼綜諸儒,以名教經世,嚴於儒、佛之辨。家貧,未嘗向人稱貸。每言立身須自節用始。人有過,於講學時以危言動之,而不明言其過。聞者內愧沾汗,退而相語曰:「比從韓先生來,不覺自失。」疾亟,謂弟子曰:「吾於文成宗旨,覺有新得。然檢點於心,終無受用,小子識之!」味其言,則知其學守仁之外,亦近硃子矣。

邵曾可,字子唯。與韓當同時。性孝友愷悌。少愛書畫,一日讀孟子「伯夷聖之清者也」句,忽有悟,悉棄去,壹志於學。姚江書院初立時,人頗迂笑之。曾可厲色曰:「不如是,便虛度此生。」遂往學。其初以主敬為宗,自師孝咸之後,專守良知。嘗曰:「於今乃知知之不可以已。日月有明,容光必照。不爾,日用跬步,鮮不貿貿者矣。」孝咸病,晨走十餘里叩床下問疾,不食而返。如是月餘,亦病。同儕共推為篤行之士焉。卒,年五十有一。曾可子貞顯,貞顯子廷採,世其學。

廷採,字允斯,又字念魯。諸生。從韓當受業,又問學於黃宗羲。初讀傳習錄無所得,既讀劉宗周人譜,曰:「吾知王氏學所始事矣。」蠡縣李恭貽廷採書,論明儒異同,兼問所學。廷採曰:「致良知者主誠意,陽明而後,原學蕺山。」又私念師友淵源,思託著述以自見。以為陽明扶世翼教,作王子傳;蕺山功主慎獨,作劉子傳;王學盛行,務使合乎準則,作王門弟子傳;金鉉、祁彪佳等能守師說,作劉門弟子傳。康熙五十年,卒,年六十四。

王朝式,字金如,山陰人。亦國模弟子。嘗入證人社,宗周主誠意,朝式守致知。曰:「學不從良知入,必有誠非所誠之蔽。」亦篤論也。順治初,卒,年三十有八。

謝文洊,字秋水,南豐人。明諸生。年二十餘,入廣昌之香山,閱佛書,學禪。既,讀龍溪王氏書,遂與友講陽明之學。年四十,會講於新城之神童峰。有王聖瑞者,力攻陽明。文洊與爭辯累日,為所動,取羅欽順困知記讀之,始一意程、硃。闢程山學舍於城西,名其堂曰尊雒。著大學中庸切已錄,發明張子主敬之旨。以為為學之本,「畏天命」一言盡之,學者當以此為心法。注目傾耳,一念之私,醒悔刻責,無犯帝天之怒。其程山十則亦以躬行實踐為主。時寧都「易堂九子」,節行文章為海內所重,「髻山七子」,亦以節概名,而文洊獨反己闇修,務求自得。髻山宋之盛過訪文洊,遂邀易堂魏禧、彭任會程山,講學旬餘。於是皆推程山,謂其篤躬行,識道本。甘京與文洊為友,後遂師之。康熙二十年,卒,年六十有七。

京,字健齋,南豐人。負氣慷慨,期有濟於世。慕陳同甫之為人,講求有用之學。與同邑封濬、曾曰都、危龍光、湯其仁、黃熙師事文洊,粹然有儒者氣象,時號「程山六君子」。著軸園稿十卷。

熙,字維緝。順治十五年進士。文洊長熙僅六歲,熙服弟子之事,常與及門之最幼者旅進退。朔望四拜,侍食起饋,唯諾步趨,進退維謹,不以為勞。彭士望比之硃子之事延平。母喪未葬,鄰不戒於火,延燎將及。熙撫棺大慟,原以身同燼。俄而風返,人以為純孝所感。

曰都,字姜公。諸生。其學務實體諸己,因自號體齋。以學行為鄉里所矜式。

龍光,字二為。善事繼母,繼母遇之非理,委曲承順,久而愛之若親子焉。

其仁,字長人。著四書切問、省克堂集。

與文洊同時者,有宋之盛、鄧元昌。

之盛,字未有,星子人。明崇禎己卯舉人。結廬髻山,足不入城巿,以講學為己任。其學以明道為宗,識仁為要,於二氏微言奧旨,皆能抉摘異同。與文水存交最篤。晚讀胡敬齋居易錄,持敬之功益密。與甘京論祭立尸喪復之禮不可廢,魏禧亟稱之。

元昌,字慕濂,贛縣人。諸生。年十七,得宋五子書,遂棄舉子業,致力於學。雩都宋昌圖以通家子謁之,元昌喜之曰:「吾小友也!」館之於家,昕夕論學為日程,言動必記之,互相考覈。一日,昌圖讀硃子大學或問首章,元昌過窗外駐聽之,謂昌圖曰:「子勉之!毋蹈吾所悔,永為硃子罪人,偷息天地也。」其互相切劘如此。

高愈,字紫超,無錫人,明高攀龍之兄孫也。十歲,讀攀龍遺書,即有向學之志。既壯,補諸生。日誦遺經及先儒語錄,謹言行,嚴取舍之辨,不尚議論。嘗曰:「士求自立,須自不忘溝壑始。」事親孝,居喪,不飲酒食肉,不內寢。晚年窮困,餟粥七日矣,方挈其子登城眺望,充然樂也。儀封張伯行巡撫江蘇,延愈主東林書院講會,愈以疾辭。平居體安氣和,有忿爭者,至愈前輒愧悔。鄉人素好以道學相詆諆,獨於愈,僉曰:「君子也。」顧棟高嘗從愈游,說經娓娓忘倦。年七十八,卒。嘗撰硃子小學注,又所著有讀易偶存、春秋經傳日鈔、春秋類、春秋疑義、周禮疏義、儀禮喪服或問。東林顧、高子弟顧樞、高世泰等,鼎革後尚傳其學。

初,世泰為攀龍從子,少侍講席,晚年以東林先緒為己任,葺道南祠、麗澤堂於梁谿,一時同志恪遵遺規。祁州刁包等相與論學。學者有南梁、北祁之稱。大學士熊賜履講學出世泰門下,儀封張伯行、平湖陸隴其亦嘗至東林講學。賜履、隴其自有傳。

顧培,字畇滋,無錫人。少從宜興湯之錡學,幡然悔曰:「道在人倫庶物而已。」之錡歿,有弟子金敞。培築共學山居以延敞,晨夕講會。遵攀龍靜坐法,以整齊嚴肅為入德之方。默識未發之中,篤守性善之旨。晚歲,四方來學日眾。張伯行頗疑靜坐之說,培往復千言,暢高氏之旨。

彭定求,字勤止,又字南畇,長洲人。父瓏授以梁谿高氏之學,又嘗師事湯斌。康熙二十五年一甲一名進士,授翰林院修撰。歷官國子監司業、翰林院侍講,充日講起居注官。前後在翰林才四年,即歸里不復出。作高望吟七章,以慕七賢。七賢者,白沙、陽明、東廓、念菴、梁谿、念臺、漳浦也。又著陽明釋毀錄、儒門法語、南畇文集。嘗與門人林云翥書云:「有原進於足下者有二:一曰無遽求高遠而略庸近。子臣弟友,君子之道。至聖以有餘不足為斤斤,孟子以堯、舜之道孝弟而已。然則舍倫常日用事親從兄之事不為,而鉤深索隱,以為聖人之道有出於人心同然之外者,必且流於異端堅僻之行矣。一曰無妄生門戶異同之見,騰口說而遺踐履。硃子之會於鵝湖也,傾倒於陸子義利之說,此陽明拔本塞源之論,致良知之指,一脈相承。其因時救弊,乃不得已之苦衷,非角人我之見。僕詠遺經,蕩滌瑕滓,因有儒門法語。足下有志聖賢,當以念臺劉子人譜、證人會二書入門,且無嘵嘵於紫陽、姚江之辨也。」定求卒年七十有八。其孫啟豐官兵部尚書,自有傳。

啟豐子紹升,頗傳家學,述儒行,有二林居集。然彭氏學兼硃、陸,識兼頓漸,啟豐、紹升頗入於禪。休寧戴震移書紹升辨之。紹升又與吳縣汪縉共講儒學。縉著三錄、二錄,尊孔子而游乎二氏。此後江南理學微矣。

湯之錡,字世調,宜興人。安貧力學,於書無所不讀,尤篤信周子主靜之說。或議其近於禪,之錡曰:「程子見學者靜坐,即嘆其善學。易言『齋戒,以神明其德』。靜坐,即古人之齋戒,非禪也。」居親喪,一循古禮,就地寢苫。事諸父如父,昆弟無間言。既而得高攀龍復七規,喟然曰:「此其入學之門乎?」仿其說為春秋兩會,聞風者不憚數千里來就學焉。明亡,之錡年二十四,即棄舉子業。嘗論出處之道曰:「『潛龍勿用』,潛要確,若不確,則遁世不見知而悔矣。古來多少高明,只為此一悔所誤。」常州知府駱鍾麟請關西李顒講學毗陵,特遣使聘之,之錡堅辭不赴;後延主東林、延陵諸講席,又不就。之錡為學,專務切近,絕無緣飾。或詢陽明致良知之說及硃、陸異同者,之錡曰:「顧吾力行何如耳,多辨論何益?」一日,抱微疾,整襟危坐而逝,年六十二。及門金敞、顧培輩,建書院於惠山之麓,奉其主祀之,著偶然雲集。

施璜,字虹玉,休寧人。少應試,見鄉先生講學紫陽,瞿然曰:「學者當如是矣!」遂棄舉業,發憤躬行。日以存何念、接何人、行何事、讀何書、吐何語五者自勘。教學者九容以養其外,九思以養其內,九德以要其成,學者稱誠齋先生。已而游梁谿,事高世泰。將歸,與世泰期某年月日當赴講。及期,世泰設榻以待,或曰:「千里之期,能必信乎?」世泰曰:「施生篤行君子也。如不信,吾不復交天下士矣。」言未既,璜果挈弟子至。著有思誠錄、小學、近思錄發明。

張夏,字秋韶,亦無錫人。隱居菰川之上,孝友力學。初從馬世奇受經,後入東林書院,從高世泰學。積十餘年,遂入世泰之室。世泰卒,其子弟相與立夏為師,事之如世泰。湯斌撫江蘇,至東林,與夏講學,韙其言。延至蘇州學宮,為諸生講孝經、小學。退而注孝經解義、小學瀹注。

吳曰慎,字徽仲,歙縣人。諸生。盡心於宋五子書。論學主乎敬,故自號曰靜菴。初游梁谿,講學東林書院。已而歸歙,會講紫陽、還古兩書院,興起者眾。

陸世儀,字道威,太倉州人。少從劉宗周講學。歸而鑿池十畝,築亭其中,不通賓客,自號桴亭。與同里陳瑚、盛敬、江士韶相約,為遷善改過之學。或橫經論難,或即事窮理,反覆以求一是。甚有商榷未定,徹夜忘寢,質明而後斷,或未斷而復辨者。著思辨錄,分小學、大學、立志、居敬、格致、誠正、修齊、治平、天道、人道、諸儒異學、經、子、史籍十四門。世儀之學,主於敦守禮法,不虛談誠敬之旨,施行實政,不空為心性之功。於近代講學諸家,最為篤實。其言曰:「天下無講學之人,此世道之衰;天下皆講學之人,亦世道之衰。嘉、隆之間,書院遍天下,呼朋引類,動輒千人,附影逐聲,廢時失事,甚有借以行其私者,此所謂處士橫議也。」又曰:「今所當學者不止六藝,如天文、地理、河渠、兵法之類,皆切於世用,不可不講。」所言深切著明,足砭虛憍之弊。其於明儒薛、胡、陳、王,皆平心論之。又嘗謂學者曰:「世有大儒,決不別立宗旨。」故全祖望謂國初儒者,孫奇逢、黃宗羲、李顒最有名,而世儀少知者。同治十一年,從祀文廟。

瑚,字言夏,號確菴。明崇禎十六年舉人。世儀格致篇首提「敬天」二字,瑚由此用力,頗得要領。因定為日紀考德法,而揭敬勝、怠勝於每日之首,格致、誠正、修齊、治平於每月之終,益信「人皆可以為堯舜」非虛語也。復取小學分為六:曰入孝,曰出悌,曰謹行,曰信言,曰親愛,曰學文;大學分為六:曰格致,曰誠意,曰正心,曰修身,曰齊家,曰治平。謂小學先行後知,大學先知後行,小學之終,即大學之始。瑚之為學,博大精深,以經世自任。值婁江湮塞,江南大饑,瑚上當事救荒書,皆精切可施行,而時不能用。明亡,絕意仕進,避地昆山之蔚村。田沮洳,瑚導鄉人築岸禦水,用兵家束伍法,不日而成。父病,刺血籥天,原以身代。父卒,遺產悉讓之弟。康熙十四年,卒,年六十有二。門人稱曰安道先生。巡撫湯斌即其故居為之立安道書院。

敬,字宗傳,號寒溪。諸生。長世儀一歲。矢志存誠主敬之學,篤於孝友。居喪三年,不飲酒食肉。有弟遇之無禮,敬終始怡怡。

士韶,字虞九,號藥園。諸生。其學以世儀為歸。同時理學諸儒多著述,士韶以為聖賢之旨,盡於昔儒之論說,惟在躬行而已。晚年取所作焚之,故不傳於後云。

張履祥,字考夫,桐鄉人。明諸生。世居楊園村,學者稱為楊園先生。七歲喪父。家貧,母沈教之曰:「孔、孟亦兩家無父兒也,只因有志,便做到聖賢。」長,受業山陰劉宗周之門。時東南文社各立門戶,履祥退然如不勝,惟與同里顏統、錢寅,海鹽吳蕃昌輩以文行相砥刻。統、寅、蕃昌相繼歿,為之經紀其家。自是與海鹽何汝霖、烏程凌克貞、歸安沈磊切劘講習,益務躬行。嘗以為聖人之於天道,「庸德之行,庸言之謹」,盡之矣。來學之士,一以友道處之。謂門人當務經濟之學,著補農書。歲耕田十餘畝,草履箬笠,提筐佐饁。嘗曰:「人須有恆業。無恆業之人,始於喪其本心,終於喪其身。許魯齋有言:『學者以治生為急。』愚謂治生以稼穡為先。能稼穡則可以無求於人,無求於人,則能立廉恥;知稼穡之艱難,則不妄求於人,不妄求於人,則能興禮讓。廉恥立,禮讓興,而人心可正,世道可隆矣。」初講宗周慎獨之學,晚乃專意程、硃。踐履篤實,學術純正。大要以為仁為本,以修己為務,而以中庸為歸。

康熙十三年,卒,年六十四。著有原學記、讀易筆記、讀史偶記、言行見聞錄、經正錄、初學備忘、近古錄、訓子語、補農書、喪葬雜錄、訓門人語及文集四十五卷。同治十年,從祀文廟。

履祥初兄事顏統。周鍾之寓桐鄉也,至其門者踵接。統曰:「鍾為人浮偽,不宜為所惑。」履祥嘗曰:「自得士鳳,而始聞過。餘不失足於周鍾、張溥之門者,皆其力也。」

寅,字子虎。與履祥為硯席交。崇禎癸未冬,海寧祝淵以抗疏論救劉宗周被逮,履祥與寅送之吳門。次年,遂偕詣宗周門受業焉。自是寅造履益謹,寇盜充斥不廢學。卒,年三十四。

汝霖,字商隱,海鹽人。嘗與友人曰:「周、程、張、硃一脈,吾輩不可令斷絕。」居喪三年,未嘗飲酒食肉。隱居澉浦紫雲村,學者稱紫雲先生。履祥子維恭,嘗受業於汝霖、克貞之門。又有吳璜、安道、邱云,皆履祥友,並命維恭師事焉。曰:「數人皆深造自得,君子人也。」璜,秀水人。剛直好義,勢利不動心。安道,嘉興人。云,桐鄉人。安道嘗言:「君子之異於小人,中國之異於夷狄,人類之異於禽獸,有禮無禮而已。士何可不學禮?」又曰:「東林諸公,大抵是重名節。然止數君子,餘皆有名而無節也。」

克貞,字渝安,烏程人。履祥交最篤。嘗謂:「父子兄弟安得人人大中、明道、伊川,夫婦安得人人伯鸞、德曜,在處之得其道耳。」與履祥游蕺山之門者,有屠安世、鄭宏。

安世,秀水人。聞宗周講學,喜曰:「茍不聞道,虛生何為!」遂執贄納拜焉。宗周既歿,從父兄偕隱於海鹽之鄉。病作,不粒食者十有七年。得宗周書,力疾鈔錄。反躬責己,無時或怠。嘗曰:「朝聞夕死,何敢不勉!」卒,年四十六。

宏,海鹽人。與弟景元俱從劉宗周受業,篤於友愛。景元短世。乙酉後絕意進取,躬灌園蔬養母,屢空,晏如也。敝衣草履,不以屑意。嘗徒跣行雨中,人不能識也。卒,年五十六。

洤,字人齋,海寧人。乾隆丙辰舉人。私淑履祥,為梓其遺書。所纂有淑艾錄。吳蕃昌、沈磊在孝友傳。

沈昀,字朗思,本名蘭先,字甸華,仁和人。劉宗周講學蕺山,昀渡江往聽。與應手為謙友。其學以誠敬為宗,以適用為主,而力排二氏。家貧絕炊,掘階前馬蘭草食之。鄰有遺之米者,昀宛轉推辭,忽僕於地,其人驚駭潛去。良久方甦,因笑曰:「其意可感,然適以困我。」手為謙嘆曰:「我於交接之際,自謂不茍。以視沈先生,猶覺愧之。」宗周身後傳其學者頗滋諍訟。昀曰:「尼父言『躬行君子』,若騰其口說以求勝,非所望於吾也。」以喪禮久廢,緝士喪禮說,以授同郡陸寅。疾革,門人問曰:「夫子今日何如?」曰:「心中無一物,惟誠敬而已。」卒,年六十三。窮無以為殮,手為謙涕泣不知所出。曰:「我不敢輕授賻衣遂,以汙先生。」其門人姚宏任趨進曰:「如宏任者,可以殮先生乎?」手為謙曰:「子篤行,殆可也。」姚遂殮之,葬於湖上。

宏任,字敬恆,錢塘人。少孤,母,賢婦也。宏任隱巿廛,其母偶見貿絲銀色下劣,慍甚,曰:「汝亦為此乎?」宏任長跪謝,原得改行,乃受業於手為謙。日誦大學一過,一言一行,服膺師說,遇事必歸於忠厚。手為謙不輕受人物,惟宏任之餽不辭。曰:「吾知其非不義也。」宏任每時其乏而致之,終身不倦。手為謙卒,執喪如古師弟子之禮。姚江黃宗炎許之曰:「是篤行傳中人也。」晚年以非罪陷縲絏。憲使閱囚入獄,宏任方朗誦大學,憲使異之,入其室,皆程、硃書;與之語,大驚,即日釋之。然宏任卒以貧死。

葉敦艮,字靜遠,西安人。劉宗周弟子。嘗貽書陸世儀,討論學術。世儀喜曰:「證人尚有緒言,吾得慰未見之憾矣。」

劉汋,字伯繩,宗周子。宗周家居講學,諸弟子聞教未達,輒私於汋。汋應機開譬,具有條理。宗周殉國難,明唐、魯二王皆遣使祭,廕汋官,水勺辭。既葬,居蕺山一小樓二十年,杜門絕人事,考訂遺經,以竟父業。有司或請見,雖通家故舊,亦峻拒之。所與接者,惟史孝感、惲日初數人。或勸之舉講會,不應。臨卒,戒其子曰:「若等安貧讀書,守人譜以終身足矣。」人譜,宗周所著書。所臥之榻,假之祁氏。疾亟,強起易之,曰:「吾豈可終於祁氏之榻?」

應手為謙,字潛齋,錢塘人。明諸生。性至孝。殫心理學,以躬行實踐為主,不喜陸、王家言。足跡不出百里,隘屋短垣,貧甚,恬如也。杭州知府嵇宗孟數式廬,欲有所贈,囁嚅未出;及讀手為謙所作無悶先生傳,乃不敢言。康熙十七年,詔徵博學鴻儒,大臣項景襄、張天馥交章薦之。手為謙輿床以告有司曰:「手為謙非敢卻薦,實病不能行耳!」客有勸者曰:「昔太山孫明復嘗因石介等請,以成丞相之賢,何果於卻薦哉?」手為謙曰:「我不能以我之不可,學明復之可。」乃免徵。二十二年,卒,年六十九。

手為謙於易、書、詩、禮、樂、春秋、孝經、四書各有著說。又撰教養全書四十一卷,分選舉、學校、治官、田賦、水利、國計、漕運、治河、師役、鹽法十考,略仿文獻通考,而於明代事實尤詳。其不載律算者,以徐光啟已有成書;不載輿地者,以顧炎武、顧祖禹方事纂輯也。又有性理大中二十八卷。門人錢塘凌嘉邵、沈士則傳其學。

硃鶴齡,字長孺,吳江人。明諸生。穎敏嗜學,嘗箋注杜甫、李商隱詩,盛行於世。鼎革後,屏居著述。晨夕一編,行不識途路,坐不知寒暑。人或謂之愚,遂自號愚菴。嘗自謂「疾惡如仇,嗜古若渴。不妄受人一錢,不虛誑人一語」云。著愚庵詩文集。初為文章之學,及與顧炎武友,炎武以本原相勖,乃湛思覃力於經注疏及儒先理學。以易理至宋儒已明,然左傳、國語所載占法,皆言象也,本義精矣,而多未備,撰易廣義略四卷。以蔡氏釋書未精,斟酌於漢學、宋學之間,撰尚書埤傳十七卷。以硃子掊擊詩小序太過,與同縣陳啟源參考諸家說,兼用啟源說,疏通序義,撰詩經通義二十卷。以胡氏傳春秋多偏見鑿說,乃合唐、宋以來諸儒之解,撰春秋集說二十二卷。又以杜氏注左傳未盡合,俗儒又以林氏注紊之,詳證參考,撰讀左日鈔十四卷。又有禹貢長箋十二卷,作於胡渭禹貢錐指之前,雖不及渭書,而備論古今利害,旁引曲證,亦多創獲。年七十餘,卒。

啟源,字長發。著有毛詩稽古編。其詮釋經旨,一準毛傳,而鄭箋佐之。訓詁聲音以爾雅為主,草木蟲魚以陸疏為則,於漢學可謂專門。又有尚書辨略二卷,讀書偶筆二卷,存耕堂槁四卷。

範鎬鼎,字彪西,洪洞人。性孝友,闡明絳州辛全之學。康熙六年進士,以母老不仕。河、汾間人士多從之受經。十八年,以博學鴻儒薦,未起。立希賢書院,置田贍學者。輯理學備考三十卷,廣理學備考四十八卷。國朝理學備考二十六卷,採辛全、孫奇逢、熊賜履、張夏、黃宗羲諸家緒綸,附以己說,議論醇正。又著五經堂文集五卷,語錄一卷。又以其父蕓茂有垂棘編,作續垂棘編十九卷,三晉詩選四十卷。

同時為辛全之學者,有絳州黨成、李生光。

成,字憲公。其學以明理去私為本。生平不求人知,鎬鼎曾揚之於人,意甚不懌,時目為狷者。其辨硃、陸異同:「論者多以陸為尊德性,硃為道問學。此言殊未然。蓋硃子之道問學,實以尊德性也,陸氏則自錮其德性矣,何尊之可云?陸子嘗曰:『不求本根,馳心外物,理豈在於外物乎?』此告子義外之學也。硃子曰:『本心物理,原無內外。以外物為外者,是告子義外之學也。』即此數語,可以見二家之異同矣。若粗論其同,二家皆欲扶世教,崇天理,去私欲,其秉心似無大異者。而實究其學,則博文約禮者,孔、顏之家法,屢見於論語,硃子得其正矣。陸氏乃言『六經皆我注腳』,又言『不識一字,管取堂堂作大丈夫』。豈不偏哉!」其辨論如此。

生光,字闇章。未冠為諸生。辛全倡學河、汾,遂往受業。篤於內行,事親至孝,全深重之。明亡,絕意仕進,自號汾曲逸民。構一草堂,日夕讀書其中,以二南大義,程、硃微言,訓門弟子。著有儒教辨正、崇正黜邪匯編,凡萬餘言。

白奐彩,字含貞,華州人。私淑於長安馮從吾,玩易洗心,詩、禮、春秋,多所自得。蓄書之富,陜以西罕儷。校讎精詳,淹貫靡遺。而沖遜自將,若一無所知。與同州黨湛、蒲城王化泰諸人相切叚。率同志結社,不入城巿,不謁官府,終日晏坐一室,手不釋卷。同知郝斌式廬,聆奐彩論議,退而嘆曰:「關中文獻也!」

湛,字子澄。嘗言:「人生須作天地間第一等事,為天地間第一等人。」故自號兩一。究宋、明以來諸儒論學語,揭其會心者於壁,默坐土室,澄心反觀,久之,恍然有契。自是動靜雲為,卓有柄持。聞李顒倡道盩厔,冒雪履冰,不憚數百里訪質所學。相與盤桓數日,每至夜半,未嘗見惰容。其志篤養邃如此。

化泰,字省庵。性方嚴峭直,面斥人過,辭色不少貸。人有一長,即欣然推遜,自以為不及。關學初以馬嗣煜嗣馮從吾,而奐彩、湛、化泰皆有名於時。武功馮雲程、康賜呂、張承烈,同州李士濱、張珥,朝邑王建常、關獨可,咸寧羅魁,韓城程良受,蒲城甯維垣,邠州王吉相,淳化宋振麟,皆篤志勵學,得知行合一之旨。至乾隆間,武功孫景烈亦能接關中學者之傳。

景烈,字酉峰。乾隆三年進士,授檢討,以言事放歸。教生徒以克己復禮。居平雖盛暑必肅衣冠。韓城王傑為入室弟子。嘗語人曰:「先生冬不爐,夏不扇,如邵康節;學行如薛文清。」又曰:「先生歸籍三十年,雖不廢講學,獨絕聲氣之交。為關中學者宗,有自來矣。」

胡承諾,字君信,天門人。明崇禎舉人。國變後,隱居不仕,臥天門巾、柘間。順治十二年,部銓縣職。康熙五年,檄徵入都。六年,至京師,以詩呈侍郎嚴正矩云:「垂老只思還舊業,暮年所急匪輕肥。」既而告歸,得請。構石莊於西村,自號石莊老人。窮年誦讀,於書無所不窺,而深自韜晦。

晚著繹志。繹志者,繹己所志也。凡聖賢、帝王、名臣、賢士與凡民之志業,莫不兼綜條貫,原本道德,切近人情,酌古而宜今,為有體有用之學。凡二十餘萬言,皆根柢於諸經,博稽於諸史,旁羅百家,而折衷於周、程、張、硃之說。承諾自擬其書於徐幹中論、顏之推家訓,然其精粹奧衍,非二書所及也。二十六年六月,卒,年七十五。所著有讀書說六卷,文體類淮南、抱樸,麟雜細碎,隨事觀理而體察之,殆繹志取材之餘,與是書相表裏。

曹本榮,字欣木,黃岡人。順治六年進士,改翰林院庶吉士。布袍蔬食,以清節自勵。八年,授秘書院編修。應詔,上聖學疏千言,其略云:「皇上得二帝三王之統,則當以二帝三王之學為學。誠宜開張聖聽,修德勤學,舉四書、五經及通鑒中有裨身心要務治平大道者,內則深宮燕閒,朝夕討論,外則經筵進講,敷對周詳。君德既修,祈天永命,必基於此。」有詔嘉納。十年,擢右春坊右贊善兼國子監司業,刊白鹿洞學規以教士。十一年,轉中允。十二年,世祖甄拔詞臣品端學裕者充日講官,本榮與焉。十三年,升秘書院侍講、左春坊左庶子兼侍讀,日侍講幄,辨論經義。敕本榮同傅以漸撰易經通注九卷,鎔鑄眾說,詞理簡明,為說經之圭臬。本榮又著五大儒語、周張精義、王羅擇編諸書。十四年八月,充順天鄉試正考官,九月,充經筵講官,十一月,以失察同考官作弊,部議革職,上以其侍從講幄日久,宥之。十八年,遷翰林院侍讀學士,改國史院侍讀學士。康熙四年,以病請回籍,卒於揚州。

本榮之學,從陽明致知之說,故論次五大儒,以程、硃、薛與陸、王並行。既告歸,宦橐蕭然,晏如也。疾革,門生計東在側,猶教以窮理盡性之學。卒之日,容城孫奇逢痛惜之。子宜溥,由廕生薦舉博學鴻儒,試,授檢討。

張貞生,字簣山,廬陵人。順治十五年進士,官翰林院侍講學士。時議遣大臣巡察,貞生上書諫。召對,所言又過戇。下考功議,革職為民,蒙恩鐫二級去官。初闡陽明良知之說,其後乃一宗考亭。居京師,寓吉安館中,蓬蒿滿徑,突無炊煙。瀕行不能具裝,故人餽贐,一無所受,其狷介如此。尋奉特旨起補原官。至京,卒。著庸書二十卷,玉山遺響集。

劉原淥,字昆石,安丘人。明末盜賊蜂起,原淥與仲兄某率鄉人壘土為堡以禦賊。賊至,守堡者多死。仲兄出鬥,身中九矢,力戰。原淥從之,發數十矢,矢盡,仲兄麾之去。原淥大呼曰:「離兄一步非死所。」乃斬二渠帥,獲馬六匹,賊遁去。亂定,以力耕致富。既而推膏腴與兄,以其餘為長兄立後,兼贍亡姊家。謝人事,求長生之術。得咯血疾,遂棄去。後讀宋儒書,乃篤信硃子之學,集硃子書作續近思錄。嘗曰:「學者居敬窮理,二者皆法先王而已。『小心翼翼,昭事上帝』,居敬之功也。『不識不知,順帝之則』,窮理之功也。」每五更起,謁祠後,與弟子講論,常至夜分。仲兄疾,籥天祈以身代。兄死,三日內水漿不入口。又為鄉人置義倉,儉歲煮粥以食饑人。嘗曰:「人與我一天而已,何畛域之有焉?」卒,年八十二。著讀書日記、四書近思續錄四卷。

後數十年,昌樂有閻循觀、周士宏,濰縣有姜國霖、劉以貴、韓夢周,德州有孫於簠、梁鴻翥,膠州有法坤宏,同縣有張貞,猶能守原淥之學。

國霖,字云一,濰縣人。父客燕中感病,國霖往省,跣走千里,至則父已歿。無錢巿棺,以衣裹尸,負之行,乞食歸里。泣告族黨曰:「父死不能斂,又不能葬,欲以身殉,又有老母在。長者何以教我?」人憐其孝,為捐金以葬。母易怒,一日怒甚,國霖作小兒嬉戲狀,長跪膝前,執母手,摑其面。母大笑,怒遂已,時年五十矣。師事昌樂周士宏,嘗與國霖至莒,樂其山川,死即葬於莒。國霖築室墓側,安貧守素,不求於人。值歉歲,莒人恐其餓死,聞於官而周之粟,亦弗卻也。昌樂閻循觀問國霖喜讀何書,曰:「論語,終身味之不盡。」

以貴,字滄嵐。康熙二十七年進士。任蒼梧令。地瑤、僮雜處,營茶山書院,以詩、書為教。歸裏後,杜門著書,有藜乘集。

夢周,字公復。乾隆二十二年進士。其學以存養、省察、致知三者為入德之資。每跬步必以禮,以恥求聞達為尚。後為來安知縣,有政聲。長洲彭紹升稱其治來如元魯山。有理堂文集,表方名,獎忠節,皆有關於世道。

鴻翥,字志南,德州人。每治一經,案上不列他書。有疑義,思之累日夜,必得而後已。益都李文藻一見奇之,為之延譽,遂知名於世。以優行貢成均。卒,年五十九。有周易觀運等書。

坤宏,字鏡野,膠州人。得傳習錄,大喜,以為如己意所出。其學以陽明為宗,以不自欺為本。乾隆六年舉人,官大理評事。卒,年八十有奇。

循觀,字懷庭,昌樂人。專志洛、閩之學,省身克己,刻苦自立。治經不立一家言,而要歸於自得。乾隆三十四年進士,吏部考功司主事。著困勉齋私記、西澗文集及尚書春秋說。

任瑗,字恕菴,淮安山陽人。年十八,棄舉子業,講學。靜坐三年,嘆曰:「聖人之道,歸於中庸,極於『精義入神以致用也,利用安身以崇德也』,豈是之謂哉?」乾隆元年,大吏舉瑗應博學鴻詞,廷試罷歸。韓夢周語人曰:「任君體用具備,有明以來無此鉅儒。」及韓將北歸,瑗語之曰:「山左人多質直,君當接引後進,以續正學。」因作反經說以示之。年八十二,卒。著有纂注硃子文類一百卷,論語困知錄二卷,反經說一卷,陽明傳習錄辨二卷,知言劄記二卷,硃子年譜一卷。

顏元,字易直,博野人。明末,父戍遼東,歿於關外。元貧無立錐,百計覓骨歸葬,世稱孝子。居喪,守硃氏家禮惟謹。古禮,「初喪,朝一溢米,夕一溢米,食之無算」。家禮刪去「無算」句,元遵之。過朝夕不敢食,當朝夕,遇哀至,又不能食,病幾殆。又喪服傳:「既練,舍外寢,始食菜果。飯素食,哭無時。」家禮改為「練後,止朝夕哭,惟朔望未除者會哭,凡哀至皆制不哭」。元亦遵之。既覺其過抑情,校以古喪禮非是。因嘆先王制禮,盡人之性,後儒無德無位,不可作也。於是著存學、存性、存治、存人四編以立教。名其居曰習齋。

肥鄉漳南書院,邑人郝文燦請元往教。有文事、武備、經史、藝能等科,從游者數十人。會天大雨,漳水溢,墻垣堂舍悉沒,人跡殆絕。元嘆曰:「天不欲行吾道也!」乃辭歸。後八年而卒,年七十。門人李恭、王源編元年譜二卷,鍾錂輯言行錄二卷,闢異錄二卷。

王源,字昆繩,大興人。兄潔,少從梁以樟游。以樟談宋儒學,源方髫齔,聞之不首肯,唯喜習知前代典要及關塞險隘攻守方略。年四十,游京師。或病其不為時文,源笑曰:「是尚需學而能乎?」因就試,中康熙三十二年舉人。或勸更應禮部試,謝曰:「吾寄焉為謀生計,使無詬厲已耳!」昆山徐乾學開書局於洞庭山,招致天下名士,源與焉。於儕輩中獨與劉獻廷善,日討論天地陰陽之變,伯王大略,兵法、文章、典制,古今興亡之故,方域要害,近代人才邪正,其意見皆相同。獻廷歿,言之輒流涕。未幾,遇李恭,大悅之,曰:「自獻廷歿,豈意復見君乎!」恭微言聖學,源聞之沛然。因持大學辨業去,是之。恭乃為極言顏元明親之道,源曰:「吾知所歸矣。」遂介恭往博野執贄元門,時年五十有六矣。後客死淮上。所著平書十卷,文集二十卷。

程廷祚,字啟生,上元人。初識武進惲鶴生,始聞顏、李之學。康熙庚子歲,恭南游金陵,廷祚屢過問學。讀顏氏存學編,題其後云:「古之害道,出於儒之外;今之害道,出於儒之中。顏氏起於燕、趙,當四海倡和翕然同氣之日,乃能折衷至當,而有以斥其非,蓋五百年間一人而已。」故嘗謂:「為顏氏其勢難於孟子,其功倍於孟子。」於是力屏異說,以顏氏為主,而參以顧炎武、黃宗羲。故其讀書極博,而皆歸於實用。乾隆元年,舉博學鴻詞,至京師,有要人慕其名,囑密友達其意,曰:「主我,翰林可得也。」廷祚拒之,卒報罷。十六年,上特詔舉經明行修之士,廷祚又以江蘇巡撫薦,復罷歸。卒,年七十有七。著易通六卷,大易擇言三十卷,尚書通議三十卷,青溪詩說三十卷,春秋識小錄三卷,禮說二卷,魯說二卷。

惲鶴生,字皋聞,武進人。因交李恭得睹顏氏遺書,自稱私淑弟子。於經長毛詩,著詩說,以毛、鄭為宗。

李恭,字剛主,蠡縣人。弱冠與王源同師顏元。躬耕善稼穡,雖儉歲必有收,而食必粢糲,妻妾子婦執苦身之役。舉康熙二十九年舉人。晚歲授通州學正,浹月,以母老告歸。恭博學工文辭,與慈溪姜宸英齊名。又嘗為其友治劇邑,逾年,政教大行,用此名動公卿間。明珠、索額圖當國,皆嘗延教其子,不就。安溪李光地撫直隸,薦其學行於朝,固辭而不謝。諸王交聘,輒避而之他。既而從毛奇齡學。著周易傳注七卷,筮考一卷,郊社考辨一卷,論語傳注二卷,大學傳注一卷,中庸傳注一卷,傳注問一卷,李氏學樂錄二卷,大學辨業四卷,聖經學規二卷,論學二卷,小學稽業五卷,恕谷後集十三卷。

恭學務以實用為主,解釋經義多與宋儒不合。又其自命太高,於程、硃之講學,陸、王之證悟,皆謂之空談。蓋明季心學盛行,儒禪淆雜,其曲謹者又闊於事情,沿及順、康朝,猶存餘說,蓋顏元及恭力以務實相爭。存其說可補諸儒枵腹之弊,然不可獨以立訓,盡廢諸家。其論易,以觀象為主,兼用互體,謂「聖教罕言性天,乾坤四德,必歸人事,屯蒙以下,亦皆以人事立言。陳摶龍圖,劉牧鉤隱,以及探無極、推先天,皆使易道入於無用」。排擊未免過激。然明人以心學竄入易學,率持禪偈以詁經,言數者反置象占於不問。誣飾聖訓,弊不可窮。恭引而歸之人事,深得垂教之旨。又以大學格物為周禮三物,謂孔子時古大學教法所謂六德、六行、六藝者,規矩尚存。故格物之學,人人所習,不必再言。惟以明德、親民標其目,以誠意指其入手而已。格物一傳,可不必補。其說本之顏元。毛奇齡惡其異己,作逸講箋以攻之。而當時學者多韙恭說焉。

刁包,字蒙吉,晚號用六居士,祁州人。明天啟舉人。再上春官,不第。遂棄舉子業,有志聖賢之學。初聞孫奇逢講良知,心鄉之。既讀高攀龍書,大喜,曰:「不讀此書,幾虛過一生。」為主奉之,或有過差,即跪主前自訟。流賊犯祁州,包毀家倡眾誓固守,城得不破。時有二璫主兵事,探卒報賊勢張甚,二璫怒其惑眾,將斬之。包厲聲曰:「必殺彼,請先殺包。」乃止。二璫相謂曰:「使若居官者,其不為楊、左乎?」賊既去,流民載道,設屋聚養之,病者給醫藥,全活尤多。有山左難婦七十餘人,擇老成家人護以歸。臨行,八拜以重託,家人皆感泣,竭力衛送。歷六府,盡歸其家。

甲申,國變,設莊烈愍皇帝主於所居之順積樓,服斬衰,朝夕哭臨如禮。偽命敦趣,包以死拒,幾及於難。遂隱居不出,於城隅闢地為齋曰潛室,亭曰肥遯。日閉戶讀書其中,無間寒暑,學者宗焉,執經之履滿戶外。居父喪,哀毀,須發盡白。三年不飲酒食肉,不內寢。及母卒,號慟嘔血,病數月,卒。

所著有易酌、四書翊注、潛室劄記、用六集,皆本義理,明白正大。又選斯文正統九十六卷,專以品行為主,若言是人非,雖絕技無取。包初與新城王餘佑為石交。

餘佑,字介祺。父延善,邑諸生,尚氣誼。當明末,散萬金產結客。有子三,長餘恪,季餘嚴,餘佑其仲也。明亡,延善率三子與雄縣馬魯建義旗,傳檄討賊。時容城孫奇逢亦起兵,共恢復雄、新、容三縣,斬其偽官。順治初,延善為仇家所陷,執赴京。餘恪揮兩弟出,為復仇計,獨身赴難,父子死燕巿。餘嚴夜率壯士入仇家,殲其老弱三十口。名捕甚急,上官有知其枉者,力解乃免。餘佑隱易州之五公山,自號五公山人。嘗受業於孫奇逢,學兵法,後更從奇逢講性命之學。隱居教授,不求聞達。教人以忠孝,務實學。卒,年七十。

李來章,字禮山,襄城人。生有神識。嘗觀石工集庭中斷石,展轉弗合,語之曰:「去宿土,當自合。是即吾學人心、道心之謂。」聞者異之。工詩古文辭。康熙十四年舉人。嘗學於魏象樞,魏戒之曰:「欲除妄念,莫如立志。」來章因作書紳語略,其持論以不背先儒有益世用為主。再學於孫奇逢、李顒。時奇逢講學百泉,來章與冉覲祖諸人講學嵩陽,兩河相望,一時稱極盛焉。再主南陽書院,作南陽學規、達天錄以教學者,士習日上。尋以母老謝歸。重葺紫雲書院,讀書其中,學者多自遠而至。母病目,來章每夙興舌舐之,目復明。

謁選廣東連山縣。連山民僅七村,丁只二千。外瑤戶大排居五,小排一十有七,數且盈萬人。重山衣復嶺,瘦石巉削,田居十分之一。瑤或負險跳梁。來章慨然曰:「瑤異類,亦有人性,當推誠以待之。」乃仿明王守仁遺意,日延耆老問民疾苦,招流亡,勸之開墾,薄其賦。復深入瑤穴,為之置約延師,以至誠相感。創連山書院,著學規,日進縣人申教之。而瑤民之秀者,亦知鄉學,誦讀聲徹巖谷。學使者交獎曰:「忠信篤敬,蠻貊信可行矣。」行取,授兵部主事,監北新倉,革運官餽遺。旋引疾歸。大學士田從典、侍郎李先復交章以實學可大用薦,得旨徵召,不出。年六十八,卒。所著有禮山園文集、洛學編、連陽八排瑤風土記、衾影錄。

冉覲祖,字永光,先賢鄆國公裔。元末有為中牟丞者,因家焉。康熙二年,鄉試第一。杜門潛居,爰取四書集注研精覃思二十年。章求其旨,句求其解,字求其訓,身體心驗,訂正群言,歸於一是,名曰玩注詳說。遞及群經,各有專書,兼採漢儒、宋儒之說。十八年,開博學鴻儒科,巡撫將薦之,欲一見覲祖。覲祖曰:「往見,是求薦也。」堅不往。少詹事耿介延主嵩陽書院,與諸生講孟子一章,剖析天人,分別理欲,眾皆悚聽。三十年,成進士,選庶吉士。三十三年,授檢討。是歲聖祖遍試翰林,御西暖閣,詢家世籍貫獨詳,有「氣度老成」之褒。越日,賜宴瀛臺,上獨識之,曰:「爾是河南解元耶?」蓋以示優異也。尋告歸。卒,年八十有二。

竇克勤,字敏修,柘城人。聞耿介傳百泉之學,從游嵩陽。六年,鄉舉至京師,謁睢州湯斌。一夕,請業,斌謂師道不立,由教官之失職。勸克勤就教職,選泌陽教諭。泌陽地小而僻,人鮮知學,克勤立五社學,月朔稽善過而勸懲之。暇則齋居讀書,雖饘粥不繼,晏如也。康熙十七年進士,選庶吉士,丁母憂歸,服除,授檢討。一日,聖祖命諸翰林作楷書,克勤書「學宗孔、孟,法在堯、舜,而其要在慎獨」十四字以進,聖祖覽而器之。尋以父老乞歸。嘗於柘城東郊立硃陽學院,倡導正學。中州自夏峰、嵩陽外,硃陽學者稱盛。卒,年六十四,著有孝經闡義。

李光坡,字耜卿,安谿人,大學士光地之弟也。生五歲,與伯叔兄弟俱陷賊壘。既脫難後,受學家庭,宗尚宋儒及鄉先正蒙引存疑諸書。次第講治十三經,濂、洛、關、閩書,旁及子、史。質不甚敏,以勤苦致熟。論學主程、硃,論易主邵子,兼取揚雄太玄,發明性理,以闡大義。壯歲專意三禮,以三禮之學至宋而微,至明幾絕,儀禮尤世所罕習,積四十年,成三禮述注六十九卷,以鄭康成為主,疏解簡明,不蹈支離,亦不侈奧博,自成一家言。其兄光地嘗著周官筆記一篇,光地子鍾倫亦著周禮訓纂二十一卷,皆標舉要旨,弗以考證辨論為長,與光坡相近,其家學如是也。

光坡家居不仁,康熙四十五年,入都,與其兄光地講貫。著性論三篇,辨論理氣先後動靜,以訂近儒之差。及歸,光地貽以詩曰:「後生茂起須家法,我老棲遲望子傳。」其惓惓於光坡如此。光地嘗論東吳顧炎武與光坡皆數十年用心經學,精勤不輟,卓然可以傳於後云。光坡天性至孝,父病篤,炷香焚掌叩天以祈延壽,病果愈。及舉孝廉方正,有司將以光坡應選,而光坡寢疾矣。卒,年七十有三。又有皋軒文編。

鍾倫,字世得。康熙三十二年舉人。初受三禮於光坡,又與宣城梅文鼎、長洲何焯、宿遷徐用錫、河間王之銳、同縣陳萬策等互相討論,其學具有本原。未仕而卒。

莊亨陽,字復齋,靖南人。康熙五十七年進士,知山東濰縣。母就養,卒於途。歸而廬墓三年,自是未嘗一日離其父。乾隆初元,禮部尚書楊名時薦士七人,亨陽與焉,授國子監助教。當是時,上方鄉用儒術,尚書楊名時、孫嘉淦,大學士趙國麟咸以耆壽名德領太學事,相與倡明正學。六堂之長,則亨陽與安溪官獻瑤、無錫蔡德晉等,皆一時之俊。每朔望謁夫子,釋菜禮畢,六堂師登講座,率國子生以次執經質疑。旬日則六堂師分占一經,各於其書齋會講南北學,弦誦之聲,夜分不絕。都下號為「四賢、五君子」。

遷吏部主事,外補德安府同知,擢徐州府。徐仍歲水災,亨陽相川澤,諮耆民,具方略,請廣開上游水道,以洩異漲,且告石林可危。未及施工而石林決,沛縣城將潰,民竄逃。亨陽駕輕舠行告父老曰:「太守來,爾民何往?」親率眾堵築,七日夜城完。在徐三年,兩遇大荒,勤賑事,幾不暇眠食。九年,遷按察司副使,分巡淮徐海道。亨陽通算術,及董河防,推究高深測量之宜,上書當路,大略謂:「淮、徐水患,在壅毛城鋪而徐州壞,壅天然減水壩而鳳、潁、泗壞,壅車邏、昭關等壩而淮、揚之上下河皆壞。宜開毛城鋪以注洪澤湖,則徐州之患息;開天然壩以注高、寶諸湖,則上江之患息;開三壩以注興、鹽之澤,則高、寶之患息;開範公堤以注之海,則興、鹽、泰諸州、縣之患息。」當路者頗韙其言,而未能用。

京察,大臣當自陳。高宗命自陳者各舉一人自代。內閣學士李清植舉亨陽,時論以為允。勘淮海災過勞,以羸疾卒。卒之日,淮海諸氓罷巿奔走,樹幟哭而投賻。訥親巡江南,監司皆鞾褲跪迎,亨陽獨長揖,訥責問,曰:「非敢惜此膝於公,其如會典所無何?」訥默然。亨陽出巡,屬吏循故事餽殽,然一切勿拒,曰:「物以烹飪,卻之是暴天物而違人情也。」所從僕皆自飲其馬,或犒之,跽而辭曰:「公視奴輩為兒子,不告而受,於心不安。告公,公必命辭,是仍虛君惠也。」強之,皆伏地,誓指其心。其感人如此。

官獻瑤,字瑜卿,安溪人。執業於漳浦蔡世遠、桐城方苞,稱高足弟子。亦以楊名時薦,補助教。甫入學,上事宜六條於其長。乾隆四年進士,選庶吉士,充三禮館纂修,授編修。九年,典試浙江。尋提督廣西、陜甘學政,遷洗馬。在關中求得宋張載二十餘代孫,囑其邑學官教之。識韓城王傑於諸生,以為大器,果如其言。獻瑤少孤,事母孝。自陜甘任滿歸,乞侍養。奉母二十餘載,母年九十乃終。撫愛諸子弟,修大小宗祠,增祭器,考禮經,遵時制以定儀式,立鄉規以教宗人,置義租以恤親族之貧者。卒,年八十。著讀易偶記三卷,尚書偶記三卷,尚書講槁,思問錄一卷,讀詩偶記二卷,周官偶記二卷,儀禮讀三卷,喪服私鈔並雜記一卷,春秋傳習錄五卷,孝經刊誤一卷,文集十六卷,詩集二卷。

王懋竑,字子中,寶應人。少從叔父式丹學,刻勵篤志,精研硃子之學,身體力行。康熙五十七年成進士,年已五十一。乞就教職,補安慶府學教授。雍正元年,以薦被召引見,授翰林院編修,在上書房行走。二年,以母憂去官,特賜內府白金為喪葬費。懋竑素善病,居喪毀瘠,服闋就職。旋以老病乞歸,越十六年卒。

懋竑性恬淡,少嘗謂友人曰:「老屋三間,破書萬卷,平生志原足矣。」歸裏後,杜門著書。校定硃子年譜,大旨在辨為學次序,以攻姚江之說。又所著白田雜著八卷,於硃子文集、語類考訂尤詳。謂易本義前九圖、筮儀皆後人依託,非硃子所作,其略云:「硃子於易,有本義,有啟蒙,與門人講論甚詳,而此九圖曾無一語及之。九圖之不合本義、啟蒙者多矣,門人何以絕不致疑也?本義之敘畫卦云:『自下而上,再倍而三,以成八卦。八卦之上,各加八卦,以成六十四卦。』初不參邵子說。至啟蒙,則一本邵子。而邵子所傳,止有先天方圓圖。其伏羲八卦圖、文王八卦圖,則以經世演易圖推而得之。同州王氏、漢上硃氏易,皆有此二圖,啟蒙因之。至硃子所自作橫圖六,則注大傳及邵子語於下,而不敢題曰伏羲六十四卦圖,其慎如此。今直云伏羲八卦次序圖、伏羲八卦方位圖、伏羲六十四卦次序圖,伏羲六十四卦方位圖,是孰受而孰傳之耶?乃云伏羲四圖,其說皆出邵氏,邵氏止有先天一圖,其八卦圖後來所推,六橫圖硃子所作。以為皆出邵氏,是誣邵氏也。」又云:「邵氏得之李之才,李之才得之穆修,穆修得之希夷先生,此明道敘康節學問源流如此。漢上硃氏以先天圖屬之,已無所據。乃今移之四圖,若希夷已有此四圖也,是並誣希夷也。文王八卦,說卦明言之。本義以為未詳,啟蒙別為之說,而不以入於本義。至於『乾,天也,故稱乎父』一節,本義以為揲蓍以求爻,啟蒙以為『乾求於坤,坤求於乾』與『乾為首』兩節,皆文王觀於已成之卦,而推其未明之象,與本義不同。今乃以為文王八卦次序圖,又孰受而孰傳之耶?卦變圖啟蒙詳之,蓋一卦可變為六十四卦,彖傳卦變,偶舉十九卦以說爾。今圖、卦皆不合,其非硃子之書明矣。」其說為宋、元儒者所未發。

又考證諸史,謂:「孟子七篇,所言齊王皆湣王,非宣王。孟子去齊,當在湣王十三四年。下距湣王之歿,更二十五六年,孟子必不及見。公孫丑兩篇,稱王不稱謚,乃其元本,而梁惠王兩篇稱宣王,為後人所增。通鑒上增威王十年,下減湣王十年,蓋遷就伐燕之歲也。」可謂實事求是矣。同邑與懋竑學硃子學者,有硃澤澐、喬僅。

澤澐,字湘陶。少勤學,得程氏讀書分年日程,尋序誦習。更學天文於泰州陳厚耀,能得其意,久之,有志於聖人之道。念硃子之學,實繼周、程,紹顏、孟,以上溯孔子。有謂硃子為道問學,陸、王為尊德性者,復取硃子文集、語類讀之,一字一句,無不精心研窮,反身體認,質之懋竑,懋竑屢答之。深信硃子居敬、窮理之學,為孔子以來相傳的緒,窮即窮其所存之心,存即存其所窮之理,止是一事,喟然嘆曰:「尊德性者,莫如硃子,道問學者,亦莫如硃子矣。」

雍正六年,詔大臣各舉所知。直隸總督劉師恕欲薦於朝,使其弟造廬請,弗應。晚年得髀疾,然猶五更起,盥沐,觀書至夜分不倦。誡其子光進曰:「聖賢工夫,正於困苦時驗之。」疾甚,謂門人喬僅曰:「死生平常事,時至則行,無所戀也。」吟邵雍詩,怡然而逝,年六十有七。所著止泉文集八卷,硃子聖賢考略十卷。

僅,字星渚。少有氣節。水決子嬰堤,眾走避,僅倡議捍塞,十日堤成。從澤澐受學,恪遵硃子教人讀書次第。取硃子書切己體察,有疑輒質澤澐,時年五十矣。澤澐稱之曰:「從吾游者眾矣,惟喬君剛甚。」因舉或問過時後學、語類訓石洪慶語告之,僅益奮。乾隆元年,舉孝廉方正,辭不就。與懋竑書,論學問之道凡再三。自謂向道晚,須用己百之功。聞弟卒江陵任,即日冒雪行數千里扶櫬歸。有潘某貸金不能償,以券與之。疾革,曰:「吾自頂至踵,無一處不痛。惟此心凝然不亂耳!」命沐浴正衣冠而逝,年六十五。著日省錄、訓子要言、困學堂遺稿,湯金釗序而行之。謂其「學術剛健篤實,發為輝光,粹然有德之言」。

李夢箕,字季豹,連城人。年十五而孤。精進學業,崇向硃子,以孝友著稱。其教人輒言為善最樂,人易而忽之。夢箕曰:「為之難,汝為之否乎?」人問之曰:「其樂何如?」曰:「不愧不怍,」「孰與孔、顏之樂?」曰:「熟之而已矣!」事兄如嚴父,撫猶子如子。每語諸子以氣質之偏,使知變化。疾亟,謂所親曰:「吾生平竭力檢身,將毋有不及省者?第言之,得聞過而終,亦云幸矣。」卒,年八十一。

子圖南,字開士。康熙六十一年舉人。能世其學。初工詩古文,既而嘆曰:「吾學自有身心性命所宜急者,可以虛名騖乎?」於是究心濂、洛、關、閩書,以反躬切己為務。居連峰、點石諸山中者久之。嘗曰:「學者唯利名之念為害最大。越此庶可與共學。」與蔡世遠講明修身窮理之要,世遠重之。雍正九年,吏部檄天下舉人需次縣令者先赴京學習政事,圖南至,觀政戶部。以母病亟歸,歸先母卒,年五十七。雷鋐謂:「學聖人必自狷者始,圖南庶足當之。」時邑人張鵬翼、童能靈皆以學行稱。

鵬翼,字蜚子。歲貢生。八歲嗜學,十餘歲通諸經。塾師教以作文取科第,心疑之。熟讀四書大全,忽悟曰:「心當在身內,身當在心內。」遂不仕。連城處萬山中,無師。鵬翼年已四十,始見近思錄及硃子全書。更十年,始見薛文清讀書錄。嘗曰:「考亭易簀之時,乃我下帷之始。」蓋俛焉日有孳孳,不知其老且耄也。所居鄉曰新泉,男女往來二橋,道不拾遺。巿中交易,先讓外客,皆服鵬翼教也。著有讀經說略、理學入門、孝子傳、歷代將相諫臣三譜、二十二史案、芝壇日讀小記。

能靈,字龍儔。貢生。好學,守程、硃家法,不失尺寸。乾隆元年,舉博學鴻詞。累舉優行,皆以母老辭。年九十,兄弟白首同居。居喪以禮,化及鄉人。能靈嘗與雷鋐論易,主河圖以明象數之學。其樂律古義,謂:「洛書為五音之本,河圖為洛書之源。河圖圓而為氣,洛書方而為體。五音者氣也,氣凝為體,體以聚氣,然後聲音出焉。蔡氏律呂新書沿淮南子、漢書之說,誤以亥為黃鍾之實。惟所約寸分釐絲忽之法,其數合於史記律書,因取其說為之推究源委以成書。」他著中天河洛五倫說、硃子為學考、理學疑問。

連城理學,始自宋之邱起潛、明之童東皋,而能靈、鵬翼繼之。力敦倫紀,嚴辨硃、陸異同。張伯行撫閩時,建文溪書院,祀起潛、東皋。後增建五賢書院,中祀宋五子,而以能靈、鵬翼配焉。

胡方,字大靈,新會人。歲貢生。方敦崇實行,處道學風氣之末,獨守堅確。總督吳興祚聞其名,使招之,方走匿,不能得也。事父母,色養靡不周,而心常如不及。遇有病,憂形於色,藥必嘗而後進。夜必衣冠侍,未嘗就寢。及居喪,藉草宿柩旁,三年不入內。先人田廬,悉以與弟,授徒自給。族★L5不能自存者,竭力資之。有達官齎重金乞其文為壽,不應;吏懾之,不應;家人告以絕糧,不應。鄉曲子弟偶蹈不韙,有原就鞭撲,不原聞其事於方者。里中語曰:「可被他人笞,勿使胡君知。他人笞猶可,胡君愧煞我。」其從學者,仕與未仕,白首猶懍懍奉其教。雖困甚,終不入公庭。聞聲向慕,以得見為喜,曰:「教我矣!」有以廕得官,則大慚曰:「吾未能信,得無辱我夫子。」方告之曰:「為官能不愛錢,致力於官守,有何不可?」其人卒不負其言。

四十後杜門著述,所居曰鹽步。元和惠士奇督學粵東,聞方名,艤舟村外,遣吳生至其家求一見,急揮手曰:「學政未蕆事,不可見!不可見!」出吳而扃其門。士奇乃索所著書而去。試事畢,仍介吳生以請,則假一冠投刺,至,長揖曰:「今日齋沐謝知己。方年邁,無受教地,不能執弟子禮。」數語遂起。惠握其手曰:「縱不欲多語,敢問先生,鄉人誰能為文者?」答曰:「並世中無人。必求之,惟明季梁朝鐘耳!」士奇遂求梁文並各家文刻之,名曰嶺南文選。既而疏薦於朝。士奇嘗語吳生曰:「胡君貌似顧炎武,豐厚端偉,必享大名。」蓋當時知方者,士奇一人而已。卒,年七十四。著有周易本義注六卷,四子書注十卷,莊子注四卷,鴻桷堂詩文集六卷。集中謁白沙祠諸作及白沙子論,具見淵源所自。粵中勵志篤行者,方後有馮成修、勞潼。

成修,字達夫,南海人。父遠出不歸,成修生有至性,語及其父,輒涕泗交頤。乾隆四年進士,選庶吉士,散館改吏部主事。晉禮部祠祭司郎中,典試福建、四川,督學貴州,揭條約十四則以訓士。成修初計偕,即遍訪其父跡。得官後,兩次乞假尋親,卒無所遇,不復出。授經裏中,粹然師範。年八十,計其父已百有一齡。乃持服三年,終身衣布。乙卯重宴鹿鳴,逾年卒,年九十有五。

潼,字潤芝,亦南海人。乾隆二十年舉人。髫齡時,母常於榻上授毛詩,長遂習焉。盧文弨視學湖南,召之往。至冬乃歸,母思念殊切。抵家時漏三下,跪母榻前,母且泣且撫之曰:「其夢也耶?」潼悲不自勝,自是絕意進取,侍養十有六年而母卒。潼哀毀骨立,杖而後起。家人或失潼所在,即於殯所覓之,則已慟哭失聲矣。又痛早孤,故以莪野為號。嘗言:「讀孔子書,得一言,曰『務民之義』;讀孟子書,得一言,曰『強為善而已矣』;讀硃子書,得一言,曰『切己體察』。」著有四書擇粹十二卷,孝經考異選註二卷,救荒備覽四卷,荷經堂古文詩稿四卷。

勞史,字麟書,餘姚人。世為農。少就傅讀書,長躬耕養父母,夜則披卷莊誦。讀硃子小學、中庸序,慨然發憤,以道自任,舉動必依於禮。繼讀硃子近思錄,立起設香案,北面稽首曰:「吾師在是矣!」常自刻責,謂:「天之命我者,若君之詔臣,父之詔子。一廢職,即膺嚴譴,一墜家業,即窮無所歸,可不慎哉!」其論學以為始於不妄語,不妄動,即極諸至誠無息。接後學,委曲進誠,雖傭工下隸皆引之鄉道,曰:「盡爾職分,務實做去,終身不懈,即聖賢矣。勿過自薄也。」聞者莫不爽然。里中負販者近史居,不敢貨偽物。芻兒牧童或折棄矰繳,毀機穽。有鬥爭,就史質,往往置酒求解。門人桑調元自錢塘來謁,論學數日。將別,送之曰:「吾壽不過三年,恐不復相見。行矣勉之!」後三年九月,謂門人汪鑒曰:「不過今月,吾將去矣!」遂遍詣親友家,與老者言所以教,少者言所以學,令家人治木飭後事。晦前一夕,沐浴更衣,移榻正寢,炳燭晏坐如平時,旋就寢。明晨,撫之冰矣。調元為刻其遺書十卷,其書謂易之為道,細無不該,遠無不屆,故多本易理以推人物之性。

調元,字弢甫,錢塘人,為孝子天顯之子。天顯親病革,合羊脂和粥以進。親死,抱鐺而哭,人為繪抱鐺圖。調元受業於史,得聞性理之學。雍正十一年,召試通知性理,欽賜進士,授工部主事,引疾歸。調元主九江濂溪書院,構須友堂,祠餘山先生,以著淵源有自,餘山,史自號也。調元東皋別業又闢餘山書屋,以友教四方之士。為人清鯁絕俗,足跡遍五嶽。晚主灤源書院,益暢師說。

鑒,餘姚人。父死於雲南,鑒護喪歸至漢川,遇大風,舟且覆,抱棺大哭,誓以身殉。忽風回得泊沙渚,眾呼為孝子。為人尚氣節,史戒之曰:「英氣,客氣也。其以問學融化之。」史之歿也,鑒實左右焉。

顧棟高,字震滄,無錫人。康熙六十年進士,授內閣中書。雍正間,引見,以奏對越次罷職。乾隆十五年,特詔內外大臣薦舉經明行修之士,所舉四十餘人。惟大學士張廷玉、尚書王安國、侍郎歸宣光舉江南舉人陳祖範,尚書汪由敦舉江南舉人吳鼎,侍郎錢陳群舉山西舉人梁錫興,大理寺卿鄒一桂舉棟高,此四人,論者謂名實允孚焉。尋皆授國子監司業。棟高以年老不任職,賜司業銜。皇太后萬壽,棟高入京祝嘏,召見,拜起令內侍扶掖。棟高奏對,首及吳敝俗,請以節儉風示海內,上嘉之。陛辭,賜七言律詩二章。二十二年,南巡,召見行在,加祭酒銜,賜御書「傳經耆碩」四字。二十四年,卒於家,年八十一。

所學合宋、元、明諸儒門徑而一之,援新安以合金谿,為調停之說。著大儒粹語二十八卷,又著春秋大事表百三十一篇,條理詳明,議論精覈,多發前人所未發。毛詩類釋二十一卷,續編三卷,採錄舊說,發明經義,頗為謹嚴。其尚書質疑二卷,多據臆斷,不足以言心得。大抵棟高窮經之功,春秋為最,而書則用力少也。

陳祖範,字亦韓,常熟人。雍正元年舉人,其秋禮部中式,以病不與殿試。歸,僦廛華匯之濱,楗戶讀書。居數年,詔天下設書院以教士,大吏爭延為師,訓課有法。或一二年輒辭去,曰:「士習難醇,師道難立。且此席似宋時祠祿,仕而不遂者處焉。吾不求仕,而久與其列為汗顏耳。」薦舉經學,祖範褒然居首。以年老不任職,賜司業銜。乾隆十八年,卒於家,年七十有九。所撰述有經咫一卷,膺薦時錄呈御覽。文集四卷,詩集四卷,掌錄二卷。祖範於學務求心得,論易不取先天之學,論書不取梅賾,論詩不廢小序,論春秋不取義例,論禮不以古制違人情,皆通達之論。同縣顧主事鎮傳其學。

吳鼎,字尊彞,金匱人。乾隆九年舉人,授司業。洊擢翰林院侍講學士,轉侍讀學士。大考降左春坊左贊善,遷翰林院侍講,旋休致。所撰有易例舉要二卷,十家易象集說九十卷。裒宋俞琰、元龍仁夫、明來知德等十家易說,以繼李鼎祚、董楷之後。其東莞學案,則專攻陳建學蔀通辨作也。兄鼐,亦通經,深於易、三禮。

梁錫興,字確軒,介休人。雍正二年舉人,亦授司業,與吳鼎同食俸辦事,不為定員。乾隆十七年,命直上書房,累遷詹事府少詹事。大考降左庶子,擢祭酒,坐遺失書籍鐫級。膺薦時,以所撰易經揆一呈御覽。鼎、錫興並蒙召對,面諭曰:「汝等以是大學士、九卿公保經學,朕所以用汝等去教人。是汝等積學所致,不是他途幸進。」又曰:「窮經為讀書根本。但窮經不徒在口耳,須要躬行實踐。汝等自己躬行實踐,方能教人躬行實踐。」鼎、錫興頓首祇謝。又奉諭:「吳鼎、梁錫興所著經學,著派翰林二十員、中書二十員,在武英殿各謄寫一部進呈。原書給還本人。所有紙札、飯食皆給於官。著梁詩正、劉統勛董理其事。」稽古之榮,海內所未有也。

孟超然,字朝舉,閩縣人。乾隆二十五年進士,選庶吉士,改兵部主事,累遷吏部郎中。三十年,典廣西試,尋督學四川,廉正不阿,遇士有禮。以蜀民父子兄弟異居者眾,作厚俗論以箴其失。旋以親老,請急歸,年甫四十二,遂不出。性至孝,侍父疾,躬執廁牏。戚族喪娶,雖空乏必應。嘗嘆服徐陵「我輩猶有車可賣」之言。其學以懲忿、窒欲、改過、遷善為主。嘗曰:「變化氣質,當學呂成公;刻意自責,當學吳聘君。」又曰:「談性命,則先儒之書已詳,不如歸諸實踐;博見聞,則將衰之年無及,不如反諸身心。」其讀商子云:「論至德者,不和於俗;成大功者,不謀於眾。聖人茍可以強國,不法其故,茍可以利民,不循其禮,以為此王介甫之先驅也。然鞅猶明於帝王霸之說,介甫乃以言利為堯、舜、周公之道,又鞅之不如矣。」其論楊時云:「龜山得伊、洛之正傳,開道南之先聲。然為人身後文,如溫州陳君、李子約、許德占、張進、孫龍圖諸墓志,往往述及釋氏之學,而贊之曰『安』、曰『定』、曰『靜』,毋惑乎後之學者,援儒入墨,紛紛不已也。」

超然性靜,家居杜門卻掃。久之,巡撫徐嗣曾請主鼇峰書院,倡明正學。閩之學者,以安溪李光地、寧化雷鋐為最。超然輩行稍後,而讀書有識,不為俗學所牽,則後先一揆也。居喪時,考士喪禮、荀子及宋司馬光、程子、硃子說,並採近代諸儒言論,以正閩俗喪葬之失,著喪禮輯略二卷。傷不葬其親者惑形家言以速禍,取孟子「掩之誠是」之語,作誠是錄一卷。他著有焚香錄、觀復錄、晚聞錄。

汪紱,初名烜,字燦人,婺源人。諸生。少稟母教,八歲,四子書、五經悉成誦。家貧,父淹滯江寧,侍母疾累年,十日未嘗一飽。母歿,紱走詣父,勸之歸。父曰:「昔人言家徒四壁,吾壁亦屬人。若持吾安歸?」叱之去。紱乃之江西景德鎮,畫碗,傭其間。然稱母喪,不御酒肉。後飄泊至閩中,為童子師。及授學浦城,從者日進。聞父歿,一慟幾殆,即日奔喪,迎櫬歸。

紱自二十後,務博覽,著書十餘萬言,三十後盡燒之。自是凡有述作,凝神直書。自六經下逮樂律、天文、地輿、陣法、術數無不究暢,而一以宋五子之學為歸。著有易經詮義十五卷,尚書詮義十二卷,詩經詮義十五卷,四書詮義十五卷,詩韻析六卷,春秋集傳十六卷,禮記章句十卷、或問四卷,參讀禮志疑二卷,樂經律呂通解五卷,樂經或問三卷,孝經章句一卷。其參讀禮志疑多得經意,可與陸隴其書並存。

紱之論學,謂學不可不知要。然所以得要,正須從學得多後,乃能揀擇出緊要處。謂易理全在象、數上乘載而來。謂書歷象、禹貢、洪範須著力去考,都是經濟。謂詩只依字句吟詠,意味自出。謂看周禮,須得周公之心,乃於宏大處見治體之大,於瑣屑處見法度之詳。謂春秋非理明義精,殆未可學。謂「格物」之「格」訓「至」,如書言「格於上下」、「格於皇天」,皆「至到」之義。上文「致知」字為「推致」,則「格物」為「窮至物理」甚明。謂「性與天道不可得聞」,直是不可得聞,陸、王家因早聞性天,而未嘗了悟,又果於自信,遺害後人也。謂周子言「一」,言「無欲」,程子言「主一」,言「無適」,微有不同。周子所謂「一」者天也,所謂「欲」者人也。純乎天,不參以人,一者即無欲也。程子所謂「一」者事也,所謂「適」者心也。一其心於所事,而不強事以成心,無適之謂一也。當時大興硃筠讀其書,稱其信乎以人任己,而頡頑古人。其後善化唐鑒亦稱其功夫體勘精密,由不欺以至誠明。紱初聘於江,比歸娶,江年二十八矣。江嘗語諸弟子曰:「吾歸汝師三十年,未嘗見一怒言、一怒色也。」乾隆二十四年,卒,年六十八。子思謙,增生,毀卒。同縣余元遴傳其學。

元遴,字秀書。諸生。著有庸言、詩經蒙說、畫脂集。

姚學塽,字晉堂,歸安人。性靜介。孩稚時,見物不取。父兄坐庭上,久侍立,足不動。既長,讀書,毅然以身學。父喪骨毀,感動鄉里。嘉慶元年進士,以中書用。時和珅為大學士,中書於大學士例執弟子禮,學塽恥之,遂歸。後四年和珅伏誅,始入都任職。十三年,主貴州鄉試。歸途聞母憂,痛父母不得躬侍祿養,遂終身不以妻子自隨。服闋,至京,轉兵部主事,遷職方司郎中。妻張有婦德,畜一妾請遣侍京寓,不許,乃歸妾父。妾方氏,十七,曰:「婦人從一者也,吾事有主矣。」竟不嫁。

學塽居京師四十年,若旅人之厄者,僦僧寺中,霜華盈席,危坐不動。居喪時有氈帽一,布羔裘一,終身服之,藍褸不改,蓋所謂終身之喪者。初彭齡掌兵部,請學塽至堂上,躬起肅揖之,學塽亦不往謝。大學士百齡兼管兵部,屢詢司員姚某何在,欲學塽詣其宅一見之,終不往也。學塽六十生辰,同里姚文田貽酒二罌為壽,固辭。文田曰:「他日以此相報可乎?」乃受之。學塽之學,由狷入中行。以敬存誠,從嚴毅清苦中發為光風霽月。闇然不求人知,未嘗向人講學。病篤,握其友潘諮手曰:「君勉矣!人生獨知之地,鮮無愧者。我生平竭蹶,竟如此止。君亦就衰盡,所得為俟年而已。」遂逝,年六十有六。

諮,字少白,會稽人。少卓犖,好獨游天下奇山水,足跡逾數萬里。與學爽友善,日求寡過,以無玷古人。與長民者言,言愛人;與里老言,言耕鑿樹畜;與士人言,言孝弟忠信。遇名下士,則告以實行為首務,尤競競於義利之辨。居惟一被,日兩蔬食。食有餘,則以給人之困者。有數人賚金為其母壽,不可返,乃各取少許。其母知之,怒曰:「汝見僧以如來像丐巿者乎?吾其為像也!」乃謝而盡散之。著有古文八卷,詩五卷,常語二卷。

唐鑒,字鏡海,善化人。父仲冕,陜西布政使,自有傳。鑒,嘉慶十四年進士,改庶吉士。十六年,授檢討。二十三年,授浙江道監察御史。坐論淮鹽引地一疏,吏議鐫級,以六部員外郎降補。會宣宗登極,詔中外大臣各舉所知,諸城劉鐶之薦鑒出知廣西平樂府,擢安徽寧池太廣道。調江安糧道,擢山西按察使。遷貴州,擢浙江布政使,調江寧,內召為太常寺卿。海疆事起,嚴劾琦善、耆英等,直聲震天下。鑒潛研性道,宗尚洛、閩諸賢。著學案小識,推陸隴其為傳道之首,以示宗旨。

時蒙古倭仁,湘鄉曾國籓,六安吳廷棟,昆明竇垿、何桂珍皆從鑒考問學業,陋室危坐,精思力踐。年七十,斯須必敬。致仕南歸,主講金陵書院。文宗踐阼,有詔召鑒赴闕,入對十五次,中外利弊,無所不罄。上以其力陳衰老,不復強之服官,令還江南,矜式多士。咸豐二年,還湘,卜居於寧鄉之善嶺山,深衣蔬食,泊然自怡。晚歲著讀易小識,編次硃子全集,別為義例,以發紫陽之蘊。十一年,卒,年八十有四。曾國籓為上遺疏,賜謚確慎。著有硃子年譜考異、省身日課、畿輔水利備覽、易反身錄、讀禮小事記等書。

吳嘉賓,字子序,南豐人。道光十八年進士,改庶吉士,授編修。既通籍,尤究心當世利弊。嘗條陳海疆事宜,上嘉納焉。二十七年,緣事謫戍軍臺,尋釋回。咸豐初,以督團兵援郡城功,賞內閣中書。同治三年。於本邑三都墟口擊賊遇害,奉旨賜恤,並建專祠。

嘉賓學宗陽明,而治經字疏句釋以求據依,非專言心學者,其要歸在潛心獨悟,力求自得。尤長於禮,成禮說二卷,自序云:「小戴記四十九篇,列於學宮。其高者蓋七十子之微言,下者乃諸博士所摭拾耳。宋以來取大學、中庸與論、孟列為四書,世無異議;則多聞擇善,固有不必盡同者。餘獨以禮運、內則、樂記、孔子閒居、表記諸篇,為古之遺言,備錄其文,以資講肄。其餘論說多者亦全錄之,否則著吾說所以與鄭君別者,以備異同焉。易曰『知崇禮卑』,又曰『謙以制禮』。夫禮者,自卑而尊人。古之制禮者上也,上之人能自卑,天下誰敢不為禮者。先王之禮,行於父子兄弟夫婦養生送死之間,而謹於東西出入升降辭讓哭泣闢踴之節,使人明乎吾之喜怒哀樂,莫敢逾夫親疏貴賤長幼男女之分;而其至約者,則在於安定其志氣而已,故曰禮、樂不可斯須去身。樂者動於內者也,禮者動於外者也。夫禮、樂不外乎吾身之自動,而奚以求諸千載而上不可究詰之名物象數也乎?」其大旨蓋如此。他著有喪服會通說四卷,周易說十四卷,書說四卷,詩文集十二卷。與嘉賓同時而專力於學者,有劉傳瑩。

傳瑩,字椒云,漢陽人。道光十九年舉人,官國子監學正。始學考據,雜載於書冊之眉,旁求秘本鉤校,硃墨並下,達旦不休。其治輿地,以尺紙圖一行省所隸之地,墨圍界畫,僅若牛毛。晨起指誦曰:「此某縣也,於漢為某縣;此某府某州也,於漢為某郡國。」凡三四日而熟一紙,易他行省亦如之。久之疾作,不良食飲。自以所業者繁雜無當於心,乃發憤嘆曰:「凡吾之所為學者何為也哉!舍孝弟取與之不講,而旁鶩瑣瑣,不亦傎乎!」於是取濂、洛以下切己之說,以意時其離合而反復之。嘗語曾國籓曰:「君子之學務本,專而已。吾與子敝精神於讎校,費日力於文辭,僥幸於身後不知誰何者之譽。自今以往,可一切罷棄,各敦內行。沒齒無聞,誓不復悔。」卒,年三十一。病中為日記一編,痛自繩檢,遺令處分無憾。國籓嘗稱其「湛深而敦厚,非其視不視,非其聽不聽,內志外體一準於法,而所以擴充官骸之用,又將推極知識,博綜百氏,以求竟乎其量」。世以為知言。硃子所編孟子要略,自來志藝文者皆不著於錄。傳瑩始於金仁山孟子集注考證內搜出之,復還其舊。

劉熙載,字融齋,興化人。十歲喪父,哭踴如禮。道光二十四年進士,改庶吉士,授編修。咸豐二年,命直上書房。與大學士倭仁以操尚相友重,論學則有異同。倭仁宗程、硃,熙載則兼取陸、王,以慎獨主敬為宗,而不喜學蔀通辨以下掊擊已甚之談。文宗嘗問所養,對以閉戶讀書。御書「性靜情逸」四大字賜之。以病乞假,巡撫胡林翼特疏薦。同治三年,徵為國子監司業,遷詹事府左春坊左中允。督學廣東,作懲忿、窒欲、遷善、改過四箴訓士,謂士學聖賢,當先從事於此。所至蕭然如寒素,未滿任乞歸,襆被篋書而已。

熙載治經,無漢、宋門戶之見。其論格物,兼取鄭義。論毛詩古韻,不廢吳棫葉音。讀爾雅釋詁至「卬、吾、臺、予」,以為四字能攝一切之音。以推開齊合撮,無不如矢貫的。又論六書中較難知者莫如諧聲,疊韻雙聲,皆諧聲也。許叔重時雖未有疊韻雙聲之名,然河、可疊韻也;江、工雙聲也。孫炎以下切音,下一字為韻,取疊韻,上一字為母,取雙聲,蓋開自許氏。又作天元正負歌,以明加減乘除相消開方諸法。生平於六經子史及仙釋家言靡不通曉,而一以躬行為重。嘗戒學者曰:「真博必約,真約必博。」又曰:「才出於學,器出於養。」又曰:「學必盡人道而已。士人所處無論窮達,當以正人心、維世道為己任,不可自待菲薄。」平居嘗以「志士不忘在溝壑」、「遯世不見知而不悔」二語自勵。自少至老,未嘗作一妄語。表裏渾然,夷險一節。主講上海龍門書院十四年,以正學教弟子,有胡安定風。著持志塾言二卷,篤近切實,足為學者法程。光緒七年,卒,年六十九。又有藝概六卷,四音定切四卷,說文雙聲二卷,說文疊韻二卷,昨非集四卷。

硃次琦,字九江,南海人。道光二十七年進士,分發山西,攝襄陵縣事,引疾歸。

次琦生平論學,平實敦大。嘗論:「漢之學,鄭康成集之;宋之學,硃子集之。硃子又即漢學而精之者也。宋末以來,殺身成仁之士,遠軼前古,皆硃子力也。然而攻之者互起,有明姚江之學,以致良知為宗,則攻硃子以格物;乾隆中葉至於今日,天下之學,以考據為宗,則攻硃子以空疏。一硃子也,攻之者又矛盾。烏乎!古之言異學也,畔之於道外,而孔子之道隱;今之言漢學、宋學者咻之於道中,而孔子之道歧。果其修行讀書蘄之於古之實學,無漢學,無宋學也。」凡示生徒修行之實四:曰敦行孝弟,曰崇尚氣節,曰變化氣質,曰檢攝威儀;讀書之實五:曰經學,曰史學,曰掌故之學,曰性理之學,曰詞章之學。一時咸推為人倫師表云。

官襄陵時,縣有平水,與臨汾縣分溉田畝,居民爭利構獄,數年不決。次琦至,博詢訟端,則豪強壟斷居奇,有有水無地者,有有地無水者。有地無水者,向無買水券,予之地,弗予之水;有水無地者,向有買水券,雖無地得以市利。於是定以地隨糧,以水隨地之制。又會臨汾縣知縣躬親履畝,兩邑田相若,稅相直也。乃定平水為四十分,縣各取其半。復於境內設四綱維持之:曰水則,曰用人,曰行水,曰陡門。實行水田三萬四百畝有奇,邑人立碑頌之。系囚趙三不夌,劇盜也,越獄逃。次琦未抵任,先出重貲購知其所適。亟假郡捕,前半夕疾馳百二十里,至曲沃郭南以俟。盜眾方飲酒家,役前持之,忽樓上下百炬齊明,則赫然襄陵縣鐙也,乃伏地就縛。比縣人迎新尹,尹已尺組系原賊入矣,遠近以為神。每行縣,所至拊循姁姁,老稚迎笑。有遮訴者,索木椅在道與決,能引服則已,恆終日不笞一人。其他頒讀書日程,創保甲,追社倉二萬石,禁火葬,罪同姓婚,除狼患,卓卓多異政。在任百九十日,民俗大化。

先是南方盜起,北至揚州。次琦猶在襄陵,謂宜綢繆全晉,聯絡關、隴,為三難、五易、十可守、八可徵之策,大吏不能用。居家時稱說浦江鄭氏、江州陳氏諸義門,及朝廷捐產準旌之例。由是宗人捐產贍族,合金數萬。次琦呈請立案,為變通範氏義莊章程,設完課、祀先、養老、勸學、矜恤孤寡諸條,刊石世守之。

同治元年,與同邑徐臺英奉旨起用,次琦竟不出。光緒七年,賞五品卿銜,逾數月卒。著有國朝名臣言行錄、五史實徵錄、晉乘、國朝逸民傳、性學源流、蒙古聞見等書。疾革,盡焚之,僅存手輯硃氏傳芳集五卷,撰定南海九江硃氏家譜十二卷,大雅堂詩集一卷,燔餘集一卷,橐中集一卷。

成孺,原名蓉鏡,字芙卿,寶應人。附生。性至孝,父歿,三日哭,氣絕而復屬者再。授經養母,歲歉,粗糲或弗繼,母所御必精鑿。事母垂六十年,起居飲食之節,有禮經所未嘗言,而以積誠通之者。早邃經學,旁及象緯、輿地、聲韻、字詁,靡不貫徹。於金石審定尤精確。久之,寢饋儒先諸書,益有所得。取紫陽日用自警詩,以「味真腴」顏其居,自號曰心巢。

孺於漢、宋兩家,實事求是,不為門戶之見。嘗曰:「為己,則治宋學真儒也,治漢學亦真儒;為人,則治漢學偽儒也,治宋學亦偽儒。」又曰:「義理,論語所謂識大是也:考證,識小是也:莫不有聖人之道焉。事父事君,識大也;多識鳥獸草木之名,識小也:皆詩教所不廢,然不可無本末輕重之差。」湖南學政硃逌然延主校經堂,孺立學程,設「博文」、「約禮」兩齋,湘中士大夫爭自興於學。著有禹貢班義述三卷,據地志解禹貢,於今、古文之同異及鄭注與班偶殊者,一一辨證。即有不合,亦不曲護其非。尚書歷譜二卷,以殷歷校殷、周歷校周,從違以經為斷。又考太初歷即三統,為太初歷譜一卷,春秋日南至譜一卷。又有切韻表五卷,二百有六表,分二呼而經以四等,緯以三十六母,審辨音聲,不容出入。晚年著述,一以硃子為宗。所編我師錄、困勉記、必自錄、庸德錄、東山政教錄,又有國朝學案備忘錄一卷,國朝師儒論略一卷,經義駢枝四卷,五經算術二卷,步算釋例六卷,文錄九卷。

邵懿辰,字位西,仁和人。性峭直,能文章,以名節自厲。於近儒尤慕方苞、李光地之學。道光十一年舉人,授內閣中書。久官京師,因究悉朝章國故,與曾國籓、梅曾亮、硃次琦數輩游處,文益茂美。折節造請高才秀士,有不可,面折之。不為朋黨,志量恆在天下。洊升刑部員外郎,入直軍機處。大學士琦善以妄殺熟番下獄,發十九事難之。

粵亂作,賽尚阿出視師,復上書次輔祁俊藻,力言不可者七端。時承平久,京朝官率雍容養望,懿辰獨無媕阿之習,一切持古義相繩責。由是諸貴人憚之,思屏於外。會粵賊陷江寧,京師震動,乃命視山東河工,未行,復命偕少詹事王履謙巡防河口。咸豐四年,坐無效鐫職。既罷歸,則大覃思經籍,著尚書通義、禮經通論、孝經通論,頗採漢學考據家言,而要以大義為歸。

十年,賊陷杭州,以奉母先去獲免。母卒,既葬,返杭州。賊再至,則麾妻子出,獨留與巡撫王有齡登陴固守。十一年,城陷,死之。時國籓督師江南,聞而嘆曰:「嗟乎!賢者之處患難,親在,則出避;親歿,則死之:義之至衷者也。」乃迎致其妻子安慶。先是懿辰以協防杭州復原官,死事聞,贈道銜,祀本省昭忠祠。其所著書,遭亂亡佚,長孫章輯錄之,為半巖廬所著書,共三十餘卷。懿辰之友,同里伊樂堯、秀水高均儒,皆知名。

均儒,字伯平。廩貢生。性狷介,嚴取與之節。治三禮主鄭氏。尤服膺宋儒,見文士蕩行檢者則絕之如讎,人苦其難近。著續東軒集。

樂堯,字遇羹。咸豐元年舉人。學術宗尚與懿辰同。值寇亂,猶商證經義危城中。城破,同殉節死。


\end{pinyinscope}