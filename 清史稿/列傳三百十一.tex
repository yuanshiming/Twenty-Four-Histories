\article{列傳三百十一}

\begin{pinyinscope}
籓部七

○唐努烏梁海阿爾泰烏梁海阿爾泰淖爾烏梁海

唐努烏梁海,在烏里雅蘇臺之北,東南土謝圖汗部,南賽音諾顏部,西阿爾泰烏梁海,西南扎薩克圖汗部,北俄羅斯。有總管五:曰唐努,曰薩拉吉克,曰托錦,曰庫布蘇庫勒諾爾,曰奇木奇克河。

康熙五十四年,扎薩克圖汗部和托輝特輔國公博貝隨大軍赴推河防準噶爾策妄阿拉布坦,言:「準噶爾不靖,恃烏梁海障之。乞往招,若抗即以兵取。扎薩克臺吉濟納彌達阿里雅及根敦羅卜藏克兵俱習戰,請與同往。」上韙其議,從之。九月,烏梁海頭目和羅爾邁率屬降。先是和羅爾邁居吹河,嘗以越界射獵為博貝縛獻,上宥其罪,諭還巢。至是將遣子瑚洛處納請降。博貝至,因遷其游牧赴特斯。冬,和羅爾邁遁,博貝追至呼爾罕什巴爾,執之。五十九年,博貝擒烏梁海逃眾,晉貝勒。時從征西將軍祁里德軍。六十年六月,議政王大臣議覆祁里德,新收烏梁海二千五百三十名,應送至巴顏諾爾克地方居住,令車臣汗等旗分派兵三百名,並派臺吉協同駐扎防守。雍正二年,諭曰:「朕詢貝勒博貝,管轄烏梁海何以資生。據奏在將軍祁里德處借餉一萬八千餘兩,買牲分給,各得產業,今勝於昔。所有借項,自以貝勒俸逐年扣抵。朕思烏梁海俱朕之百姓,豈有朕之百姓而借餉於朕之理?所借銀兩,不必扣還。諭祁里德知之。」三年,烏梁海和羅爾邁復遁,由阿哩克竄準噶爾界,博貝遣子額璘沁由托濟邀擒,而自赴克木克木齊克緝叛黨,誅之。

初額魯特與喀爾喀構兵時,錯處科布多、烏蘭固木。噶爾丹既滅,喀爾喀西境直抵阿爾泰,自唐努山陰之克木克木齊克至博木等處,皆博貝及來歸之額魯特貝凌旺布所屬烏梁海游牧。四年,策旺阿拉布坦言克木克木齊克舊隸準噶爾,乞還,上不許,慮伺間略烏梁海,詔博貝率所部兵千,隨前鋒統領定壽駐唐努山陽特斯地方防護之。尋諭理籓院曰:「朕詳思克木克木齊克烏梁海皆博貝所屬,和羅爾邁既已就擒,交博貝撫恤,居之公所。但念此等人向在喀爾喀邊外林木中射獵為生,與準噶爾所屬烏梁海接壤,又與俄羅斯連界。宜令博貝等同大臣前往曉諭,令自為預備,以防不虞。」三月,命大臣一員帶布帛茶葉賞克木克木齊克地方烏梁海,並令揀老成服眾之人作為首領。

五年,額駙策凌等與俄羅斯訂約,自恰克圖、鄂爾輝圖兩間為界,所立之鄂博,迤西至肯哲馮達霍呢音嶺、克木克木齊克之博木、沙弼納嶺。循此山梁,由正中分中劃界,其兩邊各取五貂之烏梁海,仍令照舊各歸其主,彼此各徵一貂之烏梁海。自定界之日,將各取一貂之處停止。

乾隆十六年,以和托輝特扎薩克貝勒青袞咱卜縱所屬人私出汛界與準噶爾回眾貿易,致潛居烏梁海,奪貝勒,詔額璘沁襲其爵,定烏梁海出入汛界例。二十一年,青袞咱卜脅烏梁海叛,大兵至,皆棄去。二十五年,鑄唐努烏梁海總管印給之。嘉慶二年,烏里雅蘇臺參贊大臣額樂春以需索烏梁海奪職治罪。道光三年,定禁烏梁海與商民貿易例,以山西民人私向烏梁海買取羊只涉訟。二十四年六月,烏里雅蘇臺將軍桂輪劾總管垂敦扎布需索無厭,奪職。咸豐年,奏唐努烏梁海界址。

十年,與俄國定界約,是部之沙賓達巴哈實為西疆劃界之第一地段。同治三年十一月,烏里雅蘇臺將軍明誼等奏:「唐努烏梁海游牧內,俄使前開議單,載唐努鄂拉達巴哈即系唐努山嶺,自沙賓達巴哈界牌起,先往西,後往南。亦據該使呈繪圖志,有順薩彥山嶺至奎屯鄂博所有界限地名。我國舊存圖內雖無其地名,然據該使所指方向,續經庫倫辦事大臣文盛送雍正五年已定交界圖志,名目雖殊,界限大致相似。唐努烏梁海游牧雖有被俄人包去之嫌,與西二盟游牧無礙。明年立界時,俟與麟興、車林敦多布等妥商辦理。」四年八月,麟興等奏:「據委員岳嵩武稟報,與唐努烏梁海總管凡齊爾馳赴博果素克大壩履勘起,沿站按圖詳查,行至唐努鄂拉達巴哈,核與俄國所畫唐努鄂拉達巴哈邊界相符。除薩彥山因無路徑不能履勘,其唐努鄂拉達巴哈及邊境應分之珠嚕淖爾、塔斯啟勒山、哈喇塔蘇爾山、德布色克哈山數處,擇擬立界處所,繪會勘圖志呈閱。」時俄立界使臣以事不能至。九月,明誼等以軍務緊急,請緩約俄使立界。

六年,俄人遂越界至總管邁達爾游牧內烏克果勒地方建屋種地。總理各國事務衙門照會俄使,始由庫倫俄官行文令送之回國。是年,廷旨促麟興等建立西疆毗連俄境界牌鄂博。六月,專命榮全迅與俄官會立烏里雅蘇臺邊卡界牌鄂博。八年五月,榮全與俄使穆魯木策夫至是部西南之賽留格木山嶺會立牌博,於是月二十六日起行,順賽留格木嶺至是部西南邊境盡處之博果蘇克壩,立第一界牌鄂博,科城立牌博於南,俄國立牌博於西。由此向東北約八十里,名塔斯啟勒山,於山頂立第二牌博。又向東北約九十里,至珠嚕淖爾,俄使言只就珠嚕淖爾迤北數十里唐努山之察布齊雅壩止,建立鄂博,由此直向西北,統至沙賓達巴哈,路既便捷,尤易行走。榮全以俄使所指之路俱系是部游牧內地,若照俄使所議,不惟與原圖大不相符,且將是部游牧包去大半,向俄使反復開導,仍如原圖,於珠嚕淖爾東南之哈爾根山立第三牌博。順淖爾北岸約二十餘里,至唐努山南察布齊雅壩,立第四牌博。沿唐努山南,向西過莫多圖河、扎勒都倫河、烏爾圖河、察罕扎克蘇圖河,順哈喇塔蘇爾海山,至沙克魯河,轉向東北約二百五十餘里,至庫色爾壩,系是部西方邊界,立第五牌博。向西北九十餘里,至唐努鄂拉達巴哈末處,過哈喇河偏西山下楚拉察水流之處,立第六牌博。向北又東,順薩彥山過瑪納瑚河、蒙納克河、浩拉什河,由喀喇淖爾至蘇爾大壩,約一百五十里,立第七牌博。向北又東約三百六十餘里,山脈連貫,直至沙賓達巴哈,於舊牌博之東山頂上立第八牌博。照原圖至賽留格木山博果蘇克壩上,紅線以左為中國地,紅線以右為俄國地。至六月二十二日竣事,而是部阿爾泰河、阿穆哈河區域皆入於俄。

光緒五年,烏里雅蘇臺將軍以奇木齊克河總管報俄商在唐努烏梁海屬建蓋行棧數處,及春季以來,有俄人或三五十人或八九十人不等,在奇木齊克河北一帶中唐努山內刨挖金砂,例應禁止,咨總理各國事務衙門照會俄署使凱陽德轉飭邊官查禁。七年五月,烏里雅蘇臺將軍以俄人在薩爾魯克地方居住,扎立木棚十處,附近挖過金砂大小凡一百餘處,照會俄駐庫領事迅飭邊界官嚴禁。

十四年四月,烏里雅蘇臺將軍杜嘎爾奏稱:「所轄唐努烏梁海屬地邊外自柏郭蘇克西北至沙賓達巴罕,中國設立界牌,每年夏季派員會同查閱。其嶺一東一南,至烏里雅蘇臺,即嶺之左,歸中國屬,載在條約。乃俄人竟於沙賓達巴罕以東,霍呢章達巴罕以西,唐努所屬爾里黨、薩布塔爾、都不達果勒、車爾里克、荊格等河岸地方,前經查驗過俄人挖金共四十五處,至今仍在薩布塔爾、車爾里克兩處附近河岸開挖甚多。烏克、多倫兩河地方,俄人明固賴等任意開墾地畝,長一千三百餘廣尺,寬八百二十餘廣尺。俄人雅固爾等於薩拉塔木、博木、額奇布拉克、多倫、烏克、車爾里克、托勒博、薩斯多克、密崗嚕勒、扎庫勒、哈達努額奇依斯克、木阿瑪、阿克河口、吉爾噶琥河口、吉爾扎拉克等十五處建蓋堅屋,南入我境至數百之多。本年派佐領榮昌等往烏梁海吉爾拉里克地方會俄官辯論挖金、蓋房、種地各案,俄官一味支吾,執意不辦,應由總理各國事務衙門逐件查覆。」旋由總理事務衙門覆奏:「請飭將軍等詳勘界限,研究根由,援據約章,與俄酋竭力辯論。倘彼堅執,或應知照駐俄使臣,嚴請外部妥籌辦法,或即估給蓋房之費,令從速遷徙,由將軍等就近相機籌定,奏明辦理。」十月,祥麟等奏覆派吉玉等由烏梁海印務處於六月自廕木噶拉泰起程,履勘車爾里克等處,往返兩月有餘,已將俄人在境內挖金、蓋房、種地三事詳細查明,繕單入告。命總理各國事務衙門照會俄使,將越界在唐努烏梁海挖金、蓋房、種地之背約俄人遷回本國。

二十五年八月,烏里雅蘇臺參贊大臣志銳以奇木齊克河總管請給印奏入,命連順察看情形,奏明辦理。尋覆奏,以「奇木齊克河與唐努總管相隔實在千里之外,中間橫亙賽音諾顏部之額魯特扎薩克貝子達克丹多爾濟所屬烏梁海,遇有齟,文報不通,凡事轉報總管,未能直達烏城。奇木齊克河實有二千一十三戶,丁口已幾萬人。唐努總管每年勒派各情,亦所恆有。其他毗連俄界,交涉事多。既,十蘇木連結懇求,是與唐努總管其心已離,兩不相下,倘有事故,亦難收拾。若將數十年仰希朝廷之恩,一旦下頒,必能自固籓籬,為我屏蔽。況有東烏梁海請印在前,似難以不符體制為解,請仍賞給印信」。得旨,如所請。

二十六年,詔連順等備邊。時拳匪事起,中外人心惶惑。連順檄唐努烏梁海總管棍布多爾濟、薩拉吉克烏梁海總管巴勒錦呢瑪、托錦烏梁海總管凌魁、庫布蘇庫勒諾爾烏梁海總管克什克濟爾噶勒、奇木齊克河烏梁海總管海都布調兵練團,嚴密舉辦。棍布多爾濟等均能刻日成軍。復籌幫軍食,擇要加兵防守,善待俄商,毋生邊釁。二十八年十二月,連順等再請獎敘,疏入,予克什克濟爾噶勒二品頂戴,海都布二品花翎。是年,連順以「烏梁海向風沐化幾二百年,直與喀爾喀蒙古無異。我國商民仍守舊規,不敢違禁潛往貿易。至俄商之在烏梁海貿易者,不計其數,建蓋房屋,常年居住,每年收買鹿茸、狐、狼、水獺、猞猁猻、貂皮、灰鼠,為款甚鉅,致烏梁海來烏城呈交貢皮時,竟至無貨可以貿易。惟有變通辦理,如在烏城貿易商民原赴烏梁海貿易者,準即報官前往,仍由將軍衙門照章酌給六個月限票,並嚴飭守卡官兵認真稽查,不準挾帶違禁之物」。允之。

宣統元年,烏里雅蘇臺將軍堃岫等以奇木齊克河總管海都布率奏本旗十蘇木公揀海都布長子達魯噶布音巴達爾琥辦事勤能,眾心傾服,請補總管,允之。

是部天和土腴,有灌溉之利,宜麥。有金、銅、石棉諸礦,林木亦富。達布遜山產石鹽,是部全境及科布多北部皆資之。唐努、薩拉吉克、托錦三總管各有佐領四,庫布蘇庫勒諾爾總管佐領二,奇木齊克河總管佐領十。薩拉吉克別名薩爾吉格,托錦別名陶吉,總管皆無印。庫布蘇庫勒諾爾別名庫蘇古淖爾,奇木齊克河別名肯木次克,有印。此外扎薩克圖汗部右翼右旗有五佐領:一在庫蘇古爾泊北,一在華克穆河東北,一在格德勒爾河西,一在謨什克河西,一在扎庫爾河源。賽音諾顏部額魯特貝子旗佐領十三,皆南依鄂爾噶汗山,西接阿爾泰淖爾烏梁海。哲布尊丹巴呼圖克圖徒眾所屬佐領三,西臨華克穆河。

阿爾泰烏梁海,在科布多之西,東額魯特,東南扎哈沁及布勒罕河新土爾扈特、哈弼察克新和碩特,南和博克薩裏舊土爾扈特,東北杜爾伯特,北阿爾泰淖爾烏梁海。分左右翼,左翼旗四,右翼旗三。

初屬準噶爾。乾隆十八年,喀爾喀扎薩克圖汗等臺吉達什朋素克隨北路軍營參贊大臣薩喇爾擒私入科布多汛之烏梁海人扎木圖等。十九年正月,命薩喇爾等統兵征入卡之準噶爾屬烏梁海。釋北路軍營誘捕之烏梁海禡木特等,令回部落。二月,準噶爾烏梁海庫木來降。三月,命舒赫德赴卓克索地方會薩喇爾招撫烏梁海。尋以烏梁海徙牧額爾齊斯等地,令暫撤兵。是月,以收撫烏梁海,移北路軍營於烏里雅蘇臺。七月,賽音諾顏貝子車木楚克扎布暨班第、薩喇爾等擊烏梁海宰桑於察罕烏蘇,降之。十月,班第、薩喇爾進兵降阿爾泰居住之準噶爾烏梁海宰桑禡木特及通禡木特,收戶口千餘。復由阿爾泰赴索爾畢嶺,進至布爾漢之察漢托輝額貝和碩地方,獲宰桑庫克新等。十一月,以收撫烏梁海,加和托輝特貝勒青袞咱卜郡王銜,編設烏梁海人戶旗分佐領,諭授宰桑車根、赤倫、察達克總管,命庫克新於額爾齊斯屯田。

二十年正月,察達克等兵至華額爾齊斯河收獲包沁宰桑等。授察達克副都統,予烏梁海總管赤倫副都統銜,命招撫汗哈屯之烏梁海人眾。免烏梁海等貢賦一年。二月,編察達克、赤倫所屬烏梁海為佐領七。三月,烏梁海宰桑都塔齊以指示投順之人逃竄正法,命扎薩克圖汗部扎薩克臺吉根敦等駐防海喇圖、科布多等處,管烏梁海游牧,接收降人。四月,汗哈屯地方烏梁海歸順。五月,授歸順之烏梁海宰桑圖布新為總管。十月,以烏梁海出牲畜接濟哈達哈西進之軍,嘉賚之。二十一年三月,以阿逆煽動烏梁海,哈薩克道梗。詔哈達哈剿烏梁海叛賊。有固爾班和卓者,奇爾吉斯宰桑,攜千餘戶潛赴烏梁海,賽音諾顏郡王車布登扎布及車登三丕勒邀擒之。六月,青袞咱卜叛,誘新舊烏梁海附己。大兵至,皆來效順。十月,以新舊烏梁海等備兵請討青袞咱卜,嘉賚之,授察達克內大臣。

二十二年二月,命察達克等防範準部叛賊達什車凌等逃入烏梁海。四月,以額魯特叛賊車布登多爾濟屬人分給察達克等。論察達克等俘輝特賊人功,予其子侍衛賚圖布慎、赤倫、洪郭爾等緞茶各有差。九月,命車布登扎布等防範阿逆等擾烏梁海。十月,以阿爾泰淖爾烏梁海內附,諭授官加賞,定察達克所屬烏梁海每戶歲納二貂,給俸如內地官吏之半。十一月,命烏梁海、扎哈沁人等歸還馬駝。烏梁海博和勒復降,仍授總管。二十三年二月,歸並烏梁海管轄人戶編入之,允新舊烏梁海均於烏蘭固木種地,於吹河、勒和碩等處游牧。尋命移科布多烏梁海徙就阿爾泰山陽。二十四年三月,仍命郡王車布登扎布總理烏梁海事。八月,烏梁海副都統莫尼扎布等招降鄂爾楚克人戶,附入烏梁海大臣管轄,授官有差。是年,定阿爾泰山之南額爾齊斯為是部牧地。十二月,以哈薩克人掠烏梁海,諭察達克等防剿。二十五年四月,以收撫烏梁海原任總管阿喇逃散屬人交察達克等兼管。烏梁海總管扎布罕疏脫賊犯,上以年幼宥之,命察達克派員協同辦事。

二十六年七月,禁烏梁海私向哈薩克貿易。二十七年三月,允展烏梁海卡坐。九月,嚴禁阿爾泰烏梁海竊取哈薩克馬匹。十月,以前經內附續逃入俄羅斯之烏梁海庫克新假我烏梁海名劫掠哈薩克,命察達克等領兵捕治之。十二月,鑄烏梁海左、右翼總管印,分給察達克、圖克慎,銷原領阿爾泰烏梁海總管印。二十八年正月,庫克新就擒,戮之,以招撫人戶給察達克等分轄。三十八年十二月,以新土爾扈特郡王舍楞與是部散秩大臣烏爾圖那遜為婚,諭烏梁海緊接俄羅斯,瑚圖靈阿等嗣後詳為留意。四十九年六月,給阿爾泰臺站內大臣察達克轄烏梁海官兵協濟銀兩。

道光十八年,以哈薩克潛闌入阿爾泰烏梁海,命烏里雅蘇臺參贊大臣車林多爾濟領蒙兵逐之。科布多參贊大臣毓書遣科布多主事職銜哈楚暹領兵逐入烏梁海之哈薩克依滿等於烏里雅蘇臺。八月,追敗之於沙拉布拉克。九月,又逐再入烏梁海之哈薩克,使過於庫克伸阿林,予獎。十一月,車林多爾濟奏前入烏梁海土爾扈特之哈薩克驅逐已凈,獲十餘人釋之。十二月,予烏梁海副都統車伯克達什等花翎,以論驅逐潛入游牧哈薩克勞。十九年四月,哈薩克復入烏梁海,命車林多爾濟復調兵逐之。八月,以阿爾泰烏梁海右翼散秩大臣達什濟克巴調營未到,嚴議。予驅逐哈薩克妥速之阿爾泰左翼散秩大臣達爾瑪阿扎拉頭品頂戴,仍下部優敘。二十二年,科布多參贊大臣固慶奏:「達爾瑪阿扎拉時常稱疾偷安,不善撫馭。所任散秩大臣管烏梁海四旗事務煩,游牧遼闊,且與俄羅斯接壤,責任綦重,請令離任,以參領唐嘎祿署之。」

咸豐十年,與俄羅斯定西疆界約。同治三年八月,科布多參贊大臣廣鳳等奏:「卡倫以內阿爾泰烏梁海境內奇林河等地方十七處,有哈薩克公阿吉屬下之哈濟克居住。當分界未終之際,未便一旦驅逐。倘分界後,萬不得已必須內遷,宜由塔爾巴哈臺參贊大臣酌擇地方安置。」十一月,俄人闌入是部庫什業莫多及塔布圖地方滋擾。明誼照會俄悉畢爾總督,先為查辦來我邊卡滋事官兵,俟明年兩國立界大臣會同建立牌博後,再派兵駐守。四年,以伊、塔諸城回變,命設烏梁海臺站,遞送科城至塔城文報軍餉。十二月,塔爾巴哈臺參贊大臣錫霖劾廣鳳裁撤烏梁海臺站,致文報軍餉阻滯。諭廣鳳等議處,仍令復設。五年五月,塔爾巴哈臺失守,領隊大臣圖庫爾領額魯特兵移至是部。

七年三月,命奎昌會同俄官建立科布多毗連俄境界牌鄂博。九月,奎昌等以俄使未到,奏俟明年會辦立界。八年,奎昌與俄立界使臣巴布闊福勘明自科布多東北邊界賽留格木山適中之布果素克達巴哈起,向西南順賽留格木山至奎屯鄂拉,往西沿大阿爾泰山至海留圖兩河之山;轉往南,順是山直至察奇勒莫斯鄂拉;轉往東南,沿齋桑淖爾之邊,循喀喇額爾齊斯河岸,至瑪尼圖噶圖勒幹卡倫,分為兩國交界。建牌博凡二十:首曰布果素克達巴哈,次曰杜爾伯特達巴哈,曰塔布圖達巴哈,曰博勒齊爾,曰察幹布爾哈蘇,曰烏蘭達巴哈,曰巴哈那斯達巴哈,曰薩爾那開,曰巴爾哈斯達巴哈,曰拜巴爾塔達巴哈,曰庫爾楚木,曰特勒克梯,曰固洛木拜,曰薩拉陶,曰薩勒欽車庫,曰特勒斯愛哩克,曰鄂里雅布拉克,曰奇音克裡什,曰察奇勒莫斯,末曰瑪呢圖噶圖勒幹。自五月二十五日至七月三日竣事。十月,命棍噶扎拉參赴阿爾泰山收集徒眾,妥辦安插事宜,並免是部本年例貢貂皮。其後伊犁索倫營兵移至阿爾泰山,與塔城額魯特兵皆由棍噶扎拉參暫統之。十年,署伊犁將軍榮全奏,以由科布多屬扎哈沁五臺以西至霍博克薩里一、二千里,非就地設臺,後路必斷。令烏梁海章蓋等於西翼設察罕通格、托克鄂博、德格圖阿滿三臺,於東翼設多魯圖阿滿、額爾奇賽罕、烏里雅斯三臺。自是為科、塔兩城孔道。十一年,調棍噶扎拉參所部索倫、額魯特兵赴塔城。

十二年十一月,回匪竄新土爾扈特之布爾根河,擾是部境,臺站官兵紛紛逃散。烏魯木齊領隊大臣錫綸率所部民勇自阿爾泰山南移營烏龍古河南岸,追至霍博克河下游,擊破之。匪竄綏來縣北境,科布多參贊大臣保英等飭烏梁海速將軍臺移回原處安設。

光緒七年七月,以棍噶扎拉參在烏梁海達彥地方收撫哈薩克,擅殺頭目柯伯史之子,諭錫綸飭棍噶扎拉參即回籍。八年,俄人議重劃科、塔中俄之界,欲占哈巴河一帶。科布多參贊大臣清泰等奏:「俄人數百名突至哈巴河。查新條約內,奎峒山即阿爾泰山。任其勘改,實有關礙。」八月,阿爾泰左翼散秩大臣等復呈清泰等以「前次界劃烏梁海西北境侵占已多,此次若再占哈巴河,蒙民無地自容,誓死不能退讓」。諭清安、額爾慶額會商金順、升泰妥籌。九年,額爾慶額偕參贊大臣升泰先期馳赴塞上,察邊塞沖要,辨山川主名。以棄哈巴河、奎峒山二要地烏梁海、哈薩克之眾均無所依,與俄官抗爭,相持兼旬,改以哈巴河以西阿拉喀別河為界,得展地百三十餘里,分道安設新界牌博。既竣事,額爾慶額又繞北山道大彥淖爾安插烏梁海兩翼部落,以和裏木圖河、雅瑪圖、喲洛圖、西里布拉克為夏季游牧,以罕達蓋圖河、塔裏雅圖、青格里河、烏龍古河為冬季游牧,而哈巴河仍由塔城置戍。以金順奏,諭阿爾泰山烏梁海屬一帶游牧地方,請飭棍噶扎拉參交回安插蒙民。十二年七月,以沙克都爾扎布等奏,復催棍噶扎拉參將徒眾仍回塔城。十三年,諭劉錦棠等於新疆擇安插棍噶扎拉參之地。十五年二月,劉錦棠奏移棍噶扎拉參徒眾於庫爾喀喇烏蘇屬之八英溝,讓還科布多借地。承化寺就近所招徒眾,聽留居其寺哈巴河一帶。塔城自借地以來,即已派兵駐守,未便委去,俾俄人得乘便南下,從之。十八年六月,沙克都爾扎布、額爾慶額、魁福會勘,奏哈巴河借地暫難歸還,以塔城兩次分界後,蒙、哈不敷分住,請將借地展緩三年交割。烏梁海困苦,擬令塔城哈薩克酌給牲畜,並籌安插逃戶,派兵駐守,允之。其後科城屢請收回哈巴河,塔城爭之,迄未決。

二十六年,邊防戒嚴,參贊大臣瑞洵檄烏梁海每旗挑兵二百名,半馬牛步,駐防本旗。事定,撤之。以烏梁海各旗保護俄商貨物,安全游牧,一再請獎。二十九年閏五月,予烏梁海左右翼散秩大臣額爾克、舒諾三音博勒克均頭品頂戴,左翼總管倭齊爾扎布、桑敦扎布、右翼總管棍布扎布、瓦齊爾扎布均二品頂戴,左翼副都統察罕博勒克亦予獎。二十九年,塔城以哈巴城地交還科城。三十年五月,改設科布多辦事大臣駐阿爾泰山,以錫恆為之,仍駐承化寺。三十二年七月,定阿爾泰練陸軍馬隊一標、砲隊一營,設哈巴河防營委員,及沙扎蓋臺至承化寺馬撥十六處,每處設蒙古馬兵五名,馬十匹。開辦承化寺、庫克、呼布克木、哈巴河四處屯牧,建城署房屋,撥常年經費十三萬兩,開辦經費三十一萬兩有奇。十二月,是部七旗劃隸阿爾泰。三十四年四月,錫恆奏停辦布倫托海上渠,下渠距水較近,擬再試種一年,克木奇官屯暫撥民辦。宣統三年二月,署辦事大臣延年奏開距承化寺七十里之紅墩渠,安插農民。下部知之。

地兼耕牧,有礦,有鹽。是部有佐領七,副都統暨左右翼散秩大臣均兼一旗總管。卡倫自再劃界後,南起右翼散秩大臣旗之阿拉克別克,而北曰阿克塔斯,又東北曰克雜那阿斯,曰薩斯,曰呼吉爾圖布拉克,曰烏松呼吉爾圖,轉東曰胡布蘇,訖羅蓋布,東北至左翼散秩大臣旗之霍洞淖爾止,凡八卡倫。山之著者:西吉克圖山、荄拉圖山、霍穆達山、哈喇溫爾常山。水之著者:察罕西魯河、薩格賽河、薩克布多河、青格里河、額爾齊斯河。

阿爾泰淖爾烏梁海,在科布多之西北,東唐努烏梁海,南阿爾泰烏梁海,西與北均俄羅斯。

初屬準噶爾。乾隆二十二年九月,賽音諾顏扎薩克貝勒車木楚克札布招撫阿爾泰山烏梁海。有特勒伯克扎爾納克者,阿爾泰淖爾之烏梁海宰桑,攜屬至。詔車木楚克扎布定貢賞例,宣示德意。十二月,授阿爾泰淖爾烏梁海宰桑特勒伯克等為總管。二十三年秋,烏梁海總管阿拉善、恩克等叛,車木楚克扎布剿阿拉善等,就擒。恩兄竄哈屯河,冬,擒之。尋定是部為二旗,各設總管一,歲貢貉皮如例,隸科布多參贊大臣。道光中,查邊之政漸弛,俄人始築城於是部之吹河,我查邊界鄂博者往往不至其地。

咸豐十年,定西疆界約,俄國畫界清單遂將是部包去。同治六年七月,科布多參贊大臣廣鳳等奏俄雅什達喇城衙門給阿爾泰淖爾兩旗總管文,言阿爾泰淖爾、綽羅什拜、巴什庫斯、吹河均系俄羅斯游牧。如有人言系中國游牧,拿送俄城。又俄人來綽羅什拜地方伐木,已飭總管察罕等善言開導,靜候兩國分界大臣將疆界議定換約,立界後,再按照所分界限遵行,此時不可伐木蓋房,致滋事端。時俄國官兵執去我查閱哈屯河扎薩克扎那扎布及臺吉差官、蒙古員兵等,阻我查邊道路,稱是部游牧為俄國地面,不許中國人往來。十月,阿爾泰淖爾總管莽泰等報俄官取莽泰旗下一百四十餘人及總管察罕旗下二百四十餘人手印。九月,明誼、錫霖、博勒果索與俄分界大臣照議單勘分西界,是部地遂非清有。初議遷是部誠心內附者於卡內,而總管莽泰等言兩旗人丁祈全入卡內住牧。廣鳳等諭以「所被俄國分去地面舊住人丁,隨地歸為俄國,務令安居故土,各守舊業,立界後斷不致仍前擾害」。隨令莽岱等出卡回牧,並內徙之議亦輟。

是部有佐領四。


\end{pinyinscope}