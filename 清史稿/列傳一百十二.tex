\article{列傳一百十二}

\begin{pinyinscope}
李清時姚立德李宏子奉翰孫亨特何煟子裕城吳嗣爵

薩載蘭第錫韓鑅

李清時,字授侯,福建安溪人,大學士光地從孫。乾隆七年進士,選庶吉士,授編修。十四年,授浙江嘉興知府。上南巡,或議自嘉興至杭州別闢道行民舟,清時於官塘外求得水道相屬,上通吳江平望,下達杭州壩子門,號為副河。丁父憂,去官。服除,授山東兗州知府。二十二年,擢運河道。

二十六年,河決孫家集,運河由夏鎮至南陽兩堤俱潰,清時督修築。議者或擬用椿埽,費以六十萬計;或擬建石堤,費以三百萬計。清時少時行瀕海間,見築堤捍海為田者,擲碎石積水中,潮退則以木攔之,填土其上,堅築成堤;因參用其法,以河東、西兩岸皆水,得土難,令以石壘兩旁,積葑其中,水涸,募夫起土置積葑上,費帑十四萬有奇,而兩堤成。曹縣溢,水瀉入微山湖,出韓莊湖口,閘隘,水不得洩,令於閘北毀石堤,掘地深之以洩水。事上聞,上命於其地建滾水壩,高一丈二尺餘。清時請減低為一丈,令湖水落至丈,乃閉閘蓄水。泗水經兗州西流入府河,濟寧城東舊有楊家壩,遏水使入馬場湖,蓄以濟運,遇伏秋水漲不能洩,淹民田,令改壩為閘,視水盛衰為啟閉。汶水分流入蜀山、馬踏兩湖,舊制引水使南行少北行多,後乃反之,漕船經袁口、靳口,淺澀不能進。清時規分水口,令南壩加長,北壩收短,以為節宣,並減低何家壩,使汶水南弱而北增。蜀山湖出口為利運、金線二閘,舊制開金線資南運。清時令移金線在利運北,使蜀山湖水先濟北運。壽張境有沙、趙二水,阻運河不得入海。舊於運河東岸建三空五孔橋,又於八里廟建平水三徬,使二水盛漲有所洩。清時議減低三空五孔橋,又於八里廟增建滾水壩,使漲未盛即洩,不為範、濮、壽張、東阿諸縣民田害。總督方觀承行河,用其議,二水始宣暢。衛水自館陶至臨清與汶會,舊有閘,盛漲不能御。清時令於閘南當汶、衛交流處築壩,仍歲加高厚;又議拓四女寺滾水壩。尚書裘曰修行河,用其議,衛河得安流。

二十九年,調江南淮徐道。三十年,擢河東河道總督,賜其母大緞、貂皮。清時以河堤歲修,司其事者每不度形勢,過高糜帑,而卑薄者不能大有增益,乃飭所司當水漲各具堤高水面尺寸呈報,擇堤最薄者培之。迨伏秋水發,耿家寨稱十四堡,水及舊堤上,賴豫增新築以免。清釐河工徵料諸弊,歲減派料至千餘萬斤。三十一年,運河東岸漫口,自請議處,原之。三十二年七月,授山東巡撫。高苑、博興、樂安三縣被水,清時謂小清河下流隘,故上游溢,檄所司勘驗。遽疾作,乞解任,不許。三十三年,卒。

清時治水善相度情形,窮源竟委。每乘小舟出入荒陂叢澤、支流斷港中,或徒步按行諮訪,必得要領,乃見諸建置。

姚立德,字次功,浙江仁和人。祖三辰,官吏部侍郎。立德以廕生授主事。乾隆十二年,外授江寧通判,遷知直隸景州。州俗,有人市鬻奴婢,牽就牙儈估其值,如牲畜然;親死三日,祭城隍廟獄曰「哭廟」:立德諭禁之,陋俗以革。累遷山東按察使,署河東河道總督。按行工次,見陽武汛十七堡諸地土松浮,疏請築半戧,培堤使堅。山東運河兩岸蜀山、南旺、馬場、昭陽、微山諸湖,每伏秋盛漲,水不能容,為豫籌蓄洩,壩開塞、閘啟閉惟其時。三十九年,實授,加兵部尚書銜。高雲龍者,內監高雲從弟也,立德入雲從言,薦之臨清州為傔從,坐逮,依結交近侍律論斬,命奪官,仍留任。陽穀民王倫為亂,立德分守東昌,城圮難守,引運河水繞城壕,恃以為固;檄發倫先墓,磔其尸。四十四年,儀封河決,屢築屢沖,命奪官,仍留工效力自贖。四十五年,責令回籍。旋發往南河,補淮安里河同知。四十八年,卒。

李宏,字濟夫,漢軍正藍旗人。監生,入貲授州同。效力河工,授山陽縣外河縣丞。累遷宿虹同知。乾隆十六年,授河庫道。尚書劉統勛劾河員虧帑,事連宏,解職。事白,留工。二十二年,發直隸以河務同知用,總督尹繼善疏請留南河。侍郎夢麟勘治六塘河以下,以宏從。尋復補河庫道,丁父憂,命在任守制。二十七年,調淮徐道。二十九年,擢河東河道總督。奏言:「山東運河資湖水接濟。今秋雨少,飭早閉臨運各閘。」又言:「微山湖蓄水濟運。韓莊湖口閘水深,與滾水壩脊相平,空船足敷浮送,即應堵閉。泗河會合諸泉,收入獨山湖,僅濟南運。應請於兗州府金口壩截築土偃,俾達馬場湖,俾濟寧上、下河道並資其益。蜀山、馬踏二湖專濟北運,亦須築壩收蓄。」又請增募夫役挑濬沙、趙、漳、衛、汶、泗、韓、馬諸水,均報聞。又奏:「黃河北岸耿家寨埽工為豫東第一險要,自乾隆九年下埽修防,歲費帑料。去冬於對岸引渠,冀分溜勢。今秋全河暢分入渠,險工淤閉。」得旨嘉獎。

三十年,調江南河道總督。上以宏初自監司擢用,道以下多同官,慮有瞻徇,命高晉統理南河,留宏協理河東總河。奏言:「黃河至河南武陟、滎澤始有堤防,丹、沁二水自武陟木欒店匯入,伊、洛、瀍、澗四水自鞏縣洛口匯入,設諸水並漲,兩岸節節均須防守。臣咨飭陜州於黃河出口處,鞏縣於伊、洛、瀍、澗入河處,黃沁同知於沁水入河處,各立水志,自桃汛迄霜降,長落尺寸,逐日登記具報;如遇陡漲,飛報江南總河,嚴督修防。大丹河至河內縣丹谷口,舊築攔河石壩,令由小丹河歸衛濟運,請不時察驗疏令暢達衛河。輝縣百泉為衛河之源,蘇門山下匯為巨浸。南建三斗門,中為官渠濟運,東西為民渠灌田。向例重運抵臨清,閉民渠,使泉流盡入官渠。五月後插秧,一日濟運,一日灌田。惟民渠石壩失修,泉水旁洩,應令修砌堅實。」均如議行。上以清口節宣未暢,下河田廬易湮,特定高堰五壩水志水高一尺,清口壩拆展十丈。三十一年三月,宏奏言:「清口水門因上年霜降後湖水大消,祗留十四丈。桃汛將屆,應預將東壩拆展,使口門寬二十丈,俾洪湖及早騰空,預留容納之地。」上嘉之。夏秋間湖水盛漲,續展至五十三丈。八月,河溢徐州韓家塘。宏與高晉分駐兩壩堵築,逾月工竟。奏言:「平時大展清口,騰空湖面,乃得蕆工迅速。」冬,以湖水漸落,請接築東、西壩,仍留口門二十丈,酌量收束,蓄清抵黃。三十三年,河溢王家田頭,下吏議降調,寬之。三十四年,奏言:「洪澤湖水大,將清口東、西壩遞展宣洩。適黃水驟長,灌入清口。隨閉惠濟、通濟、福興三閘,俾並力敵黃,黃水消退。」報聞。三十六年,卒。

宏嘗以明汶上老人白英立祠戴村,子孫向有廕襲,請旨仍給八品世職,上從之。

李奉翰,宏子。入貲授縣丞,補沂水。累遷江蘇蘇松太道,坐事罷。復入貲還原官,發江南河工效力,奏署河庫道。上以奉翰宏子,習河事,命真除。四十四年,署江南河道總督。四十五年二月,授河東河道總督。河溢考城芝麻莊、張家油房,奉翰督吏塞芝麻莊,工竟。上諭曰:「勉為之,莫以水弱而弛其敬謹!」旋命仍署江南河道總督。奉翰奏:「張家油房工未竟,較南河睢寧工為要。請留河東,俾蕆其役。」報可。九月,張家油房工亦竟,上為欣慰。四十六正月,調江南河道總督。二月,奏請重定南河汛員額缺,酌增河兵;移改運河閘官、運河汛員,視缺簡要,更定品秩,下大學士九卿議行。七月,河決青龍岡,命偕大學士阿桂馳赴河南會河東河道總督韓鑅督辦東、西兩壩下埽。甫合龍,壩蟄陷,乃與阿桂等議寬濬青龍岡迤下至孔家莊、榮華寺、楊家堂諸地引河,並於黃河下游北岸疏潘家屯、張家莊二引河、蘇家山水線河、宿遷十字河、桃源顧家莊引河,五道洩水。四十八年春,青龍岡工竟。方壩陷,奉翰督吏搶護,墮入金門,格於纜,傷焉,河工謂兩壩間為金門,纜所以引埽者,事聞上。四十九年,上南巡,奉翰覲行在,上獎其勤勞,賜騎都尉世職。五十年,坐清口東、西兩壩不早收束,致運道淺阻,降三品頂帶。尋命復之。秋,河水大至,奉翰督吏晝夜填築,塞李家莊、煙墩頭、司家莊、湯家莊諸漫口。五十四年,調河東河道總督。五十八年,命赴浙江會巡撫吉慶會勘海塘。奏請以範公塘及海寧石壩改築柴盤頭,並於石塘前修補坦水,三官塘柴工後加培土戧,從之。五十九年,漳水溢,臨漳三臺漲發,命馳往勘察。奏:「漳河兩岸沙土浮松,水勢驟長驟落,向無堤堰。上年大雨漫溢,應將下游淤墊處疏濬深通,再將三臺壩基填築,俾歸故道。」上從其議。嘉慶二年正月,加太子太保,授兩江總督,兼領南河事。三年,河決睢寧。四年正月,與河道總督康基田督塞睢州決口,工竟。二月,卒。

李亨特,奉翰次子。入貲授布政司理問,發河東委用,補兗州通判。累遷雲南迤西道。嘉慶初,佐平苗、惈,賜孔雀翎,加按察使銜。累遷調授江蘇按察使。九年,擢河東河道總督。十一年,河南巡撫馬慧裕劾亨特索屬吏賕不得,迫令告養諸狀,上命侍郎托津等往按,奪官,發伊犁。十三年,釋還,令至南河候差委。十四年,以河決荷花塘,追咎亨特不善料理,復發熱河效力。未幾,復釋還,授主事。十五年,選戶部主事,擢直隸永定河道。未幾,復授河東河道總督。十六年,奏南糧到通州剝運不能迅速,請在楊村全數起剝,下倉場侍郎玉寧、戴均元等議駁。上責亨特冒昧,下吏議降調,命留任。十八年秋,河溢睢寧。坐奪官,命留工效力。十九年,河道總督吳璥奏微山湖存水僅一二尺,南陽、昭陽、獨山諸湖淤成平陸,無水可導。上責亨特在官不能預籌,又聞亨特既奪官居濟寧,仍用總河儀制,斥亨特玩誤縱恣,命逮下刑部治罪,籍其家,刑部議發新疆。上命在部荷校半年,發黑龍江效力。二十年,卒於戍所。

何煟,字謙之,浙江山陰人,先世籍湖南靖州。雍正中,入貲授州同,效力江南河工。從大學士河道總督嵇曾筠修浙江尖山海塘,請補杭州東塘同知,避本籍,仍發江南河工。乾隆初,權豐碭通判,授桃源同知。十五年,擢河庫道。十六年,遷兩淮鹽運使,特敕兼管河務,以母憂去官。十九年,尚書劉統勛等奏論河庫帑項不清,奪煟官,擬徒,追償,拘留工次,久乃繳完免罪。二十二年,仍發南河以同知用。從侍郎夢麟疏濬荊山橋河工。從副總河嵇璜治淮、揚河務,超擢淮揚道。二十三年,丁父憂,總督尹繼善奏留在任守制,許之。

二十六年,以郎中內調。會河決中牟楊橋,上命大學士劉統勛等蒞工,以胃從。工竟,留煟駐工防護。旋授開歸陳許道,調山東運河道。三十年,調河南河北道,擢按察使。上以煟習河事,命兼領河工。煟信浮屠說,讞獄輒從輕比,睢州民劉玉樹謀殺人,鞫實,擬斬候,刑部改立決。上責煟寬縱,諮巡撫阿思哈,阿思哈稱其能勝任。其冬,擢布政使,仍兼理河務。兩權巡撫。三十六年,授巡撫,兼河務如故。尋又命兼領山東河道。三十七年,淅川、內鄉被水,正陽、確山風災,疏請撫恤緩徵,上賜詩,褒以「愛民知政」。

三十八年,上巡天津,閱永定河工,煟迎駕,賜孔雀翎、黃馬褂。尋命與工部尚書裘

曰修、直隸總督周元理勘永定河上游,疏言:「永定河挾沙而行,散漫無定。水性就下,本無不同;而地有高卑,沙有通塞,情因時而或異。永定河遷徙不定,其情也,非其性也。察其情,導其性,先宣後防,千古極則,雖起神禹,無以易之。永定河下口,蒙皇上指示疏導,既不阻下達之勢,更可免浸潤之虞,其法固當常守。所慮數十年後,妄生異論,別騁新奇,勢且變亂舊章,貽河防巨患。請將聖諭並議言條款勒碑垂久遠。」報聞。

三十九年,疏請各州縣常平倉溢額以四千石為限,餘循例變價。又奏河南漕穀七十九萬、薊米二十九萬,分存各州縣界。鄰省安陽等五州縣限二萬石、近水次祥符等三十五州縣限一萬石。均如所擬。加總督銜,仍領河南巡撫,又進兵部尚書銜。其秋,會剿王倫,事平,道內黃,病作。遣醫往視,未至,卒。煟贈太子太保,祀賢良祠,賜祭葬,謚恭惠。

裕城,煟子,字福天。自貢生入貲授道員。乾隆四十二年,除山東督糧道。調河南河北道。河溢儀封,大學士高晉工,以裕城從。儀封埽工蟄陷,坐奪官,命留任。四十六年,調江南河庫道。裕城侍煟治河,嘗著全河指要,謂:「治河當節宣並用,不當泥河不兩行之說,偏於節束。」並上書當事,指陳南北岸諸險工。未幾,河決青龍岡,注微山湖,沖運河。四十七年七月,河東河道總督韓鑅丁憂,青龍岡工未竟,上特命裕城署理。大學士阿桂視工曲家樓,請自蘭陽至商丘別築新堤。裕城奏:「蘭陽新開引河,其上游素稱險要,必須內

有重障,外有挑護。大堤後舊有越堤,相去遠,恐不足恃。請向東添築格堤,臨河近溜處加築挑水壩。」上從之。又奏兗州伊家河在運河八閘之西,以分洩運河及瀕湖諸水,應挑展寬深,上命速興工。又奏伊家河興工後,即往河南勘驗引水子溝;仍往來山東、河南督察:上嘉之,並諭曰:「汝若能不自滿而加以勤學,或可繼汝父也。」伊家河工竟,四十八年,賜孔雀翎。是年,青龍岡工竟,請修築運河堤岸,詣濟寧勘估,奏需帑六十四萬有奇,得旨允行。授河南巡撫。以秋審多失出,降三品頂帶,停支養廉。四十九年,運河堤岸工竟,命議敘。師討石峰堡亂回,道河南,裕城佐軍興,復頂帶、養廉。五十年,調陜西巡撫。朝邑被水,上諭裕城就被水處將淤積泥沙建築河堤。尋奏創建護城是,下部議行。調江西巡撫,五十二年,奏江西河路二千四百餘里,請以所獲盜舟改設巡船,上嘉之。又奏豐城鎮平堤中段水勢沖激,不足捍禦,請改建石是,從之。五十五年,調安徽巡撫。命來京祝八旬萬壽,行次合肥,卒。

吳嗣爵,字樹屏,浙江錢塘人。八歲而孤,母錢督之嚴,雍正八年成進士。授禮部主事,大學士張廷玉奏改吏部。再遷郎中。嗣爵彊識,嫺故事。乾隆六年,授常州知府,再授保寧,皆奏留部。旋命視學湖北,調福建。十三年,授淮安知府,遷淮揚道。洪澤湖盛漲,例當開天然壩。嗣爵曰:「開壩減暴漲,如下河州縣生靈何?」持之力,卒無恙。十六年,調兩淮鹽運使。十八年,復授淮揚道,遭母憂,上諭曰:「防河官吏叢弊,故特由運使調用。河工與地方官吏不同,畀假兩月治喪,畢,在任守制。」

擢江蘇按察使。遷布政使,調湖南,未行,奏江寧等三十五州縣積欠應徵口糧,請特旨緩徵。上諮巡撫託恩多,託恩多奏江寧等州縣年豐,不當再請緩徵。上責嗣爵藉緩徵卸過,並為有司催徵不力地,命發江南河工,以同知用。二十五年,補宿虹同知,仍授淮揚道,移淮徐道。黃河盛漲,逼徐家莊縷堤,嗣爵督吏搶護,命署理河東河道總督。旋坐官運使時商人侵蝕提引公費,坐降調,命改奪官,仍留任。三十四年,奏請修補丁廟、六里、南旺、荊門、戴村諸閘壩,並言:「運河兩岸土工,臨清以北為民堰,南旺以南為官堤,自臨清至南旺,官堤、民堰交錯。請凡民堰卑薄殘缺處,督令修築,官堤酌緩急次第培修。」上嘉之。署河南巡撫。三十五年,奏:「南旺湖北高南下,在運河西岸,值分水口之沖。伏秋汶水發,自關家、常鳴等斗門灌入,祗能收水入湖,不能出水濟運。請於南旺下游土地廟前增建石閘一,以時啟閉。」

三十六年,遷江南河道總督。四十年,奏:「丁家集黃河自北趨南,北岸新灘插入河心,致沖漫南岸民堰五百餘丈。毛城鋪過水較大,下流亦不能容。今收正河頭,測量河脣,濬

引渠,築子壩,於北岸旁黃河故道濬引河,來春相機開放,俾河改由北岸東下,不使旁注丁家集諸地。」又奏:「里河運口本設惠濟、通濟、福興三閘,惠濟尤為淮水入運關鍵,請俟春融修築。」四十一年,又奏清口通湖引河凡五,為洪澤湖尾閭,並分別籌濬,運道以濟。尋奏五引河中張家莊、裴家場二河水水曳,應濬使寬深,從之。是年,上東巡,嗣爵覲行在,入對,不能興,左右掖以出。改吏部侍郎,四十二年,乞罷,歸。四十四年,卒,年七十有三。子璥,自有傳。

薩載,伊爾根覺羅氏,滿洲正黃旗人。父薩哈岱,官鑲藍旗滿洲副都統。薩載,繙譯舉人,授理籓院筆帖式。累遷江蘇蘇松太道,管蘇州織造。果親王弘適短價令制繡緞朝衣,事發,奪官。召還京,予主事銜。尋授薩哈岱蘇州織造,命薩載侍行為佐。逾年,改授普福,命交兩江總督差委。旋授松江知府。乾隆三十年,加道銜,復署蘇州織造。三十四年,擢江蘇布政使,仍兼織造。三十五年,署巡撫。巡撫永德請以華亭、寶山土塘改建條石,薩載言條石易傾圮。按察使吳壇請裁巡檢弓兵,增州縣捕役,薩載言不便,皆寢其議。三十六年,與總督高晉奏濬海州河道,又奏江蘇社穀積至三十七萬六千餘石,請察驗,報聞。

三十七年,真除江蘇巡撫。上命察屯田,薩載奏江安糧道屬江淮、興武等六衛,蘇州糧道屬蘇州、太倉等四衛,令清釐冊報,循新例四年一編審;加給江淮、興武二衛屯丁墾田,運丁快丁終歲輓輸,請加給津貼;太倉、鎮海二衛田不隨船,私相售典,循舊例借項贖回;從之。三十九年,河溢外河老壩口,偕河道總督吳嗣爵董工事,未兩旬工竟,議敘。

四十一年,上東巡,覲行在,授江南河道總督。命與高晉察黃河海口淤沙。薩載先至,奏:「海口前在王家港,自雍正時接湧淤灘,長四十餘里;南岸為新淤尖、為尖頭洋,北岸為二泓、三泓、四泓。二泓、四泓寬二十餘丈,潮至深二三丈;三泓寬四十餘丈,潮至深三四丈。河底有高低,河脣又漸遠,淤積巳久,難以施工。」上諭曰:「此海口自然之勢,難以人力勝之。」尋與高晉奏請以清口東、西壩移建平城臺,於陶莊迤上別開引河。是夏,運河及駱馬湖水漲,薩載督吏防護,上嘉其妥協。尋開陶莊引河,四十二年二月,工竟。上諭曰:「朕屢次南巡,臨閱清、黃交匯處,慮其倒灌,思引向陶莊北流。歷任河臣未有能任此者。昨歲薩載奏請施工,與朕意合。據奏工竟,自此黃河離清口較遠,既免黃河倒灌之虞,並收清水刷沙之益,實為全河一大關鍵。視齊蘇勒例,予騎都尉世職。」入覲,上命於攔黃壩迤上加築壩為重門保障,並於舊有木龍三架迤上增設木龍。薩載回任,奏遵上指料理,上嘉之。冬,復奏:「新河河面首尾寬窄不同,請於北灘順水勢抽槽,酌留土格。俟來年水漲放溜

沖刷,使河面首尾寬闊相若。」繪圖以進,上覽圖中北岸有新淤,因慮北淤則溜必南趨,識以硃筆,命薩載疏治。四十三年,奏:「高家馬頭新淤已刷動寬深,彭家馬頭新淤前作柴枕土壩。茲於灘面抽槽,候水漲沖刷。」旋署兩江總督。四十四年,奏攔黃壩外舊河露淤灘,請於灘面築束水堤為新河保障。尋實授兩江總督。先是,高晉奏中河口門淤阻,議移下游李家莊,上命薩載勘奏。薩載請將清口東、西壩移築惠濟祠前,上從之。

四十五年,大學士阿桂奏:「陶莊引河首尾寬而中窄,河身雖已刷深,水勢尚嫌束縛。伏秋汛漲,恐宣洩不及。」命偕薩載勘覆。尋奏請河寬六十餘丈處展十餘丈,河寬不及六十丈處展二十餘丈。又奏:「雲梯關外二套以下河流現行之道,道遠而水淺,請於四泓以下增設閘壩;二套上迤西馬港河舊堤殘缺,應行修復;並於舊無堤處補築新堤,下接北潮河西堰。」上從之。

夏,河溢郭家渡,命薩載與河道總督陳輝祖督護。是歲河水盛漲,初開毛城鋪、蘇家山、峰山頭諸閘,次將清口東西壩全行拆展。薩載奏諸州縣被水,睢寧、泗州為重,邳州、宿遷、靈壁、五河次之,現在撫恤寧貼。上諭曰:「實在無善策,祗可盡力撫恤,以期補過。」復命引河水入陶莊新河。尋奏豐、碭、銅、沛險工俱次第搶護,下游洪澤、高寶諸湖亦俱平定,俟水落堵築。得旨:「覽奏深慰。」先是,上臨高堰閱洪澤湖磚石諸工,諭薩載石工卑者增

高,磚工悉改用石。薩載奏請酌量緩急,分三年修築。八月,丁父憂,命百日滿後仍署兩江總督。四十六年,奏請自李家莊至臨河集北濬引河,上命速為之。

六月,河溢魏家莊,水大至。薩載奏:「全河奔注,歸入洪澤湖。清口展寬至八十丈,山盱五壩已開智、義二壩;而高堰諸地水勢未消,盈堤推岸。未開三壩及車邏、昭關二壩,或堅守,或酌開,俟察勘後續奏。」上命堅守。尋續奏洪澤湖浪湧山盱五壩,所存仁、禮二壩,掣通過水,續開車邏、昭關二壩。上以各閘壩俱開,下河民田被淹,令察災狀速奏。八月魏家莊工竟。山東巡撫國泰奏運河積淤,水不能暢行,議於劉老澗壩旁開水口分洩,上命薩載往勘。薩載奏:「運河洩水宣暢,已開駝車頭竹簍壩洩水入駱馬湖,劉老澗九孔石閘亦過水。若議別開水口,不便使無水之區再受水患。」上韙其言。又奏:「微山湖東南兩面水色澄清,沂河及駱馬湖水不使涓滴入運,為運河騰空去路。永濟橋孔亦無橫壩攔截,水勢暢消。」上稱為有條理,命國泰聽其指授,毋持己見。

十二月,兼署安徽巡撫。四十七年,奏請濬泗州謝家溝,洩睢河及楊甿諸河水入洪澤湖;又承上命濬銅山潘家屯引河。四月,河南青龍岡漫口既堵復蟄,大溜下注。上命寬濬潘家屯、劉老澗諸河,洩水歸海。薩載請開張家莊引河與潘家屯引河分流,使湖洩入黃又多一路。上諭曰:「籌洩水之路,為今日急務,宜妥為之。」加太子少保。江蘇巡撫吳壇議開

金壇漕河,自丹徒穿句容境分水脊達江寧。薩載奏:「分水脊即茅山之麓,地峻土堅,勢不能開鑿。請濬七里橋至巷口橋河道,與上、下河道寬深一律。」又請自鎮江錢家港至江寧龍潭濬闢新河,及修濬金山對渡瓜洲城河,上嘉之。又奏請濬漣河,展駱馬湖六塘河、鹽河口門,均如議行。

四十八年正月,服闋,實授兩江總督。河南青龍岡工竟,薩載奏黃河歸故道,入江南境流行迅速,得旨:「欣慰覽之!」上命移建沛縣城。薩載奏請移舊城西南戚山,並修夏鎮文武官署,豐、沛二縣漕倉。四十九年,江西巡撫郝碩坐婪賄得罪,責薩載未奏劾,下吏議,奪官,命留任,罰養廉三年。五十年,漕艘北行,以運中河淺阻,至天津誤期。上責薩載開運中河不知建閘,水勢一洩無餘;又清口東、西壩不能及早收束預為蓄水,致運道淺阻。降三品頂帶。五十一年,足疾,請解任。遣醫往視,命復原品。尋卒,贈太子太保,賜祭葬,謚誠恪,祀賢良祠。

子薩騰安,襲騎都尉,官至廣西按察使;薩雲安,官云南迤西道,坐事戍軍臺。

蘭第錫,山西吉州人。乾隆十五年舉人,授鳳臺教諭。擢順天大興知縣。三十四年,總督楊廷璋請以第錫升補永定河北岸同知,吏部以大興非沿河州縣,議駁,再請,上特許

之。再遷永定河道。四十八年,署河東河道總督。奏請河堤分界栽柳,並禁近堤取土;又奏儀封六堡、三堡灘面淺狹,水力較悍,請於新堤南築月是為障;皆從之。四十九年,奏:「河工綢繆防護,全在平時。堤有深淺,水有變遷,及車馬踐踏,豸雚鼠洞穴,必朝夕在堤,始能目睹親切。至冬末凌汛,春初桃汛,尤應晝夜巡邏。應令駐工各員移至堤頂,禁勿私下;如有曠誤,文武得互舉。令以堤為家,庶不至疏防。」均如所請行。五十年,奏:「北岸黃沁等、南岸上南等舊堤,及蘭儀等新堤,各增卑培薄;並加築舊壩,添作挑水。」上命速行。五十二年,上以第錫署任三年,勤奮妥協,命實授。旋兼兵部侍郎。

河溢睢州十三堡,疏請罪,上以其地原無埽工,原之。工竟,議敘。五十四年,調江南河道總督。河溢睢寧周家樓,疏請罪,上以河水異漲,原之。工竟,議敘。五十六年,奏勘毛城鋪滾水壩、王平莊新挑引河,上獎第錫察驗各工不草率。五十七年,請自淮安移駐清江浦,改建衙署,允之。五十九年,奏豐北汛接築土壩過多,上游水勢不能暢達,有礙曹、單河流去路,自請下吏議奪官,上命留任。嘉慶元年,河溢豐北汛,疏請罪,諭俟工竣覈功過。工竟,賜黃辮荷包,仍以不能先事預防停甄敘。二年,卒。

三年,第錫以河溢當償帑二十萬餘兩。上以第錫尚廉潔,慮不能勝,諮山西巡撫伯麟,伯麟奏第錫遺田舍僅值一百四十餘兩。上獎第錫清慎,諭道、以上及曾任總河各員分別代償。

韓鑅,順天大興人,原籍貴州畢節。入貲授通判,揀發山東,授上河通判。累擢江南淮徐道。乾隆四十六年,授河東河道總督。奏言:「山東運河,賴汶、泗來源及各湖接濟。汶河上游東平戴村等處民堰,對岸沙淤,應鑿灘抽溝,以展河勢。泗河下游即為府河,自安居、十里二斗門入運,河淺堰卑,亦當疏治。蜀山、馬踏、馬場、南旺諸湖,現當濟運洩水,堰根顯露,正可取土培堤。」七月,河決祥符焦橋,疏請罪,上原之。工竟,命優敘。未幾,河又決儀封曲家樓、青龍岡、大李家莊、孔家莊,凡溢四口。上令江南河道總督李奉翰赴工會督。水全出青龍岡,而孔家莊等三口皆塞。又命大學士阿桂履勘,又令山東巡撫國泰赴工會督。工垂竟,壩蟄復潰。大學士嵇璜議引河北流復故道,上以諮阿桂、李奉翰及鑅。鑅疏言:「青龍岡始漫,勢甚洶湧,是以倒漾北行,分入沙、趙二河,穿運歸海。未久旋即斷流,仍行南注。地勢北高南下,若於南岸建堤堵截,欲回狂瀾使之北注,誠如聖諭必不能行。水性就下,未便輕議更張。」阿桂等所奏亦略同,乃寢璜議,惟以河水北行既已斷流,責鑅何不即時具奏。

四十七年正月,壩復蟄。上聞運道河以南深通,河以北多淤墊,命鑅往微山湖北運河

察勘。二月,赴濟寧,會國泰及巡漕御史毓奇察勘,請自濟寧在城閘至嶧縣黃林莊,築土堰、柴壩、椿埽、橋梁,設水站,置絞關;鑅並請察勘畢,還青龍岡工次。上命鑅往來督察,復勘伊家河、荊山橋諸地水勢,請濬銅山潘家屯引河益使寬深,並濬駱馬湖、六塘河及濟寧南北徒駭、馬頰、伊家等河。時青龍岡壩屢築屢蟄,鑅遵上指迅籌宣洩,使黃水漸消。復還青龍岡工次,會阿桂等於蘭陽三堡改築大堤,濬渠導水出商丘七堡入正河故道。鑅旋以父憂去。四十八年三月,青龍岡工始竟。四十九年,服闋,授工部侍郎。部議鑅任河督時應償帑十四萬餘兩,詔免十之七。五十四年,命會勘通惠、溫榆二河,及朝陽門外護城河。調戶部。五十五年,命往江南會同江南河道總督蘭第錫督防汛。嘉慶三年,調兵部。四年三月,命守護裕陵。六年,以年老休致。九年,卒。

論曰:世業尚矣,於河事尤可徵。前乎此者,嵇曾筠有子璜,高斌有從子高晉。若李氏、何氏、吳氏皆繼之而起,宏及子奉翰、煟及子裕城並有名乾隆朝,嗣爵子璥則下逮嘉慶,奉翰子亨特,貪侈隕績,忝祖父矣。清時以誠篤名,第錫以廉潔著。青龍岡塞河決,歷兩載工始竟,阿桂主之,薩載、韓鑅佐之。詳具其始末,見成功之難也。


\end{pinyinscope}