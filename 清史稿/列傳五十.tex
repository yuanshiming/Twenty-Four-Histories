\article{列傳五十}

\begin{pinyinscope}
王弘祚姚文然魏象樞硃之弼趙申喬

王弘祚,字懋自,雲南永昌人。明崇禎三年舉人。自薊州知州遷戶部郎中,督餉大同。順治元年,授岢嵐兵備道。總督吳孳昌以弘祚籌畫軍餉,請仍留大同。二年,以總督李鑒薦,仍授戶部郎中。中原初定,圖籍散佚。弘祚聰強習掌故,戶部疏請修賦役全書,以弘祚主其事。弘祚謂:「民不苦正供而苦雜派,法不立則吏不畏,吏不畏則民不安。閭閻菽帛之輸,朝廷悉知之,則可以艱難成節儉。版籍賦稅之事,小民悉知之,則可以燭照絕侵漁。」裁定賦役,一準萬歷間法例,晚末苛細巧取,盡芟除之,以為一代程式。三年,加太僕寺少卿。六年,遷太僕寺卿,仍領郎中。

十年,擢戶部侍郎。時雲、貴尚為明守,孫可望據辰州。弘祚請於江南、江西、湖廣豐稔之地,採米穀、儲糧餉為進取計。又言:「黔國公沐天波世守雲南,得民心,其僚屬有散處江寧者,宜令往招天波為內應。貴州九股黑苗,自都勻、黎平遠及慶遠、靖州,近為可望蹂躪,宜加意撫綏,俾令歸化。冠服異制,勿驟更易。」上以所言足助撫剿,下經略大學士洪承疇採行。

十一年,給事中郭一鶚劾弘祚修賦役全書逾久未成,弘祚疏辨,一鶚復劾其巧飾。下部議,以各省冊報稽遲,弘祚不舉劾,論罰俸。十二年,疏請禁有司私派累民、將領冒名領餉,皆下部議行。十三年,以河西務鈔關員外郎硃世德徵稅不如額,援赦請免議,坐降三級,命留任。十五年,賦役全書成,敘勞,還所降級。考滿,廕子。尋擢尚書,加太子少保。命同大學士巴哈納等校訂律例。十六年,進太子太保。

雲南平,迭疏上善後諸事,請開鄉試,慎署員,設重鎮,稽丁田,恤士紳,撫土司,寬新政。既,又疏言司道宜久任,州縣宜部選,投誠宜解散,荒殘宜軫恤,爐座宜多設。弘祚聞父母喪,疏乞解官奔赴,命在任守制。逾月,命出視事。十八年,聖祖即位,疏請歸葬,許之。旋諭促還朝。

康熙三年,授刑部尚書,尋復還戶部。四年,星變地震,求直言。弘祚疏言:「異星見,天失其常;地震,地失其常。挽回天地之變,首在率循人事之常。」漕糧自通州運京師,或謂水次支散,可省轉搬費。弘祚持不可,謂:「水次支散,受者艱負戴。必減直而售,則米狼戾在外。京倉頒給雖有糶者,顆粒皆在都下。根本至計,不宜以小利遽變。」又有議盡裁州縣存留與變漕糧官運為商運者,固爭不得,具疏上之,卒如弘祚議。

六年,用輔政大臣鼇拜議,戶部增設滿尚書,以授瑪爾賽,與弘祚齟。七年,戶部失察書吏假印盜帑,大學士班布爾善獨罪弘祚,坐奪官。八年,鼇拜得罪,起弘祚兵部尚書。九年,以老乞休,命馳驛歸里,食原俸。弘祚中道疾作,僑居江寧。念未終事父母,輯永思錄,自號曰思齋。十一年,疏辭俸,諭曰:「卿在官著有勞績,引年乞休,賜祿頤養,毋固辭。」十三年,卒,賜祭葬,謚端簡。

姚文然,字弱侯,江南桐城人。明崇禎十六年進士,改庶吉士。順治三年,以安慶巡撫李猶龍薦,授國史院庶吉士。五年,改禮科給事中。六年,疏請「敕撫、按、道恩詔清理刑獄,勿任有司稽玩。條赦之外,有可矜疑原宥者,許專疏上陳」。又請重定會試下第舉人選用例,以廣任使。又言:「直隸與山東、河南接壤,盜賊竊發,東西竄匿,難於越境追捕。請改保定巡撫為總督,轄直隸、山東及河南懷慶、衛輝、彰德三府。」又請敕各省督撫勿濫委私人署州縣官。諸疏皆下部議行。尋轉工科。

八年,世祖親政,疏請令都察院甄別各省巡按,下部院會議,以六等考核,黜陟有差。是歲,江南、浙江被水,文然請災地漕米改折,視災重輕定折多寡。既,又言:「折漕例新定,民未周知。官吏或折外重徵耗銀,或先已徵米而又收折,或折重運輕,其弊不一。請敕漕臣密察嚴劾。」上並採納。十年,疏言大臣得罪不當鎖禁,得旨允行。遷兵科都給事中,乞歸養。

康熙五年,起補戶科給事中。六年,疏言:「四川、湖廣諸省官吏,借殿工採木,搜取民間屋材、墓樹,宜申飭禁止。」又言:「採買官物,其由官發價者,如有駁減餘銀,例貯司庫。若價出自民,餘銀宜還之民間。」又言:「案牘煩冗滋弊,一部可逕結之事,即應一部逕結;一疏可通結之事,即應一疏通結。若各省錢糧考成已報完者,部臣宜於議覆時即予開復。」均如所請。九年,考滿內升,命以正四品頂帶食俸任事。故事,給事中內升,還籍候補。留任自文然始。文然與魏象樞皆以給事中敢言負清望,號「姚魏」。十年,兩江總督麻勒吉坐事逮詣京師,仍用鎖系例。文然復上疏論之,上諭:「自後命官赴質,概免鎖系,著為令。」

尋遷副都御史,再遷刑部侍郎。十二年,調兵部督捕侍郎。京口副都統張所養劾將軍柯永蓁徇私縱恣,令文然往按,永蓁坐罷。遷左都御史。十三年,疏言:「福建耿精忠、廣西孫延齡皆叛應吳三桂,中間阻隔,賴有廣東。精忠將士舊駐其地,熟習山川形勢,倘與延齡合謀相犄角,則廣東勢危。江西境與福建、廣東接,倘侵據贛州南安,驛道中斷,餉阻郵梗。宜駐重兵通聲援。」上嘉納之。陜西提督王輔臣叛,河南巡撫佟鳳彩引疾,上已許之;文然言河南近陜西,流言方甚,鳳彩得民心,宜令力疾視事,上為留鳳彩。

文然屢有論列,尤推本君身,請節慎起居。孝誠皇后崩,權攢鞏華城,上數臨視,文然密疏諫,且引唐太宗作臺望昭陵用魏徵諫毀臺事相擬,上亦受之,不怫也。十五年,授刑部尚書。時方更定條例,文然曰:「刃殺人一時,例殺人萬世,可無慎乎?」乃推明律意,鉤稽揅討,必劑於寬平,決獄有所平反,歸輒色喜。嘗疑獄有枉,爭之不得,退,長跪自責。又以明季用刑慘酷,奏除廷杖及鎮撫司諸非刑。十七年,卒,賜祭葬,謚端恪。

文然清介,里居幾不能自給,在官屏絕餽遺,晚益深研性命之學。子士基,官湖廣羅田知縣;士苾,官陜西朝邑知縣:皆有治行。

魏象樞,字環極,山西蔚州人。順治三年進士,選庶吉士。四年,授刑科給事中。疏言:「明季大弊未禁革者,督、撫、按聽用官舍太雜,道、府、州、縣胥隸太濫,請嚴予清釐。」報可。五年,劾安徽巡撫王懩受賕庇貪吏,懩坐罷。轉工科右給事中。時以滿、漢雜處不便,令商民徙居南城。象樞疏言:「南城地狹,商民賃買無房,拆蓋無地。請下部察官地官房,俾民輸銀承業。」復疏請更定會典。並下部議行。七年,轉刑科左給事中。

八年,世祖親政,有司有以私徵侵帑坐罪者,象樞疏陳其弊,請飭州縣依易知單造格眼冊,注明人戶姓名、糧銀、款目及蠲賑清數,上大吏覈驗,印發開徵;又請定布政使會計之法,以杜欺隱,立內外各官治事之限,以清稽滯:皆見施行。復疏言:「聖政方新,機務孔多,中外相望治平,非同昔日。上近巡京畿,輔臣當陪侍法從,盡啟沃之忠。倘遠有臨幸,亦宜諫止鑾輿,副保傅之責。」又因災變上言,謂天地之變,乃人事反常所致。語侵權貴尤急。九年,轉吏科都給事中。十年,大計,疏請復糾拾舊制,言官糾拾未得當,不宜反坐,下所司,著為令。因復疏言順治四年吏科左給事劉楗以糾拾被譴,宜予昭雪,上為復楗官。

總兵任珍失職怨望,並擅殺其家人,下九卿科道議罪,大學士陳名夏等二十八人,別為一議,象樞與焉。上責其徇黨負恩,下部議,罪應流,寬之,命留任。十一年,大學士寧完我劾名夏,辭連象樞,謂象樞與名夏姻家牛射鬥有連,象樞糾劾有誤,吏部議降級,名夏改票罰俸,命逮問。象樞自陳素不識射鬥,得免議。尋以名夏父子濟惡,言官不先事論劾,各科都給事中皆鐫秩,象樞降補詹事府主簿,稍遷光祿寺丞。十六年,以母老乞終養。

康熙十一年,母喪終,用大學士馮溥薦,授貴州道御史。入對,退而喜曰:「聖主在上,太平之業方始。不當以姑且補苴之言進。」乃分疏,言:「王道首教化,滿、漢臣僚宜敦家教。」「督撫任最重,有不容不盡之職分、不容不去之因循,宜責成互糾。」「制祿所以養廉,今罰俸例太嚴密,宜以記過示罰,增秩示恩。」「治河方亟,宜蓄人才備任使。」「戒淫侈宜正人心,勵風俗宜修禮制。」聖祖多予褒納。復疏糾湖南布政使劉顯貴侵公帑不當內升,給事中餘司仁欺罔不法,皆坐黜。十二年,以歲滿加四品卿銜,尋擢左僉都御史。

十三年,歲三遷,至戶部侍郎。會西南用兵,措兵食,察帑藏,多所規畫。疏論籌餉,請確估價直,嚴覈關稅,慎用各直省布政使。十七年,授左都御史。疏言:「國家根本在百姓,百姓安危在督撫。原諸臣為百姓留膏血,為國家培元氣。臣不敢不為朝廷正紀綱,為臣子勵名節。」因上申明憲綱十事,上嘉其切中時弊。各直省舉劾屬吏多失當,江蘇嘉定知縣陸隴其有清名而被劾罷,象樞疏薦之。鎮江知府劉鼎溺職,題升糧道;山西絳州知州曹廷俞劣跡顯著,糾察不及:象樞疏劾之。磨勘順天鄉試卷,因陳科場諸弊,請設內簾監試御史;考核各直省學道,舉勞之辨、邵嘉,劾盧元培、程汝璞,上如其議以為黜陟。

十八年,遷刑部尚書。象樞疏言:「臣忝司風紀,職多未盡,敢援漢臣汲黯自請為郎故事,留御史臺,為朝廷整肅綱紀。」上可其奏,以刑部尚書留左都御史任。分疏劾山西巡撫王克善、榷稅蕪湖主事劉源諸不法狀,皆坐黜。七月,地震,象樞與副都御史施維翰疏言:「地道,臣也。臣失職,地為之不寧,請罪臣以回天變。」上召象樞入對,語移時,至泣下。明日,上集廷臣於左翼門,詔極言大臣受賕徇私,會推不問操守;將帥克敵,焚廬舍,俘子女,攘財物;外吏不言民生疾苦;獄訟不以時結正;諸王、貝勒、大臣家人罔市利,預詞訟:上干天和,嚴飭修省。是時索額圖預政貪侈,詔多為索額圖發,論者謂象樞實啟之。

尋命舉廉吏,象樞舉原任侍郎雷虎、班迪、達哈塔、高珩,大理寺卿瑚密色,郎中宋文運,侍講蕭維豫,布政使畢振姬,知縣陸隴其、張沐凡十人。上諭曰:「雷虎朕亦聞其清,以其怠惰罷黜,既經象樞特薦,授內閣學士。班迪清慎,因使往江西按事,未能明晰,問以民間苦樂,又謝不知,以是鐫秩。餘令吏部議奏錄用。」十九年,仍授刑部尚書。尋命與侍郎科爾坤巡察畿輔,按治豪猾,還奏稱旨。

象樞有疾,上賜以人參及參膏,命內侍問飲食如何。二十三年,奏事乾清門,躓焉,即日疏乞休,再奏,乃許之,命之入對,賜御書寒松堂額,令馳驛歸。二十五年,卒,年七十一,賜祭葬,謚敏果。

象樞以馮溥薦再起。象樞見溥,問何以見知?溥曰:「昔餘為祭酒,故事,丁祭不得陪祀者,當於前一日瞻拜。君每期必至,敬慎成禮。一歲直大雨,君仍至,肅然瞻拜而去,此外無一人至者。餘以是知君篤誠。」子學誠,進士,授中書。上推象樞恩,改編修,官至諭德。嘉慶間,錄賢良祠諸臣後裔,賜象樞四世孫煜舉人。

硃之弼,字右君,順天大興人。順治三年進士,授禮科給事中,轉工科都給事中。八年,疏言:「國家宜重名器。舊制,胥吏供役年久無過,予以議敘,選用佐貳。今戶、兵等部書役別系職銜,非官非吏,有玷班行。此曹起自貧乏,不數年家貲鉅萬,衣食奢侈。非舞文作奸,何以致此?戶、兵堂司官歲有遷轉,此曹歷年久不去,官為客,吏為主,流弊何窮。請嚴察褫奪。」下部議行。九年,以父喪去。十一年,起補戶科都給事中。

十二年,疏言:「小民納糧一也,而其目有四:曰漕糧、白糧、軍糧、恤孤糧。軍糧、恤孤糧程限遲緩,無增耗之費,有力之家,往往營求撥兌;單弱之戶,派納漕、白,苦樂不均。軍糧行折色,軍得銀則妄費,生掛欠之弊。恤孤糧半飽豪強,鰥寡孤獨無由控訴。請飭漕臣下各省糧道,親督州縣畫一編徵,盡數輸納,敢有撥兌者治罪。」又言:「錢糧侵欠,兵食不充,為上所廑念。侵欠之大者,曰漕欠、糧欠。漕欠責漕督親督糧道,糧欠責督撫親督布政使,令本年附徵。某年欠項逾限不完,以溺職論,有司侵虧怠緩,糾劾不貸。如此,則年銷年欠,宿逋可清。」上韙其言,並嚴飭行。又疏言:「國家章制大備,部臣實心任事,利自知舉,弊自知革。今乃盡若事外,遇事至,才者不肯決,無才者不能決,稍重大即請會議。不然,行外察報,遷延歲月而已;不然,聽督撫參奏,科道指糾而已;不然,茍且塞責,無容再議而已:上下相諉,彼此相安。國家事安得不廢,百姓安得不困?欲致太平,必無之事也。臣愚謂今日求治,首在擇人。上面召諸大臣親試才品,因能授任;復考其歷事後興利幾何,除弊幾何,定功罪,信賞罰,則法行而事舉矣。」上納之弼言,諭六部去怠忽舊習。一歲中四遷,授戶部侍郎。十三年,河西務鈔關員外郎硃世德徵稅不如額,戶部援赦請免議,上切責譴部臣,之弼降三級。

十五年,授光祿寺少卿,再遷左副都御史。疏言:「巡按未得其人,當責都察院考核,巡按之賢不肖,即都察院堂上官賢不肖。臣與諸巡按約,操守當潔清,舉劾當得宜,撫按當互糾。臣等定差不公,考核不當,巡按賢者不薦,不肖者不糾,諸御史亦得論劾。至巡方應行諸事,當令掌河南道會諸御史各抒見聞,奏請明定畫一。」從之。

世祖惡貪吏,命官得贓十兩、役得贓一兩,皆流徙。令既行,之弼疏論其不便,略謂:「自上諭宣傳後,撫按所糾,必無以大貪入告者。何則?一經提問,有司無不圖保身命,雖盈千累百,而及其結讞,期不滿十兩而止。是未糾以前,徒層累而輸於大吏。被糾之後,又層累而輸於問官。尺籍所科,百不一二。蓋雖起龔、黃為今之有司,未有不犯十兩之令者。而今普天之下,皆不取十兩之有司,豈真出古循吏上哉?良以令嚴則思遁,徒有名而無其實也。上但擇撫按一大貪者懲之,一大廉者獎之,則眾貪懼、眾廉奮矣。」

會歲旱求言,之弼疏言:「山東巡撫耿焞、河南巡撫賈漢復以墾荒蒙賞,兩省百姓即以賠熟受困,歲增數十萬賦稅,多得之於鞭笞敲剝、呼天搶地之孑遺。怨苦之氣,積為沴厲。」又疏劾戶部賑濟需遲,救荒無術。京師既得雨,河南報彰德、衛輝以旱成災,戶部奏:「上步禱天壇,時雨方降。彰德、衛輝地接畿南,何獨請蠲恤?請覆勘。」之弼疏爭,略謂:「百里不同風,千里不同雨,安得以輦下例率土?且以撫臣疏報為不可信,而又倚以覆勘,使撫臣告災如前,部臣信之不可,不信必易人而勘,徒使地方增煩擾耳。自夏徂冬,被災州縣未盡停徵,待勘明已至來春,雖蠲免,徒飽吏橐,饑民轉為溝中瘠久矣。」與尚書王弘祚廷辨,卒從之弼議。十八年,復授戶部侍郎。

康熙四年,調吏部。五年,遷左都御史,擢工部尚書。六年,疏言:「福建官兵月米五十餘萬石,歲徵十萬餘石,餘皆糴諸市,石值銀二兩四錢。朝廷買米養兵,絕不抑值以累民。臣聞延、建、汀、邵諸府民以買米攤賠為累,有原繳田入官者。漳、泉之間,按地派米,石必加六斗,又迫令折價三四兩不等,數倍於正供,民不勝其朘削。」上特諭督撫嚴察。

七年,調刑部。八年,疏言:「各省存留錢糧,順治間軍需正迫,有裁減之令。昨年部臣又請酌減。存留各款,原為留備地方公用,事不容已,費無所出,勢不得不派之民間,不肖有司因以為利。宜復康熙七年以前存留舊例。」又疏言:「八旗家丁,每歲以自盡報部者不下二千人。人雖有貴賤,均屬赤子。請敕諭八旗,凡蓄僕婢,當時其教誨,足其衣食,恤其勞苦,減其鞭笞,使各得其所。歲終刑部列歲中自盡人數,系某旗某家,具冊呈覽,俾人知儆惕。」又言:「世祖嚴治貪官蠹役,特立嚴法,如非官役,不用此例。今不論有祿無祿,通用重典。貪蠹事發,被證畏同罪,刑訊不承,使大貪漏網。請嗣後因事納賄,仍擬同罪。如逼抑出錢,倘非官役,許用舊律。」詔並如所請。九年,調兵部。十四年,以母喪去官。十七年,起授工部尚書。二十二年,會推湖北按察使,之弼舉道員王垓,不當上意,以所舉非材,吏部議降三級調用。尋卒。

之弼內行修篤,事親孝,與其弟之佐相友愛。之佐,順治十四年進士,選庶吉士,歷官侍讀學士。嚴事之弼,雖白首,執子弟禮甚謹。

趙申喬,字慎旃,江南武進人。康熙九年進士。二十年,授河南商丘知縣,有惠政。二十五年,以賢能行取,命以主事用。二十七年,授刑部主事。三十年,遷員外郎,以病乞歸。四十年,以直隸巡撫李光地薦,召見,上察申喬敬慎,超擢浙江布政使。陛辭,上諭曰:「浙江財賦地,自張鵬翮後,錢糧多蒙混,當秉公察核,不虧帑,不累民。布政使為一省表率,爾清廉,屬吏自皆守法。」申喬頓首謝曰:「臣蒙皇上特擢,不黽勉為好官,請置重典。」申喬上官,不挾幕客,治事皆躬親,例得火耗,悉屏不取。四十一年,上諭獎申喬居官清,能踐其言,就遷巡撫。布政使舊有貼解費,歲支不過十之五,申喬積二千餘金,封識以授代者,曰:「吾奏銷不名一錢,後將難繼,得此足辦一歲事,毋以擾民也。」錢塘江潮齧塘,申喬令鎔鐵貫石,築子塘為護。

湖南鎮筸紅苗殺掠為民害,民走京師叩閽陳狀,給事中宋駿業因劾總督郭琇、巡撫金璽、提督林本植諱匿不為民去害,上命侍郎傅繼祖、甘國樞及申喬往按,盡發紅苗殺掠害民狀,琇等皆坐罷。調申喬偏沅巡撫。四十二年,疏言與總督喻成龍檄衡永道張士可入苗洞宣撫,已聽命者二十餘寨,並與提督俞益謨發兵討諸不率命者。上命尚書席爾達等率荊州駐防滿洲兵,並檄廣東、貴州、湖北三省提督,會成龍等進攻。自龍椒洞至於天星寨,分道搜剿,斬悍苗千餘,三百餘寨咸聽命受約束,苗悉定。申喬疏上善後諸事,移辰沅道駐其地。上獎征苗諸將,貴州提督李芳述功最,並褒申喬強毅。

上南巡,申喬朝行在,上以潮南地偏遠,官吏私徵、加耗倍於他省,特詔申飭。申喬還,建上諭碑亭於通衢,示屬吏,並疏劾巴陵知縣李可昌等違例苛斂,奪官逮治。四十五年,申喬疏言:「清浪、平溪二衛地處山僻,請改米徵銀,俾省運費。」四十六年,疏言:「漕運旗丁舊有耗贈、行月銀米,於起運前預發。給事中戴嵩條奏俟至通州補發,意在防其虧缺。湖南運道遠於江、浙,例本無耗贈,惟恃行月銀米為轉運之資。今既扣存,窮丁不能涉遠,必致誤漕。請仍舊例預發。」上許之,著為令。

四十七年,命赴湖北按讞荊州同知王侃等侵蝕木稅,疏請裁港口渡私稅,荊州關稅部差如故。申喬還,又請以靖州屬鸕鶿關稅並入辰州關。別疏言:「營兵給餉,每於正月支領,時地丁尚未開徵,挪移則累官,預徵則累民,請以隔歲餘存米石撥給兵餉。」並下部議行。內閣學士宋大業祭告南嶽還京師,劾申喬輕褻御書,詔詰申喬。申喬疏辨,並言:「大業初使湖南,餽金九千。此次再使湖南,餽金五百,意不慊,札布政使董昭祚,言南嶽廟工餘銀毋報部。臣仍報部充餉,以是誣劾。」大業坐奪官,申喬鐫五級留任。

四十八年,疏劾提督俞益謨取兵糧三十五石,詔詰益謨。益謨劾申喬苛刻,請並解官質訊。四十九年,上命尚書蕭永藻往按,永藻察申喬疏實,上為罷益謨,而命申喬還職。尋擢左都御史,諭曰:「申喬甚清廉,但有性氣,人皆畏其直。朕察其無私,是以護惜之。」五十年,疏請刻頒部行則例。劾編修戴名世所著南山集、孑遺錄有大逆語,下刑部,鞫實坐斬。五十一年,疏請禁營兵冒名食糧;又言上普免各省地丁錢糧,惟潼關衛、大同府徵本色,不在蠲例,請如奉天、臺灣例,一體蠲免:並允所請。

又疏言每歲農忙,京師當遵例停訟。上諭曰:「農忙停訟,聽之似有理,實乃無益。民非獨農也,商訟則廢生理,工訟則廢手藝。地方官不濫準詞狀,準則速結,訟亦少矣。若但四月至七月停訟,而平日濫準詞狀,又復何益?且此四月至七月間,或有奸民詐害良善,冤向誰訴?八月以後,正當收穫,亦非閒時。福建、廣東四季皆農時,豈終歲停訟乎?讀書當明理,事有益於民,朕即允行,否則斷乎不可也。」五十二年,廣東饑,命往督平糶。尋授戶部尚書。

五十三年,旗丁請指圈滄州民地,直隸巡撫趙弘燮議以旗退地另撥,部議不許。申喬言滄州民地有旨停圈,宜如弘燮議,上從之。時方鑄大錢,商人請納銀領易小錢送寶源局改鑄,命內務府會戶部議。申喬言:「收小錢,有司責也,商人圖利,恐近藉端擾民,不可許。」而疏已上,議準申喬奏,請罷斥。上召問狀,申喬言:「司官但送侍郎畫題,為所藐視,無顏復居職。」上曰:「君子懲忿窒欲,此語宜詳思。司官藐視,但當奏劾。爾性苛急,不能容人。天地之大德曰生,非但不殺而已。蓋於萬物皆養育而保全之。爾在官誠廉,然豈可恃廉而矯激乎?」命任事如故。卒用申喬議,罷商人納銀領錢。

申喬子鳳詔,官太原知府。上幸龍泉關,鳳詔入謁,上以申喬子優遇之。問巡撫噶禮賢否,鳳詔言噶禮清廉第一,上為擢噶禮江南總督。及噶禮以貪敗,上舉鳳詔問尚書張鵬翮,鵬翮言其貪。五十四年,山西巡撫蘇克濟劾鳳詔受賕至三十餘萬,命奪官按治。申喬疏謝不能教子,請罷斥,上責其詞意忿激,非大臣體,命任事如故。鳳詔坐贓罪至死。

五十九年,以病乞休。上仍獎申喬清廉,令在官調治。鳳詔贓未清,命免追,並諭大學士,謂「速傳此旨,使其早知,庶服藥可效也」。尋卒,年七十有七,賜祭葬,謚恭毅。雍正元年,加贈太子太保。六年,湖廣總督邁柱疏劾屬吏虧帑,有申喬在偏沅時事,例當分償。世宗特命免之。

論曰:弘祚定賦役,文然修律例,皆為一代則,其績效鉅矣。象樞廉直謇謇,能規切用事大臣,尤言人所難言。之弼意主於愛民,凡所獻替,皆切於民事。申喬名輩差後,清介絕流輩,慷慨足以任國家之重。貞元之際,自據亂入升平,開濟匡襄,諸臣與有力焉。


\end{pinyinscope}