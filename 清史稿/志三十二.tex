\article{志三十二}

\begin{pinyinscope}
地理四

△黑龍江

黑龍江:古肅慎國北境。明領於奴兒乾都司。清初有索倫、達呼爾諸部,散居黑龍江內外額爾古納河及精奇里江之地。天聰、崇德中,次第征服。康熙二十二年徵羅剎,始設鎮守黑龍江等處將軍及副都統駐江東岸之愛渾城,尋並移駐墨爾根。三十七年,副都統移駐齊齊哈爾。三十八年,將軍亦移駐,遂為省治。後增設墨爾根、黑龍江、呼蘭、呼倫貝爾、布特哈各副都統。光緒末,裁省其半,改置、府、縣有差。三十三年,罷將軍,設黑龍江巡撫,改為行省,悉裁副都統各缺,變置地方官制。宣統三年,為道三,府七,六,州一,縣七。擬設之府一,直隸十一,縣五。南至松花江與吉林界,西至額爾古訥河與俄領薩拜哈勒省及外蒙古車臣汗旗界,西南接內蒙古之烏珠穆沁左翼、科爾沁右翼中、前、後各旗界,東至松花、黑龍兩江合流處,仍界吉林,北及東北皆與俄領阿穆爾省界。廣二千八百餘里,袤一千五百餘里。北極高四十五度五十分至五十二度五十分。京師偏東三度四十分至十六度二十分。案黑龍江舊界,楊賓柳邊紀略云:「艾渾將軍所屬,東至海,西至你不楮俄羅斯界。」你不楮即尼布楚,今俄名捏爾臣斯克。艾渾將軍即黑龍江將軍,此清初界也。自安巴格爾必齊河口,即循此河上流之外興安嶺,東至於海。凡嶺以南,流入黑龍江之溪河屬中國,嶺以北屬俄羅斯。中、俄分嶺,此康熙二十八年尼布楚條約界也。自額爾古訥河入黑龍江處起,至黑龍江與松花江會流處止,以南以西屬黑龍江省,以北以東屬俄羅斯,中、俄分江,此咸豐八年璦琿條約界也。尼布楚在安巴格爾必齊河西五百餘里,本中國茂明安、布拉特、烏梁海諸部落地。崇德中,即為俄羅斯人竊據,築城居之,以侵掠索倫、達呼爾諸部,為邊患者三十餘年。康熙二十八年定界,遂捐以畀俄,已蹙舊界地五百餘里矣。若外興安嶺以南,黑龍江以北以東舊界地,殆三千七百里有奇,其境內山川、部落、城屯雖為俄有,亦並志之,不忘其朔焉。外興安嶺為昆侖北出之大幹。蓋昆侖山脈南幹,為涼州南山,為賀蘭山,為陰山,為內興安嶺。北幹為蔥嶺,為天山,為阿爾泰山,為墾特山,為外興安嶺。內外者,據黑龍江言之。餘若鬥色山、若楊山、若珠德赫山、若訥丹哈達拉山、若達勒替沙山、若阿喇拉山、若道斯哈達、若察哈彥哈達、若茂哈達,皆外興安嶺支絡,並在江東北。水以安巴格爾必齊河為康熙舊界入江之始。由此而東,曰卓魯克齊河、曰烏魯穆河、曰格爾必齊河、曰呼吉河、曰張他拉河、曰鄂爾多昆河、曰烏爾蘇河、曰波羅穆達河、曰額爾格河、曰巴爾坦河,又東少南,曰託羅河、曰臥諾河、曰巴里彥河、曰阿蘇河、曰淘斯河、曰凱蘭河、曰阿喇拉河、曰大蘭河、曰庫哷恩河、曰額蘇里河、曰多普塔拉河。又南曰精奇里江,為諸河最,源出外興安嶺極北之地,東南流,轉西南流,江形如弓。有烏爾格河、託克吉魯河、烏爾替河、克德畢河,自西北來注之。有阿爾吉河、巴里木河、塔烏爾堪河、畢奇勒圖河、欽都河、寧尼河、額勒格河、牒葉普河、鐵牛河、西里木迪河、察勒布克爾河、英肯河、們臥勒河、莫昆河、巴沙河、楊奇尼河、密奇訥河、翁額納拉河、巴里木邁庫里河、託莫臥河、伊羅河、昆貝河、屯布河、迪音河,自東北來注之。黑龍江水色微黑,精奇里江獨黃,又稱黃河。又南而東,曰謨里爾克河、曰博屯河。又南而東,曰牛滿河,源出外興安嶺,嶺東舊界吉林。西南流,東合烏旺那河、烏莫勒德河、攸瓦爾奇河、敖拉河、塔拉耐河、塔裏木河、薩公那河、吉克河,西合臥爾喜河、卓羅奇河、木爾木河、楊奇里河、珠奇河、寧那河、伊莫勒河、楚克河。牛滿河亦稱斗滿河,又南而東,曰哈拉河、曰阿拉河、曰塔拉木河、曰庫勒圖爾河、曰庫木弩河、曰珠春河、曰格林河、曰胡裕魯河、曰蘇魯河、曰伊圖裏河、曰畢占河。以上諸河,並南入黑龍江。畢占河以南,舊為吉林境。其部落,則精奇里江東西,為索倫部、達呼爾部。有索倫村,在精奇里江、額爾格河之間,南距黑龍江城五六日程。欽都河西及巴爾坦河東,為使鹿鄂倫春部。自額蘇里河口溯江而西,至額爾格河口,為庫爾喀部,其城屯有曰鐸辰城、阿薩津城、多金城、烏魯蘇穆丹城、郭博勒屯、博和哩屯、噶勒達遜屯、穆丹屯、都孫屯、烏爾堪屯、德篤勒屯、額蘇哩屯、額爾圖屯,並在江北岸東岸。雅克薩城在黑龍江城西北一千三百餘里,城東即提咸河灣城,本索倫部築。嗣因博木博果爾等據城以叛,崇德四年討平之,墟其城。順治初年,羅剎竊據,又築之。康熙二十五年,復克其城。二十八年界約,雅克薩之地俄羅斯所治之城,盡行毀除。今其地俄名阿勒巴沁云。宣統三年,編戶二十四萬一千零一十一。口一百四十五萬三千三百八十二。其名山:特爾根、佛思亨、興安嶺。黑省之山,皆脈自車臣汗部肯特山,入境則特爾根,折而東而南,綿亙嫩江、黑龍間者,以興安嶺目之。至混同、黑龍兩江將會處,乃起佛思亨。內興安嶺自索嶽爾濟山入境,為哈瑪爾,為室韋,為雅克,為西興安嶺,為伊勒呼里。分支西北迤,為治吉察。正支又東北,為嫩江源。又東南,為庫穆爾,為東興安嶺。西出一支為和羅爾。又西曰烏雲和爾冬吉。正支又東迤,為小興安嶺。又分支東北為老爺嶺。正支東盡於佛思亨。其巨川:黑龍、精奇里、松花、烏蘇諸江。其驛路:東北逾興安嶺達海蘭泡。電線:自齊齊哈爾至海蘭泡,南達吉林。鐵路:齊昂;其屬俄者,東清北段。

龍江府:沖,繁,疲,難。巡撫,民政、提學、提法司駐。即齊齊哈爾。舊曰卜魁。明,朵顏衛地。光緒三十一年,設黑水。三十四年改置府,為黑龍江省治。西南至京師三千三百餘里。廣六百六十餘里,袤六百六十餘里。北極高四十七度二十七分。京師偏東七度三十二分。北:敖寶山。西:五道梁子、碾子山、廉家大岡。東北:嫩江自訥河入,南流,至府城東北。東分一支為塔哈爾河,西南受阿倫河、音河,逕城西南。距城約五里曰船套,康熙中嫩江水師戰船泊此。光緒三十三年闢為商埠。由西南江口斜開引河縈泗城西。沿江築長堤一、小堤二。嫩江又西南納雅爾河,入安達。東:胡裕爾河自拜泉入,西流,入塔哈爾河。一支南出,歧為九道溝,西南入安達。舊設站十,在府境四:卜魁、特穆德赫、塔拉爾、寧年。西路臺十七,在府境三:七家、甘井子、那奇希。官商路二:東南東官地屯達海倫;小五明馬屯達景星鎮。卡倫三:曰莽鼐,曰綽羅,曰博爾齊勒。又和倫部卡倫三:曰拉哈鄂佛羅,曰溫德亨,曰蘇克臺蘇蘇。鐵路二:齊昂,東清。商埠,光緒三十一年中日約開。

呼蘭府:沖,繁,難。省東南八百四十里。即呼蘭副都統城。明為呼蘭山衛。光緒三十年,移呼蘭治呼蘭城,升為府。廣一千二百餘里,袤四百二十餘里。北極高四十六度十二分。京師偏東九度五十九分。領州一,縣二。西南:松花江自肇州入,東流入巴彥。呼蘭河自蘭西入,南流,大堿溝自西來注之。屈東南,逕府城南入松花江。東:漂河自巴彥入。又東,少陵河,則綽羅河亦自巴彥入,右受韓溝河,南流,同入松花江。北:濠河自綏化入,左受大荒溝河,西流入呼蘭河。府境據呼蘭河下游水域,松花襟其南,長河支港,足資灌溉,土味膏沃,號為產糧之區。雍正十三年後,移屯設莊,日事開闢。咸豐、同治之際,直隸、山東游民流徙關外者,競赴屯莊傭工,積日既久,私相售賣,占地日廣,聚徒日繁,歷任將軍乃奏辦民墾,增改民官,行省規模,府為先導焉。舊設臺三:察哈和碩;呼蘭城,即府城;新安。官商路二:西北經蘭西赴省城;東北經巴彥赴綏化。有康家井、朝陽堡文報局。舊設卡倫四:曰珊延富勒,曰綽羅河口,曰諾敏河,曰布勒嘎哩。西南:東清鐵路對青山車站,南六十里至哈爾濱。呼蘭河口有輪船埠。巴彥州繁,難。府東一百五十里。原名巴彥蘇蘇。光緒元年設呼蘭,三十年改隸府。北:青頂山、雙牙。西:少陵、泥馬爾。東北:黑山,綿亙百餘里,與木蘭青山接,故布特哈人虞獵場也,又名蒙古爾山,呼蘭民屯自山前後始。南:松花江自府境入,東入木蘭。北:少陵河自東興鎮入,西流,納布爾嘎里河、小柳樹河、硃克特河,屈西南,漂河分支曰韓溝,東流注之,為綽羅河口。又東:五岳河,出棗拉拉屯,西流屈南,逕府城西,入松花江。東:大黃泥河,左會小黃泥,又東小石頭河,皆南入松花江。北:濠河由餘慶入,合拉三太河、大荒溝入府境。西北:興隆鎮州判。舊設臺一:呼蘭,即州城。官商路三:東至木蘭;北至餘慶;北由小豬蹄山屯西行,經興隆鎮達綏化。五岳河口有輪船埠。蘭西縣沖,繁,難。府西北一百里。原名雙廟子,光緒三十年置,隸府。東呼蘭河自綏化緣海倫界,會通肯河入,屈南,右大堿溝河,左濠河,入府境。官商路四:東榆樹林達府;北至青岡;西達肇州;西北至安達。有小榆樹鎮。木蘭縣疲,難。府東二百五十四里。明,木蘭河衛。光緒三十年置,隸府。北:青山,山勢與巴彥黑山接,舊稱呼蘭青、黑二山。西北:駱駝砬子、硯臺、蒙古山。東北有玉皇閣山,皆在縣北境。南:松花江自巴彥入,東入大通。西:白楊木河;又西,大小木蘭達河,左會鎮陽河;又西,萬寶、柳樹、楊樹、大小石頭諸河,皆南入松花江。東:頭道河,左會二道河,南入大通。北:木蘭鎮巡檢協領駐。官商路三:西至巴彥;東至大通,有五站,文報局一;循大木蘭達河北東興鎮達綏化。

綏化府:沖,繁,難。省東南七百六十里。原名北團林子,隸呼蘭副都統。光緒十一年,設綏化。是時副都統治所號中路,呼蘭號南路,城號北路,名為呼蘭三城。三十年升為府。廣三百餘里,袤一百餘里。北極高四十七度三十八分。京師偏東十度五十六分。領縣一。東北:綏額楞山,尼爾吉、克音二河出。呼蘭河自餘慶入,各緣界右注之。西流,右受尼爾吉、克音河,左受津河,入蘭西。南:濠河亦緣界從之。東北:上集廠,駐經歷。官商路五:北赴海倫;南出巴彥;西至蘭西;東津河鎮赴餘慶;東北雙河鎮達鐵山包。餘慶縣繁,難。府東一百里。原名餘慶街。光緒十一年設分防經歷,屬綏化。三十年改置,隸府。南:青山、黑山山脈,跨木蘭、巴彥兩州縣界。北:呼蘭河,導源鐵山包達裡代嶺西麓,西流入境,又西入府。濠河導源極南沈萬合屯,西流入府。南:格木克河出上窖子,北至郭吳屯,屈西,逕縣治南,又西北入呼蘭河。東:拉列罕、安拜、穩水、鐵山包、尼爾吉諸河,皆北入呼蘭河。又東北額伊琿河,西南流至王家堡,合歐肯河,大伊吉密合小伊吉密河,皆入呼蘭河。官商路四:西赴府;東赴鐵山包;北出五道岡西行達海倫;東行達鐵山包。一東南黎家屯南行至東興鎮,又便道南渡格木克河、雙銀河、濠河達巴彥。民船可溯呼蘭河至鐵山包運煤。

海倫府:繁,疲,難。省東南六百里。即通肯副都統城。光緒三十年,以通肯、海倫河新墾地置海倫。三十四年升府。領縣二。東北:內興安嶺。通肯河出西麓,西流,右受十一道至八道溝,屈南流,札克河東來注。西南:七道溝自胡裕爾河分出,南流來注。南:海倫河自東來注,三道、二道、頭道、污窿河自西來注,又南會呼蘭河。呼蘭河南自綏化入,合通肯河、克音河來會,入呼蘭境。北:胡裕爾河緣訥河界入之。府境居海倫河北,有通肯協領。官商路三:西至拜泉;西南至青岡;南至綏化。東南行經綏化上集廠達餘慶。又西北海布道至布特哈,北海畢道至畢拉爾協領地,二道皆宣統中開。商船由呼蘭河入通肯河至女兒城。青岡縣疲,難。府西南二百六十里。原名柞樹岡。柞樹一名青岡柳,縣以此名。光緒三十年置,隸。東:通肯河自拜泉入,南流,與府分界,入呼蘭河。呼蘭河自府會通肯河入,西南流,與呼蘭分界,入呼蘭境。官商路四:東北駱家窩棚赴府;西大林家店赴省城;西南白家店至安達;南李春芳屯達蘭西。又縣南呂馬店、東南何小懷屯,為省城東路站道,由此赴興京。拜泉縣繁,難。府西北一百六十里。原名巴拜泉,即那吉泊,土名大泡子,縣以此名。光緒三十二年置,隸。三十四年改府,仍隸。東:通肯河自府境入,南流,與府分界。右受七道、六道、五道、四道、三道、二道、頭道溝,入青岡。北:胡裕爾河自訥河入,受印京河,西入龍江。南:雙陽河,東逕縣南,又東瀦為松津泊。巴拜泉在雙陽河南,其東南白水泉。西南:馬鞍泊、白華泊,皆平地出泉,可供汲飲,故有巴拜之稱。巴拜即「寶貝」轉音也。官商路四:東南三道溝赴府;東北李喜屯達訥河之三站,即新開海布道;西孔家地房赴省城;南菜富屯至青岡。胡裕爾河北岸有莽鼐牧場。有額魯特依克明安公府。

嫩江府:省東北四百五十里。即墨爾根副都統城。明為木里吉衛,譯即墨爾根。康熙十年,墨爾哲勒氏屯長來歸,編為墨爾根四十佐領,號新滿洲是也。光緒三十四年,以墨爾根城改置府。廣四百餘里,袤六百餘里。北極高四十九度十三分。京師偏東八度四十二分。府境為內興安嶺山脈三面環繞,嫩江縱貫其中,全境東西之水皆入嫩江,江出北伊勒呼里阿林,山脈自西而東橫亙處也。江以西山之著者,曰莽藍哈達七峰山、庫勒木爾山、穆克珠勒渾山、阿昆迪奇山、阿察特山、噶珊山、博里克山、達克固善山、東曰傅什霍山、伊勒賁孛山、勒吉勒圖山、勒吉爾山、達巴爾山、特克屯山、旺安山、圖墨爾肯山。嫩江導源東南流,逕格爾布爾山前,左受納約爾河、那昔臺河、霍吉格那彥河、額勒和肯河,右受伊斯肯。折南流,左受哈羅爾、阿魯三松哈諾勒、雅普薩臺、固巴諸河,右受喀柰、吉里克、喇都里、多布庫爾、歐肯諸河。又南屈西,江流灣環作二曲,又南謨魯爾河、和羅爾河自東來注之。又南逕府城西,又屈西,甘河自西北來注之,西南入訥河。舊設站五:自訥河博爾多站北四十三里至府屬喀木尼喀,又四十二里至依拉喀,又七十里至墨爾根,即府。又東北七十六里至科絡爾,又七十六里至喀勒塔爾奇,又東北接黑龍江城之庫穆爾。宣統元年,於兩城交界處增設陡溝子文報局。又由府北行,沿嫩江東岸,可達呼瑪金廠。卡倫九:曰諾敏河巴延和羅,曰甘河商河哈達,凡二;又和倫部曰塞楞山,曰喀穆尼峰,凡二;曰庫雨爾河,曰諾敏河,曰喀布奇勒峰,曰綏楞額山,曰布爾札木,凡五。府境為水陸通衢,沿江兩岸水土沃饒,屯地之腴,稍遜呼蘭,猶駕諸城而上。有多布庫爾協領,統鄂倫春人。

訥河直隸:省東北二百八十里。即布特哈東路總管。明,布兒哈衛。宣統二年,以東布特哈改置。廣一千一百餘里,袤七百餘里。北極高四十八度五十九分。京師偏東八度一分。東北:琉璜山、胡爾冬吉。東南:吉爾嘎爾哈瑪圖山。西:嫩江自嫩江府入,南流入龍江。東南:訥謨爾河。西北:合黑河烏德鄰池水,自東北來注。翁查爾河,自東南來注,折西,洪果爾津、芒柰、那彥、額勒合奇諸河,皆自北來注。保大泉河自東南來注。又西布拉克河,又羅洛河,皆自北來注,逕治南。又西,分二支入嫩江。東南:胡裕爾河導源胡耳山,西流入境。又西,左受印京河,右受敖倫河,入拜泉境,本索倫、達呼爾部落人打牲之所。光緒三十二年,始將南北荒段丈放。舊設站二:自龍江寧年站北八十五里至屬拉哈,又六十里至博爾多,即治。又北接嫩江喀木尼站。又東南頭、二、三站達海倫,即海布新道。舊卡倫五:喀爾開圖、烏爾布、齊吉爾吉、哈諾爾、溫托渾喀喇山。

璦琿直隸:省東北八百二十里。即黑龍江副都統城。明,考郎兀衛。光緒三十四年,以黑龍江城改置。璦琿兵備道駐。廣一千三百餘里,袤六百餘里。北極高五十度四分。京師偏東十一度。西:托列爾哈達、坤安嶺、大橫、樺皮、答儼、青泉山。南:札克達齊、博克里。東南:吉里爾哈達。黑龍江自黑河合烏克薩河入,南屈西,右受五道、四道、三道、二道諸溝,屈南,右受頭道溝,逕城東。又南,右合坤河,折東南,右合康達罕、霍爾穆勒津、博科里,左納伯勒格爾沁河,合博爾和里鄂模水,又東南合遜河,入興東。江東六十四屯在焉。精奇里江以南,黑龍江以北,東以光緒九年封堆為界,有伯勒格爾沁河、博爾和里鄂模,南北一百四十里,東西五十里至七十里,咸豐八年條約,本旗民永住之業。庚子之變。俄人違約驅奪,且擾及江右,脅耆民為官沈江者至數萬。和約成,光緒三十二年僅收回江右地,六十四屯迄未索還,今境僅西南北三鄉耳。有遜別拉荒段十餘萬晌,光緒末放墾。舊設站三:自嫩江之額勒塔爾奇東北八十五里至屬之庫木爾;又三十五里至額雨爾;又百里至黑龍江城,即治。此省城北路十站。又由南行至畢拉爾會海畢新道。又北穿森林達漠河,有新設霍爾莫津、奇勒克二卡倫。商埠,在城北頭道溝、二道溝間,光緒三十一年中日約開。按雍正中,舊設卡倫十三。咸豐八年,中、俄分江為界,如伊瑪畢拉昂阿、精奇里河、烏魯穆蘇丹、紐勒們河、黑龍混同兩江會口,五卡倫歸左岸俄境,而右岸境東增八、西增三。光緒十二年,以防護漠河金廠,增西爾根土哈達等二十三,接呼倫貝爾城之珠爾特依。又東南增車勒山、遜河、闊爾斐音河口、吉普遜河、提音河,凡五,共卡倫三十九處。庚子亂後,卡倫盡毀。遜河以南,劃歸興東道。三十四年,乃上自額爾古訥河口起,下迄遜河口止,新設卡倫二十:曰洛古河,曰訥欽哈達,曰漠河,曰烏蘇里,曰巴爾嘎力,曰阿穆爾,曰開庫康,曰安羅,曰依西肯,曰倭西門,曰安乾,曰察哈彥,曰望哈達,曰呼瑪爾,曰西爾根奇,曰奇拉,曰札克達霍洛,曰霍爾沁,曰霍爾莫津,曰奇克勒。每卡弁一、兵三十。五卡設卡官一,十卡設一總卡官。卡兵三十,以十人巡查,以二十人給荒墾種,更番輪替,所得糧即作弁兵津貼。地熟年豐,給地停餉。

黑河府:省東北九百里。原名大黑河屯。光緒三十四年置府,屬璦琿道。西:內興安嶺支山之著者,煙筒、白石、庫穆爾室韋山、額勒克爾山。黑龍江自北來,與俄分界,右受呼瑪爾河,入境。南至西爾伊奇卡倫,合丹河、寬河、奇拉、喀尼、庫倫、克魯倫、達彥、霍力戈必、法別拉、額尼、阿勒喀木諸河。又東逕城北,又南,左受精奇里江,右受烏克薩河,入璦琿。北呼瑪爾河,導源伊勒呼里山,南北四源,合而東流入境,有倭力克、庫勒郭里、綽諾、札克達齊河自西來注。又東呼爾哈,東入黑龍江。源委約七八百里,兩岸為庫瑪爾部貢貂之使馬鄂倫春人等漁獵處。南岸有呼瑪爾古城。府治舊為中、俄通商口岸,初時互市不及江海各口千分之一。分江以後,貿易遂繁。自彼銳意經營海蘭泡,又值庚子之變,華商趨附彼境,商務日興,而我驟減。然府治南屏璦琿,實邊防要沖。有法別拉荒段十餘萬晌,光緒三十四年放墾。官商路一:南八十里至愛琿城。餘皆水路,附俄輪以往。有新設卡倫四:曰西爾根奇,曰奇拉札克達,曰霍洛,曰霍爾沁。

呼倫直隸:省西北八百六十里。即呼倫貝爾副都統城。古室韋國。有室韋山。明屬朵顏三衛。光緒三十四年,以呼倫貝爾城改置。呼倫兵備道駐。廣一千一百餘里,袤一千六百里。北極高四十九度三十五分。京師偏東二度二分。內興安嶺在東。山脈自索嶽爾濟山北走,為伊勒呼里阿林,乃旋而東,餘脈西絡海拉爾河南北岸;額爾古訥河右岸為境,諸水源此。海拉爾河出嶺西麓,西逕綽羅爾,北察爾巴奇山,南納都爾、西札敦,又西特諾克,又西伊敏河,同來注。逕城北,合墨爾根河,入臚濱。西北合額爾古訥入室韋。北:根河西受鄂羅諾爾諸河,又西入額爾古訥河。南有達爾彬池,哈爾哈河出,西匯為貝爾池。烏爾順河自池出,北入呼倫池。境為索倫、新巴爾虎、厄魯特、陳巴爾虎諸旗牧場。又海拉爾河北有託河路協領,統鄂倫春人。舊設臺八:自西布特哈之牙爾伯克臺西五十里至之依爾克特,又五十里呼耳各特伊,又五十里舒都克依,又六十里牙克薩,又五十里哈拉合碩,又六十五里札拉木太,又五十二里哈克鄂模,又六十里呼倫貝爾城,即治。為省城西路十七臺。庚子之變,臺站毀,往來皆由東清鐵路。又西南三百二十里布野圖布爾都之野壽寧寺,道出張家口。寺北八里有大市場,歲八月,內外蒙古走集焉。新設卡倫三:曰孟克錫里,曰額爾得尼托羅輝,曰庫克多博,為總卡倫。西南有珠爾博特鹽池。東清鐵路自臚濱入境,逕城北,入西布特哈境。有完工、烏古諾爾、海拉爾、哈克、札爾木、牙克什、免渡河、烏諾爾、伊立克都九車站。商埠,光緒三十一年十一月中日約開。按呼倫沿邊卡倫,自雍正五年與俄勘界,設察汗敖拉、蘇克特依等卡倫十有二,名外卡倫。十一年,復於外卡倫以內設庫里多爾特勒、墨勒津等卡倫十有五,與各外卡倫相距一二百里不等,名曰內卡倫。咸豐七年,因內外相距遠,量為遷移,各三四十里,以便互巡。改三卡為三臺,另立新名,後並圮廢無考。光緒十年,防俄人越界挖金,由黑龍江城於呼倫珠爾特依卡倫北沿額爾古訥右岸,增莫里勒克等五,前後共外卡倫十有七。庚子並毀。三十四年,重行整頓,首塔爾巴幹達呼山,訖額爾古訥河口,復設二十有一,沿舊名者十有五,新命名者六:曰塔爾巴幹達呼,曰察罕敖拉,曰阿巴該圖,以上屬臚濱;曰孟克西里,曰額爾得尼托羅輝,曰庫克多博,庫克多博為總卡倫,以上屬呼倫;曰巴圖爾和碩,曰巴雅斯胡郎圖溫都爾,曰胡裕爾和奇,曰巴彥魯克,曰珠爾特依,曰莫里勒克,曰畢拉爾河,曰牛爾河,曰珠爾干河,珠爾干河為總卡倫;曰溫河,曰長甸,曰伊穆河,曰奇乾河,曰永安山,曰額勒哈達,以上屬室韋。先是俄人越界墾地刈草,至是驅逐,呼倫設邊墾總局,臚濱設分局,俄人遵章納稅,華人領票經商者,絡驛不絕。此光緒三十四年冬月事也。又呼倫西南舊十六卡倫,凡以防喀爾喀也。

臚濱府:省西北一千一百六十里。原名滿洲里,為東清鐵路入中國第一車站。光緒三十四年,初擬設滿珠府,後更名,屬呼倫道。東:額爾古訥河自呼倫入,西北流,至近阿巴該圖山,分二派。一西南流,為達蘭鄂洛木河,入呼倫池。其正流由山西東北流,為額爾古訥河。流至此作大轉折,如人曲腰以手遞物。額爾古訥,蒙古語謂以手遞物也。海拉爾河轉為額爾古訥河,分二汊,一沿東岸流,曰海拉爾河口,一沿西岸流,曰額爾古訥河,北行復合為一,入黑龍江。自阿巴該圖山以下為中、俄界水,康熙二十八年,尼布楚約立界碑。克魯倫河自喀爾喀部入,達蘭鄂洛木河自海拉爾河分出,均入呼倫池,瀦而不流,故呼倫為咸水湖。東南有烏爾順河,自貝爾池出,北流入呼倫池。其右岸為呼倫境,有新巴爾虎各旗牧場。舊設中、俄國界鄂博六:曰塔爾郭達固,曰察罕烏魯,曰博羅托羅海,曰索克圖,曰額爾底裡托羅海,曰阿巴哈依圖,此為庫倫東中、俄界第六十三鄂博。雍正五年恰克圖約鄂博止此。塔爾巴幹達呼山西南即喀爾喀界,有滿、蒙文界碑,系呼倫與喀爾喀分界,十年一換。有新設卡倫三:曰塔爾巴幹達呼,曰察罕敖拉,曰阿巴該圖。北有金源邊堡。東清鐵路自俄薩拜喀勒省入中國境,逕府治東,入呼倫。有滿洲里,咱剛,扎賚諾爾,赫勒洪德四車站。商埠,中日約開。有海關。

興東道:省東北一千五百里。明為黑龍江地面,及速溫河衛、真河所等地。光緒三十二年,移綏化城之綏蘭海道駐內興安嶺迤東,更名興東兵備道,專辦墾務、林、礦各事宜。三十四年,建署托蘿山北,為道治。領縣二。內興安嶺脈自璦琿入,南行為嫩江與黑龍江之分水嶺,至海倫東北迤東為黑龍江與松花江之分水嶺,曰布倫山,曰佛斯亨山,盡於黑龍、松花兩江匯處,謂之小興安嶺。黑龍江自璦琿合遜河入境,東南流,科爾芬河上源曰額爾皮河,又東南,右受噶其河,西都里、古勒庫拉、畢罕嘎、其達、莫里、烏雲諸河,自西南來注。又東南,右受佳勒河、輔河,屈南,嘉廕河自西來注。又南逕道治東而南,有秋臺河自西曲折來注。屈東,右受斐爾法鄂模水、布占河、伊里河,會松花江。北有遜河,東流有占河,右合阿爾沁,匯入黑龍江。其左岸為璦琿境。西:都魯,又西湯旺,右合伊春札里河,又西巴蘭河,東流屈南,皆入湯原。道治瀕黑龍江右岸,與俄屯松由子隔江對峙。西北:占河、遜河匯流,上段有畢拉爾、鄂倫春協領。鄂倫春本打牲部落,不識文字稼穡,為俄人誆誘。光緒末年,始議收撫。興東道兼署協領,創設墾務局、學堂。興安嶺嶺西有龍門鎮,黑龍江南岸有兆興鎮、裕興鎮,墾務皆盛。官商路三:舊有由齊齊哈爾至觀音山路;光緒三十四年,新開自興東逕煙筒山赴湯原,為西南路;又由觀音山經湯原境至三姓,為西路。宣統二年,新開海畢道,可由畢拉爾達海倫。大通縣道治西南五百二十里。原為崇古爾庫站,吉林江北五站之一。光緒三十一年置,為吉林依蘭屬縣。三十四年改隸。北有內興安嶺山脈縈帶,南皆平野。南:松花江自木蘭入,東流迤東北入湯原,其右岸為吉林方正。西:岔林、小橋子、富拉渾、頭道、二道、三道、四道沙河、轉心湖、二道河子,皆南入松花江。二道河子右岸為木蘭境,東有大通河,縣以此名。又東烏拉琿、大古洞、小古洞河,亦南入松花江。小古洞河左岸為湯原境。烏拉琿河西流,匯為二泊,曰三捷泡,曰二龍潭,泊旁地肥饒。站路一。乾隆二十七年,吉林借江北地設五站,由今賓州渡江東行入縣,曰佛斯心互,曰富拉渾,曰崇古爾庫,即縣治,曰鄂爾國木索,又東接今湯原之妙噶山站,以達三姓城。光緒末,各站改隸,皆設文報局。湯原縣道治西南三百五十里,明,屯河衛。屯河即湯旺河,光緒三十一年置,為吉林依蘭屬縣,三十四年改隸。北當小興安嶺山脈南麓,南近松花江,地坦平。松花江自大通入,東北流,逕縣治東,會黑龍江。松花江在縣境流甚曲,岸樹深雜,航路如蚓行。其右岸為吉林依蘭、富錦、臨江。南:湯旺河自興東入,南流,受如意河,窪丹、蘇拉巴蘭、小古洞河,皆東南流入松花江。小古洞河右岸為大通境。東北香蘭,西半節、赫金、各節、花爾布、阿凌達、鶴立諸河,左合梧桐、蒲鴨、額勒密十二入代河,皆東南入松花江。黑龍江有沱流決出,入松花江,西小黑河入之。港汊縈回,形同溝洫,為奧區上腴。有高家屯巡司。宣統二年,置額勒密河招徠鎮,有東益公司,鶴立河有興東公司,皆營墾務。縣境自西南至東北,狹長千餘里,若盡開闢,可設十縣。西南稍繁庶,東北權輿而已。站路自妙噶山站渡江至三姓,又有自興東煙筒山達縣西南,自觀音山歷縣境至三姓之西路。光緒末,新開有各節河、窪丹河文報局。

肇州直隸:繁,難。省東南六百里。明,撒察河衛,即三岔河衛。光緒三十二年,以郭爾羅斯後旗墾地置。南:松花江自吉林伯都訥入境,匯嫩江,東流,受博爾古哈泊水,逕城南,又東受蓮花泊水、下代吉船口水、三道岡子水、澇洲船口水,入呼蘭。右岸為吉林新城、雙城境。西:嫩江自安達入,南流,受烏蘭諾爾水,注松花江。右岸為大賚境。境平曠,北城泡南出匯為差達瑪泊,下流瀦於沙。東北有肇東分防經歷。舊設站三:自安達之他拉哈站南至之古魯,又南至烏蘭諾爾,又南至茂興,此南路十站。又東南路八臺,在境者四。自茂興站起,東至波爾吉哈臺,又東至察布奇爾,又東至鄂你多圖,又東至布拉克,又東入呼蘭境。官商路一:自茂興西至郭爾羅斯公府,又西由八家船口渡嫩江入大賚。東北五站。商埠,西南信宿岡子,伯都訥、哈爾濱適中地,沿江要沖,光緒末勘留商埠。東清鐵路自安達入境,逕東北入呼蘭。有酣草岡、滿溝二車站。

大賚直隸:沖,疲,難。省西南二百一十里。古靺鞨、室韋交界。明,洮兒河衛及卓兒河地面。光緒三十年,以札賚特旗莫勒紅岡子墾地置。北有索倫山脈,蜿蜒數百里,境內東流之水皆導源焉,所謂索倫圍場也。東:嫩江自龍江入,南流,匯松花江。其左岸為安達、肇州境。北:洮爾河自奉天東鎮入,東流,匯為納藍撒藍池,猶言日月池也,下流入嫩江。又北瑚爾達河、綽爾河、雅爾河,皆東南流入嫩江。雅爾河左岸為龍江境。北:塔子城、景星鎮分防二經歷。舊有蒙古站二:自卜魁站起,西至綽爾河,曰哈代罕站,曰綽羅站。又入今奉天之克爾蘇臺站。官商路三:一北出景星鎮赴省城;一東渡嫩江接茂興站;一西由二龍鎖口入奉天境,歷鎮東、靖安達洮南。嫩江沿岸哈喇和碩,有陸軍退伍兵屯田,一夫授田百畝,以火犁耕種。

安達直隸:沖,繁,難。省東南二百八十里。諳達,蒙古官名,無正譯。光緒三十二年,以杜爾伯特旗墾地置,又分省屬墾地益之。西:嫩江自龍江入,南流入肇州。北:九道溝水,西流與龍江分界,屈南入境,匯為納赫爾泊,西南:烏克吐泊,下流入嫩江。南:青肯泊,泊形如環,中有灘地,半隸肇州。放墾區域,大都在嫩江東岸及東南北三面沿邊,中部平原無河流,間有積潦,土含咸質,不宜種植。舊設站三:自龍江之特木德赫站南至屬之溫托歡,又南至多耐,又南至他拉哈,又南入肇州。官商路四:一北由林家店、九道溝赴省;一東南入呼蘭,有小林家店文報局;一東由長安堡赴青岡;一西越東清鐵路安達站至杜爾伯特貝子府,又西接多耐站。產堿,有咸廠二十五處。西北璫奈屯有鹽灘。東清鐵路自龍江入境,斜貫中部,逕治西南入肇州。有煙土屯,小河子,喇嘛甸,薩勒圖,安達五車站。

◎附志

林甸縣:光緒三十四年,擬設治林家甸,隸龍江府。在龍江府東南,安達西北。東清鐵路迤北,當省城東路孔道。光緒三十三年改訂東清鐵路合同,收回公司射占地畝,設縣墾闢。西九道溝子、東戚家店,皆東路所經,如臺站然。由此入呼蘭達興東。

諾敏縣:光緒三十四年,擬設治隸嫩江府。在嫩江府西,諾敏河東岸庫如爾其屯。西岸都克他耳屯有尼爾吉山,諾敏河上游札克奇山西有牧場,沿河有山路出呼倫。由縣南行,經西布特哈,渡嫩江,達拉哈站。

通北縣:光緒三十四年,擬設治海倫府北,通肯河北、胡裕爾河南。西:七道溝自胡裕爾河分出,南注通肯河,東至內興安嶺麓,與興東道龍門鎮界,北接訥河,即海布道所出。通肯河瀕岸多森林,土人呼曰樹川。

鐵驪縣:光緒三十四年,擬設治海倫府東南、餘慶東鐵山包。東至金牛山、興東道界。南大青山,東興鎮界。西,鐵山包河,北,依吉密河,並餘慶縣界。呼蘭河出境東達裡代嶺,西入餘慶。有協領駐河北,管理旗丁屯田。以上二縣隸海倫府。

布西直隸:光緒三十四年,擬設治西布特哈,在省城東北二百八十里嫩江西岸。西有內興安嶺,與呼倫分界。西南即索倫圍場。西北諾敏河,至南入嫩江。西有阿倫河、音河、雅爾河,皆東南入龍江。又西迤南,綽爾河入大賚。舊設臺七:自龍江之那希奇臺東至之木爾楚袞臺,又東至赫尼昂阿,又東至和尼,又東至錫伯爾,又東至巴林,又東至嘎爾甘,又東至雅勒伯霍托,又東入呼倫境。境少平原,森林之利獨饒。有土城,因起伏西去數千里,直至木蘭圍場,又西至歸化城。往時流人亡去不識途,多循此入關,蓋即金源時長城汪古部所居者也。東清鐵路自呼倫鑿興安嶺入境,橫貫中部,入龍江。有興安嶺、博爾多、雅魯、巴里木、哈拉蘇、札蘭屯、成吉思汗七車站。

甘南直隸:光緒三十四年,擬設治富拉爾基,在省城西南嫩江西。有雅爾河支津。北有東清鐵路庫勒站。由此渡嫩江達昂昂溪。富拉爾基屯開闢最先,生聚日繁,蓋鐵路交通之效。

武興直隸:光緒三十四年,擬設治多耐站,在省城南二百零五里,嫩江東路四五里,與溫托歡、他拉哈兩站首尾相接。南北長,東西窄,成一半規長梭形。向為杜爾伯特旗境。光緒三十二年,設局放荒五萬六千四百餘晌。

呼瑪直隸:光緒三十四年,擬設治西爾根卡倫。宣統二年,試辦設治,移呼瑪爾河口北岸,隸璦琿道,在道治北五百餘里。東:黑龍江。呼瑪爾河出伊勒呼里山,內興安山脈向北行者也。東行者伊勒呼里阿林,四源,合東窩集、倭勒克、庫勒都里、綽羅呼爾吉、布列斯,屈南,右受札克達奇河,又東入黑龍江。黑龍江東流,逕安羅卡倫北,屈南流,下至呼瑪爾河口。沿西岸設卡倫六:曰依西肯,曰倭西們,曰安乾,曰察哈彥,曰望哈達,曰呼瑪爾。下游接西爾根卡倫。屬黑河府。瀕臨河口駐協領,統鄂倫春人。

漠河直隸:光緒三十四年,擬設治漠河,隸璦琿道。在道治西北千餘里。漠河出治雞察山,東北入黑龍江。南額穆爾河,東北流,左受吉里瑪那裡多什都克河,屈東流,右受大札丹庫爾、小札丹庫爾,入黑龍江。又南旁烏河,東南流,左受巴達吉察,右受札克達奇,屈東北,右受布爾嘎里河、沽裡乾河,入黑龍江。又南有呼瑪爾河上源,其南為伊勒呼里阿林,乃內興安嶺自西轉東橫幹脈也。山南即嫩江源,西有額爾古訥河入黑龍江口,為璦琿與呼倫兩屬交界,即中、俄以江為界之起處。沿黑龍江南岸設卡倫八:曰洛古河,曰訥欽哈達,曰漠河,即治,曰額蘇里,曰巴爾嘎力,曰額穆爾,曰開庫康,曰安羅。有木廠一處。黑龍江由此轉南流,安羅卡倫下游接西岸之依西肯卡倫,屬呼瑪。漠河金礦,光緒十四年經始開採,庚子入於俄,光緒三十二年始行收復。漠河為省北屏障,黑龍江轉運專落俄人之手。光緒三十四年,議由嫩江之源開闢山道,卒以工費浩繁中止,故礦業衰歇而設治亦難也。

室韋直隸:光緒三十四年,擬設治吉拉林,隸呼倫道。在道治北四百餘里,額爾古訥河右岸。對岸為俄臥牛、槐敖、洛氣等屯疆域。額爾古訥河自臚濱之阿巴該圖北流,至呼倫之庫克多博,東北流,合根河、特勒布爾、胡裕爾和奇、珠魯克圖即約羅、珠爾格特依、布魯、色木特勒克諸河,皆自東南山來注,此在吉拉林以南者也。中根河最大,出內興安嶺,西北流,兩岸沃野膏原,為殖民善地。額爾古訥河逕治西,又東北流,有哈拉爾即吉拉林河,眉勒喀即尼布楚約內之河、遜河、額爾奇木、畢拉爾、畢拉克產、古爾布奇、吉林子、阿木毗、牛爾、珠爾干、溫諸河,皆自東南山來注。額爾古訥河至是屈西北流,有烏瑪、大吉嘎達、小吉嘎達,復有小河入,皆自東南來注。再折而東北流,有伊穆河,復有小河二十餘,皆自東南來注。此在吉拉林以北者也。中牛爾河最大,出內興安嶺,河口左右有平地兩區,田土肥美。額爾古訥河自受根河、牛爾河,水大而急,直注黑龍江,而吉拉林為適中地,故治在焉。新設防邊卡倫,在境內者十有五,自庫克多博總卡倫以北,曰巴圖爾和碩,曰巴雅斯胡郎圖溫都爾,曰胡裕爾和奇,曰巴彥珠魯克,曰珠爾格特依,曰莫里勒克,曰畢拉爾河,曰牛爾河,曰珠爾干河總卡倫,曰溫河,曰伊穆河,曰奇乾河,曰永安山,曰額勒哈達。珠爾干、額勒哈達為鄂倫春與俄人交易之所。尤要道路自吉拉林南至塔爾巴幹達呼山七百餘里,其北至珠爾干河三百五十餘里,則小徑不通車馬。自珠爾干至額爾古訥河口五百五十餘里,則懸崖壁立,非假道於俄,不能飛越。根河口新立官渡,為華、俄商旅必趨之路。根河上源有道出西布特哈達墨爾根,額爾古訥民船祗達吉拉林,以下溜急,民船可順流而下,不能溯流而上,非輪船不為功。冬令,河上可駕駛冰橇,每一日夜行三四百里。

舒都直隸:光緒三十四年,擬設治免渡河,隸呼倫道,在道治東二百八十餘里。河出阿爾奇山,北合札郭河,入海拉爾河。東即內興安嶺。東清鐵路經南,有免渡河車站,由境鑿興安山洞入西布特哈境。

佛山府:光緒三十四年,擬設治觀音山,隸興東道,在道治北,瀕黑龍江岸。對岸為俄吉春屯,北有輔河,南有嘉廕河。附府治有小水曰十里河,皆東入黑龍江。

蘿北直隸:光緒三十四年,擬設治托蘿山北,隸興東道,附郭,如璦琿、呼倫兩直隸之比。

烏雲直隸:光緒三十四年,擬設治烏雲河,隸興東道,在道治西北,瀕黑龍江岸。對岸為俄嘎薩得報屯。烏雲河在西,北入黑龍江。

車陸直隸:光緒三十四年,擬設治車陸,隸興東道,在道治西北遜河南。原為車勒山卡倫,音轉為車陸。東臨黑龍江,對岸為俄吉滿屯。南科爾芬河,東北流入黑龍江。

春源直隸:光緒三十四年,擬設治伊春呼蘭河源,隸興東道,在道治西南。西有布倫山,伊春河出,東流入湯旺河。布倫山西麓即呼蘭河源。南札里河,東流,左合黃泥河、報達河,入湯旺河;又南巴蘭河源在焉。

鶴岡縣:光緒三十四年,擬設治鶴立岡,隸興東道,在湯原縣北、鶴立河西。有興東墾務公司,宣統中擬移駐黑龍、松花兩江匯流處,額勒密河東,地尤沃饒,為全省冠。


\end{pinyinscope}