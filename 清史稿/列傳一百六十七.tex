\article{列傳一百六十七}

\begin{pinyinscope}
陳若霖戴三錫孫爾準程祖洛馬濟勝裕泰賀長齡

陳若霖,字宗覲,福建閩縣人。乾隆五十二年進士,選庶吉士,散館授刑部主事,累遷郎中。束鹿縣民王洪中為人聚毆,訟不得直,自經死。若霖鞫得其實,被議敘。秩滿當外用,仍留部。數從大臣赴各省讞獄,以寬恕稱。嘉慶十三年,出為四川鹽茶道,擢山東按察使。調廣東,署布政使,以佐總督百齡平海盜,賜花翎。調湖北,復調四川,就遷布政使。二十年,擢雲南巡撫。水尾土州目黃金珠結內地奸民,殺副州目李文政,掠其家,鞫實,置於法。

歷廣東、河南、浙江巡撫。浙省南北新關科罰無度,限以半正額為止,恤商而課裕。修蕭山新廟堤,建盤頭以御潮。次年,新林塘圮,親往勘,疏言:「新林塘舊為險工,今距海日遠,塘以外為灶地,外復為牧地,中有馬塘,足為新林屏蔽,宜補築以遏潮汐。疏通灶地各溝洫,引入牧地之莫家等灣以排洩之,即以灶地之土培護新林堤基。西築橫塘以禦江水。責令灶牧各戶及蕭山、山陰、會稽三縣,分別修築。」又奏修會稽、上虞等縣塘堤,並如議行。二十四年,擢湖廣總督。湖南鳳凰等屯丁額多為官占,失業者眾,悉清釐發還徵租。官入苗寨多婪索,或冒名詐財,嚴禁之。又以屯地磽瘠租額重,為奏減苗租二萬餘石,免逋賦七萬餘石,苗民感之。

道光二年,調四川。中江覃萬典、犍為道士蕭來修等假神惑眾,捕誅首犯,不坐株連。九姓長官司不諳吏治,奏請考試,獄訟別由瀘州及州判兼理。四年,召授工部尚書,調刑部,兼管順天府尹事。文安縣地形如釜底,自道光初堤防沖決,積水不能耕種,議請急行修築。七年,命勘湖北京山黃家陵堤工,疏言:「下游災民籥請修治潰堤,上游居民謂口門下游乃襄河故道,復請廢之。河流經行二百餘年,舍此不由,而別尋二百年以前故道,其說殊謬。潛江、天門、漢川俱屬下游,而天門、漢川尤當沖要,何忍委之巨浸?惟有開通江流,堵合口門,因勢利導。胡家灣沙洲當下游之沖,以四十餘丈之地束全江之水,下壅上潰,理有必然。今洲已沖潰,乘勢挑濬新灘,展寬水道,使江流無沖突之患,然後增築京山、鍾祥口門堤壩,再於潰口築石壩二,以護堤攻沙,庶可經久。」報可。十二年,乞休歸,卒於途,賜恤。

戴三錫,順天大興人,原籍江蘇丹徒。乾隆五十八年進士,授山西臨縣知縣。連丁父母憂,嘉慶六年,服闋,發四川,補南充。歷馬邊、瓘邊兩通判,署資州、眉州、工⼙州,並有政聲。工⼙州民黃子賢以治病為名,倡立鴻鈞教,捕治之。事聞,仁宗命送部引見,擢茂州直隸州知州。歷寧遠知府、建昌道、四川按察使。道光二年,遷江寧布政使,回避本籍,仍調四川。三年,署總督,五年,實授,兼署成都將軍。

三錫自牧令洊陟封疆,二十餘年,未離蜀地。盡心民事,興復通省書院,增設義學三千餘所。四川舊有義田,積儲備賑,穀多則變價添置良田。三錫以歲久將膏腴多成官產,留穀太多,又虞霉變虧挪,差定三千至萬石為額。溢額者出糶,價存司庫,以備兇歲賑恤之用。又以蜀地惟成都附近俱平疇沃野,餘多山谷磽瘠,遇水沖塞,膏腴轉為砂石,因地制宜,多設渠堰,以資捍衛宣洩。新都奸民楊守一倡立邪教,造妖書惑眾,擒誅之。越巂生番劫奪商旅,掠漢民婦女,捕駔黠者數十人置之法,救出被掠男婦,給貲安撫。屢被詔褒獎。九年,因年老召來京,署工部侍郎。尋致仕,未幾,卒。詔嘉其「宣力有年,官聲素好」,贈尚書銜,依贈銜賜恤。

孫爾準,字平叔,江蘇金匱人,廣西巡撫永清子。嘉慶十年進士,選庶吉士,授編修。十九年,出為福建汀州知府。寧化民斂錢集會,大吏將治以叛逆。爾準訊無他狀,論誅首要,鮮所株連。歷鹽法道、江西按察使,調福建,就遷布政使。道光元年,調廣東布政使,擢安徽巡撫。河南邪匪邢名章等糾眾竄潁州,檄按察使惠顯率兵馳剿,格殺名章,殲其餘黨。蠲緩被災各屬,災甚者賑恤之。先是有言賑務積弊,毋得以銀折錢,爾準疏其弗便,仍循舊章。

三年,調福建巡撫。延、建各屬山徑叢錯,多盜劫,以萬金為緝捕費,連獲賊首置之法,盜風衰息。巡閱臺灣,疏言:「臺灣南北袤延千餘里,初抵鹿耳門,可行舟楫。嗣增設鹿仔港,而淺狹多沙,內山溪水赴海,別開港在嘉彰間,曰五條港,頗利商船。又噶瑪蘭山峻路險,負戴難行,其地有烏石港、加禮遠港,可通五六百石小舟,皆宜設為正口。」

五年,擢閩浙總督。奏請噶瑪蘭收入版藉,設官治理。彰化匪徒械斗焚劫,旁近蜂起,全臺震動,檄水師提督許松年剿捕,副將邵永福等趨艋舺,阻其北竄;總兵陳化成以兵渡鹿仔,防其入海。爾準親駐廈門,遣副將佟樞等分往彰化、淡水,搜山圍捕,詗知賊黨煽誘日眾,移陸路提督馬濟勝守廈門,自渡海駐彰化督剿,賊首李通遁,捕得伏誅。令各莊舉首事,緝餘匪,閩人捕閩人,粵人捕粵人,以免誣累。

臺人有與生番貿易遂娶番婦者,俗名「番割」,其魁黃斗乃等久踞三灣,潛出為盜。當亂起時,誘生番出山助鬥,遣參將黃其漢等分路偵擊。番竄後山,士卒攀藤躡葛而登,擒黃斗乃等二十一人,斬以徇。爾準疏陳匪徒起事,由於造謠焚掠,非叛逆,當以強盜論;淡水以北分黨報復,當以械鬥論;焚殺有據者始坐闢,餘俱末減。其脅從旋解散者,多所保全。又奏臺灣北路至艋舺幾五百里,僅有守備一員,巡防難周。調南路游擊一員駐竹塹,並於大甲、銅鑼灣、斗換坪諸處添駐營汛,改建淡水土城。頭道溪為生番出入總路,亦建土城,以屯丁駐守。事平,加太子少保。七年,入覲,宣宗嘉其治臺灣匪亂悉合機宜,迅速蕆功,賜其子慧翼官主事。

木蘭陂者,創自宋熙寧間,溉民田四十萬畝,築石堤千一百餘丈以御海潮,歲久傾壞,爾準道經莆田,親勘修復。工蕆,以宋長樂室女錢創陂實功首,建祠列入祀典。爾準治閩最久,諳悉其風土人情,吏民皆相習,政從寬大,閩人安之。九年,坐失察家僕收賄,鐫二級留任。十一年,以病乞休。逾年,卒,贈太子太師,賜子慧惇進士,慧翼員外郎,謚文靖,祀福建名宦及鄉賢祠。

程祖洛,安徽歙縣人。嘉慶四年進士,授刑部主事,洊遷郎中。諳練刑名,為仁宗所知。京察記名道府,久未外簡,以截取銓授甘肅平涼知府。部臣請留,詔斥規避邊遠,撤銷記名,留部永不外用。久之,擢內閣學士。尋授江西按察使,遷湖南布政使,調山東。

道光二年,擢陜西巡撫,調河南。教匪硃麻子由新蔡竄安徽阜陽,捕獲置之法。與直隸、山東、安徽、湖北毗連諸縣素多盜,撥庫帑五萬兩生息,為緝捕經費。漳水決安陽樊馬坊,河流北徙,命大學士戴均元往會勘。祖洛周歷上下游,合疏言:「漳水自乾隆五十九年南徙合洹以來,衛水為所遏,每致潰溢。今河流既分,不可使復合。議於樊馬坊上下距洹水最近處,及南岸沖決成溝,並築土壩,使二河分流,冀減漫溢之勢。」至四年春,積水消涸,地形顯露。田市之北,漫水與溝隔斷,不能引歸正河。乃就其上游龍家莊窪地抽溝啟放,復於內黃馬家窪開引河,添築田家營大壩,使溜勢南趨。自是漳、衛合並之患遂息。虞城橫河、惠民溝,夏邑巴清河,永城減水溝,舊為豫東宣洩潦水要區,迭經黃河漫淤,濱河連歲被災,並疏濬之。初,河南、安徽治捻匪從重典,嗣部議有所減改。祖洛疏言:「匪徒結捻,倡劫黨眾,一呼而集,其豫謀早在結捻之時。新例以是否豫謀分別輕重,諸多窒礙,請復舊例。」並論匪徒拒捕及捕人治罪各條。又言:「獲盜究出舊案,免究從前失察處分。請遵嘉慶間諭旨,俾除瞻顧。」並從之。

七年,丁母憂,服闋,署工部侍郎。尋署湖南巡撫,調江蘇。十二年,擢閩浙總督。命查辦浙江鹽務,嚴定裁汰浮費章程,下部議行。臺灣奸民張丙、陳辦等倡亂,命將軍瑚松額督兵進剿,祖洛專治後路軍需。十三年,提督馬濟勝破賊,張丙等就擒,赴臺灣籌辦善後事宜,劾戰守不力之都司周進龍等,褫黜有差。改營制,增防守。優敘,賜花翎。疏陳福建吏治,略曰:「安民必先懲蠹,不可以回護瞻顧而曲縱奸惡。閩省吏治無子惠之政,而務寬大之名,始因官之庸劣,釀成頑梗之風,今又因民之譸張,遂有疲難之勢。官曰民刁,民曰吏虐,互相傳播,漸失其真。官不執法,幕不守法,因而愚民犯法,書役弄法,棍徒玩法。必先懲不執法之官,然後能治犯法、弄法、玩法之人。」於是連劾官吏不職者,略無假貸,吏治始肅。已革縣丞秦師韓京控提督馬濟勝矇奏邀功,並訐祖洛偏袒欺蒙,命侍郎趙盛奎偕學攻張鱗按鞫,白其誣,師韓遣戍新疆。十五年,疏陳閩洋形勢,以漳州之南澳、銅山為籓籬,泉之廈門、金門為門戶,興化之海壇為右翼,閩安為省會咽喉,福寧之銅山為後戶。巡緝守御,全資寨城砲臺。就最要者四十四處,由官民捐貲修築。十六年,丁父憂去官,服闋,引疾不出。二十八年,卒,宣宗甚惜之,贈太子太保,謚簡敬。

馬濟勝,山東菏澤人。以武生入伍,從剿川、陜教匪,積功累擢江蘇撫標參將。嘉慶十八年,會剿山東教匪,擢河北鎮總兵。道光初,擢浙江提督,調福建陸路提督。張丙等倡亂嘉義,臺灣鎮總兵劉廷斌困守孤城。濟勝率兵二千渡海赴援,戰於嘉義城下,大破賊,追至蘋港尾,擒斬甚眾;進屯鹽水港,分兵搜剿,張丙及其悍黨先後就擒。時命將軍瑚松額督師猶未至,詔褒成功迅速,賜雙眼花翎。餘匪萬餘復來犯,俟其怠,擊之大潰,擒頭目賴滿等,追剿盡毀其巢,賊遂平。宣宗深嘉其謀勇,錫封二等男爵。又以馭兵安靖,御書「忠勇廉明」四字賜之。召入覲,年逾七旬,猶壯健,溫詔褒獎,晉二等子爵,在御前侍衛上行走。十六年,卒於官,贈太子太保,謚昭武,四子皆予官。

裕泰,滿洲正紅旗人。由官學生考授內閣中書,遷侍讀。嘉慶末,出為四川成綿龍茂道,歷四川、湖南、安徽按察使,湖南、陜西、安徽布政使。道光十一年,擢盛京刑部侍郎,調工部,兼管奉天府尹事。查勘科爾沁蒙旗荒地,奏禁私墾。十三年,召授刑部侍郎,尋出為貴州巡撫。十六年,古州、黎平土匪起,擒其渠徐玉貴等誅之。

調湖南巡撫。鎮筸標兵滋事,劾總兵向遵化、辰沅道常慶不職,罷之。疏言:「苗疆屯田,嘉慶中道員傅鼐所經營,寓兵於農,籌邊良策。治安日久,諸弊叢生。今鎮筸標兵因借餉倡亂,苗人遂生觀望。重以苗官苛刻,屯長侵欺,後患堪虞。急應清釐損益,妥定章程,俾將弁兵練咸知經費有常,絕其覬覦,仍責成鎮道實力整飭,恩威並行。」尋議定苗疆兵勇不準客民充補,預借銀穀限以定制,拔補備弁屯長,嚴絕苞苴。辰沅道缺,以湖南知府題升。並如所議行。十七年,調江西,復調湖南。

二十年,擢湖廣總督。二十一年,湖北崇陽逆匪鍾人傑作亂,踞縣城,陷通城。裕泰馳駐咸寧,檄按察使郭熊飛率都司玉貴等進剿。崇陽在萬山中,賊盡塞孔道,築砦抗拒,選精銳出賊後夾攻,分股犯蒲圻,連為官軍所敗,踞崇陽西嶺為負嵎計。提督劉允孝迭敗之石盤山、黑橋,進毀其巢,擒人傑及其黨陳寶銘、汪敦族等。尋復通城,盡俘其孥。事平,加太子太保,賜雙眼花翎。時英吉利兵由海入江,詔募勇習水戰。裕泰仿粵艇造大船六、快船四,簡漢陽水師,每船百人,按旬操練。裁舊有巡船,以節經費。荊州駐防每出營滋事,奏請飭地方官拘拿,報將軍秉公嚴懲。乾州苗竄擾,剿撫解散。

二十九年,李沅發倡亂新寧,踞城戕官。巡撫馮德馨、提督英俊往剿,復縣城。妄傳沅發已死,而賊竄山中,勾結黔、粵交界伏莽,勢益蔓延。馮德馨逮治,專任裕泰往督師,與黔、粵諸軍合擊,數捷。三十年春,搜剿山內,擒殲多名。賊竄永福草奚塘,四面抄圍,漸窮蹙。裕泰度賊不南趨廣西全州,即入新寧瑤峒,令提督向榮由武岡進屯廣西懷遠,遇賊擊破之。賊退踞金峰嶺,分三路進擊於深箐陡石間,斬獲殆盡,沅發就擒,晉太子太傅。尋調閩浙總督。咸豐元年,調陜甘,入覲,卒,優詔以尚書例賜恤,謚莊毅。子長善,廣州將軍;長敘,侍郎。

賀長齡,字耦耕,湖南善化人,原籍浙江會稽。高祖上振,官湖南司獄,血⼙囚有隱德,貧未能歸,遂家湖南。

長齡,嘉慶十三年進士,選庶吉士,授編修,遷贊善。道光元年,出為江西南昌知府。歷山東兗沂曹濟道、江蘇按察使,就遷布政使,佐巡撫陶澍創行海運。調山東。七年,署巡撫。臨清州教匪馬進忠為逆伏誅,復有揭帖偽立名號,刻期舉事,臚列旁州縣民名數百。長齡曰:「謀不軌詎以姓名月日告?此移禍也。」詗知果出邀功者,欲藉興大獄,遂置不問。調江寧布政使,乞歸養親。十五年,母喪服闋,補福建布攻使,調直隸。

十六年,擢貴州巡撫。黔民苦訟累而多盜,以聽斷緝捕課吏,設旬報為考覈。十八年,仁懷奸民穆繼賢糾四川綦江匪肆劫,遣兵與川軍會剿,焚其巢,首從並就殲擒。郎岱、普安、清鎮諸縣多種罌粟,拔除申禁,勸民種木棉,玉屏、婺川皆有成效。黔省安置流犯三千餘人,與苗民錯處,釁隙易生,疏請改發新疆;又以鎮遠、黎平、都勻、古州苗俗桀驁,以盜為生,州縣差役緝捕難周,疏請綠營每百名內精選數名,分隸府、、州、縣文員管轄,勤加訓練,專司捕盜:並下部議行。

長齡治黔九載,振興文教,貴陽、銅仁、安順、石阡四府,普安、八寨、郎岱、松桃四,黃平、普定,天柱、永從、甕安、清平、興義、普安諸州縣,皆建書院義學;省會書院分上內外三舍,親試考覈,刊刻經籍,頒行州縣。

二十五年,擢雲貴總督,兼署云南巡撫。漢、回連歲互斗,永昌回變敗退後,復圖攻城,城回謀內應,迤西道羅天池悉捕誅之。長齡親往督剿,擊走叛回,以肅清入告。二十六年,回眾藉口善良不別,復叛,自請議處,撤銷獎敘,赴大理、永昌督剿。匪尋竄散,請免投誠張富罪,軍犯王芝異團練出力,亦請釋回。詔斥其庸懦,降補河南布政使。二十七年,乞病歸。滇回復擾雲州,多屬永昌遺孽,且得羅天池濫殺狀,追論長齡,褫職。逾年,卒。

論曰:陳若霖、戴三錫盡心民事,而三錫久任蜀疆,治效較多。孫爾準、程祖洛先後治閩有聲,寬嚴殊途,其相濟之道乎?裕泰兩殄楚寇,勛施爛然。賀長齡儒而不武,不足以奠巖疆也。


\end{pinyinscope}