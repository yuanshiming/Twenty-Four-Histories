\article{本紀二}

\begin{pinyinscope}
太宗本紀一

太宗應天興國弘德彰武寬溫仁聖睿孝敬敏昭定隆道顯功文皇帝,諱皇太極,太祖第八子,母孝慈高皇后。上儀表奇偉,聰睿絕倫,顏如渥丹,嚴寒不慄。長益神勇,善騎射,性耽典籍,諮覽弗倦,仁孝寬惠,廓然有大度。

天命元年,太祖以上為和碩貝勒,與大貝勒代善、二貝勒阿敏、三貝勒莽古爾泰為四大貝勒。上居四,稱四貝勒。

太祖崩,儲嗣未定。代善與其子岳託、薩哈廉以上才德冠世,與諸貝勒議請嗣位。上辭再三,久之乃許。

天命十一年丙寅九月庚午朔,即位於沈陽。詔以明年為天聰元年。初,太祖命上名,臆制之,後知漢稱儲君曰「皇太子」,蒙古嗣位者曰「黃臺吉」,音並闇合。及即位,咸以為有天意焉。

辛未,誓告天地,以行正道,循禮義,敦友愛,盡公忠,勖諸大貝勒等。甲戌,諭漢官民有私計遁逃及令奸細往來者,雖首告勿論,後惟已逃被獲者論死。丙子,諭曰:「工築之興,有妨農務,前以城郭邊墻,事關守御,有勞民力,良非得已。茲後止葺頹壞,不復興築,俾民專勤南畝。滿洲、漢人,毋或異視,訟獄差徭,務使均一。貝勒屬下人,毋許邊外行獵。市稅為國費所出,聽其通商貿易,私往外國及漏稅者罪之。」丁丑,令漢人與滿洲分屯別居。先是漢人十三壯丁為一莊,給滿官為奴。至是,每備禦止留八人,餘悉編為民戶,處以別屯,擇漢官廉正者理之。設八固山額真,分領八旗。以納穆泰為正黃旗固山額真,額駙達爾漢為鑲黃旗固山額真,額駙和碩圖為正紅旗固山額真,博爾晉為鑲紅旗固山額真,額駙顧三泰為鑲藍旗固山額真,托博輝為正藍旗固山額真,徹爾格為鑲白旗固山額真,喀克篤禮為正白旗固山額真。又設十六大臣,贊理庶政,聽八旗訟獄。又設十六大臣,參理訟獄,行軍駐防則遣之。乙未,蒙古科爾沁土謝圖汗奧巴遣使來吊。

冬十月己酉,以蒙古喀爾喀札魯特部敗盟殺掠,私通於明,命大貝勒代善等率精兵萬人討之,先貽書聲其罪,上送至蒲河山而還。癸丑,別遣楞額禮、阿山率輕兵六百入喀爾喀巴林地,以張軍勢。丙辰,科爾沁土謝圖汗奧巴及代達爾漢等十四貝勒各遣使來吊。達硃戶徵卦爾察部,獲其人口牲畜以歸。明寧遠巡撫袁崇煥遣李喇嘛及都司傅有爵等來吊,並賀即位。甲子,大貝勒代善等大破札魯特,斬其貝勒鄂爾齋圖,獲貝勒巴克及其二子並拉什希布等十四貝勒而還。

十一月辛未,上發沈陽迎大貝勒代善,師次鐵嶺樊河界。癸酉,行飲至禮,論功,頒賚將士。戊寅,上還沈陽。察哈爾阿喇克綽忒部貝勒圖爾濟率百戶來歸。乙酉,遣方吉納、溫塔石偕李喇嘛往報袁崇煥,且遺書曰:「頃停息干戈,遣使吊賀,來者以禮,故遣官陳謝。昔皇考往寧遠時,曾致璽書言和,未獲回答。如其修好,答書以實,勿事文飾。」崇煥不以聞,而令我使齎還。卓禮克圖貝勒之子衛徵巴拜手巂其家屬來歸。科爾沁貝勒青巴圖魯桑阿爾齋、臺吉滿珠什哩各齎鞍馬牛羊來吊。

十二月庚子,禁與蒙古諸籓售賣兵仗。壬戌,黑龍江人來朝貢。

天聰元年春正月丙子,命二貝勒阿敏,貝勒濟爾哈朗、阿濟格、杜度、嶽言乇、碩托率兵征朝鮮。上曰:「朝鮮累世得罪,今明毛文龍近彼海島,納我叛民,宜兩圖之。」復遣方吉納、溫塔石遺書明袁崇煥,言興師由七大恨,並約其議和,及每歲餽報之數。

二月己亥,以書招諭蒙古奈曼部袞出斯巴圖魯。

三月壬申,阿敏等克朝鮮義州,別遣兵搗鐵山,明守將毛文龍遁走。又克安州,進至平壤城,渡大同江。朝鮮國王李倧遣使迎師。阿敏等數其七罪,仍遣使趣和。倧懼,率妻子遁江華島,其長子李★遁全州。阿敏復遣副將劉興祚入島面諭倧。倧遣其族弟原昌君李覺獻馬百匹、虎豹皮百、錦苧各四百、布一萬五千。庚子,與朝鮮盟,定議罷兵。壬申,明袁崇煥遣杜明忠偕方吉納等以書來,並李喇嘛書,欲釋恨修好。惟請減金幣之數,而以我稱兵朝鮮為疑。辛巳,阿敏等遣使奏捷。乙酉,命留滿洲兵一千、蒙古兵二千防義州,滿洲兵三百、蒙古兵一千防鎮江城。並諭李倧曰:「我留兵義州者,防毛文龍耳。」阿敏等旋師,以李覺歸。

夏四月甲辰,遺袁崇煥書曰:「釋恨修好,固所原也。朝鮮自尊輕我,納我叛亡,我遲之數年,彼不知悔,是以興討。天誘其衷,我軍克捷。今已和矣,而爾詭言修好,仍遣哨卒偵視,修葺城堡。我國將帥,實以此致疑。夫講信修睦,必藉物以成禮,我豈貪而利此,使爾國力不支?可減其半。歲時餽答,當如前議,則兩國之福也。」書成,聞崇煥方築塔山、大凌河、錦州等城,遂罷遣使,而以書付杜明忠還。更責崇煥曰:「兩國修好,當分定疆域。今又修葺域垣,潛圖侵逼。倘戰爭不息,天以燕、雲畀我,爾主不幸奔竄,身敗名裂,為何如也。自古文臣不更事者徒為大言,每喪師殃民,社稷傾覆。前者遼左任用非人,而河東西土地盡失,今尚謂不足戒而謀動干戈耶?」癸丑,阿敏等自朝鮮凱旋,上迎於武靖營,賜阿敏御衣一襲,餘各賜馬一匹。乙卯,論征朝鮮將士功,擢賞有差。戊辰,上還沈陽。乙丑,以書諭察哈爾臺吉濟農及奈曼袞出斯巴圖魯來和。

五月戊辰,遣朝鮮國王弟李覺歸國,設宴餞之,並賜鞍馬裘帶等物。辛未,上聞明人於錦州、大凌河、小凌河築城屯田,而崇煥無報書,親率師往攻之。乙亥,至廣寧,乘夜進兵。丙子,明大凌河、小凌河兵棄城遁,遂圍錦州。明臺堡兵二千餘人來降,悉縱之歸。丁丑,明鎮守遼東太監紀用、總兵趙率教遣人詣師請命。上開誠諭之,並許紀用親來定議。用不答,遂攻錦州。垂克,明援兵至,退五里而營,遣人調沈陽兵益師。庚寅,固山額真博爾晉等以兵至。癸巳,攻寧遠城,殲其步卒千餘人。既,明總兵滿桂出城而陣,上欲擊之,三大貝勒均諫止。上怒,趣諸將戴兜鍪,率阿濟格疾馳而進,敗其前隊,追至寧遠城下,盡殪之。諸貝勒不及胄而從,濟爾哈朗、薩哈廉、瓦克達俱被創。錦州守兵亦出城合戰,我軍復迎擊之。游擊覺羅拜山、備禦巴希陣歿,上臨其喪,哭而酹之。我軍還駐雙樹鋪。乙未,復至錦州。

六月己亥,攻錦州,值天溽暑,士卒死傷甚眾。庚子,班師。丁未,上還沈陽。是歲,大饑,斗米值銀八兩,銀賤物貴,盜賊繁興。上惻然曰:「民饑為盜,可盡殺乎!」令鞭而釋之,仍發帑賑民。

秋七月己巳,蒙古敖漢瑣諾木杜棱、塞臣卓禮克圖、奈曼袞出斯巴圖魯舉國來附。朝鮮國王李倧遣使報謝,並獻方物,命阿什達爾漢等往報之,尋以義州歸朝鮮。是月,明袁崇煥罷歸。

八月辛亥,察哈爾阿喇克綽忒部貝勒巴爾巴圖魯、諾門達賚、吹爾扎木蘇率眾來歸。是月,明熹宗崩,其弟信王嗣位,是為莊烈帝。

九月甲子朔,諭國家大祀大宴用牛外,其屠宰馬騾牛驢者悉禁之。

冬十一月庚午,察哈爾大貝勒昂坤杜棱來降。辛巳,薩哈爾察部來朝貢。

十二月甲午朔,察哈爾阿喇克綽忒貝勒圖爾濟伊爾登來降。

二年春正月戊子,格伊克里部長四人率其屬來朝。

二月癸巳朔,以額亦都子圖爾格、費英東子察哈尼俱為總兵官。朝鮮國王李倧遣其總兵官李蘭等來獻方物,並米二千石,更以一千石在中江平糶。庚子,以往喀喇沁使臣屢為察哈爾多羅特部所殺,上率師親征。丁未,進擊多羅特部,敗之,多爾濟哈談巴圖魯被創遁,獲其妻子,殺臺吉古魯,俘萬一千二百人還。丁巳,以戰勝,用八牛祭天。

三月戊辰,上還沈陽,貝勒阿敏等率群臣郊迎,行抱見禮。以弟多爾袞、多鐸從征有功,賜多爾袞號墨爾根戴青,多鐸號額爾克楚虎爾。庚寅,以賜名之禮宴之。戊子,給國人無妻者金,使娶。以貝勒多爾袞為固山貝勒。

夏四月丙辰,巴林貝勒塞特爾,臺吉塞冷、阿玉石、滿硃習禮率眾來歸。明復以袁崇煥督師薊、遼。崇煥素弗善毛文龍。時文龍據皮島,招集遼民,有逃亡則殺以冒功,遂得擢總兵,便宜行事。後更致書與我通好。上遣科廓等賚書往報。既,文龍執科廓等送燕京。崇煥以文龍私通罪紿殺之。

五月辛未,明人棄錦州。貝勒阿巴泰等率兵三千略其地,隳錦州、杏山、高橋三城,毀十三站以東墩臺二十一。先是顧特塔布囊以其眾自察哈爾逃匿蒙古地,遇歸附者輒殺之。辛巳,命貝勒濟爾哈朗、豪格率兵討顧特塔布囊。乙酉,顧特伏誅,俘其人口牲畜以萬計。長白山迤東濱海虎爾哈部頭目裡佛塔等來朝。

八月辛卯,與喀喇沁部議和定盟。乙未,賜奈曼貝勒袞出斯號達爾漢,札魯特喀巴海號衛徵。乙卯,朝鮮來貢。

九月庚申,徵外籓兵共征蒙古察哈爾。癸亥,上率大軍西發。丙寅,次遼陽。敖漢、奈曼、喀爾喀、札魯特、喀喇沁諸貝勒、臺吉各以兵來會。己巳,駐師綽洛郭爾。甲戌,宴來會諸貝勒。科爾沁諸貝勒不至。土謝圖汗額駙奧巴、哈談巴圖魯、滿硃習禮如約,請先侵掠而後合軍。上怒,遣使趣之。時奧巴違命,徑歸。滿硃習禮及臺吉巴敦以所俘來獻,上賜滿硃習禮號達爾漢巴圖魯,巴敦號達爾漢卓禮克圖,厚賚之。丙子,進兵擊席爾哈、席伯圖、英、湯圖諸處,克之,獲人畜無算。

冬十月辛卯,還師。丙申,諭敖漢、奈曼、巴林、札魯特諸貝勒,毋得要殺降人,違者科∶。壬寅,上還沈陽。以劉興祚詐稱縊死,逃歸明,系其母及妻子於獄。

十二月丁亥朔,遺土謝圖汗額駙奧巴書,數其罪。巴牙喇部長伊爾彪等來朝貢。蒙古郭畀爾圖、札魯特貝勒塞本及其弟馬尼各率部來歸。

三年春正月庚申,土謝圖汗奧巴來請罪,宥而遣之。辛未,敕科爾沁、敖漢、奈曼、喀爾喀、喀喇沁諸部悉遵國制。丁丑,諭諸貝勒代理三大貝勒直月機務。

二月戊子,諭三大貝勒、諸貝勒、大臣毋得科斂民間財物,犯者治罪。己亥,合葬太祖高皇帝、孝慈高皇后於沈陽之石嘴頭山,妃富察氏祔。喀爾喀札魯特貝勒戴青、桑土、桑古爾、桑噶爾寨等率眾來附。甲辰,上南巡,閱邊境城堡,圮薄者修築之。戊申,次海州,有老人年一百三歲,妻一百五歲,子七十三歲,召見賜牛種。辛亥,上還沈陽。

三月戊午,申蒙古諸部軍令。

夏四月丙戌朔,設文館,命巴克什達海及剛林等繙譯漢字書籍,庫爾纏及吳巴什等記注本朝政事。

五月丁未,奈曼、札魯特諸貝勒越界駐牧,自請議罰。上宥之。

六月乙丑,議伐明,令科爾沁、喀爾喀、札魯特、敖漢、奈曼諸部會兵,並令預採木造船以備轉餉。丁卯,喀喇沁布爾噶都戴青、臺吉卓爾畢,土默特臺吉阿玉石等遣使朝貢。辛巳,土默特臺吉卓爾畢泰等來朝貢。

秋七月辛卯,喀爾喀臺吉拜渾岱、喇巴泰、滿硃習禮自科爾沁來朝。甲午,孟阿圖率兵征瓦爾喀。乙未,庫爾喀部來朝貢。

八月庚午,頒八旗臨陣賞罰令。乙亥,諭曰:「自古及今,文武並用,以文治世,以武克敵。今欲振興文教,試錄生員。諸貝勒府及滿、漢、蒙古所有生員,俱令赴試。中式者以他丁償之。」

九月壬午朔,初試生員,拔二百人,賞緞布有差,免其差徭。癸未,貝勒濟爾哈朗等略明錦州、寧遠諸路還,俘獲以三千計。丙戌,阿魯部杜思噶爾濟農始遣使來通好。癸卯,喀喇沁布爾噶都來朝貢。

冬十月癸丑,上親征明,徵蒙古諸部兵以次來會。庚申,次納裡特河,察哈爾五千人來歸。壬戌,次遼河。丙寅,科爾沁奧巴以二十三貝勒來會。上集諸貝勒大臣議徵明與徵察哈爾孰利,皆言察哈爾遠,於是徵明。辛未,次喀喇沁之青城。大貝勒代善、三貝勒莽古爾泰止諸貝勒帳外,入見密議班師。既退,岳託等入白諸將在外候進取。上不懌,因曰:「兩兄謂我兵深入,勞師襲遠,若糧匱馬疲,敵人環攻,無為歸計。若等見及此,而初不言,朕既遠涉,乃以此為辭。我謀且隳,何候為!」岳託堅請進師。八固山額真詣代善、莽古爾泰議,夜半議定。諭曰:「朕承天命,興師伐明,拒者戮,降者勿擾。俘獲之人,父母妻子勿使離散。勿淫人婦女,勿褫人衣服,勿毀廬舍器皿,勿伐果木,勿酗酒。違者罪無赦。固山額真等不禁,罪如之。」乙亥,次老河,命濟爾哈朗、岳託率右翼兵攻大安口,阿巴泰、阿濟格率左翼兵攻龍井關。上與大貝勒代善、三貝勒莽古爾泰率大兵繼之。丁丑,左翼兵克龍井關,明副將易愛、參將王遵臣來援,皆敗死。漢兒莊、潘家口守將俱降。戊寅,上督兵克洪山口。辛巳,上至遵化。莽古爾泰率左翼兵自漢兒莊來會。遺書明巡撫王元雅勸降。

十一月壬午朔,右翼諸貝勒率師來會。先是濟爾哈朗等克大安口,五戰皆捷,降馬蘭營、馬蘭口、大安營三城,明羅文峪守將李思禮降。山海關總兵趙率教以兵四千來援,阿濟格迎擊斬之。甲申,諸貝勒攻遵化,正白旗小校薩木哈圖先登,大兵繼之,遂克其城。明巡撫王元雅自經死。上親酌金卮賜薩木哈圖,擢備御,世襲罔替,賜號巴圖魯,有過赦免,家固貧,恤之。蒙古兵擾害羅文峪民。令曰:「凡貝勒大臣有掠歸降城堡財物者斬,擅殺降民者抵罪,強取民物,計所取倍償之。」己丑,敘克城功,將士賞賚有差。壬辰,參將英俄爾岱、文館範文程留守遵化,大軍進逼燕京。有蒙古兵殺人而褫其衣,上命射殺之。甲午,徇薊州。乙未,徇三河。丙申,左翼貝勒赴通州視渡口。明大同、宣府二鎮援兵至順義,貝勒阿巴泰、岳託擊敗之。順義降。上至通州,諭明士民曰:「我國夙以忠順守邊,葉赫與我同一國耳,明主庇葉赫而陵我,大恨有七。我知終不相容,故告天興師。天直我國,賜我河東地。我太祖皇帝猶原和好,與民休息。爾國不從,天又賜我河西地。及朕即位,復徇爾國之請,遂欲去帝稱汗,趣制國印,而爾國不從。今我興師而來,順者撫,逆者誅。是爾君好逞干戈,猶爾之君殺爾也。天運循環,無往不復,有天子而為匹夫,亦有匹夫而為天子者。天既佑我,乃使我去帝號。天其鑒之!」辛丑,大軍逼燕京。上營於城北土城關之東,兩翼營於東北。明大同總兵滿桂、宣府總兵侯世祿屯德勝門,寧遠巡撫袁崇煥、錦州總兵祖大壽屯沙窩門。上率右翼大貝勒代善,貝勒濟爾哈朗、岳託、杜度、薩哈廉等,領白甲護軍、蒙古兵進擊桂、世祿,遣左翼大貝勒莽古爾泰、阿巴泰、阿濟格、多爾袞、多鐸、豪格等,領白甲護軍、蒙古兵迎擊崇煥、大壽,俱敗之。癸卯,遣明歸順王太監賚書與明議和。乙巳,屯南海子。戊申,袁崇煥、祖大壽營於城東南隅,樹柵為衛,我軍偪之而營。上率輕騎往視。諸貝勒請攻城,諭曰:「路隘且險,若傷我士卒,雖得百城不足多也。」因止弗攻。初,獲明太監二人,令副將高鴻中,參將鮑承先、寧完我等受密計。至是,鴻中、承先坐近二太監耳語云:「今日撤兵,乃上計也。頃上單騎向敵,敵二人見上語良久乃去。意袁都堂有約,此事就矣。」時楊太監佯臥竊聽。翌日縱之歸,以所聞語明帝,遂下崇煥於獄。大壽懼,率所部奔錦州,毀山海關而出。諸貝勒大臣請攻城,上曰:「攻則可克,但恐傷我良將勁卒,餘不忍也。」遂止。

十二月辛亥朔,大軍經海子而南,且獵且行,趣良鄉,克其城。壬子,總兵吳訥格克固安。辛酉,遣貝勒阿巴泰、薩哈廉以太牢祀金太祖、世宗陵。丙寅,復趨燕京,敗明兵於盧溝橋,殲其眾。明總兵滿桂、孫祖壽、黑雲龍、麻登雲以兵四萬柵永定門之南。丁卯黎明,師毀柵入,斬桂、祖壽及副將以下三十餘人,擒黑雲龍、麻登雲,獲馬六千,分賜將士。戊辰,遣達海賚書與明議和。壬申,貝勒阿巴泰、濟爾哈朗略通州,焚其舟,攻張家灣,克之。達海賚議和書二分置安定、德勝門外。乙亥,復遣人賚書赴安定門。俱不報。丙子,駐師通州。丁丑,岳託、薩哈廉、豪格率兵四千圍永平。遂克香河、馬蘭峪諸城,復叛去。己卯,大軍趣永平。

四年春正月辛巳朔,大軍至榛子鎮、沙河驛,俱降。壬午,至永平。先是,劉興祚自我國逃歸,匿崇煥所。至是,率所手巂滿洲兵十五人、蒙古兵五百欲往守沙河。聞大兵至,改趣永平之太平寨,襲殺喀喇沁兵於途。上怒其負恩,遣貝勒阿巴泰等禽斬之,裂其尸以徇。癸丑,上授諸將方略,乘夜攻城。城中火藥自發,敵軍大亂,黎明克之。貝勒濟爾哈朗等入城安撫。丙戌,上率諸將入城,官民夾道呼萬歲。貝勒濟爾哈朗、薩哈廉守永平。以降官白養粹為永平巡撫,孟喬芳、楊文魁為副將,縱鄉民還其家。是日,上率大軍趣山海關。敖漢、奈曼、巴林、札魯特諸部兵攻昌黎,不克。臺頭營、鞍山堡、遷安、灤州以次降。建昌參將馬光遠來歸。丁酉,明兵攻遵化,貝勒杜度擊敗之。明兵入三屯營,先所下漢兒莊、喜峰口、潘家口、洪家口復叛。庚子,達海等復漢兒莊,貝勒阿巴泰守之。辛丑,喀喇沁布爾噶都為明兵所圍,遣軍往救,未至,布爾噶都自擊敗之。其帥明兵部尚書劉之綸領兵至,樹柵。我軍砲毀其柵。之綸屯山中。大貝勒代善圍之,勸之綸降,不從。破其營,之綸被箭死。壬寅,移師馬蘭峪,毀其近城屯堡。丙午,喀喇沁蘇布地上書明帝,論和好之利,且勸以愛養邊民、優恤屬國之道。不報。樂亭復叛。

二月辛亥朔,諭貝勒諸臣,凡將士驍勇立功者,勿與攻城之役。甲寅,宴明降將麻登雲等於御幄,謂之曰:「明主視爾等將士之命如草芥,驅之死地。朕屢遣使議和,竟無一言相報,何也?」登雲對曰:「明帝幼沖,大臣各圖自保,議和之事,儻不見聽,罪且不測,故懼不敢奏。」上曰:「若然,是天贊我也,豈可棄之而歸。但駐兵屯守,妨農時為可憫耳。且彼山海關、錦州防守尚堅,今但取其無備城邑可也。」己未,遺書明帝,仍申和好,並致書明諸臣,勸其急定和議,至是凡七致書矣。甲子,明榆林副將王世選來降。上班師,貝勒阿巴泰、濟爾哈朗、薩哈廉及文臣索尼、寧完我等守永平,鮑承先守遷安,固山額真圖爾格、那木泰等守灤州,察喀喇、範文程等守遵化。駐灤三日,論功行賞。壬申,諭曰:「天以明土地人民予我,其民即吾民,宜飭軍士勿加侵害,違者治罪。」上至永平,降官郎中陳此心謀遁,事覺論斬,上赦之,聽其所往。

三月壬午,上還沈陽。庚寅,遣二貝勒阿敏、貝勒碩託率兵五千往守永平四城,貝勒阿巴泰等還。庚子,阿魯四子部遣使來盟。

夏四月壬子,明兵攻灤州,不克。己卯,貝勒阿巴泰、濟爾哈朗等自永平還。上問是役俘獲較前孰多,對曰:「此行所獲人口甚多。」上曰:「財帛不足喜,惟多得人為可喜耳。」

五月己丑,諭諸臣厚撫俘眾。壬辰,阿敏、碩託等棄永平四城歸。時明監軍道張春、錦州總兵祖大壽等合兵攻灤州。那穆泰、圖爾格、湯古代等出戰,屢敗明兵,然兵少,阿敏、碩託畏不往援,明兵用砲攻灤州,那穆泰等不能支,棄城奔永平。會天雨,我軍潰圍出,無馬被創者死四百餘人。阿敏、碩託聞之恐,遂殺降官白養粹等,盡屠城中士民,收其金幣,乘夜出冷口。察哈喇等亦棄遵化歸。上方命貝勒杜度趨永平協守,且敕阿敏善撫官民,無侵暴,將整兵親往。庚子,聞阿敏棄城,且大肆屠戮,乃止。

六月甲寅,收系棄城諸將,數其罪。乙卯,御殿宣阿敏十六罪。眾議當誅。上不忍致法,幽之。碩託、湯古代、那穆泰、巴布泰、圖爾格等各奪爵、革職有差。諸將中有力戰殺敵者釋之。先是阿敏既屠永平官民,以其妻子分給士卒。上曰:「彼既屠我歸順良民,又奴其妻子耶!」命編為民戶,以房舍衣食給之。

秋九月戊戌,申諭諸大臣滿、漢官各勤職業。

冬十月辛酉,諭編審各旗壯丁,隱匿者罰之。

十一月甲午,那堪泰部虎爾噶率家屬來歸,阿魯四子部諸貝勒來歸。壬寅,阿魯伊蘇忒部聞上善養民,留所部於西拉木輪河,而偕我使臣察漢喇嘛來朝。

十二月戊辰,科爾沁貝勒圖美衛徵來朝。

五年春正月庚辰,諭已故功臣無後者,家產給其妻自贍。壬午,鑄紅衣大砲成,金雋曰「天祐助威大將軍」。軍中造砲自此始。乙未,以額駙佟養性總理漢人軍民事,漢官聽其節制。己亥,幸文館,入庫爾纏直房,問所修何書。對曰:「記註所行政事。」上曰:「如此,朕不宜觀。」又覽達海所譯武銓,見投醪飲河事,曰:「古良將體恤士卒,三軍之士樂為致死。若額駙顧三臺對敵時,見戰士歿者,以繩曳之歸,安能得人死力乎!」庚子,朝鮮貢物不及額,卻之,以書責其罪。

二月庚申,敕邊臣謹斥堠。甲戌,孟阿圖徵瓦爾喀,奏捷。

三月乙亥朔,鑲藍旗固山額真、額駙顧三臺罷,以太祖弟之子篇古代之。書諭大貝勒代善、三貝勒莽古爾泰及貝勒諸大臣,求直言過失。丁亥,閱漢兵。甲午,誅劉興祚、興治家屬,赦其母。丁酉,朝鮮復遣使來貢。辛丑,遣滿達爾漢、董訥密遺朝鮮王書,索戰船助攻明。不許。

六月癸亥,定功臣襲職例。黑龍江伊札訥、薩克提、伽期訥、俄力喀、康柱等五頭目來朝。

秋七月甲戌,黑龍江虎爾哈部四頭目來朝貢。庚辰,始設六部,以墨勒根戴青貝勒多爾袞,貝勒德格類、薩哈廉、岳託、濟爾哈朗、阿巴泰等管六部事。每部滿、漢、蒙古分設承政官,其下設參政各八員,啟心郎各一員,改巴克什為筆帖式,其尚稱巴克什者仍其舊。更定訐告諸貝勒者準其離主例,其以細事訐訴者禁之。諭貝勒審事冤抑不公者坐罪。除職官有罪概行削職律,嗣後有罪者,分別輕重降罰有差。並禁官民同族嫁娶,犯者男婦以奸論。又諭貝勒諸大臣省過改行,求極諫。甲申,鬧雷虎爾哈部四頭目來朝貢。癸巳,定小事賞罰例,令牛錄額真審理,大者送部。明總兵祖大壽等築大凌河。檄諸蒙古各率所部來會征之。己亥,大軍西發,命貝勒杜度、薩哈廉、豪格留守。庚子,渡遼河,申誡諸將恤士卒。

八月壬寅朔,次舊遼河而營,蒙古諸部率兵來會。癸卯,集蒙古諸貝勒,申前令,無擅殺掠。於是分兵兩路,貝勒德格類、岳託、阿濟格以兵二萬由義州入屯錦州、大凌河之間,上自白土場入廣寧。丁未,會於大凌河,乘夜攻城。令曰:「攻城恐傷士卒,當掘壕築壘困之。彼若出,與之戰,外援至,迎擊之。」乃分八旗兵合圍,令蒙古兵承其隙。辛亥,明馬步兵五百人出城,達爾哈擊敗之。壬子,射書城中,招蒙古人出降。癸丑,明兵出城誘戰。圖賴先入,達爾哈繼之,四面環攻,貝勒多爾袞亦率兵入。城內砲矢俱發,圖賴被創,副將孟坦、屯布祿、備禦多貝、侍衛戈裏戰歿。上以圖賴等輕進,切責之。以紅衣砲攻明臺,兵降者相繼。乙卯,遺祖大壽書曰:「往者我欲和,爾國君臣以宋為鑒,不我應。爾國非宋,我亦非金,何不達若此。朕今厭兵革,更以書往,惟將軍裁之。」大壽不答。丁巳,明松山兵二千來援,阿山、勞薩、土魯什擊敗之。甲子,貝勒阿濟格、碩托遮擊明援兵。丁卯,明錦州兵六千來攻阿濟格營。會大霧,覿面不相識。忽有青氣沖敵營,闢若門,我軍乘霧進,大戰,敗之,擒游擊一,盡獲其甲仗馬匹。辛未,上詣貝勒阿濟格營,酌金卮勞諸將。明兵突出,師夾擊,又大敗之。

九月丁亥,上以兵趨錦州,見塵起,上命諸軍勿行,自率擺牙喇兵二百,與貝勒多鐸緣山潛進。明錦州兵七千突出進上前。上甫擐甲,從者不及二百人,渡河沖敵軍。敵不能當,潰走。諸軍繼至,又大敗之,斬一副將而還。己丑,復以書招祖大壽。庚寅,上設伏山內,誘大壽出,將擒之,大壽驚遁,自是閉城不出。時城中穀止百石,馬死盡,煮馬肉為食,以鞍代爨。乙未,明太僕寺卿監軍道張春,總兵吳襄、鍾緯等,以馬步兵四萬來援,壁小凌河。戊戌,明援兵趨大凌河,距城十五里。上率兩翼騎兵沖擊之,不動。右翼兵猝入張春營,敵遂敗,吳襄及副將桑阿爾寨先奔。張春等復集潰兵立營,會大風,敵乘風縱火,將及我軍,天忽雨,反風,復戰,遂大破之,生擒張春及副將三十三人。春不屈,乞死,上赦不殺。是役也,祖大壽仍以我為誘敵,故城中無應者。是夕黑雲龍遁去。

冬十月丁未,以書招祖大壽、何可剛、張存仁。己酉,再遺大壽書。壬子,以紅衣砲攻於子章臺。臺最固,三日臺毀,守臺將王景降,於是遠近百餘臺俱下。甲寅,遣降將姜新招祖大壽。大壽亦遣游擊韓棟來會。癸亥,議三貝勒莽古爾泰上前持刃罪,降貝勒,奪所屬五牛錄。乙丑,祖大壽約我副將石廷柱議降。丙寅,大壽遣其子可法為質。戊辰,大凌河舉城降,獨副將何可剛不從。大壽掖可剛至軍前殺之,夜至御營,上優遇之,大壽遂獻取錦州策。己巳,遣兵隨大壽夜襲錦州,會大霧,失伍,還。

十一月庚午朔,縱大壽還錦州。戊寅,毀大凌河城。己卯,班師。乙酉,上還沈陽。丙戌,察哈爾侵阿魯西拉木輪地,貝勒薩哈廉、豪格移師征之,會察哈爾已去,乃還。

閏十一月庚子朔,諭曰:「我兵之棄永平四城,皆貝勒等不學無術所致。頃大凌河之役,城中人相食,明人猶死守,及援盡城降,而錦州、松、杏猶不下,豈非其人讀書明理盡忠其主乎?自今凡子弟年十五歲以下、八歲以上,皆令讀書。」遣庫爾纏等責朝鮮違約罪。庚戌,禁國中不得私立廟寺,喇嘛僧違律者還俗,巫覡星士並禁止之。

十二月壬辰,參將寧完我請設言官,定服制。上嘉納之。丙申,用禮部參政李伯龍言,更定元旦朝賀行禮班次。

六年春正月癸亥,閱漢兵。

二月壬申,定儀仗制。丁丑,謁太祖陵,行時享禮。戊子,諭海州等處城守官三年一赴沈陽考察。丁酉,諭戶部貝勒德格類以大凌河漢人分隸副將以下,給配撫養。給還貝勒莽古爾泰所罰人口。

三月戊戌,賚大凌河諸降將有差。命達海分析國書音義。庚戌,定訐告諸貝勒者輕重虛實坐罪例,禁子弟告父兄、妻告夫者,定貝勒大臣賜祭葬例。丁巳,徵察哈爾,徵蒙古兵,頒軍令。

夏四月戊辰朔,上率大軍西發,阿巴泰、杜度、揚古利、伊爾登、佟養性留守。己巳,次遼河。丙子,次西拉木輪河。己卯,次札滾烏達,諸蒙古部兵以次來會。乙酉,次哈納崖。察哈爾汗林丹聞我師至,大懼,驅歸化城富民牲畜渡河西奔,盡委輜重而去。庚寅,次都勒河,聞察哈爾林丹遠遁,上趨歸化城。丙申,大軍自阿濟格和爾戈還趨察哈爾。

五月癸卯,諭諸部貝勒大臣勿輕進,勿退縮,勿殺降,勿分散人妻子,勿奪人衣服財物。甲辰,次布龍圖布喇克。丁未,勞薩奏報察哈爾遁去已久,逐北三日無所見。上自布龍圖旋師。戊申,定議徵明。丙辰,次硃兒格土。時糧盡,忽逢黃羊遍野,遂合圍殺數萬,脯而食之。無水,以一羊易杯水而飲。上命各牛錄持水迎給之。庚申,次木魯哈喇克沁,貝勒阿濟格率左翼略宣府、大同,貝勒濟爾哈朗率右翼略歸化城,上與大貝勒代善、貝勒莽古爾泰統大軍繼進。甲子,上至歸化城,兩翼兵來會。是日,大軍馳七百里,西至黃河木納漢山,東至宣府,自歸化城南至明邊境,所在察哈爾部民悉俘之。

六月丁卯朔,蒙古部民竄沙河堡,上以書諭明守臣索之。明歸我男婦三百二十、牲畜千四百有奇。辛未,寧完我、範文程、馬國柱合疏言:「伐明之策,宜先以書議和,俟彼不從,執以為辭,乘釁深入,可以得志。」上嘉納之。甲戌,大軍發歸化城,趨明邊。丁丑,明沙河堡守臣使賚牲幣來獻。己卯,庫爾纏等自得勝堡,愛巴禮等由張家口,分詣大同、宣府議和。書曰:「我之興兵,非必欲取明天下也。遼東守臣貪黷昏罔,勸葉赫陵我,遂嬰七恨。屢愬爾主,而遼東壅不上聞。我兵至此,欲爾主察之也。及攻撫順,又因十三省商賈各遺以書,慮其不克徑達,則各以書進其省官吏,冀有一聞。乃縱之使去,寂焉不復。語云:『下情上達,天下罔不治;下情上壅,天下罔不亂。』今所在征討,爭戰不息,民死鋒鏑,雖下情不達之故,抑豈天意乎?我今聞誠相告,國雖褊小,惟欲兩國和好,互為貿易,各安★獵,以享太平。若言不由衷,天其鑒我。前者屢致書問,憤疾之詞,固所不免。此兵家之常,不足道也。幸速裁斷,實國之福。我駐兵十日以待。」庚辰,駐大同邊外。庫爾纏偕明得勝堡千總賚牲幣來獻。上不納。復遺書明守臣曰:「我仰體天意,原申和好。爾果愛民,宜速定議。若延時不報,縱欲相待,如軍中糧盡何。至書中稱謂,姑勿論,我遜爾國,我居察哈爾之上可耳。」癸未,趨宣府,守臣以明主所給察哈爾緞布皮幣一萬二千五百歸我。庚寅,駐張家口外,列營四十里。癸巳,明巡撫沈棨、總兵董繼舒遣人賚牛羊食物來獻。上宴之,遂定和議,大市於張家口。科爾沁部兵三人潛入明邊,盜牛驢,斬其首者,鞭二人,貫耳以徇。甲午,明巡撫沈棨遣使來請盟。命大臣阿什達爾哈等蒞之,刑白馬烏牛,誓告天地。禮成,遣啟心郎祁充格送明使歸。明以金幣來獻。晉封皇子豪格為和碩貝勒。是月,遼東大水。

秋七月丁酉朔,復以書約明張家口守臣信誓敦好,善保始終,且謂和議遼東地方在內,爾須遣官往告。上率大軍還。庚子,至上都河,明以和議成,來餽禮物,酌納之。辛丑,蒙古諸貝勒辭歸。庚戌,次擺斯哈兒。游擊巴克什達海卒。庚申,上還沈陽。

八月丁卯,召明諸生王文奎、孫應時、江雲入宮,問以和事成否。三人皆言,明政日紊,和議難必。且中原盜賊蜂起,人民離亂。勸上宣布仁義,用賢養民,乘時吊伐,以應天心。癸酉,六部署成,頒銀印各一。甲午,命固山額真察民疾苦,清理刑獄。察哈爾檮納楚虎爾來歸。

九月癸卯,修復蓋州城,移民實之。甲寅,命戶部貝勒德格類、兵部貝勒岳託展耀州舊界至蓋州迤南。

冬十月乙丑朔,幸開原。甲戌,還沈陽。遣衛徵囊蘇喇嘛赴寧遠,賚書致明帝曰:「我國稱兵,非不知足而冀大位,因邊臣欺侮,致啟兵釁。往征察哈爾時,過宣府定和議,我遂執越境盜竊之人戮之塞下,我之誠心可謂至矣。前邊臣未能細述,今欲備言,又恐疑我不忘舊怨,如遣信使來,將盡告之。若謂已和,不必語及往事,亦惟命。」又與明諸臣書曰:「宣府守臣與我盟時,約我毋侵遼東,誓諸天地。今爾乃有異議,天可欺乎?執政大臣宜通權變,慎勿徒事大言,坐失事機。若堅執不從,惟尋師旅,生靈荼毒,咎將誰歸?」

十一月壬寅,明寧遠守臣以我所遺書封固,不敢以陳,請露封,許之。辛亥,阿祿部都思噶爾濟農所屬祁他特吹虎爾臺吉來附。壬子,遣使往朝鮮定歲貢額。

十二月乙丑,定朝服及官民常服制。三貝勒莽古爾泰卒。乙亥,吳巴海征兀札喇遣使告捷。

七年春正月庚子,諭各牛錄額真以恤貧訓農習射。辛丑,朝鮮來貢,不及額。丁未,復書責之。戊申,皇長女下嫁敖漢部貝勒都喇爾巴圖魯子臺吉班第。乙卯,徵兀札喇師還。

二月癸亥朔,阿魯科爾沁汗車根率固木巴圖魯、達爾馬代袞等舉國來附。己卯,庫爾纏有罪,誅。癸未,土魯什、勞薩等略寧遠。

三月丁酉,築★場、攬盤、通遠堡、岫巖四城。辛丑,郭爾羅斯部臺吉固木來朝。丙辰,明故總兵毛文龍部將孔有德、耿仲明遣使來約降。

夏四月乙丑,察哈爾兩翼大總管塔什海虎魯克寨桑來附。乙亥,使參將英俄爾岱等借糧朝鮮濟孔有德軍,不從。

五月乙未,吳喇忒臺吉土門達爾漢等來朝。壬子,貝勒濟爾哈朗、阿濟格、杜度率兵迎孔有德、耿仲明於鎮江,命率所部駐東京。

六月壬戌,諭將士毋侵擾遼東新附人民,違者孥戮之。癸亥,召孔有德、耿仲明入覲,厚賚之。丙寅,遣英俄爾岱遺朝鮮王書曰:「往之借糧,貴國王以孔有德等昔隸毛氏,無輸糧養敵之理。今有德歸我,糧已足給。惟兵卒守船,輓運維艱,近距貴國,以糧給之甚便。朕思王視明為父,視朕為兄,父兄相爭數年,而王坐觀成敗,是外有父兄之名,而內懷幸禍之意。若力為解勸,息兵成好,不惟我兩國樂見太平,即貴國亦受其福。若仍以兵助明,合而御我,則構兵實自王始。」己巳,諭官民冠服遵制畫一。癸酉,以孔有德為都元帥,耿仲明為總兵官,並賜敕印。戊寅,英俄爾岱奏報朝鮮用明人計,借兵倭國,又於義州南嶺築城備我。集諸貝勒大臣議之,皆言宜置朝鮮而伐明。己卯,貝勒岳託、德格類率右翼楞額禮、葉臣,左翼伊爾登、昂阿喇及石廷柱、孔有德、耿仲明將兵取明旅順口。甲申,東海使犬部額駙僧格來朝貢。丁亥,諭曰:「凡進言者,如朕所行未協於義,宜直言勿諱。政事或有愆忌,宜開陳無隱。六部諸臣,奸偽貪邪,行事不公,宜行糾劾。諸臣有艱苦之情,亦據實奏聞。茍不務直言,遠引曲喻,剿襲紛然,何益於事?」

秋七月辛卯朔,諭滿洲各戶有漢人十丁者授棉甲一,以舊漢軍額真馬光遠統之。壬辰,阿祿部孫杜棱子臺吉古木思轄布,寨桑吳巴什、阿什圖、巴達爾和碩齊等,吳喇忒部臺吉阿巴噶爾代皆來朝貢。甲辰,貝勒岳託等奏克旅順口。

八月庚申朔,英俄爾岱等自朝鮮還,以復書允糧濟我守船軍士。壬戌,貝勒阿巴泰、阿濟格、薩哈廉、豪格等略明山海關外。庚辰,貝勒德格類、岳託師還。丁亥,以副將石廷柱為總兵官。

九月庚子,貝勒阿巴泰等師還。上以其不深入,責之。癸卯,英俄爾岱等往朝鮮互市。庚戌,明登州都司蔡賓等來降。

冬十月壬戌,遣使外籓蒙古各部,宣布法令。丙寅,大閱。丁卯,發帑賚八旗步兵。己巳,諭曰:「置官以來,吏、戶、兵三部辦事盡善,刑部訊獄稽延,罔得實情,禮部、工部皆有缺失。夫啟心郎之設,欲其隨事規諫,啟乃心也。乃有差謬而不聞開導,何耶?」又曰:「爾等動以航海取山東攻山海關為言。航海多險,攻堅易傷,是以空言相賺,不啻為敵計耳。兵事無藉爾言,惟朕與諸貝勒有過,當極言耳。」又諭文館諸儒臣曰:「太祖始命巴克什額爾德尼造國書,後庫爾纏增之。慮有未合,爾等職司紀載,宜悉心訂正。朕嗣大位,凡皇考行政用兵之大,不一一詳載,後世子孫何由而知,豈朕所以盡孝道乎?」丙子,授明降將馬光遠為總兵官,王世選、麻登雲為三等總兵官,馬光先、孟喬芳等各授職有差。癸未,明廣鹿島副將尚可喜遣使來約降。

十一月甲辰,英俄爾岱復賚書往朝鮮,責以違約十事。戊申,遣季思哈、吳巴海往征朝鮮接壤之虎爾哈部。辛亥,上獵於葉赫。

十二月辛未,上還沈陽。

八年春正月庚寅,諭蒙古諸貝勒令遵我國定制。黑龍江羌圖裏、嘛爾幹率六姓來朝貢。癸巳,詔宗人自興祖直皇帝出者為六祖後,免其徭役。乙未,正黃旗都統、一等總兵官楞額禮卒。癸卯,漢備御訴漢人徭役重於滿洲,戶部貝勒德格類以聞。上命禮部貝勒薩哈廉集眾諭其妄。漢總兵官石廷柱等執備禦八人請罪,上曰:「若加以罪,則後無復言者。」並釋之。戊申,塔布囊等徵察哈爾潰眾於席爾哈、席伯圖。己酉,蒿齊忒部臺吉額林臣來歸。丁巳,免功臣身故無嗣者丁之半,妻故始應役,著為令。

二月壬戌,定喪祭例,妻殉夫者聽,仍予旌表;逼妾殉者,妻坐死。遣貝勒多爾袞、薩哈廉往迎降將尚可喜,使駐海州。丁卯,都元帥孔有德劾耿仲明不法狀,諭解之。戊辰,遣阿山等略錦州。

三月丁亥朔,日有食之,綠虹見。辛卯,命譚泰、圖爾格略錦州。壬辰,副將尚可喜率三島官民降,駐海州。己亥,大閱。甲辰,遣英俄爾岱往朝鮮互市。令孔有德、耿仲明、尚可喜幟用白鑲皁,以別八旗。壬子,考試漢生員。

夏四月辛酉,升授太祖諸子湯古代等副將、參將、備御有差。又以哈達、烏喇二部之後無顯職,授哈達克什內為副將,烏喇巴彥為三等副將。詔以沈陽為「天眷盛京」,赫圖阿喇城為「天眷興京」。改定總兵、副將、參將、游擊、備御滿字官名。丁丑,尚可喜來朝,命為總兵官。乙亥,以太祖弟之子拜尹圖為總管。辛巳,初命禮部考試滿洲、漢人通滿、漢、蒙古書義者,取剛林等十六人為舉人,賜衣一襲,免四丁。乙酉,金繼孟等自明石城島來降,以隸尚可喜。

五月丙戌朔,黑龍江巴爾達齊來貢。庚寅,察哈爾臺吉毛祁他特來朝。定滿、漢馬步軍名。丙申,議徵明,諸貝勒請從山海關入。上曰:「不然,察哈爾為我軍所敗,其貝勒大臣將歸我,宜直趨宣、大以逆之。」乃集各都統部署軍政,遣國舅阿什達爾哈徵科爾沁兵,以書招撫遺眾之在明境者。壬寅,定百官功次,賜敕書,其世襲及官止本身者,分別開載有差。甲辰,季思哈、吳巴海征虎爾哈部奏捷。命貝勒濟爾哈朗留守盛京,貝勒杜度守海州,吏部承政圖爾格等渡遼河,沿張古臺河駐防,並扼敵兵,俱授方略。畢,上率大軍前發。己酉,次都爾鼻,諸蒙古外籓兵以次來會。甲寅,次訥裡特河。

六月辛酉,頒軍令於蒙古諸貝勒及孔有德、耿仲明、尚可喜,曰:「行軍時勿離纛,勿諠譁,勿私出劫掠。抗拒者誅之,歸順者字之。勿毀廟宇,勿殺行人,勿奪人衣服,勿離人夫婦,勿淫人婦女。違者治罪。」先是,察哈爾林丹西奔圖白特,其部眾苦林丹暴虐,逗遛者什七八,食盡,殺人相食,屠劫不已,潰散四出。至是,絡繹來附者前後數千戶。辛未,次庫黑布裏都,議覺羅布爾吉、英俄爾岱擅殺察哈爾布顏圖部眾罪,並奪其賜。甲戌,次喀喇拖落木,命貝勒德格類率兵入獨石口,偵居庸關,期會師於朔州。戊寅,諭蒙古諸貝勒曰:「科爾沁噶爾珠塞特爾等叛往索倫,為其族兄弟等追獲被殺,朕心惻然。朕欲宣布德化,使人民共登安樂。今諸貝勒雖以罪誅,亦朕教化所未洽也。」又命減阿魯部達喇海等越界駐牧罪。壬午,察哈爾土巴濟農率其民千戶來歸。喀爾喀部巴噶達爾漢來歸。甲申,命大貝勒代善等率兵入得勝堡,略大同,西至黃河,副都統土魯什、吳拜等逕歸化撫察哈爾逃民,俱會師朔州。

秋七月己丑,命貝勒阿濟格、多爾袞、多鐸等入龍門,會宣府,上親統大軍自宣府趨朔州,期四路兵克期並進。辛卯,毀邊墻。壬辰,入上方堡,至宣府右衛,以書責明守臣負盟之罪,仍諭其遣使議和。癸巳,駐城東南。時阿濟格攻龍門,未下,令略保安。丁酉,營東城,遺明代王書,復約其遣使議和。代善攻得勝堡,克之。明參將李全自縊死。進攻懷仁、井坪,皆不克,遂駐朔州。丙午,上圍應州,令代善等趣馬邑。土魯什至歸化城,察哈爾林丹之妻率其八寨桑以一千二百戶來降。庚戌,阿濟格等攻保安州,克之。壬子,德格類入獨石口,取長安嶺,攻赤城,不克,俱會師於應州。

八月乙卯,命諸將略代州。薩哈廉襲崞縣,拔之。丙辰,碩托入圓平驛。甲子,阿巴泰等取靈丘縣之王家莊,克之。禮部承政巴都禮戰歿。又攻應州之石家村堡,克之。丙寅,上發應州,聞明陽和總督張宗衡、大同總兵曹文詔駐懷仁,度是夜必奔大同,令土魯什、吳拜伏兵邀之。師行遲,宗衡等逸去。上怒責之。戊辰,上至大同,遺書文詔,令贊和議。又遺書眾官,索察哈爾餘孽之在明者。文詔挑戰,擊敗之。貝勒阿巴泰等拔靈丘。明代王母楊氏與張宗衡、曹文詔以書來請和。辛未,遣使以書報之。壬申,代善率師來會。癸酉,駐師大同,遣明宗室硃乃廷及俘獲僧人入城。三索報書,俱不答。縱乃廷妻子及硃乃振還。丁丑,營四十里鋪,得明間諜書北樓口,為書報之曰:「來書以滿洲為屬國,即予亦未嘗以為非也。惟遼東之官欺凌我國,皇帝惑於臣下之言狂,雖干戈十數年來,無一言詢及,使我國之情不達,若遣一信使判白是非,則兵戈早息矣。欲享太平,只旦暮間事。不然,爾國臣僚壅蔽欺罔,虛報斬伐,以吾小國果受傷夷,詎能數侵,豈皇帝之聰明獨不能一忖度耶?原和之誠,黑雲龍自知之,慮其恐結怨於大臣不盡告耳。」己卯,大軍至陽和。明總兵曹文詔詭以書言狂張宗衡,偽言砲傷我兵,得纛一桿等語,為我邏者所獲。上乃遺宗衡書曰:「予謂爾明當有忠臣義士實心謀國者,乃一旦虛言狂至此,豈不愧於心乎?今與公等約,我兵以一當十,能約期出戰,當勒兵以俟。若言狂言欺君,貽害生靈,禍蘗將無窮矣。」壬午,次懷遠。癸未,駐左衛。

閏八月丙戌,以書責明宣府太監欺君誤國罪。丁亥,副都統土魯什被創卒。攻萬全左衛,克之。庚寅,班師。察哈爾噶爾馬濟農等遣使乞降,言其汗林丹病殂,汗子及國人皆欲來歸,於是命阿什達爾哈等往偵之。丁酉,移軍舊上都城。庚戌,移軍克蚌。辛亥,察哈爾寨桑噶爾馬濟農等率其國人六千奉豆土門福金來歸。

九月戊辰,留守貝勒濟爾哈朗疏報季思哈、吳巴海征虎爾哈俘一千三百餘人。阿魯部毛明安舉國來附。辛未,渡遼河。壬申,上還盛京。

冬十月己丑,建太祖陵寢殿,樹松,立石獸。壬辰,論徵宣、大將士功罪。己亥,科爾沁臺吉吳克善來歸其妹,納之。庚戌,以八年征討克捷,為文告太祖。壬子,朝鮮國王李倧遣使以書來。上以其言不遜,復書切責之。

十一月乙丑,六部官考績升授有差。

十二月癸未朔,朝鮮國王以書來謝罪。壬辰,命副都統霸奇蘭、參領薩木什喀征黑龍江未服之地。丙申,分定宗室、額駙等專管佐領有差。丁酉,墨勒根喇嘛以嘛哈噶喇金像來貢,遣使迎至盛京。癸卯,察哈爾祁他特車爾貝、塞冷布都馬爾等各率所部人民來歸。遣吳巴海、荊古爾代征瓦爾喀。甲辰,佐領劉學誠疏請立郊壇,勤視朝。上曰:「疏中欲朕視朝勤政是也。至建立郊壇,未知天意所在,何敢遽行,果成大業,彼時議之未晚也。」

九年春正月丁卯,上親送科爾沁土謝圖濟農等歸國。癸酉,免功臣徭役。丁丑,詔太祖庶子稱「阿格」,六祖子孫稱「覺羅」,覺羅系紅帶以別之。有詈其祖父者罪至死。

二月壬午,令諸臣薦舉居心公正及通曉文藝可任使者。丁亥,編喀喇沁部蒙古壯丁為十一旗,每旗設都統一員,下以副都統、參領二員統之。戊子,諭曰:「邇來進言者皆請伐明,朕豈不以為念。然亦須相機而行。今察哈爾新附,人心未輯,城郭未修,而輕於出師,何以成大業。且大兵一舉,明主或棄而走,或懼而請和,攻拒之策,何者為宜?其令高鴻中、鮑承先、寧完我、範文程等酌議以聞。」己丑,沈佩瑞請屯田廣寧、閭陽,造舟輓粟,為進取計。上嘉納之。乙未,範文程、寧完我請薦舉不實宜行連坐法。丁未,命多爾袞、岳託、豪格、薩哈廉將精騎一萬,收察哈爾林丹之子額爾克孔果爾額哲。

三月戊辰,諭曰:「頃民耕耨愆期,蓋由佐領有事築城,民苦煩役所致。嗣有濫役妨農者治其罪。」庚午,察哈爾寨桑巴賴都爾等一千四百餘人來歸。

五月乙卯,霸奇蘭、薩木什喀征黑龍江虎爾哈部,盡克其地,編所獲人口以歸,論功升賞有差。癸亥,上以西征諸貝勒經宣、大境,度明必調寧、錦兵往援,遣貝勒多鐸率師入寧、錦撓之。己巳,命文館譯宋、遼、金、元四史。壬申,貝勒多鐸奏報殲明兵五百人於錦州松山城外,殺其副將劉應選。丙子,貝勒多爾袞、岳託、薩哈廉、豪格等奏報兵至西喇硃爾格,遇察哈爾囊囊太妃暨臺吉瑣諾木等以一千五百戶降,遂抵額爾克孔果爾額哲所居,其母率額哲迎降。

六月乙酉,貝勒多鐸凱旋,賜良馬五,賞從征將士有差。丁酉,吳巴海、荊古爾代師還,論功亦如之。明登州黃城島千總李進功來降。辛丑,諭曰:「太祖以人民付朕,當愛養之。諸貝勒非時修繕,勞苦百姓,民不得所,浸以逃亡,是違先志而長敵寇也。今朝鮮賓服,察哈爾舉國來附,茍不能撫輯其眾,後雖拓地,何以處之?貝勒大臣其各戢驕縱以副朕意!」壬寅,察哈爾臺吉瑣諾木率其屬六千八百人來歸。癸卯,諭曰:「太祖禁貝勒子弟郊外放鷹,慮其踐田園、擾牲畜也。今違者日眾。語曰:『涓涓不塞,將成江河。』其嚴禁之。」

秋七月癸酉,論漢人丁戶增減,擢參領李思忠等六員官,高鴻中等十一員黜罰有差。

八月庚辰,貝勒多爾袞、岳託、薩哈廉、豪格以獲傳國玉璽聞。先是元順帝北狩,以璽從,後失之。越二百餘年,為牧羊者所獲。後歸於察哈爾林丹汗。林丹亦元裔也。璽在蘇泰太妃所。至是獻之。時岳託以疾留歸化城,多爾袞等率兵略明山西,自平虜衛入邊,毀長城,略忻州、代州,至崞縣。甲申,繪太祖實錄圖成。乙巳,上率大貝勒代善及諸貝勒多爾袞等師次平虜堡。丁未,渡遼河,閱巨流河城堡。

九月癸丑,貝勒多爾袞等師還,獻玉璽,告天受之。額爾克孔果爾額哲及其母來朝。庚午,上還宮。壬申,召諸貝勒大臣數代善罪。眾議削大貝勒號及和碩貝勒,奪十佐領,其子薩哈廉奪二佐領,哈達公主降庶人,褫其夫瑣諾木濟農爵號。上皆免之。

冬十月己卯,以明議和不成,將進兵,遣使賚書諭明喜峰口、董家口諸邊將。管戶部事和碩貝勒德格類卒。癸未,命吳巴海、多濟裏、札福尼、吳什塔分將四路兵征瓦爾喀。

十一月丁未朔,命額爾克孔果爾額哲奉母居孫島習爾哈。

十二月辛巳,哈達公主莽古濟之僕冷僧機首告貝勒莽古爾泰生時與女弟莽古濟、弟德格類謀逆,公主之夫瑣諾木及屯布祿、愛巴禮與其事。會瑣諾木亦自首。訊得實,莽古濟、莽古爾泰子額必倫及屯布祿、愛巴禮皆伏誅。莽古爾泰餘子、德格類子俱為庶人。瑣諾木自首免罪。授冷僧機三等副將。丁酉,謁太祖陵。甲辰,貝勒薩哈廉與諸貝勒及大貝勒代善盟誓,請上尊號。上不許。會蒙古貝勒復來請。上曰:「朝鮮兄弟國,宜告之。」

十年春正月壬戌,皇次女下嫁額爾克孔果爾額哲。

二月丁丑,八和碩貝勒與外籓四十九貝勒各遺書朝鮮,約其國王勸進尊號。戊子,遣使至明邊松棚路、潘家口、董家口、喜峰口、賚書致明帝,索其報書。定諸臣帽頂飾。庚寅,寧完我以罪免。

三月丙午朔,清明節,謁太祖陵。辛亥,改文館為內國史、內祕書、內弘文三院。乙卯,遣貝勒阿濟格、阿巴泰築噶海城。庚申,吳什塔等征瓦爾喀,遣使奏捷。諭曰:「蒙古深信喇嘛,實乃妄人。嗣後有懸轉輪結布幡者,宜禁止之。」乙丑,英俄爾岱等自朝鮮還,言國王李倧不接見,亦不納書,以其報書及所獲倧諭邊臣書進。諸貝勒怒,欲加兵。上曰:「姑遣人諭以利害,質其子弟,不從,興兵未晚也。」丁卯,外籓蒙古十六國四十九貝勒及孔有德、耿仲明、尚可喜俱以請上尊號至盛京。

夏四月己卯,大貝勒代善,和碩貝勒濟爾哈朗、多爾袞、多鐸、岳託、豪格、阿巴泰、阿濟格、杜度率滿、漢、蒙古大臣及蒙古十六國四十九貝勒以三體表文詣闕請上尊號曰:「恭維我皇上承天眷祐,應運而興。當天下昏亂,修德體天,逆者威,順者撫,寬溫之譽,施及萬姓。征服朝鮮,混一蒙古。遂獲玉璽,受命之符,昭然可見,上揆天意,下協輿情。臣等謹上尊號,儀物俱備,伏原俞允。」上曰:「爾貝勒大臣勸上尊號,歷二年所。今再三固請,朕重違爾諸臣意,弗獲辭。朕既受命,國政恐有未逮,爾等宜恪恭贊襄。」群臣頓首謝。庚辰,禮部進儀注。壬午,齋戒,設壇德盛門外。


\end{pinyinscope}