\article{列傳九十}

\begin{pinyinscope}
錢陳群子汝誠孫臻沈德潛金德瑛錢載

齊召南陳兆侖兆侖孫桂生董邦達錢維城鄒一桂

謝墉金甡莊存與劉星煒王昶

錢陳群,字主敬,浙江嘉興人。父綸光,早卒。母陳,翼諸孤以長,語在列女傳。康熙四十四年,聖祖南巡,陳群迎駕吳江,獻詩。上命俟回蹕召試,以母陳病不赴。六十年,成進士,引見,上諭及前事。改庶吉士,授編修。雍正七年,世宗命從史貽直、杭奕祿赴陜西宣諭化導,陳群周歷諸府縣,集諸生就公廨講經,反覆深切,有聞而流涕者。使還,上諭獎為「安分讀書人」。五遷右通政,督順天學政。乾隆元年,以母喪去官。服除,高宗命仍督順天學政,除原官。陳群以母陳夜紡授經圖奏上,上為題詞。疏請增順天鄉試中額,上以官制有定,取者多,用者益遠,國家不能收科目取人之效,寢其議。

三遷內閣學士。陳群屢有建白:嘗疏請嚴治匿名揭帖,無論事鉅細,非據實首告而編造歌謠詩詞,匿名粘貼閭巷街衢,當下刑部依律治罪。疏請廣勸種植樹木,官地令官種,州郡吏種至千本以上,予紀錄;受代時具冊,備地方公用。民地令民種,至五六百本者,予扁額獎賞,成材後聽取用。疏請偏災蠲免分數,分別貧富,富者按例定分數蠲免,貧者被災幾分即蠲免幾分,使之相等。及敕詢州縣耗羨,疏言:「康熙間,州縣官額徵錢糧,收耗羨一二錢不等。陸隴其知嘉定縣止收四分,清如隴其,亦未聞全去耗羨也。議者以康熙間無耗羨,非無耗羨也,特無耗羨之名耳。世宗出自獨斷,通計外吏大小員數,酌定養廉,而以所入耗羨按季支領。吏治肅清,民亦安業。特以有徵報支收之令,不知者或以為加賦。皇上詢及盈廷,臣請稍為變通,凡耗羨所入,仍歸籓庫,各官養廉及各州縣公項,如舊支給。其續增公用,名色不能畫一,多寡亦有不同,應令直省督撫明察,某件應動正項,某件應入公用,分別報銷。各省州縣自酌定養廉,榮悴不一,其有支絀者,應令督撫確察量增,俾稍寬裕。仍飭勿得耗外加耗,以致累民。則既無加賦之名,並無全用耗羨辦公之事,州縣各有贏餘,益知鼓勵。至於施從其厚,斂從其薄,古之制也。及此倉庾充裕、民安物阜之時,大臣悉心調劑,使養廉之入,不為素餐,元氣培扶,帑藏盈溢,然後以三十年之通制國用。宋太祖能罷羨餘,臣固知皇上之聖,不必廷臣建白如張全操其人者,而德音自下也。」

七年,擢刑部侍郎。上令廷臣議州縣常平倉應行諸事,諸臣皆議歉歲減價。陳群疏言:「成熟之年,出陳易新,倉米必不及市米,而民以米值納倉,銀色當高於市易。擬令石減一錢二分,還倉時加穀四五升,以為出入耗費。」

十七年,患反穀疾,連疏乞解職,許之。命其子編修汝誠侍行,且賜詩以寬其意。陳群進途中所作詩,上為答和。時有偽為孫嘉淦疏稿語謗上,上令窮治,陳群自家密疏請省株連,上嚴飭之,而事漸解。二十二年,上南巡,令在籍食俸。二十五年,上為橋梓圖寄賜陳群。二十六年,偕江南在籍侍郎沈德潛詣京師祝皇太后七十壽,命與香山九老會,加尚書銜。上諭:「明歲南巡,諸臣今年已赴闕,毋更遠迎。」二十七年,南巡,陳群偕德潛迎駕常州,上賜詩稱為「大老」。三十年,南巡,復迎駕。是歲陳群年八十,加太子太傅。賜其子汝器舉人,汝誠扈蹕,命從還省視。

三十一年,陳群復進其母陳畫冊,冊有綸光題句。上題詩以趙孟頫、管道升為比。三十五年,上六十萬壽,命德潛至嘉興勸陳群毋詣京師,陳群獻竹根如意,上批劄云:「未頒僧紹之賜,恰致公遠之貢,文而有節,把玩良怡!今賜卿木蘭所獲鹿,服食延年,以俟清晤。」三十六年,上東巡,陳群迎駕平原,進登岱祝釐頌。是冬,復詣京師祝皇太后八十萬壽,命紫禁城騎馬,賜人葠,再與香山九老會。陳群進和詩有句云「鹿馴巖畔當童扶」,上賞其超逸,復為圖賜之。南歸,以詩餞。

陳群裏居,每歲上錄寄詩百餘篇,陳群必賡和,親書冊以進,體兼行草,屢蒙獎許。三十九年,卒,年八十九。上諭謂:「儒臣老輩中能以詩文結恩遇、備商榷者,沈德潛卒後惟陳群。」加太傅,祀賢良祠,謚文端。四十四年,上制懷舊詩,列五詞臣中。

子汝誠,字立之。乾隆十三年進士,改庶吉士,授編修,命南書房行走。四遷至侍郎,歷兵、刑、戶諸部。再典試江南,上命寄諭尹繼善,招陳群游攝山,父子可相見。汝誠試畢,迎陳群入試院,居數日乃還。三十年,乞養歸。四十一年,父喪終,授刑部侍郎,仍在南書房行走。四十四年,卒。

汝誠子臻,字潤齋。自兵馬司副指揮授河南鄧州知州,累遷江西糧道。左授山西平陽知府,復累遷直隸布政使。嘉慶二十一年,授江西巡撫。江西南昌諸府食淮鹽,而與福建、浙江、廣東三省毗連,私販侵引額。臻議疏綱額、緝私販。尋移山東巡撫。兗、曹、沂諸府民素悍,染邪教,盜甚熾。臻請就諸府增設參將以下官,上皆採其議。入覲,以衰老左授湖南布政使,休致。道光十九年,卒。

陳群詩純愨樸厚,如其為人。賡唱既久,亦頗斅禦制詩體。貳刑部十年,慎於庶獄,虛衷詳鞫。高宗嘗以於定國期之。汝誠繼貳刑部,奉陳群之教,持法明允。臻亦善治獄。在平陽,介休民被盜殺其母,攫釧去。民言姻家嘗貸釧,傭或竊釧逃,鄰家子左右之。縣捕三人,榜掠誣服。他日獲盜得釧,民乃言非其母物。獄不能決。臻微服訪得實。撫山東,清庶獄,雪非罪二十餘人,擒教訟者置於法。

沈德潛,字碻士,江南長洲人。乾隆元年,舉博學鴻詞,試未入選。四年,成進士,改庶吉士,年六十七矣。七年,散館,日晡,高宗蒞視,問孰為德潛者,稱以「江南老名士」,授編修。出禦制詩令賡和,稱旨。八年,即擢中允,五遷內閣學士。乞假還葬,命不必開缺。德潛入辭,乞封父母,上命予三代封典,賦詩餞之。十二年,命在上書房行走,遷禮部侍郎。是歲,上諭諸臣曰:「沈德潛誠實謹厚,且憐其晚遇,是以稠疊加恩,以勵老成積學之士,初不因進詩而優擢也。」

十三年,德潛以齒衰病噎乞休,命以原銜食俸,仍在上書房行走。十四年,復乞歸,命原品休致,仍令校御制詩集畢乃行。諭曰:「朕於德潛,以詩始,以詩終。」且令有所著作,許寄京呈覽。賜以人葠,賦詩寵其行。德潛歸,進所著歸愚集,上親為制序,稱其詩伯仲高、王,高、王者謂高啟、王士禎也。十六年,上南巡,命在籍食俸。是冬,德潛詣京師祝皇太后六十萬壽。十七年正月,上召賜曲宴,賦雪獅與聯句。又以德潛年八十,賜額曰「鶴性松身」,並賚藏佛、冠服。德潛歸,復進西湖志纂,上題三絕句代序。二十二年,復南巡,加禮部尚書銜。二十六年,復詣京師祝皇太后七十萬壽,進歷代聖母圖冊。入朝賜杖,上命集文武大臣七十以上者為九老,凡三班,德潛為致仕九老首。命游香山,圖形內府。

德潛進所編國朝詩別裁集請序,上覽其書以錢謙益為冠,因諭:「謙益諸人為明朝達官,而復事本朝,草昧締構,一時權宜。要其人不得為忠孝,其詩自在,聽之可也。選以冠本朝諸人則不可。錢名世者,皇考所謂『名教罪人』,更不宜入選。慎郡王,朕之叔父也,朕尚不忍名之。德潛豈宜直書其名?至世次前後倒置,益不可枚舉。」命內廷翰林重為校定。二十七年,南巡,德潛及錢陳群迎駕常州,上賜詩,並稱為「大老」。三十年,復南巡,仍迎駕常州,加太子太傅,賜其孫維熙舉人。三十四年,卒,年九十七。贈太子太師,祀賢良祠,謚文愨。禦制詩為輓。是時上命毀錢謙益詩集,下兩江總督高晉令察德潛家如有謙益詩文集,遵旨繳出。會德潛卒,高晉奏德潛家並未藏謙益詩文集,事乃已。四十三年,東臺縣民訐舉人徐述夔一柱樓集有悖逆語,上覽集前有德潛所為傳,稱其品行文章皆可為法,上不懌。下大學士九卿議,奪德潛贈官,罷祠削謚,僕其墓碑。四十四年,禦制懷舊詩,仍列德潛五詞臣末。

德潛少受詩法於吳江葉燮,自盛唐上追漢、魏,論次唐以後列朝詩為別裁集,以規矩示人。承學者效之,自成宗派。

金德瑛,字汝白,浙江仁和人。乾隆元年進士,廷對初置第六,高宗親擢第一,授修撰。是歲舉博學鴻詞科,德瑛以薦徵,既入翰林,不更試。旋命南書房行走,充江南鄉試考官。德瑛以原籍休寧辭,不許。再遷右庶子。督江西學政。任滿,上特諭「德瑛甚有操守,取士公明」,命留任。德瑛疏言:「翰林為儲才地,庶吉士宜求學有根柢,器量明達,庶可備他日任使。每科命大臣教習,大臣政事甚繁,但能總大綱。舊有分教例,但由掌院選任,時設時止。乞令掌院於翰詹中擇品學優贍、資俸較深者引見,簡畀分教。」得旨俞允。復四遷太常寺卿,命祭告山西諸行省帝王陵寢。疏言:「女媧氏陵寢殿塑女像,旁侍嬪御,民間奉為求嗣之神,實為黷褻。請毀像立主。」下部議行。督山東學政。十九年,歲饑,上發帑治賑,而鄒、滕諸縣災尤重。有司格於例限,不敢以請。德瑛任滿還京師,入對,具言狀,上特命展賑。遷內閣學士。二十一年,遷禮部侍郎。充江西鄉試考官。使還,經徐州,時河決孫家集,微山湖暴漲,入運河,江南、山東連壤諸州縣被水。德瑛諮訪形勢,入陳於上前,上嘉德瑛誠實不欺。旋命尚書劉統勛董治疏築。二十三年,督順天學政,疏言:「八旗諸生遇歲試,輒稱病諉避,甚至病者多於與試者,請下八旗都統考覈。」

二十六年,擢左都御史,疏言:「秋審舊例,凡已經秋審者謂之『舊事』,現入秋審者謂之『新事』。當九卿、詹事、科道集議時,書吏宣唱名冊,繁重淹滯。其實商榷輕重,多在新事。積年緩決之案,自按察使上巡撫,更三法司,初獄已致慎矣;況三審緩決,久成信讞。諸囚偷生囹圄,幸待十年慶典,得蒙恩赦。然亦裁自聖心,諸臣無與焉。舊事名冊宜罷宣唱。陳案既省,近事得以從容往復,盡心詳審。九卿兼有餘晷治其本職。」上韙其言,下大學士會刑部議,請如德瑛言。十二月,命稽覈通州倉儲,中寒病作,二十七年正月,卒。

德瑛端平簡直,無有偏黨,為上所知。方為少詹事,入對,上曰:「汝元年狀元,尚作四品官耶?」數日擢太常寺卿。及病,上每見廷臣問狀,且曰:「德瑛辛巳生,長朕十歲。」及病革,上方出巡幸,將啟蹕,猶曰:「德瑛久不入值,病必重。」德瑛即以其日卒。三十一年,德瑛子潔成進士,引見,上曰:「汝金德瑛子耶?」德瑛卒已將十年,上猶惓惓如是。

錢載,字坤一,浙江秀水人。雍正十年,副榜貢生,舉博學鴻詞、舉經學,就試皆未入選。乾隆十七年,成進士,改庶吉士,授編修。七遷內閣學士,直上書房。四十一年,督山東學政。四十五年,命祭告陜西、四川岳瀆及帝王陵寢。尋擢禮部侍郎,充江南鄉試考官,舉顧問為第一,四書文純用排偶,上以乖文體,命議處。

呂氏春秋堯葬穀林,史記不書其地。乾隆元年,以山東巡撫岳濬奏,自東平改祀濮州。四十一年,大理寺卿尹嘉銓疏言當在平陽,下部議駁。載督學山東,謁濮州堯陵,自四川還道平陽,得堯陵州東北;及江南典試歸,又至東平求舊時所祭堯陵,參互考訂,以為在平陽者是。史記湯、武皆未著葬地,蓋都於是葬於是則不書,堯亦其例。因疏請釐定。下大學士、九卿議駁,載奏辨;復議,仍寢不行。上諭曰:「經生論古,反覆辨證,原所不禁。但既陳之奏牘,並經廷臣集議,即不當再執成見。載斥呂不韋門下客浮說,不韋即不足取,亦尚不可以人廢言。況其門下客所著書,所謂『懸之國門,不易一字』,豈能謂不足為據?其時去古未遠,或尚有所承述。乃欲在數千年後虛揣翻駁,有是理乎?載本晚達,且其事只是考古,是以不加深問。若遇朝廷政治,亦似此嘵嘵不已,朕必重治其罪。」命傳旨申飭。載疏累數千言,語有未明,復為自注,時謂非章奏體,上亦未深詰也。

四十八年,休致。五十八年,卒,年八十有五。

子世錫,入翰林。時侍郎英廉及載充教習庶吉士,英廉語世錫曰:「君家仍世入翰林,而上命父教其子,當勉為朅、頲以報上恩。」世錫子寶甫,初名昌齡,避仁宗陵,以字行。亦以編修官至雲南布政使。

德瑛論詩宗黃庭堅,謂當辭必己出,不主故常。載初與訂交,晚登第,乃為門下門生;詩亦宗庭堅,險入橫出,嶄然成一家。同縣王又曾、萬光泰輩相與唱酬,號秀水派。語互詳文苑傳。載又為陳群族孫,從陳群母陳受畫法,蒼秀高勁,亦如其詩。

齊召南,字次風,浙江天臺人。幼而穎敏,鄉里稱神童。雍正十一年,命舉博學鴻詞,召南以副榜貢生被薦。乾隆元年,廷試二等,改庶吉士,散館授檢討。八年,御試翰詹各官,擢中允,遷侍讀。九年,以父喪去官。時方校刻經史,召南分撰禮記、漢書考證,命即家撰進。服除,起原官。十二年,遷侍讀學士。十三年,復試翰詹各官,以召南列首,擢內閣學士,命上書房行走。遷禮部侍郎。上於寧古塔得古鏡,問召南,召南辨其款識,具陳原委。上顧左右曰:「是不愧博學鴻詞矣!」上西苑射,發十九矢皆中的,顧尚書蔣溥及召南曰:「不可無詩!」召南進詩,上和以賜。十四年夏,召南散直墮馬,觸大石,顱幾裂。上聞,遣蒙古醫就視,賜以藥。語皇子宏適:「汝師傅病如何?當頻使存問!」幸木蘭,使賜鹿脯十五束。及冬,入謝,上慰勞,召南因乞歸,固請乃許。及行,賜紗、葛各二端。

上南巡,屢迎駕,輒問病狀,出禦制詩命和。上嘗詢天臺、雁宕兩山景物,召南對未嘗游覽。上問:「名勝在鄉里間,何以不往?」召南對:「山峻溪深,臣有老母,怵古人登高臨深之誡,是以未敢往。」上深嘉之。既而,以族人周華為書訕上,逮詣京師,吏議坐隱匿,當流,籍其家,上命奪職放歸,還其產十三四。召南歸,遂卒。

召南易直子諒,文辭渰雅。著水道提綱,具詳源委脈絡;歷代帝王年表,舉諸史綱要:並行於世。

陳兆侖,字星齋,浙江錢塘人。亦幼慧。雍正八年進士,福建即用知縣。舉博學鴻詞,詣京師試,授內閣中書,充軍機章京。乾隆元年,廷試二等,授檢討。十七年,上御經筵,以撰進講義稱旨,擢左中允。御試翰詹各官,復擢侍講學士。再遷順天府府尹。值大水,兆侖心計指畫,撫綏安集,無不得所。畿輔役繁,舊設官車疲敝,議僉富戶應役,兆侖奏罷之。時方西征,發禁旅,兆侖經畫宿頓儲蓄,井井有緒,軍民晏然。二十一年,遷太常寺卿。上謁陵,以同官迎駕失儀,左授太僕寺少卿。再遷太僕寺卿。三十六年,卒。

兆侖精六書之學,尤長經義,於易、書、禮均有論述。為詩文澹泊清遠。

孫桂生,字堅木。嘉慶初,自優貢生授知縣,揀發湖北。時教匪為亂,桂生從廣州將軍明亮擊賊,破孝感,殲魯惟志;戰歸州,御齊王氏:屢有功。授大冶知縣,再遷安陸知府。九年,遭母喪,湖北巡撫章煦疏請留軍。喪終,除荊州知府。三遷,再轉為江寧布政使,署江蘇巡撫。初彭齡劾桂生徵賦不力,奪職;復劾察庫帑不實,上命大學士托津、戶部尚書景安按治,疏言:「桂生察庫帑無弊,徵賦亦逾十之七。」召詣京師,旋授甘肅布政使。再轉,復遷江蘇巡撫。上六十萬壽,蠲各行省民間逋賦。桂生疏言:「曠典殊施,當令澤及於民。請自嘉慶元年起至二十二年,詳察民間逋賦,毋令官吏因緣為奸。二十二年漕項,例至二十四年奏銷,民逋請並蠲除。」又言:「民間逋賦有由州縣移他款代納者,今既蠲逋,當令現任州縣期十年償所移款。」皆議行。命署蘇州織造,兼領滸墅關,兼署兩江總督。宣宗即位,召詣京師,以三品京堂待缺,旋命休致。道光二十年,卒。桂生子憲曾,進士,官至詹事。

董邦達,字孚存,浙江富陽人。雍正元年,選拔貢生。以尚書勵廷儀薦,命在戶部七品小京官上行走。十一年,成進士,改庶吉士,授編修。乾隆三年,充陜西鄉試考官,疏言官卷數少,以民卷補中,報聞。授右中允,再遷侍讀學士。十二年,命直南書房,擢內閣學士,以母憂歸。逾年,召詣京師,命視梁詩正例,入直食俸。十五年,補原官,遷侍郎,歷戶、工、吏諸部。二十七年,遷左都御史,擢工部尚書。二十九年,調禮部。三十一年,調還工部。三十二年,仍調還禮部。三十四年,以老病乞解任,上諭曰:「邦達年逾七十,衰病乞休,自合引年之例。惟邦達移家京師,不能即還里。禮部事不繁,給假安心調治,不必解任。」尋卒。賜祭葬,謚文恪。

邦達工山水,蒼逸古厚。論者謂三董相承,為畫家正軌,目源、其昌與邦達也。子誥,自有傳。

錢維城,字宗盤,江南武進人。乾隆十年一甲一名進士,授修撰。功令,初入翰林,分習清、漢文。維城習清文,散館列三等。上不懌,曰:「維城豈謂清文不足習耶?」傅恆為之解。命再試漢文,上謂詩有疵,賦尚通順,仍留修撰。是歲即遷右中允,命南書房行走。三遷,再轉為刑部侍郎。疏請申明律例:「事主殺盜賊移尸,有司輒置勿論。本律科移尸罪,反至流徒。請凡殺人律得勿論者,雖移尸仍用本律。殺奸之獄,奸夫拒捕,有司輒用鬥殺律定讞。殺奸殺拒捕者,反重於殺不拒捕者。請用殺拒捕罪人律勿論。」下部議行。三十四年,命偕內閣學士富察善如貴州會湖廣總督吳達善按治威寧州知州劉標虧帑,巡撫良卿、前巡撫方世俊等皆坐譴。三十五年,古州苗香要為亂,復命偕吳達善及巡撫宮兆麟督剿。香要多力而狡,苗女迫根為羽翼,煽旁寨出掠。維城如古州,督總兵程國相破烏牛、佳居諸寨,獲迫根。維城乃自烏牛如佳居宣諭,解脅從。督兵破朋論大箐,香要獨身跳去。乃令先撤兵,遣詗香要,卒擒而殲之。亂定,諭議敘。三十六年,雲南龍陵戍卒四十去伍走,既就獲,大吏請悉誅之。維城入對,言:「伊犁戍卒荷校一月,今用法過重。且戮於獲所,邊兵何由知?不如械至龍陵,倍其罰,荷校三月,足以儆眾。」上從之。三十七年,丁父憂,歸,以毀卒。謚文敏。

維城工文翰,畫山水幽深沈厚。錢陳群謂維城通籍後畫益工,蓋得益於邦達云。

鄒一桂,字原褒,江南武進人。祖忠倚,順治九年一甲一名進士,官修撰。一桂,雍正五年二甲一名進士,改庶吉士,授編修。十年,授雲南道監察御史,疏禁官媒蓄婦女為奸利。乾隆七年,轉禮科給事中,疏言:「刑部諸囚已結入北監,未結羈南所。今察視監所,已未結雜收,請如例分禁。」又言:「奉命下部議諸事,科道輒於部議未上之先,攙越瀆陳,請申飭。」上韙其言。湖南巡撫許容坐誣劾糧道謝濟世罷,復命署湖北巡撫。一桂與給事中陳大玠具疏論列,謂:「容狡詐欺公,僅予奪職,已邀寬典;今復任封疆,何以訓天下?乞降旨宣示臣民,俾曉然於黜陟之所以然,斯國法昭而吏治有所率循。」上為罷容。十年,遷太常寺少卿,疏言:「律載獄具全圖,鐵索鈕金尞,俱有定式。獄官以防範為辭,匣床以束其身,鐵簫以直其項,觀音圈以攣其手足。部議禁非刑,日久復創新制,令諸囚排頭仰臥,橫穿長木,壓其手足,與匣床無異,請敕嚴禁。」從之。四遷為禮部侍郎。同部侍郎張泰開舉一桂子志伊為國子監學正,又坐徇尚書王安國、左都御史楊錫紱祀其父鄉賢,屢下部議,二十一年,左授內閣學士。二十三年,乞致仕。三十六年,詣京師祝上壽,加禮部侍郎銜,在籍食俸。三十七年,歸,卒於東昌道中。加尚書銜。

一桂畫工花卉,承惲格後為專家。嘗作百花卷,花題一詩,進上,上深賞之,為題百絕句。晚被薄譴,歸猶賦詩餞之云。

謝墉,字昆城,浙江嘉善人。乾隆十六年,上南巡,墉以優貢生召試,賜舉人,授內閣中書。十七年,成進士,改庶吉士,授編修。坐撰閩浙總督喀爾吉善碑文語失當,下部議,降調。二十四年,回部平,墉擬鐃歌上,上命復官,直上書房。五遷工部侍郎,督江蘇學政。四十三年,調禮部。四十五年,調吏部。廣西全州知州彭曰龍坐縱革役復充,奪官,詣部請捐復。大學士阿桂領吏部,將許之,墉以為不可。時有山東商河教諭侯華捐復,方議駁,墉援以例曰龍。阿桂疑墉為華地,奏聞。上命訊,華力言無囑託,乃用墉議,不許曰龍捐復。四十八年,復督江蘇學政。五十一年,任滿,還京師。上問洪澤湖運河水勢,墉奏:「洪澤湖漸高,民間傳說『昔如釜,今如盤』,請加疏濬。」五十二年,上以總督李世傑奏洪澤湖水注清口暢流,命墉往與世傑勘湖水淺深。尋奏湖水深至十丈,淺亦在一二丈間,墉自請議處。上以湖水前年較淺,墉得自傳聞,據以入告,茲既已勘明,免其議處。

墉兩任江蘇學政,士有不得志者,以偶語譏誚。阿桂偶以聞,上命巡撫閔鶚元訪察。鶚元言墉初任聲名平常,後任頗為謹飭。上命降授內閣學士。五十四年,上察直上書房諸臣多曠班,墉七日未入直,復降編修,在修書處效力。五十六年,復命直上書房。六十年,休致。尋卒。

墉在上書房久,仁宗方典學,肄習詩文,高宗命墉講授。嘉慶五年,加恩舊學,贈三品卿銜,賜祭葬。子恭銘,進士,改庶吉士,散館歸班,是歲授內閣中書。墉以督學蒙謗,然江南稱其得士,尤賞江都汪中,嘗字之曰:「予上容甫,爵也;若以學,予於容甫北面矣!」乾隆中直上書房諸臣以學行稱者,又有金甡、莊存與、劉星煒。

甡,字雨叔,浙江錢塘人。初以舉人授國子監學正。乾隆七年,舉禮部試第一,廷試復第一,授修撰。三遷侍講學士。二十二年,直上書房,擢詹事,再遷禮部侍郎。三十八年,上幸熱河,從,方入直,遘疾遽僕。大學士劉統勛以聞,命予假。甡乞休,允之。明年秋,疾間,乃得歸。四十七年,卒,年八十有一。

甡在上書房十七年,直諒誠敬,所陳說必正義法言,諸皇子皇孫皆愛重之。

存與,字方耕,江南武進人。乾隆十年一甲二名進士,授編修。四遷內閣學士。二十一年,督直隸學政。按試滿洲、蒙古童生,嚴,不得傳遞,群閧。御史湯世昌論劾,命奪存與官。上惡滿洲、蒙古童生縱恣,親覆試,搜得懷挾文字。臨鞫,童生海成最狡黠,言:「何不殺之?」上怒,立命誅之。閧堂附和者三人,發拉林種地;四十人令在旗披甲;不得更赴試。並以存與督試嚴密,仍命留任。擢禮部侍郎。遭父喪。服除,補內閣學士,仍授原官,直上書房。遭母喪。服除,補原官。五十一年,以衰老休致。五十三年,卒。

存與廉鯁。典浙江試,巡撫餽金不受,遺以二品冠,受之。及塗,從者以告曰:「冠頂真珊瑚,直千金!」存與使千餘里返之。為講官,上御文華殿,進講禮畢,存與奏:「講章有舛誤,臣意不謂爾。」奉書進,復講,盡其旨,上為留聽之。

弟培因,字本淳,乾隆十五年一甲一名進士,官至內閣學士。

劉星煒,字映榆,江南武進人。乾隆十三年進士,改庶吉士,授編修。遷侍講,督廣東學政。疏言:「鶴山立縣初,有廣州民一百五戶請修城入籍,緣是開冒考之弊,請以有廬墓、田糧在縣者為限。」丁母喪,去。服闋,補原官。督安徽學政,請童生兼試五言六韻詩。童試有詩自此始。累遷侍讀學士。二十九年,直上書房,再遷禮部侍郎。卒。

王昶,字德甫,江蘇青浦人。乾隆十九年進士。南巡,召試,授內閣中書,充軍機章京。三遷刑部郎中。三十二年,察治兩淮運鹽提引,前鹽運使盧見曾坐得罪,昶嘗客授見曾所,至是坐漏言奪職。雲貴總督阿桂帥師討緬甸,疏請發軍前自效。上命大學士傅恆出視師,嗣以理籓院尚書溫福代阿桂,皆以昶佐幕府。溫福移師討金川,昶實從,疏請敘昶勞,授吏部主事。既,復從阿桂定兩金川,再遷郎中。刑部侍郎袁守侗按事四川,上命察軍中事,還奏言昶治軍書有勞。四十一年,師凱還,擢昶鴻臚寺卿,仍充軍機章京。三遷左副都御史,外授江西按察使。數月,以憂歸。起直隸按察使,未上,移陜西按察使。

在陜西凡十年,值回田五為亂,軍興,昶繕守具,佐治軍需,疏請清釐保甲,禁民間蓄軍器。遷雲南布政使。河南伊陽民戕知縣,竄匿陜西境未獲,昶如商州督捕,上命俟得賊詣京師覲見。昶既得賊,入謁上,自陳疲憊,乞改京職,上溫旨慰遣,乃上官。以雲南銅政事重,撰銅政全書,求調劑補救之法。旋調江西布政使。五十四年,內遷刑部侍郎。屢命如江南、湖北讞獄。五十八年,以老乞罷,上許之,方歲暮,諭俟來歲春融歸里。昶歸,遂以「春融」名其堂。嘉慶元年,詣京師賀內禪,與千叟宴。四年,復詣京師謁高宗梓宮。十一年,卒。

昶工詩古文辭,通經。讀硃子書,兼及薛瑄、王守仁諸家之學。蒐採金石,平選詩文詞,著述傳於世。

論曰:國家全盛日,文學侍從之臣,雍容揄揚,潤色鴻業。人主以其閒暇,偶與賡和,一時稱盛事。未有彌歲經時,往復酬答,君臣若師友,如高宗之於陳群、德潛。嗚呼,懿矣!當時以儒臣被知遇,或以文辭,或以書畫,錄其尤著者。視陳群、德潛恩禮雖未逮,文採要足與相映,不其盛歟!


\end{pinyinscope}