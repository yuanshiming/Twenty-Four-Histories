\article{列傳二百八十六}

\begin{pinyinscope}
孝義三

岳薦張廒黃學硃吳伯宗錢天潤蕭良昌李九

張某程含光陳福譙衿黃成富李長茂任天篤

趙一桂黃調鼎楊藝咸默李晉福胡端友硃永慶王某

張瑛郭氏僕胡穆孟苑亮楊越子賓吳鴻錫

韓瑜程增李應卜塞勒王聯黎侗李秉道趙瓏

蔣堅李林孫高大鎬許所望邢清源王元鳳瑞

方元衡葉成忠楊斯盛武訓呂聯珠

岳薦,江南山陽人。明末為諸生。事父母謹,居喪哭踴,氣息僅屬,乃病羸終其身。庶弟甫生而其母暴疾死,薦亦生女,乃令妻棄女而乳其弟。弟病瘍,日夜啼,夫婦迭拊之,遂俱生瘍,血淋漓被體,不以為苦。

張廒,陜西盩厔人。順治初,山賊破其堡,殺廒兄廠,並掠廠子去。廒愍廠死且無後,負其子入山易廠子歸。方謀贖子,山賊引去,其子幼不能從,遂殺之。廒復生子,與廠子並成立。

黃學硃,福建甌寧人。諸生。順治間,縣有土寇,執學硃及其弟。度不能兩全,乃紿賊曰:「家有薄產,釋弟歸鬻產,以其值贖我,何如?」賊疑,欲遣學硃,學硃曰:「我秀才,質重於弟。」賊遂釋弟歸。實無產,贖不至,學硃遂被戕。

吳伯宗,山西稷山人。早喪父母,二弟幼,與相依。居數年,先後皆失之。伯宗求弟遍遠近,久之,得季弟京師,為高氏僕。高氏遇之厚,曰:「吾為子善撫,子求得仲弟,與之俱歸。」又久之,伯宗得仲弟消息,在寧古塔,乃躬往蹤跡之。仲弟屬將軍部,投牒訟焉。庭質,辭未畢,伯宗忽躍起,主者怒,撲之,血被面。伯宗徐曰:「民非敢與抗,適見略吾弟者,奴吾弟者,皆法所不宥,顧美衣帽,平立官側。民兄弟良家子,為奸人誘掠,萬里投命,官不明其冤,乃視若罪囚,使跪而聽命,民是以不服。」主者悟,白將軍,歸其仲弟。時正冬,兄弟相扶行冰雪中,至京師,與季弟同歸。

錢天潤,江蘇宜興人。少孤,為人傭耕,得錢必奉母。母死,以奉其兄。有女弟嫁而寡,甥二,方幼,天潤往視之。女弟泣言:「夫死子幼,不知所以為計。」天潤問其意,女弟言:「原守節,第苦貧。」天潤曰:「妹無憂!吾助汝。」遂為女弟耕以給食。三年,女弟死,撫二甥,畢姻娶。

蕭良昌,湖南邵陽人。家貧,貿漆,事父孝。兄弟四,良昌其少季。析居,伯、仲、叔皆有一子,伯、仲早卒,叔攜其子出游,良昌召伯、仲子與同居,率之貿荊、襄間。家漸起,始娶婦。歲除,具酒奉父,父語良昌曰:「兒能撫存孤侄甚善,顧安得汝叔兄父子復還耶?」良昌跪白父曰;「兒欲行求久矣。」明歲遂行。時傳叔兄在雲南,良昌行六閱月,貲且盡,途窮哭泣,目盡腫。晨行至一村,遇曉汲者,則叔兄子也,乃與見叔兄,偕歸,父乃大慰。年八十餘,乃為諸子析居,厚兄子而薄其子,其子亦受之無間言。

李九,江蘇贛榆人。家青口,兄七,與其鄰爭地而訟,知縣吳元納鄰賕,逮七,下典史費長春加楚毒焉,七自經死,九誓雪兄枉,訴州不得直,訴監司,獄下州,仍不得直。走京師,訴都察院,命下江蘇巡撫。元、長春賂承審官,責九健訟,加非刑,而令九所親關說,啗以重利,九不應。九憤且楚,發病,元等賄醫將毒九。會按察使陳繼昌至,親鞫,九得直。獄成,黜元,戍長春,誅縣役二。九嘆曰:「兄枉雪,死無憾!」歸未至,卒。青口士民具鼓樂迎其喪。

張某,甘肅通渭人。兄弟皆貧,為木工,相友愛。將析產,兄曰:「均之。」弟曰:「弟子一,而兄之子五,如兄言,弟子則富矣!諸侄獨非父母孫乎?當視人為分。」兄曰:「不可,父母先有子,未嘗有孫。」議不決,乃析為三,兄二而弟一。兄弟皆逾八十,常言:「誰先死,必呼與俱去。」兄卒,弟慟幾絕,不食七日,亦卒。

程含光,安徽休寧人。出游,得貲以養親。嘗偕弟自六安歸,策蹇經篛嶺。日暮風起,虎突出,攫弟去。含光驚墜地,持短鞭力追,左手據虎頸,右以鞭捶虎,大呼震山谷。虎舍弟嵎吼,含光負弟疾趨投嶺下旅舍。弟息僅屬,灌以湯,徐甦,肩創十餘,血淋漓。人言虎牙毒,血不盡且死,含光吮之,血盡出,乃瘥。其後含光卒,弟每言遇虎事,解衣示人,輒流涕不巳。

陳福,福建永春人。居西溪,同居十二世,家範簡肅。世以一人督家事,子孫率教醇樸,未有訟者。

譙衿,湖南沅江人。同居七世,有家訓二十條,喪祭無失禮。

黃成富,福建連江人。同居六世,子弟各執其業。方田作,諸婦饁,以一婦守家,視臥兒於筐,饑則哺,不問何人子。懸衣於桁,共衣之,垢則澣,不問何人衣。雍睦無間言。

李長茂,福建海澄人。同居四世,建祠,置祭田,立義學,著家規、法戒各十條示子孫。子五福,順治六年進士,官刑部侍郎,兄弟八人皆友愛。

任天篤,河南偃師人。乾隆中,巡撫何裕成言天篤九世同居,高宗賜以詩,賚鏹帛,表宅里。初,天篤祖開昌生五子,欲定議不析產,觀諸子意。納金麥囷中,子士堯、士舜得以告,開昌曰:「此天賜,汝二人取之!」以「子無私蓄」對。開昌悅,乃定議不析產。宗經、傳,為家訓,教子弟毋侈,毋急利,毋入城市,毋傳述時事,務耕田讀書,惟許學醫,亦毋取酬,不則執百工業以佐家。婦初至,長者以家訓教之,不率,令暫還母家,悟,乃迎歸。平居布衣椎髻操作,毋私饋,毋飾容觀,毋適私室。年五十不執役,寡毋入廚,稍厚其衣食。女適人寡,毋再嫁。至天篤,上溯開昌祖光玉,下見玄孫瑞豐,通九世男婦百六十餘人共爨。吏問天篤何術能不析產,天篤曰:「不忍也!」人傳其語,謂視張公藝書「忍」字義尤大而遠。

其後傅麟瑞、張璘,皆以七世同居賜詩旌獎。麟瑞,魯山諸生;璘,涇陽諸生。

趙一桂,不知其邑里。崇禎末,以省祭官署昌平州吏目,被檄為莊烈帝、後營葬。師入關,定京師,列狀申州,略曰:「三月二十五日奉順天府檄,穿田妃壙,葬崇禎帝、後。四月初三日發引,初四日下窆。州庫如洗,葬日促,監葬官禮部主事許作梅無策,職與義士孫繁祉等十人,斂錢三百四十千,僦夫穿壙。至初四日,羨道開通,啟壙宮門入,享殿三間,陳祭品。中設石案一,懸鐙二。旁列錦綺繒幣五色,具生存所用器物★M7具,皆貯以硃紅木笥。左傍石床一,床上氍毹衾枕。又啟中羨門,內大殿九間,中為石床,置田妃棺槨。帝、后梓宮至,停席棚,陳羊豕、金銀紙錁、祭品。率眾伏謁,哭,盡哀。職躬督夫役移田妃柩於右,奉周皇后梓宮於左,乃安先帝梓宮居中。先帝有棺無槨,移田妃槨用之。梓宮前各設香案祭器,職手燃萬年鐙,度不滅。久之,事畢,掩中羨,閉外羨門,復土與地平。初六日,又率諸人祭奠號哭,呼集居民百餘人,畚土起塚,又築塚墻高五尺有奇。幸本朝定鼎,為先帝建陵殿三間,繚以周垣,使故主陵寢,不侵樵牧,雖三代開國,無以加之。一時斂錢者:繁祉,諸生劉汝樸、白紳、徐魁、李某、鄧科、趙永健、劉應元、楊道、王政行,皆州民。」康熙中,嘉興譚吉璁至昌平,得故吏牘,採入所為肅松錄,邵長蘅又為之文,謂是時李自成據京師,禮部主事改禮政府屬,蓋一桂不知自成所改官制,而政行有子乞韓菼表墓,亦書其事。

黃調鼎,字鹽梅,河南洛陽人。諸生。其女兄,明福王由崧妃也。早卒,葬洛陽。福王稱帝南京,追爵妃父奇瑞洛中伯,以其長子九鼎襲,亦官調鼎。福王選立後、妃,巡撫山陰祁彪佳之女與焉,命以彪佳少女妻調鼎。南都破,九鼎降,馬士英挾福王母鄒太后至浙江。兵敗,太后匿山陰民家,調鼎走依祁氏,與相聞。福王死京師,求得其柩,載歸洛陽,葬故妃園。迎鄒太后奉養,至卒,葬福恭王園。調鼎棄諸生,不出。

楊藝,字碩父,廣西臨桂人,大學士瞿式耜客也。闊略無所忌諱,同幕者稱為癡藝,因以自號。已,終不合去。孔有德徇廣西,破桂林,執式耜及總督張同敞,不屈死,藝衰絰懸紙錢滿衣,號哭營、市間,請斂式耜,有德聞而義焉,遂許之,令並斂同敞。有姚端者,式耜門人。藝與謀,斂式耜及同敞,淺葬風洞山麓,築室於旁,守墓不去。時明給事中金堡去為僧,將上書有德乞斂式耜等,知藝先之,乃罷。以書稿寄式耜子,頗流傳人間,而罕知藝者。堡紀其事甚詳,且曰:「以吾書掩藝,吾為竊名,瞿氏子為負德。」

咸默,字大咸,江南山陽人。明諸生,侍郎左懋第客也。福王遣懋第等詣京師,默與司務陳用極,副將艾大選,游擊王一斌,都司張良佐、王廷佐,守備劉統從。使事畢,留勿遣。大選從令薙發,懋第怒笞之,自殺。南京破,懋第與用極、一斌、良佐、廷佐、統,皆以不屈死。默送懋第喪歸葬萊陽,又送用極喪歸葬昆山,一斌等為淺葬京師郊外。默託堪輿術游四方,嘗作哭萊陽詩以吊懋第,淒楚,人不忍讀。

李晉福,直隸景州人。事諸生趙遵譜為僮。師入塞,略地至州,遵譜方出游,騎而行,晉福從,倉卒被掠去,家人不知也。越數日,晉福潛還,告家人,即復從遵譜出塞。遵譜馬為人奪,與晉福徒跣行。久之,有騎過,則遵譜馬也。遵譜直前欲奪之,騎者抽刀斫遵譜僕,幾死。晉福負歸為里創,僅乃得愈。遵譜惷直,晉福力戒毋負氣取禍,在兵中稍久相習。晉福弟遵譜,有勞役,必代之。後三年,得間,遣遵譜亡歸。歸一年,晉福亦逃入塞。

胡端友,湖南寧鄉人,劉光初僕也。順治初,光初妻胡遇寇,以幼子付端友,端友負而逃,寇逐之,力奔得脫。至其家,釋負,僕,久之乃蘇。胡死於寇,其子得成立。至乾隆中,丁近二千,劉氏祀端友於祏。

硃永慶,字長源,順天大興人,故明宣府巡撫之馮子也。師入關,永慶見俘,隸漢軍正黃旗,僦屋居。永慶修幹美髯,負氣節,好佛,主者賢之,將賜以婦,命視諸俘,恣所擇。武進楊兆升,仕明官給事中,起兵死。妾姚見俘,薙發矢守節。永慶夙聞之,乃自名故殉難宣府巡撫子,擇姚以請,引歸所居室。向夕,姚拜永慶乞哀,永慶曰:「吾將全夫人節,非特哀之而已。」乃誦佛至旦,凡三夕,居停覘知之,問曰:「君不近婦人,安用此贅疣?」永慶曰:「此縉紳婦,吾非欲妻之,欲完其節耳。恐機洩,故且同室,然非誦佛不可。乃為君偵得,幸終為吾諱。」居停感焉,乃治別室以居姚。久之,事聞於主者,主者益賢之,令姚寄書其家,以其母若弟來,予貲遣之還。

王某,江南如皋人,隸也。順治初,縣人許德溥坐不薙發死,妻當流,王欲脫之,思不得其策,夜不寐,其妻怪問之,語以故。其妻曰:「此義舉也!然非得一人代不可。」王曰:

「安所得代者?」其妻曰:「吾當成子義舉,原代行。」王伏地叩頭謝。乃匿德溥妻,而以其妻行,行數千里,至流所。縣人義之,斂金贖歸,夫婦終老於家。

張瑛,字玉採,山西汾陽人,居西官村。順治六年,姜瓖亂,眾劫東官村趙氏,盡殺其人。獨一子亡歸瑛,瑛納之,眾索焉,瑛不與。瓖亂定,瑛助趙氏子訟於官,誅劫者。當亂急,村人將走避,瑛曰:「賊未至先走,能保必全乎?孰若為守計!」眾曰:「如無砦堡何!」瑛曰:「砦堡誠不可猝為,環村而溝焉,其可。」遂為溝,務深廣。瑛家有樓,貯村人財物其中。既而賊大至,逾溝,村人退保樓。瑛見賊渠據胡床坐而指揮,發石中之,立斃。餘賊怒攻樓,取薪將焚,眾汲井以救。持數日,乃稍稍去。瑛率眾出擊之,賊奔潰,村以得全。瑛家饒,歲終,必出粟周鄰里。康熙三十六年,饑,縣民鬻田,貶其值,瑛輒收之,得田且千畝。明年大穰,瑛榜諸村曰:「原贖者聽。」不十日盡贖去。瑛卒,年九十有一。

郭氏僕,失姓名,山西聞喜郭景汾家僕也。姜瓖反,縣人章惇為亂,殺景汾祖及父。景汾方三歲,僕負之走,得免。瓖敗,惇降,得官。景汾讀書成進士,上僕義,被旌。景汾圖復仇,顧惇已遇赦,知縣邵伯麟為之解,令惇謁景汾祖、父墓,且詣景汾謝。居無何,景汾擊殺

惇,斷其首祭祖、父,而身詣獄。伯麟義景汾,具獄辭言惇謀反,景汾率眾擊殺之。大吏覆讞惇謀反事無有,乃坐景汾擅殺,伯麟意出入人罪,皆論死。逾年遇赦,減死,戍福建。耿精忠反,官景汾,事定,逮京師,以從逆見法。僕自聞喜走京師,為具斂。惇子訐僕不當收罪人尸,下刑部,僕言:「某負三歲主艱難萬死中,辱以義被旌。景汾雖被罪死,固某主也。主死,僕不為之收,是為無義。某原死,不敢負前旌。」獄上,聖祖哀而宥之。當精忠官景汾。亦欲官伯麟,景汾言:「是不辦一縣令,何能為?」遂不用,以是免。

胡穆孟,福建人,失其縣。順治間武舉。與連江沈廷棟同歲,相善。耿精忠反,徵穆孟,避匿廷棟家。廷棟寓書於其友,詆精忠,穆孟竊見之,慮書發且得禍,易書為隱語,邏者得書,猶以詆精忠見收。穆孟以語其妻王,王謂當自承以脫廷棟。穆孟乃詣吏,吏使與廷棟各具書,辨其跡,釋廷棟而殺穆孟。穆孟死,王詣市,綴穆孟首,具衣冠為斂,囑子於其叔,且及廷棟,遂縊於尸側,市人皆感泣。師克福建,恤穆孟,廕其子焉。

苑亮,江南亳州人。事州人韓斌為僕。斌舉武科,授福建興化守備。耿精忠反,脅授副將,浙江總督李之芳討焉。移江南,錄斌子世晉。亮從之行,之芳授以札,使招斌。亮度精忠兵所置堠,為邏者所執。問誰何,亮自陳,言斌家被籍,南來投斌。主者監亮見斌,而不許交語。亮偽遺履,斌發視,得之芳札,乃單騎詣之芳降。亮陷賊中,被刑訊,終不言齎札事,遂死。之芳作傳表之。

楊越,初名春華,字友聲,浙江山陰人。所居曰安城,因以為號。為諸生,慷慨尚俠。康熙初,越友有與張煌言交通者,事發,辭連越,減死,流寧古塔。例僉妻,與其妻範偕行,留老母及二子家居。寧古塔地初闢,嚴寒,民樸魯。越至,伐木構室,壘土石為炕,出餘物易菽粟。民與習,乃教之讀書,明禮教,崇退讓,躬養老撫孤。贖入官為奴者,蕭山李兼汝、蘇州書賈硃方初及黔沐氏之裔忠顯、忠禎皆廩焉。又贖明大學士硃大典孫婦,河南李天然希聲夫婦。凡貧不能舉火及婚喪,倡出貲以周,民相助恐後。吝,則嗤之,曰:「何以見楊馬法?」馬法猶言長老,以敬越也。母終於家,年餘始聞喪,哀慟,杜門居三年。

子賓,出塞省越,越初戍年二十四,至是已六十八。賓還,叩閽乞赦越,事未行。子寶,復出塞省越。又二年,越卒於戍所,例不得歸葬,賓、寶請不已,又二年乃得請。迎範奉越喪以歸,民送者哭填路。賓撰柳邊紀略,述塞外事甚詳。

吳鴻錫,字允康,福建晉江人。父德佑,康熙初,客浙江,兵部郎中噶尼布奉命督造戰艦,延德佑入幕。數月德佑卒,鴻錫方七歲,噶尼布攜至京,將子之,鴻錫請呼以伯,曰:「父一而已。」噶尼布奇之,曰:「七歲兒能辨此耶?」噶尼布故廉,家漸困,鴻錫為督芻牧,私市書冊、弓矢習之。通滿、漢文,精騎射。噶尼布從兄云麟以平臺灣功授溫州參將,至京師,欲以鴻錫行,噶尼布諾之。鴻錫流涕曰:「我七歲育於公,今我壯而公老,父子幼,必俟其成立,我乃歸。」鎮國公海清,噶尼布壻也,義鴻錫俾入旗。

噶尼布卒,妻哀甚,得狂疾。子和順、和鼐、和麟。和順才七歲,鴻錫為治喪,持家政,延師教和順兄弟,稍長,為娶婦。和順年十六,有忌之者,授以護軍,將困苦之。每值宿,鴻錫佩刀以從,露坐終夜。

大學士阿蘭泰為噶尼布故交,鴻錫率和順兄弟候其門,和順試除中書。師征噶爾丹,和順從軍,以功擢禮部主事。有召和順飲者,佐以博,鴻錫持刀逕入坐以和順歸。他日,或問鴻錫:「人可殺乎?」鴻錫曰:「殺人罪不過死,吾受撫孤託,而坐視其溺於燕朋,誠生不如死。死而諸孤知勉,則死賢於生矣。」和順自是不復與人飲。

山東饑,遣官治賑,和順與焉,鴻錫從之。武城廩未發,出私錢散米,又慮饑者驟飽且致斃,瀹萊菔飲之,全活無算。和順尋榷密雲關,鴻錫曰:「負販小民不得取其稅,額不足,可以家財補焉。」民歡趨之,額亦足。

和鼐習舉業,鴻錫督之,慮其怠,穴幾貫鐵索自系守之,和鼐驚謝,讀益力,以副榜貢生得官。

和麟年十六,鴻錫偕詣永定河效力,水大至,巡撫於成龍夜行堤上,見有向河拜且泣者,問之,鴻錫也,解衣旌之。工竟,和麟議敘筆帖式,擢刑部郎中。

鴻錫不得歸,募工寫父母遺像,檢父遺衣冠招魂葬之。年五十八,卒。和順兄弟去纓席地,如父母喪。

韓瑜,字玉採,山東濰縣人。少孤,事母孝。母歿,哭泣三年。既除喪,祭墓未嘗不哀,年八十如故。冠時母有衣一襲,★M8篋中,賓祭則服之,衣敝不棄。將卒,命以斂,猶舉孟郊詩曰:「此慈母手中線也。」事兄謹,兄弟皆八十,無改常度。產不過中人,好施予,多蓄書,遇寒士則遺之。族黨長不能婚娶,喪不能葬,必佽以貲。族子貧,贈以秫十石,使居賈。得贏,倍以償,不受。康熙四十三年,饑,民鬻子女,罄所蓄,得九人,不立券。歲豐,悉遣還之。卒時八十有六。

程增,字維高,江南歙縣人。父朝聘,自歙移家安東。歸省墓,病作。增冒風渡江,六日夜行千五百里,至則朝聘已歿。母唐病復作,急還,又已歿,乃絕意仕進。安東地卑,母柩在堂,水大至,增與一僕力升柩木案上。既葬,復移家山陽為賈,而使二弟就學。父母之黨死而無歸者畢葬焉,餘皆定其居,使有恆業。析田立塾,以養以教。友有急難,以千金脫之,後更相背,窮復來自解,待之如初。康熙初,河、淮溢,增出家財修邗溝兩岸堤十里,河道總督張鵬翮以聞。康熙四十四年,聖祖巡視芒稻河,召增入見,書「旌勞」二字以賜。兩江總督於成龍好微行,奸人因造言傾怨家,獄或失入。增謁成龍,力言其弊,指事為徵,成龍曰:「微子言,吾安知人心抗敝至此!」久之,卒。

李應卜,河南郟縣人。早失父母,叔丕基遺側室,事如母,壽百歲終。侄緯,孤,飲食教誨之。病作,必數視之,曰:「我夜不能起,然終宵未成寢也!」弟應會亡,病甚,一夕須發皆白。侄緝幼,食必呼共案,出必視而行,返必問在何所。施及於鄉人,有典其田而遠游者,以子託焉,久之,為娶婦,且復其田。有喪其妻者,為之復娶,予田,俾資以生。有貧欲遠徙者,予之粟,留勿徙。有傭於其肆,負金,病且死者,為之蠲其逋,厚給其妻子。有持金入其肆市粟者,視金有官封,與粟,遣之去。持金詣縣庭,知縣方以庫失金笞吏,應卜以金上,具言始末,事乃白。乾隆二年,縣舉應卜行事上大吏,請旌表其門曰「義士」。

塞勒,滿洲人。官苑副。與惠色友,塞勒老無子,時引以為戚。惠色曰:「我已有二子,今婦又有身,男也,為君子。」已而得男,命曰奇豐額。既免乳,以畀塞勒,塞勒與其妻撫以為子。年十六,將應童子試,當具三代,塞勒曰:「吾寧無子,不可改祖宗,欺君父!」乃攜奇豐額還惠色。奇豐額初不自知惠色子,塞勒語以故,駷馬去。奇豐額遂還為惠色子,乾隆三十四年成進士,授刑部主事,累遷江蘇布政使。塞勒及其妻相繼卒。五十七年,奇豐額擢江蘇巡撫,入覲,涕泣陳本末,請以本身封典貤封塞勒,並以第三子廣麟為塞勒後。上命具疏,下部議,皆不許,上特允之。

奇豐額,黃氏,先世朝鮮人,隸內務府滿洲正白旗。坐事罷官,終內務府主事。

王聯,字鷺亭,江蘇泰州人。諸生。應乾隆四十五年江南鄉試,聯與友沈某偕。沈病於喉,欲歸,聯不入試,送之還。至龍潭,沈病益劇,聯伴之寢,病者口腐,穢觸鼻,不問。輿行慮其顛,徒步翼以行。沈遽死,輿者欲散,聯以義感之,乃得至丹徒,殯於僧寺,以其柩歸。論者謂新唐書以張道源送友尸歸里,列諸忠義傳,聯亦其亞也。

黎屌,安南人,故安南國王黎維祁之族也。乾隆間,廣南阮光平破安南,屌護維祁叩關乞援,上遣孫士毅率師送歸國。既,復為光平襲破,維祁出走,屌齎上所賜國王印走,間道入關,與段旺等二十九人俱。上命薙發,分置江、浙諸地,獨屌與李秉道等四人不肯從。其一為黎駟,亦維祁族,其一失姓名,四人者堅請得出關為維祁復仇。上已受光平降,不欲更為黎氏出兵。謂屌等忠於黎氏,不以盛衰為去就,諭福康安平心詢問。士毅尋奏:「屌假託忠義,意圖構釁。」上命屌等從維祁至京師,令軍機大臣傳詢。屌等力請還黎氏故土,誓以死殉。上曰:「屌等仍還安南,或為光平所戮,朕心所不忍。」命暫系刑部獄。維祁卒,葬京師郊外。

仁宗即位,命釋四人者,使居外火器營。嘉慶八年,農耐阮福映並安南,使上表乞封,屌子光倬在行,屌與秉道至涿州迓焉。仁宗責其私出,下刑部。屌等初自承出謁維祁墓,既乃具言原得歸國,並以維祁喪還葬。上許之,賚以銀,並諸黎氏舊臣入漢軍置內地者悉遣還。

趙瓏,字雨亭,安徽桐城人。倜儻重然諾。有葉暘者,與有連,官大名同知,瓏往客焉。甫逾月,暘坐事戍伊犁,童僕皆散走,暘父母老且病,日夜泣,瓏請與俱行。既至,將軍愛暘才,置幕中,瓏乃辭歸。暘泣,瓏曰:「勿爾!吾且再來。」歸一年,暘母卒,瓏復往。比出關,聞暘從將軍移駐塔爾巴哈臺,改途赴之。將軍聞,賢瓏,稱曰「義士」,以此趙義士名著關外。

有葉椿者,暘同族也,亦戍伊犁。瓏再出關,椿母附寄子書致金。瓏既改赴塔爾巴哈臺,未至伊犁。歸道呼圖壁,遇巡檢陳栻,亦皖人也,因跡椿,則死久矣。瓏曰:「椿母日夜望子歸,乃今死,當奈何?且以金附我者,為我能致之也,義不忍空返其金,令椿骨不還。顧金少,盡吾橐中貲,猶不足,又當奈何?」貸於栻,迂道八千里,載椿柩以歸。

蔣堅,字非磷,江西鉛山人。幼即有智數。七歲,從叔入寺,廡坐縣役,值與語,謂某寺僧被殺,不得其主名。堅語其叔曰:「殺人者,堂上老僧也!」方誦經,屢顧,意乃不在經。役牽去,一訊而服。年十七,附舟經瑞洪,有少年同舟,當食必出避,堅疑而問之。少年自言貧不能償舟值,舟人將不餘食焉,故出避。堅邀與共食,資以金,其人後客死,又策返其骨及餘金。長習法家言,佐幕山西,屢雪疑獄。康熙五十二年,主澤州知州佟國瓏,臨汾民迫奸胥為變,巡撫檄國瓏往按,堅從國瓏以七騎往。至則眾保山洶洶,堅以巡撫令箭先諭眾。國瓏入縣,執胥擾民者五六,笞之流血,眾就觀,歡譟悉散。國瓏乞休,堅歸。數年,聞國瓏以屬吏虧帑逮下太原獄,責償數千金。堅往省,為國瓏徵債欒城,又至澤州,貸於州民,為國瓏輸償,獄乃解。堅嘗曰:「法所以救世,心求人之生,斯善用法矣。」著求生集。

子士銓,文苑有傳。

李林孫,河南襄城人。乾隆末,教匪起,將攻河南會城。是時布政使馬慧裕主城守,顧無兵,度無以御。有陳伯瑜者,郫縣人,嘗為河南巡撫客,先事言教匪且起,以妖言下獄。川、楚亂作,諸大吏禮為上客。友林孫,言於慧裕,使率鄉兵五百人助守。教匪至,伯瑜以二百五十人面水肄戰。匪易其少,就觀之,林孫以二百五十人出其背夾擊,大破之。知縣林嵐乞其兵守盧氏,教匪渠張潮兒來攻,號十萬,嵐兵不及二千,莫敢進。嵐謝其眾曰:「公等皆林孫人,徒死無益。」指大樹曰:「我官也,死是間耳。」眾怒曰:「誰無面目者,致公為此言?今日戰,有不勝賊而生者,撞大石破腦死!」嵐拜,眾亦拜。遂戰,賊幾殲。人或以兵家言問林孫,林孫謝不省,曰:「豪傑無他,得人心耳。」

高大鎬,湖南桃源人。父陛,臨淄知縣。嘉慶初,大鎬將僕王明省父歸,道荊門,遇教匪。大鎬從容語,使引見其渠。渠疑為官軍諜,欲殺之。大鎬自言:「我盜也!奈何殺我?」渠使與其徒角,殺三人,乃錄與其徒伍。渠令攻宜城,大鎬從行,渡溪,匿橋下得脫。遇餘寇,又殺三人,乃走宜城白吏,言寇且至,為畫城守策。大鎬在賊中久,知賊畏飛石,令盡發市衢街道民家階磩碎之,置城上。寇至,見有備乃走。吏欲敘大鎬功,大鎬辭歸桃源。王明在賊中,不與大鎬相聞,既為官兵所俘,讞非盜,釋之,亦得歸。

許所望,字叔翹,安徽懷遠人。諸生。工為詩。嘉慶七年冬,宿州民王朝明、李勝才為亂,州破。所望與其戚王冠英出粟三千石佐軍,且率其徒邱惠齡、張國綱、謝崇訓等破賊陳家集。十八年秋,林清亂起,師圍滑縣,兩江總督百齡駐徐州,安徽巡撫胡克家駐亳州,為備。歸德盜楊七郎據引河集,其黨洪廣漢據保安山,與潁州亂民沙占魁等遙相結,觀變。克家知所望,以書招之。所望率八百人至亳州,以惠齡等十人為隊長。所望謀曰:「楊七郎猛且狡,宜以計誘之。」令國綱、崇訓率健兒八人偽為逃卒詣七郎,越五日誘之出,以百餘人至邱家集。七郎忽疑曰:「若為許所望來耶?」崇訓出不意斷七郎臂,眾大驚,國綱疾呼曰:「我張國綱也!」立擊殺數人、國綱與惠齡同破宿州賊,以勇聞,賊素憚之,遂大潰。所望率兵至,七郎走死,廣漢亦潰。占魁等走永城,會師克滑縣,餘賊走與合,焚會亭。所望與戰公基湖,列十火槍土埠上,令眾伏地,曰:「賊至二百步,槍發,乘煙疾進擊之。」賊潰奔,逐之數十里,亳州師乃罷。百齡在徐州,亦得河南張永祥者,以鄉兵三百助守。事定,所望辭敘功,以諸生應試如故。永祥從巡撫阮元自河南移浙江,亦罷去,人呼為張鐵槍雲。

邢清源,曹州人。入鎮標為兵數十年,老而退伍。咸豐十一年,長槍會為亂,圍曹州。時親王僧格林沁駐軍濟寧,欲乞援,無敢齎書往者,清源請行。乃裂帛為牘,置清源衣帶,清源破衣持竹杖為丐者狀,出圍達王所。王即札示發兵狀,仍置衣帶還報,兵至,城得全。王元,杭州旗營牧馬人也。粵寇陷省城,將軍瑞昌守旗營,令元持書突圍出乞援張玉良,大哭不食。玉良義之,立進兵。瑞昌夾擊,遂復省城。明年,城再陷,元已保營官,戰歿長安,附祀瑞昌祠。

鳳瑞,字桐山,瓜爾佳氏,滿洲正白旗人,乍浦駐防。粵寇來犯,與兄麟瑞戰御。城陷,麟瑞陣歿,見忠義傳。鳳瑞改隸李鴻章軍,轉戰江、浙,屢有功,而太倉一役尤著。

初,李軍以乏餉不用命,鳳瑞力保盜魁賀國賢,國賢本鹽商,官誣殺其兄,乃為盜。鳳瑞與其兄善,責以大義,立出十萬金助餉,並率所部奮攻城,遂克太倉州。國賢後官至總兵,鳳瑞以筆帖式積功累保副都統,賞花翎。

江南平,調歸杭州,遂隱居不仕。時難民遍地,鳳瑞先於上海、青浦設廠施衣食,為謀棲宿,分遣歸里。復奉詔招集旗人歸防安插,恢復營制。建昭忠祠,立忠義墳。凡杭、乍兩營死者逾萬人,尸骨狼藉,躬督檢埋,分建兩大塚於兩地。勒碑致祭,列入祀典。又採訪姓名,匯刻浙江八旗殉難錄。

乍浦副都統錫齡阿全家同殉,其僕石某獨負其幼子出,乞食養之。鳳瑞見而言於巡撫薛煥,奏請撫恤,為賦義僕行,給貲送歸。

鳳瑞義俠,好行善,歲收租穀數百石,必盡散之窮乏,數十年如一日,眾稱善人。卒,年八十有二,贈將軍。

鳳瑞博學,工書畫,游跡遍天下,嘗自刊玉章,曰「讀萬卷書,行萬里路。」著有老子解、如如老人詩草及殉難錄等。

子四,文梁年十三,母病危,剖心以救,母愈,文梁竟卒。

方元衡,字莘田,安徽桐城人。以貢生官光祿寺署正。父病失明,晨夕調護,廁牏必躬親之,終親之身不稍怠。推產給弟,惟筆耕以奉甘旨。年五十,依母懷如嬰兒。居喪不宴笑,不居內,日所行必告於主,葬則廬墓側,歲時祭,必哀戚盡禮。俗惑於風水,常停柩久不葬,請設勸葬局,限期督葬,無後者則購地代葬之,先後逾五萬具。復設採訪局,採訪全省節孝貞烈,歷二十年,匯請得旌者凡十餘萬人。建總祠總坊於省會,有司春秋致祭。著有續心學宗、孝經淺言主。卒後,皖人上其孝義行,特贈五品卿。

葉成忠,字澄衷,浙江鎮海人。世為農。六歲而孤,母洪撫以長。為農家傭,苦主婦苛,去之上海,棹扁舟江上,就來舶鬻雜具。西人有遺革囊路側者,成忠守伺而還之,酬以金不受,乃為之延譽,多購其物,因漸有所蓄。西人制物以機器,凡雜具以銅鐵及他金類造者,設肆以鬻,謂之五金。成忠肆虹口,數年業大盛,乃分肆遍通商諸埠。就上海、漢口設廠,繅絲、造火柴,貲益豐。乃置祠田,興義塾,設醫局。會朝議重學校,成忠出貲四十萬建澄衷學堂,規制宏備,生徒景從。制字課圖說、修身、輿地諸書,諸校用之,以為善本。又建懷德堂,傭於所設肆者死,育其孤,恤其嫠,困乏者歲時存問,毋俾凍餒。鄉人為之諺曰:「依澄衷,不憂窮。」凡傭於葉氏,皆為盡力。成忠屢以出貲助賑,敘勞至候選道,加二品頂戴,卒。命諸子人擇一業,行義竟其志,勿邀賞。

楊斯盛,字錦春,江蘇川沙人。為圬者至上海,上海既通市,商於此者咸受廛焉。斯盛誠信為儕輩所重,三十後稍稍有所蓄,乃以廉值市荒土營室,不數年地貴,利倍蓰。善居積,擇人而任,各從所長,設肆以取贏,迭以助賑敘官。光緒二十八年,詔廢科舉,設學校,出貲建廣明小學、師範傳習所。越三年,又建浦東中、小學,青墩小學,凡糜金十八萬有奇。上海業土木者以萬計,眾議立公所,設義學,斯盛已病,力贊其成,事立舉。海濱潮溢,居民多死者,斯盛出三千金以賑,又集貲數萬,全活甚眾。浦東路政局科渡捐急,民大譁,官至,群毀其輿。斯盛力疾往,揮眾散,捐亦罷。又出貲規築洋涇、陸家渡、六里橋南諸路,改建嚴家橋,創設上海南市醫院,諸事畢舉。建宗祠,置義田,佽故友族人,咸有恩紀。及卒,遺命散所蓄助諸不給,遺子孫者僅十一。

武訓,山東堂邑人。乞者也,初無名,以其第曰武七。七孤貧,從母乞於市,得錢必市甘旨奉母。母既喪,稍長,且傭且乞。自恨不識字,誓積貲設義學,以所得錢寄富家權子母,積三十人,得田二百三十畝有奇,乞如故。藍縷蔽骭,晝乞而夜織。或勸其娶,七謝之。又數年,設義塾柳林莊,築塾費錢四千餘緡,盡出所積田以資塾。塾為二級,曰蒙學,曰經學。開塾日,七先拜塾師,次遍拜諸生,具盛饌饗師,七屏立門外,俟宴罷,啜其餘。曰:「我乞者,不敢與師抗禮也!」常往來塾中,值師晝寢,默跪榻前,師覺驚起;遇學生游戲,亦如之:師生相戒勉。於學有不謹者,七聞之,泣且勸。有司旌其勤,名之曰訓。嘗至館陶,僧了證設塾鴉莊,貲不足,出錢數百緡助其成。復積金千餘,建義塾臨清,皆以其姓名名焉。縣有嫠張陳氏,家貧,刲肉以奉姑,訓予田十畝助其養。遇孤寒,輒假以錢,終身不取,亦不以告人。光緒二十二年,歿臨清義塾廡下,年五十九。病革,聞諸生誦讀聲,猶張目而笑。縣人感其義,鐫像於石,歸田四十畝,以其從子奉祀。山東巡撫張曜、袁樹勛先後疏請旌,祀孝義祠。

呂聯珠,字星五,漢軍正黃旗人,隸盛京內務府。所居村曰瓦子峪。貧,授徒為大父及父母養,一介不妄取。應鄉試,徒步千餘里,有富家子招與同乘,堅卻之。光緒十四年,舉於鄉,授筆帖式,補催長,不改其狷。聯珠有從叔,其一貧,無子,請兼祧侍養。叔嚴急,事之盡禮;其一出遠游,以廢疾歸,奉於家,喪葬婚嫁力任之。有田招佃以耕,鄰田鬻於人,占聯珠田五尺,聯珠言於官,讓與之。田中有他氏墓,為之掃除歲祭焉。同學坐事系獄死,為之葬。姻家有以疑獄死京師者,赴會試,為攜其骨還葬。

聯珠篤行,式於鄉人。治程、硃之學,鄉人奉其教。久之,卒。


\end{pinyinscope}