\article{列傳二百九十九}

\begin{pinyinscope}
土司一

○湖廣

西南諸省,水衣復山重,草木蒙昧,雲霧晦冥,人生其間,叢叢虱虱,言語飲食,迥殊華風,曰苗、曰蠻,史冊屢紀,顧略有區別。無君長不相統屬之謂苗,各長其部割據一方之謂蠻。若粵之僮、之黎,黔、楚之瑤,四川之惈儸、之生番,雲南之野人,皆苗之類。若漢書:「南夷君長以十數,夜郎最大。其西,靡莫之屬以十數,滇最大。自滇以北,君長以十數,工⼙都最大。」在宋為羈縻州。在元為宣慰、宣撫、招討、安撫、長官等土司。湖廣之田、彭,四川之謝、向、冉,廣西之岑、韋,貴州之安、楊,雲南之刀、思,遠者自漢、唐,近亦自宋、元,各君其君,各子其子,根柢深固,族姻互結。假我爵祿,寵之名號,乃易為統攝,故奔走惟命,皆蠻之類。

明代播州、藺州、水西、麓川,皆勤大軍數十萬,殫天下力而後剷平之。故云、貴、川、廣恆視土司為治亂。

清初因明制,屬平西、定南諸籓鎮撫之。康熙三年,吳三桂督云、貴兵兩路討水西宣慰安坤之叛,平其地,設黔西、平遠、大定、威寧四府。三籓之亂,重啗土司兵為助。及叛籓戡定,餘威震於殊俗。

至雍正初,而有改土歸流之議。四年春,以鄂爾泰巡撫雲南兼總督事,奏言:「雲貴大患,無如苗蠻。欲安民必先制夷,欲制夷必改土歸流。而苗疆多與鄰省犬牙相錯,又必歸並事權,始可一勞永逸。即如東川、烏蒙、鎮雄,皆四川土府。東川與滇一嶺之隔,至滇省城四百餘里,而距四川成都千有八百里。去冬,烏蒙土府攻掠東川,滇兵擊退,而川省令箭方至。烏蒙至滇省城亦僅六百餘里。自康熙五十三年土官祿鼎乾不法,欽差、督、撫會審畢節,以流官交質始出,益無忌憚。其錢糧不過三百餘兩,而取於下者百倍。一年四小派,三年一大派。小派計錢,大派計兩。土司一取子婦,則土民三載不敢婚。土民有罪被殺,其親族尚出墊刀數十金,終身無見天日之期。東川已改流三十載,仍為土目盤踞,文武長寓省城,膏腴四百里,無人敢墾。若東川、烏蒙、鎮雄改隸雲南,俾臣得相機改流,可設三府一鎮,永靖邊氛。此事連四川者也。廣西土府州縣峒寨等司五十餘員,分隸南寧、太平、思

恩、慶遠四府,多狄青征儂智高、王守仁征田州時所留設。其邊患,除泗城土府外,餘皆土目,橫於土司。且黔、粵向以牂牁江為界,而粵之西隆州與黔之普安州逾江互相鬥入,苗寨寥闊,文武動輒推諉。應以江北歸黔,江南歸粵,增州設營,形格勢禁。此事連廣西者也。滇邊西南界以瀾滄江,江外為車里、緬甸、老撾諸土司。其江內之滇沅、威遠、元江、新平、普洱、茶山諸夷,巢穴深邃,出沒魯魁、哀牢間,無事近患腹心,有事遠通外國,自元迨明,代為邊害。論者謂江外宜土不宜流,江內宜流不宜土。此云南宜治之邊夷也。貴州土司向無鉗束群苗之責,苗患甚於土司。而苗疆四周幾三千餘里,千有三百餘寨,古州踞其中,群砦環其外。左有清江可北達楚,右有都江可南通粵,皆為頑苗蟠據,梗隔三省,遂成化外。如欲開江路以通黔、粵,非勒兵深入,遍加剿撫不可。此貴州宜治之邊夷也。臣思前明流土之分,原因煙瘴新疆,未習風土,故因地制宜,使之鄉導彈壓。今歷數百載,相沿以夷治夷,遂至以盜治盜,苗、惈無追贓抵命之憂,土司無革職削地之罰,直至事大上聞,行賄詳結,上司亦不深求,以為鎮靜邊民無所控訴;若不剷蔓塞源,縱兵刑財賦事事整飭,皆治標而非治本。其改流之法,計擒為上,兵剿次之。令其自首為上,勒獻次之。惟制夷必先練兵,練兵必先選將。誠能賞罰嚴明,將士用命,先治內,後攘外,必能所向奏效,實云貴邊防百世之利。」世宗知鄂爾泰才,必能辦寇,即詔以東川、烏蒙、鎮雄三土府改隸雲南。六年,復鑄三省總督印,令鄂爾泰兼制廣西。

於是自四年至九年,蠻悉改流,苗亦歸化,間有叛逆,旋即平定。其間如雍正朝古州苗疆之蕩平,乾隆朝四川大小金川之誅鋤,光緒朝西藏瞻對之征伐,皆事之鉅者,分見於篇。

其土官銜號,曰宣慰司,曰宣撫司,曰招討司,曰安撫司,曰長官司。以勞績之多寡,分尊卑之等差,而府、州、縣之名亦往往有之。

今土司之未改流者,四川宣撫使二:曰工⼙部,曰沙馬。宣慰司五:曰木坪,曰明正,曰巴底,曰巴旺,曰德爾格忒。安撫使二十有一:曰長寧,曰沃日,曰瓦寺,曰梭磨,曰瓜別,曰木里,曰革布什札,曰巴底,曰綽斯甲布,曰喇袞,曰瓦述餘科,曰霍耳竹窩,曰霍耳章谷,曰霍耳孔撒,曰霍耳咱,曰林蔥,曰霍耳甘孜麻書,曰霍耳東科,曰春科,曰下瞻對,曰上納奪。長官司二十有九:曰靜州,曰隴木,曰嶽希,曰松岡,曰卓克基,曰威龍州,曰陽地隘口,曰黨壩,曰河東,曰阿都正,曰普濟州,曰昌州,曰沈邊,曰冷邊,曰瓦述啯隴,曰瓦述毛丫,曰瓦述曲登,曰瓦述色他,曰瓦述更平,曰霍耳納林沖,曰霍耳白利,曰春科高日,曰上瞻對,曰蒙葛結,曰泥溪,曰平夷,曰蠻夷,曰沐川,曰九姓。

雲南宣慰使一:曰車里。宣撫使五:曰耿馬,曰隴川,曰乾厓,曰南甸,曰孟連。副宣撫使二:曰遮放,曰盞達。安撫使三:曰路江,曰芒市,曰猛卯。副長官司三:曰納樓,曰虧容甸,曰十二關,土府四:曰蒙化,曰景東,曰孟定,曰永寧。土州四:曰富州,曰灣甸,曰鎮康,曰北勝。

貴州長官司六十有二:曰中曹,曰白納,曰養龍,曰虎墜,曰程番,曰上馬,曰小程,曰盧番,曰方番,曰違番,曰羅番,曰臥龍,曰小龍,曰大龍,曰金石,曰大平,曰小平,曰大谷龍,曰小谷龍,曰木瓜,曰麻鄉,曰新添,曰平伐,曰羊場,曰慕役,曰頂營,曰沙營,曰楊義,曰都勻,曰邦水,曰思南,曰豐寧上,曰豐寧下,曰爛土,曰平定,曰樂平,曰工⼙水,曰偏橋,曰蠻夷,曰沿河,曰郎溪,曰都坪,曰黃道,曰都素,曰施溪,曰潭溪,曰新化,曰歐陽,曰亮寨,曰湖耳,曰中林,曰八舟,曰龍里,曰古州,曰洪州,曰省溪,曰提溪,曰烏羅,曰平頭,曰垂西,曰抵寨,曰巖門。副長官司三:曰西堡,曰康莊,曰石門。

廣西土州二十有六:曰忠州,曰歸德,曰果化,曰下雷,曰下石西,曰思陵,曰憑祥,曰江州,曰思州,曰萬承,曰太平,曰安平,曰龍英,曰都結,曰結安,曰上下凍,曰佶倫,曰茗州,曰茗盈,曰鎮遠,曰那地,曰南舟,曰田州,曰向武,曰都康,曰上映。土縣四:曰羅陽,曰上林,曰羅白,曰忻城。長官司三:曰遷隆峒,曰永定,曰永順。

凡宣慰、宣撫、安撫、長官等司之承襲隸兵部,土府、土州之承襲隸吏部。凡土司貢賦,或比年一貢,或三年一貢,各因其土產,穀米、牛馬、皮、布,皆折以銀,而會計於戶部。

雍正七年,川陜總督岳鍾琪奏四川巴塘、里塘等處請授宣撫司三員、安撫司九員、長官司十二員,給與印結號紙,副土官四員、千戶三員、百戶二十四員,給以職銜,以分職守。內巴塘、里塘正副土官原無世代頭目承襲,請照流官例。如有事故,開缺題補,與他土司不同。

湖廣之西南隅,戰國時巫郡、黔中地。湖北之施南、容美,湖南之永順、保靖、桑植,境地毗連,介於嶽、辰、常德、宜昌之間,與川東巴、夔相接壤,南通黔,西通蜀。元時所置宣慰、安撫、長官司之屬,明時因之。向推永、保諸宣慰,世席富強,兵亦果敢,每遇征伐,荷戈前驅,國家倚之為重。清有天下,僅施南、散毛、容美三宣撫使,永順、保靖兩宣慰使而已。雍正年間,施南、容美、永順、保靖先後納土,特設施南一府,隸北布政使,永順一府,隸南布政使。兩府既設,合境無土司名目。後有苗寇,分見各傳,不入此篇。

施南:古巴地。秦、漢南郡蠻。唐施州。元置施南宣撫司、忠孝安撫司。明玉珍時,復置忠路宣撫司。明宣德三年,復置劍南長官司,立施州衛,領所一、宣撫司四、安撫司九、長官司十三、蠻夷官司五。清康熙三年,施州始歸順。四年,改沙溪宣慰司為宣撫司,改劍南長官司為建南長官司,而施南宣撫司、忠孝安撫司、忠路安撫司如故。雍正六年,從湖廣總督邁柱之請,裁施州衛,設恩施縣,改歸州直隸州,原管之十五土司並隸恩施縣。十二年,忠孝安撫司田璋納土,其地入於恩施縣。十三年,施南宣撫司覃禹鼎以罪改流,於是忠峒土司田光祖等並請歸流,乃以十五土司並原設恩施縣,特設施南府,領六縣。容美改鶴峰州,別隸宜昌府,領於巡荊道。

明制,施州衛,轄三里、五所、三十一土司,市郭里、都亭里、崇寧里,附郭左、右、中三所,大田軍民千戶所,支羅鎮守百戶所。

大田所,元為散毛峒。明洪武五年定其地,二十三年屬千戶所,仍名散毛。尋改為大田軍民千戶所,領百戶所一、土官百戶所十、剌惹等三峒。

支羅所,舊隸龍潭司。明嘉靖四十四年,因峒長黃中叛,討平之,遂割半置所立屯,以百戶二員世鎮之,而今峒司屬焉。

施南宣撫司,元施南道宣慰使。明洪武四年,覃大富入朝,七年,升宣撫司。清因之。雍正時,覃禹鼎襲。禹鼎,容美土司田明如婿也,有罪輒匿容美。當事以明如之先從征紅苗有功,置勿問。十三年,明如被逮,自經死。禹鼎以淫惡抗提,擬罪改流,以其地置利川縣。

東鄉安撫司,明玉珍置東鄉五路宣撫司。明洪武六年改安撫司,命覃起喇為之。清初歸附。雍正十年,覃壽椿以長子得罪正法,改流,以其地入恩施縣。

忠建宣撫司,明洪武四年,以田恩俊為之。六年,改宣撫司。清初歸附。雍正十一年,田興爵以橫暴不法擬流,以其地為恩施縣。

金峒安撫司,明洪武四年,以覃耳毛為之。清初歸附。康熙四十三年,覃世英襲。子邦舜,呈請改流,以其地為咸豐縣。

忠峒安撫司,元置湖南鎮邊宣慰司。明洪武四年,命田璽玉為宣撫司。永樂四年,改安撫司。清初田楚珍歸附,調徵播州有功,仍準襲職。雍正十二年,田光祖糾十五土司呈請納土歸流,以其地入宣恩縣。

散毛宣撫司,元為散毛府。至正六年,改宣撫司。明洪武四年,命覃野旺為宣撫司,割其半為大田所。清初覃勛麟歸附,仍準襲職。雍正十三年,覃烜納土,以其地入來鳳縣。

忠路安撫司,明洪武四年,命覃英為安撫司。清康熙元年,覃承國歸附,以征譚逆功襲前職。雍正十三年,覃楚梓納土,以其地改利川縣。

忠孝安撫司,元至正十一年,改軍民府。明洪武四年,以田墨施為安撫司。清因之。康熙八年,田京襲,累授總兵。十九年,告休。雍正十三年,田璋納土,以其地為恩施縣。

高羅安撫司,元高羅寨長官司。明洪武六年,改安撫司,以田大名為之。清順治初,田飛龍歸附,仍準世襲。雍正十三年,田昭納土,以其地入宣恩縣。

木冊長官司,元置安撫司。明永樂六年,改長官司,以田谷佐為長官司。清初,田經國歸附,仍與世襲。雍正十三年,田應鼎納土,以其地入宣恩縣。

大旺安撫司,元至正置。明洪武四年,以田驢蹄為安撫司。清康熙初,田永封歸附,仍準襲職。雍正十三年,田正元納土,以其地入來鳳縣。

臨壁長官司,原附大旺。清康熙元年,頒給田琦印信,仍與世襲。雍正十三年,田封疆納土,以其地入來鳳縣。東流安撫司,原附大旺。

唐崖長官司,元置千戶所。明洪武七年,改長官司。清初覃宗禹歸附,仍與世襲。雍正十三年,覃梓桂納土,以其地入咸豐縣。

龍潭安撫司,明洪武四年,以田應虎為安撫司。清初歸附,仍準世襲。雍正十三年,田貴龍納土,以其地入咸豐縣。

沙溪安撫司,明置。清初歸附。康熙四年,黃天奇襲安撫司。天奇子楚昌。初,楚昌入施州衛學為諸生。時諸司爭並,民鮮知禮,楚昌折節力學,有時名。及襲職,設官學,公餘與多士講肄,多所成就。楚昌死,子正爵襲。雍正十三年,改流,其地入於利川縣。

卯峒長官司,清雍正十三年,長官司向舜納土,以其地入來鳳縣。

漫水宣撫司,清初,宣撫司向國泰歸附,仍準世襲。雍正十三年,向庭官納土,其地入於來鳳縣。

西萍長官司,雍正十三年裁,其地入於咸豐縣。

建南長官司,明宣德五年置。清雍正十三年裁,其地入於利川縣。

容美土司,唐元和元年,田行皋從高崇文討平劉闢,授施溱溶萬招討把截使,仍知四州事。宋有田思政。元有田乾亨。明洪武三年,田光寶以元所授誥敕詣行在請換,乃命光寶仍為宣慰使。傳至田既霖,清順治間歸附,仍授宣慰使。子甘霖襲。甘霖字特云,著合浦集。甘霖子舜年,字九峰,受吳逆偽承恩伯敕,後繳。奉檄從征有勞績,頗招名流習文史,刻有廿一史纂。日自課,某日讀某經、閱某史至某處,刻於書之空處,用小印志之。有白鹿堂集、容陽世述錄。子明如襲職。以放肆為趙申喬劾奏,奉旨原宥。雍正十一年,再為邁柱嚴參,明如移駐平山寨儗抗拒,為石梁長官司張彤砫催迫,明如自盡。改土歸流,改司為鶴峰州,隸宜昌府。

永順:漢武陵,隋辰州,唐溪州地。宋時為永順州。元時,彭萬潛自改為永順等處軍民安撫司。明洪武五年,改宣慰使。清順治四年,恭順王孔有德至辰州,宣慰使彭宏澎率三知州、六長官、三百八十峒苗蠻歸附。十四年,頒給宣慰使印,並設流官經歷一員。康熙十年,吳三桂叛踞辰龍關,授永順宣慰使彭廷椿偽印,廷椿繳之。奉旨賞其子宏海總兵銜,令率土兵協剿,有功,授宣慰司印。雍正六年,宣慰使彭肇槐納土,請歸江西祖籍,有旨嘉獎,授參將,並世襲拖沙喇哈番之職,賜銀一萬兩,聽其在江西祖籍立產安插,改永順司為府,附郭為永順縣,分永順白崖峒地為龍山縣。

南渭州土知州,屬永順司。元至元中,置安撫司。明洪武二年,以彭萬金為土知州。傳至彭應麟,清順治四年,歸附。雍正五年,彭宗國納土,以其地入永順縣。

施溶州土知州,在永順司東南。元會溪、施溶等處長官司。明洪武二年,改州,以田建霸為土知州。傳至田茂年,清順治四年,歸附。雍正五年,田永豐納土。

上溪州土知州,屬永順司。明洪武二年,以張義保為土知州。傳至張漢卿,清順治四年,歸附。雍正五年,張漢儒納土。

臘惹峒長官司,元屬思州,以向孛爍為總管。明洪武五年,改屬永順司,以田世貴為長官司。傳至田仕朝,清順治四年,歸附。雍正五年,田中和納土。

麥著黃峒長官司,元曰麥著土村,屬思州。明洪武五年,改屬永順司,以黃谷踵為長官司。傳至黃甲,清順治四年,歸附。雍正五年,黃正乾納土。

驢遲峒長官司,元屬思州。明洪武五年,改屬永順司,以向迪踵為長官司。傳至向光胄,清順治四年,歸附。雍正五年,向錫爵納土。

施溶溪長官司,元屬思州。明初,改屬永順司,以汪良為長官司。傳至汪世忠,清順治四年,歸附。雍正五年,汪文珂納土。

白巖峒長官司,元屬葛蠻安撫司。明初,改屬永順司,以張那律為長官司。傳至張四教,清順治四年,歸附。雍正五年,張宗略納土。

田家峒長官司,明洪武三年,以田勝祖為長官司。傳至田興祿,清順治四年,歸附。雍正五年,田藎臣納土。

保靖宣慰司,亦唐溪州地。宋曰保靜州。元為保靖州安撫司。明仍為安撫使。清順治四年,明宣慰司彭象乾之子彭朝柱歸附。象乾曾孫澤虹病廢,其妻彭氏用事。漢奸高倫、張為任二人結連其舍把長官彭澤蛟、彭祖裕等,相與樹黨,以劫殺為事。雍正元年,澤虹死,子御彬幼,澤蛟欲奪其職,為御彬所遏。迨御彬襲職,肆為淫兇,澤蛟與其弟澤合謀,互相劫殺。二年,御彬以追緝澤蛟為名,潛結容美土司田旻如、桑植土司向國棟,率土兵搶虜保靖民財。七年,御彬安置遼陽,以其地為保靖縣。

大喇司,在龍山縣,屬保靖司。明正德十五年,以土舍彭惠協理巡檢事。傳至彭御佶,雍正十三年,納土。

桑植宣慰司,本慈利縣地。元有上桑植、下桑植宣慰司。明置安撫司。清順治四年,宣慰司向鼎歸附,授原職。鼎子長庚調鎮古州八萬。長庚子向國棟殘虐,與容美、永順、茅岡各土司相仇殺,民不堪命。雍正四年,土經歷唐宗聖與國棟弟國柄等相率赴愬,總督傅敏入奏,乃繳追印篆,國棟安置河南,以其地為桑植縣。

上下峒長官司,明置宣撫司,復改為長官司,而分其地為二。清康熙二年,向九鸞、向日葵歸附。二十一年,給九鸞上峒長官司印,日葵下峒長官司印。雍正十三年,上峒司向玉衡、下峒司向良佐納土,以其地屬桑植縣。

茅岡長官司,明改天平千戶所。清順治四年,石門天平所千戶覃祚昌、茅岡長官覃廕祚等相繼歸附,給與印信。雍正十二年,茅岡土司覃純一納土,石門天平所、慈利麻寮所相繼請設流官,分其地屬石門、慈利、安福三縣。


\end{pinyinscope}