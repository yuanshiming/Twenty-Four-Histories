\article{列傳一百五十一}

\begin{pinyinscope}
阮元汪廷珍湯金釗

阮元,字伯元,江蘇儀徵人。祖玉堂,官湖南參將,從征苗,活降苗數千人,有陰德。

元,乾隆五十四年進士,選庶吉士,散館第一,授編修。逾年大考,高宗親擢第一,超擢少詹事。召對,上喜曰:「不意朕八旬外復得一人!」直南書房、懋勤殿,遷詹事。五十八年,督山東學政,任滿,調浙江。歷兵部、禮部、戶部侍郎。

嘉慶四年,署浙江巡撫,尋實授。海寇擾浙歷數年,安南夷艇最強,鳳尾、水澳、箬黃諸幫附之,沿海土匪勾結為患。元徵集群議為弭盜之策,造船砲,練陸師,杜接濟。五年春,令黃巖鎮總兵岳璽擊箬黃幫,滅之。夏,寇大至,元赴臺州督剿,請以定海鎮總兵李長庚總統三鎮水師,並調粵、閩兵會剿。六月,夷艇糾鳳尾、水澳等賊共百餘艘,屯松門山下。遣諜間水澳賊先退,會颶風大作,盜艇覆溺無算,餘眾登山,檄陸師搜捕,擒八百餘人。安南四總兵溺斃者三,黃巖知縣孫鳳鳴獲其一,曰倫貴利,磔之。九月,總兵岳璽、胡振聲會擊水澳幫,擒殲殆盡。土匪亦次第殲撫。浙洋漸清,而餘盜為蔡牽所並,閩師不能制,勢益熾,復時犯浙。李長庚巳擢提督,元集貲與造霆船成,配巨砲,數破牽於海上。八年,奏建昭忠祠,以歷年捕海盜傷亡將士從祀。盜首黃葵集舟數十,號新興幫,令總兵岳璽、張成等追剿,逾年乃平之。偕總督玉德奏請以李長庚總督兩省水師,數逐蔡牽幾獲,而玉德遇事仍掣肘。十年,元丁父憂去職,長庚益無助,復與總督阿林保不協,久無成功,遂戰歿。

十一年,詔起元署福建巡撫,以病辭。十二年,服闋,署戶部侍郎,赴河南按事。授兵部侍郎,復命為浙江巡撫,暫署河南巡撫。十三年,乃至浙,詔責其防海殄寇。秋,蔡牽、硃濆合犯定海,親駐寧波督三鎮擊走之,牽復遁閩洋。時用長庚部將王得祿、邱良功為兩省提督,協力剿賊,元議海戰分兵隔賊船之策,專攻蔡牽。十四年秋,合擊於漁山外洋,竟殄牽,詳得祿等傳。元兩治浙,多惠政,平寇功尤著雲。

方督師寧波時,奏請學政劉鳳誥代辦鄉試監臨,有聯號弊,為言官論劾,遣使鞫實,詔斥徇庇,褫職,予編修,在文穎館行走。累遷內閣學士。命赴山西、河南按事,遷工部侍郎,出為漕運總督。十九年,調江西巡撫。以捕治逆匪胡秉耀,加太子少保,賜花翎。二十一年,調河南,擢湖廣總督。修武昌江堤,建江陵範家堤、沔陽龍王廟石閘。

二十二年,調兩廣總督。先一年,英吉利貢使入京,未成禮而回,遂漸跋扈。元增建大黃、大虎山兩砲臺,分兵駐守。迭疏陳預防夷患,略曰:「英吉利恃強桀驁,性復貪利。宜鎮以威,不可盡以德綏。彼之船堅砲利,技長於水短於陸。定例外國貨船不許擅入內洋,儻違例禁,即宜隨機應變,量加懲創。各國知彼犯我禁,非我輕啟釁也。」詔勖以德威相濟,勿孟浪,勿葸懦。道光元年,兼署粵海關監督。洋船夾帶鴉片煙,劾褫行商頂帶。二年,英吉利護貨兵船泊伶丁外洋,與民鬥,互有傷斃,嚴飭交犯,英人揚言罷市歸國,即停其貿易。久之拆閱多,託言兵船已歸,俟復來如命。乃暫許貿易,與約船來不交犯乃停止。終元任,兵船不至。元在粵九年,兼署巡撫凡六次。

六年,調雲貴總督。滇鹽久敝,歲絀課十餘萬,元劾罷蠹吏,力杜漏私;鹽井衰旺不齊,調劑抵補,逾年課有溢銷,酌撥邊用。騰越邊外野人時入內地劫掠,而保山等處邊夷曰枲僳,以墾山射獵為生,可用,乃募枲僳三百戶屯種山地,以御野人,即以溢課充費,歲有擴充。野人畏威,漸有降附者。十二年,協辦大學士,仍留總督任。車裏土司刀繩武與叔太康爭鬥,脅官求助,檄鎮道擊走之,另擇承襲乃安。越南保樂州土官農文云內閧,嚴邊防勿使竄入,亦不越境生事,尋文云走死。詔嘉其鎮靜得大體。十五年,召拜體仁閣大學士,管理刑部,調兵部。十八年,以老病請致仕,許之,給半俸,瀕行,加太子太保。二十六年,鄉舉重逢,晉太傅,與鹿鳴宴。二十九年,卒,年八十有六,優詔賜恤,謚文達。入祀鄉賢祠、浙江名宦祠。

元博學淹通,早被知遇。敕編石渠寶笈,校勘石經。再入翰林,創編國史儒林、文苑傳,至為浙江巡撫,始手成之。集四庫未收書一百七十二種,撰提要進御,補中秘之闕。嘉慶四年,偕大學士硃珪典會試,一時樸學高才搜羅殆盡。道光十三年,由雲南入覲,特命典試,時稱異數。與大學士曹振鏞共事意不合,元歉然。以前次得人之盛不可復繼,歷官所至,振興文教。在浙江立詁經精舍,祀許慎、鄭康成,選高才肄業;在粵立學海堂亦如之,並延攬通儒:造士有家法,人才蔚起。撰十三經校勘記、經籍篡詁、皇清經解百八十餘種,專宗漢學,治經者奉為科律。集清代天文、律算諸家作疇人傳,以章絕學。重修浙江通志、廣東通志,編輯山左金石志、兩浙金石志、積古齋鐘鼎款識、兩浙輶軒錄、淮海英靈集,刊當代名宿著述數十家為文選樓叢書。自著曰揅經室集。他紀事、談藝諸編,並為世重。身歷乾、嘉文物鼎盛之時,主持風會數十年,海內學者奉為山斗焉。

汪廷珍,字瑟庵,江蘇山陽人。少孤,母程撫之成立。家中落,歲兇,饘粥或不給,不令人知。母曰:「吾非恥貧,恥言貧,疑有求於人也。」力學,困諸生十年,始舉於鄉。成乾隆五十四年一甲二名進士,授編修。大考,擢侍讀。未幾,遷祭酒。六十年,以事忤旨,降侍講。嘉慶元年,直上書房。大考,擢侍講學士。母憂歸,服闋,補原官。七年,督安徽學政。任滿,復督江西學政。累遷侍讀學士、太僕寺卿、內閣學士,皆留任。

廷珍學有根底,初為祭酒,以師道自居,選成均課士錄,教學者立言以義法,力戒摹擬剽竊之習。及官學政,為學約五則以訓士:曰辨塗,曰端本,曰敬業,曰裁偽,曰自立。與士語,諄諄如父兄之於子弟。所刻試牘,取易修辭之旨曰立誠編。士風為之一變。萬載棚民入籍,舊分學額,後裁之,土客訐訟久不決;廷珍請復分額,爭端乃息。十六年,授禮部侍郎。復直上書房,侍宣宗學。十八年,典浙江鄉試,留學政,任滿回京。二十二年,署翰林院掌院學士,擢左都御史,充上書房總師傅。二十三年,遷禮部尚書。二十四年,仁宗六旬萬壽,慶賀期內遇孝慈高皇后忌辰,部臣未援故事疏請服色,坐率忽,降侍郎。逾年,復授禮部尚書。

道光二年,典會試,教習庶吉士。車駕謁陵,命留京辦事。三年,宣宗釋奠文廟禮成,臨幸闢雍,詔曰:「禮部尚書汪廷珍蒙皇考簡用上書房師傅,與朕朝夕講論,非法不道,使朕通經義,辨邪正,受益良多。朕親政後,畀以尚書之任,盡心厥職,於師道、臣道可謂兼備。今值臨雍,眷懷舊學,加太子太保。子報原,以員外郎即補用,示崇儒重道之意。」四年,仁宗實錄成,賜子報閏主事,孫承佑舉人。南河高堰潰決阻運,上以廷珍生長淮、揚,命偕尚書文孚往勘,劾河督張文浩、總督孫玉庭,譴黜有差。疏籌修濬事宜,交河督辦理。五年,回京,協辦大學士。七年,卒,上震悼,優詔賜恤,贈太子太師,入祀賢良祠,命大阿哥賜奠,賜銀千兩治喪,謚文端。江蘇請祀鄉賢,特詔允之。

廷珍風裁嚴峻,立朝無所親附。出入內廷,寮寀見之,莫不肅然。自言生平力戒刻薄,凡貪冒諂諛有不忍為,皆守母教。大學士阮元服其多聞淵博,勸著書,廷珍曰:「六經之奧,昔人先我言之,便何以長語相溷?讀書所以析義,要歸於中有所主而已。」服用樸儉,或以公孫弘擬之,笑曰:「大丈夫不以曲學阿世為恥,而徒畏布被之譏乎?」後進以文謁,言不宗道,曰:「異日恐喪所守。」屬官有例送御史者,持不可,曰:「斯人華而不實,何以立朝?」後皆如所言,人服其精鑒。

湯金釗,字敦甫,浙江蕭山人。嘉慶四年進士,選庶吉士,授編修。十三年,入直上書房。金釗端謹自持,宣宗在潛邸,甚敬禮之。母憂服闋,擢侍講,督湖南學政。累遷內閣學士。二十一年,復直上書房。典江南鄉試,留學政,詔勉以訓士不患無才,務培德,經學為本,才藻次之。金釗闡揚詔旨,通誡士子。會匪以禍福煽惑鄉愚,金釗著福善辨,刊發曉諭。徐州俗悍,武生不馴者,繩之以法。遷禮部侍郎,任滿,仍直上書房。

宣宗即位,調吏部,益鄉用。時用尚書英和議,命各省查州縣陋規,明定限制。金釗疏言:「陋規皆出於民,地方官未敢公然苛索者,畏上知之治其罪也。今若明定章程,即為例所應得,勢必明目張膽,求多於例外,雖有嚴旨,不能禁矣。況名目碎雜,所在不同,檢察難得真確,轉滋紛擾。無論不當明定章程,亦不能妥立章程也。吏治貴在得人,得其人,雖取於民而民愛戴之,不害其為清;非其人,雖不取於民而民嫉仇之,何論其為清?有治人無治法,惟在督撫舉措公明,而非立法所能限制。」會中外大臣亦多言其不便,金釗疏入,上手批答曰:「朝有諍臣,使朕胸中黑白分明,無傷於政體,不勝欣悅!」予議敘。

道光元年,兼署戶部侍郎。兩江總督孫玉庭以南漕浮收不能盡去,議請八折徵收,學政姚文田、御史王家相皆奏言不可。金釗既同部臣議覆,復疏爭曰:「康熙中奉永不加賦之明詔,此大清億萬年培養國脈之至計也。前有議加耗米及公費銀者,戶部以事近加賦議駁。今準其略有浮收,不肖者益無顧忌,而浮收且多於往日,雖告以收逾八折即予嚴參,然前此逾額者何嘗不干嚴譴,卒不聞為之減少,獨於新定之額,恪遵而不敢逾,此臣之所不敢信也。在督撫奏定之後,不慮控告浮收;在州縣縱有發覺,又將巧脫其罪。是限制仍同虛設,徒為盛朝開加賦之端,臣竊惜之!」疏入,下江、浙督撫妥議,事乃寢。尋以吏部事繁,罷直上書房。典江南鄉試,道經銅山,見運河支渠為黃流淤塞,歲苦潦,回京奏請疏濬,如議行。二年,典會試,調戶部,父憂歸。六年,服闋,署禮、工二部及倉場侍郎,仍直上書房,授皇長子奕緯讀。實授戶部侍郎。七年,連擢左都御史、禮部尚書、上方倚畀,迭命赴山西、直隸、四川、湖北、福建鞫獄按事,四年之中,凡奉使五次。所至持法明慎,悉當上意。充上書房總師傅,調吏部尚書。十一年,皇長子遘疾不起,忌者因以激上怒,罷總師傅,降兵部侍郎。逾兩年,復自左都御史授工部尚書,轉吏部。連典江南、順天鄉試。十六年,陜西巡撫楊名颺被劾,命偕侍郎文慶往按,暫署巡撫;又往四川按事,名颺復與臬司互訐,得其冒工庇屬狀,劾罷。會京察,以奉使公明,予議敘。又赴張家口、太原鞫獄。十八年,以戶部尚書協辦大學士,仍調吏部。

十九年,命按事安徽、江蘇、浙江。自禁煙議起,海疆久不靖。林則徐既罷,琦善主撫,復不得要領。金釗素不附和議,與穆彰阿等意齟。一日召對,上從容問廣東事可付諸何人,金釗以林則徐對,上不悅。至二十一年,事且益棘,詔予則徐四品卿銜赴浙江軍營,亦未果用之。未幾,有吏部司員陳起詩規避倉差,金釗還其呈牘禁勿遞,為所訐,坐降四級調用。逾年,授光祿寺卿。以衰老乞罷,住京養痾,許以二品頂戴致仕。久之,上仍眷念,二十九年,皇太后之喪,具疏上慰,賜頭品頂戴。咸豐四年,重宴鹿鳴,加太子太保。六年,卒,詔以尚書例賜恤,謚文端。

金釗自為翰林,布衣脫粟,後常不改。當官廉察,負一時清望,雖被排擠,卒以恩禮終。子修,通政司副使。

論曰:阮元由詞臣出膺疆寄,竟殄海寇;開府粵、滇,綏邊之績,並有足稱;晚登宰輔,與樞臣曹振鏞異趣,惟以文學裁成後進,世推耆碩。汪廷珍、湯金釗正色立朝,清節並著;金釗雖以直言被擯,宣宗終鑒其忠誠,易名曰「端」,胥無愧焉。


\end{pinyinscope}