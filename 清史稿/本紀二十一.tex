\article{本紀二十一}

\begin{pinyinscope}
穆宗本紀一

穆宗繼天開運受中居正保大定功聖智誠孝信敏恭寬毅皇帝,諱載淳,文宗長子,母孝欽顯皇後那拉氏,咸豐六年三月二十三日,生於儲秀宮。

十一年,就學,編修李鴻藻授讀。七月,文宗不豫,壬寅,疾大漸,召御前大臣載垣、端華、景壽、肅順,軍機大臣穆廕、匡源、杜翰、焦佑瀛宣諭立為皇太子。命載垣、端華、景壽、肅順、穆廕、匡源、杜翰、焦佑瀛贊襄政務。

癸卯,文宗崩,召陳孚恩、文煜赴行在。甲辰,尊皇后及聖母並為皇太后。諭軍機處於各摺片後署贊襄政務王大臣。乙巳,免惇親王、恭親王、醇郡王、鍾郡王、孚郡王尋常召對及宴賚叩拜。停各省貢獻方物。

丙午,展順天文鄉試於九月舉行,恩科武會試於十月,順天武鄉試於十一月。授駱秉章四川總督,督辦軍務。召雲貴總督劉源灝來京,以福濟代之。以崇實為成都將軍,旋命協辦四川軍務。湖北官軍復武昌、咸寧、通城等縣及江西義寧州。戊申,以景紋為駐藏辦事大臣。己酉,允恭親王赴行在叩謁梓宮。庚戌,薛煥請招商試運淮鹽濟餉。議行。辛亥,粵匪陷吉安。廣西官軍復賓州。癸丑,加上宣宗帝後尊謚。甲寅,粵匪陷靖安、武寧、義寧各州縣。乙卯,定年號祺祥。

八月丁巳朔,日月合璧,五星聯珠。粵匪陷嚴州,旋復之。戊午,官軍復新昌、奉新、瑞州、上高。己未,命景廉赴葉爾羌查辦英蘊斂錢擅殺事。允曾國籓請,以上海現舶輪船駛往皖江,歸其軍練習。辛酉,湖北官軍復德安。壬戌,江西官軍復武寧、靖安。癸亥,頒大行皇帝遺詔。勝保軍復濮州。丁卯,捻匪渡運河,諭勝保與僧格林沁等截剿,毋任北竄。戊辰,胡林翼以疾乞假,命李續宜暫署湖北巡撫。庚午,御史董元醇請皇太后權理朝政,簡親王一二人輔弼。載垣等擬旨駁飭。甲戌,曾國荃軍復安慶。戊寅,廣西官軍復潯州。庚辰,四川番賊陷松潘。辛巳,論復安慶功,加官文、曾國籓太子少保,胡林翼太子太保,並予騎都尉世職,賞李續宜黃馬褂,楊載福、多隆阿雲騎尉世職。癸未,上大行皇帝尊謚曰協天翊運執中垂謨懋德振武聖孝淵恭謙仁寬敏顯皇帝,廟號文宗。苗沛霖陷正陽、霍丘,圍壽州。

九月丙戌朔,上母後皇太后徽號曰慈安,聖母皇太后徽號曰慈禧。辛卯,楊載福軍復池州。壬辰,捻匪竄汜水、鞏縣,官軍擊退之。召張亮基來京。金州地震。甲午,川軍剿平會理回匪。丁酉,允樂斌等奏,撒拉回匪降,撤回官軍。庚子,川軍復名山。壬寅,多隆阿、曾國荃等復桐城、宿松、蘄州、黃梅、廣濟。彭玉麟、成大吉等復黃州。湖北巡撫胡林翼卒,調李續宜為湖北巡撫,仍駐鄂、皖交界,督辦軍務。擢彭玉麟為安徽巡撫。癸卯,浙江官軍復於潛、昌化。粵匪竄嚴州,張玉良等軍潰。甲辰,英、法撤廣州駐兵,英撤駐天津馬隊。乙巳,僧格林沁剿平青州等處竄捻,賞還御前大臣並黃韁。戊申,上奉大行皇帝梓宮返京師,免承德及宛平各府縣田賦。己酉,苗沛霖反,命袁甲三會賈臻諸軍討之。甲寅,上奉母後皇太后、聖母皇太后還宮。乙卯,以擅改諭旨,力阻垂簾,解載垣、端華、肅順任,罷景壽、穆廕、匡源、杜翰、焦佑瀛軍機。命恭親王會同大學士、六部、九卿、翰、詹、科道按律覈奏。賈楨、周祖培、沈兆霖、趙光疏請政權操之自上,並議皇太后召見臣工禮節及辦事章程。勝保疏請皇太后親理大政,並簡親王輔政。命王大臣、大學士等定議以聞。召醇郡王奕枻來京。是日奪載垣、端華、肅順爵職,逮問議罪。命睿親王仁壽、醇郡王奕枻逮肅順解京。詔文武各衙門自十月十六日以後輪班值日。鮑超軍復鉛山。是月,免西寧碾伯被擾額賦。

冬十月丙辰朔,命恭親王奕為議政王,在軍機處行走,大學士桂良、戶部尚書沈兆霖、侍郎寶鋆、文祥並為軍機大臣,鴻臚寺少卿曹毓瑛在軍機大臣上學習行走。召盛京戶部侍郎倭仁來京。丁巳,諭求言,申嚴門禁。戊午,大行皇帝梓宮至京,奉安於乾清宮。庚申,詔改祺祥為同治。辛酉,恭親王等擬請載垣、端華、肅順照大逆律凌遲。詔賜載垣、端華自盡,肅順處斬,褫景壽、穆廕、匡源、杜翰、焦佑瀛職,穆廕遣戍軍臺。壬戌,褫陳孚恩、黃宗漢、劉昆、成琦、德克津太、富績職。諭不究既往,諸臣毋再請察辦黨援。申誡王公、內外文武大臣招權納賄。甲子,上御太和殿即皇帝位,受朝。頒詔天下,以明年為同治元年,加恩中外,罪非常赦所不原者,咸赦除之。免惇親王、恭親王、醇郡王、鍾郡王、孚郡王諭旨及奏疏稱名。乙丑,懿旨以物力維艱,誡內務府,宮闈器用,力行節儉。賞還僧格林沁博多勒噶臺親王。命刑部覈結五宇鈔票案。通諭中外清理庶獄。丙寅,苗沛霖陷壽州。東南方有聲如雷。諭熱河未竟工程即時停止。丁卯,申誡各路統將粉飾遷延,縱寇殃民。補行咸豐十年恩科武會試。己巳,命總兵馮子材督辦鎮江軍務。庚午,諭議政王等贊理庶務,毋避小嫌。壬申,諭統兵大臣實核功罪,信賞必罰。癸酉,粵匪陷嚴州、餘杭。命曾國籓統轄蘇、皖、贛、浙軍務,節制巡撫、提督以下各官;瑞昌幫辦浙江軍務,太常寺少卿左宗棠赴浙江剿賊,調遣提、鎮以下官。丙子,申諭郊配仍以三祖五宗為定,皇考祔廟稱宗。起用予告大學士祁俊藻、翁心存、前太常寺卿李棠階。籍陳孚恩家,下獄治罪。官軍復無為及隨州。丁丑,申誡廷臣遇事因循。諭官文、曾國籓等妥籌剿撫苗練。粵匪陷蕭山、紹興及江山、常山,趣左宗棠軍速援。己卯,釋貝子德勒克色楞於獄。辛巳,廷臣議上垂簾章程,懿旨依議。詔開恩科。初,烏拉停捕珠八年。至是,諭仍停辦。壬午,陳孚恩戍新疆。命侍郎寶鋆、董恂在總理各國事務衙門辦事。甲申,法兵去天津。

十一月乙酉朔,上奉慈安皇太后、慈禧皇太后御養心殿垂簾聽政。丙戌,諭各省習教交涉,分別良莠,持平辦理。丁亥,諭定戶部五宇鈔票侵款者罪。復熙麟等官。庚寅,命各軍保薦將才。壬辰,山東教匪作亂,成祿等剿平之,匪首延秀輪伏誅。甲午,先是張亮基言雲南副將何有保戕鄧爾恆,疑徐之銘主使。至是,之銘飾奏軍功,為有保請獎,諭福濟察辦,撤任嚴參。乙未,石達開竄綏寧。庚子,諭中外舉人才,以曾國籓、胡林翼、駱秉章為法。辛丑,粵匪陷紹興、諸暨,褫王履謙職逮問。壬寅,福濟以畏葸取巧褫職。賞潘鐸二品頂戴,署云貴總督。僧格林沁剿壽張等處會匪,大捷。癸卯,命彭玉麟幫辦袁甲三軍務。官軍復來安。乙巳,給事中高延祜劾徐之銘貪淫荒謬,及滇省練黨縱恣。諭潘鐸查辦。丁未,詔各省察舉循良,並訪學行該備之士。庚戌,以吳棠為江寧布政使,兼署漕運總督,督辦江北糧臺。癸丑,粵匪陷處州。

十二月甲寅朔,諭曾國籓通籌進剿機宜。乙卯,諭譚廷襄赴東昌籌河防。濮、範教匪平。丁巳,勝保奏收撫匪首劉占考、宋景詩。戊午,國瑞軍復範縣。粵匪陷寧波、鎮海暨紹興各屬。己未,諭整頓鹽務。辛酉,命左宗棠迅速援杭,張運蘭歸調遣,得專奏軍事。壬戌,命江寧副都統魁玉幫辦鎮江軍務。以毛鴻賓言,諭督撫及統兵大臣因地選將,毋專恃楚勇。袁甲三軍復定遠。允廓爾喀例貢改丁卯年呈進。乙丑,福建會匪陷福鼎,尋復之。河南捻匪竄棗陽。丁卯,曾國籓辭節制四省軍務,不允。己巳,上孝德皇后尊謚曰孝德溫惠誠順慈莊恭天贊聖顯皇后。兵部侍郎慶英有罪褫職,戍新疆。以青海札薩克貝勒綱僧卻多布為左翼盟長。辛未,褫毓科職,擢沈葆楨為江西巡撫。命恭親王、醇親王督瑞麟、文祥等管理神機營。曾國籓奏派道員李鴻章統水陸軍赴鎮江規復蘇、常,允之。定登萊青道駐煙臺,監督東海關稅務。壬申,降端華、載垣世爵為不入八分輔國公。甲戌,免安徽、江蘇、浙江被賊來年額賦。乙亥,允江忠義終制,田興恕兼署貴州巡撫,旋以韓超署任。命張亮基督辦雲南軍務,徐之銘免雲南巡撫,以亮基署之。丁丑,多隆阿軍進攻廬州。石達開竄沅江、黔陽,偪川境,諭駱秉章、田興恕合擊之。兩淮粵匪陷杭州,瑞昌、王有齡死之。褫閩浙總督慶端職,留任。以左宗棠為浙江巡撫。彭玉麟辭巡撫,請專辦賊,許之,以為水師提督。調李續宜為安徽巡撫,嚴樹森為湖北巡撫。以鄭元善為河南巡撫。戊寅,祁俊藻以大學士銜為禮部尚書。改彭玉麟以兵部侍郎候補。庚辰,捻匪圍潁州。勝保論劾嚴樹森,諭令「反躬自責,保全名節,副皇考委任之意」。以薛煥言,諭總理各國事務衙門與英、法籌商借兵剿賊。壬午,追封皇弟二阿哥為憫郡王。趣左宗棠進取浙江。命勝保率部赴潁州。癸未,僧格林沁擊竄匪於曹州河南岸,殄之。

同治元年壬戌春正月甲申朔,慈安皇太后、慈禧皇太后御慈寧宮,上率王大臣行禮。禦乾清宮受賀。自是每歲皆如之。命麟魁、曾國籓協辦大學士。乙酉,詔酌撤畝捐、釐捐,拊循從征將士家室,撫慰傷亡兵勇子孫。以江西肅清,賞鮑超黃馬褂。李世忠復六合,賞亦如之。丙戌,諭曾國籓、左宗棠保衢州進解徽州圍。命曾國籓選將保上海。調蔣益澧部赴左宗棠軍。庚寅,勝保移軍潁州,命副都統遮克敦布、道員王榕吉接辦防務。辛卯,川軍復丹棱,匪首藍潮鼎伏誅。官軍復平越。壬辰,李世忠軍復天長。癸巳,粵匪李秀成陷奉賢、南匯、川沙。命都興阿以艇師厄吳淞口。丙申,樂斌以縱匪殃民,解任訊辦。命麟魁署陜甘總督,與沈兆霖剿撫撒回。粵匪竄偪上海。薛煥言英、法各員協同防剿。上嘉之。丁酉,初,綿性請改徵回賦,景廉赴阿克蘇勘辦之。及是,景廉覆劾,綿性坐褫職,尋戍吉林。回子郡王愛瑪特解回庫車管束。申誡回疆各大臣勿再攤徵。命英蘊察禁私典阿克蘇各城回地。戊戌,粵匪犯鎮江,馮子材軍擊退之。捻匪竄沭陽。諭僧格林沁南北兼顧。官軍復莘縣。己亥,麟魁卒。李世忠軍克江浦、浦口。撤慶端任,命耆齡赴閩接辦援浙軍務。庚子,擢鮑超為浙江提督,馮子材為廣西提督。癸卯,命喬松年督辦沿江團練。丙午,前安徽巡撫翁同書以失壽州、定遠,褫職逮問,尋論斬。丁未,加鑄阿克蘇錢。戊申,文煜等上北塘防守事務,允行。英、法留兵駐大沽砲臺。雲南官軍復麗江,回匪竄昆明。庚戌,粵匪竄松江,官軍合外兵迎剿,大敗之。洋將美人華爾原隸中國籍,賞四品頂戴、花翎。壬子,命張亮基募軍赴滇。癸丑,諭遮克敦布等會剿河套捻匪。

二月甲寅朔,官軍復來鳳。乙卯,懿旨皇帝在弘德殿入學讀書,祁俊藻、翁心存授讀。丙辰,擢曾國荃江蘇布政使,並令辦理軍務,毋庸回避。丁巳,粵匪陷黃巖。官軍解鎮江、徽州圍。辛酉,西寧辦事大臣多慧、提督成瑞以飾言撒匪投誠,並褫職議罪,尋論斬。樂斌以庇護褫職,戍新疆。壬戌,命都興阿遣兵駐天長、六合,李世忠移軍江浦、浦口,和衷共濟。粵匪陷安義,旋復之。癸亥,捻匪圍杞縣。甲子,以倭仁所進古帝王事跡及古今臣工奏議,陳弘德殿講肄。乙丑,僧格林沁軍擊捻匪大捷,賊由杞縣竄通許,追剿之。戊辰,石達開竄酆都。允田興恕請解欽差大臣,率部赴川,歸駱秉章節制。命韓超籌貴州防剿事。己巳,薛煥言會英、法軍剿高橋賊壘,克之。美人白齊文原入華籍,賞四品頂戴、花翎。壬申,金陵粵匪渡江擾江浦等處。諭曾國籓、都興阿抽調師船截擊之。癸酉,多隆阿軍進攻廬州。丙子,以上海洙涇陷,褫提督曾秉忠職。上海官軍會英、法軍剿除蕭塘賊壘。命崇厚、成明督辦天津海防。丁丑,復鄭親王、怡親王世爵。諭李續宜安集皖北流亡。是月,免汀州等處被擾額賦。

三月癸未朔,捻匪竄太和。甲申,允英、法派師船往長江協同防剿。丙戌,粵匪竄上海,薛煥軍擊敗之。戊子,賊陷青田。允鄭元善請,以丁憂布政使張曜專辦剿匪。庚寅,自正月以來不雨,詔修省,求直言。左宗棠復遂安。宋景詩降眾叛於蘭儀。壬辰,粵匪犯廬、和及江浦。甲午,勝保軍進援潁州,大捷。丙申,鄭元善言招回宋景詩,令帶罪圖效,允之。戊戌,命李續宜、鄭元善幫辦勝保軍務。辛丑,前府尹莊琦齡應召,陳崇正學,疏通正途,限制津貼、抽釐,籌軍實等十二策。議行,惟停養廉、查陋規,以妨政體不許。詔各省舉孝廉方正,務求真儒。癸卯,命沈兆霖督軍赴西寧剿撒匪。乙巳,萬壽節,停受賀。丙午,趣曾國籓分軍援湖州。丁未,匯纂帝王政治及前史垂簾事跡書成,名治平寶鑒。己酉,命副都御史晏端書赴廣東督辦釐金,吳棠督辦江北團練。命薛煥以頭品頂戴充通商大臣。以李鴻章署江蘇巡撫。京口副都統海全剿賊失利,死之。壬子,免回疆新舊應進貢物。是月,上躬詣大高殿祈雨者三。

四月甲寅,諭統兵大臣慎重餉糈,汰除浮費。景其濬上歷代君鑒,上嘉納之。乙卯,允駱秉章奏留田興恕仍辦貴州軍務。丁巳,粵匪陷宜陽,尋復之。戊午,雨。鮑超軍復青陽。曾國荃軍復巢縣、含山、和州。己未,普承堯戍軍臺。曾國籓等言蘇紳請借英、法兵規復蘇、常,斷不可行。上韙其議。令李鴻章裁制華爾常勝軍。粵匪李世賢竄江西,沈葆楨赴廣信督辦防剿。比利時請換約,諭薛煥妥酌籌辦。庚申,上孝靜成皇后尊謚曰孝靜康慈懿昭端惠弼天撫聖成皇后。壬戌,命薛煥為全權大臣,辦理比國通商事務。癸亥,賊陷漢中,乙丑,川軍復青神,左宗棠解衢州、江山圍。丙寅,捻首張洛行北竄,諭僧格林沁等籌防。以閩軍失利,慶端諱報,切責之。戊辰,曾貞幹復繁昌,鮑超復石埭、太平、涇縣。上海軍會英、法軍平南翔賊壘,復嘉定,庚午,都興阿擊敗揚州竄匪。官軍復潁上。粵匪陷孝義、鎮安。豫軍復永寧。辛未,以葉爾羌阿奇木伯克郡王阿克拉伊都違例攤捐,擅殺回眾,奪郡王,治英蘊罪,壬申,西安副都統烏蘭都剿賊失利,諭官文、鄭元善分兵赴陜。丙子,臺灣會匪陷彰化。粵匪竄偪西安,趣官文、鄭元善飭兵會剿。丁丑,上慈安皇太后、慈禧皇太后徽號,頒詔覃恩有差。戊寅,多隆阿軍克廬州,匪首陳玉成遁至壽州境,苗沛霖誘擒之。命免沛霖罪。己卯,張運蘭軍復旌德。曾貞幹軍復南陵。撒回圍攻巴燕戎格,沈兆霖援剿之。上海官軍復青浦。庚辰,何桂清逮至京,命大學士會刑部審擬。是月,免安州等州縣被水逋賦。

五月壬午朔,官軍復寧波、鎮海。癸未,鄭元善移軍汝寧。粵匪陷陜西山陽。命多隆阿督辦陜西軍務。甲申,雨。命吳振棫趣山西協辦防剿。乙酉,命明誼速赴塔城與俄會勘地界,徐宗幹剿臺灣匪。丙戌,賜徐郙等一百九十三人進士及第出身有差。丁亥,以諸暨農民包立身練勇殺賊,諭左宗棠酌用之。李世忠軍截剿江南援賊,大捷。己丑,廣西官軍復太平,劉長佑赴潯州督剿。粵匪陷渭南。壬辰,戍王履謙新疆。粵匪圍溫州、瑞安,諭慶端等進援,並令左宗棠兼顧。粵匪犯潼關,諭沈兆霖檄馬德昭援陜。乙未,彭玉麟、曾國荃各軍復太平暨蕪湖城、金柱關、東梁山各隘,賞李成謀黃馬褂。官軍會英、法軍克南橋、柘林、奉賢各城。南橋攻克時,法提督卜羅德陣沒,上嘉悼之,賜祭,賞其家屬珍物。丙申,粵匪竄陜州。以銅仁、石阡苗、教各匪猖獗,諭毛鴻賓、韓超會剿。戊戌,命侍郎恆祺會崇厚辦理葡國通商事務。時英國擬調印度兵助剿,諭曾國籓等迅克金陵、蘇、常,以杜覬覦。己亥,粵匪陷興義,官軍復霍丘。庚子,前太常寺少卿李棠階疏請於師傅匡弼之餘,預杜左右近習之漸,並講御批通鑒輯覽及大學衍義,優詔答之。辛丑,官軍復臺州府仙居、黃巖等六縣。賊目吳建瀛等以南匯降。官軍復川沙。賊陷嘉定。免直隸積欠旗租。壬寅,官軍進攻雨花臺。甲辰,允曾國籓議,仍以安慶為省治,設長江水師提督,駐蕪湖。命恆祺為辦理葡國通商全權大臣。總理各國事務衙門言法使照會,田興恕虐害教民,命駱秉章、勞崇光查辦。乙巳,陳玉成解京師,詔於中途磔之。汝州練目李瞻謀叛,官軍剿滅之。丙午,李世忠軍渡江克龍潭等處賊壘,進攻九洑洲,諭曾國籓節制。諭明誼按條約地圖與俄剖析界務,錫霖襄辦北路分界事宜。丁未,官軍復陜西山陽。戊申,踞山陽賊竄鄖西。戍英蘊盛京。川匪陷太平,竄擾陜西定遠。張芾撫叛回於臨潼縣,被執,死之。辛亥,彭玉麟、曾國荃等軍克秣陵關諸隘,進偪金陵。粵匪陷湖州,在籍福建糧道趙景賢死之。

六月壬子朔,耆齡以援浙逗留,褫職,仍留任。乙卯,諭李續宜調度淮北剿捻事,並約束苗沛霖。丙辰,僧格林沁等軍克金樓賊壘。戊午,命六部、九卿再議何桂清罪。庚申,川匪陷西鄉。官軍復定遠。李鴻章督程學啟等軍剿粵匪,大敗之。西安、同州漢、回械斗,燒殺渭北村鎮。諭分別剿撫,但辨曲直,不論漢、回。壬戌,川軍復太平。癸亥,粵匪陷鄖西。甲子,何桂清諭斬。乙丑,直隸蝗。丙寅,粵匪由伊、洛南竄,命勝保督剿之。陜回撲西安及同州,趣雷正綰入關。戊辰,申誡統兵大臣欺飾濫保,督撫嚴禁州縣藉災請緩,仍復私徵。庚午,賊匪陷天柱。癸酉,大學士桂良卒,贈太傅。頒廓爾喀王獎勵敕書。甲戌,詔難民陷賊來歸者,概予免罪。申嚴失守城池律。定比利時通商條約。常清等言俄人稱哈薩克汗阿勒坦沙拉已屬俄。諭查實酌辦,令各臺吉別舉襲汗爵者。乙亥,嚴諭文煜等緝直隸馬賊。諭譚廷襄赴兗、沂督剿各匪及竄捻。丙子,官軍復青田。丁丑,允僧格林沁請,收撫苗沛霖。己卯,石達開竄綦江,官軍大敗之,遂竄珙、高等縣。庚辰,趣多隆阿援西安剿回匪,毋為撫議所誤,仍解散被脅良回。是月,免直隸、河南逋欠及雜糧。

秋七月壬午朔。甲申,安集延匪倭里罕入喀什噶爾卡滋擾,官軍剿敗之。浩罕亂,伯克邁里被殺。丁亥,命景紋調達木蒙古兵及夥爾等族番兵赴藏。己丑,以陜回慘殺漢民,促多隆阿等入關。尋諭責其遷延,令勝保分軍援陜。袁甲三以病免,命李續宜為欽差大臣,督辦軍務。庚寅,李鴻章軍克金山衛。辛卯,甘肅撒回降。安集延賊遁出卡。俄人稱哈薩克、布魯特為其國地,命常清察覈,總理各國事務衙門剖理,明緒會明誼勘西界事宜。壬辰,命倭仁協辦大學士。甲午,川匪陷洋縣。戊戌,川軍復長寧。命愛仁、王茂廕密察陜西吏治。擢知州秦聚奎大順廣道,會遮克敦布辦直、東防務。己亥,以縱兵劫掠,褫總兵田在田職。庚子,沈兆霖督剿撒回,還至平番,山水暴發,卒。粵匪竄南陽,命勝保入陜督辦軍務,節制各軍。命熙麟為陜甘總督。允馮子材請,簡汰鎮江軍。癸卯,毛鴻賓剿黔匪連捷,諭韓超規復失地,劉長佑解散瑤人,毛鴻賓會剿黔、桂各匪。甲辰,閩軍復宣平、松陽、瑞安。以慶端為福州將軍,耆齡為閩浙總督。乙巳,李續宜母喪,詔奪情署安徽巡撫。丙午,彗星見西北方。中、葡商約成。命僧格林沁統豫、魯軍務,節制督撫以下,與李續宜商辦安徽剿匪事宜。總理各國事務衙門請設同文館,習外國語言文字,允之。丁未,鮑超軍復寧國。官軍復景寧、雲和。鄂軍復鄖西。諭刑部清理庶獄。初,廣東恩平、陽春、新興等縣土、客互斗,九年未解。至是,諭勞崇光諭止之,豫籌善後。戊申,以星變詔求直言。庚戌,林福祥、米興朝以失守逃避處斬。諭都興阿實覈沿江釐稅。雲南回匪陷永昌、龍陵、騰越。是月,免江西義寧等州縣逋賦蘆課。

八月辛亥朔,以臺州民團克復郡縣,詔蠲同治元、二年錢糧。壬子,李鴻章軍克青浦。申諭督撫痛除捐輸、抽釐、逼勒諸弊。癸丑,準京官俸減成搭放現金。甲寅,回匪圍咸陽等城,諭勝保入潼關督剿。乙卯,褫哷徵呼圖克圖名號及黃韁。以藏事敉平,停調番兵及川餉。詔順直捕蝗。己未,徐之銘請阻張亮基帶兵入滇。諭責其為回人挾制,不允。辛酉,諭嚴防陜匪句結甘回。壬戌,諭勝保分兵赴山西,英桂籌晉省防務。癸亥,諭勝保剿渭北,多隆阿剿渭南回匪,兼顧鎮平。甲子,資遣林自清練眾回滇。乙丑,陜回圍朝邑。特普欽等言呼蘭墾民日眾,請設理事同知等職,議行。命傅振邦襄辦譚廷襄軍務。丙寅,諭各省清查流品。丁卯,李續宜給假治喪,以唐訓方暫代。命福濟會景紋辦理藏事。命僧格林沁節制淮北軍,剿撫苗、捻。辛未,陜回西竄同州,朝邑路通。逆酋洪容海詣鮑超軍降,率所部克廣德。壬申,北新涇圍解,滬防肅清。癸酉,甘回竄鳳翔。粵、捻合犯淅川,陷竹谿、竹山。甲戌,允王大臣請,停送奉移山陵,命議近支親王恭代典禮。鎮江設關徵洋稅。丙子,諭勝保檄馬德昭軍駐長武一帶,防回匪竄甘。擢雷正綰陜西提督。丁丑,臺灣軍解嘉義圍。官軍復處州及縉雲。命總兵黃開榜接統田在田軍。戊寅,允直隸增募馬勇緝馬賊。官軍復青谿。命耆齡專辦援浙軍務。己卯,山東軍剿捻匪大捷。勝保奏敗回匪於斜口,西安解圍,匪竄渭北。諭以自便責之。命雷正綰襄辦勝保軍務。復浙江餘姚,廣西陽朔。以粵匪竄閿鄉,促鄭元善軍赴河、洛。

閏八月辛巳朔,慶端軍復縉雲。甲申,多隆阿軍克荊紫關。乙酉,鄂軍復竹山、竹谿。黔軍復天柱、邛水。粵匪竄老河口。回匪圍涇陽,飭雷正綰軍進剿。西安解嚴。丁亥,法庫門回民互斗,玉明等解散之。趣文煜、譚廷襄捕直、東界馬賊。戊子,回匪復攻西安。滇匪由川竄專坪。粵匪由閿鄉竄永寧。允河南收長蘆鹽釐濟餉。己丑,洪容海降眾復叛,踞廣德。辛卯,多隆阿軍剿捻匪大捷,解商南圍。調駐南苑吉、黑馬隊赴山西。壬辰,諭韓超與提督江忠義商辦貴州軍務,堵截林自清擁眾入黔。命李棠階為軍機大臣。以德勒克多爾濟等增兵巡河防。甲午,諭各省裁革州縣浮費。命京控案件專責按察使訊鞫。乙未,詔薦舉人才。命薛煥、李鴻章辦理普國換約事宜。飭各省迅解京餉。丙申,命倭仁為大學士。諭多隆阿扼守武關。戊戌,多隆阿剿亳、潁西竄捻匪大捷,賞黃馬褂。粵匪復陷慈谿,官軍合英、法軍復之,華爾沒於陣。庚子,諭勞崇光等籌濟京倉米穀,江蘇等省新漕徵收本色解京。張亮基劾徐之銘、岑毓英跋扈。允法將勒伯勒東留防寧波。諭潘鐸安撫雲南漢、回。辛丑,允袁甲三回籍,命唐訓方赴臨淮接辦軍務,馬新貽暫統甲三軍。曾國籓請簡大臣會辦軍務,上不許,仍慰勉之,並傳旨存問疾疫將士。諭景綸等嚴緝吉林教匪。壬寅,命富明阿馳赴揚州襄辦都興阿軍務。癸卯,勝保請撫三原等處回匪,不許。甲辰,以劉長佑為兩廣總督。允田興恕暫留貴州剿匪。乙巳,石達開竄綦江等處,官軍剿擊敗之。回匪竄邠州、寶雞等處。丙午,河南捻匪李如英降。戊申,石達開竄仁懷。己酉,命官文為文華殿大學士,倭仁為文淵閣大學士。

九月辛亥,孝靜成皇后升祔太廟,頒詔覃恩有差。豫捻竄內鄉、新野。壬子,御史劉慶請以招流亡、墾地畝課州縣治績,從之。甲寅,允沈葆楨請,挑練額兵,酌籌津貼。乙卯,以文宗奉移山陵,蠲經過州縣額賦。諭文煜選良有司籌辦畿輔水利。丙辰,直隸妖人王守青等編造逆書,事發伏誅。丁巳,諭鄭元善、毛昶熙夾剿西南兩路捻匪。曾國籓言馭苗沛霖,宜赦其罪而不資其力,韙之。戊午,廣東土匪黃金籠、李植槐等倡亂,官軍討平之。趣多隆阿督所部入陜,其竄隨、棗之匪,令穆圖善軍剿之。己未,勝保請調苗沛霖入陜助剿,不許。川匪竄寧陜,官軍敗之於子午谷。庚申,石達開竄桐梓。癸亥,以閩、粵、魯省玩視軍餉,予疆臣嚴議,並嚴定欠解京餉處分。甲子,粵酋李秀成大舉援金陵。陳得才陷應城、孝感,官軍復之。安徽軍克湖溝賊巢。丙寅,僧格林沁軍克亳州捻巢。陜回圍鳳翔。庚午,馮子材克湯岡賊巢。靈州回亂。趣李續宜赴軍。壬申,回眾撲同、朝,諭勝保親往督剿,雷正綰督剿咸陽以北。癸酉,浙軍復壽昌。甲戌,以勒索回商,褫庫倫大臣色克通額職,戍新疆。革庫倫茶票陋規。李鴻章軍合英、法軍復嘉定。允荷蘭立約通商。乙亥,鄂軍復京山。粵匪竄黃陂、黃安。諭曾國籓等選武弁在上海、寧波習外國兵法,令閩、粵等省仿行。丙子,豫軍克龍井賊巢。召蘇廷魁、曾望顏、劉熙載、黃彭年、硃琦等來京,仍命各省舉行團練。丁丑,詔畿輔行堅壁清野法。諭曾國籓等豫選將弁演習外國船砲。己卯,享太廟。

冬十月庚辰朔,川軍克龍場,匪首李永和等伏誅,賞提督胡中和黃馬褂。辛巳,粵匪大股圍南翔等處滬軍。勝保赴潼關剿匪。癸未,湖南援軍會復修仁。命勞崇光赴黔察辦田興恕殺教民案。以張凱嵩接辦廣西軍務。丙戌,文宗顯皇帝、孝德顯皇后升祔奉先殿,上親詣行禮。戊子,命瑞常協辦大學士。己丑,命曹毓瑛為軍機大臣。庚寅,豫軍剿捻勝之,解臨潁圍。趣勝保赴同、朝剿匪。勝保仍請調苗沛霖赴陜,諭嚴斥之。官軍復奉化。徐之銘言招撫興義回匪。諭稱其為滇回所制。令潘鐸截回委員,毋俾之銘預黔事。辛卯,延安回匪作亂。英桂辦河曲、保德團防。命李鴻章選將統常勝軍,實授江蘇巡撫。甘回竄逼花馬池。癸巳,黔軍剿敗石達開,遵義圍解。石達開竄仁懷。乙未,諭奉天嚴緝盜匪。裁故洋將華爾所部兵勇。準俄兵船在上海助剿,毋入江。定嗣後外人領兵毋易服色例。德楞額軍潰於山東,詔褫職查辦。丙申,寧夏軍剿回失利。陜回竄清水。戊戌,命僧格林沁剿山東幅匪。己亥,江南軍擊退金柱關賊。庚子,譚廷襄罷。命丁憂按察使閻敬銘署山東巡撫,辦理軍務。癸卯,命穆騰阿襄辦勝保軍務。乙巳,諭刑部:「今年例停句決,何桂清統兵失律,僅予斬候,已屬法外之仁。茲已屆期,若因停句再緩,久稽顯戮,何以謝死事者暨億萬生靈,著即處決。自後如遇停句之年,情罪重大之犯,仍特奏聞取旨。」初,徐之銘委回人馬聯升署安義鎮,回匪因踞普安城。至是,事聞。諭之銘撤回馬聯升,迅查釀變情形具奏。

十一月己酉朔,日有食之。以沈宏富署貴州提督,接辦田興恕軍務。庚戌,擢長沙知府丁寶楨署山東按察使。壬子,鄭元善以廢弛,降道員。命張之萬署河南巡撫。諭毛昶熙裁所部兵勇。臺灣會匪陷斗六門。甲寅,褫黃彬職,撤其幫辦,命吳全美接統水師,歸曾國籓、都興阿節制。丙辰,翁心存卒,贈太保。曾國荃軍剿金陵援賊大捷,賞國荃及蕭孚泗黃馬褂。戊午,官軍合英、法軍復上虞、嵊、新昌。己未,彭蘊章卒。庚申,金陵粵匪竄擾高資,馮子材軍擊退之。壬戌,勝保坐驕恣欺罔,褫職逮問。諭直隸舉行保甲。諭瑞麟嚴緝熱河匪徒。癸亥,秦聚奎剿匪冠縣沒於陣。九洑洲賊復陷和州、含山、巢縣。乙丑,宣示勝保罪狀,籍其貲,賞所部兵勇。授多隆阿欽差大臣,接統勝保所部各軍。丙寅,川匪陷佛坪,官軍復之。川匪復陷略陽。己巳,粵匪竄陷祁門。平羅回匪亂。辛未,閻敬銘請終制,不允。乙亥,山東降眾叛,陷濮州。命張亮基以總督銜署貴州巡撫,兼署提督,撤署巡撫韓超、署提督田興恕任,候查辦。丙子,石達開陷筠連。川匪陷兩當,旋復之。丁丑,法使以教士被戕,責田興恕抵償,不許。

十二月戊寅朔,諭江、浙等處被賊脅從,誠心歸順者,無論從賊久暫,均許投誠。諭曾國籓、唐訓方分軍駐正陽關、壽州。庚辰,白齊文有罪褫頂帶,逮治之。辛巳,多隆阿破回匪於同州。壬午,命荊州副都統薩薩布赴直、魯剿賊。癸未,江南軍復績溪、祁門。鮑超丁母憂,命改為署職,仍留營。官軍復濮州。乙酉,左宗棠軍復嚴州。丙戌,命雷正綰幫辦多隆阿軍務,將軍穆騰阿會瑛棨辦理省城防守事宜。丁亥,諭左宗棠等保舉湘籍人才。廣西匪陷西寧。戊子,回匪陷涇陽。宋景詩叛於山西。調阿拉善、鄂爾多斯蒙部兵助剿寧夏平羅回匪。申諭舉孝廉方正。粵匪竄平利。河州回匪肆擾,恩麟剿之。允普魯士換約。滇匪陷景東。改令席寶田軍援江西。諭江忠義節制援桂各軍。山東竄匪擾冀州、棗強,諭文煜等合剿。甲午,廣東舉人桂文燦進經學叢書,詔嘉勉。丙申,官軍復新寧,復霍丘。石達開再陷高縣,旋復之。丁酉,命侍郎崇厚幫辦直隸防剿。召劉長佑來京,命晏端書、昆壽商辦廣東軍務。戊戌,粵匪由鄖陽竄興安,諭多隆阿等會剿。庚子,賊目駱國忠等以常熟、昭文降。壬寅,諭穆騰阿、瑛棨辦理西安防剿,多隆阿兼顧省防。甘匪竄陷隴州,知州邵輔死之。癸卯,召薛煥來京,以李鴻章暫署通商大臣。甲辰,賊匪竄永年、邯鄲等處,以遷延貽誤褫文煜、遮克敦布職,並遣戍。以劉長佑為直隸總督,晏端書署兩廣總督。諭提督寶山接辦直、東交界事務。乙巳,祫祭太廟。丙午,粵匪復竄寧陜。丁未,粵匪圍興安,分竄漢中。是月,免四川榮昌等縣、福建甌寧等縣被擾額賦,江南湖灘積欠地租。

是歲,朝鮮、琉球入貢。

二年癸亥春正月戊申朔,免朝賀。授張之萬河南巡撫。辛亥,予紹興傷亡洋將勒伯勒東優恤。甲寅,詔曾國籓、都興阿等舉堪勝水師總兵者。匪陷武邑,官軍旋復之。廣西軍復蓮塘。戊午,粵匪陷興安府、鎮兩城。陜西回匪竄鄠縣,從瑛棨請,留馬德昭辦省防。丙寅,鮑超等軍復青陽。戊辰,命李桓赴陜,接辦漢南軍務。庚午,瞻對酋糾德爾格忒土司擾巴塘、里塘。辛未,畿南竄匪平。甲戌,以鳳翔困守半年,詔責瑛棨貽誤,趣雷正綰馳救解圍。

二月丁丑朔,左宗棠軍復金華、湯溪、龍游、蘭谿。戊寅,以李鴻章言,諭兩湖用漕折購米運京,免其稅。庚辰,李秀成等渡江北犯,官軍擊敗之。川軍剿石達開,破之。貴州回匪陷安南、興義。辛巳,吉林軍敗朝陽流匪於興凱湖,諭毋令竄入俄界。多隆阿剿回匪大捷,克羌白鎮等賊巢。壬午,陜西團勇復興安。粵匪竄漢陰、紫陽。李世忠請褫職贖勝保罪,不許。粵匪竄陷褒城,旋復之。癸未,復永康、武義。乙酉,譚廷襄赴東昌剿匪。丁亥,左宗棠移軍蘭谿。東陽、義烏、浦江踞賊均遁。己丑,僧格林沁軍克雉河集賊巢,捻首張洛行伏誅。得旨嘉獎,仍以親王世襲罔替。免蒙、亳等屬錢漕二年。庚寅,寧夏平羅回匪投誠。辛卯,以慶昀為寧夏將軍。癸巳,畿南匪張錫珠竄大名,以崇厚失機切責之,趣劉長佑赴直隸。馮子材敗賊於鎮江。乙未,左宗棠軍復紹興、桐廬。丙申,滿慶等剿辦瞻對逆匪。黃國瑞軍克郯城縣長城匪巢。以追賊遲延,褫崇厚職,留任。東匪竄曲周、平鄉。庚子,諭恩麟等,甘肅回匪毋輕議撫。壬寅,允平瑞請,墾烏魯木齊等處閒荒馬廠,升科濟餉,以屯田之地,分給屯兵。癸卯,粵匪陷江浦。廣東匪踞信宜,昆壽剿之。甲辰,浙東肅清,蠲新復各府州縣錢漕二年。乙巳,趣閻敬銘赴東昌辦理軍務。回匪馬化龍糾黨圍靈州,旋赴固原投誠。石達開由滇竄敘永。丙午,詔疆臣慎選牧令,薄賦輕徭,刪除煩苛,與民更始。是月,免青神兵擾二年逋賦。

三月戊申,申禁河南豫徵錢糧。辛亥,命崇厚回三口通商大臣任。壬子,命劉長佑節制直隸諸軍。諭沈葆楨辦交涉當持平,毋令紳民生釁。癸丑,諭曾國籓統籌江北軍務。乙卯,陜南粵匪陷紫陽,旋復之。雲南迤西逆匪犯昆明,潘鐸死之。以賈洪詔為雲南巡撫。丙辰,李鴻章軍克福山口。命英將戈登約束常勝軍。丁巳,捻匪陷麻城,戊午,偪武昌省垣。飭楚、豫合軍攻剿。己未,蠲浙江西安錢糧二年。庚申,丹國遣使拉斯那弗議立商約。洋將達耳第福陣亡,優恤。回匪圍平涼。以甘肅剿賊遷延,褫署提督定安職,逮問。甲子,耆齡遷福州將軍。以左宗棠為閩浙總督,節制兩省軍務。以曾國荃為浙江巡撫,仍統兵規金陵,宗棠兼署之。停福建本年例貢。乙丑,命王大臣覆覈勝保情罪。寧國粵匪竄東流、建德。予秦儒毛亨、明儒呂枬從祀文廟。丙寅,蒙城捻首賈文彬伏誅。陜南粵匪陷沔縣。貴州總兵羅孝連軍復定番、長寨、獨山、荔波。丁卯,曾國籓以失守江浦等城金雋級,褫李世忠幫辦。實授吳棠漕運總督,仍節制江北軍務。諭拊循江北難民。己巳,萬壽節,停受賀。庚午,苗沛霖復叛。官文等截剿蘄州竄賊。癸酉,褫徐之銘職,逮問。予潘鐸世職。以雨澤稀少,詔清理庶獄。甲戌,命福濟、景紋查辦西藏啟釁事。乙亥,李鴻章軍復太倉。隆德回匪亂。黃國瑞軍平沂州棍匪。丙子,詔察恤陜、甘殉難被害良善回眾,尋詔雲南亦如之。是月,上連詣大高殿祈雨。

夏四月戊寅,御史吳臺壽以疏奏袒勝保,褫職。苗沛霖陷懷遠。山東匪劉得培踞淄川。己卯,官軍剿畿南匪,張錫珠等竄高唐,尋伏誅。庚辰,粵、捻各匪竄擾廬江、桐、舒及黃州。諭曾國籓駐守安慶,勿撤金陵之圍。壬午,多隆阿軍克孝義匪巢。飭劉蓉統軍援陜。免浙江被陷各地額糧。甲申,苗沛霖圍壽州、六安,趣僧格林沁討之。粵匪踞太平、石埭,左宗棠、沈葆楨會防。多隆阿軍克倉頭匪巢,陜東肅清。苗沛霖陷潁上,犯蒙城。命劉長佑督辦直、魯、豫交界剿匪事務。乙酉,劉典軍復黟縣。命侍郎薛煥在總理各國事務衙門辦事。戊子,允英桂回駐太原。庚寅,劉長佑言匪首楊明嶺等投誠。甘肅回匪陷鹽茶,犯靜寧,馬德昭赴慶陽進剿。壬辰,贛軍敗賊祁門,逆酋胡鼎文伏誅。癸巳,李續宜請開署缺,允之。以唐訓方為安徽巡撫。李鴻章遣程學啟等軍薄昆山。涇州軍擊回匪,勝之。甲午,禮部議定先賢、先儒祀典位次,頒行各省。乙未,開墾直隸新城一帶稻田。閻敬銘赴淄川督剿。捻匪回竄河南,總兵餘際昌等死之,命張曜接統其軍。丁酉,以皖匪紛竄江、鄂,安慶可虞,詔曾國籓搘拄艱難,倍加謹慎。左宗棠軍復黟縣。以勞崇光為雲貴總督。逮治田興恕以謝法人。庚子,粵、捻各匪犯鳳臺、定遠,官軍擊退之。辛丑,賜翁曾源等二百人進士及第出身有差。停四川畝捐。癸卯,程學啟等軍復昆山、新陽。官軍敗賊酋李秀成於石澗埠。乙巳,回匪復犯西安,擊退之。是月,連祈雨。免太倉等州縣額賦。

五月戊申,苗沛霖圍蒙城。己酉,鮑超軍復巢縣。庚戌,賞郎中李云麟京卿,節制漢南防兵及川省援兵。壬子,粵、捻合犯天長,官軍擊敗之。甲寅,命江忠義統軍援江西。丁巳,鄒縣教匪平,獲匪首劉雙印。粵匪陷古州。戊午,俄兵入科布多境,執臺吉。壬戌,雨。癸亥,粵匪擾富陽,官軍擊退之,總兵熊建益等陣沒。官軍援平涼失利,趣多隆阿分軍速援。乙丑,寧夏撫回再叛。鮑超軍復克巢、和、含山。召晏端書來京,以毛鴻賓為兩廣總督,惲世臨為湖南巡撫。予明臣方孝孺從祀文廟。戊辰,諭購置輪船歸曾國籓、李鴻章節制。己巳,曾國籓為弟國荃辭浙江巡撫,上褒勉,不允所辭。西寧回句結撒匪攻丹噶爾。惠遠回匪亂,官軍捕誅之。定丹國通商條約。壬申,彭玉麟等軍復江浦、浦口及九洑洲。乙亥,廣西軍復潯州。

六月丙子朔,黔軍復普安、安南。丁丑,命明誼赴塔城會明緒等辦分界事。戊寅,詔曾國籓、左宗棠等議減江蘇常、鎮,浙江杭、嘉、湖屬漕糧。庚辰,以復城池功,賞李朝斌等及宋國永等黃馬褂。停陜西例貢。丁亥,川軍剿賊於大渡河,獲石達開,誅之。晉駱秉章太子太保銜,擢總兵唐友耕提督。辛卯,平羅回眾復叛。瓦亭回匪圍隆德,擊退之。河決開州、考城、菏澤。甲午,苗沛霖陷壽州,知州毛維翼死之。乙未,陜軍復寧羌。己亥,以俄人強占住牧,趣常清等定界,勸俄兵撤回,撫綏求內附之哈薩克、布魯特。壬寅,官軍復淄川,獲劉得培等誅之。甲辰,寶慶土匪平。命四川布政使劉蓉督辦漢南軍務。是月,免福建順昌等縣屬被擾額賦,江西義寧等州縣屬逋賦雜課。

秋七月乙巳,苗沛霖偪臨淮,唐訓方擊之。丙午,李鴻章軍復吳江、震澤。豫軍克張岡匪巢。瑛棨有罪,褫職。命劉蓉為陜西巡撫,張集馨署之。甲寅,命李鴻章暫兼南洋通商大臣。戊午,黔軍復古州。辛酉,袁甲三卒於軍。壬戌,賜勝保自盡。甲子,官軍克沙窩等處匪巢。允江北漕米仍徵折色。乙丑,命劉蓉並節制湖北援軍。丁卯,官軍擊退狼山苗眾,蒙城路通。命崇厚為全權大臣,辦理荷蘭通商條約。滇回陷平彞,岑毓英軍復之。癸酉,命明誼等會同俄使辦分界諸務。山東白蓮池教匪平。文煜予釋。捻匪逼開封。是月,免都勻等府州縣屬被擾新舊額賦,並鳳凰等縣灘地積欠租銀。

八月丙子,程學啟等軍大破賊於太湖、楓涇等處,進偪蘇州。丁丑,陜西曹克忠軍克附省等處賊巢。戊寅,西寧、狄道、河州漢、回互斗。哈薩克句結俄兵擾伊犁。趣四川何勝必軍援甘。庚辰,皖軍克長淮衛。辛巳,以畏葸褫馬德昭職。多隆阿軍抵西安,渭南肅清。命陳國瑞幫辦吳棠軍務。丙戌,蘇軍克江陰。丁亥,戍瑛棨新疆。都興阿遣軍援臨淮。己丑,以剿辦臺灣賊匪調度乖方,褫吳鴻源職,逮問。辛卯,李鴻章赴江陰督剿。諭陳國瑞援蒙城。調善慶部馬隊援臨淮。熙麟遣軍援平涼。乙未,允多隆阿請,以曹克忠補河州總兵,並令嗣後提鎮缺勿擅請簡。宋景詩竄開州。命張集馨會穆騰阿籌辦西安防守。丁酉,黔軍克桐梓賊巢。普安陷,旋復之。命劉蓉節制毛震壽、李云麟各軍。調烏魯木齊、阿克蘇兵助伊犁軍御俄。允哈薩克綽坦承襲汗爵。己亥,趣林文察渡臺剿匪。庚子,回匪陷平涼。辛丑,閻敬銘移軍東昌。定荷蘭換約。劉長佑赴景州督剿。是月,免沁州等州縣屬逋賦。

九月乙巳朔,命馬德昭赴慶陽營。沈葆楨乞病,慰留給假。戊申,允李鴻章調知縣丁日昌來滬督制火器。石泉知縣陸堃聯團剿賊,詔嘉之。庚戌,浙軍克富陽。辛亥,粵軍克廣海寨城。癸丑,諭僧格林沁以砲隊赴蒙城助剿。甲寅,粵匪陷城固。捻首張總愚等由汝州南竄。乙卯,多隆阿軍復高陵。丙辰,穆隆阿以覆奏失實褫職。調多隆阿為西安將軍。以富明阿為荊州將軍。辛酉,多隆阿軍克蘇家溝、渭城賊巢。甲子,粵匪陷會同、綏寧,旋復之。陜西兵團復沔縣。乙丑,李秀成援無錫,程學啟等擊退之。己巳,僧格林沁剿宋景詩股匪悉平。景詩遁。以援陜川軍敗,褫提督蕭慶高職,留營。以漢中失事,褫布政使毛震壽職。諭劉長佑、閻敬銘辦直、魯善後。庚午,御史馬元瑞條陳薄賦稅、慎訟獄、善拊循、勤曉諭四事,如所請行。是月,免直隸滄州等州縣,山東海豐等場未完灶課。

冬十月乙亥,閻敬銘請終制,不許。官軍獲直、東股匪硃登峰等,悉誅之。丙子,捻首張總愚由魯山、南召南竄。己卯,陶茂林軍解鳳翔圍,實授茂林甘肅提督。命丁憂總兵成祿留營。撤退李泰國,以赫德辦理總稅務司。辛巳,粵匪竄龍勝,總兵胡元昌死之。甲申,諭駱秉章分軍剿瞻對,疏通藏路。諭阻法教士入藏傳教。丁亥,朝陽餘匪竄擾昌圖。詔臣工力求節儉。趣賈洪詔赴昭通。以捐備馬匹賞扎薩克臺吉明珠爾多爾濟貝子銜。戊子,李云麟軍失利,粵匪陷陜西山陽。張總愚竄鄧州。賴、曹諸酋竄鳳縣、兩當。庚寅,左宗棠軍擊敗杭州、餘杭踞賊。壬辰,藍逆陷盩厔。癸巳,上釋服逾期,祁俊藻、倭仁、李鴻藻請黜浮靡以固聖德。懿旨:「屏斥玩好游觀興作諸務,祁俊藻等其各朝夕納誨,養成令德,以端治本而懋躬行。」逆酋古隆賢就撫,收復石埭、太平、旌德。曾國荃等軍復秣陵關。丙申,桂軍復容縣。丁酉,程學啟等軍攻克滸墅關。己亥,官軍剿昌圖匪失機,諭責玉明諱飾。辛丑,英桂遷福州將軍,以沈桂芬署山西巡撫。癸卯,李秀成援蘇州,李鶴章等軍擊敗之。命富明阿幫辦僧格林沁軍務。是月,免廣西永安等州縣被擾新舊額賦。

十一月丙午,奉天匪竄吉林,玉明等會剿。皖軍復懷遠及蚌埠。丁未,僧格林沁督諸軍攻剿苗沛霖,誅之。李鴻章督軍復蘇州,粵酋郜雲官等降。加鴻章太子少保銜,程學啟世職,並賞黃馬褂。戊申,逆酋楊友清等以高淳、寧國、建平、溧水降。李云麟等復山陽。粵軍復信宜。己酉,劉典等軍復昌化。庚戌,藍逆竄商南。癸丑,張總愚竄淅川。甲寅,僧格林沁軍復下蔡、壽州。丙辰,李鴻章誅郜雲官等,遣散降眾。丁巳,李鶴章軍克無錫、金匱。庚申,李續宜卒。丘縣匪張本功等糾眾抗糧,捕誅之。實授閻敬銘山東巡撫。汧陽回眾降。壬戌,官軍復潁上、正陽。癸亥,馬化龍陷寧夏、靈州。論平苗逆功,復李世忠職。曾國荃軍克淳化等隘,進駐孝陵。丙寅,官軍克嘉善張涇匯。丁卯,逆回圍寧夏滿城。庚午,蘇軍復平湖。賊目以乍浦、嘉善降。是月,免山東泗水等州縣災擾錢糧,直隸武清等州縣被災額賦。賑吉林打牲烏拉災。

十二月丁丑,提督江忠義卒於江西軍次。庚辰,蘇軍克平望。辛巳,唐訓方罷,以喬松年為安徽巡撫。戊子,以唐友耕為雲南提督,令赴昭通。辛卯,譚廷襄言統籌黃河下游地勢,請濬支渠以減漲水,培土墊以衛民田。諭劉長佑、閻敬銘會同籌辦。癸巳,陜回、粵匪紛竄甘境。甲午,允蘇、松、太漕糧減價折徵。乙未,上御撫辰殿大幄,賜蒙古王公宴,賞賚有差。每歲皆如之。復彰化,臺灣兩路賊平。丙申,翁同書加恩遣戍。命左宗棠剔除浙東地丁積弊。飭陜、鄂、川會剿漢南逆匪。是月,免山東、陜西被擾州縣新舊額賦,並孝義等縣倉糧。

是歲,朝鮮入貢。

三年甲子春正月癸卯朔,上率王大臣慶賀兩宮皇太后,禮成,御太和殿受朝。自是每歲皆如之。甲辰,李鴻章軍擊常州援賊於奔牛鎮,大捷。丙午,鳳翔回民乞撫,許之。商南匪竄鄖西。調湖北石清吉軍赴陜。援陜川軍失利於青石關。庚戌,河南捻匪竄隨州。癸丑,豫軍剿張總愚於趙莊山口,失利。己未,官軍復修文及冊亨。庚申,調直、晉兵援寧夏。諭阿拉善旗禁蒙民與回匪勾結。甲子,李世賢竄績溪。丙寅,命都興阿赴綏遠會辦防務。富明阿赴揚州接辦軍務。己巳,浙軍復海寧。彰化匪首戴萬生伏誅。粵匪竄石泉、漢陰、寧陜。是月,免安州等處歉收逋賦。

二月壬申朔,官軍復漢中留壩。黔軍復龍里。乙亥,粵匪竄廣信、建昌。庚辰,寧夏回匪犯中衛等處,熙麟分兵援之。壬午,廣東三山土匪平。癸未,粵匪陷鎮安,旋復之。丁亥,多隆阿圍盩厔久未下,切責之。停山東畝捐,從閻敬銘請也。戊子,桂軍克蒼梧等縣。庚寅,曾國荃等軍克鍾山石壘,合圍金陵。蔣益澧軍復桐鄉。粵匪偪閩境,張運蘭軍援之。壬辰,豫軍克息縣、光州賊寨。甲午,粵匪竄廣豐、弋陽。庚子,陜南匪竄內鄉。

三月壬寅,程學啟等軍克嘉興。贛軍復金谿。江南軍復溧陽。陜軍克盩厔,多隆阿以傷賜假,穆圖善暫督軍務。雷正綰等軍進剿逆回。川匪藍二順竄洵陽。丙午,僧格林沁統全軍赴豫,進至許州。江南軍復廣德。嘉義匪首林贛晟伏誅。己酉,戈登攻金壇受創,命慰問。岑毓英等軍克他郎、鎮沅。庚戌,命多隆阿督辦陜、甘軍務。壬子,蔣益澧各軍克復杭州及餘杭。加左宗棠太子少保銜,賞益澧黃馬褂,尋予世職。甲寅,免杭、嘉新復各地錢糧二年。命穆圖善幫辦多隆阿軍務,暫署欽差大臣。川軍攻松潘匪,復疊溪營城。丁巳,滇軍復景東、元謀及楚雄。癸亥,贛匪竄福建。乙丑,逆首藍大順伏誅。丙寅,浙軍復武康、德清、石門。諭左宗棠收養杭州難民。己巳,提督程學啟卒於軍。庚午,張總愚竄鎮平。甘肅回匪馬三娃陷赤金堡,官軍剿平之。是月,免貴州各府州縣被擾逋賦。

夏四月辛未朔,日有食之。壬申,鮑超軍復句容。丙子,命都興阿赴定邊接統訥欽所部各軍,進剿寧靈踞匪。丁丑,李世賢等竄江西。鮑超軍復金壇。捻、粵各匪合竄棗陽。陜南匪竄河南,陷荊子關。戊寅,湘軍會復古州。辛巳,覈減紹興浮收錢糧,著為永例。甲申,李鴻章督軍克常州。馮子材等軍復丹陽。以故朝鮮王李世子熙襲爵,命侍郎皁保、副都統文謙往封。丙戌,以侍郎薛煥、通政使王拯互訐,均予降調,並申誡臣工。官文赴安陸督師,嚴樹森辦省城防守。庚寅,多隆阿卒於軍。命都興阿督辦甘肅軍務,雷正綰幫辦之。辛卯,贛軍解玉山圍。癸巳,嚴樹森以官文劾降,以吳昌壽為湖北巡撫,唐訓方署之。命楊岳斌督辦江西、皖南軍務。辛卯,僧格林沁會楚軍剿粵、捻於隨州,大敗之。丁酉,以江防下游肅清,裁汰師船,並弛封江之禁。戊戌,粵匪陷弋陽。陜南粵逆竄德安府,僧格林沁軍追剿之。己亥,申誡統兵大臣奏報粉飾。是月,免武進、陽湖本年額賦。

五月庚子朔,黔匪陷長寨、定番、廣順,旋復之。甲辰,粵匪竄天門、應城、德安、隨州。乙巳,粵匪陷寧化,旋復之。熙麟病免,以楊岳斌為陜甘總督,都興阿署之。丁未,允日斯巴尼亞立約通商,命薛煥、崇厚充全權大臣,妥為辦理。諭李鴻章撥勁旅助攻金陵。己酉,李世賢犯撫州,官軍擊走之,復弋陽。賞戈登黃馬褂、花翎,並提督章服,汰留常勝軍,撤遣外國兵官。辛亥,官軍復都江、上江等城。粵匪竄逼西安。癸丑,褫劉蓉、李云麟職,留任。命穆圖善留西安會籌防剿。黔匪竄秀山。戊午,鮑超乞假葬親,詔慰留。李世賢陷宜黃、崇仁,南昌戒嚴。庚申,回匪陷狄道,旋復之。壬戌,粵匪竄黃陂,官文移軍孝感。癸亥,懿旨瑞常、寶鋆、載齡、單懋謙、徐桐輪直進講治平寶鑒。粵匪再陷建寧、寧化,旋復之。丁卯,雷正綰軍復平涼。戊辰,諭疆吏不分畛域,會緝邊匪。命李恆嵩、劉郇膏與丹使璧勒在上海換約。己巳,桂軍克貴縣賊巢,潯州肅清。

六月壬申,申誡各部院大臣毋得仍前洩沓。癸酉,粵匪竄麻城、黃岡。丁丑,雨。蘇軍復長興。黔軍復普安。馬如龍、岑毓英各軍剿迤西回匪,復中甸、維西、思茅、威遠及石膏井等賊巢。戊寅,庫車漢、回亂,辦事大臣文藝、回子郡王愛默特死之。安置哈薩克眾於齋桑淖爾東南。戊子,贛軍克貴溪賊壘。曾國荃軍克金陵外城。辛卯,雨。回匪陷布古爾、庫爾勒。諭撤訥欽等軍。癸巳,浙軍復孝豐。戊戌,官軍克復江寧,洪秀全先自盡,其子福瑱遁,獲賊酋洪仁達、李秀成,江南平。遣醇郡王詣文宗幾筵代祭告。上詣兩宮賀捷。論功,晉封曾國籓一等侯;曾國荃一等伯,加太子少保銜;提督李臣典一等子,賞黃馬褂;蕭孚泗一等男:均賞雙眼花翎。按察使劉連捷等賞世職,升敘有差。命戮洪秀全尸,傳首各省。論各路剿賊功,封僧格林沁子伯彥訥謨祜為貝勒,官文一等伯,李鴻章一等伯,駱秉章一等輕車都尉,均賞雙眼花翎,加楊岳斌、彭玉麟太子少保,並鮑超均一等輕車都尉,都興阿、富明阿、馮子材騎都尉,魁玉雲騎尉。回逆陷喀喇沙爾,辦事大臣依奇哩等均死之。是月,免福建建寧等縣屬被擾逋賦。

秋七月庚子,以江南平論功,晉封議政王恭親王子載澂貝勒,載濬不入八分輔國公,載瀅不入八分鎮國公,加軍機大臣文祥太子太保銜,寶鋆、李棠階太子少保銜,加恩宗親及御前大臣、內務府大臣,餘賚錄有差。辛丑,以歲逢甲子,詔停句情實人犯。諭:「江南新復,民生彫敝,有司招徠撫恤之。其軍務未靖諸省,統兵大臣、督撫等須激厲將士,奮勉圖效。」俄兵入科布多卡倫,執委員及扎薩克。壬寅,禁宗室、覺羅潛住外城。甲辰,追論附苗沛霖罪,總兵博崇武等戍新疆,按察使張學醇戍軍臺。粵匪竄踞羅田。桂匪陷歸順。己酉,詔修明太祖陵。裁江北釐金。復兩淮鹺務。庚戌,實授沈桂芬山西巡撫。以鄭敦謹為河東河道總督。辛亥,丹國換約成。壬子,洪仁達、李秀成伏誅。汪海洋竄踞許灣。癸丑,洪福瑱入湖州。鹽茶、固原回匪復叛,北竄寧靈,擾中衛、靖遠,撒回句結陷循化,吐魯番屬托克遜漢、回亦變亂。甲寅,戶部侍郎吳廷棟言金陵告捷,請益加敬懼,嘉納之。丁巳,以廣西道梗,止越南入貢。奇臺漢、回作亂,古城、烏魯木齊同時不靖。文光等軍進援庫車,失利,覆於烏沙塔克拉,死之。庚申,狄、河回匪結撒回擾河州。贛軍復崇仁、東鄉。辛酉,復金谿。壬戌,祁俊藻因病乞休,命仍以大學士銜直弘德殿。官軍獲昌圖盜匪劉發好等,誅之。癸亥,復鄭親王、怡親王襲爵。錄已故諸臣功,予胡林翼一等輕車都尉,李續賓二等輕車都尉,塔齊布、張國樑、江忠源、程學啟三等輕車都尉,加賞江忠濟、羅澤南、多隆阿、曾國華一雲騎尉。贛軍復宜黃,甲子,克許灣。乙丑,僧格林沁敗賊麻城。曾國荃乞病,溫諭止之。李臣典以傷卒於軍。是月,免江蘇、安徽各屬被擾逋賦。

八月己巳朔,定諸王位次,著為令。贛軍復南豐,庚午,烏魯木齊參將反,提督業普沖額死之。伊犁危急,調塔爾巴哈臺喀爾喀蒙兵援之。諭劉蓉專辦陜西軍務,穆圖善統所部赴甘,與雷正綰籌辦軍務。趣楊岳斌即赴陜甘任。辛未,諭張集馨赴固原、鹽茶辦撫回事宜。癸酉,蘇、浙官軍會克湖州及安吉。乙亥,贛軍復新城,陳炳文降。辛巳,官軍復廣德。賞郭松林世職,楊鼎勛、周盛波黃馬褂。貴縣匪平。擢劉銘傳為直隸提督。壬午,回匪陷古城漢城。癸未,雷正綰軍克張家川賊巢。甲申,僧格林沁剿羅山竄賊失利,都統舒通額等死之。丁亥,雲南巡撫賈洪詔以藉病規避,褫職。己丑,調土謝圖汗、車臣汗蒙兵赴烏魯木齊等處助剿。壬辰,浙軍追賊於昌化、淳安,擒賊酋黃文金等誅之。以林鴻年為雲南巡撫。癸巳,詔新疆各路大臣分別剿撫。以回郡王伯錫爾聯絡各城殺賊,嘉獎之。庫爾喀喇烏蘇等處回匪亂,官軍失利。甲午,命麟興辦烏里雅蘇臺立界事宜。乙未,僧格林沁剿賊失利,總兵巴揚阿等死之。丙申,雷正綰攻蓮花城不利,回匪復陷固原。丁酉,河、狄回匪竄犯蘭州及金縣。

九月己亥朔,劉銘傳各軍擊敗寧國等處竄匪。庚子,贛軍復雩都。以李云麟乞病規避,褫職,撤所統隴軍。壬寅,曾國荃以疾乞免,允之。命馬新貽為浙江巡撫,留辦安慶防守事宜。癸卯,命穆圖善幫辦都興阿軍務。甲辰,楊岳斌乞病,溫諭止之。李世賢犯南安,官軍擊走之。乙巳,回匪陷葉爾羌,署參贊奎棟死之,喀什噶爾、英吉沙爾武弁同叛。己酉,西寧回眾降。庚戌,張家川回匪犯慶陽。辛亥,贛賊竄南雄。壬子,粵匪陷開化,竄江西。黃、麻匪竄商城。乙卯,日斯巴尼亞換約。丙辰,諭內務府力求撙節。命札克通阿署哈密大臣。丁巳,西寧回匪復叛。戊午,粵匪蔡得榮等竄陷階州。庚申,詔修曲阜聖廟及各省學宮。辛酉,修浙江海塘。甲子,捻匪竄蘄水,鄂軍失利,總兵石清吉死之。乙丑,俄兵闌入阿爾泰淖爾。丁卯,沈桂芬請籌費移屯以恤旗民。

冬十月戊辰朔,允楊岳斌回籍省親,並募勇赴甘。命刑部尚書綿森、戶部侍郎吳廷棟往治察哈爾獄。己巳,改烏魯木齊提督文祥名為文祺。辛未,褫將軍常清職,命明緒代之,以聯捷為參贊大臣。命武隆阿統援救烏魯木齊各軍,節制領隊大臣以下。壬申,鮑超軍擊賊大捷,賞雙眼花翎。席寶田軍獲賊酋洪仁玕等。皖南北肅清。乙亥,回匪陷烏魯木齊滿城及綏來,都統平瑞等死之。哈密漢、回亂。命保恆署烏魯木齊都統,李鴻章署兩江總督,吳棠署江蘇巡撫,富明阿署漕運總督。戊寅,獲洪福瑱於石城,誅之。賞枕葆楨一等輕車都尉。封鮑超一等子。論恢復全浙功,封左宗棠一等伯,賞蔣益澧騎都尉。粵匪陷瑞金,旋復之。庚辰,粵匪陷漳州、龍巖、南靖、武平,按察使張運蘭等死之。劉蓉分軍守邠州等處。乙酉,明誼與俄使換分界約,科布多城卡外蒙古,阿爾泰淖爾烏梁海均屬俄。給鮑超假,所部宋國永等軍援閩,歸左宗棠節制。丁亥,雷正綰軍克蓮花城,賞曹克忠黃馬褂。僧格林沁剿賊大捷,賞郭寶昌等黃馬褂,賊首馬融和以眾降。己丑,四川援軍復仁懷。庚寅,粵匪陷平和。辛卯,陷嘉應、大埔。丙寅,諭曾國籓仍駐金陵,李鴻章等回本任。是月,免河南信陽等處被擾額賦,浙江西安等縣逋賦。

十一月己亥,豁江寧所屬糧賦三年。壬寅,回匪陷河州。癸卯,築濮州金是。乙巳,文祺、伯錫爾剿平哈密回。己酉,免江蘇歷年州縣攤賠銀兩,永禁派攤名目。壬子,沈葆楨請飭援閩,兼防賊回竄。甲寅,粵軍復武平,命閩、浙、贛軍會剿,毋縱入海。回匪陷阿克蘇、烏什,辦事大臣富珠哩、文興等死之。癸亥,僧格林沁擊襄、棗竄匪不利,發、捻各匪遂竄鄧州。甲子,諭飭劉連捷、劉銘傳各軍前進,歸僧格林沁調遣。乙丑,雷正綰等軍剿敗固原回匪。丙寅,文祺等剿巴里坤回匪,平之。回匪陷庫爾喀喇烏蘇,伊犁戒嚴。丁卯,滿慶言汪曲結布卒,請賞青饒汪曲諾們罕名號,協理西藏商上事務,允之。是月,免江蘇上元等縣被擾逋賦。

十二月戊辰朔,閩軍剿漳州匪失利,林文察等死之。己巳,命吳棠仍兼管江北事務。庚午,肇慶客匪平。都興阿等軍克清水堡。甲戌,停河南例貢棗實。築浙江海塘。乙亥,回匪陷金縣。曹克忠軍克鹽關。戊寅,伊犁官軍敗績,領隊大臣托克托奈等死之。允明緒請借俄兵助剿。己卯,濟木薩官軍失利。庚辰,予諸暨義民包立身等優恤。允吳棠請,試行河運。乙酉,陶茂林軍復金縣。丙戌,戍李元度軍臺。己丑,僧格林沁移軍寶豐剿賊,勝之。甲午,官軍剿回匪大捷,伊犁解圍,賞明緒黃馬褂。是月,免浙江瑞安被擾逋賦,江蘇太倉等州縣,淮安等衛被擾災賦。

是歲,朝鮮、琉球入貢。

四年乙丑春正月丁酉朔,官軍克靜寧賊巢。回匪陷古城漢城。庚子,巴彥岱城被圍,官軍不利。釋陳孚恩、樂斌,命襄辦伊犁兵饟事。壬寅,從曾國籓請,調劉銘傳軍赴閩,鮑超募川軍赴甘。追予死事道員何桂珍、知州劉騰鴻、游擊畢金科謚。甲辰,烏魯木齊提督文祺卒於巴里坤。回匪陷木壘等處。丁未,張集馨以罪褫職。復已革提督馬德昭原官。平、固回匪竄擾靈臺及汧陽、隴州。戊申,命伯錫爾署哈密幫辦大臣。辛亥,臺灣會匪平。甲寅,粵匪陷永定、雲霄。丙辰,復設淮揚河務兵備道,改設徐海河務兵備道。丁巳,粵、捻並竄魯山,護軍統領恆齡等死之。癸亥,回匪陷濟木薩。甲子,黔匪陷定番,旋復之,又陷黔西。乙丑,回匪竄永昌。

二月辛未,以蒙兵援古城,戰不利,諭撤已調各兵均回旗。壬申,陜軍敗回匪於醴泉,命胡中和總統進剿。戊寅,以雲南臨安官紳不附回逆,諭嘉之。己卯,允沈葆楨假歸省。癸未,以直隸諸省雷雹災異,詔修省。雷正綰軍復克固原等處。貴州參將曹元興謀逆,伏誅。甲申,長陽土匪平。丙戌,復永定、龍巖。武隆額等軍援巴彥岱城,失利。己丑,黔西匪陷大定。苗匪陷天柱、古州。以馬如龍、岑毓英肅清曲靖、尋甸,擒斬逆首馬聯升等,獎敘有差。癸巳,福建官軍剿李世賢、汪海洋各股於古田、漳州,大捷。

三月丁酉,以田興恕玩視軍務,慘殺教民,遣戍新疆。辛丑,陶茂林剿平郭家驛等處回匪。諭僧格林沁「駐軍指揮調度,勿輕臨前敵,致蹈危機」。壬寅,恭親王罷軍機,撤議政。命文祥等辦總理各國事務衙門事宜。粵匪陷詔安,知縣趙人成死之。癸卯,涼州回眾叛,剿平之。允英、法在江寧通商。命鮑超籌備西征,準專奏。惇親王言恭親王被參不實,下王公、大學士等詳議以聞。乙巳,塔城回亂。錫霖乞病,罷之,命赴伊犁,由明緒調遣。提督譚勝達以剋扣勇糧褫職,仍命赴鮑超軍。以武隆額署塔爾巴哈臺參贊大臣。丁未,巴里坤領隊大臣色普詩新以兵援古城,遇賊,失利,死之。己酉,閩軍敗汀州、連城踞賊。庚戌,甘軍擊退古浪、平番回匪。辛亥,從王大臣請,命恭親王仍在內廷行走,並管總理各國事務衙門。丙辰,諭官文簡汰兵、勇。己未,命楊岳斌赴甘。沈葆楨丁母憂,詔奪情署江西巡撫。辛酉,西寧回匪復叛,陷大通。壬戌,桂軍復永淳。癸亥,命毛昶熙回京。是春,免直、蘇、皖、贛災擾諸處額賦及逋課。

夏四月乙丑朔,禁熱河圍場墾紅椿內地。肅州回匪踞嘉峪關,圍州城,撫彞回匪亦起。丁卯,彭玉麟疏辭漕督,請專辦水師,允之。留吳棠漕運總督任,辦清、淮防務。己巳,官軍復鹽茶,免已革提督成瑞罪。庚午,回匪陷古城,領隊大臣惠慶等死之。乙亥,臺灣肅清。丁丑,黔軍復玉屏、天柱。命恭親王仍直軍機,毋復議政。甘州回匪陷永固堡。壬午,粵匪再陷沭、宿。霆軍十八營不原西征,潰於金口。止鮑超西征,命招集潰勇赴閩剿賊。乙酉,寧夏官軍剿賊大捷。丙戌,粵、捻並回竄兗、濟,命劉銘傳赴直隸設防。己丑,賜崇綺等二百六十五人進士及第出身有差。壬辰,以山東賊勢蔓延,命曾國籓出省督師,會僧格林沁軍南北合擊。癸巳,僧格林沁剿賊於菏澤南吳家店,失利,與內閣學士全順、總兵何建鰲等均死之。事聞,輟朝三日,特予配饗太廟。命曾國籓督師剿賊,李鴻章署兩江總督。

五月乙未朔,諭成祿進剿肅州踞匪。霆營叛勇由江西竄福建。粵、捻並竄開州、東明。丙申,陶茂林軍潰,回匪圍安定,蘭州戒嚴。戊戌,命曾國籓節制直、豫、魯三省軍防。甘肅潰勇竄擾陜西。乙巳,免李元度遣戍。丁未,粵、捻並渡運河,東竄濟寧、兗、泰。戊申,嚴諭盛京、吉林剿辦馬賊。己酉,以剿賊無功,褫官文、張之萬、毛昶熙職,均留任,並撤官文宮銜。趣鮑超赴江西。辛亥,命侍讀學士衛榮光赴東昌督辦沿河民團。壬子,官軍克漳州、南靖。允沈葆楨終制。曾國籓辭節制三省軍務,不許。回匪陷肅州。粵、捻分竄豐、沛。諭整頓沿海水師。竄陜潰勇平。諭劉長佑駐直境,崇厚駐東昌,部署沿河防務。黔匪陷廣順,旋復之。甲寅,雨。粵匪圍永定。乙卯,蘇軍復漳浦。以劉坤一為江西巡撫。庚申,以防剿遲延,褫提督劉銘傳職,仍留任。楊岳斌請開缺,不允,仍命赴甘。壬戌,奇臺官軍復濟木薩。癸亥,官軍復階州。

閏五月甲子朔,起沈葆楨督辦江西防剿。乙丑,粵匪由福建竄嘉應。戊辰,粵軍復平和、詔安。川軍復正安。壬申,泗城匪平。甲戌,減杭、嘉、湖屬漕米二十六萬石。丁丑,汪海洋回竄永定,官軍失利,總兵丁長勝等死之。己卯,回匪踞阜康。張總愚南竄至雉河集,諭劉銘傳、吳棠等會剿。粵匪陷廣東鎮平。遵義匪降。丙戌,鄂爾多斯蒙兵擊退花馬池回匪。黔匪陷綏陽。己丑,上臨僧忠親王喪,賜奠。賞其孫那爾蘇貝勒、溫蘇都輔國公。曾國籓駐軍臨淮。特克慎卒,命皁保查巴爾虎爭界事,恩合為吉林將軍。庚寅,以久旱,諭修省求言。癸巳,諭耆英獲咎,毋庸昭雪。禁肅順之子出仕。以耆英子慶錫鳴冤,謂其死由肅順也。

六月甲午朔,增設安徽安廬滁和道。改鳳廬潁道為鳳潁六泗道,仍兼鳳陽關監督。命劉長佑回保定,潘鼎新軍駐濟寧。丙申,甘肅民勇復嘉峪關。以安西、玉門諸縣回亂,諭楊岳斌進駐蘭州。己亥,申諭各省甄別牧令。壬寅,塔爾巴哈臺回匪誘戕參贊錫霖等,圍城,為喇嘛棍噶札拉參兵擊退。調武隆額為塔爾巴哈臺參贊大臣。以額騰額為葉爾羌參贊大臣。丙午,雨。論載華等辦工侵蝕罪,奪載華貝子、恩弼輔國公,仍圈禁二年。己酉,沈桂芬以憂免,命曾國荃為山西巡撫。黔匪復陷天柱,擾湖南會同,勞崇光、李瀚章合剿之。黔軍復黔西,在獨山失利。壬子,岷州回匪亂,戕知州增啟等,擾洮州。乙卯,援黔川軍復正安。丁巳,奇臺、哈密陷,哈密辦事大臣札克當阿死之。文麟退巴里坤。諭楊岳斌、成祿、聯捷軍進擊肅州匪。回子臺吉陸布沁投誠。丁巳,御史穆緝香阿請慎選侍御僕從。諭內務府稽察有便僻側媚者,舉實嚴懲。是夏,免陜西、浙江、福建等州縣被擾額賦,及哈密兵擾糧課。

秋七月癸亥朔,諭劉蓉嚴防定邊、鄜、延、邠、隴,楊岳斌防範回酋赫明堂。甲子,回匪陷巴燕岱,伊犁領隊大臣穆克登額等死之。褫助逆伯克都魯素等職。官軍復庫爾喀喇烏蘇。命布爾和德署領隊大臣,援塔城。雷正綰各軍攻金積堡失利,退至韋州。丁卯,武隆額剿禮拜寺回逆,平之。黔匪陷石阡,知府嚴謹陣沒,官軍旋復其城。癸酉,命董恂、崇厚為全權大臣,辦理商約事務。己卯,賞科爾沁親王伯彥那謨祜世襲博多勒噶臺親王號。壬午,御史蔡壽祺以妄言褫職。黔匪陷大定,旋復之。己丑,奉天馬賊擾遵化、薊州,罷玉明,予嚴議。以恩合署盛京將軍。換荷蘭約。庚寅,諭禁法教士干預軍事。壬辰,陳國瑞罷幫辦軍務。

八月庚子,以議撫貽誤,褫恩麟職,戍成瑞黑龍江。祁俊藻致仕。粵、捻各逆竄皖、豫境。壬寅,設機器局於上海。癸卯,回匪犯巴里坤,訥爾濟擊走之。文麟軍於奎蘇失利。甲辰,里塘夷務竣。予四川總督駱秉章假,命崇實署之。嚴諭麟興親勘唐努烏梁海立界。乙巳,命左宗棠駐粵,節制贛、粵、閩三省各軍。丙午,命曾國籓進駐許州,會剿豫捻。辛亥,令伊犁捕誅從逆官兵。予剿賊出力額魯特總管蒙庫巴雅爾等獎敘有差。癸酉,褫玉明職。郭嵩燾請開缺,以語多負氣,嚴飭之。減江西丁漕浮收。裁州縣捐攤繁費。粵酋汪海洋殺李世賢。乙卯,粵匪陷廣東長樂。英、法還天津海口砲臺。丙辰,都興阿辭督辦軍務,不許。丁巳,諭李鴻章等妥議江北新漕河海並運。庚申,蘇、松、杭、嘉、湖屬水,賑恤之。予龍溪鄉團殉難男婦建祠,賜名忠義鄉。辛酉,諭崇實等查辦酉陽教案。

九月甲子,上躬送定陵奉安,命肅親王華豐等留京辦事。長樂賊以城降粵軍。丙寅,免定陵奉安經過地方田賦。戊辰,以捻首張總愚及賴、任各逆竄擾豫、魯,命李鴻章會剿,吳棠署兩江總督,李宗羲署漕運總督。命曾國籓仍駐徐州。己巳,允招商辦雲南銅廠。庚午,調江南砲船赴山西河防教習水戰。壬申,好水川回眾降。官軍解南陽圍。陶茂林軍再潰。甲戌,官軍復鎮平。丙子,雷軍部將胡大貴、雷恆叛,圍涇州,提督周顯承擊退之。馬化龍與胡大貴等分竄陜境。授張之萬河東河道總督。己卯,上奉兩宮皇太后啟鑾。粵匪犯龍南,劉坤一赴贛州督剿。甲申,葬文宗於定陵。乙酉,回鑾。奇臺知縣恆頤以民勇復奇臺、濟木薩、古城三城。丁亥,上還宮。戊子,文宗帝後升祔太廟,翼日頒詔覃恩有差。庚寅,褫甘肅提督陶茂林職,以曹克忠代之,逮治總兵陶生林等。左宗棠辭節制三省,不允。是秋,免陜西孝義、浙江蘭谿等處被擾逋賦。

冬十月壬辰朔,藏兵克瞻對。回匪犯慶陽,官軍擊退之。癸巳,定比利時條約。甲午,命徐繼畬以三品京堂在總理各國事務衙門行走。命廓爾喀例貢俟六年並進。庚子,減浙江漕米南米浮收。壬寅,粵匪陷和平。乙巳,王榕吉言潞鹽壅滯,請分別停減續加課票,議行。丁未,回匪圍鞏昌、寧遠。己酉,浙軍克南田賊壘。辛亥,命劉蓉署陜西巡撫。壬子,以升祔禮成,祫祭太廟。醇親王辭八旗練兵。諭仍稽察校閱,勤加訓練。甲寅,馬賊偪奉天,官軍失利。庚申,命福興統吉、黑馬隊及神機營兵赴剿。辛酉,釋綿性。

十一月癸亥,賴、任各匪竄舞陽、郾城,與張總愚股合,諭鄂、豫夾擊。丙寅,減徵蘇、松、常、鎮、太倉米豆五十四萬石有奇。壬寅,奉軍剿馬賊失利。李棠階卒。命李鴻藻在軍機大臣上學習行走。湖北巡撫鄭敦謹入為戶部侍郎,以李鶴年代。乙亥,治不顧主將罪,成保論斬,戍郭寶昌新疆。丙子,奉天匪首徐點復叛於廣寧。庚辰,粵匪陷嘉應。鞏昌解圍。丙戌,官軍失利於濟木薩,恆頤死之。丁亥,諭劉長佑駐邊隘督剿馬賊。己丑,川軍剿松潘番賊,平之。黔匪犯敘永、綦江。庚寅,命左宗棠親往嘉應視師。

十二月壬辰朔,曾國籓移軍周家口。允明緒遣榮全如俄借兵貸糧。甲午,黔匪陷清鎮縣城,旋復之。命周達武為貴州提督。乙未,聯捷坐貪擾,撤幫辦軍務,以侍衛隸成祿軍。己亥,黎獻軍潰於肅州。辛丑,馬賊回竄昌圖。允戶部請,撥鹽課諸款增內廷用費三十萬。壬寅,熱河軍復朝陽。癸卯,命伯彥訥謨祜回旗會各盟長檄蒙兵協剿馬賊。以文麟為哈密辦事大臣。乙巳,瞻對逆酋工布朗結等伏誅,三瞻均歸達賴管理。丙午,金州匪偽降,竄鐵嶺,命文祥等辦奉天防守事宜。壬子,以雪澤愆期,詔清理庶獄,瘞暴露骸骨。乙卯,恩合以貽誤軍事褫職。提督成大吉軍潰於麻城。丙辰,粵軍會復越南寧海府城。調都興阿為盛京將軍。命穆圖善督辦甘肅軍務,接統都興阿所部各軍。庚申,上御保和殿,賜朝正外籓等宴。自是每歲皆如之。滇軍復麗江、鶴慶。

是冬,免四川松潘、湖南茶陵等州縣被擾逋賦。

五年丙寅春正月辛酉朔,停筵宴。甲子,捻匪擾鄂,曾國籓檄劉銘傳援黃州。馬化龍乞撫,獻寧夏漢城。乙丑,桂軍復那檀。免福建例貢。己巳,命穆圖善辦撫回善後事宜。庚午,雲南巡撫林鴻年赴昭通。乙亥,馬賊入踞伯都訥,旋及雙城堡,吉林危急。文祥、寶善檄黑龍江兵暨馬隊援之。己卯,黃巖總兵剛安泰巡洋,遇艇匪,死之。癸未,林鴻年坐畏葸貽誤褫職,劉岳昭代之。左宗棠督諸軍復嘉應,粵匪平。左宗棠以次論功賞敘。丙戌,馬賊竄陷阿勒楚喀及拉林城,富明阿往吉林剿之。命特普欽回黑龍江布防守。吳昌壽降調,調李鶴年為河南巡撫,以曾國荃為湖北巡撫。戊子,奉軍復八面城。己丑,諭嚴緝軍營哥老會匪。

二月辛卯朔,詔左宗棠等緩撤江、閩各軍,備調北路助剿捻、回諸匪。黔回陷永寧,旋復之。壬辰,命兆琛赴鎮遠辦軍務。辛丑,官軍復黃陂。丁未,都興阿坐部勇肆殺,褫職留任。戊申,命廣東陸路提督高連升赴任剿辦土匪。伯彥訥謨祜剿馬賊於鄭家屯,大捷。諭馬新貽籌辦海塘。辛亥,定安軍剿馬賊於長春,勝之,詔復副都統。壬子,德英憂免,以富明阿為吉林將軍。丙辰,召郭嵩燾來京,以蔣益澧署廣東巡撫。己未,湖南軍擊退黔苗。

三月壬戌,曾國籓移軍濟寧,督剿張總愚。乙丑,復阿勒楚喀、伯都訥、雙城堡三城。己巳,奉軍剿南北路馬賊,大敗之。蠲奉天、吉林被擾諸地銀米。庚午,明誼乞病,命麟興統蒙兵援伊犁。乙亥,賴文光等竄偪開封。戊寅,免隨征黑龍江牲丁貢貂。諭內外臣工講求律例。己卯,馬賊竄擾熱河。庚辰,允馬化龍等投誠。甲申,張總愚竄濮、範,賴文光等由豫竄鄆城、鉅野,諭曾國籓等守運河,喬松年軍截剿。乙酉,馬賊陷牛莊。丙戌,曹毓瑛卒。丁亥,閩軍復崇安、建陽。戊子,命李鴻藻為軍機大臣,胡家玉在軍機學習。是春,免河南積欠錢糧,直隸安州、奉天新民等州縣被水、被擾額賦。

夏四月己丑朔,奉天北路匪首馬傻子伏誅,降其餘眾。官軍復牛莊。粵、捻犯直隸河岸,擊退之。辛卯,允曾國荃請裁兵並餉,並調劉聯捷、彭毓橘、硃南桂、郭松林赴湖北。丙申,回目以洮州降曹克忠軍。戊戌,命馬如龍署云南提督。庚子,召文祥、福興回京,命都興阿接辦奉天軍務,節制各軍。辛丑,訥爾濟復木壘、奇臺、古城,招募民勇防守。癸卯,官軍復綏陽。甲辰,回匪陷靖遠。戊申,諭奉天、吉林會剿山內外賊匪。己酉,譚玉龍軍潰,命曹克忠兼統其軍。壬子,回匪回竄慶陽。披楞大舉悉兵眾迫布魯克巴,命景紋赴邊隘查辦。甲寅,武緣匪平。丙辰,粵、捻擾銅、沛及泗州、靈壁。勞崇光進駐昆明。杜文秀復陷麗江、鶴慶、劍川。戊午,回匪犯蘭州,官軍擊退之。

五月壬戌,黔匪復陷興義、貞豐、永寧。俄使堅請黑龍江內地通商。諭特普欽整頓營伍。乙丑,大考翰、詹,擢孫毓汶四人一等,餘升黜有差。戊辰,馬朝清降,靈州復。辛未,回匪霍三等回竄鳳、岐,官軍擊退之,諭楊岳斌、劉蓉合擊,毋再任入陜。甲戌,回匪陷塔爾巴哈臺,武隆額死之。以德興阿為參贊大臣,奎昌署科布多參贊。嚴諭成祿迅速出關。乙亥,回匪陷伊犁,明緒等死之。以榮全署伊犁將軍。命庫克吉泰督辦新疆軍務。丁丑,詔清庶獄。壬午,以久不雨,詔求直言,禁凌虐罪囚。甲申,諭保舉盡心民事官吏。丁亥,官軍復荔波。是月,免廣東嘉應等處被擾逋賦。

六月庚寅,雨。允左宗棠請,在閩建廠試造輪船。壬辰,諭內外大臣勤職。辛丑,成祿軍進圍肅州。壬寅,諭富明阿搜捕山場餘匪。甲辰,靈山匪平。戊申,烏里雅蘇臺將軍明誼病免。己酉,以德勒克多爾濟為烏里雅蘇臺將軍,福興為綏遠城將軍。庚戌,鹽、固回匪投誠。辛亥,凌雲、陽萬土匪平。乙卯,諭楊岳斌剿狄、河回匪。

秋七月庚申,褫廣鳳、圖爾庫職,逮訊。命侍郎魁齡等使朝鮮,冊封王妃。壬戌,官軍復哈密。甲子,諭整頓廣東吏治、軍務、釐稅。乙丑,李鴻藻丁母憂,懿旨令百日後仍直弘德殿、軍機處。庚午,湘軍克思南賊巢。壬申,李鴻藻請終制,不許。癸酉,減蘇、松、常、太浮收米三十七萬餘石,浮收錢百六十七萬餘貫。丙子,崇厚會日斯巴尼亞使換約。己卯,黔匪陷石阡,旋復之。庚辰,免烏梁海七旗應納半貢。乙酉,河南河決胡家屯。

八月戊子,劉蓉病免,調喬松年為陜西巡撫,以英翰為安徽巡撫。己丑,濮州河決。庚寅,潯、鬱匪平。裁山海關監督,改設奉錦山海關道。辛丑,賞李云麟頭等侍衛,幫辦新疆軍務。癸卯,楊岳斌病免,調左宗棠為陜甘總督,吳棠為閩浙總督,張之萬為漕運總督。實授瑞麟兩廣總督。甲辰,官軍克大孤山賊巢,徐宗禮伏誅。乙巳,官軍剿敗張、牛諸捻。以月食示儆,飭廷臣修省。丁未,從御史慶福請,積粟張家口、綏遠城,轉運新疆,以濟民食。

九月丁巳朔,命譚廷襄會崇厚辦義國商約事務。癸亥,福建興化土匪平。甲子,諭李云麟與麟興等整頓北路防軍。命皁保赴歸化督運新疆餉款。回匪陷阜康。祁俊藻卒。辛未,滇回陷安寧等州縣。癸未,左宗棠請將閩、浙綠營減兵加餉,就餉練兵。允之。是秋,免貴州、廣東、山東、福建被擾,江西被災等處額賦,浙江等縣逋賦。

冬十月辛卯,命劉長佑嚴覈畿輔兵額。癸巳,張總愚由陜州竄平陸,官軍擊退之。乙未,命沈葆楨總司福建船政事務。命劉典幫辦左宗棠軍務。己亥,張總愚西竄,陷華陰、渭南。甘回竄宜君、三水。詔責曾國籓任賊蔓延。辛丑,允李鴻藻病假。命富明阿辦吉林善後事宜,汪元方為軍機大臣。壬寅,黔回陷興義,旋復之,並復安平、鎮寧。乙巳,曾國籓乞病,請開各缺,在營效力,並注銷侯爵,諭慰之,命病痊陛見。諭穆圖善援應陜西。丙午,修海寧石塘。是月,免安徽、壽州等州縣被水新舊額賦。

十一月丙辰,命曾國籓回兩江總督,署通商大臣。授李鴻章欽差大臣,節制湘、淮各軍,專任剿匪。戊午,予山東巡撫閻敬銘假,以丁寶楨署之。庚申,劉銘傳等剿任、賴各匪於金鄉,大捷。乙丑,三札、兩盟西征蒙兵潰,李云麟回烏城。諭庫克吉泰統吉、黑軍速進。丁卯,川軍克桐梓賊巢。丁酉,曾國荃劾官文貪庸驕蹇。命撤任查辦。己卯,定福建船政章程。

十二月丁亥,以給事中尋鑾煒參劾失實,切責之,因諭科道慎重言事。己丑,郭松林等大破任、賴諸匪於德安。庚寅,以黃河趨北,諭蘇廷魁周歷履勘,並會同直、魯、豫三省籌辦堤工。甘回復陷哈密。罷胡家玉軍機,褫職留任,以受官文賄也。甲午,曾國籓復疏請開缺。溫旨慰留。己亥,雷正綰軍復平涼。呼蘭匪平。庚子,援黔湘軍剿苗匪於銅仁,大捷。己酉,回匪圍慶陽,提督周顯承等力戰死之。甲寅,陜軍剿張總愚,失利於灞橋,總兵蕭德陽等死之。以捻勢披猖,命曾國籓等廣籌方略。

是歲,朝鮮、琉球入貢。


\end{pinyinscope}