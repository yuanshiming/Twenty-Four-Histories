\article{列傳三}

\begin{pinyinscope}
諸王二

○太祖諸子一

廣略貝勒褚英子安平貝勒杜度敬謹莊親王尼堪杜度子?厚貝勒杜爾祜

貝子穆爾祜恪僖貝子特爾祜懷愍貝子薩弼

禮烈親王代善子巽簡親王滿達海克勤郡王岳託碩託穎毅親王薩哈璘

謙襄郡王瓦克達輔國公瑪占滿達海從子康良親王傑書

岳託子衍禧介郡王羅洛渾顯榮貝勒喀爾楚渾鎮國將軍品級巴思哈

羅洛渾曾孫平敏郡王福彭薩哈璘子阿達禮順承恭惠郡王勒克德渾

勒克德渾子勒爾錦孫錫保

太祖十六子:孝慈高皇后生太宗,元妃佟佳氏生廣略貝勒褚英、禮親王代善,繼妃富察氏生莽古爾泰、德格類,大妃烏拉納喇氏生阿濟格、睿親王多爾袞、豫親王多鐸,側妃伊爾根覺羅氏生饒餘郡王阿巴泰,庶妃兆佳氏生鎮國公阿拜,庶妃鈕祐祿氏生鎮國將軍湯古代、輔國公塔拜,庶妃嘉穆瑚覺羅氏生鎮國公巴布泰、鎮國將軍巴布海,庶妃西林覺羅氏生輔國公賴慕布,而費揚古不詳所自出。

廣略貝勒褚英,太祖第一子。歲戊戌,太祖命伐安楚拉庫路,取屯寨二十以歸。賜號。軍夜行,?洪巴圖魯,封貝勒。歲丁未,偕貝勒舒爾哈齊、代善徙瓦爾喀部蜚悠城新附之陰晦,纛有光,舒爾哈齊疑不吉,欲班師,褚英與代善持不可。抵蜚悠城,收其屯寨五百戶以先行,烏喇貝勒布占泰以萬人邀之路。扈爾漢所部止二百人,褚英、代善策?,令扈爾漢?,今日何懼?且布占泰降虜耳,乃不能復縛之耶?」?馬諭之曰:「上每征伐,皆以寡擊皆奮,因分軍夾擊,敵大敗,得其將常柱、瑚里布,斬三千級,獲馬五千、甲三千。師還,上嘉其勇,錫號曰阿爾哈圖土門,譯言「廣略」。歲戊申三月,偕貝勒阿敏伐烏喇,克宜罕山城。布占泰與蒙古科爾沁貝勒翁阿岱合兵出烏喇二十里,望見我軍,知不可敵,乃請盟。

,諸弟及?臣愬於上,上浸疏之。褚英意不自得,焚?褚英屢有功,上委以政。不恤表告天自訴,乃坐咀?,幽禁,是歲癸丑。越二年乙卯閏八月,死於禁所,年三十六。明人以為諫上毋背明,忤旨被譴。褚英死之明年,太祖稱尊號。褚英子三,有爵者二:杜度、尼堪。

安平貝勒杜度,褚英第一子。初授臺吉。天命九年,喀爾喀巴約特部臺吉恩格德爾請內附,杜度從貝勒代善迎以歸,封貝勒。天聰元年,從貝勒阿敏、岳託等伐朝鮮,朝鮮國王李倧請和,諸貝勒許之。阿敏欲仍攻王京,岳託持不可;阿敏引杜度欲與留屯,杜度亦不可:卒定盟而還。三年十一月,從上伐明,薄明都,敗明援兵。又偕貝勒阿巴泰等略通州,焚其舟,至張家灣。十二月,師還,至薊州,明兵五千自山海關來援。與代善親陷陣,傷足,駐遵化。四年正月,明兵?攻,敗之,斬其副將,獲駝馬以千計。?猶力戰,殲其

七年,明將孔有德、耿仲明降,偕貝勒濟爾哈朗、阿濟格赴鎮江迎以歸。詔問伐明及朝鮮、察哈爾三者何先,杜度言:「朝鮮在掌握,可緩;察哈爾逼則徵之;若尚遠,宜取大同邊地,秣馬乘機深入。」八年,軍海州。崇德元年,進封安平貝勒。海州河口守將伊勒慎報明將造巨艦百餘截遼河,命杜度濟師,明兵?,乃還。是冬,上伐朝鮮,杜度護輜重後行,略皮島、雲從島、大花島、鐵山。二年二月,次臨津江。前一日冰解,夕大雨雪,冰復合,師畢渡。上聞之曰:「天意也!」從睿親王多爾袞取江華島,敗其水師,遂克之。

三年,多爾袞將左翼、岳託將右翼伐明,杜度為嶽託副。師進越密雲東墻子嶺,明兵迎戰,擊敗之。進攻墻子嶺堡,分軍破黑峪、古北口、黃崖口、馬蘭峪。岳託薨於軍,杜度總軍事。會多爾袞軍於通州河西,越明都至涿州,西抵山西,南抵濟南,克城二十,降其二。凡十六戰皆捷,殺總督以下官百餘,俘二十餘萬。還,出青山口,自太平寨奪隘行。四年四月,師還,賜駝一、馬二、銀五千,命掌禮部事。略錦州、寧遠。五年,代濟爾哈朗於義州屯田,刈錦州禾,遇明兵,敗之,克錦州臺九、小凌河西臺二。明總督洪承疇以兵四萬營杏山城外,偕豪格擊敗之,追薄壕而還,又殲運糧兵三百。往錦州誘明兵出戰,復擊敗之,獲大凌河海口船,追斬敵之犯義州者。冬,再圍錦州。六年,攻廣寧,敗松山、錦州援兵。以從多爾袞離城遠駐,遣軍私還,論削爵,詔罰銀二千。復圍錦州,敗明兵於松山。是秋,復從上伐明,留攻錦州。七年六月,薨。病革時,諸王貝勒方集篤恭殿議出征功罪,上聞之,為罷朝。喪還,遣大臣迎奠。雍正二年,立碑旌其功。

杜度子七,有爵者五:杜爾祜、穆爾祜、特爾祜、杜努文、薩弼。

?厚貝勒杜爾祜,杜度第一子。初封輔國公。從太宗圍松山、錦州有功。坐事,降襲鎮國公。復以甲喇額真拜山等首告怨望,削爵,黜宗室。順治元年,從多鐸南征。二年,復宗室,封輔國公。?功,賜金五十、銀二千。五年,從濟爾哈朗徇湖廣。六年,敗敵永興,次辰州。進剿廣西,定全州。七年,賜銀六百。八年,進貝勒。十二年二月,卒,予謚。子敦達,襲貝子,謚恪恭。子孫遞降,以輔國公世襲。敦達八世孫光裕,襲輔國公。光緒二十六年,德意志等國兵入京師,死難,贈貝子銜,謚勤愍。

貝子穆爾祜,杜度第二子。天聰九年,師伐明,穆爾祜從貝勒多鐸率偏師入寧遠、錦州綴明師,抵大凌河,擊斬明將劉應選,追奔至松山,獲馬二百,克臺一,並有功。崇德元年,封輔國公。七年十月,與杜爾祜同得罪。順治元年,從多鐸南征,破李自成潼關,先後。二年,?拔兩營。賊犯我噶布什賢兵,穆爾祜擊敗之。又設伏山隘,賊自山上來襲,敗其復宗室,封三等鎮國將軍,三年,進一等。從多鐸征蘇尼特部騰機思等,敗之。四年,進輔國公。六年,從尼堪擊叛將姜瓖,進貝子。復從尼堪征湖南,賜蟒衣、鞍馬、弓矢。至衡州,尼堪戰歿。十一年,論前罪,削爵。卒,子長源,授鎮國將軍品級。子孫遞降至雲騎尉品級,爵除。

恪僖貝子特爾祜,杜度第三子。崇德四年,封輔國公。六年,從圍錦州,敗明兵於松山、杏山間。七年,移師駐塔山,克之。與杜爾祜同得罪。順治元年,從多爾袞入山海關,破李自成,逐之至慶都。復從多鐸敗自成潼關。二年,復宗室,封輔國公,賜金五十、銀二千。六年,進貝子。十五年,卒,予謚。子孫遞降,以奉恩將軍世襲。

懷愍貝子薩弼,杜度第七子。杜爾祜得罪,從坐,黜宗室。順治元年,從多爾袞入山海關,破李自成有功。二年,復宗室,封輔國公。三年,從勒克德渾南征,略荊州,屢破敵。師還,賜金五十、銀千。六年,從擊叛將姜瓖,戰朔州,敗瓖將姜之芬、孫乾、高奎等,移師攻寧武,瓖將劉偉等降,進貝子。十二年,卒,予謚。子固鼐,襲鎮國公,謚悼愍。子孫遞降,以鎮國將軍世襲。杜度諸子,惟第六子杜努文無戰功。順治初,封輔國公。卒。康熙三十七年,追封貝子,亦謚懷愍。子蘇努,初襲鎮國公。事聖祖,累進貝勒。雍正二年,坐與廉親王允禩為黨,削爵,黜宗室。

敬謹莊親王尼堪,褚英第三子。天命間,從伐多羅特、董夔諸部,有功。天聰九年,師伐明,從多鐸率偏師入錦、寧界綴明師。崇德元年,封貝子。上伐朝鮮,從多鐸逐朝鮮國王李倧至南漢山城,殲其援兵。四年,上伐明,從阿濟格等攻塔山、連山。七年,戍錦州。

順治元年四月,從多爾袞入山海關,敗李自成,復從阿濟格追擊至慶都,進貝勒。復從多鐸率師自孟津至陜州,破敵。二年,師次潼關,自成將劉方亮出御,尼堪與巴雅喇纛章京圖賴夾擊之,獲馬三百餘。又偕貝子尚善敗敵騎,趨歸德,定河南,詔慰勞,賜弓一。五月,從多鐸克明南都,追獲明福王由崧。又攻江陰,力戰,克之。師還,賜金二百、銀萬五千、鞍一、馬五。

三年,從豪格西征。時賀珍擾漢中,二隻虎、孫守法擾興安,?寇蜂起。尼堪次西安,自棧道進軍,珍自雞頭關迎拒,擊殲之,疾馳漢中躪其壘,賊走西鄉,追擊於楚湖,至漢陰,二隻虎奔四川,孫守法奔岳科寨。十一月,復從豪格入四川,斬張獻忠於西充。與貝子滿達海分兵定遵義、夔州、茂州、隆昌、富順、內江、資陽,四川平。五年,師還。偕阿濟格平天津土寇,進封敬謹郡王。六年,命為定西大將軍,討叛將姜瓖,屢敗敵。破瓖所置巡圍大同?撫姜輝,其將羅英壇以所部降。多爾袞赴大同招撫姜瓖,承制進尼堪親王。旋自左,瓖將楊振威等斬瓖以降,師還。七年,與巽親王滿達海、端重親王博洛理六部事。多爾袞遣尚書阿哈尼堪迎朝鮮王弟,阿哈尼堪啟尼堪以章京恩國泰代行,事覺,尼堪坐徇隱,降郡王。八年,復封親王。又坐不奏阿濟格私蓄兵器,降郡王。尋掌禮部。居數月,再復親王,掌宗人府事。

孫可望等犯湖南,命為定遠大將軍,率師討之。瀕行,賜御服、佩刀、鞍馬,上親送於南苑。李定國陷桂林,詔入廣西剿賊。十一月,師次湘潭,明將馬進忠等遁。師鄉衡州,噶布什賢兵擊敵衡山縣,敗敵兵千八百。尼堪督兵夜進,兼程至衡州。詰旦,師未陣,敵四萬餘猝至,尼堪督隊進擊,大破之,逐北二十餘里,獲象四、馬八百有奇。敵設伏林內,中途伏發,師欲退,尼堪曰:「我軍擊賊無退者。我為宗室,退,何面目歸乎?」奮勇直入,敵圍之數重,軍失道,尼堪督諸將縱橫沖擊,陷淖中,矢盡,拔刀戰,力竭,歿於陣。十年,喪歸,輟朝三日。命親王以下郊迎,予謚。是役也,從征諸將皆以陷師論罪。

第二子尼思哈,襲。順治十六年,追論尼堪取多爾袞身後遺財,及不劾尚書譚泰驕縱罪,以陣亡,留爵。十七年,卒,謚曰悼。第一子蘭布,襲貝勒。聖祖念尼堪以親王陣亡,進蘭布郡王,仍原號。七年,進親王。蘭布取鰲拜女,八年,鰲拜既得罪,蘭布坐降鎮國公。十三年,從尚善討吳三桂於湖南。十七年,卒於軍。十九年,追論退縮罪,削爵。子賴士,襲輔國公。乾隆四十三年,高宗以尼堪功著,力戰捐軀,進鎮國公,世襲。

禮烈親王代善,太祖第二子。初號貝勒。歲丁未,與舒爾哈齊、褚英徙東海瓦爾喀部,烏拉貝勒布占泰遣其將博克多將萬人要於路。代善見烏喇兵營山上,分兵?斐悠城新附之緣山奮擊,烏喇兵敗竄,代善馳逐博克多,自馬上左手攫其胄斬之。方雪甚寒,督戰益力,烏喇敗兵殭臥相屬,復得其將常柱、瑚哩布。師還,太祖嘉代善勇敢克敵,賜號古英巴圖魯。

歲癸丑,太祖伐烏喇,克遜扎搭、郭多、郭謨三城。布占泰將三萬人越富勒哈城而營,諸將欲戰,太祖猶持重,代善曰:「我師遠伐,利速戰,慮布占泰不出耳。出而不戰,將志在戰,復何猶豫。?謂之何?」太祖曰:「我豈怯戰?恐爾等有一二被傷,欲計萬全。今」因麾兵進,與烏喇步兵相距百步許,代善從太祖臨陣奮擊,大破之,克其城。烏喇兵潰走,代善追殪過半。布占泰奔葉赫,所屬城邑盡降,編戶萬家。天命元年,封和碩貝勒,以序稱大貝勒。

太祖始用兵於明,行二日,遇雨,太祖欲還,代善曰:「我師既入明境,遽引還,將復與修好乎?師既出,孰能諱之?且雨何害,適足以懈敵耳。」太祖從之。夜半雨霽,昧爽,圍撫順,明將李永芳以城降。東州、瑪哈丹二城及臺堡五百餘俱下。師還,出邊二十里,明將張承廕率兵來追。代善偕太宗還戰,復入邊,破其三營,斬承廕及其裨將頗廷相等。四年,命代善率諸將十六、兵五千,守扎喀關備明。尋引還。

三月,明經略楊鎬大舉來侵,遣總兵劉綎將四萬人出寬甸,杜松將六萬人出撫順,李如柏將六萬人出清河,馬林將四萬人出三岔口。太祖初聞明兵分出寬甸、撫順,以寬甸有備,親率師西御撫順明兵。代善將前軍,諜復告明兵出清河,代善曰:「清河道狹,且崎嶇,不利速行,我當御其自撫順來者。」過扎喀關,太宗以祀事後至,言界凡方築城,民應役,之。代善引兵自太蘭岡趨界凡,與築城役屯吉林崖。杜松以二萬人來攻,別軍陣薩爾?宜急滸山。代善與貝勒阿敏、莽古爾泰及諸將議以千人助吉林崖軍,使陟山下擊,餘軍張兩翼,右應吉林崖,左當薩爾滸。太祖至,以右翼兵益左翼,先趨薩爾滸。明兵出,我兵仰射,不移時破其壘。吉林崖軍自山馳而下,右翼渡河夾擊,破明兵,斬松等。馬林出三岔口,以三萬人軍於尚間崖,監軍道潘宗顏將萬人軍於飛芬山,松後部龔念遂、李希泌軍於斡琿鄂謨,太祖督兵攻之。代善將三百騎馳尚間崖,見明兵結方營,掘壕三匝,以火器居前,騎兵繼之,嚴陣而待,遣騎告太祖。太祖已擊破念遂等,親至尚間崖,令於軍,皆下馬步戰。未畢下,明兵突至,代善躍馬入陣,師奮進,斬獲過半。翌日,代善以二十騎先還,詗南路敵遠近。太祖亦還,聞劉綎兵深入,命代善率先至諸軍御之。出瓦爾喀什,綎已至阿布達哩岡,太宗率右翼陟山,代善率左翼出其西,夾擊,明兵大潰,斬綎。鎬所遣諸軍盡敗。

七月,從太祖克鐵嶺。八月,太祖伐葉赫。葉赫有二城:金臺石居其東,布揚古居其西。師至,太祖攻東城,代善攻西城。東城下,布揚古及其弟布爾杭古乞盟,代善諭而降之。復偕莽古爾泰遷金州民於復州。?。六年三月,從太祖伐明沈陽,率其子岳託戰,斬馘甚

十一年八月,太祖崩,岳託與其弟薩哈璘告代善,請奉太宗嗣位,代善曰:「是吾心也!」告諸貝勒定策。太宗辭讓再三,代善等請益堅,乃即位。是冬,伐蒙古喀爾喀扎魯特部,擒貝勒巴克等,斬鄂爾齋圖,俘所屬而歸。

天聰元年,從太宗圍錦州,拒明山海關援兵,薄寧遠,破敵,以暑還師。三年,從伐明,入洪山口,克遵化,薄明都,明總兵滿桂等赴援,擊敗之德勝門外,克良鄉,又破明兵永定門外。從上閱薊州形勢,明步兵五千自山海關至,與師遇,不及陣,列車楯、槍?而營,代善率左翼四旗擊破之。四年正月,明侍郎劉之綸率兵至遵化,營山上,代善環山圍之,破其七營,之綸走入山,射殺之。五年八月,從上圍大凌河,收城外臺堡。九月,明總兵吳襄、監軍道張春等將四萬人自錦州至,距大凌河十五里,代善從上將二萬人擊之,明兵方陣,發槍?,督騎兵突入,矢如雨,明兵大?。襄遁,春收潰兵復陣。黑雲起,風自西來,明兵,師乘之,獲春等。春見上不屈,上將?乘風縱火逼我軍。大雨反風,毀其營,明兵死者甚誅之,代善諫,乃赦之。

初,太祖命四和碩貝勒分直理政事,每御殿,和碩貝勒皆列坐。至是,禮部參政李伯龍請定朝會班制。時和碩貝勒阿敏已得罪,莽古爾泰亦以罪降多羅貝勒,諸貝勒議不得列坐。代善曰:「奚獨莽古爾泰?上居大位,我亦不當並列。自今請上南面,我與莽古爾泰侍坐於側,諸貝勒坐於下。」

六年四月,從上伐察哈爾,過興安嶺,聞林丹汗遠遁,移師攻歸化城,趨大同、宣府,出塞,與沙河堡、得勝堡、張家口諸守將議和而還。八年五月,從伐明,出榆林口,至宣府邊外,分兵自喀喇鄂博克得勝堡,遂自朔州趨馬邑,會師大同而還。

崇德元年,封和碩兄禮親王。冬,從上伐朝鮮。二年,有司論王克朝鮮,違旨以所獲溢額,上曰:「朕於兄禮親王敬愛有加,何不體朕意若是?」又曰:「?糧米飼馬及選用護王等事朕雖致恭敬,朕何所喜?必正身行義以相輔佐,朕始嘉賴焉。」四年十一月,從上獵於葉赫,射麞,馬僕,傷足。上下馬為裹創,酌金?勞之,因泣下曰:「朕以兄年高不可馳馬,兄奈何不自愛?」罷獵,還,命乘輿緩行,日十餘里,護以歸。

八年,太宗崩,世祖即位。王集諸王、貝勒、大臣議,以鄭親王濟爾哈朗、睿親王多爾袞輔政。又發貝子碩託、郡王阿達禮私議立睿親王,下法司,誅之。碩託,王次子;阿達禮,薩哈璘子,王孫也。順治元年正月朔,命上殿毋拜,著為例。二年春,至京師。五年十月,薨,年六十六。賜祭葬,立碑紀功。康熙十年,追謚。乾隆四十三年,配饗太廟。

代善子八,有爵者七:岳託、碩託、薩哈璘、瓦克達、瑪占、滿達海、祜塞。祜塞,初封鎮國公,追封惠順親王,而滿達海襲爵。

巽簡親王滿達海,代善第七子。崇德五年,從圍錦州。六年,封輔國公。從肅親王豪格圍松山,破敵。洪承疇赴援,戰,所乘馬創,豪格呼曰:「馬創矣!亟易馬!」明兵大至,力戰,殿而還。明總兵吳三桂倚山為營,滿達海合諸軍擊破之,三桂宵遁。七年,從濟爾哈朗克塔山。八年,授都察院承政。

順治元年,從入關,敗李自成,進貝子。復從英親王阿濟格逐自成趨綏德。二年,克沿邊三城及延安,自成遁湖廣,師還。三年,從豪格討張獻忠,自漢中進秦州,降獻忠將高如礪。師次西充,擊斬獻忠,與尼堪分剿餘賊。五年,師還。坐徇巴牙喇纛章京希爾根冒功,議罰銀,睿親王多爾袞令免之。六年,襲爵。降將姜瓖叛大同,滿達海與郡王瓦克達率師討之,尋授征西大將軍。克朔州、馬邑、寧武關、寧化所、八角堡、靜樂縣,遂與博洛會師,復汾州。瓖誅,大同平。遣兵圍平遙、太谷、遼沁,先後克之。屯留、襄垣、榆社、武鄉諸縣俱下。睿親王多爾袞令留瓦克達剿餘寇,滿達海還京師。

八年,世祖親政,改封號曰巽親王。諸王分治部務,滿達海掌吏部。九年二月,薨,予謚。十六年,追論滿達海於奏削多爾袞封爵後,奪其財物;掌吏部,懼譚泰驕縱,未論劾:削謚僕碑,降爵為貝勒。

子常阿岱,初襲親王。降貝勒。康熙四年,薨,謚懷愍。子星尼,襲貝子,再襲輔國公。星尼子星海,襲鎮國公。並坐事奪爵。乾隆四十三年,追錄滿達海功,命星海孫福色鏗額以輔國將軍世襲。常阿岱既降爵,以從弟傑書襲親王。

康良親王傑書,祜塞第三子。初襲封郡王。順治八年,加號曰康。十六年,襲爵,遂改號康親王。康熙十三年六月,命為奉命大將軍,率師討耿精忠。師至金華,溫州、處州已陷。精忠將徐尚朝以五萬人犯金華,王令都統巴雅爾、副都統瑪哈達迎擊,破之。尚朝復來犯,巴雅爾會總兵陳世凱破賊壘積道山,殲二萬餘,復永康、縉雲。精忠將沙有祥踞桃花嶺,梗處州道,瑪哈達率軍擊之,有祥潰走。十四年,復處州及仙居。尚朝等猶踞宣平、松陽,屢窺處州。都統拉哈達偕諸將御之,破賊於石塘,於石佛嶺,於大王嶺東隴隘口上套寨、下五塘諸地。詔寧海將軍傅喇塔自黃巖規溫州,趣傑書自衢州入,傑書疏言:「處州有警,兵單不能驟進。」上諭曰:「王守金華,將及二載,徒以文移往來,不親統兵規剿,賊何自滅?宜刻期進取。」

十五年,自金華移師衢州,精忠將馬九玉屯大溪灘拒師。傑書督諸將力擊之,伏起,兵負扉為蔽,傑書談笑自若,諸軍皆踴躍奮?刃相接。傑書坐古廟側指揮,纛為火器所穿,擊,精忠兵大敗,溪水為赤。傑書令偃旗鼓,一日夜行數百里,乘月攻克江山,進徇常山,次仙霞關。精忠將金應虎收舟泊隔岸,師不得渡。令循灘西上,視水淺亂流,涉。精忠兵不戰,潰,應虎降。進拔浦城,檄精忠諭降。師復進,拔建陽,撫定建寧、延平二府。精忠遣其子顯祚迎師,傑書承制許以不死,精忠出降。十月,師入福州,精忠請從師討鄭錦自贖,入告,詔許之。

錦將許耀以三萬人屯烏龍江南小門山、真鳳山,傑書遣拉哈達等擊走之。疏言:「精忠從師出剿,其弟昭忠、聚忠,宜留一人於福州,轄其屬。」又言:「福建制兵已設如額,精忠所率兵不少,左右兩鎮兵可並裁去。溫州總兵祖弘勛、籓下總兵曾養性,宜別除授。」上命昭忠為鎮平將軍,駐福州,餘並如所請。傑書遣兵敗錦將吳淑于浦塘,復邵武。師復進,泰寧、汀州及所屬諸縣皆下。十六年,拉哈達敗錦軍於白茅山、太平山,破二十六壘,克興化,復泉州、漳州。奏入,詔褒傑書功。傑書令拉哈達等率兵與精忠進次潮州,規廣東。錦兵陷平和,逼海澄,副都統穆赫林等守御越七旬,援不至,與長泰並陷。傑書請罪,詔俟師還議之。錦兵復破同安、惠安,傑書遣軍討復之,並復長泰,破敵於柯鏗山、萬松關,又寨。十八年,戰郭塘、歐溪頭,屢破敵。敵犯江東橋,擊?之。副?遣別將破敵江東橋、石都統吉勒塔布敗敵鰲頭山,沃申克東石城。十九年,沃申撫定大定、小定、玉洲、石馬諸地,克海澄。水師提督萬正色克海壇,拉哈達等克?門、金門,都統賚塔克銅山。錦以殘兵還臺灣。

精忠既降,復有異志,傑書疏請逮治。上令傑書諷精忠請入覲,亦召傑書師還,留八旗兵三千分守福州、泉州、漳州。十月,至京師,上率王大臣至盧溝橋迎勞之。二十一年,追論金華頓兵及遲援海澄罪,奪軍功,罰俸一年。二十九年,率兵出張家口,屯歸化城,備噶爾丹。三十六年閏三月,薨,予謚。

子椿泰,襲。椿泰豁達大度,遇下以寬。善舞六合槍,手法矯捷,敵十數人。四十八年,薨,謚曰悼。

子崇安,襲。雍正間,官都統,掌宗人府。九年,率兵駐歸化,備噶爾丹。尋命護撫遠大將軍印,召還,十一年,薨,謚曰修。傑書子巴爾圖,襲。乾隆十八年,薨,年八十,謚曰簡。

崇安子永恩,襲。四十三年,復號禮親王。永恩性寬易而持己嚴,襲爵垂五十年,淡泊勤儉,出處有恆。嘉慶十年,薨,謚曰恭。

子昭梿,襲。昭梿好學,自號汲修主人,尤習國故。二十一年,坐陵辱大臣,濫用非刑,奪爵,圈禁。二十二年,命釋之。從弟麟趾,襲,父永諲,永恩弟也。亦嗜文學,能詩。追封禮親王。麟趾,道光元年,薨,謚曰安。孫全齡,襲,父錫春,追封禮親王。全齡,三十年,薨,謚曰和。

子世鐸,襲。同治間,授內大臣、右宗正。光緒十年,恭親王奕罷政,太后諮醇親王奕枻諸王孰可任,舉世鐸對。乃命在軍機大臣上行走,並詔緊要事件會同奕枻商辦。德宗。二十年,太后萬壽,賜親?親政,世鐸請解軍機大臣,奉太后旨,不許。十九年,命增護。二十六年,上奉太后西巡,世鐸不及從。召赴行在,復以病未至。二十?王雙俸,再增護七年七月,罷直,授御前大臣。遜位後三年,薨,謚曰恪。子誠厚,襲。薨,謚曰敦。

克勤郡王岳託,代善第一子。初授臺吉。天命六年,師略奉集堡,將還,諜告明軍所在,岳託偕臺吉德格類擊敗之。上克沈陽,明總兵李秉誠引退,師從之,至白塔鋪。岳託後至,逐北四十里,殲明兵三千餘。喀爾喀扎魯特貝勒昂安執我使送葉赫,被殺。八年,岳託同臺吉阿巴泰討之,斬昂安及其子。十一年,復從代善伐扎魯特,斬其部長鄂爾齋圖,俘其。封貝勒。?

天聰元年,偕貝勒阿敏、濟爾哈朗伐朝鮮,克義州、定州、漢山三城。渡嘉山江,克安州,次平壤,其守將棄城走。再進,次中和,諭朝鮮國王李倧降。阿敏欲直攻王京,岳託密與濟爾哈朗議駐平山,再使諭倧。倧原歲貢方物,岳託謀曰:「吾曹事已集,蒙古與明皆吾敵,設有警,可不為備乎?宜與盟而歸。」既盟,告阿敏。阿敏以未與盟,縱兵掠。岳託曰:「盟成而掠,非義也。」勸之不可。復令倧弟覺與盟,乃還師。

從上伐明,又從圍寧遠,並有功。復敗明兵於牛莊。二年,略明邊,隳錦州、杏山、高橋三城。自十三站以東,毀堠二十一,殺守者三十人。師還,上迎勞,賜良馬一。三年,略明錦州、寧遠,焚其積聚。上伐明,岳託與濟爾哈朗率右翼兵夜攻大安口,毀水門入,敗馬蘭營援兵於城下。及旦,見明兵營山上,分兵授濟爾哈朗擊之,岳託駐山下以待。復見明兵自遵化來援,顧濟爾哈朗曰:「我當擊此。」五戰皆捷。尋次順義,擊破明總兵滿桂等。薄明都,復從代善擊敗援兵。偕貝勒薩哈璘圍永平,克香河。四年,還守沈陽。

五年三月,詔詢諸貝勒:「國人怨斷獄不公,何以弭之?」岳託奏:「刑罰舛謬,實在臣等。請上擢直臣,近忠良,絕讒佞,行黜陟之典,使諸臣知激勸。」是歲初設六部,命掌兵部事。上攻大凌河,趨廣寧,岳託偕貝勒阿濟格率兵二萬別自義州進,與師會。固山額真葉臣圍城西南,岳託為之應。祖大壽請降,以子可法質。可法見諸貝勒,將拜,岳託曰:「戰則仇敵,和則弟兄,何拜為?」因問何為死守空城,曰:「畏屠戮耳!」岳託善諭之,遣歸。越三日,大壽乃降。上議取錦州,命偕諸貝勒統兵四千,易漢服,偕大壽作潰奔狀,夜襲錦州。會大霧,乃止。

六年正月,岳託奏:「前克遼東、廣寧,漢人拒命者誅之,後復屠灤州、永平,是以,歸順者必多。

?人懷疑懼。今天與我大凌河,正欲使天下知我善撫民也。臣愚以為善撫此當先予以室家,出公帑以贍之。倘蒙天眷,奄有其地,仍還其家產,彼必悅服。又各官宜令諸貝勒給莊一區,每牛錄令取漢男婦二人、牛一頭,編為屯,人給二屯。出牛口之家,各牛錄復以官值償之。至明諸將士棄其鄉土,窮年戍守,畏我誅戮。今慕義歸降,善為撫恤,毋令失所,則人心附,大業成矣。」疏入,上嘉納之。

尋偕濟爾哈朗等略察哈爾部,至歸化城,俘獲以千計。又偕貝勒德格類行略地,自耀州至蓋州南。七年,又偕德格類等攻旅順口,留兵駐守。師還,上郊勞,以金?酌酒賜之。八年,上閱兵沈陽,岳託率滿洲、蒙古十一旗兵,列陣二十里許,軍容整肅,上嘉之。從上徵察哈爾,有疾先還。九年,略明山西,岳託復以病留歸化城。土默特部來告,博碩克圖汗子俄木布遣人偕阿嚕喀爾喀及明使者至,將謀我。岳託伏兵邀之,擒明使者,令土默特捕斬阿嚕喀爾喀匿馬駝者。部分土默特壯丁,立隊伍,授條約。尋與諸貝勒會師,偕還。

崇德元年四月,封成親王。八月,坐徇庇莽古爾泰、碩託,及離間濟爾哈朗、豪格,論死,上寬之,降貝勒,罷兵部。未幾,復命攝部事。二年八月,上命兩翼較射,岳託言不能執弓,上勉之再三,始引弓,弓墮地者五,乃擲去。諸王論岳託驕慢,當死,上再寬之,降貝子,罰銀五千。

三年,復貝勒。從上征喀爾喀,至博碩堆,知扎薩克圖汗已出走,乃還。八月,伐明,授岳託揚武大將軍,貝勒杜度副之,統右翼軍;統左翼者睿親王多爾袞也。至墻子嶺,明兵入堡,外為三寨,我師克之。堡堅不易拔,用俘卒言嶺東西有間道,分兵攻其前,綴明師,潛從間道逾嶺入,克臺十有一。師深入,徇山東,下濟南,岳託薨於軍。四年,多爾袞奏捷,無嶽託名。上驚問,始聞喪,大慟,輟膳,命毋使禮親王知。喪還,上至沙嶺遙奠;還宮,輟朝三日。詔封為克勤郡王,賜駝五、馬二、銀萬。康熙二十七年,立碑紀功。乾隆四十三年,配享太廟。

岳託子七,有爵者五:羅洛渾、喀爾楚渾、巴爾楚渾、巴思哈、祜里布。巴爾楚渾、祜里布並恩封貝勒,巴爾楚渾謚和惠,祜里布謚剛毅。

衍禧介郡王羅洛渾,岳託第一子。初襲貝勒。崇德五年,迎蒙古多羅特部蘇班岱、阿爾巴岱於杏山,遇明兵,搏戰破之,賜御?良馬一。尋圍錦州。復從伐明,克松山,賜蟒緞。八年,坐嗜酒妄議,敏惠恭和元妃喪不輟絲竹,削爵。旋復封,命濟爾哈朗、多爾袞戒諭之。順治元年,從定京師,進衍禧郡王。三年,偕肅親王豪格征四川,薨於軍。康熙間,追謚。

子羅科鐸,襲。八年,改封號曰平郡王。十五年,從信郡王多尼徇雲南,屢破明將李定國、白文選。十六年,賜蟒衣、弓刀、鞍馬,旌其勞。康熙二十一年,薨,謚曰比。子納爾圖,襲。二十六年,以毆斃無罪人及折人手足,削爵。弟納爾福,襲。四十年,薨,謚曰悼。子納爾蘇,襲。五十七年,從撫遠大將軍允昷收西藏,駐博羅和碩,尋移古木。六十年,攝大將軍事。雍正元年,還京。四年,坐貪婪,削爵。子福彭,襲。

平敏郡王福彭既襲爵,授右宗正,署都統。十一年,命軍機處行走。授定邊大將軍,率師討噶爾丹策零。師次烏里雅蘇臺,奏言:「行軍,駝馬為先。今喀爾喀扎薩克貝勒等遠獻駝馬,力請停償直。彼不私其所有,而宗室王、公、貝勒皆有馬,豈不內媿於心?臣有馬五百,原送軍前備用。」十二年,率將軍傅爾丹赴科布多護北路諸軍。尋召還。十三年,復命率師駐鄂爾坤,築城額爾德尼昭之北。尋以慶復代,召還。乾隆初,歷正白、正黃二旗滿洲都統。十三年,薨,予謚。

子慶寧,襲。十五年,薨,謚曰僖。無子。以納爾蘇孫慶恆襲,授右宗正。坐旗員冒借官銀,降貝子。四十年,復王爵。四十三年,復號克勤郡王。四十四年,薨,謚曰良。以訥爾圖孫雅朗阿襲。五十九年,薨,謚曰莊。子恆謹,襲。嘉慶四年,以不避皇后乘輿,奪爵。以弟恆元子尚格襲。恆元追封克勤郡王。尚格,道光四年以病乞休,十三年,薨,謚曰簡。子承碩,襲,十九年,薨,謚曰恪。

子慶惠,襲。咸豐八年,授正黃旗漢軍都統。十年,上幸熱河,命留京辦事。英國兵熸圓明園,其將巴夏禮先為我師所擒,慶惠釋之,疏請恭親王奕入城議撫。十一年,薨,內大臣。德宗大婚,加親王銜。孝?謚曰敬。子晉祺,襲。歷左宗人、右宗正、都統、領侍欽皇后萬壽,賜四團龍補服,並歲加銀二千。二十六年,薨,謚曰誠。子崧傑,襲,宣統二年,薨,謚曰順。子晏森,襲。

顯榮貝勒喀爾楚渾,岳託第三子。順治元年,從多爾袞擊李自成於山海關。二年,封降,豪格?鎮國公。三年,從豪格討張獻忠,偕貝子滿達海率師進剿。獻忠將高如礪等率殲獻忠,喀爾楚渾在事有功。五年,授都統。六年,從尼堪討叛將姜瓖,圍寧武,破敵,進貝勒。八年,攝理籓院事。卒,予謚顯榮。子克齊,方一歲,襲爵,歷七十一年卒,年七十二。子魯賓,初封貝子。事聖祖,授左宗正。從征噶爾丹,罷宗正。雍正元年,襲爵。四年,坐狂悖,削爵。復封輔國公。乾隆八年,卒,年七十四,謚恪思。子孫以奉恩將軍世襲。

鎮國將軍品級巴思哈,岳託第五子。崇德四年,封鎮國將軍。順治六年,進貝勒。九年,從尼堪征湖南,賜蟒衣、鞍馬、弓矢。尼堪戰死衡州,屯齊代為定遠大將軍,巴思哈與合軍自永州趨寶慶,敗敵周家坡。十一年,追論尼堪敗績失援罪,削爵。十二年,授都統。尋授鎮國公品級。十五年,從多尼下雲南。師次貴州,破敵。十六年,薄雲南會城,同貝勒尚善克鎮南州玉龍關、永昌府騰越州,賜蟒袍、鞍馬。十七年,師還。追議在永昌縱兵擾民,降鎮國將軍品級。十八年,卒。

碩託,代善第二子。初授臺吉。天命六年,從伐明,攻奉集堡。十年,偕貝勒莽古爾泰援科爾沁。十一年,從代善伐喀爾喀巴林部,又伐扎嚕特部,皆有功,授貝勒。天聰元年,從貝勒阿敏等伐朝鮮。又從上伐大凌河,圍錦州。四年,師克永平,偕阿敏駐守。阿敏引還,碩託坐削爵。五年,從攻錦州,明兵攻阿濟格營,碩託力戰,傷於股,上親酌金?勞之。明兵趨大凌河,碩託擊敗張春,復傷於手。?勞,賜採緞十、布百。八年,從代善自喀喇鄂博攻得勝堡,克之。又擊敗朔州騎兵。偕薩哈璘略代州,拔崞縣,分克原平驛。尋封貝子。崇德元年,從伐朝鮮,圍南漢山城,敗援兵二萬餘。二年,偕阿濟格攻克皮島。三年,偕濟爾哈朗攻寧遠。四年,坐僭上越分,降輔國公。偕阿爾格伐明,俘獲無算,論功,賜駝、馬各一。五年六月,從多爾袞圍錦州。坐離城久駐,又遣卒私歸,議削爵。上讓之曰:「爾罪多矣!朕屢宥,爾屢犯,若不關己者。後當任法司治之,不汝宥也!」改罰銀千。尋復封貝子。太宗崩,碩託與阿達禮謀立睿親王多爾袞,譴死,黜宗室。

穎毅親王薩哈璘,代善第三子。初授臺吉。天命十年,察哈爾林丹汗攻科爾沁,薩哈璘將精騎五千赴援,解其圍。十一年,從代善伐喀爾喀巴林部,又伐扎嚕特部,皆有功,授貝勒。天聰元年,上伐明,率巴雅喇精騎為前隊。上自大凌河至錦州,明兵走,薩哈璘邀擊塔山糧運,敗明兵二萬人。攻寧遠,擊明總兵滿桂,薩哈璘力戰,被創。?殲之。復率偏師三年,上伐明,次波羅河屯。代善等密請班師,上不懌。薩哈璘與岳託力贊進取,由是克遵化,薄明都。十二月,薩哈璘略通州,焚其舟,次張家灣。復圍永平,克香河。四年,永平言將屠城,斬以徇。旋諭降遷安、灤州、建?既下,薩哈璘與濟爾哈朗駐守。永平人李春旺昌、臺頭營、鞍山堡諸地。明兵自樂亭、撫寧攻灤州,薩哈璘率軍赴援,明兵引退。貝勒阿敏來代,乃還師。

五年,詔諸貝勒指陳時政,薩哈璘言:「圖治在人。人主灼知邪正,則臣下爭尚名節,惟皇上慎簡庶僚,任以政事。遇大征伐,上親在行間,諸臣皆秉方略。若遣軍,宜選賢能者為帥,給符節,畀事權,仍限某官以下乾軍令,許軍法從事。」初設六部,掌禮部事。六年,略歸化城,俘蒙古千餘。指授蒙古諸貝勒牧地,申約法。

七年六月,詔問征明及察哈爾、朝鮮三者何先,薩哈璘言:「當寬朝鮮,拒察哈爾,而專征明。察哈爾雖不加兵,如蟲食穴中,勢且自盡。至於明,我少緩,則彼守益固。臣意視今歲秋成圖進取,乘彼禾稼方熟,因糧於敵,為再進計。量留兵防察哈爾。先以騎兵往來襲擊蹂躪,再簡精兵自一片石入山海關,則寧、錦為無用;或仍自寧、錦入,斷北京四路,度地形,據糧足之地。乘機伺便,二三年中,大勛集矣。」尋略山海關。八年,偕多爾袞迎降將尚可喜,招撫廣鹿、長山二島戶口三千八百有奇。從伐明,薩哈璘自喀喇鄂博攻克得勝堡。略代州,夜襲崞縣,拔之。王東、板鎮二堡民棄堡遁。復擊敗代州兵。會上大同,籍俘獲以聞。

九年,偕多爾袞、岳託、豪格等收察哈爾林丹汗子額爾克孔果爾額哲,師次托裏圖,收其全部。師還,岳託駐歸化城。薩哈璘偕多爾袞、豪格入明邊,略山西。事詳多爾袞傳。諸貝勒大臣屢請上尊號,不許。既收察哈爾,復請,上仍不許。薩哈璘令內院大臣希福等奏曰:「臣等屢請,未蒙鑒允,夙夜悚惶,罔知所措。伏思皇上不受尊號,咎在諸貝勒不能殫竭忠信,展布嘉猷,為久大計。今諸貝勒誓改行竭忠,輔開太平之基,皇上宜受尊號。」上曰:「善。薩哈璘為朕謀,開陳及此,實獲我心。諸貝勒應誓與否,爾掌禮部,可自主之議告朝鮮,薩哈璘因言:「諸貝勒?。」翌日,薩哈璘集諸貝勒於朝,書誓詞以進。上命以亦當遣使,示以各國來附,兵力強盛。」上嘉納之。

崇德元年正月,薩哈璘有疾,上命希福諭曰:「?子弟中,整理治道,啟我所不及,助我所不能,惟爾之賴。爾其靜心調攝,以副朕望!」薩哈璘對曰:「蒙皇上溫旨眷顧,竊冀仰荷恩育,或可得生。即不幸先填溝壑,亦復何憾。但當大勛垂集,不能盡力國家,乃展轉?蓐,為可恨耳!」希福還奏,上惻然曰:「國家豈有專事甲兵以為治理者?倘疆土日闢,克成大業,而明哲先萎,孰能助朕為理乎?」病革,屢臨視,見其羸瘠,淚下,薩哈璘亦悲痛不自勝。五月,卒。上震悼,入哭者四,自辰至午乃還。仍於庭中設幄坐,不御飲食,輟朝三日。祭時,上親奠,痛哭。詔褒薩哈璘明達敏贍,通滿、漢、蒙古文義,多所贊助,追封穎親王。上御翔鳳樓,偶假寐,夢人請曰:「穎親王乞賜牛一。」故事,親王薨,初祭以牛。薩哈璘以追封,未用,上命致祭如禮。康熙十年,追謚。

薩哈璘子三:阿達禮、勒克德渾、杜蘭。杜蘭,恩封貝勒,坐事,降鎮國公。

阿達禮,薩哈璘第一子。襲郡王。崇德三年,從伐喀爾喀。五年五月,偕濟爾哈朗駐義州,迎來歸蒙古多羅特部,明錦州杏山、松山兵出拒,擊敗之。師還,賜御?良馬一。六年,圍錦州,降城中蒙古臺吉諾木齊、吳巴什等,敗明援兵於錦州南山西岡。明兵復自松山。圍松山,明兵來犯,擊敗之,斬千四百餘級。七年,明?沿海進援,我兵薄城下,擊殲其將夏承德約內應,夜半,我軍梯登,遂克松山。?功,賜鞍馬一、蟒緞九十。尋管禮部,與議政。先是,上御篤恭殿,王以下皆侍立,碩託奏定儀制,上御殿及賜宴,親王以下皆跪迎,上升階方起,駕還清寧宮亦如之。貝勒阿巴泰伐明薊州,偕多鐸屯寧遠綴明師。八年,太宗崩,坐與碩託謀立睿親王,譴死。

順承恭惠郡王勒克德渾,薩哈璘第二子。阿達禮譴死,緣坐,黜宗室。順治元年,復宗室,封貝勒。二年,命為平南大將軍,代豫親王多鐸駐江寧。時明魯王以海據浙東稱監國,其大學士馬士英等率兵渡錢塘江窺杭州,勒克德渾遣兵擊?之。復遣梅勒額真珠瑪喇擊士英餘杭,和託擊明總兵方國安富陽,兩軍合營杭州城三十里外。士英、國安復率兵渡江,又為梅勒額真濟席哈所敗,溺死者無算。十一月,明唐王聿鍵所置湖廣總督何騰蛟招李自成餘部,分據諸府縣,命勒克德渾偕鎮國將軍鞏阿岱率師討之。三年正月,師次武昌,遣護軍統領博爾輝等督兵進擊,戰臨湘,殲敵千餘。次岳州,降明將黑運昌。至石首,敵渡江犯荊州,遣尚書覺羅郎球等以偏師出南岸,伺敵渡,狙擊之。師乘夜疾馳,詰旦抵城下,。薄暮,郎球等亦盡奪敵舟以歸。翌日,分遣奉國將軍巴?分兩翼躪敵營,大破之,斬獲甚布泰等逐敵,自安遠、南漳、喜?山、關王嶺至襄陽,擊斬殆盡。次彞陵,自成弟孜及諸將、牛萬二千餘。捷聞,?田見秀、張耐、李佑、吳汝義等率馬步兵五千,詣軍前降,獲馬、優詔班師,賜金百、銀二千。五年九月,進封順承郡王。尋偕鄭親王濟爾哈朗督兵攻湘潭,拔之,擒騰蛟。移師入廣西,攻全州。破趙廉,克永安關。逐土寇曹槓子,又敗之於道州。七年,師還,賜金五十、銀五千。八年,掌刑部事。九年三月,薨。康熙十年,追謚。

子勒爾錦,襲。康熙十一年,掌宗人府事。十二年,吳三桂反,命為寧南靖寇大將軍,率師討之。十三年,駐荊州。三桂兵陷沅州、常德,分兵抵巴東,逼襄陽,遣都統鄂內率兵防守。三月,三桂將劉之復率舟師犯彞陵,夾江立五營,遣護軍統領額司泰等水陸並擊,大敗之。四月,三桂將陶繼智復自宜都來犯,又敗之。七月,敗三桂將吳應麒等。十四年五月,三桂兵犯均州,遣都統伊里布擊敗之。六月,叛將楊來嘉來犯,列陣山巔,自山溝下斷舟多,請益戰艦以斷運道。」上從?我師道,師擊之,斬三千餘級。疏言:「敵逼彞陵,兵之。七月,三桂將王會等合來嘉犯南漳,遣伊里布與總督蔡毓榮會師擊之。八月,疏言:「賊立壘掘塹,騎兵不能沖突。當簡綠旗步兵,造輕箭簾車、?車並進,填壕發?,繼以滿洲兵,庶可滅賊。」上復從之。十月,復興山。十二月,請發禁旅益師,上責其遷延。十五年,自荊州渡江,破敵於文村、於石首,復戰太平街,師敗績,退保荊州。九月,遣副都統塞格復鄖西。十八年,設隨征四營,轄新增兵萬二千。

三桂既死,復渡江克松滋、枝江、宜都及澧州,進取常德,敵焚廬舍、舟監先遁,所置巡撫李益陽、按察院陳寶鑰等降。遣兵至青石渡,吳世璠將潘龍迎戰。師左右夾擊,追至。復衡山。攻歸州,敗世璠將廖進忠於馬黃山,追至西?平峪鋪,斬馘無算,敵墮崖死者甚壤,復歸州、巴東。十九年,詔趣取重慶。疏請留將軍噶爾漢於荊州,親率師赴重慶。中途引還,具疏自劾,請解大將軍任,赴沅州軍自?,上責令還京師。下吏議,以老師糜餉,坐失事機,削爵。子勒爾貝,襲。二十一年,薨。弟揚奇,襲。二十六年,薨。弟充保,襲。三十七年,薨。弟布穆巴,襲。五十四年,坐以御賜鞍馬給優人,削爵。以從父諾羅布襲。

。累官至杭州將軍。襲爵。五十六年,薨,?諾羅布,勒克德渾第三子。初授頭等侍謚曰忠。

子錫保,嗣。雍正三年,掌宗人府事,在內廷行走。四年,諭曰:「順承郡王錫保才具優長,乃國家實心效力之賢王,可給與親王俸。」授都統。坐徇貝勒延信罪不舉劾,又逮治遲誤,奪親王俸,降左宗正。七年三月,師討噶爾丹策零,命錫保署振武將軍印,駐軍阿爾臺。九年,上以錫保治軍勤勞,進封順承親王,命守察罕叟爾。噶爾丹策零遣其將大策零敦多卜、小策零敦多卜、多爾濟丹巴入犯科布多,次克嚕倫,侵掠喀爾喀游牧。蒙古親王策棱等合師邀擊,遣臺吉巴海夜入大策零敦多卜營挑戰,擊斬其將喀喇巴圖魯,大策零敦多卜等自哈布塔克拜達克遁歸。錫保疏報,得旨嘉?。十一月,授靖遠大將軍。十年七月,策棱等敗敵額爾德尼昭。十一年,疏請城烏里雅蘇臺,從之。尋以噶爾丹策零兵越克爾森齊老,不赴援,罷大將軍,削爵。

子熙良,初封世子。以錫保罪,並奪。尋命襲郡王。乾隆九年,薨,謚曰恪。子泰斐英阿,襲。授都統、左宗正。二十一年,薨,謚曰恭。子恆昌,襲。四十三年,薨,謚曰慎。子倫柱,襲。道光三年,薨,謚曰簡。子春山,襲。咸豐四年,薨,謚曰勤。子慶恩,襲。穆宗大婚,賜三眼孔雀翎。光緒七年,薨,謚曰敏。子訥勒赫,襲。德宗大婚,賜食全俸。孝欽皇后萬壽,歲加銀二千。遜位後,薨,謚曰質。

謙襄郡王瓦克達,代善第四子。天聰元年,師攻寧遠,擊敗明總兵滿桂,瓦克達力戰,被創。崇德五年,從多爾袞圍錦州,敵兵樵採,瓦克達以十餘騎擊斬之。六年,洪承疇以十三萬人援錦州,次松山,敵騎來奪我紅衣?,瓦克達偕滿達海戰?之,天雨,復戰,又敗之。進擊承疇步兵,噶布什賢什長費雅思哈失馬,瓦克達與累騎而出。甲喇章京哈寧阿墜馬,創甚,敵圍之數重,瓦克達入其陣,挈以歸。碩託譴死,緣坐,黜宗室。

順治元年,從多爾袞入山海關,追擊李自成至慶都。復從阿濟格自邊外趨綏德。二年,?,自成遁湖廣,躡至安陸。賊方乘船遁,瓦克達偕巴牙喇纛章京鰲拜涉水登岸,射殪賊奪其船以濟大軍。三年,?功,復宗室,援三等鎮國將軍。從多鐸剿蘇尼特部騰機思、騰機特等,至圖拉河,斬騰機思孫三、騰機特子二,及喀爾喀臺吉十一,並獲其輜重。至布爾哈圖山,復與貝子博和託合軍,進斬千餘級,俘八百餘人,獲駝、馬、牛、羊無算。又擊敗喀爾喀土謝圖汗兵。四年,進封鎮國公。

五年,上念宗室貧乏,瓦克達賜銀六千,進封郡王。喀爾喀部二楚虎爾擾邊,從阿濟格防大同。復從討叛將姜瓖,圍渾源。六年,偕滿達海攻朔州,發?隳其城。移攻寧武,瓖將劉偉、趙夢龍守焉,縱火,棄城走。瓖將楊振威斬瓖降阿濟格,偉、夢龍亦降於瓦克達,靜樂及寧化所、八角堡諸寨悉平。十月,代滿達海為征西大將軍,剿山西餘寇。明大學士李建泰既降,復叛,踞太平。圍之二十餘日,窮蹙,出降。詔誅建泰及其兄弟子侄,籍家產入官。連復平陽屬縣三十六。七年,師還。八年,加封號,掌工部,預議政。九年,坐事,解部任,罷議政。薨。康熙十年,追謚。

瓦克達嘗駐軍平陽,戢軍安民。既薨,平陽人建祠以祀。薨之明年,授其子留雍、哈爾薩三等奉國將軍品級。康熙六年,留雍、哈爾薩訴瓦克達功多,授哈爾薩鎮國公,留雍鎮國將軍。八年,留雍復以己爵卑,訟不平。議政王等言前爵夤緣輔政所得,宜並黜革,上命並降奉國將軍品級。二十一年,哈爾薩復訴瓦克達爵乃功封,例得襲。命襲鎮國公,並封其子海青輔國公。哈爾薩累遷右宗正。二十五年,詔責其鉆營,與海青並奪爵。又以留雍襲鎮國公。三十七年,復以惰,奪爵。乾隆四十三年,高宗錄瓦克達功,命其四世孫洞福以鎮國將軍世襲。

輔國公瑪占,代善第六子。天聰九年,多鐸自廣寧入寧遠、錦州綴明師,瑪占在事有功。崇德元年,從阿濟格入長城,至安州,克十二城。師還,上郊勞,賜酒一金?,封輔國公。三年,從岳託自墻子嶺毀邊城,入密雲,連克臺堡,越燕京趨山東,卒於軍。四年,喪歸,賜銀二千、駝馬各一。無子,未立後。


\end{pinyinscope}