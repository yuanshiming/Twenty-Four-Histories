\article{列傳七十}

\begin{pinyinscope}
覺羅武默訥舒蘭拉錫拉錫子旺札爾孫博靈阿圖理琛何國宗

覺羅武默訥,正黃旗人,景祖第三兄索長阿四世孫也。順治四年,授世職拖沙喇哈番,累進三等阿達哈哈番,擢一等侍衛。康熙六年,授內大臣,管佐領。

十六年,命偕侍衛費耀色、塞護禮、索鼐瞻禮長白山,諭曰:「長白山祖宗發祥之地,爾等赴吉林,選識路之人,瞻視行禮,並巡視寧古塔諸處,於大暑前馳驛速往。」五月己卯,武默訥等發京師;己丑,至盛京,東行;戊戌,至吉林。詢土人,無知長白山路者。得舊居額赫訥殷獵戶岱穆布魯,言其父曾獵長白山麓,負鹿歸,道經三宿,似去額赫訥殷不遠。自吉林至額赫訥殷,陸行十日,舟行幾倍之。寧古塔將軍巴海令運米十七艘詣額赫訥殷,先發,並令協領薩布素護武默訥等行。六月丁未,武默訥等攜三月糧,陸行經溫德亨河、庫埒訥嶺、奇爾薩河、布爾堪河、納丹弗埒城、輝發江、法河、卓隆鄂河,抵訥殷江幹,米亦至。乃乘小舟,與薩布素分道行,溯訥殷江逆流上。

丙寅,會於額赫訥殷。一望深林無路,薩布素率眾前行,伐木開道。遣人還告:行三十里,得一山,升其巔,緣木而望,長白山乃在百餘里外,片片白光如積玉,視之甚晰。戊辰,武默訥前行。己巳,遇薩布素於林中。壬申黎明,大霧,莫辨山所向。聞鶴唳,尋聲往,遇鹿蹊,循行至山麓,見周遭密林,中間平迤圓繞,有草無木。前臨小林,盡處有白樺木,整若栽植,及旋行林外,仍瀰漫無所見。跪誦敕旨,拜畢,霧開,峰巒歷歷在目,登陟有路。遙望之,山修而扈,既近,則堂平而宇圜,向所睹積玉光,冰雪所凝也。山峻約百餘里,巔有池,環以五峰,其四峰臨水拱峙,正南一峰稍低,分列雙闕。池廣袤約三四十里,夾山澗水噴注,自左流者為松花江,右流者為大小訥殷河,繞山皆平林。武默訥瞻拜而下。峰巔群鹿奔逸,僕其七,墜武默訥等前。時登山者正七人,方乏食,謝山靈賜。卻行未里許,欻然霧合。癸酉,還至前望處,終不復見山光。七月庚辰,至恰庫河,馬疲甚。甲申,自恰庫河乘舟還,經色克騰、圖伯赫、噶爾漢、噶達渾、薩穆、薩克錫、法克什、多琿諸河,至松花江。八月丁未,還吉林,巡視寧古塔諸處。乙丑,還京師。

疏聞,詔封長白山之神,秩祀如五嶽。十七年,命武默訥齎敕往封,歲時望祭如典禮。十九年,召入養心殿,命工繪其像以賜。諭曰:「以此像俾爾子孫世世供享,以昭恩寵。」二十九年,卒,賜祭葬。

舒蘭,納喇氏,滿洲正紅旗人。父敦多哩,官刑部侍郎,兼佐領。坐鞫總督蔡毓榮罪,附和尚書希福從輕比,奪官,戍黑龍江。

舒蘭自理籓院筆帖式遷主事。康熙三十八年,從侍郎滿丕、都統烏達禪等,招降巴爾瑚三千餘人,安置察哈爾游牧地,編隸佐領。未幾,巴爾瑚佐領額克圖叛,戕察哈爾副總管阿必達、驍騎校班第,掠馬駝以遁。上命喀爾喀公錫卜推哈坦等率蒙古兵追剿,舒蘭持檄傳示蒙古諸貝勒臺吉,並徵察哈爾、厄魯特兵,從烏達禪會剿,擒其渠。遷內閣侍讀。

四十年,命偕侍衛拉錫往探河源,諭曰:「河源雖名古爾班索里瑪勒,其發源處人跡罕到。爾等務窮其源,察視河流自何處入雪山邊內。凡經流諸處,宜詳閱之。」四月辛酉,舒蘭等發京師。五月己亥,至青海。庚子,至庫庫布拉克。貝勒色卜騰扎勒與偕行。

六月癸亥,至鄂棱諾爾。甲子,西行至扎棱諾爾。鄂棱周二百餘里,扎棱周三百餘里,二諾爾距三十里許。乙丑,至星宿海,蒙古名「鄂敦塔拉」。星宿海之源,小泉萬億,歷歷如星,眾山環之。南有山曰古爾班圖勒哈,西南有山曰布瑚珠勒赫,西有山曰巴爾布哈,北有山曰阿克塔齊勒,東北有山曰烏闌都什,蒙古總名曰「庫爾坤」,即昆侖也。山泉出自古爾班圖勒哈者,為噶爾瑪瑭;出自巴爾布哈者,為噶爾瑪楚木朗;出自阿克塔齊勒者,為噶爾瑪沁尼。三山之泉,溢為三支河,即古爾班索里瑪勒也。三河東流入扎棱諾爾,扎棱一支入鄂棱諾爾,黃河自鄂棱出。其他山泉與平地水泉,淵淪縈繞,不可勝數,悉歸黃河東下。

丁卯,舒蘭等自星宿海還,舍故道,循河流東南行。己巳,登哈爾吉山,見黃河折而東,至庫庫陀羅海山,又南繞薩楚克山,復北流,經巴爾陀羅海山之南。庚午,達阿木尼瑪勒占穆遜山,山最高,雲霧蔽之,不可端倪。蒙古人言長三百餘里,有九高峰,積冰終古不消。常雨雪,一月得晴僅三四日。舒蘭等自此返。壬申,至錫喇庫特勒,又南過僧庫爾高嶺,更百餘里,至黃河岸。見黃河自巴爾陀羅海山東北流,經歸德堡北、達喀山南兩山峽中,流入蘭州。自京師至星宿海,七千六百餘里。寧夏西自松山至星宿海,天氣漸低,地勢漸高,人氣閉塞,行多喘息。九月,還京師,具疏述所經,並繪圖以進。

上諭廷臣曰:「朕於古今山川名號,雖在邊徼遐荒,必詳考圖籍,廣詢方言,務得其正。故遣使至昆侖,目擊詳求,載入輿圖。即如黃河源出西塞外庫爾坤山之東,眾泉渙散,燦如列星,蒙古謂之『鄂敦塔拉』,西番謂之『索里瑪勒』,中華謂之『星宿海』,是為河源。匯為扎棱、鄂棱二澤。東南行,折北,復東行,由歸德堡、積石關入蘭州,其原委可得而縷晰也。」

舒蘭累擢內閣學士。四十五年,命往西藏封拉藏為翊法恭順汗。回京得風疾,遣太醫診視。越二年,疾復發,乞休,許解任調治。五十二年,疾愈,起故官。是年以萬壽恩典,復其父敦多哩故秩。尋遷工部侍郎。未幾,坐事,降三秩調用。五十九年,卒。

拉錫,圖伯特氏,蒙古正白旗人。自親軍校三遷二等侍衛,偕舒蘭窮河源,進一等。雍正初,累擢本旗都統。以治事明敏,予拜他喇布勒哈番世職,授議政大臣。拉錫諳習旗務,奏事輒稱旨,累被褒嘉,加授拖沙喇哈番。四年,以隱匿烏梁海事,盡削官職,降授一等侍衛,管太僕寺卿。尋仍擢鑲白旗滿洲都統,迭署江寧將軍、天津滿洲水師營都統,授領侍衛內大臣。卒。

子旺札爾,初授侍衛,襲世職。使從侍郎阿克敦與噶爾丹定界。如蘇州、如浙江按事。累遷鑲白旗滿洲都統、理籓院侍郎、御前大臣。命赴金川察沿途驛站。金川平,擢領侍衛內大臣。卒,謚恪慎。

孫博靈阿,襲世職。初授侍衛,累遷正藍旗蒙古副都統。從征金川,攻當噶爾拉,撲碉受創,卒。贈都統銜,進世職一等輕車都尉,圖形紫光閣。

乾隆四十七年,高宗命侍衛阿彌達詣西寧祭河神,再窮河源。還奏:「星宿海西南有水名阿勒坦郭勒,更西有巨石高數丈,名阿勒坦噶達素齊老。蒙古語『阿勒坦』為黃金,『噶達素』為北極星,『郭勒』為河,『齊老』石也。崖壁黃金色,上有池,池中泉噴湧,釃為百道,皆黃金色。入阿勒坦郭勒,回旋三百餘里,入星宿海,為黃河真源。」高宗命四庫館諸臣輯河源紀略識其事。阿彌達更名阿必達。大學士阿桂子,附見阿桂傳。

圖理琛,阿顏覺羅氏,滿洲正黃旗人。以國子生考授內閣中書,遷侍讀。坐事,奪職。康熙五十一年,特命復職,出使土爾扈特。

初,土爾扈特汗阿玉奇從子阿喇布珠爾,假道準噶爾赴西藏謁達賴喇嘛。準噶爾臺吉策妄阿喇布坦與阿玉奇構怨,阿喇布珠爾不得歸,款關乞內屬,詔封貝子,賜牧嘉峪關外黨色爾騰。嗣阿玉奇遣使入貢,上欲歸阿喇布珠爾。命圖理琛偕侍讀學士殷扎納、郎中納顏齎敕諭阿玉奇,假道鄂羅斯。

五月,圖理琛等自京師啟行,七月,至鄂羅斯境楚庫柏興。以假道故,待其國察罕汗進止。五十二年正月,許假道,乃行。還烏的柏興,越柏海爾湖而北,抵厄爾庫。鄂羅斯託波爾噶噶林遣其屬博爾科尼來迎。噶噶林者,彼國所稱總管也。圖理琛等欲行,博爾科尼言噶噶林令天使當自水路行,而昂噶拉河冰未泮,請稍駐俟之。三月,自昂噶拉河乘舟抵伊聶謝柏興,登陸。五月,抵麻科斯科,復乘舟自揭的河順流行,經那裡穆柏興、蘇爾呼特柏興、薩瑪爾斯科、狄穆演斯科諸地。七月,至託波爾。其地噶噶林名馬提飛費多里魚赤,迎至廨,留八日。仍遣博爾科尼護之行,抵鴉班沁登陸。自費耶爾和土爾斯科越佛落克嶺,抵索里喀穆斯科,以路濘,守凍十日。復行,經改郭羅多、黑林諾付、喀山、西穆必爾斯科諸地。十一月,至薩拉託付,是為鄂羅斯與土爾扈特界。水自東北來,折而南,鄂羅斯號為佛爾格,土爾扈特號為額濟勒。阿玉奇汗駐牧地曰瑪努託海,距此十日程,以雪盛不能行。

五十三年四月,阿玉奇遣臺吉祥偉徵等來迎。五月,圖理琛等渡額濟勒河,阿喇布珠爾之父納扎爾瑪穆特遣獻馬,卻之。六年朔,至瑪努託海,阿玉奇擇日聽宣敕。圖理琛等以上意諭之曰:「阿喇布珠爾已賜爵優養,欲遣歸爾牧地,以策妄阿喇布坦方與爾交惡,恐為所戕。爾若欲令阿喇布珠爾歸,當自鄂羅斯來迎。」阿玉奇曰:「我雖外夷,然冠服與中國同。鄂羅斯乃嗜欲不同、言語不通之國也,天使歸道當察其情狀。鄂羅斯若以往來數故不假道,則我無由入貢矣。阿喇布珠爾荷厚恩,與歸土爾扈特同,復何疑慮?」阿玉奇及納扎爾瑪穆特等各贈馬及方物,圖理琛等以越境無私交,卻不受。阿玉奇待之有隆禮,留十四日,筵宴不絕。復附表奏謝。圖理琛等遂行,由舊路歸,鄂羅斯遣護如初。五十四年三月,還京師。

是役也,往返三載餘,經行數萬里。蓋土爾扈特為鄂羅斯所隔,遠阻聲教,而鄂羅斯又故導我使紆道行。圖理琛奉使無辱命,既歸國,入對,述往還事狀,並撰異域錄,首冠輿圖,次為行記,呈上覽。上嘉悅,尋授兵部員外郎。阿喇布珠爾亦遂留牧黨色爾騰不復遣,再傳至其子丹忠,雍正中,遷牧額濟內河。

圖理琛遷郎中。世宗即位,命赴廣東察籓庫,就擢廣東布政使。調陜西。三年,擢巡撫。五年,召為兵部侍郎,調吏部。偕喀爾喀郡王額駙策凌等往定喀爾喀與鄂羅斯界。仍調兵部。六年,追議前定界時,與鄂羅斯使臣薩瓦鳴砲謝天,私立木牌,並擅納鄂羅斯貿易人入界;又前任陜西巡撫時,將天下兵數繕摺私給將軍延信,逮問論斬。詔宥免,遣築扎克拜達里克城。高宗即位,授內閣學士,遷工部侍郎。乾隆元年,以老解侍郎任,仍為內閣學士。二年,引疾去。五年,卒。

何國宗,字翰如,順天大興人。康熙五十一年進士,改庶吉士,命直內廷學算法。五十二年,命編輯律歷淵源。未散館,授編修。三遷至庶子。雍正初,授侍讀學士,再遷至內閣學士。

三年,命視黃、運河道,奏請增築戴村石壩,疏濬東昌城南七里河、城北魏家灣及德州城南減河;又以汶、泗泉源紆遠,請專設管泉通判;又請修高家堰石堤。上皆允其請,並以高家堰石堤工沖要,命發帑興修。復奏言:「運河自臨清以上,賴衛水以濟。衛水發源百泉,益以丹、洹二水,其流始盛。請疏百泉為三渠,洹河亦築壩開渠引水,一分灌田,三分濟運。」上從其議。旋以山東巡撫塞楞額奏言國宗等奉使所經州縣,供億白金七千六百有奇。上責國宗不惜物力,負任使,坐降調。五年,授大理寺卿。六年,復擢內閣學士,遷工部侍郎。八年,命與侍郎牛鈕督修北運河減水壩,並濬引河。國宗等議捍護河西務北堤及耍兒渡魚鱗壩,別開塌河澱下流賈家沽洩水河,建築三里淺、筐兒港、張家莊諸處挑水壩,上命如議速行。九年,兼河東河道總督。田文鏡奏戴村初建玲瓏、亂石、滾水三壩。汶水盛漲,自壩面流入鹽河歸海。國宗等增築石壩,水不能過,瀕河連年被患。請毀石壩,復為亂石、滾水壩。上責國宗勘工錯誤,貽害民間,奪官。

乾隆初,起充算學館、律呂館總裁。九年,賜秩視三品。尋授左副都御史。十年,兼領飲天監正。十三年,遷工部侍郎。

康熙間,聖祖命制皇輿全覽圖,以天度定準望,一度當二百里,遣使如奉天,循行混同、鴨綠二江,至朝鮮分界處,測繪為圖。以鴨綠、圖門二江間未詳晰,五十年,命烏喇總管穆克登偕按事部員復往詳察。國宗弟國棟亦以通歷法直內廷。五十三年,命國棟等周歷江以南諸行省,測北極高度及日景。五十八年,圖成,為全圖一,離合凡三十二貞,別為分省圖,省各一貞。命蔣廷錫示群臣,諭曰:「朕費三十餘年心力,始得告成。山脈水道,俱與禹貢合。爾以此與九卿詳閱,如有不合處,九卿有知者,舉出奏明。」乃鐫以銅版,藏內府。

高宗既定準噶爾,乾隆二十一年,命國宗偕侍衛努克三、哈清阿率欽天監西洋人往伊犁,自巴里坤分西北兩路,測天度繪圖。既還報,命署左都御史。二十二年,授禮部尚書。以京察舉弟國棟,坐徇庇,奪官。尋授編修,直上書房。二十八年,復授內閣學士。是歲,上以諸回部悉定,復遣尚書明安圖等往測天度繪圖,是為乾隆內府皇輿圖。二十六年,遷禮部侍郎。二十七年,以老休致。三十一年,卒。

論曰:國家撫有疆宇,謂之版圖,版言乎其有民,圖言乎其有地。聖祖東訪長白山,西探河源,北撫土爾扈特,武默訥、舒蘭、圖理琛奉使稱職。觀所還奏,曲折詳盡,歷歷如繪。國宗以明算事聖祖,又幸老壽,迨高宗朝,詣新疆測繪。康熙、乾隆兩內府圖皆躬與編摹。揆之於古,其裴秀、賈耽之倫歟?


\end{pinyinscope}