\article{列傳一百七十四}

\begin{pinyinscope}
宗室肅順穆廕匡源焦祐瀛陳孚恩

宗室肅順,字雨亭,鄭親王烏爾恭阿第六子也。道光中,考封三等輔國將軍,授委散秩大臣、奉宸苑卿。文宗即位,擢內閣學士,兼副都統、護軍統領、鑾儀使。以其敢任事,漸鄉用。咸豐四年,授御前侍衛,遷工部侍郎,歷禮部、戶部。

七年,擢左都御史、理籓院尚書,兼都統。時寇亂方熾,外患日深,文宗憂勤,要政多下廷議。肅順恃恩眷,其兄鄭親王端華及怡親王載垣相為附和,擠排異己,廷臣咸側目。八年,調禮部尚書,仍管理籓院事,又調戶部。會英法聯軍犯天津,起前大學士耆英隨欽差大臣桂良、花沙納往議約。耆英不候旨回京,下獄議罪,擬絞監候,肅順獨具疏請立予正法,上雖斥其言過當,即賜耆英自盡。大學士柏葰典順天鄉試,以縱容家人靳祥舞弊,命肅順會同刑部鞫訊,讞大闢,上念柏葰舊臣,獄情可原,欲寬之;肅順力爭,遂命斬。戶部因軍興財匱,行鈔,置寶鈔處,行大錢,置官錢總局,分領其事。又設官號,招商佐出納,號「乾」字者四,「宇」字者五。鈔弊大錢無信用,以法令強行之,官民交累,徒滋弊竇。肅順察寶鈔處所列「宇」字五號欠款與官錢總局存檔不符,奏請究治,得朦混狀,褫司員臺斐音等職,與商人並論罪,籍沒者數十家。又劾官票所官吏交通,褫關防員外郎景雯等職,籍沒官吏亦數十家。大學士祁俊藻、翁心存皆因與意見不合,齮齕不安於位而去,心存且幾被重罪。

肅順日益驕橫,睥睨一切,而喜延攬名流,朝士如郭嵩燾、尹耕雲及舉人王闓運、高心夔輩,皆出入其門,採取言論,密以上陳。於剿匪主用湘軍,曾國籓、胡林翼每有陳奏,多得報可,長江上游以次收復。左宗棠為官文所劾,賴其調護免罪,且破格擢用。文宗之信任久而益專。

自八年桂良等在天津與各國議和,廷議於「遣使入京」一條堅不欲行,迄未換約。九年,乃有大沽之戰,敵卻退。十年,英法聯軍又來犯,僧格林沁拒戰屢失利,復遣桂良等議和。敵軍近逼通州,乃改命怡親王載垣、尚書穆廕往議,誘擒英官巴夏禮置之獄,而我軍屢敗之餘不能戰,車駕倉猝幸熱河,廷臣爭之不可。事多出肅順所贊畫,遂扈從。洎敵軍入京師,恭親王留京主和議,議即定,敵軍漸退。留京王大臣籲請回鑾,肅順謂獻情叵測,力阻而罷。肅順先已授御前大臣、內務府大臣,至是以戶部尚書協辦大學士,署領侍衛內大臣,行在事一以委之。

十一年七月,上疾大漸,召肅順及御前大臣載垣、端華、景壽,軍機大臣穆廕、匡源、杜翰、焦祐瀛入見,受顧命,上已不能御硃筆,諸臣承寫焉。穆宗即位,肅順等以贊襄政務多專擅,御史董元醇疏請皇太后垂簾聽政。肅順等梗其議,擬旨駁斥,非兩宮意,抑不下,載垣、端華等負氣不視事。相持逾日,卒如所擬,又屢阻回鑾。恭親王至行在,乃密定計。九月,車駕還京,至即宣示肅順、載垣、端華等不法狀,下王大臣議罪。肅順方護文宗梓宮在途,命睿親王仁壽、醇郡王奕枻往逮,遇諸密雲,夜就行館捕之,咆哮不服,械系。下宗人府獄,見載垣、端華已先在,叱曰:「早從吾言,何至今日?」載垣咎肅順曰:「吾罪皆聽汝言成之也!」讞上,罪皆凌遲。詔謂:「擅政阻皇太后垂簾,三人同罪,而肅順擅坐御位,進內廷出入自由,擅用行宮御用器物,傳收應用物件,抗違不遵,並自請分見兩宮皇太后,詞氣抑揚,意在構釁,其悖逆狂謬,較載垣、端華罪尤重。」賜載垣、端華自盡,斬肅順於市。

肅順攬權立威,數興大獄,輿論久不平;奏減八旗俸餉,尤府怨。就刑時,道旁觀者爭擲瓦礫,都人稱快。肅順既伏法,詔逮所與交結之內監杜雙奎、袁添喜等置重典;其被威脅者,概免株連。耆英子慶錫呈訴其父為肅順所陷,請昭雪,詔以耆英罪當死,肅順奏過當,文宗已斥之,特錮肅順子不得入仕以示戒。

穆廕,字清軒,托和絡氏,滿洲正白旗人。官學生,考授內閣中書,充軍機章京,遷侍讀。咸豐元年,命以五品京堂候補,在軍機大臣上學習行走。尋除國子監祭酒,故事,非科甲不與斯職,部臣執奏,特旨仍授之。歷光祿寺卿、內閣學士,兼副都統。三年,粵匪擾河南、直隸,京師戒嚴,命偕僧格林沁、花沙納、達洪阿辦理京旗各營巡防事宜。遷禮部侍郎,署左翼總兵,尋調刑部。八年,擢理籓院尚書,兼都統,調兵部。

十年,命偕怡親王載垣赴通州,與英法聯軍議和,解桂良等欽差大臣關防授之。議不諧,命擒諸酋,獲巴夏禮送京。敵軍益逼,詔斥穆廕等辦理不善,撤回,扈從熱河。丁父憂,予假十四日,命俟回京補行持服。

十一年,文宗崩,偕肅順等同受顧命,贊襄政務。十月,肅順、載垣、端華等伏法,穆廕與匡源、杜翰、焦祐瀛並罷直軍機,議罪。及議上,詔曰:「穆廕等於載垣等竊奪政柄,不能力爭,均屬辜恩溺職。穆廕在軍機大臣上行走最久,班次在前,情節尤重。王大臣等擬請將穆廕革職發往新疆效力贖罪,咎有應得。惟以載垣等兇焰方張,受其箝制,均有難與爭衡之勢,其不能振作,尚有可原,著即革職,加恩改發軍臺效力贖罪。匡源、杜翰、焦祐瀛皆革職,免其遣戍。」穆廕詣戍,同治三年,論贖歸,歿於家。杜翰,附其父受田傳。

匡源,字鶴泉,山東膠州人。道光二十年進士,選庶吉士,授編修,累官吏部侍郎。咸豐八年,入直軍機,謙退無所建白。罷官後,清貧,主講濟南濼源書院以終。

焦祐瀛,字桂樵,直隸天津人。道光十九年舉人,考授內閣中書,充軍機章京。累遷光祿寺少卿。咸豐十年,命赴天津靜海諸縣治團練,召回從幸熱河,命在軍機大臣上學習行走,遷太僕寺卿。祐瀛尤諂事肅順等,諸詔旨多出其手,為時所指目,故同敗。

陳孚恩,字子鶴,江西新城人。道光五年拔貢,授吏部七品小京官,升主事,充軍機章京。累遷郎中。大學士穆彰阿領樞務,深倚之,歷太僕寺少卿、通政司副使、太僕寺卿,皆留直。遷大理寺卿、左副都御史,兼署順天府尹、工部侍郎,擢倉場侍郎。二十七年,調署兵部侍郎,在軍機大臣上行走。偕侍郎柏葰赴山東按事,劾巡撫崇恩庫款虧缺、捕務廢弛,罷之。暫署山東巡撫。授刑部侍郎,回京面陳在署任不受公費,詔嘉之,特加頭品頂帶、紫禁城騎馬,賜匾額曰「清正良臣」,皆異數。二十九年,偕侍郎福濟赴山西按巡撫王兆琛貪婪事,得實,褫兆琛職,逮京治罪。調工部,署刑部尚書,尋實授。三十年,宣宗崩,遺命罷配郊祔廟,下王大臣議。文宗召對,孚恩與怡親王載垣等爭論於上前,載垣等以失儀自劾,詔原其小節,予薄譴,而斥孚恩乖謬,降三級留任。孚恩尋以母老乞養回籍,允之。

咸豐元年,命在籍幫辦團練。三年,九江陷,巡撫張芾出督師,孚恩與司道守省城,既而賊由安徽回竄上游,命偕芾籌防。賊犯南昌,孚恩偕芾固守,江忠源援師至,力戰,相持九十餘日,賊始引去。以守城功,賜花翎。七年,母喪畢,到京未有除授。八年,御史錢桂森疏言:「孚恩才練識明,在外數年,多所閱歷,儻仍入直樞廷,或使治洋務,必能有濟。」詔斥朋比,罷桂森言職,回原衙門。久之,命孚恩以頭品頂戴署兵部侍郎,又署禮部尚書,授兵部尚書。會鞫順天鄉試關節獄,牽涉其子景彥,自請嚴議,並回避,得旨,褫景彥職,除涉景彥者仍責會訊,僅議失察降一級,準抵銷。尋兼署刑部、戶部尚書,調授吏部尚書。

初,孚恩以議禮忤載垣、端華、肅順等,及再起,乃暱附諸人冀固位。肅順等既敗,少詹事許彭壽疏請治黨援,論形跡最著莫如孚恩,最密莫如侍郎劉昆、黃宗漢,平日所薦舉者,則有侍郎成琦、太僕寺少卿德克津太、候補京堂富績等,於是諸臣盡黜。詔謂:「孚恩當大行皇帝行幸熱河,命諸臣議可否,孚恩有『竊負而逃,遵海濱而處』之語,意在迎合載垣等。大行皇帝上賓,留京諸大臣中獨召孚恩一人赴行在,足證為載垣等心腹。革職,永不敘用。」時廷臣議郊壇配位,孚恩言:「前議宣宗配位時,大行皇帝有定為三祖六宗之諭,出於大學士杜受田所擬,非大行皇帝意。」王大臣等用其言,仍請文宗配祀。許彭壽復引據文宗御制詩有「以後無須變更」之句,請下廷臣再議,議不配祀。詔斥孚恩謬妄,又以籍肅順家得孚恩私書,有暗昧不明語,乃逮孚恩下獄,籍其家,追繳宣宗賜額,遣戍新疆。

居數年,伊犁被兵,將軍常清等奏孚恩籌餉治軍有勞,命免戍,留助理兵餉。同治五年,伊犁陷,孚恩及妾黃、子景和、媳徐、孫小連同殉難。事聞,但恤其家屬,孚恩不與焉。

論曰:文宗厭廷臣習於因循,乏匡濟之略,而肅順以宗潢疏屬,特見倚用,治事嚴刻。其尤負謗者,殺耆英、柏葰及戶部諸獄,以執法論,諸人罪固應得,第持之者不免有私嫌於其間耳。其贊畫軍事,所見實出在廷諸臣上,削平寇亂,於此肇基,功不可沒也。自庚申議和後,恭親王為中外所系望,肅順等不圖和衷共濟,而數阻返蹕。文宗既崩,冀怙權位於一時,以此罹罪。赫赫爰書,其能逭乎?穆廕諸人或以願謹取容,或以附和希進,終皆不免於斥逐。如陳孚恩者,鄙夫患失,反覆靡常,淪絕域而不返,宜哉。


\end{pinyinscope}