\article{列傳一百三十九}

\begin{pinyinscope}
姜晟金光悌祖之望韓崶

姜晟,字光宇,江蘇元和人。乾隆三十一年進士,授刑部主事,累遷郎中。擢光祿寺少卿,轉太僕寺,仍兼刑部行走。四十四年,出為江西按察使。逾年,超擢刑部侍郎,屢命赴各省按事讞獄。五十二年,授湖北巡撫。時大軍征臺灣,晟運米十萬石濟餉需,上嘉之,予議敘。五十三年,荊州江堤潰,命大學士阿桂等往勘,以晟未能疏濬上游漲沙,並坐屬吏婪索淮鹽匣費,褫頂帶。尋召授刑部侍郎。

五十六年,復出為湖南巡撫。芷江境失餉鞘久不獲,晟捕首犯置之法。洞庭湖盜董舒友等積年為商旅害,邏獲之,傳首湖干,盜風以靖。六十年,黔苗石柳鄧叛,永綏苗石三保應之,晟偕總督畢沅往剿。尋雲貴總督福康安來督師,晟駐辰州治軍需,分兵屯諸要隘,緝獲奸匪百戶楊國安父子解京,詔嘉其治軍鎮靜,下部議敘。三月,赴鎮筸查緝邊備,並撫難民,上以辰州要沖,命仍回駐。首逆吳半生就獲,予優敘。

嘉慶元年,湖北枝江、來鳳邪匪起,遣副將慶溥擊賊於龍山,走之,湖南境內獲安。是年,福康安、和琳先後卒於軍,晟偕額勒登保、德楞泰等剿撫,加總督銜。苗疆漸平,駐辰州治善後事宜。二年,兼署總督。三年,京察,予議敘。布政使鄭源鸘附和珅,以貪著,需索屬吏,必多金始得赴任。屬吏藉胥役為幹辦,縱今哧詐浮收,苦累百姓。四年,和珅敗,為言官論劾。詔「晟平日居官猶能自守,因畏和珅不敢參劾,尚非通同舞弊」,命逮訊源鸘,籍其貲財,澈底根究,具得源鸘加扣平餘、蓄養優伶、眷屬多至三百人諸罪狀,論大闢;晟坐失察,當革職留任,上特寬之。冬,鎮筸苗吳陳受倡亂,晟督師守隘,同知傅鼐以計擒斬之,加太子少保。五年,實授總督,尋調直隸。六年,畿輔久雨,永定河決。坐奏報遲延,褫職逮問,發河工效力。工竣,予主事銜,刑部行走。七年,授刑部侍郎。

晟自為曹郎,以治獄明慎受知高宗,易又歷中外,至是凡三入佐刑部。仁宗尤重刑事,晟讞鞫務得其平,多平反者。江西巡撫張誠基剿義寧州匪,飾稱自率兵臨陣,為屬吏所訐。命晟往按,得實,逮誠基,遂暫署巡撫。尋回京。九年,兼署戶部侍郎,命赴南河查勘清口運道,疏言河身淤墊,黃水增高,致清水不能暢注,宜啟祥符五瑞等閘以減黃,增運口蓋壩以蓄清,如議行。擢刑部尚書。十一年,以老疾乞休,溫詔慰留。以刑部事繁,特調工部。章再上,乃命解職在京養痾。尋以前在直隸失察籓庫虛收事,降四品京堂。歸,卒於家。

金光悌,字蘭畦,安徽英山人。由舉人授內閣中書。乾隆四十五年,成進士,轉宗人府主事。遷刑部員外郎,歷郎中。截取京察,並當外任,仍留部。五十五年,部臣奏請以四品京堂用,允之。江西舉人彭良為子賄買吏員執照,光悌與為姻親,御史初彭齡劾光悌瞻徇,坐降調,仍補刑部員外郎,留部覈辦秋審。御史張鵬展復劾之,詔:「光悌在部久,平日毀多譽少,停其兼部。」尋兼內閣侍讀學士。

嘉慶七年,授山東按察使,晉布政使。十年,召授刑部侍郎,數奉使赴山東、直隸、天津、熱河勘獄,並得實以報。十一年,授江西巡撫。疏言江西積案繁多,請設局清釐。十四年,擢刑部尚書。

光悌自居郎曹,為長官所倚,至是益自力。以當時讞獄多以寬厚為福,往往稍減罪狀上之,部臣懸千里推鞫,茍引律當毋更議。故遇事必持律,不得減比。人咸以光悌用法嚴,然亦有從寬者。舊例,監守自盜限內完贓者減等,乾隆二十六年改重不減等,光悌奏復舊例。後阿克蘇錢局章京盜官錢,計贓五百兩以上,主者引平人竊盜律,當絞情實。光悌曰:「盜官錢當擬斬監追,不決,絞情實則決矣。不得引竊盜律。」奏平之。仁宗覽奏曰:「官盜較私盜反薄耶?」對曰:「與其有聚斂之臣,寧有盜臣。律意如是。」卒如其議。光悌練習律例,議必堅執,同列無以奪之。然屢被彈劾,時論亦不盡以為平允。十七年,卒於官,詔依尚書例賜恤。

祖之望,字舫齋,福建浦城人。乾隆四十三年進士,選庶吉士,散館授刑部主事,洊升郎中。俸滿當截取外任,以諳悉部務留之。京察一等,以四五品京堂用。歷通政司參議、太常寺少卿,仍兼部務。五十八年,出為山西按察使。摘律例民間易犯罪名條列之,曰三尺須知錄,刊布於眾,俾民無誤罹法。六十年,遷雲南布政使。上以之望親老,調湖北,俾便迎養。

嘉慶元年,教匪起荊、襄,蔓延鄖、宜、施南諸郡。總督巡撫皆統師出,之望一人留武昌治事,訛言數作,時獲賊諜,偽檄遍通衢。之望靜定不驚,防御要隘,城鄉市鎮設保甲互稽,民心帖然。賊犯孝感,調師翦滅,下游五郡皆安堵。詔以之望雖未與賊戰,坐鎮根本,武、漢無虞,嘉其功,賜花翎。二年,丁父憂,命留任素服治事。四年,安襄鄖道胡齊侖侵餉事發,命之望察治,齊侖侵蝕餽送,轇轕猝不易究,上切責之,命解任來京。及讞定,之望坐徇庇降調。上知之望無染指,居官有聲,素諳刑名,以按察使降補。逾月,授刑部侍郎,予假葬父省母。

五年,授湖南巡撫。鎮筸黑苗出峒焚掠,蔓延三,遣兵擊平之。親勘常德堤圍私墾洲地百數十處,造冊立案,永息爭端。尋復召為刑部侍郎。至京,面陳永綏孤懸苗境,不足資控制,請移治花園,移協營茶洞,沿邊遍設碉卡,以永綏舊城為汛地,使苗弁駐闉,約束諸苗寨,下所司議行。六年,偕侍郎那彥寶勘近畿水災,又偕侍郎高杞監疏長辛店河道。

七年,命赴山東按皁役之孫冒考,巡撫和瑛誣斷事,和瑛譴罷,即授之望巡撫。尋調陜西。大軍剿南山餘孽,之望籌備軍食,安插鄉勇,撫恤災黎,偕總督惠齡奏籌善後事宜甚悉。調廣東,乞假省親。九年,仍授刑部侍郎。逾一年,以母老乞養歸。十四年,仁宗五旬萬壽,之望入都祝嘏。其母年八十有三,上垂問褒嘉,賚予有加。尋丁母憂,服闋,擢刑部尚書。十八年,以病解職,尋卒。

韓崶,字桂舲,江蘇元和人。父是升,客游京師,授經諸王邸,以名德稱。崶少慧能文,由拔貢授刑部七品小京官,累擢郎中。乾隆五十四年,出為河南彰德知府,遷廣東高廉道。坐失察吳川知縣庇縱私鹽事,降刑部主事,復洊遷郎中。

嘉慶六年,授湖南嶽常澧道,遷按察使,調福建,署布政使。蔡牽方擾臺灣,海疆多事,崶籌軍備杜接濟甚力,遷湖南布政使。十一年,召為刑部侍郎。十二年,命赴荊州按將軍積拉堪與知府交結事,又命勘南河。十三年,宗室敏學恃勢不法,讞擬輕比,詔斥部臣屈法縱奸,譴責有差。崶方奉使河間讞獄,未與畫諾,上以崶先於召對面陳,意存開脫,且部事素由崶先覈定,跡近專擅,降授廣東按察使。未幾,擢巡撫。

時英吉利兵船占澳門砲臺,入黃埔,久之始退。總督吳熊光不即遣兵驅逐,以罪罷,命崶兼署總督。十四年,崶查閱澳門夷民安堵,因疏陳:「西洋人於其地舊設砲臺六,請自伽思蘭砲臺迤南,加築女墻二百餘丈,於前山寨駐專營,蓮花莖增關徬石垣,新湧山口築砲臺,填蕉門海口,以資控制。」如議行。又密陳粵海形勢:「沿海村落,處處可通,外洋盜匪,易生窺伺。必先固內而後可禦外。凡屬扼要砲臺,宜簡練精銳,嚴密防守。並令沿海紳衿耆董,督率丁壯,互相捍護,自衛身家,較為得力。」百齡繼為總督,會奏:「華、洋交易章程,外國兵船停泊外洋,澳內華、洋人分別稽覈。各國商賈,止許暫留司事之人,經理債務,餘俱飭依期回國,不得在澳逗留。洋船引水人,責令澳門同知給發牌照。買辦等華人,責成地方有司慎選承充,隨時稽察。洋船起貨時,不許洋商私自分撥。」下軍機大臣採擇議行。

逾年,海盜張保仔就撫,烏石二、東海霸以次誅降,賜花翎。十六年,復署總督。疏請免米稅,以通商販、裕民食。又疏陳:「潮州多械斗,而營員無協緝之責,請令文武會拏;距省遠,請軍流以下就近由巡道覆覈。」又言:「懲治悍匪,請如四川例:初犯械系,限一年改行;積兩限如故,即治以棍徒屢次滋擾律。」皆允行。十八年,入覲,授刑部尚書。崶父是升年八十,給假三月歸為壽。二十一年,丁父憂,服闋,以一品銜署刑部侍郎,尋補刑部尚書。

道光四年,平反山西榆次縣民閻思虎獄,被議敘。初,思虎強奸趙二姑,知縣呂錫齡受賕,逼認和奸,趙二姑忿而自盡,親屬京控。命巡撫親提,仍以和奸擬結。御史梁中靖疏劾,提解刑部,審得實情是強非和,並原審各官賄囑、徇縱、回護諸弊狀,思虎論斬,趙二姑旌表,巡撫邱樹棠、按察使盧元偉及府縣各官,降革遣戍有差。詔嘉刑部堂司各官秉公申雪,並予議敘。梁中靖參奏得實,亦加四品銜。會有官犯侯際清擬流,呈請贖罪,部議因際清犯罪情重,仍以可否並請。詔斥含混取巧,命大學士托津等查訊,侍郎恩銘、常英、司員恩德等皆有賄囑情事,崶亦解任就質,坐失察司員得賄,嗣子知情,親屬撞騙,議奪職遣戍,因年老,從寬,命效力萬年吉地工程處。逾歲,召署刑部侍郎。六年,以病乞歸。十四年,卒。

論曰:有清一代,於刑部用人最慎。凡總辦秋審,必擇司員明慎習故事者為之。或出為監司數年,稍回翔疆圻,入掌邦憲,輒終其身,故多能盡職。仁宗尤留意刑獄,往往親裁,所用部臣,斯其選也。姜晟、祖之望,易又歷中外,並有政績。金光悌、韓崶,皆筦部務最久,光悌治事尤厲鋒鍔,號刻深雲。


\end{pinyinscope}