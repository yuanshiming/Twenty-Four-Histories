\article{列傳三百二}

\begin{pinyinscope}
土司四

○貴州

貴州古羅施鬼國,漢夜郎國,並牂牁、武陵郡地。唐亦置播州、思州。元置八番、順元諸軍民宣慰使司以羈縻之。明靄翠、奢香最為效忠。後則播州之楊、永寧之奢、水西之安,為西南鉅患,楊氏滅,為遵義、平越二府;奢氏滅,為永寧縣。清初,黔省安氏猶強。經孫可望之亂,未頒正朔,苗蠻蠢動,諸擅兵相攻者,蹂躪地方,無有寧日。

順治十五年,經略洪承疇定貴州。十七年四月,馬乃營土目龍吉兆等反。雲、貴既平,各土司俱奉貢賦,遵約束。龍吉兆收養亡命,私造軍器,奸民文元、胡世昌、況榮還等俱黨附之,遙結李定國為聲援,糾合鼠場營龍吉佐、樓下營龍吉祥歃血盟,掠廣西泗城州之土寨,安南衛之阿計、屯水橋、麻衣沖、下三阿、白屯等處,所過劫戮。總督趙廷臣、巡撫卞三元招諭不服,乃合疏請討。十一月十九日,廷臣破果母寨,殺賊數千,擒吉兆子、吉佐妻,殲逆黨文元、胡世昌於陣,遂乘勝破呷寨。吉兆閉寨拒守,官兵圍之。十八年二月,廷臣令官兵人持一炬,縱火焚其寨,破之。吉兆及逆黨況榮還等皆伏誅,馬乃平。

九月,劉鼎叛。康熙二年正月,丹平土官莫之廉以隱匿劉鼎伏誅。金築土官王應兆與鼎通,總督楊茂勛討平之,鼎敗逃水西。七月,被獲伏誅。

三年正月,逆賊常金印等謀反,伏誅。金印,上元人,自稱常遇春之後,從粵走黔,與水西安坤、皮熊等同謀反。金印稱「蕩虜大將軍湘平伯」,偽造印敕旗纛,聚黨陳鳳麟、高岑、吉士英、米應貴等,煽誘諸土司為亂,為同黨陳大出首,俱就擒。

二月,水西宣慰司安坤叛。初,經略洪承疇至沅,師不能進,承疇招安坤,許以如元阿盡、明靄翠故事,坤大喜,繳印歸誠,引大兵由小路進入貴陽。滇、黔底定,敘坤功,許世襲,兼賜袍帽靴服採幣。朋總兵皮熊合謀,蠢蠢思動,蹤跡頗露。總督楊葆勛曰:「水西地方沃野千里,地廣兵強,在滇為咽喉,在蜀為門戶,若於黔則腹心之蠱毒也。失今不討,養癰必大。」乃請剿。命總管吳三桂督云、貴各鎮兵分東西兩路討之。三月,三桂統十鎮兵由畢節七星關入,令總兵劉之復駐兵大方,遏其沖逸,分提督李本深統貴州四鎮兵由大方之六歸河會剿,屯糧於三岔河。而檄黔省兵書誤書「六歸」為「陸廣」,於是本深兵及黔、蜀二省所運之糧盡屯陸廣,三路氣息隔絕不相通。三桂受困兩月,食將絕,外援不至。永順總兵劉安邦戰死,受圍益迫。適水西土目安如鼎遣人偵黔營虛實,為本深所獲,始知三桂被圍已久,乃使為引導,整兵入援。副將白世彥手斬驍賊以徇於陣,賊遂敗走。總兵李如碧亦率精兵入重圍,運糧接濟,兵合為一,敗賊阿作峒,復敗之得初峒,九月又敗之紅崖峒。坤率其妻祿氏逃於木弄箐,復逃至烏蒙,烏蒙不納。坤遣漢把曾經賚印投降,不許,生擒坤於大方之杓箐,並擒皮熊、安重聖等。皮熊不食十五日而死,坤與重聖俱伏誅。

四年十二月,郎岱土司隴安籓反,命吳三桂發兵討之。籓乃安坤親黨。坤滅後,招納坤餘孽隴勝等,及安重聖妻隴氏,殺安順府經歷袁績,攻破關嶺,直犯永寧。隴勝等亦攻犯大定、威寧,殺畢節經歷秦文。五年六月,隴安籓伏誅,郎岱平。

二十四年七月,黎平賊何新瑞反。新瑞本李姓,初在靖州為僧,後至平茶所犯罪,逃至新化,乃冒姓何,稱故明總督何騰蛟子,煽惑苗民作亂,黎平官兵擊敗之。二十五年二月,新瑞伏誅,徙土司韋有能等,以其地入永從縣。

廣順州之長寨,寨據各苗之腹。前總督高其倬誘擒阿近,議設營汛,以控前後左右各寨。雍正四年夏,官兵焚其七寨,未獲首逆,副將劉業浚即退營宗角,且言三不可剿。鄂爾泰駁以三不可不剿;令總兵石禮哈搜討,盡殲首從,勒繳軍器,建參將營,分扼險要,易服薙發,立保甲,稽田戶。於是乘威招服黔邊東西南三面廣順、定番、鎮寧生苗六百八十寨,鎮寧、永寧、永豐、安順生苗千三百九十八寨,地方千餘里,直抵粵界。

鎮遠清水江者,沅水上游也,下通湖廣,上達黔、粵,而生苗據其上游,曰九股河,曰大小丹江,沿岸數百里,皆其巢窟。古州者,有里有外。里古州距黎平府百八十里,即元置古州八萬洞軍民長官司所也。地周八十餘里,戶四五千,口二萬餘。都江、溶江界其左右,合為古州江。由此東西南北各二三百里為外古州,約周千二三百里,戶數千,口十餘萬,可敵兩三州縣。環黔、粵萬山間,而諸葛營踞其中,倚山面川,尤據形勢。張廣泗守黎平,輕騎深入周勘,倡議置鎮諸葛營,扼吭控制,而其外戶為都勻、八寨,內戶為丹江、清江。乃於六年夏,先創八寨以通運道,分兵進攻大小丹江,出奇設伏,盡焚負固之雞講五寨。苗赴軍乞降,飲血刻木,埋石為誓。九年,乘勝沿九股河下抵清水江。時九股苗為漢奸曾文登所煽,言改流升科,額將歲倍,且江深崖險,兵不能入。及官軍至,以農忙佯乞撫,廣泗亦佯許之,而潛舟宵濟,扼其援竄。蘇大有、張禹謨突搗其巢,又敗其夜劫營之賊,填壕拔橛,冒險深入,苗四山號泣,縛曾文登以獻。於是清水江、丹江皆奏設重營,以控江路,令兵役雇苗船百餘,赴湖南市鹽布糧貨,往返不絕,民、夷大忭,估客雲集。

古州自昔奧樸,自清初吳三桂偽將馬寶兵由楚竄滇,取道古州,諸苗遮獲其大砲重甲火藥,由是日強,而上下江尤甚。上江為來牛、定旦,下江為溶峒。當廣泗初至,苗皆謂官兵不能久,依違從撫,及聞諸葛營建城堡,遂群起拒命。八年秋,廣泗督官兵夜半集苗船為浮橋,攻其不備,進攻上江之來牛、定旦,擒斬四千,獲砲械無算。其下江溶峒之深遠大箐,危峰障日,皆伐山通道,窮搜窟宅。乃遍勘上下江,濬灘險,置斥堠,通餉運。其都江、清水江之間,有丹江橫貫,惟隔陸路五十餘里,為之開通,於是楚、粵商艘直抵鎮城外,古州大定。

初,世宗以廣泗招撫古州,不煩兵力,由知府逾年擢至巡撫,遣侍讀春山、牧可登至軍察之,並頒犒師銀十萬兩。鄂爾泰約廣西巡撫金鉷赴貴陽會籌邊事,乃議黎平府設古州鎮,而都勻府之八寨、丹江,鎮遠府之清水江,設協營,增兵數千,為古州外衛;後復改清江協為鎮,與古州分轄。世宗嘉鄂爾泰之勞,錫封襄勤伯,世襲罔替。九年冬,入為武英殿大學士,以高其倬代之,以元展成巡撫貴州。

十二年,哈元生進新闢苗疆圖志,以尹繼善督云、貴,而復有黔苗之變。初,苗疆闢地二三千里,幾當貴州全省之半,增營設汛,凡腹內郡縣防兵大半移戍新疆。又鄂爾泰用兵招撫,止及古州、清江,未及臺拱之九股苗。有司輒稱臺拱原內屬,巡撫元展成易視苗疆,遽於十年設營駐兵。時秋稼未穫,苗佯聽版築,而刈穫甫畢,即傳集上下九股數百寨,叛圍大營,並扼排略大關之險,以阻餉道。營中樵汲皆斷,死守彌月,援至始解。提督哈元生入覲回黔,十一年春,進軍臺拱,攻賊於番招之蓮花,破之,設戍其上。

十三年春,苗疆吏以徵糧不善,遠近各寨蜂起,遍傳大刻。總兵韓勛破賊古州之王家嶺,賊復聚集清江、臺拱間,番招屯復圍於賊。巡撫元展成與哈元生不合,倉卒調兵五千,盡付副將宋朝相領之赴援,半途亦困於賊。賊探知內地防兵半戍苗疆,各城守備空虛,於是乘間大入,陷凱里,陷重安江驛,陷黃平州,陷巖門司,陷清平縣、餘慶縣,焚掠及鎮遠、思州。而鎮遠府治無城。人心恟懼,臺拱、清江各營汛亦多為賊誘陷。逆氛四起,省城戒嚴。四月,哈元生乃以親兵三百自出督師,扼清平之楊老驛。六月,詔發滇、蜀、楚、粵六省兵會剿,特授哈元生揚威將軍,湖廣提督董芳副之。七月,又命刑部尚書張照為撫定苗疆大臣,副都御史德希壽副之。時尹繼善已遣雲南兵二千星夜赴援,湖、粵兵亦繼至。生苗見各路援兵漸集,各擄掠回巢,棄城弗守。元生進軍凱里,檄各鎮克復諸城,又合攻重安江賊,以開滇師之路。生苗既回巢穴,則糾眾攻圍新疆各營汛,於是臺拱、清江、丹江、八寨諸營復同時告急。時廣西兵八千已至古州,廣東兵餉亦晝夜溯流而上,湖廣兵先後集鎮遠界。元生遣古州鎮韓勛攻毀首逆各巢,又分兵三路:一由槁貢以通臺拱,一由八弓援柳羅以通清江,一走都勻援八寨;而八寨協副將馮茂復誘殺降苗六百餘及頭目三十餘冒功,於是苗逃歸者,播告徒黨,詛盟益堅,多手刃妻女而後出抗官兵。陷青溪縣城,而清江之柳羅、都勻之丹江,自春夏被圍半載,糧盡援絕,九閱月圍始解。

張照奉命赴苗疆,且令察其利害。照至沅州、鎮遠,則密奏改流非策,致書諸將,首倡棄地之議,且袒董芳,專主招撫,與哈元生齟。楚、粵官兵皆隸芳麾下。旋議分地分兵,施秉以上用滇、黔兵,隸元生;施秉以下用楚、粵兵,隸董芳。於是已進之兵,紛紜改調互換,而哈元生、董芳遂欲將村寨道路盡畫上下界,文移辨論,致大兵雲集數月,曠久無功,賊乘間復出焚掠,清平、黃平、施秉間紛紛告警。當是時,中外畏事者,爭咎前此苗疆之不當闢,目前苗疆之不可守,全局幾大變。

八月,召張照、德希壽還。十月,授張廣泗七省經略,哈元生以下咸受節制。旋逮張照、董芳、哈元生及元展成治罪。廣泗奏言:「張照等所以無功者,由分戰兵守兵為二,而合生苗、熟苗為一也。兵本少而復分之使單,賊本眾而復驅之使合。且各路首逆,自古州敗退,咸聚於上下九股、清江、丹江、高坡諸處,皆以一大寨領數十百寨,雄長號召,聲勢犄角,我兵攻一方,則各方援應,彼眾我寡,故賊日張,兵日挫。為今日計,若不直搗巢穴,殲渠魁,潰心腹,斷不能渙其黨羽,惟有暫撫熟苗,責令繳兇獻械,以分生苗之勢。而大兵三路同搗生苗逆巢,使彼此不能相救,則我力專而彼力分,以整擊散,一舉可滅,而後再懲從逆各熟苗,以期一勞永逸。」廣泗乃調全黔兵集鎮遠,以通云、貴往來大路。以精兵四千餘攻上九股,四千餘攻下九股,而自統五千餘攻清江下流各寨,是冬,刻期並舉。

乾隆元年春,復增兵分八路排剿抗拒逆寨,遺孽盡竄牛皮大箐。箐圜苗巢之中,盤亙數百里,北丹江,南古州,西都勻、八寨,東清江、臺拱,危巖切雲,老樾蔽天,霧雨冥冥,蛇虺所國,雖近地苗蠻,亦無能悉其幽邃,故首逆諸苗咸藪伏其中,恃官兵所萬不能至,俟軍退復圖出沒。廣泗檄諸軍分扼箐口以坐困之,又旁布奇兵箐外以截逋逸,如阹獸網魚,重重合圍,以漸進偪。自四月至五月,將士犯瘴癘,冒榛莽,靡奧不搜,靡險不剔,並許其黨自相斬捕除罪。由是憝魁罔漏,俘馘萬計,其饑餓顛隕死崖谷間者,不可計數。六年,復乘兵威搜剿附逆熟苗,分首惡、次惡、脅從三等,涉秋徂暑,先後埽蕩,共毀除千有二百二十四寨,赦免三百八十有八寨,陣斬萬有七千六百有奇,俘二萬五千有奇,獲銃砲四萬六千五百有奇,刀矛弓弩標甲十有四萬八千有奇。宥其半俘,收其叛產,設九衛,屯田養兵戍之。詔盡豁新疆錢糧,永不徵收,以杜官胥之擾。其苗訟仍從苗俗處分,不拘律例。以廣泗總督貴州兼管巡撫事,世襲輕車都尉。自是南夷遂不反。

五年夏,湖南靖州、武岡瑤,城步橫嶺苗,與廣西瑤同叛。總督班第使鎮筸總兵劉策名以兵五千進剿,以五千應援,詔廣泗復以欽差大臣節制軍務。先後斬馘五千餘,俘五千餘,於十二月班師。

鄂爾泰卒於乾隆十年,以開闢西南夷功,配享太廟。

後乾隆六十年,松桃苗變;及咸豐二年,教匪變,煽及苗疆,同治十二年方定。然非土司肇事,故不錄。

貴陽府:

中曹長官司,在府南十五里。明洪武三年,以謝石寶為長官司。傳至謝正倫,清順治十五年,歸附,仍準世襲。

副司,劉氏,清雍正七年,於土權疊害案內改流官。

養龍長官司,在府北二百二十里。明洪武五年,以蔡普為長官司。傳至蔡瑛,清康熙八年,歸附,準世襲。

白納長官司,在府南七十里。元為白納縣,尋改。明初,以周可敬為長官司。傳至周爾齡,清順治十五年,歸附,仍準世襲。

副長官。趙啟賢同。

虎墜長官司,在府東六十里。明洪武三年,以宋璢為長官司。傳至宋繼榮,清順治十六年,歸附,仍準世襲。

定番州

程番長官司,唐末,程元龍平定溪洞,世守程番。元改給安撫司印。明洪武四年,改授程番長官司。傳至程民新,清順治十五年,歸附,仍準世襲。

上馬橋長官司,在州北二十里。自唐末方定遠開疆,明洪武四年,改授長官司。傳至方維新,清順治十五年,歸附,仍準世襲。

小程番長官司,在州北五里。始自唐末程鸞。明洪武四年,改授小程番長官司。傳至程登雲,清順治十五年,歸附,仍準世襲。

盧番長官司,在州北五里。始自唐末盧君聘。元置羅番靜海軍安撫司。明洪武四年,改授盧番長官司。傳至盧大用,清順治十五年,歸附,仍準世襲。

方番長官司,在州南十里。始唐末方德。明洪武四年,改授方番長官司。傳至方正綱,清順治十五年,歸附,仍準世襲。

韋番長官司,在州南五里。唐韋四海守此土。明洪武四年,改授韋番長官司。傳至韋璋,清順治十五年,歸附,仍準世襲。

臥龍番長官司,在州南十五里。唐時,龍德壽據此。明洪武四年,改授臥龍番長官司。傳至龍國瑞,清順治十五年,歸附,仍準世襲。

小龍番長官司,在州東南二十里。唐時,龍方靈據此。明洪武四年,改授小龍番長官司。傳至龍象賢,清順治十五年,歸附,仍準世襲。

金石番長官司,在州東二十五里。唐時,石寶據此。明洪武四年,改授金石番長官司。傳至龍如玉,清順治十五年,歸附,仍準世襲。

羅番長官司,在州南三十里。始自唐時龍應召。明洪武四年,改授羅番長官司。傳至龍從雲,清順治十五年,歸附,仍準世襲。

大龍番長官司,在州東三十里。始於唐時龍昌宗。明洪武四年,改授大龍番長官司。傳至龍登雲,清順治十五年,歸附,仍準世襲。

木瓜長官司,在州西七十里。始於元時石期璽。明洪武八年,改授木瓜長官司。傳至石玉林,清順治十五年,歸附,仍準世襲。

副長官,始於元時顧德。明洪武八年,改授木瓜副長官。傳至顧大維,清順治十五年,歸附,仍準世襲。

麻鄉長官司,在州西七十五里。明洪武十年,以得玉思為麻鄉長官司。傳至得志,清順治十五年,歸附,仍準世襲。

開州

乖西長官司,在州東六十里。始於唐時楊立信。明洪武四年,改授乖西長官司。傳至楊瑜,清順治十五年,歸附,仍準世襲。

副長官,始於唐時劉起昌。傳至劉國柱,清順治十五年,歸附,仍準世襲。

龍里縣

大谷龍長官司,在縣西北。始於元時宋國。明洪武十三年,授大谷龍長官司。傳至宋之尹,清順治十五年,歸附,仍準世襲。

小谷龍長官司,在縣東北。元時,宋幕授小谷龍安撫司。明嘉靖十一年,改授長官司。傳至宋景運,清順治十五年,歸附,仍準世襲。

貴定縣

平伐長官司,在縣南。唐時李保郎,以征南功授安撫司。明洪武十五年,改授平伐長官司。傳至李世廕,清順治十五年,歸附,仍準世襲。

大平伐長官司,在縣南三十里。後漢昭烈時,宋隆豆征南有功,世守茲土。明洪武四年,改授宋臣為大平伐長官司。傳至宋世昌,清順治十五年,歸附,仍準世襲。

小平伐長官司,在縣南三十里。唐時宋忠宣,以功授招討司。明洪武四年,改授小平伐長官司。傳至宋天培,清順治十五年,歸附,仍準世襲。

新添長官司,在縣東北。唐時,宋景陽據此。明洪武四年,改授新添長官司,屬新添衛。傳至宋鴻基,清順治十五年,歸附,仍準世襲。康熙十年,改隸貴定縣。

羊場長官司,在縣東北。明洪武三十二年,以郭九齡為羊場長官司。傳至郭天章,清順治十五年,歸附,仍準世襲。

修文縣

底寨長官司,唐時,蔡興隆調徵黑羊,授護國將軍,留守茲土。明洪武四年,改授底寨長官司。傳至蔡啟珵,清順治十五年,歸附,仍準世襲。

副長官,始自唐時梅天祿。明洪武四年,準世襲。傳至梅朝聘,清順治十五年,歸附,仍襲舊職。

安順府:

普定縣

西堡副長官。明洪武十二年,溫伯壽以平苗功,授西堡副長官。傳至溫捷桂,清順治十五年,歸附,仍準襲職。

鎮寧州

康佐副長官。明永樂六年,於成以功授康佐副長官。傳至於應鵬,清順治十五年,歸附,仍準襲職。

永寧州

頂營長官司,在州南一百里。明洪武十六年,羅錄以功授頂營長官司。傳至羅洪勛,清順治十五年,歸附,仍準襲職。

募役長官司,在州西一百七十里。明洪武十九年,阿辭以功授募役長官司。傳至阿更,永樂元年,賜姓禮,更名山。傳至阿廷試,清順治十五年,歸附,仍準襲職。

沙營長官司,明洪武十四年,沙先以功授沙營長官司。傳至沙裕先,清順治十五年,歸附,仍準襲職。

盤江土巡檢。明洪武八年,李當以功授盤江巡檢。傳至李桂芳,清順治十五年,歸附,仍準襲職。

平越州

楊義長官司,在州東八十里。始於唐時金密定。明洪武二十一年,改授楊義長官司。傳至金榜,清順治十五年,歸附,仍準襲職。

黃平州

巖門長官司,在州東北。明成化六年,何清以征苗有功,授凱裡安撫司左副長官。萬歷四十二年,改屬黃平州。傳至何仕洪,清順治十五年,歸附,改授巖門長官司,世襲。

重安司土吏目,在州西三十里。明洪武八年,以張佛寶、馮鐸為正、副長官司。萬歷二十七年,改土吏目。傳至張威鎮,清順治十五年,歸附,仍準襲職。

甕安縣

草塘司土縣丞。明洪武二十五年,以宋邦佐為草塘安撫司。傳至世寧,萬歷二十九年,改授土縣丞。傳至宋運鴻,清順治十五年,歸附,仍準襲職。

甕水司土縣丞,在縣西北。明洪武十七年,以猶恭為安撫司。萬歷中,改授土縣丞。傳至猶登第,清順治十五年,歸附,仍準襲職。

餘慶縣

土縣丞。唐毛巴有功,授餘慶土知府。明洪武二年,改長官司。萬歷二十九年,改為土縣丞。傳至毛鵬程,清順治十五年,歸附,準襲前職。

土主簿。元楊正寶有功,授白泥司副長官。明萬歷二十四年,改為土主簿。傳至楊璟,清順治十五年,歸附,仍襲前職。

都勻府:

都勻長官司,在府南七里。明洪武十六年,以吳賴為都勻長官司。傳至吳玉,清順治十五年,歸附,準襲前職。

副長官。王應祖,同。

邦水長官司,在府西二十里。明永樂六年,以吳珊為邦水長官司。傳至吳昌祚,清順治十五年,歸附,仍準襲職。

麻哈州

樂平長官司,在州北四十里。明洪武年間,授宋仁德為樂平司正長官。傳至宋治政,清順治十五年,歸附,仍襲前職。

平定長官司,在州北一百里。明洪武十年,授吳忠平定長官司。傳至吳士爵,清順治十五年,歸附,仍襲前職。

獨山州

土同知。明洪武十六年,以蒙聞為九姓獨山長官司,以境有九姓蠻為名。弘治八年,

改土同知。傳至蒙一龍,清順治十五年,歸附,仍襲前職。

豐寧上長官司,在州南一百二十里。明洪武二十三年,以楊萬八為豐寧上長官司。傳至楊懋功,清順治十五年,歸附,仍準世襲。

豐寧下長官司,在州東南二百四十里。明洪武二十三年,以楊萬全為豐寧下長官司。傳至楊威遠,清順治十五年,歸附,仍準世襲。

爛土長官司,在州東一百十里。明洪武二十四年,以張鈞為爛土長官司。傳至張威遠,清順治十五年,歸附,仍準世襲。

凱裡司。楊氏,清康熙四十五年,以土酋大惡案內改土歸流,入清平縣。

鎮遠府:

土同知。宋時,何永壽以功授高丹峒正長官司。明洪武三年,授何濟承為鎮遠州土同知。傳至何大昆,清順治十五年,歸附,仍準世襲。

土通判。宋時,楊從禮。明正統四年,改授楊瑄鎮遠州土通判。傳至楊龍圖,清順治十五年,歸附,仍準世襲。

土推官。宋時,楊載華。明正統十一年,改授楊忠鎮遠州土推官。傳至楊秀瑋,清順治

十五年,歸附,仍準世襲。

偏橋長官司,在府城西六十里。宋時,安崇誠。明洪武三年,改授安德可為偏橋長官司。傳至安顯祖,清順治十五年,歸附,仍準世襲。

左副長官,楊通聖;右副長官,楊毓秀:均同。

鎮遠縣

工⼙水長官司,在縣東八十里。明洪武元年,授楊昌盛為工⼙水長官司。傳至楊勝梅,清順治十六年,歸附,仍準世襲。

副長官。袁洪遠,同。

思南府:

隨府辦事長官司。宋時,田二鳳。明洪武五年,改思南宣慰司。永樂十一年,改授隨府辦事長官司。傳至田仁溥,清順治十七年,歸附,仍準世襲。

蠻夷長官司,在府城西。宋時,安仲用。明洪武二十九年,改授蠻夷長官司。傳至安於磐,清順治十七年,歸附,仍準世襲。

副長官。李際明,清順治十七年,歸附,仍準世襲。雍正八年,李慧緣事革職。

沿河祐溪長官司,在府北二百十里。元時,張仲武以功授長官司。傳至張承祿,清順治十五年,歸附,仍準世襲。

副長官。冉鼎臣,同。

朗溪長官司,在府東八十里。元時,田穀。明洪武元年,授朗溪長官司。傳至田養民,清順治十五年,歸附,仍準世襲。

副長官。任進道,同。

安化縣

土縣丞。元時,張坤元。明萬歷三十三年,改授土縣丞。傳至張試,清順治十八年,歸附,仍準世襲。

土巡檢。明洪武七年,以陸公閱為土巡檢。傳至陸陽春,清順治十五年,歸附,仍準世襲土百戶。久改流。

印江縣

土縣丞。元時,張恢留此。明嘉靖七年,改授土縣丞。傳至張應璧,清順治十五年,歸附,仍準世襲。

婺川縣

土百戶,改流。

石阡府:

石阡正長官司。清雍正八年,改土歸流。

副長官,在府城西北。元時,楊九龍以功授石阡副長官。明洪武五年,仍之。傳至楊敬勝,清順治十五年,歸附,亦準世襲。

苗民長官司,在府城西北。明洪武十年,立。清康熙四十三年,改土歸流。

思州府:

都坪長官司,在府城內。元何清授定雲路總管。明洪武七年,改授都坪長官司。傳至何學政,清順治十五年,歸附,仍準世襲。

副長官。周如,同。

都素長官司,在府西九十里。明永樂十一年,置長官司於馬口寨。傳至何起圖,清順治十五年,歸附,仍準世襲。

副長官。周之龍,同。

黃道長官司,在府東北一百二十里。明洪武五年,以黃文聽為長官司。傳至黃金印,清順治十五年,歸附,仍準世襲。

副長官。黃士元,同。

施溪長官司,在府北一百四十里。明洪武五年,以劉貴為施溪長官司。傳至劉師光,清順治十五年,歸附,仍準世襲。

銅仁府:

省溪長官司,在府西一百里。明洪武五年,以楊政為省溪長官司。傳至楊秀銘,清順治十五年,歸附,仍準世襲。

副長官。戴子美,同。

提溪長官司,在府西一百四十里。明洪武五年,以楊秀纂為提溪長官司。傳至楊通正,清順治十五年,歸附,仍準世襲。

副長官。張體泰,同。

烏蘿長官司,在府西二百里。始自唐時楊通孫。明洪武五年,改授烏蘿長官司。傳至楊洪基,清順治十五年,歸附,仍準世襲。

副長官。冉天奇,同。

平頭長官司,在府北一百二十里。明洪武二十九年,改授楊正德為平頭長官司。傳至楊昌續,清順治十五年,歸附,仍準世襲。

副長官。田茂功,同。

黎平府:

潭溪長官司,在府西南三十里。明洪武四年,以石平禾為潭溪長官司。傳至石玉柱,清順治十五年,歸附,仍準世襲。

副長官。石巖,同。

八舟長官司,在府北八十里。漢吳昌祚以功授八舟長官司。明洪武四年,仍令吳氏世襲。傳至吳遇主,清順治十五年,歸附,亦準襲職。

龍里長官司,在府西北九十里。明洪武四年,以楊光福為龍里長官司。傳至楊勝梯,清順治十五年,歸附,仍準襲職。

中林長官司,在府西北一百里。明洪武五年,以楊盛賢為中林長官司。傳至楊應詔,清順治十五年,歸附,仍準襲職。

古州長官司,在府西北八十里。元置古州八萬洞長官司,屬思州宣撫司。明洪武五年,以楊秀茂為古州長官司。永樂十年,屬府。傳至楊云龍,清順治十五年,歸附,仍準襲職。

新化長官司,在府北六十里。元時,歐陽明萬以功授軍民長官司。明洪武五年,仍襲前職。傳至歐陽瑾,清順治十五年,歸附,仍準世襲。

歐陽長官司,在府北九十里。明洪武四年,以陽都統為歐陽長官司。傳至陽運洪,清順治十五年,歸附,仍準世襲。

副長官。吳登科,同。

亮寨長官司,在府北一百里。元置。明洪武四年,以龍政忠為本司長官司。傳至龍文炳,清順治十五年,歸附,仍準襲職。

湖耳長官司,在府東北一百二十里。明洪武四年,以楊再祿為本司長官司。傳至楊通乾,清順治十五年,歸附,仍準襲職。

副長官。楊大勛,同。

洪州長官司,在府東一百五十里。元置洪州泊裡等洞軍民長官司。明洪武五年,以李氏為洪州長官司。傳至李煦,清順治十五年,歸附,仍準襲職。

副長官。林起鵬,同。

分管三郎司,在府南三十里。楊世勛襲。清康熙二十三年,改土歸流。

赤谿湳洞司,在府東北二百六十里。楊鳴鸞襲。清康熙二十三年,改土歸流。

水西宣慰司:康熙三年,吳三桂滅安坤,改設四府。二十一年十二月,諭大學士曰:「吳三桂未叛時,征討水西,曾滅土司安坤,其妻祿氏奔於烏蒙,後生子安世宗。朕觀平越、黔西、威寧、大定四府原屬苗蠻,以土司專轄,方為至便。大兵進取雲南,祿氏曾前接濟,著有勤勞,仍復設宣慰使,令世宗承襲。」四十年,總督王繼文以土司安世宗為吏民之害,仍請停襲,地方歸流官管轄。


\end{pinyinscope}