\article{志一百三十五}

\begin{pinyinscope}
○邦交八

奧斯馬加秘魯巴西葡萄牙墨西哥剛果

奧斯馬加即奧地利亞,久互市廣東,粵人以其旗識之,稱雙鷹國。同治八年,遣使臣畢慈來華,介英使阿禮國請立約,並呈其君主敕諭,欲在京議約。總署以在京議約與歷來各國成案不符,應先照會三口通商大臣,由三口通商大臣請旨。奧使遞照會三口通商大臣崇厚以聞。朝議許之,命總理各國事務衙門大臣、兵部尚書董恂會同崇厚辦理。奧使呈所遞條約四十九款,大致均從各國內採集而成。董恂等於應刪應添各節,逐一改定,而奧使於恂所添「商人不準充領事官」一語,不原列入約,於恂所刪傳教一條仍列入約。迭議不決。久之,奧使始允刪傳教一條,而於「商人不準充領事官」一節,仍欲另備照會,於畫押日一同呈遞,許之。遂訂定和約四十五款,通商章程九款,稅則一冊。是年奧夾板船名伊來撤各利亞,用英國旗號,私運外國鹽一百餘包,計重二萬餘斤,進口。天津稅務司函致總署。總署以奧船運鹽進口,顯違條約,應查拏,並知照英領事前往查起。十年九月,奧換約屆期,使臣嘉理治照會總署請換約,特旨派江蘇布政使恩錫赴上海互換。嗣因約本內漢文所載善後章程第五、第八兩款,均有引用條約「第八條」字樣,其奧文內皆誤寫作「第一條」;又稅則進口項下呀闌治木,漢文載明長不過「三十五幅地」,奧文誤寫作「五十五幅地」;又羽綾、羽紗、羽綢、小呢等類,漢文載明「每丈」,奧文誤寫作「每疋」,須更改。至十一年六月始竣事。十一年,奧使照會總署,以接奉本國文,稱去歲本國出有政令,自同治十年七月十七日起,凡量奧斯馬加各樣海船噸數之法,皆與英國丈量噸數之法相同,請劄知總稅務司轉知各口海關遵行。十二年,穆宗親政,奧隨各國公使覲見。

光緒六年,使臣李鳳苞函致李鴻章,稱:「奧君長子明年正月十六日婚期,中國雖未派駐使,宜令鄰邦駐使往賀,以盡友誼。」總署即電飭李鳳苞屆期往賀。十年夏四月,以翰林院侍講許景澄充出使德美和奧大臣,駐德。十三年,代以內閣學士洪鈞。十四年四月,洪鈞赴奧呈遞國書,見奧主於馬加行宮,頌答如禮。冬十月,奧爾而伯納親王來京,欲瞻仰天壇,許之。十六年秋七月,復以許景澄充出使俄德奧和大臣。十七年夏四月,景澄赴奧通問,覲見奧主,奧主為述前歲有兵艦抵華,承中國官員以禮相待,屬為陳謝。九月,奧使畢格哩本覲見上於承光殿。十八年,奧主以西歷六十七年即馬加王位,距今滿二十五年,西俗以為慶事,先期由奧外部通知各國公使詣馬加都城申賀。許景澄備文傳賀,旋即親赴伯達彼斯馬加都城。覲見奧主申謝悃,奧亦發電至京答謝。

二十年四月,許景澄照會奧與俄、德、和等外部,申明總署現章,酌定洋機進口稅文。十月,皇太后六旬壽,奧使隨各使呈遞國書致賀,上見之於文華殿。二十一年,奧主叔父病故,許景澄請旨致唁,許之。二十二年十月,以都察院左都御史楊儒充出使俄奧和大臣。十一月,駐德奧使送節略,稱奧廷擬派瓦耳布倫為駐京專使,請中國國家允認,中國派使駐奧亦如之。二十三年四月,奧使齊幹覲見上於文華殿。二十六年春三月,命內閣學士桂春使俄兼使奧。七月,拳匪之變,奧兵隨德、美、法、英、意、日、俄聯軍入京師。二十八年四月,三品卿吳德章充出使奧國大臣。二十九年,代以山東道員楊晟。三十年十月,奧使齊幹覲見上於皇極殿。三十一年八月,以三品京堂李經邁充出使駐奧大臣。三十二年三月,奧使顧新斯基覲見上於乾清宮。三十三年七月,以外務部參議雷補同充出使駐奧大臣。

秘魯在南亞美利加洲。同治十一年,秘魯國瑪也西船私在澳門拐華民二百餘人,行抵日本橫濱,經日本截留訊辦,知會中國派員前往。時通商大臣何璟派補用同知知縣陳福勛偕英、美兩領事派員前往,旋各運回,並謝日本。

十二年,秘魯遣使來華議立約。已而秘使葛爾西耶到津謁李鴻章,鴻章詰以虐待華工等事,不允相商。秘使旋以本國新立雇工章程,實無凌虐情事,牒鴻章。鴻章覆牒,謂:「貴國新立雇工章程雖尚公道,但查同治八年、十年間,華民公稟內所稱『苛求、打罵、枷鎖、饑寒,雖立合同,而章程虛設,雖曰送回,而限滿無歸』等語,是即保護華工未能照辦之證據也。又來文所載一千八百五十五年八月十四日議立搭客船規,不準載大幫之人。查同治十一年,日本國扣留秘魯瑪也西船,載有拐買華民二百三十人之多,據各國領事公同訊問,船主苛酷相待,飲食不繼,並有割去辮發、鞭打囚禁等事。又據粵海關稅務司報稱,『同治九年,秘魯船一只在澳門販載華工三百十三人,同治十年,秘魯船十三只在澳門販載華工五千九百八十七人,同治十一年,秘魯船十九只在澳門販載華工九千三百八十一人』。此皆系大幫,秘國並不查禁。近又據粵海關稅務司報稱,『本年七月間,廣東省城黃埔河面有秘魯船七只前來招工,因其違背通行章程,諭令驅逐出口』。以上各節,是又帆船禁載大幫華人未能照辦之證據也。查上年中國通行各國照會內載,凡系無約各國,一概不準設局招工,其船只不準搭載華工出洋。即有約各國,亦不準在澳門招工。均經各國知照在案。秘國向系無約之國,照章不準裝載華人出口。乃昨據貴大臣面稱,現載往秘魯華人已有十萬餘人,明系違背公法。況華民在秘魯重受凌虐,曾兩次公稟美國欽差轉達總理衙門,是以日前疊據英、美、法各大臣述知貴大臣欲來華議約,即經總理衙門王大臣照覆各國,以『秘魯向來專以拐販華工為事,華工受盡痛苦,其相待中國情形與別國不同,必須與伊國說明,先將所招華工全數送回中國,並聲明不準招工,方能商議立約,否則實難辦理』等語。想貴大臣必已與聞,無煩贅述。」

旋據函稱遵照總署原議,先將所招華工全數送回中國,自可妥商。鴻章訂期會議,屆時不至,鴻章責之。復請期,鴻章因再約期,至日,秘使偕愛勒謨爾秘妥士來。適同知容閎由美國回津,鴻章令閎與議。秘使將鴻章原函取出,逐條剖辨,謂無苛待情事;又謂中國既令無約之國不準招工,是以本國亟派使前來議約,以後自必照約互相稽查保護。並稱華工送回,可於約內聲明,除華人在秘魯設肆寓居,自不原歸,無庸送回,其餘工人等合同限滿,即令原主送回,分別辦理。容閎因言美國向例,無立合同年限雇工之事。華民在金山等處傭工,去留自便,美官不能勉強勒掯。即有先立合同者,若不原當,隨時將合同繳銷,作為廢紙。秘國亦應照辦。秘使允商辦。鴻章仍以拐去華民為言。秘使怫然,謂即回國。屢議不決。

十三年三月,復與秘使接議,秘使自交所擬條約,鴻章不受。久之,始定查辦華工專條,其文曰「現因秘國地方有華民多名,且有稱華民有受委屈之處。茲會同商訂,先立通商條款,和好往來,庶幾彼此同心。由中國派員往秘,將華民情形澈查,並示諭華工,以便周知一切,秘國無不力助,以禮接待。如查得實有受苦華工,合同年限未滿,不拘人數多寡,均議由委員知照地方官。雇主倘不承認,即由地方官傳案訊斷。若華工仍抱不平,立許上告秘國各大員,再為覆查。凡僑寓秘國,無論何國人民,呈稟式樣最優者,華工應一體均霑其益。自秘國核定此項章程之日起,凡華工合同已經期滿,若合同內有雇主應出回國船腳之議,該工人有原回國者,即當嚴令雇主出資送回。又各華工合同若無送回字樣,合同已滿期,該工人無力自出船資,有原回國者,秘國應將該工人等附華船送回,船資無須工人自備,秘國自行料理」雲云。

復將通商條約十九款及已訂查辦專條改定,因致總署,謂:「在秘使之意及各國公論,彼既允定查辦資遣華工專條,是秘魯已予中國以便宜,我亦當照各國和約,允以一律。現訂通商十九款,大致亦與西約詞意略同。然均經鴻章逐條酌改,如各約篇首所稱『互相較閱,俱屬妥當』或『妥善』字樣,轉覺不妥,茲將『俱屬妥協』四字刪去。各約欽差駐京往來,有彼國而無我國,茲先載明中國欽差。各約領事官無商人不準兼充明文,茲添『不得委商人代理』。各約游歷通商執照,秘使不肯刪通商貨物字樣,茲特添入『貨物應照報單章程辦理』。各約多以英、法文為憑,茲改『彼此各用本國文字,亦可兼看英文』。其餘凡通商、納稅、兵船、商船、控告、詞訟各節,均將中國一面敘入。所最要者,招工流弊無窮。澳門販運已久,華工既在秘國受苦,以前雖允查辦,以後若仍開招,害將何所底止?茲會訂第六款,上半節照美國續約,云『別有招致之法,均非所準』,下復添敘『不準在澳門及各口岸勉強誘騙中國人運載出洋,違者其人嚴懲,船只罰辦』等語。嗣後中國但能照約嚴禁,不獨秘魯不敢違犯,即各國招工之舉,亦得援引辨證。又前訂查辦華工專條,商令派員前往,秘使允即遵照。」

旋派容閎往查辦。容閎查辦訖,報告華工到彼,被賣開山、種蔗,及糖寮、鳥糞島等處虐待情形,合同限內打死及自盡、投火爐糖鍋死者甚多,實可慘憫。會屆換約之期,秘魯遣使臣愛勒謨爾來華求換約。光緒元年,派巡撫丁日昌為換約大臣。日昌謂:「去年中國所以與秘國立約者,因秘國葛使照會內言秘國設有新章新例,保護華民,盡除弊端。乃立約之後,派員前往秘國確查,始知華工受屈,顯與條約內保護優待之例相背。甫經立約,而秘國即種種違約,是不能不加一照會,聲明換約後即當遵約辦理,再不能仍照從前之凌虐。」秘使聞之,不待辭畢,即怫然去。日昌以秘使無禮,因致總署,請暫緩換約。

四年,秘魯因澳門停止招工,香港英總督又申嚴禁,秘魯乃赴廣東省城與美商同孚洋行私立運載華工合同,五年為期,每年得船費洋銀十六萬圓,設局招誘。粵督聞之,即予查禁。秘使詣天津謁鴻章,拒之。時出使大臣為陳蘭彬,雖由美使兼日秘,並未赴秘。七年,以津海關道鄭藻如為出使美日秘大臣。十年五月,始由美赴秘,謁總統遞國書,開辦使署於利馬都城,奏派參贊一員代辦使事。又於嘉理約海口設領事一員,管理華民事宜,仍禁絕招工,並咨請查拿廣州城外私設招工行棧。十二年,鄭藻如歸,迭以傅云龍、張廕桓、崔國因、楊儒充公使。又增設代理領事十,就秘籍中之廉正者充之,遇事報使署,由參贊區處,公使仍不駐秘。二十一年,秘總統即位,各國均有國書致賀,介由美使請總署代達。二十二年,始頒國書。二十三年六月,駐秘代辦李經敘行抵嘉理約,因疫疾盛行恐傳染,阻止入口,從秘制也。久之始聽入。時公使楊儒赴秘遞國書,秘外部先派護衛大臣一員在嘉理約迎伺,隨派火車接至利馬,又派副外部在車站迎伺。遞書日又用宮車迎接。公使遞國書,他國均用軍裝佩劍。中國以秘系民主,沿例用行裝,行鞠躬禮,致頌詞,秘主答頌如禮。

二十四年,利馬華人在香港辦貨,秘駐港領事照驗加戳。向例戳費值百抽一,至是增加,又改用金鎊,比前增逾倍。華商以秘違例,請秘外部飭知港領事照向例核收,又籥請於駐秘代辦謝希傅。於是照會秘外部,謂:「貨單戳費向有定章,值百抽一,又為萬國通例。貨本用金用銀,各國不一,而抽費皆按此為衡。即就利馬論,麥面一項由智利販運者,抽費俱按智洋,洋貨各項由英倫販運者,抽費俱按金錢,載在秘國稅則,眾所共知。乃同一抽費,於智於英皆就地照抽,獨至香港一處忽示歧異,於理不解。或謂香港為英屬口岸,應改金鎊,則粵商貨本亦應升算金錢,方與通商各國一律,應請批示。」秘外部不允批示。旋稱港銀成色太低,換兌金鎊虧損過多。謝希傅告以一律改從金鎊,華商亦所甚原。秘外部始允收費按照貨本,一律改從金鎊。

宣統元年五月,秘工黨仇視駐秘利馬華人暴動,秘政府特頒苛例,令進口華人每名須有英金五百鎊呈驗,始得入口。時出使美日秘大臣伍廷芳赴秘與交涉,先謁總統遞國書,即照會秘外部,謂秘所設苛例,違反兩國所立條約。旋復見秘總統辨論,請廢止飭諭。總統不允。已復由秘外部覆文。秘外部大臣玻立士謂廷芳,請先妥議限制中國工人出口來秘善法,附入條約。廷芳答以章程不應附入條約。玻立士又欲使秘領事有察驗華官所給護照是否合例之權,及到秘時,仍由地方官查驗,方準登岸。廷芳駁之,執不允。廷芳閱草案,又請加「寓智利、厄瓜多、巴拿馬等處華商欲來秘者,可由代理中國領事等官發給護照,以為入境憑據」等語,玻立士允諾。時留秘華人多吸食鴉片,廷芳請秘贊助設法限制,秘總統許之。旋復定議,廷芳與秘外部立廢除苛例證明書九條:一,中國允自限工人來秘;二至六,定非作工之華人往秘護照辦法;七,定非作工者概不限制;八,定免請護照者之資格;九,發照驗照祗須繳費五圓。並停止秘國五月十四號頒發飭諭之效力。時宣統元年七月十三日,即西歷一千九百零九年八月二十號。署押蓋印。

巴西國,南亞美利加洲共和民主新國也。光緒六年,始遣使臣喀拉多來天津,請議立和約。總署請飭南北洋大臣就近商辦。旋派李鴻章為全權大臣與議約。六月一日,喀使抵天津,照會鴻章請立約,並擬先送約稿呈閱。遂訂期接議,研商至再始定約。鴻章因上奏,言:「此次巴西議約,數易其稿。嗣以秘魯條約為底本,刪去招工各條,並參用別國條約,定為十六款。其關系中國權利者,皆力為辯論,變通酌定。如第一款『兩國人彼此皆可前往僑居』句下,添入『須由本人自原』一語,即寓禁阻設法招致之弊。第三款『設立領事官,必須奉到駐扎之國批準文憑,方得視事,如辦事不合,可將批準文憑追回』,本系西國通例。其立法之善有二:一則其人或非平素公正,或與我國向不浹洽,我皆可以不準;一則通商口岸或系新設,人情未安,不欲領事驟至,我亦可以不準。至辦事不合,追回文憑,是予奪之權我亦得而操之。第四款游歷執照一節,洋人游歷各處,多有由領事自填執照,送請關道用印,幾若內地往來,全憑領事作主。今改為『領事照會關道,請領印照』,可稍助地方官之權。第五款遵守專章一節,即是德國新約第一款之義。查『均霑』二字,利在洋人,害在中土,設法防弊,實為要圖。特聲明嗣後如有優待他國利益,彼此須將互相酬報之專條或互訂之專章,一體遵守,方準同霑優待他國之利益,似較周妥。第六款本擬照德國新約,酌用漏報捏報辦法。惟巴約系仿秘魯約本,並無通商詳細章程。若僅添漏報捏報一層,轉恐掛一漏萬。今定為『兩國商人商船,凡在此國通商口岸,即應遵從此國與各國原議續議通行商務章程辦理』。第九、第十、第十一、第十二等款,皆指問案之事。查西國案件,俱由地方官訊斷,領事不得干預。惟中西法律懸殊,各國不能聽地方官審辦,於是領事遂有其權。此次定為『被告所屬之官員專司訊斷,各依本國律例定罪』。蓋被告多系華民,前因會審掣肘,受虧不少。茲由被告所屬之官訊斷,當可持平辦理。又第十一款內『將來另議中西交涉公律,巴西亦應照辦』一節,雖公律驟難定議,究為洋務緊要關鍵,特倡其說,以作權輿。以上各節,皆按照各國約章酌議變通,期歸妥善。至洋藥一項,雖非巴西出產,惟中土受害滋深。今議令巴使知會巴國外部查酌,禁止巴商販賣,先由巴使另備照會存案,臣亦給予照覆。」約既訂,遂於八月初一日會同畫押鈐印。明年三月,喀拉多忽詣李鴻章,謂接本國電報,復請商改。於是增刪巴西原約共十七款,前約正副本作廢。八年四月,換約於上海。八月,巴西贈鴻章寶星,旋答之。

宣統元年,巴西使臣貝雷拉請與中國立公斷專約。先是巴使詣外務部,援照保和會公約,請與中國商訂一公斷條約,並呈所擬洋文約稿。遂派外務部左侍郎聯芳為全權大臣,與貝雷拉議約四條:一,兩國外交官不能和平了結之案,可向海牙所設之常川公斷衙門投控,並請審斷,但須無礙兩國利益及國權榮譽,亦不得干涉第三國之利益;二,公斷員之權限及細則,須臨時由中國皇帝及巴西總統斟酌合宜辦法;三,次約以五年為限,限滿六閱月未聲明作廢者,作為續訂五年,嗣後期限照此計算;四,本約批準後,在巴西京城換約,用華文、葡文、法文三體,而遇礙難解釋之處,則以法文為憑。此約畫押後,因事羈延,未及互換。三年十月,駐法代辦使事戴陳霖與巴西駐法代辦達旒格芬始在巴黎互換。

葡萄牙在歐羅巴極西。明正德年初至中國舟山、寧波、泉州。隆慶初,至廣東香山縣濠鏡請隙地建屋,歲納租銀五百兩,實為歐羅巴通市粵東之始。

清雍正五年夏四月,葡國遣使臣麥德樂表貢方物。抵粵,巡撫楊文乾遣員伴送至京,召見賜宴。於賞賚外,特賜人葠、緞匹、瓷漆器、紙墨、字畫、絹鐙、扇、香囊諸珍,加賞使臣,命御史常保住伴送至澳,遣歸國。麥德樂在澳天主堂,率洋商誦經行禮,恭祝聖壽。乾隆十八年夏四月,葡國遣使巴哲格、伯里多瑪諾入貢奉表,言:「臣父昔年仰奉聖主聖祖皇帝、世宗皇帝備極誠敬。臣父即世,臣嗣服以來,纘承父志,敬效虔恭。臣聞寓居中國西洋人等,仰蒙聖主施恩優眷,積有年所,臣不勝感激歡忭,謹遣一介使臣以申誠敬,因遣使巴哲格等代臣恭請聖主萬安,並行慶賀。伏乞聖主自天施降諸福,以惠小邦。至寓居中國西洋人等,更乞鴻慈優待。再所遣使臣明白自愛,臣國諸務俱令料理,臣遣其至京,必能慰悅聖懷。凡所陳奏,伏祈採納。」

道光二十九年,其酋啞嗎勒為澳民所殺,藉端尋釁,釘關逐役,抗不交租,又屯兵建臺,編牌勒稅。於是澳地關閘以內,悉被侵占,粵省大吏置之不問。

咸豐八年冬十月,葡萄牙遣人來上海請立約。時欽差大臣大學士桂良駐滬,初拒之。旋為奏聞,未許。光緒七年,葡人欲在澳門設立領事,粵督張樹森不允,欲令駐香港領事兼辦。出使大臣曾紀澤謂:「葡人之於澳門,儼然據為己有,唯租住之名尚存。若忽令香港領事兼理,將借香港領事之名,引為澳門領事之據。查澳門本有縣丞等官,似宜仿上海租界之例,設立官職較崇委員,並令督同縣丞辦理交涉事件,庶幾可圖補救。」

十二年,政府因開辦洋藥稅釐並徵新章,總署奏請飭派邵友濂,會同總稅務司赫德,前往香港會商辦法。查知洋藥自印度來華,香港為總匯之區,必須英、葡兩國一律會辦,始能得力。因與澳門總督商緝私辦法。又恐葡為無約之國,遽與商辦,或多要求。於是遣赫德與之電商,擬設稅務司,澳督亦允。乃訂草約四條:一,兩國在京互換通商條約;一,中國準葡國永駐管理澳門;一,葡國允非中國則澳地不讓與他國;一,洋藥稅徵香港如何,會同澳門即類推辦理。當派稅務司金登幹在葡國畫押,並允其派使來華,擬議詳細條約。

粵督張之洞上疏,言:「澳門為香山縣管轄,距省城二百餘里,陸路可通,實為廣東濱海門戶,非如瓊州之孤懸海外,亦非如香港之矗立海中。葡人今因事要求,曲徇其請,遷就立約,實多可慮。挽回補救之策,約有五條:一曰細訂詳約。查簡約雖經金登幹畫押,而詳細條約應刪應增,仍須俟葡使到華,會同總署核議,請旨辦理。其永駐澳門一條,原因協辦藥徵,格外見讓租銀,非畫地歸葡者可比。且約有『不得轉讓他國』之文,可見澳門系中國疆土,讓與葡國居住,應聲明葡國居住免其租銀,不得視為葡國屬地。其不讓於他國一條,應聲明澳門系中國疆土,葡國不得讓於他國。如此,則我有讓地之名,而無損權之實,仍與原約之義毫不相背。一曰畫清界限。有陸界,有水界。何謂陸界?東北枕山,西南濱海,是為澳門。其原立之三巴門、水坑門、新開門舊址,具在志乘可徵,所築砲臺、馬路、兵房,均屬格外侵占。應於立約時堅持圍墻為界,不使尺寸有逾。何謂水界?公法載地主有管轄水界之權,以砲子能及之處為止。兩國土地毗連,中隔小河,則以中流為界。此系指各國自有之地,及征伐所得者而言。澳門本系中國之地,不過準其永遠居住,葡人只能管轄所住之地。宜明立條款,所有水道,準其船只往來,不得援引公法,兼管水界。一曰界由外定。準葡住澳,免其租銀,水界仍是中國所有,自無水界之可分,陸界至舊有圍墻為止。葡人於同治初年將圍墻拆卸,希圖滅跡。然墻可拆,而舊址終不可沒。將來約有成議,似應由粵省督撫臣就近派員會同葡使親往勘驗,詳查舊址,公同立界,俾免影射逾越。一曰核對洋文。查赫德申稱所訂草約四條,與澳門洋報所載者,文義輕重懸殊。第一條派使來華擬議通商條約,洋文內加『須有利益均霑』字樣。第二條葡國永駐澳門管理一切,洋文內加『悉與葡國別處屬地無異』字樣。草約內澳門字樣凡三見,洋文皆作『澳門及澳門附地』。查『附地』二字,意極含糊,不惟將圍墻外至望廈村陰括在內,即附近小島毗連村落,皆可作附地觀。至謂『與葡國別處屬地無異』一語,措詞亦謬。雖洋報所載未盡可信,傳說必非無因。既與總署奏案不符,亦非奉旨準其永駐之本意。應請飭下總署,先將草約漢、洋文詳細核對,以防侵越。一曰暫緩批準。立約雖有成議,批準權在朝廷,此各國之通例。美國煙臺條約,光緒二年所立,有未經批準三條,直至上年始行議定,成案可據。自應明與之約,定約後,須俟稅釐款項大增、拐騙逃亡隨捉隨解諸事皆有明效可徵,兩國始行批準互換,庶彼不得終售其欺。」疏入,報可。

葡使羅沙旋來華詣總署呈節略及地圖。總署王大臣閱圖,與現在葡人所居之地界址不清,多所辯駁。復致北洋大臣李鴻章,派員赴澳確查。張之洞復上疏,請先清界址,緩議條約。略謂:「澳門水陸一帶,大抵有葡人原租之界,有久占之界,有新占之界,有圖占未得之界。除原租之圍墻以內,仍舊聽其居住外,已占者明示限制,未占者力為劃清。」又謂:「洋藥來華,皆徑到香港,分運各口,從無徑運澳門之船。是稽察之關鍵,在香港不在澳門」等語。總署因界址一時難清,仍主先議約、後劃界,久之始定。

於是總署上言曰:「向者總署兩次商辦此事,一議通商訂約,一議給價收回,迄無成說。今因洋藥緝私一事,允其重申前議。並以澳門地方界址一層,從先久經含混,因與葡使羅沙迭商,於約內言明澳門界址俟勘明再定,並聲明未經定界以前,不得有增減改變之事。仍將不得讓與他國一層專立一條,永昭信守。葡使允即電達本國,照此定議。正籌辦間,續接李鴻章函,稱粵省督撫臣分別原租、久占、新占、未占四層辦法。所謂久占者,不知何年。新占者,亦在咸豐、同治以後。委員程佐衡回津面與討論,查圍墻以內為原租,關閘以內皆所久占,譚仔、過路環則為新占。此皆已占者也。關閘以北直達前山,澳西對岸灣子、銀坑各處,遠及東南各島,皆欲占而未占者也。應俟將來派員勘界時隨時斟酌辦理。」尋報可。

嗣因交犯一條,葡使欲照英約載明華人犯罪逃至澳門者,查明實系罪犯交出。總署不允。磋商久之,始允添改華民犯案逃往澳門,官員仍照向來辦法,查獲交出。又稽查洋藥一事,復於專約內添寫「所有澳門出口前往中國各海口之洋藥,必須由督理洋藥之洋員給發準照,一面由該洋員立將轉運出口之準照,轉致拱北關稅務司辦理」。遂定議。共計條約五十四款,及緝私專約三款,當即劃押。是年葡人散鈔單于望廈,不納。明年三月,命李鴻章與葡使在天津換約,復公立換約文憑,華、洋文各一,畫押蓋印蕆事。

是月葡人出關閘外設一路燈,又修復前山營廠卡,張之洞責令撤去。旋據澳酋照稱:「關閘外至北山嶺中間一帶,向為局外之區。建廠須兩國會商,非一國所能擅主,已照會鈞署」雲云。張之洞即致總署,謂:「條約載未定界以前,俱照依現時情形勿動,自系指澳境關閘以內彼所已占者而言。同治元年,葡使來京議約,亦言關閘以外系華官把守,未敢侵及,從無『局外』之說。此次來文,實堪詫異,請折辯。」五月,葡人又欲爭執舵尾山管轄權。張之洞致總署,謂:「舵尾山在十字門小橫琴島上,為香山縣屬,向無葡人居此。此處瘋人得葡人養濟,不過尋常善舉,何得視為管治證據?如各省常有洋人施醫院,豈能即為洋界乎?請嚴切駁復。」

二十七年,與各國修改稅則,各國皆會同簽押,葡不派員。特與照會,葡使仍不至。久之,始派參贊阿梅達來,仍不主改稅則。既又請求澳門對面各島開商埠,復拒絕之。二十八年正月,葡使白朗穀來言:「本國商民原在澳門振興商務,修濬河道。前定和約,已認澳門附近屬地為葡國永居管理,應將此地之界址廣闊等項丈量妥訂。按對面山一島居澳門之西,小橫琴、大橫琴二島居澳門西南,各島系澳門生成屬地,又經和約認明,請會商妥定。」外務部王大臣等復以:「中國邊海島嶼向隸府州縣,從無此島屬於彼島之事,祗能就澳門現管界址照約勘定,不得於界之外另有屬地。」二月初,葡使復來照會,以上年各國公約第六款所載進出口稅則改為切實值百抽五,葡未與議,表明本國人民所運各項貨物,應仍照光緒十三年兩國條約所訂稅則辦理。王大臣等嚴詞駁拒,葡仍請求不已。

初,葡使面稱原將界務暫置不提,但求擴充商務,開具條款,大要照分兩端。如應允改定稅則,稽徵洋藥稅餉,在澳門設立分關,為有益中國之款。在澳門附近任便工程,由澳至廣東省城修造鐵路,為有益葡國之款。王大臣等以澳門附近任便修造工程,仍慮暗侵界址,駁令先行刪除。設關一款,札飭總稅務司赫德核辦。鐵路一款,電咨前兩廣總督陶模、督辦鐵路大臣盛宣懷分別核復。旋據赫德復稱,澳門設關,有裨稅收,但章程必須妥定。陶模復稱由澳至省修造鐵路,於地方情形尚無妨礙。盛宣懷復稱,造路於稅務有益,必須由總公司與之定立合同,不必列入約款。王大臣等得復,復與葡使一再研商,將允造鐵路另用照會聲明,不入約內。葡使亦允從,遂與定議。乃上言曰:「此次葡使來京,意在展拓澳界。磋商十餘次,始將勘界之議,商允停辦。現與議訂條款:第一款聲明舊約照舊遵守。第二款聲明上年各國公約加增稅則,大西洋國均允遵照,並與訂明該國人民所納稅項,不得較別國稍有增減,以預留日後加稅地步。第三、第四款,在澳門設分關一道,以稽查出入澳門洋藥,並徵收各項稅項。該關須在澳門界內。但使稅司稽徵得力,似於餉項不無裨益。第五、第六兩款,均申論設關事宜,章程由兩國酌定。第七款訂約文字。第八、第九款,批準互換各節,皆向來訂約應敘之款。應請簡派大臣,與葡使定期畫押,再將約本進呈,請用御寶,以憑互換。至設立中葡公司,修造由澳門至廣東省城鐵路,地僅二百餘里。現辦粵漢、九廣兩路,已議定通至省城,再添一路,亦藉以擴充商務。既與葡使訂明另用照會為憑,擬俟命下,即將照會互換,仍咨行督辦鐵路大臣盛宣懷與葡詳定合同,以期周妥。」報可。慶親王奕劻旋畫押。

三十年二月,葡駐京使臣白朗穀照稱奉本國諭,改修稅則一事,派使前赴上海畫押,並將光緒二十八年九月新訂增改條款暨是年十二月會訂分關章程條款內之意同語異之處,改為一律。其修改稅則及新定增改條款,並會訂分關章程條款,合訂一本,以歸畫一。葡使赴滬,與商約大臣呂海寰等會晤。海寰等面詢照會內所稱各節,將何者為意同語異,及如何改歸一律之處,詳為解明,以便會同辦理。葡使答以光緒二十八年新定增改條款及會訂分關章程條款,本國議院未經核準,不克互換。是以此次修改商約,另行擬送條款,即將前此條款章程意同語異之處,包括在內。海寰等以葡使晤對之詞與照會外務部文意不符,駁之。並照會詰問葡使,令其明晰照復。葡使旋復,以「本國訓諭,業在外務部聲明:一,本政府準議院所議,給權於駐華公使,新立商約,即照近日各國與中國所立之商約無異。二,現欲請立新約,包括光緒二十八年九月所立之條款,暨是年十二月會訂之專條,但內有更改者,俾中、葡兩國主權免有視為關礙之處。三,至於葡國協助中國防緝走私洋藥一事,奉本國政府訓諭,可將此項緝私之法整頓,以便全免走私。四,因今欲立之新約,應包括光緒二十八年九月所立條款,並十二月所訂專條內之宗旨,或系更改,或系推廣,悉行包括在內。所以本國之意,毋庸將前約核準。」海寰等電詢外務部,復云:「葡使並未向部聲明前約作廢。當日議約,原以分關、鐵路為彼此互換利益。儻不將光緒二十八年之約核準,藉包括為詞,以廢分關之議,則中國亦必將鐵路互換之照會聲明作廢。」海寰等即照部電直告葡使,拒不與議。葡使迭來商懇,以「澳門設立分關,實有礙於本國主權,故議院未能核準。欲明言前約作廢,又有礙於本國體制,故以包括宗旨毋庸核準為詞」。海寰等遂與議訂新約。

初,葡使送來商約款文二十條,海寰等就中摘其不能允者,往返磋商。葡使又請為寓澳華民每年準運米六百萬石,免納稅課,以資食用。海寰等以澳門華民不過十萬人,何至歲需六百萬石?拒之。旋外務部據粵督調查,每年只準運三十萬石。又購米地方,限以廣東一省。葡使不允。久之,始將各款議定。海寰乃入奏曰:「綜計釐訂條約二十款。第一款,聲明舊約照舊遵守。第二款,聲明和議所定加增稅則,葡國允遵照辦。第三款,聲明入澳門洋藥均囤於官棧。每年澳門食用洋藥,定數以外,不得再有搬出。凡報運中國各處,亦應設法以防私行運往。所有應定各項章程,應由彼此兩國商訂。又葡國迅定律例,如有犯此約章,應分別懲處。第四款,澳門水陸地方如何防緝走私,彼此派員會訂查緝之地位,並可行之辦法。第五款,照英約推廣西江各口及廣州府屬各埠行輪,惟須遵守現行一切章程。如不遵守,仍不準照辦。葡國並定律例,分別懲辦。第六款,葡萄牙酒無葡國執照,不得照本約所附稅則納稅。第七款,通商口岸地方居住貿易。第八款,華人入葡國版籍,須專定律例,杜其在內地所享利益,及藉葡國籍以脫卸在華所立有合同責任。第九款,加稅免釐。第十款,發還海關存票。第十一款,釐定國幣。第十二款,禁止嗎啡鴉片。第十三款,振興礦務。第十四款,合股經營。第十五款,保護貨牌及創蓺執照。第十六款,整頓律例。第十七款,籌安民教。第十八款,條約年限。第十九款,本約以英文為準。第二十款,在北京互換。以上各款,為我所側重者,在洋藥緝私一事。葡使立意,約文以渾括為準,免致議院再有疑阻。商酌至再,將詳細辦法另立專章。計釐定第三款專章五條,大旨在洋藥運至澳門,必須囤入官棧。其由棧報運中國,則由彼此會同稽查,必須完清海關稅釐,始準搬出。如不進官棧,私自登岸,按葡律核辦。其由原船私運中國,由拱北關緝辦。並嗣後有應行商酌加添,由澳官與稅務司商訂。第五款專章十五條,在澳門專設躉船,以便由拱北關查驗由澳門來往各處貨物為要義。其一切限制辦法,悉照英約內港行輪章程核議。迭經臣世凱、臣之洞往復籌度,公同斟酌妥善,電請外務部核準,然後與之定議。至陸路稽徵稅項,訂明設在總車站,載入鐵路合同之內。又第三款,澳門食用洋藥定數,恐將來澳督與稅司多少爭執,意見不同,特用照會聲明,可由彼此在北京之代表人細查會定。又籌安民教一款,葡使奉其政府訓條,另備照會聲明,凡有天主教堂在華之他國已經允許者,葡國始可照辦。此會訂約款章程及另備照會之情形也。伏念葡萄牙國以和約未經與議,不認各國修改稅則,而要索澳門分設鐵路與粵漢鐵路相接,是以外務部原議在澳門設關,以為互換利益。今葡國以議院未能核準,前約已不廢而廢,故此次詳訂中國海關在澳門水陸地方查緝洋藥走私辦法權限,以為補救。葡使欲以新約包括前約,誠心相助,妥訂條款章程,雖無設關之名,可收緝私之實。並由臣宣懷與葡使將粵澳鐵路合同,同兩國商董妥議,已將車站徵稅一條列入合同之內,已請外務部核準。忽接來電,謂廣東紳商不允葡運粵米,不能不俯順輿情,令再研商。適葡使急於返國,不能再候,擬將米事留後再議,先將商約暨章程先行畫押。」報可。

三十四年正月,日本船辰丸號密運槍砲彈藥向中國輸入,假泊澳門附近之過路環島東方二海裏地,為中國砲艦所捕獲。日本政府以系葡萄牙領海為詞,葡國政府亦言辰丸碇泊地系葡國領海。於是復議中、葡畫境一事。宣統二年,葡政府派海軍提督瑪喀多,中政府派雲南交涉使高而謙,為畫境全權大臣,會議於香港。葡使初要求澳門半島及拱北、小橫琴、大橫琴、譚仔、過路環諸島,與附近海面,均為葡領,謙不允。又要求譚仔、過路環二島,澳門半島,及拱北、大小橫琴諸島之一部,及附近海面為葡領,謙仍不允,只允譚仔、過路環二島承認為葡領,餘皆不承認。相持四閱月不決。葡使請付萬國和平會議解決,謙又拒之。旋停止會議,移議於北京。甫開議,會葡萄牙革命起,遂輟議,成為懸案。

墨西哥在北亞美利加洲。光緒甲申、乙酉年間,墨以立約招工,來請中國駐美公使楊儒派員赴墨察看情形,擬定約款,電請總署籌辦。久未定。二十三年,駐美公使伍廷芳與墨駐美使臣盧美路重提前議。會盧美路卒,繼使臣阿斯芘羅斯復議此事。久之,始定為二十款。初,廷芳與盧美路議也,已允將前議永行墨圓一節刪除,交犯一款,允照總署來函辦法。至是定議。廷芳乃上奏,言:「查泰西通例,領事初到,須領駐劄之國認準文憑,方得視事。大小各國,無不皆然。中國除巴西約外,各國約內皆無此款。今於第三款內訂明,『領事得有認準文憑,方能視事』;『如辦事不合,違背地方條約,可將認準文憑收回』。將來各國修訂條約,亦可視此為衡。第五款,不準誘拐華人出洋一節,是查照日斯巴尼亞約辦理。墨約之訂,實前任使臣鄭藻如首倡其議。蓋謂『出洋不必禁,誘拐則不可不防,與其受凌虐之後始行設官,不若乘未往之先妥為設法』。現定必須本人情原,不準誘令出洋,則包攬誘拐之風不禁自絕。第六款,中國人民與列國人民一律同霑利益一節,我國人民往來貿易,與別國一律無異,將來開荒種植之事,均可援照各國章程辦理。第八款,原稿『彼此土產稅則未載者,暫時免稅』。承準總理衙門電示,遵即改為『彼此進出口稅均照相待最優之國一律辦理』。此是仿照法、墨商約改訂。第十款,遇有軍務,不準勒令僑民充當兵勇,不得強令捐輸一節,此是仿照英、墨約辦理。第十五款,中國將來議立交涉公律一節,歐、美通例,凡僑居他國人民,遇有控告案件,均歸地方官訊斷。惟中國與各國定約,各歸本國領事訊斷。墨國以利益均霑為詞,不得不暫行照辦。惟於約內聲明,『若中國將來與各國議立交涉公律,以治僑居中國之外國人民,墨民亦應照辦』。第十六款,『凡船到口岸,船上諸色人等如有上岸在二十四點鐘內滋事者,準由地方官訊斷,罰鍰監禁』。此是創給中國官訊問外國人之權。如地方官辦理得宜,他日各國修約,即可循此而推。第十七款,『中國人民有事,在墨國控告,得享權利與墨國或相待最優之國人民無異』一節,查本年五月間,墨國覃壁古埠華民數百人,被工頭凌虐,剋扣工資,具詞呈訴,經臣備文由墨使轉達彼國政府,派員嚴切查辦。惟條約未立,保護莫及。今約內聲明控告事件得享權利,則遇有不平,隨時赴官剖白,於僑居商民不無裨益。以上各款,均經悉心酌定,並將漢文與英、墨文字句一一校對,皆相符合。查墨西哥國地分二十九部。其南部一歲三穫,尤為沃壤。民惰耕作,地利未興。近年新定招人開荒章程,一經開墾,即為永業。內地人稠,時虞艱食,託足海外,謀生日難,有此邦為消納之區,既可廣開利源,又可隱消患氣。歷任使臣均以訂墨約為要務,職此之由。向例草約定後,議約之員,即須會同簽押。臣隨將約本繕就,訂期十一月十二日,率同參贊隨員,將會訂條約漢文、墨文、英文各二分,覆校無訛,與墨國全權大臣阿斯芘羅斯互相畫押蓋印,咨送總理衙門,請旨批行。」報可。

二十八年,伍廷芳據粵商稟,咨外務部,謂:「自上年中墨訂約後,華人由香港搭船赴墨者日多。惟華人由香港附輪,先須假道美國舊金山埠,方能赴墨,殊非便商之道,因美正禁止華工入境故也。擬商明輪船公司,特派數艘由香港逕赴墨國口岸,俾僑民任便往來。現在中國業已換約,華人附搭輪船來往,庶不致有所窒礙。」外務部照會英公使,轉行香港總督,飭知英輪公司照辦。二十九年,出使美日秘古國大臣梁誠咨外務部,請援古巴成案,設總領事官一,兼充參贊,駐墨國薩理那古盧司海口,遇有與外部商辦事件,即可馳赴墨都,並以美使兼攝日、秘、古三國使事。外務部奏請允行。是年,墨派員充駐廣州等處領事官。尋又派領事分駐上海、福州、廈門。是年墨因防疫,禁止華人前往。梁誠與交涉,旋弛禁。墨訂立中國及東方諸國移民入境章程六條,俾共遵守。三十年,梁誠赴墨都遞國書,開辦使署分館。墨亦派使臣酈華來華遞國書,並邀覲見,請頒給墨總統暨其國各執政大臣寶星,許之。三十一年,墨前總統由國民公舉續任六年,墨致國書,由其國駐京公使烏海慕呈遞。尋由外務部擬覆國書。是年,墨開萬國地理會,請中國派員入會,許之。

剛果在亞非利加洲剛果河左右。光緒二十四年六月,遣其使臣餘式爾來華,請訂和好通商之約,許之。先是光緒十一年十一月,剛果國外部大臣伊特倭照會中國,謂:「奉命充外部大臣,原與中國開通往來,遇有交涉事件,必當妥善辦理。尚望貴王大臣推誠相待,以敦睦誼。」至是乃訂簡明條約二條:一,中國與各國所立約內,凡載身家、財產與審案之權,其如何待遇各國者,今亦可施諸剛果自主之國。二,議定中國民人可隨意遷往剛果自主之國境內僑寓居住,凡一切動產不動產,皆可購買執業,並可更易業主。至行船、經商、工藝各事,其待華民與待最優國之民人相同。各大臣先為親筆畫押,蓋用關防,以昭信守。


\end{pinyinscope}