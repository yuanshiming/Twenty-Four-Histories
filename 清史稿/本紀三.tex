\article{本紀三}

\begin{pinyinscope}
太宗本紀二

崇德元年夏四月乙酉,祭告天地,行受尊號禮,定有天下之號曰大清,改元崇德,群臣上尊號曰寬溫仁聖皇帝,受朝賀。始定祀天太牢用熟薦。遣官以建太廟追尊列祖祭告山陵。丙戌,追尊始祖為澤王,高祖為慶王,曾祖為昌王,祖為福王,考謚曰承天廣運聖德神功肇紀立極仁孝武皇帝,廟號太祖,陵曰福陵;妣謚曰孝慈昭憲純德貞順成天育聖武皇后。追贈族祖禮敦巴圖魯為武功郡王,追封功臣費英東為直義公,額亦都為弘毅公,配享。丁亥,群臣上表賀。諭曰:「朕以涼德,恐負眾望。爾諸臣宜同心匡輔,各共厥職,正己率屬,克占忠誠,立綱陳紀,撫民恤眾,使君明臣良,政治咸熙,庶克荷天之休命。」群臣頓首曰:「聖諭及此,國家之福也。」以受尊號禮成,大赦。己丑,多濟裏、扈習征瓦爾喀師還,賞賚有差。朝鮮使臣歸國。初,上受尊號,朝鮮使臣羅德憲、李廓獨不拜。上曰:「彼國王將構怨,欲朕殺其使臣以為詞耳,其釋之。」至是遣歸,以書諭朝鮮國王責之,命送子弟為質。丁酉,敘功,封大貝勒代善為和碩兄禮親王,貝勒濟爾哈朗為和碩鄭親王,多爾袞為和碩睿親王,多鐸為和碩豫親王,豪格為和碩肅親王,岳託為和碩成親王,阿濟格為多羅武英郡王,杜度為多羅安平貝勒,阿巴泰為多羅饒餘貝勒;諸蒙古貝勒巴達禮為和碩土謝圖親王,吳克善為和碩卓禮克圖親王,固倫額駙額哲為和碩親王,布塔齊為多羅札薩克圖郡王,滿硃習禮為多羅巴圖魯郡王,袞出斯巴圖魯為多羅達爾漢郡王,孫杜棱為多羅杜棱郡王,固倫額駙班第為多羅郡王,孔果爾為冰圖王,東為多羅達爾漢戴青,俄木布為多羅達爾漢卓禮克圖,古魯思轄布為多羅杜棱,單把為達爾漢,耿格爾為多羅貝勒,孔有德為恭順王,耿仲明為懷順王,尚可喜為智順王。辛丑,朝鮮使臣置我書於通遠堡,不以歸。札福尼征瓦爾喀師還。

五月丙午,以希福為內弘文院大學士,範文程、鮑承先俱為內秘書院大學士,剛林為內國史院大學士。壬子,貝勒薩哈廉卒,輟朝三日。癸丑,始薦櫻桃於太廟。丁巳,設都察院,諭曰:「朕或奢侈無度,誤誅功臣;或畋獵逸樂,不理政事;或棄忠任奸,黜陟未當;爾其直陳無隱。諸貝勒或廢職業,黷貨偷安,爾其指參。六部或斷事偏謬,審讞淹遲,爾其察奏。明國陋習,此衙門亦賄賂之府也,宜相防檢。挾劾人,例當加罪。餘所言是,即行;所言非,不問。」壬戌,追封薩哈廉為和碩穎親王。己巳,以張存仁為都察院承政,祖澤洪為吏部承政,韓大勛為戶部承政,姜新為禮部承政,祖澤潤為兵部承政,李云為刑部承政,裴國珍為工部承政。都統伊爾登罷。以圖爾格為鑲白旗都統。庚午,武英郡王阿濟格、饒餘貝勒阿巴泰、公揚古利等率師征明。上御翔鳳閣面授方略,且誡諭之。癸酉,師行。

六月甲戌朔,授蒙古降人布爾噶都等世職有差。己卯,命豫親王多鐸管禮部事,肅親王豪格管戶部事。甲申,封薩哈廉子阿達禮為多羅郡王。丙戌,以國舅阿什達爾漢為都察院承政,尼堪為蒙古承政。

秋七月己未,檄外籓蒙古兵征明。辛酉,阿濟格等會師出延慶州,俘人畜一萬五千有奇。

八月丁丑,遣官祭孔子。辛巳,成親王岳託、肅親王豪格以罪降多羅貝勒。癸未,睿親王多爾袞,豫親王多鐸,貝勒岳託、豪格舉師征明。

九月戊申,明兵入A2場,命吳善、季思哈率兵御之。己酉,阿濟格等奏我軍經保定至安州,克十二城,五十六戰皆捷,生擒總兵巢丕昌等人畜十八萬。庚申,伊勒慎等追明兵至娘娘宮渡口,見敵船甚眾,不敢進,奏聞。命宜蓀往援,復遣杜度率師助之。辛酉,蒙古達賴、拜賀、拜音代等自塔山來降。己巳,阿濟格等師還。

冬十月癸酉,多爾袞等師還。丁亥,遣大學士希福等往察哈爾、喀爾喀、科爾沁諸部稽戶口,編佐領,讞庶獄,頒法律,禁奸盜。戊戌,朝鮮國王李倧以書來。卻之。

十一月戊申,復命岳託管兵部事,豪格管戶部事。己酉,衛寨桑等自蒙古喀爾喀部還,偕其使衛徵喇嘛等來貢。辛亥,徵兵外籓。癸丑,諭曰:「朕讀史,知金世宗真賢君也。當熙宗及完顏亮時,盡廢太祖、太宗舊制,盤樂無度。世宗即位,恐子孫效法漢人,諭以無忘祖法,練習騎射。後世一不遵守,以訖於亡。我國嫻騎射,以戰則克,以攻則取。往者巴克什達海等屢勸朕易滿洲衣服以從漢制。朕惟寬衣博鮹,必廢騎射,當朕之身,豈有變更。恐後世子孫忘之,廢騎射而效漢人,滋足慮焉。爾等謹識之。」乙卯,太祖實錄成。乙丑冬至,大祀天於圜丘。以將征朝鮮告祭天地、太廟。己巳,頒軍令,傳檄朝鮮。

十二月辛未朔,外籓蒙古諸王貝勒率兵會於盛京。鄭親王濟爾哈朗留守,武英郡王阿濟格駐牛莊備邊,饒餘貝勒阿巴泰駐噶海城收集邊民防敵。壬申,上率禮親王代善等征朝鮮,大軍次沙河堡,睿親王多爾袞、貝勒豪格分兵自寬甸入長山口。癸酉,遣馬福塔等率兵三百為商賈裝,潛往圍朝鮮國都,多鐸及貝子碩言乇、尼堪以兵千人繼之,郡王滿硃習禮、布塔齊引兵來會。己卯,貝勒岳託、公揚古利以兵三千助多鐸軍。上率大軍距鎮江三十里為營,令安平貝勒杜度、恭順王孔有德等護輜重居後。庚辰,渡鎮江至義州。壬午,上至郭山城。其定州游擊來援,度不敵,自刎死。郭山降。癸未,至定州。定州亦降。乙酉,至安州,以書諭朝鮮守臣勸降。己丑,多鐸等進圍朝鮮國都。朝鮮國王李倧遁南漢山城。多鐸等復圍之,並敗其諸道援兵。辛卯,瓦爾喀葉辰、麻福塔居朝鮮,聞大軍至,以其眾來歸。丁酉,上至臨津江,會天暖冰泮,不可渡,忽驟雨,冰結,大軍畢渡。己亥,命都統譚泰等搜剿朝鮮國都,留蒙古兵與俱。上以大軍合圍南漢城。

是歲,土默特部古祿格楚虎爾,鄂爾多斯部額林臣濟農、臺吉土巴等俱來朝。

二年春正月壬寅,朝鮮全羅道總兵來援,岳託擊走之。遣英俄爾岱、馬福塔齎敕諭朝鮮閣臣,數其前後敗盟之罪。甲辰,大軍渡漢江,營於江滸。丁未,朝鮮全羅、忠清二道合兵來援,多鐸、揚古利擊走之。揚古利被創卒。庚戌,多爾袞、豪格軍克長山,連戰皆捷,以兵來會,杜度等運砲車亦至。朝鮮勢益蹙,李倧以書數乞和。上許其出降。倧上書稱臣,逡巡不敢出。壬戌,多爾袞軍入江華島,得倧妻子,護至軍前。復諭倧曰:「來則室家可完,社稷可保,朕不食言,否則不能久待。」倧聞江華島陷,妻子被俘,南漢城旦夕且下,乃請降。庚午,朝鮮國王李倧率其子★及群臣朝服出降於漢江東岸三田渡,獻明所給敕印。上慰諭賜坐,還其妻子及群臣家屬,仍厚賜之。命英俄爾岱、馬福塔送倧返其國都,留其子★、淏為質。

二月壬申,班師。貝子碩託、恭順王孔有德等率朝鮮舟師取明皮島。朝鮮國王李倧表請減貢額。詔免丁丑、戊寅兩年貢物,自己卯秋季始,仍貢如額。甲戌,諭多爾袞等禁掠降民,違者該管官同罪。辛卯,上還盛京。癸巳,諭戶部平糶勸農。

三月甲辰,殺朝鮮臺諫官洪翼漢、校理尹集、修撰吳達濟,以敗盟故。丁未,武英郡王阿濟格率師助攻皮島。戊午,罷蓋州城工。

夏四月己卯,睿親王多爾袞以朝鮮質子李★、李淏及朝鮮諸大臣子至盛京。辛巳,阿濟格師克皮島,斬明總兵沈世魁、金日觀。甲申,安平貝勒杜度率大軍後隊還。丁酉,命固山貝子尼堪、羅託、博洛等預議國政。增置每旗議政大臣三人,集★群臣諭之曰:「向者議政大臣額少,或出師奉使,而朕左右無人,卑微之臣,又不可使參國議。今特擇爾等置之議政之列,當以民生休戚為念,慎毋怠惰,有負朝廷。前蒙古察哈爾林丹悖謬不道,其臣不諫,以至失國。朕有過失,爾諸臣即當面諍。使面從而退有後言,委過於上,非純臣也。」又諭曰:「昔金熙宗循漢俗,服漢衣冠,盡忘本國言語,太祖、太宗之業遂衰。夫弓矢我之長技,今不親騎射,惟耽宴樂,則武備浸弛。朕每出獵,冀不忘騎射,勤練士卒。諸王貝勒務轉相告誡,使後世無變祖宗之制。」

閏四月癸卯,蒙古貢異獸,名齊赫特。壬子,武英郡王阿濟格師還。

五月庚午,朝鮮國王李倧遣使奉表謝恩贖俘獲。丁亥,遣朝鮮從征皮島總兵林慶業歸國,以敕獎朝鮮王。丁酉,章京尼堪等征瓦爾喀,降之,師行逕朝鮮咸鏡道,凡兩月始達,至是還。

六月辛丑,授喀喇沁歸附人阿玉石等官。明千總王國亮、都司胡應登、百總李忠國等自海島來降。莽古爾泰子光袞獲罪,伏誅。乙卯,諭曰:「頃朝鮮之役,兵行無紀,見利即前,竟忘國憲。自今宜思所以宣布法紀修明典制者。」丙辰,以臣朝鮮,克皮島,祭告太廟、福陵。丁巳,朝鮮國王李倧請平值贖俘,不許。甲子,論諸將征朝鮮及皮島違律罪。禮親王代善論革爵,宥之。鄭親王濟爾哈朗以下論罰有差。

秋七月己巳,遣喀凱等分道征瓦爾喀。癸酉,戶部參政恩克有罪,伏誅。辛巳,誡諭漢官以空言欺飾者。智順王尚可喜自皮島師還。壬午,大赦。癸未,優恤朝鮮、皮島陣亡將士揚古利等,贈官襲職有差。乙酉,明都司高繼功等自石城島來降。庚寅,追封皇后父科爾沁貝勒莽古思為和碩福親王。壬辰,以朝鮮及皮島之捷宣諭祖大壽。乙未,分漢軍為兩旗,以總兵官石廷柱、馬光遠為都統,分理左右翼。

八月丙申朔,再恤攻皮島、朝鮮陣亡將士洪文魁等,贈官襲職有差。癸丑,貝勒岳託以罪降貝子,罰金,解兵部任。丙辰,命睿親王多爾袞、饒餘貝勒阿巴泰築都爾鼻城。己未,遣阿什達爾漢等往蒙古巴林、札魯特、喀喇沁、土默特、阿魯諸部會理刑獄。

九月辛未,出獵撫安堡,以書招明石城島守將沈志祥。己丑,兵部參政穆爾泰以罪褫職。貝勒豪格以逼勒蒙古臺吉博洛罪,罰金,罷管部務。

冬十月乙未朔,初頒滿洲、蒙古、漢字歷。丙午,厄魯特顧實車臣綽爾濟遣使來貢,厄魯特道遠,以元年遣使,是年冬始至。庚申,遣英俄爾岱、馬福塔、達云齎敕冊封李倧為朝鮮國王。

十一月庚午,祀天於圜丘。朝鮮國王李倧遣使來貢,復表請歸其世子,並陳國中災變困窮狀。上不許,敕諭賜賚之。丁丑,烏硃穆秦濟農聞上善養民,率貝勒等舉國來附。癸未,追封揚古利為武勛王。庚寅,出獵打草灘。

十二月甲辰,葉克書、星訥率師征卦爾察。癸丑,徵瓦爾喀諸將奏捷。戊午,蒿齊忒部貝勒博羅特、托尼洛率屬來歸。阿濟格遣丹岱等敗明兵於清河。

是歲,虎爾哈部托科羅氏、克益克勒氏、耨野勒氏,黑龍江索倫部博穆博果爾,黑龍江巴爾達齊,精格里河扈育布祿俱來朝。

三年春正月辛未,命貝子岳託仍為多羅貝勒,管領旗務。丁亥,以德穆圖為戶部承政。甲午,皇第九子生,是為世祖章皇帝。

二月丁酉,親征喀爾喀,豫親王多鐸、武英郡王阿濟格從,禮親王代善、鄭親王濟爾哈朗、睿親王多爾袞、安平貝勒杜度居守。丁未,次喀爾占,外籓諸王貝勒等以師來會。喀爾喀聞之,遁去。上行獵達爾那洛湖西,駐蹕。乙卯,次奎屯布喇克。庚申,明東江總兵沈志祥率石城島將佐軍民來降。壬戌,遣勞薩以書告明宣府守臣趣互市,且以歲幣歸我。

三月甲子朔,次博碩堆,命留守諸王築遼陽城。甲戌,次義奚里。庚辰,至登努蘇特而還。壬午,次上都河源,河西平地湧泉高五尺。

夏四月甲午朔,次布克圖裏,葉克書等征黑龍江告捷。乙未,至遼河。丁酉,次杜棱城,明山海關太監高起潛遣人詭議和。戊戌,次札哈納里忒。己亥,次察木哈。庚子,次俄嶽博洛。都爾鼻城工竣,改名屏城。辛丑,杜爾伯特部卦爾察札馬柰等來朝貢。壬寅,至遼陽,閱新城。乙巳,上還盛京。葉克書、星訥征黑龍江師還。癸丑,命明降將沈志祥以其眾志成城居撫順。甲寅,尼噶裡等征虎爾哈師還。

五月癸酉,修盛京至遼河道路,以睿親王多爾袞、饒餘貝勒阿巴泰董其役。乙亥,禮親王代善屬下人覺善有罪,鄭親王濟爾哈朗等請誅之,議削代善爵。以細故不許,並貸覺善。

六月庚申,始設理籓院,專治蒙古諸部事。

秋七月壬戌朔,諭諸王大臣曰:「自古建國,皆立制度,辨等威。今親王、郡王、貝勒、貝子、公主、額駙名號等級,均有定制,乃皆不遵行,違棄成憲,誠何心耶?昔金太祖、太宗兄弟一心,克成大統。朕當創業之時,爾等顧不能同心體國恪守典常乎?」諸王皆引罪。丁卯,喀爾喀使臣達爾漢囊蘇喇嘛歸,諭之曰:「朕以兵討不庭,以德撫有眾。天以蒙古諸部與朕,爾喀爾喀乃興兵犯歸化,甚非分也。爾不獲已,有逃竄偷生耳。爾所能至,我軍豈不能至?其速悔罪來歸,否則不爾宥也。」壬申,達雅齊等往明張家口議歲幣及互市。丁丑,諭禮部曰:「凡有不遵定制變亂法紀者,王、貝勒、貝子議罰,官系三日,民枷責乃釋之。出入坐起違式,及官階名號已定而仍稱舊名者,戒飭之。有效他國衣冠、束發裹足者,治重罪。」又諭大學士希福等曰:「朕不尚虛文,惟務實政。今國家殷富,政在養民。凡新舊人內窮困無妻孥馬匹者,或勇敢可充伍、以貧不能披甲者,許各陳訴,驗實給與。」禁以陣獲良家子女鬻為樂戶者。丙戌,更定部院官制,專設滿洲承政,以阿拜為吏部承政,英俄爾岱為戶部承政,滿達爾漢為禮部承政,宜蓀為兵部承政,郎球為刑部承政,薩木什喀為工部承政,貝子博洛為理籓院承政,阿什達爾漢為都察院承政。命布顏為議政大臣。

八月甲午,禮部承政祝世昌以罪褫職,謫戍邊外。丙申,吳拜、沙爾虎達連擊敗明兵於紅山口、羅文峪,又敗其密雲兵,殲之。丁酉,地震。戊申,授中式舉人羅碩等十名佐領品級,免四丁,一等至三等秀才授護軍校品級,免二丁,各賜朝衣綢布有差,未入部者免一丁。庚戌,阿魯阿霸垓部額齊格諾顏等、蒿齊忒部博洛特諾木齊等並來朝貢。癸丑,以睿親王多爾袞為奉命大將軍,統左翼兵,貝勒豪格、阿巴泰副之,貝勒岳託為揚武大將軍,統右翼兵,貝勒杜度副之,分道伐明。諭之曰:「主帥為眾所瞻,自處以禮,而濟之以和,則蒙古、朝鮮、漢人之來附者,自心悅而誠服。若計一己之功,而不恤國之名譽,非所望焉。」丁巳,岳託、杜度師行。己未,以巴圖魯準塔為蒙古都統。

九月癸亥,多爾袞、豪格、阿巴泰師行。壬申,上親向山海關以撓明師。徵孔有德、耿仲明、尚可喜兵。丁丑,定優免人丁例。丁亥,幸演武場,閱兵較射。

冬十月丁酉,岳託師自墻子嶺入,遇明兵。明總兵官吳國俊敗走。戊戌,多爾袞軍入青山關。己亥,上統大軍發盛京。甲辰,次渾河,科爾沁、喀喇沁各率兵來會。丙午,遣沙爾虎達等率師趣義州。己酉,命濟爾哈朗、多鐸各率師分趣前屯衛、寧遠、錦州,上親向義州。辛亥,索海率師圍大凌河兩岸十四屯堡。壬子,上次義州,遣孔有德、耿仲明、尚可喜、石廷柱、馬光遠以砲克其五臺。乙卯,次錦州。丙辰,多鐸克桑噶爾寨堡,殺其守將。孔有德等攻石家堡、戚家堡,並克之。戊午,孔有德等攻錦州西臺,臺中砲藥自發,臺壞,克之。

十一月己未朔,多鐸將與濟爾哈朗合師徑中後所,會祖大壽往援北京,乘夜襲我師。庚申,多鐸、濟爾哈朗還至中後所。大壽懼,不敢出。石廷柱、馬光遠攻李雲屯、柏士屯、郭家堡、開州、井家堡,俱克之。孔有德招降大福堡,又攻大臺,克之。辛酉,大軍入山海關。壬戌,上次連山。癸亥,攻五里河臺,明守備李計友等率眾降。丁卯,上至中後所,遇祖大壽收兵入城。使告之曰:「別將軍數載,甚思一見。至於去留,終不相強。將軍與我角勝,為將之道應爾。朕不以此介意,亦原將軍勿疑。」戊辰,再遣使諭大壽,皆不答。己巳,濟爾哈朗克摸龍關及五里堡屯臺。庚午,班師。庚辰,次圖爾根河,遣蒙古軍各歸其部。丙戌,上還京。丁亥,地震。

十二月戊戌,刑部承政郎球有罪解任,以都察院參政索海代之。

是歲,土默特部古祿格,杜爾伯特部卦爾察札馬奈,席北部阿拜、阿閔,兀札喇部井瑙、馬考、札柰、桑吉察,鄂爾多斯部額林臣濟農,阿魯阿霸垓部額齊格諾顏,蒿齊忒部博洛特諾木齊,黑龍江博穆博果爾、瓦代噶凌阿均來朝貢。

四年春正月乙丑,貝子碩託以罪降輔國公。甲戌,皇第三女固倫公主下嫁科爾沁額駙祁他特。己卯,封沈志祥為續順公。蒙古喇克等自錦州來歸。丁亥,蘇尼特部臺吉噶布褚等率部人來歸。是月,明以洪承疇總督薊、遼。

二月丁酉,命武英郡王阿濟格率師征明。壬寅,上親統大軍繼之。丙午,次翁啟爾渾。阿濟格遣使奏捷。蒙古奈曼等部率十三旗兵來會。庚戌,營松山。孔有德、耿仲明、尚可喜、石廷柱、馬光遠以砲擊城外諸臺,克之。遣塔布囊布顏率師防烏欣河口。壬子,上登松山南岡,授諸將方略。癸丑,列砲攻城,雉堞悉毀。明副將金國鳳拒守不下。上命豎雲梯急攻之。代善請俟明日,上從之。明人復完城堞,我軍不得入。乙卯,命阿濟格、尼堪、羅托等師圍塔山、連山。

三月戊午朔,明軍援杏山,我兵邀擊之,斬五十人。己未,穿地道攻松山城。乙丑,命納海等馳略杏山。石廷柱、馬光遠攻觀民山臺,降之。丙寅,多爾袞、杜度等疏報自北京至山西界,復至山東,攻濟南府破之,蹂躪數千里,明兵望風披靡,克府一州三縣五十七,總督宣、大盧象升戰死,擒德王硃由賸、郡王硃慈漻、奉國將軍硃慈黨、總督太監馮允升等,俘獲人口五十餘萬,他物稱是。是役也,揚武大將軍貝勒岳託、輔國公瑪瞻卒於軍。上聞震悼,輟飲食三日。乙亥,多爾袞、杜度又報自遷安縣出青山關,遇明兵,二十四戰皆勝。己卯,復攻松山城。明太監高起潛、總兵祖大壽自寧遠遣副將祖克勇、徐昌永等率兵趨錦州。阿爾薩蘭等擊敗之。上聞,馳赴錦州督師,斬徐昌永於陣,擒祖克勇。甲申,解松山圍。乙酉,駐錦州。多爾袞等師還盛京。

夏四月戊子朔,阿濟格略連山。壬辰,會於錦州。癸巳,渡大凌河駐蹕。己亥,杜度等師還。辛丑,上還盛京,哭岳託而後入,輟朝三日。戊申,以庫魯克達爾漢阿賴、馬喇希為蒙古都統。甲寅,以索渾、薩璧翰為議政大臣。丙辰,追封多羅貝勒岳託為多羅克勤郡王。

五月戊午,以貝子篇古有罪,削爵。己未,鄭親王濟爾哈朗率兵略錦州、松山、杏山。辛酉,蘇尼特臺吉莽古斯、俄爾寨率眾來歸。丁卯,席特庫、沙爾虎達等敗明兵於錦州。辛未,濟爾哈朗奏入明邊,九戰皆捷。丙子,濟爾哈朗師還。庚辰,以鎮國公艾度禮為都統。辛巳,召豫親王多鐸數其罪,宥之,惟坐其徵明失利,及不親送睿親王出師,降多羅貝勒。

六月戊子,蒙古阿蘭柴、桑噶爾寨等告岳託生前與其妻父瑣諾木謀不軌。代善、濟爾哈朗、多爾袞皆請窮治。上以岳託已死,不問,並貸瑣諾木勿治。庚寅,遣馬福塔、巴哈納冊封朝鮮國王李倧妻趙氏為朝鮮王妃,其長子★為世子。丙申,分漢軍為四旗,以石廷柱、馬光遠、王世選、巴顏為都統,改纛色。辛亥,焚哈達、葉赫、烏喇、輝發前所受明敕書於篤恭殿。壬子,以伊爾登、噶爾馬為議政大臣,星訥兼議政大臣。

秋七月丁巳,遣官賚書與明帝議和,並令硃由賸等各具疏進,許其議成釋還。辛未,朝鮮國王李倧克熊島,執加哈禪來獻。乙亥,諭滿、漢、蒙古有能沖鋒陷陣先登拔城者,以馬給之。

八月己丑,授宗室固山貝子、鎮國公、輔國公、鎮國將軍、奉國將軍等爵有差。甲午,命貝勒豪格管戶部事,杜度管禮部事,多鐸管兵部事,薩爾糾等率兵征庫爾喀部。乙巳,歸化城土默特諸章京以所得明歲幣來獻。

九月乙卯朔,以孫達理等八十三人從睿親王入關有功,各授官有差,賜號巴圖魯。乙丑,都統杜雷有罪,褫職。己巳,復封貝勒豪格為和碩肅親王。癸酉,阿濟格、阿巴泰、杜度率兵略錦州、寧遠。甲戌,封嶽託子羅洛宏為多羅貝勒。丙子,以宗室賴慕布、杜沙為議政大臣,英俄爾岱為都統,馬福塔為戶部承政。

冬十月丙戌,豪格、多鐸率兵復略錦州、寧遠。庚寅,蘇尼特部墨爾根臺吉騰機思等率諸貝勒、阿霸垓部額齊格諾顏等各率部眾,自喀爾喀來歸。辛卯,出獵哈達。癸丑,以劉之源為都統,喀濟海為議政大臣。

十一月甲寅朔,豪格疏報參領阿藍泰率蒙古人來歸,遇明兵於寧遠北岡,擊敗之,斬明總兵金國鳳。辛酉,遣索海、薩木什喀等征索倫部。丁卯,出獵葉赫。

十二月甲午,上還京。

是歲,黑龍江額納布、墨音、額爾盆等,喀爾喀部土謝圖、俄木布額爾德尼等,喀爾喀、蘇尼特、烏硃穆秦、科爾沁、克西克騰、土默特諸部,遣使俱來朝貢。

五年春正月甲子,命朝鮮質子李★歸省父疾,仍令遣別子及★子來質。遣翁阿岱、多濟裡等戍錦州。

閏正月癸未朔,令各旗都統分巡所屬屯堡,察窮民,理冤獄。

二月丙辰,遣多濟裏以寧古塔兵三百往征兀札喇部。丁巳,戶部承政馬福塔卒,以車爾格代之,覺羅錫翰為工部承政。丙寅,朝鮮國王第三子橑來質。

三月丙戌,遣勞薩、吳拜等略廣寧。己丑,勞薩、吳拜以逗遛議罰有差。薩木什喀等征虎爾哈部,克雅克薩城。己亥,命濟爾哈朗、多鐸築義州城,駐兵屯田,進逼山海關。辛丑,戶部參政碩詹徵朝鮮水師糧米赴大凌、小凌二河。乙巳,索海、薩木什喀徵索倫部奏捷。

夏四月壬子朔,罷元旦、萬壽諸王貝勒獻物。乙亥,索海、薩木什喀徵索倫師還,上宴勞於實勝寺。庚辰,上視師義州。

五月癸未,渡遼河。乙酉,碩詹以朝鮮水師至。癸巳,上至義州。丁酉,蒙古多羅特部人蘇班代等自杏山遣人約降。上命濟爾哈朗等率軍迎之,戒曰:「此行勿領多人,敵見我兵少,必來拒戰。我分兵為三,以前隊拒戰,後二隊為援。」至杏山,祖大壽果遣劉周智、吳三桂列陣逼我。濟爾哈朗等偽卻,縱兵反擊,大敗之。戊戌,命勞薩、吳拜等略海邊。索倫部三百三十七戶續來降。壬寅,上率師攻克五里臺。乙巳,以紅衣砲攻錦州。丁未,刈其禾而還。庚戌,駕還京。

六月乙丑,多爾袞、豪格、杜度、阿巴泰、濟爾哈朗等屯田義州。戊辰,朝鮮世子李★至。先是,朝鮮遣總兵官林慶業等載米同我使洪尼喀等自大凌河運三山島,遇風,覆沒者半,與明兵戰又失利,乃命陵輓至蓋州、耀州,留其兵千五百人於海州。癸酉,多濟裏、喀柱征兀札喇部師還。遣朝鮮王次子李淏歸省。

秋七月庚辰朔,敘徵索倫功,索海等賞賚進秩有差。癸未,定徵索倫違律罪,薩木什喀等黜罰有差。乙酉,多爾袞等奏克錦州十一臺,請分兵為兩翼屯駐。癸巳,明總督洪承疇以兵四萬壁杏山,遣騎挑戰,多爾袞等擊敗之。乙未,遣吳拜往助多爾袞軍。丙午,席特庫、濟席哈等率師征索倫部。上幸安山溫泉。己酉,多爾袞奏敗明兵於錦州,杜度又敗之寧遠。

八月己未,遣希福等至張家口互市。乙亥,多爾袞奏敗明兵於錦州,又敗之大凌河。

九月乙酉,上還宮。丙戌,命濟爾哈朗、阿濟格、阿達禮、多鐸、羅洛宏代圍錦州、松山。辛卯,多爾袞奏敗明兵於松山。癸卯,重修鳳凰城。

冬十月壬戌,遣英俄爾岱等往朝鮮責罪。壬申,萬壽節,大赦。

十一月戊寅朔,詔免朝鮮歲貢米十之九。乙酉,濟爾哈朗奏敗明兵於塔山、杏山及錦州城下。癸巳,阿敏卒於幽所。戊戌,朝鮮國王次子李淏來質。

十二月庚戌,命多爾袞、豪格、杜度、阿巴泰代圍錦州。己未,遣朝鮮國王三子李橑歸。席特庫、濟席哈徵索倫部,擒博穆博果爾,俘九百餘人。壬申,英俄爾岱等至自朝鮮,械系其尚書金聲黑尼等四人以歸。

是歲,喀爾喀部查薩克圖遣使來朝貢。

六年春正月庚辰,朝鮮國王李倧上表謝罪。壬辰,席特庫、濟席哈等師還。癸巳,晉席特庫為三等總兵官。甲午,皇四女固倫公主雅圖下嫁科爾沁卓禮克圖親王吳克善子弼爾塔噶爾額駙。丁酉,二等副將勞薩有罪,革碩翁科羅巴圖魯號,降一等參將。

二月己未,以八旗佐領下人多貧乏,令戶部察明奏聞。諭佐領毋沉湎失職。其有因飲酒失業者四十八人並解任。諭諸王大臣教子弟習射。丙寅,多爾袞等奏敗明兵。

三月己卯,濟爾哈朗等代圍錦州。丁酉,降和碩睿親王多爾袞、肅親王豪格為多羅郡王,多羅貝勒阿巴泰、杜度以下罰銀有差。是時,祖大壽為明守錦州,屢招之不應。上令諸王迭出困之。而多爾袞等駐營錦州三十里外,又時遣軍士還家,故有是命。己亥,遣朝鮮總兵柳琳等率兵助濟爾哈朗軍。壬寅,濟爾哈朗奏克錦州外城。初,我軍環錦州而營,深溝高壘,絕明兵出入,城中大懼。蒙古貝勒諾木齊、臺吉吳巴什等請降,且約獻東關為內應。祖大壽覺之,謀執吳巴什等。於是諸蒙古大譟,與明兵搏戰。我軍自外應之,遂克其外城。大壽退保內城。甲辰,諾木齊、吳巴什等以蒙古六千餘人來歸,至盛京。

夏四月丁未,遣阿哈尼堪等率兵詣錦州助濟爾哈朗軍。濟爾哈朗奏敗明援兵於松山。庚戌,遣孔有德、尚可喜助圍錦州。多爾袞等聞錦州蒙古降,請效力贖罪。不許。

五月丁丑,明總督洪承疇以兵六萬援錦州,屯松山北崗。濟爾哈朗等擊走之,斬首二千級。丁亥,索倫部巴爾達齊降。己丑,遣希福等閱錦州屯營濠塹。壬寅,諭駐防歸化城都統古祿格等增築外城,建敵樓,浚深濠,以備守御。

六月丁未,命多爾袞、豪格代圍錦州。辛酉,濟爾哈朗、多爾袞等合軍敗明援兵於松山。丙寅,遣學士羅碩以祖澤潤書招祖大壽。庚午,多爾袞等又奏敗明援兵於松山。

秋七月戊寅,賜中式舉人滿洲鄂謨克圖、蒙古杜當、漢人崔光前等朝衣各一襲,一二三等生員緞布有差。甲申,遣孔有德、耿仲明、尚可喜下副都統率兵助圍錦州。乙酉,議圍錦州功罪,親王以下賞罰有差。

八月甲辰朔,敘克錦州外城諸將功,晉鰲拜、勞薩、伊爾登等秩,復勞薩碩翁科羅巴圖魯號。乙巳,我軍與明合戰,明陽和總兵楊國柱敗死。祖大壽自錦州分所部為三,突圍不得出。丁未,封烏硃穆秦部多爾濟濟農為和碩蘇勒親王,阿霸垓部多爾濟額齊格諾顏為卓禮克圖郡王。丁巳,上以明洪承疇、巡撫邱民仰等援錦州兵號十三萬,壁松山,上親率大軍御之。濟爾哈朗留守。諸王、貝勒、大臣以明兵勢眾,勸上緩行。上笑曰:「但恐彼聞朕至,潛師遁耳。若不去,朕破之如摧枯拉朽也。」遂疾馳而進。戊午,渡遼河。洪承疇以兵犯我右翼,豪格擊敗之。壬戌,上至戚家堡,將赴高橋,召多爾袞以兵來會。多爾袞請駐蹕松、杏間。上從之,幸松山。明以一軍駐乳峰山,由乳峰至松山,列步軍七營,騎兵則環城東西北,壁壘甚堅。我師自烏欣河南山至海,橫截大路而軍。上謂諸將曰:「敵眾,食必不足,見我斷其餉道,必無固志,設伏待之,全師可覆也。」癸亥,明兵來犯,擊卻之。又敗之塔山,獲其積粟十二屯。甲子,明兵再犯,又卻之。時承疇以餉乏,欲就食寧遠。上知其將遁,分路設伏,戒諸將嚴陣以待,扼其歸寧遠及奔塔山、錦州路。是夜,明吳三桂等六總兵果潛師先奔,昏黑中為我伏兵所截,大潰。惟曹變蛟、王廷臣返松山。乙丑,又克其四臺。王樸、吳三桂奔杏山。曹變蛟棄乳峰山,乘夜襲上營,力戰,變蛟中創走。己巳,吳三桂、王樸自杏山奔寧遠,遇我伏兵,又大敗之,三桂、樸僅以身免。是役也,斬首五萬,獲馬七千,軍資器械稱是。承疇收敗兵萬餘人入松山,嬰城守,不能戰。我軍遂掘壕圍之。是日,札魯特部桑噶爾以兵至。

九月乙亥,科爾沁卓禮克圖親王吳克善以兵至。命多爾袞、豪格分兵還守盛京。戊寅,略寧遠。乙酉,關雎宮宸妃疾。上將還京,留杜度、阿巴泰等圍錦州,多鐸、阿達禮等圍松山,阿濟格等圍杏山。丙戌,駕還。庚寅,宸妃薨。辛卯,上還京。

冬十月癸卯朔,日有食之。甲辰,遣阿拜駐錦州南乳峰山。丁未,遣孔有德、耿仲明、尚可喜等助圍錦州。己巳,追封宸妃為元妃,謚敏惠恭和。壬申,封蘇尼特墨爾根臺吉騰機思為多羅墨爾根郡王。

十一月乙亥,命多爾袞、羅託、屯齊駐錦州,豪格、滿達海等駐松山。

十二月甲寅,濟爾哈朗、多爾袞奏敗洪承疇於松山。

七年春二月癸卯,上出獵葉赫。戊申,明德王硃由賸卒,以禮葬之。戊午,阿濟格奏敗明兵於寧遠。辛酉,豪格、阿達禮、多鐸、羅洛宏奏拔松山,擒明總督洪承疇,巡撫邱民仰,總兵王廷臣、曹變蛟、祖大樂,游擊祖大名、大成等。先是,承疇援絕,屢突圍不得出,其副將夏承德約降,且請為內應,以子夏舒為質。戊午夜半,豪格等梯城破之。捷聞,上以所俘獲分賚官軍,收軍器貯松山城。壬戌,上還宮。

三月癸酉,殺邱民仰、王廷臣、曹變蛟。諭洪承疇、祖大樂來京,而縱大名、大成入錦州。己卯,克錦州,祖大壽以所部七千餘人出降。乙酉,阿濟格等奏明遣職方郎中馬紹愉來乞和,出明帝敕兵部尚書陳新甲書為驗。上曰:「明之筆札多不實,且詞意誇大,非有欲和之誠。然彼真偽不可知,而和好固朕夙原。朕為百萬生靈計,若事果成,各君其國,使民安業,則兩國俱享太平之福。爾等以朕意傳示之。」乙未,諭多爾袞、豪格駐杏山、塔山,濟爾哈朗、阿濟格、阿達禮等還京。

夏四月丁未,敕諭吳三桂等降。庚戌,大小二日並出,大者旋沒。辛亥,濟爾哈朗、多爾袞、豪格等奏克塔山。甲子,奏克杏山。毀松山、杏山、塔山三城。濟爾哈朗等班師。以阿巴泰守錦州。

五月己巳朔,濟爾哈朗等奏明遣馬紹愉來議和,遣使迓之。癸酉,洪承疇、祖大壽等至,入見請死。上赦之,諭以盡忠報效,承疇等泣謝。上問承疇曰:「明帝視宗室被俘,置若罔聞。陣亡將帥及窮蹙降我者,皆孥戮之。舊規乎?抑新例乎?」承疇對曰:「昔無此例,近因文臣妄奏,故然。」上曰:「君暗臣蔽,枉殺至此。夫將士被擒乞降,使其可贖,猶當贖之,奈何戮其妻子!」承疇曰:「皇上真仁主也。」戊寅,禁善友邪教,誅黨首李國梁等十六人。壬午,明使馬紹愉等始至。

六月辛丑,都察院參政祖可法、張存仁言:「明寇盜日起,兵力竭而倉廩虛,徵調不前,勢如瓦解。守遼將帥喪失八九,今不得已乞和,計必南遷。宜要其納貢稱臣,以黃河為界。」上不納。以書報明帝曰:「向屢致書修好,貴國不從,事屬既往,其又何言。予承天眷,自東北海濱以訖西北,其間使犬、使鹿產狐產貂之地,暨厄魯特部、斡難河源,皆我臣服,蒙古、朝鮮盡入版圖,用是昭告天地,正位改元。邇者兵入爾境,克城陷陣,乘勝長驅,亦復何畏。余特惓惓為百萬生靈計,若能各審禍福,誠心和好,自茲以往,盡釋宿怨,尊卑之分,又奚較焉。古云:『情通則明,情蔽則暗。』使者往來,期以面見,情不壅蔽。吉兇大事,交相慶吊。歲各以地所產互為餽遺,兩國逃亡亦互歸之。以寧遠雙樹堡為貴國界,塔山為我國界,而互市於連山適中之地。其自海中往來者,則以黃城島之東西為界。越者各罪其下。貴國如用此言,兩君或親誓天地,或遣大臣蒞盟,唯命之從。否則後勿復使矣。」遂厚賚明使臣及從者,遣之。後明議中變,和事竟不成。癸卯,諭諸王貝勒,凡行兵出獵,踐田禾者罪之。甲辰,設漢軍八旗,以祖澤潤等八人為都統。以貝子羅託為都察院承政,吳達海為刑部承政,郎球為禮部承政。乙巳,多羅安平貝勒杜度卒。

秋七月庚午,諭諸王、貝勒、大臣曰:「爾等於所屬賢否,當已詳悉。知而不舉,何以示勸?太祖時,蘇完札爾固齊費英東等見人有善,先自獎勵,然後舉之;見人不善,先自斥責,然後劾之。故人無矜色,無怨言。今未有若斯之公直者矣。」王貝勒等皆謝罪。辛未,承攻索海以罪褫職。壬申,以紐黑為議政大臣。丙子,敘功,晉多羅睿郡王多爾袞、肅郡王豪格復為和碩親王,多羅貝勒多鐸為多羅郡王,鄭親王濟爾哈朗以下賞賚有差。戊寅,遣輔國公博和託代戍錦州。乙酉,議濟爾哈朗以下諸將征錦州違律罪。上念其久勞,悉宥之。諭刑部慎讞獄。己丑,命多羅郡王阿達禮管禮部事。

八月己亥,鑄砲於錦州。癸卯,鎮國將軍巴布海有罪,廢為庶人。癸丑,論克錦州、松山、杏山、塔山諸將功,晉秩有差。

九月,敘外籓諸王、貝勒、大臣從征錦州功,賞賚有差。丁丑,遣貝子羅言乇等代戍錦州。壬午,命沙爾虎達等征虎爾哈部。

冬十月癸卯,遣英俄爾岱等鞫朝鮮閣臣崔鳴吉等罪。辛亥,以阿巴泰為奉命大將軍,與圖爾格率師伐明。壬子,師行。丁巳,上不豫,赦殊死以下。己未,令多鐸、阿達禮駐兵寧遠。以敕諭吳三桂降。又命祖大壽以書招之。三桂,大壽甥也。甲子,命鄭親王濟爾哈朗、睿親王多爾袞、肅親王豪格、武英郡王阿濟格裁決庶政,其不能決者奏聞。

十一月丁丑,多鐸奏擊敗吳三桂兵。丙申,阿巴泰奏自墻子嶺入克長城,敗明兵於薊州。

閏十一月甲辰,上還京。己酉,沙爾虎達等降虎爾哈部一千四百餘人。丙辰,遣巴布泰等更戍錦州。己未,以宗室韓岱為兵部承政。定圍獵誤射人馬處分例。

十二月丁卯,上出獵葉赫。乙亥,遣金維城率師戍錦州。丁丑,駐蹕開庫爾。上不豫,諸王貝子請罷獵,不許。丙戌,月暈生三珥。丁亥,日暈生三珥。癸巳,上還京。

是歲,杜爾伯特部札薩克塞冷來朝。

八年春正月丙申朔,上不豫,命和碩親王以下,副都統以上,詣堂子行禮。辛亥,沙爾虎達等師還,論功賞賚有差。甲寅,明寧遠總兵吳三桂答祖大壽書,猶豫未決,於是復降敕諭之。乙卯,遣譚布等更戍錦州。辛酉,多羅貝勒羅洛宏以罪削爵。

二月乙丑朔,日有食之。甲戌,葬敏惠恭和元妃。庚寅,禁建寺廟。

三月丙申,敕朝鮮臣民毋與明通。丙午,地震,自西隅至東南有聲。庚戌,上不豫,赦死罪以下。遣阿爾津等征黑龍江虎爾哈部,葉臣等更戍錦州。辛酉,更定六部處分例。

夏四月癸酉,遣金維城等更戍錦州。甲戌,多鐸請暫息軍興,輟工作,務農業,以足民用。-80-

五月丙申,復封羅洛宏為多羅貝勒。先是,圖白忒部達賴喇嘛遣使修聘問禮,留京八月,至是,遣還,並賚其來使。庚子,努山敗明兵界嶺口。癸卯,阿巴泰奏我軍入明,克河間、順德、兗州三府、州十八、縣六十七,降州一、縣五,與明大小三十九戰,殺魯王硃衣珮及樂陵、陽信、東原、安丘、滋陽五郡王,暨宗室文武凡千餘員,俘獲人民、牲畜、金幣以數十萬計,籍數以聞。丁巳,阿爾津征虎爾哈奏捷。

六月癸酉,多羅饒餘貝勒阿巴泰師還,鄭親王濟爾哈朗、睿親王多爾袞、武英郡王阿濟格郊迎之。甲戌,賜阿巴泰及從征將士銀緞有差。己卯,諭諸王貝勒曰:「治生者務在節用,治國者重在土地人民。爾等勿專事俘獲以私其親。其各勤農桑以敦本計。」艾度禮代戍錦州。丁亥,朝鮮國王李倧請戍錦州兵歲一更。庚寅,諭戶、兵二部清察蒙古人丁,編入佐領,俱令披甲。

秋七月戊戌,阿爾津等師還,論功賞賚有差。諭諸王勿以黃金飾鞍勒。定諸王、貝勒、貝子、公第宅制。壬寅,定諸王貝勒失誤朝會處分例。丙辰,定外籓王、貝勒、貝子、公等與諸王、貝勒、貝子、公相見禮。丁巳,以徵明大捷,宣諭朝鮮。辛酉,命滿達海掌都察院事。

八月丙寅,貝子羅託有罪論闢,免死,幽之。戊辰,以宗室鞏阿岱為吏部承政,郎球為禮部承政,星訥為工部承政。庚午,上御崇政殿。是夕,亥時,無疾崩,年五十有二,在位十七年。九月壬子,葬昭陵。冬十月丁卯,上尊謚曰應天興國弘德彰武寬溫仁聖睿孝文皇帝,廟號太宗,累上尊謚曰應天興國弘德彰武寬溫仁聖睿孝敬敏昭定隆道顯功文皇帝。

論曰:太宗允文允武,內修政事,外勤討伐,用兵如神,所向有功。雖大勛未集,而世祖即位期年,中外即歸於統一,蓋帝之詒謀遠矣。明政不綱,盜賊憑陵,帝固知明之可取,然不欲亟戰以剿民命,七致書於明之將帥,屈意請和。明人不量強弱,自亡其國,無足論者。然帝交鄰之道,實與湯事葛、文王事昆夷無以異。嗚呼,聖矣哉!


\end{pinyinscope}