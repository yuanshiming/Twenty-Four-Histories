\article{列傳一百八十六}

\begin{pinyinscope}
呂賢基鄒鳴鶴戴熙湯貽汾張芾黃琮陶廷傑馮培元

孫銘恩沈炳垣張錫庚

呂賢基,字鶴田,安徽旌德人。道光十五年進士,選庶吉士,授編修。遷御史、給事中,持正敢言,數論時政得失,多所採用。文宗即位,應詔上封事,請懋聖學,正人心,育人才,恤民隱,尤被嘉納。遷鴻臚寺卿。咸豐元年,超擢工部侍郎。二年,以時事可危,疏請下詔求言,略曰:「粵西會匪滋事,二年以來,命將出師,尚無成效,甚至圍攻省城,大肆猖獗。南河豐工未能合龍,重運阻滯,災民屯聚,在在堪虞。河工費五百萬,軍需費一千餘萬,部臣束手無措,必致掊克朘削,邦本愈搖。今日事勢,譬之於病,元氣血脈,枯竭已甚,外邪又熾,若再諱疾忌醫,愈難為救。惟有開通喉舌,廣覓良方,庶可補救萬一。請特旨令大小臣工悉去忌諱,一改洩沓之故習,各抒所見,以期集思廣益。」疏入,諭部院大臣、九卿、科道有言責者,各據見聞,直言無隱。

三年正月,命賢基馳赴安徽會同巡撫蔣文慶及周天爵辦理防剿事宜,賢基疏言:「江寧以東西梁山為要隘,必先扼守。廬州為江淮門戶,宜令重臣駐★。巢湖出江當梁山上游,地方匪徒宜招撫,免為賊用,且可與梁山為犄角。」上嘉納,不及施行,而安慶、江寧先後陷。奏調給事中袁甲三、知府趙畇幫辦團練防剿,又調編修李鴻章等襄軍事。偕周天爵疏言:「事當分任。團練專令殲除土匪;牧令守本境,統帥剿賊,不得遠駐百里之外,以免推諉。」上韙之。

安徽境內無大枝勁旅,團練亦散漫無可恃。七月,湖北敗賊竄陷英山,擾太湖,分犯洪家埠,賢基檄游擊賡音太、伍登庸擊走之。八月,賊復自江西竄踞安慶,賢基赴舒城、桐城勸募團練,為官軍聲援。賡音太、伍登庸戰歿於集賢關。賊犯桐城,紳士馬三俊率練勇迎戰失利,遂失守。已革按察使張熙宇退駐大關,賢基抗疏劾之。時方駐舒城,或告以無守土責,未轄一兵,賊鋒甚銳,可退守以圖再舉。賢基曰:「奉命治鄉兵殺賊,當以死報國。敢避寇幸免乎?」十月賊至,登陴守御,城陷,死之。

文宗初聞舒城失守,即曰:「賢基素懷忠義,必能大節無虧。」及奏上,深悼惜之,贈尚書銜,加恩於舒城建專祠,擢其子編修錦文以侍讀用,賜銀三千兩,命錦文即日回籍治喪。予騎都尉世職,祀京師及本籍府城昭忠祠。後安徽請祀鄉賢,特諭:「賢基品行端正,居官忠直,名副其實。」即報可。

鄒鳴鶴,字鍾泉,江蘇無錫人。道光二年進士,雲南即用知縣。親老告近,改發河南,署新鄭,補羅山,有惠政。母喪,去官。巡撫程祖洛疏陳鳴鶴政績,羅山紳民籥請保留河南,特旨允俟服闋以南、汝、陳、光四府州所屬酌補選缺,異數也。

尋補光山,調祥符,擢蘭儀河工同知,護開歸陳許道。以治河勞,晉秩知府。歷衛輝、陳州、開封。二十一年,河決祥符,水圍省城,鳴鶴露宿城上,盡力堵御。有議遷省城於洛陽者,鳴鶴上議有六不可。欽差大臣王鼎等據以疏陳,乃決議堅守。凡歷七十餘日,水退城安。論功,晉秩道員。二十三年,河決中牟,褫職留工,工竣,復原官,仍在工效力。丁生母憂,服闋,署彰衛懷道,尋授江西督糧道。文宗即位,詔舉賢才,戶部侍郎侯桐、兩江總督陸建瀛交章以鳴鶴薦,擢順天府尹。

咸豐元年,擢廣西巡撫。匪亂方熾,大學士賽尚阿督軍事,鳴鶴課吏治,治團練,撫恤被兵災民。二年,賊由永安突犯桂林,城中兵僅千人,倉猝防禦,提督向榮馳援,民心始定。總兵秦定三等續至,鳴鶴以諸軍無所統屬,自請督戰。分遣諸將擊賊,相持月餘,賊百計攻城,屢卻。賊遂分竄,賽尚阿促向榮追擊,鳴鶴堅留防賊回竄,互疏爭。賊尋陷興安、全州,入湖南,詔褫鳴鶴職,以守城功免治罪。

洎回籍,賊已陷武昌。三年正月,陸建瀛赴九江督師,疏請起鳴鶴籌辦沿江防務。已病,或沮其行。曰:「此吾補過報國之日也!」建瀛旋退江寧,獲罪,命鳴鶴與將軍祥厚等籌商守御。建瀛見其病甚,欲為奏請還家養痾,鳴鶴不可。及江寧陷,書絕命詞曰:「臣力難圖報稱,臣心仰答九重。三次守城盡節,庶幾全始全終。」遣人持付其子,自率隊出,至三山街,賊見識之,曰:「此守桂林之鄒巡撫也!」呼其名詬之。鳴鶴亦罵不絕口,被支解而死。事聞,贈道銜,賜恤。

同治初,江南既平,曾國籓疏陳鳴鶴生平政績及殉節狀,請加恩優恤。御史硃震言鳴鶴匿民居遇害,非臨陣捐軀者比,請罷之。編修硃福基等復以鳴鶴被難聞見各殊,呈請下兩江總督馬新貽確查。新貽覆奏紳耆咸稱鳴鶴協同防守,誓以身殉,罵賊被戕,無避匿民居之事。詔依巡撫例議恤,予騎都尉兼雲騎尉世職,謚壯節。後祀河南名宦祠。

戴熙,字醇士,浙江錢塘人。道光十二年進士,選庶吉士,授編修。大考二等,擢贊善,遷中允。十八年,入直南書房。督廣東學政,任滿,請終養。二十五年,服闋,未補官,復督廣東學政,累遷內閣學士。二十八年,授兵部侍郎,仍直南書房。

先是,廣東因士民阻英人入城,相持者數年。至二十九年,英人懾於民怒,暫罷議。宣宗嘉悅,以為奇功,錫封總督徐廣縉子爵,巡撫葉名琛男爵。會熙召對,論及之。熙言廣東民風素所諳悉,督撫所奏,恐涉鋪張,非可終恃,上不懌。尋命書扇,有帖體字,傳旨申飭。越日,命南書房書扁額,內監傳諭指派同直張錫庚,戒勿交寫誤字之戴熙。未幾,罷其入直。熙知眷衰,稱病請開缺,上益怒,降三品京堂休致。

咸豐初,詔舉人才,尚書孫瑞珍以熙薦,召來京候簡用,因病未至。粵匪踞江寧,浙江戒嚴。熙偕官紳勸諭捐輸,舉行團練。八年,粵匪由江西擾浙東,熙助巡撫晏端書籌調兵食,乞援鄰境。援師至,賊未得逞,漸退。以治團練勞,加二品頂戴。杭州初有民兵八百人,又選鋒數百,事緩,以資絀,減少半。十年,粵匪由安徽廣德入浙,連陷數縣,犯湖州、武康。熙以所部練勇付按察使段光清,會旗兵防獨松、千秋等關。賊至,斂兵入城守。熙謂用兵無獨守孤城之理,宜分營城外相犄角,又議乘賊初至迎擊,皆未行。熙與弟燾助守西北隅,砲斃黃衣賊一人,賊遽退匿山後。眾謂賊且遁,熙料其詐,偵之,果轉赴西南。晝夜環攻,久雨,兵疲。賊於宋鎮湖門故址穴地轟城,遂陷,熙赴水死之。弟煦、媳金、及甥王朝榮,同殉。事聞,贈尚書銜,建專祠,予騎都尉兼雲騎尉世職,謚文節。弟煦,精算學,自有傳。

熙雅尚絕俗,尤善畫。當視學廣東,陛辭,宣宗諭曰:「古人之作畫,須行萬里路。此行遍歷山川,畫當益進。」其見重如此。後以直言黜。及殉節,遂益為世重。同時湯貽汾畫負盛名,與熙相匹。亦殉江寧之難,同以忠義顯,世稱戴、湯云。

貽汾,字雨生,江蘇武進人。祖大奎,官福建鳳山知縣,守城殉節,父荀業同死,見忠義傳。貽汾少有俊才。家貧,以難廕襲世職,授守備,累擢浙江樂清協副將。歷官治軍捕盜有聲。尚氣節,工詩畫,政績文章為時重。晚辭官僑居江寧。及粵匪熾,貽汾見時事日亟,語人曰:「吾年七十有七,家世忠孝。脫有不幸,惟當致命遂志,以見先人。」江寧籌防,大吏每有咨詢,盡言贊畫。城陷,從容賦絕命詞,赴水死。事聞,文宗以其三世死事,特詔優恤,加一雲騎尉,謚貞愍。

張芾,字小浦,陜西涇陽人。道光十五年進士,選庶吉士,授編修。累遷庶子,直南書房。大考一等,擢少詹事,超遷內閣學士,督江蘇學政。二十五年,授工部侍郎,任滿回京,仍直南書房,調吏部。二十九年,督江西學政。文宗即位,應詔陳言,請明黜陟,寬出納,禁糜費,重海防,上嘉納。命按巡撫陳阡被劾各款,得實,罷之。阡亦訐芾收受陋規,詔免議。

咸豐二年,調刑部侍郎。任滿,留署江西巡撫,尋實授。時粵匪方圍長沙,詔芾偕在籍尚書陳孚恩籌防。未幾,岳州陷,芾駐守九江。三年正月,總督陸建瀛至九江,芾移守瑞昌,賊來犯,擊走之,而九江遂陷,革職留任,退守南昌。賊既踞江寧,分股溯江而上。芾奏調湖北按察使江忠源來援,甫至而賊船直抵城下,芾率官紳嬰城固守,賊穴道轟城,壞而復完。總兵馬濟美戰歿城外,賴江忠源迭戰卻賊,被圍凡三閱月,賊乃東走,由九江趨安徽。芾以守城勞,復原官。奏將將吏猥多,部議覈減,芾疏爭,嚴旨切責。會因截留滇、黔銅鉛銀,又陳孚恩被劾,芾為申辨,上怒,褫芾職。

芾既罷,道梗不得歸,僑居紹興。賊窺徽、寧急,巡撫駐廬州不能兼顧。侍郎王茂廕薦芾,乃命交和春、福濟差遣。芾至,練團勸捐,以千人守徽州,提督鄧紹良、總兵江長貴分扼要隘。五年,復休寧、石埭,予六品頂戴。六年,賊擾婺源、祁門,連破之於七里橋、屯溪口,徽境得安,加五品頂戴。是年冬,賊復由江西竄踞休寧,擊走之。母喪,奪情留軍,命俟服闋後以三品京堂候補。七年,鄧紹良戰歿灣沚,祁門、婺源皆告急。遣參將王慶麟破賊於清華街,又擊走祁門賊。九年,復婺源,賊西竄,授芾通政使,尋遷左副都御史。太平、石埭連戰皆捷,詔皖南四府一州軍務歸芾督辦。十年,賊復陷涇縣、旌德,由績溪進犯徽郡。芾督江長貴及知府蘇式敬、道員蕭翰慶,連克太平、旌德、石埭、涇縣,而賊由江蘇、浙江回竄,復連陷建平、廣德、涇縣。芾先以失機自劾,暫行革職留軍,至是復自請治罪,遂命以皖南軍事畀兩江總督曾國籓,召芾還京,請回籍補持服,允之。

十一年,粵匪、捻匪合擾關中,起芾助治團練御賊。事甫平,而回匪亂作,連破數州縣,逼省城,詔芾督辦陜西團練,會同巡撫瑛棨防剿。瑛棨巽懦,計無所出,謂芾大臣有鄉望,諭之宜可解。芾慨然率數騎往,歷高陵、臨潼至渭南倉頭鎮,曉以利害,回眾頗感動。其酋任老五懼搖眾心,嗾黨擁出折辱之,芾據地大罵不絕口,遂被支解。時同治元年五月十三日也。子師劬,往覓遺骸,僅得骨數節。事聞,予騎都尉兼雲騎尉世職,謚文毅。命於省城、倉頭鎮並建專祠,隨行遇害之臨潼知縣繆樹本、山西知縣蔣若訥及家屬在涇陽被害者五十二人,從死僕人金榜等六人,並附祀。賜師劬舉人。江西、徽州並建專祠,後祀江西名宦。

黃琮,雲南昆明人。道光六年進士,選庶吉士,授編修。累擢兵部侍郎,以親老乞養回籍。咸豐七年,雲南回亂方熾,命琮偕在籍御史竇垿治團練。時餉絀兵單,疆臣主且剿且撫,而漢、回仇隙素深,團練驕悍不聽約束,往往撫局將成,練勇擅殺降回,益紛擾。總督吳振棫劾琮及竇垿辦理失當,皆褫職。事稍定,振棫疏陳縱容練勇諸事,皆出竇垿主持。琮當省城被圍時,登陴固守有勞,又勸捐出力,詔復原官。同治二年,逆回馬榮詐降,入城戕總督潘鐸,肆殺掠,琮遇害,贈右都御史。光緒中,巡撫潘鼎新為請,予謚文潔。

陶廷傑,貴州都勻人。嘉慶十九年進士,由編修遷御史、給事中。道光中,出為江蘇蘇松糧儲道。歷甘肅按察使、陜西布政使,署巡撫。二十五年,休致。咸豐三年,貴州土匪起,命廷傑在籍會同地方官辦理團練。六年,古州、黃平、都勻先後陷,廷傑率團練御賊,死之,予騎都尉世職,謚文節。

馮培元,字因伯,浙江仁和人。道光二十四年一甲三名進士,授編修,入直南書房。咸豐元年,改直上書房,授惇郡王奕脤讀。二年,大考二等,擢侍講。尋督湖北學政。數月中,連擢侍講學士、光祿寺卿。

時粵匪已犯長沙,人情洶懼。培元幼孤,家貧,母何賢明苦節,撫之成立。及至湖北,將迎養。聞岳州陷,馳書止母行。母報曰:「如果有變,見危授命,大節不可奪。其遵吾教!」培元奉書,涕泣自矢。賊至攻城,培元偕在城文武登陴同守。城陷,投井死。三年正月,賊去,向榮率兵入城,有以告者,始出而殮之,尸如生。事聞,文宗以武昌之陷,闔城文武殉難,恤典特優,贈侍郎,建專祠,予騎都尉世職,謚文介。後兩子學瀚、學澧皆賜舉人。

孫銘恩,字蘭檢,江蘇通州人。道光十五年進士,選庶吉士,授編修,累遷詹事。咸豐二年,典試廣東,還京,道出九江。粵匪已由岳州東下,陷漢陽。銘恩疏上江防十二事,下江南督撫施行。三年,連擢內閣學士、兵部侍郎,督安徽學政。

時安慶已為賊踞,故事,學政駐太平府,銘恩激勵紳民,舉行團練,捐廉為倡。潰兵時至,侮官劫市,銘恩諭以大義,稍定。四年,以父病請開缺省視,會有旨命偕在籍前南河總督潘錫恩防守徽、寧,銘恩未之知也。疏入,文宗疑其規避,嚴旨切責,允其回籍,俟假滿以三四品京堂降補。未逾月,賊犯太平,從者請避之。銘恩曰:「城亡與亡,以明吾心!」城陷,賊至,衣冠坐堂上,抗罵,被執,囚於江寧,僕範源從。銘恩不食,賊脅源勸降,源叱之,斷其舌,同遇害。詔嘉其抗節不屈,遇害甚慘,贈內閣學士,入祀京師及安徽、江蘇昭忠祠,予騎都尉世職,謚文節。範源同議恤。

沈炳垣,字紫卿,浙江海鹽人。道光二十五年進士,選庶吉士,授編修,遷中允。咸豐四年,督廣西學政。廣西自洪秀全北犯後,群匪迭起。炳垣至,與巡撫勞崇光議戰守策,崇光深器之。

七年春,按試南寧畢,警報日至,居民洶洶驚避。炳垣倡言城險可保,條列守御法,捐俸濟餉,守三晝夜。賊知有備,引去。潯州陷,江路梗塞,間道至梧州。艇匪陳開等眾數萬突來犯,炳垣率知府陳瑞枝等嬰城固守,凡三閱月,糧盡援絕。官吏以炳垣無守土責,遣兵衛之出,炳垣不可。八月,城陷,仰藥未死,賊擁去,厚遇之。炳垣罵賊,求死不得。久之,乘間為書致巡撫劉長佑,請出兵襲城,密約城內民夾擊。事洩,賊恨甚,磔而焚之。有老卒睹炳垣慘死狀,走省城首於官。贈內閣學士,謚文節,建專祠桂林。

張錫庚,字星白,江蘇丹徒人,大學士玉書裔孫。道光十六年二甲一名進士,選庶吉士,授編修。遷御史,擢順天府丞,丁父憂,服闋,補原官。疏論綠營冒濫頂名及緝捕諸弊,詔下其疏於各直省,實力整頓。又疏言殿試貢士不限字數,聽其發抒,刪去頌辭,下部議行。歷太僕寺卿、左副都御史。

咸豐八年,督浙江學政,擢刑部侍郎,仍留學政任。十年,杭州陷而旋復,錫庚助城守,其子恩然率家屬自焚,錫庚以聞,予旌恤。團練大臣王履謙劾巡撫王有齡籌餉按缺派捐,命錫庚訪按。錫庚以有齡一月內更易州縣二十餘員,非政體,請予處分,從之。十一年,任滿,代者未至,杭州復被圍,錫庚同守城。城陷,或勸之去,錫庚曰:「吾大臣也,不可辱國!」遂自縊,賊稱其忠,為具棺斂。贈尚書銜,予騎都尉兼雲騎尉世職,祀浙江昭忠祠,謚文貞。

論曰:呂賢基以忠鯁受主知,其治兵安徽也,志欲大有所為,當殘破之餘,驟無藉手,倉猝殞身,文宗惜之。鄒鳴鶴久著循聲,戴熙亦負清望,張芾守江西、防皖南,雖無偉績,備歷艱難。三人以在籍搢紳治團籌防,雖久暫不同,皆事權不屬,或以城亡與亡,或以犯難遇害。黃琮初因措置失宜獲咎,繼亦原之,而終不免於難。馮培元、孫銘恩、沈炳垣、張錫庚,文學之臣,職非守土,死皆慘烈,朝廷報忠之典悉從優渥,固不以成敗論已。


\end{pinyinscope}