\article{列傳三百十二}

\begin{pinyinscope}
籓部八

○西藏

西藏,禹貢雍州之域。漢為益州沈黎郡徼外白狼、樂土諸羌地。魏、隋為附國、女國及左封、昔衛、葛延、春桑、迷桑、北利、模徒、那鄂諸羌地。唐為吐蕃,始崇佛法。既而滅吐谷渾,盡臣羊同、黨項諸羌,西鄰大食,幅員萬餘里。唐末衰弱,諸部分散。宋時朝貢不絕。

元世祖時,置烏思藏、納里、速古、魯孫等三路宣慰司,都元帥府,仍置管民萬戶諸官撫輯之。以吐蕃僧帕克斯巴為大寶法王、帝師,嗣者數世。弟子號司空、國公,佩金玉印者甚眾。

明洪武年,以攝帝師納木嘉勒藏博為熾盛佛寶國師,給玉印。置烏斯藏指揮司及宣慰司、招討司、萬戶諸官,多沿元舊,以元國公納木喀斯丹拜嘉勒藏等領之。尋改烏斯藏為行都指揮司,以班竹兒藏為烏斯藏都指揮使,自下皆令世襲。未幾,改烏斯藏俺不羅衛為行都指揮司。永樂中,增置烏斯藏牛兒宋寨行都指揮司及必里、上工⼙部二衛,復分封番僧為大寶法王、大乘法王、大慈法王、闡教王、闡化王、輔教王、贊善王、護教王,凡八王,比歲或間歲朝貢。宣德、成化間,又累加封號。其地有僧號達賴喇嘛,居拉薩之布達拉廟,號前藏;有班禪喇嘛,居日喀則城之扎什倫布廟,號後藏;番俗崇奉又在諸番王之上。西藏喇嘛舊皆紅教,至宗喀巴始創黃教,得道西藏噶勒丹寺。時紅教本印度之習,娶妻生子,世襲法王,專指密咒,流極至以吞刀吐火炫俗,盡失戒定慧宗旨。黃教不得近女色,遺囑二大弟子,世以呼畢勒罕轉生,演大乘教。呼畢勒罕者,華言「化身」。達賴、班禪即所謂二大弟子,達賴譯言「無上」,班禪譯言「光顯」。其俗謂死而不失其真,自知所往,其弟子輒迎而立之,常在輪回,本性不昧,故達賴、班禪易世互相為師。其教皆重見性度生,斥聲聞小乘及幻術小乘。當明中葉,已遠出紅教上。

達賴第一輩曰羅倫嘉穆錯,吐蕃贊普之裔,世為番王。二十歲至前藏,宗喀巴以為大弟子。年八十四。第二輩曰根登嘉木錯,在後藏札朗轉世,登布達拉、色拉、扎什倫布講經之坐。年六十七。三輩曰鎖南嘉木錯,為達賴中最著名者。置第巴,代理兵刑賦稅。弟子稱呼圖克圖,分掌教化。時黃教尚未行於蒙古。元裔俺答兼並諸部,侵掠中國,用兵土伯特,收阿木多、喀木康等部落。年老厭兵,納其侄鄂爾多斯部碩克濟農諫,往迎達賴,勸之東還。自甘州移書張居正,求通貢饋。萬歷年,遂納鎖南嘉木錯之貢,予封賚。達賴應俺答之迎,至青海,為言三生善緣。諸臺吉言:「原自今將湧血之火江,變溢乳之靜海。」俺答許立廟,一在歸化城,一在西寧,於是黃教普蒙古諸部。而藏中紅教之大寶、大乘諸法王,皆俯首稱弟子,改從黃教。化行諸部,東西數萬里,熬茶膜拜,視若天神,諸番王徒擁虛位,不復能施號令。年四十七。四輩曰榮丹嘉穆錯,年二十八。五輩曰阿旺羅布藏嘉木錯。

初,西藏俗稱其國曰圖伯特,亦曰唐古特。自達賴、班禪外有汗,則蒙古部長為之。時藏之藏巴汗與達賴所用第巴不協。額魯特部和碩汗者,名圖魯拜琥,元太祖弟哈布圖哈薩爾十九世孫也。後兼並唐古特四部,改號顧實汗。以青海地廣,令子孫游牧,而喀木、康輸其賦。衛地則第巴奉達賴居之,藏地則藏巴汗居之。第巴桑結與藏巴汗不相能,謂其虐部眾、毀黃教,乞師於顧實汗翦滅之。顧實汗遂以藏地居班禪,留長子鄂齊爾汗轄其眾,次子達賚巴圖爾臺吉佐之,皆崇德年事也。

先是天聰年間,大兵取明之東省,天現明星祥瑞。顧實汗曰:「此星系大力汗之威力星。由是觀之,非常人也。」於是遐邇蒙古共遵太宗文皇帝為和爾摩斯達額爾德穆圖博克達撤辰汗。迨崇德二年,奏請發幣使延達賴。四年,遣使貽土伯特汗及達賴書,謂「自古所制經典,不欲其泯滅不傳,故遣使敦請」云。嗣以喀爾喀有違言,不果。顧實汗復致書達賴、班禪、藏巴汗,約共遣使朝貢。達賴、班禪及藏巴汗、顧實汗遣伊喇固散胡圖克圖等貢方物,獻丹書,先稱太宗為曼殊師利大皇帝。曼殊者,華言「妙吉祥」也。使至盛京,太宗躬率王大臣迓於懷遠門。御座為起,迎於門閾,立受書,握手相見,升榻,設座於榻右,命坐,賜茶,大宴於崇政殿。間五日一宴,命王、貝勒以次宴。留八閱月乃還。八年,報幣於達賴曰:「大清國寬溫仁聖皇帝致書於金剛大士達賴喇嘛。今承喇嘛有拯濟眾生之志,欲興扶佛法,遣使通書,朕心甚悅,茲恭候安吉。凡所欲言,令察罕格龍等口授。」復貽書於班禪及紅帽喇嘛濟東胡圖古圖等,亦如之。是為西藏通好之始。於是闡化王及河州弘化、顯慶二寺僧,天全六番,烏斯藏董卜、黎州、長河西、魚通、寧遠、泥溪、蠻彞、沈村、寧戎等土司,莊浪番僧,先後入貢,獻前明敕印,請內附矣。

明年,世祖定鼎燕京,混一宇內。顧實汗復奏:「達賴功德無量,宜延至京,令其諷誦經咒,以資福佑。」乃遣使往迎。順治四年,達賴、班禪各遣使獻金佛、念珠,表頌功德。五年,遣喇嘛席喇布格隆等齎書存問達賴,並敦請之。達賴覆書,許於辰年朝覲。九年十月,達賴抵代噶,命和碩承澤親王碩塞等往迎。十二月,達賴至,謁於南苑,賓之於太和殿,建西黃寺居之。達賴尋以水土不宜,告歸,賜以金銀、緞幣、珠玉、鞍馬慰留之。十年二月,歸,復御殿賜宴,命親王碩塞偕貝子顧爾瑪洪、吳達海率八旗兵送至代噶,命禮部尚書覺羅朗球、理籓院侍郎達席禮齎金冊印,於代噶封達賴為西天大善自在佛領天下釋教普通瓦赤喇怛喇達賴喇嘛。達賴歸,興黃教,重建布達拉及前藏各寺院六十二處,又創修喀木、康等處廟,計三千七十云。

是時顧實汗先卒,達賴又年老,大權旁落於第巴桑結。桑結詭遣內安島人冒闡化王貢使,實則闡化王久經殘破,廢為喇嘛,而屢次進貢仍書王名,並請換敕印。廉得其實,斥之。吳三桂王云南,歲遣人至藏熬茶。康熙十三年,三桂反,詔青海蒙古兵由松潘入川。桑結使達賴上書尼之,且代三桂乞降。及大兵圍吳世璠於雲南,世璠割中甸、維西二地乞援於藏,其書為貝子章泰軍所獲。朝廷但駐守中甸,未深問也。康熙二十一年,在布達拉寺圓寂,年六十二。

當五世達賴之卒也,第巴桑結以議立新達賴故,與拉藏汗交惡。桑結既以己意立羅布藏仁青策養嘉錯為六世達賴,乃秘不發喪,偽言達賴入定,居高閣,不見人,凡事傳達賴命行之,自是益橫。既袒準噶爾以殘喀爾喀蒙古,復唆準噶爾以斗中國,又外構策妄阿喇布坦,內閧拉藏汗,遂招準兵寇藏之禍。凡西北擾攘數十年,皆第巴一人所致。

噶爾丹者,亦四額魯特之一,曾入藏為喇嘛,與第巴暱。歸篡其汗,自言受達賴封為準噶爾博碩克圖汗。又喀爾喀蒙古以入藏隔於額魯特,乃自奉宗喀巴第三弟子哲卜尊丹巴胡圖克圖之後身為大胡圖克圖,位與班禪亞,凡數十年矣。至喀爾喀車臣汗與土謝圖汗構兵,聖祖遣使約達賴和解之。桑結奏使噶爾丹西勒圖往。蒙語喇嘛坐床者為「西勒圖」,達賴大弟子也。而哲卜尊丹巴胡圖克圖亦奉詔蒞盟壇,與噶爾丹西勒圖抗禮。噶爾丹使其族弟隨之觀釁,因責喀爾喀待達賴無加禮,詬責之,為土謝圖汗所殺。噶爾丹遂以報仇為名,襲侵其部落。喀爾喀集眾議投俄羅斯與投中國孰利,哲卜尊丹巴曰:「俄羅斯持教不同,必以我為異類,宜投中國興黃教之地。」遂定計東走。聖祖申命桑結遣使罷兵。桑結使濟隆胡圖克圖往,反陰嗾之。二十九年,遂入寇漠南,我兵敗之烏闌布通。噶爾丹託濟隆代乞和,頂威靈佛,立誓而遁。桑結內慚,乃託達賴意,合青海蒙古及額魯特各臺吉上尊號,聖祖不受,詔曰:「朕與達賴,期於撫育眾生,而所遣堪布等故違意旨,以致喀爾喀、額魯特兩傷。如能令其修和,朕方欲加達賴嘉號,此皆任事行人不能仰副朕心及達賴意,致喀爾喀殘破,額魯特喪敗,朕心實為隱痛,復何尊號之可受乎?來使貢物其發還!」屢遣京師喇嘛入藏探之。三十四年,達賴入貢,言己年邁,國事決第巴,乞錫封爵。詔封第巴桑結為土伯特國王。

三十五年,聖祖親征噶爾丹,至克魯倫河。噶爾丹敗竄,慰其部下曰:「此行非我意,乃達賴使言南征大吉,是以深入。」上謂達賴存必無是事,乃遣使第巴桑結書曰:「朕詢之降番,皆言達賴脫緇久矣,爾至今匿不奏聞。且達賴存日,塞外無事者六十餘年,爾乃屢唆噶爾丹興戎樂禍,道法安在?達賴、班禪分主教化,向來相代持世。達賴如果厭世,當告諸護法主,以班禪主宗喀巴之教。爾乃使眾不尊班禪而尊己,又阻班禪進京,朕欲和解準噶爾部,爾乃使有虧行之濟隆以往。烏闌布通之役,為賊軍卜日誦經,張蓋山上觀戰,勝則獻哈達,不勝又代為講款,以誤我追師。繄爾袒庇噶爾丹之由,今為殄滅準夷告捷禮,以噶爾丹佩刀一及其妻阿奴之佛像一、佩符一,遣使賚往,可令與達賴相見,令班禪來京,執濟隆以畀我。如其不然,朕且檄雲南、四川、陜西之師見汝城下。汝其糾合四額魯特人以待,其毋悔!」

桑結惶恐,明年密奏言:「為眾生不幸,第五世達賴於壬戌年示寂,轉生靜體,今十五歲矣。前恐唐古特民人生變,故未發喪。今當以醜年十月二十五日出定坐床,求大皇帝勿宣洩。至班禪,因未出痘,不敢至京。濟隆,當竭力致之京師。乞全其身命戒體,並封達賴臨終尸鹽拌像。」聖祖許為秘之,待十月宣示內外。而第巴使者歸,途遇策妄阿喇布坦會擒噶爾丹之兵,復宣言:「達賴已厭世,爾部落兵毋得妄行。」策妄阿喇布坦哭而歸。聖祖以第巴始終反覆持兩端,乃追還其使,傳集各蒙古,宣示密封,則像首已墮,第巴使驚僕於地。

桑結忌策妄阿喇布坦盡收準部故地,致噶爾丹無所歸,奏防其猖獗,而策妄阿喇布坦亦奏第巴奸譎,及所立新達賴之偽,欲藉詞侵藏。聖祖以二人皆叵測,不之許也。四十四年,桑結以拉藏汗終為己害,謀毒之,未遂,欲以兵逐之。拉藏汗集眾討誅桑結,詔封為翊法恭順拉藏汗,因奏廢桑結所立達賴,詔送京師。行至青海,道死,依其俗,行事悖亂者拋棄尸骸。卒,年二十五。時康熙四十六年也。論者謂達摩創法震旦,有一花五葉之讖,至六世啟衣缽之爭,故六祖不復傳衣缽,與宗喀巴至第六世達賴之事若一轍。天數所極,佛法不能違,而況第巴詐偽出之,以尊己擅權,卒釀拉藏汗、準噶爾相尋之禍。

七輩羅布藏噶爾桑嘉穆錯於康熙四十七年在里塘轉世。生有異表,右臂紋如法輪。七歲與眾喇嘛談經,均莫能難,蓋有夙慧也。初拉藏汗既奏廢羅布藏仁青策養嘉穆錯,別立博克達山之呼畢勒罕阿旺伊什嘉穆錯為達賴,聞其名忌之,將以兵戕之,其父索諾木達爾扎襁負走,乃免。青海砲臺吉以不辨真偽爭,詔遣官率青海使人往視。拉藏汗奏:「前解偽達賴時,曾奉旨尋真達賴,訪得博克達山呼畢勒罕,以班禪言坐床。」廷議以呼畢勒罕尚幼,俟再閱數年給封,又以拉藏汗與青海臺吉不睦,遣侍郎赫壽協理藏務。是為西藏設官辦事之始,然猶不常置也。四十九年,班禪、拉藏汗會同管理藏務赫壽奏:「阿旺伊什嘉穆錯熟諳經典,青海臺吉信之,請給冊印。」詔依其請。而青海實不之信,與藏中所奏互相是非。五十三年,青海諸臺吉等遣兵取道德格,迎羅布藏噶爾桑嘉穆錯至青海坐床,請賜冊印。聖祖恐其構釁,詔徙至京,不果行。復令送紅山寺,繼請送西寧宗喀巴寺。青海貝勒察罕丹津等復尼之,且以兵脅異己者。詔大兵護送,乃居宗喀巴寺。聖祖以拉藏汗年近六旬,一子青海駐扎,一子策妄阿喇布坦就婚,恐託詞愛婿,羈留不歸,勢頗孤危。況自殺第巴,彼處人難保不生猜忌。額魯特秉性多疑,又甚疏忽,倘事出不測,相隔萬里,救之不及。諭令深謀防範。

五十六年,策妄阿喇布坦遣臺吉策凌敦多布等率兵六千,徒步繞戈壁,逾和闐南大雪山,涉險冒瘴,晝伏夜行,赴阿里克,揚言送拉藏汗長子噶爾丹忠夫婦歸。拉藏汗不知備,賊至達木始覺,偕仲子索爾扎拒,交戰兩月,不敵,奔守布達拉,始來疏乞援。賊誘噶卜倫沙克都爾扎卜,將小招獻降,唐古特臺吉納木扎勒等開布達拉北城入,戕拉藏汗,拘其季子色布騰及宰桑等,搜各廟重器送伊犁,禁阿旺伊什嘉穆錯於扎克布裏廟。索爾扎率兵三十人潰走,為所擒,其妻間道來奔,詔優養之。

西安將軍額倫特率西寧、松潘、打箭爐、噶勒丹,會同青海諸臺吉及土司屬下赴援,至喀喇河,遇伏,敗歿。賊復誘里塘營官喇嘛歸藏,於是巴塘、察木多、乍雅、巴爾喀木皆為所搖惑矣。尋詔都統法喇移打箭爐兵屯裡塘護呼畢勒罕,復令索諾木達爾扎傳諭營官喇嘛,將抗不就撫者誅之,傳檄巴塘、察木多、乍雅各籍其土及民數,遂進屯巴塘。策凌敦多布懼,返所掠。而兵自巴爾喀木歸,言唐古特有瘴癘,浮腫,難久處,青海蒙古皆憚進藏,慫恿達賴奏可隨地安禪,興大兵恐擾眾。王大臣懲前敗,亦皆言藏地險遠,不決進兵議。聖祖以西藏屏蔽青海、川、滇,若準夷盜據,將邊無寧日。且賊能沖雪縋險而至,何況我軍。策凌敦多布聞我師至,自必望風遠遁。俟定立法教後,或暫留守視,或久鎮其地。唐古特眾皆為我兵,準夷若再至,以逸待勞,何難剿滅。安藏大兵,決宜前進。詔封羅布藏噶爾桑嘉穆錯為弘法覺眾第六輩達賴喇嘛。命皇十四子允為撫遠大將軍,屯青海之木魯烏蘇治軍餉,平逆將軍延信出青海,定西將軍噶爾弼出四川,兩路搗藏。藏人亦知青海達賴之真,藏中舊立之贗,合詞請於朝,乞擁置禪榻,詔許給金冊印。於是蒙古汗、王、貝勒、臺吉各自率所部兵,或數百,或數千,隨大兵扈從達賴入藏。

策凌敦多布由中路自拒青海軍,分遣其宰桑以兵三千六百拒南路。將軍噶爾弼招撫里塘、巴塘番眾,進至察木多,奪洛隆宗嘉玉橋之險。旋奉大將軍檄,俟期並進。噶爾弼恐期久糧匱,用副將岳鍾琪以番攻番計,招土司為前馳,集皮船渡河,直搗拉薩,降番兵七千。宣諭大小第巴及喇嘛,封達賴倉庫,分兵塞險,扼賊餉道。而青海亦三敗其中途劫營之賊,斬俘千計。額魯特進退受敵,遂大潰,不敢歸藏,由克庇雅北竄,崎嶇凍餒,得還伊犁者不及半。

五十九年九月十五日,達賴至布達拉坐床,出阿旺伊什嘉穆錯於禁所,發回京師廢之,盡誅額魯特喇嘛之助逆者。留蒙古、川、滇兵四千,命公策旺諾爾布總統戍藏,額駙阿寶、都統武格參贊軍務。以藏遺臣空布之第巴阿爾布巴首向效順,同大兵取藏,阿里之第巴康濟鼐截擊準噶爾回路,俱封貝子;隆布奈歸附,授輔國公,理前藏務,頗羅鼐授扎薩克一等臺吉,理後藏務,各授噶卜倫。於是里塘所屬之上下牙色,巴塘所屬之桑阿、壩林、卡石等番,次第歸順;郭羅克之吉宜卡、納務、押六等寨先後剿撫矣。

雍正元年,召回允等,撒駐藏防兵,設戍於察木多。二年,青海喇嘛助羅卜藏丹津之叛。青海諸寺喇嘛眾各數千,群起騷動。章嘉胡圖克圖之呼畢勒罕拒戰於郭隆寺,察汗諾們汗亦黨賊助戰。石門寺喇嘛陽稱投順,陰肆劫掠,夾木燦堪布將竄藏,年羹堯等討平之。世宗謂「玷辱宗門,莫斯為甚」,乃收各寺明國師、禪師印,並定廟舍毋逾二百楹,眾毋逾三百人。

五年七月,阿爾布巴、隆布奈、扎爾鼐恃與達賴姻,爭貝子康濟鼐之權,聚兵害之,欲投準噶爾。詔吏部尚書查朗阿率川、陜、滇兵萬有五千進討。未至,而臺吉頗羅鼐率後藏及阿里兵九千,自潘玉口至喀巴,先遣兵千餘沖破喀木卡倫,與隆布奈兵交綏。夜,西藏斥堠俱歸順,頗羅鼐即率兵直抵拉薩。駐藏大臣馬喇、僧格往布達拉護達賴,各寺喇嘛將阿爾布巴等擒獻送馬喇所。查朗阿至,誅首逆及其孥。詔以頗羅鼐為貝子,總藏事。賜犒兵銀三萬兩。留大臣正副二人,領川、陜兵二千,分駐前後藏鎮撫,是為大臣駐藏三年一代之始。收巴塘、里塘隸四川,設宣撫司治之;中甸、維西隸雲南,設二治之。

是年策妄阿喇布坦死,子噶爾丹策零立,請赴藏熬茶,又聲言欲送還所虜拉藏汗二子。詔嚴兵備之,移達賴於里塘之惠遠廟。八年,遷於泰寧,護以兵千。每年夏初,西藏官兵赴防北路騰格里海之隘,以備準夷,冬雪封山,撤兵。蓋通準夷之路有三:其極西由葉爾羌至阿里,中隔大山,迂遠易備;其東路之喀喇河又有青海蒙古隔之;中路之騰格里海逼近衛地,故防守尤要。並以頗羅鼐子珠爾默特策布登統阿里諸路兵,保唐古特,授為扎薩克一等臺吉。追念康濟鼐前勛,無嗣,以其兄噶錫鼐色布登喇布陣亡阿里,封其子噶錫巴納木札勒色布騰為輔國公,尋授噶卜倫。達賴之父索諾木達爾扎亦為輔國公。晉頗羅鼐貝勒。十年,拉達克汗德忠納木札納奏:「臣理國事,尊釋教,偵準噶爾情輒以告。」優詔答之。準噶爾請和,詔果親王偕章嘉胡圖克圖送達賴由泰寧歸藏,減戍藏兵四之三。章嘉胡圖克圖為達賴請巴塘、里塘還前藏,以其為達賴所降生,諸土司建寺安禪,制最宏麗也。詔以其地商稅年銀五千兩賜之,地仍內屬。

乾隆四年,以頗羅鼐勤勞懋著,預保子襲郡王爵。頗羅鼐子二:長,珠爾默特策布登,病足;次,珠爾默特納木扎勒。兄弟互讓,而頗羅鼐愛少子,請以次子為長子,允之。又嘉長子之讓,詔封鎮國公,仍鎮守阿里。頗羅鼐善服眾,為諸噶卜倫所敬事。有綏奔喇嘛扎克巴達顏者,書其名瘞詛之。事覺,頗羅鼐欲弭變,輕議其罪。十一年,溫諭嘉獎,謂:「鎮壓左道不足患,其偕達賴協輯唐古特眾。」準噶爾使再入藏熬茶,駐藏副都統傅清等遣員率喀拉烏蘇兵監視。十二年,頗羅鼐以暴疾亡,以珠爾默特納木扎勒襲爵,兼理噶卜倫,以班第達協理藏務。高宗恐其少不更事,未能服眾,或以綏奔喇嘛扎克巴達顏故,與達賴構隙,不肖眾起而間之,不無滋事虞,諭傅清留意體察,而卒有十五年珠爾默特納木扎勒之變。

時準噶爾臺吉策妄多爾濟納木扎納復遣使赴藏熬茶,入寺詭避痘,以己卒守門,不令官兵從。詔以準噶爾狡甚,飭嚴防,雖歸巢,勿稍忽。而珠爾默特納木扎勒以駐藏大臣不便於己,乘機奏藏地靜謐,請撤駐防兵。廷議以不從撤兵請,適足滋疑,不如示之信,詔可。諭達賴勿令準噶爾入藏,雖固請弗允。珠爾默特納木扎勒又詭稱準噶爾襲唐古特,至碩翁圖庫爾,遣兵備喀拉烏蘇,徙達木番眾。不數旬,揚言準噶爾至阿哈雅克,自率兵往備。駐藏提督索拜遣旺對赴喀拉烏蘇備之。比至,無蹤。有詔撤喀拉烏蘇兵及達木番歸牧,勿惑眾。初,郡王頗羅鼐以女妻班第達,至是班第達察珠爾默特納木扎勒有逆志,不之附。珠爾默特納木扎勒惡之,奪其孥。駐藏副都統紀山劾珠爾默特納木扎勒妄戾,請檄其兄協理藏務。高宗不允,諭紀山善導之,勿露防範跡。已而珠爾默特納木扎勒以珠爾默特策布登發阿里兵擾藏告,蓋計陷之也。因諭傅清曰:「珠爾默特納木扎勒年幼躁急,性好滋事。若果無他故,兄欲進兵至藏,是特兄弟互相侵犯耳。若其兄並無此事而造言誣構,則宜相機辦理。」

十五年,珠爾默特納木扎勒以兵戕其兄珠爾默特策布登於阿里,詭以兄暴疾聞,請收葬,並育兄子。時其兄子朋素克旺布及珠爾默特旺扎勒皆居後藏。珠爾默特納木扎勒以兵往戕朋素克旺布,陽稱逃亡。珠爾默特旺扎勒依班禪為喇嘛,乃免。傅清、拉布敦以珠爾默特納木扎勒攜兵離藏告。蓋是時珠爾默特納木扎勒既襲殺其兄,復通書餽物準噶爾,請兵為外應,私攜砲至後藏,誣籍噶卜倫班第達及第巴布隆贊等旋達木,距前藏三百餘里,擁眾二千餘不歸。奏至,詔俟副都統班第自青海赴藏討罪,復諭四川總督策楞、提督岳鍾琪馳兵往會。而賊勢猖獗,驛道梗塞,軍書不通者旬日。傅清偕拉布敦計,不急誅,必據唐古特為變,召珠爾默特納木扎勒至,待諸樓。甫登,起責其罪曰:「爾違天子令,且忘爾父!無君無父,罪不可赦!」傅清趨前扼其臂,拉布敦拔佩刀剚之,諭脅從罔治。有羅卜藏扎什者,趨下呼賊,千餘突至,聚圍樓,集槁焚。達賴遣番僧往護,不得入,傅清、拉布敦死之。上嘉憫傅清等靖逆遇害,均追贈一等伯,特建雙忠祠以祀。班第達奔守達賴,集兵拒逆。即命班第達以輔國公攝噶卜倫,分其權,而總其成於達賴。設噶卜倫四、戴琫五、第巴三、堪布三,分理藏務,隸駐藏大臣及達賴轄。增駐防兵千有五百戍藏。以達木番歸駐藏大臣轄,視內地例,設佐領、驍騎校各職。並於準噶爾通藏隘設汛嚴防。二十二年,蕩平伊犁,始永無準夷患。是年,達賴在布達拉圓寂,時年五十。

八輩羅布藏降白嘉穆錯擺桑布,於乾隆二十三年在後藏拖結熱拉岡出世。二十七年,迎至布達拉坐床。三十年,由班禪班墊伊喜傳授小戒。三十三年,親至前藏攢招,隨登色拉、布賚繃、噶勒丹三大寺講經之座。四十二年,由班禪傳授格隆大戒。四十六年,頒給敕書、金冊、金印,賞達賴之兄索諾木達什輔國公。四十八年,頒玉冊、玉印,凡遇國家慶典準其鈐用,其尋常奏書文移仍用原印。

五十三年,廓爾喀侵犯藏境。初,第六輩班禪之歿,及京歸舍利於藏也,凡朝廷所賜賚,在京各王公及內外各蒙邊地諸番所供養,無慮數十萬金,而寶冠、瓔珞、念珠、晶玉之缽、鏤金之架裟,珍寶不可勝計。其兄仲巴呼圖克圖悉踞為己有,既不布施各寺,番兵、喇嘛等亦一無所與。其弟沙瑪爾巴垂涎不遂,憤唆廓爾喀籍商稅增額、食鹽糅土為詞,興兵擾邊。唐古特私和廓爾喀,朝廷所遣之侍衛巴忠、成都將軍鄂輝、總兵成德等實陰主其議,令堪布等許歲幣萬五千金,於是廓爾喀飽颺而去。巴忠等以賊降飾奏,諷廓爾喀噶箕入貢,受封國王。五十四年七月,廓爾喀遣人至藏表貢,並致駐藏大臣書,請如前約。鄂輝恐發覺私許之款,屏不奏。次年,藏中幣復爽約。

五十六年七月,廓爾喀復大入寇,占據聶拉木,誘執噶卜倫丹津班珠爾以歸。八月,復占據濟嚨。保泰等遷班禪於前藏。廓匪進擾薩加溝,遂至扎什倫布,仲巴呼圖克圖遁。九月,都司徐南騰堅守官寨,廓匪大掠扎什倫布財物以歸。巴忠扈從熱河,聞變,沉水死。鄂輝、成德奉命赴藏剿御,皆逗留不進。

十月,保泰等請移達賴、班禪於泰寧,上嚴斥之,而嘉達賴之拒其議。命嘉勇公福康安為將軍、超勇公海蘭察為參贊大臣,率索倫、達呼爾兵及屯練土兵進討。其軍餉則藏以東,四川總督孫士毅主之;藏以西,駐藏大臣和琳主之;濟嚨邊外,則前督惠齡主之。五十七年正月,鄂輝等始復聶拉木。二月,帕克哩營官率番兵收復哲孟雄、宗木地方。是月,陷寇之第巴博爾東自陽布回藏。唐古特私許歲幣事覺,詔以保泰、雅滿泰隱匿不奏,革責枷號。三月,授福康安為大將軍,逮仲巴呼圖克圖於京。四月,添調川兵三千赴藏。閏四月,福康安自定日進兵趨宗喀。五月,克擦木,復濟嚨。是月十五日,克熱索橋,遂入廓境。二十四日,克脅布魯碉卡。六月,福康安、海蘭察等進攻東覺,並雅爾賽拉、博爾東拉諸處,皆克之,成德等亦攻克扎木鐵索橋。六月,廓酋拉特納巴都爾迭遣大頭人乞降,送出丹津班珠爾及前俘之兵。七月,福康安攻克噶勒拉、堆補木,奪橋渡河,深入廓境七百餘里,將迫其都陽布。都統銜斐英阿等陣亡。成德亦進克利底大山賊卡。廓酋復呈繳唐、廓前立合同,獻所掠扎什倫布財物及沙瑪爾巴之尸。八月,廓爾喀遣使進貢。福康安以廓爾喀屢請投誠奏入,奉旨受降。時以廓境益險,八月後即雪大封山,因允所請。於是福康安率大兵凱旋,撤回藏。議定善後章程:駐藏大臣與達賴、班禪平等;噶卜倫以下由駐藏大臣選授;前後藏番歸我設之游擊、都司節制訓諫;自行設爐鼓鑄銀幣;設糧務一員監督之。至是,我國在藏始具完全之主權。

初,達賴、班禪及各大呼圖克圖之呼畢勒罕出世,均由垂仲降神指示,往往徇私不公,為世詬病。甚至哲卜尊丹巴胡圖克圖示寂,適土謝圖汗之福晉有妊,眾即指為呼畢勒罕;及彌月,竟生一女,尤貽口實。而達賴、班禪親族亦多營為大呼圖克圖,以專財利,致有仲巴兄弟爭利、唆廓夷入寇之禍。而達賴兄弟孜仲、綏繃等充商卓特巴,肆行舞弊,占人地畝,轉奉不敬黃教之紅帽喇嘛,令與第穆呼圖克圖、濟嚨呼圖克圖同坐;且與眾喇嘛斂取銀兩,並將商上物件暗中虧缺,來藏熬茶人應得路費皆減半發給,有傷達賴體制,因之特來參見者日減,殊失人心。高宗乘用兵後,特運神斷,創頒金奔巴瓶,一供於藏之大招,遇有呼畢勒罕出世,互報差異者,各書名於牙簽,封固納諸瓶中,誦經三日,大臣會同達賴、班禪,於宗喀巴佛前啟封掣之。至扎薩克蒙古所奉之呼圖克圖,其呼畢勒罕亦報名理籓院與駐京之章嘉呼圖克圖,或喇嘛印務處掌印掣定,瓶供雍和宮,而定東科爾入官之限。

嘉慶九年十月,達賴有疾,命成都副都統文弼帶醫馳往看視。未抵藏,達賴已於是月在布達拉圓寂,年四十有七。九輩阿旺隆安嘉穆錯擺桑布,於嘉慶十年在康巴墊曲科轉世。年二歲,異常聰慧,早悟前身,奉特旨即定為呼畢勒罕,毋庸入瓶簽掣。十三年九月,迎至布達拉坐床,賞達賴之叔洛桑捻扎朗結頭品頂戴。十八年,由班禪傳授小戒。時達賴幼穉,噶卜倫乘機舞弊,將達賴莊屋侵占,並將辦事人隨事更換,豢賊自肥,公肆劫掠。命成都副都統文弼、西寧辦事大臣玉寧馳藏查辦,並究噶卜倫策拔克與成林互訐。經訊噶卜倫策拔克率意更定章程四條,以內地治理民人之法概行禁止,致邀眾怨,成林挪移庫款,分別斥革,發伊犁、烏魯木齊效力贖罪。此藏事之內潰也。至外事之棼亂,則廓爾喀噶箕乃爾興戕其王,被誅。逆黨熱納畢各嚨竄逃至唐古特,又與披楞開戰,求達賴、班禪助款。布魯克巴部長曲扎曲勒請賞王爵,文弼匿不奏聞。帕克哩營官勒索其進關貨物,逞兇肇釁。哲孟雄部請賞唐古特莊田,並定邊界。緬甸國男婦私與藏中胡圖克圖文件往來。藏事已岌岌可危矣。二十年二月,在布達拉圓寂,年十一歲。

十輩阿旺羅布藏降擺丹增楚稱嘉穆錯擺桑布,於道光二年三月晦,奏明在大招金奔巴瓶內掣定。八月,迎至布達拉坐床。遣章嘉胡圖克圖由京馳藏照料。奏定噶勒丹錫埒圖薩瑪第巴克什為正師傅,噶勒丹舊池巴阿旺念扎及榮增班第達甲木巴勒伊喜丹貝嘉木磋為副師傅。尋以傳授達賴經典三年有餘,其未得諾們汗之榮增班第達亦賞給諾們汗,賞達賴之父羅布藏捻扎頭品頂戴。十四年,由班禪傳授格隆大戒。十五年,博窩滋事生番降,設曲木多寺四品番目營官一,宿凹宗、聶伊沃、有茹寺三處六品番目各一,宿木宗、普龍寺、湯堆批批三處七品番目各一。

藏西南徼外有哲孟雄者,唐古特之屏籓也。自五輩達賴以來,因其崇信黃教,歸達賴管轄。乾隆五十六年駐藏大臣奏哲卜雄、作木朗二部落每與達賴、班禪通書訊,惟不聽藏中調遣,被廓爾喀侵占已有十年。經福康安檄令協剿,奪回侵地,藉稱天熱,畏懦不前。迨聞廓爾喀歸順,復思藉天朝威勢,斷還六輩達賴所定舊界。經福康安等駁斥,畫分邊界,不能自由入藏,而夏秋之間,該部落因地方炎熱,仍準其來卓木曲批避暑。於七輩達賴時,曾將唐古特界內卓木曲批迤西奪扎之莊田賞給作為養贍,歷年自行徵收錢糧、青稞。卓木之民常至哲孟雄往來貿易。其部長之妻亦唐古特人,常遣人赴廓部長住所。距藏僅十一站,至卓木曲批避暑處,在帕克哩以外,與藏僅隔一山,不三站,設有鄂博,並無要隘,相安無事者有年。自不準赴藏,而始有請求給地之奏,及請賞卓木雅納綽之民,不得已有請賞給帕克哩營官之缺。前藏商上向與後藏商卓特巴齟。時噶勒丹錫勒圖薩瑪第巴克什尤為貪奸,不公不法,凡後藏代其陳請者,輒責其貪鄙無知。文幹等飭噶卜倫嚴斥,謂無妄求管理藏地所屬職官民人之理,並定八年來藏一次之限。廷臣不知詳情,允之。文幹等僅行文藏內文武嚴查,而不敢譯旨欽遵,蓋恐一經宣布,部長必有理申明也。而其部長每歲瀆請赴藏熬茶及入藏避暑如故。迨道光四年,松廷等始將前奉諭旨專札明示,並隨時嚴行駮飭。五年,班禪據報詳情,謂:「哲孟雄部長楚普郎結訴稱自不準赴界,上年人民病斃者一千有餘。再達賴坐床已逾數年,各部落俱得赴藏朝見,而舊所屬之人獨抱向隅,實無面目見其部民。」於是始準其暫居避暑,仍令帕克哩營官防範稽查。在當時文幹誤聽前藏一面之詞,不查實情,率行具奏。文幹等既知困難,有失字小之道,而猶遷就其詞,準其來藏熬茶一次,蓋以準噶爾視哲孟雄。而哲孟雄離心離德,甘為印度屬地,至有捻都納之敗,而西藏之門戶洞開矣。十七年,在布達拉圓寂,年二十二。

第十一輩阿旺改桑丹貝卓密凱珠嘉穆錯,於道光十八年九月朔在噶達轉世。二十一年五月,奏明在金奔巴瓶內掣定,由班禪披授戒,賞其父策旺頓柱公爵。十月,拉達克部落勾結生番占踞藏境一千七百餘里,奪據達壩、噶爾及雜仁三處營寨。經駐藏大臣派戴琫等率兵攻剿,並將矛手番兵改挑槍兵,收復補仁營寨。又噶爾布倫等帶兵四面夾攻,殄斃森巴及拉達克大小頭目四十餘、賊匪二百餘,拉達克頭人八底部長乞降,公稟投歸唐古特商上,原繳所占凡、湯及達壩、噶爾四處營寨,並準堆噶爾本挖金番民酌定五百名,由前後藏番民內擇精壯派往充當金夫,派戴琫一、如琫二、甲琫二,定駐守,教習技藝。二十二年四月,由前藏迤東日申寺迎至布達拉坐床。二十四年,以濟嚨呼圖克圖阿旺羅布藏丹貞嘉木錯為正師傅,以降孜曲結喇嘛羅布藏冷竹布為副師傅。

駐藏大臣琦善奏參噶勒丹錫埒圖薩瑪第巴克什諾們汗阿旺扎木巴勒楚勒齊木巴什擅作威福,貪黷營私,所有被控各款,訊擬結奏聞。經理籓院議得:「已革諾們汗阿旺扎木巴勒楚勒齊木巴什,洮州夷僧,本系入冊檔一微末喇嘛,自其前輩歷受三朝重恩,在雍和宮傳經,旋命赴藏坐宗喀巴床,派充達賴師傅,敕封諾們汗薩瑪第巴克什名號,遞加衍宗翊教靖遠懋功禪師,又加賞達爾汗,屢頒御書匾額以榮之,宜如何清潔潛修,公正自矢。乃竟不知守分,膽敢需索番屬財物,侵占百姓田廬,私拆達賴所建房屋,擅用未蒙恩賞轎傘。更強據商產,隱匿逃人。鈐用印信不在公所,進呈貢物不出己貲,濫支濫取,任性聽斷,恣意侵凌。甚至達賴起居不能加意照料,房內服侍無人,以致達賴頸上帶傷,流血不止,始則忽而不防,繼且知而不問。蓋當達賴受傷時,隨侍只森琫一人,此森琫即為該諾們汗之隨侍。近兩輩之達賴,每屆接辦印務以前,輒即圓寂,不得安享遐齡,其中情節,殆有不可問者。即放一扎薩克喇嘛,勒取財物,盈千累萬,尤屬駭人聽聞。」詔令將歷得職銜名號全行褫革,追敕剝黃名下徒眾全行撤出,廟內查封,發往黑龍江安置。所有財產,查抄變價,賠修藏屬各廟宇。旋命釋回,交地方嚴加管束。復捐輸銀兩請回前藏,又因廓爾喀軍事,請求開復回藏。均嚴旨不允。迨同治初元,病歿土爾扈特旗,準其留葬,不準轉世。門徒二十三人,留於該旗游牧。至光緒初年,土爾扈特王復請捐輸鉅款,代求轉世,始曲允其已轉世之呼畢勒罕得令為僧。

琦善尋奏改章程二十八條,又奏罷稽查商上出入及訓練番兵成例。故事,商上出入所有一切布施金銀,均按季奏報。自琦善奏定後,而中國御藏之財權失。又駐藏大臣及兵丁俸餉,向由福康安在廓爾喀經費內撥交商上生息,以資公用。及琦善議改章程,將生息取銷,一切由商務供給。迨後中國駐藏一切開支,藏人漸吝供給,而不知當日實有貲本發商生息,並非向商上分肥。總之,乾隆所定制度,蕩然無存矣。

是年十二月,敕諭第十一輩達賴喇嘛曰:「咨爾達賴喇嘛。朕撫綏寰宇,敷錫兆民,期一道以同風,冀九垓之遍德。亦賴洪宣梵義,普結善緣,導引群生,同參勝果。其有能通上乘,繼闡正宗,使諸部愚蒙悉資開悟者,宜加多楙獎,元沛寵封。茲以爾慧性深沉,經文諳習,既著靈蹤於齠歲,益堅戒律以壯年。承襲以來,皈依者眾。朕甚嘉之,故特依前輩達賴喇嘛例,封爾為大善自在佛所領天下釋教普通瓦赤喇呾喇嘛達賴喇嘛,改受金冊。爾尚振修黃教,主持烏斯,本利濟以佑民,迓庥祥而護國。所有圖伯特事務,其悉依例董率噶卜倫等,妥協商辦,報明駐藏大臣轉奏,俾圖伯特闔境延釐,眾生蒙福,彌勤啟迪,用副綏懷。茲隨冊齎往金銀、採幣、玻磁器皿,爾其敬承,以光我國家億萬年無疆之休命。欽哉!」

二十六年十二月,琦善以披楞,即英人,請定界通商聞,詔耆英以守成約拒之。二十七年七月,耆英復以英、德使請於西藏指明舊界派員前往聞。諭駐藏大臣斌良密查,如無流弊,自應照舊奏準允行,倘心懷詭譎,即當據理駮飭。並諭海善派員往查,事尋中輟。

二十八年,賞公爵策旺頓柱寶石頂、雙眼花翎。咸豐二年,達賴親往布賚繃、色拉、噶勒丹及南海、瓊科各寺院熬茶講經,詔幫辦大臣額勒亨額妥為照料。尋病歿,由駐藏大臣穆騰額奏駐藏守備童星魁前往護送。三年,達賴以發逆滋擾各省,虔誠念經,禱賊匪速滅,奉旨嘉獎。四年十月,理籓院議覆,淳齡奏達賴年已及歲,應宜任事。得旨:「達賴明年既已及歲,一切事務交伊掌管。所有賞給前輩之玉冊、玉印,凡遇吉祥之事準其鈐用,如常事仍用金印,以示廣興黃教至意。」五年正月,遵旨掌管政教事務。十二月,在布達拉圓寂,年十八。

十二輩阿旺羅布羅丹貝甲木參稱嘉穆錯,於咸豐六年在沃卡壩卓轉世。八年正月,奏明在金奔巴瓶內掣定。九年七月,迎至布達拉坐床。賞達賴之父彭錯策旺公爵。先是三年四月,廓爾喀商人與察木多番商索債起釁,聚眾械斗,互有殺傷,經駐藏大臣穆騰額照夷例分別罰款完案。嗣因多收稅米,阻擋商民,藉端與藏邊失和,唐古特屢戰不勝,宗喀、濟嚨、聶拉木等處均陷於賊。駐藏大臣赫特賀馳往後藏督辦防剿事宜,命成都將軍樂斌統漢土官兵繼進。廓番聞大兵將至,懼,遣其噶箕來藏上表乞和,詔許罷兵。唐古特與廓爾喀議定約十條,唐古特每年給廓爾喀稅課銀二千兩,廓爾喀將所占地方交還唐古特商上管理。同治元年,掌辦商上事務埒徵呼圖克圖因減放布施,連同色拉寺與布賚繃、噶勒丹兩寺,不勝,藏中僧俗公斥之,攜印潛逃赴京。詔黜其名號,不準轉世。命諾們汗汪曲結布協理商上事務。汪曲結布者,原系俗裝,曾為噶卜倫,即俗所謂「沙扎噶隆」是也。因與埒徵忤,辭官削發為僧,至是復起用。乃創修拉薩城垣,自西而東,工未竣而歿,遂罷役。初,駐藏官兵自游擊以下,均聚居扎什敦布營房。時駐藏大臣滿慶以藏中屢不靖,命遷拉薩市,從此僦屋而居。扎什城之營房遂廢。三年,噶勒丹池巴羅布藏青饒汪曲為達賴傳授小戒。

瞻對逆番久圍裏塘,梗塞驛路,其酋工布朗結復令期美工布大股逆賊至巴塘、里塘交界之三壩地方,劫去糧員行李,搶奪由藏發出摺報公文。其格吉地方亦有告急夷信。工布朗結曾於道光末,經前任川督琦善帶兵往剿,並未蕩平。以瞻對歸各土司侵地,奏予工布朗結職,罷兵。至是益無畏懼,將附近土司任意蠶食,川、藏商賈不通,兵餉轉運難艱,漢、番均困。駐藏大臣滿慶派番員徵兵借餉,並約三十九族調集各處土兵,防剿瞻對西北,川督駱秉章派員督飭打箭爐及巴、里各文武,同明正土司及大小金川等土司兵進攻其東南。而藏中所派之兵甫至巴塘,旋即搶掠,詔令撤回。至四年,事平。奉旨將上、中、下三瞻地方賞給達賴管理,建廟焚修。賞達賴之兄伊喜羅布汪曲承襲公爵。七年,親至前藏攢招。八年,捐修扎林噶舒金塔。十年,親往布賚繃、色拉二寺熬茶講經。十二年,親至前藏攢招。是年二月,遵旨接管政教事務。十三年及光緒元年,均親至前藏攢招。元年三月,在布達拉圓寂,年二十。

十三世阿旺羅布藏塔布克嘉穆錯,於光緒二年五月在達布甲擦營官屬下朗賴家轉世,至是呼畢勒罕訪獲,班禪率同有職各僧俗人等出具圖記公稟,懇請駐藏大臣松溎代奏。奉旨毋庸入瓶簽掣,即定為達賴之呼畢勒罕。四年正月,在貢湯德娃夫由班禪披授戒,取定法名。六月,迎至布達拉坐床,銷去呼畢勒罕名號。賞達賴之父工噶仁青公爵,寶石頂、孔雀翎。八年正月,由正師傅濟嚨呼圖克圖傳經授戒。

十年,因攢招,各處喇嘛麕集,與巴勒布商人購物起釁,將巴商八十三家全行劫毀。廓爾喀因索償損失銀三十餘萬兩,並集兵挾制。駐藏大臣色楞額奏派漢、番委員前往開導,曉以恩威,始允減為十八萬有奇。除唐古特商上捐籌及清出貨物抵價外,尚不敷銀六萬七千餘兩,奉旨由四川撥給。十一年,親至前藏攢招。十四年,工噶仁青故,賞達賴之兄頓柱奪吉公爵。是年親往布賚繃、色拉寺熬茶講經。十五年,親至前藏攢招。

當達賴降生之年,哲孟雄與布魯克巴部長因英並印度,與哲、布接壤,漸有窺藏心,籥請籌備。而廷旨不甚注重,謂披楞頭人現向布魯克巴部長租地修路,意欲來藏通商。惟布魯克巴與哲孟雄毗連,哲孟雄既已認租修路,難保不暗中勾結引進,詔松溎相機開導,務令各守疆界,勸諭阻回。哲人知中朝不知邊情,反疑其勾結滋事,於是漸暱英人,以捻都納為英租界,英竟視為保護地。藏人漸覺英之逼己,訟言哲人私結英約,屢議伐之,哲乃益親英人矣。

光緒十三年,藏人於隆吐設卡,遂與印度兵戰,敗焉。朝旨屢諭駐藏大臣文碩,令藏人撤卡。文碩奏,實藏地,卡無可撤。嚴旨責焉,以升泰代之。總署與英使議邊界通商,戒印兵毋進藏。藏番據新圖,以隆吐、日納宗為藏地,堅勿讓。文碩據以入告,而中旨謂:「向來西藏圖說藏地與哲、布分界處東西一線相齊,藏境中並無隆吐、日納宗之名。今文碩寄來新圖,隆吐、日納宗在藏南突出一塊,插入哲、布兩界之內,而布、藏分界之處,恰在捻都納修路東西一線之北,新圖以黃色為藏界,而日納宗官寨之地,註明數十年前喇嘛給與哲孟雄,現仍畫黃色,正與隆吐山相近,難保非藏人多畫此一段飾稱現界也。並著升泰詳細確查,究竟隆吐屬哲屬藏,據實覆奏,毋得稍有捏飾。」時樞廷以都察院劾文碩,革之。而升泰初到任時,猶知藏人理直,奏稱:「隆吐山南北本皆哲孟雄地方。英人雖視為保護境內,其實哲孟雄、布魯克巴皆西藏籓屬。每屆年終,兩部長必與駐藏大臣呈遞賀稟,駐藏大臣厚加賞賚以撫綏之。在唐古忒,則自達賴喇嘛以次,均有額定禮物,商上亦回賞緞疋、銀、茶,與兩部回信底稿,均呈送駐藏大臣查核,批準照繕,始行回覆。哲、布兩部遇有爭訟,亦稟由藏酌派漢、番辦理。此哲、布為藏地屬籓實在情形也。」奏上,置弗理。

藏人知文碩被議,不直中朝所為,遂自動思復仇。諭升泰嚴止之,僉憤。藏人誓眾曰:「凡我藏眾男女,誓不與英人共天地。有渝此誓,眾共殛之!」乃大集兵於帕克哩,將痛擊印軍。升泰搜得乾隆五十三年舊哲孟雄受逼於廓爾喀,達賴乃以日納宗給哲人;今哲私通英人,地應收回。升泰屢諭不從,印兵攻熱勒巴拉山,藏兵傷亡數百。印兵追入徵畢岔,印度政府令勿窮追。諭駐藏大臣赴邊界與印官會晤。英外部告駐英使劉瑞芬商議和平了結。藏人謂英若據有哲地,則誓不共立。十四年八月,印兵大隊收哲孟雄全部,攻藏兵於捻都納,藏兵敗退,咱利、亞東、朗熱諸隘並失,藏兵萬餘盡潰。印兵追噶卜倫等於仁進岡,與駐藏大臣所遣止戰武員蕭占先遇。占先豎漢字阻印兵,印兵止槍,約相見。占先約勿窮追,印兵官欲擊仁進岡民居。占先告以此為中國土,藏番違旨用兵,中國當嚴為處置,請勿進兵。印兵官諾之,要約速辦,乃退兵。藏兵既大敗失地,仍志在復仇,升泰屢嚴止之,不聽。藏人目漢官為洋黨,屢欲暴動,終為所懾而止。印官以天寒不能再緩,升泰即至邊界議約,而藏眾以噶卜倫中一二人主和,有壞黃教,群議投之藏江,力要駐藏大臣代索回哲孟雄、布魯克巴全境,否則傾眾一戰。藏兵復集大隊備四路。升泰抵藏力阻之,仍百計諭藏僧戒藏番毋妄動,乃馳赴邊界議約。

時沍寒,人馬多凍斃。抵帕克哩,隘外藏兵尚有萬人駐仁進岡。升泰命撤退,藏官言大臣尚未與印官晤,未敢遽撤,乃退扎數十里。哲孟雄部長命其弟來謁,言來見為印兵所阻。升泰與英官保爾會於納蕩。英官言:「哲孟雄與印度互立約已二十七年,應歸印度保護。藏與印構兵,藏既屢敗,我兵何難長驅卷藏全土?以邦交故。按兵靜候。」並索藏賠兵費。升泰言:「哲為藏屬。從前印、哲立約,並未見印督照會。藏番亦未赴印境滋擾,藏費無名。」英人又在布魯克巴及後藏乾灞修路,藏人又大震。英官要求甚奢,升泰力折之,藏人漸就範。

升泰屢要英撤兵,英不可。而藏眾已成軍之三大寺僧兵,及駐仁進岡之兵萬餘,皆撤退。噶卜倫及領袖僧官十餘,其他番官數十員,隨升泰至邊,皆駐仁進岡,不敢與英官晤。升泰以哲事未能即竣,大雪封山,運糧無所,亦退駐仁進岡。總署派英人赫政赴藏充通譯。哲孟雄部長之母率所屬親族連名上稟,言英官當年立約,不得過日喜曲河。哲孟雄租地與英,歲應納一萬二千圓。英人倚其國勢,歲久不給。印、藏構釁,復致殃及。伊母子親族實不原歸英,乞勿將哲境劃出版圖之外。英人既掠哲地全境,復押哲部長安置噶倫繃,以重兵駐哲境,招印度及廓爾喀游民闢地墾荒。廷議以哲事無從挽救,慮梗藏議,諭升泰勿許。布魯克巴地數倍哲孟雄,西人呼為布丹國,光緒間尚入貢。升泰至邊,部長遣兵千七百人護衛。升泰慮為英口實,謝去。並乞印綬封號,升泰允代請諸朝。藏、哲舊界本在雅納、支木兩山。其後商人往來之咱利為新闢捷徑,西人稱熱勒巴勒嶺。升泰議於咱利山先分藏、哲界以符前案,其印、哲之界在日喜曲河,擬於約中注明。印、哲立約在咸豐十一年,無案可稽,寘勿論。哲部長土朵朗思,印度稱為西金王,既被幽於噶倫繃,其母及子尚居春丕,即英人所稱徵畢也。印營假部長書取其兩子赴噶倫繃,部長母堅不可,挈其兩孫至升泰營哭訴;丐中朝作主,升泰無以援之也。英人又欲易置其部長,升泰婉止之。赫政阻雪久不達。

十五年二月,藏兵盡撤歸,升泰請總署告英電印兵速撤。三月,赫政至邊,藏兵盡撤。藏人言藏、哲本有舊界,日納宗既賜哲孟雄,其隆吐山之格壓傾倉地實有藏人游牧場,確為藏、哲舊界。至咱利山本無鄂博,不過上年實於此限止印人耳。通商極非所原,然不敢違朝命。惟咱利以內,洋人萬不可來。赫政赴營與議,英人謂咱利之界萬不可移,至哲孟雄與商上及駐藏大臣舊有禮節,均可仍之。惟西金界內藏番不得有此權,允此方可開議。升泰諾焉。印兵既撤退,英人尚久不訂約。升泰奏云:「聞藏人言,與有仇之英議和,孰若與無仇之俄通好?俄人前次來藏,我等備禮勸阻,俄即退去。今英謀吾地,偶爾戰勝,遂恣欺凌,實所不甘。查去年俄人有由和闐至藏之請。如英再延宕,則藏更生心。本年蒙古人由草地禮佛,絡繹不絕,隨來者頗類俄人。設藏番私與通款,則稽查不易。邊事久不定局,俄或私行勾結藏番,英、俄互相猜忌,則後患方長。乞告英使電催印督速定藏約。」十月,升泰奏:「英人擬撤兵之後,悉照向章,不必辦理通商,不必另立新約。通商一事,本英官初次會議即行提出。又屢言西人欲至藏貿易,答以番情疑詐,萬難辦理,然後許至江孜。力言再四,又許退至帕隘。仍復力拒,英官意拂然。彼時首重通商,否則萬難了結。臣力諭藏番,通商萬不可免,始據藏番出具遵結。今英人忽不言通商,亦自有故。當日英人深知藏番於此事力拒數年,意謂藏番必不遵行,故借以為難。今知出結遵辦,恐定約以後,他國援以為請,則藏地不能入其範圍,是以忽議中止。然英人不議通商,藏人實所深原,但能不自啟釁端,未嘗不可暫保無事。俄人亦不能有所干求,目前亦可免生枝節。惟日後防範宜嚴,未可再涉疏懈。現藏、印均已退兵,前怨已釋,自應彼此立約以昭信守。彼族恐一經定約,即不能狡焉思逞,故任意延緩。惟自入夏至今,曠日持久,虛糜時日,萬難再延。請速商英使,迅電印督,速行議結。」哲孟雄部長言原棄地居春丕,升泰止之。

十六年二月,以升泰為全權大臣,與印督定約八款:自布、坦交界之支英摯山起,至廓爾喀邊界止,分藏、哲界線;承認哲孟雄歸英保護;藏印通商、交涉、游牧三款俟議;簽約於印度孟加拉城;鈐印後,由大臣薛福成在倫敦互換。五月,給布魯克巴部長印。十七年三月,升泰奏移設納金要隘。八月,升泰奏稱改關游歷等部,藏番不遵開導,請仍在亞東立市。下所司知之。

十九年十月,派四川越巂營參將何長榮、稅務司赫政與英國政務司保爾在大吉嶺議定藏印通商交涉游牧條約九款:開亞東為商埠,聽英商貿易,添設靖西同知監督之,印政府派員駐扎,察看商務;自交界至亞東,任英商隨意來往;藏界內英人與中、藏人民訴訟,由中國邊界官與英員商辦;印度遞駐藏大臣文件,由印度駐哲孟雄之員交中國邊務委員驛遞;藏人至哲孟雄游牧,遵英國定章,與原約一律奏行。此約既訂,藏人以通商事英人獨享權利,而游牧事藏人反受限制,於亞東開埠之事不肯實行。

二十一年正月,榮增正師傅普爾覺沙布嚨為達賴傳授格隆大戒。是年掌辦商上事務前榮增師傅第穆呼圖克圖因病辭退。十一月,遵旨接管政教事務。二十四年,瞻對與川屬明正土司構辭,四川總督鹿傳霖奏明派兵攻取瞻對,成都將軍恭壽、駐藏大臣文海先後奏陳,而達賴亦密遣喇嘛羅桑稱勒等赴京呈訴。於是朝廷俯順番情,命將三瞻地方仍賞還達賴,毋庸改歸四川管理。是年,親赴色拉、布賚繃、噶勒丹三大寺熬茶講經。二十五年,親往前藏攢招。二十六年,殺其前掌辦商上事務榮增正師傅第穆呼圖克圖阿旺羅布藏稱勒饒結及其弟洛策等。第穆所居之闡宗寺財產,全行查抄入己,並咨請駐藏大臣裕鋼代奏,將第穆呼圖克圖名號永遠革除。是年,親赴南海、瓊科爾結等處熬茶講經。

二十九年,藏、英以爭界故,英兵進藏。初,達賴誤以俄羅斯為同教,親俄而遠英。雖兩次與英議定條約,迄未實行。俄員某偽作蒙古喇嘛裝束,秘密入藏,為達賴畫策,購置火器,意圖抗英,英雖偵知之而無如何也。至俄方東困於日本,不暇遠略,英遂藉事稱兵。詔裕鋼往解之。達賴恃俄員為謀主,不欲和,思與英人一戰,乃止裕鋼行,弗使番民支烏拉夫馬,並調集各路番兵。西藏番兵以乍丫為強,然無紀律。甫抵拉薩,即圍攻駐藏大臣衙署,死者數十人。後經藏官彈壓,開往前敵,未交綏,均潰變,由小路逃去。時藏兵屢敗,英兵日迫。詔解裕鋼任,尋革職。駐藏大臣有泰至藏,英軍猶駐堆補,約赴帕克里議和,照十六年條約辦理,原即休兵。有泰初與達賴商,原自往阻英兵,達賴尼之,然亦無他策,惟日令箭頭寺護法誦經詛咒英兵速死而已。既而有泰藉口商上不肯支應烏拉,不能啟程,僅以李福林往,怯不進。英軍至江孜,盼有泰赴議,有泰仍不敢行,藏人怨之。未幾,英人長驅直入,達賴聞知大懼,先一日以印授噶勒丹寺噶卜倫,倉皇北遁至青海。有泰以達賴平日跋扈妄為,臨時潛逃無蹤,請褫革達賴喇嘛名號。

榮赫鵬既得志,因列條約十款,迫噶勒丹寺噶卜倫羅生戛爾等簽約於拉薩:一、西藏允遵守光緒十六年中、英條約,並允認該第一款哲、藏邊界;二、江孜、噶大克、亞東三處開為商埠;三、四從略;五、自印邊該江孜、噶大克各通道不得阻礙;六、七從略;八、印邊至江孜、拉薩之砲臺山寨一律削平;九、以下五端,非得英國允許,不能舉辦:一西藏土地不準租讓與他國,二他國不準干涉西藏一切事宜,三他國不得派員入藏,四路礦電線及別項利權不許他國享受,五西藏進款貨物錢幣等不許給與各外國抵押撥兌。有泰往見榮赫鵬,自言無權,受制商上,不肯支應夫馬,榮赫鵬笑頷之。英人即據為中國在藏無主權之證。

其先有泰電外務部,言番眾再大敗,即有轉機。英軍進拉薩,圖壓服藏眾。及英軍至,與藏定約,誘有泰畫押,朝旨切責之。春丕暫住英兵,俟應償兵費二百五十萬盧比繳清即行撤退。朝廷以藏約損失之權太甚,命津海關道唐紹儀以三品卿加副都統銜赴藏全權議約。時議以藏事危急,宜經營四川土司,及時將三瞻收回,諭川督錫良等籌辦。錫良擬改土歸流,泰寧寺喇嘛以兵抗。朝命駐藏幫辦大臣鳳全馳往剿辦,至巴塘,為番眾所戕。錫良奏派四川建昌道趙爾豐會同四川提督馬維騏往。三十一年六月,馬維騏克復巴塘,趙爾豐繼至,接辦善後事宜,並搜捕餘匪,全境肅清。十一月,以里塘屬之鄉城桑披嶺寺嘗戕官弁,稔惡不法,派兵往討。翌年閏五月,克之,擒其渠魁,並克同惡之稻壩、貢噶嶺。詔以趙爾豐為邊務大臣。八月,至里塘,將裡塘土司改流,以防軍五營分駐里、巴改流之地。十二月,鹽井河西臘翁寺為亂,討平之。

三十三年正月,草創學務、農墾、水利、橋梁、採鐮、醫藥諸要政,粗具規模,設里化、定鄉、巴安等縣,並將應行興革諸大端次第陳奏,得部撥開辦經費一百萬兩。三十四年七月,會同川督趙爾巽奏設康安道,改打箭爐為康定府,設河口縣,里化同知,稻成縣、貢噶嶺縣丞,巴安府、三霸通判,定鄉縣、鹽井縣,並招募西軍三營。是秋因德格土司兄弟爭繼,奏明往辦。十二月,至德格,匪黨退保維渠卡,趙軍進攻,至翌年六月降之。德格肅清,土司請納土改流,乃招集百姓議定賦稅。九月,春科、高日兩土司及靈蔥土司之郎吉嶺均改流,又渡金沙口巡閱春科地方。十月,三十九族波密內附,八宿請改官,均撫循之,並派兵驅剿類伍齊、碩搬多、洛隆宗、邊壩阻路之番人,遂分兵取江卡、貢覺、桑昂、雜瑜,咸收服之。

宣統二年正月,邊軍越丹達山以西,直抵江達。是時川軍正擬入藏,特為聲援,並奏請與藏人於江達畫界,設邊北道、登科府、德化州、白玉州、同普縣、石渠縣,遂巡閱乍丫、煙袋塘、阿足,設乍丫委員。定鄉兵變,派鳳山討平之。三巖野番索戰,派傅嵩矞討平之,設三巖委員。二月,以巴塘屬之得榮、浪藏梗命,派兵攻克之,設得榮委員,並收服浪藏寺北之冷石卡。嗣趙爾豐督川,以傅嵩矞代理邊務大臣。五月,趙爾豐、傅嵩矞以兵至孔撒、麻書,收其地,設甘孜委員,並檄靈蔥、白利、倬倭、單東、魚科、明正各土司繳印,改土歸流。色達及上羅科野番來投。六月,至瞻對,逐藏官,收其地,設瞻對委員。旋返打箭爐,檄魚通、卓斯各土司繳印改流,又收復咱里、冷邊、沈邊三土司。魚科土司抗不繳印,擊破之,魚科降。於是傅嵩矞以邊地各土司先後改流,已成行省規模,乃建議,以為川邊故康地,其地在西,設行省曰西康,建方鎮以為川、滇屏蔽。以邊務大臣為西康巡撫,改邊務支局為度支司,關外學務局為提學司,康安道為提法司,邊北道為民政司。自打箭爐以西至丹達山,三千餘里,南抵維西、中甸,北至甘肅西寧,四千餘里,均為西康轄境。既入奏,於是年七月,崇喜、納奪土司先後繳印。八月,又傳檄察木多、乍丫兩呼圖克圖改流設理事官,於是西康全局遂以底定。嗣值鼎革,川局又變,建省之議卒不果行。

當唐紹儀之議約也,於光緒三十一年正月至印度,與英議約專使費利夏會議多次。英使諱言廢約,允商訂修改。紹儀易其七八,費謂無異廢約,堅拒焉。費雖名全權,而約事多主於印度總督冠仁,紹儀面揭之,費乃允商。第九款又力辨主國、上國之據,狡展不讓,乃借遼沈議約事奉命回京,留參贊張廕棠在印接議。英仍堅持初議,卒無結果。會英內閣更易,宗旨稍變,駐京英使薩道義接英政府訓,將條約稿稍有更易,命在京外務部商訂。政府以西藏與英屬印度接壤,歷年邊界交涉,爭端屢起,中國兩次與英訂約,無非以睦鄰之計為固圉之謀,英新政府既有意轉圜,仍飭該使臣在京續商。在我自當早圖結束,以保主權,因由唐紹儀與英使薩道義訂定藏、印續款六款:一光緒三十年七月英、藏所立之約暨英文、漢文約本,附入現立之約,作為附約,彼此允認,切實遵守,並將更訂批準之文據亦附入此約。如遇有應行設法,彼此隨時設法,將該約內各節切實辦理。

二英國國家允不占並藏境及不干涉西藏一切政治,中國國家亦應允不認他外國干涉藏境及一切內治。三光緒三十年七月英、藏所立之約第九款內之第四節所聲明各項權利,除中國獨能享受外,不許他國國家及他國人民享受。惟經與中國商定在該約第二款指明之各商埠,英國應得設電線通報印度境內之利益。

四所有光緒十六、十九年中國與英國所定兩次藏、印條約,其所載各款,如與本約及附約無違背者,概應切實施行。五、六從略。以挽救前約之失,藏應償兵費一百二十餘萬兩。朝廷允代籌還,英人始無辭,於北京簽押。旋有泰被言官彈劾,詔五品京堂張廕棠前往查辦。有泰及其隨員均獲罪,褫革謫戍有差。

廕棠入藏,三十二年,專辦開設商埠事。時英軍尚駐春丕,照約俟三埠開妥、賠款清交始撤兵,故開埠尤亟亟也。三十四年,政府以光緒三十二年附約第三款內載中、英條約所有更改之處另行酌辦等語,特派張廕棠為全權大臣,與英專使韋禮敦議訂藏、印通商章程十五款。其要者:二劃定江孜商埠界線。四英、印人民與中、藏人爭論,由英商務委員與中、藏官員會同查訊,面議辦法。六英軍撤退後,印邊至江孜一路旅舍,由中國贖回,所有電線,俟中國電線接修至江孜後,亦酌量售與中國。八已開及將開各埠,英商務委員因往來印邊界文件,得用傳遞夫役。又英國官商雇用中、藏人民作合法事業,不得稍加限制。九凡往來各商埠之英官民貨物,應確循印、藏邊界之商路,不得擅經他處。十英國人民可任便以貨物或銀錢交易,任便將貨物出售,或購買土產,不得限制抑勒。此約除中、英簽押外,並有西藏噶卜倫汪曲結布隨同畫押。實開三方並列先例,藏局又為一變。厥後英、藏交涉日繁,而政府撫馭藏番,既有英、藏拉薩之約在先,其事益臻艱困。至宣統季年,遂有經略川邊及達賴二次出亡之事。

自光緒三十年達賴與英境啟釁戰敗出奔後,卓錫於庫倫,意在投俄,而與哲布尊丹巴呼圖克圖不睦。經庫倫辦事大臣德麟電奏乞援,詔西寧辦事大臣延祉俟過冬後迎護至西寧。而達賴又欲在代臣王旗小住,廷旨以王旗部落甚小,達賴隨帶人眾,恐難供億。翌年,僑居塔爾寺,又與阿嘉呼圖克圖同居一處,積不相能。陜甘總督升允奏:「達賴性情貪嗇,久駐思歸,應否準其回藏?」得旨:「俟藏務大定,再行回藏。」而調阿嘉來京以和解之。旋由西寧往五臺山,折而至京,覲見於仁壽殿,如順治朝,優禮有加。三十四年十月,以萬壽節率徒祝嘏,特加封號,以昭優異。懿旨曰:「達賴喇嘛業經循照舊制,封為西天大善自在佛,茲特加封為誠順贊化西天大善自在佛,並按年賞給廩餼銀一萬兩,由四川籓庫分季支發。達賴喇嘛受封後,即令仍回西藏,經過地方,派員妥為照料。到藏以後,當確遵主國之典章,揚中朝之信義,並化導番眾,謹守法度,習為善良。所有事務,依例報明駐藏大臣,隨時轉奏,恭候定奪,期使疆宇永保治安,僧俗悉除畛域,以無負朝廷護持黃教、綏靖邊陲至意。」旋以國有大喪,受封未便舉行。達賴以不服水土請,詔令先行起程,至塔爾寺受封。又值停止筵宴之時,未便設餞,仍派大臣護送,如來時禮節。至西寧,即請將阿嘉斥革,並以此事為回藏之要挾。達賴聘練兵教習十餘人,影射蒙古,實系俄人,多購軍火回藏。

初,張廕棠以西藏地當沖要,英、俄環伺,自非早籌整頓,難以圖存。建議以漢員指揮,另派北洋新軍入藏,分駐要塞,以厚聲援。駐藏大臣聯豫疏陳藏中情形,亦有派遣軍隊之請。會川邊藏番擾亂,進攻三崖。三崖者,本巴塘屬地,與德格、多納兩土司接壤,向歸川省管轄。乃藏番察臺三大寺無端派番官帶兵占據上崖,調渣鴉、江卡各土司助兵,逼勒崖夷投降,並遍肆煽惑,打箭爐一帶均為震動。同時瞻對番官句結德格土司之弟為亂,逐其兄。爐城文武據報,派麻書土千總江文荃查辦,均被圍困。經川督入奏,廷議以三崖、德格均系川境,番官竟敢糾眾侵逼,再事優容,恐番焰日張,土司解體。命川督會同趙爾豐相機籌辦。爾豐電奏力主用兵,並稱此次藏番與達賴有關系,請飭達賴傳諭退兵。乃飭達籌、張廕棠詰問,達賴答詞閃爍,意涉支吾。政府以達賴縱肯戒飭番眾,而萬里遺書,需時甚久,三崖等處被攻正急,何能久待,遂電爾豐進剿。

三十四年冬,番兵調集益眾,近逼鹽井,並聲言索戰。雖經川軍擊敗,番眾仍未退卻,揚言阻止趙爾豐入藏。政府以藏番舉動,顯系有恃不恐,藏地介在強鄰之間,意存首鼠,自非設法經營,無以保我邊圉。因思光緒三十三、四年間聯豫等條陳有善後辦法二十四條,創財政、督練、路礦、鹽茶、學務、巡警、農務、工商、交涉九局,擬即採擇試辦。但無兵不敷彈壓,多名又恐難相安,擬先設兵三千。其一千由川督就川兵挑選精銳,厚給餉械,派得力統領率之入藏,歸駐藏大臣節制調遣。餘二千由駐藏大臣就近選募,另調川中哨弁官長,俾任訓練統率之事,以期持久。聯豫、趙爾巽覆奏贊其議,遂派知府鍾穎統領川兵,於宣統元年六月啟程入藏,取道德格,繞過江卡至察木多。藏番在恩達、類烏齊一帶,擬聚兵堵截。十一月,川軍抵類烏齊,藏番不戰自退,川軍遂由三十九族間道前進。十二月,抵拉里、江達。番兵聞川軍且至,焚其積聚,劫殺漢兵扼守。川軍進擊,大破之。

達賴自光緒三十四年由西寧入覲,出京回藏,沿途逗留,又繞道德格等處,遷延不進,其冬,始回拉薩。二年正月,達賴聞川軍將至,乘夜西奔,潛赴印度,川軍遂轉戰入藏。朝廷得聯豫奏報,降旨數達賴罪惡,革去名號,一面責成聯豫、趙爾豐會籌防務,安輯軍民;一面降旨另訪呼畢勒罕,以噶勒丹池巴羅布藏丹巴代理商上事宜,其噶卜倫以下各藏官供職如故,藏中僧俗亦安堵無事。是年三月,聯豫請於曲水、哈拉烏蘇、江達、碩般多及三十九族各設委員一。三年二月,聯豫奏裁駐藏幫辦大臣,改設左右參贊,以羅長崟、錢錫寶為之。會波密事起,聯豫遣鍾穎攻之不克,旋遣羅長崟會趙爾豐軍平之。其秋,川軍變,逐聯豫,推鍾穎代之,達賴始乘機重回拉薩。以此次出奔深賴英人保護,態度一變,於是逐鍾穎而獨立,中、英之交涉益紛紜矣。

班禪第一輩凱珠巴格勒克,為宗喀巴二弟子。出世至第五輩羅布藏伊什,仍號班禪呼圖克圖。康熙三十四年,命御史鍾申保等賚敕召來京,前藏第巴桑結以未出痘辭。五十二年,詔以班禪為人安靜,精通經典,勤修貢職,封為班禪額爾德尼,頒發金印、金冊。六輩羅布藏巴勒墊伊西,乾隆四十三年,請祝七旬萬壽,許之。迎護筵宴諸禮,概從優異,如順治九年達賴來覲例。四十五年八月,在熱河祝嘏,至京居西黃寺。是年頒賜玉印玉冊,以痘圓寂。命理籓院尚書博清額為駐藏辦事大臣,護送舍利金龕回藏。

第七輩羅布藏巴勒墊丹貝宜瑪,五十三年,以廓爾喀擾邊,命移泰寧,俟平復歸後藏。道光十五年,給金冊。二十一年,以接濟征森巴兵餉,加「宣化綏疆」封號。咸豐元年,賚七旬壽,如六旬所賜。次年,圓寂,年七十三。

第八輩羅布藏班墊格曲吉札克丹巴貝汪曲,年二十九。至第九輩羅布藏吐巴丹曲吉宜瑪格勒克拉木結,光緒十八年正月,迎至扎什倫布坐床,賞其外祖父期差汪布本身輔國公。三十一年,英人入藏,詔班禪留後藏鎮攝。十一月,班禪隨英皇子游歷印度,有泰勸阻,不從。十二月,由印回藏,諭以情詞恭順,原擅行出境之咎勿治,諄令恪供職守。張廕棠奏班禪受英唆使,屢與達賴牴牾,而全藏實權仍歸達賴替身掌握。電告外務部,請以恩澤籠絡班禪,並羈縻達賴,勿急旋藏。既而達賴將由西寧起程,班禪請自迎之,而實不行。達賴抵拉薩,班禪即請覲。諭訓聯豫等,班禪來京,於藏中情形是否相宜。其後達賴獨立,班禪亦不克安於藏矣。

統計達賴所轄寺廟三千五百五十餘所,喇嘛三十萬二千五百有奇,黑人十二萬一千四百三十八戶。班禪所轄寺廟三百二十七,喇嘛萬三千七百有奇,黑人六千七百五十二戶。西藏有爵五:輔國公三,一由貝子降襲,一由鎮國公降襲,一定世襲;一等臺吉扎薩克一;一等臺吉一。而達賴、班禪之親以恩封者不與。凡前後藏官,均由駐藏大臣分別會同達賴、班禪選補。前藏唐古特官,噶卜倫四人,三品,為總辦藏務之官,其俗稱之曰「四相」,議事之所曰噶廈。其次仔琫及商卓特巴各二人,皆四品。業爾倉巴二人,朗仔轄二人,協爾幫二人,碩第巴二人,皆五品。達琫二人,大中譯二人,卓尼爾三人,皆六品。仔琫、商卓特巴為商上辦事之官。凡喇嘛謂庫藏出納之所曰商上。業爾倉巴為管糧之官,朗仔轄為管街道之官,協爾幫為管刑名之官,碩第巴為管理布達拉一帶番民之官,達琫為管馬廠之官,大中譯、卓尼爾等為噶廈辦事之官。管兵者曰戴琫,六人,四品。如琫十二人,五品。甲琫二十四人,六品。定琫一百二十人,七品。多東科爾族任之。

其治理地方者曰營官。前藏大營十:曰乃東,曰瓊結,曰貢噶爾,曰侖孜,曰桑昂曲宗,曰工布則岡,曰江孜,曰昔孜,曰協噶爾,曰納倉,營官皆五品。後藏大營三:曰拉孜,曰練營,曰金龍,營官皆五品。前藏中營四十三:曰洛隆宗,曰角木宗,曰打孜,曰桑葉,曰巴浪,曰仁本,曰仁孜,曰朗嶺,曰宗喀,曰撒噶,曰作岡,曰達爾宗,曰江達,曰古浪,曰沃卡,曰冷竹宗,曰曲水,曰突宗,曰僧宗,曰雜仁,曰茹拕,曰鎖莊子,曰奪,曰結登,曰直谷,曰碩般多,曰拉里,曰朗,曰沃隆,曰墨竹宮,曰卡爾孜,曰文扎卡,曰轄魯,曰策堆得,曰達爾瑪,曰聶母,曰拉噶孜,曰嶺,曰納布,曰嶺噶爾,曰錯朗,曰羊八井,曰麻爾江。後藏中營十四:曰昂忍,曰仁侵孜,曰結侵孜,曰帕克仲,曰翁貢,曰乾殿熱布結,曰扎布甲,曰里卜,曰德慶熱布結,曰央,曰絨錯,曰蔥堆,曰脅,曰乾壩,營官皆六品。前藏小營二十五:曰雅爾堆,曰金東,曰拉歲,曰撒拉,曰浪蕩,曰頗章,曰札溪,曰色,曰堆沖,曰汪墊,曰甲錯,曰拉康,曰瓊科爾結,曰蔡里,曰曲隆,曰扎稱,曰折布嶺,曰扎什,曰洛美,曰嘉爾布,曰朗茹,曰里烏,曰降,曰業黨,曰工布塘;後藏小營十五:曰彭錯嶺,曰倫珠子,曰拉耳塘,曰達爾結,曰甲沖,曰哲宗,曰擦耳,曰晤欲,曰碌洞,曰科朗,曰哲喜孜,曰波多,曰達木牛廠,曰凍噶爾,曰札茹;營官皆七品。而前藏邊營十四:曰江卡,曰堆噶爾本,曰噶喇烏蘇,曰錯拉,曰帕克里,曰定結,曰聶拉木,曰濟隴,曰官覺,曰補仁,曰博窩,曰工布碩卡,曰絨轄爾,曰達巴喀爾,營官皆五品。每營營官一人或二人,以喇嘛、黑人參任之。

喇嘛之有游牧者,東起乍丫達呼圖克圖,與四川打箭爐所屬土司接,其西為察木多吧克巴拉呼圖克圖,又西為碩般多喇嘛,又西為類烏齊呼圖克圖,碩般多、類烏齊之北,皆與西藏大臣所屬土司接。碩般多之南,為八所喇嘛,又南為工布什卡喇嘛。類烏齊之西,為墨竹宮喇嘛,又西為噶勒丹喇嘛。類烏齊之西北,為贊墊喇嘛,介居西藏大臣所屬各土司之間,其西為埒徵喇嘛。噶勒丹之西為色拉喇嘛,西與布達拉接。噶勒丹之南,為瓊科爾結喇嘛,其西為丈扎卡喇嘛,又西為松熱嶺喇嘛,又西為那仁曲第喇嘛,又西南為乃東喇嘛,北與布達拉接。乃東之西,為瓊結喇嘛。布達拉之西北,為布勒繃喇嘛,又西北為羊八井喇嘛,其西為朗嶺喇嘛,西與扎什倫布接。朗嶺之南,為仁本喇嘛,其西南為江孜喇嘛,又西南為岡堅喇嘛。岡堅之西,為協噶爾喇嘛。協噶爾之西,為聶拉木喇嘛。朗嶺之西,為撒噶喇嘛,又西為雜仁喇嘛。其直屬於駐藏大臣者,有達木額魯特八旗:在喜湯者四旗,在湯寧者二旗,在佛山者一旗,皆北倚布幹山,南與前藏接;在格拉者一旗,東北濱喀喇烏蘇,西與後藏接。每旗置佐領一。

有三十九族土司:曰瓊布噶魯,曰瓊布巴爾查,曰瓊布納克魯,曰勒納夥爾,曰色裏瓊扎尼查爾,曰色裏瓊扎參嘛布瑪,曰色裏瓊扎嘛嚕,曰木硃特羊巴,曰布米特勒達克,曰木硃特尼牙木查,曰木硃特利松嘛吧,曰木硃特多嘛巴,曰勒遠夥爾,曰依戎夥爾移他瑪,曰查楚和爾孫提瑪爾,曰巴爾達山木多川目桑,曰嘛拉布什嘛弄,曰窩柱特只多,曰窩柱特娃拉,曰彭楚克夥爾,曰彭楚克彭他瑪爾,曰彭楚克拉寨,曰盆索納克書達格魯克,曰沁體牙岡納克書畢魯,曰盆沙尼牙固納克書色爾查,曰巴爾達穆納克喜奔盆,曰納格沙拉克書拉克什,曰洛克納克書貢巴,曰三渣,曰三納拉巴,曰撲旅,曰上阿扎克,曰下阿扎克,曰白獵扎嘛爾,曰上岡噶魯,曰下岡噶魯,曰上奪爾樹,曰下奪爾樹。皆土納馬賦,總之以夷情章京。

山之大者,曰岡底斯山,即昆侖,為東半球眾山眾水之祖;曰僧格山;曰郎千山;曰瑪加布山;曰達木楚克山;曰朗布山;曰巴薩通拉木山;曰諾莫渾烏巴什山,是三山即三危。川之大者,曰鄂穆河,下游為瀾滄江;曰喀喇烏蘇河,即黑水,下游為潞江;曰薄藏布河;曰雅魯藏布江,亦曰大金沙江;曰朋楚河;曰岡噶江。澤之大者,曰瑪帕本達賴池,曰郎噶池,曰牙毋魯克池,曰騰格里池,曰牙爾佳池。其物產自靖西東之堆朗至薩馬達一帶,皆有五金煤礦。其金礦最著者,曰爾倉,曰噶大克。出鹽最著者,曰勒牙,曰雅幹,凡十三。

其疆界西接印度之拉達克部,西南接洛敏湯、作木朗、廓爾喀諸部,南接哲孟雄、布魯克巴各部及珞瑜茹巴之怒江,東接四川巴塘之南墩寧靜山,東南接雲南維西,東北接西寧所管之邦木稱、巴彥諸土司,北至木魯烏蘇,接西寧所屬玉樹諸土司,西北至噶爾藏骨岔、阿爾坦諾爾一帶,接新疆和闐、莎車。


\end{pinyinscope}