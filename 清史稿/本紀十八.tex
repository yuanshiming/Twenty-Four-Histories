\article{本紀十八}

\begin{pinyinscope}
宣宗本紀二

十一年春正月辛酉,扎隆阿免,以哈哴阿、楊芳護理喀什噶爾參贊大臣。乙丑,容安論斬。丙子,以魏元烺為福建巡撫。朝鮮國王李鍚請封其孫奐為世孫,貢方物。是月,給江蘇沛縣、安徽蕪湖等八州縣衛、浙江富陽縣被災口糧。貸三姓、雙城堡兵民,直隸磁州等九州縣,湖南安鄉、華容二縣,河南武安縣,甘肅會寧等五縣被災口糧、屋費、籽種。蠲緩吉林等四處兵民新舊額賦。

二月己丑,御經筵。辛卯,以謁西陵,命奕紹等留京辦事。那彥成以驅逐安集延回民啟釁,褫太子太保,並褫其子容照侍郎。乙未,褫那彥成職,調琦善為直隸總督,王鼎署。以鄂山為四川總督,那彥寶署,史譜為陜西巡撫。戊戌,申禁各省種鬻鴉片。辛丑,上謁西陵。乙巳,謁泰陵、泰東陵、昌陵。上閱視萬年吉地,賜名龍泉峪。丙午,上再謁昌陵,行敷土禮。御隆恩殿行大饗禮。是月,貸湖北荊門營上年被水兵丁倉穀。

三月癸丑朔,釋英和並其子奎照、奎耀回京。廣東黎匪作亂,命李鴻賓剿之。辛酉,以廣東貿易英吉利人違禁令,命李鴻賓等查覆。是月,貸湖北督標、撫標暨武昌、黃州各營兵丁倉穀。

夏四月戊子,上閱健銳營兵。癸卯,上詣黑龍潭神祠祈雨。廣東黎匪平。

五月丙寅,湯金釗緣事降,並罷上書房總師傅。調潘世恩為吏部尚書,硃士彥工部尚書,白鎔都察院左都御史。戊辰,命長齡赴喀什噶爾商辦剿撫及善後事宜。辛未,雨。

六月丙申,申定官民買食鴉片煙罪例。己亥,賑安徽泗州等二十五州縣水災。庚戌,以湖北沔陽等二十州縣水災,命平糶倉穀,免湖北關津米稅。是月,給江蘇上元等九縣衛水災口糧。貸江蘇淮安衛災屯籽種。

秋七月戊午,命陶澍偕程祖洛辦江蘇災賑。以安徽水災,準鄧廷楨買鄰省米麥平糶,並備兵糈。癸酉,以誣陷回王伊薩克叛逆,扎隆阿論斬。辛未,移回疆參贊大臣及和闐領隊大臣駐葉爾羌,添設總兵駐巴爾楚克。己卯,命穆彰阿、硃士彥往江南查辦賑務。是月,給湖南武陵等五州縣,貴州桐梓縣、石峴衛水災口糧。貸江蘇江寧等六營災區兵餉。

八月己丑,萬壽節,上詣皇太后宮行禮。御正大光明殿,王以下文武各官,蒙古王公、外籓使臣等行慶賀禮,停筵宴。辛卯,晉長齡太傅。乙未,松筠病免,調穆彰阿兵部尚書,富俊工部尚書。以博啟圖為理籓院尚書。辛丑,暹羅國王遣貢使載內地遭風官民回廣東,溫諭獎賚之。癸卯,以保昌為熱河都統。以吳榮光為湖南巡撫。是月,給江蘇甘泉等十一州縣、湖北江夏等十六州縣、江西德化等二十縣水災口糧籽種。貸江南江寧駐防及溧陽營兵米。

九月甲子,福克精阿緣事褫職,以寶興為吉林將軍。丁丑,越南國王遣使送遭風難民回福建,溫諭獎賚之。

冬十月,嚴烺病免,以林則徐為河東河道總督。己丑,改喀什噶爾幫辦大臣為領隊大臣。乙未,命截留江西漕米八萬石賑南昌、九江饑民。是月,賑安徽無為等二十三州縣衛、江蘇上元等二十六州縣、浙江仁和等七縣衛、兩淮丁谿等六場水災。給安徽桐城等十州縣,湖南武陵等五縣,江西德化縣口糧、修屋費。貸甘肅皋蘭等十八州縣口糧,湖南武陵、龍陽二縣民堤修費。

十一月丙辰,大學士托津病免。授吳邦慶漕運總督。己巳,松筠罷內大臣,降三品頂戴休致。是月,貸奉天鐵嶺等五州及巨流河四處口糧,江西南昌等六縣修堤費。蠲緩寧古塔、雙城堡雹災霜災新舊額賦。

十二月乙酉,以富俊為大學士,管兵部,文孚協辦大學士。調穆彰阿為工部尚書。以那清安為兵部尚書,升寅為都察院左都御史,彥德為綏遠城將軍。乙巳,以吳邦慶為江西巡撫,蘇成額為漕運總督。是月,展賑湖北江夏等十六州縣水災。給江蘇上元等二十五州縣衛及丁谿十五場水災口糧。貸江蘇鎮江等二十七營,湖南常德、澧州各營水災兵餉。

十二年春正月辛酉,免浙江杭州等三府商船米稅。丁卯,陳若霖免,以戴敦元署刑部尚書。癸酉,王引之丁憂,以汪守和為禮部尚書。是月,賑安徽懷寧等二十一州縣水災旱災,並給懷寧等十七縣衛貧民口糧。貸直隸大名等四州縣、河南商丘等三縣災民米穀,陜西葭州等四州縣、江西南昌等十六縣、湖北江夏等二十州縣衛、湖南武陵等十州縣衛、甘肅渭源等七州縣、貴州桐梓縣災民口糧籽種。

二月戊寅,湖南江華縣瑤賊趙金龍作亂,命盧坤等剿之。己卯,御經筵。甲申,梁中靖奏查辦邪教株連冤抑,諭斥之。辛卯,鍾昌降調。授戴敦元刑部尚書。乙未,閩浙總督孫爾準卒,以程祖洛為閩浙總督,林則徐為江蘇巡撫,吳邦慶為河東河道總督,周之琦為江西巡撫。丙申,命李鴻賓剿瑤賊。壬寅,以謁東陵,命奕紹等留京辦事。

三月己酉,湖南提督海陵阿、副將馬韜等剿瑤賊於寧遠,失利,死之。壬子,上謁東陵,免經過地方額賦十分之三。乙卯,上謁昭西陵、孝陵、孝東陵、景陵、裕陵。丙辰,召瑚松額,以奕顥署盛京將軍。己未,上幸南苑行圍。庚申,召長齡。癸亥,上還京師。庚午,命戶部尚書禧恩赴湖南剿瑤賊,以文孚署戶部尚書。是月,展賑湖北江夏、漢川二縣水災。給安徽青陽縣災民口糧。貸甘肅皋蘭等七州縣災民、湖南乾州等五州縣屯丁口糧籽種,湖北督標、提標及武昌城守營被災兵丁倉穀。

夏四月癸巳,祈雨。戊戌,雨。辛丑,賜吳鍾駿等二百六人進士及第出身有差。乙巳,盧坤等敗瑤賊於羊泉,盡殲之,獲趙金龍子及賊首五十餘人。是月,再給江蘇揚州水災倉穀。

五月丁未,減福建水陸各營及浙江馬步兵有差。壬子,以趙金龍已斃,餘賊悉平,賞盧坤、羅思舉雙眼花翎、一等輕車都尉世職,加湖南提督餘步雲太子少保。乙卯,教匪尹老須等伏誅。庚申,上祈雨於黑龍潭。戊辰,上詣天神壇祈雨。己巳,詔刑部清釐庶獄。是月,貸山西大同等三縣被災兵民倉穀。

六月庚辰,上步詣社稷壇祈雨。壬午,求直言。丁亥,上詣黑龍潭祈雨。壬辰,以廣東提督劉榮慶剿連州瑤賊失利,褫職,李鴻賓褫職留任。癸巳,上步詣方澤祈雨。乙未,富俊以旱乞罷。不允。丙申,霍罕遣使進表,歸所手虜喀什噶爾回民。丁酉,復松筠頭品頂戴。癸卯,上自齋宮步詣圜丘行大雩禮。是日,雨。甲辰,命禧恩、瑚松額自湖南赴廣東剿瑤賊。是月,貸江蘇淮安衛水災屯田籽種。

秋七月丁未,宥容安,遣戍吉林。戊申,以鍾昌為科布多參贊大臣。命程祖洛清理浙江鹽政。和闐回塔瓦克等糾眾作亂,捕誅之。乙丑,廣西賀縣瑤盤均華等作亂,祁剿平之。是月,賑福建澎湖風災。給湖北天門縣水災口糧。

八月,陶澍奏英船再入內洋,或不遵約束,當嚴懲。諭以啟釁斥之。甲午,李鴻賓褫職,並提督劉榮慶逮問。調盧坤為兩廣總督。命阮元協辦大學士,仍留雲貴總督任。以訥爾經額為湖廣總督,鍾祥為山東巡撫。是月,賑山西朔州水災。蠲緩安徽懷寧等二十九州縣衛上年水災旱災額賦。

九月甲辰朔,以尹濟源為山西巡撫。丙午,南河龍窩汛堤盜決,命穆彰阿會同陶澍查辦,張井褫職留任。丁未,以英吉利船闌入內洋,命沿海整飭水師。甲寅,以特依順保為伊犁將軍。戊午,廣東連州瑤平。湖南瑤趙幅金等伏誅。是月,給江蘇桃源縣、湖北天門縣等七縣衛水災口糧。貸山西山陰縣歉收倉穀。

閏九月丁亥,上簡閱健銳營兵。壬寅,以朝鮮國王李鍚卻英吉利貿易,下詔褒獎之。是月,賑直隸阜平等十州縣災民。貸河南祥符等七州縣、陜西興安府水災口糧。貸齊齊哈爾等處被旱兵丁銀穀。

冬十月乙巳,廣東曲江、乳源兩縣盜匪作亂,剿平之。丙午,命硃士彥、敬徵往江南查辦事件。乙丑,命穆彰阿至湖北會同訥爾經額查辦事件。是月,賑直隸吳橋、東光二縣,江蘇桃源等三州縣,湖北漢川等四縣衛,安徽五河縣,兩淮板浦等三場水災旱災。給江蘇海州等四州縣,安徽五河等十一縣衛,湖南安鄉、華容二縣,奉天錦州府屬旗民口糧。貸山西大同鎮災區駐防倉穀。蠲緩直隸吳橋等十七州縣,江蘇桃源等六十三州縣衛,安徽五河等三十九州縣衛,浙江海寧等二十二州縣衛、仁和場,兩淮富安等十四場,湖南安鄉等七州縣衛,山西隰州等六州縣,湖北漢川等二十六州縣衛水災旱災雹災新舊額賦。

十一月戊寅,命署福州將軍瑚松額為欽差大臣,都統哈哴阿為參贊大臣,赴臺灣剿匪。丙申,撥京倉米一萬石賑順天府武清等八州縣災民。丁酉,李鴻賓遣戍烏魯木齊,劉榮慶遣戍伊犁。是月,貸陜西漢中等五府州屬、甘肅宜禾縣被災口糧,吉林等七處籽種。蠲緩甘肅宜禾縣逋租。

十二月甲辰,撥浙江、江西倉穀二十萬石濟福建民食。丙午,盧廕溥予假,命王鼎管刑部。己巳,以孝順岱為科布多參贊大臣。是月,貸直隸災區各營兵餉,山西豐鎮等六州縣災民倉穀。

是歲,朝鮮、南掌、琉球、暹羅入貢。

十三年春正月丁丑,臺灣嘉義匪首陳辦伏誅。己卯,升寅等查覆西安將軍徐錕贓款屬實,褫職。丁酉,以麟慶為湖北巡撫。桃南決口合龍。

二月甲辰,上御經筵。己未,四川越巂等處夷匪作亂,命那彥寶、桂涵剿之。庚申,賑被災多倫諾爾租種蒙古地貧民,並諭此後口外偏災不得援請。壬戌,以汪守和兼署吏部尚書。是月,賑直隸薊州等七州縣災民。貸陜西漢中等五府州貧民倉穀。

三月丙子,大學士盧廕溥致仕。辛巳,上閱火器營兵。丙申,盧坤奏獲越南盜陳加海等,洋面肅清。戊戌,以麟慶為江南河道總督。以鄂順安為湖北巡撫。庚子,雨。是月,貸直隸紫荊關營兵,奉天錦州府屬兵丁,湖南乾州等五縣屯丁、苗佃倉穀。

夏四月壬寅,調鄂順安為山西巡撫,尹濟源為湖北巡撫。樂善遷福州將軍。調慶山為烏里雅蘇臺將軍。丁未,雨。戊申,瑚松額遷成都將軍。調寶興為盛京將軍,保昌為吉林將軍。以蘇成額為熱河都統,貴慶為漕運總督。己酉,命潘世恩為體仁閣大學士,管戶部。調硃士彥為吏部尚書。以白鎔為工部尚書,湯金釗為左都御史。乙卯,免道光十一年十二年喀什噶爾、葉爾羌額貢。己巳,皇后佟佳氏崩。是月,貸盛京義州兵米、湖南新田縣民瑤籽種。

五月辛未朔,賜汪鳴相等二百二十人進士及第出身有差。丁丑,楊芳剿越巂夷匪,大敗之,進剿瓘邊夷匪。己丑,瓘邊匪首桑樹格等伏誅。丁酉,禧恩免御前大臣、戶部尚書,改為理籓院尚書。命大學士長齡管戶部,潘世恩管工部。調穆彰阿為戶部尚書,博啟圖為工部尚書。己亥,四川瓘邊夷匪平。

六月庚子朔,日食。是月,貸直隸博野等三縣雹災籽種。

七月甲申,御試翰林、詹事官,擢田嵩年三員為一等,餘升黜有差。壬辰,冊謚大行皇后為孝慎皇后。調祁廣東巡撫,以惠吉為廣西巡撫。是月,賑貴州古州等四縣水災。

八月,是月,賑貴州都江等二水災。

九月庚午,移孝慎皇后梓宮於田村,上臨送。乙亥,晉楊芳一等侯。壬辰,以貴慶為熱河都統。調嵩溥為漕運總督。調史譜為貴州巡撫。以楊名颺為陜西巡撫。甲午,免雲南昆明等十州縣地震災本年額賦,並賑之。是月,賑江蘇上元等六縣水災。

十月戊午,調布彥泰為伊犁參贊大臣,常德為塔爾巴哈臺參贊大臣。己未,以湯金釗為工部尚書,史致儼為左都御史。是月,賑江蘇上元等十二縣衛,湖南安鄉、華容二縣,直隸曲陽縣,黑龍江三處災民。賑湖北武昌等六縣水災。給安徽懷遠等六縣災民口糧。

十一月丙戌,上詣大高殿祈雪。以裕泰為貴州巡撫。丁亥,以武忠額為熱河都統。以凱音布署察哈爾都統。

十二月丁巳,減免直隸河間等五縣水淹地賦。是月,賑江蘇上元等十二縣衛水災。

是歲,朝鮮、越南、琉球、緬甸入貢。

十四年春正月丁卯朔,辛未,文孚免正黃旗領侍衛內大臣,以載銓代之。丁丑,緬甸貢使聶紐耶公那牙卒於京師。庚辰,廣東儋州黎匪作亂,飭盧坤剿之。甲申,以浙江杭州等府災,免外商及浙民運米關稅。福建永安等縣土匪手虜人勒贖,捕治之。允浙江杭州、湖州兩府所屬漕糧紅白兼收,秈稉並納。丁亥,命潘世恩在軍機大臣上行走。戊子,以三載考績予長齡等議敘。命松筠以都統銜休致。祁奏越南諒山解圍,七泉夷州知州阮文泉等求入關,拒之。是月,賑直隸曲陽縣貧民。給江蘇上元等八縣、浙江海寧等四州縣、江西南昌等二十二縣上年災歉口糧籽種。貸山西朔州等十州縣、陜西葭州等十四州縣、江西南昌等六縣、湖北武昌等十八州縣衛、湖南澧州等四州縣、甘肅皋蘭等九州縣上年被災倉穀口糧籽種。

二月丙申朔,硃士彥給假省親,以湯金釗署吏部尚書。改巴爾楚克換防總兵為副將。己亥,上御經筵。癸卯,升寅等查辦山東、河南事件,以敬徵署左都御史。乙巳,釋李鴻賓、劉榮慶回籍。丙午,以江蘇糧價增昂,免四川、湖廣商米各關船稅。戊申,以廣東學政李泰交自縊,命盧坤徹查。己酉,定山東運河查泉章程。庚戌,以謁西陵,命奕紹等留京辦事。壬子,命凱音布查辦烏里雅蘇臺事件。以蘇勒通阿署察哈爾都統。辛酉,硃士彥憂免,調湯金釗為吏部尚書,以汪守和為工部尚書,史致儼為禮部尚書,何凌漢為左都御史。乙丑,大學士富俊卒。是月,給江蘇上元等八縣衛上年被災口糧。貸貴州古州上年被災籽種。

三月庚午,明山病免,以成格為刑部尚書,那清安兼署。以長清為烏魯木齊都統,興德為葉爾羌參贊大臣。癸酉,上謁西陵,詣田村孝慎皇后梓宮前奠酒,免經過額賦十分之三。丁丑,上謁泰陵、泰東陵、昌陵。庚辰,上還京師。壬午,上臨故大學士富俊第賜奠。乙酉,以喀爾喀游牧被災,準凱音布請,緩勘地界。免四川夷匪滋擾之清溪等三縣,並寧越、越巂兩營上年額賦。

夏四月丁酉,以給事中黃爵滋奏,命各省督撫興復書院,選擇山長,查保甲,修水利,籌積貯,嚴禁扣餉派兵積弊,查究偷漏洋稅,並禁紋銀出洋及私鑄洋銀。戊戌,除直隸樂亭縣水沖官地租賦。丁未,儀郡王綿志卒。甲寅,臨故儀順郡王綿志第賜奠。以其子奕絪襲貝勒。丁巳,命侍郎趙盛奎、在籍前河督嚴烺會同富呢揚阿查勘浙江塘工。辛酉,以蘇清阿為伊犁參贊大臣。甲子,上詣田村孝慎皇后梓宮前行周年祭禮。是月,貸山西岳陽等十二州縣歉收民屯倉穀。

五月己巳,以恩銘署漕運總督。壬申,授凱音布察哈爾都統。癸酉,免雲南昆明等十州縣上年地震災賦。辛巳,上至田村孝慎皇后梓宮前奠酒。丙戌,命盧坤等驅逐英吉利販鴉片躉船,勿任停泊。庚寅,修山東闕里至聖孔子林、廟。甲午,申諭多爾濟喇布坦等與俄羅斯交涉事件務遵舊章。是月,貸淮安、大河二衛歉收屯田籽種。

六月戊申,以福建省城水災,準運古田、福清二縣倉穀及廈門商販米平糶。癸丑,以鄂爾多斯達拉特旗私租蒙地民人拒捕傷臺吉,命鄂順安捕治之。壬戌,實授恩銘漕運總督。是月,蠲緩葉爾羌等三城回戶逋糧。

秋七月乙丑,飭查漕運虧短積弊,並申禁京城私販接濟回漕。丁卯,博啟圖給假,以奕顥署工部尚書。戊辰,霍罕伯克以準通商免稅,遣使表貢,並請年班入覲,允之。庚午,命蘇清阿查勘巴爾楚克、喀什噶爾墾田。免福建臺匪滋擾之四縣,暨淡水抄叛各產租穀。壬申,命特依順保等妥議沿邊會哨章程。程祖洛奏獲洋盜劉四等誅之。甲戌,四川瓘邊支夷作亂,命瑚松額、楊芳等查辦。賑江西水災。丙子,工部尚書博啟圖卒,調耆英為工部尚書,升寅為禮部尚書,敬徵為左都御史。壬午,以桂良為河南巡撫。戊子,東河硃家灣決口。是月,賑江西南昌等十三縣水災。給湖南武陵等七縣衛被水軍民口糧並修屋費。

八月己酉,改建浙江北海塘為石塘。癸丑,以武忠額為烏里雅蘇臺將軍,倫布多爾濟署。以嵩溥為熱河都統。庚申,四川瓘邊支夷平。盧坤奏英商律勞卑來粵,致書稱大英國,請暫停貿易。諭是之。辛酉,上詣孝慎皇后梓宮前奠茶酒。是月,賑盛京蓋州等三處,浙江建德、淳安二縣,江西南昌等二十五縣水災。貸甘肅皋蘭等六縣旱災倉穀。蠲緩江西南昌等二十五州縣新舊額賦。

九月乙丑,英吉利兵船入廣東內河,褫盧坤職留任。庚午,上閱健銳營兵。癸酉,英吉利兵船出口,復盧坤太子少保,仍革職留任。是月,賑直隸宛平等七州縣水災,奉天新民等四州縣水災。貸廣東廣州、肇慶二府水災籽種,打牲烏拉被水旗民倉穀。蠲緩直隸大城等五十一州縣、山西太原縣水災新舊額賦。

十月己酉,立皇貴妃鈕祜祿氏為皇后,頒詔加恩有差。壬子,上皇太后徽號曰恭慈安豫康成莊惠皇太后,頒詔覃恩有差。辛酉,那清安病免,以敬徵為兵部尚書,奕顥為左都御史。是月,賑湖北黃梅等三縣衛、湖南安鄉等四縣衛水災。貸甘肅皋蘭等八州縣旱災雹災口糧。

十一月乙丑,調汪守和為禮部尚書,史致儼為工部尚書。壬申,禮部尚書升寅卒,以奕顥為禮部尚書,恩銘為左都御史。調硃為弼為漕運總督。丙子,以棍楚克策楞為塔爾巴哈臺參贊大臣。己卯,刑部尚書戴敦元卒,調史致儼代之。以王引之為工部尚書。庚辰,以烏爾恭額為浙江巡撫。丙戌,以文孚為大學士管吏部。調穆彰阿為吏部尚書、協辦大學士,耆英為戶部尚書,敬徵為工部尚書,奕顥為兵部尚書。以載銓為禮部尚書。工部尚書王引之卒。丁亥,以何凌漢為工部尚書,吳椿為左都御史。是月,賑浙江麗水縣水災。蠲緩浙江建德等十六州縣衛水災新舊額賦。

十二月癸巳,霍罕復侵色埒庫勒,命興德等諭之。命文孚為東閣大學士。丙申,四川瓘邊夷匪復叛,降楊芳二等侯,褫御前侍衛,以總兵候補。甲辰,黑龍江將軍富僧德調西安將軍,以奕經代之。癸丑,上詣大高殿祈雪。是月,貸直隸災區各營兵餉,江寧八旗官兵銀米,廣東南海等九縣籽種並圍基修費。

是歲,朝鮮、琉球、緬甸、暹羅入貢。

十五年春正月甲子,大學士曹振鏞卒。壬午,長齡以受霍罕餽送,罷御前大臣管戶部事。丙戌,陜甘總督楊遇春致仕,仍溫諭來京。以瑚松額為陜甘總督。調寶興成都將軍。以奕經為盛京將軍,保昌為黑龍江將軍,蘇清阿為吉林將軍。是月,賑奉天牛莊等三處被災旗戶。給江西南昌等九縣,甘肅靖遠等六州縣口糧。貸山西太原等三州縣,江西南昌等二十六州縣,湖南安鄉等四州縣,甘肅秦州、靖遠縣被災倉穀籽種。

二月丙申,以阮元為大學士管刑部,王鼎協辦大學士,伊里布為雲貴總督,何煊為雲南巡撫。庚子,以奇明保署黑龍江將軍。丁未,命長齡管理籓院,文孚管戶部,潘世恩管工部,阮元改管兵部,王鼎管刑部。以朝鮮世孫李奐襲封朝鮮國王。戊午,吉林將軍蘇清阿卒,調保昌代之。以祥康為黑龍江將軍。

三月,山西趙城縣匪曹順作亂,知事楊延亮死之,遂圍霍州。命鄂順安剿辦。乙亥,上親耕耤田。幸南苑行圍。庚辰,上還京師。是月,給甘肅皋蘭等五州縣災歉口糧。

夏四月,四川瓘邊支夷平,晉鄂山太子太保,賞雙眼花翎。甲寅,賜劉繹等二百七十六人進士及第出身有差。丁巳,上詣黑龍潭神祠祈雨。

五月丁卯,致仕陜甘總督楊遇春晉封一等侯,予食全俸。辛未,趙城縣匪首曹順等伏誅。丁丑,上復詣黑龍潭神祠祈雨。以慄毓美為河東河道總督。庚辰,雨。是月,貸山西鳳臺、沁水二縣被旱倉穀。

六月丙午,減江蘇丹徒被水蘆田額賦。

閏六月丁卯,敬徵降調,以載銓為工部尚書,恩銘為禮部尚書,武忠額為左都御史。調保昌為烏里雅蘇臺將軍,祥康為吉林將軍,哈豐阿為黑龍江將軍。己巳,停本年秋決。

秋七月甲辰,文孚免軍機大臣,仍命以大學士管吏部。改潘世恩管戶部,穆彰阿管工部。命刑部右侍郎趙盛奎、工部右侍郎賽尚阿在軍機大臣上學習行走。是月,給陜西沔縣、洛川縣被水,湖南華容等三縣衛被旱口糧。

八月甲子,以皇太后六旬萬壽,普免各省逋賦。庚辰,諭:「科道中馮贊勛、金應麟、黃爵滋、曾望顏擢任京卿,所以廣開忠諫,務當不避嫌怨,於民生國計用人行政闕失,仍隨時據實直陳,以資採納。」兩廣總督盧坤卒,以鄧廷楨為兩廣總督,祁署,色卜星額為安徽巡撫。甲申,上謁西陵。是日,移孝慎皇后梓宮由田村啟行,免經過地方額賦十之五。是月,給陜西府谷縣雹災口糧。

九月己丑,孝穆皇后、孝慎皇后梓宮至龍泉峪,上臨奠。庚寅,上回鑾。戊戌,授麟慶江南河道總督。丙午,硃為弼病免,以恩特亨額為漕運總督。是月,給兩淮板浦、中正二場被水灶丁口糧。緩徵陜西榆林縣、葭州雹災,江西金谿等九縣旱災額賦。

冬十月戊午,以毓書為科布多參贊大臣。甲子,以皇太后六旬聖壽,上徽號曰恭慈康豫安成莊惠壽禧皇太后。乙丑,皇太后六旬聖壽節,上率王、公、大臣詣壽康宮行慶賀禮。上御太和殿,群臣進表行慶賀禮。詔天下覃恩有差。以富呢揚阿為烏魯木齊都統。癸未,御史湯鵬以劾載銓忤旨罷。予告大學士托津卒。是月,給山西陽曲等五州縣、湖南岳州衛、浙江海寧等十三州縣被災口糧。貸奉天金州水師營兵穀。蠲緩湖南華容等十四州縣衛、浙江海寧等三十一州縣衛被災新舊額賦雜款。

十一月戊戌,臨大學士托津第賜奠。是月,給吉林等三處歉區口糧。

十二月己未,上再詣大高殿祈雪。乙丑,孝穆皇后、孝慎皇后梓宮奉安地宮。乙亥,以樂善為吉林將軍。是月,貸江蘇撫標及城守、劉河二營災區兵餉。蠲緩貴州松桃被水額賦。

是歲,朝鮮、琉球入貢。

十六年春正月乙未,以車倫多爾濟為庫倫蒙古辦事大臣。壬寅,撥山東司庫銀五萬兩賑登、萊、青三府饑。乙巳,調裕泰為湖南巡撫。以賀長齡為貴州巡撫。是月,賑浙江義烏等三縣水旱災。給奉天廣寧等處水災旗民口糧。貸甘肅金州等十四州縣、江西蓮花等五十一縣、陜西葭州等九州縣、湖南澧州等四州縣、山西保德等十五州縣水旱雹災口糧籽種倉穀。

二月丙辰,調周之琦為湖北巡撫。以陳鑾為江西巡撫。己未,以謁東陵,命肅親王等留京辦事。己巳,上閱火器營兵。癸酉,上謁東陵,免經過地方額賦十分之三。丙子,上謁昭西陵、孝陵、孝東陵、景陵、裕陵。湖南武岡州匪藍正樽等作亂,命吳榮光會同訥爾經額剿之。戊寅,免四川瓘邊逋賦。己卯,上還京師。

夏四月癸亥,以梁章鉅為廣西巡撫。丁丑,賜林鴻年等一百七十二人進士及第出身有差。是月,貸甘肅秦州等八州縣被災口糧。

五月丙申,上詣黑龍潭祈雨。戊戌,禮部尚書汪守和卒,以吳椿為禮部尚書,李宗昉為左都御史。丁未,上詣靜明園龍王廟祈雨。是月,貸直隸寶坻縣歉收口糧。

秋七月癸未,以鍾祥為閩浙總督,經額布為山東巡撫。乙酉,以哈豐阿舉發都統高喀鼐干預公事書信,加太子太保。己丑,高喀鼐褫職,遣戍熱河。丙申,大學士文孚致仕。庚子,命穆彰阿為大學士管工部,琦善協辦大學士,仍留直隸總督任。調耆英為吏部尚書,奕顥為戶部尚書,禧恩為兵部尚書,武忠額為理籓院尚書,凱音布為左都御史,樂善為察哈爾都統。壬寅,恩銘免尚書、都統,趙盛奎免軍機大臣及侍郎。以貴慶為禮部尚書。

九月壬辰,以富呢揚阿為陜西巡撫,廉敬為烏魯木齊都統。庚子,上閱健銳營兵。戊申,圓明園三殿災。己酉,以耆英受太監屬託,褫尚書、都統、內務府大臣。以奕經為吏部尚書,寶興為盛京將軍。左都御史凱音布遷成都將軍,以敬徵代之。是月,賑盛京白旗堡等處、山西朔州等十一州縣、貴州松桃災民。展賑陜西神木縣災民。蠲免山西朔州等十一州縣、陜西榆林府屬被災新舊額賦。

冬十月丙辰,加長清太子太保。貸甘肅涇州等八州縣、山西山陰縣災歉口糧倉穀籽種。蠲緩直隸景州等十二州縣水旱災新舊額賦。

十一月壬午,以敬徵為工部尚書,調武忠額為左都御史,以奕紀為理籓院尚書。癸卯,上詣大高殿祈雪。是月,給陜西府穀等四縣霜雹災口糧。蠲緩直隸安州等三州縣水災額賦。

十二月丁巳,上再詣大高殿祈雪。癸亥,雪。

是歲,朝鮮、暹羅來貢。

十七年春正月己卯朔,命奕紀為御前大臣。賞長齡四開褉袍。加潘世恩太子太保。壬辰,兵部尚書王宗誠卒,以硃士彥代之。丁酉,山東濰縣教匪馬剛等作亂,捕獲之。庚子,降訥爾經額湖南巡撫,以林則徐為湖廣總督,調陳鑾為江蘇巡撫,裕泰為江西巡撫。是月,貸山西朔州等十一州縣、陜西葭州等九州縣、甘肅金州等十三州縣水災旱災蝗災雹災霜災倉穀口糧籽種。

二月乙卯,福建嘉義縣教匪沈知等作亂,捕誅之。是月,貸山西吉州等七州縣倉穀。

三月戊寅朔,以詣棽髻山,命惇親王綿愷等留京辦事。庚寅,上奉皇太后幸棽髻山,免經過地方本年額賦十分之三。甲午,上奉皇太后還圓明園。以耆英為熱河都統。乙未,上詣明陵。丙申,上詣明長陵、獻陵、泰陵、景陵、永陵奠酒。以明裔延恩侯書桂為散秩大臣。丁酉,上還圓明園。

夏四月庚申,命彥德鞫治茂明安署札薩克貝勒丹丕勒等訐控盟長之獄。甲子,以顏伯燾為雲南巡撫。是月,貸山東濮州等二十四州縣衛、山西寧武縣倉穀。

五月戊寅,貴慶病免,調奕紀為禮部尚書,以武忠額為理籓院尚書,奎照為左都御史。以周天爵署漕運總督。

六月庚戌,以御史硃成烈奏廣東海口每歲出銀三千餘萬,福建、浙江、江蘇各海口出銀不下千萬,天津海口出銀亦二千餘萬,命沿海各督撫及各監督嚴飭稽查。戊午,命左都御史奎照、戶部侍郎文慶在軍機大臣上學習行走。己未,命琦善署直隸總督。壬申,四川馬邊夷匪作亂,命鄂山剿之。甲戌,奕山等奏獲霍罕賊目阿達那等誅之。是月,貸江蘇淮安、大河二衛被災籽種。

秋七月丙子朔,命侍郎倭什納等冊封朝鮮王妃。壬午,樂善遷荊州將軍,以賽尚阿為察哈爾都統。辛卯,諭慄毓美,東河磚工改辦碎石。丁巳,西寧辦事大臣德楞額遷荊州將軍,以蘇勒芳阿代之。甲戌,廓爾喀年貢逾例,溫諭卻之。

九月庚寅,授周天爵漕運總督。癸巳,召訥爾經額來京。甲午,以錢寶琛為湖南巡撫。甲辰,免直隸邢臺、阜城二縣被旱額賦十分之五。

冬十月丙午,上臨大學士長齡第視疾。辛未,停吉林珠貢。是月,給陜西保安縣被災籽種口糧,並貸綏德等四州縣倉穀。蠲緩山西應州等十州縣、齊齊哈爾等三城被災新舊額賦。

十一月辛卯,晉封長齡一等威勇公。是月,貸甘肅金州等九州縣貧民、江西南昌等十三縣、陜西葭州等五州縣被災籽種口糧倉穀。

十二月丁未,涼山夷匪平。己巳,李宗昉憂免,以卓秉恬為左都御史。庚午,彥德以年老留京,以棍楚克策楞為綏遠城將軍。是月,貸陜西定邊、安定二縣來春口糧籽種。

是歲,朝鮮、琉球、暹羅、越南來貢。

十八年春正月甲戌朔,命奎照、文慶為軍機大臣。乙亥,太傅、大學士、一等公長齡卒。丙子,上臨長齡第賜奠。乙酉,四川夷匪平。是月,貸甘肅固原等十四州縣、山西平定等五州縣災民口糧籽種倉穀。

二月癸卯朔,命琦善為大學士,仍署直隸總督。以雲貴總督伊里布協辦大學士,仍留任。乙巳,史致儼病免,以祁為刑部尚書,怡良為廣東巡撫。壬戌,修喀喇沙爾城。戊辰,修浙江海塘。是月,貸陜西懷遠、府穀二縣歉收籽種。

三月乙亥,以謁陵命肅親王等留京辦事。戊子,上奉皇太后謁西陵,免經過地方額賦十分之三。壬辰,上謁泰陵、泰東陵、昌陵,詣孝穆皇后、孝慎皇后陵寢奠酒。乙未,上奉皇太后還京師。丙申,上幸南苑行圍,至戊戌皆如之。庚子,上還京師。辛丑,噶勒丹錫埒圖薩瑪第巴克什入貢。是月,貸山西遼州等十三州縣上年歉收倉穀。

夏四月庚申,以富呢揚阿等建鳥魯木齊書院,議處有差。申命新疆將軍、都統、大臣認真教練,使人人習於戰陣、毋舍實政務虛名。甲子,以伍長華為湖北巡撫。丙寅,賜鈕福保等一百九十四人進士及第出身有差。辛未,以奕出為伊犁將軍,湍多布為伊犁參贊大臣。

閏四月丙子,上詣黑龍潭祈雨。辛巳,雨。鴻臚寺卿黃爵滋奏請將內地吸食鴉片者俱罪死。命盛京、吉林、黑龍江將軍,直省督撫各抒所見議奏。己丑,褫禧恩太子太保銜兵部尚書,調成格為兵部尚書,以鄂山為刑部尚書,寶興為四川總督,耆英為盛京將軍,惠吉為熱河都統。庚寅,調奕紀為戶部尚書,成格為禮部尚書,奕顥為兵部尚書。

五月丙午,上詣黑龍潭祈雨。己酉,雨。癸丑,大學士阮元致仕。命王鼎為大學士,仍管刑部,湯金釗為戶部尚書、協辦大學士,硃士彥為吏部尚書,卓秉恬為兵部尚書,姚元之為左都御史。戊辰,惇親王綿愷免內廷行走、宗令,罰親王俸三年。

六月辛未,免四川馬邊、雷波二逋賦及各廠應解銅鉛。丁丑,降惇親王綿愷為郡王。己卯,命湍多布為塔爾巴哈臺參贊大臣,關福為伊犁參贊大臣。是月,給貴州鎮遠府屬被水兵民口糧。

秋七月戊申,刑部尚書鄂山卒,以寶興為刑部尚書,蘇廷玉署四川總督。

八月丙戌,以林則徐等奏查獲煙販收繳煙具情形,諭嘉之。己丑,成格免,以奎照為禮部尚書,恩銘為左都御史。命奕紀管理籓院。以賽尚阿署理籓院尚書,布彥泰為察哈爾都統。是月,給陜西安定、府穀二縣災民口糧。

九月丙午,莊親王奕鎛等坐食鴉片革爵。丁未,上閱健銳營兵。己酉,太常寺少卿許乃濟請弛鴉片禁。命休致。召林則徐來京,以伍長華署湖廣總督。辛酉,調錢寶琛為江西巡撫,裕泰為湖南巡撫。吏部尚書硃士彥卒,調湯金釗為吏部尚書,吳椿為戶部尚書,以龔守正署禮部尚書。是月,給山東濰縣災民口糧。

冬十月庚寅,以盛貴為烏里雅蘇臺參贊大臣。是月,蠲緩直隸深州等十三州縣、江西南昌等二十二縣、安徽壽州等三十四州縣衛、河南內黃等十一縣、湖南澧州等八州縣衛、奉天寧遠州被災新舊額賦。

十一月壬寅,命伊里布等查禁雲南種罌粟。壬子,以寶興為四川總督,恩銘為刑部尚書,裕誠為左都御史。癸丑,命林則徐為欽差大臣,查辦廣東海口事件,節制該省水師。以周天爵署湖廣總督,鐵麟署漕運總督。丁巳,上詣大高殿祈雪。以固慶為科布多參贊大臣。乙丑,兵部尚書奕顥褫職,調裕誠為兵部尚書,以隆文為左都御史。丙寅,召哈豐阿來京,以舒倫保署黑龍江將軍。是月,賑陜西懷遠、安定二縣,寧古塔三姓地方兵民口糧。

十二月戊辰朔,貴州仁懷縣匪謝法真等作亂,命伊里布剿之。辛未,惇郡王綿愷卒,追復親王爵。上親臨其喪三次賜奠。乙亥,上再詣大高殿祈雪。丙戌,上復詣大高殿祈雪。庚寅,移庫倫幫辦大臣駐科布多,為科布多幫辦大臣。辛卯,授賽尚阿理籓院尚書。乙未,左都御史姚元之免,以龔守正代之。以匪亂平,賞伊里布雙眼花翎,晉餘步雲太子太保。

是歲,朝鮮、琉球、暹羅來貢。

十九年春正月戊戌朔,晉封惠郡王釂愉為親王。戊午,召奕山來京,以關福署伊犁將軍。是月,貸湖南武陵縣、陜西葭州等九州縣、甘肅固原等五州縣水旱災雹災口糧籽種。

二月壬午,御試翰林、詹事等官,擢李國杞四員為一等,餘升黜有差。丙戌,以謁東陵,命肅親王敬敏等留京辦事。命林則徐赴虎門、澳門,防外海洋船進口及內匪出洋。

三月庚子,上謁東陵,免經過地方額賦十分之三。辛丑,吳椿病免,調何凌漢為戶部尚書,以陳官俊為工部尚書,龔守正為禮部尚書,廖鴻荃為左都御史。癸卯,上謁昭西陵、孝陵、孝東陵、景陵、裕陵,詣端慧太子園寢奠酒。乙巳,陶澍病免,調林則徐為兩江總督,以陳鑾署之,裕謙署江蘇巡撫,以桂良為湖廣總督,硃樹為河南巡撫。丙午,上幸南苑行圍。辛亥,上還京師。乙卯,林則徐等奏躉船呈繳煙土,諭嘉之,予獎敘。準林則徐等奏,暫緩議斷互市。烏魯木齊都統廉敬遷成都將軍,以惠吉代之,以恩銘為熱河都統,隆文為刑部尚書。丙辰,以鐵麟為左都御史。

夏四月辛未,以吳文鎔為福建巡撫。丁丑,調周天爵為河南巡撫,硃樹為漕運總督。戊子,上詣萬壽山殿祈雨。丁酉,以直隸旱,免奉天、山東、河南來直米稅。

五月辛丑,雨。予告大學士盧廕溥卒。是月,賑雲南浪穹、鄧川二州縣地震災額賦。

六月丙寅,閩浙總督鍾祥以關防被竊褫職,以周天爵代之,以牛鑒為河南巡撫。丁亥,太子少保前兩江總督陶澍卒。辛卯,調周天爵為湖廣總督。

秋七月壬子,命林則徐以禁販鴉片檄諭英吉利國及各國在粵洋商。是月,給湖南華容縣水災口糧。

八月庚午,經額布遷成都將軍,以托渾布為山東巡撫。召烏里雅蘇臺將軍保昌來京,以廉敬代之。是月,給陜西葭州等三州縣被災口糧。

九月庚子,命托渾布查辦山東登州海賊,整頓水師。辛丑,上閱健銳營兵。己酉,哈豐阿遷廣州將軍,調棍楚克策楞為黑龍江將軍,德克金布為綏遠城將軍。

冬十月,山西巡撫申啟賢卒,賜恤如尚書例。以楊國楨為山西巡撫。是月,賑安徽無為等十一州縣及屯坐各衛水災。賑湖北黃梅等三縣災民。給湖北沔陽等九州縣,山東蒙陰縣,陜西府谷、神木二縣,湖南華容縣九州衛水災旱災口糧。蠲緩安徽無為等三十二州縣、湖北沔陽等二十六州縣、河南睢州等二十一州縣、湖南澧州等九州縣衛水災新舊額賦。

十一月庚子,英船入廣東海港,林則徐督官軍擊走之,停其貿易。以程懋來為安徽巡撫。戊申,德克金布遷廣州將軍,以松溥為綏遠城將軍,舒倫保為黑龍江將軍。庚戌,命濟克默特赴庫倫,迎哲布尊丹巴呼圖克圖來覲。是月,給江西德化等七縣、山西應州等四州縣災民口糧。蠲免江西南昌等二十三縣、山西應州等八州縣、直隸安州等五州縣被災新舊額賦。

十二月癸亥,署兩江總督陳鑾卒,調鄧廷楨為兩江總督,林則徐為兩廣總督,裕謙為江蘇巡撫。癸酉,哲布尊丹巴呼圖克圖等覲見。調伊里布為兩江總督,鄧廷楨為雲貴總督。癸未,命刑部尚書隆文在軍機大臣上行走。調鄧廷楨為閩浙總督,桂良為雲貴總督。戊子,陳官俊免,以廖鴻荃為工部尚書。軍機大臣文慶免。

是歲,朝鮮、琉球入貢。

二十年春正月壬辰朔,加王鼎太子太保。戊戌,以阿勒精阿為熱河都統。己亥,理籓院禁哲布尊丹巴呼圖克圖用旗傘,並未奏明,奕紀褫御前大臣、戶部尚書、總管內務府大臣並紫韁,免管理籓院。賽尚阿降二品頂戴。調隆文為戶部尚書。壬寅,皇后鈕祜祿氏崩。戊申,謚大行皇后為孝全皇后。庚戌,奕紀逮問。庚申,以奕紀收沙布朗餽送銀,遣戍黑龍江,賽尚阿等均下部嚴議。

二月癸亥,以阿勒精阿為刑部尚書,訥爾經額為熱河都統,哈豐阿為西寧辦事大臣。丁卯,戶部尚書何凌漢卒,以卓秉恬代之。以祁俊空話藻為兵部尚書,沈岐為左都御史。丁丑,河東河道總督慄毓美卒,以文沖為河東河道總督。是月,給安徽桐城縣貧民口糧。貸山西河保等營兵丁穀石。

三月,命何汝霖在軍機大臣上學習行走。召奕山來京,以布彥泰為伊犁將軍。辛亥,以壁昌為察哈爾都統。是月,貸山西吉州等九州縣倉穀。

夏四月辛酉朔,冊謚孝全皇后,翼日頒詔。己巳,調經額布為吉林將軍。丙子,以祥康為庫倫辦事大臣。戊寅,上詣黑龍潭祈雨。乙酉,賜李承霖等一百八十人進士及第出身有差。戊子,上詣廣潤祠祈雨。是月,貸直隸紫金關及所屬浮圖峪等三營弁兵倉穀。

六月丁卯,以色克津阿為綏遠城將軍。丁丑,林則徐等奏擊毀載煙洋艇。庚辰,英船入浙洋,圍定海縣城。命餘步雲會烏爾恭額等援之。甲申,英人陷定海縣,知縣姚懷祥等死之。褫烏爾恭額及浙江提督祝廷彪職,仍留任。調瑚松額為熱河都統,訥爾經額為陜甘總督。

秋七月癸巳,英船犯浙江乍浦海口。命奇明保率兵御之。英師犯福建廈門包臺,參將陳勝元等擊卻之。丙申,褫浙江巡撫烏爾恭額職,以劉韻珂代之。丁酉,命伊里布為欽差大臣,赴浙江剿辦。以裕謙兼署兩江總督。以湍多布為伊犁參贊大臣,花山太為塔爾巴哈臺參贊大臣。甲辰,英船泊天津口外,遞信與琦善訴屈。命琦善接收,仍飭勿進口。丙午,花山太遷喀什噶爾辦事領隊大臣,調湍多布為塔爾巴哈臺參贊大臣,以福興阿為伊犁參贊大臣。庚戌,林則徐等奏續獲販煙人犯。諭以空言搪塞,切責之。乙卯,英船至山海關等處。丙辰,命伊里布等,英人如有投遞書信,即接受馳奏。是月,賑湖北沔陽等三州縣水災。

八月甲子,以邵甲名署浙江巡撫。丙子,英人復侵福建廈門,提督陳階平等擊走之。己卯,命琦善為欽差大臣,赴廣東查辦,並諭伊里布及沿海督撫防守耍隘,洋船停泊外洋勿問。調訥爾經額署直隸總督,以瑚松額署陜甘總督。庚辰,廉敬遷成都將軍,以德楞額為烏里雅蘇臺將軍。辛巳,裕謙奏英人呈遞原書,不敢上聞。諭切責之。是月,給江蘇上元等十四縣水災口糧並修屋費。

九月庚寅,林則徐、鄧廷楨命交部嚴加議處。以琦善署兩廣總督。辛卯,以托渾布奏英船南去,命耆英、托渾布酌撤防兵。召鄧廷楨來京,以顏伯燾為閩浙總督,張澧中為雲南巡撫。甲午,諭周天爵等恤湖北各州縣水災。乙未,褫林則徐、鄧廷楨職,命赴廣東候查問。己亥,英船入浙江慈谿、餘姚二縣內洋,伊里布等擊走之。以烏爾恭額不代奏英人書信,逮問。是月,給江蘇泰興縣水災口糧。

十月壬申,以謁陵命莊親王等留京辦事。壬午,以孝全皇后梓宮奉安龍泉峪,上詣觀德殿行祖奠禮。乙酉,調祿普為烏里雅蘇臺將軍。是月,賑直隸滄州等三州縣災民。給安徽東流、含山二縣軍民口糧。貸奉天白旗堡、小黑山水災口糧。蠲緩直隸滄州等三十三州縣、湖北沔陽等八州縣衛水災新舊正雜額賦。

十一月庚寅,上謁西陵,免經過地方額賦十分之五。甲午,上謁泰陵、泰東陵、昌陵,並詣孝穆皇后、孝慎皇后陵寢奠酒,孝全皇后梓宮前行遷奠禮。乙未,孝全皇后梓宮奉安地宮,上臨視,命皇子行禮。己亥,上還京師。癸卯,祿普改荊州將軍,調奕湘為烏里雅蘇臺將軍。英人陷定海。戊申,烏爾恭額論絞。壬子,伊里布奏英人要求澳門、定海貿易。諭琦善令英人退還定海。癸丑,以周天爵擅用非刑,褫職,遣戍伊犁。以裕泰為湖廣總督,以吳其濬為湖南巡撫。是月,賑江蘇上元等十六縣、直隸天津縣水旱災。貸江蘇江寧駐防及督協各營駐扎災區兵餉、黑龍江墨爾根城水災屯丁口糧屋費。蠲緩江蘇泰州等七十二州縣衛、直隸天津縣、山西河曲縣新舊額賦。

十二月,以孝全皇后升祔奉先殿,上親詣告祭。翼日,命皇四子行禮。戊辰,調餘步雲為浙江提督。以鐵麟為察哈爾都統,恩桂為左都御史。以壁昌為伊犁參贊大臣。己卯,調吳文鎔為湖北巡撫,以劉鴻翱為福建巡撫。癸未,召瑚松額來京,以恩特亨額署陜甘總督。是月,給福建龍谿、南靖二縣水災口糧屋費。貸江蘇江陰等三營兵餉。蠲緩浙江長興等四縣水旱災新舊正雜額賦。

是歲,朝鮮入貢。


\end{pinyinscope}