\article{列傳四十二}

\begin{pinyinscope}
張勇趙良棟子弘燦弘燮王進寶子用予王萬祥孫思克馬進良

張勇,字非熊,陜西咸寧人。善騎射,仕明為副將。順治二年,英親王阿濟格師次九江,勇來降,檄令招撫,得總兵以下七百餘人。授游擊,隸陜西總督孟喬芳標下。時李自成將賀珍、賀弘器、李明義等分據漢中、興安、固原諸地,窺西安。勇與副將任珍、馬寧等御戰,屢敗之。四年,寧夏叛將馬德結弘器陷安定,勇從總兵劉芳名率師赴援,戰,馬寧陣擒德,勇攻克固原,獲弘器、明義,誅之。

四年,米喇印、丁國棟以蘭州叛,陷臨洮。勇與副將陳萬略率師夾擊,破賊,復臨洮。逐賊至岷州,敗之宮堡,又敗之馬韓山。賊分竄二崖洞,殲焉;又敗之馬家坪,獲明延長王識駉。喬芳攻拔蘭州,喇印、國棟走甘州。勇等率師與喬芳會,遂渡河而西。八月,至甘州,賊出戰,屢擊敗之。六年正月,總兵南一魁奪門入,勇入城巷戰,賊夜遁,逐之至北山,殲賊甚眾。斬喇印於水泉,國棟走肅州,師從之。五月,至肅州,伏壕外,伺賊出牧,擒斬,不使得入。十二月,勇與馬寧督兵樹雲梯登城,遂復肅州,誅國棟,超授甘肅總兵。十年,敘功,授三等阿達哈哈番。

大學士洪承疇視師湖廣,勇請自效,詔獎其忠勤,召詣京師。承疇亦薦勇智勇兼備,所部兵精馬足,請移授經略右標總兵,上許之。勇入對,賜冠服、甲胄、弓矢,加右都督。勇移家京師,乞賜宅;子云翥,以廕授陜西衛指揮,乞改隸京衛,並得旨俞允。勇將行,命內大臣索尼等傳諭曰:「當今良將如勇者甚少。軍務不可懸度,當相機而行,勿負才輕敵。」至軍,佐承疇屢破敵。十五年,從徇貴州,明將羅大順焚新添衛,勇率兵馳戰,大順走十萬谿,勇與一魁等破其壘。復從信郡王多尼下雲南,次盤江。明兵焚鐵索橋,勇夜督兵造梁,黎明,全軍皆得渡,破明將白文選於七星關。十六年,加左都督。十七年,命移鎮臨元、廣西諸處。十八年,遷雲南提督。

康熙二年,以勇久鎮甘肅,威名素著,屬番讋服,命還鎮甘肅。三年,加太子太保。西喇塔拉饒水草,號大草灘,厄魯特蒙古乞駐牧於此。勇以其地當要隘,不容逼處,自往諭之,事遂寢。因請築城其地,曰永固。旁建八寨,相聯屬為聲勢。四年,蒙古徙牧近邊,請增西寧兵四千五百二十。部議下總督覆覈,上特命允之。

十二年,吳三桂反,四川總兵吳之茂叛應之。十三年,三桂使招勇,勇執其使以聞。陜西提督王輔臣亦叛,勇督兵防禦。十四年,巡撫華善疏言:「輔臣遙應三桂,西番土回乘隙並起,河西危甚,得免淪陷,皆勇之力。請敕許勇便宜。」命授靖逆將軍,仍領提督,總兵以下聽指揮。輔臣招勇,勇斬其使,上嘉之,封靖逆侯。

勇遣西寧總兵王進寶率師攻蘭州。輔臣將潘瑀攻洮州,曾文耀攻河州,番部乘隙肆掠。勇率兵攻河州,文耀敗走。別遣土官楊朝樑攻洮州,自督兵繼其後,瑀亦敗走。上嘉勇謀略,以其次子雲翼為太僕寺卿。勇進攻鞏昌,輔臣將任國治等潛師入城,與城兵共出戰。勇與副將劉宣聖等奮擊,截其歸路,斬馘過半,獲四百七十三人。時輔臣據平涼,貝勒洞鄂督兵圍攻,久不下,上命勇率師會之。勇疏言鞏昌要地,兵力難分,下廷臣議,令勇固守鞏昌。

吳三桂遣其將吳之茂自四川北犯,為輔臣聲援,屯西和。勇與振武將軍佛尼埒及進寶等御之,三戰皆勝。寧夏兵變,戕提督陳福。勇還駐鞏昌,疏薦天津總兵趙良棟才勇,命即授寧夏提督。十五年,敘復洮、河二州功,加少保兼太子太保。

吳之茂屯樂門,分兵攻陷通渭。勇督兵道伏羌赴援,至十八盤坡,與之茂兵遇,張兩翼沖擊,之茂兵潰,乘勝復通渭。進攻樂門,之茂據險,列十一寨,勇度地,令橫營山梁。營甫立,賊齊出,勇令兵持草一束,與都統赫葉分擊南北山梁,賊亦南北應戰。火器發,賊敗走入寨,兵投草填塹直進,殺賊千餘。之茂收餘眾復戰,勇勒兵沖擊,之茂大敗。勇與佛尼勒、進寶等盡平賊寨。之茂夜走,追敗之牡丹園,又敗之西和北山,之茂僅以數騎遁。大學士圖海出視師,輔臣降,勇遣兵收平涼、慶陽、鞏昌諸屬縣。詔褒勇功,進一等侯,加少傅兼太子太師。

十七年,準噶爾臺吉噶爾丹兵入河套,厄魯特部為所敗,假道赴青海,闌入內地,勇驅令出塞。二十一年,入覲。二十二年,以老病乞休,諭留之。二十三年,聞青海蒙古游牧近邊城,率兵赴丹山防禦,至甘州,病篤。上聞,遣醫並其子雲翼馳驛往視。尋率,贈少師仍兼太子太師,賜祭葬,謚襄壯。

勇身經數百戰,克府五、州縣五十,右足中流矢,傷骨,不能履,常以肩輿督戰。臨敵若無事,而智計橫出,每以寡勝眾。居恆恂恂退讓,賓禮賢士。用人盡其材,其所甄拔,往往起卒伍為大將,良棟、進寶尤其著者也。

子雲翼,襲爵,官至江南提督。卒,謚恪定。雍正間,祀勇賢良祠。乾隆三十三年,命以一等侯世襲罔替。四十七年,詔褒勇、良棟、進寶勛績,尤稱勇有古名將風。時勇四世孫承勛襲爵,以散秩大臣曠班,降三等侍衛,命復還散秩大臣。

趙良棟,字擎宇,甘肅寧夏人,先世居榆林。順治二年,師定陜西,良棟應募,隸總督孟喬芳標下,檄署潼關守備。從征秦州、鞏昌,擊敗叛將賀珍、武大定。授寧夏水利屯田都司。五年,討河西回,擒丁國棟。良棟在行間,擢高臺游擊。十三年,以經略洪承疇薦,從征雲、貴,授督標中軍副將。康熙元年,擢雲南廣羅總兵。先後剿平馬乃、隴納、水西諸苗。四年,移鎮貴州平遠,遭父喪,吳三桂以水西未大定,留勿遣。良棟辭,忤三桂,同官為排解,乃得歸終制。八年,起山西大同總兵。十一年,移鎮直隸天津。

十二年,三桂反。十三年,寧夏兵變,戕提督陳福。甘肅提督張勇薦良棟,擢寧夏提督。入覲,奏寧夏亂兵,宜誅首惡、宥脅從,上頷之。良棟請留孥京師,賜宅以居。簡精兵百疾馳赴鎮,宣上諭撫慰。察知倡亂者把總劉德,而參將熊虎與其謀,戕福者營兵閻國賢、陳進忠。乃分兵使出防,散其黨羽,逮虎等正其罪,請旨斬之。

是時大將軍圖海督師平涼,討王輔臣,良棟及平涼提督王進寶並聽指揮,分兵定秦州、西和、禮縣。十八年,良棟疏言:「寧夏兵舊習驕縱,臣三年訓練,漸遵紀律,並嚴禁侵剋額餉,眾志思奮。臣年漸老,不乘時努力,虛負上恩。今湖南既定,宜取漢中、興安,規四川。臣原精選所部步騎五千,獨當一路。」上覽奏嘉許,下圖海。圖海議先破棧道、益門鎮諸處賊壘,分四道進取;而涼州提督孫思克疏請緩師,得旨切責。乃以十月定師期,良棟將所部出徽縣。師進破密樹關,遣兵襲黃渚關分敵勢,大戰,破三桂兵,克徽縣。思克出略陽,方次階州。良棟師自徽縣進克略陽,三桂將吳之茂敗走。良棟復進取陽平關,徇沔縣。進寶出鳳縣定漢中,良棟與會師寧羌,各奏捷。授良棟勇略將軍,仍領寧夏提督。

十九年,良棟與進寶分道進次白水壩,三桂兵夾江而陣,江水方漲,不得舟,賊矢石如雨。良棟令於眾曰:「視我鞭所向,敢退者斬!」一軍皆奮呼。良棟擐甲,驟馬亂流而渡,師從之,敵發砲,傷數十人,無回顧者。三桂兵錯愕奔潰,逐之過青川,敗之石峽溝,再敗之青箐山,下龍安府,渡明月江,經綿竹。三桂兵盡潰,所置巡撫張文德及其將汪文元等皆降,遂復成都,蓋出師甫十日。上獎良棟功,擢雲貴總督,加兵部尚書,仍領將軍。良棟念寧夏當有代者,鎮兵且不能從征,疏辭總督,上弗許。部議寧夏改設總兵,上即授良棟子廕生弘燦,仍將鎮兵從征。

時進寶亦克保寧,與建威將軍吳丹等徇順慶、重慶、遵義,皆下。良棟分遣游擊冶國用等西徇雅州,復象嶺、建昌諸衛。東略敘州,定納溪、永寧諸縣。疏請敕陜西、四川督撫諸臣合籌運餉濟軍。師自四川分道:一自保寧出永寧,達霑益;一自成都出建昌,達武定。並下雲南。上韙其言,諭諸將帥協謀定策。尋議吳丹出永寧,良棟出建昌。吳世璠遣其將胡國柱、夏國相等攻陷永寧,犯瀘州、敘州,復聚窺建昌。良棟檄總兵硃衣客將八千人援建昌,硃衣客戰不勝,退駐雅州。建昌守兵食盡,棄城走。良棟劾吳丹擁兵不進,致永寧陷賊,並及硃衣客引退狀,詔解吳丹將軍以授佛尼埒,逮硃衣客下刑部。

二十年,良棟率師次朝天關,遣弘燦出馬湖繞賊兵後,戰鳳凰村,再戰觀音崖。賊據崖,弘燦督兵攀崖襲其後,馘三百,俘八十餘。令總兵李芳述、偏圖等逐至黃茅岡,賊分三道拒戰,弘燦分兵應之,自旦至暮,大破賊,斬其將沈明、張文祥,國柱等遁走。復瀘州、敘州,遂克永寧,徇榮經。良棟與會師夾江,克雅州,進復建昌。渡金沙江,次武定。

大將軍貝子彰泰統湖廣、廣西諸路滿、漢兵四十萬下雲南,攻會城,屯城東歸化寺,西亙碧雞關,連營四十里,前臨昆明湖,湖中不設兵。世璠收餘眾固守,自水道轉運,相持數月未下。九月,良棟至軍,周視營壘,請於彰泰曰:「我師不速戰,相持日久,糧不繼,何以自存?」彰泰曰:「皇上豢養滿洲兵,豈可輕進委之於敵?且爾兵初來,亦宜體養,何可令其傷損?」良棟不從,率所部夜攻南壩,破壘奪橋,遂薄城。彰泰語良棟:「爾兵攻已瘁,宜暫退,令總督蔡毓榮代守。」良棟曰:「我兵死戰所得地,奈何令他人守乎?」於是彰泰令諸軍悉進,世璠兵出城,戰於桂花寺,諸軍皆奮斗,世璠兵大敗,乃自殺,餘眾以城降。雲南平。

自三桂鎮雲南,至世璠覆亡,歷年久,子女玉帛充積饒富。城破,諸將爭取之,獨良棟無所取,戢所部兵絲毫毋敢犯。

硃衣客就逮,具疏辨,謂良棟與兵少,又無後應,是以退還。進寶亦疏謂建昌之陷,罪在良棟。良棟復劾硃衣客欺飾狡辨,且謂辨疏出進寶。上以軍事急,命俟事平察議。雲南既定,召良棟詣京師,進寶亦入覲,諭曰:「當賊據漢中負固,諸將咸謂恢復為難,獨良棟首發議進剿,與進寶同取漢中。嗣因意見不相合,遂分道克成都,而進寶亦取保寧。成都不下,保寧未易拔;保寧不下,成都未易守:是二將並有功也。時賊皆入川抗戰,我師乘虛自沅州、鎮遠取貴陽,川中寇復張,已復之疆土幾至再陷,則二將不能和衷之所致也。二將不諳大體,私忿攻訐。朕念其功績並茂,惟欲保全,互訐章奏,皆置不問,但論失援建昌罪。」部議硃衣客論斬,吳丹奪官籍沒,良棟奪官。上命硃衣客免死為奴,吳丹奪官,良棟改授鑾儀使。

二十二年,良棟疏陳戰功,請察議,下王大臣等議:良棟失建昌,以功抵罪;止敘從征將士弘燦、芳述、偏圖,並加左都督。良棟尋乞病歸。二十五年,上念良棟克雲南,廉潔守法紀,復將軍、總督原銜。二十七年,入覲,復自陳戰功,上命還里牒部具奏。二十八年,授拜他喇布勒哈番。

三十年,噶爾丹擾邊,命西安將軍尼雅翰等出防寧夏,以軍事諮良棟。三十二年,以寧夏總兵馮德昌赴甘州,命良棟暫領鎮兵。良棟劾德昌剋軍糧,德昌坐罷。三十三年,命良棟率兵駐土喇御噶爾丹,旋召詣京師。三十四年,良棟復自陳戰功為大將軍圖海、彰泰所抑,並咎大學士明珠蔽功,上責其褊隘,還其疏,仍敕部優敘,授一等精奇尼哈番。良棟原留京師,乞田宅。御史龔翔麟劾良棟驕縱,上原之,賚白金二千,令歸里。

三十六年,良棟病,尚書馬齊自寧夏還,奏狀,手詔存問,賜人葠、鹿尾。尋卒,年七十有七。上方征噶爾丹,次榆林,諭曰:「良棟偉男子,著有功績。性躁心窄,每與人不合,奏事朕前,言語粗率。朕保全功臣,始終優容之,所請無不允。今病卒,宜為其妻子區處,使得安生。」至寧夏,命皇長子允禔臨其喪,賜祭葬,謚襄忠。五十九年,上諭群臣,猶舉良棟至雲南與彰泰議軍事,謂決於進戰乃得成切。乾隆四十七年,進一等伯,世襲罔替。

子弘燦,初以廕生特授寧夏總兵,歷川北、真定、黃巖、南贛諸鎮。康熙三十八年,授浙江提督,調廣東。四十五年,授兩廣總督。五十五年,入覲,辭還,奏言久處炎海,年事就衰,請移近地自效。尋授兵部尚書。五十六年,詣京師,至武昌,道卒,謚敏恪。

弘燮,初授完縣知縣,再遷天津道。良棟卒,襲一等精奇尼哈番,復授天津道。三遷河南巡撫,調直隸。五十四年,諭獎弘燮撫直隸十年,任事勤勞,旗、民輯睦,盜案稀少,加總督銜。六十一年,卒,謚肅敏。弘燮在官虧庫帑,特命弘燦子之垣以郎中署直隸巡撫,責完補。世宗即位,以之垣庸劣,令解任。尋命免追虧項,詔謂念良棟舊勛也。

王進寶,字顯吾,甘肅靖遠人。精騎射。順治初,從孟喬芳討定河西回,授守備,隸甘肅總兵張勇標下。十一年,勇調經略右標總兵,南征,進寶從徇湖南。十五年,下貴州,師次十萬谿,懸崖千仞,明將李定國遣其將羅大順扼險屯守。進寶率眾攀崖直上,搗其巢,大順奔潰,以功遷經略右標中營游擊。康熙二年,勇還為甘肅提督,進寶亦改授提標左營游擊,隨軍有功,遷參將。厄魯特蒙古欲得大草灘駐牧,勇用進寶議,持不可。既,城永固,以進寶為副將駐其地。十二年,擢西寧總兵。

王輔臣攻陷蘭州,勇遣進寶率師討之。次黃河,夜以革囊結筏自蔡灣渡,破賊皋蘭龍尾山,獲輔臣將李廷玉。遂東拔安定,復金縣。西攻臨洮,會大雪,言冋賊不誡備,襲破之。輔臣使持吳三桂劄招進寶,進寶以聞,加左都督。四月,進攻蘭州。輔臣遣兵開壁出戰,進寶督兵奮擊,自旦至日中,擒斬過半。賊敗入壁,為長圍困之,斷其糧運。六月,輔臣兵造筏黃河,謀潛遁。進寶緣河要之,賊計蹙,其將趙士升出降。

其秋,三桂遣其將王屏籓、吳之茂自四川入陜西,為輔臣聲援。之茂據西和鳳凰山,進寶督兵討之,初合,我師敗績;夜,之茂兵來襲,進寶以計環攻之,蹙之黨家山,大潰,多墜崖死。十五年,擢陜西提督,仍兼領西寧總兵,駐秦州。之茂進據北山,斷臨洮、鞏昌道。進寶與將軍佛尼埒分兵赴援,擊敗之,獲其將徐大仁。戰羅家堡,再戰鹽關,屢勝。之茂集潰兵萬餘屯鐵葉硤、紅山堡,築壘,護以密椿,潛出運芻糧。進寶遣兵破賊牡丹園,獲糧械。大將軍圖海進攻平涼,輔臣引四川叛將譚弘犯通渭。進寶引數十騎入自東峽口,聞將軍赫葉戰敗,寇方張,令諸軍伐木曳以行,塵大起,寇駭走,追殺數十里。分兵進攻,復靜寧,於是平涼遂下。六月,師次樂門,甫立營,之茂兵來攻,進寶督兵環擊,殲其裨將數輩。復與佛尼埒合兵,戰屢勝,之茂僅以十餘騎潰走。平原、固原悉定。論功,授二等阿思哈尼哈番。上褒進寶忠義,進一等,授奮威將軍,仍兼提督平涼諸軍事。

十七年,復慶陽,斬其將袁本秀。十八年,圖海議取漢中。圖海與總兵費雅達自棧道先驅,進寶疏乞令長子用予隨征,上授以副將。師進次寶雞,進寶遣用予擊賊紅花鋪,大敗之,克鳳、兩當二縣。復進次武關,令用予將偏師繞出關後,進寶督兵夜斬關入,獲其將羅朝興等。復進奪雞頭關,直趨漢中,屏籓率其眾自青石關走廣元,進寶遣兵追擊,其將楊永祚、孫啟耀來降,遂盡復漢中地。時趙良棟亦克略陽,命分道定四川。將軍吳丹、鄂克濟哈率滿洲兵繼進,進寶自青石關進次神宣驛,督兵奪朝天關,疾馳進,拔廣元。屏籓走保寧。

十九年,分兵趨保寧,距城二十里當孔道立營,屏籓以二萬人出戰,進寶督兵奮擊,大破之。追至錦屏山,連拔賊壘,奪浮橋。薄城,守兵貫弓注矢,進寶披襟示之曰:「何不射我?」守兵皆驚愕。用予斬門入,進寶戢諸軍毋驚井里,皆曰:「此仁義將軍也!」屏籓與其將陳君極縊焉,獲之茂與其將張起龍、郭天春等十七人,誅之。分部諸將及次子用賓復昭化,劍州、蒼溪、蓬州、廣安、合州、西充、岳池諸州縣悉定。

時良棟已克成都,授雲貴總督,移軍下雲南。詔進寶留鎮四川,駐保寧。擢用予松潘總兵。進寶疏稱疾乞休,命還固原就醫,即令用予護諸軍駐保寧。尋改用予固原總兵。良棟檄川、陜諸軍從征,進寶疏言所屬諸軍宜留鎮守,請停撥遣,從之。三桂將胡國柱、夏國相等自貴州入四川,譚弘既降復叛,陷建昌。良棟疏劾進寶,進寶言方臥疾,固原、建昌之陷,罪在良棟,詔趣進寶還保寧護諸軍。敘功,進三等精奇尼哈番,用予加左都督,授拖沙喇哈番。二十年,三桂將馬寶犯敘州,用予擊卻之,並復納溪、江安、仁懷、合江諸縣,降其將何德成等,寶竄還雲南。上命用予率所部駐永寧。

二十一年,雲南平,進寶入覲,良棟亦詣京師,命王大臣發還互劾章奏,並宣諭:「二臣功績並茂,欲矜全保護之;私忿攻訐,不諳大體,皆置不問。」語互詳良棟傳。賚服物,還鎮。二十三年,疾甚乞休,時用予已調太原總兵,命偕太醫馳驛視疾。尋移甘肅總兵,俾便奉侍。二十四年,進寶卒,贈太子太保,賜祭葬,謚忠勇。用予襲爵,進二等,尋卒官。乾隆三十三年,命世襲罔替。四十七年,進一等,用賓授侍衛。進寶所部多材武,王萬祥尤著。

萬祥,字瑞宇,會寧人。幼喪父母,依其戚郭氏,從其姓。進寶官游擊,應募入伍,屢當軍鋒。積功至副將。攻蘭州,萬祥請先取臨洮,進寶率兵以夜半至城下。萬祥見城有缺,令裨將閻潤先登,縋萬祥上,數十人從,守者驚覺,發矢石。萬祥語眾曰:「今欲退無路,惟有猛進!」手刃數人,眾繼上,遂克臨洮。

寧夏兵變,軍中流言洶洶,萬祥告進寶。翌日,陽引兵退,而置伏以待。敵來追,伏起,敵大敗。俄,至者益眾,萬祥中矢,手拔,戰益奮,左輔又創,仍力戰,敵乃潰奔,克通渭。進寶憤城人通賊,將悉按誅之,萬祥諫而止。攻漢中,將二千四百人斷敵運道,敵棄寨,屯八角原,復攻之下。土寇起,擊斬其渠。拔鳳縣,分兵取兩當。雪夜進攻武關,擒其將劉哈性。戰閻王碥,用予陷圍中,萬祥馳援,傷右股,還固原療治。進寶為疏請復姓,授定海總兵,調興化。臺灣定,復調臺灣,擢福建陸路提督。卒,贈太子少保,謚壯敏。

孫思克,字藎臣,漢軍正白旗人。父得功,以明游擊降太祖,有功,附金玉和傳。思克其次子也。初授王府護衛。順治八年,管牛錄額真,並授刑部理事官。十一年,遷甲喇額真。從軍,自湖南下貴州、雲南,轉戰有功。康熙二年,擢甘肅總兵,駐涼州。

五年,厄魯特蒙古徙牧大草灘,慰遣之。不受命,戰於定羌廟,敗去,揚言將分道入邊為寇。思克與提督張勇疏請用兵,廷議不可輕啟兵釁,令嚴防邊境,撫恤番人。思克乃偕勇修築邊墻,首扁都口西水關,至嘉峪關止,於是厄魯特蒙古入邊牧者皆徙走。思克遍視南山諸險隘,分兵固禦,乃益敕軍紀,簡將才,汰冗卒,覈餉糈,剔蠹蝕,戢兵安民,疆圉敉寧。總督盧崇峻以聞,加右都督。

十三年,提督王輔臣以平涼叛應吳三桂,臨洮、鞏昌皆附,蘭州亦陷。總督哈占檄思克赴援,思克率師道阿壩紅水蘆塘至索橋,結筏渡河,克靖遠,附近諸城堡悉下。厄魯特墨爾根臺吉乘隙毀隘,入為寇,副將陳達陣沒。思克乃留參將劉選勝等守靖遠,率師還涼州,墨爾根臺吉引去。高臺黃番復入邊為寇,攻圍暖泉、順德諸堡。思克率師赴甘州,黃番亦遠遁,乃復渡河而東,與勇會師。疏言所部兵自草地往來勞苦,乞恩加犒賞,上特許之。

思克會勇圍鞏昌,時大將軍貝勒洞鄂攻秦州未下,三桂遣兵自四川至,營南山上,勢方張。檄思克率二千人自鞏昌赴援,壁州西,與相持。輔臣將陳萬策等詣思克降,巴三綱遁走,遂克秦州。南山寇潰竄,思克與將軍佛尼埒等追擊,敗之閻關,復禮縣;復敗之西和,奪門入,斬所置吏,清水、伏羌諸縣皆下。復還軍鞏昌,遣萬策等入城諭輔臣將陳可等,以鞏昌十七州縣降。河東悉定。

乃會攻平涼,思克率師出靜寧,擊敗輔臣將李國樑,斬級五百,獲裨將三,復其城。進次華亭,輔臣將高鼎率裨將二十八、兵千餘,迎降。遂至平涼,與貝勒洞鄂師會。城兵出戰,思克徒步督所部當賊,戰南山,戰城北,八戰輒勝。又為九覆,敗賊南郭外。賊阻我軍掘壕,思克揮兵急擊,賊退復逼者三,皆敗去。攻涇州白起寨,揮兵先登,克寨,獲輔臣將李茂。又敗之甲子峪,敗之馬營子、麻布嶺,洞鄂上其功。十五年,圖海代洞鄂督師,至城北虎山墩度形勢,並偵通固原道。賊伏兵萬餘猝起,思克急擊之,逐北十餘里,被巨創。輔臣乞降,思克還涼州。詔褒思克功,擢涼州提督,授世職一等阿達哈哈番。思克疏謝,因言:「虎山墩之戰,賊斫臣右臂,傷筋骨。今已成殘疾,乞解任回旗。」溫旨慰留。十六年,敘功,進三等阿思哈尼哈番。噶爾丹為亂,諸蒙古徙入邊擾民,思克與勇遣兵驅之,乃去。

十八年,上敕圖海合諸軍下四川,定四道進兵,思克與將軍畢力克圖出略陽。會京師地震,詔內外大臣陳所見。思克疏言:「漢中、興安山嶺紆險,賊劃斷要隘,師未能直入。綠旗兵不盡強壯,馬又多羸瘦,滿洲兵亦無多。若各路調取,又恐地偪番夷,秋高馬肥,乘機思逞。秦地多山,土不生秔稻,採買麥豆,用民負載馱運,餽運維艱。諸軍聞京師地震,傾壞房屋,壓斃人口,各有內顧憂。不若今秋暫緩出師,選強壯,飼戰馬,俟來春再議進兵。」上命學士拉隆禮至涼州宣諭詰責,思克引罪。與畢力克圖率師攻階州,進克文成、沔諸縣。上命思克還涼州。尋以總督哈占奏,移駐莊浪。二十年,慶陽民耿飛糾番酋達爾嘉濟農等為亂,犯河州,思克與勇遣兵討平之。二十二年,追論請緩師罪,罷提督,奪世職,仍留總兵。二十三年,復授甘肅提督。

二十九年,學士達瑚、郎中桑格使西域歸,至嘉峪關外,為西海阿奇羅卜藏所劫。思克遣游擊硃應祥誘質其宰桑,達瑚等乃得返。又遣副將潘育龍、游擊韓成率師討之,斬四百餘級,阿奇羅卜藏敗走。復使詰責西海諸臺吉,諸臺吉懼,籍阿奇羅卜藏家償所掠。思克疏請免窮治,上嘉思克籌畫合宜,如其請。

三十年,疏言:「噶爾丹巢穴距邊三十餘程,其從子策妄阿喇布坦在西套住牧。雖叔侄為仇,慮其復合,侵掠西海,道必經嘉峪關外。今設副將,威望未尊,兵不盈千,不足資控御。請設總兵一、兵三千,以固邊圉。甘肅地瘠民貧,布種收穫,與腹地迥別。縱遇豐年,輸將國賦,僅贍八口,並無蓋藏。兵馬糧料,不敷供支。宜於河西要地,屯積糧草。本地無糧可買,輓運又恐勞民。請開事例,捐納加級、紀錄、職監。俟邊儲稍充,即行停止。」三十一年,加太子少保,予世職拜他喇布勒哈番。疏乞休,復慰留。加振武將軍。

三十二年,噶爾丹為亂,命內大臣郎岱率禁旅出駐寧夏,以思克為參贊。三十五年,上親征,大將軍費揚古當西路,思克率師出寧夏,與會於翁金。上駐蹕克魯倫河,噶爾丹遁去,費揚古督兵邀擊,戰於昭莫多。思克將綠旗兵居中,與諸軍並力奮戰,大破之,逐北三十餘里,噶爾丹引數騎走。詔褒諭,召詣京師,命侍衛迎勞,禦制詩,書箑以賜。入對暢春園,賜綏懷堂額及端罩、四團龍補服、孔雀翎、衣冠、鞍馬,並賚從入京師官兵糧料。命駐肅州,詗噶爾丹蹤跡。三十七年,敘功,加拖沙喇哈番。三十九年,以病乞休,遣醫往視,仍命留任養痾。尋卒,贈太子太保,賜祭葬,謚襄武。喪還京師,命皇長子允禔臨奠。

思克鎮邊久,威惠孚洽。喪還自甘州,至潼關,凡道所經,軍民號泣相送。上聞狀,嘆曰:「使思克平昔居官不善,何以得此?」進世職一等阿思哈尼哈番兼拖沙喇哈番。乾隆四年,定封一等男。三十二年,命世襲罔替。曾孫慶成,自有傳。

馬進良,甘肅西寧人。初入伍,隸思克軍。從攻平涼,輔臣拒戰,賊斫思克手。進良聞之,曰:「斫我總兵手,我必殺之!」乃入賊陣,逐斫思克手者殺之,身被數創。敘功,累遷游擊。思克請補中軍參將,格部議,上特允之。復再遷,授古北口總兵。上征噶爾丹,命將千五百人從。擢直隸提督,諭獎飭營伍,訓練嚴明。中軍參將缺,上特授其子龍。尋以老乞休。卒,賜祭葬,謚襄毅。

論曰:世稱河西四將,以勇為冠,忠勇篤誠,識拔裨佐,同時至專閫,奉指揮維謹。高宗許為古名將,允哉!良棟、進寶,轉戰定四川,進寶實首功,乃忼爽多所忤,聖祖力全之,始以功名終。進寶亦與良棟齟,不令並下雲南,怏怏稱疾,命其子代將。思克請緩師,雖不得與良棟、進寶同功,仍俾坐鎮,皆聖祖馭將之略也。思克戰功微不逮,而惓惓愛民,可謂知本矣。


\end{pinyinscope}