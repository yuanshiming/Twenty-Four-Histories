\article{列傳一百二十八}

\begin{pinyinscope}
慶桂劉權之戴衢亨戴均元

托津章煦盧廕溥

慶桂,字樹齋,章佳氏,滿洲鑲黃旗人,大學士尹繼善子。以廕生授戶部員外郎,充軍機章京,超擢內閣學士。

乾隆三十二年,充庫倫辦事大臣,遷理籓院侍郎。三十六年,授軍機大臣。居二載,出為伊犁參贊大臣,調塔爾巴哈臺。哈薩克巴布克詭稱阿布勒畢斯授為哈拉克齊,偕阿布勒畢斯之子博普來貢馬。慶桂以博普未至,巴布克狡詐不可信,斥之。上嘉其有識,曰:「尹繼善之子能如此,朕又得一能事大臣矣!」四十二年,授吏部侍郎。調烏里雅蘇臺將軍,授正黃旗漢軍都統,以病回京。逾年,授盛京將軍,調吉林,再調福州。四十九年,入覲,授工部尚書,仍直軍機,調兵部。逾年,署黑龍江將軍。時陜甘總督福康安赴阿克蘇安輯回眾,上以慶桂練邊事,命帶欽差關防,馳往甘肅,暫署總督。尋授塔爾巴哈臺參贊大臣。五十一年,召授兵部尚書,歷署盛京、吉林、烏里雅蘇臺將軍。五十七年,廓爾喀平,予議敘,圖形紫光閣,上親制贊。

兩淮鹽運使柴楨私挪課銀彌補浙江鹽道庫藏,命偕長麟赴浙按治,得巡撫福崧婪索侵蝕狀,讞上,福崧、楨俱伏法。尋授荊州將軍。逾年,召授正紅旗蒙古都統,命勘南河高家堰石工。嘉慶四年,授刑部尚書、協辦大學士,復直軍機。授內大臣,監修高宗實錄,加太子太保。拜文淵閣大學士,總理刑部。裕陵奉安禮成,晉太子太傅,管理吏部、理籓院、戶部三庫事。七年,三省教匪平,以贊畫功,予騎都尉世職,賜雙眼花翎。九年,授領侍衛內大臣。高宗實錄成,賞紫韁,晉太子太師。十六年,扈蹕熱河,以腿疾免從行圍,予假回京。十七年,晉太保。上念其年老,罷直軍機處,仍授內大臣。

慶桂性和平,居樞廷數十年,初無過失,舉趾不離跬寸,時咸稱其風度。逾年,命以原品休致,給予全俸。二十一年,卒,謚文恪。

劉權之,字雲房,湖南長沙人。乾隆二十五年進士,選庶吉士,授編修,累擢司經局洗馬。四十三年,督安徽學政。預修四庫全書,在事最久,及總目提要告成,以勞擢侍講。五十年,大考二等。逾年,擢大理寺卿,遷左副都御史。疏言:「大挑舉人多夤緣,請於事前一日簡派王大臣,聞命即宿朝房,以杜弊竇。」於是命在午門蒞事,御史監視,護軍巡察,步軍、五城一體嚴查,著為令。尋督山東學政。五十六年,擢禮部侍郎。六十年,典江南鄉試,留學政。嘉慶二年,調吏部。

四年,擢左都御史,典會試。疏言:「買補倉穀,地方官奉行不善,在本境採買,不論市價長賤,發銀四五錢。花戶不原納穀,惟求繳還原銀,加倍交價。富戶賄吏飛灑零戶,轉得少派。善良貧民深受其累。官以折價入己,仍無存米。遇協濟鄰省,令米商倉猝購辦,發價剋扣,起運勒掯。請飭遇應買補,向豐稔鄰縣公平採辦,不得於本縣苛派,嚴禁胥吏舞弊。」又言:「社倉大半借端挪移,管理首事與胥吏從中侵盜,至歉歲顆粒無存,以致殷實之戶不樂捐輸,老成之士不原承辦,請一律查禁。」詔韙之,飭各直省嚴禁,民得免累,湖、湘間尤稱頌焉。

編修洪亮吉上書王大臣言事戇直,成親王徑以上達,權之與硃珪未即呈奏,有旨詰問,自請嚴議。上以權之人品端正,平時陳奏不欺,寬其處分。尋遷吏部尚書。五年,典順天鄉試。六年,命為軍機大臣。越一歲,會川、楚、陜教匪戡定,權之入直未久,上嘉其素日陳奏時有所見,疊予褒敘。在吏部久,疏通淹滯,銓政號平。九年,失察書吏虛選舞弊,因兼直樞廷,薄譴之,調兵部。十年,以禮部尚書、協辦大學士,加太子少保。軍機章京、中書袁煦者,故大學士紀昀女夫也,入直已邀恩敘,權之於昀有舊恩,至是復欲以袁煦列薦。同官英和議不合,已中止,英和密請晏見,面劾權之瞻徇。上不悅,兩人同罷直,下廷議革職,念權之前勞,降編修。未幾,擢侍讀,遷光祿寺卿,歷遷兵部尚書。

十五年,協辦大學士,典順天鄉試。是年,帝以秋獮幸熱河,明年,幸五臺,並命留京辦事,拜體仁閣大學士,管理工部,復加太子少保。十八年,目疾乞假,遣御醫診視。會逆匪林清為變,事定,朝臣衰病者多罷退,詔以原品休致回籍,給半俸。二十三年,卒於家,年八十,謚文恪。

戴衢亨,字蓮士,江西大庾人。父第元,由編修官太僕寺卿。衢亨年十七,舉於鄉。乾隆四十一年,召試,授內閣中書,充軍機章京。四十三年,成一甲一名進士,授翰林院修撰,典試湖北。叔父均元、兄心亨並居館職,迭任文衡,稱「西江四戴」。尋命仍直軍機。秋獮扈蹕,射包以獻,高宗賜詩美之。累典江南、湖南鄉試,督山西、廣東學政,歷遷侍講學士。

嘉慶元年,授受禮成。凡大典撰擬文字,皆出其手。二年,命隨軍機大臣學習行走,以秩卑,特加三品卿銜。累遷禮部侍郎,調戶部。四年,仁宗始親政。衢亨以病乞假;假滿,兼署吏部侍郎。六年,擢兵部尚書,兼管順天府尹、戶部三庫。川、楚、陜教匪以次削平,以贊畫功,屢荷優褒。七年,大功戡定,詔嘉其知無不言,言無不盡,克盡忠悃,加太子少保,予雲騎尉世職。九年,失察順天府書吏盜印,罷兼尹。十年,調戶部,兼直南書房,典會試。十二年,協辦大學士,兼翰林院掌院學士,典順天鄉試。十三年,偕大學士長麟視南河。時河事日敝,帝銳意整頓,中外臣工議不一,特命查勘籌議。衢亨叔均元方以總河謝病家居,許便道省視,遂與長麟三疏陳治河要義,斟酌緩急,停修毛城鋪滾水壩,復天然閘東山罅閘壩,以減黃濟運;於王營減壩西,增築滾壩、石壩,普培沿河大堤,以淮、揚境內為尤急。雲梯關外八灘以上,接築雁翅堤以束水勢。高堰、山盱石堤加築後戧土坡,為暫救目前之計,徐辦碎石坦坡以護石工。智、禮二壩加高石基四尺,以制宣洩。疏上,帝深韙之,命嗣後考覈河工以為標準。十四年,萬壽慶典,晉太子少師。

衢亨性清通,無聲色之好。朝退延接士大夫,言人人殊,不置可否,而朝廷設施,有見之數月數年之後者。柄政既久,仁宗推心任之。給事中花傑疏論長蘆欠課,衢亨方筦戶部,議下鹽政覈辦。傑乃劾衢亨與鹽商查有圻姻親,餽送往來,助營第宅,不免徇庇;又廷試閱卷,援引洪瑩為一甲一名,有交通情狀;薦周系英、王以銜、席煜、姚元之入南書房,與英和陰附結黨。衢亨疏辨,下廷臣察詢,命二阿哥監視洪瑩覆寫試策,無誤,迭詔為衢亨湔雪;惟斥其令部員劉承澍在園寓具稿,致招物議,予薄譴,鐫級留任;坐傑污衊,承澍漏洩,降黜有差。因調衢亨工部。復以凡部臣有直軍機者,遇交議,同官每向探意旨,事後輒相推諉,特諭申儆焉。十五年,拜體仁閣大學士,管理工部,兼掌翰林院如故。

十六年春,扈蹕五臺,至正定病,先回京。尋卒,年五十有七。溫詔優恤,稱其謹飭清慎,實為國家得力大臣,親臨賜奠,贈太子太師,入祀賢良祠,謚文端。子嘉端,年甫十一,賜舉人,襲雲騎尉。

戴均元,字修原。乾隆四十年進士,選庶吉士,授編修。遷御史,迭典江南、湖北鄉試,督四川、安徽學政。嘉慶三年,由安徽任滿還京,兄子衢亨先已超授軍機大臣,故事,大臣親屬任科道者,對品回避,均元例改六部員外郎,特命以鴻臚寺少卿候補。累擢工部侍郎。

八年,偕侍郎貢楚克扎布察視張秋運河及衡家樓決口工程。歷戶部、吏部侍郎。十年,南河黃流奪運,高堰石工壞,特命馳視籌度。明年,詔以湖、河異漲,高堰堤工賴先築子堰,保衛無虞,清水暢注,河口積淤刷滌,已復三分入運、七分入黃舊制,為河事一大轉機,嘉均元盡心宣防,特復正、副總河舊制,授南河總督,以舊督徐端副之。在任三年,堵合黃河周家堡、郭家坊、王營減壩、陳家浦,及運河二堡、壯原墩,築高堰義字壩,拆修惠濟閘,以減壩合龍,加太子少保。病,乞解任,尋愈,因事降三品京堂,授左副都御史,督順天學政。未幾,遷倉場侍郎。十八年秋,河決睢州,出為東河總督。詔以均元曾任南河,許便宜調用工員,責速堵合。明年春,以吏部侍郎內召,途次擢左都御史。尋遷禮部尚書,調吏部。二十年,協辦大學士。逾年,授軍機大臣,充上書房總師傅。二十三年,拜文淵閣大學士,晉太子太保,管理刑部。二十四年,河決武陟馬營壩,自秋徂冬尚未啟工,奉命馳視,還報購料未集,詔嚴斥在事諸臣以示儆。

二十五年七月,扈從熱河,甫駐蹕,帝不豫,鄉夕大漸。均元與大學士托津督內侍檢御篋,得小金盒,啟鐍,宣示御書立宣宗為皇太子,奉嗣尊位,然後發喪。洎還京,因撰擬遺詔有「高宗降生於避暑山莊」之語,誤引禦制詩注,樞臣皆被譴鐫級,均元與托津並罷直。道光二年,裕陵隆恩殿柱蠹朽,距修建甫二十年,承辦工員俱獲罪。均元以在事未久,從寬罷管部務,奪宮銜,責同賠修,工畢復之。漳水北徙,命均元馳視。次年,因漳水下流潰直隸元城紅花堤,塞之則元城北境水無所洩,不塞則山東館陶受其害,復命均元往視。議展寬舊有引河,俾積水穿堤入衛水,別就堤下新刷水溝挑成河道,分流洩入館陶境,築堤防溢。復偕巡撫程祖洛勘上游,議:「漳水自乾隆五十一年南徙合洹水後,衛水為所格阻,頻年沖決,由於合則為患。今漳水北徙,與洹水分流入衛,當因勢利導,各完堤防,使漳、洹不再合。」疏上,詔從之。四年,予告回籍,食全俸。

先是建萬年吉地於寶華峪,均元相度選定。帝敦崇儉樸,命偕莊親王綿課、協辦大學士英和監修,面戒規制一從節減。迨七年,孝穆皇后梓宮奉安,帝親視,嘉其工程堅固,晉均元太子太師。及是,地宮有浸水,上震怒,嚴譴在事諸臣,褫均元職,逮京治罪,擬重闢,念其耄老,免罪釋歸。

均元歷官五十餘年,叔侄繼為樞相,家門鼎盛。自在翰林,數司文柄,及躋卿貳,典順天鄉試一,典會試三。晚歲獲咎家居,世猶推為耆宿。二十年,卒,年九十有五。

托津,字知亭,富察氏,滿洲鑲黃旗人,尚書博清額子。乾隆中,授都察院筆帖式,充軍機章京,累遷銀庫郎中。改御史,遷給事中。嘉慶元年,命解餉銀赴達州。五年,授副都統,留治四川軍需。疏請軍餉先一月預撥,忤旨召回。及至京,於餉數、軍事無所陳告,褫職,予頭等侍衛,充葉爾羌辦事大臣。七年,調喀什噶爾參贊大臣,復授副都統。八年,召為倉場侍郎。

十年,調吏部,命在軍機大臣上行走。偕直隸總督吳熊光往湖北,按訊鹽法道失察岸商抬價,及錢局鼓鑄偷減,治如律。時總督百齡被訐在廣東索供應、造非刑,命托津偕總督瑚圖禮治其獄,請褫百齡職。十一年,調戶部,偕侍郎廣興按東河總督李亨特勒派員,奪亨特職,遣戍。十二年,偕侍郎英和按訊熱河副都統慶傑貪婪,褫職遣戍。

十三年,偕尚書吳璥勘南河。先是,雲梯關外陳家浦漫決,由射陽湖旁趨海口,疆臣、河臣請改河道徑由射陽湖入海。托津等疏言:「馬港口、張家莊漫水西漾數十里,始折歸北潮河。如果地勢建瓴,何以轉向西流?北潮河已匯流數月,水未消涸,顯見去路不暢,改道斷不可行。請仍修故道,接築雲梯關外大是,收束水勢,較為得力。」又言:「河口高堰各工,因運河西岸堵築漫缺,頭、二壩口門較寬,不能擎托暢注,請速補築。」皆如所議行。

十四年,往江南讞獄。金山寺僧志學與王兆良爭墾沙地械斗,斃多人,依律治罪。請以蔣家沙洲歸公佃種,歲給寶晉書院及金山寺租銀各千兩。倉場書吏高添鳳舞弊,通州中、西二倉虧缺,命偕福慶勘訊,坐以奸吏骫法罪。既而,部鞫添鳳,復得私出黑檔領米狀,托津亦以久任倉場,譴責分賠。浙江學政劉鳳誥代辦鄉試監臨,有聯號弊,偕侍郎周兆基、少卿盧廕溥往按得實,論鳳誥遣戍。山西署布政使劉大觀劾前任巡撫初彭齡任性乖張,偕侍郎穆克登額往按,彭齡,大觀俱被嚴議。十五年,擢工部尚書,調戶部,兼都統。偕盧廕溥往四川按事,總督勒保寢匿名揭帖,據實上聞,罷勒保大學士職。又偕府尹初彭齡往南河清查工帑。十六年春,兩江總督松筠調任,命托津暫代。尋回京,加太子少保,兼內大臣。

十八年,扈蹕熱河,教匪林清逆黨闌入禁城,命托津回京察治。林清就獲,詔優獎,授協辦大學士。時匪黨李文成據河南滑縣,山東、直隸皆震動。那彥成督師,遷延未進,托津往代。既而那彥成連戰皆捷,命托津赴開州、大名督率提督馬瑜剿匪。十九年,授正白旗領侍衛內大臣,拜東閣大學士,管理戶部,晉太子太保。侍郎初彭齡劾兩江總督百齡、江蘇巡撫張師誠受餽送,布政使陳桂生冊報蒙混,命偕尚書景安往按。彭齡坐劾未實,被譴。二十一年,那彥成前在陜甘總督任與布政使陳祁挪賑事覺,命托津往按,那彥成逮京,即代署直隸總督,尋回京。

仁宗綜覈庶政,知托津樸誠,於行省有重事大獄,率以任之,無一歲不奉使命。二十二年,管理理籓院。二十四年,萬壽慶典,賜雙眼花翎、紫韁。二十五年,仁宗崩於熱河避暑山莊,事出倉猝,托津偕大學士戴均元手啟鐍盒,奉宣宗即位。尋因遺詔引事舛誤,詔切責,托津、均元並以年老罷軍機大臣,降四級留任。道光元年,命題仁宗神主,晉太子太傅。二年,與玉瀾堂十五老臣宴,繪像,禦制詩有「立朝正色」之褒。調管刑部。以子婦乘轎入神武門中門,坐治家不嚴,奪紫韁、雙眼花翎,尋復之。十一年,致仕,食全俸。十五年,卒,年八十有一。帝親奠,賜金治喪,贈太子太師,祀賢良祠,謚文定。

章煦,字曜青,浙江錢塘人。乾隆三十七年進士,授內閣中書,充軍機章京,累遷刑部員外郎。屢典鄉試,督陜甘學政,任滿仍留刑部,改御史。嘉慶六年,擢太僕寺少卿。詔以軍事方殷,煦習機務,仍留直。七年,三省教匪平,始罷直供本職。偕侍郎那彥寶往雲南按布政使陳孝升等冒銷軍需,治如律。歷太僕寺卿、順天府尹。十年,出為湖北布政使。逾年,擢巡撫。十三年,召為刑部侍郎。偕侍郎穆克登額往雲南按事。貢生任澍宇誣訐官吏冒銷軍需不實,論反坐。授貴州巡撫,未至,調雲南,署云貴總督。十四年,調江蘇巡撫,署兩江總督。時議行海運,下煦籌議,疏陳不便,寢之。十七年,入覲,乞改京秩,授刑部侍郎,偕侍郎景安往直隸讞獄。十八年,河南教匪起,直隸總督溫承惠赴剿,命煦代攝。尋擢工部尚書,調吏部,仍留署職。捕教匪馮克善械送京師,加太子少保。

十九年,回京,典會試。山東金鄉竊賊聚眾拒捕,巡撫同興以邪教餘黨聞。煦偕那彥寶往鞫,得狀,依律論罪。知州袁潔誣報,褫其職。上知山東吏治廢弛,命煦等嚴察以聞,遂劾同興玩洩,以致地方凋敝,倉庫空虛,及布政使硃錫爵徇私廢公狀,並褫職,命煦署巡撫,清查虧空。尋以陳大文調任,同治其事,責煦議定章程。疏言:「嘉慶十四年清查,原奏虧銀一百七十九萬有奇。今查十四年以前實虧三百四十一萬有奇,十四年以後又續虧三百三十四萬有奇。擬請清釐籓庫,嚴交代,定徵解分數,以杜新虧;立追繳及分賠限期,催徵民欠,以懲延宕;覈減提款,確查無著之虧,以示體恤;覈攤捐案,據估變流抵產物扣抵,先侭正項倉庫一律籌補,軍需墊解,查明方許列抵,以防朦混。」凡十四條,下部議行。

二十年,偕侍郎熙昌往湖北、廣東、江蘇、安徽讞獄:襄陽人吳煥章誣告易成元、易登朝等勾結謀逆,反坐論罪;襄陽知縣周以焯濫押斃命,遣戍。雷州府經歷李棠誣訐兩廣總督蔣攸銛,遣戍;雷瓊道胡大成苛派屬員,褫職;貴縣知縣吳遇坤刊書詆毀上官,遣戍;洋商盧觀恆濫祀鄉賢,黜之;江蘇知縣王保澄誣訐上官諱匿邪書,遣戍;阜陽捻匪糾搶殺人,論如律。

二十一年,調禮部尚書,授軍機大臣。調刑部,管理禮部。二十二年,病免。尋授兵部尚書、協辦大學士,兼管順天府尹事。二十三年,拜東閣大學士,管理刑部。萬壽慶典,晉太子太保。二十五年,以足疾累疏乞休,予告致仕,食全俸。居家久之,道光四年,卒,謚文簡。

煦久任樞曹,練習政事,易又歷中外,數治大獄。晚始參樞務,未久病去,再起管部,以盡心刑事,京察特被獎敘焉。

盧廕溥,字南石,山東德州人。祖見曾,康熙六十年進士,官至兩淮鹽運使。父謙,漢黃德道。

見曾起家知縣,歷官有聲。為兩淮鹽運使,以罪遣戍,復起至原官。當乾隆中葉,淮鹺方盛。見曾擅吏才,愛古好事,延接文士,風流文採,世謂繼王士禎。在揚州時,屢值南巡大典,歷年就鹽商提引,支銷冒濫,官商並有侵蝕。至三十三年,事發,自鹽政以下多罹大闢。見曾已去官,逮問論絞,死於獄中。籍沒家產,子孫連坐,謙謫戍軍臺。廕溥甫九歲,貧困,隨母歸依婦翁,讀書長山。越三年,大學士劉統勛為見曾剖雪,乞恩赦謙歸,授廣平府同知。廕溥刻苦勵學,至是始得應科舉。

乾隆四十六年,成進士,選庶吉士,授編修。阿桂為掌院,激賞其才。五十六年,大考,降禮部主事。阿桂言廕溥能事,改部可惜。帝曰:「使為部曹,正以治事也。」累司文柄,典山西鄉試,督河南學政。嘉慶五年,充軍機章京,川、楚軍事,多所贊畫。八年,孝淑睿皇后奉安山陵,故事,皇后葬禮無成式,禮臣所議未當。廕溥回直儀曹,考定禮文,草撰大儀,奏上,如議行。數隨大臣赴各省按事,累擢光祿寺少卿。十六年,大學士戴衢亨卒,仁宗以廕溥諳習樞務,數奉使有勞,加四品卿銜,命在軍機大臣上行走。歷通政司副使、光祿寺卿、內閣學士。十八年,擢兵部侍郎,調戶部。扈從熱河,會教匪起,滑縣林清入犯禁城,夜半聞報,至行在面進機宜,越日從駕還京。事平,優敘,賜子本舉人。

二十二年,擢禮部尚書,調兵部。上以廕溥實心任事,特加太子少保。尋調戶部,兼署刑、吏兩部尚書。二十三年,館臣撰進明鑒,未合上意,命廕溥偕托津、章煦、英和、和瑛為總裁,遴擇翰林才識兼長者,重加核改,書成,詔褒之。工部主事潘恭辰監督琉璃窯,不受漏規,馭吏嚴,吏誣訐侵冒,下獄。恭辰貧而無援,文書證據不得直,罪且不測,輿論憤之。上微聞,命廕溥詳鞫,得其狀,釋恭辰,置吏於法。後恭辰至雲南布政使,以清操名。二十五年,典會試,會元陳繼昌,故大學士宏謀玄孫也,鄉試、殿試皆第一。有清一代科舉得三元者,惟乾隆中錢棨及繼昌兩人。上制詩,命廕溥等賡和,以紀盛事。是年秋,帝崩,因撰擬遺詔不慎,降五級留任。尋調工部。

道光元年,調吏部,兼管順天府尹,罷軍機大臣。次年,猶以直軍機久,調任後亦能盡心,加恩予優敘。七年,協辦大學士。十年,拜體仁閣大學士,管理刑部。十三年,以疾乞休,加太子太保,食全俸。十九年,重宴鹿鳴,晉太子太傅。尋卒,年八十,贈太子太師,謚文肅。

論曰:仁宗綜覈名實,樞臣中戴衢亨最被信用,衢亨亦竭誠贊襄,時號賢相,晚遭彈劾,而睠注不移。均元繼之,卒以顧命嫌疑,不安於位。豈盈滿之不易居耶?慶桂、劉權之並以老成雍容密勿,托津、章煦、盧廕溥則奉使出入,數按事決獄,寄股肱耳目之任。因人倚畀,蓋各有所專焉。


\end{pinyinscope}