\article{列傳一百四十五}

\begin{pinyinscope}
馮光熊陸有仁覺羅瑯玕烏大經清安泰

常明溫承惠顏檢

馮光熊,字太占,浙江嘉興人。乾隆十二年舉人,考授中書,充軍機章京。累擢戶部郎中。三十二年,從明瑞赴雲南,授鹽驛道,母憂歸,坐失察屬吏科派,奪職。服闋,以員外郎起用,仍官戶部,直軍機,遷郎中。從尚書福隆安赴金川軍,授廣西右江道,署按察使兼鹽驛道。歷江西按察使、甘肅布政使。四十九年,石峰堡回民作亂,籌畫戰守,儲設餉需具備。以前江西巡撫郝碩迫索屬吏事覺,同官多獲譴,光熊亦緣坐奪官,留營效力。事平,用福康安薦,起為安徽按察使。洊擢湖南巡撫,調山西。

時議河東鹽課改歸地丁,光熊疏言:「河東鹽行山、陜、河南三省,商力積疲,易商加價,俱無所濟。若課歸地丁,聽民販運,無官課雜費、兵役盤詰、關津留難,較為便宜。山西州縣半領引行鹽,半食土鹽、蒙古鹽,仍納引稅。其間或引多而地丁少,或引少而地丁多,徵之三省皆然。請將課額四十八萬餘兩通計均攤。」允之。五十七年,上幸五臺,各疆吏先後奏陳,自鹽課改革後,價頓減落,民便安之。詔嘉光熊調劑得宜,賜花翎、黃馬褂,署工部侍郎。未幾,授貴州巡撫,調雲南。五十九年,署云南總督。明年,大塘苗石柳鄧叛擾銅仁,光熊赴松桃防禦,以思州田堧坪、鎮遠四十八溪、思南大坪,密邇楚苗,且扼銅仁後路,分兵屯守。苗匪急攻松桃、正大,不得逞。旋赴銅仁治餉需,偕總督福康安治軍設防,規畫稱旨,命留貴州巡撫任。

嘉慶二年,事平,奏請銅仁、正大改建石城,以資捍衛,從之。會仲苗又起,偕總督勒保督率鎮將,聯合滇、黔、楚、粵諸軍剿撫,事具勒保傳。光熊分檄將吏,解歸化圍,肅清播東、播西兩路,降安順、廣順所屬苗寨。仲苗平,偕勒保奏上善後四事,請隨征武舉、武生及鄉勇,就近補充弁兵餘丁,給難民棲止、牛具費用,儲糧備兵民就食,清釐田畝,靖苗、漢之爭。自軍興以來,凡所措置,多邀嘉許。勒保移師入川,善後專任光熊。三年春,復疏請申禁漢民典買苗田,及重債盤剝,驅役苗佃;禁客民差役居攝苗寨;酌裁把事土舍亭長,定夫徭工價,以利窮苗;酌設苗弁,以資管束:悉報可。五年,詔光熊治理有聲,年近八旬,召授兵部侍郎,尋擢左都御史。六年,卒,上念前勞,賜祭一壇。

陸有仁,浙江錢塘人。乾隆三十四年進士,授刑部主事,累遷郎中。四十六年,出為廣西梧州知府,調太平。五十二年,安南內訌,夷眷來奔,有仁處置得宜。會擢福建延建邵道,總督孫士毅請留防邊。尋調督糧道,歷山東按察使、直隸布政使。五十七年,坐在山東讞獄草率,降甘肅按察使。

嘉慶元年,擢刑部侍郎,留治甘肅賑務,宜綿赴陜剿教匪,命攝陜甘總督。二年,匪由河南竄硃陽關,逼雒南。疏請偕西寧鎮總兵富爾賽馳赴潼、商,又調甘涼鎮兵會剿,詔軍務責巡撫,有仁應駐甘肅,親身赴陜,跡涉張皇,命回蘭州,停止所調鎮兵。時宜綿檄調撒拉爾回兵二千赴興安,有仁並令暫停,上以漢中兵單,待回兵截剿,乃教匪竄漢陰而回兵尚滯循化,斥有仁一經申飭,於應援之兵,亦屢催罔顧,詔褫職鞫訊,尋原之,發四川效力。授陜西按察使,遷布政使。三年,襄陽賊高均德犯陜西,敘防堵功,賜花翎。四年,擢廣東巡撫。

五年,召為工部侍郎,調刑部。授陜西巡撫。先是那彥成在陜,勸民築寨堡,計藍田、郿、鄠、寶雞、商州、鎮安、商南、孝義、五郎共五百四十一處;臺布為巡撫,復議漢中二棧為軍餉要道,於寶雞、鳳縣、留壩、褒城、寧羌各驛築堡,以周三里為度,徙民屯糧。至是尚未盡實行,嚴詔切責。有仁疏言:「川、陜情形不同,四川地居天險,如大成寨、大團包、方山坪等寨,每處可容數萬人,小者亦數千人。賊據之可抗官兵,民守之亦可拒賊。如南山內層巒疊嶂,無寬敞環抱之所,止能於陡險山巔,就勢結構,每寨止容數百人至千餘人。蜀山多膏腴稻田,民居稠密,其勢易合。陜西老林,惟棚民流寓,零星墾種,隔十里數十里,始有民居十數戶。若糾合數村共築一堡,則南村之人欲近南,北村之人欲近北,惟秦隴以西,人皆土著,無不踴躍興工。秋間賊入西棧,每約彼此各不相犯,而寨民必乘間截其尾隊,奪其牲畜,不使晏然空過。其西安、同州、鳳翔三府,與漢南附近川省之區,皆多土著,審利害,每邑結有堡寨,或百餘或數百。其漢北山內近亦一律興工,又恐結寨後民丁但知守寨,而於賊出入要隘轉無堵御;復令於寨堡之外,每寨撥數百數十人合力守卡,以杜窺伺。請分區責成各道,刻期完竣。」疏入,報聞。有仁與額勒登保規畫築堡團練,著有成效。撫輯難民無歸者,以安康、白河等處叛產,及南山客民荒田,量給安插。六年,分撥兵勇防守總要隘口,奏請於五郎、孝義等處專派大員團練堵剿,以專責成。川匪逼黑河,遣總兵齊郎阿、通判雒昂截擊,餘匪東竄牛尾河,副將韓自昌殲之,被優敘。

有仁治陜三年,經理餉需,先事綢繆,撙節不濫,搜捕餘匪甚力,屢詔褒嘉。七年,卒,優恤,官其子繼祖主事。

覺羅瑯玕,隸正藍旗。捐納筆帖式,累遷刑部郎中。超擢內閣學士,出為江蘇按察使。乾隆五十年,召授刑部侍郎。逾年,授浙江巡撫。五十二年,大兵剿臺灣林爽文,瑯玕儲穀二十萬石於乍浦、寧波、溫州,由海道輸運,高宗嘉之。坐審擬海盜失當,吏議當革職,詔寬免,自請罰銀三萬兩。嘉善縣吏浮收,按問得實,上以浙漕積弊,瑯玕不勝任,命解職,予頭等侍衛,赴哈密辦事。五十六年,坐監修浙江海塘工程損壞,瑯玕在任未親勘,詔責賠修,應銀二十二萬七千有奇,免其半。歷葉爾羌辦事大臣、喀什噶爾參贊大臣。坐家人販玉,解任回京。尋予郎中銜,為熱河避暑山莊總管。

嘉慶二年,以三等侍衛充古城領隊大臣,召授刑部侍郎。五年,授貴州巡撫。剿擒廣順等寨苗楊文泰等,詔嘉獎,加總督銜。未幾,就擢雲貴總督。六年,貴州石峴苗叛,巡撫伊桑阿赴銅仁剿治,未即平,詔瑯玕往督師,而調伊桑阿云南。伊桑阿因按察使常明攻克石峴有所擒獲,遂謊奏親往督戰,苗皆歸伏,軍事已竣。及瑯玕至,難民擁道訴其誣,遂督兵進剿,攻克上潮、下潮諸寨,始肅清。會初彭齡劾伊桑阿貪劣,下瑯玕鞫實,上尤罪其欺罔,誅之。詔斥瑯玕於伊桑阿未親往石峴,避嫌瞻徇,降二品頂帶。

七年,維西夷恆乍繃與其黨臘者布作亂,禿樹、出亨附之。瑯玕率總兵張玉龍入山剿捕,克阿喃多賊寨,進攻諸別古山,獲禿樹。玉龍克小維西夷人,縛臘者布獻軍前磔之。進攻康普,恆乍繃遁瀾滄江外,獲其孥。分兵攻吉尾、樹苗,瑯玕駐劍川,斷賊後路,敗之於通甸、小川,克回龍廠。尋圍剿上江山箐賊,殲其渠,餘眾乞降。瑯玕以恆乍繃勢蹙,疏請撤兵,提督烏大經率兵二千駐防。賊詗官軍已退,乘水涸潛渡,糾江內降惈,復肆劫掠。瑯玕馳抵劍川,恆乍繃遁走。八年,上以首逆未獲,命永保接辦軍務。瑯玕已擒斬漢奸張有斌,臨江扎筏,聲言渡兵江外,惈惈震悚,詣軍門乞降,瑯玕令誘導諸寨擒賊自效。九月,恆乍繃潛匿山箐,官軍搜獲之,餘黨盡殲。事平,予議敘。

瑯玕以維西僻處邊隅,各夷雜居江內外,稽察難周,疏請於維西、麗江等五路設頭人,給頂帶,約束夷眾。又以維西南北路及鶴麗鎮、劍川諸汛皆要地,請裁馬為步,添兵八百,分布要隘,邊境遂安。九年,卒,謚恪勤。

烏大經,陜西長安人。由武進士授三等侍衛,出為山東德州營參將。乾隆三十九年,王倫倡亂,大經助守臨清,力戰保危城,功最多,高宗特獎之,立擢臨清副將。歷江西南贛鎮、貴州古州鎮總兵,廣西提督,調雲南。五十三年冬,率雲南兵從孫士毅征安南,至則士毅已克其都城。明年春,大軍為阮惠所襲,敗績,大經所部得鄉導,全師而返。尋母憂去職,起為甘肅提督,復調雲南。嘉慶四年,僧銅金與孟連土司構難,句結野惈,蔓延猛猛及緬寧內地,大經偕總兵蘇爾相進剿,克緬屬南柯、三節石、昔木、臘南、那招、霧籠、上中下寧安、臘東、困賽等地,破南灑河賊卡,肅清緬邊。署按察使屠述濂由猛猛一路會剿,連克大蚌山、南元寨。五年春,總督書麟視師,用大經計,分兩路進攻猛白山箐,大經由南路,連戰渡黑河,焚賊寨,首逆尋就擒,夷眾受撫。七年春,入覲。會維西事起,命大經馳回,從瑯玕進剿,大經偕總兵書成先清威遠惈匪,乃會兵維西,克康普。上意不欲窮兵,命大經留防。及匪復肆掠,進剿獨村坪及康普、小維西,連克之。八年春,與瑯玕分駐石鼓、橋頭,沿江督剿,至十月,恆乍繃就擒,乃班師。九年,卒。

清安泰,費莫氏,滿洲鑲黃旗人。乾隆四十六年進士,授刑部主事,擢員外郎。出為甘肅涼州知府,調署蘭州,擢湖南衡永郴桂道。六十年,苗疆事起,奉檄赴保靖撫輯降苗,以治餉功,賜花翎。

嘉慶元年,械送首逆吳半生、石三保至京,擢按察使,遷廣西布政使。七年,署巡撫。八年,調浙江布政使。十年,擢江西巡撫,調浙江。

十一年,海寇蔡牽犯浙洋,赴溫、臺防剿,嚴杜接濟,賊樵汲俱窮,竄去,詔褒之。總督阿林保劾提督李長庚因循玩寇,下清安泰密察,疏言:「長庚忠勇冠諸將,身先士卒,屢冒危險,為賊所畏。惟海艘越兩三旬若不燂洗,則苔黏旐結,駕駛不靈,其收港非逗留。且海中剿賊,全憑風力,風勢不順,雖隔數十里猶數千里,旬日尚不能到。是故海上之兵,無風不戰,大風不戰,大雨不戰,逆風逆潮不戰,陰雲蒙霧不戰,日晚夜黑不戰,颶期將至,沙路不熟,賊眾我寡,前無泊地,皆不戰。及其戰也,勇力無所施,全以大砲轟擊,船身簸蕩,中者幾何?我順風而逐,賊亦順風而逃,無伏可設,無險可扼,必以鉤鐮去其皮網,以大砲壞其舵身篷胎,使船傷行遲,我師環而攻之,賊窮投海,然後獲其一二船,而餘船已飄然遠矣。賊往來三省數千里,皆沿海內洋。其外洋灝瀚,則無船可掠,無嶴可依,從不敢往,惟遇剿急時始間為逋逃之地。倘日色西沉,賊直竄外洋,我師冒險無益,勢必回帆收港,而賊又逭誅矣。且船在大海之中,浪起如升天,落如墜地,一物不固,即有覆溺之虞。每遇大風,一舟折舵,全軍失色,雖賊在垂獲,亦必舍而收。洎易桅竣工,賊已遠遁。數日追及,桅壞復然,故常屢月不獲一戰。夫船者,官兵之城郭、營壘、車馬也。船誠得力,以戰則勇,以守則固,以追則速,以沖則堅。今浙省兵船皆長庚督造,頗能如式。惟兵船有定制,而閩省商船無定制,一報被劫,則商船即為賊船,愈高大多砲多糧,則愈足資寇。近日長庚剿賊,使諸鎮之兵隔斷賊黨之船。但以隔斷為功,不以擒獲為功。而長庚自以己兵專注蔡逆坐船圍攻,賊行與行,賊止與止。無如賊船愈大砲愈多,是以兵士明知盜船貨財充積,而不能為擒賊擒王之計。且水陸兵餉,例止發三月。海洋路遠,往返稽時,而事機之來,間不容發,遲之一日,雖勞費經年,不足追其前效。此皆已往之積弊也。非盡矯從前之失,不能收將來之效;非使賊盡失其所長,亦無由攻其所短。則岸奸濟賊之禁,必宜兩省合力,乃可期效。」奏上,詔嘉其公正。由是益鄉用長庚,清安泰之力也。

尋又條上防海事宜:「沿海居民,編造保甲。稽覈商販,以斷米糧出口;禁制火爆,防火藥透漏;斷絕採捕,以杜奸宄溷跡。」並如議行。十二年冬,蔡牽子至普陀寺,未獲,被譴責。尋以阮元代之,調河南巡撫。十四年,卒。

常明,佟佳氏,滿洲鑲紅旗人。由筆帖式授步軍統領主事,出為湖南桂陽知州,擢雲南曲靖知府。乾隆六十年,從總督福康安征苗疆,率兵屢克賊巢,賜花翎。鎮筸苗吳半生據蘇麻寨,自構皮寨進擊敗之,復破西梁賊砦,擢貴州貴東道。掩擊半生於板登寨,獲其弟吳老正等,半生復來犯,設伏大破之,乘勝奪賊卡五;尋由西梁進攻,毀其寨,賊糾夯柳苗為援,殲戮甚眾;乞降,拒不受,復大挫之:擢按察使,賜號智勇巴圖魯。詔以苗匪每遇敗乞降,叵測難信,飭各路將領以常明為法。進剿老烏廠,斬賊目隴老香,與總兵珠隆阿合剿大烏草河迤西苗,連克魚井、豆田三十餘寨。會大軍於古丈坪,半生適至,常明冒雨進攻,殲賊千餘;分兵克烏龍巖、茶它山諸寨卡,進圍高多寨,半生降,乘銳克鴨保寨。

嘉慶元年,剿下平隴苗於葫蘆坪,母憂,留營,偕副將海格破小竹山賊於墮河坡,俘賊目楊通等。上嘉常明奮勉,仍命署按察使。二年春,貴州仲苗起,從總督勒保討之,與施縉並為軍鋒,同破賊關嶺,復夾攻,連拔賊寨八,解新城圍,再敗之望城坡。賊匿巖洞以拒,設伏,斃賊千餘,環攻於卡子河,賊大潰,解南籠圍,加布政使銜。時黃草壩被圍久,滇、黔道梗,常明援之,克九頭山,獲偽將軍陸寶貴,毀馬鞭田賊柵,俘李阿六等,連戰皆捷,圍乃解。尋克馬鞍山,繞擊洞灑賊巢,連攻三晝夜,擒賊酋吳抱仙於三隴口,授布政使。進克安有山,搗當丈賊巢,獲逆首韋七綹須,又擒賊目黃阿金、梁國珍等於補衲山。三年,連拔雨薛巖等十八寨,苗境悉平。服闋,始蒞布政使任。

是年冬,署巡撫,疏薦總兵施縉率貴州兵赴四川剿教匪。五年,因縉戰歿,貴州兵不能救,常明坐褫翎頂。秋,入覲,詔念前勞,予三品頂帶,留巡撫署任。題銷軍需,詔詰貴陽賊蹤未至,募鄉勇多至五萬餘名,用銀十九萬餘兩,命總督瑯玕察覈。尋奏常明雖無冒帑,處置失宜,責賠繳賞恤銀九萬餘兩。六年,石峴苗與湖南苗句結為亂,巡撫伊桑阿檄常明率師攻克之,復原銜、花翎,尋授巡撫。七年,以挪用鉛廠帑銀,及失察幕僚私售鉛丸,抽匿案卷事,褫職,籍沒家產。既而予藍翎侍衛,充伊犁領隊大臣,調庫車辦事大臣。

十年,授湖北鹽法道,累遷湖北巡撫。上念常明久於軍事,以四川民、夷雜處,控制不易,十五年,特擢為總督,詔勉其盡職,減免賠項銀萬五千兩。寧遠府屬夷地,多募漢人充佃,自教匪之亂,川民避入者增至數十萬人,爭端漸起。十七年,常明疏請:「漢民移居夷地及佃種者,編查入冊,不追既往。此後嚴禁夷人招佃與漢民轉佃,並編保甲以資約束,增文員以便彈壓,移營汛以利控制。」報可。又請川省鹽課改歸地丁,聽民興販,詔斥其妨礙淮綱,不顧鄰省利害,降二級留任。

十八年,署成都將軍。二十年,中瞻對番酋洛布七力為亂,偕提督多隆阿、總兵羅思舉往剿,自里塘進攻,破之,搗熱籠賊巢,洛布七力舉家焚斃。詔以未生得逆首,不予議敘。二十一年,成都革兵謀變,悉捕之置於法,詔嘉其鎮靜。二十二年,寧越夷擾邊,遣將平之。尋卒,贈太子少保,優恤,謚襄恪。

溫承惠,字景僑,山西太谷人。乾隆四十二年拔貢,朝考首擢,除七品小京官,分吏部。拔貢內用自是始。累遷郎中。五十四年,出為陜西督糧道,母憂歸。高宗巡幸五臺,迎鑾召對,嘉其才。服闋,補延榆綏道。

嘉慶元年,川、陜、楚軍事急,承惠奉檄治興安、漢中團防。遭父憂,留軍,仍攝道事。賊犯平利,承惠馳剿,山水猝漲,墜水,遇救得免。趨扼險隘,獲捷。服闋,命以按察使銜仍補原官。五年,擢陜西按察使。疏言:「賊擾陜境,已歷數年。兵為牽綴,運餉往往不及。則駐兵以待,賊得乘間遠逸。三省邊境綿長,宜扼要駐兵,以逸待勞。」上韙之。殲匪首王金柱於安康,復破賊洵陽,賑撫流亡,民心漸定。遷布政使,仍留防。賊屢犯境,輒擊卻之。守禦興、漢先後凡六年,事定優敘。八年,調河南,修伊、洛舊渠。十年,擢江西巡撫。

十一年,調福建,兼署總督。海寇蔡牽犯臺灣鹿耳門,檄總兵許松年赴海壇、竿塘與提督李長庚會剿,三沙為蔡牽鄉里,增兵駐守,禁沿海接濟,詔嘉之。尋調署直隸總督。

十二年,上閱古北口兵,獎其嫺整,命實授。濬黑龍、溫榆、北運、滏陽諸河。十三年,上幸天津,賞黃馬褂。尋以巡幸點景科派,為肥鄉令所揭,褫花翎、黃馬褂,旋復之。十七年正月,以歲除得雪,加太子少保。鉅鹿縣民孫維儉等傳習大乘教,灤州民董懷信傳習金丹、八卦教,先後發覺,失察輕縱,褫宮銜、花翎、黃馬褂,革職留任。復以他事數被譴責。

十八年,河南滑縣教匪起,命偕提督馬瑜往剿,數戰滑縣近地,破賊於道口。尋命陜甘總督那彥成總統軍務,承惠為參贊。時匪首林清在京師起事,擾及宮禁,詔以林清傳教八年,承惠不能先事查緝,及剿匪逗留罪,褫職,留治糧餉。十九年,命以員外郎赴河南睢工效力,工竣,遷郎中,隨尚書戴均元襄理永定河工。

二十三年,授山東按察使。承惠前官畿輔,不孚眾望,及復起,頗思晚蓋。山東故多盜,偵知東平人廣平知府王兆奎三世窩盜,密捕治之,期年積案一清。掊擊貪酷,蘇困起敝,吏治為之一變,特詔褒獎,然卒不安其位。先是盜夜劫泰安富民徐文誥家,戕其傭柏永柱,縣以誤殺為文誥罪,實疑獄也。按察使程國仁入其言,鍛鍊定讞,承惠至,固疑不實,於他獄盜供得其情,銳意平反。巡撫和舜武惑於浮言,尼之。及偵獲盜首王壯於吉林,具承槍殺永柱狀。時國仁已擢巡撫,舊與承惠有嫌,且護前,不欲承惠竟是獄,檄勘堤工,承惠辭,乃劾承惠自以曾官總督,橫肆不受節制,褫職,薦前兗沂道童槐繼為按察使。槐復劾承惠濫禁無辜,以罪人充捕擾民,譴戍伊犁,其去也,國仁送於候館,居民洶洶詈之,不及送而歸。既而文誥訴於京,命尚書文孚往鞫,未至,槐倉卒定讞,釋文誥。二十五年,起承惠為湖北布政使。逾年,以衰老降戶部郎中。尋引疾歸,卒於家。

顏檢,字惺甫,廣東連平人,巡撫希深子。拔貢,乾隆四十二年,授禮部七品小京官,洊升郎中。五十八年,出為江西吉安知府,擢雲南鹽法道,調迤南。嘉慶二年,剿威遠介匪,擒匪首札杜。擢江西按察使,歷河南、直隸布政使。

五年,護直隸總督。東明縣民李車因奸砍傷七歲幼童,從重擬絞決。永年縣民梁自新勒斃繼妻及媳,訊因繼妻虐待前妻子有幅,縱媳與人通奸,同謀毒斃有幅,自新忿,將妻媳致死,從輕擬杖流。兩獄並為仁宗嘉許,特旨依議。梁自新加恩,再減杖徒。先是直隸回贖旗地租銀,積欠至十三萬兩,前總督胡季堂、汪承霈屢議調劑,未有善策,檢疏請復旗租原額以紓民力,積欠得全減免焉。

六年,擢河南巡撫。七年,詔檢前護直督有治績,命以兵部侍郎銜署理直隸總督。尋實授,賜黃馬褂。九年,京察,予議敘。檢歷官畿輔,頗為仁宗所信任。尋以束鹿縣民王洪中與張文觀鬥毆被傷,上控,承審官偏聽,王洪中受責自縊,獄經部鞫,詔斥檢玩視重案,下部議革職,改留任。又因他獄屢被詰責,檢具疏陳謝,諭曰:「方今中外吏治,貪墨者少,疲玩者多。因循觀望,大臣不肯實心,惟恐朕斥其專擅。小官從而效尤,僅知自保身家。此實國家之隱憂,不可不加整頓。卿系朕腹心之臣,其勉之。」

十年,坐易州知州陳渼虧空逾十萬,查辦不力,降調革任,予主事銜,效力吉地工程處。會永定河堤壞,責隨築賠修。又以刑部秋審,直隸省由緩改實者十四起,革主事銜,仍留工次,事竣,予五品銜,發南河委用。未幾,復因直隸官吏勾通侵帑事覺,革職,遣戍烏魯木齊。十三年,釋回。

十四年,命以主事充西倉及大通橋監督。十五年,授湖南嶽常澧道,遷雲南按察使。十六年,擢貴州巡撫,尋召來京。坐前在直隸失察灤州民董懷信等傳習邪教,降二級,以京員用。又坐涿州知州徐用書交代朦混,降補工部郎中。十九年,授山東鹽運使,命以三品頂戴為浙江巡撫,奏濬西湖興水利。上素稱檢操守才幹,而病其不能猛以濟寬,屢加訓戒。二十年,武平民劉奎養聽糾入添弟會傳習徒眾論斬,詔斥檢未究編造逆書之人,下部議;復因西湖厝棺被盜,言官劾其讞擬輕縱,命侍郎成格等往按,坐正犯由賄囑誣認,詔切責,褫職。二十四年,祝嘏,予官,補刑部員外郎,逾年授福建巡撫。

道光元年,疏陳歲進荔支樹、素心蘭採運艱難,詔永遠停貢,並嘉檢之直。二年,復擢直隸總督。先是籓司屠之申奏請直隸差徭,每地一畝攤徵銀一分,以示公平,詔俟檢到任定議;檢力言其不可行,請仍舊制。三年,以年老內召,授戶部侍郎,調倉場。復出為漕運總督。五年,坐河淤滯運,降三品銜休致。尋復以疏請截留漕糧忤旨,降五品銜。十二年,卒。

論曰:馮光熊治苗疆善後,陸有仁興陜境寨堡團練,瑯玕定石峴苗、維西夷,清安泰保全良將李長庚,常明佐勒保平仲苗,晚任蜀疆,鎮撫番夷,皆一時疆臣之能舉其職者。溫承惠治畿輔無異績,陳臬山東,則治盜清獄有聲,卒以平反冤獄遭傾陷,可謂能晚蓋矣。顏檢明於吏事,治尚安靜,而屢以寬縱獲譴焉。


\end{pinyinscope}