\article{列傳四十一}

\begin{pinyinscope}
賚塔穆占莽依圖覺羅舒恕勒貝佛尼埒坤鄂泰吳丹

畢力克圖噶爾漢阿密達鄂克濟哈覺羅吉哈里

拉哈達察哈泰根特華善席卜臣希爾根

賚塔,那穆都魯氏,滿洲正白旗人,康古里第四子。年十四,授三等侍衛。坐事免。崇德時,從伐明,圍錦州,擊松山、杏山敵兵,屢有斬獲。攻新城、高陽、霸州、壽光、博興,並先登,身中五創。被賞賚,授前鋒侍衛。

順治元年,從討李自成,敗之一片石,追至安肅、慶都。授巴牙喇甲喇章京。從豫親王多鐸轉戰河南、陜西,頻有功。二年,移師江南,克揚州,下江寧,追敗明福王於蕪湖,予拖沙喇哈番。三年,從端重親王博洛下福建,明唐王奔汀州,賚塔率師攻破其城,進拜他喇布勒哈番。明桂王據湖南。六年,從鄭親王濟爾哈朗進衡州,戰敗明將陶養用、胡一清;克祁陽,復戰敗明將周進唐、王進才及一清;又戰敗明將譚弘,取道州;又戰敗一清及明將焦璉,取全州。累晉二等阿達哈哈番兼世管佐領。十一年,明將李定國犯廣東,從珠瑪喇解新會圍,進三等阿思哈尼哈番,擢巴牙喇纛章京。十六年,鄭成功窺江寧,從安南將軍達素討之。比至,成功已敗遁,遂引兵下福建。十七年,戰廈門,師失利,坐免官,奪世職。康熙二年,署前鋒統領。擊李來亨等於茅麓山,數戰皆克。八年,擢正白旗蒙古都統。

十三年,耿精忠叛,遣其將馬九玉、曾養性、白顯忠分三道寇浙江。授賚塔平南將軍,赴援。寇犯金華,遣諸將瑪哈達、雅塔裏、拉哈等擊走之,復義烏、諸暨。精忠將王國斌屯金、衢接壤處,為群寇聲援。賚塔與總督李之芳駐衢州,精忠將周列擁眾二萬自常山入。賚塔遣瑚圖要之焦園,俘斬過半。精忠將桑明率眾五萬犯衢州,迎擊,斬級萬餘。十四年,督兵擊九玉,五戰皆捷,又破其將李廷魁,焚所屯木城。康親王傑書軍至衢州,賚塔依例歸將軍印,以都統參贊軍務。時九玉退據九龍山,分萬人扼大溪灘護糧運。傑書令賚塔攻之,即夕遣兵涉河,直搗九玉營,破之。九玉僅以三十騎遁,遂復常山。率瑪哈達等破仙霞關,拔浦城;又與吉勒塔布敗賊建陽,克之。進取建寧,薄延平,精忠乃迎降。

其時漳、泉、興化並為鄭錦所據,錦,成功子也。精忠導貝子傅拉塔軍攻錦。十六年,與寧海將軍拉哈達復興化,降仙游。進討叛將劉進忠於潮州,進忠亦降。康親王傑書奏仍授賚塔平南將軍,守潮州。十七年,錦將劉國軒入犯泉州,與總督姚啟聖會師赴援,復長泰,戰漳州,破敵。十八年,國軒復入犯,迎擊,敗走。十九年,克海澄,錦還臺灣。授賚塔本旗滿洲都統,守潮州如故。

尚之信之降也,仍懷貳志,返廣東,復抗命。都統王國棟首告,詔賚塔撫慰。之信已殺國棟反,賚塔率兵討擒之。

時吳世璠尚據雲南,大將軍貝子彰泰自湖南下貴州,上授賚塔平南大將軍,督滿、漢諸軍自廣西入雲南。賚塔師自田州、泗城道西隆,迭戰皆捷。石門坎者去安籠三十里,地峻★C7,世璠將何繼祖等擁眾拒守。賚塔令諸將希福、勒貝、瑪奇等率師前進,而別與總督金光祖等分兵自間道躡其後。二十年元旦,度賊無備,飭前軍進攻;繼祖等倉卒出御,後軍攀險上,前後夾擊,遂奪其隘口,復安籠所。繼祖復與詹養、王有功等以二萬人守黃草壩。賚塔督諸軍奮擊,自卯至未,破壘二十二,俘養、有功及其眾千餘,並獲其象、馬。捷聞,上以賚塔自廣西深入,先諸軍至,敗敵,溫詔嘉獎。

師至曲靖,遣諸將希福、瑪奇、碩塔等分道取霑益、雲龍、嵩明諸州及易龍所、楊林城。彰泰師自貴州至,兩軍合。未至會城三十里,世璠遣郭壯圖等迎戰,列象陣,彰泰軍其左,賚塔軍其右,自卯至午,賊五卻五進,殊死戰。過金汁河,象反踐,陣亂,師乘之,大潰,進屯城東歸化寺。九月,趙良棟師自四川至,遂合圍。賚塔軍銀錠山,運砲至,晝夜番攻,世璠將餘從龍降。詗知糧將罄,人相食,與諸將環而攻之。世璠眾內亂,欲擒世璠以降,世璠自殺。其將夏國相奔廣南,胡國柱奔雲龍州。遣諸將李國樑、希福等追襲之,擒國相,國柱自縊死。雲南大定。

二十一年,凱旋,上率群臣郊勞盧溝橋西,行抱見禮。二十二年,以隱匿之信籓下入官婦女,下所司集質。上諭賚塔有大功,勿以細事加罪。禮部議請奪官治罪,詔改降級罰俸。二十三年,卒,謚襄毅。二十五年,追授一等阿思哈尼哈番。

子費葉楞,襲。雍正五年,世宗命追封一等公,令其孫博爾屯襲。並諭:「賚塔克雲南,功績懋著。當日因其功過相掩,未予優封,欲使立功之臣,咸知儆惕收斂,不可恃功驕肆。今事歷多年,後人已知鑒戒。用特追封,示眷念舊臣。」九年,定公號曰褒績。

穆占,納喇氏,滿洲正黃旗人,南楮子也。南楮事具楊吉砮傳。穆占初任侍衛,兼牛錄額真。順治十六年,署噶布什賢章京。從都統卓洛等駐防雲南,平元江土司有功,予三等阿達哈哈番,擢本旗梅勒額真。

康熙十二年冬,吳三桂反,命授赫葉安西將軍,道陜西入四川進討,以穆占署前鋒統領,參贊軍務。十三年二月,師至陜西,時四川巡撫羅森、提督鄭蛟麟皆附賊,總兵譚弘亦叛據陽平關。穆占與西安將軍瓦爾喀率兵先驅,戰野狐嶺,敗之,克陽平關。總兵吳之茂叛據保寧,穆占進與戰,屢擊敗之。旋以賊阻餉道,引還漢中。提督王輔臣叛寧羌,與之茂、弘相應。穆占從大將軍貝勒洞鄂還西安。十四年,詔趣洞鄂討輔臣,而以穆占代赫葉為安西將軍,率師並進。輔臣將高鼎屯隴州河岸,迎戰,與達理善擊卻之。趨秦州,圍合,輔臣將陳萬策以城降。穆占復助提督張勇攻下鞏昌,還會諸軍征平涼。十五年,上遣圖海代洞鄂為大將軍,輔臣降。穆占分剿餘寇,以次復西河、清水、成、禮諸縣。輔臣將周養民等以慶陽降。

九月,詔入覲,進秩視都統,佩征南將軍印,統陜西、河南諸軍赴湖廣,討三桂,諸將塔勒岱、鄂克遜從。十六年正月,至荊州。時大將軍順承郡王勒克德渾守荊州,貝勒尚善圍岳州,安親王岳樂圍長沙,簡親王喇布守吉安。上命穆占助攻長沙,軍至,屯阿彌嶺。三桂初欲自松滋渡江,進攻荊州,相持數年不得逞。聞長沙有新軍至,亟自松滋還援,屯隔江嶽麓山。遣其將馬寶等屯城外,掘重壕,布鐵蒺藜,列象陣以守;而自從常德進,為穆占所敗,走衡州。上命穆占移兵會簡親王取衡州。十月,克茶陵,復攸、安仁、酃、永寧諸縣。十七年春,克郴州,傍縣並下。穆占守郴州,以都統宜理布守永興。三桂欲通粵東道,與尚之信、孫延齡軍合,遣其將馬寶、胡國柱等悉銳攻永興。穆占遣哈克三、碩岱來援。時喇布尚駐吉安,穆占請旨趣進徵。六月,宜理布、哈克三戰死。碩岱入城守,喇布遣薩克察來援,牒穆占請益兵。穆占謂永興軍事簡親王主之。喇布以聞,上責穆占謬戾。寶、國柱攻永興,二十餘日不下,聞三桂死,乃引還衡州。穆占率布舒庫等追擊,敗之耒陽。十八年,三桂將吳國貴為他將所𧾷戚,遁永州,穆占追剿,克之,道州、常寧、新田、永明、江華、東安皆下。師入廣西境,克全州、灌陽、興安、恭城。詔還定湖南,進克新寧。三桂將郭壯圖等擁三桂孫世璠據貴州。

十一月,上命貝子彰泰為定遠平寇大將軍,規定雲、貴,穆占參贊軍務。十九年二月,復沅州。十月,克鎮遠,並定偏橋、興隆二衛。進克平越,下貴陽。世璠奔雲南。十一月,克遵義、安順、石阡、都勻、思南諸府。二十年正月,世璠將夏國相、高起隆、王會、楊應選等擁眾二萬拒戰,屯平遠西南山。穆占與提督趙賴督諸軍奮擊,起隆等敗竄,會降,遂復平遠。分遣諸將莽奕祿等逐賊,復大定,應選亦降。遂入雲南,與廣西軍會,壁歸化寺。壯圖出兵重關,列象陣犯我軍。賚塔等縱兵夾擊,穆占戰尤力,象陣亂,反踐其軍。諸軍乘之,壯圖斂兵,止存二十七人,奔入城。九月,四川軍至,總督蔡毓榮破重關,穆占亦奪玉皇閣,猛攻東西寺。世璠、壯圖皆自殺。穆占入城,撫餘眾,籍逆產以聞。師還,授正黃旗蒙古都統、議政大臣。

二十二年,追論征保寧時奏軍事不實,徵平涼時不臨陣指揮,及不救永興,罪當絞,籍沒。上諭曰:「穆占固有罪,但其戰績多至二百六十處,此所議稍過。」命覆議,乃請奪官、削世職、沒妻子入內務府,上命但奪官,餘悉寬之。尋卒。

莽依圖,兆佳氏,滿洲鑲白旗人。父武達禪,崇德中從伐明,攻任丘、濟陽,並先登,賜號「巴圖魯」,予牛錄章京世職。既入關,授太原城守尉。卒。

莽依圖襲職,進三等阿達哈哈番。順治十五年,從征南將軍卓卜特下貴州,自都勻次盤江,破明將李定國。移師定雲南。康熙二年,李自成餘黨李來亨等據湖北茅麓山,未下,從靖西將軍穆里瑪攻克之。凱旋,授江寧協領。

十三年,吳三桂陷湖南,復從鎮南將軍尼雅翰攻岳州,砲擊寇艦,敗之七里橋。十四年,三桂構廣西總兵馬雄叛,廣東十府失其四。尚可喜請兵,上命尼雅翰率師赴廣東,以莽依圖署副都統,駐肇慶。甫至,而可喜子之信已叛應三桂。十五年,三桂將範齊韓等偪肇慶,莽依圖潰圍出,且戰且走,還駐江西。聞三桂將黃士標等攻信豐,亟率師赴援,遣奇兵出其背,與城兵衷擊之,賊大潰,遂會鎮南將軍覺羅舒恕解南康圍。

十六年三月,上命舒恕留兵守贛州,而授莽依圖署江寧副都統,代舒恕佩鎮南將軍印,帥師規復廣東,以額赫訥、穆成額參贊軍事。自南康進南安,再進南雄,三桂所遣守將皆出降,之信亦率籓屬歸順。莽依圖遂逾嶺進韶州,韶居五嶺脊,為贛、粵咽喉,賊所必爭。莽依圖以城北當敵沖,厚增土墻,夜則縋卒出城濬壕通水,並分兵斷廣州餉道。三桂將胡國柱、馬寶以萬餘人攻城,莽依圖屢擊卻之;乃扼河西斷我水運,又壁蓮花山發砲,女墻悉壞。會江寧將軍額楚赴援,莽依圖出城兵夾擊,破四壘,逐北至帽峰山,夜戰,大敗之。河西賊亦引去,餉運始通。莽依圖督軍追擊,破敵風門澳,斬二千餘級。下樂昌、仁化諸縣,乃還駐韶州。

時傅弘烈佩撫蠻滅寇將軍印,巡撫廣西,所將義兵五千人。莽依圖慮其力不支,遣副都統額赫訥將兵八千赴梧州佐弘烈,而之信不為具舟,師久不集。十七年二月,莽依圖至平樂,圍城,寇水陸拒戰,引還中山鎮,與弘烈互奏糾,上兩釋之。莽依圖復還梧州,引咎請罷將軍,上切責之,命留任圖功贖罪。十八年春,三桂從孫世琮犯梧州,莽依圖與弘烈謀合諸軍分布水陸,與戰,賊敗去,遂復桂林。語具弘烈傳。

三桂將馬承廕以南寧來降,世琮自梧州敗歸,並力攻南寧。城幾陷,莽依圖方臥病,聞警,督軍倍道赴援。賊悉銳依山列鹿角拒戰,莽依圖使額楚、額赫訥引前鋒兵沖擊之,而自與舒恕麾大軍進,預遣兵潛出山後斷歸路,盡殪之。世琮負重傷,以數十騎越山遁。南寧圍解。命進取雲、貴,莽依圖以承廕雖降,心叵測,疏請暫駐南寧。上命簡親王喇布鎮桂林,莽依圖俟都統希福軍至,合兵謀進取。十九年,授護軍統領。承廕果以柳州復叛,弘烈遇害。莽依圖軍進次宜賓,承廕驅象陣迎戰,以勁弩射之,象返奔,賊陣亂,鐵騎乘之,遂大敗。承廕復以柳州降。莽依圖疾益深,八月,卒於軍。

莽依圖母賢,嘗訓以不殺降,不掠民,莽依圖終身誦之,時稱「仁義將軍」。既卒,南寧人繪其像祀之。事平,朝議追論自平樂還梧州失律罪,當籍沒。上以莽依圖戰多,且不擾民,寬之,奪恩詔所加世職,以原授拜他喇布勒哈番兼拖沙喇哈番予其弟博和里。博和里曰:「兄平粵有功,上褒之,不可使吾子孫復襲此職。」乃撫其孫布瞻阿繼襲。乾隆元年,追謚襄壯。

三桂初反,十三年正月,上授都統尼雅翰鎮南將軍,會師德州,道安慶至武昌。尋命參贊軍務,攻岳州;旋又命進取南康,克之;又擊破三桂將黃乃忠等於袁州。十五年五月,上命哈爾哈齊率江寧兵攻吉安,解尼雅翰鎮南將軍印授之;螺子山敗,改授覺羅舒恕。

覺羅舒恕,滿洲正白旗人,武功郡王禮敦曾孫。康熙八年,自一等侍衛授兵部督捕侍郎,調吏部。十三年,命署前鋒統領,參贊定南將軍希爾根軍務。精忠遣將陷撫州,舒恕從希爾根進擊,克之。十四年,精忠兵復至,又擊破之,克新城、宜黃、崇仁、樂安諸縣。上命舒恕援廣東,授鎮南將軍。叛將馬雄及三桂將王弘勛攻高州,與戰不利,退駐肇慶。十五年,尚之信反,再退駐贛州。十六年,上命解鎮南將軍印授莽依圖,率師下廣東,令舒恕留兵佐巡撫佟國楨守贛州。尋復授安南將軍。三桂兵自宜章窺南雄、韶州,上命莽依圖赴韶州應敵,而舒恕守南雄為聲援。

十七年,穆占言郴州、桂陽新復,請敕舒恕移師駐守。舒恕疏言南韶為湖南、江西、廣東三省接壤,不可輕離。繼命進次梧州。十八年,即軍前授都察院左都御史。旋與莽依圖共擊吳世琮,解南寧之圍。舒恕以病乞還肇慶,召還京。入對,上察其神色如故,無病狀,詔詰責,命羈候宗人府,下王大臣議,奪職。三十四年,起鑲黃旗滿洲副都統,再遷寧夏將軍,參贊撫遠大將軍費揚古軍務,討噶爾丹。三十五年,上親征,授揚威將軍,從費揚古出西路。上駐棟斯拉,召費揚古議軍事,以舒恕署大將軍。師有功,予拖沙喇哈番世職,擢正藍旗滿洲都統。以病乞休。卒。

勒貝,郭絡羅氏,滿洲正藍旗人,鄂羅塞臣子。初授侍衛,兼管牛錄事。累遷正藍旗滿洲都統。三桂亂未平,康熙十六年春,上以簡親王喇布出師江西久無功,參贊均不勝任,命勒貝及哈克三、舒庫往代之。既,命與江寧將軍額楚守韶州;又詔進次梧州,與弘烈攻鬱林及北流、興業、陸川、博白,軍益振,乘勝下南寧,克象州。十九年秋,莽依圖卒於軍。詔勒貝代為鎮南將軍,從賚塔定雲南。抵西隆,詗知三桂將何繼祖等屯安籠所石門坎,與瑪奇率前鋒奮攻,次第克三峰,奪隘口,復安籠所。繼祖等堅守黃草壩,列象陣以待,復與賚塔大敗之,直抵雲南城。吳世璠自殺,滇平。師還,道卒。

佛尼埒,科奇理氏,滿洲鑲紅旗人,世居瓦爾喀。父索爾和諾,少孤,兄瑚里納撫之成立,後為仇所害,索爾和諾手刃之,祭兄墓。崇德三年,來歸。從伐明,攻河間,戰死,授牛錄章京世職。

佛尼埒襲職。授西安駐防牛錄額真,進二等阿達哈哈番。康熙初,累擢西安副都統。十三年春,從將軍瓦爾喀道四川討吳三桂。入棧道,聞四川叛附三桂,譚弘據陽平關。從瓦爾喀自野狐嶺進兵,斬三千餘級。進朝天關,屢擊敗敵軍。總兵吳之茂以保寧叛,移師往討之,弗克,鑿壕塹與相持。之茂出劫略陽糧艘,截槐樹驛運道。我師餉不繼,還漢中。之茂要於中途,與總兵王懷忠擊之,敗走。

其冬,提督王輔臣叛,連陷平涼、秦州。十四年,擢西安將軍,加振武將軍銜。命與貝勒洞鄂進討輔臣將高鼎,以四千人屯關山河岸,偕穆占整師與戰,破其壘;逐北,又敗之渭河橋,進薄秦州。壘未定,賊乘我不備,開壁出戰。佛尼埒督軍遮擊,賊不敢犯。旋攻克東西二關。賊數千掠仙逸關,佛尼埒慮斷餉道,分兵往援。賊逾山走,追躡之,殺其黨且盡,遂率師趨隴州。賊縱火焚山澤,佛尼埒曰:「是欲燒絕我輓運道也。若不增兵策應,軍食何賴焉?」因暫駐隴州。

時師攻秦州久未下,而四川及平涼諸寇挾萬餘人赴救,城寇與應者亦八千餘。佛尼埒亟還師與諸軍合,偕內大臣坤連敗賊眾,擒其將李國棟等,殪其眾三千餘。州城復,以次下禮縣、西和、清水、伏羌諸城。漢中運道阻,軍大饑。將軍席卜臣還西安,上命佛尼埒領兵開棧道,規漢中,緣塗擊賊,皆潰竄。十五年,之茂欲為輔臣援,再犯秦州。佛尼埒與護軍統領傑殷議繞賊後,絕其運道,復靜寧。大將軍圖海下平涼,之茂遁。又與傑殷乘夜追擊,及之牡丹園,遂克祁山堡。之茂僅以十餘騎走。

十六年,追論自保寧退還漢中諸罪,降世職為拜他喇布勒哈番,削振武將軍銜,仍署西安將軍。十七年,與吳丹等敗敵於牛頭山、於香泉,率師駐守寶雞,堅扼棧道諸隘。寇屢至,屢敗之。十八年,從大將軍圖海征興安,寇阻梁河關。佛尼埒領兵先驅,濟乾玉河,拔之。興安下。十九年,潼川降,並復鹽亭、中江、射洪諸縣。再敗寇豹子山,克瀘州。冬,吳世璠將胡國柱自敘州擾永寧,詔授建威將軍討之。二十年,克馬湖。世璠將宋國輔等以永寧降。國柱亦棄敘州遁,上命佛尼埒守之。尋命還鎮漢中。二十一年,卒。乾隆初,追謚恭靖。子托留,襲世職,官至黑龍江將軍。額倫特,別有傳。

坤,那木都魯氏,滿洲正黃旗人,先世居綏芬,隸瓦爾喀部。父伊訥克,太宗伐瓦爾喀,先眾降。坤事太宗,洊擢一等侍衛,兼管牛錄事。太宗伐明,圍松山。明總兵曹變蛟乘夜犯御營,迫正黃旗營門。諸侍衛及親軍等皆散列門左右,坤獨當門,力戰卻敵。上嘉其勇,賜號「巴圖魯」,賚白金四百,授一等甲喇章京世職。

世祖朝,累進一等阿思哈尼哈番兼拖沙喇哈番。尋以遣祭昭陵辭未往,扈蹕南臺不入直,又娶女子已賜配者為妻,論罪當死,上寬之,奪官,仍留世職侍衛。順治十一年,從靖南將軍珠瑪喇下廣東,命署固山額真。破明將李定國於新會,逐至橫州江岸,斬馘無算。擢內大臣。康熙十二年,獎先朝諸舊臣,坤加太子太保。

吳三桂反,授振武將軍,帥師駐汝寧。王輔臣叛,命移師西安。十四年,又命偕副都統翁愛等進駐漢中,輔臣毀鳳縣偏橋絕運道,又斷棧道,阻漢中聲援。詔趣坤援漢中,次寶雞,以道阻未克進。命罷將軍,以內大臣從軍。秦州既復,朝議規復漢中,以坤守潼關。

十八年,上念坤已老,召還。追論漢中逗留狀,當奪官、削巴圖魯號。上曰:「巴圖魯號太宗所賜,其勿削!但奪官。」仍留一等阿達哈哈番世職。二十四年,授散秩大臣,並諭年衰不能朝,聽家居。二十六年,卒。

鄂泰,瓜爾佳氏,滿洲正白旗人,世居蘇完。國初來歸,以軍功累進二等阿達哈哈番。順治間,授盛京禮部理事官,坐事黜,並奪世職,旋復起。康熙初,洊擢盛京副都統。王輔臣叛,大將軍貝勒洞鄂西討,命鄂泰率盛京兵千來京備徵發。十四年,授建威將軍,率所部兵駐太原。尋命赴西安參贊洞鄂軍務,以建威將軍印授副都統吳丹。鄂泰與副都統阿爾瑚屯寶雞,賊出棧道攻九龍山,鄂泰督兵縱擊,盡殪之。輔臣所署置總兵任德望率兵及惈惈七千餘屯益門鎮,鄂泰分兵九路進擊,自巳至未,破七壘。德望以百騎遁,驍騎校韓楚漢射中其股,乃降。十五年,復捕餘賊紅崖堡。十八年,卒,追授拜他喇布勒哈番兼拖沙喇哈番。

吳丹,納喇氏,滿洲正黃旗人,葉赫金臺石曾孫也。康熙初,以一等侍衛同學士郭廷祚視淮安河決。十三年,大將軍順承郡王勒爾錦討吳三桂,吳丹奉使軍中,宣諭機宜。王輔臣叛,命署副都統,從鄂泰駐太原。旋復命署建威將軍,移師潼關。十五年,從大將軍圖海征平涼,擊賊虎山墩,輔臣乞降,吳丹率數騎入城,安撫降人。

十七年,授護軍統領。時漢中、興安尚為三桂兵所據,上趣圖海進軍,以吳丹參贊軍務,戰於牛頭山、香泉,屢破賊。圖海入覲,命仍佩建威將軍印,暫統大兵。旋從圖海徇鎮安,偕將軍佛尼埒戰於火神崖,破賊,渡乾玉河,克梁河關,遂復興安。上命圖海還駐鳳翔,分兵畀吳丹,與將軍王進寶下四川,為後繼。十九年,與進寶擊賊蟠龍、錦屏諸山,大破之,遂復保寧,獲三桂將吳之茂等。時將軍趙良棟亦復成都,吳丹與佛尼埒分兵取順慶、重慶,並下達州、東鄉、太平諸州縣。詔取瀘州,趨雲南。吳丹復從佛尼埒戰於豹子山,破瀘州賊。會永寧復為賊得,仁懷亦不守,良棟劾吳丹不急赴援,解將軍印還漢中。事定,還京,王大臣等議罪,奪職。尋授三等侍衛兼佐領。

二十九年,喀爾喀臺吉額爾克阿海等為亂,噶爾丹亦犯邊,命從大將軍裕親王討之,戰於烏闌布通,噶爾丹敗走。裕親王命吳丹與參領色爾濟、博爾和岱言冋噶爾丹所在,知遠去已數日,乃還。途值喀爾喀叛者,並遇害,贈散秩大臣,予拖沙喇哈番世職。

畢力克圖,博爾濟吉特氏,蒙古正藍旗人,世居科爾沁。太宗時,來歸,授豫親王護衛。從伐朝鮮及明錦州,並有功。順治初,從討李自成,定西安,移師拔揚州,下江寧,以戰績著,署護軍統領,予牛錄章京世職,擢正藍旗蒙古副都統。六年,詔駐防平陽,賊犯絳州,擊卻之。李建泰叛據太平,復與協領根特等攻之,久弗下,乃穴地燃火藥隳城,擒建泰誅之。累進一等阿達哈哈番。授禮部侍郎,調戶部。

十一年,從靖南將軍珠瑪喇下廣東,明將李定國犯新會,屯縣左山峪。畢力克圖再戰敗之,追至興業,斬殺過半,趨橫州,定國渡江遁。進三等阿思哈尼哈番。坐事罷官,降二等阿達哈哈番。

十七年,命署護軍統領。從定西將軍愛星阿出師云南。時明桂王入緬甸,定國與白文選分據孟艮、木邦。十八年,會師木邦,定國走景線,文選走錫箔江,毀橋趨茶山。畢力克圖至,獲諜者,結筏以濟,次舊晚坡,去緬城六十里。緬人謀獻桂王,請大軍留駐,以百人進蘭鳩江備捍衛,於是白爾赫圖率前鋒以往,畢力克圖以護軍二百人從之。緬酋蟒猛以桂王出畀我軍,遂班師還。文選至猛養,為總兵馬寧追及,率眾降。畢力克圖撫其眾,徙之邊境。論功,進一等阿達哈哈番兼拖沙喇哈番。

康熙八年,擢正藍旗蒙古都統,列議政大臣。十二年,加太子少師。十四年,王輔臣叛,授畢力克圖平逆將軍,帥師駐大同。尋延安、綏德皆陷,命進駐榆林。詗知賊屯楊家店渡口,遂分兵三隊,乘夜疾進。黎明,鳴角濟河。賊不虞我師至也,皆駭走,遂復吳堡。進次虎爾崖口,遇賊,又擊敗之。下綏德,乘勝克延安,並招撫附近諸州縣。上命移師會揚威將軍阿密達攻平涼。將至,輔臣擁眾迎戰,與貝勒洞鄂等擊之,陣斬其將郝天祥。十五年,大學士圖海蒞師,命畢力克圖屯寧夏。輔臣降,還駐平涼。

十七年,移師守隴州、寶雞。圖海議取漢中,與鄂克濟哈等分道入,以次降靈臺、華亭、崇信諸縣。其冬,克成縣。十九年,徵還,仍任都統。二十年,卒,年七十有三,謚恪僖。孫常遠,襲職。二十五年,追錄陜西軍功,進二等阿思哈尼哈番。

噶爾漢,納喇氏,滿洲正紅旗人,尚書噶達渾子也。噶爾漢襲一等阿達哈哈番,授王府長史。康熙初,遷正紅旗滿洲副都統。

十四年,授鎮安將軍,駐守河南。時寇勢甚熾,總兵楊來嘉叛,命移師襄陽。十五年,戰南漳,破靈機寨。叛將譚弘等犯鄖陽,遣黨扼城東陡嶺,斷我輓運道。復與提督佟國瑤會師,分路進擊,賊退。十八年,謝泗、劉魁等掠竹山、竹谿諸縣,偪鄖城,與興安賊為聲援,噶爾漢往討之。時方溽暑,鄖西數百里,山逕★C7隘,草木叢塞,霪雨洪注,師阻水,弗能進。噶爾漢期以木落水涸時進師,上責其逗留,削前功。二十年,薄鄖城,時弘已死,其子天秘毀壘遁,遂克之。以次下萬、開、建始、梁山諸縣及忠州。二十二年,授荊州將軍。部議當楊來嘉攻房縣不能救,當奪職,上命降級留任。

二十六年,湖廣裁兵,夏逢龍倡亂。噶爾漢師次安陸,遣協領穆禮瑪等攻之,多所斬馘。進次應城,賊還竄武昌,會糧絕,戰艦不足用,疏言狀,召還,授正紅旗蒙古都統。比至都,論退縮玩寇罪,免官。後卒於家。

阿密達,他塔喇氏,滿洲正白旗人。順治間,授三等侍衛,洊擢正白旗滿洲副都統。康熙初,擢領侍衛內大臣、議政大臣。

十三年,吳三桂反,襄陽總兵楊來嘉以穀城叛應之。河北總兵蔡祿初與來嘉並為鄭成功將,先後來降。來嘉招同叛,祿具槍械,購騾馬,密令所部為備。聖祖聞狀,命阿密達率兵赴懷慶察視,祿不出迎,謀拒戰。阿密達疾馳入其廨,得祿及其孥,悉誅之。耿精忠亦叛,授阿密達揚威將軍,率滿洲兵千人駐江寧,命習水戰。尋授簡親王喇布揚威大將軍,阿密達歸將軍印,參贊軍務。

王輔臣叛,十四年,命阿密達仍佩揚威將軍印,率兵赴蘭州,佐以副都統鄂克濟哈、覺羅誇岱。時輔臣據平涼,蘭州諸路皆陷賊,大將軍貝勒洞鄂令阿密達徑攻平涼。五月,克寧州,薄平涼,戰失利,退駐涇州。洞鄂兵至,命參贊軍務,與總兵孫思克會師進攻,久不下。十五年,大學士圖海代為大將軍,阿密達參贊如故。既,奪虎山墩,俯攻城,輔臣乃降。

十七年,命赴湖南,從大將軍安親王岳樂討吳世璠。十八年,克武岡。諭阿密達與安親王計議,量撤滿洲兵,護還京師。十九年,授正白旗蒙古都統。部議平涼戰敗當奪職,上寬之,命降五級留任。尋復授領侍衛內大臣。

噶爾丹為亂,命詗賊狀。二十九年,命參贊大將軍裕親王福全軍務,出塞討噶爾丹,戰於烏闌布通,勝敵。師還,部議不能乘勝滅賊,福全以下皆有罪,當奪職。上以師有功,宥之。三十五年,上親征噶爾丹,阿密達請從征。上次克魯倫河,以阿密達暫充將軍,率留後滿洲兵及綠旗步兵赴克勒和碩,並命兼轄留屯各軍。尋撤還京師。四十八年,卒。

鄂克濟哈,納喇氏,滿洲正黃旗人。初任侍衛,署副都統兼佐領。康熙十三年,三桂反,陜西、湖廣並警。上命偕副統色格駐河南府。輔臣亂起,從阿密達赴西安剿御。尋赴蘭州參贊阿密達軍務,克涇州、寧州,詔嘉之。十八年,從圖海攻禮縣驛門,大破之。復塔什堡,進克興安。圖海以漢中要地,令鄂克濟哈領振武將軍,與副都統哈塔將千人守之。

十九年,提督趙良棟等徇四川,與將軍吳丹為後勁。瀘州陷,率師攻克之,又敗之托川雅。未幾,賊犯仁懷,吳丹擁兵不救,永寧復陷。命還漢中,而使鄂克濟哈領其眾。鄂克濟哈疏言建昌、永寧相去千餘里,未能兼顧,乃命佛尼埒專領永寧一路,而授鄂克濟哈宣威將軍,駐軍成都,專領建昌一路。二十年,建昌軍棄城走,自劾,解將軍印,以都統覺羅吉哈裡代,還守漢中。尋入為二等侍衛。三十年,遷正黃旗副都統。三十三年,授護軍統領。從征噶爾丹,事平,駐守寧夏。三十八年,卒。

覺羅吉哈里,滿洲正白旗人,武功郡王禮敦第三世孫。順治初,授牛錄額真,襲父拜他喇布勒哈番世職。遇恩詔,晉二等阿達哈哈番。累遷護軍參領、鑲黃旗滿洲副都統。康熙十二年,吳三桂反,京師奸民楊啟隆為亂,都統圖海、祖承烈及吉哈裡討平之。佐領鄂克遜擒其黨黃吉、陳益,吉哈里亦獲焦三、硃尚賢、張大、李柱、陳繼志、史國賓、王鎮邦等送法司,廉得實,論棄市。語互詳鄂克遜傳。十六年,命與副都統席布率師赴四川會鎮安將軍噶爾漢討賊,即軍前擢鑲黃旗蒙古都統。三桂孫吳世璠尚據有雲南、貴州,其將胡國柱、夏國相、馬寶等分犯瀘州、敘州、建昌。二十年,建昌陷,上解鄂克濟哈宣威將軍任,詔吉哈裡代之,統所部兵會提督趙良棟復建昌。良棟自雅州入,吉哈里為後,鏖戰大渡河,奪寇舟以濟。是時師下雲南,已合圍,國柱等亟引眾還,吉哈裡遂復建昌。將趨雲南,行至武家,疾作,卒於軍,恤如例。

拉哈達,鈕祜祿氏,滿洲鑲黃旗人,車爾格第五子。順治間,以侍衛襲其兄法固達三等阿達哈哈番世職,恩詔累進一等。授兵部督捕侍郎,擢工部尚書、議政大臣。康熙八年,授鑲黃旗滿洲都統。

十三年,吳三桂叛,授鎮東將軍,駐防兗州,甫至,而耿精忠叛,犯浙江。詔往署杭州將軍,與平南將軍賚塔、總督李之芳共籌防禦。賊窺金華,遣副都統沃申、副將陳世凱等擊卻之;復犯臺州,寧波、紹興皆騷動。上命康親王傑書為大將軍,貝子傅喇塔為寧海將軍,統師援浙,拉哈達以都統參贊軍務。十四年,擊處州賊,連下松陽、宣平。十五年,從康親王徇福建。精忠降,即導我師攻鄭錦。

時漳州、泉州、興化三府為錦所據,遣其將許耀以三萬人偪福州,拉哈達率師擊之,破其壘十四。其冬,傅喇塔卒於軍,授拉哈達寧海將軍。十六年,與賚塔合軍攻興化,克之,其將郭維籓以仙游降。耀奔泉州,復據以堅守。拉哈達率銳師宵加之,漏未盡,梯入,斬耀及諸偽官,入城撫定軍民。是時錦連敗,還廈門,泉州、漳州二府及海澄等十縣皆復,降將四百、兵四千有奇。移師略潮州,叛將劉進忠亦降,乃還守福州。

十七年,錦將劉國軒陷海澄,復犯泉州,斷萬安、江東二橋,扼長泰、同安諸隘,南北援絕,泉州幾不守。拉哈達駐漳州,詔責其不亟援海澄,趣戴罪赴泉州難。拉哈達議自長泰入,會江漲,軍阻水。侍讀學士李光地方居憂在籍,乃遣使導師出間道,自南靖道漳平趨安溪,遂薄泉州,圍乃解。國軒築壘濱海東石地,當金門、廈門道。十八年,拉哈達遣沃申攻克之。十九年,與巡撫吳興祚自同安至潯尾,分兵渡海,拉哈達居中,興祚自左,總兵王英自右,並趨廈門。賚塔與總督姚啟聖,提督萬正色、楊捷,總兵黃大來師來會,三面合擊,賊不能支,遂克廈門。復進攻金門,其將吳國俊等迎降,錦與國軒走歸臺灣。詔召康親王還京,命拉哈達與副都統馬思文守福州。

二十一年,撤滿洲兵還京,追論失守海澄罪,部議降世職為三等,並罷官,上以拉哈達從康親王平福建有勞,留都統任。二十四年,致仕。四十二年,病卒,恤如制。

察哈泰,薩克達氏,滿洲鑲紅旗人,世居寧古塔。事太宗,從伐明,屢有功。順治初,逐李自成,討金聲桓,皆在行間,屢擢太僕寺理事官,並授三等阿達哈哈番。復遷太僕寺卿、鑲紅旗滿洲副都統。從伐俄羅斯,將舟師,招降斐雅喀百二十餘戶。坐所部戰艦戰失利,奏不實,罷副都統,奪世職,專管牛錄事。

康熙三年,復授鑲紅旗蒙古副都統。以老乞休,上慰留之。尋遷護軍統領,加太子少保。十三年,從拉哈達出駐兗州。上命拉哈達赴杭州,以敕印留付察哈泰,繼為鎮東將軍。十四年,命仍以護軍統領帥所部赴荊州,聽順承郡王勒爾錦調度。十五年,三桂將陶繼智等犯宜昌,率兵駐江陵,通聲援。七月,卒於軍,恤如制。察哈泰調赴荊州,上命以鎮東將軍印授副都統布顏,統蒙古兵留駐兗州。事定,撤還京師。

根特,納喇氏,滿洲正黃旗人。父達雅里,國初來歸。從伐明,攻深州,先登,克之。軍功,累進一等參將世職。

根特早歲從戎,數立功績。從伐明,攻泗水縣、定州,並先登,賜號「巴圖魯」,授三等甲喇章京世職。順治元年,授刑部理事官。五年,金聲桓以南昌叛,從大將軍譚泰討之,薄南昌,攻未下,根特自城南以登,拔之。聲桓中矢死,擒王得仁。師還,擢梅勒額真,進一等阿達哈哈番。

六年,姜瓖以大同叛,其黨虞允、白漳、張萬全陷蒲州及臨晉、猗氏、河津。從總督孟喬芳濟河擊之,復蒲城,進徵平陽。白漳擁步騎六千至榮河迎戰,奮擊,大破之。迫黃河,賊未及濟,師薄之,賊多赴水死,遂斬白漳,餘奔吉鎮,悉殲焉。移師趨猗氏,瓖黨衛登芳依山結寨,與萬全為犄角,復分兵擊斬萬全,殲其眾。尋生得登芳,復進敗瓖黨郭中傑於聞喜。

康熙十三年,吳三桂反,命出駐兗州。尋以江西地要沖,命偕副都統席布徙守南昌。長沙陷,袁州、吉安二府與接壤,巡撫董衛國請發兵駐防,命根特自南昌移師,備戰御。尋以希爾根為定南將軍,根特參贊軍務。尚可喜疏請兵,上令根特俟希爾根兵至,率所部下廣東。耿精忠反,授根特平寇將軍,令仍返江西。副將柯升以廣信叛應精忠,破都昌,窺南康,復命根特先定廣信,與前鋒統領覺羅舒恕自袁州規長沙。是年八月,卒於軍,恤如制。

禮部尚書哈爾哈齊副定南將軍希爾根駐江西,根特卒,上命以平寇將軍印授之。十一月,命赴江寧,贊大將軍簡親王軍務,鎮江南。十五年五月,命率江寧兵赴廣東,授華善平寇將軍,道江西,命會師攻吉安。螺子山之敗,坐奪官,披甲。

華善,漢軍正白旗人。石廷柱第三子,為豫親王多鐸壻,授和碩額駙。三桂反,授安南將軍,守鎮江。尋命贊大將軍簡親王軍務,駐江寧。十五年,改授平寇將軍。十六年,簡親王進軍江西,命華善率所部從,以平寇將軍印留付江寧副都統科爾擴岱。十七年,授定南將軍,命守茶陵。三桂兵攻永興急,上命簡親王進次茶陵,而令華善救永興。華善不敢進,上切責之,解將軍印,令從穆占自效。事平,論罪,上命寬之。三十四年,卒。子石文炳,襲廷柱三等伯。累遷福州將軍。以華善老,召授正白旗漢軍都統。尋聞喪還京,卒於途。

席卜臣,瓜爾佳氏,滿洲鑲白旗人,費英東弟郎格之孫也。事太宗,從上征朝鮮。從睿親王多爾袞伐明,戰於通州,擊敗太監高起潛軍;再從攻錦州,屢戰破敵。順治初,從大軍入關擊李自成,戰於一片石,遂至慶都,敗賊於太原。二年,從英親王阿濟格徇陜西,逐自成至安陸。三年,從肅親王豪格下四川,殲張獻忠。五年,從討叛將姜瓖。敘功,屢遇恩詔,世職至二等拜他喇布勒哈番,官至護軍統領。十二年,與都統卓洛等出駐荊州,破孫可望。十六年,與安南將軍明安達里援江寧,敗鄭成功將楊文英等,斬馘甚多。康熙九年,擢鑲白旗蒙古都統。十二年,加太子少傅。

十三年,吳三桂反,上授都統赫葉為安西將軍,與西安將軍瓦爾喀等自漢中下四川。十四年,復授席卜臣鎮西將軍,與副都統巴喀、德業立同駐西安。尋又命大將軍貝勒洞鄂西討,赫葉歸將軍印,參贊軍務。是冬,席卜臣與赫葉會師攻保寧。三桂將王屏籓拒守,師屯蟠龍山,屏籓出戰,潛遣別將自他道絕流渡,撓我師,我師棄營退,席卜臣引還漢中。上命覈諸將罪,赫葉奪職,披甲自效。方軍退,佐領穆舒誓死決戰,將甲上記號付將軍,督兵奮斗。上聞,超擢正紅旗蒙古副都統,以獎其勇。

席卜臣至漢中,值王輔臣叛,棧道絕,餉不繼,引還西安。旋召還京。事定,王大臣追論蟠龍山戰敗罪,奪官,削世職。上以席卜臣有勞,免其籍沒。尋卒。

希爾根,覺爾察氏,滿洲正黃旗人,世居長白山。太宗居籓邸時,任護衛。天聰間,以軍功授牛錄章京世職。崇德元年,從伐明,連下昌平、寶坻十餘城,遷巴牙喇甲喇章京。擊敗明太監高起潛兵,擒總兵巢丕昌,又助譚泰設伏,敗三屯營騎兵。師還,敵躡後,諸將護輜重先行,希爾根殿,超授一等甲喇章京世職。二年,從克皮島。將行圍,選扈從,其父雅賴與焉。希爾根向睿親王多爾袞乞免,不許,紿以珠爾堪代之。事覺,坐欺罔,應罷官奪世職,從寬論罰鍰。從師圍錦州,壁山岡,明兵至,擊走之,並擊退松山援兵。復坐擅離軍伍、言事不實,停敘功。七年,師圍薊州,明總兵白騰蛟率師馳救,希爾根擊敗之。

順治二年,從英親王阿濟格討李自成,圍延安,大敗其眾。其將有一隻虎者,稱驍果,數犯我師。希爾根三戰皆克,遂至西安。自成奔湖廣,逐北至安陸,賊據城拒戰,復與鼇拜攻克之,獲戰艦八十艘。引兵武昌,賊又集艦五百浮江將東下,譚泰率眾往取,希爾根先至,獲之,進三等梅勒章京。三年,從肅親王豪格征張獻忠,與哈寧阿、阿爾津、蘇拜敗之西充。別趨涪州,討賊袁韜,斬虜多。尋坐哈寧阿陷重圍不救,復與阿爾津等爭功,論棄市,詔改罰贖,降三等甲喇章京。

六年,姜瓖叛據大同,希爾根從巽親王滿達海討之,圍太谷,以砲破其城,斬瓖所署置知縣李成沛、都司吳汝器,進克大同。以次復長子縣,渾源、朔二州。永寧州、嵐縣、潞安府並降。又與漢岱攻復遼州。山西平,當進秩,因訴前鐫秩冤,累遇恩詔,進一等阿達哈哈番。九年,擢巴牙喇纛章京,列內大臣。十二年,加太子太保。

十三年,耿精忠叛,使其將白顯忠寇廣信、建昌、撫州,授希爾根定南將軍,率師援江西,以桑格贊軍事,沃赫、伊巴罕從,次南昌,而三城已陷。是時安親王岳樂駐師省城,檄希爾根先取撫州,賊出拒,連敗之,並率沙納哈擊走援賊,城賊待援不至,棄城走。精忠將陳升構土賊郭應定等犯贛州,令副都統甘度海御之,大捷。追至龍泉,破三壘,復攻取曹林十餘寨。十四年,擊敗精忠將邵連登,復建昌。移師饒州,擊退餘干、浮梁、樂平諸縣賊。會岳樂師下湖南,命簡親王喇布赴南昌,以希爾根副之。三桂將高大節出醴陵、萍鄉,陷吉安,冀斷岳樂軍後路。我師屯螺子山,大節勇,常以少騎奔我師。喇布倉皇棄營走,希爾根從之,賊入壘,縱飲飽掠而歸。俄大節死,希爾根督師攻圍,戰又弗勝。逾歲賊遁,詔仍駐南昌。尋以老召還。十八年,卒。

子喀西泰,任護軍參領。從征四川,攻保寧,死蟠龍山之戰。

論曰:當三籓亂時,命將四出,以庶姓授大將軍,惟圖海與賚塔二人而已。賚塔自廣西,穆占自湖南,皆轉戰下雲南,削平巨憝,功最多。穆占功歸彰泰,故賞不逮;賚塔、莽依圖功與相並,惜中道先卒。佛尼埒等皆夙將,有戰績。其時雜號將軍,或出朝命,或即軍前除拜。有一人遞掌二三印者,有一印迭授二三人者,皆領異軍獨當一路。綜而觀之,當日行師應敵之大概,可以得其要矣。


\end{pinyinscope}