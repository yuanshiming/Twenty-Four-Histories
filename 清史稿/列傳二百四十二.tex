\article{列傳二百四十二}

\begin{pinyinscope}
董福祥張俊夏辛酉金運昌黃萬鵬餘虎恩桂錫楨方友升

董福祥,字星五,甘肅固原人。同治初,回亂作,鳳、邠、汧、瀧寇氛殆遍。福祥亦起安化,與其州人張俊、李雙良蹂躪陜、甘十數州縣,竊據花馬池,犯綏德,窺榆林,潰勇、饑民附之,眾常十餘萬。嗣為劉松山所敗,其父世猷降,福祥亦率眾乞歸款。乃簡其精銳者,編為董字三營:福祥居中營,俊居左,雙良居右。從攻金積堡,福祥襲卡後,被創不少卻,破其禮拜寺。頓板橋,寇來爭,與蕭章開夾擊敗之,金積堡平,超授都司。十一年,從劉錦棠至碾伯,趨峽口,與陜回禹得彥、雀三大戰,破之。進擊白彥虎於高家堡,焚其壘而還。已而偽知府高桂源構彥虎圍西寧,撲雙良營,福祥又大敗之,圍解,遷游擊。徙守向陽堡城,復討平河州叛回,積功至提督。

光緒元年,從出關,戰天山,會大風晝晦,吏士弗敢進。福祥率眾先登,一鼓殲之,又破之木壘河、古牧地,進復烏魯木齊諸城及瑪納斯南城。是時彥虎猶據開都河西岸,覬入俄。福祥自阿哈布拉緣塗置哨壘,至曲惠而營,士卒儲薪草,濬井泉,以俟錦棠軍至,破之,復喀喇沙爾。是冬,克和闐,南疆西四城告寧。繇是董軍名震西域。論功,賞黃馬褂、世職,賜號阿爾杭阿巴圖魯。

安夷既就撫,布魯特酋阿布都勒哈誘之,復入寇色勒庫爾,北走庫倫,福祥馳之,抵空谷根滿,步卒足重繭,乃遴健者乘騾隊,從騎旅及之木吉。寇方解鞍秣馬,驚起,依山而陣,俊敗之,福祥縱兵搜捕,復斬三百餘騎。自此寇不敢犯邊。授阿克蘇總兵,駐防喀城。未幾,而所部索餉譁變,戕營官胡登花,或請擊之,福祥曰:「營勇與叛勇有約,如昏夜響應,將奈何?不如閉城守,彼勢孤必自斃也。」越三日,悉為兵民擒獻,乃分別誅宥之。事定,領俊及夏辛酉移駐葉爾羌、和闐。

十六年,擢喀什噶爾提督。二十年,加尚書銜。會德攘膠澳,命率甘軍入衛。明年,西寧、碾伯又告警,督師還抵狄道。河州馬永林叛,渡洮戰卻之,連破高家集、三甲集,道始通。事寧,調甘肅。福祥自請援西寧。又明年,克上下五莊,乘勝復大通、多巴。朝命駐西寧專剿撫,以魏光燾二十七營屬之。會巴燕戎格、劉四復奔關外,福祥亟遣騎踵之,拔卡爾岡,先後夷海城、冶諸麻、甘州南山寇堡,關內外及青海悉平,加太子少保。

二十三年,入覲,命領武衛後軍,召對,福祥曰:「臣無他能,唯能殺外人耳。」榮祿頗信仗之。拳亂起,日本書記杉山彬出永定門,福祥兵殺之。於是董軍圍東交民巷,攻月餘不下。敵兵自廣渠門入,福祥走彰儀門,縱兵大掠而西。兩宮西幸,充隨扈大臣。和議成,外人堅欲誅福祥。李鴻章曰:「彼綰西陲軍寄久,慮激回變,當緩圖之。」乃褫職錮於家。

榮祿在西安綜大政,福祥移書讓之,略謂:「辱隸麾旌,任公指使,命攻使館,祥猶以殺使臣為疑。公言『僇力攘外,禍福同之』。祥本武夫,恃公在上,故敢效奔走。今公執政而祥被罪,祥死不足恤,如軍士憤懣何!」榮祿得書,置不答。三十四年,卒。其子天純,輸銀四十萬濟帑復官。

俊,字傑三。金積堡之役,與福祥並授都司。規西寧,餘虎恩困峽口,俊力戰解之。連破小峽、潤家溝,從攻河州、肅州,以戰功歷遷至副將,賜號倭興巴圖魯。光緒初,從征西陲,復烏魯木齊,擢總兵。錦棠令入關募軍,於是成定遠三營。先後從克東西四城,晉提督。安夷復叛,俊倡議主剿,眾論譁起,錦棠獨韙之。寇竄庫倫,俊追至木吉,分三路入,戰良久,手刃執紅旗悍卒,寇愕走。進至卡拉阿提,會日已入,止舍。天未曙,整軍復進,日午及之。寇不能反拒,槍矛所至,尸相填藉。抵黑子拉提、達阪,止餘數十騎,逾山入俄境,不復追。是役,四晝夜馳八百餘里,凡擒愛伊德爾呼里二人,安夷所謂「大通哈」也,胖色提以下數十人,猶華言「營官」。賜頭品服、黃馬褂,授西寧鎮總兵,調伊犁。二十一年,代福祥為喀什噶爾提督。尋還甘肅。二十五年,入都,充武衛全軍翼長,兼統中軍。逾年卒,謚壯勤,予建祠。俊好舞刀,所部衣幟皆白色,時稱「雪張」云。

辛酉,字庚堂,籍山東鄆城。初從僧格林沁討捻。宗棠西征,從討陜回,積勛至守備。攻金積,裹創力戰,稱驍果。規肅州,充前鋒,拔塔爾灣、黃草壩,關內大定。數遷至游擊。師出關,下阜康,襲黃田,破古牧,無役不從。進規南路,攻托克遜。彥虎子小虎殊死守,師行不得志,獨辛酉率游軍數戰,略有斬擒。達阪之役,與餘虎恩輕騎先涉,列城左山岡。比回覺,悉力轟拒,師少卻。辛酉斬先退者數人,乃止,卒大破之。遷副將,賜號振勇巴圖魯。

從錦棠復庫車,至拜城,履冰抵上銅廠。回出蕩,辛酉躍馬徑前,生擒貂衣賊一人。回驚走,遂下阿克蘇。是時,帕夏奔葉爾羌,彥虎奔烏什。錦棠專力討彥虎,令俊進擊,辛酉自西會之。濟胡馬納克河,行戈壁八十里,破寇什城東,城拔,擢總兵,易勇號霍伽春。南疆平,賞黃馬褂。逸寇犯三臺,辛酉隱勁騎沖殼罕山,誘之出。伏起,短兵接,斬其酋賽屹塔黑振江。俄而安酋阿里達什寇邊,從錦棠出屯玉都巴什。辛酉率二百騎為前驅,怒馬陷陣,斬執旗賊,奪其旗以歸。寇大潰,追至畢勒套格,殺其黨且盡。西陲告寧,乞歸養。甲午之役,率師鎮登州,即於軍前授廣西右江鎮,治軍如故。尋徙鎮登州。拳亂作,充武衛軍先鋒左翼長,從李秉衡禦敵,未戰而潰。後除云南提督,未到官,卒,恤如制。

金運昌,字景亭,安徽盱眙人。少孤,遭寇亂,總兵郭寶昌之母曹氏撫之,從姓郭。既長,入貲為守備。從寶昌徵發、捻,積勛至游擊。論河防功,賜號勉勇巴圖魯。平畿南,擢總兵,晉勇號鏗僧額。西捻平,遷提督,復姓金氏。從寶昌卓勝軍還陜。同治八年,寶昌創發,運昌代領其眾,調防綏德。

時湘軍已剿金積堡,運昌自清澗至,分所部略其西北,毀長墻。馬化隆勢蹙,遣黨擾北山,冀斷湘軍糧運。一自河西道葉升堡,屬劉松山;一自山西道花定,屬運昌:並達靈州。回既陷定邊,運昌所部多南人,雜食青稞、高粱,患腹病。左宗棠調寶昌來援,以河防不能赴。是時,陜回陳林、禹彥祿等十三營,益以本地土回,號稱十餘萬,卓勝軍孤立其間,幾無日不戰。

明年正月,軍益饑疲,至殺馬為食。回且決渠灌我壘,會風濤大作,運昌晝夜立水中,激厲將士,列椿囊土御之,回不得逞。適劉錦棠等越渠橫出,回大潰。因議夾河築壘護餉道,兩日壘成,回至,運昌戰卻之。湘軍開溝築堤以防水,運昌壁近棗園。冰忽解,回乃憑秦渠設卡,運昌越渠擊之,回收入堡。越二日,堡回悉眾出,騎寇趨板橋,步寇決渠水南下。運昌軍阻水,錦棠分三路泅水與合,並力轟潰其眾。未幾,回復運磚石築卡於北,環以長堤,欲引馬連水以困我。運昌亟令軍士攜鋤錨,夷其卡而還。四月,陳林率眾出花定掠食。運昌使提督王鳳鳴禦之,敗之磚井鎮。同時葉升堡道亦通,軍威復振。七月,克馬家兩寨。值新麥熟,運昌與錦棠分刈,並糜粟割之。回來爭,輒敗走。遂築壘蔡家橋。橋跨秦渠,內設卡,外障水,馬化隆前所為阻遏官軍者也。至是決水反灌,破壘三、卡十有一,乘勢下秦壩關,逼東關。議掘壕築墻久困之,與錦棠分段興工。三日畢乃事,遣兵分守之,遂合金積圍。日咯血數次,戰不少休。陳林降,運昌以西林、河州未下,宜稍示寬大。強者編籍,弱者就糧,群回多乞款。馬化隆勢蹙,亦束首歸命。於是寧、靈悉定,論功賞黃馬褂。

駐纏金,平甘回馬勝福亂,晉頭品服。徙駐包頭,數請於宗棠,願西征。光緒二年,宗棠請敕淮勇出關助剿,報可。明年夏,行抵烏魯木齊,命署提督,越二年,實授。口外經喪亂後,戶口減耗。運昌興水利,課農桑,建橋梁,皆割俸自任之。其斥巨款賑畿菑,實秉義母郭曹氏命。李鴻章為請於朝,特建坊旌異之。十一年,謝病歸。逾年卒,恤如制,入祀卓勝軍昭忠祠。妾王氏、馬氏、張氏,先後仰藥殉節,皆獲旌。

黃萬鵬,字搏九,湖南寧鄉人,本籍善化。初從曾國荃援贛、皖,積勛至都司。從克江寧,歷遷總兵,賜號力勇巴圖魯。捻入鄂,犯德安,萬鵬馳救,大敗之,又破之安陸。會師新洲,於是夾擊,大破虜,擢提督。

左宗棠西征,調赴陜,署漢中鎮總兵。同治十一年,從攻西寧,抵碾伯,戰硤口,回潰走,圍解。明年,從劉錦棠克向陽堡,進圍大通,降之。選降眾立旌善五旗,馬隊屬萬鵬領之,隨攻肅州。事寧,賜頭品服。十三年,河州閃殿臣復叛,萬鵬率崔偉等進擊,敗之城南二十里鋪。寇竄賈家集,官軍攻弗克,萬鵬從姚家嶺馳下合攻,燔其堡,更勇號為伯奇。

光緒二年,出關。時土回馬明據古牧,白彥虎聞官軍至,自紅廟子與合師,夜襲黃田。旦日,聞古牧角聲起,萬鵬與餘虎恩馳擊寇騎卻之,語詳餘虎恩傳。烏魯木齊諸城既復,追至池墩而還。捷入,賞黃馬褂。北路略定,逸寇多亡匿東南山谷。萬鵬復與虎恩取道大小鹽池墩至柴窩,略有斬擒。八月,金順攻瑪納斯南城弗勝,錦棠檄萬鵬助擊。掘隧以攻,寇死拒,矢貫萬鵬臂,拔之,更疾戰,與諸軍大破之。

三年,攻克達阪,乘勝取托克遜。至小草湖,遇伏,圍萬鵬數匝。萬鵬率隊蕩決,所向披靡。錦棠軍繼至,寇大潰,詔予雲騎尉世職。是役,帕夏知不免,飲藥死,彥虎遂奔開都河西岸。七月。師至曲惠,錦棠自向開都河,而令萬鵬道烏沙塔拉傍博斯騰淖爾西行,出庫爾勒之背。彥虎懾軍威,已先期遁。詗知脅纏回走布古爾,亟行四百里追及之,戰良久,大敗其眾。九月,馳抵托和奈,再敗之,收庫車,進駐拜城。履冰夜行至銅廠,諸軍直搏之,寇愕走。

萬鵬長驅察爾齊克臺西,斬數千級。越二日,夜抵扎木臺稍憩,即引兵阿克蘇城。未至城數里,見西南塵埃坌起,會諜報彥虎走烏什,嗾安集延走葉爾羌糸圭追師。錦棠乃舍安夷,而令萬鵬專追彥虎,阻河漲不能濟。時彥虎止隔河十里許造飯,掩襲可擒也,而我師遽返,錦棠大怒,責令復進。於是萬鵬渡胡馬納克河,行戈壁八十里,獲其後隊馬有才,進拔烏什,而彥虎已走喀什矣。東四城俱下,詔改騎都尉世職。

當是時,伯克胡裏據喀什攻漢城,彥虎至,助之,勢益張。守備何步雲告急,錦棠檄萬鵬道布魯特與虎恩期會喀城。萬鵬倍道應赴,緣雪山千餘里,每以毯鋪地濟師。十一月,抵城北麻古木,虎恩亦抵城東牌素特。寇詗騎馳歸,曰:「大軍至矣!」於是二巨酋走回城北,進搗之,則又宵遁。萬鵬向西北追彥虎,至爰岌槽,與賊後隊遇,生擒偽元帥馬元,斬其副白彥龍。次日,追至恰哈瑪納,為布魯特人所阻,彥虎遂奔俄。新疆平,改授二等輕車都尉。

四年,凱旋,乞歸省。越二年,仍赴新疆治軍。南北山邊防敉平,晉頭品秩。歷權喀什回城協副將,阿克蘇、巴里坤各鎮總兵,新疆提督。又襲其叔登和世職,並為二等男爵。二十四年,徵入京,創發,卒於道。予建祠。子鉞,道員,襲爵。

餘虎恩,湖南平江人。少孤貧,喜讀書。初從曾國籓討粵寇,積勛至副將。同治初,從劉松山征捻,蹙之沙河西,擢總兵,賜號精勇巴圖魯。張總愚與回匪合,攻破金谷、銀渠,又敗之郿縣,晉提督。寇自宜川渡河,陷山西州縣,又從劉軍追復之,易勇號奇車博。軍獲鹿,適郭松林被圍,虎恩銳身馳救,圍解。繞道長驅,騎寇雖猋騁,遇戰輒披靡。上念陜事棘,命左宗棠舉將才,乃薦松山部將尤異者十數人,虎恩與焉,寵以頭品服,令赴陜軍。靈州既克,松山進兵板橋、蔡家橋。有頃,回敗走,虎恩騎旅突之,驟若風雨,回不得歸,下其村寨三十餘。金積平,假歸。

十一年,命募軍赴甘。劉錦棠攻西寧,虎恩率軍至陜口,周覽形勢。寇出拒,被困,卒擊卻之。錦棠覘回勢盛,赴平戎驛造橋濟湟,自督師築壘北岸,令虎恩築南岸。未成,馬營灣寇突至,虎恩轟擊之,錦棠亦敗湟北寇,於是西寧告寧。論功,賞黃馬褂。隨攻肅州,軍南門,與諸將討平之,除陜安鎮總兵。

光緒二年,從出關。宗棠慮戈壁糧運艱阻,虎恩請身任之,乃絕幕而西。抵哈密,取餘糧,逾天山,遞送巴里坤古城。邊既實,襲黃田,破其卡。忽古牧寇壓師而陣,虎恩亟自山馳下,與寇騎戰良久,會董福祥軍助擊中路,寇大潰,遂合圍。帕夏遣悍黨來援,虎恩率騎旅列山岡,嚴陣以待。復麾軍截其歸路,斬關直入,城拔。度烏垣,寇且他遁,以次下烏魯木齊、迪化及偽王城,予雲騎尉世職。

明年,逾嶺而南,從錦棠趨柴窩,去達阪二十里。夜初鼓,虎恩率騎旅九營,銜枚疾走。大通哈引湖水衛城,泥深及骭。虎恩所部掠淖進,依山為陣,斬寇諜十餘騎,回方臥,未覺也,平旦始大驚,悉眾出,據險轟拒。師屹立不動,海古拉援至,虎恩又截之隘口,援騎返奔,追逐數里,斬百餘級。虎恩策城回盼援不至,必遁,預隱兵以待,寇出悉就擒。達阪復,乘勢下托克遜,予騎都尉世職。

逾月,規南路,師次曲惠,虎恩取道烏沙塔拉入庫爾勒城,地闃無人,食且盡,乃掘窖糧數萬石濟師。遂與諸將下庫車,凡六日馳九百里。已而喀什噶爾告急,錦棠令虎恩自巴爾楚取中路為正兵,黃萬鵬自烏什道布魯特為奇兵,仍歸虎恩節度。師抵巴爾楚,會天寒,冰雪凝冽,而喀城警報且日至。乃兼程應赴,軍士人人自奮,各以俘白酋取首功為利。日中,虎恩至城東牌素特,夜半時抵喀什漢城下,左右止百餘騎從,乃整兵以俟。平明,步兵至,寇騎開城出蕩,虎恩率眾大戰,刺殺回酋王元林,會萬鵬亦至,復其城。虎恩西追伯克胡里,令桂錫楨率騎旅自間道疾馳,而自率步旅繼之,前後夾擊,生擒餘小虎、馬元於陣。繼復獲金相印父子,相印者,引安集延侵占南路也。於是新疆南路平。降敕褒嘉,改予一等輕車都尉。歷駐烏什、葉爾羌、和闐,赴本官。

十一年,謝病歸。越六年,出統湘軍,駐岳州,以能捕盜名,並二世職為二等男爵。二十六年,授喀什噶爾提督,未上,留統武衛中軍十營。拳亂起,諸將多崇奉之,獨虎恩則否。福祥攻使館,虎恩與論事榮祿前,謂遍觀諸軍,實不足敵外人。福祥大怒,欲殺虎恩,榮祿以身翼蔽之,乃免。令出防獲鹿,未幾,仍還湘。三十一年,創發,卒於家。恤如制,附祀宗棠、松山祠。

桂錫楨,山東曲阜人。從軍討捻,數遷至游擊。咸豐十一年,張總愚領餘眾與陳大喜合,勢張甚。錫楨追至河間,裹創力戰,寇大敗,錫楨名始著。同治七年,從左宗棠征陜回,數獲勝,檄守同官。明年,提督高連升屯宜君,親兵丁玉龍匪首也,構回為亂,夜圍營帳,戕統帥。錫楨聞警,亟自同官馳援,擊卻之,復追剿金鎖關、米子窯。會楚軍將丁賢發等至,拘玉龍誅之,城獲全,遷參將。從克固原三營,斬其酋楊文治,進扼中衛四百戶。回酋馬光明自固原東北入同心城,復大敗之。又降元城回海生春。

十年,規河州,錫楨自中衛、靖遠南搜會寧游匪。亡何,肅州降回叛,甘、涼戒嚴,錫楨遂還軍肅州。明年,略東關,克其大卡一,寇出拒,擊退之。先是肅回敗,倚硃家堡、黃草壩、塔爾灣、文殊山各堡,互犄角,誓死守,徐占彪攻弗克,至是誘之出,錫楨隱卒深林,俟寇過半,橫截而出,諸軍繼之,寇大潰。肅州西南墩堡悉平。進屯沙子壩,去肅城三里,肅回啟南門出蕩,錫楨率騎旅突陣,回奔入城。諸軍沖殺,連克四壩、十一堡,東面寇壘亦盡,賜號精勇巴圖魯。

十二年,從出關,錫楨率四百騎歸額爾慶額節度,進古城。光緒二年,攻阜康。宗棠慮寇北竄,令錫楨扼沙山、馬橋備要擊。尋會諸軍復烏魯木齊,北路略定,唯瑪納斯南城久未下。宗棠以劉錦棠軍單,檄錫楨助擊,與諸軍轟潰之,斬其酋韓金農,更勇號業普肯,擢總兵。進規南路,三年,從錦棠攻克達阪,乘勝復吐魯番,晉提督。規喀什噶爾,錫楨統馬步二千餘人,自阿克蘇取道巴爾楚,克瑪納巴什,直抵喀城東牌素特。會黃萬鵬軍亦至麻古木,彥虎與伯克胡里潰逃,遂復西四城。論功,賞黃馬褂。回疆告寧,晉頭品秩,加賜呢鏗額勇號。五年,乞歸葬親,道陜,創發,逾歲卒。宣統改元,巡撫恩壽狀其績以上,予優恤。

方友升,湖南長沙人。咸豐末,從軍剿川寇,積勛至守備。克太平,力戰,砲彈洞■L6,繇是以饒勇名。同治七年,討陜回,克鳳翔、岐山,嘗從行,有所沖陷。入關隴,隸劉松山麾下。金積堡之役,師失利。會友升購馬張家口,逮歸,無收馬者,或勸之去,弗聽。驅馬數千入左宗棠軍,宗棠大器之,乃編所購馬為西征靖營隊,囑領之,遣赴河州,攻剿三甲集、太子寺。

十一年,會攻肅州,其關城阻壕,壕深三四丈,古所謂酒泉也。徐占彪乘回懈,逾壕進攻。友升先登,諸軍蟻附上,奪東關,回入城死守。占彪築城南二卡,回來爭,友升率騎旅下馬巷戰,彈貫脛及脊骨,創甚,猶大呼殺賊,水漿不入者七日,眾感奮,克之。捷入,晉副將,賜號哈豐阿巴圖魯。十三年,從克巴燕戎格,署鎮夷營游擊,自是別為一軍,幟色黃。每戰從後擊殺,當者輒靡,寇見黃旗隊,輒相戒無犯雲。

光緒改元,關隴平,擢總兵,從劉錦棠出關。三年,攻克達阪、托克遜,進復吐魯番,晉提督。以次下阿克蘇、烏什、庫車及天山南北二路。論功,賞黃馬褂,賜頭品秩。五年,安集延、布魯特寇邊,徑抵烏魯克恰提。友升先進烏帕爾覘虛實。越數日,報寇騎已入烏帕拉特。獲寇諜,訊知其乘夜襲營,諸軍備往擊,大破之。友升與楊金龍分左右入,軍士皆奮迅超躍,寇不能成陣,還遁俄羅斯。八年,入關,遭母憂歸。

十年,法越事起,出頓憑祥,進攻文淵,陷重圍,彈傷手斷筋,親軍五百止存二十七騎,卒潰圍出。朝廷宥其敗,且嘉勞之。復諒山、長慶,予世職,除廣東南韶連鎮總兵。十三年,入覲,上視傷痕,為惻然。尋解任。中日失和,領三千人北上,守山海關。拳亂作,復率師入衛,駐山西固關。二十七年,調浙江衢州鎮。設講武堂,以新法訓練其眾,並修復柘水故道,民德之。三十二年,創發,卒,恤如制。

論曰:初討陜回,福祥以降軍效力,名震西域,何其悍也!運昌統卓勝軍,萬鵬領旌善營,與虎恩、錫楨、友升轉戰寧、肅,皆以驍勇名,各著奇績,其勇略亦有可傳焉。其後福祥終以驕妄敗,助亂啟釁,竟免顯戮,豈非幸歟?


\end{pinyinscope}