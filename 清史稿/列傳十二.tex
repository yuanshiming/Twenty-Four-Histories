\article{列傳十二}

\begin{pinyinscope}
額亦都費英東子索海孫倭黑何和禮子多積禮和碩圖都類

安費揚古扈爾漢

額亦都,鈕祜祿氏,世居長白山。以貲雄鄉里。祖阿陵阿拜顏,移居英崿峪。父都陵阿巴圖魯。歲壬戌,額亦都生。幼時,父母為仇家所殺,匿鄰村以免。年十三,手刃其仇。有姑嫁嘉木瑚寨長穆通阿,往依焉。穆通阿子哈思護,長額亦都二歲,相得甚懽。居數歲,庚辰,太祖行經嘉木瑚寨,宿穆通阿家。額亦都與太祖語,心知非常人,遂請從,其姑止之,額亦都曰:「大丈夫生世間,能以碌碌終乎?此行任所之,誓不貽姑憂。」翌日,遂從太祖行。是歲太祖年二十二,額亦都年十九。太祖為族人所惎,數見侵侮,矢及於戶,額亦都護左右,卒弭其難。

居三年,歲癸未,太祖起兵,額亦都從,討尼堪外蘭,攻圖倫城,先登;攻色克濟城,掩敵無備,取之,獲其牛馬、甲士;又別將兵攻舒勒克布占,克其城。額亦都驍果善戰,所向克捷,太祖知其能,日見信任。歲丁亥八月,令將兵取巴爾?挽強弓十石,能以少擊達城。至渾河,秋水方至,不能涉,以繩約軍士,魚貫而渡,夜薄其城,率驍卒先登,城兵驚起拒,跨堞而戰,飛矢貫股著於堞,揮刀斷矢,戰益力,被五十餘創,不退,卒拔其城。師還,太祖迎於郊,燕勞,其所俘獲悉畀之,號為「巴圖魯」。薩克察來攻,額亦都率數卒出御,為所敗;夜入其城,進攻克尼瑪蘭、章家二城,索爾瑚寨。師還,太祖迎勞如初。界籓有科什者,以勇聞,盜九馬以遁,額亦都單騎追斬之,盡返所盜馬。嘉木瑚人貝揮巴顏謀叛附哈達,太祖命額亦都討之,誅其父子五人以徇。

歲癸巳九月,葉赫等九部合師來侵,攻我黑濟格城,太祖親御之,陣於古勒山。令額來犯,奮擊,殪九人,敵卻,我師乘之,擒葉赫貝勒布寨。九部師?亦都以百騎挑戰,敵悉皆潰,遂乘勝略諾賽寨及兆佳村。有齊法罕者,戰沒,額亦都直入敵陣,以其尸還。訥殷路守佛多和山自固。太祖命額亦都?者,九部之一也,其長搜穩塞克什,既敗歸,復聚七寨之偕噶蓋、安費揚古,以兵千人圍其寨,克之,斬搜穩塞克什,太祖以所乘馬賜之。歲己亥秋,從征哈達,滅之。

歲丁未五月,從貝勒巴雅喇等伐東海渥集部,取赫席黑、俄漠和蘇魯、佛訥赫拖克索等三路,俘二千人。九月,從征輝發,滅之。歲庚戌十一月,太祖命將兵千,撫渥集部那木都魯、綏分、寧古塔、尼瑪察四路,降其長康古禮等十九人。旋乘勝取雅攬路,俘萬人。歲辛亥,太祖命偕何和禮、扈爾漢將兵二千伐渥集部虎爾哈路,圍札庫塔城三日,招之不下,遂攻克其城,斬千級,俘二千人。環近各路悉降,令其長土勒伸、額勒伸護其民五百戶以還。歲癸丑,從征烏拉,滅之。

歲乙卯,定旗制,額亦都隸滿洲鑲黃旗。天命建元,置五大臣,以命額亦都,國語謂之「達拉哈轄」。二年,命偕安費揚古攻明馬根單、花豹沖、三岔兒諸堡,皆克之。四年,明經略楊鎬大舉來侵,總兵杜松軍自撫順入。三月甲申朔,諸貝勒帥師出御。日過午,師至太蘭岡,大貝勒代善以太祖未至,議駐軍以俟。太宗時號四貝勒,謂:「界籓有我築城夫役,宜急護之!何為次,且示弱?」額亦都大言曰:「四貝勒之言是也!」師遂進。師至界籓,築城夫役騰躍下山赴戰,太祖亦至,指揮夾擊,松軍遂覆,還破馬林於尚間崖、劉綎於阿布達里岡,額亦都並為軍鋒。

。每克敵受賜,輒散給將士之有功者,不?太祖有所征討,額亦都皆在行間,未嘗挫以自私。太祖厚遇之,始妻以族妹,後以和碩公主降焉。

額亦都次子達啟,少材武,太祖育於宮中,長使尚皇女。達啟怙寵而驕,遇諸皇子無皆愕。額亦都抽刃而言?禮,額亦都患之。一日,集諸子宴別墅,酒行,忽起,命執達啟,曰:「天下安有父殺子者?顧此子傲慢,及今不治,他日必負國敗門戶,不從者血此刃!」乃懼,引達啟入室,以被覆殺之。額亦都詣太祖謝,太祖驚惋久之,乃嗟嘆,謂額亦都為?國深慮,不可及也。

累官至左翼總兵官、一等大臣,給以百人廩食,食三世。分所部為世管牛錄三,分隸鑲黃、正白二旗。六年,克遼陽,賜第一區。六月,卒,年六十,太祖臨哭者三。天聰元年,追封弘毅公。崇德初,配享太廟。順治十一年,世祖命立碑旌功,親為制文,詳著其戰閥,以為「忠勇忘身,有始有卒,開拓疆土,厥積懋焉」。

額亦都子十六人,其知名者,徹爾格、圖爾格、伊爾登、超哈爾、遏必隆,皆自有傳。四子韓代,五子阿達海,及阿達海之子阿哈尼堪,並以從征戰死。七子謨海,蚤歲從軍,屢立戰功,仕至都統,亦戰死。十五子索渾,從太宗戰伐有功,授世管牛錄額真,累遷至議政大臣。

額亦都初授一等總兵官,康熙間改襲一等精奇尼哈番,乾隆元年改一等子。圖爾格別封公爵,以其從孫阿里袞及阿里袞子豐升額父子相繼有功,進一等果毅繼勇公。高宗諭:「額亦都後已進一等公,其初封子爵仍紹封如故。」

費英東,瓜爾佳氏,蘇完部人。父索爾果,為部長。太祖起兵之六年,歲戊子,索爾果率所部五百戶來歸。費英東時年二十有五,善射,引強弓十餘石。忠直敢言,太祖使佐理政事,授一等大臣,以皇長子臺吉褚英女妻焉。兌沁巴顏者,費英東女兄之夫也,有逆謀,費英東擒而誅之。旋授扎爾固齊,扎爾固齊職聽訟治民。

以歸。歲戊戌正月,太祖?太祖命費英東伐瓦爾喀部,取噶嘉路,殺其酋阿球,降其命費英東從臺吉褚英、巴雅喇,伐瓦爾喀部安褚拉庫路,將兵千,克屯寨二十餘,收所屬村落。歲己亥秋九月,哈達、葉赫二部構兵,哈達貝勒孟格布祿乞援於太祖,太祖命費英東及噶蓋將兵二千戍哈達;既而貳於明,費英東等以其謀聞,哈達以是亡。

歲丁未春正月,瓦爾喀部蜚悠城長策穆特黑請徙所部屬太祖,太祖命費英東從貝勒舒爾哈齊等將兵三千以往,收環城居民五百戶,分兵三百授扈爾漢,使護之先行。烏喇貝勒布占泰發兵萬人要諸途,費英東從諸貝勒督後軍至,大敗烏喇兵。夏五月,太祖命費英東從貝勒巴雅喇伐渥集部,略赫席黑等路,俘二千人以還。歲辛亥秋七月,渥集部烏爾古宸、木倫二路掠他路太祖所賜甲,太祖命費英東從臺吉阿巴泰將千人討之,俘千餘人以還。歲癸丑,從太祖伐烏喇,滅之。

歲乙卯,太祖將建號,設八旗,命費英東隸鑲黃旗,為左翼固山額真;置五大臣輔政,以命費英東,仍領一等大臣、扎爾固齊如故。明年歲丙辰,太祖遂建國,改元天命。三年,始用兵於明,費英東從攻撫順。明總兵張承廕以萬騎來援,據險而陣,火器競發。費英東馬驚旁逸,諸軍為之卻,費英東旋馬大呼,麾諸軍並進,遂破之。太祖嘆曰:「此真萬人敵也!」四年,明大舉來侵,分道深入。明總兵杜松屯薩爾滸山巔,費英東所部屬左翼,合諸旗奮擊破之,松戰死,明師以是沮敗。秋八月,太祖伐葉赫,費英東從,薄其城,城人飛石投火。太祖命且退,費英東曰:「我兵已薄城,安可退也?」又命之,費英東曰:「城垂克,必毋退!」遂拔其城。太宗諭金臺石降,費英東在側,相與詰責,卒獲金臺石,葉赫以是破。

費英東事太祖,轉戰,每遇敵,身先士卒,戰必勝,攻必克,摧鋒陷陣,當者輒披靡;國事有闕失,輒強諫,毅然不稍撓:佐太祖成帝業,功最高。五年春三月,太祖定武功爵,授費英東三等總兵官。是月,費英東卒,年五十有七。方疾革,日向西,雲起,有聲鏗鍧,雷電雨雹交至,不移時而霽。太祖將臨喪,諸貝勒以日晏諫,太祖曰:「吾股肱大臣,與同休戚,今先彫喪,吾能無悲乎?」遂往,哭之慟,至夜分始還。秋九月,太祖祭貝勒穆爾哈齊墓,出郊,因至費英東墓,躬奠酒者三,泣數行下。

天聰六年,太宗命追封直義公。崇德元年,始建太廟,以費英東配享。太宗嘗諭?臣曰:「費英東見人不善,必先自斥責而後劾之;見人之善,必先自?勸而後舉之:被劾者無怨言,被舉者亦無驕色。朕未聞諸臣以善惡直奏如斯人者也!」順治十六年,世祖詔曰:「費英東事太祖,參贊廟謨,恢擴疆土,為開創佐命第一功臣。延世之賞,勿稱其勛,命進爵為三等公。」康熙九年,聖祖親為文勒碑墓道,稱其功冠諸臣,為一代元勛。雍正九年,世宗命加封號曰信勇。乾隆四十三年,高宗復命進爵為一等公。費英東子十,圖賴自有傳。

索海,費英東第六子,襲總兵官。旋坐事,奪職。太宗天聰五年,初置六部,授刑部承政。七年,與兵部承政車爾格偵明邊,至錦州,有所俘馘,命管牛錄事。崇德三年,更定部院官制,改都察院左參政。十月,從太宗伐明,略大凌河,下屯堡十四,復授刑部承政。

四年,索倫部博木博果爾等降而復叛,命索海及工部承政薩木什喀帥師往討之,克雅克薩、兀庫爾二城。進攻鐸陳城,博木博果爾以六千人來援,乘我師後,索海設伏以待,破敵,俘四百,乘勝入其壘,博木博果爾遁去。索海率諸將攻掛喇爾屯,攻克之,屯兵五百,斬級二百,俘百三十還。逐敵額蘇里屯西、額爾圖屯東,俘六千九百五十六人,牛羊駝馬稱是。師還,命貝勒杜度、阿巴泰迎勞,太宗幸實勝寺,賜宴。?功,授二等甲喇章京。兵部劾索海行軍不立寨,俘有逋者,當奪賞,命貰之。

六年春,從睿親王多爾袞等出師圍錦州,坐私遣官兵歸,離城遠屯,徵還,與譚泰、阿山、葉克書等皆罰鍰。夏,復從多爾袞等出師圍錦州,城兵出行汲小凌河,索海以兵四百邀擊,斬九十餘級,遂從攻松山,擊破明軍。時有敏惠恭和元妃之喪,索海召降將祖大樂俳逸樂,姑自娛於家,自今毋?優至其帳歌舞,刑部論索海當死,削職。上使諭之曰:「爾既至篤恭殿及大清門前。」索海遂坐廢,終太宗世不復用。世祖順治二年,以副都統從征四川,卒於軍。子多頗羅,以從入關擊流賊有勞,授牛錄章京,進一等甲喇章京。十四年,從信郡王多尼征雲南,戰死磨盤山。

倭黑,費英東諸孫。父察哈尼。方索海嗣父爵而黜也,太宗以納海、圖賴分襲,既又以事奪爵,復以察哈尼襲。尋改三等昂邦章京。卒,子倭黑,襲。世祖初元,從入關。四年,復更定爵秩,改三等精奇尼哈番,遇恩詔累進一等。十六年,進三等公,並授內大臣。康熙八年,聖祖譴鼇拜,吏部議倭黑與同族,當黜,命罷內大臣,隸驍騎營。

吳三桂反,倭黑從征。十三年,命以署副都統率兗州駐防兵,佐定南將軍希爾根進討,敗耿精忠將左宗邦於分宜,敗吳三桂將硃君聘、黃乃忠於袁州,遂收安福。擊賊鸞石嶺、白水口,屢捷。十五年,加太子太保。從大將軍安親王岳樂復萍鄉,至長沙,擊敗吳三桂兵。十六年,岳樂分兵授倭黑,令駐茶陵。十七年,移屯攸縣。十八年,從大將軍貝子彰泰下雲南,授鑲黃旗蒙古副都統。雲南平,二十一年,擢都統。議政大臣議諸將帥功罪,以倭黑擊賊長沙嘗引退,當譴,命罷太子太保。三十年,卒。子傅爾丹,自有傳。

何和禮,棟鄂氏,其先自瓦爾喀遷於棟鄂,別為一部,因以地為姓。何和禮祖曰克徹巴顏,父曰額勒吉,兄曰屯珠魯巴顏,世為其部長。何和禮年二十六,代兄長其部。棟鄂部素強,克徹巴顏與章甲城長阿哈納相仇怨。阿哈納,興祖諸孫,為「寧古塔」六貝勒之一。棟鄂屢侵寧古塔,寧古塔借兵哈達伐棟鄂,互攻掠。

太祖初起兵,聞何和禮所部兵馬精壯,乃加禮招致之。歲戊子,太祖納哈達女為妃,行。比還,遂以所部來附,太祖以長女妻焉。何和禮故有妻,挾所部留故?何和禮率三十騎地者,求與何和禮戰,太祖面諭之,乃罷兵降。旗制初定,何和禮所部隸紅旗,為本旗總管。歲戊申,從太祖征烏喇,率本旗兵破敵有功。歲辛亥,太祖命與額亦都、扈爾漢將兵伐渥集部虎爾哈路,克扎庫塔城。歲癸丑,從太祖再征烏喇。太祖招諭布占泰,猶冀其悛悔,何和禮與諸貝勒力請進攻,遂滅烏喇。天命建元,旗制更定,何和禮所部隸正紅旗。置五大臣,何和禮與焉。四年,從破明經略楊鎬。六年,下沈陽、遼陽,何和禮皆在行間,?功,授三等總兵官。九年八月,卒,年六十有四。時費英東、額亦都、安費揚古、扈爾漢皆前卒,太祖哭之慟,曰:「朕所與並肩友好諸大臣,何不遺一人以送朕老耶?」太宗朝,進爵為三等公。順治十二年,追謚溫順,勒石紀功。雍正九年,加封號曰勇勤。子六。

多積禮,何和禮次子。初授牛錄額真。事四貝勒,從伐烏拉。天聰間,擢甲喇額真。從伐錦州,圍大凌河,授游擊世職。崇德元年,帥師伐東海瓦爾喀部,俘壯丁三百餘,擢本旗梅勒額真。四年,與鎮國公扎喀納率兵屯籓、屏二城間,卒竊馬遁去,追之勿及。論罪,奪世職,籍沒,上命留弓矢、甲胄及三馬,仍領梅勒額真事。六年,從擊洪承疇,率騎兵循。七年,以老罷。順治五年,卒。?海追捕,斬獲甚

和碩圖,何和禮四子。初襲三等總兵官。太祖以大貝勒代善女妻焉,號和碩額駙。太宗即位,授正紅旗固山額真。天聰元年,從擊朝鮮,又從伐明,攻錦州、寧遠有功。二年,從貝勒阿巴泰帥師破錦州、杏山、松山諸路。九月,復伐察哈爾,克其四路軍。以功加五牛錄,進爵三等公。三年,從貝勒岳託帥師攻大安口,敗明戍兵於馬蘭峪,再敗明援兵於石門寨。復從太宗攻遵化,率本旗兵攻其城西北,克之。師薄燕京,結營土城關,明兵來攻,擊卻之。復敗明師於盧溝橋,與副都統阿山等陣斬明武經略滿桂、總兵孫祖壽,獲黑雲龍、麻登雲。師旋,克永平,帥騎兵守灤州。五年,從圍大凌河城,以本旗兵當其西北。明兵突圍出,與都統葉臣等夾擊破之,追奔及城壕而還。七年,上詢伐明及朝鮮、察哈爾三國何先,和碩圖疏言:「宜先葺治諸城堡,乃覘明邊,乘瑕而入。若天佑我,各城納款,勢不能速歸,南界六城,立界屯耕,修築可差後。慮我兵既出,敵伺其隙,鞭長不及,難為援也。沈陽、牛莊、耀州三城宜先繕完,庶邊界內外皆可長驅。」七月,和碩圖卒,上親臨哭之。順治十二年,追謚端恪。

都類,何和禮第五子,公主出也。初為牛錄額真,洊擢本旗固山額真。以公主子,增領兩牛錄。崇德元年,從太宗伐朝鮮,薄漢城,先登,城潰,率阿禮哈超哈兵入城搜剿。以失察所部違法亂行,罰鍰,奪所分俘獲。三年,從貝勒岳託伐明,次密雲墻子嶺。明將以三千人來拒,都類與譚泰督部將夾擊,大敗之,獲馬百、駝二十。軍分四道進,所當輒摧破,略地至濟南而還。四年,從鄭親王濟爾哈朗圍錦州,坐所部退縮,又受蒙古饋遺,罰鍰。未幾,所部訐告都類在山東時,縱?養盜馬,私發明德王埋藏珍物,坐論死,上貸之,奪職,籍沒。八年,復起為固山額真,鎮錦州。順治三年,從肅親王豪格征張獻忠,分兵定慶陽,會師西充,擊殺獻忠,與貝勒尼堪等戡定川北州縣。師還,論功,並遇恩詔,累進二等伯。十三年,卒。

安費揚古,覺爾察氏,世居瑚濟寨。父完布祿,事太祖,有章甲、尼麻喇人誘之叛,不從,又劫其孫以要之,終無貳志。安費揚古少事太祖。旗制定,隸滿洲鑲藍旗。

歲癸未,太祖兵初起,仇尼堪外蘭,克圖倫城,攻甲版。薩爾滸城長諾米訥、柰喀達陰助尼堪外蘭,漏師期,尼堪外蘭得遁去。太祖憾諾米訥、柰喀達,執而殺之,使安費揚古率兵取其城。康嘉者,太祖再從兄弟也,惎太祖英武,與?從謀以哈達兵至,俾兆佳城長李岱為導,劫瑚濟寨。既,引去,安費揚古方獵,聞有兵,與巴遜以十二人追及,擊破之。歲甲申正月,從太祖攻兆佳城,獲李岱。其黨李古里扎泰走附汪泰,安費揚古以太祖命往諭,並汪泰降之。六月,從太祖攻馬兒墩寨,寨負險,守者甚備,矢石雜下,攻三日不克。安費揚古夜率兵自間道攀崖而上,拔其寨。歲丁亥六月,太祖伐哲陳部,八月,克洞城,歲戊子九月,克王甲城,安費揚古皆從戰有功。尋攻克章甲、尼麻喇、赫徹穆諸城,又取香潭寨;其長李墩拜湖遁走,追及於碩郭之陽,俘以獻。歲癸巳六月,太祖略哈達富爾佳齊寨。師還,太祖躬勒兵以殿,哈達貝勒孟格布祿率騎追至,一騎出太祖前,太祖方引弓射,復有三騎突至,太祖馬幾墜,三騎揮刀來犯,安費揚古截擊,盡斬之;太祖亦射孟格布祿中馬踣,敵騎敗走。太祖嘉其勇,賜號碩翁科羅巴圖魯。九月,太祖既破九部師,閏十一月,命與額亦都、噶蓋等攻訥殷路佛多和山寨,斬其長搜穩塞克什。歲己亥九月,從太祖滅哈達。

歲辛亥七月,命與臺吉阿巴泰等伐渥集部烏爾古宸、木倫二路,取其地,俘其人以歸。歲癸丑正月,從太祖滅烏喇,師薄城,安費揚古執纛先登。尋置五大臣,安費揚古與焉。天命元年七月,命與扈爾漢帥師伐東海薩哈連部,至兀爾簡河,刳木為舟,水陸並進,取河南北三十六寨。八月丁巳,師至黑龍江之陽,江水常以九月始冰,是日當駐師處獨冰,寬將競從之,師畢渡,冰旋解?六十步,若浮梁。安費揚古曰:「此天佑我國也!」策騎先涉,遂取江北十一寨,降使犬、諾洛、石拉忻三路。三年四月,太祖取撫順,明總兵張承廕等赴援,分為三營,安費揚古擊其左營,大破之,遂乘勝取三岔兒諸堡。四年,破明經略楊鎬,滅葉赫。六年,取沈陽、遼陽。安費揚古皆在行間。

七年七月,卒,年六十四。順治十六年,追謚敏壯,立碑紀其功。太宗嘗諭?臣曰:「昔達海、庫爾?勸朕用漢衣冠,朕謂非用武所宜。我等寬袍大袖,有如安費揚古、勞薩其人者,挺身突入,能御之乎?」當日猛士如云,而二人尤傑出雲。

子達爾岱、阿爾岱、碩爾輝。達爾岱以甲喇額真事太宗。伐明,攻大凌河,守臧家堡,取錦州、寧遠,征朝鮮,皆有功。順治二年,授拖沙喇哈番。七年,追?安費揚古功,進一等阿達哈哈番。康熙五十二年,聖祖念安費揚古開國勛,別授三等阿達哈哈番,令其孫明岱分襲。阿爾岱子都爾德及碩爾輝孫遜塔,皆有功,受爵世祖朝,別有傳。

扈爾漢,佟佳氏,世居雅爾古寨。父扈喇虎,與族人相仇,率所部來歸,是歲戊子,。旗制定,隸滿洲正白?太祖起兵之六年也,扈爾漢年十三,太祖養以為子。稍長,使為侍旗。扈爾漢感太祖撫育恩,誓效死,戰輒為前鋒。

瓦爾喀部蜚悠城初屬烏喇,貝勒布占泰待之虐,丁未正月,城長策穆特黑請徙附太祖,太祖命貝勒舒爾哈齊等將三千人迎之,扈爾漢從。既至蜚悠城,收環城屯寨凡五百戶,使扈爾漢與揚古利率兵三百,護以前行。布占泰發兵萬人邀諸路,扈爾漢結寨山巔,使蜚悠城之。自率二百人與烏喇兵萬人各據山為陣,相持,使馳告後?來附者五百戶入保,分兵百人來戰,揚古利迎擊,烏喇兵稍退,會後軍至,奮擊,大破之。夏五月,?軍。翌日,烏喇悉太祖命貝勒巴雅喇將千人伐渥集部,扈爾漢從,取赫席黑、俄漠和蘇魯、佛訥赫?克索三路,俘二千人。己酉冬十二月,復命扈爾漢將千人伐渥集部,取滹野路,收二千戶以還,太祖嘉其功,賚甲胄及馬,賜號「達爾漢」。辛亥冬十二月,復命扈爾漢及何和禮、額亦都將二千人伐渥集部虎爾哈路,克扎庫塔城,斬千餘級,俘二千人;撫環近諸路,收五百戶以還。癸丑,太祖討烏喇,扈爾漢及諸將皆從戰,奪門入,遂滅烏喇。太祖置五大臣,扈爾漢與焉。

先是太祖與明盟,畫界,戒民毋竊逾,違者殺毋赦。至天命初將十年,明民越境採參鑿礦,取樹木果蔬,殆歲有之。太祖使扈爾漢行邊,遇明民逾塞,取而殺之,凡五十餘輩。太祖遣綱古里、方吉納如廣寧,廣寧巡撫李維翰系諸獄,而使來責言,且求殺逾塞民者,太祖拒不許。既乃取葉赫俘十人戮撫順關下,明亦釋使者。是年秋七月,太祖命扈爾漢及安費揚古將二千人伐薩哈連部,道收兀爾簡河南北三十六寨;遂進攻薩哈連部,取十一寨,降其三路。語詳安費揚古傳。

四年春二月,明經略楊鎬大舉四道來侵,三月,太祖督軍御之,扈爾漢從貝勒阿敏先行,與明游擊喬一琦遇,擊敗之。時朝鮮出軍助明,其帥姜弘立屯孤拉庫嶺,一琦收殘卒匿朝鮮營。扈爾漢從諸貝勒擊明軍,戰於薩爾滸,破明將杜松等;戰於尚間崖,破明將馬林等:扈爾漢皆在行間。明將劉綎自寬奠入董鄂路,牛錄額真托保等戰不利。扈爾漢帥師與托保合軍,憑隘為伏,諸貝勒軍出瓦爾喀什林。劉綎將率兵登阿布達里岡為陣,扈爾漢引軍扼其沖,諸貝勒繼至,東西夾擊,破之,綎戰死,明兵遂熸。五年,太祖取沈陽,扈爾漢從擊明總兵賀世賢等,敗之。歷加世職至三等總兵官。八年冬十月,卒,年甫四十?八,太祖親臨其喪。

扈爾漢諸子:渾塔襲三等總兵官,其後不著;準塔別有傳;阿拉密襲準塔世職,附見準塔傳。

論曰:國初置五大臣以理政聽訟,有征伐則帥師以出,蓋實兼將帥之重焉。額亦都歸太祖最早,巍然元從,戰閥亦最多。費英東尤以忠讜著,歷朝褒許,稱佐命第一。何和禮、安費揚古、扈爾漢後先奔走,共成篳路藍縷之烈,積三十年,輔成大業,功施爛然。太祖建號後,諸子皆長且才,故五大臣沒而四大貝勒執政。他塔喇希福祖羅屯,傳言列五大臣,或初闕員時嘗簡補歟?草昧傳聞,蓋不可深考矣。


\end{pinyinscope}