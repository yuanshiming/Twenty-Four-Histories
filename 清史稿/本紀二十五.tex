\article{本紀二十五}

\begin{pinyinscope}
宣統皇帝本紀

宣統皇帝名溥儀,宣宗之曾孫,醇賢親王奕枻之孫,監國攝政王載灃之子也,於德宗為本生弟子。母攝政王嫡福晉蘇完瓜爾佳氏。光緒三十二年春正月十四日,誕於醇邸。

三十四年冬十月壬申,德宗疾大漸,太皇太后命教養宮內。癸酉,德宗崩,奉太皇太后懿旨,入承大統,為嗣皇帝,嗣穆宗,兼承大行皇帝之祧,時年三歲。

攝政王載灃奉太皇太后懿旨監國。軍國機務,中外章奏,悉取攝政王處分,稱詔行之,大事並請皇太后懿旨。詔行三年喪。

甲戌,尊聖祖母慈禧端佑康頤昭豫莊誠壽恭欽獻崇熙皇太后為太皇太后,兼祧母後皇后為皇太后。先是,太皇太后並亦違豫。是日,崩。

乙亥,申嚴門禁。丁丑,尊封文宗祺貴妃為祺皇貴太妃,穆宗瑜貴妃為瑜皇貴妃,珣貴妃為珣皇貴妃,瑨妃為瑨貴妃,大行皇帝瑾妃為瑾貴妃。戊寅,停各省進方物。己卯,誥誡群臣,詔曰:「軍國政事,由監國攝政王裁定,為大行太皇太后懿旨。自朕以下,一體服從。嗣後王公百官,儻有觀望玩違,或越禮犯分,變更典章,淆亂國是,定即治以國法,庶無負大行太皇太后委寄之重,而慰天下臣民之望。」庚辰,頒大行皇帝遺詔。安慶兵變,剿定之。

十一月乙酉,頒大行太皇太后遺誥。詔四時祭饗祝版,醇賢親王稱曰「本生祖考醇賢親王」,嫡福晉稱曰「本生祖妣醇賢親王嫡福晉」。賑湖南澧州等屬水災。戊子,皇太后懿旨,皇帝萬壽節,俟釋服後,改於每年正月十三日舉行慶賀禮。庚寅,以即位前期告祭天地、宗廟、社稷、先師孔子,告祭大行太皇太后、大行皇帝幾筵。辛卯,帝即位於太和殿,以明年為宣統元年。頒詔天下,罪非常所不原者咸赦除之。詔遵大行太皇太后懿旨,仍定於第九年內,宣統八年頒布憲法,召集議員。鑄宣統錢。己亥,頒「中和位育」扁額於文廟。壬寅,內閣等衙門會奏監國攝政王禮節總目,詔宣布之。定守衛門禁章程,命貝勒載濤、毓朗、尚書鐵良總司稽察。以副都統昆源管理察哈爾牧群。定軍機處領班章京為從三品官,幫領班章京為從四品官。福建龍溪、南靖等縣水災,發帑銀四萬兩賑之。乙巳,詔各省督撫督率司道考察屬吏,秉公甄別。不肖守令罔恤民瘼者,重治之。立變通旗制處,命貝子溥倫、鎮國公載澤、那桐、寶熙、熙彥、達壽總其事。諭內外臣工尚節儉,戒浮華。丙午,遣官告祭孔子闕里、歷代帝王陵寢、五岳、四瀆。戊申,皇太后懿旨,罷頤和園臨幸。加恩慶親王奕劻以親王世襲罔替,貝勒載洵、載濤加郡王銜,皇太后父公桂祥食雙俸,大學士以次,錫賚有差。辛亥,冬至,祀天於圜丘,莊親王載功代行禮,自是壇廟大祀皆攝。

十二月壬子朔,加上穆宗毅皇帝、孝靜成皇后、孝德顯皇后、孝貞顯皇后、孝哲毅皇后尊謚。頒宣統元年時憲書。甲寅,立禁衛軍,命貝勒載濤、毓朗、尚書鐵良專司訓練。裁湖南鎮溪營游擊、乾州協守備,減留乾州協各營兵。旌殉節故直隸提督馬玉昆妾於氏。賑黑龍江、墨爾根、布特哈、黑水、大賚等城水災。免直隸河間等八州縣被災地畝糧租。丁巳,祈雪。命張之洞兼督辦川漢鐵路大臣。庚申,致仕大學士王文韶卒,贈太保。追予故云貴總督張亮基謚。民政部上調查戶口章程表式。壬戌,袁世凱罷,命大學士那桐為軍機大臣。癸亥,以梁敦彥為外務部尚書兼會辦大臣。那桐免步軍統領,以毓朗代之。乙丑,詔定西陵金龍峪為德宗景皇帝山陵,稱曰崇陵。丁卯,復祈雪。己巳,度支部上清理財政章程。壬申,命張勛所部淮軍仍駐東三省,辦理剿撫事宜。癸酉,義大利地震災,出帑銀五萬兩助賑。憲政編查館奏,京旗初選、衣復選事宜,應歸順天府辦理。乙亥,諭各省清蠲緩錢糧積弊。丁丑,復祈雪。是日,雪。免陜西各州縣光緒三十二年逋賦。戊寅,又雪。憲政編查館上核覆城鄉地方自治,並另擬選舉章程,詔頒行之。始制寶星,賜外務部總理、會辦大臣及出使各國大臣。庚辰,設奉天各級審判、檢察。辛巳,裁江西督糧道,設巡警、勸業兩道。

宣統元年己酉春正月壬午朔,以大行在殯,不受朝賀。癸未,免江蘇長洲等二十八州縣荒廢田地,暨昭文、金壇、丹徒、昆山、新陽、靖江、溧陽等七縣漕屯銀米。戊子,置呼倫貝爾沿邊卡倫。庚寅,欽差大臣東三省總督徐世昌以病請免,不許。辛卯,皇太后聖壽節,停筵宴,不受賀。甲午,免雲南阿迷州被災逋賦。乙未,度支部奏改定幣制,請仍飭會議。下政務處覆議。開廣西富川縣錫礦。丁酉,禁置買奴婢。戊戌,以近年新設衙門,新建省分,調用人員,請加經費,不能綜覈名實,命中外切實考覈裁汰,毋漫無限制。美利堅國開萬國禁煙會議於江蘇上海,端方蒞會。乙亥,陳璧被劾罷,以徐世昌為郵傳部尚書。調錫良為欽差大臣、東三省總督,兼管三省將軍事。以李經羲為雲貴總督。壬寅,命雲南交涉使高而謙赴澳門勘界。民政部上整頓京師內外警政酌改區章程。癸卯,上大行太皇太后尊謚,翼日頒詔天下。戊申,詔籌備立憲事宜,本年各省應行各節,依限成立,不得延誤。諭核定新刑律,來年頒行。復已革廣西提督蘇元春原官。罷福建廈門貢燕。己酉,上大行皇帝尊謚廟號,翼日頒詔天下。庚戌,重整海軍,命肅親王善耆、鎮國公載澤、尚書鐵良、提督薩鎮冰籌畫,慶親王奕劻總司稽查。罷鐵良專司訓練禁衛軍大臣。

二月壬子,修德宗實錄。癸丑,諭京、外問刑衙門清訟獄,釐剔弊端。戊午,農工商部奏,和蘭將訂新律,收華僑入籍,請定國籍法。下修訂法律大臣會外務部議。庚申,免浙江仁和等場灶課錢糧。乙丑,宣示實行預備立憲宗旨,詔曰:「國是已定,期在必成。內外大小臣工,皆當共體此意,翊贊新猷。言責諸臣,亦應於一切新政得失利病,剴切敷陳。」丁卯,命熙彥、喬樹枬、劉廷琛、吳士鑒、周自齊、勞乃宣、趙炳麟、譚學衡與榮慶、陸潤庠、張英麟、唐景崇、寶熙、硃益籓分日進講。講義令孫家鼐、張之洞核定。庚午,憲政編查館上統計表式。甲戌,申鴉片煙禁。丙子,免雲南宣威州被災村莊銀米。

閏二月甲申,詔嚴預備立憲責成,戒部臣、疆臣因循敷衍,放棄責任。以服制倫紀攸關,詔自今內外遭父母喪者,滿、漢皆離任聽終制。命前內閣學士陳寶琛總理禮學館。免浙江仁和等三十二州縣並杭、嚴二衛,杭、衢、嚴三所荒廢田地山塘丁漕銀米。丙戌,軍機大臣、大學士那桐丁母憂,詔奪情,百日孝滿改署任,仍入直。戊子,置庫倫理刑司員。免廣東新礦井口稅。予死事安徽砲營管帶官陳昌鏞優恤。辛卯,監國攝政王班見王公百官於文華殿。增設海參崴總領事。頒行度支部印花票稅。置直省財政監理官。丙申,裁湖北黃州、荊門、鄖陽、宜昌、施南、德安副將、參將、游擊、都司、中軍守備各官。出使大臣伍廷芳與美國訂立公斷專約成。丁酉,修崇陵。戊戌,立法政貴胄學堂,命貝勒毓朗總理。乙巳,旌賞年逾百歲甘肅固原州回民李生潮,賜御書匾額。己酉,以大行在殯,止年班內外札薩克蒙古汗、王、貝勒、貝子、公、臺吉、塔布囊等,及呼圖克圖喇嘛,西藏堪布,察木多帕克巴拉,回子伯克,土司、土舍,廓爾喀等毋來京。

三月辛亥,增設浙江巡警道、勸業道。甲寅,復前河南巡撫李鶴年原官。庚申,皇太后懿旨,度支部每歲交進年節另款銀二十八萬兩,自今停進。辛酉,奉移德宗景皇帝梓宮於西陵梁格莊行宮。甲子,以輪船招商局歸郵傳部管轄。乙丑,復裁奉天巡警道。增設洮昌等處兵備道,臨長海等處分巡兵備道。改奉錦山海關道為錦新等處兵備道兼山海關監督,東邊道為興鳳等處兵備道。升興京為興京府。丙寅,免梓宮經過宛平、良鄉、涿州、房山、淶水五州縣本年額賦十分之五,易州十分之七,並賞民間平毀麥田銀每畝一錢。己巳,詔復前戶部尚書立山、兵部尚書徐用儀、吏部左侍郎許景澄、內閣學士聯元、太常寺卿袁昶原官,並賜謚。命陸軍協都統吳祿貞督辦吉林邊務。裁山西雁平道。辛未,以前外務部左參議楊樞充出使比國大臣。亞東、江孜、噶大克開埠設關。丙子,增置奉天輝南直隸。戊寅,四川總督趙爾巽、駐藏大臣趙爾豐助款興學,下部優敘。趙爾巽捐廉贍族,賞御書「誼篤宗親」匾額。

夏四月庚辰,以各國遣使來吊,命貝子銜鎮國將軍載振使日本,法部尚書戴鴻慈使俄羅斯報謝,他國命駐使將事。甲申,度支部立幣制調查局,鑄通行銀幣。乙酉,普免光緒十四年訖光緒三十三年直省逋賦。癸巳,裁吉林琿春、三姓、寧古塔、伯都訥、阿勒楚喀各城副都統。置琿春兵備道,三姓兵備道。升改增置綏芬、延吉、五常、雙城、賓州、臨江諸府,伊通直隸州,榆樹直隸,寶清、綏遠二州,琿春、濱江、東寧三,富錦、穆棱、和龍、樺川、臨湖、汪清、額穆諸縣。尋復設舒蘭、阿城、勃利、饒河四縣。甲午,命內閣、部院、翰林、科道會議德宗升祔大禮。乙未,祈雨。丙申,甘肅蘭州、涼州、鞏昌、碾伯、會寧各屬災,發帑銀六萬兩賑之。壬寅,裁奉天左右參贊,承宣、諮議兩。甲辰,復祈雨。戊申,諭禁煙大臣切實考驗,毋許瞻徇敷衍。外省文武職官學堂,責成督、撫、將軍、都統等嚴查禁。

五月己酉朔,日有食之。辛亥,廷試游學畢業生進士黃德章等一百二十人,授官有差。壬子,於式枚言,各省諮議局章程與普魯士國地方議會制度不符。下憲政編查館妥議。癸丑,陳啟泰卒,以瑞澂為江蘇巡撫。允浙江紳士為故兵部尚書徐用儀、吏部右侍郎許景澄、太常寺卿袁昶於浙江西湖立祠。甲寅,復祈雨。陜甘總督升允以疏陳立憲利弊罷,以長庚代之。乙卯,命廣福署伊犁將軍。丁巳,聯豫、溫宗堯奏陳西藏籌辦練兵興學事宜。己未,命世續署外務部會辦大臣。楊士驤卒,以端方為直隸總督兼辦理通商事務大臣,張人駿為兩江總督兼辦理通商事務大臣,孫寶琦署山東巡撫。辛酉,以乍丫地方曩屬四川,命畫歸邊務大臣管轄。甲子,諭農工商部趣各省興舉農林工藝各政。乙丑,復祈雨。是日雨。戊辰,復前協辦大學士、戶部尚書翁同龢原官。己巳,唐紹儀免奉天巡撫,以侍郎候補。辛未,立游美學務處。癸酉,河南省改編營制。甲戌,賑雲南南寧州地震災。丙子,詔立軍諮處,以貝勒毓朗領之。攝政王代為統率陸海軍大元帥,貝勒載洵、提督薩鎮冰俱充籌辦海軍大臣。賑湖南澧州水災。丁丑,命貝勒載濤管理軍諮處事務。

六月甲申,慶親王奕劻免管理陸軍部事。賑湖北漢陽等府水災。乙酉,伊犁始編練陸軍。丙戌,授程德全奉天巡撫,陳昭常吉林巡撫,周樹模黑龍江巡撫。丁亥,開甘肅皋蘭縣、新城、西固城渠,以工代賑。己丑,賑雲南彌勒縣習洱等處地震災。免雲南太和縣屬上年被災田糧。庚寅,復已故降調兩廣總督毛鴻賓原官。追予禦賊殉難已故江蘇常州府通判岳昌於常州府建祠。賑奉天安東水災。甲午,呂海寰罷,以徐世昌充督辦津浦鐵路大臣,沈云沛副之。更奉天錦新道名錦新營口等處分巡兵備道。乙未,吉林大水,發帑銀六萬兩賑之。賑湖南澧州、安鄉、常德、岳州等州縣水災。丁酉,湖北荊州、漢陽兩府潦,發帑銀六萬兩,並命籌銀二十萬兩急賑之。辛丑,除熱河新軍營房占用圈地額租。壬寅,賑浙江錢塘等十一縣水災。癸卯,罷張勛東三省行營翼長,命赴甘肅提督任。甲辰,命伍廷芳、錢恂俱來京,以署外務部右丞張廕棠為出使美墨秘古四國大臣,署外務部右參議吳宗濂為出使義國大臣。趙爾巽奏平四川寧遠淺水惈夷。乙巳,賞京師貧民棉衣銀,後以為常。丙午,命李準為廣東水師提督。

秋七月戊申朔,裁湖南常德、寶慶、永順、岳州、澧州、臨武、桂陽、宣奉、永州、武岡、沅州、綏靖、辰州、嶺東各協、營,暨撫標、提標副將、參將、游擊、都司、守備等官。癸丑,浚遼河。丙辰,籌辦海軍大臣上擬訂海軍長官旗式章服圖說,管理軍諮處上酌擬軍諮處暫行章程。賑江西萍鄉等縣水災。丁巳,停秋決。法部上補訂高等各級審判試辦章程及擬定外省審判編制大綱。開四川重慶江北龍王洞煤鐵礦。戊午,免雲南魯甸、鎮雄二被災田畝銀米。甲申,南洋籌設勸業會,命南洋大臣、兩江總督張人駿為會長,各省籌辦協會,出品免稅釐。辛酉,德宗景皇帝梓宮奉移山陵,免所過各州縣旗租,並賞籽種銀。甲子,裁河南糧鹽道,增置巡警、勸業二道。戊辰,諭直省整飭積穀。恤以死建言頤和園八品苑副永麟。庚午,增設南洋各島領事。壬申,學部立圖書館於京師。洪江會匪姚芢山伏誅。丙子,湖北平糶。

八月丁丑朔,考察憲政大臣李家駒進日本司法制度考等書。辛巳,開黑龍江墨爾根嫩江甘河煤礦。甲申,改吉林濱江道為西北路道,西路道為西南路道,並前設之東北路道、東南路道俱名分巡兵備道。乙酉,賑福建福州風災,熱河開魯、平泉兩州縣水災。丙戌,藏番不靖,趙爾豐剿定之。命候補內閣學士李家駒協理資政院事。戊子,京張鐵路工成。除浙江鎮海縣開浚河道挖廢民灶田地銀米。己丑,開湖南平江金礦,新化銻礦,常寧鉛礦。庚寅,予救父捐軀湖北黃陂縣舉人陳鴻偉孝行,宣付史館。丁酉,大學士孫家鼐、張之洞並以病乞休。詔慰留之。戊戌,農工商部奏試行勸業富簽公債票。己亥,大學士張之洞卒,贈太保,入祀賢良祠。命戴鴻慈在軍機大臣上學習行走。以廷傑為法部尚書,葛寶華為禮部尚書。庚子,調誠勛為熱河都統,以溥良為察哈爾都統。癸卯,京師開廠煮粥濟貧民,發粟二千五百石有奇,已改設教養局、習藝所者仍給之,歲以為常。乙巳,修訂法律大臣進編訂現行刑律,下憲政編查館核議。丙午,詔以九月初一日為各省召集議員開議之期,特申誥誡。諭曰:「諮議局議員於地方利弊當切實指陳,妥善計畫。勿挾私心以妨公益,勿逞意氣以紊成規,勿見事太易而議論稍涉囂張,勿權限不明而定法或滋侵越。各督撫亦當虛心採納,裁度施行,以期上下一心,漸臻上理。至開局以後,各督撫尤應遵照定章,實行監督,務使議決事件不稍逾越權限,違背法律。共攄忠愛,以圖富強,朕實有厚望焉。」是月,載洵、薩鎮冰出洋考查海軍。

九月丁未朔,始制爵章頒賜。辛亥,和蘭保和會條約成,分別批準畫押。癸丑,命趙爾巽兼署成都將軍。乙卯,內閣會奏德宗升祔大禮。詔穆宗毅皇帝、德宗景皇帝同為百世不祧之廟,宜以昭穆分左右,不以昭穆分尊卑。定德宗升祔太廟中殿,供奉西又次楹又五室穆位。前殿於文宗顯皇帝之次,恭設坐西東向穆位。奉先殿準此。永為定制。丁巳,賞陸軍貴胄學堂畢業學生子爵成全等侍衛,及進敘有差。己未,資政院上選舉章程。壬戌,德人游歷云南,為怒夷所害,捕誅之。甲子,豫河安瀾。賑廣東省城及南海各縣水災。乙丑,錫林果勒盟阿巴嘎、阿巴哈那爾、浩齊特、烏珠穆沁災,發帑銀三萬兩賑之。賑雲南鎮雄等州縣水災。丙寅,黃河安瀾。授鹿傳霖體仁閣大學士,吏部尚書陸潤庠協辦大學士。賞游學畢業生項驤等舉人。辛未,升翰林院侍講學士為正四品,侍讀、侍講從四品,撰文秘書郎、修撰正五品,編修、檢討從五品。頒爪哇僑民捐立學堂扁額。癸酉,南河安瀾。是月,韓人安重根戕日本前朝鮮統監伊藤博文於哈爾濱。

冬十月丁丑朔,四川西昌、會理交界二板房夷匪為亂,官軍剿平之。成都將軍馬亮卒。庚辰,葬孝欽顯皇后於菩陀峪定東陵,免梓宮經過州縣地方額賦,並賞平毀麥田籽種銀。乙酉,孝欽顯皇后神牌祔太廟,翼日頒詔天下。丙戌,定成都將軍勿庸統轄松潘、建昌。以玉昆為成都將軍。丁亥,直隸總督端方坐違制奪職。調陳夔龍為直隸總督,兼辦理通商事務大臣,瑞澂署湖廣總督,寶棻為江蘇巡撫。以孫寶琦為山東巡撫,丁寶銓為山西巡撫。己丑,詔第一、二屆籌辦憲政事宜,內外諸臣應竭誠負責,並命憲政編查館稽核所奏成績,有因循敷衍、措置遲逾者,甄劾以聞。庚寅,憲政編查館上釐定各省提法使官制章程。開庫倫哈拉格囊圍金礦。延祉以疾免,命三多署庫倫辦事大臣。辛卯,江蘇溧陽、金壇、荊溪、宜興、丹徒、丹陽、震澤等縣災,發帑銀三萬兩賑之。癸巳,民政部奏,援案請賞米石,核定各廠院實需數目,收養貧民,詔行之。賑雲南大姚、文山等縣水災。甲午,大學士孫家鼐卒,贈太傅,入祀賢良祠,賞銀治喪。詔以已故五品卿銜山西即用知縣汪宗沂經學卓越,宣付史館。賞食餉閑散宗室、覺羅人等一月錢糧,暨孤寡半月錢糧,八旗、綠、步各營官兵半月錢糧,歲以為常。丁酉,免雲南元江州屬被水田畝銀米。庚子,東明黃河安瀾。癸卯,除廣東緝匪花紅,自今文武官有再收花紅者以贓論。復前禮部尚書李端棻原官。甲辰,停今年吉林珠貢。乙巳,順天紳士請為已故戶部尚書立山、內閣學士聯元立祠,許之。

十一月戊申,免直隸武清等十一縣額賦旗租,開州、東明、長垣等三州縣額賦。己酉,上兼祧母後皇太后徽號曰隆裕皇太后,翼日頒詔天下。癸丑,民政部上府州縣自治選舉章程。癸亥,復前福建巡撫張兆棟原官。設黑龍江愛琿沿邊卡倫二十,自額爾古訥河訖於遜河口。乙丑,置督辦鹽政大臣,以載澤為之,產鹽省分督撫為會辦鹽政大臣,行鹽省分督撫俱兼銜。丙寅,授陸潤庠體仁閣大學士,戴鴻慈以尚書協辦大學士。辛未,以貝勒毓朗為步軍統領。癸酉,都察院上互選規則。乙亥,學部上女學服色章程。予絕學專家已故候選同知直隸州知州華蘅芳,與其弟故直隸州州判世芳,及已故二品封職徐壽俱宣付史館。

十二月己卯,詔求直言。辛巳,增置奉天安圖、撫松二縣。壬午,賞游學專門詹天佑等工科、文科、法科進士,工科、格致科舉人。癸未,免山東青城等八十九州縣及衛所鹽場本年錢糧。乙酉,德宗景皇帝神牌升祔奉先殿。賞一產三男河南柘城縣民婦張劉氏、通許縣民婦田厲氏米布。賑廣東佛山等十三縣災。丙戌,定太醫院院使為四品。戊子,錄咸豐、同治年間戡定發、捻、回諸匪功臣後,敘官有差。除琿春軍隊營房占用旗戶地畝租。庚寅,趙爾豐奏四川德格土司多格生吉納土,改設流官,賞土舍都司世襲。壬辰,慶親王奕劻免管理陸軍貴胄學堂,以貝勒載潤代之。癸巳,增置熱河隆化縣。乙未,憲政編查館上禁煙條例,頒行之。復故前湖南巡撫陳寶箴原官。丙申,憲政編查館上禁買賣人口條款。戊戌,法部上法官懲戒章程。己亥,憲政編查館上京師地方自治選舉章程。庚子,升太醫院左右院判為五品。壬寅,憲政編查館上府州縣地方自治章程,並府州縣議事會議員選舉章程。癸卯,憲政編查館上法院編制法,並法官考試任用、司法區城分劃、及初級暨地方審判管轄案件各暫行章程。

二年庚戌春正月丙午朔,不受朝賀。己酉,廣州新軍作亂,練軍討平之。辛亥,詔以人心浮動,黨會繁多,混入軍營,句引煽惑,命軍諮處、陸軍部、南北洋大臣新舊諸軍嚴密稽查,軍人尤重服從長官命令,如有聚眾開會演說,並嚴查禁。移吉林大通縣駐松花江南岸,更名方正縣。乙卯,廣東革命黨王占魁等伏誅。丁巳,達賴喇嘛患川兵至,出奔。諭聯豫等仍遣員迎護回藏。辛酉,詔奪阿旺羅布藏吐布丹甲錯濟寨汪曲卻勒朗結達賴喇嘛名號。鹽政處上督辦鹽政試行章程。癸亥,協辦大學士戴鴻慈卒,贈太子少保銜,賞銀治喪。呂海寰等上中國紅十字會章程,命盛宣懷充會長。監察御史江春霖以論慶親王奕劻誤國,斥回原衙門。命郵傳部尚書徐世昌協辦大學士,內閣學士吳鬱生在軍機大臣上學習行走。甲子,管理軍諮處貝勒戴濤請赴日本、美、英、法、德、義、奧、俄八國考察陸軍。辛未,英國舉行萬國刑律改良會,法部奏遣檢察長徐謙往與會。甲戌,詔:「預備立憲,宜化除成見,悉泯異同。自今滿、漢文武諸臣陳奏事件,一律稱臣,以昭畫一而示大同。」

二月乙亥朔,聯豫請以新噶勒丹池巴羅布藏丹巴代理前藏事務。丙子,禁洋商湖南購運米石。辛巳,鐵良以疾免,以廕昌為陸軍部尚書,梁敦彥為稅務處會辦大臣。免浙江仁和、海沙、鮑郎、蘆瀝四場暨江蘇橫浦、浦東二場荒蕪灶蕩宣統元年逋課。壬午,免陜西榆林等四州縣舊欠,榆林府倉糧米草束。乙酉,以內閣侍讀學士梁誠為出使德國大臣。丁亥,民政部上修正報律,下憲政編查館核奏。己丑,復發帑銀三萬兩賑安徽災。壬辰,免吉林五常、樺甸縣宣統元年逋賦。甲午,聯豫奏拉裏僧俗暨工布番兵投誠歸化。丙申,葛寶華卒,調榮慶為禮部尚書,以唐景崇為學部尚書。己亥,予故湖北提督夏毓秀優恤。癸卯,憲政編查館上行政綱目。籌辦海軍大臣奏各司名目職掌。

三月乙巳朔,王士珍以疾免,命雷震春署江北提督。己酉,雲南威寧邪匪襲昭城,官軍剿滅之,匪首李老麼伏誅。辛亥,湖南民饑倡變,諭擒首要,散脅從。壬子,湖南巡撫岑春蓂罷,命楊文鼎暫代之。遣楊士琦赴南洋充勸業會審查總長。丁巳,祈雨。庚申,雨。追復故海軍提督丁汝昌原官。廢秋審覆審舊制。諭沿江各省督撫平糶。河南巡撫吳重熹免,以寶棻代之。調程德全為江蘇巡撫。壬戌,予遺愛在民故太常寺卿袁昶安徽蕪湖縣建祠。癸亥,裁奉天巡撫。授廣福伊犁將軍。甲子,革命黨人汪兆銘、黃復生、羅世勛謀以藥彈轟擊攝政王,事覺,捕下法部獄。庚午,旌殉夫烈婦山東曲阜孔令保妻潘氏,宣付史館。

夏四月甲戌朔,詔資政院於本年九月一日開院,欽選宗室王、公世爵、宗室、覺羅各部院官暨碩學通儒議員八十八人,前期召集。丙子,裁福建督糧道,增設巡警道、勸業道。丁丑,命載濤充專使大臣,往英國吊祭。戊寅,賞游學畢業生吳匡時等七人工科進士、法政科舉人有差。庚辰,憲政編查館修訂法律大臣進現行刑律,命頒行之。詔曰:「此項刑律,為改用新律之預備。內外問刑衙門,當悉心講求,依法聽斷。毋任意出入,致枉縱。」癸未,詔:「各省增設巡警、勸業兩道,原期保衛治安,振興實業。督撫於已補人員悉心考覈,如不能勝任,或於缺不宜,即奏明另補,毋回護瞻徇。」乙酉,聯豫請西藏曲水、哈拉烏蘇、江達、山南、碩般多及三十九族地方各設委員一人,並停藏番造槍、造幣兩廠。前出使義大利大臣錢恂進和會條約譯詮。丁亥,以江北鹽梟、會匪出沒靡常,諭雷震春剿撫之。己丑,度支部上幣制兌換則例。詔:「國幣單位,定名曰圓。暫就銀為本位,以一圓為主幣,重庫平七錢二分。另以五角、二角五分、一角三種銀幣,及五分鎳幣,二分、一分、五釐、一釐四種銅幣為輔幣。圓、角、分、釐,各以十進,著為定制。」以聯芳為荊州將軍。庚寅,定續選納稅多額十人為議員。辛卯,命郵傳部侍郎汪大燮充出使日本大臣。癸巳,梁敦彥以疾免,以鄒嘉來署外務部尚書兼會辦大臣。除湖北石首縣文義洲地方租課、蘆課。丙申,湖南巡撫岑春蓂褫職。

五月丙辰,升四川寧遠阿拉所巡檢為鹽邊撫夷通判。戊午,湖南常德府水潦災,發帑銀二萬兩賑之。李經羲奏雲南永昌府屬鎮康土州改流官,增置永康州。免雲南陸涼州被旱銀糧。辛酉,賑江北海州等處水災。癸亥,都察院代遞諮議局議員孫洪伊等並直省旗籍代表等呈請速開國會。詔仍俟九年籌備完全,再行降旨定期召集議員,宣諭之。甲子,免湖南苗疆佃民欠租,湖南鳳凰、乾州、永綏、保靖、瀘溪、麻陽、古丈坪七縣積欠屯租穀石。己巳,賑湖北災。辛未,裁奉天同江河防同知。

六月壬午,黑龍江災,發帑銀二萬兩賑之。乙酉,汪大燮進考查英國憲政編輯各書。己丑,命籌辦海軍事務大臣貝勒載洵充參預政務大臣。壬辰,命外務部侍郎胡惟德充稅務處幫辦大臣。丙申,詔:「各省督撫勞於行政,亟於籌款,而恆疏於察吏。不知吏治不修,則勞民傷財,亂端且從此起,新政何由而行?其各慎選牧令,為地擇人,斯為綏靖地方至計。」戊戌,詔各部院、各督撫嚴劾貪官汙吏,並諭貴戚及中外大臣敦品勵行,整躬率屬。己亥,命載澤、壽勛會阿穆爾靈圭、載潤查辦前鋒營暨內務府三旗護軍營,釐定章程以聞。是月,山東萊陽紳民相仇,匪首曲思文聚眾萬餘,圍攻城邑,劫殺官兵,海陽亦因徵收錢糧激變,旋並平定之。

秋七月甲辰,裁福建督糧道,置勸業道。瑞興免,以志銳為杭州將軍。乙巳,瑞澂、楊文鼎奏湘省匪勢蔓延,擬行清鄉法,從之。戊申,詔農工商部會同各督撫等調查礦產,熟籌開辦。庚戌,詔趣各督撫查造官民荒田及氣候土宜圖冊,並興舉工藝實業,報農工商部奏聞。壬子,農工商部立度量權衡用器制造廠。癸丑,貝勒載濤奏考察各國軍政,軍人犯罪,統歸軍法會議處審斷,非普通裁判所得與聞。諭照行之。甲寅,世續、吳鬱生免軍機大臣,以毓朗、徐世昌為軍機大臣。命唐紹儀署郵傳部尚書。毓朗免步軍統領並專司訓練禁衛軍大臣。命烏珍兼署步軍統領。設各省交涉使。新疆陸軍營官田熙年以擅殺釀變伏誅。丙辰,安徽皖南、南陵、宿州、靈壁等屬潦災,發帑銀四萬兩賑之。丁巳,法部上秋審條款。庚申,前江西提學使浙路總理湯壽潛,以劾盛宣懷為蘇浙路罪魁禍首,奪職。辛酉,賑皖北饑民。以忠瑞為科布多辦事大臣。聯魁免新疆巡撫,以何彥升代之。改各省按察使為提法使。甲子,大學士鹿傳霖卒,贈太保,入祀賢良祠,賞銀治喪。乙丑,命外務部參議上行走沈瑞麟充出使奧國大臣,外務部右丞劉玉麟充和蘭萬國禁煙大會全權委員。戊辰,奉天開葫蘆島港。己巳,置黑龍江訥河直隸同知。是月,載洵、薩鎮冰復往美利堅、日本兩國考察海軍。

八月甲戌,置奉天鎮東縣。乙亥,清銳免,以鐵良為江寧將軍。癸未,命沈家本充資政院副總裁。甲申,以外務部右丞劉玉麟充出使英國大臣。丁亥,理籓部奏變通禁止出邊開墾地畝、民人聘娶蒙古婦女、內外蒙古不準延用內地書吏教讀、公牘不得擅用漢文,蒙古人不得用漢字命名等舊例,許之。增置四川昭覺縣。己丑,聯芳免,以鳳山為荊州將軍。命廕昌兼訓練近畿各鎮大臣。甲子,命近畿陸軍各鎮俱歸陸軍部管轄。裁近畿督練公所。增置奉天鹽運使。改四川鹽茶道為鹽運使,茶務歸勸業道管理。乙未,以奏報禁種煙苗粉飾,下吉林、黑龍江、河南、山西、福建、廣西、雲南、新疆諸省督撫部議,申諭各省嚴切查禁。丙午,授徐世昌為體仁閣大學士,以吏部尚書李殿林協辦大學士。丁酉,以廓爾喀額爾德尼王畢熱提畢畢噶爾瑪生寫熱曾噶扒噶都熱薩哈拒西藏求援兵,詔嘉獎之。庚子,賑陜西華、渭南兩州縣潦災。

九月辛丑朔,資政院舉行開院禮,監國攝政王蒞會頒訓辭。壬寅,賞游學畢業生吳乃琛等四百五十九人文、醫、格致、農、工、商、法政進士、舉人有差。癸卯,免甘肅河、金、渭源、伏羌、安定、會寧、寧靈、循化、秦九州縣上年被災地畝錢糧草束。丙午,江北徐州等屬雨潦災,命度支部發帑賑之。乙巳,署綏遠城將軍、督辦墾務大臣信勤以疾免,調堃岫代之。以奎芳為烏里雅蘇臺將軍。戊申,命度支部再發帑銀二萬兩賑皖北災。壬子,張人駿以上海市情危急,請借洋款酌劑,並輸運庫帑銀五十萬兩,許之。癸丑,永定河安瀾。賑四川釂竹等縣水災。甲寅,裁海龍圍場總管。丙辰,詔直省舉賢良方正,從嚴甄取。己未,予積貲興學山東堂邑義丐武訓事實宣付史館。裁湖南常德府同知、寶慶府長安營同知。癸亥,諭綏遠城墾務緊要,沿邊道以下官,凡關墾務者,均聽墾務大臣節制。丙寅,楊樞以疾免,命農工商部右丞李國傑充出使比國大臣。賑黑龍江水災。丁卯,袁樹勛以疾免,命張鳴岐署兩廣總督。以沈秉堃為雲南巡撫。戊辰,裁貴州副將、游擊、都司、守備等官。免新疆迪化等十一縣民欠錢糧、籽種。

十月癸酉,詔改於宣統五年開國會,以直省督撫多以為言,復據順天直隸各省諮議局人民代表請原速開國會,故有是命。甲戌,命溥倫、載澤充纂擬憲法大臣。乙亥,黃河安瀾。丁丑,廣西岑溪匪亂,官軍剿定之,匪首陳榮安伏誅。程文炳卒,以程允和為長江水師提督,命甘肅提督張勛接統江南浦口各營。免甘肅靈州水災銀米。庚辰,增韞奏浙江裁綠營改編水師。辛巳,詔以縮改宣統五年開設議院,責成各主管衙門切實籌備,民政、度支、法、學諸部俱有應負責任,提前通盤籌畫,分別最要、次要,詳細以聞。並誡勉直省督撫淬厲精神,切實遵行,毋再因循推諉,致誤限期。壬午,何彥升卒,以袁大化為新疆巡撫。戊戌,予故大學士、前署兩廣總督張之洞於江寧省城建祠。

十一月癸卯,罷陸軍尚書、侍郎及左右丞、參議,改設陸軍大臣、副大臣各一人。置海軍部,設海軍大臣、副大臣各一人。以廕昌為陸軍大臣,壽勛副之。貝勒載洵為海軍大臣,譚學衡副之。乙巳,命海軍提督薩鎮冰統巡洋長江艦隊。丙午,雲南大姚縣民亂,入城劫獄殺人,官軍剿定之,匪首陳文培、鄧良臣俱伏誅。己酉,命前安徽巡撫馮煦為江、皖籌賑大臣。壬子,農工商部進編輯棉花圖說。丁巳,資政院言軍機大臣責任不明,難資輔弼,請設立責任內閣。詔以朝廷自有權衡,非院臣所得擅預,斥之。雷震春罷,命段祺瑞署江北提督。庚申,陳夔龍奏順直諮議局呈請明年即開國會,諭提前豫備事宜已慮不及,豈能再議更張。命剴切宣示,不準再行要求瀆奏。加賞普濟教養局倉米六十石,月以為常。辛酉,置各省高等審判、檢察,設丞、長,湖南緩設。癸亥,東三省國會請原代表來京,呈請明年即開國會。軍機大臣以聞。詔民政部、步軍統領衙門勒歸籍,勿逗留,再有來京及各省聚眾者察治之。甲子,詔趣憲政編查館擬訂籌備清單,內閣官制並纂擬具奏。予故大學士張之洞於湖北省城建祠。乙丑,慶親王奕劻請免軍機大臣及總理外務部,優詔慰留之。己巳,資政院請明諭剪發易服。

十二月壬申,諭各省曉諭學堂,禁學生干豫政治,聚眾要求,違者重治。丙子,唐紹儀以疾免,以盛宣懷為郵傳部尚書。丁丑,察哈爾右翼四旗蒙古災,發帑銀一萬兩賑之。己卯,志銳請變通銷除旗檔舊制。辛巳,召增祺入覲,命孚琦署廣州將軍。壬午,召趙爾巽入覲。癸未,重申煙禁,地方官仍前粉飾者罪之,並命民政、度支二部考核。命各省總督會同憲政編查館王大臣參訂外省官制。乙酉,裁並江蘇州縣,設審判。江寧以江寧並入上元,蘇州以長洲、元和並入吳,江都並入甘泉,昭文並入常熟,新陽並入昆山,震澤並入吳江,婁並入華亭,陽湖並入武進,金匱並入無錫,荊溪並入宜興。丁亥,憲政編查館上遵擬修正逐年籌備事宜清單。裁吉林水師營官丁。戊子,四川匪踞黔江縣為亂,官軍擊卻之,復其城。己丑,考察憲政大臣李家駒進日本租稅制度考、會計制度考。癸巳,四川匪首溫朝鍾竄入湖北咸豐縣境,擒斬之。乙未,命貝子銜鎮國將軍載振充頭等專使大臣,賀英君加冕。資政院議決新刑律總則、分則,詔頒布之。丙申,免陜西咸寧等六十四府州縣光緒三十三年逋賦,並廣有倉錢糧草束。丁酉,資政院上議決統一國庫章程。戊戌,資政院奏議決宣統三年歲出歲入總豫算。廷傑卒,以紹昌為法部尚書。己亥,裁甘肅蘭州道,置勸業道。是月,江、淮饑,人相食。東三省疫。

三年辛亥春正月庚子朔,以山海關外防疫,天寒道阻,諭陳夔龍、錫良安置各省工作人。丙午,馮煦奏察勘徐、淮災狀。己酉,免江蘇長洲等四十州縣田地銀糧。庚戌,賑江蘇高郵、寶應、清河、安東、山陽、阜寧等縣水災。甲寅,度支部上全國豫算章程。丙辰,釋服。御史胡思敬劾憲政編查館,言新官不可濫設,舊官不可盡裁;起草應用正人,頒行當採眾議。下其章於政務處。庚申,調志銳為伊犁將軍,廣福為杭州將軍。乙丑,除非刑。凡遣、流以下罪,毋用刑訊。法部奏上已革綏遠城將軍貽穀罪論死。詔改戍新疆效力贖罪。乙巳,命周樹模會勘中俄邊界。是月,直隸、山東民疫。

二月庚午朔,予故大學士、前湖南巡撫王文韶於湖南省城建祠。馮煦請濬睢河。民政部上編訂戶籍法。壬申,諭所司防疫,毋藉端騷擾,並命民政部、步軍統領衙門、順天府以保衛民生之意諭人民。乙亥,四川德格、春科、高日三土司改設流官,置邊北道,登科府,德化、白玉二州,石渠、同普二縣。定應遣新疆軍臺人犯改發巴、藏。丙子,免雲南昆明等三州縣被災田地條糧銀米。丁丑,免浙江仁和等三十州縣,杭、嚴二衛,衢、嚴二所荒地錢糧漕米。戊寅,改陸軍部、海軍部大臣、副大臣為正都統、副都統,仍以廕昌、壽勛、載洵、譚學衡為之。英人占片馬。癸未,命李家駒撰擬講義輪班進呈。丙戌,裁駐藏幫辦大臣,設左右參贊。丁亥,頒浙江惠興女學堂「貞心毅力」扁額。己丑,外務部上勛章贈賞章程。命度支部右侍郎陳邦瑞、學部右侍郎李家駒、民政部左參議汪榮寶協纂憲法。以誠勛為廣州將軍,溥頲為熱河都統。以貝子溥倫為農工商部尚書,世續為資政院總裁,李家駒副之,劉若曾為修訂法律大臣。壬辰,禁洋商運鹽入口。改設英屬檳榔嶼正領事官。

三月庚子,以劉銳恆為雲南提督。裁稽察守衛處,置管理前鋒、護軍等營事務處,三旗護軍仍隸內務府。陸軍部奏,東三省測量局員焦滇賄賣秘密地圖,誅之。辛丑,裁奉天承德、錦二縣。壬寅,裁四川川北、重慶二鎮總兵官。癸卯,頒盡忠節、守禮節、尚武勇、崇信義、敦樸素、重廉恥六條訓諭軍人。丁未,賞陸軍各鎮、協統制、統領等官何宗蓮、李奎元等陸軍副都統銜、協都統有差。戊申,吉林濬圖們江航路通於海。己酉,命出使義國大臣吳宗濂充專使,賀義大利立國慶典。庚戌,革命黨人以藥彈擊殺署廣州將軍孚琦。壬子,以薩鎮冰為海軍副都統。趙爾豐奏平三巖野番,改孔撒、麻書兩土司,設流官。甲寅,授張鳴岐兩廣總督。乙卯,度支部尚書載澤與英美德法四國銀行締結借款契約。丙辰,賞伊犁將軍志銳尚書銜,伊犁地方文武各官受節制。免浙江仁和等三十七州縣並衛所田塘宣統二年銀糧。戊午,以江、皖、豫災,命馮煦會三省督撫籌春賑。己未,和蘭開禁煙會於海牙,命出使德國大臣梁誠往與會。賑科布多札哈沁蒙古游牧災。庚申,錫良以疾免,調趙爾巽為東三省總督,授欽差大臣,兼管三省將軍事。加直隸熱河道提法使銜。辛酉,命趙爾豐署四川總督,王人文為川滇邊務大臣。予哀毀殉親前浙江巡撫聶緝椝孝行宣付使館。癸亥,漢儒趙岐、元儒劉因俱從祀文廟。華商創立大同學校於日本橫濱,頒「育才廣學」扁額。丁卯,革命黨人黃興率其黨於廣州焚總督衙署,擊走之。

夏四月辛未,楊文鼎請緩裁湖南綠營及防軍。甲戌,賞游學畢業生鍾世銘、汪爔芝等法政科進士、舉人,工科舉人有差。丙子,趙爾巽奏請用人行政便宜行事,從之。丁丑,裁山東撫、鎮標營官。戊寅,詔改立責任內閣。頒內閣官制。授慶親王奕劻為內閣總理大臣,大學士那桐、徐世昌俱為協理大臣。以梁敦彥為外務大臣,善耆為民政大臣,載澤為度支大臣,唐景崇為學務大臣,廕昌為陸軍大臣,載洵為海軍大臣,紹昌為司法大臣,溥倫為農工商大臣,盛宣懷為郵傳大臣,壽耆為理籓大臣。復命內閣總、協理大臣俱為國務大臣,內閣總理大臣、協理大臣均充憲政編查館大臣,慶親王奕劻仍管理外務部。置弼德院,以陸潤庠為院長,榮慶副之。罷舊內閣、辦理軍機處及會議政務處。大學士、協辦大學士仍序次於翰林院。裁內閣學士以下官。置軍諮府,以貝勒載濤、毓朗俱為軍諮大臣,命訂府官制。趙爾巽會陳夔龍、張人駿、瑞澂、李經羲與憲政編查館大臣商訂外省官制。己卯,慶親王奕劻、大學士那桐、徐世昌俱辭內閣總理、協理,不許,趣即任事。重申鴉片煙禁,諭民政、度支二部,各省督撫剋期禁絕。詔定鐵路國有。先是,給事中石長信疏論各省商民集股造路公司弊害,宜敕部臣將全國幹路定為國有,自餘枝路準各省紳商集股自修,上韙之,下郵傳部議。至是,奏言:「中國幅員廣袤,邊疆遼遠,必有縱橫四境諸大幹路,方足以利行政而握中樞。從前規畫未善,致路政錯亂紛歧,不分枝幹,不量民力,一紙呈請,輒準商辦。乃數載以來,粵則收股及半,造路無多。川則倒帳甚鉅,參追無著。湘、鄂則開局多年,徒供坐耗。循是不已,恐曠日彌久,民累愈深,上下交受其害。應請定幹路均歸國有,枝路任民自為。曉諭人民,宣統三年以前各省分設公司集股商辦之幹路,應即由國家收回。亟圖修築,悉廢以前批準之案。」故有是詔。辛未,吉林火災,發帑銀四萬兩賑之。癸未,贈恤署廣州將軍副都統孚琦。丁亥,資政院請預算借款兩事歸院會議,不許。戊子,起端方以侍郎候補,充督辦粵漢、川漢鐵路大臣。諭裁缺候補人員毋得奏事。諭本年秋季調集禁衛軍及近畿各鎮陸軍於直隸永平府大操。己丑,恭親王溥偉以疾免禁煙大臣,以順承郡王訥勒赫代之。庚寅,郵傳大臣盛宣懷與英德法美四國銀行締結借款契約成。辛卯,龐鴻書罷,以沈瑜慶為貴州巡撫。壬辰,命督撫曉諭人民,鐵路現歸官辦,起降旨之日,川、湘兩省租股,並停罷之。宣統三年四月以前所收者,應由郵傳部、督辦鐵路大臣會督撫查奏。地方官敢有隱匿不報者詰治。楊文鼎奏湘省自聞鐵路幹路收歸國有諭旨,群情洶懼,譁噪異常,遍發傳單,恐滋煽動。諭嚴行禁止,儻有匪徒從中煽惑,意在作亂者,照懲治亂黨例,格殺勿論。硃家寶奏江、淮交會為匪黨出沒之區,比歲薦饑,盜風尤熾。請援鄂、蜀懲辦會匪、土匪章程,犯者以軍法從事。丙申,移稅務司附屬之郵政歸郵傳部管理。除云南昆明縣官用田地額賦。丁酉,賑山東滕、嶧二縣災。

五月庚子,用湖南京官大理寺少卿王世祺等言,停湖南因路抽收房捐及米鹽捐。辛酉,楊文鼎奏湖南諮議局呈湘路力能自辦,不甘借債,據情代奏,嚴飭之。恤墨西哥被害華僑銀。壬寅,裁廣西綠營都司、守備以下官及馬步兵。癸卯,山東兗、沂、曹三府,濟寧州災,發帑銀三萬兩賑之。四川諮議局以紳民自聞鐵路國有之旨,函電紛馳,請緩接收,並請停刊謄黃,呈王人文代奏。人文以聞,詔切責之,仍命迅速刊刻謄黃,遍行曉諭,並剴切開導。乙巳,免琿春貧苦旗丁承領荒地價銀。戊申,廷試游學畢業生進士江古懷等,敘官有差。乙卯,孫寶琦奏宗支不宜豫政,飭之。壬子,起復那桐,仍授文淵閣大學士。丙辰,廣東因收回路事,倡議不用官發紙幣,持票取銀。諭張鳴岐防範。丁巳,資政院上修改速記學堂章程。戊午,度支、郵傳二部會奏川、粵、漢幹路收回辦法。請收回粵、川、湘、鄂四省公司股票,由部特出國家鐵路股票換給。粵路發六成。湘、鄂路照本發還。川路宜昌實用工料之款四百餘萬,給國家保利股票,其現存七百餘萬兩,或仍入股,或興實業,悉聽其便。詔端方迅往三省會各督撫照行之。丁寶銓以疾免,以陳寶琛為山西巡撫。庚申,命於式枚總理禮學館。甲子,內閣上內閣屬官官制、法制院官制,詔頒布之。置內閣承宣,制誥、敘官、統計、印鑄四局。設閣丞、長、局長各官。並置內閣法制院院使。罷憲政編查館、吏部、中書科、稽察欽奉上諭事件處、批本處,俱歸其事於內閣。以繙書房改隸翰林院。陸軍部奏,簡各省督練公所軍事參議官。乙丑,翰林院進檢討章■J7所纂康熙政要。

六月丁卯,命資政院會內閣改訂院章。賑湖南武陵、龍陽、益陽三縣水災。保定陸軍軍械局火藥庫、陸軍第二鎮演武火藥庫俱火。庚辰,安徽水,無為州五里碑、九連等處圩壞。辛巳,以榮慶為弼德院院長,鄒嘉來副之。陸潤庠免禁煙大臣,陳寶琛免山西巡撫,以侍郎候補。伊克坦免都察院副都御史,以副都統記名。裁兼管順天府府尹。壬午,以陸鍾琦為山西巡撫。癸未,趙爾豐奏收巴塘得榮地方,戶民請納糧稅,浪莊寺喇嘛千餘人許還俗。又奏巴塘臨卡石戶民投誠,撥隸三壩管理。乙酉,伊克昭盟扎薩克固山貝子三濟密都布旗災,發帑銀一萬兩賑之。丙戌,丹噶爾及西寧縣匪黨糾眾為亂,官軍擊散之,首犯李旺、李統春、李官博儉等伏誅。辛卯,置典禮院,設掌院大學士、副掌院學士、學士、直學士各官。以李殿林為典禮院掌院學士,郭曾炘為副。壬辰,四川紳民羅綸等二千四百餘人,以收路國有,盛宣懷、端方會度支部奏定辦法,對待川民純用威力,未為持平,不敢從命,呈請裁察。王人文以聞,詔以一再瀆奏,切責之。增設和屬爪哇島總領事,泗水、蘇門答臘正領事。甲午,湖南常德府大雨河溢,浸屬縣,壞田廬,發帑銀六萬兩賑之。丙申,以禁煙與英使續訂條件,重申厲禁,諭中外切實奉行。

閏六月己亥,命寶熙充禁煙大臣。庚子,恩壽以疾免,以餘誠格為陜西巡撫。癸卯,安徽大雨,江潮暴發,濱江沿河各州縣澇災,發帑銀五萬兩賑之。庚戌,調餘誠格為湖南巡撫,楊文鼎為陜西巡撫。壬子,詔本年調集禁衛軍及近畿各鎮軍於永平府大操,命軍諮大臣貝勒載濤恭代親臨監軍。癸丑,命貝子溥倫、鎮國公載澤會宗人府纂擬皇室大典。乙卯,革命黨人以藥彈道擊廣東水師提督李準,傷而免。前吉林將軍銘安卒。丙辰,命載振、陸潤庠、增祺、陳寶琛、丁振鐸、姚錫光、沈云沛、誠勛、清銳、硃祖謀俱充弼德院顧問大臣,國務大臣奕劻、那桐、徐世昌、梁敦彥、善耆、載澤、唐景崇、廕昌、載洵、紹昌、溥倫、盛宣懷、壽耆及宗人府宗令世鐸、總管內務府大臣奎俊、繼祿俱兼任弼德院顧問大臣。丁巳,調善耆為理籓大臣,以桂春署民政大臣。調鳳山為廣州將軍,以壽耆為荊州將軍。川路股東會會長顏楷等呈劾郵傳部,趙爾豐以聞,不報。辛酉,裁各省府治首縣,改置地方審判。乙丑,內閣請修訂法規。

七月壬申,趙爾豐奏鐵路收歸國有,川民仍多誤會,相率要求。諭郵傳部、督辦鐵路大臣清理路股,明示辦法,以釋群疑。甲戌,命瑞澂、張鳴岐、趙爾豐、餘誠格各於轄境會辦鐵路事宜。命端方赴四川按查路事。丁丑,以四川人心浮動,宜防鼓惑,諭提督田振邦嚴束營伍彈壓之,趣端方速赴四川,許帶兵隊。趙爾豐、玉昆率提督、司、道奏,川民爭路激烈,請交資政院議決仍歸商辦,不許,仍責趙爾豐彈壓解散。己卯,江蘇各屬大雨,圩堤潰決,田禾淹沒,發帑銀四萬兩賑之。永定河決。端方入川,水陸新舊諸軍聽調遣。調陸徵祥為出使俄國大臣,劉鏡人為出使和國大臣。辛巳,忠瑞免,以桂芳為科布多辦事大臣。溥免,以薩廕圖為科布多參贊大臣。壬午,四川亂作,趙爾豐執諮議局議長蒲殿俊、副議長羅綸、保路同志會長鄧孝可、股東會長顏楷、張瀾及胡嶸、江三乘、葉秉誠、王銘新九人。尋同志會聚眾圍總督署,擊之始散。賑浙江杭、嘉、湖、紹四府災。癸未,帝入學,大學士陸潤庠、侍郎陳寶琛授讀,副都統伊克坦教習國語清文。賑湖北水災。甲申,廣東澄海縣堤決,發帑銀四萬兩賑之。四川旅京人民以爭路開會,具呈資政院乞代奏。命捕代表劉聲元解歸籍。諭學部約束學生勿預外事,並敕所司禁聚眾開會。丁亥,山東濟南及東西路各州縣水災,黃河上游民墊復決,發帑銀五萬兩賑之。賑福建水災。戊子,命前兩廣總督岑春煊往四川,會趙爾豐辦理剿撫事宜。己丑,監國攝政王閱禁衛軍。癸巳,以四川民亂,諭趙爾豐督飭諸軍迅速擊散,仍分別良莠剿撫,被脅者宥之。甲午,波密野番投誠。

八月丙申,總稅務司赫德卒,晉太子太保銜。予故成都將軍、前伊犁將軍馬亮於伊犁建祠。壬寅,慶親王奕劻復請免內閣總理大臣及管理外務部,不許。甲辰,裁直隸督標、提標,通永、天津、正定、大名、宣化各鎮標官弁馬步守兵,提督依舊。丙午,江南提督劉光才以疾免,調張勛代之,以張懷芝為甘肅提督。丁未,定國樂。庚戌,置鹽政院,設大臣以下官,廢鹽務處。命載澤兼任鹽政大臣。癸丑,端方、瑞澂奏,湖北境內粵漢、川漢鐵路改歸國有,取消商辦公司,議定接收路股辦法,詔嘉之,並以深明大義獎士紳。甲寅,革命黨謀亂於武昌,事覺,捕三十二人,誅劉汝夔等三人。瑞澂以聞,詔嘉其弭患初萌,定亂俄頃,命就擒獲諸人嚴鞫,並緝逃亡。乙卯,武昌新軍變附於革命黨,總督瑞澂棄城走,遂陷武昌。詔奪瑞澂職,仍命權總督事,戴罪圖功。命陸軍大臣廕昌督師往討,湖北軍及援軍悉聽節制,薩鎮冰率兵艦、程允和率水師並援之。丙辰,張彪以兵匪構變,棄營潛逃,奪湖北提督,仍責剿匪。停永平大操。弛山西、河南運糧禁。武昌軍民擁陸軍第二十一混成協統領官黎元洪稱都督,置軍政府。嗣是行省各擁兵據地號獨立,舉為魁者皆稱都督。革命軍取漢陽,襲兵工廠、鐵廠,據漢口。丁巳,起袁世凱為湖廣總督,岑春煊為四川總督,俱督辦剿撫事宜。命貝勒載濤督禁衛各軍守近畿。戊午,王人文罷,復以趙爾豐為川滇邊務大臣。停奉天今年貢。己未,岑春煊辭四川總督,詔不許。趣梁敦彥來京供職。京師開糶濟民食。壬戌,詔長江水陸諸軍俱聽袁世凱節制。諭川、楚用兵,原脅從,自拔來歸,不咎既往,原隨軍自效,能擒獻匪黨者,優賞之。獲逆黨名冊應銷毀,毋株連。兩省被擾地方撫恤之。免裁各省綠營、巡防隊。壽耆免,授連魁荊州將軍。癸亥,皇太后懿旨,發帑銀二十萬兩賑湖北遭兵難民。福建龍溪、南靖兩縣河溢堤決,發帑銀二萬兩賑之。以湖北用兵,諭山東、山西兩省購運米麥濟軍。甲子,命副都統王士珍襄辦湖北軍務。

九月乙丑朔,日有食之。資政院第二次開會,詔勖議員。湖南新軍變,巡撫餘誠格奔於兵艦,巡防營統領前廣西右江鎮總兵黃忠浩死之。丙寅,陜西新軍變,護巡撫布政使錢能訓自殺不克,遂走潼關,西安將軍文瑞、副都統承燕、克蒙額俱死之。丁卯,皇太后懿旨,發內帑二十四萬兩賑直隸、吉林、江蘇、安徽、山東、浙江、湖南、廣東諸省饑,立慈善救濟會。戊辰,張廕棠免,以施肇基充出使義墨秘魯三國大臣。革命黨人以藥彈擊殺廣州將軍鳳山。己巳,皇太后助帑於慈善救濟會。資政院言郵傳大臣盛宣懷侵權違法,罔上欺君,塗附政策,釀成禍亂,實為誤國首惡,詔奪職。端方奏,訪查川亂緣起,實由官民■J8共而成,請釋諮議局議長蒲殿俊及鄧孝可等九人,湖北拘留法部主事蕭湘並免議,從之。以唐紹儀為郵傳大臣。命陳邦瑞為江、皖賑務大臣。庚午,皇太后出內帑一百萬兩濟湖北軍。召廕昌還,授袁世凱欽差大臣,督辦湖北剿撫事宜,節制諸軍。命軍諮使馮國璋總統第一軍,江北提督段祺瑞總統第二軍,俱受袁世凱節制。以春祿為廣州將軍。贈恤遇害廣州將軍鳳山。馮國璋與革命軍戰於灄口,水陸夾擊漢口,復之。壬申,以瑞澂失守武昌,避登兵艦,潛逃出省,偷生喪恥,詔逮京,下法部治罪。癸酉,下詔罪己。命溥倫、載澤纂憲法條文,迅速以聞。資政院總裁大學士世續以疾免,以李家駒代之,達壽為副。桂春回倉場侍郎任,趙秉鈞署民政大臣。奪湖南巡撫餘誠格職,仍權管湖南巡撫事。山西新軍變,巡撫陸鍾琦死之。雲南新軍變,總督李經羲遁,布政使世增及統制官鍾麟同、兵備處候補道王振畿、輜重營管帶範鍾嶽俱死之。命湯壽潛總辦浙江團練。開黨禁。戊戌政變獲咎,及先後犯政治革命嫌疑,與此次被脅自歸者,悉原之。資政院言內閣應負責任,請廢現行章程,實行內閣完全制度,不以親貴充任。詔韙之。順天府平糶。甲戌,江西新軍變,巡撫馮汝騤走九江,仰藥死。安徽新軍犯省垣,擊散之。乙亥,授袁世凱內閣總理大臣,命組織完全內閣。慶親王奕劻罷內閣總理大臣,命為弼德院院長。那桐、徐世昌罷內閣協理大臣,及榮慶並為弼德院顧問大臣。罷善耆、鄒嘉來、載澤、唐景崇、廕昌、載洵、紹昌、溥倫、唐紹儀、壽耆國務大臣,俱解部務。載濤罷軍諮大臣,以廕昌為之。起魏光燾為湖廣總督,命速往湖北。陸海各軍及長江水師仍聽袁世凱節制調遣。丙子,召袁世凱來京。命王士珍權署湖廣總督。用張紹曾言,改命資政院制定憲法。丁丑,資政院奏採用君主立憲主義,上重大信條十九事。發內帑十萬兩賑四川遭兵難民。戊寅,詔統兵大員以朝廷與民更始,不忍再用兵力之意諭人民。諭統兵大員申明紀律,禁擾民。命第六鎮統制吳祿貞署山西巡撫。袁世凱辭內閣總理大臣,溫詔勉之。贈恤殉難山西巡撫陸鍾琦。貴州獨立,舉都督,巡撫沈瑜慶遁。革命軍陷上海。袁世凱命前敵諸軍停進兵。尋遣知府劉承恩、正參領蔡廷幹詣黎元洪勸解兵,不得要領而還。己卯,詔許革命黨人以法律組政黨。資政院言漢口之役,官軍慘殺人民,請敕停戰。諭袁世凱按治軍官罪,商民損失由國家償之。吳祿貞奏,遣員入敵軍勸告,下令停攻擊,親赴襄子關撫慰革命軍,詔嘉之。裁廣東交涉使司。江蘇巡撫程德全以蘇州附革命軍,自稱都督。浙江新軍變,巡撫增韞被執,尋縱之。庚辰,予第二十鎮統制張紹曾侍郎銜,宣撫長江。紹曾稱疾不赴。命張勛充會辦南洋軍務大臣。趙爾豐免,命端方署四川總督。趣袁世凱入京。釋政治嫌疑犯汪兆銘、黃復生、羅世勛於獄。辛巳,廣西巡撫沈秉堃自稱都督。內閣銓敘局火。壬午,江寧新軍統制徐紹楨以其軍變,將軍鐵良、總督張人駿、提督張勛拒守。鎮江陷,京口副都統載穆死之。安徽新軍變,推巡撫硃家寶為都督。癸未,詔特命袁世凱為內閣總理大臣。從資政院奏,依憲法信條公舉,故有是命。呂海寰請依紅十字會法,推廣慈善救濟會,從之。廣東獨立,舉都督,總督張鳴岐遁。福建新軍變,將軍樸壽、總督松壽死之。甲申,皇太后懿旨罷繼祿,起世續復為總管內務府大臣。召錫良入覲。以朝廷於滿、漢軍民初無歧視,命統兵大員曉諭之。乙酉,山東巡撫孫寶琦宣告獨立。順天府奏立臨時慈善普濟赤十字總會於京師。罷貝勒毓朗軍諮大臣,以徐世昌代之。丙戌,賞恤江寧戰守將士。命呂海寰充中國紅十字會會長,兼慈善救濟會事。東三省諮議局及新軍要求獨立,總督趙爾巽不從,寢其議,仍令解勸之。丁亥,命近畿各鎮及各路軍隊並姜桂題所部俱聽袁世凱節制。戊子,分遣被兵各省宣慰使,徵國民意見。命各省督撫舉足為代表者來京與會議。趙爾巽以川事引咎請罷,詔不許。吳祿貞以兵至石家莊,為其下所殺。御史溫肅劾祿貞包藏禍心,反形顯著。詔陳夔龍按查。王士珍以疾免,命段芝貴護湖廣總督。永定河合龍。袁世凱來京。己丑,以張錫鑾為山西巡撫。溥頲免,以錫良為熱河都統。庚寅,袁世凱舉國務大臣。詔命梁敦彥為外務大臣,趙秉鈞為民政大臣,嚴修為度支大臣,唐景崇為學務大臣,王士珍為陸軍大臣,薩鎮冰為海軍大臣,沈家本為司法大臣,張謇為農工商大臣,楊士琦為郵傳大臣,達壽為理籓大臣,俱置副大臣佐之。於式枚、寶熙充修律大臣。紹昌、林紹年、陳邦瑞、王垿、吳鬱生、恩順俱充弼德院顧問大臣。辛卯,命段祺瑞署湖廣總督。起升允署陜西巡撫,督辦軍務。壬辰,浙江巡撫增韞坐擅離職守奪職。癸巳,以督攻秣陵關餘黨,將士奮勇,賞張勛二等輕車都尉世職。甲午,資政院上改訂院章,頒布之。

冬十月丙申,內閣奏立憲牴觸事項,停召對奏事。弼德院、軍諮府並限制之。廢各衙門直日舊章。更命世續復為文淵閣大學士。戊戌,伍廷芳、張謇、唐文治、溫宗堯勸告攝政王,請贊共和政體。庚子,以憲法信條十九事誓告太廟,攝政王代行祀事。以勞乃宣為大學堂總監督。溥良免,命直隸宣化鎮總兵黃懋澄兼署察哈爾都統。辛丑,命甘肅提督張懷芝幫辦直隸防務。四川成都獨立,舉都督。壬寅,督辦鐵路大臣、候補侍郎、署四川總督端方率兵入川,次資州,為其下所殺。其弟端錦從,並遇害。敘復漢陽功,封馮國璋二等男爵。命科爾沁親王阿穆爾靈圭往奉天,會趙爾巽籌畫蒙古事宜。變軍犯金陵,副將王有宏戰死。甲辰,孫寶琦罷獨立,自劾待罪。詔原之,褒獎山東官商不附和者。發帑犒張勛軍。賞梁鼎芬三品京堂,會李準規復廣東。丙午,革命軍陷江寧,將軍鐵良、總督張人駿走上海,張勛以其餘眾退保徐州。袁世凱與民軍訂暫時息戰條款,停戰三日。自是展期再三,至決定國體日乃已。命徐世昌充專司訓練禁衛軍大臣。丁未,寶棻免,以齊耀琳為河南巡撫。命壽勛會袁世凱、徐世昌籌辦軍務。戊申,哲布尊丹巴胡圖克圖自立,逐庫倫辦事大臣三多。詔奪三多職。己酉,贈恤殉難江西巡撫馮汝騤。庚戌,監國攝政王載灃奏皇太后,繳監國攝政王章,退歸籓邸。皇太后懿旨,晉世續、徐世昌俱為太保,衛護皇帝。諭段祺瑞剿當陽、天門諸路土匪。辛亥,詔授袁世凱全權大臣,委代表人赴南方討論大局。以馮國璋為察哈爾都統。資政院請改用陽歷,並臣民自由剪發,詔俱行之。壬子,改訓練禁衛軍大臣為總統官,以馮國璋為之。以良弼為軍諮府軍諮使。贈恤殉難閩浙總督松壽。丙辰,開黑龍江省太平山察漢敖拉煤礦。丁巳,革命軍至荊州,署左翼副都統恆齡死之。戊午,內閣奏行愛國公債票。辛酉,孫寶琦免,以胡建樞為山東巡撫。

十一月甲子朔,袁世凱請廢臣工封奏舊制。乙丑,命前署湖北提法使施紀云、前光祿寺少卿陳鍾信四川團練。丙寅,成都尹昌衡、羅綸以同志軍入總督衙,劫前署四川總督、川滇邊務大臣趙爾豐執之,不屈,死。戊辰,贈恤死事廣東潮州鎮總兵趙國賢。壬申,皇太后命召集臨時國會,以共和立憲國體付公決。初,袁世凱遣唐紹儀南下,與民軍代表伍廷芳討論大局,以上海為議和地,一再會議,廷芳力持廢帝制建共和國,紹儀不能折,以當先奏聞取上裁,遂以入告。世凱奏請召集王公大臣開御前會議,終從其言。至是,乃定期開國民會議於上海,解決國體。甲戌,各省代表十七人開選舉臨時大總統選舉會於上海,舉臨時大總統,立政府於南京,定號曰中華民國。戊寅,勸親貴王公等輸財贍軍。大理院正卿定成免,以劉若曾代之。己卯,楊士琦免,命梁士詒署郵傳大臣。辛巳,贈恤署四川總督、督辦粵漢川漢鐵路大臣、候補侍郎端方及其弟知府端錦。罷鹽政院。灤州兵變,撫定之。伊犁新軍協統領官楊纘緒軍變,將軍志銳死之。丁亥,告諭哲布尊丹巴胡圖克圖,並賚先朝珍物。庚寅,贈恤殉難署荊州左翼副都統恆齡。辛卯,袁世凱道遇炸彈,不中。壬辰,命張懷芝兼幫辦山東防務大臣。癸巳,命所司保護外人生命財產。命舒清阿幫辦湖北防務。以烏珍為步軍統領,京師戒嚴。

十二月甲午朔,賞張懷芝巡撫銜。己未,再予前山西巡撫陸鍾琦二等輕車都尉世職,追贈同時遇害其子翰林院侍講陸光熙三品京堂,優恤賜謚,並旌恤鍾琦妻唐氏。丁酉,張人駿罷,命張勛護兩江總督。胡建樞罷,命張廣建署山東巡撫,吳鼎元會辦山東防務。己亥,贈恤殉難伊犁將軍志銳。辛丑,皇太后懿旨,以袁世凱公忠體國,封一等侯爵。命額勒渾署伊犁將軍,文琦辦塔爾巴哈臺參贊大臣事。李家駒免,以許鼎霖為資政院總裁。革命黨以藥彈擊良弼,傷股,越二日死。壬寅,袁世凱辭侯爵,固讓再三乃受。癸卯,以復潼關,賞銀一萬兩犒軍。甲辰,以敘漢陽功,復張彪提督。乙巳,以張懷芝為安徽巡撫。贈恤死事福州將軍樸壽。丁未,命張錫鑾往奉天會辦防務,李盛鐸署山西巡撫,盧永祥會辦山西軍務。贈恤遇害軍諮府軍諮使良弼。戊申,以王賡為軍諮府軍諮使。己酉,皇太后懿旨,授袁世凱全權,與民軍商酌條件奏聞。時岑春煊、袁樹勛、陸徵祥、段祺瑞等請速定共和國體,以免生靈塗炭,故不俟國會召集,決定自讓政權,遂有是命。庚戌,命昆源會辦熱河防務。辛亥,命宋小濂署黑龍江巡撫。壬子,徐世昌免軍諮大臣,贈恤雲南殉難甘肅布政使世增。乙卯,錫良免,命昆源署熱河都統。丁巳,免江南徐州府未完丁漕銀糧。戊午,袁世凱奏與南方代表伍廷芳議,贊成共和,並進皇室優待條件八,皇族待遇條件四,滿、蒙、回、藏待遇條件七,凡十九條。皇太后命袁世凱以全權立臨時共和政府,與民軍商統一辦法。袁世凱遂承皇太后懿旨,宣示中外曰:「前因民軍起義,各省響應,九夏沸騰,生靈塗炭。特命袁世凱遣員與民軍代表討論大局,議開國會、公決政體。兩月以來,尚無確當辦法。南北暌隔,彼此相持。商輟於塗,士露於野。國體一日不決,民生一日不安。今全國人民心理,多傾向共和。南中各省,既倡義於前,北方將領,亦主張於後。人心所向,天命可知。予亦何忍因一姓之尊榮,拂兆民之好惡。是用外觀大勢,內審輿情,特率皇帝將統治權公諸全國,定為立憲共和國體。近慰海內厭亂望治之心,遠協古聖天下為公之義。袁世凱前經資政院選為總理大臣,當茲新舊代謝之際,宜為南北統一之方。即由袁世凱以全權組織臨時共和政府,與民軍協商統一辦法。總期人民安堵,海宇乂安,仍合滿、蒙、漢、回、藏五族完全領土為一大中華民國。予與皇帝得以退處安閒,優游歲月,受國民之優禮,親見郅治之告成,豈不懿歟!」又曰:「古之君天下者,重在保全民命,不忍以養人者害人。現將新定國體,無非欲先弭大亂,期保乂安。若拂逆多數之民心,重啟無窮之戰禍,則大局決裂,殘殺相尋,必演成種族之慘痛。將至九廟震驚,兆民荼毒,後禍何忍復言。兩害相形,取其輕者。此正朝廷審時觀變,恫吾民之苦衷。凡爾京、外臣民,務當善體此意,為全局熟權利害,勿得挾虛矯之意氣,逞偏激之空言,致國與民兩受其害。著民政部、步軍統領、姜桂題、馮國璋等嚴密防範,剴切開導。俾皆曉然於朝廷應天順人,大公無私之意。至國家設官分職,以為民極。內列閣、府、部、院,外建督、撫、司、道,所以康保群黎,非為一人一家而設。爾京、外大小各官,均宜慨念時艱,慎供職守。應即責成各長官敦切誡勸,勿曠厥官,用副予夙昔愛撫庶民之至意。」又曰:「前以大局阽危,兆民困苦,特飭內閣與民軍商酌優待皇室各條件,以期和平解決。茲據覆奏,民軍所開優禮條件,於宗廟陵寢永遠奉祀,先皇陵制如舊妥修各節,均已一律擔承。皇帝但卸政權,不廢尊號。並議定優待皇室八條,待遇皇族四條,待遇滿、蒙、回、藏七條。覽奏尚為周至。特行宣示皇族暨滿、蒙、回、藏人等,此後務當化除畛域,共保治安,重睹世界之升平,胥享共和之幸福,予有厚望焉。」遂遜位。

論曰:帝沖齡嗣服,監國攝政,軍國機務,悉由處分,大事並白太后取進止。大變既起,遽謝政權,天下為公,永存優待,遂開千古未有之奇。虞賓在位,文物猶新。是非論定,修史者每難之。然孔子作春秋,筆則筆,削則削。所見之世且詳於所聞,一朝掌故,烏可從闕。儻亦為天下後世所共鑒歟?


\end{pinyinscope}