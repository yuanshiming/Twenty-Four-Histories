\article{列傳二百十三}

\begin{pinyinscope}
王慶雲譚廷襄馬新貽李宗羲徐宗幹

王凱泰郭柏廕

王慶雲,字雁汀,福建閩縣人。道光九年進士,選庶吉士,授編修。二十七年,大考一等,擢侍讀學士,遷通政副使。慶雲通知時事,尤究心財政,窮其利病,稽其出入。文宗即位求言,慶雲疏請通言路,省例案,寬民力,重國計。其言重國計,略謂:「今歲入四千四五百萬,歲出在四千萬以下,田賦實徵近止二千八百萬。夫旱潦事出偶然,而歲歲輪流請緩;鹽課歲額七百四十餘萬,實徵常不及五百萬。生齒日增,而銷鹽日絀。南河經費,嘉慶時止百餘萬,邇來遞增至三百五六十萬。入少出多,置之不問,思為一切茍且之計,何如取自有之財,詳細講求:地丁何以歲歲請緩?鹽課何以處處絀銷?河工何以年年報險?必得弊之所在而革除之。」奏入,上深韙焉。

時命中外大臣保薦人材,禮部侍郎曾國籓舉慶雲以應,詔擢詹事,署順天府尹。咸豐元年,授戶部侍郎,仍署府尹。內務府議令莊頭增租,佃戶不應,則勒限退地。慶雲偕直隸總督訥爾經額援乾隆間停設莊頭,嘉慶間奏禁增租奪佃兩案,奏請敕內務府不得任意加租。戶部請改河東鹽政章程,並清查山西州縣虧空,命慶雲偕浙江布政使聯英往按。

尋奏定清查虧空章程,並會山西巡撫那蘇圖奏言:「晉商賠累,一在鹽本鉅,一在浮費多,一在運腳重。官鹽既貴,私販遂乘間蔓延。從前鹽價每石三五十兩,自坐商囤積居奇,畦地錠票,租典靡常,一業數主,人人牟利。一石之鹽,貴百三四十兩,運商安得不困?河東鹽行三省,酬應繁多,總商分派者號為攤,散商自送者歲有常例,統計二十六萬餘兩,幾達歲課之半。加以石鹽腳費多至百兩,因其定價難增,遂至相率為偽,攙沙短秤,民食愈艱。臣等公同商酌,輕鹽本必先定池價,革浮費必先行票法,減運腳必先分口岸,將緝私之法並寓其中。蓋鹽有專商,票無定販,大要在留商招販,先課後鹽,而後引目不致虛懸,課額無虞短絀。向來坐商昂價,總以缺產為詞。臣覽池面寬廣,滷氣醲厚,即雨暘不齊,裒多益寡,足敷五千六百餘石之額。鹽貴不在缺產,而在售私。擬定白鹽一石貴止六十兩,青鹽遞減,坐商工本外有贏餘。令各商立法互稽,但使鹽不旁流,商鹽自富,錠票銷價,亦復刪芟。畦地租典,先侭運商,總期減輕成本,禁衛課官吏浮費,別籌公用。每票徵銀七分有奇,隨課收發,此外需索,坐贓科罪。其領票、招販、掣鹽、截角諸事,悉仿兩淮成例,微為變通,以歸簡易。河東鹽行河南引地,自嘉慶二十四年改為商運民銷,以會興鎮為發鹽口岸,商民稱便。擬將陜西、山西、會興鎮分為三路,不許攙越,鹽到發販隨銷,亦聽商人自運,兼防夾私,力杜作偽。統計河東全綱,比較昔價,裁浮改岸,年省七十餘萬。得人守法,商力不疲。即間有歇業,或運商歸並,或坐商承充,永絕舉商、保商諸弊。」下部議行。

慶雲既明習計政,主部事,先後奏請清釐江寧、蘇州、安徽三布政司例應入撥、延未造報各款,自道光三年至咸豐元年,凡千五十九款,九百三十六萬兩。又奏言:「江南賦甲他省,額徵五百二十九萬,道光十六年,豁欠五百六十餘萬,計十年蠲一年之額;二十六年,豁欠一千餘萬,計十年蠲兩年。及咸豐二年,豁欠一千三百餘萬,十年幾蠲三年。請飭江蘇督撫,熟田未完,不得混入次年緩徵。」又奏覆閩浙總督季芝昌等以閩鹺疲累,請展緩勻代額課,言:「閩鹺所以疲累,病在私鹽充斥,浮費繁重。芝昌等議停勻代課六萬餘,派認續例課二萬餘,五年之後,勻代起徵,例課仍納。朝三暮四,恐無此辦法。」又言:「芝昌等但陳料理之難,未籌補救之法,或就場徵課,或按包抽稅。應令擇一可行之策,另議具奏。」又奏覆江西巡撫張芾請撥粵鹽濟銷,言:「江西借撥粵鹽,前明總制陳南金、巡撫王守仁嘗行之,所謂不加賦而財足,不擾民而事辦,其法至善。應令速籌遵辦。」又奏:「滇、黔解運銅鉛,道遠阻兵。應令於提鎮駐扎重兵之處,籌鑄制錢,並於附近水次兼鑄大錢,運四川、兩湖易銀,並派民間交納地丁稅課。」又奏:「新疆南、北兩路駐兵四萬,歲需經費一百三四十萬,垂及百年,為數萬萬。請停陜省官兵換防喀什噶爾等八城,即由伊犁、烏魯木齊滿、綠營飭撥,五年更換,可歲省數十萬。」又奏請裁東河河督南河河庫道並兩河員修防經費,南河不得過百萬,東河不得過七八十萬,並裁漕督,歸南河總督兼管。各疏多如所議行。尋授陜西巡撫。

四年,粵匪擾河南,慶雲赴潼關,與提督豐紳、將軍扎拉芬籌防禦。又自潼關赴商南,遍歷各隘。上命豐紳率兵駐襄陽。粵匪陷武昌,慶雲請以湖北會城暫移襄陽,山西、四川協籌軍餉,保全大局。尋調山西巡撫。

五年,奏言:「潞鹽行銷山西、陜西、河南三省,陜患鹽多,晉苦值貴。擬將陜引勻銷晉省三百七十石。晉引則就地遠近,公平定價。惟河南官運已覺暢行,擬兼行民運,以廣招徠。禁止吉蘭泰、花馬池鹽侵銷。」又言:「陜省課歸地丁,輸納不前,請仿河南招販民運,於河東、河西擇地設局稽查。」又奏言:「軍興以來,各軍營用銀出納,易錢買糧,歲豐銀裕,何便如之!今用兵之地,賦稅不全,仰給鄰省,完善之區,正供不足,佐以捐輸。當此穀貴錢荒,以銀易錢,以錢易糧,耗折大半。往時兵饑,得銀可飽,恐此後以銀亦不可飽,況銀且不可常繼。擬令州縣碾動倉穀,解餉兼用制錢,舟楫可通,宜無不便。」均如所請。

又奏:「山西前明逼近三邊郡縣,率民築堡自衛。一縣十餘堡至百數十堡,星羅釭布。今惟雲中、代、朔,堡寨相連,省南各屬,則多殘缺,當令繕完。定社規,立義學,化導少壯惰游,合祭賽以聯其情,相守望以齊其力。有事則聚守,無事則散居,於無形中寓堅壁清野之法。」又以河南南陽諸地旱蝗,請飭發倉籌賑,俾災民不為土匪勾脅,以救災即以弭患。捻匪擾南陽,慶雲密陳省南分三路,遣兵巡防。

擢四川總督,貴州思南教匪為亂,慶雲遣兵防酉陽秀山,請飭總兵蔣玉龍自鎮遠規復思南。尋奏四川舊有啯匪,盜案多於他省,飭各屬行保甲,立限捕盜。又奏於酉陽設屯田,分設屯兵駐防城鄉要隘。又奏:「川省差役捕盜,傳證起贓,輒糾多人,持械搜掠,名曰『掃通』者,此與強盜無異。請照強盜律,不分首從皆斬,兵丁有犯同之。」均下部議行。

尋以黔匪焚掠,漸近綦南,遣兵出境攻層巒山、飛梯巖諸隘,又破胡家坪賊巢。九年,兼署成都將軍,調兩廣總督。行次漢陽,以病乞罷,許之。旋召詣京師,病未即行。十一年,穆宗即位,授左都御史,擢工部尚書。同治元年三月,慶雲將力疾赴召,前一日劇病,卒,謚文勤。孫仁堪,循吏有傳。

譚廷襄,字竹厓,浙江山陰人。道光十三年進士,選庶吉士,散館授刑部主事,再遷郎中。出為直隸永平知府,調保定,遷順天府尹,擢刑部侍郎。咸豐六年,出為陜西巡撫。直省採米運京倉,廷襄疏言:「陜西產米少,轉輸不便。請改折解款,由部召糴,費節而事集。」七年,署直隸總督。

是時英、法、俄、美四國合軍陷廣東省城,廷襄疏請封貨閉關,恩威並用,上以海運在途,激之生變,虛聲無實益,不允。八年四月,英兵北犯,占大沽砲臺,窺內河。大沽口外積沙,海舟不能直入,敵舟至,數以小汽船採測。時方議款,不為備,不虞其驟發。欽差大臣僧格林沁劾廷襄,奪官戍軍臺。九年,以三品頂戴署陜西巡撫。上命直省禁習天主教,廷襄疏言:「天主教流行中國二百餘年,到處窮搜,轉滋駭愕。惟有密飭官吏稽查保甲,列冊密記,乘機啟導。」時款議未定,或請西巡,偕總督樂斌疏陳三便三難,議乃寢。

十一年,授山東巡撫。頻歲軍興,山東諸郡縣群盜蜂起,皖捻入境,勾結土匪,滋擾幾遍。僧格林沁大軍駐山東督剿,廷襄率兵出省協助,並督各郡縣團練防剿兼施,具詳僧格林沁傳。同治元年,兼署河東河道總督。三年,入為刑部侍郎,調工部,又調戶部。

五年,湖北巡撫曾國荃疏劾總督官文貪庸驕蹇,並以公使錢餽四川考官胡家玉、張晉裕等,上命尚書綿森及廷襄往按,並詰家玉。家玉言自四川還京,道湖北,官文等餽贐,以道梗改水程,無州縣支應,乃受以充費。廷襄等至湖北,疏言:「丁、漕、鹽、釐、關稅、捐輸,實用實支,並無浮濫。惟漢陽竹木捐零星不請獎敘者,凡因公動用,例不報銷之項,由此動支,官文餽家玉等是實。」上為罷官文。即令廷襄署總督,家玉等並下吏議。

御史佛爾國春劾國荃,言國荃亦以竹木稅治公廨,嚴責廷襄蒙蔽。廷襄等復疏陳國荃上官未久,無以竹木稅治公廨事,因言:「湖北三次陷賊,百端草創,不循例案,諸廢具舉,隨事設施。今以動用官款,加以處分,亦足示警。若更罪及所受之人,路遠給貲,親喪承賻,皆罣吏議。王道本人情,瑣屑煩苛,似非政體。」於是諸受餽者皆置不問。六年,上用前事奪官文總督,是冬,國荃亦以病乞罷。

廷襄還京,署吏部侍郎,遷左都御史。再遷刑部尚書,兼署吏部。九年,卒,贈太子少保,謚端恪。

馬新貽,字穀山,山東菏澤人。道光二十七年進士,安徽即用知縣,除建平,署合肥,以勤明稱。咸豐三年,粵匪擾安徽,淮南北群盜並起,新貽常在兵間。五年,從攻廬州巢湖,新貽擊敗援賊,迭破賊盛家橋、三河鎮、柘皋諸賊屯,尋克廬州。積功累擢知府,賜花翎,補廬州。七年,捻匪、粵匪合陷桃鎮,分擾上下派河,新貽破賊舒城,記名以道員用。八年,署按察使。賊犯廬州,新貽率練勇出城迎擊,賊間道入城,新貽軍潰失印,下吏議,革職留任。九年,丁母憂,巡撫翁同書奏請留署。十年,欽差大臣袁甲三為奏請復官。十一年,同書復奏薦,命以道員候補。丁父憂,甲三復奏請留軍。同治元年,從克廬州,敗賊壽州吳山廟,加按察使銜,署布政使。苗沛霖叛,從署巡撫唐訓方守蒙城,屢破賊。二年,授按察使,尋遷布政使。

三年,擢浙江巡撫。浙江新定,民困未蘇,新貽至,奏蠲逋賦。四年,復奏減杭、嘉、湖、金、衢、嚴、處七府浮收錢漕,又請罷漕運諸無名之費,上從之,命勒石永禁。築海寧石塘、紹興東塘,濬三江口。岐海為盜賊窟穴,遣兵捕治,擒其魁。厚於待士,會城諸書院皆興復,士群至肄業,新貽皆視若子弟,優以資用獎勵之。嚴州、紹興被水,蠲賑覈實,災不為害。臺州民悍,輒群聚械斗,新貽奏:「地方官憚吏議,瞻顧消弭。請嗣後有諱匿不報者參處;僅止失察,皆寬貸,仍責令捕治。」下部議行。象山、寧海有禁界地曰南田,方數百里,環海土寇邱財青等處窟其中,遣兵捕得財青置之法,南田乃安。黃巖總兵剛安泰出海捕盜,為所戕,檄副將張其光等擊殺盜五十餘。上以新貽未能豫防,下吏議。嘉興、湖州北與蘇州界,皆水鄉,方亂時,民自衛置槍於船,謂之「槍船」,久之聚博行劫為民害。新貽會江蘇巡撫郭柏廕督兵擒斬其渠,及悍黨數十,槍船害始除。擢閩浙總督。

七年,調兩江總督,兼通商大臣。奏言:「標兵虛弱,無以壯根本。請選各營兵二千五百人屯江寧,親加訓練。」編為五營,令總兵劉啟發督率緝捕,盜為衰止。宿遷設水、旱兩關,淮關於蔣壩設分關,並為商民擾累。新貽奏:「蔣壩為安徽鳳陽關轄境,淮關遠隔洪澤湖,不應設為子口。當令淮關監督申明舊例,嚴禁需索。宿遷旱關非舊例,徵數微,請裁撤,專收水關。」從之。幅匪高歸等在山東、江蘇交界占民圩,行劫,新貽捕誅其渠。

九年七月,新貽赴署西偏箭道閱射,事畢步還署。甫及門,有張汶祥者突出,偽若陳狀,抽刀擊新貽,傷脅,次日卒。將軍魁玉以聞,上震悼,賜恤,贈太子太保,予騎都尉兼雲騎尉世職,謚端愍。命魁玉署總督,嚴鞫汶祥,詞反覆屢變。給事中王書瑞奏請根究主使,命漕運總督張之萬會訊。之萬等以獄辭上,略言:「汶祥嘗從粵匪,復通海盜。新貽撫浙江,捕殺南田海盜,其黨多被戮,妻為人所略。新貽閱兵至寧波,呈訴不準,以是挾仇,無他人指使。請以大逆定罪。」復命刑部尚書鄭敦謹馳往,會總督曾國籓覆訊,仍如原讞,汶祥極刑,並戮其子,上從之。

新貽官安徽、浙江皆得民心,治兩江繼曾國籓後,長於綜覈,鎮定不擾。江寧、安慶、杭州、海塘並建專祠。

李宗羲,字雨亭,四川開縣人。道光二十七年進士,安徽即用知縣,歷英山、婺源、太平。咸豐三年,粵匪陷安慶,宗羲奉檄詣廬州軍督糧械,積功累擢知府。八年,曾國籓進規安徽,調充營務處。九年,署安慶知府,以疾去官。同治元年,河南巡撫嚴樹森疏薦,命送部引見,樹森旋撫湖北,又疏調從軍。三年,曾國籓督兩江,調赴兩江筦江北釐金總局,裁定沿江釐捐科則。江寧克復,以道員歸兩江補用。四年,署兩淮鹽運使。自軍興,淮南鹽艘改道泰興,宗羲於瓜洲東別濬新河,避長江風濤之險,商民便之。遷安徽按察使,再遷江寧布政使。五年,清水潭決,被災者七州縣,宗羲工賑並行,活民甚眾。定招墾荒田酌緩升科限制章程,及江寧七屬民衛丁漕折徵等次,民皆稱便。

八年,擢山西巡撫,劾布政使胡大任廢弛因循,罷之。令按察使李慶★等率兵分地駐防,陜回乘河冰來犯,三戰皆捷;屢自延川、韓城東竄,並擊走之。丁母憂去官。

十二年,服闋,擢兩江總督。日本方構釁,宗羲治江防,增築沿江烏龍山、江陰都天廟、象山、焦山、下關砲臺。又於吳淞口及江陰北岸瀏聞沙、烏龍山北岸沙洲圩次第添築,使江、海相犄角。時詔修圓明園,宗羲疏言:「外侮內患,天時人事,皆有可慮。請省營繕,減服御。」十三年,又疏言:「星變屢見,外患方熾。上年御史沈淮奏請停止園工,臣亦冒貢愚忱。茲復有不能已於言者,時局艱難,度支短絀,特一端耳。今外人入處肘腋,圓明園距京城數十里,既無堅城管鑰之固,復少大枝護衛之兵。頻年以來,每遇民、教爭鬥,外人動挾兵船要求。天津朝警,則海澱夕驚。皇上奉皇太后於此,此臣所萬分不安者也。如蒙皇上乾綱立斷,速諭停工,天下臣民,知皇上有臥薪嘗膽之思,必共振敵愾同仇之氣。人主居崇高之位,持威福之柄,茍無敬畏之念,則驕肆之心生;茍無忠諤之臣,則讒諂之人至。近日大學士文祥引疾,侍郎桂清外調,道路頗有惜詞。臣竊謂老成憂國者,宜留之左右,以輔成聖德;忠直敢諫者,宜誘之使言,以恢張聖聽。」疏入,上嘉納之。

總理各國事務衙門籌議海防六事,下各督撫詳議,宗羲上疏曰:「萬事根本,以用人為要,而就海防言,尤以求將才為要。宋臣楊萬里有言:『相不厭舊,將不厭新。』蓋言用兵忌暮氣,宜年壯氣銳,素有遠志,未建大功之人。至宿將勛臣,帝心簡在,固無俟臣下之論列也。古有海防無海戰,今練兵仍以水陸兼練為主。水師戰艦不及輪船,輪船又不及鐵甲,而船之得力與否,仍視乎駕馭之人。今戰艦即不能一時盡易,應就弁兵中挑赴輪船學習,仍歸水師提督節制。更招集沿海熟習沙線,能耐勞苦之人,參用西法,加以訓練。然沿海地廣,勢不能遍設輪船,若敵乘無備,舍舟登陸,則我船砲皆無所用,故不可不急練陸兵。同治十年,曾國籓議沿海奉天、直隸、山東、江蘇、浙江、福建、廣東七省練陸兵九萬,沿江安徽、江西、湖北三省練陸兵三萬,合成十二萬。以陸兵為御敵之資,以輪船為調兵之用,海道雖極遼遠,血脈皆可貫通。今誠踵其議而力行之,各省分定數目,各專責成,貴精不貴多,宜聚不宜散。從前缺額之兵,不必再補,現在已募之勇,更加精練,是在平時之實力講求矣。西洋火器,日新月異,疊出不窮。今日所謂巧,即後日所謂拙。論中國自強之策,決非專恃火器所能制勝。然風會所趨,有不能不相隨轉移者。各國新出之砲,現在上海機器局已能如式制造。惟火器不難於用而難於不用。有事試演,尚可經久,無事擱置,立形銹壞。以後購造槍砲,應於操演之後,時時磨洗,不許銹壞,違者罪之,是珍惜巨帑之要義。臣聞自古覘國勢者,在人材之盛衰,不在財用之贏絀;在政事之得失,不在兵力之強弱:未聞以器械為重輕也。且西人之所以強者,其心志和而齊,其法制簡而嚴,其取人必課實用,其任事者無欺誑侵漁之習,其選兵甚精,故臨陣勇敢而不畏死。不察其所以強,而徒效其器械,豈足恃哉?自福建創設機器局,上海繼之,江寧、天津又繼之,皆由槍砲而推及輪船。臣愚以為大沽、吳淞、直、東、閩、廣等口,如能各得鐵甲一二,蚊子船三四,佐以兵輪,安配重大擊遠之砲,與砲臺相輔,亦足屹成重鎮,稍戢戎心。惟泰西各國輪船以百數十計,鐵甲船以數十計,大砲以千計,小砲以數千計,即使中國歲籌巨款,多方制造,亦必不能如彼之多且精也。臣謂船砲當量力徐圖,而仍以修政事、造人材為本,使各國鄉風慕義,或外侮可以稍紓。近年勸捐、收釐、津貼,無法不備,民力竭矣。煤、鐵乃中國自然之利,若一一開採,不獨造船造砲取之裕如,且可以致富強。現在磁州業已奏明試辦,而湖南、福建、江西、山西等省已成之煤、鐵廠,擴而行之,果能有效,何必舍近求遠,取給外國?為目前權宜計,將各口洋稅通提六成,專供海防之用,五年為限,當可集事。若夫節流之法,更非難行。節之必自朝廷始,誠能罷土木之工,省傳辦之費,減宮中之用,則一歲所省,何啻百萬?各省督撫,盡裁不急之費,錢漕稅釐,實力稽察,勿使乾沒,則一歲所增,何啻百萬?請敕下戶部,統籌全局,分別出入,於綜覈各項之外,指定籌防專款,應用若干,俾中外上下曉然於經費之有限,財用之有制,力求撙節,不必言利,而度支可裕矣。以上皆就原奏四事推廣言之,要必得人而後可以言持久。臣周諮博採,事之可行者,尚有三端。沿海各島,大都土瘠產薄,惟臺灣形勢雄勝,與廈門相犄角。東南俯瞰噶囉巴、呂宋,西南遙制越南、暹羅、緬甸、新加坡,實為中國第一門戶。其地產有山木,可採以成舟航;有煤鐵,可開以資制造。其客民多漳、泉、湖、嘉剛猛耐苦之人,足備水師之選。如得幹略大員,假以便宜,俾之輯和民、番,兼用西人機器,以取煤鐵山木之利,數年後可開制造局;練海師,為沿海各省聲援,絕東西各國窺伺。此中國防海之要略,事之可行者一也。海外新嘉坡、檳榔嶼、舊金山、新金山各埠,均有閩、廣人在彼貿易,每處不下數萬人。其為首領者,必有幹濟之才,足以提倡全埠。如派領事出洋,物色人才,不論官階文武大小,有能任此事者,給以虛銜,令前往各埠結納首領,婉轉勸導,由各省督撫奏給職官,派為練首,令其團練壯丁,隨時操演。約計經費有限,而獲益無窮,事之可行者二也。現在通商各口,外人星羅釭布,中國情事,無一不周知,而彼都情形,中國則皆未深悉。自斌椿、志剛、孫家穀出使後,至今無續往之人。竊謂宜選有才略而明大體者,隨時遣使,設有交涉,可辯論者與之辯論,可豫防者密為設防。且於彼國有用之人才,新造之精器,均可隨時採訪,以為招致購買之地,事之可行者三也。」尋乞病罷歸。

光緒四年,東鄉民亂,命宗羲按讞。宗羲以知縣孫定揚浮收激變,冒昧請兵,提督李有恆妄殺平民千餘,據實入告,獄獲平反。六年,召詣京師,以病未愈,疏請乞緩行。十年,卒,賜祭葬。

子方本,舉人,兵部郎中。有幹濟。總督鹿傳霖、錫良先後令董商務、學務。川東旱災,治賑,被疾,卒,贈太僕寺卿。

徐宗幹,字樹人,江蘇通州人。嘉慶二十五年進士,山東即用知縣,除武城,調泰安。在任十年,有政聲,遷高唐知州。道光十七年,濰縣教匪馬剛等作亂,從巡撫經額布剿擒之,議解省下獄候命。宗幹請於巡撫,即其地誅之,眾心以定。遷濟寧直隸州。金鄉民濬彭河,下游諸屯民聚眾沮之,毆官傷胥役,勢洶洶,宗幹馳往諭使解散。屯民出自首,大吏欲置重典,宗幹以為民畏水患,非與官敵,聚眾本沮工,毆官非本意,力爭戍為首者七人。署兗州知府,修滋陽河堤。

二十二年,擢四川保寧知府,兼署川北道。擢福建汀漳龍道,屬縣有械半,案久不結。宗幹率壯勇數十人直入其村,集兩造剖其曲直,令同酒食以解之,令獻犯懲治,事遂解,一時梟悍皆斂跡。總督劉韻珂密薦。二十五年,丁母憂去官,服闋,起授福建臺灣道。咸豐三年,臺灣匪洪恭等陷臺灣、鳳山兩縣,復擾噶瑪蘭,宗幹督兵平之。四年,擢按察使,為巡撫王懿德所劾,解任。旋召來京,命赴河南幫辦剿匪。六年,復命赴安徽。七年,授浙江按察使,遷布政使,以短解甘餉降調。十年,江蘇團練大臣龐鍾璐請以宗幹辦理通、泰諸州縣團練。

同治元年,擢福建巡撫。三年,粵匪李世賢、汪海洋等由廣東入閩境,逼漳州,龍巖、雲霄、武平、永定、南靖、平和相繼陷,宗幹偕閩浙總督左宗棠以次剿平。五年,卒。宗棠偕將軍英桂奏:「宗幹循良著聞,居官廉惠得民,所至有聲。」優詔褒恤,謚清惠,祀福建名宦。

王凱泰,初名敦敏,字補帆,江蘇寶應人。道光三十年進士,選庶吉士,授編修。咸豐十年,以母喪歸。粵匪分犯江北,上命大理寺卿晏端書治江北團練,大學士彭蘊章薦凱泰使佐理。敘勞,累加四品卿銜。同治二年,從巡撫李鴻章軍幕。四年,浙江巡撫馬新貽薦調,命以道員發浙江,署糧道。曾國籓、李鴻章、馬新貽交章薦舉,五年,擢浙江按察使。紹興三江閘洩山陰、會稽、蕭山三縣水入江,歲久沙積,三縣民請濬治。凱泰履勘濬治,復舊利。六年,遷廣東布政使,裁陋規,省差徭,覈釐捐,丈沙田,濬城中六脈渠,增建應元書院。七年,擢福建巡撫,課吏興學,禁械斗、火葬、溺女、淫祀舊俗,奏請撥釐金糴米二十萬石實常平倉。充鄉試監臨,奏請整飭科場積弊。臺灣獄訟淹滯,奏請勒限清釐。

十二年,應詔陳言,略謂:「宜變通者六事:一,停捐例。自捐俸減折,百餘金得佐雜,千餘金得正印,即道、府亦不過三四千金。家非素豐,人思躁進,以本求利,其弊何可勝言?今日應以停捐為急務,以江西、湖南北、四川、廣東、福建六省釐捐年提數萬,又於海關、洋稅關撥數萬,似可彌京銅局捐項。至外省籌捐雖難周知,而福建自十年至今,收銀不過數萬,他省可以類推。以涓滴之微而害吏治,得不償失,請下部覈議。一,汰冗員。捐納,軍功兩途,入官者眾,部寺額外司員,少者數十,多則數百,補缺無期,徒耗旅食。各省候補人員,較京中倍蓰。按例,各省試用佐貳雜職,視各項缺數多寡,酌留十之二。請援照大挑知縣名次在後,暫令回籍候咨之例辦理。一,限保舉。軍興後保案層疊,名器極濫,捷徑良多。請下部覈議,此後保舉只準得應升之階及應升之銜,其餘班次概予刪除。至一品封典,二、三品加銜,皆不得濫請。一,復廉俸。自咸豐間軍用浩繁,京外俸廉,分別減成,京員困苦,知縣疲累,早荷聖明鑒及。今欲砥礪廉隅,似廉俸復額,亦其一端。福建文職廉額年支十三四萬兩,計現年徵起錢糧羨耗支抵尚屬有贏,道府以下各員,似可照額全支。請中外廉俸改復舊額,或加成支放。一,重學額。近年鼓勵捐輸,有加廣中額學額之制。中額三年一試,無慮濫竽。至一州一縣,士風本有不齊,乃以文理淺陋者濫廁其間,甫得一衿,包攬詞訟,武斷鄉曲,流弊不堪指數。請嗣後各省捐輸,只加中額,不加學額,並敕各省學臣酌覈。如有不能足額,奏明立案,俟文風日上,再行如額取進。一,立練營。營兵皆招自本籍,月餉不足贍八口,勢必另習手藝,兼營負販。每逢操演,不過奉行故事。設有徵調,兼旬累月,始克成行。兵與將不相習,兵與兵亦不相識,人各一心,安能制勝?近年削平禍亂,全賴湘、淮各勇。國家養兵,糜帑歲數千百萬,竟不得其用,其弊實由於此。往年江寧克復,臣函商曾國籓,備言江寧綠營應稍變通,以現存得勝之勇,改充額兵,設營分部,一洗舊習。國籓未及議行,旋調直隸,即設練軍,蓋亦採用臣說。左宗棠在閩浙任內,奏準減兵加餉,就餉練兵,洵為救時良策。請敕下各省督撫照減兵加餉之說,而以所減之餉加於戰兵。按湘、楚營制,五百人為一營,擇地分扎,隨時互調,俾卒伍皆離原籍,不致散處市廛。餉不另增,兵有實用,庶化兵為勇,而武備可恃。」疏入,命下部議。

十三年,入覲,行至蘇州,疾作,乞罷,予假治疾。日本窺臺灣,命凱泰力疾回任。光緒元年,移駐臺灣,病劇,還福州。卒,贈太子少保,謚文勤。

郭柏廕,字遠堂,福建侯官人。道光十二年進士,選庶吉士,授編修,遷御史、給事中。出為甘肅甘涼道。二十三年,戶部銀庫虧帑事發,柏廕為御史稽察,未糾發,奪官分償,旋授主事。咸豐三年,會辦本省團練,以克廈門、防延平功,擢郎中。同治元年,引見,交欽差大臣曾國籓差委。二年,授江蘇糧道,擢按察使,遷布政使,護理巡撫。六年,擢廣西巡撫,調湖北,仍留署江蘇巡撫。方亂時,江、浙交界槍船群聚為匪,柏廕與浙江會捕,獲其首卜小二置之法。禁槍船,設牌甲,稽查約束。

是年,赴湖北任,署湖廣總督。各省遣散營勇,會匪蕭朝翥約黨分布黃梅、武穴、龍坪各水次,阻截散勇,偪令從為亂。柏廕遣兵往捕,其黨殺朝翥以降。諸縣教匪,京山吳世英、蘄水馮和義、沔陽劉維義次第擒誅。七年,奏言:「漢口鎮華、洋雜處,散勇游匪廁其間。每遇撤營,散布謠言,句結入會。疊經懲辦,在武漢、襄樊地方分設遣勇局,凡有在鄂散勇,均令赴局報名,雇船押送回籍,酌給川資,庶無業之徒,可歸鄉里,不至流而為匪。」又奏言:「淮南鹽引,楚岸為大宗。自長江被擾,運道梗阻,改用淮北票私,暫濟民食,淮南銷路遂滯。請復淮南引地,禁淮北票私,停北鹽抽課。襄、鄖、德三府前此兼銷潞鹽,亦一律禁止。」八年,多雨大水,柏廕遣吏分道治賑。九年,再署湖廣總督。十年,湖南會匪陷益陽、龍陽,柏廕分兵防守進剿,獲其渠。十二年,以病乞罷。光緒十年,卒。

子式昌,舉人。從軍積功,以知府發浙江。巡撫蔣益澧調赴廣東,署肇慶。益澧罷,式昌還浙江,補臺州。劇盜黃金滿以官吏貪酷,煽亂。式昌扼要隘,令民自守,以嚴法繩蠹吏,蠲斥苛斂。金滿乃詣彭玉麟請降。光緒二十六年,衢州民殺教士,戕西安知縣吳德潚。擢式昌金衢嚴道,諭士民安堵,得亂首誅之。三十一年,署按察使。卒。

武昌子曾炘,官至禮部侍郎。

論曰:王慶雲、譚廷襄並易又歷中外,慶雲綜覈精密,治防井井,尤為可稱。馬新貽、李宗羲皆以循吏贊畫軍事,擢任大籓,治績卓著。宗羲諫園工,籌海防,建言遠大。徐宗幹、王凱泰清節惠政,皆有時望。郭柏廕久任疆圻,澤施於後焉。


\end{pinyinscope}