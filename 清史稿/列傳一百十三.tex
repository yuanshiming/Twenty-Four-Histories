\article{列傳一百十三}

\begin{pinyinscope}
開泰阿爾泰桂林溫福

開泰,烏雅氏,滿洲正黃旗人。雍正二年進士,改庶吉士,授編修。九年,遷侍講。上御門,開泰未入侍班,黜令乾清門行走。十三年,復編修。乾隆元年,遷國子監司業。八年,遷祭酒。督江蘇學政。再遷內閣學士。三遷兵部侍郎,仍留學政任。十年,授湖北巡撫。疏言:「社倉較常平尤近於民,而弊亦易滋。湖北社倉穀麥五十二萬石有奇,散在諸鄉,恐多虧缺。應飭道府按部所至,便宜抽驗。」調江西。十三年,又調湖南。疏言:「戶部咨各省常平倉穀,以雍正舊額為準。湖南溢額穀五十五萬餘石,令糶價儲庫。臣維雍正舊額七十餘萬石。湖南夙稱產米,乾隆二年至八年,諸省赴湖南購米,先後計百七十五萬有奇。中間又撥運福建、江蘇。若盡糶溢額之穀,遇本省需用或鄰疆告糴,必致倉儲缺額,買補不易。」疏上,以留心積貯嘉之。十五年,有壽掄元者,自言南河同知,赴湖南採木,布政使孫灝諭永州府為料理。尋得其詐偽狀,開泰以聞,但言灝殊為未諳。上以灝瞻徇,何得但言未諳,知為開泰門生,斥其徇庇,下吏部嚴議,議奪官,命留任。尋調貴州。十八年,疏言:「古州募軍屯田,戶上田六畝,中田八畝,下田十畝。今食指日多,生計艱難,請準屯戶入伍充兵。」許之。擢湖廣總督,加太子少傅。

二十年,調四川。金川土司莎羅奔與革布什咱土司色楞敦多布初為婚媾,繼乃相怨構兵。旁近綽斯甲布、鄂克什、雜穀、巴旺、丹壩、明正、章谷、小金川諸土司皆不直莎羅奔。二十三年,莎羅奔攻吉地。吉地,色楞敦多布所居寨也。開泰與提督岳鍾琪檄游擊楊青、都司夏尚德等率兵分屯章谷、泰寧,令鄂克什、雜穀援革布什咱,攻金川,莎羅奔引退。尋復攻破吉地,色楞敦多布走泰寧求援,開泰復檄諸土司出兵助之,調雜穀土練千人分屯丹壩、章谷、泰寧,發黎、雅、峨邊兵屯打箭爐,諭郎卡撤兵。郎卡,莎羅奔從子,為副酋,主兵事者也。事聞,上謂:「番目相攻,於打箭爐何與?」疑郎卡擾邊,命開泰具實覆奏。開泰尋疏報章谷、巴旺土兵擊敗金川,莎羅奔焚吉地走,盡復革布什咱境,留綽斯甲布、明正兩土司兵分守之,使色楞敦多布歸寨。上諭曰:「番民挾仇攻擊,不必繩以內地官法。宜以番攻番,處以靜鎮。」旋加太子太保。二十四年,松潘鎮總兵楊朝棟入覲,開泰與鍾琪奏朝棟衰

老,難期勝任。上責開泰何以不先奏,下吏部議,奪官,命仍留任。

二十七年,莎羅奔死,郎卡應襲。例,土司承襲,鄰封諸土司具結。開泰以郎卡與諸土司皆不協,令毋取結,疏聞,上許之,命嚴諭郎卡知恩守法。未幾,郎卡侵丹壩,取所屬瑪讓,開泰檄綽斯甲布往援,使守備溫欽等赴金川詰責。上諭曰:「郎卡狼子野心,即使詰責伏罪,豈肯永守約束?諸土司援兵既集,能協力剿除,分據其地,轉可相安;若諸部不能並力剿除,而郎卡怙惡不悛,亦非開泰、岳鍾琪四川綠營兵能任其事,應臨時奏請進止。」二十八年六月,開泰奏九土司大舉擊破金川。上聞郎卡使人詣成都,開泰許進謁,撫慰之,而陰令九土司進兵,諭曰:「郎卡於綽斯甲布等屢肆欺凌,眾土司合力報復。開泰既聞其事,惟應明白宣示,諭令悉銳往攻;而於郎卡來人嚴為拒絕,且諭以爾結怨鄰境,誰肯甘心?斷不能曲為庇護。如此,則郎卡既不敢逞強,綽斯甲布等亦可洩忿。乃既用譎以籠絡郎卡,又隱為各土司援助,郎卡素狡黠,豈能掩其耳目?殊非駕馭邊夷之道。」命奪官,以頭等侍衛赴伊犁辦事。尋卒。

阿爾泰,伊爾根覺羅氏,滿洲正黃旗人。雍正間,以副榜貢生授宗人府筆帖式。乾隆中,屢遷至山東巡撫。以山東產山綢,疏請令民間就山坡隙地廣植桲欏,免其升科。歲大水,阿爾泰先後濬兗州、沂州支渠三十有九,曹州、單縣順堤河二百餘里;培南旺、蜀山湖民墊;導章丘珍珠、麻塘二泉,新城五龍河溉民田;並及高苑、博興、惠民諸縣近水地,皆令蓺稻。築洸河堤至於馬場湖,以衛濟寧州城,析白馬湖引入獨山湖以疏泗水,開汶上稻田數百頃。濟東諸州縣瀕徒駭、馬頰兩河,支流相貫注,及哨馬營、四女寺支河,皆次第疏治。濬衛河自德州至於館陶凡三百餘里。洩壽張積水自沙、趙二河入運,洩東平積水入會泉、大清諸河,洩濟南、東昌諸州縣積水。開支河三十餘,循官道為壕,引水自壕入支河,自支河入徒駭、大清諸河。漳、汶合流,開引河,增子墊,以防盛漲。阿爾泰撫山東七年,治水利有績,擢四川總督,加太子太保。

阿爾泰至四川,議平治道路:陸道北訖廣元,西達松潘,東抵夔州,護其傾欹,補其缺落,兼葺大渡河瀘定橋;水道自萬縣入湖廣境,鑿治險灘凡一百有奇。議以牧廠餘地招佃為田。議置義倉,捐穀千餘石以倡。議開南川金佛山磺礦。議築都江大堰。議松潘、雜穀、打箭爐三置倉儲麥稞,備邊儲。上皆從其請。

初,征金川,以頭人郎卡出降,罷兵。三十一年,復為亂,掠丹壩、巴旺。阿爾泰策以番攻番,令旁近綽斯甲布諸土司攻之。秋出行邊,至雜穀腦。郎卡使請還所侵丹壩碉卡。復與提督董天弼進至康巴達,郎卡出謁,阿爾泰許如所請,並畀以新印。疏聞,上戒毋遷就茍安。三十五年,小金川頭人僧格桑掠鄂克什,阿爾泰赴達木巴宗,僧格桑出謁,還侵地。尋授武英殿大學士,仍領總督。三十六年,召還京,入閣治事。既,復令出領總督。金川頭人索諾木攻革布什咱,僧格桑亦圍達木巴宗,侵明正土司。阿爾泰疏言:「兩金川相比,如議出師,需兵既多,糜餉亦鉅。茲令董天弼臨之以兵,仍使游擊宋元俊宣諭索諾木。」上責阿爾泰議非是,決策用兵,令定邊右副將軍溫福視師,佐以侍郎桂林,諭斥阿爾泰掩飾偷安,奪大學士、總督,留軍治餉,以桂林代為總督。師克約咱,上以阿爾泰鑄大砲利軍行,予散秩大臣銜。

三十七年,與總兵宋元俊劾桂林覆軍諱敗,上為罷桂林,即命阿爾泰攝總督。俄移督湖廣。阿爾泰疏言:「各路轉餉,當招商承運。西路去內地近,南路山險途長,商不肯應募,當增運值。火藥已運罄,當令雲南、陜西協助。」上謂:「阿爾泰專領轉餉,何不早籌畫?今福隆安、阿桂皆至南路,始以一奏塞責。」命毋往湖廣,仍以散秩大臣留軍督餉。未幾,阿桂疏言軍至卡丫,無五日之糧;又言綽斯甲布轉餉將一月猶未至。阿爾泰亦自陳請奪職從軍。上責其倚老負恩,始終不肯以國事為念,命逮問。

阿爾泰初至四川,上以天壇立燈竿,下四川求楠木。阿爾泰附運木材以進,言出養廉採獻。既乃私語人,謂他日且以此負累。語聞上,上心慊之。至是,詔罪狀阿爾泰,猶及此事,斥為昧良飾詐。川東道托隆入見,發阿爾泰贓私,下繼任總督富勒渾嚴鞫。三十八年,獄具,擬斬,上命賜自盡。

桂林,伊爾根覺羅氏,滿洲鑲藍旗人,兩廣總督鶴年子。桂林自廩生入貲為工部主事。累遷山西按察使。乾隆三十六年三月,擢戶部侍郎、軍機處行走。九月,命佐定邊右副將軍溫福討金川。十一月,授四川總督。小金川頭人在卡外投文餽土宜,桂林卻不受,檄罪狀其酋僧格桑。旋督兵收約咱,進克其東山梁大小碉五、石卡二十餘。疏請添調黔、陜兵五千益師,上許益陜、甘兵三千。桂林旋督總兵宋元俊攻卡丫,進據墨爾多山梁。上嘉其措置合宜,手詔謂:「無意中用汝,竟能得力。亦賴在軍機處半年,日耹朕訓也。」

三十七年,克卡丫,復破克郭松、甲木、噶爾金。進克噶爾金後山梁,分兵攻東山梁,襲阿仰,自墨壟溝進取達烏圍。是時大金川酋索諾木攻陷革布什咱,屯兵其地。桂林議乘索諾木兵力未備、革布什咱人心未定,與元俊分兵五道並進,並約將軍溫福合擊,密令革布什咱降酋旺勒丹等約其戚加琿爾為內應,遂收革布什咱寨落七十餘里。旋令元俊及守備陳定國率綽斯甲布土兵屯甲爾壟壩,進攻默資溝、吉地,斷其水道,進攻丹東。上獎桂林甚合機宜,促元俊乘勝深入取索諾木。

桂林遣裨將自東山梁墨壟溝越嶺進攻,別遣兵出間道,自札哇窠山梁縋崖設伏師。既度東山梁墨壟溝,札哇窠伏兵亦起,賊敗竄,克大碉一、石卡二十一。別遣參將常泰環攻黨哩,都司李天貴等攻沙沖,革布什咱頭人為內應,賊盡殲。黨哩、沙沖地並復。總兵英泰等復攻克達烏官寨。上嘉其功,賜御用玉韘。再進攻克格烏巴桑及那隆山嶺。元俊別攻克丹東及覺拉喇嘛寺,誅賊渠三百、番眾百三十餘。革布什咱地盡復,桂林檄定國將所調綽斯甲布兵駐界上聽調。上以革布什咱既復,正當乘勝進剿金川,攻其無備,責桂林失算。

桂林復督兵攻達烏東岸山梁,參將薛琮戰沒,琮驍將,深入糧盡。桂林既失期不會師,又不以時遣援,軍盡覆,疏請治罪,述戰狀不敢盡。元俊與散秩大臣阿爾泰劾其虛誑,並言桂林在卡丫建屋宇以居,迫屬僚供應,與副都統鐵保、提督汪騰龍等終日酣飲,諸將罕得見;密令騰龍畀總兵王萬邦白金五百,贖被掠官兵,希圖掩飾。上奪桂林職,命額駙、尚書、公福隆安馳往按治,尋奏所劾皆虛,惟官兵傷損不即察奏屬實;至贖被掠官兵,乃在軍戶部郎中汪承霈聞巴旺、布拉克底土兵歸失道,官兵告桂林,發白金五百交騰龍備賞,事為元俊構陷,請分別治罪。上以桂林在軍日親曲糵,止圖安逸,不能與士卒同甘苦,致北山梁傷損多兵,不得為無罪,命戍伊犁。三十八年七月,予三等侍衛銜,仍詣軍前督糧運。四十年,授頭等侍衛。尋授四川提督,遷兩廣總督。卒,加太子太保銜,謚壯敏。

溫福,字履綏,費莫氏,滿洲鑲紅旗人,文華殿大學士溫達孫也。自繙譯舉人授兵部筆帖式。乾隆初,累遷戶部郎中。外擢湖南布政使,歷四年;移貴州布政使,亦四年。坐平遠民閧訟庭、按治草率,奪職,戍烏里雅蘇臺。二十三年,起內閣侍讀學士。從定邊將軍兆惠討霍集占,戰葉爾羌,槍傷顴。擢內閣學士,遷倉場侍郎,予雲騎尉世職。外授福建巡撫,內遷吏部侍郎、軍機處行走,進理籓院尚書。

三十六年,師征金川,授定邊右副將軍,以侍郎桂林佐之,共討賊。溫福自汶川出西路,桂林自打箭爐出南路。時小金川頭人澤旺子僧格桑割地乞援於大金川頭人索諾木,索諾木潛遣兵助之。上命先剿小金川,且勿聲大金川罪。溫福至打箭爐,分兵三道入:溫福出巴朗拉,提督董天弼自甲金達援達木巴宗,總督阿爾泰自約咱攻僧格桑。十一月,擢武英殿大學士。十二月,至巴朗拉,戰三晝夜,賊敗去。三十七年正月,取達木巴宗。進攻斯底葉安,而分軍出別斯滿、瑪爾瓦爾濟,兩路夾擊,進克資哩。再進克東瑪,再進克路頂宗及喀木色爾,取諸碉寨。再進得博爾根山梁,並攻克得瑪覺烏寨落,攻公雅山。十二月,授定邊將軍,以阿桂、豐升額副之。進克明郭宗,再進克底木達。底木達者,僧格桑父澤旺所居寨也。師至,俘澤旺,檻致京師,誅於市,而僧格桑奔大金川。溫福檄索諾木令縛獻僧格桑,不應。

上將進討大金川,溫福等疏言:「前此張廣泗征金川,十路、七路分合不常,實祗有六路,皆以抵勒烏圍、噶爾依為主。一為卡撒正路,自美諾至噶爾依,約五程,為傅恆進兵路;一為丹壩,自維州橋經番地抵勒烏圍,約二十餘程,中有穆津岡天險,為嶽鍾琪進兵路;一地名僧格桑,自美諾抵噶爾依,六七程,即總兵馬良柱所行路;一為革布什咱,一為馬爾邦,皆距噶爾依六七程,險狹難行;一為綽斯甲布寨至勒烏圍三程,至噶爾依亦三程,均隔大河,碉寨林立,難攻。此外又有俄坡一路,從綽斯甲布寨至勒烏圍,僅二程,路較平。今當由卡撒正路進兵,其俄坡一路,既有綽斯甲布土司原出兵復其侵地,可為犄角。其餘各路,分兵牽制,使不能兼顧。」於是溫福自功噶爾拉入,阿桂自當噶爾拉入,豐升額自綽斯甲布入。溫福性剛愎,不廣咨方略,惟襲訥親、張廣泗故事,以碉卡攻碉卡,修築千計。所將兵二萬餘,強半散在各碉卡。每逾數日當奏事,即督兵攻碉。士卒多傷亡,咨怨無鬥志。溫福日置酒高會,參贊伍岱嘆曰:「焉有為帥若此而能制勝者?」因密疏聞上,溫福亦疏劾伍岱。上命豐升額及額駙色布騰巴勒珠爾按治。溫福又言色布騰巴勒珠爾朋比傾陷,上為奪伍岱職,令色布騰巴勒珠爾逮詣熱河行在,獄成,戍伍岱伊犁。

三十八年春,溫福師至功噶爾拉,賊阻險,不得進,別取道攻昔嶺,駐軍木果木;令提督董天弼分軍屯底木達。木果木、底木達皆故小金川地,索諾木陰使小金川頭人煽諸降番使復叛。諸降番以師久頓不進,遂蜂起應之。先攻底木達,天弼死之,次劫糧臺,潛襲木果木。溫福不嚴備山後要隘,賊突薄大營,奪砲局,斷汲道。時大營兵尚萬餘,運糧役數千,爭避入大營,溫福堅閉壘門不納,轟而潰,聲如壞堤,於是軍心益震。賊四面蹂入,溫福中槍死,各卡兵望風潰散。參贊海蘭察聞警赴援,殿餘兵自間道出。小金川地盡陷。上初聞溫福死,詔予一等伯爵,世襲罔替,祀昭忠祠。既,劉秉恬、海蘭察、富勒渾各疏言溫福僨事狀,命奪伯爵,予三等輕車都尉世職。四十一年,命並罷之。子勒保、永保,皆有傳。

論曰:金川再亂,開泰、阿爾泰皆主以番攻番,遲回坐誤。桂林有宋元俊不能用,反齮齕之,擁兵不進。阿爾泰與元俊劾桂林,此其意以軍國為重,不屑屑阿貴近、疏卑遠,宜若可成功,乃坐蜚語敗。溫福銳進,似勝開泰輩,乃又剛愎,有董天弼不能用,予兵至少,令僻處軍後,卒致僨潰,徒以身殉,豈不惜哉?


\end{pinyinscope}