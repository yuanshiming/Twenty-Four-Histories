\article{列傳一百八十八}

\begin{pinyinscope}
向榮和春張國樑

向榮,字欣然,四川大寧人,寄籍甘肅固原。以行伍隸提標,為提督楊遇春所識拔。從征滑縣、青海、回疆,常為選鋒。積功擢至甘肅鎮羌營游擊。道光十三年,直隸總督琦善知其才,調司教練,累遷開州協副將。海疆戒嚴,率兵駐防山海關。擢正定鎮總兵,調通永鎮。二十七年,擢四川提督。三十年,調湖南,平李沅發之亂,調固原。

廣西匪起,巡撫鄭祖琛不能制。榮於舊將中最負時望,文宗特調為廣西提督,倚以辦賊。是秋至軍,由柳州、慶遠進剿,以達宜山、象州,連破賊於索潭墟、八旺、陶鄧墟、猶山等處,賊氛稍戢。惟洪秀全等踞桂平金田,狡悍為諸賊冠。榮移兵往剿,賊以大黃江、牛排嶺為犄角。咸豐元年春,攻大黃江,賊分出誘戰,率總兵李能臣、周鳳岐合擊,大破之,殲千數百人,賜號霍欽巴圖魯。水陸合攻牛排嶺,搗其巢,又追擊於新墟、紫金山,賊乃竄踞武宣東鄉。時周天爵為巡撫,與榮同剿賊,議不合,數戰未得利,廣州副都統烏蘭泰率兵來會。四月,賊突圍竄象州。榮被譴,褫花翎,降三級留任,天爵亦罷軍事。大學士賽尚阿代李星沅督師,命榮與烏蘭泰節制鎮將以下,迭詔戒榮同心協力,以贖前愆。賊踞象州中坪,其要路東曰桐木,西曰羅秀,榮與烏蘭泰分扼之。六月,榮由桐木進兵,偕烏蘭泰合剿,迭敗賊於馬鞍山及架村、黃瓜嶺、西安村,遂回竄桂平新墟、紫金山,恃險負嵎。榮偕烏蘭泰等迭奪豬仔峽、雙髻山要隘,進破風門坳。八月,賊冒雨竄逸,官軍失利於官村,遂陷永安州,坐褫職留營。十一月,合攻永安,獲勝,復原官。

初,榮所部湖南兵,因榮子繼雄用事,軍心不服,故武宣、象州之役戰不力,皆歸咎之。文宗排眾議,仍加倚任,而調四川兵以易湖南兵。賽尚阿不知兵,專倚榮與烏蘭泰。二人復不協,圍永安久不下。榮建議缺北隅勿攻,伺賊逸擊之。二年二月,天大雨,賊由北突出,逕犯桂林。榮由間道馳援,先賊至,賊冒榮旗幟襲城,擊走之。偕巡撫鄒鳴鶴急治守具,屢出奇兵擊賊城下,俘斬甚眾。經月餘,援軍集,賊乃解圍北竄。詔嘉其保城功,已奪職復之,予議敘。賊由興安、全州入湖南。榮頓兵桂林,為總督徐廣縉論劾,褫職戍新疆。賽尚阿疏請暫緩發遣,令援湖南。九月,至長沙,破賊瀏陽門外,又破之於見家河、漁網洲、嶽麓山。至冬,圍乃解。賊北竄,陷岳州,入湖北,進犯漢陽、武昌,官軍遙尾之,莫敢擊。賽尚阿、徐廣縉先後罷黜,諸將無一能軍。詔以榮屢保危城,緩急尚欲恃之,予提督銜,幫辦軍務,責援武昌。尋復授廣西提督。榮至,數奏捷,而武昌尋陷,褫職,仍留軍。調署湖北提督,未幾實授,命為欽差大臣,專辦軍務。賊既踞武昌,勢益熾,不可復制矣。

三年正月,大舉東犯,連舟蔽江,棄城而去。榮以克復聞,詔促躡追。榮所部兵多疲弱,遣撤六千餘名,料簡精銳,率總兵和春、李瑞、秦定三、玉山、福興沿江躡賊;令提督蘇布通阿率川兵,總兵晉德布率滇兵來會。至九江,無舟,留半月,賊已掠安慶,陷江寧,為久踞計。榮至江寧,屯孝陵衛。時鎮江、揚州皆為賊踞,詔琦善剿江北,榮剿江南,分任軍事。榮所部一萬七千餘人,攻通濟門外及七橋甕賊壘,連破之,進屯紫金山,結營十八座,賜黃馬褂。江寧城內士民謀結合內應,屢爽期,迄無成功。賊已分股由安徽北擾河南,而鎮江、揚州南北互應,大江上下游賊勢相首尾。榮遣提督鄧紹良率兵八千規鎮江,總兵和春以舟師伺便夾擊,屢戰,進壁城下。六月,紹良軍為賊所襲,退守丹徒鎮,榮令和春往援,遂代領其軍。賊注意蘇、常諸郡,以和春軍相持不得進,乃欲取道東壩。十月,賊船入蕪湖,陷高淳,遣兵擊走之,令鄧紹良駐防。既而皖北賊熾,和春赴援,榮請以提督餘萬清代督鎮江軍。

四年七月,賊犯東壩,遣副將傅振邦等協剿,賊退高淳,進復其城。賊乘江寧大營空虛,大舉來撲,率諸軍拒之,擒偽丞相譚應桂,俘斬三千餘。總兵葉長春、吳全美以水師克下關水柵砲臺,殪偽燕王秦日綱,進扼三山,營江路上游。賊聚太平府,與江寧相應。張國樑連克賊壘,乘勝復太平,殲賊首韋得真等。江寧賊出營於上方橋,三路來撲,又撲七橋甕,分擊敗之,三戰殪賊二萬餘。八月,毀上方橋賊壘,進逼雨花臺,搗其巢,追奔至城下。賊復由觀音門出趨棲霞,令總兵德安追擊,敗之於高資汛,又與餘萬清合擊於夾江,擒斬殆盡。萬清亦屢敗賊於鎮江。

五年春,湖北竄賊入蕪湖,鄧紹良御之於黃池。瓜洲賊出占魚套犯高資,擊走之。五月,賊由蕪湖犯灣沚,卻之。吳全美率水師破賊於東梁山,德安、明安泰率陸師進攻蕪湖,會鄧紹良大破安慶援賊,遂復蕪湖。餘賊猶濱江結壘,以廣福磯、弋磯為犄角,數路死力來援,紹良、全美等水陸苦戰,迭敗之而不能克也。時巡撫吉爾杭阿既克上海,詔幫辦軍務,專任鎮江一路,督攻甚急,江寧賊百計赴援。十一月,榮督總兵德安、張國樑、秦如虎等,迭擊之於燕子磯、觀音門、甘家港、棲霞街、石埠橋等處,賊竄回江寧,令德安駐軍東陽鎮扼之。十二月,上游蕪湖、兩梁山、金柱關及江北瓜洲、金山、廬州、三河諸賊同趨江寧,約城中悍賊沖出:一由神策門至仙鶴門抄綴大營;一由觀音門沿江至棲霞,直趨鎮江;一由南路秣陵關來犯。榮令張國樑、秦如虎迎擊於仙鶴門,大捷,回擊石埠橋,賊亦退,又敗之於龍脖子及元山、板橋等處。檄鄧紹良自蕪湖回援,餘萬清自鎮江移駐龍潭、下蜀街。

六年春,賊踞倉頭,為往來要道,餘萬清、張國樑迭擊不退,鄧紹良至,令統前敵諸軍,屢戰不利。賊日增多,蔓延炭渚、橋頭,改以張國樑為總統,國樑力戰,連破橋頭、下蜀街、三汊河、張楊村諸壘,賊始竄走,復合鎮江賊入瓜洲,將軍托明阿軍潰,江北大震。榮令紹良援揚州,偕德興阿復其城。國梁援六合,進克江浦、浦口,江北稍定。四月,寧國告陷,蘇、浙戒嚴,令紹良馳御之。江長貴亦退守黃池,而鎮江軍事復急。國樑進攻小丹陽未下,吉爾杭阿戰歿於煙墩山,鎮江京峴山營壘皆失,榮令餘萬清代領其軍。明安泰扼小丹陽,福興、張國樑率兵防剿,以固蘇、常門戶。國樑破賊於丹徒鎮,進扼馬陵,而賊已陷溧水,由高資、下蜀街趨江寧,分屯太平、神策門外。

五月,上游賊麕至,屯城北。榮大營兵僅數千,急促國樑回援。賊分十餘路來撲,營壘盡失,退守淳化鎮,再退丹陽,自請治罪,詔原之,褫職,仍留欽差大臣,督辦軍務。丹陽當鎮江、江寧兩路要沖,榮率張國樑、虎嵩林扼守。令西林防句容,明安泰攻溧水,江長貴扼溧陽,張國樑仍總統諸軍。賊更番至,恃國樑力禦卻之。疏請增兵,未至,榮憂憤成病,七月,卒於軍。

遺疏上,文宗震悼,詔嘉其忠勤,雖未恢復堅城,數年保障蘇、常,盡心竭力,復原官,依例賜恤,予一等輕車都尉世職,謚忠武。命建專祠,又入祀江蘇名宦祠。克復江寧後,賜祭一壇,入昭忠祠。子繼雄,候選道,襲世職。

和春,字雨亭,赫舍里氏,滿洲正黃旗人。由前鋒、藍翎長授整儀尉,累遷副護軍參領。出為湖南提標中軍參將,擢永綏協副將。

咸豐元年,從向榮赴廣西剿匪,戰武宣東鄉,賜花翎。破賊於中坪,進攻紫金山,奪雙髻山、豬仔峽要隘,功最,賜號鏗色巴圖魯。又奪風門坳,克古調村賊巢,擢綏靖鎮總兵。二年,援桂林,力戰解圍,加提督銜。追賊至全州,敗之。賊入湖南,迭戰於道州、桂陽,遂犯長沙,和春從向榮赴援,數出奇破賊。賊去陷岳州,坐追剿遷延,褫職留軍。

三年春,會攻武昌。賊棄城東下,追至九江,遇賊,襲擊之。從向榮抵江寧,分軍攻通濟門外賊壘。尋偕總兵葉長春、吳全美等率舟師攻鎮江,破賊甘露寺下。駐金山扼江路,又掠占魚套,擊敗賊船。偕總兵瞿騰龍攻太平門,填壕逼城,殲賊甚眾。六月,提督鄧紹良師潰於鎮江,詔和春署江南提督,率所部廣東、湖南兵馳援。移軍丹徒鎮,進復京峴山舊壘。賊數千來爭,殲戮殆盡。賊銳稍挫,兩軍相持,蘇、常得無事。尋實授提督。

是年冬,安徽軍事急,命和春分兵移防滁州,遂進援廬州。巡撫江忠源困守危城,陜甘總督舒興阿率援軍至,不敢戰,忠源疏言和春忠勇可恃,請命總督援軍,詔允之,而所部僅千人,請舒興阿分兵,不聽。未幾,廬州陷,忠源殉。軍事專屬和春,福濟繼任巡撫,為之副。

四年,疏言:「皖省軍情重大,兵勇雖有萬餘,多未經戰陣。請調鎮江舊部湖南兵,並撥金陵得力官兵三千,交總兵秦定三、鄭魁士率之來助剿。」時廬州屬縣皆陷,與安慶踞賊連絡一氣,城大賊眾,和春駐軍三里岡,屢率鄭魁士等進剿,賊抗拒不下。乃沿河築壘構橋,分三路更番攻擊。夏,知州茅念劬率民團克六安,秦定三破賊於三連橋,進攻舒城。賊由霍山撲六安,擊走之。扎筏載大砲轟廬州城,賊分出拒戰,迭敗之。別遣軍復英山、廬江,而和州、含山一路賊時窺伺,疏請飭袁甲三嚴防烏江,以斷賊援。冬,臧紆青、劉玉豹由廬南規桐城,連奪大關等隘,逼城下,而紆青戰歿,玉豹退保六安,和春為賊牽制不能救。秦定三攻舒城,亦久不下,迭詔切責。初,和春專剿廬州,袁甲三扼臨淮,軍事多相關,而意不合。五年,偕福濟疏劾甲三,罷之,命和春遣員接統其軍。夏秋連擊敗援賊,督諸軍急攻廬州,至十月克之,城陷將兩年矣。詔嘉和春功能補過,賜黃馬褂,予騎都尉世職。六年,復舒城,大破賊於三河,克之,再復廬江。會向榮卒於軍,命和春代為欽差大臣,督辦江南軍務。

自向榮兵挫,退守丹陽,江寧賊益驕,內閧,自相殘殺,故榮歿後,張國樑等得以撫輯餘軍,規復東壩、高淳。和春至,餉械並絀,詔下各省接濟月餉四十萬兩,江蘇糧臺不能時給,疏劾總督怡良、巡撫趙德轍,詔勉其和衷。溧水、句容為賊精銳所聚,力攻數月,七年夏,先後克之,加太子少保。圍攻鎮江,賊數糾悍黨來援,督諸將迭破之。十一月,克鎮江,賜雙眼花翎。將軍德興阿督江北諸軍攻瓜洲,同日克復,軍聲大振。進攻江寧東北路,奪太平、神策兩門外賊壘。八年春,賊迭出城,力鬥卻之。合水陸諸軍克秣陵關,加太子太保。又破賊三汊河,奪要隘,江寧之圍漸合。

賊由皖南犯浙境,用以牽掣大軍。詔和春兼辦浙江軍務,先遣兵二千往援,命親往督師,以病未行。尋浙事緩,罷其行。賊復沿江來援,擊走,築長圍困之。七月,賊大舉出撲,張國樑破之城下。八月,陳玉成糾合捻匪犯江浦、浦口,德興阿兵潰,儀徵、揚州、六合先後陷。和春遣馮子材渡江赴援,復失利。張國樑繼往,力戰,復揚州、儀徵。九月,和春授江寧將軍。江寧賊乘間出撲,溧水亦陷,急調國樑回援。十月,復溧水,而上游賊犯黃池、灣沚,鄧紹良戰沒。

九年春,招降捻首薛之元,獻江浦城,復約李世忠破賊,復浦口。因劾德興阿縱寇狀,詔罷德興阿。江北不復置帥,諸軍並歸和春節制。提督鄭魁士亦克灣沚、黃池,進規蕪湖,軍事轉利。疏言:「揆察現勢,先盡力於金陵一路,絕其根株,則枝葉自萎。欲破金陵,必先斷浦口。請添募精銳萬人,由張國樑統率,一面力攻,一面進扎營壘,斷賊糧路,兼卻外援。臣當相度事機,剋期蕆功。」詔允之。是年冬,陳玉成由六合犯揚州,分黨渡江窺秣陵關,欲抄大營後路,東壩、溧水皆告警。尋大舉犯江浦,提督周天培死之,遂陷浦口。張國樑、馮子材援剿獲勝,揚州解嚴,浦口仍為賊踞。

十年春,國樑督水陸軍攻九洑洲,大捷,破其老巢。九洑洲為江寧水陸咽喉,既得,已成合圍之勢,而賊復由皖南犯浙,遽陷杭州,蘇、常震動。詔和春仍兼辦浙江軍務,先後分兵萬餘,提督張玉良總統赴援,甫至,賊即棄杭州。閏三月,由廣德分犯建平、東壩、溧陽,遂窺常州,急調張玉良回援,賊已分路逕趨江寧。時賊酋陳玉成、李秀成、李侍賢、楊輔清,糾諸路眾十餘萬,力破長圍,城賊應之。大營軍心不固,惟恃張國樑力禦。戰數晝夜,諸營同時火起。總兵黃靖、馬登富、吳天爵陣亡,全軍大潰,退守鎮江。和春坐褫職留軍。又退丹陽,賊踵至,張國樑死之。和春奪圍走常州,督兵迎敵,被重創,退至無錫,卒於軍。總督何桂清棄城走,常州、蘇州相繼陷。江南軍自向榮始任,凡歷七年,至是熸焉,蘇、浙遂糜爛。事聞,詔念和春前功,雖兵機屢挫,尚能血戰捐軀,復原官,依例賜恤,予騎都尉兼雲騎尉,合前世職並為二等男爵,謚忠壯,附祀江寧昭忠祠。子霍順武,候選參將,襲爵。

張國樑,字殿臣,廣東高要人,初名嘉祥。少材武任俠,為里豪所辱,毀其家,走山澤為盜,不妄殺。流入越南,後歸鎮南關。按察使勞崇光聞其名,招降,剿匪多得其力。咸豐元年,破劇賊顏品瑤,斬於陣,盡殲其黨。積功擢守備,繼隸向榮軍。二年,從解桂林圍,復全州、永興,擢都司。赴援湖南,迭破賊於醴陵、益陽、湘陰。援武昌,戰於洪山,皆為軍鋒。

三年,至江寧,逼城而軍。國樑屯七橋甕,攻鍾山賊壘,先登受傷,溫旨垂問,益感奮,遇艱險,一往直前。擢湖南永州營游擊。雨花臺為近城要地,屢力攻,幾克之,賜號霍羅琦巴圖魯。四年夏,復太平。太平在江寧上游,賊踞之以通糧運。府城三面阻水,惟東路通陸。賊聚千艘結四壘,設防甚密。國樑分三隊進,設伏縱火,自率精銳四百人突賊營,一戰克之,時稱奇捷。擢廣西三江協副將。又攻雨花臺,平賊壘,毀砲臺。剿南路竄賊,追入秣陵關,殲戮殆盡。五年,擢福建漳州鎮總兵。大軍急攻鎮江、瓜洲,江寧賊時出窺伺,江北賊亦乘隙進圖牽制。國樑隨方截擊,奔命不遑。六年,賊聚倉頭、炭渚、下蜀街,以斷鎮江、江寧之師。國樑總統諸軍合擊,旬日之間,殺賊萬餘,賊不得逞,乃渡江犯瓜洲,江北諸軍皆潰,又陷江浦、浦口。國樑馳援,連破賊於毛許墩、葛塘,復江浦、浦口。特詔嘉獎,加提督銜。未幾,巡撫吉爾杭阿戰歿,鎮江告急,溧水被陷,國樑回軍克之,而賊數路趨江寧,夾攻大營。向榮不能御,急調國樑回援,血戰累日,左足被槍傷,偕榮退保丹陽。時大江南北諸軍,賊所尤畏者,惟國樑一人。賊勢忽南忽北,多方肄我,皆牽制國樑之計,果為所敗。

榮既病,軍事一倚之。將軍福興與國樑不協,詔福興移軍江西,以國樑幫辦江南軍務。賊屢至,皆挫之。榮卒於軍,命和春代將,未至,國樑激勵將士,解金壇圍,復東壩、高淳,進攻句容。七年,擢湖南提督。克句容,賜黃馬褂。督諸軍規復鎮江。高資為鎮江、江寧要沖,兩路悍賊麕聚力爭,連營二十餘里,國樑大破之,斬偽安王洪仁等,又連破之於龍潭,援賊盡殲。鎮江糧盡援絕,遂克其城,城陷賊已歷五年。捷聞,文宗大悅,詔嘉國樑謀勇超群,予騎都尉世職。於是偕和春進規江寧。

八年,克秣陵關,賜雙眼花翎。復薄江寧城下,自春徂夏,迭戰破賊。築長圍,至秋乃成。皖賊大舉來援,江浦、浦口、儀徵、揚州、六合先後陷。國樑渡江援剿,復揚州、儀徵。調江南提督,晉三等輕車都尉。然賊仍踵故智,國梁兵至則走,去則復來。九年,提督周天培戰死江浦,國樑坐褫世職。

十年,合水陸諸軍克九洑洲,沿江賊爭投款,約期攻上下兩關,招撫五千餘人。軍中方謂堅城旦夕可下,而浙江告警,兵分益單,饋運不繼。和春用翼長王浚策,兵餉三分留一,約待克城後補給,士卒皆怨,國樑力諫不聽。閏三月,賊猝大至,四路受敵,大營不守,偕和春退丹陽。國樑以馮子材在鎮江未敗,進謀扼守。尋率師援丹陽,遇賊城外,兵忽潰,策馬渡河,沒於水。事聞,文宗震悼,猶冀其不死,命軍中偵訪,不得。逾數月,乃下詔優恤,追贈太子太保,祀昭忠祠,謚忠武,予騎都尉兼一雲騎尉世職。

國樑驍勇無敵,江南恃為長城。其歿也,數郡遂淪陷。士民哀思,私立廟祀。傳述戰績,與古名將同稱,往往附會過實,然益見威烈入人之深。同治三年,江寧克復,偽忠王李秀成就擒,言賊中咸重國樑,禮葬於丹陽尹公橋塔下,乃得遺骸焉。詔加給三等輕車都尉,合前世職並為一等男爵。祀江寧忠義祠,復與向榮合建專祠。子廕清,襲男爵。

論曰:粵匪初起,向榮與諸帥不和,致無成功,援桂林、長沙,為時所稱,故文宗終用之。其規江南也,近未破鎮江、瓜洲犄角之勢,遠未清長江上游,無以制賊死命,數年支拄,暫保吳疆,固昧遠猷,亦限兵力。和春繼克鎮江,又以援浙分兵,垂成之敗,禍更烈焉。張國樑一時健者,使盡其用,功不止此。善夫胡林翼之言曰:「未扼賊吭,江寧原難遽復。」觀湘軍之所以成功,與向榮、和春等之所以蹉跌,兵事固無幸焉者矣。


\end{pinyinscope}