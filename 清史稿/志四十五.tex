\article{志四十五}

\begin{pinyinscope}
地理十七

△福建

福建:禹貢揚州南境。明置福建行中書省,改承宣布政使司。清初為福建省,置閩浙總督。康熙二十三年,海島平,以其地置臺灣府。雍正十二年,升福寧州為府,永春、龍巖為直隸州。增置霞浦、屏南、福鼎。光緒十三年,升臺灣府為行省,與福建分治。後入日本。東至海;百九十里。西至江西石城;千五百五十五里。南至海;二百七十里。北至浙江景寧。四百六十里。廣九百一十里,袤九百七十五里。南至詔南縣南境,北極高二十三度四十四分。北至浦城縣北境,北極高二十八度。東至長樂縣東境,京師偏東三度一十七分。西至武平縣西境,偏西二十二分。宣統三年,編戶二百三十七萬六千八百五十五,口一千四百二十二萬九千九百六十三。領府九,直隸州二,一,縣五十七。

福州府:沖,繁,疲,難。清為省治。閩浙總督兼巡撫,布政、提法、交涉、提學四司,鹽、糧、巡警、勸業四道,福州將軍、副都統駐。道光二十三年,與英訂約五口通商之一。租界在閩江北岸,曰南臺,與府城對。航路:廈門、福州、三都澳。驛路:北逾仙霞嶺達浙江江山;西南達廣東黃岡。電線由福州北通杭州,西南通廣州,東通馬尾、川石山,東北通三都澳。海線由川石山東通臺灣淡水,由廈門東北通上海,西南通香港。北至京師六千一百三十四里。廣三百七十七里,袤四百十二里。北極高二十六度三分。京師偏東三度。領縣十。閩沖,繁,疲,難。倚。府東偏。東:鼓山,為郡之鎮。東南:九仙、大象、南臺。南:方山。海自浙江溫州迤西南入福寧,環府之羅源、連江,至縣東百九十里,為五虎門。其外大洋,其內閩江口。閩江,閩大川,上匯富屯、沙、建三溪,至侯官分二派入:北派承洪山江,東逕中洲為南臺江,至中岐為馬頭江,合大定江、演江,亦曰東峽江,至羅星塔;南派澤苗江入,為陶江,逕螺洲,左合黃山水,又東南為陰崎江,又東為烏龍江,右合榕溪,又東,亦曰西峽江,又東來會。又東過青洲,右納太平港水,逕員山,又北支津北抵亭頭鄉。又東為瑯琦江,復歧為二,一西北出五虎門,一東南與長樂分岸,為廣石江。梅花江出白猴嶼,並入焉。其下歷興化、泉、漳至粵,水程二千里,陸千二百里。閩海關總口二:一駐南臺,海防同知同;一駐閩安鎮,副將同。順治十五年築城,置戰船,南北岸砲臺。縣丞駐營前,雍正十二年徙三水部。關外、鎮口、中洲三鎮。巡司三:閩安、五虎門、永慶。三山、大田二驛。侯官沖,繁,疲,難。倚。府西偏。西南隅:閩山。北隅:越王。東北隅:冶山。南:方山。西南:怡山。西:清泉。西北:雪峰。北:蓮花、壽山。西北:梧桐嶺。城南閩江,西北上承閩清大溪入,逕大竺,左合陳溪,至小箬,仍錯出復入,右合黿溪,左大目溪。又東南,左合黃石溪,至過山洲,合陳塘溪,為馬瀆江。又東過懷安洲,歧為二:北派東南左合五峰山水,為石岊江,又為螺江、金鎖江,至城西為洪山江,分流復合,左合西湖水;南派右納樓梯嶺水,又南大樟溪自永福入,合浯溪、潢溪、喆溪、印溪,歧為澤苗港,先後來會,為澤苗江,又東南並入閩。北宦溪出蓮花山,北會板橋塘水,折西,右合長箕嶺水,逕下密,折東北為日溪,為密溪,入連江。西湖、東湖、南湖並堙。西河場。縣西:西江口,大使駐。閩畫地為埕,漉海水曝之,與江、淮、浙煮鹽異。縣丞駐大湖。梅嶺、大穆、芋原、遼沙四鎮。竹崎、五縣寨二巡司。白沙、芋原二驛。長樂疲,難。府東南百里。東:壺井山。東南:龍泉。西南:岱遇。東北:越遷。北、東、南際海。北界閩之馬江口。太平港自七巖山循界西北分入,一入其營前,一合資聖溪及文洽浦,自東水關貫城東出。又北,合考溪入洋嶼。又東至籌港為廣石江、梅花江、陳塘港,入貓嶼。其外東沙、北犬、南犬,南為磁澳。江西有仙岐寨、蕉山寨,西至漳港為漳江。又南至壺井澳為壺井江。又南至鐵爐嶼為巴頭港,三溪入。其外雙帆石、東洛嶼、西洛嶼。又南逕御國山、小祉、大祉,為松下江口,至福清界。宋建炎初,陳可大始興水利。乾道四年,徐謩為斗門及湖塘陂堰,溉田都二千八十三頃。磁澳鎮。廣石、籌港、澤里、厚福四汛。貓嶼、蕉山、小祉廢巡司。福清沖,繁,疲,難。府南少東百三十五里。城北隅:靈鷲山。東:瑞巖。東南:郭廬、海壇、南日。東南際海。自長樂迤西南為鼓嶼、貓嶼。嶼頭龍江口、海口。江上源崔溪,出西北百丈嶺,東匯龍潭山水、無患溪,曰西溪。至城南,左合東皋山水,為龍首河,瀦為琵琶洋入。又東南三山、高山、天馬山,至蓮盤。北際御國山。有大扁嶼、東沙。自鼓嶼迤東南為大練門。海壇南有三十六派湖。其北:軍山、鐘山,西:水馬山,南:南茭山、草嶼、東甲、西甲。又西:南日山,迤北:大岴、小岴,至逕江口。南日江上游,逕江上承蘇漁溪入,西南江口橋至莆田界。牛頭門、薛峰頭、上逕諸汛。錦屏、江口二廢巡司。宏路、蒜嶺二驛。連江疲,難。府東北百里。城北:龍漈山。西北:白巖、雲居。城南:金鼇。東南:定岐。東北:馬鞍。東際海。自羅源迤南為北茭。其南北竿塘山,與閩南竿塘直連江口。江即鼇江,上匯羅源羅溪、長潭溪及鳳板溪,閩清雪峰水,寧德排樓溪於五縣寨口。又東至羅侖渡為寶溪,又東為鼇江。左合財溪、利安溪及雪溪;又東逕東岱為岱江。右合蟾步江,左珠浦,又東與東北鯉溪逕燕窩並入焉。東湖溉田四萬餘畝。定海、北茭、小埕三鎮。東岱巡司。羅源沖。府東北百六十里。治鳳山南麓。城北:文殊山、席帽。西:四明。西北:洪福萬。東南:松崎。東北際海。自寧德迤南為鑒江口,東與東沖口直循東洛、西洛,又西可門、濂澳門、松崎江口。城西:羅川出蔣山,合九溪、四明溪,歧為南北溪。復合,東逕禹步跡,縣北九龍溪合起步溪來會,與白水溪、小護溪、大護溪並達於松崎江。其東南至連江界。西南:鳳板溪、長潭溪,西:霍口溪,上承侯官密溪。左納蘇洋溪,屈東,南為羅溪。左納老人山水,又東南入連江。西北楊溪入寧德。鑒江、濂澳、松山、上地四鎮。古田沖,疲。府西北二百七十里。城北:翠屏山。城西:北臺。西南:九龍。東:蓋竹。大溪二源,東溪出杉洋鎮黃居嶺,西南右會太平山水,左納甘棠溪,又西南,左合石馬山水,右納富洋溪,又東,逕城東,為東溪。屈南,西溪自其右來會。南逕鳴玉灘,錯閩清復入。劍溪自南平入,左合赤凌溪、嶺頭水,折東南,逕小武當山北來會。曰水口,亦名塔溪,又東南入閩清。東:蘇洋溪、老人山水,西南入羅源。柯潭、平湖,各溉二十頃餘。縣丞駐水口。黃田鎮。並有驛。白溪、西溪二廢巡司。屏南疲,難。府西北二百二十里。雍正十二年析古田置。東南:羅經山。南:仙字巖。西南:靈峰。城西:雙溪,南源出水竹洋,北源出天臺嶺,合為龜溪。又南至棠口為棠口溪。右合白溪,折東南,其南龍漈溪、黛溪,並入寧德。西南:甘棠溪。西:富洋溪、牛溪。閩清簡。府西北百二十里。城西南:臺山。西:鼎峰。西北:白雲。南:金鐘。東北:鳳皇。城東建江,西北自古田入,合東溪,錯入,合石步坑水;又東南,右合大雄溪,錯侯官東北,陳溪注之。又南,梅溪,出馬坑嶺,會瞿曇溪。又西,左合峰洋溪,折東北,左合演水溪。又北逕城西,環而東南,合蓋平、仁壽、孝順、金沙諸里小溪來會,曰閩清口,屈東北入侯官。清窯鎮。洋頭塘汛。永福疲。府西南百四十里。東北:摩笄山。北:文殊巖。城南:大張。南:陳山。東:觀獵。西南:高蓋,道書第七福地。城東大樟溪,西南上承德化滻溪入,合洑溪為洑口溪。又東北,左合東洋村及上下漈水,又東逕嵩口,至重光寺。右納游溪,為西溪、為東溪,又東北,左合龜洋溪、漈溪,為雙溪。右納游洋支津為大溪。又東北,左合梧嶺水,右十八溪,又東北逕大樟山北,是為大樟溪,入侯官。白葉湖,宋乾道二年修,溉十頃。大樟鎮。漈門巡司。

福寧府:沖。隸福州道。總兵駐。明,州,領縣三。雍正十二年為府,割福建之壽寧來隸,增霞浦。乾隆四年,復析置福鼎。西南距省治五百四十五里。廣三百二十四里,袤二百三里。北極高二十六度五十四分。京師偏東三度四十一分。領縣五。霞浦沖,繁。倚。西南:霞浦山,縣以此名。城北:龍首。南:羅浮。東:箬山。西:慧日。西北:望海。東、南、西際海。自福鼎迤南,小俞山、烏崎港。楊家溪納梓柏洋溪為赤溪,折東為雉溪入。又南三沙,迤西小皓,瓜溪入。又西松山,赤岸溪合倒流溪、三澗水入。又西百茶村,歐公河入。又西南漁洋垾,後壟溪入。又迤東南武崎山。又南廢大金山千戶所。又南羅浮山,是為三都澳,商埠、海關在焉。又西葉山。又南北壁。又南東沖口。又西至福安界。其北獃頭山,霞浦溪入。又北鹽田關,柘溪入。其西坪溪、坑口溪、富溪,並入福安。斗門閘溉田萬頃。東沖、大金、古鎮、斗米、牙城五鎮。柘洋、三沙二巡司。高羅、楊家溪二廢司。福鼎沖,繁。府東北二百十里。治桐山南麓。東:福鼎山,縣以此名。東:福全。東南:茭陽。西:鐵樟。南:太姥。東南際海。自浙江平陽迤西南為沙埕港。桐山溪出西北金尖山,屈東北,合金釵溪、茭溪、南溪,折南為烏溪。合透埕溪、貫嶺溪,逕城東而南,合龍山溪為夾城溪,又東南為關盤港。會三叉河、前歧溪、象溪,其西南會董江為白水江,又東南逕金嶼門入。又西屏風山,有福安塘、彈江入。又西黃崎山,筼簹溪入。又西北九曲港,王柄溪會才溪、蔗溪、躍鯉溪、秋溪入。又西峽門,硤門溪合濮陽溪入,又西至霞浦界。西南:樟柏洋溪入霞浦,管洋溪入浙江泰順。沙埕、峽門、南關三鎮。秦嶼,參將駐。有巡司。瀲城廢司。福安疲,難。府西北百三十里。城北隅:銅冠山。城東:鶴山。北:扆山。東北:大東。西:福源。東南:馬頂、城山。南:重金。南際海。自霞浦迤西南,為官井洋、白馬門口。大溪二源:東溪北上承浙江泰順,寧壽後溪,自緣界入,右會蟾溪;西溪上承寧壽托溪來會,為交溪,至城西棲雲潭,右合秦溪,是為大溪。又東南,左納坑口西坪水,折西南,會松洋溪,為三江口。又南為蘇江,右合薛阪,左赤石關水。又東南為印江,黃崎江入。又東南,逕白馬門、達官井洋,入於海。有白石關巡司。白石鎮廢司。寧德疲,難。府西南百三十里。西:白鶴山。東:官扈。東南:金甌。南:勒馬。北:霍童山,白玉蟾云「三十六洞天第一」,高二十里,周五十里。東南際海。自福安迤西南為雲淡門,松洋溪支津西北自壽寧入,納麻陽峽水為南門溪。顯聖溪緣屏南界合雙溪來會,為外渺溪。又東南,左合赤溪,逕銅鏡為金乘港入。又西南北溪,西北上承屏南龍漈溪入,逕石堂山合黛溪。又南,東為金溪。又東南,覆鼎嶼、白匏山、青山。其南,青嶼門。北溪南支合鐘洋溪,納楊溪,逕城北為藍田溪。又東南合古溪,又東合蒲嶺水,為飛鸞江,合焦溪入。又西南至羅源界。其松洋溪經流東南入福安。東湖、飛鸞鎮。霍童巡司。石堂二廢司。壽寧簡。府西北二百八十里。城北隅:真武山。北:立茂。東:叢珠。西:天馬。托溪即北溪,西北自浙江慶元入,為九嶺溪。又東南合西溪,至斜灘會南溪。其北蟾溪出西北大熟嶺,會茗溪貫城東出,逕筆架山,屈東南,緣界並入福安達交溪。又北,西溪出慶元界青田隘,東合官臺山官田洋水,又東為葛家渡溪。又北後溪自浙江景寧入,為上地溪,合小東水。又東,折北,錯泰順復入。又東南為百步溪,右合武溪,復錯泰順與西溪合。下游亦注交溪。西南松洋溪自政和入,逕芹山至泗洲橋,支津西南出,又東南至溪口,並入寧德。里老橋陂溉田二百餘畝。漁溪巡司。

延平府:沖,難。延建邵道治所。明,領縣七。雍正十二年,割大田隸永春。東南距省治三百六十里。廣三百里,袤三百十八里。北極高二十六度三十九分。京師偏東一度四十九分。領縣六。南平沖,繁,難。倚。南:九峰山。東南:屏山。城西:虎頭。東北:演仙。西北:蓮花。西南:金鳳。劍江一曰建江,為閩江上流。二源:東北東溪,上承建安建溪入,逕高桐,左合埂埕溪,右群仙洋、大小湆水,逕城東而南;西北西溪,上承順昌大溪入,逕上洋口,左合鸕甪溪,又東南,右合黃泥溪,至雙溪口,右納沙溪,又東南,折東北,逕城南來會,是為劍江。右合十里庵口溪、南平里溪、羅源溪,左納吉溪。又東南,左合岳溪,右金鋼嶺水,尤溪亦自南來會。又東,左合武步溪入古田。東溪、黯淡灘、南溪、龍窟灘險甚。嵢峽巡司。又大歷廢司。大橫、劍浦二驛。順昌繁,難。府西少北百二十里。北:華陽山。西:鳳山。西南:大明。西北:七臺。城南大溪,二源:西北富屯溪,上承邵武大溪入,東南,左合順溪,右大幹溪,逕城西而南,亦曰礶砧溪;西金溪,自將樂入,合交溪、婁杉溪來會,是為大溪。右合洄村溪水、南溪,又東,右合石溪、棋溪,入南平。鎮四:仁壽、上洋、大幹、安撫。仁壽廢巡司。將樂疲。府西二百二十里。城北:西臺山、龜山。東:蓮花。西:鐘樓。東南:天階、烏石。東北:封山、石帆。南:仙人塘。西南:五龍。城南金溪,西北上承泰寧大溪入,逕萬全北,合常溪、竹洲溪、將溪,屈東南,右合三溪寨水,左望江溪、獬村溪,又東,右納池湖溪、水口溪,折東北,至城東南,是為金溪。龍池溪合沙溪自北來會。又東,左合安福口溪,逕三澗渡,右合常口溪、漠村溪,左瀨口溪、黃坑口溪,又東入順昌。西北瓜溪,屈西北入泰寧。萬安巡司。沙繁,難。府西南百二十里。城北:鳳岡山。西:巖山。西北:陶金。西南:呂峰。北:將軍。東北:馬笠。東:玉山。南:七朵。其下沙溪,二源:一太史溪,西南上承永安燕水溪入,逕莘口,右合西霞阪水,左明溪,又北,右合蔣坑水,左斑竹溪、隴東溪,至城東南;東溪出順昌界天柱山曰半溪,東南至漈口,又會瓦溪,合幼溪,又南逕城東來會,是為太史溪。又東,左合鸕甪溪、玉溪、楊溪、下湧溪、下湖溪,右洛溪、瑯溪、丹溪、高溪、漁溪,入南平。北鄉砦巡司。永安繁,難。府西南三百里。城東二山,南登雲塔,北栟櫚。西北:黃楊巖。東南:斗山。東北:貢川。城北燕水溪,西北上承清流九龍灘水入,逕大嶺,屈東,左合羅峰溪;右櫞嶺水,至八仙巖。東連城姑田溪自西南來會,是為燕水溪。又東,右會南溪及浮流溪、林田水、桂溪,至城北,右合大梅溪,左益溪,又東北為貢溪,左納坊溪、田沙溪,右合青溪,其東南黃田嶺水,北合烏阬水為西霞阪水。其北明溪自歸化入,又東並入。鎮二:西洋、星橋。安沙、小陶二巡司。英果、黃楊、湖口三廢司。尤溪繁,難。府南百六十里。城北:永山。西南:鸕甪。西:璠山。東南:石井。南:眠象。東:參拜。尤溪城南二源:湖田溪西南上承大田縣溪入,逕高才,左合包溪,右漈頭溪及汶水,又北,左合新橋溪,右寶溪,至城南;青印溪出沙界羅巖峰,東南右合新坑水,左麻溪、小溪,逕城西來會,是為尤溪。又東為雲潭,右合雙髻山源湖水,左華南溪,又東,左合塔兜,右資壽溪,入南平。明溪北自歸化入,又東入沙。官陂西南溉田數千頃,波及德化界。高才阪巡司。

建寧府:沖,繁。隸延建邵道。明領縣八。雍正十二年,割壽寧屬福寧。東南距省治四百八十里。廣四百九十五里,袤四百三十里。北極高二十七度四分。京師偏東二度。領縣七。建安沖,疲,難。府東南。東北隅:黃華山。城南:覆船山。東北:馬鞍山。東南:象山。西南:龍池。建溪亦曰建江,二源:松溪東北自政和入,屈西南,左合川石漈,右東游、橫谷、坤口、千源諸溪,又西南,左合東萇溪、沙溪,至城東南為東溪;又西,西溪自甌寧逕城西南來會,是為建溪。又南逕太平驛,右合古老嶺、下溪、秦溪及其支津,左納百丈溪、房村口溪,入南平。雙溪、大官陂各溉田十一頃。迪口縣丞。房村巡司。太平一驛。甌寧沖,繁。附府西北。北:天湖山。西北:烏石。東:東山。城西:龍首。城南:覆船。東北:天堂。西溪二源:建陽溪自縣入,東南逕葉坊驛;柘溪自浦城入,迤西南,左合蓬嶺水來會,為雙溪口,又南,右合吉陽溪、興賢溪、躍鱗溪,左紫溪、宜均溪,至城北分流,復合於臨江門外,是為西溪。百丈溪出縣西北山,合登仙里水,東游溪出東北侖口,並入建安。巧溪出西黃源嶺,入順昌。鳳坑水出東北白石山,東入松溪。將軍山下陂溉田千餘頃。吉陽、營頭二巡司。葉坊、城西二驛。建陽沖,疲。府西北百二十里。城西隅:大潭山。西三十里,太平、九峰、唐石。西北:蘆峰。北:闌幹。東北:硯山。南:蓮臺。西南:五峰。建陽溪亦建溪,二源並西北:崇溪自崇安入,曰北溪,左合陳溪,又南折東,左合芹溪,至河船,右合石船溪,左錦溪及油溪,屈西南至城東南;西溪出西北毛虛漈山,會竹溪、瓦溪,屈南,右合化龍溪,折東,右合莒溪、馬伏溪,左龍口溪,逕玉枕峰西,又東來會,是為建溪,亦曰交溪。迤東南逕樟灘,左合窯溪、將溪,右長湍溪、吳墩溪、徐墩溪,入甌寧。油陂溉田五十頃。西:麻沙鎮。南:蓋竹鎮。南槎巡司。建溪一驛。崇安沖,繁。府西北二百四十里。南:武夷山,道書第十六洞天,周百二十里,峰三十六,巖三十七,岸壁紅膩,棱疊可愛。北:黃石。東北:濟拔。西北:三髻。西:白華。東:仙洲。東南:寨山。崇溪二源:東溪出東北石臼里。匯岑陽、寮竹諸山水,西南,左合小渾溪,右浴水溪、嵐溪、新豐溪,至大渾里,右合大漈溪,又西南至林渡;西溪出西北分水嶺,會大安源、雙溪,又東,左合溫林、觀音二寨水來會,是為崇溪。又南過押衙洲,分流復合,逕城東,右合黃龍溪。又南,左合梅溪,迤西右合九曲溪,屈南至黃庭,右合黃石溪及籍溪,又東南入建陽,為北溪。蘆陂溉田萬餘頃。鎮二:溫嶺、黃亭。五夫里巡司。興田、裴村、大安三驛。浦城沖,繁。府東北二十七里。治黃華山南麓。城東隅:越王山。城北:橫山。東北:太姥、蓋仙、仙霞嶺。南:回隆。西南:西陽。東南:金斗。柘溪、南浦溪出東北柘嶺西南,右合灰塢漈、上溪,左半源、漁倉、里洋諸水,又西,右合漁梁溪,左官田溪,側城西南,凍蕤夾岸,亦曰梅花溪。新溪出西北百丈山,合洪源溪來會,是為柘溪。又南,右合東源溪,左大石溪,又西南,右合臨江溪,左富嶺溪,逕曹村,右合石陂溪,又南入甌寧。北盆亭溪,西會小竿嶺、梨嶺水,詹溪,入江西廣豐,注信溪。富嶺,縣丞治。鎮一:漁梁。巡司二:廟灣、溪源。驛二:小關、人和。松溪簡。府東百六十里。治蹲獅子山南麓石壁山。東:王認山。東南:七峰。西:皆望。東北:鸞峰。松溪二源,分出浙江龍泉小梅、慶元溫嶼而合,逕木城隘入。西南逕舊縣塘,左納新窯水,又西,右納松源溪,又西至城東南,右合杉溪、白石溪,又西南入政和。吳村,縣丞駐。渭田巡司。政和簡。府東百四十五里。東:池棟山。東南:大風。南:飛鳳、洞宮山,道書三十七洞天。東北:天柱峰。西北:南禪。松溪自其縣入,逕常口,迤東南至西津渡。七星溪逕鐵山口,合石龜溪、胡屯溪,又西合茶溪,逕城南合官湖,又西南來會。又南,亦曰當陽溪,左合小層溪,右山表溪,入建安為東溪。東北新阬水,出天柱山,東南雙澗溪出溪門嶺,合下園溪、李洋水,並入壽寧。又南和溪出西表嶺,入寧德。範屯諸陂四十有四。下莊巡司。又苦竹廢巡司。

邵武府:沖,疲。隸延建邵道。明屬福建布政司。順治三年,隸武平道及分守建南道。康熙六年並廢,改。東南距省六百七十里。廣二百二十里,袤二百六十里。北極高二十七度二十一分。京師偏東一度五分。領縣四。邵武沖,繁,疲,難。倚。西南:殊山,為郡祖龍。西:登高。南:福山。東:雞鳴。東南:浮潭。北:雲際。東北:泉山。邵武溪即大溪,西北上承光澤交溪入,中合中坊溪,左勛溪。又東南,左合田段漠口,右和順高家渡龍鬥溪,逕紫雲灘,右合溪西鎮龍橋藥村水。又東至城北,左合石鼓溪、石樵溪,又東南,右合鹿口溪、銅青溪、大竹溪。折東,左合拿口上下溪,右密溪。又南,左合驕溪,至板孔灘,右合外石,左衛閩溪。又南,右合謝坊,左下黃溪、繡溪,至水口,右合桃溪,又東南入順昌。官坊溪出東南官尖峰,西南入泰寧,注龍湖溪。黃溪溉田四十頃。烏阪城,城東。黃土關,西南。有黃土三鹽場。水口、拿口巡司二。光澤沖,難。府西北八十里。城東:羅嘉山。西南:管蜜。西北:大和、昂山。東北:烏君塢。西溪上承馬嶺山水,自江西新城入,東逕羅家渡,右會象牙山巖嶺隘,左石螺山水,為西溪。迤北,左得小禾山水,又東至水口,左納硃溪,右陳溪,折北合大嶺水,至冊下。左合馬丫山、何家山、上下原諸水,逕城西,支津入城為九曲溪。又北,過牛洲至城東,復與雲巖水來會。又東至烏洲。北溪東北自鉛山馬鈴隘入,逕雲際關南會大棋山水。又西南,左合延寮,右火燒關水,至舉賢,右合苦株阬,左肩盤嶺水,又西逕小寺州,右納冷水坑水。又西南,右合峰坳水,又南逕城東來會,是為交溪。又東南,右合花園水,入邵武。西北大和山水,其西牛田水,並入蘆溪。松林陂溉田八頃。清化鎮。大寺巡司。建寧疲,難。府西南二百十里。西北:白鹿山。北:何家。東南:大弋。城東:南山。西:鳳山。城南:濉江,西南上承寧化寧溪入,左合都溪及裏源溪,又北,右合金鐃山水,又東北,右合百丈嶺水,是為濉溪,逕城東而北,匯為何潭。又東,右合開山溪,逕橫口,左納永城溪,右合武調溪及馮家漈溪,又北,東入泰寧。東南黃土嶺水,東入歸化。雙溪、張家陂各溉田七頃。泰寧簡。府西南百四十里。城西隅:爐峰山。北:鐘石。西:青蓮。東北:旗山。南:石山。大溪亦東溪,東北上承邵武官坊溪入,右會龍湖。東溪納龍湖,逕濟橋,右納交溪,左合梅林溪及硃口溪。又西南,合龍門溪及將溪,至山夾橋,左納黃溪,至城東,右合杉溪,匯於何潭,是為城東三澗。折西,左合福沖溪、均福溪、二十四溪,至南會保,右合瑞溪、石塘溪,建寧濉江亦自西來會,是為雙溪口。折南,右合龍安溪、金口溪,又東南為布溪,入將樂。樂思壩本鸕甪陂,溉田四十頃。

汀州府:沖,繁,疲。隸汀漳龍道。東北距省治九百七十五里。廣三百五十五里,袤四百三十里。北極高二十五度四十七分。京師偏東二分。領縣八。長汀沖,繁。倚。今治南城。北:臥龍山。東:馬鞍。城南:圓珠、宣嚴。西:玉女。東北:翠峰。東南:七寶。鄞江即汀水,東北自寧化入,右合將軍山、天井山水,又西南出龍門峽,右合梓步溪,又南為湘洪峽,右合小湘溪。折西,右合北溪及東溪,又西南,右合篁竹嶺水,逕城南,右會西溪。又南,左合南溪,又東南,左合鍾家坑諸水。又西南,左合黃風溪,濯田溪合桃楊隘水、臘溪、黃峰水、桐木坑水,逕濯田自其西來會。又南,左合羊角溪,右納小瀾溪,入上杭,下至廣東下埔為韓江,至澄海入海。東:虎忙嶺水,其南牛尾嶺水,又南八仙巖水,並入連城。其西磯頭水,入上杭。西北貢水,即湖漢水,入江西瑞金,為貢江,行七百餘里,下流與章江會為贛江。大城寨巡司。館前、臨汀、三洲三驛。寧化簡。府東北六十里。城北:翠華山。南:五靈。西南:南山。東:墨瓦。東北:寶螺。西北:西華。大溪二源:西溪西南出狐棲嶺,東匯為蛟湖,折北,左會陳家坑水、覺溪,折東至城東南;東溪,北出建寧界三都嶺,又南,左合罕坑,右若竹嶺水來會,是為大溪。有烏路峽。右合合溪及上坪村諸水,與其南安樂水、羅溪並入清流。西南龍蘿山水,入長汀。西北寧溪,東北入建寧。七里圳導竹篙嶺水自新安橋至西溪三百五十丈,溉田數千頃。縣丞駐泉上里。石牛、安遠二巡司。清流簡。府東北二百里。城北:屏山。南:龍山。東:東華。東北:國母崠。東:隘嶺。清溪即大溪,西北上承寧化大溪入,左合三港溪、鄭家坑溪,右合安樂水,逕城南,右合嚴坊水,左嵩溪、梓材坑水,又東南至羅口。文川溪,上承連城清溪入,合楮嶺水,會羅溪。又東北,合官坊溪來會。又東,左合油瓶、隔右、洞口水,又東入永安。東北夢溪合芹溪、炭山水入注之。鐵石磯廢巡司。歸化簡。府東北二百九十里。南:樓臺鼓角山。西南:銀瓶玉■D9。東:龜山。東南:南山。西:黃牛。北:蛾眉。東北:龍西嶂。明溪西出永安界五通坳,會大嶺水,逕城北,右合黃溪而東,左合隘門叉,右雪山水。又珩溪、小明溪,又東左合瀚溪,又東為沙溪,左合無塵坑,右呂源水,至紫口坊。右合夏陽溪,折東南,左合紫雲臺水,與南大吉溪、胡坊溪並入永安。西北鋪溪出黃婆山,會寧化泉、上里水。又北建寧水,自長嶺隘入,合楓溪來會,折東,左合鼇坑,右丘地、茶坑二水,又東北入將樂。東北瓦溪入沙。大陂圳西北長二十里,溉田數萬畝。夏陽巡司。明溪驛。連城簡。府東南百四十里。北:蟠龍山、蕭坑。南:銀屏。東:蓮峰。東南:天馬。東北:馬坑。文川溪源自西南五磜,東北,右會金雞山,左張坊水,逕城南而東,合草笠山水。折北,左合李坊水,至麻潭,右合楮嶺水。又西北,左納虎忙嶺、牛尾屴二水,折東北入清流。西南豐頭溪,源自郎村隘,左會岡上水,右會牛尾屴南水。又西南,左合莒溪、苧園溪,右納八仙巖水,入上杭。東南曲溪,其南賴源水,並東北入永安。少西大東溪,又西隔溪,並東南入寧洋。北團寨巡司。上杭沖,繁。府南二百四十里。北:紫金山。東北:覆籮。東:冷洋。東南:鐵障。南:橫琴。西:展旗。西南:羊廚。城南大溪二源:鄞江西北自長汀入,屈東南,左合射溪、金山溪,右九華溪,至水鋪塘,右納檀溪,至九洲關;豐頭溪東北自連城入,右納磯頭水,左合九曲溪、苦竹溪,又西南來會,是為大溪。屈南,逕城東而南,瀨溪自西南來會。橫琴岡一曰龍翔溪,又南,左合安鄉溪,至樟樹潭,左納豐稔溪,又西南,右合白沙磜水,又東南,左會永定溪,右合磜頭水,入廣東大埔,注神泉河。縣丞駐峰市。平西、籃屋二驛。武平簡。府西南二百六十里。北:交椅山。東:梁野。西:靈洞。武平溪二源:東北大豐溪,源自永平寨東,左會當風嶺水,右合漁溪,左靈聚溪,西南,左合下黃溪、黃沙溪,為化龍溪,至硃阬西北;大溪自江西會昌入,左合石徑嶺水,徑武平所東,屈南,右合溪頭水來會,是為武平溪。又西南,左合巖前寨水,入廣東鎮平,注大溪。東中保水,東南象洞水,並入上杭。北大順溪,東北,左納石子嶺水,入長汀。西南馬戰崠水,西北張阬水,分入江西長寧、會昌。象洞、永平二巡司。永定繁,難。府東南三百六十里。北:龍岡山。南:掛榜。西:印匣。西北:黎袍崠。東北:寒袍崠。東:圓嶺。永定溪東北上承龍巖、文筆山水,逕富嶺北,屈西南,左合分水嶺水,右文溪、武溪,至溪口,右合涼傘寨,左湖雷水,逕城東。又西南,左合當風坳,右分水坳水,入上杭。東南金豐溪,出巖背山,西會下佛子隘水,屈西南,左合高頭水、莒溪、香南溪,右鳴岐嶺、新村水,入廣東大埔。西北豐稔溪,出茫蕩洋山,錯上杭,會大豐壩水復入,合合溪、香溪、跳魚溪、躍鱗溪、湯湖溪,再錯復入,合小大阜漈水仍入之。三層嶺、太平砦二巡司。興化廢司。

漳州府:沖,繁,疲,難。汀漳龍道。漳州總兵駐。明,領縣十。清初因之。雍正十二年,升龍巖為直隸州。漳平、寧洋割隸。嘉慶元年,析平和、詔安地增置雲霄。東北距省治六百八十里。廣二百七十里,袤二百九十里。北極高二十四度三十二分。京師偏東一度二十分。領縣七,一。龍溪沖,繁,疲,難。倚。城西北隅:登高山。北:天柱。西北:九龍。南:名第。東:文山。東南:龍漈。西:天寶。九龍江西北上承漳平九龍溪入,逕涵口,又南為華崶溪。又東南,左合石兀山水,逕下漳,左納高層溪,右三腳灶水,入為汰溪。碧溪至香洲渡,左納龍津溪為郭溪,又東為柳營江。南門溪上承南靖大溪入為梅溪,支津入城。又東,合龍漈山水,至三叉河歧為石碼港,又東北來會,為福河。又東為錦江,過許茂烏礁歧為二,分出二洲間,並入海澄,與南溪合入海。通判駐石碼。鎮四:東關外、木屐、石尾、玉洲。驛三:江東、甘棠、丹霞。有新岱巡司。九龍、柳營江二廢司。海澄難。府東南五十里。雍正十一年割漳浦之鎮海衛來隸。西南:儒山。北:文圃。南:席帽。東:吳養。西南:侯山。東南:鹿石。東、南、北際海。北自同安迤西南為浮宮港。南溪,西南上承南靖馬坪溪,緣界為馬口溪,又東南為倒港,歧為二:一東北逕白水營北入;一東北逕城南,復歧為二,一東夾洲入,一北逕城西而北為普賢港,逕沈嶼為盧沈港,會龍溪、石碼港、錦江,又東過恆溪、玉枕二洲,逕胡使二嶼、圭嶼與海滄港,達同安鼓浪嶼入焉。鎮四:鎮海、浯嶼、海滄、海門。海門、濠門、島美三廢巡司。南靖繁,疲。府西四十里。南:林壁山、西天山、獨坐。東北:巖倉嶺。東:峽口。西南:麒麟。西北:朝天嶺。城南雙溪二源:大溪亦西溪,西南自和平入,為高港,西北逕長窯墟,折東北為小溪口,左納博平嶺水,折東南為鯉魚溪、船場溪,合象溪,納琯溪,至旗尾渡;小溪亦東溪,西北自漳平入為員沙溪,又南合沄水、阬水,折東合鵝髻山水,又南逕金山北,出湧口至太監嶺,西合涵溪,折東,左納苦竹村水,又南來會,為雙溪口。折東,逕城南,為湖山溪,又東南為峽口溪,又東入龍溪。馬坪溪自平和緣界合老灶山水,又北,東入漳浦。龍磜陂溉田三千餘畝。巡司二:和溪、永豐。又九龍砦廢司。驛一:平南。漳浦沖,繁,難。府南百二十里。城北:羅山。南:梁山,西麓盤陀嶺。西南:將軍。東:海雲。東北:太武。東南:良山。東南際海。自海澄迤西南為井尾澳,黃女江入。又南將軍澳。迤西虎頭山,山北六鼇廢所。又西竹嶼、浮頭港。李澳川西北上承平和五寨溪入,右合崎溪,又東南為秦河,又東南逕龍頭保入,又西南為古雷城,至雲霄界。其北杜潯溪入。南溪上承南靖馬坪溪,緣界合小溪,又東逕大帽山北入海澄。縣丞駐佛曇。鎮一:杜潯。驛二:臨漳、盤陀嶺。平和沖,繁。府西南二百里。東北:長盧山。東:九牙。南:天馬。西北:象湖。東南:大夆,四溪出焉。曰河頭溪,西會官寮山水為合溪、粗溪,又西逕樓宅山,至城南。又西,右合大蘆溪,入廣東大埔,注清遠河。曰高山溪,東北,左會小坪山水,右合南勝溪。又東北為琯溪,左合高溪、碧微溪,右九團溪,又東北入南靖。曰河上溪,東南會白石山水為三合溪,又東南合佛幾嶺水,入雲霄。曰徐阬溪,西南會陳溪、天馬山水,又南,左合下陂溪、馬溪入詔安。東南五寨溪,東入漳浦為李澳川。高港溪東北入南靖為大溪。南靖故城,縣丞駐。鎮二:南勝、庵後。有琯溪巡司。詔安沖,繁。府西南二百五十里。城西:良峰山。西北:金雞。北:烏山。東:奇山。東南:川陵。東南際海。自雲霄迤西南為銅山廢所,東北與古雷城直為石城嶼。其西漸山、八尺門汛。其間後澳港、大陂溪合梅洲溪入,有金石廢司。其南南浦,又西南宮前,迤西北懸鐘廢所。東溪東北上承平和徐阬溪,合下陂溪,又南,右合白葉洞水,又東南合赤溪,逕城東園林歧為二,一自東沈村循甲州而東,一自奧雅頭右合磁窯溪,逕牛姆礁,入於海。南南澳,總兵駐。又西至廣東饒平界。新陂溉二千餘畝。鎮九:懸鐘、雲澳、青澳、西砲臺、草寮尾、紅花嶺、分水關及銅林之後林村、宮前村。銅山場大使。漳潮巡司。驛二:南詔、大碑塘。長泰簡。府東北三十八里。西北:良岡山。北:董峰。東:天柱、蜈蚣。東北:內方。龍津溪出東北林口隘,東南,左合芹果溪,又南左合白桐山水,右歧為巖溪,注高層溪,入龍溪。又南,左合馬洋溪、可壟溪,折西逕城南,又西南逕鼎山北入龍溪。有朝天嶺廢巡司。雲霄中。府西南百六十里。嘉慶三年,析平和、詔安置。東:大臣山。南:馬山。西:將軍。西南:直武崎。西北:呈奇嶺。東南際海。自漳浦迤西南,杜潯港入。又西,西林溪,西北上承平和白石溪入北,又南逕大田,左合嶺腳水,右納龍頭水,又南逕為西林溪,右合將軍山水,側城東南,右合御史嶺水,與杜潯港達於漳江。陳岱港出城東南盤石,東南流,逕八尺門,入於海。石蛇尾、梅州二鎮。

龍巖直隸州:繁,難。隸汀漳龍道。清初因明制為縣,屬漳州。雍正十二年,為直隸州,割漳州之漳平、寧洋隸之。東北距省治九百二十里。廣二百二里,袤百九十一里。北極高二十五度九分。京師偏東三十九分。領縣二。城內:大崶山。城北:後山。南:奇邁。東:東寶。西:虎嶺。城南龍川,出州西九曲嶺,會大小池水,東為羅橋溪,逕城南,匯為石鼓潭。右合陳陂溪及曹溪,又東為東溪,逕觀音座山,匯為甕口潭。左合傅溪,又東北為雁石溪,左合溪、硿頭溪。其北藿溪,上承連城大東溪入,合隔溪,逕溪口,右合長阪溪,左納小東溪。有雁石巡司。適中驛。漳平難。州東百七十里。西北:古漈山。北:三山。西:龍停。東:東關。南:覆鼎。東北:凌雲。城南九龍溪二源:東源西北上承寧洋大溪入,為九鵬溪,又東南,左納藿溪、西阬水,又南逕鹽場塘西;西源雁石溪自州來會,是為九龍溪,又東,右合吳地溪,逕城南,匯為九龍潭,又東南,右合黃畬鋪水,至華口塘,左納感化溪,右合下折溪,二溪相交如十字然,又東南,與南三腳灶水並入龍溪。東北古格嶺水,入安溪。後溪洋陂溉田六頃有奇。有永福里鎮。歸化、蘆溪南廢巡司。寧洋簡。州東北百八十里。北:金鳳山。南:香寮。西:芙蓉。西北:殺狐嶺。城南大溪三源:北溪出西北梨子嶺,會百種畬洞水,逕城北,會西溪,至城南,會南溪,是為大溪;又南,右合西溪,又東南,並入漳平;東溪出縣西爐山峰,西南流,合熱水、小溪水入龍巖。

興化府:沖,繁。隸興泉永道。清初因明制。北距省治二百四十里。廣二百十里,袤八十五里。北極高二十五度二十六分。京師偏東二度四十七分。領縣二。莆田沖,繁,疲,難。倚。南:壺公山。東南:五侯。西南:天馬、龜山。西北:夾漈。北:浮山。東北:澄渚。東:持久。東南際海。自福清迤南為黃竿。北荻蘆溪,會澳溪為洙溪,匯為北洋太平陂,達迎仙港。延壽溪上承九鯉湖,東為莒溪,匯為北洋延壽陂,達涵頭港。木蘭溪上承仙游仙溪,東為瀨溪,又東堰為南洋木蘭陂,達白湖港。三港既會,是為三江口,又東入焉。又南:美南。東南:青山。瀝尋塘,唐築,溉田百四十頃。縣丞駐平海。鹽場大使三,駐涵江、前沁、東嶠。湄州、忠門二鎮。涵江二巡司。大洋凌厝廢司。莆陽驛。仙游難。府西七十里。治大飛山南麓。西北:仙游山。東:鐵山。北:將軍。東北:石所。南:白巖。城南仙溪,西北自德化入,為大目溪,右合古瀨溪,又東,右納金沙溪,左合大濟溪,為三會溪。右合神堂溪,逕城南,又東北,左合走馬山水,右納石二嶺水,至東渡,左合安吉溪。其北九鯉湖,並入莆田。其南楓亭溪入海。北游洋溪入永福。興泰、楓亭二巡司。白嶺廢司。

泉州府:沖,繁,疲,難。隸興泉永道。提督駐。通判駐蚶江。明,領縣七。雍正十二年,升永春為直隸州,割德化隸之。東北距省治四百十里。廣二百七十里,袤二百里。北極高二十四度五十六分。京師偏東二度二十五分。領縣五。晉江。沖,繁,疲,難。倚。城北:清源山。東南:法石。南:獅山。西南:石塔。北:雙陽。東北:鳳山。東南際海。自惠安迤西,其洛陽港入為洛陽江,會長溪入白嶼。晉江上承南安黃龍江,東南為筍江、浯江、溜石江,至磁灶,逕法石汛為蚶江入。少南,陳埭、玉蘭浦、植璧港入金嶼。嶼南石湖即日湖。又東東埔,東北與崇武所直。又西深滬灣,又西南圍頭鎮,又西北石菌、白沙。九溪自南安入,為安海港,合靈源山水入,又西至南安界。縣丞駐石獅。潯美場大使。鎮二:浦邊、圍頭。巡司二:鷓鴣、雒陽。又庵上廢司。驛一:晉安。南安繁,疲,難。府西四十五里。城北:葵山。西北:鵲髻。南:靈秀。西:九日。西南:覺海。城南金溪二源:自永春入者桃溪、小姑溪,合於便口,又東南歧為二,一南合高田山水,一東合瀘溪、凌斜溪。復合為雙溪口;自南安入者藍溪,東逕珠淵汛,左合洞後埔水,右英溪、歸溪,又東亦歧為二,東與永春水會,東南逕金雞山為金溪,至城南為黃龍江,一南合囷山、解阬山水,至白石復歧,一東逕娘子橋,一南逕官厝,合柏峰山水為九溪,並入晉江。縣丞駐羅溪。蓮河場大使駐營前。鎮一:洪瀨。巡司駐大盈。又澳頭、蓮河二廢司。驛一:康店。惠安沖,繁,難。府東北五十里。東北:龍屈嶺。西北:大帽山。東:五公。東南:松洋。南:錦田。西南:盤龍。城西:登科。東際海。自莆田迤西為橫嶼、洋嶼、沙格澳、傅埭、添崎港入,又南峰尾澳、峰崎港入岱嶼、吉嶼,又南黃崎澳,又南小岞,東北與莆禧所直。胡埭出石佛嶺,合籓厝水,迤西大岞,又西崇武澳、獺窟澳,又西下按澳。峰崎港支津西南逕走馬埭,合龍津溪、馬山埭入。其北洛陽港,至晉江界。又北大溪,入仙游。鎮三:崇武、沙格、黃崎。門頭鄉,鹽大使駐。良興巡司。塗嶺廢司。驛一:錦田。同安沖,繁,疲,難。府西南百三十里。北:三秀山。東北:大輪、北辰山。東:九躍。南:寶蓋。西:西山、夕陽。南際海。自南安迤南為大嶝嶼,蓮溪入。又西北石尋港,抵城南,東溪、西溪入。又南下店、潯尾,後溪、深青溪入。迤東高埔、離埔、洲嶼、白嶼。其南大島二,東曰金門,有北大武山,縣丞駐。其北官澳,其東峰上,又東料羅。西曰廈門,故嘉禾嶼,東南與澎湖直,有洪濟山。道光二十二年,金陵條約為商埠。分巡兵備道。光緒甲午後,水師提督駐。五通渡、高崎汛、筼簹港入金、廈之間。懸嶼有大擔門、小擔門,南抵海澄、浯嶼。廈門西南隅鼓浪嶼,有德、英、日、法領事署。鎮六:店頭、新墟、下店、大路尾、浯嶼、高崎。通判駐馬家巷。浯州、祥豐二鹽大使。灌口、石潯、劉五店三巡司。驛二:大輪、深青。安溪疲,難。府西一百五里。西:蓬萊、駟馬。南:黃龍山。黃蘗又名午山,為縣中眾山之宗。北:鳳山、翠屏、雪山。東南:北觀、金龜。西南:龍塘。西北:鶴頂、佛耳、朝天諸山。縣南三里,藍溪亦曰清溪,源二。西北源出龍巖、漳平東北古格嶺,東南流入縣。東南逕桃舟隘,西受梯子嶺水,南流逕連德阪,南折而東北,逕龜壩南來會,又東北錯入永春洲,永春洞口溪自北來注之,又折而東南,復入縣。又東逕小橫鄉南,受熊田溪,溪亦自永春入。又東南受漢阪水,又南至魁斗西,受東溪、三層溪,又東南逕縣治西,曰吳埔溪,又南合西源。西源出縣西南北岸山東麓,東北流,受白葉山水,又東北逕舉溪壩南,受留山水,又東北受溪益水,復合九峰山後溪、胡坑諸水,又折而東南,逕五里埔北,受龍門嶺水,又東曰澚江,又東與西北源合。合而環縣東南,是為藍溪。又東逕羅渡南、田隙鄉北,入南安。有長坑、湄上二鎮。

永春直隸州:繁,難。隸興泉永道。明,縣,屬泉州府。清初因之。雍正十二年,為直隸州。泉州德化、延平大田割隸。東北距省治四百十里。廣百八十五里,袤百八十八里。北極高二十五度一十八分。京師偏東一度一十八分。領縣二。西北:大鵬山、雪山。東:昆侖。東南:花石。西:陳巖。西南:龍山。北:浮空。東北:樂山;雪山,桃溪出,為陳巖溪,合錦溪為埔兜溪,左合新田溪,又東逕東平山為洑溪,左合冷水坑水。又北,東為磁灶溪、石鼓溪,合龜龍溪,又東為州前溪,留灣溪,左合新溪支津,又東南與小姑溪並入南安。新溪出西北天馬山,其經流東入仙游。西北熊田溪自德化入,屈西南為碧溪,右合上窯水、南洋水,其西洞口溪,並入安溪。黃阪鎮。德化難。州西北三十里。治龍潯山南麓。北:繡屏。東北:石牛。東:龍門。東南:天馬。南:雙魚。西:五華。西北:戴云,滻溪出,為東埔水,會李山水,南,西為白泉溪;又南,右合黃洋溪、花橋溪,為石溪、蘇溪、塗阪溪,右合龍潭水,左蓋竹溪、郭阪溪、丘店溪,又東為西門溪,至城南,是為滻溪。右合大雲溪、黃斜溪,左丁溪。又東北,左合龍門溪,右碧潭水,至高漈。左合龍潭水及上雲溪,又東北逕岱山,左合南埕溪,折西北,左合盧溪,又東北入永福。西北小尤溪、錦屏山水、湯嶺水,分入大田、永春、尤溪。東北石牛洞水,入仙游。內洋鎮。小尤、楊梅諸廢巡司。大田簡。州西北二百六十五里。南:大仙山。西南:臺閣。東:銀瓶。東南:文筆。北:雙髻。東北:白鶴。縣前溪上承小尤溪,東南自德化入,合龍背嶺水為梓溪。又西北,左合小坑水,折東北為湯泉溪,至城南,匯為塔兜潭。又東北逕京口,右合仙峰溪,左上華水,至漈頭,英果溪合渡頭溪自其西來會,又北入尤溪。西南沈口溪,入寧洋。其南武陵安水入漳平。有桃源巡司。花橋廢司。


\end{pinyinscope}