\article{列傳一百九十八}

\begin{pinyinscope}
李鴻章

李鴻章,字少荃,安徽合肥人。父文安,刑部郎中。其先本許姓。鴻章,道光二十七年進士,改庶吉士,授編修。從曾國籓游,講求經世之學。洪秀全據金陵,侍郎呂賢基為安徽團練大臣,奏鴻章自助。咸豐三年,廬州陷,鴻章建議先取含山、巢縣圖規復。巡撫福濟授以兵,連克二縣,逾年復廬州。累功,用道員,賞花翎。久之,以將兵淮甸遭眾忌,無所就,乃棄去。從國籓於江西,授福建延建邵道,仍留軍。

十一年,國籓既克安慶,謀大舉東伐。會江蘇缺帥,奏薦鴻章可大用,江、浙士紳亦來乞師。同治元年,遂命鴻章召募淮勇七千人,率舊部將劉銘傳、周盛波、張樹聲、吳長慶,曾軍將程學啟,湘軍將郭松林,霆軍將楊鼎勛,以行。又奏調舉人潘鼎新、編修劉秉璋,檄弟鶴章總全軍營務。時沿江賊屯林立,乃賃西國汽舟八,穿賊道二千餘里,抵上海,特起一軍,是為淮軍。外國人見其衣裝樸陋,輒笑之,鴻章曰:「軍貴能戰,非徒飾觀美。迨吾一試,笑未晚也。」旋詔署江蘇巡撫。

是時上海有英、法二國軍。美國華爾募洋兵數千,攻克松江、嘉定、青浦、奉賢,號南路軍;學啟等將湘、淮人攻南匯,號北路軍。四月,賊悉眾戰敗南路軍,嘉定、奉賢再陷,華爾棄青浦走保松江。學啟將千五百人屯新橋,賊圍之數十重,踐尸進。學啟開壁突擊,賊駭卻。鴻章親督軍來援,賊大奔,乘勝攻泗涇,解松江圍。外國軍見其戰,皆驚嘆。自此湘、淮軍威始振。詔促移師鎮江,鴻章請先圖滬而後出江。既定浦東縣,偽慕王譚紹光來援,敗之北新涇,賊走嘉定。九月,進克其城。譚紹光率數十萬眾,連營江口,犯黃渡。諸將分攻,簡精卒逾壕伏而前,斃數人,賊陣動,學啟乘之,裹創噪而進,賊大潰。捷入,授江蘇巡撫。

初,美人華爾所將兵名常勝軍,慈谿之役,歿於陣,其副白齊文懷異志,閉松江城索餉。鴻章裁其軍,易以英將戈登,常勝軍始復聽節制,命出海攻福山,不克而還。二年正月,兼署五口通商大臣。初,常熟守賊駱國忠、董正勤舉城降,福山諸海口俱下。偽忠王李秀成悉眾圍常熟,江陰援賊復陷福山。鴻章牒諭國忠固守待援,而檄鼎新、銘傳攻福山,奪石城。國忠知援至,開城猛擊,俘斬殆盡,遂解常熟圍,進復太倉、昆山。因疏陳賊情地勢,建三路進軍之策:學啟由昆山攻蘇州;鶴章、銘傳由江陰進無錫,淮、揚水軍輔之;太湖水軍將李朝斌由吳江進太湖,鼎新等分屯松江,常勝軍屯昆山為前軍援。

李秀成糾合偽納王郜雲官等水陸十萬,偪大橋角而營,鶴章擊之,敗走,九月,復集,連營互進。鶴章立八營於大橋角,與之持。源章以賊麕集西路,志在保無錫,援蘇州。乃令鶴章、銘傳守後路,抽銳卒會學啟合破賊屯,蘇、錫之賊皆大困。賊陷江寧、蘇、杭為三大窟,而蘇則其脊膂也,故李秀成百計援之。譚紹光尤兇狡,誓死守,附城築長墻石壘,堅不可猝拔。十月,鴻章親視師,以砲毀之,城賊爭權相猜,謀反正,刺殺譚紹光,開門納軍。時降酋八人皆擁重兵,號十萬,歃血誓共生死,要顯秩。學啟言不殺八人,後必為患。鴻章意難之,學啟拂衣出,鴻章笑語為解。明日,八人出城受賞,留飲,即坐上數其罪,斬之。學啟入城諭定其眾,搜殺悍黨二千餘人。捷聞,賞太子太保銜、黃馬褂。十一月,鶴章等復無錫,進攻常州,以應江寧圍軍。學啟出太湖,圖嘉興,以應浙軍。鼎新等軍先入浙,收平湖、海鹽,賊爭應官軍,所至輒下。三年二月,學啟急攻嘉興,親搏戰,登城,克之,中彈死。四月,克常州,擒斬偽護王陳坤書,賞騎都尉世職。常勝軍慚無功,戈登辭歸國,乃撤其軍。

廷議江寧久未下,促鴻章會攻,鴻章以金陵破在旦夕,託辭延師。六月,曾軍克江寧,捷書至。鴻章遂分軍令銘傳、盛波由東壩取廣德,鼎新、秉璋由松江攻湖州,松林、鼎勛由滬航海援閩。賊平,封一等肅毅伯,賞戴雙眼花翎。

四年四月,科爾沁親王僧格林沁戰歿曹州,以曾國籓為欽差大臣,督其軍。鴻章署兩江總督,命率所部馳防豫西,兼備剿京東馬賊、甘肅回匪。鴻章言:「兵勢不能遠分,且籌餉造械,臣離江南,皆無可委託。為今日計,必先圖捻而後圖回。赴豫之師,必須多練馬隊,廣置車騾,非可猝辦。」詔寢其行。時曾國籓督軍剿捻久無功,命回兩江,而以鴻章署欽差代之,敗東捻任柱、賴文光於湖北。

六年正月,授湖廣總督。賊竄河南,渡運河,濟南戒嚴。初,曾國籓議憑河築墻,遏賊奔竄。鴻章守其策,而注重運西。飭豫軍提督宋慶、張曜及周盛波、劉秉璋分守山東東平以上,自靳口至濟寧;楊鼎勛分守趙村、石佛至南陽湖;李昭慶分守攤上、黃林莊至韓莊、八牌;皖軍黃秉鈞等分守宿遷、運河上下游:互為策應,使賊不得出運。六月,抵濟寧,賊由濰縣趨竄登、萊。鴻章復議偪入海隅聚殲之,乃創膠萊河防策,令銘傳、鼎新築長墻二百八十餘里,會合豫軍、東軍分汛設守。時賊集萊陽、即墨間,屢撲堤墻不得出。七月,賊由海神廟潛渡濰河,山東守將王心安不及御,膠萊防潰。朝旨切責,將罷防,鴻章抗疏言:「運河東南北三面賊氛蹂躪,其受害者不過數府州縣,若驅過運西,則江、皖、東、豫、楚數省之地,流毒無窮。」乃堅持前議,嚴扼運防。令銘傳、松林、鼎勛三軍往來躡擊。十月,追至贛榆,降酋潘貴升斃任柱於陣,捻勢漸衰。賴文光挈眾竄山東,戰屢敗,遁入海濱,官軍圍擊之,斬獲三萬。賴文光走死揚州。東捻平,賞加一騎都尉世職。

七年正月,西捻張總愚由山右渡河,北竄定州,京師大震。詔奪職,鴻章督軍入直,疏言:「剿辦流寇,以堅壁清野為上策。東捻流竄豫東、淮北,所至民築圩寨,深溝高壘以御之。賊往往不得一飽,故其畏圩寨甚於畏兵。河北平原千里,無險可守。截此則竄彼,迎左則趨右,縱橫馳突,無處不流。且自渡黃入晉,沿途擄獲騾馬愈眾,步賊多改為騎,趨避捷,肆擾尤易。自古辦賊,必以彼此強弱饑飽為定衡。賊未必強於官軍,但彼騎多而我騎少。今欲絕賊糧、斷賊騎,惟有嚴諭紳民堅築圩寨。一聞警信,即收糧草牲畜老弱壯丁於內。賊至無所掠食,兵至轉可買食。賊雖流而其計漸窮,或可剋期撲滅也。」二月,鴻章督軍進德州,敗賊安平、饒陽。三月,賊竄晉州,渡滹沱河,南入豫,復折竄直隸,撲山東東昌;四月,趨茌平、德平,出德州,西奔吳橋、東光,偪天津。下部議處,命總統北路軍務,限一月殄滅。

鴻章以捻騎久成流寇,非就地圈圍,終不足制賊之命。三口通商大臣崇厚及左宗棠皆以為言,而直隸地平曠,無可圈圍;欲就東海南河形勢,必先扼西北運河,尤以東北至津、沽,西南至東昌、張秋為鎖鑰。乃掘滄州迤南捷地壩,洩運水入減河。河東築長墻,斷賊竄津之路。東昌運防,則淮軍自城南守至張秋,東、皖諸軍自城北守至臨清,並集民團協防。閏四月,以剿賊逾限,予嚴議。時賊為官軍所偪,奔突不常。以北路軍勢重,銳意南行,回翔陵縣、臨邑間,旁擾茌平、德平,犯臨清運防。鴻章慮久晴河涸,民團不可恃,且晝夜追奔疲士卒,議乘黃河伏汛,縮地扎圈。以運河為外圍,以馬頰河為裏圍。其時官軍大敗賊於德州揚丁莊,又追敗之商河。張總愚率悍黨遁濟陽,沿河北出德州犯運防,上竄鹽山、滄州。官軍扼截之,不得出,轉趨博平、清平。適黃、運暨徒駭交漲,東昌、臨清、張秋、徬河水深不可越。河西北岸長墻綿亙,賊竄地迫狹,勢益困。鴻章增調劉銘傳軍,期會前敵。分屯茌平之桃橋、南鎮,至博平、東昌,圈賊徒駭、黃、運之內,而令馬隊周回兜逐,賊無一生者,張總愚投水死。西捻平,詔復原官,加太子太保銜,以湖廣總督協辦大學士。八月入覲,賜紫禁城內騎馬。

八年二月,兼署湖北巡撫。十二月,詔援黔,未行,改援陜。九年七月,剿平北山土匪。值天津教堂滋事,命移軍北上。案結,調直隸總督兼北洋通商事務大臣。十月,日本請通商,授全權大臣,與定約。十二年五月,授大學士,仍留總督任。六月,授武英殿大學士。十三年,調文華殿大學士。

國家舊制,相權在樞府。鴻章與國籓為相,皆總督兼官,非真相。然中外系望,聲出政府上,政府亦倚以為重。其所經畫,皆防海交鄰大計。思以西國新法導中國以求自強,先急兵備,尤加意育才。初,與國籓合疏選幼童送往美國就學,歲百二十人。期以二十年學成歲歸為國效用,乃未及終學而中輟。鴻章爭之不能得,隨分遣生徒至英、德、法諸國留學。及建海軍,將校盡取才諸生中。初在上海奏設外國學館,及蒞天津,奏設武備海陸軍,又各立學堂,是為中國講求兵學之始。嘗議制造輪船,疏言:「西人專恃其砲輪之精利,橫行中土。於此而曰攘夷,固虛妄之論。即欲保和局,守疆土,亦非無具而能保守之也。士大夫囿於章句之學,茍安目前,遂有停止輪船之議。臣愚以為國家諸費皆可省,惟養兵設防、練習槍砲、制造兵輪之費萬不可省。求省費則必屏除一切,國無與立,終無自強之一日矣。」

光緒元年,臺灣事變,王大臣奏籌善後海防六策。鴻章議曰:「歷代備邊多在西北,其強弱之事,主客之形,皆適相埒,且猶有中外界限。今則東南海疆萬餘里,各國通商傳教,往來自如。陽託和好,陰懷吞噬,一國生事,諸國構煽,實為數千年來未有之變局。輪船電報,瞬息千里,軍火機器,工力百倍,又為數千年來未有之強敵。而環顧當世,餉力人才,實有未逮,雖欲振奮而莫由。易曰:『窮則變,變則通。』蓋不變通,則戰守皆不足恃,而和亦不可久也。近時拘謹之儒,多以交涉洋務為恥,巧者又以引避自便。若非朝廷力開風氣,破拘攣之故習,求制勝之實際,天下危局,終不可支;日後乏才,且有甚於今日者。以中國之大,而無自強自立之時,非惟可憂,抑亦可恥。」

鴻章持國事,力排眾議。在畿疆三十年,晏然無事。獨究討外國政學、法制、兵備、財用、工商、藝業。聞歐美出一新器,必百方營購以備不虞。嘗設廣方言館、機器制造局、輪船招商局;開磁州、開平煤鐵礦、漠河金礦;廣建鐵路、電線及織布局、醫學堂;購鐵甲兵艦;築大沽、旅順、威海船砲臺壘;遴武弁送德國學水陸軍械技藝;籌通商日本,派員往駐;創設公司船赴英貿易。凡所營造,皆前此所未有也。初,鴻章辦海防,政府歲給四百萬。其後不能照撥,而戶部又奏立限制,不令購船械。鴻章雖屢言,而事權不屬,蓋終不能竟厥功焉。

三年,晉、豫旱災,鴻章力籌賑濟。時直隸亦患水,永定河居五大河之一,累年漫決,害尤甚。鴻章修復金門徬及南、上、北三灰壩。盧溝橋以下二百餘里,改河築堤,緩其溜勢。別濬大清河、滹沱河、北運河、減河,以資宣洩,自是水患稍紓。

五年,命題穆宗毅皇帝、孝哲毅皇后神主,賞加太子太傅銜。六年,巴西通商,以全權大臣定約。八年,丁母憂,諭俟百日後以大學士署理直隸總督,鴻章累辭,始開缺,仍駐天津督練各軍,並署通商大臣。朝鮮內亂,鴻章時在籍,趣赴天津,代督張樹聲飭提督吳長慶率淮軍定其亂,鴻章策定朝鮮善後事宜。九年,復命署總督,累乞終制,不允。

十年,法越構兵,雲貴總督岑毓英督師援越。法乃自請講解,鴻章與法總兵福祿諾議訂條款,既竣,而法人伺隙陷越諒山,薄鎮南關,兵艦馳入南洋,分擾閩、浙、臺灣,邊事大棘。北洋口岸,南始砲臺,北迄山海關,延袤幾三千里,而旅順口實為首沖。乃檄提督宋慶、水師統領提督丁汝昌守旅順,副將羅榮光守大沽,提督唐仁廉守北塘,提督曹克忠、總兵葉志超守山海關內外,總兵全祖凱守煙臺,首尾聯絡,海疆屹然。十一年,法大敗於諒山。計窮,復尋成。授全權大臣,與法使巴德納增減前約。事平,下部議敘。是年朝鮮亂黨入王宮,戕執政大臣六人。提督吳兆有以兵入護,誅亂黨,傷及日本兵。日人要索議統將罪,鴻章嚴拒之,而允以撤兵寢其事。九月,命會同醇親王辦理海軍。

十二年,以全權大臣定法國通商滇粵邊界章程。十三年,會訂葡萄牙通商約。十四年,海軍成船二十八,檄飭海軍提督丁汝昌統率全隊,周歷南北印度各海面,習風濤,練陣技,歲率為常。十五年,太后歸政,賞用紫韁。十七年,平熱河教匪,議敘。十九年正月,鴻章年七十,兩宮賜「壽」。二十年,賞戴三眼花翎,而日朝變起。

初,鴻章籌海防十餘年,練軍簡器,外人震其名,謂非用師逾十萬,不能攻旅順,取天津、威海。故俄、法之警,皆知有備而退。至是,中興諸臣及湘淮軍名將皆老死,鮮有存者。鴻章深知將士多不可恃,器械缺乏不應用,方設謀解紛難,而國人以為北洋海軍信可恃,爭起言戰,廷議遂銳意用兵。初敗於牙山,繼敗於平壤,日本乘勝內侵,連陷九連、鳳凰諸城,大連、旅順相繼失。復據威海衛、劉公島,奪我兵艦,海軍覆喪殆盡。於是議者交咎鴻章,褫其職,以王文韶代督直隸,命鴻章往日本議和。二十一年二月,抵馬關,與日本全權大臣伊藤博文、陸奧宗光議,多要挾。鴻章遇刺傷面,創甚,而言論自若,氣不少衰。日皇遣使慰問謝罪,卒以此結約解兵。會訂條款十二,割臺灣畀之,日本悉交還侵地。七月,回京,入閣辦事。

十二月,俄皇加冕,充專使致賀,兼聘德、法、英、美諸國。二十二年正月,陛辭,上念垂老遠行,命其子經方、經述侍行。外人夙仰鴻章威望,所至禮遇逾等,至稱為東方畢士馬克。與俄議新約,由俄使經總署訂定,世傳「中俄密約」。七閱月,回京復命。兩宮召見,慰勞有加,命直總理各國事務衙門。

二十三年,充武英殿總裁。二十四年,命往山東查勘黃河工程。疏稱遷民築堤,成工匪易,惟擇要加修兩岸堤墊,疏通海口尾閭,為救急治標之策。下其奏,核議施行。

十月,出督兩廣。二十六年,賞用方龍補服。拳匪肇亂,八國聯軍入京,兩宮西狩。詔鴻章入朝,充議和全權大臣,兼督直隸,有「此行為安危存亡所系,勉為其難」之語。鴻章聞警兼程進,先以兵剿畿甸匪,孑身入京,左右前後皆敵軍,日與其使臣將帥爭盟約,卒定和約十二款。二十七年七月,講成,相率退軍。

大亂之後,公私蕩然。鴻章奏陳善後諸務。開市肆,通有無,施粥散米,中外帖然。並奉詔行新政,設政務處,充督辦大臣,旋署總理外務部事。積勞嘔血薨,年七十有九。事聞,兩宮震悼,錫祭葬,贈太傅,晉封一等侯,謚文忠。入祀賢良祠,安徽、浙江、江蘇、上海、江寧、天津各建祠以祀,並命於京師特建專祠。漢臣祀京師,蓋異數也。

鴻章長軀疏髯,性恢廓,處榮悴顯晦及事之成敗,不易常度,時以詼笑解紛難。尤善外交,陰陽開闔,風採凜然。外國與共事者,皆一時偉人。及八國定盟,其使臣大將多後進,視鴻章皆丈人行也,故兵雖勝,未敢輕中國。聞其薨,咸集吊唁,曰:「公所定約不敢渝。」其任事持大體,不為小廉曲謹。自壯至老,未嘗一日言退,嘗以曾國籓晚年求退為無益之請,受國大任,死而後已。馬關定約還,論者未已,或勸之歸。鴻章則言:「於國實有不能恝然之誼,今事敗求退,更誰賴乎?」其忠勤皆類此。居恆好整以暇,案上置宋搨蘭亭,日臨摹百字,飲食起居皆有恆晷。長於奏牘,時以曾、李並稱云。鴻章初以兄子經方為子,後生子經述,賞四品京堂,襲侯爵;經邁,侍郎。

論曰:中興名臣,與兵事相終始,其勛業往往為武功所掩。鴻章既平大難,獨主國事數十年,內政外交,常以一身當其沖,國家倚為重輕,名滿全球,中外震仰,近世所未有也。生平以天下為己任,忍辱負重,庶不愧社稷之臣;惟才氣自喜,好以利祿驅眾,志節之士多不樂為用,緩急莫恃,卒致敗誤。疑謗之起,抑豈無因哉?


\end{pinyinscope}