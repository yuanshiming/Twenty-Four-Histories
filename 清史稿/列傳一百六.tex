\article{列傳一百六}

\begin{pinyinscope}
於敏中和珅弟和琳蘇凌阿

於敏中,字叔子,江蘇金壇人。乾隆三年一甲一名進士,授翰林院修撰。以文翰受高宗知,直懋勤殿,敕書華嚴、楞嚴兩經。累遷侍講,典山西鄉試,督山東、浙江學政。十五年,直上書房。累遷內閣學士。十八年,復督山東學政。擢兵部侍郎。二十一年,丁本生父憂,歸宗持服。逾年,起署刑部侍郎。二十三年,嗣父枋歿,回籍治喪。未幾,丁本生母憂,未以上聞。御史硃嵇疏劾敏中「兩次親喪,蒙混為一,恝然赴官」。並言:「部臣與疆臣異,不宜奪情任事。」詔原之。尋實授。調戶部,管錢法堂事。二十五年,命為軍機大臣。敏中敏捷過人,承旨得上意。三十年,擢戶部尚書。子齊賢,鄉試未中式。詔以敏中久直內廷,僅一子年已及壯,加恩依尚書品級予廕生。又以敏中正室前卒,特封其妾張為淑人。三十三年,加太子太保。三十六年,協辦大學士。

三十八年,晉文華殿大學士,兼戶部尚書如故。時下詔徵遺書,安徽學政硃筠請開局搜輯永樂大典中古書。大學士劉統勛謂非政要,欲寢其議。敏中善筠奏,與統勛力爭,於是特開四庫全書館,命敏中為正總裁,主其事。又充國史館、三通館正總裁。屢典會試,命為上書房總師傅,兼翰林院掌院學士。

敏中為軍機大臣久,頗接外吏,通聲氣。三十九年,內監高雲從漏洩硃批道府記載,下廷臣鞫治。雲從言敏中嘗向詢問記載,及雲從買地涉訟,嘗乞敏中囑託府尹蔣賜棨。上面詰,敏中引罪,詔切責之曰:「內廷諸臣與內監交涉,一言及私,即當據實奏聞。朕方嘉其持正,重治若輩之罪,豈肯轉咎奏參者?於敏中侍朕左右有年,豈尚不知朕而為此隱忍耶?於敏中日蒙召對,朕何所不言?何至轉向內監探詢消息?自川省用兵以來,敏中承旨有勞。大功告竣,朕欲如張廷玉例,領以世職。今事垂成,敏中乃有此事,是其福澤有限,不能受朕深恩,寧不痛自愧悔?免其治罪,嚴加議處。」部議革職,詔從寬留任。四十一年,金川平,詔嘉其勞勩,過失可原,仍列功臣,給一等輕車都尉,世襲罔替。四十四年,病喘,遣醫視,賜人蓡。卒,優詔賜恤,祭葬如例,祀賢良祠,謚文襄。

子齊賢,前卒。孫德裕,襲世職,以主事用。敏中從侄時和,擁其貲回籍,德裕訟之。江蘇巡撫吳壇察治,罪時和,戍伊犁。所侵奪者,還德裕三萬兩,餘充金壇開河用。

蘇松糧道章攀桂為敏中營造花園,事覺,褫攀桂職。敏中受地方官逢迎,以已卒置不論。既而浙江巡撫王亶望以貪敗,上追咎敏中。五十一年,詔曰:「朕幾餘詠物,有嘉靖年間器皿,念及嚴嵩專權煬蔽,以致國是日非,朝多稗政。取閱嚴嵩傳,見其賄賂公行,生死予奪,潛竊威柄,實為前明奸佞之尤。本朝家法相承,紀綱整肅,太阿從不下移,本無大臣專權之事。原任大學士於敏中因任用日久,恩眷稍優。無識之徒,心存依附,敏中亦遂時相招引,潛受苞苴。其時軍機大臣中無老成更事之人,福康安年輕,未能歷練,以致敏中聲勢略張。究之亦止侍直承旨,不特非前朝嚴嵩可比,並不能如康熙年間明珠、徐乾學、高士奇等;即寵眷亦尚不及鄂爾泰、張廷玉,安能於朕前竊弄威福、淆亂是非耶?朕因其宣力年久,身故仍加恩飾終,準入賢良祠。迨四十六年甘肅捐監折收之事敗露,王亶望等侵欺貪黷,罪不容誅。因憶此事前經舒赫德奏請停止,於敏中於朕前力言甘肅捐監應開,部中免撥解之煩,閭閻有糶販之利,一舉兩得,是以準行。詎知勒爾謹為王亶望所愚,通同一氣,肥橐殃民。非於敏中為之主持,勒爾謹豈敢遽行奏請?王亶望豈敢肆無忌憚?於敏中擁有厚貲,必出王亶望等賄求酬謝。使於敏中尚在,朕必嚴加懲治。今不將其子孫治罪,已為從寬;賢良祠為國家風勵有位盛典,豈可以不慎廉隅之人濫行列入?朕久有此心,因覽嚴嵩傳,觸動鑒戒。恐無知之人,將以明世宗比朕,朕不受也。於敏中著撤出賢良祠,以昭儆戒。」六十年,國史館進呈敏中列傳,詔曰:「於敏中簡任綸扉,不自檢束,既向宦寺交接,復與外省官吏夤緣舞弊。即此二節,實屬辜恩,非大臣所應有。若仍令濫邀世職,何以示懲?其孫於德裕現官直隸知府,已屬格外恩施,所襲輕車都尉世職即撤革,以為大臣營私玷職者戒。」

和珅,字致齋,鈕祜祿氏,滿洲正紅旗人。少貧無藉,為文生員。乾隆三十四年,承襲三等輕車都尉。尋授三等侍衛,挑補黏桿處。四十年,直乾清門,擢御前侍衛,兼副都統。次年,遂授戶部侍郎,命為軍機大臣,兼內務府大臣,駸駸鄉用。又兼步軍統領,充崇文門稅務監督,總理行營事務。四十五年,命偕侍郎喀凝阿往雲南按總督李侍堯貪私事。侍堯號才臣,帝所倚任。和珅至,鞫其僕,得侍堯婪索狀,論重闢,奏雲南吏治廢弛,府州縣多虧帑,亟宜清釐。上欲用和珅為總督,嫌於事出所按劾,乃以福康安代之。命回京,未至,擢戶部尚書、議政大臣。及復命,面陳云南鹽務、錢法、邊事,多稱上意,並允行。授御前大臣兼都統。賜婚其子豐紳殷德為和孝公主額駙,待年行婚禮。又授領侍衛內大臣,充四庫全書館正總裁,兼理籓院尚書事,寵任冠朝列矣。

四十六年,甘肅撒拉爾番回蘇四十三等叛,逼蘭州,額駙拉旺多爾濟、領侍衛內大臣海蘭察、護軍額森特等率兵討之。命和珅為欽差大臣,偕大學士阿桂往督師。阿桂有疾,促和珅兼程先進。至則海蘭察等已擊賊勝之,即督諸將分四路進兵,海蘭察逼賊山梁,殲其伏。賊掘溝坎深數丈,並斷小道,不能度。總兵圖欽保陣亡。後數日,阿桂至,和珅委過諸將不聽調遣。阿桂曰:「是宜誅!」明日,同部署戰事,阿桂所指揮,輒應如響。乃曰:「諸將殊不見其慢,當誰誅?」和珅恚甚。上微察之,詔斥和珅匿圖欽保死事不上聞,赴師遲延,而劾海蘭察、額森特先戰顛倒是非;又謂自阿桂至軍,措置始有條理,一人足辦賊,和珅在軍事不歸一,海蘭察等久隨阿桂,易節制,命和珅速回京。和珅用是銜阿桂,終身與之齟。尋兼署兵部尚書,管理戶部三庫。

四十七年,御史錢灃劾山東巡撫國泰、布政使於易簡貪縱營私,命和珅偕都御史劉墉按鞫,灃從往。和珅陰袒國泰,即至,盤庫,令抽視銀數十封無缺,即起還行館。灃請封庫,明日盡發視庫銀,得借市銀充抵狀,國泰等罪皆鞫實。會加恩中外大臣,加太子太保,充經筵講官。四十八年,賜雙眼花翎,充國史館正總裁、文淵閣提舉閣事、清字經館總裁。甘肅石峰堡回匪平,以承旨論功,再予輕車都尉世職,並前職授一等男爵。調吏部尚書、協辦大學士,管理戶部如故。

五十一年,御史曹錫寶劾和珅家奴劉全奢僭,造屋逾制,帝察其欲劾和申,不敢明言,故以家人為由。命王大臣會同都察院傳問錫寶,使直陳和珅私弊,卒不能指實。和珅亦預使劉全毀屋更造,察勘不得直,錫寶因獲譴。逾月,授和珅文華殿大學士。詔以其管崇文門監督已閱八年,大學士不宜兼榷務,且錫寶劾其家人,未必不因此,遂罷其監督。部員湛露擢廣信知府,上見其年幼,不勝方面,斥和珅濫保。又兩廣總督富勒渾縱容家人婪索,和珅請調回富勒渾,不興大獄。京師米貴,和珅請禁囤積,逾五十石者交廠減糶,商民以為不便。廷臣遷就原議,上並切責之。五十三年,以臺灣逆匪林爽文平,晉封三等忠襄伯,賜紫韁。五十五年,賜黃帶、四開褉袍。上八旬萬壽,命和珅偕尚書金簡專司慶典事。內閣學士尹壯圖疏論各省庫藏空虛,上為動色,和珅請即命壯圖往勘各省庫,以侍郎慶成監之。慶成每至一省輒掣肘,待挪移既足,然後啟榷,迄無虧絀,壯圖以妄言坐黜。

五十六年,刻石經於闢雍,命為正總裁。時總裁八人,尚書彭元瑞獨任校勘,敕編石經考文提要,事竣,元瑞被優賚。和珅嫉之,毀元瑞所編不善,且言非天子不考文。上曰:「書為御定,何得目為私書耶?」和珅乃使人撰考文提要舉正以攻之,冒為己作進上,訾提要不便士子,請銷毀,上不許。館臣疏請頒行,為和珅所阻,中止,復私使人磨碑字,凡從古者盡改之。

五十七年,廓爾喀平,予議敘,兼翰林院掌院學士。六十年,充殿試讀卷官,教習庶吉士。時朝審停勾,情重者請旨裁定。和珅管理籓院,於蒙古重獄置未奏,鐫級留任。又廷試武舉發策,上命檢實錄。故事,實錄不載武試策問,和申率對不以實,詔斥護過飾非,革職留任。先是京察屢邀議敘,是年特停罷之。嘉慶二年,調管刑部。尋以軍需報銷,仍兼管戶部。三年,教匪王三槐就擒,以襄贊功晉公爵。

和珅柄政久,善伺高宗意,因以弄竊作威福,不附己者,伺隙激上怒陷之;納賄者則為周旋,或故緩其事,以俟上怒之霽。大僚恃為奧援,剝削其下以供所欲。鹽政、河工素利藪,以徵求無厭日益敝。川、楚匪亂,因激變而起,將帥多倚和珅,糜餉奢侈,久無功。阿桂以勛臣為首輔,素不相能,被其梗軋。入直治事,不與同止直廬。阿桂卒,益無顧忌,於軍機寄諭獨署己銜。同列嵇璜年老,以讒數被斥責。王傑持正,恆與忤,亦不能制。硃珪舊為仁宗傅,在兩廣總督任,高宗欲召為大學士,和珅忌其進用,密取仁宗賀詩白高宗,指為市恩。高宗大怒,賴董誥諫免;尋以他事降珪安徽巡撫,屏不得內召。言官惟錢灃劾其黨國泰得直,後論和珅與阿桂入直不同止直廬,奉命監察,以勞瘁死。曹錫寶、尹壯圖皆獲譴,無敢昌言其罪者。高宗雖遇事裁抑,和珅巧彌縫,不悛益恣。仁宗自在潛邸知其奸,及即位,以高宗春秋高,不欲遽發,仍優容之。

四年正月,高宗崩,給事中王念孫首劾其不法狀,仁宗即以宣遺詔日傳旨逮治,命王大臣會鞫,俱得實。詔宣布和珅罪狀,略曰:「朕於乾隆六十年九月初三日,蒙皇考冊封皇太子,尚未宣布,和珅於初二日在朕前先遞如意,以擁戴自居,大罪一。騎馬直進圓明園左門,過正大光明殿,至壽山口,大罪二。乘椅橋入大內,肩輿直入神武門,大罪三。取出宮女子為次妻,大罪四。於各路軍報任意壓擱,有心欺蔽,大罪五。皇考聖躬不豫,和珅毫無憂戚,談笑如常,大罪六。皇考力疾批答章奏,字跡間有未真,和珅輒謂不如撕去另擬,大罪七。兼管戶部報銷,竟將戶部事務一人把持,變更成例,不許部臣參議,大罪八。上年奎舒奏循化、貴德二賊番肆劫青海,和珅駁回原摺,隱匿不辦,大罪九。皇考升遐後,朕諭蒙古王公未出痘者不必來京,和珅擅令已、未出痘者俱不必來,大罪十。大學士蘇凌阿重聽衰邁,因與其弟和琳姻親,隱匿不奏;侍郎吳省蘭、李潢,太僕寺卿李光雲在其家教讀,保列卿階,兼任學政,大罪十一。軍機處記名人員任意撤去,大罪十二。所鈔家產,楠木房屋僭侈逾制,仿照寧壽宮制度,園寓點綴與圓明園蓬島、瑤臺無異,大罪十三。薊州墳塋設享殿,置隧道,居民稱和陵,大罪十四。所藏珍珠手串二百餘,多於大內數倍,大珠大於御用冠頂,大罪十五。寶石頂非所應用,乃有數十,整塊大寶石不計其數,勝於大內,大罪十六。藏銀、衣服數逾千萬,大罪十七。夾墻藏金二萬六千餘兩,私庫藏金六千餘兩,地窖埋銀三百餘萬兩,大罪十八。通州、薊州當鋪、錢店貲本十餘萬,與民爭利,大罪十九。家奴劉全家產至二十餘萬,並有大珍珠手串,大罪二十。」內外諸臣疏言和珅罪當以大逆論,上猶以和珅嘗任首輔,不忍令肆市,賜自盡。

諸劾和珅者比於操、莽。直隸布政使吳熊光舊直軍機,上因其入覲,問曰:「人言和珅有異志,有諸?」熊光曰:「凡懷不軌者,必收人心,和珅則滿、漢幾無歸附者,即使中懷不軌,誰肯從之?」上曰:「然則治之得無太急?」熊光曰:「不速治其罪,無識之徒觀望夤緣,別滋事端。發之速,是義之盡;收之速,是仁之至。」上既誅和珅,宣諭廷臣:「凡為和珅薦舉及奔走其門者,悉不深究,勉其悛改,咸與自新。」有言和珅家產尚有隱匿者,亦斥不問。和珅在位時,令奏事者具副本送軍機處;呈進方物,必先關白,擅自準駁,遇不全納者悉入私家。步軍統領巡捕營在和珅私宅供役者千餘人,又令各部以年老平庸之員保送御史。至是,悉革其弊。吏、戶兩部成例為和珅所變更者,諸臣奏請次第修正。初,乾隆中命和珅改入正黃旗,及得罪,仍隸正紅旗。

子豐紳殷德,尚固倫和孝公主,累擢都統兼護軍統領、內務府大臣。和珅伏法,廷臣議奪爵職。詔以公主故,留襲伯爵。尋以籍沒家產,正珠朝珠非臣下所應有,鞫家人,言和珅時於燈下懸掛,臨鏡自語。仁宗怒,褫豐紳殷德伯爵,仍襲舊職三等輕車都尉。嘉慶七年,川、楚、陜教匪平,推恩給民公品級,授散秩大臣。未幾,公主府長史奎福訐豐紳殷德演習武藝,謀為不軌,欲害公主。廷臣會鞫,得誣告狀。詔以豐紳殷德與公主素和睦,所作青蠅賦,憂讒畏譏,無怨望違悖;惟坐國服內侍妾生女罪,褫公銜,罷職在家圈禁。十一年,授頭等侍衛,擢副都統,賜伯爵銜。十五年,病,乞解任,賜公爵銜。尋卒。無子,以和琳子豐紳伊綿襲輕車都尉。

和珅伏法後越十五年,國史館以列傳上。仁宗以事跡疏略,高宗數加譴責,闕而未載,無以信今傳後,褫編修席煜職,特詔申戒焉。

弟和琳,自筆帖式累遷湖廣道御史。劾湖北按察使李天培私交糧艘帶運木植,鞫得兩廣總督福康安寄書索購狀,帝嘉和琳伉直,下部議敘,由是遂見擢用。自吏部給事中超擢內閣學士,兼禮部侍郎銜。尋授兵部侍郎、正藍旗漢軍副都統。廓爾喀擾後藏,將軍福康安往剿,帝命和琳督辦前藏以東臺站烏拉等事。尋命與鄂輝更番照料糧餉,擢工部尚書。疏陳賊酋拉特納巴都爾悔罪狀,詔令福康安受降,偕和琳妥籌善後。未幾,授鑲白旗漢軍都統。命偕孫士毅、惠齡覈辦察木多以西銷算事,仍理藏務。乾隆五十八年,予雲騎尉世職。五十九年,授四川總督。六十年,貴州苗石柳鄧叛,擾正大、嗅腦、松桃,湖南苗吳半生、石三保應之,圍永綏,帝命云貴總督福康安往剿。和琳時方入京,至卬州,松桃匪已闌入秀山境。和琳聞警馳往,督參將張志林、都司馬瑜擊走之;後復敗賊晏農,進攻砲木山黃陂,通道松桃:賞雙眼花翎。時福康安已解正大、嗅腦、松桃圍,攻石柳鄧於大塘汛,和琳率兵會之,遂命參贊軍事;克蝦覅碉、烏龍巖,降七十餘寨,封一等宣勇伯。復攻下巖碧山,賞上服貂褂。又以降吳半生功,賞黃帶。龍角碉、鴨保、天星諸寨大捷,加太子太保,賞玄狐端罩。嘉慶元年,克結石岡、廖家沖、連峰蜺諸隘,賞用紫韁。福康安卒,命和琳督辦軍務。時石三保已就獲,石柳鄧尚據平隴。奪尖雲山砲臺,復乾州,賞三眼花翎。八月,進圍平隴,卒於軍。晉贈一等公,謚忠壯,賜祭葬,命配饗太廟,祀昭忠、賢良等祠,準其家建專祠。四年,和珅誅,廷臣論和琳藉勢邀功,上亦追咎其會剿苗匪,牽掣福康安,師無功,命撤出太廟,毀專祠,奪其子豐紳伊綿公爵,改襲三等輕車都尉。

蘇凌阿,滿洲正白旗人。乾隆六年繙譯舉人。自內閣中書累遷江西廣饒九南道。左遷。五十年,自吏部員外郎超擢,歷兵、工、戶三部侍郎。遷戶部尚書。出為兩江總督。嘉慶二年,授東閣大學士,兼署刑部尚書。和珅誅,休致,守護裕陵。卒。

論曰:高宗英毅,大臣有過失,不稍假借。世傳敏中以高雲從事失上意,有疾,令休沐,遽賜陀羅尼經被,遂以不起聞。觀罷祠之詔,至引嚴嵩為類,傳聞有無未可知矣。和珅繼用事,值高宗倦勤,怙寵貪恣,卒以是敗。仁宗嘗論唐代宗殺李輔國,謂:「代宗為太子,不為輔國所讒者幾希。及即帝位,正其罪而誅之,一獄吏已辦。」蓋即為和珅發也。


\end{pinyinscope}