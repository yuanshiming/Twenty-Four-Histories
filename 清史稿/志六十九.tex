\article{志六十九}

\begin{pinyinscope}
樂一

記曰:「安上治民,莫善於禮。」「移風易俗,莫善於樂。」樂也者,考神納賓,類物表庸,以其德馨殷薦上帝者也。聖道四達,聲與政通,於是有綴兆之容,箾籥之音,被服其光輝,膏潤其猷烈,以與民康之,民無憔瘁揫傷之嗟,放僻嫚蕩之志,夫然後雅頌作焉。蓋三苗格而韶舞,十一稅而頌謳,玄鳥歌而商祚興,靈臺奏而周道昌。王官失守,神不降祉。移及春秋,脊脊大亂。仲尼序詩,列黍離於國風,齊王德於邦君,明其不能復雅。中更暴秦,樂經埃滅,音之鄭衛,自此而階,郊廟登歌,聲不逮下。擾民齊教,無聞焉爾。然而歷代創興,莫不鋪陳南雅,自制郊辭,繩祖業之維艱,頌帝功之有赫,考較鐘懸,裁定縟典。雖渾灝三五,炳焉同風,而寤想聞韶,跂之彌卲。是則前誥所譏,鄰於夜誦者也。

清起僻遠,迎神祭天,初沿邊俗。及太祖受命,始習華風。天命、崇德中,徵瓦爾喀,臣朝鮮,平定察哈爾,得其宮懸,以備四裔燕樂。世祖入關,修明之舊,有中和韶樂,郊廟朝會用之。有丹陛大樂,王公百僚慶賀用之。有中和清樂、丹陛清樂,宮中筵宴用之。有鹵簿導迎樂,巡蹕用之。又制鐃歌法曲,奮武敵愾,宣鬯八風,以儷漢世短簫。而滿洲舊舞,是曰莽式,率以蘭錡世裔充選,所陳皆遼沈故事,作麾旄弢矢躍馬涖陣之容,屈伸進反輕蹻俯仰之節,歌辭異漢,不頒太常,所謂纘業垂統,前王不忘者歟?

聖祖、高宗,制作自任,臣匪師曠之聰,君逾姬旦之美。考音諧金石,昭德摛天漢,帝秩皇造,於斯為盛。但觀其命伶倫使協律。召咸黑以賡歌,非不陶英鑄莖,四隅率同,而繼體再傳,頌聲浸廢。魏文聽之而思臥,季札觀之而無譏,是知樂之為懿,覘國隆洿,謳歌在民,匪所自致,而三古承流,曾靡先覺,可為惋欷者也。

稽清之樂,式遵明故,六間七始,實紹古亡,布咫𦾑禾,譣氣灰琯。斯乃神瞽以之塞(流玉),隸首由其踠步。將欲起元音之廢,復淳樸之真,弘我夏聲,秕乞西奏。澹欲繕性,一綖庶幾,有庇經誥,其或在此。必監前憲,我則優矣。國宬所書,聲容器數之次第,管律弦度之討論,煥乎秩秩,可謂有文。今掇其要,以備簡籍。

太祖肇啟東陲,戡亂用武,聲物弇樸,率緣遼舊。天命元年,即尊位沈陽,諸貝勒群臣廷賀上壽,始制鹵簿用樂。八年,定凱旋拜天行禮筵宴樂制。太宗天聰八年,又定出師謁堂子拜天行禮樂制、元旦朝賀樂制。九年,停止元旦雜劇。先是梅勒章京張存仁上言:「元旦朝賀,大體所關,雜劇戲謔,不宜陳殿陛。故事,八旗設宴,惟用雅樂。」從之。

十年,建國號曰清,改元崇德。其明年,遂有事太廟,追尊列祖,四孟時享、歲暮祫祭並奏樂。皇帝冬至、萬壽二節與元旦同。御前儀仗樂器,鑼二,鼓二,畫角四,簫二,笙二,架鼓四,橫笛二,龍頭橫笛二,檀板二,大鼓二,小銅鈸四,小銅鑼二,大銅鑼四,雲鑼二,嗩吶四。樂人綠衣黃褂紅帶,六瓣紅絨帽,銅頂上綴黃翎,從內院官奏請也。又詔公主冊封、諸王家祭、受降獻馘皆用樂。

世祖順治元年,攝政睿親王多爾袞既定燕都,將於十月告祭天地宗廟社稷,大學士馮銓、洪承疇等言:「郊廟及社稷樂章,前代各取嘉名,以昭一代之制,梁用『雅』,北齊及隋用『夏』,唐用『和』,宋用『安』,金用『寧』,元宗廟用『寧』、郊社用『咸』,前明用『和』。我朝削平寇亂,以有天下,宜改用『平』。郊社九奏,宗廟六奏,社稷七奏。」從之。於是定圜丘大祀,皇帝出宮,午門聲鐘,不作樂。致祭燔柴迎神奏始平,奠玉帛奏景平,進俎奏咸平,初獻奏壽平,亞獻奏嘉平,終獻奏雍平,徹饌奏熙平,送神奏太平,望燎奏安平。禮成,教坊司導迎,樂奏祐平。午門鐘作,還宮。方澤大祀,皇帝出宮,午門聲鐘,不作樂。致祭瘞毛血迎神奏中平,奠玉帛奏廣平,進俎奏咸平,初獻奏壽平,亞獻奏安平,終獻奏時平,徹饌奏貞平,送神望瘞奏寧平。禮成,教坊司導迎,樂奏祐平。午門鐘作,還宮。祈穀,皇帝出宮,午門聲鐘,不作樂。燔柴迎神奏中平,奠玉帛奏肅平,進俎奏咸平,初獻奏壽平,亞獻奏景平,終獻奏永平,徹饌奏凝平,送神奏清平,望燎奏太平。餘與圜丘、方澤同。太廟時享,皇帝出宮,鐘止,不作樂。致祭迎神奏開平,奠帛初獻奏壽平,亞獻奏嘉平,終獻奏雍平,徹饌奏熙平,送神望燎奏成平。禮成,教坊司導迎奏禧平,聲鐘還宮。社稷壇,皇帝出宮,聲鐘,不作樂。致祭瘞毛血迎神奏廣平,奠玉帛初獻奏壽平,亞獻奏嘉平,終獻奏雍平,徹饌奏熙平,送神望瘞奏成平。禮成,教坊司導迎奏祐平,聲鐘還宮。

舞皆八佾,初獻武舞,亞獻、終獻文舞,文武舞生各六十四人,執干戚羽籥於樂懸之次,引舞旌節四,舞生四人司之。祭之日,初獻樂作,司樂執旌節,引武舞生執干戚進,奏武功之舞。亞獻、終獻樂作,司樂執旌節,引文舞生執羽籥進,奏文德之舞。惟先師廟祗文舞六佾。

其三大節、常朝及皇帝升殿、還宮,俱奏中和韶樂,群臣行禮,奏丹陛大樂。親祭壇廟,乘輿出入,用導迎樂,樂章均用「平」字。宴享清樂,則以樂詞之首為章名。

是年世祖至京行受寶禮,先期錦衣衛設鹵簿儀仗,旗手衛設金鼓旗幟,教坊司設大樂於行殿西前導。時龜鼎初奠,官懸備物,未遑潤色,沿明舊制雜用之。教坊司置奉鑾一人,左右韶舞各一人,協同官十有五人,俳長二十人,色長十七人,歌工九十八人。宮內宴禮,領樂官妻四人,領教坊女樂二十四人。祠祭諸樂,則太常寺神樂觀司之。以協律郎教習樂生,月三、六、九日演於凝禧殿。

二年,從有司言,春秋上丁釋奠先師,樂六奏,迎神奏咸平,奠帛初獻奏寧平,亞獻奏安平,終獻奏景平,徹饌送神奏咸平。

祭歷代帝王樂六奏,迎神奏雍平,奠帛初獻奏安平,亞獻奏中平,終獻奏肅平,徹饌奏凝平,送神望燎奏壽平。

八年,制:朝日七奏,樂章用「曦」,迎神奏寅曦,奠玉帛奏朝曦,初獻奏清曦,亞獻奏咸曦,終獻奏純曦,徹饌奏延曦,送神奏歸曦。

夕月六奏,樂章用「光」,迎神奏迎光,奠玉帛初獻奏升光,亞獻奏瑤光,終獻奏瑞光,徹饌奏涵光,送神奏保光,皆中和韶樂。

皇太后、皇后三大節慶賀,皇帝大婚行禮,皆丹陛大樂。

祭真武、東嶽、城隍廟,教坊司作樂如群祀。

是年又允禮部請,更定樂舞、樂章、樂器之數,享廟大樂於殿內奏之,文武佾舞備列樂章卒歌樂器俱設,補舞生舊額五百七十人。

其後又定常朝升殿中和韶樂奏隆平,王公百官行禮丹陛大樂奏慶平,外籓行禮丹陛大樂奏治平,還宮中和韶樂奏顯平。耤田饗先農,樂章七奏,用「豐」,迎神奏永豐,奠帛初獻奏時豐,亞獻奏咸豐,終獻奏大豐,徹饌奏屢豐,送神奏報豐,望瘞奏慶豐。

禮成,御齋宮,導迎大樂奏天下樂,升座奏萬歲樂,群臣行禮丹陛大樂奏朝天子,筵宴上壽奏三月韶光,進饌清樂奏太清歌。

太廟祫祭迎神奏貞平,奠帛初獻奏壽平,亞獻奏嘉平,終獻奏雍平,徹饌奏熙平,送神奏清平。

大享殿合祀天地百神,樂章九奏,用「和」,迎神奏元和,奠玉帛奏景和,進俎奏肅和,初獻奏壽和,亞獻奏安和,終獻奏永和,徹饌奏協和,送神奏泰和,望燎瘞奏清和。

其上皇太后徽號冊寶、尊封太妃、冊立中宮、太和殿策士諸慶典,皆特詔用樂。自後幸盛京、謁陵,進實錄、玉牒亦如之。

康熙初,聖祖踐阼幼沖,率承舊憲,無所改作。八年,惟詔定皇帝、太皇太后、皇太后、皇后三大節朝賀樂,皇帝元旦升座中和韶樂奏元平,還宮奏和平,冬至升座奏遂平,還宮奏允平,萬壽節升座奏乾平,還宮奏太平,群臣行禮丹陛大樂奏慶平,外籓奏治平,太皇太后升座奏升平,還宮奏恆平,行禮奏晉平,皇太后升座奏豫平,還宮奏履平,行禮奏益平,皇后升座奏淑平,還宮奏順平,行禮奏正平。而有司肄習日久,樂句律度,凌厲失所,伶倫應官,比於制氏,但紀鏗鏘鼓舞而已。

自世祖時,已屢飭典樂官演習樂舞聲容儀節,嘗諭大學士等曰:「各處祭祀,太常寺所奏樂俱未和諧。樂乃祭祀之大典,必聲容儀節盡合歌章,始臻美善。其召太常寺官嚴飭之。」至十一年,聖祖亦諭禮臣:「慎重禋祀,勤加習練,勿仍前怠,褻越明典。」

二十一年,三籓削平,天下無事,左副都御史餘國柱首請釐正郊廟、朝賀、宴享樂章,上曰:「享祀樂章,一代制作所系,禮部、翰林院其集議以聞。」尋奏:「自古廟樂,原以頌述祖宗功德,本朝郊壇廟祀樂章,曲名曰『平』,遵奉已久。太祖、太宗、世祖同於太廟致祭,宜如舊。惟朝會、宴享等樂曲調,風雅未備,宜敕所司酌古準今,求聲律之原,定雅奏之節。」從之。因命大學士陳廷敬重撰燕樂諸章,然猶襲明故,雖務典蔚,有似徒歌,五聲二變,踵訛奪倫,黃鍾為萬事根本,臣工無能言之者。帝方謙讓,亦未暇革也。

二十三年,東巡謁闕里,躬祭孔林,陳鹵簿,奏導迎大樂樂章、樂舞,先期命太常寺遣司樂官前往肄習,與太學先師廟同。二十九年,以喀爾喀新附,特行會閱禮,陳鹵簿,奏鐃歌大樂,於是帝感禮樂崩隤,始有志制作之事。

三十一年,禦乾清宮,召大學士九卿前,指五聲八風圖示之曰:「古人謂十二律定,而後被之八音,則八音和,奏之天地,則八風和,諸福之物,可致之祥,無不畢至,言樂律所關者大也。而十二律之所從出,其義不可知。律呂新書所言算數,專用徑一圍三之法,此法若合,則所算皆合;若舛,則無所不舛矣。朕觀徑一圍三之法,必不能合,蓋徑一尺,則圍當三尺一寸四分一釐有奇,若積累至於百丈,所差當十四丈有奇,等而上之,舛錯可勝言耶?」因取方圓諸圖謂群臣曰:「所言徑一圍三,但可算六角之數,若圍圓必有奇零。朕觀八線表中半徑句股之法極精微,凡圓者可以方算,開方之術,即從此出。若黃鍾之管九寸,空圍九分,積八百一十分,是為律本,此舊說也。其分寸若以尺言,則古今尺制不同,當以天地之度數為準。惟隔八相生之說,聲音高下,循環相生,復還本音,必須隔八,乃一定之理也。」隨命樂人取笛和瑟次第審音,至第八聲,仍還本音。上曰:「此非隔八相生之義耶?」群臣皆曰:「誠如聖訓,非臣等聞見所及。」

三十四年,定大閱鳴角擊鼓聲金之制。

四十九年正月,孝惠章皇后七十萬壽,又諭禮部曰:「瑪克式舞,乃滿洲筵宴大禮,典至隆重,故事皆王大臣行之。今歲皇太后七旬大慶,朕亦五十有七,欲親舞稱觴。」是日皇太后宮進宴奏樂,上前舞蹈奉爵,極懽乃罷。

帝既妙揅鍾律,時李光地為文淵閣大學士,以耆碩被顧問,會進所纂大司樂釋義及樂律論辨,因上言曰:「禮樂不可斯臾去身,亦不可以一日不行於天下。自漢以來,禮樂崩壞,不合於三代之意者二千餘年,而樂尤甚。蓋自諸經所載節奏、篇章、器數、律呂之昭然者,而紛紛之說,終不能以相一,又況乎精微之旨,與天地同其和者哉!今四海靡靡,風聲頹敝,等威無辨,而奢僭不可止;聯屬無法,而鬥爭不可禁。記曰:『無本不立,無文不行。』神而明之者,本也;舉而措之者,文也。謂宜搜召名儒,以至淹洽古今之士,上監於夏、商,近稽自漢、唐以降,考定斟酌,成一代大典,以淑天下而範萬世。」大學士張玉書亦言:「樂律算數之學,失傳已久,承譌襲舛,莫摘其非;奧義微機,莫探其蘊。臣等躬聆訓誨,猶且一時省寤,而覆算迷蒙;中外臣民,何由共喻?宜特賜裁定,編次成書,頒示四方,共相傳習。正歷來積算之差訛,垂萬世和聲之善法,學術政事,均有裨益。」

帝重違臣下請,五十二年,遂詔修律呂諸書,於蒙養齋立館,求海內暢曉樂律者,光地薦景州魏廷珍、寧國梅成、交河王蘭生任編纂。蘭生故光地所拔士,樂律有神契,硃子琴律圖說,字多譌謬,蘭生以意是正,了然可曉。及被詔入直,所與編校者皆淹雅士,而蘭生學獨深,亦時時折中於帝,遇有疑義,親臨決焉。

其法首明黃鍾為十二律呂根源,以縱黍橫黍定古今尺度,今尺八寸一分,當古尺十寸,橫黍百粒,當縱黍八十一粒。漢志:「黃鍾之長,以子穀秬黍中者,一黍之廣度之,九十分黃鍾之長,一為一分。」廣者橫也,九十分為黃鍾之長,則黃鍾為九十橫黍所累明矣。即以橫黍之度比縱黍,為古尺之比今尺,以古尺為一率,今尺為二率,黃鍾古尺九寸為三率,推得四率七寸二分九釐,即黃鍾今尺之度。律呂新書:黃鍾九寸,空圍九分,積八百一十分,再置古尺,積八百一十分,以九十分歸之,得面冪九方分,用比例相求,面線相等,面積不同。定數圓面積一十萬為一率,方面積一十二萬七千三百二十四為二率,今面冪九方分為三率,推得四率一十一分四十五釐九十豪,開平方得三分三釐八豪五絲一忽,為黃鍾古尺徑數。求周,得十分六釐三豪四絲六忽。即以古尺之積比今尺之積,古尺一百分,自乘再乘得一百萬分為一率,今尺八十一分,自乘再乘得五十三萬一千四十一分為二率,黃鍾積八百一十分為三率,推得四率四百三十分四百六十七釐二百十一豪,即黃鍾今尺之積。以今尺長七寸二分九釐歸之,得面冪五分九十釐四十九豪,求徑得二分七釐四豪一絲九忽,而黃鍾管之縱長體積面徑定矣。

黃鍾既定,於是制律呂同徑之法,以積實容黍為數,三分損益以覈之,黃鍾三分損一,下生林鍾,林鍾三分益一,上生太簇,太簇三分損一,下生南呂,南呂三分益一,上生姑洗,姑洗三分損一,下生應鍾,應鍾三分益一,上生蕤賓,蕤賓三分益一,上生大呂,大呂三分損一,下生夷則,夷則三分益一,上生夾鍾,夾鍾三分損一,下生無射,無射三分益一,上生仲呂。又倍之,自蕤賓以下至應鍾,半之,自黃鍾以下至仲呂,皆六。不用京房變律之說,定宮聲在黃鍾、大呂之間。

黃鍾為宮,次太簇以商應,次姑洗以角應,次蕤賓以變徵應,次夷則以徵應,次無射以羽應,次半黃鍾以變宮應,所謂陽律五聲二變也。至半太簇為清宮,仍應黃鍾焉。大呂為宮,次夾鍾以商應,次仲呂以角應,次林鍾以變徵應,次南呂以徵應,次應鍾以羽應,次半大呂以變宮應,所謂陰呂五聲二變也。至半夾鍾為清宮,仍應大呂焉。旋相為宮,折中取聲,類而不雜。驗之簫笛,工為宮,則凡應商,六應角,五應變徵,乙應徵,上應羽,尺應變宮。

黃鍾為低工,大呂為高工,而分清濁。太簇為低凡,夾鍾為高凡,而分清濁。姑洗為低六,仲呂為高六,而分清濁。蕤賓為低五,林鍾為高五,而分清濁。夷則為低乙,南呂為高乙,而分清濁。無射為低上,應鍾為高上,而分清濁。倍之,則倍無射、倍應鍾為宮聲之右變宮尺字,而分清濁。倍夷則、倍南呂為變宮之右下羽上字,而分清濁。倍蕤賓、倍林鍾為下羽之右下徵乙字,而分清濁。半之,則半黃鍾、半大呂為羽聲之左變宮尺字,而分清濁。半太簇、半夾鍾為變宮之左少宮工字,而分清濁。半姑洗、半仲呂為少宮之左少商凡字,而分清濁。古樂所以起下徵而終清商也。

黃鍾一徑,別其長短,為十二律呂,復助以倍半,而得五聲二變之全,由是制以樂器,以黃鍾之積為本,加分減分,皆用黃鍾之長與徑相比,大加至八倍,則長與徑亦加一倍,小減至八分之一,則長與徑亦減其半。正律呂管十二,倍管六,半管六。黃鍾同形管五十六,亦倍管六,半管六。同形管又生同徑管十一,凡一千三百六十八管。依數立制,以考其度,以審其音。八倍黃鍾之管,聲應正黃鍾之律濁宮低工。七倍黃鍾之管,應大呂之呂清宮高工。六倍黃鍾之管,應太簇之律濁商低凡。五倍黃鍾之管,應夾鍾之呂清商高凡。四倍黃鍾之管,應姑洗之律濁角低六。三倍半黃鍾之管,應仲呂之呂清角高六。三倍黃鍾之管,應蕤賓之律濁變徵低五。三倍宜應仲呂,今高半音而應蕤賓,蓋管體漸小,聲音易別。必於三倍之積,復加正黃鍾之半積,始應仲呂之呂清角高六。半積之理,由此生也。二倍半黃鍾之管,應林鍾之呂清變徵高五。二倍加四分之一黃鍾之管,應夷則之律濁徵低乙。二倍黃鍾之管,不應夷則,而二倍半二倍之間始應之。必以半積復半之,為四分之一,加於二倍之內,其分乃合。四分之一之理,由此生焉。二倍黃鍾之管,應南呂之呂清徵高乙。正加四分之三黃鍾之管,應無射之律濁羽低上。正加四分之二黃鍾之管,應應鍾之呂清羽高上。正加四分之一黃鍾之管,應半黃鍾之律濁變宮低尺。正加八分之一黃鍾之管,應半大呂之呂清變宮高尺。此管與正黃鍾最近,欲取合清宮之分,則以四分之一復半之,為八分之一,加於正黃鍾之分,其聲始應。八分之一之理,由此生焉。

繼此則正黃鍾管聲應半太簇之律,濁宮低工乃與八倍黃鍾之管相和同聲矣。遞減之,黃鍾正積八分之七之管,應大呂之呂。八分之六之管,應太簇之律。八分之五之管,應夾鍾之呂。八分之四之管,應姑洗之律。八分之三分有半之管,應仲呂之呂。八分之三之管,應蕤賓之律。八分之二分有半之管,應林鍾之呂。八分之二又加一分之四分之一之管,應夷則之律。此一分之四分之一,乃正黃鍾三十二分之一,至此三十二分之理生焉。八分之二之管,應南呂之呂。八分之一又加一分之四分之三之管,應無射之律。八分之一又加一分之四分之二之管,應應鍾之呂。八分之一又加一分之四分之一之管,應半黃鍾之律。八分之一又加一分之八分之一之管,應半大呂之呂。此一分之八分之一,乃正黃鍾六十四分之一,至此六十四分之理生焉。而八分之一之管,又應正黃鍾,而為正黃鍾長與徑之半。

自八倍黃鍾至黃鍾八分之一,皆具同徑之十二律呂,皆成一調之五聲二變。推而演之,加黃鍾之積至六十四倍,則同形管長徑皆四倍於正黃鍾,減黃鍾之積至六十四分之一,則同形管長徑皆得正黃鍾四分之一。六十四倍積同形管應正黃鍾,五十六倍積同形管與六十四分之七同形管應大呂,四十八倍積同形管與六十四分之六同形管應太簇,四十倍積同形管與六十四分之五同形管應夾鍾,三十二倍積同形管與六十四分之四同形管應姑洗,二十八倍積同形管與六十四分之三加半同形管應仲呂,二十四倍積同形管與六十四分之三同形管應蕤賓,二十倍積同形管與六十四分之二加半同形管應林鍾,十八倍積同形管與六十四分之二加一分四之一同形管應夷則,十六倍積同形管與六十四分之二同形管應南呂,十四倍積同形管與六十四分之一加一分四之三同形管應無射,十二倍積同形管與六十四分之一加一分四之二同形管應應鍾,十倍積同形管與六十四分之一加一分四之一同形管應半黃鍾,九倍積同形管與六十四分之一加一分八之一同形管應半大呂,六十四分之一同形管仍應正黃鍾,於是十二律呂之同徑異形者,合長短倍半以成旋宮之用。而黃鍾之同形異徑者,因加減實積,亦成旋宮之用。制器求聲,齊於此矣。

雖然,五聲二變管律與弦度又各不同,漢、唐以後,皆宗司馬、淮南之說,以三分損益之術,誤為管音五聲二變之次,復執管子弦音五聲度分,而牽合於十二律呂之中。試截竹為管吹之,黃鍾半律,不與黃鍾合,而合黃鍾者為太簇之半律,則倍半相應之說,在弦音而非管音也。又黃鍾為宮,其徵聲不應於林鍾而應於夷則,則三分損益宮下生徵之說,在弦度而非管律也。以弦度取聲,全弦與半弦之音相應,而半律較全律則下一音。蓋弦之體,實藉人力鼓動而生聲,全弦長,故得音緩,半弦短,故得音急,長短緩急之間,全半相應之理寓焉。管之體虛,假人氣入之以生聲,故管之徑同者,其全半不相應,求其相應,必徑減半始得,所以正黃鍾與黃鍾八分之一之管相應同聲也。

因全半之不同,於是管律弦度首音至八音,其間所生五聲二變之度分亦異。管律黃鍾之全為宮聲首音,則太簇之半為少宮八音,其間太簇之全為商聲二音,姑洗為角聲三音,蕤賓為變徵四音,夷則為徵聲五音,無射為羽聲六音,黃鍾之半為變宮七音。自首音至第八音,得七全分。若弦度假借黃鍾全分為宮聲首音,則黃鍾之半為少宮八音,其間太簇之分為商聲二音,姑洗之分為角聲三音,蕤賓之分為變徵四音,而林鍾之分乃為徵聲五音,南呂之分為羽聲六音,應鍾之分為變宮七音。各弦之分,宮至商,商至角,角至變徵,徵至羽,羽至變宮,皆得全分,而變徵至徵,變宮至少宮,祗得半分。自首音至八音,合為六全分,故弦音不可以十二律呂之度取分。如以倍無射變宮尺字定弦,則得下徵之分。倍無射變宮尺字,即今笛與頭管之合字也。凡品樂居首一弦,必得下徵之分,而五音之位始正。故世以頭管合字定琴之一弦為黃鍾之宮者,蓋一弦不得不定以合字,正為取下徵之分也。

黃鍾宮聲工字定弦,得下羽之分;太簇商聲凡字定弦,得變宮之分;姑洗角聲六字定弦,得宮弦之分;蕤賓變徵五字定弦,得商弦之分;夷則徵聲乙字定弦,得角弦之分;無射羽聲上字定弦,得變徵之分;而半黃鍾變宮尺字定弦,仍得徵弦之分焉。今借黃鍾之分為宮弦全分,其首音仍定以黃鍾之律,則二音限於太簇之分,而聲亦應太簇之律,三音則變為夾鍾之分,而聲始應姑洗之律。如仍取姑洗之分,則聲必變而應於仲呂之呂,四音復變為仲呂之分,而聲應蕤賓之律。如仍取蕤賓之分,則聲必變而應於林鍾之呂,五音則為林鍾之分而應夷則之律,六音則為南呂之分而應無射之律,七音又變為無射之分而聲始應半黃鍾之律。如仍取應鍾之分,則聲必變而應於半大呂之呂。此宮弦之分因全弦首音定黃鍾之律,而變為羽弦之分者也。或以黃鍾之分為宮弦全分,而本弦七音欲各限以宮弦內七音之分,則首音必定以姑洗之律。以次分之,此宮弦之分因全弦首音定姑洗之律,而得宮弦之分者也。又或以笛與頭管合字為今所定倍無射之律為宮弦全分,首音依次分之,得下徵弦之分,此宮弦之分因全弦首音定以笛之合字而變為徵弦之分者也。依律呂而定弦音,則弦度之分隨之潛移,依弦度之分命為七音之次,則聲音宮調不與律呂相協。此由管律、弦度全、半生聲取分之不同,於是絲樂弦音之旋宮轉調,與竹樂管音亦異。

清濁二均各七調,中與管樂有同者,有可同者,有不可同者。同者惟宮調一調,五聲二變皆正應。可同者,商調、徵調五聲正、應二變借用;不可同者,角調、變徵調、羽調、變宮調五聲之內清濁相淆。如但以弦音奏之,而不和以管音,祗有四調,餘三調皆轉入弦音宮調。故周禮大司樂三宮,漢志三統,皆以三調為準。所謂三統,其一天統,黃鍾為宮,乃黃鍾宮聲位羽起調,姑洗角聲立宮,主調是為宮調也。其一人統,太簇為宮,乃太簇商聲位羽起調,蕤賓變徵立宮,主調是為商調也。其一地統,林鍾為宮,乃弦音徵分位羽,實管音夷則徵聲位羽起調,半黃鍾變宮立宮,主調是為徵調也。隋志鄭譯云:考尋律呂,七聲之內,三聲乖應。當時考較聲律,或以管音考核弦音,或以弦音考核管音,故得四調相和,三調乖應,即二變調與角調也。變徵調與羽調五正聲內祗一聲乖應,然羽調猶能自立一調,變徵調則轉入宮調聲字。至角調變宮調,五聲之內二三聲乖應,與宮調聲字雷同,皆不能成一調也。唐志載四宮二十八調,率以弦音之分定為十二律呂之度,故有正宮大食、高大食之名。今即弦音、管音之和不和,以辨陽律、陰呂之分用、合用,乃知唐書之二十八調獨取弦音,不在管律。而古人所用三統,實取管音、弦音之相和者用之也。

是以弦音諸樂,其要有四:一,定弦音應某律呂之聲字,即得某弦之度分。一,弦音轉調不能依次遞遷,故以宮調為準,有幾弦不移,而他弦或緊一音,或慢半音,遂成一調,而各弦七聲之分因之而變。一,弦音諸調雖無二變,而定弦取音,必審二變之聲,必計二變之分,始能得其條貫,不然,宮調無所取準。一,弦音宮調,惟宮與商徵得與律呂相和為用,餘四調陰陽乖應,或淆入宮調聲字,不得自成一調。即此四者,條分縷析,則弦音旋宮轉調之法備矣。

樂之學既微,自古言者又歧說繁滋,莫衷一是。子長、孟堅時已異同,隋、唐登歌,雜蘇祗婆龜茲樂,以律呂文之,神瞽弗世,等於詩亡。宋人李照、和峴、範鎮、蔡元定之徒,稍有志於復古,然但資肊驗,或且飾以陰陽郛廓之說,明鄭世子載堉始以勾股譚律度。

帝本長疇人術,加之以密率,基之以實測,管音弦分千載之襲繆,至是乃定。明年書成,分三編:曰正律審音,發明黃鍾起數,及縱長、體積、面冪、周徑律呂損益之理,管弦律度旋宮之法;曰和聲定樂,明八音制器之要,詳考古今之同異;曰協均度曲,取波爾都哈兒國人徐日升及意大里亞國人德里格所講聲律節度,證以經史所載律呂宮調諸法,分配陰陽二均字譜,賜名曰律呂正義。蘭生、廷珍等皆賜及第,進官有差。

既又諭改訂中和樂章聲調,曰:「殿陛所奏中和樂章,皆沿明代,句有長短,體制類詞,曾因不雅,命大學士陳廷敬等改撰,章法皆以四字為句,而樂人未嫻聲調,仍以長短句湊拍歌之。今考舊調已得,宮商節奏甚為和平,必使歌章字句亦隨韻逗,則章明而宮聲諧,其著南書房翰林同大學士詳定以聞。」是年十一月冬至,躬祀圜丘,遂用新定樂律。

五十四年,改造圜丘壇,金鐘玉磬,各十有六。五十五年,頒中和韶樂於直省文廟。初,樂章既改用「平」,而直省仍沿用「和」,至是從禮部請,始頒行焉。

世宗雍正二年,定耕耤三十六禾詞,耕耤筵宴樂制,進筵,丹陛樂奏雨暘時若之章,進酒,管弦樂奏五穀豐登之章,進饌,清樂奏家給人足之章,其辭皆大學士蔣廷錫撰。後又定祭時應宮、祭風伯廟、教坊司作樂,祭雷師、雲師廟,和聲署作樂,官民婚嫁,品官鼓樂人不得過十二,生、監、軍、民不得過八人,著為令。

高宗即位,銳意制作,莊親王允祿自聖祖時監修律算三書,至是仍典樂事。乾隆六年,殿陛奏中和韶樂,帝覺音律節奏與樂章不協,因命和親王弘晝同允祿奏試,允祿因言:「明代舊制,樂章以五、六、七字為句,而音律之節奏隨之,樂章音律俱八句,故長短相協。今殿陛樂若定以四字為句,則與壇廟無殊,惟樂章更定,大典攸關,謂宜會同大學士、禮部將樂章十二成詳議,令翰林改擬進覽。」尋大學士鄂爾泰等議:「樂章十二成內,惟淑平、順平二成每章八句,其十成樂章每章各十句,句四字,而按之音律,則每章八句,每句六、七、八字,以十句四字樂章,和以八句六、七、八字之音律,長短抑揚,宜不盡協。應將樂章字句,按音律之節奏以調和之,章酌從八句,句無拘四言。」奏可。

舊中和樂編鐘內倍夷則四鐘在黃鍾正律之前,帝疑其舛,兼詢編鐘倍律及設而不作之故於臣工,時張照以刑部侍郎副允祿管部,名知樂,奏言:「編鐘之制,以十六鐘為一架,陽律八為一懸,在上;陰律八為一懸,在下。陽自陽,陰自陰。律呂之法,必有倍、半,然後高低清濁具備,以成旋宮之用。故陽律有倍蕤賓、倍夷則、倍無射在黃鍾之前,有半黃鍾、半太簇、半姑洗在無射之後。陰律則有倍林鍾、倍南呂、倍應鍾在大呂之前,有半大呂、半夾鍾、半仲呂在應鍾之後。倍蕤賓以還,則聲過低而啞,半仲呂以還,則聲過高而促,故不用。編鐘無倍蕤賓、倍林鍾,亦無六半律,以編鐘具八,其音中和,已足於用。低不至倍蕤賓、倍林鍾,高不至六半律,其序以從低至高,濁至清,排列為次。倍夷則、倍無射當在黃鍾之前,倍南呂、倍應鍾當在大呂之前,與簫管之長短,琴弦之巨細為一例。排簫倍夷則、倍無射二管在黃鍾之前,倍南呂、倍應鍾二管在大呂之前。★A9之倍徵、倍羽二弦在宮弦之前,若琴弦簫管易位,則音不可諧,是以編鐘之次第同於弦管。」又奏:「編鐘一架,上八下八,上陽律,下陰呂。考擊之節,南郊、廟祀及臨朝大典,皆用黃鍾為宮,北郊、月壇,則用大呂為宮。用黃鍾為宮,則擊上鐘,用大呂為宮,則擊下鐘。臨朝以下鐘易置於上而擊之,非下八鐘不擊也。又八鐘原祗七音,姑洗為宮,黃鍾起調為工字,倍夷則、無射為變徵,太簇為變宮,三鐘不入調,是以不擊。工字調外,則惟二鐘不擊。如以太簇為宮,倍無射起調為尺字,則倍夷則、無射、太簇三鐘皆擊,而黃鍾為變宮,夷則為變徵,二鐘又當不擊矣。因相沿俱以黃鍾調為黃鍾宮,儒生不知音律,謂黃鍾為聲氣之元,萬物之母,郊廟、朝廷用之吉,否則兇。不知黃鍾為宮,其第一聲便是下羽,除變宮、變徵不入調,商、角、徵、羽必須迭用。若聲聲皆是黃鍾,晏子所謂琴瑟專一,誰能聽之。況大武之樂,即是無射為宮,載之國語。無射乃陽律之窮,而武王用之,則十二月各以其律為宮,無所不可,亦明矣。」上是之,命如故。

當是時,清興百餘年矣,古學萌芽,儒者毛奇齡、李恭、胡彥升、江永輩多著書言樂事,考證益邃密。帝亦慕簫韶九成之盛,剬詩緝頌,勇於改為,欲以文致太平。聖祖時雖編定樂書,大抵稽於音律,而樂章句逗無譜,不與音相應。有協律高萬霖者,耆年審音,改定宮譜,然祗壇廟之樂。朝會清歌,仍踵前繆。照遂請續纂律呂書,謂「前代墜典,宜見刊正」,許之。開館纂修,仍命允祿監其事。未幾,館臣上議:「壇廟樂章字譜,天壇、太廟、朝日壇俱黃鍾為宮,地壇、夕月壇大呂為宮,近於南齊祗用黃鍾之說,而兼清濁二均。及於大呂,雖義有可取,但編鐘器內必有設而不作者,同於隋以前啞鐘之誚。我皇上制作定世,繼述休明,允宜博考詳稽,以襄盛典。夫言禮樂必宗成周,顧周代遙邈,文不足徵,所可考者,莫如周禮。而周禮所載圜鐘為宮祭天、函鐘為宮祭地、黃鍾為宮祭宗廟之說,圜鐘、函鐘不知何律。鄭康成以圜鐘為夾鍾,函鐘為林鍾,祭地用林鍾,義則善矣。然林鍾何以又稱函鐘,則亦無所據也。惟準六樂次第論之,有函鐘而無林鍾,則知函鐘即林鍾,然六樂又有夾鍾無圜鐘,其以圜鐘為夾鍾,謂夾鍾生於房、心之間,房、心大辰,天帝之明堂,則用甘公、石申戰代星家之言,以解七百年前周公之制度,誠非篤詁。李光地謂祭天以黃鍾為宮,祭宗廟以圜鐘為宮,圜黃互錯,諸儒相承而不知改。揆以春禘之文,則夾鍾之月也,雖若近理,然亦出於肊見。周禮本言祭天以圜鐘為宮,其下即云黃鍾為角,一章之樂,斷無黃鍾既為宮,而又為角之理。六樂次第,清濁各一均,黃鍾與大呂配祀天神,太簇與應鍾配祭地祗,姑洗與南呂配祀四望,蕤賓與函鐘配祭山川,夷則與仲呂配享姜嫄,無射與夾鍾配享先祖,以律之次第分神之尊卑。顧律呂同用,而清濁之間,有同均者,有不同均者,見諸實用,難於施行。是以歷代皆欲仰法周制,而苦無所憑。惟唐貞觀時祖孝孫定為祭圜丘以黃鍾為宮,方澤以林鍾為宮,宗廟以太簇為宮,朝賀宴饗則隨月用呂為宮,最為通論。蓋黃鍾子位,天之統也。乾位在亥,亥前為子,十二辰之始。黃鍾下生林鍾,林鐘未位,地之統也。坤位在申,陽順陰逆,申前為未。自子至午七律,而天之道備,自未至丑七律,而地之道備。故黃鍾屬天,林鍾屬地,林鍾上生太簇,太簇寅位,人之統也。故以祀宗廟,先儒所謂萬物本乎天,人本乎祖之義也。光地亦稱祖孝孫特有遠識,而歷代用樂,此最近古。臣等愚見,謂宜遵聖祖律呂正義所定旋宮轉調之法,將地壇樂章改林鍾為宮,太廟樂章改太簇為宮,社稷壇亦地也,亦宜改用林鍾為宮。月生於西,酉,西方正位也。又秋分夕月,建酉之月也。夕月壇宜改用南呂為宮,朝日壇若以日東月西、日卯月酉論,應用夾鍾為宮,但夾鍾陰而日陽,衷以人心屬日之義,宜改用太簇為宮。其朝會宴享,並應依唐祖孝孫之說,各以其月之律為宮。先農壇,農事也,宜以姑洗為宮。歷代帝王廟、孔子廟祭以春秋,春夾鍾、秋南呂為宮,太歲壇宜以歲始之律太簇為宮。」奏上,而皇太后、皇后升座、還宮樂章律呂未定,因命禮臣集議。允祿議曰:「皇太后、皇后樂章應用律呂,博考前典,並無明文。惟十二律呂皆生於黃鍾,故黃鍾為聲氣之元,但既專用於南郊以尊上帝,自不便擬用。且律協於乾,呂協於坤,坤元允宜用呂。大呂為黃鍾之呂,擬皇太后樂以大呂為宮。禮記:天子日也,日月東西相從而不已,天道也。酉為月之正位,援後月之義,擬皇后樂以南呂為宮。」履親王允祹議曰:「館臣擬皇太后樂以大呂為宮,皇后樂以南呂為宮,臣愚以為大呂、南呂並是陰呂,皇上曾有『凡慶賀大典,皇太后宮應用陽律』之旨,舊制一切大典,俱以黃鍾為宮,請仍循舊制。皇上冬至、元旦、萬壽三大節,皇太后、皇后三大節,並以黃鍾為宮。」帝以「大呂者,黃鍾之呂也。既用黃鍾尊上帝,林鍾尊后土,太簇尊宗廟,而議皇太后樂用大呂,大呂之序,乃在南呂后,皇后樂已用南呂,是先於皇太后也。又方澤壇用蕤賓之呂,林鍾為宮,而社稷亦宜有別」。因命重議。於是館臣請定皇太后樂用南呂為宮,社稷壇祭以春秋二仲月上戊,宜以夾鍾南呂為宮。從之。七年,允祿等又奏:「太皇太后升座、還宮用中和韶樂,行禮用丹陛樂,與皇帝同,而皇太后、皇后俱用丹陛樂。考諸掌儀司,自來升座、還宮並用中和韶樂,緣陳廷敬撰擬樂章之時,以皇太后、皇后不敢同於太皇太后,便以丹陛名之。請仍復舊,各為樂章。」尋定皇太后御慈寧宮升座中和韶樂奏豫平;皇帝率諸王群臣行禮丹陛大樂奏益平,還宮中和韶樂奏履平,皇后率皇貴妃、貴妃、妃、嬪及公主、福晉、命婦至宮行禮並同。皇帝三大節臨軒、還宮、御內殿升座中和韶樂奏元平,皇后率皇貴妃、貴妃、妃、嬪行禮丹陛大樂奏雍平,降座中和韶樂奏和平,皇后三大節升座中和韶樂奏淑平,行禮丹陛大樂奏正平,降座中和韶樂奏順平。皇帝筵宴、進茶、賜茶丹陛清樂奏海宇升平日,進酒、賜酒奏玉殿雲開,進饌、賜食中和清樂奏萬象清寧。皇太后三大節升座、還宮行禮與慶賀同,筵宴進茶、進酒、進饌所奏歌詞與皇帝同。

時山東道監察御史徐以升奏言:「古有雩祭之典,所以為百穀祈膏雨也。其制,則為壇於南郊之旁。我朝禮制具備,惟雩祭未有壇壝,乞敕下禮臣博求典故,詳考制度,仿古龍見而雩之禮,擇地立壇。」帝下其章,大學士鄂爾泰等議曰:「孟夏之月,蒼龍宿見東方,為百穀祈膏雨,故龍見而雩。晉永和中,依郊壇制為雩壇,祈上帝百闢,旱則祈雨。唐時雩祀於南郊,後行雩禮於圜丘。歷代京師孟夏後旱雩之禮,皆七日一祈,唐制斟酌最善,臣等酌議宜仿其制。古大雩用舞童二佾,衣玄衣,各執羽翳,歌云漢之詩。今皇上仿雲漢體御制詩歌八章,聖念懇誠,宸章剴切,應用舞童十六人,玄衣,八列,執羽翳,終獻樂止,贊者贊:『舞童歌詩。』歌畢,乃望燎。令掌儀司選聲音清亮者充之,羽翳依周禮皇舞之式,禮儀與孟夏常雩同。上帝、社稷、宗廟、太歲壇俱舊有樂章,惟神祗壇闕,應敕律呂館撰進。」乃定雩祀天神從圜丘,以黃鍾為宮;地祇從方澤,以林鍾為宮。樂用七成,迎神奏祈豐,奠帛奏華豐,初獻奏安豐,亞獻奏興豐,終獻奏儀豐,徹饌奏和豐,送神奏錫豐。是年始專設樂部,凡太常寺、神樂觀所司祭祀之樂,和聲署、掌儀司所司朝會宴饗之樂,鑾儀衛所司鹵簿諸樂,均隸焉。以禮部內務府大臣及各部院大臣諳曉音律者總理之,設署正、署丞、侍從、待詔、供奉、供用官、鼓手、樂工,總曰署吏,而以所司樂器別其目。鐘曰司鐘,磬曰司磬,琴、瑟、笙、簫亦如之。又禁道士充太常寺樂員。初,明樂舞生多選道童,世祖定都,沿而用之,羽流慢褻,識者慨焉,至是其弊始革。

既又從館臣言,定耕耤之樂。耕耤前期進種,導迎樂前導,至日,和聲署率屬鵠立採棚南,採棚之制,後二十三年裁。歌禾辭者十四人,司鑼、司鼓、司版、司笛、笙、簫者各六人,搴採旗者五十人。祭畢,行耕耤禮。禮成,導迎樂作,駕涖齋宮內門,樂止,中和韶樂作。皇帝御後殿,樂止,報終畝,中和韶樂作。皇帝御齋宮,升座,樂止,群臣慶賀行禮,丹陛大樂作。進茶、賜茶中和韶樂作。皇帝乘輦出宮,和聲署鹵簿大樂並作。筵宴、進茶、賜茶改雨暘時若為喜春光。進酒、賜酒改五穀豐登為雲和迭奏。進饌、賜饌改家給人足為風和日麗,升座、還宮樂章與三月常朝同。群臣行禮丹陛樂章與元旦同。又定祀先蠶樂章器用方響十有六,雲鑼、瑟、杖鼓、拍版各二,琴四,簫、笛、笙各六,建鼓一。皇后採桑歌器用金鼓、拍版二,簫、笛、笙六。遣官致祭樂章與群祀同。

又定賜衍聖公宴樂章奏洙泗發源長。正一真人宴奏上清碧落。文進士宴奏啟天門。武進士奏和氣洽。鄉飲酒禮歌鹿鳴、四牡、皇皇者華三章,笙禦制補南陔、白華、華黍三章,閒歌魚麗、南有嘉魚、南山有臺三章,笙禦制補由庚、崇丘、由儀三章,合樂周南關雎、葛覃、卷耳三章,召南鵲巢、採蘩、採蘋三章。

八年九月,高宗東巡狩至盛京,儀仗具,馬上鼓吹導引,翼日設丹陛大樂於兩樂亭,禮部設龍亭,置慶賀表,用導迎樂。上御崇政殿,升座中和韶樂奏元平,諸王大臣行禮、宣表丹陛大樂奏慶平,朝鮮陪臣朝賀丹陛大樂奏治平,頒詔、賜茶中和韶樂奏和平。是日崇政殿筵宴所奏中和丹陛清樂與太和殿筵宴同。改瑪克式舞為慶隆之舞,又增世德之舞。旋定樂舞內大小馬護為揚烈舞,舞人所騎竹馬為禺馬,馬護為面具。大臣起舞上壽為喜起舞。歌章者曰司章,騎竹馬曰司舞,搊琵琶曰司琵琶,彈弦子曰司三弦,彈箏曰司箏,劃節曰司節,拍版曰司拍,拍掌曰司抃。

九年,親幸翰林院,詔樂部設樂,升座奏隆平,掌院大學士率百官行禮奏慶平,進茶、賜茶奏文物京華盛,進御筵宴奏玉署延英、進酒、賜酒奏延閣雲濃,百官謝恩奏慶平,還官奏顯平。

是年裁太常寺司樂人六,增設天神地祇壇樂器,諭禮臣,除夕保和殿筵宴蒙古王等,先進蒙古樂曲,次慶隆舞,元旦太和殿筵宴王大臣,互易用之,著為令。

帝自御宇,樂制屢易,因革損益,悉出睿裁,群臣希旨,務為補苴,非有張乾龜、萬寶常之識也。帝思隆巍煥,遂特詔釐定朝會宴饗諸樂章,自七年定郊廟祭祀諸樂章,至十一年始成。朝會,皇帝元旦中和樂,升座元平,還宮和平。冬至中和樂,升座遂平,還宮允平。萬壽中和樂,升座乾平,還宮泰平。上元中和樂,升座怡平,還宮升平,常朝中和樂,升座隆平,還宮顯平。內廷行禮丹陛樂雝平,諸王百官行禮丹陛樂慶平,外籓丹陛樂治平。皇太后三大節中和樂,升座豫平,還宮履平,丹陛樂益平。皇后三大節中和樂,升座淑平,還宮順平,丹陛樂正平。郊廟圜丘迎神始平,奠玉帛景平,進俎咸平,初獻壽平,亞獻嘉平,終獻永平,徹饌熙平,送神清平,望燎太平。方澤迎神中平,奠玉帛廣平,進俎含平,初獻大平,亞獻安平,終獻時平,徹饌貞平,送神、望瘞寧平。祈穀迎神祈平,奠玉帛綏平,進俎萬平,初獻寶平,亞獻穰平,終獻瑞平,徹饌渥平,送神滋平,望燎穀平。雩祭迎神靄平,奠玉帛云平,進俎需平,初獻霖平,亞獻露平,終獻霑平,徹饌靈平,送神霮平,望燎霈平。太廟時饗,迎神貽平,奠帛、初獻敉平,亞獻敷平,終獻紹平,徹饌光平,送神、還宮、望燎乂平。祫祭迎神開平,奠帛、初獻肅平,亞獻協平,終獻裕平,徹饌諴平,送神、還宮、望燎成平。社稷迎神登平,奠帛、初獻茂平,亞獻育平,終獻敦平,徹饌博平,送神樂平,望瘞徵平。社稷壇祈雨報祀迎神延豐,奠帛、初獻介豐,亞獻滋豐,終獻霈豐,徹饌綏豐,送神貽豐,望瘞博豐。朝日迎神寅曦,奠玉帛朝曦,初獻清曦,亞獻咸曦,終獻純曦,徹饌延曦,送神歸曦。夕月迎神迎光,奠帛、初獻升光,亞獻瑤光,終獻瑞光,徹饌涵光,送神保光。歷代帝王迎神肇平,奠帛、初獻興平,亞獻崇平,終獻恬平,徹饌淳平,送神、望燎匡平。先師迎神昭平,奠帛、初獻宣平,亞獻秩平,終獻敘平,徹饌懿平,送神德平。先農迎神永豐,奠帛、初獻時豐,亞獻咸豐,終獻大豐,徹饌屢豐,送神報豐,望瘞慶豐。先蠶迎神庥平,奠帛、初獻承平,亞獻均平,終獻齊平,徹饌柔平,送神洽平,天神地祇迎神祈豐,奠帛、初獻華豐,亞獻興豐,終獻儀豐,徹饌和豐,送神錫豐。太歲迎神保平,奠帛、初獻定平,亞獻嘏平,終獻富平,徹饌盈平,送神豐平。太歲壇祈雨、報祀迎神需豐,奠帛、初獻宜豐,亞獻晉豐,終獻協豐,徹饌應豐,送神洽豐。皇帝祭壇廟還宮導迎樂祐平,慶典導迎樂禧平。其詞皆命儒臣重撰,天子親裁之,分刌而節比,合則仍其故,不合則易其辭、更其調,視舊章增損有加,而律呂正義後編亦於是年書成。曰祭祀樂,曰朝會樂,曰宴饗樂,曰導迎樂,曰行幸樂。更參稽前代因革損益之異,為樂器考、樂制考、樂章考、度量權衡考。復推闡聖祖所以審音定樂制器協均者,為樂問三十五篇。大抵詳於宮譜,而於律呂之原,管音弦度之分合,一遵聖祖,無所創也。帝自製序以冠之。

十三年二月,東巡山左,祭岱嶽,大學士等上言:「泰山向不用樂,考周禮大司樂『奏蕤賓、歌函鐘、舞大夏以祭山川』。今特舉盛典,秩於岱宗,請遵古用樂,樂章飭部撰擬。」於是詔樂章六奏,用「豐」。十月,張廣泗、訥親討金川久無功,上特命大學士傅恆為經略,出師,行授鉞禮。是日御太和殿,陳法駕鹵簿樂器如常儀。升座,中和韶樂奏隆平,經略跪受敕印行禮,丹陛大樂奏慶平,經略隨奉敕印大臣由東階下,樂止,上還宮中和韶樂奏顯平。禡日建八旗大纛於堂子內門外之南,軍士執螺角列竣,上輿出宮,樂陳而不作。至紅椿,聲螺角,上入自街門降輿,螺止。行禮,復聲螺。纛前行禮畢,出至紅椿,螺止,導迎樂作。駕至東長安門外,御武帳,升座,賜經略酒,從征官皆櫜鞬,辭,啟行。還宮,導迎樂作。明年凱旋,賜宴豐澤園,駕御帳殿,進茶、賜茶奏景運乾坤泰,掌儀捧臺盞卮壺奏聖德誕敷,進饌奏日耀中天。其後兆惠平定西域,阿桂再克金川,凱旋皆用此禮,改景運乾坤泰為聖武光昭世,聖德誕敷為禹甸遐通,日耀中天為聖治遐昌。改德隆舞為德勝之舞。中和樂章皆增武成慶語,以誇膚績。上又自作凱歌三十章,增鐃歌十六章,郊勞時奏之。聲容矞厖,邁隆古矣。

二十六年,江西撫臣奏得古鐘十一,圖以進,上示廷臣,定為鎛鐘,命依鍾律尺度,鑄造十二律鎛鐘,備中和特懸。既成,帝自制銘,允祿等又請造特磬十二虡,與鎛鐘配,鑿和闐玉為之。三十三年,定關帝廟樂章,迎神、送神三獻章各一。四十五年八月,高宗七旬萬壽,增喜起舞樂九章。自是凡有大慶典,則增制樂章以為常……

五十二年,命皇子永瑢與鄒奕孝、莊存與重定詩經樂譜,糾鄭世子載堉之謬。五十八年,又命樂部肄演安南、廓爾喀、粗緬甸、細緬甸諸樂,故清之樂,終帝之世凡數變。

仁宗嘉慶元年,增制太上皇帝三大節御殿中和韶樂二章,丹陛大樂一章,宮中行禮丹陛大樂一章,筵宴中和清樂一章、丹陛清樂二章、慶隆舞樂九章,又增皇極殿千叟宴太上皇帝御殿中和韶樂二章。自後臨雍,幸翰林院、文昌廟祀,社稷壇祈晴及萬壽節,皆增制樂章。八年,命筵宴停止安南樂。十四年元旦,太和殿筵宴,命演朝鮮、回部、金川、緬甸樂舞等項,遇慶隆舞、喜起舞,即以承應。又增隊舞大臣四人,歲如故事。

宣、文之世,垂衣而治,宮懸徒為具文,雖有增創,無關宏典。德宗光緒末年,仿歐羅巴、美利堅諸邦制軍樂,又升先師大祀,增佾舞之數,及更定國歌,制作屢載不定,以訖於遜國,多未施行。


\end{pinyinscope}