\article{列傳九}

\begin{pinyinscope}
阿哈出子釋加奴猛哥不花釋加奴子李滿住李滿住孫完者禿猛哥不花子

撒滿哈失里猛哥帖木兒猛哥帖木兒弟凡察子董山董山子脫羅脫羅子脫

原保凡察子不花禿

王杲王兀堂

,始永樂元年十一月辛醜?起兵。建州設?阿哈出,遼東邊外女真頭人。太祖以建州,初為指揮使者,阿哈出也,明賜姓名李誠善,所屬授千百戶、鎮撫,賜誥印、冠服、鈔幣有差。三年十月,阿哈出朝於明。六年三月,忽的河、法胡河、卓兒河、海剌河諸女真頭人,哈喇等授千百戶。七年七月,阿哈出朝於明。?哈喇等朝於明,以其地屬建州阿哈出子二:釋加奴、猛哥不花。八年,成祖親征出塞,釋加奴率所屬從戰有功。八月乙卯,以釋加奴為都指揮僉事,賜姓名李顯忠,所屬昝卜賜姓名張志義,阿剌失賜姓名李指揮使。初?從善,可捏賜姓名郭以誠,皆為正千戶。九年九月,釋加奴舉猛哥不花為毛憐,以頭人巴兒遜為指揮使;至是從釋加奴請,以命其弟。十年,釋加奴?,永樂三年設毛憐等歲祲乏食,遼東都指揮巫凱以聞,成祖命發粟賑之。

處之?猛哥帖木兒者,亦女真頭人,其弟曰凡察,與阿哈出父子並起,明析置建州左,以為指揮使。十一年十月,與釋加奴、猛哥不花同朝於明。十四年,釋加奴、猛哥不花朝於明,為所屬乞官。十五年二月,猛哥不花朝於明。十二月,釋加奴上言:「顏春頭人月兒速哥率其孥來歸,請屬於建州。」釋加奴、猛哥不花、猛哥帖木兒屢為所屬乞官。十八年閏正月,成祖命無功不得乞官,賜敕戒諭之。十九年十月,猛哥不花朝於明。二十年正月,成祖親征出塞,猛哥不花率子弟及所屬從,賜弓矢、裘、馬。二十二年三月,成祖復親征出塞,猛哥不花使所屬指揮僉事王吉從,成祖嘉賚之。七月,成祖崩。

宣德元年正月,猛哥不花、猛哥帖木兒朝於明。是月壬子,進猛哥帖木兒為都督僉事。釋加奴已前卒,三月辛丑,以其子李滿住為都督僉事。九月丁巳,進猛哥不花為中軍都督。二年二月,猛哥不花使貢馬,旋卒。四月,命餼其孥。?同知,仍掌毛憐猛哥不花子二:撒滿哈失里、官保奴。撒滿哈失里蒙其祖阿哈出賜姓為李氏,四年三月壬子,明以為都督僉事。五年三月,官保奴朝於明。四月,李滿住上言求市於朝鮮,朝鮮不納,宣宗敕諭聽於遼東境上通市。六年正月,釋加奴妻唐氏朝於明。二月,撒滿哈失里朝於明。七年二月,猛哥帖木兒使其弟凡察朝於明;三月壬戌,明以為都指揮僉事。

八年二月庚戌,進猛哥帖木兒為右都督,凡察都指揮使。六月,撒滿哈失里朝於明。都督猛哥帖木兒及其?,殺左?頭人弗答哈等掠建州?是年,七姓野人木答忽等糾阿速江等子阿古,凡察告難於明。會明使都指揮裴俊如斡木河,中途遇寇,凡察以所屬赴援,有功。事;敕諭木答忽等還所掠人、馬、貲財,且赦其罪。?九年二月癸酉,進凡察都督僉事,掌是月,撒滿哈失里母金阿納失里朝於明。

宣德十年正月,宣宗崩。是月,李滿住、撒滿哈失里上言忽剌溫境內野人那列禿等掠所屬那顏寨,敕諭那列禿等還所掠人、馬、貲財,並以責弗答哈等。四月,撒滿哈失里朝於明。正統元年閏六月,李滿住使其子古納哈等朝於明,還遼東逃人,明英宗嘉其效誠,賜採緞、冠服;並上章言忽剌溫野人相侵,乞徙居遼陽婆?江,英宗命遼東總兵官巫凱計議安置,毋弛邊備,毋失夷情。二年正月,凡察使所屬指揮同知李伍哈朝於明,上章言:「居鄰朝鮮,為所困;欲還建州,又為所阻:乞朝命。」英宗賜敕撫諭。五月,撒滿哈失里朝於明,自陳原留京師自?。

事,賜敕遣之。是時,李滿住掌建?前,撒滿哈失裡已進都督同知,英宗命仍掌毛憐都督猛哥帖木兒死七姓?,與撒滿哈失裡並奉職貢惟謹;而故建州左?,凡察掌建州左?州指揮使。?野人之難,子阿古殉焉,諸子董山、綽顏依凡察以居。是年十一月,以董山為本三年正月,凡察朝於明。是月壬子,英宗賜以敕曰:「往者猛哥帖木兒死七姓野人之難,失二印無故事。?其印,宣德間,別鑄印畀凡察。董山上言舊印故在,而凡察復請留新印,一印自此始。六月,李滿住使所屬指揮?敕至,爾等協同署事,遣使上舊印。」凡察、董山爭趙歹因哈上章,言:「自徙居婆?江,屢為朝鮮侵掠。今復徙居?突山東南渾河上,為朝廷印為指揮阿里所匿,請別鑄印畀撒滿哈失里。」?守邊圉,罔敢或違。」別疏又言:「毛憐英宗不許,命撒滿哈失裏奏事附李滿住以達。

四年四月,李滿住上言:「都督凡察、指揮童倉為朝鮮所誘,叛去。」童倉即董山,譯音異也。英宗敕朝鮮國王李祹問狀,祹疏自明非誘。英宗命凡察、童倉即居鏡城,復敕祹韃靼相侵盜,敕遼東總兵曹義備邊。九月,朝鮮?撫諭之。五年四月,英宗以李滿住與福餘國王李祹上言凡察、童倉復逃還建州。總兵曹義亦疏陳:「凡察等去鏡城,率叛軍馬哈剌等四十家至蘇子河,乏食。」英宗敕義使編置三土河及婆?江迤西冬古河兩界間,仍依李滿住以居,發粟賑之;貰逃軍馬哈剌等,命還伍。復諭祹使歸其種人留朝鮮境者。是時,凡察,其徙居鏡城復還。六年正月戊午,進董山為都督僉事。?以都督、董山以指揮同領建州左

二月,朝鮮國王李祹上言:「凡察舊居鏡城阿木河,其兄猛哥帖木兒,臣祖授以萬戶,創公廨,與婢僕、衣糧、鞍馬,臣父又授以上將軍。及死七姓野人之難,其子阿古殉焉,屋宇、貲產焚掠殆盡。臣撫恤凡察,如先臣撫恤其兄。近歲徙居東良,後乃潛逃,與李滿住同處。此時臣不及知,安有追殺?或有留者,非懷土不去,則同類開諭而還,非臣阻之也。李滿住昔居婆?江,在臣國邊境。鹽米醯醬隨其所索,時時給與。後引忽剌溫劫掠臣邊不已。今凡察與同惡,謀與忽剌溫等來侵。請飭凡察等遄返舊居,庶小國邊民獲免寇盜。」英宗敕祹謹為備。會凡察上言不敢為非,敕遼東總兵曹義遣使諭之,並廉其情偽。

?印數年而不決。七年二月甲辰,英宗用總兵官曹義議,析置建州右?凡察、董山爭,敕分領所屬,守法安?,凡察以新印掌右?,凡察、董山皆進都督同知,董山以舊印掌左。董山、凡察及李滿住各為所屬乞官,皆許之。自是,歲有干請。久次,乞進?業,毋事爭印畀之?秩;物故,乞襲職,以為常。撒滿哈失里朝於明。三月丁丑,進右都督,別鑄毛憐指揮僉事吳良齎敕往勘。凡察所索童阿?,使錦衣?。五月,英宗以凡察等屢言朝鮮留其部哈裡等,居朝鮮久,受職事,守丘墓,皆自陳不原還,而以十人還李滿住。八年十月,李滿住使報兀良哈將入寇,英宗命僉都御史王?勒兵為備。九年正月,李滿住等上言指揮郎克苦等還自朝鮮,乞賑,英宗命發粟賑之。十二月,董山、凡察朝於明。十年正月,撒滿哈失里朝於明。十一年二月,以董山弟綽顏為副千戶。十二年正月,進李滿住為都督同知。六月,李滿住、董山、凡察等使為備。十三年正月,復敕戒李滿住等?以聞瓦剌將寇邊,敕建州三毋為北虜誘。十二月,董山、凡察朝於明。十四年,凡察妻?兒真索朝於明,進皇太后塔納亦屢犯邊。景泰中,王?巡撫遼東,使招?珠二顆,賚以紵絲表裏。既而額森入寇,建州三諭,復叩關。

天順二年正月,李滿住朝於明。二月,進董山右都督。時董山陰附朝鮮,朝鮮授以中樞密使。巡撫遼東都御史程信詗得其制書以聞,英宗使詰朝鮮及董山,皆心習服,貢馬謝。五夜至義州江,殺並江收禾民,掠男婦、牛馬。」下?年十二月,朝鮮國王李上言:「建州都督郎卜兒哈,致寇乃自取,置勿問。八年春正月,英宗崩?兵部議,以為朝鮮嘗誘殺毛憐。

成化元年正月,董山朝於明,自陳防邊有勞,乞進秩。憲宗不許,賜以採緞。十月,整飭邊備。左都御史李秉上言:「建州、毛憐、海西諸部落入貢,邊臣驗方物,貂必純黑,屢犯邊。貢使至,使者不宜過持擇,召邊釁。?馬必肥大,否則拒不納。今諸部落結福餘三犯邊,官兵擊破之。」十二月,復入犯,?」憲宗命從之。二年十一月,秉上言:「毛憐諸總兵武安侯鄭宏戰敗。三年正月,秉上言:「董山歸所掠邊人,請贖俘。」憲宗敕?董山,因戒復入鴉鶻?署都督僉事武忠將命撫諭。是月,海西、建州諸?,旋使錦衣?責建州、毛憐諸關,都指揮鄧佐御諸雙嶺,中伏死,副總兵施英不能救。三月,復入連山關,掠開原、撫順,窺鐵嶺、寧遠、廣寧。及忠至,董山等受撫。四月,偕李古納哈等朝於明,憲宗使集闕下,宣詔赦其罪,董山等頓首聽命。

五月己丑,復以左都御史李秉提督軍務,武靖伯趙輔佩靖虜將軍印,充總兵官,發兵討建州,而董山等留京師,會賜宴,其從者語嫚,奪庖人銅牌,事聞,有詔切責;既而,予馬值、賚採幣如故事。董山、李古納哈乞蟒衣、玉帶、金頂帽、銀酒器,憲宗命增賜衣、帽,人一具。董山又言指揮可昆等五人有勞,乞賜,憲宗命賜衣,人一襲。董山等辭歸,鴻臚寺通事署丞王忠奏:「董山等罵坐不敬,貪求無厭,揚言歸且復叛,請遣官防送。」憲宗命禮部遣行人護行,復賜敕戒諭。董山等既行,憲宗復用禮部主事高岡議,命趙輔縶董山塞上。輔留董山等廣寧,令遣使戒所屬毋更盜邊。七月庚申,輔召董山等聽宣敕,未畢,董山等,殺董山等二十六人。憲宗命發兵益秉、輔東征,敕安撫?為嫚語,袖出刃刺譯者,吏士格,示專討建州。九月,分道出師:左軍渡渾河,越石門,至分水嶺;右軍度?毛憐、海西諸鴉鶻關,逾鳳凰城、摩天嶺,至婆?江;中軍下撫順,經薄刀山,過五嶺,渡蘇子河,至虎城。攻破張打必納、戴咬納、朗家、嘹哈諸寨,四戰皆捷。十月,師還。秉上疏請增兵戍遼陽,於鳳凰山、鴉鶻關、撫順、奉集、通遠諸路度地築城堡,選將吏習邊事者鎮開原,憲宗悉從之。

四年正月,朝鮮國王李上言,遣中樞府知事康純等將兵征建州,渡鴨綠、潑?二江,破兀獮府諸寨,擒李滿住及其子古納哈等,多所俘馘,使獻俘。

,猛哥帖木兒領之,死,而?,傳其子釋加奴及孫李滿住。析左?自阿哈出始領建州,移凡察領之。其入邊為亂,董山為之渠。明既殺董山?弟凡察代,既復傳其子董山;析右,朝鮮亦破李滿住,其子古納哈同死,他子都喜亦的哈,後不著。凡察正統後不復見,當已前死。其子不花禿不與董山之亂,獨全。他子阿哈答嘗朝於明,爭賜幣不及例。五年六月,都指揮佟那和劄等上章,為董山子脫羅等、李古納哈子完者禿乞官。兵部請進止,?建州左憲宗命授脫羅都指揮同知、完者禿都指揮僉事。自是,凡從董山為亂者,其子姓降一等,仍襲職。

頭人沙加保等三百餘人朝於明,憲宗敕示威德,俾復奉朝貢。居?六年正月,建州三數年,太監汪直擅政,欲以邊功自重,巡撫遼東右副都御史陳鉞阿直意,十三年十二月,上為邊患,請聲罪致討。十四年六月,命兵部侍郎馬文升及鉞會議招撫,文升上?章言建州三掌印都指揮完?掌印都指揮脫羅、卜花禿等一百九十五人,建州?言:「建州左、右二者禿等二十七人,先後應命。」宣敕撫慰,遣還。卜花禿即不花禿,凡察子也,九年十二月、十一年正月,再入朝,至是同受招撫。

尋復命直詣遼東處置邊務,直至邊,鉞復請用兵。十五年十月,命直監督軍務,撫寧,並敕朝鮮國王李發兵夾擊。十一?侯硃永佩靖虜將印充總兵官,鉞參贊軍務,討建州三月,永等分道出撫順關,建州人拒守,縱擊破之,有所俘馘。師還,永等受上賞。十六年六月,建州復寇邊。巡按遼東御史強珍疏論鉞等啟釁冒功,下吏議。汪直憾珍,劾珍欺罔,逮奉朝貢如故。?治,謫戍。鉞尋罷去。十八年,直亦得罪,建州三

弘治初,脫羅、完者禿皆進都督。孝宗之世,脫羅三朝,完者禿五朝,明賜完者禿大帽、金帶。正德元年,脫羅卒,以其子脫原保襲都督僉事。二年四月,卜花禿卒,賜祭。武宗之世,脫原保三朝。

都督阿剌哈、真哥、騰力革?都督章成、古魯哥,右?都督方巾,左?嘉靖間,建州輩,見於明實錄,皆不知其世。蓋自李滿住死,復傳其孫完者禿。阿哈出之後,可紀者四世,傳子撒滿答失里,後不著。董山死,傳其子脫羅及孫脫原保。?。其別子猛哥不花領毛憐猛哥帖木兒之後,可紀者三世。其弟凡察傳子不花禿,後不著。迨嘉靖季年,王杲強,而阿哈出、猛哥帖木兒之族不復見。

王杲,不知其種族。生而黠慧,通番、漢語言文字,尤精日者術。嘉靖間,為建州右都指揮使,屢盜邊。三十六年十月,窺撫順,殺守備彭文洙,遂益恣掠東州、會安、一堵?墻諸堡無虛歲。四十一年五月,副總兵黑春帥師深入,王杲誘致春,設伏媳婦山,生得春,磔之,遂犯遼陽,劫孤山,略撫順、湯站,前後殺指揮王國柱、陳其孚、戴冕、王重爵、楊五美,把總溫欒、於欒、王守廉、田耕、劉一鳴等,凡數十輩。當事議絕貢市,發兵剿,尋又請貸,杲不為悛。隆慶末,建州哈哈納等三十人款塞請降,邊吏納焉。王杲走開原索之,勿予,乃勒千餘騎犯清河。游擊將軍曹簠伏道左,突起,斬五級,王杲遁走。

故事,當開市,守備坐聽事,諸部酋長以次序立堂上,奉土產,乃驗馬;馬即羸且跛,並予善值,饜其欲乃已。王杲尤桀驁,攫酒飲,至醉,使酒箕踞罵坐。六年,守備賈汝翼初上,為亢厲,抑諸酋長立階下,諸酋長爭非故事,盡階進一等。汝翼怒,抵幾叱之,視戲下箠不下者十餘人,驗馬必肥壯。王杲鞅鞅引去,椎牛約諸部,殺掠塞上。是時,哈達王臺方強,諸部奉約束,邊將檄使諭王杲。王杲訟言汝翼摧抑狀,巡撫遼東都御史張學顏以聞,下兵部議,令遼東鎮撫宣諭,示以恩威。於是王臺以千騎入建州寨,令王杲歸所掠人馬,盟於撫順關下而罷。學顏復以聞,賚王臺銀幣。

萬歷二年七月,建州奈兒禿等四人款寨請降,來力紅追亡至塞上,守備裴承祖勿予,追者縱騎掠行夜者五人以去。承祖檄召來力紅令還所掠,亦勿予。是時王杲方入貢,馬二百匹、方物三十馱,休傳舍。承祖度王杲必不能棄輜重而修怨於我,乃率三百騎走來力紅寨,。王杲曰:「將?諸部圍之,未敢動。王杲聞耗驚,馳歸,與來力紅入謁承祖,而諸部圍益軍幸毋畏。倉卒聞將軍至,皆匍匐原望見。」承祖知其詐,呼左右急兵之,擊殺數十人,諸,殺傷相當。來力紅執承祖及把總劉承奕、百戶劉仲文,殺之。於是學顏奏絕王杲?部皆前貢市,邊將復檄王臺使捕王杲及來力紅。王臺送王杲所掠塞上士卒,及其種人殺漢官者。

坐困,遂糾土默特、泰寧諸部,圖大舉犯遼、沈。總兵李成梁屯?王杲以貢市絕,部沈陽,分部諸將:楊騰駐鄧良屯,王維屏駐馬根單,曹簠馳大沖挑戰。王杲以諸部三千騎入五味子沖,明軍四面起,諸部兵悉走保王杲寨。王杲寨阻險,城堅塹深,謂明軍不能攻。成梁計諸部方聚處,可坐縛。十月,勒諸軍具?石、火器疾走圍王杲寨,斧其柵數重。王杲拒守,成梁益揮諸將冒矢石陷堅先登。王杲以三百人登臺射明軍,明軍縱火,屋廬、芻茭悉焚,?蔽天,諸部大潰。明軍縱擊,得一千一百四級。往時剖承祖腹及殺承奕者皆就馘,王杲遁走。明軍車騎六萬,殺掠人畜殆盡。犯邊,復為明軍所圍。王杲以蟒褂、紅甲授所親阿哈?三年二月,王杲復出,謀集餘速把亥。明軍購?納,陽為王杲突圍走,明軍追之。王杲以故得脫,走重古路,將往依泰寧王杲急,王杲不敢北走,假道於王臺。邊吏檄捕送。七月,王臺率子虎兒罕赤縛王杲以獻,檻車致闕下,磔於市。王杲嘗以日者術自推出亡不即死,竟不驗。妻孥二十七人為王臺所得,其子阿臺脫去。阿臺妻,清景祖女孫也。

王臺卒,阿臺思報怨,因誘葉赫楊吉砮等侵虎兒罕赤。總督吳兌遣守備霍九皋諭阿臺,不聽。李成梁率師御之曹子穀、大梨樹佃,大破之,斬一千五百六十三級。四年春正月,阿臺復盜邊,自靜遠堡九臺入,既又自榆林堡入至渾河,既又自長勇堡入薄渾河東岸,又糾頭人阿海居莽子寨,兩寨相與為?土蠻謀分掠廣寧、開原、遼河。阿臺居古勒寨,其黨毛憐犄角。成梁使裨將胡鸞備河東,孫守廉備河西,親帥師自撫順王剛臺出寨,攻古勒寨,寨陡峻,三面壁立,壕塹甚設。成梁麾諸軍火攻兩晝夜,射阿臺,殪。別將秦得倚已先破莽子寨,殺阿海,斬二千二百二十二級。景祖、顯祖皆及於難。語詳太祖紀。

同時又有王兀堂,亦不知其種族,所居寨距靉陽二百五十里,靉陽故通市。王兀堂初起,奉約束惟謹。萬歷三年,李成梁策徙孤山、險山諸堡,拓境數百里,斷諸部窺塞道。王杲既擒,張學顏行邊,王兀堂率諸部酋環跪馬前,謂徙堡塞道,不便行獵,請得納質子,通市易鹽、布。學顏以請,神宗許之。開原、撫順、清河、靉陽、寬奠通布市自此始。

當是時,東方諸部落,自撫順、開原而北屬海西,王臺制之;自清河而南抵鴨綠江屬建州,王兀堂制之:頗守法。已,漸竊掠東州、會安堡。七年七月,開市寬奠,參將徐國輔縱其弟若僕減直強鬻參,毆種人以回易至者幾斃,諸部皆忿,數掠寬奠、永奠、新奠諸堡。他酋佟馬兒等牧松子嶺,闌入林剛穀。巡撫都御史周詠等劾國輔,罷之,諭王兀堂戢諸部。八年三月,王兀堂及他酋趙鎖羅骨等,以六百騎犯靉陽及黃關嶺,指揮王宗義戰死。四月,又以千騎自永奠堡入,成梁帥師擊敗之,斬七百五十級,俘一百六十人。十一月,復自寬奠堡入,副總兵姚大節帥師擊敗之,斬六十七級,俘十一人。王兀堂自是遂不振,不復通於明。

則有納答哈、納木章,?以都督奉朝貢者,建州?當隆慶之世,下逮萬歷初,建州諸則有八當哈、來留住、松塔;而王杲自指揮使遷何?則有大疼克、八汗馬、哈塔臺,右?左秩,不可考見,王兀堂?不著其官,然皆強盛為大酋。自王杲就擒後五年而王兀堂敗,又後三年而阿臺死,太祖兵起。

,始自阿哈出。枝幹互生,左右析置,自永樂至嘉靖,一百五十餘?論曰:建州之為年,而阿哈出之世絕。王杲乘之起,父子弄兵十餘年乃滅。其在於清,猶爽鳩、季荝之於齊,所謂因國是也。或謂猛哥帖木兒名近肇祖諱,子若孫亦相同。然清先代遘亂,幼子範察得脫,數傳至肇祖,始克復仇,而猛哥帖木兒乃被戕於野人,安所謂復仇?若以範察當凡察,凡察又猛哥帖木兒親弟也,不得為數傳之祖。清自述其宗系,而明乃得之於簡書。春秋之義,名從主人,非得當時紀載如元秘史者,固未可以臆斷也。隆慶、萬歷間,建州諸部長未有名近興祖諱者。太祖兵起,明人所論述但及景、顯二祖,亦未有謂為董山裔者。信以傳信,疑事可考見者著於篇,以阿哈出、王杲為之綱,而其子弟?以傳疑,今取太祖未起兵前建州三及同時並起者附焉。


\end{pinyinscope}