\article{列傳一百二十六}

\begin{pinyinscope}
恆文郭一裕蔣洲楊灝高恆子高樸

王亶望勒爾謹陳輝祖鄭源鸘國泰郝碩良卿方世俊

錢度覺羅伍拉納浦霖

恆文,烏佳氏,滿洲正黃旗人。雍正初,以諸生授筆帖式,四遷兵科給事中。外授甘肅平慶道,再遷貴州布政使。乾隆初,方用兵金川,恆文奏言:「兵貴神速。臣官甘肅平慶道時,見提督以下諸營,或三之一,或四之一,擇勇健者,名為援剿兵將,備預定旗幟器械,及獎賚諸項亦預存。貴州乃無此例。本年四川調兵二千,遲至六日方得起程。請仿甘肅例預為計,提督駐安順,設重兵,請於府庫貯銀五千待用。」既又疏上行軍諸節目。上嘉其能治事,移直隸。十六年,擢湖北巡撫。疏請採漢銅廣鼓鑄,請增築武昌近城石堤,請停估變省城道倉空廒、備貯協濟鄰省米石,均得旨允行。十八年,署湖廣總督,移山西巡撫。

二十一年,擢雲貴總督。二十二年三月,疏劾貴州糧道沈遷婪索屬吏,鞫實論斬。恆文與雲南巡撫郭一裕議制金爐上貢,恆文令屬吏市金,減其值,吏民怨咨。一裕乃疏劾恆文貪污敗檢,列款以上。上命刑部尚書劉統勛會貴州巡撫定長即訊,得恆文令屬吏市金減金值,及巡察營伍縱僕婪索諸事,逮送京師。上責恆文;「為大臣,以進獻為名,私飽己橐,簠簋不飭,負恩罪大。」遣待衛三泰、扎拉豐阿乘傳就恆文所至,宣諭賜自盡。

郭一裕,湖北漢陽人。雍正初,入貲為知縣,除江南清河知縣。稍遷山西太原知府。乾隆中,累擢雲南巡撫。恆文對簿,具言貢金爐議發自一裕。統勛等察知一裕亦令屬吏市金,見恆文以減值斂怨,乃先發為掩覆計。事聞,上謂:「一裕本庸鄙,前為山東巡撫,嘗請進萬金上供。在官惟以殖產營運為事,但尚不至如恆文之狼藉。」命奪職,發軍臺效力。手詔謂:「恆文及一裕罪輕重一歸允當,毋謂一裕以漢吏劾滿洲終兩敗也。」一裕呈部請輸金贖罪,會蔣洲、楊灝皆以婪索屬吏坐誅,洲獄具,得同官朋比狀。上因謂:「恆文事發自一裕,尚彼勝於此。」特許其納贖。居數年,予三品銜,授河南按察使。以老罷。卒。

蔣洲,江南常熟人,大學士廷錫子。自主事累擢至山西布政使。二十二年,就遷巡撫,旋移山東,以塔永寧代。塔永寧劾洲貪縱,虧庫帑鉅萬。將行,令冀寧道楊龍文、太原知府七賚札諸屬吏納賕彌所虧。統勛自雲南還,上命馳往會塔永寧按治。解洲任,逮送山西嚴鞫,得實,誅洲,並及龍文、七賚論絞候。諸屬吏虧帑,文職知州硃廷揚等、武職守備武璉等,皆論罪如律。陜西巡撫明德,以前官山西嘗取洲及諸屬吏賕,亦論絞候。上命發甘肅交黃廷桂聽差遣。

楊灝,直隸曲陽人。乾隆中,官湖南布政使。時以湖南倉穀濟江南當糴補,灝發穀值百取一二,得金三千有奇。巡撫陳宏謀疏劾,讞實,坐斬。二十二年,秋讞,巡撫蔣炳以灝限內完贓,擬入緩決,上怒,命誅灝,奪炳官,逮京師,論罪坐斬。上以炳意在沽譽,尚未嘗受賄,改戍軍臺。按察使夔舒亦坐是奪職。

高恆,字立齋,滿洲鑲黃旗人,大學士高斌子也。乾隆初,以廕生授戶部主事,再遷郎中。出監山海關、淮安、張家口榷稅,署長蘆鹽政、天津總兵。二十二年,授兩淮鹽政。江蘇巡撫陳宏謀疏言:「海州產鹽盛,請令河東買運配引赴陜西引地行銷。淮北鹽賤,並令淮南商買運適中之地,作常平倉鹽備缺額補配。」命高恆會兩江總督尹繼善覆議,尋疏陳:「海洲產鹽盛衰,視天時晴雨,難定成數。距陜西三千餘里,黃河逆流而上,斷難輓運。自海州出場,經淮、徐、海各屬,皆淮北食鹽口岸;徐州以上,又系長蘆引地。恐沿途挾私,淮南額引多,鹽場廣,有盈無絀。即淮北鹽價稍賤,加以腳費折耗亦相等。若令淮南銷淮北餘鹽,尤非商情所便。縱發官帑與之收買,亦難強其領運。」疏入,上從之。湖廣總督李侍堯疏言湖北鹽驟貴,請飭淮商減價。命高恆赴湖北會議。定湖北鹽價,視淮商成本每包以二錢三分一釐為制。二十九年,授上駟院卿,仍領兩淮鹽政。三十年,以從兄高晉為兩江總督,當回避,召署戶部侍郎。疏陳整頓綱課,定分季運清獎勵之制,命以告後政普福。尋授總管內務府大臣。三十二年,署吏部侍郎。是時上屢南巡,兩淮鹽商迎蹕,治行宮揚州,上臨幸,輒留數日乃去,費不貲,頻歲上貢稍華侈。

高恆為鹽政,陳請預提綱引歲二十萬至四十萬,得旨允行。復令諸商每引輸銀三兩為公使錢,因以自私,事皆未報部。三十三年,兩淮鹽政尤拔世發其弊,上奪高恆官,命江蘇巡撫彰寶會尤拔世按治。諸鹽商具言頻歲上貢及備南巡差共用銀四百六十七萬餘,諸鹽政雖在官久,尚無寄商生息事。上責其未詳盡,下刑部鞫實,高恆嘗受鹽商金,坐誅。普福及鹽運使盧見曾等罪有差。

子高樸,初授武備院員外郎。累遷給事中,巡山東漕政。三十七年,超擢都察院左副都御史。值月食,救護未至,上諭謂:「高樸年少奮勉,是以加恩擢用,非他人比。乃在朕前有意見長,退後輒圖安逸,豈足副朕造就裁成之意?」吏議奪職,命寬之。遷兵部右侍郎。上錄諸直省道府姓名,密記治行優絀,謂之道府記載,太監高雲從偶洩於外廷。左都御史觀保,侍郎蔣賜棨、吳壇、倪承寬嘗因侍班私論其事,高樸聞,具疏劾,上怒,下刑部鞫治。尋命誅雲從,貸觀保等,不竟其事。詔謂:「雲從以賤役無忌憚,豈可不亟為整飭以肅紀綱?但不屑因此興大獄,故不復窮治。諸大臣豈無見聞,獨高樸為之陳奏,內省應自慚。若因此圖傾高樸,則是自取其死。高樸若沾沾自喜,不知謹懍,轉致妄為,則高雲從即其前車,朕亦不能曲貸也。」四十一年,命往葉爾羌辦事。距葉爾羌四百餘里,有密爾岱山,產玉,舊封禁。高樸疏請開採,歲一次。四十三年,阿奇木伯克色提巴勒底訴高樸役回民三千採玉,婪索金寶,並盜鬻官玉。烏什辦事大臣永貴以聞,上命奪官嚴鞫,籍其家,得寄還金玉;永貴又言葉爾羌存銀一萬六千餘、金五百餘。高樸坐誅。

方上誅高恆,大學士傅恆從容言乞推慧賢皇貴妃恩貸其死,上曰:「如皇后兄弟犯法,當奈何?」傅恆戰慄不敢言。至是,諭曰:「高樸貪婪無忌,罔顧法紀,較其父高恆尤甚,不能念為慧賢皇貴妃侄而稍矜宥也。」

王亶望,山西臨汾人,江蘇巡撫師子。自舉人捐納知縣,發甘肅,知山丹、皋蘭諸縣。選授雲南武定知府,引見,命仍往甘肅待缺,除寧夏知府。累遷浙江布政使,暫署巡撫。乾隆三十八年,上幸天津,亶望貢方物,範金為如意,飾以珠,上拒弗納。三十九年,移甘肅布政使。甘肅舊例,令民輸豆麥,予國子監生,得應試入官,謂之「監糧」,上令罷之。既,復令肅州、安西收捐如舊例。亶望至,申總督勒爾謹,以內地倉儲未實為辭,為疏請諸州縣皆得收捐;既,又請於勒爾謹,令民改輸銀。歲虛報旱災,妄言以粟治賑,而私其銀,自總督以下皆有分,亶望多取焉。議初行,方半載,亶望疏報收捐一萬九千名,得豆麥八十二萬。上謂:「甘肅民貧地瘠,安得有二萬人捐監?又安得有如許餘糧?今半年已得八十二萬,年復一年,經久陳紅,又將安用?即雲每歲借給民間,何如留於閭閻,聽其自為流轉?」因發「四不可解」詰勒爾謹,勒爾謹飾辭具覆。上諭曰:「爾等既身任其事,勉力妥為之可也。」

四十二年,擢浙江巡撫。四十五年,上南巡,亶望治供張甚侈。上謂:「省方問俗,非為游觀計。今乃添建屋宇,點綴鐙彩,華縟繁費,朕實所不取。」戒毋更如是。亶望旋居母喪,疏請治喪百日後,留塘工自效,上許之。浙江巡撫李質穎入覲,奏陳海塘事,因及亶望意見不相合,遂言亶望不遣妻拏還里行喪。上降旨責其忘親越禮,奪官,仍留塘工自效。

四十六年,命大學士阿桂如浙江勘工。阿桂疏發杭嘉湖道王燧貪縱、故嘉興知府陳虞盛浮冒狀,上諭曰:「朕上年南巡,入浙江境,即見其侈靡,詰亶望,言虞盛所為。今燧等借大差為名,貪縱浮冒,必亶望為之庇護。」命逮燧嚴鞫。會河州回蘇四十三為亂,勒爾謹師屢敗,亦被逮。大學士阿桂出視師,未即至,命尚書和珅先焉,和珅疏言入境即遇雨,阿桂報師行亦屢言雨。上因疑甘肅頻歲報旱不實,諭阿桂及總督李侍堯令具實以聞。阿桂、侍堯疏發亶望等令監糧改輸銀及虛銷賑粟自私諸狀,上怒甚,遣侍郎楊魁如浙江會巡撫陳輝祖召亶望嚴鞫,籍其家,得金銀逾百萬。上幸熱河,逮亶望、勒爾謹及甘肅布政使王廷贊赴行在,令諸大臣會鞫。亶望具服發議監糧改輸銀,令蘭州知府蔣全迪示意諸州縣偽報旱災,迫所轄道府具結申轉;在官尚奢侈,皋蘭知縣程棟為支應,諸州縣食鬼賂率以千萬計。獄定,上命斬亶望,賜勒爾謹自裁,廷贊論絞,並命即蘭州斬全迪;遂令阿桂按治諸州縣,冒賑至二萬以上皆死,於是坐斬者棟等二十二人,餘譴黜有差。上謂:「此二十二人之死,皆亶望導之使陷於法,與亶望殺之何異?」令奪亶望子裘等官,發伊犁,幼子逮下刑部獄,年至十二,即次第遣發,逃者斬。陜甘總督李侍堯續發得賕諸吏,又誅閔鵷元等十一人,罪董熙等六人。

五十九年,上將歸政,國史館進師傳。上覽其治績,乃赦亶望子還,幼者罷勿遣,謂「勿令師絕嗣也」。

勒爾謹,宜特墨氏,滿洲鑲白旗人。乾隆初,以繙譯進士授刑部主事,遷員外郎。外授直隸天津道。累遷陜甘總督。四十二年,河州回黃國其、王伏林為亂,馳往捕治,誅國其、伏林及其徒四百餘人。四十六年,循化回蘇四十三復起,勒爾謹令蘭州知府楊士璣、河州協副將新柱率二百人往捕,為所戕,遂破河州。勒爾謹赴援,聞賊將自小道徑攻蘭州,引還城守。上責勒爾謹觀望失機,奪官;下刑部論斬,上命改監候,卒坐亶望獄死。陳輝祖又以籍亶望家匿金玉器,譴誅。

輝祖,湖南祁陽人,兩廣總督大受子也。以廕生授戶部員外郎,遷郎中。外授河南陳州知府。累遷閩浙總督,兼領浙江巡撫。亶望獄起,輝祖弟嚴祖為甘肅知縣,獄辭連染。上以輝祖當知狀,詰之,不敢言,詔嚴切,乃具陳平日實有所聞,懼嚴祖且得罪,隱忍未聞上,因請罪,降三品頂戴留任。時安徽巡撫閔鶚元亦坐其弟鵷元,與輝祖同譴。既,布政使盛柱疏言檢校亶望家入官物與原冊有異同,命大學士阿桂按治,具得輝祖隱匿私易狀,論斬。上曰:「輝祖罪固無可逭,然與亶望較,終不同。傳云:『與其有聚斂之臣,寧有盜臣。』輝祖盜臣耳。亦命改監候。」四十七年,浙江巡撫福崧奏桐鄉民因徵漕聚眾閧縣庭,輝祖寬其罪,次年乃復閧。閩浙總督富勒渾奏兩省諸州縣虧倉穀,福建水師提督黃仕簡奏臺灣民互斗,於是上罪輝祖牟利營私,兩省庶政皆廢弛貽誤,罪無異亶望,賜自裁。五十三年,又以湖北吏治闒茸,弊始輝祖為巡撫時,戍其子伊犁。

乾隆季年,諸貪吏首亶望,次則鄭源鸘。

源鸘,直隸豐潤人。以貢生授戶部主事,累遷湖南布政使。仁宗既誅和珅,有言源鸘貪黷狀,下巡撫姜晟按治。源鸘具服收發庫項,加扣平餘,數逾八萬;署內眷屬幾三百人,自蓄優伶,服官奢侈。上宣示源鸘罪狀,因言:「諸直省大吏宴會酒食,率以囑首縣,首縣復斂於諸州縣。率皆朘小民之脂膏,供大吏之娛樂,展轉苛派,受害仍在吾民。通諭諸直省,令悛改積習。」尋命斬源鸘。

國泰,富察氏,滿洲鑲白旗人,四川總督文綬子也。國泰初授刑部主事,再遷郎中。外擢山東按察使,遷布政使。乾隆三十八年,文綬官陜甘總督,奉命按前四川總督阿爾泰縱子明德布婪索屬吏,徇不以實陳,戍伊犁。國泰具疏謝,請從父戍所贖父罪。上諭曰:「汝無罪,何必惶懼?」四十二年,遷巡撫。

國泰紈褲子,早貴,遇屬吏不以禮,小不當意,輒呵斥。布政使於易簡事之諂,至長跪白事。易簡,江蘇金壇人,大學士敏中弟也。大學士阿桂等以國泰乖張,請改京朝官。四十六年,上為召易簡詣京師問狀,易簡為國泰力辨。上降旨戒國泰馭屬吏當寬嚴得中,令警惕改悔。會文綬復官四川總督,以啯匪為亂,再戍伊犁,國泰未具疏謝。居月餘,疏謝賜鹿肉,上詰責。國泰請納養廉為父贖,並乞治罪,上寬之。

四十七年,御史錢灃劾國泰及易簡貪縱營私,徵賂諸州縣,諸州縣倉庫皆虧缺。上命尚書和珅、左都御史劉墉按治,並令灃與俱。和珅故袒國泰;墉持正,以國泰虐其鄉,右灃。驗歷城庫銀銀色不一,得借市充庫狀。語互詳灃傳。國泰具服婪索諸屬吏,數輒至千萬。易簡諂國泰,上詰不敢以實對。獄定,皆論斬,上命改監候,逮系刑部獄。巡撫明興疏言通察諸州縣倉庫,虧二百萬有奇,皆國泰、易簡在官時事。上命即獄中詰國泰等,國泰等言因王倫亂,諸州縣以公使錢佐軍興,乃虧及倉庫。上以「王倫亂起滅不過一月,即謂軍興事急,何多至二百萬?即有之,當具疏以實聞。國泰、易簡罔上行私,視諸屬吏虧帑恝置不問,罪與王亶望等均」。命即獄中賜自裁。

郝碩,漢軍鑲黃旗人。父郝玉麟,官兩江總督。郝碩襲騎都尉世職,授戶部員外郎,直軍機處,遷郎中。外授山東登萊青道,三遷江西巡撫。將朝京師,以行李不具,徵屬吏納賕。四十九年,兩江總督薩載論劾,逮京師鞫實。上謂:「郝碩罪同國泰,國泰小有才,地方事尚知料理。郝碩嘗朝行在,問以地方事,不知所對。不意復貪婪若是!且郝碩託辭求賂,正國泰事敗時,乃明知故蹈,無復忌憚。即視國泰例賜自裁。」因通諭諸直省督撫,當持名節,畏憲典,以國泰、郝碩為戒。

良卿,富察氏,滿洲正白旗人。乾隆七年進士,授戶部主事,遷郎中。外授直隸通永道,累遷貴州布政使。三十二年,命署巡撫。

師征緬甸,良卿董臺站。上諭良卿:「師行供頓有資民力者,覈實奏聞。」良卿疏言:「此項多鄉保措辦,銀數多寡參差,無從覈算。」上謂:「師行供頓有資民力,亦當官為檢覈。若以鄉保措辦遂置不問,民瘼何所仰賴?且吏役因以為奸,又何所不至耶?良卿以布政使署巡撫,何得諉為不知?」下吏議,當降調,命改奪官,仍留任。既,上發帑佐軍需,良卿請確查散給,上詰良卿:「既言無從覈算,何能確查散給?」命留供續發官軍。良卿又疏陳貴州兵極能走險耐瘴,請募五千人習槍砲、藤牌備徵發。上嘉其盡心,賜孔雀翎。尋移廣東,以募兵事未竟,仍留貴州。貴州產鉛,歲採運供鑄錢,以糧道主其事。三十四年,良卿疏劾威寧知州劉標運鉛不如額,並虧工本運值,奪標職,令良卿詳讞。良卿疏陳標虧項,並劾糧道永泰,請簡大臣會鞫,上為遣內閣學士富察善如貴州會良卿按治。永泰揭戶部陳標虧項由長官婪索,因及良卿及按察使高積貪黷狀,上解良卿職,復命刑部侍郎錢維城、湖廣總督吳達善即訊。故事,奏摺置黃木匣,外護以黃綾袱,至御前始啟。上發副將軍阿桂軍中奏,於袱內得普安民吳倎訴官吏、土目私派累民狀,命吳達善密勘;而劉標亦遣人詣戶部訴上官婪索,呈簿記,上申命吳達善嚴鞫。

吳達善先後疏言標積年虧帑至二十四萬有奇。良卿意在彌補掩覆,見事不可掩,乃以訪聞奏劾;及追繳銀六千有奇,令留抵私填公項,不入查封,始終隱飾。又及高積鬻儲庫水銀,良卿有袒庇狀。良卿長支養廉,為前布政使張逢堯及積署布政使時支放。普安州民吳國治訴知州陳昶籍軍興私派累民,良卿即令昶會鞫,不竟其事,乃致倎賄驛吏附奏事達御前。上乃責良卿負恩欺罔,罪不止於骫法婪贓,命即貴州省城處斬,銷旗籍,以其子富多、富永發伊犁,畀厄魯特為奴。積、逢堯、標皆坐譴。

方世俊,字毓川,安徽桐城人。乾隆四年進士,授戶部主事。累遷太僕寺少卿,外授陜西布政使。二十九年,擢貴州巡撫。三十二年,調湖南巡撫。劉標訐發上官婪索,言世俊得銀六千有奇,上命奪官,逮送貴州,其僕承世俊得銀千。獄成,械致刑部,論絞決,上命改監候。秋讞入情實,伏法。

錢度,字希裴,江南武進人。乾隆元年進士,授吏部主事,累遷廣西道監察御史。外授安徽徽州知府,累擢至方面。其為江安督糧道、河庫道,皆再任,歷十餘年。上嘉其久任奮勉。二十九年,授雲南布政使。三十三年,遷廣東巡撫。師方征緬甸,度主餽軍,命以巡撫銜領布政使。未歲,移廣西巡撫,乃之官,賀縣囚越獄,度請寬知縣鄭之翀罪。上命奪之翀職,責度寬縱。學政梅立本按試鬱林,索供應,民聚閧。上命度定學政供應夫船事例,度擬從寬備,失上指,仍左授雲南布政使。三十七年,監銅廠。宜良知縣硃一深揭戶部,告度貪婪,勒屬吏市金玉,上命刑部侍郎袁守侗如雲南會總督彰寶、巡撫李湖按治。貴州巡撫圖思德奏獲度僕持金玉諸器,自京師將往雲南,值銀五千以上;江西巡撫海明奏獲度僕攜銀二萬九千有奇,自雲南將往江南,並得度寄子酆書,令為衣復壁藏金,為永久計;兩江總督高晉籍度家,得窖藏銀二萬七千,又寄頓金二千。守侗等訊得度刻扣銅本平餘,及勒屬吏市金玉得值,具服,逮送京師。命軍機大臣會刑部覆讞,以度侵欺勒索贓私具實,罪當斬,命即行法。子酆亦論絞,上為改緩決。尋遇赦,仍不令應試出仕。嘉慶五年,弛其禁。

覺羅伍拉納,滿洲正黃旗人。初授戶部筆帖式,外除張家口理事同知,累遷福建布政使。林爽文之亂,伍拉納主餽軍,往來蚶江、廈門,事定,賜花翎,遷河南巡撫。乾隆五十四年,授閩浙總督。上以福建民情獷悍,戒伍拉納當與巡撫徐嗣曾商榷整飭。伍拉納督屬吏捕盜,先後所誅殺百數十人。以內地民多渡海至臺灣,疏請海口設官渡,便稽察。時定往臺灣者出蚶江,民舟或自廈門渡,亦令至蚶江報驗,疏請罷其例,俾得逕出廈門。言者以海中島嶼多,流民散處為盜藪,當毀其廬,徙其民,毋使滋蔓。下濱海諸直省議,伍拉納疏言:「福建海中諸島嶼,流民散處,凡已編甲輸糧者,當不在例中。」上命諸島嶼非例當封禁,皆任其居處。浙江嘉善縣民訴縣吏徵漕浮收,下伍拉納按治,論如律。

伍拉納治尚嚴,疏劾金門鎮總兵羅英笈巡洋兵船遇盜不以實報,英笈坐譴;又論邵武營守備餘朝武等侵餉,營吏黃國材等冒餉,黃巖右營守備葉起發屬兵遇盜不以實報,外委陳學明避盜偽為被創,營兵柯大斌誣告營官,皆傅重比。五十七年,同安民陳蘇老、晉江民陳滋等為亂,設靝雰會。「雰」字妄造,以代「天地」。伍拉納率按察使戚蓼生赴泉州捕得蘇老等,誅一百五十八人,戍六十九人。五十九年,義烏民何世來,宣平民王元、樓德新等為亂,立邪教。伍拉納率按察使錢受椿赴金華。浙江巡撫吉慶已捕誅世來、德新,伍拉納覆讞諸脅從,復誅鮑茂山、吳阿成等,還福建至浦城,捕得元,誅之。

六十年,臺灣盜陳周全為亂,陷彰化。伍拉納出駐泉州,發兵令署陸路提督烏蘭保、海壇鎮總兵特克什布赴剿,彰化民楊仲舍等擊破周全,亂已定。是歲,漳、泉被水,饑。伍拉納至,民閧集乞賑,未以聞。上促伍拉納赴臺灣,累詔詰責,伍拉納自泉州往。福州將軍魁倫疏言:「伍拉納性急,按察使錢受椿等迎合,治獄多未協。漳、泉被水,米值昂,民貧,巡撫浦霖等不為之所,多入海為盜。虎門近在省會,亦有盜舟出沒。」上為罷伍拉納、浦霖,命兩廣總督覺羅長麟署總督,魁倫署巡撫。

伍拉納至臺灣,劾鹿仔港巡檢硃繼功以喪去官,賊起,即攜眷內渡,請奪官戍新疆。上諭曰:「伍拉納為總督,臺灣賊起,陷城戕官,朕屢旨嚴飭始行,繼功丁憂巡檢,轉責其攜眷內渡,加以遠戍。伍拉納畏葸遷延,乃欲以此自掩,何其不知恥也!」伍拉納、浦霖貪縱、婪索諸屬吏,州縣倉庫多虧缺。伍拉納嘗疏陳清查諸州縣倉庫,虧穀六十四萬有奇、銀三十六萬有奇,限三年責諸主者償納。至是,魁倫疏論諸州縣倉庫虧缺,伍拉納所奏非實數。上命伍拉納、浦霖及布政使伊轍布、按察使錢受椿皆奪官,交長麟、魁倫按讞。

長麟、魁倫勘布政司庫吏周經侵庫帑八萬有奇,具獄辭以上。上疑長麟等意將歸獄於經,斥其徇隱。長麟等疏發伍拉納受鹽商賕十五萬,霖亦受二萬,別疏發受椿讞長秦械斗獄,獄斃至十人,得賕銷案。籍伍拉納家,得銀四十萬有奇、如意至一百餘柄,上比之元載胡椒八百斛;籍霖家,得窖藏金七百、銀二十八萬,田舍值六萬有奇,他服物稱是;逮京師,廷鞫服罪,命立斬。

伊轍布亦逮京師,道死。受椿監送還福建,夾二次,重笞四十,乃集在省諸官吏處斬;又以長麟主寬貸,奪官召還,以魁倫代之,遂興大獄,諸州縣虧帑一萬以上皆斬,誅李堂等十人,餘譴黜有差。

霖,浙江嘉善人。乾隆三十一年進士,授戶部主事,再遷郎中。外授湖北安襄鄖道。累遷福建巡撫,移湖南,復遷福建。及得罪,上謂:「伍拉納未嘗學問,或不知潔己奉公之義。霖以科目進,起自寒素,擢任封疆,乃貪黷無厭,罔顧廉恥,尚得謂有人心者乎?」霖及伍拉納、伊轍布、受椿諸子皆用王亶望例戍伊犁。嘉慶四年,赦還。

論曰:高宗譴諸貪吏,身大闢,家籍沒,僇及於子孫。凡所連染,窮治不稍貸,可謂嚴矣!乃營私骫法,前後相望,豈以執政者尚貪侈,源濁流不能清歟?抑以坐苞苴敗者,亦或論才宥罪,執法未嘗無撓歟?然觀其所誅殛,要可以鑒矣!


\end{pinyinscope}