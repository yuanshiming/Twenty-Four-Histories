\article{列傳一百三十}

\begin{pinyinscope}
書麟弟廣厚覺羅吉慶覺羅長麟費淳百齡伯麟

書麟,字紱齋,高佳氏,滿洲鑲黃旗人,大學士高晉子。初授鑾儀衛整儀尉,累遷冠軍使,擢西安副都統。乾隆三十八年,大軍征金川,命為領隊大臣,從參贊大臣豐升額,力戰輒先登,克堅碉數十,功最。金川平,加等議敘,圖形紫光閣。授廣西巡撫,以父憂去。起,署兵部侍郎。

四十九年,出為安徽巡撫,歲旱,請留漕糧五萬石、關稅銀三十五萬兩賑之。阜陽有荒地六千餘頃,疏請寬限清釐,民間交易用官弓丈量,以杜欺隱,期於漸復舊額。帝以書麟盡心民瘼,予優敘。黃、運兩河漫溢,帝因兩江總督李世傑未諳河工,命書麟佐之。與世傑及河督李奉翰議,漫口有四,惟司家莊、湯家莊兩處分溜,急興工堵築;又奏:「桃源境內河流因順黃壩生有淤灘,水勢紆折不暢。於玉皇閣下挑引河,俾黃流東注會清,以資宣洩。」

五十二年,擢兩江總督。書麟素行清謹,出巡屬邑,輕騎減從,民不擾累,特詔嘉之。和珅柄政,書麟與之忤。未幾,有高郵巡檢陳倚道揭報書吏假印重徵事,遣重臣鞫實,坐書麟瞻徇,下部嚴議;又失察句容書吏侵用錢糧,褫職,遣戍伊犁。尋起為山西巡撫。內閣學士尹壯圖論州縣虧空由於派累,疆臣中惟李世傑、書麟獨善其身,和珅尤忌之,命壯圖赴各省清查倉庫,自山西始,壯圖因獲譴。五十六年,仍授兩江總督。兩淮鹽政巴寧阿交結商人,坐書麟徇庇,復奪職,予三等侍衛,赴新疆效力。

嘉慶四年,和珅敗,召授吏部尚書,兼正紅旗漢軍都統,加太子少保。尋協辦大學士,授閩浙總督。弟廣興,以首發和珅奸擢官,既得官,多所彈擊,書麟不善所為,嘗於帝前言之。至是,廣興以掌四川軍需獲咎,書麟請嚴治,且自引罪,詔宥之。調雲貴,鞫前督富綱,得其貪婪狀,論如律;又按問云南巡撫江蘭諱災,得實,褫江蘭職。時惈夷不靖,疏陳江蘭所奏不實,辦理草率,帝嘉其公正。遂親赴黃草壩督兵分路進剿,擒賊首李文明等,遣降惈入箐招諭,曉以利害,夷眾五十二寨悔罪輸誠;以土司苛派擾夷,立牌申禁:優詔褒賚,加太子太保。

五年,調湖廣,督師剿襄陽青、藍、黃三號教匪。會長齡等已敗賊瓦房口,書麟以東川、保豐為糧運要路,親往截剿。帝念其年逾七旬,奔馳山谷間,賊情詭詐,戒毋冒險輕試。六年,由竹山、房縣進剿徐天德,擒斬甚眾。疏言:「剿賊之法,以固民心、培民氣為要。撫輯得宜,賊即是民;任其失所,民即是賊。」帝俞之。川匪茍文明等由陜西平利越老林圖竄房縣,偕長齡、明亮進擊,遇賊獅子崖,大敗之;復分兵伏佘家溝、高尖山,天德等來襲,卻之。疏請於襄陽添設提督,移協鎮於鄖陽、竹山二處。天德等屯聚茅倫山,令孫清元等分隊破之。因病乞解職,遣侍衛率御醫馳視。未幾,卒於軍,帝深惜之,贈太子太傅,封一等男爵,以子吉郎阿嗣,謚文勤。尋以倭什布治餉遲誤,詔斥書麟知而不舉,念其清廉公正,治軍成勞,奠醊恩禮仍有加焉。

弟廣厚,乾隆四十三年進士。由工部主事歷御史,出為江西吉南贛寧道,遷甘肅按察使。嘉慶初,偕總兵吉蘭泰擊教匪張映祥、楊天柱於鞏昌、秦州,進蹙諸白水江,殲焉。遷江西布政使,調甘肅。賊出沒於岷州、禮縣間,廣厚督兵由岷州遮羊鋪遏其沖,保完善之地,境內乂安。調廣東,坐與總督那彥成游宴,解職,予三等侍衛,為庫車辦事大臣,調哈喇沙爾。官至安徽、湖南巡撫。卒。

覺羅吉慶,隸正白旗。父萬福,騎都尉,官江寧將軍,兼散秩大臣。吉慶由官學生補內閣中書,遷侍讀,歷御史。乾隆五十年,嗣世職。擢鑲白旗蒙古副都統,累遷兵部侍郎。命赴山東、湖南、湖北、河南讞獄,均稱旨,調戶部。

五十六年,出為山東巡撫,歲祲,截留漕米三十萬石,撥豫、東軍船運米賑饑。調浙江,閩海漁船赴浙洋剽掠,吉慶於島嶴編保甲,禁米出洋,嚴緝代賣盜贓;兼署提督,獲海盜陳言等,及臨海邪匪李鶴皋,置之法。鹽政岳謙執拗病民,劾罷之,遂兼鹽政。

嘉慶元年,擢兩廣總督,劾水師提督路超吉不勝任,貶超吉秩。二年,廣西西隆亞稿寨苗匪句結貴州仲苗,竄踞八渡,率提督彭承堯進剿,克其要隘。黔苗潛渡百樂窺泗城,令副將德昌等分路攻撲,毀苗砦十有九;進攻亞稿,至戛雄遇賊,大敗之。永豐、百樂等苗目渡江降,給酒食,令回寨招撫。亞稿山路陡峻,選精卒由間道潛襲,克其巢,斬首千級,以功加太子太保,賜雙眼花翎。亞稿之捷,投誠者十餘寨,惟附近那地、小河、廣平、蒙裏等寨猶恃險抗拒,會雲南兵至,會剿,盡克之。賊首龍登連父子乞降,粵境悉平。六年,命協辦大學士、總督如故。

吉慶居官廉而察吏疏,博羅縣重犯越獄,司府徇隱;又通省贓罰銀按縣派徵,為臬司漏規。事並上聞,詔斥其因循。陳爛屐四者,於博羅山中糾眾為添弟會,知府伊秉綬請發兵往捕,吉慶為提督孫全謀所蔽,未許。七年,陳爛屐四果剽掠作亂,擾及數縣,遣師擒斬之。餘黨曾鬼六復勾結永安諸賊相繼起,吉慶馳往剿捕,請調江西兵二千為助。詔斥其張皇,始疑之。尋敗賊於義容墟,曾清浩率眾四千餘人繳械降。全謀擒賊渠薛文勝,暨匪眾四百餘,悉誅之。事聞,帝以吉慶奏報前後不符,措置失當,罷協辦大學士,留總督任,命那彥成往按。

吉慶復奏永安降匪多,請留兵防範,詔斥結局,解任聽勘。巡撫瑚圖禮素與有隙,既奉密諭詗察,遂疏劾其疲輭不職,那彥成猶未至,獨鞫之,據高坐,設囚具,隸卒故加訶辱。吉慶恚曰:「某雖不肖,曾備位政府,不可受辱傷國體!」因自戕。帝聞,命那彥成陳狀,尋以吉慶素廉潔,治匪有功,無故輕生,詔免追論。

子壽喜,仍襲世職,坐事黜,以弟常喜嗣。

覺羅長麟,字牧庵,隸正藍旗。乾隆四十年進士,授刑部主事。貌奇偉,明敏有口辯,居曹有聲。歷郎中,出為福建興泉永道,累遷江蘇布政使。五十一年,召授刑部侍郎。

五十二年,授山東巡撫,責所屬濬河道,修四十一州縣城工;捕鉅野、汶上劇盜田玉堂等,置之法:詔嘉獎。劾萊州知府徐大榕治平度州民羅有良獄,誤擬,大榕訴於京,刑部尚書胡季堂等往鞫,不直長麟。帝以防河有勞,特寬之。復以審擬濱州舉人薛對元罪失實,褫職,留修城工。未幾,授江蘇巡撫。嘗私行市井間訪察民隱,擒治強暴,禁革奢俗,清漕政,斥貪吏,為時所稱。

五十七年,調山西。入覲時,有市人董二誣告逆匪王倫潛匿山西某家,和申於宮門前言,務坐以逆黨。長麟至官,訪悉某實董仇家,故傾陷,慨然曰:「吾發垂白,奈何滅人族以媚權相?」終反坐董二,和珅大忤。

調浙江,擢兩廣總督,加太子少保。整頓水師,擒獲海盜。六十年,調署閩浙。會將軍魁倫劾總督伍拉納、巡撫浦霖貪縱,並閩省庫藏虧絀事,命長麟按治,未得實,詔切責,乃奏婪索納賄狀。伍拉納故和珅姻戚,帝疑長麟瞻徇,並斥其平日沽名取巧,奪職,予副都統銜,赴葉爾羌辦事。尋授庫爾喀拉烏蘇領隊大臣,調喀什噶爾參贊大臣。奏減回子王公年班進京行李,以恤驛站。罷回民土貢。有邊警,請調兵堵剿,詔以張皇斥之。

嘉慶四年,授雲貴總督,調閩浙。五年,調陜甘。時教匪未靖,勸民築堡團練,令川、陜、豫、楚交界處,一體仿行,募精壯難民入伍。督師敗伍金柱於唐家河,又擊於傅家鎮。將軍富成來援,戰歿。復偕固原提督慶成擊賊於沔陽乾溝河。六年,迭敗高天德、馬學禮於鐵爐川、舊州鋪、綱廠、武關,擒襄陽賊首馬應祥,詔嘉獎。尋以副將蕭福祿搜捕汧陽悄悄會匪,濫殺邀功,仁宗疑之,詗察得實,斥長麟徇庇,停其議敘。又以傅家鎮之戰,漫無籌措,致富成陣亡。七年,召回京,降署吏部侍郎,遷禮部尚書,兼都統。復命督兩廣,以母老留京。

八年,授兵部尚書,調刑部,兼管戶部三庫。十年,兼翰林院掌院學士,尋協辦大學士。十三年,命偕尚書戴衢亨察視南河。長麟至清江浦,聞安徽諸生包世臣習河事,親訪之,同視海口,實不高仰,用其說罷改道之議。與衢亨通籌河工,具得要領,帝嘉之。復偕衢亨清查兩淮鹽務,責鹽政每年雜費悉報部覈銷,以息浮議。

十五年,以目眚久在告,特詔解職。逾年,卒,謚文敏。

費淳,字筠浦,浙江錢塘人。乾隆二十八年進士,授刑部主事。歷郎中,充軍機章京。出為江蘇常州知府,父憂去。服闋,補山西太原,擢冀寧道。累遷雲南布政使,有惠政。以母老乞終養,喪除,起故官。六十年,擢安徽巡撫,調江蘇。嘉慶二年,疏言:「淮、徐、揚三府屬被水窪地,責州縣勸植蘆葦,以收地利。應納錢糧,即照蘆課改折徵輸。」詔議行。調福建,復還江蘇。四年,擢兩江總督。

淳歷官廉謹,為帝所重,兩淮鹽政徵瑞與淳為姻家,免其回避。時南河比歲漫溢,淳以江督事繁,自陳未諳河務,乞免兼管,允之。命淳與總河詳議河務工程,應行分辦事具聞,帝密詢漕督蔣兆奎等優劣,諭曰:「安民首在任賢,除弊必先去貪。汝操守雖優,察吏過寬。去一貪吏,萬姓蒙福;進一賢臣,一方受惠。其悉心訪聞,慎勿迎合朕意,顛倒是非。」淳具以實聞。有匿名訐告常州知府胡觀瀾者,下淳按治,疏糾觀瀾與江陰知縣楊世綬勒派累民,得實,請嚴譴。詔斥不先劾,以平日廉潔,覆奏無徇隱,寬之。尋劾鹽巡道彭翼蒙奢侈糜費,褫翼蒙職。復劾漕運總督富綱私受衛弁餽銀,時富綱已調雲貴總督,命吉慶嚴鞫,置諸法。漕運旗丁苦累,屢議加徵調劑,偕漕督鐵保疏陳:「原徵隨漕項下有款可撥,以裨運丁;又旗丁月米,令州縣改給折色,應領運費,責糧道放給,以免層層剝削。」如所請行。

五年,邵家壩河工合龍,加太子少保。六年,以足疾乞歸醫治,允之,命毋解職。尋稱足疾已瘳,若遵旨回籍,轉涉欺蒙,詔嘉其得大臣體,賜內府藥餌。七年,宿州土匪王潮名糾眾戕官,檄鎮將剿捕。事定,請於宿之南平集設撫民同知,裁寧國府同知,移駐其地,並調設營汛,從之。八年,召授兵部尚書。時河決河南衡家樓,橫溢張秋以南,由鹽河入海,有妨漕運,命淳往勘治,於張秋西岸加寬裹頭,東岸加高長堤,以防溜勢北掣,南口趁汶水北注之勢,引歸河身;北口自大溜迤北,分導餘流,以資挽運:並仿南河刷沙法,制混江龍鐵篦船以疏淤。明年,糧運過張秋無阻,降詔褒賚。調吏部尚書、協辦大學士。十一年,偕尚書長麟按問直隸籓司書吏侵冒錢糧獄,鞫實,論如律。

十二年,拜體仁閣大學士,管理工部,兼管戶部三庫。十四年,以庫銀被竊,鐫秩留任。已,復坐失察工部書吏冒領三庫銀,詔切責,削宮銜,左遷侍郎,調兵部。逾年,復授工部尚書。十六年,卒,復大學士,謚文恪,祀云南名宦。

百齡,字菊溪,張氏,漢軍正黃旗人。乾隆三十七年進士,選庶吉士,授編修。掌院阿桂重之,曰:「公輔器也!」督山西學政,改御史,歷奉天、順天府丞。百齡負才自守,不干進,邅回閒職十餘年。

仁宗親政後,始加拔擢。嘉慶五年,出為湖南按察使,調浙江,歷貴州、雲南布政使。八年,擢廣西巡撫。武緣縣有冤獄,諸生黃萬鏐等為知縣孫廷標誣擬大闢,百齡下車,劾廷標逮問,帝嘉之,賜花翎;洎定讞,特加太子少保。十年,調廣東。南海、番禺兩縣蠹役私設班館,羈留無辜,為民害,重懲之;劾罷縱容之知縣王軾、趙興武,嚴申禁令:詔予優敘。尋擢湖廣總督。兩湖多盜,下令擒捕,行以便宜,江、湖晏然。未幾,王軾訐百齡在粵用非刑斃命,逼勒供應,臨行用運夫二千餘名。總督那彥成疏劾,並及到湖北後,截留廣東會奏批摺。命吳熊光等按鞫,議褫職遣戍,帝原之,命效力實錄館。尋予六品頂戴,赴福建治糧餉,事竣,授汀漳龍道。擢湖南按察使,調江蘇,以病歸。病痊,授鴻臚寺卿,歷山東按察使,就擢巡撫。

十四年,擢兩廣總督。粵洋久不靖,巨寇張保挾眾數萬,勢甚張。百齡至,撤沿海商船,改鹽運由陸,禁銷贓、接濟水米諸弊。籌餉練水師,懲貪去懦,水師提督孫全謀失機,劾逮治罪。每一檄下,耳目震新。巡哨周嚴,遇盜輒擊之沉海,群魁奪氣,始有投誠意。張保妻鄭尤黠悍,遣硃爾賡額、溫承志往諭以利害,遂勸保降,要制府親臨乃聽命。百齡曰:「粵人苦盜久矣!不坦懷待之,海氛何由息?」遂單舸出虎門,從者十數人,保率艦數百,轟砲如雷,環船跪迓,立撫其眾,許奏乞貸死。旬日解散二萬餘人,繳砲船四百餘號,復令誘烏石二至雷州斬之,釋其餘黨,粵洋肅清。帝愈嘉異之,復太子少保,賜雙眼花翎,予輕車都尉世職。

十六年,再乞病,回京,授刑部尚書,改左都御史,兼都統。未幾,授兩江總督。時河決王家營,上游綿拐山、李家樓並漫溢,論者謂河患在雲梯關海口不暢,多主改由馬港新河入海。百齡親勘下游,疏言:「海口無高仰形跡,亦無攔門沙堤。其受病在上年挑河二段內積淤三千餘丈。又親至馬港口以下,見淤沙挑費更鉅,入海路窄。二者相較,仍以修濬正河為便。並請加挑灶工尾以下河身,兩岸接築新堤,於七套增建減水壩,修復王營減壩,重建磨盤埽。」詔如議。百齡年逾六旬始生子,值帝萬壽日,聞之,賜名扎拉芬以示寵異,勉其盡心治河。次年春,諸工先後竣,漕運渡黃較早,迭加優賚,賜其子六品廕生。洪湖連年水漲,五壩壞其四,詔責急修。百齡以禮壩之決,由於河督陳鳳翔急開遲閉,以致棘手,奏劾之。鳳翔被嚴譴,訴道請開禮壩時,百齡同批允;又訐淮揚道硃爾賡額為百齡所倚,司葦蕩營有弊。言官吳雲、馬履泰並論其舉劾失當,命松筠、初彭齡往按。帝意方鄉用,議上,專坐硃爾賡額罪,以塞眾謗。十八年,命協辦大學士,總督如故。

十九年,初彭齡奉命赴江蘇同查虧帑,議不合。彭齡為所掣,恚甚,遂劾百齡受鹽場稅關餽遺,按之未得實,彭齡坐誣被譴。會鹽運使廖寅捕逆犯劉第五,部鞫為偽。百齡亦坐失入,褫宮銜,罷協辦大學士。江南莠民散布逆詞,連及百齡,嚴詔責捕。二十年,獲首、從方榮升等百五十人,並抵法,復宮銜,封三等男爵,兼署安徽巡撫。是年冬,病甚,命松筠往代,卒於江寧。帝聞,悼惜,詔復協辦大學士,遣侍衛賜奠,許柩入城治喪。將遣皇子奠醊,既而以江北災民未能撫恤,停其奠醊,仍賜祭葬如例,謚文敏。子扎拉芬,襲男爵。

伯麟,字玉亭,瑚錫哈哩氏,滿洲正黃旗人。由繙譯舉人授兵部筆帖式,擢右春坊右贊善,累遷內閣學士。乾隆五十七年,授盛京兵部侍郎,尋授山西巡撫。

嘉慶九年,擢雲貴總督。十年,緬甸與暹羅屬夷戛於臘構釁,求助於孟連土司刀派功,往援遇害,失其印。伯麟以刀派功禍由自取,惟責暹羅繳所得印。十一年,緬甸請預期納貢。伯麟知其與暹羅構兵,為求助地,卻之。後緬甸為戛於臘所敗,果來乞援,伯麟拒勿應,戛於臘旋亦敗走。緬兵次車裏土司界,嚴兵守邊,移檄訓戒,緬兵遂退。迤南江外惈匪入邊劫掠,遣普洱鎮總兵那林泰剿平之。十三年,緬甸四大萬頭目來請十三板納地,伯麟責其冒昧,諭以十三板納為九龍江土司所轄,俱屬內地,毋生覬覦,詔嘉其得體。十四年,入覲,賜花翎。

十七年,騰越邊外野寨頭目拉幹出擾,遣兵擒之。緬寧、騰越要隘舊設土練一千六百名,久廢,規復其制,給曠土耕種。僧銅金從惈夷李文明為亂,已悔罪投誠,更姓名為張輔國,充南興土目;至是復勾結惈眾侵擾,伯麟赴緬寧督土司會剿。十八年正月,進逼南興,破其巢,輔國就戮,邊境肅清。增設騰越鎮馬鹿塘、大壩二汛。

二十二年,臨安邊外夷人高羅衣自稱窩泥王,偽署官職,糾眾萬餘,攻殺土目龍定國,擾瓦渣、溪處兩土司境,渡江窺伺內地,伯麟親往剿平之。議定善後條規,使各土司綏靖夷民,以安反側。敘功,加太子少保。尋命協辦大學士,仍留總督任。二十三年,羅衣從侄高老五竄藤條江外復為亂,擾及郡城。督師剿擒之,餘黨悉殲。增設臨安江內東、西兩路要隘塘汛官兵,以江外煙瘴最盛,降夷就撫,裁撤留防兵練。二十五年,召授兵部尚書,兼都統。復疏陳滇、黔邊務六事,如議行。

道光元年,拜體仁閣大學士,管理兵部。尋以年老休致,仍充實錄館總裁。三年,萬壽節,與十五老臣宴。逾年,卒,謚文慎。

伯麟任邊圻凡十六年,廉潔愛民,士林尤感戴之。還朝後,以旗人生計為憂,疏陳調劑事宜,深中利弊。論者謂有名臣風。

論曰:仁宗倚畀疆臣,膺重寄者,多參揆席。書麟、吉慶並勤勞軍事,而盡瘁辱身,有幸不幸焉。長麟、費淳先後治吳,一嚴一寬,才德互有優絀。百齡號能臣之冠,機牙鋒銳,凌轢一時,晚節乃招物議。如伯麟之安邊坐鎮,遺愛不湮,識量豈易及哉?


\end{pinyinscope}