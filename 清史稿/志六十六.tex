\article{志六十六}

\begin{pinyinscope}
禮十(賓禮)

籓國通禮山海諸國朝貢禮敕封籓服禮外國公使覲見禮

內外王公相見禮京官相見禮直省官相見禮士庶相見禮

四曰賓禮。清初籓服有二類,分隸理籓院、主客司。隸院者,蒙古喀爾喀,西藏、青海、廓爾喀是也;隸司者,曰朝鮮,曰越南,曰南掌,曰緬甸,曰蘇祿,曰荷蘭,曰暹羅,曰琉球。親疏略判,於禮同為屬也。西洋諸國,始亦屬於籓部,逮咸、同以降,歐風亞雨,咄咄逼人,覲聘往來,締結齊等,而於禮則又為敵。夫詩歌「有客」,傳載「交鄰」,無論屬國、與國,要之,來者皆賓也。我為主人,凡所以將事,皆賓禮也。茲編分著其儀節,而王公百官相見禮與士庶相見禮,亦附識焉。

籓國通禮清初,蒙古北部喀爾喀三汗同時納貢。朔漠蕩平,懷柔漸遠。北逾瀚海,西絕羌荒。青海厄魯特,西藏準噶爾,悉隸版圖。荷蘭亦受朝敕稱王,名列籓服。厥後至者彌眾,乃令各守疆圉、修職貢,設理籓院統之。

崇德間,定制,外籓諸部貝勒等有大勛績,封和碩親王,或多羅郡王,次多羅貝勒,遣使持信約往封。既入境,貝勒出迎五里外,跽俟制冊過,騎以從。抵府,設香案正中,使臣奉冊其上,退立左旁,貝勒一叩三跪。畢,興,復跪,使臣授冊。宣讀官宣畢,置原案,三叩,興。受冊如初禮。貝勒與使臣對行六叩禮。使臣坐左,貝勒坐右。事訖,躬送如前。凡有詔敕、賞賚至亦如之。

內外札薩克會盟,三年一舉。使臣齎制往,迎送禮同。自王以降,歲時朝貢者,分年番代,列班末行禮。坐次視內親王、貝勒、貝子、公降一等,宴賚有差。

康熙五十九年,定朝覲年例。蒙古二十四部為兩班,喀爾喀札薩克等為四班。雍正四年,帝念四十九旗王公臺吉遠至勤勞,詔改三班,二歲一朝。咸豐八年,以蒙古汗王等遠道輸將,諭令停止年班。御前行走者,番上如故。

其貢獻儀文,按季各旗遣一人來將事,年時貢馬匹羊酒,交理籓院轉納禮部。朝貢賞賚諸典,柔遠清吏司掌之。

山海諸國朝貢禮凡諸國以時修貢,遣陪臣來朝,延納燕賜,典之禮部。將入境,所在長吏給郵符,遴文武官數人伴送。有司供館餼,遣兵護之。按途更代,以達京畿。既至,延入賓館,以時稽其人眾,均其飲食。翼日,具表文、方物,暨從官各服其服,詣部俟階下。儀制司官設表案堂中,質明,會同四譯館卿率貢使至禮部,侍郎一人出立案左,儀制司官二人分立左右楹。館卿先升,立左楹西。通事、序班各二人,引貢使等升階跪。正使舉表,館卿祗受,以授侍郎,陳案上,復位。使臣等行三跪九叩禮,興。退,館卿率之出。禮部官送表內閣俟命,貢物納所司。

如值大朝常朝,序班引貢使等列西班末,聽贊行禮如儀。非朝期則禮部先奏,若召見,館卿豫戒習儀。屆日帝御殿,禮部尚書引貢使入,通事隨行,至丹墀西行禮畢,升自西階,通事復從之。及殿門外跪,帝慰問,尚書承傳,通事轉諭,貢使對辭,通事譯言,尚書代奏。畢,乃退。如示優異,則丹墀行禮畢,即引入殿右門,立右翼大臣末,通事立少後。賜坐、賜茶,均隨大臣跪叩,飲畢,慰問傳答如初。出朝所,賜尚方飲食,退。翼日赴午門外謝恩。

禮部疏請頒賜國王並燕賚貢使,既得旨,所司陳賜物午門道左,館卿率貢使等東面立,侍郎西面立,有司咸序。貢使詣西墀三跪九叩,主客司官頒賜物授貢使,貢使跪受。以次頒賜貢使暨從官從人,咸跪受。贊「興,叩」如儀。退,賜宴禮部。

貢使將歸國,光祿寺備牲酒果蔬,侍郎就賓館筵燕,伴送供偫如前。所經省會皆饗之,司道一人主其事,館餼日給,概從周渥焉。

順治初,定制,諸國朝貢,齎表及方物,限船三艘,艘百人,貢役二十人。十三年,俄國察罕汗遣使入貢,以不諳朝儀,卻其貢,遣之歸。明年復表貢,途經三載,表文仍不合體制。世祖以外邦從化,宜予涵容,量加恩賞,諭令毋入覲。

康熙三十二年,俄復遣使義茲柏阿朗迭義迭來朝,帝始召見,賜坐賜食。五十九年,葡萄牙使臣斐拉理入覲,帝御九經三事殿。使者入殿左門,升左陛,進表御座則膝行。帝受表,使者興,出,凡出入皆三跪九叩。賜坐賜茶,謝恩如儀。

初,琉球、安南、暹羅諸使來,議政大臣咸會集,賜坐及茶。乾隆初元,諭停止。時屬國陪臣增擴,敕所司給皇清職貢圖,以詔方來。四十七年正月,紫光閣錫燕,朝鮮、琉球、南掌陪臣與焉。燕罷,賜珍物。五十年,舉千叟宴,特命朝鮮賀正陪臣齒逾六十者充正、副使,預宴賦詩。越五年,安南國王阮光平來京祝壽,定行禮班序,列親王、郡王間,其陪臣仍附班末。五十八年,英吉利入貢,使臣瑪戛爾等覲見,自陳不習拜跪,及至御前,而跽伏自若。

嘉慶初元,再舉千叟宴,朝鮮、安南、暹羅、廓爾喀額爾德尼王吉爾巴納足塔畢噶爾瑪薩九叩,「跪奉大皇帝前:竊小臣聞湖南教匪滋事,致天威震怒,遣兵剿除。今已平定,聞之忻慰。小臣受恩深重,虔修土產微物,表文,叩賀天喜。小臣屬蒙天恩,視如子民,唯有一心歸順,和睦鄰封。小臣陽布離京遠,年尚幼,伏墾當作奴輩,曲施教導,霑恩不淺」雲云。其貢物計十二事,語質意恭如此。

二十一年,英復遣使來貢,執事者告以須行拜跪禮,司當冬等遂稱疾不入覲,帝怒,諭遣歸國,罷筵宴賜物。嗣是英使不復來庭。

道光九年,回疆敉定,上太后徽號,緬甸國王遣使進金葉表,創舉也。

故事,琉球間歲一貢,至十九年,詔改四年為期。時國王尚育咨達閩撫吳文鎔,謂琉球瀕海,地患多風,朝貢以時,風雨和順,歲則大熟。貢舶出入閩疆,歲頒時憲書,獲以因時趨事。地不產藥,賴舶載回應用。至航海針法,非隨時練習不為功。若改四年,則恐豐歉不齊,人時莫授,藥品既缺,針盤益疏,請復舊制便。報可。並令陪臣子弟得隨貢使入監讀書。

光緒三十四年,廓爾喀入貢,賞正使噶箕二品服,副使四品服,其將事時,服色即各從其品,亦前此所未有者。

凡貢期,朝鮮歲至,琉球間歲一至,安南六歲再至,暹羅三歲,荷蘭、蘇祿五歲,南掌十歲,均各一至,餘道遠貢無常期。凡貢物,各將其土實,非土產者勿進。朝鮮、安南、琉球、緬甸、蘇祿、南掌皆有常物,餘唯其所獻。

敕封籓服禮清自太宗征服朝鮮,鑱石三田渡。厥後安南、琉球諸國,先後請封,皆遣使往。其他回首內鄉者,航海匪夐,梯山忘阻,則璽書褒獎,授來使齎還而已。

崇德間,定制,凡外邦效順,俱頒冊錫爵。進奏書牘,署大清紀年。若朝貢諸國無子嗣位,則遣陪臣請朝命,禮部奏遣正、副使各一人持節往封,特賜一品麒麟服以重其行。行日,工部給旗仗,兵部給乘傳。封使詣禮部,儀制司官一人奉節,一人奉詔敕,授本部長官,以授正、副使,跪受。興,出易征衣乘傳往。將入境,其國邊吏備館傳夫馬。緣途所經,有司跪接。

及國,嗣封王遣陪臣郊迎,三跪九叩,勞使者一跪三叩。延入館,陳詔節龍亭內,行禮如儀。謁使者三叩,不答。諏日,王率陪臣詣館,禮畢,王先歸。龍亭舁行,仗樂前導,封使後隨。入門陳正中,使者及階下馬,正使奉節,副使奉詔敕,入殿陳案上,退立東旁。王率眾官北面立,三跪九叩,興,詣封位前跪。副使奉詔書付宣讀官,宣訖,王行禮如初,出俟門外。使者出,跪送。有間,適館勞之。使者還朝,乃修表文,具方物,遣陪臣詣闕謝恩。

如諭祭兼冊封,先於其祖廟將事,諭祭文陳案上,使者左右立。世子跪叩如前,退立神位左,乃宣讀,眾俯伏。宣畢,興。送燎行禮,使者退。次行冊封禮,儀與前同。

至以詔敕授使齎還,則禮部設案午門,位正中,尚書立案左。儀制司官從館卿率來使入,授詔敕,序班引詣案前跪,授受如制。退詣丹墀西,三跪九叩,禮成,歸授國王。謝恩同。

外國公使覲見禮康熙初,外洋始入貢,中朝款接,稍異籓服。南懷仁官欽天監,贈工部侍郎,凡內廷召見,並許侍立,不行拜跪禮。雍正間,羅馬教皇遣使來京,世宗許行西禮,且與握手。乾隆季葉,英使馬格里入覲,禮臣與議儀式,彼以覲見英王為言,特旨允用西禮。筵宴日,且親賜卮酒。商約既締,將命頻繁。咸、同間,外國使臣嘗求入覲,時以禮制乖異,力拒之。同治時,英、法使臣固請再四,我猶繩以華制,莫之應。彼且曰,宜亟修好,阻其入覲,是靳以客禮也。

十二年,穆宗親政,泰西使臣環請瞻覲,呈國書,先自言用西禮,折腰者三,廷臣力言其不便。直隸總督李鴻章建議,略言:「先朝召見西使時,各國未立和約,各使未駐京師,國勢雖強,不逮今日,猶得律以升殿受表常儀。然嘉慶中,英使來朝,已不行三跪九叩禮。厥後成約,儼然均敵,未便以屬禮相繩。拒而不見,似於情未洽。糾以跪拜,又似所見不廣。第取其敬有餘,當恕其禮不足。惟宜議立規條,俾相遵守。各使之來,許一見,毋再見,許一時同見,毋單班求見,當可杜其覬覦。且禮與時變通,我朝待屬國有定制,待與國無定禮。近今商約,實數千年變局,國家無此禮例,往聖亦未豫定禮經,是在酌時勢權宜以樹之準。」時總理各國事務恭親王以拜跪儀節往復申辨,而各使堅執如初。勢難終拂其意,乃為奏請,明諭允行。

其年夏,日本使臣副島種臣、俄使臣倭良嘎哩、美使臣鏤斐迪、英使臣威妥瑪、法使臣熱福理、和蘭使臣費果蓀瞻覲紫光閣,呈國書,依商訂例行事。接見時,帝坐立唯意,賜茗酒,恩自上出。使臣訊安否,謹致賀辭。未垂問,毋先言事。西例臣見君鞠躬三,今改五鞠躬。使臣初至始覲見,餘則否。嗣後親奉國書者仿此。其禮式先期繪圖試習,覲見某處所,某月日時,並候旨行。其大略也。

光緒十六年,駐英使臣薛福成奏陳:「各使覲見,須定明例。凡使臣初至一國,其君莫不延見慰勞,使臣謁畢,鞠躬退,語不及公。此通例也。頃聞駐京公使,以未蒙晝接,不無私議。昔年英使威妥瑪藉詞不令入覲,致煙臺條款多要挾,靳虛文而受實損,非計之得。今宜循同治十二年成案,援據以行。若論禮節,可於召見先敕下所司,中禮西禮,假以便宜。如是,彼雖行西禮,仍於體制無損。」雲云。自是遂為定例。

二十七年,聯軍平拳匪,各國挾求更改禮節。謂各使臣會同覲見,必在太和殿。一國使臣單行覲見者,必在乾清宮。呈遞國書,必遣乘輿往迓,至宮殿前降輿,禮成送歸。齎奏國書,必自中門入,帝必躬親接受。設宴乾清宮,帝必躬親入座。嗣復允會同覲見改在乾清宮,而轎用黃色。於是慶親王奕劻等以天澤堂廉之辨,不能每事曲從。遂與各使磋商,歷時數月,始將乘坐黃轎、太和殿覲見暨宮殿階前降輿三事酌議改易,而爭議始息。

各國親王覲見儀,始光緒二十四年。德國親王亨利入覲,帝幸頤和園,御仁壽殿,亨利公服入,遞國書,帝慰勞之。既,亨利欲覲皇太后,帝奉懿旨代見。是日巳刻,御玉瀾堂,亨利偕德使海靖等入,外部司官引殿東便門外入布幄少憩。駕至,扈從如儀,鳴鞭三,升座。慶親王等侍左右,外部長官率亨利等自中門入,北鄉一鞠躬,行數武又一鞠躬,至龍柱前又一鞠躬。然後奉國書進,慶親王降左階接受,陳玉案,亨利等又一鞠躬,帝頷首答之,操國語慰勞。慶親王跪案左聆玉音,降階,操漢語傳宣。德繙譯官譯畢,亨利等又一鞠躬,帝仍頷首答之。亨利等退數武又一鞠躬,退至堂左,又一鞠躬。禮成。

內外王公相見禮崇德初元,定宗室外籓親王、郡王、貝勒、貝子相見儀。賓及門,王府屬官入告,主人降階迎,賓辭,主人升。賓從自中門入,賓趨左,主人趨右。行相見禮,二跪六叩,即席序立。從官升東階,行禮亦如之。興,入右門,坐賓後。執事獻茶,賓受茶,叩,主人答叩。飲茶敘語畢,從官趨前楹,跪,叩,興,趨出。賓離席跪叩,主人答叩,並興。賓出,主人降階送,屬官送門外。

若外籓郡王見,則主人迎送殿外,不降階。相見,賓二跪六叩,主人答半。賓辭退,跪叩,主人答跪不叩。餘如親王儀。

外籓貝勒見,主人離坐迎,不出殿,賓北面跪叩如初,主人立受。即席正坐,賓侍坐。辭退跪叩,主人立受不送。餘如郡王儀。

外籓貝子、公見,府屬官引賓入殿,跪叩同。辭退仍跪叩,主人皆坐受。餘如貝勒儀。

外籓親王見郡王,主人迎送大門內,餘與親王相見同。郡王見郡王亦如之。

其外籓貝勒見郡王,如郡王見親王禮。以下賓主相見,降殺遞差。

外籓親王見貝勒,主人迎送門外。賓入,主人從,相見各一跪三叩。外籓郡王暨貝勒見貝勒同。

外籓貝子、公見貝勒,賓一跪三叩,主人跪拱手受。

外籓王、貝勒見貝子,賓主一跪一叩坐,此其異者也。

京官相見禮順治元年,定制,京朝官敵體相見,賓及門,主人迎大門內,揖賓入,及階,讓升,賓西主東。及事,讓入,皆北面再拜。興,主人為賓正坐西面,賓辭,主人固請,卒正坐。賓還正主人坐東面亦如之。賓就坐,受茶,揖,主人答揖。飲茶敘語畢,告辭相揖。賓降階,主人送及門,復相揖。賓辭,主人固請,送賓大門外,視賓升輿馬,乃退。

尚書、左都御史見大學士同。賓降一品者,主人趨正賓坐,辭亦如之。餘儀同。

二品以下京堂官翰詹科道見大學士,主人迎儀門內,送大門外,不視升輿馬。

科道見左都御史、副都御史、尚書儀同。

五品至八品官見大學士,主人迎堂階下,賓就東階,主人導入。賓北面拜,辭,乃三揖,主人東面答揖。賓趨正主人坐,辭,固請,卒正坐相揖。賓西面,主人東北面坐。賓啟事畢,辭退,三揖如初。主人送二門外。

翰詹科道見二三品官,如賓降一等禮。見四五品官,如同官禮。

閣部寺監屬官見其長官,初見,公服詣署,升自東階,具履行陳坐案,依次向坐三揖,長官避席答揖。退。若燕見,如五品官見大學士儀。

國學生見國子師儀,初見,具名柬,公服詣學,自東階升堂,北面三揖,師立受。侍立左旁,西面受教,畢,三揖退。若燕見,通名,俟召乃入。師迎階上,弟子升,揖。師入門,從之,北面再拜,師西面答揖。趨正師坐,師命坐,北面揖。師位東北面,弟子西面。茶至,揖,請問,揖。辭退,北面三揖,師皆答。出送,師前行,弟子後隨,及二門外,弟子三揖,俟師入始退。

翰林院庶吉士見大學士,與見教習庶吉士同。

凡京朝官途遇回避,爵秩均等,分道行,次讓道行,次勒馬俟其過,又次下馬,唯欽使即遇應回避者,分道行可也。又武職民公、侯、伯以下,男以上,文職大學士以下,九卿以上,得用引馬一騎,途遇並下馬回避雲。

直省文武官相見禮順治間,定督、撫、學政、河漕總督,鹽政,巡視御史相見,坐次平行,餘各按品秩行禮。

雍正八年,定直省官相見,位均等者,賓至署,吏入白,啟門,自中門入,至外堂簷下降輿馬。主人迎簷前,揖賓入。及事,各再拜。其正坐、就位、進茶、辭退,如京朝官儀。

屬官見長官,轅門外降輿馬,自左門入。初見具名柬,呈履行,文官司道見督撫,迎堂後屏內。及事,庭參則扶免,三揖,皆答揖。督撫正坐,司道旁坐。命坐,揖。茶至,揖。均答如儀。辭出,三揖如初。送至屏門外,司道三揖。俟督撫入,復三揖,趨出。督撫次日用名柬答拜。若公事謁見,常服通銜名,三揖就坐。餘同前。

府、、州、縣見,庭參拜則免,府、揖,答揖。州、縣揖,立受。俱不送,不答拜。

佐貳等官見,一跪三叩,不揖、不坐。府、、州、縣見司道,與司道見督撫同。佐貳等官見司道,與見督撫同。

同知、通判見知府,柬題晚生,入自中門,用賓主禮。

州、縣教職見督撫,儀如佐貳見司道,不迎送。見知府,迎送屏門外。見府倅,迎送堂簷下。餘同。見州、縣,如同、通見知府儀。

司、道、府、見學政,入中門,禮如賓主,迎送並出堂簷。學政品秩崇者,如見督撫儀。州、縣見,庭參旁坐,主人答揖不答拜。

運使見督撫、鹽院,與司道同。運、判以次遞降。

武官副將以下見提督,初見具銜名、履行,披執則傳免,易公服佩刀。都司、守備不免,跪宣名,席地坐,不進茶。餘儀按品遞降,與文職同。

順治十三年,定直省文武官相見禮,提督見總督,入中門,至儀門下馬,升堂三揖。總督正坐,提督僉坐,迎送不出堂簷。若提督兼世職者,總督西面,提督東面。辭出,送至堂簷下,視乘馬。

總兵見,儀門外下馬,坐則侍坐,迎、送止階上。與巡撫見,視賓主禮唯均,以下按品差降。

至滿、漢官相見,將軍、副都統與督、撫、提、鎮以敵體見。司道以下見將軍如總督,見副都統如總兵,協領、參領見督撫同司道,佐領、防禦同知府,驍騎校同州、縣。不相統屬者,一以賓主禮行之。

其儒學弟子員見學師,與國子生見國學師同。

士庶相見禮賓及門,從者通名,主人出迎大門外,揖入。及門、及階揖如初。登堂,各北面再拜。興,主賓互正坐。即席,賓東主西。飲茶,語畢,賓退,揖。及階、及門,揖,辭,主人皆答揖。送大門外,揖如初。卑幼見尊長禮,及門通名,俟外次,尊長召入見,升階,北面再拜,尊長西面答揖。命坐,視尊長坐次侍坐。荼至,揖,語畢,稟辭,三揖。凡揖皆答,出不送。若尊長來見,卑幼迎送大門外。餘如前儀。見父執友,與見尊長儀同。

受業弟子見師長禮,初見,師未出,先入,設席正位,俟堂下。師出召見,乃奉贄入,奠贄於席,北面再拜,師立答揖。興,謹問起居。命坐乃侍坐。有問,起而對。辭出,三揖,不送。常見侍坐,請業則起,請益則起。師有教,立聽。命坐乃坐。師問更對,仍起而對。朝入暮出均一揖。與同學弟子,以齒序之。


\end{pinyinscope}