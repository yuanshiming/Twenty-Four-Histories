\article{列傳二百七十二}

\begin{pinyinscope}
文苑二

諸錦沈廷芳夏之蓉厲鶚汪沆符曾陳撰趙昱趙信王峻王延年何夢瑤勞孝輿羅天尺蘇珥車騰芳許遂韓海劉大櫆胡宗緒王灼李鍇陳景元戴亨長海吳麟曹寅鮑珍高鶚劉文麟沈炳震弟炳謙炳巽趙一清曹仁虎吳泰來黃文蓮胡天游彭兆蓀袁枚程晉芳張問陶王又曾子復祝維誥萬光泰維誥子喆邵齊燾王太嶽吳錫麒楊芳燦楊揆吳鼒徐文靖趙青藜汪越硃仕琇高澍然蔣士銓汪軔楊垕趙由儀吳嵩梁樂鈞趙翼黃景仁呂星垣楊倫徐書受嚴長明子觀硃筠翁方綱姚鼐吳定魯九皋陳用光吳德旋宋大樽錢林端木國瑚吳文溥章學誠章宗源姚振宗吳蘭庭祁韻士張穆何秋濤馮敏昌宋湘趙希璜法式善孫原湘郭麟惲敬趙懷玉黎簡張錦芳張錦麟黃丹書呂堅胡亦常張士元張海珊張履

諸錦,字襄七,秀水人。少時家貧陋,輒就讀書肆,主人敬其勤學,恣所觀覽。顧嗣立為之延譽,名大起。雍正二年進士。乾隆初,試鴻博,授編修。閉門撰述,不詣權要。至左贊善,遂告歸。著有毛詩說、饗禮補亡、夏小正注及絳跗閣集。

先是康熙己未徵博學鴻儒,得人稱盛。高宗御極,復舉行焉,內外薦達二百六十七人,試列一等者五人,錦第三;二等十人。明年補試,續取四人,錢塘陳兆侖、仁和沈廷芳、高郵夏之蓉,皆試列二等者也。兆侖自有傳。

廷芳,字畹叔。由監生舉鴻博,授編修,遷御史。奏毀都城智化寺內明閹王振造像及李賢所撰頌德碑,報可。出為登萊青道,遷河南按察使。廷芳少從方苞游,為文無纖佻之習。詩學本查慎行。著隱拙齋集及十三經注疏正字、續經義考等書。

之蓉,字芙裳。雍正十一年進士。舉鴻博,以檢討典試福建,又督廣東、湖南學政。其校士也,必以通經學古為先。

當時試一等者,劉綸居首,次則南城潘安禮、金壇於振、錢塘杭世駿;二等自兆侖等三人外,為無錫楊度汪,菏澤劉玉麟,休寧汪士湟、程恂,錢塘陳士璠,天臺齊召南,會稽周長登。其續取者,一等宜興萬松齡,二等桐鄉硃荃、南安洪世澤、石屏張漢,凡十九人。惟綸、玉麟官最顯,而世駿、召南及兆侖尤知名於世云。

厲鶚,字太鴻,錢塘人。家貧,性孤峭,不茍合。始為詩即得佳句。於學無所不窺,一發之於詩。康熙五十九年,李紱典試浙江,得鶚卷,閱其謝表,曰:「此必詩人也!」亟錄之。計偕入都,尤以詩見賞湯右曾。再試禮部不第。乾隆元年,舉鴻博,誤寫論置詩前,又報罷。其後赴都銓,行次天津,留友人查為仁水西莊,觴詠數月,不就選,歸。卒,年六十一。

鶚搜奇嗜博。揚州馬曰琯小玲瓏山館富藏書,鶚久客其所,多見宋人集,為宋詩紀事一百卷。又南宋畫院錄、遼史拾遺、東城雜記諸書,皆博洽詳贍。詩刻鍊,尤工五言,有自得之趣。詩餘亦擅南宋諸家之長。先世本慈谿,徙居錢塘,故仍以四明山樊榭名其集云。鶚嘗與趙信、符曾等人各為南宋雜事詩一百首,自採諸書為之注,徵引浩博,考史事者重之。

汪沆,字師李。少從鶚受詩,亦試鴻博報罷。其後大學士史貽直將以經學薦,以母老辭。

同時浙江舉鴻博未錄用者,符曾,字幼魯。官戶部郎中。鄞縣陳撰最推服其詩。撰,字楞山,毛奇齡弟子。以布衣薦,未就試。仁和趙昱,字功平。貢生。弟信,字辰垣。國學生。兄弟同舉。家有池館之勝,喜購書。連江陳氏世善堂書散出,皆歸之。

王峻,字艮齋,常熟人。少與同裏宋君玉師事陳祖範,一時並稱王宋。雍正二年進士,授編修。歷典浙江、貴州、雲南鄉試。乾隆初,改御史,拜官甫三日,劾左都御史彭維新矯詐苛鄙,直聲震都下。以母憂去官,遂不出。主講安定、雲龍、紫陽書院。其學長於史,尤精地理。嘗以水經正文及注混淆,欲一一釐定之,而補唐以後水道之遷變,及地名之同異,為水經廣注,手自屬稿,未暇成也。惟成漢書正誤四卷。錢大昕謂駕三劉氏、吳氏刊誤上也。書法橅李北海,所書碑碣盛行於時。

王延年,字介眉,錢塘人。雍正四年舉人。乾隆初,舉鴻博,後官國子監學政。十七年,會試,以耆年晉司業,賜翰林院侍講銜。延年史學洽熟,嘗補袁樞通鑒紀事本末,以原書不言田制,則度地居民之法亡;不言漕運,則鑿渠引河之利塞;不言府兵,則耕牧戰守之功隳。至於耶律鴟張遼海,而陳邦瞻書不究其終;黨項虎視河、湟,薛應旂書不詳其始。紹建安者又如此,不可不亟正之也。杭世駿序之,比延年於唐杜君卿、宋劉中原父云。晚年,大學士蔣溥、劉統勛皆以經學薦,又自進呈所著書,上嘉許焉。

何夢瑤,字報之,南海人。惠士奇視學廣東,一以通經學古為教。夢瑤與同里勞孝輿、吳世忠,順德羅天尺、蘇珥、陳世和、陳海六,番禺吳秋一時並起,有「惠門八子」之目。雍正八年成進士,出宰粵西,治獄明慎,終奉天遼陽知州。性長於詩,兼通音律算術。謂蔡元定律呂新書,本原九章,為之訓釋。更取禦制律呂正義研究八音協律和聲之用,述其大要。參以曹廷棟琴學,為書一編。時稱其決擇精當。又著算迪,述梅氏之學,兼闡數理精蘊、歷象考成之旨。江籓謂近世為此學者,知有法,不知法之所以然;知之者,惟夢瑤也。

孝輿,字阮齋。乾隆元年,召試鴻博,未用。以拔貢生廷試第五,出為黔中令。治古州屯務,足繭萬山中。將去,民攀轅曰:「公勞苦以衣食我!」皆泣下。歷錦屏、龍泉、鎮遠諸邑,皆有績。卒於官。

天尺,字履先。年十七,應學使試。士奇手錄其賦、詩示諸生,名大起。徵鴻博,念親老不就,以舉人終。雍正時修一統志,與孝輿同纂粵乘。孝輿忤俗,被口語,天尺力白之。所居里曰石湖,世以前有範石湖,因稱後石湖以別之云。

珥,字瑞一。為文長於序記,詩有別趣,書法亦工。惠士奇稱之曰「南海明珠」。舉鴻博,以母老,辭不試。乾隆初鄉舉,一試禮部,遂不出。

時粵東舉鴻博者,又有番禺車騰芳,字圖南。康熙末,與里人許遂同徵。至京後期,即乞終養歸。後為海豐學官。學使吳鴻雅重之,嘗從容問其諸子頗有應試者乎,騰芳以皆失學對,吳益嘆異焉。

遂,字揚云。康熙中舉人。為清河令,蠲逋賦,民德之。坐事去職。巡撫薦應鴻博,格於部議,未試歸。

韓海,字偉五,亦番禺人也。雍正十一年進士,官封川教諭。大府欲薦應鴻博,海賦詩以見志,大府覽詩愕然,遂不復強。海亦旋卒。

劉大櫆,字才甫,一字耕南,桐城人。曾祖日燿,明末官歙縣訓導,鄉里仰其高節。其後累世皆為諸生,至大櫆益有名。始年二十餘入京師,時方苞負海內重望,後生以文謁者不輕許與,獨奇賞大櫆。雍正中,兩登副榜,竟不獲舉。乾隆元年,苞薦應詞科,大學士張廷玉黜落之,已而悔。十五年,特以經學薦,復不錄。久之,選黟縣教諭,數年告歸。居樅陽江上不復出,年八十三,卒。

大櫆修幹美髯,能引拳入口。縱聲讀古詩文,聆其音節,皆神會理解。桐城自方苞為古文之學,同時有戴名世、胡宗緒。名世被禍,宗緒博學,名不甚顯。大櫆雖游苞門,傳其義法,而才調獨出,著海峰詩文集。姚鼐繼起,其學說盛行於時,尤推服大櫆。世遂稱曰「方劉姚」。

宗緒,字襲參。康熙末,以舉人薦充明史館纂修。雍正八年進士,授編修,遷國子監司業。少孤貧,母潘苦節,課之嚴而有法。感憤勵學,自經史以逮律歷、兵刑、六書、九章、禮儀、音律之類,莫不研窮。著易管、洪範皇極疑義、古今樂通、律衍數度衍參注、晝夜儀象說、歲差新論、測量大意、梅胡問答、九九淺說、正字通芟誤、正蒙解、大學講義、方輿考、南河北河論、膠萊河考、臺灣考、兩戒辨、苗疆紀事等書。自為詩文曰環隅集,古藻過大櫆。大櫆同邑門人自姚鼐外推王灼。

灼,字濱麓。乾隆五十一年舉人,選東流教諭。嘗館於歙,與金榜、程瑤田及武進張惠言諸人相友善。一日見惠言黃山賦,曰:「子之才可追古作者,何必託齊、梁以下自域乎!」惠言遂棄儷體為古文。灼所著悔生詩文鈔,鮑桂星為刊行焉。

李鍇,字鐵君,漢軍正黃旗人。祖恆忠,副都統。湖廣總督輝祖子。鍇娶大學士索額圖女,家世貴盛,其於榮利泊如也。性友愛,兄伊山、祈山仕不遂,鍇省伊山戍所,累月乃歸。祈山罷官還,無宅,以己屋授之,並鬻產為清宿逋。嘗一充官庫筆帖式,旋棄去。乾隆元年,舉鴻博,未中選。十五年,詔舉經學,大臣交章論薦,以老疾辭。少好山水,游所至,務窮其奇。苦嗜茗,為鐵鐺瓦缶,一奴負以從。客江南,嘗月夜挾琴客泛舟採石,彈大雅之章,扣舷和之,水宿者皆驚起,人莫測其致也。鍇既以屋讓兄,乃築室盤山廌青峰下,閉戶躭吟,罕接人事。歲一至城中,一二日即去。居盤山二十載而歿。詩古奧峭削。著睫巢集,又著原易及春秋通義、尚史。

陳景元,字石閭,漢軍鑲紅旗人。詩擬孟郊、賈島。有石閭集。與戴亨、長海為「遼東三老」。

亨,字通乾,號遂堂,沈陽人,原籍錢塘。父梓,以事戍遼,見藝術傳。亨,康熙六十年進士。官山東齊河縣知縣,以抗直忤上官,解組去。寄居京師,家益貧,晏如也。為人篤於至性,不輕然諾,夙敦風義。其詩宗杜少陵,上溯漢、魏,卓然名家。有慶芝堂詩集。

長海,字匯川,納喇氏,滿洲鑲白旗人,鎮安將軍瑪奇子。例予廕,長海不就。檄補戶部庫使,又逃,曰:「庫使司帑藏,歲豐入,懼焉。逃死,非逃富貴也。」其母賢,聽之,遂布衣終其身。沖遠任真,趣無容心。博古多識,嗜金石書畫,當意則傾囊購之。嘗襲裘行吊,解裘以濟戚喪。歸塗見未見書,買之,復解其衣。由是中寒疾,乃夷然曰:「獲多矣!」中歲愛易水雷溪之勝,築大★E5菴,因以為號。晚入京居委巷,又顏其閣曰「玉衡」,懸畫四壁,對之吟諷。其詩矩矱古人,而不膠於固,斷句尤冠絕一時。論詩以性情為主,舉靡麗之習而空之。有雷溪草堂詩。乾隆九年,卒,年六十有七。

遼東以詩文名者,又有吳麟,字子瑞,號晚亭,滿洲鑲黃旗人。康熙四十九年舉人,授內閣中書。與鍇同舉鴻博,與修明史,纂本紀,充明史綱目纂修官。善詩文,兼工山水。著有黍谷山房集。

曹寅,字楝亭,漢軍正白旗人,世居沈陽,工部尚書璽子。累官通政使、江寧織造。有楝亭詩文詞鈔。

鮑珍,字冠亭,秘書院大學士鮑承先裔。乾隆初,官嘉興海防同知。有道腴堂全集。

高鶚,字蘭墅,亦漢軍旗人。乾隆六十年進士。有蘭墅詩鈔。至道光年則有劉文麟,字仙樵,遼陽人。九歲能詩。以進士用廣東知縣,總督林則徐器之。權平遠,兼長樂。俗悍,喜械斗,文麟甫任,單輿遽入解之,眾羅拜,皆釋兵,俗為之易。補文昌,丁憂。再選河南沈丘。時患匪,設方略擒其渠,盜賊息跡。以忤上官劾降,遂歸,主沈陽書院。論詩以婉至為宗,語必有寄託。英光偉氣,一發之於詩。論者謂足繼遼東三老。有仙樵詩鈔。其門人王乃新,字雪樵,承德人。亦能詩,有雪樵詩賸。

沈炳震,字東父,歸安人。少喜博覽,讀史於年月世系,人所忽者,必默識之。嘗著新舊唐書合鈔,紀傳以舊書為綱,分注新書為目;舊志多舛略,則以新書為綱,分注舊書為目。又補列方鎮表,拜罷承襲諸節目,積數十寒暑乃成。又著二十四史四譜:一紀元,二封爵,三宰執,四謚法。其體出於表歷,而變其旁行斜上為標目。乾隆元年,與弟炳謙皆以貢生試鴻博,報罷。逾年,卒,年五十九。卒後六年,侍郎錢陳群奏進其唐書合鈔,詔付書局,採錄唐書考證中。

炳謙,字幼孜,炳震季弟也。次弟炳巽,字繹旃。著水經注集釋訂譌,據明黃省曾刊本,以己意校定之。遍檢古籍,錄其文字異同者,間附諸家考訂之說。州縣沿革,則悉以今名釋焉。初未見硃謀韋本,後求得,多與之合。同時治水經者,有全祖望、趙一清。

一清,字誠夫,仁和人。國子監生。父昱,季父信,見厲鶚傳。一清稟其家學,博極群書。水經注傳寫訛奪,歐陽玄、王禕稱其經、注混淆,祖望又謂道元注中有注。一清因從其說,辨驗文義,離析之,使文屬而語不雜。又唐六典注稱桑欽所引天下之水百三十七,江、河在焉,今少二十一水。考崇文總目,水經注三十六卷,蓋宋代已佚其五卷。此二十一水,即在所佚中。於是雜採他書,證以本注,得滏、洺等十八水。又分★C1水、★C1餘水,清、濁漳,大小遼水,增多二十一,與六典注合。為水經注釋,又成水經箋刊誤,以正硃謀韋之失。方觀承督直隸,撰直隸河渠志,一清所草創,而戴震要刪之。其自著有東潛文集。

曹仁虎,字來殷,嘉定人。少稱奇才。乾隆二十二年,南巡,獻賦,召試列一等,賜舉人,授內閣中書。二十六年,成進士,選庶吉士,授編修。每遇大禮,高文典冊,多出其手。擢右中允,充日講起居注官,累遷侍講學士。五十一年,視學粵東。方按試連州,聞母訃,酷暑奔喪,晝夜號泣,竟以毀,卒於途。

仁虎以文字受主知,聲華冠都下,屢典文衡。詩宗三唐,而神明變化,一洗粗率佻巧之習。格律醇雅,醖釀深厚,為一時所推。著有宛委山房詩集、蓉鏡堂文稿。與王鳴盛、王昶、錢大昕、趙文哲及吳泰來、黃文蓮稱「吳中七子」。鳴盛等四人皆自有傳。

泰來,字企晉,長洲人。乾隆二十五年進士,用內閣中書。乞病歸,築遂初園於木瀆。藏書多宋、元善本。畢沅延主關中及大梁書院,與洪亮吉輩往還唱和。其詩一本漁洋,著有凈名軒、硯山堂等集。

文蓮,字芳亭,上海人。官知縣,有聽雨集。

胡天游,字稚威,山陰人,初姓方,名游。副榜貢生。乾隆元年,尚書任蘭枝薦舉鴻博,次年補試,鼻衄大作,投卷出。時四方文士雲集京師,每置酒高會,分題命賦,天游輒出數千言,沉博絕麗,見者咸驚服。性耿介,公卿欲招致一見,不可得。後舉經學,再報罷。客山西,卒。著有石笥山房集。

自言古文學韓愈,然往往澀險似劉蛻,非其至也。儷體文自三唐而下,日趨頹靡。清初陳維崧、毛奇齡稍振起之,至天游奧衍入古,遂臻極盛。而邵齊燾、孔廣森、洪亮吉輩繼起,才力所至,皆足名家。後數十年而有鎮洋彭兆蓀,以選聲鍊色勝,名重一時。

兆蓀,字湘涵。少有才名,久困無所遇。舉道光元年孝廉方正。胡克家為江蘇布政使,客其所。時總督以國用不足議加賦,兆蓀為克家力陳其不可,事得寢。又偕顧廣圻同校元本通鑒及文選,世稱其精槧。晚依曾燠兩淮鹽運使署。著小謨觴館集,燠為點定之。

袁枚,字子才,錢塘人。幼有異稟。年十二,補縣學生。弱冠,省叔父廣西撫幕,巡撫金鉷見而異之,試以銅鼓賦,立就,甚瑰麗。會開博學鴻詞科,遂疏薦之。時海內舉者二百餘人,枚年最少,試報罷。乾隆四年,成進士,選庶吉士。改知縣江南,歷溧水、江浦、沭陽,調劇江寧。時尹繼善為總督,知枚才,枚亦遇事盡其能。市人至以所判事作歌曲刻行四方。枚不以吏能自喜,既而引疾家居。再起發陜西,丁父憂歸,遂牒請養母。卜築江寧小倉山,號隨園,崇飾池館,自是優游其中者五十年。時出游佳山水,終不復仕。盡其才以為文辭詩歌,名流造請無虛日,詼諧詄蕩,人人意滿。後生少年一言之美,稱之不容口。篤於友誼,編修程晉芳死,舉借券五千金焚之,且恤其孤焉。

天才穎異。論詩主抒寫性靈,他人意所欲出,不達者悉為達之。士多效其體。著隨園集,凡三十餘種。上自公卿下至市井負販,皆知其名。海外琉球有來求其書者。然枚喜聲色,其所作亦頗以滑易獲世譏云。卒,年八十二。

晉芳,字魚門,江都人。家世業鹺。乾隆初,兩淮殷富,程氏尤豪侈。晉芳獨好儒,購書五萬卷,不問生產,罄其貲。少問經義於從父廷祚,學古文於劉大櫆。而與袁枚、商盤諸人往復唱和,甚相得也。乾隆七年,召試,授中書。十七年,成進士,以吏部員外郎為四庫館纂修,書成改編修。晚歲益窮,官京師至不能舉火。就畢沅謀歸計,抵關中一月卒,年六十七。晉芳於易、書、詩、禮皆有撰述,又有諸經答問、群書題跋、蕺園詩文集。

張問陶,字仲冶,遂寧人,大學士鵬翮玄孫。以詩名,書畫亦俱勝。乾隆五十五年進士,由檢討改御史,復改吏部郎中。出知萊州府,忤上官意,遂乞病。游吳、越,未幾,卒於蘇州。始見袁枚,枚曰:「所以老而不死者,以未讀君詩耳!」其欽挹之如此。著有船山集。

兄問安,字亥白。舉人。家居奉母,淡於榮利。其詩才超逸,與問陶有二難之目。

王又曾,字受銘,秀水人。乾隆十六年,南巡召試,賜舉人,授內閣中書。十九年,成進士,授刑部主事。同縣錢載論詩宗黃庭堅,務縋深鑿險,不墮臼科。又曾與硃沛然、陳向中、祝維誥和之,號「南郭五子」。又有萬光泰、汪孟鋗、仲鈖皆與同時相鏃礪,力求捐棄塵壒,毋一語相襲取。為詩不異指趣,亦不同體格。時目為秀水派,而又曾與維誥、光泰尤工。

又曾卒,其子復乞載定其詩,號丁辛老屋集。畢沅為之序,謂於漢、魏、六朝及唐、宋諸家外,能融會變化自成一家,取材於眾所不經見,用意於前人所未發,尤又曾所獨到雲。

維誥,字宣臣。乾隆三年舉人,官內閣中書。有綠谿詩鈔。光泰,字循初。乾隆元年舉人。有柘坡居士集。維誥詩,全祖望稱其俊雅,李鍇稱其醇靜。光泰詩,杭世駿稱其秀朗,載亦稱其綺麗。蓋雖宗庭堅,而鍛鍊精到,絕無西江槎枒詰屈之習。沛然,舉人,知高安縣,卒官。向中客死涼州,詩傳者差少。孟鋗,進士,吏部主事;仲鈖,舉人:皆有集。而復與載子世錫,維誥子喆相與稱。詩守家法。世錫已見載傳,有麂山老屋集。

復,字敦初。官河南鄢陵知縣。有樹萱堂、晚晴軒二集。沅採入吳會英才集。

喆,字明甫。乾隆二十五年舉人。有西澗詩鈔。

孟鋗子如洋,乾隆四十五年會試,廷試皆第一,亦與復等唱和。

邵齊燾,字叔宀,昭文人。幼異敏,甫受書即能了大義。乾隆七年進士,以編修居詞館十年。嘗獻東巡頌,時稱班、揚之亞,群公爭欲致門下。齊燾意度夷曠,殊落落也。年三十六,即罷歸。自顏其室曰「道山祿隱」。主常州龍城書院,洪亮吉、黃景仁皆從受學。善為儷體文,氣格排奡,意欲矯陳維崧、吳綺、章藻功三家之失。卒,年五十有二。著玉芝堂集。

王太嶽,字基平,定興人。齊燾同年進士,授檢討。由侍讀出補甘肅平慶道,調西安,遷湖南按察使。調雲南,擢布政使,坐事落職。命充四庫館總纂官。四十三年,仍授檢討。後遷司業,卒。太嶽蒞官有惠政,尤留心水利,與齊燾最善,駢文清剛簡直亦相近。有清虛山房集。

吳錫麒,字穀人,錢塘人。性至孝。乾隆四十年進士,授編修。累遷祭酒,以親老乞養歸。主講揚州安定樂儀書院。錫麒工應制詩文,兼善倚聲。浙中詩派,前有硃彞尊、查慎行,繼之者杭世駿、厲鶚。二人殂謝後,推錫麒,藝林奉為圭臬焉。著有正山房集。全椒吳鼒嘗輯錄齊燾、亮吉、錫麒及劉星煒、袁枚、孫星衍、孔廣森、曾燠之文為八家四六云。此八家外,有金匱楊芳燦,與弟揆並負時名。

芳燦,字蓉裳。母夢五色雀集庭樹而生。詩文華贍,學使彭元瑞大異之。乾隆四十二年拔貢生。廷試得知縣,補甘肅之伏羌。回民田五反,縣民馬稱驥應之。未發,芳燦從稱驥甥馬映龍偵得,立捕斬之,因城守。賊奄至,以無應,解圍去。憾映龍洩其謀,揚言映龍故與通,約五日後獻城也。阿桂逮映龍,將殺之,卒以芳燦言得免。敘功,擢知靈州,顧不樂外吏,入貲為戶部員外郎。與修會典,益務記覽。為詞章,嘗曰:「色不欲麗,氣不欲縱,沉博奧衍,斯駢體之能事矣。」丁母憂,貧甚,鬻書以歸。著芙蓉山館詩文鈔。

揆,字荔裳。乾隆中,召試舉人,授中書。從福康安征衛藏。官至四川布政使。有藤花館稿。

鼒,字山尊。嘉慶四年進士,終侍講學士。以母老告歸,主講揚州。亦長駢體,有夕蔡書屋集。

徐文靖,字位山,當塗人。父章達,以孝義稱鄉里。文靖務古學,無所不窺。著述甚富,皆援據經史。雍正改元,年五十七,始舉江南鄉試。侍郎黃叔琳典試還朝,以得三不朽士自矜,蓋指文靖及任啟運、陳祖範也。乾隆改元,試鴻博,不遇。詹事張鵬翀以所著山河兩戒考、管城碩記進呈,賜國子監學正。十七年,徵經學,入都。會開萬壽恩科,遂與試,年八十六,以老壽賜檢討,給假歸。卒,年九十餘。其所著又有周易拾遺、禹貢會箋、竹書統箋諸書。

趙青藜,字然一,涇縣人。九歲能文,乾隆元年,舉會試第一,選庶吉士,授編修,充浙江鄉試考官。遷御史,再充浙江考官,母憂歸,服闋,還臺,又充湖南考官。在臺前後五年,有直聲。如請清屯田、歸運丁、弛米禁、濟民食、提耗羨歸公、興西北水利;又劾總督高斌、侍郎周學健奏開捐例,啟言利之端,為害甚大。所言能持大體,不為激切之論。尋以耳疾乞休,年八十餘,卒。青藜外和內嚴,以不欺為主。受古文義法於方苞,苞稱及門中如青藜者,可信其操行之終不迷。著有漱芳居集,讀左管窺,於春秋二百四十二年穿穴甚深。

先是青藜同郡以史學稱者,推南陵汪越,字師退。康熙四十四年舉人。食貧勵節,守令咸折節致敬。不妄干謁。著綠影草堂集,沖淡典博。其讀史記十表,排比舊文,鉤稽微義,所得尤多。

硃仕琇,字斐瞻,建寧人。資性朗悟,而記誦拙,日可數十言,援筆為文輒立就。從南豐汪世麟學古文,臨別請益,世麟曰:「子但通習諸經,則世無與抗矣。」仕琇驚詫其言,遂以己意求之經傳,旁及百家諸子書,一以昌黎為宗。副都御史雷鋐見其文,嘆為醇古沖澹,近古大家,自是名大著。乾隆九年,舉鄉試第一。逾四年,成進士,選庶吉士。散館,出知夏津縣,民為之謠曰:「夏津清,我公能。」在任七年,以河決,改福寧府學教授。歸,主鰲峰講席者十年,卒,年六十六。

仕琇以古文辭自力,其意欲追古之立言者。以為清穆者惟天,澹泊者惟水,含之咀之,得其妙以為文者惟人。嘗與友人書曰:「為文在先高其志。其心有以自得,則吾心猶古人之心也,以觀古人之言,猶吾言也。然後辨其是非焉,究其誠偽焉,定其高下焉,如黑白之判於前矣。於是順其節次焉,還其訓詁焉,沉潛其義蘊焉,調和其心氣焉,久則自然合之,又久則變化生之。於是文之高也,如累土之成臺,如鴻漸之在天,有莫知其所以然者。」仕琇與大興硃筠及弟珪友善,筠推服其文甚至。著梅崖文集。

福建古文之學自仕琇。其後再傳有高澍然,字雨農,光澤人。嘉慶七年舉人,授內閣中書。未幾,移病歸。研說經傳,尤篤嗜昌黎集。其文陳義正,言不過物,高視塵壒之表。名不如仕琇,要其自得之趣,有不求人知能自樹立者。著春秋釋經、論語私記、韓文故及抑快軒文集。

蔣士銓,字心餘,鉛山人。家故貧,四歲,母鍾氏授書,斷竹篾為點畫,攢簇成字教之。既長,工為文,喜吟詠。由舉人官中書。乾隆二十二年,成進士,授編修。文名藉甚,裘曰修、彭元瑞並薦其才。旋乞病歸。帝屢從元瑞詢之,元瑞之士銓母老對。帝賜詩元瑞,有「江西兩名士」之句。士銓感恩眷,力疾起補官,記名以御史用。未幾,仍以病乞休,遂卒,年六十二。

士銓賦性悱惻,以古賢者自勵,急人之難如不及。詩詞雄傑,至★述節烈,能使讀者感泣。著忠雅堂集。少時與武寧汪軔、南昌楊垕為昆弟交,出入必偕,財物與共。

軔,字魚亭,優貢生。垕,字子載,舉人,本天全六番招討宣慰使孫,雍正初,改土歸流,安置江西,遂為南昌人。詩名與軔相埒。士銓甚推服之。同時有南豐趙由儀,字山南。與士銓等並稱四子。其後繼起者,曰東鄉吳嵩梁、臨川樂鈞。

嵩梁,字蘭雪。以舉人官中書,選知黔西州。著香蘇山館集。聲播外夷,朝鮮吏曹判書金魯敬以梅花一龕供奉之,稱為詩佛。日本賈人斥四金購其詩扇。其名重如此。

鈞,初名宮譜,字元淑。嘉慶六年舉人。與嵩梁同為翁方綱弟子。著青芝山館集。

趙翼,字耘松,陽湖人。生三歲能識字,年十二,為文一日成七篇,人奇其才。乾隆十九年,由舉人中明通榜,用內閣中書,入直軍機,大學士傅恆尤重之。二十六年,復成進士,殿試擬一甲第一,王傑第三。高宗謂陜西自國朝以來未有以一甲一名及第者,遂拔傑而移翼第三,授編修。

後出知鎮安府。粵民輸穀常社倉,用竹筐,以權代概。有司因購馬濟滇軍,別置大筐斂穀,後遂不革,民苦之。翼聽民用舊筐,自權,持羨去,民由是感激,每出行,爭肩輿過其村。先是鎮民付奉入雲南土富州為奸,捕獲百餘人,付奉顧逸去,前守以是罷官。已而付奉死,驗其尸良是。總督李侍堯疑其為前守道地,翼申辨,總督怒,劾之。適朝廷用兵緬甸,命翼赴軍贊畫,乃追劾疏還。傅恆既至滇,經略兵事,議以大兵渡戛鳩江,別遣偏師從普洱進。翼謂普洱距戛鳩江四千餘里,不如由江東岸近地取猛密,如其策入告。其後戛鳩兵遭瘴多疾病,而阿桂所統江東岸一軍獨完,卒以蕆事。尋調守廣州,擢貴西兵備道。以廣州讞獄舊案降級,遂乞歸,不復出。

五十二年,林爽文反臺灣,侍堯赴閩治軍,邀翼與俱。時總兵柴大紀城守半載,以易子析骸入告。帝意動,諭大紀以兵護民內渡。侍堯以詢翼,翼曰:「總兵欲內渡久矣,憚國法故不敢。今一棄城,則鹿耳門為賊有,全臺休矣!即大兵至,無路可入。宜封還此旨。」侍堯悟,從之,明日接追還前旨之諭,侍堯膺殊賞;而大將軍福康安續至,遂得由鹿耳門進兵破賊,皆翼計也。

事平,辭歸,以著述自娛。尤邃史學,著廿二史劄記、皇朝武功紀盛、陔餘叢考、簷曝雜記、甌北詩集。嘉慶十五年,重宴鹿鳴,賜三品銜。卒,年八十六。同時袁枚、蔣士銓與翼齊名,而翼有經世之略,未盡其用。所為詩無不如人意所欲為,亦其才優也。

其同裏學人後於翼而知名者,有洪亮吉、孫星衍、趙懷玉、黃景仁、楊倫、呂星垣、徐書受,號為「毗陵七子」。亮吉、星衍、懷玉自有傳。

景仁,字仲則,武進人。九歲應學使者試,臨試猶蒙被索句。後以母老客游四方,覓升斗為養。硃筠督學安徽,招入幕。上巳修禊,賦詩太白樓。景仁年最少,著白袷立日影中,頃刻成數百言,坐客咸輟筆。時士子試當塗,聞使者高會,畢集樓下,咸從奚童乞白袷少年詩競寫,名大噪。嘗自恨其詩無幽、並豪士氣,遂游京師。高宗四十一年東巡,召試二等。武英殿書簽,例得主簿。陜西巡撫畢沅奇其才,厚貲之,援例為縣丞,銓有日矣,為債家所迫,抱病逾太行,道卒。亮吉持其喪歸,年三十五。著兩當軒集。子乙生,通鄭氏禮,善書,早卒。

倫,字敦五。乾隆中進士,蒼梧縣知縣。著有杜詩鏡詮。

星垣,字叔諾,大學士宮五世孫。乾隆五十年,闢雍禮成,進頌冊,欽取一等一名,選訓導。後官河間縣知縣。有白雲草堂集。

書受,副貢生。葉縣知縣。有教經堂集。

嚴長明,字道甫,江寧人。幼奇慧。年十一,為李紱所賞,告方苞曰:「國器也!」遂從苞受業。尋假館揚州馬氏,盡讀其藏書。高宗二十七年南巡,以諸生獻賦,賜舉人,用內閣中書,入軍機。長明通古今,多智數,工於奏牘,大學士劉統勛最奇其才。戶部奏天下錢糧雜項名目繁多,請並入地丁徵收,長明曰:「今之雜項折徵銀,皆古正供也。若去其名,他日吏忘之,謂其物官所需,必且再徵,是使民重困也。」統勛曰善,乃奏已之。大學士溫福徵大金川,欲長明從行,長明固辭。退,有咎之者,答曰:「是將敗沒,吾奈何從之!」既而溫福果軍潰以死,隨往者皆盡。

長明在軍機七年,幹敏異眾,然亦以是見嫉。其救羅浩源事,人尤喜稱之。浩源,雲南糧道也。分償屬吏汪應繳所虧帑金,有詔逾期即誅。浩源繳不如數,逾期十日,牒請弛限。上下其議,時統勛主試禮部,秋曹無敢任其事者。長明因撾鼓入闈,見統勛,為言汪已捐復,將曳組綬出都,獨坐浩源,義未協,宜仍責汪自繳。統勛曰:「具疏稿乎?」曰:「具。」即振袖出之,辭義明晰。疏入報可,獄遂解。其他事多類此。人有圖其像祀之者。三十六年,擢侍讀。嘗扈蹕木蘭,大雪中失橐扆並所裝物,越日故吏以扆至。問「何以知為吾物」,曰:「軍機官披羊裘者獨君耳。」長明勞而遣之。

後以憂歸,遂不復出。客畢沅所,為定奏詞。又主講廬陽書院。博學強記,所讀書,或舉問,無不能對。為詩文用思周密,和易而當於情。著毛詩地理疏證、五經算術補正、三經三史答問、石經考異、漢金石例、獻徵餘錄等書。

子觀,字子進。嗜學,好金石文字。父乞歸後,築歸求草堂,藏書二萬卷,觀丹黃幾滿。著江寧金石記,錢大昕甚高其品節。

硃筠,字竹君,大興人。乾隆甲戌進士,選庶吉士,授編修。由贊善大考擢侍讀學士,屢分校鄉會試。庚寅,典福建鄉試,辛卯,督安徽學政。

詔求遺書,奏言翰林院藏永樂大典內多古書,請開局校輯。旋奉上諭:「軍機大臣議復硃筠條奏校核永樂大典一節,已派軍機大臣為總裁。又硃筠所奏將永樂大典擇取繕寫,各自為書,及每書校其得失,撮舉大旨,敘於本書卷首之處,即令承辦各員,將各原書詳細檢閱,並書中要旨總敘厓略,呈候裁定;又將來書成,著名四庫全書。」四庫全書自此始。筠又請仿漢熹平、唐開成故事,校正十三經文字,勒石太學。未幾,坐事降編修,充四庫全書纂修官,兼修日下舊聞考。高宗嘗稱筠學問文章殊過人。尋,復督學福建。歸,卒,年五十有三。

筠博聞宏覽,以經學、六書訓士。謂經學本於文字訓詁,周公作爾雅,釋詁居首;保氏教六書,說文僅存。於是敘說文解字刊布之。視學所至,尤以人才經術名義為急務,汲引後進,常若不及。因材施教,士多因以得名,時有硃門弟子之目。好金石文字,謂可佐證經史。諸史百家,皆考訂其是非同異。為文以鄭、孔經義,遷、固史書為質,而參以韓、蘇。詩出入唐、宋,不名一家,並為世重。筠銳然以興起斯文為己任,搜羅文獻,表章風化,一切破崖岸而為之。好客,善飲,談笑窮日夜。酒酣論天下事,自比李元禮、範孟博,激揚清濁,分別邪正,聞者悚然。著有笥河集等。

翁方綱,號覃溪,大興人。乾隆壬申進士,選庶吉士,授編修。擢司業,累至內閣學士。先後典江西、湖北、順天鄉試,督廣東、江西、山東學政。嘉慶元年,預千叟宴。四年,左遷鴻臚寺卿。十二年,重宴鹿鳴,賜三品銜。十九年,再宴恩榮,加二品卿,年八十二矣。又四年,卒。

方綱精研經術,嘗謂考訂之學,以衷於義理為主,論語曰「多聞」、曰「闕疑」、曰「慎言」,三者備而考訂之道盡。時錢載斥戴震為破碎大道,方綱謂:「詁訓名物,豈可目為破碎?考訂訓詁,然後能講義理也;然震謂聖人之道,必由典制名物得之,則不盡然。」

方綱讀群經,有書、禮、論語、孟子附記,並為經義考補正。尤精金石之學,所著兩漢金石記,剖析毫芒,參以說文、正義,考證至精。所為詩,自諸經注疏,以及史傳之考訂,金石文字之爬梳,皆貫徹洋溢其中。論者謂能以學為詩。他著有復初齋全集及禮經目次、蘇詩補注等。[一]

姚鼐,字姬傳,桐城人,刑部尚書文然玄孫。乾隆二十八年進士,選庶吉士,改禮部主事。歷充山東、湖南鄉試考官,會試同考官,所得多知名士。四庫館開,充纂修官。書成,以御史記名,乞養歸。

鼐工為古文。康熙間,侍郎方苞名重一時,同邑劉大櫆繼之。鼐世父範與大櫆善,鼐本所聞於家庭師友間者,益以自得,所為文高簡深古,尤近歐陽修、曾鞏。其論文根極於道德,而探原於經訓。至其淺深之際,有古人所未嘗言。鼐獨抉其微,發其蘊,論者以為辭邁於方,理深於劉。三人皆籍桐城,世傳以為桐城派。

鼐清約寡欲,接人極和藹,無貴賤皆樂與盡懽;而義所不可,則確乎不易其所守。世言學品兼備,推鼐無異詞。嘗仿王士禎五七言古體詩選為今體詩選,論者以為精當云。自告歸後,主講江南紫陽、鍾山書院四十餘年,以誨迪後進為務。嘉慶十五年,重赴鹿鳴,加四品銜。二十年,卒,年八十有五。所著有九經說十七卷,老子、莊子章義,惜抱軒文集二十卷、詩集二十卷,三傳補注三卷,法帖題跋二卷、筆記四卷。

子景衡,舉人,知縣。有雋才,鼐故工書,景衡學其筆法,能亂真。

吳定,字殿麟,歙縣人。舉孝廉方正。與姚鼐相友善,論文嚴於法。鼐每為文示定,定所不可,必盡言,得當乃止。定嘗語陳用光曰:「先生虛懷善取,為文尚如是,其為學可知矣。」著有周易集注十卷,紫石泉山房文集十二卷、詩集六卷。

魯九皋,原名仕驥,字絜非,新城人。嘗從鼐問古文法,又使其甥陳用光及鼐門。乾隆三十六年進士,選山西夏縣,以積勞致疾卒。所著曰山木居士集。

用光,字碩士。嘉慶六年進士,由編修累官禮部侍郎。篤於師友誼,嘗為姚、魯兩師置祭田,以學行重一時。著有太乙舟文集。

當嘉、道間,傳古文法者,有宜興吳德旋、上元梅曾亮諸人,曾亮自有傳。德旋,字仲倫。諸生。以古文鳴。與陽湖惲敬、永福呂璜以文相砥鏃。詩亦高澹絕俗,有初月樓集。

宋大樽,字左彞,仁和人。弱歲,刲股愈母疾,讓產其弟。乾隆三十九年舉人,為國子監助教,以母老引疾歸。豪於飲酒,善鼓琴,時時出游佳山水,助其詩興。其詩由唐人而上溯之,極於古歌謠而止,才力足以相儷。有茗香論詩、學古集、牧牛村舍詩鈔。

同縣錢林,字金粟。嘉慶十三年進士,由編修至侍讀學士,左遷庶子。林熟於本朝名臣言行,及河漕、鹽榷、錢法諸大政。詩亦醖釀於漢、魏、六朝。阮元督學浙江,稱為華實兼茂之士。著文獻徵存錄、玉山草堂詩集。

端木國瑚,青田人。青田故產鶴,國瑚生而清傲似鶴,其大父字之曰鶴田。阮元督學得之,恆言誇示人曰:「此青田一鶴也!」命賦使署定香亭,賦成,一時傳誦。國瑚好學深思,通天文之奧。嘗被召相山陵,敘勞官中書。道光十三年進士,選用知縣。性不耐劇,投牒就原官。著周易指,屬稿二十六年而後成。詩才清麗,有太鶴山人集。又著周易葬說、地理元文,後頗悔之,不輕為人營葬。

吳文溥,字澹川,嘉興貢生。亦以詩名。其為人有韜略,超然不群,能作蘇門長嘯。著南野堂集。

章學誠,字實齋,會稽人。乾隆四十三年進士,官國子監典籍。自少讀書,不甘為章句之學。從山陰劉文蔚、童鈺游,習聞蕺山、南雷之說。熟於明季朝政始末,往往出於正史外,秀水鄭炳文稱其有良史才。繼游硃筠門,筠藏書甚富,因得縱覽群籍,與名流相討論,學益宏富。著《文史通義》、《校讎通義》,推原官禮而有得於向、歆父子之傳。其於古今學術,輒能條別而得其宗旨,立論多前人所未發。嘗與戴震、汪中同客馮廷丞寧紹臺道署,廷丞甚敬禮之。

學誠好辯論,勇於自信。有實齋文集,視唐宋文體,夷然不屑。所修和州、亳州、永清縣諸志,皆得體要,為世所推。

章宗源,字逢之。乾隆五十一年,大興籍舉人,其祖籍亦浙江也。嘗輯錄唐、宋以來亡佚古書,欲撰隋書經籍志考證,積十餘年始成。稿為仇家所焚,僅存史部五卷。

後百有餘年,有姚振宗,字海槎,山陰人。著漢藝文志、隋經籍志考證,能訂宗源之失。又補後漢、三國兩藝文志。目錄之學,卓然大宗。論者謂足紹二章之傳。

而學誠同時有歸安吳蘭庭,字胥石。乾隆三十九年舉人。稽古博聞,多所纂述。嘗以宋吳縝著有五代史記纂誤,因更取薛居正舊史參校,為纂誤補四卷。同邑丁傑邃於經,蘭庭熟於史,一時有「丁經吳史」之目。嘉慶元年,與千叟宴。他所著又有五代史考異、讀通鑒筆記、南霅草堂集。

祁韻士,字鶴皋,壽陽人。乾隆四十三年進士,官編修,擢中允,大考改戶部主事。嘉慶初,以郎中監督寶泉局。局庫虧銅案發,戍伊犁。未幾,赦還。卒於保定書院,年六十五。

韻士幼喜治史,於疆域山川形勝、古人爵里名氏,靡不記覽。弱冠,館靜樂李氏,李藏書十餘楹,多善本,韻士寢饋其中五年,益賅洽。既入翰林,充國史館纂修。時創立蒙古王公表傳,計內扎薩克四十九旗,外扎薩克喀爾喀等二百餘旗,以至西藏及回部糾紛雜亂,皆無文獻可徵據。乃悉發庫貯紅本,尋其端緒,每於灰塵坌積中忽有所得,如獲異聞。各按部落立傳,要以見諸實錄、紅本者為準;又取皇輿全圖以定地界方向。其王公支派源流,則核以理籓院所存世譜,八年而後成書;又別撰籓部要略,以年月編次。蓋傳仿史記,而要略仿通鑒。李兆洛序之,謂如讀邃皇之書,睹鴻濛開闢之規模矣。及戍伊犁,有所纂述,大興徐松續修之,成新疆事略。

韻士又著西域釋地、西陲要略,皆考證古今,簡而能核。外有萬里行程記、己庚編、書史輯要、詩文集。

張穆,字石洲,平定州人。道光中,優貢生。善屬文。歙縣程恩澤見之,驚曰:「東京崔、蔡之匹也!」通訓詁、天算、輿地之學。著蒙古游牧記,用史志體,韻士要略用編年體,論者謂二書足相埒。又以魏書地形志分並建革,一以天平、元象、興和、武定為限,純乎東魏之志。其雍、秦諸州地入西魏者,遂手兌失踳駮不可讀。乃更事排纂,書未成,其友何秋濤為補輯之。又著顧炎武、閻若璩年譜,齋詩、文集。

秋濤,字原船,光澤人。道光二十四年進士,授刑部主事。留心經世之務。以俄羅斯與中國壤地連接,宜有專書資考鏡,始著北徼匯編六卷。後復詳訂圖說,起漢、晉訖道光,增為八十卷。文宗垂覽其書,賜名朔方備乘。召見,擢員外郎、懋勤殿行走,旋以憂去。同治改元,年三十九,卒。又著王會篇箋釋、一鐙精舍甲部槁。刑部奉敕撰律例根源,亦秋濤在官時創槁云。

馮敏昌,字伯術,欽州人。童年補諸生。翁方綱按試廉州,以拔貢選入國學。乾隆四十三年進士,授編修。大考,改戶部主事,調補刑部。性至孝友,聞父喪,一痛嘔血,大雪,徒跣竟日。方綱憂曰:「敏昌萬無生理!」則持其母夫人書促令歸省。及丁內艱,廬墓久,遂不復出。

平生足跡半天下,嘗登岱,題名絕壁;游廬阜,觀瀑布;抵華嶽,攀鐵纖,躋峽。在河陽時,親歷王屋、大行諸山。又以北嶽去孟縣不千里,騎駿馬直造曲陽飛石之巔,窮雁門、長城而返。最後宿南嶽廟,升祝融峰,觀雲海。其悱惻之情,曠逸之抱,一寓於詩。著有小羅浮草堂詩集、孟縣志、華山小志、河陽金石錄。學者稱魚山先生。

其後嶺南以詩名家者,有嘉應宋湘,字煥襄。嘉慶四年進士。以編修典試四川、貴州,出知曲靖府。教屬地種木棉,人稱「宋公布」。署廣南、永昌,皆有績。永昌灣甸土州知州死,遠族景在東謀襲其職,據境專殺自恣,如是者五六年。當事怯,莫敢發。民、夷赴愬,湘請諸鎮帥,不允;乃率僚屬游宴棲賢山,從容賦詩,密約鄉兵乘夜兼行,出不意,擒在東斬之,費銀八千兩,不取償公家,邊隅以靖。終湖北督糧道。詩學少陵,有不易居集。

敏昌同時又有趙希璜,字渭川,長寧人。少讀書羅浮山,與順德黎簡友善。乾隆四十四年舉人。知安陽縣,邑志久未修,希璜聘武億共成之。紀昀推其體例合古法。末附金石錄十二卷,尤精確。希璜工詩,著有四百三十二峰草堂詩鈔。

法式善,字開文,蒙古烏爾濟氏,隸內務府正黃旗。乾隆四十五年進士,授檢討,遷司業。五十年,高宗臨雍,率諸生七十餘人聽講,禮成,賞賚有差。本名運昌,命改今名,國語言「竭力有為」也。由庶子遷侍讀學士,大考降員外郎,阿桂薦補左庶子。性好文,以宏獎風流為己任。顧數奇,官至四品即左遷。其後兩為侍講學士,一以大考改贊善,一坐修書不謹貶庶子,遂乞病歸。

所居後載門北,明李東陽西涯舊址也。構詩龕及梧門書屋,法書名畫盈棟幾,得海內名流詠贈,即投詩龕中。主盟壇坫三十年,論者謂接跡西涯無愧色。著清秘述聞、槐載筆、存素堂詩集。平生於詩所激賞者,舒位、王曇、孫原湘,作三君子詠以張之。然位艷曇狂,惟原湘以才氣寫性靈,能以韻勝,著天真閣集。

原湘,字子瀟,昭文人。嘉慶十年進士。選庶吉士,未仕。

同時江蘇與原湘負才名者,有吳江郭麟,字祥伯。附監生。一眉瑩白如雪,風採超俊。家貧客游,人爭倒屣。詩學李長吉、沈下賢,詞尤清婉。著靈芬館集。嘗病潘昂霄金石例之隘,因據洪氏隸釋為金石例補,又撰詞品十二則,以繼司空表聖之詩品。

惲敬,字子居,陽湖人。幼從舅氏鄭環學,持論能獨出己見。乾隆四十八年舉人,以教習官京師。時同縣莊述祖、有可、張惠言,海鹽陳石麟,桐城王灼集輦下,敬與為友,商榷經義,以古文鳴於時。既而選令富陽,銳欲圖治,不隨群輩俯仰。大吏怒其強項,務裁抑之,令督解黔餉。敬曰:「王事也。」怡然就道。後遭父喪,服闋,選新喻。吏民素橫暴,繩以法,人疑其過猛。已乃進秀異士與論文藝,俗習大變。調知瑞金,有富民進千金求脫罪,峻拒之。關說者以萬金相啗,敬曰:「節士苞苴不逮門,吾豈有遺行耶!」卒論如法。由是廉聲大著。卓異,擢南昌同知。敬為人負氣,所至輒忤上官,以其才高優容之,然忌者遂銜之次骨。最後署吳城同知,坐奸民誣訴隸詐財失察被劾。忌者聞而喜曰:「惲子居大賢,乃以贓敗耶!」

敬既罷官,益肆其力於文。深求前史興壞治亂之故,旁及縱橫、名法、兵農、陰陽家言。會其友惠言歿,於是敬慨然曰:「古文自元、明以來漸失其傳,吾向所以不多為者,有惠言在也。今惠言死,吾安敢不並力治之?」其文蓋出於韓非、李斯,與蘇洵為近。卒,年六十一。著大雲山房稿。其治獄曰子居決事,附集後。

趙懷玉,字億孫,武進人,尚書申喬四世孫。乾隆中召試舉人,授中書。久之,出為青州府同知,以憂歸,終於家。性坦易,工古文辭。嘗自言不敢好名為欺人之事,不敢好奇為欺世之學。惲敬稱其文無有雜言詖義離真反正者。著有生齋文集。

黎簡,字簡民,順德人。十歲能詩。益都李文藻令朝陽,見簡詩,曰:「必傳之作也。」勸令就試。學使李調元得其擬昌黎石鼎聯句,奇賞之。補弟子員,人號之曰黎石鼎。久之,膺選拔。尋丁外艱,遂終於家,足不逾嶺。海內名流,欽其高節。袁枚負盛名,游羅浮,邀與相見,謝不往也。著五百四峰草堂詩文鈔。所與交同邑張錦芳、黃丹書,番禺呂堅皆以詩名。

錦芳,字粲夫。乾隆中進士,官編修。通說文,喜金石文字。弟錦麟,字瑞夫。舉人。兄弟並為翁方綱所器異。錦麟以賦「碧天如水雁初飛」句得名,時呼張碧天。早卒。錦芳著逃虛閣詩鈔,與欽州馮敏昌、同邑胡亦常稱「嶺南三子」。

丹書,字廷授。亦以詩受知調元。貢優行,事親孝,居喪能盡哀。後舉於鄉。至都,朝貴爭延之,辭不就。嘗曰:「貧與富交則損名,賤與貴交則損節。」晚官教諭,兼工書畫。著鴻雪齋詩鈔。

堅,字介卿。歲貢生,窮老不遇。著遲刪集。

亦常,字同謙。舉人。落第南歸,與戴震同舟,至富春江乃別。舟中手寫震所著書,謀刊之。多敢瓜果解渴,得胃寒疾,抵家卒。有賜書樓集。

張士元,字翰宣,震澤人。工古文辭,師法歸有光。歲正,陳其集幾上,北面拜之。又用歸氏評點史記法,上推之左氏,下逮韓、歐,無不合者。乾隆五十三年舉人,久不第,留京師館董誥第八年。誥主會試,欲令士元出門下,不能得也。姚文田督學江南,士元與有舊,戒諸子勿應試。年老,銓教諭,以耳聵謝不就。曰:「國家設學校,使師弟子相從講學,豈漫以廩祿拯寒生哉?」乃歸老爛谿之上,撰述自娛。學者稱鱸江先生。

性澹泊寡交,獨與王芑孫、秦瀛、陳用光以學問相切劘。姚鼐見其文,亦擬之震川。卒,年七十。著嘉樹山房集。

同邑張海珊,字越來;張履,字淵甫:皆舉人。海珊道光元年鄉試解首,榜發,已前卒。其論學以宋賢為歸,又恥迂儒寡效,自農田、河渠、兵制、天下形勢所在,及漕糧利弊,悉心究討。三吳亢旱港涸,一日北風大作,水入,糾眾築堤儲之,歲以有秋。著小安樂窩集、喪禮問答、火攻秘錄。

履,海珊門人也。傳海珊之學,尤精三禮。其議禮之文,皆犁然有當,非徒習訓詁名物者。官句容訓導。著積石山房集。


\end{pinyinscope}