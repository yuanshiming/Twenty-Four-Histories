\article{列傳六十六}

\begin{pinyinscope}
楊方興硃之錫崔維雅靳輔陳潢宋文運董訥熊一瀟

於成龍孫在豐開音布張鵬翮

楊方興,字浡然,漢軍鑲白旗人。初為廣寧諸生。天命七年,太祖取廣寧,方興來歸。太宗命直內院,與修太祖實錄。崇德元年,試中舉人,授牛錄額真銜,擢內秘書院學士。性嗜酒,嘗醉後犯蹕,論死,上貰之,命斷酒。

順治元年,從入關。七月,授河道總督。李自成決河灌開封,其後屢決屢塞,賊勢浸張,土寇群起,兩岸防守久廢。伏秋汛發,北岸小宋口、曹家寨堤潰,河水漫曹、單、金鄉、魚臺四縣,自蘭陽入運河,田產盡沒。方興至官,遣兵捕治土寇,掃穴擒渠,乃疏請修築。二年七月,河決流通集,分兩道入運河,運河受河水澱濁淤塞,下流徐、邳、淮、揚亦多沖決。方興以防護無功自劾,上諭以殫力河防,不必引咎。旋疏薦補管河道方大猷等。四年,流通集決口將合,河下注湍激,又決汶上入獨山湖。方興請修築通濟閘上下堤岸,並淮安東北蘇淤、馬羅等堤,又築江都、高郵諸石堤,流通集合口。進兵部尚書銜。

七年,加太子少保。八月,河決荊隆口,南岸出單家寨,北岸出硃源寨。南岸先合,河全注北岸,張秋以下堤盡潰,自大清河東入海。方興用大猷議,於上游築長縷堤遏其勢,復築小長堤塞決口,期半年蕆事。九年,方興復乞休,不許。大猷擢江南按察使,方興請以新銜管河務。九年,荊隆口工竟,方興疏言:「清口,淮、黃交匯,黃強淮弱,歲需疏濬。請於清江、通濟二閘適中處修復福興閘,啟一閉二,以時蓄洩。」從之。

給事中許作梅,御史楊世學、陳棐交章請勘九河故道,導河北流入海。方興言:「河古今同患,而治河古今異宜。宋以前治河,但令赴海有路,可南亦可北。元、明迄我清,東南漕運,自清口迄董家口二百餘里,藉河為轉輸,河可南必不可北。若欲尋禹舊跡,導河北行,無論漕運不通,恐決出之水東西奔蕩,不可收拾。勢須別築數千里長堤,較之增卑培薄,難易顯然。且河挾沙以行,束之為一,則水急沙流;播之為九,則水緩沙壅。數年後河仍他徙,何以濟運?臣愚以為河不能無決,決而不築,司河者之罪;河不能無淤,淤而不濬,亦司河者之罪。若欲保其不決不淤,誰敢任之?請敕下廷議,定畫一之規,屏二三之說,俾有所遵守。」疏入,上嘉納焉。

十年,河決大王廟,距硃源寨口不遠。給事中周體觀劾方興治河罔效,方興疏辨,因請罷斥,溫詔慰留。十一年,給事中林起龍復劾方興侵蝕工需,累民捐費至六十餘萬;並劾大猷等奸貪不法。上解方興任,命入都質對,起龍以誣譴,方興復任。既,直隸總督李廕祖復劾大猷貪婪誤工,方興亦劾大猷,上以其不先舉發,切責之。給事中董篤行又劾方興徇庇,降級留任。

十四年,乞休,上念其勞,以原官加太子太保致仕。方興還京師,所居僅蔽風雨,布衣蔬食,四壁蕭然。康熙四年,卒,賜祭葬。

硃之錫,字孟九,浙江義烏人。順治三年進士,改庶吉士,授編修。十一年七月,擢弘文院侍讀學士,四遷至吏部侍郎。十四年,楊方興乞休,上特擢之錫,以兵部尚書銜,總督河道,駐濟寧。十五年十月,河決山陽柴溝,建義、馬邏諸堤並溢。之錫馳赴清江浦築戧堤,塞決口。宿遷董家口為沙所淤,就舊渠迤東別開河四百丈通運道。十六年,條上治河諸事,言:「河南歲修夫役,近屢經奏減,宜存舊額。明制,淮工兼用民修,宜復舊例。揚屬運道與高、寶諸湖相通,淮屬運道為黃、淮交會,舊有各堤閘,宜擇要修葺。應用柳料,宜令瀕河州縣預為籌備。奸豪包占夫役,賣富僉貧,工需各物,私弊百出,宜責司、道、府、查報,徇隱者以溺職論。額設水夫,陰雨不赴工,所扣工食,謂之曠盡,宜令管河道嚴覈。河員升調降用,宜令候代始行離任。河員有專責,不宜別有差委。歲終察覈舉劾,並宜復舊例。」皆下部議行。之錫丁母憂,命在任守制,疏請歸葬,優詔給假治喪。十七年,還任。以捐金賑淮、揚、徐三府災,加太子少保。

康熙元年,河決原武、祥符、蘭陽縣境,東溢曹縣,復決石香爐村。之錫檄濟寧道方兆及董曹縣役,而赴河南督塞西閻寨、單家寨、時利驛、蔡家樓、策家寨諸決口。四年二月,疏言:「南旺為運河之脊,北至臨清,南至臺莊,四十餘閘,全賴啟閉得宜。瀕河春常少雨,伏秋雨多,東省久旱,山泉小者多枯,大者已弱。若官船經閘,應閉者強之使開,洩水下注,則重運之在上者阻;應開者強之使閉,留水待船,則重運之在下者又阻。乞飭各遵例禁。」得旨,非奉極要差遣,擅行啟閉者,準參奏。八月,疏言:「部議停差北河、中河、南河、南旺、夏鎮、通惠諸分司,歸並地方官。臣維河勢變幻,工料紛繁,天時不齊,非水則旱,或綢繆幾先,或補葺事後,或張皇於風雨倉遽之際,或調劑於左右方圓之間。北河所轄三千餘里,其間三十餘閘;中河所轄黃、運兩河,董口尤運道咽喉,清黃交接,濁流易灌;南河所轄在淮、黃、江、湖之間,相距窵遠;南旺、泉源三百餘處,近者或出道隅,遠者偏藏僻壤;夏鎮地屬兩省,鑿石通漕,形勢陡絕,節宣閘座,尤費經營;通惠浮沙易淺,峻水易沖,塞決之役,歲歲有之。若雲歸並府佐,則職微權輕,上下掣肘。至於地方監司,責以終年累月奔馳駐守,揆之事勢,萬萬不能。分司與各道界壤迥不相同,應合而分:一閘座也,上流以為應閉,下流以為應開;一額夫也,在此則欲求多,在彼又復患少。不但紛競日多,必致牽制誤事。應請仍循舊制。」得旨允行。五年二月,卒。

直隸山東河南總督硃昌祚疏言:「之錫治河十載,綢繆旱溢,則盡瘁昕宵;疏濬堤渠,則馳驅南北。受事之初,河庫貯銀十餘萬;頻年撙節,現今貯庫四十六萬有奇。覈其官守,可謂公忠。及至積勞攖疾,以河事孔亟,不敢請告。北往臨清,南至邳、宿,夙病日增,遂以不起。年止四十有四,未有子嗣。籥請恩恤,賜祭葬。」徐、兗、淮、揚間頌之錫惠政,相傳死為河神。十二年,河道總督王光裕請錫封號,部議不行。乾隆四十五年,高宗南巡視河工,始允大學士阿桂等請,封助順永寧侯,春秋祠祭。嗣加號曰「佑安」,民稱之曰硃大王。

崔維雅,字大醇,直隸大名人。順治三年舉人,授濬縣教諭,遷河南儀封知縣。儀封瀕河,歲苦泛濫,北岸三家莊當水沖,十四年,水勢北注,岸崩五里餘。維雅於上游故流疏使東行,北岸得安。復與塞封丘大王廟決口,之錫疏薦,擢開封南河同知。

康熙元年五月,曹縣石香爐村河決,士民求速塞,維雅持不可。工將成復潰,至冬乃塞,如維雅言。遷浙江寧波知府,光裕疏薦,擢河南河道副使。時沿河千餘里,險工迭出,維雅常預為之備,得無事。陽武潭口寺堤直河沖,水勢迅急,下埽輒蟄。維雅預於上流疏引河,埽定,堤得固。虞城距河堤僅數里,堤沒入河,北岸引河沖刷不利。維雅預迎河溜挑濬,及秋水歸新河,舊河為平陸。桃源七里溝河屢塞屢決,光裕檄維雅往勘,維雅言引河淺狹,流緩沙停,激蕩無力,宜令河頭加寬闊,使足翕受全河;又待河水突漲,乃使開放,建瓴直下。又言下游數十里已成平陸,而引河僅百丈,節短勢蹙,力不能刷淤,當接挑二百丈闊,損十之八而深半之。又言開放當在河頭西北,留近埽五丈勿開,則河流入口有倒瀉之勢,埽亦迎流下。光裕悉用其議。復遷河南按察使,湖南、廣西布政使,內召為大理寺卿。卒。

維雅治河主疏導引河,使水有所歸,故屢有功而後不為患。當靳輔興大工時,維雅奏上所著河防芻議、兩河治略,並詆諆輔所行諸法,列二十四事難之。輔疏辨,謂維雅說不可行,寢其議。

靳輔,字紫垣,漢軍鑲黃旗人。順治九年,以官學生考授國史館編修,改內閣中書,遷兵部員外郎。康熙初,自郎中四遷內閣學士。十年,授安徽巡撫。疏請行溝田法,以十畝為一鞬,二十鞬為一溝。溝土累為道,道高溝低,澇則洩水,旱以灌田。會三籓亂起,不果行。部議裁驛站經費,輔疏請禁差員橫索、騷擾驛遞,歲終節存驛站、摃腳等項二十四萬有奇。上獎輔實心任事,加兵部尚書銜。

十六年,授河道總督。時河道久不治,歸仁堤、王家營、邢家口、古溝、翟家壩等處先後潰溢,高家堰決三十餘處,淮水全入運河,黃水逆上至清水潭,浸淫四出。碭山以東兩岸決口數十處,下河七州縣淹為大澤,清口涸為陸地。輔到官,周度形勢,博採輿論,為八疏同日上之:首議疏下流,自清江浦至雲梯關,於河身兩旁離水三丈,各挑引河一道,俟黃、淮下注,新舊河合為一,即以所挑土築兩岸大堤,南始白洋河,北始清河縣,並東至雲梯關。雲梯關至海口百里,近海二十里,潮大土濕,不能施工;餘八十里亦宜量加疏濬,築堤以束之,限二百日畢工,日用夫十二萬三千有奇。次議治上流淤墊,洪澤湖下流自高家堰西至清口,為全淮會黃之所。當於小河兩旁離水二十丈,各挑引河一道,分頭沖洗。次議培修七里墩、武家墩、高家墩、高良澗至周橋閘臨湖殘缺堤岸,下築坦坡,使水至平漫而上,順縮而下,不至怒激崩沖。堤一尺、坦坡五尺,夯杵堅實,種草其上。次議塞黃、淮各處決口,例用埽,費鉅且不耐久;求築土御水之法,宜密下排椿,多加板纜,用蒲包裹土,麻繩縛而填之,費省而工固。次議閉通濟閘壩,濬清口至清水潭運河二百三十里,以所挑之土傾東西兩堤之外,西是築為坦坡,東堤加培堅厚,次議規畫經費,都計需銀二百十四萬八千有奇。宜令直隸、江南、浙江、山東、江西、湖北各州縣預徵康熙二十年錢糧十之一,約二百萬。工成後,令淮、揚被水田畝納三錢至一錢;運河經過,商貨米豆石納二分,他貨物斤四分;並開武生納監事例,如數補還。次議裁並冗員,明定職守,並嚴河工處分,諱決視諱盜;兼請調用官吏,工成,與原屬河官吏並得優敘。次議工竣後,設河兵守堤,裡設兵六名至二名,都計五千八百六十名。疏入,下廷議,以方軍興,復舉大工,役夫每日至十二萬餘,召募擾民,應先擇要修築。上命輔熟籌。

十七年,輔疏言:「以驢運土,可減募夫之半;初擬二百日畢工,今改為四百日,又可減募夫之半。」河工故事,大堤謂之「遙堤」,堤內復為堤逼水,謂之「縷堤」,兩堤間為橫堤,謂之「格堤」。輔疏請就原估土方加築縷堤,有餘量增格堤,南自白洋河,北自清河,上至徐州,視此興築。餘並如前議。疏入,復下廷議,允行。

上諭以治河大事,當動正項錢糧。輔疏言:「前議黃河兩岸分築遙、縷二是,勘有舊堤貼近河身,擬作為縷堤,其外更築遙堤。前議用驢運土,今議改車運。前議離堤三十丈內不許取土,今因宿遷、桃源等縣人弱工多,改令二十丈外取土。前議河身兩旁各挑引河一道,今以工費浩繁,除清河北岸淺工必須挑濬。餘俱用鐵掃帚濬深河底。」下部議,從之。

是歲吳三桂死,上趣諸將帥進兵,輔欲節帑佐軍,又以興工後需費溢出原估,均頗改前議,先開清口引河四道,塞高家堰、王家岡、武家墩諸決口,築堤束水。如所議施行。顧下流未大治,伏秋盛漲,水溢出堤上,復決碭山石將軍廟、蕭縣九里溝。輔乃議設減水壩,於蕭、碭、宿遷、桃源、清河諸縣河南北兩岸為壩十三,壩七洞,水盛藉以宣洩。輔復察清口淮、黃交會,黃漲侵灌運河,乃自新莊閘西南開新河至太平壩;又自文華寺開新河至七里閘,復折向西南,亦至太平壩;改以七里閘為運口,由武家墩爛泥淺轉入黃河。運口距黃、淮交會處約十里,自此無淤墊之患。疏報,並議行。輔勘清水潭決口屢塞屢沖,乃棄深就淺,築東西長堤二道,並挑新河八百四十丈,疏積水。山陽、高郵等七州縣民田,至是皆出水可耕。

十八年,輔疏報,並請名新河曰永安河,報聞。翟家壩淮河決口成支河九道,輔飭淮揚道副使劉國靖等督堵塞,至是工竟,輔詣勘疏報,並言:「山陽、寶應、高郵、江都四州縣瀦水諸湖,逐漸涸出。臣今廣為招墾,俾增賦足民,上下均利。」屯田之議自此起。

漕船自七里閘出口,行駱馬湖達窯灣。夏秋盛漲,冬春水涸,重運多阻。輔議濬湖旁皁河故道,上接泇河通運。疏入,下廷議,上問諸臣意若何,左都御史魏象樞曰:「輔請大修黃河,上發帑二百五十一萬,計一勞永逸。前奏堤壩已築十之七,今又欲別開河道,所謂一勞永逸者安在?臣等慮漕運有阻,故議從其請。」上曰:「象樞言良是。河雖開,必上流浩瀚,方免淤滯。今雨少水涸,恐未必有濟。即已成諸工,亦以旱易修,豈得恃為永固耶?」十九年五月,輔丁憂,命在任守制。秋,河復決,輔疏請處分,上趣輔修築。二十年三月,輔疏言:「臣前請大修黃河,限三年水歸故道。今限滿,水未歸故道,請處分。」下部議,當奪官,上命戴罪督修。

二十一年五月,上遣尚書伊桑阿、侍郎宋文運、給事中王曰溫、御史伊喇喀勘工。候補布政使崔維雅奏上所著書,議盡罷輔所行減水壩諸法,大興工,日役夫四十萬,築堤以十二丈為率。上命從伊桑阿等往與輔議之。伊桑阿等遍勘諸工,至徐州,令輔與維雅議,輔疏言:「河道全局已成十八九。蕭家渡雖有決口,而海口大闢,下流疏通,腹心之害已除。斷不宜有所更張,隳成功,釀後患。」伊桑阿等還京師,下廷議,工部尚書薩穆哈等請以蕭家渡決口責輔賠修,上以賠修非輔所能任,未允;又議維雅條奏,伊桑阿請召輔詢之。十一月,輔入對,言蕭家渡工來歲正月當竟,維雅所議日用夫四十萬、築堤以十二丈為率,皆不可行。維雅議乃寢。上命塞決口,仍動正項錢糧。二十二年四月,輔疏報蕭家渡合龍,河歸故道,大溜直下,七里溝等四十餘處險汛日加,並天妃壩、王公堤及運河閘座,均應修築。別疏請飭河南巡撫修築開封、歸德兩府境河堤,防上流疏失。上均如所請。十二月,命復輔官。

二十三年十月,上南巡,閱河北岸諸工,諭輔曰:「蕭家渡堤壩當培薄增卑,隨時修築。減水壩原用以洩水,遇泛溢橫流,安知今日減水壩不為他年之決口?且減水旁流,浸灌民田,朕心深不忍。當籌畫措置。」上見堤夫作苦,駐轡慰勞久之,諭輔戒官役侵蝕工食。復視天妃閘,諭輔宜改草壩,並另設七里、太平二閘殺水勢。舟過高郵,見田廬在水中,惻然愍念。遣尚書伊桑阿、薩穆哈察視海口。還蹕,復閱高家堰,至清口,閱黃河南岸諸工,諭輔運口當添建閘座,防黃水倒灌;復召輔入行宮慰諭,書閱河堤詩賜之。

輔以上念減水淹民,因議於宿遷、桃源、清河三縣黃河北岸堤內開新河,謂之中河。於清河西仲家莊建閘,引攔馬河減水壩所洩水入中河。漕船初出清口浮於河,至張莊運口,中河成,得自清口截流,逕渡北岸,度仲家莊閘,免黃河一百八十里之險。伊桑阿等還奏,議疏濬車路、串場諸河至白駒、丁溪、草堰諸口,引高郵等處減水壩所洩水入海。上命安徽按察使於成龍董其事,仍受輔節制,奏事由輔疏報。

二十四年正月,輔疏請徐州迤上毛城鋪、王家山諸處增建減水閘,下廷議。上諭減水閘益河工無益百姓,不可不熟計,命遣官與輔詳議,若分水不致多損民田,即令興工。九月,輔疏報赴河南勘黃河兩岸,請築考城、儀封、封丘、滎澤堤埽,下部議行。成龍議疏海口洩積水,輔謂下河地卑於海五尺,疏海口引潮內侵,害滋大;議自高郵東車邏鎮築堤,歷興化白駒場,束所洩水入海,堤內涸出田畝,丈量還民,餘招民屯墾,取田價償工費。疏聞,上謂取田價恐累民,未即許。

尋召輔、成龍馳驛詣京師廷議,成龍議開海口故道,輔仍主築長堤高一丈五尺,束水敵海潮。大學士、九卿從輔議,通政使參議成其範、給事中王又旦、御史錢鎯從成龍議,議不決。上命宣問下河諸州縣人官京師者,侍讀寶應喬萊等乃言:「從成龍議,工易成,百姓有利無害;從輔議,工難成,百姓田廬墳墓多傷損,且堤高一丈五尺,束水至一丈,高於民居,伏秋潰決,為害不可勝言。」上頗右成龍,遣尚書薩穆哈、學士穆稱額詣淮安會漕督徐旭齡、巡撫湯斌詳勘。二十五年正月,薩穆哈等還奏,謂民間皆言濬海口無益。尋授成龍直隸巡撫,罷濬海口議。四月,召斌為尚書,入對,上復舉其事以問,斌言濬海口必有益於民。上責薩穆哈、穆稱額還京時不以實奏,奪官。召大學士九卿及萊等定議濬海口,發帑二十萬,命侍郎孫在豐董其役。

工部劾輔治河已九年,無成功。上曰:「河務甚難,而輔易視之。若遽議處,後任者益難為力,今姑寬之,仍責令督修。」二十六年,輔疏言:「運堤減水以下河為壑,東即大海,濬海口似可紓水患;惟泰州安豐、東臺、鹽城諸縣地勢甚卑,形如釜底,若止就此挑濬,徒增其深。淮流甚漲,高家堰洩水洶湧而來,仍不能救民田之淹沒。臣以為杜患於流,不若杜患於源。高家堰堤外直東為下河,東北為清口,當自翟家壩起至高家堰築重堤萬六千丈,束減水北出清口,則洪澤湖不復東淹下河。下河十餘萬頃皆成沃產,而高、寶諸湖涸出田畝,可招民屯墾,以裕河庫。」上使以輔疏示成龍,成龍仍言下河宜開,重堤不宜築。上遣尚書佛倫,侍郎熊一瀟,給事中達奇納、趙吉士與總督董訥,總漕慕天顏會勘。佛倫等皆欲用輔議,天顏、在豐與相左。佛倫等還奏,下廷議,會太皇太后崩,議未上。

二十七年春,給事中劉楷,御史郭琇、陸祖修交章論輔,琇辭連輔幕客陳潢,祖修請罷輔,至以舜殛鯀為比;天顏、在豐亦疏論屯田累民,及輔阻撓開濬下河狀。琇旋劾大學士明珠等,語復及輔。輔入覲,亦疏訐成龍、天顏、在豐等朋比謀陷害。上曰:「輔為總河,挑河築堤,漕運無誤,不可謂無功;但屯田、下河二事,亦難逃罪。近因被劾,論其過者甚多。人窮則呼天,輔若不陳辨朕前,復何所控告耶?」三月,上禦乾清門,召輔與成龍、琇等廷辨,輔、成龍各持所見不相下。琇言輔屯田害民,輔言屬吏奉行不善致民怨,因引咎,坐罷,以王新命代,佛倫、訥、在豐、達奇納皆左遷,天顏、吉士並奪官,陳潢亦坐譴。

時中河工初竣,上遣學士開音布、侍衛馬武往勘,還奏中河商賈舟楫不絕。上諭廷臣曰:「前者於成龍奏河道為靳輔所壞,今開音布等還奏,數年未嘗沖決,漕運亦不誤。若謂輔治河全無所裨,微特輔不服,即朕亦不愜。」因遣尚書張玉書、圖納,左都御史馬齊,侍郎成其範、徐廷璽閱工,遍察輔所繕治,孰為當改,孰為不當改,詳勘具奏。玉書等還言河身漸次刷深,黃水汎溜入海,兩岸閘壩有應循舊者,有應移改者,多守輔舊規。

十一月,上遣尚書蘇赫等閱通州運河,命輔偕往,請於沙河建閘蓄水,通州下流築堤束水,從之。二十八年正月,上南巡閱河,輔扈行。閱中河,上慮逼近黃河,水漲堤潰;輔對若加築遙堤即無患。還京師,諭獎輔所繕治河深堤固,命還舊秩。二十九年,漕運總督董訥以北運河水淺,擬盡引南旺河水北流;倉場侍郎開音布復疏請濬北運河。上諮輔,言南旺河水盡北流,南河必水淺,惟從北河兩旁下埽束水,自可濟運。上命偕開音布董理。

三十一年,王新命坐事罷,上曰:「朕聽政後,以三籓及河務、漕運為三大事,書宮中柱上。河務不得其人,必誤漕運。及輔未甚老而用之,亦得紓數年之慮。」令仍為河道總督,輔以衰弱辭,命順天府丞徐廷璽為協理。會陜西西安、鳳翔災,上命留江北漕糧二十萬石,自黃河運蒲州。輔疏言水道止可至孟津,親詣督運,上嘉之。輔疏請就高家堰運料小河培堤使高廣,中河加築遙堤,並增建四閘,堵塞張莊舊運口,皆前此繕治所未竟者。別疏請復陳潢官,並起用熊一瀟、達奇納、趙吉士。輔病劇,再疏乞解任,命內大臣明珠往視,傳諭調治。十一月,卒,賜祭葬,謚文襄。三十五年,允江南士民請,建祠河干。四十六年,追贈太子太保,予拜他喇布勒哈番世職。雍正五年,復加工部尚書。

子治豫,襲職。世宗以其侍父在官,知河務,命自副參領加工部侍郎銜,協理江南河工。

陳潢,字天一,浙江錢塘人。負才久不遇,過邯鄲呂祖祠,題詩壁間,語豪邁。輔見而異焉,蹤跡得之,引為幕客,甚相得。凡輔所建白,多自潢發之。康熙二十三年,上巡河,問輔:「孰為汝佐?」以潢對。二十六年,輔疏言潢十年佐治勤勞,下部議,授潢僉事道銜。二十七年,郭琇劾輔,辭連潢。輔罷,潢削職銜,逮京師,未入獄,以病卒。輔復起,疏請復潢官,部議以潢已卒,寢其奏。

潢佐治河,主順河性而利導之,有所患必推其致患之由;工主覈實,料主豫備,而估計不當過省,省則速敗,所費較所省尤大;慎固堤防,主潘季馴束水刷沙之說,尤以減水壩為要務;有潰決,先固兩旁,不使日擴,乃修復故道,而疏引河以注之;河流今昔形勢不同,無一勞永逸之策,在時時謹小慎微,而尤重在河員之久任。張靄生採潢所論,次為治河述言十二篇。高宗以靄生河圖能得真源,命採其書入四庫,與輔治河奏績並列。

宋文運,字開之,直隸南宮人。順治六年進士,授山東滋陽知縣,行取刑部主事。再遷吏部郎中,掌選政,清直守正。以魏象樞薦,擢鴻臚寺少卿,累擢刑部侍郎。命佐伊桑阿行河,上特諭之曰:「爾有所見,當堅持詳議,毋以伊桑阿為尚書而阿其意也。」以病乞休,加太子少保,致仕。卒,謚端愨。久之,上猶謂文選司事要,文運操守聲名,無能及之者。

董訥,字茲重,山東平原人。康熙六年一甲三名進士,授編修。累擢至江南總督。為政持大體,有惠於民。左遷去,江南民為立生祠。二十八年,上南巡,民執香跪訥生祠前,求復官訥江南。上還蹕,笑謂訥曰:「汝官江南惠及民,民為汝建小廟。」旋以侍讀學士復出為漕運總督。卒。

熊一瀟,字蔚懷,江西南昌人。康熙三年進士,改庶吉士,授浙江道監察御史。請罷投誠武官改授文官例,並議裁並各關,皆下部議行。累官工部尚書,坐奪官。以輔遺疏薦,起太常寺卿,復至工部尚書。致仕,卒。孫學鵬,進士,官廣東巡撫。

於成龍,字振甲,漢軍鑲黃旗人。康熙七年,自廕生授直隸樂亭知縣。八年,署灤州知州。以逸囚當降調,樂亭民列善政,兩叩閽籥留,下巡撫金世德勘實,得復任。十三年,以緝盜逾限未獲,又當降調,世德疏請留,上特許之。十八年,遷通州知州。

二十年,直隸巡撫於成龍遷兩江總督,疏薦可大用;會江寧府缺員,疏請敕廷臣推清操久著與相類者,上即以命成龍。二十三年,上南巡至江寧,嘉成龍廉潔,親書手卷賜之。超擢安徽按察使。上還京師,賜其父參領得水貂裘,並諭八旗諸大臣有子弟為外吏者,各貽書訓勉,視得水之教成龍。上以江南下河諸州縣久被水,敕議疏濬,命成龍分理,仍聽河道總督靳輔節制。輔請於上流築堤束水;成龍擬疏海口,濬下河水道,持異議。上遣尚書薩穆哈、學士穆稱額往諮於民,薩穆哈等還奏,言眾謂濬海無益,乃命緩興工。

二十五年二月,授成龍直隸巡撫。入對,上問:「治畿輔利弊應興革者宜何先?」成龍對:「弭盜為先。奸宄倚旗下為淵藪,有司莫敢誰何,臣當執法治之。」瀕行,賜白金千、表裏二十端。上官,疏言:「弭盜當力行保甲,旗下莊屯不屬於州縣,本旗統領遠在京師,僅有撥什庫在屯,未能約束。應令旗人與民戶同編保甲,撥什庫、鄉長互相稽察,盜發,無問所劫為旗為民,協力救護。得盜,賞;藏盜、縱盜,罰。」又疏言:「燕山六衛,所轄遼闊,與州縣不相統屬,盜發止責汛弁捕治,而衛官置不問。請以衛地屬所近州縣同編保甲,並於通州、盧溝橋、黃村、沙河各設捕盜同知,守備以下分汛、墩、臺及旗下莊屯,悉歸稽察。」並下部議行。先後捕治旗丁沈顛、太監張進升及大盜司九、張破樓子等,置於法。二十六年,上獎成龍廉能,加太子少保。幸霸州,成龍朝行在,賜白金千、馬具黃鞍轡。湖廣巡撫張汧以貪被劾,命與副都御史開音布、山西巡撫馬齊往按,得實,論如律。

初,成龍分理下河,未興工而罷。上又以湯斌言,復命濬治,以侍郎孫在豐董其役。輔仍主重堤束水,並議開中河,疏攔馬河減水壩所洩水。上命學士禪布以疏示成龍,成龍力主濬下河,罷築重堤,並謂中河雖開無益。輔詣京師,疏言在豐及總漕慕天顏附和成龍,朋謀陷害。成龍自湖廣還,上命諸臣廷辨之。輔言濬海口慮倒灌,成龍言高家堰築堤,縱上流水不來,而秋雨時至,天長、六合諸水洩歸何處,故海口仍當濬。上罷輔,代以王新命。及中河工竟,遣學士開音布、侍衛馬武閱視,還奏天顏令漕船退出中河。上逮問天顏,天顏發成龍私書,囑毋附輔。下廷臣議,削太子少保,降調,命留任。二十九年,遷左都御史,兼鑲黃旗漢軍都統。

三十一年,新命罷,輔復為河督,旋卒,上以命成龍。輔領帑購柳束,工部駁減,成龍覈無虛冒。輔築高家堰重堤,募夫遠方,預給銀安家,工中止,未扣抵。新命題銷,格部議,成龍復以請,上並與豁免。三十三年,召詣京師,疏言運河自通州至嶧縣,黃河自滎澤至碭山,堤卑薄者皆宜加築高厚,並高家堰諸處改石工,毛城鋪諸處疏引河,及清江浦迤下並江都、高郵諸堤工,策大舉修治。別疏請設道員以下各官,又計工費,請開捐例,減成核收;並推廣休革各員,上至布政使,皆得捐復。上召成龍入,問:「開捐例得無累民?」成龍言:「無累。」請益力,上廷折之,成龍乃請罪。上因問:「爾嘗短靳輔,謂減水壩不宜開,今果何如?」成龍曰:「臣彼時妄言,今亦視輔而行。」廷臣議成龍懷私妄奏,當奪官,上命留任。仍興舉簡要各工,乃請先將高家堰土堤改築石工。

三十四年,命復官。旋丁父憂,還京師,以董安國代。上親征噶爾丹,再出塞,命成龍以左都御史銜督餉,噶爾丹竄死,予拜他喇布勒哈番世職。三十七年,命以總督銜管直隸巡撫,請修永清、固安渾河堤,並加以濬治,上為改河名曰永定。旋疏請設南北岸分司。董安國罷,復授河道總督。三十八年,上南巡,臨閱高家堰、歸仁堤諸處,諭以增築疏濬諸事。尋以病乞假,命在任調治,遣醫往視。三十九年,卒,賜祭葬,謚襄勤。

孫在豐,字屺瞻,浙江德清人。康熙九年一甲二名進士,授編修。直起居注,充日講官,進講屢稱旨。累遷工部侍郎,仍兼翰林院學士。二十六年,命率郎中鄂素等赴淮、揚濬海口,鑄監修下河工部印授之。在豐疏言開新不如循舊,築高不如就低,迤遠不如取近。施工以岡門鎮為最先,次白駒場,次丁溪場,次草堰。上悉從之,並以在豐請,令輔閉高家堰及高郵諸減水壩。輔仍主築堤束水。上令輔會總督董訥、總漕慕天顏及在豐集議,遂會疏用輔議。在豐監修海口岡門鎮、白駒工已畢,丁溪、草堰工俱停。上以諮成龍,成龍言:「上遣在豐監修下河,萬民歡頌。今岡門、白駒諸工將竣,而輔又以為無益,欲於高家堰等處築堤。在豐先經履勘,始行興工;若果無益,何待開濬年餘又會議請停?此實臣所不能解也。」二十七年,在豐疏劾輔阻撓下河,輔亦劾在豐與天顏結姻,附和成龍。下廷臣議,輔罷,成龍坐鐫秩,責在豐前後言不仇,降調。上命仍以翰林官用,俄授侍讀學士。二十八年,遷內閣學士。卒。

開音布,西林覺羅氏,滿洲正白旗人。自筆帖式授內閣中書,累遷至左副都御史。康熙二十六年,偕成龍按湖廣巡撫張汧,論罪如律。二十七年,擢戶部侍郎,命監理高郵、寶應下河工程。二十八年,上南巡,成龍扈行,命與侍郎徐廷璽閱視下河,還奏丁溪至白駒,水三道入海,上流馮家壩引河當仍開濬,餘工悉可停。乃召開音布還,授正白旗滿洲副都統。尋擢步軍統領,遷兵部尚書,授鑲白旗滿洲都統。三十八年,命專管步軍統領。四十一年,卒,謚肅敏。

張鵬翮,字運青,四川遂寧人。康熙九年進士,選庶吉士。改刑部主事,累遷禮部郎中。十九年,授江南蘇州知府,丁母憂。除山東兗州知府,舉卓異,擢河東鹽運使,內遷通政司參議,轉兵部督捕副理事官。從內大臣索額圖等勘定俄羅斯界,還擢大理寺少卿。二十八年,授浙江巡撫。疏言紳民原畝捐穀四合,力不能者聽。旋以杭州、嘉興等府秋收歉薄,請暫免輸穀。上曰:「昨歲浙江被災,循例蠲賦,並豁免錢糧,豈可強令捐輸?鵬翮原題力不能者聽,自相矛盾。」下部議,奪官,上寬之。尋授兵部侍郎,督江南學政。三十六年,遷左都御史。三十七年,遷刑部尚書,授江南江西總督。三十八年,上南巡,命鵬翮扈從入京,賜朝服、鞍馬、弓矢。

初,陜西巡撫布喀劾四川陜西總督吳赫等侵蝕貧民籽粒銀兩,命鵬翮與傅臘塔往按。還奏未稱旨,命鵬翮與傅臘塔復往陜西詳審。三十九年春,還奏布喀、吳赫及知州藺佳選、知縣張鳴遠等侵蝕挪用,各擬罪如律。上諭大學士曰:「鵬翮往陜西,朕留心訪察,一介不取,天下廉吏無出其右。」

尋授河道總督,入辭,上諭令毀攔黃壩通下流,濬芒稻河、人字河湖入江。鵬翮到官,請撤協理徐廷璽及河工隨帶人員,並乞敕工部毋以不應查駁之事阻撓,並從之。尋疏言:「臣過雲梯關,見攔黃壩巍然如山,下流不暢,無怪上流之潰決。應拆攔黃壩,挑濬河身,與上流一律寬深。」又言清口淤墊,應於張福口開引河,引清水入運敵黃,建閘以時啟閉。又言人字河至芒稻山分二派,又名芒稻河,應濬使暢流;並濬鳳凰橋引河及雙橋、灣頭二河,皆匯芒稻河入江。俱下部議行。尋以攔黃壩既撤,河身開濬深通,暢流入海,疏請賜名大通口。上嘉鵬翮章奏詞簡意明,治事精詳,遣員外郎拖抗拖和、中書張古禮馳驛令鵬翮舉所規畫入奏。鵬翮疏陳開濬引河、運口,培修河岸堤壩諸事,並下部速議行。尋又疏陳河工諸弊,並請河員承挑引河,偶致淤墊,免其賠修;夫役勞苦,工成日請給印票免雜徭。上嘉其陳奏切要周備。尋又請於歸仁堤五堡建磯心石閘,並於三義壩舊中河築堤,改入新中河,合為一河,便糧艘通行。上謂所議甚當,並如所請。

上倚鵬翮治河,謂鵬翮得治河秘要,諭大學士曰:「鵬翮自到河工,日乘馬巡視堤岸,不憚勞苦。居官如鵬翮,更有何議?」鵬翮以修治事狀遣郎中王進楫入奏,上諭進楫歸語鵬翮,加意防守高家堰。鵬翮乃增築月堤及旁近諸堤壩。洪澤湖溢,泗州、盱眙被災,上詢修治策,鵬翮言:「泗州、盱眙屢被災,即開六壩亦不能免。」上怒曰:「塞六壩乃於成龍題請,不自鵬翮始。頃因泗州、盱眙災,令與阿山議修治,非欲開六壩救泗州、盱眙而令淮、揚罹水患也。鵬翮何昏憒乃爾!」四十一年,鵬翮疏請加築清河縣黃河南北岸戧堤,天妃閘改築運口,草壩建石壩,改卞家莊土堤為石堤,皆議行。又以桃源城西煙墩黃水大漲,請加築衛城月堤,並於邵家莊、顏家莊開引河,上慮部議遲延,特允之。四十二年,上南巡視河,制河臣箴、淮黃告成詩以賜,並書榜賚鵬翮父良。

山東泰安、沂州等州饑,上命截漕二萬石交鵬翮往賑。鵬翮令河員動常平倉穀二十八萬餘石散賑,疏請以山東各官俸工補還。上責鵬翮河員發倉穀邀譽,乃令山東各官補還,鵬翮謝罪,仍以「殫心宣力、清潔自持」,加太子太保。

河決時家馬頭,數年未堵塞。鵬翮以淮安道王謙言劾山安同知佟世祿冒帑誤工,奪官追償。世祿再叩閽,上令尚書徐潮按治,鵬翮、謙坐誣劾當譴,上特寬鵬翮。工部侍郎趙世芳又劾鵬翮浮銷十三萬有奇,請逮治。上曰:「河工錢糧原不限數,水大所需多,水小所需少。如謂鵬翮以十三萬入己,必無之事。河工恃用人,鵬翮用人不勝事,故至此耳。」因還世芳疏。上南巡,閱清口,見黃水倒灌,詰鵬翮,鵬翮不能對。上曰:「汝為王謙輩所欺,流於刻薄。大儒持身如光風霽月,況大臣為國,若徒自表廉潔,於事何益?」上舟渡河閱九里岡,嘉鵬翮修治如法,禦制詩書扇以賜。及秋,淮、黃並漲,古溝、清水溝、韓家莊並溢,廷臣議奪官,上命仍留任。尋督塞諸處漫口。

四十五年,疏請開鮑家營引河,尋用通判徐光啟言,擬開引河出張福口,分洪澤湖異漲,即為高家堰保障,謂為溜淮套。鵬翮與總督阿山、總漕桑額合疏請上蒞視。四十六年,上南巡,閱所擬引河道,諭曰:「朕自清口至曹家廟,見地勢甚高,標竿錯雜。依此開河,不惟壞田產,抑且毀塚墓。鵬翮讀書人,乃為此殘忍事,讀書何為?」詰責鵬翮,鵬翮謝罪。上以議為河山所主,非鵬翮意,削太子太保,奪官,仍留任。四十七年,以黃、運、湖、河修防平穩,命復官,並免應追帑銀。尋遷刑部尚書。四十八年,調戶部。

五十一年,江南總督噶禮與巡撫張伯行互劾,命鵬翮與總漕赫壽往按。鵬翮等右噶禮,請罷伯行。五十二年,調吏部。伯行劾布政使牟欽元,赫壽時為總督,與異議。五十三年,命鵬翮與副都御史阿錫鼐往按,復請雪欽元,議伯行罪斬。事互詳伯行傳。尋丁父憂,以原官回籍守制,服闋還朝。

六十年,汶水旱涸阻運,命往勘。請疏濬坎河、雞爪諸泉分注南旺,而於彭口築堤,障沙水入微山湖。河決開州,橫流至山東張秋,阻運,命往勘。請築南旺、馬場等湖堤,蓄水濟運;並陳引沁入運利害,謂地勢西北高於東南,若沁水從高直下,而河躡其後,害且叵測。

六十一年,世宗即位,加太子太傅。雍正元年,授武英殿大學士。河決馬營口,久未塞,命往勘。議並塞詹家店四口,濬治黃、沁合流處積沙,從之。三年,卒,加少保,命於定例外加祭,漢堂上官、科道皆會賜葬,謚文端。

論曰:明治河諸臣,推潘季馴為最,蓋借黃以濟運,又借淮以刷黃,固非束水攻沙不可也。方興、之錫皆守其成法,而輔尤以是底績。輔八疏以濬下流為第一,節費不得已而議減水。成龍主治海口,及躬其任,仍不廢減水策。鵬翮承上指,大通口工成,入海道始暢。然終不能用輔初議,大舉濬治。世以開中河、培高家堰為輔功,孰知輔言固未盡用也。


\end{pinyinscope}