\article{列傳一百五十七}

\begin{pinyinscope}
琦善伊里布宗室耆英

琦善,字靜庵,博爾濟吉特氏,滿洲正黃旗人。父成德,熱河都統,以先世格得理爾率屬歸附,世襲一等侯爵。

琦善由廕生授刑部員外郎,累遷通政司副使。嘉慶十九年,出為河南按察使,歷江寧、河南布政使。二十四年,擢河南巡撫。河決馬營壩,偕尚書吳璥督工,甫塞而儀封南岸又決,奪職,予主事銜留工。尋授河南按察使,調山東。道光元年,就擢巡撫。父憂,奪情任事,襲侯爵。捕治臨清教匪馬進忠,又籌濟高家堰工費八十萬。

五年,京察,詔嘉其明幹有為,能任勞怨,加總督銜。尋擢兩江總督,兼署漕運總督。時高堰屢決,淤運阻漕。琦善請用盤運法,並暫行海運,如議行。七年,議啟王家營舊減壩,大濬正河,尋以減壩堵合,黃水倒漾,復閉禦黃壩,漕船倒塘灌放,詔斥失機,議革職,寬之,降授內閣學士。尋復授山東巡撫。九年,擢四川總督。十一年,調直隸。十六年,協辦大學士。十八年,拜文淵閣大學士,仍留總督任。

琦善久膺疆寄,為宣宗所倚任。二十年,海疆事急,駐天津籌辦防務。八月,英兵船至海口,投書乞通商,訴林則徐、鄧廷楨等燒煙啟釁。琦善招宴英領事義律及兵官,許以代奏。遂入覲面陳,授欽差大臣,赴廣東查辦。諭沿海疆吏但防要隘,遇英船毋開砲,義律乃率船回粵。尋罷則徐、廷楨,命琦善署兩廣總督兼粵海關監督。密疏臚陳粵事,略曰:「林則徐示令繳煙,許以賞犒,洋人頗存奢望。迨後每煙一箱,僅給茶葉五斤,所得不及本銀百分之一;又勒具『再販船貨入官、人即正法』甘結,迄未遵依,此釁所由起也。當義律具稟繳煙,距撤退買辦五日,非出情原。時義律僅止孤身,設有黨援,未必降心俯首。英吉利國王無給林則徐文書之事,惟呂宋國王曾有來文,或因此誤傳。林則徐稱定海陰濕,洋人病死甚多。咨查洋人米穀牲畜尚充,疫癘病斃者多水手舵工,頭目死者不過數人。從前外洋來信,祗言貿易。自林則徐欲悉外情,多方購求漁利之人,造作播傳,真偽互見,此時紛紛查探,適墮術中。林則徐奏各國憤英人阻其貿易,美利堅、法蘭西將遣船來與理論。訪聞各國曾有此說,然迄未見兵船來粵。前有美國二船,乘英人不備,進口,至今未敢駛出。畏葸如斯,縱力足頡頏,恐未肯傷其同類。虎門燒煙時,洋人觀者撰文數千言紀事,事誠有之,語多含譏刺,非心服。林則徐稱具結之後,查驗他國來船,絕無鴉片。如指上年而言,事屬以往,船貨無憑;若指本年而言,來船尚未進口,不能知其有,亦安能信其無?」並言將軍阿精阿請團練水勇,及林則徐請鼓勵員弁,俟事定再議。疏入,報聞,則徐以是獲罪。

時廣東撤水師歸營,猝被敵轟擊,掠去米艇兵丁,巡撫怡良以聞。琦善又陳:「英人回粵,詞氣傲慢,義律託疾將回國,且兵船日增。」得旨,仍暫停貿易,一面與議,一面籌防。義律堅持索還煙價,並增廈門、福州通商,嚴旨拒不許。十二月,義律見防禦漸撤,數遣挑戰,琦善諭止之。義律曰:「戰後再議,未為遲也。」乃犯虎門外沙角、大角兩砲臺,副將陳連升力戰死之,遂陷。提督關天培守靖遠砲臺,總兵李廷鈺守威遠砲臺,並請援,琦善不敢明發兵,夜遣二百人往。二十一年正月,事聞,上震怒,下琦善嚴議,命御前大臣貝子奕山為靖逆將軍,戶部尚書隆文、湖南提督楊芳副之,率師赴粵協剿。

義律數索香港,志在必得,琦善當事急,佯許之而不敢上聞。至是,義律獻出所踞砲臺,並原繳還定海以易香港全島,別議通商章程。琦善親與相見蓮花城定議,往返傳語,由差遣之鮑鵬將事,同城將軍、巡撫皆不預知。及英人占踞香港,出示安民,巡撫怡良奏聞,琦善方疏陳:「地勢無可扼,軍械無可恃,兵力不固,民情不堅,如與交鋒,實無把握,不如暫事羈縻。」上益怒,詔斥琦善擅予香港,擅許通商之罪,褫職逮治,籍沒家產。英兵遂奪虎門靖遠砲臺,提督關天培死之。

奕山等至,戰復不利,廣州危急,許以煙價六百萬兩,圍始解,而福建、浙江復被擾。琦善逮京,讞論大闢,尋釋之,命赴浙江軍營效力,未至,改發軍臺。二十二年,浙師復敗,吳淞不守,英兵遂入江,江寧戒嚴,於是耆英、伊里布等定和議,海內莫不以罷戰言和歸咎於琦善為作俑之始矣。是年秋,予四等侍衛,充葉爾羌幫辦大臣。

二十三年,以三品頂戴授熱河都統。御史陳慶鏞疏論僨事諸臣罪狀,上重違清議,再褫琦善職,意仍鄉用,未幾,予三等侍衛,充駐藏大臣。二十六年,授四川總督。二十八年,詔嘉其治蜀於吏治營伍實心整頓,復頭品頂戴。尋協辦大學士,留總督任。以平瞻對野番功被議敘。二十九年,調陜甘總督,兼署青海辦事大臣,剿雍沙番及黑城撒拉回匪。既而言官劾其妄殺,命都統薩迎阿往按,革職逮問。咸豐二年,定讞發吉林效力贖罪,尋釋回。

時粵匪已犯湖南,勢日熾,屢易帥皆不能制。起琦善署河南巡撫,駐防楚、豫界上。以捐餉加都統銜,授欽差大臣,專辦防務。湖北省城失守,觀望不能救。三年春,賊遂連陷安徽、江寧省城,分擾鎮江、揚州,命琦善偕直隸提督陳金綬防江北。三月,連敗賊於浦口雷塘,進剿揚州,分屯寶塔山、司徒廟,五戰皆捷。秋,破浦口援賊,合圍揚州。十二月,賊突圍出竄瓜洲,以收復揚州入告,詔斥勇潰縱賊,責令進剿瓜洲、儀徵,儀徵克復。四年夏,連戰金川、瓜洲、三汊河,屢奏斬獲。自琦善與向榮分主大江南北軍事,攻戰年餘,鎮江、瓜洲迄未克復,無得力水師,不能扼賊,琦善雖議增水師,亦未果。是年秋,卒於軍,贈太子太保、協辦大學士,依總督例賜恤,謚文勤。

子恭鏜,黑龍江將軍。孫瑞洵,烏里雅蘇臺參贊大臣;瑞澂,兩湖總督。瑞澂自有傳。

伊里布,字莘農,鑲黃旗紅帶子。嘉慶六年進士,授國子監學正,改補典簿。出為雲南府南關通判,署澂江知府,遷騰越知州。二十四年,總督伯麟薦其熟練邊務,能馭土司,治緬匪有功,以應升用。道光元年,從總督慶保剿平永北大姚夷匪,賜花翎,署永昌知府。擢安徽太平知府。歷山西冀寧道,浙江按察使,湖北、浙江布政使。五年,擢陜西巡撫,調山東。丁父憂,署云南巡撫。服闋,乃實授。時阮元為總督,伊里布和而廉,有政聲。回疆兵事起,自請從軍,詔斥不諳回情,妄行陳奏,奪職留任,尋復之。十三年,擢雲貴總督。京察,以久任邊疆,鎮撫得宜,被議敘。十八年,協辦大學士,留總督任。四川綦江奸民穆繼賢仇殺貴州仁懷武生趙應彩,遂糾眾踞方家溝為亂,伊里布率提督餘步雲、布政使慶祿等破其巢,斬獲千餘,誅賊首穆繼賢、謝法真等,餘匪悉平,賜雙眼花翎。

十九年,調兩江總督。二十年秋,英兵陷定海,命為欽差大臣,赴浙江查辦。時已有論致寇由斷絕貿易燒煙起釁者,密諭察訪確情毋回護。尋以琦善代林則徐,命沿海遇敵勿擊。伊里布初至浙,駐鎮海籌防,疏報擊沉敵船,有所擒獲,命慰諭英人攻擊出於誤會,促令退兵交地,俘虜俟敵退釋還。伊里布遣家丁張喜偕員弁赴定海犒師,英人亦答餽,奏聞,諭卻勿受。請增調安徽、兩湖兵,允之。

裕謙方代署兩江總督,疏言:「各省皆可議守,獨浙江必應速戰。」且言:「定海西境岑港為第一險要,應以精兵先據之。」下伊里布體察辦理。既而琦善在粵議款不得要領,兵端又開,二十一年正月,詔促伊里布進兵規復定海。二月,義律既踞香港,盡調英船赴粵,以交還定海告。詔斥附和琦善,以兵砲未集,藉詞緩攻,致敵船遁去,褫協辦大學士、雙眼花翎,暫留兩江總督任,以裕謙代為欽差大臣督浙師。裕謙論劾伊里布遣家丁赴敵船事,命解任,帶張喜來京,下刑部訊鞫,褫職,遣戍軍臺。未幾,定海、鎮海、寧波相繼陷,裕謙殉之。

二十二年春,揚威將軍奕經援浙,復挫敗。巡撫劉韻珂疏陳浙事危急,薦伊里布無急功近名之心,為一時僅見,請發軍營效力贖罪。於是予七品頂戴,隨杭州將軍耆英赴浙,密諭相機辦理。及英兵犯乍浦,耆英遣往設計退兵。五月,署乍浦副都統,復令張喜傳語,英兵遂去乍浦,犯吳淞,由海入江,鎮江失守。伊里布奉命偕耆英赴江寧議和,事詳耆英傳。和議既成,英兵退,約於廣東議稅則,命偕耆英詳慎酌商,授廣州將軍、欽差大臣,辦理善後事宜。二十三年,至粵,見民心不服,夷情狡橫,憂悴。逾月病卒,贈太子太保,謚文敏。

宗室耆英,字介春,隸正藍旗。父祿康,嘉慶間官東閣大學士。耆英以廕生授宗人府主事,遷理事官。累擢內閣學士,兼副都統、護軍統領。道光二年,遷理籓院侍郎,調兵部。四年,送宗室閒散移駐雙城堡。五年,授內務府大臣,歷工部、戶部。七年,授步軍統領。九年,擢禮部尚書,管理太常寺、鴻臚寺、太醫院,兼都統。十二年,畿輔旱,疏請察吏省刑,嘉納之,授內大臣。十四年,以管理步軍統領勤事,被議敘。歷工部、戶部尚書。十五年,以相度龍泉峪萬年吉地,加太子少保。命赴廣東、江西按事。十七年,內監張道忠犯賭博,耆英瞻徇釋放,事覺,降兵部侍郎。尋出為熱河都統。十八年,授盛京將軍。詔嚴禁鴉片,無論宗室、覺羅,按律懲治。疏請旗民十家聯保,以憑稽察。二十年,海疆戒嚴,疏請旅順口為水路沖衢,當扼要籌備。英船入奉天洋面,先後游弋山海關、秦皇島等處,錦州、山海關皆設防。

二十二年正月,粵事急,琦善既黜,調耆英廣州將軍,授欽差大臣,督辦浙江洋務。因御史蘇廷魁奏英吉利為鄰國所破,詔促耆英赴廣州本任,乘機進剿,尋知其訛傳,仍留浙江。五月,吳淞失守,命偕伊里布赴江蘇相機籌辦。英兵已入江,越圌山關,陷鎮江,踞瓜洲,耆英與揚威將軍奕經先後奏請羈縻招撫。七月,英兵薄江寧下關,伊里布先至,英人索煙價、商欠、戰費共二千一百萬兩,廣州、福州、廈門、寧波、上海五港通商,英官與中國官員用平行禮,及劃抵關稅、釋放漢奸等款。越三日,耆英至,稍稍駁詰之。英兵突張紅旗,置砲鍾山上臨城,急止之,遣侍衛咸齡、江寧布政使恩彤、寧紹臺道鹿澤良,偕伊里布家丁張喜,詣英舟,許據情奏聞。宣宗憤甚,大學士穆彰阿以糜餉勞師無效、剿與撫費亦相等為言,乃允之。耆英等與英將濮鼎查、馬利遜會盟於儀鳳門外靜海寺,同簽條約,先予六百萬,餘分三年給,和議遂成。九月,英兵盡數駛出吳淞,授兩江總督,命籌辦通商及浙江、福建因地制宜之策。

二十三年,授欽差大臣,赴廣東議通商章程,就粵海關稅則分別增減,各口按新例一體開關,臚列整頓稅務條款,下廷議施行。又奏美利堅、法蘭西等國一體通商,允之。美國請入京瞻覲,卻不許。二十四年,調授兩廣總督,兼辦通商事宜。二十五年,協辦大學士,留總督任。比利時、丹麥等國請通商,命體察約束。二十六年,京察,以殫心竭慮坐鎮海疆,被議敘。疏上練兵事宜,繕呈唐臣陸贄守備事宜狀,請下各將軍督撫置諸座右。英國請於西藏定界通商,諭耆英堅守成約,毋為搖惑。

故事,廣東洋商居住澳門,貿易有定界,赴洋行發貨,不得擅入省城。自江寧和議有省城設立棧房及領事入城之約,粵民猶持舊例,愬於大吏,不省,乃舉團練,眾議洶洶,不受官吏約束。二十三年,濮鼎查將入城,粵民不可,逡巡去。二十五年,英船復至,耆英遣廣州知府餘保純詣商,粵民鼓噪,安撫乃罷。英人以登岸每遭窘辱,貽書大吏誚讓,群情憤激,不可曉諭。至二十七年,英船突入省河,要求益堅,耆英謾許兩年後踐約,始退,自請議處。諭嚴為防備,務出萬全。耆英知終必有釁。

二十八年,請入覲,留京供職,賜雙眼花翎,管理禮部、兵部,兼都統。尋拜文淵閣大學士,命赴山東查辦鹽務,校閱浙江營伍。三十年,文宗即位,應詔陳言,略曰:「求治莫先於用人、理財、行政諸大端。用人之道,明試以功。人有剛柔,才有長短。用違其才,君子亦恐誤事;用得其當,小人亦能濟事。設官分職,非為眾人藏身之地。實心任事者,雖小人當保全;不肯任怨者,雖君子當委置。行政在於得人,迂腐之說,無裨時務,泥古之論,難合機宜,財非人不理。今賦額四千餘萬,支用有餘,不能如額,以致短絀。致絀之由,非探本窮源,不能通盤清釐。與其正賦外別費經營,不如於正賦中覈實籌畫。」疏入,特諭曰:「身為端揆,一言一動,舉朝所矜式。耆英率意敷陳,持論過偏,顯違古訓,流弊曷可勝言。」傳旨申飭。耆英不自安,屢稱病。是年十月,上手詔揭示穆彰阿及耆英罪狀,斥「耆英在廣東抑民奉夷,謾許入城,幾致不測之變。數面陳夷情可畏,應事周旋,但圖常保祿位。穆彰阿暗而難明,耆英顯而易見,貽害國家,其罪則一」。猶念其迫於時勢,從寬降為部屬。尋補工部員外郎。

咸豐三年,粵匪北犯,耆英子馬蘭鎮總兵慶錫奏請父子兄弟同赴軍前,命耆英隨巡防王大臣效力,以捐餉予四品頂戴。五年,慶錫向屬員借貸被劾,耆英坐私告,革職圈禁。

八年,英人糾合法、美、俄諸國兵船犯天津,爭改條約,命大學士桂良、尚書花沙納扆往查辦。巡防王大臣薦耆英熟悉情形,召對,自陳原力任其難,予侍郎銜,赴天津協議。初耆英之在廣東也,五口通商事多由裁決,一意遷就。七年冬,廣州陷,檔案為英人所得,譯出耆英章奏,多掩飾不實,深惡之。及至天津,英人拒不見,惶恐求去,不候旨,回通州,於是欺謾之跡益彰,為王大臣論劾,嚴詔逮治,賜自盡。

論曰:罷戰言和,始發於琦善,去備媚敵,致敗之由。伊里布有忍辱負重之心,無安危定傾之略,且廟謨未定,廷議紛紜,至江寧城下之盟,乃與耆英結束和議,損威喪權,貽害莫挽。耆英獨任善後,留廣州入城之隙,兵釁再開,浸致庚申之禍。三人者同受惡名,而耆英不保其身命,宜哉。


\end{pinyinscope}