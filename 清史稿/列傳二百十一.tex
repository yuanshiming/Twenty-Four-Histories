\article{列傳二百十一}

\begin{pinyinscope}
吳振棫張亮基毛鴻賓張凱嵩

吳振棫,字仲云,浙江錢塘人。嘉慶十九年進士,選庶吉士,授編修。道光二年,出為雲南大理知府,歷山東登州、沂州、濟南,安徽鳳陽知府;山東登萊青道;貴州糧儲道;貴州按察使;山西、四川布政使。咸豐二年,擢雲南巡撫。尋甸、東川回匪蠢動,粵匪由廣西闌入開化、廣南境,偕總督吳文鎔先後遣將擊平之。四年,調陜西巡撫,未行,署云貴總督。貴州興義、普安匪起,檄安義鎮總兵金剛保等剿之。遵義亦被匪圍,合滇、黔兵力,迭戰獲勝,擒匪首楊鳳先於石阡葛莊司。五年秋,始抵陜西任。匪首陳通明受粵匪指揮,於潼關糾眾謀響應,以計擒之,並獲其黨張順、羅吉祥等置諸法,被詔嘉獎。鹽課攤歸地丁,數倍於昔,奏請改行招販,先課後鹽,民便之。未幾,擢四川總督。

七年,調雲貴總督。雲南漢、回積仇,自中原兵事亟,協餉不至,回亂愈恣。團練跋扈,動相殺掠,省城戒嚴。前任總督恆春不能制,夫婦同縊,巡撫舒興阿亦以病求去,惟布政使桑春榮困守危城。文宗知振棫熟悉滇省情形,故以代之。命選川兵三千,攜餉五萬馳往,調前山東巡撫張亮基幫辦軍務以副之。振或至,先駐宣威,進次曲靖。疏言:「先剿后撫,勢順而易,不待智者而知。兵盛餉足,必應如是。前督臣林則徐剿永昌回匪,兵、練萬餘,本省有餉可籌;彌渡獲勝,匪旋受撫,其地祗迤西一隅中之一隅。此次匪遍三迤,情形迥不相同,非數千之兵、十數萬之餉所能蕆事。如率意逕行,徒損國威,於事無補。臣初到滇,於漢、回兩無嫌怨,惟憑藉兵威,結以恩信,有所申訴,處以公平。省城為根本重地,省回解散,此外漸次籌辦,其負嵎抗拒者,仍當力剿。匪勢漸孤,較易得手。否則不自量度而急乘之,更無轉圜地步,禍更烈矣。現在兵無可調,餉無可籌,宵旰焦勞,事非一省。臣為雲南一省計,並當為天下全局計,豈容再有貽誤,致令徵調無休?故未言剿先言撫,有萬不得已之苦衷,雖成敗利鈍難以逆料,舍此亦別無良策也。」

又奏:「在籍侍郎黃琮、御史竇墉、總兵周鳳岐奉命團練,設總局於省城。周鳳岐意見不合,引嫌不肯與聞。黃琮、竇墉聯銜出示,專主痛剿,民間紛紛集練,回眾疑忌日深。地方官苦心解散,漢民往往閧堂塞署,逼官殺回。故團練在他省為要務,在滇省竟為大患。黃琮等每言省團可得六十萬人,無虞寇警。回匪初至城外,不及千人,團丁招之不來,來即奔潰。近日省練一萬餘人,月需餉數萬,經費不敷。練頭自行管帶,不盡官派。回眾有求撫之意,梗議者忽用練往剿,妄殺邀功,致可撫者終不能撫。黃琮、竇墉系特派人員,非臣力所能制,請旨定奪。臣已咨桑春榮嚴覈守城之練,裁汰冗濫,以節糜費。練歸官統,如不奉調派,自行出隊,即按軍法從事,庶一事權而免掣肘。」疏入,詔褫黃琮、竇墉職,許回民悔悟自新,其負固不服者,痛加剿辦。漢民借團練為名肆行殺掠者,以軍法從事。於是振棫遣漢、回委員赴省城曉諭漢、回,解釋猜嫌,分畫所居街道,撥抵難民遺產,議定章程,遣散歸業。先後剿平霑益回匪,殲咸寧土匪李廣沅。八年四月,撫局粗定,入駐省城,偕張亮基籌辦迤西剿撫事宜。臨安回匪攻府城,遣兵擊走之,又敗之於阿迷州,解河西縣之圍。

是年冬,以病乞罷,因子春傑官雁平道,就養山西。同治元年,命會同巡撫英桂防河,尋命赴陜西會辦軍務。十年,卒,詔依例賜恤。

張亮基,字石卿,江蘇銅山人。道光十四年舉人,入貲為內閣中書。從大學士王鼎赴河南治河,督築西壩。工竣,賜花翎,擢侍讀。二十六年,出為雲南臨安知府,總督林則徐曾與共事河工,知其才,密薦可大用,調署永昌。邊夷滋擾,亮基用土弁左大雄擒匪首,事乃定。超擢雲南按察使,就遷布政使。三十年,擢雲南巡撫,兼署云貴總督。粵匪漸熾,嘗密疏論軍事,文宗韙之。

咸豐二年,調湖南巡撫,在途聞賊圍長沙,疏請駐守常德。詔趣進解省城之圍,至則梯城而入,屢出隊與城外援軍夾擊,賊解圍去。破岳州,入湖北,漢陽、武昌相繼陷,湖廣總督徐廣縉以罪罷,命亮基代之,規進剿。亮基疏言宜防賊回竄,意在專顧湖南,詔趣速進。三年春,賊棄武漢東下,亮基抵湖北籌辦收復撫恤事宜。通城、崇陽、嘉魚、廣濟土匪起,平之。賊自下游分竄江西,亮基督師扼道士洑、黃石港,分兵赴援。秋,賊之分竄河南者,由羅山入湖北黃安、麻城境,水陸夾擊,殲之。

調山東巡撫,未行,江西賊由九江來犯,令道員徐豐玉御之於田家鎮,戰失利,豐玉陣亡,亮基坐降四級留任。時粵匪李開芳等犯畿輔,踞靜海。亮基至山東,奉命扼德州,防其南逸。南路賊欲由淮、徐窺伺北犯為應援,令按察使厲恩官率兵駐宿遷之北以防之。四年,賊入山東境,亮基馳扼濟寧,杜其北竄。尋陷鄆城,擾範縣、壽張、東平,繞出賊前截擊,敗之於臨清黑家莊。既奏捷,幫辦軍務大臣勝保劾其取巧冒功,詔斥亮基欺罔,並追論初赴湖南不急趨長沙,及去湖北時但求自全,居心狡詐,職,遣戍軍臺。逾年,給事中毛鴻賓言臨清之役,勝保妄劾,御史宗稷辰亦言亮基能任事,未盡其用,乃釋回,發東河差遣,尋命往安徽隨辦軍務。

七年,予五品頂戴,命赴雲南幫辦剿匪事宜。雲南回匪方熾,團練橫行省會,總督吳振棫初至,駐曲靖,裁抑練勇,招撫回眾。霑益回最悍,集眾犯宣威,亮基督按察使徐之銘等率兵擊走之。八年春,又敗之於袁家屯,殲賊甚眾,餘黨就撫,詔嘉之,授雲南巡撫。既而振棫乞罷,擢雲貴總督,亮基薦徐之銘代為巡撫。臨安回匪攻城,擾及阿迷,剿平之。九年,省回就撫後,踞碧雞關,劫奪近郊,分剿乃散。又剿平彞、安寧、緬寧、楚雄諸匪,武定、羅次、富民、祿豐、祿勸諸州縣先後克復。然回、練互相猜忌,亂機時起。

徐之銘既為巡撫,貪縱險狠,與亮基陰不相能,時構煽其間。十年秋,回人掌教馬德新、徐元吉,武生馬現,率各屬回民來省乞撫,住城外江右館,亮基約之銘同詣撫諭。之銘陰嗾已散練丁擁至督署阻撓,諭之不可,殺通海知縣雷焱於門,遂逼殺招撫委員紳士馬椿齡、孫鈞。亮基為所脅持,不敢入告,以病乞罷,命劉源灝代之。源灝久不至,亮基逕去。十一年,至湖北,乃疏陳滇事,劾之銘不法。會布政使鄧爾恆升任陜西巡撫,去滇,之銘嗾匪戕於路。於是罷源灝,以潘鐸署總督,命亮基赴滇查辦,督師剿匪。亮基疏請發部照募損充餉,募勇千人然後行,與潘鐸先後至四川,欲資其餉力、兵力。四川兵事未定,無以濟之。林自清者,亮基之舊部,方署云南提督,與之銘及馬如龍等皆不協,回人仇之。聞亮基在四川,擅率所部號萬人入川求效用,阻之不聽。詔亮基撫諭解散,而之銘嗾馬如龍等聲言拒亮基不使入境,相持久之。同治元年,潘鐸先抵任,請暫留之銘以畢撫局,遂改命亮基以總督銜署貴州巡撫。未幾,之銘復陰嗾回眾為變,鐸被戕,而雲南之亂愈亟矣。

二年,亮基至貴州,黃號、白號、苗、教諸匪並熾,上下游遍地皆賊。亮基令總兵沈宏富等攻遵義螺螄堰,破之,殲餘匪於上稽場。令總兵劉義方等剿思南教匪,復普安、安南,又連破苗匪於桐梓鼎城及水城馬龍胯,擒匪首何潤科等於黔西,降萬人。三年,尚大坪匪犯省城,督沈宏富等戰於郊,殲賊千計,復修文。總兵林自清、趙德昌克龍里,又復興義,解清鎮之圍,收復定番、廣順、長寨諸城,破龍泉、湄潭黃、白號匪老巢,克滇西衛城。四年,克黔西石阡、永寧、荔波,貴州地瘠財匱,饑軍索餉,時虞譁噪。亮基撫馭防剿,僅得粗安,而所部諸將多驕蹇,輿論不協,為侍讀學士景其濬論劾。亮基乃劾總兵林自清、劉有勛,副將池有連等劫掠扣餉,不聽調度,請嚴治。詔布政使嚴樹森察奏,亮基復具疏自陳,言樹森規避貴州,安坐鄰省不親至,於是亮基、樹森並褫職。

十年,卒。湖南巡撫王文韶、貴州巡撫曾璧光先後請復原銜,各建專祠。光緒三十四年,湖南、貴州京官合詞臚陳功德在民,追謚惠肅。

毛鴻賓,字翊雲,山東歷城人。道光十八年進士,選庶吉士,授編修。遷御史、給事中,數上封事論軍務。咸豐三年,以尚書孫瑞珍薦,命回籍治團練。四年,劾幫辦軍務大臣勝保罪狀,請嚴旨查辦。五年,授湖北荊宜施道,調安襄鄖荊道,歷安徽按察使、江蘇布政使。

十一年,署湖南巡撫,尋實授。疏言:「湖南地居僻遠,向非富強,自前撫臣張亮基、駱秉章等於吏治民風實力講求,用能削平寇盜,屹為上游重鎮,用人之效,有明徵矣。臣以為名將不過收戰陣之功,得賢督撫,斯能造封疆之福。如左宗棠識略過人,其才力不在曾國籓、胡林翼之下,今但使之帶勇,殊不足以盡其長,倘畀以封疆重任,必能保境安民,兼顧大局。前任雲貴總督張亮基,果決有為,雲南壤接邊陲,餉糈不給,漢、回仇釁相尋,即令經營盡善,亦僅有益一隅,似不若任以要地,俾展所長。但使東南日有轉機,則雲、貴游氛無難迅掃,此輕重之機宜審者也。」時湘軍所至有功,各省多往召募,鴻賓疏陳招勇流弊,請慎選將領以收實效,並被嘉納。

石達開竄湖南,鴻賓遣知府席寶田、副將周達武、總兵趙福元分路進擊,解會同、黔陽之圍。同治元年,進復來鳳,貴州提督田興恕兼署巡撫,軍報不實,信用左右,鴻賓疏劾之。遣兵越境剿貴州竄匪,復天柱縣城。又剿銅仁張家寨,匪首蕭文魁率眾降,克大小青兩堡。江藍同知椿齡指團紳為土匪,鴻賓廉知椿齡有酷刑逼借事,劾罷之。椿齡京控,訐鴻賓借貸不遂,鴻賓自請查辦,下總督官文鞫訊,得白。

擢兩廣總督,英德土匪起,令按察使張運蘭剿平之。偕巡撫郭嵩燾奏定變通緝捕章程,獲大盜者予優擢,允之。

三年,江南既復,浙、贛餘氛未靖。鴻賓疏言:「江西南路之防猶有未備,閩、粵交界均無防兵,慮賊上竄,以粵東為尾閭。江西當四沖之地,宜合數省兵力,乘大勝餘威,聚而殲之。已咨曾國籓調撥勁旅,繞越寧郡、石城一帶,扼賊南竄之路,臣派一軍於閩、粵交界會同進剿。並請敕曾國籓嚴守南贛,俾毋竄越。」

四年,坐前在湖南,道員胡鏞請咨引見,繳回咨文,委署道缺,降一級調用,回籍。七年,卒。宣統初,山東巡撫袁樹勛疏陳鴻賓功績,復原官,祀鄉賢祠。

張凱嵩,字雲卿,湖北江夏人。道光二十五年進士,廣西即用知縣,歷宣化、懷集、臨桂知縣。李星沅、勞崇光並薦其能,咸豐五年,擢慶遠知府。剿平土匪王得勝等,擢左江道,調署右江道。慶遠失守,革職留任。八年,偕按察使蔣益澧破賊,克慶遠,復原官,署按察使,尋實授,遷布政使。同治元年,巡撫劉長佑赴潯州籌剿撫,留凱嵩經畫後路。荔浦張皋友陷陽朔,遣兵敗賊於鷓鴣巖,復其城,就擢巡撫。諸匪中黃鼎鳳、張皋友最猖獗,分陷貴縣、陽朔、麕集大鹿灘、馬瀨,檄總兵李明惠、提督江忠義先剿馬瀨,進規貴縣,破之於桂嶺,殲擒賊首張皋友、陳土養。二年,檄布政使劉坤一攻黃鼎鳳於登龍橋。賊走覃塘,進圍之。信都賊陳金剛等來援,道員蔣澤春逆擊敗之,進克容縣,坤一克覃塘。三年,克天平寨,擒黃鼎鳳。貴縣平,加頭品頂戴。

疏陳左右江積匪未清,議三路進兵,以劉坤一統七營留防潯州,易元泰統十一營由賓州、遷江達思恩,李士恩統水陸八營由橫州達南寧,節節進剿。四年,坤一攻克大廟、江口、平菼,斬賊首梁安邦,南寧河道始通。元泰剿上林,平之。坤一擢江西巡撫去,以同知劉培一代領其軍,將親赴南寧督戰,會偽康王汪海洋竄粵,將入廣西,詔凱嵩駐防潯州。五年,凱嵩至南寧,進攻山澤,督諸軍穴地轟城,奪山入,擒偽平章蘇仲熙等。孫仁廣單騎走旺隴,追斬之。山澤為賊所踞十餘年,至此悉平。

六年,擢雲貴總督。自潘鐸被戕,滇事益紛。行至巴東,稱病,三疏請罷,坐規避,褫職。光緒六年,以五品京堂起用,授通政使參議,遷內閣侍讀學士,署順天府尹,授貴州巡撫。十年,調雲南。請於省城設開採五金總局,以興礦利,偕內閣學士周德潤勘越南界務。十二年,卒於官。廣西巡撫李秉衡疏陳凱嵩政績,請建專祠,廣西京官論其不當,罷之。子仲炘,光緒三年進士,由翰林御史官至通政司參議,敢言有聲。

論曰:雲南地居邊遠,回、漢積仇,中原多故之秋,幾為王靈所不及。吳振棫兼籌剿撫,實體中朝措置之難。張亮基才足有為,誤用徐之銘,受其排擠,遂至不可收拾。自潘鐸被戕之後,無人敢任其艱危。毛鴻賓疏言內地寇平,邊方自靖,誠為確論。張凱嵩因規避黜,後仍以舊勞起用,朝廷固鑒其情已。


\end{pinyinscope}