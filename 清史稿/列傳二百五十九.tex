\article{列傳二百五十九}

\begin{pinyinscope}
陸潤庠世續伊克坦梁鼎芬徐坊勞乃宣沈曾植

陸潤庠,字鳳石,江蘇元和人。父懋修,精醫,見藝術傳。潤庠,同治十三年一甲一名進士,授修撰。光緒初,屢典試事,湖南、陜西皆再至。入直南書房,洊擢侍讀。出督山東學政。父憂服闋,再遷祭酒,典試江西。以母疾乞養歸。二十四年,起補祭酒,擢內閣學士,署工部侍郎。兩宮西巡,奔赴行在,授禮部侍郎,充經筵講官。擢左都御史,管理醫局,典順天鄉試,充會試副總裁,署工部尚書。

三十二年,充釐訂官制大臣。已而工部裁省,以尚書兼領順天府尹事。明年,授吏部尚書、參預政務大臣,謂:「捐例開,仕途雜,膺民社者或不通曉文義,因訂道府以下考試章程,試不及格者停其分發,設仕學館教習之。」潤庠為陸贄後,嘗奏進文集,參以時事,大意謂:「成規未可墨守,而新法亦須斟酌行之。若不研求國內歷史,以為變通,必至窒礙難行,且有變本加厲之害。」

宣統元年,協辦大學士,由體仁閣轉東閣大學士,充弼德院院長。皇帝典學,充毓慶宮授讀,兼顧問大臣。疏陳:「曲阜篤生聖人之地,今新建曲阜學堂,必須闡明經術,提倡正學。若雜聘外人,異言異服,喧賓奪主,將來聖教澌滅,亦朝廷之憂。」又陳:「釐訂官制,宜保存臺諫一職。說者謂既有國會,不須復有言官。豈知議員職在立法,言官職在擊邪。議院開會,不過三月,臺諫則隨時可以陳言。行政裁判,系定斷於事後,言官則舉發於事前。朝廷欲開通耳目,則諫院不可裁;諸臣欲鞏固君權,則亦不可言裁。即使他時國會成立,亦宜使該院獨立,勿為邪說所淆。」又言:「游學諸生,於實業等事學成而歸者,寥寥可數,而又用非所學。其最多者惟法政一科。法政各國歧異,悉就其本國人情風俗以為制。今諸生根柢未深,於前古聖賢經傳曾未誦習,道德風尚概未聞知,襲人皮毛,妄言改革;甚且包藏禍心,倡民權革命之說,判國家與君主為兩途,布其黨徒,潛為謀主。各部院大臣以為朝廷銳意變法,非重用學生不足以稱上旨,遂乃邪說詖行,遍播中外,久之必致根本動搖,民生塗炭。」

又疏陳財用枯竭,請酌停新政,謂:「今日之害,先由於督撫無權,漸而至於朝廷無權。庫儲之困難,寇賊之充斥,猶其顯而易見者也。鎮兵之設也,所用皆未經歷練之學生,韜略則紙上空談,作用則徒取形式,甚至持不擊同胞之謬說。一旦有事,督撫非但不能調遣,甚且反戈相向,其不可用明矣。則莫如停辦鎮兵,仍取巡防隊而整理之。審判之立也,所授皆未曾聽訟之法官,黑白混淆,是非倒置。舊時諳練之老吏,督撫不得用之,散遣州縣捕役,以緝盜責之巡警。巡警無能也,且不過省會及通商口岸有巡警,豈能分布鄉閭?將來必至遍地皆盜,人民無可控訴。則莫如停辦審判,仍以聽斷緝捕歸之州縣。諮議局之設也,所舉皆不諳掌故之議員,逞臆狂談,箝制當道,督撫莫能禁之。於是借籌款之名,魚肉鄉里,竊自治之號,私樹黨援。上年資政院開議,竟至戟手漫罵,藐視朝廷。以辯給為通才,以橫議為輿論,蜩螗沸羹,莫可究詰。則莫如停辦國會,仍以言事責之諫院。學堂之設也,所聘皆未通經史之教員,其沿用教科書,僅足啟發顓蒙,廢五經而不讀,禍直等於秦焚。暑假、星期,毫無拘束,彼血氣未定者,豈不結黨為非?又膳學費百倍於前,致使貧寒聰穎之士流,進身無路。則莫如停辦中小學堂,仍用經策取士。凡此皆於財政有關,而禍不僅在財政,使不早為之所,必至權柄下移,大局不可收拾。」疏上,多不報。時建設立憲內閣,宰輔擁虛名而已。

武昌兵變,官軍既克漢陽,武昌旦夕下。而新內閣又成立,總理大臣袁世凱議修和息戰禍,取隆裕太后懿旨,頒示天下,改建國體,於是遜位詔下矣。潤庠以老瞶辭授讀差,奉懿旨仍照料毓慶宮,給月俸如故,授太保。越二年,病卒,年七十五,贈太傅,謚文端。

潤庠性和易,接物無崖岸,雖貴,服用如為諸生時。遇變憂鬱,內結於胸而外不露。及病篤,竟日危坐,瞑目不言,亦不食,數日而逝。

世續,字伯軒,索勒豁金氏,隸內務府滿洲正黃旗。光緒元年舉人,以議敘主事歷內務府郎中,擢武備院卿,授內閣學士。二十二年,為總管內務府大臣,兼工部侍郎。二十六年,各國聯軍入京,兩宮西狩,適遭父喪,命留京辦事。即日縗墨詣聯軍請保護宮廷,日為宮中備飲饌,並保壇廟。晉理籓院尚書,調禮部。兩宮回鑾,賞黃馬褂,轉吏部,兼都統。內務府三旗甲米向歸吏胥代領折價,名曰「米折」,所得甚微。世續商之倉場,飭旗丁自領,眾感實惠。纂呈四書圖說,特旨褒嘉。三十年,以吏部尚書協辦大學士,尋授體仁閣大學士。三十二年,命為軍機大臣。歷轉文華殿大學士,充憲政編查館參預政務大臣。念八旗生計日艱,奏設工藝廠,俾習工藝贍身家。德宗崩,議繼體,世續獨言國事艱危,宜立長君,不能用。

宣統改元,以疾乞休。三年,復起原官,仍兼總管內務府大臣。及議遜位,世續首贊之。太后令磋商優待條件,授太保。接修崇陵工程,加太傅。丁巳復闢,懼禍及,力阻之。事變亟,入宿衛,並以殮服自隨。頻年以經費拮據,支持尤苦,纂修德宗實錄,始終其事,及書成,已病不能起矣。辛酉年,卒,年六十九。贈太師,謚文端。

伊克坦,字仲平,瓜爾佳氏,滿洲正白旗人,西安駐防。光緒十二年進士,以編修歷至都察院副都御史,充滿蒙文學堂監督。有請達海從祀文廟者,伊克坦以達海創定國書,繙譯經史,有功聖教,允宜附祀,即為代奏,略言:「學官立於漢京,而配享實始於唐代,宋、元以來,迭有增祀,大率以闡明聖學,有功經訓為斷。漢儒許慎,特因說文解字,功在經籍,專隆升祔。我太祖高皇帝、太宗文皇帝指授文臣創立國書,傳譯經史,宣布文教,尤極千古未有之盛。夫國書字體,創自文臣額爾德尼及噶蓋等,而仰承聖意,匯集大成,詳定頒行者,實唯儒臣達海。達海以肇造貞元之佐,擅閎通著述之才,歷相兩朝,瞻言百里。其初奉命詳定國書,重加圈點,發明音義;又以國書漢字對音未全,於十二字頭之外有所增加,而國書之用乃廣。復定兩字切音之法,較之漢文切音,更為精當,而國書之制乃備。繙譯經典,昭示群倫,功不在傳經諸儒下。崇德十年,既蒙賜謚文成,康熙九年,復奉賜文立碑,隆德報功,永受恩澤。旋有學士阿理瑚奏請從祀文廟,禮臣復奏,以為創造國書,一藝之長,不當從祀,未經議準。查達海詳定國書字體,實稟太宗指示而成。作者為聖,述者為明,非唯羽翼六經,抑且昭示百世。部議謂僅一藝之長,實未深知大體。達海於聖經有表章之力,於後學有津逮之功。方今宗學、旗學兼重國書,並奉旨特設滿蒙文學堂於京師,奉省亦經奏立八旗滿蒙文中學堂。揆諸古者釋奠祭師之誼,達海應得附祀,核與漢儒許慎從祀之例亦屬相符。仰懇俯準達海附祀文廟,並請敕建專祠於盛京,以昭矜式。查盛京東門外尚有達海塋墓,榛莽荒蕪,碣碑剝落,並請敕下所司修治看護,用示朝廷崇尚實學、藎念儒臣之至意。」

又代陳典學事宜,略言:「伏讀雍正三年世宗憲皇帝諭:『帝王御宇膺圖,咸資典學。我聖祖仁皇帝天亶聰明,而好古敏求,六十餘年孜孜不倦。』又喜慶二十四年仁宗睿皇帝:『帝王之學,在於貫徹天人,明體達用,以見諸施行,與經生尋章索句者不同。』仰見列聖相承,重視典學之至意。我皇上睿哲性成,聰明天縱,沖齡踐祚,洪業肇基,當此春秋典學之時,實為聖敬日躋之始,伏維監國攝政王薰陶德性,輔養聖躬,慎選侍從,左右將護,亦既淵沖翕受,法戒靡遺。唯是皇上一念之張弛,系萬機之治忽;一朝之規制,系薄海之觀瞻。有不得不慎之又慎者,謹為我皇上詳晰陳之:一,請崇聖學。易端蒙養,禮重師教,書述遜敏,詩頌緝熙,聖學精微,非尋常科學範圍之所能及。宋儒有言『帝王之學,與儒生異尚』,與我仁宗睿皇帝典學之諭用意正符。今我皇上典學之初,應定教學科目,自應會通今古,融貫中西,不可拘於舊例。伏乞簡派儒臣,詳細籌訂,鑒成憲,酌時宜,毋徒陳進講之空文,毋虛循延英之故事,庶足以開張聖聽,裨益亶聰,以立聖學聖治之基。一,請擇賢傅。舊制師傅向以大臣選充,期於老成典型,成就君德,然或入官從政,講學非其所長。老師大儒,潛德隱而勿耀,而教育精深,尤非研究有素,不能取益。擬請敕下內外大臣,各舉所知,勿拘資格,略仿乾隆十四年詔舉經學人員成例,擇其品端學粹、教育卓著成績者,請旨召用,隆以師傅之任,分門講教,而仍派大臣總司其成,俾專日講於經筵,不必更勞以職事。其任彌專,其責彌重,其效彌速,使天下曉然於尊師崇儒之意,庶儒林有所矜式,而聖德日進高明矣。一,請肅規制。古者聖王教胄,必選端方正直、道術博聞之士,與之居處,是以習與智長,化與心成。我皇上毓德方新,始基宜固,舊制選派內監伴讀,似不足以肅學制而廣箴規。擬請改選王公大臣之賢子弟昕夕侍從,斅學相長,並參考學校制度,建設講堂,陳列圖書彞器,觀摩肄習,以收敬業樂群之效。以上三事,僅舉大綱。我皇上今日之言動起居,罔有勿敬,即異日之立政敷教,罔有勿臧,此尤根本之至計,不可不謹之於漸,而慎之於始者也。伏念朝廷廣勵人才,振興教育,侁侁學子,爭自濯磨,皇上典學伊始,益宜宏茲遠謨,以慰天下士民之望。」

宣統三年,伊克坦與大學士陸潤庠及侍郎陳寶琛,同奉命直毓慶宮,朝夕入講,遇事進言,憂勤彌甚。丁巳復闢,潤庠已前卒,寶琛為議政大臣,伊克坦一不爭權位,日進講如故。及事變,誓臨危以身殉。伊克坦忠直有遠識,主開誠布公,集思廣益;而左右慮患深,務趨避,時復相左。伊克坦憂鬱遂久病,日寄於酒。癸亥,卒,年五十有八,謚文直。

梁鼎芬,字星海,廣東番禺人。光緒六年進士,授編修。法越事亟,疏劾北洋大臣李鴻章,不報。旋又追論妄劾,交部嚴議,降五級調用。張之洞督粵,聘主廣雅書院講席;調署兩江,復聘主鍾山書院;又隨還鄂,皆參其幕府事。之洞銳行新政,學堂林立,言學事惟鼎芬是任。

拳禍起,兩宮西幸,鼎芬首倡呈進方物之議。初以端方薦,起用直隸州知州;之洞再薦,詔赴行在所,用知府,發湖北,署武昌,補漢陽。擢安襄鄖荊道、按察使,署布政使。奏請化除滿、漢界限。三十二年,入覲,面劾慶親王奕劻通賕賄,請月給銀三萬兩以養其廉。又劾直隸總督袁世凱「權謀邁眾,城府阻深,能諂人又能用人,自得奕劻之助,其權威遂為我朝二百年來滿、漢疆臣所未有,引用私黨,布滿要津。我皇太后、皇上或未盡知,臣但有一日之官,即盡一日之心。言盡有淚,淚盡有血。奕劻、世凱若仍不悛,臣當隨時奏劾,以報天恩」。詔訶責,引疾乞退。兩宮升遐,奔赴哭臨,越日即行,時之洞在樞垣,不一往謁也。明年,聞之洞喪,親送葬南皮。

及武昌事起,再入都,用直隸總督陳夔龍薦,以三品京堂候補。旋奉廣東宣慰使之命,粵中已大亂,道梗不得達,遂病嘔血。兩至梁格莊叩謁景皇帝暫安之殿,露宿寢殿旁,瞻仰流涕。及孝定景皇后升遐,奉安崇陵,恭送如禮,自原留守陵寢,遂命管理崇陵種樹事。旋命在毓慶宮行走。丁巳復闢,已臥病,強起周旋。事變憂甚,逾年卒,謚文忠。

徐坊,字梧生,山東臨清州人,巡撫延旭子。少納貲為戶部主事。光緒十年,法陷諒山,延旭逮問,下刑部獄。坊侍至京師,入則慰母,出則省延旭於獄,橐饘之事,皆自任之,布衣蔬食,言輒流涕。延旭戍新疆,未出都卒,坊扶柩歸葬,徒行泥淖中,道路嘆為孝子。二十六年,奔赴西安行在。明年,扈駕返,以尚書榮慶薦,超擢國子丞。鄂變起,連上五封事,俱不報。遜位詔下,遂棄官。旋命行走毓慶宮,坊已久病,力疾入直。未幾,卒,謚忠勤。

勞乃宣,字玉初,浙江桐鄉人。同治十年進士,以知縣分直隸。查淶水禮王府圈地,力請減租蘇民困。光緒五年,初任臨榆,日晨起坐堂皇治官書,啟重門,民有呼籥者,立親訊之,使閽者不能隔吏役,吏役不能隔人民。其後居官二十餘年皆如之。曾國荃督師山海關,檄司文案。歷南皮等縣,畿輔州縣遇道差,咸科於民有定額,而官取其贏。乃宣任蠡縣,值謁陵事竣,贏支應錢千餘緡,儲庫備公用。任完縣,購書萬餘卷庋尊經閣。任吳橋,創里塾,農事畢,令民入塾,授以弟子規、小學內篇、聖諭廣訓諸書,歲盡始罷。先是寧津奸民陳二糾黨為州郡害,土人稱曰黑團,勢甚熾。嘗至南皮劫殺,乃宣會防營掩捕,擒陳二及其黨數人磔於市,黑團遂絕。

二十五年,義和拳起山東,蔓延於直、東各境,乃宣為義和拳教門源流考,張示曉諭,且申請奏頒禁止,不能行。景州有節小廷者,匪首也,號能降神。乃宣飭役捕治,縱士民環觀,既受笞,號呼不能作神狀,梟示之,匪乃不敢入境。明年,拳黨入京,乃宣知大亂將作,適調吏部稽勛司主事,遂請急南歸,浙撫任道鎔延主浙江大學堂。尋入江督李興銳幕,端方、周馥繼任,咸禮重之。周馥從乃宣議,設簡字學堂於金陵。初,寧河王照造官話字母,乃宣增其母韻聲號為合聲簡字譜,俾江、浙語音相近處皆可通。三十四年,召入都,以四品京堂候補,充憲政編查館參議、政務處提調。

宣統元年,詔撰經史講義,輪日進呈,疏請造就保姆,輔養聖德。二年,欽選資政院碩學通儒議員。法律館奏進新刑律,乃宣摘其妨於父子之倫、長幼之序、男女之別者數條,提議修正之。授江寧提學使。三年,召為京師大學堂總監督,兼學部副大臣。遜位議定,乞休去,隱居淶水。時士大夫多流寓青島,德人尉禮賢立尊孔文社,延乃宣主社事,著共和正解。丁巳復闢,授法部尚書,乃宣時居曲阜,以衰老辭。卒,年七十有九。

乃宣誦服儒先,踐履不茍,而於古今政治,四裔情勢,靡弗洞達,世目為通儒。著有遺安錄、古籌算考釋、約章纂要、詩文稿。

沈曾植,字子培,浙江嘉興人。光緒六年進士,用刑部主事。事親孝,母多疾,醫藥必親嘗,終歲未嘗解衣安臥,遂通醫。遷員外郎,擢郎中。居刑曹十八年,專研古今律令書,由大明律、宋律統、唐律上溯漢、魏,於是有漢律輯補、晉書刑法志補之作。曾植為學兼綜漢、宋,而尤深於史學掌故,後專治遼、金、元三史,及西北輿地,南洋貿遷沿革。尋充總理衙門章京。中日和議成,曾植請自借英款創辦東三省鐵路,時俄之韋特西比利亞鐵路尚未建議也,不果行。母憂歸,兩湖總督張之洞聘主兩湖書院講席。

拳亂啟釁,曾植與盛宣懷等密商保護長江之策,力疾走江、鄂,決大計於劉坤一、張之洞,而以李鴻章主其成,所謂「畫保東南約」也。旋還京,調外交部。出授江西廣信知府,曾植為政,知民情偽,而持之以忠恕,故事治而民親。歷署督糧道、鹽法道,擢安徽提學使,赴日本考察學務。三十二年,署布政使,尋護巡撫。值江、鄂、皖三省軍會操太湖,而適遭國恤,群情忷忷,民一日數驚,城外砲馬兵又譁變。曾植聞之,登城守御,檄協統余大鴻馳入江防,楚材兵艦擊毀東門外砲兵壁壘,黃鳳岐奪回菱湖嘴火藥局,一日而亂定。

曾植在皖五年,重治人而尚禮治,政無鉅細,皆以身先。其任學使,廣教育,設存古學堂。又興實業,創造紙諸廠。會外人要我訂約開銅官山礦,曾植嚴拒之。未幾,貝子載振出皖境,當道命籓庫支巨款供張,曾植不允,遂與當道忤。宣統二年,移病歸。遜位詔下,痛哭不能止。丁巳復闢,授學部尚書。事變歸,臥病海上,壬戌冬,卒,年七十三。著有海日樓文詩集。

論曰:辛壬之際,世變推移,莫之為而為,其中蓋有天焉。潤庠、世續諸人非濟變才,而鞠躬盡瘁,始終如一,亦為人所難者也。乃宣、曾植皆碩學有遠識,惓惓不忘,卒憂傷憔悴以死。嗚呼,豈非天哉!


\end{pinyinscope}