\article{列傳二百五十四}

\begin{pinyinscope}
李秉衡王廷相聶士成羅榮光壽山族孫瑞昌鳳翔崇玉等

李秉衡,字鑒堂,奉天海城人。初入貲為縣丞,遷知縣。光緒五年,除知冀州。歲饑,發倉粟,不給。州俗重紡織,布賤,為醵金求遠遷,易糧歸,而裁其價以招民,民獲甦。越二年,擢知永平府。部議追論劫案,貶秩。李鴻章上其理狀,請免議,不獲。時稱「北直廉吏第一」。以張之洞薦,超授浙江按察使,未到官,移廣西。十年,平峒寨亂,晉二品秩。

明年,法人假越事寇邊,秉衡主龍州西運局。是時財匱,戰士不得餉,蹂尸輿廝,無人過問。秉衡益節儉,汰浮費,無分主客軍,給糧不絕,戰恤功賞力從厚。復創設醫局,治負傷軍士,身自拊循之,日數四,雖末弁,亦延見,殷殷勖以殺敵報國。護撫命下,驩聲若雷動。與馮子材分任戰守。諒山之捷,彭玉麟等疏言:「兩臣忠直,同得民心,亦同功最盛。」予優敘。重申前命為護撫,整營制,舉賢能,資遣越南游眾,越事漸告寧。新任巡撫沈秉成蒞官,乃乞病去。

二十年,東事棘,召為山東巡撫。至則嚴紀律,杜苞苴。以威海、旅順筦鑰北門,遂率師駐煙臺。聞旅順不守,劾罷丁汝昌、龔照嶼等,以警威海守將。既而日軍浮三艦窺登州,秉衡悉萃精兵於西北,而榮城以戎備寡,為日軍所誘而獲,時論詬之。其時大刀會起,主仇教,勢漸張。二十三年,會眾戕德國教士,德使海靖要褫秉衡職,編修王廷相力爭之,徙督四川。海靖請益堅,乃罷免。於是秉衡隱安陽,居三年,剛毅入樞廷,薦之起,入都。廷相慕其名,往訪,遂訂交。朝命秉衡詣奉天按事,奏廷相自隨。既至,糾不職者數人,皆廷相微服所言冋知者。還,會御史彭述疏請整飭長江水師,詔使秉衡往,秉衡固辭,太后責勉之,遂行。

歲餘,拳禍作,枋事者矯詔趣戰,電各省,諸疆臣失措,商之鴻章。於是定畫保東南約,秉衡與焉。無何,又請募師入衛。至京,入覲太后,力主戰,遂命統張春發、陳澤霖、夏辛酉、萬本華四軍,出屯楊村、河西塢。戰才合,張、萬二軍先潰,澤霖自武清移壁,聞砲聲,軍皆走。秉衡不得已,退通州,疾書致各將領,述諸軍畏葸狀,飲金死。事聞,優詔賜恤,謚忠節。聯軍索罪魁,請重治,以先死免議,詔褫職,奪恤典。

廷相,字梅岑,直隸承德人,本籍山東。少劬學,以孝稱。光緒十三年進士,以編修督山西學政。口外七洊饑,有司匿不聞,為上流民殘弊狀,獲賑如腹地。二十三年,轉御史,敢言事。時宗室、覺羅官學久廢不葺,廷相謂培材宜自近始,請依八旗官學新章,求實際,議行。國用患不足,計臣議加賦,廷相力申李鴻藻議,為民請命,事遂寢。二十四年元旦,日食,疏請勤修省,條上七事,而尤以進賢退不肖為國家治亂之源。因劾張廕桓媚外人、交近侍,並以浙江學政徐致祥秩滿調安徽,外似優隆,內實屏絕。嚴旨下吏議,敕還原衙門行走。拳亂起,秉衡出禦聯軍,廷相從。及敗,尋秉衡不遇,還至倉頭橋,赴河死。子履豐,拯之不及,從之,遇救免。贈五品卿,予世職,賞履豐主事。

聶士成,字功亭,安徽合肥人。初從袁甲三軍討捻,補把總。同治初,改隸淮軍,從劉銘傳分援江、浙、閩、皖,累遷至副將。東捻敗,賜號力勇巴圖魯,擢總兵。西捻平,晉提督。光緒十年,法人據基隆,率師渡臺灣,屢戰卻敵。還北洋,統慶軍駐旅順。十七年,海軍大閱禮成,晉頭品秩。調統蘆臺淮、練諸軍,擊熱河朝陽教匪,擒斬其酋楊悅春,賞黃馬褂,易勇號曰巴圖隆阿。明年,授山西太原鎮總兵,仍留蘆臺治軍。請單騎巡邊,歷東三省俄羅斯東境、朝鮮八道,圖其山川厄塞,著東游紀程。

逾歲,日韓亂起,隨提督葉志超軍牙山。聞高升兵艦毀,語志超曰:「海道梗,牙山不可守。公州背山面江,勢便利。」從之。士成乃先諸軍發,次成歡,遇伏,迷失道,吏士無人色。士成見二鶴立岡阜,語眾曰:「彼處無隱兵也!」遂出險,往就志超。志超已棄公州行,追及之。士成議趨平壤合大軍,而鴻章檄令內渡,以故平壤陷,得免議。志超逮問,宋慶接統諸軍,遣士成守虎山。未幾,銘軍潰,諸軍皆走,士成猶悉力以御。日軍大集,力不支,退扼大高嶺。是時遼西危棘,士成請奇兵出敵後截其運道,諸帥不從,乃自率師偪雪裡站而陣。除夕,置酒飲將士,預設伏以待,日軍果來襲,大敗之分水嶺,斬日將富剛三造。優詔褒勉,授直隸提督。

和議成,還駐蘆臺。北洋創立武衛軍,改所部三十營為前軍,與宋慶、董福祥、袁世凱並為統帥。慶、福祥用舊法訓練,世凱軍仿日式,士成軍則半仿德式,是為武衛四軍。

二十六年,拳匪亂,戕總兵楊福同,命士成相機剿辦。匪焚黃村、廊坊鐵軌,士成阻止之,弗應,擊殺數十人。其黨大恨,訴諸朝,朝旨訶責士成。時匪麕集天津可二萬,遇武衛軍輒詬辱,士成檢勒部下毋妄動。榮祿慮激變,馳書慰解之,士成覆書曰:「匪害民,必至害國!身為提督,境有匪不能剿,如職何?」乃鬱鬱駐楊村觀變。會英、法諸國聯軍至,士成三分其軍,一護鐵路,一留蘆臺,而自率兵守天津。連奪陳家溝、跑馬廠、八里臺,徑攻紫竹林,喋血八晝夜,敵來益眾,燃毒煙砲,我軍稍卻。士成立橋上手刃退卒,顧諸將曰:「此吾致命之所也,逾此一步非夫矣!」遂殞於陣,腸胃洞流。詔賜恤。閱二載,以世凱言,贈太子少保,謚忠節,建專祠。

羅榮光,湖南乾州人。初隸曾國籓麾下,補把總。同治初,李鴻章規三吳,從西將華爾克青浦,攻南橋鎮、柘林,直搗其巢,大敗之。乘勝復沙川、金山,遷守備。又從西將戈登釋常、昭圍,以次下太倉、昆山諸邑。累擢參將。攻常州,先登,城復,遷副將,賜號果勇巴圖魯。除狼山鎮右營游擊。蘇軍分援浙、皖、閩,連克湖州、長興、廣德、漳州、漳浦諸城,與有功,擢總兵。六年,東捻擾魯疆,榮光以偏師游弋淮南北,敗捻於運。東捻回竄江、淮,分寇海、沭、邳、宿,並擊退之。明年,西捻窺滑、濬,我師躡之,榮光戰數挫,而勇氣彌厲。鴻章謀困之黃、運間,緣河築長壘,榮光壁當敵沖,相持凡三閱月。會霖雨,寇多陷淖死,榮光復躡之東北,勢益蹙,張總愚自沉於河。事寧,晉記名提督。自是徙防金陵、武昌、西安,凡二年。移駐天津,補大沽協副將。

光緒七年,創設水雷營,遴各營將士演習,兼授化電測量諸學。既而北塘、山海關相繼設,皆受成於榮光。醇親王閱北洋軍,以其教練有方,薦授天津鎮總兵。位漸顯,服食儉約若老兵然。二十六年,擢喀什噶爾提督,未之官而拳亂起,八國兵艦入寇,榮光守大沽砲臺。大沽水深廣,河道縈曲,曲有臺,備險奧,外兵懾其勢,弗敢進。榮光備益嚴,乃佯就款,使人言於裕祿,謂第得四五艘入口護僑商,無他意,裕祿許之。榮光聞而大驚,力阻,而敵艦已踵入,將及臺,遽出砲仰擊。榮光再謁裕祿乞發戰令,諜者已報臺毀,榮光憤極,歸,拔刀殺眷屬,曰:「毋令辱外人手!」遂出赴難,一僕隨之,不知所終。他日得其尸臺下,僕尸亦在焉。沒三日而天津陷,時年六十有七。

壽山,字眉峰,袁氏,漢軍正白旗人,黑龍江駐防,吉林將軍富明阿子。以父任為員外郎,兼襲騎都尉世職,遷郎中。光緒二十年,日軍犯奉天,自請赴前敵,充步隊統領。弟永山領馬隊,數與日軍戰,復草河嶺,克連山關,進薄鳳凰城。敵援至,永山歿於陣,壽山被重創。以敢戰,兼領鎮邊軍馬隊。逾歲,降敕褒嘉。官軍既克海城,壽山領七十騎詣遼南詗敵勢,遇之湯岡子,搏戰,槍彈入右腹,貫左臀出,戰愈猛,敵稍卻,馳還壁,血縷縷滿衣褲。上嘉其勇,遷知府,賞花翎。

二十三年,調充鎮邊軍左路統領,徙駐黑龍江城。越二年,除知開封遺缺知府,未之官,值東北邊防亟,超改黑龍江副都統。明年春,入覲,垂詢邊情甚悉,命佐將軍恩澤治軍。疏請增募十五營,調諳邊事者十餘人,躬詣上海購軍械,自長崎、海參崴、伯利循海歸,潛度形勢,備戰守。新軍成,而恩澤卒於任,朝命代之。既蒞事,鏟奸弊,明賞罰,圖要塞;手訂行陣操法,頒之各將領,使番上,授以方略;雖末弁亦接見,籍記備器使。

二十六年夏,拳亂作,俄軍數千聲為保護哈爾濱鐵軌,紛集海蘭泡,乞假道。壽山曰:「敵偪我都,我假敵道,如大義何!」拒之。遂檄愛琿副都統鳳翔禦北路,呼倫貝爾副都統依興阿御西路,通肯副都統慶祺御東路,令各嚴戒備毋浪戰;並牒俄勿進兵,願負保路責。而俄軍已分道進,重以鐵路土工可十餘萬索值,倡罷工,揚言與俄為難。壽山亟下令軍中曰:「保鐵路,護難民,全睦誼,違者殺無赦!」復使統領吉祥約富拉爾基監工蓋爾肖甫入城,俾釋疑懼,而蓋爾肖甫乃擊殺工人宵遁。壽山猶強為容忍也,慎導俄民出境,籍錄其財物備還,然俄軍不為止,入寇愛琿及黑河屯,華人被迫赴水者,尸蔽江下。

三姓、呼倫貝爾又紛紛告警,壽山亟電吉林將軍長順會攻哈爾濱,然猶囑其語俄總監工,謂若罷兵,願以全家質。當是時,諸路軍皆潰敗,北路統領崇玉,營官德春、瑞昌,西路統領保全,東路營官保林,並陷陣死,於是俄遂偪齊齊哈爾省城。既而聞聯軍媾和,乃遣同知程德全往商和議,而自守「軍覆則死」之義,命妻及子婦先裁,手繕遺疏,猶惓惓於墾政,並致書俄將領囑勿戕民。閱日,具衣冠,飲金,臥柩中,不死;呼其屬下材官擊以槍,不忍,手顫機動,彈出中左脅,猶不死;更呼材官擊小腹,仍不死;呼益厲,又擊之,氣始絕。先是詔責其開邊釁,部議奪職。後以總督徐世昌請復官,予騎都尉兼雲騎尉世職,附祀富明阿祠。

族孫瑞昌,充北路營官,俄陷黑河,與統領崇玉同戰歿。

鳳翔,字集庭,漢軍鑲黃旗人,吉林駐防。累官協領。光緒二十一年,中日事起,將軍長順赴奉督師,鳳翔任餽運,給食不乏。尋擢愛琿副都統。二十六年,俄將固畢乃脫爾來假道,壽山令愛琿戒備。俄軍已自黑龍江下駛,翼日,俄官廓米薩爾名闊利士密德者,浮軍艦至,鳳翔遣軍拒之三道溝。闊利士密德來謁,申前請,弗允,赬怒去,令舟師擊我,而我師已先發,殲其軍官二,闊利士密德被重創,奔還海蘭泡,旋卒。於是黑河軍與海蘭泡俄軍相轟擊者數日。鳳翔令統領王仲良率騎旅三百渡江擊之,始小挫,繼獲大勝。俄軍緣江遁,師往馳之,會其軍艦泊江岸,載歸。閱二日,又渡江來,擊卻之。遲明,又率步旅六千自五道河濟,右路統領崇玉望見之,其軍皆樹我幟,衣我衣,意為漠河護礦兵也,弗敢擊,既登岸始覺,而勢已不可遏,我師敗績,崇玉殞於陣,愛琿陷。壽山聞之,亟令鳳翔回援,弗及。鳳翔駐兜溝子,去愛琿七十里。

逾月,俄軍復至,槍彈雨下,鳳翔以戰為守,相持累日。黑龍江行軍故無棚帳,戰罷露宿,眾苦寒,以是軍有怨聲,鳳翔慮譁潰,復以地勢平衍難扼守,乃請壽山結陣徐退,抵內興安嶺軍焉,去兜溝子又百六十里。未幾,俄軍爭上嶺,勢洶洶,師失利,仍扼嶺拒之。敵攻益亟,鳳翔悉甲出,令曰:「有後者斬!」而自赴前敵督懾。有材官稍卻,立使飛騎斬之。材官懼,大呼陷陣,俄軍少卻,復進,遂大敗,署北路翼長恆玉斷一臂,俄將卒死傷無算。鳳翔戰既酣,右臂左足兩受彈傷,墜馬者三,輒復躍上,鏖戰不少休,既還,嘔血數升而死。事聞,優恤如制。

崇玉,通肯正藍旗佐領。時同死事者,玉慶,黑龍江城世管佐領。城陷被執,詈不絕口,死最慘。扎魯布,黑龍江城水師四品官。懷印以殉,死後猶手握印不可脫。又段國英,宜黃人,以縣丞榷鹽阿什河。俄兵至,令讓所處屯兵,嚴詞拒之,縛而去;旋釋歸,則俄兵已占其地,且懸俄幟,國英大哭曰:「中國亡矣!」觸石,頭裂,死。俄人觀者皆嘆息。

論曰:秉衡清忠自矢,受命危難,大節凜然,此不能以成敗論也。聯軍之占津、海也,長驅而入,唯士成阻之;俄兵之侵龍江也,乘隙以進,唯壽山拒之:固知必不能敵,誓以一死報耳。榮光爭大沽,鳳翔守愛琿,雖已無救於大局,而至死不屈,外人亦為之奪氣,何其壯哉!


\end{pinyinscope}