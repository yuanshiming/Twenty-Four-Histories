\article{列傳一百七十六}

\begin{pinyinscope}
柏葰麟魁瑞常全慶

柏葰,原名松葰,字靜濤,巴魯特氏,蒙古正藍旗人。道光六年進士,選庶吉士,授編修。累遷內閣學士,兼正紅旗漢軍副都統。十八年,出為盛京工部侍郎,調刑部,兼管奉天府尹。二十年,召授刑部侍郎,調吏部,又調戶部。二十三年,充諭祭朝鮮正使,例有餽贐,奏卻之。二十五年,充總管內務府大臣。二十六年,典江南鄉試。疏言:「徵漕大戶短欠,取償小戶,劣紳挾制官吏,大戶包攬小戶,畸輕畸重,旗丁需索,加增津貼諸弊,請嚴禁。」如議行。尋偕倉場侍郎陳孚恩盤查山東籓庫,劾布政使王篤濫用幕友及地方官縱盜,巡撫崇恩以下議譴有差。二十八年,擢左都御史。三十年,遷兵部尚書,授內大臣。尋調吏部,管理三庫,兼翰林院掌院學士。咸豐三年,命偕侍郎善燾赴盛京按協領塔芬布輕聽謠言,調兵護宅,幾至激變,得實,論遣戍。將軍奕興坐袒護,革任。尋以前在鑲白旗蒙古都統任揀選承襲有誤,罷內務府大臣,降授左副都御史。未幾,出為馬蘭鎮總兵。五年,擢熱河都統,搜捕山匪。疏言:「熱河將惰兵疲,州縣不諳吏治。行使大錢,民皆罷市。礦匪占踞山場,委員侵蝕商款。」詔嚴切查辦。召授戶部尚書,兼正黃旗漢軍都統。六年,命在軍機大臣上行走,兼翰林院掌院學士。尋以戶部尚書協辦大學士。八年,典順天鄉試,拜文淵閣大學士。

柏葰素持正,自登樞府,與載垣、端華、肅順等不協。會御史孟傳金疏劾本科士論未孚,命覆勘試卷,應議者五十卷,文宗震怒,褫柏葰等職,命載垣等會鞫,得柏葰聽信家人靳祥言,取中羅鴻繹情事,靳祥斃於獄。九年,讞上,上猶有矜全之意,為肅順等所持。乃召見王大臣等諭曰:「科場為掄才大典,交通舞弊,定例綦嚴。自來典試諸臣,從無敢以身試法者。不意柏葰以一品大員,辜恩藐法,至於如是!柏葰身任大臣,且系科甲進士出身,豈不知科場定例?竟以家人干請,輒即撤換試卷。若使靳祥尚在,加以夾訊,何難盡情吐露?既有成憲可循,即不為已甚,就所供各節,情雖可原,法難寬宥,言念及此,不禁垂淚!」柏葰遂伏法。

十一年,穆宗即位,肅順等既敗,御史任兆堅疏請昭雪,下禮、刑兩部詳議,議上,詔曰:「柏葰聽受囑託,罪無可辭。惟載垣、端華、肅順等因律無僅關囑託明文,比賄買關節之例,擬以斬決。由載垣等平日與柏葰挾有私仇,欲因擅作威福,竟以牽連蒙混之詞,致罹重闢。皇考聖諭有『不禁垂淚』之語,仰見不為已甚之心。今兩宮皇太后政令維新,事事務從寬大平允。柏葰不能謂無罪,該御史措詞失當。念柏葰受恩兩朝,內廷行走多年,平日勤慎,雖已置重典,當推皇考法外之仁。」於是錄其子候選員外郎鍾濂賜四品卿銜,以六部郎中遇缺即選。鍾濂後官盛京兵部侍郎。

麟魁,字梅谷,索綽羅氏,滿洲鑲白旗人。道光六年二甲一名進士,選庶吉士,散館改刑部主事,遷中允。歷庶子、侍講學士、詹事、通政使、左副都御史。十七年,出為盛京刑部侍郎。十八年,召授刑部侍郎,兼鑲紅旗漢軍副都統。二十年,署倉場侍郎。命偕侍郎吳其濬赴湖北按事,劾總督周天爵酷刑,罷之,其濬留署總督。麟魁復往江西鞫鬧漕京控之獄,及江蘇邳州知州賈輝山被劾濫用非刑等事,並治如律。調戶部,又調吏部,充總管內務府大臣。二十二年,出署山東巡撫。英兵犯江南,疏陳登州突出黃、渤,三面環海,敵兵船砲堅利,計難與爭,請移兵扼陸路險要。尋偕侍郎王植赴湖南鞫獄,並勘湖南、江蘇、山東水災,奏請蠲緩,如所請行。二十三年,擢禮部尚書,管理太常寺、鴻臚寺。河決中牟,命偕尚書廖鴻荃往督工,東西兩壩成而屢蟄,褫職,予七品頂戴,仍留工,以料缺水增請緩,復褫頂戴。召還,予三等侍衛,充葉爾羌參贊大臣,調烏里雅蘇臺參贊大臣。

二十七年,召授禮部侍郎,調刑部。二十八年,復授禮部尚書,兼翰林院掌院學士。以前在山東收受陋規,降三級調用,予副都統銜,充烏什辦事大臣。咸豐元年,疏陳時事,略曰:「廣西逆匪,勞師糜餉。其始不過星星之火,當時牧令茍安畏事,諱盜不言;久之蒂固蔓延,養成巨患。請飭封疆大吏嚴查地方,如有教匪、土匪聚眾以及搶劫,隨時查拏,視緝捕之勤惰以為勸懲。近開捐例,實朝廷萬不得已之舉,各省清查,屢經申令。宜飭部臣按時詳覈徵解多寡,實行賞罰章程,俾生愧奮。否則名託清查,事仍敷衍,國儲不裕,官紀益荒,甚非朝廷澄清吏治之意。」奏入,下所司議行。授察哈爾副都統,召為戶部侍郎。

二年,命在軍機大臣上行走,擢工部尚書。三年,調禮部,充總管內務府大臣,罷直軍機,調刑部。八年,復調禮部,補內大臣。十年,因謝恩摺失檢,降授刑部侍郎。是年秋,車駕幸熱河,命署右翼總兵,充巡防大臣。英法兵入京師,麟魁部勒僚屬,戒都人守望相助,令家人閉戶厝薪,曰:「事急即燔!」自宿於巡防廨中,相持數月。和議成,赴行在,籥請回鑾,為載垣、端華、肅順等所阻。十一年,遷左都御史,兼正白旗蒙古都統,尋授兵部尚書。同治元年,協辦大學士。時方奉命偕尚書沈兆霖赴甘肅按事,至蘭州,數日遽卒,詔依大學士例賜血⼙,賜其子恩壽舉人,謚文端。恩壽,同治十三年進士,官至陜西巡撫。

瑞常,字芝生,石爾德特氏,蒙古鑲紅旗人,杭州駐防。道光十二年進士,選庶吉士,授編修。大考二等,六遷至少詹事。二十四年,連擢光祿寺卿、內閣學士。二十五年,遷兵部侍郎,兼鑲紅旗漢軍副都統。二十九年,充冊封朝鮮正使。調吏部,歷兼左、右翼總兵。咸豐元年,典江南鄉試,就勘徐州豐北河決,疏陳災情、賑務、漕務,請飭地方官嚴防匪徒蠢擾,報聞。定郡王載銓管步軍統領,越次題升主事,瑞常力爭不得。尋解左翼總兵職。七年,擢左都御史。八年,遷理籓院尚書,兼正藍旗漢軍都統,署步軍統領,調刑部尚書。十年,寶源局監督張仁政因侵蝕畏罪自盡,命瑞常偕尚書沈兆霖按之,得前任監督奎麟、瑞琇贓私狀,並論大闢,追贓後遣戍。文宗幸熱河,留京辦事,督防巡防。十一年,調工部,又調戶部。

同治元年,以吏部尚書協辦大學士。皇太后命南書房、上書房翰林纂輯史事以昭法戒,書成,賜名治平寶鑒,遴擇大臣輪班進講,瑞常與焉。四年,充總管內務府大臣。時陜西巡撫劉蓉驟起膺疆寄,為編修蔡壽祺所劾,蓉自陳辯,疏中引及胡林翼密薦之詞,又倚任布政使林壽圖,為人所忌。言官遂劾壽圖湎酒廢事,舉劾不公,並訐蓉漏洩之罪,於是命瑞常偕尚書羅惇衍往按之,疏白其無罪,惟坐壽圖演戲及蓉陳奏失當,並予薄譴。定陵奉安禮成,題神主,加太子少保。歷工部、刑部尚書,兼翰林院掌院學士,管理戶部三庫。六年,赴天津驗收漕糧,復命盤查北新倉,得虧米六萬餘石狀,論所司罪如律。十年,拜文淵閣大學士,管理刑部。

瑞常歷事三朝,端謹無過,累司文柄,時稱耆碩。十一年,卒,贈太保,祀賢良祠,謚文端。子文暉,官至盛京禮部侍郎。

全慶,字小汀,葉赫納喇氏,滿洲正白旗人,尚書那清安子。道光九年進士,選庶吉士,授編修,累遷侍講。大考二等,擢侍讀學士。歷少詹事、詹事、大理寺卿。以誤班鐫級。二十一年,予頭等侍衛,充古城領隊大臣,調喀喇沙爾辦事大臣。召還,未行,會回疆興墾,伊犁將軍布彥泰疏留全慶偕林則徐往勘。二十五年,至葉爾羌,疏言:「和爾罕地膏腴,哈拉木札什水渠可資灌溉。又巴爾楚克為回疆扼要之地,道光十二年已奏開墾屯田,未種者尚多,應先侭安插民戶,俾成重鎮。」詔如所請行。先是,全慶疏陳喀喇沙爾環城荒地,及庫爾勒、北山根,可墾田萬餘畝,命辦事大臣常清籌辦。至是復偕則徐詳勘,疏言:「庫爾勒應於此大渠南岸接開中渠,引入新墾之地,分開支渠二。其北山根展寬開都河龍口,別開大渠,與舊渠並行;再分支渠四,別開退水渠一。」又疏言:「伊拉里克在吐魯番托克遜軍臺西,地平土潤,土人謂之『板土戈壁』。其西為『沙石戈壁』,有大小阿拉渾兩水,匯為一河。此次引水自西而東,鑿成大渠,復多開支渠以資灌溉。伊拉里克西南沿山為蒙古出入之路,墾地在滿卡南附近,東西兩面,以『人壽年豐』四字分號,各設正副戶長一,鄉約四,擇誠實農民充之,承領耕種。又吐魯番為南北樞紐,應安置內地民戶,戶領地五十畝,農田以水利為首務。此次開渠,自龍口至黑山頭,地勢高低,碎石夾沙,渠身易淤,酌定經久修治章程。」並如所請行。自是回疆南路凡墾田六十餘萬畝。

回京,擢內閣學士,兼正紅旗漢軍副都統。歷刑部、吏部、戶部、倉場侍郎。咸豐四年,擢工部尚書,兼正紅旗漢軍都統。七年,調兵部。九年,命赴天津驗收漕糧。時英兵犯大沽,僧格林沁擊卻之。全慶疏陳兵事,略謂:「敵軍戰敗之後,不進不退,心實叵測。竊恐別有舉動,未必從此就撫而去。我之精銳,盡萃大沽,旁無應援,後無擁護。雙港之旅,已調前敵;津門之備,但資土練;北塘一帶,又頗空虛。應請速簡重臣,發勁旅,嚴近畿海口之備,為僧格林沁之援,令廣東義勇搗香港以牽其援兵,登州水師合旅順以截其歸路,然後國威可振,撫局可成。」疏入,被嘉納。調吏部尚書。

十年,授內大臣,兼翰林院掌院學士。十一年,充總管內務府大臣。同治元年,追論大學士柏葰科場之獄原讞未允,全慶坐附和定讞,鐫四級,降授大理寺卿。歷內閣學士、工部侍郎、左都御史。五年,授禮部尚書,調刑部。十一年,協辦大學士,兼翰林院掌院學士。十二年,典順天鄉試,以中式舉人徐景春試卷疵謬,鐫二級去職。

全慶易又歷清要,累掌文衡,更閱四朝,雖屢黜,尋即錄用。光緒元年,授內閣學士。復歷禮部侍郎、左都御史、刑部尚書、協辦大學士。五年,鄉舉重逢,加太子少保。六年,拜體仁閣大學士。七年,致仕,食全俸。八年,卒,晉贈太子太保,祀賢良祠,謚文恪。

論曰:自道光以來,科場請託,習為故常,寒門才士,為之抑遏。柏葰立朝正直,且所不免,其罹大闢也,出於肅順等之構陷。然自此司文衡者懍懍畏法,科場清肅,歷三十年,至光緒中始漸弛,弊竇復滋,終未至如前此之甚者,實文宗用重典之效,足以挽回風氣也。麟魁、瑞常、全慶皆起家文學,洊陟綸扉,其建白猶有可紀焉。


\end{pinyinscope}