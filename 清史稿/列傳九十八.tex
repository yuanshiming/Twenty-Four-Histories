\article{列傳九十八}

\begin{pinyinscope}
哈攀龍子國興任舉冶大雄馬良柱本進忠劉順

哈攀龍,直隸河間人,其先出回部。乾隆二年一甲一名武進士,授頭等侍衛。以副將發福建,除興化城守副將。遷總兵,歷河南南陽,福建海壇、漳州諸鎮。以母喪去官。十三年,高宗東巡,攀龍迎鑾,命往金川,隸總督張廣泗軍,署松潘鎮總兵。出美諾溝,取撒臥山、大松林、噶達諸寨。分兵出馬溝右梁,察形勢,得其險要,搜截松林,賊蔽松設卡。毀其二,徑左梁山溝,砲斃賊數十。進克渴足寨,焚碉寨四、水城一,殺賊二十餘。尋與署重慶鎮總兵任舉合兵攻色爾力石城,舉沒於陣。攀龍入林,殪賊三十餘,奪舉尸回。復偕都統班第、署重慶鎮總兵段起賢、侍衛富成分道夜襲色爾力,焚木卡三,殺賊五十餘。進破石梁、雙溝諸壘。經略訥親、總督張廣泗劾攀龍攻色爾力不能下,兵部議左遷。上責攀龍自陳,攀龍言屢克卡殺賊報廣泗,廣泗不以入告。會訥親、廣泗皆得罪去,上知攀龍枉,命罷議。尋從經略大學士傅恆夜攻色爾力,先登,拔石卡,殪賊數十。十四年,金川事定,命署固原提督。十六年,移湖廣提督,陳整飭弁兵諸事,上嘉勉之。尋命真除。復移貴州提督。入陛見,病留京師,卒。

子國興,乾隆十七年武進士,授三等侍衛。出為雲南督標右營游擊,遷東川營參將。緬甸頭人召散據孟艮為亂。總督楊應琚檄國興佐軍,戰楞木,進克猛卯,督戰被槍,創右輔及臂。應琚以聞,賜孔雀翎。尋署騰越營副將。時副將趙宏榜以偏師深入,與緬人戰於新街,師敗績。國興師至蠻暮,詗新街無備,督兵潛入,緬人乃引退。從將軍明瑞進克木邦,戰於蠻暮,大破之。復偕侍衛莽克察擊斬守隘賊六十餘。擢楚姚鎮總兵。入陛見,命在乾清門行走,賚銀幣。還軍,移普洱鎮總兵,遷貴州提督。經略傅恆議用水師,令國興赴銅壁關外野人山督造船。移雲南提督,加太子少保。船成,從傅恆出猛拱、孟養、南豐、猛烈、猛壩,次老官屯。緬人水陸備甚固,攻之不時下。頭人諾爾塔以其酋懵駁命,遣使得魯蘊詣軍乞解兵。傅恆令國興出見,曉以利害,令具約十年一貢,毋更擾邊,歸所掠內地人。緬人誓奉約。時傅恆方病,將軍阿桂召從征諸大臣議,皆言許之便,遂與定約解兵。既而貢弗至,總督彰寶遣都司蘇爾相諭意,留不遣,揚言國興許以木邦、猛拱、蠻暮三土司予緬人,請如議。彰寶劾國興與緬人議具約不以實,上召國興至京師,詰國興,國興自陳未嘗有此議。上責國興遷就畢事,奪太子少保,左授貴州古州鎮總兵。移雲南臨元鎮。後二年,得魯蘊復至老官屯,請如前誓三事。

時師征金川,上命國興從將軍溫福進討。三十七年,遷西安提督,命盡護陜西、甘肅從征諸軍。尋令偕總兵董天弼自曾頭溝取底木達、布朗郭宗。溫福以國興能軍,令自策卜丹徑取美諾當一面。國興自阿喀木雅山溝紆道徑瑪爾迪克山寨,察策卜丹地勢,林深徑狹,不宜於行師,乃將二千人佐海蘭察攻瑪爾迪克。溫福再疏聞上。金川賊千餘屯貢噶山左,謀劫糧,國興馳擊,賊敗匿。師還,經瑪爾迪克,賊自林中出,復擊敗之,上賚荷包四。進攻貢噶山,設伏,斬賊百餘,搜箐奪碉卡。九月,金川酋索諾木使詣國興,請獻鄂克什地以降。國興令並割南北兩山美美卡、木蘭壩及瑪爾迪克。越日,賊盡撤諸柵。國興以兵入鄂克什舊寨,賊退守路頂宗。十月,使歸墨壟溝師敗時所掠外委臧儒,且言嘗勸僧格桑同降。溫福以聞,上令國興檄諭索諾木聲其罪。時國興及海蘭察將五千人屯貢噶山,謀攻策卜丹,阻冰雪未進。上命還師攻路頂宗。路頂宗山麓有巨溝,溝源出南山。海蘭察紆道出山後,侍衛額森特自小徑為應,國興前越溝攻碉。師繼進,遂克路頂宗,破卡五十餘、碉三百餘,俘獲甚眾。復自喀木色爾北山攻穆拉斯郭大寨,進據兜烏山巔,與總兵馬彪軍合,奪附近碉卡,克額爾奔木柵。復將千人渡水,自南山鄂爾濟仰攻,克諸寨,與大軍會,進攻明郭宗。別以兵襲擊公雅山,克木爾古魯寨,並奪據嘉巴山麓。廷議既定小金川,分命將帥三道進討金川。上曰:「國興雖綠營漢員,熟軍事;又嘗為乾清門侍衛,與滿洲大臣無異。」授參贊大臣,佐副將軍豐升額。是月克明郭宗,焚念經樓。整兵進取日果爾烏谷山麓,攻美諾。上嘉國興功,官其子文虎守備。攻克布朗郭宗,僧格桑遁金川。我軍直抵底木達,僧格桑父澤旺出降。小金川平。

國興卒於軍,賜白金千,存恤其家,加贈太子太保,謚壯武。祀昭忠祠,圖形紫光閣。文虎授陜西提標右營守備,從軍攻木果木,陣沒,從祀昭忠祠。復官次子文彪千總。

任舉,山西大同人。雍正二年武進士。以守備發陜西。累遷固原提標左營游擊,署城守營參將。乾隆十一年十二月,固原兵變,夜攻提督許仕盛,毀轅門將入。舉聞亂,單騎詣鼓樓鳴角,招營兵未變者才五十人,部勒使成列。變兵懼,退掠市廛。舉追及,手刃十餘人,擒四十餘人。變兵出城南門,還攻東西二門。舉守東門,右營游擊鐵保守西門,御戰,變兵潰。事定,總督慶復以聞,擢中軍參將。

十二年,命征金川,隸總督張廣泗軍。尋授西鳳協副將。舉至軍,與總兵許應虎、副將高宗瑾、參將買國良攻色底賊碉,擊以砲二百餘發,碉一角圮,垣鑿孔發砲,密如鱗比。舉度我軍砲小不能下,將移軍退守,賊出戰,再設伏敗之。十三年,上諭謂:「在軍諸將狃於瞻對之役,庸懦欺蒙,已成夙習。今別用舉等,皆未從征瞻對,無所掣肘,宜鼓勵勇往。」廣泗亦奏在川鎮將,忠誠勇幹無出舉右者,令率漢、土兵三千取道攻昔嶺。尋又奏令署重慶鎮總兵。

舉與參將王愷自牛廠至素可尼山。時五月,遇大雪,闢道以行。經撒烏山,至昔嶺山梁,山北曰木岡,孤峰當道,賊置城卡守隘。舉督兵攻卡,憑高發砲洞其垣,令土兵緣溝潛進,毀賊碉。師循出山腰,克賊卡,遂陟中峰,以千人駐守,進攻木岡。時總兵哈攀龍師至馬溝右梁,阻松林不得進。廣泗令自納喇溝出昔嶺右,與舉合攻木岡賊所署城卡,力戰未即下。舉察昔嶺左有道通卡撒,中經得思東、木達溝,賊皆置碉焉。總兵冶大雄方自卡撒進,舉與合軍,焚木達溝諸碉,圍得思東,斷其汲道,督兵挾斧斫賊,賊墮巖遁,得大小碉三。進攻色爾力石城,分兵為三道:舉督兵直攻石城,攀龍出其右,副將唐開中及國良出其左。越溝度林,攻賊所置木城,國良戰死。六月己巳,舉與攀龍、開中合攻石城,城堅甚。我師方力攻,賊三百餘自西南林內出,舉督兵與戰,被創;戰益力,槍復中要害,遂卒。攀龍入林殺賊,以其尸還。

時上方命舉真除,經略大學士訥親以舉死事聞,上閱疏為泣下,並諭:「舉忠憤激發,甘死如飴,而朕以小醜跳梁,用良臣於危地,思之深惻!」命視提督例賜恤,加都督同知,謚勇烈,祀昭忠祠,官其子承恩都司,承緒千總。承恩喪終入謝,上以尚幼,命傳諭其母善教之。二十四年,授三等侍衛。累遷福建陸路提督。五十二年,臺灣林爽文為亂,承恩請往討之,師無功,逮詣京師,罪當死,上寬之。五十三年,赦出獄。五十五年,復授巡捕營參將,遷副將。卒。承緒官巡捕營游擊,市中火,赴救被創,卒。上之赦承恩,謂其未有子,承緒又死勤事,不可使舉無嗣也。

冶大雄,四川成都人。康熙季年入伍,從征西藏,克里塘、巴塘,降結敦落籠宗、說板多打籠宗諸寨,獲為亂喇嘛五。雍正初,從軍出松潘黃勝關,剿撫熱當十二部落。攻郭隆寺,攻嶺三,破寨十五,追斬康布喇嘛於西海。又從征桌子山、棋子山,戮頭人。追剿羅卜藏丹津,擒丹津琿臺吉。川陜總督岳鍾琪疏薦,引見,特授藍翎侍衛。累遷陜西莊浪營參將。加副將銜,賜孔雀翎,命赴巴里坤軍,檄署川陜標中軍副將。

準噶爾犯克什圖、峨侖磯諸卡倫。大雄偕總兵樊廷以二千人當賊二萬,轉戰七晝夜,拔守卡倫兵以出。與總兵張元佐等師會,力戰殺賊。賜拜他喇布勒哈番世職,賚白金五千。尋授直隸山永協副將。命署湖北彞陵鎮總兵。上言:「彞陵距省千餘里,兵餉歲以四季支給,請改夏秋、冬春二次匯支。」下督撫議行。尋調署山西大同鎮總兵。與前任總兵李如栢互劾,均奪職。乾隆元年,以副將發湖廣,尋授衡州協副將。城綏苗、瑤為亂,大雄駐長安堡,焚賊寨,戮其渠,餘相率就撫。擢鎮筸鎮總兵。總督孫嘉淦劾大雄貪縱,奪職。湖南巡撫蔣溥言讞無貪縱跡,引見,復授雲南昭通鎮總兵。敘剿苗功,加都督僉事銜。

十三年,從征金川,至卡撒,統云南、貴州諸軍進攻色底、光多諸寨。引兵出昔嶺中峰之西,與署總兵哈攀龍、任舉師會,克大小碉十、石城一,墮碉百三十。同攻克昔嶺溝底石城水卡。經略大學士傅恆奏大雄歷經戰陣,令總理營壘,措置妥協,賜孔雀翎。金川頭人莎羅奔等乞降,師還。授雲南提督,加左都督銜。入覲,官其子繼鈞藍翎侍衛,命送大雄上官。疏言:「西藏喀拉烏蘇諸地與準噶爾連界,盜竊紛擾,是其故習。今藏北鄙即我邊地,防邊自可弭盜。請駐藏大臣仍設重兵,循大道置臺站,以資防守。」上嘉其留心。

繼鈞至常德迎家,中途假回民金,大雄以聞。上以大雄知事不可揜乃始奏劾,左授哈密總兵。命署安西提督,赴巴里坤驗馬駝,疏報四千餘。會總督方觀承核參將鍾世傑等至巴里坤領馬千九百餘,途中馬多死,論罪。上以大雄疏不實,下部議;總督黃廷桂復劾大雄,命奪官,逮京師治罪。二十一年四月,行至西安,卒。三十二年,上以綠營世職不得世襲罔替,下兵部察諸將有功者,俟襲次畢,賜恩騎尉世襲罔替,大雄與焉。

馬良柱,甘肅張掖人,其先本回部。康熙季年,從軍征吐魯蕃;雍正初,將軍阿爾邦檄赴插漢麥裡幹討賊:皆有功。復從安西鎮總兵孫繼宗攻羅卜藏丹津,降臺吉三十三。戰於哈馬兒打布罕噶斯,擒其渠,授藍翎侍衛,賜白金百,遷三等侍衛。外授四川提標游擊,賜貂皮、數珠。命將兵屯西藏。旋以兵擾民,左降,聽四川巡撫、提督調遣。

八年,瞻對土司為亂,提督黃廷桂檄良柱討之。賊堅守石碉,督兵仰攻,槍殪所乘馬,易馬進,再殪,乃步行督戰。碉上投石如雨,傷面,搏賊益奮,火其碉,並焚擦馬、擦牙諸寨,殲賊無算。側冷邦諸頭人皆降。復授松潘鎮左營游擊。三遷夔州協副將。

乾隆十年,師復徵瞻對,破直達、松多諸寨,奪碉七十餘。進攻下密左山梁,獲頭人噶籠丹坪。再進克下密等百餘寨,獲頭人塔巴四交。渡丫魯河,遂破瞻對,焚其寨。其渠姜錯太死於火。十二年,大金川酋莎羅奔攻革布什咱土司,並掠明正土司所屬魯密、章穀諸地。巡撫紀山移良柱威茂協副將,督兵防禦。莎羅奔糾小金川土司澤旺侵沃日各寨,都司馬光祖赴援,賊大至,光祖困於熱籠。良柱率輕騎馳救,敗賊巴納山,進克石卡二百二十三。光祖等出應,賊潰,圍解。澤旺降,並還所侵沃日三寨。詔嘉其奮勇,遷重慶鎮總兵。再進復孫克宗官寨,攻江卡,戰屢勝,克大小碉寨百餘,降二十餘寨。進克丹噶山,分兵焚撒籠等七寨,噶固等寨先後降。賊守石達大碉,良柱冒雨進,數十戰,賊乘夜來撲營,設伏,殲焉。馬邦頭人思錯已降,總兵許應虎馭之不以道,復叛,圍應虎於的交,良柱馳救。賊退入戎布寨,攻之未下。旋復犯馬邦,副將張興被圍。良柱請移戎布師赴援,總督張廣泗不許,興陷於賊。侵噶固,守兵叛附賊,奪卡倫七。廣泗令良柱往攻,力戰,賊未卻。值大雪二十餘日,糧匱,煮鎧弩以食。力不支,廣泗檄退師。倉卒移營,砲械為賊得。

廣泗劾之,命逮詣京師,良柱陳糧絕狀,上特原之。命在香山教禁軍雲梯,親臨觀之。良柱起舞鞭,稱旨,賜大緞、荷包。命仍赴金川軍,以副將、參將等官酌量委用。尋授泰寧協副將,大學士傅恆視師,檄良柱攻昔嶺,克之。莎羅奔請降,良柱以十餘騎入其營宣諭。授建昌鎮總兵,賜孔雀翎。母憂去官。召入京師,仍令教禁軍習雲梯。服闋,授松潘鎮總兵。雜穀土司蒼旺為亂,偕提督岳鍾琪討平之。尋請老,改籍四川華陽。卒,年八十一。

良柱額然,大目虯髯,邊人畏之,號為獅子頭。善戰,臨陣手鐵鞭一,馬上旋轉如飛。其攻噶固,廣泗不為策應,餉又不時至,上知廣泗忌其成功,故特輕其罰云。子應詔,官直隸河間副將。孫瑜,自有傳。

本進忠,甘肅西寧人。初入伍,冒姓名曰張元吉,尋請復姓名。雍正中,從揚武將軍張廣泗援吐魯番,屯魯克沁。準噶爾來侵,邀擊,擒賊七。復追敗之哈喇和卓。乾隆十三年,檄赴金川,從征囊得山梁。攻碉先登,奪矛,中石,傷。從攻普沾,擲火彈入碉,焚碉十三,奪木城。進戰於樂利噶爾堤克,殪賊。攻碉,右股中槍,傷。錄功,擢四川威茂協右營都司。引見,賜大緞。雜穀土司蒼旺為亂,提督岳鍾琪檄進忠討之,奪銅砲一,斬馘數十,生擒二十五,降茶堡番民二千餘。自角木角溝入雜穀,獲蒼旺。累擢永寧協副將。

三十年,從將軍明瑞徵緬甸,進攻蠻結,克木卡十六,殪賊三,傷額,明日,仍裹創出戰。事聞,賜孔雀翎,號法式善巴圖魯。擢雲南臨元鎮總兵。明瑞令將五千人屯龍陵關備調遣。召詣熱河行在,入見,命乾清門行走,賜貂皮、銀幣,令還軍。旋移普洱鎮總兵,擢雲南提督。卒,加太子太保,謚勤毅。

劉順,順天人。雍正五年武進士,授藍翎侍衛。以守備發陜西。累遷至金塔協副將。乾隆十三年,令將千五百人赴金川,偕副將高雄自甲索攻囊得,道松林。賊百餘出戰,擊之遁,毀賊碉。從大軍自卡撒左山梁進,諸碉以次皆下。惟普瞻雙、單二碉守甚堅。日暮,將收兵,順潛率所部逼單碉,縱火攻之,賊潰,並奪雙碉。師繼進,遂克色底。普瞻西有山曰阿利,賊碉林立。順冒雨奮攻,奪山梁木卡,破碉。發砲,殪賊數十,復破大碉一、石卡四。

經略訥親屢奏順奮勇。金川平,擢貴州威寧鎮總兵。上以順熟邊情,移甘肅西寧鎮總兵。入見,賜孔雀翎。擢安西提督。病,乞罷。卒,加太子太保,謚壯靖。

論曰:初征金川,攀龍、舉、大雄皆以勇略著。舉尤驍桀為軍鋒,訥親、張廣泗督戰急,鼓銳攻堅,遂以身殉,傷已!良柱善戰,又以廣泗牽制,不能盡其材。進忠、順力戰破堅碉,亦攀龍輩之亞也。


\end{pinyinscope}