\article{列傳一百五十二}

\begin{pinyinscope}
覺羅寶興宗室敬徵宗室禧恩陳官俊卓秉恬

覺羅寶興,字獻山,隸鑲黃旗。嘉慶十五年進士,選庶吉士,授編修。累遷少詹事,入直上書房。十八年,仁宗幸熱河,林清逆黨突入禁城,寶興散直,至東華門與賊遇,急入告警。宣宗方在上書房,聞警戒備,賊不得逞。上還京,擢寶興內閣學士。十九年,授禮部侍郎。以事忤旨,詔斥寶興不學,降大理寺卿,罷直書房。復坐部刊科場條例誤「高宗」為「高祖」,降二級調用。尋予三等侍衛,充吐魯番領隊大臣。

道光二年,召為大理寺少卿。復因事降通政司參議,歷左副都御史、兵部侍郎,出為泰寧鎮總兵。八年,授理籓院侍郎,調兵部。迭命偕戶部尚書王鼎察治長蘆、兩淮鹽務,籌議整頓,詳王鼎傳。十年,出為吉林將軍,疏言:「松花江西岸、輝發河北岸舊例封禁,其餘閒曠山場均設卡倫,惟許兵丁打捕牲畜,以備貢品。民人無照,私出挖蓡斫木者,查拏治罪。」又言:「伯都訥珠爾山荒田先後開墾五千二百六十二晌,其租息請自道光十五年為始,以其半分賞兵丁,半存備報修工程。此外尚有可墾荒地五萬六千餘晌,作為官荒,將來奏請招佃徵租。烏拉涼水泉已墾七萬三千九百餘晌,請撥二道河東二萬晌,以七成給烏拉總管衙門,三成給協領衙門,資為津貼。餘未墾地五萬三千餘晌,亦作官荒。」並從之。調盛京,又調成都。

十七年,署四川總督,逾年實授。時馬邊、越巂邊外夷匪數出為患。十九年,疏言:「御邊之策,不外剿、撫、防三者,撫之之道,在施於平時,斷無失利之後轉而就撫之理。比來勞師糜餉,迄無成功。為今計者,以修邊防為急務,陳防邊五事:一、增兵額,請於馬邊增兵千二百,雷波、普安、安阜、越巂、寧越各增兵八百,瓘邊、屏山各增兵四百;一、改營制,請以綏定協副將移駐馬邊城,游擊、都司以下各增設移駐有差;一、築碉堡,飭各縣因地制宜,多修堡寨,責令各集團練,官給抬砲,督率教演,擇要隘築砲臺,增設大砲;一、定期巡閱,歲春夏之交,建昌道赴越巂、瓘邊,永寧道赴馬邊、雷波、屏山,周歷巡閱各一次,秋冬責成提督與建昌總兵分赴巡行察勘邊隘;一、優獎邊吏,馬邊、越巂兩同知,請三年俸滿,以題調選缺知府升補。」疏下議行。言官論奏四川提督應如湖南例,半年駐越巂等處。寶興議:「馬邊、越巂相距遼遠,請於春秋夷匪出沒之時,提督往駐馬邊、瓘邊、雷波三,建昌總兵往駐越巂、寧越。」又言:「越巂邊防以大路為重,麥子營、利濟站均應增駐弁兵,乾溝諸汛應酌量移撤,分設於馬日槓諸處。越巂、寧越兩營相距頗遠,聲勢不能相及。前請以建昌左營游擊移駐大菩薩地,遠在寧越之東,而越巂營參將復與游擊不相統屬。請越巂、寧越適中之界牌樓,以建昌鎮右營都司移駐,專管麥子營、利濟站兩汛。」並從之。

先是寶興以馬邊諸縣增設防兵,籌議邊防經費,請按糧津貼,計可徵銀百萬兩,以三十萬為初設防兵之需。每歲經費,即以餘銀七十萬兩生息,置田供支。上以津貼病民,撥部帑銀百萬。翰林院侍讀學士王炳瀛奏:「四川前買義田,遍及百餘州縣,若更以數十萬帑銀於各州縣買田收租,膏腴將盡歸公產。請限於四近邊地收買,安置屯防。」下寶興妥議,疏言:「邊防完竣,用銀二十二萬兩有奇,以三十七萬發鹽茶各商,歲得息三萬七千餘兩,足敷增設練勇餉械之需。餘銀四十萬,聽部撥別用。」遂罷買田議,二十一年,拜文淵閣大學士,留四川總督任。時大學士琦善、協辦大學士伊里布相繼罷,在朝滿洲大臣鮮當上意,故有是授。二十六年,入覲,命留京管理刑部,充上書房總師傅,兼翰林院掌院學士。二十八年元旦,加恩年老諸臣,加太保。十月,卒,年七十二,謚文莊。

宗室敬徵,隸鑲白旗,肅親王永錫子。嘉慶十年,封輔國公,授頭等侍衛,兼委散秩大臣、副都統。十九年,授內閣學士,兼鑾儀使,充總族長。二十二年,失察宗室海康等習紅陽教,褫職,謫居盛京。尋予四等侍衛,乾清門行走。道光初,累遷工部侍郎,授內務府大臣,調戶部。八年,偕尚書王鼎察治長蘆鹽務,奏定歸補帑課章程,詳王鼎傳。十二年,南河奸民陳堂等盜決於家灣官堤,命偕尚書硃士彥往勘。疏陳:「諸口已合,壩下尚未閉氣,間有蟄陷。陳堂等聽從逸犯陳端糾眾,以為從例問擬,疏防各官遣戍。通判張懋祖賠修壩工不實,罰賠枷號。覆勘湖河各工,請擇要興修,高堰、山盱卑矮石工,分年改砌碎石;信壩補還石工,智壩、仁河、義河壩改修石底;里河福興閘塌卸,急築;揚河西岸加高磚工,改拋碎石。」並從之。又會同兩江總督陶澍議定淮鹽票引兼行,言官所論官票運私、侵礙暢岸、爭占馬頭三者皆可無慮,詔如原議行。

十四年,授左都御史。偕侍郎吳椿勘浙江海塘,疏言:「念里亭至尖山柴工尚資御溜,石塘仍當修整,鎮海及戴家橋汛議改竹簍,塊石不如條石坦水舊法為堅實。烏龍廟以東,冬工暫緩。」回京,擢兵部尚書,調工部。十五年,以孝穆皇后、孝慎皇后梓宮奉安龍泉峪,諏日不慎,罷尚書、都統,仍充內務府大臣。十六年,署戶部侍郎,累遷工部尚書,兼都統。東河總督慄毓美多用磚工,御史李蓴言其不便,命敬徵偕蓴往勘。疏陳:「已辦磚工尚屬整齊,輿論謂保灘護崖可資其力。水深溜急之處,不及埽工鞏固,搶辦險工,未可深恃。請停止燒磚,改辦碎石。」從之。十八年,調戶部。

二十二年,南河揚河漫口,水由灌河入海。有議即改新河,河督麟慶以河流未定,遽難決議,命敬徵偕尚書廖鴻荃往勘。疏言:「改河之議,在因勢利導。今查灌河海口至蕭莊口門三百六十餘里。新河正溜,由六塘出達灌口,其下游東北一百十里,滔滔直注。惟當潮漲時,黃水相逼,壅閼不前,而上游自口門至響水口二百餘里,支流忽分忽合,必須兩岸築堤束水,方免汎濫。計工長三百餘里,經費難籌。且中河運道為黃流橫截,不得不移塘灌運。清水本弱,仍恃借黃以濟。空船引轉需時,重運更形艱滯。是移塘乃權宜之計,常年行之,恐妨運道。舊黃河自蕭莊迄舊海口四百二十餘里,尾閭寬暢。自漫口斷流,河身益淤。若挽歸故道,堵口挑河,共費五六百萬,較改河築堤撙節實多。請定明歲春融興工,俟軍船回空後築壩合龍。」詔如議行。尋以戶部尚書協辦大學士。

二十三年,偕侍郎何汝霖赴南河勘工,又赴河南察視中河漫口。疏陳築壩挑河工費需銀五百十八萬兩,較祥符工費為節省,允之。二十五年,奏:「河南下北河廟工,乃北岸七適中之所,河臣宜常年駐此,便於控制。」詔河督每於伏汛前移駐廟工,立冬後仍回濟寧。尋坐濫保駐藏大臣孟保,降內閣學士。未幾,復授工部尚書。又坐濫保科布多參贊大臣果勒明阿,褫職。三十年,署正白旗滿洲副都統。咸豐元年,卒,詔念前勞,予一品銜,依尚書例賜恤,謚文★。子恆恩,左副都御史;孫盛昱,自有傳。

宗室禧恩,字仲蕃,隸正藍旗,睿親王淳穎子。嘉慶六年,賜頭品頂戴,授頭等侍衛,乾清門行走。十年,晉御前侍衛,兼副都統、鑾儀使、上駟院卿,轉奉宸院卿,遷內閣學士。十八年,擢理籓院侍郎。二十年,授內務府大臣,調戶部侍郎。二十五年,仁宗崩於熱河避暑山莊,事出倉猝,禧恩以內廷扈從,建議宣宗有定亂勛,當繼位。樞臣托津、戴均元等猶豫,禧恩抗論,眾不能奪。會得秘匱硃諭,乃偕諸臣奉宣宗即位,命在御前大臣、領侍衛大臣上行走。

道光二年,擢理籓院尚書。時哈薩克部眾潛聚烏梁海,議遷徙安置,增設卡倫。吏部尚書松筠諳習邊事,上每垂詢,禧恩因以諮之。松筠素坦率,遂代刪改疏稿。禧恩怒,以上聞,松筠坐越職干預被譴。尋調工部,仍兼署理籓院尚書。六年,調戶部。八年,加太子少保,署吏部尚書。九年,隨扈盛京,詔念睿親王多爾袞數定大勛,加恩後裔,賜禧恩雙眼花翎。

十二年,湖南江華瑤趙金龍作亂,命禧恩偕盛京將軍瑚松額督師,未至,總督盧坤、提督羅思舉已平之,殲金龍。禧恩素貴倨,奉命視師,意氣甚盛,嗛諸將不待而告捷,謂金龍死未可信。思舉以金龍焚骸及佩物為證,議始息。廣東瑤匪趙仔青竄入湖南,率提督餘步雲、總兵曾勝追剿之;偕巡撫吳榮光疏陳善後事。湖南既定,而兩廣總督李鴻賓剿連山瑤,閱半年,軍屢挫。詔逮鴻賓,以禧恩署總督,由湖南進兵。遣步雲、勝等先後破賊,擒首逆鄧三、盤文理,毀其巢。甫一月,諸瑤乞降。詔嘉其奏功迅速,賜三眼花翎,封不入八分輔國公。班師,途次丁母憂,溫諭慰之。

十三年,孝慎皇后薨,命理喪儀,坐議禮徵引違制,褫御前大臣、戶部尚書、內務府大臣。尋復授理籓院尚書。以生日受屬員饋送,為御史趙敦詩所劾,疏辯得直,敦詩坐譴。十四年,因相度龍泉峪萬年吉地,加太子太保。調兵部尚書,兼署禮部戶部。十八年,詔以南苑牲畜不蕃,禧恩久管奉宸苑,廢弛疏懈,罷其兼領。尋得員司積弊狀,盡罷諸兼職,降內閣學士。二十二年,署盛京將軍,授理籓院侍郎,留將軍署任。英吉利內犯,海疆戒嚴,命治盛京防務。既而和議成,疏陳善後十事,並巡洋章程,如議行。

二十五年,以病解職。坐失察內地民人越朝鮮界墾地,削公爵,降二等輔國將軍。三十年,起署馬蘭鎮總兵、密雲副都統。咸豐元年,召授戶部侍郎。二年,擢戶部尚書,協辦大學士,管理籓院事。尋卒,贈太子太保,謚文莊。

禧恩自道光初被恩眷,及孝全皇后被選入宮,家故寒素,賴其資助,遂益用事。遍膺禁近要職,兼攝諸部,凌轢同列,人皆側目。後晚寵衰,禧恩亦數獲譴罷斥。文宗即位,乃復起,不兩年登協揆焉。

陳官俊,字偉堂,山東濰縣人。嘉慶十三年進士,選庶吉士,授編修,遷贊善。二十一年,入直上書房。大考二等,擢洗馬,累遷右庶子。典陜西鄉試,督山西學政。道光元年,命各省明定陋規,中外臣工多言窒礙,官俊亦疏陳不可行,詔嘉之,予議敘。會密諭留心察訪官吏賢否、政治得失,官俊恃曾直內廷為宣宗所眷,意氣甚張。尋遷侍講學士,命回京,仍直上書房。山西巡撫成格追劾官俊在學政任毆差買妾,妄作威福,大開奔競。上以官俊於毆差買妾已自承不諱,曾薦舉魏元烺、邱鳴泰,人材尚不繆;惟所述太監往河東查訪鹽務控案,事出無稽,解職就質,命長齡道出山西,傳旨面詰成格,亦以不能指實引咎,遂兩斥之。

官俊降編修,罷直上書房。連典貴州、江西鄉試,歷中允、祭酒、侍講學士、內閣學士。十六年,授禮部侍郎,調吏部。十九年,擢工部尚書。東陵郎中慶玉侵帑籍沒,主事全孚預告,多所寄頓。事覺,語由官俊閒談漏洩,回奏復諱飾,詔斥失大臣體,褫職。二十一年,起為通政使。歷戶部、吏部侍郎,管理三庫。擢禮部尚書,調工部。二十四年,以吏部尚書協辦大學士。

官俊再起,歷典鄉會試、殿廷御試,每與衡校。充上書房總師傅。編修童福承素無行,直上書房授皇子讀。給事中陳壇劾之,語及福承為官俊妻作祭文,措詞過當。福承譴黜,詔斥官俊容隱不奏,罷總師傅,議降三級調用,從寬留任。二十九年,卒,優詔賜恤,稱其心田坦白,贈太子太保,入祀賢良祠,謚文愨。賜其孫厚鍾、厚滋並為舉人。

官俊初直上書房,授宣宗長子奕緯讀,宣宗嘉其訓迪有方。後皇長子逾冠而薨,上深以為恫,故遇官俊特厚,屢獲咎而恩禮始終不衰。

子介祺,道光二十五年進士,官編修。咸豐中,助軍餉,加侍講學士銜。後在籍治團練,守城,賑饑,賜二品頂戴。介祺績學好古,所藏鐘鼎、彞器、金石為近代之冠。

卓秉恬,字靜遠,四川華陽人。嘉慶七年進士,選庶吉士,年甫逾冠,授檢討。典陜西鄉試。十八年,改御史,歷給事中,章疏凡數十上。論盜風未息,由捕役與盜賊因緣為奸,捕役藉盜賊以漁利,盜賊仗捕役為護符,民間控告,官不為理,盜賊結恨,又召荼毒;直隸之大名、滄州,河南之衛輝、陳州、山東之曹州、東昌、武定,江蘇之徐州最甚,請飭實力禁懲。巡漕山東,履勘泰安、兗州各屬,探濬新泉四十三處,定名勒石。歷鴻臚寺少卿、順天府丞。

二十五年,疏言:「由陜西略陽迄東至湖北鄖西,謂之南山老林;由陜西寧羌迄南而東,經四川境至湖北保康,謂之巴山老林。地皆磽瘠,糧徭極微。無業游民,給地主錢數千,即租種數溝數嶺。歲薄不收則徙去,謂之棚民。良莠莫辨,攘奪時聞。一遇旱澇,一二奸民為之倡,即蟻附蜂起。州縣以地方遼闊,莫能追捕,遂至互相容隱。迨釀成大案,即加參劾,事已無濟。且事連三省,大吏往返咨商,州縣奉文辦理,恆在數月之後。與其即一隅而專謀之,何如合三省而共議之。請於扼要之地,專設大員控制。」宣宗深韙之,詔下三省會議,未果行,僅將邊境文武酌就要地改駐添設。

道光四年,調奉天府丞,丁父憂去。服闋,歷太僕寺、大理寺少卿,太僕寺卿,宗人府丞,內閣學士,典江南鄉試。十五年,遷禮部侍郎,調吏部。督浙江學政。擢左都御史,召還京,兼管順天府尹事。歷兵部、戶部、吏部尚書、協辦大學士。二十四年,拜文淵閣大學士,晉武英殿。歷管兵部、戶部、工部,賜花翎。咸豐五年,卒,年七十四,贈太子太保,謚文端。

秉恬兼管京尹最久,凡十有八年。時九卿會議,一二王公樞相主之,餘率占位畫諾。秉恬在列,時有辯論,不為用事者所喜。子枟,道光二十年進士,官至吏部侍郎。

論曰:自設軍機處,閣臣不預樞務。始猶取名德較著者表望中朝,繼則旅進旅退之流,且以年資眷睞,馴躋鼎鉉矣。寶興號嫺吏事,而蒙簠簋不飭之聲;敬徵數視河工,差著勞勩;禧恩、陳官俊並恃恩私,崛而復起;卓秉恬以言官進,視緘默自安者稍表異焉。


\end{pinyinscope}