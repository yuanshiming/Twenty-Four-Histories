\article{列傳八十}

\begin{pinyinscope}
李紱蔡珽謝濟世陳學海

李紱,字巨來,江西臨川人。少孤貧,好學,讀書經目成誦。康熙四十八年,成進士,改庶吉士,散館授編修。累遷侍講學士。五十九年,擢內閣學士,尋遷左副都御史,仍兼學士。六十年,充會試副考官。出榜日,黃霧風霾,上語大學士等曰:「此榜或有亂臣賊子,否亦當有讀書積學之士不得中式,怨氣所致。」命磨勘試卷,劣者停殿試。又賜滿洲舉人留保、直隸舉人王蘭生進士。下第舉子群聚紱門,投瓦石喧閧。御史舒庫疏劾,下部議,責紱匿不奏,奪官,發永定河工效力。雍正元年,特命復官,署吏部侍郎,赴山東催漕。尋授兵部侍郎。上令截留湖南等省漕糧於天津收貯,旋又命估價出糶。

二年四月,授廣西巡撫。奏言:「廣西賀縣大金、蕉木二山產礦砂,五十里外為廣東梅峒汛,又數里為宿塘寨,礦徒盤據,時時竊發。臣方擬嚴禁,聞總督孔毓珣條陳開採,因而中止。將來或恐滋事。」毓珣奏同時至,廷議寢其事。上命以諭毓珣者示紱,令協力禁止。紱疏陳練兵,列舉嚴賞罰、演陣法、習用槍砲、豫備帳房鑼鍋諸事,上嘉其留心武備。康熙中,巡撫陳元龍奏請開捐,都計收穀百十七萬石有奇,石折銀一兩一錢,而發州縣買穀石止三錢,不足以糴。至紱上官,尚虧四萬餘石,紱奏請限一月補足。會提督韓良輔條奏墾荒,下紱議,紱請以桂林、柳州、梧州、南寧四府收貯捐穀動支為開墾費。上曰:「朕觀紱意,不過借開墾以銷捐穀。當時陳元龍等首尾不清,朕知之甚詳。應令元龍等往廣西料理。」並諭紱詳察,毋隱諱瞻徇,自承虧空。尋紱奏察出督撫、司道、府分得羨餘銀八十二萬有奇,勒限分償,上嘉紱秉公執正。紱在吏部時,年羹堯子富等捐造營房,下部議敘,不肯從優,為羹堯所嫉;及上命天津截漕估糶盈餘銀五千交守道桑成鼎貯庫,紱至廣西,成鼎使齎以畀紱。紱具摺送直隸巡撫李維鈞會奏。維鈞匿不上,紱乃奏聞。先是,羹堯朝京師,入對,舉此訐紱,謂紱乾沒。上以問維鈞,維鈞言紱取數百金治裝,餘尚貯庫。紱奏至,上謂維鈞與羹堯比,欲陷紱。諭獎紱,命留充公用。

三年六月,紱奏言:「太平、思恩府界流言安南內亂。有潘騰龍者,自言為莫姓後,其黨黃把勢、陳亂彈等煽誘為亂。嚴飭將吏捕治。」上諭曰:「封疆之內,宜整理振作。至於安邊柔遠,最忌貪利圖功,當慎之又慎!」九月,奏:「瑤、僮頑梗,修仁十排、天河三甿為尤甚,常出劫掠。臣遣吏入十排,捕得其渠。三甿阻萬山中,所種田在隘外。臣發兵守隘,斷其收穫。其渠今亦出自歸。」上獎其辦理得宜。

旋授直隸總督。四年,紱入覲。初,左都御史蔡珽薦起其故吏知縣黃振國授河南信陽知州,忤巡撫田文鏡。文鏡馭吏嚴,尤惡科目,劾振國貪劣。紱過河南,詰文鏡胡為有意蹂踐士人。入對,因極言文鏡貪虐,且謂文鏡所劾屬吏,如振國及邵言綸、汪諴皆枉,振國已死獄中。文鏡因紱語,先密疏聞,謂紱與振國同年袒護。紱疏辨,上不直紱,而振國實未死,逮至京師,上更謂紱妄語。良輔奏雲南、廣西所屬土司與貴州接壤者,皆改歸貴州安籠鎮節制,命紱往與雲貴總督高其倬會勘,疏請循舊制,從之。

紱還直隸,時上譴責諸弟允禩、允禟等,更允禟名塞思黑,幽諸西寧,復移置保定,命胡什禮監送。紱語胡什禮:「塞思黑至,當便宜行事。」胡什禮以聞,上以為不可,命諭紱,紱奏初無此語。塞思黑至保定,未幾,紱以病聞,尋遂死。是冬,御史謝濟世劾文鏡貪虐,仍及誣劾振國等。上奪濟世官,下大學士九卿會鞫,戍濟世阿爾泰軍前。上以濟世奏與紱語同,疑紱與為黨,召紱授工部侍郎。紱在廣西捕亂苗莫東旺置天河縣獄,獄未竟,紱移督直隸去。久之,蠻、僮集眾破獄,劫東旺去。五年春,良輔署廣西巡撫,奏聞。上以詰紱,下部察議。會都察院奏廣西州判程旦詣院訴土司羅文剛掠村落抗官兵,上責紱與繼任巡撫甘汝來逡巡貽害,命紱與汝來至廣西捕治,不獲,當重譴。紱至廣西,東旺聞而自歸,文剛亦捕得。直隸總督宜兆熊劾知府曾逢聖、知縣王游虧空錢糧,上以逢聖、游皆紱所薦,命詰紱。戶部議覆,紱在直隸奏報懷來倉圮,穀為小民竊食,當下直隸總督詳察。上曰:「穀至六千餘石,豈能竊食至盡?明系紱市恩,為縣吏脫罪。當責紱償補,以成其市恩。」兆熊又劾知縣李先枝私派累民,上以先枝亦紱所薦,責紱欺罔,奪官;下刑部、議政大臣等會鞫,紱罪凡二十一事,當斬。上諭曰:「紱既知悔過,情詞懇切,且其學問尚優,命免死,纂修八旗通志效力。」

七年,又以順承郡王錫保奏濟世在阿爾泰供言劾文鏡實受紱及珽指,下紱等刑部。會曾靜、張熙獄起,上召王大臣宣諭,並命紱入,諭曰:「朕在籓邸,初不知珽、紱姓名。有馬爾濟哈者,能醫。朕問:『更有能醫者否?』以珽對。召珽來見,珽謂不當與諸王往來,辭不至,以是朕重之。年羹堯來京,亟稱珽,朕告以嘗招之不來,羹堯以語珽,珽復辭不至,以是朕益重之。及出為四川巡撫,詣熱河行在,始與相見,為朕言李紱。朕知紱自此始。既即位,延訪人才,起紱原官。旋自侍郎出撫廣西,至為直隸總督,徇私廢公,沽名邀譽,致吏治廢弛,人心玩愒。又如塞思黑自西大通調回,令暫住保定。未幾,紱奏言遘病,不數日即死。奸黨遂謂朕授意於紱,使之戕害。今紱在此,試問朕嘗授意否乎?塞思黑罪本無可赦,豈料其遽死?紱不將其病死明白於眾,致生疑議,紱能辭其過乎?田文鏡公忠,而紱與珽極力陷害,使濟世誣劾,必欲遂其私怨。此風何可長也?」復下紱刑部嚴鞫,獄上,請治罪,上寬之。

高宗即位,賜侍郎銜,管戶部三庫,尋授戶部侍郎。乾隆元年,方開博學鴻辭科,紱所舉已眾,又以所知囑副都御史孫國璽薦舉,事聞上,上詰紱,紱自承妄言,上謂「紱乃妄舉,非止妄言,避重就輕」。降授詹事。二年,以母憂歸。六年,補光祿寺卿,遷內閣學士。

紱偉岸自喜。其論學大指,謂硃子道問學,陸九淵尊德性,不可偏廢,上聞而韙之。八年,以病致仕,入辭,上問:「有欲所陳否?」紱以慎終如始對,賜詩獎及之。十五年,卒。

孫友棠,乾隆十年進士,自編修累遷至工部侍郎。新昌舉人王錫侯撰字貫,坐悖逆死。友棠有題詩,並奪官,賜三品卿銜。卒。

蔡珽,字若璞,漢軍正白旗人,雲貴總督毓榮子。康熙三十六年進士,改庶吉士,散館授檢討。洊擢少詹事,進翰林院掌院學士,兼禮部侍郎。時世宗在潛邸,聞其能醫,欲見之,珽謝不往。六十年,四川巡撫年羹堯入覲,世宗命達意,仍堅辭。六十一年,羹堯授川陜總督,以珽代為四川巡撫,覲聖祖熱河行在,世宗方扈從,乃詣謁而去。雍正二年,羹堯請川、陜開採鼓鑄,珽疏言四川不產鉛,開採非便,羹堯劾珽阻撓,下部議,當奪官。珽辱重慶知府蔣興仁,憤自殺,珽以病卒聞,羹堯劾之,上詰責再三,始自承。下部議,擬斬,詔逮至京師,召入見,具言羹堯貪暴及所以抗拒羹堯狀,上諭曰:「珽罪應如律,然劾之者羹堯,人將謂朕以羹堯故殺珽,是羹堯得操威福柄也。其免珽罪。」特授左都御史,兼正白旗漢軍都統。尋進兵部尚書,仍兼左都御史。會羹堯得罪,直隸總督李維鈞隱其財產,上命珽偕內大臣馬爾賽往按,得實,奪維鈞官,以珽署總督。

直隸方被水,議蠲賑,復發帑修河間、靜海諸城,俾饑民就傭受食。珽奏言省會米貴,令按察使浦文倬至天津運截留漕米二萬石,以萬石運保定平糶,留萬石賑經過諸地,上如所請,敕再運通倉米十萬石往天津,加賑一月。珽奏:「請察地方官侵冒,懲胥役虛報,訪衿棍挾制,貧民戶給印券,每村給村名紙旗,以次給領。賑滿,續修城工,即以賑時所給印券交驗受傭。」從之。調補吏部尚書,仍兼領兵部、都察院及都統事。四年,以珽所領事多,先後解左都御史、都統、吏部尚書,專任兵部尚書。旋以在直隸時徇庇昌平營參將楊雲棟,坐奪官,上命降授奉天府尹。

初,上以岳鍾琪代年羹堯為川陜總督,珽入對,言鍾琪叵測。鍾琪入覲,過保定,珽方署直隸總督,造蜚語,冀以撼鍾琪。事聞,上嚴旨詰責。五年,召回京按訊,上閱羹堯幕客舉人汪景祺所著書,載珽撫四川時得夔州府知府程如絲賄,保治行第一。如絲守夔州,鬻私鹽,而捕湖廣民鬻私鹽者得輒殺之,為羹堯劾罷。珽入對,言其冤。上命免如絲罪,且擢為四川按察使。至是,上頗疑景祺言。會巡撫馬會伯劾如絲營私網利疏至,命侍郎黃炳如四川按其事,以珽偕炳還奏,事實,下法司匯讞。尋議珽挾詐懷私,受夔關稅銀、富順縣鹽規,冒銷庫帑,並得如絲銀六萬六千、金九百,讒毀鍾琪,交結查嗣庭,凡十八事,應斬決,妻子入辛者庫,財產沒入官,命改斬監候。

六年,管理正白旗信郡王德昭又奏珽家藏硃批奏摺三件未繳進,大不敬,應立斬,詔逮至京師。初,珽故吏知縣黃振國坐事奪官,珽薦起河南信陽知州,巡撫田文鏡劾貪劣不法。李紱自廣西巡撫遷直隸總督,入對,力陳振國無罪,御史謝濟世劾文鏡亦及之,言與紱合。上疑紱與濟世為黨,召紱還京師,戍濟世。及珽至,諭暴珽等結黨欺罔、傾陷文鏡諸罪狀,命斬振國,珽仍改斬監候,下獄。十三年,高宗即位,赦免。乾隆八年,卒

謝濟世,字石霖,廣西全州人。康熙四十七年,舉鄉試第一。五十一年,成進士,改庶吉士,授檢討。雍正四年,考選浙江道御史。未浹旬,疏劾河南巡撫田文鏡營私負國,貪虐不法,列舉十罪。上方倚文鏡,意不懌,命還濟世奏,濟世堅持不可。上諭曰:「文鏡秉公持正,實心治事,為督撫中所罕見者,貪贓壞法,朕保其必無,而濟世於督撫中獨劾文鏡,朕不知其何心?朕訓誡科道至再至三,誠以科道無私,方能彈劾人之有私者。若自恃為言官,聽人指使,顛倒是非,擾亂國政,為國法所不容。朕豈不知誅戮諫官史書所戒?然誅戮諫官之過小,釀成人心世道之害大。禮義不愆,何恤於人言,朕豈恤此區區小節哉?」奪濟世官,下大學士、九卿、科道會鞫,濟世辨甚力。刑部尚書勵杜訥問:「指使何人?」對曰:「孔、孟。」問:「何故?」曰:「讀孔、孟書,當忠諫。見奸弗擊,非忠也!」讞上,以濟世所言風聞無據,顯系聽人指使,要結朋黨,擬斬。

文鏡劾屬吏黃振國、邵言綸、汪諴等,李紱訟言其枉,並謂河南諸吏張球最劣,文鏡縱弗糾。入對,具為上言之。上先入文鏡言,不直紱,而濟世罪狀文鏡又及枉振國、言綸、諴庇球諸事。上召大學士、九卿、科道等入見,舉前事,謂:「濟世言與紱奏一一吻合,今詰濟世劾文鏡諸事,濟世皆茫無憑據,俯首無詞,是其受人指使,情弊顯見。」命奪濟世官,往阿爾泰軍前效力贖罪。濟世至軍,大將軍平郡王福彭頗敬禮之,濟世講學著書不稍輟。七年,振武將軍順承郡王錫保以濟世撰古本大學注毀謗程、硃,疏劾,請治罪。上摘「見賢而不能舉」兩節注,有「拒諫飾非,拂人之性」語,責濟世怨望謗訕,下九卿、翰詹、科道議罪。有陸生棻者,自舉人選授江南吳縣知縣,引見,上有所詰問,不能對,改授工部主事。復引見,上見其傲慢,以其廣西人,疑與濟世為黨,命奪官發軍前,令與濟世同效力。生棻撰通鑒論十七篇,錫保以為非議時政,別疏論劾。上並下九卿、翰詹、科道議罪,尋議濟世詆訕怨望,怙惡不悛,生棻憤懣猖狂,悖逆恣肆,皆於軍前正法。上密諭錫保誅生棻,縛濟世使視,生棻既就刑,宣旨釋之。

濟世在戍九年,高宗即位,詔開言路,為建勛將軍欽拜草奏,請責成科道嚴不言之罰,恕妄言之罪,上嘉納焉。旋召濟世還京師,復補江南道御史。濟世以所撰大學注、中庸疏進上,略言:「大學注中,九卿、科道所議諷刺三語,臣已改刪,惟分章釋義,遵古本不遵程、硃,習舉業者有成規,講道學者無厲禁。千慮一得,乞舍其瑕而取其瑜。」得旨嚴飭,還其書。乾隆二年,濟世疏曰:「臣今所言者有二:一曰去邪勿疑,一曰出令勿貳。有罪而復用,如程元章、哈元生者,輿論猶有恕詞;至於隆升,國人皆曰不可,猶未罷斥。不惟不罷斥隆升而已,如王士俊以加賦為墾荒,肆毒中州,又請為田文鏡立賢良祠。皇上既深惡之,乃調回而仍用,逮勘而復赦,乃者清問及之,議者謂將用為籓臬。籓臬總一省刑名錢穀,豈辜恩負罪之督撫所能勝任乎?易言渙汗,禮稱綸綍,信而已矣。今則元年諭旨,二年即廢格或改易矣;特諭停止在任守制,近日督撫又漸次請行。天下之大,何患無才?記曰『金革無闢』,又曰『君子不奪人之親』,安用此食祿忘親者為哉?特諭監生準入場不準考職。昨世宗升祔恩詔,監生仍準考職。考職者入仕之門,既準捐監,又準考職,復開捐例之張本也。即止給虛銜,不準實授,而後命前命相違,亦不宜如此。臣聞不退不遠,大學所譏,世間君子少、小人多,已敗露者不行放流,未敗露者益無忌憚。若發號施令,小人得以搖奪,君子無所適從,國事未有不隳者也。」

三年,疏言:「母蔣年七十一,行動艱難,耳目昏憒。臣欲歸養,則貧不能供甘旨;欲迎養,則老不能任舟車;欲歸省,則往返動經半年。在家不過數月,乍逢又別,既別難逢,慈母之涕淚轉添,游子之方寸終亂。臣才不稱道府,例又從無自請遷轉。乞敕部以州縣降授湖南、廣東,量予近地,臣得母子聚首,無任哀懇。」上特授濟世湖南糧儲道。

八年,濟世聞衡陽知縣李澎徵賦縱丁役索浮費,易服偽為鄉民納賦者以往,察得實,善化知縣樊德貽與同弊,濟世詳劾。巡撫許容庇德貽等,以濟世蕩檢逾閑列狀入告。上命解任,交總督孫嘉淦會鞫,濟世捕衡陽丁役下長沙知府張琳,讞得徵收浮費有據。容令岳常澧道倉德代濟世,布政使張璨附容指,貽書倉德,令更易長沙府詳牒。倉德初官給事中,嘗劾濟世奏事失儀,至是不直璨所為,發其書上嘉淦及漕運總督顧琮,嘉淦庇容,寢其事。諭倉德委曲善處,琮咨都察院奏聞。御史胡定糾容挾私誣劾,採湖南民謠,斥容與璨等朋謀傾濟世。上命侍郎阿里袞如湖南會嘉淦按治,而倉德以嘉淦寢其事,復揭都察院奏聞。上責嘉淦草率扶同,召還京師,解容、璨任,奪琳、德貽、澎官。阿里袞尋奏濟世被誣劾,請復官,容、璨及按察使王玠皆坐奪官,上命並罷嘉淦,而獎倉德及定,調濟世驛鹽道。

蔣溥代為巡撫,嗛濟世密進所著書,斥為離經畔道,上曰:「朕不以語言文字罪人。」置不問。未幾,復言其老病,乃命休致。歸家居十二年,卒,年六十有八。

陳學海,字志澄,江西永豐人。康熙五十二年進士,改庶吉士。與濟世友,授山東恩縣知縣,行取刑部主事,遷員外郎。文鏡劾振國等,上遣侍郎海壽、史貽直往按,請以學海從,得文鏡欺罔狀,將以實入告,繼乃反之,學海爭不得。使還,擢御史,嘗以語濟世,濟世用是劾文鏡。既譴,學海不自安,次年,以病告。都察院劾偽病,並及與濟世交關狀,奪官,命與濟世同效力軍前。雍正七年,召還,授檢討。十一年,卒。

論曰:田文鏡與鄂爾泰、李衛同為世宗所激賞。高宗謂三人者文鏡為最下,允哉!文鏡馭屬吏苛急,待士尤虐。紱固以好士得時譽,宜其惡之深,而所爭以為枉者,為珽所薦吏。濟世又繼以為言,世宗疑珽使紱入告,不納;又嗾濟世露章論劾,互相結,務欲傾文鏡。獄遂不可解,然終未即誅死。高宗嗣服,諸人皆得湔祓,紱復起,濟世亦見用。孰謂世宗嚴?不肯戮諫臣,固明言之矣。


\end{pinyinscope}