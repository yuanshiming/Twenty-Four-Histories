\article{列傳一百八十九}

\begin{pinyinscope}
烏蘭泰長瑞長壽董光甲邵鶴齡鄧紹良石玉龍

周天受弟天培天孚饒廷選文瑞彭斯舉

張玉良魯占鼇劉季三雙來瞿騰龍

王國才虎坤元戴文英

烏蘭泰,字遠芳,滿洲正紅旗人。由火器營鳥槍護軍從征回疆有功,升藍翎長,累擢護軍參領、營總、翼長。軍政卓異,道光二十七年,擢廣東副都統。善訓練,講求火器。

咸豐元年,廣西匪熾,詔烏蘭泰幫辨軍務,選帶適用器械及得力章京兵丁赴軍,以廣東綠營精兵五百人隸之。四月,偕向榮、秦定三等圍賊於武宣,賊竄象州,自請治罪。詔以其初至,免議,命偕向榮節制鎮將。時軍中將帥不和,文宗憂之,密諭烏蘭泰實陳勿隱。上疏略曰:「周天爵奏向榮曲徇其子,致失眾心,不為無因。武宣之役,秦定三、周鳳岐、張敬修連營防禦,其堵剿不利,追賊遲延,咎當同任。天爵劾定三、鳳岐,不及敬修,人心不服。向榮將官傅春、和春失利,天爵責定三不並力,後訪知實非退縮,諉為向榮推卸之言。因之天爵、榮、定三皆有隙。天爵年老,直強、耳輭,其子光岳干預,致失人心。」又言:「向榮初剿賊屢捷,未免輕賊。及其子招嫌,楚兵藉口,遂多諉卸。然在軍鎮將無及榮者。更易其兵,仍可立功。」上下其疏,命賽尚阿覈奏,賽尚阿請不咎既往,令烏蘭泰與向榮分任軍事,以專責成。

賊踞象州中坪,烏蘭泰督貴州三鎮兵,由羅秀進梁山村,逼近賊巢。賊乘駐營未定,猛撲,連擊敗之,殪賊千餘。是年秋,賊竄桂平新墟,烏蘭泰分四路進攻,破伏賊於莫村,一日七戰皆捷,斬級數千,賜花翎。賊屯紫荊山,新墟為山前門戶,雙髻山、豬仔峽為山後要隘,負隅死拒。向榮偕巴清德連奪雙髻山、豬仔峽,合攻風門坳,破之。進逼新墟,迭攻不下,其附近村落掃蕩幾盡。閏八月,賊編木牌欲渡河,烏蘭泰迭擊,大敗之,詔嘉獎,加都統銜。於是賊棄新墟他竄,向榮等追至平南,敗績,賊遂陷永安州。烏蘭泰追至,戰於水竇、圞嶺,皆大捷,賜黃馬褂。永安地險,賊皆死黨固結,僅烏蘭泰一軍久戰已疲,故不能制之。

向榮自平南敗後被譴,託病逗留梧州、平樂者兩月有餘。至冬始抵永安,攻北路,烏蘭泰攻南路,毀水竇賊巢。向榮亦進奪槓嶺要隘,合擊迭挫賊。賽尚阿親蒞督戰,期在必克。江忠源號知兵,隸烏蘭泰軍,倚其贊助;每言賊兇悍,久蔓將不可制,必聚而殲之。烏蘭泰主鎖圍困賊,向榮謂圍城缺一面,乃古法,宜縱賊出擊,兩人意不合。會榮克城西砲臺,二年元旦,同詣賽尚阿賀歲。賽尚阿遇榮特優,烏蘭泰憤甚,忠源解之,然益不相能。忠源以母憂,辭歸。時嚴詔促戰,春雨連旬,士卒疲困。二月,賊棄城冒雨夜走,北犯桂林。烏蘭泰率兵急追至昭平山中,路險雨滑,為賊所乘,敗績,總兵長瑞、長壽、董光甲、邵鶴齡死之。向榮徑收州城,由間道趨桂林,先賊至。烏蘭泰踵賊後,戰於南門外,爭將軍橋,砲中右腿,創甚,退屯陽朔,越二十日卒於軍。烏蘭泰忠勇為諸將冠,文宗深惜之,賜銀一千兩治喪,予輕車都尉世職,謚武壯。

長瑞、長壽,瓜爾佳氏,滿洲正白旗人。父塔思哈,道光初,官喀什噶爾幫辦大臣。叛回張格爾作亂,殉難,予騎都尉世職。長瑞襲世職,授三等侍衛,累擢直隸天津鎮總兵;長壽以廕授藍翎侍衛,累擢甘肅涼州鎮總兵:並從賽尚阿赴廣西剿匪,同領湖南兵。長瑞戰風門坳有功,新墟禦賊失利,奪職留營。及賊由永安出竄,從烏蘭泰躡追至龍寮嶺,地險,左右止勿進。長瑞曰:「軍令孰敢違者!死耳,勿復言。」以母老,令長壽毋相從,長壽泣曰:「貪生忘國,非孝也。」卒偕行。值大霧,賊以巨砲扼山間。軍士兩日不得食,為賊沖潰踐踏,死無算。長壽墜馬,長瑞挺矛救之,身被數十創,同遇害。文宗以其父子兄弟皆死難,深惜之,並贈提督,予騎都尉兼雲騎尉世職。存問其母,賜銀三百兩。長瑞謚武壯,長壽謚勤勇,於永安建祠曰雙忠,同死者附祀焉。

董光甲,直隸河間人。嘉慶十四年武進士,授守備。累擢河南河北鎮總兵。從向榮攻永安,奪槓嶺、摩天嶺、天鵝嶺諸要隘。追賊至昭平,迭擊賊於古束、龍寮嶺,次黃茆嶺。賊反撲,力戰死之,贈提督,予騎都尉兼雲騎尉世職,謚勇烈。

邵鶴齡,山東招遠人。嘉慶二十五年進士,授三等侍衛。累擢湖北鄖陽鎮總兵。偕長瑞等同追賊龍寮嶺,殞於陣,予騎都尉兼雲騎尉世職,謚威確。

鄧紹良,字臣若,湖南乾州人。由屯弁累擢守備。從剿崇陽土匪李沅發,率五百人破賊金峰嶺,擒沅發,擢都司,賜花翎、揚勇巴圖魯名號。遂從向榮赴廣西剿賊,潯州牛排嶺之戰,以精騎張左右翼,擊兩路賊,皆挫之。又戰象州、永安州,皆有功。咸豐元年,授楚雄協副將。二年,援桂林,屯西門,力戰卻賊。追賊入湖南,援長沙,入任城守,地雷發,持刀屹立,砲洞左臂,不動,殪先登賊,賊退,城復完,軍中稱其勇。洎賊解圍竄湖北,巴陵土匪晏仲武勾結肆掠,紹良偕總兵阿勒經阿剿平之。

三年,擢安徽壽春鎮總兵,詔率所部從向榮援江南,廷臣多薦紹良者,尋擢江南提督。榮令分剿鎮江踞賊,進擊觀音山,合攻瓜洲,皆捷。逼城而軍,賊設伏北固山下,而自城突出撲營,火四起,官軍不能御。退守丹陽,褫職議罪,仍隸向榮軍,帶罪自效。賊兩次窺伺東壩,榮令紹良擊走之。四年,克太平,紹良移軍駐守,又破賊採石。向榮疏陳戰功,為乞免罪,允之。時賊由蕪湖窺徽州、寧國,紹良屯黃池,賊酋石國宗糾各路賊萬餘來犯。紹良兵少,設伏山溝,多張疑兵,誘賊入,痛殲之。五年春,賊復乘夜撲營。伏槍砲,俟近驟發,殲賊無算。詔嘉紹良力遏賊鋒,保全甚大,予三品頂戴,復花翎。賊既退歸,復圖襲徽、寧,以窺浙境。紹良奉命馳往,統各路援兵,至則簡精銳,伏要隘伺擊,屢破賊,克婺源、黟、石埭諸縣,復提督銜。賊聚於蕪湖,窺南陵、黃池。紹良由灣沚進剿,連破賊,焚其舟,遂克蕪湖,授陜西提督。

六年春,江寧賊上竄,踞倉頭鎮,勢甚熾。向榮令紹良往督戰,而諸將意不愜,轉不盡力,於是戰不利,紹良受傷,坐褫花翎。德興阿軍潰,揚州陷。詔紹良渡江赴援,幫辦江北軍務。破藥王廟賊壘,環攻揚州六晝夜,克之,又追破賊於三汊河。會寧國告陷,復命幫辦皖南軍務。移軍赴援,扼金河橋,大破賊於東溪橋,又迭擊賊於涇縣,挫之,調浙江提督。賊糾黨數萬來援,敗之於楊柳鋪。副將周天受遇賊夏家渡,戰未利。紹良乘隙縱擊,賊大潰,遂督諸軍連奪夏家渡、團山諸賊壘,破七里岡賊巢,進攻寧國,十二月,克之。七年,丁母憂,奪情留軍。紹良以寧國為浙之屏蔽,而涇縣為咽喉要沖,屯軍扼之,賊屢犯不得逞。既而大軍復鎮江、瓜洲,急攻江寧,賊圖牽掣,大舉犯南陵,紹良擊走之。八年,進屯灣沚。賊合捻匪踞黃池,紹良回援,出賊不意,大破其眾,復黃池。會浙江軍事日棘,分兵赴援。十一月,賊乘虛悉馬步數萬鵕而涉水,斷黃池山後接應,突攻灣沚營壘。總兵戴文英由江寧來援,戰歿,遂合圍。軍中餉絀食盡,紹良舉火自燔其營,率親兵血戰,死之。

事聞,詔念紹良桂林、長沙保城前功,轉戰徽、寧之間,凡歷五載,力竭捐軀,深致憫惜。贈太子少保,予騎都尉兼雲騎尉世職,謚忠武。於殉難地方建專祠,並賜其父白金四百兩,子亨先候錄用。尋以遺骸不得,文宗尤憫之,賜亨先員外郎銜。後湖南巡撫駱秉章疏請附祀表忠祠,允之。

石玉龍,湖南鳳凰人。以練勇從征,隸向榮、鄧紹良軍,積功至游擊。咸豐六年,總兵秦如虎駐防涇縣,以憂去,代者難其人,紹良薦玉龍,以游擊充統將。玉龍感奮,遇戰益力。從紹良復灣沚、黃池,又破賊萬級嶺,累擢副將。九年冬,賊大舉犯涇縣,迎擊於藍山嶺,初勝,賊至益眾,圍之數重,身被十餘創而殞,贈總兵加提督銜,謚剛介。

周天受,字百祿,四川巴縣人。咸豐初,從向榮剿賊廣西,轉戰湖南、湖北、江南,積功至游擊,賜號沙拉嗎依巴圖魯。五年,皖南軍事亟,前江西巡撫張芾治徽、寧兩郡防務,乞援於向榮,乃令天受率川兵赴援,偕諸軍克婺源、休寧、石埭。六年,援太平,連破賊於花橋、西溪,進規涇縣。大敗賊於雙坑寺,復其城,擢副將。會休寧復為賊踞,官軍戰不利,張芾檄天受助剿,連捷。進毀石嶺、萬安街賊壘,會攻休寧,再復之,以總兵記名。七年,再復婺源,授福建漳州鎮總兵。賊踞陵陽鎮,值中秋令節,夜半出不意縱火攻之,盡毀賊營。復破賊於祁門五里牌,搗其巢,擒斬甚眾。八年,援浙江,將軍福興令守衢州。天受以浙西完善地,不可為賊擾,主扼樟樹潭。賊竄龍游,天受留軍守壘,自率千人趨湯溪、宣平,賊引去。

和春疏言天受知兵,能占先著,而力單,遣其弟天培往助之,詔加提督銜,督辦浙江防剿事宜。天受嚴守金華,令天培復武義,又會江南軍復永康。張芾劾其驕縱,縱兵搶掠,詔罷總統,仍責剿賊。天受方連克縉雲、宣平、溫州,於是浙江巡撫晏端書疏陳援浙功,為白被劾之枉。詔以浙事漸平,命偕弟天培及總兵饒廷選等進援福建,連戰皆捷,復浦城,而賊回竄江西,復犯皖南。命署湖南提督,回軍防徽州,節制諸軍,從張芾之請也。九年,進軍寧國,賊犯石埭、太平、涇縣,皆遣將擊走之。十年春,官軍連捷於涇縣、旌德,賊復入浙境,坐防剿不力,褫勇號,革職留任。

時江南大營再潰,軍事愈棘。張芾疏言:「寧防將弁大半籍隸湖南,皆鄧紹良舊部,習氣甚深。天受雖力求整頓,轉滋疑謗,請歸曾國籓節制。」國籓亦言其兵不可用,別調募新軍,倉猝不能至。天受偕江長貴再復涇縣,而賊糾大股犯寧國,勢甚張,天受激勵饑軍力禦。既而徽州陷,餉道梗絕,遣去城中居民萬餘,誓以身殉。八月,兵敗於廟埠。天受督隊守北門,大雨,火器不燃,城陷,巷戰死之。詔復天受原官,予騎都尉兼雲騎尉世職,謚忠壯;以其弟天培、天孚先皆殉節,命於四川省城及本縣合建專祠。

天培,由行伍從征廣西,累擢守備,隸向榮軍。咸豐六年,從破高資蔡家窯及壩西賊壘,賜號衛勇巴圖魯。七年,克東壩,平寶堰賊巢。連戰於鄔山、尖山,克溧水,又破賊於鎮江虎頭山,累擢貴州定廣協副將。克瓜洲,以總兵記名。八年,授雲南鶴麗鎮總兵。先後偕張國樑破賊秣陵關及江寧南門外,功皆最。和春知其善戰,令赴浙江援其兄天受,迭克武義、龍泉,追賊入閩,克浦城。會江南、北軍事急,天培回援。九年春,賊分六路攻浦口,張國樑督諸軍御戰,天培首先躍馬沖陣,各軍乘之,殲賊無算。賊築壘於雙陽、蕭家圩,別由九洑洲出悍眾來撲,天培分兵擊之,三戰三捷,功出諸將上,擢湖北提督,遂駐防浦口。是年冬,匪首陳玉成糾眾十餘萬犯江浦,天培乘其初至,痛殲之。既而賊麕集,後路為所抄襲,裹創血戰,力竭陣亡。優詔賜恤,贈太子少保,予騎都尉兼雲騎尉世職,謚武壯。

天孚,從兄天受軍轉戰,以功洊保參將,留江蘇補用。咸豐九年,賊犯皖南,副將石玉龍戰死涇縣南山嶺。天孚屯灣沚,馳百里往援,要擊於章家渡,大破之,由是以驍勇名。尋援金壇,會諸軍連戰解其圍。十年,江南大營潰,閏三月,賊首李世賢大舉復圍金壇。天孚偕總兵蕭知音、參將艾得勝、知縣李淮同守之。淮素得人心,兵民合力,屢卻賊。時江南軍事大壞,孤城援絕。天孚馳書兄天受,始疏聞,屢詔促鎮江副都統巴棟阿偕總兵馮子材赴援,卒不至。凡守百四十餘日,糧盡,軍無固志。知音等原率兵民突圍走鎮江,淮不可,誓死守,乃中止。屢獲賊內應,斬之。城陷先一日,偵知將有變,竟夜登陴,至旦,分半隊休息,值大霧,叛兵遽起,先戕天孚。賊乃梯登,知音、得勝突圍出,淮死之。事聞,贈天孚總兵,予騎都尉世職,謚威毅。

饒廷選,字枚臣,福建侯官人。以行伍洊升千總。道光中,從剿臺灣有功,擢守備。從水師提督竇振彪出洋擒海盜,擢漳州營都司。遷游擊,治匪無株連,得民心。咸豐三年,奉檄赴詔安治械斗,而潮州會匪襲漳州,伏兵於城中突起,鎮道皆遇害。廷選聞變,間道馳還,號召鄉民千餘,城民應之。賊遁,旋復大至。廷選率鄉團固守,迭戰破賊,擒賊首謝厚等,遂署漳州鎮總兵。外剿內撫,期年始平。總督王懿德薦其才可大用,四年,授貴州安義鎮總兵,留署福建陸路提督。

五年,粵匪陷廣信,浙江戒嚴。廷選赴援,扼衢州。尋楚軍克廣信,賊知浙境有備,走徽州。六年,賊酋楊輔清復圖廣信以擾浙。廣信兵僅數百,知府沈葆楨馳書告急。廷選方駐甲玉山,曰:「賊得廣信,則玉山不守,而浙危矣。」值大雨水漲,駛舟急行,抵廣信。賊已至城西太平橋,初諜城中無兵,及見旌旗,賊為奪氣。廷選所部僅千餘人,屢出奇擊賊。既而賊大至,部將畢定邦、賴高翔皆勇敢,獻計曰:「今賊不知我虛實,以我能戰,後路必有大兵。若稍退,賊追我,且立盡。當速決死戰。」廷選用其言,明日開城奮擊,自晨至日暮,毀其長圍,軍聲大振。越二日,賊引去,賜號西林巴圖魯。閩、浙大吏與江西督防者不慊,檄廷選速回師保浙。廷選待接防兵至始行,廣信民感其義。

七年,調衢州鎮總兵,會皖軍克婺源。八年,賊首石達開大舉犯浙,廷選分軍援廣豐,自守衢州。賊驟至,穴地攻城,城圮者三,皆擊卻之,守七十餘日。巡撫晏端書劾其久未解圍,又失江山、常山、開化三縣,奪職。未幾,圍解,三縣皆復,授南贛鎮總兵。王懿德檄召回援閩境,以病未行,遽劾,革職留營。八年,會克連城、龍巖,仍補南贛鎮。曾國籓奏以代沈葆楨守廣信,從民望也。

十年,粵匪復犯浙,廷選赴援,復淳安,擢浙江提督。十一年秋,攻克嚴州,進規浦江,賊大至,不敵,退保諸暨,而杭州被圍急。巡撫王有齡促回援,廷選舊部僅漳勇數百、楚軍二千。事急,收集江南潰卒,皆不任戰,徒激忠義,勉以當賊。賊於城外海潮寺、鳳凰山為堅壁,隔絕內外。困守七十餘日,糧盡,士卒饑餓。十一月,城陷,巷戰死之,贈太子少保,予騎都尉兼雲騎尉世職,謚果壯。入祀昭忠祠,於杭州建專祠。兄廷傑,弟廷夔,同戰死,附祀焉。既而曾國籓、沈葆楨以廷選守廣信功,奏請建祠廣信,以副將畢定邦、賴高翔附祀。

文瑞,克什克特恩氏,蒙古鑲藍旗人,荊州駐防。由驍騎校從軍,轉戰湖北、安徽,累擢江西撫標中軍參將。咸豐十年,赴援浙江,克餘杭,以總兵記名。解湖州圍,賜號唐木濟特依巴圖魯。授處州鎮總兵,進剿金華。賊圍浦江,文瑞嬰城固守,屢出奇破賊營,逾月乃陷,詔免其處分。回援杭州,入城助守,城陷死之,予騎都尉兼雲騎尉世職,謚果毅。

彭斯舉,湖南平江人。以團練剿賊,從李元度為平江軍營官。戰湖口、東鄉、貴溪、安仁、玉山,積功晉秩知府。元度罷去,留所部五營隸斯舉,始獨將一軍。會攻景德鎮,饒廷選見而器之,調援浙江,破賊於淳安,復其城,擢道員,留浙補用。駐守千秋關,賊大至,搏戰竟日,潰圍出,移防海寧。會攻嚴州,下之。進援廣信,而所部留駐常山者索餉譁潰,斯舉率親兵赴杭州,乞解軍事回籍,巡撫王有齡留管營務處。斯舉建議,省城米糧來自寧、紹,錢塘江距城三里,當築甬道,兵護之,運道乃無虞。未及行而賊至,城中竟以絕糧陷。斯舉分守湧金門,死之。

張玉良,字璧田,四川巴縣人。咸豐初,由行伍從征廣西,積功至千總。四年,從向榮至江南,戰江寧城外,屢有功,累擢永州左營游擊。六年,敗賊於丹陽、金壇,賜號黽勇巴圖魯。又敗賊於溧水西門,毀其砲臺,擢處州營參將。七年,克句容,加總兵銜,擢三江口協副將。破鎮江援賊於江濱,克鎮江,敘功以總兵記名。八年,大破江寧援賊,擢甘肅巴里坤總兵。攻太平、金川諸門,賊眾突出,痛殲之。馳援溧水,毀紅藍埠賊壘,克其城,斬賊千餘級,加提督銜。九月,會攻浦口,大捷。而九洑洲之賊來援,玉良率後隊截擊,賊大潰。十年春,遂乘勝克九洑洲,詔遇提督缺出題奏,尋調肅州鎮總兵。

江南大營諸將善戰者,向榮舊部多蜀將,張國樑所部多粵將。蜀將以虎坤元為首,周天培及玉良次之。時浙江軍事亟,議分軍赴援,咸屬望於張國樑,而圍攻江寧,功在垂成,國樑為全軍所系,不克行。坤元、天培已前歿,乃命玉良總統援浙諸軍,專辦浙江軍務,未至而杭州陷,將軍瑞昌獨堅守駐防內城,與賊相持。玉良率六百人馳至,出賊不意,毀武林、錢塘諸門外賊壘,梯城而上,遂復杭州。捷聞,詔嘉為奇功,賜黃馬褂,予騎都尉世職,擢廣西提督。

賊之擾浙也,原以牽制江南軍,故見玉良至,則不戰遽去,由廣德分路趨江寧。總督何桂清駐常州,檄玉良回援,而賊別隊已侵江南大營後路。桂清留玉良於常州以自衛。未幾,江寧兵潰,張國樑、和春先後殉,詔玉良代節制其軍。常州陷,禦賊於無錫高橋,賊由間道出九龍山襲無錫。玉良前後受敵,退保蘇州,入城計守御,未定,潰兵應賊,蘇州亦陷。玉良奔杭州,褫職,隸瑞昌軍。瑞昌令規復嚴州,繼克常山,復原官。十一年,復遂安,而嚴州又陷。玉良自江南敗衄後,兵心已渙,不能復振。賊再攻杭州,馳援,軍不用命,自知事不可為,戰杭州城下,輒身臨前敵,力鬥,中飛砲,歿於軍。贈太子少保,予騎都尉兼雲騎尉世職,祀本籍昭忠祠,謚忠壯。

魯占鼇,四川成都人。由行伍官平番營守備,從向榮剿賊廣西、江南。繼從吉爾杭阿克上海,攻鎮江,戰皆力,累擢川北鎮總兵,調建昌鎮。蘇州陷,為賊所執,罵賊被臠割,死之。贈提督銜,予騎都尉兼雲騎尉世職。

劉季三,廣西武宣人。以武舉從右江道張敬修戰桂林、全州,授左江鎮標守備。從向榮至江南,積功至副將,賜號直勇巴圖魯。咸豐八年,大兵攻秣陵關,季三於葛塘寺設伏,出賊不意,斬關入,火之,又破六郎橋賊巢,功皆最,擢直隸通永鎮總兵。十年,張國樑督諸軍攻江寧,季三任上關一路,壽德洲守賊秦禮國獻壘內應,破上關,拔出難民千餘,解散脅從五千餘人。從張玉良援浙江,克餘杭、臨安,進秩提督。是年秋,賊陷嚴州,掠富陽,季三孤軍往援,戰竟日,死之。予騎都尉兼雲騎尉世職,謚忠毅。

雙來,徐氏,漢軍正白旗人。由拜唐阿累遷鑾儀衛治儀正,出為甘肅碾伯營都司,擢秦州營游擊。道光二十七年,赴援回疆,行至黑孜布依遇賊。兵少,被圍,相持十餘日。援至,合力破賊。方圍急,賊塞水源以斷汲路,越日泉湧盈塘。宣宗聞之,嘉嘆曰:「此將士忠義所感也!」命以參將用,賜花翎、法福哩巴圖魯勇號。尋敗賊於駱駝脖子,加副將銜。歷靈州營參將、永固協副將。

咸豐二年,調赴欽差大臣琦善軍,擢肅州鎮總兵。三年,從琦善攻揚州,勇銳為一軍之冠,戰輒手執大旗以先,迭破賊,毀西北隅土城,悉奪其營壘。賊遁入城死守,圍攻兩閱月。雙來發砲壞城垣丈餘,作桴渡河,逼城布雲梯,鼓勇先登,縱火,賊於城上苦鬥,槍彈如雨。雙來傷頰,折二齒,暈跌,扶下,從卒多傷亡,以無繼援而退。特詔褒獎,加提督銜,他將觀望者並被譴。越旬日,雙來復督隊攻城,力戰逾時,中砲,洞穿右股,猶大呼登城殺賊。翌日,創甚,卒於軍。

文宗素知其勇,事聞,震悼,手批其疏曰:「雙來何如是不幸?朕隕涕覽奏,不勝悲憤!然視彼貪生退縮者,奚啻霄壤。」詔依提督例優恤,賜銀一千兩,命柩歸時專奏入城治喪,予騎都尉兼雲騎尉世職,謚忠毅。後都統德興阿疏言雙來與總兵瞿騰龍戰績尤異,先後於江北陣亡,請在揚州建雙忠祠合祀,詔允之。

瞿騰龍,字在田,湖南善化人。由行伍補千總,剿瑤匪趙金龍及乾州苗有功,累擢古丈坪營都司,署鎮筸鎮標游擊。咸豐元年,率標兵赴廣西剿匪,迭破賊於武宣桐木、馬鞍山,永安古排塘。二年,援桂林,以巨砲擊賊於文昌門,殲斃甚眾,賜號莽阿巴圖魯,擢永綏協副將。追賊入湖南,迭戰於寧遠、耒陽、永興、安仁。賊圍攻長沙,騰龍率苗兵千人赴援,偕鄧紹良破南門外賊柵。賊以地雷轟城,圮十餘丈,騰龍守城缺力禦,斬悍賊三百餘人,城復完,加總兵銜。

三年,從向榮戰武昌,遂尾賊東下,擢湖北鄖陽鎮總兵。抵江寧,賊已分黨北犯,命率所部馳赴山東、河南防剿。行至高郵,琦善疏留其軍會攻揚州。騰龍身先士卒,與總兵雙來並號軍鋒。既而雙來以傷殞,遂兼領其軍,充翼長,琦善甚倚之。揚州久不克,而賊之踞瓜洲者盡力來援。騰龍扼三汊河,賊至,十倍我軍,騰龍下令「有進無退,回顧者斬」,下馬持大刀闖入陣,士卒皆喋血戰,賊退,夜乘雷雨突之,賊不辨眾寡,自相踐殺,及曉,尸骸狼藉,斃賊二千有奇。尋賊揚帆逕趨揚州南門,登東岸,復馳擊走之。於是樹巨椿以阻河路,城賊屢突圍,擊退。十一月,賊全隊沖出,並入瓜洲,乃復郡城。

初,向榮疏調騰龍回軍江寧,不許。至是詔率兵援安慶,琦善奏三汊河要沖,恃騰龍力守,仍請留。賊於運河南岸築數壘以逼三汊河,進攻破之。四年正月,進攻瓜洲,設伏誘賊出,伏起,大破之。二月,復進攻,乘夜雪襲賊,連破二壘,深入,賊傾巢出,鈔官軍後,圍數重,戰竟日,被傷,下馬步戰,力竭死之,年六十有四。贈提督,予騎都尉兼雲騎尉世職,謚威壯。

王國才,字錦堂,原姓羅氏,雲南昆明人。以武舉效力督標,洊升守備。道光末,剿彌渡回匪,擒賊首海老陜,賜號勝勇巴圖魯,擢都司。從剿廣西賊,轉戰大黃江、永安州有功。尋撤滇軍歸伍。咸豐二年,平尋甸回匪,擢山東青州參將。

三年,吳文鎔移督兩湖,疏調率所部赴湖北,行至天門,遇賊,以親兵七十人擊走之。會文鎔戰歿黃梅,國才將返滇,過荊州,將軍官文留之,予兵千二百、練勇五百,守城北龍會橋。賊萬餘猝至,軍士氣沮,國才曰:「賊如潮湧,不進何以求生!」親以鳥槍斃執旗賊,大呼陷陣,賊披靡,墜河無算。追至馬湟山,賊敗竄,軍中稱其勇。官文令整飭諸縣團練,荊州獲安,賜花翎,以副將升用。四年,署督標中軍副將,從總督楊霈防德安。

會湘軍規大冶,國才當右路,連破賊,克蘄州。楊霈以川練千人益其軍,進攻九江。五年,率部將畢金科戰城下,數捷。會揚霈師潰,國才回援武昌,夜至,城已陷,未知也;先驅入城,始覺。賊由漢陽悉眾來拒,國才突圍出,駐金口,進大軍山。尋屯沌口,偕水師合攻漢陽,設伏誘賊出,殲之。賊屢襲金口、沌口,皆擊退。破大別山賊壘,授竹山協副將,署鄖陽鎮總兵。總督官文進逼漢陽,國才屢從破賊。六年,諸軍合攻,國才越壕逼城下,一擁而入,巷戰,殺賊甚眾,加總兵銜,記名簡放。復黃梅,守之,改隸將軍都興阿。七年,賊由太湖來犯,以空城誘賊入,斬獲無算。追至九江對岸,連破賊段窯、楓樹坳、狗山鎮。雲南回匪熾,調回援,官文、胡林翼疏留不遣。黃梅城僻隘,國才謂不足屏蔽,請守雙林驛。都興阿不許,乃屯城西,分副將石清吉守城,賊屢犯,卻之,授貴州安義鎮總兵。六月,皖賊陳玉成糾賊數十萬上犯,國才被圍,力戰,歿於陣。贈提督,予騎都尉兼雲騎尉世職,建專祠,謚剛介。

虎坤元,字子厚,四川成都人。父嵩林,咸豐初,以湖南游擊調廣西剿匪,從向榮戰紫荊山,攻永安,解桂林、長沙圍,並有功。從至江南,累擢湖北宜昌鎮總兵。偕巡撫吉爾杭阿克上海,遂從攻鎮江,屢破賊於寶蓋山、倉頭、下蜀街、高資。在江南軍中稱宿將。

坤元,年十七,從父軍,勇力過人,戰輒先登,軍中號曰「小虎」。初至江寧,奪鍾山賊壘,功最,擢守備。四年,克高淳、太平,賜花翎、鼓勇巴圖魯勇號,擢川北鎮標都司。五年,援灣沚,焚賊舟,乘勝取蕪湖,坤元躍登城,殺守陴賊,遂克之。六年,江寧賊出援鎮江,坤元元旦馳至三汊河,擊敗之。又戰於下蜀街,破賊壘,追賊直至仙鶴門,擢建昌鎮標游擊。從總兵秦如虎援浙江,而寧國告陷,遇賊於宣城紅林橋,設伏,身率數騎誘賊,敗之。進攻寧國未下,回援鎮江,嵩林為賊所困,馳入重圍掖之出。會江南大營潰,向榮等退守丹陽,賊躡至,勢甚張,坤元偕張國樑力戰卻之。遂從嵩林移駐珥陵,扼賊犯常州之路。未幾,國樑戰五里牌,傷胯,急召坤元夜至,簡精銳,未曉即出,逾簡瀆河,東攻黃土臺賊壘,躍上壘墻,毀其柵,大隊擁入,勁騎鈔截,賊無脫者,連破五壘。國樑亦破河西賊壘,賊勢始挫。

坤元以是名出諸將上,乘勝進兵,逾月遂解金壇之圍,擢參將。進攻東壩,填壕登城,負創力戰,手斬悍賊,復之。又克高淳,以副將侭先升用。七年,會攻溧水。賊屢來援,與城賊夾攻官軍。坤元迭破之於鄔山、拓塘、博望、天裏山、小茅山,凡十餘戰,殲戮無算,擒偽迓天侯陳士章,鏖鬥城下四晝夜,躍登南門,復溧水,授貴州定廣協副將。又敗賊於高陽橋,克湖墅、龍都。張國樑攻句容,賊堅守未下,檄坤元往助。值賊出撲,率數十騎突之,進逼南門,縱火焚城樓,大軍繼之,遂克句容。敘功,以總兵記名簡放。從國樑規鎮江,時賊由江寧來援,蟻聚七星觀、倉頭。坤元以輕騎誘敵入伏中,大敗之,追擊,立破三壘。賊退至三汊河,伏兵又起,無去路。坤元大呼:「棄戈者免死!殺賊首者賞!」降者數百人。是役斬馘及淹斃者三千有奇,生擒三百。尋敗賊於西堰岡,援賊復於倉頭、顧家壩築壘。坤元於山後樹幟為疑兵,自率小隊沖鋒,殪悍賊。而賊以大隊來拒,諸軍環擊,乘勢全毀賊營,鎮江守賊遂遁。追至龍潭,痛殲之。坤元甫授直隸通永鎮總兵,文宗手批其謝恩疏曰:「聞汝父子在軍營甚為奮勇。汝年未三十,已膺顯秩。務自勉勵,以副朕望。」至是復下部優敘。尋丁母憂,奪情留軍。

八年,攻秣陵關,逼賊巢為壘。賊出鬥,敗之,窮追,單騎獨前,惟游擊劉萬清從,疑有伏,止之,勿聽,進至石橋,中槍而殞,萬清奪其尸還。和春疏聞,言:「坤元從軍八載,忠勇性成。善以少擊眾,自為都守。父子所入之貲,悉以養勇士。故旌旗所指,無不披靡。歷經頒給御賜金牌六次,受二等傷四次,頭等傷十二次。灼頸落指,瀕死者屢矣。既歿,大江南北同聲悼惜。」詔從優恤,於溧水、湖墅及死事地方建專祠,謚忠壯,予騎都尉兼雲騎尉世職。未娶無子,以弟坤岡襲。是年,其父嵩林守溧水,為賊陷,坐褫職,以坤元陣亡故,獲免治罪。嵩林回籍助剿滇匪,命襄治團練,尋卒於家。

戴文英,廣東羅定人。由行伍從剿羅鏡凌十八有功,擢千總。咸豐三年,從向榮援江寧。初至進攻,文英偕張國樑穿越深林叢葦十餘處,潛襲雨花臺賊營後,大敗之,賜號色固巴圖魯。四年,剿賊七橋甕,往來沖鋒,又偕總兵德安破賊營。五年,戰高資,皆以勇銳稱,累擢惠州營都司。六年,攻鎮江,戰於京峴山。馳馬入賊陣,槍斃悍賊甚多,擢南詔營游擊。從張國樑援金壇,率精銳過河奮擊,解金壇之圍。兩江總督怡良薦舉將才,文英與其選。大兵克東壩賊壘,平寶堰賊巢,文英率茅村團練獨當一路,斬獲多,擢淮安營參將。七年,從張國樑攻鎮江,駐紅花山。賊眾來撲,文英沖入賊中,手刺殺悍酋數人,賊大敗,擢江南督標中軍副將。是年冬,攻克鎮江府城,記名總兵。八年,克秣陵關,授直隸通永鎮總兵。

時江寧長圍漸合,賊百計潰圍,屢出沖突。文英從張國樑四面兜剿,直抵外壕,焚毀望樓。皖北援賊陷溧水,文英偕張玉良馳赴會剿,分攻紅藍埠,逼河砲擊,乘夜渡河,踏平賊壘,遂復溧水。而賊復自西路來援,文英自督前隊,以劈山砲迎擊,騎兵包抄,斃賊無算。會提督鄧紹良在寧國為賊所困,文英馳援,遇賊於灣沚,連戰皆捷,而賊至愈眾,力竭,歿於陣。

文英在軍中以善戰名,為張國樑所倚,甫擢專閫而殞。文宗惜之,優詔賜恤,稱其所向有功,克溧水,破援賊,功為尤著,予騎都尉世職,謚武烈。

論曰:烏蘭泰忠勇冠軍,與向榮不合,致無成功,時論多右之。鄧紹良、周天受老於軍事,保障皖南,軍律不嚴,終不能保全浙境。張玉良後起,號驍健,江南師潰之後,竟不復振。諸人皆當一面,以死勤事,其成敗有足鑒者。雙來、瞿騰龍、王國才、虎坤元、戴文英並以善戰名,志決身殲,時論惜焉。


\end{pinyinscope}