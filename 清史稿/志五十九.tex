\article{志五十九}

\begin{pinyinscope}
禮三(吉禮三)

歷代帝王陵廟傳心殿先師孔子元聖周公關聖帝君

文昌帝君祭纛祀砲京師群祀附五祀八蠟直省祭厲

歷代帝王廟順治初,建都城西阜成門內,南鄉,正中景德崇聖殿,九楹,東西二廡各七楹,燎爐各一。後為祭器庫,前景德門。門外神庫、神廚、宰牲亭、井亭、鐘樓、齋所咸備。初,明祀歷代帝王,元世祖入廟,遼、金諸帝不與焉。至是用禮臣言,以遼、金分統宋時天下,其太祖應廟祀。元啟疆宇,功始太祖,禮合追崇。從祀諸臣,若遼耶律赫嚕,金尼瑪哈、斡里雅布,元穆呼哩、巴延,明徐達、劉基並入之。

屆日,大臣一人祭正殿,殿祀伏羲,神農,黃帝,少昊,顓頊,帝嚳,唐堯,虞舜,夏禹,商湯,周武王,漢高祖、光武,唐太宗,宋、遼、金太祖、世宗,元太祖、世祖,明太祖,凡廿一帝,祀以太牢。分獻官四人祭兩廡,廡祀風後、力牧、皋陶、夔、龍、伯益、伯夷、伊尹、傅說、周公旦、召公奭、太公望、召虎、方叔、張良、蕭何、曹參、陳平、周勃、鄧禹、馮異、諸葛亮、房玄齡、杜如晦、李靖、郭子儀、李晟、張巡、許遠、耶律赫嚕、曹彬、潘美、張浚、韓世忠、岳飛、尼瑪哈、斡里雅布、穆呼哩、巴延、徐達、劉基,凡功臣四十一,祀以少牢。

十四年,聖祖躬祭,屆時致齋畢,翼日昧爽,駕出西華門,至廟降,入幄次盥訖,入直殿就位上香。三皇位前,二跪六拜,奠帛、爵,讀祝,俱初獻時行。凡三獻,分獻官祀兩廡如儀。遣官則衣朝服。王、公承祭,入景德左門,升左階,位階上,餘入右門,位階下,俱三跪九拜,不飲酒、受胙,不陪祀。

十七年,禮臣議言廟祀帝王,止及開創,應增守成令闢,並罷宋臣潘美、張浚祀,從之。於是增祀商中宗、高宗,周成王、康王,漢文帝,宋仁宗,明孝宗。而遼、金、元太祖皆罷祀。聖祖嗣服,以開創功復之。

六十一年,諭:「帝王崇祀,代止一二君,或廟饗其臣子而不及其君父,是偏也。凡為天下主,除亡國暨無道被弒,悉當廟祀。有明國事,壞自萬歷、泰昌、天啟三朝,神宗、光宗、憙宗不應崇祀,咎不在愍帝也。」於是廷臣議正殿增祀夏啟、仲康、少康、杼、槐、芒、洩、不降、扃、廑、孔甲、皋、發,商太甲、沃丁、太庚、小甲、雍己、太戊、仲丁、外壬、河亶甲、祖乙、祖辛、沃甲、祖丁、南庚、陽甲、盤庚、小辛、小乙、武丁、祖庚、祖甲、廩辛、庚丁、太丁、帝乙,周成王、康王、昭王、穆王、共王、懿王、孝王、夷王、宣王、平王、桓王、莊王、僖王、惠王、襄王、頃王、匡王、定王、簡王、靈王、景王、悼王、敬王、元王、貞定王、考王、威烈王、安王、烈王、顯王、慎靚王,漢惠帝、文帝、景帝、武帝、昭帝、宣帝、元帝、成帝、哀帝、明帝、章帝、和帝、殤帝、安帝、順帝、沖帝、桓帝、靈帝、昭烈帝,唐高祖、高宗、睿宗、玄宗、肅宗、代宗、德宗、順宗、穆宗、文宗、武宗、宣宗、懿宗、僖宗,遼太宗、景宗、聖宗、興宗、道宗,宋太宗、真宗、仁宗、英宗、神宗、哲宗、高宗、孝宗、光宗、寧宗、理宗、度宗、端宗,金太宗、章宗、宣宗,元太宗、定宗、憲宗、成宗、武宗、仁宗、泰定帝、文宗、寧宗,明成祖、仁宗、宣宗、英宗、景帝、憲宗、孝宗、武宗、世宗、穆宗、愍帝,凡百四十三位。其從祀功臣,增黃帝臣倉頡,商仲虺,周畢公高、呂侯、仲山甫、尹吉甫,漢劉章、魏相、丙吉、耿弇、馬援、趙雲,唐狄仁傑、宋璟、姚崇、李泌、陸贄、裴度,宋呂蒙正、李沆、寇準、王曾、範仲淹、富弼、韓琦、文彥博、司馬光、李綱、趙鼎、文天祥,金呼嚕,元博果密、托克托,明常遇春、李文忠、楊士奇、楊榮、於謙、李賢、劉大夏,凡四十人。是歲,世宗御極,依議行,增置神主,為文鑱之石。

乾隆元年,謚明建文帝曰恭閔惠皇帝,廟祀之,位次太祖。復定帝王廟鹿脯、鹿醢,增鹿一,兩廡易醓醢,增豕一。十四年,以唐、虞五臣唯契未祀,乃建殿成湯廟後,有司致饗,如孔廟崇聖祠制。初,帝王廟正殿用青綠琉璃瓦,至十八年重修,改覆黃瓦。

四十九年,諭廷臣:「曩時皇祖敕議增祀,聖訓至公,而陳議者未能曲體,乃列遼、金二朝,而遺東西晉、元魏、前後五代。謂南北朝偏安,則遼、金亦未奄有中夏。即兩晉諸代,因篡而斥,不知三國正統,本在昭烈。至司馬氏以還,南朝神器數易,宋武帝手移晉祚,篡奪無所逃罪,其他祖宗得國不正,子孫但能守成,即為中主。且蜀漢至初唐不乏賢君,安可闕略!洎硃溫以下,或起寇竊,或為叛臣,五十餘年,國統不絕如線。周世宗藉餘業,擴疆宇,卓然可稱,而斥擯弗列,此數百年間,祀典闃如,又豈千秋公論?他若元魏雄據河北,太武、道武,胥勤治理,並宜表章。昔楊維楨著正統辨,謂正統在宋不在遼、金、元,其說甚當。今通禮祀遼、金,黜兩晉諸代,使後世疑本朝區分南北,非禮意也。明神、憙二宗,法紀墜失,愍帝嗣統,事無可為,雖國覆身殉,未可以荒淫例。皇祖徹神、憙,祀愍帝,具見大公。乃議者因復推祀桓、靈,亦思漢之所由亡乎?其再詳議。」尋議增祀兩晉、元魏、前後五代各帝王,並以唐憲宗平亂,金哀宗殉國,亦宜列祀。允行。

同治四年,以散宜生配饗,位次畢公高。高允配饗,位次趙雲。

陵寢之祭,太宗徵明,至燕京,即遣貝勒阿巴泰等赴金太祖、世宗陵致祭。順治建元,禮葬明崇禎帝、後,復詔明十二陵絜禋祀,禁樵牧,給地畝,置司香官及陵戶。歲時祭品,戶部設之。明年,定春、秋仲日致祭,遣官行。六年,定明陵仍設太監,並置房山、金陵陵戶。

八年,定帝王陵寢祀典,淮寧伏羲,滑縣顓頊、帝嚳,內黃商中宗,西華商高宗,孟津漢光武,鄭周世宗,鞏宋太祖、太宗、真宗、仁宗,趙城女媧,榮河商湯,曲阜少昊,東平唐堯,中都軒轅,咸陽周文、武、成、康,涇陽漢高祖、唐宣宗,咸寧漢文帝,長安宣帝,富平後魏孝文帝,三原唐高祖,醴泉太宗,蒲城憲宗,酃神農,寧遠虞舜,會稽夏禹,江寧明太祖,廣寧遼太祖,房山金太祖、世宗,宛平元太祖、世祖,昌平明宣宗、孝宗、世宗,各就地饗殿行之,或因陵寢築壇,惟元陵望祭。十六年,幸畿輔,親酹崇禎帝陵,謚曰莊烈愍皇帝。

凡巡幸所蒞,皆祭陵、廟,有大慶典,祭告亦如之。康熙二十一年,滇亂平,遣官致祭,頒冊文、香、帛,給黃傘一,御仗、龍纛各二,凡成武功,皆祭如典。二十三年,南巡,道江寧,詣明太祖陵,拜奠。諭有司巡察,守陵人防護。越五年,巡會稽,祭禹陵,祝文書御名,行三跪九拜禮。蹕江寧,祭明太祖陵,如祀禹儀。凡時巡祭帝王陵寢,儀同祭廟,率二跪六拜,茲蓋殊典云。三十八年,復南巡,見明太祖陵圮剝,詔依周封杞、宋例,授明裔一官,俾世守弗替。四十二年,西巡,遣祭女媧氏陵,幸陜,遣祭所經諸陵,惟祀周文、武祝文書御名,尊聖也。

六十一年,遺諭,言:「明太祖起布衣,統方夏,駕軼漢、唐、宋諸君。末葉災荒,臣工內訌,寇盜外起,以致社稷顛覆。考其嗣主,未有荒墜顯跡,蓋亦歷數使然。且其制度規模,我朝多所依據。允宜甄訪支派,量授爵秩,俾奉春秋饗祀。」世宗纘緒,遂授硃之璉一等侯世襲,往江寧、昌平致祭,自是歲舉以為常。

帝堯陵向有二:一在平陽,一在濮州。濮州東南穀林,古雷澤也。乾隆元年,修葺釐正,定穀林為舊址,平陽時奠如故。並修神農、虞舜陵廟,置陵戶典守。十一年,以陜西古建都地,帝王陵墓多,命疆吏考其不載會典者,所在令有司防護。十三年,車駕幸曲阜,奠少昊陵,嗣是東巡皆躬祭。十六年,選姒氏子姓一人,授世襲八品官,奉祀禹陵。趙城女媧陵,廟中故有塑像,帝斥其黷慢,徹之,改立神位,禁私禱。

十八年,謁泰陵,禮畢,詣房山祭金太祖陵,賚其裔完顏氏官爵、幣帛。

二十六年,定帝王陵寢與岳鎮海瀆、先師闕里皆遣官行。四十一年,禮臣言:「堯陵見正史者,兩漢地理志云:『濟陰郡成陽有堯塚靈臺。』劉向傳稱『葬濟陰』。晉地理志:『成陽舜所漁,堯塚在西。』宋史禮志:『在濮州雷澤東穀林山。』呂氏春秋,帝王世紀,水經注所引述征記,括地志,太平寰宇記,路史,集古錄諸說,皆與正史符。後漢元和以來,祀典並於其地行。明洪武雖改祀東平,而隸魯境則一。乾隆初,定穀林為堯陵,稽古正訛,萬世可守。嗣後祭告,率由舊章。其平陽一陵,有司祀之,如東平例。」

已,大理寺卿尹嘉銓請罷明宣宗、世宗二陵祭告,廷議以為:「宣宗有善政,不應以一二事生訾議,唯世宗戮忠親佞,實與史合,應停饗祀。」從之。

四十九年,南巡至江寧,祭明太祖陵,禮臣具儀上,三奠酒,每奠一拜。帝命用祀少昊陵例,二跪六拜,不必奠酒,著為令。

五十年,幸湯山,道昌平,親酹明成祖陵,繕葺之,仍建定陵饗殿,並復世宗祀事。

嘉慶元年,罷遣官,敕各省副都統、總兵官舉行。九年,謁東陵,道盤山,閱明陵。故事,往長陵奠醊,遣王大臣致奠餘陵。是日仁宗躬詣,三奠畢,乃三拜。

望祭元太祖、世祖陵,向在德勝門外,位暢春園、圓明園南,帝以為乖制。命嗣後行慶典,改於清河以北、昌平以南擇地行禮。

道光十六年,定明陵春秋致祭,由襲侯往行,餘以其族官品峻者攝之,或遣散秩大臣,為永制。

光緒七年,諭禁開墾明陵旁近地畝。

傳心殿順治十四年,沿明制舉經筵,祭先師孔子弘德殿。康熙十年續舉,遣官告祭。二十四年,規建傳心殿,位文華殿東。正中祀皇師伏羲、神農、軒轅,帝師堯、舜,王師禹、湯、文、武,南鄉。東周公,西孔子。祭器視帝王廟。歲御經筵,前期遣大學士祗告。祭傳心殿自此始。

明年,帝將御經筵,詔言:「先聖、先師,傳道垂統,炳若日星。朕遠承心學,效法不已,漸近自然。施之政教,庶不與聖賢相悖,其躬詣行禮。」祀日具香燭,鉶一,籩、豆各二,奠帛、爵,讀祝,以祭。帝御袞服,行二跪六拜禮。太子春秋會講,亦先祭告焉。月朔望遣太常卿供酒果上香。雍正四年,定本日行祗告禮,自是以為常。

乾隆六年,親祭傳心殿,六十年歸政,再行之。歷仁宗、宣宗、文宗,並親詣祗告,後不復行。經筵儀制,別詳嘉禮。

至聖先師孔子崇德元年,建廟盛京,遣大學士範文程致祭。奉顏子、曾子、子思、孟子配。定春秋二仲上丁行釋奠禮。世祖定大原,以京師國子監為大學,立文廟。制方,南鄉。西持敬門,西鄉。前大成門,內列戟二十四,石鼓十,東西舍各十一楹,北鄉。大成殿七楹,陛三出,兩廡各十九楹,東西列舍如門內,南鄉。啟聖祠正殿五楹,兩廡各三楹,燎爐、瘞坎、神庫、神廚、宰牲亭、井亭皆如制。

順治二年,定稱大成至聖文宣先師孔子,春秋上丁,遣大學士一人行祭,翰林官二人分獻,祭酒祭啟聖祠,以先賢、先儒配饗從祀。有故,改用次丁或下丁。月朔,祭酒釋菜,設酒、芹、棗、慄。先師四配三獻,十哲兩廡,監丞等分獻。望日,司業上香。

正中祀先師孔子,南鄉。四配:復聖顏子,宗聖曾子,述聖子思子,亞聖孟子。十哲:閔子損、冉子雍、端木子賜、仲子由、卜子商、冉子耕、宰子予、冉子求、言子偃、顓孫子師,俱東西鄉。西廡從祀:先賢澹臺滅明、宓不齊、原憲、公冶長、南宮適、公晳哀、商瞿、高柴、漆雕開、樊須、司馬耕、商澤、有若、梁鱣、巫馬施、冉孺、顏辛、伯虔、曹血阜、冉季、公孫龍、漆雕徒文、秦商、漆雕哆、顏高、公西赤、壤駟赤、任不齊、石作蜀、公良孺、公夏首、公肩定、後處、鄡單、奚容■E9、罕父黑、顏祖、榮旗、句井疆、左人郢、秦祖、鄭國、縣成、原亢、公祖句茲、廉潔、燕伋、叔仲會、樂欬、公西輿如、狄黑、邽巽、孔忠、陳亢、公西■E9、琴張、顏之僕、步叔乘、施之常、秦非、申棖、顏噲、左丘明、周敦頤、張載、程顥、程頤、邵雍、硃熹,凡六十九人;先儒公羊高、穀梁赤、伏勝、孔安國、毛萇、後蒼、高堂生、董仲舒、王通、杜子春、韓愈、司馬光、歐陽修、胡安國、楊時、呂祖謙、羅從彥、蔡沈、李侗、陸九淵、張栻、許衡、真德秀、王守仁、陳獻章、薛瑄、胡居仁,凡二十八人。

啟聖祠,啟聖公位正中,南鄉。配位:先賢顏無繇、曾點、孔鯉、孟孫氏,東西鄉。兩廡從祀:先儒周輔成、程、蔡元定、硃松。

九年,世祖視學,釋奠先師,王、公、百官,齋戒陪祀。前期,衍聖公率孔、顏、曾、孟、仲五氏世襲五經博士,孔氏族五人,顏、曾、孟、仲族各二人,赴都。暨五氏子孫居京秩者咸與祭。是歲授孔氏南宗博士一人,奉西安祀。

十四年,給事中張文光言:「追王固誣聖,而『大成文宣』四字,亦不足以盡聖,宜改題『至聖先師』。」從之。康熙六年,頒太學中和韶樂。二十二年,御書「萬世師表」額懸大成殿,並頒直省學宮。二十六年,禦制孔子贊序、顏曾思孟四贊鑱之石。揭其文頒直省。

五十一年,以硃子昌明聖學,升躋十哲,位次卜子。尋命宋儒範仲淹從祀。

雍正元年,詔追封孔子五代王爵,於是錫木金父公曰肇聖,祈父公曰裕聖,防叔公曰詒聖,伯夏公曰昌聖,叔梁公曰啟聖,更啟聖祠曰崇聖。肇聖位中,裕聖左,詒聖右,昌聖次左,啟聖次右,俱南鄉。配饗從祀如故。

二年,視學釋奠,世宗以祔饗廟庭諸賢,有先罷宜復,或舊闕宜增,與孰應祔祀崇聖祠者,命廷臣考議。議上,帝曰:「戴聖、何休非純儒,鄭眾、盧植、服虔、範甯守一家言,視鄭康成淳質深通者有間,其他諸儒是否允協,應再確議。」復議上。於是復祀者六人:曰林放、蘧瑗、秦冉、顏何、鄭康成、範甯。增祀者二十人:曰孔子弟子縣亶、牧皮,孟子弟子樂正子、公都子、萬章、公孫丑,漢諸葛亮,宋尹焞、魏了翁、黃幹、陳淳、何基、王柏、趙復,元金履祥、許謙、陳澔,明羅欽順、蔡清,國朝陸隴其。入崇聖祠者一人,宋橫渠張子迪。

尋命避先師諱,加「邑」為「邱」,地名讀如期音,惟「圜丘」字不改。

四年八月仲丁,世宗親詣釋奠。初,春秋二祀無親祭制,至是始定。犧牲、籩豆視丁祭,行禮二跪六拜,奠帛獻爵,改立為跪,仍讀祝,不飲福、受胙。尚書分獻四配,侍郎分獻十一哲兩廡。明年,定八月二十七日先師誕辰,官民軍士,致齋一日,以為常。又明年,御書「生民未有」額,頒懸如故事。十一年,定親祭儀,香案前三上香。

乾隆二年,諭易大成殿及門黃瓦,崇聖祠綠瓦。復元儒吳澄祀。三年,升有子若為十二哲,位次卜子商。移硃子次顓孫子師。

是歲上丁,帝親視學釋奠,嚴駕出,至廟門外降輿。入中門,俟大次,出盥訖,入大成中門,升階,三上香,行二跪六拜禮。有司以次奠獻。正殿,分獻官升東、西階,入左、右門,詣四配、十二哲位前,兩廡分獻官分詣先賢、先儒位前,上香奠獻畢,帝三拜,亞獻、終獻如初。釋奠用三獻始此。其祭崇聖祠,拜位在階下,承祭官升東階,入左門,詣肇聖王位前上香畢,分獻官升東、西階,入左、右門,分詣配位及兩廡從位前上香,三跪九拜。奠帛、讀祝,初獻時行。凡三獻,禮畢。自是為恆式。

十八年,改正太學丁祭牲品,依闕里例用少牢,十二哲東西各一案,兩廡各三案。崇聖祠四配,兩廡東西各一案,十二哲位各一帛,東西共二篚。其分獻,正殿東西,翰林官各奠三爵;西廡國子監四人,共奠三爵;十二哲兩廡奉爵用肄業諸生。定兩廡位序,按史傳年代先後之。

三十三年,葺文廟成,增大門「先師廟」額,正殿及門曰「大成」,帝親書榜,制碑記。選內府尊彞中十器,凡犧尊、雷文壺、子爵、內言卣、康侯爵、鼎盟簋、雷紋觚、召仲簋、素洗、犧首罍各一,頒之成均。

五十年,新建闢雍成,親臨講學,釋奠如故。嘉慶中,兩舉臨雍儀。

道光二年詔劉宗周,三年湯斌,五年黃道周,六年陸贄、呂坤,八年孫奇逢,從祀先儒。八年,湖北學政王贈芳請祀陳良,帝以言行無可考,寢其議。未幾,御史牛鑒以李顒請,部議謂然,帝斥之。十六年,詔祀孔子不得與佛、老同廟。是後復以宋臣文天祥、宋儒謝良佐侑饗云。咸豐初,增先賢公明儀,宋臣李綱、韓琦侑饗。

三年二月上丁,行釋菜禮,越六日,臨雍講學,自聖賢後裔,以至太學諸生,圜橋而聽者雲集。

七年,增聖兄孟皮從祀崇聖祠,先賢公孫僑從祀聖廟,宋臣陸秀夫、明儒曹端並入之。

十年,用禮臣言,從祀盛典,以闡聖學、傳道統為斷。餘各視其所行,分入忠義、名宦、鄉賢。至名臣碩輔,已配饗帝王廟者,毋再滋議。同治二年,御史劉毓楠以祔祀新章過嚴,如宋儒黃震輩均不得預,恐釀人心風俗之憂,帝責其迂謬。

是歲魯人毛亨,明呂棻、方孝孺並侑饗。於是更訂增祀位次,各按時代為序。乃定公羊高、伏勝、毛亨、孔安國、後蒼、鄭康成、範甯、陸贄、範仲淹、歐陽脩、司馬光、謝良佐、羅從彥、李綱、張栻、陸九淵、陳淳、真德秀、何基、文天祥、趙復、金履祥、陳澔、方孝孺、薛瑄、胡居仁、羅欽順、呂棻、劉宗周、孫奇逢、陸隴其列東廡,穀梁赤、高堂生、董仲舒、毛萇、杜子春、諸葛亮、王通、韓愈、胡瑗、韓琦、楊時、尹焞、胡安國、李侗、呂祖謙、黃幹、蔡沈、魏了翁、王柏、陸秀夫、許衡、吳澄、許謙、曹端、陳獻章、蔡清、王守仁、呂坤、黃道周、湯斌列西廡,並繪圖頒各省。七年,以宋臣袁燮、先儒張履祥從祀。光緒初元,增入先儒陸世儀。自是漢儒許慎、河間獻王劉德,先儒張伯行,宋儒輔廣、游酢、呂大臨並祀焉。

二十年仲秋上丁,親詣釋奠,仍用飲福、受胙儀。

三十二年冬十二月,升為大祀。先師祀典,自明成化、弘治間,已定八佾,十二籩、豆。嘉靖九年,用張璁議,始釐為中祀。康熙時,祭酒王士禛嘗請酌採成、弘制,議久未行。至是命禮臣具儀上,奏言:「孔子德參兩大,道冠百王。自漢至明,典多缺略。我聖祖釋奠闕里,三跪九拜,曲柄黃蓋,留供廟庭。世宗臨雍,止稱詣學。案前上香、奠帛、獻爵,跪而不立。黃瓦飾廟,五代封王。聖誕致齋,聖諱敬避。高宗釋奠,均法聖祖,躬行三獻,垂為常儀。崇德報功,遠軼前代。已隱寓升大祀至意。世宗諭言:『堯舜禹湯文武之道,賴孔子以不墜。魯論一書,尤切日用,能使萬世倫紀明,名分辨,人心正,風俗端,此所以為生民未有也。』聖訓煌煌,後先一揆。近雖學派紛歧,而顯示欽崇,自足收經正民興巨效。」疏上,於是文廟改覆黃瓦,樂用八佾,增武舞,釋奠躬詣,有事遣親王代,分獻四配用大學士,十二哲兩廡用尚書。祀日入大成左門,升階入殿左門,行三跪九拜禮。上香,奠帛、爵俱跪。三獻俱親行。出亦如之。遣代則四配用尚書,餘用侍郎,出入自右門,不飲福、受胙。崇聖祠本改親王承祭,若代釋奠,則以大學士為之。分獻配位用侍郎,西廡用內閣學士。餘如故。三十四年,定文廟九楹三階五陛制。

御史趙啟霖請以王夫之、黃宗羲、顧炎武從祀。下部議。先是署禮部侍郎郭嵩燾、湖北學政孔祥霖請夫之從祀,江西學政陳寶琛請宗羲、炎武從祀,並被駁。至是部議謂:「三人生當明季,毅然以窮經為天下倡,德性問學,尊道並行,第夫之黃書,原極諸篇,託旨春秋;宗羲明夷待訪錄,原君、原臣諸篇,取義孟子,似近偏激。惟炎武醇乎其醇,應允炎武從祀,夫之、宗羲候裁定。」帝命並祀之。

闕里文廟,有事祭告,具前祭告篇。春、秋致祭同太學。康熙中,聖祖東巡親祭,禮部具儀。駐蹕次日,帝服龍袞,行在儀仗具陳,行禮二跪六拜,配位、十哲、兩廡、啟聖祠,皆遣官分獻。扈從諸臣,文官知府、武官副將以上,衍聖公暨各氏子孫在職者,咸陪祀。聖心猶未安,命更議。尋定迎神、送神俱三跪九拜,惟樂章與國學小異,可令太常司樂及樂舞生先往肄習。帝親制祝文。祀日詣廟,至奎文閣前降輦,如齋所小憩,自大次出,入大成門,登殿釋奠畢,御詩禮堂講書。禮成,周視廟庭車服、禮器。更常服,駕如孔林,跪奠酒,三爵,三拜,賜衍聖公以下銀幣有差。留曲柄黃蓋陳廟庭。擴孔林地畝,蠲其稅。建廟碑,御書文鑱石。又建子思子廟,仿顏、曾、孟三廟制。

三十二年,修文廟成,皇子往祭,行禮杏壇。雍正二年,曲阜廟災,遣官詣闕里祭慰,敕大臣重建,並令闕裏司樂遣人赴太常習樂舞,冠服悉準太學式為之。八年,廟成,黃瓦畫棟,悉仿宮殿制。凡登、簋、簠、鉶、籩、豆、尊、爵,頒自上方。勒碑如故事。特詔皇五子往祭。

乾隆八年,定闕里聖廟樂章。二十三年,東巡親祭如往制。遣大臣祭顏、曾、思、孟專廟。勒御制四賢贊於石。其盛京學宮所需樂器,乾隆中始敕府尹遵皇朝禮器圖造作,鎛鐘、特磬,制出內廷,特頒太學暨各省學宮,並令府丞選佾生精音律者送太常習舞。厥後以熱河為時巡所,黌序肇興,定大成殿龕案如太學式,祭器、樂器亦如之。

至各省府、州、縣釋奠,以所在印官承祭,禮如太學,順治初行之。雍正五年,定制各省督、撫、學政上丁率屬致祭。學政蒞試時,先至文廟行禮,府、州、縣官率屬於治所文廟行。乾隆六年,敕直省學宮設先賢、先儒神位。同治初,頒從祀先儒位次圖。光緒末,升大祀,各省文廟規制、禮器、樂舞暨崇聖祠祭品,並視太學,禮節悉從舊。

元聖周公順治十七年,給事中黏本盛奏請文廟後別立傳聖祠。下部議,禮臣言:「祭祀周公,向在太學。至唐顯慶間,以公制禮作樂,功侔帝王,就饗儒宮,欲尊反貶。始定配饗帝王廟,既不與孔子並祭太學,乃反立傳聖祠於其後,殊失尊崇本意也。」事遂寢。康熙二十三年,聖祖祀闕里,詔言:「周公古大聖人,制禮作樂,垂法萬世,廟在曲阜,應行致祭。」乃遣親王及禮部尚書往焉。親制祝文。祭禮,三獻。祭品:羊一、豕一、果五盤、尊一、爵三,敕有司治辦。明年,授東野氏一人博士,奉祀祠廟。二十六年,御書周公廟碑文,依文廟式,勒之貞氏。乾隆十二年,東巡,增登一,鉶二,簋、簠各二,籩、豆各八,遣親王一人行禮。其祀配饗魯公,遣禮部尚書行。明年,幸曲阜,親詣上香,一跪三拜。自是東巡親詣以為常。四十三年,依孔氏南宗例,置當陽博士,奉祀陵墓。

關聖帝君清初都盛京,建廟地載門外,賜額「義高千古」。世祖入關,復建廟地安門外,歲以五月十三日致祭。順治九年,敕封忠義神武關聖大帝。雍正三年,追封三代公爵,曾祖曰光昭,祖曰裕昌,父曰成忠,供後殿。增春、秋二祭。洛陽、解州後裔並授五經博士,世襲承祀。尋定春、秋祀儀,前殿大臣承祭,後殿以太常長官。屆日質明,大臣朝服入廟左門,升階就拜位,上香,行三跪九拜禮。三獻,不飲福、受胙。祭後殿二跪六拜。十一年,增當陽博士一人奉塚祀。

乾隆三十三年,以壯繆原謚,未孚定論,更命神勇,加號靈佑。殿及大門,易綠瓦為黃。四十一年,詔言:「關帝力扶炎漢,志節懍然,陳壽撰志,多存私見。正史存謚,猶寓譏評,曷由傳信?今方錄四庫書,改曰忠義。武英殿可刊此旨傳末,用彰大公。」嘉慶十八年,以林清擾禁城,靈顯翊衛,命皇子報祀如儀,加封仁勇。道光中,加威顯。咸豐二年,加護國。明年,加保民。於是躋列中祀,行禮三跪九叩,樂六奏,舞八佾,如帝王廟儀。五月告祭,承祭官前一日齋,不作樂,不徹饌,供鹿、兔、果、酒。旋追封三代王爵,祭品視崇聖祠。加精誠綏靖封號,御書「萬世人極」額,摹勒頒行。同治九年,加號翊贊。光緒五年,加號宣德。

直省關帝廟亦一歲三祭,用太牢。先期承祭官致齋,不理刑名,前殿印官,後殿丞、史,陳設禮儀,略如京師。

文昌帝君明成化間,因元祠重建。在京師地安門外,久圮。嘉慶五年,潼江寇平,初寇闚梓潼,望見祠山旗幟,卻退。至是御書「化成耆定」額,用彰異績。發中帑重新祠宇,明年夏告成,仁宗躬謁九拜,詔稱:「帝君主持文運,崇聖闢邪,海內尊奉,與關聖同,允宜列入祀典。」於是大學士硃珪撰碑記,略言:「文昌星載天官書,所謂『斗魁六星,戴匡曰文昌宮』是也。尚書『禋六宗』,孔疏引鄭玄云:『皆天神,司中、司命,文昌第五、第四星也。』周禮大宗伯:『以燎祀司中、司命。』鄭注謂文昌星。然則文昌之祀,始有虞,著周禮,漢、晉且配郊祀。元命苞云:『上將建威武,次將正左右,貴相理文緒,司祿賞功進士。』是爵祿、科舉職司久矣。又言帝君周初為張仲,孝友顯化,隋、唐為王通,徵李商隱張亞子廟詩,讀孫樵祭梓潼神君文,化書:唐開元命為左丞,通考:僖宗封為濟順王,宋真宗改號英顯,哲宗加封輔元開化文昌司祿帝君,元加號宏仁,蓋可考見雲。」禮官遂定議。

歲春祭以二月初三誕日,秋祭,仲秋諏吉將事,遣大臣往。前殿供正神,後殿則祀其先世,祀典如關帝。咸豐六年,躋中祀,禮臣請崇殿階,拓規制,遣王承祭,後殿以太常長官親詣,二跪六拜,樂六奏,文舞八佾,允行。直省文昌廟有司以時饗祀,無祠廟者,設位公所祭之。畢,徹位隨祝帛送燎。

旗纛之祭天命十年,定沈陽,還軍扈渾河,刲牛祭纛。天聰元年征朝鮮,明年凱旋,並立纛拜天。自是出征班師祭纛以為常,時旗纛附祀關帝廟也。世祖入關後,始行望祭。

凡親征諏吉啟行,先於堂子內門外建御營黃龍大纛,按翼分設八旗大纛、火器營大纛各八,列其後,並北鄉。帝御戎服佩刀,出宮乘騎,入堂子街門降。圜殿禮畢,出內門致禮纛神,率從征將士三跪九拜,不贊。禮成樂作,鑾駕啟行,領侍衛內大臣、司纛侍衛率親軍舉纛從。

凱旋致祭,屆日陳法駕鹵簿,自郊外五里訖堂子門外。駕至郊,降輿拜纛如儀。命將出師亦如之。聖祖征噶爾丹凱旋,翼日為壇安定門外,致祭隨營旗纛,用太牢,始遣大臣行禮。雍正初,定三年一祭。

凡旗纛皆庋內府,祭則設之。各省祭旗纛,則遣武官戎服行禮焉。

砲位之祭,天聰五年,造紅衣砲,鐫曰天佑助威大將軍,遂攜以毀於子章臺,克大凌河,行軍攜紅衣砲始此。

厥後敕漢軍齎砲進關。世祖奠鼎燕京,定制以歲季秋朔,陳砲位盧溝橋沙鍋村,席地為壇,西鄉,以八旗漢軍都統將事。分旗翼列,用果品、少牢。屆時先鑲黃旗砲位,都統御補服,上香,三跪九拜,三獻,讀祝。餘七砲位亦如之。副都統以次陪祀。聖祖凱旋,設壇德勝門外,祭品如祭纛。世宗亦定三年一祭。

乾隆十四年,滿洲火器營始祭八旗子母砲神,總統承祭,如漢軍祀砲儀。其後定滿洲祀砲依漢軍例,季秋赴盧溝橋演砲,即以其日祭焉。三十年,祀砲始用祝版,並專設祭器。

群祀先醫,初沿明舊,致祭太醫院景惠殿,歲仲春上甲,遣官行禮。祀三皇,中伏羲,左神農,右黃帝。四配:句芒、風後、祝融、力牧。東廡僦貸季、岐伯、伯高、少師、雷公、伊尹、淳於意、華陀、皇甫謐、巢元方、韋慈藏、錢乙、劉完素、李果十四人,西則鬼臾區、俞跗、少俞、桐君、馬師皇、扁鵲、張機、王叔和、葛洪、孫思邈、王冰、硃肱、張元素、硃彥修十四人。禮部尚書承祭。兩廡分獻,以太醫院官。禮用三跪九拜。三獻。雍正中,命太醫院官咸致齋陪祀。

都城隍廟有二,舊沈陽城隍廟,自元訖明,祀典勿替。清初建都後,升為都城隍廟,有司以時致祭。其在燕京者,建廟宣武門內。順治八年仲秋,遣太常卿致祭,歲以為常。用太牢,禮獻如祀先醫。萬壽節遣祭,加果品。雍正中,改遣大臣,嗣復命親王行禮。禁城城隍廟建城西北隅。皇城城隍廟建西安門內,曰永佑宮,萬壽節或季秋,遣內府大臣承祭,用少牢。

北極佑聖真君廟,建地安門外日中坊橋東,曰靈明顯佑宮。順治中,定制萬壽節遣官祭,後改遣大臣。設果盤五、餅餌盤十五、茶■D9三、行禮三跪九拜。

火神廟,建日中坊橋西。康熙初,定歲六月二十三日遣太常卿祭,後改遣大臣。用少牢。雍正中,改太牢。帛初用白,乾隆中改用赤。餘如祀北極儀。

東嶽廟,在朝陽門外,歲祭以萬壽節。

龍神之祭,黑龍潭廟建西北金山巔,聖祖、世宗親制碑記。乾隆五年,錫號「昭靈沛澤」。玉泉山廟,九年錫號「惠濟慈佑」。昆明湖祠,舊曰廣潤靈雨祠,錫號「安佑普濟」,嘉慶中,加「沛澤廣生」。京畿旱,帝親禱黑龍潭廟。乾隆四十六年,錫號「昭靈廣濟」。嘉慶間,始列祀典,遣散秩大臣往祭惠濟祠。河神廟建綺春園內,祀天后、龍神、河神,並春、秋致祭,遣圓明園大臣將事。儀品俱視都城隍廟。

其祀之無定時、定所,及有司以時專祭者,後土司工之神,順治初制,凡大興作,因其方築左右壇,建採棚,遣官往祭,用少牢餅果。若大工迎吻,祭琉璃窯神暨各門神,如祭司工禮。咸豐間,錫號圓明園春雨軒司工神曰昭休敷禧真君,土母曰夫人。命內府大臣春、秋奉祀。司機神,順治季年設織造局,始行祭告,禮部長官主之。司倉神,通州三倉,舊惟西倉有祠。京內七倉,惟右翼興平倉有祠,雍正間重葺。繇是左翼置廟海運倉。京外五倉,置廟儲濟倉,並立神位。倉場侍郎承祭,用少牢、果品,倉監督陪祀,二跪六拜。諸祭將事以黎明,與祭者咸朝服,此其大凡也。至特旨建祠京師者,具見後簡。

若夫直省御災捍患有功德於民者,則錫封號,建專祠,所在有司秩祀如典。

世祖朝,宿遷祀河神宋謝緒。

聖祖朝,成都祀諸葛亮;福建暨各省祀天後宋林氏女。

世宗朝,各省祀猛將軍元劉承忠。先是直隸總督李維鈞奏:「蝗災,土人禱猛將軍廟,患輒除。」於是下各省立廟祀。已,兩江總督查弼納亦言:「猛將軍廟祀所在無蝗害,無廟處皆為災。」被訶責。詔言:「水旱蝗災,疆吏當修省,勿專事祈禱。」錢塘祀伍員,封英衛公;臨安祀錢鏐,封誠應王;蕭山祀宋張夏,封靜安公;紹興祀明知府湯紹忠,封寧江伯,後司事莫龍附焉;汶上祀明尚書宋禮,封寧漕公,老人白英封永濟神附焉;灌縣祀秦蜀守李冰,封敷澤興濟通裕王,子二郎,為承績廣惠英顯王;德清祀元戴繼元,封保濟顯佑侯;徐聞祀故水師副將江啟龍,封英佑驍騎將軍,後附祀張瑜,錫號「襄靖普佑」;江南山陽祀唐許遠,封威靈顯佑王;浮梁祀張巡,錫號「顯佑安瀾」。

高宗朝,陳留祀河神守才,後建廟江南,曰靈佑觀;清河祀明張襄,封彰靈衛漕將軍;廣西祀蜀將武當,封顯佑英濟廣福王;濱河各縣祀故河督硃之錫,封助順永寧侯。

仁宗朝,追封天后父積慶公,母曰夫人;永綏鎮筸祀宋楊灝,封宣威助順靖遠侯;蕪湖祀蜀漢孫夫人;曹縣祀張桓侯飛、趙將軍云;江南山陽祀湖神譚氏,封昭靈顯佑水府都君;南昌祀旌陽令許遜,封靈感普濟神;直省祀純陽演正警化孚佑帝君唐呂巖;仁和祀孚順侯宋蔣崇仁,弟孚惠侯崇義、孚佑侯崇信;會稽祀漢曹娥,封福應夫人;慈谿祀天井潭神宋劉揚祖;義烏祀明漕運總管陳道興;都昌左蠡鎮祀元將軍長興;湖州、蘇州祀太湖神明王天英;高郵祀露筋祠神;淮揚運河祀康澤靈應侯宋耿裕德;漢城祀竇孝婦;錢塘祀金華將軍五代曹杲。

宣宗朝,翁源祀元詹姓三神,並封侯。建德祀故知府王光鼎;浙江新城祀宣靈王周雄;黔陽祀殉難知縣周文煜;鄞縣祀濱江靈廟神宋晁說之,封孚惠侯;白鶴山廟神唐任侗;茅山廟神張仁皓;長沙祀元李育萬,封廣濟李真人;莆田祀宋長樂錢氏室女;蕭山祀江塘神元楊伯遠妻王氏;又祀唐董戈管、張實、張耀、張聖,宋盧萬,故知縣賈國楨、姚文熊;浙江祀太湖神晉張賁;鄒溪廟神宋裴肅;仁和祀宋施全為興福廟神;奉化祀元馬稱德為進林廟神;滕縣祀明馮克利為三界廟神;慈谿祀漢張竟暨子齊芳;杭州祀靈感廣大觀音大士,加封慈濟;郫縣祀古蜀王杜宇,開明;綿州祀漢蔣琬;新寧祀宋陳仲真;欽州祀故副將景懋;永定河、張秋鎮並祀九龍陳將軍;福建歸化祀福順夫人莘氏。

文宗朝,臨清、東昌、河南正陽關並祀金龍四大王,靖遠、鎮遠、綏遠三侯,俱晉王爵;永城祀觀音大士、孚佑帝君;潮陽及江南高堰祀顯佑安南神;潮陽祀威顯靈佑王;廣東祀明石康令羅神;長沙祀晉陶淡暨侄烜,並號陶真人;桂平祀孚應惠濟王宋甘佃;連江祀崇福昭惠慈濟夫人唐陳昌女,孚濟將軍黃助暨弟昭遠將軍;會稽祀回向廟神漢陳德道;杭州、嘉興、湯陰、武昌並祀宋岳飛;三水祀玄壇正一真神;靈山祀明硃將軍統鑒;潮州祀安濟王漢王伉;奉化祀漢陳鴻;歸善祀明王守仁、後唐何澤、元譚道;歙縣祀唐汪華,陳程靈洗暨子文季;嚴州祀孚惠王唐邵仁祥;鎮洋祀元忠正王李祿、宋忠惠侯楊滋;壽寧祀懿政天仙馬氏女;全州祀無量壽佛唐周全真,威信侯柴崇泆;攸縣祀唐杉仙真人陳皎;淳安祀吳山陰侯賀齊;宜章祀唐武陵侯黃師浩;四會祀宋阮大師子鬱、梁化師慈能;南雄祀聖化夫人練氏;淮安祀周王子晉;封普惠祖師。

穆宗朝,加金龍四大王封號至四十字,廟祀封丘、臨清、張秋鎮、六塘河;封故河督慄毓美誠孚慄大王,附祀鄆城神廟;廣東祀大鑒禪師盧惠能,靈通侍者陳道明;寶山祀故知縣胡仁濟;廣州祀唐陳四公、五公;廣豐祀明太保胡德濟;瀏陽祀宋指揮溫康孟;襄垣祀昭澤王唐焦姓神;山陰祀元楊興嗣;福建永安祀唐田王李肅;廣東祀石龍太夫人馮洗氏,錫號「慈佑夫人」;上饒祀鷹武將軍唐李德勝;善化祀朗公普濟真君唐邱姓神,明李真人潤濟;羅定祀殉難州同金芳,封護國神;貴州祀唐南霽雲;會昌祀晉賴公神;新會祀宋戴存仁;上虞祀顯應侯宋陳賢,封護國潮神;張秋鎮祀明楊四將軍,故河督黎世序,封孚惠河神;長沙祀周真人福壽,瞿真人餐岑;溫州祀唐楊精義;陽曲祀晉大夫竇犨;孟縣祀晉趙武;上虞祀唐桑憲保,封桑王神;濱河祀故祥河同知王仁福,封將軍;南安祀宋廣澤尊王郭忠;棲霞祀元邱真人處機;麻城祀五腦山土主神張瑞;高要祀太保神宋盧僧;邵陽祀唐鄭洞天;黔陽祀唐孝子劉三將軍;江都祀漢杜女仙暨康女仙紫霞;平江祀唐楊孝仙耀庭。

德宗朝,甌寧祀三聖夫人;福建祀白玉蟾真人葛長庚;增城祀賓公佛;上杭祀黃仙師、幸仙師;介休祀空王古佛田志超;雙流祀僧大朗;廣德祀漢張渤;項城祀傅宗龍;寧武祀明周遇吉;封丘祀漢百里嵩;長樂祀唐郭子儀;長沙祀雷萬春;交城祀晉大夫狐突;潞城祀唐李靖;臨海祀唐林洪;雲陽祀張飛;廣西祀漢馬援,明王守仁。

光緒二十七年,兩宮西狩,回鑾,御舟濟河,波濤不驚,特加大王、將軍諸封號。凡予祀皆有封號,不悉紀,紀其著者。或前朝已封,今復加號,或當代始封,後屢加號,則悉略之。定例,封號至四十字不復加,間有之,非常制,止金龍四大王四十字外加號錫祜,天後加至六十字,復錫以嘉佑云。

五祀,初循舊制,每歲暮合祭太廟西廡下。順治八年定制,歲孟春宮門外祭司戶神,孟夏大庖前祭司灶神,季夏太和殿階祭中霤神,孟秋午門西祭司門神,孟冬大庖井前祭司井神,中霤門、午門二祀,太常寺掌之,戶、灶、井三祀,內務府掌之,於是始分祭,旋復故。逮聖祖釐祀典,再罷之,並停專祀。惟十二月二十三日,宮中祀灶以為常。

八蠟之祭,清初關外舉行,廟建南門內,春、秋設壇望祭。世祖入關,猶踵行之。乾隆十年,詔罷蠟祭。時廷臣猶力請行古蠟祭,高宗諭曰:「大蠟之禮,昉自伊耆,三代因之,古制夐遠,傳注參錯。八蠟配以昆蟲,後儒謂害稼不當祭。月令:『祈年於天宗。』蠟祭也。注云『日、月、星、辰』,則所主又非八神。至謂合聚萬物而索饗之,神多位益難定。蠟與臘冠服各殊,或謂臘即蠟,或謂蠟而後臘。自漢臘而不蠟,魏、晉以降,廢置無恆。或溺五行家言,甚至天帝、人帝及龍、麟、硃鳥,為座百九十二,議者謂失禮。蘇軾曰:『迎貓則為貓尸,迎虎則為虎尸,近俳優所為。』是其跡久類於戲也,是以元、明廢止不行。況蠟祭諸神,如先嗇、司嗇、日、月、星、辰、山、林、川、澤,祀之各壇廟,民間報賽,亦借蠟祭聯歡井閭。但各隨其風尚,初不責以儀文,其悉罷之。」自是無復蠟祭矣。

祭厲明制,自京師訖郡、縣,皆祭厲壇。清初建都盛京,厲壇建地載門外。自世祖入關後,京師祭厲無聞焉。唯直省城隍合祀神祇壇,月朔、望有司詣廟上香,二跪六拜,暘雨愆期則禱。復以城隍主厲壇祀。

順治初,直省府、州、縣設壇城北郊,歲以清明日、七月十五日、十月朔日,用羊三、豕三、米飯三石、香燭、酒醴、楮帛祭本境無祀鬼神。府曰郡厲,縣曰邑厲。先期備祭物,有司詣城隍廟以祭厲告。屆日設燎爐壇南,奉城隍神位安壇正中。詣神位前跪,三上香,行禮用三拜。送燎,奠三爵,退,神位復初。


\end{pinyinscope}