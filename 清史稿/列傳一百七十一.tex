\article{列傳一百七十一}

\begin{pinyinscope}
林培厚李象鵾李宗傳王鳳生黃冕俞德淵姚瑩

林培厚,字敏齋,浙江瑞安人。嘉慶十三年進士,選庶吉士,授編修。出為四川重慶知府。啯匪帶刀異常制,禁鍛者毋制賣,有犯則坐。沿江渡船為盜資,籍而稽其出入,刻姓名船側,盜為衰息。民習天主教,搜其書,批抉繆妄,聞者多悔悟。署川東道,所屬雷波民、夷忿爭,或覬覦邀功,請發兵,培厚不應,立縛治其魁,餘悉貸遣。總督蔣攸銛器之,稱為蜀中良吏之最。母憂歸,服闋,授直隸天津府。畿輔大水,天津地窪下,災尤劇,培厚遍行屬縣,賑活饑民七萬有奇。奉天、臺灣商米先後抵海口,議以官錢收買,委曲劑量,商民交利,而官不費。時蔣攸銛移督直隸,詔舉賢吏,遂薦之,不旬日,擢大順廣道。畿南澇後,大興水利。培厚先在天津治澱河,至大名治新衛河、洺河,浚築悉中程度。培厚數以時事利病、屬吏賢否語攸銛,為布政使屠之申所忌。及攸銛入相,那彥成代之,坐河北旱荒施賑不如法,解培厚任,宣宗夙知其能,改授湖北糧儲道。時河患淺涸,漕舟數阻。攸銛以大學士出督兩江,期八省漕以首夏畢渡河,乘清水盛漲,浮渡遄利。培厚所部尤速達,為嘉慶以來數十年所未有,攸銛特疏陳給敘。歷三運無誤,上意方鄉用,以勞卒於通州運次。

李象鵾,字云皋,湖南長沙人。嘉慶十六年進士,選庶吉士,授編修。道光二年,出為直隸宣化知府。歲饑,禁奸販,安屯戶,煮粥以賑,民無失所。課士有法,一變邊郡弇陋之習。調正定,再調保定。蔣攸銛、那彥成先後為總督,皆倚如左右手。象鵾持正無撓,擢通永道,調河南鹽道。治漕嚴,弁丁懍懍,禁胥役藉雇剝船擾民,請潞鹽仍歸商運,民便之。丁父艱歸,服闋,補江西吉南贛寧道。轄境與粵東犬牙相錯,多伏莽,屬縣僻瘠,幾不可治,象鵾掃除積弊,境內秩然。擢江蘇按察使,署江寧布政使。時陶澍為總督,賴其佐理焉。調貴州按察使。仁懷奸民為亂,株連眾,治之無枉縱。擢布政使,禁漢奸盤剝苗民,多惠政。二十四年,以假去職。洎入覲,詔以三品京堂候補。未幾,乞歸。

李宗傳,字孝曾,安徽桐城人。嘉慶三年舉人。授浙江上虞知縣先攝麗水、平湖、瑞安、建德、平陽,所至求民隱,鋤豪強,平反冤獄。在麗水斷積案七百餘事,捐貲河工,敘知府,擢浙江督糧道。道光三年,杭、嘉、湖三府大水,宗傳建議,浙西諸水尾閭,下由江蘇入海,必宜江、浙兩省通籌疏濬,大吏用其言,疏請合治。坐事左遷,巡撫程含章薦之,以知府用,授湖南永州,葺濂溪書院,崇節義,勸種植。擢四川成綿龍茂道,累攝鹽道、布政使。

十三年,瓘邊屬惈夷降復叛,勢甚張,總督鄂山既奏劾提督楊芳,檄宗傳往察治。宗傳上言:「四夷環山為巢,嗜利頑鈍,愈撫愈囂。去年添兵設防,夷轉四出焚掠,攻壘窺城,略無忌憚。雖擾一,實四安危所系,不可姑息貽患。」乃建三路進剿之策,倡助軍需,治兵選士,聲威大振。三路大軍猶未至,宗傳先以計誘降十三支夷,縶之,勒還所掠人口,有業者復之,無業者給貲,縱俘歸,使諭威德。夷猶豫未決,大軍由冷跡關逼老林巢藪,大破之於石門坎,擒斬數百,毀賊寨二百餘所,夷落悉平。論功最,擢山東按察使。捕大盜劉二鞍子置之法,群盜遠遁,遷湖北布政使。年逾七十,引疾歸。

宗傳征叛夷出奇有功,然居恆時以計取傷仁,意不自慊。嘗從同縣姚鼐游,能文章。

王鳳生,字竹嶼,安徽婺源人。父友亮,乾隆四十六年進士。由中書充軍機章京,累遷刑部郎中,精究法律,治獄矜慎。改御史,巡城、巡漕,官至通政司副使,有清直聲。以詩名。

鳳生,嘉慶中,入貲為浙江通判,屢攝知縣事。任蘭溪僅數月,清積案七百餘事。任平湖,有民數百戶,誦經茹素,傳授邪教,鳳生憫其愚惑,開諭利害,治為首數人罪,餘釋之。補嘉興府通判。道光初,浙江清查倉庫,以鳳生總其事。署嘉興知府,遷玉環同知。會浙西大水,江、浙兩省議合治,調鳳生乍浦同知,勘水道,乃由天目山歷湖州、嘉興,沿太湖以達松江。計畫甫就,事未行,值淮南高堰潰決,江南大吏疏調鳳生赴南河。未幾,擢河南歸德知府,濬虞城、夏邑、永城三縣溝渠。尋擢彰衛懷道,道屬河工五,歲修糜費,春秋防汛,虛應故事,鳳生力矯積習,事必躬親。以歲修有定例,另案無定例,在任三年,力刪另案以杜弊。尋以疾乞歸。

九年,兩江總督蔣攸銛薦起原官,署兩淮鹽運使。鳳生以淮鹽極敝,條上十八事。攸銛採其議,改灶鹽,節浮費,濬河道,增屯船,緝場私、鄰私之出入,禁江船、漕船之夾帶,及清查庫款,督運淮北諸條,疏陳待施行,會詔捕鹽梟巨魁黃玉林,鳳生計招出首,責緝私贖罪。攸銛已入告,旋因告訐置之獄,又得玉林所寄其黨私書,意反復,密疏請處以重法。上以前後歧異,譴攸銛,鳳生亦降調。陶澍繼督兩江,與尚書王鼎、侍郎寶興會籌鹽法,合疏留鳳生襄議,於是大有興革,略與鳳生初議相出入;又奏以鳳生察湖廣銷引,勘議淮北改票事,鳳生雖去官,仍與鹽事終始。十二年,湖北大潦,總督盧坤疏留鳳生治江、漢堤工,袤亙數百里,半載告竣,秋水至,新堤有潰者,鳳生引咎乞疾歸。尋淮北票鹽大暢,陶澍以鳳生首議功上聞,促之出,未行而卒。

鳳生以仕為學,尤篤好圖志,成浙西水利圖說備考、河北採風錄、江淮河運圖、漢江紀程、江漢宣防備考、淮南北場河運鹽走私道路圖,每吏一方,必能指畫其形勢,與所宜興革。四方大吏爭相疏調,少竟其用,惟治淮鹽尤為陶澍所倚藉焉。

黃冕,字服周,湖南長沙人。年二十,官兩淮鹽大使,治淮、揚賑有聲。初行海運,巡撫陶澍使赴上海集沙船與議,盡得要領,授江都知縣。歷元和、上海,署太倉州,擢蘇州府同知,晉秩知府,署常州、鎮江,有大興作,大吏悉倚以辦。疏治劉河海口,上海蒲匯塘,常州芙蓉江、孟河,冕皆躬任之。海疆兵事起,從總督裕謙赴浙江。裕謙死難,冕牽連遣戍伊犁,既而林則徐亦至戍,議興屯田,冕佐治水利有功,赦還。江蘇巡撫陸建瀛復調冕治海運,革漕費,歲省銀數十萬,為忌者所中,劾罷歸。咸豐初,粵匪圍長沙,冕建守御策。及曾國籓治兵討賊,冕創釐稅,興茶鹽之利,軍餉取給焉。又開東征局,專餉曾國籓一軍。起授江西吉安知府,復以事劾免歸,仍以餉事自任,湘軍賴以成功。尋授雲南迤西道,辭病不赴,卒於家。

冕仕宦初為陶澍、林則徐所知,晚在籍為駱秉章所倚任。時稱其幹濟,被謗亦甚云。

俞德淵,字陶泉,甘肅平羅人。嘉慶二十二年進士,選庶吉士,散館授江蘇荊溪知縣。始至,遮訴者百十輩,逾年,前訴者又易名來控,一見即識之,群驚為神。調長洲,甚得民心。遷蘇州督糧同知。道光六年,初行海運,以德淵董其役,章程皆出手定,以憂去。八年,服闋,擢常州知府,調江寧。

十年,宣宗以兩淮鹽法大壞,授陶澍為兩江總督,命尚書王鼎、侍郎寶興赴江南會議改革。時議者多主罷官商鹽,歸場灶科稅,以德淵有心計,使與議。德淵具議數千言,略謂:「鹽歸場灶,其法有三:一曰歸灶丁按金敝起科,然其中有難行者三:一在灶丁之逋欠,一在金敝鑊之私煎,一在災祲之藉口;二曰歸場官給單收稅,難行者亦有三:一在額數之難定,一在稽查之難周,一在官吏之難恃;三曰歸場商認金敝納課,難行者亦有三:一在疲商之鉆充,一在殷戶之規避,一在垣外之私售。以上三法,共有九難。如就三者兼權之,則招商認金敝,猶為此善於彼。茍得其人,或可講求盡善。顧事關圖始,果欲行之,則宜先定章程。清灶僉商、改官易制諸事,非三年不能就緒。此三年中,額課未可長懸也,場鹽未可停售也,各岸食鹽未可久缺也。新舊接替之時,非熟思審處,何能變通盡利乎?向來捆鹽之夫,淮北永豐有萬餘人,淮南老虎頸不下數萬人,皆無賴游民以此為事業。一旦失所,此數萬眾將安往?其患又不止私梟拒捕已也。」議上,陶澍深然之,乃與朝使定議,不歸場灶,仍用官商如故;惟奏罷鹽政,裁浮費,減窩價,凡積弊皆除之。薦德淵超擢兩淮鹽運使。

德淵精會計,又知人善任。諸滯岸商憚往運,改以官督辦,千里行鹽,稽覈價用,瑣屑悉當。每運恆有餘利,盡以充庫,無私取。兩淮本脂膏地,運使多以財結權貴及四方游客,餘贍給寒畯,取聲譽,皆出商貲。德淵謹守筦鑰,失望者眾,言者時相攻訐,不顧也。在任五年,力崇節儉,妻子常衣布素,揚州華侈之俗為之一變。尚書黃鉞子中民為場大使,欲得美職,德淵曰:「美職以待有功,中民無功不可得。」堅不與。陶澍益賢之,薦其才可大用,以循良久在鹽官可惜,上亦嘉之,未及擢用而卒。

姚瑩,字石甫,安徽桐城人。嘉慶十三年進士,授福建平和知縣。調龍溪,俗健悍,械斗仇殺無虛日。瑩擒巨惡立斃之,收豪猾為用,予以自新。親巡問疾苦,使侵奪者各還舊業,誓解仇讎。擇強力者為家長,約束族眾,籍壯丁為鄉勇,逐捕盜賊,有犯,責家長縛送。械斗平,盜賊亦戢,治行為閩中第一。調臺灣,署海防同知、噶瑪蘭同知,坐事落職。尋以噶瑪蘭獲盜功,復官。父憂歸,服闋,改發江蘇,歷金壇、元和、武進。遷高郵知州,擢兩淮監掣同知,護鹽運使。先後疆吏趙慎畛、陶澍、林則徐皆薦其可大用。

道光十年,特擢臺灣道。及海疆戒嚴,瑩與總兵達洪阿預為戰守計。達洪阿性剛,與同官鮮合,瑩推誠相接,一日謁謝曰:「武人不學,為子所容久矣,自今聽子而行。」二十一年秋,英兵兩犯雞籠海口,明年正月,又犯大安港。瑩設方略,與達洪阿督兵連卻之,大有斬獲,收前所失寧波、廈門砲械甚多。敵構奸民煽亂,海寇亦竊發,皆即捕戮,一方屹然,詔嘉獎,加二品銜,予雲騎尉世職。

洎江寧議款求息事,遂有臺灣鎮道冒功之獄。故事,臺灣以懸隔海外,加兵備道按察使銜,得與鎮臣專奏事。雞籠、大安之捷,飛章入告,總督怡良心不平。英兵留駐鼓浪嶼,前獲俘欲解內地,勢不能達,奏請便宜誅之,以絕內患,已報可,怡良仍令解省。瑩與達洪阿謀曰:「大府意欲市德,藉以退鼓浪嶼之兵。兵不可退,徒示弱,不如殺之!」怡良愈怒,諸帥並忌之。款議既成,交還敵俘,以妄殺被劾,逮問。瑩與達洪阿約,義不與俘虜質,即自引咎。宣宗心知臺灣功,入獄六日,特旨以同知直隸州知州發往四川效用,至則復為總督寶興所忌。會西藏兩呼圖克圖相爭,檄往平之。瑩謂:「夷人難以德化。失職下僚,孑身往,徒損國威。」不聽。及至乍雅,果不得要領而返。總督劾其畏難規避,責再往。事竣,補蓬州。在州二年,引疾歸。

文宗即位,黜大學士穆彰阿,詔宣示中外,並及瑩與達洪阿被陷狀,於是復起用,授湖北武昌鹽法道,未行,擢廣西按察使,命參大學士賽尚阿軍事。時廣西寇漸熾,諸將不合,師久無功。瑩至,任為翼長。大軍圍賊紫金山,瑩言流賊如水,必環攻以斷其逸,不聽,賊遂竄永安。又上書請斬僨事將,復不聽。永安城小,都統烏蘭泰軍西南,提督向榮軍東北,合滇、黔、楚、蜀兵四萬餘人,賊數千壁險死鬥。水竇者,永安東北之隘也,緣山徑可達桂林。瑩與烏蘭泰皆主擊水竇,絕賊外援,向榮不從,自由龍寮嶺進而敗,乃議開水竇一路縱賊逸,尾追擊之。瑩力辯其失,賽尚阿仍用向榮策,賊果突圍出犯桂林,烏蘭泰戰死,賽尚阿逮問。賊勢益熾,連陷興安、全州,犯湖南,遂不可制。瑩隨軍至湖南,巡撫張亮基奏署按察使,憂憤致疾,卒於官。

瑩師事從祖鼐,不好經生章句,務通大意,見諸施行。文章善持論,指陳時事利害,慷慨深切。所著東溟文集、奏稿、後湘詩集、東槎紀略、康輶紀行及雜著諸書,為中復堂全集,行於世。

子濬昌,能繼家學。曾國籓以名家子留佐幕,官江西安福、湖北竹山知縣。工詩,有五瑞堂集。

論曰:林培厚救荒治河有實績,而以察吏招忌。李宗傳便宜平夷,功在邊方。王鳳生、俞德淵佐陶澍治淮鹽,尤濟時之才。姚瑩保巖疆,挫強敵,反遭讒譴,然朝廷未嘗不諒其忠勤,海內引領望其再用,亦不可謂不遇矣。


\end{pinyinscope}