\article{志一百十七}

\begin{pinyinscope}
○刑法一

中國自書契以來,以禮教治天下。勞之來之而政出焉,匡之直之而刑生焉。政也,刑也,凡皆以維持禮教於勿替。故尚書曰:「明於五刑,以弼五教。」又曰:「士制百姓於刑之中,以教祗德。」古先哲王,其制刑之精義如此。周衰禮廢,典籍散失。魏李悝著法經六篇,流衍至於漢初,蕭何加為九章,歷代頗有增損分合。至唐永徽律出,始集其成。雖沿宋迄元、明而面目一變,然科條所布,於扶翼世教之意,未嘗不兢兢焉。君子上下數千年間,觀其教化之昏明,與夫刑罰之中不中,而盛衰治亂之故,綦可睹矣。

有清起自遼左,不三四十年混一區宇。聖祖沖年踐阼,與天下休養,六十餘稔,寬恤之詔,歲不絕書。高宗運際昌明,一代法制,多所裁定。仁宗以降,事多因循,未遑改作。綜其終始,列朝刑政,雖不盡清明,然如明代之廠衛、廷杖,專意戮辱士大夫,無有也。治獄者雖不盡仁恕,然如漢、唐之張湯、趙禹、周興、來俊臣輩,深文慘刻,無有也。德宗末葉,庚子拳匪之變,創巨痛深,朝野上下,爭言變法,於是新律萌芽。迨宣統遜位,而中國數千年相傳之刑典俱廢。是故論有清一代之刑法,亦古今絕續之交也。爰備志之,俾後有考焉。

清太祖嗣服之初,始定國政,禁悖亂,戢盜賊,法制以立。太宗繼武,於天聰七年,遣國舅阿什達爾漢等往外籓蒙古諸國宣布欽定法令,時所謂「盛京定例」是也。嗣復陸續著有治罪條文,然皆因時立制,不盡垂諸久遠。

世祖順治元年,攝政睿親王入關定亂,六月,即令問刑衙門準依明律治罪。八月,刑科給事中孫襄陳刑法四事,一曰定刑書:「刑之有律,猶物之有規矩準繩也。今法司所遵及故明律令,科條繁簡,情法輕重,當稽往憲,合時宜,斟酌損益,刊定成書,布告中外,俾知畫一遵守,庶奸慝不形,風俗移易。」疏上,攝政王諭令法司會同廷臣詳繹明律,參酌時宜,集議允當,以便裁定成書,頒行天下。十月,世祖入京,即皇帝位。刑部左侍郎黨崇雅奏,在外官吏,乘茲新制未定,不無憑臆舞文之弊。並乞暫用明律,候國制畫一,永垂令甲。得旨:「在外仍照明律行,如有恣意輕重等弊,指參重處。」二年,命修律官參稽滿、漢條例,分輕重等差,從刑科都給事中李士焜請也。

三年五月,大清律成,世祖御製序文曰:「朕惟太祖、太宗創業東方,民淳法簡,大闢之外,惟有鞭笞。朕仰荷天休,撫臨中夏,人民既眾,情偽多端。每遇奏讞,輕重出入,頗煩擬議。律例未定,有司無所稟承。爰敕法司官廣集廷議,詳譯明律,參以國制,增損劑量,期於平允。書成奏進,朕再三覆閱,仍命內院諸臣校訂妥確,乃允刊布,名曰大清律集解附例。爾內外有司官吏,敬此成憲,勿得任意低昂,務使百官萬民,畏名義而重犯法,冀幾刑措之風,以昭我祖宗好生之德。子孫臣民,其世世守之。」十三年,復頒滿文大清律。

康熙九年,聖祖命大學士管理刑部尚書事對喀納等將律文復行校正。十八年,特諭刑部定律之外,所有條例,應去應存,著九卿、詹事、科道會同詳加酌定,確議具奏。嗣經九卿等遵旨會同更改條例,別自為書,名為現行則例。二十八年,臺臣盛符升以律例須歸一貫,乞重加考定,以垂法守。特交九卿議,準將現行則例附入大清律條。隨命大學士圖納、張玉書等為總裁。諸臣以律文昉自唐律,辭簡義賅,易致舛訛,於每篇正文後增用總注,疏解律義。次第酌定名例四十六條,三十四年,先行繕呈。三十六年,發回刑部,命將奏聞後更改之處補入。至四十六年六月,輯進四十二本,留覽未發。

雍正元年,巡視東城御史湯之旭奏:「律例最關緊要,今六部見行則例,或有從重改輕,從輕擬重,有先行而今停,事同而法異者,未經畫一。乞簡諳練律例大臣,專掌律例館總裁,將康熙六十一年以前之例並大清會典,逐條互訂,庶免參差。」世宗允之,命大學士硃軾等為總裁,諭令於應增應減之處,再行詳加分晰,作速修完。三年書成,五年頒布。蓋明律以名例居首,其次則分隸於六部,合計三十門,都凡四百六十條。順治初,釐定律書,將公式門之信牌移入職制,漏洩軍情移入軍政,於公式門刪漏用鈔印,於倉庫門刪鈔法,於詐偽門刪偽造寶鈔。後又於名例增入邊遠充軍一條。雍正三年之律,其刪除者:名例律之吏卒犯死罪、殺害軍人、在京犯罪軍民共三條,職制門選用軍職、官吏給由二條,婚姻門之蒙古、色目人婚姻一條,宮衛門之懸帶關防牌面一條。其並入者:名例之邊遠充軍並於充軍地方,公式門之毀棄制書印信並二條為一,課程門之鹽法並十二條為一,宮衛門之沖突儀仗並三條為一,郵驛門之遞送公文並三條為一。其改易者:名例之軍官軍人免發遣更為犯罪免發遣,軍官有犯更為軍籍有犯;儀制門之收藏禁書及私習天文生節為收藏禁書。其增入者:名例之天文生有犯充軍地方二條。總計名例律四十六條。吏律:曰職制十四條,曰公式十四條。戶律:曰戶役十五條,曰田宅十一條,曰婚姻十七條,曰倉庫二十三條,曰課程八條,曰市廛五條。禮律:曰祭祀六條,曰儀制二十條。兵律:曰宮衛十六條,曰軍政二十一條,曰關津七條,曰廄牧十一條,曰郵驛十六條。刑律:曰賊盜二十八條,曰人命二十條,曰鬥毆二十二條,曰罵詈八條,曰訴訟十二條,曰受贓十一條,曰詐偽十一條,曰犯奸十條,曰雜犯十一條,曰捕亡八條,曰斷獄二十九條。工律:曰營造九條,曰河防四條。蓋仍明律三十門,而總為四百三十六條。律首六贓圖、五刑圖、獄具圖、喪服圖,大都沿明之舊。納贖諸例圖、徒限內老疾收贖圖、誣輕為重收贖圖,銀數皆從現制。其律文及律注,頗有增損改易。律後總注,則康熙年間所創造。律末並附比引律三十條。此其大較也。自時厥後,雖屢經纂修,然僅續增附律之條例,而律文未之或改。惟乾隆五年,館修奏準芟除總注,並補入過失殺傷收贖一圖而已。

例文自康熙初年僅存三百二十一條,末年增一百一十五條。雍正三年,分別訂定,曰原例,累朝舊例凡三百二十一條;曰增例,康熙間現行例凡二百九十條;曰欽定例,上諭及臣工條奏凡二百有四條,總計八百十有五條。其立法之善者,如犯罪存留養親,推及孀婦獨子;若毆兄致死,並得準其承祀,恤孤嫠且教孝也。犯死罪非常赦所不原,察有祖父子孫陣亡,準其優免一次,勸忠也。枉法贓有祿人八十兩,無祿人及不枉法贓有祿人一百二十兩,俱實絞,嚴貪墨之誅也。衙蠹索詐,驗贓加等治罪,懲胥役所以保良懦也。強盜分別法無可貸、情有可原,殲渠魁、赦脅從之義也。復仇以國法得伸與否為斷,杜兇殘之路也。凡此諸端,或隱合古義,或矯正前失,皆良法也。而要皆定制於康、雍時。

又國初以來,凡纂修律例,類必欽命二三大臣為總裁,特開專館。維時各部院則例陸續成書,茍與刑律相涉,館員俱一一釐正,故鮮乖牾。自乾隆元年,刑部奏準三年修例一次。十一年,內閣等衙門議改五年一修。由是刑部專司其事,不復簡派總裁,律例館亦遂附屬於刑曹,與他部往往不相關會。高宗臨御六十年,性矜明察,每閱讞牘,必求其情罪曲當,以萬變不齊之情,欲御以萬變不齊之例。故乾隆一朝纂修八九次,刪原例、增例諸名目,而改變舊例及因案增設者為獨多。

嘉慶以降,按期開館,沿道光、咸豐以迄同治,而條例乃增至一千八百九十有二。蓋清代定例,一如宋時之編敕,有例不用律,律既多成虛文,而例遂愈滋繁碎。其間前後牴觸,或律外加重,或因例破律,或一事設一例,或一省一地方專一例,甚且因此例而生彼例,不惟與他部則例參差,即一例分載各門者,亦不無歧異。展轉糾紛,易滋高下。雍正十三年,世宗遺詔有曰:「國家刑罰禁令之設,所以詰奸除暴,懲貪黜邪,以端風俗,以肅官方者也。然寬嚴之用,又必因乎其時。從前朕見人情淺薄,官吏營私,相習成風,罔知省改,不得不懲治整理,以戒將來。今人心共知警惕矣,凡各衙門條例,有前嚴而改寬者,此乃從前部臣定議未協,朕與廷臣悉心酌核而後更定,自可垂諸永久。若前寬而改嚴者,此乃整飭人心風俗之計,原欲暫行於一時,俟諸弊革除,仍可酌復舊章,此朕本意也。向後遇事斟酌,如有應從舊例者,仍照舊例行。」惜後世議法諸臣,未盡明世輕世重之故,每屆修例,第將歷奉諭旨及議準臣工條奏節次編入,從未統合全書,逐條釐正。穆宗號稱中興,母後柄政,削平發、捻、回疆之亂,百端待理,尚於同治九年纂修一次。德宗幼沖繼統,未遑興作。兼之時勢多故,章程叢積,刑部既憚其繁猥,不敢議修,群臣亦未有言及者,因循久之。

逮光緒二十六年,聯軍入京,兩宮西狩。憂時之士,咸謂非取法歐、美,不足以圖強。於是條陳時事者,頗稍稍議及刑律。二十八年,直隸總督袁世凱、兩江總督劉坤一、湖廣總督張之洞,會保刑部左侍郎沈家本、出使美國大臣伍廷芳修訂法律,兼取中西。旨如所請,並諭將一切現行律例,按照通商交涉情形,參酌各國法律,妥為擬議,務期中外通行,有裨治理。自此而議律者,乃群措意於領事裁判權。

是年刑部亦奏請開館修例。三十一年,先將例內今昔情形不同,及例文無關引用,或兩例重衣復,或舊例停止者,奏準刪除三百四十四條。三十三年,更命侍郎俞廉三與沈家本俱充修訂法律大臣。沈家本等乃徵集館員,分科纂輯,並延聘東西各國之博士律師,藉備顧問。其前數年編纂未竣之舊律,亦特設編案處,歸並分修。十二月,遵旨議定滿、漢通行刑律,又刪並舊例四十九條。宣統元年,全書纂成繕進,諭交憲政編查館核議。二年,覆奏訂定,名為現行刑律。

時官制改變,立憲詔下,東西洋學說朋興。律雖仍舊分三十門,而芟削六部之目。其因時事推移及新章遞嬗而刪者,如名例之犯罪免發遣、軍籍有犯、流囚家屬、流犯在道會赦、天文生有犯、工樂戶及婦人犯罪、充軍地方,職制之大臣專擅選官、文官不許封公侯、官員赴任過限、無故不朝參公座、奸黨,公式之照刷文卷、磨勘卷宗、封掌印信,戶役之丁夫差遣不平、隱蔽差役、逃避差役,田宅之任所置買田宅,婚姻之同姓為婚、良賤為婚姻,課程之監臨勢要中鹽、阻壞鹽法、私礬、舶商匿貨,禮制之朝見留難,宮衛之內府工作人匠替役,軍政之邊境申索軍需、公侯私役官軍、夜禁,關津之私越冒度關津、詐冒給路引、遞送逃軍妻女出城、私出外境及違禁下海、私役弓兵,廄牧之公使人等索借馬匹,郵驛之占宿驛舍上房,賊盜之起除刺字,鬥毆之良賤相毆,訴訟之軍民約會、詞訟誣告、充軍及遷徙,受贓之私受公侯財物,犯奸之良賤相奸,雜犯之搬做雜劇,捕亡之徒流人逃,斷獄之徒囚不應役,營造之有司官吏不住公廨是也。其緣政體及刑制遷變而改者,如名例之化外人有犯改為蒙古及入國籍人有犯,徒流遷徙地方改為五徒三流二遣地方,婚姻之娶樂人為妻妾改娶娼妓為妻,人命之殺子孫及奴婢圖賴人節去「及奴婢」字,鬥毆之奴婢毆家長改為雇工人毆家長,罵詈之奴婢罵家長改為雇工人罵家長,犯奸之奴婢奸家長妻改為雇工人奸家長妻是也。綜計全律仍存三百八十有九條,而比引律則刪存及半,依類散入各門,不列比附之目。舊例除刪並外,合續纂之新例,統一千六十六條。其督捕則例一書,順治朝命臣工纂進,原為旗下逃奴而設。康熙十五年重加酌定,乾隆以後續有增入,計條文一百一十,亦經分別去留,附入刑律,而全書悉廢。律首仍載服制全圖,以重禮教。是年冬頒行焉。若蒙古治罪各條,載諸理籓院則例,及西寧番子治罪條例,別行諸岷、洮等處者,以其習俗既殊,刑制亦異,未敢輕議更張。

新律則光緒三十二年法律館撰上刑民訴訟律,酌取英、美陪審制度。各督撫多議其窒礙,遂寢。三十三年,復先後奏上新刑律草案,總則十七章:曰法例,曰不論罪,曰未遂罪,曰累犯罪,曰俱發罪,曰共犯罪,曰刑名,曰宥恕減輕,曰自首減免,曰酌量減輕,曰加減例,曰緩刑,曰暫釋,曰恩赦,曰時效,曰時期計算,曰文例。分則三十六章:曰關於帝室之罪,曰關於內亂之罪,曰關於國交之罪,曰關於外患之罪,曰關於漏洩機務之罪,曰關於瀆職之罪,曰關於妨害公務之罪,曰關於選舉之罪,曰關於騷擾之罪,曰關於逮捕監禁者脫逃之罪,曰關於藏匿罪人及湮滅證據之罪,曰關於偽證及誣告之罪,曰關於放火決水及水利之罪,曰關於危險物之罪,曰關於往來通信之罪,曰關於秩序之罪,曰關於偽造貨幣之罪,曰關於偽造文書及印文之罪,曰關於偽造度量衡之罪,曰關於祀典及墳墓之罪,曰關於鴉片煙之罪,曰關於賭博彩票之罪,曰關於奸非及重婚之罪,曰關於飲料水之罪,曰關於衛生之罪,曰關於殺傷之罪,曰關於墮胎之罪,曰關於遺棄之罪,曰關於逮捕監禁之罪,曰關於略誘及和誘之罪,曰關於安全信用名譽及秘密之罪,曰關於竊盜及強盜之罪,曰關於詐欺取財之罪,曰關於侵占之罪,曰關於贓物之罪,曰關於毀棄損壞之罪。兩編合共三百八十七條,經憲政編查館奏交部院及疆臣覈議,簽駁者夥。

宣統元年,沈家本等匯集各說,復奏進修正草案。時江蘇提學使勞乃宣上書憲政編查館論之曰:「法律大臣會同法部奏進修改刑律,義關倫常諸條,未依舊律修入。但於附則稱中國宗教遵孔,以綱常禮教為重。如律中十惡親屬容隱,干名犯義,存留養親,及親屬相奸、相盜、相毆,發塚犯奸各條,未便蔑棄。中國人有犯以上各罪,應仍依舊律,別輯單行法,以昭懲創。竊維修訂新律,本為籌備立憲,統一法權。凡中國人及在中國居住之外國人,皆應服從同一法律。是此法律,本當以治中國人為主。今乃依舊律別輯中國人單行法,是視此新刑律專為外國人設矣。本末倒置,莫此為甚。草案案語謂修訂刑律,所以收回領事裁判權。刑律內有一二條為外國人所不遵奉,即無收回裁判權之實。故所修刑律,專以摹仿外國為事。此說實不盡然。泰西各國,凡外國人居其國中,無不服從其國法律,不得執本國無此律以相爭,亦不得恃本國有此律以相抗。今中國修訂刑律,乃謂為收回領事裁判權,必盡舍固有之禮教風俗,一一摹仿外國。則同乎此國者,彼國有違言,同乎彼國者,此國又相反,是必窮之道也。總之一國之律,必與各國之律相同,然後乃能令國內居住之外國人遵奉,萬萬無此理,亦萬萬無此事。以此為收回領事裁判權之策,是終古無收回之望也。且夫國之有刑,所以弼教。一國之民有不遵禮教者,以刑齊之。所謂禮防未然,刑禁已然,相輔而行,不可缺一者也。故各省簽駁草案,每以維持風化立論,而案語乃指為渾道德法律為一。其論無夫奸曰:『國家立法,期於令行禁止。有法而不能行,轉使民玩法而肆無忌憚。和奸之事,幾於禁之無可禁,誅之不勝誅,即刑章具在,亦祗具文。必教育普及,家庭嚴正,輿論之力盛,廉恥之心生,然後淫靡之風可少衰。』又曰:『防遏此等醜行,不在法律而在教化。即列為專條,亦無實際。』其立論在離法律與道德教化而二之,視法律為全無關於道德教化,故一意摹仿外國,而於舊律義關倫常諸條棄之如遺,焉用此法為乎?」謂宜將舊律有關禮教倫紀各節,逐一修入正文,並擬補幹名犯義、犯罪存留養親、親屬相奸相毆、無夫奸、子孫違犯教令各條。法律館爭之。明年資政院開,憲政編查館奏交院議,將總則通過。時勞乃宣充議員,與同院內閣學士陳寶琛等,於無夫奸及違犯教令二條尤力持不少怠,而分則遂未議決。餘如民律、商律、刑事訴訟律、民事訴訟律、國籍法俱編纂告竣,未經核議。惟法院編制法、違警律、禁煙條例均經宣統二年頒布,與現行刑律僅行之一年,而遜位之詔下矣。


\end{pinyinscope}