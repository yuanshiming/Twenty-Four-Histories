\article{列傳二百七十}

\begin{pinyinscope}
儒林四

孔廕植

孔廕植,字對寰,孔子六十五代孫,世居曲阜。明天啟初,襲封衍聖公。清順治元年,世祖定鼎京師,山東巡撫方大猷疏言開國之初,首宜尊崇先聖。下禮部議,衍聖公爵及其官屬,悉循明舊制。廕植朝京師,遣官迎勞。入朝,班列大學士上,賜宴,恩禮有加。四年,卒,遣山東布政使致祭。子興燮襲。

興燮,字起呂。時年十三,生母陶撫以成立。稍長,事母甚孝,凝重有器識。飭廟庭,修禮樂,諸廢悉舉。累加少保兼太子太保。康熙六年,卒。子毓圻襲。

毓圻,字鐘在。方幼,年十一,朝京師。聖祖召見瀛臺,禮度如成人,奏對稱旨。越二年,上幸學,召毓圻陪祀,太皇太后召入見,賜坐,問家世,具以對,賜茶及克食。辭出,命內臣送至宮門外,傳諭從官善輔翼之。上御殿,毓圻從諸大臣朝參,及退,命自御道行,逡巡辭,上敦諭之,乃趨出。加太子少師。二十三年,上東巡,釋奠孔子廟,留曲柄黃蓋。謁林,周覽遺跡,每事問,毓圻謹以對。因請擴林地,置守衛,除租賦,設百戶,官秩視衛守備,皆許之。毓圻輯幸魯盛典以進,復奏請重修孔子廟,白巡撫及河道總督,免縣人河工應役。雍正元年,世宗命追封先聖五代王爵。十月,毓圻詣闕謝,疾作,上命醫診視,賜參餌。十一月,卒於京師,上遣內大臣奠茶酒。喪歸,命皇三子及莊親王允祿臨奠,行人護行,賜葬,謚恭愨。毓圻工書,愛蘭,自號曰蘭堂。子傳鐸襲。

傳鐸,字振路。康熙間賜二品冠服,襲爵後一年,世宗幸學,召傳鐸陪祀。傳鐸老,病足,命其子繼溥代行禮。六月,孔子廟災,傳鐸用明弘治間故事,率族人素服三日哭,疏引咎,上遣侍郎王景曾祭告。並傳旨慰問。尋發帑重建,命侍郎留保會巡撫岳濬、前巡撫陳世倌庀工役,而以傳鐸董其事。詔詢傳鐸,有當增設者言無隱。因請增設樂器庫直房,上許之。八年,廟成。九年,上命修孔林,仍與世倌監理,疾作乞休,上允之。子繼濩前卒,命以孫廣棨襲。十年,孔林工竟,復開館輯闕里盛典。十三年,卒,賜祭葬。傳鐸工詩詞,有集。

廣棨,字京立。雍正初,授二品冠服,襲爵。以孔林工竟,率族人詣闕謝。上御圓明園正大光明殿,召入對,命坐賜茶,諭曰:「汝為先聖後,當存聖賢心,行聖賢事,秉禮守義,以驕奢為戒。汝年方少,尤宜勤學讀書,敦品勵行,與汝族人相勸戒,相砥礪,為端人正士。」廣棨頓首謝。賜松花江石硯及錦幣,賜宴,遣歸。十三年,世宗崩,入臨。高宗復召入對,以覃恩贈父繼濩如其爵。乾隆三年,上幸學,召廣棨陪祀。獻親耕耤田頌、視學大禮慶成賦。四年,朝京師,祝上萬壽。會舉經筵,令侍班,因奏請著為令。六年,疏劾曲阜知縣毓琚不職,毓琚亦訐廣棨居鄉不法,下巡撫按治,上原廣棨而譴毓琚。八年,卒。子昭煥襲。

昭煥,字顯明,十三年正月,上東巡,釋奠孔子廟,御詩禮堂。昭煥方幼,命其族人舉人繼汾等進講。是日並謁林,還,復留曲柄黃蓋。賜昭煥宴,賚書籍、文綺、貂幣,官繼汾中書,族人有官者皆進秩。親制孔子廟碑,勒石大成門外。二十一年,昭煥疏言:「皇莊戶丁蒙恩免役,歷來地方官額外雜派,每事調劑非易,請酌留五十戶,餘改歸民籍,交地方官編審應役。」上諭曰:「昭煥疏言皇莊,此必沿前代舊習,然亦止應稱官莊。子不云乎:『甚矣由之行詐,無臣而為有臣。』昭煥可謂不能讀其祖書矣。此時丁銀已停徵,地方官安得更令百姓應役?且取役何事?若為朕東巡修道,則皆發帑雇役,初未累百姓。朕展謁先師,衍聖公督令廟戶除道清產,理所應爾,豈當轉庇廟戶,並發帑雇役亦不肯應耶?」下吏議,當奪爵,上命寬之。以昭煥年少,歸咎繼汾及其兄繼涑,皆譴黜。三月,上東巡,釋奠孔子廟,謁林。二十二年,上奉皇太后東巡釋奠。三十六年,復東巡釋奠。既還京師,出內府所藏周銅器木鼎、亞尊、犧尊、伯彞、冊卣、蟠夔敦、寶簠、夔鳳豆、饕餮甗、四足鬲,凡十事,置廟庭。四十一年,兩金川平。三月,復奉皇太后東巡釋奠,告成功。次日,謁林。四十八年,昭煥卒,子憲培襲。

憲培,字養元。乾隆五十九年,卒。子慶鎔襲。

慶鎔,字陶甫。道光二十一年,卒。子繁灝襲。

繁灝,字文淵。同治二年,卒,謚端恪。子祥珂襲。

祥珂,字覲堂。光緒三年,卒,謚莊愨。子令貽襲。

令貽,字穀孫。國變後,襲爵,奉祀如故。

當唐末五季,以文宣公兼曲阜令。宋用孔氏支子,明至清初因之。自毓琚與廣棨互訐坐罷官,廷議以衍聖公咨送易涉私,孔氏子領鄉縣,所隸皆親屬,審斷亦未能悉公,擬更前例。御史衛廷璞疏言宜仍舊貫,鴻臚寺卿林令旭又請以衢州孔氏子孫為曲阜知縣,下廷臣議,用廷璞言,仍令衍聖公咨送,巡撫考試題補。後十餘年,巡撫白鍾山奏請改題缺。上諭曰:「闕里毓聖之鄉,唐、宋以來,率以聖裔領縣事。大宗主鬯,爵列上公。而知縣以民事為職,奉法令,則以裁制傷恩;厚族黨,則以偏私廢事;非古易地而官之道,當如鍾山議。仍別設世襲六品官,選孔氏子充補。」

明制,五經博士,孔氏南宗一,奉衢州孔子廟祀;北宗一人,奉述聖祀。顏氏復聖後,曾氏宗聖後,孟氏亞聖後,仲氏子路後,各一人。道州周氏元公後,江寧、嵩縣程氏皆正公後,洛陽邵氏康節後,建安、婺源硃氏皆文公後,各一人。清因之。又增設咸陽姬氏文王後,曲阜東野氏周公後,濟寧閔氏子騫後,濬縣端木氏子貢後,常熟言氏子游後,鉅野卜氏子夏後,蕭縣顓孫氏子張後,菏澤、肥城兩冉氏伯牛、仲弓後,肥城有氏有子後,鄒平伏氏伏生後,孟縣韓氏文公後,郿縣張氏明公後,各一人。而程氏改純公後一人。又崇關侯祀事,亦錄其後,洛陽、解州、江陵各一人。明史衍聖公附儒林傳後,今仿其例,並五經博士有增設者亦附焉。


\end{pinyinscope}