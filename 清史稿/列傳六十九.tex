\article{列傳六十九}

\begin{pinyinscope}
姜希轍餘縉德格勒陳紫芝笪重光任弘嘉高層雲

沈愷曾龔翔麟高遐昌

姜希轍,字二濱,浙江會稽人。明崇禎間舉人。順治初,除溫州教授。五年,以瑞安知縣缺員,令暫攝。鄭成功兵來犯,攻城,希轍督民守,遇事立應。援至,破成功兵齊雲江上。九年,遷直隸元城知縣。畿北饑,流民至者日以萬計。逃人令方嚴,民慮溷入為累,輒拒不予食。希轍令察非逃人,使墾縣中荒田,田闢,饑民以活。善決獄,民稱之。

十五年,授工科給事中。吏得盜,自列義王孫可望家人,為買馬,鑲白旗丁為之因緣。希轍疏言:「可望來歸本朝,湔滌不暇,尚敢收亡命相關通?身為旗丁,豈復應桀驁冒法網?夫盜有根柢,有黨羽,臣請收義王家人及旗丁窮治之。」上下其疏,罪人皆抵法。國初考功法,獲逃人、闢荒田、督運漕糧,皆躐等升擢。希轍疏爭非政體,不當開幸進。上方嚴罪貪吏,吏往往曲法罰鍰。希轍疏言:「例贖杖分有力無力,所輕重不過銖兩間。今乃倍五倍十,不拘成數,不應則敲樸隨之。是昔以罰省刑,今以罰濟刑也。」命仍如定例。

十七年,上詔求言,希轍疏言:「臣聞君臣一德,原未嘗以憂勞之任獨歸之君父,為人臣諉卸責地也。臣觀今日積習病根,大要有二:巧於卸肩者,假詳慎以行推諉;畏於任事者,飾持重以蹈委靡。請進一德之箴,為中外諸臣誡。」師自江西下廣東,州縣供億繁重。南贛巡撫報曲江、始興兩知縣同時自戕。希轍疏言:「大兵所集,米豆、草束、槽秬、釜鑊,自所必需。然先時傳檄,使之預備,供億雖艱,何至捐僨?行兵不嚴,責在總督;立法不預,責在巡撫:二者必居一於是。請飭察究。」尋更歷兵、禮二科。時會計法嚴,錢糧完欠,每項各限十分定考成,條例繁賾,有司救過不給。希轍疏請:「總歸十分,以一歲之徵收,計一歲之款項,起訖既清,稽核亦便。」自此部計稍紓,有司得久任。

康熙元年,考滿,內升,回籍待缺。九年,詣京師,復授戶科都給事中。具三疏:請增科員;請令巡撫得轄兵,防地方竊發;請緩奏銷之期,使催科不迫。遷順天府丞。遭父喪歸。十七年,授奉天府丞。乞養母歸。三十七年,卒於家。

餘縉,字仲紳,浙江諸暨人。順治九年進士,授河南封丘知縣。兵後流亡未復,棄地彌望,朝議興屯,設道、董之。民田徵賦,屯田徵租,租視賦為重,民棄屯不耕。府縣吏急考成,以屯租散入田賦,民失業。總督李廕祖行部至縣,縉導觀民間困苦狀,廕祖疏聞,興屯道、悉罷。十七年,行取授山西道御史,乞養歸。起河南道御史。

康熙初,鄭成功已死,其子錦屯廈門。有議棄舟山者,縉上疏爭之,略謂:「浙江三面環海,寧波尤孤懸海隅,以舟山為外籓。不知行間諸臣何所見而倡捐棄之議?江海門戶,斂手委之逆豎。夫閩海祗一廈門,數萬之眾,環而攻之,窮年不能下。奈何以已克之舟山增其巢穴?」福建總督李率泰議遷海濱居民,縉復疏爭之。略謂:「海濱之民,與賊狎處。一二冥頑貪狡,嗜厚利,通消息,以相接濟者,固未必無之。但據所稱排頭、方田諸處,民或盜牧馬,或縛窮民潛送廈門。當此兩軍相望,巡徼嚴密,雖有奸宄,安能飛渡?是其號令不肅,已可概見。」又云:「派撥舵工、水手,公然不應。海上舵工名曰「老大」,其人必少長海舟內,外洋島嶼徑路,靡不熟歷,而後駕風使舵,操縱自如。奈何責之素不練習之民,視同里役,橫加僉派?彼即勉強應役,技既不精,心復叵測。萬一變從中起,將置數十萬奮戈持滿之士於何地?」兩疏語皆切至。

聖祖親政,順治間建言諸臣坐遷謫者,次第赦還,惟議及逃人不在赦例。居數年,詔寬逃人禁。縉疏請敕部察當日建言被謫諸臣,存者召還錄用,歿者歸葬贈恤。尋命巡視長蘆鹽政。以改葬乞歸。二十八年,卒於家。

縉廉而能,治事尤持正。妖人硃方旦言禍福,朝士多信之。縉曰:「此妄男子耳,於法當誅。」方旦卒坐死。

德格勒,滿洲鑲藍旗人。康熙九年進士,選庶吉士,授編修。累擢侍讀學士,充日講起居注官、掌院學士。李光地亟稱其賢。聖祖時,召見講論經史,嘗扈從巡行。大學士明珠柄政,務結納士大夫,將餽金為治裝。德格勒以裝具,固辭不受。會久旱,上命德格勒筮,遇夬。問其占,曰:「澤上於天,將降矣!而卦義五陽決一陰。小人居鼎鉉,故天屯其膏。決去之,即雨。」上愕然,曰:「安有是?」德格勒遂以明珠對。明珠聞,大惡之,時以蜚語上聞,謂德格勒與侍講徐元夢互相標榜。徐元夢亦不附明珠者也,故並嫉之。二十六年,光地乞假歸,入辭,面奏德格勒、徐元夢學博文優。逾月,上召尚書陳廷敬、湯斌等及德格勒、徐元夢試於乾清宮。閱卷畢,諭曰:「朕政暇好讀書,然不輕評論古人。評論古人猶易,評論時人更難。如德格勒每評論時人,朕心不謂然,故召爾等面試。妍媸優劣,今已判然。學問自有分量,毋徒肆議論為也。」二十七年,明珠罷。

未幾,掌院學士庫勒訥劾德格勒私抹起居注,並與徐元夢互相標榜,下刑部論罪。故事,起居注數易槁然後登籍,德格勒所刪易者,實未定槁也。讞上論斬,命改監候秋後處決,徐元夢亦坐譴。語詳徐元夢傳。光地還京師,上命尚書張玉書等以德格勒試卷示九卿,並詰光地。於是玉書等奏稱德格勒文實鄙陋,光地亦以妄奏引罪,命從寬免究。德格勒尋遇赦,釋歸本旗。卒。

陳紫芝,字非園,浙江鄞縣人。康熙十八年進士,選庶吉士。改陜西道御史,力持風紀,絕外僚餽遺。巡視南城,捕大猾鄧二置諸法。疏言:「朝章國典宜畫一,民間冠昏喪祭未有定制,請編纂禮書,頒行天下。」又請裁屯衛:「以屯務屬州縣,則田賦可覈,逃盜可清。」詔並允行。

時督、撫、監司皆由廷臣保舉。湖廣巡撫張汧,大學士明珠所私也,恃勢貪暴,言路莫敢摘發。二十六年,紫芝上疏劾之,言:「汧蒞任未久,黷貨多端,凡地方鹽引、錢局、船埠,靡不搜括,甚至漢口市肆招牌,亦按數派錢。當日保舉之人,必有賄囑情弊,請一並敕部論罪。」上命奪汧官,遣直隸巡撫於成龍、山西巡撫馬齊、副都御史開音布往按治。復諭廷臣,謂汧貪婪無人敢言,紫芝獨能彈劾,即予內升。成龍等按得汧以前官福建布政使虧帑令屬吏彌補,又派收鹽商銀九萬,上荊南道祖澤深婪取於民又八萬,讞上,論絞。保舉汧為巡撫者,侍郎王遵訓、學士盧琦、大理寺丞任辰旦,皆坐奪官。擢紫芝大理少卿。每讞獄,稍涉矜疑,即為駁正,多所平反。

紫芝以峭直受上知,同朝多側目。無何,卒。或傳紫芝一日詣朝房,明珠延坐進茗,飲之,歸遂暴卒云。

笪重光,字在辛,江南句容人。順治九年進士。自刑部郎中考選御史。巡按江西,與明珠忤,罷歸。初,鄭成功犯鎮江,重光縋城乞援。事平,賜御書榜。卒,祀鄉賢。

任弘嘉,字葵尊,江南宜興人。初以舉人官行人。康熙十五年,成進士。十八年,考選江南道御史。巡南城,疏言:「各州縣宜有講堂書院,庶人知鄉學。」又言:「學道不惟受制籓司,抑且受制知府。蓋府道階級不甚懸,無以資表率。部郎聲望不甚重,又無由達封章。求其公明,實不可得,乞重其選。」改巡北城,疏陳五城應行事,謂:「盜風未靖,由保甲不行。稽察未清,由旗、民雜處。司坊未潔,由勸懲不當。」又言:「州縣昏夜比較,鄉民託宿無地,饑寒受杖,往往殞命。又或因分釐火耗之輕,受僉役橫索之累。」又言:「朝廷清丈,所以為民,而籓府駁冊,上下動費累百。津梁有關,所以禦暴,今小港皆設巡攔,旱路亦行堵截,檢索至負擔,稅課遍雞豚。」所言皆痛切。弘嘉一日巡城,有錦衣駿馬突其前,訶叱之。隸卒白曰:「此王府優也。」弘嘉趨王府,索優出,杖之四十。上聞,直弘嘉。由是貴戚斂跡,毋敢玩法。

尋掌山東道,兼江南道如故。上十漸疏:「一曰,朋黨交結之漸。始因交際為餽遺,漸以愛憎成水火。二曰,奢侈僭逾之漸。物力既殫,等威亦紊。三曰,文武訐訕之漸。督、撫、提、鎮挾私互訐,小吏效尤,何以使民無訟?四曰,紳士吹求之漸。有司視如仇讎,奸民以為魚肉。五曰,上下奉違之漸。國家良法美意,奉行者徒有虛文,過當者反成弊政。六曰,名器混淆之漸。為生養萬民計,守令宜用正途。七曰,常平侵漁之漸。貯穀久易浥損,又難盤察,不若聽民輸錢,數易稽而無朽蠹。八曰,河工興建之漸。從古無不徙之河,治河惟去其太甚,不必議開議塞,借一勞永逸之辭,為逐利幸功之術。九曰,情罪過當之漸。如逃人止於鞭刺,過宿反至竄流,輕重不平,枉誣尤甚。十曰,積習膠固之漸。升遷則趕缺壓缺,處分則忽重忽輕,視為故常,營競特甚。」復疏論銓政不平,並下部議行。三十三年,遷奉天府府丞,兼學政。轉通政司參議,署通政使。丁母憂歸。服闋,病目,卒於家。

弘嘉素慎,疏上言過直,輒戰慄。或曰:「子葸若此,何如不言?」曰:「弘嘉之戰慄,氣不足也。然知其當言,不敢欺吾心,尤不敢負吾君耳。」

高層雲,字二鮑,江南華亭人。康熙十五年進士。授大理寺評事。二十五年,授吏科給事中。二十六年,太皇太后崩,詔王大臣集永康左門外議喪禮。大學士王熙等向諸王白所議,跪移時,李之芳年老,起而踣。層雲曰:「是非國體也。」即日疏言謂:「天潢貴胄,大臣禮當致敬。獨集議國政,無弗列坐,所以重君命、尊朝廷也。況永康左門乃禁門重地,太皇太后在殯,至尊居廬,天威咫尺,非大臣致敬諸王之地。大學士為輔弼大臣,固當自重,諸王亦宜加以禮節,不可驕恣倨慢,坐受其跪,失籓臣體。」疏入,上曰:「朕召大臣議事,如時久,每賜墊坐語。今大臣為諸王跪,於禮不合。」下宗人府,吏、禮二部議,嗣後大臣與諸王會議,不得引身長跪,著為令。

二十八年,京師旱,詔求言。層雲疏論江、淮間行屯田擾民,請急停蘇民困,上嘉納之。遷通政司參議。二十九年,遷太常寺少卿,卒官。

沈愷曾,字樂存,浙江歸安人。康熙二十六年進士,選庶吉士。三十年,改山東道御史。喀爾喀率屬內附,上親出塞拊循。愷曾疏言:「巡行口外,為蒙古諸臣定賞罰,編戶口,安插新附。但聖躬遠出,間關崎嶇,乘輿勞頓於外,群臣晏息於家,臣心何安?宜遣部院大臣經理,令逐一奏聞,仍與皇上親行無異。乞傳旨暫緩此行。」疏入,不報。上還京師,召愷曾入對,賜宴。三十五年,上親征噶爾丹,歲暮,以餘孽未靖,復出塞。愷曾復上疏請回鑾,語甚剴切。

順天學政侍郎李光地有母喪,命奪情視事,光地請給假九月,言路大譁。愷曾疏言:「學臣關系名教,表率士子。使衰絰者衣錦論文,其何以訓?宜令終喪,以隆孝治。閣臣職司票擬,理應委曲奏請,始不當有在任守制之票,既不當有仍遵前旨之擬。科臣職司封駁,閣臣票擬不當,科臣繳旨覆奏,固其職也。乃亦復默然,不知其所謂封駁者何在也?臣不敢以妄擬閣臣為嫌,劾奏同列為咎。」疏入,下九卿議,尋用彭鵬言,令解任在京守制。陜西提督孫思克請令富民納粟佐軍,愷曾論奏乞敕部停止,上是之。

入臺七年,疏數十上,伉直敢言。歷掌山西、江南、浙江、河南道事,管登聞院。三十八年,巡兩廣鹽課,多惠政,商民德之。報滿,留任一年。還京,復掌山西道。丁父憂,以廣東運使罣誤事連坐,罷官。四十四年,上南巡,召試行在稱旨,賜御書。尋卒。

龔翔麟,字蘅圃,浙江仁和人。父佳育,字祖錫。自龍驤衛經歷出知安定縣,又自兵部郎中出為分巡通永道僉事,擢江南布政使,所至有聲績。入授光祿寺卿。命修賦役簡明書,未竟。卒。

翔麟自副貢生授兵部主事,出榷廣東關稅。沿海諸稅口,遠者去省二千里,吏役苛索,商民重困。翔麟嚴其禁,並移行府縣察究。康熙三十三年,考選陜西道御史,遂疏請以諸稅口交府縣徵收,著為令。

尋命巡視西城。大學士熊賜履以誤擬旨罷,復起為吏部尚書。翔麟疏劾:「賜履竊講學虛聲,前因票擬錯誤,嚼毀草簽,卸過同官。皇上從寬,放歸田里。旋賜起用,晉位塚宰,毫無報稱。其弟賜瓚包攬捐納,奉旨傳問,賜履不求請處分,猶泰然踞六卿之上。乞賜罷斥。」右通政張雲翮,故靖逆侯勇子。勇妻李卒,雲翮不居喪。翔麟疏劾:「雲翮縱非李出,嫡母、繼母並制三年,豈可視為陌路?乞嚴加議處,以儆敗類。」雲貴總督趙良棟討吳三桂,定雲南,以敘功未允,為部下乞恩,屢有求請。翔麟疏劾:「良棟效力行間,悉由皇上指授方略。蕩平後敘功,既經廷議,重以睿裁,輕重無不允當。事閱十年,而良棟猶嘵嘵不已,妄肆薦揚,市恩於眾,借矜己功。且越例求賜莊田、房屋,言詞狂悖,大不敬。乞下所司定罪。」賜履雅負清望,良棟功臣,雲翮功臣子,翔麟論列無所避,以是得直聲。俄又劾賜履及侍郎趙士麟亂銓政,條列以上。

官御史十年,乞歸,貧至不能舉火,蕭然不改恆度。尋卒。

高遐昌,字振聲,河南淇縣人。康熙十五年進士,授湖南龍陽知縣。以屯賦重,請減與民田同額。父憂去。服闋,補廣東東莞知縣,歷茂名、信宜,護高州知府,皆有聲。行取,擢刑部主事,累遷戶部郎中。

四十六年,授戶科給事中。時提督九門步軍統領託合齊恃權不法,給事中王懿德列款疏劾。上方幸熱河,遐昌詣行在繼劾之。略言:「託合齊欺罔不法,經懿德糾參,臣又何敢置喙?伏念其所以橫恣,皆緣握權太過。自督捕裁,而所轄三營改歸提督,悍將驕兵,毫無忌憚。請仍歸兵部擇司官督率,考勤惰、禁勒索,營務防汛,晝夜巡邏,即有奸匪,不得妄牽無辜,私刑酷訊。提督干預詞訟,奸民構弁兵,擇人而噬,民不聊生。請仍歸大、宛二縣,五城司坊、巡城御史以及府尹、治中。逃盜命案,歸於刑部,一秉國法。提督管理街道,縱其兵丁肆為貪噬,勢壓官民。請五城分治,仍歸司坊。每年工部保題司官督理,庶法官守制,無復軼越。此皆本朝舊例,當歸所司,防微杜漸,不致成積重之勢。」疏上,上以巡捕三營並步軍統領,非自託合齊始。司坊管街道,畏懼顯要,止知勒索鋪戶,故亦歸並步軍統領。今既累商民,即以遐昌兼管,期一年責以肅清。遐昌既任事,革除陋規,街道溝渠次第平治,兵民以安。兩屆報滿,仍命接管。

託合齊陰圖報復,欲伺隙中傷。五十年,上自暢春園還,見內城街道被侵占甚窄,召託合齊詰責之。託合齊奏外城尤窄。命尚書赫碩色等察勘,託合齊故引視僻巷,民居占官街得三百餘間,謂皆遐昌任內所造,逮下刑部獄。尚書齊世武,託合齊黨也,將刑訊,主事蔣晟持不可。乃議遐昌以官街邀民譽,應發奉天安置。託合齊黨復譁,言遐昌受賂。嚴訊家屬,定爰書,謂據供雖未受賂,但風聞街道舊規,鋪戶修房,每間與胥役錢二三百,以此例之,房三百餘間,計錢七百五十千,當枉法贓律處絞。朝審,具冤狀。尚書王掞、李天馥謂遐昌廉能為上知,宜從寬典,富寧安贊之,獄乃緩。會託合齊以病乞假,隆科多攝其職,因言託合齊罔上行私,橫恣貪婪,及誣陷遐昌狀。上命釋遐昌,都人爭赴獄舁之出,擁赴闕謝。及出都,送者填溢,醵金完懸贓。遐昌歸,未幾卒。

論曰:康熙間以直言著者,魏象樞、郝浴、楊素蘊、彭鵬、趙申喬輩,易又歷中外,卓然為名臣。希轍、縉,自世祖朝巳在諫垣,有獻替。弘嘉論十漸,層雲爭國體,陳義皆甚高。若德格勒、紫芝、重光忤明珠,愷曾彈李光地,翔麟論熊賜履、趙良棟,遐昌抗託合齊,雖所糾繩賢不肖不同,謇謇匪躬,不為名懾,不為勢撓,謚為「遺直」,殆無愧歟?


\end{pinyinscope}