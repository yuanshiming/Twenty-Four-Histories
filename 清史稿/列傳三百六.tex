\article{列傳三百六}

\begin{pinyinscope}
籓部二

○敖漢柰曼巴林扎嚕特阿嚕科爾沁翁牛特

克什克騰喀爾喀左翼烏珠穆沁浩齊特

蘇尼特阿巴噶阿巴哈納爾

敖漢部,在喜峰口外,至京師千有十里。東西距百六十里,南北距二百八十里。東柰曼,西喀喇沁,南土默特,北翁牛特。

內扎薩克二十四部,自科爾沁、扎賚特、杜爾伯特、郭爾羅斯、喀喇沁、土默特左翼、阿嚕科爾沁、翁牛特、阿巴噶、阿巴哈納爾、四子部落、茂明安、烏喇特外,皆元太祖十五世孫達延車臣汗之裔。達延車臣汗子十一:長圖嚕博羅特,其嗣為敖漢、柰曼、烏珠穆沁、浩齊特、蘇尼特五部;第三子巴爾蘇博羅特,其嗣為土默特右翼一旗及鄂爾多斯部;第五子阿爾楚博羅特,其嗣為巴林、扎嚕特二部;第六子鄂齊爾博羅特,其嗣為克什克騰部;第十一子格哷森扎扎賚爾琿臺吉,其嗣為喀爾咯左翼、喀爾喀右翼二部;餘皆不著。圖嚕博羅特子二:長博第阿喇克,詳烏珠穆沁傳;次納密克,生貝瑪土謝圖。子二:長岱青杜楞,號所部曰敖漢;次額森偉徵諾顏,詳柰曼傳。

岱青杜楞子索諾木杜棱及塞臣卓哩克圖,初皆服屬於察哈爾。以林丹汗不道,天聰元年,偕柰曼部長袞楚克率屬來歸,詔索諾木杜棱居開原,塞臣卓哩克圖還舊牧。二年,偕柰曼、巴林、扎嚕特諸臺吉剿察哈爾,諭勿妄殺降,嚴汛哨。後索諾木杜棱以私獵哈達、葉赫山罪,議奪開原地。塞臣卓哩克圖卒,子旺第繼為部長。八年冬,遣大臣赴碩翁科爾定諸籓牧,以扎哈蘇臺、囊嘉臺為敖漢界。崇德元年,詔編所部佐領,設扎薩克,以旺第領之,爵多羅郡王。

順治元年,從入山海關,擊流賊李自成。康熙十三年,請選兵隨剿逆籓吳三桂,詔還牧聽調。十四年,隨大軍剿察哈爾叛人布爾尼。十五年,徵兵赴河南,尋調荊州。越三年,凱旋。二十八年秋,詔發喜峰口倉粟賑所屬貧戶。三十七年冬,遣官往教之耕,曰:「朕巡幸所經,見敖漢及柰曼諸部田土甚嘉,百穀可種。如種穀多穫,則興安嶺左右無地可耕之人,就近貿糴,不須入邊巿米矣。其向因種穀之地不可牧馬,未曾墾耕者,今酌留草茂之處為牧地,自兩不相妨。且敖漢、柰曼蒙古以捕魚為業者眾,教之以引水灌田,彼亦易從。凡有利益於蒙古者,與王、臺吉等相商而行。」雍正五年,以所部災,賜帑賑之。九年,隨大軍剿噶爾丹策凌。

所部一旗,駐固爾班圖爾噶山,與柰曼、翁牛特、巴林、扎嚕特、喀爾喀左翼、阿嚕科爾沁諸部統盟於昭烏達。爵五:扎薩克多羅郡王一;附多羅郡王一;附固山貝子二,一由貝勒降襲;鎮國公一,由貝子降襲。

是旗墾事最在先。嘉慶以後,屢申嚴禁。光緒十七年,金丹道匪楊悅春等糾眾為亂。十月,攻貝子德克沁府踞之,戕德克沁,四出紛擾,喀喇沁、土默持、翁牛特、柰曼諸部皆被兵。脅漢人為匪,遇蒙人則殺,占官署,毀教堂,蹂躪甚慘。命直隸提督葉志超等剿之,至十二月始平。詔賑恤之,凡敖漢等五部八旗,為銀十七萬兩有奇,全濟民、蒙三十萬口有奇。李鴻章會都統奎斌奏:「蒙古、客民結怨已深,一在佃種之交租,一在商賈之積欠。應更定新章,佃種蒙地者,由地方官徵收,蒙古王公派員領取;商民領取蒙古貲本貿易,或彼此賒欠致有虧折,亦應送地方官持平論斷,毋稍偏倚。」此敖漢諸部蒙古、客民結隙根本所在,故鴻章等欲更張救之。二十四年,扎薩克郡王達木林達爾達克以充昭烏達盟長擾累屬下,違例科派,奪盟長及扎薩克。三十一年,扎薩克郡王勒恩扎勒諾爾贊復被護衛刺死。三十三年,都統廷傑以置嗣未定,請理籓院慎擇親賢,速為承襲。宣統元年,以族人棍布札布襲。二年,分置左、右二旗,以原有扎薩克者為左旗,別授郡王色凌端嚕布為右旗扎薩克。左旗有佐領三十五,右旗有佐領二十。

柰曼部,在喜峰口外,至京師千有百一十里。東西距九十五里,南北距二百二十里。東喀爾喀左翼,西敖漢,南土默特,北翁牛特。

元太祖嘗偕弟哈布圖哈薩爾平柰曼部,三傳至額森偉徵諾顏,即以為所部號。子袞楚克嗣,稱巴圖魯臺吉,服屬於察哈爾。以林丹汗不道,天聰元年,偕從子鄠齊爾等率屬來歸,詔還舊牧。鄂齊爾以卒巡徼,斬察哈爾兵百,獲牲畜百餘獻,賜號和碩齊,賚甲一。八年,遣大臣赴碩翁科爾定諸籓牧,以巴克阿爾和碩、巴噶什魯蘇臺為柰曼界。崇德元年,授扎薩克,爵多羅達爾漢郡王。先是,所部阿邦和碩齊從大軍剿茂明安部逃賊有功,至是以宣諭朝鮮,袞楚克遣屬岱都齊齎書從。遇明皮島兵,狙擊之,斬賊二,被創還,悉蒙獎賚。五年,遣屬扎丹隨大軍征索倫,凱旋,得優賜。七年,復遣屬善丹、薩爾圖隨征明,由黃崖口入邊,下薊州,趨山東,攻克袞州。八年,善丹來獻俘,賜宴。

順治元年,從入山海關,擊流賊李自成。康熙十四年,察哈爾布爾尼叛,扎薩克郡王扎木三應之,徙察罕郭勒,與布爾尼賊壘聯聲援,且遣黨煽諸扎薩克。詔撫遠大將軍信郡王鄂扎率師討,至達祿,布爾尼敗遁,為科爾沁額駙沙津陣斬。扎木三蹙縛乞罪,特旨貸死。更優獎不附逆諸臺吉,鄂齊爾由一等臺吉襲扎薩克郡王爵,烏勒木濟由二等臺吉晉貝子,格哷爾由二等臺吉晉輔國公,烏爾圖納素圖由三等臺吉晉一等臺吉,鄂齊爾長子額爾德尼授三等臺吉。二十年,詔發喜峰口倉粟賑所屬貧戶。雍正五年,所部歉收,賜帑賑之。九年,隨大軍剿噶爾丹策凌。初,柰曼與敖漢逢國家典禮及征伐事,先後偕來,位秩如一。獨扎木三懷貳,遂不齒於敖漢。迨鄂齊爾重膺錫封,奉職惟謹,而荷恩亦如故焉。

所部一旗,駐彰武臺,其爵為扎薩克多羅達爾漢郡王。道光二十七年,以壽安固倫公主指配柰曼扎薩克郡王阿完都窪第扎布之子德木楚克扎布,授固倫額駙。旋襲爵職。同治四年,卒,追賜親王銜。光緒十七年,金丹道匪之變,是部亦被擾。事平,賑恤之。有佐領五十。

巴林部,在古北口外,至京師九百六十里。東西距二百五十一里,南北距二百三十三里。東阿嚕科爾沁,西克什克騰,南翁牛特,北烏珠穆沁。

元太祖十六世孫阿爾楚博羅特生和爾朔齊哈薩爾。子蘇巴海,稱達爾漢諾顏,號所部曰巴林。子巴噶巴圖爾嗣。有子三:長額布格岱洪巴圖魯,次和托果爾昂哈,次色特爾。初皆服屬於喀爾喀。

天命四年,額布格岱洪巴圖魯偕喀爾喀部長遣使乞盟,允之。十一年春,以背盟私與明和,大軍往討,陣斬臺吉囊努克。冬,討扎嚕特,詔分軍入部境以張兵勢,焚原驅哨而還。會察哈爾林丹汗掠其諸部,臺吉皆奔依科爾沁。天聰二年,色特爾率子色布騰及額布格岱洪巴圖魯子色棱、和托果爾昂哈子滿珠習禮等,自科爾沁來歸,優賚撫輯之。三年,從征明,由養息穆河入大安口,克遵化。四年,攻昌黎,與扎嚕特兵圍城北。六年,從略大同、宣府邊。八年五月,會兵扎木哈克徵察哈爾,賜宰桑布兌山津雕鞍良馬,遂由獨石口徵明朔州,克堡八。十月,遣大臣赴碩翁科爾定諸籓牧,以扈拉瑚琥、呼布裏都、克哩葉哈達、瑚濟爾阿達克為巴林界。崇德元年,選兵從征明。三年,自墻子嶺入明邊,樹雲梯攻城,臺吉阿玉什屬索爾古先登,克之。四年,圍錦州。六年,圍松山。七年,獻俘,賚將弁幣。

順治元年,從入山海關,擊流賊李自成。五年,詔編所部佐領,以滿珠習禮掌左翼,爵固山貝子;色布騰掌右翼,爵多羅郡王:各授扎薩克。康熙二十三年,上幸塞外,駐蹕烏拉岱,兩翼扎薩克率諸臺吉來朝,賜冠服、弓矢、銀幣有差。二十八年,詔發古北口倉粟賑所屬貧戶。二十九年,命額駙阿喇布坦率兩翼兵四百,赴葫蘆郭勒偵噶勒丹。是役也,色布騰子格哷爾圖、納木扎,孫納木達克、桑哩達、烏爾袞,暨族臺吉沙克塔爾等皆從。格哷爾圖尤沖鋒奮擊,師旋,得優賚。三十四年,以噶勒丹掠喀爾喀至巴顏烏蘭,詔檄敖漢、柰曼兵赴阿喇布坦軍,並命納木達克、烏爾袞等防烏珠穆沁汛。是年所部歉收,詔發坡賴屯米賑之。三十八年,命護軍統領鄂克濟哈、學士蘇赫納往會扎薩克等,將現貯巴林米千石散賑。若人眾米寡,再運坡賴米賑給。雍正九年,隨大軍剿噶勒丹策凌。二等臺吉璘瞻追賊察巴罕河,護駝馬;又擊之於塔爾勒圖、固爾班什勒諸處。敘功,晉授一等臺吉。

所部二旗:右翼駐托缽山,左翼駐阿察圖拖羅海。爵四:親王品級扎薩克多羅郡王一,扎薩克固山貝子一,附固山貝子二。光緒十七年,金丹道匪之變,賊渠李國珍擾至是部那林溝地,葉志超遣軍擊平之。三十三年,以是部墾地置林西縣,隸赤峰直隸州。左翼有佐領十六,右翼有佐領二十六。

扎嚕特部,在喜峰口外,至京師千五百一十里。東西距百二十五里,南北距四百六十里。東科爾沁,西界阿嚕科爾沁,南喀爾喀左翼,北烏珠穆沁。

元太祖十八世孫烏巴什稱偉徵諾顏,號所部曰扎嚕特。子二:長巴顏達爾伊勒登,次都喇勒諾顏。巴顏達爾伊勒登子五:長忠圖,傳子內齊,相繼稱汗;次賡根;次忠嫩;次果弼爾圖,次昂安。都喇勒諾顏子二:長色本,次瑪尼。初皆服屬於喀爾喀。

太祖高皇帝甲寅年,內齊以其妹歸我貝勒莽古爾泰;忠嫩及從弟額爾濟格亦來締★L5。天命四年秋,大軍征明鐵嶺,從。色本偕從兄巴克等隨喀爾喀臺吉宰賽以兵萬餘助明,為我軍陣擒。冬,內齊、額爾濟格、額騰、鄂爾齋圖、多爾濟桑、阿爾齋弼登圖偕喀爾卓哩克圖洪巴圖魯等遣使乞盟,許之,遣大臣往蒞盟。其宰桑扣肯屬有來奔者,上以盟不可渝,拒弗納。旋釋色本、巴克歸。八年,巴克來朝,命釋其質子鄂齊爾桑與俱歸。而忠喇、昂安等屢以兵掠我使齎往科爾沁之服物及馬牛。上遣軍征之,斬昂安,俘其眾。忠嫩子桑圖以孥被擒,來朝乞哀,詔歸令完聚。未幾,所部諸臺吉復背盟,襲我使固什於漢察喇及遼河畔,掠財物。十一年,命大貝勒代善率師往討,斬鄂爾齋圖,擒巴克等凡十四臺吉。師還,仍詔釋歸。尋為察哈爾林丹汗所掠,往依科爾沁。

天聰二年,內齊、色本等先後率屬來歸。臺吉喀巴海殺察哈爾臺吉噶爾圖,以俘七百獻,賜號偉徵。三年,奉敕定隨征軍令。以越界駐牧自議罪,內齊、色本、瑪尼及果弼爾圖、巴雅爾圖、岱青,請各罰駝十、馬百,詔寬之,各罰馬一。是年冬,隨征明,入龍井關,克遵化,圍其都。明兵屯城東,蒙古諸部不俟整隊,驟進失利,惟色本及瑪尼敗敵,得優賚。五年春,詔議臺吉岱青罪。先是大貝勒代善陣擒岱青子善都,往奔科爾沁。越二年歸,詔留贍養。嗣從大軍征明,貝勒莽古爾泰與明兵戰都城東,岱青、善都遁走。又誣訐貝勒阿濟格縱屬殺人。至是,論罪應斬,上特宥之,奪所屬人戶分給莽古爾泰、阿濟格。六年,內齊、色本、瑪尼、喀巴海等從征察哈爾,諭獎其實心效力。尋隨貝勒阿濟格略明大同、宣府邊。八年,由獨石口進攻朔州。是年冬,遣大臣赴碩翁科爾定諸籓牧,以諾綽噶爾多布圖烏魯木為扎嚕特界。崇德二年,由朝鮮進徵瓦爾喀。三年,隨征喀爾喀扎薩克圖汗。五年春,從征索倫,賜臺吉桑古爾及阿玉什、琥賴、阿爾蘇瑚、嶽博果等蟒服、貂裘、甲胄、弓矢。冬,以臺吉肯哲赫追擒茂明安逃人功,賜號達爾漢。

順治元年,從入山海關,擊流賊李自成。五年,詔編所部佐領。時內齊、色本卒,以內齊子尚嘉布掌左翼,色本子桑噶爾掌右翼,各授扎薩克貝勒。康熙十四年,察哈爾布爾尼叛,且陰煽諸部。二等臺吉根翼什希布以不附逆,封鎮國公。後停襲。二十九年,隨大軍征噶爾丹,二等臺吉科克晉、四等臺吉袞楚克色爾濟額爾德尼陣歿,俱贈一等臺吉,賜號達爾漢。雍正元年,所部歉收,詔發帑賑之。十一年,選兵隨剿噶爾丹策凌,隸敖漢臺吉羅卜藏軍。

所部二旗,左翼駐齊齊靈花拖羅海山北,右翼駐圖爾山南。爵四:扎薩克多羅貝勒一,扎薩克多羅達爾漢貝勒一,附鎮國公一,輔國公一。是部產鹼,初禁開取。光緒二十一年,都統松壽以部議主開,奏定納課章程,由各旗選派公正蒙員試辦。三十三年,都統廷傑奏,以是部及阿嚕科爾沁墾地置開魯縣,隸赤峰直隸州。是部左右翼旗各有佐領十六。

阿嚕科爾沁部,在古北口外,至京師千三百四十里。東西距百三十里,南北距四百二十里。東扎嚕特,西巴林,南喀爾喀左翼,北烏珠穆沁。

元太祖弟哈布圖哈薩爾十三傳至圖美尼雅哈齊。子三:長奎蒙克塔斯哈喇,游牧嫩江,號嫩科爾沁;次巴袞諾顏;次布爾海,游牧呼倫貝爾。巴袞諾顏子三:長昆都倫岱青,號所部曰阿嚕科爾沁,以別於嫩科爾沁。子達賚,稱楚琥爾,嗣為部長;次哈貝,子巴圖爾,裔不著;次諾顏泰,子四,號四子部落。布爾海裔號烏喇特,詳各部傳。

阿嚕科爾沁與四子部落、烏喇特、茂明安、翁牛特、阿巴噶、阿巴哈納爾及喀爾喀內外扎薩克統號阿嚕蒙古,初皆服屬於察哈爾。以林丹汗不道,天聰四年,達賚暨子穆彰率屬來歸,命諸貝勒郊迎五里,賜宴。八年,遣大臣赴碩翁科爾定諸籓牧,以兩白旗外塔拉布拉克遜島為其部界。崇德元年,宣諭朝鮮,其部德赫拜達爾齎書從。遇明皮島兵,狙擊敗之。還,得優賚。先是阿嚕科爾沁設兩旗,達賚、穆彰各領一。至是始並兩旗為一,以穆彰領之。嗣從征朝鮮、瓦爾喀、索倫、喀爾喀,及明濟南、錦州、松山、薊州。

順治元年,從入山海關,擊流賊李自成。敘功授扎薩克,爵固山貝子。康熙二十七年,噶爾丹侵喀爾喀,諭所部兵防蘇尼特汛。二十八年,部眾乏食,賜粟賑之。二十九年,二等臺吉棟紐特從征噶爾丹,見賊勢熾,慷慨謂眾曰:「我等受恩深,若稍退,何面目見聖顏乎?」率兵三百趨前戰,皆歿。三十年,贈一等臺吉,世襲達爾漢號。是冬,理籓院議給所部貧戶米穀。諭曰:「賞給米穀,應調蒙古駝馬運送。時值隆冬,輸輓殊艱,恐領米之人不能運到,必致沿邊私糶,不如量米給銀,到彼甚易,貧人得霑實惠。」三十五年,上親征噶爾丹,偵賊沿克嚕倫河至額哲特圖哈布齊爾地,諭嚴防汛界。

四十三年,遣大臣往訊盜案,宣諭扎薩克戢所部,務令無盜。四十八年,固山額駙巴特瑪妻縣君以屬人不遵令,請獻戶口,諭暫遣官理,後不為例。雍正五年,賜所部貧戶銀。九年,從大軍剿噶爾丹策凌。十三年,遣官齎銀賑饑。

所部一旗,駐牧琿圖山東,隸昭烏達盟。其爵為扎薩克多羅貝勒,由固山貝子晉襲。是部亦產鹼。光緒三十一年,定蒙員自辦納課章程。是部一旗,有佐領五十。

翁牛特部,在古北口外,至京師七百六十里。東西距三百里,南北距百六十里。東阿嚕科爾沁,西承德府,南喀喇沁及敖漢,北巴林及克什克騰。

元太祖弟諤楚因,稱烏真諾顏。其裔蒙克察罕諾顏。有子二:長巴顏岱洪果爾諾顏,號所部曰翁牛特,次巴泰車臣諾顏,別號喀喇齊哩克部,皆稱阿嚕蒙古。巴顏岱洪果爾諾顏再傳至圖蘭,號杜棱汗。子七:長遜杜棱,次阿巴噶圖琿臺吉,次棟岱青,次班第偉徵,次達拉海諾木齊,次薩揚墨爾根,次本巴楚琥爾巴泰車臣諾顏。三傳至努綏,子二:長噶爾瑪,次諾密泰岱青。皆初服屬於察哈爾。以林丹汗不道,天聰六年,遜杜棱、棟岱青暨喀喇齊哩克臺吉噶爾瑪率屬來歸。是年,上親征察哈爾,各選兵從。林丹汗遁;復從貝勒阿濟格赴大同、宣府,收察哈爾部眾之竄入明邊者。師旋,優賚遣歸。自是其部稱翁牛特,以喀喇齊哩克附之,不復冠阿嚕舊稱。

七年春,棟岱青、噶爾瑪來朝,班第偉徵等相繼獻駝馬。冬,遜杜棱復率眾來朝。八年,遣大臣赴碩翁科爾定諸籓牧,以扈拉瑚、琥呼布哩都為翁牛特部界。是冬,班第偉徵、達拉海諾木齊以越界游牧罪,議罰駝百、馬千。詔從寬,罰十之一。復以罰奈曼部駝馬命分給遜杜棱、棟岱青。崇德元年,詔編新部佐領,以遜杜棱掌右翼,爵多羅杜棱郡王;棟岱青掌左翼,子多羅達爾漢岱青,各授扎薩克。三年,喀爾喀扎薩克圖汗擁眾偪歸化城,上親征之,棟岱青、班第偉徵、達拉海諾木齊等以兵會偵,扎薩克圖汗遁,乃還。四年,棟岱青率宰桑烏巴什、和尼齊等從大軍征明。六年,圍錦州、松山,設伏高橋大路及桑阿爾齋堡,遇杏山逃卒,追擊之,斬獲甚眾。七年,敘功,賜棟岱青、噶爾瑪、和尼齊等布幣有差。復追議松山掘壕時,宰桑烏巴什以誦經故不親督兵,及暮又失守望罪,論死,詔宥之。達拉海諾木齊及綽克圖巴木布等復從貝勒阿巴泰徵明。八年,來獻俘,賜宴。

順治元年,從入山海關,擊流賊李自成,復追敘部將噶勒嘛從征明功,賜號達爾漢。康熙十五年,以剿逆籓吳三桂,詔選兵赴河南駐防。十六年,調荊州。十八年,撤還。二十二年,以其部多盜,諭撫眾及弭盜法。二十六年,上閱兵盧溝橋,命其部來朝人從觀。二十七年,選兵赴蘇尼特汛防禦噶爾丹。三十四年,所部乏食,遣官往賑。三十五年,上親征噶爾丹,詔徵兵五百,運中路軍糈給器備。三十六年,朔漠平,賚運糧兵銀。五十六年,理籓院奏翁牛特及克什克騰諸扎薩克請令公勘地址有越界伐木者論罪,從之。雍正五年,賜銀賑所屬貧戶。九年,隨大軍剿噶爾丹策凌。乾隆二十年,從征達瓦齊。

所部二旗,右翼駐英什爾哈齊特呼朗,左翼駐扎喇峰西。爵四,扎薩克多羅杜棱郡王一,附固山貝子一,鎮國公一,扎薩克多羅達爾漢岱青貝勒一。光緒十七年,金丹道匪之變,賊渠李國珍等擾是部,焚王府,踞烏丹城,即元全寧路治,實熱河北路門戶。葉志超遣副將潘萬才等率軍先克之,餘遂迎刃而解。是部二旗,蹂躪均重。事平,賑恤之。左翼有佐領二十,右翼有佐領三十八。

克什克騰部,在古北口外,至京師八百有十里。東西距三百三十四里,南北距三百五十七里。東翁牛特及巴林,西浩齊特及察哈爾正藍旗牧廠,南翁牛特,北烏珠穆沁。

元太祖十六世孫鄂齊爾博羅特,再傳至沙喇勒達,稱墨爾根諾顏,號所部曰克什克騰。子達爾瑪,有子三:長索諾木,次巴本,次圖壘。服屬於察哈爾。天聰八年,索諾木率屬來歸。崇德六年,臺吉沙哩、博羅和、雲敦等奉命赴董家、喜峰諸口偵明兵,俘斬甚眾。順治九年,詔編所部佐領,以索諾木領之,授扎薩克。康熙二十六年,上閱兵盧溝橋,命其部來朝人從觀。二十七年,噶爾丹侵喀爾喀,詔選兵防蘇尼特汛。二十九年,四等臺吉穆倫噶爾弼以偵擊噶爾丹功,晉一等臺吉。三十五年,上親征噶爾丹。凱旋,以其部設站兵無誤驛務,賚銀幣。雍正五年,賜銀賑其屬貧戶。

所部一旗,駐牧吉拉巴斯峰,隸昭烏達盟。其爵為扎薩克一等臺吉。是部墾事最早。嘉慶中,設白岔巡檢治之。同治中,回匪東竄熱河,設戍其地。

又經棚當直隸多倫諾爾東北,商民萃處,號稱蕃盛。光緒十七年,金丹道匪之變,是部曾以兵協剿烏丹城等處之匪,得捷。有佐領十。

喀爾喀左翼部,在喜峰口外,至京師千二百有十里。東西距百二十五里,南北距二百三十里。東科爾沁,西柰曼,南土默特,北扎嚕特及翁牛特。

元太祖十六世孫格哷森札扎賚爾琿臺吉居杭愛山,始號喀爾喀。有子七,部族繁衍,分東、西、中三路,以三汗掌之。其長子阿什海達爾漢諾顏。生子二:長巴顏達喇,為西路扎薩克圖汗祖;次圖捫達喇岱青,子碩壘烏巴什琿臺吉。生子三:長俄木布額爾德尼,次杭圖岱,次袞布伊勒登,皆為喀爾喀西路臺吉,隸扎薩克圖汗。

康熙三年,袞布伊勒登以其汗旺舒克為同族羅卜藏臺吉額璘沁所戕,部眾潰,窮無依,乃越瀚海來歸。先是喀爾喀中路土謝圖汗下臺吉本塔爾攜眾內附,封扎薩克親王爵,駐牧張家口外。至是詔袞布伊勒登扎薩克多羅貝勒賜牧喜峰口外察罕和碩圖,以所居地分東西,故本塔爾稱喀爾喀右翼,袞布伊勒登稱喀爾喀左翼。蓋自國初以來,喀爾喀相繼歸誠,名凡三:曰舊喀爾喀,歸誠最早,後編入蒙古八旗;曰內喀爾喀,即今隸內扎薩克之喀爾喀左右翼二部;曰外喀爾喀,其歸誠較後,即今隸外扎薩克之喀爾喀土謝圖汗、車臣汗、扎薩克圖汗、賽因諾顏四部。二十九年,以額魯特臺吉噶爾丹侵喀爾喀土謝圖汗、車臣汗、扎薩克圖汗,所居皆被掠,先後乞降。詔袞布伊勒登備兵要汛,偵御噶爾丹。三十五年,上由克嚕倫河親征,諭其部選兵赴烏勒輝聽調。噶爾丹敗遁,撤兵還。雍正元年,所屬歉收,賜帑賑之。九年,大軍剿噶爾丹策凌,選兵赴歸化城駐防。尋以護外扎薩克游牧,移駐克嚕倫河。乾隆初撤之。

所部一旗,駐察罕和碩圖。其爵為扎薩克多羅貝勒。有佐領一。是部與敖漢、柰曼、巴林、翁牛特、扎嚕特、喀爾喀左翼、阿嚕科爾沁七部十一旗,統盟於卓索圖。道光末籌海防,咸豐中剿粵匪,皆徵其兵。至同治初,科爾沁親王僧格林沁陣亡,乃撤歸。清代蒙古留京王公,以是盟與哲裏木、卓索圖為多,大都額駙子孫。錫林郭勒、烏察布、伊克昭三盟則鮮見焉。

烏珠穆沁部,在古北口外,至京師千一百六十三里。東西距三百六十里,南北距四百二十五里。東索倫,西浩齊特,南巴林,北瀚海。

元太祖十六世孫圖嚕博羅特由杭愛山徙牧瀚海南,子博第阿喇克繼之。有子三,分牧而處。長庫登汗,詳浩齊特部傳。次庫克齊圖墨爾根臺吉,詳蘇尼特部傳。次翁袞都喇爾,號其部曰烏珠穆沁。子五:長綽克圖,號巴圖爾諾顏;次巴雅,號賽音冰圖諾顏;次納延泰,號伊勒登諾顏;次彰錦,號達爾漢諾顏。皆早卒。次多爾濟,號車臣濟農,與察哈爾同族,為所屬。以林丹汗不道,多爾濟偕綽克圖子色棱徙牧瀚海北,依喀爾喀。

天聰九年,大軍收服察哈爾,多爾濟偕喀爾喀部車臣汗碩壘、浩齊特部策夌伊勒登土謝圖、蘇尼特部叟塞巴圖魯濟農、阿巴噶部都思噶爾扎薩克圖巴圖爾濟農等表貢方物。崇德元年,命舊自察哈爾來歸之偉宰桑等齎敕往諭,遂偕其使納木渾津等至。自是貢物不絕。二年八月,臺吉伊什喀布、烏喇垓增格、阿津、鏗特克等來貢,賚冠服、甲胄、弓矢、布幣。十一月,多爾濟、色棱各率屬由克嚕倫來歸。三年,喀爾喀扎薩克圖汗擁眾偪歸化城,上統師親征,多爾濟、色棱以兵會偵,扎薩克圖汗遁,乃還。賜貢馬臺吉巴甘冠服、鞓帶。五年,賜來朝臺吉固穆、塔布囊阿哈圖等蟒服、採幣。六年,詔授多爾濟扎薩克和碩車臣親王。順治三年,詔授色棱扎薩克多羅額爾德尼貝勒。以多爾濟掌左翼,色棱掌右翼。是年大軍剿蘇尼特部騰機思,至喀爾喀,以多爾濟屬達喇海鄉導功,賜號達爾漢。

康熙二十年,以所部牧鄰喀爾喀,因互竊駝馬,王大臣等遵旨議邊汛形勝處各屯兵百許,按旗設哨,嗣後扎薩克能撫眾戢盜者予敘,否則論罪。二十七年,噶爾丹侵喀爾喀,遣大臣赴烏珠穆沁宣諭扎薩克等防汛。三十年,阿巴噶臺吉奔塔爾首烏珠穆沁臺吉車根等叛附噶爾丹,語涉扎薩克王素達尼妻。命大臣往勘,得車根等私給噶爾丹駝馬,又令部校阿爾塔等往通信狀,罪應死。素達尼妻預知,應削封號、奪所屬人戶。素達尼已故,應除爵。議上,詔治車根等罪,免奪人戶。素達尼未預謀,免除爵,襲如初。三十一年,素達尼弟協理臺吉烏達喇希妻以烏達喇希證車根等從逆狀,乞予敘。理籓院議烏達喇希故,應贈輔國公,令子袞布扎偵襲,從之。後停襲。三十四年,噶爾丹復侵喀爾喀,詔所部選兵駐汛。三十五年,偵噶爾丹至額哲特圖,哈卜濟爾赴烏爾輝聽調。是年,上親征噶爾丹還,賜坐塘諸弁兵銀。五十五年,選兵隨大軍防御策妄阿喇布坦。雍正九年,議剿噶爾丹策凌,詔徵烏珠穆沁西各扎薩克兵三千駐烏喇特汛防四十九旗游牧,復諭烏珠穆沁別以兵駐克嚕倫河。十年,移駐達哩剛愛。十三年,撤還。乾隆十二年,詔嘉兩翼扎薩克,值所屬災,贍貧戶二萬餘,王貝勒以下各賜俸半年,無俸臺吉俱賜幣有差。

所部二旗:右翼駐巴克蘇爾哈臺山,左翼駐魁蘇陀羅海,與浩齊特、蘇尼特、阿巴噶、阿巴哈納爾諸部統盟於錫林郭勒。爵四:扎薩克和碩車臣親王一,附鎮國公一,輔國公一,扎薩克多羅額爾德尼貝勒一。左旗扎薩克貝勒色楞傳至達克丹都克雅扎布。咸豐十年,以報效軍需駝馬,予郡王銜。是部左翼有固爾班泊,產鹽,由巴林橋烏丹城運售內地,西出圍場,分銷承德、豐、灤各屬;東出建平,分銷建昌、朝陽各屬;遠者更可銷至奉天突泉諸縣,西南可由多倫至山西豐鎮、寧遠諸。光緒三十二年,都統廷傑奏定試辦蒙鹽章程。宣統二年,度支部尚書載澤奏定山西蒙鹽辦法,謂東路以烏珠穆沁蒙鹽為主,以蘇尼特部鹽附之。左翼有佐領二十一,右翼有佐領九。

浩齊特部,在獨石口外,至京師千八百一十五里。東西距百七十里,南北距三百七十五里。東及北烏珠穆沁,西阿巴噶,南克什克騰。

元太祖十六世孫圖嚕博羅特,再傳至庫登汗,號其部曰浩齊特。庫登汗孫德格類,號額爾德尼琿臺吉。子五:長奇塔特扎幹杜棱土謝圖,次巴斯琫土謝圖,次策凌伊勒登土謝圖,次奇塔特昆杜棱額爾德尼車臣楚琥爾,次茂海墨爾根。與察哈爾同族,為所屬。以林丹汗不道,徙牧瀚海北,依喀爾喀。

天聰八年,所部臺吉額琳臣及塔布囊巴特瑪班第圖嚕齊、宰桑僧格布延徹臣烏巴什等,攜戶口駝馬自喀爾喀內附,遣使迎宴,賚甲胄、雕鞍、蟒服、銀幣。額琳臣屬有先附者五十三戶,仍命轄之。九年,大軍收服察哈爾,策凌伊勒登土謝圖偕烏珠穆沁諸部長表貢方物。崇德元年,巴斯琫土謝圖偕蘇尼特部來貢。二年,奇塔特昆杜棱額爾德尼車臣楚琥爾子博羅特率屬來歸。順治三年,詔授扎薩克多羅額爾德尼貝勒,後晉封郡王。八年,奇塔特扎幹杜棱土謝圖子噶爾瑪色旺攜眾至。十年,詔授扎薩克多羅郡王,以博羅特掌左翼,噶爾瑪色旺掌右翼。

康熙二十七年,詔發拜察儲粟賑其部貧戶,復命給銀。三十四年,噶爾丹侵喀爾喀,詔兩翼扎薩克選兵駐界偵御之。三十五年,上親征噶爾丹,牧馬郭和蘇臺,諭偕蘇尼特、阿巴哈納爾部長董牧務。凱旋,兩翼扎薩克率臺吉等歡迎道左,諭獎飼秣得宜,並優賚監牧及修道鑿井諸弁兵。五十四年,所部歉收,以唐三營儲粟賑之,並遣官往教之漁。雍正九年,大軍剿噶爾丹策凌,詔選兵分駐克嚕倫河。十年,移駐達哩剛愛。十三年,撤還。

所部二旗:左翼駐特古哩克呼都克瑚欽,右翼駐烏默赫塞哩,隸錫林郭勒盟。爵二:扎薩克多羅額爾德尼郡王一,扎薩克多羅郡王一。是部左右翼有佐領各五。

蘇尼特部,在張家口外,至京師九百六十里。東西距四百六里,南北距五百八十里。東阿巴噶,西四子部落,南察哈爾正藍旗牧廠,北瀚海。

元太祖十六世孫圖嚕博羅特,再傳至庫克齊圖墨爾根臺吉,號其部曰蘇尼特。庫克齊圖墨爾根臺吉子四:長布延琿臺吉,子綽爾袞,居蘇尼特西路;次布爾海楚琥爾,子塔巴海達爾漢和碩齊,居蘇尼特東路。初皆服屬於察哈爾。以林丹汗不道,徙牧瀚海北,依喀爾喀。

天聰九年,綽爾袞子叟塞偕喀爾喀車臣汗碩壘遣使貢方物。崇德二年,塔巴海達爾漢和碩齊子騰機思、騰機特、莽古岱、哈爾呼喇偕臺吉、偉徵等,各遣使來朝,賜朝鮮貢物。三年,臺吉務善伊勒登、多爾濟喀喇巴圖魯、色棱、達爾瑪等從征喀爾喀扎薩克圖汗,偵遁,仍還。四年春,臺吉超察海、噶爾楚、瑭古特、卓特巴、什達喇、莽古思、鄂爾齋、巴圖賴、額思赫爾、僧格等來朝,賚冠服、甲胄、弓矢。冬,騰機思、叟塞各率屬自喀爾喀來歸,入覲,獻駝馬。五年正月,賜叟塞、騰機思、騰機特、莽古岱、哈爾呼喇及臺吉布達什希布、阿玉什、噶爾瑪色棱、額爾克、辰寶、茂海、伊勒畢斯等甲胄、銀幣。十月,臺吉烏班岱、棟果爾、鄂爾齊、博希、沙津等來貢馬,賚冠服、鞍轡。六年,授騰機思扎薩克多羅郡王。七年,授叟塞扎薩克多羅杜棱郡王。以騰機思掌左翼,叟塞掌右翼。

順治三年,騰機思以車臣汗碩壘誘叛,率弟騰機特及臺吉烏班岱、多爾濟斯喀等逃喀爾喀。上遣師偕外籓軍由克嚕倫追剿至諤特克山及圖拉河,騰機思、騰機特遁,獲其孥。烏班岱、多爾濟斯喀為四子部落軍陣斬。師旋,以烏班岱從子託濟弗從叛,且隨剿,賜所俘。五年,騰機思及騰機特悔罪乞降,詔宥死,仍襲爵如初。康熙十年,所部歉收,詔發宣化府及歸化城賑粟儲之,復酌給馬牛羊。二十年,遣官察給兩翼災戶銀米。

二十七年,噶爾丹侵喀爾喀,詔選兵二千防汛。二十九年,噶爾丹襲喀爾喀昆都倫博碩克圖袞布,詔新部王以下原效力者,赴軍聽用。尋噶爾丹入烏珠穆沁界,諭還駐本旗要汛。三十五年,上親征噶爾丹,詔選兵赴烏勒輝聽調,以牧馬郭和蘇臺,偕浩齊特、阿巴噶、阿巴哈納爾諸部長董牧務。凱旋,諭★飼牧得宜,並優賚監牧及修道鑿井諸弁兵。以右翼扎薩克屬旺舒克、左翼扎薩克屬博羅扎布鄉導功,賜號達爾漢。復詔郡王薩穆扎之第三子多爾濟思喀布貝勒、博木布之長子素岱會師圖拉河,緝噶爾丹。尋分右翼兵赴珠勒輝克爾阿濟爾罕、左翼兵赴伊察扎罕,以不見賊蹤,撤還。五十四年,災,詔發張家口儲粟並帑十萬,自臺吉下六萬四千九百餘丁遍贍之。

雍正元年,右翼二等臺吉進達克以追捕叛賊遇害,晉贈一等臺吉,命視公爵致祭。子三:長噶爾瑪遜多布,封輔國公;次噶爾瑪策布騰;次恭格垂穆丕勒。以隨捕賊功,各晉臺吉秩有差。噶爾瑪遜多布爵後停襲。二年,所部災,賜銀賑之。九年,調兵屯克嚕倫河,防禦噶爾丹策凌。十年,有奏商都達布遜諾爾牧廠應移蘇尼特汛者,上飭止之,令各居其牧。十二年,所部兵駐防達哩剛愛。十三年,撤還。乾隆十二年,以災告饑,遣官往賑。

所部二旗:左翼駐和林圖察伯臺岡,右翼駐薩敏西勒山,隸錫林郭勒盟。爵四:扎薩克多羅郡王一;附多羅貝勒一;扎薩克多羅杜棱郡王一;附輔國公一,由貝勒降襲。洎五十六年,以是部連年被旱,又特賑之。道光十三年,右翼郡王與喀爾喀親王爭界,詔察哈爾都統凱音布往勘。尋以喀爾喀災,緩之。其地當漠南北之沖,歷代由漠南用兵漠北者,多出其途。光緒末,於蘇尼特右翼王府東北七十里置電報局,曰滂江,以通烏得叨林之電。是部亦產鹽,西南行銷山西豐寧諸。左翼有佐領二十,右翼有佐領十三。

阿巴噶部,在張家口外,至京師千里。東西距二百里,南北距二百有十里。東阿巴哈納爾,西蘇尼特,南察哈爾正藍旗牧廠,北瀚海。

元太祖弟布格博勒格圖,十七傳至巴雅思瑚布爾古特。子二,長塔爾尼庫同,號所部曰阿巴噶。塔爾尼庫同子二:長素僧克偉徵,子額爾德尼圖捫,號扎薩克圖諾顏;次揚古岱卓哩克圖,子多爾濟,號額齊格諾顏。初稱阿嚕蒙古,服屬於察哈爾。以林丹汗不道,徙牧瀚海北克嚕倫河界,依喀爾喀車臣汗碩壘。

天聰二年,偕喀喇沁、土默特、鄂爾多斯諸部長擊察哈爾眾四萬於土默特之趙城,復約喀爾喀偕喀喇沁乞師問察哈爾罪。六年,臺吉奇塔特楚琥爾攜眾五百內附。九年,大軍收服察哈爾,額爾德尼圖捫孫都思噶爾等附車臣汗碩壘表貢方物。崇德四年,額齊格諾顏多爾濟自喀爾喀來歸。時有同名多爾濟者,號達爾漢諾顏,率眾皆至。六年,詔授額齊格諾顏多爾濟為扎薩克多羅卓哩克圖郡王。順治八年,都思噶爾自喀爾喀來歸,詔授扎薩克多羅郡王。以多爾濟掌左翼,都思噶爾掌右翼,遣官定牧地。康熙六年,阿巴哈納爾部乞降,以阿巴噶牧地賜之。遣官視浩齊特、蘇尼特界外水草豐美地,指給阿巴噶移牧。二十九年,噶爾丹侵喀爾喀,詔所部王以下原效力者,赴軍聽用。復諭偕阿巴哈納爾供軍糈,兼防新降喀爾喀掠諸內扎薩克牧產。三十一年,以臺吉班第額爾德尼岱青、根敦、巴雅爾、納木塔爾、扎木素、齊達什等導烏梁海眾內附,均授二等臺吉。三十五年,上親征噶爾丹,牧馬郭和蘇臺,諭偕浩齊特、蘇尼特、阿巴哈納爾諸部長董牧務。凱旋,諭獎飼牧得宜,並優賚監牧及修道鑿井諸弁兵。復以所部達濟桑阿鄉導功,賜號達爾漢。三十六年,王、貝子、臺吉等朝正,請備馬從軍,慰令各歸所部。時有二等臺吉圖把扎布色臣楚琥爾者,年八十八,諭嘉其奮志報效,優賚之。五十四年,以災歉收,詔發唐三營儲粟賑之,復賜無產臺吉牧牲。雍正二年,遣官齎銀賑所部貧戶。九年,大軍剿噶爾丹策凌,徵兵駐達哩剛愛。十三年,撤還。乾隆十一年,旱災,賑之。五十四年,扎薩克卓里克圖郡王喇特納什第以事奪扎薩克,予其弟巴勒丹僧格一等臺吉扎薩克。

所部二旗,左翼駐科布爾塞哩,右翼駐巴顏額倫。爵五:扎薩克多羅郡王一;扎薩克一等臺吉一;附多羅卓里克圖郡王一;固山達爾漢貝子一;輔國達爾漢公一,由貝子降襲。右翼扎薩克巴勒丹僧格三傳至杜噶爾布木。咸豐七年,以報效軍需,予鎮國公銜。是部左右翼有佐領各十一。

阿巴哈納爾部,在張家口外,至京師千五十里。東西距百八十里,南北距四百三十六里。東浩齊特,西阿巴噶,南察哈爾正藍旗牧廠,北瀚海。

元太祖弟布格博勒格圖,十八傳至諾密特默克圖,號所部曰阿巴哈納爾。再傳至多爾濟伊勒登。子二:長色棱墨爾根,次棟伊思喇布。初稱阿嚕蒙古,依喀爾喀車臣汗碩壘。駐牧克嚕倫河界,其地在瀚海北。

崇德七年,有和碩泰者,臺吉達喇務巴三察屬也,攜孥內附。嗣托克托伊達嚕噶、達賴等至,皆優養之。康熙元年,臺吉阿喇納、噶爾瑪,宰桑固英等越瀚海南牧綽諾陀羅海近內汛。三年,色棱墨爾根復如之。守臣以聞,上知為喀爾喀所脅,宥罪遣歸。因諭喀爾喀以噶爾拜、瀚海為牧界,繼此有越者留勿遣。四年,喀爾喀復違諭,令阿巴哈納爾臺吉牧瀚海南。棟伊思喇布弗之從。尋偕臺吉阿喇納、噶爾瑪等率眾來歸,詔授扎薩克固山貝子。阿喇納、噶爾瑪以各攜丁七百餘,均授一等臺吉。五年,色棱墨爾根亦來歸。六年,詔授扎薩克多羅貝勒,遣官指示阿巴噶部移牧他所,以舊牧地給阿巴哈納爾。色棱墨爾根掌左翼,棟伊思喇布掌右翼。二十七年,噶爾丹侵喀爾喀,哲卜尊丹巴呼圖克圖奔赴內汛,所部班第岱青、車凌岱青奉詔督兵二百往護,復選兵千三百由瀚海偵噶爾丹。先是色棱墨爾根、棟伊思喇布來歸,阿巴哈納爾諸臺吉有留居喀爾喀者,至是隨哲卜尊丹巴呼圖克圖、額爾德尼臺吉納木扎勒等至,曰根敦額爾克,曰阿海烏巴什,曰伊克岱青,曰額爾克烏巴什,挈屬戶千餘,詔納之。二十九年,噶爾丹復侵喀爾喀,至烏勒札河,所部選兵四千,從大軍迎擊。復以所部索諾木伊嚕爾圖鄉導功,賜號達爾漢。五十四年,以災歉收,詔發唐三營儲粟賑之,復賜無產臺吉牲牧。雍正二年,遣官齎銀賑所部貧戶。九年,大軍剿噶爾丹策凌,檄兵駐達哩剛愛。十三年,撤還。

所部二旗:右翼駐昌圖山,左翼駐烏勒扈陀羅海。爵二:扎薩克多羅貝勒一,扎薩克固山貝子一。扎薩克貝子棟伊思喇布十傳至東林多爾濟。宣統元年,以報效軍需,賜郡王銜,世襲貝勒。左翼有佐領九,右翼有佐領七。

是部與烏珠穆沁、浩齊特、蘇尼特、阿巴噶四部合為十旗,統盟於錫林郭勒。於內扎薩克東四盟中距京稍遠,風氣獨守舊,迄清季無招墾之事。察哈爾都統行文令辦新政,其盟覆文頗不遜。咸豐中,嘗徵其兵備防,旋以不得力,撤之。同治中,以回匪東竄,徵其盟駝只濟軍。


\end{pinyinscope}