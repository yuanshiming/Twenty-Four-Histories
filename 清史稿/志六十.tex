\article{志六十}

\begin{pinyinscope}
禮四(吉禮四)

堂子祭天坤寧宮祀神令節設供求福祀神奉先殿壽皇殿

安佑宮綏成殿附滿洲跳神儀

堂子祭天清初起自遼沈,有設桿祭天禮。又於靜室總祀社稷諸神祇,名曰堂子。建築城東內治門外,即古明堂會祀群神之義。世祖既定鼎燕京,沿國俗,度地長安左門外,仍建堂子。正中為饗殿,五楹,南鄉,匯祀群神,上覆黃琉璃。前為拜天圜殿,北鄉。中設神桿石座,稍後,兩翼分設各六行,行各六重,皇子列第一重,次親王、郡王、貝勒、貝子、公,各按行序,均北鄉。東南為上神殿,三楹,南鄉。祭禮不一,而以元旦拜天、出征凱旋為重,皆帝所躬祭。其餘月祭、桿祭、浴佛祭、馬祭,則率遣所司。崇德建元,定制,歲元旦,帝率親王、籓王迄副都統行禮。尋限貝勒止,已復限郡王止,並遣護衛往掛紙帛。

凡親祭,前期十二月二十六日,內府官赴坤寧宮請朝祭、夕祭神位,安奉神輿,內監舁行。前引御仗八、鐙四,司俎官六人,掌儀司一人,侍衛十人,導至饗殿供奉。朝夕獻香如儀。故事,神位所懸紙帛,月終積貯盛以囊,除夕送堂子,與凈紙、神桿等同焚。時內府大臣率長史、護衛掛新紙帛各二十有七。昧爽,帝乘輿出宮,陪祀王公等隨行。至堂子內門降,入中門,詣圜殿就拜位,南鄉,率群臣行三跪九叩禮。畢,回鑾。翼日,奉神位還宮。康熙十一年,詔元旦拜堂子禮宜明備,用鳴贊官。明年,罷漢官與祭。二十九年,諭令皇子隨行禮,內府大臣圜殿進楮帛畢,次進皇太子楮帛。

月祭,歲正月初旬諏吉,餘月朔日。司俎二人,就杉柱上掛紙帛數等。元旦,案陳時食盤一、醴酒■D9一。司香上香,內監執三弦、琵琶,坐甬道西,守堂子人持拍板坐其東。司祝進跪,司香授■D9,司祝受之,獻酒。奏神弦,鳴拍板,拊掌應節。凡六獻,皆贊歌「鄂囉羅」,守堂子人亦歌。獻畢,一叩,興,合掌致敬。弦、板止,司祝執神刀進,奏弦、拍板如初。司祝一叩,興,司俎贊歌「鄂囉羅」,眾和歌。司祝舉神刀誦神歌曰:「上天之子,紐歡臺吉,武篤本貝子,某年生小子,某年生小子,今敬祝者,豐於首而仔於肩,衛於後而護於前。畀以嘉祥兮,齒其兒而發其黃兮,偕老而成雙兮,年其增而歲其長兮,根其固而神其康兮。神兮貺我,神兮佑我,永我年而壽我兮。」凡三禱,如前儀,誦贊者九。司祝跪,一叩,興,誦贊三。弦、板止,復跪,一叩,興,合掌退。

立桿大祭,歲春、秋二季月朔,或二、四、八、十月上旬諏吉行,桿木以松,長三丈,圍徑五寸。先一月,所司往延慶州屬採★E0,樹梢留枝葉九層,架為桿,齎至堂子。前期一日,樹之石座。崇德初,定親王、郡王、貝勒祭三桿,貝子、鎮國、輔國公二,鎮國、輔國將軍一。月朔大內致祭,初二日後依次祭,凡祭三桿者,定期內祭其一,過旬祭其二。祀日有數家同者,仍按位為等差,違例多祭與爭先越祭並處罰。後改定大內至入八分公俱祭一桿,將軍不祭。

屆日,司香豫懸神幔,炕上置漆案,陳碟三。前置棻案,黃磁碗二。圜殿置二棻案,高者陳爐,卑者陳碗,前設採氈。司俎二人赴坤寧宮請佛亭及菩薩、關帝像,舁至堂子。安佛亭於座,像懸幔以三繩,系兩殿神桿間。懸黃幡,掛紙帛,圜殿掛帛亦如之。饗殿北炕案上陳打糕、搓條餑餑盤九,酒■D9三,圜殿高案則盤三■D9一。每獻,司祝挹碗酒注■D9,兩殿祭獻歌禱如前儀。祝辭曰:「上天之子,佛及菩薩,大君先師、三軍之師、關聖帝君,某年生小子,某年生小子,今敬祝者,貫九以盈,具八以呈,九期屆滿,立桿禮行。爰系索繩,爰備粢盛,以祭於神靈。」餘辭同月祭。卒事,司香卷幔、徹像奉入宮。

若帝親祭,殿內敷採席,覆紅氈,甬道布椶薦。屆時乘輿出宮,滿大臣隨扈至堂子街,王公跽俟,興,從之。帝降輿入中門,詣饗殿前東鄉坐,司祝獻酒,舉神刀,禱祝,奏弦、拍板,拊掌,歌「鄂囉羅。」帝入,一跪三叩。圜殿同。畢,升座,賜王公等炕前坐。尚膳正、司俎官進胙糕,尚茶正獻福酒,帝受胙,分賜各王公。禮成,還宮。遇壇、廟齋期或清明節,再涓吉以祀。

月朔祀東南隅尚錫神亭,即堂子上神殿也。神曰田苗,神案上盤一、■D9一,分陳時食醴酒,司香上香,司俎掛凈紙杉柱上,諸王護衛依次掛之。內管領一人入,除冠服,解帶,跪叩,祝辭曰:「上天之子,尚錫之神,月巳更矣,建始維新,某年生小子,敬備粢盛兮,潔楮並陳。惠我某年生小子,貺以嘉祥兮,畀以康寧。」畢,退。或謂祀明副總兵鄧子龍也,以與太祖有舊誼,故附祀之。

四月八日佛誕,祭祀前期,饗殿懸神幔,選覺羅妻正、副贊祀二人為司祝。祭日,不祈報,不宰牲,不理刑名。屆時赴坤寧宮請佛亭及菩薩、關聖像,司俎內監置椴葉餑餑、釀酒、紅蜜於盒以從,至則陳香鐙,獻糕酒,取紅蜜暨諸王供蜜各少許,注黃磁浴池。司祝請佛,浴畢,以新棉承座,還奉佛亭,陳椴葉餑餑九盤,酒■D9、香碟各三,並諸王所供餑餑、酒。圜殿亦如之。司香上香,司祝獻酒九巡,餘略如月祭、桿祭。崇德元年,定八旗王、貝勒各一人,依次供獻。厥後唯親王、郡王行之。

馬祭,歲春、秋季月,為所乘馬祀圜殿。正日,司俎掛紙帛如常數,陳打糕一盤、醴酒一■D9,縛馬鬃、尾綠紬二十對。司香上香,牧牽十馬,色皆白,立甬道下。司祝六獻酒,奏樂如儀。所禱之神同月祭,唯祝辭則易為所乘馬。「敬祝者,撫脊以起兮,引鬣以興兮,嘶風以奮兮,噓霧以行兮,食草以壯兮,齧艾以騰兮。溝穴其弗逾兮,盜賊其無擾兮。神其貺我,神其佑我。」禱訖,取紬條就香爐薰禱,司俎以授牧長,系之馬尾。是日,馬神室並奉朝祭、夕祭神位,遣內府大臣行禮。朝祭豫懸幔,舁供佛小亭奉炕上,案陳香、酒、食品。司俎進二豕,熟而薦之。司香上香,舉■D9授司祝,司祝進跪三獻,歌奏如前。訖,授■D9司香,一叩,興,合掌致敬。復跪,祝,一叩,興。取縛馬鬃、尾紅紬條七十對,就香碟薰禱,授司俎官,轉授上駟院侍衛,分給各廠、院。卿、侍衛、廄長入,隨食肉。

其夕祭儀略如朝祭,候肉熟分陳案上,進跪叩祝同。司祝坐杌置夕祭定處,設小案、小腰鈴,別置神鈴。案東展背鐙布幕,振鈴桿,搖腰鈴,誦神歌,前後所禱所祝之神詳下。

背鐙祭,其辭禱同朝祭,祈請者四,禱後跪祝辭、供肉祝辭亦如之。畢,取縛馬鬃、尾青紬條三十對,仍就香碟薰禱授如初。翼日,為牧群滋息,復行朝、夕祭如初禮。唯祝辭易「今為牧群繁息」六字,「溝穴」二句易為「如萌芽之發育兮,如根本之滋榮兮」,餘辭並同。又司香取縛馬鬃、尾紬條二百八十對,皆青色。崇德初制,為馬群致祭,唯親王至輔國公得行。乾隆三十六年,定春、秋騸馬致祭,薩滿叩頭。薩滿者,贊祀也。訖,取所送青色十馬系綠紬條如數。又定朝祭御馬拴紅紬條,大凌河騾馬拴青紬條,為恆制。

凡出師凱旋,皆有事堂子。崇德元年,太宗徵明及朝鮮,明年班師,並告祭。世祖定中原,建堂子。嗣是聖祖平吳三桂、察哈爾,迄歷朝靖亂,皆以禮祗告。

凡親征告祭命下,涓吉,屆期兵部建大纛,具祀纛篇。帝御戎服,出宮乘騎,前後翊衛,午門鳴鐘鼓,法駕鹵簿為導,饒歌大樂,備而不作。至玉河橋,軍士鳴角螺,帝入堂子街門降騎,角螺止。入中門,詣圜殿就拜位,南鄉立,率群臣行三跪九叩禮。角螺齊鳴。出內門,致禮纛神。禮成,樂作,車駕啟行。凱旋日,率大將軍及從征將士詣堂子告成。若命重臣經略軍務以討不庭,禮亦如之。

乾隆十四年,詔言:「堂子致祭,所祭即天神也。列祖御宇,稽古郊禋,燔柴鉅典,舉必以時。堂子則舊俗相承,凡遇大事,及春、秋季月上旬,必祭天祈報,歲首尤先展禮。定鼎以來,恪遵舊制。考經訓祭天,有郊、有類,有祈穀、祈年,禮本不一。兵戎國之大事,命將先禮堂子,正類祭遺意,禮纛即禡也。或在行營別有征討,不及祭告堂子,則行望祭,其誠敬如此。夫出師告遣,凱旋即當告至。乃天地、宗社皆已祝冊致虔,且受成太學,而堂子則弗及,禮官疏略,如神貺何?其詳議以聞。」尋奏凱旋、告祭之禮。報可。

坤寧宮祀神昉自盛京。既建堂子祀天,復設神位清寧宮正寢。世祖定燕京,率循舊制,定坤寧宮祀神禮。宮廣九楹,東暖閣懸高宗御製銘,略言:「首在盛京,清寧正寢,建極熙鴻,貞符義審。思媚嗣徽,松茂竹苞,神罔時恫,執豕酌匏。」其眷眷祀神如此。

宮西供朝祭神位,北夕祭神位,廷樹桿以祀天。朝祭神為佛、為關聖,夕祭神為穆哩罕諸神,祝辭所稱納丹岱琿為七星之祀,喀屯諾延為蒙古神,並以先世有功而祀者。餘如年錫、安春阿雅喇諸號,「納爾琿、安哲、鄂囉羅」諸字,雖訓義未詳,而流傳有自。

綜其所祀,曰元旦行禮,曰日祭,曰月祭及翼日祭,曰報祭,曰大祭,曰背鐙祭及翼日祭,曰四季獻神。其儀節大率類堂子。茲略舉其小異者。

元旦子刻,司香上香,帝、後行禮。日祭,順治初,定大內日祭,朝以醜、寅,夕以未、申。

朝祭,司香豫懸黃幔,奉菩薩、關帝像,東鄉。左、右炕上置低桌二,陳爐、■D9各三,時果九,糕十。炕前置獻案,黃磁碗二,虛其一,以一實酒。案下列樽酒,前設採氈。昧爽,司俎等進二豕,司香獻香,執弦板內監暨司俎官帥屬進,奏神弦,拍板,拊掌應節。司祝跪六獻,酒灌虛碗中,一叩,興,合掌致敬。餘如堂子朝祭儀。司祝復跪,一叩,興。又誦贊三,弦板止,侍側。帝親詣,入門,立神位前。司祝先跪,帝跪。司祝致辭,帝行禮,興,司祝叩,興,合掌致敬。後隨行禮。將事者俱退,留司俎、司祝、司香婦人侍行禮。時帝南後北,帝不與祭。司祝叩興後,徹■D9,奉神像納黃筒,位西楹大亭中。

徙幔稍南,安關帝像正中,執弦板者進,跪坐,司香斂氈三折之,奏弦拍板如初。司祝跽氈上,致辭,獻香酒,司祝酌酒,執豕耳灌之,一叩,弦板止。司祝舉豕於俎、復奏、拍,灌如初,一叩,興,退。司俎如法刲牲,熟而薦之。司香獻香,司俎進跪,凡三獻,俱奏弦、拍板、拊掌。畢,徹饌,列胙長案上,或帝率後受胙,或率王、公等食肉,否則大臣侍衛進食之。

夕祭,司香豫懸青幔,西樹桿,懸大小神鈴七。幔內奉穆哩罕神、畫像神、蒙古神,南鄉。前低桌二,陳爐、■D9各五。別懸菩薩像西楹大亭,鋪油紙,設案如朝祭。既上香,司祝系裙、束腰鈴、擊手鼓,坐杌上誦神歌祈請曰:「自天而降,阿琿年錫之神,與日分精,年錫之神,年錫唯靈。安春阿雅喇、穆哩穆哩哈、納丹岱琿、納爾琿軒初、恩都哩僧固、拜滿章京、納丹威瑚哩、恩都蒙鄂樂、喀屯諾延,某年生小子,今為所乘馬敬祝者」雲云。辭同馬祭,擊鼓拍板和之。初禱曰納丹岱琿、納爾琿軒初,二禱曰恩都哩僧固,三禱曰拜滿章京、納丹威瑚哩、恩都蒙鄂樂、喀屯諾延,三禱並為馬祝云云。皆擊鼓為節,內監亦擊拍板以和,止,退,釋手鼓腰鈴,司香設採氈,帝親行禮如朝祭儀。後隨行,則帝東後西。刲牲、薦俎暨叩跪、致辭如初。畢。遇齋期、國忌,不宰牲。並十二月二十六日請神送堂子後,宮內均停祭。

乾隆十二年,制定坤寧宮祭神背鐙供獻,其儀,夕祭薦肉後,司香斂氈,展青綢幕,掩鐙火,眾出闔戶,留司祝及執板鼓內監侍。司祝坐杌上振桿鈴,初向神鈴致祈請,辭曰:「哲,伊埒呼,哲,納爾琿。掩戶牖以迓神兮,納爾琿。息甑灶以迓神兮,納爾琿。來將迎兮,侑坐以俟,納爾琿。秘以俟兮,幾筵具陳,納爾琿。納丹岱琿藹然降兮,納爾琿。卓爾歡鐘依惠然臨兮,納爾琿。感於神靈兮來格,蒞於神鈴兮來歇,納爾琿。」二次搖神鈴致禱,辭曰:「納丹岱琿、納爾琿軒初、卓爾歡鐘依、珠嚕珠克特亨,某年生小子,今為所乘馬祝者」雲云。餘辭同馬祭。三次向腰鈴致祈請,辭曰:「哲,伊埒呼,哲,古伊雙寬。列幾筵以敬迓,古伊雙寬。潔粢盛兮以恭延,古伊雙寬。來將迎兮盡敬,古伊雙寬。秘以俟兮申虔,古伊雙寬。乘羽葆兮陟於位,古伊雙寬。應鈴響兮降於壇,古伊雙寬。」四次搖腰鈴,復致禱,辭曰:「籥者唯神,迓者斐孫,犧牲既陳,奔走臣鄰。仍為所乘馬敬祝者」雲云。每次並擊鼓拍板以和。畢,啟扉明鐙,司俎徹俎,司香卷幔,奉神像納硃匱。

月祭略同日祭,唯食品因月而殊,灌豕耳以酒不以水。如為皇子祭祀,則司祝禱祝,皇子叩拜。

翼日祭天,安佛、菩薩像西楹大亭,神桿東北置案一,西鄉。奉桿倚柱座前,桿首鄉東仰。案陳銀盤三,一實米居中。西北置幔架,覆紅罽。東北置牲案。昧爽,司俎進一豕。司香設採氈閾內,帝行禮,鄉神桿南面跪。司俎進,舉盤中米灑之。祝禱畢,興。不親祭,則司祝奉御衣叩拜。後隨行,帝居中,後傍西。刲牲熟薦,陳頸、膽左右銀盤,縷肉為膾,列碗二,佐以箸;炊稗為飯,列碗二,佐以匙;相間以獻。帝復行禮,灑米如初。禮成,司俎奉頸骨桿端,膽、膾及米置桿碗,桿遂立。以所獻肉飯進,帝後受胙,退。如為皇子祭天,則皇子叩拜。不親祭,則司祝奉皇子衣服叩拜。

報祭,歲春、秋二季,立桿大祭。前期四旬,釀酒西炕上,祭前一日漉之。司香染布為神冠,制楮帛。

大祭日,司俎婦人打糕作穆丹條子,餘如前儀。其翼日祭天,與月祭翼日同。

四孟月大祭,亦曰四季獻神,懸朝祭、夕祭神幔,並同日祭儀。涓吉,具馬二,牛一,金、銀錠各二,蟒緞、龍緞、片金倭緞、閃緞、各色緞十,毛青布十,置案。掌儀官等前引,內府大臣、上駟院卿同行。自乾清右門舁入,逕交泰殿,至坤寧宮門外。陳馬於西,列牛於東。司俎等奉金銀緞布入,司香陳案上,奉朝祭神位前,加金銀其上。司祝跪致辭,一叩,興。復舉案夕祭神位前,如上儀。帝親祭,禮同月朔。陳獻畢,司香舉金銀緞布貯案下,侍衛等牽牛馬出。越三日,宮殿監諸神位前,以金銀緞布及牛馬授會計司發售,計直購豕以祭。故事,帝獵南苑或他所,射得包、鹿,如尾氾腑臟無傷者,雖小創必整潔之,備供獻,傷多體缺者舍之。至四時進獻,按時以奉,春雛雞二,夏子鵝一,秋魚一,冬雉二,選肥且澤者以將誠焉。

令節設供萬壽節、元旦節,宮殿監率各首領設供案天香亭內,北鄉。奉安神牌、香屬、鐙爐、斗香、拜褥各具,陳祭品七十有五。屆時帝拈香行禮,畢,送燎還宮。

冬至、夏至或未親行郊禮,則設供宮中。宮殿監設供案,冬至北鄉,夏至南鄉。奉安神牌、祭品同,拈香送燎亦如之。

立春、立夏、立秋、立冬設案如前儀。春東鄉,夏南鄉,秋西鄉,冬北鄉。陳祭品三十有六,羊、豕各一,儀如初。

仲春朔祭日,仲秋望祭月,七月七夕祭牛、女,陳祭品四十有九。帝行禮畢,宮殿監奏請皇后、皇貴妃、貴妃、妃、嬪行禮,畢,帝送燎還宮。

求福祀神所稱佛立佛多鄂謨錫瑪瑪者,知為保嬰而祀也,亦名換索。其儀,諏吉有期,豫釀醴酒。前期數日,選無事故滿洲九家,攢取棉線綢片,捻線索二紐,小方戒綢三。先一日,司俎官偕奉宸苑官赴西苑斫取柳條全株,高九尺,圍徑三寸。屆期赴坤寧宮廊下,樹柳枝於石,懸凈紙、戒綢。幔懸神像。炕上設低案一,陳香碟、醴酒各三,豆糕、煤糕、打糕各九。西炕設求福高案,陳鯉魚、稗米飯、水訄子各二,醴酒、豆糕等皆九數。稍北植神箭,懸線索其上,用三色綢片夾系之,令穿出戶,系之柳枝。司香展採氈,帝、後親詣行禮,如朝祭儀。

內監司俎官率屬進,奏神弦,鳴拍板,司祝執神刀進,誦神歌禱辭曰:「聚九家之採線,樹柳枝以牽繩。舉揚神箭,以祈福佑,以致敬誠。某年生小子,綏以多福,承之於首,介以繁祉,服之於膺。千祥薈集,九敘阜盈。亦既孔皆,福祿來成。神兮貺我,神兮佑我。豐於首而仔於肩,衛於後而護於前。畀以嘉祥兮,偕老而成雙兮。富厚而成雙兮,富厚而豐穰兮。如葉之茂兮,如木之榮兮。食則體腴兮,飲則滋營兮。甘旨其獻兮,硃顏其鮮兮。歲其增而根其固兮,年其永而壽其延兮。」如是者三,眾歌「鄂囉羅」和之。

禱畢,司香舉線索、神箭授司祝,司香舁高案出戶外,列柳枝前。司祝左執神刀,右執神箭,立案前。帝立正中,後立檻內東次。皆跪,司祝對柳枝舉揚神箭,以練麻拭其枝。初次誦禱畢,舉箭奉練麻進,帝三埒而懷之,歌如前。帝、後一叩,興,柳枝上灑以酒,夾以糕,司祝揚箭歌禱如式。凡三。帝詣神位前跪,司祝以箭上線索二分奉帝、後,致辭,叩,興,合掌致敬。帝、後同一叩,興。司祝進神胙,帝、後受之,還宮。祀肉與糕不出門,則分給諸人,令戶內盡食之。

其夕祭求福,帝、後行禮如夕祭儀。柳枝所系線索貯於囊,懸西壁上。其枝司俎官齎送堂子。至除夕,與神桿紙帛爇化之。

奉先殿順治十三年,詔建景運門東北,前後各九楹,如太廟寢制。中為堂,左神庫,右神廚。明年殿成,世祖躬妥神位,讀祝大饗。定制,元旦、冬至、歲除、萬壽、冊封、月朔、望,奉神位前殿,帝親行禮,供獻如太廟大饗儀。唯立春、上元、四月八日、端陽、重陽皆尋常節,國忌、清明、霜降、十月朔屬哀慕期,親祭,不贊禮、作樂。七夕如常供。四月八日、七月望日陳素果。月薦新,帝親獻。

凡常例供獻,後殿行之。饗太廟畢,行躬告禮,上香燭。又定日供湯、飯、果、肉各五盤。元旦、萬壽,請太廟後殿四祖、四后神位至奉先殿,與列聖、列後合饗。其後罷奉請,就太廟後殿祀之。是歲冬,御經筵,上親祭焉。

十七年,以並夾室乖制,諭令夾室行廊外中通為敞殿九楹,★A1改建如旨。

明年,聖祖嗣服,用禮臣言,依明洪武三年例,朝夕焚香,朔、望瞻拜,時節獻新,生、忌致祭,具常饌,行家人禮。其冬世祖升祔,奉神位至前殿行大饗。禮成,還奉後殿神龕。厥後祔饗仿此。康熙十三年,罷日供食,早、晚燃香燭。十五年,罷冊封大饗,遣官祗告後殿。凡上徽號、冊立、御經筵、耕耤、謁陵、巡狩、回鑾亦如之。雍正十三年,準太廟時饗例,增上香儀。

乾隆二年修殿,徙神位暫安太廟。其秋會值太祖、太宗忌辰,帝擬親饗,群臣言故事無素服入廟,乃止。道光元年,增修後殿龕座。中室列龕三,奉太祖、太宗、世祖。左一室龕二,奉聖祖、高宗。右一室龕二,奉世宗、仁宗。昭、穆仍舊制。餘四室分列八龕焉。

凡親饗,先三日致齋。先一日,掌儀司進祝版,割牲瘞毛血,潔治祭品。屆日昧爽,內監啟寢室神龕,執事官各事咸備。內府官省畢,分詣寢室前,跪上香,三叩,興。奉列聖、列後神位以次行。皇后祔饗,同至前殿,安於座位,南鄉,祔後西鄉。詣各香案前跪,三叩,興。屆時帝袞服出宮,至誠肅門降輿,入左門,盥訖,就拜位,北面立,迓神,奏貽平章。導詣太祖香案前,跪上炷香一,瓣香三。旋位,行三跪九叩禮。導詣皇後香案前,立上香,旋位。行初獻禮,奏敉平章,舞干戚,有司揭尊冪,勺挹實爵,司帛、司爵以次至各案前。獻訖,司祝詣祝案前跪,三叩,興。跪案左,奉祝版。帝跪,司祝讀祝,興,安於篚,叩如初。帝三叩,興。行亞獻禮,奏敷平章,舞羽籥,獻爵,儀如初。行終獻禮,奏紹平章,餘並同初獻。徹饌,奏光平章。畢,請神還寢室,三叩,退。贊「舉還宮樂」,奏乂平章,帝復行三跪九叩。司祝、司帛以次送燎所,帝轉立東旁。禮成,仍出左門。餘如來儀。

或遣皇子代祭,前諸儀同。殿門外正中設拜位,入右門,至西階下盥手,升階詣拜位行禮。祝、帛送燎,避立西旁,仍自西階退。

其月朔薦新,正月鯉魚、青韭、鴨卵,二月萵苣、菠菜、小蔥、芹菜、鱖魚,三月王瓜、蔞萵、蕓薹、茼蒿、蘿卜,四月櫻桃、茄子、雛雞,五月桃、杏、李、桑葚、蕨香、瓜子、鵝,六月杜梨、西瓜、葡萄、蘋果,七月梨、蓮子、菱、藕、榛仁、野雞,八月山藥、慄實、野鴨,九月柿、雁,十月松仁、輭棗、蘑菇、木耳,十一月銀魚、鹿肉,十二月蓼芽、綠豆芽、兔、蟫蝗魚。其豌豆、大麥、文官果諸鮮品,或廷旨特薦者,隨時內監獻之。順治十四年,定月薦鮮獻粢盛牲品。康熙十三年,定薦新日,掌儀司詣後殿行禮。獻帛爵用侍衛。

壽皇殿舊制三室,在景山東北。太祖、太宗、世祖及列後聖容,向奉體仁閣。雍正元年,命御史莽鵠立繪聖祖御容,供奉壽皇殿中殿,遇聖誕、忌辰、元旦、令節,率皇子、近支王公展謁奠獻。凡奉安山陵、升祔太廟禮成,皆親詣致祭。蓋月必瞻禮,或至三詣焉。

乾隆元年,奉世宗聖容東一室,嗣後列朝聖容,依次奉東西室,為恆例。三年,定謁陵、省方啟蹕、回鑾均詣壽皇殿行禮。尋定萬壽節行禮如諸令節儀。十三年,徙建景山正中,如安佑宮制。大殿九室,左右殿各三楹,東西配殿各五楹,其冬成。高宗親制碑記,其頌曰:「唯堯巍巍,唯舜重華,祖考則之。不競不絿,仁漸義摩,祖考式之。弘仁皇仁,明憲帝憲,小子職之。是繼是繩,曰明曰旦,小子忽之。天游雲殂,春露秋霜,予心惻惻。考奉祖御,於是壽皇,予仍即之。制廣而正,爰經爰營,工勿亟之。陟降依憑,居歆攸室,羹墻得之。佑我後嗣,綿於萬,匪萬億之。觀德於茲,無然畔援,承欽識之。」

十五年,諭:「前代安奉神御,率在寺中,別殿凈宇,本無定所。敬念列祖創垂,顯承斯在。永懷先澤,瞻仰長新。式衷廟祫之儀,斯協家庭之制。應迎列祖、列後聖容奉壽皇殿,歲朝合請懸供,肅將祼獻。」於是奉聖祖、世宗御容,並自體仁閣迎太祖、太宗、世祖御容,乃定除夕敬懸,供鮮果、肉醬。元旦大饗,獻磁器籩豆供品,並上香行禮。初二日如除夕供。禮畢尊藏。

又元旦帝有事堂子、奉先殿,訖,詣壽皇殿行禮。除夕、初二日,命皇子番行。上元節供餅餌,秋季展聖容,宮殿監敬謹將事。是歲繪列朝聖容成,親詣奉安,行大饗。嘉慶四年,詔壽皇殿供奉神御,始自聖祖,凡遇忌辰、誕辰,皆應躬親展敬,示子孫遵行,安佑宮亦如之。

安佑宮在圓明園西北隅,建工始乾隆五年,迄八年蕆事。大殿九室,硃扉黃甍,如寢廟制。中龕懸聖祖御容,左世宗,右高宗。龕前陳彞器、書冊、佩用服物,合設中和韶樂一列。帝臨御園中,遇列聖誕辰,忌辰,令節,朔、望,並拈香行禮。謁陵、省方啟鑾、回蹕,皆躬詣祗告焉。高宗親制碑記,略言:「朔酌望獻,西漢原廟遺制。宋時神御殿亦本斯義,蓋奉安列朝御容所也。上元結鐙樓,寒食設秋千,視漢已備。而崇建遍郡國,奉祀在禪院,識者譏之。我皇祖聖祖,恩澤旁覃,僻邑窮谷,飲其德而不知,子孫臣庶,躬被教育者,宜其謳歌慨慕而未有已也。是以皇考世宗謹就壽皇殿奉安御容,朔望瞻禮,而於皇祖所幸暢春園,亦陳薦如儀。有漢、宋備物備禮之誠,無宋代祀繁致褻之弊。予小子心懍紹庭,念茲圓明園為我皇考囿沼地,築室九楹,敬奉皇祖其中,奉皇考配東一室。所謂禮緣義起,有其舉之,莫敢廢也。」

永佑寺在熱河避暑山莊萬樹園旁,乾隆十六年建。有樓五楹,奉聖祖、世宗、高宗御容,雲山勝地樓奉仁宗御容,陳設一如安佑宮。車駕蒞至輒懸奉,回蹕後庋藏。丹墀列高宗御製碑文,略言:「創立精藍,爰名永佑。固不特鐘魚梵唄,足令三十六景借證聲聞;而皇祖聖日所照,千秋萬歲後,子孫臣庶,莫不永如在之思。是即釋迦之耆闍崛山,金剛法座,天龍擁護;而所以繩武寧親,祝釐養志,亦於是託焉云爾。」仁宗御制永佑寺瞻禮敬紀,亦頗惓惓祖若父焉。

道光時,移供聖御繼德堂,更題曰綏成殿。中室聖祖,左世宗,右高宗,左次室仁宗,以後列朝御容,仍依次懸左右室雲。

滿洲俗尚跳神,其儀,內室供神牌,或用木龕,室正中、西北龕各一。凡室南鄉北鄉,以西方為上;東鄉西鄉,以南方為上:頗與禮經合。南龕下懸簾幕,黃雲緞為之。北龕上置杌,杌下陳香盤三,木為之。春、秋擇日致祭,謂為跳神。前一月,造酒神房。前三日,朝暮獻牲各二,名曰烏雲,即引祀也。前一日,神前供打糕各九盤,以為散獻。大祀日,五鼓獻糕,主人吉服鄉西跪,設神幄鄉東,中設如來、觀音神位。女巫舞刀祝曰:「敬獻糕餌,以祈康年。」主人跪擊神版,諸護衛亦擊,並彈弦、箏、月琴和之,其聲嗚嗚然。巫歌畢,主人一叩,興。司香婦請神出。戶牖西設龕,南鄉奉之。司俎者呼「進牲」,牲入,主人跪,家人皆跪。巫者前致辭,以酒灌牲耳,牲耳甡,司俎高聲曰:「神已領牲。」主人叩謝。庖人刲牲,熟而薦之。主人再拜謁,巫致辭。主人叩畢,巫以系馬吉帛進,祝如儀。主人跪領帛,以授司牧,一叩,興。乃集宗人食胙肉,令毋出戶庭。

其夕供七仙女、長白山神、遠祖、始祖,位西南鄉。以神幙蔽窗牖,舞刀進牲致祝如朝儀。唯伐銅鼓作淵淵聲,主家亦擊手鼓、架鼓,以銅鼓聲為應。誦益急,跳益甚。禮成,眾受福。次早設位庭院前,位北鄉,主人吉服如儀。用男巫致辭畢,灑以米,趨退。主人叩拜。牲肉皆刲為菹醢,和稻米以進。名曰祭天還原。

又明日,神位前祈福,供餅餌,綴五色縷。祝辭畢,以縷系主人胸,謂之受福。三日祭乃畢。

長白滿洲舊族近興京城者,祀典禮儀皆同。唯舒穆祿氏供昊天上帝、如來、菩薩諸像,又供貂神其側。納蘭氏則供羊、雞、魚、鴨諸品,巫者身系銅鈴跳舞,以鈴墜為宜男兆。蒙古跳神用羊、酒,輝和跳神以一人介胄持弓矢坐墻堵,蓋先世有劫祀者,故豫使人防之,因沿為制。跳神之舉,清初盛行,其誦祝辭者曰薩嗎,迄嘉慶時,罕用薩嗎跳神者,然其祭固未嘗廢也。


\end{pinyinscope}