\article{列傳七十二}

\begin{pinyinscope}
王紫綬袁州佐黎士弘多弘安佟國聘王繻田呈瑞張孟球

王紫綬,字金章,河南祥符人。順治三年進士,選庶吉士。散館,授編修。乞養歸,僑寓蘇門山中,從孫奇逢講學。居十有七年,母歿,服闋,康熙十二年,授江西贛南道副使。

吳三桂反,贛南總兵劉進寶有謀略,紫綬推誠結納,預籌防禦。既而江西降眾屯墾者相繼叛,惟贛南尚未動。紫綬與進寶謀:「閩、粵反已見端,贛南扼其間,應援前朝故事,設巡撫以資鎮攝。」申疆吏上請,允之。十四年,賊勢益熾,山寇蜂起,鎮兵疲於奔命,乃練鄉勇以輔之,屢殺賊有功。十五年,巡撫白色純及進寶先後卒官,參將周球領鎮兵。三桂將高得捷、韓大任據吉安,餉道絕,屬縣相繼陷。大任屢致書勸降,送偽署巡撫劄,紫綬斬其使。球以乏餉為難,紫綬集士商勸輸間架稅,得白金四萬畀球,餉以無缺。鎮南將軍覺羅舒恕率禁旅下廣東,為尚之信將嚴自明所敗,兵退,距贛州三十里。自明約得捷由吉安會師夾擊。紫綬薦降將許盛率所部漳州水兵五百人益師,夜泅江斫賊營,禁旅繼之,擊敗自明。得捷等勢孤,不敢復窺贛。鎮兵出剿土寇,掠村民,紫綬曰:「鄉民脅從,若並以賊論,贛南二府十六縣將無孑遺。」戒鎮將毋妄發兵,飭有司招撫,分別留遣,賑濟難民,境內稍安。乃規復萬安、泰和兩縣。自螺山間道達墨潭,可登舟,於是南昌道始通,運餉銀十萬至。又發附近倉穀贍軍,人心大定。巡撫佟國禎亦自間道至,始知紫綬已擢浙江督糧道參政。贛南久不通驛報,大學士李霨言於朝曰:「紫綬死守危疆,三年於茲。為國惜才,援而出之,猶可大用。」故有是擢。紫綬聞命泣下。

十六年,上官,察積弊,嘆曰:「糧官不可為也!漕截減而軍困,白折浮而民困,吾安忍竭東南之澤而漁之?」一月即引疾去。迨開博學鴻詞科,魏象樞以紫綬與湯斌同薦入試。放還。卒。

袁州佐,字左之,山東濟寧人。順治十二年進士,授陜西乾州知州。入為工部員外郎,遷郎中。有清直聲,胥吏不敢牟利。時山陵工巨,經費浩穰,州佐曰:「民困極矣,寸縑尺縷,皆閭閻膏血!」力清乾沒,司焚帛,省金錢鉅萬。出為陜西甘山道僉事。青海蒙古諸部覬得大草灘為牧地。康熙九年,偕提督張勇度地畫界,堅拒,寢其議。自後青海蒙古諸部人不敢復窺邊。歲協西寧餽運,負載千里,甘州民苦之,州佐力請得罷。甘州駐兵數千,待餉急,力為籌備,軍得宿飽。十年,遷直隸口北道參議。地確民貧,逋課積累,倉儲歷歲侵漁,耗蝕無算。州佐請按籍覈實,清宿蠹。大吏懼以失察得罪,陽韙而陰沮之。州佐擘畫盤錯,致疾乞休,未去官,卒。

州佐在甘州久,言邊境要害戰守狀,原委斠然。謂邊地民稀,宜用開中法,分河東鹽引三之一輸粟河西資軍食;又宜簡練鄉勇,拔置卒伍,不待召募,可坐收精銳。時詔簡監司具才望者入為卿貳,州佐在選,會卒,未及用。

黎士弘,字媿曾,福建長汀人。少讀書山中二十年,篤於孝友。順治十一年,舉順天鄉試,授江西廣信府推官。鋤強糾貪,奸宄斂戢。理讞牘,脫無罪數百人,時為語曰:「遇黎則生。」署玉山縣事。兵後城中草三尺,不辨街巷,居民才三十二家。士弘立學建治,招集流亡,墾田定賦,民復舊業。裁缺,改授永新知縣。政清獄簡,與民休息。舊例,二月開徵,五月解其半。士弘陳於上官曰:「縣小民窮,二月寫租十石,貸銀一兩,三月可減至六石,四月則三石。請以四月開徵,五月解,展兩月之徵,已為窮民留數萬之糧。」布政使劉楗素寬仁,即允之。

甲訴乙悔婚。鄉俗婚書各裝為卷,書男女生辰。兩造固鄰舊,女生辰所素悉,偽為卷為證。士弘先問媒證:「乙得甲聘禮若干?行聘時有何客?」媒證出不意,妄舉以對。復問甲,所對各異。擘視卷軸,竹猶青,笑詰之曰:「若訂婚三載,卷軸竹色猶新,此非臨訟偽造者乎?」甲乃服罪。縣吏左梅伯有叔富而無子,梅伯糾賊劫殺之,獲賊而梅伯逃。士弘抵任,叔妻哭訴,陰跡梅伯匿安福勢宦家,故緩詞曰:「此舊事。前官不了,餘安能按之?」數月,梅伯歸,叔妻復訴,置不問。梅伯且出收叔遺產,叔妻號於庭曰:「公號廉明,今寬殺人者罪,且占寡婦田,何得為廉明!」陽怒,批其牘曰:「止問田土,不問人命。」梅伯益自得,赴縣訴理,乃笑謂曰:「候汝三載矣!」批其牘曰:「止問人命,不問田土。」梅伯遂伏法。其善斷獄多類此。考最,擢陜西甘州同知。復考最,擢江南常州知府。

吳三桂亂起,關隴震動,大吏疏請擢洮西道副使,未到官而洮、岷陷。邊外群番乘亂內犯,肆剽掠,調署甘山道。王輔臣叛,河東失守。士弘以兵集當謀帥,言於巡撫,謂:「恢復河東,非用河西兵不可;用河西兵,非責之提督張勇不可。」疏入,授勇靖逆將軍,節制諸鎮。復蘭州,士弘贊畫功為多。署甘肅按察使,按失守官吏罪,務平允。寧夏兵叛,殺提督陳福,調寧夏道。嚴守御,安反側,免衛所逋糧七萬五千石。康熙十六年,寇平,以功進布政使參議。母老乞歸,家居幾三十年。卒,年八十。

士弘備兵甘山時,取晉辛憲英語:「軍旅之間可以濟者,惟仁與恕。」因以名其堂。

多弘安,字君修,直隸阜城人。順治五年,選拔貢生。康熙初,授廣東靈山知縣。兵後荒殘,居無衙舍。弘安請免積年逋賦,招撫流移,捐給牛種,民得安耕稼。葺城垣,創學宮,繕官廨,捕除盜賊,靈山大治,士民刊石紀其績。七年,遷奉天承德知縣。旗、民抗法者,送部懲治,皆懾服。十年,擢陜西延安靖邊同知。十六年,補江南淮安山盱河務同知。時高堰長堤潰決,淮水注寶應、高郵,不復出清口敵黃。黃水直注裏河,運道淤淺,復隨淮入堰,無由會清口下雲梯關入海,近海口盡淤墊。弘安與河督靳輔籌策築高堰,束淮敵黃,治爛泥淺諸故道,導清水入里河,運道乃通。修築兩岸及河口清江大閘,與淮工相表裏。清河達雲梯關數百里,葭葦榛蕪,壅塞故道。用以水攻沙法,塞周橋、高澗諸閘,使清淮無旁洩,蓄全力攻積沙。十七年,大雨,淮盛漲,與黃並入海。治淮、治黃、治運,並收成效。十九年,擢淮安知府。二十年,擢淮揚道。二十四年,擢安徽按察使。時方議浚下河、治高堰。弘安入覲,疏陳:「高堰宜急治,無論下河開浚與否。治堰法,砌石先安地釘,湖底水深,費帑甚繁。如用板若掃,水勢蕩掣,尤易摧殘。惟密釘排椿,內實以碎石,庶可敵風浪,省金錢。十餘年後,黃河刷深,則湖、河水俱卑,高堰既固,下河亦漸就理。」二十八年,遷江西布政使,乞歸。後值黃、運兩河潰溢,起用弘安。會病卒,祀靈山名宦。

佟國聘,字君莘,奉天人。以廕生補吏部筆帖式。康熙十年,授江南碭山知縣,縣當黃河沖,研求治河方略。擢歸仁堤同知,調宿桃同知。擢貴州平遠知府,河督靳輔疏留任,十餘年倚如左右手。塞楊家莊、蕭家渡決口,建硃家堂、溫家廟二石壩,浚白洋引河九道,築黃河南、北兩岸堤,浚中河,靡役不從。久之,擢山東濟寧道副使。道地為漕運樞紐,恤夫役,減苛稅,除冗費,能舉其職。復調監督高堰工程。三十八年,卒於官。

王繻,字慎夫,河南睢州人。少學於湯斌。康熙二十五年,以歲貢生授直隸東明知縣。糧賦多欺隱,易甲長,大戶使族長督之,飛灑不行,流亡來歸。民間養官馬為累,力除之。撫盜魁,責以緝捕,盜絕跡。逃人誣攀良民,雪之。民有繼妻素淫,欲並亂前妻女,不從,戕之死。繻謂母道絕,當故殺妻前夫子律論斬,報可,因著為例。母憂去,服闋,補獲鹿。治驛有法,民不累於供億。內遷戶部員外郎,擢郎中。三十八年,出為江南糧儲道。道舊有倉規銀鉅萬,繻一擯勿取。將徵漕,扁舟行縣,懲其濫收者。至宜興,宜興民曰:「吾民四十年不見糧道,今飛來耶?」號曰「飛糧道」。道庫歲收銀八十五萬兩,為修船及弁丁運費。運丁預支行糧,例扣月息,丁益困,繻悉罷之。

四十年,擢江蘇按察使。治獄仁恕,多所平反。宿州生攜妻子出客授,妻兄女來視,居數日,妻子並中毒死,妻兄素有隙,疑其女置毒,告官,被刑誣服。繻疑之,問其室來往復何人,得十二歲學徒畏師嚴置鏚食中狀,事乃白。無錫民毆攻皮匠,匠死,僧與民仇,證為鬥毆殺。繻察鬥毆日月在保辜限外,詰曰:「傷重何不醫?」出醫方,則匠死於傷寒,僧乃服。上南巡,入覲,顧宋犖曰:「朕聞繻督糧時官聲甚好。」時繻已病,遣御醫視之,賜德里雅噶藥一器,溫旨慰諭,復賜御書。繻曰:「按察任大責重,臥治即辜恩。」引疾歸,年甫五十。久之,卒於家。

田呈瑞,字介璞,山西汾陽人。康熙中,仕為中書舍人。出襄南河事。有堤當水沖,曰:「此堤一壞,萬家其魚矣!土堤易修易敗,宜更以石。」家素豐,出私錢成之。以功擢大名道,未之任,調陜西臨洮道。遇饑治賑,策馬行郡縣山谷間,豪右胥吏不敢為奸弊。呈瑞念救荒無善策,於蘭州西石佛灣鑿渠,教民造水車,引以溉田,歲增粟十餘萬石,民為建生祠。調浙江金衢嚴道,署糧儲道,徵漕積弊盡洗滌之。值旱,冒暑省荒,感疾,乞歸不得。五十九年,卒於官。

張孟球,字夔石,江南長洲人。康熙二十四年進士,授山東昌樂知縣。入為工部主事。累遷禮部郎中。出督云南學政,父憂去,服闋,補福建糧驛道。駐防軍食取給於漕。上游四郡阻灘險。故事,徵解折色,官為採置,輒抑勒病商。孟球於延、建產米地平價購米,僦民船運省城,不假吏胥,諸弊盡絕。地多山嶺,官吏濫用驛夫,孟球禁革私冒。遇大徭,預期發雇值,終其任無擾驛者。

調河南糧儲道。河南漕糧,就衛輝水次收兌。舊無倉廒,又無額役,運船調之他省。天寒水涸,糧不時至,宿河干以待,遇雨雪則米濕霉變,又患盜竊。孟球始以羨餘建倉。署布政使。

西藏用兵,調河南馬騾萬,凡騾馬三需一夫,剋期兩月。孟球止宿郊外,躬自檢閱,西路近陜諸郡遣吏往督之,盡除需索留難諸弊。凡五十四日,馬驢如數遣赴軍,而民不擾。擢按察使。蘭陽民硃復業附白蓮教,自稱明裔,煽惑數縣。孟球檄杞縣知縣寧君佐馳往捕治,盡獲其黨。上命尚書張廷樞往按,從孟球議,誅其與逆謀者,愚民被誘悉釋之。淅川營兵博,知縣崔錫執而罪之,兵譁,執南陽知府沈淵,眾辱之,總兵高成不能治。時巡撫張聖佐坐譴,孟球護巡撫,曰:「南陽地連襄、鄖,急則鋌而走險,事未可知。」密令附近諸縣嚴守御,諭:「止誅首惡,自首免罪。」得倡亂者七人誅之,不數日而事定。

康熙末,乞歸,不復出。乾隆初,卒,年八十。

論曰:官監司卓卓有名氏,即平進至督撫,易耳。如紫綬等皆早退,遂以監司終。紫綬崎嶇兵間,捍偏隅為民保障;州佐、士弘勤勤重民事;弘安贊治河;繻善斷獄;孟球能應變:使得為督撫,其績效當有大於是者。時方承平,仕得行其意,知止知足,必有說以自處矣。


\end{pinyinscope}