\article{列傳七十五}

\begin{pinyinscope}
鄂爾泰弟鄂爾奇子鄂弼鄂寧

張廷玉子若靄若澄若渟從子若溎

鄂爾泰,字毅庵,西林覺羅氏,滿洲鑲藍旗人,世居汪欽。國初有屯泰者,以七村附太祖,授牛錄額真。子圖捫,事太宗,從戰大凌河,擊明將張理,陣沒,授備御世職。雍正初,祀昭忠祠。

鄂爾泰,其曾孫也。康熙三十八年舉人。四十二年,襲佐領,授三等侍衛。從聖祖獵,和詩稱旨。五十五年,遷內務府員外郎。世宗在籓邸,偶有所囑,鄂爾泰拒之。世宗即位,召曰:「汝為郎官拒皇子,其執法甚堅。」深慰諭之。雍正元年,充雲南鄉試考官,特擢江蘇布政使。於廨中建春風亭,禮致能文士,錄其詩文為南邦黎獻集。以應得公使銀買穀三萬三千四百石有奇,分貯蘇、松、常三府備賑貸。察太湖水利,擬疏下游吳淞、白茆,役未舉。

三年,遷廣西巡撫,甫上官,調雲南,以巡撫治總督事。貴州仲家苗為亂二十餘年,巡撫石禮哈、提督馬會伯請用兵,上未即許。巡撫何世璂疏言仲家苗藥箭銛利,地勢險阻,用兵不易,上即命世璂招撫,久未定,詔諮鄂爾泰。四年春,疏言:「雲、貴大患無如苗、蠻。欲安民必制夷,欲制夷必改土歸流。而苗疆多與鄰省相錯,即如東川、烏蒙、鎮雄,皆四川土府,東川距雲南四百餘里。去冬烏蒙攻掠東川,滇兵擊退,而川省令箭方至。烏蒙距雲南省城亦僅六百餘里,錢糧不過三百餘兩,取於下者百倍。一年四小派,三年一大派,小派計錢,大派計兩。土司娶子婦,土民三載不敢婚。土民被殺,親族尚出墊刀數十金,終身不見天日。東川雖已改流,尚為土目盤據,文武長寓省城,膏腴四百里無人敢墾。若改隸雲南,俾臣得相機改流,可設三府、一鎮。此事連四川者也。廣西土府、州、縣、峒、寨等一百五十餘員,分隸南寧、太平、思恩、慶遠四府。其為邊患,自泗城土府外,皆土目橫於土司。黔、粵以牂牁江為界,而粵屬西隆州與黔屬普安州越江互相鬥入。苗寨寥闊,將吏推諉。應以江北歸黔,江南歸粵,增州設營,形格勢禁。此事連廣西者也。滇邊西南界以瀾滄江,江外為車里、緬甸、老撾諸境,其江內鎮沅、威遠、元江、新平、普洱、茶山諸夷,巢穴深邃,出沒魯魁、哀牢間,無事近患腹心,有事遠通外國。論者謂江外宜土不宜流,江內宜流不宜土。此云南宜治之邊夷也。貴州土司向無鉗束群苗之責,苗患甚於土司。苗疆四圍幾三千餘里,千三百餘寨,古州踞其中,群寨環其外。左有清江可北達楚,右有都江可南通粵,蟠據梗隔,遂成化外。如欲開江路通黔、粵,非勒兵深入遍加剿撫不可。此貴州宜治之邊夷也。臣思前明流、土之分,原因煙瘴新疆,未習風土,故因地制宜,使之鄉導彈壓。今歷數百載,以夷治夷,即以盜治盜,苗、惈無追贓抵命之憂,土司無革職削地之罰。直至事上聞,行賄詳結,上司亦不深求,以為鎮靜,邊民無所控訴。若不剷蔓塞源,縱兵刑財賦事事整理,皆非治本。改流之法:計擒為上,兵剿次之;令其自首為上,勒獻次之。惟剿夷必練兵,練兵必選將。誠能賞罰嚴明,將士用命,先治內,後攘外,實邊防百世之利。」疏入,上深然之。

會石禮哈疏報遣兵擊破穀隆、長寨、者貢、羊城諸隘,擒其渠阿革、阿給及諸苗之從為亂者,上命交鄂爾泰按讞。五月,鄂爾泰遣兵三道入:一自穀隆,一自焦山,一自馬落孔。破三十六寨,降二十一寨,撫苗民五百餘戶、二千餘口,察出荒熟田地三萬畝。又以鎮遠土知府刁澣、霑益土知州安於籓素兇詐,計擒之;者樂甸土司刁聯鬥乞免死,改土歸流。鄂爾泰疏報仲家苗悉定。上嘉其成功速,令議敘。旋條上經理仲苗諸事,報可。十月,真除云貴總督。

四川烏蒙土司祿萬鍾為亂,侵東川。鄂爾泰請以東川改隸雲南,上從之。仍命會四川總督岳鍾琪按治,招其渠祿鼎坤出降。鄂爾泰令鼎坤招萬鍾,數往不就撫,乃檄總兵劉起元率師討之,破其所居寨。萬鍾走匿鎮雄土司隴慶侯所。五年,萬鍾詣鍾琪降,慶侯亦詣鍾琪請改土歸流。上命鍾琪以萬鍾、慶侯交鄂爾泰按讞。敘功,授世職拜他喇布勒哈番。三月,鎮沅惈刁如珍等戕官焚掠,遣兵討平之,獲如珍。泗城土知府岑映宸縱其眾出掠,又發兵屯者相,立七營。鄂爾泰疏劾,令諸道兵候檄進討,映宸乞免死存祀,改土歸流。鄂爾泰請映宸送浙江原籍,留其弟映翰奉祀。七月,發兵與湖北師會討定謬沖花苗,獲其渠,降其餘眾。威遠惈札鐵匠等、新平惈李百疊等應如珍為亂。九月,鄂爾泰檄臨元總兵孫宏本率師討之,獲札鐵匠,降李百疊。威遠、新平皆定。十一月,招降長寨後路苗百八十四寨,編戶口,定額賦。得旨嘉獎,進世職一等阿達哈哈番。十二月,攻破雲南惈窩泥種,取六茶山地千餘里,劃界建城,置官吏。

雲南南徼地與安南接,前總督高其倬疏言安南國界應屬內地者百二十里,請以賭咒河為界。安南國王黎維祹奏辯,上命鄂爾泰清察。鄂爾泰請與地八十里,於鉛廠山下小河內四十里立界,上從之,敕諭安南。六年,維祹表謝,上嘉其知禮,命復與四十里。旋討擒東川法戛土目祿天佑、則補土目祿世豪;按治米貼土目祿永孝,論斬。永孝妻陸氏結惈儸為亂,檄總兵張耀祖討之,攻克門坎山。師入,獲陸氏。米貼平。廣西八達寨儂顏光色等為亂,提督田畯不能討。鄂爾泰遣兵往,儂殺光色以降。上命鄂爾泰總督云、貴、廣西三省,發帑十萬犒師。旋又撫貴州拜克猛、長寨、古羊等生苗百四十五寨。十月,萬壽節,雲南卿雲見,鄂爾泰疏聞。

七年正月,命超授三等阿思哈尼哈番,雲、貴兩省巡撫、提督、總兵,文知縣、武千總以上,皆加級。三月,令按察使張廣泗率師攻貴州丹江雞溝生苗,破其寨,種人悉降。上下九股、清水江、古州諸地以次定。下部議敘,鄂爾泰疏辭,而乞予曾祖圖捫封典,俾昭忠祠位得改書贈官,列大臣之末,上允其請,仍命議敘。七月,招安順、高耀等寨生苗及儂、仲諸種人內附。十月,雲南趙州醴泉出,鄂爾泰疏聞。上褒鄂爾泰化民成俗,格天致瑞,尋加少保。八年五月,招黎平、都勻等寨生苗內附。鄂爾泰既討定群苗為亂者,諸土司懾軍威納土,疆理其地,置郡縣,設營汛,重定三省及四川界域,而諸土司世守其地,一旦歸版籍,其渠誅夷、遷徙皆無幸。

屬苗內憤奰,烏蒙惈最狡悍,總兵劉起元移鎮其地,恣為貪虐。六月,祿鼎坤及其族人鼎新、萬福遂糾眾攻城,劫殺起元及游擊江仁、知縣賽枝大等,盡戕其孥。鄂爾泰疏聞,請罷斥,上慰諭之。烏蒙既陷,江外涼山、下方、阿驢,江內巧家營、者家海諸寨及東川祿氏諸土目皆起而應之,又令則補、以址諸寨要截江路,以則、以擢諸寨窺伺城邑,東川境內挖泥、矣氏、歹補、阿汪諸寨,東川境外急羅箐、施魯、古牛、畢古諸寨,及武定、尋甸、威寧、鎮雄所屬諸夷,遠近響應,殺塘兵,劫糧運,堵要隘,毀橋樑,所在屯聚為亂。鄂爾泰集官兵萬數千人,土兵半之,分三路進攻:令總兵魏翥國攻東川;哈元生攻威寧,副將徐成貞副之;參將韓勛攻鎮雄。翥國師行,土目祿鼎明遣行刺,被創,以總兵官祿代將。師進,焚苗寨十三。遣游擊何元攻急羅箐,殺三百餘,降一百三十餘。游擊紀龍攻者家海,破寨,盡殲其眾。勛與苗兵遇於莫都,戰一晝夜,破寨四,殺數百人。進攻奎鄉,戰三日,殺二千餘。元生、成貞自威寧攻烏蒙,射殺其渠黑寡、暮末,連破寨八十餘,擊敗其眾數萬,遂克烏蒙。鄂爾泰檄提督張耀祖督諸軍分道窮搜屠殺,刳腸截脰,分懸崖樹間,群苗讋慄。上獎鄂爾泰及諸將,以元生、成貞、勛為功首,發帑犒師。隴慶侯庶母二祿氏、四川沙馬土婦沙氏以不從亂,給誥命,賚銀幣。於是苗疆復定。鄂爾泰令於雲、貴界上築橋,命曰庚戌橋,以年紀其績也。

是歲,永昌邊外孟連土司請歲納廠課六百,鶴慶邊外皦子請歲貢土物,鄂爾泰疏聞。上以邊外野夷向化,命減孟連廠課之半。皦子入貢,犒以鹽三百斤。九年,疏請重定烏蒙、鎮遠、東川、威寧營汛。別疏請興雲南水利,濬嵩明州楊林海,開墾周圍草塘,疏宜良、尋甸諸水,耕東川城北漫海,築浪穹羽河諸堤,修臨安諸處工,暨通粵河道,皆下部議行。十年,召拜保和殿大學士,兼兵部尚書,辦理軍機事務。敘定苗疆功,部議進世職一等精奇尼哈番,上特命授一等伯爵,世襲。

師討準噶爾,六月,命鄂爾泰督巡陜、甘,經略軍務。九月,師破敵額爾德尼昭,鄂爾泰檄大將軍張廣泗遣兵截袞塔馬哈戈壁,斷敵北遁道。尋疏請屯田。十一年六月,還京師。入對,言準部未可驟滅,用兵久,敝中國,無益,上頗然之。

十三年,臺拱苗復叛。上命設辦理苗疆事務處,以果親王、寶親王、和親王、鄂爾泰及大學士張廷玉等董其事。苗患日熾,焚掠黃平、施秉諸地。鄂爾泰以從前布置未協,引咎請罷斥,並削去伯爵。上曰:「國家錫命之恩,有功則受,無功則辭,古今通義。」允其請,予休沐,仍食俸。尋命留三等阿思哈尼哈番。

八月,世宗疾大漸,鄂爾泰仍以大學士與莊親王允祿,果親王允禮,大學士張廷玉,內大臣豐盛額、訥親、海望同被顧命。鄂爾泰與廷玉捧御筆密詔,命高宗為皇太子。俄,皇太子傳旨命鄂爾泰等輔政。世宗崩,宣遺詔以鄂爾泰志秉忠貞,才優經濟,命他日配享太廟。高宗即位,命總理事務,進一等精奇尼哈番。乾隆二年十一月,辭總理事務,授軍機大臣;又辭兼管兵部,上不許,加拜他喇布勒哈番,合為三等伯,賜號襄勤。迭主會試,充領侍衛內大臣、議政大臣、經筵講官。

四年,南河河道總督高斌請開新運口,河東河道總督白鍾山請復漳河故道,命鄂爾泰按視。尋加太保。七年,副都御史仲永檀以密奏留中事告鄂爾泰長子鄂容安,命王大臣會鞫,請奪鄂爾泰官逮問,上不許。十年,以疾乞解任。上慰留,加太傅。卒,命遵遺詔配享太廟,並祀賢良祠,賜祭葬,謚文端。二十年,內閣學士胡中藻以詩辭悖逆獲罪,中藻出鄂爾泰門下,鄂爾泰從子甘肅巡撫鄂昌與唱和,並坐譴。上追咎鄂爾泰植黨,命撤出賢良祠。

鄂爾泰弟鄂爾奇,康熙五十一年進士,改庶吉士,散館授編修。雍正中,四遷至侍郎,歷工、禮二部,署兵部。五年,擢戶部尚書,兼步軍統領。十一年,直隸總督李衛論劾壞法營私、紊制擾民諸狀,鞫實,當治罪,上推鄂爾泰恩,宥之。十三年,卒。

鄂爾泰子鄂容安,鄂實,鄂弼,鄂寧,鄂圻,鄂謨。鄂容安自有傳。鄂實與高天喜同傳。

鄂弼初授三等侍衛,遷正紅旗漢軍副都統。出為山西巡撫,調陜西,署西安將軍。擢四川總督,未上官,卒,賜祭葬,謚勤肅。

鄂寧,舉人,初授戶部筆帖式。屢以員外郎署副都統,復自郎中擢禮部侍郎。出為湖北巡撫,調湖南,再調雲南。師征緬甸,雲南總督楊應琚戰失利,鄂寧以實疏聞。明端代應琚,深入戰死。鄂寧劾參贊額勒登額、提督譚五格逗遛失機。上獎鄂寧,加內大臣銜,即命代明瑞為雲貴總督。尋以與參贊舒赫德合疏議撫失上指,奪內大臣銜,左授福建巡撫,迭降藍翎侍衛。卒。

張廷玉,字衡臣,安徽桐城人,大學士英次子。康熙三十九年進士,改庶吉士。散館授檢討,直南書房,以憂歸。服除,遷洗馬,歷庶子、侍講學士、內閣學士。五十九年,授刑部侍郎。山東鹽販王美公等糾眾倡邪教,巡撫李樹德令捕治,得百五十餘人。上命廷玉與都統託賴、學士登德會勘,戮七人、戍三十五人而讞定。旋調吏部。

世宗即位,命與翰林院學士阿克敦、勵廷儀應奉幾筵祭告文字,賜廕生視一品,擢禮部尚書。雍正元年,復命直南書房。偕左都御史硃軾充順天鄉試考官,上嘉其公慎,加太子太保。尋兼翰林院掌院學士,調戶部。疏言:「浙江衢州,江西廣信、贛州,毗連閩、粵,無藉之徒流徙失業,入山種麻,結棚以居,號曰『棚民』。歲月既久,生息日繁。其強悍者,輒出剽掠。請敕督撫慎選廉能州縣,嚴加約束。其有讀書向學,膂力技勇,察明考驗錄用,庶生聚教訓,初無歧視。」下督撫議行。命署大學士事。四年,授文淵閣大學士,仍兼戶部尚書、翰林院掌院學士。五年,進文華殿大學士。六年,進保和殿大學士,兼吏部尚書。七年,加少保。

八年,上以西北用兵,命設軍機房隆宗門內,以怡親王允祥、廷玉及大學士蔣廷錫領其事。嗣改稱辦理軍機處。廷玉定規制:諸臣陳奏,常事用疏,自通政司上,下內閣擬旨;要事用摺,自奏事處上,下軍機處擬旨,親御硃筆批發。自是內閣權移於軍機處,大學士必充軍機大臣,始得預政事,日必召入對,承旨,平章政事,參與機密。

廷玉周敏勤慎,尤為上所倚。上偶有疾,獎廷玉等翊贊功,各予一等阿達哈哈番,世襲。廷玉請以子編修若靄承襲。十一年,疏言:「諸行省例,凡罪人重者收禁,輕者取保。獨刑部不論事大小、人首從,皆收禁,累無辜。請如諸行省例,得分別取保。刑部引律例,往往刪截,但用數語,即承以所斷罪;甚有求其仿彿,比照定議者:高下其手,率由此起。請敕都察院、大理寺駁正;扶同草率,並予處分。」命九卿議行。大學士英祀京師賢良祠,復即本籍諭祭,命廷玉歸行禮,並令子若靄從;弟廷璐督江蘇學政,亦命來會。發帑金萬為英建祠,並賜冠帶、衣裘及貂皮、人參、內府書籍五十二種。十二月,廷玉疏言:「行經直隸,被水諸縣已予賑,尚有積潦不能種麥,請敕加賑一月。」並議以工代賑。得旨允行。十二年二月,還京師,上遣內大臣、侍郎海望迎勞盧溝橋,賜酒膳。十三年,世宗疾大漸,與大學士鄂爾泰等同被顧命。遺詔以廷玉器量純全,抒誠供職,命他日配享太廟。高宗即位,命總理事務,予世職一等阿達哈哈番,合為三等子,仍以若靄襲。

乾隆元年,明史成,表進,命仍兼管翰林院事。二年十一月,辭總理事務,加拜他喇布勒哈番,特命與鄂爾泰同進三等伯,賜號勤宣,仍以若靄襲。四年,加太保。尋諭:「本朝文臣無爵至侯伯者,廷玉為例外,命自兼,不必令若靄襲。」又諭:「廷玉年已過七十,不必向早入朝,炎暑風雪無強入。」十一年,若靄卒。上以廷玉入內廷須扶掖,命次子庶吉士若澄直南書房。十三年,以老病乞休。上諭曰:「卿受兩朝厚恩,且奉皇考遺命配享太廟,豈有從祀元臣歸田終老?」廷玉言:「宋、明配享諸臣亦有乞休得請者。且七十懸車,古今通義。」上曰:「不然。易稱見幾而作,非所論於國家關休戚、視君臣為一體者。使七十必令懸車,何以尚有八十杖朝之典?武侯鞠躬盡瘁,又何為耶?」廷玉又言:「亮受任軍旅,臣幸得優游太平,未可同日而語。」上曰:「是又不然。皋、夔、龍、比易地皆然。既以身任天下之重,則不以艱鉅自諉,亦豈得以承平自逸?朕為卿思之,不獨受皇祖、皇考優渥之恩,不可言去;即以朕十餘年眷待,亦不當言去。朕且不忍令卿去,卿顧能辭朕去耶?朕謂致仕之義,必古人遭逢不偶,不得已之苦衷。為人臣者,設預存此心,必將漠視一切,泛泛如秦、越,年至則奉身以退,誰復出力為國家治事?是不可以不辨。」因命舉所諭宣告朝列,並允廷玉解兼管吏部,廷玉自是不敢言去。然廷玉實老病,十四年正月,命如宋文彥博十日一至都堂議事,四五日一入內廷備顧問。是冬,廷玉乞休沐養痾,上命解所兼領監修、總裁諸職,且令軍機大臣往省。廷玉言:「受上恩不敢言去,私意原得暫歸。後年,上南巡,當於江寧迎駕。」上乃許廷玉致仕,命待來春冰泮,舟行歸里。親制詩三章以賜,廷玉入謝,奏言:「蒙世宗遺命配享太廟,上年奉恩諭,從祀元臣不宜歸田終老,恐身後不獲更蒙大典。免冠叩首,乞上一言為券。」上意不懌,然猶為頒手詔,申世宗成命,並制詩示意,以明劉基乞休後仍配享為例。次日,遣子若澄入謝。上以廷玉不親至,遂發怒,命降旨詰責。軍機大臣傅恆、汪由敦承旨,由敦為乞恩,旨未下。又次日,廷玉入謝,上責由敦漏言,降旨切責。廷臣請奪廷玉官爵,罷配享。上命削伯爵,以大學士原銜休致,仍許配享。十五年二月,皇長子定安親王薨,方初祭,廷玉即請南還,上愈怒,命以太廟配享諸臣名示廷玉,命自審應否配享。廷玉惶懼,疏請罷配享治罪。上用大學士九卿議,罷廷玉配享,仍免治罪。又以四川學政編修硃荃坐罪,荃為廷玉姻家,嘗薦舉,上以責廷玉,命盡繳歷年頒賜諸物。二十年三月,卒,命仍遵世宗遺詔,配享太廟,賜祭葬,謚文和。

乾隆三年,上將臨雍視學,舉古禮三老五更,諮鄂爾泰及廷玉。廷玉謂無足當此者,撰議以為不可行。四十三年,上撰三老五更說,闢古說踳駁,命勒碑闢雍。五十年,復見廷玉議,以所論與上同,命勒碑其次,並題其後,謂「廷玉有此卓識,乃未見及。朕必遵皇考遺旨,令其配享。古所謂老而戒得,朕以廷玉之戒為戒,且為廷玉惜之。」終清世,漢大臣配享太廟,惟廷玉一人而已。

子若靄,字晴嵐。雍正十一年進士。廷試,世宗親定一甲三名。拆卷知為廷玉子,遣內侍就直廬宣諭。廷玉堅辭,乃改二甲一名,授編修,直南書房,充軍機章京。乾隆間,屢遷至內閣學士。若靄工書畫,內直御府所藏,令題品鑒別,詣益進。十一年,扈上西巡,感疾,歸卒。

若澄,字鏡壑。乾隆十年進士,改庶吉士,直南書房,累遷至內閣學士。卒。若澄亦工畫,亞若靄。

若渟,字聖泉。入貲授刑部主事,充軍機章京,再遷郎中。出為雲南澂江知府、四川建昌道。內擢太僕少卿,五遷至侍郎,歷工、刑、戶諸部。嘉慶五年,授兵部尚書,調刑部。七年,卒,贈太子少保,賜祭葬,謚勤恪。

從子若溎,字樹穀。雍正八年進士,授兵部主事。考選江西道御史。擢鴻臚寺少卿,六遷刑部侍郎,擢左都御史。上命旌恤勝朝殉節諸臣,若溎請遍行採訪。下大學士、九卿議,以為明史外兼採各省通志,專謚、通謚已至千五六百人,不必更行採訪。若溎以老乞休。上南巡,屢迎謁。五十年,與千叟宴,御書榜以賜。歸,又二年,卒。

論曰:世宗初即位,擢鄂爾泰於郎署,不數年至總督。廷玉已貳禮部,內直稱旨,不數年遂大拜。軍機處初設,職制皆廷玉所定。鄂爾泰稍後,委寄與相埒。庶政修舉,宇內乂安,遂乃受遺命,侑大烝,可謂極心膂股肱之重矣。顧以在政地久,兩家子弟賓客,漸且競權勢、角門戶,高宗燭幾摧萌,不使成朋黨之禍,非二臣之幸歟?


\end{pinyinscope}