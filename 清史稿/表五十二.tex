\article{表五十二}

\begin{pinyinscope}
交聘年表一(中國遣駐使)

交聘之典,春秋為盛。南北史本紀書交聘頗詳。其時中土分裂,與列國之敵體相交,無以異也。宋與遼、金,歲賀正旦、賀生辰外,有泛使,今謂之專使。然皆事畢即行,不常駐。金史始有盟聘表。清有中夏,沿元、明制,視海內外莫與為對。凡俄、英之來聘者,國非一,而遣使駐?史皆書曰「來貢」。洎道光庚子訂約,始與敵體相等。咸豐庚申之役,肇京,未允實行者,亦一大端。自是而後,有約各國率遣使駐京。同治中,志剛、孫家穀之出,是為中國遣專使之始。光緒建元,郭嵩燾、陳蘭彬諸人分使英、美,是為中國遣駐使之始。其時以使俄者兼德、奧,使英者兼法、義、比,使美者兼日斯巴尼亞、秘魯,而日本無附近之國,則特置使。甲午以後,增置漸多,迄於宣統,俄、英、法、德、和、比、義、奧、日本皆特置使,日斯巴尼亞則改以法使兼,秘魯、墨西哥、古巴則以美使兼。韓國置使旋廢。有約之國,惟葡萄牙、瑞典、那威、丹馬諸國無駐使,有事則以就近駐使任之。國際交涉,大至和戰之重,細至節文之末,為使者罔弗與聞,關國家休戚者固至重也。作交聘表。

表略


\end{pinyinscope}