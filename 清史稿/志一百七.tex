\article{志一百七}

\begin{pinyinscope}
○兵三

△防軍陸軍

防軍初皆召募,於八旗、綠營以外,別自成營,兵數多寡不定,分布郡縣,遇寇警則隸於專征將帥,二百年間,調發徵戍,咸出於此。若乾隆年臺灣之役,乾、嘉間黔、楚征苗之役,嘉慶間川、陜教匪之役,道光年洋艘征撫之役,皆暫募勇營,事平旋撤。故嘉慶七年,楚北初設提督,即以勇丁充補標兵,道光十七年,以練勇隸於鎮筸鎮標,二十三年,以防守海疆之水陸義勇三萬六千人仍遣回本籍,無防、練軍之名也。道、咸間,粵匪事起,各省多募勇自衛,張國樑募潮州勇丁最多。咸豐二年,命曾國籓治湖南練勇,定湘軍營哨之制,為防軍營制所昉。迨國籓奉命東征,湘勇外益以淮勇,多至二百營。左宗棠平西陲,所部楚軍亦百數十營。軍事甫定,各省險要,悉以勇營留防,舊日綠營,遂同虛設。綠營兵月餉不及防勇四分之一,升擢擁滯,咸辭兵就勇。粵、捻既平,左宗棠諸臣建議,防營誠為勁旅,有事則兵不如勇,無事則分汛巡守,宜以制兵為練兵,而於直隸、江、淮南北扼要之處,留勇營屯駐,遂有防軍之稱。

練軍始自咸豐間,以勇營日多,屢令統兵大臣以勇補兵額,而以餘勇備緩急,尚無別練之師。至同治元年,始令各疆吏以練勇人數口糧,悉數報部稽核。是年於天津創練洋槍隊。二年,以直隸額兵酌改練軍。四年,兵部、戶部諸臣會議選練直隸六軍,始定練軍之名。各省練軍乃踵行之。練軍雖在額設制兵內選擇,而營哨餉章,悉準湘、淮軍制,與防軍同。其綠營制兵,分布列郡汛地,練軍則屯聚於通都重鎮,簡器械,勤訓練,以散為整,重在屯防要地,其用亦與防軍同,故練軍亦防軍也。

同治、光緒間,各省所增編防、練軍,兵部、戶部於光緒二十四年核其總數,直隸練軍一萬一千人,留防淮軍三萬一千人,新軍一萬一千四百人,毅軍一萬人,奉天練軍一萬一千四百人,吉林防軍八千五百九十八人,練軍四千四百三十八人,黑龍江練軍七千九百七十一人,山西練軍四千九百人,河南防軍九千一百九十人,陜西防、練軍一萬四千四百五十人,甘肅防軍一萬二千五百人,新疆防軍二萬七千八百四十五人,塔爾巴哈臺勇營二千四百三十二人,四川勇營一萬五千六百九十八人,雲南防軍一萬五千三十三人,貴州練軍九千四百八十六人,廣東勇營一萬一千八百人,廣西勇營一萬六千九百四十人,湖南練軍一萬二千九百七十人,湖北勇營一萬二千六百九十人,新軍一千九十三人,江西防軍九千三百六十三人,安徽防、練軍一萬一千二百九十人,江蘇防軍二萬三千七百九十人,自強軍三千一百七十人,得勝軍三千人,浙江防軍二萬一千三百人,山東防軍一萬三千九百五十人,福建防軍一萬五百四十人,各省防軍、練勇凡三十六萬餘人,歲需餉銀二千餘萬兩。其後綠營兵屢加裁汰,各省衛戍之責,遂專屬於防、練軍。光緒中葉後,防、練軍改為巡防隊。光、宣之間,又改為陸軍。至宣統三年,各省巡防隊猶未裁盡也。茲列同、光、宣三朝改設防、練軍規畫於篇,而以陸軍新制附焉。

防軍,同治元年,直隸省於大沽協標六營內選練五百人,復增至二千五百人,分為五營,營分十隊,設總統一人,翼長二人,各營管帶一人,副管帶二人,正副令官二人,帶隊官十人,分隊官二十人。沈葆楨於江西省額兵一萬二千人內,嚴汰老弱,增補精銳,分為二班,一班調至省城操練,一班留防汛地,半年換班。其赴操者,酌加練費,較募勇之費不及其半,練成即調赴前敵助戰。

二年,劉長佑以直隸省營務積年廢弛,各營兵數多寡懸殊,號令不一,乃改仿湘軍成規,以五百人為一營,設營官、哨隊官及親兵,分別隊伍旂幟,申明號令,改設六軍,凡築營結陣諸法,一律講求。其步隊營制,設營官一人,哨官四人,哨長五人,什長四十人,正兵三百六十人,營官親兵五十人,哨官護兵四十人,營官自率中哨,凡五百人。其馬隊營制,設營官一人,幫辦二人,督隊官五人,每哨五棚,每棚什長一人,正兵九人,營官自率中哨,合伙兵、馬夫凡三百十六人。保定練軍,馬、步、守兵一千九百五十人為一軍,宣化練軍,一千四百八十人為一軍,古北口練軍,二千四百十人為一軍,大名練軍,一千二百三十四人為一軍,正定練軍,一千四百八十人為一軍,通永練軍,一千七百五十四人為一軍,共編為六軍。

五年,令遵化等處各駐防軍,每軍定為步隊二千人,馬隊五百人,在督標、提標內選取,凡一萬五千人,分為六軍,頒練兵章程十七條,隸總督節制,以防畿輔。又於六軍外續練防勇二軍。以奉天留防隊伍調補直隸練軍缺額。其訓練京營,由神機營量增兵額。是年,左宗棠以福建省綠營額冘餉薄,乃裁兵十成之四,即以裁餉加留營之兵,並營操練。

六年,丁寶楨於山東省增練馬隊三千人。

七年,以各省綠營日益孱弱,令各省以壯健練勇易之。令曾國籓經理直隸省練兵事宜,就全省綠營內抽練六千人,仿勇營規制,分地巡防。海防議起,調駐天津,分中、左、右、前、後五營,與勇營相犄角。

八年,曾國籓以軍事既竣,宜練兵不宜練勇,而勇營良法為練軍所當參用者,一、文法宜簡,一、事權宜專,一、情意宜洽。減兵增餉,汰弱留強,嚴杜頂替之弊。於原有練軍四千人外,古北口、正定、保定各練千人,統以東南戰將。練成之後,分為四軍。以二軍駐京北,二軍駐京南,俟功效既著,增練五千人。全省防營於未撤之九營外,以劉銘傳全部淮軍駐防張秋,以督標親軍砲隊營及前營副營駐天津,以親軍砲隊營駐大沽砲臺,以盛字中軍六營、左軍三營,仁軍二營,馬隊五營駐馬廠、青縣,於運河西岸築砲臺五座,駐盛字前軍三營、右軍三營、老左軍一營,於滄洲駐樂字中、左各一營,其盛字營兼辦屯田,以衛畿輔。是年,丁日昌以江蘇省自淮軍全部撤防以後,江蘇撫標兵僅有一千六百餘人,乃裁汰老弱,補以勇丁,分左右二營,練習洋槍及開花砲諸技。馬新貽以江南全省額兵一萬二千七百餘人,分防各處,徒有其名,必須化散為整,始能轉弱為強,乃於督標內選千人為左右營,浦口、瓜洲營內選五百人為中營,揚州、泰州營內選五百人為前營,駐省城訓練,於徐州鎮標內選千人為徐防新兵左右營,以地方之輕重,定練兵之多寡。劉錦棠以新疆全境自回民亂後,旗營零落殆盡,乃於烏魯木齊創設標兵,於天山南北路各置額兵,新疆所有駐防旗兵,歸並伊犁整頓,別以精騎重兵居中屯駐,為南北各路策應之師。崇實以四川省軍事漸定,酌裁防軍,選練旗、綠各營。

九年,曾國籓於直隸省增募馬勇千人,分為四營,原有額兵,增足萬人,分練馬隊、步隊,奏定各營哨之制,及底餉、練餉、出征加餉之制,為北方重鎮。

十年,鮑源深以山西省撫標兵仿曾國籓直隸練兵之法,選練馬隊一營,步隊二營,以次推行各鎮。吳棠以四川全省額兵類多疲弱,乃歸並訓練,得精壯萬人。王文韶以苗疆戡定,所有湖南省留防軍三十營,分布於湖南、貴州接壤之區,又於撫標、提標內各選練精壯一營。

十二年,令陜甘督臣左宗棠、雲貴督臣岑毓英各選所部勇丁,以補營兵之額。是時中外臣工皆注意練兵。李宗羲謂勇與兵有主客聚散勤惰之異,未可易勇為兵。王凱泰謂各省練兵,宜令更番換防,雲、貴蕩平以後,兩省制兵亦宜換防調操,以杜久駐疲惰之漸。兵部諸臣會議,以同治初年創議練兵,京師神機營及直隸省六軍,別籌練餉,特立營制。福建、浙江、廣東、江蘇等省,皆就所減之餉加於練軍。河南、山西、山東、湖南等省,則按直隸之法,於額兵內抽練,於正餉外略加練費。甘肅省則因軍事初定,先練千五百人。但各省所抽撥之兵,不過原額十之二三。若其餘之兵,置之不問,終成疲弱。應令各省統兵大臣,已練之兵,以時休息,其未練者,次第調操,期通省額兵咸成勁旅。

十三年,都興阿於奉天各城額兵內選練馬隊二千人,於各城八旗內選蘇拉千人為餘兵,俟客兵裁撤,再行增練。

光緒二年,崇實因奉天換防旗兵日久弊生,乃於岫巖、熊岳、大孤山、青堆子等處改設練軍。

三年,允李慶翱之議,於河南省增設練軍步隊。

六年,令各疆臣酌量裁兵。各省防軍自裁撤後,為數尚多。直隸、陜、甘須辦邊防,雲南、貴州則防軍較少,此外各省,均應大加裁汰。水師自設兵輪船後,舊式戰船水師,亦分別去留。旋廣西撫臣慶裕以廣西省兵單餉薄,乃酌裁防軍,以所節之餉,仿直隸練兵章程,在省標、提標內各選練二營,左右江兩鎮各選練一營。岐元以奉天省自同治間馬賊四出肆擾,先後商調客軍,增練旗、綠各營,而營制餉章未能畫一。光緒五年,乃以直隸客軍歸並奉天省,合槍砲馬步各隊,釐定營制,編為奉字中、左、右、前、後馬步隊五營,中軍增步隊一營。丁寶楨因四川省自軍興以後,川勇而外,益以湖南、貴州各軍,多至六萬餘人,事定次第裁並,至光緒三年,實存防軍一萬餘人,須分守要隘,未可再裁。貴州防軍,較他省為少,李明墀於光緒五年後,陸續裁汰四千餘人。李瀚章以湖北省防軍,若升字三營、忠義八營、武毅七營、水師七營,皆扼要駐守,不宜裁汰,就湖北通省額兵酌量裁去三千餘人。裕祿以安徽省自捻寇平後,駐防皖南、皖北各軍,凡一萬八千餘人,次第歸並訓練,實存水陸防軍萬人。

七年,岑毓英因苗亂已平,貴州之屯軍、防勇,量為裁並,屯軍裁去九千人,以裁軍補額兵,酌改練軍。旋移撫福建,乃率貴州練軍二千人赴閩,教練閩省制兵。譚鍾麟以浙江省防軍於光緒六年募足三十營,旋裁去四營,以練軍十營駐溫州,海門、省垣各一營,餘皆歸守汛地。是年,以各省防軍歲餉甚鉅,令統兵大臣一律嚴核,不得有吞蝕空額諸弊。

八年,崇綺裁並奉天各軍,於八旗捷勝營及東邊道標兵、蒙古練勇外,所有馬步營中之南方防勇,遷地勿良,乃裁並為一營,餘悉遣歸原省。任道鎔於山東省撫標及兗、曹鎮標內抽調步兵千二百人,分為三營,加餉訓練。張曜、劉錦棠以伊犁收復,就關外營勇選練制兵,改行餉為坐糧,略更舊制,增馬隊重火器,設游擊之師,復參用屯田法,以足軍食。

九年,張之洞練山西省軍隊,由省標先練,掃除積習,為全省軍營模範。李鴻章裁撤直隸省防軍,除裁撤外,實存直字、榮字、義勝各營數千人,與淮軍之親兵及仁軍、盛軍、銘軍、楚軍等馬、步、水師三十九營,分防各地。岑毓英以貴州苗疆多事,原設重兵數逾三萬,積久廢弛,專恃防軍定亂,事定後,以防軍歸入制兵。雲南省制兵,凡戰兵九千餘人,守兵七千餘人,塘汛堆卡,零星散布,而巡防緝捕,專任練軍,乃以戰兵屯聚於統將駐所,隨時整飭。潘霨裁並江西省防軍,實存七千八百餘人,每哨續裁十餘人,量為省並。曾國荃綜核廣東省募兵之數,於光緒六年,張之洞曾募沙民千人守虎門,楊玉科增募千人及惠清營五百人,鄭紹忠募安勇二千人,八年,募勁勇千人駐欽州,鄧安邦續募千人,散布廣州各屬,其廣東額兵實存九千餘人。

十年,奎斌裁汰山西省兩鎮兵三千餘人,挑練大同鎮馬步隊各一營,太原鎮步隊一營。

十一年,卞寶第裁湖南省綠營,選精壯為練軍,給以雙餉,其未足之額,以營勇補之。希元等抽撥吉林防軍左右路馬步營千五百人,又於未練之兵及八旗臺站西丹內選三千人,編為吉字營,分左右二翼,修築壁壘,歸營訓練。岑毓英以雲南省沿邊之防軍一萬六千人分編三十營,於每年瘴消之際,親歷邊疆,巡視防務。卞寶第分湖南全額兵之半,加以訓練,編為巡防營。

十二年,劉秉璋以四川省防營漸染習氣,所有壽字、武字等十營,巡鹽五營,一律選練整飭。

十三年,穆圖善整理東三省練兵事宜,每省挑練馬隊二起,步隊八營,奉天、吉林、黑龍江各足成四千五百人,以克魯伯砲六十尊,分配三省防營。剛毅裁並山西省額兵六千人,就餉練兵,撫標馬隊一旗,步隊三營,太原鎮馬隊二旗,步隊四營,大同鎮馬隊七旗,步隊二營,編列成軍,其北路則以樹字各營分地巡防。

十四年,岑毓英就云南省內地防軍及邊關勇營內共選練九千六百餘人,以符通省戰兵五成之數。而邊境遼闊,分防尚屬不敷,乃增練三十營,凡一萬五千四百餘人,分防騰越、蒙自各邊及大理、普洱各府。

十五年,譚鈞培更定雲南省營制。雲南防軍,於光緒二年,劉長佑挑練戰兵,以三百七十人編為一營。十年,岑毓英以督師出關,改編二百二十人為一小營,營分五哨,哨各四隊,隊各十人。十一年,合練軍各營,以半防內地,半防邊境,仍以二百人上下為一小營。凡調防八成戰兵七十七營,留防粵勇十二營,惈黑防勇六營,西南土防二十五營。乃裁汰三成,歸並整齊,以三營為一營,每營分編五哨,中哨六隊,餘各三隊,以散合整。凡戰兵二十六營,粵勇五營,惈黑勇二營,土勇十三營。

十六年,張曜練山東省步隊一營。

十七年,福潤增練步隊左營。鹿傳霖以陜西省自經亂後,兵制未復,乃酌留馬步防軍並練軍各營,居中策應,各路馬隊,利於巡緝,乃改步隊為馬隊以節餉糈,凡防、練軍馬隊千五百人,在平原及北山扼要駐守。張煦以湖南省自湘勇回籍後,專恃防軍彈壓各路,凡防軍萬人,水勇二千四百餘人,乃歸並損益,互為聲援。

二十一年,依克唐阿編定奉天省砲兵三哨,合原有之防軍為五營,又以效力獵戶二千人編為四營。是年張之洞創練自強軍十三營於江南,器械訓練,悉仿歐洲。

二十二年,張之洞練洋操隊二營於湖北。聶士成於直隸駐防淮軍內選練馬步隊三十營,仿德國營制操法,編為武毅軍。

二十三年,張之洞以練軍重在操演,令分防各營,以十之一更番來省,教以新操,俟練成後,轉授各營。

二十四年,王毓藻練貴州軍隊,先就省防三營改習洋操,次第推及各營。王文韶挑留直隸全省淮、練各軍二萬餘人,編為二十營,分左右翼,駐守大沽口及山海關,以練軍三十三營分防內地及熱河等處。色楞額以熱河兼轄蒙古兩盟十七旗,而馬步防兵僅有千人,乃增練壯丁五百人為一營,馬隊五百人為二營,佐以砲隊百人。增祺以福建省多山,新練防軍,宜重步隊,參以砲隊,增制過山快砲十二尊。胡聘之以東、直、秦、豫各省皆有防軍,支餉自數十萬至百萬不等,而山西省屏蔽畿疆,僅有練軍五千人,乃增練新軍,固西路之防。榮祿因北洋四大軍訓練已成,分路駐防,以武毅軍駐蘆臺為前軍,甘軍駐薊州為後軍,毅軍駐山海關為左軍,新建軍駐小站為右軍,別練萬人駐南苑為中軍,軍械不足,令江南機器局撥解新式快槍三千枝,快砲七尊,原有之淮軍一萬二千人,防、練軍一萬九千人,歸並訓練。劉坤一以江南省之江寧、鎮江、吳淞、江陰、徐州五路防軍悉改習洋操,所用軍械,統歸一律。是年,令王大臣選京師神機營馬步萬人為選鋒營。令北方各省營伍,由新建軍遣員教習,南方各省營伍,由自強軍遣員教習。東三省防練各營伍,由北洋武備學堂遣人教習。

二十五年,李秉衡上言奉天仁、育二軍,訓練已成,應擇地修築營壘,俾成重鎮。裕祿以直隸防、練各軍為數太多,乃挑留馬步精兵一萬八百餘人,編為練軍步隊十二營,馬隊二十營,更定營制,步隊以三百人為一營,馬隊以二百餘人為一營,凡三十二營,分為直隸練軍左右翼,以通永鎮總兵統左翼,天津鎮總兵統右翼,其新建等軍,仍與宋慶之二十五營各守原防。劉樹堂以浙江防軍雲字、吉字、勝字、旅字各營凡十一營二十三旗,並為五軍,名為兩浙新軍,用北洋武毅軍操法訓練。松壽以江西省防軍有忠新等營二千餘人,內江及贛防水師二千四百餘人,武威等營旗三千餘人,分布各路,乃在省城設全省營務處,為訓練各軍之總匯。劉坤一以江南各軍歸並為三十七營,加以新法教練,漸有成效。文興以盛京八旗制兵,汰弱留強,仿北洋練軍新法教練。裕祥就四川駐防旗兵內選精銳為一營,陣法營制,與防軍一式。松壽以江西省新練防軍三千人,撥解南北洋新式槍砲,以資操練。黃槐森選廣西省各軍,先就省標、提標及左右江各營挑練一千四百人,為各軍模範。廖壽豐以浙江省寧波、鎮海各營次第改習洋操,省防各軍先練步隊三哨,砲隊一哨,凡標營及防、練軍,俟四哨教成,更番改練,推及全省。

二十六年,端方以陜西新練洋操之馬步十三旗,分防南北山隘。是年,令各省疆臣嚴定將弁貪墨之刑,並整理浙江省防營積弊。

二十七年,李興銳以江西防軍人數不一,乃分為五路,釐定人數,以中軍為常備軍,前、後、左、右軍為續備軍,軍各五營,營各五哨。劉坤一以江南武衛先鋒軍、江勝軍各二千人為常備左右軍,其餘防軍四十餘營悉編為續備軍。岑春煊以山西省兵制紛歧,有練軍、防軍、晉威軍之判,乃仿北洋武衛軍制,以省標三千人分左右翼為常備軍,以太原、大同二鎮兵共練三千人為續備軍。魏光燾以雲南省防軍二十四營,營各二百五十人,改編為常備軍十二營,營各三百人,舊有練軍改為續備軍,均練習洋操。丁振鐸於廣西省防軍三營內選千人為常備軍,各屬防軍,就人數多寡,練一、二隊不等。鄧華熙以貴州防軍及威遠營並練五營,凡千五百人,為常備軍,東西路練軍及緝捕營共二十九營,選練五千七百人,為續備軍,分防各隘。是年,設軍政司於天津,總司直隸省淮、練各防軍操防事宜。

二十八年,升允以陜西省新舊各軍均已改習洋操,乃選精銳六旗為常備新軍,其忠靖八旗兩翼步隊,武威兩翼馬隊,改為步隊十二旗,以六旗為續備防軍,六旗為續備長軍,防軍有地方之責,長軍為開荒之需,以馬隊砲隊佐之。

二十九年,夏時以江西省新軍僅有千二百人,江防重要,殊苦不足,九江為全省門戶,乃別募一軍,亦為常備軍,合中、前常備兩軍共十營,專防省城及九江二處,以左、右、後續備三軍分防各地。

三十年,曹鴻勛以貴州各軍於光緒二十六年改編為常備軍、續備軍,共二十四營,嗣因沿邊戒嚴,增募防勇十九營,而籌餉艱難,遂每營酌減人數,凡防、練軍及親兵減存一萬五百餘人,次第改習洋操。潘效蘇於新疆標、防、巡、練各軍三萬二千餘人內,選存正勇一萬三千餘人,於南北各路勺配分防。

三十一年,練兵處王大臣以山東省武衛先鋒隊二十營分防散漫,令擇地屯駐,增募成鎮。是年,命鐵良校閱江蘇、安徽、江西、湖北各省防軍、練軍、陸軍、旗兵、巡警兵。鐵良遍閱各軍,大都軍械不一,操法亦未盡嫺,舊營改練,進步甚遲。惟安徽練軍二隊,九江常備五營,湖北二鎮,較為生色。

三十三年,張之洞以沿江督捕營、下游緝匪營改編為水陸巡緝隊。王士珍以江北巡防隊改為步隊六營,馬隊二營,其餘淮海水師、練軍衛隊,悉仍其舊。錫良以雲南防軍二十七營,鐵路巡防十一營,土勇一營,凡三十九營,次第改編新軍,以全省防軍每營二百五十人為定額,分南防、西防、普防、江防、鐵路巡防為五路,凡四十七營。

宣統元年,以熱河巡防強勝營改編常備軍,以察哈爾原有之精壯、精健等營改編為巡防馬隊一營,步隊二營。徐世昌以奉天巡防隊分駐五路剿匪,旋合編為步隊一標,其河防營亦一律改編。王士珍因江南防軍步隊六營、砲隊二營改隸江北,乃合原有之巡防隊及留防各營編為巡防第七營,共巡防步隊八營,以備練成一鎮,原有衛隊,增募一哨,編為一營,尚有練軍三百人,水師十棚,均改為巡防隊。沈秉堃以雲南防軍內有各屬之保衛隊,系舊日團營,名為營隊,實即鄉團,未能遽改為巡防隊。廣福以伊犁軍標漢隊,系金順西征營勇之舊,其營制餉章,均仿湘軍,乃遵新章,以步隊一營、馬隊二旗為左路巡防隊,馬隊二旗為右路巡防隊,分駐惠遠、惠寧各城。袁樹勛以山東省原駐淮軍,於光緒二十四年移防長江,新增防兵二營駐兗、沂二府及德州,均當南北要道,未能遽裁。聯魁以新疆籌餉維艱,就原有防營改編為步隊三營、馬隊二營,又增編工程兵一隊,馬隊一營,勉成一協。寶棻以山西省軍隊,向分太原、大同、口外三大支巡防隊,乃歸並分編為中、前、後三路,各以統領節制之,凡馬步二十二隊。吳重熹以河南省巡防營不合部章,就通省巡防步隊二十八營、馬隊十二營分為五路,豫正左軍為中路,南陽鎮為前路,歸德鎮為左路,河州鎮為右路,豫正右軍為後路。趙爾巽以四川省防軍二十九營,編為六軍,每軍六營,分中、前、後、左、右、副中為六路,分駐防境。其防守寧遠之靖字二營、游擊步隊二營,增募寧遠之靖字後營,改為巡防副左路、副右路兩軍,每三營為一軍。成都駐防滿營亦改編巡防隊三營,俾臻一律。瑞澂以江蘇省各營練成一協外,尚有太湖水師巡防隊、陸師左右巡防隊,系陸路三旗及蘇捕營衛隊等先後改編者,乃次第換防調操,以免弛懈。

二年,岑春蓂改編湖南省巡防隊,酌定餉章,即日成軍,其餘緝私三旗,改為南路巡防隊。孫寶琦改編山東省巡防隊,所有中、前、後、左、右五路,各就坐營之中哨改編,其砲隊以快砲六尊為一隊,各府州縣巡勇悉改為巡防隊,兗、沂、曹三府原有之巡防營,亦遵新章編練。恩壽以陜西省巡警軍悉改為巡防隊。楊文鼎以湖南省巡防隊分為中、東、西、南四路,駐防各府。昆源以察哈爾八旗壯丁編練巡防馬隊。松壽以裁撤福建全省之綠營兵改為巡防隊十六營,分五路駐防各府。張人駿以兩江巡緝隊及師船十艘改為探訪隊,其沿江巡防隊深資得力,以協解北洋之淮軍餉為巡防軍餉,並以江防軍分駐江寧省城。錫良以奉天原有之協巡隊、備補隊、砲隊、衛隊各防營,遵章改編為陸軍步隊一標、砲隊一營。是年,山東、山西撫臣咸擬緩裁巡防軍,以靖地方。

三年,張人駿以兩江巡防軍關系重要,其屬於江寧者,馬步三十二營,屬於江蘇者,步隊六營,屬於江北者,步隊八營一哨、馬隊一營,江南北地勢扼要,未可議裁,並擬以新兵中副二營留防三隊改為第一、二、三巡防隊,以一哨為提督衛隊。丁寶銓以山西太原滿營,於光緒二十八年已改練新操,乃遵章改編為巡防隊。恩壽以陜西省巡警軍已改編巡防隊,並設馬步巡防營務處。慶恕以青海墾荒,已開墾六萬餘畝,原有巡防隊不敷分布,增練防軍一旗。誠勛以熱河雖有直隸練軍八營,僅防朝、建一帶,其先後所練巡防隊十三營,分防各屬,未能遽改陸軍。張勛以長江巡防馬、步、砲隊十三營,分駐浦口、六合、江寧、蘇州、懷遠各府縣,並在沿江一帶廣布偵探,以靖盜源。瑞澂以湖南六營已裁,所有撫標之兵,選精壯編巡防一營。此改設防練軍之大略也。

自咸豐軍興,由綠營改為勇營,為留防營,為練軍,為巡防隊,為陸軍,兵制變而益新。至宣統年,非特綠營盡汰,即湘、淮營勇駐防南北洋者,所存亦無幾矣。

陸軍新制,始於甲午戰後,步軍統領榮祿疏保溫處道袁世凱練新軍,是曰新建陸軍。復練兵小站,名曰定武軍。兩江總督張之洞聘德人教練新軍,名曰江南自強軍。其後榮祿以兵部尚書協辦大學士節制北洋海陸各軍,益練新軍,是為武衛軍。

庚子亂後,各省皆起練新軍,或就防軍改編,或用新式招練。至光緒三十年,畫定軍制,京師設練兵處,各省設督練公所,改定新軍區為三十六鎮,新軍制始畫一。

三十三年,京、外新練陸軍,除禁衛軍外,統計近畿第一鎮駐京北仰山窪,官七百四十八員,兵一萬一千七百六十四名。第六鎮駐南苑,官七百四十七員,兵一萬一千八百四十六名。直隸第二鎮駐保定、永平等府,官七百三十七員,兵一萬一千七百三十一名。第四鎮駐馬廠,官七百四十八員,兵一萬一千七百五十六名。山東第五鎮駐省城、濰縣、昌邑等處,官七百四十八員,兵一萬一千七百六十四名。江蘇第二十三混成協駐蘇州等處,官二百七十四員,兵四千三百四十五名。江北第十三混成協駐清江浦,官三百七十六員,兵二千四百八十一名。安徽步隊二標、馬隊一營、砲隊一隊駐省城,官二百五十三員,兵四千一百五十五名。江南第九鎮步隊一營、馬隊二隊駐省城,官七百八十九員,兵八千二百五十五名。江西步隊一協、馬隊二隊駐省城,官二百三十一員,兵四千二百八十七名。河南第二十九混成協駐省城,官三百三十八員,兵五千六百十八名,步隊一協、馬砲隊各一營調駐京城,官一百六十二員,兵三千八十五名。湖南步隊一協、砲隊一營駐省城,官二百四十八員,兵四千五十六名。湖北第八鎮駐省城,官七百二員,兵一萬五百二名,第二十一混成協駐武昌、漢陽及京漢鐵路,官二百八十八員,兵四千六百十二名。浙江步隊一協駐省城,官一百五十九員,兵二千三百八十四名。福建第十鎮駐省城及福寧、延平等處,官四百五十五員,兵六千七百八十八名。雲南步隊一協、砲隊一營駐省城及臨安,官二百三十八員,兵四千二百四十八名。貴州步隊一標、砲隊一隊駐省城,官一百七員,兵一千八百四十六名。四川步隊駐省城,官十二員,兵六十一名。山西步隊二標、馬砲隊各一營駐省城,官二百六十二員,兵四千五百五十七名。陜西步隊一協、砲隊一隊駐省城,官二百二十員,兵三千九百三十六名。甘肅步隊二標、砲隊一營駐省城、河州、固原、西寧,官二百二十一員,兵四千一百二十八名。新疆步隊一協、馬隊一標、砲隊一營駐省城,官一百六十七員,兵二千三百二十二名。東三省第三鎮駐吉林省城、長春、寧安、延吉及奉天錦州等處,官七百五十三員,兵一萬一千八百八十三名,第一混成協駐奉天省城,官三百三員,兵三千五十九名,第二混成協駐奉天新民等處,官三百四員,兵五千五十三名,步隊一協一標、砲隊一營駐吉林,官三百六十一員,兵七千八百七十名。宣統三年統計,除前列外,浙江成第二十一鎮,雲南成第十九鎮,四川成第十七鎮,奉天成第二十鎮,吉林成第二十三鎮,廣東成第二十六鎮駐省城,廣西成第二十五鎮駐省城及桂林等處,先後共成二十六鎮。未幾,武昌陸軍先變,各省應之,而三十六鎮卒未全立雲。


\end{pinyinscope}