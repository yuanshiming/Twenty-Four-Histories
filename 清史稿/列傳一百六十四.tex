\article{列傳一百六十四}

\begin{pinyinscope}
鮑桂星顧★吳孝銘陳鴻鄂木順額徐法績

鮑桂星,字雙五,安徽歙縣人。嘉慶四年進士,選庶吉士,授編修,遷中允。九年,典試河南,留學政。十三年,典試江西。十五年,督湖北學政。累遷至內閣學士。十八年,任滿,既受代,聞林清之變,疏陳十事,急馳至京,仁宗嘉之,曰:「汝所奏已次第施行矣。」擢工部侍郎,充武英殿總裁。桂星性質直,勇於任事。十九年,疏陳刊書及校勘事宜。又劾提調劉榮黼等不職,命王大臣按之。榮黼面訐桂星曾言滿總裁熙昌所校,不過偏旁點畫,修改徒延時日;且言近日有旨,旗人不足恃,故督撫多用漢人。上聞之,怒,命傳詢。桂星對聞自侍郎周兆基,且言在部與滿員共事,多有徇私背公,而兆基不承;又指同官熙昌及慶溥囑託部事,兩人亦不承。以任性妄言,下部嚴議,詔斥桂星指訐慶溥、熙昌囑託無據,其咎小;妄言朝廷輕滿洲重漢人,亂政之大者:革職,不準回籍,令在京閉門思過,責五城御史嚴察;如私著詩文有怨望誹謗之詞,從重治罪。越五年,上意解,復官編修。宣宗即位,召對,諭曰:「汝昔所劾,今已罷斥。」擢侍講,又擢通政司副使,意頗鄉用。道光四年,擢詹事。未幾,卒。

桂星少從同縣吳定學,後師姚鼐,詩古文並有法,著有進奉文及詩集,又嘗用司空圖說輯唐詩品。

顧蓴,字南雅,江蘇吳縣人。嘉慶七年進士,選庶吉士,授編修。十七年,大考一等,擢侍讀。督云南學政,道經河南,見吏多貪墨,奸民充斥,密疏陳謂不早根治,恐釀巨患。仁宗問樞臣,樞臣微其事,不以為意,明年遂有滑縣之亂。在雲南,課士嚴而有恩,以正心術端行誼為首,次治經史、辨文體。按試所至,聞賢士必禮遇之,士風丕振。任滿,充日講官。二十五年,遷侍講學士。值宣宗初政,疏請停捐例。再疏陳崇君德、正人心、飭官方三事。上召對,嘉納其言。故事,大臣子弟不得充軍機章京,時值考選,許一體與試。蓴謂貴介不宜與聞樞要,請收回成命。事尋止。

左都御史松筠出為熱河都統,蓴上疏,謂松筠正人,宜留置左右,失上意,降編修,九歲不調。先是嘉慶中蓴在史館,撰和珅傳,及進御,經他人竄改,和珅曾數因事被高宗詰責,並未載入傳。仁宗怒其失實,嚴詔詰問。大臣以蓴原稿進,仁宗深是之,而奪竄改者官。宣宗一日閱實錄至此事,嘉蓴直筆,因言前保留松筠,必非阿私,特擢蓴右中允。未一歲,復侍講學士原職。

時回疆張格爾亂甫定,蓴疏:「請於喀什噶爾沿邊增重兵,以控制安集延,杜回人窺伺;又其地密邇英吉沙爾、葉爾羌、和闐,皆有水草可耕牧,宜募民屯田,為戰守備。更請慎選大臣,無分滿、漢,務得讀書知大體有方略者任之,而以廉靜明信能拊循民、回者為之佐,庶可永永無事。」

道光十一年,遷通政司副使。湖南北、江南、江西、浙江大水,蓴疏言:「饑民與鹽梟糾合易生事,鹽梟不盡去,終為巨患。緩治之則養禍深,急治之則召禍速,欲禁其妄行,必先謀其生路。現兩淮鹽場漂沒,三江、兩湖勢必仰給蘆、粵之鹽,宜聽民往販,隨時納課,收課後,不問所之,俟鹽產盛,丁力紓,即令課歸丁,不限疆域。」事下所司,格未行。

蓴性嚴正,尚氣節,晚益負時望,從游者眾,類能砥勵自立,滇士尤歸之,其秀異者至京師多就問業焉。十三年,卒。

吳孝銘,字伯新,江蘇陽湖人。嘉慶十四年進士,選庶吉士,散館授工部主事,充軍機章京。十八年,林清之亂甫定,大軍會攻滑縣,孝銘從大臣行,參軍事。累遷郎中。道光中,回疆用兵,首逆張格爾潛逋未獲,議者欲以克復四城,分封回部酋長。孝銘密言於樞臣曰:「是可行於乾隆時,不可行於今日,行之邊患且益甚。」議中止。張格爾旋就俘,賜花翎。

瀕年大水,江、浙、兩湖被災尤數,承回疆兵事後,度支大絀。戶部擬議,宗室日以蕃衍,衣食悉仰之官,耗財之大者,請自系出世祖以上子孫皆改為覺羅,為覺羅者以次遞革。孝銘曰:「茲事當密陳,不宜顯言。法當緩更,不宜驟易。宗室久受恩養,一旦降爵減糧令下即大困,因而呼籥,朝廷不得已,將必復之,是良法美意終於不行也。」部臣是其言,即使草奏上之。歷鴻臚寺少卿、光祿寺少卿、通政司參議、順天府丞,仍留直軍機處。十四年,擢太僕寺卿,再遷宗人府丞。

孝銘前後在樞廷二十餘年,練於掌故,持議悉合機宜;屢膺文衡,有公明稱。母憂,以毀致疾,服闋,至京。尋乞病歸,卒於家。

陳鴻,字午橋,浙江錢塘人。嘉慶十四年進士,選庶吉士,授編修。遷御史,剛直有聲。典試山西還,力陳驛站煩擾,請申定例,肅郵政。二十五年,疏陳浙江水利,略曰:「杭城地當省會,用上下兩塘之水,溉仁和、錢塘、海寧之田數萬餘頃。源出西湖,近廢不治。水淤葑積,塘河津耗,夏旱少雨,上塘枯涸,菑害尤劇。海寧長安鎮號產米之鄉,許村黃灣場為產鹽之地,杭、嘉、湖、寧、紹諸郡賴是挽運。擬請仿江蘇浚吳淞例,歸民間按畝出貲,並飭疆臣躬履屬境,凡堤塘徬壩,悉復舊制,俾農田旱潦有備。」又請:「北省多闢水田,兼收秔稻之利,庶使畿輔為沃野,無兇年。」皆被採納。道光初年,疏陳浙鹺不綱,請裁鹽政,歸巡撫兼理,令整頓緝私,嚴禁掣規重斤科派供應諸弊,如議行。糾劾工部弊竇最多,不避權貴。遷給事中。

二年,奉命稽察銀庫,其妻固賢明,曰:「今而後可送妾輩歸矣!」驚問之,曰:「銀庫美差也,茍為所染,暱君者麕至。禍且不測,妾不忍見君菜市也。」鴻指天自誓,禁絕賂遺。中庭已列花數盆,急揮去,墮地盆碎,中有藏鏹,益聳懼。遂奏庫衡年久鐵陷,請敕工部選精鐵易之。送庫日,責成管庫大臣率科道庫員較驗,然後啟用。禁挪壓餉銀、空白出納及劈鞘諸弊。庫吏百計餂之,不動。復請戶部逐月移送收銀總簿,別立放銀簿,鈐用印信,以資考覈。先是御史趙佩湘馭吏嚴,其死也,論者疑其中毒。鴻蒞庫,勺水不敢飲。出督云南學政,奏革陋規,嚴束書吏,弊風頓革。遷通政司參議,卒於官。

鄂木順額,字復亭,鈕祜祿氏,滿洲正藍旗人。父明安泰,江蘇按察使。鄂木順額,嘉慶二十五年進士,選庶吉士,授編修,累遷右庶子。道光四年,大考一等,擢翰林院侍講學士,遷少詹事。扈從東巡,命分視御道,內監前驅者多率意馳踐,鄂木順額執而鞭之,則愬於御前。召問,鄂木順額對曰:「關外地與關內異,先驅蹂踐則路壞,慮驚乘輿。且御道非大駕不得行,臣不敢不執法。」上韙之。命為湖南學政,以在母憂,引禮力辭。服闋,督安徽學政,遷光祿寺卿。十一年,大雨江溢,學政駐當塗,鄂木順額捐廉以賑,督守令勸捐,士民踴躍。知縣趙汝和盡心民事,而戇直忤大吏,調為鄉試同考官。鄂木順額堅留治賑,事得辦,後上聞。宣宗以為賢,期滿留任,遷大理寺卿。十二年,鄉試,往江寧考錄遺才,卒於試院。

鄂木順額以氣節自勵,在滿洲京僚中稱最。大學士松筠尤重之,曰:「君光明挺直,行且大用,原自愛。」為英和門下士,在翰林,非有故不通謁。及英和謫戍,獨送至數十里外。英和太息曰:「吾愧不知人,平日何曾好待君耶?」嘗謁掌院學士玉麟,閽人弗為通,怒叱曰:「英相國獲罪,即若曹為之,奈何猶不知儆!」翼日,玉麟自往謝。

徐法績,字熙庵,陜西涇陽人。嘉慶二十二年進士,選庶吉士,授編修。以親老歸養,家居十年。道光九年,遷御史,謂諫臣當識大體,不宜毛舉細故瀆上聽,致久浸生厭。疏陳求人才、捐文法、重守令、繩貪墨四事。會直隸、河南地震成災,劾罷監司不職者二人。遷給事中,稽察銀庫,無所染。十二年,分校會試,同官與吏乘隙為奸,匿雲南餉銀,法績出闈亟按之,謀始沮。典試湖南,其副病歿,獨專校閱,遍搜遺卷,拔取多知名士,而得於遺卷者六人,大學士左宗棠其首也。以薦赴東河,學習河工,周歷兩岸,詳詢利弊,著錄為東河要略一篇。十四年,遷太常寺少卿。尋以病乞歸,逾二年卒。

論曰:鮑桂星、顧蓴以鯁直獲譴,卒見諒於明主,蓴之建白,尤卓卓矣。吳孝銘通達政體,鄂木順額樸誠持正,陳鴻、徐法績清操相繼,冀挽頹風,而庫藏大獄,卒發於十數年之間,甚矣實心除弊之罕覯其人也!


\end{pinyinscope}