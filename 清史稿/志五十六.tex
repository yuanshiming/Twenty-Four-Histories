\article{志五十六}

\begin{pinyinscope}
地理二十八

△察哈爾

察哈爾八旗:東南距京師四百三十里。當直隸宣化、山西大同邊外。明插漢,本元裔小王子後。嘉靖間,布希駐牧察哈爾之地,因以名部。天聰六年,征林丹汗,走死。其子孔果爾額哲來降,即其部編旗,駐義州。康熙十四年,其子布爾呢兄弟叛,討誅之,遷部眾駐牧宣化、大同邊外。又以來降之喀爾喀、厄魯特編為佐領隸焉。乾隆二十六年,設都統,駐張家口。其地東界克什克騰,西界歸化城土默特,南界直隸獨石、張家二口及山西大同、朔平,北界蘇尼特及四子部落。袤延千里。北極高四十二度二十分。京師偏西十分。鑲黃旗察哈爾駐蘇明峰,在張家口北三百四十里。東南距京師七百五十里。明,萬全右衛邊外。漢,上谷郡。牧地當張家口之北。東界正白旗察哈爾,西界正黃旗察哈爾,南界鑲黃旗牧廠,北界蘇尼特右翼,廣一百六十里,袤一百九十里。其山:東曰漠爾圖山。南,哈石郎山。北,青羊山蒙名博羅虎插、紅羊山蒙名烏蘭虎插。東南,阿哈魯虎山、駱駝山。西南,額類山。東北,白鹿山蒙名布虎圖。西北,衣爾哈圖山。東南:大紅泉蒙名伊克烏蘭。西南:滾布拉克泉。北:小紅泉。正黃旗察哈爾駐木孫忒克山,在張家口西北三百二十里。東南距京師七百六十里。漢,且如縣地。牧地當張家口之西北,喀喇烏納根山南。東界鑲黃旗察哈爾,西界正紅旗察哈爾,南界陸軍部右翼牧廠,北界四子部落。廣一百一十里,袤二百八十里。其山:東曰額爾吉納克山。南,烏爾虎拖羅海山。北,大鮮卑山蒙名伊克阿勒特、興安山。西南,插漢和邵山。東北,榆樹山蒙名烏里雅蘇臺。西:七金河,蒙名賀爾博金,源出賀爾博金山,南流入希爾池。東南:兆哈河,源出平地,南流,會烏爾古河。又南,蒙古幾河自西來注之。又南,蘇爾扎河自東北來注之。又南流,從大同天鎮入邊,逕柴溝堡,西北入懷安,為東洋河。蒙古幾河源出平地,東流會兆哈河,南入邊城,弩里河南流從之。鑲紅旗察哈爾駐布林泉,在張家口西北四百二十里。東南距京師八百三十里。漢,雁門郡北境。牧地當山西陶林之東北代哈泊。東界正紅旗察哈爾,西界鑲藍旗察哈爾,南界豐鎮,北界四子部落。廣五十里,袤二百里。其山南曰鴨兒山。北,阿爾達布色山。東南,格爾白山。西南,烏爾姑蘇臺山。北:漠惠圖河,源出敖托海泉,西流入鑲藍旗察哈爾,會安達河。東南:莽喀圖河,源出正紅旗察哈爾,西北流,會阿拉齊河,入黛哈池,即奄遏下水海。正紅旗察哈爾駐古爾板拖羅海山,在張家口西北三百七十里。東南距京師八百里。漢,雁門郡北境。牧地當山西陶林之東北、豐鎮之北,奇爾泊。東界正黃旗察哈爾,西界鑲紅旗察哈爾,南界陸軍部右翼牧廠,北界四子部落。廣五十五里,袤二百八十里。其山:東曰阿拍撻蘭臺山。北,伊克和洛圖山。東北,哈撤克圖山。西北,插漢峰。南:昆都倫泉、葫蘆蘇臺泉。北:諾爾孫泉,東南流入正黃旗察哈爾,為納林河,又東南注希爾池。鑲白旗察哈爾駐布雅阿海蘇默,在獨石口西北二百四十五里。東南距京師七百七十里。明,開平衛西北邊。漢,上谷郡北境。牧地當獨石口治西北。東及南界陸軍部牧廠,西界正白旗察哈爾,北界正藍旗察哈爾。廣五十六里,袤一百九十七里。其山:南曰巴漢得兒山。西北,鐵柱山蒙名阿爾坦噶達蘇。其北,西爾哈池。西北:紅鹽池蒙名烏蘭池、魁素池。正白旗察哈爾駐布爾噶臺,在獨石口西北二百九十里。東南距京師八百二十里。明,龍門衛邊外。漢,上谷郡北境。牧地當獨石口治之西北。東及北界鑲白旗察哈爾,西及南界鑲黃旗察哈爾。廣七十八里,袤二百九十五里。其山:南曰清涼黑山蒙名魁屯喀喇。西,喀喇峨博圖山,一名黑山。東南,伊克得兒山,一名大馬鬣山。西北:翁翁泊、黑水灤蒙名喀喇烏蘇。鑲藍旗察哈爾駐阿巴漢喀喇山,在殺虎口東北九十里。東南距京師一千里。明,大同府西北邊外。漢,雁門郡沃陽縣地。牧地當山西寧遠之北。東界鑲紅旗察哈爾,西界山西歸化,南界山西大同,北界四子部落。廣一百一十五里,袤一百六十里。其山:東曰克丑山。西,烏蘭插伯山。東北,衣馬圖山。東南,朔隆峰。其水:南曰察哈音圖河,源出阿爾站嶺,西南流,會弩衡格爾、虎虎烏蘇二河,入烏蘭木倫河。東南,阿拉齊河,源出朔隆峰,東流至鑲紅旗察哈爾,納巴爾哈孫河,入黛哈池。東北,硃喇馬臺河,源出席喇峰,西南流,會喀喇烏蘇河、納札海河,為土爾根河,即黑河之上源。黑河,源出海拉蘇臺坡,與鑲紅旗察哈爾接界,西北流,有納札海、硃喇馬臺等河,皆自東北來,與黑水河會。又西流,受德布色黑河,折西南,合哲爾德河,始名伊克土爾根河,又西入歸化。正藍旗察哈爾駐札哈蘇臺泊,在獨石口東北三百六十里。東南距京師八百九十里。明,開平衛北境。金,桓州地。牧地當直隸獨石口治之北。東界克什克騰,西界鑲白旗察哈爾,南界內務府正白旗羊群牧廠,北界阿巴噶左翼。廣二百六十五里,袤九十五里。其水:東曰戈賀蘇臺河,源出額默黑特站西,北流,會察察爾臺、戈賀蘇臺、奴黑特等河,入阿霸垓右翼。


\end{pinyinscope}