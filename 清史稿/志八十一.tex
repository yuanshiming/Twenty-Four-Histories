\article{志八十一}

\begin{pinyinscope}
○選舉一

古者取士之法,莫備於成周,而得人之盛,亦以成周為最。自唐以後,廢選舉之制,改用科目,歷代相沿。而明則專取四子書及易、書、詩、春秋、禮記五經命題試士,謂之制義。有清一沿明制,二百餘年,雖有以他途進者,終不得與科第出身者相比。康、乾兩朝,特開制科。博學鴻詞,號稱得人。然所試者亦僅詩、賦、策論而已。洎乎末造,世變日亟。論者謂科目人才不足應時務,毅然罷科舉,興學校。採東、西各國教育之新制,變唐、宋以來選舉之成規。前後學制,判然兩事焉。今綜其章制沿革新舊異同之故著於篇。

學校一

有清學校,向沿明制。京師曰國學,並設八旗、宗室等官學。直省曰府、州、縣學。

世祖定鼎燕京,修明北監為太學。順治元年,置祭酒、司業及監丞、博士、助教、學正、學錄、典籍、典簿等官。設六堂為講肄之所,曰率性、修道、誠心、正義、崇志、廣業,一仍明舊。少詹事李若琳首為祭酒,請仿明初制,廣收生徒,官生除恩廕外,七品以上官子弟勤敏好學者,民生除貢生外,廩、增、附生員文義優長者,並許提學考選送監。又言學以國子名,所謂國之貴游子弟學焉。前朝公、侯、伯、駙馬初襲授者,皆入國學讀書。滿洲勛臣子弟有志向學者,並請送監肄業。詔允增設滿洲司業、助教等官,是為八旗子弟入監之始。厥後定為限制,條例屢更,益臻詳備。肄業生徒,有貢、有監。貢生凡六:曰歲貢、恩貢、拔貢、優貢、副貢、例貢。監生凡四:曰恩監、廕監、優監、例監。廕監有二:曰恩廕、難廕。通謂之國子監生。

六堂肄業,分內、外班。初,內班百五十名,堂各二十五名;外班百二十名,堂各二十名。戶部歲發帑銀,給膏火,獎勵有差,餘備周恤。乾隆初,改內班堂各三十名,內、外共三百名。既而裁外班百二十名,加內班膏火,撥內班二十四名為外班。嘉慶初,以八旗及大、宛兩縣肄業生距家近,不住舍,不許補內班。補班之始,赴監應試,曰考到。列一、二等者再試,曰考驗。貢生一、二等,監生一等,乃許肄業。假滿回監曰復班。內班生原依親處館,滿、蒙、漢軍恩監生習繙譯或騎射,不能竟月在學者,改外班。曠大課一次,無故離學至三次以上,例罰改外。置集愆冊,治諸不帥教者。出入必記於簿,監丞掌之。省親、完姻、丁憂、告病及同居伯、叔、兄長喪而無子者,予假歸里,限期回監。遲誤懲罰,私歸黜革,冒替除名。

課士之法,月朔、望釋奠畢,博士集諸生,講解經書。上旬助教講義。既望,學正、學錄講書各一次。會講、覆講、上書、覆背,月三回,周而復始。所習四書、五經、性理、通鑒諸書,其兼通十三經、二十一史,博極群書者,隨資學所詣。日摹晉、唐名帖數百字,立日課冊,旬日呈助教等批晰,朔、望呈堂查驗。祭酒、司業月望輪課四書文一、詩一,曰大課。祭酒季考,司業月課,皆用四書、五經文,並詔、誥、表、策論、判。月朔,博士課經文、經解及策論。月三日,助教課,十八日,學正、學錄課,各試四書文一、詩一、經文或策一。

積分歷事之法,國初行之。監生坐監期滿,撥歷部院練習政體。三月考勤,一年期滿送廷試。其免坐監,或免歷一月二月者,恩詔有之,非常例也。順治三年,祭酒薛所蘊奏定漢監生積分法,常課外,月試經義、策論各一,合式者拔置一等。歲考一等十二次為及格,免撥歷,送廷試超選。十五年,祭酒固爾嘉渾議:「令監生考到日,拔其尤者許積分;不與者,期滿咨部歷事。積分法一年為限。常課外,月試一等與一分,二等半分,二等以下無分。有五經兼通,全史精熟,或善摹鍾、王諸帖,雖文不及格,亦與一分。積滿八分為及格,歲不逾十餘人。恩、拔、歲、副,咨部歷滿考職,照教習貢生例,上上卷用通判,上卷用知縣。例監歷滿考職,與不積分貢生一體廷試。每百名取正印八名,餘用州、縣佐貳。積分不滿數,原分部者,咨部不得優選。原再肄業滿分者聽。」從之。是年,科臣王命岳以貢途壅塞,請暫停恩、拔、歲貢。於是坐監人少,難較分數。十七年,固爾嘉渾奏停積分法,後遂不復行。康熙初,並停撥歷,期滿咨部考試,用州同、州判、縣丞、主簿、吏目。自是部院諸司無監生,惟考選通文理能楷書者,送修書各館,較年勞議敘,照應得職銜選用,優者或加等焉。

監生坐監期,恩貢六月,歲貢八月,副貢廩膳六月,增、附八月,拔貢廩膳十四月,增、附十六月,恩廕二十四月,難廕六月,例貢廩膳十四月,增、附十六月,俊秀二十四月。例監計捐監月分三十六月。雍正五年,定除監期計算。各監生肄業,率以連閏扣滿三年為期。告假、丁憂、考劣、記過,則扣除月日。告假依限到監,或逾限而本籍有司官具牘者,仍前後通算。

舊制,祭酒、司業總理監務。雍正三年,始設管理監事大臣。乾隆二年,孫嘉淦以刑部尚書管監事。初嘉淦在世宗朝官司業,奏言:「學校之教,宜先經術,請敕天下學臣,選拔諸生貢太學,九卿舉經明行修之士為助教,一以經術造士。三年考成,舉以待用。」議未及行,遷祭酒,申前請,世宗韙之。先是太學生名為坐監肄業,率假館散處。遇釋奠、堂期、季考、月課,暫一齊集。監內舊有號房五百餘間,修圮不時,且資斧不給,無以宿諸生。嘉淦言:「各省拔貢云集京師,需住監者三百餘人。六堂祗可誦讀,不能棲止。乞給監南官房,令助教等官及肄業生居住。歲給銀六千兩為講課、桌飯、衣服、賑助之費。」允之。是為南學。

至是,請仿宋儒胡瑗經義、治事分齋遺法。明經者,或治一經,或兼他經,務取御纂折中、傳說諸書,探其原本,講明人倫日用之理。治事者,如歷代典禮、賦役、律令、邊防、水利、天官、河渠、算法之類。或專治一事,或兼治數事,務窮究其源流利弊。考試時,必以經術湛深、通達事理、驗稽古愛民之識。三年期滿,分別等第,以示勸懲。從之。令諸生有心得或疑義,逐條劄記,呈助教批判,按期呈堂。季考月課,改四書題一,五經講義題各一,治事策問一。時高宗加意太學,嘉淦嚴立課程,獎誘備至,六堂講師,極一時之選。舉人吳鼎、梁錫興,皆以薦舉經學授司業。進士莊亨陽,舉人潘永季、蔡德峻、秦蕙田、吳鼐,貢生官獻瑤、王文震,監生夏宗瀾,皆以潛心經學,先後被薦為本監屬官。分長六堂,各占一經,時有「四賢五君子」之稱。師徒濟濟,皆奮自鏃礪,研求實學。而祭酒趙國麟又以經義、治事外,應講習時藝,請頒六堂欽定四書文資誦習。並報可。

清代臨雍視學典禮綦重。順治九年,世祖首視學。先期行取衍聖公、五經博士率孔氏暨先賢各氏族裔赴京觀禮。帝釋奠畢,詣彞倫堂御講幄。祭酒講四書,司業講經。宣制勉太學諸生。越日,賜衍聖公冠服,國子監官賞賚有差。各氏後裔送監讀書。嗣是歷代舉行以為常。乾隆四十八年諭曰:「稽古國學之制,天子曰闢雍,所以行禮樂、宣德化、昭文明而流教澤,典至鉅也。國學為人文薈萃之地,規制宜隆。闢雍之立,元、明以來,典尚闕如,應增建以臻美備。」命尚書德保,尚書兼管國子監事劉墉,侍郎德成,仿禮經舊制,於彞倫堂南營建。明年,落成。又明年,高宗駕臨闢雍行講學禮。命大學士、伯伍彌泰,大學士管監事蔡新,進講四書。祭酒覺羅吉善、鄒奕孝,進講周易。頒禦論二篇,宣示義蘊。王、公、衍聖公、大學士以下官,暨肄業觀禮諸生,三千八十八人,圜橋聽講。禮成,賜燕禮部,恩賚有加。是時天子右文,群臣躬遇休明,翊贊文化,彬彬稱極盛矣。嘉慶以後,視學典禮,率循不廢。咸豐初,猶一舉行焉。

道光末,詔整飭南學,住學者百餘人,監規頹廢已久,迄難振作。咸豐軍興,歲費折發,章程亦屢更。同治初元,以國學專課文藝,無裨實學,令兼課論、策。用經、史、性理諸書命題,獎勵留心時務者。明年,增發歲費三千兩。九年,乃復舊額。選文行優者四十人住南學,厚給廩餼,文風稍稍興起。光緒二年,增二十名。十一年,許各省舉人入監,曰舉監。其後無論舉人、貢監生,凡非正印官未投供,舉、貢未傳到教習,均得入監,以廣裁成。

貢監生諸色目多沿明制,歲貢,取府、州、縣學食廩年深者,挨次升貢。順治二年,命直省歲貢士京師。府學歲一人,州學三歲二人,縣學二歲一人,一正二陪。學政嚴加遴選,濫充發回原學。五名以上,學政罰俸。十五年,令到部時詳查,年力強壯者,乃許送監。康熙元年,減貢額,府三歲二人,州二歲一人,縣三歲一人。八年,復照順治二年例。二十六年,罷歲貢廷試。其後但由學政挨序考準咨部選授本省訓導。得缺後,巡撫一加考驗,原入監者益鮮矣。恩貢,因明制,國家有慶典或登極詔書,以當貢者充之。順治元年,詔直省府、州、縣學,以本年正貢作恩貢,次貢作歲貢。歷代恩詔皆如之。九年,五氏子孫觀禮生員十五人,送監讀書,準作恩貢。乾隆後,恩賜臨雍觀禮聖賢後裔廩、增、附生入監以為常。至康、乾間,天子東巡,親詣闕里,拔取五氏、十三氏子孫生員貢成均,則加恩聖裔,非恆制也。拔貢,因明選貢遺制,順治元年舉行。順天六人,直省府學二人,州、縣學各一人。康熙十年,令學臣於考取一、二等生員內,遴選文行兼優者貢太學,從祭酒查祿請也。明年,始選拔八旗生員,滿洲、蒙古二人,漢軍一人。時各省選貢多冒濫,三十七八年間,祭酒特默德、孫岳頒面試山西選拔張漢翀等六名,陜西呂爾恆等四名,廣東陳其瑋等三名,均文理不堪,字畫舛謬,原卷駁回,學臣參處,遂停選拔。雍正元年,禮部尚書陳元龍疏請嚴成均肄業之規。部議,太學監生,皆由捐納,能文之士稀少,應令學臣照舊例選拔送監。從之。五年,世宗以歲貢較食廩淺深,多年力衰憊之人,欲得英才,必須選拔。命嗣後六年選拔一次。明年,又諭學政選拔不拘一、二等生員,酌試時務策論,果有識見才幹,再訪平日品行,即未列優等,亦許選拔。故雍、乾間充貢國學,以選拔為最盛。

乾隆初定朝考制,列一、二等者,揀選引見錄用。三等劄監肄業。尋停揀選例。三年期滿,祭酒等分別等第,覈實保薦,用知縣、教職。七年,帝以拔貢六年一舉,人多缺少,妨舉人銓選之路。且生員優者,應科舉時,自可脫穎而出,不專藉選拔為進身。改十二年一舉。遂為永制。十六年,以天下教官多昏耄,濫竽戀棧。雖定例六年甄別,長官每以閒曹,多方寬假。諭詳加澄汰。廷臣議,督、撫三年澄汰教職員缺,以朝考揀選拔貢充補。未入揀選者,劄監肄業如舊。四十一年,定朝考優等兼用七品小京官。五十五年,朝考始用覆試。學政選拔分二場,試四書文、經文、策論。乾隆十七年,經文改經解。二十三年,增五言八韻詩。會同督、撫覆試。朝考試書藝一、詩一。副榜入監,順治二年,令順天鄉試中式副榜增、附,準作貢監。廩生及恩、拔、歲貢,免坐監,與廷試。十五年,他貢停,惟副榜照舊解送。康熙元年,停副貢額。十一年,以查祿奏復,舊制優貢之選,與拔貢並重。

順治二年,令直省不拘廩、增、附生,選文行兼優者,大學二人、小學一人送監。康熙二十四年,以監生止輸納一途,貧窶之士無由觀光,令照順治二年例選送。雍正間,始析貢監名色,廩、增準作優貢,附生準作優監。乾隆四年,限大省無過五、六名,中省三、四名,小省一、二名,任缺無濫。學政三年會同督、撫保題,分試兩場,略同選拔。試四書文、經解、經文、策論,後增詩。二十三年,定優生到部,如拔貢朝考例。試書藝一、詩一,文理明通者升太學;荒疏者發回,學政議處。二十九年,學臣有以拔貢年分暫停舉優為請者,部議拔貢十二年一舉,而學臣三年任滿,宜舉優黜劣,通省不過數名,應仍舊例。嘉慶十九年,御史黃中傑條奏,請與拔貢一體廷試錄用。禮部議駁。請免來京朝考,示體恤。帝以優生經朝考準作貢生,斯合貢於王廷之義。停朝考,名實不符。弗許。然卒以無錄用之條,多不赴京報考。同治二年,議定甲子科始廷試優生,仿順天鄉試例,分南、北、中卷。八旗、奉天、直隸、山東、山西、河南、陜西、甘肅為北卷,江蘇、江西、浙江、安徽、福建、湖北、湖南為南卷,四川、廣東、廣西、雲南、貴州為中卷。考列一、二等用知縣、教職,三等用訓導。恩、拔、副、歲、優,時稱「五貢」。科目之外,由此者謂之正途。所以別於雜流也。

恩監,由八旗漢文官學生、算學滿、漢肄業生考取。又臨雍觀禮聖賢後裔,由武生、奉祀生、俊秀入監者,皆為恩監。例貢與例監相仿,由廩、增、附生或俊秀監生援例報捐貢生者,曰例貢;由俊秀報捐監生者,曰例監。凡捐納入官必由之。或在監肄業,或在籍,均為監生。恩廕,凡滿、漢子弟奉敕送監讀書,恩詔分別內外文武品級,廕子入監。順治二年,定文官京四品、外三品以上,武官二品以上,俱送一子入監。十一年,覺羅廕生照各官廕生例,一體送監。包衣佐領下官子弟,向例不得為廕監。康熙九年,例除。宗室給廕入監,自康熙五十二年始也。難廕始順治四年,以殉難陜西固原道副使呂鳴夏子入監讀書。九年,定內、外滿、漢三品以上官,三年任滿,勤事以死者,廕一子入監。後廣其例,凡三司首領,州、縣佐貳官死難者,亦得廕子矣。

外國肄業生,康熙二十七年,琉球國王始遣陪臣子弟梁成楫等隨貢使至,入貢肄業。雍正六年,鄂羅斯遣官生魯喀等留學中國,以滿、漢助教等教之,月給銀米器物,學成遣歸,先後絡繹。至同治間,琉球官生猶有至者。

他如順治二年,於隨征入關奉天十五學,取三十人入監,為天下勸。十一年,定隨征廩生準作貢監。生員有軍功二等,準作生監。更有軍功二等,準作貢生,謂之功貢。未幾例停,則開國時權宜之制也。

考送校錄,始於乾隆三年,令國子監選正途貢生,年力少壯、字畫端楷者十人,送武英殿備謄錄。年滿議敘。三十四年例停,歸吏部謄錄貢生內選取。嗣以吏部無合例者,仍由在監拔、副、優貢生考選。嘉慶間增十名,後不復行。

五貢就職,學政會同巡撫驗看,咨部依科分名次、年分先後,恩、拔、副貢以教諭選用,歲貢以訓導選用。康熙中,捐納歲貢,並用訓導。雍正初,捐納貢生,教諭改縣丞,訓導改主簿。既仍許廩生捐歲貢者,用訓導;恩、拔、副貢年力富強者,得就職直隸州州判。嘉慶以後,凡朝考未錄之拔貢及恩、副、歲、優貢生,遇鄉試年,得具呈就職、就教。優貢就教,附歲貢末用訓導。道光初,許滿、蒙正途貢生就職,與滿員通較年分先後選用。貢監考職,定例必監期已滿,乃許送考。惟特恩考職,不論監期滿否。凡正途、捐納各項貢、監生,及候補言謄錄、教習、校錄,一體送考。其已就教、就職及捐職、襲世職者不許。初制,考職歲一舉,貢、監一例以州同、州判、縣丞、主簿、吏目錄用。乾隆元年,定考職以鄉試年,恩科不考。恩、拔、副貢考列一等以州同、二等以州判、三等以縣丞選用。歲貢一等以主簿、二等以吏目選用。原就教者聽。捐納貢監考取如歲貢例。五十六年停考職。嘉慶五年,僅一行之。光緒三十一年,直隸總督袁世凱等奏停科舉摺寬籌舉貢生員出路一條,「請十年三科內優貢加額錄取。己酉選拔如舊,朝考用京官知縣。督、撫、學政三科內考選學貢通算學、地理、財政、兵事、交涉、鐵路、礦務、警察、外國政法之一者,三年一次,保送若干名,略視會試中額兩三倍。赴京試取者,用主事、中書、知縣」。詔議行。明年,政務處詳議,己酉拔貢,照向額倍取,本年丙午考優。以後三年一考,視例額加四倍。廩生出貢許倍額。部院考用謄錄,分舉人、五貢、生員三等。二年期滿獎敘。舉人、優、拔,擇尤改用七品小京官。又為廣就職之例,五貢一體以直隸州州判,按察、鹽運經歷,散州州判、經歷,縣丞,分別註選,或分發試用。蓋五貢終清之世,未嘗廢棄也。

算學隸國子監,稱國子監算學。乾隆四年,額設學生滿、漢各十二,蒙古、漢軍各六。續設漢肄業生二十四。遵禦制數理精蘊,分線、面、體三部。部限一年通曉。七政限二年。有季考、歲考。五年期滿考取者,滿、蒙、漢軍學生咨部,以本旗天文生序補。漢學生舉人用博士,貢監生童用天文生。

此外隸國學者,為八旗官學。順治元年,若琳奏:「臣監僻在城東北隅,滿員子弟就學不便,議於滿洲八固山地方各立書院,以國學二、六堂教官分教之,以時赴監考課。」下部議行。於是八旗各建學舍。每佐領下取官學生一名,以十名習漢書,餘習滿書。二年,從所蘊言,合兩旗為一學。每學教習十人,教習酌取京省生員。其後學額屢有增減,教習於國學肄業生考選,止用恩、拔、副、歲貢生。如無其人,準例監生亦得考取。舉人原就,一例考選。雍正元年,於八旗蒙古護軍、領催、驍騎內,選熟練國語、蒙古語者十六人,充蒙古教習。向例官學生分佐領選送。五年,定每旗額設百名。滿洲六十,習清、漢書各半。蒙古、漢軍各二十,通一旗選擇,不拘佐領。年幼者習清書,稍長者習漢文。撥八旗教養兵額滿洲三十,蒙古、漢軍各十名錢糧分給學生。定漢教習每旗五人。乾隆初,定官學生肄業以十年為率,三年內講誦經書,監臣考驗,擇材資聰穎有志力學者,歸漢文班;年長原學繙譯者,歸滿文班。三年,欽派大臣考取漢文明通者,拔為監生,升太學。與漢貢監究心明經治事,期滿,擇尤保薦,考選錄用。八年,定漢教習三年期滿,分等引見。一等用知縣,二等用知縣或教職銓選。一等再教習三年,果實心訓課者,知縣即用。蒙古教習五年期滿實心訓課者,用護軍校、驍騎校。滿助教每旗二人,以八旗文進士、舉人,繙譯進士、舉人,恩、拔、副、歲貢生,文生員,繙譯生員,廢員,筆帖式考取。三十三年,下五旗包衣每旗增設學生十名。滿洲六,蒙古、漢軍各二,不給錢糧。五十四年,於每旗百名內裁十名,選取經書熟、文理優者二十人,加給膏火資鼓勵。嘉、道以後,官學積漸廢弛,八旗子弟僅恃此進身。教習停年期滿予錄用例,月課虛應故事。雖明諭屢督責,迄難振刷。光緒初,力籌整頓。每學以滿、漢科甲官一人為管學官,專司考覈學生課程,教習勤惰。簡派滿、漢進士出身大員二人為管理八旗官學大臣。每學添設翰林編、檢一員。月課季考,分司考校。春秋赴監會考如舊。

同、光間,國學及官學造就科舉之才,亦頗稱盛。然囿於帖括,舊制鮮變通。三十一年,監臣奏於南學添設科學,未幾,裁國子監,並設學部。文廟祀典,設國子丞一人掌之。八旗官學改並學堂,算學亦改稱欽天監天文算學,隸欽天監。而太學遂與科舉並廢云。

宗學肇自虞廷,命夔典樂,教胄子。三代無宗學名,而義已備。唐、宋後,有其名而制弗詳。清順治十年,八旗各設宗學,選滿洲生員為師。凡未封宗室子弟,十歲以上,俱入學習清書。雍正二年定制,左、右兩翼設滿、漢學各一,王、公、將軍及閒散宗室子弟十八歲以下,入學分習清、漢書,兼騎射。以王、公一人總其事。設總、副管,以宗室分尊齒長者充之。清書教習二人,選罷閒旗員及進士、舉人、貢生、生員善繙譯者充之。騎射教習二人,選罷閒旗員及護軍校善射者充之。每學生十人,設漢書教習一人,禮部考取舉、貢充之。三年期滿,分別等第錄用。十一年,兩學各以翰林官二人董率課程,分日講授經義、文法。乾隆初,以滿、漢京堂各一人總稽學課,月試經義、繙譯及射藝。九年,定每屆五年,簡大臣合試兩翼學生,欽定名次,以會試中式註冊。俟會試年,習繙譯者,與八旗繙譯貢生同引見,賜進士,用府屬額外主事。習漢文者,與天下貢士同殿試,賜進士甲第,用翰林部屬等官。十年,考試漢文、繙譯無佳作。諭曰:「我朝崇尚本務,宗室子弟俱講究清文,精通騎射。誠恐學習漢文,流於漢人浮靡之習。世祖諭停習漢書,所以敦本實、黜浮華也。嗣後宗室子弟不能習漢文者,其各嫺習武藝,儲為國家有用之器。」明年,定學額,左翼七十,右翼六十。二十一年,裁漢教習九人,改繙譯教習。增騎射教習,翼各一人。嘉慶初,畫一兩翼學額,增右翼十名。定每學教習滿三人,漢四人。十三年,兩翼各增學額三十,足百名,為永制。

覺羅學,雍正七年,詔八旗於衙署旁設滿、漢學各一,覺羅子弟八歲至十八歲,入學讀書習射,規制略同宗學。總管王、公,春秋考驗。三年欽派大臣會同宗人府考試,分別獎懲。學成,與旗人同應歲、科試及鄉、會試,並考用中書、筆帖式。學額鑲黃旗六十一,正黃旗三十六,正白旗、正紅旗各四十,鑲白旗十五,鑲紅旗六十四,正藍旗三十九,鑲藍旗四十五。滿、漢教習,旗各二人。惟鑲白旗各一。

景山官學,康熙二十四年,令於北上門兩旁官房設官學,選內府三旗佐領、管領下幼童三百六十名。清書三房,各設教習三人。漢書三房,各設教習四人。初,滿教習用內府官老成者,漢教習禮部考取生員文理優通者。尋改選內閣善書、射之中書充滿教習,新進士老成者充漢教習。雍正後,漢教習以舉人、貢生考取,三年期滿,咨部敘用。學生肄業三年,考列一等用筆帖式,二等用庫使、庫守。乾隆四十四年,許回子佐領下選補學生四名。嘉慶間,定額鑲黃旗、正白旗均百二十四,正黃旗百四十,回童四。

咸安宮官學,雍正六年,詔選內府三旗佐領、管領下幼童及八旗俊秀者九十名,以翰林官居住咸安宮教之。漢書十二房,清書三房,各設教習一人,教射、教國語,各三人,如景山官學考取例。五年欽派大臣考試,一、二等用七、八品筆帖式。漢教習三年、清語騎射教習五年,分別議敘。乾隆初,定漢教習選取新進士,不足,於明通榜舉人考充。期滿,進士用主事、知縣,舉人用知縣、教職。二十三年以後,不論年分,許學生考繙譯中書、筆帖式、庫使。定教習漢九人,滿六人。

宗學、覺羅學隸宗人府,景山學、咸安宮學隸內務府。諸學總管、教習等,類乏通才,經費徒糜。甚者黌舍空虛,期滿時,例報成就學生若干名而已。光緒二十八年,翰林院侍讀寶熙奏請援同文館歸並大學堂例,將宗室、覺羅、八旗等官學改並中、小學堂,均歸管學大臣辦理。從之。

他如世職官學,八旗及禮部義學,健銳營、外火器營、圓明園、護軍營等學,皆清代特設,習滿、蒙語言文字。

府、州、縣、衛儒學,明制具備,清因之。世祖勘定天下,命賑助貧生,優免在學生員,官給廩餼。順治七年,改南京國子監為江寧府學。尋頒臥碑文,刊石立直省學宮。諭禮部曰:「帝王敷治,文教為先。臣子致君,經術為本。自明末擾亂,日尋干戈,學問之道,闕焉未講。今天下漸定,朕將興文教,崇經術,以開太平。爾部傳諭直省學臣,訓督士子,凡理學、道德、經濟、典故諸書,務研求淹貫。明體則為真儒,達用則為良吏。果有實學,朕必不次簡拔,重加任用。」初,各省設督學道,以各部郎中進士出身者充之。惟順天、江南、浙江為提督學政,用翰林官。宣大、蘇松、江安、淮揚、肇高先皆分設,既乃裁並。上下江、湖南北則裁並後仍分設。雍正中,一體改稱學院,省設一人。奉天以府丞、臺灣以臺灣道兼之。甘肅自分闈後,始設學政。

各學教官,府設教授,州設學正,縣設教諭,各一,皆設訓導佐之。員額時有裁並。生員色目,曰廩膳生、增廣生、附生。初入學曰附學生員。廩、增有定額,以歲、科兩試等第高者補充。生員額初視人文多寡,分大、中、小學。大學四十名,中學三十名,小學二十名。嗣改府視大學,大州、縣視中學減半,小學四名或五名。康熙九年,大府、州、縣仍舊額,更定中學十二名,小學七名或八名。後屢有增廣。滿洲、蒙古、漢軍子弟,初歸順天考試取進,滿洲、漢軍各百二十名,蒙古六十名。康熙中減定滿、蒙四十名,漢軍二十名。旋復增為滿、蒙六十,漢軍三十。學政三年任滿。歲、科兩試。順治十五年停直省科試,康熙十二年復之。

儒童入學考試,初用四書文、孝經論各一,孝經題少,又以性理、太極圖說、通書、西銘、正蒙命題。嗣定正試四書文二,覆試四書文、小學論各一。雍正初,科試加經文。冬月晷短,書一、經一。尋定科試四書、經文外,增策論題,仍用孝經。乾隆初,覆試兼用小學論。中葉以後,試書藝、經藝各一。增五言六韻詩。聖祖先後頒聖諭廣訓及訓飭士子文於直省儒學。雍正間,學士張照奏令儒童縣、府覆試,背錄聖諭廣訓一條,著為令。凡新進生員,如國子監坐監例,令在學肄業,以次期新生入學為滿。

教官考校之法,有月課、季考,四書文外,兼試策論。翌日講大清律刑名、錢穀要者若干條。月集諸生明倫堂,誦訓飭士子文及臥碑諸條,諸生環聽。除丁憂、患病、游學、有事故外,不應月課三次者戒飭,無故終年不應者黜革。試卷申送學政查覆。訖於嘉慶,月課漸不舉行。御史辛從益以為言,詔令整頓。嗣是教官多闒茸不稱職,有師生之名,無訓誨之實矣。

學政考覈教官,按其文行及訓士勤惰,隨時薦黜。康熙中,令撫臣考試。嗣教職部選後,赴撫院試。四等以上,給憑赴任;五等學習三年再試,六等褫職。雍正初,定四、五等俱解任學習。六年考成俸滿,盡心訓導,士無過犯者,督、撫、學政保題,擢用知縣。

學臣按臨,謁先師,升明倫堂,官生以次揖見。生員掣簽講書,各講大清律三條,西鄉立;講畢,東鄉立:俟行賞罰。

考試生員,舊例歲、科試俱四書文二、經文一。自有給燭之禁,例不出經題。雍正元年,科試增經文,冬月一書、一經。六年,更定歲試兩書、一經,冬月一書、一經。科試書一、經一、策一,冬月減經文。乾隆二十三年,改歲試書一、經一,科試書一、策一、詩一,冬月亦如之。欠考,勒限補行。三次,黜革。後寬其例,五次以上乃黜。

駐防考試,清初定制,各省駐防弁兵子弟能讀書者,詣京應試。乾隆時,參領金珩請許歲、科試將軍先試騎射,就近送府院取進。嚴旨切責。嘉慶四年,湖南布政使通恩奏如金珩言,詔議行。應試童生,五六名取進一名,佐領約束之。訓習清語、騎射,府學課文藝。明年諭曰:「我滿洲根本,騎射為先。若八旗子弟專以讀書應試為能,輕視弓馬,怠荒武備,殊失國家設立駐防之意。嗣後各省駐防官弁子弟,不得因有就近考試之例,遂荒本業。」

漢軍設廩、增,自順治九年始。康熙十年,滿、蒙亦設廩、增。初制各二十名,嗣減漢軍十名。雍正間定額,滿、蒙六十,漢軍三十。直省廩、增額,府四十,州三十,縣二十,衛十。其新設者,府學視州學,州學視縣學。其一學分兩學,則均分其額,或差分之。

六等黜陟法,視明為繁密。考列一等,增、附、青、社俱補廩。無廩缺,附、青、社補增。無增缺,青、社復附,各候廩。原廩、增停降者收復。二等,增補廩,附、青、社補增。無增缺,青、社復附。停廩降增者復廩。增降附者復增,不許補廩。三等,停廩者收復候廩。丁憂起復,病痊考復,緣事辨復,增降附者許收復,青衣發社者復附,廩降增者不許復。四等,廩免責停餼,不作缺,限讀書六月送考。停降者不許限考。增、附、青、社俱撲責。五等,廩停作缺。原停廩者降增,增降附,附降青衣,青衣發社,原發社者黜為民。六等,廩膳十年以上發社,六年以上與增十年以上者,發本處充吏,餘黜為民。入學未及六年者發社。科試一、二等送鄉試,幫補廩、增,如歲試大率祗列三等,八旗生員給錢糧,考列四等以下停給,次屆列一、二、三等給還。優等補廩、增,劣等降青、社,如漢生員。八旗故重騎射,往往不苛求文藝,但置後等。

凡優恤諸生,例免差徭。廩生貧生給學租養贍。違犯禁令,小者府、州、縣行教官責懲,大者申學政,黜革後治罪,地方官不得擅責。學政校文外,賞黜優劣,以為勸懲。如教官徇庇劣生不揭報,或經揭報,學政不嚴加懲處,分別罰俸、鐫級、褫職。其大較也。

光緒末,科舉廢,丙午並停歲、科試。天下生員無所託業,乃議廣用途,許考各部院謄錄。並於考優年,令州縣官、教官會保申送督、撫、學政,考取文理暢達、事理明晰者,大省百名,中省七十名,小省五十名,咨部以巡檢、典史分別註選,或分發試用。各省學政改司,考校學堂。未幾學政裁,教官停選。在職者,凡生員考職、孝廉方正各事屬之,俸滿用知縣,或以直州同、鹽庫大使用。儒學雖不廢,名存實亡,非一日矣。

武生附儒學,通稱武生。順治初,京衛武生童考試隸兵部。康熙三年,改隸學院,直省府、州、縣、衛武生,儒學教官兼轄之。騎射外,教以武經七書、百將傳及孝經、四書。學政三年一考。順天舊設武學,自八旗設儒學教官,兼轄滿洲、蒙古、漢軍武生,裁武學官。大、宛兩縣武生,順天教官轄之,學額如文生童例,分大、中、小學。自二十名遞減至七八名。考試分內、外場,先外場騎射,次內場策論。歲試列一、二等,準作科舉。故武生有歲試無科試。

各省書院之設,輔學校所不及,初於省會設之。世祖頒給帑金,風勵天下。厥後府、州、縣次第建立,延聘經明行修之士為之長,秀異多出其中。高宗明詔獎勸,比於古者侯國之學。儒學浸衰,教官不舉其職,所賴以造士者,獨在書院。其裨益育才,非淺鮮也。

又有義學,社學。社學,鄉置一區,擇文行優者充社師,免其差徭,量給廩餼。凡近鄉子弟十二歲以上令入學。義學,初由京師五城各立一所,後各省府、州、縣多設立,教孤寒生童,或苗、蠻、黎、瑤子弟秀異者。規制簡略,可無述也。


\end{pinyinscope}