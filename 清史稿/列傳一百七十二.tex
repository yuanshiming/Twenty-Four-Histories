\article{列傳一百七十二}

\begin{pinyinscope}
杜受田子翰祁俊藻子世長翁心存彭蘊章

杜受田,字芝農,山東濱州人。父堮,嘉慶六年進士,由翰林院編修累官禮部侍郎,重宴鹿鳴,加太子少保,卒贈太傅,謚文端。

受田,道光三年進士,會試第一,殿試二甲第一,選庶吉士,授編修。大考擢中允,遷洗馬,督山西學政。十五年,特召還京,直上書房,授文宗讀。四遷內閣學士,命專心授讀,毋庸到閣批本。十八年,擢工部侍郎,調戶部。二十四年,連擢左都御史、工部尚書,尋充上書房總師傅。文宗自六歲入學,受田朝夕納誨,必以正道,歷十餘年。至宣宗晚年,以文宗長且賢,欲付大業,猶未決。會校獵南苑,諸皇子皆從,恭親王奕獲禽最多,文宗未發一矢,問之,對曰:「時方春,鳥獸孳育,不忍傷生以干天和。」宣宗大悅,曰:「此真帝者之言!」立儲遂密定,受田輔導之力也。

三十年,文宗即位,加太子太傅,兼署吏部尚書,調刑部尚書、協辦大學士。受田雖未入樞廷,國家大政及進退大臣,上必諮而後行。廣西軍事亟,受田數陳方略,薦林則徐、周天爵,先後起用。提督向榮老於軍事,以同列不和被謗,力陳輿論,數保全之。咸豐元年,調管禮部。二年,因河決豐北久未塞,山東、江北被災重,命偕福州將軍怡良往治賑務。疏言:「災廣民眾,賑恤不可緩,尤在得人。」薦山東布政使劉源灝、江寧布政使祁宿藻,皆持正有為,責成專任;請截留江、廣漕米六十萬石分給兩省;詔並允行。

受田自侍文宗學,未嘗離左右,當陛辭,不覺感戀流涕。在途觸暑染疫,力疾治事,與源灝、宿藻等覈定施賑章程,疏陳而不言病,至清江浦遽卒。遺疏念賊氛未靖,河患未平,尤以敬天法祖、勤政愛民、崇節儉、慎好惡、平賞罰為言。文宗震悼,贈太師、大學士,入祀賢良祠,賜金五千兩治喪,遣近臣慰視其父堮,擢其子檢討翰為庶子,孫三人並賜舉人。復特詔曰:「杜受田品端學粹,正色立朝,皇考深加倚重,特簡為朕師傅。憶在書齋,凡所陳說,悉本唐、虞、三代聖聖相傳之旨,實能發明蘊奧,體用兼賅。朕即位後,周諮時政利弊,民生疾苦,盡心獻替,啟沃良多!援嘉慶朝大學士硃珪故事,特謚文正。」謂其公忠正直,足當「正」字而無愧。柩至京,上親奠,撫棺哭甚哀,晉其父堮禮部尚書銜。明年,上臨雍講學,復詔褒受田曩日講貫之功,即家賜祭一壇。及柩歸,命恭親王奠送,遣官到籍致祭,飾終之典,一時無與比。子,由翰林院編修累官戶部侍郎,督辦山東團練。

翰,字繼園。道光二十四年進士,選庶吉士,授檢討。咸豐三年,降。服闋,補庶子。文宗念受田舊勞,數月間迭擢工部侍郎,命在軍機大臣上行走,辦理京城巡防事宜。翰勇於任事,甚被倚任。十年,隨扈熱河,以勞賜花翎。上崩於行在,穆宗即位。御史董元醇疏請兩宮皇太后垂簾聽政,載垣、端華、肅順等持不可,翰附之,抗言甚力,遂黜元醇議。肅順曰:「君誠不愧杜文正之子也!」既而載垣等以竊奪政柄被罪,翰連坐,議革職戍新疆,詔原之,褫職,免其發遣。同治五年,卒。

祁俊藻,字春圃,山西壽陽人。父韻士,官戶部郎中,以事系獄。俊藻方幼,隨侍讀書不輟,賦春草詩以見志。嘉慶十九年,成進士,選庶吉士,授編修。道光元年,直南書房。督湖南學政,累遷庶子。十年,以母病陳情歸養,宣宗不許,予假省親。逾年回京,補原官,遷侍講學士。尋復予假省母,不開缺。歷通政司副使、光祿寺卿、內閣學士。母憂歸,十六年,將屆服闋,預授兵部侍郎,督江蘇學政。歷戶部、吏部侍郎,留學政任,未滿,十九年,命偕侍郎黃爵滋視福建海防及禁煙事,連擢左都御史、兵部尚書。迭疏陳總宜駐泉州治防務,改海口砲臺為墩,查禁煙販,捕治漢奸,並禁漳、泉兩府行使夷錢,夾帶私鑄者治罪,嚴懲械斗,並得旨允行。在閩半載,還經浙江,按臺、溫兩府私種罌粟,劾罷臺州知府潘盛;又劾溫州知府劉煜試行票鹽不善,被議,自呈枉屈,戍新疆。時鄧廷楨奏擊英吉利兵船於廈門走之,忌者謂其不實,命俊藻復往按,具陳戰勝狀。回京,仍直南書房。二十一年,調戶部,命為軍機大臣。

二十六年,偕尚書文慶按長蘆鹽運使陳鑒挪撥鹽課,彌補加價,褫其職,歷任鹽政運司議譴有差。二十九年,以戶部尚書協辦大學士,命赴甘肅偕琦善按前任總督布彥泰清查舛誤、縱容家丁,下嚴議。回京,請便道省墓,途次聞宣宗崩,過里門不入。文宗即位,拜體仁閣大學士,仍管戶部。俊藻自道光中論洋務與穆彰阿不合,至是文宗銳意圖治,罷穆彰阿,俊藻遂領樞務,開言路,起用舊臣,俊藻左右之。

咸豐元年,調管工部,兼管戶部三庫事務。二年,復調戶部。廣西匪日熾,出湖南,遂不可制,湖北、江南數省先後淪陷。軍興財匱,議者試行鈔法,又鑄當百、當五百大錢,皆行之未久而滋弊。尚書肅順同掌戶部事,尚苛刻。又湘軍初起,肅順力言其可用,上鄉之,俊藻皆意與齟,屢稱病請罷,溫詔慰留。四年冬,復堅以為請,乃允致仕。十年,英法聯軍犯天津,車駕將幸熱河,俊藻密疏切諫。又言關中形勝可建都,釐捐病民,北省尤宜急停,並報聞。

十一年,穆宗即位,特詔起用。疏陳時政六事:曰保護聖躬以崇帝學;曰綏輯民心以清盜源;曰重守令以固民心;曰開制科以收人才;曰速剿山東、河南賊匪,嚴防山西、陜西要隘,以衛畿輔;曰敦崇節儉以培元氣。言甚切摯,並被嘉納,次第施行。命以大學士銜授禮部尚書。同治元年,穆宗入學,命直弘德殿,偕翁心存、倭仁、李鴻藻同授讀,摘錄經史二帙進呈。上讀大學畢,俊藻具疏推陳為人君止於仁之義,略曰:「大學一書,皇上已成誦,凡制治保邦之道,用人行政之源,胥在於是。為人君之道,止於仁而已。治國平天下兩章,言仁者六,終之以未有上好仁而下不好義。蓋仁者必以仁親為寶,故能愛人,能惡人。不好仁,則好人之所惡,惡人之所好。仁者必以貪為戒,故忠信以得之,不仁者則驕泰以失之矣。仁者以義為利,不以利為利,故以財發身,不仁者則以身發財,菑害並至矣。千古治亂之機,判於義利,而義利之判,則由於上之好仁不好仁也。如近日所講帝鑒圖說,下車泣罪,解網施恩,澤及枯骨等事,斯即帝王仁心所見端也。若納諫求賢,尊儒遠佞,則仁親為寶,能好能惡之說也。露臺罷工,裘馬卻獻,則以義為利,不以利為利之說也。帝鑒圖說講畢,請進講輿地,以會典諸圖簡明,易於指畫。又耕織圖及內府石刻宋馬遠豳風圖為農桑衣食之原,皇上讀書之暇,隨時講求,庶知稼穡之艱難,懍守成之不易也。」

二年,上服除,俊藻偕倭仁、李鴻藻上疏曰:「皇上沖齡踐阼,智慧漸開。當此釋服之初,吉禮舉行,聖心之敬肆於此分,風會之轉移即於此始,則玩好之漸可慮也,游觀之漸可慮也,興作之漸可慮也。嗜好之端一開,不惟分誦讀之心,海內之窺意旨者,且將從風而靡。安危治亂之機,其端甚微,所關甚鉅,可無慎乎?方今軍務未平,生民塗炭,正君臣交儆之時,非上下恬熙之日。伏原皇上恪遵慈訓,時時以憂勤惕厲為心,以逸樂便安為戒。凡內廷服御一切用項,稍涉浮靡,概從裁減;向例所有,不妨量為撙節。如是,則外務之紛華不接於耳目,詩書之啟迪益斂夫心思,聖學日新,聖德日固,而去奢崇儉之風,自不令而行矣。」疏上,優詔褒答焉。

俊藻提倡樸學,延納寒素,士林歸之。疏言:「通經之學,義理與訓詁不可偏重。後學不察,以訓詁專屬漢儒,義理專屬宋儒,使畫分界限,學術日歧。」因舉素所知寒士端木埰、鄭珍、莫友芝、閻汝弼、王軒、楊寶臣,經明行修,堪資器使。又疏言:「軍興以來,不講吏治,請下中外大臣,保舉循吏及伏處潛修之士,以備任用。」自舉原任同知劉大紳、按察使李文耕、大順廣道劉煦,請宣付史館入循吏傳。又薦直隸知縣張光藻、陳崇砥、王蘭廣,山東知縣蔣慶第,山西知縣程豫、吳輝祖及江南優貢端木埰,山西舉人秦東來。並嘉納允行。屢以病乞休,三年,詔致仕,食全俸。五年,卒,晉贈太保,祀賢良祠,命鍾郡王奠醊,謚文端。擢其子編修世長以侍讀用。

世長,字子禾。咸豐十年進士。年十三,侍父江蘇學政任,幕客俞正燮、張穆、苗夔諸人,並樸學通儒,世長濡染有素,尤篤守宋儒義理之說。同治九年,服闋,補侍讀。累遷內閣學士。光緒初,連督安徽、順天、浙江學政,清勤愛士,一守俊藻舊規。歷禮部、吏部侍郎,擢左都御史。十年,命偕尚書延煦勘山東河工,疏言:「非疏海口不能洩盛漲。修防以民墊為第一層屏障,守民墊即以守大堤。巡撫陳士傑築修民墊多在大堤既決之後,殊為失計。請乘時興修。」從之。迭疏陳時務,多持正議。十六年,遷工部尚書,兼管順天府尹。兩典會試,皆得士。世長清操自勵,累世官卿貳,家如寒素,時以稱焉。十八年,卒,優詔賜恤,謚文恪。賜其孫師曾員外郎,子友蒙主事。

翁心存,字二銘,江蘇常熟人。父咸封,官海州學正。知州唐仲冕見心存有異才,奇之,授之學。道光二年,成進士,選庶吉士,授編修。大考擢中允,督廣東學政。任滿,入直上書房,授惠郡王讀。尋督江西學政,累遷大理寺少卿。十七年,復直上書房,授六阿哥讀。逾年,以母老乞養。家居十年,終母喪。會子同書督貴州學政,陛辭,宣宗命傳諭促之來。二十九年,至京,仍入直,授八阿哥讀。補祭酒。歷內閣學士、工部侍郎,調戶部。江蘇巡撫請蘇州、松江、太倉漕米改徵折色,心存謂:「三屬額徵米一百十四萬餘石,一旦改折,慮京倉不敷支放,州縣假折色抑勒倍徵,便民適以累民。」主駁議,事乃寢。

咸豐元年,擢工部尚書。三年,江寧陷,心存疏陳兵事,請乘賊勢未定,飭向榮渡江,陳金綬進屯浦口,以上海水師溯流沖其前,江忠源、鄧紹良之師掩其後,四路進攻;增重兵守江、淮杜北竄;急清兗、豫、鳳、潁捻匪,毋令與粵寇合勢;並覈軍需,恤災黎;籌京倉積貯,整飭紀綱,以維根本。疏上,多被採用。又薦湖北按察使江忠源,請畀統帥重任,尋即擢為巡撫。調刑部,再調工部,兼管順天府尹。

粵匪北犯,心存疏言賊氛逼近,請扼河而守,畿南直駐重兵,河南、山西、陜西各要隘並力堵截,速調駐熱河、綏遠之蒙古馬隊進口內衛京畿;京師九門嚴緝奸宄,運通倉存糧入城;並敕琦善、鄧紹良規復揚州、鎮江,為會剿江寧之計。又疏陳順天防務,畫分汛地,舉行團練;府屬各營舊隸總督管轄,請旨暫歸調遣。未幾,賊犯天津,僧格林沁率師進剿,命順天府設糧臺。心存請發內帑三十二萬兩、京倉米二千六百石以給軍食,添制軍需火藥。又偕團防大臣會議京城防守事宜,舉光祿寺卿宋晉、太僕寺卿王茂廕綜理其事,並詔允行。時議行鈔幣,心存疏言:「軍營搭放票鈔,諸多窒礙。鈔幣之法,施行當有次第,此時甫經頒發,並未試用,勢難驟用之軍營。」詔斥為阻撓,即責籌次等施行之法,俾無阻滯。會言官論通州捕役勾結土匪行劫,命刑部侍郎文瑞鞫得實,心存以徇庇革職。

四年,起授吏部侍郎,調戶部,擢兵部尚書,調吏部。六年,疏陳江南軍事,略曰:「蘇、松、常、太三府一州,及浙之杭、嘉、湖三府,久為賊所窺伺。今寧國先陷,逼近宜興,向榮近守丹陽,溧水、句容相繼失守,宜責向榮嚴扼丹陽,令張國樑率精兵駐宜興扼東壩,別簡水師駐太湖,庶蘇、常兩郡可保無事。又近有按畝捐輸,失政體,竭民財,請查明停止。」是年冬,兼翰林院掌院學士,以吏部尚書協辦大學士,尋調戶部。

八年,充上書房總師傅。英法聯軍北犯,天津戒嚴。心存疏請聖駕還宮,以定眾志,力言京師重地,不可駐外國領事;長江形勢不可失;綏芬邊地不可捐;兵費不可再償;傳教不可推廣;和議難成,宜速進剿。湖北巡撫胡林翼奏除漕務中飽之弊,請改徵折色。心存力贊其議;由部定章程五事,滿、漢兵糧折價支給,上下衙門一切陋規概行裁革焉。拜體仁閣大學士,管理戶部。與肅順同官不相能,屢乞病,不許。九年,復固請,乃予告去職。

十年,戶部迭興大獄,肅順主之,多所羅織。怡親王載垣等會鞫,謂司員忠麟、王熙震以短號鈔兌換長號,曾面啟心存,心存回奏部院事非一二人所能專政,斷無立談數語改舊章之理。載垣等遂請褫頂帶歸案訊質,文宗鑒其誣,僅以失察議處,免傳訊,議降五級,改俟補官,革職留任。復以五「宇」商號添支經費,心存駁令議減,未陳奏,司員即列入奏銷,下嚴議,革職留任。是年秋,車駕將幸熱河,心存上疏切諫。

十一年,文宗崩於行在,梓宮還京,心存偕諸臣迎謁,特詔起用,以大學士銜管理工部。疏舉人材,詔嘉其不失以人事君之義。又疏言:「東南之民鄉義甚堅,各郡縣陷後,流亡渡江者,日夜思招練義勇,克復鄉里。請敕曾國籓擇能辦賊者馳赴通州東臺,收拾將散之人心,激勵方興之義旅,進搗蘇、常,退保下河。上海一隅賦稅所出,宜取江海關無窮之利,以供曾國籓有用之兵。」疏上,被嘉納。同治元年,入直弘德殿,偕祁俊藻等授穆宗讀。兩宮皇太后慎重師傅之選,倚畀彌篤。是年冬,寢疾,子安徽巡撫同書方緣事系獄,詔暫釋侍疾。尋卒,優詔賜恤,稱其「品端學粹,守正不阿」,贈太保,入祀賢良祠,謚文端。賜其孫曾源進士,曾榮舉人,曾純、曾桂並以原官即用,曾翰賜內閣中書。逾年,文宗實錄告成,以心存曾充監修總裁,賜祭一壇。子同書、同龢自有傳,同爵官湖北巡撫。

彭蘊章,字詠莪,江蘇長洲人,尚書啟豐曾孫。由舉人入貲為內閣中書,充軍機章京。道光十五年,成進士,授工部主事,仍留直軍機處。累遷郎中,歷鴻臚寺少卿、光祿寺少卿、順天府丞、通政司副使、宗人府丞。督福建學政,遷左副都御史。二十八年,疏言:「漕船衛官需索旗丁日益增多,沿途委員及漕運衙門、倉場花戶皆有費,欲減旗丁幫費,宜探本窮源。又州縣辦漕,應令督撫察其潔己愛民者,每歲酌保一二員;辦理不善者,劾一二員。運漕官及坐糧如能潔己剔弊,準漕督、倉場保奏,不稱職者劾罷。」下部議行。擢工部侍郎,仍留學政任。咸豐元年,命在軍機大臣上行走。四年,調禮部,尋擢工部尚書。五年,協辦大學士。六年,拜文淵閣大學士,管理工部及戶部三庫事務,充上書房總師傅。

八年,京師旱,糧價踴貴,旗民生計益艱,蘊章奏請撥款採米,允之。復疏言:「自改用大錢,城中米貴,疊荷加恩賑濟,又加米折,然民生疾苦未見轉機。臣聞兵丁所領止有實米二成,其餘折色定價,每石京錢四千至三千不等,大米一石市價京錢三十千。持此折價買米,不過升斗。民生之蹙,不獨在無銀,並在無米。本年海運多於上年,可將兵米酌量加增。又各營養育兵及鰥寡孤獨小口米不過四萬餘名,每名歲支一石六斗,擬請此項酌給米,毋庸折色。自前年以來,有提存部庫採買銀,又存四川、山東、山西、河南、陜西解京米價銀,共有四十七萬餘兩,堪以採買米石,加放兵米。又有河南停運節省運腳銀二萬兩,堪為轉運之用。伏乞飭部採買,以資搭放,實於旗兵生計大有裨益。」疏入,下部議行。

蘊章久直樞廷,廉謹小心,每與會議,必持詳慎。鈔票、科場諸大獄,婉體調護,與肅順等意忤。兩江總督何桂清素以才敏自負,蘊章誤信之,數於上前稱薦。十年,江寧大營潰,蘊章猶言桂清可恃。未幾,蘇、常相繼陷,桂清逮治。文宗以蘊章無知人鑒,眷注浸衰。適有足疾,扶掖入直,命毋庸在軍機大臣上行走,以示體恤。尋奏乞罷職,出都就醫。詔曰:「卿久任樞垣,備悉時事。現在軍務如有見及,並採訪輿論民情,隨時具疏交地方官大吏代遞。」蘊章密陳時務六則,報聞。十一年,病痊,署兵部尚書,尋兼署左都御史。同治元年,復以病乞休。未幾,卒,依大學士例賜恤,謚文敬。子祖賢,官至湖北巡撫。

論曰:文宗初政,杜受田以師傅最被信任,贊畫獨多。祈俊藻、彭蘊章皆久領樞務,翁心存數論軍事,久筦度支。三人者並與肅順不協,先後去位;同治初元,聯翩復起。俊藻、心存三朝耆碩,輔導沖主,一時清望所歸焉。


\end{pinyinscope}