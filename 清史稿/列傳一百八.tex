\article{列傳一百八}

\begin{pinyinscope}
裘曰修吳紹詩子垣壇閻循琦王際華曹秀先

周煌子興岱曹文埴杜玉林王士棻金簡子縕布

裘曰修,字叔度,江西新建人。乾隆四年進士,改庶吉士。自編修五遷至侍郎,歷兵、吏、戶諸部。胡中藻以賦詩訕上罪殊死,事未發,曰修漏言於鄉人。上詰曰修,不敢承,逮所與言者質實,上謂「曰修面欺。」二十年五月,下部議奪職,左授右中允。十二月,擢吏部侍郎。二十一年,令在軍機處行走。師討準噶爾,命如巴里坤董軍儲。二十二年,疏言:「西陲回民數十部落,厄魯特人介其中。當策妄阿喇布坦時恣殺掠,回民久切齒。請敕伯克額敏和卓,厄魯特竄入境當擒戮,予賞賚,勿被煽生疑懼。」尋還京師。

河屢決山東、河南、安徽境,積水久不去。是歲上南巡蒞視,既返蹕,命曰修會山東、河南、安徽諸巡撫周行積水諸州縣,畫疏濬之策。曰修至安徽,偕巡撫高晉疏言:「安徽宿、靈壁、虹三州縣頻年被水,上承河南虞城、夏邑、商丘、永城四縣積水,下注畢匯於宿州。宿州有睢河,虹縣有潼河,泗洲與宿遷、桃源接壤處有安河,皆境內大水,與靈壁、虹縣諸支港當次第疏濬,俾入洪澤湖。洪澤以清口為出路,上令去草壩使暢流,江南之民,仰頌聖明,宜令每歲應期開放。」

曰修至河南,偕巡撫胡寶瑔疏陳:「黃河南岸,自滎澤以下諸水,東入睢,東南入淮,皆淺阻不能宣洩。東境幹河,在商丘為豐樂河,在夏邑為響河,在永城為巴河,實即一水,次則賈魯河,又次則惠濟河、渦河,皆當疏濬。自永城至汝寧府支河當施工者凡十二,導積水自支河入於幹河。其不能達者,或多作溝渠,或渟為藪澤,潢汙野潦,有所約束而不為民害。」

曰修至山東,偕巡撫鶴年疏請培館陶、臨清濱運河諸州縣民墊,官給夫米,令實力修補。復偕巡撫蔣洲疏言:「山東當疏濬諸水,以兗州為要,曹州次之。兗州宜治者九水,曹州西南境當濬順堤河,東北境當於八里廟建壩,俾沙河、趙王河水入運,賴以節宣。」曰修諸議皆稱上意,命及時修築。

曰修復至安徽,議濬潁州府境與河南連界者六水,在府境者四水,加疏宿州境睢河,並寬留清口壩口門。上獎所議甚合機宜。還河南,諸幹河工竟,議續濬商丘、遂平、上蔡、新蔡諸支流凡五水,並築諸堤堰。調戶部侍郎。二十三年,諸水畢治,禦制詩褒之。疏言:「諸行省偏災,米豆例免稅。但以免稅故,稽查繁密。欲通商而商反以為累,卻顧不前。請如常收稅。」下九卿議行。京師平糶,曰修言糶價過減,適令商家乘機居積,請石減百錢,數日後市價稍平,以次漸減。會天津民訟鹽商牛兆泰,兆泰與曰修有連,曰修嘗寄書,上命不必在軍機處行走。二十五年,授倉場侍郎。

二十六年,河決楊橋,命如河南勘災賑,並議疏洩。曰修請廣設粥廠,饑民便就食;量增料價,料易集,工可速蕆:上皆可其奏。上遣大學士劉統勛、兆惠督塞河。曰修勘下游,疏言:「黃水悉入賈魯、惠濟二河,二河倘不能容,為患滋大。宜察堤墊為河水所從入,悉堵御,俾中流不至復決。」曰修還楊橋,疏言河流逼北岸,當挽行中道;又請培補沁水堤,並賑流民:得旨嘉允。曰修子編修麟,卒於京師。上念曰修所領事將竟,有子喪,母老,召還京師。工竟,上制中州治河碑,褒曰修及寶瑔不惜工,不愛帑,不勞民,上源下流,以次就治。旋居母喪,歸。

二十八年,上以直隸連年被水,曰修服將除,召來京督直隸水利。署吏部侍郎。河渠工畢,曰修請迎生母就養。上令會高晉籌濬睢河,曰修言當厚蓄清水以刷淤泥,秋冬水弱,南北築壩堵截,至四月水漲,啟壩分洩,上採其議。二十九年,福建提督黃仕簡疏論總督、巡撫得廈門洋行歲餽,命曰修偕尚書舒赫德往按,並命曰修暫署福建巡撫。讞定,還京師,署倉場侍郎。三十年,授戶部侍郎。

三十一年,上以江南淮、徐諸河堤前令曰修等經營修築,為時已久;復命曰修及高恆往勘山東、河南毗連處,並令巡視。曰修等疏言:「諸水自二十二年大治後,歲於農隙疏濬,堤岸亦以時培補,現無淤墊殘缺。」報聞。遷尚書,歷禮、工、刑三部。三十三年,丁生母憂,歸。三十四年,召授刑部尚書。初,江南、山東蝗起,命曰修捕治。是歲畿南蝗,復命捕治。曰修至武清,令順天府尹竇光鼐行求蝗起處。上責曰修不親勘,左授順天府府尹。尋遷工部侍郎。

三十六年,命如滄州勘運河,疏請改低壩基殺水勢,疏下流引河,移捷地閘,裁曲就直,疏減河使順流達海,上從之。遷工部尚書,命南書房行走。命督濬北運河。三十七年,又命督濬永定、北運諸河,疏言:「治河不外疏築,而築不如疏。直省近水居民與水爭地,水退即占耕,升科築墊。有司見不及遠,以為糧地自當防護,逼水為堤墊墊,水乃橫決為災。請敕所司,澱泊毋得報墾升科,橫加堤墊,使水有所歸。」上降旨嚴禁。

三十八年四月,曰修病噎乞歸,上以「錢陳群嘗病此,以老許其歸;今曰修方六十,不當如陳群之引退。」賜詩慰之,屢遣存問,御醫視疾。旋加太子少傅。卒,謚文達。子行簡,自有傳。

吳紹詩,字二南,山東海豐人。諸生。雍正二年,世宗命京官主事以上、外官知縣以上,舉品行才猷備任使,即親戚子弟不必引避。時紹詩世父象寬官湖北黃梅知縣,遂以紹詩應詔,引見,分刑部學習。十二年,授七品小京官。乾隆初,累遷至郎中。外擢甘肅鞏昌知府,遷陜西督糧道。總督永常劾紹詩採兵米侵帑,奪職,下巡撫鍾音鞫治。紹詩以市米貴賤不齊,為中價具報,非侵帑。狀聞,發軍臺效力,以母病許贖。

二十二年,高宗南巡,紹詩迎蹕。起貴州督糧道。遷雲南按察使。調甘肅按察使,就遷布政使。疏言寧夏駐防將軍以下官祿應給粳米,請改徵諸民應納粟米石者,改交粳米七斗,上命寧夏駐防官祿如涼州、莊浪例,改折價。又疏鎮番縣柳林湖招墾地,請如安西瓜州屯田例,升科納賦,較前此徵租歲計有盈,且民戶世業,俾可盡心耕耨,下總督楊應琚等議行。甘、涼諸縣旱,紹詩復疏言張掖、永昌、鎮番、碾伯、高臺五縣舊無城,撫彞、隆德、涇州城已損壞,請以時修築,使饑民就工授食,下巡撫常鈞議行。旋以憂歸,三十一年,服除,擢刑部侍郎。

出為江西巡撫。以南昌、九江二衛屯田租過重,贛州、袁州、鉛山三衛所租重而田缺,疏請減租,下總督高晉詳勘量減。上猶產鐵砂,民爭取滋事,疏請募民淘採,募商設廠收鎔,為之條例。九江關監督舒善、建昌府知府黃肇隆皆以不職為上聞,責紹詩不先事論劾,部議奪職,命寬之。三十四年,召為刑部尚書,未上,調禮部尚書。是歲南昌等縣被水,十月,紹詩將受代,始奏請緩徵。上諭曰:「災地收薄,小民豈能復事輸將?紹詩遷延不問,直至開徵將及一月,始以一奏塞責。現雖傳諭停緩,急公者納糧不免拮據,疲窘者徒受催科之累。此皆紹詩全不知以民事為重有以誤之也。紹詩累經部議降革,並從寬留任。此則玩視民瘼,難復曲貸。」因命奪職。

三十五年,起刑部郎中,三十六年,擢侍郎。皇太后八十萬壽,列香山九老,賜以宴賚。三十七年,調吏部侍郎。三十九年,乞致仕。四十一年,上東巡,迎蹕,加尚書銜。卒,年七十八,謚恭定。子垣、壇。

垣,自舉人入貲授兵部郎中,三十五年,特命調刑部。三十六年,紹詩為侍郎,上以垣本特調,命毋回避。三十七年,弟壇為侍郎,乃調吏部。遷監察御史,以憂歸。服除,補原官。遷給事中。以弟壇為巡撫,例不為言官,署吏部郎中。壇卒,復為給事中。五遷為吏部侍郎。四十九年,外授廣西巡撫。五十年,入覲,與千叟宴。調湖北巡撫。江夏等州縣旱,疏請緩徵平糶,募商赴四川買米。五十一年,卒,上賜恤,猶獎其實心治災賑也。

壇,二十六年進士,授刑部主事,再遷郎中。三十一年,紹詩為侍郎,上以壇治事明敏,毋回避。三十二年,超授江蘇按察使,就遷布政使。江寧、蘇州兩布政所屬,互支官俸兵米,壇疏請更定;江蘇賦重甲諸行省,每遇奏銷,款目繁衣復,壇疏請分別總案、專案,以便察覈:皆議行。三十七年,內擢刑部侍郎。三十九年,太監高雲從以洩道府記載誅,京朝諸臣從問消息者皆奪職,壇亦與。上謂:「不意壇竟至於此!念其練習刑名,廢棄可惜。左授刑部主事。」遷郎中。四十四年,授江南河庫道,遷江蘇布政使。四十五年,擢巡撫。疏言:「吳縣舊有公田萬二千五百畝,銀漕外歲納租息佐轉漕,逋租甚鉅。以非正賦,遇蠲免不得與。請並予豁除,災歉隨賦蠲緩。」又疏言:「江、河險處設救生船五十六,今裁存二十八。請增募四十,分泊京口、瓜州、金山諸處。」並從之。旋卒。

紹詩父子明習法律,為高宗所器。紹詩兩為侍郎,垣、壇在後在郎署,特命毋相避。及紹詩移貳吏部,以壇繼其後。父子相代,尤異數。乾隆初,重修大清律例,紹詩充纂修官,綱目二卷,實所釐定。壇復著大清律例通考三十九卷。

閻循琦,字景韓,山東昌樂人。乾隆七年進士,改庶吉士。散館,授工部主事。三遷廣東道御史,仍兼工部行走。疏言:「江南諸行省水災治賑,應照戶口秤定銀封。主其事者每假手胥吏,不能無扣減,甚或私用輕戥。宜令督撫派專員監封,仍令道府以時抽驗。貧民以銀易錢買米,當禁奸民剝削。富家積錢,亦應令其散易,以平市價。」上曰:「循琦所言,頗中情弊。但若明降諭旨,不肖者未必畏憚;本無此弊者,或轉因此啟其舞弊。當抄循琦奏寄諸行省督撫,令加意體察。」又疏言八旗義學教習多不實心督課,請歲派大臣會禮部堂官嚴察,上為罷八旗義學,令董理各官學大臣盡心教育。遷轉吏科掌印給事中。

三十四年,特命兼吏部文選司郎中。遷內閣侍讀學士,仍兼吏部行走。京西門頭溝煤窯歲久淤塞,有議他處營採者,因緣為利,命循琦會勘。謂舊窯產煤本旺,鑿溝隧,疏積水,淤去而煤暢;他處有可採,當以時招商。議上,大學士傅恆覆奏如循琦言。三十六年,超擢工部侍郎。會試知貢舉,事畢入對,上問:「諸臣知貢舉每有條奏,汝獨無,何也?」循琦對:「科場條例已甚詳備,諸臣實力奉行自足,不敢毛舉一二端自謂曉事也。」上曰:「汝言是。凡事皆當如此,非獨知貢舉而已。」三十八年,遷工部尚書。四十年,卒,贈太子太保,謚恭定。

王際華,字秋瑞,浙江錢塘人。乾隆十年一甲三名進士,授編修。十三年,大考翰詹,擢侍讀學士、上書房行走。廣東舊設兩學政,十五年,以侍讀程巖督廣韶學政,際華督肇高學政,旋用巖議裁並,以憂歸。服除,起原官。三遷至侍郎,歷工、刑、兵、戶、吏諸部。在兵部,疏言:「武鄉會試舊例,外場挑雙好、單好、合式三類入內場,雙、單好列東號,合式列西號。不肖者見列西號,知不能幸中,紛紛求出。即有歸號,終日喧嘩。請嗣後武鄉會試,但挑雙、單好,毋更挑合式。」在吏部,疏請在京文武官吏議處,及各部會議外省文武官吏議處,當分別定限,皆如所議。三十四年,遷禮部尚書。三十八年,加太子少傅,調戶部尚書。四十一年,卒,贈太子太保,謚文莊。賜其子朝梧內閣中書,官至山東兗沂曹道。

程巖,字巨山,江西鉛山人。以檢討督廣東肇高學政,移督廣韶學政。建議裁並,即以命巖。官至禮部侍郎。

曹秀先,字恆所,江西新建人。乾隆元年,舉博學鴻詞,未試,成進士,改庶吉士,授編修。十年,遷浙江道御史。十七年八月,舉恩科會試,秀先從子詠祖坐關節誅,秀先當奪職,上以秀先初不與知,但失察,命寬之。十八年,近畿蝗,秀先請御制文以祭,舉蠟禮;州縣募捕蝗,毋藉吏胥。上曰:「蝗害稼,惟實力捕治,此人事所可盡。若欲假文辭以期感格,如韓愈祭噩魚,噩魚遠徙與否,究亦無稽。朕非有泰山北斗之文筆,好名無實,深所弗取。」下部議,罷蠟禮,餘如所請。七遷至侍郎,歷工、戶、吏諸部。三十九年,遷禮部尚書、上書房行走,命為總師傅。四十六年,禮部議四十七年祀祈穀壇日用次辛。上曰:「朕御極以來,遇正月上辛在初三日前,當隔歲齋戒,改用次辛。其有初四日上辛亦改次辛者,以為聖母皇太后祝釐,朕率王公大臣拜賀東朝,禮不可闕。至明歲正月上辛,則非向年可比矣。如謂不敢輕易朝正令典,亦當備稽往例,具奏請旨。乃遽行題達,何昧昧至此!」禮部堂官悉下部議,秀先當奪職,復命寬之。四十七年,罷上書房總師傅。四十九年,卒,贈太子太傅,謚文恪。

秀先少孤,事母胡孝,嘗為吮疽。母卒,庶母龔為攜持,事如母。學於兄茂先,事之如嚴師。既貴,收宗族,弭鄉里水患。蒞政勤慎廉儉,罣吏議數四,輒命減免。秀先顏其堂曰「知恩」,紀上眷也。

子師曾,自兵部郎中屢遷至侍郎,歷禮、兵二部。嘉慶二十五年,以兵部失行在印,左授太常寺少卿。道光初,再遷太常寺卿。請修墓,歸。卒。

周煌,字景垣,四川涪州人。乾隆二年進士,改庶吉士,散館授編修。二十年,命偕侍講全魁冊封琉球國王尚穆。尋遷右中允,再遷侍講。二十二年,使還,奏上琉球國志略,命以武英殿聚珍板印行。以從兵在琉球失約束,下吏議,當奪官,上以煌遠使,且在姑米山遇風險,命寬之,仍留任。二十三年,大考二等,開復。尋遷左庶子,命上書房行走。累遷兵部侍郎。三十八年五月,命如四川按壁山民訟武生勒派;十月,復命如四川按蓬溪諸生訟縣吏勒派:俱鞫虛,罪如律。四十四年,擢工部尚書。四十五年,調兵部尚書。四十六年,上幸熱河,煌詣行在入對。四川方多盜,號為侂嚕子。總督文綬疏報,遣將吏捕治。上以諮煌,煌對:「侂嚕子所在多有,縣輒百十人,其渠號『朋頭』。白日劫掠,將吏置不問。甚且州縣胥役亦為之,大竹縣役子為盜渠,號一隻虎。」上為罷文綬,調福康安督四川,命防護煌所居村。四十七年,命為上書房總師傅,未逾年,以煌不勝總師傅,罷之。四十九年,調左都御史。五十年,以病乞休,詔以兵部尚書加太子少傅致仕。尋卒,進太子太傅,賜祭葬,謚文恭。

子興岱,字冠三。乾隆三十六年進士,改庶吉士,散館授編修。累遷侍講學士。超授內閣學士。擢侍郎,歷禮、吏、戶諸部。命在南書房行走。嘉慶四年,祭告川、陜岳瀆。川、楚教匪亂方急,上命興岱經被寇州縣宣諭慰恤,並傳詔招撫;復以軍中諸將勇怯諮興岱。興岱奏:「臣行次廣元,民言總兵硃射鬥在高院場戰敗,總督魁倫未遣兵應援,又不嚴守潼關。賊夜掠太和鎮,焚殺甚酷。行次梓潼,賊正擾縣境,民紛紛徙避。臣在縣督率嚴防,駐二日乃行,途中宣上指慰諭。民言川軍逐賊,德楞泰最奮勇,且能於臨陣廣布德意,解散脅從。但賊勢方張,一人不能兼顧。請敕督兵諸大臣同心協力。」上奪魁倫官,逮詣成都,命興岱會勒保按鞫。事畢,還京師。煌嘗兩使四川按事,興岱復繼之,時以為榮。六年,充江西考官,坐受餽,並索取衣裘,命退出南書房,左授侍讀學士。八年,大考,以老乞休,上從之。旋復授編修,遷侍講。擢內閣學士,復再遷左都御史。十四年,卒。

曹文埴,字竹虛,安徽歙縣人。乾隆二十五年二甲一名進士,改庶吉士,授編修。直懋勤殿,四遷翰林院侍讀學士,命在南書房行走。再遷詹事府詹事。居父喪,歸。四十二年,詣京師,謁孝聖憲皇后梓宮。喪終,仍在南書房行走。授左副都御史。遷侍郎,歷刑、兵、工、戶諸部,兼管順天府府尹。軍機章京、員外郎海升毆殺其妻,以自縊報,其妻弟貴寧爭非是。命左都御史紀昀等驗尸,仍以自縊具獄。貴寧復爭言:「海升與大學士阿桂有連,驗不實。」更命文埴與侍郎伊齡阿覆驗,得毆殺狀,以聞。上獎文埴等不徇隱,公正得大臣體。阿桂以嘗奏及語袒海升,坐罰俸,昀下吏議,刑部侍郎景祿、杜玉林及郎中王士棻等皆遣戍。擢文埴戶部尚書。復命與伊齡阿如通州督漕政,漕船回空較早,命議敘。

五十一年,命如浙江察倉庫虧缺。旋復命阿桂會文埴董理。浙江濱海建石塘,外積柴為障,是為柴塘。外又累土為坡以護,是為坦水。巡撫福崧疏請籌歲修,命文埴並按。文埴言:「柴塘日受潮汐,往來汕刷,勢不能無蹲蕣。今既為坦水,若不以時補修,不足當潮勢而為石塘之保障。」得旨,如所議。文埴還京師。上以阿桂及文埴鞫平陽知縣黃梅未得實,下部議,降二級,命寬之。

五十二年,文埴以母老乞歸養,俞其請,加太子太保,御書賜其母。五十四年,上以明年八十萬壽,命文埴毋詣京師。文埴疏言:「母健在,明年當詣京師祝嘏。至時如未能遠離,當自審度。上體聖意,下順親心,諸事皆從實。」得旨:「卿能來,朕誠喜,但毋稍勉強。」五十五年,文埴詣京師祝嘏,上賜文埴母大緞、貂皮。五十六年,御試翰詹,文埴子編修振鏞列三等。上以才可造,又為文埴子,擢侍講。寄賜文埴禦制文勒石拓本。六十年,以上御極周甲子,文埴詣京師賀,上復賜文埴母御書、文綺、貂皮。嘉慶三年,卒。高宗方有疾,恤典未行。五年,仁宗命予恤,謚文敏,並賜文埴母大緞、人參。

乾隆之季,和珅專政,嫉阿桂功高位其上。海升妻之獄,辭連阿桂。和申妄謂文埴能立異同,欲引以為重。文埴特持正,故非阿和珅,母老決引退,恩禮弗替。子振鏞,自有傳。

杜玉林,字凝臺,江蘇金匱人。乾隆十九年進士,授刑部主事,再遷郎中。外授江西南康知府,三遷四川布政使。四十四年,內擢刑部侍郎。四十五年,命如四川按會理州沙金鳳訴其兄土司金龍占田獄。讞定,金鳳復詣京師呈訴,覆讞如玉林議分田,惟獄情未盡,又知州徐士勛當劾,玉林以同鄉置不問。吏議當左遷,上授玉林工部侍郎,仍領刑部事。旋復還刑部,迭使湖南北、江南讞獄。尚書福隆安僕笞殺役夫,賄他人自代,玉林不能察,降三品冠服。旋命復本秩。五十年,坐海升妻獄,戍伊犁。明年,召還。授刑部郎中。行至涇州,卒。

玉林善治獄,嘗曰:「刑一成而不變。治律例猶善醫,貴不泥於方書,而察其受病之實。不如是無以臨民。」

王士棻,字蘭圃,陜西華州人。乾隆十九年進士,改庶吉士,授刑部主事。再遷郎中。和珅為步軍統領,寵其役,役占通州車行。州民訴刑部,士棻為定讞,戍其役黑龍江。上詣碧雲寺禮佛,訝池涸,問其故。僧言寺後開煤礦,引水別流。上怒,逮主其事者下刑部,則和珅奴也。諸曹憚和珅,不欲竟其獄,士棻復為定讞。上責和珅而誅其奴。五十年四月,海升妻之獄,刑部侍郎杜玉林坐驗尸不以實,當譴。上欲以士棻代,而士棻亦佐驗。上諭曰:「王士棻在刑部年久,前因召對,觀其人尚有才,方欲量加擢用。乃覆驗回護,逢迎阿桂,罪無可逭。」遂與玉林戍伊犁。明年,召還。授刑部員外郎。五十二年六月,特擢江蘇按察使。五十五年,高郵州吏以偽印徵賦,事發,巡撫閔鶚元以下皆坐重譴。上以按察使得奏事,士棻見巡撫以下互相徇隱,置若罔聞,士棻本起廢籍,尤負恩,命奪職;總督書麟等請遣戍,上許納贖。尋復授刑部員外郎。五十七年,以病乞歸。嘉慶元年,卒。

士棻治獄,虛公周密,每有所平反。章丘民辛存義索逋於屠者,死於途,旁置屠刀。縣吏坐屠殺人。士棻奉命詣讞,躬訪於村女,別得罪人,屠乃雪。旗丁有兄弟異母而同居者,兄鰥,弟有婦,夜為人戕,母訴長子奸殺。士棻蒞視,長子伏地哭,無一語。在側指畫者,母之侄也。士棻審視良久,叱其侄曰:「殺人者汝也!」侄股慄具伏。泰安嫠顏氏富而子幼,夫弟強之嫁,走訴部。或餽士棻白金五千,士棻拒之,卒論如律。邳州民有舅訟甥者,謂其發母墓,罪殊死。士棻疑之,為覆讞。蓋甥為前母子,舅則後母兄。後母憎長子,舅誑之曰:「汝母墓有蛇跡。」甥與其妻往視,舅伺叢墓間,執詣縣。士棻得其情,白長子枉。士棻嘗曰:「刑官之弊,莫大於成見。聽訟有成見,強人從我,不能盡其情,是客氣也。斷罪有成見,或偏於嚴明,因求能折獄名;或偏於寬厚,自以為陰德:皆私心也。」高宗知其才,屢坐譴,終不使廢棄,仍俾為刑官。世傳其再起復欲用為侍郎,和珅實尼之云。

金簡,賜姓金佳氏,滿洲正黃旗人,初隸內務府漢軍。父三保,武備院卿。金簡,乾隆中授內務府筆帖式,累遷奉宸院卿。三十七年,授總管內務府大臣。監武英殿刻書,充四庫全書副總裁,專司考覈督催。三十九年,授戶部侍郎,管錢法堂,鑲黃旗漢軍副都統,賜孔雀翎。四十年,奏:「京局鼓鑄,每年七十五卯,錢九十二萬七千三百五十千。歲餘二萬餘千,加以節年餘存,遇閏侭可抵放。請裁去閏月四卯。」從之。四十三年,命纂四庫薈要,署工部尚書。命赴盛京察平允庫項虧短,關防拉薩禮等治罪如律。奏定盛京銀庫章程,下部議行。四十六年,命總理工部。四十八年,擢工部尚書、鑲黃旗漢軍都統。四十九年,請疏濬盧溝橋中泓五孔水道,並請定三、四年疏濬一次。五十年,與千叟宴。四庫全書成,議敘。命修葺明陵,請加築思陵月臺,並拓享殿、宮門。五十六年,故安南國王黎維祁聽所屬黃益曉、黎光霽等稟請歸國,命金簡察治,益曉、光霽等並發遣。五十七年,調吏部尚書。五十九年,卒,令皇孫綿懃奠醊,賜祭葬,謚勤恪。金簡女弟為高宗貴妃。嘉慶初,仁宗命其族改入滿洲,賜姓。

縕布,金簡子。初授拜唐阿,擢藍翎侍衛。乾隆四十八年,授泰寧鎮總兵。六十年,召授總管內務府大臣。嘉慶三年,授鑲紅旗漢軍副都統。四年,授工部侍郎,賜孔雀翎。奏請增設內務府養育兵,上斥其例外乞恩,意在沽名。俄以清字摺誤書孝聖憲皇后徽號,奪官,予四品頂帶,留佐領。旋復授正紅旗蒙古副都統、總管內務府大臣。五年,授兵部侍郎。六年,擢工部尚書、鑲紅旗漢軍都統。九年,署戶部尚書。十四年,卒。

論曰:曰修奉使治水,利澤施於生民;紹詩疏律義,尚平恕:皆有子克承厥緒。循琦、際華、秀先回翔臺省,以篤謹被主知;文埴眷尤厚,不阿時相,潔其身以去:皆彬彬平世令僕才也。乾隆之季,民窮盜起,煌父子言鄉里民間疾苦,高宗不以為忤。金簡起戚畹,所論鑄錢、葺明陵,及黎維祁乞歸國,並關國故,故比而次之。


\end{pinyinscope}