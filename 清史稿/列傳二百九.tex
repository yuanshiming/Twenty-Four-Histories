\article{列傳二百九}

\begin{pinyinscope}
王茂廕宋晉袁希祖文瑞毓祿徐繼畬王發桂

廉兆綸雷以諴陶樑吳存義殷兆鏞

王茂廕,字椿年,安徽歙縣人。道光十二年進士,授戶部主事,升員外郎。咸豐元年,遷御史。疏請振獎人才,鄉會試務覈實,殿試、朝考重文義,造就宗室、八旗人才,以有裨實用為貴。戶部議開捐納舉人、生員例,茂廕疏爭,且言:「籌餉之法,不徒在開源,而在於善用。委諸盜賊之手,靡諸老弱之兵,銷諸不肖之員弁,雖日言推廣捐輸,何濟?」又極論:「銀票虧商,銀號虧國。經國謀猷,下同商賈,體至褻而利實至微。初時虧不能見,及虧折已甚,雖重治其罪,亦復奚補!」其言皆驗。

二年,粵匪自長沙趨岳州,茂廕疏言:「安徽防務,以宿松為要沖,小孤山為鎖鑰。設險非難,得人為難。請起前署廣西巡撫周天爵幫辦防堵,扼要駐守陸路,令府縣勸諭紳民團練守助,用明金聲禦流賊保鄉里之法,最為簡易。」武昌既陷,茂廕又疏言:「賊勢猖獗,宜急收人心,籌儲積,講訓練,求人才。」三年,戶部奏試行鈔法,上命左都御史花沙納與茂廕會議,奏行簡要章程,並繪鈔式以進。疏言:「皖北蒙、亳,捻匪蜂起,萬一粵賊勾結,更為心腹巨患。夫欲平盜賊,尤在守令得人。廬、鳳、潁諸郡,守令貪鄙者,實繁有徒。請嚴飭大吏從嚴劾汰,以治賊之源。」又曰:「兩湖、江、皖處處言防,而處處不守。請嚴飭各督撫專主剿辦,一處賊平,則他處之賊不敢復起;鄰省賊滅,則本省之賊無自而來。是不言防而防自固也。」三月,揚州陷,茂廕疏言:「寇氛將偪山東,巡撫以剿賊出省,籓臬漫無布置,城內團丁不滿七百。乞特簡重臣防守,以固畿南屏蔽。」又言:「陜西設防,兵為民害,請諭飭按治。」茂廕屢上疏,言事侃侃,文宗頗鄉用。擢太常寺卿,遷太僕寺卿。

粵匪犯畿輔,參贊大臣科爾沁親王僧格林沁駐師涿州,諸軍咸觀望不肯前。茂廕疏言:「賊既渡滹沱而北,回翔於深、晉之郊,而不遽北犯者,懼吾兵出也。吾兵出而遷延不進,賊有以知我之勇怯矣。臣竊謂賊自桂林北竄,諸帥喪師左次,皆為一守字所誤。賊屯一日,可資休息;我屯一日,銳氣日隳。賊所過劫掠,行不裹糧;我軍坐食縣官,日需鉅餉。相持數月,餉絕兵匱,不待交綏而勝負已判。請密飭王大臣等明發號令,按兵拒守,而陰選健將率死士數千,潛師出彼不意,麾兵急擊,一鼓可殲。如此,則大河以南,諸賊心怵膽落,不敢復圖北犯矣。」

尋命會辦京城團防保甲,擢戶部侍郎,兼管錢法堂。戶部奏鑄當十、堂五十大錢,王大臣又請增鑄當百、當千,謂之四項大錢。當千者,以二兩為率,餘遞減。茂廕上疏爭之曰:「大錢之鑄,意在節省,由漢訖明,行之屢矣。不久即廢,未能有經久者。今行大錢,頗見便利,蓋喜新厭故,人情一概。及不旋踵,棄如敝屣。稽諸往事,莫非如是。錢法過繁,市肆必擾,折當過重,廢罷必速,此人事物理之自然。論者謂國家此制,當十則十,當千則千,孰敢有違?不知官能定錢直而不能定物直,錢當千,民不敢以為百;物直百,民不難以為千。自來大錢之廢,多由私鑄繁興,物價騰踴。宋沈畸之言曰:『當十錢鑄,召禍導奸,游手之徒,爭先私鑄。無故而有數倍之息,雖日斬之,勢不可遏。』張方平之議曰:『奸人盜鑄,大錢之用日輕。比年以來,虛高物估,增直於下,取償於上,有折當之虛名,罹虧損之實害。』大觀錢鑄自蔡京,而其子絳作國史補敘:『始之得息流通,繼之盜鑄多弊,終之改當折閱。』事皆目睹,尤為詳盡。古所不能行,而謂可通行於今乎?信者國之寶。大錢鈔票,皆屬權宜之計,全在持之以信,庶可冀數年之利。今大錢輕重程式,甫經頒行,未及數月,忽盡更變。商民惶恐,群疑朝廷為不可信,此非細故也。或慮銅短停鑄,故須及時變通,顧變通欲其能行,不行則亦與不鑄等。逆賊一平,不患無銅,若賊不能平,銅不能運,雖侭現有之銅,悉鑄當千,恐亦無濟,可慮者不僅停鑄而已。」上命王大臣及戶部秉公定議,王大臣終執原議。

四年,戶部會奏推廣大錢辦法,茂廕復疏爭曰:「臣疏陳大錢利弊,未奉諭旨,臣職司錢法,夙夜思維,實覺難行。當百以上大錢,與原行當五十者無甚分別,此何以貴,彼何以賤,難一;以易市物,則難分折,以易制錢,莫與兌換,難二;大錢雖準交官項,然準交五成者,已有寶鈔官票,大錢何能並搭?難三。此猶其小者耳,最大之患,莫如私鑄。奸人以銅四兩鑄大錢兩枚,即抵交官銀一兩,是病國也。蓋行制錢,每千重百二十兩,鎔之可得六十兩,以鑄當十錢可得三十千。設奸人日銷制錢以鑄大錢,民間將無制錢可用,是病民也。寶鈔官票,其省遠過大錢,果能推行盡利,裨益亦非淺鮮,大錢之行,似可已也。」疏入,仍不報。其後大錢終廢,如茂廕言。

又疏論鈔法利病,略曰:「上年初用銀鈔,雖未暢行,亦未滋累。及臘月行錢鈔,至今已發百數十萬,為累頗多。向來鈔法,唐、宋之飛錢、交子、會子,皆有實以運之。元廢銀錢不用而專用鈔,上下通行,為能以虛運實。明專以虛責民,以實歸上,勢遂不行。臣元年所奏,皆以實運虛之法。今時勢所迫,前法不行,議者雖專於收鈔時設法,然京師放多收少,軍營有放無收,直省州縣有收無放,非有商人運於其間,則皆不行。非與商人以可運之方、能運之利,亦仍不能行。」因擬上四事,務在通商情,利轉運。奏入,上斥其為商人指使,不關心於國是,命恭親王奕、定郡王載銓覈議。議上,謂茂廕所論,窒礙難行,嚴旨切責。尋調兵部。

粵匪踞池州、太平,皖南隔絕,茂廕奏請以徽州暫歸浙江統轄,上命浙江巡撫黃宗漢體察酌行。初,茂廕疏言:「賊脅良民,驅為前鋒。請特降諭旨,自拔來歸,均從寬貸。殺賊來獻,均加爵賞。」京師久不雨,上命清釐庶獄,減免情節可矜者,茂廕又疏言:「可矜者莫如賊中逃出之難民,各處捕獲難民,指為形跡可疑,嚴訊楚毒。此輩於法不為無罪,於情實有可矜,請敕暫緩定擬。皇上御極以來,屢詔求言,言或無當,奉旨明斥;斥其無當,非禁使不言也,然言者即因以見少。即如諸路僨軍失地之將帥,未敗之始,其措置乖方,人言藉藉;而無敢為皇上言者,或慮無實據也,或雖有實據而慮查辦時化為子虛也,或慮不用而徒招怨也,或謂聖心自有權衡也,是以皆不敢言。至用人進退之際,臣子每不敢盡言,淺者懼干聖怒而見斥,深者懼激上意而難回。皇上披覽奏章,纖悉必邀批示,勤亦至矣。臣以為精神貴於不紛,原務其遠者大者,舍其近者小者。明主勞於求賢,而逸於任人。今天下人才不足,此誠可憂。雖然,非無才也。如羅澤南,人無不知為將材矣,初不過一貢生耳。湖南一省,既有江忠源兄弟,又有羅澤南諸人,則他省可知。惟賢知賢,惟才愛才,是在聖心之誠求耳。方今武昌未下,江西又復危急,兩省之民,向也與賊為仇,今乃竟有從逆者。此中轉移之故,宜深思也。列聖仁漸德被,人心斷不能忘。然此時不亟維系,使賊得徐出假仁假義以為市,恐民心將為所搖而難挽矣。」奏入,上嘉納之。

八年,病免。十一年,穆宗即位,以茂廕忠直,命俟病痊聽候簡用。同治元年,上疏陳時政,言:「天象示警,宜廑修省。議政王責任重大,宜專心機務,餘事綜其大綱。言官宜加優容。順天府事繁,府尹石贊清不宜兼部。各國通商事務衙門司員甫及一年,即得優保,恐各衙門人員皆以營求保送為得計,宜防其漸。」署左副都御史,命偕兵部尚書愛仁往山西按事。授工部侍郎。二年,調吏部。丁繼母憂,歸。四年,卒於家。

宋晉,字錫蕃,江蘇溧陽人。道光二十四年進士,選庶吉士,授編修。二十七年,大考二等,擢中允。二十九年,典河南鄉試,因命題錯誤議處,諭不得更與考試差。咸豐二年,大考二等,擢侍讀學士,遷光祿寺卿。三年,命會辦京城團防保甲,署禮部侍郎。四年正月,疏言:「去冬圜丘大祭,適值聖體違和,禮臣以登降繁縟,於親詣壇位及奠帛後諸儀節,更加酌定,奏請允行,旋以遣親王恭代而止。惟詳稽典禮,祀天鉅典,尤為慎重。偶遇服色不宜,興居未適,有遣代,無議減。現值祈年大祀,伏原皇上飭停新議,仍遵成憲。」五年,遷宗人府丞。

六年,疏言:「自江寧失陷,上自九江,下及鎮江、瓜洲,寇勢水陸相援。現聞向榮兵力不支,情形危急,今即分路赴援,仍恐緩不濟事。請飭江督、浙撫,雇用輪船載兵,由圌山關入江,焚攻金、焦賊船。再由儀徵溯浦口,與六合諸軍相為犄角,則江寧、鎮江對岸之賊,節節防我,必不敢離巢東竄。是不特解江南之急,即江北亦愈寧謐。又聞廣東新至紅單船二十餘艘,請飭德興阿、向榮將紅單船並歸一處,力扼蕪湖江面。如能克復蕪湖,則拊賊之背,寧國不攻自下。」薦道員繆梓、楊裕深、金安清通達治體,洞悉夷情,請以雇船籌費諸事責成辦理。疏上,諭兩江總督怡良與向榮、德興阿酌行。

宣宗實錄告成,敘勞,擢內閣學士,迭署戶、工二部侍郎。八年,授工部侍郎。文宗頻歲抱病,未能親行祀典,十年,晉疏言:「近年郊壇大祀,聖躬以步履失常,偶緩親行,而於遣恭代外,仍先期躬詣皇乾殿拈香,仰見寅畏深衷。惟每屆大祀,皇上於前一日辰巳間躬詣拈香,即在齋宮祇宿。今則先期即如臨事,請於前一日寅卯間先行詣殿拈香,然後還宮辦事。臣尤原慎攝聖躬,養元氣,節峻伐之味,復健行之常,於下屆郊祀大典照常親行。」上嘉納之。

十一年,疏言:「江寧失陷已將十載,總督曾國籓經營防剿,與官文、胡林翼會合攻復安慶,惟所部不足二萬人。若合四川、湖北、湖南、江西、安徽五省歲入,養兵勇十三萬人,以七萬分駐防剿,六萬大舉東征,餉足兵增,庶可一舉集事。」又言:「江西首當賊沖,巡撫毓科、布政使慶善皆失人望,請以太常寺卿左宗棠簡署巡撫,而於督糧道李桓、前廣饒道沈葆楨、浙江道員史致諤三人中簡擇擢授籓司。」又請以曾國籓總統四川、湖北、湖南、江西、安徽五省督辦東征軍務。上以所籌不為無見,下官文、國籓等議奏。又疏言:「慕陵規制,儉約樸實,萬世可法。定陵工程請仿行勿改。」格於部議,不行。

同治元年,調倉場侍郎。南漕初改海運,歲額三百萬石,自天津運京倉,偷漏飛灑,歲損米綦鉅。迨軍興,江、浙郡邑淪陷,南漕起運才二十餘萬石,而偷漏飛灑如故。十年以來,侍郎及監督官凡數易。晉受事,深悉其弊,因循未奏舉。六年,事發,左遷內閣學士,償米二萬石。十二年,遷戶部侍郎。十三年,卒。

袁希祖,字荀陔,湖北漢陽人,原籍浙江上虞。道光二十七年進士,選庶吉士,授編修。咸豐二年,大考二等,擢侍講。三遷侍講學士。八年,超擢內閣學士。迭署禮、工、刑諸部侍郎。九年,疏言:「咸豐初以道梗銅少,改鑄大錢,未幾,當百、當五皆不行,惟當十行之。始直制錢三五,近則以十當一。銀直增貴,百物騰踴,民間重困。旗餉月三兩,改折錢十五千,致無以自活。向日制錢重一錢二分,大錢重四錢八分,以之當十,贏五錢四分。今以十當一,是反以四錢八分銅作一錢二分用也。民間私鎔改鑄,百弊叢生。今天下皆用制錢,獨京師一隅用大錢,事不畫一。請悉復舊規,俾小民易於得食,盜源亦以稍弭。」

十年,疏言軍事,略謂:「數年以來,地方軍事所謂失守,無所為守也,但聽其失。即坐以罪,僅革職留營而已。所謂收復,不見其收,自然而復。俟賊自去,即虛報克捷,上狀列保,以樹植私人。似此用兵,安有成功之一日?臣愚以為今雖敗裂,機尚可轉。賊窺蘇、常久,一旦得之,子女玉帛,其意已饜,不特金陵老賊全股爭趨,即天長、六合之賊,亦涎其利。宜乘彼勢方散緩,請特選重臣駐清、淮要地,統籌全局。頃諭旨令曾國籓赴兩江署任,規復蘇、常,自寧國進兵,前後受敵,非萬全之計。莫如令胡林翼自江北進攻,牽制安慶;令楊載福以水師直下大江,互相策應;令李若珠力攻天長、六合,以出江浦,遙立聲援。密飭國籓潛引銳兵,倍道以取金陵,方為上策。今日勞師糜餉,勢無窮已,兼各路統帥散而無紀,其賢者往往深入援絕,血戰殞身;其不肖者坐擁厚兵,遇敵輒避;必得重臣領兵統馭,積弊既除,精神乃奮,此轉移之機也。」尋署戶部侍郎。

時各直省行團練,分遣大臣督辦,希祖疏言:「團者一時可集,練非經久不能。即雲團練,非五六千人不可。計口授食,費已不貲。即使練成,而此五六千人制敵不足,騷動有餘,坐食貲詘,終虞譁潰。且遴往大臣,萬一與有司齟,必至互為水火,轉貽大局之憂。請頒明諭,使知團練乃以自衛鄉閭,並不以此科斂,亦不必日給口糧,坐守困耗。否則用多費溢,正供無可挹注,不得不取諸民。輕則聚眾,重則返戈,大可慮也。」

英、法、俄、美四國合軍內犯,天津不守,希祖請暫就和議,遷延旬日,俾部署得以周詳。僧格林沁獲英官巴夏裏,希祖疏請殺之。未幾,敵軍深入,上巡幸熱河。希祖屢疏諫,不報,屢北望痛哭,遂得疾。已而和議成,兼署兵部侍郎。尋卒。

文瑞,字叔安,烏蘇氏,滿洲鑲紅旗人。道光二十一年進士,選庶吉士,授編修。擢侍講,五遷至左副都御史。文宗即位求言,文瑞疏陳四事,請選賢才,明賞罰,廣聽納,謹調攝,並錄乾隆元年左都御史孫家淦三習一弊疏以進,上嘉之。咸豐三年,粵匪陷武昌東下,疏請於上海、鎮江雇用廣東紅單船,擇員統帶,以防江面;並密察京師流言,以消逆萌、靖畿輔。上命諸大臣集議增兵籌餉,文瑞疏言:「兵餉為國家大政,遵旨會議,乃大學士等絕無一語及公,言笑晏晏,不知內閣何地,不詢會議何事。臣臚舉搘持之策,尚書孫瑞珍竟閒辭支吾,自述家私,形同市井。大臣如此,深堪悼嘆。」又言:「二月朔為領俸定期,戶部款絀,早應籌畫。乃於是日清晨請旨,冀以停俸上諉朝廷。又議行鈔法,並徵鋪稅,商民驚懼。請發帑三十萬支放春俸,暫可流通,俾商民安業,鈔法鋪稅,暫從緩議。」從之。又疏言:「鈔法之弊,放多收少,半為廢紙。放少收多,民間鈔無從得。若收放必均,是與之甲而取之乙,徒擾無益,非易銀鈔為錢票不可。擬就道光年間所設官號錢鋪五處,分儲戶、工兩局卯錢。京師俸餉,照公費發票之案,按數支給,以錢代銀。」並具條目六事。疏入,議行。

尋兼署大理寺卿,以天變奏請修省,上嘉納之。刑部罪人劉秋貴死於獄,文瑞奏:「秋貴無病,一夕而死。刑部後四日入奏,改易日期,塗飾操縱,請嚴飭根究。」山西崞縣民婦王劉氏拒奸死,罪人從輕比,刑部題駮,文瑞復奏:「原擬知州失出,請飭山西巡撫嚴劾。」上並從之。

粵匪入山西境,陷平陽等處,文瑞奏請飭督兵大臣嚴防入直隸要路。尋自臨洺關竄逼天津,命文瑞率兵駐通州。奏言:「通州城垣樓櫓損壞,請集款建復。」諭:「此守土之責,統兵大臣不必兼轄。」擢刑部右侍郎。四年,以病乞罷。

先是文瑞偕克勤郡王慶惠請捐銅鑄四項大錢濟兵餉,上從其請。及還京,病痊,命仍與慶惠董其事,設局開爐。上命尚書阿靈阿、御史範承典往銅廠查驗,文瑞奏劾阿靈阿等擅開爐房,恐有偷漏,上斥其負氣任性,降二級調用。同治元年,卒。

毓祿,字曉山,舒穆魯氏,滿洲正白旗人。道光二十一年進士,授刑部主事。累升郎中,遷御史。軍興,安徽、江蘇、山東諸省皆暫停秋審。毓祿奏言:「寇蹤所至,每先釋獄囚,脫其死而置之生,自必原為賊用。雖有投首減罪之例,而愚頑類多不知大義。聞直隸近因賊擾,將秋審諸囚,酌覈情罪,其謀、故、兇、盜、拒捕、殺人重囚,立即正法。其情有可矜及例應緩決諸囚,即予減等發配,誠為權宜變通之道。現有軍務省分,應令一體遵辦。」

京師行用大錢,當百、當五十二種壅滯不行,毓祿疏請商民應納旗租、地丁、關稅,於例定收鈔五成數內專收當百、當五十大錢二成,部收捐項應交錢票,亦一律納大錢。七年,擢工科給事中,歷內閣侍讀學士、太僕寺少卿、通政司副使、內閣學士。同治三年,擢工部侍郎,兼管錢法堂。五年,奏言:「寶源局鑄當十錢,向系滇省解銅,以銅七鉛三配鑄。近因滇銅久未解局,市銅低雜,致錢文輕小,例定每錢應重三錢二分。請每屆收錢,以三錢為率,不及者即飭改鑄。」上斥寶泉、寶源二局不職之兩侍郎監督,並下吏議。

徐繼畬,字松龕,山西五臺人。道光六年進士,選庶吉士,授編修,遷御史。迭疏劾忻州知州史夢蛟、保德知州林樹雲營求升遷,登州知府英文諱災催徵,榮河知縣武履中藉事科斂。又疏請除大臣回護調停積習。

又疏陳政體宜崇簡要,略謂:「皇上廣開言路,諸臣條奏茍有可取,無不通行訓諭,惟是積習疲玩已久,煌煌聖諭,漠不經意,輕褻甚矣。臣以為諸臣條奏,或非大體所關,或非時務所急,原不必悉見明文。若事關切要,聖慮折中,期於必行者,即降諭旨,宜重考成。度其事之難易,限年興革。如仍前玩視,於本案外重治以違旨之罪。此教令之宜簡也。六部則例日增,律不足,求之例;例不足,求之案:陳陳相因,棼亂如絲。論者謂六部之權,全歸書吏。非書吏之有權,條例之煩多使然也。臣以為當就現行事例,精審詳定,取切於事理者,事省十之五,文省十之七,名曰簡明事例,使當事各官得以知其梗概,庶不至聽命於書吏。此則例之宜簡也。考功、職方,議功議過,使百僚知勸懲也。現行之條,苦於太繁太密,不得大體。嘗見各直省州縣有蒞任不及一年,而罰俸至數年十數年者,左牽右掣,動輒得咎。且議處愈增愈密,規避亦愈出愈奇,彼此相遁,上下相詭,非所以清治道也。臣以為各官處分,凡關於國計民生,官箴品行,不妨從重從嚴;其事涉細微,無關治體,與夫苛責太深,情勢所難者,當準情酌理,大加刪削。此處分之宜簡也。」疏入,上嘉納。旋召入對,論時事至為流涕。

十六年,出為廣西潯州知府,擢福建延邵道,調署汀漳龍道。海疆事起,敵艦聚廈門,與漳州隔一水,居民日數驚。繼畬處以鎮定,民賴以安。二十二年,遷兩廣鹽運使,旬日擢廣東按察使。二十三年,遷福建布政使。二十六年,授廣西巡撫,未赴官,調福建。閩浙總督劉韻珂以病乞假,繼畬暫兼署總督。福州初通商,英吉利人僦居會城烏石山神光寺,士民大譁,言路以入告,上命韻珂、繼畬令其遷徙,久之乃移居道山觀。士民以繼畬初不力拒,終不慊,言者屢論劾。繼畬初入覲,宣宗詢各國風土形勢,奏對甚悉,退遂編次為書曰瀛寰志略,未進呈而宣宗崩,言者抨擊及之。

咸豐元年,文宗召繼畬還京,召對,稱其樸實,尋授太僕寺少卿。詔求言,繼畬上疏,略謂:「國家崇尚儉樸,大內宮殿,一仍明舊。惟圓明園為三時聽政之地,避暑山莊為秋獮駐蹕之地,兩處規模,至乾隆間而備。宣宗皇帝暫停秋獮,熱河工程一切報罷,惟自正月至十月恆駐圓明園。然三十年中,未嘗增一堵一椽,游觀不及諸坐落,或報應修,輒令撤去,以故內帑發出外庫前後凡千數百萬。數年以來,園亭久曠,或謂先朝堂構,不應坐聽彫殘。方今軍務未完,河工未畢,亦料無暇及此。將來兩事告蕆,內庫稍充,保無以營繕之說嘗試者,伏望皇上堅持,茍非萬不得已之工程,一切停罷。至於裝修陳設,珍奇玩好,可省則省,無取鋪張,此土木之漸宜防也。孔子刪詩,以關雎為首,義取摯而有別。匡衡之說有曰:『情欲之感,無介於容儀;宴安之私,不形於動靜。』其言有別,可謂深切著明。第以事涉宮闈,絕於聽睹,非臣子之所敢言。雖有折檻之忠,牽裾之直,止能言得失於殿廷,豈能爭是非於宮壼?故聖帝明王,即以是為修省最切之地。皇上至剛無欲,邇者釋服禮成,將備周官九御之制,衍大雅百男之祥。竊以為聖德日新,肇基於此,此宴安之漸宜防也。自古壅蔽之患,由於言路不通,然亦有言路既通,而壅蔽轉生於不覺者。皇上御極之初,即以開言路為務。自倭仁一疏,手詔褒嘉,言事者紛紛而起。邇因天旱求言,又復諄諄獎誘,舉空言塞責、受人指揮、激直沽名三弊為戒。臣庶大半中材,臣以為空言塞責,事出庸愚,一覽擲之,無關輕重。激直沽名,由於器小,皇上予以優容,適足以見聖度。至受人指揮,事涉營私,果其確有可憑,必當明正其罪。總之群言淆亂,衷諸聖人,亦在皇上權衡酌量而已。臣竊計在京言事者,約分三等:以章奏陳者,曰九卿、科道;以章奏陳兼得面陳者,曰部院大臣;不以章奏陳而時得面陳者,曰內廷王公。此三者各有所優,亦各有所蔽。九卿、科道,爵秩未崇,少回翔之意,聞見較廣,多採訪之途,以風節相磨,以彈劾為職,此其所優也;其所蔽則前之三弊是也。部院大臣,久在朝列,受恩效忠,明習時事,此其所優也;然階級既崇,天顏日接,顧忌矜慎,胸臆所存,莫能傾吐其十一,此則其所蔽也。內廷王公,國家肺腑,外無私交黨援之患,內無希幸爵賞之心,此其所優也;然法制綦嚴,例不與外人交接,廷評輿論,所不盡聞,此則其所蔽也。皇上明目達聰,幽隱畢照,而臣乃鰓鰓過慮者,誠恐言事者限於才識,未能仰副淵衷,致皇上察納虛懷,不免悵然而思返,此壅蔽之漸宜防也。昔唐臣魏徵有十漸之疏,太宗嘉納,千古以為美談。夫漸者,已然之詞也。正之於已然,何如防之於未然。臣謹師其意,衍為三防之說,極知迂陋,無補高深,伏冀幾餘採納。」上優詔報之。

咸豐二年,吏部追論繼畬在巡撫任逮送罪人遲誤,請議處,乃罷歸。尋丁母憂。粵匪北犯,攻懷慶,山西巡撫哈芬檄太原總兵烏勒欣泰率兵防澤州,遷延未即赴。賊渡河陷垣曲,哈芬出駐陽城,布政使郭夢齡疏乞援,繼畬亦具疏借布政使印馳奏,上為罷哈芬巡撫,以王慶云代之。繼畬條舉防守諸事以告,尋奏請令繼畬督辦防堵。事定,居數年,回、捻交亂,又命督率官紳總辦各府州團防。繼畬駐潞安年餘,親歷遼州、上黨、陽城諸要隘,措置詳備,署巡撫沈桂芬甚重之。同治二年,召詣京師,命在總理各國事務衙門行走。尋授太僕寺卿,加二品頂戴。五年,以老疾乞歸。

繼畬父潤第,治陸王之學。繼畬承其教,務博覽,通時事。在閩、粵久,熟外情,務持重,以恩信約束。在官廉謹。罷歸,主平遙書院以自給。尋卒。

王發桂,字笑山,直隸清苑人。道光十六年進士,授禮部主事,充軍機章京,累遷郎中。咸豐三年,上疏言軍事,被嘉納。尋遷御史。

洪秀全既踞江寧,分兵北犯,發桂疏言:「順德、正定地當沖要,請屯兵扼隘。」並條列六事,曰:謹偵報,嚴催儹,慎查勘,明曉諭,廣撫恤,籌協濟。又疏薦貴州道員胡林翼知兵能勝重任,請超擢,俾任軍旅,上命林翼留湖北襄軍事。迭疏請令各省汰舊伍,練新兵,設鄉團,值有事則新軍進戰,鄉團設防,以明戚繼光紀效新書、練兵實紀訓練將士。賊渡河偪近畿輔,疏請蒐簡軍實,選精銳為後備,並蠲貧民房稅,撫流亡以安人心,下所司議行。疏言:「軍興以來,大臣獲罪,多以從軍自效,位崇性驕,不可任使,坐耗糧糈,無裨軍政。且主將曲庇,輒請起用,有罪幾同無罪,圖功適以冒功。頃副都統達洪阿退縮失律,致知縣謝子澄、副都統佟鑒同時死寇。欽差大臣勝保賜以神雀刀,原令便宜行事,乃自入直境,未戮一人;而於獲戾大臣,多所論薦,以私廢公,抑阻士氣。請按治達洪阿以下,行軍法。紀律既嚴,軍威自振。」並被採納。累遷給事中、鴻臚寺卿。

八年,復疏論時事,言:「宜上廉恥,重訓練,以求將帥之才。李續賓、唐訓方起自末僚,能自張一軍,轉戰千里。敦樸廉潔,勇往任事之人,隨地而有,請飭督撫採訪奏聞。物力艱窘,莫甚於湖南;軍餉糜費,莫甚於江蘇。自湖南得左宗棠,江蘇得王有齡,而餉源日裕。夫興利莫如去★L2,今司計者日言捐餉,而鹽、漕、糧稅,凡國家自然之利,一任廢弛。請下所司議整飭。兩廣總督黃宗漢赴粵,遷延六月,遲不之官。城淪於敵,巡撫柏貴莫知為計。城東居民殺敵數百,柏貴輒為懸賞緝殺人者。貴州巡撫蔣霨遠當叛苗、教匪日久鴟張,未聞有所措施。此皆才力不逮,遂使一方塗炭。聖主恩威並用,尤所仰望。」

歷太僕寺卿、通政使、左副都御史。同治二年,署工部侍郎。疏薦戶部郎中王正誼守潔才優,以忤肅順得罪,請復其官,報可。授禮部侍郎,調刑部,又調工部。五年,以疾乞免。九年,卒。

廉兆綸,初名師敏,字葆醇,順天寧河人。道光二十年進士,選庶吉士,授編修。宣宗知其賢,將擢用,以父憂歸,遺命諸臣可大用者,兆綸與焉。咸豐元年,服除。二年,大考二等。三年,直南書房。四年,授右贊善,超擢翰林院侍講學士,督江西學政,轉侍讀學士,再擢內閣學士。五年,授工部侍郎。

時粵匪石達開擾江西,侍郎曾國籓率師御之,寇張甚,陷州縣五十餘,逼會城。上命兆綸幫辦廣信、饒州防剿,兆綸奏言:「江西通省募勇計一萬五六千人,各不相統屬。地方有警,勝則互訐以競功,敗則爭潰而不相救。甚且擾民冒餉,乘便營私,其弊不勝枚舉。今賊勢日張,瑞州、臨江相繼失守,設有倉卒,以此散而無紀者當之,何恃不恐?惟有將所募之勇,裁去一切名號,並為三四軍,每軍得四五千人,統以監司方面素有威望者,庶可責成功。」

六年三月,兆綸按試廣信,賊陷吉安、撫州,進據安仁,兆綸上疏請援,並以練勇千守貴溪。賊竄德興,陷建昌,廣信勢益孤,兆綸督諸生集鄉團,與廣信知府沈葆楨、上饒知縣楊升籌防禦。遣上饒諸生郭守謙率鄉勇三百夜襲金谿,諸生曾守誠奮勇先入城,賊不虞兵至,奪西南門逸,克其城。乘勝會攻建昌,而饒州又陷,官軍敗績,廣信益危。兆綸與國籓等合疏請截留閩兵一千六百專攻建昌,分檄守謙與在籍道員石景芬防剿。六月,國籓遣都司畢金科復饒州,兆綸飭景芬、守謙等馳攻撫州。會賊連陷廣昌、南豐、新城、瀘溪四縣,八月,守謙軍撫州張家橋,三接皆捷,窮追遇伏,力戰死。時兆綸方赴鉛山,道梗,咨衢州鎮總兵饒廷選乞援。廷選率兵二千一百至,兆綸冒雨穿敵壘,復入廣信,共謀守御,寇屢攻不下。凡七戰,捕斬其渠六,斬六千餘級。廷選與游擊穆隆阿、都司賴高翔等又屢擊破之。賊走玉山,廣信始解嚴。兆綸防守危城,盡出俸銀餉軍,貧困至不能自給,尋以病告歸。

七年,病痊,仍直南書房,署工部侍郎。八年,授戶部侍郎,調倉場侍郎。時軍事方急,兆綸疏請責成督撫辦賊,略曰:「今於督撫外另設統兵大員,其本省督撫雖有會剿之名,其實專為籌餉之事。統兵者往往以呼應不靈,餉糈不給,漸至遷延;而督撫又往往以事權不一,供億不貲,各生意見。及至城池失守,統兵者無地方之責,或邀寬大之恩,而並未帶兵之督撫,轉受其咎。名實不符,事多掣肘,賊氛之熾,職此之由。臣惟督撫大吏,類皆朝廷簡拔之人,設其人未盡知兵,不妨擇統兵大員,畀以督撫之任,使之各清各省,而責其成功。方今川、黔、閩、廣,並未另派統兵大員,而本境漸就肅清。湖南北之專任督撫討賊者,轉有餘力助剿鄰境。至於江蘇一省,統兵者不一而足,而潰敗糜爛至今。平心而論,統兵大員中,豈乏公忠體國之臣?所以然者,抑其所處之地不同,用情亦異,此其故不可不深長思也。清、淮一帶,實為南北要沖,漕運總督不兼管地方,宜此時權設江北巡撫,抑或將漕運總督權改斯缺,所有江北各路軍務,悉歸統制,庶可控扼江、淮,聲援汝、潁。不惟江南群逆絕其覬覦之心,即豫東會、捻各匪出沒之區,亦可斷其一臂矣。」疏上,不報。

九年,英吉利兵北犯,疏請以戰為和。十年,英兵掠豐益倉,兆綸疏自劾,上寬之。又疏言:「軍興以來,各省兵不足,因招募鄉勇。比來兵日少,勇日增,不可不預為之計。此後勇丁如有技藝精嫺,戰陣得力者,請令統兵督撫大臣,即於存營缺額挑選充補。軍事既定,原歸農者遣散,原效力者分隸各標,序補額兵。」上韙之。兆綸以交河糧商囤積穀秕,遣勇目捕治,糧商訴勇目索詐,辭連兆綸,事上聞,命刑部逮問。同治元年,京察休致。二年,諭責兆綸在任用人不當,奪職銜。

兆綸感知遇,遇事敢言,以是多齟。罷官歸,讓產諸弟,主問津書院,以修脯自給。六年,卒。

雷以諴,字鶴皋,湖北咸寧人。道光三年進士,授刑部主事,洊升郎中。遷御史、給事中,擢內閣侍讀學士,三遷奉天府府丞。咸豐元年,應詔陳言,請任賢能,覈名實。二年,復授太常寺少卿,屢上疏陳軍事。三年,遷左副都御史,命會同河道總督楊以增巡視黃河口岸,迭疏請撫恤茌平、東平、東阿、汶上饑民,撤山東防河兵,省各渡口冗費,皆報可。

粵匪陷揚州,以諴自請討賊,募勇屯萬福橋,扼揚州東南。賊窺里下河,以諴屢擊走之,通、泰十餘城賴以保全。授刑部侍郎,幫辦軍務。與琦善、陳金綬會攻揚州,以諴分兵駐守要隘,焚浦口賊舟。屢會諸軍擊賊,而揚州久攻不能下,諸將以總兵瞿騰龍最勇敢足恃,詔命援安徽。以諴疏言:「臨陣易將,兵家所忌。」琦善亦以為言,乃留勿遣。其冬,賊陷儀徵,偪運河西岸,官軍屢擊走之。以諴與浙閩總督慧成合駐軍灣頭六徬,未幾,賊援至,鄉勇潰散,琦善奏劾,奪官留軍自效。嗣琦善請移灣頭大營,以諴與慧成力爭,琦善復劾以諴諱飾。上責琦善諉過,飭以諴仍守灣頭及萬福橋諸隘。賊既自揚州退瓜洲,時來攻,以諴與陳金綬合擊敗之,加三品頂戴。尋授江蘇布政使,屢督砲船渡江會剿,攻北固山,破其土城,乘勝逐至金山,敗之。

六年,托明阿兵潰瓜洲,揚州復陷,詔責以諴等擁兵不援。又疏辨冒功,為德興阿所劾,褫職戍新疆。以諴在戍所,呈請將軍扎拉芬代奏,言江北軍事。尋赦還,賜四品頂戴,授陜西按察使。遷布政使,入為光祿寺卿。同治元年,京察,休致。光緒五年,以重宴鹿鳴還原銜。八年,又以重宴恩榮,加頭品頂戴。十年,卒,年七十九。

以諴在江北,用幕客錢江策,創收釐捐。錢江者,浙江長興諸生,嘗以策干揚威將軍奕經,不能用。林則徐戍伊犁,從之出關,以是知名。謁以諴於邵伯,留佐幕,餉絀,江獻策,遣官吏分駐水陸要沖,設局卡,行商經過,視貨值高下定稅率,千取其一,名曰「釐捐」,亦並徵坐賈,歲得錢數千萬緡。江與同幕五人赴下河督勸,不從者脅以兵,民間目為「五虎」。江自以為功,累保獎至道員,氣矜益盛,以諴不能堪。會飲,江使酒罵坐,以諴執而殺之,以跋扈狂肆、謀不軌聞。後各省皆仿其例以濟軍需,為歲入大宗焉。

陶樑,字鳧薌,江蘇長洲人。嘉慶十三年進士,選庶吉士,授編修,纂修皇清文穎。十九年,林清之變,逆黨闌入禁城,樑方在館修書,其僕駱升聞警,匿樑於書櫥,自當戶立,賊刃之,僕,越日事定,樑出,救之甦。仁宗回鑾聞之,召樑問狀,曰:「義僕也!」賜之金。

二十一年,以知府發直隸,補永平,調正定。道光四年,擢清河道,署按察使。新城縣失過境餉鞘,歸罪外委白勤,逮訊,死於刑。上遣尚書松筠、侍郎白鎔按治,察其枉,樑坐降四級,捐復知府,留直隸。十二年,補大名知府。十八年,遷湖北荊宜施道,萬城堤決,樑復坐降調,捐復。二十二年,補湖南糧儲道,調湖北漢黃德道。二十八年,遷甘肅按察使,調山西。二十九年,遷江西布政使。入覲,授太常寺卿。

文宗即位,樑疏言:「宣宗成皇帝天錫智勇,嘉慶十九年八月之變,當時但傳發槍斃賊,不知首逆林清姓名地址,亦由宮中訊得,立時遣捕,故渠魁不致遠颺,餘孽不致滋蔓。請敕載入實錄,以揚聖武。」上從之。咸豐二年,擢內閣學士。四年,遷禮部侍郎。六年,以病乞罷。七年,卒,年八十六。

樑早有文名,曾從侍郎王昶助其纂述。歷官所至,提倡風雅,賓接才俊,輯畿輔詩傳行世。晚登朝右,時值軍興,耆舊凋落,其猶見乾、嘉文物之盛者,惟大學士祁藻與樑二人,為士林所歸仰雲。

吳存義,字和甫,江蘇泰興人。道光十八年進士,選庶吉士,授編修。二十二年,督云南學政。邊徼士風敦樸,存義力為提倡,文風改觀。回民煽亂,存義按試永昌竟,出郭數里,城中火起,待學使去而始發也。二十八年,丁母憂歸。會江北大水洊饑,存義議賑,躬詣富室勸捐,多感其誠,出貲購米穀。存義棹小舟散給饑民,全活甚眾。服闋,直南書房,擢侍講。咸豐五年,典試雲南,復留督學政,士益親之。回亂益棘,圍會城,城中兵閧,掠官署民居,獨未入學政廨,民間婦孺匿考院避難者千人。存義在雲南久,習知民情,比復命奏對,陳變亂始末甚詳。累遷侍讀學士,署順天府丞。

十年,英法聯軍入京師,上幸熱河,京朝官多挈家出走,存義屬疾,語家人毋隨人妄動。事定,敘城守勞,將入存義名,存義聞之,力疾起,署牘曰:「府丞吳存義抱病家居,干掫詰奸皆無與。今病未愈,不敢冒受賞。」

未幾,擢太僕寺卿,遷通政使,署禮部侍郎。存義以文廟從祀位次多舛,奏請審定,繪圖頒行。又以諸儒增祀既繁,漸失世用其書、垂諸國胄之義,奏飭中外臣工不得濫請。署刑部侍郎。

同治二年,署工部侍郎,迭署禮、戶二部。出督浙江學政,軍事甫定,人士離散初歸,存義寬大拊循,歲考既周,秀良者始奮於學,乃導以經、史、小學,文風復興。三年,調吏部,留學政任。六年,任滿,以病乞歸。七年,卒。

殷兆鏞,字譜經,江蘇吳江人。道光二十年進士,選庶吉士,授編修。咸豐四年,遷侍講,直上書房,授惠親王子奕詳等讀。擢侍講學士,命授孚郡王奕硉讀,累遷大理寺少卿。八年,英吉利兵犯天津,兆鏞力主戰,疏請黜邪謀,決不計,詆斥主和諸臣甚力,擢詹事。九年,署兵部侍郎。詔江蘇諸省治團練,兆鏞疏言其弊,舉四害,言甚切。上海欲借英、法人助戰,兆鏞亦以為不可。

十一年,丁本生母憂,同治元年,服除,仍直上書房。疏言:「江、皖軍威既震,大局漸有轉機。臣來自災區,敢就見聞真切關系重大者為皇上陳之:一,宜飭戎行。上海兵勇號稱四萬,皆不堪用,何以今年經英、法人管帶,便成勁旅?華爾親兵六百,盡中國人,戰無不勝。無他,挑選慎,約束嚴,器械精,賞罰信耳。請敕將帥講求武備,漸事安攘。提鎮中如曾秉忠水師通賊焚掠;馬德昭掠蘇州、上海;李定泰掠湖州、嘉興;向奎每戰輒敗,敗輒行劫;馮日坤部兵掠婦女。李恆嵩兵不行劫,已共推良將。竊謂行師首禁焚掠,克城先謀戍守,否則旋得旋失,民間無孑遺矣。一,宜澄吏治。上海諸官吏,惟劉郇膏得民心,已蒙特簡。薛煥統馭無能;吳煦精心計,在上海設銀號,繳捐者非所出銀票不收;新授糧儲道楊坊,由洋行擔水夫致巨富,為洋人所鄙;浙江布政使林福祥,杭州破後降賊,送王有齡、張錫庚柩至上海。臣意此等悖員,宜分別懲創,稍申憲典。一,宜清釐餉款。上海左近官卡、賊卡、槍船卡林立,卡稅之外,釐捐、月捐、船捐、畝捐、房捐日增月益,臣聞官吏紳商皆云日可收銀二萬,月得六十萬。兵勇四萬人,日餉三錢,月止三十六萬,而當局猶入不敷出。請敕曾國籓、李鴻章嚴密清釐。蘇、松、嘉、湖賦額甲天下,近三十年,年年蠲緩,官民交欠,賦成虛額。現經大亂,田荒戶絕,可否俟軍務大定,敕督撫覈計,酌留商稅,核減農賦,以羨補不足,勿逾定則。一,宜撫恤遺民。江、浙交界莠民設槍船,所至焚掠,此輩視官兵盛衰以為向背,克復時必為內應。請敕督撫從宜處置,或令歸農,或籍為兵,勿貽後患。至失守郡縣,陷賊士民商賈,茍非出自甘心,僅止偷生畏死,可否援脅從罔治之義,乞恩原宥。一,宜防維外人。上海孤城克保,不得謂非外人之力。自經助剿,所向無前,或云實出義舉,或云欲通商販,或云日後恃功索償,臣俱不敢逆億。各處通商,尊奉外人太過。猶幸我國新政清明,未萌覬覦。日久相習,利權盡歸,人情益附,而謂狼子必無野心,實難深信。撫御得體,尤在博知外情。請敕各口通商衙門,譯述各國新聞有關時事者,書記大則奏聞,藉資豫備。」上以所陳不為無見,下國籓、鴻章等籌畫,並將福祥等察劾按治。尋授詹事,遷內閣學士,迭署兵、禮諸部侍郎。

四年,編修蔡壽祺疏劾恭親王,命大學士倭仁等察奏。兆鏞與左都御史潘祖廕疏言:「恭親王輔政以來,功過久蒙睿照,重臣進退,關系安危。尚祈持平用中,熟思審處,察其悔過,予以轉圜。庶無紊黜陟大綱,滋天下後世之惑。」上納其言。六年,督安徽學政。七年,授禮部侍郎,任滿,仍直上書房,迭署兵、工二部侍郎。尋授吏部侍郎,調戶部,再調禮部。光緒七年,以病乞罷。九年,卒。

論曰:咸豐中四方多故,文宗悒悒,恆抱疾。京師用不足,大錢鈔票,法立弊滋。王茂廕屢進讜言,均中利害,清直為一時之最,宋晉亦其次也。袁希祖、文瑞皆有所論列,而徐繼畬直箴君德,所舉三防,陳義尤高,發桂言軍事亦有識。廉兆綸助守江西,雷以諴分防江北,並著事功。陶樑為文學老宿,吳存義、殷兆鏞並侍從清望,存義視學滇、浙,能得士心,兆鏞慷慨論事,於鄉邦疾苦冀有補苴,何言之深也!


\end{pinyinscope}