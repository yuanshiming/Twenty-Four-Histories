\article{志一百二十八}

\begin{pinyinscope}
○邦交一

中國古重邦交。有清盛時,諸國朝聘,皆與以禮。自海道大通而後,局勢乃一變。其始葡萄牙、和蘭諸國,假一席之地,遷居貿易,來往粵東;英、法、美、德諸大國連袂偕來,鱗萃羽集,其意亦僅求通市而已。洎乎道光己亥,禁煙釁起,倉猝受盟,於是畀英以香港,開五口通商。嗣後法蘭西、美利堅、瑞典、那威相繼立約,而德意志、和蘭、日斯巴尼亞、義大里、奧斯馬加、葡萄牙、比利時均援英、法之例,訂約通商,海疆自此多事矣。俄羅斯訂約在康熙二十八年,較諸國最先,日本訂約在同治九年,較諸國最後,中國逼處強鄰,受禍尤烈。其他若秘魯、巴西、剛果、墨西哥諸小邦,不過尾隨大國之後,無他志也。咸豐庚申之役,聯軍入都,乘輿出狩,其時英、法互起要求,當事諸臣不敢易其一字,講成增約,其患日深。至光緒甲午馬關之約,喪師割地,忍辱行成,而列強據利益均霑之例,乘機攘索,險要盡失。其尤甚者,則定有某地不得讓與他國之條,直以中國土疆視為己有,辱莫大焉。庚子一役,兩宮播遷,八國連師,勢益不支,其不亡者幸耳。

夫中國幅員之廣,遠軼前古,幽陵、交阯之眾,流沙、蟠木之屬,莫不款關奉贄,同我版圖。乃康、乾以來所力征而經營者,任人蠶食,置之不顧,西則浩罕、巴達克山諸部失之於俄,南則越南、緬甸失之英、法,東則琉球、朝鮮失之日本,而朔邊分界,喪地幾近萬里,守夷守境之謂何,此則尤令人痛心而疾首者也。爰志各國邦交始末,以備後人之考鏡焉。

俄羅斯

俄羅斯,地跨亞細亞、歐羅巴兩洲北境。清初,俄東部有羅剎者,由東洋海岸收毳礦之貢,抵黑龍江北岸,據雅克薩、尼布楚二地,樹木城居之,侵擾諸部。嗣又越興安嶺南向,侵掠布拉特烏梁海四佐領。崇德四年,大兵再定黑龍江,毀其城,兵退而羅剎復城之。

順治中,屢遣兵驅逐,以餉不繼而返。十二年及十七年,俄察罕汗兩附貿易人至京奏書,然不言邊界事。康熙十五年,帝召見其商人尼果賚,貽書察罕汗,令管束羅剎,毋擾邊陲。既而羅剎復肆擾,帝命黑龍江將軍薩布素圍雅克薩城。會荷蘭貢使至,乃賜書付荷蘭轉達其汗。二十五年九月,其新察罕汗復書至,言:「中國前屢賜書,本國無能通解者。今已知邊人構釁之罪,自當嚴治,即遣使臣詣邊定界,請先釋雅克薩之圍。」許之,遂詔薩布素退師。

二十八年冬十二月,與俄定黑龍江界,立約七條。先是俄使臣費嶽多羅額里克謝等由陸路至喀爾喀土謝圖汗境,文移往復。至是始與領侍衛內大臣索額圖等會議於黑龍江:一,循烏倫穆河相近格爾必齊河上游之石大興安嶺以至於海,凡山南流入黑龍江之溪河盡屬中國,山北溪河盡屬俄。一,循流入黑龍江之額爾古訥河為界,南岸盡屬中國,北岸盡屬俄。乃歸中國雅克薩、尼布楚二城。定市於喀爾喀東部之庫倫。立石於黑龍江兩岸,刊泐會議條款,用滿、漢、拉提諾、蒙古、俄羅斯五體文字。是為尼布楚條約。自後貿易之使每歲間歲一至,未嘗稍違節制。

三十三年,遣使入貢。時有二犯逃入俄,俄遣人送回,理籓院行文獎之,遂復遣使入貢。帝閱其章奏,諭大學士曰:「外籓朝貢,雖屬盛事,恐傳至後世,未必不因此反生事端。總之,中國安寧則外釁不作,故當以培養元氣為根本要務。」三十九年,遣使齎表至。

雍正五年秋九月,與俄訂恰克圖互市界約十一條。俄察罕汗卒後,其妃代臨朝,為叩肯汗。遣使臣薩瓦暨俄官伊立禮,與理籓院尚書圖禮善、喀爾喀親王策凌在恰克圖議定。喀爾喀北界,自楚庫河以西,沿布爾固特山至博移沙嶺為兩國邊境,而互市於恰克圖。議定,陳兵鳴砲,謝天立誓。是月,定俄人來京就學額數。俄國界近大西洋者崇天主教,其南境近哈薩克者崇回教,其東境近蒙古者崇佛教。康熙間,嘗遣人至中國學喇嘛經典,並遣子弟入國子監,習滿、漢語言文字,居舊會同館,以滿、漢助教各一人教習之。至是,定俄人來學喇嘛者,額數六人,學生額數四人,十年更代為例。

乾隆二十三年春正月,俄人獻叛人阿睦爾撒納尸。初,厄魯特輝特部阿睦爾撒納背準噶爾來附,帝封為親王,命副定北將軍班第征準噶爾,降其部眾。已復叛歸,逃入俄,索之,以渡河溺死聞。既而患痘死,遂移尸至恰克圖來獻。未幾,厄魯特臺吉舍楞戕中國都統唐喀祿,叛逃入俄,索之又不與,絕其恰克圖貿易。三十年秋八月,俄綽爾濟喇嘛丹巴達爾扎等請附,又恐俄人追索,中國擒送,遣人來探。瑚圖靈阿以聞,帝命納之。三十三年秋八月,復俄恰克圖互市,理籓院設庫倫辦事大臣掌之。四十四年,再停互市,次年復之。五十四年,又以納叛人閉市,嚴禁大黃、茶葉出口,俄人復以為請。五十七年,乃與訂恰克圖市約五條。

嘉慶七年秋七月,喀爾喀親王蘊端多爾濟請巡查恰克圖兩國邊界,帝命逾十年與庫倫辦事大臣輪次往查。十年冬十二月,俄商船來粵請互市,不許。

道光二十五年,俄進呈書籍三百餘種。二十八年,俄商船來上海求互市,不許。初嘉、道間,俄由黑海沿里海南侵游牧各回部。英吉利既據東南兩印度,漸拓及溫都斯坦而北。於是怱嶺西自布哈爾、浩罕諸部皆並於俄,夾恆河城郭回國半屬於英,英、俄邊界僅隔印度歌士一大山,連年爭戰。俄思結援中國,遣使約中國以兵二萬由緬甸、西藏夾攻印度。事未行。英旋助土耳其與俄戰,始講和而罷。逮江寧撫議定,法、美未與議者,亦照英例,並在五口通商。而俄人自嘉慶十一年商船來粵駮回後,至是有一船亦來上海求市,經疆臣奏駮,後遂有四國聯盟合從稱兵之事。

咸豐元年,俄人請增伊犁、塔爾巴哈臺、喀什噶爾互市,經理籓院議允伊、塔而拒喀什噶爾。文宗即位,命伊犁將軍奕山等與之定約,成通商章程十七條。三年,俄人請在上海通商,不許。又請立格爾畢齊河界牌,許之。至五年,俄帝尼哥拉斯一世始命木喇福岳福等來畫界。

先是木喇福岳福至莫斯科議新任地諸事,以為欲開西伯利亞富源,必利用黑龍江航路;欲得黑龍江航路,則江口及附近海岸必使為俄領,而以海軍協力助之。俄帝遂遣海軍中將尼伯爾斯克為貝加爾號艦長,使視察堪察加、鄂霍次克海,兼黑龍江探險之任。與木喇福岳福偕乘船入黑龍江,由松花江下駛,即請在松花江會議。八月開議,以三款要求,既指地圖語我,謂格爾畢齊河起,至興安嶺陽面各河止,俱屬俄界,而請將黑龍江、松花江左岸及海口分給俄;又以防備英、法為辭,且登岸設砲,逼遷屯戶。迭由奕山、景淳與之爭議,迄不能決。六年四月,俄人復率艦隊入黑龍江。七年,木喇福岳福歸伊爾庫次克。

時英法聯軍與中國開釁,俄人乘英國請求,遣布恬廷為公使,來議國境及通商事宜。中國拒之。布恬廷遂下黑龍江,由海道進廣東,與英、法、美公使合致書大學士裕誠,請中國派全權大臣至上海議事。答以英、法、美三國交涉事由廣東總督辦理,俄國交涉事由黑龍江辦事大臣辦理。布恬廷乃與三國公使進上海。木喇福岳福乘機擴地於黑龍江左岸,並廣築營舍。遣使詰責,則答以與俄公使在上海協商。尋遣使告黑龍江將軍奕山,在愛琿議界。奕山遂迎木喇福岳福至愛琿會議。木喇福岳福要求以黑龍江為兩國國境,提出條件。明年四月,遂定愛琿條約,先劃分中俄東界,將黑龍江、松花江左岸由額爾古訥河至松花江海口為俄界,右岸順江流至烏蘇里河為中國界;由烏蘇里河至海之地,有接連兩國界者,兩國共管之。於是繪圖作記,以滿、漢、俄三體字刊立界碑。

時英法聯軍已陷大沽砲臺,俄與美藉口調停,因欽差大臣桂良與英、法締約,遂援例增通商七海口。初,中、俄交涉,向由理籓院行文,至是往來交接用與國禮,前限制條款悉除焉。是年,議結五年塔爾巴哈臺焚俄貨圈案,俄屢索償,至是以茶箱貼補之。九年五月,俄遣伊格那提業福為駐北京公使。十年秋,中國與英、法再開戰,聯軍陷北京,帝狩熱河,命恭親王議和。伊格那提業福出任調停,恭親王乃與英、法訂北京和約。伊格那提業福要中國政府將兩國共管之烏蘇里河以東至海之地域讓與俄以為報。十月,與訂北京續約。其重要者:一,兩國沿烏蘇里河、松阿察河、興凱湖、白琳河、瑚布圖河、琿春河、圖們江為界,以東為俄領,以西為中國領;二,西疆未勘定之界,此後應順山嶺、大河,及中國常駐卡倫等處,立標為界,自雍正五年所立沙賓達巴哈之界碑末處起,往西直至齋桑淖爾湖,自此往西南,順天山之特穆爾圖淖爾,南至浩罕邊境為界;三,俄商由恰克圖到北京,經過庫倫、張家口地方,準零星貿易,庫倫設領事官一員;四,中國許喀什噶爾試行貿易。十一年夏五月,倉場侍郎成琦與俄人勘分黑龍江東界。秋七月,俄設領事於漢陽。八月,俄人進槍砲。是年,俄人請進京貿易,不許;後援英、法例,改至天津。

同治元年春二月,與俄訂陸路通商章程。俄人初意欲納稅從輕,商蒙古不加限制,張家口立行棧,經關隘免稽查。總署以俄人向在恰克圖等處以貨易華茶出口,今許其進口貿易,宜照洋關重稅,免礙華商生計。又庫倫為蒙古錯居之地,其為庫倫大臣所屬者,向止車臣汗、圖什業圖汗等地,此外各游牧處所地曠族繁,不盡為庫倫大臣所轄,若許俄隨地貿易,稽查難周。又張家口距京伊邇,嚴拒俄商設立行棧。久之,始定章程二十一款於天津,續增稅則一冊。三月,俄人以喀什噶爾不靖,請暫移阿克蘇通商,不許。

時俄人在伊犁屬瑪呢圖一帶私設卡倫,阻中國赴勒布什之路,復於沙拉托羅海境率兵攔阻查邊人,聲稱哈薩克、布魯特為其屬國,又於各卡倫外壘立鄂博。烏里雅蘇臺將軍明誼等詰責之,不聽。八月,明誼等與俄人會議地界。俄使以續約第二條載有「西疆尚在未定之界,此後應順山嶺、大河之流,及現在中國常駐卡倫」之語,執為定論,並出設色地圖,欲將卡外地盡屬俄國。明誼等以為條約內載自沙賓達巴哈界牌末處起至浩罕邊為界,袤延萬里,其中僅有三處地名,未詳逐段立界之處。況條約內載「現在中國常駐卡倫等處」並無「為界」之語,自不當執以為詞。屢與辨論,不省。忽遣兵隊數百人,執持器械砲車,於伊犁卡倫附近伐木滋擾。是月,俄人請派兵船至滬助剿粵賊,許之。十月,俄人復進槍砲。是年,俄人越界盜耕黑龍江右岸地畝,詰之。

二年四月,俄官布色依由海蘭泡遣人到齊齊哈爾省城借用驛馬,並求通商,請假道前往吉林自松花江回國。黑龍江將軍特普欽以非條約所載,不許。是月,俄人復遣兵隊數百人至塔爾巴哈臺巴克圖卡倫住牧。中國諭令撤回,不聽。又遣隊往伊犁、科布多,又派兵數千分赴齋桑淖爾等地耕種建屋,遣兵四出潛立石壘,為將來議界地步。明誼等議籌防,並與交涉,不省。五月,俄人以哈薩克兵犯伊犁博羅胡吉爾卡倫,擊之始退。六月,復來犯沿邊卡倫,復擊之。七月,俄使進議單,仍執條約第二款為辭。又以條約所載「西直」字為「西南」字誤,必欲照議單所指地名分界,不許更易。乃許照議單換約。於是烏里雅蘇臺將軍明誼上言:「照議單換約,實與烏梁海蒙古及內服之哈薩克、布魯特並伊犁距近邊卡居住之索倫四愛曼人等生計有妨,請籌安插各項人眾及所有生計。」廷諭令與俄人議,須使俄人讓地安插,及中國人照舊游牧。俄人仍不許。

三年秋八月,俄人復遣兵進逼伊犁卡倫。九月,俄使雜哈勞至塔爾巴哈臺與明誼會,仍執議單為詞。時新疆回氛甚熾,朝廷重開邊釁,遂照議單換約。綜計界約分數段:一為烏里雅蘇臺所屬地,即烏城界約所立為八界牌者,自沙賓達巴哈起,往西南順薩彥山嶺至唐努額拉達巴哈西邊末處,轉往西南至賽留格木山嶺之柏郭蘇克山為止,嶺右歸俄,嶺左歸中國。二為科布多所屬地,即科城界約所立牌博二十處者,自柏郭蘇克山起,向西南順賽留格木山嶺至奎屯鄂拉,即往西行,沿大阿勒臺山,至海留圖河中間之山,轉往西南,順此山直至察奇勒莫斯鄂拉,轉往東南,沿齋桑淖爾邊順喀喇額爾齊斯河岸,至瑪呢圖噶圖勒幹卡倫。三為塔爾巴哈臺所屬地,即自瑪呢圖噶圖勒幹卡倫起,先往東南,後向西南,順塔爾巴哈臺山嶺至哈巴爾蘇,轉往西南,順塔境西南各卡倫以迄於阿勒坦特布什山嶺,西北為俄地,東南為中國地。四為伊犁所屬地,即順阿勒坦特布什等山嶺以北偏西偏屬俄,再順伊犁以西諸卡倫至特穆爾圖淖爾,由喀什噶爾邊境迤邐達天山之頂而至蔥嶺,倚浩罕處為界,期明年勘界立牌。會回亂亟,中、俄道阻,界牌遷延未立。

四年,伊犁將軍明緒因回亂,請暫假俄兵助剿,許之。然俄人延不發兵,僅允饟需假俄邊轉解,及所需糧食槍砲火藥允資借。五年春正月,伊犁大城失守,俄允借兵,仍遲延不至。三月,與俄議改陸路通商章程。俄人欲在張家口任意通商,及刪去「小本營生」、天津免納子稅二事。中國以張家口近接京畿,非邊疆可比,不可無限制。「小本營生」字樣若刪去,則俄商貨色人數無從稽考。惟天津免納子稅,與他國販土貨出口僅納一正稅相合,遂議免天津子稅。而張家口任意通商,及刪去「小本營生」事,並從緩商。五月,俄人請往黑龍江內地通商,不許。是月,俄人占科布多所屬布克圖爾滿河北境。六年六月,俄使倭良嘎哩以西疆不靖,有妨通商,貽書總署責問。是月,俄人占科布多所屬霍呢邁拉扈卡倫及烏里雅蘇臺所屬霍呢音達巴罕之烏克果勒地。詰之,不省。

七年二月,俄人越界如庫倫所屬烏雅拉噶哈當蘇河等處採金,阻之,不聽,反以為俄國游牧地,不認雍正五年所定界址及嘉慶二十三年兩國所繪地圖界址。中國屢與爭議,不決。時新疆毗連俄境未立界牌鄂博,烏里雅蘇臺將軍麟興等請派大員會定界址,許之。然遲久未勘。俄人又私伐樹株,標記所侵庫倫所屬地。又於朝鮮慶興府隔江遙對之處建築房屋,朝鮮國王疑懼,咨中國查詢。七月,俄人又如呼倫貝爾所屬地盜伐木植,阻之,不聽。

八年春三月,與俄國續訂陸路通商條約。五月,榮全等與俄立界大臣巴布闊福等會立界牌鄂博,至烏里雅蘇臺所屬賽留格木,俄官藉口原約第六條謂非水源所在,辯議三日,始遵紅線條約,於博果蘇克壩、塔斯啟勒山各建牌博,其由珠嚕淖爾至沙賓達巴哈分界處,原圖所載,險阻難行。俄官輒欲繞道由珠嚕淖爾迤北數十里唐努山之察布雅齊壩上建立鄂博,由此直向西北,繞至沙賓達巴哈。朝旨不許,乃改由珠嚕淖爾東南約十數里哈爾噶小山立第三牌博。又順珠嚕淖爾北唐努山南約二百里察布雅齊壩上立第四牌博,照原圖所繪紅線以外珠嚕淖爾圈出為俄國地,哈爾噶小山以東、察布雅齊壩以北,為中國地。又順珠嚕淖爾北唐努山南直向西行,至珠嚕淖爾末處轉折而北而東,均系紅線以外科屬阿勒坦淖爾烏梁海地,已分給俄,至庫色爾壩上已接唐努烏梁海向西偏北極邊地,於此壩上立第五牌博。由此向西,無路可通,乃下壩向東北入唐努烏梁海,復轉折而西而北,至唐努鄂拉達巴哈末處,迤西有水西流,名楚拉察河,亦系紅線以外分給俄者,於此立第六牌博。其東南為唐努烏梁海邊境,其西北為俄地。又由楚拉察河順薩勒塔斯臺噶山至蘇爾壩上,立第七牌博。由此壩前進,直至沙賓達巴哈山脈,一線相連,此處舊有兩國牌博。與此壩相接,因不再立。榮全仍欲復增牌博,俄官允出具印結,聽中國自立,榮全乃遣人立焉。

八月,科布多參贊大臣奎昌又與俄官議立俄屬牌博,俄官仍欲以山形水勢為憑。奎昌等抗辯,非按原圖限道建立不可,遂於科布多東北邊末布果素克嶺至瑪呢圖噶圖勒幹各立牌博,至塔爾巴哈臺所屬布倫托海分界。中國因塔城未經克復,道途梗塞,未暇辦理。俄使遽欲於塔城所屬瑪呢圖噶圖勒幹至哈巴爾蘇從北起先建鄂博,並稱無中國大臣會辦,亦可自行建立。中國以分界關兩國地址,決無獨勘之理,允俟明年春融,派員會勘。是年,俄人輪船由松花江上駛抵呼蘭河口,要求在黑龍江內地通商。黑龍江將軍德英以聞,朝旨以非條約所載,不許。

九年正月,俄人來言哈巴爾蘇牌博已於去秋自行建立。中國以不符會辦原議詰之,並命科布多大臣奎昌按圖查勘。二月,俄人復請派員赴齊齊哈爾、吉林與將軍議邊事,命禁阻之。秋八月,奎昌至塔城所屬瑪呢圖噶圖勒幹卡倫,與俄立界大臣穆魯木策傅會勘俄自立牌博,中國亦於俄國自立牌博內建立牌博。復往塔爾巴哈臺山嶺等處勘查,直至哈巴爾蘇,共立牌博十。至是分界始竣。十月,庫倫辦事大臣張廷岳等以烏里雅蘇臺失陷,烏梁海與俄界毗連,請防侵占。

十年夏五月,俄人襲取伊犁,復欲乘勝收烏魯木齊。帝命將軍、參贊大臣等止其進兵,不省。既又出兵二千,欲剿瑪納斯賊,以有妨彼國貿易為詞。中國命榮全、奎昌、劉銘傳等督兵圖復烏魯木齊,規收伊犁。俄人既得伊犁,即令圖爾根所駐索倫人移居薩瑪爾屯。又於金頂寺造屋,令漢、回分駐綏定城、清水河等處。復遣人赴喀喇沙爾、晶河,勸土爾扈特降。又說瑪納斯賊投降。事聞,命防阻。十二月,俄人請援各國例通商瓊州,許之。是年,俄人帶兵入科布多境。諭令退兵,久之始去。

十一年四月,伊犁將軍榮全與俄官博呼策勒傅斯奇會於俄國色爾賀鄂普勒,議交還伊犁事。俄官置伊犁不問,僅議新疆各處如何平定,並以助兵為言,要求在科布多、烏里雅蘇臺、烏魯木齊、哈密、阿克蘇、喀什噶爾等處通商、設領事,及賠補塔城商館,及匡蘇勒官龐齡等被害各節,並請讓科布多所屬喀喇額爾濟斯河及額魯特游牧額爾米斯河歸俄。榮全等拒之。博呼策勒傅斯奇遂置伊犁事不議。已忽如北京總署,請仍與榮全會議。博呼策勒傅斯奇又忽辭歸國。至是接收伊犁又遲延矣。

八月,俄人載貨入烏魯木齊所屬三塘湖,請赴巴里坤、哈密等處貿易。阻之,不聽。既聞回匪有由哈密東山西竄察罕川古之信,乃折回。已復有俄官來文,謂伊犁所屬土爾扈特游牧西湖、晶河、大沿子居民均歸順俄國,中國軍隊不得往西湖各村。中國以當初分界在伊犁迤西,並無西湖之名,西湖系烏魯木齊所屬軍隊,原由總署與俄使議有大略,何可阻止?拒之。時榮全將帶兵由塔赴伊安設臺站,俄人以越俄國兵所占地,不許。又阻榮全接濟錫伯銀兩。十月,俄商赴瑪納斯貿易,中途被殺傷五十餘人。十二年夏四月,俄人忽帶兵及哈薩克、漢、回等眾,入晶河土爾扈特游牧,索哈薩克所失馬,並執貝子及固山達保來綽囉木等,又修治伊犁迤東果子溝大路,更換錫伯各官,圖東犯,又於塔爾巴哈臺所屬察罕鄂博山口駐兵,盤詰往來行旅。十三年八月,俄人自庫倫貿易入烏里雅蘇臺建房,詰以非條約所載,不省。旋命陜甘總督左宗棠督辦新疆軍務。

光緒元年夏五月,俄游歷官索思諾等來蘭州,言奉國主之命,欲與中國永敦和好,俟中國克復烏魯木齊、瑪納斯,即便交還。左宗棠以聞。既而左宗棠以新疆與俄境毗連,交涉事繁,請旨定奪。帝命左宗棠主辦。

三年,議修陸路通商章程。俄使布策欲於伊犁未交之先,通各路貿易。中國不允,僅允西路通商,而仍以交收伊犁與商辦各事並行為言。俄人又以榮全張示激伊犁人民不遵俄令,烏里雅蘇臺官吏擅責俄人,江海關道扣留俄船,英廉擅殺哈薩克車隆,及徵收俄稅,指為違約,謂非先議各事不可。會新疆南路大捷,各城收復,回匪白彥虎等竄入俄,中國援俄約第八款,請其執送。屢與理論,未決。

四年五月,命吏部左侍郎崇厚使俄,議還伊犁及交白彥虎諸事。十二月抵俄。五年二月,與俄外部尚書格爾斯開議。格爾斯提議三端:一通商,一分界,一償款。而通商、分界又各區分為三。通商之條:一,由嘉峪關達漢口,稱為中國西邊省分,聽其貿易;一,烏魯木齊、塔爾巴哈臺、伊犁、喀什噶爾等處,稱為天山南北各路,妥議貿易章程;一,烏里雅蘇臺、科布多等處,稱為蒙古地方,及上所舉西邊省分,均設立領事。分界之條:展伊犁界,以便控制回部;一,更定塔爾巴哈臺界,以便哈薩克冬夏游牧;一,新定天山迤南界,以便俄屬浩罕得清界線。崇厚皆允之,惟償款數目未定。崇厚以聞,命塔爾巴哈臺參贊大臣錫綸接收伊犁及分界各事。既議償款盧布五百萬圓,俄亦遣高復滿等為交還伊犁專使。

崇厚將赴黑海畫押回國,而恭親王奕等以崇厚所定條款損失甚大,請飭下李鴻章、左宗棠、沈葆楨、金順、錫綸等,將各條分別酌核密陳。於是李鴻章等及一時言事之臣交章彈劾,而洗馬張之洞抗爭尤力。略謂:「新約十八條,其最謬妄者,如陸路通商由嘉峪關、西安、漢中直達漢口,秦隴要害、荊楚上游,盡為所窺。不可許者一。東三省國家根本,伯都訥吉林精華,若許其乘船至此,即與東三省任其游行無異,是於綏芬河之西無故自蹙地二千里;且內河行舟,乃各國歷年所求而不得者,一許俄人,效尤踵至。不可許者二。朝廷不爭稅課,當恤商民。若準、回兩部,蒙古各盟,一任俄人貿易,概免納稅,華商日困;且張家口等處內地開設行棧,以後逐漸推廣,設啟戎心,萬里之內,首尾銜接。不可許者三。中國屏籓,全在內外蒙古,沙漠萬里,天所以限夷狄。如蒙古全站供其役使,一旦有事,音信易通,必撤籓屏,為彼先導。不可許者四。條約所載,俄人準建卡三十六,延袤廣大,無事而商往,則譏不勝譏;有事而兵來,則御不勝禦。不可許者五。各國商賈,從無許帶軍器之例。今無故聲明人帶一槍,其意何居?不可許者六。俄人商稅,種種取巧,若各國希冀均霑,洋關稅課必至歲絀數百萬。不可許者七。同治三年新疆已經議定之界,又欲內侵,斷我入城之路。新疆形勢,北路荒涼,南城富庶,爭磽瘠,棄膏腴,務虛名,受實禍。不可許者八。伊犁、塔爾巴哈臺、科布多、烏里雅蘇臺、喀什噶爾、烏魯木齊、古城、哈密、嘉峪關等處準設領事官,是西域全疆盡由出入。且各國通例,惟沿海口岸準設外邦領事。若烏里雅蘇臺等,乃我邊境,今日俄人作俑,設各國援例,又將何以處之?不可許者九。名還伊犁,而三省山嶺內卡倫以外盤踞如故,割霍爾果斯河以西、格爾海島以北,金頂寺又為俄人市廛,約定俄人產業不更交還,地利盡失。不可許者十。」又言:「改議之道:一在治崇厚以違訓越權之罪;一在請諭旨將俄人不公平,臣民公議不原之故,布告中外,行文各國,使評曲直;一在據理力爭,使知使臣畫押,未奉御批示覆,不足為據;一在設新疆、吉林、天津之防,以作戰備。」疏入,命與修撰王仁堪等及庶吉士盛昱所奏,並交大學士等議,並治崇厚罪。

六年正月,命大理寺少卿曾紀澤為使俄大臣,續議各款。時廷臣多主廢約,曾紀澤以為廢約須權輕重,因上疏曰:「伊犁一案,大端有三:曰分界,曰通商,曰償款。三端之中,償款固其小焉者也。即通商一端,亦較分界為稍輕。查西洋定約之例有二,一則長守不渝,一可隨時修改。長守不渝者,分界是也。分界不能兩全,此有所益,則彼有所損,是以定約之際,其慎其難。隨時修改者,通商是也。通商之損益,不可逆睹,或開辦乃見端倪,或久辦乃分利弊,是以定約之時,必商定年限修改,所以保其利而去其弊也。俄約經崇厚議定,中國誠為受損,然必欲一時全數更張,而不別予一途以為轉圜之路,似亦難降心以相從也。臣以為分界既屬永定,自宜持以定力,百折不回。至於通商各條,惟當即其太甚者,酌加更易,餘者宜從權應允。」

時俄人以中國治崇厚罪,增兵設防,為有意尋釁,欲拒紀澤不與議事。英、法二使各奉本國命,亦以因定約治使臣罪為不然,代請寬免。中國不得已,允減崇厚罪,詔仍監禁。已又與俄使凱陽德先議結邊界各案。

六年七月,紀澤抵俄,侍郎郭嵩燾疏請準萬國公法,寬免崇厚罪名,紀澤亦請釋崇厚,許之。初紀澤至俄,俄吉爾斯、布策諸人咸以非頭等全權大臣,欲不與議,遣布策如北京議約。已成行,而朝旨以在俄定議為要,命紀澤向俄再請,始追回布策。紀澤與議主廢約。俄人挾崇約成見,屢與忤。紀澤不得已,乃遵總署電,謂可緩索伊犁,全廢舊約。尋接俄牒,允還帖克斯川,餘不容議。布策又欲俄商在通州租房存貨,及天津運貨用小輪船拖帶。紀澤以非條約所有,拒之。而改約事仍相持不決。

十一月,俄牒中國,允改各條,其要有七:一,交還伊犁;二,喀什噶爾界務;三,塔爾巴哈臺界務;四,嘉峪關通商,允許俄商由西安、漢中行走,直達漢口;五,松花江行船至伯都訥;六,增設領事;七,天山南北路貿易納稅。曾紀澤得牒,以俄既許讓,則緩索之說,自可不議。於是按約辯論:於伊犁,得爭回南境;喀什噶爾,得照兩國現管之地,派員再勘;塔爾巴哈臺,得於崇厚、明誼所訂兩界之間,酌中勘定;嘉峪關通商,得仿照天津辦理,西安、漢中兩路及漢口字均刪去;松花江行船,因愛琿條約誤指混同江為松花江,又無畫押之漢文可據,致俄人歷年藉口,久之始允將專條廢去,聲明愛琿舊約如何辦法,再行商定;增設領事,俄人請設烏魯木齊一處,總署命再商改,始將烏魯木齊改為吐魯番,餘俟商務興盛時再議增設;天山南北路貿易納稅,將原約「均不納稅」字改為「暫不納稅,俟商務興盛再訂稅章」。此外,償款,崇厚原約償五百萬盧布,俄人以伊犁南境既已讓還,欲倍原數,久之始允減定為盧布九百萬。紀澤又以此次改約並未用兵,兵費之名絕不能認。於是將歷年邊疆、腹地與俄人未結之案,有應賠應恤者一百九案,並入其中,作為全結。又於崇厚原訂俄章字句有所增減。如條約第三條刪去伊犁已入俄籍之民,入華貿易游歷許照俄民利益一段;第四條俄民在伊犁置有田地,照舊管業,聲明伊犁遷出之民,不得援例,且聲明俄民管業既在貿易圈外,應照中國民人一體完納稅餉;並於第七條伊犁西境安置遷民之處,聲明系安置因入俄籍而棄田地之民;第六條寫明所有前此各案,第十條吐魯番非通商口岸而設領事,暨第十三條張家口無領事而設行棧,均聲明他處不得援以為例;第十五條修約期限,改五年為十年。章程第二條貨色包件下添言主牲畜字樣,其無執照商民,照例懲辦,改為從嚴罰辦;第八條車腳運夫,繞越捷徑,以避關卡查驗,貨主不知情,分別罰辦之下,聲明海口通商及內地不得援以為例。是為收回伊犁條約。又同時與俄訂陸路通商章程。七年正月,與俄外部尚書吉爾斯及前駐京使臣布策,在俄都畫押鈐印,旋批準換約。七月,賀俄君即位,遞國書。索逆犯白彥虎等,俄以白彥虎等犯系屬公罪,不在條約所載之列,不允交還,允嚴禁。

尋命伊犁將軍金順、參贊大臣升泰接收伊犁。八年二月,接收訖。金順進駐綏定城。升泰會同俄官勘分地界,並以哈密幫辦大臣長順會辦西北界務,巴里坤領隊大臣沙克都林扎布會辦西南界務。四月,俄人帶兵潛入科布多所屬哈巴河,清安等以聞。因言圖內奎峒山、黑伊爾特什河、薩烏爾嶺等處形勢,與積年新舊圖說不符。朝旨命就原圖應勘之處,力與指辯,酌定新界。

十一月,分界大臣長順等與俄官佛哩德勘分伊犁中段邊界。先是距那林東北百餘里之格登山有高宗平準噶爾銘勛碑,同治三年已畫歸俄,至是爭回,立界約三條。

九年,督辦新疆軍務大臣劉錦棠以新疆南界烏什之貢古魯克地為南北要津,請按約索還。先是,舊約所載伊犁南界,系指貢古魯克山頂而言。上年沙克都林扎布與俄使勘分南界,由貢古魯克等處卡倫繞貢古魯克山麓至別疊里達阪設立界牌,侵占至畢底爾河源,故錦棠以為言。朝旨命長順等據理辯論。既而沙克都林扎布又與俄官咩登斯格勘伊犁南界,俄人必欲以薩瓦巴齊為界,沙克都林扎布以為薩瓦巴齊在天山之陽,距天山中梁尚遠,不許,乃以天山中梁為界。又立牌博於別疊里達阪,是為喀什噶爾界約。

七月,分界大臣升泰等與俄官巴布闊福等勘分科、塔界務。巴布闊福等欲照圖中直線,以哈巴河為界。升泰等以哈巴河地居上游,為科境之門戶,塔城之籓籬,若劃分歸俄,不惟原住之哈薩克、蒙、民等無地安插,即科屬之烏梁海、塔屬之土爾扈特等處游牧之所,亦俱受逼,界址既近,釁端必多,拒之。俄使乃允退離哈巴河迤西約八十餘里之畢裏克河劃分。升泰等以畢裏克系小河,原圖並未繪刊,若以此劃界,則哈巴河上游仍為俄所占,復與力爭。俄使乃允復退出五十里,議定在於阿拉喀別克河為界,計距哈巴河至直線共一百三十餘里,即原圖黃線之旁所開之小河也。餘均照黃線所指方位劃分。至兩國所屬之哈薩克,原歸俄者歸俄,原歸中國者歸中國。如有人歸中國而產業在俄,或人居俄而產業在中國,均照伊犁辦法,以此次議定新界換約日為始,限一年遷移。約定,又與俄官斐里德勘塔城西南未分之界。俄使意欲多分,升泰以此段界務,新約第七條內業經指明,系順同治三年塔城界約所定舊界,即原約第二條內所指依額爾格圖巴爾魯克、莫多巴爾魯克等處卡倫之路辦理,是原有圖線條約可循,非若他處尚須勘酌議分可比,不許。俄使乃以巴爾魯克山界內住牧之哈薩克久已投俄,一經定界,不免遷移,請借讓安插,許之。仍援舊約第十條所開塔屬原住小水地方居民之例,限十年外遷,隨立牌博。

九月,分界大臣額爾慶額等與俄官撇斐索富勘分科布多界。自阿拉克別克河口之喀拉素畢業格庫瑪小山梁起,至塔木塔克薩斯止,共立牌博四,又立牌博於阿克哈巴河源。先是喀什噶爾西邊界務已經長順與俄人劃分,以依爾克池他木為界,而幫辦軍務廣東陸路提督張曜以為有誤,請飭覆查。長順以勘界系依紅線,依爾克池他木雖舊圖不載,而新圖正在紅線界限,不容有誤。尋總署以約內有現管為界一語,意曾紀澤定約時,必因新圖不無縮入,又知左宗棠咨報克復喀城,有占得安集延遺地,邊界展寬之說,故約內添此一語。既以現管為界,即可不拘紅線,仍命長順與爭。俄人以喀拉多拜、帖列克達灣、屯木倫三處雖現為中國所管,然均在線外百數十里,執不允,仍依紅線履勘,自喀克善山起,至烏斯別山止,共立牌博二十二,指山為界者七,遂定議。是為續勘喀什噶爾界約。是年,塔爾巴哈臺參贊大臣錫綸與俄人會議俄商在塔貿易新圈地址。

十年三月,塔爾巴哈臺參贊大臣錫綸與俄人會定哈薩克歸附條約,凡在塔城境內混居之哈薩克提爾賽哷克部、拜吉格特部、賽波拉特部、托勒圖勒部、滿必特部、柯勒依部、圖瑪臺部各大小鄂拓克,約五千餘戶,除原遷回俄境外,其自原歸中國者一千八百戶,均由中國管轄,並訂管轄條款。七月,法因越南與中國開釁,法人請俄國保護在華之旅人教士及一切利益,俄使允保護,牒中國。

十一年三月,總署以吉林東界牌博中多舛錯,年久未修,請簡大員會勘,據約立界。先是俄人侵占琿春邊界,將圖們江東岸沿江百餘里誤為俄國轄地,並於黑頂子安設俄卡,招致朝鮮流民墾地。前督辦寧古塔等處事宜吳大澂,請飭查令俄人交還。朝廷乃命吳大澂等為欽差大臣,與俄人訂期會勘。大澂等以咸豐十年北京條約中俄東界順黑龍江至烏蘇里河及圖們江口所立界牌,有俄國「阿」「巴」「瓦」「噶」「達」「耶」「熱」「皆」「伊」「亦」「喀」「拉」「瑪」「那」「倭」「怕」「啦」「薩」「土」「烏」十二字頭,十一年成琦勘界圖內尚有「伊」「亦」「喀」「拉」「瑪」「那」「倭」「怕」「啦」「薩」「土」「烏」十二字頭,何以官界記文內僅止「耶」「亦」「喀」「拉」「那」「倭」「怕」「土」八字頭?圖約不符。又界牌用木難經久,應請易石,及補立界牌。又以俄人所占黑頂子地,即在「土」字界牌以內,尤為重要。又以自琿春河源至圖們江口五百餘里,處處與俄接壤,無一界牌。又成琦所立界牌八處,惟「土」字一牌之外,尚有「烏」字一牌。以交界記文而論,圖們江左邊距海不過二十里,立界牌一,上寫俄國「土」字頭,是「土」字一牌已在交界盡處,更無補立「烏」字界牌之地,二者必有一誤。又補立界牌,無論「烏」字、「土」字,總以圖們江左邊距海二十里之地為斷。十二年夏,吳大澂等赴俄境巖杵河,與俄勘界大員巴啦諾伏等商議界務。大澂等首議補立「土」字界牌,因咸豐十一年所立「土」字界牌之地,未照條約記文「江口相距二十里」之說。大澂等與之辯論,俄員以為海灘二十里,俄人謂之海河,除去海河二十里,方是江口。大澂等以為江口即海口,中國二十里即俄國十里,沙草峰原立「土」字界牌,既與條約記文不符,此時即應更正。巴啦諾伏仍以舊圖紅線為詞。久之,始允於沙草峰南越嶺而下至平岡盡處立「土」字牌,又於舊圖內「拉」字、「那」字兩牌之間,補立「瑪」字界牌,條約內「怕」字、「土」字兩牌之間,補立「啦」「薩」二字界牌,悉易以石。又於界牌相去甚遠之處,多立封堆,或掘濠為記。至俄人所占黑頂子地,亦允交還。大澂等又以寧古塔境內「倭」字、「那」字二界牌,與記文條約不符,請更正,緣「倭」字界牌本在瑚布圖河口,因當時河口水漲,木牌易於沖失,權設小孤山頂,離河較遠。大澂等以為若以立牌之地即為交界之所,則小孤山以東至瑚布圖河口一段又將割為俄地。乃與巴啦諾伏議定,將「倭」字石界牌改置瑚布圖河口山坡高處,「那」字界牌原在橫山會處,距瑚布圖河口百餘里,僅存朽爛木牌二尺餘,因易以石,仍立橫山會處,迤西即系小綏芬河源水向南流處,又於交界處增立銅柱。是為中俄琿春東界約。

是年,俄莫斯克瓦商人欲攜貨赴科布多、哈密、肅州、甘州、涼州、蘭州等處貿易。中國以科布多、哈密、肅州皆系條約訂明通商處所,自可前往;甘州、涼州、蘭州系屬內地,非條約所載,不許。十四年,俄人在烏梁海所屬,掘金開地建房,阻之不聽。十五年,俄人越界入黑龍江所屬,以刈草為名,搭棚占地。總署以詢北洋大臣李鴻章,鴻章請但許刈草,不許搭棚,切與要約,以示限制,從之。十六年,俄商請照約由科布多運貨回國,許之。初,俄商由陸路運貨回國,舊章祗有恰克圖一路。光緒七年,改訂新約,許由尼布楚、科布多兩路往來運貨。至是,許由科布多行走,其收繳執照諸辦法,由科布多參贊大臣派員查驗。是年,出使大臣洪鈞以俄人在恰克圖境穴地取金,請自設廠掘金,不果。俄人又勾結藏番私相餽贈。十七年,俄遣兵至海參崴開辦鐵路。是年,俄太子來華游歷,命李鴻章往煙臺款接。初俄欲中國簡親籓接待,未允,乃遣鴻章往,有加禮。

十八年,與俄人議接琿春、海蘭泡陸路電線。先是中國陸路電線創自光緒六年,惟丹國大北公司海線,先於同治十年由香港、廈門迤邐至上海,一通新加坡、檳榔嶼以達歐洲,名為南線;一通海參崴,由俄國亞洲旱線以達歐洲,名為北線。俄、丹早有連線之約。嗣丹復與英合辦水線。逮各省自設陸線,並拆去英、丹在滬、粵已成之陸線。迨中國吉林、黑龍江線成,與俄之東海濱境內近接。大北公司等深慮中俄線接,分奪其利,屢起爭議。至是,命鴻章與俄使喀希呢議約,酌擬滬、福、廈、港公司有水線處,不與爭減,此外各口電價,亦不允水線公司爭減,遂定議。是為中俄邊界陸路電線相接條約。

是年,俄入帕米爾。帕米爾高原在中國回疆邊外,舊為中國所屬。自俄、英分爭,而迤北、迤西稍稍歸屬於俄,迤南小部則附於英屬之阿富汗,惟東路、中路久服中國,迄今未變。俄欲取帕米爾以通印度,英人防之,以劃清阿富汗邊界為辭,欲使中國收轄帕境中間之地,勘明界址;俄人亦欲會同中國勘界分疆,不使英與聞。至是,俄兵入帕,英領事璧利南以從前英、俄立約,喀什噶爾、阿富汗之間並無俄地,原出作證,又據所繪圖,力闢俄圖。俄人不顧,欲以郎庫郎里湖為界,移軍而南,將據色勒庫爾。色勒庫爾乃莎車境,益逼近新疆南境。陜甘總督楊昌濬請設防,許之。既因出使大臣洪鈞所繪地圖有誤,李鴻章據薛福成所寄圖,謂:「喀約既稱烏斯別里南向系中國地界,自應認定『南向』二字方合,若無端插入『轉東』二字,所謂謬以千里;況烏斯別里為蔥嶺支脈,如順山梁為自然界,以變一直往南之說,不特兩帕盡棄,喀什噶爾頓失屏蔽,葉爾羌、西藏等全撤籓籬,且恐後此藉口於交界本循山脊而行,語更寬混,尤難分劃,此固萬難允也。如彼以喀約語太寬混為辭,擬仿照北亞墨利加英、美用經緯度分界之法,以烏斯別里山口之經線為界,北自烏斯別里山口一直往南,至阿富汗界之薩雷庫里湖為止,方與經線相合。如此,則大帕米爾可得大半,小帕米爾全境俱在線內,其簡當精確,更勝於自然界,而與原議之約亦相符合。否則阿里楚爾山環三面,惟東一面與喀境毗連,界亦自然。何彼竟舍外之山梁,而專用內之山梁,以求多占地界耶?」議久不決。是年,俄茶在戈壁被焚,索償,允由攬運俄茶之人分償,俄使欲公家代償,不允。

十九年四月,議收俄國借地。初,俄借塔爾巴哈臺所屬之巴爾魯克山,給所屬哈薩克游牧,限十年遷回。至是限滿,伊犁將軍長庚請遣員商辦,俄人請再展十年,不許。久之,俄始允還地遷民,遂立交山文約,聲明限滿不遷,即照人隨地歸之約。又續立收山未盡事宜文約,以清釐兩屬哈薩克欠債及盜牲畜等事。

二十年,與俄復議帕界。俄初欲據郎庫里、阿克塔什,出使大臣許景澄以此為中國地,力爭不許。既而俄允於色勒庫爾山嶺之西,請中國指實何地相讓,中國仍以自烏仔別里至薩雷庫里湖為言,俄人不允。總署欲改循水為界,擬循阿克拜塔爾河,南逾阿克蘇河,東南循河至阿克塔什平地,轉向西南,循伊西提克河,直至薩雷庫里湖,各將分界水名詳敘,仍未決。是年俄嗣皇即位,遣布政使王之春為專使往賀。

明年春,與日本講成,割臺灣及遼河以南地,俄聯法、德勸阻遼南割地,日本不允。俄忽調戰艦赴煙臺,日本允還遼,惟欲於二萬萬外加償費。俄皇特命戶部大臣威特見出使大臣許景澄,雲欲為中國代借鉅款,俾早日退兵。許景澄以聞。總署命與俄商辦,遂訂借法銀四萬萬佛郎,以海關作保,年息四釐,分年償還。是為中俄四釐借款合同。

九月,俄人分赴東三省勘路。初俄興造悉畢爾鐵路,欲在滿洲地方借地接修。總署議自俄境入華境以後,由中國自造。十月,俄水師輪船請暫借山東膠澳過冬,許之。山東巡撫李秉衡上言:「煙臺芝罘島並非不可泊船,膠州向非通商口岸,應請飭俄使進泊後,退出須定期限。」報可。十二月,賞俄使喀希呢及法、德二使頭等第三寶星。

二十二年四月,俄皇尼哥拉斯二世加冕,命李鴻章為專使,王之春為副使,贈俄皇頭等第一寶星。九月,與俄訂新約。時李鴻章尚未回國,俄使喀希呢特密約求總署奏請批準。約成,俄使貴族鄔多穆斯契以報謝加冕使來北京,議立華俄銀行,遂命許景澄與俄結華俄道勝銀行契約,中國出股本銀五百萬兩,與俄合辦。別立中國東省鐵路公司,又立條例九章,其第二章銀行業務之第十項,規定對於中國之業務:一,領收中國內之諸稅;二,經營地方及國庫有關系之事業;三,鑄造中國政府允許之貨幣;四,代還中國政府募集公債之利息;五,布設中國內之鐵道電線,並訂結東清鐵道會社條約,以建造鐵路與經理事宜悉委銀行。

二十三年十一月,俄以德占膠州灣為口實,命西伯利亞艦隊入旅順口,要求租借旅順、大連二港,且求築造自哈爾濱至旅順之鐵道權。十二月,俄以兵入金州城徵收錢糧,阻之,不省。鄉民聚眾抗拒,俄人遂於貔口槍斃華民數十。奉天將軍依克唐阿以聞,命出使大臣楊儒迅與俄人商辦,議久不決。俄皇謂許景澄曰:「俄船借泊,一為膠事,二為度冬,三為助華防護他國占據。」景澄再與商,不應。二十四年二月,命許景澄專論旅、大俄船借泊及黃海鐵路事,俄以德既占膠州,各國均有所索,俄未便不租旅、大。又鐵路請中國許東省公司自鴨綠江至牛莊一帶水口擇宜通接,限三月初六日訂約,過期俄即自行辦理,詞甚決絕。既而俄提督率兵登岸,張接管旅、大示,限中國官吏交金州城。中國再與交涉,俄始允兵屯城外。遂訂約,將旅順口及大連灣暨附近水面租與俄。已畫押遣員分勘,將軍伊克唐阿以「附近」二字太寬泛,電總署力爭,謂金西、金東各島,離岸一二十里、三四十里不等,謂之「附近」尚可,至索山以南廟兒七島,近者三四十里,遠者二百餘里,在山東登萊海面,非遼東所屬,不得謂之「附近」。爭之再三,俄請將廟群島作為隙地,免他國占據。總署告以中國但可允認不讓與他國享用並通商等利益,不能允作隙地,致損主權。俄人又請允許立字不設砲臺、不駐兵。總署仍與力駁,不省。久之,始允照中國議,刪去「作為隙地」及「不設砲臺」等語;復於專條廟群島下增繕「不歸租界之內」字,而金州東海海陽、五蟒二島仍租俄。

七月,出使大臣許景澄、楊儒與東省鐵路公司續訂合同。初,中、俄會訂條約,原許東省鐵路公司由某站起至大連灣,或酌量至遼東半島營口、鴨綠江中間沿海較便地方,築一枝路,未行。至是與議,許景澄與俄外部商明枝路末處在大連灣海口,不在遼東半島沿海別處,列入專條訂合同。俄人嗣以造路首重運料,擬照原合同所許各陸路轉運之事,訂定暫築通海口枝路暨行船辦法,並自行開採煤礦木植等事。許景澄等以原合同第一款,載明中國在鐵路交界設關,照通商稅則減三分之一,此系指陸路而言,今大連灣海口開作商埠,貨物來往內地,竟援減徵稅,恐牛莊、津海兩關必致掣礙。至內地與租地交界,視中俄兩國交界有別,設關處所亦須變通,擬改定專款。俄人尚欲並開各礦產,拒之,並議限制轉運開採各事。又商加全路工竣年限,俾暫築枝路屆期照拆。凡七款:一,枝路名東省鐵路南滿洲枝路;二,造路需用料件,許公司用輪船及別船樹公司旗,駛行遼河並枝河及營口並隙地各海口,運卸料件;三,公司為運載料件糧草便捷起見,許由南路暫築枝路至營口及隙地海口,惟造路工竣,全路通行貿易後,應將枝路拆去,不得逾八年;四,許公司採伐在官樹株,每株由總監工與地方官酌定繳費,惟盛京御用產物,暨關系風水,不得損動,並許公司所過開採煤礦,亦由總監工與地方官酌定,計斤納稅;五,俄可在租地內自酌稅則,中國可在交界徵收貨物從租界運入內地,或由內地運往租地之稅,照海關進出口稅則無增減,並允俄在大連灣設關,委公司代徵,別遣文官駐扎為稅關委員;六,許公司自備行海商船,照各國通商例,如有虧折,與中國無涉,應照原合同十二條價買及歸還期限辦理;七,造路方向所過地方,應俟總監工勘定,由公司或北京代辦人與鐵路總辦公司商定。復定鐵路經過奉天,應繞避陵寢,俄允繞距三十里,遂畫押。

二十五年,盛京將軍文興等遣知府福培、同知塗景濤與俄員倭高格伊林思齊等,勘分旅大租界。俄員擬先從租地北界西岸亞當灣起勘。福培等以中國輿圖無亞當灣地名,應照總署電,亞當即普蘭店之文為憑,當從普蘭店西海灣之馬虎島起。俄員以續約明言西從亞當灣北起,無普蘭店字,堅不允改。遂從北界西岸起,次第立碑,至大海濱,凡三十有一碑,北刻漢文,南鐫俄國字母。復立小碑八,以數目為號。界線由西至東,長九十八里餘九十四弓。界既定,與俄員會議分界專條,又將所繪界圖,用華、俄文註明,畫押蓋印,互換後,分呈俄使及總署批定完結。初由李鴻章、張廕桓與俄使巴布羅福訂此約於北京,至是,命王文韶、許景澄加押。

時中國欲自造山海關至營口枝路,英欲投資。俄使牒總署,謂借用外國資本,與續約相背。俄人又以東省鐵路將興工,擬在北京設東省鐵路俄文學堂,招中國學生學習俄國語言文字,以備鐵路調遣之用。許之。是年,俄以遼東租借地為「關東省」。

二十六年,拳匪亂,各國聯軍入北京,俄乘勢以兵占東三省,藉口防馬賊、保鐵路。初,奉天土匪先攻俄鐵道警衛兵,亂兵燒天主教堂,破毀鐵嶺鐵道,掠洋庫;旋攻遼陽鐵道,俄鐵道員咸退去,同時黑龍江亦砲擊俄船。俄聞警,遣軍分道進攻,由璦琿、三姓、寧古塔、琿春進據奉天,乃迫將軍增祺訂奉天交地約,擬在東三省駐兵,政賦官兵均歸俄管轄。時朝廷以慶親王、李鴻章為全權與各國議款,並命駐俄欽使楊儒為全權大臣,與俄商辦接收東三省事。楊儒與爭論久,始允作廢。而俄人別出約稿相要,張之洞等連電力爭,遂暫停議。

二十七年七月,各國和議成,李鴻章乃手擬四事:一,歸地;二,撤兵;三,俄國在東三省,除指定鐵路公司地段,不再增兵;四,交還鐵路,償以費用。與俄使開議於北京。講未成而鴻章卒,王文韶繼之。二十八年三月,訂約四條。

四月,俄人強占科布多所屬阿拉克別克河,參贊大臣瑞洵以聞,命外務部商辦,不得要領。七月,鐵路公司與華俄道勝銀行訂立正太鐵路借款及行車合同,又與俄續訂接線展限合同。九月,交還關外鐵路及撤退錦州遼河西南部之俄軍,是為第一期撤兵。至翌年三月第二期,金州、牛莊、遼陽、奉天、鐵嶺、開原、長春、吉林、寧古塔、琿春、阿拉楚喀、哈爾濱駐扎之俄兵仍不如期撤退,俄代理北京公使布拉穆損向外務部新要求七款,拒之,俄使撤回要求案。會俄使雷薩爾復任,復提新議五款,宣言東省撤兵,斷不能無條件,縱因此事與日本開戰,亦所不顧。

三十年,日、俄開戰,中國守中立。是年,俄造東三省鐵路成,又改定中俄接線續約,議照倫敦萬國公會所訂條例各減價。三十一年,日本戰勝,旅順、大連租借權移歸日本,俄專力於東清鐵道。於是有哈爾濱行政權之交涉。哈爾濱為東清鐵道中心地,初祗俄人住居。自三十一年開放為通商口岸,各國次第置領事,按中國各商埠辦法,中國有行政權。乃俄人謂哈爾濱行政權當歸諸東清鐵道會社,中國拒之。既而俄領事霍爾哇拖忽布東清鐵道市制,凡居住哈爾濱市內中外人民,悉課租稅。命東三省總督徐世昌與俄人交涉,不洽。宣統元年,俄領事赴北京與外務部議,外務部尚書梁敦彥與霍爾哇拖議設自治會於東清鐵道界內,以保中國主權,亦不違反東清鐵道會社諸條約,遂議結。而松花江航權之議又起。

初,中俄條約所指之松花江,系指黑龍江下流而言,未許在內地松花江通航也。俄謂咸豐八年、光緒七年所結條約,系指松花江全部而言。至是,命濱江關道施肇基與俄領事開議,俄人仍執舊約為詞。中國以日、俄訂立樸資茅斯約,已將中、俄在松花江獨得行船之權利讓出,舊約不適用。相與辯論不決。既而俄人又欲干預中國管理船舶之權,及防疫並給發專照等事,復嚴拒之。俄人仍執全江貿易自由,不認商埠、內地之區別,又以江路與陸路為一類,不與海路並論,久之始就範。明年締約:一,滿洲界內之松花江,許各國自由航行;二,船泊稅依所載貨物重量收納;三,兩國國境各百里之消費貨各免稅;四,穀物稅比從來減三分之一;五,內地輸出貨在松花江稅關照例納稅。此約成,於是各國得航行於松花江內,而北滿之局勢一變。時中國與俄訂東省鐵路公議會大綱,俄人以中國開放商埠,與東清鐵路地段性質不同,東清鐵路地段內有完全行政之權,意在於東清鐵路界內施行其行政權。政府以俄侵越主權,嚴拒之。並通告各國曰:「東清鐵路合同首段即載明中政府與華俄道勝銀行合夥開設生意,曰『合夥開設生意』,明系商務之性質,與行政上之權限絲毫不得侵越。乃俄引此合同第六條為據,謂有『由公司一手經理』字樣為完全行政之權,不知其一手經理,即合同所指鐵路工程實在必需之地段,而公司經理之權限,不得越出鐵路應辦之事,絕無可推移到行政地位。又宣統元年中、俄兩國所訂東省鐵路界內公議會大綱條款,自第一條以至第五條,均系聲明鐵路界內中國主權不得稍有損失。又光緒三十一年俄、日在美國議定條約,第三條載明俄、日兩國政府統行歸還中國全滿洲完全專主治理之權。又俄政府聲明俄國在滿洲並無地方上利益或優先及獨得讓與之件,致侵害中國主權,或違背機會均等主義。豈能強解商務合同,並以未經中國明認宣布之言為依據,而轉將兩國之約廢棄不論耶?」俄人屈於詞,乃定議。

宣統二年,屆中俄通商條約期滿,應改訂,因與駐京俄使交涉,俄使堅執舊約。正爭議間,俄使奉本國政府電旨,轉向中國提出要求案:一,兩國國境各百里內,俄制定之國境稅率,不受限制,兩國領土內之產物及工商品,皆無稅貿易;二,旅中國俄人訟案,全歸俄官審理,兩國人民訟案,歸兩國會審;三,蒙古及天山南北兩路,俄人得自由居住,為無稅貿易;四,俄國於伊犁、塔爾巴哈臺、庫倫、烏里雅蘇臺、喀什噶爾、烏魯木齊、科布多、哈密、古城、張家口等處,得設置領事官,並有購置土地建築房屋之權。久之,始復俄使云:一,國境百里內,中國確遵自由貿易之約,並不限制俄國之國境稅率;二,兩國人民訟案,應照舊約辦理;三,蒙古、新疆地方貿易,原定俟商務興盛,即設定稅率;四,科布多、哈密、古城三處,既認為貿易隆盛,中國依俄國設領事之要求,俄國亦應依原約,允中國制定關稅。俄使以告本國政府,俄以制定關稅不應與增設領事並提,更向中國質問,並命土耳其斯坦駐軍進伊犁邊境,遂允之。俄人又遣兵駐庫倫,向外務部邀求開礦優先權,拒之。會革命軍興,庫倫獨立,事益不可問矣。


\end{pinyinscope}