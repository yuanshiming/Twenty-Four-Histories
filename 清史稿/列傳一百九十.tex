\article{列傳一百九十}

\begin{pinyinscope}
勝保托明阿陳金綬德興阿

勝保,字克齋,蘇完瓜爾佳氏,滿洲鑲白旗人。道光二十年舉人,考授順天府教授。遷贊善,大考二等,擢侍講,累遷祭酒。屢上疏言事,甚著風採。歷光祿寺卿、內閣學士。

咸豐二年,因天變上疏論時政,言甚切直,略謂:「廣西賊勢猖獗,廣東、湖南皆可憂。賽尚阿督師無功,請明賞罰以振紀綱。河決不治河員之罪,刑輕盜風日熾,應明敕法以肅典常。臣工奏摺多留中,恐滋流弊。一切事務,硃批多而諭旨少。市井細民,時或私論聖德。」疏入,下樞臣傳問疏末兩端,令直言無隱。覆奏曰:「硃批因事垂訓,臣工奉到遵行,他人不與聞,非若諭旨頒示天下。近日諸臣條奏雖依議,而原奏之人不知;交部重案,覆奏依議,外人並不知作何發落。古者象魏懸書,俾眾屬目。似宜通行宣示,以昭朝廷之令甲,而杜胥吏之蔽欺。至愚賤私議,或謂皇上勵精之心不如初政,或謂勤儉之德不及先皇。今游觀之所,煥然一新。釋服之後,必將有適性陶情之事,現在內府已有採辦犁園服飾以備進御者。夫鼓樂田獵,何損聖德。然自古帝王必先天下之憂而憂,後天下之樂而樂。書曰:『無於水監,當於民監。』誠不可不察也。」文宗不懌,明諭指駁,以其意存諷諫,不之罪也。尋因自行撤回封奏,降四品京堂。

會粵匪犯武昌,勝保疏陳辦賊方略,命馳往河南,交欽差大臣琦善差遣。三年春,偕提督陳金綬率兵援湖北、安徽,而江寧告急。至則城已陷,駐兵江浦。勝保疏陳軍事稱旨,命以內閣學士會辦軍務,克浦口而賊陷揚州,偕陳金綬進剿。擊賊鎮海寺南,破之,薄揚州城下,賜花翎。又連破賊於天寧、廣儲門外。

奉命赴安徽剿賊,而賊已入河南,渡河圍懷慶。勝保會諸軍進擊,將軍托明阿軍其東,勝保軍其南。時督師大學士訥爾經額遙駐臨洺關,援軍數路久頓城下,惟二軍戰較力,命勝保幫辦河北軍務。七月,分三路進攻賊壘,大破之,懷慶圍解,加都統銜,賜黃馬褂,予霍鑾巴圖魯名號。賊竄山西,連陷數縣,諸軍遷延,惟勝保率善祿、西凌阿兵四千尾追,一破之封門山口,再破之平陽,繞出賊前,扼韓侯嶺,尋復洪洞、平陽。劾逗留諸將托云保、董占元、烏勒欣泰等,罪之;詔嘉勝保果勇有為,授欽差大臣,代訥爾經額督師,節制各路,特賜康熙朝安親王所進神雀刀,凡貽誤軍情者,副將以下立斬以聞。

賊既不得北竄,轉而南,由澤、潞間道入直隸境。訥爾經額師潰於臨洺關,賊復猖獗,竄順德、趙州、正定。勝保由井陘一路迎截,坐追賊不力,鐫二級。命惠親王綿愉為大將軍,科爾沁郡王僧格林沁為參贊大臣,駐軍涿州,直隸軍務仍責勝保專任,而以西凌阿、善祿副之。賊東竄,由深州、河間窺天津,勝保轉戰追賊至靜海。賊由獨流分踞楊柳青,迭擊之,遂聚於靜海、獨流,負嵎久踞。詔僧格林沁進軍合剿。四年春,賊突圍走阜城,追擊,殲賊數千,陣斃悍酋吉文元。而援賊由江北偷渡黃河擾山東,命勝保移兵往剿,臨清失守,坐褫職,戴罪自效。尋破賊,克臨清,餘賊南走,追擊迭破之,解散甚眾。及竄入豐縣,僅千餘人,蹙之河岸,悉數殲除。捷聞,復職,加太子少保。僧格林沁圍林鳳祥、李開芳於連鎮,久未下,命勝保回軍會剿。開芳突出,分股竄山東,勝保親率輕騎追之,賊陷高唐踞守,圍之數月不能克。迭詔詰責,褫職逮京治罪,遣戍新疆。直隸、山東賊既平,予藍翎侍衛,充伊犁領隊大臣。

六年,召還,發往安徽軍營差遣。七年,予副都統銜,幫辦河南軍務。捻匪方熾,勝保至,連破之方家集、烏龍集、柳溝集,克三河尖老巢。又克河關,復霍丘,大捷於正陽關,斬捻首魏藍奇等,加頭品頂戴。八年,平酆家集、喬家廟、趙屯諸捻巢。粵匪大股圍固始,擊破之,殲賊萬餘,斬偽顯天侯卜占魁等,固始圍解。詔嘉謀勇兼優,遇都統缺出題奏,復黃馬褂、巴圖魯,免其弟廉保遣戍罪。粵匪陳玉成、李侍賢合陷廬州、鳳陽,授勝保鑲黃旗蒙古都統,命為欽差大臣,督辦安徽軍務,連破賊於定遠池河、高橋。督軍抵三河,賊遁走。捻首李兆受久踞江、淮間,與粵匪勾結。及見粵匪屢挫,漸持兩端。勝保親至清流關密招之,許歸誠後免罪授官。兆受以其部下家屬在江寧,請緩發。至是進攻天長,兆受內應,克之,遂獻滁州,奏授參將職,改名世忠,安置降★,自為一軍。九年,克六安,捻首張元龍以鳳陽降,復臨淮關。進克霍山、盱眙,破賊清水鎮,斬其酋吳加孝,遂克懷遠,而廬州、定遠久未下,賊仍蔓延。丁母憂,奪情留軍。

十年,罷欽差大臣,命赴河南剿匪。御史林之望論劾,降授鑲藍旗漢軍副都統。復坐剿匪不力,降授光祿寺卿,召回京。甫至,會英法聯軍內犯,命率八旗禁軍駐定福莊,偕僧格林沁、瑞麟進戰通州八里橋,敗績,勝保受傷,退保京師。停戰議和,勝保收集各路潰軍及勤王師續至者共萬餘人。疏陳京兵亟應訓練,擬議章程以進。命兼管圓明園八旗、內務府包衣三旗,親督操練,是為改練京兵之始。

十一年,擢兵部侍郎,捻匪擾山東,詔分所部五千人畀僧格林沁往剿。尋命勝保赴直、東交界治防,連克丘縣、館陶、冠縣、莘縣,破賊老巢。招降捻首宋景詩,率眾隨軍。復朝城、觀城,命督辦河南、安徽剿匪事宜。河北肅清,予優敘。

是年七月,文宗崩於行在,穆宗嗣位,肅順、載垣、端華等輔政專擅。勝保昌言將入清君側,肅順等頗忌憚之。洎回鑾,上疏曰:「政柄操之自上,非臣下所得專。皇上沖齡嗣位,輔政得人,方足以資治理。怡親王載垣、鄭親王端華等非不宣力有年,赫赫師尹,民具爾瞻;今竟攬君國大權,以臣僕而代綸音,挾至尊而令天下,實無以副寄託之望,而饜四海之心。該王等以承寫硃諭為辭,居之不疑。不知皇上纘承大統,天與人歸,原不以硃諭之有無為重。至贊襄政務,當以親親尊賢為斷,不當專以承寫為憑。先皇帝彌留之際,近支親王多不在側。仰窺顧命苦衷,所以未留親筆硃諭者,未必非以輔政之難得其人,待皇上自擇而任之,以成未竟之志也。嗣聖既未親政,皇太后又不臨朝,是政柄盡付之該王等數人。其託諸掣簽簡放,鈐用符信圖章,以此取信於人,無如人皆不信,民碞可畏,天下難欺。近如御史董元醇條陳,極有關系,應準應駁,惟當斷自聖裁,廣集廷議,以定行止。乃徑行擬旨駁斥,已開矯竊之端,大失臣民之望。道路之人皆曰:『此非吾君之言也,非母後聖母之意也。』一切發號施令,真偽難分。眾情洶洶,咸懷不服。夫天下者,宣宗成皇帝之天下,傳之文宗顯皇帝以付之我皇上者也。昔我文皇后雖無垂簾之明文,而有聽政之實用。為今之計,非皇太后親理萬幾,召對群臣,無以通下情而正國體;非特簡近支親王佐理庶政,盡心匡弼,無以振紀綱而順人心。惟有籥懇皇上俯察芻蕘,即奉皇太后權宜聽政,而於近支親王擇賢而任,仍秉命而行,以成郅治。」奏上,會大學士周祖培等亦以為言,下廷議,從之。肅順等並伏法。尋授鑲黃旗滿洲都統兼正藍旗護軍統領。

時捻匪肆擾皖、豫間,以張洛行為最強。苗沛霖自踞壽州,逼走巡撫翁同書後,佯稱就撫,陰與粵匪陳玉成勾結。署巡撫賈臻被圍於潁州,久不解。楚軍已克安慶,陳玉成退踞廬州。朝廷本意安徽軍事屬之李續宜,用為巡撫。沛霖舊隸勝保部下,心憚楚軍,揚言勝保來始薙發。賈臻以聞,詔促勝保援潁州。同治元年,遣軍先進,為賊所挫。三月,勝保至,擊破賊壘,圍乃解,加兵部尚書銜。多隆阿等克廬州,陳玉成遁走,沛霖誘擒之,獻於勝保軍。詔於軍前誅玉成,赦沛霖罪,許立功後復官。沛霖擁眾號十萬,所屬二百餘圩。與張洛行勢敵相仇,自請剿之,心實叵測。詔詢曾國籓、官文、李續宜、袁甲三等,皆主剿。獨勝保一意主撫,上疏言事權不一,身為客軍,地方掣肘,請以安徽、河南兩巡撫幫辦軍務,允之。迭詔訓飭,褒其才略,戒其驕愎。卒不悛,力言沛霖無他,而為李續宜所疑,恐激變。續宜奉旨進駐潁州,亦迄不至。

會陜西回亂熾,多隆阿援軍阻隔不能遽達。遂授勝保欽差大臣,督辦陜西軍務。八月,轉戰至西安,解其圍。降捻宋景詩中途率眾叛走。東路同州、朝邑猶為回踞,詔責勝保專剿東路,命多隆阿進軍分任西路。勝保力不能制賊,而忌多隆阿,擅調苗沛霖率兵赴陜,嚴詔斥阻,不聽。命僧格林沁大軍監制,乃止。於是中外交章劾勝保驕縱貪淫,冒餉納賄,擁兵縱寇,欺罔貽誤,下僧格林沁及山西巡撫英桂、西安副都統德興阿察實奏上,密詔多隆阿率師至陜,傳旨宣布勝保罪狀,褫職逮京,交刑部治罪,籍其家。

二年,王大臣會鞫,勝保僅自承攜妾隨營,呈訴參劾諸人誣告之罪。詔斥其貪污欺罔,天下共知,苗沛霖已戕官踞城,宋景詩反覆背叛,皆其養筴貽患,不得謂無挾制朝廷之意;念其戰功足錄,從寬賜自盡,並逮其從官論罪有差。當其被逮也,降捻李世忠已擢至提督,請黜己官為之贖罪,不許。御史吳臺壽疏言勝保有克敵禦侮之功,無失地喪師之罪,請從末減。臺壽兄臺朗在勝保軍中,詔斥黨附,褫臺壽職。

托明阿,棟鄂氏,滿洲正紅旗人。由侍衛擢護軍參領,出為山東兗州營游擊。從巡撫武隆阿徵回疆,以功賜花翎。累擢曹州鎮總兵,調四川松潘、重慶二鎮。道光二十四年,擢四川提督,以病去職。二十七年,起授烏魯木齊提督。調陜西,擢綏遠城將軍,整飭戎政,勤於訓練。

咸豐三年,粵匪林鳳祥等陷揚州,逼淮、徐,命率所部赴江南、山東交界防堵,進屯清江浦。賊竄滁州,托明阿赴援,與周天爵會剿。遂追賊至河南,迭戰於睢州、杞縣、陳留、中牟,進克汜水,殲賊千餘,被珍賚,命襄辦軍務。賊竄河北,圍懷慶,乃渡河會諸軍分路進攻,迭有斬獲。賊築土城樹木柵以拒,合攻破之,擒斬數千。賊始遁,懷慶圍解。論功,賜黃馬褂,予西林巴圖魯名號。追賊山西,詔以勝保督師,命托明阿襄辦。賊竄入直隸境,坐降五級留任,尋以傷劇解職回旗。四年春,病痊,命赴直隸,仍幫辦僧格林沁、勝保軍務。賊方踞阜城,堅守不出,諸軍圍之。托明阿屯東北,賊來撲,輒擊退,突由東南隅竄出,踞連鎮,夾運河。托明阿與都統西凌阿軍東西兩岸,圍復合。

會琦善督師揚州,卒於軍,命托明阿為欽差大臣,馳往代之,授江寧將軍。自賊踞江寧,鎮江、揚州皆陷,南北梗阻,大軍分兩路,向榮軍江南,琦善軍江北。江北軍攻揚州不能克,賊棄城去,聚於瓜洲,與南岸鎮江相犄角。江寧賊時乘鉅簰順流而下,陸師不能扼,水師力薄,亦不能制賊。上游浦口最當沖要,賊於沙洲結營,時圖進竄。恃總兵武慶一軍及道員溫紹原六合練勇為屏蔽,亦不能進取。托明阿至軍,令副將鞠殿華毀運河鐵金巢,提督陳金綬循東岸進攻,小有斬獲。又截擊賊簰,斃偽丞相黃起茅。自督舟師渡江,略北固山、金山而還。五年,瓜洲、鎮江賊合犯儀徵,令副都統德興阿、總兵李志和擊退。又進軍三汊河,誘賊敗之。托明阿見僧格林沁於連鎮、馮官屯皆以圍墻制賊,議仿其法,於瓜洲築長圍以困之。然瓜洲濱大江,江路不斷,且地勢袤長不易守,實無足恃。圍成,屢偕陳金綬進攻,無大勝利。江寧賊踞江浦石磯橋,武慶、西昌阿等馳擊,克之。巡撫吉爾杭阿督師攻鎮江甚急,於是議南北同時進剿。

六年二月,江寧賊大舉援鎮江,未得逞。渡江與瓜洲賊合,突越土圍,四出縱火。官軍戰土橋竟日,傷亡多。托明阿營壘被毀,退三汊河,又退秦家橋,幾不能軍。陳金綬、雷以諴等亦退走,揚州遂陷。諸營潰散,惟德興阿猶整軍力戰。向榮遣鄧紹良渡江來援。越十日,復揚州,而江浦亦為賊踞。詔褫托明阿職,留營效力,尋以病歸。

八年,予頭等侍衛,率兵駐楊村防英兵內犯,授直隸提督,遷西安將軍。同治元年,以傷病乞休,四年,卒。

陳金綬,四川岳池人。從剿教匪,授把總,積功至都司。道光初,從征回疆,破賊於佳噶賴,功最,賜號逸勇巴圖魯,擢留壩營游擊。十三年,直隸總督琦善調司教練,累擢督標中軍副將,琦善倚之,以堪勝總兵薦,擢天津鎮。

二十二年,擢直隸提督。及琦善督師剿粵匪,率所部三千以從。詔金綬為楊遇春舊部,命幫辦軍務,率兵先發。又以其不諳文字,命勝保偕行。咸豐三年春,趣援江寧,偕勝保克浦口,詔責專防江北。揚州陷,由六合、儀徵趨援。琦善大軍始至,合攻揚州。琦善軍其北,金綬、勝保軍其西,累戰皆捷。賊堅守數月不下,而瓜洲一路通江,兵少不能合圍。賊分犯浦口踞之,進陷滁州,遂北竄。勝保率兵赴安徽應援,迭詔以孤城久抗,責攻益急。總兵雙來奮進,緣梯登城,金綬策應。兵不聽命,雙來以無援負創退,尋歿於軍,自此不敢力攻,而賊時由瓜洲窺伺來援,屢卻之。十一月,賊陷儀徵,兩路同時來犯。參將馮景尼守楊子橋,先潰,諸軍多失利。城賊擁輜重突出趨瓜洲,琦善、金綬不能截擊,並坐褫職留軍。揚州雖復,賊久踞瓜洲。四年春,琦善卒於軍,金綬暫署關防。托明阿至,偕金綬進攻瓜洲,毀賊砲臺。尋攻新橋賊壘,金綬之侄能義及游擊海明殞於陣。

江北軍多疲玩,金綬年老,文宗以其謹願,姑容之。閱時輒報小捷,屢以虛飾被斥。至托明阿兵潰土橋,金綬及雷以諴駐萬福橋,望風而走。事後飾辭自辨,又奏隨同克復揚州,為德興阿論劾,應治罪,金綬已先歿於軍矣。

德興阿,喬佳氏,滿洲正黃旗人,黑龍江駐防。道光末,由駐京前鋒授藍翎侍衛、乾清門行走,累擢頭等侍衛。以善騎射受文宗知,曾手擒奔馬,賜黃馬褂。

咸豐二年,命率黑龍江兵赴琦善軍。三年,從攻揚州,屯蔣家廟,為通儀徵要路,城賊竄出,奮擊敗之。瓜洲援賊進踞虹橋,與守備毛三元夾擊於三汊河。德興阿單騎陷陣,射殪其酋,大破賊,加副都統銜。別賊破儀徵,分兩路來犯。德興阿急趨東石人頭,毀賊浮橋。而瓜洲賊又進築土城於河西,偪三汊河,與儀徵賊相犄角。德興阿偕總兵瞿騰龍渡河毀賊營,賊乃不能西進。是年冬,賊棄揚州城退踞瓜洲,官軍進復儀徵,授正白旗漢軍副都統。四年,偕瞿騰龍進攻瓜洲,騰龍深入,為賊所襲,殞於陣。德興阿率勁騎馳援,賊敗走,軍賴以全,賜號博奇巴圖魯。尋復敗賊三汊河,賊埋地雷誘官軍,德興阿偵知,揮軍繞路而前,賊伏壘不出,遂分軍兩路夾攻,斬馘過當,奪獲大砲地雷。捷聞,晉御前侍衛。五年,迭攻瓜洲賊壘,又截擊竄賊於虹橋、八江口等處,皆獲勝。六年,托明阿兵敗於土橋,揚州復陷,諸軍渙散,獨德興阿軍未動。詔黜托明阿,以德興阿為欽差大臣,加都統銜。敗賊薛家樓,進規郡城。賊萬人迎敵,德興阿身先士卒,斬賊酋一,諸軍乘之,賊大潰,乘勝復揚州。同時江浦、浦口並為賊踞,令總兵武慶攻克之。

德興阿戰功素為江北諸軍冠,惟不曉漢文,命少詹事翁同書為幫辦。添調新兵,軍聲稍振,進規瓜洲。七年,參將富明阿破賊於土橋、四里鋪,水師又擊沉賊船,斬偽將軍陳磊。是年夏,合水陸諸軍進攻,毀賊艦及砲臺。德興阿親督戰,更番進逼,至十一月,大破之,復瓜洲。賊負嵎歷四年,至是始克。詔嘉調度有方,賜雙眼花翎,予騎都尉世職。乘勝逼金山,剿平新河口、龍王廟等處餘匪。江南軍亦同日克鎮江,專力進攻江寧。八年春,德興阿進軍江浦,獲勝。江寧賊勢日蹙,悍黨陳玉成等由安徽糾眾來援,德興阿兵敗於浦口,退保六合,褫雙眼花翎,革職留任。賊連陷江浦、天長、儀徵,德興阿不能救,揚州亦陷,褫世職。尋張國樑率兵渡江復揚州,而德興阿擁兵邵伯,觀望不前,嚴旨斥責。溫紹原守六合歷數年,為江北屏蔽,至是亦以援絕被陷,紹原死之。翌日而張國樑馳至,已無及。國樑以江寧軍事急,移軍渡江,詔責德興阿規復六合,軍已不振,迄無功。

何桂清疏劾:「德興阿秉性粗率,初賴翁同書相助,得克瓜洲。自同書調任安徽巡撫去後,左右無人,毫無謀略,貽誤軍事。」和春亦劾其舉動乖謬,難以圖功。文宗猶念其前勞,未遽加譴,九年,以圍攻六合久不下,革任召還。自此江北不置帥,軍務統歸和春節制。尋予六品頂戴,交僧格林沁差遣。

十一年,署密雲副都統。同治初,授西安右翼副都統,留辦山西防務,又移駐陜西同、朝一帶防剿。五年,充塔爾巴哈臺參贊大臣,授正紅旗漢軍副都統,幫辦新疆北路軍務。六年,丁母憂回旗。尋卒,依都統例賜恤,謚威恪。

論曰:勝保初以直諫稱。及出治軍,膽略機警,數著功績。然負氣凌人,雖僧格林沁不相下。自餘疆臣共事,無不齟互劾。文宗嚴馭之,屢躓屢起,蓋惜其才也。始終以客軍辦賊,無自練之兵,無治餉之權;撫用悍寇而紊紀律,濫收廢員而通賄賂,又縱淫侈不自檢束。卒因袒庇苗沛霖,與楚軍不相能,朝廷苦心調和而不之喻,遂致獲罪,功過固莫掩也。托明阿、德興阿皆戰將,非獨當一面之才,負乘僨事,宜哉。斯又不足與勝保並論矣。


\end{pinyinscope}