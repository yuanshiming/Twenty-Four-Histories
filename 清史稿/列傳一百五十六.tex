\article{列傳一百五十六}

\begin{pinyinscope}
林則徐鄧廷楨達洪阿

林則徐,字少穆,福建侯官人。少警敏,有異才。年二十,舉鄉試。巡撫張師誠闢佐幕。嘉慶十六年進士,選庶吉士,授編修。歷典江西、雲南鄉試,分校會試。遷御史,疏論福建閩安副將張寶以海盜投誠,宜示裁抑,以防驕蹇,被嘉納。未幾,出為杭嘉湖道,修海塘,興水利。道光元年,聞父病,引疾歸。二年,起授淮海道,未之任,署浙江鹽運使。遷江蘇按察使,治獄嚴明。四年,大水,署布政使,治賑。尋丁母憂,命赴南河修高家堰堤工,事竣回籍。六年,命署兩淮鹽政,以未終制辭,服闋,補陜西按察使。遷江寧布政使,父憂歸。十年,補湖北布政使,調河南,又調江寧。十一年,擢河東河道總督。疏陳稭料為河工第一弊藪,親赴各察驗;又言碎石實足為埽工之輔,應隨宜施用。十二年,調江蘇巡撫。吳中洊饑,奏免逋賦,籌撫恤。前在籓司任,議定賑務章程,行之有效,至是仍其法,宿弊一清。賑竣,乃籌積穀備荒。清釐交代,盡結京控諸獄。考覈屬吏,疏言:「察吏莫先於自察,必將各屬大小政務,逐一求盡於心,然後能以驗群吏之盡心與否。如大吏之心先未貫徹,何從察其情偽?臣惟持此不敢不盡之心,事事與僚屬求實際。」詔嘉之,勉以力行。

先是總督陶澍奏濬三江,則徐方為臬司,綜理其事,旋以憂去。至是黃浦、吳淞工已竣,則徐力任未竟者,劉河工最要,撥帑十六萬五千有奇,白茆次要,官紳集捐十一萬兩,同時開濬,以工代賑。兩河舊皆通海,易淤,且鑿河工鉅,改為清水長河,與黃埔、吳淞交匯通流。各於近海修閘建壩,潮汐泥沙不能壅入,內河漲,則由壩洩出歸海。復就原河逢灣取直,節省工費三萬餘兩,用濬附近劉河之七浦河,及附近白茆之徐六涇、東西護塘諸河。又濬丹徒、丹陽運河,寶帶橋泖澱諸工,以次興舉,為吳中數十年之利。兩署兩江總督。

十七年,擢湖廣總督。荊、襄歲罹水災,大修堤工,其患遂弭。整頓鹽課,以減價敵私無成效,專嚴緝私之禁,銷數大增。湖南鎮筸兵悍,數肇釁,巡閱撫馭,密薦總兵楊芳,擢為提督,移駐辰州,慎固苗疆屯防。

十八年,鴻臚寺卿黃爵滋請禁鴉片煙,下中外大臣議。則徐請用重典,言:「此禍不除,十年之後,不惟無可籌之餉,且無可用之兵。」宣宗深韙之,命入覲,召對十九次。授欽差大臣,赴廣東查辦,十九年春,至。總督鄧廷楨已嚴申禁令,捕拏煙犯,洋商查頓先避回國。則徐知水師提督關天培忠勇可用,令整兵嚴備。檄諭英國領事義律查繳煙土,驅逐躉船,呈出煙土二萬餘箱,親蒞虎門驗收,焚於海濱,四十餘日始盡。請定洋商夾帶鴉片罪名,依化外有犯之例,人即正法,貨物入官,責具甘結。他國皆聽命,獨義律枝梧未從。於是閱視沿海砲臺,以虎門為第一門戶,橫檔山、武山為第二門戶,大小虎山為第三門戶。海道至橫檔分為二支,右多暗沙,左經武山前,水深,洋船由之出入。關天培創議於此設木排鐵練二重,又增築虎門之河角砲臺,英國商船後至者不敢入。義律請令赴澳門載貨,冀囤煙私販,嚴斥拒之,潛泊尖沙嘴外洋。

會有英人毆斃華民,抗不交犯,遂斷其食物,撤買辦、工人以困之。七月,義律藉索食為名,以貨船載兵犯九龍山砲臺,參將賴恩爵擊走之。疏聞,帝喜悅,報曰:「既有此舉,不可再示柔弱。不患卿等孟浪,但戒卿等畏葸。」御史步際桐言出結徒虛文,則徐以彼國重然諾,不肯出結,愈不能不向索取,持之益堅。尋義律浼澳門洋酋轉圜,原令載煙之船回國,貨船聽官查驗。九月,商船已具結進口,義律遣兵船阻之,開砲來攻,關天培率游擊麥廷章奮擊敗之。十月,又犯虎門官湧,官軍分五路進攻,六戰皆捷。詔停止貿易,宣示罪狀,飭福建、浙江、江蘇嚴防海口。先已授則徐兩江總督,至是調補兩廣。府尹曾望顏請罷各國通商,禁漁船出洋。則徐疏言:「自斷英國貿易,他國喜,此盈彼絀,正可以夷制夷。如概與之絕,轉恐聯為一氣。粵民以海為生,概禁出洋,其勢不可終日。」時英船寄椗外洋,以利誘奸民接濟銷煙。二十年春,令關天培密裝砲械,雇漁船戶出洋設伏,候夜順風縱火,焚毀附夷匪船,接濟始斷。五月,再焚夷船於磨刀洋。諜知新來敵船揚帆北鄉,疏請沿海各省戒嚴。又言夷情詭譎,若逕赴天津求通貿易,請優示懷柔,依嘉慶年間成例,將遞詞人由內地送粵。

六月,英船至廈門,為閩浙總督鄧廷楨所拒。其犯浙者陷定海,掠寧波。則徐上疏自請治罪,密陳兵事不可中止,略曰:「英夷所憾在粵而滋擾於浙,雖變動出於意外,其窮蹙實在意中。惟其虛憍性成,愈窮蹙時,愈欲顯其桀驁,試其恫喝,甚且別生秘計,冀售其奸;一切不得行,仍必帖耳俯伏。第恐議者以為內地船砲非外夷之敵,與其曠日持久,不如設法羈縻。抑知夷情無厭,得步進步,威不能克,患無已時。他國紛紛效尤,不可不慮。」因請戴罪赴浙,隨營自效。七月,義律至天津,投書總督琦善,言廣東燒煙之釁,起自則徐及鄧廷楨二人,索價不與,又遭詬逐,故越境呈訴。琦善據以上聞,上意始動。

時英船在粵窺伺,復連敗之蓮花峰下及龍穴洲。捷書未上,九月,詔曰:「鴉片流毒內地,特遣林則徐會同鄧廷楨查辦,原期肅清內地,斷絕來源,隨地隨時,妥為辦理。乃自查辦以來,內而奸民犯法不能凈盡,外而興販來源並未斷絕,沿海各省紛紛徵調,糜餉勞師,皆林則徐等辦理不善之所致。」下則徐等嚴議,飭即來京,以琦善代之。尋議革職,命仍回廣東備查問差委。琦善至,義律要求賠償煙價,廈門、福州開埠通商,上怒,復命備戰。二十一年春,予則徐四品卿銜,赴浙江鎮海協防。時琦善雖以擅與香港逮治,和戰仍無定局。五月,詔斥則徐在粵不能德威並用,褫卿銜,遣戍伊犁。會河決開封,中途奉命襄辦塞決,二十二年,工竣,仍赴戍,而浙江、江南師屢敗。是年秋,和議遂成。

二十四年,新疆興治屯田,將軍布彥泰請以則徐綜其事。周歷南八城,濬水源,闢溝渠,墾田三萬七千餘頃,請給回民耕種,改屯兵為操防,如議行。二十五年,召還,以四五品京堂候補。尋署陜甘總督。二十六年,授陜西巡撫,留甘肅,偕布彥泰治叛番,擒其酋。

二十七年,授雲貴總督。雲南漢、回互斗焚殺,歷十數年。會保山回民控於京,漢民奪犯,毀官署,拆瀾滄江橋以拒,鎮道不能制。則徐主止分良莠,不分漢、回。二十八年,親督師往剿,途中聞彌渡客回滋亂,移兵破其巢,殲匪數百。保山民聞風股慄,縛犯迎師,誅其首要,散其脅從,召漢、回父老諭以恩信。遂搜捕永昌、順寧、雲州、姚州歷年戕官諸重犯,威德震洽,邊境乃安。加太子太保,賜花翎。二十九年,騰越邊外野夷滋擾,遣兵平之。以病乞歸。逾年,文宗嗣位,疊詔宣召,未至,以廣西逆首洪秀全稔亂,授欽差大臣,督師進剿,並署廣西巡撫。行次潮州,病卒。則徐威惠久著南服,賊聞其出,皆震悚,中道遽歿,天下惜之。遺疏上,優詔賜恤,贈太子太傅,謚文忠。雲南、江蘇並祀名宦,陜西請建專祠。

則徐才識過人,而待下虛衷,人樂為用,所蒞治績皆卓越。道光之季,東南困於漕運,宣宗密詢利弊,疏陳補救本原諸策,上畿輔水利議,文宗欲命籌辦而未果。海疆事起,時以英吉利最強為憂,則徐獨曰:「為中國患者,其俄羅斯乎!」後其言果驗。

鄧廷楨,字嶰筠,江蘇江寧人。嘉慶六年進士,選庶吉士,授編修。屢分校鄉、會試,稱得士。十五年,授臺灣遺缺知府,浙江巡撫蔣攸銛請留浙,補寧波。母憂歸,服闋,補陜西延安府,歷榆林、西安,以善折獄稱。平反韓城、南鄭冤獄,又全同州嫠婦母子,陜民歌頌,傳播京師。二十五年,超擢湖北按察使,權布政使。沿江民田歷年沉沒,而賦額仍在,為民累,悉請免之。道光元年,遷江西布政使。以前在西安失察渭南令故出縣民柳全璧殺人罪,罣誤,奪職。議戍軍臺,宣宗知其無私,特免遣戍,予七品銜,發直隸委用。尋授通永道。四年,擢陜西按察使,遷布政使。

六年,擢安徽巡撫。自嘉慶時,安徽多大獄,鳳、潁兩郡俗尤悍,常以兵定,責繳兵械,私藏尚多。廷楨乃立限,責成保長,逾限及私造者置之法。任吏皆得人,刁悍之風稍戢。舊例,潁州屬三人以上兇器傷人者,極邊煙瘴充軍,僉妻發配。廷楨疏言:「悍俗誠宜重懲,婦女顧名節,多自殘求免,或自盡傷生,情在可矜,請停其例。」遇水災,親乘舟勘賑。修復安豐塘、芍陂水門,濬鳳陽沫河,加築堤閘。嚴緝捕,屢獲劇盜。以獲南河掘堤首犯陳端,詔嘉獎。治皖十載,政尚安靜,境內大和。

十五年,擢兩廣總督。鴉片煙方盛行,漏銀出洋為大患。十六年,英吉利商人以躉船載煙,廷楨禁止不許進口,猶泊外洋,嚴旨驅逐。沿海奸民勾結,禁令猝難斷絕。廷楨與提督關天培整備海防,迭於大嶼山口、急水洋獲蟹艇,載銀鉅萬,盡數充賞,破獲囤煙私販。十八年,英船載屬番男婦五百餘人赴澳門居住,驅令回國。詔下禁煙議,疏言:「法行於豪貴,則小民易從;令嚴於中土,則外貨自絀。」十九年,林則徐奉命至廣東,廷楨與之同心協力,盡獲躉船積煙,焚之,嚴私販之罪;臨以兵威,屢戰皆捷,事詳則徐傳。奸民因失業,遍騰蜚語。廷楨疏陳,略曰:「臣緝懲鴉片,三載於茲。豪猾之徒,刑僇逋逃,身家既失,怨讟遂興。查檢為希旨,掩捕為貪功,偵伺為詭謀,推鞫為酷罰。誣以納賄,目以營私。譏建議為急於理財,訾新例為輕於改律,狂悖紛熒,無非為煙匪洩憤。」詔慰勉之。

調兩江、雲貴,皆未赴,閩防方急,遂調閩浙總督。購洋砲十四運閩,以閩洋無內港,砲臺建於海灘,沙浮不固,奏改為砲墩,囊沙堆築,外護以船。募水勇飾商船出洋巡緝。二十年三月,英船窺廈門,遣提督程恩高等迎敵於梅林澳,擊走之。奸民勾通出洋運煙,分責水陸師嚴緝,遇即攻擊,迭有殲擒。六月,敵船駛入廈門,求通貿易,阻之,遂開砲,來撲砲臺,參將陳勝元、守備陳光福奮擊,斃其前隊數人,發砲傷敵甚眾,乃遁。其分犯浙洋者,陷定海,廷楨率師赴剿,行次清風嶺,詔以閩防緊要,止其赴浙,遂駐兵泉州,招募練勇。疏言:「英船二十餘艘聚泊定海,內地師船恐難驟近,必改造堅大之船,多配砲火,間道而進,方能制勝。」

九月,詔以廷楨等在粵辦理不善,轉滋事端,與林則徐同奪職。二十一年,琦善撤沿海兵備,虎門失守,復追論廷楨久任兩廣,廢弛營務,與則徐同戍伊犁。二十三年,釋還。尋予三品頂戴,授甘肅布政使。議清查荒地,親往歷勘,由銀州東盡洮、隴,西極酒泉,得田一萬九千四百餘頃,又番貢地一千五百餘頃,寧夏馬廠地歸公一百餘頃,熟地升科,荒者招墾,詔嘉其勤,復二品頂戴。二十五年,擢陜西巡撫,署陜甘總督。番匪擾蒙部,遣兵邀擊於硫磺溝,平之。尋回任。二十六年,卒於官。

廷楨治行早為時稱,屢躓屢起,宣宗知之深,故卒用之。績學好士,幕府多名流,論學不輟。尤精於音韻之學,所著筆記、詩、詞並行世。子爾恆,亦官至陜西巡撫,自有傳。

當廷楨之去福建也,逾年,英兵復至,陷廈門,遂窺臺灣。總兵達洪阿偕臺灣道姚瑩屢卻之。及和議成,同獲譴。

達洪阿,字厚庵,富察氏,滿洲鑲黃旗人。由護軍洊擢總兵。道光十五年,調臺灣鎮。十八年,剿嘉義縣匪沈和等,賜花翎,加提督銜。二十一年八月,英兵船至雞籠海口,達洪阿與姚瑩督兵御之。副將邱鎮功燃巨砲折其桅,敵船沖礁破碎,擒斬甚眾,賜雙眼花翎。九月,敵船再至雞籠三沙灣,復卻之。剿平嘉義、鳳山土匪,予騎都尉世職。二十二年,敵船犯淡水、彰化間之大安港,欲入口。達洪阿謀於姚瑩,瑩曰:「此未可與海上爭鋒,必以計殲之。」乃募漁舟投敵任鄉導,誘令從土地公港入,擱淺中流,伏發,大破之,落水死者無算,其竄入漁舟者,擊斬殆盡。詔嘉臺灣三次破敵,達洪阿等智勇兼施,大揚國威,賜號阿克達春巴圖魯,加太子太保銜。敵船游奕外洋,乘間掩擊,迭有俘獲,遂不復至。

既而英師再陷定海,浙江、江蘇軍屢挫,乃議和。英將濮鼎查訴稱臺灣所戮皆遭風難民,達洪阿等冒功捏奏,命總督怡良赴臺灣查辦。至即傳旨革職逮問,兵民不服,勢洶洶,達洪阿等撫慰乃散。至京,下刑部獄,尋釋之,予三等侍衛,充哈密辦事大臣。歷伊犁參贊大臣,西寧辦事大臣。二十六年,偕陜甘總督布彥泰剿平黑錯寺番匪。三十年,授副都統。

咸豐元年,從大學士賽尚阿剿賊廣西,破紫金山西南砲臺。以病回京。三年,粵匪犯畿輔,率八旗兵赴臨洺關進剿。從欽差大臣勝保擊賊靜海,四戰皆捷,追至下西河,副都統佟鑒、天津知縣謝子澄陣亡。詔斥達洪阿先退,革職,留營效力。四年,敗賊獻縣,復原官。尋追賊阜城,受傷,卒於軍。贈都統銜,予騎都尉兼一雲騎尉世職,謚壯武。姚瑩自有傳。

論曰:林則徐才略冠時,禁煙一役,承宣宗嚴切之旨,操之過急;及敵氛蹈瑕他犯,遂遭讒屏斥。論者謂粵事始終倚之,加之操縱,潰裂當不致此。則徐瀕謫,疏陳:「自道光元年以來,粵關徵銀三千餘萬兩,收其利必防其害。使以關稅十分之一制砲造船,制夷已可裕如。」誠為讜論。惟當時內治廢弛,外情隔膜,言和言戰,皆昧機宜,其禍豈能幸免哉?鄧廷楨與則徐同心禦侮,克保巖疆。若達洪阿、姚瑩卻敵臺灣,固由守御有方,亦因敵非專注,朝廷皆不得已而罪之,諸人卒皆復起,而名節播宇內、煥史冊矣。


\end{pinyinscope}