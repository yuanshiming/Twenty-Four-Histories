\article{列傳二百二十七}

\begin{pinyinscope}
英桂宗室載齡恩承宗室福錕崇禮裕德

英桂,字香巖,赫舍哩氏,滿洲正藍旗人。道光元年舉人,以中書充軍機章京,晉侍讀。授山東青州知府,遷登萊青道。擢山西按察使,調山東,署布政使。咸豐三年,擢河南巡撫。粵匪擾湖北,英桂抵南陽籌防,匪踞安徽六安州,馳防汝寧。捻首張洛行竄踞雉河集,命英桂督三省軍務,疊敗賊於三河尖、潁上,捕獲教匪陳太安、王庭貞。遷山西巡撫。同治元年,欽差勝保被逮,多隆河代領其軍,多所裁撤,部將宋景詩復叛。英桂奏言:「勝保舊部雖多烏合降眾,久經戰陣。多隆阿到營旬日,遣歸七起,未免操之過急,窮無所歸,乘機走險。應遵前諭,如能隨同立功,仍準一體保奏,以安眾心。」報聞。遷福州將軍。

七年,署閩浙總督,奏言:「前督左宗棠議減兵者,為增餉也;議增餉者,為練兵也。應就地勢情形,以定經久之制。浙省依山阻海,馬步水陸額兵三萬七千五十九名,而駐於杭、嘉、湖、寧、溫、紹、臺海濱七府者三萬餘名,分駐湖、金、衢、嚴、處五府者七千餘名。海疆偏重,形勢了然。加餉為人情所原,減兵又為人情所難。各屬形勢不同,參以變通,庶臻妥善。今擬分別減兵增餉,以本省應裁之餉,加本省應存之兵。至練兵擬照楚、湘兵制,整器械,精技藝,庶兵氣可揚,水師戰船,寬籌經費,期復舊模。」又言:「輪船之設,利於巨洋。駕駛之法,迥異長江。」並擬定外海砲艇章程十二條,上均嘉納。召為內大臣。

十一年,授兵部尚書,兼總管內務府大臣。調吏部,兼步軍統領。光緒元年,協辦大學士。三年,授體仁閣大學士。四年,以病乞休。五年,卒,贈太子太保,謚文勤。

宗室載齡,字鶴峰,隸鑲藍旗,誠隱郡王允祉五世孫。道光二十一年進士,改庶吉士,授檢討。遷洗馬,累至內閣學士。以題定郡王載銓息肩圖稱門生違例,鐫三級。除光祿寺卿。咸豐三年,擢都察院副都御史,授工部左侍郎。粵匪北竄,踞河間、阜城,命載齡督防固安,匪南竄,撤防。會川督裕瑞被劾,命載齡往勘。因疏陳山西、陜西、四川捐輸款項侵蝕、濫銷諸弊,請敕各督撫嚴查參辦,並條上章程五則,議行。時黔匪偪近蜀境,詔載齡嚴飭地方勸諭鄉團助聲勢。尋署陜西巡撫。調刑部侍郎,仍留陜。五年,疏言:「前撫臣王慶雲請準遣戍新疆官犯捐輸,改發內地。捐數無多,何裨國計?此端一開,行險徼幸之徒,將肆意妄為,絕無忌憚。所得小而所失大,請停止以儆官邪。」上韙之。

尋詔回京,授泰寧鎮總兵,兼總管內務府大臣。以病乞休。病痊,署禮部侍郎,授刑部,調吏部。同治元年,擢都察院左都御史,遷兵部尚書。九年,丁父憂,襲輔國公。光緒三年,調吏部,協辦大學士。明年,授體仁閣大學士。六年,因病屢疏乞休,允之。九年,卒,贈太子太保,謚文恪。

恩承,字露圃,葉赫那拉氏,滿洲正白旗人。以筆帖式歷禮部郎中。隨僧格林沁剿賊,賞四品京堂。授侍讀學士,仍留營充翼長。解山東滕縣圍,克沙溝營、臨城驛,破賊曹州,又敗之臨朐縣南。晉三品京堂,授太常寺卿。同治二年,捻首張洛行伏誅,賞黃馬褂,擢內閣學士,授鑲紅旗蒙古副都統。以僧格林沁遇害,坐革職。旋以剿奉天馬賊,復原官。授理籓院侍郎。七年,捻匪張總愚北竄,恩承總統神機營馬步兵往雄、霸扼防。捻平,還京。歷調工部、禮部、刑部、吏部。

光緒元年,兼總管內務府大臣,擢都察院左都御史、正藍旗漢軍都統,遷禮部尚書。命與侍郎童華往四川查辦總督丁寶楨等被劾案,覆奏寶楨交部議。恩承言:「從古言利之臣,咸以不加賦而財用足,為動人聽聞之具。溯自軍興以來,川省釐、捐兩項,協撥餉需,以千百萬計。茍非國家深仁厚澤,何以人樂輸將?方今軍務肅清,民氣未復,乃川省設立官運局,所徵正款,已暗寓加釐;所收雜款,更巧為攤派。下與小民爭利,而司、道兩庫懸欠百萬有奇。正款反形支絀,似於國計民生兩無裨益。」疏入,敕部覈覆。復命赴雲南查辦事件,以侍郎閻敬銘劾恩承入川時失察家人需索,部議革職留任。

回京,授步軍統領。十年,遷刑部尚書,調吏部,協辦大學士。明年,授體仁閣大學士。十三年,命赴廣西、湖南、河南按事。十五年,轉東閣。十八年,卒,謚文恪。

宗室福錕,字箴庭,隸鑲藍旗,理密親王允礽六世孫。咸豐九年進士,授吏部主事,晉員外郎。光緒四年,授右庶子,遷侍讀學士,擢太僕寺卿。六年,賞副都統,充西寧辦事大臣。八年,召授兵部侍郎,歷調刑部、戶部。十年,擢工部尚書,兼步軍統領。命在總理各國事務衙門行走,兼管內務府大臣。調戶部,協辦大學士。以部駁機器鼓鑄,福錕議革職,改留任,旋復官。十五年,加太子太保,詹事府右庶子。崇文疏劾大學士張之萬交納外官,命福錕偕尚書潘祖廕勘之,奏言:「之萬住居湫隘,門無雜賓。樞臣接見外僚,藉以考覈人才。不得以因公謁見,謂為接納營私。惟僧靜洲以方外浮屠往來仕宦之家,易招物議,請驅逐回籍。」報可。十七年,授體仁閣大學士。二十年,皇太后萬壽,賞雙眼花翎。時京師盜風甚熾,福錕初禁步軍訊盜用嚴刑,盜益肆。至是奏請變通緝捕章程,允之。二十一年,疏請乞休。卒,謚文慎。

崇禮,字受之,姜氏,內務府漢軍正白旗人。咸豐七年,以拜唐阿為清漪園苑丞。文宗巡幸,嘗詢以事,奏對稱旨,嘉獎之。由員外郎歷內務府卿,加內務府大臣。光緒元年,授山海關副都統,乞病歸。五年,歷遷內閣學士,命在總理各國事務衙門行走,補禮部右侍郎。坐事,議革職,改降三級。九年,授光祿寺卿。歷理籓院侍郎,轉兵部、戶部。二十年,加太子少保,賞黃馬褂。旋擢理籓院尚書。出為熱河都統,再乞病。二十四年,授刑部尚書,兼步軍統領。

崇禮勤於職事,太后念先帝識拔,頗推恩遇。及政變起,太后復訓政,參預新政。楊銳等獲罪,崇禮以案情重大,請欽派大學士、軍機大臣會同審訊,始命軍機會刑部、都察院嚴審。已,又傳旨即行正法。二十六年,調戶部,協辦大學士。二十九年,授東閣大學士,轉文淵閣。三十一年,以病乞罷。又二年,卒,謚文恪。

裕德,字壽田,喜塔臘氏,滿洲正白旗人,湖北巡撫崇綸子。光緒二年進士,改庶吉士,授編修。累遷侍讀。八年,充咸安宮總裁,偕詹事府少詹事寶昌等疏請整頓咸安宮官學凡六事,下部議行。五轉至內閣學士,督山東學政。十六年,擢工部侍郎,調刑部。二十年,授都察院左都御史,命偕侍郎廖壽恆赴四川按事。二十四年,遷理籓院尚書,調兵部。二十八年,赴哲裏木盟查辦事件,因條上領荒招墾事宜,如所議行。二十九年,協辦大學士,授體仁閣大學士。三十年,充會試總裁。明年,改東閣。卒,謚文慎。

裕德持躬謙謹,禮賢下士,有一得之長,譽之不容口,時皆稱之。

論曰:大學士滿、漢並重,非有資望,不輕予大拜。內閣不兼軍機者,不參機務,相業無聞焉。英桂諸人或起軍功,或承世廕,或嫺文學,或優政事,雖未能顯有名績,而舊德老成,雍容臺鼎,亦不愧宰相之器者歟!


\end{pinyinscope}