\article{列傳二百九十七}

\begin{pinyinscope}
列女三

韋守官妻梁歸昭妻陸昭弟繼登妻張羅仁美妻李仁美弟妻劉

妾梅李等錢應式女王氏三女沈華區妻潘陳某妻伍

孫諤妻顧等洪志達妻葉羅章袞妻杜章袞侄群聘妻田等

王磐千妻顏何大封妻阮方希文妻項廖愈達妻李

妾汪張葉芊妻謝姚文璚妻劉毛翼順妻陳王三接妻黃

劉琰妻邢王躋聖妻韓等程顯妻硃劉元鏜妻吳妾硃等

應氏婦平陽婦殷壯猷妻李楊昌文妻袁諶日升妻陳

陳某妻萬林應雒妻莫梁學謙女吳師讓妻某黃某妻李

文秉世妻梁文氏女文樞妻陸何氏女王氏三女陳心俊妻馬

郭俊清女張問行妻楊張聯標妾傅林乾妻程楊應鶚妾佟

黃居中妻吳胡守謙妻黃沈棠妻俞陳得棟妻蔣等

汪二蛟母徐妻戴劉章壽妻徐黃嘉文妻蔡徐明英妻吳

長清嶺烈婦韓昌有妻李馬雄鎮妻李妾顧等沈瑞妻鄭

傅璇妻黃劉昆妻張妾吳及二女楊天階妻關及二女烏蒙女

劉亨基女滕士學妻滿向宗榜妻滕滕作賢妻楊滕家萬妻黃

高村婦陳世章妻硃薛中傑女傅瑛妻周任寨村二十烈女

王自正妻馬強逢泰妻徐方振聲妻張陳玉威妻唐寶豐二婦

戴鈞衡妻李妾劉陳吉麟妻周凌傳經妻楊秦耀曾妻畢

曹士鶴妻管謝石全妻廖曾石泰妻黃葉金題母胡繆勝云妻黃

石時稔聘妻劉章瑤圃女戴可恆妻硃金福曾妻李

張福海妻姚邵順年妻伊順年弟順國妻劉陳某聘妻酆

胡金題妻俞王氏女鄭德高妻阮方其蓮妻阮周小梅妻湯

楊某妻沈周世棣妻胡蔡以瑩妻曹妾馬王永喜妻盧

劉崇鼎母張武昌女子滄州女子費某妻吳冷煜瀛妻盧

陳兆吉妻餘蔡法度妻簡張守一女王占元妻楊王秉堃女

魏克明女劉慶耀妻廖歐陽維元妻曹李盤龍妻鄧等黃氏女

程氏女韓肖硃妻郗張醴仁妻王許氏女李氏女楊某妻吳

康創業妻邸李鴻業妻邸王書云妻谷王有周妻楊子漢連妻張

漢元妻李漢科妻李等張金鑄妻段王氏二女馬安娃妻趙

王之綱妻李穆氏女張某妻蔡程丁兒妻黃張氏女趙貴賜妻任

楊貴升妻劉多寶聘妻宗室氏子英爚妻鄂卓爾氏公額布妻

音德布女良奎妻連惠妻根瑞妻松文母吳姚葉敏妻耿

陳某妻殷黃晞妻周鄒延玠妻吳陳生輝妻侯

田一朋妻劉蔣世珍妻劉王有章妻羅有章妹

樓文貴妻盧沙木哈妻鄭榮組妻徐張翼妻戴

詹允迪妻吳蔡以位妻孫楊春芳妻王王尊德妾唐

竇鴻妾郝章學閔妻董杜聶齊妻何張氏婦寧化二婦

韋守官妻梁,長清人。明季饑,女未行,從父流轉河南,婢於富室。及笄,主為擇婿,梁泣言幼嘗受韋氏聘,死,不敢別嫁。主使求得守官,守官迎以歸。已而守官卒,家人欲使別嫁,梁自沉大清河,救,不死。乃自治棺,曰:「有欲娶我,以此畀之!」家人不復言。寇亂,匿棺以免。順治二年,師南行,過其村,梁懼,積薪於戶下,舉火,乃入棺,自焚死。

歸昭妻陸,弟繼登妻張,昭,昆山人;陸、張皆太倉人。昭仕明為監紀,順治二年,死揚州;繼登為教諭,長興民亂,戕焉。二婦未得問。昆山兵起,舅姑避於鄉,舟迎二婦,二婦不果行。師至,城閉,城西砲如雷。二婦夜登樓,環坐諸兒女酌酒,戒積薪樓下,城破則縱火。一老僕進,謂城破當兵沖,慮不及死,城北比丘尼故與主母善,菴後有池,倉卒可得死,從之。城破,兵掠菴,張入池,陸視其女,一卒前犯,陸力拒,被二矛,僕,又亂箠之,乃絕。張以水淺,不即死。兵去,潛視陸,陸亦蘇,乃與尼共掖起之。兵復至,張輒避諸池,一卒索得張,欲執以去,張力拒,見殺。陸創重卒。

羅仁美妻李,仁美,揚州人,失其縣;李,龍游人。家揚州廣儲門。師下揚州,李方娠,積薪所居樓下,呼諸婦曰:「原死者共死,毋辱!」於是姒劉、仁美妾梅、李,前室女宦姑及諸婦,從李登樓,凡十二人。呼婢菊花舉火,前室子哭,從李俱上,李顧見,啟牖呼仁美,擲兒下。仁美負母手挈兒,哭出巷,回首,見黑煙出瓦隙,火合樓摧,聞屟聲沸火中。仁美行,遇兵,僅得脫。兵去,發樓燼,拾殘骼,惟菊花遺肢衣可辨。乃叢葬十三人西華門外。

同時錢應式女淑賢,丹徒人。聞城破,數自殺,未絕。雨甚,門外萬馬聲,比屋殺人,火四起。淑賢以紙漬水塞口鼻,持父手壅其氣,父手悸不能舉,又解衣帶,強母使縊。母哭走,出,聞足擊床閣閣,入視,已絕。

王氏三女,金壇人。其二為同產,其一為群從姊妹,年皆十六七,以王師下江南,諸州縣盜群起,王氏避長蕩湖。晝延緣葦間,夜復其居。一日,盜至,劫三女子,縛置筏上。三女子號泣跌蕩,筏覆,三女子死焉,賊十數輩亦溺。明日,尸浮水上,縛盡弛,三女子攜手,發相縻。亂中無棺,得故篋三重以斂,墓於湖濱,墓木枝蘗皆三,相樛。

沈華區妻潘,海寧人,居硤石。順治二年六月,舉人周宗彞起兵硤石。八月望,師宵乘北關破之,華區與潘皆被俘。過南市橋,潘睨水欲自沉,華區密止之,曰:「汝死,兵且殺我!」潘乃語兵:「我從汝去,原得釋我夫。」兵釋華區,驅潘入舟,舟行十八里,至王店。水次,觀者方集,潘忽躍起,曰:「我硤石沈華區妻,義不任受辱!」奮入水。兵驚,捽其發出水,潘力自沉,發斷,系以糸墨,益力自沉,纆絕,如是三,兵以刃舂其喉,遂死。師中有裨將嘆其烈,出千錢為斂。

陳某妻伍,華亭人。師下松江,陳家璜溪,兵至,斧陳首,伍奔救,兵舍其夫而縶之。伍曰:「毋縛我,我從汝去!」將登舟,躍入溪,死。

當時死於溪者,諸生孫諤妻顧、徽州商孫氏之媼。

洪志達妻葉,歙人。順治二年,徽州初定,盜賊所在多有,志達偕葉避兵淳安鄭家村。明年二月,村人譁言兵至,志達與葉倉皇走,匿草中,游騎過,自草中曳葉出。志達習拳勇有力,踴自草中,奮擊一騎,僕,眾騎拔刀赴志達。志達徒手與鬥,眾騎且僕且起,環射之,矢中志達目,貫腦死。葉抱尸慟,眾騎挾之行,葉輟哭。馬行漸緩,度懸崖,葉曰:「勿持我急,我自能乘。」賊信之,遂縱馬向崖,眾騎自後從之,葉自馬上擲崖下,死。淳安人言其死且為神,為之祠焉。

羅章袞妻杜,群聘妻田,淳化人。群聘,章袞從子也,皆早卒。順治三年,寇至,城破,杜指墻間井,語養女淑明、淑儀曰:「此吾曹死所也!」遂入井。淑明、淑儀相向哭,從之下。田與杜連墻居,聞哭,呼其女優姐,亦趨井死。

先一年,縣兵譟變,章袞侄女竇芳墮樓死,竇芳有從姊雁珠,明崇禎間死寇,竇芳方在娠,其母夢雁珠偕一女至,謂唐奉天竇烈女也,故命曰竇芳。既長,嫁三原房大猷。其死後雁珠十七年,俱以正月十五日死,死時年俱十八,鄉人合前後稱「七烈」。

王磐千妻顏,江西安福人。順治三年,遇寇,■M0其臂索賄,顏詫曰:「此手乃為賊執耶?」投水死。

何大封妻阮,無為人,早寡。有授物誤觸其手者,引刀斷指,血濺尺許。

方希文妻項,名淑美,淳安人。順治三年,明潰師掠縣,希文攜家避兵西坑。以妾子病,謁醫。兵驟至,縱火。火將及,婢請項出避,項曰:「出,死於兵;不出,死於火。死同,死火不辱。若能死,則從;不能,亟出!」希文故有藏書,項積書左右,坐其中,火及,書燼,項殉焉。

廖愈達妻李,妾汪、張,泰寧人。李讀書通大義,教二妾章句。愈達從外歸,聞李疏「仁」字,教二妾,語諄諄。愈達入而笑,李正色曰:「志士仁人,有殺身以成仁,毋求生以害仁!」順治三年,愈達將妻妾避兵,或傳崇禎十七年京師破時,檢討汪偉與其妻耿殉國事,李以告二妾,相持而哭。師漸逼,愈達與妻妾夜走南石砦,師至,攻砦,愈達率妻妾避砦口。或呼師自砦後入,李即從砦口展手投崖下。愈達挈二妾匿巖石中,搜山兵至,張亦投崖死。愈達出金遣兵去,汪堅持愈達衣,伏其後,頃之,遙見師中出裨將,硃纓窄袖,指揮從卒巡山。汪大哭曰:「君善自保!」亦投崖,激於石,身裂若支解。師退,愈達及諸同避砦中者皆得脫。

葉芊妻謝,寧都人。六年冬十月,明將揭重熙等以師赴南昌,駐寧都兵掠得謝,部曲將悅其色,問家世,謝從容具以對,因乞得沐浴,部曲將許之,遂入室,以鬢刀自揕其喉,死。

姚文璚妻劉,名滿,福清人。文璚鬻香於市,順治三年,海寇至,索錢無所得,截文璚首去。滿舁尸還,舐血縛布綴於頸,斂畢,乃言曰:「我恨不能手刃賊,獨以死報君。」首觸棺,僕,久之,甦,請以兄公子為後,盡鬻衣珥營葬。越三年,清明上塚,歸,屑金咽之,死。

毛翼順妻陳,亦福清人。順治四年,翼順死於寇,舁尸還,血溢於鼻,陳舐血,斂畢,不食七日,自經。

王三接妻黃,曹縣人。三接官汾西知縣,黃侍姑田家居。順治五年,李化鯨亂,破城,姑、婦皆被執。黃語賊曰:「釋我姑,我與金帛,惟爾欲!」賊釋其姑,黃度姑行遠,乃罵曰:「吾家清白吏,安有厚藏?吾名家女,命婦,豈肯從賊?有死而已!」賊磔之。

當時為賊殺者,劉琰妻邢等九人;投水死者,王躋聖妻韓等七人。

程顯妻硃,新建人,明宗室女也。以其侄為子婦。順治五年,金聲桓為亂,顯自南昌將家人入山,道遇兵相失。或傳顯已死,硃謂子婦:「翁死,吾不獨生,汝奈何?」婦曰:「死耳!」硃縊樹上,已絕,兵救之,甦,復觸樹死。婦亦起觸樹,兵前持婦,婦齧其指,奪刀自剄死。

劉元鏜妻吳,妾硃,南昌人。元鏜亦將家人避兵,兵及,棄抱中兒道旁而走,吳伏溝草。硃為兵得,縶以行,經溪,躍,縶絕,兵斫其頰,死。吳出草,行數十武,遇鄰媼,脫簪求扶持。兵復至,吳握發仰天號曰:「夫邪子邪!吾其死邪!」兵挾刃逐之,行赴陂死。

是役諸女婦死者至眾,靖安舒調熙妻硃,救夫;豐城熊嗣蕃妻胡,及從子有恆妻沈,從夫救舅:皆死。而新建徐文璠妻硃,割乳斷首;進賢胡永益妻胡,刃出背:死尤烈。

應氏婦,鄞人。貧行乞。順治六年,海寇至,匿郭東廟。寇欲汙之,堅不從。既仍偽諾,出廟,將入井,寇復牽以入,終號泣不就,死亂刃。

平陽婦,不知其姓氏。順治七年,姜瓖亂,為其徒所掠,過定州唐城村,刺血題詩於壁,並為序自述,略言:「明月在天,清水在旁。得自盡於此,上不媿父母,次不慚婿,庶幾與水同清,與月同明。」遂自經死。

殷壯猷妻李,豐潤人。順治中,壯猷為臨藍參將。十一年,孫可望攻臨藍,壯猷築城以守,圍久不解,出戰,死。李以印畀次子質,揮使出避,而與長子文自剄死。

楊昌文妻袁,安義人,或曰建昌人。順治間兵亂,父母迎袁歸,袁不可,曰:「棄姑避兵,不義。」兵至,伏地請死,斫數刃去。家人歸,努目問:「姑無恙乎?」曰「無恙」,乃瞑。

諶日升妻陳,高安人。順治間,金聲桓亂,為兵掠挾上馬,力拒,中八刃,剖心斷脰刳孕死。

陳某妻萬,萬縣人。康熙間,譚弘亂,被執,殺其懷中子。萬詭言家有藏金強,賊使其徒從以往,過懸崖,奮起,擠賊墮,亦自投死。

林應雒妻莫,梁學謙女,吳師讓妻某,黃某妻李,皆新會人。應雒、學謙、師讓皆諸生。順治十一年,明將李定國攻新會,城守閱八月,食盡,殺人馬為食。莫代姑,梁女年十一代父,黃、李代夫,皆死。李之死,兵持首還其夫,使葬焉。

文秉世妻梁,鬱林人。李定國掠州,梁為兵掠,迫上馬。梁哭,據地罵,兵殺之。越二日,秉世得其尸,目未瞑也。

文氏女兆祥,文樞妻陸,灌陽人。定國兵至,姑嫂避火星山箐中,兵入,自殺。

何氏女,昭平人。是歲師逐定國,避兵思庇沖。或迫之,死。

王氏三女:長亥娘,次竹姑,次酉娘,博白農家女。康熙十九年,避寇宴石巖,寇攻巖,姊妹皆投崖死。

陳心俊妻馬,伏羌人。年十九,寡。順治初,流寇據城,其渠聞馬有色,遣人強致之。馬居樓上,揮雜器物擲樓下,厲聲叱其人曰:「白若渠,欲強污我,惟有頭可斷耳!」渠聞,亦愕曰:「烈婦!烈婦!」卒得免。

郭俊清女蓮姑,巴州人。嘉慶二年九月,教匪破城,掠以去,女罵不絕。賊褫其衣,罵愈厲,殺之,書其背曰「烈女尸」。

張問行妻楊,秦州人。同治間回亂,破其堡。楊遣三子行,持廚刀倚扉罵賊,賊剺其口至耳際,罵猶不已,遂死。賊舉扉掩其尸,書其上曰:「此張監生妻楊烈婦,毋損其尸。」

張聯標妾傅,泰順人。聯標為羅陽知縣,傅從,年方笄。山寇破縣,被執。賊渠令其徒百方誘之,不從。一夕,擁至渠所,諸賊執刀夾左右,怵以死,終不屈,乃縊殺之。

林乾妻程,漳浦人。有殊色。康熙元年,縣有劉暢者,為盜馬婆山。掠程至,將★之,不從。使他婦惎之曰:「我曹已至此,即完節,誰復能信?」程曰:「吾自行吾志,非求人信,豈能效汝曹無恥耶!」暢殺之。

楊應鶚妾佟,奉天人。應鶚官貴陽同知,吳三桂叛,檄署官,應鶚力拒,乃置諸順寧。師將入滇,郭壯圖使殺之。應鶚罵使者。佟曰:「大丈夫當毅然引決,無戀戀如兒女子!我請為公先,不使公遺憾。」遂縊,應鶚亦縊。

黃居中妻吳,居中失其裏貫;吳豐順人,廣東饒平鎮總兵六奇女也。康熙中,居中為蒼梧教諭。十三年,孫延齡叛,梧州戍兵應之,入其室,吳曰:「封疆之事,固知非若曹所能,若曹其俘我乎?我將待之!」奮擊,殺二人,自伏劍死。

胡守謙妻黃,閩人。守謙武舉。當耿精忠叛,守謙投書城外,言賊必敗,狀為守者所收,送郊外殺之。黃請代,不許。乃求得守謙首,綴於尸。葬畢,自具棺衾,飲藥死。

沈棠妻俞,莆田人。年十八,美。耿精忠兵至,執俞,並及棠。俞計脫棠,乃抗賊。賊威以刃,就刃;迫以火,赴火;幽之,遂自縊,賊磔其尸。

同時福清陳得棟妻蔣,陳云元妻周,皆為賊磔。莆田林振先妻鄭支解,永安黃尾四妻鄭刳孕,貴溪傅護妻薛剖腹,臟腑盡出。

汪二蛟母徐,妻戴,開化人。康熙十三年,耿精忠兵入浙江境,開化陷,二蛟及母、妻行避賊。賊至,縛二蛟,驅其母、妻以行。行過大澤,戴厲聲曰:「得死所矣!」徐應曰:「待我!」賊持戴袖,戴絕袖,抱子自投澤中,徐與俱下。二蛟大呼,縛盡絕,亦赴水死。

後二年,開化復陷,劉章壽妻徐,為賊渠所得,置樓上,令兩卒為守。婦陽謂守者:「事已至此,幸語若主,欲婚我,當具禮。」卒告渠,渠盛服佩刀上,婦迎坐,解刀置案上。復陽言:「奈何不為我具衣飾?」渠諾而下,婦取刀弄之,拔出鞘,忽引自刺。守者前奪刀,婦揮刀斷其臂,遂自剄,渠裂其尸。

黃嘉文妻蔡,名慧奴,黃巖人。康熙十三年,耿精忠之徒陷黃巖,明年,師復黃巖,以黃巖民嘗麗賊,俘焉。蔡及其子女屬杭州駐防將,將艷蔡,欲以為子婦。九月壬申,將召蔡喻指,蔡取壁間刀自剄死,將投其尸於江。時軍中得俘輒責金贖,嘉文方求金杭州,至,則蔡已死,乃贖子女還。蔡父行求蔡尸,十二月丙子,風作,江潮湧,蔡尸乃出,距蔡死九十有九日。嘉文還,言子女得贖正同日。

徐明英妻吳,名宗愛,字絳雪,永康人。宗愛幼慧,九歲通音律,十餘歲即能詩,善寫生,間作設色山水。明英卒。康熙十三年,耿精忠將徐尚朝攻處州,略金華。六月,游兵至永康。尚朝嘗官浙東,聞宗愛才色,乃使脅宗愛族人,求宗愛,勢洶洶。宗愛乃曰:「未亡人終一死耳,行矣,復何言!」賊遣迎宗愛,以兩騎翼宗愛行。至三十里坑,宗愛紿騎取飲,投崖死。宗愛二女兄皆能詩,而宗愛尤工,所著詩二卷。

長清嶺烈婦,不知其氏,諸暨人。康熙十三年,盜硃德甫占縣紫閬山為亂,吏發兵討之,婦見掠,與其並縶。婦好謂兵:「吾既被獲,復何言?吾夫祗此子,請俟其追至,以子歸之,吾從汝去耳。」行至長清嶺,其夫奔而至,婦復請以子授其夫。度父子行已遠,自擲崖下死。

韓昌有妻李,秦州人。康熙十四年六月,遇寇,李負幼子,行遲,為賊及。李批賊頰罵,賊刃之七創,項未殊。昌有舁之歸,夜而蘇,謂昌有曰:「必葬我松下!」又七日乃絕,昌有葬之松下。

馬雄鎮妻李,雄鎮自有傳,李不知其裏貫。雄鎮為廣西巡撫,孫延齡反,遣子世濟如京師告變,旋見執,幽四歲。康熙十六年,吳世琮攻殺延齡,遂戕雄鎮及其二子。李及妾顧、劉,女子子二,世濟妻董,妾苗,同日死。雄鎮初見執,置其孥別室,妾趙及世濟子一、女三皆以饑寒死。於是雄鎮二女相要同死,妾顧亦原從。及雄鎮見執,守者梯垣以告,二女謂顧:「今日當踐約。」為繯於梁,語顧曰:「夫人諸母行,宜位於中,雖顛沛,不可失序。」顧曰:「我妾也,又無出,何敢與諸母齒?」再讓,乃先縊,幼女年十五,弱,手不勝綆,久之,環不就,呼曰:「姊助我!」長女年十八,應曰:「妹怖死耶?吾助妹!」已,皆縊。董先二女縊,綆再絕,再僕地,傷額及足,三縊乃絕。苗與劉後二女縊,李視諸人皆死,曰:「姑婦子女,皆幸不辱身,我無憾矣!」乃亦縊。

顧,名荃,字芬若,豐潤人,能詩畫。

沈瑞妻鄭,瑞附見其從祖志祥傳。鄭父斌事鄭錦,私署禮官,蓋亦錦族。瑞嗣封續順公,駐潮州。錦兵破潮州,送瑞臺灣,時瑞年十五,斌蓋以此時婿瑞。居數年,錦部有傅為霖者謀為反間,事洩,辭連瑞,錦系瑞及其孥,而以鄭歸斌。鄭泣謂斌曰:「兒既歸沈氏,生死與共!請遣兒同系。」斌使處於別室。及瑞將死,問:「夫人安在?」或以告,解帶使訣鄭,鄭遂自經。

傅璇妻黃,名棄娘,臺灣人。璇,為霖子也。為霖事敗,錦俘其孥,棄娘有兄銓為營救得免。為霖、璇皆被殺,棄娘矢殉,銓寬譬之。棄娘曰:「今日之事,子為父死,妻為夫死,復何言!」卒自經。

劉昆妻張,保寧人。昆死烏蒙之難,語在忠義傳。昆既死,賊遂破城,張冠帔坐中堂,呼女易璋、可璋及妾吳,戒毋辱,出昆佩刀示易璋,易璋泣而跪,張斫其肩死。可璋亦跪,張慄,刀墮,可璋曰:「母怖耶?」拾刀自鏨,亦死。張語吳:「汝將三歲兒,好自匿,存張氏後。」吳號,抱張膝,張且嘆且回刀自殊,頸且斷,危坐幾上。吳揮乳母抱兒速去,拜張前,引刀沖喉,死幾下。雍正八年八月事也。乳母逃山中,卒全張氏後。師定烏蒙,錄昆死事,張、吳易璋、可璋旌贈如例。

楊天階妻關,開化人。天階為烏蒙守備,城破時戰死。亦有女子子二,長曰鳳,次無名,關聞天階死,謂二女曰:「我當死,汝姊妹宜求自脫。」二女泣曰:「父已死,兄不知存亡,何以為生?」遂對縊。關自剄死。

烏蒙女,不知姓氏,里居烏蒙。惈亂,掠子女財物,女子年少者,頭人自取之。女與其曹二十餘輩立棚下,日暮,頭人持刀入,叱諸女去衣,不從。擊以刀脊,次及女,女年十五六,有容色,堅不從。頭人欲擊輒復止,小惈告有以酒食賀者,頭人擲刀出。惈營中為坑,爇薪炭禦寒,女挾頭人所棄刀立坑後。頭人醉,復入就女,張兩手將抱持,女迎刺洞其胸,僕地死。眾惈驚,就視,女已自剄,群碎其尸。

劉亨基女,字滿,湘潭人。亨基官臺灣府同知,權知彰化縣。林爽文之難,亨基殉焉。滿年十六,自沉後池,池淺不得死,展轉泥中。賊大至,曳之上,滿罵曰:「我名家女,豈懼死乎?汝曹生太平,乃為逆亂,官軍至,汝曹當萬段!」賊劙其口,劓其鼻,罵愈厲,乃殺之。臺灣平,得旌,臺灣之民私謚曰貞烈。

滕士學妻滿,向宗榜妻滕,滕作賢妻楊,滕家萬妻黃,皆麻陽高村人。乾隆六十年,苗亂,掠高村,入士學家,擊滿以梃。滿怒罵,苗抉其目。罵愈厲,遂斷舌剖腹,寸磔死。滕繃其兒走水次,求舟將渡。苗逐之,執其手,滕怒罵,苗殺其子,滕躍入水死。作賢、家萬皆為苗殺,楊自剄殉。黃為苗掠至八斗山,紿苗入深林,解刀揕其胸,殺之。走求家萬尸巖下,亦自經殉。高村又有婦,以舅方病,不忍去。苗至,將殺其舅,婦奪刀刺苗,殪,遂自剄。

陳世章妻硃,義寧人。世章為湖北保康知縣。嘉慶元年,曾世興為亂,保康故無城,賊驟至,硃懷印坐。賊挾刃索印,硃曰:「我命婦,印在此!汝曹何敢奪?」賊以矛貫其胸死。

薛中傑女,洋縣人。嘉慶二年,教匪掠縣境,女年十六七,從家人行避賊。為賊得,置馬上,女罵,躍,僕地,賊掖之起行。經益水濱,自擲入水。方冬,水落,不即死。賊岸上立,好語招使上,女益匍匐求深處,賊攢矛刺之,死。

傅瑛妻周,寶慶人。道光間,教匪起,周方在母家,從母匿叢慄中。賊擁入,鄰婦先匿者群叩頭乞哀,周語母曰:「死生命也!奈何降志於此曹乎?」乃舉袂蒙其首伏母懷。賊迫視之,美,挾上馬,二賊挾以行。周罵賊,賊撫其背為好語,周以指剺面罵益急。賊刺其肋,推墜馬,死亂刃下。

任寨村二十烈女,任寨村寶豐縣村也。嘉慶五年,教匪至,距村不十里,村民出御。此二十人者,與同村諸婦避於樓。教匪入村,攻樓,不能克,乃收禾黍積樓下,環而焚焉。火熾,樓中諸婦有穴墻而跳者,或欲與二十人俱,二十人同聲曰:「教匪盈野,理難自拔,萬一求死不能得,何顏食息於人世?死於刃,死於水,死於火,死同也。惟畢命於此,吾儕志決矣!」俄而風起,火益怒,樓燼,二十人熸。二十人中已適人者,何李氏、張王氏、劉王氏、馮劉氏、傅李氏、任趙氏、任周氏、任宋氏、任邱氏、任張氏、任趙氏、趙葉氏、李張氏、張趙氏、崔郝氏;未字者,何氏、馮氏、傅氏、熊氏、崔氏。

王自正妻馬,秦安人。嘉慶五年,教匪破縣,馬被掠,罵不已,刀脅之,益厲,眥裂血,賊積薪焚殺之。

強逢泰妻徐,韓城人。逢泰父克捷,嘉慶間官滑縣知縣。十八年九月庚午,李文成之徒為亂,克捷及其妻殉焉。前一月,逢泰將其弟望泰歸取婦。亂作,徐罵賊不為屈,賊縶徐釘著事柱上,臠割之,棄其骨。事聞,仁宗以徐死事烈,命謚節烈,贈恭人,附祀克捷祠。

方振聲妻張,大興人;陳玉威妻唐,臺灣人。振聲官嘉義縣斗六門縣丞,玉威官臺灣北路協把總。道光十二年十一月,盜張炳為亂,遣其徒黃城攻斗六門,振聲、玉威與千總唐步衢拒戰,皆死之,張、唐殉焉。張罵賊,劓鼻剜舌死尤慘。其幼女亦從死。

宣宗命張、唐並謚節烈,附祀振聲、玉威祠。終清世,婦人得謚者凡三人。克捷、振聲、玉威語在忠義傳。

寶豐二婦,不知其氏,縣察河寨人。道光中,教匪為亂,官軍逐捕,以車載火藥留置寨中,為教匪所詗,將攘而有之。攻寨急,墮其一隅為陂陀,肉薄以登。二婦見賊入,大呼曰:「寨破矣!火藥且資賊,奈何?」寨中人皆潛避,無應者。二婦從風而火,藥盡焚,煙湧塵起,蓬勃雺晦如夜,賊自相鬥殺,二婦燔焉。

戴鈞衡妻李、妾劉,桐城人。鈞衡,文苑有傳。咸豐初,洪秀全之徒攻縣,鈞衡避舒城,李、劉及二女居。寇至,仲女年十六,抗刃死,李、劉皆被掠。寇使他所掠婦與李處,李陽與諸婦語,納手入袖。忽口噴血僕地,視之,刃刺喉死。寇欲褫其衣,其侶呼曰:「此烈婦!汝褫其衣,吾斬汝!」諸婦防劉益嚴,劉受李誡,以間脫其幼女囚。兩月餘,不言,不櫛發。一日,寇欲污之,乃大罵。寇怒,殺諸東郊外,罵不絕。曰:「吾今可以報女君矣!」遂死。

陳吉麟妻周,臨川人。咸豐間,洪秀全之徒破縣,周與女仙英走銅嶺,賊及之,加劍於項,逼之,不肯從。殺仙英,愈怒,批賊頰,賊殺之,尸提其首而立,賊為之驚走。

同時凌傳經妻楊,彭澤人。與姑匿山中,賊搜得姑,楊持刀奔赴。賊舍姑與鬥,力盡,為賊支解。楊同縣又有賈蓮品妻韓,摑賊,為所磔。

秦耀曾妻畢,耀曾,江寧人;畢,鎮洋人,湖廣總督沅女也。耀曾以舉人官郎中。咸豐三年二月,洪秀全攻江寧。畢年將八十,城破,集家人告曰:「吾家人受朝廷恩,於義當死。爾曹皆朝廷百姓,平日受承平之福,今寇亂,可愛死乎?且為賊得,必有求死不得者,悔何及!」乃服命服,扶杖赴水死。從者數十人。

曹士鶴妻管,名懷珠,字藏真,亦江寧人。士鶴官陜西清澗知縣。城將破,與士鶴兄妻李縊硃氏祠樹上,自書衣襟曰:「陜西清澗縣知縣曹士鶴妻管氏為國死於此。」

謝石全妻廖,曾石泰妻黃,葉金題母胡,繆勝云妻黃,皆定南人。咸豐六年,粵賊攻城,廖、黃皆助城守。廖執刃登陴,歷數十晝夜。一夕,依堞視賊,為飛砲所中,遂卒。黃佐石泰殺賊,賊攻城東南隅,黃赴救,中火槍,猶大呼殺賊,死城上。八年,賊復至,攻胡所居村,金題從鄉兵禦賊,胡握析薪斧,踣賊十餘。力鬥,被重創,與金題俱死。勝雲所居曰繆家莊,土寇作,黃與妯娌發火箭殪賊。賊逾屋入,勝雲與其父皆死。黃揮刀巷戰,久之,賊大至,自剄死。

石時稔聘妻劉,名敏和,吳縣人,家洞庭山。時稔卒,劉得請於父母,奔喪,奉姑居。咸豐十年夏,洪秀全之徒破蘇州,洞庭山民拒守。閱歲餘,力盡。賊自山前入,劉盛服待水次,誓死。居三日,賊不至,姑挽令入室,劉問:「何以得免?」則曰:「率錢輸賊兵。」劉躍起,哭曰:「是乃降也!降則此賊土,吾賊人矣。吾以為三日中,若輩與賊決死戰耳。今若此,何用生為?」姑與家人輩力勸毋死,劉好謂曰:「我三日不入戶,憊矣!且少休。」入室,即夜自經死。留一紙,自書生死年月日。

章瑤圃女亥姑,餘杭人。咸豐十年,年十五。六月庚午,賊至,亥姑抱柱堅不釋,賊擊之,十指皆創,抱柱如故。賊斫其肩背,亥姑罵曰:「恨不為男子殺爾輩盡!」賊勒其頸死。

戴可恆妻硃,可恆,仁和人;硃,長興人。可恆父熙自有傳。咸豐十年,杭州破,熙殉。硃具衣衾,視斂如禮,從可恆轉徙。明年,復還,賊復至。圍急,硃方為詩詞自若,曰:「我自為計久矣,何懼!」城破,硃語可恆速將子出避,賦詩矢死。不食兩日,未絕;自經,絙斷,又未絕;夜入池死,即熙死節處。熙死時,少子穗孫妻孫,方歸省,聞即仰藥殉。其祖母姚、母閔,及諸弟、妹皆死,凡七人。

金福曾妻李,福曾,秀水人,有傳;李,餘杭人。福曾父鼎燮,官臨安訓導,寄孥杭州。洪秀全之徒再攻杭州,圍久食盡,雜啖草木,甚至煠雨屐緣革為食。城將破,李與福曾矢必死。尚餘銀餅一,為福曾縫置衣復絮中,謂窮途得此,猶可旦夕活也。俄,賊大至,投姻家洪氏屋後池死。同時鼎燮殉臨安,鼎燮弟鴻僖妻胡,避臨安村間,為賊所迫,矛舂其喉死。咸豐十年,賊破嘉興,福曾之族諸婦女死者,衍芹妻倪、衍科妻鍾、鴻鑒妻徐、鴻墀妻許、鴻勛妻潘、鴻勩妻胡、鴻綬妻顧、鴻紱妻屈。徐、許皆有女從死。振聲妻張,賊將至時先自經殉。

張福海妻姚,錢塘人。福海官廣東曲江知縣。姚家居,寇至,城圍合,米盡食麥,麥盡食糠粃,糠粃盡食馬料豆。城破,賊脅姚行,姚奮起擊賊,被殺。同死者娣、姒孫、王,女杏珠,侄女滿、文、月。

邵順年妻伊,仁和人。順年,懿辰子,懿辰自有傳。杭州被圍,伊炊粥奉舅姑,輒忍饑不食。城破,俟其姑既出,入井死。巡撫馬新貽上懿辰死事狀,附陳伊「生則以孝事親,臨難不求茍活,深明大義」,得旌。

順年弟順國妻劉,亦仁和人。順國為六合知縣,卒。劉父堃方為漢中知府,令以二子往。劉謂異鄉非可久居,以順國喪還葬。蒐先世藏書授二子,督就學甚嚴,二子皆成立。

陳某聘妻酆,海寧長安鎮人。未行而夫死,誓不嫁,奉父;父卒,為立後。年四十餘,賊至,焚其村,酆自沉水甕中。賊去,戚族往視之,其廬燼,甕水沸,尸為糜矣。

胡金題妻俞,金題,烏程人;俞,歸安人:家雙林。賊以有色,驅使行,不從,持刃哧

之,張目以頸就刃。賊笑曰:「癡女子!」乃縶以行。行數十步,有橋橫水,俞好語賊曰:「雨後泥濘,縶不可以行,乞舍我,我自從汝去。」復請以兩矛夾持以上,示無死意。至橋半,奮躍入水,賊怒其紿,矛刺之,死。

王氏女婉容,亦家雙林。賊掠其父母,婉容請於賊:「釋父母,我從汝去。」賊釋其父母。已入舟,婉容出戶呼曰:「我猶有語,請少待!」且呼且行,近水,疾躍自沉。賊操矛拯之,不上,遂死。

鄭德高妻與方其蓮妻,皆阮氏兄弟也,蘭谿人。賊破縣,德高、其蓮將其孥避北山。久之,德高、其蓮偕入縣,為賊殺。二婦慟,誓死。一日賊奄至,二婦堅坐不為動。一賊持矛入,倚矛於壁,呼二婦具茗,二婦不應。賊解佩刀擲地,曰:「不應且死!」二婦厲聲答曰:「我曹畏死,尚坐待汝耶?吾夫死於賊,今當殺汝!」遂躍起,即取刀矛擊賊,賊徒手,被數創,大呼,群賊皆至,二婦力鬥死。

周小梅妻湯,名碩人,常熟人。咸豐十年,洪秀全之徒陷常熟,小梅方赴鄉,湯率子漣香、女淑貞及幼子、女入井死。將入井,囑長子於鄰翁;脫戒指付老僕,囑持書報小梅,書曰:

「昨君出門,飯後即失常熟,一夜未眠。今水窮山盡,當死義,恨不能一言為別。原君平安,勿以妾母子為念。寄戒指一枚,見此如見妾!」

楊某妻沈,名彩霞,金華人。生農家,有力,能舞大刀,重百斤。俗鬥牛,牛奔,彩霞手挽之,牛不得動。咸豐十一年,賊將至,鄉人集團練得數百人,推彩霞主之。時蘭溪諸葛燾團練過萬人,與相犄角,賊至則互救。洪秀全將李世賢自龍游至,彩霞乘其未定擊之,敗走。總督張玉良至蘭溪,暴於民,燾惎之。兵有自賊降者,偽為諸葛氏之幟過金華索犒,彩霞察其詐,擊殺數百人。玉良告巡撫,謂團練殺官軍,互訐不已。賊又至,偽為官軍裝,吏不復察,金華破,彩霞自剄死。楊某亦死亂軍中。

周世棣妻胡,鎮海人。咸豐十一年,賊掠世棣去,使市馬,以三賊監之行。世棣曰:「吾鄉故多馬,四人乃不足。」賊令募壯夫偕,世棣得鄉人同陷賊者六,導之至鄞東鄉。地僻,遂手刃三賊,其一實陽死,世棣未察也。遣鄉人自歸,矯賊令入寧波,出被掠男婦數十輩。夜半,陽死賊歸告其渠,將群賊捕世棣,世棣逃走。賊執世棣母及胡,胡語賊曰:「吾家有藏鏹,請以吾質,遣吾姑發藏金強,饋諸公。」姑已去,胡仰藥死,世棣母子皆得免。

蔡以瑩妻曹、妾馬,蕭山人。咸豐十一年,賊自嚴州循江薄蕭山,以瑩將妻妾子女避兵王家橋。遇賊,劫曹,將犯之,且罵且入水死。子景軾、女景良奔赴,與俱死。女景李為賊掠,語賊:「勿相強,我固原從汝。」賊稍寬之。行近水,亦疾躍自沉。馬抱三歲子匿葦間,以瑩還,求得馬。賊復至,馬視道旁舍有採菱者所遺木罌,折枯木授以瑩使乘以渡。以瑩要馬偕,馬曰:「此非舟,不能勝二人。」出懷中兒投以瑩,曰:「以此子隨君去。」以瑩渡未半,回望賊垂及,馬呼:「君勿念我,今與君永別!」赴水死,以瑩得免。

王永喜妻盧,永喜,開州人;盧,清豐人。咸豐十一年四月乙巳,盜李古考圍州城,永喜將出助守,語盧曰:「若聞砲,即城破,吾家世清白,慎勿為賊汙!」盧曰:「諾。」賊至,舉砲相擊,城得全。永喜歸,則盧率二女自經死矣。二女:長曰印,次曰改。又有張氏婦,村居,賊執以去。見井,紿曰:「我渴甚,乞解縛飲我!」賊解縛,入井死。

劉崇鼎母張,都昌人。咸豐間,洪秀全之徒攻縣,縣人治鄉兵,推崇鼎主其事,崇鼎謝母在。張曰:「人誰無母,皆以母謝,誰當殺賊者?」崇鼎受命主鄉兵,張出家財佐餉。賊至,崇鼎請母避賊,張泫然曰:「未戰而先策敗,人心散矣!有進尺,無退寸,此外復何顧?」崇鼎雪涕出戰,敗死。張聞敗,曰:「崇鼎死矣!」遂自經,未絕;賊已入,張出,坐堂上,罵賊,死之。

武昌女子,不知其姓氏,在賊中號為硃九妹。咸豐間,洪秀全破武昌,驅以東,至江寧,楊秀清欲納之。女侍飲驩甚,潛置毒酒食中進秀清,持之急,秀清察有異,磔死。

滄州女子,亦不知其姓氏,同治七年,張總愚北攻滄州,其黨得此女,獻總愚,總愚使執役。女袖出剪刺總愚,傷其臂,群賊集,立醢之。

費某妻吳,費某,德清人;吳,處州人,失其縣。父景籓,為湖州運糧千總,因以女歸費。早寡,事祖姑甚謹。洪秀全之徒陷德清,景籓他徙,吳囑以子而留事祖姑。賊大至,追吳,將汙之,不從。賊抽刃出,祖姑與相向哭,吳慷慨求死。賊系之樹上,曰:「我出汝心,觀汝心堅否?」刃剚胸,出心,堅如石,賊大驚。就德清人求其姓氏,曰:「此婦殆有神!」

冷煜瀛妻盧,義寧人。煜瀛官都昌訓導,洪秀全之徒破縣,死之。盧伏哭煜瀛側,為煜瀛理須,厲聲罵賊。賊斷其舌,死,手猶握須弗釋也。

陳兆吉妻餘,亦義寧人。義寧破,賊殺兆吉。餘方姙,罵賊,賊刳其腹,兒逐刃墮,呱呱泣,賊驚走。其渠聞,為之少戢。

蔡法度妻簡,新淦人。簡早寡,美。洪秀全之徒攻縣,名索簡,言不得屠蔡氏。蔡氏大忷,簡曰:「是無難。」艷服乘輿出,方度谿橋,驟自輿躍出,入溪水。溪水急,求其尸,勿能得。

張守一女春英,山西人,寓海城。同治二年,回亂,守一已卒,弟、妹幼,母悲泣。春英陽語回:「能脫我母及弟、妹,原相從。」回遣兩騎使守一舊僕護之行。春英度去遠,入井死。

王占元妻楊,皋蘭人。同治四年,回亂,楊從家人匿山穴中,為回所得。楊曰:「如愛我,幸毋傷我姑。」回驅楊去,至一村,回入掠。楊語途人曰:「我王占元妻,將死於此。乞寄語吾夫,速負母遠遁!」遂入井死。

王秉堃女翠環,固原人。亦為回得,欲挾之去,翠環曰:「釋我父、兄,可。」回釋其父、兄,曰:「我弱不任騎,原以輿行。」回喜,俾以輿行,女輿中餌毒,未至回所,死輿中。

魏克明女秀蓮,涇州人。同治七年二月,從兩兄行避兵。回至,次兄中矛死。秀蓮跪請活長兄,回許之。長兄脫走至山麓,遙望回迫秀蓮乘馬渡水,至中流,墜水死。

劉慶耀妻廖,龍南人。慶耀貰酒自給。同治三年,賊至,廖持刃衛姑出。賊執姑,廖揮刃斷賊腕,姑得脫。賊鬥廖,廖殺二賊,力盡,刳腹斷舌死。

歐陽維元妻曹,崇仁人。姑早寡,年九十九矣,賊急,曹奉姑走太浮山,遇賊,姑見殺。曹與維元擊賊,皆死。

李盤龍妻鄧,永新人。賊攻縣,鄧與族娣、姒走,遇賊仕坪。三婦共鬥賊,皆死。娣、姒失其氏。

黃氏女,名婉梨,江寧人。咸豐三年,洪秀全破江寧,婉梨方五歲,有母,與兄弟居。同治四年,師克江寧,有兵入其室,殺其母及其兄弟,縛婉梨置舟中,謂將歸湖南。婉梨好語兵:「至汝家,當妻汝,舟中毋相逼。」時有金眉姑者,亦被掠,自沉於江,婉梨舉以怵兵,兵不敢犯。月餘,將至其家,驅就陸,兵遇其侶,與俱投逆旅,二人方共飲,婉梨見牖上有毒鼠藥,潛置食中。夜分,一人毒發死,一人毒淺,未即死,婉梨掣所佩刀剚其腹,題詩壁間,述始末,自經死。

程氏女,名季玉,歸安人,從父居蘇州。蘇州陷,其父以醫卜自活。師克蘇州,季玉與其父相失,就鄰媼匿桃花塢。其女兄為部曲將所得,脅季玉去。季玉自經,不死,作絕命詩畀媼,使他日告其父,入井死。

韓肖硃妻郗,趙州人。姑瞽,張總愚自柏鄉向趙州,郗奉姑走欒城。賊驟至,姑曰:「我瞽不能行,汝可疾逃,無以我累汝!」郗侍姑終不去。賊見其少,將縶以去,郗請訣於姑,賊稍緩,郗急趨赴井。賊持矛遂之,郗張兩手以拒,回身墮井死。賊去,出其尸,矛創七。

張醴仁妻王,武強人。張總愚之徒入縣境,王避亂深州。賊至,王與婦女數百自沉于滹沱,水淺,不即死。賊據河濱村二日,饑凍顛踣,一婦哭曰:「此不即死,不如死賊刃!」王曰:「見殺於賊辱甚,不如水死!」三日殭立死。

同縣許氏女,從其父避賊。行遇賊,女促父速去。父陟岡望之,賊授女鞭令上馬,女持鞭鞭賊,罵曰:「子!安敢爾?」賊縶女,挾刃迫之,女罵如故。刺其腕,刺其肩,罵如故,遂見殺。

李氏女,名蒲,饒陽人。亦從父避賊。賊至,將劫之去,女抱持父,坐於地不起。父令從賊行,道側有井,父顧曰:「蒲,井也!」蒲疾入井。賊並擠其父入井,同死。

楊某妻吳,武進人。子傳第,以舉人官知府。客河道總督幕,迎吳居黑堈。黑堈在開封北,濱河。同治三年八月,捻匪攻開封,未下,掠黑堈,吳罵賊死。傳第從河道總督在開封,聞母死,大戚。以為不能豫戒,陷母死,為母撰行述,成,抑藥死。

康創業妻,與李鴻業妻,皆邸氏,兄弟也,深澤人。同治七年,張總愚黨掠縣境,方歸寧,從其父半千登屋避。賊登,刺半千死,姊持梃擊賊,妹奪賊刀殪之。賊踵登,揮刀墮梯下,斃。賊發槍,妹僕,姊被數十創,亦死。

王書云妻谷,亦縣人。書云精針灸,谷傳其術,活婦女無算。賊至,矛刺其子鳳銜僕,穀操杖擊賊酋。賊縱火,與其子鳳德、鳳桐,女然文,皆死。

王有周妻楊,玉門人。早寡,撫三子漢連、漢元、漢科,皆長。同治三年正月,回攻所居堡,急。楊使漢元間道詣肅州請兵,漢連以其人出御。楊聞砲聲急,意堡破,將二女孫入井死。漢連妻張挈次女自經,漢元妻李率次女飲酖,漢科妻李及子三、從女一、女甥一皆自剄。逾時回敗去,漢連歸,則家人皆狼藉死矣。

張金鑄妻段,平涼人。同治間,回亂,金鑄跳而逸,段未得從。回至,脅以刃,不為屈。砍項折,未殊,猶怒罵。復斷其左臂,乃僕,回委之去。金鑄歸,段尚能語,曰:「我家長物,盡為寇掠去,惟敝書數帙,我取置懷中,君可將去!」又曰:「我且死,君當速行!勿以我故留,寇復至,君將不免。」金鑄取懷中書欲去,返顧,段已絕。

王氏二女,香蘭、纏娃,秦州人。同治八年,回亂,掠香蘭。悅其色,以好言誘,不從;刃脅,不屈。欲走投崖,為賊追及,支解死。纏娃年十六,尤麗。賊縶以行,纏娃唾賊面罵,不少怯,亦見殺。

馬安娃妻趙,秦州人。莊而有容。回亂,見執,纏賊,劙口,被數十創而死。安娃母田、兄妻趙皆死。

王之綱妻李,亦秦州人。扶姑避賊,賊及之,李捍賊刃,乞代姑,姑得間走,李乃罵賊。賊剜其左目,被十餘刃而死。

穆氏女,名芝,束鹿人。幼慧。同治七年,年十八,捻匪至,欲縶以去。女哀之,不聽,乃呼其父曰:「速去!勿相顧,兒自有以處之。」父行稍遠,芝厲聲詬賊,賊鞭之僕。賊曰:「汝陽死,豈拾汝耶?」就曳之,芝驟舉足創賊目,賊連刃刺之死。

張某妻蔡,秦安人。同治中,回亂,蔡有色,回使執爨,不可;與語兼嘲謔,蔡奪他賊刀刺之,傷賊手,見殺。

同縣程丁兒妻黃,執廚刀擊賊,不中,賊刳其腹,引腸懸樹上。

張氏女,小字純秀,年十七,有色。為回得,堅縶之。女止哭,求弛縛,度峭巖,聳身自擲巖下死。

趙貴賜妻任,甘肅安化人。同治間,回亂,貴賜為團勇,戰死。回入其家,任執廚刀伏戶側,回先入者,出不意,斫之,踣。餘賊挺矛入,任反刃自殺。

楊貴升妻劉,伏羌人。回執其姑,將捶楚,劉請代,不聽,取廚刀殲一賊,因自殺。

多寶聘妻,宗室氏,多寶,赫舍里氏,失其所隸旗;宗室氏,正藍旗人,大學士靈桂兄女。未行,多寶卒,易衰絰,赴吊,立從子英爚為後。靈桂以聞,穆宗書「未吉完貞」四字以賜。

英爚亦早卒,妻鄂卓爾氏,蒙古正白旗人,大學士榮慶女弟。婚甫逾月,姑、婦食貧守節。光緒二十六年,義和拳為亂,各國合軍入京師,城破,多寶弟和寶妻,率傭婦入井;多寶妻起,引藥飲其婦,視既絕,乃自飲,同殉。

公額布妻,西安駐防,失其所隸旗。善事姑,三十而寡,教二子奎亮、奎喜,有禮法。宣統三年九月,亂作,戒二子曰:「此我完節時,汝曹當努力報朝廷,毋念我!」城破,率二子婦及孫定炎、成惠、孫女三入井死。清中葉後,八旗多從漢姓,公額布妻姓關桑氏,奎亮妻關鄂氏,奎喜妻關白氏。

音德布女雪雁,西安駐防,正紅旗人。幼慧,粗解文字。亂作,從家人出避。行遇兵,有誘之者,雪雁引刀斷其指,血沾衣,誘者驚卻。又遇兵,強脅之,女大詬曰:「吾頭可斷,志不可奪!」兵群起抶之,無完膚,女罵不絕,刃洞胸死之。

良奎妻,從漢姓曰石甘氏,荊州駐防,滿洲鑲黃旗人,為駐藏大臣鳳全女兄。鳳全自有傳。貧,躬織紉供朝夕,諸子佐軍,迎母居武昌。宣統三年八月,武昌兵起,諸子將奉母出避,力拒曰:「吾七十老婦,死何憾!」諸子哭,麾之出,遽闔戶。翌日,兵大掠,與子婦二、女子一、孫及女孫三,皆死之。

連惠妻,從漢姓曰趙那氏,京口駐防,失其所隸旗。連惠咸豐間以前鋒從攻鎮江,戰死。連惠妻以節旌。宣統三年,年已逾八十。九月兵起,出走,兵抽刃擊之,未殊,罵不絕,被數刃,乃絕。血肉狼藉,白發為之赤。

根瑞妻,從漢姓曰王劉氏,京口駐防,鑲白旗人。父德永,有文譽,客授學子。根瑞妻服父訓,早寡,以節旌。無子,有女已嫁,依以居。聞兵起,語女及女夫曰:「吾年六十二,被旌,當殉變。爾曹將子女村居,得田十畝,耕且食,毋更求仕。」俄聞副都統載穆死官,即求死,輒救免;號泣不食,女及女夫跪進食,終不食,七日乃絕。

松文母吳,松文,荊州駐防,鑲藍旗人。同治初,徙江寧,從漢姓為馮氏。吳,荊州士人女也。事姑孝,早寡,無子,松文,其族子也,立為後。松文子富倫渾,才而早卒,松文哭子慟,亦卒。松文妻康,富倫渾妻石,仍世守節。宣統三年,兵起,江寧駐防軍潰,松文母年九十三矣,慟哭,以仍世守節,義不辱,首觸墻死。康與婦石將諸孫自沉於水。康死,石與子、女遇救免,康與石不詳其族系。

姚葉敏妻耿,襄城人。葉敏早卒,事舅姑盡禮。立兄子為後。武漢兵起,耿方病,襄城土豪為暴,掠婦子為質,耿懼辱,飲藥死。

陳某妻殷,秀水人。宣統三年,殷從夫在郴州。九月,長沙兵起,湘南諸府州應之,郴屬縣宜章、永興皆變,殷告夫誓相守以死。夫趣殷將子女徙湘鄉,依戚屬避兵,殷不可;強之,乃行。瀕行,部署瑣雜事井井,入舟,抑鬱,語子女:「若曹免矣,若父奈何?」湘鄉距郴千餘里,俄傳郴破,殷憂悸不食,面深墨,戚屬相慰藉,陽為酬答。十月壬子夕,戚屬同居者,聞啟戶聲,旋聞其季女驚呼阿母起,燭之,就堂後門衡自罄死矣。

黃晞妻周,江陰人。晞父毓祺,明諸生,能文,明亡,發狂亡命。有司得晞系諸獄,周聞自經,婢救之,不死;乃日餽獄饘粥,夏不施帷,恣蚊嚙,曰:「我遙與獄中共辛苦也!」晞入獄十閱月,事小解,得出。居無何,怨家告毓祺所在,死江寧按察使獄中。有司籍其家,捕晞兄弟,兼收周,周夜投水,不死;茹金屑,亦不死;乃詣府,藏刃刺喉,血沖溢僕地。知府驚其烈,問晞有女兄為女僧,命舁置所居庵,上按察使請釋周,按察使不許,下縣令再收周。周創漸合,乃自歸,語縣役曰:「我不累若輩,第徐之,俟我死,持片紙去公家,事易了也。」手檢晞單衣一,付老僕曰:「主人行久,無衷衣備澣濯,汝以此寄之!」徐入室,闔戶自經乃死。時順治七年十月丁巳,年二十八。晞尚系按察使獄,聞周死,為文述其事,略言:「古成仁取義之士,所以趨死之道不一,由其一,皆可得死。婦獨多途遍歷,靡苦不嘗,而顛跌頓撼,卒死於家。一以顯百折不回之苦節,一以遂正命內寢之初心,天不可謂無意云。」晞輸八旗為官奴,鄉人贖出之,得歸,為童子師,至七十餘乃卒。

鄒延玠妻吳,武進人。延玠,明諸生,順治八年,逮系江寧獄。十年,見法。吳自經,救不死。十二年,延玠喪還葬。十三年,有司復議收延玠家北徙。吳乃迎母至,夜將半,起,請母所曰:「兒今固必死,安能俯首求旦夕活,作長安累囚婦耶?原母稍忍,成兒死。」母泣不能言。吳更衣拜佛,復向母曰:「兒欲為母拜,恐傷母心,兒不敢。母老矣,勿以兒故過哀!」因出一扇,曰:「此夫子南京寄我者。」出一囊,曰:「有醫方,夫子所手校。有書,夫子生平所習。有發,夫子獄中所留也,仍乞以殉。」復呼婢戒毋號。乃自燃燭,持囊及扇還入室。時雞甫鳴,母及婢傍徨哭,不敢出聲。少頃,視吳,自經已絕。死前一日,苦熱,吳祝曰:「安所得甘雨乎?」遂雨竟日,人謂「節婦雨」。

陳生輝妻侯,單縣人。順治初,盜掠生輝使牧馬。縣北郭秦氏有馬,為盜掠,生輝乘以歸。秦氏見馬訟生輝,生輝坐通寇死。侯事姑,喪葬畢,並葬生輝,設祭自剄。

田一朋妻劉,通江人。國初,一朋不從薙發令,坐當死,吏並縶劉去。劉挾毒自隨,聞一朋將就刑,先服毒死。

蔣世珍妻劉,揚州人,失其縣。世珍,順治中為廣東連平知州,有惠於民。嶺海初定,土寇數發,諜報旁縣賊數千人鄉連平,行至。世珍曰:「賊至,驚吾民,吾且往,權順逆強弱而為之所。」單騎入賊中,諭其渠降,其渠為引退。世珍宿賊營,翌旦乃還。守備吳章者,故與世珍有隙,誣世珍通寇,告總兵黃應傑,應傑啟平南王尚可喜,捕世珍赴惠州獄,劉系置守備廨旁舍。章將無禮於劉,劉怒叱去。又遣婢說劉,劉曰:「死不可緩矣!」遂縊而死。世珍入獄病,亦死。連平民葬劉州南烏石坳,為之碣,曰「正烈劉宜人之墓」。嘉慶二十三年,知州陳鵬來上其事,乃得旌。

王有章妻羅,益陽人。順治七年,盜殺有章父賡及家人男婦二十餘輩。越三年,又殺有章,惟餘羅及有章妹頭貞,皆斷發剺面,號於有司。歷八年,乃論殺盜渠。羅謂頭貞曰:「我當報汝兄地下!」因不食死。

頭貞初字曹氏子,曹氏子以其毀容也,遂罷婚。頭貞徙長沙,仇家有子赴試,誘至家,殪之。

樓文貴妻盧,東陽人。文貴,農也,有鵝啄其麥,文貴驅鵝,傷鄰兒。鄰兒呼,遂毆之,投水死。里豪喝文貴,使鬻妻以為解。盧曰:「吾不忍生離!」文貴怵得罪,因求死,盧曰:「吾與汝同死!」遂入林偕縊。

沙木哈妻哈里克,滿洲鑲白旗人。沙木哈,兵也,為弟三太所擊,垂斃,沙木哈妻誓身殉。沙木哈言曰:「我止一弟,我死,弟抵罪。守先墓,撫諸孤,復何人?汝當言於官,曲貰三太死。」沙木哈遂死。沙木哈妻叩閽,述沙木哈遺言,乞貰三太,聖祖命許之。沙木哈妻得請,即自裁。康熙三年正月壬午,禮部疏請旌表,聖祖令立石塚上,書其事始未。

鄭榮組妻徐,西安人。榮組有族叔,無狀,毆其父,赴救,為所殺。其子五元、七元遇仇於途,嚙其鼻。仇愬於縣,縣吏逮五元、七元,徐以冤白吏,吏不省,撞縣門碑死,時康熙二十七年六月事也。典史某為具槥,露置城西鐵塔。越七年,知縣陳鵬年為營葬,立祠於墓側。

張翼妻戴,名禮,烏程人。翼父韜,嘗知休寧縣,託翼於其友王毅,毅以女妻焉。韜卒,毅女亦死,繼室於戴。毅子覬翼產,康熙六十年五月,誘至其家,迫作券,毆之垂斃,擠墮水。舁歸,不能語,瞠視戴。戴泣曰:「我一弱女子,不能為君復仇,當以死從君。」齧指以誓。越七日,翼死;又十七日,戴自經,衣帶間得絕命詩三章。

詹允迪妻吳,東陽人。允迪不嗛於族人,為所中,坐危法下獄,吳期與俱死。至其日,盡出金珠畀所識貧乏者,散諸婢僕,詣獄與允迪訣,瞠視不語者久之,歸自剄。

蔡以位妻孫,侯官人。以位佐鹺商與私販者鬥而死,孫迎喪河干,自擲入水,以救免。其娣,即其姊也,責以撫孤,乃不復言死。官捕得私販者,法當檢驗,讞乃定,孫曰:「是重僇吾夫也!」乃大戚。官悲其意,為杖殺私販者。喪再期,從容語其姊曰:「兒稍長,履可取諸市,不煩手自制矣。兒昔病瘍,今愈矣。不累我姊矣!」或曰:「姑在,既祥,當更淺色履。」孫曰:「然,姑徐之!」至大祥,奠竟,入戶自經死。

楊春芳妻王,銅梁人。乾隆十七年,其家火,春芳臥病,王入戶,負以行。火逼不能出,子女奔赴,皆死。

王尊德妾唐,臨桂人。尊德年八十,病劇,鄰家火,唐欲負以避,力不勝。火迫,尊德揮使出,唐身翼蔽尊德,皆死。

竇鴻妾郝,字湘娥,保定人。十六為鴻妾,能詩善弈,畫兼工花草、士女。有繩其才者,豪家謀奪之,不能。嗾盜誣鴻死,湘娥因自經。將死,為絕命詞,矢為厲以報。

章學閔妻董,名合珠,連江人。故為婢,嫁學閔。學閔貧不自聊,走死深山中。董號泣求之,不知其存亡。逾年,有樵入山,若有聲,行見遺骼委於地,只履在側。出以語人,董聞曰:「得非吾夫乎?」亟往視履,其手制也,拾餘骨瘞焉,即夕自經死。

杜聶齊妻何,聶齊,泰寧人;何,將樂人。聶齊死於虎,何求得尸,解衣拭其血。斂畢,斥家財以葬,悉以其餘分戚族,遂自經。

張氏婦,宿州人。夫樵於野,遇狼,為所噬。婦求得夫尸,以鐮絕脰死。

寧化二婦,不知其氏。其一,夫嗜博,母閉諸室中,不與飲食,婦導使出亡。既,夫死於途,婦聞,自殺。其一,夫行竊,父將殺之,婦泣為請免。生二子,婦攜就母家,父卒殺其夫,婦聞,亦自殺。


\end{pinyinscope}