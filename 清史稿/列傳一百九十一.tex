\article{列傳一百九十一}

\begin{pinyinscope}
僧格林沁舒通額恆齡蘇克金何建鼇全順史榮椿樂善

僧格林沁,博爾濟吉特氏,蒙古科爾沁旗人。本生父畢啟,四等臺吉,追封貝勒。族父索特納木多布齋,尚仁宗女。公主無出,宣宗為選於族眾,見僧林格沁儀表非常,立為嗣。道光五年,襲封科爾沁札薩克多羅郡王爵。十四年,授御前大臣,補正白旗領侍衛內大臣、正藍旗蒙古都統,總理行營,調鑲白旗滿洲都統。出入禁闈,最被恩眷。

咸豐三年,粵匪林鳳祥、李開芳等北犯,命僧格林沁偕左都御史花沙納等專辦京師團防。八月,欽差大臣訥爾經額師潰臨洺關,賊竄正定。詔授惠親王綿愉為奉命大將軍,僧格林沁為參贊大臣,上禦乾清宮親頒關防,賜納庫素光刀,命率京兵駐防涿州。十月,賊陷靜海,窺天津。兵進永清,又進王家口。賊不得前,乃踞獨流鎮。四年正月,僧格林沁會欽差大臣勝保軍乘夜越壕燔其壘,賊西南逸,追擊之子牙鎮南,擒斬甚眾,賜號湍多巴圖魯。復連敗賊於河間束城村、獻縣單家橋、交河富莊驛。賊竄踞阜城縣城,附城村堡皆為賊屯。僧格林沁偕勝保率副都統達洪阿、侍郎瑞麟、將軍善祿等諸軍圍擊,毀堆村、連村、杜場諸賊屯,砲殪悍酋吉文元,賊猶頑抗,攻之累月不下。粵匪復自江北豐縣渡河擾山東,浸近直隸境,欲以牽掣大軍,勝保及善祿先後分兵迎剿,迭詔責僧格林沁速攻阜城,於是穴地為重壕長圍困之。四月,賊乘風突圍出,竄東光連鎮。連鎮跨運河,分東西兩鎮,村落相錯,賊悉踞之。僧格林沁自率西凌阿屯河東,令托明阿屯河西,別遣馬隊扼桑園。會勝保已破賊山東,回軍合攻連鎮。五月,賊酋李開芳以馬隊二千餘由連鎮東突出趨山東,勝保率騎兵追之,遂竄踞高唐州。詔斥僧格林沁疏防,責速攻連鎮自贖。會霖雨河漲,賊聚高阜,官軍屯窪地,勢甚棘。於是議開壕築堤,以水灌賊營。堤成,蓄水勢如建瓴,賊大困,屢出撲,皆擊退。九月,東西鎮各出賊數千,欲突圍而竄,為官軍所扼,糧盡勢蹙。附近村莊皆收復,合力急攻,凡數十戰。十二月,斃偽檢點黃某。悍黨詹啟綸出降,焚西連鎮賊巢,僅餘死黨二千餘人,以大砲環擊。五年正月,破東連鎮木城,賊冒死沖突,盡殲之,擒林鳳祥,檻送京師誅之。畿輔肅清,錫封僧格林沁為博多勒噶臺親王,擢其子二等侍衛伯彥訥謨祜御前行走,敕移師赴高唐州督辦軍務。

先是,勝保圍攻高唐久不下,密詔僧格林沁查辦,至即劾罷之。賊聞連鎮既下,喪膽欲遁。大軍數日即至,故疏其防。賊果乘隙夜走,親率五百騎追奔五十里,至茌平馮官屯,賊踞以守。合軍圍攻,四面砲擊,賊掘地為壕,盤旋三匝,穴堀潛藏,穿孔伺擊,攻者傷亡甚多。復議用水攻,挑河築壩,引徒駭河水灌之。賊屢沖突,皆擊退。四月,水入賊窖,紛紛出降。擒李開芳及其死黨黃懿端等八名,械送京師誅之。北路蕩平,文宗大悅,加恩世襲親王罔替。五月,凱撤回京,上御養心殿,行抱見禮,賜朝珠及四團龍補褂。又禦乾清宮,恭繳參贊大臣關防,賜宴勤政殿,從征將士、文武大臣並預焉。林鳳祥、李開芳為粵匪悍黨,狡狠善戰,兩年之中,大小數百戰,全數殄滅,無一漏網,僧格林沁威名震於海內。

時英吉利在粵東開釁,乘東南軍事方棘,多所要挾,每思北犯。故近畿肅清後,命西凌阿分得勝之師赴援湖北,而僧格林沁遂留京師。六年,丁本生母憂,予假百日,在京持服。尋調正黃旗領侍衛內大臣。七年四月,英吉利兵船至天津海口,命僧格林沁為欽差大臣,督辦軍務,駐通州,托明阿屯楊村,督前路。倉猝徵調,兵難驟集,敵兵已占海口砲臺,闖入內河。議掘南北運河洩水以阻陸路,別遣議和大臣桂良、花沙納赴天津與議條約。五月,議粗定,英兵退。未盡事宜,桂良等赴上海詳議。於是籌議海防,命僧格林沁赴天津,勘築雙港、大沽砲臺,增設水師。以瑞麟為直隸總督,襄理其事。奏請提督每年二月至十月駐大沽,自天津至山海關海口,北塘、蘆臺、澗河口、蒲河口、秦皇島、石河口各砲臺,一律興修。九年,桂良等在上海議不得要領。五月,英、法兵船犯天津,毀海口防具,駛至雞心灘,轟擊砲臺,提督史榮椿中砲死。別以步隊登岸,僧格林沁督軍力戰,大挫之,毀敵船入內河者十三艘。持數日,敵船引去。

九年六月,英、法、俄、美四國兵百餘艘復來犯,知大沽防禦嚴固,別於北塘登岸,我軍失利。敵以馬步萬人分撲新河、軍糧城,進陷唐兒沽,僧格林沁力扼大沽兩岸。文宗手諭曰:「天下根本在京師,當迅守津郡,萬不可寄身命於砲臺。若不念大局,只了一身之計,有負朕心。」蓋知其忠憤,慮以身殉也。尋於右岸迎戰失利,砲臺被陷,提督樂善死之。僧格林沁退守通州,奪三眼花翎,褫領侍衛內大臣及都統。迭命大臣議和,不就。敵兵日進,迎擊,獲英人巴夏禮送京師。戰於通州八里橋,敗績。瑞麟又敗於安定門外,聯軍遂入京。文宗先幸熱河,圓明園被毀,詔褫僧格林沁爵、職,仍留欽差大臣。

十年九月,和議成,命遣撤殘軍,馳赴行在,未行,會畿南土匪蜂起,山東捻匪猖肆,復僧格林沁郡王爵,命偕瑞麟往剿。師至河間,匪多解散。詔促赴濟寧、兗州督師。十一月,至濟寧,賊已他竄回巢。疏陳軍事,略曰:「捻首張洛行、龔瞎子、孫葵心等,各聚匪黨無數。此外大小頭目,人數不少。每年數次出巢打糧,輒向無兵處所。迨官兵往剿,業經飽掠而歸。所至搶擄貲財糧米,村舍燒為赤地,殺害老弱,裹脅少壯。不從逆,亦無家可歸。故出巢一次,即增添人數無算。此捻匪眾多之情形也。匪巢四面一二百里外,村莊焚燒無存,井亦填塞。官兵裹糧帶水,何能與之久持?一經撤退,匪緊躡,往往因之失利。此各路官兵僅能堵御,不能進攻之情形也。每次出巢,馬步數十萬,列隊百餘里。兵賊眾寡懸殊,任其猖獗,無可如何。前此粵、捻各樹旗幟,近年彼此相通,聯為一氣。官兵在北,粵匪在南,捻匪居中,以為粵匪屏蔽。若厚集兵力,分投進剿,捻匪一經受創,粵匪蠢動,非竭力相助,即另圖北犯,以分我兵勢。此剿捻不易之情形也。臣原帶馬步六千,續調陜甘、山東綠營及青州旗兵,共一萬二千餘人。擬俟齊集,會合傅振邦、德楞額二軍,相機直搗老巢。」疏入,詔:「捻匪正圖北犯,應坐鎮山東,以杜窺伺,毋輕舉以誤全局。」尋捻匪由徐州北竄,迎擊於鉅野羊山,親率西凌阿、國瑞當其東,瑞麟及副都統格繃額當其西,殺賊甚眾,而格繃額陣亡。瑞麟傷退,劾罷之,薦西凌阿、國瑞幫辦軍務。又劾團練大臣杜不能禦賊,供應擾民,罷其任,團練歸巡撫督辦。鄒縣教匪宋紹明集眾數千戕官,令國瑞、西凌阿擊剿解散。

十一年,捻匪五旗並出,僧格林沁率諸將由金鄉迎剿。遇賊於菏澤李家莊,戰失利,察哈爾總管伊什旺布陣亡,回師駐唐家口。二月,令西凌阿馳赴汶上,會都統伊興額、總兵滕家勝追賊至楊柳集,戰歿。僧格林沁親駐汶上,令西凌阿回守濟寧。賊由沙溝渡運河,盤踞東平、汶上。德楞額追擊於小汶河北岸,破之,賊始東竄。四月,令舒通額進剿,解滕縣圍。德楞額克沙溝營、臨城驛,賊分兩路奔竄。其入曹州境者,勾結長槍會匪擾鄆城、鉅野,令知府趙康侯集諸縣鄉團御之。教匪宋繼明復糾眾踞鄒縣鳳凰山,令國瑞、德楞額攻之,連破賊圩,繼明尋遁走乞撫。六月,親赴曹州進剿會匪,連破之於曹縣安陵集、濮州田潭,擒其渠李燦祥、陳懷五等。八月,捻匪渡運河,犯泰安、濟南。僧格林沁親率大軍追躡,敗之於孫家鎮,賊走青州。九月,襲擊於臨朐縣南,沿諸城至沂水,黑旗捻黨跨河抗拒,分兵擊之,追及蘭山蘭溪鎮殲焉。捷聞,復御前大臣,賞還黃韁,授正紅旗漢軍都統,管理奉宸苑。穆宗即位,特詔嘉其勤勞,復博多勒噶臺親王爵。

是年冬,會東軍攻曹郡會匪,破濮州紅川口賊圩,搜斬無遺。毀劉家橋、郭家唐房賊巢,又破定陶賊於大張寺,復範縣。西凌阿等攻捻匪於鉅野境,大捷,定陶踞匪聞風遁走。會匪郭秉鈞自河西來犯,連擊之於崔家壩,至黃河南岸,屢挫賊鋒,曹郡漸清。疏陳軍事,略曰:「捻匪老巢多在宿州、蒙城、亳州境內,其北來,每由歸德之虞、永、夏,徐州之豐、沛、蕭、碭,直入山東之曹、單、魚臺,或由宿、徐北至韓莊、八閘。今領重兵進駐亳州,偏於西南一隅。北至徐州三百餘里,再東更慮鞭長莫及。如派隊輪轉,由西路進攻賊圩,即使得手,距亳州尚遠,東路捻眾豈能坐待,勢必由豐、碭、韓莊鈔襲我軍之後,我軍不得不回顧北路。一經移動,則亳東之賊尾隨,受其牽掣。故屯兵亳州之議,在豫省為良策,若欲衛東省兼顧北路籓籬,則未可行也。臣擬俟曹屬肅清,移營單縣,觀皖捻動靜,剿撫兼施。鄒縣教匪踞險難攻,暫準投誠,以示羈縻,留兵鎮壓。待南捻稍松,相機辦理。滕、嶧之匪,德楞額招安劉雙印、牛際堂等,若有反側,仍應往剿。河北教、捻各匪,本年兩次鴟張,眾不過一二萬。臣令西凌阿、國瑞兩次會剿,勝保等方能得手。勝保於此匪尚不能獨力剿除,豈能當十餘萬之捻眾?壽張及曹屬一帶,臣已辦理就緒,毋須勝保前來會剿。」疏上,詔從之。

同治元年正月捻匪二萬餘由江北豐縣犯金鄉、魚臺,令翼長蘇克金擊走之。二月,亳捻張洛行合長槍會匪西竄,勢甚張。僧格林沁率馬隊追至河南杞縣許岡,賊列隊橫亙十餘里。蘇克金等奮擊,斃賊二千餘。西路援賊至,豫軍亦來會剿,嬰城而守,連日鏖戰。以馬隊伏壕邊伺賊懈,城中突出勁騎沖賊營,伏赴夾擊,毀賊壘七,斬馘千餘。越日餘際昌率步隊至,與蘇克金合擊,沖賊為兩,追殺二千餘。於是先破趙圩賊寨,合攻焦寨,援賊數至,皆擊卻,賊宵遁。是役三路合剿,殲匪萬餘,捷聞,特詔褒獎。僧格林沁督率諸將窮追竄匪,破之於尉氏東。賊踞民寨堅守,圍攻之,旋虛東面誘之出,至樊家樓,盡殲焉。五月,補正黃旗領侍衛內大臣。長槍匪黨董智信竄東明,蘇克金馳剿,受降。營總富和破坦頭集捻巢,招撫被脅數十圩寨。恆齡破焦桂昌於曹州,乞降,誅之。

六月,進攻商丘金樓寨。教匪郝姚氏及金鳴亭久踞金樓,其黨尤本立、常立身尤兇悍,官軍屢攻不克。僧格林沁先遣諜用間,諭令投誠,金鳴亭潛允降而不出,其子線駒居郭家老寨,密捕之。會有賊黨通教匪,以鳴亭稟詞示常立身,立身遂殺鳴亭,賊中自相疑忌。至是合兵進攻,游擊許得等率降人為導,先攻入,大軍繼之,巷戰,斬郝姚氏及其兩子,常立身、尤本立、楊玉聰同授首,餘賊盡殲,夷其寨。乘勢連破援賊於邢家圩、吳家廟、營廓集,前鋒直抵亳州境。僧格林沁移駐夏邑,疏陳將帥市恩麾下,督撫見好屬員,保舉冗濫,吏治廢弛,州縣捏災私徵,軍餉不足,言甚切至。詔嘉其公忠,命統轄山東、河南軍務,並直隸、山西四省督、撫、提、鎮統兵大員均歸節制。

八月,令恆齡、卓明阿等追捻匪姜臺凌至裕州博望驛,大破之,餘眾遁入山。別股李城、趙浩然等乘大軍分隊西行,糾眾擾永城,復由碭山北竄。副都統色爾圖喜追至魚臺羅家屯,戰不利。僧格林沁促恆齡等回援,親督進戰於鉅野滿家洞,令馬隊誘賊深入,回擊之,恆齡、國瑞分合沖突,斃賊數千。復連敗之於子山集,賊東南竄。亳北白旗捻首李廷彥以邢大莊為老巢,附近賊圩互相首尾。九月,僧格林沁自攻盧廟,令國瑞、恆齡攻邢大莊及張大莊。廷彥見事急,詐稱投誠,誘出誅之,黨羽多乞降,惟孫老莊匪首孫彩蘭不肯出。令降匪李匊奇為導,攻入寨,擒斬彩蘭,諸寨皆下。亳東黑旗捻首宋喜沅,因與蘇天柏相仇殺,諸悍黨攻破王大莊、劉大莊兩寨來降。諸小寨頭目聞風歸順,亳北肅清。於是諸捻懾震兵威,多思反正。

二年正月,馬林橋、唐家寨、張家瓦房、孟家樓、童溝集諸賊巢先後剿平,著名捻首魏喜元、蘇天才、趙浩然、李大個子、田現、李城等或降或遁。張洛行為巨憝首惡,見勢敗,時思竄逸。會孫丑、劉大、劉二、楊二等由鹿邑西竄,令舒通額、蘇克金等追之,戰於魏橋,殲戮甚眾。洛行欲由宿州趨徐州,為知州英翰所截。又聞西路諸匪被創,洛行遂潛回雉河集老巢。尹家溝、白龍廟與雉河集為犄角,二月,令舒通額等進攻尹家溝。賊出撲,擊潰,遂攻雉河集。洛行夜遁,追至淝河北岸,拒戰,殲賊過千,擒斬捻首韓四萬等。逸匪多潛匿各莊寨,分軍駐索。西洋寨捻首李勤邦投誠,誘擒張洛行及其子張憙以獻,磔之。捻匪自蒙、亳創亂,已歷十年,至是掃除。詔嘉僧格林沁謀勇兼備,加恩仍以親王世襲罔替,並準服用上賜章服,以示優異。

時北路竄捻與教、會各匪句結肆擾,僧格林沁回師,令恆齡、蘇克金馳赴直、東交界會剿,自剿淄川踞匪劉德培。六月,賊傾巢出撲,追敗之於田莊,遂克縣城。德培遁大白山,擒斬之,進攻鄒縣。白蓮池匪首宋繼明屢降屢叛,擁眾二萬餘,恃險抗拒。令總兵陳國瑞、郭寶昌猛攻,破其山寨,敗竄紅山,死守經月,糧盡欲遁。令舒通額等設伏嶺下,陳國瑞於山北攻上焚其寨,殺賊過半。其竄山下者,伏起並殲。擒匪首李九,獲宋繼明尸及其家屬。留國瑞暫駐,搜緝餘匪。即日令陳國瑞赴皖剿苗沛霖。

沛霖倔強淮北,當張洛行伏誅,懼,請散練歸農。及僧格林沁北行,又襲攻蚌埠、懷遠、壽州,圍蒙城,皖軍不能制。至是僧格林沁督軍討之。陳國瑞先至,連戰皆捷,匪黨喪膽。十月,大軍進亳州,連克蔣集、楊家寨。與陳國瑞合攻,絕其糧道,破蔡家圩,淮河兩岸賊壘悉盡。沛霖昏夜越壕出竄,為其黨刺殺。總兵王萬清斬首以獻,逆黨苗憬開等均伏法。尋破西洋集,擒匪首葛春元,潁、亳、壽境圩寨悉定,淮甸漸清。

時捻匪張洛行之侄總愚擾河南,令蘇克金率馬隊往會剿,而降捻李世忠,官至江南提督,素跋扈,盤踞淮南,將為隱患。詔曾國籓密為處置,命僧格林沁駐軍鎮懾。三年春,世忠自請解兵柄。會漢南粵、捻諸匪糾合下竄,與張總愚相應接,將圖南犯,為江寧踞賊聲援。僧格林沁乃督師赴許州,進南陽,與河南、湖北諸軍會剿,迭破賊於信陽、應山、鄖陽之間。六月,江寧克復,大賚諸軍,詔嘉僧格林沁轉戰勛勤,加一貝勒,命其子伯彥訥謨祜受封,復以所部蒙古馬隊最得力,保舉素無冒濫,命擇尤奏獎,賞兵丁銀一萬兩。

七月,粵、捻諸匪麕聚麻城,令蘇克金、張曜、英翰等分路進擊,破賊壘數十。捻首陳得才以萬眾來撲,戰於紅石堰。蘇克金力戰,殲賊甚眾,遽病暍卒,以成保代之。賊竄麻城南境閔家集,結壘為固,成保攻破之。總兵郭寶昌克蔡家畈,賊竄河南光山、羅山。僧格林沁親督馬隊追擊,戰於蕭家河,援賊大至,稻隴地狹,馬隊失利,自翼長舒通額以下,陣亡將領十二人。八月,復戰於光山柳林寨,先勝,中伏,為賊所圍,力戰始退,總兵巴揚阿死之。九月,張總愚東竄,與上巴河、蘄州之賊勾合,踞風火山,僧格林沁會鄂軍進剿,連戰破之。賊趨安徽境,分竄潛山、太湖、英山。十月,連破之於土漠河、樂兒嶺、陶家河。匪目黃中庸率千人來降,追至黑石渡,令黃中庸為前鋒,襲賊營,大軍繼之,沖賊為兩段,賊目溫其玉等率九千餘人投械乞降。偵知賊分三路,遣兵分剿,捻首馬融和率黨七萬人投誠,原為前敵。賊黨甘懷德誘擒偽端王藍成春出獻,磔於軍前。餘黨汪傳第、吳青泉、吳青泰、範立川等各率眾乞撫,先後受降十數萬人,著名匪首僅存數人。陳得才尋亦窮蹙自盡,惟張總愚、陳大憙西竄河南、湖北境,復猖獗。

十一月,僧格林沁督軍追剿,敗之於光山境,進至棗陽。粵匪賴文光、邱元才,捻匪牛洛紅、任柱、李允等竄踞襄陽黃龍垱、峪山,官軍進擊小挫,而張總愚、陳大憙乘間與合,圖犯樊城。大軍追擊於鄧州唐坡,賊傾巢出撲,兩面包鈔,官軍失利,傷亡甚多。僧格林沁自請嚴議,詔寬之,乃駐軍南陽。十二月,賊由南召、魯山竄踞寶豐張八橋。大軍進逼,令郭寶昌、何建鼇分南北兩路,恆齡、成保以馬隊護之。北路逼賊而營,賊來撲,成保橫出鈔襲,乘勝壓過山岡;南路誘賊深入,從旁更番進擊:兩路皆捷,合軍追擊,直抵張八橋。賊夜遁入山,北趨河、洛。僧格林沁督軍由洛陽取道宜陽,駐韓城鎮。

四年正月,賊折而南犯魯山,大軍追及,戰於城下。前鋒得利窮追,後路為賊鈔襲,翼長恆齡等陣亡。舒倫保、常順馬隊接應,陳國瑞橫突扼橋上,始得全師退,而舒倫保、常順亦以傷殞。賊遂竄葉縣、襄城,陳國瑞乘雪夜襲攻,縱火焚之。賊東北竄新鄭、尉氏,追及於雙溪河,翼長諾林丕勒等擊走之。賊南趨,由臨潁、郾城擾西平,裹脅愈眾,遂犯汝寧。二月,僧格林沁進抵汝寧,賊由息縣、羅山竄信陽。大軍抵信陽,賊又北竄,追至確山。陳國瑞等步隊亦到,令與全順、何建鼇、常星阿、成保數路合擊。郭寶昌設伏山口,僧格林沁登山督戰,諸悍賊齊集,合力死鬥。國瑞鏖戰最力,寶昌伏起沖突,賊大敗,尸橫遍野,由遂平、西平、郾城、許州、扶溝直走睢州。官軍追至,又奔入山東境,渡運河至寧陽,折向曲阜。

官軍馳追匝月,日行百里,往返三千餘里,馬力久疲。自蘇克金、舒通額、恆齡等歿後,得力戰將漸稀。朝命先調湘淮軍著名兵將,多觀望不至,僧格林沁亦不原用之。至是匪蹤剽忽,盤旋於兗、沂、曹、濟之間。由汶上竄鄆城水套,句結伏莽,眾至數萬。僧格林沁督師猛進,再戰再捷。至曹州北高莊,賊拒戰。軍分三路合擊,皆挫敗,退扎荒莊,遂被圍,兵不得食,夜半突圍亂戰,昏黑不辨行,至吳家店,從騎半沒。僧格林沁抽佩刀當賊,馬蹶遇害。時四月二十四日也。內閣學士全順、總兵何建鼇同殉於陣。

事聞,兩宮震悼,詔嘉其忠勇性成,視國事如家事,飾終典禮視親王,從優議恤。命侍衛馳驛迎柩至京,上奉兩宮皇太后親奠,賜金治喪,祀昭忠祠,於立功地方建專祠,配享太廟,謚曰忠,預繪像紫光閣。七年,捻平,遣官賜祭一壇。光緒十五年,皇太后歸政,敕於京師安定門內建專祠,祠曰顯忠。子伯彥訥謨祜襲親王爵,孫那爾蘇襲封貝勒,次孫溫都蘇封輔國公。

僧格林沁所部騎兵最號勁旅,驍將以舒通額、恆齡、蘇克金為最,均先殞。及從難,僅全順、何建鼇二人。兩次治海防,倚提督史榮椿、樂善,先後死事焉。其將勇營者,陳國瑞、郭寶昌最有名,並自有傳。

舒通額,蘇里氏,滿洲鑲白旗人,齊齊哈爾達呼爾。咸豐三年,以領催從軍江北,隸德興阿部下。攻江浦,矢殪黃衣執纛賊。迭著戰功,洊升協領,賜號圖薩泰巴圖魯。九年,僧格林沁督師天津,調充馬隊營總。十年冬,從赴山東剿捻匪,捻首趙浩然犯濟寧,舒通額敗之羊山。十一年春,戰於渮澤李家莊,分三路進擊,不利。舒通額將右翼,獨殺賊多,全師而退,擢充翼長。敗賊於泰安、寧陽,解滕縣圍。捻竄豐、沛,阻於水,復折而西,分竄鉅野,合長槍會匪,甚張,舒通額破之,斬馘數千。擊會匪郭秉鈞、劉占考於城武柳林集,復破賊徐官莊。偕協領色爾固善敗捻匪於郯城紅花埠、馬陵山,擒賊首李燦漳於曹州安陵集。復破郭秉鈞田潭老巢,追剿捻匪於青、沂之間。累功記名副都統,加頭品頂戴,賜黃馬褂。敗捻匪劉天祥於滕縣岡山,敗會匪劉占考於範縣,又破劉天祥於曹州袁家園。

同治元年,授阿勒楚喀副都統,從剿商丘金樓寨教匪,克之。偕恆齡平亳州張大莊捻巢,偕蘇克金敗捻魁張洛行於張橋。二年,捻匪劉狗、孫丑犯鹿邑,復與蘇克金要擊於魏橋。破尹家溝、雉河集賊巢,張洛行就擒。六月,捻首張守義陷淄川,他軍戰不利。舒通額突擊之,沖賊為四。守義棄城遁入鳳凰山白蓮池寨,與李成、宋繼明、劉雙印合,眾二萬餘,負嵎抗拒。舒通額攻其北,奪西寨門、棗園諸隘,總兵陳國瑞由東南登山,縱火焚之。繼明自殺,餘賊奔潰,舒通額覆諸山下,俘斬數千,擢正黃旗漢軍都統。從剿苗沛霖,平之。三年,粵、捻諸匪合擾豫、皖、楚三省間。八月,追至羅山,賊退蕭家河。舒通額躡其後,悍黨四面至,援軍阻絕,騎兵不得馳騁。舒通額下馬持短刀搏斗,突圍不出,遂戰死,優恤,予騎都尉兼雲騎尉世職,謚威毅。

恆齡,郭貝爾氏,滿洲鑲黃旗人,呼倫貝爾達呼爾。咸豐九年,以佐領從提督傅振邦剿捻匪,破賊於夏邑李家窪,勇常冠軍,擢協領。十年,振邦遣率兵千五百人入衛京畿。尋從巡撫文煜折回山東剿捻,解濟寧圍,遂從僧格林沁充營總。十一年,迭敗賊於東昌、青州、沂州,積功記名副都統,賜黃馬褂、達春巴圖魯名號。是年冬,會匪劉占考竄範縣,副都統舒明阿戰死,恆齡突擊走之。援賊至,賊返斗,恆齡與舒通額夾擊,追至簸箕營。舒通額攻其圩,恆齡逐逸賊至範縣西,斬千餘級。同治元年,敗長槍會匪於曹州楊家集,殲焦桂昌。侍郎國瑞攻亳州邢大莊不下,恆齡夜襲克之。二年,偕舒通額破捻匪於鹿邑魏橋,偕侍衛卓明阿敗賊於杞縣許岡,圍其寨。賊三路來援,偕蘇克金、卓明阿分擊,斬馘二千,又追敗之於博望驛。賊走山東,恆齡回援,大戰於鉅野大義渠。賊翻山遁,偕國瑞逐北,殲五千人。駐軍永城,撫定亳北諸圩寨。偕舒通額、蘇克金毀渦河南北捻巢,躡追至肥河北,張洛行就擒,伏誅。時降捻張錫珠、宋景詩復叛,擾畿南。恆齡偕蘇克金率馬隊馳援,署直隸提督。擊散張錫珠黨眾,進剿宋景詩於堂邑。三路合擊,景詩遁走,畿輔解嚴。從僧格林沁剿苗沛霖,奏充翼長。會諸軍克蔡圩,沛霖就殲。三年,從剿粵、捻諸匪於河南、湖北邊境,破賊於隨州,授正黃旗護軍統領。迭戰麻城、羅山間,賊北趨,恆齡與何建鼇等敗之張八橋。四年三月,追賊抵魯山城下,賊潮至,恆齡將右翼,與常星阿、成保合蹙賊。賊逾沙河走,恆齡追之,反斗,伏起,殞於陣。予騎都尉兼雲騎尉世職,謚壯烈。

蘇克金,倭勒氏,滿洲正黃旗人,愛琿駐防。咸豐初,以驍騎校從僧格林沁剿粵匪,克連鎮、馮官屯,積功擢佐領。五年,從都統西凌阿剿賊湖北,克德安。七年,從副都統德楞額剿潁上捻匪,轉戰河南,肅清河、陜、汝三郡,擢協領,加副都統銜。八年,阜陽教匪王廷楨擾洛陽、新蔡,蘇克金破西爐賊巢,斃王廷楨於陣。會德楞額疾,代領所部,追賊寨河集、陳家阪,盡殲之,賜號伊固木圖巴圖魯。邀擊捻匪於夏邑、寧陵,走之。尋又自亳州竄入河南境,敗之鄧六莊。坐赴援周家口失期,革職留營。尋破賊虞城,復原官。九年,克睢州。十年,僧格林沁調充天津行營翼長,遂從剿捻山東。十一年,從攻紅川口,殲賊渠劉占考、梁繼海,賜黃馬褂,記名副都統。

同治元年,從剿張洛行於河南杞縣、尉氏,屢敗之。攻金樓寨教匪,先登,斬郝姚氏及其二子,授福州副都統。二年,偕舒通額敗捻匪於鹿邑魏橋,破尹家溝賊巢,擒捻首韓四萬、陳二坎,蒙、亳悉平,加頭品頂戴。偕恆齡赴援畿輔,駐防河間。時河北多伏莽,鄉團跋扈。蘇克金謂疆吏姑息所致,言於僧格林沁,劾之。從剿苗沛霖,克淮南北各圩寨。餘捻走河南,張總愚最狡悍。三年,僧格林沁督師進剿,令蘇克金先驅扼魯山。賊畏大軍馬隊,盤旋山地。蘇克金在諸將中號持重,善審地勢,持數月未戰。詔屢促之,會張總愚出鄧州,急起追擊,連破之赤眉城、雙橋、安春寨,總愚負傷遁。而粵匪陳得才、藍成春等由漢中回竄,麕集麻城,蘇克金偕皖、豫諸軍進攻,力戰兼旬,毀賊壘數十。七月,戰於紅石堰,蘇克金指揮列陣,忽中暍,疾作,墜馬,舁歸遽卒。詔依都統例賜恤,謚壯介。

何建鼇,漢軍鑲紅旗人。由武舉補京營把總,初從達洪阿赴廣西剿匪,繼從僧格林沁戰阜城、連鎮、馮官屯,積功擢守備,回京營供職。咸豐七年,調赴河南,從剿角子山捻匪、阜陽教匪,洊升游擊。九年,調守天津大沽,擊退英國兵艦,加副將銜。及從剿捻匪,轉戰山東、江北,以破曹州紅川口會匪,擢副將。殲亳州捻首李廷彥,記名總兵。平張洛行,賜號雄勇巴圖魯,授中營副將。歷從剿捻豫、楚之交,常為軍鋒。曹州之敗,兵分三路,建鼇當其西,中路失利,賊萃於建鼇,士卒多死,退從僧格林沁守空堡,短刀殺賊,歿於陣。詔嘉其至死不離主帥,依提督例優恤,予騎都尉兼雲騎尉世職,謚果毅。

全順,薩爾圖拉氏,蒙古正藍旗人。咸豐六年,繙譯進士,歷官中允。十年,僧格林沁治防天津,疏調從軍,累遷翰林院侍讀學士。在軍充翼長,從剿商丘金樓寨、亳州邢大莊,及平張洛行,並著戰績,賜黃馬褂。擢內閣學士,授西安左翼副都統。從僧格林沁陣亡,恤典加等,依尚書例,予騎都尉兼雲騎尉世職,謚忠壯。舒通額、恆齡、蘇克金、何建鼇、全順並附祀僧格林沁專祠。

史榮椿,順天大興人。由行伍洊升京營參將,歷從揚威將軍奕經、大學士賽尚阿軍中。繼從都統勝保剿粵匪,攻獨流賊壘,戰阜城,破賊堆村,賜號洽希巴圖魯。僧格林沁薦其堪膺專閫,咸豐五年,擢大名鎮總兵。洎近畿軍事平,都統西凌阿率師移剿湖北,留馬隊千五百人隸榮椿防畿輔。尋赴援河南、安徽,連破捻匪於鹿邑、歸德。調徐州鎮,破捻匪於沁治天津海防。九年,英國兵艦犯海口,榮椿偕大沽協副將龍汝元力戰,中砲,同歿於陣。予騎都尉兼雲騎尉世職,建專祠,謚忠壯。

樂善,伊勒忒氏,蒙古正白旗人。由拜唐阿洊升雲麾使。揀發陜甘參將,剿番匪有功。從勝保剿粵匪,戰獨流、阜城,賜號巴克敦巴圖魯。咸豐六年,率馬隊剿捻匪河南,連破賊於鹿邑、潁川。七年,擢河北鎮總兵。克方家集捻巢,從勝保克正陽關,解固始圍,賜黃馬褂。九年,命赴僧格林沁天津軍營,擢直隸提督。英兵闖入海口,樂善扼擊,敵不得逞,尋退去。論功最,被優敘。十年七月,英兵復至,大沽砲臺陷,樂善力戰,死之。贈太子少保,予騎都尉兼雲騎尉世職,於海口建專祠,謚威毅。尋封二等男爵,子成友襲。

論曰:僧格林沁忠勇樸誠,出於天性,名震寰宇,朝廷倚為長城。治軍公廉無私,部曲誠服,勞而不怨。其殄寇也,惟以殺敵致果,無畏難趨避之心。剿捻凡五年,掃穴擒渠,餘孽遂為流寇,困獸之斗,勢更棘焉。繼事者變通戰略,以持重蕆功,則僧格林沁所未暇計及者也。然燕、齊、皖、豫之間,謳思久而不沫,於以見功德入人之深。有清籓部建大勛者,惟僧格林沁及策凌二人,同膺侑廟曠典,後先輝映,旂常增色矣。


\end{pinyinscope}