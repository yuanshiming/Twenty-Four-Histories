\article{列傳二百九十二}

\begin{pinyinscope}
藝術四

王來咸褚士寶馮行貞甘鳳池曹竹齋潘佩言

江之桐梁九張漣葉陶劉源唐英戴梓

丁守存徐壽子建寅華封

王來咸,字征南,浙江鄞縣人。先世居奉化,自祖父居鄞,至來咸徙同,從同里單思南受內家拳法。內家者,起於宋武當道士張三峰,其法以靜制動,應手即僕,與少林之主於搏人者異,故別少林為外家。其後流傳於秦、晉間,至明中葉,王宗岳為最著,溫州、陳州同受之,遂流傳於溫州。嘉靖間,張松溪最著,松溪之徒三四人,寧波葉繼美為魁,遂流傳於寧波。得繼美之傳者,曰吳昆山、周雲泉、陳貞石、孫繼槎及思南,各有授受。思南從征關白,歸老於家,以術教,頗惜其精微。來咸從樓上穴板窺之,得其梗概。以銀卮易美檟奉思南,始盡以不傳者傳之。

來咸為人機警,不露圭角,非遇甚困不發。凡搏人皆以其穴,死穴、暈穴、啞穴,一切如銅人圖法。有惡少侮之,為所擊,數日不溺,謝過,乃得如故。牧童竊學其法,擊伴侶,立死。視之,曰:「此暈穴。」不久果甦。任俠,嘗為人報仇,有致金以讎其弟者,絕之,曰:「此以禽獸待我也!」明末,嘗入伍為把總,從錢肅樂起兵浙東,事敗,隱居於家。慕其藝者,多通殷勤,皆不顧。鋤地擔糞,安於食貧。未嘗讀書,與士大夫談論蘊藉,不知為粗人。黃宗羲與之游,同入天童,僧少焰有膂力,四五人不能掣其手,稍近來咸,蹶然負痛。來咸嘗曰:「今人以內家無可炫耀,於是以外家羼之,此學行衰矣!」因為宗羲論述其學源流。康熙八年,卒,年五十三。宗羲子百家從之學,演其說為內家拳法一卷,百家後無所傳焉。

清中葉,河北有太極拳,云其法出於山西王宗岳,其法式論解,與百家之言相出入。至清末,傳習者頗眾云。

褚士寶,字復生,江南上海人。家素封,膂力過人,好技擊,游學四方。與畢昆陽、武君卿為友,遂精槍法,名曰四平槍,旋轉如風,人莫能近。同邑有張擎者,虎頸板肋,力舉百鈞,橫行為閭里患,眾請士寶除之。同飲酒,擎自言誇其勇,酒酣,攘臂作勢,士寶徐以箸點其胸,曰:「子盍坐而言乎?」擎遂默然,少頃辭去,越日,死於橋亭。明季福王南渡,兵部員外郎何剛薦士寶為伏波營游擊。未之官,南都陷,終老於家。所傳弟子有王聖蕃、池天榮。天榮又傳浙江提督喬照。其槍譜二種及治傷藥酒方,世猶有藏之者。

馮行貞,字服之,江南常熟人。父班,以文學者。兄行賢,傳其學。行貞少亦喜讀書,工小詞,性倜儻不羈。善射,能以後矢落前矢,投石子於百步外無不中。實雞卵殼以礦灰,遇劇盜,輒先發雞卵中其目。山東響馬老瓜賊為行旅患,聞馮氏名,莫敢攖。從休寧程打虎及張老受槍法,馳突無敵。山行遇虎,以短槍斃之。嘗為客報仇。康熙中,從康親王傑書軍南征,有功,當得官,尋棄歸。僑居吳中婁門外村落,以經書教授,詩畫自娛。年七十餘,卒。以槍法授同縣陶元淳,元淳後無傳者。

甘鳳池,江南江寧人。少以勇聞。康熙中,客京師貴邸。力士張大義者慕其名,自濟南來見。酒酣,命與鳳池角,鳳池辭,固強之。大義身長八尺餘,脛力強大,以鐵裹拇,騰躍若風雨之驟至。鳳池卻立倚柱,俟其來,承以手,大義大呼僕,血滿鞾,解視,拇盡嵌鐵中。即墨馬玉麟,長軀大腹,以帛約身,緣墻升木,捷於猱。客揚州巨賈家,鳳池後至,居其上。玉麟不平,與角技,終日無勝負。鳳池曰:「此勁敵,非張大義比!」明日又角,數蹈其瑕,玉麟直前擒鳳池,以駢指卻之,玉麟僕地,慚遁。鳳池嘗語人曰:「吾力不逾中人,所以能勝人者,善借其力以制之耳。」手能破堅,握鉛錫化為水。又善導引術,同里譚氏子病瘵,醫不效,鳳池於靜室窒牖戶,夜與合背坐,四十九日而痊。

喜任俠,接人和易,見者不知為賁、育。雍正中,浙江總督李衛捕治江寧顧云如邪術不軌獄,株連百數十人,鳳池亦被逮,讞擬大闢。世宗於此獄從寬,未盡駢誅。或云鳳池年八十餘,終於家。江湖間流傳其佚事多荒誕,著其可信者。

曹竹齋,以字行,佚其名,福建人。老而貧,賣卜揚州市。江、淮間健者,莫能當其一拳,故稱曹一拳。少年以重幣請其術,不可。或怪之,則曰:「此皆無賴子,豈當授藝以助虐哉?拳棒,古先舞蹈之遺也,君子習之,所以調血脈,養壽命,其粗乃以禦侮。必彼侮而我御之,若以之侮人,則反為人所御而自敗矣。無賴子以血氣事侵凌,其氣浮於上,而立腳虛,故因其奔赴之勢,略藉手而僕耳。一身止兩拳,拳之大才數寸,焉足衛五尺之軀,且以接四面乎?惟養吾正氣,使周於吾身,彼之手足近吾身,而吾之拳,即在其所近之處。以彼虛囂之氣,與吾靜定之氣接,則自無幸矣。故至精是術者,其徵有二:一則精神貫注,而腹背皆乾滑如臘肉;一則氣體健舉,而額顱皆肥澤如粉粢。是皆血脈流行,應乎自然,內充實而外和平,犯而不校者也。」嘉慶末,歿於揚州,年八十餘。

潘佩言,亦以字行,安徽歙縣人。以槍法著稱,稱潘五先生。其言:「槍長九尺,而桿圓四五寸,然槍入手,則全身悉委於桿。故必以小腹貼桿,使主運;後手必盡錞,以虎口實擫之;前手必直,令盡勢。以其掌根與後手虎口反正擰絞,而虛指使主導。兩足亦左虛右實,進退相任以趨勢。使槍尖、前手尖、前足尖、鼻尖五尖相對,而五尺之身,自託廕於數寸之桿,遮閉周匝,敵仗無從入犯矣。其用,有戳、有打;其法,曰二、曰叉。二以取人,叉以拒人。此叉則彼二,此二則彼叉。叉二循環,兩槍尖交如繞指,分寸間,出入百合,不得令相附。桿一附,則有僕者,故曰『千金難買一聲響』。手同則爭目,目同則爭氣。氣之運也,久暫稍殊,而勝敗分焉。故其術為至靜。」「吾授徒百數,而莫能傳吾術。吾之術,受於師者才十之三,其十之七,則授徒時被其非法相取之勢迫而得之於無意者也。是故名師易求,佳徒難訪。佳徒意在得師,以天下之大,求之無不如意者。至名師求徒,雖遇高資妙質,足以授道,而非其志之所存,不能耐勞苦以要之永久,則百貢而百見卻矣。」

佩言與竹齋同時處揚州,後歸歙,不知所終。

江之桐,字蘭崖,安徽和州人。年十餘歲,傭於江寧賣餅家,嗜讀書,其主人異之。招至家,居之樓上數年,讀左傳、國語、戰國策、史記、漢書、三國志畢。乃謝主人去,自設小肆於市。更習武藝,手臂刀矛,皆務實用,變通成法。且讀書,且習藝,讀稍倦,則趫舉翕張,以作其氣。已而默坐,以凝其神,晝夜無間。至百日乃睡,睡十餘日,復如之。讀史善疑,質之儒生,往往無以答。其藝通綿長、俞刀、程棓、瓘嵋十八棍,多取洪門,敵硬鬥強,以急疾為用。復及陣圖、形勢、器械,皆有理解。

年六十餘,始遇荊溪周濟。濟故績學,自負經世之略,通武藝,好談兵。與語大悅,延教其孫,三年而之桐卒。濟之言曰:「兵事至危,非得練士能臨敵苦鬥歷三十刻,及選鋒一可當三者,雖上有致果之志,下有死長之心,遇強敵不能必克。以力為本,以技濟之,謂之練士;作其勇者,謂之選鋒。世之便騎射、習火器,以為士卒程,事取捷速,恆不能持久。洎乎接刃,則霍然而去。故曰『巧不勝拙』。若之桐,庶為知務。」

梁九,順天人。自明末至清初,大內興造匠作,皆九董其役。初,明時京師有工師馮巧者,董造宮殿,至崇禎間老矣。九往執業門下,數載,終不得其傳,而服事左右,不懈益恭。一日九獨侍,巧顧曰:「子可教矣!」於是盡授其奧。巧死,九遂隸籍工部,代執營造之事。康熙三十四年,重建太和殿,九手制木殿一區,以寸準尺,以尺準丈,大不逾數尺許,四阿重室,規模悉其,工作以之為準,無爽。

張漣,字南垣,浙江秀水人,本籍江南華亭。少學畫,謁董其昌,通其法,用以疊石堆土為假山。謂世之聚危石作洞壑者,氣象蹙促,由於不通畫理。故漣所作,平岡小阪,陵阜陂紘,錯之以石,就其奔注起伏之勢,多得畫意,而石取易致,隨地材足,點綴飛動,變化無窮。為之既久,土石草樹,咸識其性情,各得其用。創手之始,亂石林立,躊躕四顧,默識在心。高坐與客談笑,但呼役夫,某樹下某石置某處,不假斧鑿而合。及成,結構天然,奇正罔不入妙。以其術游江以南數十年,大家名園,多出其手。東至越,北至燕,多慕其名來請者,四子皆衣食其業。晚歲,大學士馮銓聘赴京師,以老辭,遣其仲子往。康熙中,卒。後京師亦傳其法,有稱山石張者,世業百餘年未替。吳偉業、黃宗羲並為漣作傳,宗羲謂其「移山水畫法為石工,比元劉元之塑人物像,同為絕技」云。

葉陶,字金城,江南青浦人,本籍新安。善畫山水,康熙中,祇候內廷。奉敕作暢春園圖本稱旨,即命佐監造,園成,賜金馳驛歸。尋復召,卒於途。

劉源,字伴阮,河南祥符人,隸漢軍旗籍。康熙中,官刑部主事,供奉內廷,監督蕪湖、九江兩關,技巧絕倫。少工畫,曾繪唐凌煙閣功臣像,鐫刻行世,吳偉業贈詩紀之。及在內廷,於殿壁畫竹,風枝雨葉,極生動之致,為時所稱。手制清煙墨,在「寥天一」、「青麟髓」之上。於一笏上刻滕王閣序、心經,字畫嶄然。奉敕制太皇太后及皇貴妃寶範,撥蠟精絕。時江西景德鎮開御窯,源呈赩樣數百種。參古今之式,運以新意,備諸巧妙。於彩繪人物山水花鳥,尤各極其勝。及成,其精美過於明代諸窯。其他御用木漆器物,亦多出監作,聖祖甚眷遇之。及卒,無子,命官奠茶酒,侍衛護柩,馳驛歸葬,恩禮特異焉。

唐英,字俊公,漢軍旗人。官內務府員外郎,直養心殿。雍正六年,命監江西景德鎮窯務,歷監粵海關、淮安關。乾隆初,調九江關,復監督窯務,先後在事十餘年。明以中官督造,後改巡道,督府佐司其事,清初因之。順治中,巡撫郎廷佐所督造,精美有名,世稱「郎窯」。其後御窯興工,每命工部或內務府司官往,專任其事。年希堯曾奉使造器甚夥,世稱「年窯」。

英繼其後,任事最久,講求陶法,於泥土、釉料、坯胎、火候,具有心得,躬自指揮。又能恤工慎帑,撰陶成紀事碑,備載經費、工匠解額,臚列諸色赩釉,仿古採今,凡五十七種。自宋大觀,明永樂、宣德、成化、嘉靖、萬歷諸官窯,及哥窯、定窯、均窯、龍泉窯、宜興窯、西洋、東洋諸器,皆有仿制。其釉色,有白粉青、大綠、米色、玫瑰紫、海棠紅、茄花紫、梅子青、騾肝、馬肺、天藍、霽紅、霽青、鱔魚黃、蛇皮綠、油綠、歐紅、歐藍、月白、翡翠、烏金、紫金諸種。又有澆黃、澆紫、澆綠、填白、描金、青花、水墨、五彩、錐花、拱花、抹金、抹銀諸名。

奉敕編陶冶圖,為圖二十:曰採石制泥,曰淘煉泥土,曰煉灰配釉,曰制造匣缽,曰圓器修模,曰圓器拉坯,曰琢器做坯,曰採取青料,曰煉選青料,曰印坯乳料,曰圓器青花,曰制畫琢器,曰蘸釉吹釉,曰金旋坯挖足,曰成坯入窯,曰燒坯開窯,曰圓琢洋採,曰明爐暗爐,曰束草裝桶,曰祀神酬原。各附詳說,備著工作次第,後之治陶政者取法焉。英所造者,世稱「唐窯」。

戴梓,字文開,浙江錢塘人。少有機悟,自制火器,能擊百步外。康熙初,耿精忠叛,犯浙江,康親王傑書南征,梓以布衣從軍,獻連珠火銃法。下江山有功,授道員劄付。師還,聖祖召見,知其能文,試春日早朝詩,稱旨,授翰林院侍講。偕高士奇入直南書房,尋改直養心殿。梓通天文算法,預纂修律呂正義,與南懷仁及諸西洋人論不合,咸忌之。陳弘勛者,張獻忠養子,投誠得官,向梓索詐,互毆構訟。忌者中以蜚語,褫職,徙關東。後赦還家,留於鐵嶺,遂隸籍。

所造連珠銃,形如琵琶,火藥鉛丸,皆貯於銃脊,以機輪開閉。其機有二,相銜如牝牡,扳一機則火藥鉛丸自落筒中,第二機隨之並動,石激火出而銃發,凡二十八發乃重貯。法與西洋機關槍合,當時未通用,器藏於家,乾隆中猶存。西洋人貢蟠腸鳥槍,梓奉命仿造,以十槍賚其使臣。又奉命造子母砲,母送子出墜而碎裂,如西洋炸砲,聖祖率諸臣親臨視之,錫名為「威遠將軍」,鐫制者職名於砲後。親征噶爾丹,用以破敵。

丁守存,字心齋,山東日照人。道光十五年進士,授戶部主事,充軍機章京。守存通天文、歷算、風角、壬遁之術,善制器。時英吉利兵犯沿海數省,船砲之利,為中國所未有。守存慨然講求制造,西學猶未通行,凡所謂力學、化學、光學、重學,皆無專書,覃思每與闇合。大學士卓秉恬薦之,命繕進圖說,偕郎中文康、徐有壬赴天津,監造地雷、火機等器,試之皆驗。

咸豐初,從大學士賽尚阿赴廣西參軍事,會獲賊黨胡以暘,使招降其兄以晄,守存制一

匣曰手捧雷,偽若緘書其中,俾以晄致之賊酋,酋啟匣炸首死。尋檻送賊渠洪大全還京,遷員外郎。

從尚書孫瑞珍赴山東治沂州團防,造石雷、石砲以禦賊。尋調直隸襄辦團練,上戰守十六策。十年,回山東,創議築堡日照要塞,曰濤雒。賊大舉來犯,發石砲,聲震山谷,賊闢易,相戒無犯。丁家堡附近之民歸之,數年遂成都聚。

同治初,復至直隸,留治廣平防務,築堡二百餘所。軍事竣,授湖北督糧道,署按察使。充鄉試監試,創法,以竹筒引江水注闈中,時以為便。瀕江諸省,率仿行之。尋罷歸。所著書曰丙丁秘籥,進御不傳於外;所傳者曰造化究原,曰新火器說。

徐壽,字雪村,江蘇無錫人。生於僻鄉,幼孤,事母以孝聞。性質直無華。道、咸間,東南兵事起,遂棄舉業,專研博物格致之學。時泰西學術流傳中國者,尚未昌明,試驗諸器絕鮮。壽與金匱華蘅芳討論搜求,始得十一,苦心研索,每以意求之,而得其真。嘗購三棱玻璃不可得,磨水晶印章成三角形,驗得光分七色。知槍彈之行拋物線,疑其仰攻俯擊有異,設遠近多靶以測之,其成學之艱類此。久之,於西學具窺見原委,尤精制器。咸豐十一年,從大學士曾國籓軍,先後於安慶、江寧設機器局,皆預其事。

壽與蘅芳及吳嘉廉、龔蕓棠試造木質輪船,推求動理,測算汽機,蘅芳之力為多;造器罝機,皆出壽手制,不假西人,數年而成。長五十餘尺,每一時能行四十餘里,名之曰黃鵠。國籓激賞之,招入幕府,以奇才異能薦。既而設制造局於上海,百事草創,壽於船砲槍彈,多所發明。自制強水棉花藥、汞爆藥。

創議繙譯西書,以求制造根本。於是聘西士偉力亞利、傅蘭雅、林樂知、金楷理等,壽與同志華蘅芳、李鳳苞、王德均、趙元益孳孳研究,先後成書數百種。壽所譯述者,曰西藝知新及續編,化學鑒原及續編、補編,化學考質,化學求數,物體遇熱改易說,汽機發軔,營陣揭要,測地繪圖,寶藏興焉。法律、醫學,刊行者凡十三種,西藝知新、化學鑒原二書,尤稱善本。

同治末,與傅蘭雅設格致書院於上海,風氣漸開,成就甚眾,壽名益播。山東、四川仿設機器局,爭延聘壽主其事,以譯書事尤急,皆謝不往,而使其子建寅、華封代行。大冶煤鐵礦、開平煤礦、漠河金礦經始之際,壽皆為擘畫規制。購器選匠,資其力焉。無錫產桑宜蠶,西商購繭奪民利,壽考求烘繭法,倡設烘灶,及機器繅絲法,育蠶者利驟增。

壽狷介,不求仕進,以布衣終。光緒中,卒,年六十七。子建寅、華封,皆世其學。

建寅,字仲虎。從父於江寧、上海,助任制造。尋充山東機器局總辦,福建船政提調,出使德國二等參贊,洊擢直隸候補道。光緒末,張之洞調至湖北監造無煙火藥,已成,藥炸裂,殞焉,賜優恤。

華封,字祝三。性敏,為父所愛,秘說精器多授之,以制造為治生。建寅、華封並從父譯書行於世。


\end{pinyinscope}