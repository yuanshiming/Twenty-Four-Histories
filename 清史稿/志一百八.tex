\article{志一百八}

\begin{pinyinscope}
○兵四

△鄉兵

鄉兵始自雍、乾,旋募旋散,初非經制之師。嘉慶間,平川、楚教匪,鄉兵之功始著。道光之季,粵西寇起,各省舉辦團練,有駐守地方者,有隨營征剿者。侍郎曾國籓以衡、湘團練討寇,練鄉兵為勇營,以兵制部勒之,卒平巨憝,其始皆鄉兵也。而邊徼之地,剿有鄉兵。其在東三省者,則寧古塔以東之赫哲部、克雅克部,混同江東北之鄂倫春部,不設佐領,惟設鄉兵姓長。其在黑龍江者,有打牲人,在江以南之錫伯、卦勒察,江以北之索倫、達瑚爾,則附屬於滿營。在蒙古者,蒙兵而外,有奇古民勇。在山、陜邊外者,有番兵,有僧俗兵。在四川、雲南、貴州邊境者,有夷兵,有土司兵,有黑惈勇丁。在西藏者,有藏番兵。皆與內地鄉兵不同,故不詳。其各直省之鄉兵,曰屯練,曰民壯,曰鄉團,曰獵戶,曰漁團,曰沙民。額數之多寡不齊,器械之良窳不一,餉章之增減不定,良以聚散無恆,故與額兵迥異,無編制之可紀。茲特志其始末於後焉。

雍正八年,鄂爾泰平西南夷烏蒙之亂,調官兵萬餘人,鄉兵半之,遂定東川,是為鄉兵之始。

乾隆三十八年,用兵小金川,定邊將軍溫福、定西將軍阿桂疏言,調滿洲兵道遠費重,不如多用鄉兵,人地相宜。四川鄉兵,以金川屯練為強,自平定金川以後,設屯練鄉兵,其糧餉倍於額兵,分屯大小金川兩路,春夏訓練,秋冬蒐獵,有戰事則搜剿山路,退兵則為殿後之用。

嘉慶初,苗疆事起,傅鼐以鄉兵平苗,功冠諸將。詔以鼐總理邊務,令各省督撫以鼐練鄉兵之法練官兵。川、楚教匪之役,官兵征討,而鄉兵之功為多。其勛績最著者,文臣則四川按察使劉清,武臣則四川提督桂涵,湖北提督羅思舉,各統鄉兵,分路剿寇,大小數百戰,遂奏膚功。嘉慶十七年,以雲南邊外野夷惈匪肆擾,而緬寧、騰越各隘皆瘴癘之地,難駐官兵,乃練鄉兵一千六百人,以八百人駐守緬寧之丙野山梁等處,以八百人駐守騰越蠻章山等處,省官兵徵調之勞。其時苗疆底定,亦增設鄉兵,凡屯丁七千人,訓練之暇,開墾屯防田數十萬頃。

道光二年,令直隸疆臣招集團練,修築土堡,互為策應。十五年,令各州縣額設民壯,一律充補訓練,復令各省民壯每月隨營操演,範以紀律。是年,調大小金川鄉兵千名,給以千人之糧,隨營征戰,歸屯則仍食五百人之糧。二十一年,令山東巡撫於蓬萊、黃縣、榮成、寧海、掖縣、膠州、即墨所屬之十三梟,編練鄉兵,互相防衛。又令沿海疆臣仿浙江定海縣土堡之法,凡近海村落,招募鄉兵,興築土堡,以聯聲勢。二十三年,令廣東省以團練助防海口。旋疆吏疏言廣東民風宜於鄉團,招集已得十萬人,以升平學社為團練總匯之區,推及韶州、廉州等處,一律舉行。二十六年,令各州縣民壯隨營考察技藝。是年,甘肅沿邊番賊肆擾,令疆臣召募獵戶千人,編為一軍,供遠探近防之用。及道光季年,張國樑募廣東潮州鄉兵追逐粵寇,轉戰東下,卒以獷悍不馴,遂至潰散。

咸豐二年,令在籍侍郎曾國籓辦理湖南鄉團。旋國籓疏言先行練勇一千人,所辦者乃官勇非團丁,是為鄉團改勇營之始。三年,令山東登、萊、青三府舉辦聯莊團練,給以兵械。四年,令甘肅沿邊增募獵戶三千人以防番騎。八年,安徽巡撫翁同書疏言皖省定遠、壽州、合肥等縣舉辦團練,旬日之間,遠近響應,和州踞賊屢出焚掠,多被鄉團擊回,以其深明大義,踴躍同仇,凡董事團總人等,傳諭嘉勉。九年,河南巡撫恆福疏言,皖寇進偪豫境,令道府大員於接近皖寇之地,舉辦鄉團,睢州等州縣興築堡寨已數十處。旋諭河南官紳訓練鄉勇,聯村築寨,迅速舉行。

十年,諭勝保等督辦鄉團,以資統率,酌定章程,凡辦團州縣一律遵行,惟鄉團更番調營,所領糧餉,易滋流弊,毋得冒濫。又諭:「江蘇等省在籍紳士,除已經辦理團練外,其明曉大義律身公正者,自不乏人,所有在京直隸、江蘇、安徽、浙江、河南等省之大小官員,將如何舉行鄉團,隨同官兵剿賊,及防守等一切事宜,各陳所見,各舉所知,迅即上聞。」

尋侍郎沈兆霖疏陳:「自咸豐三年以後,迭奉朝旨舉行鄉團,已至再至三,各省官紳士民,未嘗不遵旨辦理,而賊勢披猖,卒無成效。良由茍且塗飾,未經實力講求,或募勇以充數,徒取外觀,或藉端以營私,轉成欲壑,無事則恃為威脅,擾害鄉閭,有警則首先遁逃,流為盜賊。議者幾謂鄉團之無益而有損矣。不知名為民團,即當以民為團,而不可以募勇塞責也。民統於紳,則紳之邪正宜慎擇也。紳倚於官,則官之賢否宜嚴辨也。不歸並於一路,則督察無人,必不能一律堅固。不專力於四鄉,則城守雖嚴,已難免四面受敵。官與紳宜兩相孚,不宜兩相厄。兵與民宜兩相顧,不宜兩相仇。任封疆者,當知民本吾民,用兵數少,何如用民數多。任將帥者,當知兵本吾民,我能救民,自然民能救我。現在賊氛猖獗,非實辦民團,更無安全之法。」乃擬上事宜十二條:「一、民團須招本地有業之民,不可招市井無賴也。一、宜分別地段,以近賊一、二百里為最要,距賊稍遠,中隔一、二縣者為次要,其遠在三、四百里外者,則從緩辦團也。一、各州縣要地,宜一律辦團,無使一處疏漏,俾寇得乘隙而入也。一、辦團宜四鄉加密,有警則互相應援,無事則嚴詰奸宄,庶城守完固也。一、牧令宜擇賢能,與辦團之紳,不得各存意見,亦不得任用劣紳也。一、宜簡道府大員分路辦團,俾各縣聯為一氣也。一、民團有急,官兵速往救援,不得觀望也。一、宜擇要設卡盤查也。一、民團祗可就地助戰,不宜調遣,變為練勇,失其恆業也。一、立功宜即獎勵,視官兵稍優也。一、團費宜自捐自辦,不得藉端漁利也。一、民團辦成,則分防之兵可省,集合成軍,攻剿更為得力也。」

同時應詔陳言者,有載垣等所議團練章程十條,賈楨等所擬辦理章程八條。旋命順天府府丞毛昶熙為督辦河南團練大臣,南汝光道鄭元善幫辦團練事宜,按照怡親王載垣等所擬章程辦理。命戶部右侍郎杜為督辦山東團練大臣,登萊青道貢璜、登州府知府盧朝安幫辦團練事宜,按照大學士賈楨等所擬條款,並參酌河南章程,體察情形辦理。又以皖南地方緊要,應一律辦團,令兩江總督曾國籓察看情形,擇其諳練軍務素有人望者,酌保一人,即令督辦皖南團練事宜。

旋曾國籓覆陳:「鄉團本是良法,然奉行不善,縣官徒借以斂費,局紳亦從而分肥,賊至則先行潰逃,賊退則重加苛派,轉為地方之弊。所經過各省,從未見有鄉團能專打一股、專克一城者,不過隨官兵之後,勝則貪財,敗則先奔,常藉口於工食之太少。而辦理歧異者,每多給錢文,團丁所領之餉,與官勇例價相同,且有過之。其取之民間,無非勸捐抽釐之類。是於團練已失其本義,於軍餉又大有妨礙。今奉諭舉行皖南團練,皖南嶺隘紛歧,若築碉設卡,有險可憑,徽、寧各要隘,宜擇地築碉,以資防守。有在籍翰林院編修宋夢蘭當賊由太平縣竄擾徽州,宋夢蘭督帶練丁協力嚴守,眾論翕然。請即以該員辦理皖南團練事宜,會同委員,董勸各屬紳民,興築碉塞。其未經克復者,官兵攻剿,概不令團丁隨往。其已經克復者,紳耆修碉,團丁守之。庶幾軍民兩利,名實相符矣。」

又以四川地屬巖疆,毗連雲、貴,滇匪滋擾,未能肅清。嘉慶間,四川舉辦鄉團,行堅壁清野之法,著有成效,自應仿辦。所有應行事宜,諭四川在京各員,就地方情形,各抒所見。官紳中有練達時務者,各舉所知,以俟後命。同時尚書陳孚恩等以江西毗連安徽、浙江、廣東等省,疏請辦理團練,酌保辦事人員,並擬團練事宜八條。疏入,允行。命在籍翰林院修撰劉繹為江西督辦團練大臣,吉南贛寧道沈葆楨、甘肅安肅道劉於潯幫辦團練事宜,按照陳孚恩等所擬章程,妥為辦理。

同時督辦河南團練大臣、順天府府丞毛昶熙疏陳團練辦法,並酌擬規條:一、添築堡寨以扼要隘,一、講求險要以便堵御,一、慎擇首事以資統率,一、分選團丁以備訓練,一、攤派練費以備公用,一、互為聲援以資聯絡,一、申明號令以壹眾志,一、嚴定約束以禁頑暴,一、秉公賞罰以示勸懲,一、嚴察奸宄以防內應,一、旌表忠義以作民氣,一、事貴實力以冀成功。疏入,允行。

毛昶熙又疏陳河南團練,以歸、陳二府為先。前統兵大臣勝保,因調團不齊,勒派百姓出資雇丁,統計勇糧運費,較正供多至倍蓰,百姓苦累,紛紛稟請,以抽丁一項,民力已竭,鄉團勢難再辦。其開封等府百姓聞歸、陳雇勇之苦,亦復觀望,不肯實辦。團練之事,仍恐有名無實。尋奉諭:「用民之法,總宜深得民心。勝保等所辦章程既與民心不洽,自應改弦更張,以期得力。慶廉即體察情形,將此項雇勇酌量裁撤。毛昶熙按照載垣等會議章程,速即集團練勇,以輔兵力。」又以甘肅控制西陲,地方遼闊,且與陜西、四川毗連,匪患未靖,自應一律舉辦團練,以靖邊陲。所有甘肅省團練事宜,即命陜甘總督樂斌督辦,並命甘涼道蕭浚蘭、刑部員外郎吳可讀、江西候補道楊升幫辦團練。

十一年,以歸化之番眾僧俗兵四千餘人,馬四千餘匹,防禦抱罕羌人。是年,奉諭:「鄉團之設,原以濟兵力所不逮。必須官紳一體,兵勇同心協力,內靖土匪,外禦賊氛,於地方庶有裨益。若如清盛疏劾山東章丘縣之水賽街、新城之南婁裡等莊,以及博山、萊蕪等縣鄉團,遇有經過客商,往來差役,輒敢擅行殺戮,害及無辜。撫署之差弁馬匹,亦被劫奪。是團練御賊尚無成效,而抗官滋事竟有滋蔓之勢。巡撫譚廷襄速將清盛所奏各情,嚴密查訪。如有藉團為名,肆行不法,及私立黑團,聚眾抗官,立即嚴懲。」又諭浙江巡撫:「前以浙省軍務未平,籌辦團練,勸諭捐輸,原以保衛民生。若如王履謙疏劾辦團情形,雜亂無章,勸捐委員,令捐戶加捐至數十倍之多,並於捐戶加以威逼。今賊氛逼近浙東,若因勸捐辦理不善,致失人心,必致激成內訌。巡撫王有齡速即會同王履謙妥為勸辦,議定章程,不得徇私委派貪劣之員。」

是年,左副都御史潘祖廕疏言:「各省設立團練大臣,辦理年餘,曾無一效,請獎請敘,紛紛效尤,並未克復一城,其為無益,已可概見。應將團練大臣分別裁撤,以一事機而節糜費。」翰林院侍講學士顏宗儀疏言:「鄉團之設,原以百姓之財力,衛百姓之身家,果能眾志成城,同仇敵愾,即一舉、貢、生、監,足以統領之,無俟大員為之督率。若必以大僚綜任之,幫辦司員分理之,是督撫之外又設督撫,僚屬之外又增僚屬,徒滋紛擾。上年豫省辦團,各省團練大臣亦紛紛四出。旋因浙江、四川、陜西、甘肅等省情形不同,旋即裁撤。而直隸、山東、江南、江北等處,則仍歸由團練大臣辦理。於是幫辦人員假公濟私,百端紛擾。或偪勒州縣供應,或苛派民間銀錢,或於官設捐局之外,團練再設捐局,或於官抽釐金之外,團練再抽釐金,或查閱各處團防,支應紛煩,地方告乏,或任令家人奴僕勒索規費,約束不嚴。幫辦人員或十餘人,或數十人,薪水所出,皆刻剝民間。刁生劣監,因以把持地方;狡吏貪夫,藉以希圖名利,流弊實多。各省團練大臣,直隸桑春榮操守尚嚴,山東杜已嘖有煩言,至於江北、江南所辦鄉團,自上年至今,未聞有團練大臣收復一州一縣者,徒以騷動天下,無益有損。今山東杜已經撤回,河南毛昶熙較有成效,其直隸、江南、江北等處團練大臣,宜一並撤回。其各省州縣距賊較遠者,停止辦團,以安民業。其距賊較近之處,仍責地方官切實辦團,而以本省督撫總其成,庶事權不至紛歧,商民可免滋擾。」

旋奉諭:「直隸團練大臣桑春榮回京供職,直隸團練事宜,責成文煜辦理。江西團練大臣劉繹來京任用,江西團練事宜,責成毓科督同官紳辦理。其二省京官如有回籍辦團者,各部院查取職名,飭令來京供職。江北團練大臣晏端書,江西團練大臣龐鍾璐,其辦理團練,是否仍須該員經理,抑或即可裁撤,令曾國籓、薛煥速議以聞。王履謙幫辦浙江團練,兼辦浙東捐務,今浙江軍務方殷,自難遽撤。令王有齡會同王履謙切實籌辦,以固疆圉。毛昶熙在河南歸德著有成效,應否仍令毛昶熙督辦團練,及有無把握之處,令嚴樹森速議以聞。」

旋兩江總督曾國籓覆陳:「團練之設,只能防小支千餘之游匪,不能剿大股數萬之悍賊。其練丁口糧,若太多,則與募勇之價相等,不必僅以團名;若太少,則與官勇之餉迥殊,不能得其死力。其團防經費,若取諸丁、漕、釐、捐四者之中,則有礙督撫籌款之途;若設法四者之外,則更無措手之處。事權既無專屬,剛柔實覺兩難。晏端書在江北不設餉局,但勸各邑築圩自保,龐鍾璐在江南激勸鄉民,俾知同仇敵愾之義,辦理極有斟酌。今之賊勢,決非鄉團所能奏功。應俟賊氛稍衰,大功將成,然後辦團練以善其後。晏端書、龐鍾璐二員,清操雅望,內任最宜。應請裁去團練差使,回京供職。」疏入,允之。

同治元年,諭:「鄉團之設,原以使民自衛身家,藉可保全地方,以輔官兵。前因各路辦理團練大臣隨帶多員,任意騷擾,有害無利,是以陸續裁撤,仍責令地方官切實經理。乃邇來統兵大員,守土牧令,或恐其分餉而輕為裁撤,或疑其無益而視為具文,於是民心不固,盜賊橫行,所過州縣村莊,動遭劫掠,是又地方官不能因地制宜舉行團練之所致,因噎廢食,貽誤殊多。嗣後各省團練,仍由督撫臣通飭各州縣,選公正紳士,實力興辦。務使官不掣肘,民悉同心,城市鄉村,聲勢聯絡。其有認真辦理保全地方者,將其實在勞績,聲明保獎。」

二年,以都察院代遞山東貢生硃德秀條陳團練事宜,語多可採,命硃德秀回籍,隨同英桂、趙德轍辦理團練,並命英桂督飭官紳,就地方情形,認真辦團,毋得有名無實。

是年,統兵大臣僧格林沁疏言:「各省練團築寨,本以助守望而御寇盜,為權宜補救之法。乃各團每以有寨可據,輒藐視官長,擅理詞訟,或聚眾抗糧,或挾仇械斗,甚至謀為不軌,踞城戕官,如山東之劉德培,河南之李瞻,先後倡亂,而安徽之苗沛霖,尤為梟桀反復,勞師糜餉,始得次第翦除。辦團之舉,始則合一鄉為一團,繼則聯眾團為一練,地廣人多,良莠不齊,不肖團長有跋扈情形,承辦團練紳士又不能杜漸防微,隨時舉發,致有尾大不掉之勢。況捻匪屢經竄擾之區,亦未見各團堵禦得力。其河南團練,均由侍郎毛昶熙管理。毛昶熙於通省地方,勢難周歷兼顧,而各練既有專管大員,地方官轉至呼應不靈。今賊氛漸平,請命毛昶熙回京供職,所有團練,視直隸、山東之例,歸地方官經理,以一事權。並請飭河南巡撫嚴查各團,如有增置軍械等事,均責令稟請地方官允準置備。如不肖團長借修圍制械,種種斂錢,以致苦累鄉民,即從嚴懲辦,庶幾權歸於上,免滋流弊。」御史裘德俊疏言:「團練之舉,本屬有治人無治法。今直隸善後章程,有抽撥鄉團訓練之議。但抽撥鄉兵,必得賢明牧令,駕馭有方,乃能權不下移,民無擾累。若遇不肖州縣,借端苛斂,抽丁派費,吏胥因緣為奸,上下咸思中飽,小民已不聊生;加以每縣聚眾數百人,游手無著,以強凌弱,甚或恃眾把持,一有亂萌,尤易響應,不可不遠慮及之。」

旋奉諭:「山東鄉團已由官為經理,所有河南省團練事宜,亦統歸官辦,以一事權。其直隸抽練團丁,督臣劉長佑權其利害,是否可行,如有窒礙之處,即據實以聞。」

六年,李云麟招募奇古民勇駐八里岡,與科布多、塔爾巴哈臺蒙兵為犄角。

七年,諭各疆臣:「捻寇蕩平,勇丁亦各還鄉里,誠恐江南、安徽、河南、山東從前被兵處所,不免伏莽潛匿,乘隙為害。江、皖等省督撫,於徐、海、潁、亳、歸、汝、曹、沂等處,飭各地方官勸諭民間照舊修理圩寨,整頓鄉團,互相保衛。此外各處民團,亦應一律整飭,慎選牧令,安良除暴,以靖地方。」

十二年,因四川峨邊蠻族投誠,擇充千百戶等職,編制夷兵,建修碉堡。

光緒六年,兩廣總督張之洞募沙民千人助守虎門,楊玉科增募千人及惠清營五百人,鄭紹忠募安勇二千人,所募鄉兵,以防勇規制編之。是年,命黑龍江將軍於增練馬隊外,秋冬之季,招集打牲人等,加以訓練。

八年,兩江總督左宗棠以江蘇沿江海州縣捕魚為業者甚多,於內江外海風濤沙線無不熟諳,而崇明尤為各海口漁戶爭趨之所。其中有技勇而悉洋務者,所在不乏。外洋船駛入內江者,每用漁戶為導。江蘇自川沙迄贛榆二十二州縣,濱臨江海,漁戶約數萬人。乃令蘇松太道員為沿海漁團督辦,於漁戶每百人中,選壯健三十人,練漁團五千名,設總局於吳淞口,設分局於濱海各縣,每月操練二次,習水勇技藝,用以捕盜緝私,兼備水師之選。

十一年,雲貴總督岑毓英釐定雲南通省營制,惈黑勇丁,編為六營,西南土防,編為二十五營。又因云南沿邊,由西而東南,皆野人山寨,布列於九隘之外,乃調兵二千人,與原有防軍及鄉團土司,協力警備。督辦廣東軍務大臣彭玉麟以欽州、廉州地廣兵單,招募鄉團協守。是年,吉林將軍增練防軍,佐以烏拉牲丁,凡萬五千人。

二十四年,都察院代陳湖南舉人何鎮圭條陳團練事宜,命兵部議奏。又諭:「侍郎張廕桓疏請實行團練,同時臣工屢有仿西法練民兵之請。若各省實行團練,即以鄉團為民兵,用更番替換之法,較諸遽練民兵為有把握。廣西會匪滋事,尤宜速辦,以收捍禦之功。各省督撫一律切實籌辦。各省於三月內,廣東、廣西於一月內,將辦理情形,具疏以聞。」

三十年,廣西巡撫柯逢時令廣西各州縣增募鄉勇八千人,給以毛瑟後膛槍,並令民間多築碉堡,共御外侮。

三十一年,兩廣總督李經羲增練防營,並募土著鄉兵,備廣西邊境。新疆巡撫潘效蘇以新疆兵費太重,改募土著,仿勇營訓練,次第遣散客軍。

三十四年,雲南防軍裁並,於騰越、臨安兩路創設團練,藉資捍衛。

宣統元年,各省改防營為巡防隊。雲貴總督沈秉堃以雲南防軍內有各屬之保衛團,系昔日之鄉團,名為營隊,實即鄉兵,未能遽改為巡防隊,乃仍其舊。此鄉兵舉廢之概略也。


\end{pinyinscope}