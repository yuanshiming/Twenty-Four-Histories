\article{列傳二百五十六}

\begin{pinyinscope}
恩銘孚琦鳳山端方弟端錦劉燧赫成額

松壽趙爾豐馮汝騤陸鍾琦子光熙等

恩銘,字新甫,於庫里氏,滿洲鑲白旗人,錦州駐防。以舉人納貲為知縣,累官至知府。光緒十一年,權知兗州,晉道員。二十一年,改官山西。二十六年,署按察使。拳匪擾晉,恩銘請巡撫毓賢陰護送教士出境,弗聽。兩宮西幸,毓賢率師赴固關,恩銘兼攝撫、籓事。車駕至太原,召見,奏對,聲淚俱下。補歸綏道。先是口外七殺教士四十餘、教民二千餘,待撫者眾且亟,到官後,即發帑金倉粟濟之。會聯軍至大同,民駭走。復令教士諷喻,並與執爭,乃引兵去。

二十八年,調直隸口北道。時經拳亂後,十三、州、縣教民洶洶圖報復,宣化華教士且強逼民入教,恩銘患之,與西教士反覆辨論,始允約束,民、教始安。遷浙江鹽運使。二十九年,調兩淮,晉江蘇按察使。辦鹽務如故,杜私販,恤煎丁,歲增國課三十萬。時論欲請改場垣為公司,並創煤煎輪運議,恩銘力陳其弊,事乃寢。授布政使,錄山西協餉功,晉頭品服。三十二年,署安徽巡撫,修廣濟圩,賑皖北水菑,民德之。紅蓮會匪自贛入,毀建德教堂,同時楚民寄居霍山者,亦與教堂啟釁,匪黨乘之,勢漸熾。恩銘分軍援剿,並劾有司之釀禍者,地方以靖。

是時廷議行新政,銳意興警察,於是承上指,整頓巡警學堂。適王之春薦道員徐錫麟才,遂畀以會辦。復念政劇財匱,援例清丈緣江洲地,按年收科,墾牧與樹藝並舉。朝旨又以民刑事訴訟法參用東、西律,下其議督撫。恩銘慮皖北民悍,為擇其不便者六事具以報。明年夏,巡警學生卒業,恩銘詣校試驗,錫麟乘間以槍擊之,被重創。知縣陸永頤銳身救護,先殞。錫麟令經歷顧松閉校門,不從,亦斃之。從者負恩銘還署,遂卒。事聞,贈太子少保,謚忠愍,予皖省建祠,賞騎都尉兼一雲騎尉世職,子咸麟襲。恩銘既死,錫麟亦被獲。

錫麟者,浙江山陰人。就學日本,以貲為道員。志在謀綰軍隊,便起事,倉卒發難,卒被擒僇。閱數年,復有孚琦、鳳山被刺事。

孚琦,字樸孫,西林覺羅氏,隸滿洲正藍旗。以工部筆帖式充軍機章京,累官郎中。三遷至內閣學士。光緒二十八年,授刑部右侍郎。三十二年,出為廣州副都統。頗以興學為己任,嘗設八旗工藝學校,整麗中小各學堂。明年,權將軍。將軍事故簡,孚琦慮即偷惰,日必讀書臨池,暇輒躬執勞役。宣統二年,再攝將軍篆。明年春,赴城東燕塘勘旗地,兼閱試演軍用飛機。有溫生才者,隸革命黨,事暗殺。會日將暮,伏道左,俟其至,轟擊之,遂殞命。生才被執,論棄市。事聞,上憫惻,謚恪愍,命鳳山代之。

鳳山,字禹門,劉氏,隸漢軍鑲白旗。以繙譯舉人襲佐領,充驍騎營翼長、印務章京。累遷參領,總辦東安巡捕分局。聯軍入京,法人在其轄境刃傷商民,縛致總局,請毋少貸,論如律。擢副都統,訓練近畿陸軍,著聲績。除西安將軍,仍留治兵事。宣統初,改練軍歸部節度,始解兵柄。三年,授廣州將軍,未行而武昌事起。香港為粵民黨藪,謀攻省城,眾阻其勿往,曰:「吾大臣也,不可不奉詔。」遂毅然去。將至時,總督及布、按以下官皆不敢出迓,或勸宜微服先入城,毋蹈孚將軍覆轍,鳳山不可。日午,輿衛導行,抵南城外,黨人匿市廛簷際擲炸彈,屋瓦摧壓,從者死十餘人,街石寸寸裂。暮得鳳山尸,焦爛無完膚。事聞,贈太子少保,謚勤節,予騎都尉世職。

端方,字午橋,托忒克氏,滿洲正白旗人。由廕生中舉人,入貲為員外郎,遷郎中。光緒二十四年,出為直隸霸昌道。京師創設農工商局,徵還,筦局務,賞三品卿銜。上勸善歌,稱旨。除陜西按察使,晉布政使,護巡撫。兩宮西幸,迎駕設行在。調河南布政使,擢湖北巡撫。二十八年,攝湖廣總督。三十年,調江蘇,攝兩江總督。尋調湖南。顓志興學,資遣出洋學生甚眾。逾歲,召入覲。擢閩浙總督,未之官,詔赴東西各國考政治。既還,成歐美政治要義,獻上,議改立憲自此始。三十二年,移督兩江,設學堂,辦警察,造兵艦,練陸軍,定長江巡緝章程,聲聞益著。

宣統改元,調直隸。孝欽皇后梓宮奉安,端方輿從橫沖神路,農工商部左丞李國傑劾之,坐違制免。既而御史胡思敬又彈其貪橫凡十罪,事下張人駿,覆奏入,以不治崖檢被訶斥,因已罷官,貸勿問。

三年,命以侍郎督辦川漢、粵漢鐵路。時部議路歸國有,而收路章條湘、川不一致,川人大譁。川、鄂為黨人所萃,乘機竊發。端方行次漢口,亟入川,並劾川督趙爾豐操切。命率師往按,尋詔代攝其事。所過州縣,輒召父老宣喻威德。至資州,所部鄂軍皆變,軍官劉怡鳳率眾入室,語不遜,端方以不屈遇害。

端方性通侻,不拘小節。篤嗜金石書畫,尤好客,建節江、鄂,燕集無虛日,一時文採幾上希畢、阮云。

弟端錦,字叔絅。河南知府。赴東西各國考路政,箸日本鐵道紀要。從兄入川,變作,以身蔽其兄,極口詈軍士無良,同被殺。事聞,贈端方太子太保,謚忠敏;端錦謚忠惠。

其時轉餉官劉燧,荊州駐防、舉人、都司赫成額,並赴水死。

松壽,字鶴齡,滿洲正白旗人。以廕生官工部筆帖式,累遷郎中。出為陜西督糧道。光緒二十一年,晉山東按察使。明年,調江西,晉江寧布政使。二十四年,擢江西巡撫。越三載,移撫江蘇,歷河南,加尚書銜,所蒞皆稱職。二十八年,召為工部右侍郎,兼正藍旗蒙古副都統,尋授熱河都統。疏陳續修礦章四條,允行。復以地控蒙部,號難治,條上吏治、軍政、興學、理財方略甚悉。又召還,拜兵部尚書。明年,調工部。又明年,出為察哈爾都統。三十三年,授閩浙總督。

居官垂二十年,不務赫赫名,然律己以廉,臨下以寬,為時論所美。宣統三年秋,鄂、湘、江、浙新軍踵變,閩軍乘之,將舉事,使人要松壽,令繳駐防營軍械,斥之,遂決戰,初獲勝,繼乃大挫,憤甚,飲金以殉。事聞,贈太子少保,予二等輕車都尉世職,謚忠節。

趙爾豐,字季和,漢軍正藍旗人。以山西知縣累保道員。四川總督錫良疏薦其才,權永寧道,剿匪嚴誅捕。駐藏大臣鳳全遇害,調建昌。會克巴塘,建議籌邊,充川滇邊務大臣,護總督,改駐藏大臣。以兵至打箭爐,改設康定、登科等府。宣統元年,仍專任邊務。藏兵犯巴塘,擊敗之,乘勢收江卡等四部。於是爾豐軍越丹達山而西,直抵江達,達賴喇嘛逃入印度。爾豐請一舉平藏,革教易俗,廷意不欲開釁,阻之。爾豐盡克三崖野番,決收回瞻對。三年,署四川總督,檄番官獻瞻對。爾豐遂入瞻對,設官治之。進克波密,並取白馬崗,收明正等土司,皆改流。計所收邊地縱橫三四千里,設治者三十餘區,一時皆懾於兵力,不敢抗。

會川亂起,爾豐還省,集司道聯名奏請變更收路辦法,不允。商民罷市,全省騷動。廷寄飭拏禍首,捕蒲殿俊等拘之,其黨圍攻省城。督辦川路大臣端方劾爾豐操切,詔仍回邊務大臣,以岑春煊代總督。武昌變作,資政院議爾豐罷黜待罪,而朝旨已不能達川。重慶兵變,會匪蜂起,軍民環請獨立,爾豐遽讓政權於殿俊,殿俊自稱都督。防軍復變,殿俊走匿,全城無主。商民請爾豐出定亂,因揭示撫輯變兵。而標統尹昌衡率部入城,自為都督,羅綸副之,以兵攻督署,擁爾豐至貢院,爾豐罵不絕口,遂被害。[一]

馮汝騤,字星巖,河南祥符人。光緒九年進士,選庶吉士,散館授戶部主事,充軍機章京,累遷郎中。出知四川順慶府,遭母憂去。服闋,起山東青州知府,調直隸大名。三十一年,遷湖北鹽法道。明年,調安徽徽寧池太道,遷甘肅按察使。未幾,晉陜西布政使,擢浙江巡撫。三十四年,移撫江西,整稅務,省不急,官稱治辦。朝議方厲行新政,乃復察民情,量財力,從容施設,士民安之。宣統元年,御史江春霖上其溺職徇私狀,事下安徽巡撫硃家寶覈覆,得白。坐疏忽幹吏議,奪俸三月。

三年,武昌變起,下游皆震。南昌軍相應和,脅汝騤為都督,號獨立,峻拒之。贛人故感其賢,導之出。至九江,乃仰藥以殉。詔旨軫惜,謚忠愍。

陸鍾琦,字申甫,順天宛平人,本籍浙江蕭山。父春榮,績學不遇,祭酒盛昱其弟子也。鍾琦少劬學,以孝稱。光緒十五年進士,以編修辦直隸賑災,徐桐亟賞之。拳禍起,桐惑焉,鍾琦持異議,弗聽。聯軍入,同年王懿榮、熙元、寶豐輩先後皆殉節。鍾琦聞之,泣,闔戶自經,遇救獲免。二十九年,除江蘇督糧道。越五載,遷江西按察使,調湖南,察吏嚴,定州縣結案功過章條,月計勘案數與其鞫訊狀限期報司,繇是獄鮮積滯。再移江蘇,多平反。

宣統改元,晉布政使。三年,擢山西巡撫。到官未逾月,而武昌難作。鍾琦語次子敬熙曰:「大事不可為矣!省垣倘不測,吾誓死職。汝曹讀書明大義,屆期毋效婦仁害我!」又曰:「生死之事,父子不相強,任汝曹自為之。但吾孫毋使同盡,以斬宗祀。」敬熙知父意決,入告母。母曰:「汝父殉國,吾惟從之而已。」敬熙以事亟,赴京語其兄光熙,偕還晉。鍾琦馭新軍嚴,至是調兩營赴南路,時九月七日也。夜發餉,將以翼日行,而遲明變作,新軍突入撫署。鍾琦出堂皇,僕李慶雲從,麾之弗去,且挺身出,先被戕。鍾琦叱曰:「爾輩將反邪?」語未竟,遽中槍而殞。光熙奔救,亦被擊死。叛軍入內室,其妻唐氏抱雛孫起,並遇害。詔褒其忠孝節義萃於一門,予謚文烈。妻唐旌表。

光熙,本名惠熙,字亮臣。少從盛昱游,勵學。鍾琦遘危疾,嘗刲股和藥以進。光緒三十年,成進士,選庶吉士。東渡日本學陸軍,卒業歸,授編修,擢侍講。贈三品京堂,謚文節。

論曰:恩銘遇刺,實在辛亥之前,蓋亂機已久兆矣。武昌變起,各行省大吏惴惴自危,皆罔知所措。其死封疆者,唯松壽、鍾琦等數人,或慷慨捐軀,或從容就義,示天下以大節,垂絕綱常,庶幾恃以復振焉。

[一]按;趙爾豐傳,關內本與關外一次本相同,較此為詳。全文附錄於後,作為參考。

趙爾豐,字季和,漢軍正藍旗人。父文穎,見忠義傳。爾豐以鹽大使改知縣,選山西靜樂,歷永濟。清獄治盜,匪絕跡。躬捕蝗,始免災。擢河東監掣同知,護河東道,以憂去。光緒二十六年,聯軍入晉邊,山西巡撫錫良檄總營務處嚴防密偵圓咄酥N記ê擁雷芏劍魑庸ぃ郾5澇保創又寥群印N級醬ǎ杓銎洳牛ㄓ濫饋J被岱宋跡要莧渭辭壯鱍步耍舶嗽腦攏錁薹稅兮湃耍袷及慘怠?br>三十一年,駐藏大臣鳳全被害於巴塘,錫良以爾豐為建昌道,會提督馬維騏往討。維騏軍先發,爾豐從之,遂克巴塘。爾豐接辦善後,移兵討鄉城,匪退喇嘛寺,據碉死守。爾豐斷水道,圍攻,番眾悉降。於是爾豐建籌邊議,錫良以聞,加爾豐侍郎,充川滇邊務大臣。爾豐會錫良暨雲貴總督丁振鐸奏陳改流設官、練兵、招墾、開礦、修路、通商、興學諸端,廷議準撥開邊費銀百萬兩。三十三年,錫良移任去,爾豐護四川總督。於是遙策邊事,凡前所奏陳,皆以次舉,察吏尤嚴,多所舉劾,僚屬肅然。川南邊地多匪,移興文縣於建武,移永寧縣於古藺。時外人議輪運入川,爾豐令川商自辦淺水輪以阻之,是為川江駛輪之始。

三十四年,以爾豐兄爾巽督川,改爾豐駐藏大臣,仍兼邊務,專邊藏事。爾豐以經營全藏,宜以殖民為主,特慮恩信未孚,藏人疑阻,請仍責駐藏大臣聯豫駐守,而自巡視邊藏。先以巴塘為根據,寓遷民於兵墾,漸及藏地。又與爾巽會奏,設安康道,改打箭爐為康定府,設河口縣、里化同知、稻成縣、貢噶嶺縣丞,巴安府三壩通判,定鄉縣,鹽井縣。詔促爾豐出關,因就成都駐防旗兵中選練西軍三營自隨。藏人聞之,聚兵三崖以阻。爾豐至打箭爐,適德格土司爭襲構亂,乃請旨往辦,迭敗之贈科、麻木,追奔至卡納沙漠地,眾悉降。爾豐分其地為五區,設登科府德化、白玉兩州,石渠、普同兩縣,置邊北道。德格地大,包有春科、高日兩土司,遂與靈蔥土司之郎吉嶺等地並改歸流。宣統元年,朝意務懷柔藏人,採爾巽議,以經營西藏責聯豫暨幫辦溫宗堯,改爾豐專任邊務,駐巴塘,為藏聲援,劃察木多、乍丫歸邊轄。

川軍協統鍾穎率新軍三千入藏,被困察木多。爾豐聞報,立馳往援,鍾穎軍出,並驅剿類伍齊、碩般多、洛隆宗、邊壩各部落逆番殆盡,三十九族波密、八宿等部咸納款。而江卡藏兵忽抄邊軍後路,犯巴塘,爾豐分兵擊敗之,乘勢收江卡、貢覺、桑昂、雜瑜四部落。於是爾豐軍越丹達山而西,直抵江達,距藏都拉薩僅六日程矣。二年,達賴喇嘛聞川軍將至,逃入英屬印度。爾豐請乘勝一舉平藏,革教易俗,廷意不欲開釁,阻之。爾豐上疏力爭,略言:「我國幅瀼遼闊,強鄰環伺,屬地多有侵占。自革達賴喇嘛,阿旺郎結叛逆,不惟藏人搖動,即外人覬覦之心亦因而愈熾。今我兵雖已入藏,然阿旺郎結已入英手,英人必挾以圖藏。若再姑容,將成大患。臣因一面由巴塘進兵攻破南墩,一面由察木多進兵貢覺、桑昂、曲宗,我兵所到,番人親附,即洛隆宗、碩板多等亦皆遠來輸誠,備陳藏中苛虐情形,堅墾內屬。臣初意務在保境息民,並無開疆拓土之念。唯桑昂、曲宗屬地雜瑜與惈儸野番接壤,時有英人潛伏。惈儸之南,為阿撒密,西為波密。英人若得雜瑜,即可直接波密,由工布入藏,與印度聯成一片。則波密不可不收入版圖,其勢至迫。請及此將邊兵所到之地,概收歸邊。並函商聯豫以烏蘇里江以東隸邊,以西屬藏。」疏入,樞府以外交責言為慮,聯豫亦不允劃界。然邊軍所得江達以內地,爾豐已逐漸改流,早成轄境矣。

爾豐巡視各地,經貢覺、乍丫、江卡三部落,群以討三崖為請。三崖者野番也,地險人悍,三部落苦其侵掠,嘗合攻之,反為所敗,官軍久不能討。爾豐策三崖四周皆已改流,必為我用,遂派知府傅嵩矞率兵五路進攻,苦戰兩月,盡克上中下三崖全境,設官治之。初,藏人占瞻對,爾豐屢請收回,廷議責聯豫議贖,久不得要領。至是邊地略定,獨瞻對為藏有,梗塞其中,爾豐乃決以策取之。三年,爾豐調署四川總督,因薦嵩矞以道員用,代理邊務大臣,同行閱邊,繞道北路,先至孔撒、麻書,設甘孜委員,靈蔥、白利、倬倭、東科、單東、魚科各土司繳印改流,並受色達及上羅科野番降,瞻對民皆聞風請附。爾豐乃檄番官曰:「瞻對原系川屬,朝廷前以賞藏,設官徵糧。光緒二十年,瞻人叛藏,則藏已失瞻;川兵取瞻,則瞻為川有。乃藏人久占不歸,迄今又十餘年矣,厚斂橫徵,民不堪命。應將瞻對仍獻朝廷,以表恭順。」藏官畏爾豐威,獻戶籍去。瞻對民歡呼出迎,爾豐遂入瞻對,設官治之。野番俄落、色達均望風降。又波密自言其先為入藏漢兵,別成部落。爾豐前至察木多,波密呈驗所產棉布、糧食,證明確由漢出,並述其地與白馬崗接壤,在英、藏間,力請內附。及爾豐師還,聯豫忽遣兵攻之,大敗乞援。至是,爾豐派鳳山由巴塘率邊兵二千往與聯豫參贊羅長崟軍共克波密,並取白馬崗。爾豐至打箭爐,收明正土司地及魚通、冷邊、沈邊、咱裡等土司印,皆改流。計爾豐所收邊地,東西三千餘里,南北四千餘里,設治者三十餘區,詳土司傳。

會川亂起,爾豐還省。初,商辦川漢鐵路公司集股銀二千餘萬,忽奉旨收歸國有,咸大譁,倡保路同志會,好事者爭附和,勢張甚。爾豐至成都,察亂已成,思弭解,集司道聯名電奏,請變更收路辦法,不允。商民罷市,同志會捧德宗神牌沖入督署,與護兵相持,頗有死傷,全省騷動。廷寄飭拏禍首正法,爾豐不得已捕會首蒲殿俊等九人拘之。其黨圍攻省城,兵皆川產,不用命。督辦川漢鐵路大臣端方方奉命援川,滯重慶,劾爾豐操切,詔仍回邊務大臣,以岑春煊代為總督。武昌變作,春煊阻不得往,端方至資州,遇害。資政院劾爾豐,罷黜待罪,而朝旨已不能達川。重慶兵變,會匪蜂起,軍民環請獨立,爾豐遽讓政權於殿俊,殿俊自稱都督,防軍復變,殿俊走匿,全城無主,商民請爾豐出定亂,因揭示撫輯變兵。而標統尹昌衡率部入城,自為都督,羅綸副之,以兵攻督署,擁爾豐至貢院,爾豐罵不絕口,遂被害。


\end{pinyinscope}