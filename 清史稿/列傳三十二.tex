\article{列傳三十二}

\begin{pinyinscope}
剛林祁充格馮銓孫之獬李若琳陳名夏陳之遴劉正宗張縉彥

剛林,瓜爾佳氏,字公茂,滿洲正黃旗人,世居蘇完。初來歸,隸正藍旗,屬郡王阿達禮。授筆帖式,掌繙譯漢文。天聰八年,以漢文應試,中式舉人,命直文館。崇德元年,授國史院大學士,與範文程、希福等參與政事。疏請重定部院承政以下官各五等,又疏請定試士之法,皆報可。太宗四征不庭,疆宇日闢。剛林屢奉使軍前,宣布威德,咸稱上旨。積功,授世職牛錄章京。八年,阿達禮有罪,改隸正黃旗。

世祖定鼎,進世職二等甲喇章京。三年、四年,迭主會試。考滿,進世職一等阿達哈哈番。五年,復進三等阿思哈尼哈番,賜號「巴克什」。六年,充太宗實錄總裁,復主會試。疏請令六科錄諸臣章奏並批答,月送史館,備纂修國史,報可。八年,以編撰明史闕天啟四年至七年實錄,請敕懸賞購求;崇禎一朝事𧾷責無考,其有野史、外傳,並令訪送。章下所司。

睿親王多爾袞薨,得罪。剛林阿附睿親王,參與移永平密謀,又與大學士祁充格擅改太祖實錄,為睿親王削匿罪愆、增載功績,坐斬,籍沒。

祁充格,烏蘇氏,滿洲鑲白旗人,世居瓦爾喀。國初從其族吉思哈等來歸。太宗時號「四貝勒」,以祁充格嫺習文史,令掌書記。天聰五年,初設六部,授禮部啟心郎。八年,考績,授牛錄額真。崇德元年,睿親王多爾袞伐明,攻錦州,命鞏阿岱往濟師,祁充格從師有功,還報捷。三年,睿親王復伐明,太宗親餞於郊。祁充格以不啟豫親王多鐸從上出送,又於是日私往屯莊,坐死,命寬之,奪官,貫耳鞭責,以隸睿親王。順治二年,授弘文院大學士,充明史總裁官、冊封朝鮮世子正使。四年,考滿,加授牛錄額真。六年,充太宗實錄總裁官,與剛林等同主會試。八年,與剛林同誅。

馮銓,字振鷺,順天涿州人。明萬歷進士,授檢討。諂事魏忠賢,累遷文淵閣大學士兼戶部尚書,加少保兼太子太保,以微忤罷去。莊烈帝既誅忠賢,得銓罷官後壽忠賢百韻詩,論杖徒,贖為民。

順治元年,睿親王既定京師,以書徵銓,銓聞命即至,賚冠服、鞍馬、銀幣。令以大學士原銜入內院佐理機務,與大學士洪承疇疏請復明票擬舊制,又與大學士謝升等議定郊社、宗廟樂章。十月朔,世祖御皇極門受賀,給事中孫承澤疏糾朝班雜亂,語侵內院。銓與升、承疇乞罷,諭令益殫忠猷,以襄新治。

二年,授弘文院大學士兼禮部尚書。御史吳達劾銓向降將姜瓖索銀三萬,許以封拜,未稱其意;內院政本所關,乃令其子源淮擅入,張宴歡飲。給事中許作梅、莊憲祖、杜立德,御史王守履、羅國士、鄧孕槐、桑蕓等亦交章劾銓得招撫侍郎江禹緒金;為源淮賄招撫侍郎孫之獬充標下中軍;禮部侍郎李若琳為銓黨羽,庸懦無行。御史李森先疏繼入,語尤峻,略謂:「明二百餘年國祚,壞於忠賢,而忠賢當日殺戮賢良,通賄謀逆,皆成於銓。此通國共知者。請立彰大法,戮之於市。」疏並下刑部鞫問,刑部以所劾不實,啟睿親王。王集廷臣覆讞,以銓降後與之獬、若琳皆先薙發,之獬家男婦並改滿裝,諸臣遂謀陷害。王謂三人者皆恪遵本朝法度,詰責科道諸臣。給事中龔鼎孳言銓附忠賢作惡,銓亦反詰鼎孳嘗降李自成。王問鼎孳:「銓語實否?」鼎孳曰:「豈惟鼎孳,魏徵亦嘗降唐太宗。」王因斥鼎孳,遂寢其事。以森先言過甚,奪官,互見森先傳。

三年正月,銓疏言:「臣蒙特召入內院,列同官舊臣之前,臣固辭不敢。攝政王面諭:『國家尊賢敬客,卿其勿讓!』今海宇漸平,制度略定。金臺駿骨,暫示招徠。久假不歸,實逾涯分。況叨承寵命,賜婚滿洲,理當附籍滿洲編氓之末。回繹尊賢敬客之諭,展轉悚懼,特懇改列範文程、剛林後。如以新舊為次,並當列祁充格、寧完我後。」得旨:「天下一統,滿、漢無分別。內院職掌等級,原有成規,不必再定。」是年命典會試,列範文程、剛林後,寧完我前。四年,復典會試。六年,加少傅兼太子太傅。

八年,上親覈諸大臣功績,諭:「銓先經吳達奏劾得叛將姜瓖賄,便當引去;乃隱忍居官,七年以來,無所建白:令致仕。李若琳憸險專擅,與銓朋比為奸,奪官,永不敘用。」銓既罷,代以陳名夏,坐事奪官;代以陳之遴,亦不久罷。上復召銓還,諭曰:「國家用人,使功不如使過。銓素有才學,博洽諳練,朕特召用,以觀自新。」銓至,召見,又與承疇、文程等同夕對論翰林官賢否,上曰:「朕將親試之!」銓奏曰:「南人優於文而行不符,北人短於文而行或善。今取文行兼優者用之可也。」上頷之,仍授弘文院大學士。以議總兵任珍罪坐欺飾論絞,上命寬之。銓入謝,奏對失旨,諭誡之。

龔鼎孳為左都御史,復劾銓,上命指實。鼎孳言銓罪過頗多,惟以密勿票擬,非如諸曹有實可指,上切責鼎孳。十二年,居母喪,命入直如故。尋加少師兼太子太師。十三年,上以銓衰老,加太保致仕,仍令在左右備顧問,銓疏請回籍,許之。十六年,改設內閣,命以原銜兼中和殿大學士。康熙十一年,卒,謚文敏。旋命削謚。

孫之獬,山東淄川人。明天啟進士,授檢討,遷侍讀。以爭毀三朝要典入逆案,削籍。順治元年,侍郎王鰲永招撫山東。土寇攻淄川,之獬斥家財守城。山東巡撫方大猷上其事,召詣京師,授禮部侍郎。二年,師克九江,之獬奏請往任招撫,從之,加兵部尚書銜以行。三年,召還。總兵金聲桓劾之獬擅加副將高進庫、劉一鵬總兵銜,市恩構釁;之獬議撫諸將懷觀望,不力攻贛州。之獬疏辨,下兵部議,奪之獬官。四年,土寇復攻淄川,之獬佐城守,城破,死之,諸孫從死者七人,下吏部議恤。侍郎陳名夏、金之俊議復之獬官,予恤;馬光輝及啟心郎寧古里議之獬已削籍,不當予恤。兩議上,命用光輝議。

李若琳,山東新城人。明天啟進士,授檢討。順治元年,起原官,累遷少詹事,兼國子監祭酒。詹事府裁,改翰林院侍讀學士,兼祭酒如故。二年,請更定孔子神牌,復元制曰大成至聖文宣王,下禮部議,定稱大成至聖先師。再遷禮部侍郎。五年,進尚書。六年,加太子太保。既罷歸,未幾卒。

陳名夏,字百史,江南溧陽人。明崇禎進士,官修撰,兼戶、兵二科都給事中。降李自成。福王時,入從賊案。順治二年,詣大名降。以保定巡撫王文奎薦,復原官。入謁睿親王,請正大位。王曰:「本朝自有家法,非爾所知也。」旋超擢吏部侍郎,兼翰林院侍讀學士。師定江南,九卿科道議南京設官。名夏言:「國家定鼎神京,居北制南。不當如前朝稱都會,設官如諸行省。」疏入稱旨。三年,居父喪,命奪情任事,請終制,賜白金五百,暫假歸葬,仍給俸贍其孥在京者。五年,初設六部漢尚書,授名夏吏部尚書,加太子太保。八年,授弘文院大學士,進少保,兼太子太保。

名夏任吏部時,滿尚書譚泰阿睿親王,擅權,名夏附之亂政。睿親王薨,是夏,御史張煊劾名夏結黨行私,銓選不公,下王大臣會鞫,譚泰袒名夏,坐煊誣奏,論死。語詳煊傳。是時御史盛復選亦以劾名夏坐黜。迨秋,譚泰以罪誅,九年春,復命王大臣按煊所劾名夏罪狀,名夏辨甚力。及屢見詰難,詞窮,泣訴投誠有功,冀貸死。上曰:「此展轉矯詐之小人也,罪實難逭!但朕已有旨,凡與譚泰事干連者,皆赦勿問。若復罪名夏,是為不信。」因宥之,命奪官,仍給俸,發正黃旗,與閒散官隨朝,諭令自新。

十年,復授秘書院大學士。吏部尚書員缺,侍郎孫承澤請以名夏兼攝,上責承澤以侍郎舉大學士,非體。翌日,命名夏署吏部尚書。上時幸內院,恆諭諸臣:「滿、漢一體,毋互結黨與。」名夏或強辭以對,上戒之曰:「爾勿怙過,自貽伊戚。」諸大臣議總兵任珍罪,皆以珍擅殺,其孥怨望,宜傅重比。名夏與陳之遴、金之俊等異議,坐欺蒙,論死,復寬之,但鐫秩俸,任事如故。

十一年,大學士寧完我劾之,略言:「名夏屢蒙赦宥,尚復包藏禍心。嘗謂臣曰:『留發復衣冠,天下即太平。』其情叵測。名夏子掖臣,居鄉暴惡,士民怨恨。移居江寧,占入官園宅,關通納賄,名夏明知故縱。名夏署吏部尚書,破格擢其私交趙延先,給事中郭一鶚疏及之,名夏欲加罪,以劉正宗不平而止。浙江道員史儒綱為名夏姻家,坐事奪官逮問,名夏必欲為之復官。給事中魏象樞與名夏姻家,有連坐事,應左遷,僅票俸。護黨市恩,於此可見。臣等職掌票擬,一字輕重,關系公私;立簿注姓,以防推諉。名夏私自塗抹一百十四字。上命誥誡科道官結黨,名夏擅加抹改,其欺罔類是。請敕大臣鞫實,法斷施行。」疏下廷臣會鞫,名夏辨諸款皆虛,惟「留發復衣冠」,實有其語。完我與正宗共證名夏諸罪狀皆實,讞成,論斬,上命改絞。掖臣逮治,杖戍。

陳之遴,字彥升,浙江海寧人。明崇禎進士,自編修遷中允。順治二年,來降,授秘書院侍讀學士。五年,遷禮部侍郎。六年,加右都御史。八年,擢禮部尚書。御史張煊劾大學士陳名夏,語涉之遴,鞫不實,免議,加太子太保,九年,授弘文院大學士。

時捕治京師巨猾李應試,王大臣會鞫,之遴默不語,王大臣詰之,之遴曰:「上立置應試於法則已,如或免死,則必受其害,是以不言。」王大臣等以聞,上以詰之遴,疏引罪。上以之遴既悔過,宥之。調戶部尚書。議總兵任珍罪,與名夏及金之俊持異議,坐罪,寬貸如名夏。十二年,奏請依律定滿臣有罪籍沒家產、降革世職之例,下所司議行。復授弘文院大學士,加少保兼太子太保。

十三年,上幸南苑,召諸大臣入對,諭之遴曰:「朕不念爾前罪,屢申誥誡,嘗以朕言告人乎?抑自思所行亦曾少改乎?」之遴奏曰:「上教臣,臣安敢不改?特臣才疏學淺,不能仰報上恩。」上曰:「朕非不知之遴等朋黨而用之,但欲資其才,故任以職。且時時教飭之者,亦冀其改過效忠耳。」因責左副都御史魏裔介等媕阿緘默,裔介退,具疏劾之遴植黨營私,當上詰問,但云「才疏學淺」,良心已昧;並言之遴諷禮部尚書胡世安舉知府沈令式,旋為總督李輝祖所劾,是為結黨之據。給事中王楨又劾之遴市權豪縱,昨蒙詰責,不思閉門省罪,即於次日遨游靈佑宮,逍遙恣肆,罪不容誅。之遴疏引罪,有云:「南北各親其親,各友其友。」上益不懌,下吏部嚴議,命以原官發盛京居住。是冬,復命回京入旗。十五年,復坐賄結內監吳良輔,鞫實,論斬,命奪官,籍其家,流徙尚陽堡,死徙所。

劉正宗,字可宗,山東安丘人。明崇禎進士,自推官授編修。福王時,授中允。順治二年,以薦起國史院編修。累遷秘書院學士。十四年,授吏部侍郎,擢弘文院大學士。吏部尚書缺員,諭以「正宗清正耿介,堪勝此任,加太子太保,管吏部尚書」。

御史楊義論部推越次,正宗與辨,執相詬詈。給事中周曾發,御史姜圖南、祖建明交章劾之。御史張嘉復以正宗昏庸衰老,背公徇私,疏請罷斥。下部議,以無實據,寢其事。給事中硃徽復劾正宗擅擬僉事許宸遷通政司參議,不由會推,又未專疏題明。正宗以疏忽引咎,當俸,援恩詔以免。旋引疾乞休,不允。辭尚書,命以兼銜回內院,加少保兼太子太保。十四年,考滿,進少傅兼太子太傅。十五年,改文華殿大學士。

十六年,上以正宗器量狹隘,終日務詩文,廷議輒以己意為是,降旨嚴飭,並諭曰:「朕委任大臣,期始終相成,以愜簡拔初念。故不忍加罪,時加申戒。當痛改前非,稱朕優容寬恕之意。」十七年,自陳乞罷,不允。左都御史魏裔介劾「正宗自陳奏內不敘上諭切責,無人臣禮。李昌祚叛案有名,票擬內升。先後薦董國祥、梁羽明,今皆事敗,被劾不自檢舉。欺君之罪何辭?正宗與張縉彥為友,縉彥序正宗詩曰『將明之才』,詭譎尤不可解。正宗弟正學,為鄭成功總兵,正宗囑巡撫耿焞躐升中軍。蠹國亂政,其事非一端。請乾斷以杜禍萌」。御史季振宜繼劾,亦及國祥、正學,並正宗貪賄營利諸事。正宗疏辨,略謂:「李昌祚為叛黨,裔介身為法司,何不早行糾參?例凡薦舉之官,在本任不職,追坐舉主。國祥、羽明皆升任後得罪。縉彥序臣詩有曰『將明之才』,臣詩稿見存,縉彥序未見此語。」疏入,上奪正宗官,下王大臣會鞫;亦責裔介、振宜不早糾參,並奪官待質。旋議上裔介、振宜劾正宗罪狀鞫實者十一事,罪當絞。上斥「正宗性質暴戾,器量褊淺,持論偏私,處事執謬。惟事沽名好勝,罔顧大體,罪戾滋甚。從寬免死,籍家產之半,入旗,不許回籍」。十八年,聖祖即位,以世祖遺詔及正宗罪狀,當置重典,愍其衰老,貸之。未幾病卒。

張縉彥,河南新鄭人。明崇禎進士,自知縣行取授主事。再授編修,擢兵部尚書。順治元年,詣固山額真葉臣軍前納款,福王授以總督,乃遁去。既,復受洪承疇招降。九年,以薦下吏部考核。十年,授山東右布政。十五年,擢工部侍郎。十七年,甄別三品以上大臣,降授江南徽寧道。裔介劾正宗,詞連縉彥,奪官逮訊。御史蕭震疏劾縉彥編刻無聲戲,自稱「不死英雄」,惑人心,害風俗。王大臣會鞫,論斬,上命貰死,籍其家,流徙寧古塔。尋死於戍所。

論曰:剛林相太宗,與範文程、希福並命,祁充格掌記室,於創業宜皆有功。銓故明相,諳故事,與名夏皆善占對。名夏勸進雖不用,以此邀峻擢。之遴、正宗各有所援引,知當時亦頗用事。際初運,都高位,而不足以堪之。誅夷削奪,曾莫之惜。正宗傾名夏,亦不免於罪,尤可鑒矣。


\end{pinyinscope}