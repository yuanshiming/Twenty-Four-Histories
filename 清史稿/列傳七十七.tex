\article{列傳七十七}

\begin{pinyinscope}
楊名時黃叔琳子登賢方苞王蘭生留保胡煦

魏廷珍任蘭枝蔡世遠沈近思雷鋐

楊名時,字賓實,江南江陰人。康熙三十年進士,改庶吉士。李光地為考官,深器之,從受經學。散館,授檢討。四十一年,督順天學政,用光地薦也。尋遷侍讀。四十二年,上西巡,肥鄉武生李正朝病狂,沖突儀仗。光地時為直隸巡撫,請罪正朝,因劾名時。上斥名時督學,有意棄富錄貧,不問學業文字,但不受賄囑,從寬恕宥。四十四年,任滿,命河工效力。旋連遭父母喪,以憂歸。五十一年,服除,候補。五十三年,命直南書房。名時不投牒吏部,因不得補官,上特命充陜西考官。五十六年,授直隸巡道。時沿明制,直隸不設兩司,以巡道任按察使事。政劇,吏為奸,名時革宿弊殆盡。五十八年,遷貴州布政使。

五十九年,擢雲南巡撫。師征西藏,留駐雲南,名時為營館舍,明約束,無敢叫囂。名時疏言:「雲南兵糧歲需十四萬九千餘石,俱就近支放。兵多米少,諸州縣例四年折徵一次,請改每年給本色三季,折色一季。」部議如所請行。雍正元年,名時奏請安,世宗諭曰:「爾向日居官有聲。茲當加勉,莫移初志。」尋疏言:「雲南巡撫一切規禮,臣一無所取。惟鹽規五萬二千兩,除留充恤灶、修井諸用,餘四萬六千兩。累年供應在藏官兵軍需賞賚,撥補銀廠缺課,及公私所用,皆取於此。藏兵撤後,請仍留臣署若干,餘悉充公用。」上諭曰:「督撫羨餘,豈可限以規則?取所當取,用所當用,全在爾等揆情度理而行,無煩章奏也。」名時迭疏請調劑鹽井,改行社倉,皆下部議行。雲南自亂後田賦淆亂,往往戶絕田去而丁未除,至有一人當數十丁者,累代相仍,名曰「子孫丁」。名時疏請照直隸例,將通省丁額攤入田糧完納。雲南舊例,地方應辦事,皆取諸民間,謂之「公件」。胥役科斂,指一派十,重為民累。名時議核實州縣需款,酌定數目徵收,不得再有加派。檄行所屬諸州縣,核數開報。

三年,擢兵部尚書,改授雲貴總督,仍管巡撫事。時上令諸督撫常事疏題,要事摺奏。名時洩密摺,上令悉用題本,名時乞遇事仍得摺奏,許之。四年,轉吏部尚書,仍以總督管巡撫。名時具題本,誤將密諭載入,上嚴責,命解任,以硃綱代為巡撫。未至,仍令名時暫署。俄,綱上官,劾名時在任七載,徇隱廢弛,庫帑倉穀,借欠虧空。上命名時自陳,綱代名時奏謝罪,上責其巧詐,諭總督鄂爾泰嚴訊。名時自承沽名邀譽,斷不敢巧詐。讞上,部議以名時始終掩護,朦朧引咎,無人臣事君禮,坐挾詐欺公,當斬。上命寬免,復遣侍郎黃炳會綱按治。炳等欲刑訊,鄂爾泰持不可,乃坐名時得鹽規八萬,除捐補銀廠缺課,應追五萬八千餘兩。上令名時留雲南待後命。

高宗即位,召詣京師。乾隆元年,名時至,賜禮部尚書銜,兼領國子監祭酒,兼直上書房、南書房。名時以前在雲南令諸州縣核實需款定數徵收,去公件之弊,事未竟而去,奏請下督撫勘定。總督尹繼善、巡撫張允隨奏請以額編條糧重輕,與原定公件多寡,兩相比並,就中攤減,下部議行。視未定議前取諸民者去十之七,雲南民困以蘇。

苗疆用兵久,名時疏言:「御夷之道,貴在羈縻,未有怨毒猜嫌而能長久寧貼者。貴州境內多與苗疆相接,生苗在南,漢人在北,而熟苗居中,受雇直為漢人傭,相安已久。生苗所居深山密箐,有熟苗為之限,常聲內地兵威以懾之,故亦罔敢窺伺。自議開拓苗疆,生苗界上常屯官兵,干戈相尋,而生苗始不安其所。至熟苗無事則供力役,用兵則為鄉導,軍民待之若奴隸,生苗疾之若寇仇。官兵勝,則生苗乘間抄殺以洩忿;官兵敗,又或屠戮以冒功。由是熟苗怨恨,反結生苗為亂。如臺拱本在化外,有司迎合要功,輒謂苗民獻地。上官不察,竟議駐師。遂使生苗煽亂,屢陷官兵,蹂躪內地;間有就撫熟苗,又為武臣殘殺,賣其妻女。是以賊志益堅,人懷必死。為今日計,惟有棄苗疆而不取,撤重兵還駐內地,要害築城,俾民有可依,兵有可守。來則御之,去則舍之。明懸賞格,有能擒首惡及率眾歸順者,給與土官世襲,分管其地。更加意撫綏熟苗,使勿為生苗所劫掠,官兵所侵陵,庶有俯首向化之日。不然,臣恐兵端不能遽息也。」二年,卒,贈太子太傅,賜祭葬,謚文定。

黃叔琳,字昆圃,順天大興縣人。康熙三十年一甲三名進士,授編修,累遷侍講。丁父憂,服除,起原官,遷鴻臚寺少卿。五遷刑部侍郎。雍正元年,調吏部。命偕兩淮鹽政謝賜履赴湖廣,與總督楊宗仁議鹽價,革除陋規,從所請。疏言:「各省支撥兵糧,布政使、糧道為政,先期請託,方撥近營。否則撥遠汛,加運費,民既重累輸輓,兵亦苦待餉。請敕督撫察兵數,先撥本州縣衛、所,不敷,於附近州縣撥運。」下部議行。旋授浙江巡撫。時御史錢廷獻請濬浙江東西湖,蓄水灌田,命叔琳會總督滿保勘議。叔琳等奏言:「西湖居會城西,周三十餘里,南北山泉入湖處,舊皆設閘以阻浮沙,水得暢流;又有東湖為之停蓄,湖水分出上下塘河,農田資以灌溉。自閘廢土淤,民占為田,築埂圍蕩,栽荷蓄魚。請照舊址清釐,去埂建閘,濬城內河道,並疏治上塘河各支港,及自會城至江南吳江界運河港汊壩堰。」部議從之。

叔琳疏薦人才,有廷臣嘗言於上者,上疑叔琳請託先容,諭戒鄭重。會有言叔琳赴湖廣時,得鹽商賕,俾充總商,及為巡撫,庇海寧陳氏僕;其弟御史叔敬巡視臺灣,過杭州,僕閧於市,叔琳皆以罪商,有死者,商為罷市。上命解叔琳任,遣侍郎李周望與將軍安泰分案按治。安泰等奏叔琳以陳氏僕與商爭毆,逮商杖斃,事實,無與叔璥事,亦未嘗罷市。周望等奏叔琳貸金鹽商,非行賄,上命毋窮究。三年,命赴海塘效力。

乾隆元年,授山東按察使。疏言:「舊例州縣命案,印官公出,由鄰封相驗。嗣廣西巡撫金鉷奏請改委佐雜,夤緣賄囑,難成信讞。」又言:「審案舊有定限,逾限議處。嗣河東總督田文鏡題定分立解府、州、司、院限期,雖意在清釐,適啟通融挪改之弊,請皆仍舊為便。」從之。二年,遷布政使。四年,丁母憂。服除,授詹事。以在山東誤揭屬吏諱盜,奪官。叔琳登第甫二十,十六年,重遇登第歲,命給侍郎銜。二十一年,卒,年八十三。

叔琳富藏書,與方苞友。苞治諸經,叔琳皆與商榷。

子登賢,字筠盟。乾隆元年進士,授戶部主事。累遷左副都御史,督山東學政。康熙間,叔琳來督學,立三賢祠,祀胡瑗、孫復、石介,以式諸士。後六十年,登賢繼之,訓士遴才,皆循叔琳訓。四十九年,卒。

方苞,字靈皋,江南桐城人。父仲舒,寄籍上元,善為詩,苞其次子也。篤學修內行,治古文,自為諸生,已有聲於時。康熙三十八年,舉人。四十五年,會試中式,將應殿試,聞母病,歸侍。五十年,副都御史趙申喬劾編修戴名世所著南山集、孑遺錄有悖逆語,辭連苞族祖孝標。名世與苞同縣,亦工為古文,苞為序其集,並逮下獄。五十二年,獄成,名世坐斬。孝標已前死,戍其子登嶧等。苞及諸與是獄有干連者,皆免罪入旗。聖祖夙知苞文學,大學士李光地亦薦苞,乃召苞直南書房。未幾,改直蒙養齋,編校御制樂律、算法諸書。六十一年,命充武英殿修書總裁。世宗即位,赦苞及其族人入旗者歸原籍。

雍正二年,苞乞歸里葬母。三年,還京師,入直如故。居數年,特授左中允。三遷內閣學士。苞以足疾辭,上命專領修書,不必詣內閣治事。尋命教習庶吉士,充一統志總裁、皇清文穎副總裁。乾隆元年,充三禮義疏副總裁。命再直南書房,擢禮部侍郎,仍以足疾辭,上留之,命免隨班行走。復命教習庶吉士,堅請解侍郎任,許之,仍以原銜食俸。苞初蒙聖祖恩宥,奮欲以學術見諸政事。光地及左都御史徐元夢雅重苞。苞見朝政得失,有所論列,既,命專事編輯,終聖祖朝,未嘗授以官。世宗赦出旗,召入對,慰諭之,並曰:「先帝執法,朕原情。汝老學,當知此義。」乃特除清要,馴致通顯。

苞屢上疏言事,嘗論:「常平倉穀例定存七糶三。南省卑濕,存糶多寡,應因地制宜,不必囿成例。年饑米貴,有司請於大吏,定值開糶,未奉檄不敢擅。自後各州縣遇穀貴,應即令定值開糶,仍詳報大吏。穀存倉有鼠耗,盤量有折減,移動有運費,糶糴守局有人工食用。春糶值有餘,即留充諸費。廉能之吏,遇秋糴值賤,得穀較多,應令詳明別貯,備歉歲發賑。」下部議行。又言民生日匱,請禁燒酒,禁種煙草,禁米穀出洋,並議令佐貳官督民樹畜,士紳相度濬水道。又請矯積習,興人才,謂:「上當以時延見廷臣,別邪正,示好惡。內九卿、外督撫,深信其忠誠無私意者,命各舉所知。先試以事,破瞻徇,繩贓私,厚俸而久任著聲績者,賜金帛,進爵秩。尤以六部各有其職,必慎簡卿貳,使訓厲其僚屬,以時進退之,則中材咸自矜奮。」乾隆初,疏謂:「救荒宜豫。夏末秋初,水旱豐歉,十已見八九。舊例報災必待八九月後,災民朝不待夕,上奏得旨,動經旬月。請自後遇水旱,五六月即以實奏報。」並言:「古者城必有池,周設司險、掌固二官,恃溝樹以守,請飭及時修舉。通川可開支河,沮洳可興大圩,及諸塘堰宜創宜修,若鎮集宜開溝渠、築垣堡者,皆造冊具報,待歲歉興作,以工代賑。」下部議,以五六月報災慮浮冒,不可行;溝樹塘堰諸事,令各督撫籌議。

高宗命苞選錄有明及本朝諸大家時藝,加以批評,示學子準繩,書成,命為欽定四書文。苞欲仿硃子學校貢舉議立科目程式,及充教習庶吉士,奏請改定館課及散館則例,議格不行。苞老多病,上憐之,屢命御醫往視。

苞以事忤河道總督高斌,高斌疏發苞請託私書,上稍不直苞。苞與尚書魏廷珍善,廷珍守護泰陵,苞居其第。上召苞入對,苞請起廷珍。居無何,上召廷珍為左都御史,命未下,苞移居城外。或以訐苞,謂苞漏奏對語,以是示意。庶吉士散館,已奏聞定試期,吳喬齡後至,復補請與試。或又以訐苞,謂苞移居喬齡宅,受請託。上乃降旨詰責,削侍郎銜,仍命修三禮義疏。苞年已將八十,病日深,大學士等代奏,賜侍講銜,許還里。十四年,卒,年八十二。苞既罷,祭酒缺員,上曰:「此官可使方苞為之。」旁無應者。

苞為學宗程、硃,尤究心春秋、三禮,篤於倫紀。既家居,建宗祠,定祭禮,設義田。其為文,自唐、宋諸大家上通太史公書,務以扶道教、裨風化為任。尤嚴於義法,為古文正宗,號「桐城派」。

苞兄舟,字百川,諸生,與苞同負文譽。嘗語苞,當兄弟同葬,不得以妻祔。苞病革,命從舟遺言;並以弟林早卒未視斂,斂袒右臂以自罰。

王蘭生,字振聲,直隸交河人。少穎異。李光地督順天學政,補縣學生,及為直隸巡撫,錄入保陽書院肄業,教以治經,並通樂律、歷算、音韻之學。光地入為大學士,薦蘭生直內廷,編纂律呂正義、音韻闡微諸書。康熙五十二年,賜舉人,以父憂歸。服除,仍直內廷。六十年,應會試,未第。上以蘭生內直久,精熟性理,學問亦優,賜進士,殿試二甲一名,改庶吉士。雍正元年,散館授編修。三年,署國子監司業。四年,真除,督浙江學政。五年,遷侍講。六年,轉侍讀。時查嗣庭、汪景祺以誹謗得罪,停浙江士子鄉會試。蘭生奏言:「諸生當立品奉公,如有潛通胥役,欺隱錢糧,察出黜懲。臣按考所至,嚴加曉諭,並令地方官開報,必使輸糧乃得入試。」上深嘉之,命浙江士子準照舊鄉會試。七年,擢侍讀學士,督安徽學政。九年,遷內閣學士,仍留學政。十年,命再留任三年。尋充江南鄉試考官,調陜西學政。十三年,以所舉士得罪,左授少詹事。高宗即位,召入都,復授內閣學士。乾隆元年,遷刑部侍郎,兼署禮部侍郎。二年春二月,上奉世宗葬泰陵,蘭生扈行。次良鄉,發,病遽作,卒於肩輿中。賚白金五百,治喪涿州,待家人奔赴,賜祭葬如例。

蘭生為學原本程、硃,光地授以樂律,與共校硃子琴律圖說,刻本多謬誤,以意詳正,遂可推據。既入直,聖祖授以律管、風琴諸解,本明道程子說,以人之中聲定黃鐘之管,積黍以驗之,展轉生十二律,皆與古法相應;又至郊壇親驗樂器,推匏土絲竹諸音與黃鐘相應之理,其說與管子、淮南子相合。音韻亦授自光地,謂邵子經世詳等而略韻,顧炎武音學五書詳韻而略等,兼取其長,以國書五字類為聲韻之元以定韻,又用連音為紐均之法以定等,皆發前人所未及。聖祖深賞之,禁中夜讀書,惟蘭生侍左右,巡幸必以從,亟稱其賢。

留保,字松裔,完顏氏,滿洲正白旗人。祖阿什坦,字金龍,順治初,授內院六品他敕哈哈番,繙譯大學、中庸、孝經、通鑒總論諸書;九年,成進士,授刑科給事中。留保,康熙五十三年舉人。六十年,與蘭生同賜進士,改庶吉士。雍正元年,散館授檢討。累遷通政使。六年,廣東巡撫楊文乾劾總督阿克敦侵蝕粵海關火耗,並令家人索暹羅米船規禮諸事,上命總督孔毓珣及文乾按治。尋文乾卒,改命留保及郎中喀爾吉善會毓珣按治。毓珣以上怒,將刑訊,留保爭之,乃免。讞定,阿克敦罪當死,尋復起,語詳阿克敦傳。留保遷侍郎,歷禮、吏、工三部。乾隆初,乞病,致仕。卒,年七十七。

胡煦,字滄曉,河南光山人。初以舉入官安陽教諭。治周易,有所撰述。康熙五十一年,成進士,散館授檢討。聖祖聞煦通易理,召對乾清宮,問河、洛理數及卦爻中疑義。煦繪圖進講,聖祖賞之,曰:「真苦心讀書人也。」五十三年,命直南書房。上方纂周易折中,大學士李光地為總裁,命煦分纂。尋命直蒙養齋,與修卜筮精蘊。五十七年,遷洗馬,與修卜筮匯義。轉鴻臚寺少卿。六十一年,遷光祿寺少卿,再遷鴻臚寺卿。雍正元年,擢內閣學士,命與刑部侍郎馬晉泰如盛京按鞫私刨人葠,錄囚百五十八人,論罪如律。煦還奏:「刨葠俱貧民,羈候按鞫,自春夏至九、十月,往往瘐斃。請歸盛京刑部及將軍、府尹,以時定讞。」上如所請,命嗣後停遣部院堂官按鞫。五年,擢兵部侍郎,兼署戶部。時諸部院每於員外增置佐正員治事,煦協理副都御史,又協辦禮部侍郎。八年,命直上書房,充明史總裁。九年,授禮部侍郎。旋以衰老奪官。十年,河東總督田文鏡劾煦長子孟基本邱氏子,冒姓,以官卷得鄉舉,下部議黜。乾隆元年,煦詣闕召見,命還原銜,復孟基舉人,賜其幼子季堂廕生。煦疾作,卒於京師,賚銀五百治喪,賜祭葬。

煦正直忠厚,所建白必歸本於教化。嘗奏:「請敕州縣歲舉孝子悌弟,督撫旌其門,免徭役,見長官如諸生。其有慈惠廉節,篤於交友,下逮僕婢,行有可稱,皆得申請獎勸,庶化行俗美,人知自愛。」又請敕州縣勸農桑,或別設農官以專其任。又言:「督撫於命、盜重案,每用『自行招認』四字,援以定罪。夫民奸黠者抵死不服,愚懦者畏刑自誣。請嗣後必證據確然,然後付法司閱實。一有不當,旋即駁正,庶得慎刑之意。」他所陳奏,如廣言路,裕積儲,汰浮糧,省冗官,平權量,多切於世務。乾隆間,高宗詔求遺書,徵煦著述。時季堂官江蘇按察使,以煦著周易函書進。五十九年,特命追謚,謚文良。季堂自有傳。

魏廷珍,字君璧,直隸景州人。李光地督學,招入幕閱卷,旋以舉人薦直內廷,與王蘭生、梅成校樂律淵源。五十二年,成一甲三名進士,授編修。五十四年,遷侍講,直南書房。五十六年,轉侍讀。五十九年,轉擢詹事,復遷內閣學士。六十一年,命領兩淮鹽政。

雍正元年,授偏沅巡撫。世宗諭曰:「爾清正和平,但不肯任勞怨。今為巡撫,宜剛果嚴厲,不宜因循退縮。」二年,以辰谿諸生黃先文故殺人,讞鬥殺擬絞,遇赦請免;會同民譚子壽等因奸斃三命,擬斬候,皆失出;又以撥綠旗兵餉未具題:部議降調。上諭:「廷珍學問操守勝人,乃料理刑名錢穀,非過則不及。」召回京,授盛京工部侍郎。三年,授安徽巡撫,又以按治涇縣吏王時瑞等假印徵賦,寬徇,為部駁,上戒其毋姑息。廷珍疏言:「清釐錢糧,官吏侵蝕,往往匿民欠中,不易清察。請視民欠多少,多限一年,少限半年,分別詳察。官吏侵蝕,循例責償,如實欠在民,督徵催解,州縣有逋賦,繼任受代,許以時察報。」詔如所請行。嗣以清察限促,敕部更定。廣東總督孔毓珣入對,言道經宿州靈壁,積潦妨稼,上責廷珍怠玩,令出俸疏濬。廷珍乞內補,上不許。八年,調湖北。九年,召回京,授禮部尚書。十年,授漕運總督,署兩江總督。十二年,授兵部尚書。十三年,仍調禮部。

高宗即位,命以尚書銜守護泰陵。乾隆三年,授左都御史。四年,遷工部尚書。五年,以老病乞休。上以:「廷珍在世宗朝服官中外,不克舉其職,屢奉申誡,今以老病乞休,似此因循懈怠、持祿保身之習,斷不可長。」命奪官。時方苦旱,太常寺卿陶正靖謝上入對,上問:「今苦旱,用人行政或有闕失,宜直言。」正靖因奏:「廷珍負清望,無大過。近日放還,天語峻厲,非所以優老臣。」上霽顏聽之。後數日,上以語禮部尚書任蘭枝,蘭枝言正靖其門生也。上知蘭枝與廷珍為同年進士,因不懌,諭:「朝臣師友門生援引標榜,其端不可開。」命蘭枝書上諭戒正靖,蘭枝書上諭,言:「上問正靖,知為蘭枝門生。」上詰蘭枝,蘭枝對「年老耳聾,一時誤聽。」上愈怒,責蘭枝詐偽,對稱「老」,以舊臣自居,下吏議,蘭枝、正靖皆奪官。上命留蘭枝,正靖降調。

十三年,上東巡,過景州,廷珍迎謁,命還原銜,賜以詩,有句曰:「皇祖栽培士,於今賸幾人?」並書「林泉耆碩」榜賚之。十六年,又賜詩,予其子錫麟廕生。二十一年,復東巡,廷珍迎謁,年已將九十,又賜詩,予錫麟員外郎銜。尋卒,賜祭葬,謚文簡。

任蘭枝,字香谷,江蘇溧陽人。康熙五十二年一甲二名進士,授編修。雍正元年,命直南書房。累遷內閣學士。五年,與安南定界,偕左副都御史杭奕祿齎詔宣諭,語詳杭奕祿傳。使還,遷兵部侍郎。命如江西按南昌總兵陳玉章侵餉。調吏部。高宗即位,命充世宗實錄總裁。擢禮部尚書,歷戶、兵、工部,復調禮部。十年,以老致仕。十一年,卒。

蔡世遠,字聞之,福建漳浦人。父璧,拔貢生,官羅源訓導,有學行,巡撫張伯行延主鼇峰書院,招世遠入使院校訂先儒遺書。

世遠,康熙四十八年進士,改庶吉士。大學士李光地以宋五子之書倡后進,得世遠,深器之。四十九年,乞假省親。五十年,遭父喪,服除,赴京師。以假逾期,於例當休致,世遠不欲以父喪自列。會上命纂性理精義,光地充總裁,薦世遠分修,書成,世遠不欲以編輯敘勞,辭歸。巡撫呂猶龍延主鼇峰書院,以正學教士。居久之,雍正元年,特召授編修,直上書房,侍諸皇子讀。尋遷侍講。四年,遷右庶子,再遷侍講學士。五年,遷少詹事,再遷內閣學士。六年,遷禮部侍郎。

七年,上將設福建觀風整俗使,諮世遠,命與同籍京朝官議之。僉謂:「福建自海疆平定後,泉、漳將吏因功驟擢通顯,子弟驕悍,無所懍畏。皇上飭官方,興民俗,上年學政程元章奏以泉、漳風俗未醇,責成巡道整飭,自此益加儆戒。但人有賢愚,士或鄙劣薄行,民又多因怒互爭,未必洗心滌慮。應請設觀風整俗使,防範化導,於風俗人心有益。」得旨允行。八年,福建總督高其倬劾世遠長子長漢違例私給船照,上以疏示世遠。世遠奏言:「臣子長漢現在京邸。此所給照,不知何人所為。但有臣官銜圖書,非臣族姓,即臣戚屬,請敕鞫治。」部議坐失察,降調。十年,特旨復原職。十二年,卒。

世遠侍諸皇子讀,講四子、五經及宋五子書,必引而近之,發言處事,所宜設誠而致行者;於諸史及他載籍,則即興亡治亂,君子小人消長,心跡異同,反覆陳列。十餘年來,寒暑無或間。十三年,高宗即位,贈禮部尚書,謚文勤。所著二希堂集,禦制序弁首。「二希」者,謂功業不敢望諸葛武侯,庶幾範希文;道德不敢望硃子,庶幾真希元。上制懷舊詩,稱為聞之蔡先生。六十年,上將歸政,釋奠於先師,禮成,推恩舊學,加贈太傅。

子長澐,諸生。乾隆三年,以學行兼優薦,發江南以知縣用。歷甘泉、石埭、句容、無錫諸縣。兩江總督德沛稱其廉明,再遷江寧知府。調廬州、松江諸府,遷四川按察使。二十七年,特擢兵部侍郎。逾年,卒。上屢念世遠舊勞,推恩其諸子,觀瀾、長汭及孫本崇皆賜舉人。

沈近思,字位山,浙江錢塘人。康熙三十九年進士。四十五年,授河南臨潁知縣。潁水經許州東入臨潁,許州孔家口下距臨潁境僅百餘步,堤屢圮,水入臨潁,害禾稼。近思請築堤,臨潁任夫十之七,士民爭輸穀。日役千三百人,人穀二升,二十日而堤成。水至不為患,歲大熟。近思立紫陽書院,教士以正學。縣西葛岡村俗最惡,近思為置塾,課村童,立書程簿,躬教督之。化行於其鄉,俗日馴。五十二年,巡撫鹿祐薦卓異,遷廣西南寧同知。病,告歸。

五十九年,以浙江巡撫硃軾薦,敕部調取引見,命監督本裕倉。浙江福建總督滿保奏請以知府揀發福建,檄署臺灣知府。近思議析置數縣,道鎮彈壓,府治駐兵三千,分布營汛,收材勇入行伍,嚴加操練,以漸移充內地各標。流民至者,必審籍貫、稽家口,方授以田土,否則悉驅過洋。議未即行,雍正元年,召授吏部文選司郎中,賜第,賚帑金四百。尋授太僕寺卿,仍兼領文選司事。二年,超授吏部侍郎,命與尚書阿爾松阿如河南按治諸生王遜等糾眾罷考,論如律。

四年,充江南鄉試考官。例以鄉試錄進呈,上嘉近思命題正大,策問發揮性理,諭獎之。時侍郎查嗣庭、舉人汪景祺以誹謗獲罪,停浙江人鄉會試。近思疏言:「浙省乃有如嗣庭、景祺者,越水增羞,吳山蒙恥!」因條列整飭風俗,約束士子,凡十事。上曰:「浙省有近思,不為習俗所移,足為越水、吳山洗其羞恥!」所陳委曲詳盡,下巡撫李衛、觀風整俗使王國棟,如議施行。五年,擢左都御史,仍兼領吏部事。卒,命平郡王福彭往奠,加禮部尚書、太子少傅。以其子方幼,令吏部遣司官為治喪,賜祭葬,謚端恪。

近思少孤貧,為僧靈隱寺。世宗通佛理,嘗以問近思,近思對曰:「臣少年潦倒時,嘗逃於此。幸得通籍,方留心經世事以報國家。亦知皇上聖明天縱,早悟大乘,然萬幾為重,臣原皇上為堯、舜,不原皇上為釋迦。即有所記,安敢妄言以分睿慮?」上為改容。及耗羨歸公議起,上意在必行,近思獨爭之,言:「耗羨歸公,即為正項,今日正項之外加正項,他日必至耗羨之外加耗羨。臣嘗為縣令,故知其必不可行。」上一再詰之,近思陳對侃侃,雖終不用其言,亦不以為忤也。

子玉璉,世宗命地方官加意撫養成立。乾隆中,授廣西桂林同知。

雷鋐,字貫一,福建寧化人。為諸生,究心性理。庶吉士蔡世遠主鼇峰書院,從問學。雍正元年,舉於鄉。世遠時為侍郎,薦授國子監學正。十一年,成進士,改庶吉士,乞假歸。十三年,高宗即位,召來京,命直上書房。乾隆元年,散館,以病未入試,特授編修。二年,大考二等一名,賜筆、墨、硯、葛紗。同直編修余棟以憂歸,端慧皇太子喪,入臨,上欲留之。鋐疏言:「侍學之臣,當明大義,篤人倫。使棟講書至『宰我問三年喪』,何以出諸口?」楊名時亦諍之,事遂寢。四年,遷諭德。尋以父憂歸。九年,召來京,仍直上書房,賞額外諭德食俸。

十年,三遷通政使。上以言事者多沽直名,自規便利,詔訓飭。鋐疏言:「皇上裁成激勸,俾以古純臣為法,意至深厚。然臺諫所得者名,政事所得者實。論臣子之分,不惟不可計利,並不可好名;而在朝廷樂聞讜言,不必疑其好名,並不必疑其計利。孔子稱舜大知曰隱惡揚善,則知當時進言者不皆有善無惡,惟舜隱之揚之,所以嘉言罔攸伏,成執兩用中之治。」得旨嘉獎。十四年,乞假省母。十五年,還京,命督浙江學政。十六年,上南巡,賜以詩,謂:「浙江近福建,為汝便養母也。」尋調江蘇。十八年,擢左副都御史,仍留督學。復調浙江。杭州、嘉興災,致書巡撫周人驥議蠲賑。人驥以時已隆冬,例不得補報,難之。鋐遂疏聞,上命蠲賑。二十一年,乞養母歸。二十二年,上南巡,鋐迎謁,上書榜賜其母。二十四年,丁母憂。二十五年,鋐未終喪,卒,年六十四。

鋐和易誠篤,論學宗程、硃。督學政,以小學及陸隴其年譜教士。與方苞友,為文簡約沖夷得體要。

論曰:聖祖以硃子之學倡天下,命大學士李光地參訂性理諸書,承學之士,聞而興起。苞與光地誼在師友間,名時、蘭生、廷珍、世遠皆出光地門。煦亦佐光地修書,得受裁成於聖祖。叔琳,苞友,鋐又出世遠門,淵源有自。獨近思未與光地等游,而學術亦無異,雍正初,與世遠、苞先後蒙特擢。壽考作人,成一時之盛,聖祖之澤遠矣。


\end{pinyinscope}