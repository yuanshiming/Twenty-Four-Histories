\article{志一百六}

\begin{pinyinscope}
○兵二

△綠營

綠營規制,始自前明。清順治初,天下已定,始建各省營制。綠營之制,有馬兵、守兵、戰兵。戰守皆步兵。額外外委皆馬兵。綜天下制兵都六十六萬人,安徽最少,閩、廣以有水師故最多,甘肅次之。綠營隸禁旅者,惟京師五城巡捕營步兵。將軍兼統綠營者惟四川。有屯兵者惟湖南、貴州。其新疆之綠營屯防,始乾隆二十五年,由陜、甘陸續移往駐防。各省標兵規制,督撫得隨時疏定。綠營戰功,自康熙征三籓時,用旗、綠兵至四十萬,雲、貴多山地,綠營步兵居前,旗兵繼之,所向輒捷。其後平定準部、回疆、金川,咸有勛績。乾隆四十六年增兵,而川、楚教匪之役,英、法通商之役,兵力反遜於前。迨粵寇起,廣西綠營額兵二萬三千,土兵一萬四千,遇敵輒靡。承平日久,暮氣乘之,自同治迄光緒,疊經裁汰,綠營之制,僅存而已。

京師巡捕五營,設步軍統領一人,統左右翼總兵官以及十六門門千總,海澱、暢春園、樹村汛、靜宜園、樂善園設副將或守備各官不等,置兵共三千人。京城內九門、外七門,每門設千總二,門甲十或二十,門軍四十人。左翼總兵統步軍營巡捕南、左二營各汛官,凡兵三千六百有奇。右翼總兵統步軍營巡捕北、右二營各汛官,凡兵二千五百有奇。

各直省營制,順治元年,定直隸官兵經制,設直隸巡撫,標兵分左、右二營,游擊以下八人。設宣府、真定、薊州、通州、天津、山海關六鎮總兵官及鎮標守備、游擊等,設紫荊關等七協副將及協標官兵,設拱極城等十七處參將,山永等營游擊,鞏華城等處守備、都司,分領各營兵。

定山東官兵經制,設河道總督,標兵分中、左、右三營,設副將或游擊以下將領八,兵凡三千,備河防護運。山東巡撫標兵分左右二營,設游擊以下將領八,兵凡二千。設臨清、沂州二鎮總兵官及將領八,兵共二千四百有奇。設德州、青州、武定三營參將或守備將領八或六,兵共二千二百有奇。設登州水師營守備,登州、萊州、臨清、濟南各營游擊或守備四,兵共一千二百有奇。初,山東與直隸、河南共一總督,康熙元年,設山東提督,尋並裁去,以巡撫兼任。

山西、江南、陜西官兵經制,並於順治二年定之。山西設宣大總督及巡撫,督標分中、左、右三營,撫標分左右營,各設將領八,兵凡二千。設太原、平陽二協副將及協標官兵。設汾州等營參將、游擊、守備,分領營兵。十三年,裁宣大總督,康熙元年,設山西提督,迭裁迭復,雍正九年仍裁之,以巡撫兼任。

江南設漕運總督,江蘇、鳳廬二巡撫,標兵及左右營如制,將領九或八人,兵共四千有奇,並設奇兵營、游兵營。設江南漢兵提督,分中、左、右、前、後五營,分設將領八,兵凡四千。設蘇州、鎮江、浦口、安慶、池太、東山、廣德八鎮總兵官,鎮標兵及將領。設狼山等七協副將,金山、常州各營參將、游擊、守備,分領營兵。國初設江南江西河南總督。其後分合不常。康熙間,定為兩江總督。又裁鳳廬巡撫歸並江蘇。設蘇松提督。尋定為江寧提督,增安徽提督,分轄營務。又裁安徽提督,改江南水陸提督,統全省官兵。先是設操江巡撫,轄安慶等五府,滁、和等三州兵。後改安徽巡撫,以鳳廬兵並屬之。

陜西初設川陜總督,並轄四川兵,標兵分五營。別設西安、延綏、甘肅、寧夏四巡撫,標兵各分左右營,將領略如諸省。設延綏、固原、臨鞏、鳳翔、漢羌、甘肅六鎮總兵官,鎮標兵亦分五營,將領如之,延綏又分設東西二協。設西安、慶陽等八處副將,宜君、階州等各營參將、游擊、都司及守備,分領營兵。康熙時,迭改川陜總督,並轄山、陜、甘。尋改川陜甘總督。乾隆間,甘肅分設總督,以四川總督兼轄陜西兵,為川陜總督,復改陜甘總督。國初設甘肅巡撫,其寧夏、延綏巡撫先後裁撤,寧夏歸甘肅,延綏歸陜西。後又裁甘肅巡撫,陜甘總督兼統撫標兵。甘州置甘肅總兵官,尋改設甘肅提督。初設陜西漢兵提督及寧夏提督,分五營,皆設將領八,兵凡四千人。後改西安提督,又移駐固原,改固原提督云。

順治三年,定河南、江西、湖廣官兵經制。河南設巡撫,標兵分左右營,將領八,兵二千,制同上。設河南提督,標兵分中、左、右三營,設將領分統。設河北、南陽、開歸三鎮總兵官,標兵各分左右營,將領兵數如撫標制。設開封副將、守備以下將領七,兵一千人,河南衛輝、汝寧、歸德各營各參將等,兵各一千。設磁州營都司,兵五百人,後屬直隸嵩縣等二營守備,兵三百或二百。先是河南與直隸、山東共一總督,兼轄河南官兵。其後或專設河南總督,或裁改之。至雍正十三年,仍為河南巡撫。

江西初設巡撫及南贛巡撫,標兵分左右營,設將領五人,兵凡千五百人。設江西提督,標兵分五營,營設將領八,兵凡五千人。設南贛、九江二鎮總兵官,標兵分五營,各設游擊以下將領官,兵如提標之數。設袁州等四協副將,分左右營,將領各八,兵凡二千人。設廣德各營參將,撫州各水師營守備,兵六百人,南康等營守備兵三百人。康熙初年,裁南贛巡撫,以標兵屬江西巡撫。七年,裁提督。十三年,復設。嗣增設撫建提督,旋裁之,並裁江西提督,以巡撫兼任。

湖廣設總督,標兵分中、左、右營,將領各八,兵凡三千人。設湖北巡撫、鄖陽巡撫、偏沅巡撫,撫標兵分左右營,將領官兵如江西撫標例。設湖廣提督,標兵分五營,將領官兵如江西提標例。設荊州、鄖陽、長沙三鎮總兵官,辰州協副將,標兵各分中、左、右營,各設將領八,兵凡三千人。設黃州、承天、常德三協副將,協標兵各設將領七,兵凡千二百人。承天協後改安陸營。設漢陽等營參將將領各四,兵六百人。夷陵等營游擊各設將領三,兵四百人。設三江口等營守備、把總,兵各二百人。康熙初,並湖廣總督為川湖總督。其後四川總督不轄湖廣,復設湖廣總督。裁鄖陽巡撫,以湖北巡撫統轄標兵。

順治四年,定四川官兵經制。設四川巡撫,標兵分左右營,各設將領八,兵凡千三百人。設建昌、保寧、永寧、夔州四鎮總兵官,鎮標分三營,設將領八,兵凡二千人。設松潘、成都、重慶三協副將,協標兵分二營,設游擊以下將領官兵。設威茂等各營參將、游擊、守備,分領營兵。四川初僅設巡撫,駐成都府。川陜總督駐陜西,兼轄四川十四年。嗣設四川總督,駐重慶府。其間或並為川湖總督,駐荊州九年,移駐重慶十九年。或云川陜甘,或云川陜,遷改靡常。至乾隆間,定為四川總督。

順治五年,定浙江官兵經制。設總督,標兵分三營,設副將或游擊將領各八,兵共三千。設浙江巡撫,標兵二營,將領各八,兵共二千。浙江提督標兵三營,營設將領八,兵共三千。設定海、衢州二鎮總兵官,標兵皆三營,營設將領八,共兵各三千。錢塘水師二營,臺州水師三營,營設將領八,共兵各三千。衢州設水師左右路總兵官,標兵三營,游擊以下將領分統營兵。設衢州、湖州、嘉興等七協副將,標兵皆三營,營皆設將領八,每協共兵二千五、六百。設金華、嚴州、處州三協副將,標兵二營,將領各八,共兵皆千六百。設吉安等各營守備、參將,分統營兵。先是設浙江總督,其後改稱閩浙,兼轄福建,裁改不常。雍正間,定為閩浙總督。

順治七年,定福建官兵經制。設福建巡撫,標兵二營,將領八,兵凡二千。設福建水陸提督,標兵三營,營設將領八,兵凡三千。設汀州、泉州、銅山三鎮總兵官,及援剿總兵官、中路總兵官,標兵各二營,各設將領八,兵二千。設福州、漳州、建寧三協副將,標兵三營,各設將領八,兵凡三千。設福州水師,及汀州、興化、邵武、延平、閩安、同安七協副將標兵,各設將領八,兵凡二千。設福寧協副將二營,將領七,兵凡千八百人。設泉州等營參將、長樂等營游擊,將領各八,共兵各一千。

順治八年,定兩廣官兵經制。廣東設巡撫,標兵二營,將領八,兵凡二千。設廣東提督,標兵五營,將領八,兵凡五千。設廣東水師總兵官,標兵六千,分左右二協,中、左、右三營。二協設副將,復分二營,設將領八,兵一千五百。三營水師,各設將領八,兵各一千。設肇慶、潮州、瓊州三鎮總兵官,標兵二營,將領八,兵凡二千。設韶州、惠州、高州、南雄四協副將,協標兵皆二營,將領各八,共兵各二千。惟南雄為一千六百。設肇慶、高州水師及吳川等營參將,柘林鎮各營游擊,將領各七,共兵各一千。設東莞、始興等州縣守備以下將領,兵二百至五百有差。廣西設巡撫,標兵二營,將領八,兵凡千五百。廣西提督標兵分五營,將領八,兵凡四千有奇。設左右翼總兵官,並桂林暨南寧城守營。九年,增設潯梧、柳慶、思南三協副將以下將領,兵各千二百;鬱林、新太、河池三營參將以下將領,兵各六百;永寧、昭平二營參將以下將領,兵各四百;上思、三里二營守備以下將領,兵各二百;賀縣營守備,兵百人。十年,定兩廣總督標兵分五營,中營設將領八,左、右、前、後營共將領八,兵凡五千。國初置兩廣總督,康熙二年,專轄廣東,四年,兼轄兩廣,雍正元年,復專轄廣東,十三年,仍兼轄兩廣。

順治十六年,定雲、貴官兵經制。設雲貴總督,標兵分中、左、右、前四營,中營設將領八,餘三營將領八,兵凡四千。設雲南巡撫,標兵二營,將領八,兵一千五百。先一年,貴州設巡撫,營制亦同。及是設貴州提督,標兵分左、右、前、後四營,左營設將領八,餘三營將領八,兵凡三千。設大定、黔西、鎮遠、威寧四鎮總兵官,標兵三營,將領八,兵各二千有奇。設貴陽城守協及平遠、定廣、銅仁、平越、安南五協副將,標兵二營,游擊以下將領。設思南營等處參將、游擊、守備,分統官兵。國初云貴總督,兩省互駐。康熙元年,分置兩省總督,自後或改或並。迨乾隆中,仍定為雲貴總督。此直省綠營初制也。

雍正四年,靖逆將軍富寧安於哈密置大小卡路八,西安總兵潘之善於沙州西南諸隘設哨探、置臺站防夷。五年,以浙江綠營積弱,選山、陜、甘兵壯健者移駐之。十年,以苗疆遼闊,貴州改設總兵、游擊,統轄丹江、臺拱等營,及銅仁、鎮遠、石阡各協,並新設上江、下江諸營協,隸古州,以鎮攝之。十一年,諭各總兵官巡察營伍。乾隆五年,用湖廣總督那蘇圖言,裁虛設戰船,除私立提塘,及字識占冒口糧之弊。十六年,定哈密駐防兵制,於安、甘、涼、肅四提、鎮營分遣將弁廿餘,兵二千往駐。二年一受代,四月、八月迭更半數,新舊相間,以資教練。回營時,鎮臣核其勤惰,分別擢用之。十八年,陜甘總督尹繼善疏陳西陲防務,宜慎選安西將材,多備槍彈,預蓄資糧,築城垣,擇畜牧,允行。二十四年,改安西提督為巴里坤提督,設哈密副將以下將領八,兵八百,餘裁改有差。尋改設烏魯木齊總兵官,分中、左、右營及城守營,隸巴里坤提督。凡巴里坤、烏魯木齊將領官兵,歸陜甘總督統屬。乾隆四十一年,大小金川平,新入版圖,屯兵駐守,制同內地,設懋功、綏靖、崇化、撫邊、慶寧等營,置游擊、守備等官,兵共二千六百有奇。四十九年,以陜甘總督福康安言,甘肅原設額兵五萬六千六百人,陜西額兵三萬四千五百九十人,迭經移駐裁並,存兵五萬五千九百餘,減原額過半。嗣增兵萬二千七百餘,合舊存兵額凡七萬人。而州縣墩戍兵力猶單,請於平涼等府州縣各增兵額,墩堡四十四座,於各標兵內酌選移駐,從之。旋議再增兵三千。又議陜、甘各營兵習弓矢、鳥槍、馬上槍箭,每日在本營習技,五日小合操,十日大合操,演九進十連環陣法,練勁旅三萬人。五十三年,諭提、鎮不得私立旗牌、伴當等名,致侵兵額。嘉慶四年,以剿辦教匪,各省額兵徵調四出,令各省召募補充。五年,陜西設寧陜鎮總兵、副將以下官,咸如昔制。十年,諭各督、撫、提、鎮,以練習鄉勇法練習綠旗兵。道光五年,諭直隸備戰兵萬五千三百有奇,演習車砲陣式。旋即議裁。十六年,諭直隸營兵以四成習弓矢,二成習步槍兼馬槍,其刀矛二技,令藤牌軍盡習之。二十二年,直隸蘆臺增設通永鎮總兵官,以北塘、海口等十五營均歸統屬,分三營,設游擊、守備等將領,新鎮標兵凡五千四百餘,專操水陸技藝。咸豐八年,河南歸德營升為鎮,設總兵官、左右營都司、游擊等,馬兵五百八十,步兵千一百有奇。同治元年,諭專設總督之直隸、江南、四川、甘肅及督、撫同城之福建、廣東、湖北由總督會同提督節制。其江蘇、浙江、安徽、江西、陜西、湖南、廣西、貴州各鎮兵,就近由巡撫節制。四年,增安徽皖南鎮總兵官,設將領弁兵如制。六年,諭寧夏鎮綠營兵原額七千,陜西定邊協原額千人,回匪亂後,存者寥寥,咸令補足。九年,改廣東赤溪營為水師,隸陽江鎮統轄,變通巡洋舊章。又移湖北武昌城守營分防金口、簰洲二汛。十二年,於山西南北二鎮選兵一千,分二營,設將領訓練。光緒十一年,以廣西南邊二千餘里,原設隘一百九,分卡六十六,兵力猶單,分要處為三路,鎮南關口關前隘憑祥土州為中路,自關以東諸隘為東路,以西諸隘為西路,就原有防軍二十二營並為二十四營,以十二營專防中路,餘十二營分防東、西路。廣西提督自柳州移駐龍州。其城守營設游擊及守備等。增設柳慶鎮總兵官,駐柳州。綠營歷年增損規制,大略如是。其移駐編改,節目不能覙縷以詳也。

若其裁汰之數,自順治中,所裁山西標兵四千餘,陜、甘將領四十八,兵一萬六百餘,河南五百,湖廣五千,江西三千,將領八,江南萬九千餘,將領百十七,其最多者也。餘者海州一協,裁將領七,兵六百餘,臨清一鎮,裁將領五,兵一千,三營兵五百,沂州鎮裁將領九,臨清城守營將領五,兵三百,壽張營兵二百。又裁江西及南贛撫標二營官兵,四川撫標、湖北及鄖陽撫標各二營官兵,多少不等。康熙八年,裁辰常鎮總兵,設辰州協標官兵。二十三年,裁崇明提督,設崇明水師總兵,定三營及奇兵營制。三十四年後,計所裁標兵,南贛鎮千餘,九江協九百餘,銅鼓營兵八百餘為最多,餘者自四、五百以下,少至六、七人。乾隆中,裁撫標新設二營,餘所裁最多三百餘,最少十人、九人。嘉慶十九年,諭各標額兵六十二萬四千餘,較雍、乾以來所增實多,令督、撫、提、鎮量加裁汰。於是次第減萬四千有奇。二十五年,又諭各省勿糜餉以養額兵。道光中,裁陜、甘綠營馬兵三千六百餘。又裁山東、山西撫標,及兗州等三鎮,太原、大同二鎮,東河河標,雲、貴督、撫、鎮、協各標兵額,暨福建水陸各營,浙江馬、步兵,兩廣、江蘇、安徽馬、步、守兵各有差。

咸豐元年,曾國籓疏言:「八旗勁旅,以強半翊衛京師,以少半駐防天下,而山海要隘,往往布滿,其額數常不過三十五萬。綠營兵名為六十餘萬,其實缺額常六、七萬人。乾隆中葉,增兵議起。向之空名坐糧,悉令補足,一舉而增兵逾六萬。經費驟加,大學士阿桂爭之不得。至嘉慶、道光間,睹帑藏之漸絀,思阿桂之遠慮,特詔裁兵,而兩次所裁僅一萬六千。請飭各省留強汰弱,復乾隆初制。」諭如所請,命各督、撫分三年裁復舊額,所裁之數,年終匯陳,不得再有空糧之弊。四年,裁山西馬、步、守兵五千八百餘,雲南步、守兵三千九百餘。同治八年,裁九江、洞庭、岳州、荊州等水師營,改城守營,並酌設陸汛。

光緒五年,左宗棠、楊昌濬疏言:「軍興未收制兵之效,由餉薄而額多,不能應時精練,兵不練與無兵同,練不精與不練同。甘肅賦少兵多,軍實向資他省,餉源稍絀,動滋事端。亟宜量減可裁之兵,以節餉糈,即以所裁軍餉加所留之兵,庶可責其勤練。雍正中,甘兵定額較內地為多,後雖陸續裁減,計尚存馬、步、守兵五萬七千餘。即須分成核減。」六年,丁寶楨言:「四川自軍興後,招募營勇,裁者少而增者多。同治間,楚、黔、川勇多至六萬餘。次第裁撤,至今存營勇二千九百餘,尚可裁其什一。」是歲,湖南各營弁兵及水陸防勇次第裁者四千三百餘,湖北裁者三千二百餘,安徽陸續裁者約九千餘。八年,張曜疏言:「裁汰勇丁,即可規復兵額,變通營制,方能永固邊防。」九年,張之洞奏整頓山西綠營練軍,裁湘軍正勇千人,設籌資遣,尋復裁汰,綜合前後裁兵約及六千人。時貴州制兵裁汰二成,守兵裁者三千二百餘,戰兵二千九百餘。江西額兵萬一千九百餘,近始以制兵作練軍,然長年調練,冒替弊生,遂有「兵止一人,人已三變」之誚。因定撫標選鋒仍舊操練,裁外屬各營抽練之軍,悉回原汛。

十一年,諭直省裁汰綠營。卞寶第言:「廣西額兵二萬三千,土兵一萬四千。粵逆初起,不過二千人。合此巨數之兵,不能擊少數之賊。廣西如此,他可類推。自後發、捻、回、苗恣亂,綠營戰績無聞。今宜以漸變通營制,裁額並糧,以兩餉挑一兵。如額兵一萬,分二十營,一半駐守,一半巡防。無事則計日操防,有警則隨時援應。綠營積習,無許復存。」

二十二年,諭:「近者戶部奏請裁兵,宜汰綠營七成,勇營三成。通諭以來,惟山東陳明分限五年裁減五成,此外酌裁無幾。綜各省兵勇尚八十萬有餘,歲餉約共三千餘萬。綠營積惰,久成虛設。當茲借款期迫,棄有用之餉,養無用之兵,因之國窮民蹙。各將軍、督、撫亟應定限切實裁減以聞。」

二十四年,從胡燏棻等言,裁並綠營、練勇,選練新操。時山東兵額已陸續裁十之三。至是以不敷分配,未裁之二成,仍止不裁。於是山西以汰存兵額不敷防卡之用,請增練新軍數營。恭壽亦言綠營弊深,屢裁而益弱,須藉民力以輔之,宜急行團練。

二十七年,劉坤一、張之洞奏汰綠營,言:「綠營官皆選補,兵皆土著。兵非弁之所自招,弁非將之所親信,既無恩義,自難鈐束。以傳舍之官,馭世業之兵,亦如州縣之於吏役,欲其整飭變化,服教從風,此必無之事。況綠營將弁,薰染官習,官弁且不易教,況於兵乎!層層積弊,已入膏肓,既甚驕頑,又極疲弱,本難練成可用之兵,自非裁汰不可。惟有分年漸裁一策,不分馬、步、戰、守,每年裁二十分之一,計百人裁五,限二十年而竣。計成扣餉,按次銷除,即以節省之餉,作緝捕營察之用。惟湖南鎮筸鎮,系改土歸流,無土著農戶,除苗產外,地皆屯田,民皆兵籍,綏靖鎮亦然,請於此兩鎮兵額不再裁汰,但將綠營改為勇營。所裁將領,可用者改隸勇營,不能帶勇者,開缺或改官。使武職無把持之弊,合天下兵出於勇營之一途。更定營名,以符名實。」

二十九年,從徐世昌等言,以綠營挑改巡警。

宣統元年,步軍統領衙門疏言:「巡捕五營,原設馬、戰制兵萬人。嗣因屢次裁並,中營現兵千五百人,內分馬兵五百四十,戰兵八百六十,簡差戰兵百人。南營兵千二百五十人,內分馬兵三百二十,戰兵三百三十,簡差戰兵百人。左營兵八百人,內分馬兵三百二十,戰兵三百八十,簡差戰兵百人。右營兵七百人,內分馬、戰兵各三百,簡差戰兵百人。惟南營汛地設巡警後,差務較簡,請撥南營兵三百七十五人隸北、左、右三營,每營馬兵各三百六十五人,戰兵四百十人。」是年,免裁之鎮筸、綏靖二鎮,定議改為續備軍。此外乾州、永綏、常德諸協,河溪、保靖等營,留兵各三、四百人,去綠營之名,改勇營規制,作為續備軍。岳州、澧州等營,各裁將弁,存兵六十四人或至九十三人。其餘撫、提、鎮、協諸營,各裁統將,一以同城將領兼統餘兵。湖北通省將領,副將五人裁去一人,參將七人裁二人,游擊十七人裁五人,都司十一人裁三人,守備三十三人裁十人。其撫標各營尚未盡裁,俟分軍裁汰。是年,裁江北舊役衛兵左右二哨兵。貴州綠營已裁二成,尋裁副將以下各官,歸並四營,酌改六營,惟邊防要地佐防軍所不及者緩裁。

二年,浙江綠營裁汰後,尚餘將領三百九十九,兵七千餘,一律裁盡,收取馬匹軍械,改編巡防隊八營。四川綠營次第裁盡,挑選精壯改練防軍。湖廣營已裁十成之七,一、二年後,即可裁盡。湖北自咸豐八年裁馬兵改步兵,同治八、九年,先後裁撤水陸軍二千一百有奇,馬二百餘匹,光緒十一年以來,又裁二千九百有奇,馬、步、戰、守兵七千六百有奇,馬八百八十餘匹,實存馬、步、守兵共七千餘,馬千六百六十匹,以後分年裁盡。尋湖北之漢陽協興國等營,湖南之衡州協保靖等營,副將以下各官,一律停補。裁福建綠營,計至宣統六年裁盡,現存將領三百八十人,步、戰、守、舵、炊、兵夫五千九百有奇。直隸綠營,於同治年間改為練軍。光緒以來,通永等鎮分年裁減,至二十九年,實存馬、步、戰、守兵二萬六千餘人。其天津城守及葛沽、通永、通州、北塘等凡十一營,當庚子之變,潰散無餘,遂悉裁撤。此外各營均十裁其三,復裁將弁三百十四人。其大沽六營,庚子年傷亡過甚,亦全裁之,改設巡警。

三年,直隸綠營尚存官弁七百餘,兵六千六百餘,實行裁汰,惟淮、練、巡防各營,暫仍其舊。四川關外原設臺兵,向由綠營撥派,共三十九臺,將弁兵丁,一律裁撤。福建綠營,豫定裁盡年限,所節之餉,編練巡防隊。江西亦擬裁盡綠營。甘肅邊要,陸軍尚未成鎮,僅存馬、步、守兵萬七千餘,資其防制之力,暫從緩裁。山西綠營所存無幾,分三年盡裁之。江南綠營亦然,惟徐州鎮標緩撤。山東以全裁綠營情事窒礙,因請緩裁。廣東綠營,三江、崖州二協,儋州營,督標中營均免裁。其餘十減其四,將領五百餘,除邊要及兼防營之缺緩裁,餘悉停補,改練陸軍。廣西綠營,自光緒二十九年裁後,僅存撫、提標將領五或四人,兵四五十人,左江、右江兩鎮將領各二人,兵各二十人。此歷朝裁兵大較也。

綠營積重,沿數百年。同治中興以後,疆臣列帥,懲前毖後,漸改練勇巡防之制。光、宣間屢加裁汰。宣統三年,武昌事起,陸軍部疏言時局艱危,各省綠營、巡防隊一律從緩裁撤。綠營之制,遂與有清相終始云。

直隸總督統轄督標四營,節制一提督、七總兵,兼轄保定城守,熱河喀喇沁,吉林、奉天捕盜,永定河、運河等營。

督標四營。左營,右營,前營,後營。保定城守等營。新雄營,涿州營,拱極營,良鄉營,中路,東路,南路,西路,北路,張家口,獨石口。熱河喀喇沁等營。烏蘭哈達,塔子溝,承德府,平泉州,三座塔,多倫諾爾。吉林捕盜營。賓州,五常,敦化縣,雙城,伊通州。奉天捕盜營。昌圖府,新民,海城,承德縣,開原縣,鐵嶺,遼陽州,錦縣,寧遠州,義州,廣寧縣,蓋平縣,復州,金州,懷德縣,奉化縣,唐平縣,海龍,鳳凰,安東縣,寬甸縣,懷仁縣,通化縣,興京,岫巖州。永定河、運河等營。北運河務關,楊村,通惠河漕運,南運河。

直隸古北口提督統轄提標四營,節制七鎮,兼轄河屯一協、三屯等營。提標中營、左營、右營、前營,密雲城守營,順義營,承德府河屯協左營、右營,唐三營,三屯營,喜峰路,燕河路,建昌路,八溝營,建昌營,赤峰營,朝陽營,昌平營,居庸路,鞏華營,懷柔路,湯泉營,古北口。

馬蘭鎮總兵統轄鎮標二營,兼轄遵化等營。鎮標左營、右營,遵化營,薊州營,曹家路,墻子路,黃花山,餘丁營。

泰寧鎮總兵統轄鎮標二營,兼轄紫荊關等營。鎮標左營、右營,水東★營,紫荊關,白石口營,廣昌營,插箭嶺,礬山營,易州營,房山營,淶水營,馬水口,沿河口。

宣化鎮總兵統轄鎮標三營,兼轄獨石口、多倫諾爾二協,蔚州等營。鎮標中營、左營、右營,獨石口協左營、右營,鎮安營,龍門所營,雲州堡,馬雲堡,鎮寧堡,松樹堡,滴水堡,赤城堡,君子堡,靖安堡,多倫諾爾協中營、左營、右營,蔚州營,東城營,宣化城守營,懷來營,懷來城守營,岔道營,龍門路營,懷安營,左衛營,柴溝營,西陽河堡營,張家口營,萬全營,膳房堡營,新河口堡營,洗馬林堡營。

天津鎮總兵統轄鎮標二營,兼轄河間、大沽二協,務關等營。鎮標左營、右營,四黨口營,河間協左營、右營,鄭家口營,景州營,大沽協前左及中左、後左、前右、中右、後右六營,葛沽營,祁口營,務關營,霸州營,武清營,靜海營,舊州營,天津城守營。

正定鎮總兵統轄鎮標二營,兼轄固關等營。鎮標左營、右營,固關營,龍泉關營,倒馬關營,忠順關營,龍固城守營。

大名鎮總兵統轄鎮標三營,兼轄開州協、大名城守等營。鎮標中營、左營、右營,開州協,杜勝營,東明營,長垣營,大名城守營,廣平營,順德營,磁州營。

通永鎮總兵統轄鎮標二營,兼轄通州、山永二協,北塘等四營。鎮標左營、右營,通州協左營、右營,張家灣營,採育營,三河營,山永協左營、右營,山海路營,石門路營,蒲河營,樂亭營,北塘營,豐順營,玉田營,寶坻營。

山東巡撫兼提督,駐濟南府,節制三鎮,統轄撫標二營,兼轄登榮水師一協。

撫標左營、右營,登榮水師練軍營。

兗州鎮總兵統轄鎮標二營,兼轄沂州一協、泰安等六營。鎮標左營、右營,沂州協,泰安營,臺莊營,濟南城守營,武定營,安東營,沙溝營。

登州鎮總兵統轄鎮標二營,兼轄文登等七營。鎮標左營、右營,文登營,膠州協,萊州營,即墨營,青州營,寧福營,壽樂營。

曹州鎮總兵統轄鎮標二營,兼轄臨清協、德州等營。鎮標中營、右營,臨清協,德州營,東昌營,單縣營,壽張營,濮州營,高唐營,梁山營,鉅野營,桃源營。

河東河道總督統轄河標三營,兼轄濟寧城守及運河、懷河、豫河等營。

河標中營、左營、右營,濟寧城守營,運河營,懷河營黃河北岸祥河、下北河、黃沁河、陽封,豫河營上南河、中河、下南河。

山西巡撫兼提督,節制二鎮,統轄撫標二營,兼轄精兵兩哨、口外七捕盜營。

撫標左營、右營,精兵兩哨,歸化標,薩拉齊標,豐鎮標,寧遠標,和林格爾標,托克托城標,清水河標。

太原鎮總兵統轄鎮標二營,兼轄蒲州、潞安二協,太原等營。鎮標左營、右營,蒲州協,運城營,吉州營,潞安協,澤州營,東陽營,粟城營,太原營,平陽營,隰州營,汾州營,平垣營,盂壽營,東灘營,平定營。

大同鎮總兵統轄鎮標三營、殺虎口一協、新平路等營。鎮標中營、左營、右營,殺虎協左營、右營,寧武營,偏關營,鎮西城,河保營,保德營,水泉營,平魯營,靖遠營,歸化城,新平路,天城營,陽和營,渾源營,得勝路,豐川營,助馬路,懷仁城,北樓營,東路,忻州營,靈丘路,山陰路。

河南巡撫兼提督,節制三鎮,統轄撫標二營,兼轄開封營。

撫標左營、右營,開封城守營。

河北鎮總兵統轄鎮標二營,兼轄河南城守等營。鎮標左營、右營,河南城守營左營、右營,衛輝營,彰德營,陜州營,內黃營,嵩陽營,王祿店營,滑縣營。

南陽鎮總兵統轄鎮標二營,兼轄荊子關、信陽二協,汝寧等營。鎮標左營、右營,荊子關協,盧氏營,信陽協左營、右營,汝寧營,鄧新營,襄城城守營,新野營,光州營,固始縣營。

歸德鎮總兵統轄鎮標二營,兼轄永城等營。鎮標左營、右營,永城營,考城營,陳州營。

兩江總督統轄督標二營,節制三巡撫、一提督、九總兵,兼轄江寧城守一協、揚州、鹽捕二營。

督標中營、左營,江寧城守協左、右兩營,奇兵營,青山營,浦口營,溧陽營,瓜州營,揚州營,鹽捕營。

漕運總督統轄各衛所外,復統轄旗、綠、漕標三營,兼轄淮安城守等營。

漕標中營、左營、右營,淮安城守營,海州營,鹽城水師營,東海水師營。

江蘇巡撫節制三鎮,統轄撫標二營,兼轄蘇州城守營。

撫標左營、右營,蘇州城守營。

江南水陸提督節制五鎮,統轄提標五營,兼轄太湖、松北二協,松江城守等營。提標中營、左營、右營、前營、後營,太湖協左營、右營,松北協,松江城守營,金山營,柘林營,青村營,平望營,江陰營,靖江營,孟河營,常州營,鎮江營,松南水師營,南匯水師營。

狼山鎮總兵統轄鎮標二營,兼轄通州等營。鎮標中營、右營,通州水師營,掘港水師營,泰州營,泰興營,三江水師營。

蘇松鎮水師總兵統轄鎮標三營,兼轄海門一協。鎮標中營、左營、右營,海門協。

徐州鎮總兵統轄鎮標中營,兼轄徐州城守等營。鎮標中營,徐州城守營,蕭營,宿州營。

淮揚鎮總兵統轄鎮標三營,兼轄清江城守等營。鎮標中營、左營、右營,清江城守營,宿遷營,廟灣水師營,佃湖營,洪湖水師營,葦蕩左營,葦蕩右營。

福山鎮總兵統轄鎮標二營,吳淞、川沙二營。鎮標左營、右營,吳淞水師營,川沙水師營。

安徽巡撫兼提督,節制二鎮,統轄撫標二營,兼轄安慶一協,游兵、潛山二營。

撫標左營、右營,安慶協左營、右營,游兵營,潛山營。

壽春鎮總兵統轄鎮標二營,兼轄六安等營。鎮標中營、右營,六安營,潁州營,泗州營,廬州營,亳州營,龍山營。

皖南鎮總兵統轄鎮標二營,兼轄徽州等營。鎮標中營、右營,徽州營,池州營,蕪採營,廣德營。

江西巡撫兼提督,節制二鎮,統轄撫標二營,兼轄南昌城守一協。

撫標左營、右營,南昌城守協。

九江鎮總兵統轄鎮標二營,兼轄九江城守等營。鎮標前營、後營,九江城守營,廣信營,鉛山營,饒州營,浮梁營,建昌營,廣昌營,武寧營,瑞州營,撫州營,銅鼓營,南康營。

南贛鎮總兵統轄鎮標三營,兼轄袁州一協、贛州城守等營。鎮標中營、左營、後營,袁州協,臨江營,贛州城守營,寧都營,南安營,吉安營,龍泉營,萬安營,永豐營,蓮花營,興國營,文英營,永鎮營,橫岡營,羊角營。

長江水師提督節制四鎮,統轄提標五營,兼受兩江總督、湖廣總督節制。提標中營,金陵營,裕溪營,大通營,蕪湖營。

長江水師岳州鎮總兵統轄鎮標四營。鎮標中營,荊州營,沅江營,陸溪營。

長江水師漢陽鎮總兵統轄鎮標四營。鎮標中營,田鎮營,蘄州營,巴河營。

長江水師湖口鎮總兵統轄鎮標五營。鎮標中營,安慶營,吳城營,饒州營,華陽營。

長江水師瓜洲鎮總兵統轄鎮標四營。鎮標中營,江陰營,三江營,孟河營。

閩浙總督節制二巡撫、三提督、十二鎮,統轄督標三營,兼轄撫標二營、南臺水師營。

督標三營。中營、左營、右營,撫標左營、右營,南臺水師營。

福州將軍除統轄八旗駐防官兵外,兼轄福州城守營,節制福寧鎮標、福州城守及同安等營。

福建陸路提督節制四鎮,統轄提標五營,兼轄福州城守、興化城守二協、泉州城守等營。提標中營、左營、右營、前營、後營,福州城守協左營、右營,興化城守協左營、右營,泉州城守營,長福營。

福寧鎮總兵統轄鎮標三營,其左營系水師提督節制,兼轄海壇、閩安二協,烽火門四營。鎮標中營、左營、右營,海壇協左營、右營,閩安水師協左、右兩營,烽火門水師營,桐山營,連江營,羅源營。

汀州鎮總兵統轄鎮標三營,兼轄邵武城守營。鎮標中營、左營、右營,邵武城守營左營、右營。

建寧鎮總兵統轄鎮標三營,兼轄延平城守協、楓嶺營。鎮標中營、左營、右營,延平城守協左營、右營,楓嶺營。

漳州鎮總兵統轄鎮標三營,兼轄順昌協、同安等營。鎮標中營,順昌協,同安營,詔安營,平和營,雲霄營,龍巖營,漳州城守營。

福建水師提督節制三鎮,及福寧鎮左營、廣東南澳鎮左營,統轄提標五營,兼轄金門協,銅山、湄州等營。鎮標中營,左、右、前、後四營,金門協,銅山水師營,湄州水師營。

閩粵南澳鎮外海水師總兵。左營。

福建臺灣巡撫節制二鎮。

臺灣鎮總兵統轄鎮標中營,兼轄臺灣北路、臺灣水師二協、臺灣城守及臺灣南路等營。鎮標中營,臺灣北路協中營、右營,臺灣水師協中營、左營、右營,臺灣城守營左營、右營,臺灣南路營,臺灣嘉義營,臺灣艋舺水師營,滬尾水師營,噶嗎蘭營,臺灣恆春營,臺灣道標,臺灣南路下淡水營。

澎湖鎮外海水師總兵統轄鎮標二營。鎮標左營,右營。

浙江巡撫統轄撫標二營,兼轄海防營。

撫標左營、右營。巡鹽營,海防營。

浙江水陸提督節制五鎮,統轄提標五營,兼轄杭州等協、太湖等營。提標中營、左營、右營、前營、後營,杭州城守協,錢塘水師營,嘉興協左、右兩營,湖州協左、右兩營,安吉營,紹興協左營、右營,乍浦水師協左營、右營,太湖水師營,寧波城守營,澉浦水師營,海寧水師營。

定海鎮總兵統轄鎮標三營,兼轄象山協,鎮海、定海城守營。鎮標中營、左營、右營,象山協左營、右營,石浦水師營,鎮海水師營,定海城守營。

海門鎮總兵統轄鎮標三營,兼轄臺州協、海門城守等營。鎮標中營、左營、右營,臺州協中營、左營、右營,海門城守水師營,寧海營,太平營。

溫州鎮總兵統轄鎮標三營,兼轄樂清、瑞安、平陽三協,玉環、溫州城守等營。鎮標中營、左營、右營,樂清協,大荊營,磐石營,瑞安協左營、右營,平陽協左營、右營,玉環營左營、右營,溫州城守營。

處州鎮總兵統轄鎮標三營,兼轄金華協、麗水營。鎮標中營、左營、右營,金華協左營、右營,麗水營。

衢州鎮總兵統轄鎮標三營,兼轄嚴州協,楓嶺、衢州城守等營。鎮標中營、左營、右營。嚴州協左、右兩營,楓嶺營,衢州城守營。

湖廣總督節制二巡撫、二提督、五鎮,統轄督標三營。

督標中營、左營、右營。

湖北巡撫統轄撫標二營。

撫標左營、右營。

湖北提督節制二鎮,統轄提標五營,兼轄黃州、漢陽二協,荊州城守等營。提標中營、左營、右營、前營、後營,黃州協,蘄州營,漢陽協,荊州城守營,武昌城守營,德安營,興國營,均光營,襄陽城守營,荊門營,安陸營。

鄖陽鎮總兵統轄鎮標四營,兼轄竹山協、鄖陽城守營。鎮標中營、左營、右營、前營,竹山協,鄖陽城守營。

宜昌鎮總兵統轄鎮標四營,兼轄施南協、遠安等營。鎮標中營、左營、前營、後營,施南協左營、右營,遠安營,衛昌營,宜都營,荊州堤防營。

湖南巡撫節制三鎮,統轄撫標二營,兼轄鳳凰等屯軍營。

撫標左營、右營,鳳凰屯,永綏屯,乾州屯,古丈坪屯,保靖屯。

湖南提督節制三鎮,統轄提標五營,兼轄長沙等協、澧州等營。提標中營、左營、右營、前營、後營,長沙協左營、右營,乾州協左營、右營,鎮溪營,河溪營,永順協,常德協,龍陽城守營,澧州營,岳州營,九溪營,永定營,辰州城守營,古丈坪營。

鎮筸鎮總兵統轄鎮標四營,兼轄沅州、靖州二協,綏寧、長安等營。鎮標中營、左營、右營、前營,沅州協,晃州營,靖州協,綏寧營,長安營。

永州鎮總兵統轄鎮標三營,兼轄寶慶、衡州二協,臨武等營。鎮標中營、左營、右營,寶慶協,衡州協,臨武營,宜章營,桂陽營,武岡營,嶺東營。

綏靖鎮總兵統轄鎮標二營,兼轄永綏協、保靖營。鎮標左營、右營,永綏協中營、左營,芭茅坪營,保靖營左營、右營。

陜甘總督節制二巡撫、三提督、十一鎮,統轄督標五營。

督標中營、左營、右營、前營、後營。

陜西巡撫統轄撫標三營。

撫標中營、左右兩營。

陜西固原提督節制四鎮,統轄提標五營,兼轄靖遠等協、靜寧等營。提標中營、左、右、前、後四營,靖遠協,蘆塘營,鹽茶營,下馬關營,八營,潼關協,金鎖關,三要司,商州協中營、左營、右營,西安城守協左營、右營,盩厔營,靜寧營,馬營監營,安定營,隆德營,西鳳營,邠州營,長武營,慶陽營,涇州營,紅德城守營,固原城守營,硝河城汛,平涼城守營,秦州營,利橋營,宜君營,化平營。

延綏鎮總兵統轄鎮標三營,兼轄定邊協、神木等營。鎮標中營、左營、右營,定邊協,靖邊營,鎮靖營,安邊營,神木營,黃甫營,麻池潢營,高家營,鎮羌營,波羅營,綏德城守營,延安營,鄜州營,延綏城守營。

陜安鎮總兵統轄鎮標三營,兼轄鎮安城守等營。鎮標中營、左營、右營,鎮安城守營,磚坪營,興安城守營,鎮坪營,孝義城守營,紫陽營,白河營,洵陽營。

河州鎮總兵統轄鎮標二營,兼轄洮岷協、循化等營。鎮標左營、右營,洮岷協,階州營,文縣營,西固營,岷州營,舊洮營,循化營,保安營,起臺營,蘭州城守營,鞏昌營,臨洮營,河州城守營。

漢中鎮總兵統轄鎮標三營,兼轄寧陜等營。鎮標中營、左營、右營,寧陜營,陽平關營,寧羌營,略陽營,留壩營,定遠營,西鄉營,華陽營,東江口營,漢中城守營,漢鳳營,鐵爐川營,佛坪營。

甘肅提督統轄提標五營,兼轄永固城守協,節制西寧等四鎮。提標中營、左營、右營、前營、後營,永固城守協,甘州城守營,梨園營,洪水營,南古城營,山丹營,硤口營,大馬營,察漢俄博營。

西寧鎮總兵統轄鎮標五營,兼轄鎮海協、西寧城守等營。鎮標中營、左營、右營、前營、後營,鎮海協,哈拉庫圖爾營,西寧城守營,巴燕戎格營,巴暖三川營,貴德營,南川營,大通營,永安營,白塔營,碾伯營,威遠營。

寧夏鎮總兵統轄鎮標五營,兼轄中衛協、花馬池等營。鎮標左營、右營、前營、後營兼管城守營、城守營,中衛協,石空寺堡,古水井堡,花馬池營,安定堡,靈武營,靈州營,同心營,平羅營,洪廣營,玉泉營,廣武營,興武營,橫城營。

涼州鎮總兵統轄鎮標五營,兼轄永昌、莊浪二協。鎮標中營、左營、右營、前營、後營,西把截堡,永昌協,寧遠營,水泉營,新城營,張義營,鎮番營,安城營,大靖營,土門營,莊浪協,俄博嶺營,松山營,鎮羌營,岔口營,紅城堡,紅水營,三眼井營。

肅州鎮總兵統轄鎮標三營,兼轄金塔、安西二協,肅州城守等營。鎮標中營、左營、右營,金塔協,鎮彞營,清水營,高臺營,撫彞營,紅厓堡,安西協,布隆吉爾營,橋灣營,肅州城守營,嘉峪關營,沙州營,靖逆營,赤金營。

甘肅新疆巡撫節制三鎮,統轄撫標四營、瑪納斯協、濟木薩等營。

撫標中營、左營、右營,城守協中營,喀喇巴爾噶遜營,瑪納斯協,濟木薩營,庫爾喀喇烏蘇營,精河營,吐魯番營。

新疆喀什噶爾提督節制三鎮,統轄提標五營,兼轄回城、莎車二協,英吉沙爾等營。提標中營、左右兩營、前營、城守營,回城協中營、左右兩旗,莎車協中營、中左右三旗,英吉沙爾營,和闐營,瑪喇巴什營。

新疆阿克蘇鎮總兵統轄鎮標四營,兼轄烏什協、哈喇沙爾等營。鎮標中左右三營、城守營,烏什協,哈喇沙爾營、庫車營。

新疆巴里坤鎮總兵統轄鎮標四營,兼轄哈密協、古城等營。鎮標中營、左右兩營、城守營,哈密協,古城營,塔爾納沁營,木壘營。

伊犁將軍節制一鎮,統轄軍標二營。軍標中營、左營。

伊犁鎮總兵統轄鎮標四營,兼轄塔爾巴哈臺協、霍爾果斯等營。鎮標中營、左營、右營、綏定城守營,塔爾巴哈臺協,霍爾果斯營,寧遠城營。

四川總督節制一提督、四鎮,統轄督標三營。

督標中營、左營、右營。

成都將軍除統轄八旗駐防官兵外,統轄軍標綠營二營,節制建昌、松潘二鎮。軍標左營、右營。

四川提督節制四鎮,統轄提標三營,兼轄阜和、懋功、馬邊三協,成都城守等營。提標中營、左營、右營,阜和協左營、右營,黎雅營,泰寧營,懋功協,崇化營,綏靖營,慶寧營,撫邊營,馬邊協左營、右營,存城營,萬全營,平安營,成都城守營、右營,永寧營,瀘州營,敘馬營,建武營,普安營、右營,安阜營,峨邊營、右營,鎮遠營,綿州營。

川北鎮總兵統轄鎮標三營,兼轄綏定等營。鎮標中營、左營、右營,綏定營,順慶營,太平營,巴州營,廣元營,潼川營,城口營,通江營。

重慶鎮總兵統轄鎮標三營,兼轄夔州、綏寧二協,忠州營。鎮標中營、左營、右營,夔州協左營、右營,巫山營,梁萬營,鹽廠營,綏寧協左營、右營,酉陽營,黔彭營,邑梅營,忠州營。

建昌鎮總兵統轄鎮標二營,兼轄會川等營。鎮標中營、左營,會川營,永定營,越巂營,寧越營,保安營,靖遠營,瀘寧營,會鹽營,懷遠營,冕山營。

松潘鎮總兵統轄鎮標三營,兼轄維州協、漳臘等營。鎮標中營、左營、右營,維州協左營、右營,茂州營,漳臘營,疊溪營,龍安營,平番營。

兩廣總督節制二巡撫、三提督、九鎮,統轄督標五營,兼轄本標水師、綏瑤等營。

督標中營、左營、右營、前營、後營,督標水師營,綏瑤營。

廣州將軍除統轄八旗駐防官兵外,節制南韶連鎮標、潮州鎮標、高州鎮標、瓊州鎮標、惠州協標、肇慶協標、廣州城守協、三江口協、黃岡協、羅定協、增城各二營,南雄協、欽州各一營,雷州左營、前山、永靖、連陽、惠來、驍平、潮陽、廉州、儋州、萬州、和平、四會、那扶、永安、興寧、平鎮、潮州城守、石城、陽春、三水、徐聞、綏瑤等營。

廣東巡撫統轄撫標二營。

撫標左營、右營。

廣東陸路提督節制五鎮,統轄提標五營,廣州城守等協、增城等營。提標中營、左營、右營、前營、後營,廣州城守協左營、右營,三水營,惠州協左營、右營,和平營,肇慶城守協左營、右營,四會營,那扶營,增城營左營、右營,永靖營,永安營。

南韶連鎮總兵統轄鎮標三營,兼轄三江口、南雄二協,清遠、佛岡等營。鎮標中營、左營、右營,三江口協左營、右營,連陽營,南雄協,清遠營左軍、右軍,佛岡營。

潮州鎮總兵統轄鎮標三營,兼轄黃岡協、惠來等營。鎮標中營、左營、右營,黃岡協左營、右營,惠來營,饒平營,潮陽營,興寧營,平鎮營,潮州城守營。

高州鎮水師兼陸路總兵統轄鎮標二營,兼轄羅定協、陽江等營。鎮標左營、右營,羅定協左營、右營,陽江營,硇州營,吳川營,電白營,東山營,陽春營。

廣東水師提督節制五鎮,統轄提標五營,香山等四協,新會、前山等營。提標中營、左營、右營、前營、後營,香山協左營、右營,順德協左營、右營,大鵬協左營、右營,赤溪協左營、右營,新會營左營、右營,前山營。

碣石鎮總兵統轄鎮標三營,兼轄平海營。鎮標中營、左營、右營,平海營。

瓊州鎮水師兼陸路總兵統轄鎮標二營,兼轄崖州協、海口等營。鎮標左營、右營,崖州協,海口營,萬州營,儋州營,海安營。

南澳鎮總兵分管閩、粵二省,統轄鎮標二營,兼轄澄海等營。鎮標左營隸福建水師提督節制,右營,澄海營左營、右營,海門營,達濠營。

北海鎮水陸總兵統轄鎮標二營,兼轄龍門協、雷州等營。鎮標左營、右營,龍門協左營、右營,雷州營,欽州營,白龍營,徐聞營,石城營,靈山營。

廣西巡撫統轄撫標二營。

撫標左營、右營。

廣西提督節制三鎮,統轄提標中軍一營,兼轄平樂、新太二協,全州等營。提標中軍,平樂協左營、右營,富賀營,麥嶺營,新太協,馗纛營,全州營,賓州營,三里營,上思營,東蘭營,桂林城守營,龍州城守營。

左江鎮總兵統轄鎮標三營,兼轄梧州、潯州二協,南寧城守等營。鎮標中營、左營、右營,梧州協左營、右營,懷集營,潯州協左營、右營,南寧城守營,鬱林營。

右江鎮總兵統轄鎮標三營,兼轄鎮安協、思恩等營。鎮標中營、左營、右營,鎮安協左營、右營,思恩營,隆林營,上林營,恩隆營。

柳慶鎮總兵統轄鎮標二營,慶遠、義寧二協,融懷等營。鎮標左營、右營,慶遠協左營、右營,義寧協左營、右營,融懷營,永寧營,柳州城守營。

雲貴總督節制二巡撫、二提督、十鎮,統轄本標三營,兼轄曲尋協、雲南城守、尋霑等營。

督標中營、左營、右營,曲尋協左營、右營,雲南城守營,尋霑營。

雲南巡撫統轄撫標二營。

撫標左營、右營。

雲南提督節制六鎮,統轄提標三營,兼轄楚雄協,武定、大理城守等營。提標中營、左營、右營,楚雄協,武定營,大理城守營。

臨元鎮總兵統轄鎮標四營,兼轄元新、澂江等營。鎮標中營、左營、右營、前營,元新營,澂江營。

開化鎮總兵統轄鎮標四營,兼轄廣南、廣西等營。鎮標中營、左營、右營、後營,廣南營,廣西營。

騰越鎮總兵統轄鎮標三營,兼轄永昌等二協、龍陵營。鎮標中營、左營、右營,永昌協左營、右營、順云協中營、左營、右營,龍陵營。

鶴麗鎮總兵統轄鎮標三營,兼轄維西協、永北、劍川等營。鎮標中營、左營、右營,維西協左營、右營,永北營,劍川營。

昭通鎮總兵統轄鎮標四營,兼轄東川、鎮雄等營。鎮標中營、左營、右營,東川營,鎮雄營。

普洱鎮總兵統轄鎮標三營,兼轄威遠、景蒙等營。鎮標中營、左營、右營,威遠營,景蒙營。

貴州巡撫統轄撫標二營,兼轄古州等十衛、都江、下江等營。

撫標左營、右營,古州左衛、右衛,八寨衛,臺拱衛,黃施衛,丹江衛,凱里衛,清江左衛、右衛,石峴衛,都江標,下江標。

貴州提督節制四鎮,統轄提標三營,兼轄大定等協、羅斛等營。提標左營、右營、前營,大定協左營、右營,平遠協左營、右營,遵義協左營、右營,定廣協左營、右營,羅斛營左營,右營,貴陽營,平越營,歸化營,黔西營,安順城守營,仁懷營,新添營。

安義鎮總兵統轄鎮標三營,兼轄永安協、長壩等營。鎮標中營、左營、右營,永安協左營、右營,長壩營,普安營,安南營,冊亨營。

古州鎮總兵統轄鎮標三營,兼轄上江、都勻二協,朗洞等營。鎮標中營、左營、右營,上江協左營、右營,都勻協左營、右營,朗洞營左營、右營,黎平營左營、右營,荔波營,下江營。

鎮遠鎮總兵統轄鎮標三營,兼轄清江等三協、臺拱等營。鎮標中營、左營、右營,清江協左營、右營,松桃協左營、右營,銅仁協左營、右營,臺拱營左營、右營,丹江營左營、右營,思南營,凱里營,黃平營,天柱營,石阡營。

威寧鎮總兵統轄鎮標二營,兼轄畢赤、水城等營。鎮標左營、右營,畢赤營,水城營。

綠營兵額,清初未定。考明代京軍二十萬餘,外軍九十九萬餘。順治間不可考,大約視舊額約裁減十三四。康熙兵制,京巡捕三營經制馬步兵三千三百,直隸各標兵三萬七百,山西二萬五千,川陜總督,陜、甘兩巡撫及提鎮各標兵八萬五千九百七十八,四川四萬,雲南四萬二千,貴州二萬,廣西二萬,湖廣四萬,廣東七萬三千一百十人,江南總督,總漕,江寧、安徽兩巡撫,京口將軍四萬九千八百五十,浙江四萬三千四百五十,江西萬五千,福建六萬九千七百二十六,山東總河及撫、鎮標兵二萬,河南一萬,都各省經制馬步兵五十九萬四千四百十四。逮乾隆二十九年,次第增加,各省多者一千至六千餘,惟貴州加至萬八千二百餘,減者江西七百餘,廣東四百餘,浙江二千餘,福建三千餘,都六十三萬七千三百二十三。

至五十年,各省綠營兵額,京巡捕五營一萬,直隸三萬九千四百二,山東萬七千五百四,山西二萬五千七百五十二,河南萬一千八百七十四,江南四萬八千七百四十七,江西萬三千九百二十九,福建六萬三千一百十九,浙江四萬三十七,湖北萬七千七百九十四,湖南二萬三千六百四,四川三萬千一百十二,陜、甘八萬四千四百九十六,廣東六萬八千九十四,廣西二萬三千五百八十八,雲南四萬千三百五十三,貴州三萬七千七百六十九,都五十九萬九千八百十四,綜計數減於舊者凡四萬餘。各省減者,自數百至數千不等,惟陜、甘減至萬二千,則以四十六年新增者不在此數,而山東、河南、江南視舊額轉多,蓋河、漕標兵本定分額,此實並入各省中也。

嘉慶十七年,綠營都數為六十六萬千六百七十一,視乾隆中葉增額六萬餘,各省均所有益,惟浙江減額千餘。其江南總額,此分江寧七千三十九,南河萬五千六百六十六,漕運三千六百八十一,江蘇二萬三千七百四十八,安徽八千七百三十八,總為五萬六千八百七十二,增舊額八千餘。又舊額但舉山東,此分山東萬五千九百三十三,東河四千二百四十一,增額三千餘,略可考見。十九年,山西等省共裁兵萬五千四百餘,內改馬戰兵為步守兵共千二百餘。

道光初元,諭行裁汰,減額萬餘,復議裁改。二十九年兵額,直隸四萬千三百三十五,山東二萬五十七,河南萬五千三百八十一,東河並入河南、山東。山西二萬二千八百五,江蘇三萬八千一百八,安徽九千四百四十二,南河、漕運並入江南。江西萬二千四百七十二,福建六萬千六百七十五,浙江三萬七千五百六十五,陜西二萬四千七百二十,甘肅六萬八千八百六十二,湖北二萬五百五,湖南二萬七千百十五,四川三萬三千八百十一,廣東六萬八千三百二十二,廣西二萬二千四百七十二,雲南三萬九千七百六十二,貴州三萬六千四百七十七,都五十八萬五千四百十二,京營萬名在外。減於乾隆舊額且逾萬矣。

咸豐軍興以來,綠營議裁。迄同治、光緒間,兵制一變,直省厲行簡汰,顧不能悉廢,存額尚不為少。再綜近時綠營兵額,京巡捕營一萬外,十六門門甲三百十,門軍六百四十,凡萬九百五十,直隸四萬二千八百十,山東萬七千八百七十五,山西萬六千四十五,河南萬四百六十八,江蘇二萬五千七百七十,安徽九千三百六十四,江西萬一千七百四十,長江水師萬一千六十四,福建二萬三千六百七十八,臺灣八千二百六十八,浙江二萬三千四百九,湖北萬五千三百四十三,湖南三萬零二十四,陜西萬八千六百八十七,甘肅萬二千七百二十五,新疆二萬六千五百十五,四川三萬千二百八十一,廣東四萬六千七百七十四,廣西萬四千一百十五,雲南萬二千五百七十二,貴州四萬二千九百五,都四十六萬二千三百八十二。取道光末年額較之,減於舊者幾十二萬,但舊額不及長江水師與臺灣云。


\end{pinyinscope}