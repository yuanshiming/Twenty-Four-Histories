\article{列傳二百七十八}

\begin{pinyinscope}
忠義五

王淑元高延祉黃為錦瑞麟曹燮培楊映河等

劉繼祖翟登峨等劉作肅沈衍慶李仁元李福培王恩綬

李右文從弟載文李榞陳肖儀萬成袁祖德李大均

於松尚那布李淮唐治鍾普塘等林源恩唐德升

畢大鈺湯世銓劉福林謝子澄周憲曾等文穎徐鳳喈等

張積功傅士珍瞿濬冒芬施作霖韓體震德克登額

蔣嘉穀鄧玲筠承順托克清阿馮元吉

平源張寶華王泗周來豫餘寶錕王汝揆

王淑元,字秋查,浙江鄞縣人。以舉人知縣,分發廣西。歷權柳城、雒容、平南、馬平等縣事,授博平,調天保。會臨桂縣民以糧價重不輸稅,大吏欲懾以兵,淑元在省,進議曰:「民固有所苦,得平自服。」遂調臨桂。既蒞任,為汰浮收,民便之,無逋賦者。

道光季年,粵匪洪秀全始謀逆,其黨李嘉耀潛入省垣煽土匪內應,發覺,淑元鞫得餘匪匿所,悉數掩擒,敘功升龍州同知,以肅清會匪獎知府銜。二十九年,擢太平府知府,旋任龍州。時廣西巡撫鄭祖琛懦而黯,群盜蜂起,輒務諱匿。三十年,賊潘寶源等來犯,淑元率練丁御諸距州十里之灣道,以次子光頡自隨。戰不利,雨甚,藥濕砲不及發,因退回城。及門,賊已由間道入,擁眾逼官廨。淑元立堂上,罵不絕口,呼家眾殺賊。賊斫淑元僕地,擄之去,光頡奔奪,賊殺光頡,而投淑元於勤村河。盜去三日,始出之,身首皆裂,獨面色如生。

高延祉,字筠坡,浙江蕭山人。由舉人充官學教習,期滿,用知縣。道光二十一年,英夷犯浙江沿海,舉行團練,延祉率義勇為前驅,擊毀夷船。咸豐元年,揀發廣西,與都下親友別,即以致身奉國自誓。尋署隆安縣事,土賊陸鵬理與其黨乃利中、凌阿東等,屢為邑害。延祉集團練,遣間諜,以計誘殺鵬理,捕獲其家屬黨與二十餘人,並毀利中巢窟及塚。阿東亡命,謀復仇,糾眾千餘,據白山之感墟,與歸德接壤。延祉偕歸德土知州黃為錦率練勇四百進攻,沿途搜戮賊探多名,行抵袍墟,距感墟十餘里,遇伏,軍潰。復激勵練勇奮擊,殪匪二百餘。賊眾蜂至,延祉挺刃督戰,被賊矛中腹遇害,為錦亦戰死,僕隸多殉之者。

延祉任隆安七十餘日,無日不在外治戰事,居縣廨僅數日耳。民感其保衛之恩,爭賻其孤,孤;乃以其貲建祠祀之,以從死之僕隸、壯勇附。同治十一年,追謚壯節。

為錦,山東人。

瑞麟,白氏,漢軍鑲白旗人。由謄錄議敘巡檢,道光五年,選廣西鎮峽寨巡檢。調主簿,擢州判、知州。咸豐元年,授西隆州知州。咸豐二年二月,洪秀全自永安犯桂林,敗竄全州,瑞麟已卸州事,繼任知州曹燮培知瑞麟才,深相結納,約共守御。時都司武昌顯以楚兵四百援桂林,道經全州,燮培留助守。四月,賊薄城下,發砲轟擊,斃賊甚夥。越日攻益急,歷十一晝夜,提督餘萬清、劉長清來援,分駐城北太平堡,城西魯班橋,距十五里外,牽制弗能進,守者憊甚,子藥不繼。賊穴城,地雷發,城崩,賊乘入,千總葉永林、把總張元福死之。瑞麟素驍勇,遇賊中衢,手刃數人,力竭身死。燮培亦巷戰死。

賊攻城時,多死傷,恨甚,城陷,屠之,焚屋舍幾盡。文武官紳同時死者:署全州營參將楊映河,把總卜有祥,解餉官四川知縣盧金第,安徽府經歷陳堯,湖南游擊餘遇升,都司武昌顯,千總田慶華、馬瑞龍,把總盧先振、黃志林、韓大興,外委孫紹全、楊清麒、田宏義、楊大賓、龔心仁、田宗南,武舉唐殿試,生員蔣成龍、金建勛,武生張以敬,幕友黃柏彬、祝永文、硃福坪、周希齡、孫培駒、楊菱舟、金家駒、硃澤,凡三十餘人。學正農賢託,年七十,甫歿,棺毀,妻殉之。瑞麟謚壯節,與燮培並贈道銜,詔祀京師昭忠祠。建專祠全州,曰愍忠,祀燮培及諸死事者。

燮培,字理村,浙江仁和人。選柳州通判,攝西隆州、賓州事,除東蘭州知州,權全州。性倜儻,有吏才,不拘節目,聲伎滿前,然無廢事。或規之,引文信國公少年時事自解,曰:「他日能學文山足矣!」人謂燮培無負素志云。

時死廣西者,又有署永安州吳江龍門司巡檢馮元等。

劉繼祖,江西玉山人。增貢生。道光十一年,以同知分福建。十九年,除淡水同知,以憂去。服闋,借揀知州,發廣西。二十七年,授永康,尋署藤縣。時灌陽、平樂、陽朔等處匪徒肆擾,偕知府張熙宇督剿,殲擒殆盡,進知府。咸豐元年,金田賊敗竄大黃江,繼祖率水陸壯勇乘夜攻擊,焚其巢。以所部練勇失鈐束,奪職。四年,巡撫勞崇光奏請留藤協辦團練,尋艇賊梁培友糾眾攻藤,繼祖偕知縣翟登峨等嬰城固守,設間出奇擊卻之。旋以土賊馮六、戴九等接踵至,據河干,盡焚沿岸舟,鄉團來援者不得渡。賊眾兵單,城陷,繼祖受重創,與登峨子襄採、團長梁文軾等巷戰,力竭,死之。登峨被擄,罵賊不屈,被害,棄尸於河。典史冉正棠鬥死獄門。詔復繼祖原官,賞世職,登峨以次死者恤有差。

登峨,字眉峰,山東章丘人。進士,截取選藤縣。

劉作肅,字敬亭,奉天承德縣人。道光元年舉人,選知縣,授天河縣。歷寧明知州,兼明江同知。咸豐三年,賊眾萬餘攻城,相持五月餘,解圍去。以城守功加知府,賞花翎。六年,署太平知府,賊屢來犯,御卻之。十年,復來。城中無儲粟,賊圍亟,守陴者皆走。城陷,作肅投池,水淺,不能死,為賊擁去。以其居官清廉,不忍害之。作肅乃吟絕命辭,絕粒死。其弟與僕姚雲、吳貴同殉。妻趙及子家祥、女等皆先自盡。以子家鳳被執不屈死,恤如制。賞世職,建祠府城,二僕並賜恤。

沈衍慶,字槐卿,安徽石埭人。道光十五年進士,以知縣發江西,署興國,補泰和。二十五年,調鄱陽,縣濱湖,盜賊所出沒。衍慶編漁戶,仿保甲法行之,屢獲劇盜。俗悍好鬥,輒輕騎馳往,竭誠開導,事浸息。兩遇水災,盡力賑撫,存活無算。舉卓異。咸豐二年,粵匪陷湖北武昌,衍慶請兵守康山,控鄱陽門戶。三年,九江陷,譌言四起,居民逃亡,不可禁止。衍慶率練勇巡東門,見糧船中數百人譟而前,衍慶手刃二人,餘黨心習服,人心始定。賊圍南昌,巡撫張芾檄衍慶赴援,會合省防諸軍與賊戰,大破之。賊將東竄,衍慶慮賊犯鄱陽,請於巡撫,馳歸。時樂平令李仁元攝鄱陽事,同商守御。賊至,與仁元同力戰,城陷,死之。贈道銜,立祠鄱陽。

仁元,字資齋,河南濟源人。道光二十七年進士,內閣中書,改知縣。咸豐元年,授樂平。民俗剽悍,以禮讓教之,多感悟。有素習械鬥者,仁元曰:「民不畏死,然後可以致死。今天下多事,正此輩效順之時也。」簡驍健得六百人,日加訓練,土匪畏之,斂跡。樂平與鄱陽為鄰境,仁元政聲亦與衍慶相埒,至是南昌戒嚴,衍慶助剿,仁元移攝鄱陽以代之。未幾,衍慶以防賊擾,馳歸縣。因仁元父母妻子在樂平,亟趣仁元去。仁元曰:「賊旦夕且至,臨敵易令,是謂我不丈夫也。」遂議並力戰守。值久雨湖漲,城圮,無險可攖。於是審度地勢,衍慶軍南門,仁元軍北門,為犄角。部署甫定,賊揚帆大至。麾軍燃砲,碎賊艦,賊繞東門登岸,入城,衍慶迎擊,賊稍卻。又繞而北,仁元率樂平勇巷戰,矛刺仁元,踣,臠割之。所部猶力戰,死者過半,卒得仁元尸以出。

初,樂平土匪度仁元去必復來,伏不敢動。及聞殉難,乃倡議迎賊。仁元母顧其婦及女曰:「禍將及矣,曷早計!」皆死之。城陷,仁元父及弟並不屈死。事聞,詔贈知府銜。與衍慶合祠於鄱陽,別於樂平建仁元專祠,父予墀、母陳氏、妻金氏、弟誠元、妹三人、妾楊氏及僕、婦等均附祀。

李福培,字仲謙,江蘇無錫人。道光二年舉人。會試十三次不遇,考教習,補左翼宗學教習。期滿,用知縣,咸豐元年,選授廣東從化縣。時廣西賊起,廣州為賊出沒所,從化界連七邑,距府城百七十里。四年,賊偪廣州,福培以花縣之石角及縣境之太平場為從化及諸邑屏障,請大府屯兵二千,兼可斷賊糧道,不報。乃自募壯丁數百人,與典史趙應端及從弟性培分將之。七月,賊數千直薄城下,福培登陴固守,率兵民力戰,凡七捷,斬八百餘級。九月,援賊大至,急解縣印授其子送省會,而誓以死守。賊舁砲攻城,裂數丈,賊蜂擁入。與應端、性培等巷戰,身受數傷,退至學宮尊經閣,猶投石殪賊,賊焚閣,三人同死之。僕周鏞、勇丁蘇兆英等皆殉難。恤福培贈知府銜,建專祠,特謚剛烈。福培就義處有血影漬地,如人形,濯之愈顯,後任建石欄護之,榜曰「忠跡昭然」。

王恩綬,字樂山,亦無錫人。與福培為中表昆弟。少以諸生受知巡撫林則徐,招入節署讀書,稱為篤行君子。道光二十九年順天鄉試舉人,考充宗學教習,勤其職。惠親王稽察宗學,語人曰:「不視此職為具文,孜孜不倦者,王教習一人而已。」期滿,以知縣候選。恩綬幼與福培同學,長以氣節相砥礪。同居京師,夜分論時事,慷慨罵諸將吏棄城與軍,輒面發赤。戟手搏案,聲震鄰舍,童僕為驚起。福培仕廣東,恩綬與之書曰:「大丈夫當此時,與其老死牖下,孰若埋骨疆場耶?」及福培殉,益躍躍欲得一當。

咸豐四年秋,武昌克復,大吏以湖北缺員,請吏部揀發選人。方是時,武漢再陷再復,寇尚蠢蠢至,選人皆畏沮不欲行,多稱疾謁假。恩綬慨然曰:「若仕必擇地,則夷艱搘危杖節之士不復見於今世矣!寇何由平?」冠帶往聽旨,果發湖北。或言「寇深入,道且梗,盍徐徐行」。恩綬不可。攜一子一僕,間道疾驅,五年二月始至,則武昌已被圍。巡撫陶恩培嬰城守,兵弱餉絀,勢岌岌不保。官吏藉口出請援師,乞大吏檄引去者相屬。布政使胡林翼駐師城外,恩綬往謁,林翼惜其才,留贊畫軍事,恩綬辭,竟縋城入。恩培詫曰:「此旦夕死地,人患不得出,君獨患不得入,今何時,乃有此義烈男子耶?」溫語慰遣之曰:「君無守土責,尚可出,就胡營,留此身以待用。」恩綬固不可,遂奉檄登陴守御,翼日城陷,恩培殉黃鶴樓。恩綬與武昌府知府多山督兵巷戰,同時死之。仲子燮及二僕皆殉。

明年冬,武漢克復,當事以恩綬死事狀上聞,得旨賜祭葬,予謚武愍。既而御史汪朝棨疏言:「恩綬無守土責,而視死如歸,不特與草間偷活判若天淵,即較之城亡與亡亦分難易。且忠孝一門,僕從皆知赴戰,尤足扶植綱常。請於本籍建專祠。」會巡撫郭柏廕亦疏請建祠武昌,詔並許之。

李右文,字伯蘭,順天通州人。道光十一年舉人。咸豐三年,選授湖南東安知縣。粵匪犯天津,留辦本籍團練,以功賞知州銜。五年,赴官,值湘南道梗,諸弟馳書尼其行,不聽。至楚,權新寧。邑屢被寇,戶口流亡,右文招集撫循,凋敝以振。七年,以最調祁陽,時從弟載文殉難廣西,弟復馳書勸歸,慨然曰:「死生命也,脫捐頂踵報國,是得死所也,何慮為!」尋回東安任。八年,湖南境賊退,右文謂眾曰:「賊敗他竄,不可恃。」亟訓練民團,置倉穀數千石,備不虞。

九年春,石達開由江西回竄湖南,逼近東安,新寧紳眾數百人來迎,請避賊新寧。右文曰:「吾去,誰為守此土者?」已,復請護家口出境,又曰:「是為民望也。」卻其請。眾泣,誓死不忍去。三月,賊麕至,城卑,四面皆山,賊環瞰之。右文集城中官民登陴固守,親冒矢石,歷七晝夜,轟斃城下賊甚夥。城陷,與賊巷戰署東,身被重創,猶手刃數賊,力竭遇害,賊燔其尸,僅得脊骨歸葬。子傑、妻郝、子婦王,及僕婢,皆從死。新寧紳眾數百,亦先後戰死。詔視道員例賜恤,建祠本籍,隨殉親丁、紳勇附祀。

子傑,字小蘭。縣丞。有幹略,侍父湖南,襄督練勇,進知縣。方賊之回竄也,右文知不免,作書與諸弟訣,命傑齎往,意欲生之也。傑不忍去,又重逆父命,潛避署左右,觀賊變。賊至,率練勇守南門,城陷,聞警馳父所,未至,遇害,尸同被焚。視同知例賜恤。

載文,字潞帆,右文從弟。道光二十四年舉人。三十年,以知縣發廣西,咸豐元年,權馬平縣。時洪秀全犯桂林,馬平賊李志信響應,載文率兵剿捕,殲之。尋調平南,五年,艇匪梁培友由梧州上竄,陷潯州,擾平南,載文禦之渡口,砲轟沉其船,追擊斃匪無數。賊屢分撲南北岸,悉卻之。累以功擢同知直隸州。

六年五月至七月,賊麕至,水陸環攻,載文偕參將曾廷相、張遇清,都司唐文燦等,嬰城固守,困重圍七十餘日。乞援、乞餉,告急文數十上,大吏但空言慰藉。載文知事不可為,遣親僕間道以縣印檄送桂林,獨激勵兵勇與賊相持,教諭傅揚清,把總呂耀文,生員傅揚芬、吳國霖先後戰死。賊攻益急,載文中砲傷腿,痛哭,北面頓首曰:「臣力盡,惟以一死報國,然不忍百姓屠戮也。」縱之去。千總方源開城私遁,賊乘隙入。載文、廷相率勇巷戰,手刃數十賊,力竭,自剄不殊,賊擁至船中,抗罵不屈,並臠割之。

是役也,遇清守北門,持大刀斫賊三十餘,被賊攢刺無完膚,死。文燦守南城,率外委張珽巷戰死。守備張彪守火藥局,燃火轟斃賊百餘,亦戰死。載文、廷相死尤慘。先是巡撫勞崇光奏薦載文堪勝道府,兵部侍郎王茂廕亦奏保循聲卓著,擢桂林遺缺知府。命下,載文已遇害。贈太僕寺卿銜,賞世職,建祠本籍。同治十年,追謚壯烈。

李榞,字紫籓,安徽宣城人。以監生入貲為知縣,道光二十六年,選授湖北公安,賑災有惠政。調孝感,再調鍾祥。咸豐二年,粵匪自長沙躪岳州,犯武昌,所在奸民競起,鍾祥馬騾子、襄陽郭大安、天門蓋天王皆盜魁,黨眾大者萬餘,小乃數千。榞教練壯士千餘人,捕馬騾子及其黨數十人斬之。偵知郭大安方謀以眾投粵賊,設伏間道擒之。乘大霧掩擊蓋天王,悉俘其眾。時武昌、漢陽相繼陷,上游諸郡帖然無恐者,榞平諸盜力也。既而武昌復,大吏上榞功,擢荊門州知州,調署江夏縣,鍾祥民萬眾攀留不得。

會粵匪林鳳祥等北犯,其後隊自河南折入湖北,陷黃梅,趨麻城。榞率提標兵千人往援,擊賊黃岡之鵝公頸江口,大破之,窮追至安慶,與安慶兵夾擊,殲賊殆盡。還值宿松警,復破賊下倉埠,詔以知府升用。逾月,賊復自江西大至,寇廣濟之田家鎮,大吏檄榞往,連戰皆捷。最後戰,他將懦不進,榞率所部渡江擊賊。賊敗走,孤軍追賊,至興國州富池口,賊知榞軍無繼者,分舟中賊登岸襲其後。榞引就水軍,水軍走左,陷淖中,與所部二百人皆鬥死,咸豐三年九月十日也。事聞,贈道銜,予世職。公安、鍾祥之民,家祭巷哭,奉木主祀之。

始榞為縣,所至必於其地夷險豐耗、民俗醇訛、奸蠹根株、人所疾苦盡知之。為治行之出於至誠,人樂為用,原效死力。及其殉難,久而思之。同治二年,湖北大吏復奏榞死事甚烈,在官政績尤著,請宣城及死事所建專祠,詔可,予謚剛介。

陳肖儀,字幼泉,江西弋陽人。嘗遭母喪,扶柩舟行江中。夜火發,四面皆烈焰,肖儀以身伏柩上,隨江流飄蕩,不死,柩亦無恙,一時稱奇孝。年十九,官湖北縣丞,擢廣濟知縣。咸豐三年,粵匪破田家鎮,去縣七十里,縣故無城垣,召募鄉兵,皆望風走。肖儀知事不可為,持刃坐堂皇,賊入,數其罪,即抽刃自剄,未殊。賊縛之,曳於市。子恩藻奮臂擊賊,賊立殺之,肖儀罵益烈。賊鑿齒刳頰,膚盡見骨,三日乃死,賊解其體為五。縣民悲憤,賊去始斂焉。

萬成,滿洲鑲白旗人。道光二十四年舉人,揀發湖北知縣,署漢川,調安陸。咸豐四年,匪由武昌北竄,陷雲夢。時總督臺湧駐兵德安,萬成陳戰守二策,湧不能用,欲退守三關,徐圖克復,且諷與俱去。萬成垂涕曰:「棄而不守,如百姓何?與城存亡,守土之義也!」其僕復勸之,並以主人無嗣為辭,萬成厲聲曰:「我家世受國恩,若臨難偷生,無以對國家,即無以對祖、父!」遂致書邑紳曰:「禍在旦夕,誰之責歟?一死塞責,不可為臣;有辱於親,不可為子。原不歸櫬於先人之墓,留葬於此,以志吾恨。」是夜警報沓至,萬成召團練諸紳,告以在城兵勇俱隨總督北發,己當以死守城。又知事必不濟,復作絕命書,與士民訣,略曰:「賊已至雲夢,勢必來德安,我惟攖城固守。不能,則以死繼之。諸君不我遐棄,能尋我遺骸,葬於碧霞臺下,常此北望神京,則九原之下,感不忘矣!」逾日,賊距城二十餘里,臺湧擁兵徑去。萬成謁知府議救急策,甫出署,紅巾賊數十突至。知城陷,抽佩刀與戰,手刃數人,力竭死之,賊焚裂其尸。德安復,縣民卒收葬殘骨於碧霞臺下,以遂其志。

袁祖德,字又村,浙江錢塘人。祖枚,以詩文名,官江寧,因家焉。祖德早慧,入貲為江蘇寶山縣丞。兵備道某稔其才,以上海縣令姚某漕事詿誤去,檄祖德擢縣事,且代姚辦漕,未五月,難作。先是縣中團練多閩、廣無賴,本地游民和之,漫無紀律。粵匪據江寧為偽都,人心益搖,於是小刀會起事。小刀會者,即無賴游民所結合,黨羽散布,官役皆為耳目。道故粵產,謂中多粵人,置不為備;先發難嘉定,戕縣官,道仍不為備。咸豐三年八月初五日為上丁祀事日,黎明,祖德肅衣冠出,賊蟻擁入署。一賊號小禁子者,祖德嘗因案懲之,首犯祖德,刃交於胸,被十餘創,罵不絕口,死。

守備李大均得訊,躍馬呼殺賊,手無械,不能戰,自經死。

於松,漢軍正黃旗人。以廕授藍翎侍衛,出為江蘇松江糧。咸豐元年春,南漕改海運,漕船水手將譁變,大吏檄松資遣,變遂定。明年,大吏復以資遣事檄松,時粵賊已踞江寧為偽都,水手環而嘯呼,勢倍前。松為上息內閧計,藉其精壯而訓練之,不旬日,得勁卒二千人。會向榮躡賊圍江寧,江蘇境內稍安。六年,率所籍卒從巡撫吉爾杭阿剿鎮江,既成營,搏賊銀山下,戰屢捷。鎮江賊仰息江寧,既屢創,閉壘,潛略高資鎮。松以千人馳擊,渡夾江,平賊營。改攻鎮江城,以眾夜薄城下,梯垣縱火,潮勇噪而驚賊。賊起,燃巨砲,登者紛墜。松督隊在前,鉛丸中額,僕牙旗下,旋卒。潮勇故剽椎名盜,居嘗啗賊金,故為賊用,敗官軍。松死,麾下士千餘人,悲憤痛哭不忍聞。

尚那布,國羅落氏,滿洲鑲黃旗人。咸豐三年,由舉人揀發江蘇知縣,八年,署溧陽。僕從蕭然,日集士紳議戰守,不退食。兵勇踐境,親立城卡彈壓,出境乃已。創義學,築舍數十楹,集諸生講肄,購田百餘畝供膏火。修葺文廟,庀材鳩胥,捐廉為之倡。疏濬城河,懋遷稱便。迭以軍需籌防、催徵力最,賞知府銜。十年賊陷廣德,溧陽界其北,尚那布誓死守。賊眾逼城下,急切無援,督練勇擊賊退。未幾,賊復大集,攻城愈迫,越日城陷。尚那布厲聲叱曰:「我溧陽知縣,練勇殺賊,我作主,速殺我,勿傷百姓!」遂遇害。恤贈太僕寺卿銜,賞世職。時署金壇縣知縣李淮同以城陷殉難。

淮,字小石,浙江鄞縣人。固守至百餘日,賊乘霧登城,淮朝服坐堂皇,罵賊死之。

唐治,字魯泉,江蘇句容人。道光五年舉人,大挑知縣,分安徽,補桐城縣。歲大水,請帑勸分,按口賑施,不假手胥吏,一月須發為白。調祁門,舊有東山書院,生童膏火取給鹽釐,治別籌捐項充經費,士商兩便之。又立義廒,積穀至數萬石。時粵賊據江寧,安徽改省治廬州,賊船上下無所忌。上書陳利害,不報;祁門無兵,依山為城,徽州以富名,賊欲圖徽,必道祁,請以兵守,又不報;而祁之奸民前苦治嚴緝者,遂為賊鄉道。道光四年正月,賊入縣屬櫸根嶺,治招集團丁,激以大義,誓共城存亡。時大洪司巡檢鍾普塘亦帶勇入城協守,賊偪西門,治督眾登陴迎戰,砲轟斃匪數十人。大股猝至,城遂陷,猶奮勇巷戰,力竭馬蹶,與普塘同時被執。誘降不可,凌辱之,不屈,以禮遇之,終不食飲,卒罵賊死。普塘同時遇害,沉尸於河。

普塘,紹興人。賊欲說降之,曰:「吾年逾六十矣,即不知羞恥事,能再活六十餘耶?」傳其罵賊尤烈云。同治二年,曾國籓請於祁門建專祠,以鍾普塘附祀。

賊躪安徽,守土吏殉節死者,又有泗州知州鄭沅,六安州知州金寶樹,蒙城縣知縣宋維屏,望江縣知縣衛君選,盱眙縣知縣許垣。沅,順天大興人;寶樹,江蘇元和人;君選,河南趙城人;垣,江蘇上元人;維屏籍未詳。

林源恩,字秀三,四川達州人。拔貢生。舉道光二十三年順天鄉試,咸豐元年,選湖南平江縣知縣。二年秋,粵賊犯長沙,瀏陽、通城匪徒皆為亂,三縣皆與平江接壤。源恩詰奸守隘,如防禦水,截然不得蟄。江忠源以為才,保奏知州銜,又以書播告士友,道「林某堪軍旅」也。時曾國籓治兵長沙,檄源恩募平江勇五百人以從。旋有他賊自崇陽、通城犯平江,檄源恩回援,壁北鄉之上塔市。三月四日,賊大至,環源恩壘,源恩逆戰,大捷,追奔數十里。既而塔齊布、胡林翼師克通城,平江解嚴,師別剿,則賊仍麕至,源恩屢戰卻之。會有忌源恩者,功不得敘,又別摭他事中之。源恩憤甚,詣大府自陳,而謇於辭,卒莫能自達。

遂從國籓九江軍,命治羅澤南糧臺。乙卯春,從克廣信,賞花翎。又治塔齊布糧臺,旋任水師營務。十一月,又攝理陸軍於廬山之麓、姑塘之南,而江西巡撫聞源恩賢,飛檄至南昌,付以所新募之平江營者。源恩在廬山,又與共事武夫不相能,憤彌甚,嘗獨嘆曰:「丈夫一死強寇耳,終不返顧矣!」

明年,石達開犯江西,連陷八府五十餘州縣。六年三月,李元度率師自湖口南來,源恩與鄧輔綸自南昌而東,兩軍會於撫州,克進賢東鄉,進破文昌橋堅壘五,赭其巢。既薄城,源恩壁南門,元度壁西南隅,相去四里。賊嬰城拒守,堅不可拔。當是時,江、楚道梗,瑞、臨、袁、吉四郡無一官軍。援賊不時至,至則合城賊來犯,所部迎擊三十里外,輒重創之,破賊壘者九,大小戰五十有六,皆告捷。然部下血戰久,疲不得休,裹創者十之三,病者十四五。會輔綸中蜚語去,在事者多告退,源恩勢益孤,餉日絀。

宜黃、崇仁兩縣來乞師,謂克宜、崇則能拊撫賊之背,且勸士民輸餉,可得十數萬。源恩與元度遂分江、楚軍共五千徇西路。九月三日,克宜黃,九日,克崇仁,俘斬各數百。忽皖賊數千自景德鎮來援,急撤宜、崇軍,官民苦留不遣。將士亦以久饑甫得一飽,不能行。賊趨撫州,十六日,扼河而戰。水涸,賊駷馬飛渡,追而敗諸城下。

先是源恩所部之右護軍遣赴崇仁,留三百人守壘,賊詗知之,詰旦出犯,先陷右軍,遂圍源恩壁。源恩慷慨諭將士曰:「好男子,努力殺賊,無走也!」眾皆應曰:「惟公命。」都司唐德升馳入壁,掖源恩上馬,源恩曰:「此吾死所也,子受事日淺,其行乎!」德升曰:「君能死,吾獨不能死耶?」從容解金條脫畀其從子某,曰:「若馳去,吾與林公死此矣!」壘破,源恩手劍鏦賊,力竭,死之。德升素驍健,格殺十餘賊,始被害。從死者三百餘人。源恩年僅四十。追贈道員,賜恤如例。

德升,字彥遠,寧遠人。舊隸副將周鳳山部下,以十五日奉檄來軍,十七日及難,贈游擊。

畢大鈺,湖南長沙人。咸豐二年,以附生守長沙南關,粵賊砲轟城塌,大鈺斂空棺實土為墻,頃刻成三十餘丈。隨提督鄧紹良堅拒八十餘日,殲賊數千。賊自湖北回竄,湘潭、靖港均陷,大鈺復以防省功選用府經歷、縣丞。湖北崇陽、通城陷,大鈺復領兵赴剿。諜知賊由平江搗長沙,絕饋道,厲兵為備,賊不得逞。行軍禁騷擾,一蔬一木無妄取。通城亂久無官,為立團防,鋤土匪,通人安業。因其歸,報金巨萬,大鈺卻不受。四年,保用知縣,授浙江仙居知縣,案無留牘。地瘠民貧,逋賦多,大鈺在官,民爭輸納。尋捐知府,浙江巡撫何桂清留筦糧臺,檄赴於潛防堵。又以開化疊警,調防婺源。初戰屢捷,尋賊以三千人圍南關,大鈺偕胞侄候選通判榮清合剿,賊大至,力竭,均死之。恤贈太僕寺卿,賞世職。

湯世銓,字彥聲,順天大興籍,江蘇武進人。道光二十六年舉人。咸豐三年,以知縣發浙江,七年,署開化縣。時粵匪闌入浙境,由常山窺開化,委署者多不肯往,世銓獨毅然請行,至則募勇防守。八年三月,賊首石達開擾浙,衢州鎮總兵饒廷選戰敗,遂偪開化。世銓聞警登陴,賊突至,城陷,世銓拔佩刀自剄,為紳民奪刀擁出,不得死。陰約各都結團,且飛書請兵,會鶴麗鎮總兵周天孚督軍追擊,賊奔處州,世銓率團沿途截殺。

六月,縣城復,仍因失守褫職,代未至,仍帶勇守御。七月,賊由常山復攻開化,江蘇候補知縣劉福林帥鄉勇方檄赴寧國。世銓請於大府,留籍防禦,而以城守囑縣丞某,且出印印其衣,畢,遣人賚印至府授代者,遂出禦賊於華埠。賊至,疊擊敗之。會貴州定遠協副將硃貴統兵三千夾援,戰失利,世銓急整隊出,倉猝不能成陣,力鬥,與福林同歿於戰所。以印衣覓得尸,胸腹腰肋創十數。勇目方忠同死於其側。事聞,復原官,恤如例,給世職。

謝子澄,字雲航,四川新都人。道光十二年舉人,大挑知縣,分直隸。咸豐元年,署無極縣,二年,補天津。天津地濱海,獷悍難治,市有所謂「混混」者,健武善鬥。子澄至,見前令系諸混混,嘆曰:「是奚不可化者?」籍其名,縱之。未幾,縱者閧於市,子澄按名捕,殛其魁,地面遂靖。時粵匪出擾湘、漢,順流而東,遣酋林鳳祥、李開芳分兵渡河,莫測所向。人方謂南北道隔,賊不敢犯,子澄深以為憂。捐金倡團練,召所縱諸混混,以周處故事喻之,眾皆為用。回民劉繼德復集回民千餘人應之,遂率赴教場,授器械,教戰陣之法,其妻亦撤簪珥以助。長蘆鹽政文謙歸財與糧,隨時協濟,子澄得一意練兵。

未幾,賊圍懷慶。逾月,渡臨洺關,總督納爾經額帥師遁,遂經順河、柏鄉、欒城入深州。主閫者務持重,雖數奉詔夾剿,而習於潰逃,數避賊。其奮勇者尾追數千里,氣亦餒,賊勢益橫。又經獻縣、交河,以薄滄州,滄州號有備,亦為所拔。津地大震。

九月,賊至梢直口,大吏不知所為,議嬰城守。子澄以負郭居民數十萬,不應棄之,力爭。遂用沿河棹小舟以火器取野鶩者,又火會會眾萬人,合水陸拒賊,而別向火會首事張錦文籌貲。先是錦文輸家財濬壕,壕成,運河水大至,環城窪下成巨浸,而葡萄窪尤甚。子澄阻壕守,渡壕擊賊。賊酋開山王小禿子手黃旗指揮,迅奮剽疾,能一躍丈餘,避槍擊。子澄先伏打野鴨船於岸外,賊以為民船也,呼渡,船槍發,殪小禿子,群賊奪氣。伏舟進擊餘賊,血流染波。日晡,軍餒,錦文又齎糗糧至,戰益奮。勇目餘鵬龍等相繼陷陣,復斬級無算,賊遁。是役也,子澄功最,旨以知府用,留本任。

時賊退踞靜海及獨流鎮,子澄奉調赴勝保營,列營河西。賊由獨流出撲,屢擊退之。嗣靜海賊傾巢出援,子澄追剿,賊竄,正窘,會都統佟鑒思絕賊歸路,進掣壕板,以路滑失足踣地,賊刃交下。子澄單騎馳救,砲洞馬腹,身受七傷。鵬龍負之趨,子澄曰:「憊矣!爾亟行,毋顧我。」賊酋高剛頭薄之急,子澄恐為所辱,沉於河。鵬龍率從子陳梁等皆戰死。事聞,加布政使銜,謚忠愍,建專祠。喪車還津,無貴賤皆往吊,哭如私親。天津祠落成,蠡縣人李某,生致高剛頭,剖心以祭。

子澄好為小詩,工駢體文,為政有聲,卒以殺賊致殞。人謂賊自河北經山西,所至席卷無堅城,獨受挫於子澄,使京師得以為備,其關系尤重雲。

先是賊過臨洺關,同知周憲曾公服坐餉鞘上,罵賊死。後子澄以知縣死直隸者:江安瀾,廣西臨桂人。舉人,挑教職,保知縣,發直隸,補柏鄉。咸豐元年,調靜海。賊北犯,靜海為畿南沖要,大軍援剿,供應無乏。賊入境,偕署都司潘宗得等擒斬偽司馬陳得旺,大隊麕至,官軍眾寡不敵,遽潰。城陷,赴水死。破沙河,王衡身中七刃死;破欒城,唐盛朝服罵賊,賊縛之柱上死,典史陳虎臣從死。

又馬雲嵐,慶雲人。州判。賊犯縣城,率鄉團出御,被執,不屈死。子龍文從死。恤如例,予世職。

文穎,字魯齋,趙氏,漢軍正藍旗人。道光二十五年進士,用知縣,發山東,補蒙陰。邑患蝗,兩以文籥神,皆應。調陽信,弭抗漕釁。又調商河,濬徒駭河,境免積潦。時粵匪已竄直隸之建通鎮,去商河百里,募練鄉勇,民恃無恐。調省主糧臺事,適股匪入東境,金鄉、鄆城皆陷,而陽穀當其沖。大吏以文穎有幹才,檄令往署,至則城備久弛,急號召鄉團為守禦計。是時將軍善祿擁重兵駐東昌,飛牒請援,置不應。憤極,抵案曰:「死耳,復何言!」或諷以出城待援者,怒斥曰:「與城俱存亡,豈有臨難茍免之文某哉?」

未幾,賊大至,割半袖付僕馳報父母,即懷印上城,與典史徐鳳喈從容出印相視。賊入城,怒馬馳入賊隊,被七創,罵不絕口死。鳳喈及教官李文綬同遇害。文穎抵任才五日,時咸豐四年二月二十九日。事聞,優恤,立專祠,予世職。文穎嘗過泰山,題句有云:「此行不了封侯業,原把頑軀竊比君。」蓋以泰山自矢,見危授命,其志素定云。子四,三爾豐,自有傳。

張積功,江蘇儀徵人。嘉慶二十三年舉人。道光十年大挑知縣,發山東,歷州縣吏。二十年,初權臨淄。前政不善,多流亡,以誠招徠之,皆歸故業。即墨饑民滋擾,檄往辦理而定。朝城民變,民聞積功治臨淄事,即首行館請死,喻以理,懲以法,皆歡呼去。咸豐四年三月,賊攻臨清州,積功適知州事,守御十四晝夜。十四日,城陷,闔門死難。初,賊過冠縣,知縣傅士珍自經死。

典史瞿濬,字菊坪,江蘇武進人。帥鄉勇出敵,遇賊城闉,中鳥槍,洞其肋,墜馬。欲退保於司獄,賊追及,刃倳其胸,罵不絕口,剖腹死之。妻呂氏,罵賊,被寸磔。亦全家遇害,時三月朔也。

冒芬,江蘇如皋人。巡檢,發廣東,補北寨司巡檢,調五斗口。緝獲盜匪傅敏南、烏石姊等,有能名。擢廣州府經歷,調海豐縣丞。英吉利擾廣州,以守城功進知縣,授開平縣。縣介新會、鶴山間,盜賊出沒,芬嚴為條約,捕甚多。歷權高要、曲江、乳源等縣。

咸豐二年,洪秀全陷仁化、樂昌兩縣,分股攻乳源,芬募勇三百,約都司車定海扼河為守,使鄉勇繞出河岸設伏。凌晨賊至,官軍隔河砲斃騎馬賊一,伏軍薄其後,夾擊之,賊大潰。渡河追擊,斬甚眾。

餘匪吳煥中、黃老滿等潛聚曲江龍歸墟,結連羅鏡墟凌十八,圖復逞。煥中潛至乳源,為邏者獲。芬訊得實,偕千總張鷹揚馳往,捕獲黃老滿等頭目十三名,解經曲江寺前村,猝與羅鏡賊遇。鷹揚所部潰散,芬率親軍百餘人與賊戰,軍火盡,芬被創,賊奪黃老滿去。芬裹創為書,上總督葉名琛,極言兩粵賊勢急,宜聯絡官民,早繕備具。越數日,傷劇卒。恤如例,後建專祠。

施作霖,浙江蕭山人。道光二十九年拔貢,用知縣,發陜西。咸豐三年,粵賊竄河南,奉檄督練勇防陜境,署城固縣知縣。七年,河南角子山捻匪擾南陽府,將竄陜,巡撫曾望顏以作霖練勇有紀律,令防商南。馳抵清油河,距武關三十里,賊已潛由天橋河陷武關,作霖偕候補同知曾兆蓉夜冒風雪抵頭條嶺,擊卻賊前敵。越四道嶺,賊蜂至。作霖奮下擊,義勇厲進,作霖手殲悍賊王黨。餘賊卻拒守關,作霖直逼關前,賊復三面撲。熸賊二十餘騎,賊攻愈猛。作霖分隊擊,身受重創,力竭死。家丁王建、義勇馬永剛等十三人皆死之。賜恤,謚剛毅,賞世職,建專祠。

韓體震,字省齋,河南夏邑人。道光二十五年,捐州吏目,補直隸祁州吏目。因父作謀任文安主簿,回避,補山東德州吏目,捐升知縣。以防堵功,獎開缺即選,選湖北通城縣。防堵鄰境要隘出力,保同知。同治元年,鄂督官文調赴軍營差委,嗣權孝感縣事。孝感屢經殘破,城缺不完,體震修葺之,招鄉勇城守。閏八月,捻匪大股分擾京山、應城一帶,闌入縣境,遂撲縣城。體震與護軍統領舒保善因請入城同守,始解鞍,而賊由缺口入城,體震率勇巷戰,眾寡不敵,身受十傷,刀矛槍子無不備,大呼殺賊而死。詔照知府例賜恤,給世職。

德克登額,字靜庵,滿洲某旗人。由筆帖式從將軍都興阿軍,累保至副都統記名。嘗從攻廣濟,守營壘,不眠者七晝夜。為人沉靜,溽暑不去長衣,每曰:「賊平即回家授徒,暇則垂釣黑龍江。」又曰:「世受國恩,得一日授命疆場,則吾事畢矣!」與體震同守城,城陷,死之。

蔣嘉穀,順天大興籍,浙江山陰山。以府經歷發貴州,旋保知縣。咸豐三年,署荔波縣。縣毗連粵西,粵氛近偪,土匪乘之。嘉穀內守外御,境內安堵。始之任,獄多繁,囚半逆黨脅從,復有挾私誣告人從匪者。嘉穀訊得實,俱決釋之。時芻糧告匱,或以勸捐進,嘉穀曰:「民被蹂躪久矣,忍朘其生而激變乎?」事遂寢。五年六月,水匪復叛,與廣匪合,約五六千人,薄城下,嘉穀募勇五百人擊退之。時土匪遍地,餉需匱乏,嘉穀毀家募勇,妻陸氏亦出金義釧佐軍,眾感奮,守愈堅。以故附近州縣皆不保,獨荔波得存。十月,賊復至,嘉穀部署城防,誓師出營於水堡,與賊遇,戰捷,賊小卻,後見師乏援,始無忌,麾眾並進。嘉穀鏖戰終日,傷亡略盡,猶裹創刃賊,俄被執。賊乘勝攻城,城以有備,卒不破。嘉穀既陷賊,怒罵不屈。賊束薪漬油遍體灼之,死而復甦,甦則罵,罵則復灼,如是數次,乃絕。貴州巡撫蔣霨遠以嘉穀善政得民,力捍疆圉,被害尤慘。奏入,恤世職。紬士請捐建專祠,允之。

鄧玲筠,字治薌,湖南寧鄉人。道光二十三年舉人。咸豐六年,以知縣發貴州。七年,擢知印江縣,時黔中苗、教匪充斥,匪酋以邪教蠱亂,民有黃號、白號等目,鄉團多叛應之。玲筠銳意圖治,周巡轄境,與田更畬叟握手詢利病,手疏小冊,用是能摘發民隱,訟者神之。思南賊熾,地連印江,亟行保甲法。單騎詣各鄉,手自敦率,給門牌如式。署紙尾十則:曰忤逆,曰習邪教,曰私結盟黨,曰劫掠,曰藏匪類,曰竊盜,曰容留娼妓,曰賭博,曰鬥毆生事,曰唆訟。各擇士紳董之,犯者同甲勿與齒。改悔者許具狀於各條下,加小印曰「自新」;其頑抗及無人敢具保者治之。且計月以驗紳董之能否,加勸懲焉。又加意課士,割俸給書院餐錢,與講求正學,並及軍政,士皆畏愛之。勸民修水利,立法詳盡易曉,或親履指示,不以勺水擾民。邪教惑眾,為文告抉摘其謬,婦孺能解。簡壯丁數百,親教之擊刺法。

是年十二月,賊陷思南,將犯印江。印江故無城,出營於雲泮禦賊。賊以書請假道,焚書,斬其人。賊從間道襲治所,玲筠袖銅椎斃三賊。賊環攻,復出銅鐧格鬥,賊莫敢近。忽四山火起,乃突圍,抵銅仁乞師,得練總王士秀領五百人,一日夜行三百里。民見玲筠歸,奮躍,復得壯士千餘,仍從至云泮。是日大霧,人馬對立不相見,譟而進。賊奔,自相蹴蹋,墜崖死者無算。復追百餘里,戰中壩,戰螺生溪,戰袁家灣,皆捷。

八年春,知府令玲筠越境剿賊,知府先聞賊畏玲筠,立鄧字旗懾賊,故嚴檄三至。縣民苦留,玲筠慨然曰:「郡守檄,縣令安敢違?且殺賊固無分畛域也。」以千三百人往。師次分水埡,賊混運糧者入營門,變作,眾驚潰。玲筠親搏戰,飛石中首,手格殺一賊,足後被創,遂及於難,喪其元。後軍聞失事,憤極,殊死鬥,殺聲與哭聲並,卒奪玲筠尸還。乃樹「忠憤」幟,誓復仇,賊懼,退屯八十里。喪歸,士民大慟,爭致賻賵。有負販傭,挈錢四,將運鹽,悉以充賻。或曰:「如爾家何?」傭哭曰:「公死,吾屬無葬所矣!何家為?」民懷其德,立祠祀之。並刻遺集,曰鉅業堂稿。

承順,佟佳氏,漢軍正藍旗人。由文生於咸豐四年隨其父甘肅寧夏鎮總兵定安出征湖北,累功擢至通判,發甘肅。歷權寧夏鹽捕通判、平番縣事,授甘州撫彞通判,所至有聲。同治元年,西寧撒回就撫,大吏以貴德孤懸大河以外,漢民與番、回雜處,治理不易,檄承順往署。適番、回械斗,承順為之平怨息爭,番、回悅服。值河州回匪倡亂,甘、涼、寧、肅一帶響應,貴德回民洶洶欲動,承順勸導解散,以被難婦孺置署中別院,撫養數年。有主者認還,無主者擇配。由是漢、番感戴,回民亦懾其威。

時西寧所屬各相繼淪陷,貴德一孤立賊中者六年。城中回民暗結陜回謀亂,承順密調兵勇入城,嚴為之備。回首馬朵三等率眾千餘人攻城,承順登陴抵御,砲石雨下,斃賊頗多。城內回民開門應賊,城遂陷。承順率勇巷戰,身受重創,厲聲罵賊,賊怒,斷其左臂,罵愈厲;復斷其右足,罵如故;遂斷其首而支解之。其弟議敘知縣崇順、監生吉順扶其母薩克達氏至尸所哭詈,皆遇害。家丁李文忠等七名,同時死之。事聞,恤贈道銜,給世職。

貴德士民復以死事狀赴都申訴,御史吳可讀疏言:「青海辦事大臣玉通疏報,祗及承順被害情形,猶惑於當時『回眾拘集漢民、勒寫官偪民反,漢、回同謀戕官』之說,後經查覆,於精忠大節,仍未述及。在承順為國捐軀,光明俊偉,於原遂矣。遺愛在民,漢、番男婦老幼呼為活佛。誤觸其名,即童子皆呵禁之。在朝廷為有臣,定安為有子,甘肅為有官。闔門全節,允為一代完人,再懇優恤。」光緒初元,陜甘總督左宗棠覆奏,謂:「承順死節奇偉,一時僅見。綱常名義,不因品秩等差而別,則表揚較名位尊顯為尤亟。請官為建祠,並予謚法,以勵人心。」疏上,允之,謚勤愍。

托克清阿,字凝如,滿洲正藍旗人。道光十四年舉人,大挑知縣,發甘肅,署環縣、安化知縣,及土魯番同知。以清查事鐫級。咸豐元年,捐復原官,補皋蘭。時回、捻擾陜、甘,土匪聞風響應。侍郎梁瀚治團練,疏薦之。總督樂斌亦以其任事果敢,檄署秦州直隸州知州,尋實授。同治二年,逆回竄甘南,州境戒嚴,托克清阿募壯勇,繕器械,力籌守御。賊竄秦安,率軍迎剿,屢挫賊。賊糾大股至,眾寡不敵,力戰死之。事聞,詔以道員從優議恤,秦州及本旗立專祠。後秦州承其規畫,防禦嚴密,境獲安全。四年,秦州士民以托克清阿忠貞孝友,慈惠嚴明,潔己愛民,御災捍患,在任時民皆安業,賊不犯境。遺愛餘威,實足固民心而寒賊膽,籥請加恩賜謚。總督恩麟據以入告,特詔允之,予謚剛烈。

馮元吉,字景梅,浙江山陰人。由供事議敘從九品,分廣西,歷署貴縣五山汛、凌雲平樂司巡檢。道光二十八年,授宜山龍門巡檢。咸豐元年,金田賊由武宣東鄉逃竄,都統烏蘭泰、提督向榮、總兵秦定三等節節追剿。賊竄象州,兵勇不能御,直至大樂墟,轉掠龍門。元吉率鄉兵御之,戰敗,馳回署,衣冠坐堂皇,二子澍、溥侍立。家人請暫避,元吉厲聲曰:「身為命官,不能殺賊安民,走避偷生,吾不為也!」麾二子出,皆痛哭不去。賊至,父子抗罵,同遇害。家丁嚴祿、夏玉俱死。詔以元吉微員,從容盡節,澍、溥從父殉難,忠孝堪嘉。贈鹽運使司知事銜,賞世職,建專祠,澍、溥附祀。

平源,字沛霖,順天大興籍,浙江山陰人。由吏員敘典史,發安徽。咸豐二年,署懷寧縣典史,恤獄囚,嘗曰:「囚死於法,可也;死於非法,不可也。」眠食皆躬察之。粵匪犯安慶,事急,囚譁,欲脫械去。源至,囚曰:「此何時也,公胡弗自便?」源曰:「此若輩所以犯刑也,死可茍免耶?」囚曰:「公不去,囚何忍去?」俄而城陷,巡撫蔣文慶遇害,餘官皆走。源獨冠服坐獄門外,賊至,脅之曰:「若降,官;若不然,飲吾刃!」源曰:「刃則刃耳,吾豈受汝脅者?」賊曳至懷寧縣署外殺之,逮死罵不絕口。安慶人思之,為立石於殉節處。

時又有張寶華者,為望江縣典史。聞城陷,視其妻賈氏自經畢,冠服坐堂上罵賊,死。華陽鎮巡檢王泗同時殉難,盱眙縣典史周來豫後於九年助守縣城,力戰墜馬死。

餘寶錕,江西德化人。附貢生,捐知縣。道光十六年,選授浙江景寧縣,以才力不及降調。復捐縣丞,發貴州。咸豐五年三月,署麻哈州吏目。四月,仁懷縣教匪楊漋喜竄麻哈,隨知州何鋌擊卻之。尋盜魁陳大陸糾苗匪來犯,復隨鋌出戰。賊退,遂率眾攻拔下司巖、下雞場等處,扼茅坪山,悉力堵御。未幾,賊聚益眾,勢不能敵。退州城,賊旋陷都勻府。提督孝順兵至茅坪被圍,寶錕率團兵隨總兵佟攀梅援剿,圍解。自是無日不戰,互有勝負。巡撫蔣霨遠檄雲南降將陳得功隨孝順攻克都勻,進援麻哈,官軍勢復振。得功旋叛去,孝順軍潰,賊大股圍州城三日,寶錕率鄉兵登陴固守,賊不得逞。七年,城中糧匱,兵益單,寶錕自誓與城存亡。八年正月,賊悉眾來攻,寶錕出北門迎敵,不利,入城,賊已自他門入。寶錕持矛巷戰,賊不忍害,揮令去。寶錕怒罵,掣矛刺之,賊奪矛還刺,死之。

王汝揆,甘肅伏羌人。道光二十年舉人,揀選知縣,親老改教職。咸豐間,授平涼縣教諭。同治元年,陜西回匪竄乾鳳,偪甘肅境,汝揆上書平慶涇道萬金鏞曰:「賊西偪鳳翔,必分黨由汧、隴間道趨秦安東北,構煽醜類。宜及其未至,扼險嚴防。不然,內應且四起,平涼擾則靈、固、狄、河等州縣亦危矣!」言未及用,賊尋由固關逾隴,張家川、蓮花城土回應之,陷鹽茶及固原,金鏞死之,平涼戒嚴。汝揆議盡毀城外民舍,無令賊倚為障蔽,議不行。未幾,賊圍平涼,汝揆協同守令,督率生徒,登陴固守,衣不解帶者六閱月。一日,偵西北二路賊少可擊,謁知府田增壽請率壯士縋城出剿,又不許。二年,賊匿民舍掘地道,納火藥轟之,城遂陷。人皆泣曰:「早從教諭言,事豈至此乎?」汝揆還署,易朝服,北向叩首訖,妻汪氏暨女一、孫女一皆死,乃從容就縊於孔廟鍾虡以殉。

汝揆性質實,敦孝友。居親喪,不入內,不御酒肉。弟印揆,客西寧久,音信乏絕,汝揆往尋之,風雪中徒步千餘里,卒挈其弟以歸。平生肆力於經籍,家居課徒,以窮經為急,輒點勘善本授之,勖以立品敦行。其官平涼,亦以是為教。期年,訟庭無士子跡。當城未陷之先兩月,有門人馳書勸引疾歸,謂可免難。汝揆曰:「無疾而稱疾,是欺也;食祿而茍免,非義也。」乃為書與戚友訣,略曰:「我生不辰,逢天癉怒,向者耳聞之,今則目睹之。平郡自二月以來,圍困日迫,飛書告急,援兵無一至者。汝揆妻、女,行當自盡,決不受辱於賊手。死者士之終,今誠獲死所矣。惟官卑不得展一籌以報國,死有餘憾耳。」三年,官軍克平涼,總督楊岳斌請優恤。六年,總督穆圖善疏陳汝揆死事狀,請照陣亡例議恤,贈國子監助教銜,給世職,又命於本籍建祠,以從死之妻、女等附祀。


\end{pinyinscope}