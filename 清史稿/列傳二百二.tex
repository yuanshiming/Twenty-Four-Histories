\article{列傳二百二}

\begin{pinyinscope}
黃翼升丁義方王吉吳家榜李成謀李朝斌江福山劉培元

黃翼升,字昌岐,湖南長沙人。少孤,育於鄧氏,冒其姓,入長沙協標充隊長。咸豐初,從征廣西,曾國籓創水師,調為哨長。四年,從楊岳斌下嶽州,敘千總。戰於城陵磯,賊以十餘舟來誘,翼升知其詐,追至擂鼓臺、荊河腦,伏賊突出,翼升駕舢板奮擊,後隊繼之,賊大敗。轉戰至金口,值賊下游被圍,力戰卻之。積功擢守備。克武漢後,進攻蘄州,翼升自蒜花出戰敗賊,焚其舟,擢都司。復蘄州,拔充營官。

攻湖口,毀賊船十餘艘。沖入內湖,賊塞隘口不得退,泊姑塘,迭戰於都昌縣河、雞公湖,焚賊舟。時水師在內湖者無大船,既與外江阻絕,曾國籓令添造,並撥江西長龍、快蟹諸船,以翼升及蕭捷三分領之,各為一軍。五年,屢會諸軍攻湖口,未克,蕭捷三戰死,翼升大憤,沖入賊卡,盡毀下鐘巖賊船。夜出奇兵數驚賊,賊不出,仍駐軍姑塘。

六年,賊犯撫州,南昌戒嚴,翼升奉檄泊吳城鎮,衛省城。湖口之賊尾至,結土匪窺吳城,翼升分兵由前河包鈔,自赴後河擊陸路之賊,走之。會彭玉麟至軍,令翼升專攻陸路,敗賊於塗家埠,毀浮橋二、船百餘。賊復冒民船來犯,翼升合軍圍擊,敗之。追至德河口,遂會攻南康,直薄城下,火賊船,城賊遁走。

七年,授直隸提標左營游擊。楊岳斌師至九江,彭玉麟與約夾攻湖口,軍分六隊,翼升率內湖右營當其沖,轉鬥而前。砲丸冒船過,他營失利,賊逐之,翼升待其還,縱擊,斬殺過當。賊復乘夜劫營,滅炬待之,殲賊無算,盡毀梅家洲賊船。東岸諸軍亦斷湖口鐵鎖,遂克湖口,內外水師復合。越日,進奪彭澤賊舟,破小孤山,擢副將。

九年,池州守城賊韋志俊投誠,彭玉麟令翼升往受降,賊酋古隆賢、楊輔清等來爭,擊卻之。旋有奸人內應,池州復陷。

十年,曾國籓規江南,奏設淮陽水師,薦翼升領之,即授淮陽鎮總兵。十一年,破賊於黃盆鎮,又敗之方村。進攻銅陵,決城東北堤,從決口入據之。又進攻無為州,毀泥汊口、神塘河賊壘,無為、銅陵同復,賜號剛勇巴圖魯。偕王明山循沿江郡縣,克池州,銅陵亦失而復得。運漕鎮濱江通湖,賊踞之以通接濟。翼升進擊,諸軍乘之,焚賊舟,賊遁銅城徬。又偕陳湜攻東關,克之,加提督銜。

同治元年,追賊入巢湖,賊聚湖口以遏歸路,翼升掘堤岸引船出,反拊賊背夾擊,大敗之,城賊遁。進克含山、和州。四月,會攻金柱關,李朝斌臨上游,翼升等遏下游,賊牽於水師,不暇內顧。曾國荃襲克太平,並趨金柱關合攻。翼升夜督隊逾壕,縱火焚西門,賊突出,揮士卒登岸短兵接,立克金柱關。襲東梁山,一鼓下之。移師攻蕪湖,賊棄城走。又擊賊清水河,俘馘千計,以提督記名。

五月,克秣陵關、江心洲諸隘,血戰奪九洑洲,軍聲大振。時李鴻章至上海,規蘇、常,翼升移師會剿,詔署江南水師提督,松江、上海諸水軍悉歸節制。翼升所部十營,分二營駐浦口,四營駐揚州,親率四營,六月,抵松江,就上海增造舢板、飛劃諸船,移守青浦。賊酋譚紹光合嘉、湖、蘇、昆諸賊圖犯上海,屢撲青浦,翼升與陸軍合擊走之。賊繞犯北新涇大營,又走吳淞,翼升駛往,相持至夜,毀賊營七。賊犯嘉定及青浦、張堰,分隊往援,且戰且進,至白鶴江,毀橋而還。翼升兵少,調揚州駐營來會剿。鴻章約合攻黃渡,翼升由趙屯橋截擊,追至三江口,盡平沿岸橋、壘。

十月,破賊蘆墟、尤家莊、汾湖、三官塘,進距蘇州三十里。常熟守賊駱國忠以城降,譚紹光來爭,陷福山,翼升赴援,進攻河西、白茅、徐六涇諸口。二年正月,翼升會常勝軍克福山,駱國忠見西山火起,突圍出,圍乃解。楊舍汛為沿江沖要,賊守之以蔽江陰,翼升沿江兜剿,迭破援賊,克之。乃會攻江陰,迭破蠡口、陳市。賊酋陳坤書來援,翼升扼江幹誘賊出戰,與郭松林、劉銘傳合擊,大破之。克江陰,賜黃馬褂。九月,由無錫進攻蘇州,詔翼升赴臨淮會剿苗沛霖,鴻章疏留勿遣。諸軍合圍蘇州,薄城下,當齊門、閶門之間,截賊竄路,城賊乞降,予雲騎尉世職。是年冬,再克無錫,率五營赴臨淮,苗沛霖尋走死,餘黨瓦解,翼升仍回江蘇。

三年,陳坤書犯常熟,偕郭松林等合擊,賊敗走。遣部將王東華等助攻常州,克之,被優敘,詔授江南水師提督。曾國籓奏:「江南額設提督一員,兼轄水陸。翼升所授,當是新設,請敕部鑄頒新印。」從之。會楊岳斌督師江西,翼升接統外江水師。江寧復,加一等輕車都尉世職。

四年,詔翼升赴清江浦防捻匪,至則賊已敗竄山東,進駐邳、宿之間。會僧格林沁戰歿,捻氛益熾,犯雉河,翼升駛援,賊又走。五年,回駐江寧。六年,調守清江,東捻賴文光敗竄淮安,翼升督諸軍追擊,文光為道員吳毓蘭所擒。東捻平,論功,被珍賚。七年,西捻張總愚竄畿輔,諸軍為長圍困之,鴻章調翼升率師船入運河設防。六月,乘伏汛入張秋口,至德州。張總愚奔至,冒官軍喚渡,翼升部將徐道奎察其偽,轟擊之,大軍環集,總愚溺水死。西捻平,加雲騎尉世職,合並為三等男爵。

長江水師營制定,仍以翼升為提督。彭玉麟終制回籍,長江事宜悉付翼升主之。十一年,詔起玉麟巡閱,劾不職將弁百餘人。翼升以傷病請代奏乞退,詔斥馭軍不嚴,濫收候補將弁二百餘人之多,念前功,從寬免議,許其開缺回籍養痾。光緒十五年,皇太后歸政,以翼升舊勛,予議敘,繪像紫光閣。十八年,復授長江水師提督,入覲,賜紫禁城騎馬。二十年,皇太后萬壽慶典,加尚書銜。日本兵事起,翼升由岳州赴江寧籌江防,卒於軍,賜恤,謚武靖,立功地建專祠。子宗炎,襲男爵,官廣西桂平梧鹽法道。

丁義方,湖南益陽人。入水師,隸彭玉麟部下,積功至守備。咸豐八年,克九江,擢都司。十年,克建德,賜花翎。尋建德復陷,賊數萬上犯湖口,勢甚張。義方收建德潰兵,簡精壯五百人,分布守御,自率水師駐西北門。賊乘銳攻城,義方登陴躬自搏戰,會副將成發翔來援,賊引去。曾國籓疏言義方膽識過人,部署迅速,詔超擢參將,加副將銜。十一年,駐防小池口,賊自興國來犯,擊卻之。馳援都昌,解其圍。同治元年,從彭玉麟迭克沿江諸隘,擢副將。二年,要擊都昌敗賊,毀其舟,尋解青陽圍,以總兵記名,賜號壯勇巴圖魯。七年,授湖口鎮總兵。光緒十九年,卒官。

王吉,湖南衡陽人。由馬兵累擢守備。咸豐九年,入水師,隸彭玉麟部下。從屯黃石磯,擊蕪湖賊,戰蟂磯、殷家匯、樅陽,皆有功,擢都司。十一年,從克孝感,戰最力,擢游擊,賜號猛勇巴圖魯。克德安、黃州,累擢副將。同治元年,金柱關之戰,吉率隊蛇行而進,躍上堤埂,破賊壘,以總兵記名。尋賊復由太平來犯,多方窺伺,吉駕飛劃入湖迎擊,又登岸馳逐。經月餘,賊蹤始凈。援無為州,率水勇登陸,會諸軍夾擊敗賊。破銅城徬水卡,結小劃船為橋以濟陸師。復破陶家嘴、大甲村、岷山岡賊壘。二年,曾國籓、彭玉麟合疏薦吉勇敢誠樸,堪勝總兵之任,授狼山鎮總兵。從克江浦、浦口,奪下關、草鞋峽、燕子磯諸隘,進拔九洑洲,以提督記名。八年,水師凱撤,乞假修墓,乃赴狼山鎮任。光緒七年,卒,賜恤。

吳家榜,湖南益陽人。入水師,初隸楊岳斌營。咸豐十年,從黃翼升破賊殷家匯,樅陽,遂歸其部下。菱湖、銅陵、泥汊口、運漕鎮、東關諸戰,皆有功,累擢守備。同治元年,從攻金柱關、東梁山、蕪湖,擢都司。從黃翼升援上海,迭破賊北新涇、四江口,敗援賊於江陰,賜號敢勇巴圖魯。領淮陽水師前營,克無錫,擢副將。三年,江寧復,錄功,以總兵記名。四年,追敘克宜興、荊溪、溧陽功,以提督記名。七年,從黃翼升赴直隸防運河。捻匪平,晉號訥恩登額巴圖魯,授瓜洲鎮總兵。光緒二年,兼署長江水師提督。十八年,卒,附祀彭玉麟祠。

李成謀,字與吾,湖南芷江人。咸豐四年,投效水師充哨長。從楊岳斌克湘潭、岳州,敘千總。轉戰湖北,敗賊於倒口,拔沿江木柵,毀鹽關賊船。克武漢,擢守備。從克田家鎮,成謀追賊,上至武穴,下至龍坪,往來擊賊,殲斃甚眾,擢都司。五年,從戰塘角,焚賊舟二百餘,乘風夜抵武昌城下,砲擊賊船,擢游擊。攻金口,循北岸進拔賊壘。又連破賊於壇角、占魚套,擢參將,賜號銳勇巴圖魯。

成謀身長八尺,力能一手豎大桅,素為胡林翼所器重。至是薦其每戰沖鋒,廉明愛士,堪勝水陸方鎮之任,詔記名,俟軍事稍閒,送部引見。

六年,扼沙口,斷賊糧道,破賊小河口、青山,燔其輜重。轉戰蘄州、黃州、廣濟、武穴,下至九江,毀賊舟數百,獲糧械以資軍用。武漢復,擢副將。七年,會攻九江,追賊至湖口,前隊銳進失利,成謀突入陣中,奪回所失四艘。尋授江蘇太湖協副將。既克湖口,從楊岳斌順流而下,登陸克望江、東流,疾趨安慶,復銅陵,會江南水師於峽口。紅單船方攻泥汊賊壘不能下,岳斌令成謀急棹薄壘,擲火焚其火藥庫,賊遁走,獲其糧械船艦。胡林翼奏「肅清江面,成謀之功為最,平日事親孝」,特給二品封典。八年,擢福建漳州鎮總兵。

十年,進攻池州,拔殷家匯,毀城外賊壘,破樅陽偽城,加提督銜。十一年,陳玉成圍樅陽,擊卻之。同治元年,會陸師拔巢縣、雍家鎮,薄西梁山,斷橫江鐵鎖,奪回要隘,以提督記名。破賊於魯港、採石磯,克金柱關、蕪湖,賜黃馬褂。三年,援湖北,破捻匪於羅田。五年,署福建水師提督,尋實授。

時軍事漸定,整頓營制,會奏裁金門鎮總兵,改為水師副將。裁左營游擊,移右營駐湄州,歸提標統轄。徙前營游擊駐谾口,後營游擊駐鎦門。變通巡哨章程。十一年,彭玉麟整頓長江水師,罷提督黃翼升,薦成謀樸誠堪膺重任,即以代之。光緒二年,丁母憂,奪情留任。兩江總督曾國荃奏請江南兵輪悉歸成謀統轄。十六年,萬壽推恩,加太子少保。十八年,以病乞歸,尋卒。詔嘉其在任十餘年,馭軍有法,江面乂安。賜恤,建專祠,謚勇恪。

李朝斌,字質堂,湖南善化人。由行伍隸長沙協標。咸豐四年,曾國籓調充水師中營哨官,從楊岳斌克武昌、田家鎮各城隘,累功擢至參將。六年,會內湖水師攻克湖口及梅家洲,從楊岳斌乘勝循下游,埽蕩江面,擢副將。八年,會攻九江,朝斌以水師登陸助戰,克之。復從楊岳斌進攻安慶,拔樅陽、銅陵賊壘,賜號固勇巴圖魯。十年冬,間道援南陵,回軍攻克東流。十一年,下茯苓洲、白茅嘴賊壘,會陸軍克無為州,以總兵記名。再復銅陵,迭克泥汊、神塘河、運漕鎮、東關,加提督銜,授湖北竹山協副將。同治元年,擢浙江處州鎮總兵。

彭玉麟督水師會陸軍進規沿江要隘,令朝斌率所部游奕上下游,兜剿環攻,連克金柱關、蕪湖、東梁山,以提督記名。曾國籓奏設太湖水師,以朝斌將,令赴湖南造船募勇。二年,成軍東下,會諸軍克江浦、浦口,連破草鞋峽、燕子磯賊屯,戰九洑洲,功最,賜黃馬褂。

朝斌一師,原為規復江、浙而設,九洑洲既克,會黃翼升淮揚水師同援上海,由長江直下,與總兵程學啟會師夾浦,督水師百艘攻沿湖賊壘,下之,進破澹臺湖賊壘;直逼蘇州,破盤門外賊壘。賊酋李秀成率眾七八萬奪寶帶橋,朝斌會陸師合擊,血戰挫之,賊始退。破援賊於葉澤湖,截竄賊於覓渡橋。會克五龍橋賊壘,分攻葑門、閶門,晝夜轟擊,李秀成先逸,餘黨以城降。李鴻章奏捷,言朝斌迭次苦戰,謀勇兼優,予雲騎尉世職。

是年冬,會陸師剿賊江、浙之交,克平望鎮,又破賊九里橋,署江南提督。三年,偕程學啟會攻嘉興,朝斌水師由官塘進,破其七壘。湖州援賊圖竄盛澤以牽圍師,為朝斌所扼,不得逞,遂克嘉興,實授江南提督。進規湖州,由夾浦逼長興,賊眾數萬,依山築壘,楊鼎勛、劉士奇等與之相持,朝斌督水師登陸襲賊後,夾擊之,盡毀西北沿水賊壘。乘勝克長興,復湖州,被珍賚。

五年,移駐蘇州。軍事甫平,江、浙湖蕩盜多出沒,捕著匪卜小二誅之,轄境晏然。八年,請設經制水師,著為成例,移駐松江。光緒四年,兩江總督沈葆楨疏請以外洋兵輪統歸朝斌節制,允之。十二年,以病乞歸。二十年,卒於家,賜恤,建專祠。

朝斌本姓王氏,父正儒,生子四,朝斌最幼,襁褓育於李氏。朝斌官江南提督時,牒請歸宗,曾國籓引金史張詩事,謂:「朝斌所處相同,定例出嗣之子,亦視所繼父母有無子嗣為斷。今若準歸宗,王氏不過於三子外又增一子,李氏竟至斬焉不祀。參考古禮今律,朝斌應於李氏別立一宗,於王氏不通婚姻。一以報顧復之恩,一以別族屬之義。王氏本生父母由朝斌奉養殘年,庶為兩全之道。」詔如議行。

江福山,湖南清泉人。咸豐五年,應募入水師,積功敘把總。十一年,克赤岡嶺、菱湖賊壘。安慶復,累擢游擊。同治元年,改隸太湖水師,從李朝斌回籍造船,領前營。浦口、下關、草鞋峽、燕子磯、九洑洲諸戰皆有功,擢參將。從援上海,破賊於楓涇、烏涇塘。蘇州復,擢副將,賜號強勇巴圖魯。三年,從攻嘉興,砲穿左臂,裹創而進,克郡城,擒賊酋,以總兵記名。攻太湖夾浦鎮,砲斷左手指,奮擊破之。進攻湖州久不下,郡東晟舍賊壘最堅,請以偏師往攻,使賊互救,然後大軍乘之。福山首先躍壕而入,諸軍繼進,悉毀賊壘。援賊大至,福山摧鋒直前,中砲洞腹,歿於陣。事聞,詔視提督例賜恤,死事地建專祠,入祀京師昭忠祠,予騎都尉兼雲騎尉世職,謚武烈。

劉培元,湖南長沙人。咸豐初,以武生入水師,從克湘潭、岳州,敘千總。戰金口,沉賊船,登岸縱擊,斬賊酋一人。克嘉魚、蒲圻,擢守備。戰田家鎮,培元率十舟窮追四十餘里,毀賊船,擢都司。會攻湖口,斧斷鎖筏,毀湖口賊船。五年,回援武漢,擊賊占魚套,又會鮑超攻小河口,毀賊舟二百有奇。

六年,改陸軍,領長字營,從劉長佑援江西。由瀏陽攻萬載,破賊荊樹鋪、慄樹坳,駐大橋。賊潛來襲,培元出奇兵擊之,斬級八百。又破援賊於高城、竹埠。克萬載,營西門外,賊數路來爭,多於官軍數倍。培元開壁大戰,斬級千計,擢游擊。進攻袁州,破南門嶺上賊壘,會蕭啟江破吉安臨江援賊,城賊遁走。克袁州,以參將留湖南補用。七年,會攻吉安,偕曾國荃迎擊援賊於三曲灘,追至硃山槽,賊援復集,夾擊破之,擢副將。八年,水陸合攻吉安,賊結大筏沖官軍浮橋,培元督師船截擊,砲傷胸,裹創血戰,盡毀其筏。尋克吉安,以總兵記名。是年冬,軍中大疫,培元病,回籍。

九年,石達開犯湖南,培元率千人扼桂陽,眾寡不敵,桂陽遂陷。尋率師船溯資水進援寶慶,會諸軍扼河而戰,數破賊,寶慶圍解,援浙江處州鎮總兵,仍留湖南領水師。

十一年,左宗棠進規浙江,立衢州水師,疏薦培元熟諳水陸軍事,請以署衢州鎮,募勇三千赴浙。同治元年,培元率安武水陸全軍駐常山,控衢州北路,進江山,破大洲賊營。賊竄龍游,會攻之,賊酋李侍賢大股來援,培元與諸軍合擊,賊敗走。二年,克湯溪、龍游,斃賊酋陳廷秀,加提督銜,賜號銳勇巴圖魯。迭克桐廬、富陽,會攻杭州,破賊於萬松嶺,攻清泰門外觀音堂,平其壘。城賊出戰,敗之。舁舢板入西湖,砲擊杭城。左宗棠以衢州後路要沖,令培元返鎮,其所部水師留攻杭州。三年,杭州復。培元丁母憂歸,遂不出。光緒十七年,卒。湖南巡撫陳寶箴疏陳培元戰績,賜恤。

論曰:自湘軍水師興,而後得平寇要領。後又設淮揚、太湖兩水師,平吳及浙西賴其力。黃翼升、李朝斌當其任。其後設長江水師為經制,翼升與李成謀迭相更代,為東南重鎮。平浙東專在陸師,故水師僅有衢州一軍。劉培元亦彭、楊舊部,戰績可稱,用並列之。


\end{pinyinscope}