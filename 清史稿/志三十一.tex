\article{志三十一}

\begin{pinyinscope}
地理三

△吉林

吉林:古肅慎國之域。明初,奴兒乾都司地,領衛百八十四、所二十。後為長白山三部、扈倫四部所屬輝發、烏拉、葉赫,兼有哈達北境及東海部地。清初,建滿洲城於俄漠惠之野鄂多理城。順治十年,置昂邦章京及副都統二人鎮守寧古塔。康熙元年,改寧古塔將軍。十五年徙,改吉林將軍。先是十年徙副都統一人駐吉林,三十三年徙伯都訥。雍正三年,復置吉林阿勒楚喀副都統。五年,增三姓副都統。光緒七年,置琿春副都統,吉林、賓州、五常三。八年,吉林升府。後增長春、新城、依蘭,各領縣有差。三十三年建行省,改將軍為巡撫,盡裁副都統等。宣統三年,定西南、西北、東南、東北四路為四道。凡轄府十有一,州一,五,縣十八。西至伊通州,界盛京;東至烏蘇里江,界俄領東海濱省;北至松花江,界黑龍江;南至圖們、鴨綠江,界朝鮮。廣二千四百餘里,袤千五百餘里。北極高四十一度三十分至四十五度四十分。京師偏東九度八分至十三度十分。宣統三年,編戶七十三萬九千四百六十一,口三百七十三萬五千一百六十七。案:吉之舊界,東至寧古塔八百餘里,又烏扎庫邊卡七百餘里,又松阿察河三百里,又千餘里至海,凡三千里有奇。其東北至三姓千二百里,又五百餘里富克錦,又七百餘里烏蘇里江口,又二千餘里至廟爾,實四千四百里有奇。又自富克錦逾混同,循黑龍江東界,北至外興安嶺,二千里有奇。又自琿春而東至海參崴,又東七百里有奇錫林河。其中部落,若費雅,居圖庫魯、鄂古二河之間,在混同江北海濱;若費雅喀,居額濟第河西;若貢豹,居約色河北;若奇雅喀喇,居約色河南,並混同江東南海濱。其自混同江口西至黑勒爾,則濟勒彌部居之,即金史之濟勒敏;自黑勒爾西至阿吉大山,沿混同江兩岸,則額登喀喇部居之,即不薙發黑斤;自阿吉大山西至伯利,則赫哲喀喇居之,即薙發黑斤;並久隸版圖,比於編戶。咸豐八年愛琿之約,以烏蘇里江口為新界,失地二千餘里,然於吉只東北一隅。十一年北京之約,自烏蘇里江口溯流至松阿察河,逾興凱湖西至白棱河口,又逾大綏芬河而南至瑚布圖河口,又南而西至圖們江口以東舊界屬俄,以烏札庫邊卡瑚布圖河口為新界,又失數千里,遂無復有江口入海為吉轄境者。光緒十二年,黑頂子勘界,定琿春之海口屬俄,則圖們江口內去海三十里「土」字界碑為中俄新界矣。又東北海中庫葉島,一曰黑龍嶼,廣三四百里,袤二千餘里。西北圖克蘇圖山,山陰社瓦狼、陽費雅喀部,南有阿當吉山,山東嵩闊洛、南俄倫春部,又南雅丹部,並天命中內附。遼遠不克時至,歲以夏六月遣使至寧古塔東北三千里普魯鄉貢獻,頒賚有差。後屬三姓。今亦為俄有矣。又東南海中蝦夷島,康熙中屢偕庫葉人至混同江境內貢貂受賞,後亦隸日。其名山:長白。北迤者,黑山、平頂。歧為二:西支,西北迤為色齊窩集、張廣才嶺,至拉林;東支,東北迤為哈爾巴嶺、老松嶺。至綏芬河源復歧,一東訖俄東海濱省,一東北為察庫蘭嶺、哈達嶺、阿爾哈山。其巨川:松花、混同、嫩江、牡丹、烏蘇里、圖們諸江。其驛路:西達盛京開原;北齊齊哈爾;西南達琿春。電線:東達海參崴,北齊齊哈爾,西南達奉天。

吉林府:繁,疲,難。總督駐奉天。巡撫兼副都統,民政、交涉、提學、提法、度支司,勸業道駐。明,烏拉等衛。後屬扈倫族之烏拉部。本吉林烏拉,一曰烏拉雞林,又名船廠。清初,隸寧古塔將軍。康熙十五年徙駐。雍正五年置永吉州,隸奉天。乾隆十二年改吉林,仍隸將軍。光緒八年升為吉林省治,領伊通、敦化,後削。西南至京師二千三百里。距盛京八百二十餘里。隸西南路道。廣四百九十里,袤五百餘里。北極高四十五度四十九分。京師偏東十度二十七分。東:團山,尼什哈龍潭。西南:溫德亨,亦望祭山,有殿祀長白,雍正十一年建;壽山。東南:松花江自額穆入,右合海青溝,左溫德亨河。東北逕城東,又北,右合莽牛、四家子,左鼇龍、興隆河,緣舒蘭界入德惠。西南驛馬河,即伊勒們,自磐石緣界合岔路河,又北緣雙陽界,西北木石河,並從之。打牲烏拉,城北七十里,本烏拉國,舊曰布特哈烏拉。太祖先後克其宜罕山、臨河、金州、遜扎塔諸城,遂平之。柳邊四圍長六百二十二里,柵高四尺五寸,壕寬深各一丈,插柳結繩以定內外,曰「柳條邊」,亦新邊。東北接舒蘭,西南至雙陽。農事試驗場,桑蠶山蠶林業土山分局,松花江官輪局,歡喜嶺稽查所。商埠,光緒三十一年中日約開。舊設站五:東尼什哈、額赫,北金珠鄂佛羅,西蒐登、伊勒們。官商路四:南並溫德河達樺甸官街;東南歷大小風門達敦化;西南雙河鎮、磐石;北渡鼇龍達德惠。吉長鐵路站九:吉林、孤店子、樺皮廠、趙家店、土門嶺、馬鞍山、營城子、下九臺、驛馬河。

長春府:繁,疲,難。省西二百四十里。古扶餘國地。明初,三萬衛。後屬蒙古科爾沁部。清初屬蒙古郭爾羅斯前旗,曰寬城子。嘉慶五年,於長春堡置長春。道光五年徙治,仍舊名。光緒十五年升。宣統元年,設西南路分巡兵備道,駐府。廣三百二十里,袤一百七十里。北極高四十三度四十一分。京師偏東八度三十三分。西南:白龍駒山。俄築東清鐵路採石。光緒三十四年,與日本交涉封禁。西人謂世界最古石山,與英阿爾蘭為二。西:龍泉、大青、對龍。南:伊通河自其州邊門入,逕城東,又北,左會新開河,東北緣農安界,逕潘家嶺,入德惠。東:驛馬河,自雙陽緣界,右岸及霧海河並北從之。硃家城照磨,光緒十六年由農安徙。官商路四:南入伊通門,達其州;東南十里堡達雙陽;西:萬家橋達奉天懷德;北:萬寶山鎮達農安。吉長鐵路自吉林歷德惠入。站四:飲馬河、卡倫、長春、頭道溝。在府西北與東清接。日俄戰後,長春以北屬俄之東清,以南屬日本南滿鐵道會社。俄站寬城子,曰長春驛。商埠,光緒三十一年中日約開。

伊通州:沖,繁,難。省西偏南二百八十里。渤海長嶺府地。明初,塔山、雅哈河、伊敦、拉克山、發河等衛。後屬扈倫族之葉赫部。雍正六年,由吉林鑲黃、正黃二旗各撥一旗駐之。嘉慶十九年,置伊通河巡司。光緒八年為州,屬吉林。宣統元年直隸。二年降隸西南路道。北極高四十三度四十分。京師偏東八度五十分。西南:龍潭山。西:摩里、青、馬鞍。北:勒克。東:尖山。東南:大星嶺,其東板石屯,伊通河出西北,逕城東,右合伊巴丹河,出邊入長春。西:小伊通河,自奉天東平錯入,為新開河,入懷德。太平河從之。又西,東遼河自西豐入,右合大小雅哈河,入奉化。昭蘇太及條子河亦入焉。左納陽斯河,一曰赫爾蘇河。又西,清河入為葉赫河,入開原。其瞻河錯入從之。赫爾蘇,州同,光緒二十八年由磨盤山徙。舊設站五:東自雙陽蘇瓦延入境,六十里伊巴丹;又西百里阿勒坦額墨勒,即大孤山站;又西六十里赫爾蘇;又八十里葉赫;又五十里蒙古霍羅,即蓮花街站。官商路四:北達長春;東南營城子達磐石;西赫爾蘇站達奉天奉化;西南蓮花街達昌圖。

濛江州:省南四百六十里。明,鄂爾琿山所。後屬訥音部。光緒三十四年,析吉林極南地置。宣統元年,隸西南路道。北:那爾轟嶺。南:長半城山、五金頂子。東南:頭道江,自奉天撫松緣界,為湯河口,屈北,合花園河。其西裴德里山,頭道濛江出,州以此得名。右合二道、三道水,左珠子河來會。又東北合那爾轟河,其右岸會二道江,是為松花江,入樺甸。官商路四:北達樺甸官街至省;東北至夾皮溝;西達奉天輝南;東南湯河口入長白。

農安縣:疲,難。省西北三百六十里。古扶餘國都。明置三萬衛。清初,郭爾羅斯前旗地。光緒八年置照磨,十五年改,仍隸長春。宣統元年,隸西南路道。東:臥牛石山、紅石砑。西:太平嶺、伏山、大青。東:松花江,自德惠入。城南伊通河自長春緣界注之,西北入蒙古郭爾羅斯前旗。舊有蒙古站路,共十一站,三百九十里。

長嶺縣:省西北五百二十里。蒙古郭爾羅斯前旗地,曰長嶺子。光緒三十三年,析農安之農家、農齊、農國三區置,隸西南路道。南:硃克山、團山。境無河流。北有大漠如瀚海,俗呼北海。冬夏恆苦風沙,惟東、南二鄉繁盛。新安鎮,主簿。官商路四:東南至長春;西北至奉天開通;北歷郭爾羅斯前旗達安廣;南歷科爾沁達爾罕旗達遼源。

樺甸縣:省南偏東二百七十里。明,法河衛。末屬長白山之訥音部。清初禁地。光緒三十四年,置治樺皮甸子,徙樺樹林子。宣統元年,隸西南路道。西北:趙大吉山、慶嶺。西:杉松、天平。南:帽山、猴嶺。東南:金銀壁嶺。二道江自奉天安圖緣界,富爾河合古洞、黃泥、蒲岑諸水注之,為上兩江口。又西仍緣安圖及撫松界,至下兩江口。左岸合頭道江,是為松花江。合境內柳河五。又葦沙色勒河,復緣濛江界入,右合穆奇河,逕城西,左會輝發河,為大渡口。又北右合漂河,逕樺皮甸子入額穆。側有常山屯,扼琿春、敦化西至奉天孔道。官商路五:西至官街,折北入吉林;北出大鷹溝,並達省;西南至濛江;東至敦化;東南延吉洞河入延吉。

磐石縣:省南偏西三百里。明,扈倫族輝發部。清初,北境屬吉林,南奉天圍場。光緒八年,置磨盤山巡司,隸伊通。十三年改州同。二十八年為縣,隸吉林。宣統元年,隸西南路道。磨盤山,東北二里。北:雞冠。東北:老茅。西:大紅石磖、庫勒嶺。東南:輝發江自奉天海龍亮子河入,東北,左合石頭、富大都嶺,右蝦蟆、獨木河,逕黑石鎮,左合硃其、呼蘭,右大小色力河。五道至頭道荒溝,入樺甸。東北:呼蘭嶺,驛馬河出西北,左合黃河,入吉林,岔路河從之。官商路三:北小城子達省城;西由朝陽山達伊通;東南黑石鎮西達海龍;南濛江。

舒蘭縣:省北偏東百六十里。明,阿林衛地。康熙二十年,置巴彥鄂佛羅防禦旗員,屬烏拉總管。宣統元年,置於舒蘭站。二年,徙治朝陽川南,隸西南路道。南:北慶嶺。東南:玲璫嶺。西:松花江自吉林緣界西北入德惠。卡岔河北入榆樹達之。東:蘭陵河,自額穆緣界入五常。東南:馬鞍山,溪浪河出東北,逕秋千嶺,合呼蘭河從之。有巴彥鄂佛羅邊門,即法特哈,康熙中更名。舊設站二:舒蘭、法特哈。南接吉林金珠鄂佛羅,北達榆樹盟溫。官商路三:西南達烏拉街;東北水曲達五常;東南小城子鎮達額穆。

德惠縣:省北偏西百四十里。蒙古郭爾羅斯前旗地。舊屬長春。宣統二年,析長春沐德、懷惠二鄉置,治大房身,隸西南路道。南:狼洞嶺。西:團山、雙山。西南:土牛。東南:松花江自吉林緣界合木石河,入新城。西北伊通河,自長春緣界,納驛馬及霧海河從之。官商路三:南五臺達省;東岔路口達榆樹;西雙山崖鎮達農安。吉長,東清鐵路。

雙陽縣:省西百九十五里。明,依爾們、蘇完河二衛。宣統二年,析吉林西界、長春東界、伊通北界置,治蘇斡延,隸西南路道。西南:黑頂子。南:土頂子、將軍嶺,光僻山,雙陽河出焉。東南驛馬河、自磐石緣界,合杜帶、雙陽、放牛、溝河入長春,西北霧海河從之。舊設站一;蘇斡延。官商路三:南皇營;東南五家子鎮,並達磐石;北奢嶺口達長春。

新城府:繁,疲,難。省西北省六百里。即伯都訥副都統城。古扶餘國地。明,三岔河衛。後屬烏拉部。嘉慶十五年,置伯都訥。光緒三十二年,改隸西北路道。廣四百二十里,袤一百七十里。北極高四十五度十五分。京師偏東八度三十七分。南:大青山、鷹山。東南:松花江自德惠緣界農安,又西,左岸逕城南,又西北,左岸蒙古郭爾羅斯前旗界,至三岔口會嫩江。折東,緣黑龍江界,右會拉林河。自榆樹緣界,復緣雙城,合灰塘、薛家窩鋪河,入雙城府境。松花環其南、北、西三面,拉林流其東,川原廣衍,水陸輻輳,富庶甲全省。舊設站五:自榆樹盟溫西北五十里入陶賴昭,又西五十里遜札保,又四十里伯都,又五十里社哩,又北八十里伯都訥,至松花渡口出境。官商路四:東北長春嶺達雙城;東南集廠達榆樹;西渡江歷郭爾羅斯前旗達奉天洮南;一由社哩渡江至郭爾羅斯鎮國公府。有輪船埠。東清鐵路站三:蔡家溝、石頭城、陶賴昭。有通松花鐵橋。

雙城府:沖,繁,難。省北五百里。明,拉林河衛。有古城二,舊曰雙城子。嘉慶十九年,置委協領,隸阿勒楚喀副都統。光緒八年,置雙城。宣統元年,改隸西北路道。廣二百四十里,袤一百四十里。北極高四十五度四十分。京師偏東九度二十分。東南:砍戶德山。西北:松花江自新城會拉林河,東南自五常入,緣榆樹、新城界,合朝陽葦塘河,入濱江。東:阿什河自賓州緣界,合混元河,逕小青頂子,合大紅黃泥河,屈北緣阿城界入之。拉林城巡司。舊設站二:多歡、雙城。官商路三:東東官所達阿城;東南至拉林;西西官所達新城。東清鐵路二:西路站二,雙城堡、五家子;東路站一,帽兒山。

賓州府:沖,繁,難。省北偏東六百十里。古挹婁國地。明,費克圖河衛。光緒六年,建城葦子溝,置賓州。二十八年直隸。宣統元年升府,隸西北路道。廣四百三十里,袤二百六十里。北極高四十五度五十一分。京師偏東十一度五分。東:海裏渾山、太平、大青。南:黃頭、混元。西北:團山。松花江自阿城入,合裴克圖河,出南釣水湖嶺,緣阿城界。又東合烏爾海裏琿夾板。有新甸鎮,江運巨埠。陶淇、擺渡諸河,入方正。東南:墨爾根阿什河出,西緣雙城界合混元河入之。舊設站三:裴克圖、葦子溝、色勒佛特庫,東入松花北岸之佛斯亨。官商路三:東廟嶺達長壽;西北滿井達阿城;南古道嶺達五常。東清鐵路,府南。站:小嶺。

五常府:繁,疲,難。省北偏東三百八十里。渤海上京屬境。明屬摩琳衛。同治八年,置五常堡協領。光緒六年,建城歡喜嶺。八年改五常。宣統元年升府,隸西北路道。廣二百一十二里,袤二百三十五里。北極高四十五度。京師偏東十度二十七分。東:螞蜒窩集。東北:索多和。東南:九十五頂子。蘭林河自額穆緣界,又西北緣舒蘭界,合響水、寒蔥河入。右合渾水、黃泥,左納石頭、溪浪河,逕城西,復緣榆樹界。東南摩琳莫勒恩河出,右合沖河、香水、大小泥,左小黑、取才、條子、藤子河,逕五常堡來會,為拉林河。又西北入雙城。山河屯經歷,南六十里。藍採橋巡司。舊設站一:五常。官商路五:北達賓州;南達舒蘭;東南向陽山街達額穆;東北太平山達長壽;東南沖河鎮達寧安。

榆樹直隸:繁,疲,難。省北二百八十里。舊孤榆樹屯,屬伯都訥部。光緒八年,伯都訥同知徙駐。三十二年,置榆樹縣。宣統元年升。二年,隸西北路道。東:龍首山。西南:松花江自舒蘭入,西北緣德惠界,逕五棵樹鎮,入新城。有渚曰巴彥通。東北:蘭棱河,自五常緣界,迤西北緣雙城界,為拉林河。至牛頭山鎮,南卡岔河自舒蘭入,右合二、三、四道河注之。舊設站三。登伊勒哲庫即秀水甸子,西接蒙古喀倫,西北接拉林多歡,東達五常盟溫,南接舒蘭之法特哈,西北接新城之陶賴昭。

濱江:省北五百五十里。即哈爾濱,本松花江右灘地。光緒三十二年,置治傅家甸,為江防同知,駐濱江關道,分隸黑龍江省。宣統元年,劃雙城東北境益之,江防改撫民,專屬吉林,分巡西北路道駐。東:秦家岡。北:松花江自雙城合葦塘溝河緣界入。左岸黑龍江哈爾濱。總車站,城西。自此西南雙城、新城、德惠,達長春。東南阿城、賓州、雙城、長壽、寧安、穆棱,達東寧之交界驛。商埠,光緒三十一年中日約開。海關。兩江郵船總局。

長壽縣:疲,難。省東北八百六十里。明,螞蜒河衛。光緒八年,置燒鍋甸子巡司,屬賓州。二十八年改置,隸賓州直隸。宣統二年,隸西北路道。南:花曲柳山。東:西老嶺。東南:螞蜒窩集嶺,螞蜒河出,屈西,左合小石頭、七道、葦沙、西烏吉密,右養魚池、蒿麥、棱河。折東北,左合西亮珠,右黃玉、長壽。逕城東,又東北逕夾信鎮,右合東亮珠、大石頭、大黃泥河,入方正。一面坡,巡司。官商路三:西黑龍宮達賓州;南一面坡達五常;東黃泥鎮達方正。東清鐵路站五:烏吉密、一面坡、葦沙河、石頭河、交嶺子。

阿城縣:省北四百八十里。即阿勒楚喀副都統城。渤海海古勒地。明,岳希、河突二衛。宣統元年,裁改阿勒楚喀副都統置,隸西北路道。東南:牛角、廢兒諸嶺。北:松花江自濱江入,納阿什,合裴克圖河自賓州。舊設站一:薩庫哩。東清鐵路站二:阿什河、三層甸子。

延吉府:繁,疲,難。省東南七百六十里。東南路道駐。明,錫璘、布爾哈通、愛丹三衛。清初,為南荒圍場。光緒七年,弛墾。二十八年,置延吉。宣統元年升。西:哈爾巴嶺,布爾哈通河出其東,東南匯太平、倒木、岔條、簸箕、葦子諸溝,細鱗河,逕銅佛寺,至朝陽川,左合朝陽延吉河,至城南,右納海蘭河,又東北,左合一兩溝,抵汪清界。頭道嘎雅會二道嘎雅河錯出,仍緣界來會,折東南入圖們江。舊設站三:老松嶺、薩奇庫、瑚珠。官商路四:西南東古城達樺甸;西銅佛寺達敦化;南六道溝達和龍;東北小盤嶺達琿春。商埠,頭道溝、龍井村、局子街,三。宣統元年中日間島約開。

寧安府:省東八百里。即寧古塔副都統城。其舊城,西北五十里舊街鎮。康熙五年徙之。古肅慎國都。明,奴兒乾都指揮使司。光緒二十八年置綏芬,駐三岔口,尋徙寧古塔城。宣統元年升,二年更名,隸東南路道。廣八百餘里,袤六百里。北極高四十四度四十六分。京師偏東十三度三十五分。西:茨老茅山。東北:卡倫。西北:瑪展窩集。南:老松、瑪爾瑚哩窩集諸嶺。西南:牡丹河自額穆入,匯為鏡泊。右受大小夾溪、松陰河,左布尼、畢拉罕河。復北出,左合沙蘭,右馬連河,逕東京城,至府治東。右合蛤螞,左海浪河,逕乜河鎮。右合乜,左頭、二、三道河,入方正分界磖子。舊設站九:西必爾罕、沙蘭、寧古臺,北鰟頭岔、沙河子、細鱗、三道河,分自吉林三姓達寧古塔,南新官地、瑪爾瑚哩,則自塔達琿春。東清鐵路,橫道河、山巖、海林、牡丹江站四。商埠,光緒三十一年中日約開。

東寧:省東千四百里。明,綏芬河地,置率賓江衛。光緒二十八年,置綏芬撫民同知。宣統元年,改通判,更名。隸東南路道。北:黃窩集山。南:通肯。西北:萬鹿溝。西:穆棱窩集、老松諸嶺。西南:大綏芬河自汪清入,左合蛤螞、黃泥、寒蔥河,右葦子諸溝。又東北,左合小綏芬河,逕城北,南大瑚布圖河,北緣俄東海濱省界,合小瑚布圖河來會入之。官商路四:西北萬鹿溝達東清鐵路;西屯田營達寧安;西南達汪清;南沿瑚布圖河達於琿春。東清鐵路,六、小、五站三。界碑「倭」、「那」字二。綏芬河稅關。

琿春:省東南千二百里。明,琿春衛。後屬瓦爾喀部。清初,南荒圍場。光緒七年,始弛禁設墾局。宣統元年,改副都統,置同知,隸東南路道。廣二百五十里,袤三百餘里。北極高四十三度,京師偏東十四度三十分。東:分水嶺長嶺子。西北:圖們江自汪清、朝鮮緣界,合乾密江,至紅旗河口,即琿春河。出東北土門嶺,屈南,逕太平川,左合官道,右六道、五道諸溝。又西,左合夾心子、胡盧別、瓦岡寨、大小紅旗河,右四、三、頭、二道、車擔溝。逕城南,右合二道、罕通河來會。又南,逕黑頂子,合圈河,出境入海。舊設站二:北密江,中阻大盤嶺,恆假道朝鮮鍾城達慶源;東路三道溝、哈達門、二道河並達俄。界碑:南「土」,東「薩」、「啦」、「帕」字,凡四。商埠,光緒三十二年中日約開。

敦化縣:疲,難。省東南四百七十里。古挹婁國。明,建州左衛。後屬窩集部之赫席赫路。清始祖居鄂多哩城,即此。初為額穆赫索羅堧地。光緒八年建新城置,隸吉林。宣統元年,改隸東南路道。西南:牡丹嶺。牡丹江出東北,左會小牡丹江,右合四、三、二、大荒溝。又東北,左合黃泥、大石頭河,逕城東。又北,左合小石頭、雷風氣河,入額穆。東大沙、西北河並從之。舊設站二:自額穆通溝西南八十里至城;又東八十里滴媟嘴達寧古塔。官商路三:西半截河出新開道達樺甸;西南逾牡丹嶺達濛江;東黃土腰子達延吉。

穆棱縣:省東偏北千里。明,木倫河衛。清初穆棱路。光緒二十八年,置穆棱河分防知事,屬綏芬。宣統元年,改隸東南路道。穆棱窩集,鎮南。穆棱河出嶺北,屈折東北,左合泉水、大小石頭,偽臉河,右廟溝,逕城南。又東北,左合柳毛河、坎椽子、扣河溝,右太平、朝陽川。馬橋河出四頂子山,合狐貍密河。又北,左合膻羊磖子河、雷風氣、百草溝。右上亮子河,出鐵鍬背,逕下管入密山。官商路三:西泰東站入寧安;東北下城子逾青溝嶺達密山;一東渡細鱗逾鐵路至東寧。東清鐵路,磨刀石、臺馬溝、美嶺、馬橋河、太平嶺站五。

額穆縣:省東三百八十里。明,斡朵里、禿屯河二衛。後屬窩集部之鄂謨和蘇魯路。清始祖所居俄漠惠,即此。舊曰額穆赫索羅。乾隆三年,置佐領。宣統三年,改隸東南路道。西:嵩領。蘭陵河出其北,曰黃泥河,會大石頭河,緣五常界入。西南:松花江自樺甸入,左合拉發及嘎雅河,折西北入吉林。南:牡丹江自敦化入,右合大沙河,左硃爾德河,納戚虎河注入,屈東,左合馬鹿溝、都林、塔拉泡,右朝陽、大小空心木河,入寧安。舊設站六:西拉法,距吉林額赫穆站八十里;又東六十五里退摶;八十里伊壽松;又四十里至城,即額穆赫索羅站;又東八十里塔拉達寧安;一東南八十里通溝達敦化。

汪清縣:省東南千二十三里。明,阿布達哩衛。清初庫雅拉部鈕呼特居之,為世管佐領。宣統二年置,隸東南路道。北:老松嶺。南:圖們江自和龍入,二道嘎雅河自嶺西合樺安溝,緣延吉界,合藥水河,至摩天嶺仍入。左合大小汪清溝,逕城東,又南復緣延吉界注之,入琿春。東北:荒溝嶺。大綏芬河出東北,左合大石頭、老母豬河、太平溝,入東寧。舊設站三:東哈順,北至延吉瑚珠嶺站六十里;又南四十五里德通;西北逾高麗嶺至牛什哈嶺為分站。官商路二:南逾吉清嶺至延吉;東北歷綏芬甸子入東寧。商埠,百草溝,宣統元年中日間島協約開。

和龍縣:省東南八百里。明,賡金河衛地。光緒十一年,吉、韓通商,和龍峪與光霽峪西步江互市。二十八年,置分防經歷,屬延吉。宣統二年,改隸東南路道。西:秫秸嶺。迤東北雞冠磖子,又北窩集嶺,其東三、二道溝並入延吉。西南:圖們江自奉天安圖入,合紅旗河外六、五、四道溝,逕東景德,至汗王習射臺。又北逕光霽峪入汪清。官商路二:一北至延吉;一南至火狐貍溝,渡江達朝鮮會寧。又西北由窩集嶺出長白北麓,沿古洞、富爾河,歷樺甸、磐石,達奉天海龍,俗呼盤道,清初為通衢。後別為圍場,禁塞。光緒中復通。

依蘭府:繁,疲,難。省東北千四十里。東北路道駐。即三姓副都統城。古肅慎國地。明,和屯衛。清初稱依蘭喀喇。光緒三十一年改置,隸東北路道。東:大德依亨山、阿爾布善。東南:察庫嶺。西北:松花江自方正入。西南牡丹江自寧安緣界,又北緣方正界,合阿什明達、烏斯渾、伯利,逕城西注之。東:倭肯河自樺川入,合奇塔、庫倫、連珠岡、大小八浪,納七八虎力,又西北合蘇木,至城東來會。舊設站九:西妙嘎山,又西鄂爾國木索,崇古爾庫、富拉渾、佛斯恆,並江北岸,約二百八十餘里;南太平莊、烏斯渾、小巴彥蘇、蓮花泡接寧安。官商路三:西珠淇河達方正;東阿穆達樺川;東南土龍山達密山。有護江關。

臨江府:繁,疲,難。省東北二千里。金,黑水靺鞨部。清初黑哲喀喇人所居,即薙發黑斤。曰拉哈蘇蘇。光緒初,始由三姓副都統編戶入旗,分三佐領。三十二年,置臨江州。宣統元年升,隸東北路道。廣四百三十里,袤四百餘里。北極高四十六度二十分。京師偏東十三度二十分。東:街津山、小白。南:西太平。西南:葛蘭棒子。西:烏爾古力。松花江自富錦入,左會黑龍江,曰黑河口,為混同江。又東合街津河,出向陽山,入綏遠。南:饒力河自密山緣界,合依瓦魯河,又東緣饒河界,大七里星河入之。西南:倭肯河,西入樺川。官商路四:西圖斯科達富錦;東睦鄰鎮達綏遠;東南寒蔥山達饒河;又由二道岡西歷駝腰子,亦達富錦。

密山府:省東北千三百里。渤海湖州地。明,木倫河★及松阿察河堧地。清初瓦爾喀部人所居,隸寧古塔副都統。光緒三十四年置。蜂蜜山南十餘里,脈與西南黃窩集接,■三百里。隸東北路道。西南:穆棱河自其縣入。右合小穆棱、滴道哈達嶺水,左下亮子,逕城西。又東北,左合大穆棱河,其北七虎林河,其東南阿松察河,出興凱湖,東北緣界,並入虎林。北:饒力河,東緣臨江界入饒河。官商路六:西大柞木臺達穆棱;東楊木岡達虎林;西北太平砬子達依蘭;北達臨江;南至快當別;東南龍王廟達俄。界碑:興凱湖東「亦」字、西「喀」字,又西「拉」字、「瑪」字。

虎林:省東北千九百里。宣統元年,置呢嗎口。二年更名。署西南關帝廟榜題「嘉慶己巳重修」,則漢民足跡早至。隸東北路道。西:七虎林山。西南:半拉窩集、蘇爾德。西北:安巴倭克里。北:那丹哈達拉嶺。南:烏蘇里江自俄東海濱省緣界,納松阿察及小黑河,又北納大小穆棱河,逕城東。又北納七虎林河,合阿布沁、小大木克、獨木、外七里星河,入饒河。官商路三:南至大穆棱河,西歷索倫營達密山;南歷倒木溝至龍王廟;一城北下水撈達饒河。惟烏蘇裏時溢,沿江哈湯多,足礙行旅。又由治至渡江,溯呢嗎,即至烏蘇里鐵路伊曼站。

綏遠州:省東北二千五百里。清初使犬部額真喀喇人居之,隸三姓副都統,曰伊力嘎。宣統元年置,隸東北路道。北極高四十度四十九分。京師偏西四度四十八分。西南:秦得力山、額圖、昂古喀蘭、太平。南:完達、科勒木蘇拉立喀蘭。北:混同江自臨江入,合二吉利、秦得力、沃泥河、濃江。南:烏蘇里江自饒河入。右畢拉音畢爾竇,屈東北,右東海濱省,分二支來會,折西北亦入之。官商路三:西秦皇、魚通,西小白山,並達臨江;東南窩集口達饒河。烏蘇里下口西岸有「耶」字界碑。

方正縣:繁,疲,難。省東北九百二十里。清初呼爾哈部人居之,隸三姓副都統。光緒三十二年置大通,隸依蘭,治江北崇古爾庫站。宣統元年,徙治江南方正泡,割濱州長壽東境益之,更名,隸東北路道。西:萬寶山。東:雙鳳、鳥槍頂子。南:東老龍爪溝嶺。北:松花江自賓州入,納螞蜒及柳樹、黃泥河,逕城北,合二古力、德墨里、大小羅拉蜜。又東北,納珠淇河,入依蘭。東南:牡丹江自寧安緣界,合大小營門石,四、五、三道諸河從之。官商路三:西新安入賓州;東達溝達依蘭,舊阻哈湯,近通利;西南黃泥河入長壽。船埠:德墨里屯。

樺川縣:省東北千三百十八里。清初黑哲喀喇人居之,隸三姓副都統。宣統二年置,治佳木斯。三年,徙悅來鎮,隸東北路道。西:格布蘇嶺、猴石山。南:巴虎。東:馬庫力。南:筆架、哈達密。東南:倭肯河,自臨江緣界,及七八虎力河入依蘭,注松花江。西北合音達木、小鈴鐺麥河,入富錦。東南:柳樹河從之。官商路三:西蘇蘇屯達依蘭;東汶登岡,東南寶山鎮,並達富錦。船埠:佳木斯屯,瀕江。

富錦縣:省東北千八百里。清初黑哲喀喇人本部,曰富克錦。光緒七年置協領。三十三年置巡司,隸臨江州。宣統元年改隸東北路道。南:對錦山、別拉音、四方臺。西南:雙崖。東:烏爾古力。北:松花江自樺川入,納柳樹、哈達密河,入臨江。西南七星砬子,大七里星河出東北,緣界合砭石河,逕對面城屯,流分復合。官商路四:西霍悅路達樺川;東古必扎拉達臨江;東南歷臨江二龍山鎮達饒河;南懷德鎮達密山。

饒河縣:省東北二千百四十里。明,尼瑪河堧地。後為窩集部之諾羅路。清初瓦爾喀部人居之,隸寧古塔副都統。宣統元年置,隸東北路道。南:佛力山、大頂。西:小菜根。西南:雙呀堪達。東:東老營盤。東南:烏蘇里江自虎林入,合外七里星、大小別拉、大帶、小安河,北至斯莫勒山。西南:饒河自密山緣臨江界,合大索倫、蛤螞罣、寶清、貛子、里七里星、大佳氣河,入逕城北,又東,右合小佳氣、蛤螞河,逕饒力葛山來會。又東北,入綏遠。官商路二:東沿烏蘇里,分達綏遠、虎林;西沿饒力,分達臨江、密山。

◎附志

寶清州:宣統元年擬置於饒河西境寶清河西。

勃利州:宣統元年擬置於依蘭東南倭肯河上游,即古勃利州地。

臨湖縣:宣統元年擬置於密山東,南臨興凱湖,有小興凱湖。


\end{pinyinscope}