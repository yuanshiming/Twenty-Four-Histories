\article{列傳一百一}

\begin{pinyinscope}
策楞子特通額特清額特成額玉保達爾黨阿哈達哈子哈寧阿

永常覺羅雅爾哈善富德薩賴爾

策楞,鈕祜祿氏,滿洲鑲黃旗人,尹德長子。乾隆初,為御前侍衛。二年秋,永定河決,上出帑命策楞如盧溝橋賑災民。累遷為廣州將軍,授兩廣總督。廣東巡撫託庸劾布政使唐綏祖贓私,下策楞勘讞。策楞雪綏祖枉,上嘉其秉公。尋加太子少傅,移兩江總督。其弟訥親承父爵進為一等公,以征金川失律坐譴。十三年十月,命策楞襲爵,仍為二等公,復移川陜總督。旋以川、陜轄地廣,析置二督,策楞專領四川。時大學士傅恆代訥親為經略,命策楞參贊軍務。傅恆受金川降,班師行賞,策楞加太子太保。

西藏郡王珠爾默特那木札勒狡暴,謀為亂,上命策楞戒備。十五年冬,駐藏大臣傅清、拉布敦誅珠爾默特那木札勒,為其黨所戕,西藏亂,上命策楞及提督岳鍾琪督師戡難。時西藏公班第達獲逆渠卓呢、羅卜藏扎布,戢兵待命。策楞以聞,請率八百人以往,留軍駐打箭爐待徵發。策楞至西藏,與鍾琪及侍郎兆惠,駐藏大臣納穆札爾、班第等審定規制,為西藏善後章程,語詳西藏傳。

雜穀土司蒼旺侵梭磨、卓克基二土司為亂,策楞與鍾琪發兵討之。上以川兵弱,當瞻對、金川用兵後,元氣未復,誡慎重。師戰勝,獲蒼旺,收其地內屬。策楞丁母憂,解官還京師。江南淮、揚水災,命偕尚書劉統勛往勘。因疏河工積習,總督高斌以下皆坐黜,即令策楞署南河總督。河決銅山張家馬路,上以河工非所習,改授兩廣總督。時準噶爾酋達瓦齊庸懦,所部內訌。上銳意用兵,十九年二月,召策楞,命出視師,授定邊左副將軍。阿睦爾撒納之降也,尚書舒赫德在軍察其狙詐,慮且復叛,策楞與共議,以所攜部族置戈壁南,而留阿睦爾撒納及諸頭人丁壯勝兵者從軍。上聞阿睦爾撒納降,將倚以取達瓦齊,得策楞等疏,怒甚,命削職,以閒散在參贊上效力贖罪,發諸子各行省駐防披甲。上遂用阿睦爾撒納為定邊左副將軍,導我師討達瓦齊。二十年五月,師定伊犁,上降詔猶責策楞、舒赫德恇怯乖張,幾僨事。旋以師有功,予策楞副都統銜,令率偏師戍巴里坤。

九月,阿睦爾撒納叛去,上以永常為定西將軍,命策楞參贊大臣上行走。既,聞當阿陸爾撒納叛時,永常引師自穆壘左次巴里坤,罷永常將軍,以命策楞。旋詔逮永常,授扎拉豐阿為將軍。策楞疏言待軍士器械,隨將軍進兵。詔並逮策楞,謂懲其懦也。尋以罪在永常,貸策楞,令屬扎拉豐阿督餉。會準噶爾宰桑克什木等陷伊犁,定北將軍班第等死事。策楞馳疏聞,請合兵進討。上復授策楞副都統銜參贊大臣,扎拉豐阿未至,攝將軍。策楞與喀爾喀諸部貝勒合兵擊敗準噶爾部落,授內大臣,真除定西將軍。上督諸將逐捕阿睦爾撒納甚急。二十一年二月,策楞聞臺吉諾爾布等已得阿睦爾撒納,騰章奏捷,上告於陵廟。進策楞一等公,賜雙眼孔雀翎、寶石帽頂、四團龍補服。三月,策楞復疏言前奏非實,上命停封賞,嚴促進兵逐捕。是月,復克伊犁,阿睦爾撒納走哈薩克。四月,命大學士傅恆視師,逮策楞及參贊大臣玉保。旋得策楞奏,方督兵壓哈薩克境,令擒阿睦爾撒納以獻。上乃令傅恆還京師。時達爾黨阿出西路,哈達哈出北路,與策楞合軍以進,師久次,不得阿睦爾撒納蹤跡。九月,達爾黨阿、哈達哈引兵還屯哈薩拉克。十一月,復命逮策楞、玉保檻送京師,途遇準噶爾兵,為所戕。

子特通額,初發黑龍江披甲。二十三年,以侍衛從將軍兆惠討霍集占,戰黑水,與總兵高天喜等同戰死。圖形紫光閣,列後五十功臣。

特清額,初發杭州披甲。自上虞備用處拜唐阿,十一遷,至嘉慶間,授成都將軍。嘗兩攝四川總督。會有為蜀都賦訐長吏者,給事中胡大成以聞。仁宗命工部尚書託津、光祿寺少卿盧廕溥詣勘,特清額坐徇隱,降三級留任。未幾,卒。

特成額,初發西安披甲。自黏竿處拜唐阿,再遷三等侍衛。師討大金川酋索諾木,高宗命特成額從征。轉戰兩年,自資理北山下克美美卡諸地;攻榮噶爾博最高★,奪康薩爾山半石碉;破密拉噶拉木山梁木城:特成額皆有功,授貴州威寧鎮總兵。乾隆四十二年,上以勛舊世家有世為領侍衛內大臣,因以豐升額遺缺授特成額。三遷授禮部尚書,為成都將軍,三攝總督。尋除湖廣總督。五十年,歲旱,湖北、江蘇、浙江皆饑,特成額疏請發湖南倉穀賑湖北。有餘平值以糶,使商自四川販米至者,見湖北穀值低,得輸以濟江、浙。上獎其不分畛域,得大臣體。尋移雲貴總督,以李侍堯代督湖廣。侍堯疏發上年旱饑,孝感民無食,掠富家儲穀;諸生梅調元者,糾眾與抗,生瘞二十三人。上震怒,逮特成額,籍其家。旋予副都統銜,充烏什辦事大臣。又坐在湖廣失察屬吏侵帑、案牘壅積,屢被譴責。及荊州堤決,復逮下獄論絞,久之,赦。授頭等侍衛、烏魯木齊辦事大臣。嘉慶初,自科布多參贊大臣授兵部侍郎,未上,卒。

玉保,烏朗罕濟勒門氏,蒙古鑲白旗人。自理籓院筆帖式三遷郎中。乾隆三年,擢侍郎。八年,率準噶爾使者入藏熬茶,賜孔雀翎。十二年,復率準噶爾使者入藏熬茶,疏言:「前次入藏,自巴延喀喇納木齊圖穆倫至穆魯烏蘇渡口,道甚險,時方秋冬間少雪,行旅尚便。今冬令大雪,擬改道逾哈什哈嶺左巴延喀喇巴山後,自布魯爾仍至穆魯烏蘇渡口。」報可。十六年,遷正黃旗蒙古都統。十七年,達瓦齊為亂,命偕尚書舒赫德赴北路防邊。十八年,杜爾伯特臺吉策凌等來降,命馳赴犒勞。上以玉保習準噶爾事,命以參贊大臣佐軍事。十九年,輝特臺吉阿睦爾撒納來降,復命馳赴犒勞,率以入覲。

二十年,阿睦爾撒納叛,命仍以侍郎、參贊大臣出北路。師次哈齊克,遣兵至鄂什默納河,收阿睦爾撒納所屬三百餘戶。搜山,獲阿睦爾撒納黨得木齊班咱等。進次安集雅哈,殲阿巴噶齊所屬三百餘戶,圍班雜游牧。尋從逆喇嘛達什藏布,並收其妻子。擢內大臣。二十一年,策楞疏報已獲阿睦爾撒納,行賞,封玉保三等男世襲。玉保獲從賊達永阿,言阿睦爾撒納相距僅一日,玉保執送策楞。又得從賊烏遜,言阿睦爾撒納方出痘,所部尚有厄魯特兵八千、哈薩克兵三千,亦執送策楞。上責玉保退縮,玉保師復進。遣諸將烏爾登等追至庫隴癸嶺,得從賊額林沁,言阿睦爾撒納已逾嶺入哈薩克境,引還,次固勒扎。上怒策楞、玉保不得阿睦爾撒納。策楞又疏言玉保馳檄謂阿睦爾撒納即日就擒,無煩大軍深入,因是勒兵未進,遂命並逮詣京師,旋命姑寬之。玉保疏辨未嘗馳檄阻策楞進兵,上謂:「玉保即未阻策楞進兵,阿睦爾撒納脫於誰手?」因斥其畏葸欺飾,削男爵,奪參贊大臣,改授領隊大臣。玉保疏言阿睦爾撒納僅餘從賊二三人,投哈薩克汗阿布賚,正督兵往索。上以玉保明知叛賊孑身無助,始直前追逐,斥其取巧。命尚書阿里袞詣軍逮策楞,並諭:「玉保已率兵向哈薩克,免其罪,未行則並逮。」尋達爾黨阿疏報玉保師已臨哈薩克,命授頭等侍衛。旋以師久次不得阿睦爾撒納,命仍逮治,與策楞同送京師。道死。

達爾黨阿,鈕祜祿氏,滿洲鑲黃旗人,理籓院尚書阿靈阿次子。初襲曾祖額亦都一等子爵,累官吏部尚書。訥親得罪,請從軍。師還,加太子少保。乾隆十九年,出為黑龍江將軍。策楞得罪,命襲封二等公。是年十二月,上用阿睦爾撒納討達瓦齊,以班第為定北將軍,授達爾黨阿參贊大臣。二十年正月,命將索倫、巴爾呼兵詣軍。五月,定伊犁。師還,命協辦大學士。

及阿睦爾撒納叛,授定邊左副將軍,偕參贊大臣哈達哈,出北路,率師逐捕。十月,改授右副將軍,出西路,而以哈達哈當北路。十二月,復以將印授扎拉豐阿,達爾黨阿仍為參贊大臣。二十一年正月,又以鄂勒哲依、薩賴爾同掌將印。達爾黨阿帥師至珠勒都斯迎薩賴爾。及策楞報獲阿睦爾撒納,達爾黨阿亦賜雙眼孔雀翎。尋自特訥格爾赴安集海,分兵略唐古特游牧。旋以阿睦爾撒納竄入哈薩克,上命西路專任達爾黨阿,北路專任哈達哈,督兵壓哈薩克境,使擒阿睦爾撒納以獻。五月,復授右副將軍。時策楞駐登努勒臺,令達爾黨阿還師。達爾黨阿不從,上即解策楞定西將軍以命達爾黨阿。

八月,師次雅爾拉,哈薩克汗阿布賚遣頭人和集博爾根率四千騎分二隊從阿睦爾撒納走魯臘,而自率千餘騎西行,會於毫阿臘克山下。達爾黨阿師至,遇和集博爾根前隊,自山谷中誘使出,突其中堅,斬五百七十餘級,獲頭人楚魯克。逐敵至努喇,遇和集博爾根後隊,復戰陷陣,得其纛,斬三百四十餘級。阿睦爾撒納部宰桑言阿睦爾撒納易藍纛以戰,戰敗,易服遁。哈達哈亦擊破阿布賚軍,獲頭人昭華什。兩軍合,遣楚魯克、昭華什還諭其渠。時阿睦爾撒納走不過一二里許,遇楚魯克等,使還報偽為哈薩克頭人語,待其汗阿布賚至,且執阿睦爾撒納以獻。達爾黨阿信之,按兵以待。阿睦爾撒納從容捆載去。上聞不得阿睦爾撒納,命繳雙眼翎,召還京師,罷協辦大學士。二十二年二月,奪爵,左授正白旗滿洲副都統。八月,軍中俘阿睦爾撒納從子達什,策楞檻致京師。上始聞達爾黨阿、哈達哈緩追逸賊狀,俱奪官,發熱河披甲。二十三年,授三等侍衛,率西安駐防兵赴軍,師有功,進二等侍衛。卒。

哈達哈,瓜爾佳氏,滿洲鑲藍旗人,黑龍江將軍傅爾丹子。傅爾丹初襲曾祖費英東二等信勇公,乾隆元年,追論失律罪,黜,以哈達哈襲。是時哈達哈已自侍衛累遷領侍衛內大臣,兼勛舊佐領。既,襲爵,復遷鑲紅旗滿洲都統、工部尚書,加太子少保,署兵部尚書、步軍統領。

十九年,師討達瓦齊,授參贊大臣,佐定北將軍班第出北路。尋改領隊大臣。二十年,達瓦齊就俘。再出師討阿睦爾撒納,復授參贊大臣,佐定邊左副將軍達爾黨阿出北路。哈達哈請將索倫、喀爾喀兵為前鋒,上獎其奮勉。尋命代達爾黨阿為定邊左副將軍當北路,移軍布延圖。南自伊克斯淖爾,北至烏哈爾喀碩及烏里雅蘇臺、劄卜堪諸形勝地,皆分兵列戍。二十一年,命自阿爾泰進兵,詔以北路專任哈達哈。特楞古特宰桑敦多克、固爾班和卓等與我師遇,偽請降。哈達哈察其詐,斬敦多克,縶固爾班和卓等,殪其眾。上嘉其勇,再授領侍衛內大臣,賜雙眼孔雀翎。

師至嵩哈薩拉克山,遇哈薩克汗阿布賚擁眾自巴顏山西行,與戰,敗之。復遣諸將瑚爾起、鄂博什、奇徹布等追擊,斬百餘級,獲馬二百。哈達哈不知阿布賚在軍,未窮追;而達爾黨阿與阿睦爾撒納遇,戰既勝,縱使脫去。兩軍合,引還。奪雙眼孔雀翎,命以參贊大臣屯科布多。尋論失阿布賚罪,奪爵,罷領侍衛內大臣,左授兵部侍郎。旋就進尚書,徙屯烏里雅蘇臺。二十二年八月,詔罪狀達爾黨阿、哈達哈,謂:「二臣皆勛舊子孫,襲爵專閫,而因循觀望,坐失軍機若此。」盡奪其官,發熱河披甲。二十三年,與達爾黨阿同授三等侍衛從軍,同進二等侍衛。

子哈寧阿,自藍翎侍衛累遷寧夏副都統。哈達哈為定邊左副將軍,哈寧阿為領隊大臣。尋命以參贊大臣佐定西將軍達爾黨阿出西路。旋令詣伊犁佐定邊右副將軍兆惠。兆惠困濟爾哈朗,力戰突圍出,哈寧阿與焉,予三等輕車都尉世職。又從兆惠擊巴雅爾,功最,賜玉韘、荷包、鼻煙壺。哈達哈奪爵,以哈寧阿襲,擢鑲黃旗漢軍都統。乾隆二十三年,復授參贊大臣,佐靖逆將軍雅爾哈善討霍集占。圍庫車,霍集占脫去,與雅爾哈善同逮送京師。二十四年正月,雅爾哈善棄市。上以哈寧阿為參贊,責薄於將軍,又念濟爾哈朗力戰有勞,命系獄待秋決。十一月,富德師至巴達克山,遣使令縛送霍集占。上以達爾黨阿、哈達哈皆在軍,不自奮請行,詔詰責,因言:「哈寧阿秋讞本當決,哈達哈稍有事效,尚當寬宥,今豈可曲貸?重念費英東勛勞,不忍刑諸市。」命賜自盡,且令馳諭哈達哈,哈達哈已先以十月卒於軍。

永常,董鄂氏,滿洲正白旗人。自三等侍衛累遷鑲紅旗滿洲都統。乾隆五年,命如安西按事,即授安西提督,屯哈密,賜孔雀翎、紅絨結頂冠。十五年,授湖廣總督。羅田民馬潮柱為亂,討平之。十八年,上將征準噶爾,命為欽差大臣,駐安西。旋移陜甘總督,加太子少保。

輝特臺吉阿睦爾撒納來降,言達瓦齊昏暴。上決策用兵,召永常詣京師,諭行軍機宜,遂以內大臣授定西將軍。時上倚阿睦爾撒納及來降宰桑薩賴爾取達瓦齊,以阿睦爾撒納副定北將軍班第出北路,以薩賴爾副永常出西路,仍諭阿睦爾撒納、薩賴爾為軍鋒,敕永常督軍鋒先發。永常令諸道軍兼程並進,上責其誤。永常師次巴里坤,命還肅州。永常還督餉,有所計畫,上皆不謂然。師定伊犁,俘達瓦齊,詔責:「永常但知師行糧隨,沾沾議接濟。今功已成,何慮糧不足?因糧於敵,從來勝算。如永常奏,展轉挽運,動逾數十日,庸有濟乎?」因左授吏部侍郎。

阿睦爾撒納叛,犯伊犁,永常師左次,上責其怯懦,罷內大臣、定西將軍,以副都統銜為參贊。厄魯特諸臺吉有不從阿睦爾撒納叛者,宰桑扎木參等率數千人詣永常請附屯。永常疑其詐,挾宰桑為質,兼程卻走,恐賊躡其後,徵策楞赴援,並檄阿敏道引還,同駐巴里坤。上命奪官逮京師,行至臨潼,道卒。仍籍其家,戍其子拉林。

覺羅雅爾哈善,字蔚文,滿洲正紅旗人。雍正三年繙譯舉人,自內閣中書四遷,乾隆三年,授通政使。御史邱玖華疏論九卿議事不公,別疏請錄用賢良祠大臣子孫。雅爾哈善劾玖華為原任侍郎勵宗萬門生,宗萬祖杜訥為賢良祠大臣,玖華劾九卿議事不公,示剛正,實為起宗萬地。上謂:「錄用賢良祠大臣子孫,不過虛銜微秩,視其材可用然後用之。豈有嘗為侍郎獲罪因賢良祠大臣子孫而輒起者?勵宗萬雖愚,計不出此。玖華所論九卿議事不公,切中時弊。諸臣見之,宜深自儆省。若遷怒建言者,是為不知恥!」命解雅爾哈善任。令莊親王允祿、平郡王福彭會大學士以下嚴鞫,雅爾哈善言語得之右通政陳履平,因請皆奪官。上責王大臣議不當,命奪雅爾哈善官,履平下吏議。四年,特起四川龍安知府。五年,以憂去。六年,授江南松江知府,移蘇州知府。九年,遷福建汀漳道。雅爾哈善在松江、蘇州皆有聲績,其去,民思之。十三年,以福建按察使署江蘇巡撫。上元民毀制錢,雅爾哈善論如律,復以數少乞原,上責其寬縱,命奪職留任。十五年,雅爾哈善議經徵未完不及一分知縣許惟枚等,皆劾罷。總督黃廷桂劾不當下吏議,當奪官,仍命留任。尋入為戶部侍郎。十六年,復出為浙江巡撫。十九年,復入為戶部侍郎,命軍機處行走,旋授兵部侍郎。

二十年,師討阿睦爾撒納,授參贊大臣,出北路。二十一年,命改赴西路,令駐巴里坤辦事。疏請徙布庫努特降人於烏蘭烏蘇,與前降噶勒雜特人同牧。未幾,綽羅斯汗噶勒藏多爾濟叛,噶勒雜特人哈薩克錫喇等與為響應,回部降人莽噶里克亦從之。雅爾哈善擒其黨並其子白和卓。十二月,上獎雅爾哈善實心治軍事,加內大臣銜。和碩特降酋沙克都爾曼吉不與阿睦爾撒納之亂,率所部徙巴里坤附城為牧地以居。噶勒藏多爾濟巴雅爾之叛,上寄諭雅爾哈善,令密察沙克都爾曼吉蹤跡。雅爾哈善方內疑,又以餉不時至,沙克都爾曼吉請糧不能給,乃使裨將閻相師將五百人入其壘,若迷途借宿者。夜大雪,相師吹笳,督兵襲其廬。沙克都爾曼吉驚起,其妻與相抱持,至死不釋,其眾四千餘人殲焉。雅爾哈善疏報沙克都爾曼吉與綽羅斯叛黨扎那噶爾布相通,戮以杜後患。又遣兵赴魯克察克剿莽噶里克,上嘉其奮往。

二十二年春,定邊右副將軍兆惠自伊犁率師逐捕噶勒藏多爾濟等,雅爾哈善遣侍衛圖倫楚將八百人益兆惠軍。提督傅魁師至鹽池,遇莽噶里克率三十二人入塞探白和卓消息,傅魁執而殺之,雅爾哈善疏聞。上以莽噶里克為叛首,當讞定行誅,命逮傅魁送京師。兆惠師自濟爾哈朗至特納勒爾,為敵圍,得圖倫楚援乃解。尋召雅爾哈善還京師,授戶部侍郎。四月,復授參贊大臣,令駐濟爾哈朗。九月,擢兵部尚書。十二月,令移駐魯克察克,總理屯田。

二十三年二月,命為靖逆將軍,帥師討霍集占。五月,師至庫車,霍集占所屬頭人阿卜都克勒木城守。雅爾哈善督師合圍,斷其水草,城賊出戰,屢敗之。六月,敗援賊於托木羅克。霍集占自將八千人,具最精巴拉鳥槍,行阿克蘇戈壁來援。雅爾哈善督兵戰庫車南,斬千餘級。霍集占負傷入庫車,獲其纛。庫車依岡為城,以柳枝、沙土密築甚堅,砲攻不能入。提督馬得勝策穴地入城,距城北一里為隧,已及城。雅爾哈善督之急,我兵夜秉燧入穴。城賊見火光,於城內為橫溝,水入隧,我兵皆沒。頭人鄂對告雅爾哈善曰:「庫車食且盡,霍集占必出走。城西鄂根河水淺可涉,北山通戈壁走阿克蘇。宜分兵屯此二隘,霍集占可擒也。」雅爾哈善以鄂對新降,不可信。越八日,霍集占乘夜引四百騎啟西門,涉鄂根河遁。又數日,阿卜都克勒木復夜遁。餘頭人阿拉難爾等率老弱出城降。雅爾哈善雜訊城人,謂沙呢雅斯等五人為阿卜都克勒木死黨,因殺之。

疏入,上聞不得霍集占,盛怒,奪雅爾哈善官。雅爾哈善劾副都統順德訥疏縱,又劾馬得勝失機。上曰:「雅爾哈善始劾順德訥,繼劾馬得勝,無一語引罪。不思身任元戎,指麾諸將者誰之責歟?此而不置之法,國憲安在?」命兆惠至軍斬順德訥以徇,逮雅爾哈善及得勝送京師。二十四年正月,逮至,命王公大臣會鞫,以雅爾哈善老師糜餉失機事,論斬,遂見法。後二日,並斬得勝。自雅爾哈善死,高宗知沙克都爾曼吉無叛狀,賦詩斥其殺降。

富德,瓜爾佳氏,滿洲正黃旗人,駐防吉林。乾隆初,自護軍擢至三等侍衛。十三年,從經略大學士傅恆征金川,擒賊黨阿扣,遷二等侍衛。師還,累遷副都統。二十年,師征準噶爾,命送綽羅斯臺吉噶勒藏多爾濟等赴軍。擢參贊大臣,督西路臺站。阿睦爾撒納所屬唐古忒部見阿睦爾撒納入伊犁,謀遁去。二十一年,富德帥師至鄂塔穆和爾,遇唐古忒眾千餘營樹林蒲葦中,擊殺二十餘人,追至色白口山內。賊據險分隊抵御,奪寨六,斬獲無算。唐古忒部遁伊犁,追至察罕鄂博,復遇哈薩克兵千人與唐古忒隊合。富德奮勇沖擊,斬百餘級,奪回被掠集賽噶雜特三十餘戶,擒臺吉恩克巴雅爾等四十餘人。上獎富德奮勉,授正黃旗蒙古都統。

二十二年,定邊將軍成袞扎布赴巴里坤,以富德為參贊大臣。定邊右副將軍兆惠疏報與成袞扎布分道進兵,命富德從兆惠軍。阿睦爾撒納還掠扎那噶爾布游牧,富德追剿,收復巴爾達穆特各鄂拓克。得叛酋巴雅爾蹤跡,遂深入逐捕,奪隘五。至愛登蘇,哈薩克汗阿布賚遣使降。阿睦爾撒納逃入俄羅斯,尋死。叛酋哈薩克錫喇、布庫察罕未獲,命富德逐捕。二十三年,招右部哈薩克圖裏拜及塔什罕回人圖爾占俱來降,遣使入覲。上以富德在軍久,招撫西哈薩克有勞,予雲騎尉世職。

是時雅爾哈善討霍集占無功,兆惠代將,師銳進,被圍,命富德為定邊右副將軍赴援。二十四年正月,軍次呼爾璊,遇賊騎五千,轉戰五日四夜。會參贊大臣阿里袞送馬至,分翼馳突,賊眾大潰,殺巴爾圖十五人、大伯克數十人、賊千餘。酋布拉呢敦中槍傷劇,舁入城,旋遁喀什噶爾。兆惠解圍出,以功封三等伯。師進次葉爾羌河岸,復戰敗賊,進封一等成勇伯。霍集占黨侵和闐,富德赴援,破賊。進攻葉爾羌,霍集占兄弟棄城遁,追敗之於阿勒楚爾,又敗之於伊西洱庫爾淖爾,竄巴達克山。軍從之,令擒獻,巴達克山汗素勒坦沙獻霍集占首。師還,進封一等靖遠成勇侯,賜雙眼孔雀翎,官其子侍衛,授領侍衛大臣。二十五年,復授御前大臣,圖形紫光閣,賜紫禁城騎馬,命軍機處行走。尋授理籓院尚書、正黃旗蒙古都統。副都統老格盜官駝事發,鞫實,言寄馬富德牧廠,有牲畜數千。上以富德暴貴,安得有牧廠,命都統巴爾品勘驗,旋奏富德家產擁貲至三萬餘。命和親王等會鞫,得富德出兵時留官馬,索蒙古王公牲畜,並攜緞、布、煙、茶牟利狀,下獄,吏議當斬,上命改監候。二十八年,赦,授散秩大臣。三十三年,將軍明瑞徵緬甸死綏,參贊大臣額勒登額坐逗遛得罪。額勒登額亦吉林駐防,與富德有連,富德坐誤舉,罷散秩大臣,下獄,吏議當斬,上命入緩決。三十六年,赦,授三等侍衛。

三十八年,將軍溫福征金川,軍潰木果木。發健銳、火器兩營兵益阿桂軍,授富德頭等侍衛,為領隊大臣,從副將軍明亮出南路。富德自真登、梅列舊卡進兵,克得布甲喇嘛寺、得裏兩面山梁、日寨、策爾丹色木諸隘,復進克僧格宗、馬柰、絨布寨、卡卡角諸隘,授副都統,待缺。復進克沙錫理穆當噶爾碉卡、羊圈河橋。四十四年,請撥兵三千往宜喜助明亮,允之。攻噶咱普得婁,奪卡五;攻布咱爾尼山梁,奪沿河卡五;攻庚額特山梁,奪大碉三、卡八;攻噶咱普得爾窩,賊棄碉竄,追至馬爾邦,乞降。富德從軍二年,未能大有摧破,屢下詔敦責之,至是,命下部敘功。

金川平,阿桂劾富德濫賞,侵土兵鹽菜銀兩彌不足,下桂林覈實,復命袁守侗如川會阿桂具獄。富德密上清字疏訐阿桂,上命檻送京師。廷訊,乃具服濫賞,並以銀六鋌入己;又受知府曾承謨餽金五十兩,並劾副將廣著,不待命即令其充兵,廣著自戕死。清字疏復稱「阿桂手持黃帶,語不遜」,坐誣告大逆,例當斬,遂見法。

薩賴爾,蒙古正黃旗人。本厄魯特頭人,隸準噶爾臺吉達什達瓦為宰桑。乾隆十五年,準噶爾內亂,薩賴爾率所屬四十七戶降,安置察哈爾。命入旗,授散秩大臣。準噶爾臺吉喇嘛達爾扎請遣薩賴爾歸,不許。授參贊大臣,出北路。十九年,烏梁海得木齊扎木參入邊,薩賴爾以五百人御之,擒扎木參,而遣收凌、朔岱、訥庫勒等十人還。事聞,授內大臣。既,遣還諸人來告宰桑雅爾都、得木齊阿茂海欲來歸,乞駐牧烏蘭固木、克木克木齊克。薩賴爾言雅爾都等親至,許駐特斯河,否則驅之阿爾臺山外;並請發厄魯特兵聽調。尚書舒赫德以為未便,上諭薩賴爾相機而行。命舒赫德會同薩賴爾及車凌等選臺吉、宰桑可信任者將兵二百人,並令侍衛永柱會總管阿敏道選察哈爾八旗兵五百,交薩賴爾為招諭驅逐之用。

薩賴爾兵至卓克索,烏梁海宰桑雅爾都、車根、赤倫、察達克、圖布慎、瑪濟岱各鄂拓克竄徙阿爾臺山外。薩賴爾奏:「烏梁海等已遠遁,但貪戀故土,必仍回牧。彼時整兵速出,易於收服。請暫撤兵還。」允之。輝特臺吉阿睦爾撒納來降,命薩賴爾迎勞頒賞。旋偕喀爾喀貝子車木楚克扎布等以千八百人擊雅爾都、車根、赤倫、察達克四宰桑於察罕烏蘇,敗之,獲牛馬無算。初,有扎哈沁宰桑庫克新瑪木特者犯卡倫,追之弗獲,達青阿誘執之。上責其不武,令縱之去。瑪木特移牧布拉罕託輝,不即降。道遇通瑪木特,被擒,縶之諾海克卜特勒。薩賴爾詗知之,自烏蘭山後掩擒通瑪木特,並護庫克新瑪木特送軍營,安置其戶畜於庫卜克爾克勒。上嘉之,授子爵世襲,遷正白旗領侍衛內大臣。

時定議徵達瓦齊,命薩賴爾為定邊右副將軍。二十年正月,率師偕參贊大臣鄂容安等出西路。師行,厄魯特降者於途中肆劫。上戒鄂容安,以己意喻薩賴爾使自斂戢。阿睦爾撒納請移牧烏里雅蘇臺,招輝特部眾。上察其意叵測,諭薩賴爾令防範,並促其進兵。薩賴爾等疏報扎哈沁得木齊巴哈曼集以三百餘戶,宰桑敦多克以千餘戶來降。復遣侍衛瑚集圖招諭達瓦齊同族臺吉噶勒藏多爾濟,尋率臺吉諾海奇齊等三十餘人來降,詔封為綽羅斯汗。上諭獎薩賴爾,解所佩荷包以賜,並賜雙眼孔雀翎。三月,薩賴爾與諸將和起、齊努渾自羅克倫督兵赴博羅塔拉,與北路班第等軍合。疏言:「招撫綽羅斯臺吉袞布扎卜等,皆率所屬來降,凡四千餘戶。葉爾羌、喀什噶爾和卓木獻玉盤請降,令各回原牧;降人請與地耕牧,令往吐魯番、莽阿里克處受地。阿睦爾撒納屬人二百餘及額林哈畢爾噶窮夷八百餘戶,令附屬扎哈沁宰桑,有牲畜者,畀耔種,令其耕牧。並自羅克倫啟行,馳檄達瓦齊,曉諭利害。」上獎其籌畫妥協,以御用寶石朝珠賜之。

薩賴爾兵至登努勒臺,將軍班第等亦至尼楚袞,兩軍合。達瓦齊居伊犁河西格登,不設備。五月,西路軍自固勒扎渡口越推墨爾里克嶺直抵格登,達瓦齊驚遁,未幾就擒。伊犁平,詔封薩賴爾一等超勇公,賜寶石頂、四團龍服。六月,軍還。徵阿睦爾撒納入覲,薩賴爾同班第、鄂容安駐守伊犁,留兵五百為衛。七月,阿睦爾撒納謀叛,逗遛途中。班第等屢疏入告,薩賴爾亦以為言。上密諭諸臣擒治,弗能決,阿睦爾撒納遂遁。其徒克什木等為亂,班第、鄂容安死之,薩賴爾更衣降。十二月,薩賴爾遣使詣巴里坤辦事大臣和起,以阿睦爾撒納蹤跡告,請發兵往擊。和起以聞,上令將軍策楞傳諭慰勞,賚荷包、鼻煙壺,俟其至賜之。又命理籓院員外郎唐喀祿董其游牧。

二十一年正月,薩賴爾脫出,至吐魯番。巴里坤參贊大臣達爾黨阿率兵往會。薩賴爾疏請罪,上令駐特訥格爾,仍授定邊右副將軍。三月,策楞疏言:「侍衛巴寧阿自伊犁歸,言克什木之亂,將軍班第等自固勒扎赴崆格斯禦之。賊甫至,薩賴爾欲奔。鄂容安曰:『賊來當戰,胡急走?』薩賴爾答言:『爾何知?』遂策馬去,眾從之。班第等僅餘司員侍衛及衛卒六十人。夜賊至,班第等遂自殺。」上命逮薩賴爾入都,鞫實,以薩賴爾降人,貸其死,命錮之獄。班第等喪還,執克什木馘以祭,令薩賴爾觀之。尋以叛黨漸次就擒,釋出獄。二十四年,授散秩大臣、鑲白旗蒙古副都統、乾清門行走。旋擢內大臣,復封二等超勇伯。卒。圖形紫光閣。

論曰:國重有世臣,然承平久,富貴宴安,恆不足任使;出任軍旅,兵未接,將已內怯,幾何不僨事耶?策楞輩擁兵玩寇,其病正坐此。雅爾哈善文墨吏,其殺降亦以內怯。富德族微,力戰致通顯,有功而不善居,卒以遘禍。薩賴爾反覆,★甚著,獨以降人蒙寬典,幸矣!


\end{pinyinscope}