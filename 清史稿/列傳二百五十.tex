\article{列傳二百五十}

\begin{pinyinscope}
唐景崧劉永福

唐景崧,字維卿,廣西灌陽人。同治四年進士,選庶吉士,改吏部主事。光緒八年,法越事起,自請出關招致劉永福,廷旨交岑毓英差序。景崧先至粵,謁曾國荃,韙其議,資之入越。明年,抵保勝,見永福,為陳三策,謂:「據保勝十州,傳檄而定諸省,請命中國,假以名號,事成則王,此上策也;次則提全師擊河內,中國必助之餉;若坐守保勝,事敗而投中國,策之下也。」永福從中策。戰紙橋,敵潰,為作檄文布告內外,檄出,遠近爭響應。越嗣君為法脅,莫能自振,景崧乘間勸內附。永福意猶豫,景崧曰:「子能存亡繼絕,即所以報故主也。且阮福時已薨,無背主嫌。」永福意稍動,於是廣招戎幕謀大舉。上念景崧勞,賞四品銜。

景崧上書言:「越南半載之內,三易國王,欲靖亂源,莫如遣師直入順化,扶翼其君,以定人心。若不為籓服計,不妨直取為我有,免歸法奪,否則首鼠兩端,未有不敗者也。」十年,駐興化,會北寧告急,毓英令景崧導永福往援。初,桂軍黃桂蘭等方守北寧,劉團被困山西,坐視不救,永福憾之深。至是景崧力解之,始往;並勸桂蘭離城擇隘而守,弗聽。景崧輕騎入諒山,與徐延旭量戰守。適扶良警,請還犒劉軍,行至郎甲,湧球陷,阻弗達。回諒,謂延旭曰:「寇深矣!亟宜收潰卒,定人心,備糗糧,集軍械,分兵守險,以保茲土。」於是令綜前敵營務,扼巴塘嶺。敵再至,再卻之,廣軍氣稍振。

會張之洞令其募勇入關,乃編立四營,號景字軍,為規越廣軍之一。朝廷賞加五品卿。景崧遂取道牧馬,行千二百里,箐壑深岨,多瘴厲,人馬顛隕不可稱計。既至,數挫敵鋒。毓英高其能,復以潘德繼滇軍屬之,兵力乃益厚,進頓三江口。逾月,法人攻劉軍吳鳳典營,景崧率談敬德馳救,大捷。敵既退,遂先薄宣光。城外地故荒服,乃督軍開山斬道,首龍州,訖館司,創設臺站,滇桂道始達。已而軍其南門,敵開壁出蕩,疾擊之,逼城而壘,槍彈雨坌,攻益力。是時天霪雨,運饋絕,吏士無人色。逾歲,滇軍丁槐攻城,桂軍雖饑疲,然猶據山巔轟擊。法人殊死鬥,不可敗。毓英慮其斷後援,令勿拚孤注,於是退頓牧馬。有旨罷戰,遂入關。論宣光獲勝功,賞花翎,賜號霍伽春巴圖魯,晉二品秩,除福建臺灣道。十七年,遷布政使。二十年,代邵友濂為巡撫。

臺灣自設巡撫,首任劉銘傳,治臺七年,頗有建設,詳銘傳傳。銘傳去,友濂繼之,丈地清賦,改則啟徵,迭平番亂,建基隆砲臺。及景崧蒞任,日韓啟釁,亟起籌防。永福分鎮南澳。景崧自與永福共事,積不相能,乃徙永福軍臺南,而自任守臺北,未幾而李文奎變作。文奎故直隸匪,從淮軍渡臺,居景崧麾下為卒。有副將餘姓者,緣事再革之,文奎忿甚,即撫署前斬其頭,護勇內應,爭發槍,將入殺景崧。景崧出,叛卒見而怖之,斂刃立,並告無事。景崧慰之,翻令文奎充營官,出駐基隆。於是將領多離心,兵浸驕不可制。

割臺議起,主事邱逢甲建議自主,臺民爭贊之。乃建「民國」,設議院,推景崧為總統。和議成,抗疏援贖遼先例,請免割,不報,命內渡。臺民憤,乃決自主,制藍旗,上印綬於景崧,鼓吹前導,紳民數千人詣撫署。景崧朝服出,望闕謝罪,旋北面受任,大哭而入。電告中外,有「遙奉正朔,永作屏籓」語,置內部、外部、軍部以下各大臣。命陳季同介法人求各國承認,無應者。無何,日軍攻基隆,分統李文忠敗潰。景崧命黃義德頓八堵,遽馳歸,詭言獅球嶺已失,八堵不能軍,且日人懸金六十萬購總統頭,故還防內亂,景崧不敢詰也。是夜,義德所部譁變。平旦,日軍果占獅球嶺,潰兵爭入城,城中大驚擾亂,客勇、土勇互仇殺,尸遍地。總統府火發,景崧微服挈子遁,附英輪至廈門,時立國方七日也。二十八年,卒。

劉永福,字淵亭,廣西上思人,本名義。幼無賴,率三百人出關,粵人何均昌據保勝,即取而代之。所部皆黑旗,號黑旗軍。

同治末,法人陷河內,法將安鄴構越匪黃崇英謀占全越,擁眾數萬,號黃旗。越王諭永福來歸,永福遂繞馳河內,與法人抗,設伏以誘斬安鄴,覆其全軍。法人大舉入寇,永福軍頻挫。越人懼,乃行成,而授永福為三宣副提督,轄宣光、興化、山西三省,設局保勝,榷釐稅助餉。有黃佐炎者,越駙馬,以大學士督師。永福數著戰功,匿不聞,永福銜之。越難深,國王責令佐炎發兵,六調永福不至,然越王始終思用之。

光緒七年,法人藉詞前約互市紅河,脅越王逐永福。越王佯調解,而陰令勿徙。法大怒,逾歲,入據河內。永福憤,請戰,出駐山西,逕諒山,謁提督黃桂蘭,乞援助。會唐景崧至,面陳三策,永福曰:「微力不足當上策,中策勉為之!」朝旨賞十萬金犒軍,永福入貲為游擊。戰懷德紙橋,陣斬法將李威利,越王封一等男。既又敗之城下,法人決堤掩其軍,越人具舟拯之出,退頓丹鳳,與法人水陸相持,苦戰三日,部將黃守忠攻最力。敵大創,乃浮艦攻越都,懸萬金購永福,越乞降。永福欲退保勝,黑旗軍皆憤懣,守忠自請以全師守山西,功不居,罪自坐,永福乃不復言退。無何,聞法軍至,遂出駐水田中,而軍已罷困,及戰,大潰,退保興化。

九年,法人要議越事,岑毓英力言土寇可驅,永福斷不宜逐,上韙之,命永福相機規河內,並濟以餉。十年,毓英次嘉喻關,永福往謁,毓英極優禮之,編其軍為十二營。法人聞之,改道犯北寧。永福馳援,逕永祥金,英、法教民梗阻,擊卻之。比至,粵軍已大潰,永福奪還扶朗、猛球砲臺。俄北寧失,力不支,再還興化。復以糧運艱阻,改壁文盤洲大灘,候進止。

毓英奏言:「永福為越官守越地,分所應為,若畀以職,將來邊徼海澨,皆可驅策。」於是擢提督,賞花翎。而李鴻章堅持和議,猶責其騷動。已,和局中變,上令永福軍先進。法人擾宣光,永福窖地雷待之,連日隱卒以誘敵,不敢出。復徙營偪城,三戰皆利。敵援至,毓英遣水師溯河而上,永福夾流截擊,奪其船二十餘艘,斬馘數十級,法人愕走。逾月,法艦入同章,毓英遣將分伏河東西,永福居中策應,兩岸轟擊,敗之,復以全力扼河道。十一年,法軍攻左域,守忠失同章不守,諸軍敗挫,永福退浪泊。停戰詔已下未至,猶大捷臨洮。論勝宣、臨功,賜號依博德恩巴圖魯。和議成,法人要逐如故。張之洞令永福駐思欽,不肯行。景崧危詞脅之,乃勉歸於粵,授南澳鎮總兵。

二十年,中日釁起,命守臺灣,增募兵,仍號黑旗。景崧署巡撫,徙其軍駐臺南。及臺北陷,景崧走,臺民以總統印綬上永福,永福不受,仍稱幫辦。日艦駛入安平口,擊沉之。攻新竹,相持月餘,兵疲糧絕,永福使使如廈門告急,並電緣海督撫乞助餉,無應者。而臺南土寇為內間,引日軍深入,破新化,陷雲林,掇苗慄,轟嘉義,孤城危棘,永福猶死守。日臺灣總督樺山資紀貽書永福勸其去,峻拒之。日軍乃大攻城,城陷,永福亡匿德國商輪,日軍大搜不獲。內渡後,詔仍守欽州邊境。後卒於家。

永福骨瘦柴立,而膽氣過人,重信愛士,故所部皆盡死力雲。

論曰:清初平定臺灣,用兵數十載,始入版圖。甲午議和,遽許割讓,天下莫不同憤焉。臺民奮起,擁景崧為總統,建號永清,此實國民自主之始,七日遽亡。景崧初說永福王越,乃自為之,竟不可以終日,雖有知慧,不如乘勢,豈不然哉?永福戰越,名震中外,談黑旗軍,輒為之變色。及其渡臺,已多暮氣,景崧又不與和衷,卒歸同敗,此不僅一隅之失也,惜哉!


\end{pinyinscope}