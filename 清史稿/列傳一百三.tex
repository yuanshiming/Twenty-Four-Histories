\article{列傳一百三}

\begin{pinyinscope}
瑚爾起愛隆阿弟巴靈阿舒明福祿齊里克齊

閻相師伊柱努三烏勒登

瑚爾起,瓜爾佳氏,滿洲鑲藍旗人。自筆帖式累遷協領。乾隆十三年,從征金川。遷呼倫貝爾總管。二十年,從征準噶爾,加副都統銜。二十一年,從參贊大臣達爾黨阿自珠爾都斯逐捕阿睦爾撒納,詗知阿睦爾撒納竄哈薩克,從定邊左副將軍哈達哈以師臨之。哈薩克汗阿布賚拒戰,擊敗之,斬百餘級,得馬二百餘。獲其頭人,言阿睦爾撒納方在泥雅斯圖山,檄阿布賚擒獻。杜爾默特貝勒巴圖、伯羅特等潛通阿睦爾撒納,瑚爾起與戰輝巴朗山,執伯羅特,盡殲其部眾,及阿睦爾撒納所留烏梁海五十餘戶。

沙喇斯、瑪呼斯既降復叛,掠臺站,而布魯古特臺吉琿齊等戕察哈爾總管巴寧阿以叛。上命瑚爾起偕鄂實、三格副哈寧阿,將千人駐濟爾哈朗、巴里坤適中地,捕琿齊及沙喇斯、瑪呼斯部眾。瑚爾起偕鄂實追剿扎哈沁逃賊,又偕副都統巴圖濟爾噶勒自呼斯坦至尼勒喀河,偵琿齊等百餘戶游牧,突擊,執之。

尋從師自伊犁逐剿諸回部,至善塔斯巔,招降布魯特頭人圖魯啟拜、鄂庫及其部眾,搜捕阿里瑪圖河逸賊。上以索倫兵從征久,召瑚爾起及副都統鄂博什率以還,瑚爾起等仍請從軍。將軍兆惠攻霍集占於葉爾羌,被圍,定邊右副將軍富德檄瑚爾起及巴圖濟爾噶勒率索倫兵自伊拉里克赴援,以馬駝未至,負糧械步行戈壁中。上獎諭,即授正白旗蒙古副都統。師至巴爾楚克,兆惠圍已解,與富德軍合。霍集占之徒阿卜都克勒木等侵和闐,攻哈拉哈什,侍衛齊凌扎卜請援,兆惠令瑚爾起與巴圖濟爾噶勒督兵赴援。齊凌扎卜馳告,夜行至伊立齊,賊聞兵至,引退。詗知賊騎七百餘屯博爾齊,天大霧,瑚爾起督兵突擊,賊潰走,退至皁窪勒河,斬百餘級,收回人四千餘戶,和闐遂平。上賦博羅齊行紀事,賜瑚爾起雲騎尉世職。

師自喀喇烏蘇逐捕霍集占,至阿爾楚爾。賊設伏兩山間,我軍張兩翼擊之,賊敗走三十里,負山而屯。瑚爾起等自山麓橫沖入陣,師夾擊,賊大敗,越山遁,師從之,至伊西洱庫爾淖爾。瑚爾起等為伏東山,側擊,賊復大敗,霍集占竄入巴達克山。巴達克山汗素勒坦沙獻霍集占首。瑚爾起將索倫兵還,賚銀幣,圖形紫光閣,列前五十功臣。瑚爾起疏言:「呼倫貝爾多水泉,可耕。請選塔裏雅沁降回百戶往耕。」上命瑚爾起以副都統為呼倫貝爾總管,董其事。移黑龍江副都統。從征緬甸,收猛拱、猛養諸地。卒於軍。賜騎都尉,並前世職為一等輕車都尉,祀昭忠祠。

愛隆阿,覺爾察氏,滿洲正黃旗人。自前鋒侍衛累遷齊齊哈爾副都統。乾隆二十一年,授領隊大臣,赴巴里坤軍營。偕參贊大臣富德逐捕巴雅爾,至愛登蘇,遇阿布賚部眾突出,數與戰,卻之。自巴爾楚克至濟爾哈朗置臺站,逐賊沙喇博和什嶺,遇都爾伯特納木奇游牧,乞降,旋遁去,愛隆阿追及之,殺千餘人,納木奇遂納款。師至察罕烏蘇,收厄魯特宰桑烏魯木游牧百餘戶。師屯濟爾哈朗,命愛隆阿駐守濟爾哈朗、巴里坤適中地。尋從靖逆將軍雅爾哈善討霍集占。先是愛登蘇之戰,侍衛奇徹布戰沒,至是愛隆阿上言:「前擒巴雅爾,奪還奇徹布尸,富德未及疏列。」定邊將軍兆惠疏言:「愛隆阿原報所無,事後追論,顯為爭功,請嚴議。」詔原之。

師圍庫車,賊來援,愛隆阿等與戰於戈壁,殲賊甚眾。霍集占將五千人續至,愛隆阿等率吉林及索倫兵千騎逐賊至鄂根河側,與戰,迫賊入水,死者三千餘人。拔其纛,驛致京師。上為賦回纛行,獎其能戰。旋從將軍兆惠至葉爾羌,與霍集占部眾戰,當左翼。兆惠被困,靖逆將軍納穆札爾赴援,愛隆阿將兵截喀什噶爾賊援路。徼巡臺站,至托罕塔罕,遇賊,剿殺百餘人。上授愛隆阿參贊大臣,令與定邊右副將軍富德援兆惠。愛隆阿戰呼爾璊,再戰葉爾羌河,遂與兆惠軍合。尋引兵駐烏什,兼防喀什噶爾,予雲騎尉世職。復從富德逐霍集占,戰於伊西洱庫爾淖爾。徼巡臺站,值嗎唬斯、賓巴等謀劫察罕烏蘇臺站,以兵追襲,斬獲殆盡,進騎都尉世職。師還,授正白旗護軍統領,兼鑲白旗蒙古副都統。圖形紫光閣,列前五十功臣。再進一等輕車都尉兼一雲騎尉世職。授伊犁參贊大臣。卒。

弟巴靈阿,自親軍校累遷二等侍衛,授察哈爾總管。賜坤都爾巴圖魯名號,授領隊大臣。在博羅齊搜捕厄魯特部眾,遇伏戰死,賜雲騎尉世職,圖形紫光閣,列後五十功臣。

舒明,烏梁海濟勒莫特氏,蒙古正黃旗人。自二等侍衛累遷都察院左副都御史、正黃旗護軍統領。命赴北路軍,為諸部降人董理游牧。旋授吏部侍郎。詗知降人訥默庫戕臺站侍衛,謀以所部叛,馳奏。敕參贊大臣阿蘭泰往捕治,阿蘭泰請益兵,上責其紛擾。訥默庫就擒,上以舒明籌策得宜,而阿蘭泰推諉遲誤,奪阿蘭泰三等男爵畀舒明。

舒明在邊,諸部降人至者,為之拊循。噶勒雜特宰桑根敦降,上授佐領,使與丹畢游牧同處。都爾伯特臺吉伯什阿噶什、烏巴什降,上授伯什阿噶什親王、烏巴什貝子,游牧額爾齊斯,舒明為陳請留屯哈達青吉勒。達什達瓦部降,編為三旗,移阿爾臺;其續至者,使處扎哈沁舊游牧地。策凌烏巴什、巴圖博羅特及達瑪林等部眾貧甚,疏請賑,上為發米六百石。上聞和托輝特青滾雜卜將叛,命舒明詗之。舒明言叛已著,命會將軍成袞札布等捕治。授參贊大臣,成袞札布令將科布多兵二百以往。上命侍衛巴寧阿勒泰將三百人為舒明佐。旋命偕成袞札布駐烏里雅蘇臺。授理籓部侍郎。再遷綏遠城將軍,兼領歸化城都統。二十七年,卒。

子雅滿泰,襲三等男。累遷正白旗蒙古副都統。坐事左授頭等侍衛。與保泰同充駐藏大臣。廓爾喀侵後藏,與保泰同得罪,荷校被杖。復起至頭等侍衛。卒。

福祿,旺察氏,蒙古正白旗人。自護軍校累遷福建建寧鎮總兵。內移正藍旗蒙古副都統。外授直隸宣化、廣東右翼諸鎮總兵。又內移正紅旗漢軍副都統。乾隆二十三年,授參贊大臣,駐烏里雅蘇臺。旋命將索倫兵二千人赴巴里坤。時定邊左副將軍成袞札布與參贊大臣阿桂會討舍楞,福祿請具三月糧,自科布多輸送,從之。至海拉爾,與御前侍衛敦察會師進。旋佐將軍兆惠討霍集占,偕定邊右副將軍富德帥師次呼爾璊。霍集占以五千餘人來犯,福祿偕領隊大臣永慶率索倫、察哈爾兵擊之,自巳至申,與賊戰十餘次,賊潰去。進次葉爾羌河岸,城賊突圍出,富德與福祿等領中軍自右進,追賊渡河,賊屢敗。兆惠自葉爾羌出,至阿爾吉什,偵鄂斯璊方侵和闐,疏請富德、福祿帥師策應。上命福祿偕策布登札布以兵堵霍集占竄俄羅斯路。旋命駐軍和闐,予雲騎尉世職。遷杭州將軍。準噶爾平,圖形紫光閣。上巡浙江,福祿督駐防兵肄武,制閱武詩獎之。調西安將軍。授領侍衛內大臣。以老乞休。卒

齊里克齊,蒙古鑲黃旗人。初為額魯特人,以地為氏。乾隆二十年,師征準噶爾,來降。準噶爾平,從定邊將軍兆惠擊霍集占,戰於霍爾果斯。霍集占敗走,降頭人圖魯啟拜等,授藍翎侍衛。護哈薩克使臣詣京師,遷三等侍衛。復從定邊右副將軍富德擊霍集占,至色勒庫爾,敵踞山以拒。齊里克齊偕前鋒參領喀木齊布督健銳營兵自山陰攀登仰擊,霍集占敗遁。降所部二千餘人,獲軍器、駝騾,賜布哈巴圖魯勇號。師還,命在乾清門行走,圖形紫光閣。再遷頭等侍衛,予雲騎尉世職。三十二年,從將軍明瑞徵緬甸,遇賊於底麻,敗之。賜副都統銜。召回京,再遷鑲黃旗蒙古副都統。三十七年,師征金川,命督健銳營從參贊大臣阿桂出南路。授領隊大臣,攻美諾,克之。金川平,師還,領健銳營。

嘉慶初,教匪起,送察哈爾馬如湖北軍,事竟即還。上以未請從軍,詔詰責,奪官,削世職。尋授鑲黃旗蒙古副都統。卒。

閻相師,字渭陽,陜西高臺人。入伍。累遷安西前營游擊。雅爾哈善謀誅厄魯特降人沙克都爾曼吉。天大雪,相師將五百人,偽為失道,求寄宿其壘。夜分,鳴笳驟起,殺沙克都爾曼吉,殲其部眾四千餘人。尋偕副將醜達將千人赴魯克察克同額敏和卓逐回酋莽阿里克。錄功,遷金塔寺營副將。屯田吐魯番。擢甘肅肅州鎮總兵,賜花翎。從雅爾哈善討霍集占,授領隊大臣。圍庫車,力戰被創。師克阿克蘇,以相師駐守。已,復隨剿霍集占於葉爾羌。授安西提督,駐喀什噶爾。未幾,改甘肅提督,移駐庫車。上命屯田烏魯木齊。凱旋,入覲,賚銀幣,圖形紫光閣。引疾罷,予食全俸。旋卒,贈太子太保,謚桓肅。

相師軀幹修偉,有至性。既貴,念親不逮養,每食泣下。得俸與兄弟,不問出入。所居鎮夷堡地萬畝,為濬渠灌溉,數百家利賴之。

伊柱,薩克達氏,滿洲正白旗人。父塔勒馬善,雍正間,以副都統將歸化城兵從征噶爾丹策凌。將軍達爾濟駐伯格爾,世宗命塔勒馬善參贊軍務。署前鋒統領,逐賊至額得爾河源,駐軍烏里雅蘇臺。乾隆初,權定邊左副將軍,召還。師復徵準噶爾,命赴額爾齊斯屯田。二十一年,授北路參贊大臣。復召還,授護軍統領。卒。

伊柱,自佐領再遷索倫總管。偕副都統濟福、侍衛德爾森保赴喀爾喀車臣部捕盜,得逋賊。二十四年,從將軍兆惠討霍集占。霍集占之棄葉爾羌走也,副將軍富德等逐之,至阿爾楚爾。賊設伏兩山間,師分三隊奮擊,伊柱領右翼,戰自辰至午,賊大潰。翌日,至巴達克山界伊西洱庫爾淖爾,賊據險守。師分道進攻,樹白纛,降賊萬餘。伊柱偕巴圖濟爾噶勒等堵山後策應。富德遣侍衛賽音圖等諭巴達克山汗,使擒霍集占以獻。伊柱駐兵卡倫為聲援。瓦罕伯克率所部降。尋,巴達克山汗素勒坦沙函獻霍集占首。回部平。伊柱將千人駐喀什噶爾,護諸降人屯田伊犁。師還,上御豐澤園宴勞,賜伊柱緞十二、白金五百。伊柱復出領屯田,為置臺守望,疏渠灌溉,農隙督佃伐木作屋以居,上諭令加意開拓。遷鑲藍旗蒙古副都統。從將軍明瑞徵緬甸,擊賊老官屯。卒於軍,進三等輕車都尉世職。

努三,瓜爾佳氏,吉林滿洲正黃旗人。自前鋒再遷頭等侍衛、御前行走。乾隆十一年,四川總督慶復剿下瞻對頭人班滾,命努三如慶復軍。慶復疏報班滾焚死,罷兵。張廣泗代慶復,言班滾現在。慶復坐得罪,努三罷御前行走。尋授鑲白旗蒙古副都統、正藍旗護軍統領。十八年,師征準噶爾,命從湖廣總督永常籌軍事。旋帥師駐鄂爾坤。準噶爾宰桑瑪木特闌入卡倫。授參贊大臣,命會將軍成袞札布逐捕。努三與參贊大臣薩賴爾、護軍統領烏勒登合軍,軍不戢,雜取牲畜。努三獲逃人特赫拜哈都,未聞上。烏勒登收烏梁海,縱逃人巴朗。上詰責努三、烏勒登,下定北將軍班第等按治。努三、烏勒登自陳收牲畜匿以自私事始薩賴爾,上以薩賴爾新降,不知法度,責努三等不得以此諉過。尋讞上,坐失巴朗,罪當斬。詔錄其前勞,恕死,留軍,仍籍其家。

旋授藍翎侍衛。再遷頭等侍衛,命與左都御史何國宗赴伊犁,測天度,繪地圖。送兵詣巴里坤,請回京。左授藍翎侍衛,留巴里坤差遣。招撫巴爾達穆特各鄂拓克有勞,三遷鑲藍旗護軍統領,督巴里坤屯田。兆惠被圍黑水,努三從定邊左副將軍富德往援,至呼爾璊,分兩翼擊賊,與兆惠軍會,賜騎都尉世職。師還,賜銀幣。累遷領侍衛內大臣、正藍旗滿洲都統。卒,謚恪靖。

烏勒登,烏禮蘇氏,滿洲正白旗人。自前鋒累遷鑲黃旗蒙古副都統、護軍統領。乾隆十三年,從征金川。經略大學士傅恆至軍,令駐軍馬奈。十八年,師征準噶爾,授參贊大臣,駐烏里雅蘇臺。扎哈沁宰桑瑪木特等闌入卡倫,烏勒登偕喀爾喀副都統策登扎卜將五百人,與參贊大臣努三分道捕治。參贊大臣薩賴爾收烏梁海,烏勒登自索郭克策應,俘獲甚眾。尋坐縱逃人巴朗,並與努三匿所獲烏梁海牲畜,罪當斬,貸死從軍。尋授頭等侍衛,命選厄魯特宰桑厄勒錐音等兵赴伊犁討賊。加副都統銜,授領隊大臣,進剿阿巴噶斯、哈丹等游牧。

阿睦爾撒納竄哈薩克,定西將軍策楞遣烏勒登將千人從參贊大臣玉保逐捕,玉保中道引還。烏勒登師至庫隴癸嶺,阿睦爾撒納脫走。逮詣京師,廷鞫,言:「初聞阿睦爾撒納遁,請發兵速追之。策楞、玉保俱不允。後從玉保往,復請追擊。玉保止發兵五十,至庫隴癸嶺,僅餘二十人,駝復乏。阿睦爾撒納於師行日已過嶺竄哈薩克。」上以其言實,貸死,授三等侍衛,在乾清門行走。尋仍遣赴軍。定邊將軍兆惠招降布勒特部頭目圖魯啟拜,令烏勒登自珠木罕至圖固斯塔老宣詔,護降人入覲。擢頭等侍衛,授參贊大臣。令捕瑪哈沁,並截霍集占逃路。尋以捕瑪哈沁不力,令在領隊大臣上行走。師還,累遷鑲黃旗蒙古都統、左翼前鋒統領。卒。

論曰:從兆惠、富德討霍集占有功諸將校,若瑚爾起、愛隆阿殲敵搴旗,見於詠歌,厥績懋焉。舒明逐叛拊降,以勞受爵。福祿、努三與呼爾璊之役,齊里克齊佐色勒庫爾之戰,相師助庫車之圍,伊柱收伊西洱之降,錄功皆居最,抑亦其次也。


\end{pinyinscope}