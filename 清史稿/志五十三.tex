\article{志五十三}

\begin{pinyinscope}
地理二十五

△外蒙古

外蒙古喀爾喀:古北狄地。唐、虞,山戎。夏,獯鬻。周,玁狁。秦、漢曰匈奴。漢初冒頓並有漠南,旋復北徙。後漢仍為北匈奴地。元魏曰蠕蠕,後入突厥。唐初入回紇。貞觀四年來朝,以其地為瀚海、燕然、金微、幽陵、龜林、盧山六都督府,又置皋蘭、高闕、雞田、榆溪、雞鹿、蹛林、寘顏等七州,皆隸燕然都護府。其後並有九姓諸部,盡得匈奴故地。五代至宋,回紇漸衰,與室韋嫗厥律諸部散居其地,羈屬於遼。金大安初,蒙古始盛。元太宗七年,建都和林,初立元昌路,後改轉運和林使司,前後五朝都焉。世祖遷都大興,於和林置都元帥府。大德十一年,立和林等處行中書省,統和林總管府。皇慶元年,改和林路為和寧路。順帝太子阿裕錫哩達賴汗依王保保於此,明兵破之,順帝孫特古斯特穆爾汗遁於土喇河。七傳至本雅失里,又為明所敗。後諸部共立託克託布哈之子號小王子。又數傳,徙幕東方,其留漠北部落曰喀爾喀。清崇德三年,遣使朝貢。康熙二十八年,厄魯特噶爾丹興兵攻破喀爾喀,七旗舉族款塞內附,安置喀倫邊內,噶爾丹遂並其地。三十五年,聖祖親征,噶爾丹竄死,朔漠平。喀爾喀諸部復還舊牧,為部三:一曰土謝圖汗,一曰車臣汗,一曰札薩克圖汗。又善巴自為一部,曰賽音諾顏。共部四,為旗八十有六。東至黑龍江呼倫貝爾城,南至瀚海,西至阿爾臺山,北至俄羅斯。廣五千里,袤三千里。北極高四十二度至五十一度三十分。京師偏東三度至偏西二十六度。人約七十萬口。

土謝圖汗部:駐土拉河。直大同邊外漠北。至京師二千八百餘里。南界瀚海,西界翁金河,北界楚庫河,東南界蘇尼特、四子部落諸部,西北界唐努烏梁海。所部佐領積三十七旗,以分設賽音諾顏部,析二十一旗隸之,後增四旗,凡二十旗。乾隆四十六年,詔世襲。北極高四十五度三十三分。京師偏西十一度二十四分。土謝圖汗本旗其汗為噶爾丹所破,來降。康熙三十年,許仍舊號世襲。佐領一。牧地在杭愛山東、喀里雅爾山南,跨鄂爾坤、喀魯哈二河。西:杭愛山,在鄂爾坤河源之北,其山最高大,山脈自西北阿爾泰山來,東趨,逾鄂爾坤、杜喇諸水,為大興安、肯特諸山。又自山西庫庫嶺北折,環繞色楞格河上流諸水發源之處。杭愛譯言「橐駝」也,山形似之。當即古之燕然山。有鄂爾坤河,自附牧賽音諾顏之額魯特旗界,東北經章鄂山東麓,又經西爾哈阿濟爾罕山西麓,又東北出山,折而西北流,有濟爾瑪臺河自南來會。喀魯哈河,源出翁金河北土喇、鄂爾坤二河間平地,西北流,轉東北,入土喇河。鄂爾坤河又東北經吉拉哈吉圖布拉克地南,有西拉索博太河,北自布龍山南支阜,合三水南流來注之,又東北經喀里雅拉山西南麓,中有大洲。又北流,有伊奔河,自西北布龍山東南支阜,合三水來注之。又東北循山,會哈拉河、衣魯河。又正北流,至布龍山東北支阜,入色楞格河。右翼左旗土謝圖汗之從子,康熙三十年授札薩克一等臺吉。傳至乾隆二十一年,其孫累以功晉和碩親王,世襲。佐領七有半。牧地跨色楞格河、土喇河之合流,南至達什爾嶺,北至罕臺山。色楞格河自賽音諾顏部東北入,有厄赫河自西北大山東南流,合翁佳河諸水來會,水勢始盛。稍東,有市呼圖河自南合三水來注之。又東北,受北來一水。又東北,有一河自西南沙昆沙拉之北,東北流,合東南一水,北來注之。又東,受西南一小水,又東逕布龍山北麓。山脈西南自巴顏濟魯克山、賽堪山綿亙而東北,為厄魯墨得依山。又東為西拉克山、布昆沙拉山,又東北為此山蜿蜒至兩河合處,為色楞格、鄂爾坤界。色楞格河自山北麓,又東北,鄂爾坤河自南合土喇諸河,東北流來會。土喇河東南來,納喀魯哈河,東北折而北流,又合鄂爾坤河,當西十度北極出地四十九度處。東有布噶勒臺河。中右旗土謝圖汗之弟,康熙二十五年授札薩克,三十年封多羅貝勒。雍正元年,晉其子郡王,世襲。佐領三。牧地當土喇河曲處。東北:達什隆山。土喇河循都蘭喀拉折而西北流,曲曲四百餘里,有喀魯哈河自西南來會。左翼中旗土謝圖汗裔,康熙三十年封多羅郡王,兼札薩克,世襲。佐領十四。牧地當阿爾泰軍臺所經。北緯四十四度二十分。西經七度五十分。東北有札爾噶山。中旗土謝圖汗裔,康熙三十年封多羅郡王。乾隆二十二年,改為札薩克固山貝子,世襲。佐領四。牧地在肯特山西南;當土喇河源。西北:哈麻爾嶺。西南:達什隆山。東北:肯特山,山高大,為漠北群山東至大海之祖。山西阜曰即龍嶺,又西曰特勒爾濟嶺。凡諸嶺以南,水皆流入克魯倫河,以北,水皆流入敖嫩河。敖嫩河源在克魯倫河源西北小肯特山;土人呼為阿即格肯特山,山南為喀爾喀地,山北為俄邊。嶺北麓水即楚庫河源,北流入色楞格河者。嶺南幹山西南麓水,即土喇河源,西南流,折而西北,會鄂爾坤河入色楞格河者。此嶺為漠北一大分水嶺也。自小肯特山東北行,為大興安山,包絡黑龍江諸水之北而東入海。一支折而南,分為二幹:一東南,為大肯特山起頂,又東南為必爾喀嶺諸山,為北黑龍、南喀魯倫諸水之界,綿亙千餘里,至會合處;一西南為圖拉源山,又南為噶拉泰嶺,折而西南為興安嶺,為東克魯倫、西土喇諸水,又西南而西北,至土喇會鄂爾坤河處。自此而西北,群山皆以阿爾泰山為祖。若論漠北大分水之處,一東至東海,一北至北海,則莫高肯特山矣。汗山,在興安嶺北、土喇河南岸,元祕史謂之不兒罕山。天山,在圖拉河之西,約出長城三千里。山不甚高,籓名汗河嶺。汗山之北為庫倫,即苦另山,山甚峻。土喇河即圖拉河,發源敖嫩河源之西南數十里許、特勒爾濟嶺之西,曰土喇色欽。色欽,蒙古語「河源」也。西南流,與北源喀拉圖魯河會。西南流,哈溪河自西北合東來喀拉鄂模水、西北來空烏魯河,東南流來會。又南,噶爾泰河自東南大山西流來會。又西南,逕啟拉薩山西。又西南,阿拉克他河自北來注之。又西,特勒爾濟河合東占河二水,東南流,會奎羅河。左翼後旗土謝圖汗裔。康熙三十二年授札薩克一等臺吉。乾隆十九年封輔國公,尋晉貝子、貝勒、郡王。五十七年降鎮國公,世襲。佐領四。牧地當阿爾泰軍臺所經。翁金河至是瀦於胡爾哈鄂倫諾爾。諾爾直漠南河套八百里許,舊作呼拉喀五郎鄂模,周二十餘里。諾爾東北有哈喇哈達山、徹徹山、上凱山,皆沙海中孤嶼也。翁金水,源西十三度三分,極四十六度九分。諾爾西九度四分,極四十五度二分。自西北而東南,行大漠中,近千里也。中右末旗土謝圖汗裔。康熙五十八年授札薩克一等臺吉。乾隆二十四年封輔國公,世襲。佐領一。牧地跨土喇河。西北:達什隆山。土喇河自中旗汗山北麓,會色勒弼河,又西至色勒弼嶺南,曲曲西南,至杜蘭喀喇山之北,山南即大漠。西十度,極四十七度五分。南經寧夏九度,經套北陰山六度。河隨山折,西北流入中右旗境,南岸即度蘭支阜,綿亙北岸,即色勒弼嶺支阜,又北行為查木勒山。左翼左中末旗土謝圖汗裔。康熙五十年封札薩克輔國公,世襲。佐領一。牧地當喀魯哈河源。喀魯哈河流出平地,在翁金河之北二百里,鄂爾坤河北折之東四百里。西十二度,極四十六度七分。有二泉,西北流而合,又西北,有一水西南自科洛爾昆山東北流來會。山在額爾德尼昭之東南。又北流,逕昆庫勒山,西折,東北經科克內山西。又北流,折而東北,曲曲數百里,與土喇河會。水口東即查木勒山西麓也。水源流長七百餘里。右翼右旗土謝圖汗裔。康熙三十年授札薩克一等臺吉。乾隆二十年封輔國公,世襲。佐領一。牧地東至錫伯格圖,南至諾昆陀羅海,西至烏遜珠爾東山,北至齊克達噶圖嶺。左翼前旗土謝圖汗裔。康熙三十年授札薩克一等臺吉。乾隆三年封輔國公,世襲。佐領三。牧地跨喀魯哈河。西北:烏噶勒札山。右翼右末旗土謝圖汗裔。雍正九年,以功授札薩克一等臺吉。十年封輔國公,世襲。佐領一。牧地當哈拉河源。東:恰克圖山。南:烏里雅呼嶺。北:諾不圖布拉克山。東南:達喇勒濟山。西南:哈瑪爾嶺。哈拉河源出土喇河北與汗山相對之色勒弼嶺。北有那林河、布勒哈太二河,阿達海河、松納拉河均來會。又北,通勒河。東北至阿即格肯特山西麓,合三源,西南流,又折西北,有一河自東北合數水來注之。又西逕陀羅什山北、哈達圖爾山南,納博羅河、查克都勒河,西北折,逕都拉遜那拉酥查克丹地之東,大松林也。又北逕喀里雅喇山東麓,又北入鄂爾坤河。源委六百餘里。中左旗土謝圖汗裔。初授一等臺吉。乾隆三年,晉輔國公、貝子品級。二十三年授札薩克。後遂以功品級一等臺吉,世襲。佐領一。牧地東至察奇爾哈喇,南至善達勒,西至阿爾噶棱,北至阿魯哈朗。左翼右末旗土謝圖汗裔。康熙三十六年,授札薩克一等臺吉,世襲。佐領五。牧地當阿爾泰軍臺之東。達庫倫之驛於是分道。左翼末旗土謝圖汗裔。康熙三十年授札薩克一等臺吉,世襲。佐領一。牧地當阿爾泰軍臺之東。左翼中左旗土謝圖汗裔。雍正十年授札薩克一等臺吉,世襲罔替。佐領一。牧地當阿爾泰軍臺之西。中次旗土謝圖汗裔。康熙五十八年授札薩克一等臺吉,世襲。佐領一。牧地當左翼中旗之東。右翼右末次旗土謝圖汗裔。康熙三十五年授札薩克一等臺吉,世襲。佐領一有半。牧地跨鄂爾坤河、色楞格河。東:薩爾金河。西:塔裏雅那臺河。北:札勒圖爾河。東北:桑喀勒圖河。東南:札克圖勒河。右翼左後旗土謝圖汗裔。雍正八年授札薩克一等臺吉,世襲。佐領一。牧地當土喇河、喀魯哈河之合流。南:達什隆山。西:珠格楞嶺。中左翼末旗土謝圖汗裔。康熙三十三年授車棱札布一等臺吉,兼札薩克,世襲。佐領四。牧地當鄂爾坤河、色楞格河之合流。鄂爾坤河自東南向西流入色楞格河。色楞格河自西南來,環繞山北,東北流,過俄羅斯之楚庫柏興,又北流入柏海兒湖。東:烏雅勒噶河。西:薩爾金河。北:察罕烏蘇河。東北:博拉河。右翼左末旗土謝圖汗裔。康熙三十年封札薩克輔國公,後降一等臺吉兼札薩克,世襲。佐領一。牧地當哈拉河、伊遜河東南哈臺山北二百里。有哈拉河南流,受南來揆河,折西北,逕右翼右末旗東北。左得博羅河、查克杜兒河,又北注鄂爾坤河。土喇河北岸諸山,有色爾畢谷口三處,及松吉納山嶺三處,皆自各山發源,流入土喇、鄂爾坤。又東北,衣魯河,自東南合三水來注之。又正北流至布龍山支阜,與色楞格河會。東北:敏吉河。西北:札克都勒河。以上統盟於汗阿林。滿語「山」。在庫倫南。

賽音諾顏部:直甘肅涼州邊外西套之北。至京師三千餘里。格埒森札之孫圖蒙肯護持黃教,唐古特達賴喇嘛賢之,授賽音諾顏號。康熙中,其孫善巴來歸,旋以善巴從弟策凌從征有功,始自為一部。乾隆中,以善巴曾孫諾爾布札布襲賽音諾顏號,世襲與三汗同。所部東界博羅布爾哈蘇多歡,南界齊齊爾里克,西界庫勒薩雅孛郭圖額金嶺,北界齊老圖河。轄旗二十二。北極高四十五度四十四分。京師偏西十二度五十分。賽音諾顏本旗初,信順額爾克岱青諾顏善巴率屬來歸。康熙三十五年封和碩親王。乾隆三十一年,許仍其賽音諾顏舊號,世襲。佐領四有半。牧地當鄂爾坤河源,在北緯四十七度、西經十四度五十分處。西北:庫爾布拉克灰圖山。鄂爾坤河出旗境,二水合東流,北納一水,入土謝圖汗部。西:塔楚河,源出都蘭喀喇山東南大幹南麓,二水南流而合,會東北來三水,折流逕塔奇驛,西南至阿勒察圖山。中左末旗善巴再從弟策凌,康熙六十年授札薩克。雍正元年封多羅郡王。九年,晉和碩親王,世襲。佐領四。牧地當塔米爾、哈綏、齊老圖三河源。北:伊克沙巴爾山。東北:綽嚨山。西北:弇克嶺。西南:庫克嶺。塔米爾河亦曰他米勒,有南北兩源。南源出杭愛山北麓,在鄂爾坤河之西者曰阿索郭特河,西北流,合三澗而東北流,有西北來二水皆會,又東北,始曰塔米爾河,又北而會阿索郭特河,皆杭愛以北水也。又東北,會東南來一水,其東即蘇巴勒乾山。又東北受朝木多河、齊齊爾里克河,並會諸小水,東北與北源合。北源出枯庫嶺東麓,在杭愛山西北,有二澗,東北流而合,又東北合三澗水,並納諸小水,始曰塔米爾河,北岸連山,即哈瑞河諸源也。又東流,受四水,瀦為臺魯勒倭黑池,廣數十里,中有一山。又東流,有察罕烏倫河,自西北來會,其南岸即布拉幹北山也。又東北百數十里,而南源自西南來會,又東折北,會鄂爾坤河。此水兩源,俱五百餘里始合入鄂爾坤。自杭愛山以北、枯庫嶺以東,諸泉皆會焉。喀綏河亦曰哈瑞河,即古和林河,出杭愛西南榦山,在齊老圖源之南,流數百里,合北來伊遜都蘭喀喇地山南二水,又東北,有一河合二水自南來會,始曰喀綏河。又東北,有硃薩蘭河自西合二水東流來會,又東北,會瑚伊努河,入色楞格河。河源流都長九百里。齊老圖河即石河,源出杭愛西界山下之額爾哲伊圖察罕泊,泊周六十里,在鄂勒白稽山之南榦大山下,西北經隔山之桑錦達賚泊。自泊東北流出,逕烏爾圖烏雅山南麓,稍東,會西北來一水,又東,會西南來二水,始曰齊老圖河。右翼右後旗賽音諾顏之裔。康熙三十年授札薩克鎮國公。雍正二年封固山貝子。乾隆二十一年,晉多羅貝勒。尋以功晉郡王,世襲。佐領二。牧地當拜塔裏克河源。北:札克額沁山。拜塔裏克舊作貝德勒克,源出枯庫嶺南麓,其北麓隔山即塔米爾河源也。三水南流,合而西南,有查克河自北山合五水南流三百餘里來會,逕庫倫伯勒齊爾之地。又南有察罕帖睦爾河,東北自索阿都依嶺合二水西南流來會。又南出兩山間,西南流平地中百數十里,西彌河自南合一水北流來會,又西南瀦為察罕泊。源流八百餘里。中右旗親王策凌次子。雍正十年封輔國公。乾隆二十年封多羅貝勒。二十一年,晉郡王,世襲。佐領一。牧地當推河源。北:庫克嶺。推河亦曰頹河,舊作拖衣河,源出杭愛山尾南麓,西南流,會三小水,又西南,有烏可克河,西北自烏可克嶺合三水東南流來會。嶺在杭愛山西南,嶺南水入推河,嶺北水為塔米爾河南源。推河又南,有雅馬圖河自東北合三水西流來會,即鄂爾吉圖都蘭喀喇山西水也。又南受庫塞楞圖河。稍南,有一水自東合二澗來會,又南逕兩山間,額勒屯圖河自東合三水來會,皆都蘭喀喇山西南麓水也。又南出山,曲曲流平地中百八十里,逕博濟和碩驛東,又南折西流,瀦為鄂洛克泊,形東西長四十里。西十五度五分,極四十五度六分。源流五百餘里。此水東三百里為塔楚河。中前旗賽音諾顏之裔。康熙三十年授札薩克鎮國公。雍正元年,晉固山貝子。乾隆二十年,晉貝勒,世襲。佐領一。牧地跨濟爾瑪臺河、鄂爾坤河、翁金河。濟爾瑪臺河出右翼中右旗,東流,逕額魯特旗入土謝圖汗部界。鄂爾坤河自與姑洛河會,東南流兩山間,折而東北,入額魯特旗境。北岸山即杭愛東南支阜,南岸即西自都蘭喀喇綿亙而東之杭亦哈馬勒山。隔山而南即翁金河也。翁金河出右翼左末旗,二水合東流,逕右翼中左旗、中前旗,北合二水,亦入土謝圖汗部界。中左旗賽音諾顏之裔。康熙二十五年授札薩克。三十年,封多羅郡王,後降貝勒,世襲。佐領三。牧地有特爾克河、伊第爾河,合於齊老圖河,為色楞格河。伊第爾舊作厄得勒,亦作依得爾。色楞格河南源有四,稍北者曰厄得勒河,源出喀爾喀西界鄂勒伯稽山,共合七水,行四百餘里,而齊老圖河合諸源水自西南來會。又東北,受南來一水,疑即特爾克河也。又東北,循山麓流百餘里,而烏里雅蘇臺河自西南來會。又東北三十里,而阿濟勒克河自南來會,始曰色楞格河。中末旗賽音諾顏之裔。康熙三十一年授一等臺吉兼札薩克。雍正二年封輔國公。乾隆二年,晉鎮國公,世襲。佐領一。牧地哈綏河至是合於色楞格河。右翼中左旗賽音諾顏之裔。康熙四十六年授札薩克一等臺吉,後晉輔國公,世襲。佐領四。牧地當翁金河源。南:阿哈爾山。翁金河亦作翁吉,又作甕金,兩源出鄂爾吉圖都蘭喀喇山東行大榦山中。其西隔山即塔楚河源也。其北隔山即鄂爾坤河,東南流出平地合焉。又東南,會西南來一水,又東,會北來一水,又東南,逕杭亦哈馬勒山前,受二水。又東南,曲曲流八百餘里,於大漠瀦為呼拉喀烏浪諾爾,周二十餘里。右翼末旗賽音諾顏之裔。康熙三十年授札薩克一等臺吉。雍正十年封輔國公,世襲。佐領二。牧地墨特河至是合於拜塔裏克河。北:札木圖嶺。東北:庫首庫爾嶺。墨特河疑即察罕帖睦爾河也,東北自索阿都依嶺合二水西南流來會。南有繃察罕諾爾,廣二十餘里。其北三十里有濟爾哈朗圖池,廣十里許。又東北有伊洛河,北自山麓克庫池南流,逕哈拉圖科山西麓,又南數十里涸。哈拉圖科山南有鄂洛克池,山東百里即推河也。右翼前旗賽音諾顏之裔。康熙三十年授札薩克一等臺吉。三十五年封輔國公,世襲。佐領一。牧地胡努伊河至是合於哈綏河。胡努伊舊作呼納衣,又作庫諾衣,源自西南山中,東北四百里,逕賽坎山北麓,又東北入哈綏河。賽坎山甚高大,即巴顏濟魯克山之北行正榦,又折而東北,為厄勒黑圖諸山。中後旗賽音諾顏之裔。康熙五十一年授札薩克一等臺吉。乾隆元年封輔國公,世襲。佐領一。牧地有布爾噶蘇臺河合於札布噶河。布爾噶蘇臺河出旗北馬喇噶山,山脈自阿爾泰頂南行,分一榦東行,為烏蘇郭瑪山。又東連峰相接,東南數百里,為伯勒奇那克科克伊山。又東為昂奇山。又東北行為馬喇噶山。此水源即馬喇噶山東北將折東南之南麓也。出山南流,會東來二水、西北來一水。又南有烏海河,西北自昂奇山兩源合東南流來會。又南與西喇河會。二源既合,西南流,逕巴顏山北麓,曰札布噶河。又有烏里雅蘇臺河,出旗境,西流八百餘里,納蘇布拉河來會。北有布音圖河源。左翼左旗賽音諾顏之裔。乾隆三十一年封札薩克輔國公,世襲。佐領二。牧地當札布噶河源。札布噶舊作查巴哈,又作札布堪,源有二,最東者曰西喇河,出庫倫伯勒齊爾西北大山,凡四水,南流並為二支,又西南百餘里合焉。又西南受北來一水,又南受東來之西喇河,又西受北來一水。又西南,布爾噶蘇臺河自北來會,即西源也,出北馬喇噶山南麓,南流會二水,又南有烏海河,兩源合東南流來會,又南流與西喇河會。二源既合,逕巴顏山北麓,曰札布噶河。又南入札薩克圖汗旗南界。左翼中旗賽音諾顏之裔。初授一等臺吉。乾隆二十二年,晉貝子品級,授札薩克。後降襲公品級,世襲。佐領一。牧地跨哈綏河。左翼右旗賽音諾顏之裔。康熙三十年授札薩克一等臺吉,世襲。佐領三。牧地在哈魯特山。左翼左末旗賽音諾顏之裔。康熙三十五年授札薩克一等臺吉,世襲。佐領一。牧地跨塔米爾河、胡努伊河。右翼中末旗賽音諾顏之裔。康熙五十一年授札薩克一等臺吉,世襲。佐領一。牧地拜塔裏克河東支至是瀦於察罕諾爾,其西支在青素珠克圖諾們罕游牧諾爾,當西十度、北極出地四十五度七分,庫倫伯勒齊爾地南界,形如瓜,周百里,東西長,諾爾東有呼里圖克白爾池,廣十餘里。又東為西彌河源。又東為一小河,又東為繃察罕諾爾。右翼左末旗賽音諾顏之裔。康熙三十六年授札薩克一等臺吉,世襲。佐領一。牧地跨翁金河。東有圖魯根山。右末旗賽音諾顏之裔。乾隆三年授一等臺吉。四年授札薩克,世襲。佐領一。牧地當伊第爾河源。南:雪山。西北:索郭圖嶺。伊第爾河出鄂勒白稽山,即杭愛山頂之西南大榦也。隔山西即桑錦達賚泊,西十六度九分,北極出地四十九度。兩水自山麓東流而合,又東,會七水,名伊第爾河。又東北會齊老圖河,以入於色楞格河。右翼中右旗賽音諾顏之裔。康熙三十五年授札薩克一等臺吉,世襲。佐領無。牧地當濟爾瑪臺河源。濟爾瑪臺舊作硃勒馬臺,亦作硃爾馬臺,源出額黑鐵木兒山南麓,東南流,繞布庫鐵木兒山足三面,東北流,曲曲二百餘里,瀦為池,曰察罕鄂模,廣數十里。又東北流,有布勒哈爾臺河,南自達爾湖喀喇巴冷孫地之池水東北流來會。又東北入鄂爾坤河。右翼後旗賽音諾顏之裔。康熙三十一年授一等臺吉兼札薩克,世襲。佐領一。牧地當哈綏河北岸、色楞格河南岸。中後末旗賽音諾顏之裔。康熙四十八年授札薩克一等臺吉,世襲。佐領一。牧地跨齊老圖河。中右翼末旗賽音諾顏之裔。康熙三十五年授札薩克一等臺吉,世襲。佐領無。牧地當塔米爾河南岸。東北:烏爾圖特莫爾河。附額魯特部本旗準噶爾之裔。康熙三十六年來降。四十四年封札薩克輔國公。雍正元年,晉固山貝子,世襲。佐領一。牧地跨濟爾瑪臺河、鄂爾坤河。西:察汗山。東南:博勒克山。鄂爾坤河自中前旗境折而東北,逕西爾哈阿濟爾罕山西麓之額爾德尼昭,即大喇嘛寺也。河逕其西及章鄂山之東麓。山亦高大,即杭愛之東支阜,唐時回鶻牙帳西之烏德鞬山也。又東北出山,折而西北流三百餘里,濟爾瑪臺河自西南來會。額魯特前旗噶爾丹同祖弟丹津之孫,號丹津阿喇布坦,康熙四十一年來降,封多羅郡王。四十二年授札薩克。乾隆十三年,降固山貝子,世襲。佐領一。牧地當塔米爾河北岸,隸賽音諾顏部。東南有溫奎諾爾。以上統盟於齊齊爾里克。

車臣汗部:駐克魯倫翁都爾多博,直古北口邊外漠北。至京師三千五百里。格埒森札之孫謨羅貝瑪號車臣汗。東界額爾德尼陀羅海,南界塔爾滾柴達木,西界察罕齊老圖,北界溫都爾罕。轄旗二十三。北極高四十五度三十四分。京師偏西五度三十四分。車臣汗本旗故車臣汗阿喇布坦之子,康熙二十七年,率眾十餘萬戶來降,仍其故號。雍正六年,賜印文曰格根車臣汗,世襲。佐領二。牧地跨喀魯倫河。東:烏蘭溫都爾山。南:阿爾圖山。西:塔奇勒噶圖山。北:哈喇莽鼐山。東北:色勒格圖山。東南:鄂爾楚克山。西南:庫特肯額里雅山。喀魯倫河自右翼中前旗境拖諾山南麓,稍折東北流數十里,又東北逕克勒和碩山北麓,入左翼右旗境。左翼中旗烏默客之叔,康熙二十八年授札薩克。三十年封多羅郡王。乾隆二十年,晉和碩親王,世襲。佐領二。牧地在科勒蘇河之東,跨喀魯倫河。東:卜固尼和碩山。西有特克瑪爾圖山。西北:圖木斯泰山。科勒蘇河出西南大山,兩源,東北合二水,北入敖嫩河。東北:喀魯倫河,入旗南界,有固爾班博爾龍山,三峰並峙,在南岸沙中,至庫魯諾爾南,入中左旗境。中右旗烏默客之叔,康熙二十八年授札薩克。三十年封固山貝子。三十五年,晉多羅郡王,世襲。佐領四。牧地喀爾喀河至是瀦於貝爾諾爾。喀爾喀河在齊齊哈爾城西,源出摩克託里山,西北流入於貝爾諾爾。又北流出,曰鄂爾順河,入呼倫諾爾。貝爾諾爾舊作布伊爾湖,亦作布育里鄂模,元之捕魚兒海子也。明藍玉破脫古思帖木兒處。東北有沙喇勒濟河。右翼中旗烏默客之族叔,康熙二十八年授札薩克。三十年封多羅貝勒,世襲。佐領八。牧地在喀魯倫河之南烏純地。西:伊克噶札爾阿齊圖山。中末旗烏默客之族,康熙三十年授札薩克固山貝子,世襲。佐領三。牧地在喀魯倫河之南博羅布達。北:庫特肯額里雅山。東北:伊克阿爾圖山。西北:額爾克納克山。東南:鄂斯奇山。中左旗烏默客之族,康熙二十八年授札薩克。三十年封固山貝子,世襲。佐領二有半。牧地在喀魯倫河之布色鄂埒客。東:和爾蓋山。北:伯爾克山。中後旗烏默客之族,康熙二十八年授札薩克。三十年封固山貝子,後降輔國公,世襲。佐領一有半。牧地跨敖嫩河。南:色勒格圖山。北:達喇特河。東北:莽阿泰河。敖嫩河自大肯特山北麓會北來一水,又東有一河,西北合二水,東南流來會。稍東南,啟查魯河西南自大肯特山折向東南支阜,東北流來會,折東北流,又折東,巴拉喀河合二水自西南畢爾喀嶺東北流來會。又東南流,呼瑪拉堪河自南大山合兩源北流來會。又東北流,有一河合兩源西北自大興安山東南流來會。大興安山,土人曰阿母巴興安,甚高大,自此綿亙而東,直抵黑龍江入海處。山之南為喀爾喀界,山之北為俄界。又南,北合科勒蘇河。左翼前旗烏默客之族,康熙二十八年授札薩克。三十年封鎮國公,世襲。佐領一有半。牧地當索嶽爾濟山北,濱喀爾喀河。索嶽爾濟山袤延數百里,其西麓臨大漠,東北與齊齊哈爾城相近。喀爾喀河有數源,最東者出阿魯特拉奇嶺西麓,有池廣數十里,西南流,南源合三水來會。又西南流,有一河自北合三源來會。又西分為二支,一南流,有阿母巴哈爾渾河合三水自南來會。又西,合北支西流,伊蘭塞罕河自北大山西南流來會。又一河自西北合三源南流注之。又西南,受哈爾渾河。又西,噶爾查布魯克圖河自東南合噶爾圖思臺及噶爾巴哈尼二河北流注之。又西會和爾和河,折西北,逕喀勒河朔之北,其北岸有小山,受東北來之呼魯思太河,折而西流,曰喀爾喀河。西南流,分支渠,匯為貝爾諾爾。右翼中右旗烏默客之族,康熙五十年授一等臺吉。五十一年授札薩克。雍正二年封輔國公,世襲。佐領一有半。牧地在達爾漢徹根。東:依札噶爾山。南:巴噶額里彥山。西:鄂羅克依山。西北:依爾蓋山。左翼後旗烏默客之族祖,康熙三十年授札薩克一等臺吉,世襲。佐領二有半。牧地在察漢布爾噶蘇臺。東有鄂爾布勒山。西有布哈山。北:烏蘭溫都山。西南:布勒格圖山。左翼後末旗烏默客族,康熙五十年授札薩克一等臺吉,世襲。佐領一有半。牧地在烏爾圖。西:鄂爾布勒山。右翼後旗烏默客族,康熙三十年授札薩克一等臺吉,世襲。佐領三。牧地在巴顏濟魯克。西:阿克索那山。南:烏尼格特山。中末右旗烏默客族,雍正十三年授一等臺吉。乾隆十四年授札薩克,世襲。佐領一。牧地東至特克什烏蘇,南至多木達哲爾克特山,西至鄂爾和山,北至庫登圖山。東北:託克臺山。西北:阿爾圖山。東南:布哈山。西南:烏斯奇山。右翼中左旗烏默客族,康熙五十二年授札薩克一等臺吉,世襲。佐領一。牧地在騰格里克。東南:庫里彥山。北:僧庫爾河。右翼前旗烏默客族,康熙三十年授札薩克一等臺吉,世襲。佐領一有半。牧地在喀喇莽鼐。西北:色布素勒山。東:薩喇克河。右翼左旗烏默客之叔,康熙四十年授札薩克一等臺吉,世襲。佐領半。牧地在額爾得墨。東:鄂博克圖山。北:得勒山。西南:鄂爾楚克山。中末次旗烏默客族,康熙三十四年授札薩克一等臺吉,世襲。佐領一有半。牧地在白爾格庫爾濟圖。東:哈爾噶朗圖山。南:圖木斯圖山。西北:得勒山。左翼右旗烏默客之叔,康熙四十年授札薩克一等臺吉,世襲。佐領一。牧地跨喀魯倫河。東:特格裏木圖山。西:哈噶勒噶山。北:瑪勒胡爾山。東北:圖木斯圖山。西南:託克特依山。喀魯倫河自喀勒和朔北麓,又東北會塔爾河,舊名他拉即兒即河,自畢爾喀嶺西南麓,合二源東南流沙土中,隱見不常。又東北數十里,逕厄窩得哈爾哈小山西北麓,即北岸厄莫勒山之西南麓也。折東流,至東南麓,兩岸沙漠,又東北入左翼中旗境。中右後旗烏默客族,康熙三十六年授札薩克一等臺吉,世襲。佐領半。牧地在肯特山東,當喀魯倫、敖嫩二河源。東:得勒格爾罕山。南:巴顏烏蘭山。西北:罕臺山。西:塔尼特河。東北:塔喇塔河。有喀魯倫河,即臚朐河,北史之怯綠憐河也。源出肯特山東南支峰西南麓。兩源西流而合,又西,有一河,東北亦自肯特山南麓西南來注之。又西南流,逕肯特山頂之南,受北來衣魯河。又西南,受西北即龍河。又西南,至布塞山東南麓,受撒內河,東自畢爾喀嶺西麓西流合東南一水來會。又東南,有一河,北自忒勒兒吉嶺東南流來會。又西南,白勒肯河自土喇色欽東麓東南流來會。又西南,至噶拉太嶺之東,循兩山間,折而東南流,逕巴顏烏蘭山西麓,入右翼中前旗境。又東經車臣汗旗、左翼右旗、左翼中旗、中左旗、左翼左旗、中左前旗、中前旗境,凡二千數百里,東北入枯倫湖。敖嫩河乃黑龍江上源,亦名俄儂河,元之斡難河也,自肯特山西忒勒爾吉嶺西北小肯特山東麓,折東南流,納東北一水,經忒勒爾吉嶺北麓,有一水自嶺西北東流來會,亦敖嫩一源也,又東入中後旗境。左翼左旗烏默客叔,康熙三十五年授一等臺吉。四十年授札薩克,世襲罔替。佐領一有半。牧地跨喀魯倫河。南:巴彥罕山。西:鄂喇霍圖山。喀魯倫河自庫魯鄂模南稍東,逕西拉得克西博格山之陰,又東百里,中有沙洲曰術爾呼術,東北流,入旗境必拉城南。隔河而南,有乾諸可客蒲山,綿亙東北百里許,即塔本陀羅海也。又東逕杜勒鄂模南,入中左前旗境。中左前旗烏默客裔,康熙三十六年,授貢楚克一等臺吉兼札薩克,世襲。佐領一。牧地跨喀魯倫河。喀魯倫河自杜勒鄂模南入旗境。又東,河心有沙洲,南岸為塔本陀羅海之北麓。折東南流,又東入中前旗境。中前旗烏默客裔,康熙二十八年授濟農及札薩克。三十年封固山貝子。乾隆二十二年,降一等臺吉兼札薩克,世襲。佐領五。牧地跨喀魯倫河。東:札爾噶山。北:鄂克託木山。喀魯倫河自塔本陀羅海北麓,折東南流,又東逕南岸小山北麓,折東北至南岸大山東北麓,東南流,折向正北,又東北流,中有沙洲,其東南岸外,則杜勒鄂模也。又東北,曲曲注阿勒坦厄莫爾山東北,瀦為枯倫湖,在黑龍江齊齊哈爾城西千三百餘里也。湖自西南而東北,長徑二百餘里,東西闊百餘里,周可五六百里。枯倫今作呼爾,即古之具倫泊也。右翼中前旗烏默客裔,初授二等臺吉。乾隆十九年晉一等臺吉。二十年,封輔國公兼札薩克,後降一等臺吉,世襲。佐領一。牧地當喀魯倫河曲處。東:庫里葉山。北:巴顏烏蘭山,綿亙東南二百里許。喀魯倫河自噶拉太嶺之東,西南至兩山間,循山麓東南流,逕巴顏烏蘭山西麓,至南岸山盡處,稍折東流,有僧庫爾河南流沙中來注之。喀魯倫河又東南,自沙地經拖諾山南麓,入車臣汗旗境。以上統盟於巴爾和屯。即巴拉斯城。

札薩克圖汗部:駐杭愛山陽,直甘肅、寧夏邊外漠北。至京師四千餘里。東界翁錦錫爾哈勒珠特,西界喀喇烏蘇額埒克諾爾,南界阿爾察喀喇託輝,北界特斯河,接唐努烏梁海。本元裔,號札薩克圖汗。康熙二十七年,沙喇兵敗,為噶爾丹所戕。其弟策旺札布率族來歸,封和碩親王,詔仍襲汗號。轄旗十九。北極高四十三度三十五分。京師偏西十九度九分。札薩克圖汗兼管右翼左旗策旺札布,以從征退縮削爵。雍正四年,詔其族格埒克延丕勒襲汗號,兼郡王爵,領右翼左旗札薩克事,世襲。佐領三。牧地有博格爾諾爾。東南:札布噶河,自賽音諾顏部左翼左旗界西南流,逕巴顏山北麓尼魯班禪喇嘛游牧,折西流,席喇烏蘇河南自阿爾洪山水所匯之大泊來會。又西北流,烏里雅蘇臺河東來入之。博格爾諾爾,舊作白格爾察罕鄂模,在庫克西勒克山之南、都忒嶺之東。又有都魯泊。中左翼左旗札薩克圖汗之族。康熙三十五年封多羅貝勒兼札薩克。乾隆二十二年,以功晉郡王品級。四十六年,詔以貝勒世襲。佐領二。牧地當特斯河源。東:庫蘭阿濟爾噶山。北:伯爾克山。東北:巴彥集魯克山。特斯河源出阿爾泰東北大幹之唐努山西南麓,西南流山中,受南北來四水,又西南入烏梁海境。曲曲西瀦為烏布薩泊。泊在阿爾泰頂之東南麓六十里。左翼中旗札薩克圖汗之族。雍正五年授札薩克二等臺吉。乾隆二十一年晉一等臺吉。二十三年封輔國公,復晉鎮國公,世襲。佐領一。右翼後旗札薩克圖汗之族。康熙三十年授札薩克一等臺吉,世襲。佐領一。與左翼中旗同游牧。牧地當札布噶河西岸。左翼右旗札薩克圖汗之族。康熙二十九年授札薩克。三十年封多羅貝勒。雍正十二年降鎮國公,世襲。佐領一。牧地在都爾根諾爾之南。諾爾在科布多城西、伊克阿拉克泊之西南,北與喀喇諾爾相聯,形如葫蘆,亦札布噶河之支流所匯也。左翼前旗札薩克圖汗之族。康熙二十八年授札薩克。三十年授一等臺吉。五十年封輔國公,世襲。佐領二。左翼後末旗札薩克圖汗之族。雍正四年授札薩克一等臺吉,世襲。佐領一。與左翼前旗同游牧。牧地在奇勒稽思諾爾之東,一作柯爾奇思諾爾,在阿爾泰頂東南,去兩旗札薩克駐處八百里。東南:札布噶河、空歸河。西南:伊克阿拉克池水所匯也,周百數十里,西南相聯一泊曰愛拉克諾爾,南與喀喇諾爾相直。右翼右末旗札薩克圖汗之族。雍正二年授札薩克一等臺吉,世襲。佐領二。牧地當德勒格爾河西岸、桑錦達賚之東。德勒格爾河一作哈喇臺爾河,源出唐努山南、錫巴里喀倫北,東北流,當阿哈里喀倫之北,有一小水西北來入之。折東南流,與德勒格爾河會。又東南流,託爾和里克河北自博爾圖斯喀倫,兩源並導,百里而合,又南,德勒格爾河自西來會。又南流,布克綏河自西北來會。又南入齊老圖河。中左翼右旗札薩克圖汗之族。初授二等臺吉。乾隆二十一年封輔國公並札薩克,世襲。佐領一。牧地當桑錦達賚之南。桑錦達賚泊在旗境及中左翼左旗之間。西南有色楞格河。右翼右旗札薩克圖汗之族。康熙二十八年授札薩克。三十年封固山貝子,後降輔國公,世襲。佐領一。牧地在烏喇特界內庫埒謨多。左翼後旗札薩克圖汗之族。康熙三十年授札薩克一等臺吉。三十六年晉輔國公,世襲。佐領一。牧地在伊克敖拉里克察罕郭勒。北:烏蘭泊。中右翼末旗札薩克圖汗之族。康熙四十三年授一等臺吉。五十三年授札薩克。雍正二年封輔國公,世襲。佐領一。牧地當濟爾哈河,至是瀦於察罕諾爾。所部察罕諾爾有二,一在左翼右旗之西,其南為齊齊克泊,接科布多界;一即此,濟爾哈河所瀦也。右翼後末旗札薩克圖汗之族。康熙三十六年授札薩克一等臺吉,世襲。佐領一。牧地在奇齊格訥洪果爾阿齊喇克。中右翼左旗札薩克圖汗之族。乾隆二十年授札薩克一等臺吉,世襲。佐領一。牧地在左翼左旗西南。右翼前旗札薩克圖汗之族。康熙二十八年授札薩克。三十年授一等臺吉,世襲。佐領一有半。牧地在阿爾察圖、和嶽爾敖拉、雅蘇圖、鄂和多爾、納默格爾諸界。左翼左旗札薩克圖汗之族。乾隆二十一年授札薩克一等臺吉,世襲。佐領一。牧地在奇勒稽思諾爾、愛拉克諾爾之南,跨空歸河。空歸河又名空陰河,舊作空格依河,出昂奇山南麓,合三水西南流,入札布噶河。中右翼末次旗羅卜藏臺吉之孫。康熙四十八年授札薩克一等臺吉,世襲。佐領一。牧地有特們諾爾、委袞諾爾,兩諾爾水皆發源烏里雅蘇臺軍營城北大山,東北流,瀦為兩大泊,委袞在北,特們在南,中隔一嶺,南北相望,形擬蝌蚪也。中左翼末旗羅卜藏臺吉之裔。雍正十二年授二等臺吉。乾隆二十二年授一等臺吉兼札薩克,世襲。佐領一。牧地當德勒格爾河東岸。附輝特一旗額魯特部輝特族人羅卜藏,為噶爾丹所虐,來歸。乾隆二十年授其孫一等臺吉。三十年授札薩克,世襲。佐領一。牧地當濟爾哈河東岸。濟爾哈河自旗南界合三源東北流,至札薩克圖汗部中右翼末旗界,瀦為察罕諾爾。以上統盟於札克畢賴色欽畢都爾諾爾。

喀爾喀四部八十六旗,統稱外札薩克。自雍正中用兵準噶爾,即於烏里雅蘇臺築城駐兵,城以木為之,中實以土,高丈六尺,厚一丈,在烏里雅蘇臺河北岸。光緒七年,收還伊犁,改訂條約,許俄人在烏里雅蘇臺通商,俟商務興旺,再設領事。定邊副將軍治之。總統四部兵,內蒙古各部兵統於各部札薩克。蓋內札薩克多從龍功臣,而游牧之地悉附近盛京、直隸、山西、陜西一帶,與外札薩克之後來歸附遠在漠北者有別。兼理札薩克圖汗、賽音諾顏兩部事。又設庫倫辦事大臣,庫倫在土喇河上游西岸,人三萬口,喇嘛教徒甚眾。其胡土克圖殿宇嚴莊,蒙民每夏從諸部來頂禮者,道路不絕。理俄羅斯邊事。康熙六十年與俄立約,定為陸路通商埠,各遣官監視。乾隆二年,並停京師貿易,統歸恰克圖辦理,總其權於庫倫大臣。互市處在恰克圖南買賣城,有路南通庫倫,北達上烏丁斯克,與新修鐵路接。有俄國領事署。貿易茶最盛。車臣汗、土謝圖汗兩部事亦歸監理。

杜爾伯特部十六旗:至京師六千餘里。元臣孛罕之裔,姓綽羅斯。六傳至額森,即乜先,生二子。長伯羅納哈勒,為杜爾伯特部祖;次額斯墨特達爾諾顏,為準噶爾部祖。杜爾伯特本分牧額爾齊斯河。乾隆十八年,為準噶爾所逼,率族來歸,編所部佐領左翼旗十一,特固斯庫魯克達賴汗旗、中旗、中左旗、中前旗、中後旗、中上旗、中下旗、中前左旗、中前右旗、中後左旗、中後右旗。右翼旗三,前旗、前右旗、中右旗。附輝特旗二。下前旗俱在科布多河,下後旗俱在烏布薩泊南、杜東輝西。授札薩克,世襲。設科布多參贊大臣以轄之。同游牧科布多金山之東烏蘭固木地。東至薩拉陀羅海、納林蘇穆河,南至哈喇諾爾、齊爾噶圖山,西至索果克河,北至阿斯哈圖河。北極高四十九度十分至二十分。京師偏西二十四度至二十七度二十分。科布多一作和卜多,其水源名索果克河,蓋即索和克薩里也。東流,南合瑚爾噶泊、輝美泊、和通泊水,東北流,西合噶斯河,折而東南流,逕輝特下前旗、杜爾伯特右翼旗,南合塔爾巴泊、託爾博泊水,北合烏里雅蘇圖河、根德克圖泊、戴舒爾泊水,遂名科布多河。東南流,經科布多城西,布彥圖河出阿爾泰烏梁海旗西北流來會。又東流入阿勒克泊。納林蘇穆河,發源特斯河南沙地,西南流,與古薩爾泊水會,西北入烏布薩泊。烏布薩泊在左翼旗北,西與北接唐努烏梁海界。喀喇奇拉河、古薩爾泊水,俱出左翼界,北流,薩克里哈拉河亦出左翼界,東流,俱瀦於烏布薩泊。又東,特斯河、和賴河,東北特里河,北伊爾河、博爾河、札爾河、齊塔齊河,西有哈拉莽鼐山水,俱流入烏布薩泊。南:哈喇泊水、札布噶河,自札薩克圖汗部西北流,東納空歸河,又西北會奇勒稽思泊、愛拉克泊水,西流,南合都爾根泊、哈喇泊水,逕明阿特旗,匯於阿拉克泊。

明阿特部一旗:系出於烏梁海。後為札薩克圖汗部中左翼左旗之屬。乾隆三十年,撤出設一旗,隸科布多大臣轄。牧地在科布多城西。東界起塔拉布拉克至齊爾噶圖山、科布多河止,南界起齊爾噶圖山至茂垓止,北界起茂垓至塔拉布拉克止,俱與杜爾伯特連界。北極高四十八度五十分。京師偏西二十六度二十分。

阿爾泰烏梁海七旗:東界起都嚕淖爾至哈叨烏里雅蘇臺止,與額魯特連界;南界起烏蘭波木、烏龍古河至巴噶諾爾止,與塔爾巴哈臺所屬土爾扈特連界;西界起碑爾素克託羅垓至巴爾哈斯淖爾止,與喀倫連界;北界起巴爾哈斯淖爾至哈竇里達巴止,與喀倫連界。曰左翼副都統旗、散秩大臣旗各一,總管旗二;右翼散秩大臣旗一,總管旗二。北極高四十九度二十分。京師偏西二十九度十分。哈屯河二源,東曰喀喇河,西曰噶老圖河,俱出阿爾泰烏梁海旗北境阿爾泰山北麓,二源合為納爾噶河,東北流,鄂依滿河入之。又東北流,札爾滿河入之。折東流,達爾欽圖河自西南來匯。又東北流,始曰哈屯河。又東北流,逕阿爾泰諾爾烏梁海旗,西納烏賴河、僧瑪爾達河,東納喀達林河。又北流,會亨吉河,入唐努烏梁海界。阿爾泰河亦自科布多西北流來會,又西北入俄羅斯界。西南:華額爾齊斯河,源出阿爾泰山。

阿爾泰諾爾烏梁海部二旗:在索果克喀倫外。東界起哈勒巴哈雅山至布古素山、博羅布爾噶蘇河止,南界起博羅布爾噶蘇至託申圖山、習伯圖山、達爾欽圖河止,西界起達爾欽圖河至阿爾占山、巴勒塔爾罕山、呼巴圖嚕山止,北界起呼巴圖嚕山至阿爾泰諾爾、伯勒山、楚勒坤諾爾、哈勒巴哈雅山止。北極高五十三度。京師偏西二十五度四十分。旗東北有阿爾泰泊,綽爾齊河、沙爾河、巴什庫斯河、阿斯巴圖河,合北流瀦焉。東納格吉河,西納巴哈齊里河、伊克齊里河、郭爾達爾河,北流為阿爾泰河,又西北入唐努烏梁海界,會哈屯河。

博東齊旗、布圖庫旗:均杜爾伯特族。乾隆二十一年來歸,編置佐領。同牧於呼倫貝爾。隸呼倫貝爾都統轄,黑龍江將軍節制。

新土爾扈特部二旗:在科布多城西南。至京師七千餘里。元為乃蠻國,太祖滅之。後為和林行省所屬地。明屬衛拉特。初,始祖翁罕裔舍棱為準噶爾臺吉。七傳至貝果鄂爾勒克。其長子卓立甘鄂爾勒克,即徙牧俄國一支之祖。數傳至渥巴錫,來款,賜牧新疆,號舊土爾扈特。其次子衛袞察布察一支,依準噶爾,傳至舍棱,為準噶爾臺吉。大軍征準噶爾,舍棱奔俄。乾隆三十六年來歸,編佐領,設札薩克,賜牧,號新土爾扈特。二旗:曰新左旗,曰新右旗。自為一盟,曰青色特啟勒圖。隸科布多大臣兼轄。光緒三十二年,劃隸阿爾泰辦事大臣。牧地當金山南、烏隆古之東。東至奔巴圖、捫楚克烏蘭、布勒幹和碩,南至胡圖斯山、烏龍古河,西至清依勒河、昌罕阿璊、那彥鄂博,北至綽和爾淖爾、那郭幹諾爾之中山。北極高四十六度。京師偏西二十七度二十分。拜塔克,地以山名,其山至哈布塔克西、青吉斯河南岸。由拜塔克西南行,至奇臺界,唐時以沙陀部為沙陀州,此其故壤也。烏隆古河二源,東曰布爾干河,西曰青吉斯河。布爾干河出新和碩特旗北,合喀喇圖泊水,南流,經札哈沁旗東南流。青吉斯河出旗境北,合哈泊水,西南流,合哈弼察克河。又東南,與布爾干河合,為烏隆古河。折西流,逕阿爾泰烏梁海旗,瀦為赫薩爾巴什泊。

新和碩特部一旗:在科布多城南。至京師七千餘里。和碩特臺吉巴雅爾拉瑚之族蒙袞。乾隆三十七年來歸,附新土爾扈特貝子旗。後為所虐,移牧杜爾伯特近處。嘉慶元年,給札薩克印,隸科布多大臣兼轄。光緒三十二年,劃隸阿爾泰辦事大臣。牧地當金山東南哈弼察克,西臨青吉斯河。東至和託昂鄂博,西至捫楚克烏蘭,北至奔巴圖、哈弼察克河。北極高四十七度。京師偏西二十七度。哈弼察克一作哈布塔克,地以山名,在鎮西府西北四百里。北六十里即布拉乾郭勒河南山北之地,饒水草,宜畜牧。

札哈沁部一旗:初為準噶爾宰桑。乾隆十九年,大軍獲之。其隨來之札哈沁,即令統轄。四十年,設一旗。嘉慶五年,增設一旗。隸科布多大臣。牧地在科布多城南。東界起德杜庫庫圖勒至巴爾嚕克止,與喀爾喀連界;南界起昂吉爾圖至哈布塔克山止,與巴爾庫勒連界;西界起和託昂鄂博至布爾干河東岸止,與阿爾泰烏梁海連界;北界起惠圖僧庫爾至土古里克止,與喀爾喀屯田兵官廠連界;東北界由土古里克起至德杜庫庫圖勒止,與喀爾喀連界。北極高四十六度五十分。京師偏西二十六度十分。

科布多額魯特部一旗:本臺吉達木拜屬,有罪削爵,以其眾屬科布多大臣轄。東界起齊爾噶朗圖至布古圖和碩止,南界起布古圖和碩至哈叨烏里雅蘇臺止,東南均與喀爾喀屯田兵連界,西界起哈叨烏里雅蘇臺至都嚕諾爾止,北界起都嚕諾爾至習集克圖河止,西北均與阿爾泰烏梁海連界。北極高四十八度五十分。京師偏西二十七度三十分。以上並隸科布多大臣定邊左副將軍轄。

阿拉善額魯特部一旗:在河套以西,袤延七百餘里。至京師五千里。本漢北地郡西境,及武威、張掖二郡北境地。晉為前涼、後涼、北涼所有。唐屬河西節度使。廣德初,陷於西番。宋景德中,陷於西夏。元屬甘肅行中書省。明末為額魯特蒙古所據。元太祖弟哈布圖哈薩爾之裔,世駐牧河西套。後為噶爾丹所滅,其酋逃竄近邊。康熙二十五年,上書求給牧地,詔於寧夏、甘州邊外畫疆給之。東至寧夏府邊外界;南至涼州、甘州二府邊外界,西至古爾鼐接額濟納土爾扈特界,北逾戈壁接札薩克圖汗部界。三十六年,編佐領,授札薩克,封多羅貝勒,駐定遠城。雍正二年,晉郡王。乾隆三十年,晉和碩親王,世襲。佐領八。牧地當賀蘭山西、龍頭山北。北極高三十八度至四十二度。京師偏西十度至十八度。城北有吉蘭泰鹽池,名曰「吉鹽」,歸阿拉善王管轄。自為部,不設盟。賀蘭山在旗東,土人名阿拉善山。山有樹木,青白如駮馬,北人呼駮為「賀蘭」。其山與河東望雲山形勢相接,邐迤向北,經靈武西北,逕保靜西,又北逕懷遠西,又北逕定遠,又東北抵河。抵河之處名乞伏山,在黃河西,從首至尾像月形,南北約長五百餘里,邊城之鉅防也。山之東,山口自北而南曰寧靖、鎮北,至獨樹,凡十九口。又南接邊城曰青羊溝、乾溝,至小關兒,凡十九口。又南則石空寺堡及勝金關也。西山口自北而南曰歸德、紅兒,至黃峽,凡十三口。又南,山勢迤邐而西,其南曰山嘴口、金塔口、杏樹口、赤木口,東接邊城曰大佛寺口、三岔溝口。其西曰靖湖埻,至崇慶,凡六口,鎮北口、寧安口、向陽埻口、殺虎埻口。龍首山一名龍頭山,俗呼為甘峻山,在旗西南,與山丹接界,蒙名阿喇克鄂拉,綿亙廣遠,東大山之脈絡也。距山丹城三里。山盡處為寧遠堡。山南為內地,蒙古俱於山北游牧。旗南有松陜水,自古浪縣北流,逕縣東,又東北至土門堡流出邊。又東北至旗界,瀦為澤。漢志:「蒼松縣南山,松陜水所出,北至揟次入海。」一統志:「按陜音峽,松陜水即今古浪河,邊外積水處總曰海。」有谷水,即三岔河,自河州府城東,東北流,逕鎮番東北出邊,土人呼為郭河,至旗界入白亭海子。地形志:「武威郡襄城縣有武始澤。」水經注:「馬城河又東北逕武威縣故城,東屆此水流兩分,一水北入休屠澤,一水又東流入瀦野。」有水磨川,一名雲川,自永昌城西,東北流,逕新城堡北、水磨堡西,又東流逕永昌城北、寧遠堡西,北流出邊。經旗界,瀦為大澤,蒙古名沙喇鄂模。有休屠澤,即古瀦野。漢志:「武威縣,休屠澤在東北,古文以為瀦野澤。」水經注:「武威北有休屠澤,俗謂之西海,其東有瀦野澤,俗謂之東海,通謂之瀦野。」有魚海,即白亭海,一名小闊端海子,五澗穀水流入此海。有沙喇鄂模,在休屠澤西。水磨川自寧遠堡北出邊,注入其中,方廣三四十里。有昌寧湖,直永昌東北、寧遠堡北四十里,東至鎮番界,多水草楊木。明季青把都游牧於此。有長草湖,在寧羅山北。有伯顏湖,直平番東北邊外。有雙泉,直永昌西北,亦名雙井。有馬跑泉,直永昌北。有高泉、平泉、赤諾泉。有三井,直鎮番西北,有亂井兒。有青鹽池、鴛鴦白鹽池、小白鹽池,皆在鎮番西北邊外。有紅鹽池,在山丹城北,池產紅鹽,其根可作器。定遠城北有鹽池,所謂吉蘭泰池也。

額濟納舊土爾扈特部一旗:在阿拉善旗之西,當甘肅甘州府及肅州邊外。袤延八百里。至京師五千五百餘里。本漢居延縣地,張掖郡都尉治此。後漢安帝時,改置張掖居延屬國,別領居延一城。獻帝建安末,立為西海郡。魏、晉因之。永嘉以後,地屬前涼、後涼、北涼、西涼,相繼割據。元魏為涼州所轄地。隋、唐為甘州、肅州北境。大歷中,陷於吐蕃。宋景德中,地屬西夏,曰威福軍。元,亦集乃路,屬甘肅行中書省。明,甘州、肅州二衛邊外地。元臣翁罕裔。明季為準噶爾所偪,徙居俄境之額濟拉河。額濟拉即窩爾吉譯音之變。土爾扈特居俄久,常遣使入貢。康熙四十二年,其汗阿玉奇之嫂攜其子阿喇布珠爾入藏禮佛,準噶爾阻其歸路,乃款塞乞內屬,賜牧色爾騰。旋定牧額濟納河。雍正七年,封多羅貝勒。乾隆十八年,授札薩克,世襲。佐領一。以來歸在先,故亦稱舊土爾扈特。不設盟長。牧地跨昆都倫河。東至古爾鼐,南至毛目縣丞民地,西至大戈璧,北至阿濟山。北極高四十一度。京師偏西十七度。旗境有掃林山。明馮勝拔肅州,進至掃林山亦集乃路,即此。別篤山今曰畢道山。明紀,洪武五年,副將軍傅友德下額濟納路,次別篤山,即此。東:旗桿山。北:阿濟山。自哈密北逾天山,至巴里坤池,又北渡大砂磧幾三四百里,有阿吉山,亦曰阿濟山。山脈自西北阿爾泰山南來,蜿蜒東趨,橫帶瀚海中,起伏不斷,為喀爾喀西路之南境,其長殆三四千里。東南:合黎山,即禹貢弱水所經也。水經云,「合離山在酒泉會水縣東北」,注以為即合黎山。史記正義,山在張掖縣西北二百里。行都司志云在高臺所北十里、鎮夷所東北三十里,與黑山相接。黑山在鎮夷所東北,屹立沙漠中,一名紫塞。其山口東南至肅州百四十里。東北有狼心山,在金塔寺堡北,南去鎮夷所城五百里,為往來要路。又有孤仁山,在金塔寺堡東北三百五十里,凡往來哈密北山者,必聚於此。南有毛目城。額濟納河在西套額魯特西界。又弱水源出山丹西南,自與張掖河合,其下通名為張掖河。又討來河發源肅州西南番界中,有三派,最西曰討來河,其西又有哈土巴爾呼河,北流百餘里,與討來河合,又東北百餘里,南有巴哈、額濟納二河,合流而北,與討來河會為一,又東北流入邊,繞州南至州東北,合西來之水,又東北出邊,過金塔寺,折北轉東,與張掖河合,又北入居延海。昆都倫河自甘肅肅州北流,經旗境,分二道,匯為澤,俱曰居延海。旗東有澤曰大苦水,南直甘肅張掖縣邊外。大苦水之東有二澤,曰騙馬湖,東南有澤曰沙棗湖,亦曰沙棗泉,在肅州東北金塔寺北,沙棗湖之東,直山丹縣邊外,有澤曰豐盈大泉。以上諸澤,皆瀦於沙。又東有昌寧湖、魚海、白海,其上源皆在甘州府、涼州府界。

南路舊土爾扈特部四旗:在喀喇沙爾城北,當天山之南,珠勒都斯。至京師八千六百餘里。本古西戎地。漢及魏、晉為烏孫國地。北魏,高車國地。周,突厥地。隋,西突厥地。唐,鷹娑都督府地。宋屬西州回鶻。明為回部所據。乾隆二十三年,回疆平,入版圖。三十六年,元臣翁罕裔渥巴錫挈所部內附,遂以其地賜之,是為南路舊土爾扈特,與中路和碩特同游牧,編置佐領。設旗四:曰南路汗旗,曰中旗,曰右旗,曰左旗。授札薩克,世襲。隸伊犁將軍轄。牧地有珠勒都斯河,東逾天山,至博爾圖嶺,南至扣克納克嶺,西至天山,北至喀倫。北極高四十二度五十分。京師偏西三十度四十分。天山一名祁連,一名雪山,一名白山,又曰折羅漫山。自葉爾羌西南蜿蜒而來,曰蔥嶺,至闢勒玉山分脈。其東南一支,繞和闐而東行,其西北一支,繞英吉沙爾、喀什噶爾之西,又北行,達布魯特境,東行繞烏什之北,又逕阿克蘇之北,又逕庫車、喀喇沙爾、吐魯番之北,綿亙七八千里,而至哈密東北百餘里,為北天山,又百餘裏截然而止,則在巴里坤之東,名鹽池山,伏入地中矣。此山為南路回疆、西路伊犁之分界。山陽為自哈密至葉爾羌南路,山北則由巴里坤至伊犁北路也。鹽池山之南,沙磧漫野,即希爾哈戈壁,所謂「千里瀚海」也。其山伏地千餘里,至嘉峪關外沙州之東,突兀起頂,東行名祁連山,所謂南天山也。再東行至洞素達巴罕過脈,東北行至巴圖爾達巴罕,北分一支,至八寶山,形如蓮華,尊成嶽體,乃西寧、涼州、甘州、肅州四郡之鎮山也。又自鎮素達巴罕東行,至野馬川之東,景陽嶺自南而北,東分一支結涼州諸山,西分一支與察罕鄂博過脈,西行至祁連達巴罕,過脈向北,分一支結甘州諸山。珠勒都斯山,在喀喇沙爾城北珠勒都斯之地,北連雪山,回環千餘里,水草豐茂。博爾圖嶺亦名博羅圖塔克,在闢展西南,當喀喇沙爾東北境,其山與阿勒癸山南北相接,形如鎖鑰,西通準部,南界回疆,天山南路一大關隘也。山多積雪,博羅圖河發源北麓,入北谷口西行,通珠勒都斯,出西南谷口,西南行,即喀喇沙爾境。扣克納克嶺亦名庫克納克達巴罕,在愛呼木什嶺西五十里,額什克巴什河發源南麓。山脈自天山正幹之額什克巴什山分支,東行六十里至此。

中路和碩特部三旗:至京師八千六百餘里。舊為四衛拉特之一。牧青海、伊犁諸境,後徙俄羅斯。乾隆三十六年,從土爾扈特汗渥巴錫來歸,詔附南路土爾扈特部同游牧珠勒都斯,編置佐領。設旗三:曰中路中旗,曰中路右旗,曰中路左旗。授札薩克,世襲。歸伊犁將軍轄。牧地在南路舊土爾扈特部之西。東至烏沙克塔爾,南至開都河,西至小珠勒都斯,北至察汗通格山。北極高四十二度五十分。京師偏西三十一度十分。察汗通格山在烏沙克塔勒西,西南距喀喇沙爾城百九十五里,地有廢城,城西有泉,委折而南,經烏沙克塔勒城東,分導灌田,自闢展西入納林奇喇塔克、博羅圖塔克谷口,循博羅圖郭勒,逾塔什海,至其地,為喀喇沙爾東北境。開都河俗名通天河,源出大雪山,經喀喇沙爾西門外,水勢甚寬。東南流,上源曰珠勒都斯河,出布古爾東北山,數水合西南流,西納達賴克河。折東流,歧為二,復合,南北納十餘水而東,北納瑪爾什河,經庫勒爾北,折東南流,注塔裏木河。一統志載葉爾欽有塔裏母河,下流與西北來之海多河合。海多河即開都河,塔裏母河即塔裏木河也。小珠勒都斯河出自阿爾泰陰克遜之北源處,極四十三度十分,西三十一度三十分,即和碩特牧地也。

北路舊土爾扈特部三旗:在塔爾巴哈臺城東,當金山之西南霍博克薩里。至京師九千七百餘里。本漢時匈奴西境、烏孫北境。北魏,蠕蠕地。後周時入於突厥。唐,西突厥地。明時為衛拉特地。舊為準噶爾臺吉游牧處。乾隆二十年,準部平,入版圖。三十六年,元臣翁罕裔袞札布來歸,遂以其地賜之,是為北路舊土爾扈特部,編置佐領。設旗三:曰北路旗,曰右旗,曰左旗。授札薩克,世襲。隸塔爾巴哈臺大臣轄,伊犁將軍節制。牧地東至噶札爾巴什諾爾,西至察漢鄂博,南至戈壁,北至額爾齊斯河。北極高四十六度三十分。京師偏西二十九度十分。有薩里山,即賽兒山。東:噶札爾巴什諾爾,即赫薩爾巴什泊,在哈莽奈山北,凡金山東南烏龍古河、布爾干河、青吉斯河皆匯焉。廣七十里,袤三十里,餘波入於沙磧。泊以東即新土爾扈特牧地。北有額爾齊斯河,一源為華額爾齊斯河,一源為喀喇額爾齊斯河,均出阿爾泰山,二河合為額爾齊斯河。西北流,納蘇布圖河,罕達海圖河、奇喇河,與克木齊克河、固爾圖河、博喇河、哈布河、喀喇哈布河、訥恰庫河、塔爾巴哈臺河。又西北,瀦為宰桑諾爾。俄儂河、果莫孫河匯其東南,納林河、哈流圖河匯其東北,阿布達爾摩多河匯其西。復從諾爾西北溢為額爾齊斯河,科爾沁河入之。又西北布昆河,又北烏柯爾烏蘇,又東北流,納林河、莫依璘河、布克克圖爾瑪河皆入之。又東北流,經塔爾巴哈臺北境、科布多西北境,入俄羅斯界。

東路舊土爾扈特部二旗:在庫爾喀喇烏蘇城西南,當天山之北,濟爾噶朗。至京師九千五百餘里。本漢時烏孫國地。北魏為蠕蠕地。後周時入於突厥。唐為西突厥地。後為嗢鹿州都督府地。明時為衛拉特地。舊為準噶爾各鄂拓克及各臺吉游牧處。乾隆二十年,準部平,入版圖。元臣翁罕裔納札爾瑪穆特來歸,遂以其地賜之,是為東路土爾扈特部,編置佐領。設旗二:曰右旗,曰左旗。授札薩克,世襲。統隸伊犁將軍節制。牧地跨濟爾噶朗河。東至奎屯河,南至南山,西至庫爾喀喇烏蘇屯田,北至戈壁。北極高四十四度二十分。京師偏西三十一度二十分。濟爾噶朗河三源,發庫爾喀喇烏蘇南山,名古爾班恰克圖水。山中產金,置濟爾噶朗金廠。古爾班恰克圖水北流,逕土爾扈特喇嘛寺,又西北流,逕布爾哈齊軍臺西,為濟爾噶朗河,又曰多木達喀喇烏蘇,言於三喀喇烏蘇居中也。布爾哈齊莊南五里許,沙阜湧泉,勢甚湍急,北逕莊東,為布爾哈齊水,西北流,入於濟爾噶朗河。濟爾噶朗河又西北流,入庫爾喀喇烏蘇河。濟爾噶朗廠西南有山曰額布圖嶺,發泉,東北流為額布圖河,又曰固爾班喀喇烏蘇,其水自東北折而西北流,入庫爾喀喇烏蘇河,又西入喀喇塔拉額西柯諾爾。東:奎屯河,在庫爾喀喇烏蘇城東南,源出額林哈畢爾噶山。山產金,置廠。奎屯河北流出山,疏西流渠一,曰樹窩子商戶渠。又北流,逕庫爾喀喇烏蘇城東。又北流,東西各引渠一,東曰河沿子商戶渠,西曰民戶渠。戶屯之北為兵屯河,逕兵屯東,折而西北流,逕軍臺西,為庫爾喀喇烏蘇河。

西路舊土爾扈特部一旗:在伊犁城東,當天山之北,晶河東岸。至京師一萬餘里。本漢時烏孫國地。北魏時為悅般國。尋為蠕蠕所並。後周時入於突厥。唐初西突厥地,後為嗢鹿州都督府地。元,阿勒穆爾地。明時為衛拉特地。舊為準噶爾各鄂拓克及各臺吉游牧處。乾隆二十年,準部平,入版圖。元臣翁罕裔羅卜藏諾顏來歸,遂以其地賜之,是為西路舊土爾扈特部,編置佐領。設西路旗一,授札薩克,世襲。隸伊犁將軍節制。牧地東至精河屯田,南至哈什山陰,西至託霍木圖臺,北至喀喇塔拉額西柯諾爾。北極高四十四度四十分。京師偏西三十二度五十分。哈什山在慶綏城西南,山之陽即伊犁哈什河源所出,合十餘水,西流來會,曰伊犁河。有晶河,舊作精河,源出安阜城南山,其山即伊犁哈什河北岸山陰也。山有峽口,曰登努勒臺。新唐書地理志云,黑水守捉又七十里有東林守捉,又七十里有西林守捉。又經黃草泊、大漠小磧,渡石漆河,逾車嶺,至弓月城。過思渾川、蟄失密城,渡伊麗河,蓋即由登努勒臺至伊犁矣。石漆河或晶河之舊稱,河三源並出,為古爾班晶河,準語晶,謂「蒸籠」也。河濱沙土,濕暖如蒸,故名。西北流出山,經西路一旗土爾扈特游牧一百科樹之西,北距安阜城九十里。又西北流,導西流渠一。又西北流,導東流渠一。又西北流,逕晶河舊城西。又北流,入喀喇塔拉額西柯諾爾。喀喇塔拉額西柯諾爾即鹽海子也,在精河城北。庫爾喀喇烏蘇河出庫爾喀喇烏蘇城南山中,三水合北流,逕城東及北,合南來一水;又西北,濟爾噶朗河自其南注之。又西,敦穆達河亦自其南注之,合流瀦焉,曰鹽海子。

唐努烏梁海部:本明時兀良哈部族。至京師八千餘里。清初來附,屬烏里雅蘇臺定邊副將軍轄。共二十五佐領。二佐領在德勒格爾河東岸;二佐領在庫蘇古爾泊東北;四佐領當貝克穆河折西流處;四佐領當噶哈爾河源;三佐領當謨和爾阿拉河源;十佐領在西北,跨阿爾泰河、阿穆哈河。又附札薩克圖汗部所屬烏梁海五佐領,賽音諾顏部所屬烏梁海十三佐領,哲布尊丹巴呼圖克圖門徒所屬烏梁海三佐領。東南至土謝圖汗及賽音諾顏部、札薩克圖汗部,西南至科布多,北至俄羅斯。北極五十五度四十分。京師偏西二十四度二十分。南:唐努山,延亙千餘里。又有穆遜山。西北:敖蘭烏納瑚山、鄂爾噶漢山,與唐努山相接。阿努河、察罕米哈河、阿穆哈河,皆出其北麓。北:塔爾噶克山,其南為額爾齊克山。有克穆河,即劍河,元史謙河,亦即此水。河出穆遜山西北之託羅斯嶺南麓,曰華克穆河,南流,經哲布尊丹巴呼圖克圖門徒所屬烏梁海三佐領之西。又南流,陶託泊水自東來匯。陶託泊水出穆遜山西麓,兩源並發,合流曰烏魯河,西流瀦為陶託泊。和金哈河匯其北,有二水匯其南。復從泊西北流出,入於華克穆河。華克穆河折西流,逕札薩克圖汗部所屬烏梁海一佐領之西北,又西流,布斯河出章哈山北麓自南來匯。又西,多集瑪河自北來會。又西流,哈爾吉河自南來匯。又西流,有札噶泊,周數十里,當唐努山北,近吉里克卡倫隔山之東,瀦為泊。其水東北流,哈拉穆楞河自東南來匯。又東北流,南入於華克穆河。又折而北流,經札薩克圖汗部所屬烏梁海一佐領境,納東來一小水,又北流,會貝克穆河。自發源至此,一千一百餘里。貝克穆河源出託羅斯嶺南麓,在華克穆河源之西,水南流瀦為伯魯克泊。復南流,博爾魯克河自南來匯。折西流,阿薩斯河亦出託羅斯嶺,瀦為圖集泊,從泊流出,自北來會。又西流,庫克穆河自南來匯。又西流,哈彥薩拉克穆出託羅斯嶺西麓,瀦為特爾里克泊,復從泊中流出,與北來之伯集克穆合,入於貝克穆河。克穆齊克河出唐努山北麓,其南隔山即烏布薩泊也。克穆齊克河東北流,巴爾魯克河自南合一水來匯。又東北,阿克河自西來匯。又東流,北納一小水,南納集爾噶瑚河。又東流,北納一小水,南納札達克河,東入大克穆河。大克穆河西流,謨什克河、巴拉克河皆自南來入之。又西流,烏蘭烏蘇河自北來入之。又西流,謨和爾阿拉河、額錫穆河、察漢河、拉爾河、特穆爾烏蘇河、札庫爾河合三水,皆來匯。圖蘭河出塔爾噶克山西南麓,南流,合鄂克河,入於大克穆河。察漢米哈河發源鄂爾噶漢山北麓,北流,逕敖蘭烏納瑚山西,西北流入阿努河。阿穆哈河亦發源鄂爾噶漢山西北麓,北流,逕烏梁海十佐領之東,折而東北流,入阿努河。特里泊出唐努山北麓,西北流為泊,又西北流,入於華克穆河。額赫河即厄赫河,上源為庫蘇古爾泊,在唐努山烏梁海東南境。伊克杭哈河、納林杭哈河、哈拉錫爾河、納林和羅河俱出穆遜山南麓,南流瀦焉。復自泊東南流出,曰額赫河,南北合數水。庫克陀羅蓋河、達爾沁圖河、鄂依拉噶河、阿勒渾博勒爾河俱出卡倫外,東南流來會。又東逕札薩克圖汗部、賽音諾顏部境,又東南入土謝圖汗部界,北納努拉河、布科倭河,東南會色楞格河。有德勒格爾河,出唐努山東南,東流逾卡倫,東南流,西納伊克河,羅河、託爾和里克河,出德勒格爾河源東,皆東南流,入札薩克圖汗部界。哈屯河自科布多北流入界。阿爾泰河亦自科布多西北流來會,又西北入俄界。蘇特泊在鄂爾噶漢山南。以上隸伊犁將軍節制。


\end{pinyinscope}