\article{列傳一百十九}

\begin{pinyinscope}
富勒渾文綬劉秉恬查禮鄂寶顏希深徐績

覺羅圖思德彰寶徐嗣曾陳步瀛孫永清

郭世勛畢沅

富勒渾,章佳氏。初自舉人授內閣中書。累遷戶部郎中。乾隆二十八年,授山西冀寧道。遷山東按察使。以在冀寧道失察陽曲知縣段成功虧帑,左授山西雁平道。再遷浙江布政使。三十五年,署巡撫。奏劾總督崔應階僕誣指錢塘民為賊,擅刑致斃,論罪如律。三十七年,調陜西。尋擢湖廣總督,入覲,賜孔雀翎。四川總督阿爾泰坐貪黷玩縱得罪,上命富勒渾如四川,會總督文綬按治。阿爾泰縱子明德布與布政使劉益相結受賕,明德布在京師,上令軍機大臣傳訊,自承,富勒渾奏論益立斬。上以為過重,改監候,獄連署布政使李本,富勒渾奏本罪當奪職,枷示不足蔽辜,請留軍效力。上責其名重實寬,意存取巧,命枷示期滿,留軍效力。

三十八年,師征金川,四川總督劉秉恬出駐美諾,命富勒渾留署四川總督,總理各路軍需。秉恬奏:「揀發往川省各員視軍營為畏途,惟恐出口辦差不通聞問。」上以責富勒渾,富勒渾奏陳:「司道公議,新到各員出口辦差,未免竭蹶。請以現任各員調赴,而令新到者分別署理。」上責富勒渾玩公沽譽,令劾倡議者,富勒渾奏司道公議,並無倡始。上益不懌,謂:「富勒渾竟敢以罰不及眾哧朕!」下部議,奪官,命寬之。

木果木師潰,底木達被陷。富勒渾率新至貴州兵馳赴蒙固橋防守,事聞,上嘉之。旋奪秉恬官,即以富勒渾實授,令駐美諾,以欽差大臣關防督餉。時美諾亦被陷,富勒渾屯明郭宗河口,據山梁設卡防守,復發兵分駐路頂宗、巴朗拉。將軍阿桂進攻小金川,上命富勒渾與提督王進泰統兵策應。師克美諾,上令富勒渾、進泰嚴守美諾,並分兵駐僧格宗、明郭宗。阿桂奏富勒渾、王進泰通慎而葸,於山川形勢、行軍機要均未能悉,請令副都統成果、雲南提督常青駐守後路,上從之,諭戒富勒渾等勿存畛域。奏新開楸底至色利溝運道,軍糧歸此路運送。瑪爾當、明郭宗諸地存米,借防兵一月糧,餘俱運軍前,請撤前設臺站;又奏分兵駐防大板昭及梭格泊古諸地。四十年,奏阿桂等督兵進搗賊巢,應用糧餉、軍火、銅片、砲料,儲備充裕,並造皮船濟師;又奏調梭格泊古、瑪爾當兵分防沙壩、三松坪,以護運道:皆稱旨。上命富勒渾駐布朗郭宗,富勒渾奏阿桂、明亮合攻甲索山梁,布朗郭宗距軍五百餘里,慮難於策應。上諭曰:「阿桂進攻勒烏圍,自應隨軍督餉。兵事移步換形,不必泥前旨也。」師克勒烏圍,奏請撤前設卓克採一路臺站。四十一年,復授湖廣總督,命師還上官。金川平,議敘。

四十二年,授禮部尚書。四十三年,調工部。授鑲藍旗蒙古都統。四十四年,復授湖廣總督。四十五年,調閩浙,上南巡,迎謁。時李侍堯以貪縱得罪,富勒渾入對,上諭及之。富勒渾對:「侍堯實心體國,為督撫中所罕見。」及上命各督撫議罪,又請行誅,上責其前後歧異。浙江巡撫王亶望丁憂,留辦塘工,攜家居杭州。亶望得罪,上又責富勒渾未劾。大學士阿桂赴浙江閱海塘,疏劾杭嘉湖道王燧,又責富勒渾徇庇。奪孔雀翎,降三品頂帶,授河南巡撫。河溢萬錦灘,富勃渾親赴防護;又溢青龍岡,四十七年,工竟,還現任頂帶。

復授閩浙總督。臺灣漳、泉民械斗,劾總兵金蟾桂、知府蘇泰等,並奪官。五十年三月,入京,與千叟宴。調兩廣。粵海關監督穆騰額入覲,上詢富勒渾操守,對:「未敢深信。」及命軍機大臣詰之,又發富勒渾縱僕殷士俊納賕狀,下巡撫孫士毅按治。士俊常熟人,並令江蘇織造四德等籍其家資累萬;士毅奏亦發富勒渾與士俊等關通納賄事實,上奪富勒渾官,遣尚書舒常如廣東會訊。大學士阿桂方按事浙江,又命士毅逮富勒渾監送阿桂鞫治,論斬,下刑部獄。五十二年,詔釋之。五十三年,坐在閩浙失察總兵柴大紀貪劣,復下刑部論紋,仍釋之。五十四年,羅源盜發,上追論富勒渾廢弛玩誤,戍伊犁。五十五年,釋回。六十一年,又發熱河,是年即釋回。卒。

文綬,富察氏,滿洲鑲白旗人。雍正十三年,自監生授內閣中書。再遷禮部員外郎,改內閣侍讀。乾隆十一年,授甘肅涼州知府。累遷轉山西布政使。三十一年,坐迎合巡撫和其衷徇陽曲知縣段成功虧帑,奪官,戍軍臺。旋授道銜,往哈密辦事。三十三年,授河南巡撫,未上官,調陜西。三十六年,署陜甘總督。土爾扈特內附,命赴齊齊哈爾犒勞。授四川總督,未行,仍調授陜甘。

師征金川,奏陜、甘發兵三千,延綏鎮總兵書明阿以千人赴維州,興漢總兵張大經以二千人入四川從征,文綬如鞏昌、安定視師行。三十七年,疏言:「巴里坤、烏魯木齊年來日繁盛。招民墾地,戶給三十畝,並農具籽種,視新疆例,六年升科。瑪納斯城南可二萬餘畝,瑚圖璧城西北可六千餘畝,巴里坤城外及傍近諸地五千九百餘畝,玉門、酒泉、敦煌三縣可五千餘畝。往時嘉峪關恆閉,過者候譏察,今關外已同內地,請令辰開酉閉;兼開烏魯木齊城南七達色巴山梁以利行旅。」又酌定收捐監糧,籌備巴里坤移駐滿洲兵糧料;並於巴里坤山灣設廠牧羊,令滿洲兵子弟取乳剪毛,以廣生計。均如所請行。

三十七年,調四川總督。前政阿爾泰坐誤軍興,又縱其子明德布婪索,得罪,上命文綬察明德布婪索狀。文綬言:「明德布侍阿爾泰日久,與屬吏往還,尚無婪索事。」而明德布在京師,上命軍機大臣按鞫,具服,乃責文綬袒護,奪官,往伊犁效力。三十八年,木果木師潰,總督富勒渾奏報金川酋攻明郭宗河口,上授文綬頭等侍衛,佐富勒渾治軍。未幾,授湖廣總督,仍署四川總督。偕富勒渾奏言:「增兵需餉,請令商民原自湖廣運糧入四川者,視乾隆十三年範毓馪助餉加銜例,穀一石當銀九錢,授以貢監職銜。」並議行。四十一年,實授。四十四年,入覲。子國泰,官山東巡撫,召詣京師相見。四十五年,疏言:「雲南昭通、東川諸屬改食川鹽,應於川、滇交界隘口設稽察。」上可其奏,並諭云貴總督福康安一律嚴防。四十六年,詔停打箭爐收稅部員,由總督委員管理,因條奏裁改諸事,從之。四川多盜,民間號啯嚕子,闌入鄰近諸省。湖廣總督舒常、湖南巡撫劉墉、貴州巡撫李本先後疏言盜自四川入境,遣將吏捕治。文綬奏後入,上責其玩縱,降三品頂帶。尚書周煌復陳盜為民害,將吏置不問,甚或州縣吏胥身為盜擾民,上以文綬因循貽患,奪官,往伊犁效力。四十八年,釋回。四十九年,卒。子國泰,自有傳。

劉秉恬,字德引,山西洪洞人。乾隆二十一年舉人。二十六年,明通榜,授內閣中書,充軍機處章京。再遷郎中。三十二年,考選福建道御史,轉吏科給事中。大學士傅恆督師討緬甸,以秉恬從,擢鴻臚寺少卿。師還,超擢左副都御史。遷刑部侍郎,調工部,再調倉場。

三十七年,師征金川,大學士溫福出西路,總督桂林出南路,授秉恬欽差大臣,督西路糧運。尋以南路徑僻站長,輓運尤艱,命改赴南路。秉恬以西路需餉急,請暫留料理,上韙之。又奏:「南路運糧,人俱畏其難。臣非敢言易,然天下無必不可辦之事。」上諭令勉為之。尋奏:「師自甲爾木進攻小金川,道路險阻,唯羊可陟。乃招蠻民販羊至軍,以六羊當米一石。」又奏:「師攻克僧格宗,距達烏圍六十餘里。臣往勘,擬於策爾丹色木設站。其地有喇嘛寺,糧至即貯寺,以蔽風雨。」旋赴美諾督運。上嘉秉恬不辭勞瘁,賜孔雀翎,授四川總督,仍留美諾督運。

三十八年,師克小金川,溫福督兵進攻昔嶺。上命秉恬將美臥溝、曾頭溝兩路酌量形勢,分別駐守,赴木果木及功噶爾拉兩地察勘。秉恬奏至,與上諭正合,深嘉之,諭謂:「勤勞軍務,與統兵督戰無異。命交部照軍功議敘。」秉恬途中得綽斯甲布土司遣頭人投稟,訐綽斯甲布與金川親暱,雖從征未嘗盡力,並請歸金川所侵噶爾瑪六宗諸地。秉恬諭:「師討金川,斷不中止。噶爾瑪六宗諸地,事平後當有公斷。爾土司從征未得一地,且縱金川人在境內為盜,所謂盡力者安在?」頭人語塞,奉檄而去。疏聞,上嘉秉恬甚合機宜。秉恬至木果木,復奏:「臣自崇德抵功噶爾拉,地氣極寒,四山皆雪,甫經設站,以篾席支棚,使人畜暫有棲止。至簇拉角克為布朗郭宗運糧要道,兩口東西相距六七十里,開修土路,通至木波,即合帛噶爾角克碉及布朗郭宗大道。又自功噶爾拉至木果木,路陡雪滑,已飭修路鑿冰,不致少誤糧道。」報聞,加太子少保。木果木師潰,以提督董天弼失守底木達、布朗郭宗責秉恬不先奏劾,奪官,予按察使銜留軍。旋並削銜,命佐按察使郝碩督西路運糧。

三十九年,奏面視米易取攜,已由四川採辦十數萬斤;又奏修整楸坻至日爾拉薩拉驛道,並與總督富勒渾議以北路軍餉歸西路遞運:上並嘉納。四十年,以督運無誤,授兵部郎中,仍賜孔雀翎,以欽差關防督餉。未幾,擢吏部侍郎。以母病召還京師,旋丁憂。未幾,起署陜西巡撫。四十五年,召入覲,調署云南巡撫。

四十六年,署云貴總督。安南國王以內地人民出邊居住,脅制土民欠稅,且動稱內地差委,徵索租賦,大為民擾,咨請防禁。秉恬擬照會,略謂:「內地百姓緣爾國需用貨物,特準開關通市,為爾國利賴。本非在外墾田種地,無應納租賦,焉有脅制土民欠稅之理?如滋生事端,惟有責令爾國察出送回內地究治。」奏聞,上嘉其得體,仍令軍機大臣刪改,寄秉恬具答。累年以運銅妥速,議敘。五十一年,召授兵部侍郎。五十二年,調倉場。嘉慶四年,復調兵部。五年,卒。

查禮,字恂叔,順天宛平人。少劬學。乾隆元年,應博學鴻詞科,報罷。入貲授戶部主事,揀發廣西,補慶遠同知。舉卓異,上命督撫舉堪任知府者。巡撫定長、李錫秦先後以禮薦。十八年,擢太平知府,母憂去。服闋,補四川寧遠。三十三年,擢川北道。三十四年,調松茂道,

小金川用兵,總督阿爾泰檄禮治餉;將軍溫福師進巴朗阿,大營以禮從,令修建汶川桃關索橋,逾月工竟,上嘉之,命專司督運西路糧餉。三雜穀土司為小金川煽惑,頗懷疑懼。禮諭以利害,眾感服。時溫福出雜穀腦,遣提督董天弼分兵自間道出曾頭溝。軍需局以儲米半運雜穀腦,曾頭溝軍糧不足,禮坐奪官,仍留軍效力。師克美諾,溫福令禮與天弼清察戶口地糧,總兵五福自美諾移軍丹壩。總督劉秉恬奏禮雖文員,頗強幹,諳番情,命署松茂道,代五福駐美諾撫降番。

三十八年,木果木師潰,禮偕游擊穆克登阿赴援,至蒙固橋,聞喇嘛寺糧站陷,士卒狼顧;會松茂總兵福昌至,遂復進,遇伏,禮率督兵擊之,擒砦首,餘寇驚遁。美諾已陷賊,阿桂馳援,以達圍垂陷,檄禮駐守,尋命真除。三十九年,阿桂師再進,令禮專任臥龍關路糧餉。阿桂秉上旨,以南路陰翳,設疑兵牽綴,奇兵自北山入。禮請自楸坻至薩拉站開日爾拉山,山高五十里,冰雪六七尺,故無行徑。禮登高相度,以火融積凍,鑿石為磴,不匝月通路二百餘里。自楸坻達西北兩路軍營,視故道皆近十餘站,省運費月以鉅萬計,特旨嘉獎。

郭羅克掠蒙古軍牲畜,殺青海公里塔爾,富勒渾令禮及游擊龔學聖捕治,復盜二,還牛馬五百餘,盜渠牛獲。富勒渾以禮行後糧運漸遲誤,奏促禮還。四十一年,金川平,禮留辦兵屯,拊循降番,敘功,賜孔雀翎。上遣理籓院郎中阿林、知府倭什布、參將李天貴出黃勝關捕郭羅克盜渠,未得,皆坐奪官;仍令禮往捕,禮調三雜穀土兵四千,先令裹糧疾進。禮至,宣布上意,郭羅克酋瑪克蘇爾袞布來謁,問盜渠所在,諉不知;禮執送內地,責其弟索朗勒爾務捕盜。四十三年,瑪克蘇爾袞布病死,上責禮失撫馭番夷之道。四十四年,擢按察使。瞻對番劫里塘熱砦喇嘛寺,禮往按,得盜,寘於法。

四十五年,遷布政使。尋擢湖南巡撫。入覲,四十六年,卒於京師。子淳,大理寺少卿。

鄂寶,鄂謨託氏,滿洲鑲黃旗人。父西柱,官西安將軍。鄂寶自官學生授內閣中書。再遷戶部員外郎。乾隆十六年,授奉天府尹。二十年,署廣西巡撫。二十六年,總督李侍堯劾陸川知縣應斯鳴等縱賊害民,鄂寶奏前後相歧,奪官,以三品銜往庫車辦事。三十一年,召還,署左副都御史。仍授巡撫,歷湖北、貴州、福建、廣西、山西諸省。內遷刑部侍郎。

金川用兵,三十七年七月,命侍郎劉秉恬及鄂寶督餉,秉恬主西路,鄂寶及散秩大臣阿爾泰主南路,尋令改主西路。鄂寶議人負米五斗,日行一站,騾負米石,日行可二三站,改以騾運,軍糈得無缺,賜孔雀翎。三十八年,仍授山西巡撫,督餉如故。溫福師自功噶爾拉入,阿桂自當噶爾拉入,豐升額自綽斯甲布入。鄂寶駐大板昭主餽溫福軍,秉恬駐底木達主餽阿桂軍;而豐升額軍出綽斯甲布,南路自打箭爐往,秉恬兼任之,西路自三雜穀、丹壩往,鄂寶兼任之。木果木師潰,底木達、大板昭皆陷賊。上命阿桂整兵復進,鄂寶仍駐覺木交督餉。旋進翁古爾壟,疏調副將董果護後路。上又命原任江西布政使顏希深馳驛往佐之。副將軍明亮等又請令鄂寶駐丹東,上念鄂寶兵少,命以湖廣續調兵千人屬鄂寶。阿桂又疏請桂林率李世傑主南路,令鄂寶主西路。丹壩至綽斯甲布糧運,鄂寶請以丹東屬桂林兼領。旋詣丹壩置臺站,副將軍豐升額自凱立葉進兵。鄂寶請自三雜穀、梭磨、卓克採轉輸凱立葉,較丹壩道為近。豐升額進攻穀噶,鄂寶請自梭落柏古轉輸色木多,凱立葉留少兵,即裁站夫,省糜費。會明亮自宜喜進兵,既克達爾圖,兩路軍合師沙壩,克勒烏圍。鄂寶請將西路臺站以次裁撤。

四十一年,金川平,軍功加一級。七月,調湖南巡撫,仍留辦軍需奏銷。十月,授漕運總督。四十四年,大學士於敏中等議報銷四川軍需不符,請令鄂寶等分償,得旨豁免。四十八年,授盛京戶部侍郎,兼奉天府府尹。五十二年,卒。子文通,官內閣侍讀學士,兼公中佐領。

顏希深,字若愚,廣東連平州人。入貲授山西太原同知。累遷山東泰安知府。建考棚、書院,清察徵漕浮收諸弊。高宗東巡,召對,褒以「他時可大用」。乾隆二十七年,授四川按察使,入覲,上以希深母老,尚欲隨任,希深亦不敢奏請改補近地,母子知大義,命調希深江西。二十八年,遷福建布政使。三十二年,調江西,丁母憂去。三十四年,仍授江西布政使,又丁父憂去。三十八年,詣京師,命赴金川軍佐鄂寶治餉,援河南布政使,仍留軍。疏言:「糧臺設木池,因限於山,與軍營相隔,將山地開平安營。臣與黃巖總兵李時擴督兵防護,時令將弁操演,不但技藝熟練,而槍聲遠近相聞,亦可牽綴賊勢。」又言:「覺木交深林密箐,賊易以藏身。臣督兵斬伐林木,使附近賊碉有徑可通處,絕無遮蔽,藉免竊發。」皆稱旨,賜孔雀翎。木池站焚毀火藥,希深請與時擴分償。師深入,山重雪積,希深催督拊循,恆終夜露宿。四十二年,擢湖南巡撫。旋入為兵部侍郎。四十五年,復出署貴州巡撫,調雲南。卒。

徐績,漢軍正藍旗人。乾隆十二年舉人。入貲授山東兗州泉河通判。累遷山東濟東泰武道。三十四年,擢按察使,丁父憂,命以按察使銜往哈密辦事,賜孔雀翎。三十五年,擢工部侍郎、烏魯木齊辦事大臣。三十六年,奏:「瑪納斯在伊犁、塔爾巴哈臺之間,請駐兵,使聲勢聯絡。」從之。授山東巡撫。三十八年,上幸天津,迎謁,賜黃馬褂。

三十九年,壽張民王倫為亂,績率兵捕治,次臨清城南,為倫所圍,總兵惟一赴援,戰敗。上遣左都御史阿思哈率兵援績,並令大學士舒赫德視師。諭曰:「績為巡撫,地方有此奸民,不早覺察,不為無罪;但以民亂將巡撫治罪,適足長其刁頑,事定,功過自不能掩。」尋事定,命解任,責捕倫餘黨,捕得倫弟柱、林等二十餘人。上嘉績黽勉,授河南巡撫,仍繳進孔雀翎示儆。四十二年,奏按察使趙銓健忘,上責績於銓應否去留不置一辭,下吏議,奪官,命寬之。召授禮部侍郎。四十七年,坐雩祭禮器誤,奪官,以三品頂帶往和闐辦事。召授正黃旗漢軍副都統,遷正紅旗漢軍都統。六十年,上詢前政弘旴在官事跡,奏不實,奪官,以六品頂帶往和闐辦事。

嘉慶元年,授三等侍衛、烏什辦事大臣。召授大理寺少卿,還孔雀翎。再遷宗人府府丞。十年,以病乞休。十二年,重與鹿鳴宴,賜二品銜。十六年,績子錕,授建寧總兵,入覲,上以績年逾八十,調錕直隸正定總兵,俾就養。卒,錕官至直隸提督。

覺羅圖思德,滿洲鑲黃旗人。初自諸生授光祿寺筆帖式。累遷戶部員外郎。外授江南常鎮道。再遷貴州布政使。乾隆三十七年,擢巡撫。疏言:「貴州威寧瑪姑柞子廠,水城福集廠產黑、白鉛,歲供京局及各省鼓鑄。廠員營私滯運,請立條款,嚴處分。」並下部議行。三十九年,署云貴總督。上令出駐永昌,並諭以防邊事重,視前政彰寶舊日章程益加奮勉。抵任後,疏言:「清釐彰寶移交文牘,永昌軍需造銷牽混,應請各歸各款,以清眉目。造解京箭,各鎮協稱現多損壞,與彰寶原奏不符;又有批準保山等縣添買倉穀,亦滋疑義。」尋劾保山知縣王錫、永平知縣沈文亨侵虧倉穀,請奪官鞫治。上命侍郎袁守侗馳驛往按,錫言彰寶勒索供應四萬餘,致虧短兵糧,上震怒,逮彰寶治罪。圖思德以箭二十萬解四川軍營,上嘉之。十一月,兼署云南巡撫。

自傅恆征緬甸還師,緬甸貢使久不至,閉關絕市年久。圖思德奏言:「偵知緬民亟盻開關,緬酋亦窘迫有投誠意。惟風聞難信,但當簡練軍實,使聞風生畏。」上韙之。及兼署巡撫,自永昌還會城,令提督錦山等董理邊防,疏報,怫上意,嚴旨促仍赴永昌督辦邊防。四十一年,復奏:「偵知緬酋懵駁已死,子贅角牙嗣立,方幼,頭人得魯蘊將遣使叩關納貢。」上以緬甸初無悔罪輸誠之意,諭勿輕聽。尋奏:「得魯蘊遣使投稟,原送還內地官人,貢象,乞開關。已飭龍州將吏與以回文。」上以圖思德示緬甸有遷就結案之意,斥為大謬。四十二年,又奏得魯蘊欲將所留楊重英、蘇爾相、多朝相等送還,並叩關納貢。上念受降事重,圖思德不能勝其任,命大學士阿桂赴雲南主持。調李侍堯雲貴總督,圖思德回貴州巡撫任。四十四年,擢湖廣總督。卒,賜祭葬,謚恭愨。

彰寶,鄂謨託氏,滿洲鑲黃旗人。乾隆十三年,自繙譯舉人授內閣中書。十八年,授江蘇淮安海防同知。累遷江寧布政使。三十年,授山西巡撫。陽曲知縣段成功虧帑事發,具得巡撫和其衷畀銀五百為彌補及布政使文綬等知情狀,奏聞。上遣侍郎四達會鞫得實,其衷、成功論斬,文綬等戍軍臺。安邑知縣馮兆觀揭河東鹽政達色累商及受贄禮、門包,又遣四達會鞫,並得河東運使吳雲從因被四達糾參,嗾兆觀揭發狀,達色論死,雲從、兆觀治罪如律。三十二年,調江蘇。兩淮鹽政尤拔世奏繳本年提引徵銀,上以此項歷年均未奏明,自乾隆十一年起,應有千餘萬,命彰寶會同詳察。前任鹽政高恆、普福、運使盧見曾均坐是得罪;又發前任監掣同知楊守英詐取商銀:並論如律。

三十四年,命馳驛往雲南署巡撫。師征緬甸,署云貴總督,命出駐老官屯督餉,加太子太保。三十五年,奏:「永昌沿邊千餘里,山深徑僻,應於曩宋關、緬箐山、隴川、龍陵、姚關及順寧篾笆橋設卡駐兵。」上令實力督率。又奏:「貴州調至兵間有老弱,現加甄汰。」上責:「彰寶現為總督,兩省皆所轄,何不劾奏?」三十七年,劾雲南巡撫諾木親才識不能勝任,召還;又奏車裏宣慰土司刀維屏逃匿,請裁土缺設專營,上從其議,定營名曰普安。尋實授雲貴總督。三十九年,以病請解任。王錫事發,奪官,逮京師論斬。四十二年,卒於獄。

徐嗣曾,字宛東,實楊氏,出為徐氏後,浙江海寧人。乾隆二十八年進士,授戶部主事。再遷郎中。四十年,授雲南迤東道。累遷福建布政使。五十年,擢巡撫。五十二年,臺灣民林爽文為亂,調浙江兵,經延平吉溪塘,兵有溺者,嗣曾坐不能督察,下吏議。亂既定,五十三年,命赴臺灣勘建城垣,因命偕福康安、李侍堯按柴大紀貪劣狀,上責嗣曾平日緘默不言。尋疏言大紀廢弛行伍,貪婪營私,事跡昭著。又奏:「撫恤被難流民,給銀折米,福建舊例,石準銀二兩;今以米貴,請改為三兩。」上以福康安奏晴雨及時,歲可豐收,仍令視舊例。偕福康安等奏清察積弊,籌酌善後諸事,均得旨允行。嘗以臺灣吏治廢弛,不能早行覺察,自劾,上原之。命臺灣建福康安、海蘭察生祠,以嗣曾並列。尋奏臺灣海疆刁悍,治亂用嚴,民為盜及殺人者,役殃民,兵冒糧,及助戰守義民或挾嫌害良,皆立置典刑,以是稱上旨,嘉嗣曾不負任使。事觕定,命內渡,尋又命俟總兵奎林至乃行。莊大田者,與爽文同亂,坐誅,嗣曾捕得其子天畏及用事者黃天養送京師,又得海盜,立誅之。五十四年,賜孔雀翎、大小荷包。圖像紫光閣。

請入覲,未行,安南阮光平據黎城,福康安督兵赴廣西,嗣曾署總督。福康安瀕行,奏福建文武廢弛,宜大加懲創,上諭嗣曾振刷整頓。嗣曾奏許琉球市大黃,限三五百斤,諭不可因噎廢食。又奏:「福建民多聚族而居,有為盜,責族正舉首,教約有方,給頂帶;盜但附從行劫未殺人拒捕,自首,擬斬監候,三年發遣,免死。」上諭曰:「捕盜責在將吏。令族正舉首,設將吏何用?族正皆土豪,假以事權,將何所不為?福建多盜,當嚴治。若行劫後尚許自首免死,何以示儆?二條俱屬錯謬。」

五十五年,高宗八旬萬壽,臺灣生番頭人請赴京祝嘏,嗣曾以聞,命率詣熱河行在瞻覲。十一月,回任,次山東臺莊,病作,遂卒。

陳步瀛,字麟洲,江南江寧人。乾隆二十六年會試第一,選庶吉士,改兵部主事。累擢郎中,外授河南陳州知府。再遷山西按察使。尋以山西獄訟繁多,改命長麟,仍留步瀛蘭州道。旋授甘肅按察使。

薩拉爾回蘇四十三亂既定,四十九年,鹽茶回田五復據石峰堡為亂,總督李侍堯率兵討之,以步瀛從,捕治諸亂回家屬。旋奏令赴安定、會寧督餉,行次隆德,聞副都統明善戰死高廟山,步瀛以靜寧、隆德、平涼諸州縣當下隴要沖,靜寧駐兵三百,請益兵。步瀛調固原兵五百赴平涼、隆德守,為犄角;復往靜寧收明善餘兵守隘,上獎許之,尋諭:「步瀛兵事徑行陳奏,不必拘體制。」步瀛奏:「臣收明善餘兵,尚存九百有奇。石峰堡回越隆德犯靜寧,平涼知府王立柱督兵民擊之,回退據翠屏山。靜寧距省五百餘里、中間會寧、安定為糧運要道。慮回自靜寧南竄襲我師之後,已稟督臣發重兵防護。」旋疏報靜寧圍解,並籌濟南、西二路官軍糧餉藥彈,稱上旨。上命大學士阿桂視師,以福康安代侍堯為總督。上諭以軍事諮步瀛,擢布政使。福康安奏:「步瀛明白誠實,督餉甚力,但才具不如浦霖。」命調安徽布政使。事定論功,賜孔雀翎。

江、淮大饑,民脅眾劫奪。步瀛行縣,督吏賑恤,而捕治其不法者,自夏迄秋,事漸定。步瀛以勞瘁致疾,五十四年,擢貴州巡撫,疾大作,卒。

孫永清,字宏度,江南金匱人。乾隆三十三年舉人,授內閣中書。永清未入官,嘗佐廣東布政使胡文伯幕。土司以爭襲相訐,驗文牒皆明印,大吏欲以私造符信罪之。永清具稿請文伯力陳,得免者二百餘人。旋充軍機處章京,撰擬精當,事至輒倚以辦。遷侍讀。四十二年,雲南總督圖思德奏緬甸將遣使入貢,上遣大學士阿桂往蒞,以永清從。緬甸使不至,阿桂令永清撰檄諭之,送所留守備蘇爾相還。四十四年,授刑部郎中。考選江西道監察御史。四十五年,超授左副都御史。授貴州布政使。奏言柞子廠產黑鉛,課餘三十餘萬斤,請以十萬斤運廣。四十九年,署巡撫。又奏:「柞子廠黑鉛,例於四川永寧設局收發,課餘三百萬斤,請歲以五十萬運存永寧。」

五十年,擢廣西巡撫。劾新寧知州金自等逋稅,按察使杜琮、鹽道周延俊等並坐奪官。五十二年,臺灣民林爽文為亂,徵廣西兵,永清奏:「兵出征,在例馬兵賞、借銀各十兩,步兵賞、借銀各六兩,請於借銀留三兩為制衣。」命議敘。五十三年,藤縣獄系盜梁美煥謀穴墻逃,捕得,永清令立誅之,奏聞,上諭曰:「獄囚反獄劫獄當立誅,若鉆穴越墻,祗求茍免,不得與此同科。今之督撫皆好殺弄權,永清失之太過。」

安南阮惠為亂,國王黎維祁出亡,其臣阮輝宿護維祁母、妻、宗族至龍州,永清及總督孫士毅疏聞。士毅尋發兵討惠,永清出駐南寧,奏太平設軍需局,以福建延建邵道陸有仁、桂林知府查淳董其事。五十四年,維祁復國,使迎其母、妻、宗族,永清為具行李,並傳上旨賚錦緞、綢、布及白金四百。諭獎永清自駐南寧,彈壓邊關,籌辦餉糈,措置得宜,賜孔雀翎。

士毅師敗還,福康安代為總督。永清與福康安奏:「安南用兵,關內外支放銀百萬、米八萬餘,逐款詳覈,例可用而未用,或用不及數者,以實用之數具報。如有軍行緊急,略有變通。與例不符者,仍如例覈減。」上諭令以實為之。秋,以廣西秋審冊自緩決改情實凡三案,諭責永清寬縱。東蘭州安置臺灣降人鄭管、陳廷乘舟走,追捕,以溺水報。上命奪知州黃圖等官逮訊,永清坐降調,命留任。

是時阮惠更名光平,上封為安南國王,請以來年詣京師祝萬壽,使阮宏匡等叩關入貢。永清令在太平候旨,疏聞。上令光平使臣於來年燈節前至京師,與外籓蒙古等一體入宴,責永清拘泥。永清旋奏光平使臣自桂林北行。上察廣西學政潘曾起不稱職,以諮永清,永清言曾起性情褊急,未愜士心。上責永清不先奏劾,以方料理安南內附,光平將入覲,不遽易人,罰養廉二年。五十五年春,光平又以新賜印並御制詩使叩關入貢,永清疏以應否令光平使詣京師請旨。上諭曰:「光平遣使陳貢,自應令詣京師,何必奏請?」永清又奏太平、南寧、鎮安三府與安南接壤,請屯兵防隘,立柵開壕,分隸龍憑、馗纛二營管轄,報聞。四月,光平入關,以其子光垂、臣吳文楚從,奏聞,上嘉之。尋卒。

弟籓,監生。以四庫館議敘,授中書科中書。官至安徽布政使。子爾準,自有傳。

郭世勛,漢軍正紅旗人。初自筆帖式擢吏部主事。選福建龍巖知州。五遷湖南布政使。乾隆五十四年,擢貴州巡撫,調廣東。上諭曰:「廣東有洋商鹽務,為腥膻之地。世勛操守廉潔,治事勤實,務慎持素履。」監臨鄉試,奏額送科舉多取數百名,經費由督撫捐貲備辦,諭國家無此政體,不允。奏禁大黃出洋,西洋各國歲不過五百斤,瓊州、臺灣亦如之;暹羅、安南貢船至,亦五百斤。五十五年,總督福康安入覲,命世勛署兩廣總督。劾雷瓊鎮總兵葉至剛誤民為匪,左江鎮總兵普吉保濫刑斃命,皆論罪如律。參將錢邦彥巡洋崖州,遇盜被戕,上以福康安詣京師後,世勛不能整飭,嚴斥之。

暹羅國王鄭華咨:「乾隆三十一年被烏圖構兵圍城,國君被陷。其父昭克復舊基,十僅五六。舊有丹蓍氏、麻叨、塗坯三城,仍被占據。請代奏令烏圖割還三城。」烏圖即緬甸。世勛以其非禮妄干,留其使廣東,奏聞。上命軍機大臣擬檄,略謂:「故緬甸酋懵駁與暹羅詔氏構兵,非今國王孟隕事。暹羅又系異姓繼立,不直追問詔氏已失疆土。天朝撫馭萬國,緬甸固新封,暹羅亦至華嗣掌國始加封爵,宜釋嫌修好,共沐寵榮,不得以非分干求,妄行瑣瀆。」命世勛與福康安聯銜照會,並告來使,但云:「札商福康安,未經代奏。」

五十六年,世勛奏洋船準攜砲,內地商船不準攜砲。上諭之曰:「商船出洋,攜砲御盜。不特各國來船未便禁止,即內地商船遇盜不能御,豈有束手待斃之理?祗令海口將吏察驗,不可因噎廢食。」上以廣東多械斗,諭世勛稽察化導。有步文斌者,以罪配德慶州,傳習邪教,世勛捕得四十餘人送京師。上諭以其渠送京師,餘令世勛系獄,候刑部擬罪。

五十七年,安南國王阮光平咨言:「國境嵩陵等七州毗連雲南開化,莫氏舊人黃公瓚父子據守,夤緣內附,籥懇代奏詳察。」使至龍州,龍州通判王撫棠以所請非分,發書駁還。世勛奏聞,上嘉撫棠,賜大緞獎之。光平又以黎維祁弟維祗結土酋農福縉為亂,遣兵剿滅,具表獻捷。表內並言:「維祗為亂,因維祁從人丁迓衡等為維祁通消息,請按治維祁罪。」世勛以光平所言臆度無憑,對揚失體,照會令將表文刪節,繕正奏聞。上已先得巡撫陳用敷奏,令諭光平具確據,並通消息者何人,送京師按治,命世勛遵前旨照會光平。五十八年,暹羅、安南貢使至,世勛遣吏伴送詣京師。上以所派職卑才庸,慮為外籓所輕,降旨申飭。潮州總兵託爾歡請觀,例具清字摺,硃批令來見。世勛奏委署總兵,譯漢文為俚語,上賜荷包愧之。

英吉利遣使入貢,請遣人留京居住,上不許,慮英吉利貢使還經廣東復多所陳乞,時已授長麟兩廣總督,命與世勛和衷商榷。尋奏英吉利貢使請在黃埔蓋房居住,已嚴行拒絕,並禁內地奸民指引勾結,上賜荷包獎之。五十九年,入覲,途次病作,至京師卒,賜祭葬。

畢沅,字纕蘅,江南鎮洋人。乾隆十八年舉人,授內閣中書,充軍機處章京。二十五年一甲一名進士,授修撰。再遷庶子。三十一年,授甘肅鞏秦階道。從總督明山出關勘屯田,調安肅道。擢陜西按察使。上東巡,覲行在,備言甘肅旱。諭治賑,並免逋賦四百萬。擢布政使,屢護巡撫。師征金川,遣沅督餉,軍無匱,授巡撫。河、洛、渭並漲,朝邑被水。治賑,全活甚眾。幕民墾興平、盩厔、扶風、武功荒地,得田八十餘頃。濬涇陽龍洞渠,溉民田。嘉峪關外鎮西、迪化士子赴鄉會試者,奏請給驛馬。置姬氏五經博士,奉祀文、武、成、康四王及周公陵墓。修華嶽廟暨漢、唐以來名跡,收碑碣儲學宮。屢署總督。四十一年,賜孔雀翎。四十四年,丁母憂,去官。四十五年,陜西巡撫缺員,諭:「沅在西安久,守制將一年。命往署理,非開在任守制例也。」

四十六年,甘肅撒拉爾回蘇四十三為亂,沅會西安軍伍彌泰、提督馬彪發兵討之。事平論功,賜一品頂帶。甘肅冒賑事發,御史錢灃劾沅瞻徇,降三品項戴。四十八年,復還原品,尋實授巡撫。四十九年,甘肅鹽茶回田五復亂,沅遣兵分道搜剿。上命大學士阿桂視師,沅治軍需及驛傳供億,屢得旨獎勵。

沅先后撫陜西十年,嘗奏:「足民之要,農田為上。關右大川,如涇、渭、灞、滻、灃、滈、潦、潏、河、洛、漆、沮、汧、汭諸水,流長源遠。若能就近疏引,築堰開渠,以時蓄洩,自無水旱之虞。古來雲中、北地、五原、上郡諸處畜牧,為天下饒,若酌籌閒款,市牛羊駝馬,為畀民試牧;俟有孳生,交還官項,餘則畀其人以為資本。耕作與畜牧相兼,實為邊土無窮之利。」議未行。

五十年,調河南巡撫。奏:「河北諸府患旱,各屬倉儲,蠲緩賑恤,所存無多,請留漕糧二十萬備賑。」既又請緩徵民欠錢糧,並展賑,上溫諭嘉之。命詣胎簪山求淮水真源,禦制淮源記以賜。五十一年,賜黃馬褂。授湖廣總督。伊陽盜秦國棟戕官,上責沅捕治未得,命仍回巡撫。五十三年,復授湖廣總督。江決荊州,發帑百萬治工。沅奏:「江自松滋下至荊州萬城堤,折而東北流,南逼窖金,荊水至無所宣洩。請築對岸楊林洲土壩、雞嘴石壩,逼溜南趨,刷洲沙無致雍遏。」又請修襄陽老龍堤、常德石櫃堤、潛江仙人堤,鑿四川、湖北大江險灘,便云南銅運。

五十九年,陜西安康、四川大寧邪教並起,稱傳自湖北,沅赴襄陽、鄖陽按治,降授山東巡撫。上以明年歸政,令督撫察民欠錢糧豁免,奏蠲山東積逋四百八十七萬、常平社倉米穀五十萬四千餘石。六十年,仍授湖廣總督。

湖南苗石三保等為亂,命赴荊州、常德督餉,以運輸周妥,賜孔雀翎。嘉慶元年,枝江民聶人傑等挾邪教為亂,破保康、夾鳳、竹山,圍襄陽,沅自辰州至枝江捕治。當陽又陷,復移駐荊州,上命解沅總督。旋克當陽,獲亂渠張正謨等,復命沅為總督如故,予二等輕車都尉世職。尋奏亂渠石三保、吳半生、吳八月等皆就獲,惟石柳鄧未獲;請撤各省兵,留二三萬分駐苗疆要隘。上諭曰:「撤兵朕所原,但平隴未克,石柳鄧未獲,豈能遽議及此?」尋獲石柳鄧。上命沅馳赴湖南鎮撫。疏言:「樊城為漢南一都會,請建磚城,以工代賑。」二年,請以提督移辰州,增設總兵駐花園汛。尋報疾作,手足不仁,賜活絡丸。旋卒,贈太子太保。四年,追論沅教匪初起失察貽誤,濫用軍需帑項,奪世職,籍其家。

沅以文學起,愛才下士,職事修舉;然不長於治軍,又易為屬吏所蔽,功名遂不終。

論曰:富勒渾、秉恬、鄂寶餫金川之軍,績當臨清之亂,圖思德招緬甸之使,步瀛御石峰堡之變,嗣曾肅臺灣之政,永清受安南之降,世勛屢卻暹羅、安南干請。若英吉利入貢,中外交涉,於此萌芽。川、楚教匪,沅當其始,久而後定。諸人者皆身膺疆寄,與兵事相表裏,功罪不同,賞罰或異;欲求其事始末,固不可略焉,故類而錄之。


\end{pinyinscope}