\article{列傳一百九十三}

\begin{pinyinscope}
駱秉章胡林翼

駱秉章,原名俊,以字行,改字籥門,廣東花縣人。道光十二年進士,選庶吉士,授編修。遷御史,稽察銀庫,卻陋規,嚴檢閱。吏不便其所為,欲齮齕去之,會發其奸,不得逞。歷給事中、鴻臚寺少卿、奉天府丞兼學政。二十三年,銀庫虧帑事發,坐失察,褫職,罰分賠。及讞定,宣宗知秉章獨持正無私,特旨以庶子用。尋丁母憂。服闋,補右庶子,先後命赴山東、河南、江蘇按事。詞臣奉使出異數,所治獄悉稱旨。二十八年,擢侍講學士。出為湖北按察使,遷貴州布政使,調雲南。三十年,擢湖南巡撫。

咸豐元年,廣西匪熾,詔湖廣總督程矞採赴湖南督辦防務,秉章及提督餘萬清副之。大學士賽尚阿督師過境,以供張薄,有嫌,密奏湖南吏治廢弛。二年,詔秉章開缺來京,而粵匪已由桂林北竄入湖南。矞採聞警,由衡州退長沙,尋復往駐。萬清守道州,被賊陷。江華、嘉禾、桂陽、郴州、攸縣相繼失,萬清逮治。秉章坐未能預防,革職留任。先議修長沙城,甫畢工,而賊由醴陵突犯長沙。秉章嬰城固守,悍賊蕭朝貴預詗城壞,故以輕軍來襲,未得逞,尋斃於砲。副將鄧紹良赴援最先至,入城任戰守。賊屢以地雷壞城,皆擊卻之。新授巡撫張亮基至,秉章奉旨暫留同守城。及賊首洪秀全大舉來攻,援軍向榮、和春、張國樑等亦並集,且守且戰,歷八十餘日。賊引去,陷岳州,趨湖北。賽尚阿、程矞採並坐失機罷譴。秉章以守城功,免議,召來京。尋命留湖北襄辦防守事宜,未至而武昌陷。三年春,官軍收復武昌,暫署湖北巡撫。詔赴徐州筦糧臺,未行,復署湖南巡撫,尋實授。

在籍侍郎曾國籓奉命治團練,始立湘軍,秉章力贊成之。又延湘陰舉人左宗棠襄理戎幕,廣羅英俊之士,練勇助剿,軍威漸振。先清境內,遣軍分路破江西賊於桂陽,破廣西賊於永明、零陵、江華,破廣東賊於興寧,又破江西賊於茶陵,而常寧、永興土匪皆平。賊由湖北進陷岳州,令王珍、曾國葆水陸截擊,敗之,岳州遂復。令貴州道員胡林翼率黔勇追賊逼界口。四年,總督吳文鎔師潰黃州,漢陽復陷。曾國籓水師成,進援湖北,前敵失利,岳州復陷。賊犯靖港及樟樹港,距長沙數十里,並陷寧鄉、湘潭。秉章調撫標兵益塔齊布軍,令偕楊岳斌、彭玉麟同援湘潭。國籓親率水師戰靖港,復失利。布政使徐有壬、按察使陶恩培請奏劾罷其軍。秉章曰:「曾公謀國之忠,不可以一時勝敗論也。」會次日塔齊布等大破賊於湘潭,復其城,靖港賊亦遁走,長沙獲安。賊繞西湖陷華容、龍陽、常德,令胡林翼專剿此路。塔齊布、羅澤南進規岳州、崇陽、通城,未幾,各城皆復,而武昌再陷。國籓整軍東征,餉械悉力資之無缺,十月,遂克武昌。湘軍之名自此顯。

五年,武昌三陷,胡林翼署巡撫,飛書告急。秉章令鮑超率水師先赴,彭玉麟募勇繼之。起楊岳斌於家,統其眾以固北路,而南路廣東、廣西群賊擾境,土匪紛起應之。令田興恕御東路,王珍剿南路,先清土匪,克東安,斬廣西賊首胡有祿。餘賊復擾永明、江華,擊走之。克桂陽、永興、茶陵、郴州、宜章,斃廣東賊首何祿,南路遂定。貴州苗犯晃州、沅州、麻陽,並擊走之。當武昌陷後,總督楊霈奏飭胡林翼渡江上扼漢川,以固荊襄。秉章上疏爭之,略曰:「楊霈始終堅執防賊北竄,然以現在形勢論之,江西、湖南尚稱完地。若使湖北水陸兩軍移駐漢川,長江千里,盡委之賊,其將置東南於不問乎?未解者一也。移駐漢川,祗能御上竄襄陽之路,其於荊州並無輕重。若賊水陸並進,荊州門戶,其孰當之?未解者二也。水陸兩軍相為依附,胡林翼既駐漢川,則水軍非退守監利,即移泊岳州,為湖南門戶計,尚未為失。然武漢門戶豈能度外置之乎?未解者三也。若謂賊眾兵單,不思廣濟失利之初,以總督萬餘之兵,不能當千餘之賊,乃退守黃州,未一日即退漢川,由此而德安,而隨州,今又退至棗陽。北竄者賊也,引之北竄者誰歟?未解者四也。扼賊北竄,必固荊襄,欲保荊襄,必守武漢,此一定之局。漢陽未復,不能繞至漢川,況武漢均為賊屯,胡林翼縱至漢川,以孤軍駐四面皆賊之地,又能為荊襄門戶計乎?未解者五也。」霈之專防北竄,原出迎合上意。疏入,詔斥所詆霈者過當。然上意開悟,未久罷霈,以官文代之,與胡林翼合規武漢。秉章悉力資給林翼軍,如所以助曾國籓者。洎林翼與羅澤南破石達開於咸寧,達開折入江西,連陷瑞州、臨江,而吉安、撫州、建昌屬城多被擾。

國籓自上年九江之挫,久留南昌,孤軍難進展。秉章至是銳意東援,令江忠濟出通城以固岳州,令劉長佑、蕭啟江率軍分路入江西。六年,劉長佑等連克萍鄉、萬載,進攻袁州。江忠濟戰歿通城,以王珍代之,連克通城、崇陽、蒲圻、通山諸縣。至冬,長佑克袁州、分宜、新喻,趙煥聯自茶陵收永寧,餘星元自酃縣收永新、蓮花。初議規江西分三路,北路出瑞州,中路出袁州,南路出吉安。劉長佑袁州一路兵逾九千,餉難再籌。至是始令周鳳山、曾國荃各募勇二千,合趨吉安。詔嘉秉章不分畛域,越境殄寇,賜花翎。

七年,武漢既復,下游無警,湘軍乃四出。以蔣益澧率永州軍援廣西,以王珍軍增援江西,以兆琛等軍援貴州,需餉益鉅。湖南自軍興停漕運,米賤,而徵折猶沿舊價,民困賦絀。秉章減浮折,覈中飽,民減納而賦增。仿揚州例,抽收鹽貨釐金,歲入百數十萬,給軍無缺。王珍戰江西,屢破悍寇,克樂安,尋卒於軍,以張運蘭及珍弟開化分統其眾。劉長佑攻臨江,至十二月克之。八年,京察敘功,加頭品頂戴。劉長佑以疾歸,以劉坤一代領其軍。進規撫州、建昌,先後克復。八月,諸軍齊集,克吉安。石達開敗竄浙江,江西略定。秉章以兵合不易,應乘勝進取。疏請起曾國籓督師援浙,留蕭啟江、張運蘭兩軍隨征,餘軍盡撤。蓋自五年援江西,糜湖南餉凡二百六十萬,協濟之數不預焉。

石達開由浙入閩、粵,徘徊五嶺之上。九年春,復由江西入湖南。秉章調魏喻義、陳士傑扼巋河,起劉長佑於家,令與劉坤一募勇四萬備迎擊。調蕭啟江、張運蘭於江西,調田興恕於貴州,未集而賊至,陷桂陽、宜章、興寧,窺衡州,為巋河之軍所扼,回竄嘉禾、新田、臨武、寧遠。達開大隊竄永興,以據上游。劉長佑出祁陽,與之相持。回犯東安、新寧,劉坤一再挫之,乃趨寶慶,眾號三十萬,多烏合。秉章下免死令,散數萬人。時趙煥聯、田興恕等軍先至,營城外。賊營環二百里,包諸軍於中。胡林翼遣李續宜率軍赴援,秉章令劉長佑、劉岳昭、何紹彩分三路進。六月,戰寶慶城下,內外夾擊。賊人眾乏食,再戰再敗,遂東竄。蕭啟江軍遇於永州,又擊敗之。乃由全州竄廣西,啟江尾追,劉長佑繼進,敗之於大榕口,又敗之桂林,賊竄慶遠。秉章令長佑留鎮廣西,田興恕回貴州,蕭啟江出沅江,兼顧川、黔。時廣東賊又擾邊境,令張運蘭、黃淳熙分擊於江華、宜章,並殲之。

十年,命赴四川督辦軍務。時左宗棠已奉命募勇援浙,聘湘鄉劉蓉贊軍事。湘軍名將多從曾國籓、胡林翼,惟劉岳昭、黃淳熙在湖南。調兩軍隨行,受代將發,石達開復由廣東犯湖南境,吏民乞留。遣岳昭、淳熙會剿,賊尋引去。十一年正月,始啟行,抵宜昌,聞陳玉成犯湖北,分遣岳昭赴援,自率五千人入川。

四川之亂,始於咸豐九年。滇匪藍大順又名朝柱,李短搭又名永和。結黨私販鴉片,其黨被捕,聚眾陷宜賓,攻敘州,擾嘉定,眾號十餘萬,群盜遂四起。總督有鳳、曾望顏等不能制,徵兵湖南,先遣蕭啟江一軍赴之。啟江尋病歿,詔曾國籓赴川督師,中止未行,成都將軍崇實署總督。秉章奉命後,慮客軍易遭齮齕,猶觀望。崇實馳書促行,開誠迎候,發夔關稅以給軍,軍至,乃出望外。時賊首李永和、卯得興踞青神,藍朝柱圍綿州,張第才、何國樑圍順慶,蹂躪四十餘縣,將逼成都。秉章至萬縣,即令黃淳熙援順慶,戰於定遠,陣斬何國樑,賊大敗。追至潼川二郎場,中伏,淳熙陣亡,然賊驚湘軍勇銳,引去。秉章由順慶進駐潼川,令胡中和、蕭慶、何勝必率蕭啟江舊部,曾傳理代領黃淳熙之眾,劉德謙率親軍,唐友耕率川軍,合萬九千人,援綿州,別以他軍綴青神,分扼東北。會穆宗即位,擢授秉章四川總督。八月,師會綿州城下,連破賊十餘壘,賊敗退,渡涪水屯守。官軍作五浮橋以濟,又擊敗之。賊遁走,由什邡、崇慶趨丹棱,秉章始入成都。

蒞任,奏劾布政使祥奎、中軍副將張定川不職,罷之。薦劉蓉,詔超擢署布政使。軍事吏治,振刷一新,於是分剿諸賊,急攻藍、李二股。令唐友耕扼眉州洪堰,斷青神之援,胡中和等諸軍圍丹棱,作長壕木城,節節進逼。賊棄城走,追斃藍朝鼎於陣。餘賊分路逃散,為民團汛兵截殺幾盡。藍朝柱率二百人遁入山,尋出合諸匪陷新寧,復為官軍擊散。其後陜西盩厔匪潰走興安,為民團所獲,有自稱為藍大順及弟三順至九順,並戮之。李永和見丹棱已克,亦遁走,分軍追擊,圍之於鐵山。同治元年,京察,詔嘉秉章殄寇迅速,整頓地方,加太子少保。尋克青神,李永和、卯得興由鐵山遁走,追至宜賓,擒之。道員張由庚克新寧,賊分竄,張第才遁陜西,曹燦章入老林。總兵周達武解涪州圍,追擒周紹湧於大竹,又擒郭刀刀於巴州。周毚毚由雲南入岳池、合州、新寧,張由庚擊走之,諸城皆復。至冬,川南北一律肅清,詔嘉調度有方,予優敘。

石達開見川中兵事方殷,屢由黔、楚窺伺來犯。是年春,陷石柱,撲涪州,為劉岳昭軍所阻,竄黔境。尋又入敘永,攻江安,陷長寧,分擾珙、高、慶符,劉岳昭、曾傳理等擊敗之。退滇境,分竄筠連、高縣,官軍扼金沙江以守。賊謀三路入川,秉章調諸將及土司兵分防。二年正月,賴裕新自寧遠犯冕寧,至越巂,為工⼙部土司嶺承恩擊斃。餘賊散擾川西十餘縣,多為官軍民團截殺,盡殲於平武山谷中。三月,石達開渡金沙江,為唐友耕等軍所扼,由小徑趨土司紫打地。大渡河水漲,官軍伺半濟擊之,退撲松林、小河,又為土司王應元所扼。嶺承恩夜襲破馬鞍山賊營,斷其糧道。復連撲兩河,皆不得渡,糧盡,殺馬採樹葉而食。唐友耕等漢、土官兵合擊,焚其巢,墮巖落水無數。餘七八千人奔老鴉漩,復為土兵所阻。達開率一子及其黨三人乞降,解散四千人,餘黨盡誅之。五月,檻送達開至成都,磔於市。捷聞,詔深嘉之,加太子太保,將士獎擢有差。李福猷為達開死黨,初約由黔入川。令劉岳昭與黔軍合剿,尋於黔境就殲。達開餘孽遂盡。

粵匪擾陜西,圍漢中,秉章令道員易佩紳率軍解其圍,張由庚駐防川境。至是復令蕭慶高、何勝必赴剿。詔擢劉蓉為陜西巡撫,督諸軍。秉章病目請告,命力疾視事。三年,江寧克復,詔錄前後功,予一等輕車都尉世職,賜雙眼花翎。四年,陜西粵匪為諸軍擊敗,竄甘肅階州,令周達武會剿平之。回剿南坪番匪,匪首歐利哇降。又剿馬邊,擒匪首宋士傑,邊境悉平。令劉岳昭援黔,由綏陽抵遵義,道路始通,後由黔規滇,皆秉章遺策也。六年夏,疾愈視事,命以四川總督協辦大學士。十一月,卒於官,優詔賜恤,稱其「公忠誠亮,清正勤明」,贈太子太傅,入祀賢良祠,四川、湖南建專祠。賜其子天保郎中、天詒舉人,諸孫並賜官,謚文忠。

秉章晚年愈負重望,朝廷要政多諮決,西南軍事胥倚之。所論薦人才,悉被任用,著勛名。川民感其削平寇亂,出於水火,及其歿,巷哭罷市。遺愛之深,世與漢諸葛亮、唐韋皋並稱云。

胡林翼,字潤之,湖南益陽人。父達源,嘉慶二十四年一甲三名進士,官至少詹事,學宗宋儒。林翼少時,即授以性理諸書,而林翼負才不羈,娶總督陶澍女,習聞緒論,有經世志。

道光十六年,成進士,選庶吉士,授編修。二十年,充江南副考官,坐失察正考官文慶攜舉人熊少牧入闈,降一級調用。丁父憂,服闋,捐納內閣中書,改貴州知府。署安順、鎮遠,皆盜藪,用明戚繼光法練勇士,搜捕林箐,身與同甘苦。屢擒劇盜,靖苗氛,以功賜花翎。又因防剿新寧匪李沅發,以道員用。總督吳文鎔、巡撫喬用遷並薦堪大用。咸豐元年,補黎平,實行保甲團練,千五百餘寨,建碉樓四百餘座,嚴扼要隘,儲穀備城守。地鄰湘、桂,匪戢而民安。三年,剿甕安榔匪,誅其魁。湖南巡撫張亮基、駱秉章兩次奏調,以貴州留不行。御史王發桂疏薦林翼剿匪成效,詔赴湖北委用。

四年,擢貴東道,率黔勇千人行次通城,而總督吳文鎔戰歿黃州,遂進援武昌。賊尋犯湖南,駱秉章調林翼回防,平安化土匪,擢四川按察使,尋調湖北。曾國籓既克武昌,檄林翼與羅澤南會攻九江,屯湖口,破賊梅家洲。五年春,擢湖北布政使。總督楊霈師潰黃梅,林翼率所部回援武昌,別以副將王國才一軍隸之,未至,漢陽陷,會攻不克,屯沌口。武昌復陷,潛師渡江規武昌,為賊所圍,兵少食盡,退金口。詔林翼署理湖北巡撫,楊霈奏令上扼漢川。林翼疏陳形勢,宜急攻武漢,方能內固荊襄,上俞之。時武、漢、黃、德四郡皆為賊踞,後路崇陽、通城多伏莽,公私赤立,兵餉皆絀。林翼馳書四出乞貸,發家穀給軍。添募兵勇,兼顧南北兩路,凡數十戰,時有克捷,亦屢瀕於危。七月,攻克漢口鎮,奪大別山賊卡。未幾,援賊由漢川至,焚漢口。崇、通匪勾結武昌城賊,撲金口大營。詔念林翼素善用兵,勉以重整散卒。尋退奓山,餉絕兵潰,下部議處。林翼移營大軍山,收集潰兵,駐新堤、嘉魚。水陸合萬人,半出新募,賊至常數萬,軍中奪氣。林翼鎮靜相持,以忠義激勵將士,始漸定。奏調羅澤南由江西來援,連克通城、崇陽,林翼自往迎之於蒲圻。合破援賊韋俊、石達開於咸寧,復其城。乘勝進攻武昌,自率所部普承堯、唐訓方軍由中路,羅澤南當西路,楊岳斌以水師會金口,總督官文亦令都興阿率騎兵駐北岸。林翼和輯諸將,軍勢遂日振,屢戰皆捷。

六年三月,羅澤南急攻城,傷於砲,驟卒。以李續賓代領其軍,攻戰不少輟。石達開自咸寧敗後,竄江西,連陷數郡。曾國籓屢調羅澤南回援,不克往。林翼分遣劉騰鴻、普承堯兩軍赴之。詔以武漢久不克,督戰急。林翼疏陳,略曰:「臣頓兵城下五月餘矣。血肉之軀,日當砲石,傷亡水陸士卒三千餘,喪將領羅澤南、周得魁百餘人,李續賓中丸墮馬者數矣。夫兵易募而將難求,臣觀前史,李左車告韓信,以頓兵城下,情見事絀為戒。戰易攻難,自昔已然。故臣自四月後乃禁仰攻,分兵咸、蒲以取義寧,四戰皆捷。分水師以清下游,直達九江。臣自率兵五千扼武昌南路,李續賓率六千三百扼洪山東,分剿北路。水師六營下駐沙口。賊由九江、興國分路來援,臣豫撥三千餘人戰於百里之外。微臣之志,誓與兵事相終始。萬一變生意外,決不敢退怯茍且,自取羞辱。」文宗覽奏,特慰勉之。

五月,賊於武昌城外豹子澥等處增壘掘壕,林翼抽調諸軍擊之,遂於要隘掘壕困賊。賊屢撲,皆擊退。諜知九江賊古隆賢來援,已至樊口,先遣黨數千進踞葛店。令蔣益澧率精銳迎擊,戰於葛店,大破賊,焚其舟。追至樊口,楊載福水師亦至,合擊,斃賊數千。攻克武昌縣城,遂渡江攻黃州。而石達開由江西竄江寧,復糾眾上犯,分數路。七月,急調黃州軍回援。賊由金牛趨葛店,古隆賢亦起應之。林翼督水陸軍分御,連戰於油坊嶺、魯家港、姚家嶺、窯灣、沙子嶺、小龜山,旬日內二十餘捷,擒斬無算,解散脅從萬餘,追奔百餘里,至華容,賊悉遁。九月,楊岳斌追賊至蘄州,焚其舟,直抵田家鎮。賊援既絕,添募陸勇五千,水師六營,為長圖計。十一月,咨會官文克期大舉。楊岳斌斷攔江鐵鎖,焚賊船盡。賊傾城出撲,鏖戰三時,大敗狂奔,諸軍逐之,遂復武昌。擒賊酋古文新等,駢誅數百人,生降四千。同日官文亦克漢陽。詔實授林翼湖北巡撫,加頭品頂戴。遂分兵收復武昌縣、黃州府及興國、大冶、蘄水、蘄州、黃梅。令李續賓乘勝規九江,都興阿、楊岳斌、鮑超屯小池口,自駐武昌籌全局。

上疏論軍事吏治,略曰:「湖北軍務不飭已久,無論賊之多寡強弱,聞警先驚,接仗即潰。上下相蒙,恬不知恥。誤於使貪使詐,而實為貪詐所使。川、楚、河南勇目,招合無賴投效,以一報十,冒領口糧。交綏即敗,又顧之他。帑項至艱,徒飽無賴欲壑。遣散不得其方,又相聚為盜。近年湖北募勇之大患,綠營則怯懦若性,正額虛浮,軍政營制,蕩然無存。此為兵事急應整頓之要。自古用武之地,荊襄為南北關鍵,武漢為荊襄咽喉。武漢有警,則鄰疆胥震。四年之中,武昌三陷,漢陽四陷。東南數省,受害惟武漢為甚。夫善鬥者必扼其吭,善兵者必審其勢。今於武漢設重鎮,則水陸東征之師,恃為根本軍火米糧委輸不絕,傷痍疾病休養得所。平吳之策,必先保鄂,明矣。保鄂必先固漢陽。湖北之失,在漢陽無備。下游小挫,賊遂長驅直入。應請於武漢設陸師八千,水師二千,日夜訓練。平時有藜藿不採之威,臨事有千里折沖之勢。且東征之師,孤軍下剿,苦戰必傷,久役必疲。傷病之人,留於軍中,不但誤戰,亦且誤餉。若以武漢之防兵更番迭代,則士氣常新,軍行必利。此武漢宜急設防練之要。湖北莠民從賊者多,兵勇搜捕,徒滋擾害。惟有保甲清釐,族戶綑獻,分別斬釋。然牧令不得其人,則法不能行。官吏之舉動,為士民所趨向;紳士之舉動,又為愚民所趨向。未有不養士而能致民,不察吏而能安民者。五年大熟,州縣乃或報災,六年大饑,州縣轉或徵賦。以豐為歉,是病國計;以歉為豐,是害民生,而終害於國計。歉歲官吏私收蠲緩,實惠不及於民。有所謂挖徵、急公等名目,無一非蠹國病民。凡下與上交接之事,諉之幕友;官與民交接之事,諉之門丁。詞訟案牘,病在積壓;盜賊奸宄,弊在因循。州縣之小事,即百姓之大事,今日之小賊,即異日之大賊。厝火積薪,隱憂方大。又如捐輸則有踩堂、贄見之費,牙帖則有勒索之費,釐金則有私設之費。臣受事以來,迭次特參,在國自有刑章,在臣甘為怨府。惟思劾貪非難,求才為難。前者劾去,後者踵事,而巧避其名,弊將不可勝言。臣愚以為必嚴禁官場應酬陋習,與群吏更始,崇尚敦樸,屏退浮華。行之數年,庶可改觀。目下州縣懸缺待人,請敕下部臣,暫勿拘臣文法資格。此吏治急應整飭之要。武漢甫經收復,人或以為已治已安,臣竊憂之。如以為治安,則前收復已二次矣。況江西七府俱淪於賊,旁軼橫出,不僅九江、安慶為足慮也。未收復之前,事勢極難,文武尚有懼心;收復之後,布置尚易,而特恐文武均萌肆志。外省粉飾之習,久在聖明洞鑒。不揣愚昧,用以直陳。」疏入,上嘉納焉。於是裁浮勇,練新軍,蠲四十六州縣田賦以蘇民困。設清查局,稽核全省倉庫盈虛之數;設節義局,表彰死難官紳士女;設軍需局,以備東征餉械。嚴課吏治,糾劾文武數十人,推廉尚能,手書戒勉將吏如子弟。初,將吏頗構督、撫異同,下令曰:「敢再言北岸兵事吏事長短者,以造言論罪。」官文亦開誠相與,無掣肘。軍政吏治,皆林翼主稿,林翼推美任過,督撫大和。湖北振興,實基於此。襄陽土匪猖獗,擾及河南境,令唐訓方等剿之。

七年春,擒匪首高先二等。陳玉成由皖北上犯,諸軍不能御。林翼赴黃州督師,賊眾十餘萬環踞巴河東。會水漲,林翼令毀三臺河石橋,扼河而守。潛師出回龍山,遏賊上竄。調李續宜率湖勇馳至,督諸軍合擊於孫家嘴、馬家河、月山,賊大敗遁走。都興阿、李續賓亦連破賊於黃梅、宿松,楚北肅清。遂視師九江,定合圍方略而還。八年四月,李續賓等攻九江,克之,磔賊首林啟榮。詔嘉林翼調度有方,加太子少保。林翼乃急規安慶,楊岳斌率水師出九江,都興阿出宿松、望江,逼安慶為圍師。李續賓規復太湖、潛山、桐城,與都軍為犄角。五月,丁母憂,詔予假百日治喪,假滿仍署巡撫。七月,廬州陷,李續賓輕軍赴援,戰歿三河。林翼方奉母柩回籍,詔急起視師,林翼聞命,痛哭起行,逕次黃州,軍心始定。

九年,進屯上巴河,與李續宜整飭部伍,日夜訓練,謀大舉。會石達開由江西犯湖南,圍寶慶。林翼令李續宜率所部赴援,舒保馬隊助之,又以水師分扼河道,寶慶圍得解,於是與曾國籓合力圖復安徽。國籓循江而下為第一路,多隆阿、鮑超攻取潛山、太湖為第二路,林翼自出英山、霍山為第三路,李續宜由松子關出商城、固始為第四路。十月,由黃州移營英山。陳玉成在賊中最狡悍,見太湖圍急,糾合捻匪張洛行、龔瞎子眾數十萬來援。林翼集諸軍精銳全力備戰,欲一鼓殲之。與曾國籓部署諸將,指揮戰略。謀前敵總統,以多隆阿謀勇兼優,而鮑超素不相下,手書勸勉,十數往復,始定議。又備意外,令金國琛、餘際昌以八千人出潛山天堂拊賊背。十二月,賊至,鮑超營小池驛,當其沖,賊聚攻之。多隆阿慮分兵掣全勢,置不救,調唐訓方往助。事且急,金國琛等由山中鼓行而出,賊乃奪氣。十年正月,多隆阿攻羅山沖為西路,鮑超出小池驛為東路,硃品隆、蔣凝學、唐訓方等合擊,金國琛等亦同時並進,大破賊,殲斃先後二萬餘,遂克太湖城,潛山亦復。是役為僅見之大捷,安慶之勢遂孤。

既而江南大軍潰,蘇、常盡陷,曾國籓授兩江總督,督師。林翼為畫分路大舉之策,國籓不盡用,率鮑超等次祁門,為規復江南計,以其弟國荃圍安慶。林翼令多隆阿圍桐城,李續宜屯青草塥,為兩軍援,都興阿別出師江北,分兵濟餉,林翼悉任之。十月,多隆阿、李續宜大破賊於桐城掛車河。林翼進駐太湖,度賊援安慶不利,必深入湖北腹地以分我軍勢。令餘際昌屯霍山樂兒嶺,成大吉屯羅田松子關,戒賊至勿浪戰,堅守待援。十一年春,賊果合捻匪西犯,成大吉破之松子關,殲捻渠龔瞎子。霍山守者違節度,為賊所敗,遂進陷黃州、德安、孝感、隨州,林翼令李續宜回援。賊復分股回略蘄、黃,趨安慶,約城賊夾擊。檄成大吉下援,鮑超亦由南岸至,破賊於集賢關,擒斬數千,磔其渠劉瑲林。多隆阿亦破援賊於桐城,賊計不得逞,城中糧將盡,勢益慼。南岸之賊復由江西犯興國、大冶,南及崇、通,武漢震動。林翼方病咯血,自率師回援,而圍攻安慶益急。及抵湖北,賊已聞風遁。八月朔,遂克安慶。曾國籓推林翼為首功,詔加太子太保,予騎都尉世職。桐城、廬江、舒城以次復,黃州、德安之賊先後擒斬,楚境悉平。

林翼久病,聞文宗崩於熱河行在,大慟嘔血,八月,卒。詔贈總督,祀賢良祠,湖北、湖南並建專祠,賜其子子勛舉人,謚文忠。同治元年,復詔:「林翼未竟全功,遽就溘逝,跡其功勛卓越,名播寰區,至今江、鄂士民稱頌。命於原籍家祠賜祭一壇。」洎江南平,加予一等輕車都尉世職,子子勛襲,後並兩世職為男爵。光緒中,以孫祖廕襲,官郵傳部參議。

林翼貌英偉,目巖巖,威棱懾人。事至立斷,無留難。尤長綜覈,釐正湖北漕糧積弊,以部定漕折為率,因地量加輕重,民歲減錢百餘萬緡,歲增帑四十餘萬兩,提存節省銀亦三十餘萬兩。兩湖自淮鹽阻絕,率食川鹽,於宜昌、沙市、武穴、老河口設局徵稅,視舊課增至倍蓰。時東南各省皆抽釐助餉,惟湖北多用士人司榷,覈實無弊。其治軍務明紀律,手訂營制,留意將才。嘗曰:「兵之囂者無不罷,將之貪者無不怯;觀將知兵,觀兵知將。為統將必明大體,知進退緩急機宜;其次知陣法,臨敵決勝;又其次勇敢:此大小之分也。」馭將以誠,因材而造就之,多以功名顯。察吏嚴而不沒一善,手書褒美,受者榮於薦剡,故文武皆樂為之用。士有志節才名不樂仕進者,千里招致,於武昌立寶善堂居之,以示坊表。嘗曰:「國之需才,猶魚之需水,鳥之需林,人之需氣,草木之需土。得之則生,不得則死。才者無求於天下,天下當自求之。」薦舉不盡相識,無一失人。曾國籓稱其薦賢滿天下,非虛語。嘗自以聞道晚,刻自繩檢,欿然常若不足。家有田數百畝,初筮仕,誓先墓,不以官俸自益。父著弟子箴言行世,承其志為箴言書院,教人務實學。病革,曰:「吾死,諸君賻吾,惟修書院,無贍吾家。」所著讀史兵略、奏議、書牘,皆經世精言。

論曰:駱秉章休休有容,取人為善。胡林翼綜覈名實,幹濟冠時。論其治事之寬嚴疏密若不相侔,而皆以長駕遠馭,驅策群材,用能丕樹偉績。所蒞者千里方圻,規畫動關軍事全局。使無其人,則曾國籓、左宗棠諸人失所匡扶憑藉,其成功且較難。緬懷中興之業,二人所關系者豈不鉅哉?


\end{pinyinscope}