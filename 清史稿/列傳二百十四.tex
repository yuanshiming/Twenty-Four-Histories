\article{列傳二百十四}

\begin{pinyinscope}
王懿德曾望顏覺羅耆齡福濟翁同書嚴樹森

王懿德,字紹甫,河南祥符人。道光三年進士,授禮部主事,再遷郎中。出為湖北襄陽知府,擢山東兗沂曹濟道。歷山東鹽運使、浙江按察使,調山東。三十年,擢陜西布政使。咸豐元年,護巡撫,奏請豁免積年民欠常平倉糧八萬餘石,擢福建巡撫。

二年,奏言:「漢患錢乏,造幣贍國;宋有交引、錢引、交鈔;元、明制鈔法,或直千文、五百不等。我朝準歲入為出,因民利而利,帑項夙充,奚庸過慮?自海防多事,銷費漸增,粵西軍務,河工撥款,不下千數百萬,目前已艱,善後何術?捐輸雖殷,僅同勺水。督催稍迫,且礙閭閻。與其籌畫多銀,不若改行鈔引。歷考畿輔、山左以及關東,多用錢票,即福建各屬,銀錢番票參互行使,便於攜取,視同現金,商民亦操紙幣信用。況天下之主,國庫之重,飭造寶鈔,尤易流轉。惟鈔式宜簡,一兩為率,頒發籓庫,通喻四民,準完丁糧關稅,自無窒滯。或疑庫銀溢出,悉成鈔引,銀日以少,鈔日以賤。豈知朝廷不蓄為寶,以天下之財供天下之用,能收能發,自能左右逢源也。」疏入,諭軍機大臣同戶部議行。兼署閩浙總督。三年,奏福建匪徒糾結滋擾,請寬地方官失察處分,俾獲盜自贖,允之。

時會匪四起,突入海澄縣劫獄戕官,又掠同安、安溪,遣兵會剿。漳州猝為匪陷,鎮、道皆遇害。游擊饒廷選方率兵他出,聞警回援。近城鄉民及城中紳士密約,啟廷選入,擒匪首謝厚等,殲匪數百,復其城。延平亦被匪攻,副將李壽春擊走之。大田、德化有匪闌入,紳士率鄉團殺賊數百。永春為匪所踞,游擊恩霈等會勇破賊,擒其渠,餘黨遁走,被詔嘉獎。臺灣南路亦有匪擾,懿德奏陳防剿情形,諭曰:「福建紳練素諳大義。前同安縣義民殺賊,泉州在籍副將呂大升等自原募勇渡臺,是其明驗。務當激揚士氣,滅此群醜。」尋以海澄、同安、廈門、安溪、仙游相繼陷,疏請治罪,下吏議。令參將李煌、都司顧飛熊破賊,尤溪縣城失而旋復。水師提督施得高、金門鎮總兵孫鼎鼇擊賊於金門,破之。廈門、仙游皆復。四年,上游以次定,賊首林俊尚焚斃,實授閩浙總督。

戶部議限制行鈔,奏言:「鈔之能行,不在於發而在於收。內自部庫以及各關稅務,外則丁耗錢糧、鹽典契紙各稅,果能悉收鈔票,不限成數,且示以非鈔不用,則百姓爭相買鈔。有銀之家,以鈔輕而易藏;納課之氓,以率定而無損;貿遷之商,以利運而省費。部臣見未及此,惟恐解鈔而不解銀,故限以成數。夫以為無用,則鈔、銀均非可食可衣;以為有用,則鈔、銀不能畸輕畸重。今於領鈔之時,區以一省,由部知照,方能行用。己不自信,人豈可強?徒開藉端漁利之門。請飭部臣及各省督撫,以此發即以此收,無論各項度支,示天下非鈔不用。新收買鈔銀兩,積於部庫、籓庫,以為母金。行鈔不分畛域,則銀日豐而本源厚。」疏入,下部議,格不行。

五年,因病請改京職,不許。七年,粵匪自江西竄入境,陷光澤、汀州,尋先後克復。遣總兵饒廷選進援浙江、江西。八年,京察,詔以懿德攘外安內,布置咸宜,予議敘。粵匪復自江西竄陷浦城、松溪、政和等縣,邵武、光澤、連城亦被賊擾。周天培軍赴援,賊復回竄江西,諸城皆復。十年,以病乞罷。十一年,卒,謚靖毅。

曾望顏,字瞻孔,廣東香山人。道光二年進士,選庶吉士,授編修,遷御史。十五年,條奏整飭科場凡十四事,皆如所請行。遷給事中,再遷光祿寺少卿。上以望顏遇事敢言,褒勉之,轉太常寺少卿。十六年,擢順天府尹。二十年,出為福建布政使。二十三年,戶部銀庫虧帑事發,望顏嘗以御史察庫,未糾發,坐奪官分償。旋授主事。咸豐三年,命以五品京堂候補,補通政司參議。六年,復授順天府尹,擢陜西巡撫。七年,粵匪自湖北竹山擾陜西平利,望顏遣游擊常有等會湖北軍克竹山。賊竄均州武當山,又遣總兵龍澤厚會湖北軍進剿,殲賊殆盡。八年,粵匪入雞頭關,侵商南,遣兵擊走之。

九年,署四川總督。粵匪入四川,攻敘州,尋引去。滇匪藍朝柱、李永和倡亂,與敘州土匪勾結肆擾。望顏遣兵進攻,斬賊目李祖資等。十年,遣提督孔廣順等攻大巖尖山賊寨,獲其渠王帶周。滇匪攻犍為,自箭板場竄至河口,將縛筏以渡,提督皁升督兵水陸夾擊,走之。望顏又慮賊渡河犯嘉定,遣總兵占泰等截擊。賊據觀音場,師自黃閣寺進攻,戰於羅城鋪,敗之。賊竄踞貢井、天池寺諸地,為壘數十,飭占泰等剿之。黔匪李志高等據長阡壩諸寨,遣兵攻毀長阡壩。總兵虎嵩林自程家場進攻貢井,又遣兵攻濯水賊,獲其渠賀世愚等。諸路雖有斬獲,而滇匪勢日熾,藍朝柱擾青神、敘州,李永和攻嘉定,省城戒嚴。詔斥望顏不能制賊,下吏議。

給事中李培祜疏劾任性妄為,濫保浮銷,縱子干預。命陜甘總督樂斌偕署巡撫譚廷襄按治。覆奏望顏尚無贓私,惟舉劾屬吏多粗率謬誤,不能約束子弟僕隸。部議褫職,命暫留署任。復為知府翁祖烈所訐,下將軍崇實按治,辭復連子捷魁及其僕,乃命解任,仍留四川。十一年,命回籍。同治元年,召詣京師,以四品京堂候補。五年,補內閣侍讀學士。九年,卒。

覺羅耆齡,字九峰,正黃旗人。初授工部筆帖式,中式道光十七年舉人,升刑部主事,累遷郎中。出為江西廣信知府,調南安。歷署建昌、撫州、吉安、袁州諸府。咸豐三年,調赴省城筦官團局。粵匪攻南昌,耆齡佐守御,賜花翎。尋補贛州知府。五年,擢吉南贛寧道。賊竄義寧,耆齡率兵赴援。六年,擢布政使,命駐防饒州,偕畢金科等分屯扼守。賊三路來犯,金科乘勝追賊,而贛軍營壘被襲,城遂陷。旋即合攻破賊,復之。奉檄移軍南昌。侍郎曾國籓奏:「耆齡在饒州聯絡鄉團,屏障東北。今九江重兵已盡赴省城,耆齡宜仍駐饒州,毋庸移調。」時江西司道多統軍,曾國籓及學政廉兆綸皆以耆齡為善,而訾議巡撫文俊。七年,詔罷文俊,擢耆齡為巡撫。

江西郡縣半淪於賊,存者惟南昌、廣信、饒州、贛州數郡,戰事多倚湘軍。未幾,曾國籓偕弟國荃以奔喪歸湖南。圍吉安久不下,國荃去後,軍無所統,益疲。耆齡奏起國荃仍督吉安軍,乃復振。七月,劉騰鴻克瑞州。十二月,劉長佑克臨江。八年四月,李續賓克九江,蕭啟江、劉坤一克撫州。八月,曾國荃克吉安。詔起曾國籓督師規浙江,於九月至南昌。國籓前於五年初至江西,兵餉俱困,地方官吏狎侮掣肘,事多艱阻。至是,耆齡奉令惟謹,主客大和,軍事日有起色。九年三月,克南安。六月,克景德鎮。江西全境暫告肅清。九月,調廣東巡撫。粵匪翟明開自南雄攻江西安遠,耆齡遣兵越境解圍。十一年,賊自安遠敗竄平遠,入福建,陷武平,耆齡分兵收復。

同治元年,命督軍入福建援浙江,擢閩浙總督。粵匪陷處州,耆齡遣總兵秦如虎等分道進攻,直偪城下。賊竄縉雲,遂克處州,進收縉雲,再進復奉化。二年,復進克湯溪、永康、武義、龍游、蘭谿諸縣,及金華府城,浙東略定。調福州將軍。尋卒,賜恤,謚恪慎。

福濟,字元修,必祿氏,滿洲鑲白旗人。道光十三年進士,選庶吉士,授編修。擢侍講,四遷少詹事,大考二等,復三遷兵部侍郎,兼鑲白旗蒙古副都統、總管內務府大臣。調工部,復調吏部,兼右翼總兵。二十八年,命偕右庶子駱秉章往河南、江蘇、山東按事。歸德知府胡希周貪劣,鞫實,論如律。河南賈魯河工糜費虛報,工竣河復淤,巡撫鄂順安以下皆坐譴。蘇州知府鍾殿選等濫刑諱盜,鞫實,論如律。又按山東鹽運使韋德成訐巡撫崇恩,勒令開缺,請交刑部逮治。復調戶部。二十九年,授正白旗護軍統領。命偕刑部侍郎陳孚恩按山西巡撫王兆琛贓污,兆琛坐譴。三十年,轉左翼總兵。醫士薛執中坐妖言得罪,事牽福濟,奪官。尋予四品頂戴,署山西按察使,授山東按察使。咸豐二年,授奉天府尹,擢南河河道總督。三年,調漕運總督,命暫行督辦淮北鹽務。

時粵匪踞江寧,擾江北,福濟會琦善敗賊揚州,授安徽巡撫。福濟調漕河標兵六百自臨淮關赴廬州,疏請飭琦善撥精兵二千扼關山、澗溪,防賊北竄;又請仍兼督淮北鹽課,藉濟安徽軍餉:皆允之。四年,至廬州,土匪陷六安,下部議處。福濟奏言:「抵廬後,統計調兵約二萬餘,月餉不下十五六萬。請飭浙、魯、秦、晉各撫臣協濟。」復請以前江南河道總督潘錫恩、安徽學政孫銘恩會辦徽州、寧國、廣德三府州防剿,俱從之。提督和春以欽差大臣督辦軍務,福濟與會師克六安,收英山、霍山。五年十月,克廬州,加太子少保、頭品頂戴。於是廬江、巢縣、無為相繼克復,被優敘,賜御用棉袍、翎管、搬指、荷包。十一月,移軍桐城。

七年,無為、廬州附近各縣復為賊陷,桐城被圍,屢擊卻之。二月,賊大至,福濟率兵潰圍出,還駐廬州。詔斥調度無方,下部議處。未幾,六安復陷,福濟因病請開巡撫缺,專辦軍務,不許。時安徽本省無兵,軍務實主於和春。賊踞安慶,皖南數郡懸隔,遙轄於浙江。淮北捻匪蔓延,袁甲三任之,巡撫號令所及,僅十餘縣。兵後荒蕪,賦稅無出,餉絀兵譁,遺失巡撫關防,自請嚴議,上原之,薄譴而已。會江南大營潰,和春移赴督師,惟總兵秦定三、鄭魁士兩軍仍留,倚以戰守。粵匪大股由湖北入皖,捻匪縱橫於皖、豫之交,省爭調定三、魁士二人。奏上,皆報可,福濟依違無可否。定三久攻桐城未下,魁士亦奉命而至,兩軍爭餉生嫌,賊乘隙撲營,遂致大潰。八年,滁州、來安、鳳陽、懷遠相繼失陷。福濟以病乞假,詔斥日久無功,褫宮銜、頭品頂戴,命來京。尋授內閣學士,予副都統銜,充西寧辦事大臣。九年,以安插投誠野番功,還頭品頂戴。十年,授工部侍郎,署陜甘總督,兼正黃旗漢軍都統。十一年,授成都將軍,調雲貴總督。文宗崩,福濟奏請謁梓宮,不許,詔斥規避滇、黔軍務,褫職,予四品頂戴,仍赴雲南,交署總督潘鐸差遣。

同治元年,予副都統銜赴西藏查辦事件,道梗未往。四年,還京。六年,授科布多幫辦大臣,調布倫托海辦事大臣。八年,授烏里雅蘇臺將軍。九年,回匪陷烏里雅蘇臺,褫職。十二年,捐銀助賑。直隸總督李鴻章為陳在安徽前勞,還原銜。光緒元年,卒,依巡撫例賜恤。

翁同書,字藥房,江蘇常熟人,大學士心存子。道光二十年進士,選庶吉士,授編修。大考屢列二等,擢中允。咸豐元年,應詔陳四事:請撫恤失業良民;察舉潔己愛民守令;興修江、浙、湖廣水利;訓練嶺海水師。三年,命赴江南佐欽差大臣琦善軍事。擢侍講學士,轉侍讀學士,遷少詹事。六年,自軍中奏言:「安民先足兵,足兵先理財。雲南運銅道梗,請於滇中設局鼓鑄,運錢至荊州充軍需及河工之用。沿江戒嚴,淮南鹽引不行,請以浙鹽行江西,而以蘇、常、鎮、太四府州改食淮鹽。江、浙漕米改由海運,數不及全漕之半,請分米雇民船仍由運河行轉搬之法。馬政廢弛,請令營馬量減數成,牧馬除借營用,令變價解庫。各省營兵應調赴戰,請飭將傷病撤回。空糧缺伍,實力整頓。軍興各省州縣倉穀或遭蹂躪,或備供億,實存綦少,請令地方官勸富民納粟入倉,量予獎勵。」又疏陳江防五事,曰:扼要津,聯陸路,斷岸奸,議火攻,增小船,並下部議行。琦善卒,托明阿為欽差大臣,同書仍留佐軍事。粵匪再陷揚州,托明阿坐罷,德興阿代之,詔同書幫辦軍務。德興阿連復揚州、浦口,進規瓜洲、鎮江,軍事日有起色,多出同書贊畫。克瓜洲,命以侍郎候補,賜黃馬褂。

八年六月,授安徽巡撫。時廬州再陷,粵匪、捻匪相勾結,淮南北蹂躪殆遍。上命同書幫辦欽差大臣勝保軍務,安徽境各軍均歸節制。同書移軍定遠,賊自天長犯三河集,擊破之,復天長。捻匪擾定遠,粵匪亦來犯,同書督兵擊卻之。九年,捻匪大舉陷六安,攻定遠,同書與勝保夾擊,大破之,復六安。捻匪復合粵匪數萬人來犯,定遠陷,同書移軍壽州,下吏議,革職留任。同書奏:「近來可用之兵,莫如楚師。諜聞楚師順江而下,已破石牌。倘別遣勁旅間道急趨英、霍,徐圖懷、定,此上策也。如楚師轉戰未能深入,用苗沛霖輔以官軍,先拔懷遠,此中策也。若二者皆不能行,則以勝保攻明光,李世忠逾清流關以保東路,臣守壽州,與傅振邦、關保相應援,制孫葵心、劉添福二巨捻以保西路,此下策也。」葵心攻潁州,同書遣兵擊之,敗走,復霍山。十年,遣兵攻爐橋,焚賊壘,進擊舒城援賊,破王家海賊圩。勝保議招葵心,上諮同書,同書言師方攻程家圩賊巢,不必曲意招撫。俄拔程家圩。

英法聯軍犯京師,勝保請召苗沛霖練勇入援,命同書傳旨;同書亦自請開巡撫缺,率之同行:尋並諭止之。粵匪陳玉成攻壽州,同書力禦,尋退。苗沛霖本懷反側,見時方多故,益猖恣,因與壽州團練徐立壯、孫家泰等有嫌,會其所部數人為立壯所殺,遂圍攻壽州。同書密疏陳沛霖跋扈,詔飭會袁甲三查辦。沛霖抗不聽命,圍攻益急,縱兵四擾。立壯所部多舊捻,素騷擾為民怨,十一年,坐其通捻,殺之。又下孫家泰於獄,家泰自殺。以蒙時中付沛霖,沛霖仍不息兵。召同書還京,以賈臻代署巡撫。同書令署布政使張學鵬勸諭沛霖,始撤圍。奏言:「沛霖過猶知改,請量加撫慰,責剿捻贖罪,俾袁甲三、賈臻籌辦善後事宜。」

同治元年,曾國籓奏劾同書於定遠失守時棄城走壽州,復不能妥辦,致紳練有仇殺之事。迨壽州城陷,奏報情形前後矛盾,命褫職逮問。王大臣會鞫,擬大闢。父心存病篤,暫釋侍湯藥。心存卒,復命持服百日仍入獄。二年,改戍新疆。三年,都興阿請留甘肅軍營效力,以花馬池戰捷,獲賊渠孫義保,賜四品頂戴。尋卒,復原官,贈右都御史,謚文勤。

嚴樹森,初名澍森,字渭春,四川新繁人,原籍陜西渭南。道光二十年舉人,入貲為內閣中書。改知縣,銓授湖北東湖,捐升同知。以防剿功,晉秩知府,署武昌府。巡撫胡林翼薦之,八年,擢荊宜施道,遷按察使。十年,遷布政使,擢河南巡撫。

時皖捻縱橫於河南境內,又有汝寧土匪陳大喜、金樓教匪郜永清皆猖獗。十一年正月,捻匪姜臺凌自歸德犯省城,援軍集,遂南趨陷唐縣,攻南陽府城,圍鄧州、裕州,三月,始回巢。孫葵心犯光州、陳州,亦至三月始出境。苗沛霖黨勾結陳大喜等擾陳州、汝寧邊境。五月,雷彥等圍鹿邑,經月始回巢。七月,劉狗大股分黑、白、花三旗擾歸德,結金樓教匪攻馬牧寨。樹森出駐陳州督剿。八月,劉狗竄硃仙鎮,犯省城。樹森率兵回援,賊竄汜水、鞏縣,掠黑石關,回竄鄭州,仍由歸德回巢。姜臺凌亦犯沈丘、裕州,越樊城,復入荊子關,擾南、汝兩郡,由柘城、鹿邑回巢。十月,劉狗復大舉援金樓寨,為官軍所阻,未得逞。時苗沛霖復叛,結張洛行,與汝寧、正陽、息縣諸匪連絡,將犯河南。樹森偕團練大臣毛昶熙合疏請調宜昌鎮總兵李續燾及鮑超部將陳由立,各募楚勇三千赴豫,又調吉林馬隊一千,以資防剿,請增兵之後,山西、陜西月協銀各二萬兩,允之。樹森老於吏事,在湖北從胡林翼治兵久,堅愎自是,與毛昶熙不合,事相掣肘。治河南年餘,禦賊雖有擒斬,軍事不得要領,迄無起色,調湖北巡撫。

同治元年,粵匪陳得才自南陽趨陜邊,捻匪竄永寧,延及雒南。樹森疏言:「當今賊勢,不患其並力南趨,特慮其潛窺陜境。西、同、鳳三府為全陜菁華所萃,宜急驅出關,會合夾擊,以保完善之區。」五月,賊犯鄖西,令總兵何紹彩敗之何家店。會道員金國琛赴鄖策應,令周鳳山分兵剿正陽、羅山,破賊巢,克邢家集、龍井、陡溝、明港。敘、捻諸匪合陷隨州,陳大喜陷京山,馬融和陷德安,令舒保擊敗德安賊,穆正春復京山、應城,襄北稍定。因星變,奏劾欽差大臣勝保。又奏言:「籓、臬任重,不得以軍功擅請記名。標兵缺額,請以戰勇充補。陣亡恤賞欠發,許作子孫捐項,敘給官階職銜貢監。京官五品以下,官俸實發不折。」下部分別議行。

二年,捻匪竄城,樹森赴黃州視師,督舒保、穆正春等擊走之。三年,粵、捻諸匪由陜南合趨湖北,詔總督官文出省督師,樹森留防省城。官文奏劾樹森把持兵柄,舊營悉改隸撫標。上斥其任意妄為,降道員。四年,授廣西按察使,貴州巡撫張亮基被劾玩兵侵餉。縱暴殃民諸款,命樹森馳往查奏。五年,授貴州布政使。樹森逗遛不進,未至,即奏覆參案。六年,疏請開缺,詔斥其規避取巧,褫職,發往雲南差遣委用。十一年,予四品頂戴,署廣西按察使。光緒元年,遷布政使,就擢巡撫。二年,卒,賜恤。

論曰:王懿德治閩,悍寇未深入。鎮輯萑苻,尚能保境。曾望顏在言路有聲,治兵無術,蜀亂遂成。耆齡輯睦湘軍,因人成事。安徽兵餉俱絀,四郊多壘,福濟固一籌莫展。翁同書亦據蒺終兇。嚴樹森恃才器小,效胡林翼而適得其反者也。


\end{pinyinscope}