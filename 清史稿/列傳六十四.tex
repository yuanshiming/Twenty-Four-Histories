\article{列傳六十四}

\begin{pinyinscope}
於成龍孫準彭鵬陳瑸陳鵬年施世綸

於成龍,字北溟,山西永寧人。明崇禎間副榜貢生。順治十八年,謁選,授廣西羅城知縣,年四十五矣。羅城居萬山中,盛瘴癘,瑤、僮獷悍,初隸版籍。方兵後,遍地榛莽,縣中居民僅六家,無城郭廨舍。成龍到官,召吏民拊循之,申明保甲。盜發即時捕治,請於上官,讞實即處決,民安其居。鄰瑤歲來殺掠,成龍集鄉兵將搗其巢,瑤懼,誓不敢犯羅山境。民益得盡力耕耘。居羅山七年,與民相愛如家人父子。牒上官請寬徭役,疏鹺引,建學宮,創設養濟院,凡所當興罷者,次第舉行,縣大治。總督盧興祖等薦卓異。

康熙六年,遷四川合州知州。四川大亂後,州中遺民裁百餘,正賦僅十五兩,而供役繁重。成龍請革宿弊,招民墾田,貸以牛種,期月戶增至千。遷湖廣黃岡同知,駐岐亭。岐亭故多盜,白晝行劫,莫敢誰何。成龍撫其渠彭百齡,貰罪,令捕盜自贖。嘗察知盜所在,偽為丐者,入其巢,與雜處十餘日,盡得其平時行劫狀。乃出呼役械諸盜,具獄辭,駢縛坑之,他盜皆遠竄。嘗微行村堡,周訪閭里情偽,遇盜及他疑獄,輒蹤跡得之,民驚服。巡撫張朝珍舉卓異。

十三年,署武昌知府。吳三桂犯湖南,師方攻岳州,檄成龍造浮橋濟師,甫成,山水發,橋圮,坐奪官。三桂散偽劄遍湖北州縣,麻城、大冶、黃岡、黃安諸盜,皆倚山結寨應三桂。妖人黃金龍匿興寧山中,謀內亂。劉君孚者,嘗為成龍役,善捕盜,亦得三桂劄,與金龍等結大盜周鐵爪,據曹家河以叛。朝珍以成龍舊治得民心,檄往招撫。成龍詗知君孚雖反,眾未合,猶豫持兩端。兼程趨賊砦,距十里許止宿,榜示自首者免罪,來者日千計,皆貸之。先遣鄉約諭君孚,降者待以不死。乃策黑騾往,從者二,張蓋鳴鉦,逕入賊舍。呼君孚出見,叩頭受撫,降其眾數千,分立區保,籍其勇力者,督令進討。金龍走紙棚河,與其渠鄒君申往保山砦,成龍擒斬之。朝珍以聞,請復官,即擢黃州知府,上允之。

諸盜何士榮反永寧鄉,陳鼎業反陽邏,劉啟業反石陂,周鐵爪、鮑世庸反泉畈,各有眾數千,號東山賊,遙與湖口、寧州諸盜合,將趨黃州。時諸鎮兵皆從師徇湖南,州中吏民裁數百,議退保麻城。成龍曰:「黃州,七郡門戶,我師屯荊、岳,轉運取道於此。棄此不守,荊、岳且瓦解。」誓死不去。遂集鄉勇得二千人,遣黃岡知縣李經政攻陽邏,得鼎業誅之。士榮率賊數犯,自牧馬崖分兩路來犯。成龍遣千總羅登雲以千人當東路,而自當西路。令千總吳之蘭攻左,武舉張尚聖攻右,成龍力沖其中堅。戰合,之蘭中槍死,師少卻;成龍策馬冒矢石逕前,顧千總李茂升曰:「我死,汝歸報巡撫!」茂升戰甚力,尚聖自右出賊後,賊大敗,生致士榮,檻送朝珍,遂進克泉畈。凡二十四日,東山賊悉平。十五年,歲饉,訛言復起。成龍修治赤壁亭榭,日與僚吏歗詠其中,民心大定。會丁繼母憂,總督蔡毓榮奏請奪情視事。十六年,增設江防道,駐黃州,即以命成龍。

十七年,遷福建按察使。時鄭成功迭犯泉、漳諸郡,民以通海獲罪,株連數千人,獄成,當駢戮。成龍白康親王傑書,言所連引多平民,宜省釋。王素重成龍,悉從其請。遇疑獄,輒令訊鞫。判決明允,獄無淹滯。軍中多掠良民子女沒為奴婢,成龍集資贖歸之。巡撫吳興祚疏薦廉能第一,遷布政使。師駐福建,月徵莝夫數萬,累民,成龍白王罷之。

十九年,擢直隸巡撫,蒞任,戒州縣私加火耗餽遺上官。令既行,道府劾州縣,州縣即訐道府不得餽遺挾嫌,疏請嚴定處分,下部議行。宣化所屬東西二城與懷安、蔚州二衛舊有水沖沙壓地千八百頃,前政金世德請除糧,未行,為民累;成龍復疏請,從之。又以其地夏秋屢被災,請治賑。別疏劾青縣知縣趙履謙貪墨,論如律。二十年,入覲,召對,上褒為「清官第一」,因問剿撫黃州土賊狀,成龍對:「臣惟宣布上威德,未有他能。」問:「屬吏中亦有清廉否?」成龍以知縣謝錫袞,同知何如玉、羅京對。復諭劾趙履謙甚當,成龍奏:「履謙過而不改,臣不得已劾之。」上曰:「為政當知大體,小聰小察不足尚。人貴始終一節,爾其勉旃!」旋賜帑金千、親乘良馬一,制詩褒寵,並命戶部遣官助成龍賑濟宣化等處饑民。成龍復疏請緩真定府屬五縣房租,並全蠲霸州本年錢糧,均報可。是年冬,乞假喪母,優詔許之。

未幾,遷江南江西總督。成龍先後疏薦直隸守道董秉忠、阜城知縣王燮、南路通判陳天棟。瀕行,復薦通州知州於成龍等。會江寧知府缺,命即以通州知州於成龍擢補。成龍至江南,進屬吏誥誡之。革加派,剔積弊,治事嘗至達旦。好微行,察知民間疾苦、屬吏賢不肖。自奉簡陋,日惟以粗糲蔬食自給。江南俗侈麗,相率易布衣。士大夫家為減輿從、毀丹堊,婚嫁不用音樂,豪猾率家遠避。居數月,政化大行。勢家懼其不利,構蜚語。明珠秉政,尤與忤。二十二年,副都御史馬世濟督造漕船還京,劾成龍年衰,為中軍副將田萬侯所欺蔽。命成龍回奏,成龍引咎乞嚴譴,詔留任,萬侯降調。二十三年,江蘇巡撫餘國柱入為左都御史,安徽巡撫塗國相遷湖廣總督,命成龍兼攝兩巡撫事。未幾,卒於官。

成龍歷官未嘗攜家屬,卒時,將軍、都統及僚吏入視,惟笥中綈袍一襲、床頭鹽豉數器而已。民罷市聚哭,家繪像祀之。賜祭葬,謚清端。內閣學士錫住勘海疆還,上詢成龍在官狀,錫住奏甚清廉,但因輕信,或為屬員欺罔。上曰:「於成龍督江南,或言其變更素行,及卒後,始知其始終廉潔,為百姓所稱。殆因素性鯁直,不肖挾仇讒害,造為此言耳。居官如成龍,能有幾耶?」是年冬,上南巡至江寧,諭知府於成龍曰:「爾務效前總督於成龍正直潔清,乃為不負。」又諭大學士等曰:「朕博採輿評,咸稱於成龍實天下廉吏第一。」加贈太子太保,廕一子入監,復制詩褒之。雍正中,祀賢良祠。

孫準,字子繩。自廕生授山東臨清知州,有清操。舉卓異,入為刑部員外郎,遷戶部郎中。出為江南驛鹽道,再遷浙江按察使,居成龍喪歸,起四川布政使。康熙四十三年,授貴州巡撫。飭州縣立義學,令土司子弟及苗民俊秀者悉入肄業,送督學考試。調江蘇,歲饑,請發帑賑濟上元等十五縣及太倉、鎮海二衛。濱江海田畝被潮汐沖擊,多坍沒,疏請豁免錢糧,詔允行。以布政使宜思恭為總督噶禮所劾,準坐失察,罷歸。雍正三年,復職銜。尋卒。

彭鵬,字奮斯,福建莆田人。幼慧,有與其父仇,欲殺鵬,走匿得免。順治十七年,舉鄉試。耿精忠叛,迫就偽職,鵬陽狂示疾,椎齒出血,堅拒不從。事平,謁選,康熙二十三年,授三河知縣。三河當沖要,旗、民雜居,號難治。鵬拊循懲勸,不畏強御。有妄稱御前放鷹者,至縣索餼牽,鵬察其詐,縶而鞭之。治獄,摘發如神。鄰縣有疑獄,檄鵬往鞫,輒白其冤。二十七年,聖祖巡畿甸,召問鵬居官及拒精忠偽命狀,賜帑金三百,諭曰:「知爾清正不受民錢,以此養爾廉,勝民間數萬多矣!」尋順天府尹許三禮劾鵬匿報控案,命巡撫於成龍察之。成龍奏:「鵬訊無左驗,方緝兇,非不報也。」吏議奪官,詔鐫級留任。嗣以緝盜不獲,累被議,積至降十三級,俱從寬留任。

二十九年,詔舉廉能吏,用尚書李天馥薦,鵬與邵嗣堯、陸隴其、趙蒼璧並行取,擢為科道。尋乞假歸,明年,即家起工科給事中。三十二年,陜西西安、鳳翔,山西平陽災,發帑賑之。又命運河南米十萬石畀陜西散饑民。鵬疏論陜西、山西、河南三省有司不恤民狀,語甚切,下所司,並令鵬指實以聞。鵬因奏涇陽知縣劉桂剋扣籽粒,猗氏知縣李澍杖殺災民,磁州知州陳成郊濫派運價,夏邑知縣尚崇震派銀包運,南陽知府硃璘曖昧分肥,並及聞喜、夏縣匿災不報狀。詔三省巡撫察審,事不皆實,鵬例當譴,上貰之。

三十三年,疏劾順天鄉試中式舉人李仙湄闈墨刪改過多,楊文鐸文謬妄,給事中馬士芳磨勘通賄。下九卿等察議,以鵬奏涉虛,因摘疏語有「臣言如妄,請劈臣頭,半懸國門,半懸順天府學」,以為狂妄不敬,應奪官。命鵬回奏,鵬疏言:「會議諸臣,徇試官徐倬、彭殿元欺飾,反以臣為妄,乞賜罪斥。」上不問,而予倬、殿元休致。

是年,順天學政侍郎李光地遭母喪,上命在任守制,光地乞假九月。鵬劾光地貪戀祿位,不請終制,應將光地解任,留京守制,上從之。會廷臣集議,鵬追論楊文鐸文謬妄,與廷臣忿爭,事聞,命解職,以原品效力江南河工。三十六年,召授刑科給事中。三十七年,出為貴州按察使。

三十八年,擢廣西巡撫。湖廣總督郭琇請除學政積弊,給事中慕琛、滿晉,御史鄭惟孜等亦疏列順天鄉試事。上以李光地。張鵬翮、郭琇與鵬俱清廉,命各抒所見。鵬疏言:「琇請嚴督撫處分,學政貪贓,提問督撫,需索陋規,視貪贓治罪,久有定例,請敕榜示律條。維孜請令各省監生回籍鄉試,九卿慮成均空虛,應責成祭酒司業,就坐監讀書者講習考課,各省學政擇諸生有文行者送入成均,何慮空虛?琛、晉請察封坐號以防換卷,臣謂換卷多在入門暗約出號交卷時,請嚴稽於此。」又言:「文官子弟請皇上親試,臣謂當另立考場,去取聽睿裁。」與光地等疏皆下九卿詳議。互詳光地等傳。時河南巡撫徐潮之任,上諭曰:「爾能如李光地、張鵬翮、郭琇、彭鵬,不但為今之名臣,亦足重於後世矣。」鵬在官省刑布德,減稅輕徭。廣西舊供魚膠、鐵葉,非其土物,赴廣東採運,鵬疏請免之。

尋移撫廣東,瀕行,疏言:「廣西州縣借端私派,名曰均平。臣到任,劾罷賀縣、荔浦、懷集、武緣諸貪吏。前此諸州縣大者派至三千兩,其次一二千兩。不肖官吏,往往先徵均平而後正課,甚者均平入己,遇事復行苛派。其不派均平者,又取盈於火耗。且均平所入,費於公者十之二三,費於饋遺者十之六七。欲去舊弊、甦民困,必先養州縣之廉。請於徵糧之內,明加火耗一分。其餘陋規,概行禁止。」疏入,下部議,謂火耗不可行,但嚴禁加派。廣西舊未設武科,鵬奏請行之。時與蕭永藻互調,上勉永藻效鵬,又諭大學士曰:「彭鵬人才壯健,前知三河,聞有賊,即佩刀乘馬馳捕,朕所知也。」御史王度昭劾鵬在廣西知布政使教化新虧帑,不即糾舉;迨離任始奏聞,又掩護其半。廣西糧道張天覺改徵兵米浮銷九十餘萬,部勒追完,而鵬反以天覺署布政使。兵米之案,必由籓司審詳,是直以天覺察天覺也。命鵬回契,鵬疏辨,並訐度昭。上以其辭忿激,降旨嚴飭。

廣東因借兵餉,改額賦徵銀為徵米,較估報時值浮多,戶部屢飭追完。鵬至官,是年歲稔米價低,以米計銀少七萬三千有奇,疏請令經管各官扣追存庫,並議嗣後額賦仍依原則徵銀,採購兵米;其按年應追完之銀,實因豐歉不同,米價無定,乞免重追:詔允行。鵬視事勤敏,遇墨吏糾劾無少徇。歲旱,步禱日中,詣獄慮囚,開倉平糶,旋得雨,民大稱頌。四十三年,卒官,年六十八,上深悼惜,稱其勤勞,賜祭葬。尋祀廣東名宦。

陳瑸,字眉川,廣東海康人。康熙三十三年進士,授福建古田知縣。古田多山,丁田淆錯,賦役輕重不均,民逋逃遷徙,黠者去為盜。瑸請平賦役,民以蘇息。調臺灣,臺灣初隸版圖,民驍悍不馴。瑸興學廣教,在縣五年,民知禮讓。四十二年,行取,授刑部主事,歷郎中,出為四川提學道僉事。清介公慎,杜絕苞苴。上以四川官吏加派厲民,詔戒飭,特稱瑸廉。未幾,用福建巡撫張伯行薦,調臺灣廈門道。新學宮建硃子祠於學右,以正學厲俗,鎮以廉靜,番、民帖然。在官應得公使錢,悉屏不取。

五十三年,超擢偏沅巡撫。蒞任,劾湘潭知縣王爰溱縱役累民,長沙知府薛琳聲徇庇不糾劾,降黜有差。尋條奏禁加耗,除酷刑,糶積穀,置社倉,崇節儉,禁餽送,先起運,興書院,飭武備,停開採,凡十事。詔嘉勉,諭以躬行實踐,勿騖虛名。旋入覲,奏言:「官吏妄取一錢,即與百千萬金無異。人所以貪取,皆為用不足。臣初任知縣,即不至窮苦,不取一錢,亦自足用。」比退,上目之曰:「此苦行老僧也!」

尋調撫福建,上諭廷臣曰:「朕見瑸,察其舉止言論,實為清官。瑸生長海濱,非世家大族,無門生故舊,而天下皆稱其清。非有實行,豈能如此?國家得此等人,實為祥瑞。宜加優異,以厲清操。」陛辭,上問:「福建有加耗否?」瑸奏:「臺灣三縣無之。」上曰:「火耗盡禁,州縣無以辦公,恐別生弊端。」又曰:「清官誠善,惟以清而不刻為尚。」瑸為治,舉大綱,不尚煩苛。修建考亭書院及建陽、尤溪硃子祠,疏請御書榜額,並允之。復疏言:「防海賊與山賊異,山賊嘯聚有所,而海賊則出沒靡常。臺灣、金、廈防海賊,又與沿海邊境不同,沿海邊境患在突犯內境,而臺、廈患在剽掠海中。欲防臺、廈海賊,當令提標及臺、澎水師定期會哨,以交旗為驗。商船出海,令臺、廈兩汛撥哨船護送。又令商船連環具結,遇賊首尾相救,不救以通同行劫論罪。」下部議,以為繁瑣,上韙其言,命九卿再議,允行。

是年冬,兼攝閩浙總督。奉命巡海,自齎行糧,屏絕供億。捐穀應交巡撫公費,奏請充餉。上曰:「督撫有以公費請充餉者,朕皆未之允。蓋恐準令充餉,即同正項錢糧,不肖者又於此外婪取,重為民累。」令瑸遇本省需款撥用。賓又請以司庫餘平賞賚兵役,命遵前旨。廣東雷州東洋塘堤岸,海潮沖激,侵損民田,瑸奏請修築,即移所貯公項及俸錢助工費。堤岸自是永固,鄉人蒙其利。五十七年,以病乞休,詔慰留之。未幾,卒於官。遺疏以所貯公項餘銀一萬三千有奇充西師之費。命以一萬佐餉,餘給其子為葬具。尋諭大學士曰:「陳賓居官甚優,操守極清,朕所罕見,恐古人中亦不多得也。」追授禮部尚書,廕一子入監讀書,謚清端。

瑸服御儉素,自奉惟草具粗糲。居止皆於事,昧爽治事,夜分始休。在福建置學田,增書院學舍,聘主講,人文日盛。雍正中,入祀賢良祠。乾隆初,賜其孫子良舉人;子恭員外郎,官至知府。

陳鵬年,字滄洲,湖廣湘潭人。康熙三十年進士。授浙江西安知縣,當兵後,戶口流亡,豪強率占田自殖。鵬年履畝按驗,復業者數千戶。烈婦徐冤死十年,鵬年雪其枉,得罪人置諸法。禁溺女,民感之,女欲棄復育者,皆以陳為姓。河道總督張鵬翮薦調赴江南河工,授江南山陽知縣,遷海州知州。四十二年,聖祖南巡閱河,以山東饑,詔截漕四萬石,令鵬翮選賢幹吏運兗州分賑,以鵬年董事,全活數萬人。上回鑾,召見濟寧舟次,賦詩稱旨,賜御書。

尋擢江寧知府。四十四年,上復南巡,總督阿山召屬吏議增地丁耗羨為巡幸供億,鵬年力持不可,事得寢。阿山嗛之,令主辦龍潭行宮,侍從徵餽遺,悉勿應,忌者中以蜚語。會致仕大學士張英入對,上問江南廉吏,舉鵬年;復詢居官狀,英言:「吏畏威而不怨,民懷德而不玩,士式教而不欺,廉其末也。」上意乃釋。幸京口閱水師,先一日,阿山檄鵬年於江幹疊石為步,江流急,施工困難,胥徒惶遽。鵬年率士民親運土石,詰旦工成。顧阿山憾不已,疏劾鵬年受鹽、典各商年規,侵蝕龍江關稅銀,又無故枷責關役,坐奪職,系江寧獄。命桑額、張鵬翮與阿山會鞫,江寧民呼號罷市,諸生千餘建幡將叩閽。鵬年嘗就南市樓故址建鄉約講堂,月朔宣講聖諭,並為之榜曰「天語丁寧」。南市樓者故狹邪地也,因坐以大不敬,論大闢。上與大學士李光地論阿山居官,光地言阿山任事廉幹,獨劾陳鵬年犯清議,上頷之。讞上,鵬年坐奪官免死,徵入武英殿修書。

四十七年,復出為蘇州知府。禁革奢俗,清滯獄,聽斷稱神。值歲饑,疫甚,周歷村墟,詢民疾苦,請賑貨,全活甚眾。四十八年,署布政使。巡撫張伯行雅重鵬年,事無鉅細,倚以裁決。總督噶禮與伯行忤,並忌鵬年。已,劾布政使宜思恭、糧道賈樸,因坐鵬年覈報不實,吏議奪官,遣戍黑龍江,上寬之,命仍來京修書。噶禮復密奏鵬年虎丘詩,以為怨望,欲文致其罪,上不報。俄,噶禮與伯行互訐,屢遣大臣按治,議奪伯行職。上以伯行清廉,命九卿改議,並諭曰:「噶禮曾奏陳鵬年詩語悖謬,宵人伎倆,大率如此。朕豈受若輩欺耶?」因出其詩畀閣臣共閱。五十六年,出署霸昌道,仍回京修書。

六十年,命隨尚書張鵬翮勘山東、河南運河,時河決武陟縣馬營口,自長垣直注張秋,命河督趙世顯塞之。議久不決,鵬年疏言:「黃河老堤沖決八九里,大溜直趨溢口,宜於對岸上流廣武山下別開引河,更於決口稍東亦開引河,引溜仍歸正河,方可堵築。」奏入稱旨。世顯罷,即命鵬年署河道總督。六十一年,馬營口既塞復決,鵬年謂:「地勢低窪,雖有引河,流不能暢。惟有分疏上下,殺其悍怒。請於沁、黃交匯對岸王家溝開引河,使水東南行,入滎澤正河,然後堤工可成。」詔如議行。先是,馬營決口因桃汛流激,難以程工;副都御史牛鈕奉命閱河,奏於上流秦家廠堵築,工甫竟,而南壩尾旋決一百二十餘丈,入馬營東下。鵬年與巡撫楊宗義謀合之。既,北壩尾復潰百餘丈,鵬年乃建此議。世宗即位,命真除。時南北壩尾合而復潰者四,至是以次合龍,而馬營口尚未塞。鵬年止宿河堧,寢食俱廢,浸羸憊。雍正元年,疾篤,遣御醫診視。尋卒,上聞,諭曰:「鵬年積勞成疾,沒於公所。聞其家有八旬老母,室如懸罄。此真鞠躬盡瘁、死而後已之臣。」褒錫甚至。賜帑金二千,錫其母封誥,視一品例廕子,謚恪勤。祀河南、江寧名宦。

子樹芝、樹萱,聖祖時,以諸生召見,令隨鵬年校書內廷。樹芝官至平越知府,樹萱官至戶部侍郎。

施世綸,字文賢,漢軍鑲黃旗人,瑯仲子。康熙二十四年,以廕生授江南泰州知州。世綸廉惠勤民,州大治。二十七年,淮安被水,上遣使督堤工,從者數十輩,驛騷擾民,世綸白其不法者治之。湖北兵變,官兵赴援出州境,世綸具芻糧,而使吏人執梃列而待,兵有擾民,立捕治,兵皆斂手去。二十八年,以承修京口沙船遲誤,部議降調。總督傅臘塔疏陳世綸清廉公直,上允留任。擢揚州知府。揚州民好游蕩,世綸力禁之,俗為變。三十年八月,海潮驟漲,泰州範公堤圮,世綸請捐修。三十二年,移江寧知府。三十五年,瑯卒,總督範成勛疏以世綸輿情愛戴,請在任守制;御史胡德邁疏論,世綸乃得去官,復居母喪。歲餘,授蘇州知府,仍請終制,辭不赴。三十八年,既終制,授江南淮徐道。

四十年,湖南按察使員缺,九卿舉世綸,大學士伊桑阿入奏,聖祖諭曰:「朕深知世綸廉,但遇事偏執,民與諸生訟,彼必袒民;諸生與搢紳訟,彼必袒諸生。處事惟求得中,豈偏執?如世綸者,委以錢穀之事,則相宜耳。」是歲授湖南布政使。湖南田賦丁銀有徭費,漕米有京費。世綸至,盡革徭費,減京費四之一,民立石頌之。四十三年,移安徽布政使。

四十四年,遷太僕寺卿。四十五年,坐湖南任內失察營兵掠當鋪,罷職。三月,授順天府府尹,疏請禁司坊擅理詞訟、奸徒包攬捐納、牙行霸占貨物、流娼歌舞飲宴,飭部議,定為令。四十八年,授左副都御史,兼管府尹事。四十九年,遷戶部侍郎,督理錢法。尋調總督倉場。五十四年,授雲南巡撫,未行,調漕運總督。世綸察運漕積弊,革羨金,劾貪弁,除蠹役,以嚴明為治。歲督漕船,應限全完,無稍愆誤。

時西陲用兵,轉輸餽運,自河南達陜西。陜西旱饑,五十九年,上命世綸詣陜西佐總督鄂海督軍餉,並令道中勘河南府至西安黃河輓運路徑,並察陜西現存穀石數目陳奏。世綸乃溯河西上,疏言:「河南府孟津縣至陜西太陽渡,大小數十餘灘,纖道高低不等,或在河南,或在河北。澠池以下,舟下水可載糧三百餘石,上水載及其半;澠池以上,河流高迅,僅可數十石。自砥柱至神門無纖道,惟路旁石往往有方眼,又有石鼻,從前輓運,其跡猶存。自陜州至西安府,河水平穩,俱有輓運路徑。謹繪圖以聞。」又言:「河南府至陜州三門,今乃無舟。請自太陽渡以下改車運,太陽渡至西安府黨家馬頭舟行為便。黨家馬頭入倉復改車運,穀二十萬石都銀十萬三千兩有奇。但運穀二十萬,止得米十萬。請令河南以二穀易一米,則運價可省其半。若慮米難久貯,請照例出陳易新。」奏入,上念陜西災,發帑金五十萬,並令酌發常平倉穀;又以地方官吏大半在軍前,令選部院司官詣陜西,命世綸總其事。世綸令分十二路察貧民,按口分給,遠近皆遍。六十年春,得雨,災漸澹。上命世綸還理漕事。六十一年四月,以病乞休,溫旨慰留,令其子廷祥馳驛省視。五月,卒。遺疏請隨父瑯葬福建,上允之,詔獎其清慎勤勞,予祭葬。

世綸當官聰強果決,摧抑豪猾,禁戢胥吏。所至有惠政,民號曰「青天」。在江寧以憂歸,民乞留者逾萬。既不得請,人出一錢建兩亭府署前,號一文亭。官府尹,步軍統領託合齊方貴幸,出必擁騶從。世綸與相值,拱立道旁俟。託合齊下輿驚問,世綸抗聲曰:「國制,諸王始具騶從。吾以為諸王至,拱立以俟,不意為汝也!」將疏劾,託合齊謝之乃已。賑陜西,陜西積儲多虛耗,將疏劾。鄂海以廷祥知會寧,語微及之,世綸曰:「吾自入官,身且不顧,何有於子?」卒疏言之。鄂海坐罷去。

論曰:於成龍秉剛正之性,苦節自厲,始終不渝,所至民懷其德。彭鵬拒偽命,立身不茍,在官亦以正直稱。陳瑸起自海濱,一介不取,行能踐言。陳鵬年、施世綸明愛人,不畏強御。之五人者,皆自牧令起,以清節聞於時。成龍、世綸名尤盛,閭巷誦其績,久而弗渝。康熙間吏治清明,廉吏接踵起,聖祖所以保全諸臣,其效大矣。


\end{pinyinscope}