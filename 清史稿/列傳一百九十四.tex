\article{列傳一百九十四}

\begin{pinyinscope}
江忠源弟忠濟族弟忠信羅澤南

江忠源,字岷樵,湖南新寧人。道光十七年舉人。究心經世之學,伉爽尚義。公車入京,初謁曾國籓,國籓曰:「吾生平未見如此人,當立名天下,然終以節烈死。」大挑教職,回籍。察教匪亂將作,陰以兵法部勒鄉里子弟。既而黃背峒盜雷再浩果勾結廣西莠民為亂,一戰破其巢,擒再浩戮之。以功擢知縣,揀發浙江。秀水災,奉檄往賑,遂權縣事。賑務畢舉,擒劇盜十數,邑大治。巡撫吳文鎔待以國士,補麗水,檄治海塘。文宗即位,曾國籓應詔薦其才,送部引見,尋以父憂去官。

咸豐元年,大學士賽尚阿督師剿粵匪,調赴軍前,副都統烏蘭泰深倚重,事必諮而行。忠源招舊所練鄉兵五百人,使弟忠濬率以往,號「楚勇」。賊氛方熾,官兵莫攖其鋒。忠源勇始至,偪賊而壘。賊輕其少,且新集,急犯之。堅壁不出,逼近始馳突,斬級數百,一軍皆驚。累功賜花翎,擢同知直隸州。賊聚永安,向榮與烏蘭泰不協,忠源調和,勿聽,知必敗,引疾回籍。

二年春,賊果突圍出犯桂林。忠源聞警,增募千人,偕劉長佑兼程赴援,未至,烏蘭泰傷歿於軍,自是獨領一軍,進扼桂林城外鸕鶿洲,三戰皆捷,圍尋解,擢知府。賊竄全州,將趨湖南,忠源偕諸軍進擊。賊陷城不守,復出竄,悉載輜重舟中,期水陸並下。忠源發樹塞河,截賊蓑衣渡,鏖戰兩晝夜,悍酋馮雲山中砲死。賊棄舟夜遁,盡獲其輜重。忠源先請扼東岸,未用其策,賊由東竄入湖南,陷道州。又議賊眾不滿萬,慮日久裹脅眾,分防不如合剿,遠堵不如近攻。於是諸軍合攻道州,賊堅壁,意在久踞。購城中內應,約期襲之。賊走藍山、嘉禾,犯桂陽,陷郴州。忠源謂後路進剿愈急,前路攻陷愈多,請仍申合剿之議,當事不省,賊益張,徑犯長沙。忠源偕總兵和春馳援,至則賊已踞城南,窟穴民廛,攻城甚急。忠源望見天心閣地勢高,賊柵其上,驚曰:「賊據此,長沙危矣!」率死士爭之,賊敗退。趣移壘逼賊,共汲一井,擊柝相聞。忠源弟忠濟自郴州尾賊至,約夾擊,為伏賊所傷。縋入城商方略,因語眾曰:「官軍四面集,惟河西一路空虛。賊奪民舟渡江掠食,食盡將他竄。宜重兵扼回龍塘。」巡撫張亮基韙之,而諸將逡巡莫前。時賽尚阿罷,徐廣縉代之,未至,城內外巡撫三,提督二,總兵十,莫相統攝。忠源赴湘潭,請於廣縉,不省。賊卒由回龍塘竄陷岳州,遂破武昌。忠源痛謀不見用,不欲東。張亮基奏留守湖南,剿平巴陵土匪,調赴瀏陽剿徵義堂會匪周國虞,斬馘七百,解散萬人。瀏陽平,擢道員。

三年正月,授湖北按察使,張亮基署總督,兵事悉倚之。剿平通城、崇陽、嘉魚、蒲圻諸匪,擒其渠劉立簡、陳百斗、熊開宇等。文宗知忠源忠勇可恃,命率所部赴向榮軍,尋命幫辦江南軍務。瀕行,上疏切論軍事,略曰:「粵寇之亂,用兵數年,糜餉二千萬,人無固志,地罕堅城。臣出入鋒鏑,於今三年,謹策其大端,惟聖明裁察:一曰嚴軍法。將不行法,是謂無將;兵不用法,是為無兵。全州以失援陷而左次相仍,道州以棄城陷而潰逃踵接;岳州設防而不能為旦夕之守,九江列艦而不能遏水陸之沖。豈有他哉?畏賊之念中之也。賊嘗致死於我,而我不能致死於賊。賊之戰也,驅新附於前,以故黨乘其後,卻則擊殺。故賊退必死而進乃生,我退必生而進則死,不待戰陣,而勝負分焉已。誠欲反怯為強,莫若易寬為猛。皇上執法以馭將帥,將帥執法以馭偏裨,偏裨執法以馭兵士。避寇者誅,不援者誅,未令而退者誅。法令既嚴,軍聲自壯。此討賊之大端也。一曰撤提鎮。承平既久,宿將凋亡,提鎮大臣,積資可待。位尊則意為趨避,偏裨不敢與爭;權重則法難驟加,督撫不能擅決。人情當齒壯官卑之日,輒思發奮為雄,位高則進取念衰,必不能踔厲以赴時會。且軍興數載,饋餉滋艱,提鎮所需,較副參懸絕。裁一提鎮,養精兵二百而有餘。奚取以有限脂膏,奉此無益之提鎮?誠擇一深明將略者統制其間,餘則悉歸休致。副將以下,量擢其才。此整軍之要道也。一曰汰冗兵。選兵膽氣為上,堅樸次之,技藝又次之。質實耐苦之人,令進則進,令退則退,其身聽命於將而不知它。浮怯之徒,無事則趨蹌觀美,臨陣則退縮旁徨,論功則鉆刺以圖美官,遇敗則推諉以逃咎戾,宜汰者一也。徵調頻煩,或羸老備籍,坐耗資糧,或部曲散亡,驚魂甫定。當此餉糈匱絀,豈容更益虛糜,宜汰者二也。誠敕各營將領,討部曲而嚴察之,氣充膽壯者備攻剿,樸實堅苦者備屯防。舍此二端,盡歸釐汰,此致強之急務也。一曰明賞罰。勝有賞,敗有罰,亙古不變之常經也。顧勝有賞而賞非勝,則不如無賞;敗有罰而罰非敗,則不如無罰。無賞無罰,人猶冀賞罰之時;賞非其功,罰非其罪,則懲勸之用乖,怨讟之聲作,而軍事不可為矣。今戰勝有功,固當賞錄,左右侍從,獎敘尤多;且未嘗行一失律之誅,按一縱寇之罪。勝敗本兵家之常,主兵者每言勝而諱敗;功過本無妨互見,主兵者輒匿過而言功。治承平天下且不可,況危亂之世哉?夫軍中賞罰未可一概論。勝固當賞,或旅進取斬級以冒功,或追擊貪貨財而得小,則當罰;敗固當罰,或邁勇先驅,後援不繼,或大軍已卻,一將獨前,則當賞。今大帥據營將之言,營將恃左右之口。功罪之實,非採訪所可知,好惡之心,因毀譽而多舛。求是非洽乎人心,難矣。自非親歷行陣,開誠布公,何以慰軍士之心而振披靡之習?此風氣不可不急為振拔者也。一曰戒浪戰。用兵之道,能守而後能戰,能制人而後不制於人,能避賊之長而後可用吾之短。臣自廣西以來,深觀賊勢,結營則因地築壘,環以深壕;置陣則正兵敵前,奇兵旁襲;止則遍購徒黨,伺吾虛實;行則遙壯聲威,乘吾張皇。故嘗以為賊止則當扼要以斷其饋濟,嚴兵以截其奔逃;賊行則當逆擊以遏其鋒,設伏以撓其勢。乃我之圍賊不嚴守而攻堅,追賊不截歸而尾擊,小有挫失,士氣先頹。此兵法不可不變計者也。一曰察地勢。勢者非圖史所載山川一定之險也。視賊出入之途,先為之防,察賊分合之機,遙為之制;則漸車之澮,數仞之岡,茍形勢在所必爭,即事機不容或失。全州蓑衣渡之戰,寇焰已摧,宜速壁河東斷其右臂;道州之役,寇鋒已挫,宜分屯七里橋扼其東趨;長沙將解圍,則宜堅壁回龍潭、土橋頭,使賊不得西犯。它若道州蓮花池、蓮濤灣,死地六十里,而縱之使生;湘陰臨資口、岳州城陵磯皆必爭之區,而縱之使遁。禍機在咫尺之間,流毒遂在千里之外。此敗轍之不可不深鑒者也。一曰嚴約束。殺賊所以安民,安民乃可殺賊。粵寇慘虐,不可勝言,然擇肥而噬,窮簷不暇搜求。或偽結民心,多償市直。兵則攫取奸污,窮戶且難幸免。故於賊且有恕詞,於兵能無怨毒。且長夫估客,游蕩無常,託偽營裝,恣行淫掠,鄉民畏懼,莫敢誰何。應敕諸營首嚴防制,備冊時稽。犯則軍法按行,絕其芽蘗。此結民心毖後患之要圖也。一曰寬脅從。粵寇徒黨,喪亡實多,煨燼之餘,類多附脅。平昔會徒盜賊,寬典相蒙,監禁軍流,乘時放逸,命為前導,尤所甘心。凡此法無可逭,自爾獲焉必殺。至若良民驅迫,骨肉羈縻,此中進退維谷之忱,艱苦顛連之狀,每一念及,輒用隱傷。宜敕各營刊示射達,臨陣建免死之旗,令其倒戈以赴,曲賜保全。既可探賊情,復以攜賊黨。此尤好生盛德,討賊機宜之大權也。行此八者,破格以攬奇才,便宜以畀賢帥,擇良吏以固根本,嚴綜覈以裕餉源。如此而盜賊不滅,盛治不興,原斬臣首以謝天下。」疏入,上嘉納之。

行至九江,聞南昌被圍,方有旨促援鳳陽,疏請先援江西,率兵千三百人,三晝夜馳抵南昌。巡撫張芾舉王命旗牌授忠源,戰守事悉聽指揮。忠源火城外廛廬,斬逃者,謂章江門最受敵,自當之,日登城督戰。賊穴地轟城,崩數十丈。刃斃先登賊,囊土填缺。數突門出戰,夜遣死士縋下焚賊營。詔嘉獎,被珍賚。尋湖南援師至,分軍扼樟樹鎮,遣羅澤南剿平泰和、萬安、安福土匪。守南昌九十餘日,至八月,屢砲毀賊壘,沉賊船,乘風縱火,賊乃遁。詔嘉其功,加二品頂戴。賊退據九江,分擾湖北興國,逕犯田家鎮。忠源赴援,部兵二千,途阻不能遽達,先挈親兵數十人抵田家鎮。甫一日,賊舟乘風大至,道員徐豐玉等死之。忠源自劾,詔原之,降四級留任,尋擢安徽巡撫。

賊已陷黃州、漢陽,圍武昌。沿江擊賊,敗之,武昌解嚴。疏請增兵萬人,當淮南一路,而湖北留其兵不盡遣,僅率兵二千冒雨行。將士疲頓,忠源亦遘疾。至六安,賊已陷桐城、舒城。吏民遮留,不可,留千人守六安,舁疾抵廬州。部署未定,賊已大至。城中合援兵團勇僅三千人,忠源力疾守陴,迭挫撲城之賊。地道轟城屢圮,皆奮擊卻之。詔嘉忠源力保危城,躬馳戰陣,賜號霍隆武巴圖魯。時陜甘總督舒興阿兵萬餘,畏葸不進。忠源弟忠濬偕劉長佑來援,駐城外五里墩,阻不得前。被圍月餘,廬州知府胡元煒陰通賊,賊知城中食乏,軍火將盡,攻益急。水西門圮,且戰且修築。賊突自南門緣梯入,忠源掣刀自刎。左右持之,一僕負之行,忠源奮脫。轉戰至水閘橋,身受七創,投古塘死之。布政使劉裕珍,池州知府陳源兗,同知鄒漢勛、胡子雝,縣丞興福、艾延輝,副將松安,參將馬良、戴文淵,同時殉難。胡元煒竟降賊。忠濬募人求其尸。後八日,部卒周昌跡得之,負出,面如生。

事聞,文宗震悼,贈總督,予騎都尉兼雲騎尉世職,入祀昭忠祠,謚忠烈。同治初,江南平,追念前功,予三等輕車都尉世職,湖南、江西並建專祠,湖北省城與羅澤南合祀三忠祠。忠源歿逾年,湖南有寇警,弟忠淑奉檄募勇助剿。母陳出私財助餉,並懸重賞以勵眾。事定,巡撫駱秉章以聞,特旨予忠源父母三代一品封典。忠源弟三人,忠濬、忠濟、忠淑,族弟忠義、忠信,皆自忠源初起即從軍中。忠濬、忠義自有傳。

忠濟,從守長沙,城壞,堵缺口,殺登城賊數十,以勇名。三年,忠源赴湖北,以舊部千人付忠濟留長沙。忠源剿賊通城,兵單不利,忠濟倍道赴援,戰於桂口,斬賊首陳申子於陣,又破何田俊等,焚其巢;及援南昌,兩塞城缺,斬賊之先登者。巡撫張芾疏稱其精敏勇敢,軍中畏服,累功擢候選知府。江西解嚴後,忠濟回籍侍母。忠源既歿,有旨仍用忠濟及忠濬率兵剿賊。忠濬方赴援廬州,從和春攻剿。忠濟為駱秉章調赴藍山、寧遠剿土匪,連破賊解圍,擢道員。五年,駐防岳州。胡林翼攻武昌未下,賊勾結崇陽、通城土匪,忠濟遣兵復通城,遂留駐。六年春,江西賊由義寧竄至,忠濟進擊,連破賊壘,而悍黨集數萬,為所圍,力戰三日,營陷,死之。贈按察使銜,予騎都尉世職,謚壯節。

忠信,少跅弛不羈,年十六,從忠源赴廣西軍。犯軍令,忠源將斬之,眾為乞免。及遇賊,驍捷敢戰,常為軍鋒,累加擢千總。聞忠源被圍廬州,從忠濬赴援。比至,壁西門外五里墩不得進。忠信夜率壯士十餘人,潛越賊營,縋入城,告以援至。留城中,屢完城缺,縋出攻賊壘,殺賊,擢守備,賜花翎。及城陷,忠源揮之去。五年,從忠濬復廬州,功多,擢游擊,賜號毅勇巴圖魯。忠濬假歸,代統其眾。六年,從和春克三河、巢縣,累擢副將。從秦定三規桐城,建議出奇兵夾擊,連破賊營十有六,進逼城下,賊大出,迎擊,進至東門外,躍馬越壕擒賊將,砲丸中左腋,殞於陣。予雲騎尉世職,謚忠節。忠濟、忠信並附祀忠源專祠。

羅澤南,字仲岳,湖南湘鄉人。諸生,講學鄉里,從游甚眾。咸豐元年,舉孝廉方正。二年,粵匪犯長沙,澤南在籍倡辦團練。三年,以勞敘訓導。曾國籓奉命督鄉兵,檄剿平桂東土匪,擢知縣。江忠源援江西,乞師於國籓,乃令澤南率以往。所部多起書生,初臨行陣,戰南昌城下,爭奮搏,死者數人。國籓聞之,喜曰:「湘軍果可用。」及圍解,剿安福土匪,以三百人破賊數千,擢同知直隸州。歸湖南,剿平永興土匪,所部增至千人,屯衡州。與國籓簡軍實,更營制,教練歷半載。

四年六月,偕塔齊布進攻岳州,以大橋為賊所必爭,堅扼不動,伺便突出擊之,三戰皆捷,殲賊千。閏七月,破高橋賊壘九,賊退踞城陵磯,偕塔齊布乘勝進擊,連破賊營,賊遂遁走,擢知府,賜花翎。自是湘軍名始播,以澤南與塔齊布並稱。轉戰而東,復崇陽,擊走咸寧賊,再敗之金牛,進駐紫坊。曾國籓會諸將於金口,議攻武昌。澤南繪圖獻方略,謂由紫坊出武昌有二道,請以塔齊布扼洪山,而自攻花園。賊萬餘踞花園,築堅壘,一枕大江,一瀕青林湖,一跨長堤,深溝重柵,峙江東岸,與蝦蟆磯對壘。列巨砲向江內外,分阻水陸兩路。澤南率隊直趨花園,賊憑木城發砲。士卒蛇行而進,三伏三起,已逼賊壘,分兵奪賊舟,舟賊退,營賊亦亂,三壘同下。翌日又破占魚套賊營,其竄洪山者,為塔齊布所扼,賊夜棄城走。武昌、漢陽皆復,距會議僅七日。捷聞,以道員記名,尋授浙江寧紹臺道,國籓請仍留軍。

賊據興國,分陷大冶。澤南馳克興國,塔齊布亦克武昌、大冶,乃規取田家鎮。賊以鐵鎖截水師,而踞半壁山為犄角,夾江而守。澤南進駐馬嶺坳,距半壁山三里許。賊數千突來犯,而由田鎮渡江來援者近萬人。澤南兵僅二千,令堅伏,度賊懈,奮擊,賊大潰,後路為我軍所阻,墜崖死者數千,遂奪半壁山,水師斷橫江鐵鎖,燔賊舟,克田家鎮,賜號普鏗額巴圖魯,加按察使銜。時議水陸軍分三路進剿,總督楊霈督江北岸軍,澤南偕塔齊布攻其南,曾國籓督水師循江下。霈不能軍,賊復北趨,乃偕塔齊布改北渡江,復廣濟、黃梅。賊退踞孔隴驛、小池口,澤南約諸軍會攻。渡江未半,賊來犯,軍少卻,澤南傷臂,仍指揮沖突,分兵破街口賊壘,賊酋羅大綱引去。是役也,五千人破賊二萬,賊乃盡撤沿江諸營,並守九江。塔齊布圍攻之,澤南別剿盔山,遏湖口援賊。會水師入鄱陽湖,為賊所襲,輜重皆失。國籓馳入澤南營,而水師阻湖口不得出。

五年,湖北官軍屢敗,武昌復陷。澤南從國籓入南昌,赴援饒州,戰於陳家山、大松林,大破賊,復弋陽。又援廣信,破賊於城西烏石山,復之。連復興安、德興、浮梁,進剿義寧。敗賊於梁口、鼇嶺,復義寧,加布政使銜。澤南見江西軍事不得要領,上書國籓,略曰:「九江逼近江寧,兼牽制武昌,故賊以全力爭之。犯弋陽,援廣信,從信水下彭蠡,抄我師之右;據義寧,守梅嶺,從修水下彭蠡,抄我師之左。今兩處平定,九江門戶漸固,惟湖北通城等處群盜如毛。江西之義寧、武寧,湖南之平江、巴陵,終無安枕之日。欲制九江之命,宜從武昌而下;如解武昌之圍,宜從崇、通而入。為今之計,當以湖口水師、九江陸師截賊船之上下,更選勁旅掃崇、通以進武昌,由武昌以規九江。東南全局,庶有轉機。」國籓據以上聞,遂命澤南移師湖北會剿,以塔齊布舊將彭三元、普承堯所部寶勇隸之,凡五千人。

九月,至通城。賊號數萬,皆烏合,一戰而潰。進奪桂口要隘,克崇陽,駐軍羊樓峒。悍賊韋俊、石達開合黨二萬餘自蒲圻來犯,截擊走之。胡林翼來勞師,合攻蒲圻,復其城,乘霧進克咸寧。自是武昌以南無賊蹤。十一月,師抵紫坊,與林翼議進取次第。澤南屯洪山,林翼屯城南堤上,水師駐金口。賊於城外築堅壘十三,與城埒。初戰,賊二萬出十字街,林翼與交綏,數卻數進。澤南與李續賓分兩路潛抄賊壘,破十字街營,盡毀城東南諸壘。八步街口為我軍通江要路,塘角為賊糧運所出,先後攻破之,焚其船廠,環西北賊壘亦盡。賊又由望山門外葺石壘二,揮軍蹋平之;又迭於窯灣、塘角逐賊,殲戮數千,賊遂閉城不出。

石達開自崇陽敗後,竄入江西,勢復張。曾國籓檄澤南回援,澤南以武漢為南北樞紐,若湘勇驟撤,胡林翼一軍不能獨立,現在賊糧將盡,功在垂成,舍之非計。其父年八十,貽書軍中勖以忠義,林翼以聞,六年二月,詔特予澤南祖父母、父母二品封典,以示旌異。三月,賊開門出撲,澤南親督戰。援賊大隊繼至,我軍自洪山馳下,奮擊追逐,直抵城下,飛砲中澤南左額,血流被面。駐馬一時許,歸洪山,猶危坐營外,指畫戰狀。翌日,卒於軍。文宗震悼,詔依巡撫例議恤。賜其父嘉旦頭品頂戴,子兆作、兆升皆舉人,予騎都尉世職。入祀昭忠祠,本籍、湖北、江西建立專祠,謚忠節。及江南平,穆宗追念前勞,加一雲騎尉世職。

澤南所著有小學韻語、西銘講義、周易附說、人極衍義、姚江學辨、方輿要覽諸書。體用兼備,一宗程、硃,學者稱羅山先生。嘗論兵略,謂大學首章「知止」數語盡之,左傳「再衰」、「三竭」之言,其注腳也。弟子從軍多成名將,最著者李續賓、李續宜、王珍、劉騰鴻、蔣益澧,皆自有傳。其早死兵事名未顯者,有鍾近衡,少事澤南,以克己自勵,日記言動,有過立起自責。澤南語劉蓉曰:「吾門為己之學,鍾生其庶幾乎!」從平郴、桂土匪,敘從九品。咸豐四年,粵匪由江寧上竄犯岳州,偕弟近濂各將五百人從王珍破賊於靖港,追至蒲圻羊樓峒,戰失利,死之。王珍退保岳州,賊又大至,近濂亦戰歿。易良幹、謝邦翰,並戰死南昌城下。邦翰死後,李續賓代領其眾,所稱「湘右營」者是也。諸人皆湘鄉人,後並附祀澤南專祠。

論曰:湖南募勇出境剿賊,自江忠源始。曾國籓立湘軍,則羅澤南實左右之。樸誠勇敢之風,皆二人所提倡也。忠源受知於文宗,已大用而遽殞。澤南定力爭上游之策,功未竟而身殲,天下惜之。忠源言兵事一疏,澤南籌援鄂一書,為大局成敗所關,並列之以存龜鑒。此大將風規,不第為楚材之弁冕已。


\end{pinyinscope}