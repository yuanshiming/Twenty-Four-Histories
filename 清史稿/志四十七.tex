\article{志四十七}

\begin{pinyinscope}
地理十九

△廣東

廣東:禹貢揚州之南裔。明置布政使司,治廣州。清初因明制,定為省。雍正中,升連州及程鄉為嘉應州,並直隸。嘉慶中,南雄降直隸州,尋並復故,增佛岡,南雄仍降州,增連山。同治中,陽江升,增赤溪。光緒中,升欽州、崖州,降萬州。為道六,為府九,直隸州七,直隸三,散州四,散一,縣七十九。東至福建詔南;千里。西至廣西宣化;千五百里。南至海;三百里。北至湖南桂陽;七百八十里。東南至海;二百八十里。西南至崖州海;二千四百里。東北至江西長寧;八百里。西北至廣西賀。七百三十里。廣二千五百里,袤一千八百里。東北距京師七千五百七十里。宣統三年,編戶五百四萬一千七百八十,口二千八百一萬五百六十四。其名山:靈洲、黃嶺、羅浮。其巨川:西江、北江、東江。鐵路:粵漢南段,自廣州西逕三水,又北清遠、英德、曲江至樂昌,與湖南興寧路接。

廣州府:沖,繁,疲,難。隸廣肇羅道。兩廣總督舊駐肇慶,乾隆十一年徙。光緒二十四年裁巡撫,尋復。三十一年,仍與粵海關監督、糧道同裁。布政、提學、提法、鹽運四司,巡警、勸業二道,廣州將軍,滿洲、漢軍副都統,廣東水師提督駐。明領縣十三。康熙中增置花縣。廣四百二十里,袤五百二十二里。北極高二十三度十一分。京師偏西三度三十三分。領縣十四。有三:曰佛山,雍正十一年置;曰前山,乾隆八年置;曰虎門,道光二十五年置。有粵海關,康熙二十四年置。廣州商埠,道光二十二年英南京條約訂開。南海沖,繁,疲,難。倚。府西偏。粵之山,五嶺據其三。北:越秀。西北:靈洲。西南:西樵山。北江自三水入,東南流,東別出為紫洞水,至番禺,合珠江入順德。西江自三水入,東南過九江,亦入順德。西北馬逕水,首受蘆包水,南與三江水會。屈東北流,左合黃洞水,南流溪水自番禺西南注之。南出石門山為石門水,過府治西南,屈東為珠江,入番禺。有九江浦主簿。三江、金利、神安、黃鼎、江浦、五斗口六巡司。西粵漢,西南三佛鐵路。番禺沖,繁,難。倚。在城有番、禺二山,縣以是名。北:白雲。東南:浮練,一名浮蓮岡。東南濱海。南有珠江,上承南海石門水,東南流,歧為二,至長洲復合。又東南為波羅江,左合東江,為三江口。又東南,獅子洋合沙灣水入於海。有獅子營。西:永靖營。有慕德、鹿步、沙灣、茭塘四巡司。魚雷營有船塢在黃埔。番禺、五羊二驛。東廣九,西北粵漢鐵路。順德繁,疲,難。府南百里。北:都寧。西:天湖。西北:西淋山。北江自南海入,為河澎海,東南流,屈北為扶閭海,又東疊石海,東別為沙灣水,合珠江。至半江為板沙海,入香山。西江自南海入,東別出為甘竹灘水,與板沙海合,過仰船岡,別出為仰船海,至新會入海。縣丞一,治容奇。有紫泥、江村、馬寧,又北都寧四巡司。有甘竹商埠,光緒二十三年中英緬甸條約開。東莞沖,繁,疲,難。府東南百八十里。南:黃嶺。東南:寶山。西南濱海。海中秀山,東西峙若門然,曰虎頭門,珠江出其中,又謂之珠江口。有砲臺五:曰威遠、上橫檔、下橫檔、大角、沙角。東江自博羅入,合瀝林水、九江水,西過黃家山,南別出為到湧水,會珠江,皆入海。石龍鎮,縣丞一。京山、缺口、中堂巡司三。鐵岡驛一。廣九鐵路。從化簡。府北百三十里。東北:五指山。又十八山,流溪水出焉,南合陳峒水、玉溪,合過縣治東南,左納曲江,右納黎塘,至番禺入石門。有流溪巡司。石岐驛。龍門簡。府東北二百一十里。西:藍糞山。西北:分水凹山。山西之水匯流溪入北江,山東之水匯西林水入東江。西林水一名九淋水,出西北三角山,合高明、白沙,屈西南,納群溪水、永清水,入增城為增江。有熱水湖在西北。有廟子角巡司,後遷永清墟。新寧疲,難。府西南三百六十里。北:三臺。東:百峰山。西南:大隆山。南濱海。海中有上川洲、下川洲。北:長沙河,即恩平江,自開平入,東南流,合南門河,西北合紫霞河,入新會。東南:泥湧河,南合牛角水,至烽火角入海。西:那扶水,亦南至獅子洲入海。又有潭■E1河,康熙二十六年總督吳在南鑿,西引泥湧河,東北達新會崖門,以通舟楫。溫泉、醴泉在西南。廣海寨,縣丞駐。有上川鹽巡司。有公益商埠。寧陽鐵路。增城簡。府東百六十二里。西:雲母。西南:南樵。東北:羅浮山。東江自博羅入,西流屈南,至番禺合珠江。增江上流為龍門水,南與派潭水合,又南至三江口,右納澄溪水,左納九曲水,過縣治東南,分流入東江。綏福水出西北青幽山,亦東南入於東江。有茅田巡司,新塘墟主簿。香山疲,繁,難。府東南二百廿里。北:浮虛。東南:五桂。又濠鏡澳山,山突出海中成半島形,曰澳門。光緒十三年入於葡萄牙。其北濠鏡澳關。又西,拱北灣有關。東南濱海。海中有東澳山、九星山,下曰九星洋。又有老萬、九澳、橫琴、三灶、浪白諸山在海中。西北:板沙海,自順德入,東南至潭洲。木頭海首受仰船水,東南分流入海。古鎮海首受西江,亦自順德入,東南至螺洲,與石岐水合,南出磨刀門入海。前山寨城,縣丞駐。黃梁都城,都司、巡檢駐。又淇澳、香山、黃圃三巡司。有香洲商埠,宣統元年奏開。新會繁,疲,難。府西南二百三十里。北:黃雲、圭峰。東南:崖山,與西南湯瓶嘴山對峙,熊海出其中,曰崖門。南濱海。西江自南海入,為天河海,東屈而南,過豬頭山,歧為二:東南出者曰荷塘水,合古鎮海東南入香山,又西別為外海水,西南至虎跳門入海;西南出者曰分水江,合潖水,南過江門,注熊海。又西,恩平江自開平入,與潭江合,東北流,為青膽洋,左納橋亭水,東南合分水江,出崖門入海。江門,縣丞駐。潮連、牛肚灣、沙村三巡司。大瓦司,廢。蜆岡、東亭二驛。江門商埠,光緒二十八年中英商約訂開。有寧陽鐵路。三水沖,難。府西北二百七十里。南:昆都。北:龍坡山。北江西南流,至胥江口東別出為蘆包水,又西南至四會,合綏江,別為思賢■E1水,會西江。東過縣治南,為肄江,至於西南潭入南海。北江自西南潭別出為三江水,與蘆包水合,至南海,出石門,其下流為珠江。西有西江,自高要入。青岐水首受綏江,東南過金洲山,亦入南海。西南鎮,縣丞駐。有胥江、三水二巡司。三水口亦名河口,有商埠,光緒二十三年中英緬甸條約訂開。有三水、西南二驛。三佛鐵路。清遠沖,難。府北三百四十里。西:秦王。東:中宿峽,一名飛來峽。北江自英德入,西南流,潖江水東來注之,曰潖五江口。至縣治西南,合政賓江。屈南,右納山塘水,左納大燕水,過回岐山,入三水。有回岐、潖江、濱江三巡司。有清遠驛,安遠廢驛。有粵漢鐵路。新安疲,難。府東南二百六十里。康熙六年省入東莞,八年復置。南:杯渡,一名聖山,古謂之屯門山。東南:官富。東北:大鵬山。其南曰老大鵬山,有東湧所城。東、西、南三面濱海。海中有零丁山,其下曰零丁洋。又南,頭沱濘、佛堂門、急水門、大嶼山、榕樹灣等澳。西北:永平河,首受東莞九江水,東南至碧頭汛入海。大鵬所,縣丞駐。有福永、九龍二巡司。其南:香港島,道光二十二年割於英。咸豐十年,又割九龍寨屬焉。光緒二十四年,又拓租九龍司屬地二百方英里,訂九十九年之約,置九龍關榷稅。有廣九鐵路。花簡。府北九十里。康熙二十四年,以番禺縣平嶺置,析南海縣地益之,來屬。東北:花山,縣以是名。西北:盤古洞,黃洞水出焉,西南流,右納橫潭水、羅洞水,屈南曰泥水,出清遠,自西北來注之,又東南入南海。有獅嶺、水西二巡司。有粵漢鐵路。

肇慶府:沖,繁,疲,難。廣肇羅道治所。初沿明制,領州一,縣十一。雍正九年增鶴山。同治九年,陽江升直隸。光緒三十二年,復改直隸州,陽春、恩平割隸。東距省治二百九十里。廣一百一十九里,袤三百九十五里。北極高二十三度五分。京師偏西四度八分。領州一,縣九。高要沖,繁,疲,難。倚。北:定山。東北:頂湖,有高峽。西北:騰豺山。西江自德慶入縣西北境曰端溪,北屈而東,都偃水、筍洞水南流入焉。東合大湘水,屈南,合小湘水,過府治南,新興江自西南來注之,謂之新江口。又東北,與宋崇水合。過羚羊峽,左納長利水,右納蒼梧水,入三水。縣丞駐金利墟。有橫槎、祿步二巡司。高要、新村二驛。四會簡。府東北百三十里。北:金雞山。南:貞山。東南:北江自三水入。西北:綏江,一名綏建水,自廣寧入,東南流,至縣治東南,龍江水西北來注之。過消息嶺,南別出為青岐水,至三水合西江,東至南津口合北江,入三水。有南津巡司。新興沖,難。府西南百三十里。北:巨福、雲斛。南:龍山。東北:利山。北:新興江,源出縣南六阬頂山,屈西北流,入東安。逕縣西南為錦水,東北至洞口,盧溪水北流合焉。又北與通利水合,是為新興江也。又西北,入東安。西南:立將巡司治天堂墟。有腰古廢驛。高明疲。府東南七十里。西北:老香山。東北:凌雲。西南:表山。西江自三水入。南滄江一名倉步水,出高要,東南流,合雲宿水、屏山水,逕縣治東南,左納北港水,右納清泰水,又東南合西江水入南海。有三洲巡司。廣寧疲。府西北二百九十里。東北:大羅山。西南:高望山。西:綏江自廣西懷集入,南流出峽山,南鄉水東北流合焉。又南,與顧水合,屈東南,右納金場水、新招水,左納東鄉水、扶羅水,東南入四會。又龍江水出東北石馬山,亦至四會合綏江。開平疲,難。府東南二百六十里。順治六年,以新興縣開平屯置,析新會、恩平二縣地益之,來屬。東北:梁金山。西南:北獵山、羅漢山。蜆江水上承恩平江,東南流,右納長塘水,東南至赤磡為赤磡水。北雙橋水,南流入焉,至縣治南,與獨鶴水合,是為尖石水也。又東南流為長沙河,過赤水口,入新會。有松柏、沙岡二巡司。鶴山疲,難。府東南二百六十里。雍正九年,以廣州府新會縣大官田置,析開平縣地益之,來屬。在城有鶴山,縣以是名。東北:昆侖。西北:雲宿。西江自南海入,過縣東北境曰古勞河,又曰蘇海,合古勞小河,東過大雁山入新會。潭江出縣西馬耳山,東南至鑼鼓潭,屈西錯入開平,至新會合恩平江。官田水出東北嶂背山,東南與嵐洞水合,入新會為橋亭水也。雙橋水出西北雲蓋村,西南流,至水坪墟曰水坪江,西南過胡盧山入開平。有雙橋、藥徑二巡司。德慶州沖。府西百八十里。西北:香山,一名利人山。東北:西源山。南:西江自封川入,東流過錦石山曰錦水,又東與淥水合,過州治,端溪水南流入焉。又東過南江口,合馬墟水、悅城水。悅城水上源曰靈溪,又曰靈陵水也,東北入高要。有悅城巡司。德慶驛。舊壽康驛,廢。封川沖。府西北三百三十里。東:封門山。東北:白馬、留連大山。西江,古鬱水,合黔水、桂水自廣西蒼梧入,東南至靈州。賀江自開建入,左合寧洞、文德水,右合東安江,又東南,右納蟠龍,左世陽水,逕圓珠山,屈西南入德慶。淥水出東北豐壽山,亦南至德慶入西江。有文德巡司。封川驛。舊麟山驛,廢。開建簡。府西北四百一十里。西北:圓珠山。東北:忠讜山。開江在西,即賀江,古謂之封溪水,自廣西賀縣入,東南至潭霜山,潭霜水合金裝水南流入焉。又南,與蓮塘水合,過縣治西南,左納金縷水、黎水,右納大小玉水,屈東南入封川。

羅定直隸州:繁,疲,難。隸廣肇羅道。東北距省治六百八十九里。廣一百八十四里,袤二百里。北極高二十三度四十二分。京師偏西五度十三分。沿明制,領縣二。西:云致山。西南:雲際山,一名雲沙山。瀧水源出西寧縣榃棉村,東北流,入州西南分界墟,東南過羅鏡所城,屈北與石印水合,又西南合三都水,過州治,入西寧為南江。東水出州南沙墟,亦東北入西寧合南江。州判治羅鏡墟。晉康巡司治連灘墟。有晉康廢驛。東安難。州東北百六十里。西南:雲霧山。西江自西寧入,東南至絳水口,大絳水自西南來注之,又東北入高要。東南:新興江出新興,東北流,左納客朗水,過腰古汛,入高要合西江。有西山巡司。西寧難。州北百二十里。北:玉枕山。西江自封川入,至羅旁口,文昌水合寶珠水、桂河水北流入焉。又南,南江上源瀧水出西南榃棉村,東北入羅定,過連灘墟合西江,入於東安。西南:到沙水,出羅雲山,東南至羅定入瀧水。又西,蟠龍水,出大筍嶺,東北入封川。有夜護巡司。都城巡司,廢。

佛岡直隸:難。隸廣肇羅道。明大埔坪地,分屬清遠、英德。雍正九年置同知,隸廣州府。乾隆七年廢。嘉慶十六年復置,更名。南距省治四百四十里。廣五十七里,袤四十八里。北極高二十三度五十分。京師偏西二度五十九分。北:觀音山。東北:獨凰山,水頭汛河出焉,北合高江水,至燕嶺墟為燕嶺水。又西北至英德,合羅紋水,入翁江。吉河水亦出獨凰山,迤西流,神逕水自北來注之,南別出為達溪,瀦為潭。過治北,屈南,右納黃沙河,出大廟峽入清遠。黃華水出東南羊角山,亦西南入清遠,合於吉河水。其下流是為潖江也。

赤溪直隸:要。隸廣肇羅道。同治七年,析新寧縣赤溪、曹沖等地置。東北距省治四百一十五里。廣二十里,袤二十里。北極高二十一度五十四分。京師偏西三度三十五分。東、西、南三面濱海。南:曹沖山。西南:銅鼓山,其下曰銅鼓海。又有黃茅、青洲、大金、小金諸山,在海中。

韶州府:沖,疲,難。韶連道治所。南距省治八百七十里。廣一百九十五里,袤三百一十一里。北極高二十四度五十五分。京師偏西三度二十一分。領縣六。有太平橋鈔關,舊在南雄,後遷府治西南。又有太平分關,在英德。曲江繁,難。倚。北:浮岳。東北:韶石。西:芙蓉山。東南:南華山。湞水在東,一名湘江,自始興入,西南流,合錦江、零溪,逕府治東南,武水自北來會,曰曲江,又謂之始興大江也。又西南,過虎榜山,屈東南,右納瀧水,左納曹溪水、宣溪水,南入英德為北江。縣丞治蓮花嶺村。有濛浬、平圃二巡司。曲江縣驛。舊芙蓉驛,廢。有粵漢鐵路。樂昌沖,難。府西北八十里。東:昌山,縣以是名。北:桂山。東北:冷君。西北:九峰山。武水在西,一名虎溪,古謂之溱水,出湖南臨武,東北至宜章。屈而南,入縣西北境,武陽溪自乳源東流合焉。屈東南,歷藍毫山,為三瀧水,與羅渡水、九峰水合。過縣治西南,蓮花江分流注之。又東,屈而南,左納長垑水,右納楊溪水,入曲江。有九峰、羅家渡二巡司。有粵漢鐵路。仁化簡。府東北百里。西北:黃嶺山。東南:丹霞山。東:錦江出分水坳,西南至恩口,與恩溪水合,即藍田水也。西南流,左納扶溪水、康溪水,過縣治東南,澌溪水合潼陽水自西北來注之。屈東南入曲江。有扶溪巡司。仁化縣驛。乳源簡。府西九十里。北:雲門山。西南:臘嶺。武陽溪自湖南宜章入,東北逕武陽司,右合七姑灘水,左納瀔溪,屈東至樂昌入武水。楊溪水出西北神仙坪,亦至樂昌入武水。瀧水一名洲頭水,出西南梯子山,北屈而東,左納員子山水,右納湯盤水,過縣治南,大布水北流合焉,又東南入於曲江。南有武陽巡司。世襲撫瑤一,管埠巿。翁源沖,難。府東南百八十里。嘉慶十六年改隸江西南安府,十七年仍來屬。北:雞籠。東:玉華。東北:婆髻山,羅江水所出,西南逕翁山南,浦水自東南來注之。屈南,右納芙蓉水,左納龍仙水,又西南與周陂水合,迤西過三華鎮入英德。又西,太平水,一名江鎮水,出東北桂袨山,南流至英德合羅江水,是為翁江也。桂山、磜下二巡司。英德沖,難。府南二百二十里。北:英山。南:南山。又南:皋石山,一名湞陽峽。北江在北,自曲江入,過湞石山,屈西至縣治東南,東有翁江,右合曲潭水,左合羅紋水,西南流合焉。南至洸口,洭水合波羅水自西北來會。洭水者,湟水也,亦曰洸水,東南流入清遠。有洸口、象岡二巡司。英德縣驛。舊湞陽驛,廢。有粵漢鐵路。

南雄直隸州:沖,繁,疲。隸南韶連道。初沿明制為府,領縣二,治保昌。嘉慶十一年,降為直隸州,省保昌縣。十六年,復升為府。十七年,又降為直隸州。西南距省治千一百七十里。廣一百七十里,袤一百二十一里。北極高二十五度十五分。京師偏西二度三十分。領縣一。大庾嶺在東北,一名梅嶺,有梅關。東:天柱。東南:青嶂山。南有湞水,出東北油山,南逕漿田鎮,與昌水合。西南流,左合平田水、芙蓉水,右合東溪水,至長浦橋,北坑水合橫水南流入焉。水出梅嶺,又謂之大庾河水也。又西合長潭水,過州治南,樓船水自西北來注之,西南與修仁水合。又北納半徑水,入始興。又西北,分水坳,石峽水出,為康溪水,入仁化。有平田、紅梅、百順三巡司。有保昌驛。舊臨江驛,廢。始興沖,繁。北:丹鳳山。南:機山。北:湞水自州入,西南至圓嶺鋪,躍溪水北流合焉。又南,墨江,出西南沙子嶺,迤東為清化水,屈西北為涼傘水,右合翔水為始興水,即古斜階水也。又西北過縣治南,與官石水合,又西北合水貞水入曲江。有清化徑巡司。在城驛。

連州直隸州:沖,難。隸南韶連道。初沿明制,隸廣州府。雍正五年,升為直隸州,其陽山、連山割隸。嘉慶中,連山直隸。東南距省治七百六十里。廣八十里,袤一百六十八里。北極高二十四度四十八分。京師偏西四度十七分。領縣一。南:楞枷,一名貞女山。西南:昆湖。西北:桂陽。湟水在西,一名洭水,漢志以為匯水。上源為盧溪,出西北黃蘗嶺,又曰蘗水,南迤東過圭峰山,東北合奉化、潭源、黃嬌諸水,至州治西南,高良水自連山西來注之,東南過同冠峽,入陽山。州判治皇子墟。有硃岡巡司。陽山難。州東南二百里。雍正十五年自廣州府來屬。北:騎田嶺。西北:陽巖。東北:寶源山。湟水自州入,一名陽谿,南合同冠水,又東南過縣治南,通儒水自馬丁嶺東流注之,又東與青蓮水合。水出縣北大陂墟,又謂之大陂水也。又東南,過三峽入英德。有淇潭、七鞏二巡司。

連山直隸:繁,難。隸南韶連道。本連山縣,隸廣州府。雍正五年,改隸連州。嘉慶二十一年,升為綏瑤。東南距省治八百七十里。廣一百里,袤一百二十六里。北極高二十四度四十九分。京師偏西四度三十五分。北:昆湖山。西北:鍾留、大霧。南:黃帝源山,一名黃連山,中有大排瑤五,小排瑤二十四。高良水在南,一名大獲水,上源為橫水,出西北天堂嶺,東南流,逕治南,屈東北,與茂古水合。過雞鳴關入連州,合於湟水。又,上吉水出西分水坳,西南流,至木羌墟,八排瑤水自東南來注之,屈西北,過鐘山,入廣西賀縣,又為賀江別源也。有宜善巡司。

惠州府:沖,繁,難。隸惠潮嘉道。西距省治三百九十里。廣四百五十里,袤四百里。北極高二十四度八分。京師偏西二度三十七分。領州一,縣九。有通判一,治碣石衛城,道光二十一年置。有惠州商埠,光緒二十八年中英商約訂開。歸善沖,繁,難。倚。東北:歸化山,一名雞籠山。東南:平海山。東南濱海,中有霞湧、吉頭、澳頭諸港。東江在北,一名龍江,自河源入,西南流,至府治東北。西江出縣東龍頭石山,西南合長塘水、上下淮水,入博羅。西豐湖、潼湖,皆引流入於東江。內外管、平山、平政、平海、碧甲五巡司。欣樂司,廢。博羅繁,疲。府西北三十里。西北:羅浮山。東北:象山。東江自河源入,中與歸善分界。合公莊水,逕縣治南,右納榕溪水,過缸瓦洲入東莞。其支渠,西北至黃家山,與羅陽水合,過石灣鎮入增城。有石灣、善政、蘇州三巡司。莫村廢驛。長寧簡。府西北四百里。北:玉女峰、雲髻山。東北:雪洞山。新豐水在南,出西北分水凹,屈東與沙羅山水合。一東逕縣治,又東,左合羌阬水,逕馬頭墟,左納密溪、大席、忠信水,右納錫場水,過立溪口,至河源入東江。羅紋水出縣西宋洞山,西北至來石汛,屈西南入英德合翁江。有乍坪巡司。永安簡。府東北二百里。西南:越王山。東南:南嶺。南:秋香江,一名欖溪,出縣東雞公嶺,西南流,與南山水合,至河源入東江。又西,神江、義容江從之。南琴江,源出公阬嶂,南流至米潭,又東北入長樂。北琴江亦至長樂,合於南琴江,其下流是為梅江也。有馴雉里、寬仁里二巡司。海豐難。府東南三百里。東:龍山。西北:五坡嶺。南濱海。有麗江,一名長沙港,上流曰龍津水,出西北蓮花山,東南會黃姜水,南屈而西,至鹿鏡山,匯為青草澳,合大液水,逕大金籠山入海。東北有熱水,南流過九龍山,屈東為大德港,至陸豐,合內河水入海。西:鳳河水,南與鵝埠水合為小漠港入海。東:汕尾鎮,縣丞駐。有鵝埠巡司。平安廢驛。陸豐難。府東南三百五十里。雍正九年析海豐縣地置,治東海■E1,來屬。東北:內洋山。南:虎頭山。濱海。北:內河水,一名羅江,源出東北旗頭嶂,與吉石溪合。南過石頭山,分流,至大德港、烏敢港入海。又東:草洋水,東南流,屈西為華清港,至甲子港入海。上沙墟水出東北赤嶺,至普寧合南溪。有甲子、黃沙阬、河田三巡司。有法留鋪在縣西,道接海豐,又東至惠來百六十里。有鹽場三:曰石橋、海甲、小靖。龍川簡。府東北四百里。東:霍山。東北:龍穴,一名龍川山。西北:山。龍川水在東,又名合河,上源為定南水,自和平入,東南合河口會杜田河,西南流,與浰溪合,逕縣治東南,雷江水南流入焉。又西南,合合溪入河源為東江。又練溪出東北鵝石嶂,西南流,右納通衢水,入長樂。有老隆、通衢、十一都三巡司。雷鄉廢驛。連平州簡。府北四百里。東北:九連山。南:戈羅、筆山。有密溪水,出分水坳,東南流,與楊梅坪水合。又過州治南,納內管水、九嶺水,東南至長寧入新豐水。東大席水從之。又忠信水,西南入河源。有忠信、上坪、長吉三巡司。河源沖,難。府北百五十五里。西:桂山。東:古云。東北:藍溪山。東江一名槎江,西南至藍鎮墟,左納藍溪水,右納曾田水,又西南與康禾水合。過縣治東南,新豐江自長寧東來注之,西南合秋香江入歸善。西北:忠信水,出連平,西南過楓木鎮,合二龍岡水,至長寧入新豐水。噩湖東為河源舊城,今謂之下城也。有藍口巡司。義合、寶江二驛,後廢。和平簡。府東北四百二十里。北:紫雲山。西:九連山。東北:定南水,自江西定南入,東南流,右納烏虎水,又東北過江口,屈東南入龍川。浰水出西北羊角山,東南至合水口,湯坊水自東北來注之,過林鎮墟,與九龍水合。屈東至龍川,入於定南水。有浰頭巡司,後廢。

潮州府:沖,繁,難。隸惠潮嘉道。西距省治千一百八十五里。廣二百五十五里,袤三百里。北極高二十三度二十七分。京師偏東十二分。領一,縣九。有黃岡同知,康熙五十七年置。有通判一,治菴埠鎮。海陽沖,繁,難。倚。東:韓山。南:桑浦山。西:湖山。西北:海陽山。韓江在東,一名意溪,上承留隍河,自豐順入,東南過蒲都山,分流為三:正渠東南流為東溪;東北出者曰涸溪,舊名噩溪,屈東南,過七屏山至饒平為後溪;西南出者曰西溪,過府治東南,右納白茫洲水,屈南,北溪水自揭陽來注之,屈東,與東溪合,南流入於澄海。縣丞一,治菴埠鎮。有浮洋巡司。鳳城廢驛。豐順疲,難。府西北百九十里。乾隆三年以海陽縣豐順鎮置,析嘉應州及揭陽、大埔二縣地益之,來屬。南:瘦牛山,一名雲落山。東北:銅鼓嶂。東:留隍河自大埔入,西南合豐溪水,又南合九河水,入海陽為韓江。又南湯溪,一名湯阬水,下流至揭陽為北溪。有湯阬、留隍二巡司。潮陽繁,疲,難。府南百四十里。東:東山。東南:錢澳。西北:曾山,一名雙髻山。北、東、南三面濱海。海中有東沙島。練江在西,首受揭陽南溪,自普寧入,至縣治南合後溪,西南出海門入海。西北:後溪水亦出揭陽,東南過石井山為鋪前水,過潯洄山,別出為後溪,引流入練江,過磊口山為招沙水,屈南,至河渡門入海。有招寧、吉安、門闢三巡司。有靈山驛。揭陽繁,難。府西南八十里。西:獨山。西北:揭陽山。東南濱海。南:南溪,出縣西明山,東南流入普寧,又東北入縣。西南:古溪水北流合焉。迤東逕縣治南,與北溪別派合,東南過雙溪口入海。北溪出豐順南,屈東分流注南溪,又東北至海陽合韓江。縣丞駐棉湖寨。有河婆、北寨二巡司。饒平難。府東北百五十里。北:九峻。西:鳳凰山。東南:紅螺山。南濱海。海中有井洲、信洲、浮潯、牛心石諸澳。東南:黃岡溪,出東北界山,西屈而南至望海嶺,姚源水自西北來注之,南與飛龍徑水合,屈東南為大石溪,至黃岡鎮分流入海。西南:韓江,自海陽入,合後溪水,東入澄海。東南有黃岡鎮城,其東南為大城所城,又南為柘林,有柘林巡司。海山、東界二鹽場。惠來難。府西南二百七十里。西:龍溪。西南:釣鼇山。東南濱海。南:神泉港,上流為龍江溪,出西北南陽山,東南合葵潭水、梅林水,迤東過龍江關,林招溪自西北來注之,東注神泉港。東福溪、祿昌溪皆流合焉,又南入海。有神泉、葵潭二巡司。北山驛。惠來柵鹽場。大埔簡。府東北百六十里。西:陰那山。汀水自福建上杭入,一名神泉河,東南流,逕縣治東北,屈西,漳溪水東流北屈注之。又西過大河山,屈南與小河水合,又南至三河市,清遠河西北流合焉。河出福建平和,其上源曰河頭溪也,東南入豐順。有三河、白堠二巡司。烏槎司一,廢。澄海繁,難。府東南六十里。康熙五年省入海陽縣,八年復置。北:管隴山。西南:龍泉山。東南濱海。海中有鳳嶼,其下曰侍郎洲、大萊蕪、小萊蕪山。西北:橫隴溪,首受東溪,自海陽入,西南別出為新港水,分流入海。正渠迤東流,南別出為玉帶溪,至縣治東南入海。又東逕獅子山,與饒平後溪合,東至東隴關為東隴港入海。有漳林、鮀浦二巡司。商埠曰沙汕頭,咸豐八年英天津條約訂開。有潮海關。潮汕鐵路。小江鹽場。普寧繁,疲,難。府西南百二十里。南:鐵山。西北:官人望山。南溪自揭陽入,歧為二:一東逕馬嘶巖山,東北入揭陽;一西南逕鯉湖埠為鯉湖水,屈東南,與上沙墟水合,過望夫石山,為寒婆徑水,東北為白阬湖,又東入潮陽為練江。又東:普寧港,一名通潮港,東北入揭陽為古溪。有雲落徑巡司。南澳中。府東南百五十里。本南澳鎮地。分四澳。雲、青二澳隸閩之詔安,隆、深二澳隸粵之饒平。雍正十年置海防同知,為南澳治,深澳來屬。南:金山。東南:雲蓋山。四面濱海。北臘嶼、虎嶼,在海中。西南有赤嶼、白嶼,其田產鹽。有南澳巡司。

嘉應直隸州:沖,繁,難。隸惠潮嘉道。舊程鄉縣,隸潮州府。雍正十一年,升為嘉應州,直隸廣東布政使司。嘉慶十二年,升為嘉應府,復置程鄉縣為府治。十七年,仍改為直隸州,省程鄉縣。西南距省治千二百八十二里。廣百五十七里,袤百五十四里。北極高二十四度十二分。京師偏西十九分。領縣四。東:百花。東南:酉陽,一名九峰山。東北:王壽山。南:梅江即興寧江,東北流,逕州治南,左納程江水,屈東與周溪水合,東北至丙市,石窟溪西北自鎮平來注之。東北合松源溪,屈東南,過蓬辣灘入大埔,是為小河水也。州同駐松口。有豐順、太平二巡司。程鄉、武寧二驛,後廢。長樂沖,難。州西南百一十里。舊隸惠州府。雍正十一年來隸。北:五華山。東南:嵩螺山。西南:龍村河自永安入,東北至琴口鄉,華陽水首受北琴江,東流合焉,東北至七都河口會岐嶺河。河出龍川曰練溪,其下流又謂之清溪也;又東北流為長樂河,入興寧。有十二都巡司。興寧難。州西七十里。舊隸惠州府。雍正十一年來屬。東:雞靈山。北:大望山。其西麓羅岡水,合龍歸水、楊梅砦水,西南流為大河水,又逕縣治西為西河,亦名通海河,屈東南至水口鎮,長樂河自西南來會,是為興寧江也,東北入嘉應為梅江。西北:杜田河,出江西長寧,西南過杜田汛入龍川。有十三都、水口二巡司。平遠簡。州西北七十里。舊隸潮州府。雍正十一年來屬。東北:頂山、五子石山。西:鳳頭嶂。其東麓曰分水坳,縣前水出焉,東南流,左納頂山水,過卓筆山,至福建武平,合於武平溪。又河頭溪,源出西南九鄉堡,東南過石鎮山,大拓水東流合焉,東南流為橫梁溪,與長田水合,東入鎮平為徐溪。有壩頭巡司。鎮平簡。州北六十里。舊隸潮州府。雍正十一年來屬。西:鐵山嶂。東:大峰嶂。西北:石窟溪出平遠,自福建武平入,合楊子山水,過縣治西,與東山水合,南至小誥山納徐溪,至嘉應入於梅溪。又東北:松源溪,源出玉華峰,亦至嘉應入梅溪。有羅岡巡司。

高州府:沖,繁,難。高雷陽道治所。東北距省治千六十里。廣三百一十五里,袤二百三十里。北極高二十一度四十九分。京師偏西五度四十分。領州一,縣五。有通判一,治梅菉。茂名繁,難。倚。高涼山在東北,州以是名。東山在東。南濱海。北:竇江自信宜入,東南流。左納雙柘水,至府治東北,鑒江水西流合焉,今又謂之石骨水也。屈西南,過那射嶺入化州。東南有浮山水,即三橋河,出電白,西南至赤嶺為赤嶺水,又西南入吳川。有赤水、平山二巡司。大陵廢驛。電白繁,疲,難。府東南百六十里。北:浮山。南:蓮頭山,其下曰蓮頭港。又西南,有赤水港。南濱海。有博賀島在海中。東北有儒垌河,源出分水凹,西南流,過望夫山曰望夫水,屈南與界頭河合,又南為五藍河,入於海。又三橋河,出東北木力嶺,西南至潭儒山為潭儒河,合龍珠河,西南入於茂名,其下流是為浮山水也。有沙瑯巡司。鹽場二,曰博茂、電茂。信宜難。府東北八十里。東:龍山。東北:雲開。西川水出大人山,西南過舊潭峨縣曰潭峨江,至縣治西南,東川水來會。屈南,過羅竇洞為竇江,又南入茂名。東:雙龍水出長充坑,西南至古丁墟,屈東入陽春,合雙■E1水。又東北,雙床水出大水嶺,南合吐珠水,屈東北流為石印水,至羅定入瀧水。又懷鄉水出東北黃陂嶺,會扶龍水、石人水,西北與響水合,為黃華江,入廣西岑溪。又金洞水出縣北雷公嶺,水西北至廣西容縣為渭龍江也。有懷鄉巡司。化州簡。府西南九十里。北:浮梁山。東北:龍王,一名來安山。茂名水在東北,即竇江,又東北有陵水,源出廣西北流,入,屈西南至合江墟,羅水亦自陸水入,其合流曰羅江,又謂之陵羅水也。屈東南,逕州治北合竇江,又東南流為平源江,入吳川。有梁家沙巡司。吳川簡。府西南百二十里。北:麗山。西北:特思山。東南濱海。南為利劍門,至匈州,又西南至於通明港,謂之廣州灣。光緒二十五年租於法。吳川水在東北,一名吳江,自化州入,東南過三江嶺,浮山水西流合焉。屈西南為木棉江,與平城江合,分流至限門港入海。石門港源出石城東橋水,東南流,山角水自東北來注之,又東南至麻斜入海。有塘綴巡司。鹽場一,曰茂暉。石城簡。府西南百九十里。北:謝建山。西南:敷復山,濱海。西有南廉江,即烏江,自廣西陸川入,西南流,至石角墟曰石角水,又西南與武陵江合,為合江,青榕水西流合焉。又西南為九洲江,賀江水自西北來注之,至鯉魚潭入海。又西:洗米河,出廣西博白,迤南流,為英羅港,入海。又東,東橋水,出雞頭嶺,東南過兩家灘,入吳川,是為石門港也。有凌綠巡司。息安廢驛。

雷州府:簡。隸高雷陽道。東南距省治千五百一十里。廣九十五里,袤二百二十九里。北極高二十度四十九分。京師偏西六度二十八分。領縣三。府境突出海中作半島形。東為廣州灣,西為東京灣,其南則瓊州海峽也。同知一,治海安所城,後廢。海康疲。倚。西:博袍山。南:擎雷山。東、西濱海。有北莉埠、新𦫼埠諸島,在東海中。西北:南渡水,出博政村,東南流,屈北,西別出為東亭水,瀦為湖。屈東,過縣治南,又別出為大肚河,北至遂溪入海。又東南流為雙溪港,擎雷水自西南來注之,又東北入海。有清道巡司。雷陽廢驛。武郎廢鹽場。遂溪簡。府東北百八十里。東:石門嶺,其下曰石門港,東、西濱海。海中有東山島,一名湛川島,島北為分流港,其西則通明港也。西北有西溪水,出分界村,東南流,與東溪水合,屈東過縣治南,東北合石門港入海。又城月水,出西南螺岡嶺,南屈而東為庫竹港,入海。又牛鼻水亦出螺岡嶺,迤西流為樂民港,入海。縣丞治楊柑墟。有湛川廢司。城月廢驛。調樓、蠶村二廢鹽場。徐聞簡。府西南百六十里。西:冠頭嶺。東、西、南三面濱海。北:遇賢水出石灣嶺,會青桐港水,又西合濂濱水,為流沙港,入海。又東,大水溪,出東北龍床嶺,西南與葫蘆溪合,西南流為海安港,入海。有寧海、東場二巡司。又有新興鹽場,後廢。

陽江直隸州:繁,難。隸高雷陽道。舊陽江縣,隸肇慶府。同治五年,升為直隸。光緒三十二年,改為直隸州。東北距省治七百三十里。廣一百三十里,袤一百一十五里。北極高二十一度五十二分。京師偏西四度三十分。領縣二。北:北甘山。東南:北津山。又海朗,一名鎮海山。南濱海。海陵山在海中。西:漠陽江自陽春入,左合輪水河,東南至河口市,左合第八河,右歧為西河,又東南至州治南為鼉江,亦謂之恩江也。左納那龍河,為北津港,西河水自西南來注之,東南過虎頭山入海。紫蘿水源出紫蘿山,下流為三鴉港,入海。坡尾河出羅王嶂,與織篢河合。又東南為豐頭港,亦入海。又西南有雙魚港。又有北額港,上源即望夫水也。有太平巡司、那龍巡司,後廢。有太平驛、蓮塘驛,亦廢。鹽場一,曰雙恩。陽春沖,難。州西北百七十里。舊隸肇慶府。光緒三十二年來屬。東南:射木。東北:銅石。西:漠陽江,源出縣北雲浮山,曰雲浮水,東南流,合雲霖水,屈西南,左納羅鳳水,右納博學水,至縣治西北,北瀧水西流合焉,東南入陽江。又西,雙■E1水,出東安,南合雙龍水,又南屈而東,麻陳水自西南來注之,又東過古良鎮,屈東北,合於漠陽江。有古良、黃泥灣二巡司。樂安廢驛。恩平簡。州東北百五十一里。舊隸肇慶府。光緒三十二年來屬。石神山在北,一名鼇山。龍鼉山在西南。南有恩平江,亦曰錦水,上源為岑洞水,出西北雙穴,逕東南至平城山,君子河東流合焉。又東與橫槎水合,屈東北,左納牛岡水,右納金雞水,又東入開平。又東南,長塘水,亦至開平合於恩平江。又西南:那吉水,南至陽江,其下流為那龍水也。

廉州府:繁,難。隸廉欽道。初沿明制。領州一,縣二。光緒十四年,欽州直隸。東北距省治千八百里。廣一百六十里,袤二百二十六里。北極高二十一度二十四分。京師偏西七度十九分。領縣二。合浦疲。倚。東北:大廉山,州以是名。又北:五黃山。南:冠頭嶺。東南濱海。海中有珠池,曰珠海。又有潿洲、蛇洋洲,在海中。廉江在北,一名西門江,自廣西博白入,迤西流,右納小江水,又西合張黃江,屈西南為羅成江。武利江自東北來注之,至府治西北合洪潮江,又西南分流入海。又東北,漆桐江自廣西興業入,左合六硍江,又西北入廣西貴縣,是為武思江也。縣丞駐永安所城。珠場、高仰、潿洲、永平四巡司。北海市稅關。商埠,光緒二年英煙臺會議條約訂開。有還珠廢驛。靈山簡。府西北百八十里。北:洪崖山。西:六峰山。西南:林冶山。南:陸屋江,一名南岸大江,源出縣東羅陽山,西南至欽州為欽江。西北:那良江,出那良山,南流過太平墟曰太平江,又東北入廣西橫州為平塘江也。又黃橄江出西北英雄山,亦東北入廣西永淳為秋風江。有西鄉巡司。太平廢驛。

欽州直隸州:沖,繁,難。廉欽道治所。初沿明制,屬廉州府。光緒十四年,升為直隸州,析靈山縣林墟司隸之,又析州屬防城、如昔二司置防城縣來屬。東北距省治千九百里。廣二百二十四里,袤一百九十五里。北極高二十一度五十五分。京師偏西七度五十分。領縣一。北:銅魚山。東南:烏雷嶺,其下曰烏雷港。南濱海。海中有牙山、龍門諸島。東:欽江,自靈山入,迤西南至州治南,歧為二,又西南匯為貓尾海,屈東南,過龍門入海。北:那蒙江,源出靈山高塘嶺,西南流,右合長潭水,至三門灘,大寺江自西來注之,又南為漁洪江,又東南合於欽江。又篆嶺江亦出靈山,西南至平銀渡曰平銀江,屈東南與丹竹江合,南流為大觀港,入海。又那陳江出西北心嶺,東北至那陳墟為那陳江,又東北復入宣化為八尺江也。有沿海、林墟、長墩三巡司。那陳司廢。防城沖,繁,難。州西南百里。十萬大山在西北。白龍山在西南。山麓斗入海,向隸越南,光緒十三年來屬。又西南,分茅嶺,與越南界。南濱海。防城江出西北稔賓山,東南流,右納滑石江,逕縣治南,過石龜頭汛入海。北:大直江,出虎豹隘,南與賣竹江合。又東南過獅子嶺,那良江東北流合焉。又東為鳳凰江,又東南合於漁洪江,至欽州入海。又西潭洪江,出大勉山,東南過銅皮山為潭洪港,入海。北侖河,其上源曰文義河,出拷邦嶺,東北至北侖汛,屈而南,嘉隆江自西南來注之。其南岸則越南界也。又東與那良江合,逕越南海寧府北境入海。東興,縣丞駐。有如昔、永坪二巡司。

瓊州府:繁,疲,難。瓊崖道治所。東北距省治千八百一十里。廣一百五十二里,袤二百一十里。北極高二十度一分。京師偏西六度五分。領州一,縣七。府及崖州在南海中,曰海南島,中有五指山,綿亙數邑。山南隸崖州,山北隸府。環山中生黎,其外熟黎,又外各州縣。山峒深阻,黎、岐出沒為患,光緒十五年,總督張之洞始開五指山道為大路十二:東路三,西路三,南路、北路、東南路、東北路、西南路、西北路各一。奧區荒徼,闢為坦途,人以為便。瓊州,商埠,咸豐八年英天津條約訂開。有瓊海關。瓊山繁。倚。南:瓊山,縣以是名。北濱海。海西南白石河即建江,自澄邁入,北屈而東,入定安。又北入縣東南,為南渡江,又北為北沙河,屈西北至白沙門入海。縣丞駐海口所城。有水尾巡司。感恩鹽場。澄邁簡。府西六十里。邁山在南。北濱海。西南:建江,一名新安江,自臨高入,東南過黎母嶺,右納新田溪,入于瓊山。又澄江出東南獨珠嶺,西北流,至縣治西,合九曲水,又西為東水港,入海。稍陽水上源為南滾泉,北合沙地水,過石矍嶺為石矍港,入海。有澄邁巡司。定安簡。府南八十里。西南有五指山,一名黎母山,綿亙而東,為光螺嶺。又東為南閭嶺,南遠溪出焉。北:建江自瓊山入,東合南遠溪,過縣治東北,潭覽溪、仙客溪北流入焉,東北入于瓊山為南渡江。西南有萬全河,出喃嘮峒,東南流,入樂會。有太平巡司。文昌難。府東南百六十里。北:玉陽。南:紫貝山。東北濱海。海中有浮山,其下曰分洲洋。南:文昌溪,出縣西白玉嶺,東南流,右納白石溪、白芒溪,屈東,平昌溪自西北來注之。又南為清瀾港,入海。又南,白延溪,出八角山,東南為長岐港,入海。又北,三江水,即羅漢溪,出抱虎嶺,西北流,為鋪前港,入海。有鋪前、青藍二巡司。有樂會鹽場。會同簡。府東南二百九十里。東:多異嶺,濱海。西:龍角溪,源出西崖嶺,東南至嘉積市為嘉積溪,黎盆溪西流合焉。又東南為五灣溪水,入樂會。樂會簡。府東南三百三十里。西:白石嶺。西南:縱橫嶺。東濱海。西:萬全河自定安入,迤東流,屈而北,會太平水。又東南,會五灣水,逕龍磨山,分流環縣治,復合,又東過蓮花峰,屈東南為博鼇港,入海。又流馬河,源出西南龍巖嶺,東南入萬縣,與龍滾河合。又東北復入縣境,為嘉濂河。又東北為九曲河,納蓮塘溪。又東北會萬全河入海。臨高疲。府西南百八十里。南:那盆嶺。西:毗耶山。北濱海。南:大江即建江,自儋州入,北至腰背嶺,西別出為縣前江,屈東北流,至文瀾村,為文瀾水。透灘水北流合焉,亦謂之迎恩水也。又北為博鋪港,入海。其正渠,東北過白石嶺入澄邁,有和舍巡司。馬裊鹽場。儋州要。府西南三百里。儋耳山在北,一名松林山,又名藤山。西北濱海。獅子山在海中。東南建江,亦曰黎母江,西北過龍頭嶺,歧為二:東出曰大江,東北入臨高;西出曰北門江,一名倫江,西北流,至州治東北,屈而西,為新英港,新昌江自東南來注之,又西南入海。東北有榕橋江,西南有沙溝江,皆西北流入海。有薄沙巡司。鎮南司,廢。鹽場曰蘭馨。

崖州直隸州:沖,繁。隸瓊崖道。崖州舊隸瓊州府。光緒三十一年,升為直隸州。東北距省治二千六百八十里。廣二百四十二里,袤一百七十五里。北極高十八度二十七分。京師偏西七度三十六分。領縣四。東:回風嶺。西南:澄島山,一名澄崖山。東南濱海。東北:安遠水自陵水入,西南流,至郎勇嶺,歧為二:一西南至大村入海;一西北流為抱漾水,過州治北,屈南為保平港,入港。北:樂安河,西南過多港嶺,屈西北入感恩。東:多銀水,一名臨川水,出黎峒,東南與三亞水合,又東南為榆林港,入海。有樂安、永寧二巡司。鹽場曰臨川。感恩難。州西北百九十五里。舊隸瓊州府。光緒三十一年來屬。東:大雅山。東北:九龍山。西濱海。東南:龍江,出小黎母山,西南流,別出為感恩水,迤西至縣治北為縣門港,入海。其正渠西北過北黎市為北黎港,又西南入海。樂安河出州,西北流,入昌化。昌化簡。州西北三百六十里。舊隸瓊州府。光緒三十一年來屬。東北:峻靈山。東南:九峰山。西北濱海。南:昌江即樂安河,自感恩入,至縣治東南,歧為二,西南出曰南崖江,北出曰北江,皆入海。又安海江出東北歌謗嶺,西北至儋州入海。陵水難。州東北二百一十里。舊隸瓊州府。光緒三十一年來屬。西:獨秀山。南:多雲嶺。東南濱海。有加攝嶼、雙女嶼,在海中。西北:大河水,出七指嶺,東南過博吉嶺,屈南為桐棲港,又東入海。又南,青水塘水出西北狼牙村,東南流,至縣治西,別出為筆架山水,與大河水合,瀦為灶仔港。屈西南,至新村港口入海。有寶停巡司。萬沖,繁。州東北三百七十里。萬州舊隸瓊州府,光緒三十一年降為縣,來屬。東:東山。北:六連嶺。東南濱海。海中有獨洲山,其下曰獨洲洋。西北:龍滾河,出縱橫峒,南屈而東,與流馬河合,又東北入樂會,屈東南復入縣北。東別出為蓮塘溪,屈北至樂會,合萬全河。其正渠,東南過連岐嶺入海。又都封水亦出縱橫峒,東南流,歧為四派:曰和樂港,曰港北港,曰石狗澗,曰金仙河,至縣治東北入海。又南,踢容河,出西北鷓鴣山,東南至瘦田村分流,與石龜河合,又東南流入海。有龍滾巡司。鹽場一,曰新安。


\end{pinyinscope}