\article{列傳二百三十九}

\begin{pinyinscope}
洪汝奎楊宗濂史樸史克寬沈保靖

硃其昂弟其詔宗源瀚徐慶璋徐珍蒯光典

陳遹聲潘民表嚴作霖唐錫晉婁春蕃

洪汝奎,字琴西,湖北漢陽人。道光二十四年舉人。咸豐初,考取官學教習,期滿以知縣用。參曾國籓軍事。同治初,洊保至江南道員。總理糧臺,供應防軍及他省協餉。又籌還西征洋債,出入逾二千萬,綜核名實,不避嫌忌。光緒中,沈葆楨為兩江總督,尤倚任之。葆楨治尚威猛,因疾在告,輒疏請汝奎代治事,聲望益起。會詔求人才,大臣交章論薦。五年,特擢廣東鹽運使。調兩淮,裁冗費,建義倉,濬揚州城河。方欲大有為,而江寧三牌樓之獄起。

先是有棄尸三牌樓竹園旁,汝奎令參將胡金傳偵獲僧紹宗等仇殺謝姓男子,又稱薛姓,名亦屢易,汝奎請覆訊。葆楨以會匪自相殘,即置大闢。逾三年,得真盜周五、沈鮑洪等殺硃彪事,時地悉合。事聞,命尚書麟書、侍郎薛允升往江南即訊,金傳坐濫刑失入,治如律;汝奎失察,褫職遣戍;葆楨以前卒,免議。於是朝旨申戒各行省慎重刑獄,並禁嗣後武員毋庸會鞫。汝奎至戍所,未幾赦歸,遽病卒。宣統初,總督端方疏陳其治行,復原官。

楊宗濂,字藝芳,江蘇無錫人。咸豐末,以戶部員外郎在籍治團練。時錢鼎銘乞師於曾國籓,宗濂偕行。及李鴻章以援師東下,宗濂率舊部為軍導,屢著戰績。劉銘傳進剿江陰,宗濂率濂字營守楊舍。賊來犯,宗濂領沙團擊卻之。沙團者,起於江岸集眾自衛,以技勇名,賊皆畏之。攻無錫,宗濂任前鋒。與賊酋黃子澄鏖戰,夜奪北門入,拔其城。合攻常州,宗濂督戰西門,架浮橋,獨騎先進。馬驚逸,墮河,躍起易騎再進,揮兵肉薄登,遂擒陳坤書。江南平,隨鴻章移師剿捻,總理營務處。軍興所至索官車,吏民交困,宗濂創立車營,行軍所需,預為儲峙,隨時無不備。諸軍仿其制,皆稱便。積功擢道員。

同治十一年,權湖北荊宜施道,被劾罷。鴻章創北洋武備學堂,奏起宗濂總其事,成材甚眾。光緒十六年,授直隸通永道。時畿輔大潦,宗濂主賑事,假便宜發緡粟。復大治水利,修潮白、青龍、薊運、北運、通惠、永清各河。疏渠樹防,闢膏腴數萬頃,士民刊碑頌德。以憂歸。再起,為山西河東道,歷權布政使、按察使,遷長蘆鹽運使。二十六年,聯軍犯天津,宗濂督蘆勇登陴固守,飛丸裂左脛,血流如沈,猶裹創治軍。城陷,巷戰,又傷右股。命駐保定督糧臺,旋隨鴻章入都議和。事定,賞三品京堂。未幾,以病乞休,卒。

史樸,字蘭畦,直隸遵化州人。以進士用知縣,分廣東,歷惠來、乳源、南海等縣,所至有威惠。潮陽盜鄭段基殺前令,樸蒞任,立捕誅之。晉羅定知州,留省捕劇盜劉亞才及餘盜九百,並置諸法。粵省海盜久為患,樸航海往剿,降盜魁張十五仔等,盡散其黨數千,有不受撫者剿平之,擢知府。剿英德土匪,遇伏佛岡,沒深澗,罣木得不死。賊踵至,睨之,曰:「史公也!」爭引出,跪進飲食。樸責以大義,數且詈,誓絕粒。賊益敬畏,羅拜感泣,原縛渠自效。會救至,舁之歸。詔革職,仍令自效。匪平,復故官。

粵東匪起,省城戒嚴。守獵德砲臺,連破沙灣、茭塘、新灶各賊巢,賞孔雀翎,知肇慶府。梧州被圍,督兵往援,拔其眾還軍封川,且戰且守。會英人陷廣州,大府不遑西顧。樸與賊相持五閱月,大小數十戰,殺賊數千人。其後賊大至,會提督昆壽水陸並進,大戰封川江口,連捷殲賊,軍遂復梧州。晉道員,再權肇羅道。同治二年,移廣州,攝按察使,旋署糧儲道。樸在粵前後垂四十年,善治盜,尤善用人。南海隸為盜誣,特出之,後督捕得其死力。撫瓊盜,易名入伍,多死敵。省圍乏餉,出勸募,立集百萬金。賊初起,獨主撫,及踞梧,則主剿,皆得其機宜。光緒二年,以籌解西征協餉,加鹽運使。鄉舉重逢,賞二品服。四年,卒。

史克寬,字生原,安徽六合人。咸豐中,與兄克諧辦鄉團禦賊。太湖陷,克諧殉。克寬從克太湖、宿松,解六合圍,以國子監典簿保知縣。同治初,劉銘傳剿捻,移征西回,皆挾克寬與俱,為司饋運及營務處。敘功,擢知府。光緒中,李鴻章督畿輔,檄董工程局,掌河事,治滹沱,於獻縣硃家口闢減河三十里,循子牙河故道入海。鴻章上其績狀,因奏任清河道,民立石頌其德。旋以他事被劾奪職,遂歸。

沈保靖,字仲維,江蘇江陰人。咸豐八年舉人。父燿鋆,湖北通判,武昌陷,罵賊被害。保靖出入賊中覓遺骸,三載始得死事狀,得賜恤立祠。李鴻章督師上海,招參幕事,積功至道員。同治十一年,授江西廣饒九南道。時英使訂約煙臺,議於江西湖口輪舟停泊起卸貨物,保靖以有礙九江關稅務,力爭之,總署卒廢約。擢按察使,攝布政使。光緒七年,遷福建布政使。法越事起,方事急,城閉,錢米歇業,居民洶洶將為亂。保靖出諭,發庫款三十萬以濟市面,人心始定。以他事被劾奪職,旋復官,遂不復出。所著有讀孟集說、韓非子錄要、怡雲堂內外編等。

硃其昂,字云甫,江蘇寶山人。同治初,從軍攻南匯。城賊原降,要一人入盟,無敢往者,其昂毅然請入受其降,城始下。旋納貲為通判,累至道員。北洋大臣李鴻章頗奇其才。福州船政造軍艦不適用,奏改商船。其昂與其弟其詔創議官商合辦,請設輪船招商局,鴻章上其事,遂檄為總辦。御史董俊翰劾以力小任重,下鴻章查覆,仍力贊其成。於是官商合力開局集股,並收並外人所設旗昌輪船公司以保航權。數年,成效大著。光緒初,直、晉災,其昂輸私財力任賑撫,以勞致疾。鴻章特委權津海關道,越三日卒,詔優恤,贈光祿寺卿。

其詔,字翼甫。納貲為知縣,累至道員。歷充江、浙漕運事。輪船招商局既成,復請以額定漕運費給輪船代為海運,局基始固。再權永定河道,時出巡河堤上下,務盡其利弊。遇伏汛暴漲,嘗三晝夜不交睫,親督弁兵搶護,始免潰決,民皆德之。擴充天津電報學堂,成材益廣。時方議辦海軍醫學堂,其詔復捐自置天津法租界地四十畝為校址以成之,其急公好義類如此。未幾,卒,贈內閣學士。

宗源瀚,字湘文,江蘇上元人。少佐幕,洊保至知府。光緒初,官浙江,歷署衢州、湖州、嘉興府事,敏於吏事,判牘輒千言。在湖州濬碧浪湖,興水利。時太湖漊港淤塞,前守楊榮緒疏濬無功,會有疏陳治法者,下郡,源瀚乃議大興工役,所規畫甚備。榮緒回任,卒成之,補嚴州。兵後凋敝,多溫、臺客民寄墾,習於剽劫,廉治其魁,遣散歸者六千人。治嚴五載,煦嘔山民,穿渠灌田,引東、西湖以洩新安江之暴漲,旱潦不害。每巡行田野,勸民力穡。調寧波,通商事繁。有戈鯤者,素豪猾,為英國領事主文牘,積為奸利病民。源瀚發其罪狀,牒上大吏及南、北洋大臣,逐鯤海外。法國兵船犯浙洋,源瀚從寧紹臺道薛福成籌海防,多所贊畫,數有功。晉道員,署杭嘉湖道。二十年,日本構兵,調溫處道,沿海戒嚴,處以鎮靜,清內匪,捕誅盜渠十餘人,疆圉晏然。又三年,卒於官。

源瀚優文學,尤精輿地,所繪浙江輿圖世稱之。

徐慶璋,字興齋,浙江山陰縣人。初佐都興阿戎幕,累保知縣,歷任奉天寬甸、蓋平、義州,晉興京同知。所歷多善政。常微行市中,遇有訟爭者,輒為剖其曲直而遣之。倡修養濟院,收養貧民。興俗春耕遲,慶璋集村氓語以農事不可違時之義,眾承其訓,有「早種一天早收十天」之諺,至今誦之。

光緒二十年,由鳳凰調遼陽知州。值中日戰亟,省東南各縣相繼淪陷。僅遼陽為盛京門戶,賴先事籌備。募餉練兵,號鎮東軍,沿邊設防。自遼陽而岫嚴、海城、復縣三千六百村士民,編團數萬人,以遼南靦峒徐珍為練長,勒以兵法。日兵至,慶璋語眾曰:「敵迫矣!援師未集,汝等自為計,毋與我偕亡。我死,分也!」眾感奮,皆請殺敵,遂迭敗日兵,俘百數十人。戰守歷五越月。長順、依克唐阿方督戰,皆倚以為重,屢詔嘉獎。是時州西連年水災,復募款捐濟,全活無算。慶璋才而負氣,其平日為政寬猛兼施,眾畏之如秋霜,愛之如冬日,有徐青天之稱。如議成,擢甘肅慶陽知府,遷甘涼道,積勞致病,卒於官。

徐珍,字聘卿,遼陽人。剛正多勇略,日軍犯遼,珍獨率民團守吉洞峪,扼險堅持,敵不得逞。慶璋既屬以練長,會將兵者忌之,飭散團眾,防務遂弛,而吉洞峪鄉團之名,乃著於中外。事定,以抗敵出力,保用縣主簿。拳匪亂作,珍復辦民團,聯數百村,有匪即剿捕,不分畛域。匪攻騰鰲堡及荒溝,先後剿平之。日俄之戰,珍嚴守中立,不稍假藉。總督趙爾巽嘉珍功,以辦團成績上,有「上不支官款,下不取民財,徒以忠義之故,護衛鄉閭,保全無算」之語。歷保至知府。武昌變起,土匪假革命名嘯聚煽亂。爾巽知珍義勇,委充巡防營

幫統,分防遼陽、海城、岫巖、本溪四城,地方賴以安謐。尋以巡防改編陸軍,遂辭職。卒後,州人建專祠祀之。

蒯光典,字禮卿,安徽合肥人。父德模,見循吏傳。光典幼慧,八歲能詩,隨父官江南,所師友多當代名儒,聞見日擴,名亦日起。其論學務明群經大義,而以六書、九數為樞紐,治六書則必求義類以旁通諸學,識雙聲以明假借。性強記,有口辯,尤熟於目錄掌故。有所論難,援據該洽,莫能窮也。

光緒九年進士,授檢討。典貴州鄉試,與其副不相下,以狂倨見譏,然榜發稱得士。充會典館圖繪總纂,精密勝於舊。中東兵起,發憤上書,不報,遂乞假歸。總督劉坤一聘主尊經書院講席。光典念國勢弱,在列諸人惟鄂督張之洞有大略,又嘗所從受業師也,因往說之洞慎選才俊,習武備,為異日革新庶政之用。之洞韙之,卒不果,而聘為兩湖書院監督。二十四年,敘會典館勞,以道員發江南,創辦江寧高等學堂。大學士剛毅按事江南,司道百餘人同詣謁,獨延光典密室縱談國事,語切直。剛毅大憾,即議裁高等學堂。光典力爭,不能得,拂衣去。坤一兩解之,檄赴鹽城丈樵地,樵地者,故鹽場葦蕩也。年餘得可耕之地七萬五千頃,收入荒價亦鉅萬。領正陽關督銷局,歲增銷官引百數十萬。會之洞代坤一為總督,以江南財匱,用不足,議增貨釐。光典謂增新釐則病商,毋寧整齊鹽課。之洞因奏陳兩淮鹽事衰旺,謂:「北鹽視正陽銷數,南鹽視儀棧出數。光典為江南治鹽第一,督正陽既有績,請使主儀棧。期三年,成效必可睹。」詔允之。光典既蒞事,以輪船駐大江三要區,首金、焦,次三江口,次沙漫洲,輔以兵艇,私梟斂跡。始儀棧出數不足四十萬引,比三年,增引十餘萬,歲益課釐銀百五十餘萬。乃益增募緝私兵隊,日夕訓練成勁旅,又於十二圩設學堂,建工廠,遂隱然為江防重鎮。

三十二年,按淮揚海道,加按察使銜。寶應饑民劫米,令潛逃。適光典舟至,剴切諭解之。而揚州亦以饑民劫米告,詗知猾胥陰煽眾,即擒治胥。大吏怒,將窮其獄,以光典言得免。運河盛漲,光典先分檄河員增修堤,而自泊舟高郵守視。壩險工迭出,大吏以故事,視節候測水,檄啟壩,不為動。歷月餘啟二壩,七月杪乃啟三壩,下河六縣獲有秋。建言淮海災區廣,宜寬籌賑金,不宜設粥廠,使災民麕集,費不貲,且生事。與布政使繼昌議不合,會奉檄入都參議改定官制,遂去任。後江北賑事款絀而費糜,一如光典言。

三十四年,命赴歐洲監督留學生。諸生不樂受約束,輒相訾謷,歲餘謝職歸。詔以四品京堂候補,充京師督學局長。宣統二年,赴南洋提調勸業會,卒於江寧。

陳遹聲,字蓉曙,浙江諸暨人。光緒十二年進士,改庶吉士,授編修。出為松江知府,鹽梟久為患。遹聲到官,密致其黨為導,帥健卒策疾騎踔百餘里,掩其魁捕之,寘諸法。松窪下,數苦潦,濬支河三十餘,並籌歲修費數萬金以澤農。以憂歸。拳禍起,暨俗素強,與教仇。不逞者轉相煽惑,眾至千餘,城鄉約期將為亂。遹聲獨命輿往喻之,途與眾遌,勢洶洶,斫輿前衡深寸許,正告之曰:「吾楓橋陳某也,來活爾!」為指陳利害。眾悟且泣,皆羅拜,爭棄械而走。而城中莠民忽蜂起,遹聲促官守閉城,捕其魁五人斬以徇,事立平。縣北江藻村,賭窟也。每歲十月,吳、越賭徒紛集,一擲累千金,破家者無算。遹聲請於大吏,屆時縣官蒞村坐禁,著為例,數百年敝俗至此而革。服除,以勞遷道員,入參政務、練兵、稅務諸政。

三十三年,授川東道。川東,盜藪也,蒞任未浹旬,開縣寇萬餘躪旁縣,立平之。次年,黔中盜魁劉天成結蜀邊逋寇撓川南,防軍數為所敗。省檄練軍七營剿之,寇至,委械去。遹聲立募精勇數百人,部以兵法,疾馳赴援,未匝月,生縛天成歸。江北產煤,礦■L5綿延數百里,至合州。奸民私售龍王洞於英商,外務部與訂租約,胥江北礦產授之;復要展拓至石牛溝,且蔓及兩川。川人憤,將與英商角。遹聲力爭之英領事,並密囑川人收石牛溝左右地。英商以無佗地可得,得溝與洞,猶石田也,恫喝百端,不為動,卒以賤值贖回。治渝兩載,大吏交章論薦,遽引疾歸。當軸數招之,謝不出。著有明逸民詩、畸廬稗說及詩集等。

潘民表,字振聲,江蘇陽湖人。同治十二年舉人。光緒初,數募金賑直隸、河南、山西諸行省。十五年,山東河決,凡賑歷城、齊河、臨邑、齊東、濟陽、惠民、商河、青城、濱、霑化、海豐、陽信、蒲臺十三州縣。閱四年始竣。災民無歸者眾,民表於歷城臥牛山建屋五百間、窩棚千間居之,使植桑麻,興耕織,疾病婚嫁,皆有資助。別建工廠百間,義塾八所,設教養局董之。因其規畫,歷十年之久,多有藝成自給者,乃以經費改設蒙養學堂。十九年,賑山西大同邊外豐鎮諸,亦仿臥牛山成法,收集教養之,尋以州同就職山東,署恩縣,補平度,擢泰安知府。二十八年,河決利津,詔頒內帑十萬,大吏檄民表去任專賑事。晉道員,發陜西,筦農工商礦局。民表諗同官縣土質宜磁,建磁窯同官,興大利。貲竭將中輟,請兼盩厔釐榷,以羨餘助磁業,仍不給,且虧稅,計無所出,竟仰藥死,時論惜之。

民表瘁於賑務二十餘年,每遇災祲,呼籥奔走,置身家不顧,敝衣草履,躑躅泥塗,面目黧黑,非人所堪,貲斧悉自貸。及服官,俸入悉以償賑債,充賑用。自義賑風起,或從事數年,由寒儒而致素豐。如民表之始終無染,歿無餘貲者,蓋不數覯。

嚴作霖,字佑之,丹徒人。以儒生奮起司賑事。自光緒二年始至三十年,歷賑山東、河南、山西、安徽、江蘇、直隸、廣西、奉天、陜西數行省。每兼濬河修堤,以工代賑。作霖性強毅,赴事勇決,綜覈無糜費,久而為人所信,故樂輸者眾。其施賑不拘成法,隨時地而取其宜。當時疆吏以義賑可矯官吏拘牽延緩積習,樂倚以集事。作霖不求仕進,輒辭薦剡,僅受國子監助教銜,數被溫詔嘉焉。積賑餘貲興揚州、鎮江兩郡善舉。及歿,子良沛出二十餘萬金為恤嫠、保節、備荒等用,成其遺志云。

唐錫晉,字桐卿,無錫人。父文源,闔門殉粵難,積尸滿井。亂平,錫晉拾親骨,瀝血取驗,誓奉遺訓力行善。光緒初,聞豫、晉災,始募義賑。十四年,以恩貢授安東縣教諭。時淮、徐、海大水,錫晉棹小舟往賑,憂勞甚,須發為白。明年,安東澇,益募金賑之。冬,復賑山東沿海諸郡災,為置常平倉。二十六年,兩宮西狩,關中大饑,人相食,錫晉醵金四十萬往賑,歷二州八縣,艱困不少阻。災區廣,賑款且匱,乃單車詣行在,請於大學士王文韶,得二十萬金益之。事竣,返安東。坐劾安東知縣貪殘,同落職。兩江總督端方等奏復錫晉官,改銓長洲,後以輸金助賑保道員。三十二年,湘中災,官紳復以賑事囑。秋,淮浦被水,流民數十萬洶聚,喻遣勿散,咸曰:「有司行賑不足恃,必得唐公。」時錫晉臥病,猶強扶而至,眾見其來,驩曰:「吾生矣!」乃各還歸待賑,遂以無事。

宣統三年,方籌賑江、皖,而武昌變起。錫晉憂憤,病日劇,越歲卒。錫晉治賑,自乙亥至辛亥,凡三十有七年,其賑地為行省八:山西、河南、江蘇、山東以及陜西、湖南,東至吉林,西至甘肅;其賑款過百萬以上。義賑之遠且久,無過錫晉。歿後眾思其德,受賑各省咸請立祠祀之。

婁春蕃,字椒生,浙江紹興人。以貢生納貲為同知,歷保道員。久參北洋幕府,李鴻章尤重之,常倚以治繁劇。春蕃熟諳直隸水利,永定河常歲決,思患預防,以時消息之,河不數病。長蘆鹽商久困增釐,春蕃務為寬大,課裕而商不撓。尤精刑律,審覈維慎,直省遂無冤獄。拳亂作,力主剿辦。為總督裕祿草奏,痛陳邪術萬不可信,戰釁萬不可開,以一服八,決無幸理。裕祿初頗信之,不能堅持,卒致敗裂。匪以通敵誣紳富,請搜殺。春蕃力阻,多保全。事亟,春蕃首請召鴻章北上停戰議和。及聯軍猝至,同僚皆走,春蕃獨留不去,艱苦謀搘拄,至一月之久。鴻章至,復參和議,約成,辭優保。辛亥事起,人心惶惑,春蕃夙夜籌慮,獨為地方謀保安。焦勞益甚,猝病卒。

春蕃敦節操,有經濟才。自鴻章延入直幕,先後垂三十年。歷任總督如王文韶、榮祿、袁世凱、楊士驤、端方、陳夔龍等,皆敬禮之。雖不樂仕進,未親吏治,而論治佐政,留意民生,各郡縣皆奉為圭臬。歿後,直人思其德,公請附祀鴻章祠。

論曰:各省監司能著聲績者,大抵多起於守令,蓋親民之效焉。及兵事興而有事功幕職,捐例開而有輸餉助賑,雖其初不必盡親吏治,而以實心行實政,流愛於人,民之感之,亦豈有異?自汝奎、宗濂以至錫晉、春蕃諸人,德惠在人,後人稱之至今,不可敬哉!


\end{pinyinscope}