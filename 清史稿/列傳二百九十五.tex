\article{列傳二百九十五}

\begin{pinyinscope}
列女一

田緒宗妻張嵇永仁妻楊妾蘇張英妻姚

蔡璧妻黃子世遠妻劉尹公弼妻李錢綸光妻陳

胡彌禪妻潘張棠妻金洪翹妻蔣張蟾賓妻姜

施曾錫妻金廷璐妻惲汪楷妻王妾徐馮智懋妻謝

鄭文清妻黎程世雄妻萬高學山妻王

王氏女張天相女周氏女王孜女繆滸妻蔡濮氏女

李氏女來氏二女曾尚增女王氏女劉魁妻徐薛中奇女

呂氏女佘長安女王法夔女武仁女唐氏女

張桐女汪儼聘妻周劉氏女吳某聘妻周李薦一聘妻曾

袁斯鳳女丁氏女硃棫之女杜仲梅女方氏二女劉可求女

楊泰初女孫承沂女趙承穀聘妻丁彭爵麒女陳寶廉女

吳士仁女王濟源女董桂林女耿恂女吳芬女

邵氏二女蔣遂良女徐氏二女李鴻普妻郭牛輔世妻張

高位妻段鄭光春妻葉子文炳妻吳屈崇山妻劉

謝以炳妻路弟仲秀妻鄭季純妻吳王鉅妻施陳文世妻劉

張守仁妻梁韓守立妻俞路和生妻吳諸君祿妻唐

牛允度妻張游應標妻蕭蔣廣居妻伍

周學臣妻柳王德駿妻盛張茂信妻方林經妻陳

張德鄰妻李武烈妻趙孫朗人妻吳李天挺妻申劉與齊妻魏

周志桂妻馮歐陽玉光妻蔡子惟本妻蔡蕭學華妻賀

張友儀妻陳馮氏王鉞妻隋林云銘妻蔡

陳龍妻胡王懃妻岳魯宗鎬妻硃馬叔籥妻丁

許光清妻陳黃開鼇妻廖黃茂梧妻顧高其倬妻蔡

陳之遴妻徐詹枚妻王柯蘅妻李艾紫東妻徐

郝懿行妻王汪遠孫妻梁陳裴之妻汪汪延澤妻趙

吳廷珍妻張諸妹章政平妻等程鼎調妻汪

陳瑞妻繆馬某妻阮富樂賀妻王仁興妻瓜爾佳氏耀州三婦

杉松郵卒婦楊芳妻龍崔龍見妻錢沈葆楨妻林

王某妻陳李某妻趙羅傑妻陳楊某妻唐姚旺妻潘蓋氏

積家而成國,家恆男婦半。女順父母,婦敬舅姑,妻助夫,母長子女,姊妹娣姒,各盡其分。人如是,家和;家如是,國治。是故匹婦黽勉帷闥之內,議酒食,操井臼,勤織紝組紃,乃與公卿大夫士謀政事。農勞稼穡,工業勢曲,商賈通貨財,同有職於國,而不而闕。晚近好異議,以謂女豢於父,妻豢於夫,戚戚求自食。或謂女制於父母,婦制於舅姑,妻制於夫,將一切排決,舍家而躐國,務為閎大,其過不及若殊,要為自棄其所職而害中於家國則均。嗚呼,何其誣也!古昔聖王經國中而為之軌,億萬士女毋或逾焉。是故矜其變,所以誨其正;愍其異,所以勵其庸:範而趨於一。使凡為女若婦者,循循各盡其職。則且廣之為風俗,永之為名教。有國者之事,以權始,以化終。權故行,化故成,國以治平。

清制,禮部掌旌格孝婦、孝女、烈婦、烈女、守節、殉節、未婚守節,歲會而上,都數千人。軍興,死寇難役輒十百萬,則別牘上請。捍強暴而死,爰書定,亦別牘上請,皆謹書於實錄。其徵之也廣,其襮之也顯,流風餘韻,綿綿延延,風雨如晦,雞鳴不已。故知權所以能行,化所以能成,尤必有當於人人之心,固不可強而致也。列女入史,始後漢書,用其例,擇尤炳著如乾人,賢母、孝女、孝婦、賢婦、節婦、貞婦、貞女、烈婦義行,邊徼諸婦,以類相從,其處變事相亞者,厭而比焉。纂昔懿,傃來淑,敬我彤管,宜有助於興觀。

田緒宗妻張,德州人。緒宗,順治九年進士,官浙江麗水知縣,有聲。卒官。張預戒管庫,謹視賦徭所入,發牘覈其數。代者至,請知府臨察,無稍舛漏,乃持喪歸。教三子雯、需、霡,皆有文行。張通詩、春秋傳,能文。

年七十,里黨將為壽,誡諸子曰:「禮,婦人無夫者稱未亡人,凡吉兇交際之事不與,亦不為主名,故春秋書『紀履緰來逆女』。公羊傳曰:『紀有母,何以不稱母?母不通也。』何休云:『婦人無外事,所以遠別也。』後世禮意失,始有登堂拜母之事。戰國時,嚴仲子自觴聶政母前,且進百金為壽。蓋任俠好交之流,有所求而然耳,豈禮意當如是耶?吾自汝父之歿於官,攜扶小弱,千里歸櫬,含艱履戚,三十年餘。闔戶闢績,以禮自守。幸汝曹皆得成立,養我餘年,然此中長有隱痛。每歲時膢臘,兒女滿前,牽衣嬉笑,輒怦怦心動,念汝父之不及見。故或中坐嘆息,或輟箸掩淚。今一旦賓客填門,為未亡人稱慶,未亡人尚可以言慶乎?三十年吉兇交際之事不與知,而今日更強我為主名,其可謂之禮乎?處我以非禮,不足為我慶,而適足增我悲耳。汝曹官於朝,宜曉大體,其詳思禮意,以安老人之心!」

張年七十七而卒,有茹荼集,雯官至戶部侍郎。

嵇永仁妻楊,永仁,無錫人;楊,長洲人。永仁死福建總督範承謨之難,楊時年二十七,子曾筠生七年。舅姑皆篤老,黽勉奉事,喪葬謹如禮。福建定,永仁僕程治乃克以其喪還,楊質衣營葬。葬竟,撫曾筠而泣曰:「我前所以不死,以有舅姑在。舅姑既而葬,今又葬汝父,我可以死,則又有汝在。汝父以諸生死國事,汝未成人,當如何?」則又嗚咽曰:「我其如何?」曾筠長而力學,楊日織布易米以為食,指謂曾筠曰:「汝能讀書,乃得啖此,未亡人則歠粥。」及曾筠官漸顯,恆誡以廉慎。雍正十一年,卒,年八十有四。永仁、曾筠皆有傳。

永仁妾蘇,字瑤青。從永仁福州,臨難,取帶面永仁而縊,年十七。

張英妻姚,桐城人。英初官翰林,貧甚,或餽之千金,英勿受也。故以語姚,姚曰:「貧家或餽十金五金,童僕皆喜相告。今無故得千金,人問所從來,能勿慚乎?」居恆質衣貰米。英祿稍豐,姚不改其儉,一青衫數年不易。英既相,彌自謙下。戚黨或使婢起居,姚方補故衣,不識也。問:「夫人安在?」姚逡巡起應,婢大慚沮。英年六十,姚制棉衣貸寒者。子廷玉繼入翰林,直南書房,聖祖嘗顧左右曰:「張廷玉兄弟,母教之有素,不獨父訓也!」卒,年六十九,有含章閣詩。女令儀,為同縣姚士封妻,好學,有蠹窗集。英、廷玉皆有傳。

蔡璧妻黃,漳浦人,世遠母也。璧喪妻,以為妾。耿精忠為變,璧方客京師,黃奉璧父母避山中。璧母老不能粒食,輟女子子乳乳之。璧父母命璧以為妻。

世遠妻劉,事舅姑孝。世遠既貴,家人謀買婢,勿許。謀傭乳母,劉曰:「吾六子四女皆自乳,吾不以貴易其素。」世遠有傳。

尹公弼妻李,博野人。公弼早卒,家貧,舅姑老,父母衰病,無子。養生送死,拮據黽勉。教子會一有法度,通籍,出為襄陽府知府,李就養。雨暘不時,必躬自跽禱,禳疫驅蝗亦如之。冬寒,民六十以上,量予布帛。襄陽民德之,為建賢母堂。李賦詩辭之,不能止。會一移揚州府知府,揚州俗奢,李為作女訓十二章,教以儉。累遷河南巡撫,所至節俸錢,畀高年布帛,周貧民,佐軍餉,皆以母命為之。民間輒為立生祠,如在襄陽時。會一內擢左副都御史,李以疾不能入京師,陳情歸養。復以母命,里塾社倉次第設置。居數年,高宗賜詩嘉許,榜所居堂曰「荻訓松齡」。卒,年七十八。

公弼曾孫溯醇妻徐,亦早寡,與其族公亮妻高、公聘妻楊、德一妻韓、成一妻李、多福妻劉、林妻王、二喜妻硃,合稱「尹氏九節」。會一有傳。

錢綸光妻陳,名書。綸光,嘉興人;陳,秀水人。幼端靜,讀書通大義。初婚,綸光侍其父瑞徵出上塚,陳從樓上望見少年毆佃客幾死,咯血,方大雪,血沾衣盡赤,佃客家以其族黨至,洶洶。陳遣蒼頭問,少年,從子也。乃舁佃客入室,召醫予藥,畀其母錢米,呼從子使受杖,眾乃散。瑞徵還,亟賢之。陳善事舅姑,助綸光款賓客,周鄰里,曲盡恩意。綸光卒,教子尤有法度。子陳群,自有傳;界,官陜西醴泉知縣,有賢聲。陳晚為詩,號復庵;署畫,號南樓老人。詩三卷,戒陳群毋付刻。畫尤工,山水、人物、花草皆清迥高秀,力追古作者。

曾孫女與齡,字九英,為廣西太平府同知吳江蒯嘉珍妻。亦能畫,題所居曰仰南樓。

胡彌禪妻潘,桐城人。彌禪卒,遺三子,長子宗緒,方十歲。貧,遣就學村塾,旦倚閭泣而送之,逾嶺不見,乃返,暮復迎之而泣。三年,貧益甚,罷學,潘不知書,使兒誦,以意為解說。一日,聞程、硃語,嘆且起立曰:「我固謂世間當有此!」聞誦司馬相如美人賦則怒,禁毋更讀。諸子出必告,襟濡露,則笞之,問:「奈何不由正路?」歲饑,潘日茹瓜蔓,而為麥粥飯兒,有餘,以周里之餓者。嘗命僕治室,發地得千金,獻宗緒,宗緒不受,母聞乃喜。宗緒成雍正八年進士,官至國子監司業,篤學行,有所述作。

張棠妻金,秀水人。棠卒,金作苦奉姑,晨炊偶有餘,日午復以進。姑呼金共食,金慮姑不足,輒以腹痛辭。姑病,侍食嘗藥,搔癢滌牏,鬋發拭垢,靡不躬焉。夜坐床下,聞呻吟即起。姑歿,哭之痛,曰:「吾將何怙,以冀孤兒長乎?」則愈益作苦。方冬捆屨,兩手龜且裂,敷以醬及蠟淚,痛如割,必畢事乃寢。子庚,稍長有文行,客游以為養。一日,金晨起,理發竟,登案扳甍西南望曰:「我安得望見江西?」時庚方客南昌,南昌於浙為西南,故云。既得旌,泣而言曰:「我姑亦早寡,徒以年已逾三十,不中令甲,而我得被旌,我於是有私痛也。」年七十九而卒。

洪翹妻蔣,武進人。翹尚義而貧,僦居臨大池,隘且濕,蔣擇處其尤陋者,暴雨,水浸淫床下。翹既不第,客游養父母。俄書報病且歸,蔣挾二子舟迎,聞來舟哭聲,審其僕也,號而自擲於水,女傭持之,免。自是率諸女針紉組織,力以自食。授其子禮吉讀,至禮經「夫者婦之天」,哭絕良久,呼曰:「吾何戴矣!」遂廢其句讀。禮吉稍長,出就裏中師,里中師不辨音訓,母為正其誤,日數十字。母織子誦,往往至夜分。翹大父嘗守大同,父公寀獨償大同官逋十有餘萬,不以累弟昆。受託趙氏孤,坐累家破,卒全之,以此名孝義,蔣恆舉以勖禮吉。喪舅姑,毀甚,既復喪母,疾作遂卒。禮吉更名亮吉,有傳。

張蟾賓妻姜,武進人。蟾賓父金第客死京師,妻白,食貧撫諸孤。蟾賓復早卒,姜撫二子惠言、翊。貧,惠言就其世父讀,歸省姜,無食,明日,惠言餓不能起,姜撫之曰:「兒不慣餓,憊耶?吾與而姊、而弟時時如此也!」惠言稍長,使授翊書,姜與女課女紅,常數線為節,晨起,盡三十線乃炊。夜燃燈視二子讀,恆至漏四下,里黨稱姜苦節如其姑。惠言有傳。

施曾錫妻金,名鏡淑。曾錫,桐鄉人;金,震澤人。曾錫故有文行,以副榜貢生終。孤福元生七年矣,教之嚴,夜篝燈讀書,福元稍怠,欲撲之,撲未下,涕泗交於頤,輒罷。初曾錫喪父母及所生父,金撤簪珥以佐葬;及葬曾錫,家益貧。紡績,冬寒皸瘃,十指皆流血。所生姑亦卒,乃還依母。歲大無,具飯飯母,並及福元,而自食豆粥雜糠覈。母病,侍尤謹。福元以舉人知西江安福縣,而金已前卒。

廷璐妻惲,廷璐,完顏氏,滿洲鑲黃旗人。惲,陽湖人,名珠,字珍浦。惲自壽平以畫名,其族多能畫。毛鴻調妻惲冰,字清於,畫尤工粉墨,映日有光,於珠為諸姑。珠亦能畫,善為詩。廷璐為泰安知府,卒官。珠撫諸子麟慶、麟昌、麟書,教之嚴。持家政,肅而恕。嘗擬列女傳為蘭閨寶錄。撰定清女子詩,為國朝女士正始集。校刻壽平父日初遺書及李顒集,皆傳世。麟慶有傳。

汪楷妻王、妾徐,蕭山人。楷為河南淇縣典史,嘗廉民冤,白令為平反。既去官,客死廣東。母七十,徐有子輝祖,幼。喪歸,索債者至,王鬻田、出嫁時衣裝以償。楷弟不肖,恆求錢以博,甚或篡輝祖去,得錢乃歸之。已,將以母遷,王與徐力請留,奉侍甚謹。母垂歿,嘆其賢孝。教輝祖讀,或不中程,徐奉箠呼輝祖跪受教,王涕泣戒督,往往棄箠罷。貧益甚,互稱疾減食食輝祖。

輝祖長,出游,佐州縣治刑名,王戒之曰:「汝父嘗言生人慘怛,無過囹圄中,偶撲一人,輒數日不怡,曰:『彼得無恚恨戕其生乎?』汝佐人當知此意。」輝祖自外歸,必問:「不入人死罪否?破人家否?」曰:「無。」則喜。即言法不免,王與徐輒相視為流涕。王尤不喜言人過,輝祖或偶及之,必曰:「汝能不爾即佳,此何與汝事?」徐居常布衣操作,歲饑,日織布一疋,易三斗粟,雖瘧不為止。一絮被,餘二十年,輝祖請易,曰:「此汝父所予,不可易也!」徐病,輝祖進蓡,卻之,曰:「汝父客死,吾不獲以此進,吾何忍飲?」王強之,微啜而罷。徐卒十餘年,輝祖成進士而王卒。輝祖有傳。

馮智懋妻謝,智懋,長洲人;謝,嘉興人。智懋家中落,再遇火,謝處貧,黽勉無所恨。子桂芬,入學為諸生,謝喜曰:「汝家久無秀才,汝繼之,甚善。原世世為秀才,毋覬科第也!」及得第,訓之曰:「人必有職,女紅中饋,婦職也,易盡耳;汝當思盡其職。」又曰:「好官不過多得錢,然則商賈耳,何名官也?汝謹,當不至是,勉旃!」蘇州、嘉興,皆困重賦,謝氏以催科破家。謝每謂桂芬:「汝他日為言官,此第一事也!」同治初,江、浙初定,桂芬佐江蘇巡撫李鴻章幕,成減賦之議。蘇州、松江、太倉三府、州,減三之一;常州、鎮江減十之一。浙江巡撫左宗棠繼請嘉興亦得量減,時謝已前卒。桂芬有傳。

鄭文清妻黎,遵義人。事祖姑及姑能得其歡心。貧,令長子珍就傅,諸子力田,教督之甚肅。珍錄平生所訓誡為母教錄。嘗曰:「婦人舍言、容、工,無所謂德。言只柔聲下氣,容只衣飾整潔,工則針黹、紡績、酒漿、菹醢,終身不能盡。」又曰:「人雖貧,禮不可不富;禮不富,是謂真貧。」珍,儒林有傳。

程世雄妻萬,衡陽人。世雄兄世英早卒,妻何無子,世雄旋亦卒。子學伊弱,族有爭嗣者,萬以學伊兼承世英後。姑喪未殯,火發,何、萬與諸婢號泣奉柩出,火為之止。萬善治家,學伊長,家漸起。咸豐間軍興,諸將唐訓方、陳士傑、彭玉麟皆倚學伊籌兵食。萬日具

百人饌,為規畫周至,賢母名益聞。力施與,贍諸戚族,教孫曾,皆成立。年八十九卒。

高學山妻王,瀘州人。王歸學山,視前室子四皆羸弱,鞠育甚至。長子病且死,泣語申母恩,原再來為母子。第三子病,亦如之。逾年,學山夢二子者至,即夕,王孿生二子。王教諸子讀書、擇友有法度,多取科目,孿生子同舉於鄉。

王氏女娥,九江屠者女也。順治十四年,火,屠者方醉臥,娥奔火中,呼不起,遂並焚死。

張天相女巧姑,儀徵人。乾隆十年正月庚寅,火,天相方病,巧姑年十四,負父欲出,同死。明日得其尸,猶負父也。

周氏女,六安人。父瞽,女八歲,火作,母抱女出,問:「父胡不出?」母曰:「父瞽不能行,奈何?」女入火中,導父行,火烈迷路,俱死。

王孜女,慈谿人。康熙十六年七月乙未,乙夜慈谿火。女方居母喪,停棺於堂。火至,女呼舁棺,無應者,伏棺上泣。父從火光中遙見之,抱之出,則已死。灌以礬水,稍甦,聲出喉間,僅屬。問:「母棺出否?」家人不答,遂哽咽而絕。女年十五。

薩玉瑞妻許,閩人。夫亡,姑初喪,火發,護姑柩不去,同燼。

繆滸妻蔡,名蕙,泰州人。父孕琦,生五女,而蕙為長。字滸,未行,孕琦坐法論死,系獄待決。蕙絕嗜好,屏服飾,寢不解衣,嚴寒不設爐火。居四年,滸請婚,蕙謝不行。康熙二十八年,聖祖巡江南,蕙伏道旁上疏,略云:「妾聞在昔淳于緹縈為父鳴冤贖罪,漢文帝憐而釋之,載之前史,傳為盛典。今妾父孕琦被仇害,自逮獄以來,妾日夜悲號,籥天無路。每夕遙望宸闕,禮拜數千,於今三年,寒暑靡輟。今幸駕臨淮海,是誠千載奇逢,妾原效緹縈之故事,冒死鳴哀,伏維天鑒。」上下其疏江南江西總督覆讞,二十九年,讞上,孕琦得減死。蕙歸滸,未一年,卒。

濮氏女,桐鄉人。其父無子,而母妒,不使置媵侍,家萬金悉畀女。嫁吳生,予田宅、奴婢、什物皆具。女獨愍父未有子,嘗從容諫母,母怒,罵曰:「吾萬金餉汝,犬豕猶知人意,況人乎?」女不敢復言。乃為父置婢其家,時父至,使侍父。歲餘,果生男,載而之母家,會濮氏長老,見男於廟。具白母,賀母有子,母憾女,盡收田宅、奴婢、什物,驅就他舍,屏勿復相見。吳生既以婦富,乃驟貧,憤恚欲殺女,女度無所容,自經死。

李氏女,鹿邑人,次三。父麒生與族人礎、挺九有隙,挺九語礎,若與麒生有殺姊仇,不先之,終為害。礎與其子兆龍行求麒生,共毆之,垂死,乃棄去。三時年十九,麒生將死,唶曰:「仇殺我,我無子能報者,尚何言!」呼:「天,天!」遂絕。三請於母,訟縣及府,皆不省。訟巡撫,下開封府同知治,挺九好語三,原養母,請得息訟,三扼其吭,齧面盡壞,卒脫去。獄上,當礎死,礎自殺。兆龍杖,創甚,亦死。三以禍始挺九,顧無罪,走京師,擊登聞院鼓自列。下巡撫覆按,會挺九亦死。三泣告父墓曰:「仇雖盡,然不棄於市,恨未雪也!」乃不嫁養母。居十五年,康熙三十七年八月,母卒,三治喪葬竟,自經死。乾隆中,知縣海寧許菼表其墓,環墓為之田,曰「李孝女墓田」。

來氏二女,蕭山人。姊曰鳳筠,年十四。父客福建,從渡古田篛洋。父墮水,鳳筠方臥病,聞遽起,躍入水,呼救。魚舟集,援出水,鳳筠慄無人色,猶為父易衣。夜半,遂死。鳳蓀,其女弟也。父病,露禱百餘夕,不勝寒,亦死。

曾尚增女衍綸,長清人。尚增以庶吉士改官,遷知郴州,衍綸從。母病瘓不能起,衍綸日夜侍。居四年,一夕,母命衍綸少休,女傭就床下熏衣,遺火灼帷。衍綸突火入抱母號,救者以衍綸出,復入,哭且呼曰:「速救夫人!夫人出,我乃出。」火冪床,救者不得入,尚增厲聲呼衍綸出,不應,火益熾,遂殉。既滅火,見衍綸身覆母,兩體膠結不可解。時乾隆二十三年十二月乙亥,衍綸年十五。

又有王氏女,懷遠人,母亦病瘓,火作,女突火入負母,俱燼。

劉魁妻徐,霍丘人。既嫁而歸省,火作,負父出,復入負母,病瘓不能起,俱焚。火熄,見徐跪床下,猶執母手。

薛中奇女,宿州人。侍祖母,火作,扶祖母出,梁折,承以肩,焚死,祖母得免。

呂氏女,平陸人。父卒,母且嫁,女生七年,痛哭諫其母,母不聽,則日長跽母前,且哭且言,母意終不回。一日晨,潛出,家人求之勿得;暮,途人或言墦間有幼女死焉。家人就視,則女哭父瘞所,死矣,淚血溢兩眶,遍地盡碧。及斂,視其寢處,枕上血深漬數重。

佘長安女,名酉州,四川重慶人。長安妄訟人聚博宰耕牛,坐誣,戍湖北。嘉慶十六年,酉州走京師,詣都察院,自陳祖父、母年皆逾八十,乞赦其父得侍養。事聞,仁宗以長安罪非常赦所不原,至配所已九年,其女年甫十一,不遠數千里匍匐奔訴,情可愍,命赦長安。

王法夔女,名淑春,揚州人。法夔老而貧,淑春誓不嫁,力針黹為養。方冬,手龜身寒顫,工不輟。法夔至七十餘卒,淑春以首觸壁,額裂死。

武仁女,名端,錢塘人。能讀書,原不嫁事父母,父母不可。少長,母偶疾,夜求藥,墜樓,折脊,則喜曰:「吾今形殘,不可匹人,吾自是得終事父母矣!」仁客死貴州,端從母迎喪,至則貲已盡,力針黹奉母,而蓄其餘。居十有七年,始克以喪歸。

唐氏女,名素,無錫人。貧無昆弟,亦不嫁,鬻畫以贍父母。

張桐女,名富,蔚州人。道光九年,山水暴發,家人皆走避。桐方病臥,富將負父出,弱不勝。水大至,父揮之去,號泣,俱溺。水退,家人至,見富兩手猶握父臂不釋。

汪儼聘妻周,劉氏女名密,吳某聘妻周,皆六安人。儼卒,周歸注氏,事舅姑,水至,周從姑乘屋,攀樹,姑墮水,周躍下拯之,與俱死。密與母同墮,得板扉,緣以上,扉欹屢墮。母呼密速上,密曰:「扉狹不足全我母女,冀活母,兒不上矣!」遂死。周既入水,或援之登舟,問:「父母存否?」皆曰:「不知。」復躍入水死。

李薦一聘妻曾,南豐人。未行,遇水,室盡圮,母投水死。女援不及,入水殉。

袁斯鳳女璱,字儀貞,江蘇華亭人。斯鳳官河南懷慶府黃沁同知,璱事父母孝,視疾尤謹。母陳有寒疾,璱榻母側,視起居。母命之臥,頃輒起。八年,陳疾少瘥,璱乃曰:「世無不可治之疾,人力未至,而委之天命,則以為不可治爾。」斯鳳疾作,乍劇乍瘥,夜靜或大雪,璱嚴立窗外,伺聲息,往往不眠。道光十四年,斯鳳疾大作,醫謝不治。璱聞涕泣,已而怒曰:「誰謂不可愈,吾必欲愈之!」而斯鳳竟卒。後四日,璱闔扉欲自經,嫂過而勸之,璱泣誓死。嫂喻以殺身非孝,璱作色曰:「吾自欲死,此時雖孔子、硃子以吾為不孝,吾亦惟死爾!」嫂曰:「獨不念病母乎?」璱曰:「有汝在。」乃告其母,共諭慰之。又二日,璱竟死。死後,母察奩具,得斷釧。

丁氏女,鶴慶人。父貧,段石為灰以自給,女助之。年十六,父卒,女力作養母。嘗負重而躓,遂痀僂。為傭,食於傭家,每飯思母,輒哽咽。人憐之,許其分食以遺。否必為母炊竟乃出傭。居四十餘年,母卒女亦卒。

硃棫之女,武清人。字縣諸生曹文甲。早喪父,母病,奉事良謹。將婚,女堅請留侍母。母卒,治喪葬,請旌母節,奉母主入祠,見祠有孝女,為低徊甚久,歸遂自裁。遺書告文甲曰:「君家孝娥以身殉父,兒愚祗知有母,深負舅姑慈,原更得賢婦奉饔飧也。」

杜仲梅女末姑,安徽太平人。賊至,刃其母,抱持乞代,刃及,終不釋。賊去,母創死,女抱母尸泣,達旦,尋毀卒。同時二方氏女,一年十四,一方九歲,皆代母死。

又有劉可求女,亦太平人。弟被掠,女請於父易弟歸,即夕自殺。

楊泰初女徽德,孫承沂女錦宜,皆休寧人。徽德年十二,母死寇,抱尸不食死。錦宜七歲,寇殺其祖母,守尸側五日,賊與食,卻之,餓死。

趙承穀聘妻丁,名畹芬,武進人。父士衍,官蠡縣知縣,母趙及畹芬從。咸豐十年,洪秀全兵破常州,承穀大父起殉焉。或傳承穀亦見執,母感傷發病卒。明年二月壬子夕,畹芬自經死。將死,書所為思親賦及詞六篇,字畫端靜如平時。

彭爵麒女,名詠春,懷寧人;陳寶廉二女慧莊、慧敬,侯官人:皆殉母。詠春哭母殯僧寺,登浮屠自投死。慧敬請以身代母,慧莊居母喪,皆仰藥死。

吳士仁女,獻縣人。幼喪父,無兄弟,誓不嫁養母。會寇至,女求利刃置袖中,扶母出避,遇二寇,擠母僕,母怒詈,寇持刃欲斫,女急呼曰:「毋殺我母!我從若,不則死。」寇乃止。扶母還其家,藏母於室,出問寇饑否?具食使食。食畢,一方飲,一出臥他室中,女躡飲者後,挾刃刺其頸,貫喉,嘶而僕。女陽為嬉笑,拔所佩刀至他室,臥者方起立,遽前剚其胸,亦死,乃負母出走。

王濟源女,棗強人。幼即能事父母。寡兄弟,遂矢不嫁。嘗有盜,夜破門入,女持火槍立暗陬,擊一盜斃,盜乃去。喪父母,葬祭皆如禮,為立後。同治間,寇至,負父母木主行避寇。逾六十,父母忌日,歲時祭墓,猶號泣哽咽。

董桂林女,樂亭人。桂林卒,女十二,矢不嫁,耕織以養母。昌黎富家子,聞其賢,請婚,原代之養,女堅拒不許。母卒,女五十餘矣,鬻田以為斂,存屋數椽,田一畝,杏五樹,女即牖外置母棺,手畚土以封。獨處,晝夜懸刀自衛。又十餘年,鄰里高其義,醵金為營葬。

耿恂女,名一圭,望都人。恂舉人,無子,客授保定。母劉病痺,一圭按摩抑搔,嘗六七晝夜不少休。母少間,因臥床下,恂自外至,誤踐其手,指甲脫,血流至肘,倦不自知也。嘗議婚某氏子,未聘而某氏子夭,女聞泣曰:「我得終事父母矣!」遂矢不字。劉病垂二十年,哽噎不能食,食必女口哺。恂卒,持喪奉病母歸里。逾年,劉亦卒,一圭營喪葬,自為文以表於阡。一圭嘗以生日上塚,掬土以益墓,憊僕墓側,家人掖以歸,數日卒。

吳芬女,開縣人,女次第二。芬,光緒二十三年拔貢生,以知縣發山東,女留侍母。芬病,女聞,夜輒焚香露禱。三十一年,芬卒,女聞大悲,且恚曰:「人謂天有眼,我夜焚香露禱,叩頭至數百,乃漠然不一顧耶?」越日飲藥死,時年十三。

邵氏二女,黟人,長名媚,十五;次名揚,十三。從父入山樵,虎出噬其父,媚持父揮樵斧斫虎,虎負創去,父女皆不死。

蔣遂良女,城步人,虎挾其母去,女奪以還。

徐氏二女,淑雲、淑英,溫江人。父瞽,兄登雲早卒。嫂凌疾革,撫子成龍而泣,淑雲、淑英在側,曰:「我二人在,當扶持以長,嫂何虞?」時成龍方二歲,淑雲、淑英皆不嫁,以女紅事蓄,卒扶持以長。

李鴻普妻郭,禹州人。鴻普母王,明季流賊破州,自經死,失其尸。鴻普將斫檀為之像,未成而卒。郭力紡織,奉其舅及後姑。子以達,稍長,喻以父意。求檀,輒不中像材。郭乃刺左腕,出血盈盂,和香屑為像,復剪發飾其首。以達驚,叩首泣,郭曰:「我姑以節死,我何愛發若血不以奉姑?吾無恙,汝又何悲?」像成,藏潔室,日上飲膳,事如生。

其後又有牛輔世妻張,太原人。姑卒,刻木祀之,飲食必祭。

高位妻段,宛平人。位卒,段年十七,二子幼,依其兄以居。兄勸改嫁,段不可,攜二子徙居小市板屋中。長子早死,次子為吏,以罪徙遼左,乃復撫諸孫。段年九十,孫裔成進士,贖其父以歸。

裔母谷,事姑孝。始處賤,躬灑掃。晨侍盥櫛,食時,就灶下作羹,親上之。食畢,然後退,日以為常。既貴,終不改。或以為言,穀曰:「若毋言,吾與姑故寒苦,姑習我,非我供事,姑終不適。吾老矣!灑掃盥饋以事我姑,此日可多得耶?」康熙二十七年,段卒,年九十六。

鄭光春妻葉,莆田人。光春游湖南,久不歸,葉以紡績養姑。子文炳幼,或不率教,輒拊心號天,文炳懼,向學。姑老病痺,葉負以出入。七年,姑乃卒。

文炳長,娶於吳,念父不歸,婚夕惘惘無歡。吳逡巡得其故,勸文炳行求父,曰:「事姑,我任之!」文炳行求得父以歸,吳已卒,猶處子。文炳子任仁,婦張,能繩其孝。

屈崇山妻劉,鄠縣人。崇山卒,劉奉姑以居。康熙三十年,歲兇,姑勸之嫁,不從。饑益甚,姑泣語劉曰:「我旦暮且死,盍自鬻,尚可活我!」劉泣不應。姑大慟曰:「死耳,夫何言!」劉哽咽久之,乃曰:「如姑命。」自鬻於豪家,得金畀姑,號泣登車去。豪家方具酒食為賀,劉入廁自經死。豪家大恨,以敝槁裹尸棄野外。

謝以炳妻路,仲秀妻鄭,季純妻吳,湖口人。以炳兄弟並早卒,三婦勵節事姑,姑病癰,迭吮之,良愈。

王鉅妻施,鉅,蕭山人;施,富陽人。姑嚴,小不當意,輒呵斥,施屏息不敢聲。姑病反胃甚,醫以為不治,施刲股和藥進,病良已,姑遇施如故。鉅疾作,施視疾憊,病瘵卒,姑猶不善施。鉅以刲股事告,視其尸,信,乃大慟曰:「吾負孝婦!」及疾篤,出珠花付鉅曰:「汝婦孝,以此志吾痛,使汝子孫勿忘。」蕭山人因稱鉅後為珠花王氏。

陳文世妻劉,鄖人。陳、劉皆農家,劉待年於陳。既婚,姑年七十二,病噎,劉割臂和藥以進,疾少間;既而復作,不食已十日,垂盡矣。劉夜屏人,殺雞誓於神,持小刀自劙其胸二寸許,出肝刲半,取布束創,以肝與雞同瀹湯奉姑。姑久不言,忽曰:「湯香甚!」飲之竟,病良愈,劉亦旋平。為乾隆四十四年夏六月事。知縣嘉興李集出俸為買田宅,宅北有大陂,幾三頃,因命曰孝婦陂。

張守仁妻梁,獻縣人。守仁卒,祖姑穆,耄而瞽且痿,日偃仰床蓐,梁傭力以養。或諷梁嫁,梁曰:「我今日嫁,明日祖姑饑且死,義不忍。」祖姑善恚,小不當意,則怒詈,或攫其面,血出,梁事之自若。祖姑卒,依其女以終。

縣又有韓守立妻俞,祖姑及姑皆瞽,或妄言割肉以燃燈可愈,守立原試之,俞請代,刲右股燃之,盡十餘日,祖姑目復明。

路和生妻吳,靖遠人。善事姑。姑喪明,吳侍左右,非整衣不入。或言姑無見也,吳曰:「吾心自不可欺耳。」

諸君祿妻唐,零陵人。姑胡,老無齒,兼病痺,唐日操作畢,輒跪而乳之。或曰:「坐可也。」唐曰:「是乳小兒也,乳姑不可。」

牛允度妻張,通渭人。三十而寡,奉姑謹。嘉慶六年,大祲,求野菜以食。姑老病,久之,不能復食。張貸錢得市脯進姑。又久之,貸不繼,姑病欲絕,張慰之曰:「姑稍待,婦制草笠,可得錢數十,猶足為數日供也。」笠成,賣得錢,姑已死。乃求市脯祭,朝夕哭,以餕餘活夫弟。

游應標妻蕭,新都人。應標出耘,蕭居績。火發翁室,翁老病不能行,蕭冒火入,負翁,將及門,門焚,俱死。

蔣廣居妻伍,桐城人。寡,奉姑徐。嘉慶二十四年,火作,徐年九十六矣,臥不能起。伍自火中奔赴,負徐至灶前,火逼,俱死。伍尸倚墻,背負徐,俱殭立不僕,面如生。

又有扶溝蔣有廣妻陳,救翁;洧川閻惠妻李,救姑:皆火死。

周學臣妻柳,湖口人。早寡。夜,虎突門,翁出視,驚僕。柳徒手擊虎,虎自去。

王德駿妻盛,益陽人。事祖姑孝,病噎,哺以乳。寇掠縣,負姑夜遁,墮虎穴,禱於虎,虎不咥。

張茂信妻方,茂信,河津人;方儀徵人。方嘗割股愈舅疾,舅與茂信皆卒,奉姑劉。姑嚴,方事之謹。當夏,姑病暴下,方躬滌茵席,不以為穢。夜與姑共枕寢,微呻輒起,撫摩抑搔五十餘日,姑愈,亟稱其孝。

林經妻陳,連江人,姑盲性卞,常臆婦藐己,陳斷三指自明,姑為之悔。經病,刲股;經卒,以節終。

張德鄰妻李,遷安人。寡,從弟欲奪其志,力拒。歲饑,驅驢鬻石灰易米以養姑。一日遇盜,泣曰:「驢可將去,丐留囊中物俾我姑,不即餓死!」盜舍之去。

武烈妻趙,烈,永年人;趙,宣化人。趙事姑孝,姑病,夜露禱,得寒嗽疾。烈病疫,或謂口吮胸,汗出則愈,而吮者當病,趙曰:「果爾,死不恤。」卒吮之,烈竟卒,趙病幾殆。貧,操作紡績,諸子成進士,自奉恆觳。親族有緩急,往往傾其貲。出千金置義學,卒,遂祠焉。

孫朗人妻吳,連江人。姑陳,早寡,遺腹生朗人。性嚴急,有不當意,輒堅臥,朗人偕吳跪床下,俟意解,命之起,乃起。朗人卒,吳以節終。

李天挺妻申,日照人。天挺早卒,姑嚴,申年六十,猶終日跪庭中。居姑喪,以毀卒。

劉與齊妻魏,秦州人。既寡,事姑,日被笞罵,歡顏受之。躬薉賤,十餘年不怠。

周志桂妻馮,湘鄉人。姑暴,忍饑以養,猶時時加箠楚。姑病瘓,不能舉杖,叱馮跪自撾,流血,不敢怨。歷三十餘年,人名其里曰孝婦村。

歐陽玉光妻蔡,湘鄉人。玉光母劉,治家有法度。玉光居父喪,以毀卒。蔡承姑教,董家事,率妯娌,與子侄傭奴,各有專職,家漸起。

子惟本,亦娶於蔡。婦家貧,將嫁,宗族周焉,得錢三千有奇,陰置稈薦中,而系鑰其端。父送女還,入室,引鑰,則錢在焉。曰:「孝哉我女,留此以活我!」惟本亦早卒,從姑敬事祖姑,祖姑興,姑執笄侍左,婦自右為約發。盥,姑奉水,婦奉槃。及食,婦具饌,姑侑之。寢,三世連床。一夕,姑起,墮床折肋,婦號泣就援,姑戒勿聲,毋令祖姑驚也。祖姑晚喪明,手足痿痺,挽箯輿,日游庭中,姑肩前,婦肩後。祖姑劉,年至九十,姑蔡,九十六,婦蔡,八十三。曾國籓為之傳,謂:「歐陽姑、婦,雖似庸行無殊絕者,而純孝兢兢,事姑至六十年、五十年之久而不渝,天下之至難,無以逾此。」

蕭學華妻賀,湖南安化人。賀父徙陜西,學華贅其家。年餘,學華歸省母,賀欲與俱,父不許,賀割股肉付夫以奉姑。姑適病,學華烹肉進,病良已。後學華攜賀歸,事姑以孝稱。

張友儀妻陳,福建永定人。事姑孝,姑嘗稱曰:「諸婦汝最樸訥,然酒漿筐篋瑣碎無不治,得吾意者,汝也!」友儀早卒,陳未三十,勉痛事姑,撫孤子。同治初,寇至,負姑入山避,徒行數十里,踵裂血流,屢踣屢起。匿深林中,燃枯枝,採野蓛以活,卒得免。

子日焜妻李,嘗刲股愈母病,事祖姑及姑孝。姑病,割臂進,病目,舐以舌,良已。嘗赴族人飯,心動,歸,正姑病。又嘗宿姻家,夜半,索輿還。姑曰:「吾正念汝,知汝必念我速歸也。」

馮氏,武進人。嫁吉龍大,事舅姑謹。姑病偏廢,飲食臥起皆需馮,而龍大游蕩,欲衒馮以媒估客,馮不可。龍大引外婦入室,舅怒而逐之,馮曰:「姑病,婦終日侍,苦為他事閒,得一人分其勞,甚善。」因持臥具從姑寢。龍大時時毆辱馮,馮未嘗有怨色。舅病,龍大市毒藥授馮,令飲其父,馮擲藥,跪諫數日,龍大別市藥,毆而逼之,馮嘆曰:「我所以不死,為舅姑耳,今無冀矣!」入視姑寢,至龍大所,舉藥盡飲之。謂龍大曰:「我代舅矣,後毋萌此念!」須臾毒發死。

王鉞妻隋,諸城人。敏而有定識。明季,奉姑避兵,航海行數千里。寇至,負姑夜逾垣匿谷中以免。鉞成進士,為廣東西寧知縣。康熙十三年,吳三桂反,鉞城守,賊至,鉞謂隋:「當奈何?」隋出匕首曰:「有此何懼!」賊去,鉞行取主事,隋請以諸子先行。是時賊方盛,行人道絕,隋得敝舟,挾幼子經肇慶、度大庾、入鄱陽湖,水陸行數千里,率僕婢佩刀晝夜警備。家居,地震,自樓墮,血淋漓,持子泣,地搖搖未已,子請避,隋曰:「諸婢壓其下,吾去,死矣!」督家僮發磚石出之,皆復活。火發於樓,煙蔽梯不可登,命以水濡被予諸婢,身持濕衣障火先登,諸婢汲水次第上,火遂得熸。子沛恩、沛憻、沛恂,皆成進士,官於朝,隋益勤儉自斂抑,鄉人稱老實王家。

林云銘妻蔡,雲銘,閩人;蔡名捷,字步仙,侯官人。雲銘,順治十五年進士,授江南徽州推官。鄭成功兵入江,徽州兵叛,蔡矢死不去。官省,還居建寧。耿精忠反,下雲銘獄,蔡憂之,嘔血殷紫,女瑛佩剜臂肉入藥,旋蘇。師至,雲銘乃出獄。雲銘無子,蔡為買妾七,乃生子。蔡御諸妾有恩,所親有婦妒,而五十無子者,蔡延至家,與處三日,歸為夫買妾生子。里婦忤其夫,共指蔡以勸,曰:「毋令林孺人知。」瑛佩為閩清鄭郯妻。

陳龍妻胡,龍溪人。龍少恃勇,為暴於鄉里,父老群謀去害。時胡未嫁,使密勸乘時立

功名。龍亡命為盜海島,父母將別字,胡堅拒。聞龍娶,不貳。龍降,官金門總兵,知胡猶未字,乃成婚。海澄許貞嘗以逋餉系獄,胡告龍代償其負,釋使去,貞卒為名將。

王懃妻岳,曲周人。嶽奉舅姑篤謹,若不能言。懃移家臨清,而商於天津。王倫為亂,將攻臨清,臨清民爭走避,岳請於舅姑曰:「賊將以臨清為窟,必不剪居民以自弱。從眾以行,不死於奔竄,必死於蹂藉,宜若可緩然。」舅姑用其言,出者爭道,多擠入水死。岳曰:「乃今宜可徙,官軍且至,賊方謀出御,不暇捕逃人。且徙者已十八九,今行,無慮蹂藉;今不行,免於賊,或不免於官軍。」遂相將潛出城,還曲周,懃亦歸。人稱其能量事,岳篤謹如故。

魯宗鎬妻硃,名如玉,字又寒,仁和人。事舅姑孝。或以賄干宗鎬,有所關說,硃勸毋受。宗鎬曰:「我度是無利害。」硃曰:「諸為不義事,皆以為無利害耳,奈何以貧隳素行!」宗鎬悟,謝之。

馬叔籥妻丁,揚州舊城人,事舅姑甚謹。叔籥兄弟三,既分,而伯兄以訟破家,丁義不

己食,雖壺酒豆肉必以分。一日,語叔籥,請致家於伯氏,叔籥許之。丁事伯如舅,姒如姑,米鹽纖悉一關姒,嫁時衣裝飾首約臂皆不私。家故賈也,叔籥兄善賈,遂以其家富。叔籥有所請於姒,姒不時給,叔籥怒曰:「乃我家所有,嫂何與?」丁曰:「始讓而終怒,人其謂我何?」勸叔籥毋校。

許光清妻陳,海寧人。善持家。戚有鬻婦者,婦誓死不從,陳偕姒婦硃醵金畀其夫,要之署券。曰:「彼人游蕩,金盡終且鬻婦,不如是,婦不免。」乃招婦至,善視之。其夫死,復醵金贖所居,遣婦還,並前券焚之。鄰童入其室竊壺去,陳戒家人勿言,曰:「彼何以為人?」御婢寬,聞有虐婢者,必以陶潛語勸曰:「彼亦人子也!」

黃開鼇妻廖,開鼇,高安人;廖,沔陽人。開鼇善為針,設肆衡州,廖佐以紡績。開鼇病瘓,廖習為針,針成,置諸版,摩以掌,針乃澤,數以是創,不懈。

開鼇卒,子長發幼,婦劉,監利人,待年於姑氏。稍長,夫婦共為針,長發截鐵,圓本而銳末,持就段,睨火察純窳。劉削竹,綴以鋼,懸雙絙環竹,曳則竹轉以穿針鼻。針良,市者多,家漸裕。洪秀全之徒躪湖南,家破,長發治針益力。當冬,得敝羊裘奉廖,與劉皆敝

褐短褌,手足龜,不敢怠。

長發旋卒,子才三歲,被火,家再破。於是廖語劉曰:「天乎!此誠不可再活,盍同死?」劉對曰:「火,亦常也,姑、婦惟當復食苦耳。」鬻簪珥為貿遷,居賤鬻貴。廖持算,劉主議值。又數年,家復裕。廖老而卞,易怒,劉進淡巴菰,徐言他事輒解;不解,即跪謝,相持泣乃已。廖七十六而卒。

劉既善貿遷,鄰家就求術,劉為謀至詳,貧者貸以貲。同巷居五十餘家,多以貿遷富。開鼇初設肆,才錢六千四百,劉晚年積白金至十萬,督子孫就學,取科目,家益大,年七十九而卒。

黃茂梧妻顧,名若璞,字和知,仁和人。顧好言經世之學,為詩、古文辭,自為集序曰:「若璞不才,少不若於母訓,笄事東生,十有三年。閒事詠歌,大抵與東生相對憂苦之所為作也。東生溘逝,帷殯而哭,不如死之久矣。徒以藐諸孤在。發藏書,日夜披覽,二子從外傅,入輒令隅坐,為陳說吾所明。日月漸多,聞見與積,聖賢經傳,旁及騷雅詞賦,冀以自發其哀思。題曰臥月軒稿。軒為東生所嘗憇,志思也。」東生,茂梧字。顧至康熙中乃卒,年九十。

子燦妻丁,從顧學,亦好言經世,先顧卒。

高其倬妻蔡,名琬,字季玉,漢軍正白旗人,綏遠將軍毓榮女也。毓榮、其倬皆有傳。琬諳政事,其倬章疏文檄每與商榷。能詩,有蘊真軒詩鈔。集中辰龍關、關鎖嶺、江西坡、九峰寺諸篇,追懷其父戰績,尤悲壯,為世傳誦。嘉慶間,鐵保錄滿洲、蒙古、漢軍旗人詩,為熙朝雅頌集,以琬為餘集首。同入選者,珠亮妻、嵩山妻皆宗室女。張宗仁妻高,名景芳,詩最多。珠亮妻有養易齋詩,嵩山妻有蘭軒詩,景芳有紅雪軒詩。

陳之遴妻徐,名燦,字明霞,吳縣人。之遴自有傳。徐通書史,之遴得罪,再遣戍,徐從出塞。之遴死戍所,諸子亦皆歿。康熙十年,聖祖東巡,徐跪道旁自陳。上問:「寧有冤乎?」徐曰:「先臣惟知思過,豈敢言冤?伏惟聖上覆載之仁,許先臣歸骨。」上即命還葬。徐晚學佛,更號紫䇾,有拙政園詩詞集。詞尤工,陳維崧推為南宋後閨秀第一。畫得北宋法。

詹枚妻王,名貞儀,字德卿。枚,無為人;貞儀,泗州人,而家江寧,祖者輔,官宣化知府,坐事戍吉林,貞儀年十一。者輔卒戍所,從父錫琛奔喪,因僑居吉林,侍祖母董,讀書學騎射。十六還江南,又從錫琛客京師,轉徙陜西、湖北、廣東,二十五歸於枚。後五年,嘉慶二年,卒。

貞儀通天算之學,能測星象,旁及壬遁,且知醫。為詩文皆質實說事理,不為藻採。撰星象圖釋二卷,歷算簡存五卷,籌算易知、重訂策算證訛、西洋籌算增刪,皆一卷,象數窺餘四卷,女蒙拾誦、沉痾囈語,皆一卷,繡紩餘箋、文選詩賦參評,皆十卷,德風亭集二十卷。

貞儀病且死,謂枚曰:「君門祚薄,無可為者。我先君死,不為不幸。平生手稿,為我盡致蒯夫人,蒯夫人能彰我。」蒯夫人者,吳江蒯嘉珍妻錢,附見曾伯母錢綸光妻陳傳中,時僑居江寧,貞儀與相習,枚以貞儀書歸焉。錢侄儀吉,為歷算簡存序,言:「貞儀有實學,不可沒,班惠姬後一人而已。」女子治歷算蓋至鮮。

咸豐間,膠州柯蘅妻李,名長霞,邃於選學,著文選詳校八卷。工詩,有錡齋詩集。

光緒間,濟陽艾紫東妻徐,名桂馨,治音韻之學,有切韻指南四卷。

郝懿行妻王,名照圓,字瑞玉,一字婉佺,福山人。懿行見儒林傳。照圓文辭高曠,得六朝人遺意。懿行有所述作,照圓每為寫定題識。其所自為書有列女傳補註八卷,序曰:「列女傳補註者,補曹大家註也。照圓六歲而孤,母林夫人恩勤鞠育,教以讀書。嘗從燕間,顧照圓而命之曰:『昔班氏註列女傳十五卷,今其書亡,如能補為之註,是餘所望於汝也。』照圓謹志之不敢忘。分陰遄邁,奄忽四七,寸草盟心,遂成銜恤。追省前言,隕越滋懼。不揣愚蒙,略依先師之詁,用達作者之意,凡所詮釋,將以通其隱滯,取供吟諷。至於義所常行,或傳記成文,舊人已注,則皆闕而弗論。誠知疏陋,無能纂續前修,庶幾念昔先人,少酬明發之懷。補註成,請夫子辨析疑義,時加訂正,無隱乎爾,竊所慕焉!」

又校正列仙傳二卷,舊有贊,考以隋書經籍志,知為晉郭元祖撰,復別出為一卷。又集傳記言占夢者為夢書一卷,皆自為序,附懿行書以行。尤喜言詩,著葩經小記,書未成。懿行撰詩問,謂與照圓相問答,條其餘義,別為詩說,皆採照圓說為多。光緒間,其孫聯薇以書進,因誤為照圓著雲。自照圓為列女傳補注,其後又有汪遠孫妻梁校注。

梁,名端,字無非,錢塘人。幼為祖玉繩所愛。元和顧之逵校刻列女傳,玉繩為審定,端輒臚其同異,退而筆之,玉繩為之折衷。既歸遠孫,與參酌增損。端既卒,遠孫為刻行。

陳裴之妻汪,名端,字允莊。七歲賦春雪詩,擬以謝道韞,因又字小韞,錢塘人。長為

詩,旨遠而辭文,嘗撰定明詩初、二集,上始開國,下逮遺民,都三十家,附錄又七十人。自定凡例,以為:「初集,猶主盟之晉、楚;二集,猶列國之宋、鄭、魯、衛;附錄,猶附庸之邾、莒、杞、薛。」梁德繩稱其宗尚清蒼雅正,能掃前後七子門徑。吳振棫稱其論一代升降正變,元元本本,縱橫莫當。端所自為詩,有自然好學齋集。裴之卒,子又有疾,舅文述素奉道,端詩亦多為道家語。既卒,諸侄重定其集,盡刪晚作,二本並行於世。

汪延澤妻趙,名棻,字儀姞。延澤,烏程人;趙,上海人,戶部侍郎秉沖女也。幼讀書,能詩文,有濾月軒詩集四卷,文集二卷,詞一卷。自為序,略曰:「宋後儒者多言文章吟詠非女子所當為,故今世女子能詩者,輒自諱匿,以為吾謹守『內言不出於閫』之禮。反是,則迋欺炫鬻於世,以射利焉耳。是二者,胥失之也。禮昏義女師之教,婦言居德之次,鄭君注云:『婦言,辭令也。』夫言之不文,行而不遠,文章吟詠,非言辭之遠鄙倍者歟?何屑屑諱匿為!」

子曰楨,撰二十四史日月考,趙為之序,曰:「劉羲叟撰劉氏輯術,迄於五季,書久佚,僅存通鑒目錄。自宋迨明,六百餘年,未有續為之者。曰楨好史學,習算,考當時行用本術,如法推步,得其朔閏。自史記至新、舊唐書,屬草已一百餘卷,餘亟欲睹其成,預為此序,俾寫定冠諸簡端。」

吳廷珍妻張,廷珍,常熟人,道光六年進士,官至刑部員外郎。張名糸習英,字孟緹,陽湖人。世父惠言,父琦,皆博通能文章。糸習英與諸女弟承其教,咸有述作,皆能詩。糸習英兼為詞,秀逸有王沂中、張炎遺意。妹惸英亦能詩詞;綸英尤工書,傳琦筆法,真書出入歐陽、顏、楊諸家,分書自北碑上溯晉、漢,遒麗沉厚;紈英兼治古文。糸習英嘗編次國朝列女詩錄,紈英為作傳,簡雅合法度。惸英,江陰章政平妻;綸英,同縣孫劼妻;紈英,太倉王曦妻。

程鼎調妻汪,名嫈,字雅安,歙人。好學,通儒家言,詩文皆雅正。病將卒,為詩曰:「秋風一葉落,餘亦歸荒墟。」遺書戒其子葆,言家事至詳。復謂:「武侯著書,內有八務、七戒、六恐、五懼,武侯第一流人,務一,而戒恐懼居其三,可不識所致力耶!」葆編其所作為雅安書屋詩文集。

陳瑞妻繆,名嘉蕙,字素筠,昆明人。工書、善畫。光緒中,召入宮供奉,為皇太后嘉賞,特賜三品服。

時同被召者,馬某妻阮,字蘋香,儀徵人,賜名玉芬。富樂賀妻王,名韶,字矞雲,杭州駐防滿洲人,著有冬青館詩。仁興妻瓜爾佳氏,名畫梁,亦杭州駐防滿洲人,著有超範室畫範。

耀州三婦:一青嘉努妻,一納岱妻,一邁圖妻,所居寨曰蕎麥沖,在耀州城南。天命十年六月癸卯,明將毛文龍遣兵三百夜薄寨,方逾墻入,寨兵未即出,三婦者見之,倚車轅於墻,以為梯,青嘉努妻持利刃先偕登城奮擊,三百人皆驚,墜墻走。耀州守將揚古利以兵至,追擊,盡殲之。太祖召三婦,賚金、帛、牛、馬,賜青嘉努、納岱妻備御,邁圖妻千總。

杉松郵卒婦,祿勸人,失其姓。康熙五十七年正月,有常應運者為亂,逼杉松,諸郵卒方耕於山,無御者。婦曰:「此可計走也。」挾鉦鳴山巔,若且集眾,賊引去,婦乃走告夫,州始為備。事定,知州李廷宰聚父老賚婦酒食,具鼓吹,簪勝披錦,以矜於市民。

楊芳妻龍,芳,松桃人;龍,華陽人。芳有傳。龍善鼓琴,工畫蘭。嘉慶十一年,芳自寧陜鎮總兵署固原提督,龍留寧陜。是歲秋,鎮兵以餉不給,將叛。龍使告署總兵楊之震,之震不之省。或請龍行避亂,龍曰:「不可,若我出而兵叛,是知其叛也,人其謂我何?」七月辛亥夕,亂作,芳素得兵心,兵有以匪降者,尤感芳不殺,皆入署為龍衛。民婦就避兵,廊廡盈焉。龍嚴戒奴婢毋號泣,鄉明,叛兵叩閤請謁,諸避兵者忷懼,請毋納。龍曰:「愚哉!彼輩且自入,孰能御之?」乃啟門,納其渠數十人,咸泣謝,且請龍行。龍謂之曰:「若曹雖叛戕官,其渠罪不逭,於多人何尤?主將旦夕歸,白若曹於朝,非盡殲也,可各罷歸伍。」叛兵不欲罷,堅請龍行,龍命以輿來,盡出諸避難者,而殿其後。叛兵送至清澗,哭而返。龍兄為興安知府,乃之興安。芳自固原至,撫叛兵,復定。

蒲大芳者,叛兵渠也,請於芳,迎龍歸。芳遣大芳等二十輩以往,龍初舉子,即冒雪就道,道中大芳與其曹詬爭,舉刀傷其曹。行至漢陰,龍使假刑具於有司,召大芳責曰:「汝叛,幸不死,更弄刀杖,又待叛耶?」杖之四十,械而行。三日,將至寧陜,其曹十九人者為之請,乃令脫械。

龍至,語芳曰:「事雖事,然君且有遠行。」芳曰:「何至是?」龍曰:「朝廷自有法度,兵叛事大,不容無任其咎者。」果有命戍伊犁。龍歸侍姑,姑風緩不能言,惟龍達其意,左右在視。居姑喪盡禮。芳復起,遷湖南提督,道光五年,龍卒。

崔龍見妻錢,名孟鈿,字冠之,一字浣青。龍見,永濟人;錢,武進人,侍郎維城女。九歲刲臂療父疾。歸龍見,事姑謹,龍見以進士官州縣,為四川順慶知府。川東啯匪為亂,龍見師出御,賊自間道來襲,吏民驚擾。錢詗賊自府西至,遣人掣渡舟泊東岸。賊至,不得渡,遂引去。

及為湖北荊宜施道,值白蓮教匪為亂,龍見出督餉,錢居危城中,烽火四偪,以龍見指發書,戒所屬州縣,令收附郭積聚,謹守備,毋與賊浪戰。賊偵有備,亦引去。

龍見在官廉,錢每出餘財周戚黨。自四川還,泊燕子磯,見渡舟覆溺,出錢募救者,活十餘人,皆應試士也,羅拜岸上。龍見卒,教諸子成立。錢工詩詞,即以「浣青」名其集。

沈葆楨妻林,名普晴,字敬紉,侯官人,雲貴總督則徐女也。則徐、葆楨皆有傳。葆楨故則徐甥,林六七歲時,嘗侍諸姑坐,臧否戚黨諸子弟。戲以諮林,輒曰:「無逾沈氏兄賢。」及歸葆楨,葆楨貧,董中廚,斥奩具佐饈,能得姑歡。

咸豐六年,葆楨知廣信府,八月,出行縣,洪秀全將楊輔清自吉安潛師越山谷入。戊子,破貴溪,己丑,破弋陽。吏具舟促林避寇,林勿行。庚寅,葆楨還,時遵義鎮總兵饒廷選駐軍玉山,乃為書乞援,而輔清兵益進,去廣信八十里。辛卯,廷選報書,言水涸,師不得下。僕役散走,林懷印倚井坐誓死。乙夜,城南火,達曙,大雨火滅。林謂葆楨曰:「城中炊煙斷,火何由起?此賊諜所為,以空城告也。今日賊當至,吾殉君固其所。」解劍授葆楨曰:「雨甚,吾不可露坐,賊至,君以劍當之,使吾倉卒得入井也。」賊得諜,知城無人,易之,待霽乃發。癸巳,輔清兵復進四十里,而廷選師至,葆楨徒步迎以入。甲午,輔清兵薄城,廷選軍出御,其裨將畢定邦、賴高翔戰甚力,林煮粥啖士卒,士卒益奮。丁酉,賊大至,圍合,文吏竄伏,饋運犒勞,皆林會計而出納之。乙亥望,大戰,解圍,輔清乃引去。

自是葆楨治軍日有聲,擢江西巡撫。治船政,林佐治官書,一一中條理。治家尤有節度,斷線殘紙,必儲以待用。方葆楨試禮部,鬻金條脫治行,代以蜀藤,雖貴,弗易也。光緒三年,卒。

王某妻陳,皋蘭人。同治六年,河州回攻蘭州,師自平番來援,阻黃河不得渡。陳家河北,令其子化鳳集族黨,以舟濟師,蘭州以全。

李某妻趙,營山人。縣多虎,李子赴市,暮未還,李立村外待。虎驟至,李驚呼,趙聞,持梃出,與虎鬥,虎弭尾去。

羅傑妻陳,安徽太平人。傑與陳共入山採薪,虎攫傑,陳與爭,不得脫,急觸虎口,虎舍傑咥陳,陳死,傑得脫。

楊某妻唐,衡陽人。夫婦偕耘,虎攫其夫去,唐曳虎尾不舍,三逾嶺,傷左臂,卒負夫歸。數日夫死,以節終。

姚旺妻潘,旌德人。旺遇虎,潘奔救,同死。

蓋氏,吉林涼水泉金廣年妻也。廣年貧,眇一目,有友與狎。一日,戲語廣年:「汝何修得美婦?」廣年心動,即曰:「若艷我婦,予我百金,以婦與若。」遂與友偕還語蓋,蓋曰:「貧死命也!以貧而鬻其婦,生何心矣?」噭然哭。廣年出以語友,聞哭止,入視,則自罄死矣。呼友共解之,友因摩其足,蓋蘇,以足抵友僕,走廚下,取刀自斫其足,立斷。昏臥血中,鄰里趨視,唾廣年。其友懼,請以百金療,廣年亦悔,力負販,育子姓甚繁。


\end{pinyinscope}