\article{列傳一百四十}

\begin{pinyinscope}
達椿子薩彬圖鐵保弟玉保和瑛覺羅桂芳

達椿,字香圃,烏蘇氏,滿洲鑲白旗人。乾隆二十五年進士,選庶吉士,散館授戶部主事,遷員外郎。歷翰林院侍講、侍讀、國子監祭酒、詹事府詹事、大理寺卿。二十九年,入直上書房,充四庫全書總閱,累擢禮部侍郎,兼副都統。四十五年,坐會同四譯館屋壞,斃朝鮮使臣,革職留任。五十四年,左遷內閣學士。達椿直內廷,不附和珅,數媒孽其短,以曠直褫職,仍留上書房效力行走。尋授翰林院侍講學士,復迭以大考降黜授檢討。仁宗知其屈抑,至嘉慶四年,詔:「達椿因曠班被譴,其過輕,當時劉墉亦緣此降官;今劉墉已為大學士,達椿尚未遷擢,加恩補授內閣學士兼副都統。」子薩彬圖,時亦同官,命達椿班次列薩彬圖之前。歷禮部、吏部侍郎,兼翰林院掌院學士,擢左都御史兼都統,遷禮部尚書。六年,典會試。七年,卒。

薩彬圖,乾隆四十五年進士,授戶部主事,遷員外郎。典貴州鄉試,改歷翰詹,累遷內閣學士兼副都統。和珅既伏法,仁宗不欲株連興獄,而薩彬圖屢疏言和珅財產多寄頓隱匿,有嘗管金銀使女四名,請獨至慎刑司訊鞫。詔嚴斥之,命從王大臣訊,不得實,議革職,予七品筆帖式,效力萬年吉地。尋以其父年老,召還京,授戶部主事,累擢倉場侍郎。十二年,出為漕運總督。逾三歲,京倉虧缺事覺,降光祿寺卿。遷盛京戶部侍郎,十六年,坐奉天災民流徙出邊,褫職。尋卒。

鐵保,字冶亭,棟鄂氏,滿洲正黃旗人。先世姓覺羅,稱為趙宋之裔,後改今氏。父誠泰,泰寧鎮總兵,世為將家。鐵保折節讀書,年二十一,成乾隆三十七年進士,授吏部主事,襲恩騎尉世職。於曹司中介然孤立,意有不可,爭辯勿撓。大學士阿桂屢薦之,遷郎中,擢少詹事,因事罷。尋補戶部員外郎,調吏部。擢翰林院侍講學士,仍兼吏部行走,歷侍讀學士、內閣學士。五十四年,遷禮部侍郎,兼副都統。校射中的,賜花翎。調吏部。

嘉慶四年,奏劾司員,帝責其過當,左遷內閣學士,轉盛京兵部、刑部侍郎,兼奉天府尹。尋復召為吏部侍郎,出為漕運總督。五年,值車駕將幸盛京,疏請御道因舊址,勿闢新道;裁革餽送扈從官員土儀;禁從官妄拿車馬:上嘉納之。七年,遷廣東巡撫,調山東。河決衡家樓,詔預籌運道。九年三月,漕運迅速,加太子少保。尋以水淺船遲,革職留任。十年,擢兩江總督,命覆鞫安徽壽州武舉張大有妒奸毒斃族侄獄,蘇州知府周鍔受賄輕縱,及初彭齡為安徽巡撫,勘實置法。鐵保坐失察,褫宮銜,降二品頂戴,尋復之。

十二年,疏請八旗兵米酌給二成折色,詔斥妄改舊章,革職留任。先後疏論治河,請改建王營減壩,培築高堰、山盱堤后土坡及河岸大堤,修復雲梯關外海口,遣大臣勘議,並採其說施行。十四年,運河屢壞堤,荷花塘決口合而復潰,鐫級留任。山陽知縣王伸漢冒賑,酖殺委員李毓昌,至是事覺,詔斥鐵保偏聽固執,河工日壞,吏治日弛,釀成重獄,褫職,遣戍烏魯木齊。逾年,給三等侍衛,充葉爾羌辦事大臣。尋授翰林院侍講學士,調喀什噶爾參贊大臣。授浙江巡撫,未之任,改吏部侍郎。擢禮部尚書,調吏部。請芟吏、兵兩部苛例,條陳時政,多見施行。林清之變,召對,極言內監通賊有據,因窮治逆黨,內監多銜恨,遍騰謗言。會伊犁將軍松筠劾鐵保前在喀什噶爾治叛裔玉素普之獄,誤聽人言,枉殺回民毛拉素皮等四人,上怒,追念江南李毓昌之獄,斥其屢蹈重咎,褫職,發往吉林效力。二十三年,召為司經局洗馬。道光初,以疾乞休,賜三品卿銜。四年,卒。

鐵保慷慨論事,高宗謂其有大臣風。及居外任,自欲有所表見,倨傲,意為愛憎,屢以措施失當被黜。然優於文學,詞翰並美。兩典禮闈及山東、順天鄉試,皆得人。留心文獻,為八旗通志總裁。多得開國以來滿洲、蒙古、漢軍遺集,先成白山詩介五十卷,復增輯改編,得一百三十四卷,進御,仁宗製序,賜名熙朝雅頌集。自著曰懷清齋集。

弟玉保,字閬峰。乾隆四十六年進士,入翰林,有才名。高宗親試八旗翰詹,與兄鐵保並被擢,時比以郊、祁,軾、轍。官至兵部侍郎,究心兵家言。川、楚教匪起,嘗原自效行間。會上欲用為巡撫,為和珅所阻,鬱鬱卒,年甫四十。

和瑛,原名和寧,避宣宗諱改,字太葊,額勒德特氏,蒙古鑲黃旗人。乾隆三十六年進士,授戶部主事,歷員外郎。出為安徽太平知府,調潁州。五十二年,擢廬鳳道,歷四川按察使,安徽、四川、陜西布政使。五十八年,予副都統銜,充西藏辦事大臣。尋授內閣學士,仍留藏辦事。和瑛在藏八年,著西藏賦,博採地形、民俗、物產,自為之注。

嘉慶五年,召為理籓院侍郎,歷工部、戶部,出為山東巡撫。七年,金鄉皁役之孫張敬禮冒考被控,知縣汪廷楷置不問,學政劉鳳誥以聞,下和瑛提鞫,誤聽濟南知府德生言誣斷,為給事中汪鏞所糾。上以和瑛日事文墨,廢弛政務,即解職,命鏞從侍郎祖之望往按,得實,褫和瑛職,又以匿蝗災事覺,譴戍烏魯木齊。尋予藍翎侍衛,充葉爾羌幫辦大臣,調喀什噶爾參贊大臣。

九年,授理籓院侍郎,仍留邊任。疏言:「喀什噶爾、英吉沙爾倉儲足供軍食,請減運伊犁布疋,改徵雜糧四千石,減價出糶,且請嗣後折收制錢,以免運費。」允之。劾喀喇沙爾歷任辦事大臣私以庫款貸與軍民,及土爾扈特、回子取息錢入己,降革治罪有差。十一年,召還京為吏部侍郎,調倉場。未幾,復出為烏魯木齊都統。十三年,塔爾巴哈臺參贊大臣愛星阿欲調瑪納斯戍兵四百人番上屯田,和瑛謂瑪納斯處極邊,戍兵專事操防,不諳耕作,咨駁以聞,上韙之。

十四年,授陜甘總督。坐前在倉場失察盜米,降大理寺少卿。十六年,遷盛京刑部侍郎。復州、寧海、岫巖饑,將軍觀明以匿災罷免,授和瑛為將軍,廉得邊門章京塔清阿等承觀明意,諱災不報,降革有差。尋以誤捕屯民張建謨為盜,鍛鍊成獄,刑部覆訊雪其冤,議革和瑛職,詔寬之,留任。調熱河都統,未上,召為禮部尚書,調兵部。坐失察盛京宗室裕瑞強娶有夫民婦為妾,降盛京副都統,遷熱河都統。二十一年,授工部尚書。命赴甘肅按倉庫虧缺,得總督先福徇庇及貪縱狀,治如律。二十二年,調兵部,加太子少保,歷禮部、兵部。二十三年,授軍機大臣、領侍衛內大臣,充上書房總諳達、文穎館總裁。逾一歲,調刑部,罷內直。道光元年,卒,贈太子太保,謚簡勤。

和瑛嫺習掌故,優於文學,著書多不傳。久任邊職,有惠政。後其子璧昌治回疆,回部猶歸心焉。璧昌自有傳。

覺羅桂芳,字香東,隸鑲藍旗,總督圖思德孫。嘉慶四年進士,選庶吉士,授檢討。嘗召對,仁宗曰:「奇才也!」不數年,累擢內閣學士。十一年,入直上書房,遷禮部侍郎,歷吏部、戶部侍郎,兼副都統、總管內務府大臣、翰林院掌院學士。迭典順天、江南鄉試,兼直南書房。桂芳家素貧,有門生餽納,曰:「執贄禮甚古。某忝佐司農,俸入粗給,無藉乎此。」封還之。大學士祿康輿夫聚博,命偕侍郎英和按治,無所徇。上嘉其不避嫌怨。

十八年,教匪林清逆黨闌入禁城,桂芳方直內廷,偕諸王大臣率兵殲捕,敘勞,加二級。上遇變修省,訓誡臣工,頒禦制文七篇,示內廷諸臣,命各抒所見,書以進御。桂芳書罪己詔後曰:「皇上臨御以來,承列聖深仁厚澤,日以愛民為政,四海之內,莫不聞睹。今茲事變,豈不怪異?而臣竊以為此未足為聖德之累。昔孔子論仁至於濟眾,論敬至於安百姓,皆曰:『堯、舜其猶病諸。』豈真以堯、舜之聖為未至哉?夫天下之大,萬民之眾,而決其無一夫之梗者,蓋自古其難之。然而揆之人事,則實有未盡者。夫林清先以習教被系,既釋歸,轉益煽亂。數年之間,往來糾結於曹、衛、齊、魯之間,其黨至數千人。閹寺職官,竟有與其謀者,而未事之先,曾無一人抉發,是吏無政也。藏利刃,懷白幟,度越門關,飲於都市,無詗而知者,是邏者、門者無禁也。禁兵千計,賊不及百,闔門而擊之,俄頃可盡,乃兩日一夜始悉擒戮,是軍無律也。夫吏惰卒驕,文武並弛,而法制禁令為虛器,則事之可憂,豈獨在賊?我皇上觀微知著,洞悉天下之故,詔曰『方今大弊在因循怠玩』,至哉言乎!臣敬繹之,蓋因循怠玩,亦有所由。無才與識,則有因循而已;無志與氣,則有怠玩而已。是故得人而任之,則因循怠玩之習不患不除。儻非其人,微獨不能除其習而已;就令除之,不因循而且為煩苛,不怠玩而且為躁競,其無裨於治則均耳。是在皇上詢事考言,循名責實,器使之以奏其能,專任之以收其效,因小失而崇丕業,在陛下一旋轉間耳。」

書行實政論後曰:「實心者何?忠是也。忠者一於為國,而不亟亟於求上之知。其所以急於公者如急於己,一政而便於民,其行之而恐不及也;一政而不便於民,其去之恐不速也。不以避疑謗而易其是非之公,不以處疏逖而違其夙夜之志。故其於政也,籌之至審,而不為旦夕之謀;行之務當,而不揣詔旨之合;惟力是視,不必其事之諒於人;惟善之從,不必其謀之出於己。若是者謂之實政。夫為臣之道,疇不當忠,然而忠之實蓋如此。非然者,初無寸勞,而已為見功之地;未必加譴,而已存巧避之心。取容於唯諾,而不以國事為憂;快意於愛憎,而不以人才為惜。如斯人者,雖我皇上日討而訓之,尚望其能行實政乎?夫政者,上所以治天下之具。然而行之以實,乃能有功,不則文具而已。官無實政,民乃不治,非細故也。皇上震動恪恭,求賢納諫,敕中外諸臣,改慮易志。稍有人心者,疇敢不勉;而臣所欲言者,則又在陛下之心矣。臣昨歲恭錄乾隆朝臣孫嘉淦三習一弊疏於御制養心殿記冊末,伏原萬幾之暇,時賜觀覽。用其說以考諸臣之政,因以識諸臣之心,則賢才不患其不思奮,庶績不患其不咸熙。較臣管蠡之見,似更有助於高深焉。」

又論致變之源,由於民窮,民窮由於幣輕,幣輕則國與民交病。論刑用重典而不得其平,則不能格奸定亂。論民惑邪教,由士大夫好言因果利益有以導之。因事納規,所言多切中時弊。於是復條陳時事,或見之,謂其未必盡合上意。桂芳慨然曰:「此何時,尚以迎合為言耶?」及上,嘉納之,命暫在軍機處學習行走。未幾,授軍機大臣。

十九年,軍事竣,以贊畫功賜桂芳子炳奎七品小京官。尋命往廣西按事,授漕運總督。未至廣西,於武昌途次病疫,卒。上以桂芳明慎直爽,方鄉用,至是優詔褒恤,嘆為「良才難得」,贈太子少保,加尚書銜;復以曾授三阿哥讀書,喪至京師,命三阿哥往奠,禦制詩悼之,謚文敏。著有經進、敬儀堂詩存,才華豐贍,為時所稱。

論曰:承平既久,八旗人士起甲科、列侍從者,亦多以文字被恩眷。達椿忤權相,晚乃見用,其守正有足稱。鐵保、和瑛並器識淵雅,述作斐然。桂芳通達政體,建言諤諤,最為一時俊才,年命不永,未竟其用,惜哉!


\end{pinyinscope}