\article{列傳二百七十三}

\begin{pinyinscope}
文苑三

張澍邢澍莫與儔子友芝陸繼輅從子耀遹彭績

洪頤煊兄坤煊弟震煊鄧顯鶴萬希槐周濟陳鶴

徐松沈堯陳潮李圖李兆洛承培元宋景昌繆尚誥六承如

錢儀吉從弟泰吉包世臣齊彥槐姚椿顧廣譽

張鑒楊鳳苞施國祁黃易瞿中溶張廷濟沈濤陸增祥

董祐誠方履籛周儀暐俞正燮趙紹祖汪文臺湯球

潘德輿吳昆田張維屏譚敬昭彭泰來梅曾亮管同劉開

毛岳生湯鵬張際亮龔鞏祚魏源方東樹從弟宗誠

蘇惇元戴鈞衡魯一同子蕡譚瑩熊景星黃子高瑩子宗浚

吳敏樹楊彞珍周壽昌李希聖斌良錫縝李云麟

何紹基孫維樸李瑞清馮桂芬王頌蔚葉昌熾管禮耕袁寶璜

李慈銘陶方琦譚廷獻李稷勛張裕釗範當世硃銘盤楊守敬

吳汝綸蕭穆賀濤劉孚京林紓嚴復辜湯生

張澍,字介侯,武威人。父應舉,有孝行。嘉慶四年,澍年十八,成進士。是科得人最盛,澍選庶吉士,文詞博麗。散館改知縣,初令玉屏,以病歸。敘防河勞,選屏山,攝興文,丁父艱。再起,知永新。署臨江通判,坐徵解緩,罷官。開復,補瀘溪,復以憂去。

澍性亢直,所至輒有聲。在黔時,巡撫初彭齡過縣,澍杖其僕之索金者。座主蔣攸銛督四川,甫下車,舉劾屬吏,風採嚴峻。澍上書論其循情市恩,黜陟不當,以此官不遂。務博覽經史,皆有纂著。游跡半天下,詩文益富。留心關、隴文獻,蒐輯刊刻之。纂五涼舊聞、三古人苑、續黔書、秦音、蜀典,而姓氏五書尤為絕學。自著詩文外,又有詩小序翼、說文引經考證。

同時甘肅有與之同名者,曰邢澍,字雨民,階州人也。兩人學派亦略相近。乾隆五十五年進士,官至南安知府。好古博聞,孫星衍輯寰宇訪碑錄,多資於澍。著有關右經籍考、兩漢希姓錄、金石文字辨異、守雅堂集。

莫與儔,字猶人,獨山州人。少有志操,兄歿,持期服,不與試。嘉慶四年,硃珪、阮元總裁會試,所拔取多樸學知名士,與儔亦以是年成進士,選庶吉士。散館,改令鹽源縣。俗,富民買田好擇取無稅者,貧民往往鬻產存賦,久輒逃亡。與儔責賦富人,而貰其隱占罪。又上言河西寧遠子稅所府隸橫征病民,得裁去。木裡喇嗎左所有山產銀銅,布政使符縣開礦,與儔持不可,以為礦山實土官經堂所據,奸民所呈地圖距經堂遠,實無礦,開廠聚眾,滋擾夷境,貪小利,賈大釁,事誠不便。大吏檄與儔覆勘,至則礦山果在經堂右。其眾嚴兵以待,既瞻與儔貌,聆其溫語,皆解甲羅拜。縣令至,土司例有供餽,盡卻之,又懸諸禁。比還,老幼遮道獻酒,填咽不得前。舉治行卓異,以父憂去。母老,遂請終養。

久之,被吏部檄復起,自請改教授,選遵義。士人聞其至,爭請受業。學舍如蜂房,猶不足,僦居半城市。旦暮進諸生而詔之:「學以盡其下焉者而已,上焉者聽其自至可也。程、硃氏之論,窮神達化,不越灑埽應對日用之常。至六藝故訓,則國朝專經大師,實邁近古。」其稱江、閻、惠、陳、段、王父子,未嘗隔三宿不言,聽者如旱苗之得膏雨。其後門人鄭珍及子友芝遂通許、鄭之學,為西南大師。與儔著二南近說,詩文散佚。友芝記其言行為過庭碎錄。

友芝,字子偲。家世傳業,通會漢、宋。工詩。真行篆隸書不類唐以後人,世爭寶貴。友芝亦樂易近人,臒貌玉立,而介特內含。道光十一年舉人,在京師遠跡權貴。胡林翼、曾國籓皆其舊好,留居幕府,評騭書史外,榮利泊如也。咸豐時,嘗選取縣令,棄去。至是中外大臣密疏薦其學行,有詔徵至,復謝不就。卒,年六十一。著黔詩紀略、遵義府志、聲韻考略、郘庭詩鈔、宋元舊本經眼錄、樗繭譜注、唐本說文木部箋異。

陸繼輅,字祁孫,陽湖人。幼孤,生母林嚴督之,非其人,禁勿與游。甫成童,出應試,得識丁履恆,歸告母,母察其賢,始令與結。其後益交莊曾詒、張琦、惲敬、洪飴孫輩,學日進。嘉慶五年舉人,選合肥訓導。以修安徽省志敘勞,遷貴溪令,三年引疾歸。繼輅儀幹秀削,聲清如唳鶴。不以塵務經心,惟肆力於詩。清溫多風,如其人也。

常州自張惠言、惲敬以古文名,繼輅與董士錫同時並起,世遂推為陽湖派,與桐城相抗。然繼輅選七家古文,以為惠言、敬受文法於錢伯坰,伯坰親業劉大櫆之門;蓋其淵源同出唐、宋大家,以上窺史、漢,桐城、陽湖,皆未嘗自標異也。繼輅著崇百藥齋集、合肥學舍札記。

從子耀遹,字劭文。縣學生。工為詩,喜金石文字,與繼輅齊名。其為人韜斂精採,而遇事侃侃無所撓。游公卿間,尤長尺牘。嘗客陜西巡撫幕,教匪反滑縣,那彥成過長安,聞耀遹名,即請見,為陳機宜數十事,因囑具草以聞,多施行。道光初,舉孝廉方正,選阜寧教諭,卒。有雙白燕堂集、金石續編。

繼輅所鈔七家文者,大櫆、惠言、敬外,則方苞、姚鼐、硃仕琇、彭績也。

績,字秋士,長洲人。品詣孤峻。乾隆末,窮而客死。無子,年四十四。族子紹升曰:「人之吊先生者,悲其窮。吾獨謂先生竹柏之性,有節有文採,其英亦元結、孟郊之匹,未見其窮也。」有秋士遺集。餘六人皆自有傳。

洪頤煊,字旌賢,臨海人。少時自力於學,與兄坤煊、弟震煊讀書僧寮,夜就佛鐙講誦不輟。學使阮元招頤煊、震煊就學行省,名日起。嘉慶六年,充選拔貢生。入貲為州判,權知新興縣事。適阮元督粵,知頤煊學優非吏才,延致幕府,相與諮諏經史。後卒於家。性喜聚書,廣購嶺南舊本至三萬餘卷,碑版彞器多世所罕覯。著禮經宮室答問、孔子三朝記、管子義證、漢志水道疏證、讀書叢錄、臺州札記、筠軒詩文集。

坤煊,字載厚。乾隆末,以拔貢生舉鄉試,題名後十餘日卒。

震煊,字百里。精選學,詩才敏贍。阮元修經籍篡詁、十三經校勘記皆任其役。後頤煊十二年充選拔貢生。既廷試,貧不克歸,遂以客死。著夏小正疏義。

鄧顯鶴,字子立,新化人。少與同里歐陽紹洛以詩相勵,游客四方,所至傾動。嘉慶九年舉人。厭薄仕進,一以篡著為事,系楚南文獻者三十年,學者稱之曰湘皋先生。內行修,事兄白首無間,撫其子勤於己子。尤篤於師友風義。嘗以為洞庭以南,服嶺以北,屈原、賈誼傷心之地也,歷代通人志士相望,而文字湮鬱不宣。乃從事搜討,每得貞烈遺行於殘簡斷冊中,為之驚喜狂拜,汲汲彰顯,若大譴隨其後。凡所著有資江耆舊集、沅湘耆舊集、楚寶增輯考異、武岡志、寶慶志、硃子五忠祠傳略及續傳、明季湖南殉節傳略。又易述、毛詩表、南村草堂詩文集,共數百卷。晚授寧鄉訓導。卒,年七十五。

同時萬希槐,字蔚亭,黃岡人。以廩膳生官南漳訓導。通經史百家言,著十三經證異。困學紀聞集證,陳嵩慶推為王氏功臣。

周濟,字保緒,荊溪人。好讀史,喜觀古將帥兵略,騎射擊刺藝絕精。嘉慶十年進士。或謂之曰:「對策語幸無過激。」濟曰:「始進,敢欺君乎!」及廷對,縱言天下事,字逾恆格。以三甲歸班選知縣,改就淮安府學教授。上丁釋奠,禮畢,知府王轂就殿門外升輿,濟趨前阻之,知府不懌去,濟遂引疾歸。是秋冒賑事發,自轂以下吏皆得罪,濟以先去免。淮南北鹽梟充斥,總督孫玉庭知濟能,以防撫事屬之。濟集營弁,勒以兵法,奸民皆斂跡。已而嘆曰:「鹽務不理其本,徒緝私,私不可勝緝也。」因謝去。濟與李兆洛、張琦、包世臣訂交。當是時,數吳中士有裨世用者,必首及世臣、濟兩人。

濟雖以才自喜,一日盡屏豪習,閉門撰述,成晉略八十卷,例精辭潔,於攻取防守地勢多發明論贊中,非徒考訂已也。晚復任淮安教授,遴秀童教以樂舞,禮成,觀者盈千。周天爵移督湖廣,邀濟偕行。道卒,年五十九。

陳鶴,字鶴齡,元和人。操行修潔,亦精史學。嘉慶元年進士,以主事分工部,出無車馬。與棲霞牟昌裕、陽山鄭士超有「工部三君子」之目。熟於明代事,輯明紀六十卷。未成,卒。後八卷其孫克家續成之。克家,道光末舉人。官中書。後參張國樑軍事,殉難,贈知府銜

徐松,字星伯,大興人。嘉慶十年進士,授編修。簡湖南學政,坐事戍伊犁。松留心文獻,既出關,置開方小冊,隨所至圖其山川曲折,成西域水道記,擬水經;復自為釋,以比道元之注。又以新疆入版圖數十年,視同畿甸,而未有專書,乃纂述成編,於建置、控扼、錢糧、兵籍,言之尤詳。將軍松筠奏進其書,賜名新疆事略,特旨赦還,禦制序付武英殿刊行。道光改元,起內閣中書,洊擢郎中,補御史,出知榆林府。未幾,卒。他所著有新斠注地理志集釋、漢書西域傳補注、唐兩京城坊考、唐登科記考、新疆賦共數十卷。

松喜延譽後進。其客有沈堯者,字子惇,烏程人。優貢生。性沉默,足不越關塞,好指畫絕域山川。初為何凌漢、陳用光所賞拔。入京師,館於松。松稱其地學之精。歙程恩澤嘗讀西游記,擬為文疏通其說。及見堯所撰西游記金山以東釋,嘆曰:「遐荒萬里在目前矣!」遂閣筆。堯客死,張穆裒其遺著,為落颿樓槁。

陳潮,字東之,泰興人。通經,工小篆,又擅周髀之學。嘗夜登高臺窺星象,不寐。游京師,亦卒於松寓。

李圖,字少伯,掖縣人。以拔貢生官直隸無極縣知縣,謝病歸。圖讀書十行俱下,天才卓越。工詩古文詞,力屏近世浮靡之習。嘗曰:「文非司馬子長,詩非蘇、李,不足為師法也。」徐松為濟南濼源書院山長,見圖詩,嘆曰:「三百年來無此作矣!」著有鴻桷齋詩文集。山左稱詩者,王士禎、趙執信以後,以圖為巨擘雲。

李兆洛,字申耆,陽湖人。嘉慶十年進士,選庶吉士。改令鳳臺,俗獷悍多盜,地接蒙城、阜陽,遠者至百八十里,官或終任不一至。兆洛親行縣,辨其里落繁耗、地畝廣袤饒瘠,次第經理之。焦岡湖,漢芍陂也,濱淮,易為災。乃增堤防,設溝閘,歲以屢豐。擇耆老勸民孝謹,優敘之。於僻遠設義學,為求良師。其捕盜,尤為人所喜稱。嘗騎率健勇出不意得其魁,因察而撫之。兆洛嘗曰:「鳳、潁、泗民氣可用,揀集五千人,方行天下有餘矣。然唯其豪能使之,官帥至千里外,必客兵勢勝足相鈐制乃可。」兆洛在縣七年,以父憂去,遂不出。主講江陰書院幾二十年,以實學課士,其治經學、音韻、訓詁,訂輿圖,考天官歷術及習古文辭者輩出。如江陰承培元、宋景昌、繆尚誥、六承如等,皆其選也。

兆洛短身碩腹,豹顱剛目,望之若不可近,而接人和易,未嘗疾言遽色。資恤故舊窮乏無不至。藏書逾五萬卷,皆手加丹鉛,尤嗜輿地學。其論文欲合駢散為一,病當世治古文者知宗唐、宋不知宗兩漢,因輯駢體文鈔。其序略云:「自秦迄隋,其體遞變,而文無異名。自唐以來,始有古文之目,而目六朝之文為駢體。為其學者,亦自以為與古文殊路。夫氣有厚薄,天為之也;學有純駁,人為之也;體格有遷變,人與天參焉者也;義理無殊途,天人合焉者也。得其厚薄純雜之故,則於其體格之變,可以知世焉;於其義理之無殊,可以知文焉。文之體至六代而其變盡,夫沿其流極而溯之以至乎其源,則其所出者一也。」卒,年七十一。其自著曰養一齋集。所輯有皇朝文典、大清一統輿地全圖、鳳臺縣志、地理韻編。

培元,字守丹。優貢生。著說文引經證例、籀雅、經滯揭豬木。

景昌,字冕之。縣學生。著星緯測量諸篇。

尚誥,字芷卿。舉人。著古韻譜、雙聲譜、經星考。

承如及族人嚴,皆貢生。兆洛訂輿地圖,六氏兩生所手繪也。

錢儀吉,字衎石,嘉興人,尚書陳群曾孫。父福胙,侍讀學士。儀吉生有五色文禽翔其室,故初名逵吉,後易焉。嘉慶十三年進士,選庶吉士。改戶部主事,累遷至工科給事中。皆能舉其職,因公罷歸。

儀吉治經,先求古訓,博考眾說,一折衷本文大義,不持漢、宋門戶。嘗著經典證文、說文雅厭。雅厭者,以十九篇之次,寫九百四部之文,而以經籍傳注推廣之。其讀史,補晉兵志、朔閏諸表,撰三國晉南北朝會要,體例視徐天麟有所出入,不限斷以本書。又仿宋杜大珪名臣琬琰碑傳集,得清臣工文儒等八百餘人,輯錄之為碑傳集。後卒於大梁書院,年六十八。

從弟泰吉,字警石。少孤,執喪盡哀禮。與儀吉以學行相磨,遠近盛稱「嘉興二石」。為詩文原本情性,讀其辭,知其於孝友最深也。以廩貢生得海寧州學訓導。居間務讀書,自經史百氏下逮唐、宋以來詩文集,靡不博校。以其學語諸生,諸生之賢且文者大附。嘗修學宮,以費所羨修海昌備志。既又得民間節孝行者千餘事為旌之,曰:「吾職也。」再三請,必得乃已。為訓導幾三十年,不以枝官自放曠。粵寇陷浙,往依曾國籓,卒於安慶。著曝書雜志、甘泉鄉人稿。儀吉子寶惠,泰吉子炳森,皆能世其學。

包世臣,字慎伯,涇縣人。少工詞章,有經濟大略,喜言兵。嘉慶十三年舉人,大挑以知縣發江西。一權新喻,被劾去。復隨明亮征川、楚,發奇謀不見用,遂歸,卜居金陵。世臣精悍有口辯,以布衣遨游公卿間。東南大吏,每遇兵、荒、河、漕、鹽諸鉅政。無不屈節諮詢,世臣亦慷慨言之。

初,海盜蔡牽犯上海,鎮道迎世臣閱沿海島嶼。見黃浦停泊商船千艘,遂建海運可救漕弊之議。游袁浦,值河事亟,箸策河四略。是時鹽法以兩淮為大,私梟充斥,議者爭言緝私。世臣擬多裁鹽官,惟留運司主錢糧,場大使督灶戶,不分畛域,仿現行鐵硝之例,聽商販領本地官印照,赴場繳課買鹽。州縣具詳,運司存核,則場官不能乾沒正課;而轉輸迅速,則鹽價必銳減;私鹽皆輸官課,課入必倍。以之津貼辦公,並增翰、詹、科、道廉俸,為計甚便。

其論西北水利曰:「今國家南漕四百萬石,中歲腴田二百萬畝所產也。有田四百萬畝,歲入與佃半之,遂當全漕。先減運十之一,糶其穀及運資置官屯,遞減至十年,則漕可罷,賦可寬。以其盈餘量加賦餉,而官可廉,兵可練。不然,漕東南以贍西北,浮收勒折,日增一日,竭民力,積眾怒。東南大患,終必在此。」

世臣能為大言。其論書法尤精,行草隸書,皆為世所珍貴。著有小倦游閣文集,別編為安吳四種。

齊彥槐,字梅麓,婺源人。嘉慶十三年召試舉人,明年成進士,選庶吉士。散館,授金匱令。毀淫祠,歲旱,勤賑務。擢蘇州府同知,陳海運策,巡撫召詰之,條舉以對,巡撫不能難,終以更張寢其事。後十餘年,改行海運,仍仿其法焉。嘗制渾天儀、中星儀,並各為之說,及龍尾、恆升二車,便民運水。又著北極星緯度分表、海運南漕叢議、梅麓詩文集。

姚椿,字春木,婁縣人。父令儀,四川布政使,又屢參戎幕。椿高才博學,幼隨父游歷諸行省,洞知閭閻疾苦,慨然欲效用於世。

以國子監生試京兆,日與洪亮吉、楊芳燦、張問陶輩文酒高會,才名大起。顧試輒不遇。既,又受學於姚鼐,退而發宋賢書讀之,屏棄夙習,壹意求道,泊如也。嘗得寶應硃澤澐遺著,嘆曰:「此真為程、硃之學者!」親詣其墓拜之,申私淑之禮。道光元年,舉孝廉方正,不就。主書院講席,以實學勵諸生。其論文必舉桐城所稱,曰:「好學深思,心知其意。」又曰:「文之用有四:曰明道,曰記事,曰考古有得,曰言詞深美。」其錄清代人文八十餘卷,一本此旨。著有通藝閣錄、晚學齋文錄。

顧廣譽,字維康,平湖人。優貢生,舉咸豐元年孝廉方正。寇亂,未廷試。廣譽慕其鄉張履祥、陸隴其之為人,刻意厲行。其治經一依程端禮讀書分年日程遺法。著學詩詳說,用力至勤。又憫晚近喪祭禮廢,恩紀衰薄,婚娶僭侈逾度,乃變通古禮,酌時俗之宜,成四禮榷疑八卷。姚椿推為一時宗匠。有悔過齋文稿。卒於上海龍門書院。

張鑒,字春冶,歸安人。巡撫阮元築詁經精舍西湖,鑒及同里楊鳳苞、施國祁肄業其中,皆知名。嘉慶初,副榜貢生。元剿海寇,賑兩浙水災,一資鑒贊畫。時方議海運,鑒力主之。以為河運雖安,費鉅;海運費省,得其人熟習海道,未嘗不安。乃著海運芻言,凡料淺占風之法,定盤望星之規,放洋泊舟之處,考之甚悉,侍郎英和亟稱其書。道光四年,河決高家堰,漕運阻。英和遂奏行海運,多採用鑒說。卒,年八十三。著十五經叢說、西夏紀事本末、眉山詩案廣證。

鳳苞,字傅九。元編經籍篡詁,鳳苞與分纂。熟明季事,嘗為南疆逸史跋十二篇,傳於時。晚館郡城陳氏,其書室為鄭元慶魚計亭,人以為元慶復生云。

國祁,字非熊。與鳳苞皆廩膳生。國祁病金史蕪雜,積二十餘年,成金史詳校。以其帙繁,乃列舉條目為金源劄記。又作元遺山集箋、金源雜事詩。國祁工詩文,善填詞。家貧,為人主計市肆中。有一樓,顏曰吉貝居,著書其中,毀於火,著述多燼。

黃易,字小松,錢塘人。父樹穀,以孝聞,工隸書,博通金石。易承先業,於吉金樂石,寢食依之,遂以名家。官山東運河同知,勤於職事。嘗得武班碑及武梁祠堂石室畫像於嘉祥,乃即其地起武氏祠堂,砌石祠內。又出家藏精拓雙鉤鋟木。凡四方好古之士得奇文古刻,皆就易是正,以是所蓄甲於一時。自乾、嘉以來,漢學盛行,群經古訓無可蒐輯,則旁及金石,嗜之成癖,亦一時風尚然也。

瞿中溶,字木夫,嘉定人。為錢大昕女夫。尤邃金石之學。官湖南布政司理問,搜奇訪僻於人跡罕至之境,所獲益多。著有孔廟從祀弟子辨證、漢魏蜀石經考異辨正、說文地名考異、古泉山館彞器圖錄、錢志補正集、古官印考證、古鏡圖錄、續漢金石文編,凡二十餘種。

張廷濟,字叔未,嘉興人。嘉慶三年,舉鄉試第一。應禮部試輒躓,遂歸隱,以圖書金石自娛。建清儀閣,藏庋古器,名被大江南北。

沈濤,字西雝。與廷濟同邑。嘉慶十五年舉人。咸豐初,署江西鹽法道。粵賊攻南昌,隨巡撫張芾城守。圍解,授興泉永道,未到官,卒。濤尚考訂之學,喜金石,著常山貞石志、說文古本考。

陸增祥,字星農,太倉人。道光三十年一甲一名進士,授修撰,至辰永沅靖道。踵王昶金石萃編成金石補正百二十卷,凡三千五百餘通。又著磚錄一卷。其訂正金石款識名物,何紹基服其精。

董祐誠,字方立,陽湖人。生五歲,曉九九數。稍長,善屬文。游陜西,成華山神廟賦,一時傳誦。其學於典章、禮儀、輿地、名物皆肆力探索,而尤精歷算,盡通諸家法。特善深沉之思,書之鉤棘難讀者,一覽輒通曉。復能出新意,闡曲隱,補罅漏。嘉慶二十三年舉人。越五年卒,年三十三。

祐誠讀諸史歷志,因著三統衍補。復取三統以次迄明大統、萬年、回回各術,擬撰五十三家歷術,屬稿未成,其兄基誠取已成五種附水經注圖說刊之。其所著算學,有割圜連比例術圖解、斜弧三邊求角補術、堆垛求積術若干種。

基誠,字子詵。進士。由刑部郎中出知開封府。工詞章,與祐誠文合刊曰多華館駢體文。

方履籛,字彥聞,大興人。與祐誠同年舉人,為令閩中。初試吏署永定,里豪胡鳳兆掘族人父棺,並殺其子,名捕不得。履籛至,為書諭之,鳳兆自首,遂論如法。調閩縣,會旱,禱兩烈日中,體豐碩,中暑卒。履籛亦以駢文著稱。尤嗜金石文字,所積幾萬種,有伊闕石刻錄、富蘅齋碑目、河內縣志、萬善花室集。

周儀暐,字伯恬,陽湖人。嘉慶初舉人,宣城訓導。擢知山陽縣,調鳳翔。能詩。有夫椒山館集。

其後又有吳頡鴻,字嘉之。道光中進士,官代州知州;莊縉度,字眉叔。進士,戶部主事;趙申嘉,字蕓酉;陸容,字蓉卿;徐廷華,字子楞;汪士進,字逸云;周儀顥,字叔程,舉人,即儀暐弟也。號「毗陵後七子」,其名位亞於前七子。

俞正燮,字理初,黟縣人。性彊記,經目不忘。年二十餘,北走兗州謁孫星衍。時星衍為伏生建立博士,復訪求左氏後裔。正燮因作邱明子孫姓氏論、左山考,星衍多據以折衷群議,由是名大起。道光元年舉人。明年,阮元主會試,士相謂曰:「理初入彀矣!」後竟落第。其經策淹博,為他考官所乙,元未之見也。房考王藻嘗引為恨。

正燮讀書,置巨冊數十,分題疏記,積歲月乃排比為文,斷以己意。藻為刻十五卷,名曰癸巳類稿,又有存稿十五卷,山西楊氏刻之。弟正禧,亦舉人。多義行,文學與正燮齊名。

趙紹祖,字琴士,涇縣人。年十二,受知學使硃筠,補諸生。筠授以說文,曰:「讀此日無過十字。讀注疏,亦無過十葉。必精造乃已。」紹祖熟於史事,嘗應布政使陶澍聘,修安徽省志,詳贍有法。道光初,年七十,舉孝廉方正。又十二年,卒。注有通鑒注商、新舊唐書互證、金石跋、安徽金石記、涇川金石記、金石文正續鈔。

汪文臺,字士南。與正燮同縣,相善。宗漢儒,以論語邢疏疏略,因取證古義,博採子史箋傳,依韓嬰詩傳例作論語外傳。見阮元十三經注疏校勘記,謂有益於後學,然成於眾手,時有駮文,別為表識,作校勘記識語,寄示阮元,元服其精博,禮聘之。又嘗纂輯七家後漢書、淮南子校勘記及脞稿,皆行於世。道光二十四年,卒,年四十九。

湯球,字伯玕,亦黟人。少耽經史,從正燮、文臺游,傳其考據之學。通歷算星緯,恥以藝名。嘗輯鄭康成逸書九種、劉熙孟子注、劉珍等東觀漢記、皇甫謐帝王世紀、譙周古史考、傅子、伏侯古今注。球讀史用力於晉書尤深,廣蒐載籍,補晉史之闕,成書數種。同治六年,舉孝廉方正。光緒七年,卒,年七十八。

潘德輿,字四農,山陽人。年五六歲,母病不食,亦不食。父咯血,刲臂肉和藥進,父察其色動,泣曰:「固知兒有是也!」既孤,大母猶在堂,孝敬彌至。居喪一遵禮制,柴瘠劚然。著喪禮正俗文、祭儀,為家法。撫寡妹嗣子,教養盡二十年。其他行多類此。嘗以挽回世運,莫切於文章,文章之根本在忠孝,源在經術。其說經,不袒漢、宋,力求古人微言大義。其論治術,謂天下大病不外三言:曰「吏」、曰「例」、曰「利」。世儒負匡濟大略,非雜縱橫,即陷功利,未有能破「利」字而成百年休養之治者。道光八年,舉江南鄉試第一。入都,座主侍郎鍾昌館德輿於家,語人曰:「四農乃吾師也。」大挑以知縣分安徽,未到官卒,年五十五。

初,阮元總督漕運,招之,謝不往。後硃桂楨、周天爵皆號為名臣,折節原納交,德輿遠引避之,以為義無所居也,天爵喟然有望塵之嘆。其所與游,若永豐郭儀霄、建寧張際亮、震澤張履、益陽湯鵬、歙徐寶善,皆一時之選。德輿詩文精深博奧,有養一齋集。

門人清河吳昆田,字雲圃。舉人,刑部員外郎。晚年家居,賊犯清河,團練防守,邑賴以安。著漱六軒集。

張維屏,字子樹,番禺人。工詩,計偕入都,翁方綱賞異之。與黃培芳、譚敬昭稱「粵東三子」。道光二年進士,改官知縣,署黃梅。江水潰堤,乘小舟勘災,水急舟沖溜,掛樹免。民為謠曰:「犯急湍,官救民,神救官。」調補廣濟,公費一資漕折,民苦之,勢不可革,引疾去。汪廷珍語人曰:「縣官不原收漕,世罕見也!」丁艱服闋,原就閒,援例改郡丞,權南康。建太白、東坡祠廬山,暇則集諸生談藝,以風雅寓規勸焉。未一載,復罷歸。築聽松園,頹然不與世事,癖愛松,又號松心子。見松形奇古,輒下拜。精書法,朝鮮、小呂宋得其書,咸寶愛之。卒,年八十。有松心草堂集、國朝詩人徵略。培芳,香山人。

敬昭,字子晉,陽春人。順德黎簡者,以詩名海內,敬昭賦鵬鶴篇投之,簡嘆為異才。嘉慶二十二年進士,官戶部主事。著聽雲樓集。

同時廣東以學行名者,又有高要彭泰來,字子大。生二十月,能即事誦古經,語無不切。嘉慶十八年拔貢生。絕意進取,學使李棠階高其品,屏騶從徒步就見,詢以挽回風俗之道。泰來為書數千言復之,棠階表其廬,下教高要令,歲時存問。自惠士奇禮下胡方後,此為再見焉。著端州金石略、昨夢齋、詩義堂各集。

梅曾亮,字伯言,上元人。少時工駢文。姚鼐主講鍾山書院,曾亮與邑人管同俱出其門,兩人交最篤,同肆力古文,鼐稱之不容口,名大起。間以規曾亮,曾亮自喜,不為動也。久之,讀周、秦、太史公書,乃頗寤,一變舊習。義法本桐城,稍參以異己者之長,選聲練色,務窮極筆勢。道光二年進士,用知縣,授例改戶部郎中。居京師二十餘年,與宗稷辰、硃琦、龍啟瑞、王拯、邵懿辰輩游處,曾國籓亦起而應之。京師治古文者,皆從梅氏問法。當是時,管同已前逝,曾亮最為大師;而國籓又從唐鑒、倭仁、吳廷棟講身心克治之學,其於文推挹姚氏尤至。於是士大夫多喜言文術政治,乾、嘉考據之風稍稍衰矣。未幾,曾亮依河督楊以增。卒,年七十一。以增為刊其詩文,曰柏見山房集。

同,字異之。少孤,母鄒以節孝聞。同善屬文,有經世之志,稱姚門高足弟子。嘗擬言風俗書、籌積貯書,為一時傳誦。道光五年,陳用光典試江南,同中式。用光語人曰:「吾校兩江士,獨以得一異之自憙耳。」用光亦鼐弟子也。同卒,年四十七,著因寄軒集。子嗣復,字小異。能世其業,兼通算術。

鼐門下著籍者眾,惟同傳法最早。其於同里,則亟稱劉開之才。

開,字明東。以孤童牧牛,聞塾師誦書,竊聽之,盡記其語。塾師留之學,而妻以女。年十四,以文謁鼐,有國士之譽,盡授以文法。游客公卿,才名動一時。年四十,卒。著孟塗集。子繼,字少塗。有信義。遍走貴勢求刻其父書,以此孟塗集益顯。

寶山毛岳生,字申甫。用難廕改文學生。孤貧,以孝聞。自力於學,未弱冠,賦白雁詩,得名。亦從鼐學古文,以鉤棘字句為工。有休復居集。

湯鵬,字海秋,益陽人。道光二年進士。初喜為詩,自上古歌謠至三百篇、漢、魏、六朝、唐,無不形規而神絜之,有詩三千首。既,官禮部主事,兼軍機章京。旋補戶部主事,轉員外郎,改御史。意氣蹈厲,其議論所許可,惟李德裕、張居正輩,徒為詞章士無當也。於是勇言事,未逾月,三上章。最後以言宗室尚書叱辱滿司官非國體,在已奉旨處分後,罷御史,回戶部,轉郎中。是時英吉利擾海疆,求通市。鵬已黜,不得言事,猶條上三十事於尚書轉奏,報聞。

鵬負才氣,鬱不得施,乃著之言,為浮邱子一書。立一意為幹,一幹而分數支,支之中又有支焉,支幹相演,以遞於無窮。大抵言軍國利病,吏治要最,人事情偽,開張形勢,尋躡要眇,一篇數千言者九十餘篇,最四十餘萬言。每遇人輒曰:「能過我一閱浮邱子乎?」其自喜如此。二十四年,卒。同時有張際亮者,亦以才氣磊落聞。

際亮,字亨甫,建寧人。少孤,伯兄業賈,以其才,資之讀書。補諸生,肄業福州鼇峰書院,院長陳壽祺器之。尋試拔貢,入京師,朝考報罷,而時皆嘖嘖稱其詩。鹺使曾燠以事至,召之飲。燠以名輩自處,縱意言論,同坐贊服,際亮心薄之。燠食瓜子粘須,一人起為拈去,際亮大笑,眾慚。既罷,復投書責燠不能教後進,徒以財利奔走寒士門下。燠怒,毀於諸貴人,由是得狂名,試輒不利。乃遍游天下山川,窮探奇勝,以其窮愁慷慨牢落古今之意,發為詩歌,益沉雄悲壯。十八年,鄉試者約:「張際亮狂士不可中。」而際亮已易名亨輔,中式。拆卷,疑欲去之,副考官申解而止。及來謁,果際亮也,主試者愕然。會試復報罷。際亮故與桐城姚瑩善。二十三年,聞瑩以守土事被誣下獄,入都急難。及事白而際亮疾篤,以所著思伯子堂詩集囑瑩,遂卒。其後瑩子濬昌輯而刊之,都三十二卷。

龔鞏祚,原名自珍,字璱人,仁和人。父麗正,進士,官蘇松兵備道,為段玉裁婿,能傳其學。鞏祚十二歲,玉裁授以說文部目。鞏祚才氣橫越,其舉動不依恆格,時近俶詭,而說經必原本字訓,由始教也。初由舉人援例為中書。道光時成進士,歸本班。洊擢宗人府主事,改禮部。謁告歸,遂不出。官中書時,上書總裁論西北塞外部落源流、山川形勢,訂一統志之疏漏,凡五千言。後復上書論禮部四司政體宜沿革者,亦三千言。其文字驁桀,出入諸子百家,自成學派。所至必驚眾,名聲藉藉,顧仕宦不達。年五十,卒於丹陽書院。著有尚書序大義、大誓答問、尚書馬氏家法、左氏春秋服杜補義、左氏決疣、春秋決事比、定菴詩文集。

魏源,字默深,邵陽人。道光二年,舉順天鄉試。宣宗閱其試卷,揮翰褒賞,名藉甚。會試落第,房考劉逢祿賦兩生行惜之。兩生者,謂源及龔鞏祚。兩人皆負才自喜,名亦相埒。源入貲為中書,至二十四年成進士。以知州發江蘇,權興化。二十八年,大水,河帥將啟閘。源力爭不能得,則親擊鼓制府,總督陸建瀛馳勘得免,士民德之。補高郵,坐遲誤驛遞免。副都御史袁甲三奏復其官。咸豐六年,卒。

源兀傲有大略,熟於朝章國故。論古今成敗利病,學術流別,馳騁往復,四座皆屈。嘗謂河宜改復北行故道,至咸豐五年,銅瓦廂決口,河果北流。又作籌篇上總督陶澍,謂:「自古有緝場私之法,無緝鄰私之法。鄰私惟有減價敵之而已。非裁費曷以輕本減價?非變法曷以裁費?」顧承平久,撓之者眾。迨漢口火災後,陸建瀛始力主行之。

源以我朝幅員廣,武功實邁前古,因借觀史館官書,參以士大夫私著,排比經緯,成聖武記四十餘萬言。晚遭夷變,謂籌夷事必知夷情,復據史志及林則徐所譯西夷四州志等,成海國圖志一百卷。他所著有書古微、詩古微、元史新編、古微堂詩文集。

方東樹,字植之,桐城人;宗誠,字存之,從兄弟也:皆諸生。東樹曾祖澤,拔貢生,為姚鼐師。東樹既承先業,更師事鼐。當乾、嘉時,漢學熾盛,鼐獨守宋賢說。至東樹排斥漢學益力。阮元督眾,闢學海堂,名流輻湊,東樹亦客其所,不茍同於眾。以謂:「近世尚考據,與宋賢為水火。而其人類皆鴻名博學,貫穿百氏,遂使數十年承學之士,耳目心思為之大障。」乃發憤著漢學商兌一書,正其違謬。又著書林揚觶,戒學者勿輕事著述。

東樹始好文事,專精治之,有獨到之識,中歲為義理學,晚躭禪悅,凡三變,皆有論撰。務盡言,惟恐詞不達。年八十,卒於祁門東山書院。他所著有大意尊聞、向果微言、昭昧詹言、儀衛軒集,凡數十卷。東樹博極群書,窮老不遇,傳其學宗誠。既歿,宗誠刊布其書,名乃大著。

宗誠能古文,熟於儒家性理之言,欲合文與道為一。咸豐時寇亂,轉徙不廢學,益留心兵事吏治。著俟命錄,以究天時人事致亂之原,大要歸於植綱常、明正學,志量恢如也。山東布政使吳廷棟見之,聘為子師。倭仁、曾國籓皆因廷棟以知宗誠。倭仁為師傅,寫其書數十則,進御經筵。國籓督直隸,奏以自隨。令棗強十餘年,設鄉塾,創敬義書院,刻邑先正遺著,舉孝子、悌弟、節婦,建義倉,積穀萬石,皆前此未有也。國籓去,李鴻章繼任,亦不以屬吏待之,有請輒施行。嘗歲旱,已逾報災期,手書為民請,並及鄰郡邑,不以侵官自嫌,卒得請普免焉。舉治行卓異,不赴部,自免歸。以學行詔後進,人有一善,獎譽之不容口。勤於纂述,逾時越月輒成帙。著柏堂經說、筆記、文集百五十餘卷。詔加五品卿銜,從安徽學政請也。其同縣友人又有蘇惇元,字厚子;戴鈞衡,字存莊:皆東樹弟子。

惇元,咸豐元年孝廉方正。其學近張楊園,文似方望溪。編有楊園、望溪年譜。所著曰四禮從宜、遜敏錄、詩文集。

鈞衡,道光二十九年舉人。自謂生方、姚之鄉,不敢不以古文自任。與惇元重訂望溪集,增集外文十之四。其後榮成孫葆田更得遺稿若干篇刻之,方氏一家之言備矣。鈞衡有經濟才,與國籓為友,著書傳補商,國籓亟稱之。避寇臨淮,妻李、妾劉皆殉難,鈞衡嘔血卒,年未四十。有蓉州集、味經山館詩文鈔。

魯一同,字通甫,清河人。善屬文,師事潘德輿。道光十五年舉人。時承平久,一同獨深憂,謂:「今天下多不激之氣,積而不化之習;在位者貪不去之身,陳說者務不駭之論。風烈不紀,一旦有緩急,莫可倚仗。」既,再試不第,益研精於學。凡田賦、兵戎諸大政,及河道遷變、地形險要,悉得其機牙。為文務切世情,古茂峻厲,有杜牧、尹洙之風。漕督周天爵見之,曰:「天下大材也,豈直文字哉!」曾國籓尤嘆異之。

試禮部,入都,國籓數屏騶從就問天下事。粵逆踞金陵也,同年生吳棠方宰清河,一同為草檄,傳示列縣,辭氣奮發,江北人心大定。江忠源師抵廬州,友人戴鈞衡為書通國籓之指,欲其起佐忠源。一同謝不出,復書極論用兵機宜,謂當緩金陵,專攻旁郡。其後大兵築長圍,期旦夕破金陵,一同獨決其必敗,未幾,果潰裂,蘇、浙淪陷。已而國籓克安慶,復金陵,一如所論。同治二年,卒,年五十九。著邳州志、清河志、通甫類稿。

子蕡,字仲實。諸生,文有家法。善綜核,知府章儀林議減清河賦,苦繁重,叩蕡。蕡為剖析條目,退草三千言,明旦獻之。儀林驚喜,因請主辦,三年而成。又佐修安東水道,役竣,費無毫發溢。

譚瑩,字玉生,南海人。弱冠應縣試,總督阮元游山寺,見瑩題壁詩,驚賞,告縣令曰:「邑有才人,勿失之!」令問姓名,不答。已而得所為賦以告元,元曰:「是矣。」逾年,元開學海堂課士,以瑩及侯康、儀克中、熊景星、黃子高為學長。瑩性強記,述往事,雖久遠,時日不失。博考粵中文獻,友人伍崇曜富於貲,為匯刻之,曰嶺南遺書五十九種,曰粵十三家集,曰楚南耆舊遺詩,益擴之為粵雅堂叢書。瑩為學長三十年,英彥多出其門。道光二十四年,舉於鄉,官化州訓導。久之,遷瓊州教授,加中書銜。少與侯康等交莫逆,晚歲陳澧與之齊名。著樂志堂集。

景星,字伯晴,亦南海人也。以詩見賞於元。顧其意恨文士綿弱,學騎射技擊。以舉人終學官,無所試,一假書畫自娛。

子高,字叔立,番禺人。優貢生。精小篆,喜考證金石。藏書多異本。

瑩子宗浚,字叔裕。工駢文。同治十三年一甲二名進士,授編修。初舉於鄉,齒尚少。瑩課令讀書十年,乃許出仕。授以馬氏通考,略能記誦。既,入翰林,督學四川,又充江南副考官。以伉直為掌院所惡,出為雲南糧儲道。宗浚不樂外任,辭,不允。再權按察使,引疾歸,鬱鬱道卒。

吳敏樹,字本深,巴陵人。父達德,歲歉,貸貧民穀逾萬石,不償,有名湖、湘間。敏樹生而好學,為文章力求岸異,刮去世俗之見。道光十二年,舉於鄉。時梅曾亮倡古文義法京師,傳其師姚氏學說。敏樹起湖湘,不與當世士接手,錄明昆山歸氏文成冊。既,入都,與曾亮語合。於是京師盛傳敏樹能古文。曾國籓官京師,與敏樹交最篤,既出治軍,欲使參幕事,辭不赴。

敏樹貌溫而氣夷,意趣超曠,視人世忻戚得喪無累於其心。以大挑選瀏陽訓導,旋自免去。時登君山江樓,徜徉吟嘯。學者稱南屏先生。著柈湖文錄。卒,年六十九。

敏樹之友以文名者,曰楊彞珍,字性農,武陵人。父丕復,舉人,官石門訓導,著歷代輿地沿革。彞珍,道光末進士,選庶吉士,改兵部主事。與曾國籓、左宗棠往還,好奔走聲氣。重宴鹿鳴,賞四品卿。年九十餘,卒。有移芝室集。

周壽昌,字應甫,長沙人。道光二十五年進士,選庶吉士,授編修。咸豐初,洊擢至侍讀。時粵寇犯湖南,督師賽尚阿逗遛不戰,上疏劾之,一時推為敢言。迨寇踞金陵,分黨北犯,命隨辦京畿防務。鄉民十七人闌入城,當事者偵獲,以賊諜論,壽昌廉得實,趣令釋之;或疑失要人旨,且得罪,壽昌曰:「我豈以人命阿權貴哉?」卒釋之。穆宗親政,疏請躬行典禮,戒逸豫,報聞。

壽昌精核強記,雖宦達,勤學過諸生。篤嗜班固書,塗染無隙紙,成漢書注校補五十卷,易槁十有七。又有後漢注補正、三國志注證遺、思益堂集。官終內閣學士。

李希聖,字亦園,湘鄉人。以進士官刑部主事。嗜學,初治訓詁,通周官、春秋、穀梁,史習新舊唐書,文法騷、選,詩多淒艷,似玉谿。好讀書,通古今治法,慨然有經世之志。嘗纂光緒會計錄以總綜財賦。又草律例損益議,張百熙等皆極重之。光緒末,卒。

斌良,字笠畊,號梅舫,瓜爾佳氏,滿洲正紅旗人,閩浙總督玉德子。由廕生歷官刑部侍郎,為駐藏大臣。善為詩,以一官為一集,得八千首。其弟法良匯刊為抱沖齋全集,稱其早年詩,風華典贍,雅近竹垞、樊榭。迨服官農部,從軍滅滑,詩格堅老。古體胎息漢、魏、韓、杜、蘇、李,律詩則純法盛唐。秉臬陜、豫,奉召還都,時與陳荔峰、李春湖、葉筠潭、吳蘭雪唱酬,詩境益高。奉使蒙籓,跋馬古塞,索隱探奇,多詩人未歷之境,風格又一變,以薩天錫、元遺山自況。阮元為序,亦頗稱之。

法良,字可盦。梅曾亮稱其詩學東坡,得清曠之氣,而運以唐賢優游平夷之情。有漚羅盦詩集。

錫縝,原名錫淳,字厚安,博爾濟吉特氏,滿洲正藍旗人。咸豐六年進士。由戶部郎中授江西督糧道,為駐藏大臣,乞病歸。工書,善詩文。著有退復軒詩文集。

李云麟,字雨蒼,漢軍正白旗人。以諸生從曾國籓督師剿粵匪,累功至副都統。時新疆設布倫托海辦事大臣,以云麟任之。署伊犁將軍。治邊皆著績,為言官劾罷。云麟性剛使氣,少好游,遍歷五岳,歸著曠游偶筆一卷。紀游詩有奇氣。初謁國籓,適遇其子不為禮,云麟怒批之。國籓延入謝過,使獨領一軍。左宗棠奏調,亦稱其有將才。云麟時被酒狂言,與世多忤。罷歸後,卒貧困死。有詩集,西陲紀行。

道、咸以來,滿洲如觀成,字葦杭,瓜爾佳氏。有瓜亭雜錄、語花館詩集。鄂恆,字松亭,伊爾根覺羅氏。有求是山房集。震鈞,字在廷,改名唐宴,瓜爾佳氏。有渤海國志、天咫偶聞。英華,字斂之,赫佳氏,正紅旗人。博學善詩文,工書法。著書立說,中外知名。有安蹇齋集、萬松野人言善錄等。蒙古盛元,字愷廷,巴魯特氏。有南昌府志、杭營小志、怡園詩草。漢軍宗山,字歗梧,魯氏。有窺生鐵齋詩集、希晦堂遺文。皆以詩文名。

何紹基,字子貞,道州人,尚書凌漢子。道光十六年進士,選庶吉士,授編修。紹基承家學,少有名。阮元、程恩澤頗器賞之。歷典福建、貴州、廣東鄉試,均稱得人。咸豐二年,簡四川學政。召對,詢家世學業,兼及時務。紹基感激,思立言報知遇,時直陳地方情形,終以條陳時務降歸。歷主山東濼源、長沙城南書院,教授生徒,勖以實學。同治十三年,卒,年七十又五。

紹基通經史,精律算。嘗據大戴記考證禮經,貫通制度,頗精切。又為水經注刊誤。於說文考訂尤深。詩類黃庭堅。嗜金石,精書法。初學顏真卿,遍臨漢、魏各碑至百十過。運肘斂指,心摹手追,遂自成一家,世皆重之。所著有東洲詩文集四十卷。

弟紹京,字子愚。亦工書,筆法頗似其兄。

孫維樸,字詩孫。以副貢為中書,累至道員。工書畫,字摹其祖。久寓滬,國變後,卒,年八十餘。

與維樸同時以書名海上者李瑞清,字梅盦,臨川人。光緒二十年進士,選庶吉士。改道員,分江蘇,攝江寧提學使,兼兩江師範學堂監督。宣統三年,武昌亂起,江寧新軍亦變,合浙軍攻城。官吏潛遁,瑞清獨留不去,仍日率諸生上課如常。布政使樊增祥棄職走,以瑞清代之。急購米三十萬斛餉官軍,助城守,設平糶局,賑難民。城陷,瑞清衣冠坐堂皇,矢死不少屈。民軍不忍加害,縱之行。乃封籓庫,以鑰與籍囑之士紳,積金尚數十萬也。自是為道士裝,隱滬上,匿姓名,自署曰清道人,鬻書畫以自活。瑞清詩宗漢、魏,下涉陶、謝。書各體皆備,尤好篆隸。嘗謂作篆必目無二李,神游三代乃佳。丁巳復闢,授學部侍郎。又三年卒,謚文潔。

馮桂芬,字林一,號景亭,吳縣人。道光二十年一甲二名進士,授編修,充廣西鄉試正考官,丁母憂。服闋,文宗御極,用大臣薦召見。旋丁父憂,服甫闋而金陵陷。詔募貲團練於鄉,以克復松江府諸城功晉五品銜,擢右中允。赴京,期年告歸。同治元年,以治團功加四品銜。亂定,復以耆宿著書裨治加三品銜。

桂芬少工駢體文,中年後乃肆力古文辭。於書無所不窺,尤留意天文、地輿、兵刑、鹽鐵、河漕諸政。初佐某邑令治錢穀,以事不合拂衣去,入兩江總督陶澍幕。自未仕時已名重大江南北。及粵賊陷蘇州,避居上海。時大學士曾國籓治軍皖疆。蘇州士大夫推錢鼎銘持書乞援,陳滬城危狀,及用兵機宜,累數千言,其稿,桂芬所手創也。國籓讀之感動,乃遣李鴻章率師東下。既解滬上圍,進克蘇州,皆闢以為助。桂芬立會防局,調和中外雜處者。設廣方言館,求博通西學之才,儲以濟變。嘗從容為鴻章言吳人糧重之苦,往往因催科破家。會松江知府方傳書亦上書,謂:「江蘇自南宋籍沒諸王大臣田,官徵其租,延及元代,官田民田淆亂,租額浸淫入賦額,民既苦之;其後張士誠又盡攘諸豪田為官產,明太祖平吳,怒吳人附士誠,依田租私籍數定稅,乃重困。雍正、乾隆間,嘗再議減,然但及地丁。今儻乘民亂後覈減浮糧,疲民大悅,賊勢且益衰。」鴻章以聞。有詔減蘇、松、太米賦三之一,常、鎮十一,著為令。

桂芬性恬澹,服官僅十年,然家居遇事奮發,不避勞怨。凡濬河、建學、積穀諸舉,條議皆出其手。先後主講金陵、上海、蘇州諸書院,與後進論學,昕夕忘倦。精研書數,嘗以意造定向尺及反羅經,以步田繪圖。又以江南清丈用部頒五尺步弓,田多溢額,乃考會典定用舊行六尺步弓量舊田,新頒者量新漲沙田。著說文解字段注考證、弧矢算術細草圖解、西算新法直解、校邠廬抗議、顯志堂詩文集,都數十卷。同治十三年,卒。

王頌蔚,字芾卿,長洲人。光緒五年進士,選庶吉士。吳縣潘祖廕、常熟翁同龢皆稱頌蔚才。散館,改官戶部,補軍機章京。暇輒從事著述。嘗於方略館故紙堆中見殿板初印明史殘本,眉上黏有黃簽,審為乾隆朝擬撰考證未竟之本。因多方搜求,逐條釐訂,芟其繁冗,採其精要,成明史考證攟逸四十餘卷。光緒十八年,試御史第一,軍機處奏留。頌蔚思立言抒忠讜,轉鬱鬱不樂。嘗派充工程監督差,例有分饋,頌蔚獨卻之,曰:「我輩取與之間,貴自審慎,不可隨俗浮沉。昔陳稽亭先生官部曹時,印結公項,且猶不取。矧此實為廠商之賄賂乎?」

二十一年,中日釁起,戰事多北洋大臣主之。會翁同龢復入軍機,乃進言曰:「讀聖祖、高宗聖訓,凡事關軍務者,皆由中朝謀定後動。今日戰局既成,非直隸一省事,豈能悉諉之北洋乎?」及議和,頌蔚益為悲憤,嘗曰:「今之敗績,徒歸咎於師之不練、器之不利,猶非探本之論。頻年以來,盈廷習洩沓之風,宮中務游觀之樂,直臣擯棄,賄賂公行,安有戰勝之望?此後償金既巨,民力益疲,恐大亂之不在外患而在內憂矣。」明年,卒。著有寫禮廎文集、詩集、讀碑記、古書經眼錄各一卷,明史考證攟逸四十二卷。

葉昌熾,字鞠裳,元和人。光緒十六年進士,選庶吉士,授編修。累至侍講,督甘肅學政,邊地樸陋,昌熾校閱盡職。以裁缺歸,著書終老。國變後五年,卒。著有藏書紀事詩六卷,語石十卷,邠州大佛寺題刻考二卷,均考訂精確。

管禮耕,字申季。歲貢生。父慶祺,從陳奐游。禮耕篤守家學,尤長訓詁。嘗言唐以正義立學官,漢、魏、六朝遺說,積久泰半闕不完。凡所考見,獨存釋文,而今本踳駮非其舊,思綜稽群籍為校證,未及半而卒。

袁寶璜,字朅禹,元和人。光緒二十一年進士,官刑部主事。通經、小學,兼及算術。著書亦未成而卒。

李慈銘,字愛伯,會稽人。諸生,入貲為戶部郎中。至都,即以詩文名於時。大學士周祖培、尚書潘祖廕引為上客。光緒六年,成進士,歸本班,改御史。時朝政日非,慈銘遇事建言,請臨雍,請整頓臺綱。大臣則糾孫毓汶、孫楫,疆臣則糾德馨、沈秉成、裕寬,數上疏,均不報。慈銘鬱鬱而卒,年六十六。

慈銘為文沉博絕麗,詩尤工,自成一家。性狷介,又口多雌黃。服其學者好之,憎其口者惡之。日有課記,每讀一書,必求其所蓄之深淺,致力之先後,而評騭之,務得其當,後進翕然大服。著有越縵堂文十卷,白華絳趺閣詩十卷、詞二卷,又日記數十冊。弟子著錄數百人,同邑陶方琦為最。

方琦,字子珍。光緒二年進士,選庶吉士,授編修。督學湖南。年四十,卒於京邸。方琦學有本末,汲汲於古,述造無間歲時。治易鄭注,詩魯故,爾雅漢注,又習大戴禮記。其治淮南王書,力以推究經訓,蒐採許注,拾補高誘。再三屬草,矻矻十年,實事求是。有淮南許注異同詁、許君年表、漢孳室文鈔、駢文、詩詞。

譚廷獻,字仲修,仁和人。同治六年舉人。少負志節,通知時事。國家政制典禮,能講求其義。治經必求西漢諸儒微言大義,不屑屑章句。讀書日有程課,凡所論箸,隱栝於所為日記。文導源漢、魏,詩優柔善入,惻然動人。又工詞,與慈銘友善,相唱和。官安徽,知歙、全椒、合肥、宿松諸縣。晚告歸,貧甚。張之洞延主經心書院,年餘謝歸,卒於家。

李稷勛,字姚琴,秀山人。光緒二十四年二甲一名進士,改庶吉士,授編修。充會試同考官,精衡鑒,重實學,頗得知名士。累官郵傳部參議,總川漢路事。博學善古文,嘗受詩法於王闓運,而不囿師說。專步趨唐賢,意致深婉,得風人之遺。慈銘嘗稱賞之。有甓盦詩錄四卷。

張裕釗,字廉卿,武昌人。少時,塾師授以制舉業,意不樂。家獨有南豐集,時時竊讀之。咸豐元年舉人,考授內閣中書。曾國籓閱卷賞其文,既,來見,曰:「子豈嘗習子固文耶?」裕釗私自喜。已而國籓益告以文事利病及唐、宋以來家法,學乃大進,寤前此所為猶凡近,馬遷、班固、相如、揚雄之書,無一日不誦習。又精八法,由魏、晉、六朝以上窺漢隸,臨池之勤,亦未嘗一日輟。國籓既成大功,出其門者多通顯。裕釗相從數十年,獨以治文為事。國籓為文,義法取桐城,益閎以漢賦之氣體,尤善裕釗之文。嘗言「吾門人可期有成者,惟張、吳兩生」,謂裕釗及吳汝綸也。

裕釗文字淵懿,歷主江寧、湖北、直隸、陜西各書院,成就後學甚眾。嘗言:「文以意為主,而辭欲能副其意,氣欲能舉其辭。譬之車然,意為之御,辭為之載,而氣則所以行也。欲學古人之文,其始在因聲以求氣,得其氣,則意與辭往往因之而益顯,而法不外是矣。」世以為知言。著濂亭文集。

裕釗門下最知名者,有範當世、硃銘盤。當世,字肯堂,江蘇通州諸生。能詩,汝綸嘗嘆其奇橫不可敵。著範伯子詩文集。銘盤,字曼君,泰興舉人。敘知州。其學長於史,兼工詩古文。著晉會要一百卷,朝鮮長編四十卷,及桂之華軒詩文集。

與裕釗同時者,有楊守敬,字惺吾,宜都人。為文不足躋裕釗,而其學通博。精輿地,用力於水經尤勤。通訓詁,考證金石文字。能書,摹鐘鼎至精。工儷體,為箴銘之屬,古奧聳拔,文如其人。以舉人官黃岡教諭,加中書銜。嘗游日本,搜古籍,多得唐、宋善本,辛苦積貲,藏書數十萬卷,為鄂學靈光者垂二十年。卒,年七十有七。著有水經注圖、水經注要刪、隋書地理志考證、日本訪書志、晦明軒稿、鄰蘇老人題跋、望堂金石集等。

吳汝綸,字摯父,桐城人。少貧力學,嘗得雞卵一,易松脂以照讀。好文出天性,早著文名。同治四年進士,用內閣中書。曾國籓奇其文,留佐幕府,久乃益奇之,嘗以漢禰衡相儗。旋調直隸,參李鴻章幕。時中外大政常決於國籓、鴻章二人,其奏疏多出汝綸手。

尋出補深州,丁外內艱。服除,補冀州。其治以教育為先,不憚貴勢,籍深州諸村已廢學田為豪民侵奪者千四百餘畝入書院,資膏火。聚一州三縣高材生親教課之,民忘其吏,推為大師。會以憂去,豪民至交通御史以壞村學劾奏,還其田。及蒞冀州,仍銳意興學,深、冀二州文教斐然冠畿輔。又開冀、衡六十里之渠,洩積水於滏,以溉田畝,便商旅。時時求其士之賢有文者禮先之,得十許人。月一會書院,議所施為興革於民便不便,率不依常格。稱疾乞休。

鴻章素重其人,延主蓮池講席。其為教,一主乎文,以為:「文者,天地之至精至粹,吾國所獨優。語其實用,則歐、美新學尚焉。博物格致機械之用,必取資於彼,得其長乃能共競。舊法完且好,吾猶將革新之,況其窳敗不可復用。」其勤勤導誘後生,常以是為說。嘗樂與西士游,而日本之慕文章者,亦踔海來請業。會朝旨開大學堂於京師,管學大臣張百熙奏薦汝綸加五品卿銜總教務,辭不獲,則請赴日本考學制。既至其國,上自君、相及教育名家,婦孺學子,皆備禮接款,求請題詠,更番踵至。旋返國,先乞假省墓,興辦本邑小學堂。規制粗立,遽以疾卒,年六十四。

汝綸為學,由訓詁以通文辭,無古今,無中外,唯是之求。自群經子史、周、秦故籍,以下逮近世方、姚諸文集,無不博求慎取,窮其原而竟其委。於經,則易、書、詩、禮、左氏、穀梁、四子書,旁及小學音韻,各有詮釋。於史,則史記、漢書、三國志、新五代史、資治通鑒、國語、國策皆有點校,尤邃於史記,盡發太史公立言微旨。於子,則老、莊、荀、韓、管、墨、呂覽、淮南、法言、太玄各有評騭,而最取其精者。於集,則楚辭、文選,漢魏以來各大家詩文皆有點勘之本。凡所啟發,皆能得其深微,整齊百代,別白高下,而一以貫之。盡取古人不傳之蘊,昭然揭示,俾學者易於研求;且以識夫作文之軌範,雖萬變不窮,而千載如出一轍。

其論文,嘗謂:「千秋蓋世之勛業皆尋常耳,獨文章之事,緯地經天,代不數人,人不數篇,唯此為難。」又謂:「中國之文,非徒習其字形而已,綴字為文,而氣行乎其間,寄聲音神採於文外。雖古之聖賢豪傑去吾世邈矣,一涉其書,而其人之精神意氣若儼立乎吾目中。」務欲因聲求氣,凡所為抗墜、詘折、斷續、斂侈、緩急、長短、伸縮、抑揚、頓挫之節,一循乎機勢之自然,以漸於精微奧之域。乃有以化裁而致於用,悉舉學問與事業合而為一;而尤以瀹民智自強亟時病為兢兢云。著有易說二卷、寫定尚書一卷、尚書故三卷、夏小正私箋一卷、文集四卷、詩集一卷、深州風土記二十二卷,及點勘諸書,皆行於世。

汝綸門下最著者為賀濤,而同時有蕭穆,亦以通考據名。

穆,字敬孚。縣學生。其學博綜群籍,喜談掌故,於顧炎武、全祖望諸家之書尤熟。復多見舊槧,考其異同,硃墨雜下。遇孤本多方勸刻,所校印凡百餘種。有敬孚類槁十六卷。

濤,字松坡,武強人。光緒十二年進士,官刑部主事。以目疾去官。初,汝綸牧深州,見濤所為反離騷,大奇之,遂盡授以所學,復使受學於張裕釗。濤謹守兩家師說,於姚鼐義理、考據、詞章三者不可偏廢之說,尤必以詞章為貫澈始終,日與學者討論義法不厭。與同年生劉孚京俱治古文,濤言宜先以八家立門戶,而上窺秦、漢;孚京言宜先以秦、漢為根柢,而下攬八家,其門徑大略相同。濤有文集四卷。

孚京,字鎬仲,南昌人。有文集六卷。

林紓,字琴南,號畏廬,閩縣人。光緒八年舉人。少孤,事母至孝。幼嗜讀,家貧,不能藏書。嘗得史、漢殘本,窮日夕讀之,因悟文法,後遂以文名。壯渡海游臺灣,歸客杭州,主東城講舍。入京,就五城學堂聘,復主國學。禮部侍郎郭曾炘以經濟特科薦,辭不應。

生平任俠尚氣節,嫉惡嚴。見聞有不平,輒憤起,忠懇之誠發於至性。念德宗以英主被扼,每述及,常不勝哀痛。十謁崇陵,匍伏流涕。逢歲祭,雖風雪勿為阻。嘗蒙賜御書「貞不絕俗」額,感幸無極,誓死必表於墓,曰「清處士」。憂時傷事,一發之於詩文。

為文宗韓、柳。少時務博覽,中年後案頭唯有詩、禮二疏,左、史、南華及韓、歐之文,此外則說文、廣雅,無他書矣。其由博反約也如此。

其論文主意境、識度、氣勢、神韻,而忌率襲庸怪,文必己出。嘗曰:「古文唯其理之獲,與道無悖者,則味之彌臻於無窮。若分畫秦、漢、唐、宋,加以統系派別,為此為彼,使讀者炫惑莫知所從,則已格其途而左其趣。經生之文樸,往往流入於枯淡,史家之文則又隳突恣肆,無復規檢,二者均不足以明道。唯積理養氣,偶成一篇,類若不得已者,必意在言先,修其辭而峻其防,外質而中膏,聲希而趣永,則庶乎其近矣。」紓所作務抑遏掩蔽,能伏其光氣,而其真終不可自閟。尤善敘悲,音吐淒梗,令人不忍卒讀。論者謂以血性為文章,不關學問也。

所傳譯歐西說部至百數十種。然紓故不習歐文,皆待人口達而筆述之。任氣好辯,自新文學興,有倡非孝之說者,奮筆與爭,雖脅以威,累歲不為屈。尤善畫,山水渾厚,冶南北於一爐,時皆寶之。紓講學不分門戶,嘗謂清代學術之盛,超越今古,義理、考據,合而為一,而精博過之。實於漢學、宋學以外別創清學一派。時有請立清學會者,紓撫掌稱善,力贊其成。甲子秋,卒,年七十有三,門人私謚貞文先生。有畏廬文集、詩集、論文、論畫等。

嚴復,初名宗光,字又陵,一字幾道,侯官人。早慧,嗜為文。閩督沈葆楨初創船政,招試英俊,儲海軍將才,得復文,奇之,用冠其曹,則年十四也。既卒業,從軍艦練習,周歷南洋、黃海。日本窺臺灣,葆楨奉命籌防,挈之東渡詗敵,勘測各海口。光緒二年,派赴英國海軍學校肄戰術及砲臺建築諸學,每試輒最。侍郎郭嵩燾使英,賞其才,時引與論析中西學術同異。學成歸,北洋大臣李鴻章方大治海軍,以復總學堂。二十四年,詔求人才,復被薦,召對稱旨。諭繕所擬萬言書以進,未及用,而政局猝變。越二年,避拳亂南歸。

是時人士漸傾向西人學說,復以為自由、平等、權利諸說,由之未嘗無利,脫靡所折衷,則流蕩放佚,害且不可勝言,常於廣眾中陳之。復久以海軍積勞敘副將,盡棄去,入貲為同知,累保道員。宣統元年,海軍部立,特授協都統,尋賜文科進士,充學部名詞館總纂。以碩學通儒徵為資政院議員。三年,授海軍一等參謀官。復殫心著述,於學無所不窺,舉中外治術學理,靡不究極原委,抉其失得,證明而會通之。精歐西文字,所譯書以朅辭達奧旨。

其天演論自序有曰:「仲尼之於六藝也,易、春秋最嚴。司馬遷曰:『易本隱而之顯,春秋推見至隱。』此天下至精之言也。始吾以為本隱之顯者,觀象系辭,以定吉兇而已;推見至隱者,誅意褒貶而已。及觀西人名學,則見其格物致知之事,有內籀之術焉,有外籀之術焉。內籀云者,察其曲而知其全者也,執其微以會其通者也。外籀云者,援公理以斷眾事者也,設定數以逆未然者也。是固吾易、春秋之學也。遷所謂『本隱之顯』者外籀也,所謂『推見至隱』者內籀也,二者即物窮理之要術也。夫西學之最為切實,而執其例可以禦蕃變者,名、數、質、力四者之學而已。而吾易則名、數以為經,質、力以為律,而合而名之曰『易』。大宇之內,質、力相推,非質無以見力,非力無以呈質。凡力皆乾也,凡質皆坤也。奈端動之例三,其一曰:『靜者不自動,動者不自止,動路必直,速率必均。』而易則曰:『乾,其靜也專,其動也直。』有斯賓塞爾者,以天演自然言化,其為天演界說曰:『翕以合質,闢以出

力,始簡易而終雜糅。』而易則曰:『坤,其靜也翕,其動也闢。』至於全力不增減之說,則有自強不息為之先;凡動必復之說,則有消息之義居其始。而『易不可見,乾坤或幾乎息』之旨,尤與熱力平均、天地乃毀之言相發明也。大抵古書難讀,中國為尤。二千年來,士徇利祿,守闕殘,無獨闢之慮,是以生今日者,乃轉於西學得識古之用焉。」凡復所譯著,獨得精微皆類此。

世謂紓以中文溝通西文,復以西文溝通中文,並稱「林嚴」。辛酉秋,卒,年六十有九。著有文集及譯天演論、原富、群學肄言、穆勒名學、法意、群己權界論、社會通詮等。

同時有辜湯生,字鴻銘,同安人。幼學於英國,為博士。遍游德、法、意、奧諸邦,通其政藝。年三十始返而求中國學術,窮四子、五經之奧,兼涉群籍。爽然曰:「道在是矣!」乃譯四子書,述春秋大義及禮制諸書。西人見之,始嘆中國學理之精,爭起傳譯。庚子拳亂,聯軍北犯,湯生以英文草尊王篇,申大義。列強知中華以禮教立國,終不可侮,和議乃就。張之洞、周馥皆奇其才,歷委辦議約、濬浦等事。旋為外務部員外郎,晉郎中,擢左丞。

湯生論學以正誼明道為歸,嘗謂:「歐、美主強權,務其外者也;中國主禮教,修其內者也。」又謂:「近人欲以歐、美政學變中國,是亂中國也。異日世界之爭必烈,微中國禮教不能弭此禍也。」湯生好辯,善罵世。國變後,悲憤尤甚。窮無所之,日人聘講東方文化,留東數年,歸。卒,年七十有二。


\end{pinyinscope}