\article{本紀十四}

\begin{pinyinscope}
高宗本紀五

四十一年春正月癸酉朔,富德克打噶咱普德爾窩、馬爾邦等碉卡。明亮等克獨松等碉卡。甲戌,定郡王綿德以交結禮部司員削爵,命綿恩承襲。阿桂克喇烏喇等碉卡及舍齊等寺。己卯,阿桂率諸軍進圍噶喇依,索諾木之母及其姑姊妹出降。命封阿桂一等誠謀英勇公,予四團龍補服、金黃帶。加賞果毅繼勇公豐升額一等子。封明亮一等襄勇伯,海蘭察一等超勇侯,額森特一等男,和隆武三等果勇侯,福康安、普爾普三等男。加賞奎林一等男。豐升額、明亮、海蘭察、奎林、和隆武仍各予雙眼花翎,賞於敏中一等輕車都尉,均世襲。阿桂請安插降眾於綽斯甲布十二土司地方,從之。壬午,賞阿桂紫韁。甲申,調明善為科布多參贊大臣。以法福里為烏里雅蘇臺參贊大臣。己丑,吏部尚書、協辦大學士官保以病乞休,允之。以阿桂為吏部尚書、協辦大學士。調豐升額為戶部尚書,福隆安為兵部尚書。以綽克托為工部尚書。庚寅,嘉謨遷倉場侍郎。命阿思哈署漕運總督,永貴署吏部尚書,英廉署戶部尚書。

二月己酉,授文綬四川總督,調富勒渾為湖廣總督。庚戌,命嗣後社稷壇祭時,或值風雨,於殿內致祭。蠲江蘇上元等三十九州縣、鎮江等五衛四十年旱災額賦。辛亥,上謁東陵。以祇謁兩陵,並巡幸山東,免經過州縣本年額賦十分之三。甲寅,上謁昭西陵、孝陵、孝東陵、景陵,詣孝賢皇后陵奠酒。阿桂等奏索諾木等出降,檻送京師,兩金川平。乙卯,命永貴回禮部尚書,仍兼署吏部事。丙辰,命圖平定金川前後五十功臣像於紫光閣。命新設將軍駐雅州,四川提督桂林駐金川。丁巳,上還京師。戊午,上謁泰陵。命袁守侗赴四川,會同阿桂查辦參贊大臣富德。壬戌,上謁泰陵。設雲南騰越鎮總兵官。丁卯,上奉皇太后巡幸山東。己巳,免順天直隸通州等二十八州縣未完地糧倉穀。庚午,停湖北勘丈湖地。免直隸霸州等二十一州縣未完地糧倉穀。辛未,減直隸軍流以下人犯罪。

三月丁丑,免山東泰安、曲阜二縣本年額賦。戊寅,免山東鄒平等三十九州縣衛各項民欠額賦。己卯,增設成都將軍,以明亮為之。辛巳,減山東軍流以下人犯罪。壬午,免山東德州等十一州縣緩徵漕米漕項。癸未,以薩載為江南河道總督,楊魁為江蘇巡撫。甲申,勒爾謹陛見,命畢沅署陜甘總督。丙戌,上駐蹕泰安,謁岱廟。命還督撫貢物,仍嚴飭之。設金川勒烏圍總兵。丁亥,上登泰山。辛卯,戶部尚書王際華卒,以袁守侗代之。免四川通省上年額賦及本年夷賦有差。蠲河南武陟縣四十年水災額賦。乙未,上至曲阜,謁孔子廟。蠲安徽懷寧等三十二州縣、建陽等七衛四十年水旱額賦。丙申,釋奠先師孔子,告平兩金川功。丁酉,上謁孔林。調李質穎為廣東巡撫,以閔鶚元為安徽巡撫。戊戌,富德褫職逮治。己亥,雲南車裏逃夷刀維屏等悔罪自歸,諭免死,錮之。庚子,命戶部侍郎和珅軍機處行走。辛丑,上奉皇太后自濟寧登舟。

夏四月癸卯,以平定金川,遣官祭告天地、太廟、社稷。以英廉兼署戶部尚書。命劉墉會同陳輝祖查勘湖北沔陽州沖潰堤工。甲辰,予告協辦大學士、吏部尚書官保卒。丁未,上閱臨清州舊城。辛亥,命阿桂仍在軍機處行走。癸丑,蠲直隸霸州等五十二州縣四十年水災額賦有差。乙卯,以平定金川,遣官告祭昭西陵、孝陵、孝東陵、景陵、泰陵、孝賢皇后陵。丙辰,遣官告祭孔子闕里。壬戌,遣官告祭永陵、福陵、昭陵。甲子,以阿思哈為漕運總督,素爾訥為左都御史,索琳為理籓院尚書,仍留庫倫辦事,命豐升額署理籓院尚書。乙丑,上送皇太后自寶稼營還京師。丙寅,獻金川俘馘於廟社。丁卯,定西將軍阿桂等凱旋。戊辰,上幸良鄉城南行郊勞禮,賜將軍及隨征將士等宴,並賞阿桂等御用鞍馬各一。上還京師。己巳,受俘。上御瀛臺,親鞫俘囚。索諾木等皆磔於市。上御紫光閣,行飲至禮,賜凱旋將士及王大臣等宴,賜將軍阿桂以下銀幣有差。庚午,斬番目布籠普占巴、雅瑪朋阿庫魯等於市。

五月辛未朔,上奉皇太后御慈寧宮,上徽號曰崇慶慈宣康惠敦和裕壽純禧恭懿安祺寧豫皇太后,頒詔覃恩有差。戊寅,富德以誣訐阿桂悖逆,處斬。辛巳,豁山西石樓等三縣丁徭虛額銀。癸未,上奉皇太后啟鑾,秋獮木蘭。己丑,上駐蹕避暑山莊。

六月庚子朔,定文淵閣官制。壬子,以甘肅皋蘭等二十九州縣旱災,命多留市米以供民食。庚申,黃邦寧論斬,逮治前護廣西巡撫蘇爾德、署按察使廣德。

秋七月庚申,索琳以不職金雋級,以伍彌泰為理籓院尚書。丁亥,授巴延三山西巡撫,調鄂寶為湖南巡撫。

八月丁未,召瑚圖靈阿,以巴林王巴圖為定邊左副將軍,以額駙拉旺多爾濟為伊犁參贊大臣。乙卯,上幸木蘭行圍。

九月丙子,上回駐避暑山莊。庚辰,上送皇太后回鑾。庚寅,上奉皇太后還京師。

冬十月己亥朔,命豐升額為步軍統領,福隆安仍兼管。壬寅,綏遠城將軍容保罷,以伍彌泰代之。甲辰,命英誠公阿克棟阿在領侍衛內大臣上行走,以奎林為理籓院尚書。戊申,左都御史張若溎病免。辛亥,調崔應階為左都御史,以餘文儀為刑部尚書。壬子,阿思哈病免,以鄂寶為漕運總督。癸丑,以敦福為湖南巡撫。丙辰,命三寶查浙江漕糧積弊。甲子,以甘肅皋蘭等二十九州縣旱災,豁歷年積欠倉糧四百萬有奇。

十一月甲申,命四庫全書館詳覈違禁各書,分別改毀。諭曰:「明季諸人書集詞意抵觸本朝者,如錢謙益等,均不能死節,妄肆狂狺,自應查明毀棄。劉宗周、黃道周立朝守正,熊廷弼材優幹濟,諸人所言,若當時採用,敗亡未必若彼其速,惟當改易字句,無庸銷毀。又直臣如楊漣等,即有一二語傷觸,亦止須酌改,實不忍並從焚棄。」

十二月庚子,命戊戌年八月舉行繙譯鄉試,次年三月舉行會試。丙午,命明亮軍機處行走,伍彌泰遷西安將軍,博成署綏遠城將軍。戊申,以雅朗阿為綏遠城將軍。甲寅,蠲山東德州等三十州縣衛所本年被災額賦。丙辰,緬目得魯蘊請送還內地官人,準其入貢。諭令進京乞恩。戊午,上幸瀛臺。庫車阿奇木伯克、哈薩克使人,及四川明正土司等瞻覲,各賜冠服有差。

四十二年春正月戊辰朔,蠲甘肅乾隆二十三年至三十五年民欠銀八十四萬兩有奇。丙子,上御閱武樓閱兵,命諸王、大臣、外籓蒙古及回部、庫車、哈薩克使臣、金川土司等從觀。辛巳,以皇太后不豫,詣長春仙館問安,奉皇太后幸同樂園,侍晚饍。自是每日詣長春仙館請安。乙酉,以圖思德奏緬番內附,命阿桂往雲南籌辦。調李侍堯為雲貴總督,以楊景素為兩度總督,郝碩為山東巡撫,圖思德回貴州巡撫,裴宗錫回雲南巡撫。己丑,宥熊學鵬罪,蘇爾德、廣德論斬。庚寅,皇太后崩,奉安於慈寧宮正殿,上以含清齋為倚廬,頒大行皇太后遺詔。諭穿孝百日,王大臣官員等二十七日除服。辛卯,尊大行皇太后謚號為孝聖憲皇后,推恩普免錢糧一次。壬辰,定二十七日內郊廟社稷遣官致祭用樂之制。乙未,尊大行皇太后陵曰泰東陵。丙申,移大行皇太后梓宮於申昜春園,奉安於九經三事殿。上居圓明園。

二月丁酉朔,上詣安佑宮行告哀禮。上居無逸齋苫次。己亥,上還居圓明園。庚子,上詣九經三事殿大行皇太后梓宮前供奠。諸王大臣請間一二日行禮,不允。甲辰,諭二十七月內停止元旦朝賀。其百日後,尋常御殿視朝,屆日請旨。乙巳,定百日內與二十七月內御用服色及臣下服色制。甲寅,高晉會同阿揚阿赴安徽查案,楊魁兼署兩江總督。蠲安徽宿州等八州縣、鳳陽等三衛四十一年水災額賦。丁巳,上詣九經三事殿大行皇太后梓宮前行月祭禮。以顏希深為湖南巡撫。

三月辛未,左都御史素爾訥、大理寺卿尹嘉銓休致。壬申,以薩載赴京,命德保兼署江南河道總督。戊寅,以邁拉遜為左都御史。壬午,上大行皇太后尊謚曰孝聖慈宣康惠敦和敬天光聖憲皇后。戊子,以恆山保為烏里雅蘇臺參贊大臣。

夏四月戊戌,以緬番投誠反覆,召阿桂回京,留緬目所遣孟幹等。戊申,上詣九經三事殿孝聖憲皇后梓宮前行祖奠禮。己酉,孝聖憲皇后發引,上送往泰東陵,免經過州縣本年額賦十分之七。癸丑,上謁泰陵。是日,孝聖憲皇后梓宮至泰東陵,奉安於隆恩殿。丙辰,上詣泰東陵孝聖憲皇后梓宮前行百日祭禮。丁巳,大學士舒赫德卒。戊午,命永貴署大學士兼吏部尚書。辛酉,蠲安徽宿州等八州縣、長河等三衛四十一年水災額賦。壬戌,命福隆安兼署吏部尚書。甲子,上還京師。

五月乙丑朔,孝聖憲皇后神牌升祔太廟。翌日,頒詔覃恩有差。戊辰,上臨舒赫德喪次賜奠。壬申,蠲直隸清苑等十州縣逋賦。戊寅,以普蠲全國錢糧,免福建臺灣府屬官莊租息十分之三。甲申,馬蘭鎮總兵滿斗於東陵掘墻通路,論斬。丁亥,命阿桂為武英殿大學士,兼管吏部事,英廉協辦大學士。命尚書果毅繼勇公豐升額之父阿里袞原襲果毅公爵號,亦加「繼勇」二字。調永貴為吏部尚書,以富勒渾為禮部尚書,三寶為湖廣總督,王亶望為浙江巡撫。蠲順天直隸大興等三十三州縣被災額賦。

六月乙卯,以吉林將軍富椿調杭州將軍,命福康安代之。己未,上詣黑龍潭祈雨。

秋七月,蠲甘肅皋蘭等二十九州縣四十一年被災額賦。丙戌,命甘肅應徵各屬番糧草束免十分之三。暹羅頭目鄭昭進貢,送所獲緬番,諭楊景素以請封檄諭之。

八月庚子,免烏魯木齊各州縣戶民額糧十分之三。庚申,命侍郎金簡赴吉林,會同福康安查辦事件。

九月丙子,上謁泰陵、泰東陵。壬午,上還京師。

冬十月戊戌,戶部尚書果毅繼勇公豐升額卒,調英廉為戶部尚書,仍兼管刑部,命德福為刑部尚書。乙巳,詔陜西民屯租糧草束屆輪免錢糧之年,一體蠲免。庚申,設密雲副都統一,駐防兵二千。辛酉,命袁守侗赴浙江查審歸安縣知縣劉均被控案。命侍郎周煌、阿揚阿赴四川查審大足縣知縣趙憲高被控案。

十一月丙寅,廣德處斬。戊辰,海成以縱庇王錫侯褫職,以郝碩為江西巡撫,國泰為山東巡撫。壬申,刑部尚書餘文儀乞休,允之。甲戌,調袁守侗為刑部尚書,梁國治為戶部尚書。乙酉,蠲甘肅寧夏等七縣本年被災額賦。

十二月丁酉,蠲甘肅皋蘭等十七州縣四十一年被災額賦。癸丑,賑甘肅皋蘭等三十二州縣被旱災民。

四十三年春正月壬戌朔,免朝賀。癸亥,以鄭大進為河南巡撫。辛未,追復睿親王封爵及豫親王多鐸、禮親王代善、鄭親王濟爾哈朗、肅親王豪格、克勤郡王岳託原爵,並配享太廟。己卯,上謁西陵,免經過地方本年額賦十分之三。癸未,上謁泰陵、泰東陵。甲申,上謁泰東陵行期年禮。

二月丁酉,朝鮮、琉球入貢。己酉,以特成額為禮部尚書。調綽克托為吏部尚書,富勒渾為工部尚書。特成額遷成都將軍,以鍾音為禮部尚書。調楊景素為閩浙總督,桂林為兩廣總督,李質穎護之。戊午,以諴親王弘申昜為正白旗領侍衛內大臣。

三月甲子,上詣西陵。戊辰,上謁泰陵、泰東陵。己巳,上親祭泰東陵。乙亥,上閱健銳營兵。己丑,以李湖為湖南巡撫。

夏四月辛卯,以河南旱,命減開封等五府軍流以下罪。壬寅,命先免河南四十五年田賦。癸卯,肅親王蘊著卒。乙巳,上詣黑龍潭祈雨。辛亥,命減河南軍流以下罪。乙卯,賜戴衢亨等一百五十七人進士及第出身有差。

五月庚申朔,以山東荒歉,命預免四十五年錢糧。丁卯,命山西巡撫兼理河東鹽政。戊辰,怡親王弘曉卒。

六月乙未,以九江關監督全德浮收,逮治之。

閏六月癸亥,河南祥符河決。

秋七月癸巳,河南儀封考城河決。乙未,命袁守侗往河南,會同河督姚立德、巡撫鄭大進查辦河工。戊戌,命高晉督辦堤工。丁未,上詣盛京謁陵,免經過直隸、奉天各州縣本年額賦十分之三。

八月癸酉,以儀封決河下注安徽鳳陽各州縣,諭薩載等賑災民。甲戌,上謁永陵。乙亥,行大饗禮。己卯,上謁福陵。免奉天所屬府州縣明年丁賦。庚辰,行大饗禮。上謁昭陵。辛巳,行大饗禮。命奉天、吉林、黑龍江各屬已結未結死罪均減等,軍流以下悉宥之。癸未,上臨奠克勤郡王岳託墓。甲申,上臨奠武勛王揚古利、弘毅公額亦都、直義公費英東墓。乙酉,上詣文廟行禮。

九月甲午,錦縣生員金從善,以上言建儲立後,納諫施德,忤旨,論斬。戊戌,禮部尚書鍾音卒。金從善以妄肆詆斥,處斬。己亥,以德保為禮部尚書。丁未,申諭立儲流弊,及宣明歸政之期。壬子,上還京師。甲寅,高樸以婪贓論斬。綽克托以失察高樸褫職。命永貴為吏部尚書。乙卯,命邁拉遜署吏部尚書。

冬十月己未,以庚子年七旬萬壽,巡幸江、浙,命舉恩科鄉會試,並普蠲錢糧。甲戌,江蘇布政使陶易以徇縱徐述夔,褫職論斬。丙子,免甘肅皋蘭等三十二州縣四十二年旱災額賦。

十一月戊子,禁貢獻整玉如意及大玉。壬辰,定驛務歸巡道分管,裁甘肅驛傳道。賑廣西興安等九州縣本年旱災。庚子,免甘肅寧夏等七州縣四十二年被災額賦。

十二月庚申,河南儀封堤工塌壞,高晉等下部嚴議。丙寅,諭國泰嚴治山東冠縣義和拳教匪。甲戌,賑安徽當塗等三十四州縣衛本年水旱災、湖南湘陰等十五州縣衛旱災,並蠲額賦有差。

四十四年春正月丙戌朔,調陳輝祖為河南巡撫,鄭大進為湖北巡撫。乙未,大學士、兩江總督高晉卒。命三寶為東閣大學士,仍留湖廣總督任,薩載為兩江總督,李奉翰為江南河道總督。癸卯,上詣西陵,免經過地方本年丁賦十分之三。裁福州副都統。乙巳,命阿桂赴河南查勘河工。丁未,上謁泰陵、泰東陵。辛亥,上還京師。

二月癸亥,左都御史邁拉遜病免。丙子,以增福為福建巡撫,申保為左都御史。庚辰,命輯明季諸臣奏疏。諭曰:「各省送到違礙應毀書籍,如徐必達南州草,蕭近高疏草,宋一韓掖垣封事,切中彼時弊病者,俱無慚骨更。雖其君置若罔聞,而一時廢弛瞀亂之跡,痛切敷陳,足資考鏡。朕以為不若擇其較有關系者,別加編錄,名為明季奏疏,勒成一書,永為殷鑒。諸臣在勝國言事,於我國家間有干犯之語,不宜深責,應量為改易選錄,餘仍分別撤毀。」壬午,建江南龍泉莊等處行宮。

三月丙申,命英廉署直隸總督。丁酉,命德福署協辦大學士。調楊景素為直隸總督,三寶為閩浙總督。以圖思德為湖廣總督,舒常為貴州巡撫。乙巳,以譚尚忠署山西巡撫。己酉,賑湖北江夏等三十九州縣衛上年旱災。

夏四月己未,改闢展辦事大臣為吐魯番領隊大臣。戊辰,上詣西陵。壬申,上謁泰陵、泰東陵。丁丑,改甘肅驛傳道為分巡蘭州道。戊寅,以袁守侗為河東河道總督,胡季堂為刑部尚書。己卯,上閱健銳營兵。庚辰,上還京師。

五月乙未,上秋獮木蘭,免經過地方本年丁賦十分之三。丙申,以李世傑為廣西巡撫。辛丑,上駐避暑山莊。丙午,以富綱為福建巡撫。丁未,上詣文廟行釋奠禮。

六月丁卯,免甘肅乾隆二十七年至三十七年逋賦銀二十三萬五千兩、糧一百零五萬石各有奇。戊辰,河南武陟、河內沁河決。庚辰,建吐魯番滿城。

秋七月乙未,以孫士毅為雲南巡撫。

八月戊辰,上幸木蘭行圍。辛未,命和珅在御前大臣上學習行走。甲戌,以宗室永瑋為黑龍江將軍。乙亥,寧壽宮成。

九月庚子,上還京師。

冬十月壬戌,免陜西延安等三府州屬乾隆二十年至三十七年民欠社倉穀。免西藏那克舒三十九族番子等應交馬銀。乙亥,免甘肅莊浪等十七州縣被災額賦。

十一月甲申,免安徽亳州等十一州縣額賦。戊戌,杭州將軍嵩椿坐耽於逸樂褫職,仍通諭申儆。癸卯,賑甘肅皋蘭等十二州縣災民,並蠲本年額賦。丙午,以姚成烈為廣西巡撫。以伍彌泰護送班禪至熱河,給欽差大臣關防。

十二月癸丑,命侍郎德成至河南會辦河工。甲寅,命戶部侍郎董誥在軍機處行走。乙卯,兩廣總督桂林卒,以巴延三代之,雅德為山西巡撫。戊午,大學士於敏中卒。湖廣總督圖思德卒,以富勒渾代之,綽克托代為工部尚書。丙寅,賑湖北沔陽等七州縣衛本年水災。己巳,命程景伊為文淵閣大學士,調嵇璜為吏部尚書、協辦大學士,周煌為工部尚書。辛未,直隸總督楊景素卒,以袁守侗代之。調陳輝祖為河東河道總督,榮柱為河南巡撫。

四十五年春正月庚辰朔,以八月七旬萬壽,頒詔覃恩有差。辛巳,免河南儀封等十三州縣被災額賦。辛卯,上巡幸江、浙,免直隸、山東經過地方本年額賦十分之三。壬辰,免直隸順德等四府屬逋賦。己亥,免山東歷城等二十八州縣逋賦及倉穀。己酉,朝鮮國王李算表賀萬壽,優詔答之。修浙江仁和、海寧塘工。

二月癸丑,命舒常同和珅、喀寧阿查辦海寧劾李侍堯各款。甲寅,免江南、浙江經過地方本年額賦十分之三。免兩江所屬四十三年以前逋賦。丙辰,調李奉翰為河東河道總督,陳輝祖為江南河道總督。丁巳,免臺灣府屬本年額穀,免兩淮灶戶災欠及川餉未繳銀。己未,上渡江,閱清口東壩堤工。甲子,免江南、浙江省會附郭諸州縣本年額賦。戊辰,上幸焦山。壬申,上幸蘇州府。儀封決口合龍。己卯,免浙江仁和等縣逋賦。

三月辛巳,上幸海寧州觀潮。壬午,上幸尖山。召索諾木策凌來京,以奎林為烏魯木齊都統。癸未,上幸杭州府。甲申,上幸秋濤宮閱水師。以博清額為理籓院尚書。壬辰,調李質穎為浙江巡撫,李湖為廣東巡撫,以劉墉為湖南巡撫。以京察屆期,予阿桂等議敘,左都御史崔應階等原品休致。癸巳,以羅源漢為左都御史。丁酉,李侍堯褫職逮問。孫士毅褫職,發伊犁效力。以福康安為雲貴總督,索諾木策凌為盛京將軍。辛丑,命英廉為東閣大學士,和珅為戶部尚書。丙午,上詣明太祖陵奠酒。

夏四月己酉朔,上渡江。壬子,山東壽光人魏塾以著書悖妄,處斬。丁巳,上至武家墩,閱高家堰堤工,渡河。免山西太原等十六府州並歸化城等應徵額賦十分之三,大同、朔平及和林格爾等屬全免之。辛酉,調楊魁為陜西巡撫,劉秉恬署云南巡撫,顏希深為貴州巡撫,吳壇為江蘇巡撫。丁卯,調楊魁為河南巡撫,雅德為陜西巡撫,喀寧阿為山西巡撫。

五月甲申,以大學士、九卿改和珅所擬李侍堯監候為斬決,諭各督撫各抒所見,定擬題奏。丁亥,上還京師。癸巳,賜汪如洋等一百五十五人進士及第出身有差。丁酉,宥孫士毅罪。己亥,上秋獮木蘭。乙巳,上駐蹕避暑山莊。甲寅,免湖北沔陽等五州縣本年水災額賦。乙卯,召大學士三寶入閣辦事。調富勒渾為閩浙總督,舒常為湖廣總督。丁卯,以和珅為正白旗領侍衛內大臣。庚午,江蘇睢寧郭家渡河決。

秋七月丁丑,起孫士毅為編修。丁酉,班禪額爾德尼自後藏入覲,上御清曠殿,賜坐,賜茶。戊戌,順天良鄉永定河決。庚子,上御萬樹園,賜班禪額爾德尼及王、公、大臣,蒙古王、貝勒、貝子、公、額駙、臺吉等宴,並賜冠服金幣有差。辛丑,山東曹縣及河南考城河決。壬寅,以李本為貴州巡撫。

八月戊申,賑河南寧陵等四縣水災。乙卯,大學士程景伊卒。丁巳,永定河決口合龍。湖北巡撫鄭大進貢金器,不納,切責之。己未,上七旬萬壽節,御澹泊敬誠殿,王、公、大臣及蒙古王、貝勒、貝子、額駙、臺吉等行慶賀禮。癸酉,調閔鶚元為江蘇巡撫,農起為安徽巡撫。甲戌,上詣東西陵,免經過地方本年額賦十分之三。賑浙江諸暨等七縣水災。

九月,以嵇璜為文淵閣大學士,蔡新為吏部尚書、協辦大學士。調周煌為兵部尚書,以周元理為工部尚書。壬午,上謁昭西陵、孝陵、孝東陵、景陵,詣孝賢皇后陵奠酒。辛卯,上謁泰陵、泰東陵。睢寧郭家渡決口合龍。乙未,上還京師。乙巳,賑吉林琿春水災。

冬十月戊申,定李侍堯斬監候。調雅德為河南巡撫。辛酉,免河南儀封等六縣本年水災額賦。壬戌,免直隸霸州等六十三州縣本年水災額賦。免江蘇清河等八州縣衛本年水旱額賦。免甘肅皋蘭等三十五州縣四十四年水災額賦。甲戌,命博清額署左都御史,和珅仍兼署理籓院尚書。

十一月庚辰,命博清額為欽差大臣,護送班禪額爾德尼往穆魯烏蘇地方。壬午,以慶桂為烏里雅蘇臺將軍。癸未,班禪額爾德尼卒於京師。

十二月乙卯,賑甘肅皋蘭等十八州縣饑民。庚申,以會同四譯館屋壞,壓斃朝鮮人,禮部尚書等下部嚴議。丁卯,命阿桂會同陳輝祖、富勒渾、李質穎勘視海塘。

四十六年春正月己卯,定蒙古喀爾喀,青海杜爾伯特、土爾扈特、和碩特,回部王、公、札薩克、臺吉等世襲爵秩。丙申,朝鮮國王李算表謝賜緞匹,仍貢方物,溫諭受之。癸卯,召富勒渾、李質穎來京。以陳輝祖為閩浙總督,兼管浙江巡撫,督辦塘工。調李奉翰為江南河道總督,韓鑅為河東河道總督。

二月丙辰,免浙江諸暨水災額賦。癸亥,命阿桂勘視江南、河南河工。乙丑,上西巡五臺山,免經過地方本年額賦十分之三。丙寅,免順天保定七府州縣逋賦。己巳,調雅德為山西巡撫。庚午,以富勒渾為河南巡撫。王燧論絞。

三月甲戌朔,上幸正定府閱兵。乙亥,免安徽亳州等九州縣、鳳陽等三衛水災額賦有差。丙子,免江蘇清河等八州縣衛水災額賦有差。戊寅,召慶桂來京,以巴圖署烏里雅蘇臺將軍。辛巳,上駐蹕五臺山。己丑,免甘肅皋蘭等十五州縣雹災額賦有差。甲午,以宗室嵩椿為綏遠城將軍。庚子,上還京師。壬寅,甘肅循化撒拉爾回匪蘇四十三等作亂,陷河州,命西安提督馬彪同勒爾謹剿之。癸卯,回匪犯蘭州,命阿桂往甘肅調度剿賊機宜。

夏四月甲申朔,命尚書和珅、額駙拉旺多爾濟、領侍衛內大臣海蘭察,並巴圖魯侍衛等,赴甘肅剿賊。乙巳,命安徽巡撫農起往甘肅辦理軍需,宥李侍堯罪,賞三品頂戴赴甘肅。己酉,甘肅官軍收復河州,仁和進援省城。庚申,休致大理寺卿尹嘉銓坐妄請其父從祀孔廟及著書狂悖,處絞。免直隸霸州等五十州縣水災額賦。戊辰,賜錢棨等一百六十九人進士及第出身有差。庚午,逮勒爾謹,以李侍堯管理陜甘總督事,未至,以阿桂兼管之。召和珅回京。辛未,免安徽壽州等十二州縣衛、河南儀封等五縣水災額賦。

五月辛卯,諭阿桂等除回民新教。

閏五月癸卯朔,勒爾謹論斬。己酉,免江蘇阜寧等七縣衛逋賦。庚戌,上秋獮木蘭。丙辰,上駐蹕避暑山莊。

六月庚辰,江蘇睢寧魏家莊河決。己丑,以甘肅累年冒賑,命刑部嚴鞫勒爾謹,逮王亶望至都。壬辰,免陜西西安等十二府州民欠倉穀。癸巳,甘肅回匪蘇四十三等伏誅。

秋七月壬寅朔,江蘇崇明、太倉等州縣海溢。甘肅布政使王廷贊,以冒賑浮銷,褫職逮治。丙午,以奎林為烏里雅蘇臺將軍,明亮為烏魯木齊都統。己酉,河南萬錦灘及儀封曲家樓河決。庚申,暹羅國長鄭昭遣使齎表貢方物。辛酉,命阿桂閱視河南、山東河工。乙丑,南掌國王弟召翁貢方物。庚午,王亶望處斬,賜勒爾謹自盡,王廷贊論絞。免江蘇崇明縣本年額賦。賑江蘇崇明等九州縣、河南儀封縣水災。

八月甲戌,賑甘肅隴西等四縣水災。免金縣等七縣額徵半賦。己卯,袁守侗等坐查監糧失實,下部嚴議。壬午,調福康安為四川總督,以富綱為雲貴總督,楊魁署福建巡撫。乙酉,賑湖北潛江等四州縣水災。丙戌,上幸木蘭行圍。魏家莊決口合龍。

九月戊申,王廷贊處絞。丁卯,賑山東金鄉水災。

冬十月丙子,賑江蘇銅山等縣水災。丁丑,賑山東鄒平等二十九州縣、濟寧等三衛、永阜等三場水災。乙酉,賑直隸滄州等四州縣、嚴鎮等四場水災。戊子,賑河南祥符十三縣水災。庚寅,賑湖北江夏等十七州縣水旱災。癸巳,賑安徽靈壁等二十四州縣衛水旱災。丁酉,上以御史劉天成奏,諭曰:「均田之法,勢必致貧者未富,富者先貧。我君臣惟崇儉尚樸,知愧知懼,使四民則效而已。」罷陜西貢皮。

十一月庚子,工部尚書周元理予告,以羅源漢代之。以劉墉為左都御史,仍暫管湖南巡撫。丙午,以李世傑為湖南巡撫。戊辰,以鄭大進為直隸總督。

十二月己巳朔,調姚成烈為湖北巡撫。以硃椿為廣西巡撫。丁丑,以雅德為廣東巡撫,譚尚忠為山西巡撫。戊子,大學士等議駁嵇璜請復黃河故道,上韙之。庚寅,畢沅以御史錢灃劾,降三品頂戴留任。辛卯,調農起為山西巡撫,譚尚忠為安徽巡撫。

四十七年春正月庚子,陳輝祖、閔鶚元降三品頂戴留任。乙卯,建盛京文溯閣。丙寅,四庫全書成。

二月己巳,上御文淵閣,賜四庫全書總裁等官宴,賞賚有差。丁亥,命乾清門侍衛阿彌達致祭河神。

三月庚子,上幸盤山。壬寅,上駐蹕盤山。癸丑,調雅德為福建巡撫,以尚安為廣東巡撫。甲寅,上還京師。乙卯,免甘肅積年逋賦糧二百四十五萬石、銀三十萬兩各有奇。戊午,免江蘇常熟等二十八州縣衛水災額賦。癸亥,免直隸天津等三十九州縣水災額賦。

夏四月戊辰,命和珅、劉墉同御史錢灃查辦山東虧空。戊寅,免山東壽光等五縣水災額賦。己卯,山東巡撫國泰褫職逮問,以明興代之。辛巳,上閱火器營兵。甲申,免山西永濟縣水災額賦。丁亥,上閱健銳營兵。壬辰,協辦大學士、吏部尚書蔡新乞假,允之。以劉墉署吏部尚書。甲午,羅源漢罷,以劉墉為工部尚書,王傑為都察院左都御史,慶桂為盛京將軍。

五月丁酉,召阿桂來京,命韓鑅、富勒渾籌辦河工。己亥,賑山東曹州、兗州、濟寧等府州,江蘇徐州、豐、沛等縣水災。辛丑,免河南祥符等六縣水災額賦。定新建巴爾噶遜城名曰嘉德。戊申,上幸木蘭。庚戌,免安徽懷寧等十八州縣、安慶等五衛水災額賦。甲寅,上駐蹕避暑山莊。

六月丙子,國泰、於易簡論斬。以富躬為安徽巡撫。

秋七月丙申朔,命阿桂仍督辦河工。戊戌,索諾木策凌論斬。癸卯,國泰、於易簡賜自盡。甲辰,以李侍堯、國泰所辦貢物過優,皆致罪戾,諭各督撫等惟當潔清自矢,毋專以進獻為能。己未,以何裕城署河東河道總督。癸亥,免甘肅隴西等四縣四十六年水災額賦。

八月丁卯,以福康安為御前大臣。癸酉,以宗室永瑋為吉林將軍,宗室恆秀為黑龍江將軍。甲戌,加英廉、嵇璜、和珅、李侍堯、福康安太子太保,梁國治、鄭大進太子少傅,薩載太子少保。壬午,賑江蘇沛縣等州縣,山東鄒、嶧二縣被水災民。癸未,上幸木蘭行圍。乙酉,賜索諾木策凌自盡。壬辰,賑山東兗州等府縣被水災民。

九月丙申,建浙江文瀾閣。壬寅,上回駐避暑山莊。癸卯,刑部尚書德福卒,以喀寧阿代之。命英廉暫管刑部。乙巳,調宗室永瑋為盛京將軍,慶桂為吉林將軍。辛亥,陳輝祖褫職逮問,調富勒渾為閩浙總督,福長安署之。調李世傑為河南巡撫,以查禮為湖南巡撫。己未,賑浙江玉環等處海溢災民。辛酉,免奉天承德等五縣水災額賦。

冬十月癸酉,新建庫爾喀喇烏蘇城名曰慶綏,晶河城名曰安阜。丁卯,賑河南汝陽等十六縣水災。甲申,直隸總督鄭大進卒,以袁守侗署之。以福崧為浙江巡撫。賑安徽壽州等十六州縣衛水旱災。

十二月癸亥朔,陳輝祖及國棟等論斬。甲申,常青遷杭州將軍。以烏爾圖納遜為察哈爾都統。

四十八年春正月甲午,以伊星阿為湖南巡撫。戊申,以薩載為兩江總督,畢沅為陜西巡撫,劉秉恬為雲南巡撫。

二月甲子,賜陳輝祖自盡,王燧處斬。乙丑,以毓奇為漕運總督。丙寅,以拉旺多爾濟為御前大臣。戊辰,命建闢雍於太學。辛未,上詣西陵,免經過地方額賦十分之三。乙亥,上詣泰陵、泰東陵。戊子,賜明遼東經略熊廷弼五世孫泗先為儒學訓導。

三月辛丑,予大學士阿桂等議敘。禮部侍郎錢載等原品休致。予總督袁守侗等、巡撫農起等議敘。召硃椿來京,以劉瓘為廣西巡撫。甲寅,免江蘇銅山等十九州縣、淮安等三衛水旱災額賦。

夏四月乙丑,御前大臣喀喇沁郡王札拉豐阿卒,以拉旺多爾濟為御前大臣。乙亥,上閱火器營兵。辛巳,召福康安來京。

五月壬辰,以福康安為正黃旗領侍衛內大臣。予李奉翰兵部尚書、右都御史銜。甲辰,以硃椿為左都御史。丙午,協辦大學士、吏部尚書永貴卒。免安徽壽州等十一州縣上年水災額賦。丁未,直隸總督袁守侗卒,以劉瓘代之。以孫士毅為廣西巡撫,伍彌泰為吏部尚書、協辦大學士。己酉,上有疾,命永瑢代祀方澤。癸丑,上幸木蘭。庚申,上駐蹕避暑山莊。

六月乙丑,體仁閣火。乙酉,免山東永阜等五場上年水災額賦。丁亥,賑湖北廣濟等六州縣水災。

秋七月戊戌,命海祿署伊犁將軍,圖思義署烏魯木齊都統。乙卯,命蔡新為文華殿大學士,梁國治協辦大學士,劉墉為吏部尚書。

八月甲午,賜達賴喇嘛玉冊玉寶。甲戌,明亮、巴林泰等褫職逮問,以海祿為烏魯木齊都統。乙亥,上自避暑山莊詣盛京謁陵,免經過地方本年額賦十分之五。庚辰,太子太保、大學士英廉卒。辛巳,上駐蹕哈那達大營。喀喇沁郡王喇特納錫第等迎駕,賞賚有差。丁亥,上駐五里屯大營,科爾沁親王恭格喇布坦、巴林郡王巴圖等迎駕,賞賚有差。戊子,予明遼東經略袁崇煥五世孫炳以八九品官選補。

九月己丑朔,上駐蹕四堡子東大營閱射。命皇十一子永瑆等迎冊寶至盛京,藏於太廟。癸巳,上駐老邊大營閱射。朝鮮國王遣使貢方物。乙未,免奉天各屬乾隆四十九年額賦。戊戌,上謁永陵。己亥,行大饗禮。閱興京城。免盛京戶部各莊頭倉糧。免盛京等處旗地應納米豆草束十分之五。減奉天等處死罪,免軍流以下罪。癸卯,上謁福陵。甲辰,行大饗禮。上謁昭陵,臨奠武勛王揚古利墓。乙巳,行大饗禮。丙午,上臨奠克勤郡王岳託墓。丁未,上臨奠弘毅公額亦都、直義公費英東墓。戊申,上御崇政殿受慶賀。御大政殿賜扈從皇子、王、公、大臣等宴,賞賚有差。己酉,上詣清寧宮祭神,賜皇子、王、公、大臣等食胙。庚戌,上回蹕。戊午,申諭詹事府備詞臣升轉之階,及建儲之必不可行。

冬十月壬戌,賑陜西榆林八州縣等旱災。癸亥,上駐蹕文殊菴行宮。壬申,上謁昭西陵、孝陵、孝東陵、景陵。乙亥,上還京師。

十一月己亥,釋國棟。庚子,以福隆安病未痊,命福康安協同辦理兵部尚書。辛丑,命劉瓘飭玉田附近州縣掘蝗蝻。壬寅,命劉瓘查辦南宮縣義和拳邪教。己酉,以阿克棟阿為烏里雅蘇臺參贊大臣,那爾瑚善為塔爾巴哈臺參贊大臣。

十二月丙寅,命福康安赴廣東,會同永德讞鹽商獄。

四十九年春正月丁未,上南巡,免直隸、山東經過地方本年錢糧十分之三。戊申,免直隸順天等十二府州屬逋賦。甲寅,調孫士毅為廣東巡撫,以吳垣為廣西巡撫。丙辰,免山東利津等二十一州縣衛逋賦。召巴延三來京,調舒常為兩廣總督。以特成額為湖廣總督,保寧為成都將軍。

二月壬戌,上幸泰安府,詣岱廟行禮。丙寅,上謁少昊陵。至曲阜謁先師廟。丁卯,釋奠先師,詣孔林酹酒。祭元聖周公廟。壬申,免江寧、蘇州、安徽各屬逋賦。免江南、浙江經過地方本年錢糧十分之三。以永保為貴州巡撫。賚江南、浙江耆民。戊寅,祭河神。上渡河。減江蘇、安徽、浙江三省軍流以下罪。壬午,免江南江寧、蘇州,浙江杭州等附郭諸縣額賦。甲申,免兩淮灶戶四十五、六兩年逋賦。

三月丙戌朔,祭江神。上渡江,幸金山。丁亥,上幸焦山。調周煌為左都御史。己丑,以王傑為兵部尚書,俟服闋後供職。辛卯,上幸蘇州府。壬辰,免湖北江夏等二十四州縣衛三十年至四十四年逋賦。乙未,上詣文廟行禮。丁酉,再免浙江杭州、嘉興、湖州三府屬額賦十分之三。己亥,上幸海寧州祭海神。以福建欽賜進士郭鍾嶽年屆一百四歲,來浙迎鑾,賞國子監司業。庚子,上幸尖山觀潮。閱視塘工。辛丑,上幸杭州府。癸卯,上詣聖因寺祭聖祖神御。戊申,上閱福建水師。庚戌,上自杭州回鑾。改慶桂為福州將軍。以都爾嘉為吉林將軍。增西安副都統一。甲寅,上駐蹕蘇州府。巴延三褫職。

閏三月丙辰朔,兵部尚書福隆安卒,以福康安為兵部尚書,復興署工部尚書。壬戌,上幸江寧府。甲子,祭明太祖陵。乙丑,上閱江寧府駐防兵。戊辰,上渡江。丙子,上祭河神,渡河。以伊齡阿為總管內務府大臣。是月,免江蘇上元等八州縣衛,安徽懷寧等十州縣、安慶等三衛上年水旱災額賦。

夏四月丙戌,免直隸宛平等五州縣上年水災額賦。庚寅,上祭禹廟。壬寅,以李綬為江西巡撫。甲辰,以河南衛輝等屬旱,免汲縣等十六縣逋賦。乙巳,免直隸大名等七州縣逋賦。丙午,甘肅新教回人田五等作亂,命李侍堯、剛塔剿之。丁未,上還京師。以海祿為烏什參贊大臣。庚戌,免陜西、甘肅三十八年至四十六年逋賦。辛亥,調李綬為湖南巡撫,以伊星阿為江西巡撫。甲寅,賜茹棻等一百十二人進士及第出身有差。是月,免湖北黃梅等四縣、武昌等三衛上年水災額賦。

五月丙辰,綽克托以緣事褫職逮問,以慶桂為工部尚書。調常青為福州將軍,以永鐸為杭州將軍。己未,命慶桂在軍機處行走。壬戌,上秋獮木蘭。癸亥,免陜西延安等三府州逋賦。戊辰,上駐蹕避暑山莊。己巳,命福康安、海蘭察赴甘肅剿捕回匪。甲戌,命阿桂領火器、健銳兩營兵往甘肅剿叛回。以阿桂為將軍,福康安、海蘭察、伍岱並為參贊大臣。乙亥,甘肅回匪陷通渭縣,尋復之。以舒亮為領隊大臣。庚辰,李侍堯坐玩誤褫職,以福康安為陜甘總督。剛塔以失機褫職逮問。辛巳,調慶桂為兵部尚書,復興為工部尚書。以阿揚阿為左都御史。癸未,江南巡撫郝碩坐貪婪逮問。是月,免山東兗州等三府州屬上年水災額賦。

六月庚寅,免甘肅本年額賦。甲午,賑湖南茶陵、攸縣水災。壬寅,東閣大學士三寶卒。戊申,以書麟為安徽巡撫。是月,免安徽懷寧等十三州縣衛上年水旱額賦。

秋七月甲寅朔,日食。丁巳,禮部尚書曹秀先卒,以姚成烈為禮部尚書。調李綬為湖北巡撫,以陸燿為湖南巡撫。己未,賜郝碩自裁。甲子,甘肅石峰堡回匪平,俘賊首張文慶等。予阿桂輕車都尉,晉封福康安嘉勇侯,擢海蘭察子安祿二等侍衛,授伍岱都統,俱給騎都尉,和珅再給輕車都尉,餘各甄敘有差。丙寅,以常青為烏魯木齊都統。癸酉,以伍彌泰為東閣大學士。調和珅為吏部尚書、協辦大學士,兼管戶部。以福康安為戶部尚書,仍留陜西總督任。戊寅,命頒行軍紀律。癸未,李侍堯論斬。宥剛塔罪,戍伊犁。是月,免陜西榆林等八州縣上年旱災額賦。

八月己丑,河南睢州河決,命阿桂督治之。癸巳,免甘肅積年逋賦銀三十五萬兩、糧四十七萬石各有差。乙未,以河南偃師縣任天篤九世同居,賜禦制詩御書扁額。己亥,上幸木蘭行圍。辛丑,張文慶等伏誅。甲辰,暹羅國長鄭華遣陪臣貢方物,乞封。

九月癸丑朔,賑安徽宿州等處水災。乙卯,以回匪平,封和珅一等男。庚申,上駐蹕避暑山莊。甲子,調烏爾圖納遜為察哈爾都統,積福為綏遠城將軍。甲戌,上還京師。丙子,宥綽克托罪。庚辰,命內大臣西明、翰林院侍讀學士阿肅使朝鮮,冊封世子。是月,賑陜西華州等三州縣水災。冬十月辛卯,命重舉千叟宴。戊戌,賑江西南昌等六縣水災。己酉,減京師朝審情實句到逾三次人犯罪。

十一月乙丑,諭秋審、朝審各犯緩決至三次者,分別減等。壬申,睢州河工合龍。庚辰,命留保住為駐藏大臣,以福祿為西寧辦事大臣。

十二月甲辰,諭預千叟宴官民年九十以上者,許其子孫一人扶掖;大臣年逾七十者,如步履稍艱,亦許其子孫一人扶掖。

是歲,朝鮮、琉球、暹羅、安南來貢。

五十年春正月辛亥朔,上以五十年國慶,頒詔覃恩有差。丙辰,舉千叟宴禮,宴親王以下三千人於乾清宮,賞賚有差。丁巳,左都御史周煌致仕,以紀昀為左都御史。調吳垣為湖北巡撫,以孫永清為廣西巡撫。戊辰,召奎林來京,以拉旺多爾濟署烏里雅蘇臺將軍。甲戌,喀什噶爾阿奇木伯克阿裏木以潛與薩木薩克交通事覺,處斬。乙酉,賑江西萍鄉等三縣水災。丁亥,上釋奠先師,臨闢雍講學。戊子,免河南汲縣等十四縣逋賦。己丑,御試翰林院、詹事府官,擢陸伯焜、吳璥為一等,餘升黜有差。試六部升用翰詹等官,擢慶齡為一等,餘升黜有差。辛卯,調畢沅為河南巡撫,何裕城為陜西巡撫。甲辰,免江南江寧等六府州逋賦。是月,賑江西萍鄉等三縣、福建建安等二縣水災,河南汲縣等十四縣旱災。

三月壬子,上幸盤山。甲寅,上詣明長陵奠酒。丁巳,上駐蹕盤山。辛酉,截河南、山東漕糧三十萬石,賑河南衛輝旱災。甲子,免江蘇安東、阜寧逋賦。丙寅,上還京師。丁卯,以永鐸為伊犁參贊大臣,常青為西安將軍,奎林為烏魯木齊都統,復興為烏里雅蘇臺將軍。以舒常為工部尚書,孫士毅兼署兩廣總督。乙亥,免直隸霸州等四十九州縣逋賦。丙子,免河南商丘等六州縣上年水災額賦。

夏四月甲申,甘肅肅州等處地震,賑恤之。壬辰,上閱健銳營兵。丁酉,刑部尚書喀寧阿、胡季堂,侍郎穆精阿、姜晟以檢驗失實,降四品頂戴。戊戌,大學士蔡新致仕。是月,免河南汲縣等旱災額賦。賑祥符等州縣旱災。

五月壬子,免河南祥符等十六州縣、鄭州等三十二州縣新舊額賦積欠。甲寅,調永保為江西巡撫,陳用敷為貴州巡撫。己未,撥兩淮運庫銀一百萬兩交河南備賑。丙寅,上秋獮木蘭。丁卯,山西平陽等屬饑,給貧民兩月糧。壬申,上駐蹕避暑山莊。丙子,命梁國治為東閣大學士,兼戶部尚書,劉墉協辦大學士。以曹文埴為戶部尚書。丁丑,柘城盜匪平。是月,賑江蘇銅山等十六州縣、山東陵縣等四十州縣旱災。

六月壬午,以漕運遲誤,薩載等下部嚴議,分別賠償。乙酉,理籓院尚書博清阿卒。丙戌,以留保住為理籓院尚書。辛丑,以奎林署伊犁將軍,永鐸署烏魯木齊都統。乙巳,命再截留江西漕糧十萬石於安徽備賑。是月,賑安徽亳州等八州縣旱災。

秋七月己酉,調富勒渾為兩廣總督,以雅德為閩浙總督,浦霖為福建巡撫。庚戌,調浦霖為湖南巡撫,以徐嗣曾為福建巡撫。辛酉,以李慶棻為貴州巡撫。乙丑,撥戶部銀一百萬兩交河南備賑。辛未,賑山西代州等六州縣水災。乙亥,以奎林為伊犁將軍,永鐸為烏魯木齊都統。

八月乙酉,命阿桂赴河南勘災,兼赴江南、山東查辦河運。癸巳,上幸木蘭行圍。庚子,賑陜西朝邑縣水災。癸卯,以伊桑阿為山西巡撫。

九月己酉,命福康安赴阿克蘇安輯回眾。以慶桂為烏什參贊大臣,署陜甘總督。降海祿為伊犁領隊大臣。命明亮以伊犁參贊大臣署烏什參贊大臣。甲寅,上駐蹕避暑山莊。戊午,調永保為陜西巡撫,何裕城為江西巡撫。戊辰,上還京師。壬申,賑江蘇長洲等五十六州縣衛旱災。

冬十月丁丑朔,召勒保、松筠回京,命佛住駐庫倫,會同蘊端多爾濟辦事。庚辰,賑湖南巴陵等十州縣旱災。辛丑,賑安徽亳州五十一州縣並鳳陽等九衛旱災。是月,免甘肅皋蘭等十二州縣衛本年雹水災額賦。賑直隸平鄉等十六州縣水旱災,河南永城等十二州縣旱災。

十一月乙亥,以乾隆六十年乙卯正旦推算日食,宣諭定次年歸政。是月,賑山東嶧縣等九州縣旱災,甘肅河州等七州縣水雹災。

十二月丁丑,以御史富森阿條陳地丁錢糧請收本色,諭斥為斷不可行,罷之。丙戌,以明亮為烏什參贊大臣,慶桂為塔爾巴哈臺參贊大臣。壬寅,禁廣東洋商及粵海關監督貢獻。是月,賑陜西朝邑等三縣水災。

是歲,朝鮮來貢。


\end{pinyinscope}