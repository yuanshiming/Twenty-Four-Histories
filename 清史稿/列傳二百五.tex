\article{列傳二百五}

\begin{pinyinscope}
袁甲三子保恆毛昶熙

袁甲三,字午橋,河南項城人。道光十五年進士,授禮部主事,充軍機章京,累遷郎中。三十年,遷御史、給事中,疏劾廣西巡撫鄭祖琛慈柔釀亂,又劾江西巡撫陳阡賄賂交通,皆罷之。戶部復捐例,疏請收回成命。咸豐元年,粵匪起,南河豐北決口,上疏極論時事,皆切中利弊。二年,粵匪竄湖南,疏言:「總督程矞採為守土之臣,責無旁貸。若復令賽尚阿持節移軍,誠恐諉過爭功,互相掣肘。請命賽尚阿回京,專責程矞採便宜行事,如有疏虞,按律定罪。」並言:「湖北巡撫龔裕聞賊入境,託疾乞休,尤宜嚴懲,以昭炯戒。」又列款奏劾定郡王載銓賣弄橫勢,擅作威福,及刑部侍郎書元貪鄙險詐,諂事載銓狀,詔詰載銓所收門生實據,疏請飭呈出所繪息肩圖,事皆得實。載銓坐罰王俸,奪領侍衛大臣兼官,書元及尚書恆春降調,題圖者降謫罰俸有差。於是直聲震中外。

三年,命赴安徽佐侍郎呂賢基軍務。粵匪陷鳳陽府,踞明淮關,煽動土匪,連陷蒙城、懷遠。甲三至軍,疏言:「賊勢未遽北犯,請飭諸臣勿涉張皇,急圖制賊。」命權廬鳳道。漢、回相閧,圍潁州,遣兵解散,誅首亂,事即定。會漕運總督周天爵卒於亳州,命代領其軍。時土匪合五十八捻為一,勢甚張。甲三至王市集,收散勇,整民團,擊賊高公廟,破之,加三品卿銜。命署布政使,疏辭不赴,請專治兵事,允之;命專剿捻匪,破賊標裡鋪,擒其渠鄧大俊。鄉團先後擒獻者二千餘,悉置之法。

十月,粵匪由安慶竄踞桐城,尋陷舒城,呂賢基死之。上命移軍桐城,甲三疏言:「捻首張茂踞懷、蒙間,窺廬郡,請先赴蒙、亳為諸郡聲援。」時捻匪麕聚雉河集,甲三令縣丞徐曉峰擊破之,擒賊渠孫重倫。分兵擊敗臨湖鋪竄匪,擒賊渠宮步雲、馬九,並其目數十人。令游擊錢朝舉、知縣米鎮攻懷遠,大破之,張茂負傷遁。十二月,賊陷廬州,巡撫江忠源戰歿。甲三劾陜甘總督舒興阿擁兵坐視,褫其職;並請撥兵防壽卅、六安以杜旁擾。

四年二月,粵匪陷六安,竄蒙城,甲三進搗蒙城。賊走永城,甲三恐其趨宿、徐,阻糧道,急躡之,賊已濟河,不及而還。賊尋復南竄,連敗之潁州、正陽關,餘賊退六安。還軍蒙、亳剿捻匪,搗臨湖鋪,進偪雉河集。賊空巢誘官軍,甲三偵知,盡泊船南岸,令知州張家駒陣河干,參將硃連泰、李成虎敗賊馬家樓,迫之渦河,殲賊殆盡,遂破義門集,捻首張捷三遁去。

甲三移屯臨淮,地數被兵,比戶凋敝。既至,討軍實,撫殘黎,眾皆樂為之用,超擢左副都御史。疏言:「皖軍以克復廬州為急,宜出偏師赴南路斷賊接濟。」尋以賊陷和州,窺江浦,將北竄。分遣將扼關山,赴滁河鳩團練為聲援。十月,北路捻匪復熾,令張家駒、硃連泰率軍破之於寺覺集。粵匪踞烏江,令廬鳳道張吉第擊敗之。賊夜結五壘於駐馬河,乘其初至薄之,殲擒甚眾。令參將劉玉豹、舉人臧紆青規桐城,連奪大小關,擊走廬江援賊。紆青戰甚銳,進攻桐城西門,賊由安慶、潛山來援,城賊出應之,紆青戰死,玉豹收餘眾退保六安。

五年,疏陳軍事,略曰:「北路以臨淮為要,正陽次之。臣駐臨淮,牛鑒扼正陽,以防賊北渡。廬州為中路,和春、福濟師老力疲,久攻不下。西路蘄、黃無處非賊,兵力過單。東路沿江針魚嘴、西梁山,賊船賊壘,來去無常。張光第等分軍進攻,然無水師夾擊,終難收效。目下悍賊力爭江路,群聚上游,廬州有機可乘,請益厚兵力,分扼廬城東南,或增兵並剿舒、巢,俾其應接不暇,庶可一鼓而下。」

時淮北官吏,甲三欲有更調,和春、福濟意不合,甲三專奏,詔仍飭會銜。於是和春、福濟疏劾甲三堅執己見,並劾其株守臨淮,粉飾軍情,擅裁餉銀,冒銷肥己。召回京,部議褫職。甲三呈訴被誣,下兩江總督按治,事得白。甲三在淮北得軍民心,其去也,軍民泣留者塞道。未幾,捻首張洛行勾結皖、豫諸捻,勢益熾。懷遠民胡文忠鬻子女,徒步京師,控都察院求以甲三回鎮,格不達,懷狀自縊。言官孫觀、曹登庸、宗稷辰先後疏請起用;疆臣怡良、吉爾杭阿、何桂清亦交章論薦。

六年二月,命隨同英桂剿捻河南。甲三赴歸德,招集舊部,三戰三捷,進解亳州之圍,毀白龍王廟砦,破燕家小樓賊數萬,直搗雉河集,擒蘇天福,洛行僅身免,特詔嘉獎,命以三品京堂候補。洛行尋復糾黨犯潁州,擊走之,又踞雉河集。七年,平王、鄧、宋、姚諸圩,誅捻渠李寅等百餘人,授太僕寺卿,賜花翎。勝保督師攻張洛行於正陽關,久不下,奏請甲三合剿,令部將硃連泰、史榮椿攻韓圩,克之。八年,偕勝保解固始之圍,復六安。史榮椿破捻匪於銅山,斬其渠孫大旺。移軍宿州,襲賊王家圩,誅賊首王紹堂等,乘勝復七圩。七月,命代勝保督辦三省剿匪事宜。張洛行方踞陳家莊,擊走之,分兵復豐縣。未幾,蒙、亳諸捻入歸德,窺周家口,甲三令子保恆偕總兵傅振邦馳援。賊遽趨西北,偪開封,振邦追賊,及之太和李興集。保恆集團勇扼橋口,馬步合擊,大破之,殲斃數千,逐賊出河南境,賜號伊勒圖巴圖魯。疏言:「兵分則勢孤,合則勢盛。捻匪踞地千餘里,臣兵不過數千,不能制賊死命。請敕各督撫合力大舉,為掃穴擒渠之計。」

九年正月,擊張洛行於草溝,破其巢,追至沱河,多溺水死,復擊之雙渡口,洛行泅水免。勝保與甲三意不合,屢疏詆之,詔斥「甲三督剿半載,但防徐、宿,不搗賊巢,日久無效」。召回京,入覲,面陳軍事。四月,命署漕運總督。尋勝保以母憂歸,命署欽差大臣,督辦安徽軍務,實授漕運總督。進攻臨淮關,軍南岸,斷其糧道,降捻內應,斬關而入,生擒賊首顧大隴等,遂克之。

十年,進規鳳陽,屢戰皆捷。鄧正明以府城乞降,張元隆猶據縣城,誘出誅之,並誅悍賊三百餘人。未匝月,拔兩城,詔嘉調度有方,賜黃馬褂,命其子保恆赴軍差遣。

捻匪陷清江浦,窺淮安,令道員張學醇擊走之,乘勝復全椒。粵匪陳玉成來援,分擾滁州,令李世忠夾擊走之。是時江北無統帥,揚州叛將薛成良擁眾剽掠,亟發舟師扼高、寶諸湖。成良走依李世忠,甲三責以大義,即縛獻成良,斬之以徇。令保恆合總兵張得勝、副都統花尚阿各軍圍定遠,陳玉成糾眾來援,會合捻匪撲鳳陽,據九華諸山,連營數十里。城中食且盡,甲三令參將黃國瑞潛率銳卒四百夜薄九華山,躍入壘,城上發砲應之,賊大亂,棄營走,圍乃解。

是年秋,英法聯軍入京師,車駕幸熱河,甲三請率兵入衛,詔以臨淮為南北筦鑰,止勿行。和議定,條上四事,曰:慎採納,節糜費,精訓練,選將才,下所司議行。復疏請還京,泰西諸國欲助兵討賊,甲三力陳非策,皆報聞。十一年,張洛行屯聚渦河北,令李世忠擊走之。

練總苗沛霖者,鳳臺諸生,健猾為閭里雄。以團練功累擢川北道,加布政使銜,然不冠服,令其下稱「先生」。所平賊圩輒置長,收其田租。緣道設關隘,壟斷公私。渦河、澮、潁之間,跋扈自恣。甲三屢羈縻之,用以牽制捻匪。勝保尤信用沛霖,沛霖亦深與結納,內懷反側,憚威不敢猝發。至是藉口其練勇被害,據懷遠,圍壽州,巡撫翁同書為所劫持,殺壽州團練徐立壯;囚孫家泰,亦自盡,而壽州之圍仍不撤;遣其黨茍憬開犯河南,受粵匪封職,令練眾蓄發,四出擾掠。於是詔褫沛霖職,命甲三會諸軍進剿,同書罷去,賈臻代署巡撫,復於潁州被圍。會張洛行大舉渡淮,甲三移軍擊之,洛行敗走。甲三屯長淮衛,解散沛霖屬圩二百餘處。十一月,保恆偕總兵張得勝等克定遠,粵匪遁走,進拔六合、天長。

同治元年,會克江浦、浦口,移軍會多隆阿軍攻廬州,克之。陳玉成走壽州投苗沛霖,執送勝保軍,誅之。於是勝保為沛霖乞恩免罪,責剿捻自效,佯奉命而倔殭如故。甲三策沛霖終為患,疏陳大勢,先剿群捻,次沛霖。薦李續宜撫皖,而自移師會僧格林沁軍擊捻匪,上報可。尋以病劇乞罷,允之。前因壽州失陷,部議革職,特詔寬免。

既受代,行至歸德,疏陳四事,請崇聖學;議政親臣專心國事;用人宜審;聽言宜斷:上嘉納之。復奏苗練終難就撫。二年春,沛霖復叛,圍蒙城,群捻助之,詔甲三在籍會籌防剿。臨淮軍苦饑乏。甲三奉命急籌接濟,乃倡捐募敢死士出間道,運至蒙城。捻匪兩犯陳州,甲三病已亟,榻前授將吏方略,擊走之。尋卒,優詔賜恤,謚端敏。擢其子保恆侍講學士,保齡內閣中書。陳州、臨淮、淮安並建專祠。後淮安請祀名宦,河南請祀鄉賢。

子保恆,字小午,道光三十年進士,選庶吉士,授編修。從父軍中,咸豐五年,詔允留軍差遣。七年,從解亳州圍,拔白龍王廟、寺兒集、雉河集賊壘,進攻三圩,戰最力。勝保以聞,加侍講銜,賜花翎。八年,會攻懷遠捻首李大喜,奪其輜重,又大破孫葵心、劉狗於太和,賜號伊勒圖巴圖魯。九年,甲三罷軍事,保恆回京供職。十年,復命保恆赴甲三軍,破賊定遠,幫辦軍務穆騰阿上其功,甲三力辭,上諭甲三不必引嫌。十一年,破苗沛霖黨張士端於懷遠,會克定遠。同治元年,連擢侍講、侍讀、庶子。甲三以病解職,命保恆仍留軍。尋丁繼母憂,歸。二年,從甲三督治陳州團防。甲三尋卒,恤典推恩,命保恆以侍講學士即補。

淮北初平,保恆疏陳善後八策,請以逆產、絕產募民屯墾,整頓兩淮鹽務,以濟屯田經費;又密陳李世忠驕恣難制,請加裁抑。三年,保恆以屯田議未即行,請詣京與廷臣面議。詔斥不諳體制,下部議降一級,以鴻臚寺少卿候補。服闋赴京,廷臣交薦其才。七年,捻匪犯畿輔,保恆自請效力戎行,命赴李鴻章軍委用。捻平,加三品銜,授侍講學士。從陜甘總督左宗棠赴陜西,八年,命筦西征糧臺,許專摺奏事。十一年,遷詹事。肅州克復,加頭品頂戴。十三年,連擢內閣學士、戶部侍郎。保恆督餉凡五載,諸軍欠餉糾轕,騰挪無缺。及大軍出關,詔襄辦左宗棠轉餉事,進駐肅州。保恆請入覲,未許。光緒元年,召回京,兼署吏部侍郎。二年,調刑部侍郎。

保恆久歷兵間,審於世變,屢上疏論時事,請辨人材,厲士氣,收人心,言甚切直。又言:「歷觀各國情形,惟俄為最強最狡,往往不動聲色,布局於十數年以前,肆毒於十數年以後。履霜有象,桑土宜先。伏原特簡久經戰陣熟習韜略之治兵重臣,專辦東三省練兵事務。凡屬兵馬餉糈邊防之事,悉以屬之。重以事權,寬以歲月,無事則可消覬覦之萌,有事則可為撻伐之助。用以拱衛神京,懾服他族。根本至計,未可委之一二不相統轄之武臣,謂可威強鄰而彌外患也。福建之臺灣,僻處海澨,物產豐饒,民、番逼處。非專駐大臣,鎮以重兵,孚以威信,舉民風、吏治、營制、鄉團,事事實力整頓,未易為功。若以福建巡撫每歲半載駐臺,恐閩中全省之政務,道路懸隔,而轉就拋荒。臺灣甫定之規模,去住無常,而終為具文。請改福建巡撫為臺灣巡撫,駐臺灣,而以總督辦福建全省事,各專責成。」疏入,下部議行。

三年,河南大旱,命保恆襄辦賑務。既至,疏陳沿途流民狀,先令州縣停徵。四年,奏請截留江南漕糧九萬石,不許;請借直隸平糶餘米三萬石,許之。又請借用江蘇義倉積穀及臺灣捐修鐵路洋銀五十萬圓,下部議。令籌歸還之法。保恆請緩禁川鹽行楚,加抽鹽釐,備抵賑需,為兩全之計。疏入,仍下部議。保恆查賑所至,屏絕供張,服食粗糲,刊賑章二十二則頒行,就孔道設粥廠,就食省城者凡十餘萬人,棲息得所。時親視察,感疫病卒,優詔賜恤,謚文誠。河南省城建專祠,附祀陳州、臨淮甲三祠。

毛昶熙,字旭初,河南武陟人。父樹棠,官至戶部侍郎。昶熙,道光二十五年進士,選庶吉士,授檢討,咸豐五年,遷御史,轉給事中。屢上疏論軍事吏治,劾步軍統領聯順徇私廢治,罷之,甚負清望。八年,授順天府丞,胡林翼密疏薦之。十年,加左副都御史銜,命督辦河南團練,至則規畫全局,定條規十二事:築堡寨,扼要隘,擇首事,選團丁,籌公費,互救援,定約束,申號令,公賞罰,詰奸宄,旌忠義,而終之以實力奉行;並疏陳調練民勇苦累之弊,亟宜改辦鄉團,以紓民力。尋命督辦剿匪事宜,駐軍歸德。亳州捻匪犯鹿邑,督練勇擊走之,分路馳剿,九戰皆捷。

十一年,疏言:「捻騎逾萬,官軍馬隊過單,皖、豫交界之區,皆平原曠野,步隊無以制賊死命。今豫境修築寨堡,已有成效,應責令寨長各選壯丁一名、馬一匹,投效來營。歸、陳兩屬,約可得馬隊三四百名。」上命推廣其法行之。捻匪偪省城,圍通許,昶熙檄軍援之,圍立解。因疏言:「軍令不一,將士無所適從,宜會合撫臣以一事權。」上命巡撫嚴樹森督辦河南剿匪事宜,昶熙副之,仍兼辦團練。三月,克唐縣。捻匪趙國良犯光州,陳大喜犯汝陽,並擊走之。尋以誤用逃犯李占標,降三級調用,暫免開缺。大河以南府、、州、縣團練皆成立,屢敗賊,詔開復處分。連擢順天府尹、太僕寺卿、內閣學士,仍留軍。

穆宗即位,昶熙請謁文宗梓宮,面陳機要,未許,命以軍事密疏入告。疏上制捻要策,略曰:「年來剿捻未得要領,其誤有二:一在專言防堵。潁、徐、歸、陳,平原千里,無險可扼,捻數路同發,分而愈多。官軍分堵則兵單,合堵則力疏,猶之院無墻垣,徒守門戶,不能遏盜也。一在無成算而輕戰。賊眾數倍於我,馬則十倍過之。我無必勝之術,僥幸一戰,一旦敗潰,賊焰愈張。至會師搗老巢,實為平賊要策。皖捻雖以張洛行為主,而陳、宋、潁、壽、淮、徐方數百里,無處非賊巢,即無處無賊首。官軍即能次第掃除,勢難刻期凈盡。若繞過小捻,徑搗大捻老巢,舍近攻遠,而近賊襲我於後,我必不支,此會搗老巢之難遽奏效也。然捻匪與粵匪不同,粵匪蜂屯蟻聚,其勢合;捻匪散處各圩,其勢分。其出竄也,必須裝旗糾合各圩賊目,約期會舉,常十餘日始得出。其竄山東者,每會於保安山、龍山;竄汴梁者,會於小奈集、大寺集;竄陳州者,會於南十字河、張信溜:地皆偪近亳州,亳州者,賊之吭也。計莫若擇重臣素有威望者,統步隊數萬、馬隊數千,屯軍於此。用伍員多方誤楚之法,分所部為數起,此歸彼出,此出彼歸,循環馳突於各捻賊圩之間,使大捻無從勾結,小捻聲息不通,惴惴焉日防官兵之至,自不能裝旗出竄,四出打糧。俟其饑困,然後以重兵次第圍剿。賊無外援,則小股膽落,大股易平,招撫兼施,立可解散,不必盡煩兵力矣。夫防賊於既出之後,何如遏賊於未出之先?剿賊於既聚之餘,何如蹙賊以難聚之勢?而又無勞師襲遠之危、輕進損威之失,所謂不戰而屈人之兵者是也。今日大計,以衛畿輔固根本為先。豫東者,畿輔之門戶也。亳州者,豫東之賊源也。亳州之賊不除,則豫東之匪難絕,即畿輔之地不安。重兵駐豫,不能兼顧東省,駐東亦不能兼顧豫防。惟亳為諸捻匯處之區,拔本塞源,實在於此。且蒙、亳百姓,祗以偪處賊巢,呼訴無門,不得不茍全性命,非盡甘心為逆也。若官軍聲勢一振,隨撫隨剿,不但忠義良民同心殺賊,即附賊之堡寨,亦相率就撫,輔助官兵。彼久經兵革之地,人習戰爭,附賊則為悍賊,反正則為勁兵,奪賊焰而益軍威,計無便於此者。前勝保、袁甲三累獲大勝,皆由屯駐亳州,扼其要害,並賴關保、德楞額馬隊之力,是以所向有功。前事不遠,可為券證。」奏入,上韙之。

時粵、捻合擾潁州,命昶熙出境會剿。昶熙兵僅五千,且無馬隊,疏請調總兵李續燾等募精壯六千來豫,以厚兵力,如所請行。上復敕西安將軍托明阿選西安馬隊一千赴豫。

同治元年春,亳捻劉大淵糾黨趨河南,昶熙在省聞警,馳至杞縣,賊已圍城,會僧格林沁軍自山東進至,敗賊許岡,昶熙會所部合擊之,克復所占民圩,斬馘逾萬,餘賊引去。檄諸路團勇截殺之,還駐歸德,扼賊歸路。四月,會同僧軍合擊金樓教匪楊玉驄,盡殲其眾,授禮部侍郎,仍命督團剿賊,歸僧格林沁節制。赴汝寧督兵團剿陳大喜諸匪,克正陽,收寨、圩多處。二年,誅賊首張鳳林、張福林,克邢集、尚店賊巢,陳大喜竄湖北,汝寧、陳州所屬踞賊,殲除殆盡。調吏部。亳捻犯陳州,為官軍所扼擊,四竄。昶熙屯鹿邑,盡平亳北賊寨。

是年冬,苗沛霖伏誅,淮北肅清。詔:「昶熙部勇原助兵力所不足,今兵力足敷應用,飭散遣歸農。」命昶熙回京供職。會陳大喜勾結苗練餘黨趨汝南,陷正陽、信陽、新蔡、息縣各民寨,乃暫留剿賊。三年,進屯息縣,擒誅捻首趙國良、徐文田十餘名,盡復諸寨。十一月,僧格林沁敗陳大喜、張總愚於光山,賊西竄,偪南陽。昶熙調張曜回屯唐縣,知府湯聘珍扼宛南。四年,僧格林沁戰歿曹州,諸軍並被譴,坐革職留任,詔回京。六年,調戶部。七年,擢左都御史,兼署工部尚書。

時捻匪戡定,疏陳軍務漸平,宜益思寅畏,略曰:「功成而喜者,常人之同情;功成而懼者,聖人之遠慮。今日巨寇甫平,兵戈未息,滇、黔、秦、隴,烽火驚心;皖、豫、直、東,瘡痍滿目。戡亂安民,一一尚煩宸慮,敬肆之機,間不容發。萬一大捷之餘,偶忘乾惕,則患機之萌,恐有伏於無形者。今之所急:一在勤聖學。皇上春秋鼎盛,典學日新。但恐親師講學,為時無多,還宮之後,左右近習,或以功業日盛,間進諛詞,意氣漸盈,懋修或懈。昔宋莊獻皇后臨朝,仁宗聽內侍之言,欲觀寶玩,莊獻太后為言祖宗創業之艱。臣亦伏原皇太后於皇上還宮之餘,殷殷以時事艱難,勤加啟迪。至於近侍,尤宜擇老成有識之人,服事起居,將見養正之功,日臻堅定矣。一在崇節儉。今寇亂雖平,而流離之民,未盡歸農,荒蕪之田,尚多未墾。非力加撙節,不足以廣積儲而備緩急。臣前管三庫事務,見內務府借撥部庫銀兩,逐歲加增。竊恐中原底定,踵事增華,財源未開,財流不節,度支告匱,為患匪輕。伏原皇太后、皇上崇尚節儉,為天下先。一切不急之務,可罷則罷之,可緩則緩之,庶國用可充,而風俗亦漸歸質厚矣。一在飭吏治。發、捻之禍,實由不肖州縣所激而成。正供之外,百計誅求;私派私罰,自營囊橐,以致民氣不伸,釀成巨患。用兵以來,此風尤甚。即如釐金一項,奉行不善,百弊叢生。病商病民,莫此為甚。今日之封疆大吏,以地方多事,喜用精明強幹之員,而不求愷悌循良之吏。斯民元氣,剝削愈甚,其禍遂不可勝言。今東南初定,畿甸甫清,兵燹遺黎,不堪再擾。應令各省督撫慎選良吏,與民休息,以復富庶之舊。一在固根本。陜西回逆、土匪,麕聚北山,現聞大軍乘勝西征,恐至窮而思竄。其或由晉省撲河,或由草地北擾宣、大,畿輔兵單地廣,在在須防。直隸提督劉銘傳謀勇兼優,應令迅回本任,並帶所部萬人,留直屯守,以壯聲威。並將綠營兵丁,練成勁旅,庶諸賊不敢萌心北擾,而諸將亦得專意西徵矣。」疏入,上嘉其言剴切,優詔答之。

八年,授工部尚書,命在總理各國事務衙門行走。九年,天津民、教構釁,命偕直隸總督曾國籓按治,暫署三口通商大臣。事定回京,請裁歸總督兼理,從之。十一年,調吏部。十二年,上謁東陵,命留京辦事。十三年,兼翰林院掌院學士。光緒四年,丁母憂,服闋,命仍在總理各國事務衙門行走,兼翰林院掌院學士。八年,授兵部尚書。尋卒,優詔賜恤,贈太子少保,謚文達。

昶熙屢掌文衡,兩典會試,凡朝、殿考試,閱卷歷二十餘次,士論歸之。

論曰:袁甲三、毛昶熙並以謇諤著聲,出膺軍寄。甲三孤軍支拄淮壖,與捻事相終始,驕帥傾排,狡寇反覆,卒能保障巖疆,其堅毅不可及也。昶熙事權未專,同時疆吏非辦賊才,補苴之功,亦不可沒。所陳平捻方略,具得要領。賊平之後,懃懃以寅畏納諫,老成謀國,於斯見之。保恆濟美戎行,立朝侃侃,家英國幹,鬱有風規已。


\end{pinyinscope}