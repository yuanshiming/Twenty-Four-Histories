\article{志一百}

\begin{pinyinscope}
○食貨六

△征榷會計

征榷清興,首除煩苛,設關處所,多仍明制。自海禁開,常關外始建洋關,而釐局之設,洋藥之徵,亦相繼而起。三者皆前代所無,茲列著於篇。至印花稅、煙酒加徵,均試行旋罷,不具載。

常關。順治初,定各省關稅,專差戶部司員督徵。左、右兩翼,張家口稅,差滿官督徵。時京師初定,免各關徵稅一年,並豁免明季稅課虧欠。嗣浙、閩以次蕩平,復禁革明末加增稅額,及各州縣零星落地稅。三年,革明末加增太平府姑溪橋米稅、金柱山商稅。四年,定戶、工各關,兼差滿洲、漢軍、漢官。八年,減定關差員數,並停止關差議敘。九年,並西新關、江寧倉為一差,停獨石口差。嚴關差留用、保家委官之禁。凡額設巡攔,各制號衣、腰牌。

十年,令各關刊示定則,設櫃收稅,不得勒扣火耗、需索陋規,並禁關役包攬報單。十一年,用給事中杜篤祜言,清釐關弊四事:一,裁吏役;一,查稅累;一,關差回避本藉;一,批文核對限期。十六年,移潘桃口於永平,移古北口於密雲,並設關徵木植稅,十分取二。十七年,裁永平、密雲新關,歸並古北口兼督管理。十八年,定各口木植什一而稅。停臨清磚差。其板閘稅交北河分司徵收。

康熙元年,移設河西務於天津,更名天津關。更定各關兼差滿、漢官筆帖式各一,由六部咨送輪掣,停蒙古、漢軍差。其張家、殺虎二口,專差滿、蒙官。二年,定盤詰漕船,止於儀真、瓜州、淮安、濟寧、天津五關。免外國貨物入崇文門稅。四年,嚴禁各關違例徵收,永免溢額議敘之例。五年,命各關稅均交地方官管理。於是崇文門歸治中,天津歸天津道,龍泉等歸井陘道,紫荊歸直隸守道,臨清歸東昌道,挖運歸通薊道,居庸歸昌密道,西新歸鎮江道,蕪湖歸池太道,揚州歸驛傳道,滸墅歸蘇昌道,淮安歸淮海道,北新歸浙江布政使,荊歸荊州同知,九江歸九江道,贛歸吉南贛道,太平橋歸南雄知府,遇仙橋、浛光廠歸韶州知府,各稽徵稅課。又裁古北口差歸密雲縣管理,惟兩翼、張家口、殺虎口如故。只差戶部司員,申令直省關刊示稅則。罷崇文門出京貨物稅。

八年,臨清倉歸並臨清關。以給事中蘇拜言「地方官兼關稅,事務繁多,且恐畏懼上司,希圖足額,派累商民」,復定稅額較多之滸墅、蕪湖、北新、九江、淮安、太平橋、揚州、贛、西新、臨清、天津、鳳陽倉,仍差部員督徵,餘如故。是年定關差缺出,以六部俸深司員輪掣,其差過之員,不準重差。又定關差考覈法:欠稅不足半分者罰俸;半分至四分,分別降調;五分以上革職。旋又定不及半分者降留,全完者紀錄。凡部差官員,不令督撫管轄。

九年,定淮安關兼轄淮安倉及工部清江廠,兩翼專差滿官筆帖式。十年,裁西新戶關歸並龍江工關,裁蕪湖工關歸並蕪湖戶關,各兼理。既而改鳳陽倉歸鳳陽知府,正陽歸通判,臨淮交大使徵收,停差部員。十七年,裁北河分司,臨清閘稅歸濟寧道兼管。十九年,開山東海禁,令查船戶匿稅。差滿部員督收潼關、山海關稅課,潼關兼轄大慶關、龍駒寨稅務。二十一年,移九江關駐湖口,停潼關、山海關部員差,仍歸地方官管理。鳳陽仍差六部滿員。二十三年,更定各關輪差各部院司員例。

是時始開江、浙、閩、廣海禁,於雲山、寧波、漳州、澳門設四海關,關設監督,滿、漢各一筆帖式,期年而代。定海稅則例,免海口內橋津地方抽稅,分設西新、龍江二關課稅專官。二十四年,西新仍歸戶部。免外國貢船稅,減洋船丈抽例十之三。二十五年,定州縣海船隱匿處分。時海禁初開,沿海漁船,州縣既徵漁課,海關復稅梁頭,民甚苦之。上用福建巡撫張仲舉言,定漁船五尺以上,梁頭稅統歸地方官徵收。先是康熙四年罷抽稅溢額議敘例。至十四年,又定溢額多寡,分別加級升用。及是,上以苛取累商,復停止溢額議敘。二十六年,滸墅監督桑額徵收溢額二萬一千有奇,上以擾累閭閻,罪之。永減閩海稅額六千四百兩有奇。二十八年,蠲沿海魚蝦船及民間日用物糊口貿易之稅,著為令。先是沙溝於二十六年歸並淮關,其朦朧、軋東、岔河等處悉免稽查。至是以沙溝系朦朧、軋東總匯,不宜再增一稅,將朦朧歸海關,軋東歸淮關,沙溝免稅。復歸並西新戶關於龍江工關。

三十三年,仍差部員督收山海關稅,張家口稅歸宣化府兼收。三十四年,分設浙海關署於寧波、定海,令監督往來巡視。三十五年,定洋海商船往天津運米至奉天者,但收貨物正稅。三十六年,嚴關差官自京私帶年滿舊役謀占總科庫頭之禁。三十七年,永減粵海關額稅三萬二百兩有奇。三十八年,上恐各關差苛取瘠商,停罷額外盈餘銀。設河寶營,差滿官督收大青山木稅。四十年,裁陜西三原縣商稅,歸潼關、龍駒寨、大慶關兼收。裁通會河分司,通州木廠歸通永道管理。四十一年,大青木稅歸並殺虎口兼轄。

四十六年,以金州、牛莊交山海關監督巡察越關漏稅。設渝關於重慶,歸川東道徵收木稅。四十七年,仍差工部司員督收荊關稅。五十三年,以臨清關稅缺額,改歸巡撫監收。未幾,鳳陽、天津、杭州、荊州、江海、浙海、淮安、板閘及淮關,先後改交各巡撫監收。停瓜州稅,裁稅課大使。定臺灣收泊江、浙等省商船,經過廈門就驗者不重徵。福建糖船至廈門者,赴關納稅,其往江、浙貿易者免徵。設橫城稅口,歸山海關監督監收,增稅千兩,作為定額。六十一年,禁各番部落夾帶硝磺軍器出邊,其進口稅許從輕減。

雍正元年,移湖口關於九江,並設大孤塘分口。裁淮安、北新、鳳陽、天津、臨清、江海、浙海、荊州各關加增贏餘銀。嚴禁各省關及崇文門胥役分外苛求。是年定各關稅務俱交地方官管理,惟崇文門仍差內務府官;山海關、兩翼,古北、潘桃、殺虎三口,暨打箭爐,仍差部員。盛京呼爾哈河木稅,亦交將軍、府尹委沿河官徵收。明年,淮安仍差部員,滸墅改歸蘇州織造,鳳凰城中江稅,派盛京部員各督收。河西務運糧船料,改於通州徵收。三年,以暹羅進獻稻種果樹等物,免回空壓載貨物稅。禁邊關城門索取蒙古貢物稅,其假名匿稅者罪之。五年,宿遷關歸並淮關徵收,由徬稅交地方官管理。河寶營木稅,由殺虎口監督徵收。奉天牛馬稅,改差部院司員。

六年,更定臨清關米麥雜糧船稅。定各關稅則。龍江、西新二關,交江寧織造兼管。永免暹羅米稅。七年,夔州關改委專員督收。南北二新關交杭州織造兼收。移荊州之徐關於田家洲,更名田關。江蘇廟灣稅歸淮關兼管。定閩海關減折船稅丈尺例。裁古北口監督,交密雲縣徵收。以潼關商稅浮於部例,相安已久,照現徵之數,著為令。移潘桃口正關於潘家、桃林二分口徵收。八年,減各關餘平銀之半,革除天津戥耗例外徵收。定落地稅搜求溢額議處例。嚴黃金出洋之禁。十年,設交城縣水泉灘木廠,武元城設立稅口徵收。十一年,改天津關歸長蘆鹽政管理。十三年,設居庸關稅課大使,定潘桃、古北、殺虎三口給商印票,兼滿、漢、蒙三體文字。山東海口各州、縣、衛設兩聯印票,填註客商年貌籍貫、船隻字號、梁頭丈尺、豆石數目、出口年月,分給商船,回日查銷。

乾隆元年,革除龍江、西新二關衙規票銀。初,外洋夾板船到粵,起其砲位,候交易事畢給還。其稅法每船按梁頭徵銀二千兩左右,貨稅照則徵收。革除額稅外另徵置貨銀加一繳送稅。定閩省漁船稅,分上、中、下三則起科,除額外重徵。定各省稅課則例頒行。定九江、贛州二關三聯稅單例,一給商人,一交撫署,一存稅署。準張家口、居庸關收取車馱貨物過稅飯錢,以資養贍。禁偷運米穀接濟外洋,分別擬罪有差。免沿海採捕魚蝦單桅船稅。二年,定米穀稅,凡遇地方旱澇,米穀船到即放行,俟成熟後照舊徵收。永停徵廣東開建、恩平二縣米船稅。三年,裁滸墅關之轉水、柏瀆二口,改瓜州由閘稅歸兩淮鹽政。九江差內務府司員,蕪湖、鳳陽派部員,各管理監督。四年,定歸化城木稅額,歸殺虎口徵收。五年,復差部員監督荊關。用御史陸尹耀言,嚴捏名討關之禁。

六年,復定考覈關稅贏餘例,清查外省私增口岸。免領帑採銅錫鉛及米穀稅,仍徵船料,惟黃豆非麥秫比,雖歉歲照常徵收。改宿遷之豐、沛、蕭、碭四縣陸稅,仍交各縣分徵。永禁龍江關木稅飛量法。定各關贏餘,比較上年數目考核,著為令。七年,永免直省關豆米額稅。復設通州分司之黃村,臨清關之德州、魏家灣、尖塚、樊口等口岸。免徵臨清關船料。以揚州關歸兩江總督遴員徵收。停止閩海關之南山邊口徵稅,專司稽查。八年,定官運米穀免徵船料。九年,嚴蒙古來京漏稅,及為奸商私運貨物之禁。

十年,交阯亂平,復開徵雲南馬臼稅。禁止宿遷關通船一載收稅例,改按擔數徵收。定一官兼管兩關,其徵額有此贏彼絀者,準其抵補;再有短歉,仍著追賠。移福建詔安之雅溪稅館於懸鐘,以閩省舊有子頭船包攬走私,永禁制造。十三年,復徵米豆稅。十四年,定各關贏餘,以雍正十三年為準,短少者按分數分別議處,罰俸降調有差。十五年,移福建寧德縣稅口於酒嶼。十七年,改渝關木稅歸並夔關徵收,十八年,移廈門查稅之玉洲館駐石美,鳳陽關查稅之濉陽口駐虹縣,改虹縣徵稅之青陽鎮駐濉河口。二十年,移淮南關之流均口駐涇河。

二十二年,增定浙、閩二海關稅則,照粵海關例。尋又申禁洋船不準收泊浙海,有駛至者,乃令回粵貿易納稅。二十四年,定葉爾羌、喀什噶爾牲畜稅二十取一,緞布皮張稅十之一,自外番販入者倍徵,嚴絲斤出洋之禁。二十五年,始派員徵收多倫諾爾皮張等稅,並設盛京拉林、阿勒楚喀稅局,派員徵收,如寧古塔、伯都訥例。革除粵海關陋規銀,歸公造報。二十六年,設淮安關石稅口,又設歸化城總稅局,並綏遠、歸化、和林格爾、托克托、薩拉齊、西包頭、昆都侖、八十家子等口,差蒙古筆帖式二員,分督徵收牲畜稅。

二十七年,以龍江、淮安二關歸兩江總督,滸墅歸江蘇巡撫,各稽查嚴禁榷關漏稅積弊,並定漏稅罰數。江蘇巡撫陳宏謀條上滸墅關四弊:一,鋪戶代客完稅,包攬居奇,仍令商人自行完納,按簿親填;一,貨船抵關,簽驗納稅,給票後始準過關,以杜偷越;一,官員遴委佐雜官,半年而代;一,督撫與監督原相助為理,所徵數目,應令監督按月知會督撫,再於年滿奏報時統咨知會。從之。是年弛絲斤出洋之禁,仍示限制。定崇文門、兩翼稅差期滿,由部開列滿、蒙大學士、尚書、都統、侍郎、副都統等職名,請簡更代,遂為永制。二十八年,畫一天津各口稅則。定商販山東豆石由海運浙,照運赴江南例輸稅。張家口出口鐵器,照殺虎口例納稅。革除蕪湖關之戶、工幫貼飯費,江海關之駁票給單掛號、油燭飯費、看驗艙錢文、揚關由閘之給串錢。

二十九年,更定臨清關船隻補稅例。定外番商貨至回部貿易者,三十抽一,皮貨二十抽一;回商往外番貿易,二十抽一,皮貨十之一;其牲畜貨物不及抽分之數,視所值折算。三十年,更定吉林等處稅額,裁潘桃口監督稅歸張家口徵收,所屬六小口,改歸通永道管理。明年,復改潘桃口稅歸多倫諾爾同知徵收。設局大河口,差理籓院司員督收歸化城稅。既而改歸山西巡撫遴員徵收。岫巖城屬之鮑家馬頭等七口岸海船商稅,歸山海關監督設局徵收。三十三年,定山海關、張家口、八溝、塔子溝、三座塔、烏蘭哈達、多倫諾爾交直隸總督,殺虎口、歸化城交山西巡撫,盛京牛馬稅、中江稅交盛京戶部侍郎,坐糧交倉場侍郎,打箭爐交四川總督,荊關交湖廣總督,均兼管稽差。各監督有侵蝕、情弊、參處後不能完項者,即令兼管之員代賠。三十四年,準九江關正稅一兩加平餘一分,以供飯食費需之用。停洋船入口夾帶硫磺之禁,著為令。三十五年,裁潯、梧二廠公費歸入正稅。

三十八年,裁多倫諾爾監督,歸多倫諾爾同知管理。移由閘、南壩稅口於中閘。四十年,封閉廣西由村溢口,禁內地商民越關交易。四十一年,改通州分司及河西務計價科稅為計數科稅,並革除張家灣油面等出店進店稅。改定打箭爐商貨按數徵稅例。明年,定打箭爐稅差,照山海關例,於宗人府及部院司員內選派。四十五年,停荊關、打箭爐司員差,交各督撫遴員管理。四十六年,裁荊關監督養廉銀,於荊宜施道、荊州知府遴派一員監收。四十九年,定粵海關珍珠寶石概不徵稅,著為令。五十一年,裁荊州之郝關及郝支關,另設口於越市,更名越關。移楊關於調賢口,更名調關。定除暹羅貢使船外,其帶貨私船,照例徵收。

五十二年,定各關預期請領收稅冊檔,及請領遲延,擅用本關簿冊參處例。以安南奉貢請封,弛水口等關之禁。越四年,緬甸效順,亦準開關通市,於永昌、騰越、順寧收徵出口稅,杉木籠、暮福、南河口徵收入口稅。以福建五虎門與臺彎淡水八里岔設口開渡,由閩安鎮徵收進口稅,南臺口徵收出口稅。貨物進口,復運往他處,限一月內免重徵;若逾限出口,或限內移貨別船,均徵出口稅。

五十七年,定粵海關到關船貨,責成督撫查明,按月冊咨。一年期滿,與監督清冊覈對不符,參辦。五十八年,定西洋除貢船外,別項商船不得免徵。以杭州織造改歸鹽政,南北二新關交巡撫管理。開山西得勝口歸殺虎口監督稽徵。時英吉利貨船求往江、浙寧波、珠山及天津、廣東等處收泊交易,上不許,仍令照例於澳門互市,向粵海關納稅,並徵船料。

嘉慶二年,並左、右翼為一差。越二年,復簡派二員。定辰關、渝關、潘家口、通永道、古北口五處各關例。是年命覈減各關贏餘額數,於是定戶關之坐糧六千兩,天津二萬,臨清一萬一千,江海四萬二千,滸墅二十三萬五千,淮安十一萬一千,海關廟灣口二千二百,揚州六萬八千,西新二萬九千,九江三十四萬七千八百,贛關三萬八千,閩海十一萬三千,浙海三萬九千,北新六萬五千,武昌一萬二千,夔關十一萬,粵海八十五萬五千五百,太平七萬五千五百,梧州七千五百,潯州五千二百,歸化城一千六百,山海關四萬九千四百八十七,殺虎口一萬五千四百十四,張家口四萬五百六十一,打箭爐侭收侭解;工關之辰關三千八百兩,宿遷七千八百,蕪湖四萬七千,龍江五萬五千,荊關一萬三千,通永道三千九百;閘、南新、渝三關,潘桃、古北、殺虎三口,竹木稅向無贏餘,無庸更議。

五年,議準回空漕船於六十石例額外夾帶二十石,均免輸稅。嚴禁崇文門、盧溝橋及各省關役訛索行旅。以辰州知府李大(隆)接管稅額外贏餘萬兩有奇,下部議敘。六年,定盛京牛馬稅差,於盛京五部侍郎內簡派。定打箭爐正稅額二萬兩。革除閩海徵收二八添平銀。七年,改密雲縣徵收古北口木稅為侭收侭解,並繳銷原額監督關防。九年,復增定各關贏餘額數,浙海四萬四千,西新三萬三千,九江三十六萬七千,滸墅二十五萬,淮安十三萬一千。十一年,定辰關歲徵加一耗銀二千七百七十餘兩。十五年,定崇文門蓡稅則例。令營汛官分查崇文門私放私收冒充白役之弊。二十二年,飭各海關查禁例不出洋之貨。

道光元年,裁浙江鹽政,改設杭州織造,兼管南北新關稅務。三年,飭各省關整頓奸蠹包攬、書吏徇縱等積弊,嚴各關員例外橫徵及糧船夾帶偷漏之禁。定多倫諾爾木稅。更定浙海關稅則。九年,申定回疆稅課三十抽一。時英吉利大班等以洋行閉歇,拖欠貨銀,商船停泊外洋,延不進口。每言在粵海關年納稅銀六七十萬,以為居奇。上曰:「洋商私帶鴉片入口,偷買紋銀出洋,得不償失。倘故刁難,即不準開艙。少此一國貨稅,所損幾何!至請分別商船大小納餉,尚可變通。」

十年,定各關盈餘銀以六成為額內,四成為額外,覈其溢額絀額分別功過例。先是御史許乃濟言崇文門稅局需索,曾令巡視五城御史隨時稽查。至是,御史晉昌復言巡役勒索,胥吏賣放,特派滿、漢御史各一,專司稽查,一年而代。十一年,減滸墅盈餘二萬兩,淮安二萬一千兩。定賠繳短徵關稅,按數多寡分別限期久暫例。命廣東嚴緝快蟹船為洋商運私偷稅。十二年,停止白鉛出洋。十三年,革除各關標禮並查船謝儀,及地棍報單等名目。以霍罕悔罪輸誠,復準入卡貿易,並免稅課。十四年,嚴禁各關家丁需索賣放,及書役盤踞、地棍包攬之弊。又查禁粵商增收洋商私稅。定貢物到京,崇文門免稅驗放。

十七年,嚴禁紋銀出洋。查辦粵省匪艇及窯口走私漏稅。十九年,設韶州、東江二關,歸南韶連道管理。二十一年,移設荊州正關於柳家集,更名柳關,並改支關為柳支關。二十四年,免暹羅接正貢使船貨稅。二十五年,裁龍江關查驗木植稅局。

咸豐二年,查禁沿海各關走私積弊。三年,以捻匪擾江南,滸墅、淮安、蕪湖、鳳陽等關,紛請侭徵侭收,漫無限制,令仍遵定額照常徵收。六年,定打箭爐稅額二萬兩。八年,定盛京盈餘稅以錢抵銀,及漁船、大小牛船交納船規例。九年,設山東煙臺稅局。十年,以士子會試入京,照例驗放,嚴禁崇文門巡役訛詐。更定奉天海口稅則,增收黃豆、豆餅、包頭、油簍四稅,加贏餘八萬兩。又定各關監督未及一年離任者,交後任接徵,扣足一年分晰匯報例。革除北新關南北二口貨稅過關五日十日之限。是年,俄羅斯於黑龍江互市免稅課。

同治二年,免巴爾楚克過稅,加徵葉爾羌正稅。三年,設福建臺南之打狗口海關,歸巡撫管理。暫停北新關徵稅。四年,暫停龍江、西新關、滸墅三關徵稅。湖北新關竹木稅,遴本省道府一員督徵。先是粵海關額徵,常洋不分。至是,定貨由華船裝運者為常稅,額徵五萬六千五百餘兩,贏餘十萬兩;再有贏餘,侭徵侭解。是年裁革太平關文武各署規費,並飭粵海關嚴查各口偷漏隱匿。裁山海關監督,改設奉錦山海關道稽徵。七年,定太平關歸南韶連道專管,其四分廠委員,仍由巡撫遴派。八年,申定貢物解京,崇文門放行,毋許留難勒索。十一年,停江蘇淮關傳辦活計。

光緒二年,復開蕪湖、鳳陽兩關。三年,嚴定考核各關章程。四年,定輝發、穆欽等處及寧古塔、三姓稅務,均由吉林將軍委員徵收。山西交城縣木稅,由知縣設口於武元城故交村徵收。八年,定蕪湖關稅額十三萬六千餘兩。九年,中江稅務改歸東邊道徵收。十三年,改廣東黃江廠稅委員專管,裁廠書、簽子、官房、總散房名目,並革除額外加平、辦用官錢、釐頭、船錢、墟艇錢、黑錢、包攬錢七項陋規,榜示通衢。定梧、潯二廠贏餘六萬兩。改滬尾、打狗兩關歸臺灣巡撫監督。二十五年,敕各將軍、督撫綜核各關卡陋規中飽之數,酌量歸公,勒限稟報。三十四年,減崇文門華商稅為值百抽三,如洋商稅例;免日食蔬菜等物稅。宣統元年,設立吉林省稅務處,分設稽徵、庶務、支應、核銷四所,所有捐稅各局、所、公司概行裁撤歸並。更定四川常關徵收章程及辦事規則。

洋關之設,自五口通商始。前此雖有洋商來粵貿易,惟遵章向常關納稅而已。道光十九年,有躉船繳煙之役。是秋各商船來粵者,皆為英兵船所阻,不得入口。粵海稅課,以洋貨為大宗,至是徵收短絀。二十二年秋,英人要求通商口岸,允於沿海廣州、福州、廈門、寧波、上海五口開埠通商。明年,定洋貨稅則值百徵五,先於廣州、上海開市。洋貨進口,按則輸納。後由華商運入內地,所過稅關,只照估價若干,每兩加稅不過某分。

二十四年,定法商條約:一,允法人赴五口通商船隻,不得進別口及沿海岸私行交易,違者貨沒官;一,法商出入五口,照則輸貨稅船鈔外,不再收別項規費;一,商船進口,二日不繳船牌貨單,由領事照會海關者,每逾一日罰洋五十元,但不得過二百元,倘未領海關牌照,擅自開艙卸貨,罰銀五百元,貨並沒官;一,船進口未卸貨,在二日內可往別口,即在彼口納稅;一,船進口二日外全完船鈔,百五十噸以上噸納銀五錢,以下噸納一錢;一,估價之貨有損壞者,得核減稅銀;一,船進口按卸貨之多寡輸納,餘貨如帶往別口卸賣,即在彼口輸稅。二十五年,定比利時商約,照章納稅輸鈔。二十七年,定瑞典那威商約,稅鈔亦如之。

咸豐四年,設江海關於上海。八年,復定英約:一,牛莊、臺灣、登州、潮州、瓊州等口,均準開埠通商;一,值百抽五之貨,多有價值漸減者,應將舊則重修,此次新定稅則,如欲重修,以十年為限,須先六月知照,否則照前章完納,復俟十年;一,子口稅按值百抽二五,如原一次輸納,洋貨在進口、土貨在經過第一關納稅給票後,他口不再徵;一,英船納鈔給照後,四月內不重徵;一,貨船進口二日,即全納鈔;一,英商自用艇,如帶例應納稅之貨,每四月納鈔一次;一,商貨納稅後,改運他口,系原包,免重徵。是年,允法商于潮州、瓊州、臺灣之淡水、登州、江寧通市,納稅輸鈔均同有約國。

九年,設粵海關於廣州。允俄人於上海、寧波、福州、廈門、廣州、臺灣、瓊州七口通商,稅則視各國例。定美約亦如之,並允於潮州、臺灣兩口開市,照新章完納稅鈔。十年,設潮海關於汕頭。允英人於漢口、九江通商。以英人李泰國為總稅務司,幫司各口稅務。設天津、牛莊、登州三口通商大臣。十一年,設浙海關,歸寧紹臺道監督,津海關歸通商大臣統轄。並設閩海、鎮江、九江三關。定各國洋稅自上年八月始,每三月結報一次,四結奏銷一次。英、美二國於九江、漢口開埠,俄亦於漢口通商,於是定長江及各口通商章程。洋貨入江,於上海納正稅及子口稅;土貨出口,納出口稅;復進口時,完一正稅,準扣二成;若完半稅,不扣二成,再入內地,仍照納稅釐。又定德商約,其稅約與英同。

同治元年,設廈門關。以五口商務歸通商大臣兼理。二年,設東海、臺南、淡水三關。免英租界洋貨釐金,並準添開宜昌、蕪湖、溫州、北海四口岸,其沿江之大通、安慶、湖口、武穴、陸溪口、沙市,均準英輪船暫時停泊,用民船上下貨物。除洋貨半稅單照章查驗外,土貨只準上船,不準卸賣。又英商自置土貨,非運出海口,不得援子口半稅例。是年定丹麥及荷蘭商約,輸納稅鈔如英例。三年,設山海關於牛莊。定日斯巴尼亞稅則,視咸豐八年各國例。明年,定比約,稅鈔亦如之。又改定法船鈔章程:凡商船進口已納稅,往他口,並往來安南之法國各埠,與附近之日本碼頭,由海關給照,逾四月再納鈔。初粵海關稅常、洋不分,至是始定由洋船裝運者為洋稅。五年,定義商約稅鈔,商船入口漏捏者,罰船主五百兩,餘如法約。

八年,定奧商稅鈔,均視義約。又定俄商約:一,邊界百里內及往蒙古各盟貿易者,不納稅;一,俄商運貨至天津,納進口稅減三之一,其酌留張家口之貨納正稅,如再運赴通州、天津,不再徵,並將張家口多納之一分補還;一,由天津運俄貨至各口,須補足減一之稅,他口不再徵,如由他口復入內地,另納子口半稅;一,運土貨及洋貨由水路進口,納稅視各國例;一,在天津、通州運土貨由陸路返國,照例納正稅,不再徵,沿途不得銷賣;一,在津運復進口土貨由陸返國,納稅後,限一年起運,不再徵,並給還復進口稅,沿途不得銷賣;一,在津或他口運別國貨由陸返國,已交正稅子稅,不再徵,如只交正稅,應補交子稅;一,議定稅章,試行五年,限滿欲修改,先六月照會。九年,設江漢關。裁三口通商大臣,東海、山海二關均歸直隸總督統轄,另設津海關道,監督新、鈔兩關。

光緒三年,設蕪湖、宜昌二關,歸徽寧池太廣道、荊宜施道各監督;瓊海、北海二關,歸粵海關兼理。又設甌海關於溫州。六年,續定德商約:一,中國允除宜昌、蕪湖、溫州、北海前已添開岸並沿江之大通、安慶、湖口、武穴、陸溪口、沙市前已作為上下客貨之處外,又允德船於吳淞口停泊,上下貨物;一,夾板進口,停泊十四日,應納減半之鈔;一,船貨報關有漏捏,應罰船主,不得過五百兩;一,德商運土煤出口,噸納正稅三錢;一,無照冒充引水者,罰銀不得過百兩;一,船只損壞,準在各口修理,飾詞偷漏,罰倍圖免噸鈔之數;一,中船掛德旗而德人知情,與德船掛中旗而貨主知情,貨均沒官。是年定美商約,稅鈔視各國例。

七年,設嘉峪關,歸安肅道監督。改定俄陸路商約:一,俄貨至嘉峪關,照天津關例,納三分減一之稅,再運內地,納稅亦視天津例;一,貨至天津與原照不符者,沒官,查僅繞越避查驗者,罰令完一正稅;一,在通州運土貨回國,完出口正稅,在張家口運回,暨在內地運土貨至通州、張家口回國者,均納子口稅,沿途不得售賣。餘同前約。

十二年,復定法商約:一,中國準於北圻界擇開兩處通商,設關徵稅;一,洋貨入雲南、廣西兩邊關,納減半正稅三之一;一,洋貨入此關納稅,轉往彼關者,三十六月內不再徵,如轉入各口,另納正稅,土貨在此關納稅復轉彼關,只徵復進口稅,如轉入各口,另徵正稅,入內地仍納子口稅;一,進出口貨到關逾十八時不報驗,日罰五十兩,惟不得過二百兩,報有漏捏,貨並沒官。餘同前約。十三年,允法人於廣西之龍州、雲南之蒙自及蠻耗,開埠通商,並減洋貨進口稅十之三,出口稅十之四。尋改蠻耗為河內,並添雲南之思茅口岸。由通商各口運土貨前往四口時,徵出口十成正稅,到四口照十分減四徵復進口半稅。又定葡約,其稅鈔及罰例均視上年法約。是年設拱北關於澳門,九龍關於香港,由粵海關監督。改臺南、淡水兩稅歸臺灣巡撫監督。十五年,設鎮南、蒙自二關。十六年,設重慶關。

二十年,開西藏之亞東關,允英通商。除禁運貨物外,自開關日始,免進口稅五年。限滿再定稅章照納。又允由蠻允、盞西兩路販運各貨,限六年內減進口稅十之四。二十一年,設思茅關及猛烈、易武二分關,歸思茅同知兼理。

二十二年,定日本商約:一,進出口貨視各國例,只輸進口或出口稅;一,已進口貨再運各口,不論貨主及運貨系何國人及何國船,所有鈔稅釐金雜派各項一概豁免;一,運貨入內地,再納子口稅,系免稅者,按值百抽二五;一,出口土貨,完正稅子稅後,限十二月運往外國,如系禁運出外洋之物,出口時只完正稅;一,洋貨已完進口稅,三年內復運出口,不再徵;一,船鈔視各國例。是年設杭州、蘇州及沙市三關。明年,設梧州、三水二關,並甘竹、江門二分關。改定英人長江通商章程:一,在長江貿易輪船,由上海稅務司給專照,年換一次,或在漢口及宜昌換領亦可,船鈔在給照之關交納,違者照罰,再犯繳銷專照;一,撤銷出口正稅復進口半稅同時完納之例,有專照江輪,出口及復進口稅照各口例,在裝貨起貨之口分次完納,至裝貨撥貨卸貨,亦如各口例。

二十四年,設嶽州關及江海之吳淞分關。明年,設膠州關。與德會定徵稅辦法:一,青海設關,應揀派德人充稅務司;一,海運進口之貨不徵稅,若膠州界口運赴內地,徵進口稅,惟無海關準單不準出膠州界;一,土貨陸運入租界,再水運他口,徵出口稅,惟租界內產土貨並土產,及海運入口之物料制成各貨,出口時不徵;一,土貨進口復運內地,照約納稅;一,土貨納出口稅,復運他口納半稅。又定韓暨墨國稅鈔及各費,悉視海關例。是年設金陵關。又設福海關於三澳。二十六年,設騰越關及蠻允、弄璋二分關。二十七年,定常關距口岸在五十里內,稅由洋關兼徵。二十八年,設秦皇島分關。

先是商約大臣盛宣懷、聶緝椝等言,稅務司赫德籌擬洋貨進口稅,援照洋藥稅釐並徵之法,核估時值,按正稅子口稅七二五,統加釐金一倍,為值百抽十五,由海關並徵,以免各處釐局留難紛雜,貨可暢銷,洋商或可允從。並擬出口土貨向完半稅者,改完釐金,以抵洋貨釐捐改歸海關並徵之數,於各省釐金亦無所損。上以此事利害出入關系甚大,下南北洋大臣、各督撫參酌各省情形,妥議具陳。至是,始與英定裁釐加稅之約:一,約款照行時,中國允除現有各常關外,向設各釐卡及抽類似釐捐之關,概行裁撤;一,英國允於進口洋貨增至切實值百抽五加一額外倍半之稅,以抵撤釐金子口稅及各項稅捐,至土貨出口稅總數,不得逾值百抽七五之數;一,現有常關仍舊存留,其有海關而無常關,及沿海沿邊非通商口岸處,均可添設常關,如新開口岸應設海關者,可並設常關;一,民帆各船運貨所納出入口稅,不得少於輪船進口正稅及添加稅之總數,土貨運出至第一常關,照海關例徵出口加稅,給照單,限一年內無論經何關出口,不再徵,如運出各租界外銷售,應納銷場稅;一,土貨運出,除正稅外,加徵半稅,以為裁撤釐捐之抵補,至絲斤出口正稅,不得逾值百抽五之數;一,向不出洋之貨,於銷售處徵銷場稅,凡民船運至口岸之土貨將銷售本地者,無論貨主何國,均徵銷場稅,惟不在租界內徵收;一,華洋各商在內地用機器紡織之紗布,只納出場稅,餘概豁免,凡機器織成類似之洋貨視此,惟漢陽大冶鐵廠,及國有免稅各廠,與後設之制造局、船澳等廠所出物件,不在此例。尋與美、日、大西洋各國均定此約,卒以事費調查,迄未能實行也。

二十九年,與俄協定北滿稅關試辦章程:一,鐵路運貨減三之一納稅,指定界限,按車站大小,四面各距十里或五里三里不等,如運出指定界限外,應補足正稅,並照運貨入內地章程辦理;一,鐵路運貨減價,此中俄特約,除俄貨外,各國貨經東省鐵路運入者準此;一,章程稅項有應更改者,俟一年再商定。又定通商進口善後章程:一,進口洋貨稅則不載者,照值百抽五例,按市價估貨,以市平合足關平,並扣除使費,方為貨物實價;一,貨未報關已售於華商,即視合同價值之總數為市價;一,由海關估定之價與該商不合,即由海關與該商本國領事,並領袖領事,各派一人公同斷定,若查出該商所報每百少至二十四兩,按估定價值徵正稅,並按所報應完之正稅罰繳四倍;一,洋船專載免稅之米糧等仍稅鈔。是年設澳門分關。

三十年,與德會定青島設關徵稅辦法,附件一。無論華洋輪船,行駛內港,應領關牌,一年而易。初次納牌費十兩,換領只納二兩,每四年納鈔一次。明年,與德修改青島徵稅辦法:一,改青島口岸,概行免稅,惟擇定稅界內一區為無稅地,餘均起徵;一,無稅區外制成各貨,出口納稅,不得逾運原料應完之稅數。改三水之江門為正關。三十三年,設南寧、大連二關,又設安東關及大東溝分關。三十四年,設濱江關及滿洲里、綏芬河二分關。宣統元年,設璦琿、三姓二分關。二年,設琿春關及延吉分關。三年,更定東海關各口稅則為值百抽二五,再收一二五內地捐,所有規費概行裁免。

自光緒二十二年裁撤臺南、淡水、漢城各關外,為關二十七。宣統三年,續增南寧、梧州、三水、岳州、福海、吳淞、金陵、膠海、騰越、江門、安東、大東溝、大連、濱江、滿洲里、綏芬河、璦琿、三姓、琿春、延吉等,為關四十七。

先是土藥各稅列入進口。同治十二年始列專款,合計洋關歲徵各稅。咸豐末年,只四百九十餘萬。同治末年,增至千一百四十餘萬。光緒十三年,兼徵洋藥釐金,增為二千五十餘萬。三十四年,增至三千二百九十餘萬。宣統末年,都三千六百十七萬有奇,為歲入大宗云。

釐金抽捐,創始揚州一隅,後遂推行全國。咸豐三年,刑部右侍郎雷以諴治軍揚州,始於仙女廟等鎮創辦釐捐。是年蘇、常疊陷,丁、漕無收,乃設釐局於上海,藉資接濟;又設江北釐捐,歸大營糧臺經理。五年,江西設六十五局卡,湖北設四百八十餘局卡,湖南亦設城內外總分各局,江蘇揚、常、鎮各府屬添設小河口、普安、新港、三江營、荷花池五局。御史宗稷辰言:「大江南北設卡過多,收捐太雜。」刑部左侍郎羅惇衍亦言:「泰州仙女廟釐局官紳弁兵,刁難勒索。」上令酌量裁並,嚴禁查辦。

六年,盛京抽收商貨及糧石捐,值百取一,吉林亦如之。烏魯木齊之吐魯番亦抽收棉花釐金。七年,設湖北釐金總局。八年,定豫省釐捐除水煙、藥材、茶葉外,餘概不抽收,並裁撤陜州、荊子關及沿河各局卡。是年福建、廣西均設局卡,抽收貨釐。九年,登、萊、青三府屬海口設局抽釐。山西設籌餉局,收行商藥稅及百貨釐捐,於各隘口設七總卡及各分卡。十年,以張家口辦理釐金不善,激成事變,文武各員俱獲嚴譴。兩江總督曾國籓以湘軍援鄂,請於長沙設東征局,克復一處,即酌添局卡,以濟軍儲。凡貨物皆於本省釐金外加抽半釐。允之。是時江北八里鋪及廣東韶關、肇慶府俱設局卡抽釐。十一年,改山西行商藥釐為稅。安徽抽收釐金,設立正卡,省局所屬四,皖南及淮北局屬各三,並設分卡分巡五十九。貴州亦設貨釐局於川、楚鄰近之區。時各省釐局過多,上恐有累商民,命除各省通衢要口外,其餘局卡概行裁撤。

同治元年,以廣東官紳辦理釐捐,營私病民,特命三品京堂晏端書駐扎韶關,督辦廣東釐金。四川總督駱秉章亦以粵省釐捐積弊為言。上誡端書以「釐捐原出於不得已,總期有益軍餉。無戾民情」。御史丁紹周言:「釐捐各委員徒事中飽,民怨沸騰。」命裁革各委員,統歸地方官管理。其通都大邑釐捐事繁,著派道府等官辦理,並照部章分晰開載,榜示通衢。是年設江寧大勝關釐卡。河南禹州、陜州暨河內縣、清化鎮均設藥材釐金分局,禹州並抽收百貨。移設衢州府牙釐總局於浙江省城。除杭州、金華、嚴州三府外,餘八府均設分局分卡。設周口、三河尖兩釐稅局。

二年,幫辦揚州軍務漢軍統領富明阿言:「里下河一帶,南北糧臺設立捐卡百餘處。有一處而設數卡,一卡而分數局。每月局用少者二百金,多者至千餘金。委員既繁,局費尤濫。」上以江北如此,他省可知,嚴飭各督撫歸並裁革,遴委賢能地方官經理。尋湖北巡撫嚴樹森言:「胡林翼創辦湖北釐金,仿劉晏用士類不用吏胥之法,歷久著有成效。若改歸地方官,諸多窒礙。」並臚陳八弊,請仍舊章。又言:「湖北釐金年收百三十四萬,全賴分設小局,稽查偷漏,大局之徵收始旺,零卡勢難議裁。且以一省之財力,協濟數省軍餉,多藉資釐金,輕議更張,恐入款頓減。」均允之。是年江北設釐捐總局,裁並各卡,留存大勝關等二十六卡。江蘇亦設牙釐捐總局,裁並各卡,留存蘇城等十四卡。浙江定百貨釐捐值百抽九,浙東兩起兩驗,間卡抽收,貨值千文,起卡抽三十,驗卡減半,捐足兩起兩驗不重徵。浙西則一起一驗,由第一卡並徵,餘皆驗放。

三年,直隸設天津雙廟卡。淮南亦設卡抽收鄰私釐金。浙江定絲斤捐。河南以捻匪肆擾,停止禹州釐捐,尋復之。時湖廣總督官文言:「直隸、山東、山西、河南、陜、甘、雲、貴、廣西等省釐金不多,軍務告竣,即可議撤。其餘東南各省釐金,不可驟裁,留作善後之費。」曾國籓則以江寧克復,請停廣東釐金。上恐餉項不繼,未之許也。四年,撤湖南東征局,改江北總局為金陵釐捐總局。福建設稅釐總局,徵收百貨及茶釐。六年,湖北裁存釐局分卡八十六。湖南合並分局,統名釐金鹽茶總局。七年,定釐金報部,照兩淮鹽釐排式,年分兩次。時軍務漸平,督撫、臺諫屢以裁撤釐金為言。上飭各省酌留大宗,裁去零星分局。於是湖北又裁去五十四局卡,浙江裁並十六卡。

八年,甘肅開百貨及鹽茶釐捐。定廣東省城及佛山、江門、陳村各繁盛處所,補抽百貨坐釐,由商承辦。九年,廣西減釐,改徵西稅。十年,用御史黃槐森言,禁革廣東釐局幫費名目,並裁汰吏胥。直隸改天津府捐輸義館為百貨釐捐局,設東河、西河、南河、海河四分卡,並於東關設洋藥釐捐局。十三年,停止山海關之臨榆縣釐局。雲南省城設牙釐總局,各府屬設分局二十三,及各井鹽釐局。

光緒元年,浙江復裁並十四卡,存留六十五卡。免湖北米穀釐金。二年,安徽規復,蕪湖、鳳陽兩關分別裁撤釐卡,永免湖南境內運售米穀釐金,販運出境者,仍於首卡完釐一次。三年,山西大祲,商貨滯銷,裁並各路添設之分卡。吉林於雙城堡、農安城抽收七釐貨捐。四年,貴州貨釐減收二成五。七年,給事中劉瑞祺言釐捐無裨國計,飭各督撫酌量截留。山西以釐金減收,復設各分卡。八年,江、揚裁撤分卡一、巡卡二,滬釐局裁並布貨捐局,閩、廣三幫雜貨捐局暨東溝四釐卡,並撤古山、水橋巡卡。明年,滬局又裁東溝、大涇兩巡卡。十年,陜西裁留二十八卡。十三年,貴州增設二十五分局。先是各省局卡林立,擾民病商,屢經奉飭裁並,而江西一省尚多至七十餘局。御史鄭思賀又以為言,核實刪減。

二十三年,戶部疏言:「各省釐局中飽,弊在承辦之員不肯和盤托出。各省例不應支而非得已者,輒於釐稅收款提留濟用,所謂外銷者也。院司類有存案,原非自謀肥己。然既有外銷之事,即有匿報之款,否則從何羅掘?無惑乎人言藉藉,僉謂各省實收之數,竟數倍報部之數。現在中飽之弊,已諭飭各將軍、督撫認真整頓,自不至仍前洩沓。惟外銷之數若不和盤托出,臣部總握度支,歲入歲出,終於無可句稽。即外銷款目不能驟議全裁,亦宜咨報臣部,權衡緩急,內外一氣,共濟時艱。擬準將外銷最要之款,切實聲明,量予留支,使無窘公用。此後再有隱匿,甚或巧立名目,謬稱入不敷出,則典守之官,不能辭咎。」上下大學士及廷臣議。越二年,上從諸臣議,飭各將軍、督撫詳細稽核,究竟裁去陋規中飽之數若干,酌量提歸公用之數若干,勒限奏明。其外銷款項,應準臚列報部,以昭核實。所有水陸總分各局卡,應如何因地制宜,官紳並委,著體察情形辦理。

二十九年,江西巡撫柯逢時言:「江西釐局積弊過深,改辦統捐,凡納捐貨物,黏貼印花,概不重徵。」報聞。宣統元年,四川以實行禁煙,籌抵土藥各稅釐,加倍徵收肉釐,允之。二年,貴州三江釐局改辦木植統捐局。陜西百貨釐捐亦改照統稅辦法,減為二十七局。

洋藥。道光初,英吉利大舶終歲停泊零丁洋、大嶼山等處,名曰躉船,凡販鴉片煙至粵者,先剝赴躉船,然後入口。省城包買戶謂之窯口。議定價值,同至夷船兌價給單,即雇快艇至躉船,憑單取土。其快艇名快蟹,械砲畢具,行駛如飛,兵船追捕不及。灌輸內地,愈禁愈多。各項貨物,亦多從躉船私售。紋銀之出洋,關稅之偷漏,率由於此。疊經諭飭驅逐嚴拿,而躉船停泊、快蟹遞私如故。

十八年,鴻臚卿黃爵滋言:「自煙土入中國,粵奸商句通巡海弁兵,運銀出洋,運土入口。查道光初年,歲漏銀數百萬;十四年以前,歲漏二千餘萬;近年歲漏三千餘萬。此外各海口合之亦數千萬。年復一年,伊於胡底。耗銀之多,由於販煙之盛。販煙之盛,由於食煙之眾。實力查禁,宜加重罪名。」上韙其言,特命林則徐為欽差大臣,赴粵查辦。明年,截獲躉船煙土二萬八百八十餘箱,焚之。時定禁煙章程,凡開設窯口及煙館,與興販吸食,無論華洋,均擬極刑。

咸豐七年,閩浙總督王懿德等,始有軍需緊要,暫時從權,量予抽捐之請。朝旨允行。八年,與法定約。向來洋藥不準通商,現稍寬其禁,聽商貿易。每百斤納稅銀三十兩,只在口銷售,離口即屬中國貨物,準華商運往內地,法商不得護送。嗣與各國定約皆如之。九年,上以洋藥未定稅前,地方官多有私收情弊,現既議定稅章,自應一律遵辦。上海為各商薈萃之區,尤宜及早奉行,不得以多報少,藉肥私囊。兩江總督何桂清請減輕洋藥稅,下廷議。尋議:「洋藥稅則,各省關均照辦,江蘇何得獨異?所徵稅銀,每三月報解,不準留支。至洋藥釐捐,與關稅有別,原定銀二十兩,毋庸再加十兩,惟不得以洋稅抵作釐捐。」允之。雲貴總督張亮基言滇省向無洋藥,上命先將所產土藥分別徵收稅釐,不得以洋藥混土藥。

十一年,上海新行洋藥稅章程,而普魯斯領事密迪樂以洋商既定進口稅,重徵華商,有礙洋商貿易。上曰:「洋商進口,華商出口,兩稅各不相礙。」不允其請。時稅務司赫德言:「洋藥抽稅,今昔情形不同,收稅愈重,則走漏愈甚。」上以其言可採,下所司酌議施行。

光緒初元,廣東招商包收洋藥捐,年認交四十二萬元,五年限滿,每年遞增二萬元。二年,與英定約,洋藥入口,由官稽查,封存棧房或躉船,俟售賣時,照則納稅;並令購者輸納例稅,以防偷漏:其數由各省酌定。六年,廣東新商接辦洋藥捐,年認交九十萬元,仍五年為限。

七年,大學士左宗棠言:「禁食鴉片,宜先增稅。洋藥百斤,擬徵稅釐百五十兩。土藥價低,準依洋藥推算。」上命將軍、督撫及海關監督各就情形妥議以聞。尋直隸總督李鴻章言:「洋藥既難驟禁,只可先加稅釐。煙價增,則吸者漸減,未始非徐禁示罰之意。惟釐稅太重,恐偷漏愈多,亦須通盤籌計。查洋藥由印度先到香港,然後分運各口,奸商即于該港私相授受。檢閱貿易總冊,同治十三年至光緒四年,到港洋藥,每年八萬四千至九萬六千餘箱,運銷各口有稅者,只六萬五千至七萬一千餘箱。五年到港十萬七千餘箱,運銷各口有稅者,只八萬六千餘箱,年計私銷二萬數千箱。加捐易辦,偷漏難防。擬於洋藥每百正稅三十兩外,加徵八十兩,統計釐稅百一十兩。土藥不論價之高下,每百徵四十兩。」帝用其議。又以洋藥來自英商,命出使大臣曾紀澤與英確商。至九年,始如前議定約,並在進口時輸納。

十年,定不分洋土藥,給華商行坐部票例。其行票每限十斤,斤捐銀二錢,經過關卡,另納稅釐。無票,貨沒官。其行店坐票,無論資本大小,年捐二十兩,換票一次。無票不得售賣。十一年,定洋藥入口,由官驗明封存,俟每箱百斤,完納正稅三十兩、釐金八十兩,方允出運。十三年,與葡定議,在澳門協助中國徵收運往各口之洋藥稅釐,一如英香港辦法。

二十八年,定洋藥稅釐並徵,仍照現行約章,嗣後應以釐金作為加稅。又定英商莫啡鴉之禁。其為醫藥用者,進口仍照則納稅,俟領海關專單,方準起岸,違者沒官。是年裁浙江洋藥釐金局,歸海關釐稅並徵。三十二年,德宗銳意圖強,命限十年將洋藥一律革除凈盡。又以鴉片為生民之害,禁吸尤必禁種,為清源辦法,務令遞年減種,統限十年將洋土藥盡絕根株。是年開廣西巡撫柯逢時缺,賞侍郎銜,督辦各省土藥統稅,設總局於湖北,各省並設分局。逾年,以洋土兩藥稅釐為歲入鉅款,既嚴行禁斷,應預籌的款以資抵補。初定莫啡鴉進口每兩徵稅三兩,至是以既準醫藥需用,減輕照百貨例,值百徵五。

宣統二年,度支部奏言:「各省土藥減收,業將浙江、福建、江蘇、安徽、山東、山西土藥統稅分局先後裁撤。其兩湖、陜、甘、兩粵,略有收數,自應及時收束。惟稅局之應否裁撤,以有無稅項為斷,而統稅之應否停徵,以有無產土為衡。」於是分遣司員,派赴各省調查。明年,又奏言:「現在擬裁土藥統稅分局,尚未據各省議定辦法,派員接收。而洋藥進口,已與英定約,稅釐並徵,每百兩增收二百五十兩,土藥亦須同時比例加稅。查土藥價值不及洋藥三分之二。以徵為禁,稅則無妨略重,即照洋藥稅推算,定土藥百斤加徵二百三十兩。凡未禁運及本產本銷地方,即按新章徵收。」從之。時與英議定,禁煙遞減,已滿三年,如於未滿之七年期內,土藥禁絕,則洋藥亦禁進口。以洋藥加稅實行,停止各項捐收。

會計順治初,既除明季三餉,南服諸省尚未底定,歲入本少,而頻年用兵,經營四方,供億不貲,歲出尤鉅。至九年,海宇粗定,歲入則地丁等款徵銀二千一百二十六萬兩有奇,鹽課徵銀二百一十二萬兩有奇,關稅等銀一百餘萬兩,米、麥、豆之徵本色者五百六十二萬石有奇。歲出則諸路兵餉需千三百餘萬兩,王公官俸各費需二百餘萬兩,各省留支驛站等款三百餘萬兩。其後兵餉增至二千四百萬兩,地丁亦至二千五百餘萬兩。

康熙之初,三籓叛逆,歲入地丁等款,自二千六百餘萬減至二千一百餘萬。二十一年,三籓削平,歲入地丁等銀復至二千六百三十四萬兩有奇,鹽課銀亦至二百七十六萬兩有奇,關稅等銀二百餘萬兩,米、麥、豆之徵本色者為六百三十四萬石有奇。雍正初年,整理度支,收入頗增。

至乾隆三十一年,歲入地丁為二千九百九十一萬兩有奇,耗羨為三百萬兩有奇,鹽課為五百七十四萬兩有奇,關稅為五百四十餘萬兩有奇,蘆課、魚課為十四萬兩有奇,茶課為七萬兩有奇,落地、雜稅為八十五萬兩有奇,契稅為十九萬兩有奇,牙、當等稅為十六萬兩有奇,礦課有定額者八萬兩有奇,常例捐輸三百餘萬,是為歲入四千數百餘萬之大數,而外銷之生息、攤捐諸款不與焉。

歲出為滿、漢兵餉一千七百餘萬兩,王公百官俸九十餘萬兩,外籓王公俸十二萬兩有奇,文職養廉三百四十七萬兩有奇,武職養廉八十萬兩有奇,京官各衙門公費飯食十四萬兩有奇,內務府、工部、太常寺、光祿寺、理籓院祭祀、賓客備用銀五十六萬兩,採辦顏料、木、銅、布銀十二萬兩有奇,織造銀十四萬兩有奇,寶泉、寶源局工料銀十萬兩有奇,京師各衙門胥役工食銀八萬兩有奇,京師官牧馬牛羊象芻秣銀八萬兩有奇,東河、南河歲修銀三百八十餘萬兩,各省留支驛站、祭祀、儀憲、官俸役食、科場廩膳等銀六百餘萬兩,歲不全支,更定漕船歲約需銀一百二十萬兩,是為歲出三千數百餘萬之大數,而宗室年俸津貼、漕運旗丁諸費之無定額者,各省之外銷者不與焉。

自是至道光之季,軍需、河工、賑務、賠款之用,及歷次事例之開,鹽商等報效修河工料之攤徵,凡為不時之入與供不時之出者,為數均鉅。然例定之歲入歲出,仍守乾隆之舊。是以乾隆五十六年,歲入銀四千三百五十九萬兩,歲出銀三千一百七十七萬兩。嘉慶十七年,歲入銀四千十三萬兩,歲出銀三千五百十萬兩。道光二十二年,歲入銀三千七百十四萬兩,歲出銀三千一百五十萬兩,均有奇。咸豐初年,粵匪驟起,捻、回繼之,國用大絀。迄於同治,歲入之項,轉以釐金洋稅大宗,歲出之項,又以善後籌防為鉅款。

光緒五年八月,翰林院侍讀王先謙奏:「舊入之款,如地丁雜稅、鹽務雜款等,共四千萬,今止入二千七八百萬。新入之款,如洋稅一千二百萬,鹽釐三百萬。舊出款,如兵餉、河工、京餉、各省留支四千萬,今止支二千四五百萬。新有出款,如西征、津防兩軍約一千萬,各省防軍約一千萬。」

十七年,戶部奏更定歲出歲入,以光緒七年一年歲出入詳細冊底為據。言:「臣部為錢糧總匯之區,從前出入均有定額。入款不過地丁、關稅、鹽課、耗羨數端,出款不過京餉、兵餉、存留、協撥數事,最為簡括。乃自軍興以來,出入難依定制。入款如扣成、減平、提解、退回等項,皆系入自出款之中。出款如撥補、籌還、移解、留備等項,又皆出歸入款之內。匯核良非易易。此次所辦冊籍,以地丁、雜賦、地租、糧折、漕折、漕項、耗羨、鹽課、常稅、生息等十項為常例徵收,以釐金、洋稅、新關稅、按糧津貼等四項為新增徵收,以續完、捐輸、完繳、節扣等四項為本年收款。除去蠲緩未完各數,通計實入共收銀八千二百三十四萬九千一百九十八兩,是為銀收。以陵寢供應、交進銀、祭祀、儀憲、俸食、科場、餉乾、驛站、廩膳、賞恤、修繕、河工、採辦、辦漕、織造、公廉、雜支等十七項為常例開支,以營勇餉需、關局、洋款、還借息款等四項為新增開支,以補發舊欠,豫行支給兩項為補支豫支,以批解在京各衙門銀兩一項為批解支款。除去欠發未報各數,通計實出共支銀七千八百十七萬一千四百五十一兩,是為銀支。原奏並及錢收、糧收、錢支、糧支,實為明覈。今按十七年歲入歲出之籍,入項為地丁二千三百六十六萬六千九百一十一兩,雜賦二百八十一萬有一百四十四兩,租息十四萬一千六百七十二兩,糧折四百二十六萬二千九百二十八兩,耗羨三百萬四千八百八十七兩,鹽課七百四十二萬七千六百有五兩,常稅二百五十五萬八千四百一十兩,釐金一千六百三十一萬六千八百二十一兩,洋稅一千八百二十萬六千七百七十七兩,節扣二百九十六萬四千九百四十四兩,續完七百十二萬八千七百四十四兩,捐繳一百八十七萬五千五百七十六兩,均有奇。統為歲入八千九百六十八萬四千八百兩有奇。出項為陵寢供應等款十三萬五百五十九兩,交進十八萬兩,祭祀三十三萬六千七百三十三兩,儀憲七萬四千八百七十九兩,俸食三百八十四萬一千四百二十四兩,科場十萬五千二百七十兩,餉乾二千三十五萬六千一百五十九兩,驛站一百七十三萬四千七百有九兩,廩膳十一萬二千有二十九兩,賞恤五十二萬五千二百十六兩,修繕二百二十萬九千七百四十八兩,採辦四百有三萬三千九百有三兩,織造一百有三萬四千九百十五兩,公廉四百五十七萬五千七百八十三兩,雜支三十萬三千二百七十八兩,勇餉一千八百二十六萬八千三百十三兩,關局經費三百十四萬四千六百十六兩,洋款三百八十六萬一千五十一兩,補支一千二百七十七萬五千五百二十五兩,豫支一百七十四萬二千七十三兩,解京各衙門飯食經費各項支款三百四十七萬二千五百三十三兩。統為歲出七千九百三十五萬五千二百四十一兩。」

再三年為甲午,朝鮮役起,軍用浩繁,息借洋款、商款。及和議既定,又借俄、法、英、德之款付日本賠款,增攤各省關銀一千二百萬兩,益以匯豐、克薩、華商各款本息,及新增宋慶等軍餉,共八百萬。蓋歲出之增於前者二千萬。迨於庚子,復釀兵禍。辛丑約成,遂有四萬五千萬之鉅,派之各省者一千八百萬兩有奇。二十九年,以練新軍,復攤各省練兵經費,而各省以創練新軍,辦巡警教育,又有就地自籌之款。奉天一省警費至三百餘萬兩。湖北一省撥提地丁錢價充學費者六十萬兩。捐例停於二十七年,以練兵復開,至三十二年復停。

庚子以後新增之徵收者,大端為糧捐,如按糧加捐、規復徵收丁漕錢價、規復差徭、加收耗羨之類;鹽捐如鹽斤加價、鹽引加課、土鹽加稅、行鹽口捐之類;官捐如官員報效、酌提丁漕盈餘、酌提優缺盈餘之類;加釐加稅如菸酒土藥之加釐稅、百貨稅之改統捐、稅契加徵之類;雜捐如彩票捐、房鋪捐、漁戶捐、樂戶捐之類;節省如裁節綠營俸餉、節省河工經費、核扣驛站經費、節省各署局經費之類;實業如鐵路、電局、郵政收入,及銀行、銀銅元局、官辦工廠商局餘利之類。出款自賠款、練兵費、學、警、司法諸費外,各官署新增費亦為大端。

宣統二年,度支部奏試辦宣統三年預算,歲入為類八:曰田賦,經常四千六百十六萬四千七百有九兩,臨時一百九十三萬六千六百三十六兩,皆有奇。曰鹽茶課稅,經常四千六百三十一萬二千三百五十五兩。曰洋關稅,經常三千五百十三萬九千九百十七兩。曰常關稅,經常六百九十九萬一千一百四十五兩,臨時八千五百二十四兩。曰正雜各稅,經常二千六百十六萬三千八百四十二兩。曰釐捐,經常四千三百十八萬七千九百七兩。曰官業收入,經常四千六百六十萬八百九十九兩。曰雜收入,經常一千九百十九萬四千一百有一兩,臨時一千六百有五萬六百四十八兩。附列者為類二:曰捐輸,五百六十五萬二千三百三十三兩。曰公債,三百五十六萬兩。皆臨時歲入。歲出為類十八:曰行政,經常二千六百六萬九千六百六十六兩,臨時一百二十五萬八千一百八十四兩。曰交涉,經常三百三十七萬五千一百有三十兩,臨時六十二萬六千一百七十七兩。曰民政,經常四百四十一萬六千三百三十八兩,臨時一百三十二萬四千五百三十一兩。曰財政,經常一千七百九十萬三千五百四十五兩,臨時二百八十七萬七千九百有四兩。曰洋關經費,經常五百七十四萬八千二百三十七兩,臨時九千一百六十三兩。曰常關經費,經常一百四十六萬三千三百三十二兩。曰典禮,經常七十四萬五千七百五十九兩,臨時五萬四千有三十七兩。曰教育,經常二百五十五萬三千四百十六兩,臨時一百四萬一千八百九十二兩。曰司法,經常六百六十一萬六千五百七十九兩,臨時二十一萬八千七百四十六兩。曰軍政,經常八千三百四十九萬八千一百十一兩,臨時一千四百萬有五百四十六兩。曰實業,經常一百六十萬三千八百三十五兩。曰交通,經常四千七百二十二萬一千八百四十一兩,臨時七百有八十萬四千九百有八兩。曰工程,經常二百四十九萬三千二四百兩,臨時二百有二萬二千有六十四兩。曰官業支出,經常五百六十萬四百三十五兩。曰各省應解賠款、洋款,三千九百有十二萬九百二十二兩。曰洋關應解賠款、洋款,一千一百二十六萬三千五百四十七兩。曰常關應解賠款、洋款,一百二十五萬六千四百九十兩。曰邊防經費,一百二十三萬九千九百有八兩。附列者為類一:曰歸還公債,四百七十七萬二千六百十三兩。統為歲入二萬九千六百九十六萬二千七百兩有奇。歲出三萬三千八百六十五萬兩有奇。十二月,資政院覈覆,於歲入有增加,於歲出有減削。次年即值變更國體,故有預算而無決算。蓋自光緒三十三年,度支部即奏準令京師各衙署及各省實報歲入歲出,又於各省設財政監理官以督之。凡昔日外銷之款項,與夫雜捐陋規之類,及新定之教育、司法、實業、軍政、外債諸費,皆列於簿書期會,故較順治、康熙之出入多至十倍。茲錄之以見一代財政之盈虧焉。

其軍需、河工、賑務、賠款之鉅者,乾隆初次金川之役,二千餘萬兩。準回之役,三千三百餘萬兩。緬甸之役,九百餘萬兩。二次金川之役,七千餘萬兩。廓爾喀之役,一千有五十二萬兩。臺灣之役,八百餘萬兩。嘉慶川、湖、陜教匪之役,二萬萬兩。紅苗之役,湖南一省請銷一千有九十萬。洋匪之役,廣東一省請銷三百萬兩。道光初次回疆之役,一千一百餘萬兩。二次回疆之役,七百三十萬兩。英人之役,一千數百萬兩。咸豐初年粵匪之役,二千七百萬,其後江南大營月需五十萬兩,徽寧防營月需三十萬兩,則一年亦千萬。湖北供東征之需者,歲四百餘萬,湖南亦不貲。而北路及西南各省用兵之費不與焉。同治中,曾國籓奏湘軍四案、五案,合之剿捻軍需,共請銷三千餘萬兩。李鴻章奏蘇滬一案、二案,合之淮軍西征兩案,共請銷一千七百餘萬兩。左宗棠奏西征兩案,共請銷四千八百二十餘萬兩。此外若福建援浙軍需,合之本省及臺灣軍需,截至三年六月,已逾六百萬兩。四川、湖南援黔軍需,歲約四百萬兩,積五年二千萬兩。雲南自同治二年至同治十二年,請銷軍需一千四百六十餘萬兩。而甘肅官紳商民集捐銀糧供軍需者,五千餘萬兩,再加各省廣中額學額計之,當不下數萬萬。光緒中,惟中法之役用三千餘萬兩。若西征之餉,海防之餉,則已入年例歲出,不衣復列。

河工,自康熙中即趨重南河。十六年大修之工,用銀二百五十萬兩。原估六百萬兩,迨蕭家渡之工,用銀一百二十萬兩。自乾隆十八年,以南河高郵、邵伯、車邏壩之決,撥銀二百萬兩。四十四年,儀封決河之塞,撥銀五百六十萬兩。四十七年,蘭陽決河之塞,自例需工料外,加價至九百四十五萬三千兩。浙江海塘之修,則撥銀六百餘萬兩。荊州江堤之修,則撥銀二百萬兩。大率興一次大工,多者千餘萬,少亦數百萬。嘉慶中,如衡工加價至七百三十萬兩。十年至十五年,南河年例歲修搶修及另案專案各工,共用銀四千有九十九萬兩,而馬家港大工不與。二十年睢工之成,加價至三百餘萬兩。道光中,東河、南河於年例歲修外,另案工程,東河率撥一百五十餘萬兩,南河率撥二百七十餘萬兩。逾十年則四千餘萬。六年,撥南河王營開壩及堰、盱大堤銀,合為五百一十七萬兩。二十一年,東河祥工撥銀五百五十萬兩。二十二年,南河揚工撥六百萬兩。二十三年,東河牟工撥五百十八萬兩,後又有加。咸豐初,豐工亦撥四百萬兩以上。同治中,山東有侯工、賈莊各工,用款二百餘萬兩。光緒十三年,河南鄭州大工,請撥一千二百萬兩。其後山東時有河溢,然用款不及道光之什一。

賑務,康熙中,賑陜西之災,用銀至五百餘萬兩。乾隆七年,江蘇、安徽夏秋大水,撫恤、正賑、加賑,江蘇給被災軍民等米共一百五十六萬石有奇,銀五百五萬兩有奇。安徽給被災軍民等米八十三萬石有奇,銀二百三十三萬兩有奇。十八年,以高郵運河之決,撥米穀一百十萬石,銀四百萬兩,賑江蘇災,此其最鉅者。其後直隸、山東、江蘇、河南、湖北、甘肅諸省之災,發帑截漕及資於捐輸者,不可勝舉。嘉慶初,山東曹、單等縣災,賑銀米合計三四百萬兩。六年,以直隸水災,撥賑銀一百萬兩,截漕米六十萬石。江蘇、安徽、山東、河南諸省之因災賑恤者,節次糜帑,均不下數十百萬。資於捐輸者,如十九年江蘇、安徽之災,至二三百萬兩。道光十一年,撥江蘇賑需銀一百餘萬兩。二十七年,賑河南災銀一百餘萬兩。二十八年,賑河北災銀一百三十八萬兩。二十九年,撥江蘇等四省賑災銀一百萬兩。而安徽、浙江之截留辦賑者,皆近百萬,江蘇一省則一百四十餘萬,此外尚多,而官紳商民捐輸者尚不與。光緒初,山西、河南、陜西之災,撥帑截漕為數均鉅,合官賑、義賑及捐輸等銀,不下千數百萬兩。鄭州河決,賑需河南用銀二百五十餘萬兩。時各省有水旱之災,輒請開賑捐。直隸自十六年之水至二十一年海嘯之災,用銀七百餘萬兩。山東自十一年後,頻年河溢,至二十五年,用銀七百餘萬兩。江蘇自十五年之水至二十四年淮、徐、海之災,用銀五百餘萬兩。二十七年秦、晉之災,則開實官捐以濟之,為數至七百六十萬兩有奇。

賠款,始於道光壬寅江寧之約,二千一百萬兩。咸豐庚申之約,一千六百萬兩。光緒辛巳伊犁之約,六百餘萬兩。乙未中日之約,並遼南歸地,二萬三千萬兩。至辛丑公約,賠款四萬五千萬兩而極。以息金計之,實九萬萬餘兩。

清代田賦徵糧之數,乾隆三十一年,為八百三十一萬七千七百石有奇。江蘇、安徽、山東、河南、浙江、江西、湖北、湖南八省,自歲漕京師外,留充本省經費。直隸、奉天、山西、陜西、甘肅、福建、四川、廣東、廣西、雲南、貴州則全充本省經費。光緒十年,新疆改行省,歲徵糧二十七萬一千石有奇,亦全充本省經費。吉林、黑龍江之徵米者亦如之。各省駐防旗營官兵、綠營兵丁皆支月米。凡留充本省經費者,大率供旗綠營月支米豆之需,有餘則報糶易銀候撥雲。


\end{pinyinscope}