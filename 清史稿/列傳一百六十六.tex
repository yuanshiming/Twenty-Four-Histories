\article{列傳一百六十六}

\begin{pinyinscope}
趙慎畛盧坤曾勝陶澍

趙慎畛,字笛樓,湖南武陵人。為諸生時,學政錢灃器之,曰:「人英也!」嘉慶元年,成進士,選庶吉士,授編修。遷御史、給事中。條上川、楚善後屯田保甲事宜。巡通州漕,革陋規,廉得楊村通判科索剝船,奏褫其職。湖南學政徐松矜愎失士心,欲附慎畛自固,常列其弟子優等,慎畛列款糾劾罷之。兩廣總督蔣攸銛薦其才可大用。

十七年,出為廣東惠潮嘉道。嚴治械斗,捕南澳、澄海、潮陽盜甚眾;沿海民寮居為逋逃藪,悉編入保甲。逾年,擢廣西按察使。天地會匪結黨構亂,脅有貲者入其中,慎畛惟嚴罪匪首,被脅者不坐。廣東洋匪投誠後,漸入廣西為盜。設水路巡船以護商旅,督守令以捕盜多少為殿最。遠郡招解重囚煩費,吏因諱盜,省文法,嚴舉劾,緝捕始力。二十年,遷廣東布政使。州縣多積虧,展轉相承,悉心鉤稽,除其糾轕,庫儲頓增。南海、高要瀕河堤防多圮,民苦水患,籌款生息資歲修,屯田五千餘頃。賦重為累,請減糧額,攤抵於沙坦輕則之地。粵俗奢靡,刊發陳宏謀行政訓俗遺規,躬行節儉以示勸。

二十三年,擢廣西巡撫。習知粵西地勢如建瓴,旬日不雨即旱竭,勸民修是塘,造龍骨車,開廕井,設井筒架,皆頒式俾仿行。地連黔、楚,群盜出沒,宜山會匪廖五桂、藍耀青分踞新、舊兩墟,糾眾分黨,偽立名目,勒索殷戶,爭利相擾,親往捕誅之。飭屬行保甲,置望樓,練民壯互相守望,縣建卡房數十座,府各督屬會營巡緝。柳州至省千餘里,設水汛四十三所,終任凡獲盜千七百餘人。盜多出於流匪,編客民籍,驅其單身游蕩者,礦廠窯榨傭丁皆立冊,有保者留,否則逐。故事,梧、潯二關,巡撫例得動用盈餘。慎畛曰:「吾家衣食粗足,身為大臣,取盈將安用之?當為國家布仁澤耳。」乃於桂林設預備倉,增設書院,柳州、慶遠、思恩三府皆創設之;繕城濬河,廣置棲流所,並取給焉。

道光二年,入覲,宣宗嘉其誠實不欺,溫諭褒勉,擢閩浙總督。嚴申軍律,課諸鎮營汛勤訓練。浙江提督沈添華玩縱,劾罷之。責水師緝海盜,盜多就擒。上游四府多山,客民租山立廠,游匪群聚,遣兵搜山,捕誅其魁。閩安所轄有瑯琦島,居民多為奸利,擒治之,移駐水師,建砲臺,遂為省城門戶。臺灣自來多亂,動煩大兵,慎畛尤以為慮,盡選賢能以治。鳳山莠民楊良斌煽眾起事,檄巡道孔昭虔、知府孔傳穟剿治,未一月而定,不煩一兵渡海。驩瑪蘭初設治,部議賦則較重,奏減之。民入山伐木,歲供道廠船料,匠首苛斂激變,捕誅首亂,更定採木章程,乃相安。戍兵萬四千,更代時皆赴廈門,由提督點驗,遠者千里,改由各提鎮分驗,兵困以蘇。臺灣產米,漳、泉數郡仰給商運,江、浙、天津民無蓋藏,米貴輒生亂,於海口稽米船出數,酌豐歉為限制,常留有餘。疏請漳浦明儒黃道周從祀文廟,下廷臣議行。侯官謝金鑾、德化鄭兼才皆以學行著,素所敬禮,歿而舉祀鄉賢。又旌表義烈,以振風俗。

五年,調雲貴總督。銅礦、鹽務積疲,疏陳變通整頓之法。以邊防莫便於屯田,方考訪形勢利便,未及議行而疾作。病中拜疏劾貪黷不職者數十人。未幾,遂卒。代者急遞追回原疏,滇人惜之。遺疏上,優詔賜恤,贈太子少保,謚文恪,祀名宦、鄉賢祠。

慎畛服膺儒先,凡有益身心可致用者,皆身體力行。好善嫉惡,體恤屬僚,訓懇切,如師之於弟子。所至於文武官吏,常能識別其才否,人亦樂為之用。所著奏議、從征錄、載年錄、讀書日記、惜日筆記等書及詩文集凡數十卷。

盧坤,字厚山,順天涿州人。嘉慶四年進士,選庶吉士,散館授兵部主事,洊遷郎中。扈隨木蘭,校射,賜花翎。十八年,出為湖南糧儲道,丁本生父母憂,服闋,歷廣東惠潮嘉道、山東兗沂曹濟道、湖北按察使、甘肅布政使。道光元年,護理陜西巡撫。二年,擢廣東巡撫,未之任,調陜西。議者謂南山老林易藪奸,不宜開墾。坤歷陳漢、蜀、唐、宋史事,及漢李翕郙閣頌,以徵墾治之利;專任嚴如熤,假以便宜,墾務大興。勘修南山各屬城工,漢江堤岸,築壩濬淤,審度形勢,移駐文武,增改官制。又修復咸寧、長安、涇陽、盩厔、岐山、寶雞、華州、榆林河渠水利,籌補榆林、綏德兩屬常平倉穀,勸民捐建社倉。疏陳:「察吏之要,不獨親民,官貪廉為民身家所系,其勤惰、明昧、寬嚴,皆關民生休戚。」宣宗深韙之。五年,以母憂去官。

六年,回疆用兵,特起駐肅州,偕總督鄂山治轉餉。以托古遜為運糧首站,自烏魯木齊至阿克蘇,置三十二站,大兵五萬餘,日需糧五百石,每站備駝五百有奇,由山西、陜西採購;又蒙古阿拉善部進駝千,烏里雅蘇臺調撥官駝四千。疏請軍需從寬籌備;兵丁量增口糧;給皮衣皮帽,以禦寒;出口駝馬芻秣;時給買補缺額營馬,預備續調;監造軍械務期堅實;撥運陜省制錢,平市價;添設臺站夫馬;雇用車輛,定例價;招募護臺民丁;後路糧臺亦添兵守護:凡十一事,並如議行。回疆平,加太子少保。及張格爾就擒,賜頭品頂戴。服闋,授山東巡撫,調山西。八年,裁撤肅州軍局。始抵任,尋調廣東巡撫。

十年,又調江蘇,未至,擢湖廣總督。兩湖鹺務,狃於封輪之例,道光初議散輪,七年復因加價,仍改封輪,引滯商疲。坤至,疏請實行散輪,建鹽倉於漢岸,俾商船源源攬運。尋量減售價,以銷楚岸積鹽。設塘角總卡,按船編號,以杜內私外私之弊。復湖南永興粵鹽定額,以保淮綱。湖北水災,請免米稅,借帑十萬兩,購川米平糶。疏調前兩淮鹽運使王鳳生綜理水利,擇要疏濬河道,修築堤堰,皆以次舉行。

十二年,湖南江華瑤趙金龍作亂,粵瑤應之,湖南提督海凌阿及副將、游擊等皆戰歿,坤親往督師,密陳湖北提督羅思舉能辦賊。時桂陽、常德諸瑤蜂起應賊,常德水師、荊州駐防兵皆不習山戰,坤至,悉罷之,改調鎮筸苗疆兵,分屯要隘,堅壁清野,與賊相持。俟兩湖兵大集,貴州提督餘步雲、雲南副將曾勝亦率軍至,乘雷雨襲擊洋泉街。羅思舉督諸將晝夜環攻,斃賊數千,破其巢,擒金龍子女及頭目數百人。金龍乘間逸,為亂軍所殲,獲其尸及劍印木偶諸物。捷聞,賜雙眼花翎,世襲一等輕車都尉。尚書禧恩、將軍瑚松額方奉命視師,未至,賊已平。粵瑤趙青仔糾眾數千入楚界,聲言為金龍復仇,連敗之於濠江、銀江,擒青仔磔於市。廣東連山黃瓜寨瑤猶猖獗,兩廣總督李鴻賓剿治不力,以罪逮,調坤代之。偕禧恩等先後往督諸將進剿,瑤疆悉平。合疏陳兩省善後事宜,改移文武官制駐所,並允行。

十三年,越南盜陳加海結邊地游民嘯聚狗頭山,潛入內洋,遣水師擊沉八船,擒加海誅之。尋越南內訌,慎固邊防,拒其請兵,詔嘉得大體。

英吉利兵船擅入海口,要乞推廣通商,坤依故事停其貿易。領事律勞卑挾二船入虎門,砲擊不退,且以砲拒,進泊黃埔。坤設方略扼其歸路,斷其接濟,集水陸師臨以兵威,律勞卑窮蹙,引罪求去。澳門洋商代請命,坤持之良久,乃驅之出口。疏聞,詔嘉獎,先奪宮銜、花翎並復之。於是嚴海防,勤訓練,自南山至大虎分三段,與沙角、大角相聯絡。省河中流沙地增建砲臺,以資保障,夷情斂懾。坤久任封圻,所皆有名績,宣宗深倚之。十五年,卒,贈太子太師、兵部尚書,從優恤,謚文肅。子端黼,襲世職。

曾勝,廣西馬平人。以行伍從剿湖南苗匪、川、楚教匪,積功至都司。累遷雲南參將,以計擒梟渠徐黑二及宣威小梁山匪首,為時稱。遷維西協副將。瑤匪趙金龍之亂,率師會剿,擢湖南永州鎮總兵,殲金龍,及擒粵瑤趙青仔,戰皆力。尋赴廣東剿連山瑤,迭戰大拱橋、分水嶺、砲臺山、火燒坪、軍僚里、大厓沖、上坻園。瑤平,論功最,加提督銜,賜號瑚爾察圖巴圖魯,予雲騎尉世職。調南韶連鎮,擢廣東陸路提督。當英吉利兵船入內河,水師提督李增堦不能阻,勝獻策,以巨船載石沉塞海口老洲岡隘道,聚草船數百橫內河,備火攻,勝率兵臨之,英領事律勞卑悚懼聽令,事乃定。十七年,卒於官,謚勤勇。

陶澍,字雲汀,湖南安化人。嘉慶七年進士,選庶吉士,授編修,遷御史、給事中。疏劾吏部重簽,河工冒濫,及外省吏治積弊。巡中城,決滯獄八百有奇。巡南漕,革陋規,請濬京口運河。二十四年,出為川東道,日坐堂皇,剖決獄訟如流。請減鹽價,私絕課增。總督蔣攸銛薦其治行為四川第一。歷山西按察使、安徽布政使。

道光三年,就擢巡撫。安徽庫款,五次清查,未得要領。澍自為籓司時,鉤覈檔案,分別應劾、應償、應豁,於是三十餘年之糾轕,豁然一清。嚴交代,禁流攤,裁捐款,至是奏定章程,俾有司釋累,得專力治民。瀕江水災,購米十萬石,勸捐數十萬金,賑務覈實,災民賴之無失所。治壽州城西湖、鳳臺蕉岡湖、鳳陽花源湖;又懷遠新漲沙洲阻水,並開引河,導之入淮。淮水所經,勸民修是束水,保障農田。各縣設豐備倉於鄉村,令民秋收後量力分捐,不經吏役,不減糶,不出易,不假貸,歲歉備賑,樂歲再捐,略如社倉法而去其弊。創輯安徽通志,旌表忠孝節烈以勵風俗。

五年,調江蘇。先是洪澤湖決,漕運梗阻,協辦大學士英和陳海運策,而中外紛議撓之。澍毅然以身任,奏請蘇、松、常、鎮、太五府州漕糧百六十餘萬石歸海運,親赴上海,籌雇商船,體恤商艱,群情踴躍。六年春,開兌,至夏全抵天津,無一漂損者,驗米色率瑩潔,過河運數倍。商船回空,載豆而南,兩次得值船餘耗米十餘萬石,發部帑收買,由漕項協濟天津、通倉之用,及調劑旗丁,尚節省銀米各十餘萬。事竣,優詔褒美,賜花翎。明年,遂偕總督蔣攸銛合疏陳海運章程八條,冀垂令甲,永紓漕累,格於部議,未果行。又以紳衿包完漕米,橫索陋規,為漕務之害,奏請懲辦。學政辛從益意不合,爭之。澍復疏言:「陋規日增,勢必取償小民。若預計有司不減浮收,置陋規於不問,非釜底抽薪之計。」仍執前議,治包抗從嚴焉。

江蘇頻遭水患,由太湖水洩不暢。疏言:「太湖尾閭在吳淞江及劉河、白茆河,而以吳淞江為最要。治吳淞以通海口為最要。」於是以海運節省銀二十餘萬興工,擇賢任事,至八年工竣。又以江以南運道,徒陽運河最易淤阻,而練湖為其上游,孟瀆為其旁支。澍自巡漕時,條奏利害,至是先濬徒陽河,將以次舉劉河、白茆、練湖、孟瀆諸工。後在總督任,與巡撫林則徐合力悉加疏濬,吳中稱為數十年之利,語詳則徐傳。

十年,以捕獲戶部私造假照要犯,加太子少保銜,署兩江總督,尋實授。時淮鹽敗壞,商困課絀,岌岌不可終日。澍疏陳積弊,請大刪浮費,以為補救。議者多主改法課歸場灶,命尚書王鼎、侍郎寶興赴江南查議。澍謂除弊即以興利,無事輕改舊制,偕鼎等合疏臚陳利害,條上十五事。鼎等復請裁鹽政歸總督管理,報可。澍受事,繳還鹽政養廉五千兩,裁減衙門陋規十六萬兩有奇,凡淮南之窩價,淮北之壩槓,兩淮之岸費,分別減除,歲計數百萬兩,分設內外二庫,正款貯內庫,雜項貯外庫,杜絕挪墊。革總商以除把持,散輪規以免淹滯,禁糧船回空帶蘆鹽,及商船借官行私,令行禁止,弊肅風清。淮北尤疲累,先借款官督商運,繼仿山東、浙江票引兼行之法,於海州所屬中正、板浦、臨興三場擇要隘設局給票,注明斤數運地,無票越境以私論。仍留暢銷之岸,江運八州縣、湖運十一州縣,歸商運。十二年,奏準開辦,越半歲,溢銷逾額,復推廣於江運、湖運各岸,減價裁費,商販爭趨,而窩商蠹吏、壩夫岸胥一旦盡失其中飽需索之利,群議沸騰。言官摭浮言,屢事彈劾,賴宣宗鑒其忠誠,倚畀愈專。屢請復鹽政專職,皆不許,澍益感奮,力排眾議,毅然持之,卒獲成效。道光元年至十年,淮南行六綱,淮北僅行三綱。澍承極弊之後,自十一年至十七年,淮南已完六綱有餘,淮北率一歲行兩綱之鹽,盡完從前滯欠,且割淮南懸引,兩淮共完正雜銀二千六百四十餘萬兩,庫貯實存三百餘萬兩。兩屆京察,並被褒獎優敘。晚年將推淮北之法於淮南,已病風痺,未竟其施,然天下皆知票鹽減價敵私,為正本清源之計。後咸豐中乃卒行之。十九年,卒。遺疏上,優詔軫惜,稱其「實心任事,不避嫌怨」,晉贈太子太保,依尚書例賜恤,賜其子桄主事,謚文毅。祝名宦祠,於海州建專祠。

澍見義勇為,胸無城府。用人能盡其長,所拔取多至方面節鉞有名。在江南治河、治漕、治鹽,並賴王鳳生、俞德源、姚瑩、黃冕諸人之力。左宗棠、胡林翼皆識之未遇,結為婚姻,後俱為名臣。所著奏議、詩文集、蜀輶日記、陶桓公年譜、陶淵明詩輯注並行世。

論曰:趙慎畛學有本源,察吏治民,嚴而能恕,所至政無不舉。盧坤治回疆軍需,平湖南瑤,馭廣東夷商,皆有殊績。陶澍治水利、漕運、鹽政,垂百年之利,為屏為翰,庶無媿焉。道光中年後,海內多事,諸臣並已徂謝,遂無以紓朝廷南顧之憂。人之云亡,邦國殄瘁,其信然哉!


\end{pinyinscope}