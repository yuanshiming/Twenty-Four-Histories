\article{列傳二百二十五}

\begin{pinyinscope}
閻敬銘張之萬鹿傳霖林紹年

閻敬銘,字丹初,陜西朝邑人。道光二十五年進士,選庶吉士,散館改戶部主事。咸豐九年,湖北巡撫胡林翼奏調赴鄂,總司糧臺營務。累遷郎中,擢四品京堂。林翼請病,復疏薦敬銘才,授湖北按察使。同治元年,嚴樹森繼為巡撫,亦推敬銘湖北賢能第一,署布政使。以丁本生父憂歸,命治喪畢赴軍,未行,詔署山東鹽運使,擢署巡撫,疏乞終制,不許。時山東教匪入新泰,捻、幅各匪犯鄒、曲阜,降眾竄陽穀、聊城。敬銘既受任,檄總兵保德等進剿,而自督軍規淄川,克之。已革參將宋景詩引降眾屯東昌,復叛,飭按察使丁寶楨討之。景詩竄莘,敬銘檄軍防運河,令之曰:「使一匪潛渡者,殺無赦!」而自移軍博平。已而保德、寶楨連敗賊唐邑馬橋,克王家海,別軍克甘官屯,賊遁開州。事平,再請終制,仍不許。三年,服除,實授。

奏言抽調綠營兵練騎隊,朝旨允行,令即遣散募勇。敬銘言:「東省變故頻仍,亂甫定,降眾未必革心。綠營廢弛已久,驟裁勇易啟戎心。臣不敢為節嗇帑項浮詞遺後患。」又言:「兵之能強,端恃將領。將領之材,亦資汲引。如胡林翼、曾國籓、左宗棠倡率鄉里,楚將之名遂著。前者僧格林沁奏稱不宜專用南勇,啟輕視朝廷之漸。老成謀國,瞻言百里。自古名將,北人為多。臣北人也,恥不知兵。以在軍久,見諸軍之成敗利鈍,必求其所以然之故。深知不求將而言兵,有兵與無兵等。今北方雖所在募勇,皆烏合耳。為將者貪婪欺飾,不知尊君親上為何事,使握兵符,民變兵譁,後患滋大。故欲強兵必先儲將。北人之智勇兼備者,推多隆阿。請飭多隆阿募北方將士,教之戰陣,擇其忠勇者,補授提、鎮、參、游,俾綠營均成勁旅,何必更募勇丁?」時捻患熾,臺臣議行團練。敬銘言:「斂鄉里之財以為餉,集耕種之民以為兵,於事有害無益,不如力行堅壁清野之法。」事遂寢。

四年,僧格林沁戰歿曹州,賊勢張,益趨張秋南,將犯省城。敬銘督師東昌,還軍御之,增設砲劃防河,賊折而東。移軍兗州,賊竄豐、沛。乃檄總兵楊飛熊間道趨滕,防賊還竄。賊果入湖濱,以飛熊扼運河,不得逞,竄徐州。明年,賊入鉅野,游擊王心安失利。敬銘方臥疾,強起視師東平。兗沂曹濟道文彬督團勇擊賊,賊引去。敬銘赴濟寧,會曾國籓商定分扼黃、運之議。賊復大股趨鉅野、金鄉,分擾運西。遣知府王成謙等要擊,而自督軍巡河,露宿四晝夜,賊連敗,始西遁。有張積中者,結寨肥城黃崖,集眾自保,以不受撫,夷之。六年,移疾歸,居久之,以工部侍郎召,不起。

光緒三年,山西大饑,奉命察視賑務。奏劾侵帑知州段鼎耀,置之法。請裁減山、陜諸省差徭,並追彈尚書恩承、童華前奉使四川過境擾累狀,均下吏議。八年,起戶部尚書,甫視事,以廣東布政使姚覲元、荊宜施道董俊漢賄結前任司員骫法,咸劾罷之。兼署兵部。疏陳興辦新疆屯田。明年,充軍機大臣、總理衙門行走,晉協辦大學士。十一年,授東閣大學士,仍筦戶部,賜黃馬褂。自陳衰老,辭軍機大臣。時上意將修圓明園,而敬銘論治以節用為本,會廷議錢法,失太后旨,因革職留任。十三年,復職,遂乞休,章四上,乃得請。十八年,卒,贈太子少保,謚文介。

敬銘質樸,以潔廉自矯厲,雖貴,望之若老儒。善理財,在鄂治軍需,足食足兵,佐平大難。及長戶部,精校財賦,立科條,令出期必行。初直樞廷,太后頗信仗之,終以戇直早退云。

張之萬,字子青,直隸南皮人。道光二十七年,以一甲一名進士授修撰。咸豐二年,出督河南學政。粵賊破歸德,近偪開封,之萬條上防剿事宜,多允行。俄,召還,授鍾郡王讀。由侍讀累遷內閣學士。同治元年,擢禮部侍郎,兼署工部。嘗被詔偕太常寺卿許彭壽等匯輯前代帝王及垂簾事跡可法戒者上之,錫名治平寶鑒。會河南州縣以苛派擅殺為御史劉毓楠奏劾,命之萬往按,得實,巡撫鄭元善以下降黜有差,即以之萬署巡撫事。疏陳軍興財匱,請仿湖北變通漕折,言:「汴漕一石舊折銀四兩,今請令州縣留辦公費七錢,實解司庫三兩三錢,以二兩購米實倉,餘一兩充汴餉,其三錢為通省公費。」允行。

捻酋陳大喜犯南陽,之萬親赴汝州督師。大喜竄阜陽,勾結皖捻,一由嶽城趨楊莊偪雷堰,一入張岡,總兵張曜馳擊破之。團練大臣毛昶熙諸軍相繼至,連戰皆捷,斬逆酋張鳳舞,汝南肅清。之萬駐軍許州,既分遣諸將設防,自引軍還省;而亳捻乘虛襲許,陷兩寨,坐降二級留任。西捻張總愚竄鄧州,藍大順走西坪,謀與合。張曜既敗總愚重陽店,乘勝襲西坪,大順亦敗走。之萬復進汝州。三年,移屯南陽,賊犯開封,還軍擊走之。四年,遷河道總督。僧格林沁戰歿曹州,督兵大臣皆獲咎。之萬亦革職留任,以助防省城功,給二品頂戴。五年,移督漕運。捻入徐州,之萬以里下河為財賦所出,嚴防清、淮及六塘河諸要地。六年,淮軍獲賴文光於揚州,東捻平。捷聞,賜之萬花翎、頭品頂戴。七年,會剿西捻,總愚溺死,東南大定。之萬疏陳江北善後事宜。九年,調江蘇巡撫。遷浙閩總督,以母老乞養歸。

光緒八年,起兵部尚書,調刑部。十年,入軍機,兼署吏部,充上書房總師傅、協辦大學士。十五年,授體仁閣大學士,轉東閣。賜雙眼花翎、紫韁。二十年,免直軍機。

之萬入直凡十年,領樞密者為禮親王世鐸,治尚安靜,故得無事。及日韓事棘,之萬乃先罷退。又二年,以病致仕。卒,年八十七,贈太傅,謚文達。

鹿傳霖,字滋軒,直隸定興人。父丕宗,官都勻知府,死寇難,謚壯節,傳霖其第五子也。當丕宗守都勻時,叛苗麕聚城下,傳霖方率健卒迎餉,聞警,馳還助城守,相持十閱月,援絕城陷。傳霖投總督告父死狀,大兵攻復都勻,奉父母遺骸歸葬,時年甫二十,由是知名。以舉人從欽差大臣勝保征捻,授同知。同治元年,成進士,選庶吉士,散館改廣西知縣。以督剿柳、雒土匪功,賜孔雀翎,擢桂林知府。光緒四年,調廉州。時李揚才將叛擾越南,急捕之,立散其黨。旋升惠潮嘉道。擢福建按察使,調四川,遷布政使。九年,授河南巡撫,清釐州縣納糧積弊,歲增三十餘萬。十一年,調陜西,引疾歸。十五年,再出撫陜。值黃河西嚙,將與洛通。傳霖增築石壩三十餘座,得無患。中日構釁,遣兵入衛,命兼攝西安將軍。二十一年,擢四川總督。蜀故多盜,特立一軍捕治之。夔、萬大饑,發上游積穀,又採湖北糧米平糶。

是時英、俄交窺西藏,藏番恃俄援,梗英畫界。英嗾廓爾喀與藏構兵,而瞻對土民苦藏官苛虐,思內附。傳霖以瞻對為蜀門戶,瞻不化服,無以威藏番;藏番不聽命,則界無時定。而英之忌俄者益急圖藏,藏亡瞻必隨亡,行且及於蜀。會硃窩、章穀土司爭襲事起,傳霖檄知府羅以禮、知縣穆秉文往諭,以提督周萬順統防邊各軍進駐打箭爐。瞻酋仔仲則忠札霸以兵侵章谷,抗我軍。傳霖乘機進發,迭克諸要害。各土司讋服,率兵聽調。渡雅龍江抵瞻巢,斬馘過當,盡收三瞻地,乃請歸流改漢,條陳善後之策,疏十數上。會成都將軍恭壽、駐藏辦事大臣文海交章言其不便,達賴復疏訴於朝,廷議中變,傳霖解職去。

二十四年,召授廣東巡撫,旋移江蘇,攝兩江總督。二十六年,拳匪亂作,傳霖募三營入衛,奔及乘輿於大同。至太原,授兩廣總督。旋命入直軍機,從幸長安。擢左都御史,遷禮部尚書,兼署工部。明年,回蹕,兼督辦政務大臣。凡疏陳加賦括財、損民以益上者,傳霖率擯勿用;而務汰冗費,去中飽,並奏罷不急之工:均報可。有詔自後宮內供需皆取給內務府,戶部專掌軍國大計,實傳霖發之也。三十年,轉吏部。三十二年,新官制成,乃退直,專治部事。尋仍入直,解部務,以尚書協辦大學士。命查辦歸化城墾務大臣貽穀,論遣戍,參劾不職者數十人。

宣統嗣立,與攝政醇親王同受遺詔,加太子少保,晉太子太保。歷拜體仁閣、東閣大學士,兼經筵講官。二年春,疾作,章四上,皆溫諭慰留。七月,卒,年七十五,贈太保,謚文端。

傳霖起外吏,知民疾苦。所至廉約率下,尤惡貪吏,雖貴勢不稍貰。其在軍機,凡事不茍同,喜扶持善類。晚病重聽,屢乞休不獲,居恆鬱鬱雲。

林紹年,字贊虞,福建閩縣人。同治十三年進士,以編修歷充鄉會試同考官。光緒十四年,改御史。時議修頤和園,先是疆吏籌設海軍經費,輸存北洋,及園工興,陰移其費以助工,號為「進獻」。紹年極陳:「生民疲敝,當以儉化天下,使督撫愛養百姓。若誅求進獻,未足以言忠。請即下詔停輸,還所進奉。」得旨嚴飭。會以憂去,服除,補山西監察御史。疏嚴門禁,杜宦寺交通之漸。十九年,陜西考官丁維禔夤緣內監得試差,復疏論之。

俄,授雲南昭通府知府。邊瘠難治,土目祿爾泰橫暴,睚眥殺人,莫敢訴,猝捕戮之,眾懾而定。期年劾罷文武吏不職者五人。調攝云南府,甫受事,安寧州盜劫貨戕人於塗,州牧以總督崧蕃怒緝捕不力,妄系平民二十餘。紹年覆按,疑其枉,謁總督廷爭,卒獲正犯,出二十餘人者於死。崧蕃愧謝,密疏薦紹年可大用。擢迤南道,未之任,擢貴州按察使。二十六年,遷雲南布政使,就擢巡撫,兼署云貴總督。廣西游匪侵滇邊,遣將擊卻之。招撫八達河村民之陷匪者,以斷賊接濟,益大出兵合剿。滇境既清,乃以全力赴援廣西,而蒙自土匪乘間復發,連陷臨安、石屏。紹年會商總督丁振鐸,檄按察使劉春霖扼通海,廣南軍躡其後,不兩月事平。疏言督撫同城任事非便,自請裁缺,從之。移撫貴州,而湖北、廣東兩巡撫旋亦議裁。印江團首呂志禮、楊鑫不相能,積十餘載,相殘殺。紹年至,以兵脅之降,仍擁眾不散,遂案誅之。

紹年默察大勢,非立憲不足以救亡,請預定政體以系人心,不報。三十一年,移廣西。明年,內召,以侍郎充軍機大臣,兼署郵傳部尚書,授度支部侍郎。時黑龍江新設行省,驟擢道員段芝貴為巡撫。紹年言芝貴望輕,不稱邊帥任。御史趙啟霖劾芝貴,因及慶親王奕劻子載振納賄漁色事,命大臣按驗所劾,稱無左證,褫啟霖職,而芝貴亦由是罷。紹年言御史風聞言事,啟霖無罪,爭之不得,遂稱疾。

出為河南巡撫。以州縣吏罄貲遠宦,人地不習,無益於杜弊。請援漢、唐故事,免避本籍。部議自縣丞以下,如所請行。益飭吏治,得朝貴請託書輒焚之。兩疏糾彈百餘人。調倉場侍郎。

宣統元年,徙民政部侍郎。時奕劻握政柄,陜西巡撫恩壽與有連,總督升允劾其贓私,不報。俄,解升允職。紹年召對論其事,以為賞罰不當,則是非不明。退復具疏言之,不省。二年,充經筵講官,署學部侍郎,改弼德院顧問大臣。以病請告。卒,年六十八,謚文直。

論曰:同、光以後,世稱軍機權重,然特領班王大臣主其事耳。次者僅乃得參機務。光、宣之際,政既失馭,權乃益紛,雖當國無以為治焉。敬銘質樸,之萬練達,傳霖廉約,紹年勁直,其任封疆、治軍旅多有績,而立朝不復有所建樹。敬銘初欲得君專國政,為勢所限,終不能行其志,世尤惜之。


\end{pinyinscope}