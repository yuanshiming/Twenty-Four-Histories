\article{列傳二百六十八}

\begin{pinyinscope}
儒林二

顧炎武張爾岐馬驌萬斯大兄斯選子經侄言

胡渭子彥升葉佩蓀毛奇齡陸邦烈閻若璩李鎧吳玉搢

惠周惕子士奇孫棟餘蕭客陳厚耀臧琳玄孫庸禮堂

任啟運全祖望蔣學鏞董秉純沈彤蔡德晉盛世佐

江永程瑤田褚寅亮盧文弨顧廣圻錢大昕族子塘坫

王鳴盛金曰追吳凌云戴震金榜段玉裁鈕樹玉徐承慶

孫志祖翟灝梁玉繩履繩汪家禧劉臺拱硃彬孔廣森

邵晉涵周永年王念孫子引之李惇賈田祖宋綿初

汪中江德量徐復汪光爔武億莊述祖莊綬甲莊有可

戚學標江有誥陳熙晉李誠丁傑周春

孫星衍畢亨李貽德王聘珍凌廷堪洪榜汪龍

桂馥許瀚江聲孫沅錢大昭子東垣繹侗硃駿聲

顧炎武,字寧人,原名絳,昆山人。明諸生。生而雙瞳,中白邊黑。讀書目十行下。見明季多故,講求經世之學。明南都亡,奉嗣母王氏避兵常熟。昆山令楊永言起義師,炎武及歸莊從之。魯王授為兵部司務,事不克,幸而得脫,母遂不食卒,誡炎武弗事二姓。唐王以兵部職方郎召,母喪未赴,遂去家不返。炎武自負用世之略,不得一遂,所至輒小試之。墾田於山東長白山下,畜牧於山西雁門之北、五臺之東,累致千金。遍歷關塞,四謁孝陵,六謁思陵,始卜居陜之華陰。謂「秦人慕經學,重處士,持清議,實他邦所少;而華陰綰轂關河之口,雖足不出戶,亦能見天下之人、聞天下之事。一旦有警,入山守險,不過十里之遙;若有志四方,則一出關門,亦有建瓴之便」。乃定居焉。

生平精力絕人,自少至老,無一刻離書。所至之地,以二騾二馬載書,過邊塞亭障,呼老兵卒詢曲折,有與平日所聞不合,即發書對勘;或平原大野,則於鞍上默誦諸經注疏。嘗與友人論學云:「百餘年來之為學者,往往言心言性,而茫然不得其解也。命與仁,夫子所罕言;性與天道,子貢所未得聞。性命之理,著之易傳,未嘗數以語人。其答問士,則曰『行己有恥』,其為學,則曰『好古敏求』。其告哀公明善之功,先之以博學。顏子幾於聖人,猶曰『博我以文』。自曾子而下,篤實無如子夏,言仁,則曰『博學而篤志、切問而近思』。今之君子則不然,聚賓客門人數十百人,與之言心言性;舍多學而識以求一貫之方,置四海之困窮不言,而講危微精一;是必其道高於夫子,而其弟子之賢於子貢也。孟子一書,言心言性亦諄諄矣,乃至萬章、公孫丑、陳代、陳臻、周霄、彭更之所問,與孟子之所答,常在乎出處去就辭受取與之間。是故性也、命也、天也,夫子之所罕言,而今之君子之所恆言也。出處去就辭受取與之辨,孔子、孟子之所恆言,而今之君子之所罕言也。愚所謂聖人之道者如之何?曰『博學於文,行己有恥』。自一身以至於天下國家,皆學之事也。自子臣弟友以至出入往來辭受取與之間,皆有恥之事也。士而不先言恥,則為無本之人;非好古多聞,則為空虛之學。以無本之人,而講空虛之學,吾見其日從事於聖人,而去之彌遠也。」

炎武之學,大抵主於斂華就實。凡國家典制、郡邑掌故、天文儀象、河漕兵農之屬,莫不窮原究委,考正得失,撰天下郡國利病書百二十卷;別有肇域志一編,則考索之餘,合圖經而成者。精韻學,撰音論三卷。言古韻者,自明陳第,雖創闢榛蕪,猶未邃密。炎武乃推尋經傳,探討本原。又詩本音十卷,其書主陳第詩無協韻之說,不與吳棫本音爭,亦不用棫之例,但即本經之韻互考,且證以他書,明古音原作是讀,非由遷就,故曰本音。又易音三卷,即周易以求古音,考證精確。又唐韻正二十卷,古音表二卷,韻補正一卷,皆能追復三代以來之音,分部正帙而知其變。又撰金石文字記、求古錄,與經史相證。而日知錄三十卷,尤為精詣之書,蓋積三十餘年而後成。其論治綜覈名實,於禮教尤兢兢。謂風俗衰,廉恥之防潰,由無禮以權之,常欲以古制率天下。炎武又以杜預左傳集解時有闕失,作杜解補正三卷。其他著作,有二十一史年表、歷代帝王宅京記、營平二州地名記、昌平山水記、山東考古錄、京東考古錄、譎觚、菰中隨筆、亭林文集、詩集等書,並有補於學術世道。清初稱學有根柢者,以炎武為最,學者稱為亭林先生。

又廣交賢豪長者,虛懷商榷,不自滿假。作廣師篇云:「學究天人,確乎不拔,吾不如王寅旭;讀書為己,探賾洞微,吾不如楊雪臣;獨精三禮,卓然經師,吾不如張稷若;蕭然物外,自得天機,吾不如傅青主;堅苦力學,無師而成,吾不如李中孚;險阻備嘗,與時屈伸,吾不如路安卿;博聞強記,群書之府,吾不如吳志伊;文章爾雅,宅心和厚,吾不如硃錫鬯;好學不倦,篤於朋友,吾不如王山史;精心六書,信而好古,吾不如張力臣。至於達而在位,其可稱述者,亦多有之,然非布衣之所得議也。」

康熙十七年,詔舉博學鴻儒科,又修明史,大臣爭薦之,以死自誓。二十一年,卒,年七十。無子,吳江潘耒敘其遺書行世。宣統元年,從祀文廟。

張爾岐,字稷若,濟陽人。明諸生。父行素,官石首縣丞,罹兵難,爾岐欲身殉,以母老止。順治七年,貢成均,亦不出。遜志好學,篤守程、硃之說,著天道論、中庸論,為時所稱。又著學辨五篇:曰辨志,曰辨術,曰辨業,曰辨成,曰辨徵。又著立命說辨,斥袁氏功過格、立命說之非。年三十,覃思儀禮,以鄭康成注文古質,賈公彥釋義曼衍,學者不能尋其端緒;乃取經與注章分之,定其句讀,疏其節,錄其要,取其明注而止,有疑義則以意斷之,亦附於末:成儀禮鄭注句讀十七卷,附以監本正誤、石經正誤二卷。顧炎武游山東,讀而善之,曰:「炎武年過五十,乃知『不學禮無以立』。若儀禮鄭注句讀一書,根本先儒,立言簡當,以其人不求聞達,故無當世名,然書實可傳,使硃子見之,必不僅謝監獄之稱許矣。」爾岐又著周易說略八卷,詩說略五卷,蒿菴集三卷,蒿菴閒話二卷。所居敗屋不修,藝蔬果養母,集其弟四人,講說三代古文於母前,愉愉如也。妻硃,婉娩執婦道,勸爾岐勿出,取蓼莪詩意,題其室曰蒿庵,遂教授鄉里終其身。康熙十六年,卒,年六十六。乾隆中,按察使吳江陸燿建蒿菴書院以祀之,而顏其堂曰辨志。山東善治經者,爾岐同時有馬驌。

驌,字宛斯,鄒平人。順治十六年進士,除淮安府推官。尋推官議裁,補靈壁縣知縣。蠲荒除弊,流亡復業。康熙十二年,卒於官,年五十四。士民奉祀名宦祠。驌於左氏融會貫通,著左傳事緯十二卷,附錄八卷,所論有條理,圖表亦考證精詳。驌又撰繹史一百六十卷,纂錄開闢至秦末之事,博引古籍。疏通辨證,非路史、皇王大紀所可及也。時人稱為馬三代。四十四年,聖祖命大學士張玉書物色驌所著書,令人至鄒平購板入內府。

萬斯大,字充宗,鄞縣人。父泰,明崇禎丙子舉人,與陸符齊名。寧波文學風氣,泰實開之。以經、史分授諸子,使從黃宗羲游,各名一家。

斯大治經,以為非通諸經不能通一經;非悟傳注之失,則不能通經;非以經釋經,則亦無由悟傳注之失。其為學尤精春秋、三禮。於春秋,則有專傳論世、屬辭比事、原情定罪諸義;於三禮,則有論社、論禘、論祖宗、論明堂泰壇、論喪服諸義;其辨正商、周改月改時,周詩周正及兄弟同昭穆,皆極確實。宗法十餘篇,亦頗見推衍。答應手為謙書,辨治朝無堂,尤為精覈。根柢三禮,以釋三傳,較宋、元以後空談書法者殊。然其說經以新見長,亦以鑿見短,置其非存其是,未始非一家之學。

斯大性剛毅,慕義若渴。明臣張煌言死後棄骨荒郊,斯大葬之南屏。父執陸符死無後,斯大為葬其兩世六棺。所著有學春秋隨筆十卷,學禮質疑二卷,儀禮商三卷,禮記偶箋三卷,周官辨非二卷。康熙二十二年,卒,年六十。

兄斯選,字公擇。學於黃宗羲。嘗謂學者須驗之躬行,方為實學。於是切實體認,知意為心之存主,非心之所發。理即在氣中,非理先氣後。涵養純粹,年六十,卒。宗羲哭之慟,曰:「甬上從游,能續蕺山之傳者,惟斯選一人,而今已矣!」

斯大子經,字授一。黃宗羲移證人書院於鄞,申明劉宗周之學。經侍席末,與聞其教。及長,傳父、叔及兄言之學,又學於應手為謙、閻若璩。康熙四十二年,成進士,選庶吉士,散館授編修。五十年,充山西鄉試副考官。五十三年,提督貴州學政。及還,以派修通州城工罄其家。素工分隸,經乃賣所作字,得錢給朝夕。晚增補斯大禮記集解數萬言,春秋定、哀二公未畢,又續纂數萬言。又重修斯同列代紀年,又續纂兄言尚書說、明史舉要,皆先代未成之書。乾隆初,舉博學鴻詞科,不就。年八十二,家遭大火,遺書悉焚。經終日涕洟,自以為負罪先人,逾年卒。著有分隸偶存二卷。

言,字貞一,斯選兄斯年子。副榜貢生。少隨諸父講社中,號精博。著有尚書說、明史舉要。嘗與修明史,獨成崇禎長編,故國輔相子弟多以賄求減先人罪,言悉拒之。尤工古文,同縣李鄴嗣嘗曰:「事古而信,篤志不分,吾不如充宗;粹然有得,造次儒者,吾不如公擇;學通古今,無所不辨,吾不如季野;文章名世,居然大家,吾不如貞一。吾邑有萬氏,誠天下之望。」有管村文集。晚出為五河知縣,忤大吏,論死,子承勛,狂走數千里,裒金五千贖之歸,時稱孝子。

承勛,字開遠。諸生。以薦,用為磁州知州。工詩,有冰雪集。

胡渭,初名渭生,字朏明,德清人。渭年十二而孤,母沈,攜之避亂山谷間。十五為縣學生,入太學,篤志經義,尤精輿地之學。嘗館大學士馮溥邸。尚書徐乾學奉詔修一統志,開局洞庭山,延常熟黃儀、顧祖禹,太原閻若璩及渭分纂。渭著禹貢錐指二十卷,圖四十七篇。謂漢、唐二孔氏,宋蔡氏,於地理多疏舛。如三江當主鄭康成說;禹貢「達於河」,「河」當從說文作「菏」;「滎波既豬」,當從鄭康成作「播」;梁州黑水與導川之黑水,不可溷為一。乃博稽載籍,考其同異而折衷之。山川形勢,郡國分合,道里遠近夷險,一一討論詳明。又漢、唐以來,河道遷徙,為民生國計所系,故於導河一章,備考決溢改流之跡,留心經濟,異於迂儒不通時務。間有千慮一失,則不屑闕疑之過。

又撰易圖明辨十卷,專為辨定圖、書而作。初,陳摶推闡易理衍為諸圖,其圖本準易而生,故以卦爻反覆研求無不符合。傳者務神其說,遂歸其圖於伏羲,謂易反由圖而作。又因系辭「河圖、洛書」之文,取大衍算數作五十五點之圖,以當河圖;取乾鑿度太乙行九宮法,造四十五點之圖,以當洛書;其陰陽奇偶,亦一一與易相應。傳者益神其說,又真以為龍馬神龜之所負,謂伏羲由此而有先天之圖。實則唐以前書絕無一字符驗,而突出於北宋之初,由邵子以及硃子,亦但取其數之巧合,而未暇究其太古以來從誰授受,故易學啟蒙、易本義前九圖皆沿其說。同時袁樞、薛季宣皆有異論,然宋史儒林傳:易學啟蒙硃子本囑蔡元定創槁,非硃子自撰,晦菴大全集載答劉君房書曰:「啟蒙本欲學者且就大傳所言卦畫蓍數推尋,不須過為浮說。而自今觀之,如河圖、洛書,亦不免尚有賸語。」至於本義卷首九圖,為門人所依附,硃子當日未嘗堅主其說。元陳應潤作爻變義蘊,始指諸圖為道家假借。吳澄、歸有光諸人亦相繼排擊,毛奇齡、黃宗羲爭之尤力。然皆各據所見抵其罅隙,尚未能窮溯本末,一一抉所自來。渭則於河圖、洛書,五行、九宮,參同、先天、太極,龍圖,易數鉤隱圖,啟蒙圖、書,先天、後天、卦變、象數流弊,皆引據舊文,互相參證,以箝依託之口。使學者知圖、書之說,乃修鍊、術數二家旁分易學之支流,非作易之根柢,視禹貢錐指尤為有功經學。

又撰洪範正論五卷,謂漢人專取災祥,推衍五行,穿鑿附會,事同讖緯,亂彞倫攸敘之經,其害一;洛書本文具在洪範,非龜文,宋儒創為黑白之點,方員之體,九十之位,變書為圖,以至九數十數,劉牧、蔡季通紛紜更定,其害二;洪範元無錯簡,王柏、胡一中等任意改竄,其害三。渭又撰大學翼真七卷,大旨以硃子為主,僅謂格致一章不必補傳,力闢王學改本之誤。所見切實,視泛為性命理氣之談者,勝之遠矣。

渭經術湛深,學有根柢,故所論一軌於正。漢儒傅會之談,宋儒變亂之論,掃而除焉。康熙四十三年,聖祖南巡,渭以禹貢錐指獻行在,聖祖嘉獎,御書「耆年篤學」四大字賜之,儒者咸以為榮。五十三年,卒,年八十有二。

渭子彥升,字國賢。雍正八年進士,授刑部主事,改山東定陶縣知縣。著春秋說、四書近是、叢書錄要。又於樂律尤有心得,著樂律表微八卷。

渭同郡葉佩蓀,字丹穎,歸安人。亦治古易,不言圖、書,著易守四十卷。於易中三聖人所未言者不加一字,故曰「守」。

毛奇齡,字大可,又名甡,蕭山人。四歲,母口授大學即成誦。總角,陳子龍為推官,奇愛之,遂補諸生。明亡,哭於學宮三日。山賊起,竄身城南山,築土室,讀書其中。

順治三年,明保定伯毛有倫以寧波兵至西陵,奇齡入其軍中。是時馬士英、方國安與有倫犄角,奇齡曰:「方、馬國賊也,明公為東南建義旗,何可與二賊共事?」國安聞之大恨,欲殺之,奇齡遂脫去。後怨家屢陷之,乃變姓名為王士方,亡命浪游。及事解,以原名入國學。康熙十八年,薦舉博學鴻儒科,試列二等,授翰林院檢討,充明史纂修官。二十四年,充會試同考官,尋假歸,得痺疾,遂不復出。

初著毛詩續傳三十八卷,既以避仇流寓江、淮間,失其槁,乃就所記憶著國風省篇、詩札、毛詩寫官記。復在江西參議道施閏章所與湖廣楊洪才說詩,作白鷺洲主客說詩一卷。明嘉靖中,鄞人豐坊偽造子貢詩傳、申培詩說行世,奇齡作詩傳詩說駁議五卷,引證諸書,多所糾正。洎通籍,進所著古今通韻十二卷,聖祖善之,詔付史館。

歸田後,僦居杭州,著仲氏易,一日著一卦,凡六十四日而書成,託於其兄錫齡之緒言,故曰「仲氏」。又著推易始末四卷,春秋占筮書三卷,易小帖五卷,易韻四卷,河圖洛書原舛編一卷,太極圖說遺議一卷。其言易發明荀、虞、乾、侯諸家,旁及卦變、卦綜之法。奇齡分校會闈時,閱春秋房卷,心非胡傳之偏,有意撰述,至是乃就經文起義,著春秋毛氏傳三十六卷,春秋簡書刊誤二卷,春秋屬辭比事記四卷,條例明晰,考據精核。又欲全著禮經,以衰病不能,乃次第著昏、喪、祭禮、宗法、廟制及郊、社、禘、祫、明堂、學校諸問答,多發先儒所未及。至於論語、大學、中庸、孟子,各有考證,而大學證文及孝經問,援據古今,辨後儒改經之非,持論甚正。

奇齡淹貫群書,所自負者在經學,然好為駁辨,他人所已言者,必力反其詞。古文尚書自宋吳棫後多疑其偽,及閻若璩作疏證,奇齡力辨為真,遂作古文尚書冤詞。又刪舊所作尚書廣聽錄為五卷,以求勝於若璩,而周禮、儀禮,奇齡又以為戰國之書。所作經問,指名攻駁者,惟顧炎武、閻若璩、胡渭三人。以三人博學重望,足以攻擊,而餘子以下不足齒錄,其傲睨如此。

素曉音律,家有明代宗籓所傳唐樂笛色譜,直史館,據以作竟山樂錄四卷。及在籍,聞聖祖論樂諭群臣以徑一圍三隔八相生之法,因推闡考證,撰聖諭樂本解說二卷,皇言定聲錄八卷。三十八年,聖祖南巡,奇齡迎駕於嘉興,以樂本解說二卷進,溫諭獎勞。聖祖三巡至浙,奇齡復謁行在,賜御書一幅。五十二年,卒於家,年九十一。門人蔣樞編輯遺集,分經集、文集二部,經集自仲氏易以下凡五十種,文集合詩、賦、序、記及他雜著凡二百三十四卷。四庫全書收奇齡所著書目多至四十餘部。奇齡辨正圖、書,排擊異學,尤有功於經義。弟子李恭、陸邦烈、盛唐、王錫、章大來、邵廷寀等,著錄者甚眾。李恭、廷寀自有傳。

邦烈,字又超,平湖人。嘗取奇齡經說所載裒為聖門釋非錄五卷,謂聖問口語未可盡非雲。

閻若璩,字百詩,太原人。世業鹽筴,僑寓淮安。父修齡,以詩名家。若璩幼多病,讀書闇記不出聲,年十五,以商籍補山陽縣學生員。研究經史,深造自得。嘗集陶弘景、皇甫謐語題其柱云:「一物不知,以為深恥;遭人而問,少有暇日。」其立志如此。海內名流過淮,必主其家。年二十,讀尚書至古文二十五篇,即疑其譌。沉潛三十餘年,乃盡得其癥結所在,作古文尚書疏證八卷。引經據古,一一陳其矛盾之故,古文之偽大明。所列一百二十八條,毛奇齡尚書古文冤詞百計相軋,終不能以強辭奪正理,則有據之言先立於不可敗也。

康熙元年,游京師,旋改歸太原故籍,補廩膳生。十八年,應博學鴻儒科試,報罷。昆山顧炎武以所撰日知錄相質,即為改定數條,炎武虛心從之。編修汪琬著五服考異,若璩糾其謬,尚書徐乾學嘆服。及乾學奉敕修一統志,開局洞庭山,若璩與其事。若璩於地理尤精審,山川形勢,州郡沿革,了如指掌,撰四書釋地五卷,及於人名物類訓詁典制,事必求其根柢,言必求其依據,旁參互證,多所貫通。又據孟子七篇,參以史記諸書,作孟子生卒年月考一卷。又著潛丘劄記六卷,毛硃詩說一卷,手校困學紀聞二十卷,因浚儀之舊而駮正箋說推廣之。又有日知錄補正,喪服異注,宋劉攽、李燾、馬端臨、王應麟四家逸事,博湖掌錄諸書。

世宗在潛邸聞其名,延人邸中,索觀所著書,每進一篇必稱善。疾革,請移就城外,以大床為輿,上施青紗帳,二十人舁之出,安穩如床簀。康熙四十三年,卒,年六十九。世宗遣使經紀其喪,親制詩四章,復為文祭之。有云:「讀書等身,一字無假,孔思周情,旨深言大。」僉謂非若璩不能當也。

子詠。康熙四十八年進士,官中書舍人,亦能文。同時山陽學者,有李鎧、吳玉搢。

鎧,字公凱。順治十八年進士,補奉天蓋平縣知縣。康熙十八年,薦應博學鴻儒科試,授翰林院編修,與修明史,洊官內閣學士。所著有讀書雜述、史斷,王士禎稱為有本之學。

玉搢,字藉五。官鳳陽府訓導。著山陽志遺、金石存、說文引經考、六書述部敘考,又著別雅五卷,辨六書之假借,深為有功,非俗儒剽竊所能徬彿也。

惠周惕,字元龍,原名恕,吳縣人。父有聲,以九經教授鄉里,與徐枋善。周惕少從枋游,又曾受業於汪琬。康熙十八年,舉博學鴻儒科,丁憂,不與試。三十年,成進士,選翰林院庶吉士。散館,改密雲縣知縣,有善政,卒於官。

周惕邃於經學,為文章有矩度,著有易傳、春秋三禮問及硯谿詩文集。其詩說二卷,謂大、小雅以音別,不以政別。謂正雅、變雅美刺錯陳,不必分六月以上為正、六月以下為變;文王以下為正、民勞以下為變。謂二南二十六篇,皆房中之樂,不必泥其所指何人。謂天子諸侯均得有頌,魯頌非僭,其言並有依據。清二百餘年談漢儒之學者,必以東吳惠氏為首。惠氏三世傳經,周惕其創始者也。

子士奇,字天牧。康熙五十年進士,選翰林院庶吉士,授編修。兩充會試同考官。聖祖嘗問廷臣,誰工作賦,內閣學士蔣廷錫以王頊齡、湯右曾及士奇三人對。五十七年,孝惠章皇后升祔禮成,特命祭告炎帝陵、舜陵。故事,祭告使臣,學士以上乃得開列,士奇以編修與,異數也。五十九年,充湖廣鄉試正考官,尋提督廣東學政,以經學倡多士,三年之後,通經者多。又謂:「校官古博士也,校官無博士之才,弟子何所效法?」訪得海陽進士翁廷資,即具疏題補韶州府學教授,部議格不行。聖祖曰:「惠士奇所舉,諒非徇私,著如所請,後不為例。」

雍正初,復命留任。召還,入對不稱旨,罰修鎮江城,以產盡停工削籍,乾隆元年,復起為侍讀,免欠修城銀,令纂修三禮。越四年,告歸,卒於家。

士奇盛年兼治經史,晚尤邃於經學,撰易說六卷,禮說十四卷,春秋說十五卷。於易,雜釋卦爻,以象為主,力矯王弼以來空疏說經之弊。於禮,疏通古音、古字,俱使無疑似,復援引諸子百家之文,或以證明周制,或以參考鄭氏所引之漢制,以遞觀周制,而各闡其制作之深意。於春秋,事實據左氏,論斷多採公、穀,大致出於宋張大亨春秋五禮例宗、沈棐春秋比事,而典核過之。大學說一卷晚出,「親民」不讀「新民」。論格物不外本末終始先後,即絜矩之不外上下前後左右,亦能根極理要,又著交食舉隅三卷,琴笛理數考四卷。子七人,棟最知名。

棟,字定宇。元和學生員。自幼篤志向學,家多藏書,日夜講誦。於經、史、諸子、稗官野乘及七經毖緯之學,靡不津逮。小學本爾雅,六書本說文,餘及急就章,經典釋文,漢、魏碑碣,自玉篇、廣韻而下勿論也。乾隆十五年,詔舉經明行修之士,陜甘總督尹繼善、兩江總督黃廷桂交章論薦。會大學士、九卿索所著書,未及呈進,罷歸。

棟於諸經熟洽貫串,謂詁訓古字古音,非經師不能辨,作九經古義二十二卷。尤邃於易,其撰易漢學八卷,掇拾孟喜、虞翻、荀爽緒論,以見大凡。其末篇附以己意,發明漢易之理,以辨正河圖、洛書、先天、太極之學。易例二卷,乃鎔鑄舊說以發明易之本例,實為棟論易諸家發凡。其撰周易述二十三卷,以荀爽、虞翻為主,而參以鄭康成、宋咸、干寶之說,約其旨為注,演其說為疏。書垂成而疾革,遂闕革至未濟十五卦及序卦、雜卦兩傳,雖為未善之書,然漢學之絕者千有五百餘年,至是而粲然復明。撰明堂大道錄八卷,禘說二卷,謂禘行於明堂,明堂法本於易。古文尚書考二卷,辨鄭康成所傳之二十四篇為孔壁真古文,東晉晚出之二十五篇為偽。又撰後漢書補注二十四卷,王士禎精華錄訓纂二十四卷,九曜齋筆記、松崖文鈔諸書。嘉定錢大昕嘗論:「宋、元以來說經之書盈屋充棟,高者蔑古訓以言誇心得,下者襲人言以為己有。獨惠氏世守古學,而棟所得尤精。擬諸前儒,當在何休、服虔之間,馬融、趙岐輩不及也。」卒,年六十二。其弟子知名者,餘蕭客、江聲最為純實。

蕭客,字古農,長洲人。撰古經解鉤沉三十卷,凡唐以前舊說,自諸家經解所引,旁及史傳、類書,片語單詞,悉著於錄。清代經學昌明,著述之家,爭及於古,蕭客是書其一也。蕭客又撰文選紀聞三十卷,文選音義八卷。聲自有傳。

陳厚耀,字泗源,泰州人。康熙四十五年進士,官蘇州府學教授。大學士李光地薦其通天文、算法,引見,改內閣中書。上命試以算法,繪三角形,令求中線及弧背尺寸,厚耀具劄以進,皆如式。授翰林院編修,入直內廷。厚耀學問淵博,直內廷後,兼通幾何算法,於是其學益進。遷國子監司業,轉左春坊左諭德,以老乞致仕,卒於家。

厚耀以天算之法治春秋,嘗補杜預長歷為春秋長歷十卷,其凡有四:一曰歷證,備引漢書、續漢書、晉書、隋書、唐書、宋史、元史、左傳注疏、春秋屬辭、天元歷理諸說,以證推步之異。其引春秋屬辭載杜預論日月差謬一條,為注疏所無。又引大衍歷義春秋歷考一條,亦唐志所未錄。二曰古歷,以古法十九年為一章,一章之首,推合周歷正月朔日冬至,前列算法,後以春秋十二公紀年,橫列為四章,縱列十二公,積而成表,以求歷元。三曰歷編,舉春秋二百四十二年,推其朔閏及月之大小,而以經、傳干支為證佐,述杜預之說而考辨之。四曰歷存,古歷推隱公元年正月庚戌朔,杜氏長歷則為辛巳朔,乃古歷所推上年十二月朔,謂元年以前失一閏,蓋以經、傳干支排次知之。厚耀則謂如預之說,元年至七年中書日者雖多不失,而與二年八月之庚辰、四年二月之戊申又不能合。且隱公三年二月己巳朔日食,桓公三年七月壬辰朔日食,亦皆失之。蓋隱公元年以前非失一閏,乃多一閏,因定隱公元年正月為庚辰朔,較長歷退兩月,推至僖公五年止。以下朔、閏,一一與杜歷相符,故不復續推焉。

又撰春秋戰國異辭五十四卷、通表二卷、摭遺一卷,春秋世族譜一卷。鄒平馬驌為繹史,兼採三傳、國語、國策,厚耀則皆摭於五書之外,獨為其難。氏族一書,與顧棟高大事表互證,春秋氏族之學,幾乎備矣。厚耀又著禮記分類、十七史正譌諸書,今不傳。

臧琳,字玉林,武進人。諸生。治經以漢注唐疏為主,教人先以爾雅、說文,曰:「不解字,何以讀書?不通訓詁,何以明經?」鍵戶著述,世無知者。有尚書集解百二十卷,經義雜記三十卷。閻若璩稱其深明兩漢之學,錢大昕校定其書,云:「實事求是,別白精審,而未嘗輕詆前哲,斯真務實而不近名者。」

玄孫庸,本名鏞堂,字在東。與弟禮堂俱事錢塘盧文弨。沉默樸厚,學術精審。續其高祖將絕之學,儗經義雜記為拜經日記八卷,高郵王念孫亟稱之。其敘孟子年譜,辨齊宣王、湣王之譌,閩縣陳壽祺嘆為絕識。又著拜經文集四卷,月令雜說一卷,樂記二十三篇注一卷,孝經考異一卷,子夏易傳一卷,詩考異四卷,韓詩遺說二卷、訂譌一卷,校鄭康成易注二卷。其輯子夏易傳,辨此傳為漢韓嬰作,非卜子夏。其詩考異大旨如王伯厚,但逐條必自考輯,不依循王本。庸初因寶應劉臺拱獲交儀徵阮元,其後館元署中為多。元寫其書為副本,以原本還其家。嘉慶十六年,卒,年四十五。

禮堂,字和貴。事親孝。父繼宏,久瘧,冬月畏火,禮堂潛以身溫被。居喪如禮,笑不見齒。母遘危疾,刲股合藥,私禱於神,減齒以延親壽。娶婦胡,初婚夕教以孝弟,長言令熟聽,乃合巹,一家感而化之。尤精小學,善讎校,為四方賢士所貴。師事錢大昕,業益進。好許氏說文解字,為說文經考十三卷。慕古孝子、孝女、孝婦事,作孝傳百數十卷。尚書集解案六卷,三禮注校字六卷,春秋注疏校正六卷。卒,年三十。

任啟運,字翼聖,宜興人。少讀孟子,至卒章,輒哽咽,大懼道統無傳。家貧,無藏書,從人借閱。夜乏膏火,持書就月,至移墻不輟。事父母孝以聞。年五十四,舉於鄉。雍正十一年,計偕至都,會世宗問有精通性理之學者,尚書張照以啟運名上。特詔廷試,以「太極似何物」對,進呈御覽,得旨嘉獎。會成進士,遂於臚唱前一日引見,特授翰林院檢討,在阿哥書房行走。上嘗問以「朝聞夕死」之旨,啟運對以「生死一理,未知生,焉知死」。上曰:「此是賢人分上事,未到聖人地位。從此作去,久自知之。」逾年抱疾,賜藥賜醫,越月謝恩,特諭繞廊而進。面稱:「知汝非堯、舜不敢以陳於王前。」務令自愛。令侍臣扶掖以出,且遙望之。

高宗登基,仍命在書房行走,署日講起居注官,尋擢中允。乾隆四年,遷侍講,晉侍講學士。七年,擢都察院左僉都御史。八年,充三禮館副總裁官,尋升宗人府府丞。九年,卒於賜第,年七十五。賜帑金治喪具,賜祭葬。

啟運學宗硃子,嘗謂諸經已有子硃子傳,獨未及禮經,乃著肆獻祼饋食禮三卷。以儀禮特性、少牢、饋食禮皆士禮,因據三禮及他傳記之有關王禮者推之,不得於經,則求諸注疏以補之,凡五篇:一曰祭統,二曰吉蠲,三曰朝踐,四曰正祭,五曰繹祭。其名則取周禮「以肆獻祼享先王」、「以饋食享先王」之文,較之黃幹所續祭禮,更為精密。又宮室考十三卷,於李如圭釋宮之外別為類次:曰門,曰觀,曰朝,曰廟,曰寢,曰塾,曰寧,曰等威,曰名物,曰門大小廣狹,曰明堂,曰方明,曰闢雍,考據頗為精核。儀禮一經,久成絕學,啟運研究鉤貫,使條理秩然,不愧窮經之目。又禮記章句十卷,以大學、中庸,硃子既成章句,則曲禮以下四十七篇,皆可釐為章句。但所傳篇次序列紛錯,爰仿鄭康成序儀禮例,更其前後,並為四十二篇。其有關倫紀之大,而為秦、漢、元、明輕變易者,則眾著其說,以俟後之論禮者酌取。外有周易洗心九卷,四書約指十九卷,孝經章句十卷,夏小正注,竹書紀年考,逸書補,孟子時事考,清芬樓文集等書,其周易洗心則年六十時作,觀象玩辭,時闡精理。

啟運研窮刻苦,既受特達之知,益思報稱。年七十二,猶書自責語曰:「孔、曾、思、孟,實惟汝師。日面命汝,汝頑不知,痛自懲責,涕泗漣洏。嗚呼老矣,瞑目為期。」及總裁三禮館,喜甚,因盡發中秘所儲,平心參訂,目營手寫,漏常二十刻不輟。論必本天道,酌人情,務求合硃子遺意,而心神煎耗,竟以是終。

十四年,詔舉經學,上諭有「任啟運研窮經術,敦樸可嘉」之語。三十七年,命中外蒐集古今群書,高宗諭曰:「歷代名臣,洎本朝士林夙望,向有詩文專集及近時沉潛經史,原本風雅,如顧棟高、陳祖範、任啟運、沈德潛輩,亦各著成編,並非剿說卮言可比。均應概行查明,在坊肆者或量為給價,家藏者或官為裝印。至有未經鐫刊祗系鈔本存留者,不妨鈔錄副本,仍將原本給還。庶幾副在石渠,用儲一覽。」於是上啟運所著書四種,入四庫中。

全祖望,字紹衣,鄞縣人。十六歲能為古文。討論經史,證明掌故。補諸生。雍正七年,督學王蘭生選以充貢,入京師,旋舉順天鄉試。戶部侍郎李紱見其文,曰:「此深寧、東發後一人也!」乾隆元年,薦舉博學鴻詞。是春會試,先成進士,選翰林院庶吉士,不再與試。時張廷玉當國,與李紱不相能,並惡祖望,祖望又不往見,二年,散館,寘之最下等,歸班以知縣用,遂不復出。方詞科諸人未集,紱以問祖望,祖望為記四十餘人,各列所長。性伉直,既歸,貧且病,饔飧不給,人有所餽,弗受。主蕺山、端谿書院講席,為士林仰重。二十年,卒於家,年五十有一。

祖望為學,淵博無涯涘,於書無不貫串。在翰林,與紱共借永樂大典讀之,每日各盡二十卷。時開明史館,復為書六通移之,先論藝文,次論表,次論忠義、隱逸兩列傳,皆以其言為韙。生平服膺黃宗羲,宗羲表章明季忠節諸人,祖望益廣修枌社掌故、桑海遺聞以益之,詳盡而核實,可當續史。宗羲宋元學案甫創草槁,祖望博採諸書為之補輯,編成百卷。又七校水經注,三箋困學紀聞,皆足見其汲古之深。又答弟子董秉純、張炳、蔣學鏞、盧鎬等所問經史疑義,錄為經史問答十卷。儀徵阮元嘗謂經學、史才、詞科三者得一足傳,而祖望兼之。其經史問答,實足以繼古賢,啟後學,與顧炎武日知錄相埒。晚年定文槁,刪其十七,為鮚埼亭文集五十卷。

弟子同縣蔣學鏞,字聲始。乾隆三十六年舉人。從祖望得聞黃、萬學派,學鏞尤得史學之傳。

董秉純,字小鈍。乾隆十八年拔貢,補廣西那地州州判,升秦安縣知縣。全祖望文內、外集,均秉純一手編定。

沈彤,字果堂,吳江人。自少力學,以窮經為事。貫串前人之異同,折衷至當。乾隆元年,薦舉博學鴻詞報罷,與修三禮及一統志。書成,授九品官,以親老歸。

彤淹通三禮,以歐陽修有周禮官多田少,祿且不給之疑,後人多沿其說,即有辨者,不過以攝官為詞。乃詳究周制,撰周官祿田考,以辨正歐說。分官爵數、公田數、祿田數三篇,積算至為精密。其說自鄭注、賈疏以後,可云特出。又撰儀禮小疏一卷,取士冠禮、士昏禮、公食大夫禮、喪服、士喪禮為之疏箋,足訂舊義之譌。其果堂集十二卷,多訂正經學之文,若周官頒田異同說,五溝異同說,井田軍賦說,釋周官地征等篇,皆援據典核。又撰春秋左氏傳小疏,尚書小疏,氣穴考略,內經本論。

彤性至孝,親歿,三年中不茹葷,不內寢。居恆每講求經世之務,所著保甲論,其後吳德旋見之,稱為最善云。卒,年六十五。

蔡德晉,字仁錫,無錫人。雍正四年舉人。乾隆二年,禮部尚書楊名時薦德晉經明行修,授國子監學正,遷工部司務。德晉嘗謂橫渠以禮教人,最得孔門博約之旨,故其律身甚嚴。其論三禮,多前人所未發。著禮經本義十七卷,禮傳本義二十卷,通禮五十卷。

盛世佐,字庸三,秀水人。官貴州龍裡知縣。撰儀禮集編四十卷,集眾解而研辨之,持論謹嚴。又楊復儀禮圖久行於世,然其說本注疏,而時有並注疏之意失之者,一一是正,至於諸家謬誤,辨之尤詳焉。

江永,字慎修,婺源人。為諸生數十年,博通古今,專心十三經注疏,而於三禮功尤深。以硃子晚年治禮,為儀禮經傳通解。書未就,黃氏、楊氏相繼纂續,亦非完書。乃廣摭博討,大綱細目,一從吉、兇、軍、嘉、賓五禮舊次,題曰禮經綱目,凡八十八卷。引據諸書,釐正發明,實足終硃子未竟之緒。嘗一至京師,桐城方苞、荊谿吳紱質以禮經疑義,皆大折服。讀書好深思,長於比勘,明推步、鐘律、聲韻。歲實消長,前人多論之者,梅文鼎略舉授時,而亦疑之。永為之說,當以恆氣為率,隨其時之高沖以算定氣,而歲實消長勿論,其說至為精當。其論黃鍾之宮,據管子、呂氏春秋以正淮南子,其論古韻平、上、去三聲,皆當為十三部,入聲當為八部,而三代以上之音,始有條不紊。晚年讀書有得,隨筆撰記。謂周易以反對為次序,卦變當於反對取之。否反為泰,泰反為否,故「小往大來」,「大往小來」,是其例也。凡曰來、曰下、曰反,自反卦之外卦來居內卦也。曰往、曰上、曰進、曰升,自反卦之內卦往居外卦也。又謂兵、農之分,春秋時已然,不起於秦、漢。證以管子、左傳,兵常近國都,野處之農固不隸於師旅也。其於經、傳稽考精審多類此。

所著有周禮疑義舉要七卷,禮記訓義擇言六卷,深衣考誤一卷,律呂闡微十卷,律呂新論二卷,春秋地理考實四卷,鄉黨圖考十一卷,讀書隨筆十二卷,古韻標準四卷,四聲切韻表四卷,音學辨微一卷,河洛精蘊九卷,推步法解五卷,七政衍、金水二星發微、冬至權度、恆氣注歷辨、歲實消長辨、歷學補論、中西合法擬草各一卷,近思錄集注十四卷,考訂硃子世家一卷。乾隆二十七年,卒,年八十二。弟子甚眾,而戴震、程瑤田、金榜尤得其傳。雲、榜自有傳。

瑤田,字易疇,歙人。讀書好深沉之思,學於江氏。乾隆三十五年舉人,選授太倉州學正。以身率教,廉潔自持。告歸之日,錢大昕、王鳴盛皆贈詩推重,至與平湖陸隴其並稱。嘉慶元年,舉孝廉方正。同時舉者,推錢大昭、江聲、陳鱣三人,阮元獨謂瑤田足以冠之。平生著述,長於旁搜曲證,不屑依傍傳注,所著曰喪服足徵記,宗法小記,溝洫疆里小記,禹貢三江考,九穀考,磬折古義,水地小記,解字小記,聲律小記,考工創物小記,釋草釋蟲小記。年老目盲,猶口授孫輩成琴音記。東原戴氏自謂尚遜其精密。

褚寅亮,字搢升,長洲人。乾隆十六年召試舉人,授內閣中書,官至刑部員外郎。寅亮少以博雅名,心思精銳,於史書魯魚,一見便能訂其誤謬。中年覃精經術,一以注疏為歸。從事禮經幾三十年,墨守家法,專主鄭學。鄭氏周禮、禮記注,妄庸人群起嗤點之,獨儀禮為孤學,能發揮者固絕無,而謬加指摘者亦尚少。惟敖繼公集說,多巧竄經文,陰就己說。後儒苦經注難讀,喜其平易,無疵之者。萬斯大、沈彤於鄭注亦多所糾駮,至張爾岐、馬駉但粗為演繹,其於敖氏之似是而非,均未能正其失也。寅亮著儀禮管見三卷,於敖氏洞見其癥結,驅豁其雺霧。

時公羊何氏學久無循習者,所謂五始、三科、九旨、七等、六輔、二類之義,不傳於世,惟武進莊存與默會其解,而寅亮能闡發之,撰公羊釋例三十篇。謂三傳惟公羊為漢學,孔子作春秋,本為後王制作,訾議公羊者,實違經旨。又因何劭公言禮有殷制,有時王之制,與周禮不同,作周禮公羊異義二卷,世稱為絕業。又長於算術,著句股廣問三卷,校正三統術衍刊本誤字甚多,其中月相求六扐之數句,六扐當作七扐;推閏餘所在加十得一句,加十當作加七:皆寅亮說也。

著有十三經筆記十卷,諸史筆記八卷,諸子筆記二卷,名家文集筆記七卷,藏於家。四十六年,以病告歸,主常州龍城書院八年。五十五年,卒,年七十六。

盧文弨,字召弓,餘姚人。父存心,乾隆初舉博學鴻詞科。文弨,乾隆十七年一甲進士,授翰林院編修,上書房行走。歷官左春坊左中允、翰林院侍讀學士。三十年,充廣東鄉試正考官。三十一年,提督湖南學政,以條陳學政事宜,部議降三級用。三十三年,乞養歸。

文弨孝謹篤厚,潛心漢學,與戴震、段玉裁友善。好校書,所校逸周書、孟子音義、荀子、呂氏春秋、賈誼新書、韓詩外傳、春秋繁露、方言、白虎通、獨斷、經典釋文諸善本,鏤板惠學者。又苦鏤板難多,則合經、史、子、集三十八種而名之曰群書拾補。所自著書有抱經堂集三十四卷,儀禮注疏詳校十七卷,鍾山劄記四卷,龍城劄記三卷,廣雅釋天以下注二卷,皆使學者諟正積非,蓄疑渙釋。其言曰:「唐人之為義疏也,本單行,不與經注合。單行經注,唐以後尚多善本,自宋後附疏於經注,而所附之經注非必孔、賈諸人所據之本也,則兩相齟矣。南宋後又附經典釋文於注疏間,而陸氏所據之經注,又非孔、賈諸人所據也,則齟更多矣。淺人必比而同之,則彼此互改,多失其真,幸有改之不盡,以滋其齟,啟人考核者,故注疏、釋文合刻,似便而非古法也。」其特識多類此。

文弨歷主江、浙各書院講席,以經術導士,江、浙士子多信從之,學術為之一變。六十年,卒,年七十九。

文弨校書,參合各本,擇善而從,頗引他書改本書,而不專主一說,故嚴元照詆其儀禮詳校,顧廣圻譏其釋文考證,後黃丕烈影宋刻書,各本同異另編於後,兩家各有宗旨,亦互相補苴云。

顧廣圻,字千里,元和人。諸生。吳中自惠氏父子後,江聲繼之,後進翕然多好古窮經之士。廣圻讀惠氏書,盡通其義。論經學云:「漢人治經,最重師法。古文今文,其說各異。若混而一之,則轇轕不勝矣。」論小學云:「說文一書,不過為六書發凡,原非字義盡於此。」

廣圻天質過人,經、史、訓詁、天算、輿地靡不貫通,至於目錄之學,尤為專門,時人方之王仲寶、阮孝緒。兼工校讎,同時孫星衍、張敦仁、黃丕烈、胡克家延校宋本說文、禮記、儀禮、國語、國策、文選諸書,皆為之札記,考定文字,有益後學。乾、嘉間以校讎名家,文弨及廣圻為最著雲。又時為漢學者多譏宋儒,廣圻獨取先儒語錄,摘其切近者,為遯翁苦口一卷,以教學者。著有思適齋文集十八卷。道光十九年,卒,年七十。

錢大昕,字曉徵,嘉定人。乾隆十六年召試舉人,授內閣中書。十九年進士,選翰林院庶吉士,散館授編修。大考二等一名,擢右春坊右贊善。累充山東鄉試、湖南鄉試正考官,浙江鄉試副考官。大考一等三名,擢翰林院侍講學士。三十二年,乞假歸。三十四年,補原官。入直上書房,遷詹事府少詹事,充河南鄉試正考官。尋提督廣東學政。四十年,丁父艱,服闋,又丁母艱,病不復出。嘉慶九年,卒,年七十七。

大昕幼慧,善讀書。時元和惠棟、吳江沈彤以經術稱,其學求之十三經注疏,又求之唐以前子、史、小學。大昕推而廣之,錯綜貫串,發古人所未發。任中書時,與吳烺、褚寅亮同習梅氏算術。及入翰林,禮部尚書何國宗世業天文,年已老,聞其善算,先往見之,曰:「今同館諸公談此道者鮮矣。」

大昕於中、西兩法,剖析無遺。用以觀史,自太初、三統、四分,中至大衍,下迄授時,朔望薄蝕,凌犯進退,抉摘無遺。漢三統術為七十餘家之權輿,訛文奧義,無能正之者。大昕衍之,據班志以闡劉歆之說,裁志文之訛,二千年已絕之學,昭然若發蒙。大昕又謂:「古法歲陰與太歲不同,淮南天文訓攝提以下十二名,皆謂歲陰所在。史記太初元年年名焉逢、攝提格者,歲陰,非太歲也。東漢後不用歲陰紀年,又不知太歲超辰之法,乃以太初元年為丁丑歲,則與史、漢之文皆悖矣。」又謂:「尚書緯四游升降之說,即西法日躔最高、卑之說,宋楊忠輔統天術以距差乘躔差,減氣汎積為定積,梅文鼎謂郭守敬加減歲餘法出於此。但統天求汎積,必先減氣差十九日有奇,與郭又異,文鼎不能言。大昕推之同,凡步氣朔,必以甲子日起算,今統天上元冬至乃戊子日,不值甲子,依授時法當加氣應二十四日有奇,乃得從甲子起。今減去氣差,是以上元冬至後甲子日起算也。既如此,當減氣應三十五日有奇,今減十九日有奇者,去躔差之數不算也。求天正經朔又減閏差者,經朔當從合朔起算。今推得統天上元冬至後第一朔乃乙丑戌初二刻弱,故必減閏差而後以朔實除之,即授時之朔應也。」

大昕始以辭章名,沈德潛吳中七子詩選,大昕居一。既乃研精經、史,於經義之聚訟難決者,皆能剖析源流。文字、音韻、訓詁、天算、地理、氏族、金石以及古人爵里、事實、年齒,了如指掌。古人賢奸是非疑似難明者,典章制度昔人不能明斷者,皆有確見。惟不喜二氏書,嘗曰:「立德立功立言,吾儒之不朽也。先儒言釋氏近於墨,予以為釋氏亦終於楊氏為己而已。彼棄父母而學道,是視己重於父母也。」

大昕在館時,常與修音韻述微、續文獻通考、續通志、一統志、天球圖諸書。所著有唐石經考異一卷,經典文字考異一卷,聲類四卷,廿二史考異一百卷,唐書史臣表一卷,唐五代學士年表二卷,宋學士年表一卷,元史氏族表三卷,元史藝文志四卷,三史拾遺五卷,諸史拾遺五卷,通鑒注辨證三卷,四史朔閏考四卷,吳興舊德錄四卷,先德錄四卷,洪文惠、洪文敏、王伯厚、王弇州四家年譜各一卷,疑年錄三卷,潛揅堂文集五十卷,詩集二十卷,潛揅堂金石文跋尾二十五卷,養新錄二十三卷,恆言錄六卷,竹汀日記鈔三卷。族子塘、坫,能傳其學。

塘,字學淵。乾隆四十五年進士,改教職,選江寧府學教授。塘少大昕七歲,相與共學,又與大昕弟大昭及弟坫相切磋,為實事求是之學,於聲音文字、律呂推步尤有神解。著律呂古義六卷,據所得漢慮俿銅尺正荀勖以劉歆銅斛尺為周尺之非。謂周本八寸尺,不可以制律,律必用十寸尺,即昔人所云夏尺。周因夏、商,夏、商因唐、虞,古律當無異度。又史記三書釋疑三卷,於律歷天官家言皆究其原本,而以他書疏通證明之。律書「上九、商八、羽七、角六、宮五、徵九」數語,注家皆不能曉,小司馬疑其數錯。塘據淮南子、太玄經證之,始信其確。又著泮宮雅樂釋律四卷,說文聲系二十卷,淮南天文訓補注三卷。其所作古文曰述古編凡四卷。卒,年五十六。

坫,字獻之。副榜貢生。游京師,硃筠引為上客。以直隸州州判官於陜,與洪亮吉、孫星衍討論訓詁輿地之學,論者謂坫沉博不及大昕,而精當過之。嘉慶二年,教匪擾陜西,坫時署華州,率眾乘城,力遏其沖。城無弓矢,仿古為合竹強弓,厚背紙為翎,二人共發之,達百五十步;又以意為發石之法,石重十斤,達三百步:前後斃賊無算,城獲全。以積勞得末疾,引歸。著史記補注百三十卷,詳於音訓及郡縣沿革、山川所在。陜甘總督松筠重其品學,親至臥榻問疾,索未刊著述,坫取付之。曰:「三十年精力,盡於此書矣!」十一年,卒,年六十六。又有詩音表一卷,車制考一卷,論語後錄五卷,爾雅釋義十卷,釋地以下四篇注四卷,十經文字通正書十四卷,說文斠銓十四卷,新斠注地理志十六卷,漢書十表注十卷,聖賢塚墓志十二卷。

王鳴盛,字鳳喈,嘉定人。幼從長洲沈德潛受詩,後又從惠棟問經義,遂通漢學。乾隆十九年,以一甲進士授翰林院編修,大考翰詹第一,擢侍讀學士。充福建鄉試正考官,尋擢內閣學士,兼禮部侍郎。坐濫支驛馬,左遷光祿寺卿。丁內艱,遂不復出。

鳴盛性儉素,無聲色玩好之娛,晏坐一室,吚唔如寒士。嘗言:「漢人說經必守家法,自唐貞觀撰諸經義疏而家法亡,宋元豐以新經學取士而漢學殆絕,今好古之儒皆知崇注疏矣,然注疏惟詩、三禮及公羊傳猶是漢人家法,他經注則出魏、晉人。未為醇備。」著尚書後案三十卷,專述鄭康成之學,若鄭注亡逸,採馬、王注補之。孔傳雖出東晉,其訓詁猶有傳授,間一取焉。又謂東晉所獻之太誓偽,而唐人所斥之太誓非偽,故附書今文太誓一篇,存古之功,自謂不減惠氏周易述也。又著周禮軍賦說四卷,發明鄭氏之旨。又十七史商榷一百卷,於一史中紀、志、表、傳互相稽考,因而得其異同,又取稗史叢說以證其舛誤,於輿地、職官、典章、名物每致詳焉。別撰蛾術編一百卷,其為目十:說錄、說字、說地、說制、說人、說物、說集、說刻、說通、說系,蓋仿王應麟、顧炎武之意,而援引尤博。詩以才輔學,以韻達情。古文用歐、曾之法,闡許、鄭之義,有詩文集四十卷。嘉慶二年,卒,年七十六。

弟子同縣金曰追,字對揚。諸生。深於九經正義,每有疑譌,隨條輒錄,先成儀禮注疏正偽十七卷。阮元奉詔校勘儀禮石經,多採其說。

時同縣通經學者,有吳凌雲,字得青。嘉慶五年歲貢。讀書深造,經師遺說,靡不通貫。嘗假館錢大昕孱守齋,盡讀所藏書,學益邃。所著十三經考異,援據精核,多前人所未發。又經說三卷,小學說、廣韻說各一卷,海鹽陳其幹為合刊之,題曰吳氏遺著。

戴震,字東原,休寧人。讀書好深湛之思,少時塾師授以說文,三年盡得其節目。年十六七,研精注疏,實事求是,不主一家。與郡人鄭牧、汪肇龍、方矩、程瑤田、金榜從婺源江永游,震出所學質之永,永為之駭嘆。永精禮經及推步、鐘律、音聲、文字之學,惟震能得其全。

性特介。年二十八補諸生,家屢空,而學日進。與吳縣惠棟、吳江沈彤為忘年友。以避仇入都,北方學者如獻縣紀昀、大興硃筠,南方學者如嘉定錢大昕、王鳴盛,餘姚盧文弨,青浦王昶,皆折節與交。尚書秦蕙田纂五禮通考,震任其事焉。

乾隆二十七年,舉鄉試,三十八年,詔開四庫館,徵海內淹貫之士司編校之職,總裁薦震充纂修。四十年,特命與會試中式者同赴殿試,賜同進士出身,改翰林院庶吉士。震以文學受知,出入著作之庭。館中有奇文疑義,輒就咨訪。震亦思勤修其職,晨夕披檢,無間寒暑。經進圖籍,論次精審。所校大戴禮記、水經注尤精核。又於永樂大典內得九章、五曹算經七種,皆王錫闡、梅文鼎所未見。震正譌補脫以進,得旨刊行。四十二年,卒於官,年五十有五。

震之學,由聲音、文字以求訓詁,由訓詁以尋義理。謂:「義理不可空憑胸臆,必求之於古經。求之古經而遺文垂絕,今古懸隔,必求之古訓。古訓明則古經明,古經明則賢人聖人之義理明,而我心之同然者,乃因之而明。義理非他,存乎典章制度者也。彼歧古訓、義理而二之,是古訓非以明義理,而義理不寓乎典章制度,勢必流入於異學曲說而不自知也。」

震為學精誠解辨,每立一義,初若創獲,乃參考之,果不可易。大約有三:曰小學,曰測算,曰典章制度。

其小學書有六書論三卷,聲韻考四卷,聲類表九卷,方言疏證十卷。漢以後轉注之學失傳,好古如顧炎武,亦不深省。震謂:「指事、象形、諧聲、會意四者為書之體,假借、轉注二者為書之用。一字具數用者為假借,數字共一用者為轉注。初、哉、首、基之皆為始,工⼙、吾、臺、予之皆為我,其義轉相注也。」又自漢以來,古音浸微,學者於六書之故,靡所從入。顧氏古音表,入聲與廣韻相反。震謂:「有入無入之韻,當兩兩相配,以入聲為之樞紐。真至仙十四韻,與脂、微、齊、皆、灰五韻同入聲;東至江四韻及陽至登八韻,與支、之、佳、咍、蕭、宵、肴、豪、尤、侯、幽十一韻同入聲;浸至凡九韻之入聲,則從廣韻,無與之配。魚、虞、模、歌、戈、麻六韻,廣韻無入聲,今同以鐸為入聲,不與唐相配。而古音遞轉及六書諧聲之故,胥可由此得之。」皆古人所未發。

其測算書原象一卷,迎日推策記一卷,句股割圜記三卷,歷問一卷,古歷考二卷,續天文略三卷,策算一卷。自漢以來,疇人不知有黃極,西人入中國,始云赤道極之外又有黃道極,是為七政恆星右旋之樞,詫為六經所未有。震謂:「西人所云赤極,即周髀之正北極也,黃極即周髀之北極璿璣也。虞書『在璿璣玉衡,以齊七政』,蓋設璿璣以擬黃道極也。黃極在柱史星東南,上弼、少弼之間,終古不隨歲差而改。赤極居中,黃極環繞其外,周髀固已言之,不始於西人也。」

震所著典章制度之書未成。有詩經二南補注二卷,毛鄭詩考四卷,尚書義考一卷,儀經考正一卷,考工記圖二卷,春秋即位改元考一卷,大學補注一卷,中庸補注一卷,孟子字義疏證三卷,爾雅文字考十卷,經說四卷,水地記一卷,水經注四十卷,九章補圖一卷,屈原賦注七卷,通釋三卷,原善三卷,緒言三卷,直隸河渠書一百有二卷,氣穴記一卷,藏府算經論四卷,葬法贅言四卷,文集十卷。

震卒後,其小學,則高郵王念孫、金壇段玉裁傳之;測算之學,曲阜孔廣森傳之;典章制度之學,則興化任大椿傳之:皆其弟子也。後十餘年,高宗以震所校水經注問南書房諸臣曰:「戴震尚在否?」對曰:「已死。」上惋惜久之。王念孫、段玉裁、孔廣森、任大椿自有傳。

金榜,字輔之,歙縣人。乾隆二十九年召試舉人,授內閣中書,軍機處行走。三十七年一甲一名進士,授翰林院修撰。散館後,養痾讀書不復出,卒於家。師事江永,友戴震,著禮箋十卷,刺取其大者數十事為三卷,寄硃珪,珪序之,以為詞精義核。榜治禮最尊康成,然博稽而精思,慎求而能斷。嘗援鄭志答趙商云:「不信亦非,悉信亦非。」曰:「斯言也,敢以為治經之大法。故鄭義所未衷者必糾正之,於鄭氏家法不敢誣也。」

段玉裁,字若膺,金壇人。生而穎異,讀書有兼人之資。乾隆二十五年舉人,至京師見休寧戴震,好其學,遂師事之。以教習得貴州玉屏縣知縣,旋調四川,署富順及南溪縣事,又辦理化林坪站務。時大兵征金川,輓輸絡繹,玉裁處分畢,輒篝鐙著述不輟。著六書音均表五卷。古韻自顧炎武析為十部,後江永復析為十三部,玉裁謂支、佳一部也,脂、微、齊皆、灰一部也,之、咍一部也,漢人猶未嘗淆借通用。晉、宋而後,乃少有出入。迄乎唐之功令,支注「脂、之同用」,佳注「皆同用」,灰注「咍同用」,於是古之截然為三者,罕有知之。又謂真、臻、先、與諄、文、殷、魂、痕為二,尤、幽與侯為二,得十七部。其書始名詩經韻譜,群經韻譜。嘉定錢大昕見之,以為鑿破混沌,後易其體例,增以新加,十七部蓋如舊也。震偉其所學之精,雲自唐以來講韻學者所未發。尋任巫山縣,年四十六,以父老引疾歸,鍵戶不問世事者三十餘年。

玉裁於周、秦、兩漢書,無所不讀,諸家小學,皆別擇其是非。於是積數十年精力,專說說文,著說文解字注三十卷,謂:「爾雅以下,義書也;聲類以下,音書也;說文,形書也。凡篆一字,先訓其義,次釋其形,次釋其音,合三者以完一篆,故曰形書。」又謂:「許以形為主,因形以說音、說義。其所說義,與他書絕不同者,他書多假借,則字多非本義,許惟就字說其本義。知何者為本義,乃知何者為假借,則本義乃假借之權衡也。說文、爾雅相為表裏,治說文而後爾雅及傳注明。」又謂:「自倉頡造字時至唐、虞、三代、秦、漢以及許叔重造說文,曰『某聲』、曰『讀若某』者,皆條理合一不紊。故既用徐鉉切音,又某字志之曰古音第幾部,後附六書音均表,俾形、聲相為表裏。始為長編,名說文解字讀,凡五百四十卷。既乃隱括之成此注。」玉裁又以:「說文者,說字之書,故有『讀如』、無『讀為』,說經、傳之書,必兼是二者。漢人作注,於字發疑正讀,其例有三:『讀如』、『讀若』者,擬其音也,比方之詞;『讀為』『讀曰』者,易其字也,變化之詞;『當為』者,定為字之誤、聲之誤,而改其字也,救正之詞:三者分,而漢注可讀,而經可讀。」述漢讀考,先成周禮六卷,又撰禮經漢讀考一卷,其他十六卷未成。儀徵阮元謂玉裁書有功於天下後世者三:言古音一也,言說文二也,言說文二也,漢讀考三也。其他說經之書,以漢志毛詩經、毛詩古訓傳本各自為書,因釐次傳文,還其舊著,重訂毛詩古訓傳三十卷。以諸經惟尚書離厄最甚,古文幾亡,賈逵分別古今,劉陶是正文字,其書皆不存。乃廣蒐補闕,正晉、唐之妄改,存周、漢之駮文,著古文尚書撰異三十二卷。又錄左氏經文,取鄭注禮、周禮,存古文、今文故書之例,附見公羊、穀梁經文之異,著春秋左氏古經十二卷,而以左氏傳五十凡附後。外有毛詩小學三十卷,汲古閣說文訂六卷,經韻樓集十二卷。嘉慶二十年,卒,年八十一。

初,玉裁與念孫俱師震,故戴氏有段、王兩家之學,玉裁少震四歲,謙,專執弟子禮,雖耄,或稱震,必垂手拱立,朔望必莊誦震手札一通。卒後,王念孫謂其弟子長洲陳奐曰:「若膺死,天下遂無讀書人矣!」玉裁弟子,長洲徐頲、嘉興沈濤及女夫仁和龔麗正俱知名,而奐尤得其傳,奐自有傳。

鈕樹玉,字匪石,吳縣人。篤志好古,不為科舉之業,精研文字聲音訓詁。謂說文懸諸日月而不刊者也,後人以新附淆之,誣許君矣。因博稽載籍,著說文新附考六卷,續考一卷。又著說文解字校錄三十卷。樹玉後見玉裁書,著段氏說文注訂八卷,所駮正之處,皆有依據。

徐承慶,字夢祥,元和人。乾隆五十一年舉人,官至山西汾州府知府。著段注匡謬十五卷,其攻瑕索瘢,尤勝鈕氏之書,皆力求其是,非故為吹求者。

孫志祖,字詒穀,仁和人。乾隆三十一年進士,改刑部主事,洊升郎中,擢江南道監察御史,乞養歸。志祖清修自好,讀經史必釋其疑而後已,著讀書脞錄七卷,考論經、子、雜家,折衷精詳,不為武斷之論。又家語疏證六卷,謂王肅作聖證論以攻康成,又偽撰家語,飾其說以欺世。因博集群書,凡肅所剿竊者,皆疏通證明之。又謂孔叢子亦王肅偽託,其小爾雅亦肅借古書以自文,並作疏證以辨其妄。幼熟精文選,後乃仿韓文考異之例,參稽眾說,正俗本之誤,為文選考異四卷。又輯前人及朋輩論說,為文選注補正四卷。又有文選理學權輿補一卷。輯風俗通逸文一卷,補正姚之駰輯謝承後漢書五卷。嘉慶六年,卒,年六十五。

翟灝,字大川,亦仁和人。乾隆十九年進士,官金華、衢州府學教授。水顥見聞淹博,又能搜奇引痺,嘗與錢塘梁玉繩論王肅撰家語難鄭氏,欲搜考以證其譌,因握筆互疏所出,頃刻數十事。時方被酒,旋罷去,未竟槁,其精力殊絕人也。著有爾雅補郭二卷,以爾雅郭注未詳、未聞者百四十二科,邢疏補言其十,餘仍闕如,乃參稽眾家,一一備說。又云:「古爾雅當有釋禮篇,與釋樂篇相隨。祭名與講武、旌旂三章,乃釋禮之殘缺失次者。」又著四書考異七十二卷,皆貫串精審,為世所推。他著又有家語發覆、通俗篇、湖山便覽、無不宜齋詩文槁。五十三年,卒。

梁玉繩,字曜北,錢塘人。增貢生。家世貴顯,玉繩不志富貴,自號清白士。嘗語弟履繩曰:「後漢襄陽樊氏,顯重當時。子孫雖無名德盛位,世世作書生門戶,原與弟共勉之!」故玉繩年未四十,棄舉子業,專心撰著。其瞥記七卷,多釋經之文,有裨古義。玉繩尤精乙部書,著史記志疑三十六卷,據經、傳以糾乖違,參班、荀以究同異,錢大昕稱其書為龍門功臣。著人表考九卷,謂班氏借用禹貢田賦九等之目,造端自馬遷。史記李將軍傳云:「李蔡為人在下中。」其說頗是。

履繩,字處素。乾隆五十三年舉人。與兄玉繩相礱錯,有元方、季方之目。其於眾經中尤精左氏傳,謂隋志載賈逵解詁、服虔解義各數十卷,今俱亡佚。杜氏參用賈、服,仲達作疏,間有稱引,未睹其全。亦如馬融諸儒之說,僅存單文只義。唐以後注左氏者,惟張洽、趙汸最為明晰,大抵詳書法而略紀載。履繩綜覽諸家,旁採眾籍,以廣杜之所未備,作左通補釋三十二卷。又有未成者五門:曰廣傳、考異、駁證、古音、臆說。錢大昕見其書,嘆為絕恉。通說文,下筆鮮俗字。年四十六,卒。

汪家禧,字漢郊,仁和人。諸生。穎敏特異,通漢易,作易消息解。所著書數十卷,毀於火。其友秀水莊仲方、門人仁和許乃穀輯其遺文,為東里生燼餘集三卷。文多說經,粹然有家法。

劉臺拱,字端臨,寶應人。性至孝,六歲,母硃氏歿,哀如成人。事繼母鍾氏,與親母同。九歲作顏子頌,斐然成章,觀者稱為神童。中乾隆三十五年舉人,屢試禮部不第。是時朝廷開四庫館,海內方聞綴學之士雲集。臺拱在都,與學士硃筠、編修程晉芳、庶吉士戴震、學士邵晉涵及其同郡御史任大椿、給事中王念孫等交游,稽經考古,旦夕討論。自天文、律呂至於聲音、文字,靡不該貫。其於漢、宋諸儒之說,不專一家,而惟是之求。精思所到,如與古作者晤言一室而知其意指之所在,比之閻若璩,蓋相伯仲也。段玉裁每謂「潛心三禮,吾所不如」。

選丹徒縣訓導。取儀禮十七篇除喪服外各繪為圖,與諸生習禮容,為發明先王制作之精意。迎兩親學署,雍雍色養,年雖五十,有孺子之慕。嘗客他所,忽心痛驟歸,母病危甚,乃悉心奉湯藥,衣不解帶者數旬,母病遂愈。逮丁內外艱,水漿不入口。既斂,枕苫、啜粥,哭泣之哀,震動鄰里。居喪蔬食五年,出就外寢,以哀毀過情卒,年五十有五。

與同郡汪中為文章道義交,中歿,撫其孤喜孫,賴以成立。武進臧庸常以說經之文請益,臺拱善之。恤其窮,周其困,飲食教誨,十七年如一日,庸心感焉。臺拱慕黃叔度之為人,王昶稱其有曾、閔之孝。著有論語駢枝、經傳小記、國語補校、荀子補注、方言補校、淮南子補校、漢學拾遺、文集,都為端臨遺書凡八卷。

同邑硃彬,字武曹。乾隆六十年舉人。彬幼有至行,年十一喪母,哀戚如成人。長丁父憂,斂葬盡禮,三年蔬食居外。自少至老,好學不厭。承其鄉王懋竑經法,與外兄劉臺拱互相切磋。每有所得,輒以書札往來辨難,必求其是而後已。於訓詁、聲音、文字之學,用力尤深。著有經傳考證八卷,禮記訓纂四十九卷,虎觀諸儒所論議,鄭志弟子之問答,以及魏、晉以降諸儒之訓釋,書鈔、通典、御覽之涉是書者,一以注疏為主,擷其精要,緯以古今諸說。其附以己意者,皆援據精碻,發前人所未發。他著有游道堂詩文集四卷。道光十四年,卒,年八十有二。子士彥,吏部尚書,自有傳。

孔廣森,字眾仲,曲阜人,孔子六十八代孫,襲封衍聖公傳鐸之孫,戶部主事繼汾之子。乾隆三十六年進士,選翰林院庶吉士,散館授檢討。年少入官,性淡泊,躭著述,不與要人通謁。告養歸,不復出。及居大母與父喪,竟以哀卒,時乾隆五十一年,年三十五。

廣森聰穎特達,嘗受經於戴震、姚鼐之門,經史、小學,沉覽妙解。所學在公羊春秋,嘗以左氏舊學湮於征南,穀梁本義汨於武子。王祖游謂何休志通公羊,往往為公羊疚病。其餘啖助、趙匡之徒,又橫生義例,無當於經,唯趙汸最為近正。何氏體大思精,然不無承訛率臆。於是旁通諸家,兼採左、穀,擇善而從,著春秋公羊通義十一卷,序一卷。凡諸經籍義有可通於公羊者,多著錄之。

其不同於解詁者,大端有數事:謂古者諸侯分土而守,分民而治,有不純臣之義,故各得紀年於其境內。而何劭公謂唯王者然後改元立號,經書元年,為託王於魯,則自蹈所云反傳違戾之失。其不同一也。謂春秋分十二公而為三世,舊說「所傳聞之世」,隱、桓、莊、閔、僖也;「所聞之世」,文、宣、成、襄也;「所見之世」,昭、定、哀也。顏安樂以為:襄公二十三年「邾婁鼻我來奔」,云「邾婁無大夫,此何以書?以近書也」;又昭公二十七年「邾婁快來奔」,傳云「邾婁無大夫,此何以書?以近書也」:二文不異,同宜一世,故斷自孔子生後,即為「所見之世」,從之。其不同二也。謂桓十七年經無夏,二家經皆有夏,獨公羊脫耳。何氏謂:「夏者陽也,月者陰也,去夏者,明夫人不系於公也。」所不敢言。其不同三也。謂春秋上本天道,中用王法,而下理人情。天道者:一曰時,二曰月,三曰日。王法者:一曰譏,二曰貶,三曰絕。人情者:一曰尊,二曰親,三曰賢。此三科九旨。而何氏文謚例云:「三科九旨者,新周故宋,以春秋當新王,此一科三旨也。」又云:「所見異辭,所聞異辭,所傳聞又異辭。」三科六旨也。又「內其國而外諸夏,內諸夏而外夷狄,是三科九旨也」。其不同四也。他如何氏所據間有失者,多所裨損,以成一家之言。又謂左氏之事詳,公羊之義長,春秋重義不重事。皆好學深思,心知其意。其為說能融會貫通,使是非之旨不謬於聖人大旨,見自序中。儀徵阮元謂讀其書始知聖志之所在。

又著有大戴禮記補注十四卷,詩聲類十三卷,禮學卮言六卷,經學卮言六卷,少廣正負術內外篇六卷。駢體兼有漢、魏、六朝、初唐之勝,江都汪中讀之,嘆為絕手。然廣森不自足,作堂於其居,名曰「儀鄭」,自庶幾於康成。桐城姚鼐謂其將以孔子之裔傳孔子之學,雖康成猶不足以限之。惜奔走家難,勞思夭年,不充其志,藝林有遺憾焉。

邵晉涵,字二云,餘姚人。乾隆三十六年進士,歸班銓選。會開四庫館,特詔徵晉涵及歷城周永年、休寧戴震、仁和餘集等入館編纂,改翰林院庶吉士,授編修。四十五年,充廣西鄉試正考官。五十六年,大考遷左中允。擢侍講學士,充文淵閣直閣事日講起居注官。

晉涵左目眚,清羸。善讀書,四部、七錄,靡不研究。嘗謂爾雅者,六藝之津梁,而邢疏淺陋不稱;乃別為正義二十卷,以郭璞為宗,而兼採舍人、樊、劉、李、孫諸家,郭有未詳者,摭他書附之。自是承學之士,多舍邢而從邵。

尤長於史,以生在浙東,習聞劉宗周、黃宗羲諸緒論,說明季事,往往出於正史之外。在史館時,見永樂大典採薛居正五代史,乃薈萃編次,得十之八九,復採冊府元龜、太平御覽諸書,以補其缺。並參考通鑒長編諸史及宋人說部、碑碣,辨證條系,悉符原書一百五十卷之數。書成,呈御覽,館臣請仿劉昫舊唐書之例列於廿三史,刊布學宮,詔從之。由是薛史與歐陽史並傳矣。嘗謂宋史自南渡後多謬,慶元之間,褒貶失實,不如東都有王偁事略也。欲先輯南都事略,使條貫粗具,詞簡事增,又欲為趙宋一代之志,俱未卒業。其後鎮洋畢沅為續宋、元通鑒,囑晉涵刪補考定,故其緒餘稍見於審正續通鑒中。

晉涵性狷介,不為要人屈。嘗與會稽章學誠論修宋史宗旨,晉涵曰:「宋人門戶之習,語錄庸陋之風,誠可鄙也。然其立身制行,出於倫常日用,何可廢耶?士大夫博學工文,雄出當世,而於辭受取與、出處進退之間,不能無簟豆萬鐘之擇。本心既失,其他又何議焉!此著宋史之宗旨也。」學誠聞而聳然。他著有孟子述義、穀梁正義、韓詩內傳考,並足正趙岐、範甯及王應麟之失,而補其所遺。又有皇朝大臣謚跡錄、方輿金石編目、輶軒日記、南江詩文槁。嘉慶元年,卒,年五十有四。

周永年,字書昌,歷城人。博學貫通,為時推許。乾隆三十六年進士,與晉涵同徵修四庫書,改翰林院庶吉士,授編修。四十四年,充貴州鄉試副考官。永年在書館好深沉之思,四部兵、農、天算、術數諸家,鉤稽精義,褒譏悉當,為同館所推重。見宋、元遺書湮沒者多見採於永樂大典中,於是抉摘編摩,自永新劉氏兄弟公是、公非集以下,凡得十餘家,皆前人所未見者,咸著於錄。又以為釋、道有藏,儒者獨無。乃開借書園,聚古今書籍十萬卷,供人閱覽傳鈔,以廣流傳。惜永年歿後,漸就散佚,則未定經久之法也。

王念孫,字懷祖,高郵州人。父安國,官吏部尚書,謚文肅,自有傳。八歲讀十三經畢,旁涉史鑒。高宗南巡,以大臣子迎鑾,獻文冊,賜舉人。乾隆四十年進士,選翰林院庶吉士,散館,改工部主事。升郎中,擢陜西道御史,轉吏科給事中。嘉慶四年,仁宗親政,時川、楚教匪猖獗,念孫陳剿賊六事,首劾大學士和珅,疏語援據經義,大契聖心。是年授直隸永定河道。六年,以河堤漫口罷,特旨留督辦河工。工竣,賞主事銜。河南衡家樓河決,命往查勘,又命馳赴臺莊治河務。尋授山東運河道,在任六年,調永定河道。會東河總督與山東巡撫以引黃利運異議,召入都決其是非。念孫奏引黃入湖,不能不少淤,然暫行無害,詔許之。已而永定河水復異漲,如六年之隘,念孫自引罪,得旨休致。道光五年,重宴鹿鳴,卒,年八十有九。

念孫故精熟水利書,官工部,著導河議上下篇。及奉旨纂河源紀略,議者或誤指河源所出,念孫力辨其譌,議乃定,紀略中辨譌一門,念孫所撰也。既罷官,日以著述自娛,著讀書雜志,分逸周書、戰國策、管子、荀子、晏子春秋、墨子、淮南子、史記、漢書、漢隸拾遺,都八十二卷。於古義之晦,於鈔之誤寫,校之妄改,皆一一正之。一字之證,博及萬卷,其精於校讎如此。

初從休寧戴震受聲音文字訓詁,其於經,熟於漢學之門戶,手編詩三百篇、九經、楚辭之韻,分古音為二十一部。於支、脂、之三部之分,段玉裁六書音均表亦見及此,其分至、祭、盍、輯為四部,則段書所未及也。念孫以段書先出,遂輟作。

又以邵晉涵先為爾雅正義,乃撰廣雅疏證。日三字為程,閱十年而書成,凡三十二卷。其書就古音以求古義,引伸觸類,擴充於爾雅、說文,無所不達。然聲音文字部分之嚴,一絲不亂。蓋藉張揖之書以納諸說,而實多揖所未知,及同時惠棟、戴震所未及。

嘗語子引之曰:「詁訓之旨,存乎聲音,字之聲同、聲近者,經傳往往假借。學者以聲求義,破其假借之字而讀本字,則渙然冰釋。如因假借之字強為之解,則結夋不通矣。毛公詩傳多易假借之字而訓以本字,已開改讀之先。至康成箋詩注禮,屢云某讀為某,假借之例大明。後人或病康成破字者,不知古字之多假借也。」又曰:「說經者,期得經意而已,不必墨守一家。」引之因推廣庭訓,成經義述聞十五卷,經傳釋辭十卷,周秦古字解詁,字典考證。論者謂有清經術獨絕千古,高郵王氏一家之學,三世相承,與長洲惠氏相埒云。

引之,字伯申。嘉慶四年一甲進士,授編修。大考一等,擢侍講。歷官至工部尚書。福建署龍溪令硃履中誣布政使李賡蕓受賕,總督汪志伊、巡撫王紹蘭劾之。對簿無佐證,而持之愈急。賡蕓不堪,遂自經。命引之讞之,平反其獄,罷督撫官。為禮部侍郎時,有議為生祖母承重丁憂三年者,引之力持不可。會奉使去,持議者遽奏行之。引之還,疏陳庶祖母非祖敵體,不得以承重論。緣情,即終身持服不足以報罔極;制禮,則承重之義,不能加於支庶。請復治喪一年舊例,遂更正。道光十四年,卒,謚文簡。

同州李惇,字成裕。乾隆四十五年進士。惇與同縣王念孫、賈田祖同力於學。始為諸生,為學使謝墉所賞。將選拔貢,會田祖卒於旅舍,惇經營殯事,不與試,墉嘆為古人。江籓好詆訶前人,惇謂之曰:「王子雍若不作聖證論以攻康成,豈非醇儒?」其面規人過如此。著有群經識小八卷,考諸經古義二百二十餘事,多前人所未發。四十九年,卒,年五十一。

田祖,字稻孫。諸生。通左氏春秋,有春秋左氏通解。

宋綿初,字守端,亦高郵人。乾隆四十二年拔貢生,官五河、清河訓導。邃深經術,長於說詩,著韓詩內傳徵四卷。又有釋服二卷。

汪中,字容甫,江都人。生七歲而孤,家貧不能就外傅。母鄒,授以四子書。稍長,助書賈鬻書於市,因遍讀經、史、百家,過目成誦,遂為通人。年二十,補諸生。乾隆四十二年拔貢生,提學使者謝墉,每試別置一榜,署名諸生前。嘗曰:「余之先容甫,爵也。若以學,當北面事之。」其敬中如此。以母老竟不朝考。五十一年,侍郎硃珪主江南試,謂人曰:「吾此行必得汪中為選首。」不知其不與試也。

中顓意經術,與高郵王念孫、寶應劉臺拱為友,共討論之。其治尚書,有尚書考異。治禮,有儀禮校本,大戴禮記校本。治春秋,有春秋述義。治小學,有爾雅校本,及小學說文求端。中嘗謂國朝古學之興,顧炎武開其端。河、洛矯誣,至胡渭而絀。中、西推步,至梅文鼎而精。力攻古文者,閻若璩也。專治漢易者,惠棟也。凡此皆千餘年不傳之絕學,及戴震出而集其大成。擬作六儒頌,未成。

又嘗博考先秦古籍三代以上學制廢興,使知古人所以為學者。凡虞、夏第一,周禮之制第二,周衰列國第三,孔門第四,七十子後學者第五。又列通論、釋經、舊聞、典籍、數典、世官,目錄凡六。而自題其端曰:「觀周禮太史雲云,當時行一事則有一書,其後執書以行事,又其後則事廢而書存。至宋儒以後,則並其書之事而去之矣。」又曰:「有官府之典籍,有學士大夫之典籍,故老之傳聞。行一事有一書,傳之後世,奉以為成憲,此官府之典籍也。先王之禮樂政事,遭世之衰廢而不失,有司徒守其文,故老能言其事。好古之君子,憫其浸久而遂亡也,而書之簡畢,此學士大夫之典籍也。」又曰:「古之為學士者,官師之長,但教之以其事,其所誦者詩書而已。其他典籍,則皆官府藏而世守之,民間無有也。茍非其官,官亦無有也。其所謂士者,非王侯公卿大夫之子,則一命之士,外此則鄉學、小學而已。自闢雍之制無聞,太史之官失守,於是布衣有授業之徒,草野多載筆之士。教學之官,記載之職,不在上而在下。及其衰也,諸子各以其學鳴,而先王之道荒矣。然當諸侯去籍,秦政焚書,有司所掌,蕩然無存。猶賴學士相傳,存其一二,斯不幸中之幸也。」又曰:「孔子所言,則學士所能為者,留為世教。若其政教之大者,聖人無位,不復以教子弟。」又曰:「古人學在官府,人世其官,故官世其業。官既失守,故專門之學廢。」其書槁草略具,亦未成。後乃即其考三代典禮及文字訓詁、名物象數,益以論撰之文,為述學內、外篇,凡六卷。

其有功經義者,則有若釋三九,婦人無主答問,女子許嫁而壻死從死及守志議,居喪釋服解義。其表章經傳及先儒者,則有若周官徵文,左氏春秋釋疑,荀卿子通論,賈誼新書序。其他考證之文,亦有依據。

中又熟於諸史地理,山川厄要,講畫了然,著有廣陵通典十卷,秦蠶食六國表,金陵地圖考。生平於詩文書翰無所不工,所作廣陵對、黃鶴樓銘、漢上琴臺銘,皆見稱於時。他著有經義知新記一卷,大戴禮正誤一卷,遺詩一卷。五十九年,卒,年五十一。

中事母以孝聞,左右服勞,不辭煩辱。居喪,哀戚過人,其於知友故舊,沒後衰落,相存問過於從前。道光十一年,旌孝子。中子喜孫,自有傳。同郡人為漢學者,又有江德量、徐復、汪光爔。

德量,字量殊,江都人。父恂,好金石文字。伯父昱,通聲音訓詁之學。德量少承家學,及長,與汪中友,勵志肄經,學益進。乾隆四十四年一甲進士,授翰林院編修,改江西道御史。居朝多識舊聞,博通掌故。公餘鍵戶,以文籍自娛。著有古泉志三十卷。五十八年,卒,年四十二。

復,字心仲,亦江都人。通九章算術。

光爔,字晉蕃,儀徵人。廩生。博通經史,嘗辨惠氏易爻辰圖之謬,又作荑稗釋,時人服其精核。

武億,字虛谷,偃師人。父紹周,進士,官吏部郎中。億居父母喪,哀痛毀瘠,以讀書自勵。時伊、洛溢,屋圮,架洿以居,斧朽木燎寒,誦讀不輟。已,復從大興硃筠游,益為博通之學。乾隆四十五年進士,五十六年,授山東博山縣知縣。縣山多土瘠,民不務農。地產石炭、石礬,燒作玻璃器皿,商賈輻輳。億問土俗利病,免玻璃入貢,革煤炭供饋,里馬草豆不以累民。創範泉書院,進其秀者與之講敦倫理,務實學。而決辭無留獄,禱雨即沛。有以賄乾者,未敢進,億廉知之,值迅雷,曰:「汝不聞雷聲乎?吾矢禱久矣。」賄者惶悚而止,輿情大洽。

五十七年,大學士和珅領步軍統領事,聞妄人言山東逆匪王倫未定死,密遣番役四出蹤跡之。於是番役頭目杜成德等十一人橫行州縣,入博山境,手鐵尺飲博,莫敢誰何,億悉執之,成德尤倔強,按法痛杖之。喧傳其事者曰:「億鹵莽刑無罪,將累上官。」巡撫吉慶遂以濫責平民劾罷之,而不直書其事。億蒞任僅七月,及去,民攜老弱千餘人走大府乞留「我好官」,不可得,則日為運致薪米,門如市焉。吉慶亦感動,因入覲,偕億行,為籌捐復。大學士、公阿桂謂吉慶曰:「例禁番役出京畿,奈何責縣令按法之非,且隱其實而劾強項吏,何也?」吉慶深自悔,而格於部議,遂歸。嘉慶四年十月,仁宗諭朝臣密舉京、外各員內操守端潔、才猷幹濟、於平日居官事跡可據者,得赴部候旨召用,億在所舉中。十一月,縣令捧檄至門,而億先以十月卒矣,年五十有五。

億學問醰粹,於七經注疏、三史、涑水通鑒,皆能闇誦。既罷官,貧不能歸,所至以經史訓詁教授生徒。勇於著錄,有群經義證七卷,經讀考異九卷,金石三跋十卷,金石文字續跋十四卷,偃師金石記四卷,安陽金石錄十三卷。又有三禮義證、授堂劄記、詩文集等書,皆旁引遠徵,遇微罅,輒剖抉精蘊,比辭達意,以成一例。大興硃珪稱億不愧好古遺直云。

莊述祖,字葆琛,武進人。世父存與,官禮部侍郎,自有傳。述祖,乾隆四十五年進士,官山東濰縣知縣。明暢吏治,刑獄得中,豪猾斂跡。嘗勘鹼地,眾以為斥鹵也,述祖指路旁草問何名,曰馬帚。述祖笑曰:「此於經名荓,夏正『荓秀』記時,凡沙土草荓者宜禾,何謂鹼?」眾皆服。甲寅,以卓異引見,還,檄授桃源同知。不一月,乞養歸。著書色養者十六年,未嘗一日離左右。二十一年,卒。

述祖傳存與之學,研求精密,於世儒所忽不經意者,覃思獨闢,洞見本末。著述皆義理宏達,為前賢未有。以為連山亡而尚存夏小正,歸藏亡而尚有倉頡古文,略可稽求義類。故著夏小正經傳考釋,以斗柄南門織女記天行之不變,以參中大中記日度之差,以二月丁卯知夏時,以正月甲寅啟蟄為歷元,歲祭為郊,萬用入學為禘。著古文甲乙篇,謂許叔重始一終亥,偏旁條例所由出,日辰幹支,黃帝世大撓所作,沮誦、蒼頡名之以易結繩,伏羲畫八卦作十言之教之後,以此三十二類為正名百物之本。故歸藏為黃帝易,就許氏偏旁條例,以幹支別為序次,凡許書所存及見於金石文字者,分別部居,書未竟,而條理粗具。其餘五經,悉有撰著。旁及逸周書、尚書大傳、史記、白虎通,於其舛句訛字,佚文脫簡,易次換弟,草薙腋補,咸有證據,無不疏通,曠然思慮之表,若面稽古人而整比之也。所著夏小正經傳考釋十卷,尚書今古文考證七卷,毛詩考證四卷,毛詩周頌口義三卷,五經小學述二卷,歷代載籍足徵錄一卷,弟子職集解一卷,漢鐃歌句解一卷,石鼓然疑一卷,文鈔七卷,詩鈔二卷。

存與孫綬甲,字卿珊。盡通家學,尤為述祖所愛重。著尚書考異三卷,釋書名一卷。

同族莊有可,字大久。勤學力行,老而彌篤。取諸注、傳,精研義理,句櫛字比,合諸儒之書以正其是非,而自為之說。於易、書、詩、禮、春秋皆有撰述,凡四十二種,四百三十餘卷。

戚學標,字鶴泉,太平人。幼從天臺齊召南游,稱高第。高宗巡江、浙,學標獻南巡頌。乾隆四十五年,成進士,官河南涉縣知縣。縣苦闊布徵,學標請於大府得減額。權林縣,有兄弟爭產者,集李白句為斗粟謠以諷,皆感悔。性強項,多與上官齟,卒以是罷。後改寧波教授,未幾歸,從事撰述。

精考證,著漢學諧聲二十三卷、總論一卷。用說文以明古音,謂六書之學,三曰形聲,聲不離形,形者聲之本也。而聲又隨乎氣,氣有陰有陽,故一字之音,或從陰,或從陽,或陽而陰,或陰而陽,或陰陽各造其偏。昔人知其然,故但以某聲者明字音所出,以耑其本。以讀若某設為譬況之詞,使人依類而求。即離絕遠去,而因此聲之本以究此聲之變,無患其不合。說文從某某聲,從某某亦聲,從某某省聲,從某讀若某,從某讀與某某同,並二端兼舉。聲音之學,莫備於此。後人惑於徐氏所附孫愐音切,不究本讀,而一二宿儒言古音如吳棫、陳第、顧炎武、江永之流,亦第就韻書辨析。不知說文形聲相系,韻書就聲言聲;說文聲氣相求,韻書祗論同聲之應。其部居錯雜分合,類出肊見。學者茍趣其便,衷於一讀。且狃於平上去入之界之不可移易,諧聲之法廢,而說文之學晦矣。其書論聲一本許氏,由本聲以推變聲,既列本注,旁搜古讀以為之證。末附說文補考二卷,多辨正二徐謬誤。

又有毛詩證讀若干卷,詩聲辨定陰陽譜四卷,四書偶談四卷,內外篇二卷,字易二卷,鶴泉文鈔二卷。

江有誥,字晉三,歙縣人。通音韻之學,得顧炎武、江永兩家書,嗜之忘寢食。謂江書能補顧所未及,而分部仍多罅漏,乃析江氏十三部為二十一,與戴震、孔廣森多暗合。書成,寄示段玉裁,玉裁深重之,曰:「餘與顧氏、孔氏皆一於考古,江氏、戴氏則兼以審音。晉三於前人之說擇善而從,無所偏徇,又精於呼等字母,不惟古音大明,亦使今韻分為二百六部者得其剖析之故,韻學於是大備矣。」著有詩經韻讀、群經韻讀、楚辭韻讀、先秦韻讀、漢魏韻讀、唐韻四聲正、諧聲表、入聲表、二十一部韻譜、唐韻再正、唐韻更定部分,總名江氏音學十書,王念孫父子胥服其精。晚歲著說文六書錄、說文分韻譜。道光末,室災,焚其稿。有誥老而目盲,鬱鬱遂卒。

陳熙晉,原名津,字析木,義烏人。優貢生。以教習官貴州開泰、龍里、普定知縣,仁懷同知,擢湖北宜昌府知府。權開泰時,教匪蔣昌華擾黎平,將興大獄,熙晉縛其渠而貸諸脅從,全活無算。龍里民以釘奚殺人,已誣服,而兇驗不合,心疑焉。一日,方慮囚,見叢人中有曳釘奚竊睨者,命執而鞫之,痕宛合,遂款服。普定俗糾聚相雄長,號其魁曰「牛叢」。其獲盜,不謁之官,輒積薪焚殺之。先是有挾仇焚三尸者,吏不敢捕。熙晉期必得,重繩以法,風頓革。其守宜昌也,楚大水,流民聚宜昌,畢力撫綏,繕城垣,以工代賑。會秩滿將行,為留六閱月,蕆其事。送者數千人,皆泣下。乞養歸,未幾卒。

熙晉邃於學,積書數萬卷,訂疑糾謬,務窮竟原委,取裁精審。嘗謂杜預解左氏有三蔽,劉光伯規之,而書久佚。惟正義引一百七十三事,孔穎達皆以為非,乃刺取經史百家及近儒著述,以明劉義。其杜非而劉是者申之,杜是而劉非者釋之,杜、劉兩說義俱未安,則證諸群言,斷以己意,成春秋規過考信九卷。又謂隋經籍志載光伯左氏述義四十卷,不及規過,據孔穎達序稱習杜義而攻杜氏,疑規過即在述義中。舊唐書經籍志載述義三十七卷,較隋志少三卷,而多規過三卷,此其證也。正義於規杜一百七十三事外,又得一百四十三事,蓋皆述義之文。其異杜者三十事,駁正甚少。殆唐初奉敕刪定,著為令典,黨同伐異,勢會使然。乃參稽得失,援據群言,成春秋述義拾遺八卷。

他著有古文孝經述義疏證五卷,帝王世紀二卷,貴州風土記三十二卷,黔中水道記四卷,宋大夫集箋注三卷,駱臨海集箋注十卷,日損齋筆記考證一卷,文集八卷,征帆集四卷。

李誠,字靜軒,黃巖人。嘉慶十八年拔貢生,官云南姚州州判,終順寧知縣。撰十三經集解二百六十卷,首臚漢、魏諸家之說,次採近人精確之語,而唐、宋諸儒之徵實者亦不廢焉。嘗謂記水之書,自酈道元下,代不乏人,而言山者無成編,乃作萬山綱目六十卷。又水道提綱補訂二十八卷,宦游日記一卷,微言管窺三十六卷,醫家指迷一卷。

丁傑,原名錦鴻,字升衢,歸安人。乾隆四十六年進士,官寧波府學教授。傑純孝誠篤,嘗奔走滇南迎父柩歸葬。少家貧,就書肆中讀。肆力經史,旁及說文、音韻、算數。初至都,適四庫館開,任事者延之佐校,遂與硃筠、戴震、盧文弨、金榜、程瑤田等相講習。

傑為學長於校讎,與盧文弨最相似。得一書必審定句讀,博稽他本同異。於大戴禮用功尤深,著有大戴禮記繹。又易鄭注久佚,宋王應麟裒輯成書,惠棟復有增入。傑審視兩本,以為多羼入鄭氏易乾鑿度注,又漢書注所云鄭氏,乃即注漢書之人,非康成。乃刊其譌,定其是,復摘補其未備,著周易鄭注後定凡十二卷。胡渭禹貢錐指號為絕學,傑摘其誤甚多。嘗謂緯書「移河為界,在齊呂填閼八流以自廣」。河患之棘,由九河堙廢,而害始於齊。管仲能臣,必不自貽伊戚。班固敘溝洫志云:「商竭周移,秦決南涯,自茲距漢,北亡八支。」則九河之塞,當在秦、楚之際矣。惠棟尚書大傳輯本,傑以為疏舛,如「鮮度作荊,以詰四方」,誤讀困學紀聞,此謬之甚者。五行傳文不類,讀後漢書注,始知誤連皇覽也。傑嘗與翁方綱補正硃彞尊經義考序年月,博採見聞,以相證合。又與許言彥闡繹墨子上、下經,大有端緒。方言善本,始於戴震,傑採獲裨益最多,盧文弨以為不在戴下。漢隸字原考正,錢塘謂得隸之義例。

傑又言字母三十六字不可增並,不可顛倒:見、端、知、邦、非、精、照為孤清,不可增濁聲也;疑、泥、襄、明、微、來、日為孤濁,不可增清聲也;非即邦之輕脣,不可並於專又;微即明之輕脣,不可並於奉;影為曉之深喉,喻為匣之深喉,曉、匣、影、喻不可顛倒為影、曉、喻、匣也。所著書有小酉山房文集,嘉慶十二年,卒,年七十。

子授經,嘉慶三年優貢;傳經,六年優貢。皆能世其家學,有「雙丁」之目。授經佐其友嚴可均造甲乙丙丁長編,以校定說文。

周春,字松靄,海寧人。乾隆十九年進士,官廣西岑溪縣知縣。革陋規,幾微不以擾民,有古循吏風。以憂去官,岑溪人構祠祀焉。嘉慶十五年,重赴鹿鳴。二十年,卒,年八十七。春博學好古,兩親服闋,年未五十,不謁選。著十三經音略十三卷,專考經音,以陸氏釋文為權輿,參以玉篇、廣均、五經文字諸書音,字必審音,音必歸母,謹嚴細密,絲毫不假。他著又有中文孝經一卷,爾雅補注四卷,小學餘論二卷,代北姓譜二卷,遼金元姓譜一卷,遼詩話一卷,選材錄一卷,杜詩雙聲疊韻譜括略八卷。

孫星衍,字淵如,陽湖人。少與同里楊芳燦、洪亮吉、黃景仁文學相齊。袁枚品其詩,曰「天下奇才」,與訂忘年交。星衍雅不欲以詩名,深究經、史、文字、音訓之學,旁及諸子百家,皆必通其義。乾隆五十二年,以一甲進士授翰林院編修,充三通館校理。五十四年,散館,試厲志賦,用史記「如畏」,大學士和珅疑為別字,置三等改部。故事,一甲進士改部,或奏請留館,又編修改官可得員外,前此吳文煥有成案。珅示意欲使往見,星衍不肯屈節,曰:「主事終擢員外,何汲汲求人為?」自是編修改主事遂為成例。

官刑部,為法寬恕,大學士阿桂、尚書胡季堂悉器重之。有疑獄,輒令依古義平議,所平反全活甚眾。退直之暇,輒理舊業。洊升郎中。六十年,授山東兗沂曹濟道。

嘉慶元年七月,曹南水漫灘潰,決單縣地,星衍與按察使康基田鳩工集夫,五日夜,從上游築堤遏禦之,不果決。基田謂此役省國家數百萬帑金也。尋權按察使,凡七閱月,平反數十百條,活死罪誣服者十餘獄。濰縣有武人犯法,賄和珅門,囑託大吏。星衍訪捕鞫之,械和門來者於衢。及回本任,值曹工漫溢,星衍以無工處所得疏防咎,特旨予留任。曹工分治引河三道,星衍治中段。畢工,較濟東道、登萊道上下段省三十餘萬。先是河工分賠之員或得羨餘,謂之扣費,星衍不取,悉以給引河工費。時曹工尚未合,河督、巡撫亟奏合龍,移星衍任,尋又奏稱合而復開。開則分賠兩次壩工銀九萬兩,當半屬後任,而司事者並以歸星衍。星衍亦任之,曰:「吾既兼河務,不能不為人受過也。」

四年,丁母憂歸,浙撫阮元聘主詁經精舍。星衍課諸生以經史疑義及小學、天部、地理、算學、詞章,不十年,舍中士皆以撰述名家。服闋入都,仍發山東,十年,補督糧道。十二年,權布政使。值侍郎廣興在省,按章供張煩擾,星衍不肯妄支。後廣以賄敗,豫、東兩省多以支庫獲罪,星衍不與焉。十六年,引疾歸。

星衍博極群書,勤於著述。又好聚書,聞人家藏有善本,借鈔無虛日。金石文字,靡不考其原委。嘗病古文尚書為東晉梅賾所亂,官刑部時,即集古文尚書馬鄭注十卷、逸文二卷。歸田後,又為尚書今古文注疏三十九卷,其序例云:「尚書古注散佚,今刺取書傳升為注者五家三科之說:一,司馬遷從孔氏安國問故,是古文說;一,書大傳伏生所傳歐陽高、大夏侯勝、小夏侯建,是今文說;一,馬氏融、鄭氏康成雖有異同,多本衛氏宏、賈氏逵,是孔壁古文說:皆疏明出典。其先秦諸子所引古書說及緯書、白虎通等,漢、魏諸儒今文說、許氏說文所載孔壁古文,注中存其異文、異字,其說則附疏中。」其意在網羅放失舊聞,故錄漢、魏人佚說為多,又兼採近代王鳴盛、江聲、段玉裁諸人書說。惟不取趙宋以來諸人注,以其時文籍散亡,較今代無異聞,又無師傳,恐滋臆說也。凡積二十二年而後成。

其他撰輯,有周易集解十卷,夏小正傳校正三卷,明堂考三卷,考注春秋別典十五卷,爾雅廣雅詁訓韻編五卷,魏三體石經殘字考一卷,孔子集語十七卷,晏子春秋音義二卷,史記天官書考證十卷,建立伏博士始末二卷,寰宇訪碑錄十二卷,金石萃編二十卷,續古文苑二十卷,詩文集二十五卷。二十三年,卒,年六十六。星衍晚年所著書,多付文登畢亨、嘉興李貽德為卒其業。

亨,原名以田,字恬谿。初從休寧戴震游,精漢人古訓之學,尤長於書。星衍撰尚書今古文注疏,多採亨說,每稱以為經學無雙。中嘉慶十二年舉人,道光六年,以大挑知縣分發江西,署安義縣。有兄殺胞弟案,亨執「不念鞠子哀,泯亂倫彞,刑茲無赦」義,不準援赦。大府怒,將劾之,會歙程恩澤重亨,事乃解。後補崇義,以積勞卒官,年且八十矣。著有九水山房文存二卷。

貽德,字次白。嘉慶二十三年舉人。館星衍所,相得甚歡。著春秋左氏解賈服注輯述二十卷。其書援引甚博,字比句櫛,於義有未安者,亦加駁難。雖使沖遠復生,終未敢專樹征南之幟而盡棄舊義也。又有詩考異、詩經名物考、周禮賸義、十七史考異、攬青閣詩鈔、夢春廬詞。

王聘珍,字貞吾,南城人。自幼以力學聞。乾隆五十四年,學使翁方綱拔貢成均,為謝啟昆、阮元參訂古籍。嘗客浙西,與歙凌廷堪論學,廷堪深許之。為人厚重誠篤,廉介自守。

治經確守後鄭之學,著大戴禮記解詁十三卷、目錄一卷。其言曰:「大戴與小戴同受業於後倉,各取孔壁古文說,非小戴刪大戴、馬融足小戴也。禮察、保傅,語及秦亡,乃孔襄等所合藏。是賈誼有取於古記,非古記採及新書也。三朝記、曾子,乃劉氏分屬九流,非大戴所裒集也。」

又曰:「近代校讎,不知家法,王肅本點竄此經,私定孔子家語,反據肅本改易經文。又或據唐、宋類書如藝文類聚、太平御覽之流,增刪字句,或云據永樂大典改某字作某。凡茲數端,率以今義繩古義,以今音證古音,以今文易古文,遂使孔壁古奧之經,變而文從字順,經義由茲而亡。」故其發凡大旨,禮典器數,墨守鄭義,解詁文字,一依爾雅、說文及兩漢經師訓詁。有不知而闕,無杜撰之言。如「五義」義字,據周禮注讀若儀,「五鑿」五字釋若忤,青史子引漢書「君子養之」,讀若「中心養養」之養。皆能根據經史,發蒙解惑。江都焦循稱其不為增刪,一仍其舊,列為三十二讀書贊之一。他著經義考補,九經學。

凌廷堪,字次仲,歙縣人。六歲而孤,冠後始讀書,慕其鄉江永、戴震之學。乾隆五十五年進士,改教職,選寧國府學教授。奉母之官,畢力著述者十餘年。嘉慶十四年,卒,年五十三。

廷堪之學,無所不窺,於六書、歷算以迄古今疆域之沿革、職官之異同,靡不條貫。尤專禮學,謂:「古聖使人復性者學也,所學者即禮也。顏淵問仁,孔子告之者惟禮焉爾,顏子嘆道之高堅前後。迨『博文約禮』,然後『如有所立』,即『立於禮』之立也。禮有節文度數,非空言理者可託。」著禮經釋例十三卷,謂:「禮儀委曲繁重,必須會通其例。如鄉飲酒、鄉射、燕禮、大射不同,而其為獻酢酬旅、酬無算爵之例則同;聘禮、覲禮不同,而其為郊勞執玉、行享庭實之例則同;特牲饋食、少牢饋食不同,而其為尸飯主人初獻、主婦亞獻、賓長三獻、祭畢飲酒之例則同。」乃區為八例,以明同中之異,異中之同:曰通例,曰飲食例,曰賓客例,曰射例,曰變例,曰祭例,曰器服例,曰雜例。禮經第十一篇,自漢以來說者雖多,由不明尊尊之旨,故罕得經意,乃為封建尊尊服制考一篇,附於變例之後。大興硃珪讀其書,贈詩推重之。

廷堪禮經而外,復潛心於樂,謂今世俗樂與古雅樂中隔唐人燕樂一關,蔡季通、鄭世子輩俱未之知。因以隋沛公鄭譯五旦、七調之說為燕樂之本,又參考段安節琵琶錄、張叔夏詞源、遼史樂志諸書,著燕樂考原六卷。江都江籓嘆以為「思通鬼神」。他著有元遺山年譜二卷,校禮堂文集三十六卷、詩集十四卷。儀徵阮元常命子常生從廷堪授士禮,又稱其鄉射五物考、九拜解、九祭解、釋牲、詩楚茨考諸說經之文,多發古人所未發。其尤卓然者,則復禮三篇云。

同邑洪榜,字汝登。乾隆二十三年舉人。四十一年,應天津召試第一,授內閣中書。卒,年三十有五。粹於經學,著明象未成,終於益卦。因鄭康成易贊作述贊二卷。又明聲均,撰四聲均和表五卷,示兒切語一卷。江氏永切字六百十有六,是書增補百三十九字,又以字母見、溪等字注於廣韻之目每字之上,以定喉、吻、舌、齒、脣五音,蓋其書宗江、戴二家之說而加詳焉。為人律身以正,待人以誠。生平服膺戴震。戴震所著孟子字義疏證,當時讀者不能通其義,惟榜以為功不在禹下。撰震行狀,載與彭紹升書,硃筠見之曰:「可不必載,戴氏可傳者不在此。」榜乃上書辨論。江籓在吳下見其書,嘆曰:「洪君可謂衛道之儒矣。」

汪龍,字辰叔,亦廷堪同邑人。乾隆五十一年舉人。嗜古博學,尤精於詩,嘗讀詩生民、玄鳥二篇,疑鄭箋跡乳卵生之說,不若毛詩謂姜嫄、簡狄從帝嚳祀郊禖之正。遂稽傳、箋同異,用力於是經者數十年,成毛詩異義四卷,毛詩申成十卷。卒,年八十二。

桂馥,字冬卉,曲阜人。乾隆五十五年進士,選云南永平縣知縣,卒於官。

馥博涉群書,尤潛心小學,精通聲義。嘗謂:「士不通經,不足致用;而訓詁不明,不足以通經。」故自諸生以至通籍,四十年間,日取許氏說文與諸經之義相疏證,為說文義證五十卷。力窮根柢,為一生精力所在。

馥與段玉裁生同時,同治說文,學者以桂、段並稱,而兩人兩不相見,書亦未見,亦異事也。蓋段氏之書,聲義兼明,而尤邃於聲;桂氏之書,聲亦並及,而尤博於義。段氏鉤索比傅,自以為能冥合許君之旨,勇於自信,自成一家之言,故破字創義為多;桂氏專佐許說,發揮旁通,令學者引申貫注,自得其義之所歸。故段書約而猝難通闢,桂書繁而尋省易了。夫語其得於心,則段勝矣;語其便於人,則段或未之先也。其專臚古籍,不下己意,則以意在博證求通,展轉孳乳,觸長無方,亦如王氏廣雅疏證、阮氏經籍篡詁之類,非以己意為獨斷者。

及馥就宦滇南,追念舊聞,隨筆疏記十卷,以其細碎,比之匠門木材,題曰札樸。然馥嘗引徐幹中論:「鄙儒博學,務於物名,詳於器械,考於訓詁,摘其章句而不能統其大義之所極,以獲先王之心。故使學者勞思慮而不知道,費日月而無功成。」謂近日學者風尚六書,動成習氣,偶涉名物,自負倉、雅,略講點畫,妄議斯、冰,叩以經典大義,茫乎未之聞也。此尤為同時小學家所不能言,足以針肓起廢。他著有晚學集十二卷。

許瀚,字印林,日照人。道光十五年舉人,官嶧縣教諭。博綜經史及金石文字,訓詁尤深。至校勘宋、元、明本書籍,精審不減黃丕烈、顧廣圻。晚年為靈石楊氏校刊桂馥說文義證於清河,甫成而板毀於捻寇,並所藏經籍金石俱盡,遂挹鬱而歿,年七十。他著有韓詩外傳勘誤,攀古小廬文。

江聲,字叔澐,元和人。七歲就傅讀書,問讀書何為,師以取科第為言,聲求所以進於是者。年二十九,遭父疾,晨夕侍衛褥,不解衣帶,至自滌穠窬,視穢以驗疾進退。及居憂,哀毀骨立,逾三年,容戚然如新喪者。侍母疾,居喪,亦如父歿時。族黨哀其至行。既孤,因不復事科舉業。

讀尚書,怪古文與今文不類。又怪孔傳非安國所為。年三十五,師事同郡通儒惠棟,得讀所著古文尚書考及閻若璩古文疏證,乃知古文及孔傳皆晉時人偽作,於是集漢儒之說,以注二十九篇,漢注不備,則旁考他書。精研古訓,成尚書集注音疏十二卷,附補誼九條、識偽字一條,尚書集注音疏前後述外編一卷,尚書經師系表也。經文注疏,皆以古篆書之。疑偽古文者,始於宋之吳才老,硃子以後,吳草廬、郝京山、梅鷟皆不能得其要領。至本朝閻、惠兩徵君所著之書,乃能發其作偽之跡、剿竊之原。若刊正經文,疏明古注,則皆未之及也,及聲出而集大成焉。

聲又病後世深求考老轉注之義,至以篆跡求之,因為六書說,謂建類一首,即始一終亥五百四十部之首,同意相受,即凡某之屬皆從某也。陽湖孫星衍亦推其說,以為爾雅肇、祖、元、胎之屬,始也。始亦建類一首,肇、祖、元、胎皆為始,亦同意相受。說文此類亦甚多,推考老之訓,如口部之咽,嗌也;嗌,咽也。走部之走,趨也;趨,走也。猶之考注老,老轉注考矣。其同在口部、走部,即建類一首也。聲亦以為然,而戴震以為貫全部則義太廣。聲折之曰:「若止考老為轉注,不已隘乎?且諧聲一義,不貫全部乎?」聲與震以學問相推重,其不相附和如此。

生平不作楷書,即與人往來筆札,皆作古篆,俗儒往往非笑之,而聲不顧也。其寫尚書瀍水字,■L0字,不在說文,瀍據淮南作廛,水廛據爾雅義作孟,人始或怪之,後服其非臆說。顧其書終以時俗不便識讀,不甚行於時。

聲性耿介,不慕榮利。交游如王鳴盛、王昶、畢沅,皆重其品藻,而聲未嘗以私事干之,當事益重其人。嘉慶元年,舉孝廉方正。四年,卒,年七十有九。晚年因不諧俗,動與時違,取周易艮背之義,自號艮庭,學者稱為艮庭先生云。

子鏐,吳縣學生。孫沅,優貢生。世傳其學。

沅,字子蘭。金壇段玉裁僑居蘇州,沅出入其門者數十年。沅先著說文釋例,後承玉裁囑,以段書十七部諧聲表之列某聲某聲者為綱,而件系之;聲復生聲,則依其次第,為說文解字音均表凡十七卷。沅於段紕譌處略箋其失,其言曰:「支、脂、之之為三,真、臻、先與諄、文、欣、魂、痕之為二,皆陸氏之舊,而段氏矜為獨得之秘,嚴分其界以自殊異。凡許氏所合韻處,皆多方改使離之,而一部之與十二部,亦不使相通。故皕之讀若秘,改為逼;肊之乙聲,刪去聲字;必之弋亦聲,改為八亦聲。而於開章一篆說解極一物三字,即是一部、十二部、十五部合韻之理,於是絕不敢言其韻,直至亥字下重文說之也。十二、十三兩部之相通者,惟民、昬二字為梗,故力去昬字,以就其說。畀字田聲,十五部也,綥從畀得聲,而翏即古綦字,在一部,遂改畀字為★M1聲,以避十五部與一部之合音。凡此皆段氏之癥結處也。」又曰:「段氏論音謂古無去,故譜諸書平而上入。沅意古音有去無入,平輕去重,平引成上,去促成入。上入之字,少於平去,職是故耳。北人語言入皆成去,古音所沿,至今猶舊,非敢茍異,參之或然。」沅當時面質玉裁,親許駮勘,故有不同雲。卒,年七十二。

錢大昭,字晦之,嘉定人,大昕弟。大昕深於經史,一門群從,皆治古學,能文章,為東南之望。大昭少於大昕者二十年,事兄如嚴師,得其指授,時有兩蘇之比。壯歲游京師,嘗校錄四庫全書,人間未見之秘,皆得縱觀,由是學問益浩博。又善於決擇,其說經及小學之書,能直入漢儒閫奧。嘗欲從事爾雅,大昕與書,謂:「六經皆以明道,未有不通訓詁而能知道者。欲窮六經之旨,必自爾雅始。」大昭乃著爾雅釋文補三卷及廣雅疏義二十卷。

又著說文統釋六十卷,其例十:一曰疏證以佐古義,凡經典古義與許合者在所必收。二曰音切以復古音,以徐鉉、徐鍇等不知古音,往往誤讀,又許君言讀若某者,即有某音,今並補正;又說文本有舊音,隋書經籍志有說文音隱,顏氏家訓引之。唐以前傳注家多稱說文音某,今並採附本字之下。三曰考異以復古本,凡古本暨古書所引有異同者,悉取以折中。四曰辨俗以正譌字,凡經典相承俗字,及徐氏新補、新附字,皆辨證詳明,別為一卷附後。五曰通義以明互借,凡經典之同物同音,於古本是通用者,皆引經證之。六曰從母以明孳乳,如完、刓、髡、軏等字,皆於元下注云從此。七曰別體以廣異義,凡重文中之籀、篆、古文、奇字,皆有所從,其許君未言者,亦略釋之;經典兩用者,則引而證焉。八曰正譌以訂刊誤,凡許君不收之字,注中不應有,又字畫脫誤者,並校正之。九曰崇古以知古字,如鷐、鴠、、之類,經典有不從鳥者,此古今字,今注曰古用某。十曰補字以免漏略,如由、希、免、畾等三十九字,從此得聲者甚多,而書中脫落,有子無母,非許例,今酌補之,亦別為一卷附後。

大昭於正史尤精兩漢,嘗謂注史與注經不同,注經以明理為宗,理寓於訓詁,訓詁明而理自見。注史以達事為主,事不明,訓詁雖精無益也。每怪服虔、應劭之於漢書,裴駰、徐廣之於史記,其時去古未遠,稗官、載記、碑刻尚多,不能會而通之,考異質疑,徒戔戔於訓詁,乃著兩漢書辨疑四十卷,於地理、官制皆有所得。又仿其例著三國志辨疑三卷。又以宋熊方所補後漢書年表祗取材範書、陳志,乃於正史外兼取山經、地志、金石、子集,其體例依班氏之舊,而略變通之,著後漢書補表八卷。計所補王侯,多於熊書百三十人,論者謂視萬斯同歷代史表有過之無不及。他著有詩古訓十二卷,經說十卷,補續漢書藝文志二卷,後漢郡國令長考一卷,邇言二卷。

生平不嗜榮利,名其讀書之所曰可廬,欲蘄至於古之隨遇自足者。嘉慶元年,舉孝廉方正。

子東垣,字既勤。嘉慶三年舉人。官浙江松陽縣知縣,以艱歸。服闋,補上虞縣。東垣與弟繹、侗,皆潛研經、史、金石,時稱「三鳳」。嘗與繹、侗及同縣秦鑒勘訂鄭志,又與繹、侗、鑒及桐鄉金錫鬯輯釋崇文總目,世稱精本。東垣為學沉博而知要,以世傳孟子注疏繆舛特甚,乃輯劉熙、綦毋邃、陸善經諸儒古注及顧炎武、閻若璩、同時師友之論,附以己見。並正其音讀,考其異同,為孟子解誼十四卷。他著有小爾雅校證二卷,補經義考四十卷,列代建元表,勤有堂文集。

侗,字同人。於歷算之學,亦能究其原本。大昕撰宋遼金元四史朔閏考,未竟而卒,侗證以群書、金石文字,增輯一千三百餘條。日夕檢閱推算,幾忘寢食,卒因是感疾而歿。

硃駿聲,字豐芑,吳縣人。年十三,受許氏說文,一讀即通曉。從錢大昕游,錢一見奇之,曰:「衣缽之傳,將在子矣!」嘉慶二十三年舉人,官黟縣訓導。咸豐元年,以截取知縣入都,進呈所著說文通訓定聲及古今韻準、柬韻、說雅,共四十卷。文宗披覽,嘉其洽,賞國子監博士銜。旋遷揚州府學教授,引疾,未之官。八年,卒,年七十一。

駿聲著述甚博,不求知於世,兼長推步,明通象數。嘗論爾雅太歲在寅,推大昕說,謂其時自以實測之歲星在亥,定太歲在寅,命之曰攝提格以紀年,歲星所合之辰,即為太歲。然歲星閱百四十四年而超一辰,在秦、漢而甲寅之年歲星在丑,太歲應在子。漢詔書以太初元年為攝提格者,因六十紀年之名,歷年以次排敘,不能頓超一辰,故仍命以攝提格也。於是後人以寅、卯等為太歲,強以攝提格等為歲陰。其實爾雅所云歲陽、歲陰,非如後人說也。他著有左傳旁通十卷,左傳識小錄三卷,夏小正補傳一卷,離騷補注一卷。

子孔彰,字仲我。能傳父業,著有說文粹三編,十三經漢注,中興將帥別傳。


\end{pinyinscope}