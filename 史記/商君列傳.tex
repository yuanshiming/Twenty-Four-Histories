\article{商君列傳}

\begin{pinyinscope}
商君者,衛之諸庶孽公子也,名鞅,姓公孫氏,其祖本姬姓也。鞅少好刑名之學,事魏相公叔座為中庶子。公叔座知其賢,未及進。會座病,魏惠王親往問病,曰:「公叔病有如不可諱,將柰社稷何?」公叔曰:「座之中庶子公孫鞅,年雖少,有奇才,願王舉國而聽之。」王嘿然。王且去,座屏人言曰:「王即不聽用鞅,必殺之,無令出境。」王許諾而去。公叔座召鞅謝曰:「今者王問可以為相者,我言若,王色不許我。我方先君後臣,因謂王即弗用鞅,當殺之。王許我。汝可疾去矣,且見禽。」鞅曰:「彼王不能用君之言任臣,又安能用君之言殺臣乎?」卒不去。惠王既去,而謂左右曰:「公叔病甚,悲乎,欲令寡人以國聽公孫鞅也,豈不悖哉!」

公叔既死,公孫鞅聞秦孝公下令國中求賢者,將修繆公之業,東復侵地,乃遂西入秦,因孝公寵臣景監以求見孝公。孝公既見衛鞅,語事良久,孝公時時睡,弗聽。罷而孝公怒景監曰:「子之客妄人耳,安足用邪!」景監以讓衛鞅。衛鞅曰:「吾說公以帝道,其志不開悟矣。」後五日,復求見鞅。鞅復見孝公,益愈,然而未中旨。罷而孝公復讓景監,景監亦讓鞅。鞅曰:「吾說公以王道而未入也。請復見鞅。」鞅復見孝公,孝公善之而未用也。罷而去。孝公謂景監曰:「汝客善,可與語矣。」鞅曰:「吾說公以霸道,其意欲用之矣。誠復見我,我知之矣。」衛鞅復見孝公。公與語,不自知厀之前於席也。語數日不厭。景監曰:「子何以中吾君?吾君之驩甚也。」鞅曰:「吾說君以帝王之道比三代,而君曰:『久遠,吾不能待。且賢君者,各及其身顯名天下,安能邑邑待數十百年以成帝王乎?』故吾以彊國之術說君,君大說之耳。然亦難以比德於殷周矣。」

孝公既用衛鞅,鞅欲變法,恐天下議己。衛鞅曰:「疑行無名,疑事無功。且夫有高人之行者,固見非於世;有獨知之慮者,必見敖於民。愚者闇於成事,知者見於未萌。民不可與慮始而可與樂成。論至德者不和於俗,成大功者不謀於眾。是以聖人茍可以彊國,不法其故;茍可以利民,不循其禮。」孝公曰:「善。」甘龍曰:「不然。聖人不易民而教,知者不變法而治。因民而教,不勞而成功;緣法而治者,吏習而民安之。」衛鞅曰:「龍之所言,世俗之言也。常人安於故俗,學者溺於所聞。以此兩者居官守法可也,非所與論於法之外也。三代不同禮而王,五伯不同法而霸。智者作法,愚者制焉;賢者更禮,不肖者拘焉。」杜摯曰:「利不百,不變法;功不十,不易器。法古無過,循禮無邪。」衛鞅曰:「治世不一道,便國不法古。故湯武不循古而王,夏殷不易禮而亡。反古者不可非,而循禮者不足多。」孝公曰:「善。」以衛鞅為左庶長,卒定變法之令。

令民為什伍,而相牧司連坐。不告姦者腰斬,告姦者與斬敵首同賞,匿姦者與降敵同罰。民有二男以上不分異者,倍其賦。有軍功者,各以率受上爵;為私斗者,各以輕重被刑大小。僇力本業,耕織致粟帛多者復其身。事末利及怠而貧者,舉以為收孥。宗室非有軍功論,不得為屬籍。明尊卑爵秩等級,各以差次名田宅,臣妾衣服以家次。有功者顯榮,無功者雖富無所芬華。

令既具,未布,恐民之不信,已乃立三丈之木於國都市南門,募民有能徙置北門者予十金。民怪之,莫敢徙。復曰「能徙者予五十金」。有一人徙之,輒予五十金,以明不欺。卒下令。

令行於民朞年,秦民之國都言初令之不便者以千數。於是太子犯法。衛鞅曰:「法之不行,自上犯之。」將法太子。太子,君嗣也,不可施刑,刑其傅公子虔,黥其師公孫賈。明日,秦人皆趨令。行之十年,秦民大說,道不拾遺,山無盜賊,家給人足。民勇於公戰,怯於私斗,鄉邑大治。秦民初言令不便者有來言令便者,衛鞅曰「此皆亂化之民也」,盡遷之於邊城。其後民莫敢議令。

於是以鞅為大良造。將兵圍魏安邑,降之。居三年,作為筑冀闕宮庭於咸陽,秦自雍徙都之。而令民父子兄弟同室內息者為禁。而集小(都)鄉邑聚為縣,置令、丞,凡三十一縣。為田開阡陌封疆,而賦稅平。平斗桶權衡丈尺。行之四年,公子虔復犯約,劓之。居五年,秦人富彊,天子致胙於孝公,諸侯畢賀。

其明年,齊敗魏兵於馬陵,虜其太子申,殺將軍龐涓。其明年,衛鞅說孝公曰:「秦之與魏,譬若人之有腹心疾,非魏并秦,秦即并魏。何者?魏居領阨之西,都安邑,與秦界河而獨擅山東之利。利則西侵秦,病則東收地。今以君之賢聖,國賴以盛。而魏往年大破於齊,諸侯畔之,可因此時伐魏。魏不支秦,必東徙。東徙,秦據河山之固,東鄉以制諸侯,此帝王之業也。」孝公以為然,使衛鞅將而伐魏。魏使公子卬將而擊之。軍既相距,衛鞅遺魏將公子卬書曰:「吾始與公子驩,今俱為兩國將,不忍相攻,可與公子面相見,盟,樂飲而罷兵,以安秦魏。」魏公子卬以為然。會盟已,飲,而衛鞅伏甲士而襲虜魏公子卬,因攻其軍,盡破之以歸秦。魏惠王兵數破於齊秦,國內空,日以削,恐,乃使使割河西之地獻於秦以和。而魏遂去安邑,徙都大梁。梁惠王曰:「寡人恨不用公叔座之言也。」衛鞅既破魏還,秦封之於、商十五邑,號為商君。

商君相秦十年,宗室貴戚多怨望者。趙良見商君。商君曰:「鞅之得見也,從孟蘭皋,今鞅請得交,可乎?」趙良曰:「仆弗敢願也。孔丘有言曰:『推賢而戴者進,聚不肖而王者退。』仆不肖,故不敢受命。仆聞之曰:『非其位而居之曰貪位,非其名而有之曰貪名。』仆聽君之義,則恐仆貪位貪名也。故不敢聞命。」商君曰:「子不說吾治秦與?」趙良曰:「反聽之謂聰,內視之謂明,自勝之謂彊。虞舜有言曰:『自卑也尚矣。』君不若道虞舜之道,無為問仆矣。」商君曰:「始秦戎翟之教,父子無別,同室而居。今我更制其教,而為其男女之別,大筑冀闕,營如魯衛矣。子觀我治秦也,孰與五羖大夫賢?」趙良曰:「千羊之皮,不如一狐之掖;千人之諾諾,不如一士之諤諤。武王諤諤以昌,殷紂墨墨以亡。君若不非武王乎,則仆請終日正言而無誅,可乎?」商君曰:「語有之矣,貌言華也,至言實也,苦言藥也,甘言疾也。夫子果肯終日正言,鞅之藥也。鞅將事子,子又何辭焉!」趙良曰:「夫五羖大夫,荊之鄙人也。聞秦繆公之賢而願望見,行而無資,自粥於秦客,被褐食牛。期年,繆公知之,舉之牛口之下,而加之百姓之上,秦國莫敢望焉。相秦六七年,而東伐鄭,三置晉國之君,一救荊國之禍。發教封內,而巴人致貢;施德諸侯,而八戎來服。由余聞之,款關請見。五羖大夫之相秦也,勞不坐乘,暑不張蓋,行於國中,不從車乘,不操干戈,功名藏於府庫,德行施於後世。五羖大夫死,秦國男女流涕,童子不歌謠,舂者不相杵。此五羖大夫之德也。今君之見秦王也,因嬖人景監以為主,非所以為名也。相秦不以百姓為事,而大筑冀闕,非所以為功也。刑黥太子之師傅,殘傷民以駿刑,是積怨畜禍也。教之化民也深於命,民之效上也捷於令。今君又左建外易,非所以為教也。君又南面而稱寡人,日繩秦之貴公子。《詩》曰:『相鼠有體,人而無禮,人而無禮,何不遄死。』以詩觀之,非所以為壽也。公子虔杜門不出已八年矣,君又殺祝懽而黥公孫賈。《詩》曰:『得人者興,失人者崩。』此數事者,非所以得人也。君之出也,後車十數,從車載甲,多力而駢脅者為驂乘,持矛而操闟戟者旁車而趨。此一物不具,君固不出。《書》曰:『恃德者昌,恃力者亡。』君之危若朝露,尚將欲延年益壽乎?則何不歸十五都,灌園於鄙,勸秦王顯巖穴之士,養老存孤,敬父兄,序有功,尊有德,可以少安。君尚將貪商於之富,寵秦國之教,畜百姓之怨,秦王一旦捐賓客而不立朝,秦國之所以收君者,豈其微哉?亡可翹足而待。」商君弗從。

後五月而秦孝公卒,太子立。公子虔之徒告商君欲反,發吏捕商君。商君亡至關下,欲舍客舍。客人不知其是商君也,曰:「商君之法,舍人無驗者坐之。」商君喟然嘆曰:「嗟乎,為法之敝一至此哉!」去之魏。魏人怨其欺公子卬而破魏師,弗受。商君欲之他國。魏人曰:「商君,秦之賊。秦彊而賊入魏,弗歸,不可。」遂內秦。商君既復入秦,走商邑,與其徒屬發邑兵北出擊鄭。秦發兵攻商君,殺之於鄭黽池。秦惠王車裂商君以徇,曰:「莫如商鞅反者!」遂滅商君之家。

太史公曰:商君,其天資刻薄人也。跡其欲干孝公以帝王術,挾持浮說,非其質矣。且所因由嬖臣,及得用,刑公子虔,欺魏將卬,不師趙良之言,亦足發明商君之少恩矣。余嘗讀商君開塞耕戰書,與其人行事相類。卒受惡名於秦,有以也夫!


\end{pinyinscope}