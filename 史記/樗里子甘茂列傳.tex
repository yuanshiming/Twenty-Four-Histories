\article{樗里子甘茂列傳}

\begin{pinyinscope}
樗里子者,名疾,秦惠王之弟也,與惠王異母。母,韓女也。樗里子滑稽多智,秦人號曰「智囊」。

秦惠王八年,爵樗里子右更,使將而伐曲沃,盡出其人,取其城,地入秦。秦惠王二十五年,使樗里子為將伐趙,虜趙將軍莊豹,拔藺。明年,助魏章攻楚,敗楚將屈丐,取漢中地。秦封樗里子,號為嚴君。

秦惠王卒,太子武王立,逐張儀、魏章,而以樗里子、甘茂為左右丞相。秦使甘茂攻韓,拔宜陽。使樗里子以車百乘入周。周以卒迎之,意甚敬。楚王怒,讓周,以其重秦客。游騰為周說楚王曰:「知伯之伐仇猶,遺之廣車,因隨之以兵,仇猶遂亡。何則?無備故也。齊桓公伐蔡,號曰誅楚,其實襲蔡。今秦,虎狼之國,使樗里子以車百乘入周,周以仇猶、蔡觀焉,故使長戟居前,彊弩在後,名曰衛疾,而實囚之。且夫周豈能無憂其社稷哉?恐一旦亡國以憂大王。」楚王乃悅。

秦武王卒,昭王立,樗里子又益尊重。

昭王元年,樗里子將伐蒲。蒲守恐,請胡衍。胡衍為蒲謂樗里子曰:「公之攻蒲,為秦乎?為魏乎?為魏則善矣,為秦則不為賴矣。夫衛之所以為衛者,以蒲也。今伐蒲入於魏,衛必折而從之。魏亡西河之外而無以取者,兵弱也。今并衛於魏,魏必彊。魏彊之日,西河之外必危矣。且秦王將觀公之事,害秦而利魏,王必罪公。」樗里子曰:「柰何?」胡衍曰:「公釋蒲勿攻,臣試為公入言之,以德衛君。」樗里子曰:「善。」胡衍入蒲,謂其守曰:「樗里子知蒲之病矣,其言曰必拔蒲。衍能令釋蒲勿攻。」蒲守恐,因再拜曰:「願以請。」因效金三百斤,曰:「秦兵茍退,請必言子於衛君,使子為南面。」故胡衍受金於蒲以自貴於衛。於是遂解蒲而去。還擊皮氏,皮氏未降,又去。

昭王七年,樗里子卒,葬于渭南章臺之東。曰:「後百歲,是當有天子之宮夾我墓。」樗里子疾室在於昭王廟西渭南陰鄉樗裏,故俗謂之樗里子。至漢興,長樂宮在其東,未央宮在其西,武庫正直其墓。秦人諺曰:「力則任鄙,智則樗裏。」

甘茂者,下蔡人也。事下蔡史舉先生,學百家之說。因張儀、樗里子而求見秦惠王。王見而說之,使將,而佐魏章略定漢中地。

惠王卒,武王立。張儀、魏章去,東之魏。蜀侯煇、相壯反,秦使甘茂定蜀。還,而以甘茂為左丞相,以樗里子為右丞相。

秦武王三年,謂甘茂曰:「寡人欲容車通三川,以窺周室,而寡人死不朽矣。」甘茂曰:「請之魏,約以伐韓,而令向壽輔行。」甘茂至,謂向壽曰:「子歸,言之於王曰『魏聽臣矣,然願王勿伐』。事成,盡以為子功。」向壽歸,以告王,王迎甘茂於息壤。甘茂至,王問其故。對曰:「宜陽,大縣也,上黨、南陽積之久矣。名曰縣,其實郡也。今王倍數險,行千里攻之,難。昔曾參之處費,魯人有與曾參同姓名者殺人,人告其母曰『曾參殺人』,其母織自若也。頃之,一人又告之曰『曾參殺人』,其母尚織自若也。頃又一人告之曰『曾參殺人』,其母投杼下機,踰墻而走。夫以曾參之賢與其母信之也,三人疑之,其母懼焉。今臣之賢不若曾參,王之信臣又不如曾參之母信曾參也,疑臣者非特三人,臣恐大王之投杼也。始張儀西并巴蜀之地,北開西河之外,南取上庸,天下不以多張子而以賢先王。魏文侯令樂羊將而攻中山,三年而拔之。樂羊返而論功,文侯示之謗書一篋。樂羊再拜稽首曰:『此非臣之功也,主君之力也。』今臣,羈旅之臣也。樗里子、公孫奭二人者挾韓而議之,王必聽之,是王欺魏王而臣受公仲侈之怨也。」王曰:「寡人不聽也,請與子盟。」卒使丞相甘茂將兵伐宜陽。五月而不拔,樗里子、公孫奭果爭之。武王召甘茂,欲罷兵。甘茂曰:「息壤在彼。」王曰:「有之。」因大悉起兵,使甘茂擊之。斬首六萬,遂拔宜陽。韓襄王使公仲侈入謝,與秦平。

武王竟至周,而卒於周。其弟立,為昭王。王母宣太后,楚女也。楚懷王怨前秦敗楚於丹陽而韓不救,乃以兵圍韓雍氏。韓使公仲侈告急於秦。秦昭王新立,太后楚人,不肯救。公仲因甘茂,茂為韓言於秦昭王曰:「公仲方有得秦救,故敢捍楚也。今雍氏圍,秦師不下殽,公仲且仰首而不朝,公叔且以國南合於楚。楚、韓為一,魏氏不敢不聽,然則伐秦之形成矣。不識坐而待伐孰與伐人之利?」秦王曰:「善。」乃下師於殽以救韓。楚兵去。

秦使向壽平宜陽,而使樗里子、甘茂伐魏皮氏。向壽者,宣太后外族也,而與昭王少相長,故任用。向壽如楚,楚聞秦之貴向壽,而厚事向壽。向壽為秦守宜陽,將以伐韓。韓公仲使蘇代謂向壽曰:「禽困覆車。公破韓,辱公仲,公仲收國復事秦,自以為必可以封。今公與楚解口地,封小令尹以杜陽。秦楚合,復攻韓,韓必亡。韓亡,公仲且躬率其私徒以閼於秦。願公孰慮之也。」向壽曰:「吾合秦楚非以當韓也,子為壽謁之公仲,曰秦韓之交可合也。」蘇代對曰:「願有謁於公。人曰貴其所以貴者貴。王之愛習公也,不如公孫奭;其智能公也,不如甘茂。今二人者皆不得親於秦事,而公獨與王主斷於國者何?彼有以失之也。公孫奭黨於韓,而甘茂黨於魏,故王不信也。今秦楚爭彊而公黨於楚,是與公孫奭、甘茂同道也,公何以異之?人皆言楚之善變也,而公必亡之,是自為責也。公不如與王謀其變也,善韓以備楚,如此則無患矣。韓氏必先以國從公孫奭而後委國於甘茂。韓,公之讎也。今公言善韓以備楚,是外舉不僻讎也。」向壽曰:「然,吾甚欲韓合。」對曰:「甘茂許公仲以武遂,反宜陽之民,今公徒收之,甚難。」向壽曰:「然則奈何?武遂終不可得也?」對曰:「公奚不以秦為韓求潁川於楚?此韓之寄地也。公求而得之,是令行於楚而以其地德韓也。公求而不得,是韓楚之怨不解而交走秦也。秦楚爭彊,而公徐過楚以收韓,此利於秦。」向壽曰:「柰何?」對曰:「此善事也。甘茂欲以魏取齊,公孫奭欲以韓取齊。今公取宜陽以為功,收楚韓以安之,而誅齊魏之罪,是以公孫奭、甘茂無事也。」

甘茂竟言秦昭王,以武遂復歸之韓。向壽、公孫奭爭之,不能得。向壽、公孫奭由此怨,讒甘茂。茂懼,輟伐魏蒲阪,亡去。樗里子與魏講,罷兵。

甘茂之亡秦奔齊,逢蘇代。代為齊使於秦。甘茂曰:「臣得罪於秦,懼而遯逃,無所容跡。臣聞貧人女與富人女會績,貧人女曰:『我無以買燭,而子之燭光幸有餘,子可分我餘光,無損子明而得一斯便焉。』今臣困而君方使秦而當路矣。茂之妻子在焉,願君以餘光振之。」蘇代許諾。遂致使於秦。已,因說秦王曰:「甘茂,非常士也。其居於秦,累世重矣。自殽塞及至鬼谷,其地形險易皆明知之。彼以齊約韓魏反以圖秦,非秦之利也。」秦王曰:「然則柰何?」蘇代曰:「王不若重其贄,厚其祿以迎之,使彼來則置之鬼谷,終身勿出。」秦王曰:「善。」即賜之上卿,以相印迎之於齊。甘茂不往。蘇代謂齊湣王曰:「夫甘茂,賢人也。今秦賜之上卿,以相印迎之。甘茂德王之賜,好為王臣,故辭而不往。今王何以禮之?」齊王曰:「善。」即位之上卿而處之。秦因復甘茂之家以市於齊。

齊使甘茂於楚,楚懷王新與秦合婚而驩。而秦聞甘茂在楚,使人謂楚王曰:「願送甘茂於秦。」楚王問於范蜎曰:「寡人欲置相於秦,孰可?」對曰:「臣不足以識之。」楚王曰:「寡人欲相甘茂,可乎?」對曰:「不可。夫史舉,下蔡之監門也,大不為事君,小不為家室,以茍賤不廉聞於世,甘茂事之順焉。故惠王之明,武王之察,張儀之辯,而甘茂事之,取十官而無罪。茂誠賢者也,然不可相於秦。夫秦之有賢相,非楚國之利也。且王前嘗用召滑於越,而內行章義之難,越國亂,故楚南塞厲門而郡江東。計王之功所以能如此者,越國亂而楚治也。今王知用諸越而忘用諸秦,臣以王為鉅過矣。然則王若欲置相於秦,則莫若向壽者可。夫向壽之於秦王,親也,少與之同衣,長與之同車,以聽事。王必相向壽於秦,則楚國之利也。」於是使使請秦相向壽於秦。秦卒相向壽。而甘茂竟不得復入秦,卒於魏。

甘茂有孫曰甘羅。

甘羅者,甘茂孫也。茂既死後,甘羅年十二,事秦相文信侯呂不韋。

秦始皇帝使剛成君蔡澤於燕,三年而燕王喜使太子丹入質於秦。秦使張唐往相燕,欲與燕共伐趙以廣河閒之地。張唐謂文信侯曰:「臣嘗為秦昭王伐趙,趙怨臣,曰:『得唐者與百里之地。』今之燕必經趙,臣不可以行。」文信侯不快,未有以彊也。甘羅曰:「君侯何不快之甚也?」文信侯曰:「吾令剛成君蔡澤事燕三年,燕太子丹已入質矣,吾自請張卿相燕而不肯行。」甘羅曰:「臣請行之。」文信侯叱曰:「去!我身自請之而不肯,女焉能行之?」甘羅曰:「大項橐生七歲為孔子師。今臣生十二歲於茲矣,君其試臣,何遽叱乎?」於是甘羅見張卿曰:「卿之功孰與武安君?」卿曰:「武安君南挫彊楚,北威燕、趙,戰勝攻取,破城墮邑,不知其數,臣之功不如也。」甘羅曰:「應侯之用於秦也,孰與文信侯專?」張卿曰:「應侯不如文信侯專。」甘羅曰:「卿明知其不如文信侯專與?」曰:「知之。」甘羅曰:「應侯欲攻趙,武安君難之,去咸陽七里而立死於杜郵。今文信侯自請卿相燕而不肯行,臣不知卿所死處矣。」張唐曰:「請因孺子行。」令裝治行。

行有日,甘羅謂文信侯曰:「借臣車五乘,請為張唐先報趙。」文信侯乃入言之於始皇曰:「昔甘茂之孫甘羅,年少耳,然名家之子孫,諸侯皆聞之。今者張唐欲稱疾不肯行,甘羅說而行之。今願先報趙,請許遣之。」始皇召見,使甘羅於趙。趙襄王郊迎甘羅。甘羅說趙王曰:「王聞燕太子丹入質秦歟?」曰:「聞之。」曰:「聞張唐相燕歟?」曰:「聞之。」「燕太子丹入秦者,燕不欺秦也。張唐相燕者,秦不欺燕也。燕、秦不相欺者,伐趙,危矣。燕、秦不相欺無異故,欲攻趙而廣河閒。王不如齎臣五城以廣河閒,請歸燕太子,與彊趙攻弱燕。」趙王立自割五城以廣河閒。秦歸燕太子。趙攻燕,得上谷三十城,令秦有十一。

甘羅還報秦,乃封甘羅以為上卿,復以始甘茂田宅賜之。

太史公曰:樗里子以骨肉重,固其理,而秦人稱其智,故頗采焉。甘茂起下蔡閭閻,顯名諸侯,重彊齊楚。甘羅年少,然出一奇計,聲稱後世。雖非篤行之君子,然亦戰國之策士也。方秦之彊時,天下尤趨謀詐哉。


\end{pinyinscope}