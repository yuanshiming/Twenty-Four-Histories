\article{劉敬叔孫通列傳}

\begin{pinyinscope}
劉敬者,齊人也。漢五年,戍隴西,過洛陽,高帝在焉。婁敬脫輓輅,衣其羊裘,見齊人虞將軍曰:「臣願見上言便事。」虞將軍欲與之鮮衣,婁敬曰:「臣衣帛,衣帛見;衣褐,衣褐見:終不敢易衣。」於是虞將軍入言上。上召入見,賜食。

已而問婁敬,婁敬說曰:「陛下都洛陽,豈欲與周室比隆哉?」上曰:「然。」婁敬曰:「陛下取天下與周室異。周之先自后稷,堯封之邰,積德累善十有餘世。公劉避桀居豳。太王以狄伐故,去豳,杖馬箠居岐,國人爭隨之。及文王為西伯,斷虞芮之訟,始受命,呂望、伯夷自海濱來歸之。武王伐紂,不期而會孟津之上八百諸侯,皆曰紂可伐矣,遂滅殷。成王即位,周公之屬傅相焉,乃營成周洛邑,以此為天下之中也,諸侯四方納貢職,道裏均矣,有德則易以王,無德則易以亡。凡居此者,欲令周務以德致人,不欲依阻險,令後世驕奢以虐民也。及周之盛時,天下和洽,四夷鄉風,慕義懷德,附離而并事天子,不屯一卒,不戰一士,八夷大國之民莫不賓服,效其貢職。及周之衰也,分而為兩,天下莫朝,周不能制也。非其德薄也,而形勢弱也。今陛下起豐沛,收卒三千人,以之徑往而卷蜀漢,定三秦,與項羽戰滎陽,爭成皋之口,大戰七十,小戰四十,使天下之民肝腦涂地,父子暴骨中野,不可勝數,哭泣之聲未絕,傷痍者未起,而欲比隆於成康之時,臣竊以為不侔也。且夫秦地被山帶河,四塞以為固,卒然有急,百萬之眾可具也。因秦之故,資甚美膏腴之地,此所謂天府者也。陛下入關而都之,山東雖亂,秦之故地可全而有也。夫與人鬬,不搤其亢,拊其背,未能全其勝也。今陛下入關而都,案秦之故地,此亦搤天下之亢而拊其背也。」

高帝問群臣,群臣皆山東人,爭言周王數百年,秦二世即亡,不如都周。上疑未能決。及留侯明言入關便,即日車駕西都關中。

於是上曰:「本言都秦地者婁敬,『婁』者乃『劉』也。」賜姓劉氏,拜為郎中,號為奉春君。

漢七年,韓王信反,高帝自往擊之。至晉陽,聞信與匈奴欲共擊漢,上大怒,使人使匈奴。匈奴匿其壯士肥牛馬,但見老弱及羸畜。使者十輩來,皆言匈奴可擊。上使劉敬復往使匈奴,還報曰:「兩國相擊,此宜夸矜見所長。今臣往,徒見羸瘠老弱,此必欲見短,伏奇兵以爭利。愚以為匈奴不可擊也。」是時漢兵已踰句注,二十餘萬兵已業行。上怒,罵劉敬曰:「齊虜!以口舌得官,今乃妄言沮吾軍。」械系敬廣武。遂往,至平城,匈奴果出奇兵圍高帝白登,七日然後得解。高帝至廣武,赦敬,曰:「吾不用公言,以困平城。吾皆已斬前使十輩言可擊者矣。」乃封敬二千戶,為關內侯,號為建信侯。

高帝罷平城歸,韓王信亡入胡。當是時,冒頓為單于,兵彊,控弦三十萬,數苦北邊。上患之,問劉敬。劉敬曰:「天下初定,士卒罷於兵,未可以武服也。冒頓殺父代立,妻群母,以力為威,未可以仁義說也。獨可以計久遠子孫為臣耳,然恐陛下不能為。」上曰:「誠可,何為不能!顧為柰何?」劉敬對曰:「陛下誠能以適長公主妻之,厚奉遺之,彼知漢適女送厚,蠻夷必慕以為閼氏,生子必為太子。代單于。何者?貪漢重幣。陛下以歲時漢所餘彼所鮮數問遺,因使辯士風諭以禮節。冒頓在,固為子婿;死,則外孫為單于。豈嘗聞外孫敢與大父抗禮者哉?兵可無戰以漸臣也。若陛下不能遣長公主,而令宗室及後宮詐稱公主,彼亦知,不肯貴近,無益也。」高帝曰:「善。」欲遣長公主。呂后日夜泣,曰:「妾唯太子、一女,柰何棄之匈奴!」上竟不能遣長公主,而取家人子名為長公主,妻單于。使劉敬往結和親約。

劉敬從匈奴來,因言「匈奴河南白羊、樓煩王,去長安近者七百里,輕騎一日一夜可以至秦中。秦中新破,少民,地肥饒,可益實。夫諸侯初起時,非齊諸田,楚昭、屈、景莫能興。今陛下雖都關中,實少人。北近胡寇,東有六國之族,宗彊,一日有變,陛下亦未得高枕而臥也。臣願陛下徙齊諸田,楚昭、屈、景,燕、趙、韓、魏後,及豪桀名家居關中。無事,可以備胡;諸侯有變,亦足率以東伐。此彊本弱末之術也」。上曰:「善。」乃使劉敬徙所言關中十餘萬口。

叔孫通者,薛人也。秦時以文學徵,待詔博士。數歲,陳勝起山東,使者以聞,二世召博士諸儒生問曰:「楚戍卒攻蘄入陳,於公如何?」博士諸生三十餘人前曰:「人臣無將,將即反,罪死無赦。願陛下急發兵擊之。」二世怒,作色。叔孫通前曰:「諸生言皆非也。夫天下合為一家,毀郡縣城,鑠其兵,示天下不復用。且明主在其上,法令具於下,使人人奉職,四方輻輳,安敢有反者!此特群盜鼠竊狗盜耳,何足置之齒牙閒。郡守尉今捕論,何足憂。」二世喜曰:「善。」盡問諸生,諸生或言反,或言盜。於是二世令御史案諸生言反者下吏,非所宜言。諸言盜者皆罷之。乃賜叔孫通帛二十匹,衣一襲,拜為博士。叔孫通已出宮,反舍,諸生曰:「先生何言之諛也?」通曰:「公不知也,我幾不脫於虎口!」乃亡去,之薛,薛已降楚矣。及項梁之薛,叔孫通從之。敗於定陶,從懷王。懷王為義帝,徙長沙,叔孫通留事項王。漢二年,漢王從五諸侯入彭城,叔孫通降漢王。漢王敗而西,因竟從漢。

叔孫通儒服,漢王憎之;乃變其服,服短衣,楚製,漢王喜。

叔孫通之降漢,從儒生弟子百餘人,然通無所言進,專言諸故群盜壯士進之。弟子皆竊罵曰:「事先生數歲,幸得從降漢,今不能進臣等,專言大猾,何也?」叔孫通聞之,乃謂曰:「漢王方蒙矢石爭天下,諸生寧能鬬乎?故先言斬將搴旗之士。諸生且待我,我不忘矣。」漢王拜叔孫通為博士,號稷嗣君。

漢五年,已并天下,諸侯共尊漢王為皇帝於定陶,叔孫通就其儀號。高帝悉去秦苛儀法,為簡易。群臣飲酒爭功,醉或妄呼,拔劍擊柱,高帝患之。叔孫通知上益厭之也,說上曰:「夫儒者難與進取,可與守成。臣願徵魯諸生,與臣弟子共起朝儀。」高帝曰:「得無難乎?」叔孫通曰:「五帝異樂,三王不同禮。禮者,因時世人情為之節文者也。故夏、殷、周之禮所因損益可知者,謂不相復也。臣願頗采古禮與秦儀雜就之。」上曰:「可試為之,令易知,度吾所能行為之。」

於是叔孫通使徵魯諸生三十餘人。魯有兩生不肯行,曰:「公所事者且十主,皆面諛以得親貴。今天下初定,死者未葬,傷者未起,又欲起禮樂。禮樂所由起,積德百年而後可興也。吾不忍為公所為。公所為不合古,吾不行。公往矣,無汙我!」叔孫通笑曰:「若真鄙儒也,不知時變。」

遂與所徵三十人西,及上左右為學者與其弟子百餘人為綿蕞野外。習之月餘,叔孫通曰:「上可試觀。」上既觀,使行禮,曰:「吾能為此。」乃令群臣習肄,會十月。

漢七年,長樂宮成,諸侯群臣皆朝十月。儀:先平明,謁者治禮,引以次入殿門,廷中陳車騎步卒衛宮,設兵張旗志。傳言「趨」。殿下郎中俠陛,陛數百人。功臣列侯諸將軍軍吏以次陳西方,東鄉;文官丞相以下陳東方,西鄉。大行設九賓,臚傳。於是皇帝輦出房,百官執職傳警,引諸侯王以下至吏六百石以次奉賀。自諸侯王以下莫不振恐肅敬。至禮畢,復置法酒。諸侍坐殿上皆伏抑首,以尊卑次起上壽。觴九行,謁者言「罷酒」。御史執法舉不如儀者輒引去。竟朝置酒,無敢讙譁失禮者。於是高帝曰:「吾乃今日知為皇帝之貴也。」乃拜叔孫通為太常,賜金五百斤。

叔孫通因進曰:「諸弟子儒生隨臣久矣,與臣共為儀,願陛下官之。」高帝悉以為郎。叔孫通出,皆以五百斤金賜諸生。諸生乃皆喜曰:「叔孫生誠聖人也,知當世之要務。」

漢九年,高帝徙叔孫通為太子太傅。漢十二年,高祖欲以趙王如意易太子,叔孫通諫上曰:「昔者晉獻公以驪姬之故廢太子,立奚齊,晉國亂者數十年,為天下笑。秦以不蚤定扶蘇,令趙高得以詐立胡亥,自使滅祀,此陛下所親見。今太子仁孝,天下皆聞之;呂后與陛下攻苦食啖,其可背哉!陛下必欲廢適而立少,臣願先伏誅,以頸血汙地。」高帝曰:「公罷矣,吾直戲耳。」叔孫通曰:「太子天下本,本一搖天下振動,柰何以天下為戲!」高帝曰:「吾聽公言。」及上置酒,見留侯所招客從太子入見,上乃遂無易太子志矣。

高帝崩,孝惠即位,乃謂叔孫生曰:「先帝園陵寢廟,群臣莫(能)習。」徙為太常,定宗廟儀法。及稍定漢諸儀法,皆叔孫生為太常所論箸也。

孝惠帝為東朝長樂宮,及閒往,數蹕煩人,乃作複道,方筑武庫南。叔孫生奏事,因請閒曰:「陛下何自筑複道高寢,衣冠月出游高廟?高廟,漢太祖,柰何令後世子孫乘宗廟道上行哉?」孝惠帝大懼,曰:「急壞之。」叔孫生曰:「人主無過舉。今已作,百姓皆知之,今壞此,則示有過舉。願陛下原廟渭北,衣冠月出游之,益廣多宗廟,大孝之本也。」上乃詔有司立原廟。原廟起,以複道故。

孝惠帝曾春出游離宮,叔孫生曰:「古者有春嘗果,方今櫻桃孰,可獻,願陛下出,因取櫻桃獻宗廟。」上乃許之。諸果獻由此興。

太史公曰:語曰「千金之裘,非一狐之腋也;臺榭之榱,非一木之枝也;三代之際,非一士之智也。」信哉!夫高祖起微細,定海內,謀計用兵,可謂盡之矣。然而劉敬脫輓輅一說,建萬世之安,智豈可專邪!叔孫通希世度務,制禮進退,與時變化,卒為漢家儒宗。「大直若詘,道固委蛇」,蓋謂是乎?


\end{pinyinscope}