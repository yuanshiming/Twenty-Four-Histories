\article{伯夷列傳}

\begin{pinyinscope}
夫學者載籍極博,猶考信於六藝。詩書雖缺,然虞夏之文可知也。堯將遜位,讓於虞舜,舜禹之閒,岳牧咸薦,乃試之於位,典職數十年,功用既興,然後授政。示天下重器,王者大統,傳天下若斯之難也。而說者曰堯讓天下於許由,許由不受,恥之逃隱。及夏之時,有卞隨、務光者。此何以稱焉?太史公曰:余登箕山,其上蓋有許由冢云。孔子序列古之仁聖賢人,如吳太伯、伯夷之倫詳矣。余以所聞由、光義至高,其文辭不少概見,何哉?

孔子曰:「伯夷、叔齊,不念舊惡,怨是用希。」「求仁得仁,又何怨乎?」余悲伯夷之意,睹軼詩可異焉。其傳曰:

伯夷、叔齊,孤竹君之二子也。父欲立叔齊,及父卒,叔齊讓伯夷。伯夷曰:「父命也。」遂逃去。叔齊亦不肯立而逃之。國人立其中子。於是伯夷、叔齊聞西伯昌善養老,盍往歸焉。及至,西伯卒,武王載木主,號為文王,東伐紂。伯夷、叔齊叩馬而諫曰:「父死不葬,爰及干戈,可謂孝乎?以臣弒君,可謂仁乎?」左右欲兵之。太公曰:「此義人也。」扶而去之。武王已平殷亂,天下宗周,而伯夷、叔齊恥之,義不食周粟,隱於首陽山,采薇而食之。及餓且死,作歌。其辭曰:「登彼西山兮,采其薇矣。以暴易暴兮,不知其非矣。神農、虞、夏忽焉沒兮,我安適歸矣?于嗟徂兮,命之衰矣!」遂餓死於首陽山。

由此觀之,怨邪非邪?

或曰:「天道無親,常與善人。」若伯夷、叔齊,可謂善人者非邪?積仁絜行如此而餓死!且七十子之徒,仲尼獨薦顏淵為好學。然回也屢空,糟糠不厭,而卒蚤夭。天之報施善人,其何如哉?盜蹠日殺不辜,肝人之肉,暴戾恣睢,聚黨數千人橫行天下,竟以壽終。是遵何德哉?此其尤大彰明較著者也。若至近世,操行不軌,專犯忌諱,而終身逸樂,富厚累世不絕。或擇地而蹈之,時然後出言,行不由徑,非公正不發憤,而遇禍災者,不可勝數也。余甚惑焉,儻所謂天道,是邪非邪?

子曰「道不同不相為謀」,亦各從其志也。故曰「富貴如可求,雖執鞭之士,吾亦為之。如不可求,從吾所好」。「歲寒,然後知松柏之後凋」。舉世混濁,清士乃見。豈以其重若彼,其輕若此哉?

「君子疾沒世而名不稱焉。」賈子曰:「貪夫徇財,烈士徇名,夸者死權,眾庶馮生。」「同明相照,同類相求。」雲從龍,風從虎,聖人作而萬物睹。」伯夷、叔齊雖賢,得夫子而名益彰。顏淵雖篤學,附驥尾而行益顯。巖穴之士,趣舍有時若此,類名堙滅而不稱,悲夫!閭巷之人,欲砥行立名者,非附青雲之士,惡能施于後世哉?


\end{pinyinscope}