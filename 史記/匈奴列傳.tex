\article{匈奴列傳}

\begin{pinyinscope}
匈奴,其先祖夏后氏之苗裔也,曰淳維。唐虞以上有山戎、獫狁、葷粥,居于北蠻,隨畜牧而轉移。其畜之所多則馬、牛、羊,其奇畜則橐駞、驢、驘、駃騠、騊駼、騨騱。逐水草遷徙,毋城郭常處耕田之業,然亦各有分地。毋文書,以言語為約束。兒能騎羊,引弓射鳥鼠;少長則射狐兔:用為食。士力能毋弓,盡為甲騎。其俗,寬則隨畜,因射獵禽獸為生業,急則人習戰攻以侵伐,其天性也。其長兵則弓矢,短兵則刀鋋。利則進,不利則退,不羞遁走。茍利所在,不知禮義。自君王以下,咸食畜肉,衣其皮革,被旃裘。壯者食肥美,老者食其餘。貴壯健,賤老弱。父死,妻其後母;兄弟死,皆取其妻妻之。其俗有名不諱,而無姓字。

夏道衰,而公劉失其稷官,變于西戎,邑于豳。其後三百有餘歲,戎狄攻大王亶父,亶父亡走岐下,而豳人悉從亶父而邑焉,作周。其後百有餘歲,周西伯昌伐畎夷氏。後十有餘年,武王伐紂而營雒邑,復居于酆鄗,放逐戎夷涇、洛之北,以時入貢,命曰「荒服」。其後二百有餘年,周道衰,而穆王伐犬戎,得四白狼四白鹿以歸。自是之後,荒服不至。於是周遂作甫刑之辟。穆王之後二百有餘年,周幽王用寵姬褒姒之故,與申侯有卻。申侯怒而與犬戎共攻殺周幽王于驪山之下,遂取周之焦穫,而居于涇渭之閒,侵暴中國。秦襄公救周,於是周平王去酆鄗而東徙雒邑。當是之時,秦襄公伐戎至岐,始列為諸侯。是後六十有五年,而山戎越燕而伐齊,齊釐公與戰于齊郊。其後四十四年,而山戎伐燕。燕告急于齊,齊桓公北伐山戎,山戎走。其後二十有餘年,而戎狄至洛邑,伐周襄王,襄王奔于鄭之氾邑。初,周襄王欲伐鄭,故娶戎狄女為后,與戎狄兵共伐鄭。已而黜狄后,狄后怨,而襄王後母曰惠后,有子子帶,欲立之,於是惠后與狄后、子帶為內應,開戎狄,戎狄以故得入,破逐周襄王,而立子帶為天子。於是戎狄或居于陸渾,東至於衛,侵盜暴虐中國。中國疾之,故詩人歌之曰「戎狄是應」,「薄伐獫狁,至於大原」,「出輿彭彭,城彼朔方」。周襄王既居外四年,乃使使告急于晉。晉文公初立,欲修霸業,乃興師伐逐戎翟,誅子帶,迎內周襄王,居于雒邑。

當是之時,秦晉為彊國。晉文公攘戎翟,居于河西圁、洛之閒,號曰赤翟、白翟。秦穆公得由余,西戎八國服於秦,故自隴以西有綿諸、緄戎、翟、獂之戎,岐、梁山、涇、漆之北有義渠、大荔、烏氏、朐衍之戎。而晉北有林胡、樓煩之戎,燕北有東胡、山戎。各分散居谿谷,自有君長,往往而聚者百有餘戎,然莫能相一。

自是之後百有餘年,晉悼公使魏絳和戎翟,戎翟朝晉。後百有餘年,趙襄子踰句注而破并代以臨胡貉。其後既與韓魏共滅智伯,分晉地而有之,則趙有代、句注之北,魏有河西、上郡,以與戎界邊。其後義渠之戎筑城郭以自守,而秦稍蠶食,至於惠王,遂拔義渠二十五城。惠王擊魏,魏盡入西河及上郡于秦。秦昭王時,義渠戎王與宣太后亂,有二子。宣太后詐而殺義渠戎王於甘泉,遂起兵伐殘義渠。於是秦有隴西、北地、上郡,筑長城以拒胡。而趙武靈王亦變俗胡服,習騎射,北破林胡、樓煩。筑長城,自代并陰山下,至高闕為塞。而置雲中、鴈門、代郡。其後燕有賢將秦開,為質於胡,胡甚信之。歸而襲破走東胡,東胡卻千餘里。與荊軻刺秦王秦舞陽者,開之孫也。燕亦筑長城,自造陽至襄平。置上谷、漁陽、右北平、遼西、遼東郡以拒胡。當是之時,冠帶戰國七,而三國邊於匈奴。其後趙將李牧時,匈奴不敢入趙邊。後秦滅六國,而始皇帝使蒙恬將十萬之眾北擊胡,悉收河南地。因河為塞,筑四十四縣城臨河,徙適戍以充之。而通直道,自九原至雲陽,因邊山險塹谿谷可繕者治之,起臨洮至遼東萬餘里。又度河據陽山北假中。

當是之時,東胡彊而月氏盛。匈奴單于曰頭曼,頭曼不勝秦,北徙。十餘年而蒙恬死,諸侯畔秦,中國擾亂,諸秦所徙適戍邊者皆復去,於是匈奴得寬,復稍度河南與中國界於故塞。

單于有太子名冒頓。後有所愛閼氏,生少子,而單于欲廢冒頓而立少子,乃使冒頓質於月氏。冒頓既質於月氏,而頭曼急擊月氏。月氏欲殺冒頓,冒頓盜其善馬,騎之亡歸。頭曼以為壯,令將萬騎。冒頓乃作為鳴鏑,習勒其騎射,令曰:「鳴鏑所射而不悉射者,斬之。」行獵鳥獸,有不射鳴鏑所射者,輒斬之。已而冒頓以鳴鏑自射其善馬,左右或不敢射者,冒頓立斬不射善馬者。居頃之,復以鳴鏑自射其愛妻,左右或頗恐,不敢射,冒頓又復斬之。居頃之,冒頓出獵,以鳴鏑射單于善馬,左右皆射之。於是冒頓知其左右皆可用。從其父單于頭曼獵,以鳴鏑射頭曼,其左右亦皆隨鳴鏑而射殺單于頭曼,遂盡誅其後母與弟及大臣不聽從者。冒頓自立為單于。

冒頓既立,是時東胡彊盛,聞冒頓殺父自立,乃使使謂冒頓,欲得頭曼時有千里馬。冒頓問群臣,群臣皆曰:「千里馬,匈奴寶馬也,勿與。」冒頓曰:「柰何與人鄰國而愛一馬乎?」遂與之千里馬。居頃之,東胡以為冒頓畏之,乃使使謂冒頓,欲得單于一閼氏。冒頓復問左右,左右皆怒曰:「東胡無道,乃求閼氏!請擊之。」冒頓曰:「柰何與人鄰國愛一女子乎?」遂取所愛閼氏予東胡。東胡王愈益驕,西侵。與匈奴閒,中有棄地,莫居,千餘里,各居其邊為甌脫。東胡使使謂冒頓曰:「匈奴所與我界甌脫外棄地,匈奴非能至也,吾欲有之。」冒頓問群臣,群臣或曰:「此棄地,予之亦可,勿予亦可。」於是冒頓大怒曰:「地者,國之本也,柰何予之!」諸言予之者,皆斬之。冒頓上馬,令國中有後者斬,遂東襲擊東胡。東胡初輕冒頓,不為備。及冒頓以兵至,擊,大破滅東胡王,而虜其民人及畜產。既歸,西擊走月氏,南并樓煩、白羊河南王。[侵燕代]悉復收秦所使蒙恬所奪匈奴地者,與漢關故河南塞,至朝那、膚施,遂侵燕、代。是時漢兵與項羽相距,中國罷於兵革,以故冒頓得自彊,控弦之士三十餘萬。

自淳維以至頭曼千有餘歲,時大時小,別散分離,尚矣,其世傳不可得而次云。然至冒頓而匈奴最彊大,盡服從北夷,而南與中國為敵國,其世傳國官號乃可得而記云。

置左右賢王,左右谷蠡王,左右大將,左右大都尉,左右大當戶,左右骨都侯。匈奴謂賢曰「屠耆」,故常以太子為左屠耆王。自如左右賢王以下至當戶,大者萬騎,小者數千,凡二十四長,立號曰「萬騎」。諸大臣皆世官。呼衍氏,蘭氏,其後有須卜氏,此三姓其貴種也。諸左方王將居東方,直上谷以往者,東接穢貉、朝鮮;右方王將居西方,直上郡以西,接月氏、氐、羌;而單于之庭直代、雲中:各有分地,逐水草移徙。而左右賢王、左右谷蠡王最為大(國),左右骨都侯輔政。諸二十四長亦各自置千長、百長、什長、裨小王、相、封都尉、當戶、且渠之屬。

歲正月,諸長小會單于庭,祠。五月,大會蘢城,祭其先、天地、鬼神。秋,馬肥,大會蹛林,課校人畜計。其法,拔刃尺者死,坐盜者沒入其家;有罪小者軋,大者死。獄久者不過十日,一國之囚不過數人。而單于朝出營,拜日之始生,夕拜月。其坐,長左而北鄉。日上戊己。其送死,有棺槨金銀衣裘,而無封樹喪服;近幸臣妾從死者,多至數千百人。舉事而候星月,月盛壯則攻戰,月虧則退兵。其攻戰,斬首虜賜一卮酒,而所得鹵獲因以予之,得人以為奴婢。故其戰,人人自為趣利,善為誘兵以冒敵。故其見敵則逐利,如鳥之集;其困敗,則瓦解雲散矣。戰而扶輿死者,盡得死者家財。

後北服渾庾、屈射、丁零、鬲昆、薪犁之國。於是匈奴貴人大臣皆服,以冒頓單于為賢。

是時漢初定中國,徙韓王信於代,都馬邑。匈奴大攻圍馬邑,韓王信降匈奴。匈奴得信,因引兵南踰句注,攻太原,至晉陽下。高帝自將兵往擊之。會冬大寒雨雪,卒之墮指者十二三,於是冒頓詳敗走,誘漢兵。漢兵逐擊冒頓,冒頓匿其精兵,見其羸弱,於是漢悉兵,多步兵,三十二萬,北逐之。高帝先至平城,步兵未盡到,冒頓縱精兵四十萬騎圍高帝於白登,七日,漢兵中外不得相救餉。匈奴騎,其西方盡白馬,東方盡青駹馬,北方盡烏驪馬,南方盡騂馬。高帝乃使使閒厚遺閼氏,閼氏乃謂冒頓曰:「兩主不相困。今得漢地,而單于終非能居之也。且漢王亦有神,單于察之。」冒頓與韓王信之將王黃、趙利期,而黃、利兵又不來,疑其與漢有謀,亦取閼氏之言,乃解圍之一角。於是高帝令士皆持滿傅矢外鄉,從解角直出,竟與大軍合,而冒頓遂引兵而去。漢亦引兵而罷,使劉敬結和親之約。

是後韓王信為匈奴將,及趙利、王黃等數倍約,侵盜代、雲中。居無幾何,陳豨反,又與韓信合謀擊代。漢使樊噲往擊之,復拔代、鴈門、雲中郡縣,不出塞。是時匈奴以漢將眾往降,故冒頓常往來侵盜代地。於是漢患之,高帝乃使劉敬奉宗室女公主為單于閼氏,歲奉匈奴絮繒酒米食物各有數,約為昆弟以和親,冒頓乃少止。後燕王盧綰反,率其黨數千人降匈奴,往來苦上谷以東。

高祖崩,孝惠、呂太后時,漢初定,故匈奴以驕。冒頓乃為書遺高后,妄言。高后欲擊之,諸將曰:「以高帝賢武,然尚困於平城。」於是高后乃止,復與匈奴和親。

至孝文帝初立,復修和親之事。其三年五月,匈奴右賢王入居河南地,侵盜上郡葆塞蠻夷,殺略人民。於是孝文帝詔丞相灌嬰發車騎八萬五千,詣高奴,擊右賢王。右賢王走出塞。文帝幸太原。是時濟北王反,文帝歸,罷丞相擊胡之兵。

其明年,單于遺漢書曰:「天所立匈奴大單于敬問皇帝無恙。前時皇帝言和親事,稱書意,合歡。漢邊吏侵侮右賢王,右賢王不請,聽後義盧侯難氏等計,與漢吏相距,絕二主之約,離兄弟之親。皇帝讓書再至,發使以書報,不來,漢使不至,漢以其故不和,鄰國不附。今以小吏之敗約故,罰右賢王,使之西求月氏擊之。以天之福,吏卒良,馬彊力,以夷滅月氏,盡斬殺降下之。定樓蘭、烏孫、呼揭及其旁二十六國,皆以為匈奴。諸引弓之民,并為一家。北州已定,願寢兵休士卒養馬,除前事,復故約,以安邊民,以應始古,使少者得成其長,老者安其處,世世平樂。未得皇帝之志也,故使郎中系雩淺奉書請,獻橐他一匹,騎馬二匹,駕二駟。皇帝即不欲匈奴近塞,則且詔吏民遠舍。使者至,即遣之。」以六月中來至薪望之地。書至,漢議擊與和親孰便。公卿皆曰:「單于新破月氏,乘勝,不可擊。且得匈奴地,澤鹵,非可居也。和親甚便。」漢許之。

孝文皇帝前六年,漢遺匈奴書曰:「皇帝敬問匈奴大單于無恙。使郎中系雩淺遺朕書曰:『右賢王不請,聽後義盧侯難氏等計,絕二主之約,離兄弟之親,漢以故不和,鄰國不附。今以小吏敗約,故罰右賢王使西擊月氏,盡定之。願寢兵休士卒養馬,除前事,復故約,以安邊民,使少者得成其長,老者安其處,世世平樂。』朕甚嘉之,此古聖主之意也。漢與匈奴約為兄弟,所以遺單于甚厚。倍約離兄弟之親者,常在匈奴。然右賢王事已在赦前,單于勿深誅。單于若稱書意,明告諸吏,使無負約,有信,敬如單于書。使者言單于自將伐國有功,甚苦兵事。服繡袷綺衣、繡袷長襦、錦袷袍各一,比余一,黃金飾具帶一,黃金胥紕一,繡十匹,錦三十匹,赤綈、綠繒各四十匹,使中大夫意、謁者令肩遺單于。」

後頃之,冒頓死,子稽粥立,號曰老上單于。

老上稽粥單于初立,孝文皇帝復遣宗室女公主為單于閼氏,使宦者燕人中行說傅公主。說不欲行,漢彊使之。說曰:「必我行也,為漢患者。」中行說既至,因降單于,單于甚親幸之。

初,匈奴好漢繒絮食物,中行說曰:「匈奴人眾不能當漢之一郡,然所以彊者,以衣食異,無仰於漢也。今單于變俗好漢物,漢物不過什二,則匈奴盡歸於漢矣。其得漢繒絮,以馳草棘中,衣袴皆裂敝,以示不如旃裘之完善也。得漢食物皆去之,以示不如湩酪之便美也。」於是說教單于左右疏記,以計課其人眾畜物。

漢遺單于書,牘以尺一寸,辭曰「皇帝敬問匈奴大單于無恙」,所遺物及言語云云。中行說令單于遺漢書以尺二寸牘,及印封皆令廣大長,倨傲其辭曰「天地所生日月所置匈奴大單于敬問漢皇帝無恙」,所以遺物言語亦云云。

漢使或言曰:「匈奴俗賤老。」中行說窮漢使曰:「而漢俗屯戍從軍當發者,其老親豈有不自脫溫厚肥美以齎送飲食行戍乎?」漢使曰:「然。」中行說曰:「匈奴明以戰攻為事,其老弱不能鬬,故以其肥美飲食壯健者,蓋以自為守衛,如此父子各得久相保,何以言匈奴輕老也?」漢使曰:「匈奴父子乃同穹廬而臥。父死,妻其後母;兄弟死,盡取其妻妻之。無冠帶之飾,闕庭之禮。」中行說曰:「匈奴之俗,人食畜肉,飲其汁,衣其皮;畜食草飲水,隨時轉移。故其急則人習騎射,寬則人樂無事,其約束輕,易行也。君臣簡易,一國之政猶一身也。父子兄弟死,取其妻妻之,惡種姓之失也。故匈奴雖亂,必立宗種。今中國雖詳不取其父兄之妻,親屬益疏則相殺,至乃易姓,皆從此類。且禮義之敝,上下交怨望,而室屋之極,生力必屈。夫力耕桑以求衣食,筑城郭以自備,故其民急則不習戰功,緩則罷於作業。嗟土室之人,顧無多辭,令喋喋而佔佔,冠固何當?」

自是之後,漢使欲辯論者,中行說輒曰:「漢使無多言,顧漢所輸匈奴繒絮米糱,令其量中,必善美而己矣,何以為言乎?且所給備善則已;不備,苦惡,則候秋孰,以騎馳蹂而稼穡耳。」日夜教單于候利害處。

漢孝文皇帝十四年,匈奴單于十四萬騎入朝那、蕭關,殺北地都尉卬,虜人民畜產甚多,遂至彭陽。使奇兵入燒回中宮,候騎至雍甘泉。於是文帝以中尉周舍、郎中令張武為將軍,發車千乘,騎十萬,軍長安旁以備胡寇。而拜昌侯盧卿為上郡將軍,甯侯魏遬為北地將軍,隆慮侯周灶為隴西將軍,東陽侯張相如為大將軍,成侯董赤為前將軍,大發車騎往擊胡。單于留塞內月餘乃去,漢逐出塞即還,不能有所殺。匈奴日已驕,歲入邊,殺略人民畜產甚多,雲中、遼東最甚,至代郡萬餘人。漢患之,乃使使遺匈奴書。單于亦使當戶報謝,復言和親事。

孝文帝後二年,使使遺匈奴書曰:「皇帝敬問匈奴大單于無恙。使當戶且居雕渠難、郎中韓遼遺朕馬二匹,已至,敬受。先帝制:長城以北,引弓之國,受命單于;長城以內,冠帶之室,朕亦制之。使萬民耕織射獵衣食,父子無離,臣主相安,俱無暴逆。今聞渫惡民貪降其進取之利,倍義絕約,忘萬民之命,離兩主之驩,然其事已在前矣。書曰:『二國已和親,兩主驩說,寢兵休卒養馬,世世昌樂,闟然更始。』朕甚嘉之。聖人者日新,改作更始,使老者得息,幼者得長,各保其首領而終其天年。朕與單于俱由此道,順天恤民,世世相傳,施之無窮,天下莫不咸便。漢與匈奴鄰國之敵,匈奴處北地,寒,殺氣早降,故詔吏遺單于秫糱金帛絲絮佗物歲有數。今天下大安,萬民熙熙,朕與單于為之父母。朕追念前事,薄物細故,謀臣計失,皆不足以離兄弟之驩。朕聞天不頗覆,地不偏載。朕與單于皆捐往細故,俱蹈大道,墮壞前惡,以圖長久,使兩國之民若一家子。元元萬民,下及魚鱉,上及飛鳥,跂行喙息蠕動之類,莫不就安利而辟危殆。故來者不止,天之道也。俱去前事:朕釋逃虜民,單于無言章尼等。朕聞古之帝王,約分明而無食言。單于留志,天下大安,和親之後,漢過不先。單于其察之。」

單于既約和親,於是制詔御史曰:「匈奴大單于遺朕書,言和親已定,亡人不足以益眾廣地,匈奴無入塞,漢無出塞,犯(令)[今]約者殺之,可以久親,后無咎,俱便。朕已許之。其布告天下,使明知之。」

後四歲,老上稽粥單于死,子軍臣立為單于。既立,孝文皇帝復與匈奴和親。而中行說復事之。

軍臣單于立四歲,匈奴復絕和親,大入上郡、雲中各三萬騎,所殺略甚眾而去。於是漢使三將軍軍屯北地,代屯句注,趙屯飛狐口,緣邊亦各堅守以備胡寇。又置三將軍,軍長安西細柳、渭北棘門、霸上以備胡。胡騎入代句注邊,烽火通於甘泉、長安。數月,漢兵至邊,匈奴亦去遠塞,漢兵亦罷。後歲餘,孝文帝崩,孝景帝立,而趙王遂乃陰使人於匈奴。吳楚反,欲與趙合謀入邊。漢圍破趙,匈奴亦止。自是之後,孝景帝復與匈奴和親,通關市,給遺匈奴,遣公主,如故約。終孝景時,時小入盜邊,無大寇。

今帝即位,明和親約束,厚遇,通關市,饒給之。匈奴自單于以下皆親漢,往來長城下。

漢使馬邑下人聶翁壹奸蘭出物與匈奴交,詳為賣馬邑城以誘單于。單于信之,而貪馬邑財物,乃以十萬騎入武州塞。漢伏兵三十餘萬馬邑旁,御史大夫韓安國為護軍,護四將軍以伏單于。單于既入漢塞,未至馬邑百餘里,見畜布野而無人牧者,怪之,乃攻亭。是時鴈門尉史行徼,見寇,葆此亭,知漢兵謀,單于得,欲殺之,尉史乃告單于漢兵所居。單于大驚曰:「吾固疑之。」乃引兵還。出曰:「吾得尉史,天也,天使若言。」以尉史為「天王」。漢兵約單于入馬邑而縱,單于不至,以故漢兵無所得。漢將軍王恢部出代擊胡輜重,聞單于還,兵多,不敢出。漢以恢本造兵謀而不進,斬恢。自是之後,匈奴絕和親,攻當路塞,往往入盜於漢邊,不可勝數。然匈奴貪,尚樂關市,嗜漢財物,漢亦尚關市不絕以中之。

自馬邑軍後五年之秋,漢使四將軍各萬騎擊胡關市下。將軍衛青出上谷,至蘢城,得胡首虜七百人。公孫賀出雲中,無所得。公孫敖出代郡,為胡所敗七千餘人。李廣出鴈門,為胡所敗,而匈奴生得廣,廣后得亡歸。漢囚敖、廣,敖、廣贖為庶人。其冬,匈奴數入盜邊,漁陽尤甚。漢使將軍韓安國屯漁陽備胡。其明年秋,匈奴二萬騎入漢,殺遼西太守,略二千餘人。胡又入敗漁陽太守軍千餘人,圍漢將軍安國,安國時千餘騎亦且盡,會燕救至,匈奴乃去。匈奴又入鴈門,殺略千餘人。於是漢使將軍衛青將三萬騎出鴈門,李息出代郡,擊胡。得首虜數千人。其明年,衛青復出雲中以西至隴西,擊胡之樓煩、白羊王於河南,得胡首虜數千,牛羊百餘萬。於是漢遂取河南地,筑朔方,復繕故秦時蒙恬所為塞,因河為固。漢亦棄上谷之什辟縣造陽地以予胡。是歲,漢之元朔二年也。

其后冬,匈奴軍臣單于死。軍臣單于弟左谷蠡王伊稚斜自立為單于,攻破軍臣單于太子於單。於單亡降漢,漢封於單為涉安侯,數月而死。

伊稚斜單于既立,其夏,匈奴數萬騎入殺代郡太守恭友,略千餘人。其秋,匈奴又入鴈門,殺略千餘人。其明年,匈奴又復復入代郡、定襄、上郡,各三萬騎,殺略數千人。匈奴右賢王怨漢奪之河南地而筑朔方,數為寇,盜邊,及入河南,侵擾朔方,殺略吏民其眾。

其明年春,漢以衛青為大將軍,將六將軍,十餘萬人,出朔方、高闕擊胡。右賢王以為漢兵不能至,飲酒醉,漢兵出塞六七百里,夜圍右賢王。右賢王大驚,脫身逃走,諸精騎往往隨后去。漢得右賢王眾男女萬五千人,裨小王十餘人。其秋,匈奴萬騎入殺代郡都尉朱英,略千餘人。

其明年春,漢復遣大將軍衛青將六將軍,兵十餘萬騎,乃再出定襄數百里擊匈奴,得首虜前后凡萬九千餘級,而漢亦亡兩將軍,軍三千餘騎。右將軍建得以身脫,而前將軍翕侯趙信兵不利,降匈奴。趙信者,故胡小王,降漢,漢封為翕侯,以前將軍與右將軍并軍分行,獨遇單于兵,故盡沒。單于既得翕侯,以為自次王,用其姊妻之,與謀漢。信教單于益北絕幕,以誘罷漢兵,徼極而取之,無近塞。單于從其計。其明年,胡騎萬人入上谷,殺數百人。

其明年春,漢使驃騎將軍去病將萬騎出隴西,過焉支山千餘里,擊匈奴,得胡首虜(騎)萬八千餘級,破得休屠王祭天金人。其夏,驃騎將軍復與合騎侯數萬騎出隴西、北地二千里,擊匈奴。過居延,攻祁連山,得胡首虜三萬餘人,裨小王以下七十餘人。是時匈奴亦來入代郡、鴈門,殺略數百人。漢使博望侯及李將軍廣出右北平,擊匈奴左賢王。左賢王圍李將軍,卒可四千人,且盡,殺虜亦過當。會博望侯軍救至,李將軍得脫。漢失亡數千人,合騎侯後驃騎將軍期,及與博望侯皆當死,贖為庶人。

其秋,單于怒渾邪王、休屠王居西方為漢所殺虜數萬人,欲召誅之。渾邪王與休屠王恐,謀降漢,漢使驃騎將軍往迎之。渾邪王殺休屠王,并將其眾降漢。凡四萬餘人,號十萬。於是漢已得渾邪王,則隴西、北地、河西益少胡寇,徙關東貧民處所奪匈奴河南、新秦中以實之,而減北地以西戍卒半。其明年,匈奴入右北平、定襄各數萬騎,殺略千餘人而去。

其明年春,漢謀曰「翕侯信為單于計,居幕北,以為漢兵不能至」。乃粟馬發十萬騎,(負)私[負]從馬凡十四萬匹,糧重不與焉。令大將軍青、驃騎將軍去病中分軍,大將軍出定襄,驃騎將軍出代,咸約絕幕擊匈奴。單于聞之,遠其輜重,以精兵待於幕北。與漢大將軍接戰一日,會暮,大風起,漢兵縱左右翼圍單于。單于自度戰不能如漢兵,單于遂獨身與壯騎數百潰漢圍西北遁走。漢兵夜追不得。行斬捕匈奴首虜萬九千級,北至闐顏山趙信城而還。

單于之遁走,其兵往往與漢兵相亂而隨單于。單于久不與其大眾相得,其右谷蠡王以為單于死,乃自立為單于。真單于復得其眾,而右谷蠡王乃去其單于號,復為右谷蠡王。

漢驃騎將軍之出代二千餘里,與左賢王接戰,漢兵得胡首虜凡七萬餘級,左賢王將皆遁走。驃騎封於狼居胥山,禪姑衍,臨翰海而還。

是後匈奴遠遁,而幕南無王庭。漢度河自朔方以西至令居,往往通渠置田,官吏卒五六萬人,稍蠶食,地接匈奴以北。

初,漢兩將軍大出圍單于,所殺虜八九萬,而漢士卒物故亦數萬,漢馬死者十餘萬。匈奴雖病,遠去,而漢亦馬少,無以復往。匈奴用趙信之計,遣使於漢,好辭請和親。天子下其議,或言和親,或言遂臣之。丞相長史任敞曰:「匈奴新破,困,宜可使為外臣,朝請於邊。」漢使任敞於單于。單于聞敞計,大怒,留之不遣。先是漢亦有所降匈奴使者,單于亦輒留漢使相當。漢方復收士馬,會驃騎將軍去病死,於是漢久不北擊胡。

數歲,伊稚斜單于立十三年死,子烏維立為單于。是歲,漢元鼎三年也。烏維單于立,而漢天子始出巡郡縣。其後漢方南誅兩越,不擊匈奴,匈奴亦不侵入邊。

烏維單于立三年,漢已滅南越,遣故太仆賀將萬五千騎出九原二千餘里,至浮苴井而還,不見匈奴一人。漢又遣故從驃侯趙破奴萬餘騎出令居數千里,至匈河水而還,亦不見匈奴一人。

是時天子巡邊,至朔方,勒兵十八萬騎以見武節,而使郭吉風告單于。郭吉既至匈奴,匈奴主客問所使,郭吉禮卑言好,曰:「吾見單于而口言。」單于見吉,吉曰:「南越王頭已懸於漢北闕。今單于(能)即[能]前與漢戰,天子自將兵待邊;單于即不能,即南面而臣於漢。何徒遠走,亡匿於幕北寒苦無水草之地,毋為也。」語卒而單于大怒,立斬主客見者,而留郭吉不歸,遷之北海上。而單于終不肯為寇於漢邊,休養息士馬,習射獵,數使使於漢,好辭甘言求請和親。

漢使王烏等窺匈奴。匈奴法,漢使非去節而以墨黥其面者不得入穹廬。王烏,北地人,習胡俗,去其節,黥面,得入穹廬。單于愛之,詳許甘言,為遣其太子入漢為質,以求和親。

漢使楊信於匈奴。是時漢東拔穢貉、朝鮮以為郡,而西置酒泉郡以鬲絕胡與羌通之路。漢又西通月氏、大夏,又以公主妻烏孫王,以分匈奴西方之援國。又北益廣田至胘雷為塞,而匈奴終不敢以為言。是歲,翕侯信死,漢用事者以匈奴為已弱,可臣從也。楊信為人剛直屈彊,素非貴臣,單于不親。單于欲召入,不肯去節,單于乃坐穹廬外見楊信。楊信既見單于,說曰:「即欲和親,以單于太子為質於漢。」單于曰:「非故約。故約,漢常遣翁主,給繒絮食物有品,以和親,而匈奴亦不擾邊。今乃欲反古,令吾太子為質,無幾矣。」匈奴俗,見漢使非中貴人,其儒先,以為欲說,折其辯;其少年,以為欲刺,折其氣。每漢使入匈奴,匈奴輒報償。漢留匈奴使,匈奴亦留漢使,必得當乃肯止。

楊信既歸,漢使王烏,而單于復肴甘言,欲多得漢財物,紿謂王烏曰:「吾欲入漢見天子,面相約為兄弟。」王烏歸報漢,漢為單于筑邸于長安。匈奴曰:「非得漢貴人使,吾不與誠語。」匈奴使其貴人至漢,病,漢予藥,欲愈之,不幸而死。而漢使路充國佩二千石印綬往使,因送其喪,厚葬直數千金,曰「此漢貴人也」。單于以為漢殺吾貴使者,乃留路充國不歸。諸所言者,單于特空紿王烏,殊無意入漢及遣太子來質。於是匈奴數使奇兵侵犯邊。漢乃拜郭昌為拔胡將軍,及浞野侯屯朔方以東,備胡。路充國留匈奴三歲,單于死。

烏維單于立十歲而死,子烏師廬立為單于。年少,號為兒單于。是歲元封六年也。自此之後,單于益西北,左方兵直雲中,右方直酒泉、燉煌郡。

兒單于立,漢使兩使者,一弔單于,一弔右賢王,欲以乖其國。使者入匈奴,匈奴悉將致單于。單于怒而盡留漢使。漢使留匈奴者前後十餘輩,而匈奴使來,漢亦輒留相當。

是歲,漢使貳師將軍廣利西伐大宛,而令因杅將軍敖筑受降城。其冬,匈奴大雨雪,畜多饑寒死。兒單于年少,好殺伐,國人多不安。左大都尉欲殺單于,使人閒告漢曰:「我欲殺單于降漢,漢遠,即兵來迎我,我即發。」初,漢聞此言,故筑受降城,猶以為遠。

其明年春,漢使浞野侯破奴將二萬餘騎出朔方西北二千餘里,期至浚稽山而還。浞野侯既至期而還,左大都尉欲發而覺,單于誅之,發左方兵擊浞野。浞野侯行捕首虜得數千人。還,未至受降城四百里,匈奴兵八萬騎圍之。浞野侯夜自出求水,匈奴閒捕,生得浞野侯,因急擊其軍。軍中郭縱為護,維王為渠,相與謀曰:「及諸校尉畏亡將軍而誅之,莫相勸歸。」軍遂沒於匈奴。匈奴兒單于大喜,遂遣奇兵攻受降城。不能下,乃寇入邊而去。其明年,單于欲自攻受降城,未至,病死。

兒單于立三歲而死。子年少,匈奴乃立其季父烏維單于弟右賢王呴犁湖為單于。是歲太初三年也。

呴犁湖單于立,漢使光祿徐自為出五原塞數百里,遠者千餘里,筑城鄣列亭至廬朐,而使游擊將軍韓說、長平侯衛伉屯其旁,使彊弩都尉路博德筑居延澤上。

其秋,匈奴大入定襄、雲中,殺略數千人,敗數二千石而去,行破壞光祿所筑城列亭鄣。又使右賢王入酒泉、張掖,略數千人。會任文擊救,盡復失所得而去。是歲,貳師將軍破大宛,斬其王而還。匈奴欲遮之,不能至。其冬,欲攻受降城,會單于病死。

呴犁湖單于立一歲死。匈奴乃立其弟左大都尉且鞮侯為單于。

漢既誅大宛,威震外國。天子意欲遂困胡,乃下詔曰:「高皇帝遺朕平城之憂,高后時單于書絕悖逆。昔齊襄公復九世之讎,春秋大之。」是歲太初四年也。

且鞮侯單于既立,盡歸漢使之不降者。路充國等得歸。單于初立,恐漢襲之,乃自謂「我兒子,安敢望漢天子!漢天子,我丈人行也」。漢遣中郎將蘇武厚幣賂遺單于。單于益驕,禮甚倨,非漢所望也。其明年,浞野侯破奴得亡歸漢。

其明年,漢使貳師將軍廣利以三萬騎出酒泉,擊右賢王於天山,得胡首虜萬餘級而還。匈奴大圍貳師將軍,幾不脫。漢兵物故什六七。漢復使因杅將軍敖出西河,與彊弩都尉會涿涂山,毋所得。又使騎都尉李陵將步騎五千人,出居延北千餘里,與單于會,合戰,陵所殺傷萬餘人,兵及食盡,欲解歸,匈奴圍陵,陵降匈奴,其兵遂沒,得還者四百人。單于乃貴陵,以其女妻之。

後二歲,復使貳師將軍將六萬騎,步兵十萬,出朔方。彊弩都尉路博德將萬餘人,與貳師會。游擊將軍說將步騎三萬人,出五原。因杅將軍敖將萬騎步兵三萬人,出鴈門。匈奴聞,悉遠其累重於余吾水北,而單于以十萬騎待水南,與貳師將軍接戰。貳師乃解而引歸,與單于連戰十餘日。貳師聞其家以巫蠱族滅,因并眾降匈奴,得來還千人一兩人耳。游擊說無所得。因杅敖與左賢王戰,不利,引歸。是歲漢兵之出擊匈奴者不得言功多少,功不得御。有詔捕太醫令隨但,言貳師將軍家室族滅,使廣利得降匈奴。

太史公曰:孔氏著春秋,隱桓之閒則章,至定哀之際則微,為其切當世之文而罔褒,忌諱之辭也。世俗之言匈奴者,患其徼一時之權,而務莖讇納其說,以便偏指,不參彼己;將率席中國廣大,氣奮,人主因以決策,是以建功不深。堯雖賢,興事業不成,得禹而九州寧。且欲興聖統,唯在擇任將相哉!唯在擇任將相哉!


\end{pinyinscope}