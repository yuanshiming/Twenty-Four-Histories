\article{萬石張叔列傳}

\begin{pinyinscope}
萬石君名奮,其父趙人也,姓石氏。趙亡,徙居溫。高祖東擊項籍,過河內,時奮年十五,為小吏,侍高祖。高祖與語,愛其恭敬,問曰:「若何有?」對曰:「奮獨有母,不幸失明。家貧。有姊,能鼓琴。」高祖曰:「若能從我乎?」曰:「願盡力。」於是高祖召其姊為美人,以奮為中涓,受書謁,徙其家長安中戚裏,以姊為美人故也。其官至孝文時,積功勞至大中大夫。無文學,恭謹無與比。

文帝時,東陽侯張相如為太子太傅,免。選可為傅者,皆推奮,奮為太子太傅。及孝景即位,以為九卿;迫近,憚之,徙奮為諸侯相。奮長子建,次子甲,次子乙,次子慶,皆以馴行孝謹,官皆至二千石。於是景帝曰:「石君及四子皆二千石,人臣尊寵乃集其門。」號奮為萬石君。

孝景帝季年,萬石君以上大夫祿歸老于家,以歲時為朝臣。過宮門闕,萬石君必下車趨,見路馬必式焉。子孫為小吏,來歸謁,萬石君必朝服見之,不名。子孫有過失,不譙讓,為便坐,對案不食。然後諸子相責,因長老肉袒固謝罪,改之,乃許。子孫勝冠者在側,雖燕居必冠,申申如也。僮仆訢訢如也,唯謹。上時賜食於家,必稽首俯伏而食之,如在上前。其執喪,哀戚甚悼。子孫遵教,亦如之。萬石君家以孝謹聞乎郡國,雖齊魯諸儒質行,皆自以為不及也。

建元二年,郎中令王臧以文學獲罪。皇太后以為儒者文多質少,今萬石君家不言而躬行,乃以長子建為郎中令,少子慶為內史。

建老白首,萬石君尚無恙。建為郎中令,每五日洗沐歸謁親,入子舍,竊問侍者,取親中帬廁牏,身自浣滌,復與侍者,不敢令萬石君知,以為常。建為郎中令,事有可言,屏人恣言,極切;至廷見,如不能言者。是以上乃親尊禮之。

萬石君徙居陵裏。內史慶醉歸,入外門不下車。萬石君聞之,不食。慶恐,肉袒請罪,不許。舉宗及兄建肉袒,萬石君讓曰:「內史貴人,入閭里,里中長老皆走匿,而內史坐車中自如,固當!」乃謝罷慶。慶及諸子弟入里門,趨至家。

萬石君以元朔五年中卒。長子郎中令建哭泣哀思,扶杖乃能行。歲餘,建亦死。諸子孫咸孝,然建最甚,甚於萬石君。

建為郎中令,書奏事,事下,建讀之,曰:「誤書!『馬』者與尾當五,今乃四,不足一。上譴死矣!」甚惶恐。其為謹慎,雖他皆如是。

萬石君少子慶為太仆,御出,上問車中幾馬,慶以策數馬畢,舉手曰:「六馬。」慶於諸子中最為簡易矣,然猶如此。為齊相,舉齊國皆慕其家行,不言而齊國大治,為立石相祠。

元狩元年,上立太子,選群臣可為傅者,慶自沛守為太子太傅,七歲遷為御史大夫。

元鼎五年秋,丞相有罪,罷。制詔御史:「萬石君先帝尊之,子孫孝,其以御史大夫慶為丞相,封為牧丘侯。」是時漢方南誅兩越,東擊朝鮮,北逐匈奴,西伐大宛,中國多事。天子巡狩海內,修上古神祠,封禪,興禮樂。公家用少,桑弘羊等致利,王溫舒之屬峻法,兒寬等推文學至九卿,更進用事,事不關決於丞相,丞相醇謹而已。在位九歲,無能有所匡言。嘗欲請治上近臣所忠、九卿咸宣罪,不能服,反受其過,贖罪。

元封四年中,關東流民二百萬口,無名數者四十萬,公卿議欲請徙流民於邊以適之。上以為丞相老謹,不能與其議,乃賜丞相告歸,而案御史大夫以下議為請者。丞相慚不任職,乃上書曰:「慶幸得待罪丞相,罷駑無以輔治,城郭倉庫空虛,民多流亡,罪當伏斧質,上不忍致法。願歸丞相侯印,乞骸骨歸,避賢者路。」天子曰:「倉廩既空,民貧流亡,而君欲請徙之,搖蕩不安,動危之,而辭位,君欲安歸難乎?」以書讓慶,慶甚慚,遂復視事。

慶文深審謹,然無他大略,為百姓言。後三歲餘,太初二年中,丞相慶卒,謚為恬侯。慶中子德,慶愛用之,上以德為嗣,代侯。後為太常,坐法當死,贖免為庶人。慶方為丞相,諸子孫為吏更至二千石者十三人。及慶死後,稍以罪去,孝謹益衰矣。

建陵侯衛綰者,代大陵人也。綰以戲車為郎,事文帝,功次遷為中郎將,醇謹無他。孝景為太子時,召上左右飲,而綰稱病不行。文帝且崩時,屬孝景曰:「綰長者,善遇之。」及文帝崩,景帝立,歲餘不噍呵綰,綰日以謹力。

景帝幸上林,詔中郎將參乘,還而問曰:「君知所以得參乘乎?」綰曰:「臣從車士幸得以功次遷為中郎將,不自知也。」上問曰:「吾為太子時召君,君不肯來,何也?」對曰:「死罪,實病!」上賜之劍。綰曰:「先帝賜臣劍凡六,劍不敢奉詔。」上曰:「劍,人之所施易,獨至今乎?」綰曰:「具在。」上使取六劍,劍尚盛,未嘗服也。郎官有譴,常蒙其罪,不與他將爭;有功,常讓他將。上以為廉,忠實無他腸,乃拜綰為河閒王太傅。吳楚反,詔綰為將,將河閒兵擊吳楚有功,拜為中尉。三歲,以軍功,孝景前六年中封綰為建陵侯。

其明年,上廢太子,誅栗卿之屬。上以為綰長者,不忍,乃賜綰告歸,而使郅都治捕栗氏。既已,上立膠東王為太子,召綰,拜為太子太傅。久之,遷為御史大夫。五歲,代桃侯舍為丞相,朝奏事如職所奏。然自初官以至丞相,終無可言。天子以為敦厚,可相少主,尊寵之,賞賜甚多。

為丞相三歲,景帝崩,武帝立。建元年中,丞相以景帝疾時諸官囚多坐不辜者,而君不任職,免之。其後綰卒,子信代。坐酎金失侯。

塞侯直不疑者,南陽人也。為郎,事文帝。其同舍有告歸,誤持同舍郎金去,已而金主覺,妄意不疑,不疑謝有之,買金償。而告歸者來而歸金,而前郎亡金者大慚,以此稱為長者。文帝稱舉,稍遷至太中大夫。朝廷見,人或毀曰:「不疑狀貌甚美,然獨無柰其善盜嫂何也!」不疑聞,曰:「我乃無兄。」然終不自明也。

吳楚反時,不疑以二千石將兵擊之。景帝後元年,拜為御史大夫。天子修吳楚時功,乃封不疑為塞侯。武帝建元年中,與丞相綰俱以過免。

不疑學老子言。其所臨,為官如故,唯恐人知其為吏跡也。不好立名稱,稱為長者。不疑卒,子相如代。孫望,坐酎金失侯。

郎中令周文者,名仁,其先故任城人也。以醫見。景帝為太子時,拜為舍人,積功稍遷,孝文帝時至太中大夫。景帝初即位,拜仁為郎中令。

仁為人陰重不泄,常衣敝補衣溺袴,期為不絜清,以是得幸。景帝入臥內,於後宮祕戲,仁常在旁。至景帝崩,仁尚為郎中令,終無所言。上時問人,仁曰:「上自察之。」然亦無所毀。以此景帝再自幸其家。家徙陽陵。上所賜甚多,然常讓,不敢受也。諸侯群臣賂遺,終無所受。

武帝立,以為先帝臣,重之。仁乃病免,以二千石祿歸老,子孫咸至大官矣。

御史大夫張叔者,名歐,安丘侯說之庶子也。孝文時以治刑名言事太子。然歐雖治刑名家,其人長者。景帝時尊重,常為九卿。至武帝元朔四年,韓安國免,詔拜歐為御史大夫。自歐為吏,未嘗言案人,專以誠長者處官。官屬以為長者,亦不敢大欺。上具獄事,有可卻,卻之;不可者,不得已,為涕泣面對而封之。其愛人如此。

老病甐,請免。於是天子亦策罷,以上大夫祿歸老于家。家於陽陵。子孫咸至大官矣。

太史公曰:仲尼有言曰「君子欲訥於言而敏於行」,其萬石、建陵、張叔之謂邪?是以其教不肅而成,不嚴而治。塞侯微巧,而周文處讇君子譏之,為其近於佞也。然斯可謂篤行君子矣!


\end{pinyinscope}