\article{五宗世家}

\begin{pinyinscope}
孝景皇帝子凡十三人為王,而母五人,同母者為宗親。栗姬子曰榮、德、閼于。程姬子曰餘、非、端。賈夫人子曰彭祖、勝。唐姬子曰發。王夫人兒姁子曰越、寄、乘、舜。

河閒獻王德,以孝景帝前二年用皇子為河閒王。好儒學,被服造次必於儒者。山東諸儒多從之游。

二十六年卒,子共王不害立。四年卒,子剛王基代立。十二年卒,子頃王授代立。

臨江哀王閼于,以孝景帝前二年用皇子為臨江王。三年卒,無後,國除為郡。

臨江閔王榮,以孝景前四年為皇太子,四歲廢,用故太子為臨江王。

四年,坐侵廟壖垣為宮,上徵榮。榮行,祖於江陵北門。既已上車,軸折車廢。江陵父老流涕竊言曰:「吾王不反矣!」榮至,詣中尉府簿。中尉郅都責訊王,王恐,自殺。葬藍田。燕數萬銜土置冢上,百姓憐之。

榮最長,死無後,國除,地入于漢,為南郡。

右三國本王皆栗姬之子也。

魯共王餘,以孝景前二年用皇子為淮陽王。二年,吳楚反破後,以孝景前三年徙為魯王。好治宮室苑囿狗馬。季年好音,不喜辭辯。為人吃。

二十六年卒,子光代為王。初好音輿馬;晚節嗇,惟恐不足於財。

江都易王非,以孝景前二年用皇子為汝南王。吳楚反時,非年十五,有材力,上書願擊吳。景帝賜非將軍印,擊吳。吳已破,二歲,徙為江都王,治吳故國,以軍功賜天子旌旗。元光五年,匈奴大入漢為賊,非上書願擊匈奴,上不許。非好氣力,治宮觀,招四方豪桀,驕奢甚。

立二十六年卒,子建立為王。七年自殺。淮南、衡山謀反時,建頗聞其謀。自以為國近淮南,恐一日發,為所并,即陰作兵器,而時佩其父所賜將軍印,載天子旗以出。易王死未葬,建有所說易王寵美人淖姬,夜使人迎與姦服舍中。及淮南事發,治黨與頗及江都王建。建恐,因使人多持金錢,事絕其獄。而又信巫祝,使人禱祠妄言。建又盡與其姊弟姦。事既聞,漢公卿請捕治建。天子不忍,使大臣即訊王。王服所犯,遂自殺。國除,地入于漢,為廣陵郡。

膠西于王端,以孝景前三年吳楚七國反破後,端用皇子為膠西王。端為人賊戾,又陰痿,一近婦人,病之數月。而有愛幸少年為郎。為郎者頃之與後宮亂,端禽滅之,及殺其子母。數犯上法,漢公卿數請誅端,天子為兄弟之故不忍,而端所為滋甚。有司再請削其國,去太半。端心慍,遂為無訾省。府庫壞漏盡,腐財物以巨萬計,終不得收徙。令吏毋得收租賦。端皆去衛,封其宮門,從一門出游。數變名姓,為布衣,之他郡國。

相、二千石往者,奉漢法以治,端輒求其罪告之,無罪者詐藥殺之。所以設詐究變,彊足以距諫,智足以飾非。相、二千石從王治,則漢繩以法。故膠西小國,而所殺傷二千石甚眾。

立四十七年,卒,竟無男代後,國除,地入于漢,為膠西郡。

右三國本王皆程姬之子也。

趙王彭祖,以孝景前二年用皇子為廣川王。趙王遂反破後,彭祖王廣川。四年,徙為趙王。十五年,孝景帝崩。彭祖為人巧佞卑諂,足恭而心刻深。好法律,持詭辯以中人。彭祖多內寵姬及子孫。相、二千石欲奉漢法以治,則害於王家。是以每相、二千石至,彭祖衣皁布衣,自行迎,除二千石舍,多設疑事以作動之,得二千石失言,中忌諱,輒書之。二千石欲治者,則以此迫劫;不聽,乃上書告,及汙以姦利事。彭祖立五十餘年,相、二千石無能滿二歲,輒以罪去,大者死,小者刑,以故二千石莫敢治。而趙王擅權,使使即縣為賈人榷會,入多於國經租稅。以是趙王家多金錢,然所賜姬諸子,亦盡之矣。彭祖取笔江都易王寵姬王建所盜與姦淖姬者為姬,甚愛之。

彭祖不好治宮室、禨祥,好為吏事。上書願督國中盜賊。常夜從走卒行徼邯鄲中。諸使過客以彭祖險陂,莫敢留邯鄲。

其太子丹與其女及同產姊姦,與其客江充有卻。充告丹,丹以故廢。趙更立太子。

中山靖王勝,以孝景前三年用皇子為中山王。十四年,孝景帝崩。勝為人樂酒好內,有子枝屬百二十餘人。常與兄趙王相非,曰:「兄為王,專代吏治事。王者當日聽音樂聲色。」趙王亦非之,曰:「中山王徒日淫,不佐天子拊循百姓,何以稱為藩臣!」

立四十二年卒,子哀王昌立。一年卒,子昆侈代為中山王。

右二國本王皆賈夫人之子也。

長沙定王發,發之母唐姬,故程姬侍者。景帝召程姬,程姬有所辟,不願進,而飾侍者唐兒使夜進。上醉不知,以為程姬而幸之,遂有身。已乃覺非程姬也。及生子,因命曰發。以孝景前二年用皇子為長沙王。以其母微,無寵,故王卑溼貧國。

立二十七年卒,子康王庸立。二十八年,卒,子鮒鮈立為長沙王。

右一國本王唐姬之子也。廣川惠王越,以孝景中二年用皇子為廣川王。

十二年卒,子齊立為王。齊有幸臣桑距。已而有罪,欲誅距,距亡,王因禽其宗族。距怨王,乃上書告王齊與同產姦。自是之後,王齊數上書告言漢公卿及幸臣所忠等。

膠東康王寄,以孝景中二年用皇子為膠東王。二十八年卒。淮南王謀反時,寄微聞其事,私作樓車鏃矢戰守備,候淮南之起。及吏治淮南之事,辭出之。寄於上最親,意傷之,發病而死,不敢置後,於是上(問)[聞]。寄有長子者名賢,母無寵;少子名慶,母愛幸,寄常欲立之,為不次,因有過,遂無言。上憐之,乃以賢為膠東王奉康王嗣,而封慶於故衡山地,為六安王。

膠東王賢立十四年卒,謚為哀王。子慶為王。

六安王慶,以元狩二年用膠東康王子為六安王。

清河哀王乘,以孝景中三年用皇子為清河王。十二年卒,無後,國除,地入于漢,為清河郡。

常山憲王舜,以孝景中五年用皇子為常山王。舜最親,景帝少子,驕怠多淫,數犯禁,上常寬釋之。立三十二年卒,太子勃代立為王。

初,憲王舜有所不愛姬生長男棁。棁以母無寵故,亦不得幸於王。王後修生太子勃。王內多,所幸姬生子平、子商,王后希得幸。及憲王病甚,諸幸姬常侍病,故王后亦以妒媢不常侍病,輒歸舍。醫進藥,太子勃不自嘗藥,又不宿留侍病。及王薨,王后、太子乃至。憲王雅不以長子棁為人數,及薨,又不分與財物。郎或說太子、王后,令諸子與長子棁共分財物,太子、王后不聽。太子代立,又不收恤棁。棁怨王后、太子。漢使者視憲王喪,棁自言憲王病時,王后、太子不侍,及薨,六日出舍,太子勃私姦,飲酒,博戲,擊筑,與女子載馳,環城過市,入牢視囚。天子遣大行騫驗王后及問王勃,請逮勃所與姦諸證左,王又匿之。吏求捕勃大急,使人致擊笞掠,擅出漢所疑囚者。有司請誅憲王後修及王勃。上以修素無行,使棁陷之罪,勃無良師傅,不忍誅。有司請廢王後修,徙王勃以家屬處房陵,上許之。

勃王數月,遷于房陵,國絕。月餘,天子為最親,乃詔有司曰:「常山憲王蚤夭,后妾不和,適孽誣爭,陷于不義以滅國,朕甚閔焉。其封憲王子平三萬戶,為真定王;封子商三萬戶,為泗水王。」

真定王平,元鼎四年用常山憲王子為真定王。

泗水思王商,以元鼎四年用常山憲王子為泗水王。十一年卒,子哀王安世立。十一年卒,無子。於是上憐泗水王絕,乃立安世弟賀為泗水王。

右四國本王皆王夫人兒姁子也。其後漢益封其支子為六安王、泗水王二國。凡兒姁子孫,於今為六王。

太史公曰:高祖時諸侯皆賦,得自除內史以下,漢獨為置丞相,黃金印。諸侯自除御史、廷尉正、博士,擬於天子。自吳楚反後,五宗王世,漢為置二千石,去「丞相」曰「相」,銀印。諸侯獨得食租稅,奪之權。其後諸侯貧者或乘牛車也。


\end{pinyinscope}