\article{宋微子世家}

\begin{pinyinscope}
微子開者,殷帝乙之首子而帝紂之庶兄也。紂既立,不明,淫亂於政,微子數諫,紂不聽。及祖伊以周西伯昌之修德,滅黎國,懼禍至,以告紂。紂曰:「我生不有命在天乎?是何能為!」於是微子度紂終不可諫,欲死之,及去,未能自決,乃問於太師、少師曰:「殷不有治政,不治四方。我祖遂陳於上,紂沈湎於酒,婦人是用,亂敗湯德於下。殷既小大好草竊姦宄,卿士師師非度,皆有罪辜,乃無維獲,小民乃并興,相為敵讎。今殷其典喪!若涉水無津涯。殷遂喪,越至于今。」曰:「太師,少師,我其發出往?吾家保于喪?今女無故告予,顛躋,如之何其?」太師若曰:「王子,天篤下菑亡殷國,乃毋畏畏,不用老長。今殷民乃陋淫神祇之祀。今誠得治國,國治身死不恨。為死,終不得治,不如去。」遂亡。

箕子者,紂親戚也。紂始為象箸,箕子嘆曰:「彼為象箸,必為玉桮;為桮,則必思遠方珍怪之物而御之矣。輿馬宮室之漸自此始,不可振也。」紂為淫泆,箕子諫,不聽。人或曰:「可以去矣。」箕子曰:「為人臣諫不聽而去,是彰君之惡而自說於民,吾不忍為也。」乃被髪詳狂而為奴。遂隱而鼓琴以自悲,故傳之曰箕子操。

王子比干者,亦紂之親戚也。見箕子諫不聽而為奴,則曰:「君有過而不以死爭,則百姓何辜!」乃直言諫紂。紂怒曰:「吾聞聖人之心有七竅,信有諸乎?」乃遂殺王子比干,刳視其心。

微子曰:「父子有骨肉,而臣主以義屬。故父有過,子三諫不聽,則隨而號之;人臣三諫不聽,則其義可以去矣。」於是太師、少師乃勸微子去,遂行。

周武王伐紂克殷,微子乃持其祭器造於軍門,肉袒面縛,左牽羊,右把茅,膝行而前以告。於是武王乃釋微子,復其位如故。

武王封紂子武庚祿父以續殷祀,使管叔、蔡叔傅相之。

武王既克殷,訪問箕子。

武王曰:「於乎!維天陰定下民,相和其居,我不知其常倫所序。」

箕子對曰:「在昔鯀陻鴻水,汨陳其五行,帝乃震怒,不從鴻範九等,常倫所斁。鯀則殛死,禹乃嗣興。天乃錫禹鴻範九等,常倫所序。

「初一曰五行;二曰五事;三曰八政;四曰五紀;五曰皇極;六曰三德;七曰稽疑;八曰庶徵;九曰嚮用五福,畏用六極。

「五行:一曰水,二曰火,三曰木,四曰金,五曰土。水曰潤下,火曰炎上,木曰曲直,金曰從革,土曰稼穡。潤下作鹹,炎上作苦,曲直作酸,從革作辛,稼穡作甘。

「五事:一曰貌,二曰言,三曰視,四曰聽,五曰思。貌曰恭,言曰從,視曰明,聽曰聰,思曰睿。恭作肅,從作治,明作智,聰作謀,睿作聖。

「八政:一曰食,二曰貨,三曰祀,四曰司空,五曰司徒,六曰司寇,七曰賓,八曰師。

「五紀:一曰歲,二曰月,三曰日,四曰星辰,五曰歷數。

「皇極:皇建其有極,斂時五福,用傅錫其庶民,維時其庶民于女極,錫女保極。凡厥庶民,毋有淫朋,人毋有比德,維皇作極。凡厥庶民,有猷有為有守,女則念之。不協于極,不離于咎,皇則受之。而安而色,曰予所好德,女則錫之福。時人斯其維皇之極。毋侮寡而畏高明。人之有能有為,使羞其行,而國其昌。凡厥正人,既富方穀。女不能使有好于而家,時人斯其辜。于其毋好,女雖錫之福,其作女用咎。毋偏毋頗,遵王之義。毋有作好,遵王之道。毋有作惡,遵王之路。毋偏毋黨,王道蕩蕩。毋黨毋偏,王道平平。毋反毋側,王道正直。會其有極,歸其有極。曰王極之傅言,是夷是訓,于帝其順。凡厥庶民,極之傅言,是順是行,以近天子之光。曰天子作民父母,以為天下王。

「三德:一曰正直,二曰剛克,三曰柔克。平康正直,彊不友剛克,內友柔克,沈漸剛克,高明柔克。維辟作福,維辟作威,維辟玉食。臣無有作福作威玉食。臣有作福作威玉食,其害于而家,凶于而國,人用側頗辟,民用僭忒。

「稽疑:擇建立卜筮人。乃命卜筮,曰雨,曰濟,曰涕,曰霧,曰克,曰貞,曰悔,凡七。卜五,占之用二,衍貣。立時人為卜筮,三人占則從二人之言。女則有大疑,謀及女心,謀及卿士,謀及庶人,謀及卜筮。女則從,龜從,筮從,卿士從,庶民從,是之謂大同,而身其康彊,而子孫其逢吉。女則從,龜從,筮從,卿士逆,庶民逆,吉。卿士從,龜從,筮從,女則逆,庶民逆,吉。庶民從,龜從,筮從,女則逆,卿士逆,吉。女則從,龜從,筮逆,卿士逆,庶民逆,作內吉,作外凶。龜筮共違于人,用靜吉,用作凶。

「庶徵:曰雨,曰陽,曰奧,曰寒,曰風,曰時。五者來備,各以其序,庶草繁廡。一極備,凶。一極亡,凶。曰休徵:曰肅,時雨若,曰治,時暘若;曰知,時奧若;曰謀,時寒若;曰聖,時風若。曰咎徵:曰狂,常雨若;曰僭,常暘若;曰舒,常奧若;曰急,常寒若;曰霧,常風若。王眚維歲,卿士維月,師尹維日。歲月日時毋易,百穀用成,治用明,畯民用章,家用平康。日月歲時既易,百穀用不成,治用昏不明,畯民用微,家用不寧。庶民維星,星有好風,星有好雨。日月之行,有冬有夏。月之從星,則以風雨。

「五福:一曰壽,二曰富,三曰康寧,四曰攸好德,五曰考終命。六極:一曰凶短折,二曰疾,三曰憂,四曰貧,五曰惡,六曰弱。」

於是武王乃封箕子於朝鮮而不臣也。

其後箕子朝周,過故殷虛,感宮室毀壞,生禾黍,箕子傷之,欲哭則不可,欲泣為其近婦人,乃作麥秀之詩以歌詠之。其詩曰:「麥秀漸漸兮,禾黍油油。彼狡僮兮,不與我好兮!」所謂狡童者,紂也。殷民聞之,皆為流涕。

武王崩,成王少,周公旦代行政當國。管、蔡疑之,乃與武庚作亂,欲襲成王、周公。周公既承成王命誅武庚,殺管叔,放蔡叔,乃命微子開代殷後,奉其先祀,作微子之命以申之,國于宋。微子故能仁賢,乃代武庚,故殷之餘民甚戴愛之。

微子開卒,立其弟衍,是為微仲。微仲卒,子宋公稽立。宋公稽卒,子丁公申立。丁公申卒,子湣公共立。湣公共卒,弟煬公熙立。煬公即位,湣公子鮒祀弒煬公而自立,曰「我當立」,是為厲公。厲公卒,子釐公舉立。

釐公十七年,周厲王出奔彘。

二十八年,釐公卒,子惠公覵立。惠公四年,周宣王即位。三十年,惠公卒,子哀公立。哀公元年卒,子戴公立。

戴公二十九年,周幽王為犬戎所殺,秦始列為諸侯。

三十四年,戴公卒,子武公司空立。武公生女為魯惠公夫人,生魯桓公。十八年,武公卒,子宣公力立。

宣公有太子與夷。十九年,宣公病,讓其弟和,曰:「父死子繼,兄死弟及,天下通義也。我其立和。」和亦三讓而受之。宣公卒,弟和立,是為穆公。

穆公九年,病,召大司馬孔父謂曰:「先君宣公捨太子與夷而立我,我不敢忘。我死,必立與夷也。」孔父曰:「群臣皆願立公子馮。」穆公曰:「毋立馮,吾不可以負宣公。」於是穆公使馮出居于鄭。八月庚辰,穆公卒,兄宣公子與夷立,是為殤公。君子聞之,曰:「宋宣公可謂知人矣,立其弟以成義,然卒其子復享之。

殤公元年,衛公子州吁弒其君完自立,欲得諸侯,使告於宋曰:「馮在鄭,必為亂,可與我伐之。」宋許之,與伐鄭,至東門而還。二年,鄭伐宋,以報東門之役。其後諸侯數來侵伐。

九年,大司馬孔父嘉妻好,出,道遇太宰華督,督說,目而觀之。督利孔父妻,乃使人宣言國中曰:「殤公即位十年耳,而十一戰,民苦不堪,皆孔父為之,我且殺孔父以寧民。」是歲,魯弒其君隱公。十年,華督攻殺孔父,取其妻。殤公怒,遂弒殤公,而迎穆公子馮於鄭而立之,是為莊公。

莊公元年,華督為相。九年,執鄭之祭仲,要以立突為鄭君。祭仲許,竟立突。十九年,莊公卒,子湣公捷立。

湣公七年,齊桓公即位。九年,宋水,魯使臧文仲往弔水。湣公自罪曰:「寡人以不能事鬼神,政不修,故水。」臧文仲善此言。此言乃公子子魚教湣公也。

十年夏,宋伐魯,戰於乘丘,魯生虜宋南宮萬。宋人請萬,萬歸宋。十一年秋,湣公與南宮萬獵,因博爭行,湣公怒,辱之,曰:「始吾敬若;今若,魯虜也。」萬有力,病此言,遂以局殺湣公于蒙澤。大夫仇牧聞之,以兵造公門。萬搏牧,牧齒著門闔死。因殺太宰華督,乃更立公子游為君。諸公子奔蕭,公子御說奔亳。萬弟南宮牛將兵圍亳。冬,蕭及宋之諸公子共擊殺南宮牛,弒宋新君游而立湣公弟御說,是為桓公。宋萬奔陳。宋人請以賂陳。陳人使婦人飲之醇酒,以革裹之,歸宋。宋人醢萬也。

桓公二年,諸侯伐宋,至郊而去。三年,齊桓公始霸。二十三年,迎衛公子燬於齊,立之,是為衛文公。文公女弟為桓公夫人。秦穆公即位。三十年,桓公病,太子茲甫讓其庶兄目夷為嗣。桓公義太子意,竟不聽。三十一年春,桓公卒,太子茲甫立,是為襄公。以其庶兄目夷為相。未葬,而齊桓公會諸侯于葵丘,襄公往會。

襄公七年,宋地茀星如雨,與雨偕下;六鶂退蜚,風疾也。

八年,齊桓公卒,宋欲為盟會。十二年春,宋襄公為鹿上之盟,以求諸侯於楚,楚人許之。公子目夷諫曰:「小國爭盟,禍也。」不聽。秋,諸侯會宋公盟于盂。目夷曰:「禍其在此乎?君欲已甚,何以堪之!」於是楚執宋襄公以伐宋。冬,會于亳,以釋宋公。子魚曰:「禍猶未也。」十三年夏,宋伐鄭。子魚曰:「禍在此矣。」秋,楚伐宋以救鄭。襄公將戰,子魚諫曰:「天之棄商久矣,不可。」冬,十一月,襄公與楚成王戰于泓。楚人未濟,目夷曰:「彼眾我寡,及其未濟擊之。」公不聽。已濟未陳,又曰:「可擊。」公曰:「待其已陳。」陳成,宋人擊之。宋師大敗,襄公傷股。國人皆怨公。公曰:「君子不困人於阸,不鼓不成列。」子魚曰:「兵以勝為功,何常言與!必如公言,即奴事之耳,又何戰為?」

楚成王已救鄭,鄭享之;去而取鄭二姬以歸。叔瞻曰:「成王無禮,其不沒乎?為禮卒於無別,有以知其不遂霸也。」

是年,晉公子重耳過宋,襄公以傷於楚,欲得晉援,厚禮重耳以馬二十乘。

十四年夏,襄公病傷於泓而竟卒,子成公王臣立。

成公元年,晉文公即位。三年,倍楚盟親晉,以有德於文公也。四年,楚成王伐宋,宋告急於晉。五年,晉文公救宋,楚兵去。九年,晉文公卒。十一年,楚太子商臣弒其父成王代立。十六年,秦穆公卒。

十七年,成公卒。成公弟御殺太子及大司馬公孫固而自立為君。宋人共殺君御而立成公少子杵臼,是為昭公。

昭公四年,宋敗長翟緣斯於長丘。七年,楚莊王即位。

九年,昭公無道,國人不附。昭公弟鮑革賢而下士。先,襄公夫人欲通於公子鮑,不可,乃助之施於國,因大夫華元為右師。昭公出獵,夫人王姬使衛伯攻殺昭公杵臼。弟鮑革立,是為文公。

文公元年,晉率諸侯伐宋,責以弒君。聞文公定立,乃去。二年,昭公子因文公母弟須與武、繆、戴、莊、桓之族為亂,文公盡誅之,出武、繆之族。

四年春,(鄭)[楚]命(楚)[鄭]伐宋。宋使華元將,鄭敗宋,囚華元。華元之將戰,殺羊以食士,其御羊羹不及,故怨,馳入鄭軍,故宋師敗,得囚華元。宋以兵車百乘文馬四百匹贖華元。未盡入,華元亡歸宋。

十四年,楚莊王圍鄭。鄭伯降楚,楚復釋之。

十六年,楚使過宋,宋有前仇,執楚使。九月,楚莊王圍宋。十七年,楚以圍宋五月不解,宋城中急,無食,華元乃夜私見楚將子反。子反告莊王。王問:「城中何如?」曰:「析骨而炊,易子而食。」莊王曰:「誠哉言!我軍亦有二日糧。」以信故,遂罷兵去。

二十二年,文公卒,子共公瑕立。始厚葬。君子譏華元不臣矣。

共公(元)[十]年,華元善楚將子重,又善晉將欒書,兩盟晉楚。十三年,共公卒。華元為右師,魚石為左師。司馬唐山攻殺太子肥,欲殺華元,華元奔晉,魚石止之,至河乃還,誅唐山。乃立共公少子成,是為平公。

平公三年,楚共王拔宋之彭城,以封宋左師魚石。四年,諸侯共誅魚石,而歸彭城於宋。三十五年,楚公子圍弒其君自立,為靈王。四十四年,平公卒,子元公佐立。

元公三年,楚公子棄疾弒靈王,自立為平王。八年,宋火。十年,元公毋信,詐殺諸公子,大夫華、向氏作亂。楚平王太子建來奔,見諸華氏相攻亂,建去如鄭。十五年,元公為魯昭公避季氏居外,為之求入魯,行道卒,子景公頭曼立。

景公十六年,魯陽虎來奔,已復去。二十五年,孔子過宋,宋司馬桓魋惡之,欲殺孔子,孔子微服去。三十年,曹倍宋,又倍晉,宋伐曹,晉不救,遂滅曹有之。三十六年,齊田常弒簡公。

三十七年,楚惠王滅陳。熒惑守心。心,宋之分野也。景公憂之。司星子韋曰:「可移於相。」景公曰:「相,吾之股肱。」曰:「可移於民。」景公曰:「君者待民。」曰:「可移於歲。」景公曰:「歲饑民困,吾誰為君!」子韋曰:「天高聽卑。君有君人之言三,熒惑宜有動。」於是候之,果徙三度。

六十四年,景公卒。宋公子特攻殺太子而自立,是為昭公。昭公者,元公之曾庶孫也。昭公父公孫糾,糾父公子褍秦,褍秦即元公少子也。景公殺昭公父糾,故昭公怨殺太子而自立。

昭公四十七年卒,子悼公購由立。悼公八年卒,子休公田立。休公田二十三年卒,子辟公辟兵立。辟公三年卒,子剔成立。剔成四十一年,剔成弟偃攻襲剔成,剔成敗奔齊,偃自立為宋君。

君偃十一年,自立為王。東敗齊,取五城;南敗楚,取地三百里;西敗魏軍,乃與齊、魏為敵國。盛血以韋囊,縣而射之,命曰「射天」。淫於酒婦人。群臣諫者輒射之。於是諸侯皆曰「桀宋」。「宋其復為紂所為,不可不誅」。告齊伐宋。王偃立四十七年,齊湣王與魏、楚伐宋,殺王偃,遂滅宋而三分其地。

太史公曰:孔子稱「微子去之,箕子為之奴,比干諫而死,殷有三仁焉」。春秋譏宋之亂自宣公廢太子而立弟,國以不寧者十世。襄公之時,修行仁義,欲為盟主。其大夫正考父美之,故追道契、湯、高宗,殷所以興,作商頌。襄公既敗於泓,而君子或以為多,傷中國闕禮義,褒之也,宋襄之有禮讓也。


\end{pinyinscope}