\article{吳太伯世家}

\begin{pinyinscope}
吳太伯,太伯弟仲雍,皆周太王之子,而王季歷之兄也。季歷賢,而有聖子昌,太王欲立季歷以及昌,於是太佰、仲雍二人乃奔荊蠻,文身斷發,示不可用,以避季歷。季歷果立,是為王季,而昌為文王。太伯之奔荊蠻,自號句吳。荊蠻義之,從而歸之千餘家,立為吳太伯。

太伯卒,無子,弟仲雍立,是為吳仲雍。仲雍卒,子季簡立。季簡卒,子叔達立。叔達卒,子周章立。是時周武王克殷,求太伯、仲雍之後,得周章。周章已君吳,因而封之。乃封周章弟虞仲於周之北故夏虛,是為虞仲,列為諸侯。

周章卒,子熊遂立,熊遂卒,子柯相立。柯相卒,子彊鳩夷立。彊鳩夷卒,子餘橋疑吾立。餘橋疑吾卒,子柯盧立。柯盧卒,子周繇立。周繇卒,子屈羽立。屈羽卒,子夷吾立。夷吾卒,子禽處立。禽處卒,子轉立。轉卒,子頗高立。頗高卒,子句卑立。是時晉獻公滅周北虞公,以開晉伐虢也。句卑卒,子去齊立。去齊卒,子壽夢立。壽夢立而吳始益大,稱王。

自太伯作吳,五世而武王克殷,封其後為二:其一虞,在中國;其一吳,在夷蠻。十二世而晉滅中國之虞。中國之虞滅二世,而夷蠻之吳興。大凡從太伯至壽夢十九世。

王壽夢二年,楚之亡大夫申公巫臣怨楚將子反而奔晉,自晉使吳,教吳用兵乘車,令其子為吳行人,吳於是始通於中國。吳伐楚。十六年,楚共王伐吳,至衡山。

二十五年,王壽夢卒。壽夢有子四人,長曰諸樊,次曰餘祭,次曰餘眛,次曰季札。季札賢,而壽夢欲立之,季札讓不可,於是乃立長子諸樊,攝行事當國。

王諸樊元年,諸樊已除喪,讓位季札。季札謝曰:「曹宣公之卒也,諸侯與曹人不義曹君,將立子臧,子臧去之,以成曹君,君子曰『能守節矣』。君義嗣,誰敢干君!有國,非吾節也。札雖不材,願附於子臧之義。」吳人固立季札,季札棄其室而耕,乃捨之。秋,吳伐楚,楚敗我師。四年,晉平公初立。

十三年,王諸樊卒。有命授弟餘祭,欲傳以次,必致國於季札而止,以稱先王壽夢之意,且嘉季札之義,兄弟皆欲致國,令以漸至焉。季札封於延陵,故號曰延陵季子。

王餘祭三年,齊相慶封有罪,自齊來奔吳。吳予慶封朱方之縣,以為奉邑,以女妻之,富於在齊。

四年,吳使季札聘於魯,請觀周樂。為歌周南、召南。曰:「美哉,始基之矣,猶未也。然勤而不怨。」歌邶、鄘、衛。曰:「美哉,淵乎,憂而不困者也。吾聞衛康叔、武公之德如是,是其衛風乎?」歌王。曰:「美哉,思而不懼,其周之東乎?」歌鄭。曰:「其細已甚,民不堪也,是其先亡乎?」歌齊。曰:「美哉,泱泱乎大風也哉。表東海者,其太公乎?國未可量也。」歌豳。曰:「美哉,蕩蕩乎,樂而不淫,其周公之東乎?」歌秦。曰:「此之謂夏聲。夫能夏則大,大之至也,其周之舊乎?」歌魏。曰:「美哉,沨沨乎,大而寬,儉而易,行以德輔,此則盟主也。」歌唐。曰:「思深哉,其有陶唐氏之遺風乎?不然,何憂之遠也?非令德之後,誰能若是!」歌陳。曰:「國無主,其能久乎?」自鄶以下,無譏焉。歌小雅。曰:「美哉,思而不貳,怨而不言,其周德之衰乎?猶有先王之遺民也。」歌大雅。曰:「廣哉,熙熙乎,曲而有直體,其文王之德乎?」歌頌。曰:「至矣哉,直而不倨,曲而不詘,近而不偪遠而不攜,而遷不淫,復而不厭,哀而不愁,樂而不荒,用而不匱,廣而不宣,施而不費,取而不貪,處而不底,行而不流。五聲和,八風平,節有度,守有序,盛德之所同也。」見舞象箾、南籥者,曰:「美哉,猶有感。」見舞大武,曰:「美哉,周之盛也其若此乎?」見舞韶護者,曰:「聖人之弘也,猶有慚德,聖人之難也!」見舞大夏,曰:「美哉,勤而不德!非禹其誰能及之?」見舞招箾,曰:「德至矣哉,大矣,如天之無不燾也,如地之無不載也,雖甚盛德,無以加矣。觀止矣,若有他樂,吾不敢觀。」

去魯,遂使齊。說晏平仲曰:「子速納邑與政。無邑無政,乃免於難。齊國之政將有所歸;未得所歸,難未息也。」故晏子因陳桓子以納政與邑,是以免於欒高之難。

去齊,使於鄭。見子產,如舊交。謂子產曰:「鄭之執政侈,難將至矣,政必及子。子為政,慎以禮。不然,鄭國將敗。」去鄭,適衛。說蘧瑗、史狗、史、公子荊、公叔發、公子朝曰:「衛多君子,未有患也。」

自衛如晉,將舍於宿,聞鐘聲,曰:「異哉!吾聞之,辯而不德,必加於戮。夫子獲罪於君以在此,懼猶不足,而又可以畔乎?夫子之在此,猶燕之巢于幕也。君在殯而可以樂乎?」遂去之。文子聞之,終身不聽琴瑟。

適晉,說趙文子、韓宣子、魏獻子曰:「晉國其萃於三家乎!」將去,謂叔向曰:「吾子勉之!君侈而多良,大夫皆富,政將在三家。吾子直,必思自免於難。」

季札之初使,北過徐君。徐君好季札劍,口弗敢言。季札心知之,為使上國,未獻。還至徐,徐君已死,於是乃解其寶劍,系之徐君冢樹而去。從者曰:「徐君已死,尚誰予乎?」季子曰:「不然。始吾心已許之,豈以死倍吾心哉!」

七年,楚公子圍弒其王夾敖而代立,是為靈王。十年,楚靈王會諸侯而以伐吳之朱方,以誅齊慶封。吳亦攻楚,取三邑而去。十一年,楚伐吳,至雩婁。十二年,楚復來伐,次於乾谿,楚師敗走。

十七年,王餘祭卒,弟餘眛立。王餘眛二年,楚公子棄疾弒其君靈王代立焉。

四年,王餘眛卒,欲授弟季札。季札讓,逃去。於是吳人曰:「先王有命,兄卒弟代立,必致季子。季子今逃位,則王餘眛後立。今卒,其子當代。」乃立王餘眛之子僚為王。

王僚二年,公子光伐楚,敗而亡王舟。光懼,襲楚,復得王舟而還。

五年,楚之亡臣伍子胥來奔,公子光客之。公子光者,王諸樊之子也。常以為吾父兄弟四人,當傳至季子。季子即不受國,光父先立。即不傳季子,光當立。陰納賢士,欲以襲王僚。

八年,吳使公子光伐楚,敗楚師,迎楚故太子建母於居巢以歸。因北伐,敗陳、蔡之師。九年,公子光伐楚,拔居巢、鐘離。初,楚邊邑卑梁氏之處女與吳邊邑之女爭桑,二女家怒相滅,兩國邊邑長聞之,怒而相攻,滅吳之邊邑。吳王怒,故遂伐楚,取兩都而去。

伍子胥之初奔吳,說吳王僚以伐楚之利。公子光曰:「胥之父兄為僇於楚,欲自報其仇耳。未見其利。」於是伍員知光有他志,乃求勇士專諸,見之光。光喜,乃客伍子胥。子胥退而耕於野,以待專諸之事。

十二年冬,楚平王卒。十三年春,吳欲因楚喪而伐之,使公子蓋餘、燭庸以兵圍楚之六、𤅬。使季札於晉,以觀諸侯之變。楚發兵絕吳兵後,吳兵不得還。於是吳公子光曰:「此時不可失也。」告專諸曰:「不索何獲!我真王嗣,當立,吾欲求之。季子雖至,不吾廢也。」專諸曰:「王僚可殺也。母老子弱,而兩公子將兵攻楚,楚絕其路。方今吳外困於楚,而內空無骨鯁之臣,是無柰我何。」光曰:「我身,子之身也。」四月丙子,光伏甲士於窟室,而謁王僚飲。王僚使兵陳於道,自王宮至光之家,門階戶席,皆王僚之親也,人夾持鈹。公子光詳為足疾,入于窟室,使專諸置匕首於炙魚之中以進食。手匕首刺王僚,鈹交於匈,遂弒王僚。公子光竟代立為王,是為吳王闔廬。闔廬乃以專諸子為卿。

季子至,曰:「茍先君無廢祀,民人無廢主,社稷有奉,乃吾君也。吾敢誰怨乎?哀死事生,以待天命。非我生亂,立者從之,先人之道也。」復命,哭僚墓,復位而待。吳公子燭庸、蓋餘二人將兵遇圍於楚者,聞公子光弒王僚自立,乃以其兵降楚,楚封之於舒。

王闔廬元年,舉伍子胥為行人而與謀國事。楚誅伯州犁,其孫伯嚭亡奔吳,吳以為大夫。

三年,吳王闔廬與子胥、伯嚭將兵伐楚,拔舒,殺吳亡將二公子。光謀欲入郢,將軍孫武曰:「民勞,未可,待之。」四年,伐楚,取六與𤅬。五年,伐越,敗之。六年,楚使子常囊瓦伐吳。迎而擊之,大敗楚軍於豫章,取楚之居巢而還。

九年,吳王闔廬請伍子胥、孫武曰:「始子之言郢未可入,今果如何?」二子對曰:「楚將子常貪,而唐、蔡皆怨之。王必欲大伐,必得唐、蔡乃可。」闔廬從之,悉興師,與唐、蔡西伐楚,至於漢水。楚亦發兵拒吳,夾水陳。吳王闔廬弟夫欲戰,闔廬弗許。夫曰:「王已屬臣兵,兵以利為上,尚何待焉?」遂以其部五千人襲冒楚,楚兵大敗,走。於是吳王遂縱兵追之。比至郢,五戰,楚五敗。楚昭王亡出郢,奔鄖。鄖公弟欲弒昭王,昭王與鄖公奔隨。而吳兵遂入郢。子胥、伯嚭鞭平王之尸以報父讎。

十年春,越聞吳王之在郢,國空,乃伐吳。吳使別兵擊越。楚告急秦,秦遣兵救楚擊吳,吳師敗。闔廬弟夫見秦越交敗吳,吳王留楚不去,夫亡歸吳而自立為吳王。闔廬聞之,乃引兵歸,攻夫。夫敗奔楚。楚昭王乃得以九月復入郢,而封夫槩於堂谿,為堂谿氏。十一年,吳王使太子夫差伐楚,取番。楚恐而去郢徙鄀。

十五年,孔子相魯。

十九年夏,吳伐越,越王句踐迎擊之檇李。越使死士挑戰,三行造吳師,呼,自剄。吳師觀之,越因伐吳,敗之姑蘇,傷吳王闔廬指,軍卻七里。吳王病傷而死。闔廬使立太子夫差,謂曰:「爾而忘句踐殺汝父乎?」對曰:「不敢!」三年,乃報越。

王夫差元年,以大夫伯嚭為太宰。習戰射,常以報越為志。二年,吳王悉精兵以伐越,敗之夫椒,報姑蘇也。越王句踐乃以甲兵五千人棲於會稽,使大夫種因吳太宰嚭而行成,請委國為臣妾。吳王將許之,伍子胥諫曰:「昔有過氏殺斟灌以伐斟尋,滅夏后帝相。帝相之妃后緡方娠,逃於有仍而生少康。少康為有仍牧正。有過又欲殺少康,少康奔有虞。有虞思夏德,於是妻之以二女而邑之於綸,有田一成,有眾一旅。後遂收夏眾,撫其官職。使人誘之,遂滅有過氏,復禹之績,祀夏配天,不失舊物。今吳不如有過之彊,而句踐大於少康。今不因此而滅之,又將寬之,不亦難乎!且句踐為人能辛苦,今不滅,後必悔之。」吳王不聽,聽太宰嚭,卒許越平,與盟而罷兵去。

七年,吳王夫差聞齊景公死而大臣爭寵,新君弱,乃興師北伐齊。子胥諫曰:「越王句踐食不重味,衣不重采,弔死問疾,且欲有所用其眾。此人不死,必為吳患。今越在腹心疾而王不先,而務齊,不亦謬乎!」吳王不聽,遂北伐齊,敗齊師於艾陵。至繒,召魯哀公而徵百牢。季康子使子貢以周禮說太宰嚭,乃得止。因留略地於齊魯之南。九年,為騶伐魯,至與魯盟乃去。十年,因伐齊而歸。十一年,復北伐齊。

越王句踐率其眾以朝吳,厚獻遺之,吳王喜。唯子胥懼,曰:「是棄吳也。」諫曰:「越在腹心,今得志於齊,猶石田,無所用。且盤庚之誥有顛越勿遺,商之以興。」吳王不聽,使子胥於齊,子胥屬其子於齊鮑氏,還報吳王。吳王聞之,大怒,賜子胥屬鏤之劍以死。將死,曰:「樹吾墓上以梓,令可為器。抉吾眼置之吳東門,以觀越之滅吳也。」

齊鮑氏弒齊悼公。吳王聞之,哭於軍門外三日,乃從海上攻齊。齊人敗吳,吳王乃引兵歸。

十三年,吳召魯、衛之君會於橐皋。

十四年春,吳王北會諸侯於黃池,欲霸中國以全周室。六月(戊)[丙]子,越王句踐伐吳。乙酉,越五千人與吳戰。丙戌,虜吳太子友。丁亥,入吳。吳人告敗於王夫差,夫差惡其聞也。或泄其語,吳王怒,斬七人於幕下。七月辛丑,吳王與晉定公爭長。吳王曰:「於周室我為長。」晉定公曰:「於姬姓我為伯。」趙鞅怒,將伐吳,乃長晉定公。吳王已盟,與晉別,欲伐宋。太宰嚭曰:「可勝而不能居也。」乃引兵歸國。國亡太子,內空,王居外久,士皆罷敝,於是乃使厚幣以與越平。

十五年,齊田常殺簡公。

十八年,越益彊。越王句踐率兵(使)[復]伐敗吳師於笠澤。楚滅陳。

二十年,越王句踐復伐吳。二十一年,遂圍吳。二十三年十一月丁卯,越敗吳。越王句踐欲遷吳王夫差於甬東,予百家居之。吳王曰:「孤老矣,不能事君王也。吾悔不用子胥之言,自令陷此。」遂自剄死。越王滅吳,誅太宰嚭,以為不忠,而歸。

太史公曰:孔子言「太伯可謂至德矣,三以天下讓,民無得而稱焉」。余讀春秋古文,乃知中國之虞與荊蠻句吳兄弟也。延陵季子之仁心,慕義無窮,見微而知清濁。嗚呼,又何其閎覽博物君子也!


\end{pinyinscope}