\article{楚世家}

\begin{pinyinscope}
楚之先祖出自帝顓頊高陽。高陽者,黃帝之孫,昌意之子也。高陽生稱,稱生卷章,卷章生重黎。重黎為帝嚳高辛居火正,甚有功,能光融天下,帝嚳命曰祝融。共工氏作亂,帝嚳使重黎誅之而不盡。帝乃以庚寅日誅重黎,而以其弟吳回為重黎後,復居火正,為祝融。

吳回生陸終。陸終生子六人,坼剖而產焉。其長一曰昆吾;二曰參胡;三曰彭祖;四曰會人;五曰曹姓;六曰季連,羋姓,楚其後也。昆吾氏,夏之時嘗為侯伯,桀之時湯滅之。彭祖氏,殷之時嘗為侯伯,殷之末世滅彭祖氏。季連生附沮,附沮生穴熊。其後中微,或在中國,或在蠻夷,弗能紀其世。

周文王之時,季連之苗裔曰鬻熊。鬻熊子事文王,蚤卒。其子曰熊麗。熊麗生熊狂,熊狂生熊繹。熊繹當周成王之時,舉文、武勤勞之後嗣,而封熊繹於楚蠻,封以子男之田,姓羋氏,居丹陽。楚子熊繹與魯公伯禽、衛康叔子牟、晉侯燮、齊太公子呂伋俱事成王。

熊繹生熊艾,熊艾生熊亶,熊亶生熊勝。熊勝以弟熊楊為後。熊楊生熊渠。

熊渠生子三年。當周夷王之時,王室微,諸侯或不朝,相伐。熊渠甚得江漢閒民和,乃興兵伐庸、楊蠆,至于鄂。熊渠曰:「我蠻夷也,不與中國之號謚。」乃立其長子康為句亶王,中子紅為鄂王,少子執疵為越章王,皆在江上楚蠻之地。及周厲王之時,暴虐,熊渠畏其伐楚,亦去其王。

後為熊毋康,毋康蚤死。熊渠卒,子熊摯紅立。摯紅卒,其弟弒而代立,曰熊延。熊延生熊勇。

熊勇六年,而周人作亂,攻厲王,厲王出奔彘。熊勇十年,卒,弟熊嚴為後。

熊嚴十年,卒。有子四人,長子伯霜,中子仲雪,次子叔堪,少子季徇。熊嚴卒,長子伯霜代立,是為熊霜。

熊霜元年,周宣王初立。熊霜六年,卒,三弟爭立。仲雪死;叔堪亡,避難於濮;而少弟季徇立,是為熊徇。熊徇十六年,鄭桓公初封於鄭。二十二年,熊徇卒,子熊咢立。熊咢九年,卒,子熊儀立,是為若敖。

若敖二十年,周幽王為犬戎所弒,周東徙,而秦襄公始列為諸侯。

二十七年,若敖卒,子熊坎立,是為霄敖。霄敖六年,卒,子熊眴立,是為蚡冒。蚡冒十三年,晉始亂,以曲沃之故。蚡冒十七年,卒。蚡冒弟熊通弒蚡冒子而代立,是為楚武王。

武王十七年,晉之曲沃莊伯弒主國晉孝侯。十九年,鄭伯弟段作亂。二十一年,鄭侵天子之田。二十三年,衛弒其君桓公。二十九年,魯弒其君隱公。三十一年,宋太宰華督弒其君殤公。

三十五年,楚伐隨。隨曰:「我無罪。」楚曰:「我蠻夷也。今諸侯皆為叛相侵,或相殺。我有敝甲,欲以觀中國之政,請王室尊吾號。」隨人為之周,請尊楚,王室不聽,還報楚。三十七年,楚熊通怒曰:「吾先鬻熊,文王之師也,蚤終。成王舉我先公,乃以子男田令居楚,蠻夷皆率服,而王不加位,我自尊耳。」乃自立為武王,與隨人盟而去。於是始開濮地而有之。

五十一年,周召隨侯,數以立楚為王。楚怒,以隨背己,伐隨。武王卒師中而兵罷。子文王熊貲立,始都郢。

文王二年,伐申過鄧,鄧人曰「楚王易取」,鄧侯不許也。六年,伐蔡,虜蔡哀侯以歸,已而釋之。楚彊,陵江漢閒小國,小國皆畏之。十一年,齊桓公始霸,楚亦始大。

十二年,伐鄧,滅之。十三年,卒,子熊艱立,是為莊敖。莊敖五年,欲殺其弟熊惲,惲奔隨,與隨襲弒莊敖代立,是為成王。

成王惲元年,初即位,布德施惠,結舊好於諸侯。使人獻天子,天子賜胙,曰:「鎮爾南方夷越之亂,無侵中國。」於是楚地千里。

十六年,齊桓公以兵侵楚,至陘山。楚成王使將軍屈完以兵御之,與桓公盟。桓公數以周之賦不入王室,楚許之,乃去。

十八年,成王以兵北伐許,許君肉袒謝,乃釋之。二十二年,伐黃。二十六年,滅英。

三十三年,宋襄公欲為盟會,召楚。楚王怒曰:「召我,我將好往襲辱之。」遂行,至盂,遂執辱宋公,已而歸之。三十四年,鄭文公南朝楚。楚成王北伐宋,敗之泓,射傷宋襄公,襄公遂病創死。

三十五年,晉公子重耳過楚,成王以諸侯客禮饗,而厚送之於秦。

三十九年,魯僖公來請兵以伐齊,楚使申侯將兵伐齊,取穀,置齊桓公子雍焉。齊桓公七子皆奔楚,楚盡以為上大夫。滅夔,夔不祀祝融、鬻熊故也。

夏,伐宋,宋告急於晉,晉救宋,成王罷歸。將軍子玉請戰,成王曰:「重耳亡居外久,卒得反國,天之所開,不可當。」子玉固請,乃與之少師而去。晉果敗子玉於城濮。成王怒,誅子玉。

四十六年,初,成王將以商臣為太子,語令尹子上。子上曰:「君之齒未也,而又多內寵,絀乃亂也。楚國之舉常在少者。且商臣蜂目而豺聲,忍人也,不可立也。」王不聽,立之。後又欲立子職而絀太子商臣。商臣聞而未審也,告其傅潘崇曰:「何以得其實?」崇曰:「饗王之寵姬江羋而勿敬也。」商臣從之。江羋怒曰:「宜乎王之欲殺若而立職也。」商臣告潘崇曰:「信矣。」崇曰:「能事之乎?」曰:「不能。」「能亡去乎?」曰:「不能。」「能行大事乎?」曰:「能。」冬十月,商臣以宮衛兵圍成王。成王請食熊蹯而死,不聽。丁未,成王自絞殺。商臣代立,是為穆王。

穆王立,以其太子宮予潘崇,使為太師,掌國事。穆王三年,滅江。四年,滅六、蓼。六、蓼,皋陶之後。八年,伐陳。十二年,卒。子莊王侶立。

莊王即位三年,不出號令,日夜為樂,令國中曰:「有敢諫者死無赦!」伍舉入諫。莊王左抱鄭姬,右抱越女,坐鐘鼓之閒。伍舉曰:「願有進隱。」曰:「有鳥在於阜,三年不蜚不鳴,是何鳥也?」莊王曰:「三年不蜚,蜚將沖天;三年不鳴,鳴將驚人。舉退矣,吾知之矣。」居數月,淫益甚。大夫蘇從乃入諫。王曰:「若不聞令乎?」對曰:「殺身以明君,臣之願也。」於是乃罷淫樂,聽政,所誅者數百人,所進者數百人,任伍舉、蘇從以政,國人大說。是歲滅庸。六年,伐宋,獲五百乘。

八年,伐陸渾戎,遂至洛,觀兵於周郊。周定王使王孫滿勞楚王。楚王問鼎小大輕重,對曰:「在德不在鼎。」莊王曰:「子無阻九鼎!楚國折鉤之喙,足以為九鼎。」王孫滿曰:「嗚呼!君王其忘之乎?昔虞夏之盛,遠方皆至,貢金九牧,鑄鼎象物,百物而為之備,使民知神姦。桀有亂德,鼎遷於殷,載祀六百。殷紂暴虐,鼎遷於周。德之休明,雖小必重;其姦回昏亂,雖大必輕。昔成王定鼎于郟鄏,卜世三十,卜年七百,天所命也。周德雖衰,天命未改。鼎之輕重,未可問也。」楚王乃歸。

九年,相若敖氏。人或讒之王,恐誅,反攻王,王擊滅若敖氏之族。十三年,滅舒。

十六年,伐陳,殺夏徵舒。徵舒弒其君,故誅之也。已破陳,即縣之。群臣皆賀,申叔時使齊來,不賀。王問,對曰:「鄙語曰,牽牛徑人田,田主取其牛。徑者則不直矣,取之牛不亦甚乎?且王以陳之亂而率諸侯伐之,以義伐之而貪其縣,亦何以復令於天下!」莊王乃復國陳後。

十七年春,楚莊王圍鄭,三月克之。入自皇門,鄭伯肉袒牽羊以逆,曰:「孤不天,不能事君,君用懷怒,以及敝邑,孤之罪也。敢不惟命是聽!賓之南海,若以臣妾賜諸侯,亦惟命是聽。若君不忘厲、宣、桓、武,不絕其社稷,使改事君,孤之願也,非所敢望也。敢布腹心。」楚群臣曰:「王勿許。」莊王曰:「其君能下人,必能信用其民,庸可絕乎!」莊王自手旗,左右麾軍,引兵去三十里而舍,遂許之平。潘尪入盟,子良出質。夏六月,晉救鄭,與楚戰,大敗晉師河上,遂至衡雍而歸。

二十年,圍宋,以殺楚使也。圍宋五月,城中食盡,易子而食,析骨而炊。宋華元出告以情。莊王曰:「君子哉!」遂罷兵去。

二十三年,莊王卒,子共王審立。

共王十六年,晉伐鄭。鄭告急,共王救鄭。與晉兵戰鄢陵,晉敗楚,射中共王目。共王召將軍子反。子反嗜酒,從者豎陽穀進酒醉。王怒,射殺子反,遂罷兵歸。

三十一年,共王卒,子康王招立。康王立十五年卒,子員立,是為郟敖。

康王寵弟公子圍、子比、子皙、棄疾。郟敖三年,以其季父康王弟公子圍為令尹,主兵事。四年,圍使鄭,道聞王疾而還。十二月己酉,圍入問王疾,絞而弒之,遂殺其子莫及平夏。使使赴於鄭。伍舉問曰:「誰為後?」對曰:「寡大夫圍。」伍舉更曰:「共王之子圍為長。」子比奔晉,而圍立,是為靈王。

靈王三年六月,楚使使告晉,欲會諸侯。諸侯皆會楚于申。伍舉曰:「昔夏啟有鈞臺之饗,商湯有景亳之命,周武王有盟津之誓,成王有岐陽之閒,康王有豐宮之朝,穆王有涂山之會,齊桓有召陵之師,晉文有踐土之盟,君其何用?」靈王曰:「用桓公。」時鄭子產在焉。於是晉、宋、魯、衛不往。靈王已盟,有驕色。伍舉曰:「桀為有仍之會,有緡叛之。紂為黎山之會,東夷叛之。幽王為太室之盟,戎、翟叛之。君其慎終!」

七月,楚以諸侯兵伐吳,圍朱方。八月,克之,囚慶封,滅其族。以封徇,曰:「無效齊慶封弒其君而弱其孤,以盟諸大夫!」封反曰:「莫如楚共王庶子圍弒其君兄之子員而代之立!」於是靈王使(棄)疾殺之。

七年,就章華臺,下令內亡人實之。

八年,使公子棄疾將兵滅陳。十年,召蔡侯,醉而殺之。使棄疾定蔡,因為陳蔡公。

十一年,伐徐以恐吳。靈王次於乾豀以待之。王曰:「齊、晉、魯、衛,其封皆受寶器,我獨不。今吾使使周求鼎以為分,其予我乎?」析父對曰:「其予君王哉!昔我先王熊繹辟在荊山,蓽露藍蔞以處草莽,跋涉山林以事天子,唯是桃弧棘矢以共王事。齊,王舅也;晉及魯、衛,王母弟也:楚是以無分而彼皆有。周今與四國服事君王,將惟命是從,豈敢愛鼎?」靈王曰:「昔我皇祖伯父昆吾舊許是宅,今鄭人貪其田,不我予,今我求之,其予我乎?」對曰:「周不愛鼎,鄭安敢愛田?」靈王曰:「昔諸侯遠我而畏晉,今吾大城陳、蔡、不羹,賦皆千乘,諸侯畏我乎?」對曰:「畏哉!」靈王喜曰:「析父善言古事焉。」

十二年春,楚靈王樂乾豀,不能去也。國人苦役。初,靈王會兵於申,僇越大夫常壽過,殺蔡大夫觀起。起子從亡在吳,乃勸吳王伐楚,為閒越大夫常壽過而作亂,為吳閒。使矯公子棄疾命召公子比於晉,至蔡,與吳、越兵欲襲蔡。令公子比見棄疾,與盟於鄧。遂入殺靈王太子祿,立子比為王,公子子皙為令尹,棄疾為司馬。先除王宮,觀從從師于乾豀,令楚眾曰:「國有王矣。先歸,復爵邑田室。後者遷之。」楚眾皆潰,去靈王而歸。

靈王聞太子祿之死也,自投車下,而曰:「人之愛子亦如是乎?」侍者曰:「甚是。」王曰:「余殺人之子多矣,能無及此乎?」右尹曰:「請待於郊以聽國人。」王曰:「眾怒不可犯。」曰:「且入大縣而乞師於諸侯。」王曰:「皆叛矣。」又曰:「且奔諸侯以聽大國之慮。」王曰:「大福不再,祗取辱耳。」於是王乘舟將欲入鄢。右尹度王不用其計,懼俱死,亦去王亡。

靈王於是獨傍偟山中,野人莫敢入王。王行遇其故鋗人,謂曰:「為我求食,我已不食三日矣。」鋗人曰:「新王下法,有敢閒王從王者,罪及三族,且又無所得食。」王因枕其股而臥。鋗人又以土自代,逃去。王覺而弗見,遂饑弗能起。芋尹申無宇之子申亥曰:「吾父再犯王命,王弗誅,恩孰大焉!」乃求王,遇王饑於釐澤,奉之以歸。夏五月癸丑,王死申亥家,申亥以二女從死,并葬之。

是時楚國雖已立比為王,畏靈王復來,又不聞靈王死,故觀從謂初王比曰:「不殺棄疾,雖得國猶受禍。」王曰:「余不忍。」從曰:「人將忍王。」王不聽,乃去。棄疾歸。國人每夜驚,曰:「靈王入矣!」乙卯夜,棄疾使船人從江上走呼曰:「靈王至矣!」國人愈驚。又使曼成然告初王比及令尹子皙曰:「王至矣!國人將殺君,司馬將至矣!君蚤自圖,無取辱焉。眾怒如水火,不可救也。」初王及子皙遂自殺。丙辰,棄疾即位為王,改名熊居,是為平王。

平王以詐弒兩王而自立,恐國人及諸侯叛之,乃施惠百姓。復陳蔡之地而立其後如故,歸鄭之侵地。存恤國中,修政教。吳以楚亂故,獲五率以歸。平王謂觀從:「恣爾所欲。」欲為卜尹,王許之。

初,共王有寵子五人,無適立,乃望祭群神,請神決之,使主社稷,而陰與巴姬埋璧於室內,召五公子齋而入。康王跨之,靈王肘加之,子比、子皙皆遠之。平王幼,抱其上而拜,壓紐。故康王以長立,至其子失之;圍為靈王,及身而弒;子比為王十餘日,子皙不得立,又俱誅。四子皆絕無後。唯獨棄疾後立,為平王,竟續楚祀,如其神符。

初,子比自晉歸,韓宣子問叔向曰:「子比其濟乎?」對曰:「不就。」宣子曰:「同惡相求,如市賈焉,何為不就?」對曰:「無與同好,誰與同惡?取國有五難:有寵無人,一也;有人無主,二也;有主無謀,三也;有謀而無民,四也;有民而無德,五也。子比在晉十三年矣,晉、楚之從不聞通者,可謂無人矣;族盡親叛,可謂無主矣;無釁而動,可謂無謀矣;為羈終世,可謂無民矣;亡無愛徵,可謂無德矣。王虐而不忌,子比涉五難以弒君,誰能濟之!有楚國者,其棄疾乎?君陳、蔡,方城外屬焉。苛慝不作,盜賊伏隱,私欲不違,民無怨心。先神命之,國民信之。羋姓有亂,必季實立,楚之常也。子比之官,則右尹也;數其貴寵,則庶子也;以神所命,則又遠之;民無懷焉,將何以立?」宣子曰:「齊桓、晉文不亦是乎?」對曰:「齊桓,衛姬之子也,有寵於釐公。有鮑叔牙、賓須無、隰朋以為輔,有莒、衛以為外主,有高、國以為內主。從善如流,施惠不倦。有國,不亦宜乎?昔我文公,狐季姬之子也,有寵於獻公。好學不倦。生十七年,有士五人,有先大夫子餘、子犯以為腹心,有魏犫、賈佗以為股肱,有齊、宋、秦、楚以為外主,有欒、郤、狐、先以為內主。亡十九年,守志彌篤。惠、懷棄民,民從而與之。故文公有國,不亦宜乎?子比無施於民,無援於外,去晉,晉不送;歸楚,楚不迎。何以有國!」子比果不終焉,卒立者棄疾,如叔向言也。

平王二年,使費無忌如秦為太子建取婦。婦好,來,未至,無忌先歸,說平王曰:「秦女好,可自娶,為太子更求。」平王聽之,卒自娶秦女,生熊珍。更為太子娶。是時伍奢為太子太傅,無忌為少傅。無忌無寵於太子,常讒惡太子建。建時年十五矣,其母蔡女也,無寵於王,王稍益疏外建也。

六年,使太子建居城父,守邊。無忌又日夜讒太子建於王曰:「自無忌入秦女,太子怨,亦不能無望於王,王少自備焉。且太子居城父,擅兵,外交諸侯,且欲入矣。」平王召其傅伍奢責之。伍奢知無忌讒,乃曰:「王柰何以小臣疏骨肉?」無忌曰:「今不制,後悔也。」於是王遂囚伍奢。[而召其二子而告以免父死]乃令司馬奮揚召太子建,欲誅之。太子聞之,亡奔宋。

無忌曰:「伍奢有二子,不殺者為楚國患。盍以免其父召之,必至。」於是王使使謂奢:「能致二子則生,不能將死。」奢曰:「尚至,胥不至。」王曰:「何也?」奢曰:「尚之為人,廉,死節,慈孝而仁,聞召而免父,必至,不顧其死。胥之為人,智而好謀,勇而矜功,知來必死,必不來。然為楚國憂者必此子。」於是王使人召之,曰:「來,吾免爾父。」伍尚謂伍胥曰:「聞父免而莫奔,不孝也;父戮莫報,無謀也;度能任事,知也。子其行矣,我其歸死。」伍尚遂歸。伍胥彎弓屬矢,出見使者,曰:「父有罪,何以召其子為?」將射,使者還走,遂出奔吳。伍奢聞之,曰:「胥亡,楚國危哉。」楚人遂殺伍奢及尚。

十年,楚太子建母在居巢,開吳。吳使公子光伐楚,遂敗陳、蔡,取太子建母而去。楚恐,城郢。初,吳之邊邑卑梁與楚邊邑鐘離小童爭桑,兩家交怒相攻,滅卑梁人。卑梁大夫怒,發邑兵攻鐘離。楚王聞之怒,發國兵滅卑梁。吳王聞之大怒,亦發兵,使公子光因建母家攻楚,遂滅鐘離、居巢。楚乃恐而城郢。

十三年,平王卒。將軍子常曰:「太子珍少,且其母乃前太子建所當娶也。」欲立令尹子西。子西,平王之庶弟也,有義。子西曰:「國有常法,更立則亂,言之則致誅。」乃立太子珍,是為昭王。

昭王元年,楚眾不說費無忌,以其讒亡太子建,殺伍奢子父與郤宛。宛之宗姓伯氏子及子胥皆奔吳,吳兵數侵楚,楚人怨無忌甚。楚令尹子常誅無忌以說眾,眾乃喜。

四年,吳三公子奔楚,楚封之以捍吳。五年,吳伐取楚之六、潛。七年,楚使子常伐吳,吳大敗楚於豫章。

十年冬,吳王闔閭、伍子胥、伯與唐、蔡俱伐楚,楚大敗,吳兵遂入郢,辱平王之墓,以伍子胥故也。吳兵之來,楚使子常以兵迎之,夾漢水陣。吳伐敗子常,子常亡奔鄭。楚兵走,吳乘勝逐之,五戰及郢。己卯,昭王出奔。庚辰,吳人入郢。

昭王亡也至雲夢。雲夢不知其王也,射傷王。王走鄖。鄖公之弟懷曰:「平王殺吾父,今我殺其子,不亦可乎?」鄖公止之,然恐其弒昭王,乃與王出奔隨。吳王聞昭王往,即進擊隨,謂隨人曰:「周之子孫封於江漢之閒者,楚盡滅之。」欲殺昭王。王從臣子綦乃深匿王,自以為王,謂隨人曰:「以我予吳。」隨人卜予吳,不吉,乃謝吳王曰:「昭王亡,不在隨。」吳請入自索之,隨不聽,吳亦罷去。

昭王之出郢也,使申鮑胥請救於秦。秦以車五百乘救楚,楚亦收餘散兵,與秦擊吳。十一年六月,敗吳於稷。會吳王弟夫概見吳王兵傷敗,乃亡歸,自立為王。闔閭聞之,引兵去楚,歸擊夫概。夫概敗,奔楚,楚封之堂谿,號為堂谿氏。

楚昭王滅唐九月,歸入郢。十二年,吳復伐楚,取番。楚恐,去郢,北徙都鄀。

十六年,孔子相魯。二十年,楚滅頓,滅胡。二十一年,吳王闔閭伐越。越王句踐射傷吳王,遂死。吳由此怨越而不西伐楚。

二十七年春,吳伐陳,楚昭王救之,軍城父。十月,昭王病於軍中,有赤雲如鳥,夾日而蜚。昭王問周太史,太史曰:「是害於楚王,然可移於將相。」將相聞是言,乃請自以身禱於神。昭王曰:「將相,孤之股肱也,今移禍,庸去是身乎!」弗聽。卜而河為祟,大夫請禱河。昭王曰:「自吾先王受封,望不過江、漢,而河非所獲罪也。」止不許。孔子在陳,聞是言,曰:「楚昭王通大道矣。其不失國,宜哉!」

昭王病甚,乃召諸公子大夫曰:「孤不佞,再辱楚國之師,今乃得以天壽終,孤之幸也。」讓其弟公子申為王,不可。又讓次弟公子結,亦不可。乃又讓次弟公子閭,五讓,乃後許為王。將戰,庚寅,昭王卒於軍中。子閭曰:「王病甚,捨其子讓群臣,臣所以許王,以廣王意也。今君王卒,臣豈敢忘君王之意乎!」乃與子西、子綦謀,伏師閉涂,迎越女之子章立之,是為惠王。然後罷兵歸,葬昭王。

惠王二年,子西召故平王太子建之子勝於吳,以為巢大夫,號曰白公。白公好兵而下士,欲報仇。六年,白公請兵令尹子西伐鄭。初,白公父建亡在鄭,鄭殺之,白公亡走吳,子西復召之,故以此怨鄭,欲伐之。子西許而未為發兵。八年,晉伐鄭,鄭告急楚,楚使子西救鄭,受賂而去。白公勝怒,乃遂與勇力死士石乞等襲殺令尹子西、子綦於朝,因劫惠王,置之高府,欲弒之。惠王從者屈固負王亡走昭王夫人宮。白公自立為王。月餘,會葉公來救楚,楚惠王之徒與共攻白公,殺之。惠王乃復位。是歲也,滅陳而縣之。

十三年,吳王夫差彊,陵齊、晉,來伐楚。十六年,越滅吳。四十二年,楚滅蔡。四十四年,楚滅杞。與秦平。是時越已滅吳而不能正江、淮北;楚東侵,廣地至泗上。

五十七年,惠王卒,子簡王中立。

簡王元年,北伐滅莒。八年,魏文侯、韓武子、趙桓子始列為諸侯。

二十四年,簡王卒,子聲王當立。聲王六年,盜殺聲王,子悼王熊疑立。悼王二年,三晉來伐楚,至乘丘而還。四年,楚伐周。鄭殺子陽。九年,伐韓,取負黍。十一年,三晉伐楚,敗我大梁、榆關。楚厚賂秦,與之平。二十一年,悼王卒,子肅王臧立。

肅王四年,蜀伐楚,取茲方。於是楚為捍關以距之。十年,魏取我魯陽。十一年,肅王卒,無子,立其弟熊良夫,是為宣王。

宣王六年,周天子賀秦獻公。秦始復彊,而三晉益大,魏惠王、齊威王尤彊。三十年,秦封衛鞅於商,南侵楚。是年,宣王卒,子威王熊商立。

威王六年,周顯王致文武胙於秦惠王。

七年,齊孟嘗君父田嬰欺楚,楚威王伐齊,敗之於徐州,而令齊必逐田嬰。田嬰恐,張丑偽謂楚王曰:「王所以戰勝於徐州者,田盼子不用也。盼子者,有功於國,而百姓為之用。嬰子弗善而用申紀。申紀者,大臣不附,百姓不為用,故王勝之也。今王逐嬰子,嬰子逐,盼子必用矣。復搏其士卒以與王遇,必不便於王矣。」楚王因弗逐也。

十一年,威王卒,子懷王熊槐立。魏聞楚喪,伐楚,取我陘山。

懷王元年,張儀始相秦惠王。四年,秦惠王初稱王。

六年,楚使柱國昭陽將兵而攻魏,破之於襄陵,得八邑。又移兵而攻齊,齊王患之。陳軫適為秦使齊,齊王曰:「為之柰何?」陳軫曰:「王勿憂,請令罷之。」即往見昭陽軍中,曰:「願聞楚國之法,破軍殺將者何以貴之?」昭陽曰:「其官為上柱國,封上爵執珪。」陳軫曰:「其有貴於此者乎?」昭陽曰:「令尹。」陳軫曰:「今君已為令尹矣,此國冠之上。臣請得譬之。人有遺其舍人一卮酒者,舍人相謂曰:『數人飲此,不足以遍,請遂畫地為蛇,蛇先成者獨飲之。』一人曰:『吾蛇先成。』舉酒而起,曰:『吾能為之足。』及其為之足,而後成人奪之酒而飲之,曰:『蛇固無足,今為之足,是非蛇也。』今君相楚而攻魏,破軍殺將,功莫大焉,冠之上不可以加矣。今又移兵而攻齊,攻齊勝之,官爵不加於此;攻之不勝,身死爵奪,有毀於楚:此為蛇為足之說也。不若引兵而去以德齊,此持滿之術也。」昭陽曰:「善。」引兵而去。

燕、韓君初稱王。秦使張儀與楚、齊、魏相會,盟齧桑。

十一年,蘇秦約從山東六國共攻秦,楚懷王為從長。至函谷關,秦出兵擊六國,六國兵皆引而歸,齊獨後。十二年,齊湣王伐敗趙、魏軍,秦亦伐敗韓,與齊爭長。

十六年,秦欲伐齊,而楚與齊從親,秦惠王患之,乃宣言張儀免相,使張儀南見楚王,謂楚王曰:「敝邑之王所甚說者無先大王,雖儀之所甚願為門闌之廝者亦無先大王。敝邑之王所甚憎者無先齊王,雖儀之所甚憎者亦無先齊王。而大王和之,是以敝邑之王不得事王,而令儀亦不得為門闌之廝也。王為儀閉關而絕齊,今使使者從儀西取笔秦所分楚商於之地方六百里,如是則齊弱矣。是北弱齊,西德於秦,私商於以為富,此一計而三利俱至也。」懷王大悅,乃置相璽於張儀,日與置酒,宣言「吾復得吾商於之地」。群臣皆賀,而陳軫獨弔。懷王曰:「何故?」陳軫對曰:「秦之所為重王者,以王之有齊也。今地未可得而齊交先絕,是楚孤也。夫秦又何重孤國哉,必輕楚矣。且先出地而後絕齊,則秦計不為。先絕齊而後責地,則必見欺於張儀。見欺於張儀,則王必怨之。怨之,是西起秦患,北絕齊交。西起秦患,北絕齊交,則兩國之兵必至。臣故弔。」楚王弗聽,因使一將軍西受封地。

張儀至秦,詳醉墜車,稱病不出三月,地不可得。楚王曰:「儀以吾絕齊為尚薄邪?」乃使勇士宋遺北辱齊王。齊王大怒,折楚符而合於秦。秦齊交合,張儀乃起朝,謂楚將軍曰:「子何不受地?從某至某,廣袤六里。」楚將軍曰:「臣之所以見命者六百里,不聞六里。」即以歸報懷王。懷王大怒,興師將伐秦。陳軫又曰:「伐秦非計也。不如因賂之一名都,與之伐齊,是我亡於秦,取償於齊也,吾國尚可全。今王已絕於齊而責欺於秦,是吾合秦齊之交而來天下之兵也,國必大傷矣。」楚王不聽,遂絕和於秦,發兵西攻秦。秦亦發兵擊之。

十七年春,與秦戰丹陽,秦大敗我軍,斬甲士八萬,虜我大將軍屈丐、裨將軍逢侯丑等七十餘人,遂取漢中之郡。楚懷王大怒,乃悉國兵復襲秦,戰於藍田,大敗楚軍。韓、魏聞楚之困,乃南襲楚,至於鄧。楚聞,乃引兵歸。

十八年,秦使使約復與楚親,分漢中之半以和楚。楚王曰:「願得張儀,不願得地。」張儀聞之,請之楚。秦王曰:「楚且甘心於子,柰何?」張儀曰:「臣善其左右靳尚,靳尚又能得事於楚王幸姬鄭袖,袖所言無不從者。且儀以前使負楚以商於之約,今秦楚大戰,有惡,臣非面自謝楚不解。且大王在,楚不宜敢取儀。誠殺儀以便國,臣之願也。」儀遂使楚。

至,懷王不見,因而囚張儀,欲殺之。儀私於靳尚,靳尚為請懷王曰:「拘張儀,秦王必怒。天下見楚無秦,必輕王矣。」又謂夫人鄭袖曰:「秦王甚愛張儀,而王欲殺之,今將以上庸之地六縣賂楚,以美人聘楚王,以宮中善歌者為之媵。楚王重地,秦女必貴,而夫人必斥矣。夫人不若言而出之。」鄭袖卒言張儀於王而出之。儀出,懷王因善遇儀,儀因說楚王以叛從約而與秦合親,約婚姻。張儀已去,屈原使從齊來,諫王曰:「何不誅張儀?」懷王悔,使人追儀,弗及。是歲,秦惠王卒。

二十[六]年,齊湣王欲為從長,惡楚之與秦合,乃使使遺楚王書曰:「寡人患楚之不察於尊名也。今秦惠王死,武王立,張儀走魏,樗裏疾、公孫衍用,而楚事秦。夫樗裏疾善乎韓,而公孫衍善乎魏;楚必事秦,韓、魏恐,必因二人求合於秦,則燕、趙亦宜事秦。四國爭事秦,則楚為郡縣矣。王何不與寡人并力收韓、魏、燕、趙,與為從而尊周室,以案兵息民,令於天下?莫敢不樂聽,則王名成矣。王率諸侯并伐,破秦必矣。王取武關、蜀、漢之地,私吳、越之富而擅江海之利,韓、魏割上黨,西薄函谷,則楚之彊百萬也。且王欺於張儀,亡地漢中,兵銼藍田,天下莫不代王懷怒。今乃欲先事秦!願大王孰計之。」

楚王業已欲和於秦,見齊王書,猶豫不決,下其議群臣。群臣或言和秦,或曰聽齊。昭雎曰:「王雖東取地於越,不足以刷恥;必且取地於秦,而後足以刷恥於諸侯。王不如深善齊、韓以重樗裏疾,如是則王得韓、齊之重以求地矣。秦破韓宜陽,而韓猶復事秦者,以先王墓在平陽,而秦之武遂去之七十里,以故尤畏秦。不然,秦攻三川,趙攻上黨,楚攻河外,韓必亡。楚之救韓,不能使韓不亡,然存韓者楚也。韓已得武遂於秦,以河山為塞,所報德莫如楚厚,臣以為其事王必疾。齊之所信於韓者,以韓公子眛為齊相也。韓已得武遂於秦,王甚善之,使之以齊、韓重樗裏疾,疾得齊、韓之重,其主弗敢棄疾也。今又益之以楚之重,樗里子必言秦,復與楚之侵地矣。」於是懷王許之,竟不合秦,而合齊以善韓。

二十四年,倍齊而合秦。秦昭王初立,乃厚賂於楚。楚往迎婦。二十五年,懷王入與秦昭王盟,約於黃棘。秦復與楚上庸。二十六年,齊、韓、魏為楚負其從親而合於秦,三國共伐楚。楚使太子入質於秦而請救。秦乃遣客卿通將兵救楚,三國引兵去。

二十七年,秦大夫有私與楚太子鬬,楚太子殺之而亡歸。二十八年,秦乃與齊、韓、魏共攻楚,殺楚將唐眛,取我重丘而去。二十九年,秦復攻楚,大破楚,楚軍死者二萬,殺我將軍景缺。懷王恐,乃使太子為質於齊以求平。三十年,秦復伐楚,取八城。秦昭王遺楚王書曰:「始寡人與王約為弟兄,盟于黃棘,太子為質,至驩也。太子陵殺寡人之重臣,不謝而亡去,寡人誠不勝怒,使兵侵君王之邊。今聞君王乃令太子質於齊以求平。寡人與楚接境壤界,故為婚姻,所從相親久矣。而今秦楚不驩,則無以令諸侯。寡人願與君王會武關,面相約,結盟而去,寡人之願也。敢以聞下執事。」楚懷王見秦王書,患之。欲往,恐見欺;無往,恐秦怒。昭雎曰:「王毋行,而發兵自守耳。秦虎狼,不可信,有并諸侯之心。」懷王子子蘭勸王行,曰:「柰何絕秦之驩心!」於是往會秦昭王。昭王詐令一將軍伏兵武關,號為秦王。楚王至,則閉武關,遂與西至咸陽,朝章臺,如蕃臣,不與亢禮。楚懷王大怒,悔不用昭子言。秦因留楚王,要以割巫、黔中之郡。楚王欲盟,秦欲先得地。楚王怒曰:「秦詐我而又彊要我以地!」不復許秦。秦因留之。

楚大臣患之,乃相與謀曰:「吾王在秦不得還,要以割地,而太子為質於齊,齊、秦合謀,則楚無國矣。」乃欲立懷王子在國者。昭雎曰:「王與太子俱困於諸侯,而今又倍王命而立其庶子,不宜。」乃詐赴於齊,齊湣王謂其相曰:「不若留太子以求楚之淮北。」相曰:「不可,郢中立王,是吾抱空質而行不義於天下也。」或曰:「不然。郢中立王,因與其新王市曰『予我下東國,吾為王殺太子,不然,將與三國共立之』,然則東國必可得矣。」齊王卒用其相計而歸楚太子。太子橫至,立為王,是為頃襄王。乃告于秦曰:「賴社稷神靈,國有王矣。」

頃襄王橫元年,秦要懷王不可得地,楚立王以應秦,秦昭王怒,發兵出武關攻楚,大敗楚軍,斬首五萬,取析十五城而去。二年,楚懷王亡逃歸,秦覺之,遮楚道,懷王恐,乃從閒道走趙以求歸。趙主父在代,其子惠王初立,行王事,恐,不敢入楚王。楚王欲走魏,秦追至,遂與秦使復之秦。懷王遂發病。頃襄王三年,懷王卒于秦,秦歸其喪于楚。楚人皆憐之,如悲親戚。諸侯由是不直秦。秦楚絕。

六年,秦使白起伐韓於伊闕,大勝,斬首二十四萬。秦乃遺楚王書曰:「楚倍秦,秦且率諸侯伐楚,爭一旦之命。願王之飭士卒,得一樂戰。」楚頃襄王患之,乃謀復與秦平。七年,楚迎婦於秦,秦楚復平。

十一年,齊秦各自稱為帝;月餘,復歸帝為王。

十四年,楚頃襄王與秦昭王好會于宛,結和親。十五年,楚王與秦、三晉、燕共伐齊,取淮北。十六年,與秦昭王好會於鄢。其秋,復與秦王會穰。

十八年,楚人有好以弱弓微繳加歸鴈之上者,頃襄王聞,召而問之。對曰:「小臣之好射鶀鴈,羅鸗,小矢之發也,何足為大王道也。且稱楚之大,因大王之賢,所弋非直此也。昔者三王以弋道德,五霸以弋戰國。故秦、魏、燕、趙者,鶀鴈也;齊、魯、韓、衛者,青首也;騶、費、郯、邳者,羅鸗也。外其餘則不足射者。見鳥六雙,以王何取?王何不以聖人為弓,以勇士為繳,時張而射之?此六雙者,可得而囊載也。其樂非特朝昔之樂也,其獲非特鳧鴈之實也。王朝張弓而射魏之大梁之南,加其右臂而徑屬之於韓,則中國之路絕而上蔡之郡壞矣。還射圉之東,解魏左肘而外擊定陶,則魏之東外棄而大宋、方與二郡者舉矣。且魏斷二臂,顛越矣;膺擊郯國,大梁可得而有也。王綪繳蘭臺,飲馬西河,定魏大梁,此一發之樂也。若王之於弋誠好而不厭,則出寶弓,碆新繳,射噣鳥於東海,還蓋長城以為防,朝射東莒,夕發浿丘,夜加即墨,顧據午道,則長城之東收而太山之北舉矣。西結境於趙而北達於燕,三國布鶴,則從不待約而可成也。北遊目於燕之遼東而南登望於越之會稽,此再發之樂也。若夫泗上十二諸侯,左縈而右拂之,可一旦而盡也。今秦破韓以為長憂,得列城而不敢守也;伐魏而無功,擊趙而顧病,則秦魏之勇力屈矣,楚之故地漢中、析、酈可得而復有也。王出寶弓,碆新繳,涉鄳塞,而待秦之倦也,山東、河內可得而一也。勞民休眾,南面稱王矣。故曰秦為大鳥,負海內而處,東面而立,左臂據趙之西南,右臂傅楚鄢郢,膺擊韓魏,垂頭中國,處既形便,勢有地利,奮翼鼓鶴,方三千里,則秦未可得獨招而夜射也。」欲以激怒襄王,故對以此言。襄王因召與語,遂言曰:「夫先王為秦所欺而客死於外,怨莫大焉。今以匹夫有怨,尚有報萬乘,白公、子胥是也。今楚之地方五千里,帶甲百萬,猶足以踴躍中野也,而坐受困,臣竊為大王弗取也。」於是頃襄王遣使於諸侯,復為從,欲以伐秦。秦聞之,發兵來伐楚。

楚欲與齊韓連和伐秦,因欲圖周。周王赧使武公謂楚相昭子曰:「三國以兵割周郊地以便輸,而南器以尊楚,臣以為不然。夫弒共主,臣世君,大國不親;以眾脅寡,小國不附。大國不親,小國不附,不可以致名實。名實不得,不足以傷民。夫有圖周之聲,非所以為號也。」昭子曰:「乃圖周則無之。雖然,周何故不可圖也?」對曰:「軍不五不攻,城不十不圍。夫一周為二十晉,公之所知也。韓嘗以二十萬之眾辱於晉之城下,銳士死,中士傷,而晉不拔。公之無百韓以圖周,此天下之所知也。夫怨結兩周以塞騶魯之心,交絕於齊,聲失天下,其為事危矣。夫危兩周以厚三川,方城之外必為韓弱矣。何以知其然也?西周之地,絕長補短,不過百里。名為天下共主,裂其地不足以肥國,得其眾不足以勁兵。雖無攻之,名為弒君。然而好事之君,喜攻之臣,發號用兵,未嘗不以周為終始。是何也?見祭器在焉,欲器之至而忘弒君之亂。今韓以器之在楚,臣恐天下以器讎楚也。臣請譬之。夫虎肉臊,其兵利身,人猶攻之也。若使澤中之麋蒙虎之皮,人之攻之必萬於虎矣。裂楚之地,足以肥國;詘楚之名,足以尊主。今子將以欲誅殘天下之共主,居三代之傳器,吞三翮六翼,以高世主,非貪而何?《周書》曰『欲起無先』,故器南則兵至矣。」於是楚計輟不行。

十九年,秦伐楚,楚軍敗,割上庸、漢北地予秦。二十年,秦將白起拔我西陵。二十一年,秦將白起遂拔我郢,燒先王墓夷陵。楚襄王兵散,遂不復戰,東北保於陳城。二十二年,秦復拔我巫、黔中郡。

二十三年,襄王乃收東地兵,得十餘萬,復西取秦所拔我江旁十五邑以為郡,距秦。二十七年,使三萬人助三晉伐燕。復與秦平,而入太子為質於秦。楚使左徒侍太子於秦。

三十六年,頃襄王病,太子亡歸。秋,頃襄王卒,太子熊元代立,是為考烈王。考烈王以左徒為令尹,封以吳,號春申君。

考烈王元年,納州于秦以平。是時楚益弱。

六年,秦圍邯鄲,趙告急楚,楚遣將軍景陽救趙。七年,至新中。秦兵去。十二年,秦昭王卒,楚王使春申君弔祠于秦。十六年,秦莊襄王卒,秦王趙政立。二十二年,與諸侯共伐秦,不利而去。楚東徙都壽春,命曰郢。

二十五年,考烈王卒,子幽王悍立。李園殺春申君。幽王三年,秦、魏伐楚。秦相呂不韋卒。九年,秦滅韓。十年,幽王卒,同母弟猶代立,是為哀王。哀王立二月餘,哀王庶兄負芻之徒襲殺哀王而立負芻為王。是歲,秦虜趙王遷。

王負芻元年,燕太子丹使荊軻刺秦王。二年,秦使將軍伐楚,大破楚軍,亡十餘城。三年,秦滅魏。四年,秦將王翦破我軍於蘄,而殺將軍項燕。

五年,秦將王翦、蒙武遂破楚國,虜楚王負芻,滅楚名為[楚]郡云。

太史公曰:楚靈王方會諸侯於申,誅齊慶封,作章華臺,求周九鼎之時,志小天下;及餓死于申亥之家,為天下笑。操行之不得,悲夫!勢之於人也,可不慎與?棄疾以亂立,嬖淫秦女,甚乎哉,幾再亡國!


\end{pinyinscope}