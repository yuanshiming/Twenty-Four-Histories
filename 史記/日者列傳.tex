\article{日者列傳}

\begin{pinyinscope}
自古受命而王,王者之興何嘗不以卜筮決於天命哉!其於周尤甚,及秦可見。代王之入,任於卜者。太卜之起,由漢興而有。

司馬季主者,楚人也。卜於長安東市。

宋忠為中大夫,賈誼為博士,同日俱出洗沐,相從論議,誦易先王聖人之道術,究遍人情,相視而嘆。賈誼曰:「吾聞古之聖人,不居朝廷,必在卜醫之中。今吾已見三公九卿朝士大夫,皆可知矣。試之卜數中以觀采。」二人即同輿而之市,游於卜肆中。天新雨,道少人,司馬季主閒坐,弟子三四人侍,方辯天地之道,日月之運,陰陽吉凶之本。二大夫再拜謁。司馬季主視其狀貌,如類有知者,即禮之,使弟子延之坐。坐定,司馬季主復理前語,分別天地之終始,日月星辰之紀,差次仁義之際,列吉凶之符,語數千言,莫不順理。

宋忠、賈誼瞿然而悟,獵纓正襟危坐,曰:「吾望先生之狀,聽先生之辭,小子竊觀於世,未嘗見也。今何居之卑,何行之汙?」

司馬季主捧腹大笑曰:「觀大夫類有道術者,今何言之陋也,何辭之野也!今夫子所賢者何也?所高者誰也?今何以卑汙長者?」

二君曰:「尊官厚祿,世之所高也,賢才處之。今所處非其地,故謂之卑。言不信,行不驗,取不當,故謂之汙。夫卜筮者,世俗之所賤簡也。世皆言曰:『夫卜者多言誇嚴以得人情,虛高人祿命以說人志,擅言禍災以傷人心,矯言鬼神以盡人財,厚求拜謝以私於己。』此吾之所恥,故謂之卑汙也。」

司馬季主曰:「公且安坐。公見夫被髪童子乎?日月照之則行,不照則止,問之日月疵瑕吉凶,則不能理。由是觀之,能知別賢與不肖者寡矣。

「賢之行也,直道以正諫,三諫不聽則退。其譽人也不望其報,惡人也不顧其怨,以便國家利眾為務。故官非其任不處也,祿非其功不受也;見人不正,雖貴不敬也;見人有污,雖尊不下也;得不為喜,去不為恨;非其罪也,雖累辱而不愧也。

「今公所謂賢者,皆可為羞矣。卑疵而前,孅趨而言;相引以勢,相導以利;比周賓正,以求尊譽,以受公奉;事私利,枉主法,獵農民;以官為威,以法為機,求利逆暴:譬無異於操白刃劫人者也。初試官時,倍力為巧詐,飾虛功執空文以罔主上,用居上為右;試官不讓賢陳功,見偽增實,以無為有,以少為多,以求便勢尊位;食飲驅馳,從姬歌兒,不顧於親,犯法害民,虛公家:此夫為盜不操矛弧者也,攻而不用弦刃者也,欺父母未有罪而弒君未伐者也。何以為高賢才乎?

「盜賊發不能禁,夷貊不服不能攝,姦邪起不能塞,官秏亂不能治,四時不和不能調,歲穀不孰不能適。才賢不為,是不忠也;才不賢而託官位,利上奉,妨賢者處,是竊位也;有人者進,有財者禮,是偽也。子獨不見鴟梟之與鳳皇翔乎?蘭芷芎藭棄於廣野,蒿蕭成林,使君子退而不顯眾,公等是也。

「述而不作,君子義也。今夫卜者,必法天地,象四時,順於仁義,分策定卦,旋式正棋,然後言天地之利害,事之成敗。昔先王之定國家,必先龜策日月,而後乃敢代;正時日,乃后入家;產子必先占吉凶,后乃有之。自伏羲作八卦,周文王演三百八十四爻而天下治。越王句踐放文王八卦以破敵國,霸天下。由是言之,卜筮有何負哉!

「且夫卜筮者,埽除設坐,正其冠帶,然後乃言事,此有禮也。言而鬼神或以饗,忠臣以事其上,孝子以養其親,慈父以畜其子,此有德者也。而以義置數十百錢,病者或以愈,且死或以生,患或以免,事或以成,嫁子娶婦或以養生:此之為德,豈直數十百錢哉!此夫老子所謂『上德不德,是以有德』。今夫卜筮者利大而謝少,老子之云豈異於是乎?

「莊子曰:『君子內無饑寒之患,外無劫奪之憂,居上而敬,居下不為害,君子之道也。』今夫卜筮者之為業也,積之無委聚,藏之不用府庫,徙之不用輜車,負裝之不重,止而用之無盡索之時。持不盡索之物,游於無窮之世,雖莊氏之行未能增於是也,子何故而云不可卜哉?天不足西北,星辰西北移;地不足東南,以海為池;日中必移,月滿必虧;先王之道,乍存乍亡。公責卜者言必信,不亦惑乎!

「公見夫談士辯人乎?慮事定計,必是人也,然不能以一言說人主意,故言必稱先王,語必道上古;慮事定計,飾先王之成功,語其敗害,以恐喜人主之志,以求其欲。多言誇嚴,莫大於此矣。然欲彊國成功,盡忠於上,非此不立。今夫卜者,導惑教愚也。夫愚惑之人,豈能以一言而知之哉!言不厭多。

「故騏驥不能與罷驢為駟,而鳳皇不與燕雀為群,而賢者亦不與不肖者同列。故君子處卑隱以辟眾,自匿以辟倫,微見德順以除群害,以明天性,助上養下,多其功利,不求尊譽。公之等喁喁者也,何知長者之道乎!」

宋忠、賈誼忽而自失,芒乎無色,悵然噤口不能言。於是攝衣而起,再拜而辭。行洋洋也,出門僅能自上車,伏軾低頭,卒不能出氣。

居三日,宋忠見賈誼於殿門外,乃相引屏語相謂自嘆曰:「道高益安,勢高益危。居赫赫之勢,失身且有日矣。夫卜而有不審,不見奪糈;為人主計而不審,身無所處。此相去遠矣,猶天冠地屨也。此老子之所謂『無名者萬物之始』也。天地曠曠,物之熙熙,或安或危,莫知居之。我與若,何足預彼哉!彼久而愈安,雖曾氏之義未有以異也。」

久之,宋忠使匈奴,不至而還,抵罪。而賈誼為梁懷王傅,王墮馬薨,誼不食,毒恨而死。此務華絕根者也。

太史公曰:古者卜人所以不載者,多不見于篇。及至司馬季主,余志而著之。

褚先生曰:臣為郎時,游觀長安中,見卜筮之賢大夫,觀其起居行步,坐起自動,誓正其衣冠而當鄉人也,有君子之風。見性好解婦來卜,對之顏色嚴振,未嘗見齒而笑也。從古以來,賢者避世,有居止舞澤者,有居民閒閉口不言,有隱居卜筮閒以全身者。夫司馬季主者,楚賢大夫,游學長安,通易經,術黃帝、老子,博聞遠見。觀其對二大夫貴人之談言,稱引古明王聖人道,固非淺聞小數之能。及卜筮立名聲千里者,各往往而在。傳曰:「富為上,貴次之;既貴各各學一伎能立其身。」黃直,大夫也;陳君夫,婦人也:以相馬立名天下。齊張仲、曲成侯以善擊刺學用劍,立名天下。留長孺以相彘立名。滎陽褚氏以相牛立名。能以伎能立名者甚多,皆有高世絕人之風,何可勝言。故曰:「非其地,樹之不生;非其意,教之不成。」夫家之教子孫,當視其所以好,好含茍生活之道,因而成之。故曰:「制宅命子,足以觀士;子有處所,可謂賢人。」臣為郎時,與太卜待詔為郎者同署,言曰:「孝武帝時,聚會占家問之,某日可取婦乎?五行家曰可,堪輿家曰不可,建除家曰不吉,叢辰家曰大凶,歷家曰小凶,天人家曰小吉,太一家曰大吉。辯訟不決,以狀聞。制曰:『避諸死忌,以五行為主。』」人取於五行者也。


\end{pinyinscope}