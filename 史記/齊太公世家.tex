\article{齊太公世家}

\begin{pinyinscope}
太公望呂尚者,東海上人。其先祖嘗為四嶽,佐禹平水土甚有功。虞夏之際封於呂,或封於申,姓姜氏。夏商之時,申、呂或封枝庶子孫,或為庶人,尚其後苗裔也。本姓姜氏,從其封姓,故曰呂尚。

呂尚蓋嘗窮困,年老矣,以漁釣奸周西伯。西伯將出獵,卜之,曰「所獲非龍非彨非虎非羆;所獲霸王之輔」。於是周西伯獵,果遇太公於渭之陽,與語大說,曰:「自吾先君太公曰『當有聖人適周,周以興』。子真是邪?吾太公望子久矣。」故號之曰「太公望」,載與俱歸,立為師。

或曰,太公博聞,嘗事紂。紂無道,去之。游說諸侯,無所遇,而卒西歸周西伯。或曰,呂尚處士,隱海濱。周西伯拘羑里,散宜生、閎夭素知而招呂尚。呂尚亦曰「吾聞西伯賢,又善養老,盍往焉」。三人者為西伯求美女奇物,獻之於紂,以贖西伯。西伯得以出,反國。言呂尚所以事周雖異,然要之為文武師。

周西伯昌之脫羑里歸,與呂尚陰謀修德以傾商政,其事多兵權與奇計,故後世之言兵及周之陰權皆宗太公為本謀。周西伯政平,及斷虞芮之訟,而詩人稱西伯受命曰文王。伐崇、密須、犬夷,大作豐邑。天下三分,其二歸周者,太公之謀計居多。

文王崩,武王即位。九年,欲修文王業,東伐以觀諸侯集否。師行,師尚父左杖黃鉞,右把白旄以誓,曰:「蒼兕蒼兕,總爾眾庶,與爾舟楫,後至者斬!」遂至盟津。諸侯不期而會者八百諸侯。諸侯皆曰:「紂可伐也。」武王曰:「未可。」還師,與太公作此太誓。

居二年,紂殺王子比干,囚箕子。武王將伐紂,卜龜兆,不吉,風雨暴至。群公盡懼,唯太公彊之勸武王,武王於是遂行。十一年正月甲子,誓於牧野,伐商紂。紂師敗績。紂反走,登鹿臺,遂追斬紂。明日,武王立于社,群公奉明水,衛康叔封布采席,師尚父牽牲,史佚策祝,以告神討紂之罪。散鹿臺之錢,發鉅橋之粟,以振貧民。封比干墓,釋箕子囚。遷九鼎,修周政,與天下更始。師尚父謀居多。

於是武王已平商而王天下,封師尚父於齊營丘。東就國,道宿行遲。逆旅之人曰:「吾聞時難得而易失。客寢甚安,殆非就國者也。」太公聞之,夜衣而行,犁明至國。萊侯來伐,與之爭營丘。營丘邊萊。萊人,夷也,會紂之亂而周初定,未能集遠方,是以與太公爭國。

太公至國,修政,因其俗,簡其禮,通商工之業,便魚鹽之利,而人民多歸齊,齊為大國。及周成王少時,管蔡作亂,淮夷畔周,乃使召康公命太公曰:「東至海,西至河,南至穆陵,北至無棣,五侯九伯,實得征之。」齊由此得征伐,為大國。都營丘。

蓋太公之卒百有餘年,子丁公呂伋立。丁公卒,子乙公得立。乙公卒,子癸公慈母立。癸公卒,子哀公不辰立。

哀公時,紀侯譖之周,周烹哀公而立其弟靜,是為胡公。胡公徙都薄泵,而當周夷王之時。

哀公之同母少弟山怨胡公,乃與其黨率營丘人襲攻殺胡公而自立,是為獻公。獻公元年,盡逐胡公子,因徙薄泵都,治臨菑。

九年,獻公卒,子武公壽立。武公九年,周厲王出奔,居彘。十年,王室亂,大臣行政,號曰「共和」。二十四年,周宣王初立。

二十六年,武公卒,子厲公無忌立。厲公暴虐,故胡公子復入齊,齊人欲立之,乃與攻殺厲公。胡公子亦戰死。齊人乃立厲公子赤為君,是為文公,而誅殺厲公者七十人。

文公十二年卒,子成公脫立。成公九年卒,子莊公購立。

莊公二十四年,犬戎殺幽王,周東徙雒。秦始列為諸侯。五十六年,晉弒其君昭侯。

六十四年,莊公卒,子釐公祿甫立。

釐公九年,魯隱公初立。十九年,魯桓公弒其兄隱公而自立為君。

二十五年,北戎伐齊。鄭使太子忽來救齊,齊欲妻之。忽曰:「鄭小齊大,非我敵。」遂辭之。

三十二年,釐公同母弟夷仲年死。其子曰公孫無知,釐公愛之,令其秩服奉養比太子。

三十三年,釐公卒,太子諸兒立,是為襄公。

襄公元年,始為太子時,嘗與無知鬬,及立,絀無知秩服,無知怨。

四年,魯桓公與夫人如齊。齊襄公故嘗私通魯夫人。魯夫人者,襄公女弟也,自釐公時嫁為魯桓公婦,及桓公來而襄公復通焉。魯桓公知之,怒夫人,夫人以告齊襄公。齊襄公與魯君飲,醉之,使力士彭生抱上魯君車,因拉殺魯桓公,桓公下車則死矣。魯人以為讓,而齊襄公殺彭生以謝魯。

八年,伐紀,紀遷去其邑。

十二年,初,襄公使連稱、管至父戍葵丘,瓜時而往,及瓜而代。往戍一歲,卒瓜時而公弗為發代。或為請代,公弗許。故此二人怒,因公孫無知謀作亂。連稱有從妹在公宮,無寵,使之閒襄公,曰「事成以女為無知夫人」。冬十二月,襄公游姑棼,遂獵沛丘。見彘,從者曰「彭生」。公怒,射之,彘人立而啼。公懼,墜車傷足,失屨。反而鞭主屨者茀三百。茀出宮。而無知、連稱、管至父等聞公傷,乃遂率其眾襲宮。逢主屨茀,茀曰:「且無入驚宮,驚宮未易入也。」無知弗信,茀示之創,乃信之。待宮外,令茀先入。茀先入,即匿襄公戶閒。良久,無知等恐,遂入宮。茀反與宮中及公之幸臣攻無知等,不勝,皆死。無知入宮,求公不得。或見人足於戶閒,發視,乃襄公,遂弒之,而無知自立為齊君。

桓公元年春,齊君無知游於雍林。雍林人嘗有怨無知,及其往游,雍林人襲殺無知,告齊大夫曰:「無知弒襄公自立,臣謹行誅。唯大夫更立公子之當立者,唯命是聽。」

初,襄公之醉殺魯桓公,通其夫人,殺誅數不當,淫於婦人,數欺大臣,群弟恐禍及,故次弟糾奔魯。其母魯女也。管仲、召忽傅之。次弟小白奔莒,鮑叔傅之。小白母,衛女也,有寵於釐公。小白自少好善大夫高傒。及雍林人殺無知,議立君,高、國先陰召小白於莒。魯聞無知死,亦發兵送公子糾,而使管仲別將兵遮莒道,射中小白帶鉤。小白詳死,管仲使人馳報魯。魯送糾者行益遲,六日至齊,則小白已入,高傒立之,是為桓公。

桓公之中鉤,詳死以誤管仲,已而載溫車中馳行,亦有高、國內應,故得先入立,發兵距魯。秋,與魯戰于乾時,魯兵敗走,齊兵掩絕魯歸道。齊遺魯書曰:「子糾兄弟,弗忍誅,請魯自殺之。召忽、管仲讎也,請得而甘心醢之。不然,將圍魯。」魯人患之,遂殺子糾于笙瀆。召忽自殺,管仲請囚。桓公之立,發兵攻魯,心欲殺管仲。鮑叔牙曰:「臣幸得從君,君竟以立。君之尊,臣無以增君。君將治齊,即高傒與叔牙足也。君且欲霸王,非管夷吾不可。夷吾所居國國重,不可失也。」於是桓公從之。乃詳為召管仲欲甘心,實欲用之。管仲知之,故請往。鮑叔牙迎受管仲,及堂阜而脫桎梏,齋祓而見桓公。桓公厚禮以為大夫,任政。

桓公既得管仲,與鮑叔、隰朋、高傒修齊國政,連五家之兵,設輕重魚鹽之利,以贍貧窮,祿賢能,齊人皆說。

二年,伐滅郯,郯子奔莒。初,桓公亡時,過郯,郯無禮,故伐之。

五年,伐魯,魯將師敗。魯莊公請獻遂邑以平,桓公許,與魯會柯而盟。魯將盟,曹沬以匕首劫桓公於壇上,曰:「反魯之侵地!」桓公許之。已而曹沬去匕首,北面就臣位。桓公後悔,欲無與魯地而殺曹沬。管仲曰:「夫劫許之而倍信殺之,愈一小快耳,而棄信於諸侯,失天下之援,不可。」於是遂與曹沬三敗所亡地於魯。諸侯聞之,皆信齊而欲附焉。七年,諸侯會桓公於甄,而桓公於是始霸焉。

十四年,陳厲公子完,號敬仲,來奔齊。齊桓公欲以為卿,讓;於是以為工正。田成子常之祖也。

二十三年,山戎伐燕,燕告急於齊。齊桓公救燕,遂伐山戎,至于孤竹而還。燕莊公遂送桓公入齊境。桓公曰:「非天子,諸侯相送不出境,吾不可以無禮於燕。」於是分溝割燕君所至與燕,命燕君復修召公之政,納貢于周,如成康之時。諸侯聞之,皆從齊。

二十七年,魯湣公母曰哀姜,桓公女弟也。哀姜淫於魯公子慶父,慶父弒湣公,哀姜欲立慶父,魯人更立釐公。桓公召哀姜,殺之。

二十八年,衛文公有狄亂,告急於齊。齊率諸侯城楚丘而立衛君。

二十九年,桓公與夫人蔡姬戲船中。蔡姬習水,蕩公,公懼,止之,不止,出船,怒,歸蔡姬,弗絕。蔡亦怒,嫁其女。桓公聞而怒,興師往伐。

三十年春,齊桓公率諸侯伐蔡,蔡潰。遂伐楚。楚成王興師問曰:「何故涉吾地?」管仲對曰:「昔召康公命我先君太公曰:『五侯九伯,若實征之,以夾輔周室。』賜我先君履,東至海,西至河,南至穆陵,北至無棣。楚貢包茅不入,王祭不具,是以來責。昭王南征不復,是以來問。」楚王曰:「貢之不入,有之,寡人罪也,敢不共乎!昭王之出不復,君其問之水濱。」齊師進次于陘。夏,楚王使屈完將兵捍齊,齊師退次召陵。桓公矜屈完以其眾。屈完曰:「君以道則可;若不,則楚方城以為城,江、漢以為溝,君安能進乎?」乃與屈完盟而去。過陳,陳袁濤涂詐齊,令出東方,覺。秋,齊伐陳。是歲,晉殺太子申生。

三十五年夏,會諸侯于葵丘。周襄王使宰孔賜桓公文武胙、彤弓矢、大路,命無拜。桓公欲許之,管仲曰「不可」,乃下拜受賜。秋,復會諸侯於葵丘,益有驕色。周使宰孔會。諸侯頗有叛者。晉侯病,後,遇宰孔。宰孔曰:「齊侯驕矣,弟無行。」從之。是歲,晉獻公卒,裏克殺奚齊、卓子,秦穆公以夫人入公子夷吾為晉君。桓公於是討晉亂,至高梁,使隰朋立晉君,還。

是時周室微,唯齊、楚、秦、晉為彊。晉初與會,獻公死,國內亂。秦穆公辟遠,不與中國會盟。楚成王初收荊蠻有之,夷狄自置。唯獨齊為中國會盟,而桓公能宣其德,故諸侯賓會。於是桓公稱曰:「寡人南伐至召陵,望熊山;北伐山戎、離枝、孤竹;西伐大夏,涉流沙;束馬懸車登太行,至卑耳山而還。諸侯莫違寡人。寡人兵車之會三,乘車之會六,九合諸侯,一匡天下。昔三代受命,有何以異於此乎?吾欲封泰山,禪梁父。」管仲固諫,不聽;乃說桓公以遠方珍怪物至乃得封,桓公乃止。

三十八年,周襄王弟帶與戎、翟合謀伐周,齊使管仲平戎於周。周欲以上卿禮管仲,管仲頓首曰:「臣陪臣,安敢!」三讓,乃受下卿禮以見。三十九年,周襄王弟帶來奔齊。齊使仲孫請王,為帶謝。襄王怒,弗聽。

四十一年,秦穆公虜晉惠公,復歸之。是歲,管仲、隰朋皆卒。管仲病,桓公問曰:「群臣誰可相者?」管仲曰:「知臣莫如君。」公曰:「易牙如何?」對曰:「殺子以適君,非人情,不可。」公曰:「開方如何?」對曰:「倍親以適君,非人情,難近。」公曰:「豎刀如何?」對曰:「自宮以適君,非人情,難親。」管仲死,而桓公不用管仲言,卒近用三子,三子專權。

四十二年,戎伐周,周告急於齊,齊令諸侯各發卒戍周。是歲,晉公子重耳來,桓公妻之。

四十三年。初,齊桓公之夫人三:曰王姬、徐姬、蔡姬,皆無子。桓公好內,多內寵,如夫人者六人,長衛姬,生無詭;少衛姬,生惠公元;鄭姬,生孝公昭;葛嬴,生昭公潘;密姬,生懿公商人;宋華子,生公子雍。桓公與管仲屬孝公於宋襄公,以為太子。雍巫有寵於衛共姬,因宦者豎刀以厚獻於桓公,亦有寵,桓公許之立無詭。管仲卒,五公子皆求立。冬十月乙亥,齊桓公卒。易牙入,與豎刀因內寵殺群吏,而立公子無詭為君。太子昭奔宋。

桓公病,五公子各樹黨爭立。及桓公卒,遂相攻,以故宮中空,莫敢棺。桓公尸在床上六十七日,尸蟲出于戶。十二月乙亥,無詭立,乃棺赴。辛巳夜,斂殯。

桓公十有餘子,要其後立者五人:無詭立三月死,無謚;次孝公;次昭公;次懿公;次惠公。孝公元年三月,宋襄公率諸侯兵送齊太子昭而伐齊。齊人恐,殺其君無詭。齊人將立太子昭,四公子之徒攻太子,太子走宋,宋遂與齊人四公子戰。五月,宋敗齊四公子師而立太子昭,是為齊孝公。宋以桓公與管仲屬之太子,故來征之。以亂故,八月乃葬齊桓公。

六年春,齊伐宋,以其不同盟于齊也。夏,宋襄公卒。七年,晉文公立。

十年,孝公卒,孝公弟潘因衛公子開方殺孝公子而立潘,是為昭公。昭公,桓公子也,其母曰葛嬴。

昭公元年,晉文公敗楚於城濮,而會諸侯踐土,朝周,天子使晉稱伯。六年,翟侵齊。晉文公卒。秦兵敗於殽。十二年,秦穆公卒。

十九年五月,昭公卒,子舍立為齊君。捨之母無寵於昭公,國人莫畏。昭公之弟商人以桓公死爭立而不得,陰交賢士,附愛百姓,百姓說。及昭公卒,子舍立,孤弱,即與眾十月即墓上弒齊君舍,而商人自立,是為懿公。懿公,桓公子也,其母曰密姬。

懿公四年春,初,懿公為公子時,與丙戎之父獵,爭獲不勝,及即位,斷丙戎父足,而使丙戎仆。庸職之妻好,公內之宮,使庸職驂乘。五月,懿公游於申池,二人浴,戲。職曰:「斷足子!」戎曰:「奪妻者!」二人俱病此言,乃怨。謀與公游竹中,二人弒懿公車上,棄竹中而亡去。

懿公之立,驕,民不附。齊人廢其子而迎公子元於衛,立之,是為惠公。惠公,桓公子也。其母衛女,曰少衛姬,避齊亂,故在衛。

惠公二年,長翟來,王子城父攻殺之,埋之於北門。晉趙穿弒其君靈公。

十年,惠公卒,子頃公無野立。初,崔杼有寵於惠公,惠公卒,高、國畏其偪也,逐之,崔杼奔衛。

頃公元年,楚莊王彊,伐陳;二年,圍鄭,鄭伯降,已復國鄭伯。

六年春,晉使郤克於齊,齊使夫人帷中而觀之。郤克上,夫人笑之。郤克曰:「不是報,不復涉河!」歸,請伐齊,晉侯弗許。齊使至晉,郤克執齊使者四人河內,殺之。八年。晉伐齊,齊以公子彊質晉,晉兵去。十年春,齊伐魯、衛。魯、衛大夫如晉請師,皆因郤克。晉使郤克以車八百乘為中軍將,士燮將上軍,欒書將下軍,以救魯、衛,伐齊。六月壬申,與齊侯兵合靡笄下。癸酉,陳于礱逄丑父為齊頃公右。頃公曰:「馳之,破晉軍會食。」射傷郤克,流血至履。克欲還入壁,其御曰:「我始入,再傷,不敢言疾,恐懼士卒,願子忍之。」遂復戰。戰,齊急,丑父恐齊侯得,乃易處,頃公為右,車絓於木而止。晉小將韓厥伏齊侯車前,曰「寡君使臣救魯、衛」,戲之。丑父使頃公下取飲,因得亡,脫去,入其軍。晉郤克欲殺丑父。丑父曰:「代君死而見僇,後人臣無忠其君者矣。」克捨之,丑父遂得亡歸齊。於是晉軍追齊至馬陵。齊侯請以寶器謝,不聽;必得笑克者蕭桐叔子,令齊東畝。對曰:「叔子,齊君母。齊君母亦猶晉君母,子安置之?且子以義伐而以暴為後,其可乎?」於是乃許,令反魯、衛之侵地。

十一年,晉初置六卿,賞鞌之功。齊頃公朝晉,欲尊王晉景公,晉景公不敢受,乃歸。歸而頃公弛苑囿,薄賦斂,振孤問疾,虛積聚以救民,民亦大說。厚禮諸侯。竟頃公卒,百姓附,諸侯不犯。

十七年,頃公卒,子靈公環立。

靈公九年,晉欒書弒其君厲公。十年,晉悼公伐齊,齊令公子光質晉。十九年,立子光為太子,高厚傅之,令會諸侯盟於鐘離。二十七年,晉使中行獻子伐齊。齊師敗,靈公走入臨菑。晏嬰止靈公,靈公弗從。曰:「君亦無勇矣!」晉兵遂圍臨菑,臨菑城守不敢出,晉焚郭中而去。

二十八年,初,靈公取魯女,生子光,以為太子。仲姬,戎姬。戎姬嬖,仲姬生子牙,屬之戎姬。戎姬請以為太子,公許之。仲姬曰:「不可。光之立,列於諸侯矣,今無故廢之,君必悔之。」公曰:「在我耳。」遂東太子光,使高厚傅牙為太子。靈公疾,崔杼迎故太子光而立之,是為莊公。莊公殺戎姬。五月壬辰,靈公卒,莊公即位,執太子牙於句竇之丘,殺之。八月,崔杼殺高厚。晉聞齊亂,伐齊,至高唐。

莊公三年,晉大夫欒盈奔齊,莊公厚客待之。晏嬰、田文子諫,公弗聽。四年,齊莊公使欒盈閒入晉曲沃為內應,以兵隨之,上太行,入孟門。欒盈敗,齊兵還,取朝歌。

六年,初,棠公妻好,棠公死,崔杼取之。莊公通之,數如崔氏,以崔杼之冠賜人。待者曰:「不可。」崔杼怒,因其伐晉,欲與晉合謀襲齊而不得閒。莊公嘗笞宦者賈舉,賈舉復侍,為崔杼閒公以報怨。五月,莒子朝齊,齊以甲戌饗之。崔杼稱病不視事。乙亥,公問崔杼病,遂從崔杼妻。崔杼妻入室,與崔杼自閉戶不出,公擁柱而歌。宦者賈舉遮公從官而入,閉門,崔杼之徒持兵從中起。公登臺而請解,不許;請盟,不許;請自殺於廟,不許。皆曰:「君之臣杼疾病,不能聽命。近於公宮。陪臣爭趣有淫者,不知二命。」公踰墻,射中公股,公反墜,遂弒之。晏嬰立崔杼門外,曰:「君為社稷死則死之,為社稷亡則亡之。若為己死己亡,非其私暱,誰敢任之!」門開而入,枕公尸而哭,三踴而出。人謂崔杼:「必殺之。」崔杼曰:「民之望也,捨之得民。」

丁丑,崔杼立莊公異母弟杵臼,是為景公。景公母,魯叔孫宣伯女也。景公立,以崔杼為右相,慶封為左相。二相恐亂起,乃與國人盟曰:「不與崔慶者死!」晏子仰天曰:「嬰所不(獲)唯忠於君利社稷者是從!」不肯盟。慶封欲殺晏子,崔杼曰:「忠臣也,捨之。」齊太史書曰「崔杼弒莊公」,崔杼殺之。其弟復書,崔杼復殺之。少弟復書,崔杼乃捨之。

景公元年,初,崔杼生子成及彊,其母死,取東郭女,生明。東郭女使其前夫子無咎與其弟偃相崔氏。成有罪,二相急治之,立明為太子。成請老於崔[杼],崔杼許之,二相弗聽,曰:「崔,宗邑,不可。」成、彊怒,告慶封。慶封與崔杼有郤,欲其敗也。成、彊殺無咎、偃於崔杼家,家皆奔亡。崔杼怒,無人,使一宦者御,見慶封。慶封曰:「請為子誅之。」使崔杼仇盧蒲嫳攻崔氏,殺成、彊,盡滅崔氏,崔杼婦自殺。崔杼毋歸,亦自殺。慶封為相國,專權。

三年十月,慶封出獵。初,慶封已殺崔杼,益驕,嗜酒好獵,不聽政令。慶舍用政,已有內郤。田文子謂桓子曰:「亂將作。」田、鮑、高、欒氏相與謀慶氏。慶舍發甲圍慶封宮,四家徒共擊破之。慶封還,不得入,奔魯。齊人讓魯,封奔吳。吳與之朱方,聚其族而居之,富於在齊。其秋,齊人徙葬莊公,僇崔杼尸於市以說眾。

九年,景公使晏嬰之晉,與叔向私語曰:「齊政卒歸田氏。田氏雖無大德,以公權私,有德於民,民愛之。」十二年,景公如晉,見平公,欲與伐燕。十八年,公復如晉,見昭公。二十六年,獵魯郊,因入魯,與晏嬰俱問魯禮。三十一年,魯昭公辟季氏難,奔齊。齊欲以千社封之,子家止昭公,昭公乃請齊伐魯,取鄆以居昭公。

三十二年,彗星見。景公坐柏寢,嘆曰:「堂堂!誰有此乎?」群臣皆泣,晏子笑,公怒。晏子曰:「臣笑群臣諛甚。」景公曰:「彗星出東北,當齊分野,寡人以為憂。」晏子曰:「君高臺深池,賦斂如弗得,刑罰恐弗勝,茀星將出,彗星何懼乎?」公曰:「可禳否?」晏子曰:「使神可祝而來,亦可禳而去也。百姓苦怨以萬數,而君令一人禳之,安能勝眾口乎?」是時景公好治宮室,聚狗馬,奢侈,厚賦重刑,故晏子以此諫之。

四十二年,吳王闔閭伐楚,入郢。

四十七年,魯陽虎攻其君,不勝,奔齊,請齊伐魯。鮑子諫景公,乃囚陽虎。陽虎得亡,奔晉。

四十八年,與魯定公好會夾谷。犁鉏曰:「孔丘知禮而怯,請令萊人為樂,因執魯君,可得志。」景公害孔丘相魯,懼其霸,故從犁鉏之計。方會,進萊樂,孔子歷階上,使有司執萊人斬之,以禮讓景公。景公慚,乃歸魯侵地以謝,而罷去。是歲,晏嬰卒。

五十五年,范、中行反其君於晉,晉攻之急,來請粟。田乞欲為亂,樹黨於逆臣,說景公曰:「范、中行數有德於齊,不可不救。」及使乞救而輸之粟。

五十八年夏,景公夫人燕姬適子死。景公寵妾芮姬生子荼,荼少,其母賤,無行,諸大夫恐其為嗣,乃言願擇諸子長賢者為太子。景公老,惡言嗣事,又愛荼母,欲立之,憚發之口,乃謂諸大夫曰:「為樂耳,國何患無君乎?」秋,景公病,命國惠子、高昭子立少子荼為太子,逐群公子,遷之萊。景公卒,太子荼立,是為晏孺子。冬,未葬,而群公子畏誅,皆出亡。荼諸異母兄公子壽、駒、黔奔衛,公子駔、陽生奔魯。萊人歌之曰:「景公死乎弗與埋,三軍事乎弗與謀,師乎師乎,胡黨之乎?」

晏孺子元年春,田乞偽事高、國者,每朝,乞驂乘,言曰:「子得君,大夫皆自危,欲謀作亂。」又謂諸大夫曰:「高昭子可畏,及未發,先之。」大夫從之。六月,田乞、鮑牧乃與大夫以兵入公宮,攻高昭子。昭子聞之,與國惠子救公。公師敗,田乞之徒追之,國惠子奔莒,遂反殺高昭子。晏圉奔魯。八月,齊秉意茲。田乞敗二相,乃使人之魯召公子陽生。陽生至齊,私匿田乞家。十月戊子,田乞請諸大夫曰:「常之母有魚菽之祭,幸來會飲。」會飲,田乞盛陽生橐中,置坐中央,發橐出陽生,曰:「此乃齊君矣!」大夫皆伏謁。將與大夫盟而立之,鮑牧醉,乞誣大夫曰:「吾與鮑牧謀共立陽生。」鮑牧怒曰:「子忘景公之命乎?」諸大夫相視欲悔,陽生前,頓首曰:「可則立之,否則已。」鮑牧恐禍起,乃復曰:「皆景公子也,何為不可!」乃與盟,立陽生,是為悼公。悼公入宮,使人遷晏孺子於駘,殺之幕下,而逐孺子母芮子。芮子故賤而孺子少,故無權,國人輕之。

悼公元年,齊伐魯,取讙、闡。初,陽生亡在魯,季康子以其妹妻之。及歸即位,使迎之。季姬與季魴侯通,言其情,魯弗敢與,故齊伐魯,竟迎季姬。季姬嬖,齊復歸魯侵地。

鮑子與悼公有郤,不善。四年,吳、魯伐齊南方。鮑子弒悼公,赴于吳。吳王夫差哭於軍門外三日,將從海入討齊。齊人敗之,吳師乃去。晉趙鞅伐齊,至賴而去。齊人共立悼公子壬,是為簡公。

簡公四年春,初,簡公與父陽生俱在魯也,監止有寵焉。及即位,使為政。田成子憚之,驟顧於朝。御鞅言簡公曰:「田、監不可并也,君其擇焉。」弗聽。子我夕,田逆殺人,逢之,遂捕以入。田氏方睦,使囚病而遺守囚者酒,醉而殺守者,得亡。子我盟諸田於陳宗。初,田豹欲為子我臣,使公孫言豹,豹有喪而止。後卒以為臣,幸於子我。子我謂曰:「吾盡逐田氏而立女,可乎?」對曰:「我遠田氏矣。且其違者不過數人,何盡逐焉!」遂告田氏。子行曰:「彼得君,弗先,必禍子。」子行舍於公宮。

夏五月壬申,成子兄弟四乘如公。子我在幄,出迎之,遂入,閉門。宦者御之,子行殺宦者。公與婦人飲酒於檀臺,成子遷諸寢。公執戈將擊之,太史子餘曰:「非不利也,將除害也。」成子出舍于庫,聞公猶怒,將出,曰:「何所無君!」子行拔劍曰:「需,事之賊也。誰非田宗?所不殺子者有如田宗。」乃止。子我歸,屬徒攻闈與大門,皆弗勝,乃出。田氏追之。豐丘人執子我以告,殺之郭關。成子將殺大陸子方,田逆請而免之。以公命取車於道,出雍門。田豹與之車,弗受,曰:「逆為余請,豹與余車,余有私焉。事子我而有私於其讎,何以見魯、衛之士?」

庚辰,田常執簡公于俆州。公曰:「余蚤從御鞅言,不及此。」甲午,田常弒簡公于俆州。田常乃立簡公弟驁,是為平公。平公即位,田常相之,專齊之政,割齊安平以東為田氏封邑。

平公八年,越滅吳。二十五年卒,子宣公積立。

宣公五十一年卒,子康公貸立。田會反廩丘。康公二年,韓、魏、趙始列為諸侯。十九年,田常曾孫田和始為諸侯,遷康公海濱。

二十六年,康公卒,呂氏遂絕其祀。田氏卒有齊國,為齊威王,彊於天下。

太史公曰:吾適齊,自泰山屬之瑯邪,北被于海,膏壤二千里,其民闊達多匿知,其天性也。以太公之聖,建國本,桓公之盛,修善政,以為諸侯會盟,稱伯,不亦宜乎?洋洋哉,固大國之風也!


\end{pinyinscope}