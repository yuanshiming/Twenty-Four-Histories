\article{荊燕世家}

\begin{pinyinscope}
荊王劉賈者,諸劉,不知其何屬初起時。漢王元年,還定三秦,劉賈為將軍,定塞地,從東擊項籍。

漢四年,漢王之敗成皋,北渡河,得張耳、韓信軍,軍修武,深溝高壘,使劉賈將二萬人,騎數百,渡白馬津入楚地,燒其積聚,以破其業,無以給項王軍食。已而楚兵擊劉賈,賈輒壁不肯與戰,而與彭越相保。

漢五年,漢王追項籍至固陵,使劉賈南渡淮圍壽春。還至,使人閒招楚大司馬周殷。周殷反楚,佐劉賈舉九江,迎武王黥布兵,皆會垓下,共擊項籍。漢王因使劉賈將九江兵,與太尉盧綰西南擊臨江王共尉。共尉已死,以臨江為南郡。

漢六年春,會諸侯於陳,廢楚王信,囚之,分其地為二國。當是時也,高祖子幼,昆弟少,又不賢,欲王同姓以鎮天下,乃詔曰:「將軍劉賈有功,及擇子弟可以為王者。」群臣皆曰:「立劉賈為荊王,王淮東五十二城;高祖弟交為楚王,王淮西三十六城。」因立子肥為齊王。始王昆弟劉氏也。

高祖十一年秋,淮南王黥布反,東擊荊。荊王賈與戰,不勝,走富陵,為布軍所殺。高祖自擊破布。十二年,立沛侯劉濞為吳王,王故荊地。

燕王劉澤者,諸劉遠屬也。高帝三年,澤為郎中。高帝十一年,澤以將軍擊陳豨,得王黃,為營陵侯。

高后時,齊人田生游乏資,以畫干營陵侯澤。澤大說之,用金二百斤為田生壽。田生已得金,即歸齊。二年,澤使人謂田生曰:「弗與矣。」田生如長安,不見澤,而假大宅,令其子求事呂后所幸大謁者張子卿。居數月,田生子請張卿臨,親修具。張卿許往。田生盛帷帳共具,譬如列侯。張卿驚。酒酣,乃屏人說張卿曰:「臣觀諸侯王邸弟百餘,皆高祖一切功臣。今呂氏雅故本推轂高帝就天下,功至大,又親戚太后之重。太后春秋長,諸呂弱,太后欲立呂產為[呂]王,王代。太后又重發之,恐大臣不聽。今卿最幸,大臣所敬,何不風大臣以聞太后,太后必喜。諸呂已王,萬戶侯亦卿之有。太后心欲之,而卿為內臣,不急發,恐禍及身矣。」張卿大然之,乃風大臣語太后。太后朝,因問大臣。大臣請立呂產為呂王。太后賜張卿千斤金,張卿以其半與田生。田生弗受,因說之曰:「呂產王也,諸大臣未大服。今營陵侯澤,諸劉,為大將軍,獨此尚觖望。今卿言太后,列十餘縣王之,彼得王,喜去,諸呂王益固矣。」張卿入言,太后然之。乃以營陵侯劉澤為瑯邪王。瑯邪王乃與田生之國。田生勸澤急行,毋留。出關,太后果使人追止之,已出,即還。

及太后崩,瑯邪王澤乃曰:「帝少,諸呂用事,劉氏孤弱。」乃引兵與齊王合謀西,欲誅諸呂。至梁,聞漢遣灌將軍屯滎陽,澤還兵備西界,遂跳驅至長安。代王亦從代至。諸將相與瑯邪王共立代王為天子。天子乃徙澤為燕王,乃復以瑯邪予齊,復故地。

澤王燕二年,薨,謚為敬王。傳子嘉,為康王。

至孫定國,與父康王姬姦,生子男一人。奪弟妻為姬。與子女三人姦。定國有所欲誅殺臣肥如令郢人,郢人等告定國,定國使謁者以他法劾捕格殺郢人以滅口。至元朔元年,郢人昆弟復上書具言定國陰事,以此發覺。詔下公卿,皆議曰:「定國禽獸行,亂人倫,逆天,當誅。」上許之。定國自殺,國除為郡。

太史公曰:荊王王也,由漢初定,天下未集,故劉賈雖屬疏,然以策為王,填江淮之閒。劉澤之王,權激呂氏,然劉澤卒南面稱孤者三世。事發相重,豈不為偉乎!


\end{pinyinscope}