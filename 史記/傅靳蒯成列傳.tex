\article{傅靳蒯成列傳}

\begin{pinyinscope}
陽陵侯傅寬,以魏五大夫騎將從,為舍人,起橫陽。從攻安陽、杠裏,擊趙賁軍於開封,及擊楊熊曲遇、陽武,斬首十二級,賜爵卿。從至霸上。沛公立為漢王,漢王賜寬封號共德君。從入漢中,遷為右騎將。從定三秦,賜食邑雕陰。從擊項籍,待懷,賜爵通德侯。從擊項冠、周蘭、龍且,所將卒斬騎將一人敖下,益食邑。

屬淮陰,擊破齊歷下軍,擊田解。屬相國參,殘博,益食邑。因定齊地,剖符世世勿絕,封為陽陵侯,二千六百戶,除前所食。為齊右丞相,備齊。五歲為齊相國。

四月,擊陳豨,屬太尉勃,以相國代丞相噲擊豨。一月,徙為代相國,將屯。二歲,為代丞相,將屯。

孝惠五年卒,謚為景侯。子頃侯精立,二十四年卒。子共侯則立,十二年卒。子侯偃立,三十一年,坐與淮南王謀反,死,國除。

信武侯靳歙,以中涓從,起宛朐。攻濟陽。破李由軍。擊秦軍亳南、開封東北,斬騎千人將一人,首五十七級,捕虜七十三人,賜爵封號臨平君。又戰藍田北,斬車司馬二人,騎長一人,首二十八級,捕虜五十七人。至霸上。沛公立為漢王,賜歙爵建武侯,遷為騎都尉。

從定三秦。別西擊章平軍於隴西,破之,定隴西六縣,所將卒斬車司馬、候各四人,騎長十二人。從東擊楚,至彭城。漢軍敗還,保雍丘,去擊反者王武等。略梁地,別將擊邢說軍菑南,破之,身得說都尉二人,司馬、候十二人,降吏卒四千一百八十人。破楚軍滎陽東。三年,賜食邑四千二百戶。

別之河內,擊趙將賁郝軍朝歌,破之,所將卒得騎將二人,車馬二百五十匹。從攻安陽以東,至棘蒲,下七縣。別攻破趙軍,得其將司馬二人,候四人,降吏卒二千四百人。從攻下邯鄲。別下平陽,身斬守相,所將卒斬兵守、郡守各一人,降鄴。從攻朝歌、邯鄲,及別擊破趙軍,降邯鄲郡六縣。還軍敖倉,破項籍軍成皋南,擊絕楚馕道,起滎陽至襄邑。破項冠軍魯下。略地東至繒、郯、下邳,南至蘄、竹邑。擊項悍濟陽下。還擊項籍陳下,破之。別定江陵,降江陵柱國、大司馬以下八人,身得江陵王,生致之雒陽,因定南郡。從至陳,取楚王信,剖符世世勿絕,定食四千六百戶,號信武侯。

以騎都尉從擊代,攻韓信平城下,還軍東垣。有功,遷為車騎將軍,并將梁、趙、齊、燕、楚車騎,別擊陳豨丞相敞,破之,因降曲逆。從擊黥布有功,益封定食五千三百戶。凡斬首九十級,虜百三十二人;別破軍十四,降城五十九,定郡、國各一,縣二十三;得王、柱國各一人,二千石以下至五百石三十九人。

高后五年,歙卒,謚為肅侯。子亭代侯。二十一年,坐事國人過律,孝文後三年,奪侯,國除。

蒯成侯緤者,沛人也,姓周氏。常為高祖參乘,以舍人從起沛。至霸上,西入蜀、漢,還定三秦,食邑池陽。東絕甬道,從出度平陰,遇淮陰侯兵襄國,軍乍利乍不利,終無離上心。以『緤為武侯,食邑三千三百戶。高祖十二年,以緤為蒯成侯,除前所食邑。

上欲自擊陳豨,蒯成侯泣曰:「始秦攻破天下,未嘗自行。今上常自行,是為無人可使者乎?」上以為「愛我」,賜入殿門不趨,殺人不死。

至孝文五年,緤以壽終,謚為貞侯。子昌代侯,有罪,國除。至孝景中二年,封緤子居代侯。至元鼎三年,居為太常,有罪,國除。

太史公曰:陽陵侯傅寬、信武侯靳歙皆高爵,從高祖起山東,攻項籍,誅殺名將,破軍降城以十數,未嘗困辱,此亦天授也。蒯成侯周緤操心堅正,身不見疑,上欲有所之,未嘗不垂涕,此有傷心者然,可謂篤厚君子矣。


\end{pinyinscope}