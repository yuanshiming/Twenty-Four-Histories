\article{田單列傳}

\begin{pinyinscope}
田單者,齊諸田疏屬也。湣王時,單為臨菑市掾,不見知。及燕使樂毅伐破齊,齊湣王出奔,已而保莒城。燕師長驅平齊,而田單走安平,令其宗人盡斷其車軸末而傅鐵籠。已而燕軍攻安平,城壞,齊人走,爭涂,以折車敗,為燕所虜,唯田單宗人以鐵籠故得脫,東保即墨。燕既盡降齊城,唯獨莒、即墨不下。燕軍聞齊王在莒,并兵攻之。淖齒既殺湣王於莒,因堅守,距燕軍,數年不下。燕引兵東圍即墨,即墨大夫出與戰,敗死。城中相與推田單,曰:「安平之戰,田單宗人以鐵籠得全,習兵。」立以為將軍,以即墨距燕。

頃之,燕昭王卒,惠王立,與樂毅有隙。田單聞之,乃縱反閒於燕,宣言曰:「齊王已死,城之不拔者二耳。樂毅畏誅而不敢歸,以伐齊為名,實欲連兵南面而王齊。齊人未附,故且緩攻即墨以待其事。齊人所懼,唯恐他將之來,即墨殘矣。」燕王以為然,使騎劫代樂毅。

樂毅因歸趙,燕人士卒忿。而田單乃令城中人食必祭其先祖於庭,飛鳥悉翔舞城中下食。燕人怪之。田單因宣言曰:「神來下教我。」乃令城中人曰:「當有神人為我師。」有一卒曰:「臣可以為師乎?」因反走。田單乃起,引還,東鄉坐,師事之。卒曰:「臣欺君,誠無能也。」田單曰:「子勿言也!」因師之。每出約束,必稱神師。乃宣言曰:「吾唯懼燕軍之劓所得齊卒,置之前行,與我戰,即墨敗矣。」燕人聞之,如其言。城中人見齊諸降者盡劓,皆怒,堅守,唯恐見得。單又縱反閒曰:「吾懼燕人掘吾城外冢墓,僇先人,可為寒心。」燕軍盡掘壟墓,燒死人。即墨人從城上望見,皆涕泣,俱欲出戰,怒自十倍。

田單知士卒之可用,乃身操版插,與士卒分功,妻妾編於行伍之閒,盡散飲食饗士。令甲卒皆伏,使老弱女子乘城,遣使約降於燕,燕軍皆呼萬歲。田單又收民金,得千溢,令即墨富豪遺燕將,曰:「即墨即降,願無虜掠吾族家妻妾,令安堵。」燕將大喜,許之。燕軍由此益懈。

田單乃收城中得千餘牛,為絳繒衣,畫以五彩龍文,束兵刃於其角,而灌脂束葦於尾,燒其端。鑿城數十穴,夜縱牛,壯士五千人隨其後。牛尾熱,怒而奔燕軍,燕軍夜大驚。牛尾炬火光明炫燿,燕軍視之皆龍文,所觸盡死傷。五千人因銜枚擊之,而城中鼓譟從之,老弱皆擊銅器為聲,聲動天地。燕軍大駭,敗走。齊人遂夷殺其將騎劫。燕軍擾亂奔走,齊人追亡逐北,所過城邑皆畔燕而歸田單,兵日益多,乘勝,燕日敗亡,卒至河上,而齊七十餘城皆復為齊。乃迎襄王於莒,入臨菑而聽政。

襄王封田單,號曰安平君。

太史公曰:兵以正合,以奇勝。善之者,出奇無窮。奇正還相生,如環之無端。夫始如處女,適人開戶;後如脫兔,適不及距:其由單之謂邪!

初,淖齒之殺湣王也,莒人求湣王子法章,得之太史嬓之家,為人灌園。嬓女憐而善遇之。後法章私以情告女,女遂與通。及莒人共立法章為齊王,以莒距燕,而太史氏女遂為后,所謂「君王后」也。

燕之初入齊,聞畫邑人王蠋賢,令軍中曰「環畫邑三十里無入」,以王蠋之故。已而使人謂蠋曰:「齊人多高子之義,吾以子為將,封子萬家。」蠋固謝。燕人曰:「子不聽,吾引三軍而屠畫邑。」王蠋曰:「忠臣不事二君,貞女不更二夫。齊王不聽吾諫,故退而耕於野。國既破亡,吾不能存;今又劫之以兵為君將,是助桀為暴也。與其生而無義,固不如烹!」遂經其頸於樹枝,自奮絕脰而死。齊亡大夫聞之,曰:「王蠋,布衣也,義不北面於燕,況在位食祿者乎!」乃相聚如莒,求諸子,立為襄王。


\end{pinyinscope}