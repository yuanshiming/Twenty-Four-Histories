\article{魯周公世家}

\begin{pinyinscope}
周公旦者,周武王弟也。自文王在時,旦為子孝,篤仁,異於群子。及武王即位,旦常輔翼武王,用事居多。武王九年,東伐至盟津,周公輔行。十一年,伐紂,至牧野,周公佐武王,作牧誓。破殷,入商宮。已殺紂,周公把大鉞,召公把小鉞,以夾武王,釁社,告紂之罪于天,及殷民。釋箕子之囚。封紂子武庚祿父,使管叔、蔡叔傅之,以續殷祀。遍封功臣同姓戚者。封周公旦於少昊之虛曲阜,是為魯公。周公不就封,留佐武王。

武王克殷二年,天下未集,武王有疾,不豫,群臣懼,太公、召公乃繆卜。周公曰:「未可以戚我先王。」周公於是乃自以為質,設三壇,周公北面立,戴璧秉圭,告于太王、王季、文王。史策祝曰:「惟爾元孫王發,勤勞阻疾。若爾三王是有負子之責於天,以旦代王發之身。旦巧能,多材多藝,能事鬼神。乃王發不如旦多材多藝,不能事鬼神。乃命于帝庭,敷佑四方,用能定汝子孫于下地,四方之民罔不敬畏。無墜天之降葆命,我先王亦永有所依歸。今我其即命於元龜,爾之許我,我以其璧與圭歸,以俟爾命。爾不許我,我乃屏璧與圭。」周公已令史策告太王、王季、文王,欲代武王發,於是乃即三王而卜。卜人皆曰吉,發書視之,信吉。周公喜,開籥,乃見書遇吉。周公入賀武王曰:「王其無害。旦新受命三王,維長終是圖。茲道能念予一人。」周公藏其策金縢匱中,誡守者勿敢言。明日,武王有瘳。

其後武王既崩,成王少,在彊葆之中。周公恐天下聞武王崩而畔,周公乃踐阼代成王攝行政當國。管叔及其群弟流言於國曰:「周公將不利於成王。」周公乃告太公望、召公奭曰:「我之所以弗辟而攝行政者,恐天下畔周,無以告我先王太王、王季、文王。三王之憂勞天下久矣,於今而后成。武王蚤終,成王少,將以成周,我所以為之若此。」於是卒相成王,而使其子伯禽代就封於魯。周公戒伯禽曰:「我文王之子,武王之弟,成王之叔父,我於天下亦不賤矣。然我一沐三捉發,一飯三吐哺,起以待士,猶恐失天下之賢人。子之魯,慎無以國驕人。」

管、蔡、武庚等果率淮夷而反。周公乃奉成王命,興師東伐,作大誥。遂誅管叔,殺武庚,放蔡叔。收殷餘民,以封康叔於衛,封微子於宋,以奉殷祀。寧淮夷東土,二年而畢定。諸侯咸服宗周。

天降祉福,唐叔得禾,異母同穎,獻之成王,成王命唐叔以餽周公於東土,作餽禾。周公既受命禾,嘉天子命,作嘉禾。東土以集,周公歸報成王,乃為詩貽王,命之曰鴟鸮。王亦未敢訓周公。

成王七年二月乙未,王朝步自周,至豐,使太保召公先之雒相土。其三月,周公往營成周雒邑,卜居焉,曰吉,遂國之。

成王長,能聽政。於是周公乃還政於成王,成王臨朝。周公之代成王治,南面倍依以朝諸侯。及七年後,還政成王,北面就臣位,匔匔如畏然。

初,成王少時,病,周公乃自揃其蚤沈之河,以祝於神曰:「王少未有識,奸神命者乃旦也。」亦藏其策於府。成王病有瘳。及成王用事,人或譖周公,周公奔楚。成王發府,見周公禱書,乃泣,反周公。

周公歸,恐成王壯,治有所淫佚,乃作多士,作毋逸。毋逸稱:「為人父母,為業至長久,子孫驕奢忘之,以亡其家,為人子可不慎乎!笔昔在殷王中宗,嚴恭敬畏天命,自度治民,震懼不敢荒寧,故中宗饗國七十五年。其在高宗,久勞于外,為與小人,作其即位,乃有亮闇,三年不言,言乃讙,不敢荒寧,密靖殷國,至于小大無怨,故高宗饗國五十五年。其在祖甲,不義惟王,久為小人于外,知小人之依,能保施小民,不侮寡,故祖甲饗國三十三年。」多士稱曰:「自湯至于帝乙,無不率祀明德,帝無不配天者。在今後嗣王紂,誕淫厥佚,不顧天及民之從也。其民皆可誅。」(周多士)「文王日中昃不暇食,饗國五十年。」作此以誡成王。

成王在豐,天下已安,周之官政未次序,於是周公作周官,官別其宜,作立政,以便百姓。百姓說。

周公在豐,病,將沒,曰:「必葬我成周,以明吾不敢離成王。」周公既卒,成王亦讓,葬周公於畢,從文王,以明予小子不敢臣周公也。

周公卒後,秋未穫,暴風雷[雨],禾盡偃,大木盡拔。周國大恐。成王與大夫朝服以開金縢書,王乃得周公所自以為功代武王之說。二公及王乃問史百執事,史百執事曰:「信有,昔周公命我勿敢言。」成王執書以泣,曰:「自今後其無繆卜乎!昔周公勤勞王家,惟予幼人弗及知。今天動威以彰周公之德,惟朕小子其迎,我國家禮亦宜之。」王出郊,天乃雨,反風,禾盡起。二公命國人,凡大木所偃,盡起而筑之。歲則大孰。於是成王乃命魯得郊祭文王。魯有天子禮樂者,以褒周公之德也。

周公卒,子伯禽固已前受封,是為魯公。魯公伯禽之初受封之魯,三年而後報政周公。周公曰:「何遲也?」伯禽曰:「變其俗,革其禮,喪三年然後除之,故遲。」太公亦封於齊,五月而報政周公。周公曰:「何疾也?」曰:「吾簡其君臣禮,從其俗為也。」及後聞伯禽報政遲,乃嘆曰:「嗚呼,魯後世其北面事齊矣!夫政不簡不易,民不有近;平易近民,民必歸之。」

伯禽即位之後,有管、蔡等反也,淮夷、徐戎亦并興反。於是伯禽率師伐之於肸,作肸誓,曰:「陳爾甲胄,無敢不善。無敢傷牿。馬牛其風,臣妾逋逃,勿敢越逐,敬復之。無敢寇攘,踰墻垣。魯人三郊三隧,峙爾芻茭、糗糧、楨榦,無敢不逮。我甲戌筑而征徐戎,無敢不及,有大刑。」作此肸誓,遂平徐戎,定魯。

魯公伯禽卒,子考公酋立。考公四年卒,立弟熙,是謂煬公。煬公筑茅闕門。六年卒,子幽公宰立。幽公十四年。幽公弟沸殺幽公而自立,是為魏公。魏公五十年卒,子厲公擢立。厲公三十七年卒,魯人立其弟具,是為獻公。獻公三十二年卒,子真公濞立。

真公十四年,周厲王無道,出奔彘,共和行政。二十九年,周宣王即位。

三十年,真公卒,弟敖立,是為武公。

武公九年春,武公與長子括,少子戲,西朝周宣王。宣王愛戲,欲立戲為魯太子。周之樊仲山父諫宣王曰:「廢長立少,不順;不順,必犯王命;犯王命,必誅之:故出令不可不順也。令之不行,政之不立;行而不順,民將棄上。夫下事上,少事長,所以為順。今天子建諸侯,立其少,是教民逆也。若魯從之,諸侯效之,王命將有所壅;若弗從而誅之,是自誅王命也。誅之亦失,不誅亦失,王其圖之。」宣王弗聽,卒立戲為魯太子。夏,武公歸而卒,戲立,是為懿公。

懿公九年,懿公兄括之子伯御與魯人攻弒懿公,而立伯御為君。伯御即位十一年,周宣王伐魯,殺其君伯御,而問魯公子能道順諸侯者,以為魯後。樊穆仲曰:「魯懿公弟稱,肅恭明神,敬事耆老;賦事行刑,必問於遺訓而咨於固實;不干所問,不犯所(知)[咨]。」宣王曰:「然,能訓治其民矣。」乃立稱於夷宮,是為孝公。自是後,諸侯多畔王命。

孝公二十五年,諸侯畔周,犬戎殺幽王。秦始列為諸侯。

二十七年,孝公卒,子弗湟立,是為惠公。

惠公三十年,晉人弒其君昭侯。四十五年,晉人又弒其君孝侯。

四十六年,惠公卒,長庶子息攝當國,行君事,是為隱公。初,惠公適夫人無子,公賤妾聲子生子息。息長,為娶於宋。宋女至而好,惠公奪而自妻之。生子允。登宋女為夫人,以允為太子。及惠公卒,為允少故,魯人共令息攝政,不言即位。

隱公五年,觀漁於棠。八年,與鄭易天子之太山之邑祊及許田,君子譏之。

十一年冬,公子揮諂謂隱公曰:「百姓便君,君其遂立。吾請為君殺子允,君以我為相。」隱公曰:「有先君命。吾為允少,故攝代。今允長矣,吾方營菟裘之地而老焉,以授子允政。」揮懼子允聞而反誅之,乃反譖隱公於子允曰:「隱公欲遂立,去子,子其圖之。請為子殺隱公。」子允許諾。十一月,隱公祭鐘巫,齊于社圃,館于蒍氏。揮使人殺隱公于蒍氏,而立子允為君,是為桓公。

桓公元年,鄭以璧易天子之許田。二年,以宋之賂鼎入於太廟,君子譏之。

三年,使揮迎婦于齊為夫人。六年,夫人生子,與桓公同日,故名曰同。同長,為太子。

十六年,會于曹,伐鄭,入厲公。

十八年春,公將有行,遂與夫人如齊。申繻諫止,公不聽,遂如齊。齊襄公通桓公夫人。公怒夫人,夫人以告齊侯。夏四月丙子,齊襄公饗公,公醉,使公子彭生抱魯桓公,因命彭生摺其脅,公死于車。魯人告于齊曰:「寡君畏君之威,不敢寧居,來修好禮。禮成而不反,無所歸咎,請得彭生除丑於諸侯。」齊人殺彭生以說魯。立太子同,是為莊公。莊公母夫人因留齊,不敢歸魯。

莊公五年冬,伐衛,內衛惠公。

八年,齊公子糾來奔。九年,魯欲內子糾於齊,後桓公,桓公發兵擊魯,魯急,殺子糾。召忽死。齊告魯生致管仲。魯人施伯曰:「齊欲得管仲,非殺之也,將用之,用之則為魯患。不如殺,以其尸與之。」莊公不聽,遂囚管仲與齊。齊人相管仲。

十三年,魯莊公與曹沬會齊桓公於柯,曹沬劫齊桓公,求魯侵地,已盟而釋桓公。桓公欲背約,管仲諫,卒歸魯侵地。十五年,齊桓公始霸。二十三年,莊公如齊觀社。

三十二年,初,莊公筑臺臨黨氏,見孟女,說而愛之,許立為夫人,割臂以盟。孟女生子斑。斑長,說梁氏女,往觀。圉人犖自墻外與梁氏女戲。斑怒,鞭犖。莊公聞之,曰:「犖有力焉,遂殺之,是未可鞭而置也。」斑未得殺。會莊公有疾。莊公有三弟,長曰慶父,次曰叔牙,次曰季友。莊公取齊女為夫人曰哀姜。哀姜無子。哀姜娣曰叔姜,生子開。莊公無適嗣,愛孟女,欲立其子斑。莊公病,而問嗣於弟叔牙。叔牙曰:「一繼一及,魯之常也。慶父在,可為嗣,君何憂?」莊公患叔牙欲立慶父,退而問季友。季友曰:「請以死立斑也。」莊公曰:「曩者叔牙欲立慶父,柰何?」季友以莊公命命牙待於鍼巫氏,使鍼季劫飲叔牙以鴆,曰:「飲此則有後奉祀;不然,死且無後。」牙遂飲鴆而死,魯立其子為叔孫氏。八月癸亥,莊公卒,季友竟立子斑為君,如莊公命。侍喪,舍于黨氏。

先時慶父與哀姜私通,欲立哀姜娣子開。及莊公卒而季友立斑,十月己未,慶父使圉人犖殺魯公子斑於黨氏。季友奔陳。慶父竟立莊公子開,是為湣公。

湣公二年,慶父與哀姜通益甚。哀姜與慶父謀殺湣公而立慶父。慶父使卜齮襲殺湣公於武闈。季友聞之,自陳與湣公弟申如邾,請魯求內之。魯人欲誅慶父。慶父恐,奔莒。於是季友奉子申入,立之,是為釐公。釐公亦莊公少子。哀姜恐,奔邾。季友以賂如莒求慶父,慶父歸,使人殺慶父,慶父請奔,弗聽,乃使大夫奚斯行哭而往。慶父聞奚斯音,乃自殺。齊桓公聞哀姜與慶父亂以危魯,及召之邾而殺之,以其尸歸,戮之魯。魯釐公請而葬之。

季友母陳女,故亡在陳,陳故佐送季友及子申。季友之將生也,父魯桓公使人卜之,曰:「男也,其名曰『友』,閒于兩社,為公室輔。季友亡,則魯不昌。」及生,有文在掌曰「友」,遂以名之,號為成季。其後為季氏,慶父後為孟氏也。

釐公元年,以汶陽鄪封季友。季友為相。

九年,晉裏克殺其君奚齊、卓子。齊桓公率釐公討晉亂,至高梁而還,立晉惠公。十七年,齊桓公卒。二十四年,晉文公即位。

三十三年,釐公卒,子興立,是為文公。

文公元年,楚太子商臣弒其父成王,代立。三年,文公朝晉襄父。

十一年十月甲午,魯敗翟于咸,獲長翟喬如,富父終甥舂其喉,以戈殺之,埋其首於子駒之門,以命宣伯。

初,宋武公之世,鄋瞞伐宋,司徒皇父帥師御之,以敗翟于長丘,獲長翟緣斯。晉之滅路,獲喬如弟棼如。齊惠公二年,鄋瞞伐齊,齊王子城父獲其弟榮如,埋其首於北門。衛人獲其季弟簡如。鄋瞞由是遂亡。

十五年,季文子使於晉。

十八年二月,文公卒。文公有二妃:長妃齊女為哀姜,生子惡及視;次妃敬嬴,嬖愛,生子俀。俀私事襄仲,襄仲欲立之,叔仲曰不可。襄仲請齊惠公,惠公新立,欲親魯,許之。冬十月,襄仲殺子惡及視而立俀,是為宣公。哀姜歸齊,哭而過市,曰:「天乎!襄仲為不道,殺適立庶!」市人皆哭,魯人謂之「哀姜」。魯由此公室卑,三桓彊。

宣公俀十二年,楚莊王彊,圍鄭。鄭伯降,復國之。

十八年,宣公卒,子成公黑肱立,是為成公。季文子曰:「使我殺適立庶失大援者,襄仲。」襄仲立宣公,公孫歸父有寵。宣公欲去三桓,與晉謀伐三桓。會宣公卒,季文子怨之,歸父奔齊。

成公二年春,齊伐取我隆。夏,公與晉郤克敗齊頃公於砹齊復歸我侵地。四年,成公如晉,晉景公不敬魯。魯欲背晉合於楚,或諫,乃不。十年,成公如晉。晉景公卒,因留成公送葬,魯諱之。十五年,始與吳王壽夢會鐘離。

十六年,宣伯告晉,欲誅季文子。文子有義,晉人弗許。

十八年,成公卒,子午立,是為襄公。是時襄公三歲也。

襄公元年,晉立悼公。往年冬,晉欒書弒其君厲公。四年,襄公朝晉。

五年,季文子卒。家無衣帛之妾,廄無食粟之馬,府無金玉,以相三君。君子曰:「季文子廉忠矣。」

九年,與晉伐鄭。晉悼公冠襄公於衛,季武子從,相行禮。

十一年,三桓氏分為三軍。

十二年,朝晉。十六年,晉平公即位。二十一年,朝晉平公。

二十二年,孔丘生。

二十五年,齊崔杼弒其君莊公,立其弟景公。

二十九年,吳延陵季子使魯,問周樂,盡知其意,魯人敬焉。

三十一年六月,襄公卒。其九月,太子卒。魯人立齊歸之子裯為君,是為昭公。

昭公年十九,猶有童心。穆叔不欲立,曰:「太子死,有母弟可立,不即立長。年鈞擇賢,義鈞則卜之。今裯非適嗣,且又居喪意不在戚而有喜色,若果立,必為季氏憂。」季武子弗聽,卒立之。比及葬,三易衰。君子曰:「是不終也。」

昭公三年,朝晉至河,晉平公謝還之,魯恥焉。四年,楚靈王會諸侯於申,昭公稱病不往。七年,季武子卒。八年,楚靈王就章華臺,召昭公。昭公往賀,賜昭公寶器;已而悔,復詐取之。十二年,朝晉至河,晉平公謝還之。十三年,楚公子棄疾弒其君靈王,代立。十五年,朝晉,晉留之葬晉昭公,魯恥之。二十年,齊景公與晏子狩竟,因入魯問禮。二十一年,朝晉至河,晉謝還之。

二十五年春,鸜鵒來巢。師己曰:「文成之世童謠曰『鸜鵒來巢,公在乾侯。鸜鵒入處,公在外野』。」

季氏與郈氏鬬雞,季氏芥雞羽,郈氏金距。季平子怒而侵郈氏,郈昭伯亦怒平子。臧昭伯之弟會偽讒臧氏,匿季氏,臧昭伯囚季氏人。季平子怒,囚臧氏老。臧、郈氏以難告昭公。昭公九月戊戌伐季氏,遂入。平子登臺請曰:「君以讒不察臣罪,誅之,請遷沂上。」弗許。請囚於鄪,弗許。請以五乘亡,弗許。子家駒曰:「君其許之。政自季氏久矣,為徒者眾,眾將合謀。」弗聽。郈氏曰:「必殺之。」叔孫氏之臣戾謂其眾曰:「無季氏與有,孰利?」皆曰:「無季氏是無叔孫氏。」戾曰:「然,救季氏!」遂敗公師。孟懿子聞叔孫氏勝,亦殺郈昭伯。郈昭伯為公使,故孟氏得之。三家共伐公,公遂奔。己亥,公至于齊。齊景公曰:「請致千社待君。」子家曰:「棄周公之業而臣於齊,可乎?」乃止。子家曰:「齊景公無信,不如早之晉。」弗從。叔孫見公還,見平子,平子頓首。初欲迎昭公,孟孫、季孫後悔,乃止。

二十六年春,齊伐魯,取鄆而居昭公焉。夏,齊景公將內公,令無受魯賂。申豐、汝賈許齊臣高龁、子將粟五千庾。子將言於齊侯曰:「群臣不能事魯君,有異焉。宋元公為魯如晉,求內之,道卒。叔孫昭子求內其君,無病而死。不知天棄魯乎?抑魯君有罪于鬼神也?願君且待。」齊景公從之。

二十八年,昭公如晉,求入。季平子私於晉六卿,六卿受季氏賂,諫晉君,晉君乃止,居昭公乾侯。二十九年,昭公如鄆。齊景公使人賜昭公書,自謂「主君」。昭公恥之,怒而去乾侯。三十一年,晉欲內昭公,召季平子。平子布衣跣行,因六卿謝罪。六卿為言曰:「晉欲內昭公,眾不從。」晉人止。三十二年,昭公卒於乾侯。魯人共立昭公弟宋為君,是為定公。

定公立,趙簡子問史墨曰:「季氏亡乎?」史墨對曰:「不亡。季友有大功於魯,受鄪為上卿,至于文子、武子,世增其業。魯文公卒,東門遂殺適立庶,魯君於是失國政。政在季氏,於今四君矣。民不知君,何以得國!是以為君慎器與名,不可以假人。」

定公五年,季平子卒。陽虎私怒,囚季桓子,與盟,乃捨之。七年,齊伐我,取鄆,以為魯陽虎邑以從政。八年,陽虎欲盡殺三桓適,而更立其所善庶子以代之;載季桓子將殺之,桓子詐而得脫。三桓共攻陽虎,陽虎居陽關。九年,魯伐陽虎,陽虎奔齊,已而奔晉趙氏。

十年,定公與齊景公會於夾谷,孔子行相事。齊欲襲魯君,孔子以禮歷階,誅齊淫樂,齊侯懼,乃止,歸魯侵地而謝過。十二年,使仲由毀三桓城,收其甲兵。孟氏不肯墮城,伐之,不克而止。季桓子受齊女樂,孔子去。

十五年,定公卒,子將立,是為哀公。

哀公五年,齊景公卒。六年,齊田乞弒其君孺子。

七年,吳王夫差彊,伐齊,至繒,徵百牢於魯。季康子使子貢說吳王及太宰嚭,以禮詘之。吳王曰:「我文身,不足責禮。」乃止。

吳為鄒伐魯,至城下,盟而去。齊伐我,取三邑。十年,伐齊南邊。

齊伐魯。季氏用冉有有功,思孔子,孔子自衛歸魯。齊田常弒其君簡公於俆州。孔子請伐之,哀公不聽。

十五年,使子服景伯、子貢為介,適齊,齊歸我侵地。田常初相,欲親諸侯。

十六年,孔子卒。

二十二年,越王句踐滅吳王夫差。

二十七年春,季康子卒。夏,哀公患三桓,將欲因諸侯以劫之,三桓亦患公作難,故君臣多閒。公游于陵阪,遇孟武伯於街,曰:「請問余及死乎?」對曰:「不知也。」公欲以越伐三桓。八月,哀公如陘氏。三桓攻公,公奔于衛,去如鄒,遂如越。國人迎哀公復歸,卒于有山氏。子寧立,是為悼公。

悼公之時,三桓勝,魯如小侯,卑於三桓之家。

十三年,三晉滅智伯,分其地有之。

三十七年,悼公卒,子嘉立,是為元公。元公二十一年卒,子顯立,是為穆公。穆公三十三年卒,子奮立,是為共公。共公二十二年卒,子屯立,是為康公。康公九年卒,子匽立,是為景公。景公二十九年卒,子叔立,是為平公。是時六國皆稱王。

平公十二年,秦惠王卒。二十(二)年,平公卒,子賈立,是為文公。文公(七)[元]年,楚懷王死于秦。二十三年,文公卒,子讎立,是為頃公。

頃公二年,秦拔楚之郢,楚頃王東徙于陳。十九年,楚伐我,取徐州。二十四年,楚考烈王伐滅魯。頃公亡,遷於下邑,為家人,魯絕祀。頃公卒于柯。

魯起周公至頃公,凡三十四世。

太史公曰:余聞孔子稱曰「甚矣魯道之衰也!洙泗之閒龂龂如也」。觀慶父及叔牙閔公之際,何其亂也?隱桓之事;襄仲殺適立庶;三家北面為臣,親攻昭公,昭公以奔。至其揖讓之禮則從矣,而行事何其戾也?


\end{pinyinscope}