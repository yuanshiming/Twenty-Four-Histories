\article{東越列傳}

\begin{pinyinscope}
閩越王無諸及越東海王搖者,其先皆越王句踐之後也,姓騶氏。秦已并天下,皆廢為君長,以其地為閩中郡。及諸侯畔秦,無諸、搖率越歸鄱陽令吳芮,所謂鄱君者也,從諸侯滅秦。當是之時,項籍主命,弗王,以故不附楚。漢擊項籍,無諸、搖率越人佐漢。漢五年,復立無諸為閩越王,王閩中故地,都東冶。孝惠三年,舉高帝時越功,曰閩君搖功多,其民便附,乃立搖為東海王,都東甌,世俗號為東甌王。

後數世,至孝景三年,吳王濞反,欲從閩越,閩越未肯行,獨東甌從吳。及吳破,東甌受漢購,殺吳王丹徒,以故皆得不誅,歸國。

吳王子子駒亡走閩越,怨東甌殺其父,常勸閩越擊東甌。至建元三年,閩越發兵圍東甌。東甌食盡,困,且降,乃使人告急天子。天子問太尉田蚡,蚡對曰:「越人相攻擊,固其常,又數反覆,不足以煩中國往救也。自秦時棄弗屬。」於是中大夫莊助詰蚡曰:「特患力弗能救,德弗能覆;誠能,何故棄之?且秦舉咸陽而棄之,何乃越也!今小國以窮困來告急天子,天子弗振,彼當安所告愬?又何以子萬國乎?」上曰:「太尉未足與計。吾初即位,不欲出虎符發兵郡國。」乃遣莊助以節發兵會稽。會稽太守欲距不為發兵,助乃斬一司馬,諭意指,遂發兵浮海救東甌。未至,閩越引兵而去。東甌請舉國徙中國,乃悉舉眾來,處江淮之閒。

至建元六年,閩越擊南越。南越守天子約,不敢擅發兵擊而以聞。上遣大行王恢出豫章,大農韓安國出會稽,皆為將軍。兵未踰嶺,閩越王郢發兵距險。其弟餘善乃與相、宗族謀曰:「王以擅發兵擊南越,不請,故天子兵來誅。今漢兵眾彊,今即幸勝之,后來益多,終滅國而止。今殺王以謝天子。天子聽,罷兵,固一國完;不聽,乃力戰;不勝,即亡入海。」皆曰「善」。即鏦殺王,使使奉其頭致大行。大行曰:「所為來者誅王。今王頭至,謝罪,不戰而耘,利莫大焉。」乃以便宜案兵告大農軍,而使使奉王頭馳報天子。詔罷兩將兵,曰:「郢等首惡,獨無諸孫繇君丑不與謀焉。」乃使郎中將立丑為越繇王,奉閩越先祭祀。

餘善已殺郢,威行於國,國民多屬,竊自立為王。繇王不能矯其眾持正。天子聞之,為餘善不足復興師,曰:「餘善數與郢謀亂,而後首誅郢,師得不勞。」因立餘善為東越王,與繇王并處。

至元鼎五年,南越反,東越王餘善上書,請以卒八千人從樓船將軍擊呂嘉等。兵至揭揚,以海風波為解,不行,持兩端,陰使南越。及漢破番禺,不至。是時樓船將軍楊仆使使上書,願便引兵擊東越。上曰士卒勞倦,不許,罷兵,令諸校屯豫章梅領待命。

元鼎六年秋,餘善聞樓船請誅之,漢兵臨境,且往,乃遂反,發兵距漢道。號將軍騶力等為「吞漢將軍」,入白沙、武林、梅嶺,殺漢三校尉。是時漢使大農張成、故山州侯齒將屯,弗敢擊,卻就便處,皆坐畏懦誅。

餘善刻「武帝」璽自立,詐其民,為妄言。天子遣橫海將軍韓說出句章,浮海從東方往;樓船將軍楊仆出武林;中尉王溫舒出梅嶺;越侯為戈船、下瀨將軍,出若邪、白沙。元封元年冬,咸入東越。東越素發兵距險,使徇北將軍守武林,敗樓船軍數校尉,殺長吏。樓船將軍率錢唐轅終古斬徇北將軍,為御兒侯。自兵未往。

故越衍侯吳陽前在漢,漢使歸諭餘善,餘善弗聽。及橫海將軍先至,越衍侯吳陽以其邑七百人反,攻越軍於漢陽。從建成侯敖,與其率,從繇王居股謀曰:「餘善首惡,劫守吾屬。今漢兵至,眾彊,計殺餘善,自歸諸將,儻幸得脫。」乃遂俱殺餘善,以其眾降橫海將軍,故封繇王居股為東成侯,萬戶;封建成侯敖為開陵侯;封越衍侯吳陽為北石侯;封橫海將軍說為案道侯;封橫海校尉福為繚嫈侯。福者,成陽共王子,故為海常侯,坐法失侯。舊從軍無功,以宗室故侯。諸將皆無成功,莫封。東越將多軍,漢兵至,棄其軍降,封為無錫侯。

於是天子曰東越狹多阻,閩越悍,數反覆,詔軍吏皆將其民徙處江淮閒。東越地遂虛。

太史公曰:越雖蠻夷,其先豈嘗有大功德於民哉,何其久也!歷數代常為君王,句踐一稱伯。然餘善至大逆,滅國遷眾,其先苗裔繇王居股等猶尚封為萬戶侯,由此知越世世為公侯矣。蓋禹之餘烈也。


\end{pinyinscope}