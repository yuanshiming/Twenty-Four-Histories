\article{酈生陸賈列傳}

\begin{pinyinscope}
酈生食其者,陳留高陽人也。好讀書,家貧落魄,無以為衣食業,為里監門吏。然縣中賢豪不敢役,縣中皆謂之狂生。

及陳勝、項梁等起,諸將徇地過高陽者數十人,酈生聞其將皆握齱好苛禮自用,不能聽大度之言,酈生乃深自藏匿。後聞沛公將兵略地陳留郊,沛公麾下騎士適酈生裏中子也,沛公時時問邑中賢士豪俊。騎士歸,酈生見謂之曰:「吾聞沛公慢而易人,多大略,此真吾所願從游,莫為我先。若見沛公,謂曰『臣里中有酈生,年六十餘,長八尺,人皆謂之狂生,生自謂我非狂生』。」騎士曰:「沛公不好儒,諸客冠儒冠來者,沛公輒解其冠,溲溺其中。與人言,常大罵。未可以儒生說也。」酈生曰:「弟言之。」騎士從容言如酈生所誡者。

沛公至高陽傳舍,使人召酈生。酈生至,入謁,沛公方倨床使兩女子洗足,而見酈生。酈生入,則長揖不拜,曰:「足下欲助秦攻諸侯乎?且欲率諸侯破秦也?」沛公罵曰:「豎儒!夫天下同苦秦久矣,故諸侯相率而攻秦,何謂助秦攻諸侯乎?」酈生曰:「必聚徒合義兵誅無道秦,不宜倨見長者。」於是沛公輟洗,起攝衣,延酈生上坐,謝之。酈生因言六國從橫時。沛公喜,賜酈生食,問曰:「計將安出?」酈生曰:「足下起糾合之眾,收散亂之兵,不滿萬人,欲以徑入彊秦,此所謂探虎口者也。夫陳留,天下之衝,四通五達之郊也,今其城又多積粟。臣善其令,請得使之,令下足下。即不聽,足下舉兵攻之,臣為內應。」於是遣酈生行,沛公引兵隨之,遂下陳留。號酈食其為廣野君。

酈生言其弟酈商,使將數千人從沛公西南略地。酈生常為說客,馳使諸侯。

漢三年秋,項羽擊漢,拔滎陽,漢兵遁保鞏、洛。楚人聞淮陰侯破趙,彭越數反梁地,則分兵救之。淮陰方東擊齊,漢王數困滎陽、成皋,計欲捐成皋以東,屯鞏、洛以拒楚。酈生因曰:「臣聞知天之天者,王事可成;不知天之天者,王事不可成。王者以民人為天,而民人以食為天。夫敖倉,天下轉輸久矣,臣聞其下乃有藏粟甚多。楚人拔滎陽,不堅守敖倉,乃引而東,令適卒分守成皋,此乃天所以資漢也。方今楚易取而漢反卻,自奪其便,臣竊以為過矣。且兩雄不俱立,楚漢久相持不決,百姓騷動,海內搖蕩,農夫釋耒,工女下機,天下之心未有所定也。願足下急復進兵,收取滎陽,據敖倉之粟,塞成皋之險,杜大行之道,距蜚狐之口,守白馬之津,以示諸侯效實形制之勢,則天下知所歸矣。方今燕、趙已定,唯齊未下。今田廣據千里之齊,田閒將二十萬之眾,軍於歷城,諸田宗彊,負海阻河濟,南近楚,人多變詐,足下雖遣數十萬師,未可以歲月破也。臣請得奉明詔說齊王,使為漢而稱東藩。」上曰:「善。」

乃從其畫,復守敖倉,而使酈生說齊王曰:「王知天下之所歸乎?」王曰:「不知也。」曰:「王知天下之所歸,則齊國可得而有也;若不知天下之所歸,即齊國未可得保也。」齊王曰:「天下何所歸?」曰:「歸漢。」曰:「先生何以言之?」曰:「漢王與項王力西面擊秦,約先入咸陽者王之。漢王先入咸陽,項王負約不與而王之漢中。項王遷殺義帝,漢王聞之,起蜀漢之兵擊三秦,出關而責義帝之處,收天下之兵,立諸侯之後。降城即以侯其將,得賂即以分其士,與天下同其利,豪英賢才皆樂為之用。諸侯之兵四面而至,蜀漢之粟方船而下。項王有倍約之名,殺義帝之負;於人之功無所記,於人之罪無所忘;戰勝而不得其賞,拔城而不得其封;非項氏莫得用事;為人刻印,刓而不能授;攻城得賂,積而不能賞:天下畔之,賢才怨之,而莫為之用。故天下之士歸於漢王,可坐而策也。夫漢王發蜀漢,定三秦;涉西河之外,援上黨之兵;下井陘,誅成安君;破北魏,舉三十二城:此蚩尤之兵也,非人之力也,天之福也。今已據敖倉之粟,塞成皋之險,守白馬之津,杜大行之阪,距蜚狐之口,天下後服者先亡矣。王疾先下漢王,齊國社稷可得而保也;不下漢王,危亡可立而待也。」田廣以為然,乃聽酈生,罷歷下兵守戰備,與酈生日縱酒。

淮陰侯聞酈生伏軾下齊七十餘城,乃夜度兵平原襲齊。齊王田廣聞漢兵至,以為酈生賣己,乃曰:「汝能止漢軍,我活汝;不然,我將亨汝!」酈生曰:「舉大事不細謹,盛德不辭讓。而公不為若更言!」齊王遂亨酈生,引兵東走。

漢十二年,曲周侯酈商以丞相將兵擊黥布有功。高祖舉列侯功臣,思酈食其。酈食其子疥數將兵,功未當侯,上以其父故,封疥為高梁侯。後更食武遂,嗣三世。元狩元年中,武遂侯平坐詐詔衡山王取百斤金,當棄市,病死,國除也。

陸賈者,楚人也。以客從高祖定天下,名為有口辯士,居左右,常使諸侯。

及高祖時,中國初定,尉他平南越,因王之。高祖使陸賈賜尉他印為南越王。陸生至,尉他魋結箕倨見陸生。陸生因進說他曰:「足下中國人,親戚昆弟墳在真定。今足下反天性,棄冠帶,欲以區區之越與天子抗衡為敵國,禍且及身矣。且夫秦失其政,諸侯豪桀并起,唯漢王先入關,據咸陽。項羽倍約,自立為西楚霸王,諸侯皆屬,可謂至彊。然漢王起巴蜀,鞭笞天下,劫略諸侯,遂誅項羽滅之。五年之閒,海內平定,此非人力,天之所建也。天子聞君王王南越,不助天下誅暴逆,將相欲移兵而誅王,天子憐百姓新勞苦,故且休之,遣臣授君王印,剖符通使。君王宜郊迎,北面稱臣,乃欲以新造未集之越,屈彊於此。漢誠聞之,掘燒王先人冢,夷滅宗族,使一偏將將十萬眾臨越,則越殺王降漢,如反覆手耳。」

於是尉他乃蹶然起坐,謝陸生曰:「居蠻夷中久,殊失禮義。」因問陸生曰:「我孰與蕭何、曹參、韓信賢?」陸生曰:「王似賢。」復曰:「我孰與皇帝賢?」陸生曰:「皇帝起豐沛,討暴秦,誅彊楚,為天下興利除害,繼五帝三王之業,統理中國。中國之人以億計,地方萬里,居天下之膏腴,人眾車轝,萬物殷富,政由一家,自天地剖泮未始有也。今王眾不過數十萬,皆蠻夷,崎嶇山海閒,譬若漢一郡,王何乃比於漢!」尉他大笑曰:「吾不起中國,故王此。使我居中國,何渠不若漢?」乃大說陸生,留與飲數月。曰:「越中無足與語,至生來,令我日聞所不聞。」賜陸生橐中裝直千金,他送亦千金。陸生卒拜尉他為南越王,令稱臣奉漢約。歸報,高祖大悅,拜賈為太中大夫。

陸生時時前說稱詩書。高帝罵之曰:「乃公居馬上而得之,安事詩書!」陸生曰;「居馬上得之,寧可以馬上治之乎?且湯武逆取而以順守之,文武并用,長久之術也。昔者吳王夫差、智伯極武而亡;秦任刑法不變,卒滅趙氏。鄉使秦已并天下,行仁義,法先聖,陛下安得而有之?」高帝不懌而有慚色,乃謂陸生曰:「試為我著秦所以失天下,吾所以得之者何,及古成敗之國。」陸生乃粗述存亡之徵,凡著十二篇。每奏一篇,高帝未嘗不稱善,左右呼萬歲,號其書曰「新語」。

孝惠帝時,呂太后用事,欲王諸呂,畏大臣有口者,陸生自度不能爭之,乃病免家居。以好畤田地善,可以家焉。有五男,乃出所使越得橐中裝賣千金,分其子,子二百金,令為生產。陸生常安車駟馬,從歌舞鼓琴瑟侍者十人,寶劍直百金,謂其子曰:「與汝約:過汝,汝給吾人馬酒食,極欲,十日而更。所死家,得寶劍車騎侍從者。一歲中往來過他客,率不過再三過,數見不鮮,無久慁公為也。」

呂太后時,王諸呂,諸呂擅權,欲劫少主,危劉氏。右丞相陳平患之,力不能爭,恐禍及己,常燕居深念。陸生往請,直入坐,而陳丞相方深念,不時見陸生。陸生曰:「何念之深也?」陳平曰:「生揣我何念?」陸生曰:「足下位為上相,食三萬戶侯,可謂極富貴無欲矣。然有憂念,不過患諸呂、少主耳。」陳平曰:「然。為之柰何?」陸生曰:「天下安,注意相;天下危,注意將。將相和調,則士務附;士務附,天下雖有變,即權不分。為社稷計,在兩君掌握耳。臣常欲謂太尉絳侯,絳侯與我戲,易吾言。君何不交驩太尉,深相結?」為陳平畫呂氏數事。陳平用其計,乃以五百金為絳侯壽,厚具樂飲;太尉亦報如之。此兩人深相結,則呂氏謀益衰。陳平乃以奴婢百人,車馬五十乘,錢五百萬,遺陸生為飲食費。陸生以此游漢廷公卿閒,名聲藉甚。

及誅諸呂,立孝文帝,陸生頗有力焉。孝文帝即位,欲使人之南越。陳丞相等乃言陸生為太中大夫,往使尉他,令尉他去黃屋稱制,令比諸侯,皆如意旨。語在南越語中。陸生竟以壽終。

平原君朱建者,楚人也。故嘗為淮南王黥布相,有罪去,後復事黥布。布欲反時,問平原君,平原君非之,布不聽而聽梁父侯,遂反。漢已誅布,聞平原君諫不與謀,得不誅。語在黥布語中。

平原君為人辯有口,刻廉剛直,家於長安。行不茍合,義不取容。辟陽侯行不正,得幸呂太后。時辟陽侯欲知平原君,平原君不肯見。及平原君母死,陸生素與平原君善,過之。平原君家貧,未有以發喪,方假貸服具,陸生令平原君發喪。陸生往見辟陽侯,賀曰:「平原君母死。」辟陽侯曰:「平原君母死,何乃賀我乎?」陸賈曰:「前日君侯欲知平原君,平原君義不知君,以其母故。今其母死,君誠厚送喪,則彼為君死矣。」辟陽侯乃奉百金往稅。列侯貴人以辟陽侯故,往稅凡五百金。

辟陽侯幸呂太后,人或毀辟陽侯於孝惠帝,孝惠帝大怒,下吏,欲誅之。呂太后慚,不可以言。大臣多害辟陽侯行,欲遂誅之。辟陽侯急,因使人欲見平原君。平原君辭曰:「獄急,不敢見君。」乃求見孝惠幸臣閎籍孺,說之曰:「君所以得幸帝,天下莫不聞。今辟陽侯幸太后而下吏,道路皆言君讒,欲殺之。今日辟陽侯誅,旦日太后含怒,亦誅君。何不肉袒為辟陽侯言於帝?帝聽君出辟陽侯,太后大驩。兩主共幸君,君貴富益倍矣。」於是閎籍孺大恐,從其計,言帝,果出辟陽侯。辟陽侯之囚,欲見平原君,平原君不見辟陽侯,辟陽侯以為倍己,大怒。及其成功出之,乃大驚。

呂太后崩,大臣誅諸呂,辟陽侯於諸呂至深,而卒不誅。計畫所以全者,皆陸生、平原君之力也。

孝文帝時,淮南厲王殺辟陽侯,以諸呂故。文帝聞其客平原君為計策,使吏捕欲治。聞吏至門,平原君欲自殺。諸子及吏皆曰:「事未可知,何早自殺為?」平原君曰:「我死禍絕,不及而身矣。」遂自剄。孝文帝聞而惜之,曰:「吾無意殺之。」乃召其子,拜為中大夫。使匈奴,單于無禮,乃罵單于,遂死匈奴中。

初,沛公引兵過陳留,酈生踵軍門上謁曰:「高陽賤民酈食其,竊聞沛公暴露,將兵助楚討不義,敬勞從者,願得望見,口畫天下便事。」使者入通,沛公方洗,問使者曰:「何如人也?」使者對曰:「狀貌類大儒,衣儒衣,冠側注。」沛公曰:「為我謝之,言我方以天下為事,未暇見儒人也。」使者出謝曰:「沛公敬謝先生,方以天下為事,未暇見儒人也。」酈生瞋目案劍叱使者曰:「走!復入言沛公,吾高陽酒徒也,非儒人也。」使者懼而失謁,跪拾謁,還走,復入報曰:「客,天下壯士也,叱臣,臣恐,至失謁。曰『走!復入言,而公高陽酒徒也』。」沛公遽雪足杖矛曰:「延客入!」

酈生入,揖沛公曰:「足下甚苦,暴衣露冠,將兵助楚討不義,足不何不自喜也?臣願以事見,而曰『吾方以天下為事,未暇見儒人也』。夫足下欲興天下之大事而成天下之大功,而以目皮相,恐失天下之能士。且吾度足下之智不如吾,勇又不如吾。若欲就天下而不相見,竊為足下失之。」沛公謝曰:「鄉者聞先生之容,今見先生之意矣。」乃延而坐之,問所以取天下者。酈生曰:「夫足下欲成大功,不如止陳留。陳留者,天下之據衝也,兵之會地也,積粟數千萬石,城守甚堅。臣素善其令,願為足下說之。不聽臣,臣請為足下殺之,而下陳留。足下將陳留之眾,據陳留之城,而食其積粟,招天下之從兵;從兵已成,足下橫行天下,莫能有害足下者矣。」沛公曰:「敬聞命矣。」

於是酈生乃夜見陳留令,說之曰:「夫秦為無道而天下畔之,今足下與天下從則可以成大功。今獨為亡秦嬰城而堅守,臣竊為足下危之。」陳留令曰:「秦法至重也,不可以妄言,妄言者無類,吾不可以應。先生所以教臣者,非臣之意也,願勿復道。」酈生留宿臥,夜半時斬陳留令首,踰城而下報沛公。沛公引兵攻城,縣令首於長竿以示城上人,曰:「趣下,而令頭已斷矣!今後下者必先斬之!」於是陳留人見令已死,遂相率而下沛公。沛公舍陳留南城門上,因其庫兵,食積粟,留出入三月,從兵以萬數,遂入破秦。

太史公曰:世之傳酈生書,多曰漢王已拔三秦,東擊項籍而引軍於鞏洛之閒,酈生被儒衣往說漢王。乃非也。自沛公未入關,與項羽別而至高陽,得酈生兄弟。余讀陸生新語書十二篇,固當世之辯士。至平原君子與余善,是以得具論之。


\end{pinyinscope}