\article{魏豹彭越列傳}

\begin{pinyinscope}
魏豹者,故魏諸公子也。其兄魏咎,故魏時封為寧陵君。秦滅魏,遷咎為家人。陳勝之起王也,咎往從之。陳王使魏人周市徇魏地,魏地已下,欲相與立周市為魏王。周市曰:「天下昏亂,忠臣乃見。今天下共畔秦,其義必立魏王後乃可。」齊、趙使車各五十乘,立周市為魏王。市辭不受,迎魏咎於陳。五反,陳王乃遣立咎為魏王。

章邯已破陳王,乃進兵擊魏王於臨濟。魏王乃使周市出請救於齊、楚。齊、楚遣項它、田巴將兵隨市救魏。章邯遂擊破殺周市等軍,圍臨濟。咎為其民約降。約定,咎自燒殺。

魏豹亡走楚。楚懷王予魏豹數千人,復徇魏地。項羽已破秦,降章邯。豹下魏二十餘城,立豹為魏王。豹引精兵從項羽入關。漢元年,項羽封諸侯,欲有梁地,乃徙魏王豹於河東,都平陽,為西魏王。

漢王還定三秦,渡臨晉,魏王豹以國屬焉,遂從擊楚於彭城。漢敗,還至滎陽,豹請歸視親病,至國,即絕河津畔漢。漢王聞魏豹反,方東憂楚,未及擊,謂酈生曰:「緩頰往說魏豹,能下之,吾以萬戶封若。」酈生說豹。豹謝曰:「人生一世閒,如白駒過隙耳。今漢王慢而侮人,罵詈諸侯群臣如罵奴耳,非有上下禮節也,吾不忍復見也。」於是漢王遣韓信擊虜豹於河東,傳詣滎陽,以豹國為郡。漢王令豹守滎陽。楚圍之急,周苛遂殺魏豹。

彭越者,昌邑人也,字仲。常漁鉅野澤中,為群盜。陳勝、項梁之起,少年或謂越曰:「諸豪桀相立畔秦,仲可以來,亦效之。」彭越曰:「兩龍方鬬,且待之。」

居歲餘,澤閒少年相聚百餘人,往從彭越,曰:「請仲為長。」越謝曰:「臣不願與諸君。」少年彊請,乃許。與期旦日日出會,後期者斬。旦日日出,十餘人后,後者至日中。於是越謝曰:「臣老,諸君彊以為長。今期而多後,不可盡誅,誅最後者一人。」令校長斬之。皆笑曰:「何至是?請後不敢。」於是越乃引一人斬之,設壇祭,乃令徒屬。徒屬皆大驚,畏越,莫敢仰視。乃行略地,收諸侯散卒,得千餘人。

沛公之從碭北擊昌邑,彭越助之。昌邑未下,沛公引兵西。彭越亦將其眾居鉅野中,收魏散卒。項籍入關,王諸侯,還歸,彭越眾萬餘人毋所屬。漢元年秋,齊王田榮畔項王,(漢)乃使人賜彭越將軍印,使下濟陰以擊楚。楚命蕭公角將兵擊越,越大破楚軍。漢王二年春,與魏王豹及諸侯東擊楚,彭越將其兵三萬餘人歸漢於外黃。漢王曰:「彭將軍收魏地得十餘城,欲急立魏後。今西魏王豹亦魏王咎從弟也,真魏後。」乃拜彭越為魏相國,擅將其兵,略定梁地。

漢王之敗彭城解而西也,彭越皆復亡其所下城,獨將其兵北居河上。漢王三年,彭越常往來為漢游兵,擊楚,絕其後糧於梁地。漢四年冬,項王與漢王相距滎陽,彭越攻下睢陽、外黃十七城。項王聞之,乃使曹咎守成皋,自東收彭越所下城邑,皆復為楚。越將其兵北走穀城。漢五年秋,項王之南走陽夏,彭越復下昌邑旁二十餘城,得穀十餘萬斛,以給漢王食。

漢王敗,使使召彭越并力擊楚。越曰:「魏地初定,尚畏楚,未可去。」漢王追楚,為項籍所敗碧陵。乃謂留侯曰:「諸侯兵不從,為之柰何?」留侯曰:「齊王信之立,非君王之意,信亦不自堅。彭越本定梁地,功多,始君王以魏豹故,拜彭越為魏相國。今豹死毋後,且越亦欲王,而君王不蚤定。與此兩國約:即勝楚,睢陽以北至穀城,皆以王彭相國;從陳以東傅海,與齊王信。齊王信家在楚,此其意欲復得故邑。君王能出捐此地許二人,二人今可致;即不能,事未可知也。」於是漢王乃發使使彭越,如留侯策。使者至,彭越乃悉引兵會垓下,遂破楚。(五年)項籍已死。春,立彭越為梁王,都定陶。

六年,朝陳。九年,十年,皆來朝長安。

十年秋,陳豨反代地,高帝自往擊,至邯鄲,徵兵梁王。梁王稱病,使將將兵詣邯鄲。高帝怒,使人讓梁王。梁王恐,欲自往謝。其將扈輒曰:「王始不往,見讓而往,往則為禽矣。不如遂發兵反。」梁王不聽,稱病。梁王怒其太仆,欲斬之。太仆亡走漢,告梁王與扈輒謀反。於是上使使掩梁王,梁王不覺,捕梁王,囚之雒陽。有司治反形己具,請論如法。上赦以為庶人,傳處蜀青衣。西至鄭,逢呂后從長安來,欲之雒陽,道見彭王。彭王為呂后泣涕,自言無罪,願處故昌邑。呂后許諾,與俱東至雒陽。呂后白上曰:「彭王壯士,今徙之蜀,此自遺患,不如遂誅之。妾謹與俱來。」於是呂后乃令其舍人彭越復謀反。廷尉王恬開奏請族之。上乃可,遂夷越宗族,國除。

太史公曰:魏豹、彭越雖故賤,然已席卷千里,南面稱孤,喋血乘勝日有聞矣。懷畔逆之意,及敗,不死而虜囚,身被刑戮,何哉?中材已上且羞其行,況王者乎!彼無異故,智略絕人,獨患無身耳。得攝尺寸之柄,其雲蒸龍變,欲有所會其度,以故幽囚而不辭云。


\end{pinyinscope}