\article{梁孝王世家}

\begin{pinyinscope}
梁孝王武者,孝文皇帝子也,而與孝景帝同母。母,竇太后也。

孝文帝凡四男:長子曰太子,是為孝景帝;次子武;次子參;次子勝。孝文帝即位二年,以武為代王,以參為太原王,以勝為梁王。二歲,徙代王為淮陽王。以代盡與太原王,號曰代王。參立十七年,孝文後二年卒,謚為孝王。子登嗣立,是為代共王。立二十九年,元光二年卒。子義立,是為代王。十九年,漢廣關,以常山為限,而徙代王王清河。清河王徙以元鼎三年也。

初,武為淮陽王十年,而梁王勝卒,謚為梁懷王。懷王最少子,愛幸異於他子。其明年,徙淮陽王武為梁王。梁王之初王梁,孝文帝之十二年也。梁王自初王通歷已十一年矣。

梁王十四年,入朝。十七年,十八年,比年入朝,留,其明年,乃之國。二十一年,入朝。二十二年,孝文帝崩。二十四年,入朝。二十五年,復入朝。是時上未置太子也。上與梁王燕飲,嘗從容言曰:「千秋萬歲後傳於王。」王辭謝。雖知非至言,然心內喜。太后亦然。

其春,吳楚齊趙七國反。吳楚先擊梁棘壁,殺數萬人。梁孝王城守睢陽,而使韓安國、張羽等為大將軍,以距吳楚。吳楚以梁為限,不敢過而西,與太尉亞夫等相距三月。吳楚破,而梁所破殺虜略與漢中分。明年,漢立太子。其後梁最親,有功,又為大國,居天下膏腴地。地北界泰山,西至高陽,四十餘城,皆多大縣。

孝王,竇太后少子也,愛之,賞賜不可勝道。於是孝王筑東苑,方三百餘里。廣睢陽城七十里。大治宮室,為複道,自宮連屬於平臺三十餘里。得賜天子旌旗,出從千乘萬騎。東西馳獵,擬於天子。出言蹕,入言警。招延四方豪桀,自山以東游說之士。莫不畢至,齊人羊勝、公孫詭、鄒陽之屬。公孫詭多奇邪計,初見王,賜千金,官至中尉,梁號之曰公孫將軍,梁多作兵器弩弓矛數十萬,而府庫金錢且百巨萬,珠玉寶器多於京師。

二十九年十月,梁孝王入朝。景帝使使持節乘輿駟馬,迎梁王於關下。既朝,上疏因留,以太后親故。王入則侍景帝同輦,出則同車游獵,射禽獸上林中。梁之侍中、郎、謁者著籍引出入天子殿門,與漢宦官無異。

十一月,上廢栗太子,竇太后心欲以孝王為後嗣。大臣及袁盎等有所關說於景帝,竇太后義格,亦遂不復言以梁王為嗣事由此。以事秘,世莫知。乃辭歸國。

其夏四月,上立膠東王為太子。梁王怨袁盎及議臣,乃與羊勝、公孫詭之屬陰使人刺殺袁盎及他議臣十餘人。逐其賊,未得也。於是天子意梁王,逐賊,果梁使之。乃遣使冠蓋相望於道,覆按梁,捕公孫詭、羊勝。公孫詭、羊勝匿王後宮。使者責二千石急,梁相軒丘豹及內史韓安國進諫王,王乃令勝、詭皆自殺,出之。上由此怨望於梁王。梁王恐,乃使韓安國因長公主謝罪太后,然后得釋。

上怒稍解,因上書請朝。既至關,茅蘭說王,使乘布車,從兩騎入,匿於長公主園。漢使使迎王,王已入關,車騎盡居外,不知王處。太后泣曰:「帝殺吾子!」景帝憂恐。於是梁王伏斧質於闕下,謝罪,然後太后、景帝大喜,相泣,復如故。悉召王從官入關。然景帝益疏王,不同車輦矣。

三十五年冬,復朝。上疏欲留,上弗許。歸國,意忽忽不樂。北獵良山,有獻牛,足出背上,孝王惡之。六月中,病熱,六日卒,謚曰孝王。

孝王慈孝,每聞太后病,口不能食,居不安寢,常欲留長安侍太后。太后亦愛之。及聞梁王薨,竇太后哭極哀,不食,曰:「帝果殺吾子!」景帝哀懼,不知所為。與長公主計之,乃分梁為五國,盡立孝王男五人為王,女五人皆食湯沐邑。於是奏之太后,太后乃說,為帝加壹湌 。

梁孝王長子買為梁王,是為共王;子明為濟川王;子彭離為濟東王;子定為山陽王;子不識為濟陰王。

孝王未死時,財以巨萬計,不可勝數。及死,藏府餘黃金尚四十餘萬斤,他財物稱是。

梁共王三年,景帝崩。共王立七年卒,子襄立,是為平王。

梁平王襄十四年,母曰陳太后。共王母曰李太后。李太后,親平王之大母也。而平王之后姓任,曰任王后。任王后甚有寵於平王襄。初,孝王在時,有罍樽,直千金。孝王誡後世,善保罍樽,無得以與人。任王后聞而欲得罍樽。平王大母李太后曰:「先王有命,無得以罍樽與人。他物雖百巨萬,猶自恣也。」任王后絕欲得之。平王襄直使人開府取罍樽,賜任王后。李太后大怒,漢使者來,欲自言,平王襄及任王后遮止,閉門,李太后與爭門,措指,遂不得見漢使者。李太后亦私與食官長及郎中尹霸等士通亂,而王與任王后以此使人風止李太后,李太后內有淫行,亦已。後病薨。病時,任后未嘗請病;薨,又不持喪。

元朔中,睢陽人類犴反者,人有辱其父,而與淮陽太守客出同車。太守客出下車,類犴反殺其仇於車上而去。淮陽太守怒,以讓梁二千石。二千石以下求反甚急,執反親戚。反知國陰事,乃上變事,具告知王與大母爭樽狀。時丞相以下見知之,欲以傷梁長吏,其書聞天子。天子下吏驗問,有之。公卿請廢襄為庶人。天子曰:「李太后有淫行,而梁王襄無良師傅,故陷不義。」乃削梁八城,梟任王后首于市。梁餘尚有十城。襄立三十九年卒,謚為平王。子無傷立為梁王也。

濟川王明者,梁孝王子,以桓邑侯孝景中六年為濟川王。七歲,坐射殺其中尉,漢有司請誅,天子弗忍誅,廢明為庶人。遷房陵,地入于漢為郡。

濟東王彭離者,梁孝王子,以孝景中六年為濟東王。二十九年,彭離驕悍,無人君禮,昏暮私與其奴、亡命少年數十人行剽殺人,取財物以為好。所殺發覺者百餘人,國皆知之,莫敢夜行。所殺者子上書言。漢有司請誅,上不忍,廢以為庶人,遷上庸,地入于漢,為大河郡。

山陽哀王定者,梁孝王子,以孝景中六年為山陽王。九年卒,無子,國除,地入于漢,為山陽郡。

濟陰哀王不識者,梁孝王子,以孝景中六年為濟陰王。一歲卒,無子,國除,地入于漢,為濟陰郡。

太史公曰:梁孝王雖以親愛之故,王膏腴之地,然會漢家隆盛,百姓殷富,故能植其財貨,廣宮室,車服擬於天子。然亦僭矣。

褚先生曰:臣為郎時,聞之於宮殿中老郎吏好事者稱道之也。竊以為令梁孝王怨望,欲為不善者,事從中生。今太后,女主也,以愛少子故,欲令梁王為太子。大臣不時正言其不可狀,阿意治小,私說意以受賞賜,非忠臣也。齊如魏其侯竇嬰之正言也,何以有後禍?景帝與王燕見,侍太后飲,景帝曰:「千秋萬歲之後傳王。」太后喜說。竇嬰在前,據地言曰:「漢法之約,傳子適孫,今帝何以得傳弟,擅亂高帝約乎!」於是景帝默然無聲。太后意不說。

故成王與小弱弟立樹下,取一桐葉以與之,曰:「吾用封汝。」周公聞之,進見曰:「天王封弟,甚善。」成王曰:「吾直與戲耳。」周公曰:「人主無過舉,不當有戲言,言之必行之。」於是乃封小弟以應縣。是後成王沒齒不敢有戲言,言必行之。《孝經》曰:「非法不言,非道不行。」此聖人之法言也。今主上不宜出好言於梁王。梁王上有太后之重,驕蹇日久,數聞景帝好言,千秋萬世之後傳王,而實不行。

又諸侯王朝見天子,漢法凡當四見耳。始到,入小見;到正月朔旦,奉皮薦璧玉賀正月,法見;後三日,為王置酒,賜金錢財物;後二日,復入小見,辭去。凡留長安不過二十日。小見者,燕見於禁門內,飲於省中,非士人所得入也。今梁王西朝,因留,且半歲。入與人主同輦,出與同車。示風以大言而實不與,令出怨言,謀畔逆,乃隨而憂之,不亦遠乎!非大賢人,不知退讓。今漢之儀法,朝見賀正月者,常一王與四侯俱朝見,十餘歲一至。今梁王常比年入朝見,久留。鄙語曰「驕子不孝」,非惡言也。故諸侯王當為置良師傅,相忠言之士,如汲黯、韓長孺等,敢直言極諫,安得有患害!

蓋聞梁王西入朝,謁竇太后,燕見,與景帝俱侍坐於太后前,語言私說。太后謂帝曰:「吾聞殷道親親,周道尊尊,其義一也。安車大駕,用梁孝王為寄。」景帝跪席舉身曰:「諾。」罷酒出,帝召袁盎諸大臣通經術者曰:「太后言如是,何謂也?」皆對曰:「太后意欲立梁王為帝太子。」帝問其狀,袁盎等曰:「殷道親親者,立弟。周道尊尊者,立子。殷道質,質者法天,親其所親,故立弟。周道文,文者法地,尊者敬也,敬其本始,故立長子。周道,太子死,立適孫。殷道。太子死,立其弟。」帝曰:「於公何如?」皆對曰:「方今漢家法周,周道不得立弟,當立子。故春秋所以非宋宣公。宋宣公死,不立子而與弟。弟受國死,復反之與兄之子。弟之子爭之,以為我當代父後,即刺殺兄子。以故國亂,禍不絕。故春秋曰『君子大居正,宋之禍宣公為之』。臣請見太后白之。」袁盎等入見太后:「太后言欲立梁王,梁王即終,欲誰立?」太后曰:「吾復立帝子。」袁盎等以宋宣公不立正,生禍,禍亂後五世不絕,小不忍害大義狀報太后。太后乃解說,即使梁王歸就國。而梁王聞其義出於袁盎諸大臣所,怨望,使人來殺袁盎。袁盎顧之曰:「我所謂袁將軍者也,公得毋誤乎?」刺者曰:「是矣!」刺之,置其劍,劍著身。視其劍,新治。問長安中削厲工,工曰:「梁郎某子來治此劍。」以此知而發覺之,發使者捕逐之。獨梁王所欲殺大臣十餘人,文吏窮本之,謀反端頗見。太后不食,日夜泣不止。景帝甚憂之,問公卿大臣,大臣以為遣經術吏往治之,乃可解。於是遣田叔、呂季主往治之。此二人皆通經術,知大禮。來還,至霸昌廄,取火悉燒梁之反辭,但空手來對景帝。景帝曰:「何如?」對曰:「言梁王不知也。造為之者,獨其幸臣羊勝、公孫詭之屬為之耳。謹以伏誅死,梁王無恙也。」景帝喜說,曰:「急趨謁太后。」太后聞之,立起坐湌,氣平復。故曰,不通經術知古今之大禮,不可以為三公及左右近臣。少見之人,如從管中闚天也。


\end{pinyinscope}