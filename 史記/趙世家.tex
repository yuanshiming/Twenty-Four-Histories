\article{趙世家}

\begin{pinyinscope}
趙氏之先,與秦共祖。至中衍,為帝大戊御。其後世蜚廉有子二人,而命其一子曰惡來,事紂,為周所殺,其後為秦。惡來弟曰季勝,其後為趙。

季勝生孟增。孟增幸於周成王,是為宅皋狼。皋狼生衡父,衡父生造父。造父幸於周繆王。造父取驥之乘匹,與桃林盜驪、驊騮、綠耳,獻之繆王。繆王使造父御,西巡狩,見西王母,樂之忘歸。而徐偃王反,繆王日馳千里馬,攻徐偃王,大破之。乃賜造父以趙城,由此為趙氏。

自造父已下六世至奄父,曰公仲,周宣王時伐戎,為御。及千畝戰,奄父脫宣王。奄父生叔帶。叔帶之時,周幽王無道,去周如晉,事晉文侯,始建趙氏于晉國。

自叔帶以下,趙宗益興,五世而(生)[至]趙夙。

趙夙,晉獻公之十六年伐霍、魏、耿,而趙夙為將伐霍。霍公求奔齊。晉大旱,卜之,曰「霍太山為祟」。使趙夙召霍君於齊,復之,以奉霍太山之祀,晉復穰。晉獻公賜趙夙耿。

夙生共孟,當魯閔公之元年也。共孟生趙衰,字子餘。

趙衰卜事晉獻公及諸公子,莫吉;卜事公子重耳,吉,即事重耳。重耳以驪姬之亂亡奔翟,趙衰從。翟伐廧咎如,得二女,翟以其少女妻重耳,長女妻趙衰而生盾。初,重耳在晉時,趙衰妻亦生趙同、趙括、趙嬰齊。趙衰從重耳出亡,凡十九年,得反國。重耳為晉文公,趙衰為原大夫,居原,任國政。文公所以反國及霸,多趙衰計策,語在晉事中。

趙衰既反晉,晉之妻固要迎翟妻,而以其子盾為適嗣,晉妻三子皆下事之。晉襄公之六年,而趙衰卒,謚為成季。

趙盾代成季任國政二年而晉襄公卒,太子夷皋年少。盾為國多難,欲立襄公弟雍。雍時在秦,使使迎之。太子母日夜啼泣,頓首謂趙盾曰:「先君何罪,釋其適子而更求君?」趙盾患之,恐其宗與大夫襲誅之,乃遂立太子,是為靈公,發兵距所迎襄公弟於秦者。靈公既立,趙盾益專國政。

靈公立十四年,益驕。趙盾驟諫,靈公弗聽。及食熊蹯,胹不熟,殺宰人,持其尸出,趙盾見之。靈公由此懼,欲殺盾。盾素仁愛人,嘗所食桑下餓人反捍救盾,盾以得亡。未出境,而趙穿弒靈公而立襄公弟黑臀,是為成公。趙盾復反,任國政。君子譏盾「為正卿,亡不出境,反不討賊」,故太史書曰「趙盾弒其君」。晉景公時而趙盾卒,謚為宣孟,子朔嗣。

趙朔,晉景公之三年,朔為晉將下軍救鄭,與楚莊王戰河上。朔娶晉成公姊為夫人。

晉景公之三年,大夫屠岸賈欲誅趙氏。初,趙盾在時,夢見叔帶持要而哭,甚悲;已而笑,拊手且歌。盾卜之,兆絕而後好。趙史援占之,曰:「此夢甚惡,非君之身,乃君之子,然亦君之咎。至孫,趙將世益衰。」屠岸賈者,始有寵於靈公,及至於景公而賈為司寇,將作難,乃治靈公之賊以致趙盾,遍告諸將曰:「盾雖不知,猶為賊首。以臣弒君,子孫在朝,何以懲罪?請誅之。」韓厥曰:「靈公遇賊,趙盾在外,吾先君以為無罪,故不誅。今諸君將誅其後,是非先君之意而今妄誅。妄誅謂之亂。臣有大事而君不聞,是無君也。」屠岸賈不聽。韓厥告趙朔趣亡。朔不肯,曰:「子必不絕趙祀,朔死不恨。」韓厥許諾,稱疾不出。賈不請而擅與諸將攻趙氏於下宮,殺趙朔、趙同、趙括、趙嬰齊,皆滅其族。

趙朔妻成公姊,有遺腹,走公宮匿。趙朔客曰公孫杵臼,杵臼謂朔友人程嬰曰:「胡不死?」程嬰曰:「朔之婦有遺腹,若幸而男,吾奉之;即女也,吾徐死耳。」居無何,而朔婦免身,生男。屠岸賈聞之,索於宮中。夫人置兒叱罅祝曰:「趙宗滅乎,若號;即不滅,若無聲。」及索,兒竟無聲。已脫,程嬰謂公孫杵臼曰:「今一索不得,後必且復索之,柰何?」公孫杵臼曰:「立孤與死孰難?」程嬰曰:「死易,立孤難耳。」公孫杵臼曰:「趙氏先君遇子厚,子彊為其難者,吾為其易者,請先死。」乃二人謀取他人嬰兒負之,衣以文葆,匿山中。程嬰出,謬謂諸將軍曰:「嬰不肖,不能立趙孤。誰能與我千金,吾告趙氏孤處。」諸將皆喜,許之,發師隨程嬰攻公孫杵臼。杵臼謬曰:「小人哉程嬰!昔下宮之難不能死,與我謀匿趙氏孤兒,今又賣我。縱不能立,而忍賣之乎!」抱兒呼曰:「天乎天乎!趙氏孤兒何罪?請活之,獨殺杵臼可也。」諸將不許,遂殺杵臼與孤兒。諸將以為趙氏孤兒良已死,皆喜。然趙氏真孤乃反在,程嬰卒與俱匿山中。

居十五年,晉景公疾,卜之,大業之後不遂者為祟。景公問韓厥,厥知趙孤在,乃曰:「大業之後在晉絕祀者,其趙氏乎?夫自中衍者皆嬴姓也。中衍人面鳥噣,降佐殷帝大戊,及周天子,皆有明德。下及幽厲無道,而叔帶去周適晉,事先君文侯,至于成公,世有立功,未嘗絕祀。今吾君獨滅趙宗,國人哀之,故見龜策。唯君圖之。」景公問:「趙尚有後子孫乎?」韓厥具以實告。於是景公乃與韓厥謀立趙孤兒,召而匿之宮中。諸將入問疾,景公因韓厥之眾以脅諸將而見趙孤。趙孤名曰武。諸將不得已,乃曰:「昔下宮之難,屠岸賈為之,矯以君命,并命群臣。非然,孰敢作難!微君之疾,群臣固且請立趙後。今君有命,群臣之願也。」於是召趙武、程嬰遍拜諸將,遂反與程嬰、趙武攻屠岸賈,滅其族。復與趙武田邑如故。

及趙武冠,為成人,程嬰乃辭諸大夫,謂趙武曰:「昔下宮之難,皆能死。我非不能死,我思立趙氏之後。今趙武既立,為成人,復故位,我將下報趙宣孟與公孫杵臼。」趙武啼泣頓首固請,曰:「武願苦筋骨以報子至死,而子忍去我死乎!」程嬰曰:「不可。彼以我為能成事,故先我死;今我不報,是以我事為不成。」遂自殺。趙武服齊衰三年,為之祭邑,春秋祠之,世世勿絕。

趙氏復位十一年,而晉厲公殺其大夫三郤。欒書畏及,乃遂弒其君厲公,更立襄公曾孫周,是為悼公。晉由此大夫稍彊。

趙武續趙宗二十七年,晉平公立。平公十二年,而趙武為正卿。十三年,吳延陵季子使於晉,曰:「晉國之政卒歸於趙武子、韓宣子、魏獻子之後矣。」趙武死,謚為文子。

文子生景叔。景叔之時,齊景公使晏嬰於晉,晏嬰與晉叔向語。嬰曰:「齊之政後卒歸田氏。」叔向亦曰:「晉國之政將歸六卿。六卿侈矣,而吾君不能恤也。」

趙景叔卒,生趙鞅,是為簡子。

趙簡子在位,晉頃公之九年,簡子將合諸侯戍于周。其明年,入周敬王于周,辟弟子朝之故也。

晉頃公之十二年,六卿以法誅公族祁氏、羊舌氏,分其邑為十縣,六卿各令其族為之大夫。晉公室由此益弱。

後十三年,魯賊臣陽虎來奔,趙簡子受賂,厚遇之。

趙簡子疾,五日不知人,大夫皆懼。醫扁鵲視之,出,董安于問。扁鵲曰:「血脈治也,而何怪!在昔秦繆公嘗如此,七日而寤。寤之日,告公孫支與子輿曰:『我之帝所甚樂。吾所以久者,適有學也。帝告我:「晉國將大亂,五世不安;其後將霸,未老而死;霸者之子且令而國男女無別。」』公孫支書而藏之,秦讖於是出矣。獻公之亂,文公之霸,而襄公敗秦師於殽而歸縱淫,此子之所聞。今主君之疾與之同,不出三日疾必閒,閒必有言也。」

居二日半,簡子寤。語大夫曰:「我之帝所甚樂,與百神游於鈞天,廣樂九奏萬舞,不類三代之樂,其聲動人心。有一熊欲來援我,帝命我射之,中熊,熊死。又有一羆來,我又射之,中羆,羆死。帝甚喜,賜我二笥,皆有副。吾見兒在帝側,帝屬我一翟犬,曰:『及而子之壯也,以賜之。』帝告我:『晉國且世衰,七世而亡,嬴姓將大敗周人於范魁之西,而亦不能有也。今余思虞舜之勳,適余將以其胄女孟姚配而七世之孫。』」董安于受言而書藏之。以扁鵲言告簡子,簡子賜扁鵲田四萬畝。

他日,簡子出,有人當道,辟之不去,從者怒,將刃之。當道者曰:「吾欲有謁於主君。」從者以聞。簡子召之,曰:「譆,吾有所見子晣也。」當道者曰:「屏左右,願有謁。」簡子屏人。當道者曰:「主君之疾,臣在帝側。」簡子曰:「然,有之。子之見我,我何為?」當道者曰:「帝令主君射熊與羆,皆死。」簡子曰:「是,且何也?」當道者曰:「晉國且有大難,主君首之。帝令主君滅二卿,夫熊與羆皆其祖也。」簡子曰:「帝賜我二笥皆有副,何也?」當道者曰:「主君之子將克二國於翟,皆子姓也。」簡子曰:「吾見兒在帝側,帝屬我一翟犬,曰『及而子之長以賜之』。夫兒何謂以賜翟犬?」當道者曰:「兒,主君之子也。翟犬者,代之先也。主君之子且必有代。及主君之後嗣,且有革政而胡服,并二國於翟。」簡子問其姓而延之以官。當道者曰:「臣野人,致帝命耳。」遂不見。簡子書藏之府。

異日,姑布子卿見簡子,簡子遍召諸子相之。子卿曰:「無為將軍者。」簡子曰:「趙氏其滅乎?」子卿曰:「吾嘗見一子於路,殆君之子也。」簡子召子毋卹。毋卹至,則子卿起曰:「此真將軍矣!」簡子曰:「此其母賤,翟婢也,奚道貴哉?」子卿曰:「天所授,雖賤必貴。」自是之後,簡子盡召諸子與語,毋卹最賢。簡子乃告諸子曰:「吾藏寶符於常山上,先得者賞。」諸子馳之常山上,求,無所得。毋卹還,曰:「已得符矣。」簡子曰:「奏之。」毋卹曰:「從常山上臨代,代可取也。」簡子於是知毋卹果賢,乃廢太子伯魯,而以毋卹為太子。

後二年,晉定公之十四年,范、中行作亂。明年春,簡子謂邯鄲大夫午曰:「歸我衛士五百家,吾將置之晉陽。」午許諾,歸而其父兄不聽,倍言。趙鞅捕午,囚之晉陽。乃告邯鄲人曰:「我私有誅午也,諸君欲誰立?」遂殺午。趙稷、涉賓以邯鄲反。晉君使籍秦圍邯鄲。荀寅、范吉射與午善,不肯助秦而謀作亂,董安于知之。十月,范、中行氏伐趙鞅,鞅奔晉陽,晉人圍之。范吉射、荀寅仇人魏襄等謀逐荀寅,以梁嬰父代之;逐吉射,以范皋繹代之。荀櫟言於晉侯曰:「君命大臣,始亂者死。今三臣始亂而獨逐鞅,用刑不均,請皆逐之。」十一月,荀櫟、韓不佞、魏哆奉公命以伐范、中行氏,不克。范、中行氏反伐公,公擊之,范、中行敗走。丁未,二子奔朝歌。韓、魏以趙氏為請。十二月辛未,趙鞅入絳,盟于公宮。其明年,知伯文子謂趙鞅曰:「范、中行雖信為亂,安于發之,是安于與謀也。晉國有法,始亂者死。夫二子已伏罪而安于獨在。」趙鞅患之。安于曰:「臣死,趙氏定,晉國寧,吾死晚矣。」遂自殺。趙氏以告知伯,然後趙氏寧。

孔子聞趙簡子不請晉君而執邯鄲午,保晉陽,故書春秋曰「趙鞅以晉陽畔」。趙簡子有臣曰周舍,好直諫。周舍死,簡子每聽朝,常不悅,大夫請罪。簡子曰:「大夫無罪。吾聞千羊之皮不如一狐之腋。諸大夫朝,徒聞唯唯,不聞周捨之鄂鄂,是以憂也。」簡子由此能附趙邑而懷晉人。

晉定公十八年,趙簡子圍范、中行于朝歌,中行文子奔邯鄲。明年,衛靈公卒。簡子與陽虎送衛太子蒯聵于衛,衛不內,居戚。

晉定公二十一年,簡子拔邯鄲,中行文子奔柏人。簡子又圍柏人,中行文子、范昭子遂奔齊。趙竟有邯鄲、柏人。范、中行餘邑入于晉。趙名晉卿,實專晉權,奉邑侔於諸侯。

晉定公三十年,定公與吳王夫差爭長於黃池,趙簡子從晉定公,卒長吳。定公三十七年卒,而簡子除三年之喪,期而已。是歲,越王句踐滅吳。

晉出公十一年,知伯伐鄭。趙簡子疾,使太子毋卹將而圍鄭。知伯醉,以酒灌擊毋卹。毋卹群臣請死之。毋卹曰:「君所以置毋卹,為能忍。」然亦慍知伯。知伯歸,因謂簡子,使廢毋卹,簡子不聽。毋卹由此怨知伯。

晉出公十七年,簡子卒,太子毋卹代立,是為襄子。

趙襄子元年,越圍吳。襄子降喪食,使楚隆問吳王。

襄子姊前為代王夫人。簡子既葬,未除服,北登夏屋,請代王。使廚人操銅枓以食代王及從者,行斟,陰令宰人各以枓擊殺代王及從官,遂興兵平代地。其姊聞之,泣而呼天,摩笄自殺。代人憐之,所死地名之為摩笄之山。遂以代封伯魯子周為代成君。伯魯者,襄子兄,故太子。太子蚤死,故封其子。

襄子立四年,知伯與趙、韓、魏盡分其范、中行故地。晉出公怒,告齊、魯,欲以伐四卿。四卿恐,遂共攻出公。出公奔齊,道死。知伯乃立昭公曾孫驕,是為晉懿公。知伯益驕。請地韓、魏,韓、魏與之。請地趙,趙不與,以其圍鄭之辱。知伯怒,遂率韓、魏攻趙。趙襄子懼,乃奔保晉陽。

原過從,後,至於王澤,見三人,自帶以上可見,自帶以下不可見。與原過竹二節,莫通。曰:「為我以是遺趙毋卹。」原過既至,以告襄子。襄子齊三日,親自剖竹,有朱書曰:「趙毋卹,余霍泰山山陽侯天使也。三月丙戌,余將使女反滅知氏。女亦立我百邑,余將賜女林胡之地。至于後世,且有伉王,赤黑,龍面而鳥噣,鬢麋髭髯,大膺大胸,修下而馮,左袵界乘,奄有河宗,至于休溷諸貉,南伐晉別,北滅黑姑。」襄子再拜,受三神之令。

三國攻晉陽,歲餘,引汾水灌其城,城不浸者三版。城中懸釜而炊,易子而食。群臣皆有外心,禮益慢,唯高共不敢失禮。襄子懼,乃夜使相張孟同私於韓、魏。韓、魏與合謀,以三月丙戌,三國反滅知氏,共分其地。於是襄子行賞,高共為上。張孟同曰:「晉陽之難,唯共無功。」襄子曰:「方晉陽急,群臣皆懈,惟共不敢失人臣禮,是以先之。」於是趙北有代,南并知氏,彊於韓、魏。遂祠三神於百邑,使原過主霍泰山祠祀。

其後娶空同氏,生五子。襄子為伯魯之不立也,不肯立子,且必欲傳位與伯魯子代成君。成君先死,乃取代成君子浣立為太子。襄子立三十三年卒,浣立,是為獻侯。

獻侯少即位,治中牟。

襄子弟桓子逐獻侯,自立於代,一年卒。國人曰桓子立非襄子意,乃共殺其子而復迎立獻侯。

十年,中山武公初立。十三年,城平邑。十五年,獻侯卒,子烈侯籍立。

烈侯元年,魏文侯伐中山,使太子擊守之。六年,魏、韓、趙皆相立為諸侯,追尊獻子為獻侯。

烈侯好音,謂相國公仲連曰:「寡人有愛,可以貴之乎?」公仲曰:「富之可,貴之則否。」烈侯曰:「然。夫鄭歌者槍、石二人,吾賜之田,人萬畝。」公仲曰:「諾。」不與。居一月,烈侯從代來,問歌者田。公仲曰:「求,未有可者。」有頃,烈侯復問。公仲終不與,乃稱疾不朝。番吾君自代來,謂公仲曰:「君實好善,而未知所持。今公仲相趙,於今四年,亦有進士乎?」公仲曰:「未也。」番吾君曰:「牛畜、荀欣、徐越皆可。」公仲乃進三人。及朝,烈侯復問:「歌者田何如?」公仲曰:「方使擇其善者。」牛畜侍烈侯以仁義,約以王道,烈侯逌然。明日,荀欣侍,以選練舉賢,任官使能。明日,徐越侍,以節財儉用,察度功德。所與無不充,君說。烈侯使使謂相國曰:「歌者之田且止。」官牛畜為師,荀欣為中尉,徐越為內史,賜相國衣二襲。

九年,烈侯卒,弟武公立。武公十三年卒,趙復立烈侯太子章,是為敬侯。是歲,魏文侯卒。

敬侯元年,武公子朝作亂,不克,出奔魏。趙始都邯鄲。

二年,敗齊于靈丘。三年,救魏于廩丘,大敗齊人。四年,魏敗我兔臺。筑剛平以侵衛。五年,齊、魏為衛攻趙,取我剛平。六年,借兵於楚伐魏,取棘蒲。八年,拔魏黃城。九年,伐齊。齊伐燕,趙救燕。十年,與中山戰于房子。

十一年,魏、韓、趙共滅晉,分其地。伐中山,又戰於中人。十二年,敬侯卒,子成侯種立。

成侯元年,公子勝與成侯爭立,為亂。二年六月,雨雪。三年,太戊午為相。伐衛,取鄉邑七十三。魏敗我藺。四年,與秦戰高安,敗之。五年,伐齊于鄄。魏敗我懷。攻鄭,敗之,以與韓,韓與我長子。六年,中山筑長城。伐魏,敗湪澤,圍魏惠王。七年,侵齊,至長城。與韓攻周。八年,與韓分周以為兩。九年,與齊戰阿下。十年,攻衛,取甄。十一年,秦攻魏,趙救之石阿。十二年,秦攻魏少梁,趙救之。十三年,秦獻公使庶長國伐魏少梁,虜其太子、痤。魏敗我澮,取皮牢。成侯與韓昭侯遇上黨。十四年,與韓攻秦。十五年,助魏攻齊。

十六年,與韓、魏分晉,封晉君以端氏。

十七年,成侯與魏惠王遇葛孽。十九年,與齊、宋會平陸,與燕會阿。二十年,魏獻榮椽,因以為檀臺。二十一年,魏圍我邯鄲。二十二年,魏惠王拔我邯鄲,齊亦敗魏於桂陵。二十四年,魏歸我邯鄲,與魏盟漳水上。秦攻我藺。二十五年,成侯卒。公子紲太子肅侯爭立,紲敗亡奔韓。

肅侯元年,奪晉君端氏,徙處屯留。二年,與魏惠王遇於陰晉。三年,公子范襲邯鄲,不勝而死。四年,朝天子。六年,攻齊,拔高唐。七年,公子刻攻魏首垣。十一年,秦孝公使商君伐魏,虜其將公子卬。趙伐魏。十二年,秦孝公卒,商君死。十五年,起壽陵。魏惠王卒。

十六年,肅侯游大陵,出於鹿門,大戊午扣馬曰:「耕事方急,一日不作,百日不食。」肅侯下車謝。

十七年,圍魏黃,不克。筑長城。

十八年,齊、魏伐我,我決河水灌之,兵去。二十二年,張儀相秦。趙疵與秦戰,敗,秦殺疵河西,取我藺、離石。二十三年,韓舉與齊、魏戰,死于桑丘。

二十四年,肅侯卒。秦、楚、燕、齊、魏出銳師各萬人來會葬。子武靈王立。

武靈王元年,陽文君趙豹相。梁襄王與太子嗣,韓宣王與太子倉來朝信宮。武靈王少,未能聽政,博聞師三人,左右司過三人。及聽政,先問先王貴臣肥義,加其秩;國三老年八十,月致其禮。

三年,城鄗。四年,與韓會于區鼠。五年,娶韓女為夫人。

八年,韓擊秦,不勝而去。五國相王,趙獨否,曰:「無其實,敢處其名乎!」令國人謂已曰「君」。

九年,與韓、魏共擊秦,秦敗我,斬首八萬級。齊敗我觀澤。十年,秦取我中都及西陽。齊破燕。燕相子之為君,君反為臣。十一年,王召公子職於韓,立以為燕王,使樂池送之。十三年,秦拔我藺,虜將軍趙莊。楚、魏王來,過邯鄲。十四年,趙何攻魏。

十六年,秦惠王卒。王遊大陵。他日,王夢見處女鼓琴而歌詩曰:「美人熒熒兮,顏若苕之榮。命乎命乎,曾無我嬴!」異日,王飲酒樂,數言所夢,想見其狀。吳廣聞之,因夫人而內其女娃嬴。孟姚也。孟姚甚有寵於王,是為惠后。

十七年,王出九門,為野臺,以望齊、中山之境。

十八年,秦武王與孟說舉龍文赤鼎,絕臏而死。趙王使代相趙固迎公子稷於燕,送歸,立為秦王,是為昭王。

十九年春正月,大朝信宮。召肥義與議天下,五日而畢。王北略中山之地,至於房子,遂之代,北至無窮,西至河,登黃華之上。召樓緩謀曰:「我先王因世之變,以長南藩之地,屬阻漳、滏之險,立長城,又取藺、郭狼,敗林人於荏,而功未遂。今中山在我腹心,北有燕,東有胡,西有林胡、樓煩、秦、韓之邊,而無彊兵之救,是亡社稷,柰何?夫有高世之名,必有遺俗之累。吾欲胡服。」樓緩曰:「善。」群臣皆不欲。

於是肥義侍,王曰:「簡、襄主之烈,計胡、翟之利。為人臣者,寵有孝弟長幼順明之節,通有補民益主之業,此兩者臣之分也。今吾欲繼襄主之跡,開於胡、翟之鄉,而卒世不見也。為敵弱,用力少而功多,可以毋盡百姓之勞,而序往古之勳。夫有高世之功者,負遺俗之累;有獨智之慮者,任驁民之怨。今吾將胡服騎射以教百姓,而世必議寡人,柰何?」肥義曰:「臣聞疑事無功,疑行無名。王既定負遺俗之慮,殆無顧天下之議矣。夫論至德者不和於俗,成大功者不謀於眾。昔者舜舞有苗,禹袒裸國,非以養欲而樂志也,務以論德而約功也。愚者闇成事,智者睹未形,則王何疑焉。」王曰:「吾不疑胡服也,吾恐天下笑我也。狂夫之樂,智者哀焉;愚者所笑,賢者察焉。世有順我者,胡服之功未可知也。雖驅世以笑我,胡地中山吾必有之。」於是遂胡服矣。

使王紲公子成曰:「寡人胡服,將以朝也,亦欲叔服之。家聽於親而國聽於君,古今之公行也。子不反親,臣不逆君,兄弟之通義也。今寡人作教易服而叔不服,吾恐天下議之也。制國有常,利民為本;從政有經,令行為上。明德先論於賤,而行政先信於貴。今胡服之意,非以養欲而樂志也;事有所止而功有所出,事成功立,然後善也。今寡人恐叔之逆從政之經,以輔叔之議。且寡人聞之,事利國者行無邪,因貴戚者名不累,故願慕公叔之義,以成胡服之功。使紲謁之叔,請服焉。」公子成再拜稽首曰:「臣固聞王之胡服也。臣不佞,寢疾,未能趨走以滋進也。王命之,臣敢對,因竭其愚忠。曰:臣聞中國者,蓋聰明徇智之所居也,萬物財用之所聚也,賢聖之所教也,仁義之所施也,詩書禮樂之所用也,異敏技能之所試也,遠方之所觀赴也,蠻夷之所義行也。今王捨此而襲遠方之服,變古之教,易古人道,逆人之心,而怫學者,離中國,故臣願王圖之也。」使者以報。王曰:「吾固聞叔之疾也,我將自往請之。」

王遂往之公子成家,因自請之,曰:「夫服者,所以便用也;禮者,所以便事也。聖人觀鄉而順宜,因事而制禮,所以利其民而厚其國也。夫翦發文身,錯臂左衽,甌越之民也。黑齒雕題,卻冠秫絀,大吳之國也。故禮服莫同,其便一也。鄉異而用變,事異而禮易。是以聖人果可以利其國,不一其用;果可以便其事,不同其禮。儒者一師而俗異,中國同禮而教離,況於山谷之便乎?故去就之變,智者不能一;遠近之服,賢聖不能同。窮鄉多異,曲學多辯。不知而不疑,異於己而不非者,公焉而眾求盡善也。今叔之所言者俗也,吾所言者所以制俗也。吾國東有河、薄洛之水,與齊、中山同之,無舟楫之用。自常山以至代、上黨,東有燕、東胡之境,而西有樓煩、秦、韓之邊,今無騎射之備。故寡人無舟楫之用,夾水居之民,將何以守河、薄洛之水;變服騎射,以備燕、三胡、秦、韓之邊。且昔者簡主不塞晉陽以及上黨,而襄主并戎取代以攘諸胡,此愚智所明也。先時中山負齊之彊兵,侵暴吾地,系累吾民,引水圍鄗,微社稷之神靈,則鄗幾於不守也。先王丑之,而怨未能報也。今騎射之備,近可以便上黨之形,而遠可以報中山之怨。而叔順中國之俗以逆簡、襄之意,惡變服之名以忘鄗事之丑,非寡人之所望也。」公字成再拜稽首曰:「臣愚,不達於王之義,敢道世俗之聞,臣之罪也。今王將繼簡、襄之意以順先王之志,臣敢不聽命乎!」再拜稽首。乃賜胡服。明日,服而朝。於是始出胡服令也。

趙文、趙造、周袑、趙俊皆諫止王毋胡服,如故法便。王曰:「先王不同俗,何古之法?帝王不相襲,何禮之循?虙戲、神農教而不誅,黃帝、堯、舜誅而不怒。及至三王,隨時制法,因事制禮。法度制令各順其宜,衣服器械各便其用。故禮也不必一道,而便國不必古。聖人之興也不相襲而王,夏、殷之衰也不易禮而滅。然則反古未可非,而循禮未足多也。且服奇者志淫,則是鄒、魯無奇行也;俗辟者民易,則是吳、越無秀士也。且聖人利身謂之服,便事謂之禮。夫進退之節,衣服之制者,所以齊常民也,非所以論賢者也。故齊民與俗流,賢者與變俱。故諺曰『以書御者不盡馬之情,以古制今者不達事之變』。循法之功,不足以高世;法古之學,不足以制今。子不及也。」遂胡服招騎射。

二十年,王略中山地,至寧葭;西略胡地,至榆中。林胡王獻馬。歸,使樓緩之秦,仇液之韓,王賁之楚,富丁之魏,趙爵之齊。代相趙固主胡,致其兵。

二十一年,攻中山。趙袑為右軍,許鈞為左軍,公子章為中軍,王并將之。牛翦將車騎,趙希并將胡、代。趙與之陘,合軍曲陽,攻取丹丘、華陽、鴟之塞。王軍取鄗、石邑、封龍、東垣。中山獻四邑和,王許之,罷兵。二十三年,攻中山。二十五年,惠后卒。使周袑胡服傅王子何。二十六年,復攻中山,攘地北至燕、代,西至雲中、九原。

二十七年五月戊申,大朝於東宮,傳國,立王子何以為王。王廟見禮畢,出臨朝。大夫悉為臣,肥義為相國,并傅王。是為惠文王。惠文王,惠后吳娃子也。武靈王自號為主父。

主父欲令子主治國,而身胡服將士大夫西北略胡地,而欲從雲中、九原直南襲秦,於是詐自為使者入秦。秦昭王不知,已而怪其狀甚偉,非人臣之度,使人逐之,而主父馳已脫關矣。審問之,乃主父也。秦人大驚。主父所以入秦者,欲自略地形,因觀秦王之為人也。

惠文王二年,主父行新地,遂出代,西遇樓煩王於西河而致其兵。

三年,滅中山,遷其王於膚施。起靈壽,北地方從,代道大通。還歸,行賞,大赦,置酒酺五日,封長子章為代安陽君。章素侈,心不服其弟所立。主父又使田不禮相章也。

李兌謂肥義曰:「公子章彊壯而志驕,黨眾而欲大,殆有私乎?田不禮之為人也,忍殺而驕。二人相得,必有謀陰賊起,一出身徼幸。夫小人有欲,輕慮淺謀,徒見其利而不顧其害,同類相推,俱入禍門。以吾觀之,必不久矣。子任重而勢大,亂之所始,禍之所集也,子必先患。仁者愛萬物而智者備禍於未形,不仁不智,何以為國?子奚不稱疾毋出,傳政於公子成?毋為怨府,毋為禍梯。」肥義曰:「不可,昔者主父以王屬義也,曰:『毋變而度,毋異而慮,堅守一心,以歿而世。』義再拜受命而籍之。今畏不禮之難而忘吾籍,變孰大焉。進受嚴命,退而不全,負孰甚焉。變負之臣,不容於刑。諺曰『死者復生,生者不愧』。吾言已在前矣,吾欲全吾言,安得全吾身!且夫貞臣也難至而節見,忠臣也累至而行明。子則有賜而忠我矣,雖然,吾有語在前者也,終不敢失。」李兌曰:「諾,子勉之矣!吾見子已今年耳。」涕泣而出。李兌數見公子成,以備田不禮之事。

異日肥義謂信期曰:「公子與田不禮甚可憂也。其於義也聲善而實惡,此為人也不子不臣。吾聞之也,姦臣在朝,國之殘也;讒臣在中,主之蠹也。此人貪而欲大,內得主而外為暴。矯令為慢,以擅一旦之命,不難為也,禍且逮國。今吾憂之,夜而忘寐,饑而忘食。盜賊出入不可不備。自今以來,若有召王者必見吾面,我將先以身當之,無故而王乃入。」信期曰:「善哉,吾得聞此也!」

四年,朝群臣,安陽君亦來朝。主父令王聽朝,而自從旁觀窺群臣宗室之禮。見其長子章傫然也,反北面為臣,詘於其弟,心憐之,於是乃欲分趙而王章於代,計未決而輟。

主父及王游沙丘,異宮,公子章即以其徒與田不禮作亂,詐以主父令召王。肥義先入,殺之。高信即與王戰。公子成與李兌自國至,乃起四邑之兵入距難,殺公子章及田不禮,滅其黨賊而定王室。公子成為相,號安平君,李兌為司寇。公子章之敗,往走主父,主主開之,成、兌因圍主父宮。公子章死,公子成、李兌謀曰:「以章故圍主父,即解兵,吾屬夷矣。」乃遂圍主父。令宮中人「後出者夷」,宮中人悉出。主父欲出不得,又不得食,探爵鷇而食之,三月餘而餓死沙丘宮。主父定死,乃發喪赴諸侯。

是時王少,成、兌專政,畏誅,故圍主父。主父初以長子章為太子,後得吳娃,愛之,為不出者數歲,生子何,乃廢太子章而立何為王。吳娃死,愛弛,憐故太子,欲兩王之,猶豫未決,故亂起,以至父子俱死,為天下笑,豈不痛乎!

(主父死惠文王立立)五年,與燕鄚、易。八年,城南行唐。九年,趙梁將,與齊合軍攻韓,至魯關下。及十年,秦自置為西帝。十一年,董叔與魏氏伐宋,得河陽於魏。秦取保陽。十二年,趙梁將攻齊。十三年,韓徐為將,攻齊。公主死。十四年,相國樂毅將趙、秦、韓、魏、燕攻齊,取靈丘。與秦會中陽。十五年,燕昭王來見。趙與韓、魏、秦共擊齊,齊王敗走,燕獨深入,取臨菑。

十六年,秦復與趙數擊齊,齊人患之。蘇厲為齊遺趙王書曰:

臣聞古之賢君,其德行非布於海內也,教順非洽於民人也,祭祀時享非數常於鬼神也。甘露降,時雨至,年穀豐孰,民不疾疫,眾人善之,然而賢主圖之。

今足下之賢行功力,非數加於秦也;怨毒積怒,非素深於齊也。秦趙與國,以彊征兵於韓,秦誠愛趙乎?其實憎齊乎?物之甚者,賢主察之。秦非愛趙而憎齊也,欲亡韓而吞二周,故以齊餤天下。恐事之不合,故出兵以劫魏、趙。恐天下畏己也,故出質以為信。恐天下亟反也,故徵兵於韓以威之。聲以德與國,實而伐空韓,臣以秦計為必出於此。夫物固有勢異而患同者,楚久伐而中山亡,今齊久伐而韓必亡。破齊,王與六國分其利也。亡韓,秦獨擅之。收二周,西取祭器,秦獨私之。賦田計功,王之獲利孰與秦多?

說士之計曰:「韓亡三川,魏亡晉國,市朝未變而禍已及矣。」燕盡齊之北地,去沙丘、鉅鹿斂三百里,韓之上黨去邯鄲百里,燕、秦謀王之河山,閒三百里而通矣。秦之上郡近挺關,至於榆中者千五百里,秦以三郡攻王之上黨,羊腸之西,句注之南,非王有已。踰句注,斬常山而守之,三百里而通於燕,代馬胡犬不東下,昆山之玉不出,此三寶者亦非王有已。王久伐齊,從彊秦攻韓,其禍必至於此。願王孰慮之。

且齊之所以伐者,以事王也;天下屬行,以謀王也。燕秦之約成而兵出有日矣。五國三分王之地,齊倍五國之約而殉王之患,西兵以禁彊秦,秦廢帝請服,反高平、根柔於魏,反坙分、先俞於趙。齊之事王,宜為上佼,而今乃抵罪,臣恐天下後事王者之不敢自必也。願王孰計之也。

今王毋與天下攻齊,天下必以王為義。齊抱社稷而厚事王,天下必盡重王義。王以天下善秦,秦暴,王以天下禁之,是一世之名寵制於王也。於是趙乃輟,謝秦不擊齊。

王與燕王遇。廉頗將,攻齊昔陽,取之。

十七年,樂毅將趙師攻魏伯陽。而秦怨趙不與己擊齊,伐趙,拔我兩城。十八年,秦拔我石城。王再之衛東陽,決河水,伐魏氏。大潦,漳水出。魏冉來相趙。十九年,秦(敗)[取]我二城。趙與魏伯陽。趙奢將,攻齊麥丘,取之。

二十年,廉頗將,攻齊。王與秦昭王遇西河外。

二十一年,趙徙漳水武平西。二十二年,大疫。置公子丹為太子。

二十三年,樓昌將,攻魏幾,不能取。十二月,廉頗將,攻幾,取之。二十四年,廉頗將,攻魏房子,拔之,因城而還。又攻安陽,取之。二十五年,燕周將,攻昌城、高唐,取之。與魏共擊秦。秦將白起破我華陽,得一將軍。二十六年,取東胡歐代地。

二十七年,徙漳水武平南。封趙豹為平陽君。河水出,大潦。

二十八年,藺相如伐齊,至平邑。罷城北九門大城。燕將成安君公孫操弒其王。二十九年,秦、韓相攻,而圍閼與。趙使趙奢將,擊秦,大破秦軍閼與下,賜號為馬服君。

三十三年,惠文王卒,太子丹立,是為孝成王。

孝成王元年,秦伐我,拔三城。趙王新立,太后用事,秦急攻之。趙氏求救於齊,齊曰:「必以長安君為質,兵乃出。」太后不肯,大臣彊諫。太后明謂左右曰:「復言長安君為質者,老婦必唾其面。」左師觸龍言願見太后,太后盛氣而胥之。入,徐趨而坐,自謝曰:「老臣病足,曾不能疾走,不得見久矣。竊自恕,而恐太后體之有所苦也,故願望見太后。」太后曰:「老婦恃輦而行耳。」曰:「食得毋衰乎?」曰:「恃粥耳。」曰:「老臣閒者殊不欲食,乃彊步,日三四里,少益嗜食,和於身也。」太后曰:「老婦不能。」太后不和之色少解。左師公曰:「老臣賤息舒祺最少,不肖,而臣衰,竊憐愛之,願得補黑衣之缺以衛王宮,昧死以聞。」太后曰:「敬諾。年幾何矣?」對曰:「十五歲矣。雖少,願及未填溝壑而託之。」太后曰:「丈夫亦愛憐少子乎?」對曰:「甚於婦人。」太后笑曰:「婦人異甚。」對曰:「老臣竊以為媼之愛燕后賢於長安君。」太后曰:「君過矣,不若長安君之甚。」左師公曰:「父母愛子則為之計深遠。媼之送燕后也,持其踵,為之泣,念其遠也,亦哀之矣。已行,非不思也,祭祀則祝之曰『必勿使反』,豈非計長久,為子孫相繼為王也哉?」太后曰:「然。」左師公曰:「今三世以前,至於趙主之子孫為侯者,其繼有在者乎?」曰:「無有。」曰:「微獨趙,諸侯有在者乎?」曰:「老婦不聞也。」曰:「此其近者禍及其身,遠者及其子孫。豈人主之子侯則不善哉?位尊而無功,奉厚而無勞,而挾重器多也。今媼尊長安君之位,而封之以膏腴之地,多與之重器,而不及今令有功於國,一旦山陵崩,長安君何以自託於趙?老臣以媼為長安君之計短也,故以為愛之不若燕后。」太后曰:「諾,恣君之所使之。」於是為長安君約車百乘,質於齊,齊兵乃出。

子義聞之,曰:「人主之子,骨肉之親也,猶不能持無功之尊,無勞之奉,而守金玉之重也,而況於予乎?」

齊安平君田單將趙師而攻燕中陽,拔之。又攻韓注人,拔之。二年,惠文后卒。田單為相。

四年,王夢衣偏裻之衣,乘飛龍上天,不至而墜,見金玉之積如山。明日,王召筮史敢占之,曰:「夢衣偏裻之衣者,殘也。乘飛龍上天不至而墜者,有氣而無實也。見金玉之積如山者,憂也。」

後三日,韓氏上黨守馮亭使者至,曰:「韓不能守上黨,入之於秦。其吏民皆安為趙,不欲為秦。有城市邑十七,願再拜入之趙,財王所以賜吏民。」王大喜,召平陽君豹告之曰:「馮亭入城市邑十七,受之何如?」對曰:「聖人甚禍無故之利。」王曰:「人懷吾德,何謂無故乎?」對曰:「夫秦蠶食韓氏地,中絕不令相通,固自以為坐而受上黨之地也。韓氏所以不入於秦者,欲嫁其禍於趙也。秦服其勞而趙受其利,雖彊大不能得之於小弱,小弱顧能得之於彊大乎?豈可謂非無故之利哉!且夫秦以牛田之水通糧蠶食,上乘倍戰者,裂上國之地,其政行,不可與為難,必勿受也。」王曰:「今發百萬之軍而攻,踰年歷歲未得一城也。今以城市邑十七幣吾國,此大利也。」

趙豹出,王召平原君與趙禹而告之。對曰:「發百萬之軍而攻,踰歲未得一城,今坐受城市邑十七,此大利,不可失也。」王曰:「善。」乃令趙勝受地,告馮亭曰:「敝國使者臣勝,敝國君使勝致命,以萬戶都三封太守,千戶都三封縣令,皆世世為侯,吏民皆益爵三級,吏民能相安,皆賜之六金。」馮亭垂涕不見使者,曰:「吾不處三不義也:為主守地,不能死固,不義一矣;入之秦,不聽主令,不義二矣;賣主地而食之,不義三矣。」趙遂發兵取上黨。廉頗將軍軍長平。

七(年)[月],廉頗免而趙括代將。秦人圍趙括,趙括以軍降,卒四十餘萬皆阬之。王悔不聽趙豹之計,故有長平之禍焉。

王還,不聽秦,秦圍邯鄲。武垣令傅豹、王容、蘇射率燕眾反燕地。趙以靈丘封楚相春申君。

八年,平原君如楚請救。還,楚來救,及魏公子無忌亦來救,秦圍邯鄲乃解。

十年,燕攻昌壯,五月拔之。趙將樂乘、慶舍攻秦信梁軍,破之。太子死。而秦攻西周,拔之。徒父祺出。十一年,城元氏,縣上原。武陽君鄭安平死,收其地。十二年,邯鄲廥燒。十四年,平原君趙勝死。

十五年,以尉文封相國廉頗為信平君。燕王令丞相栗腹約驤,以五百金為趙王酒,還歸,報燕王曰:「趙氏壯者皆死長平,其孤未壯,可伐也。」王召昌國君樂閒而問之。對曰:「趙,四戰之國也,其民習兵,伐之不可。」王曰:「吾以眾伐寡,二而伐一,可乎?」對曰:「不可。」王曰:「吾即以五而伐一,可乎?」對曰:「不可。」燕王大怒。群臣皆以為可。燕卒起二軍,車二千乘,栗腹將而攻鄗,卿秦將而攻代。廉頗為趙將,破殺栗腹,虜卿秦、樂閒。

十六年,廉頗圍燕。以樂乘為武襄君。十七年,假相大將武襄君攻燕,圍其國。十八年,延陵鈞率師從相國信平君助魏攻燕。秦拔我榆次三十七城。十九年,趙與燕易土:以龍兌、汾門、臨樂與燕;燕以葛、武陽、平舒與趙。

二十年,秦王政初立。秦拔我晉陽。

二十一年,孝成王卒。廉頗將,攻繁陽,取之。使樂乘代之,廉頗攻樂乘,樂乘走,廉頗亡入魏。子偃立,是為悼襄王。

悼襄王元年,大備魏。欲通平邑、中牟之道,不成。

二年,李牧將,攻燕,拔武遂、方城。秦召春平君,因而留之。泄鈞為之謂文信侯曰:「春平君者,趙王甚愛之而郎中之,故相與謀曰『春平君入秦,秦必留之』,故相與謀而內之秦也。今君留之,是絕趙而郎中之計中也。君不如遣春平君而留平都。春平君者言行信於王,王必厚割趙而贖平都。」文信侯曰:「善。」因遣之。城韓皋。

三年,龐煖將,攻燕,禽其將劇辛。四年,龐煖將趙、楚、魏、燕之銳師,攻秦蕞,不拔;移攻齊,取饒安。五年,傅抵將,居平邑;慶舍將東陽河外師,守河梁。六年,封長安君以饒。魏與趙鄴。

九年,趙攻燕,取貍、陽城。兵未罷,秦攻鄴,拔之。悼襄王卒,子幽繆王遷立。

幽繆王遷元年,城柏人。二年,秦攻武城,扈輒率師救之,軍敗,死焉。

三年,秦攻赤麗、宜安,李牧率師與戰肥下,卻之。封牧為武安君。四年,秦攻番吾,李牧與之戰,卻之。

五年,代地大動,自樂徐以西,北至平陰,臺屋墻垣太半壞,地坼東西百三十步。六年,大饑,民訛言曰:「趙為號,秦為笑。以為不信,視地之生毛。」

七年,秦人攻趙,趙大將李牧、將軍司馬尚將,擊之。李牧誅,司馬尚免,趙怱及齊將顏聚代之。趙怱軍破,顏聚亡去。以王遷降。

八年十月,邯鄲為秦。

太史公曰。吾聞馮王孫曰:「趙王遷,其母倡也,嬖於悼襄王。悼襄王廢適子嘉而立遷。遷素無行,信讒,故誅其良將李牧,用郭開。」豈不繆哉!秦既虜遷,趙之亡大夫共立嘉為王,王代六歲,秦進兵破嘉,遂滅趙以為郡。


\end{pinyinscope}