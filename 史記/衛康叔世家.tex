\article{衛康叔世家}

\begin{pinyinscope}
衛康叔名封,周武王同母少弟也。其次尚有冉季,冉季最少。

武王已克殷紂,復以殷餘民封紂子武庚祿父,比諸侯,以奉其先祀勿絕。為武庚未集,恐其有賊心,武王乃令其弟管叔、蔡叔傅相武庚祿父,以和其民。武王既崩,成王少。周公旦代成王治,當國。管叔、蔡叔疑周公,乃與武庚祿父作亂,欲攻成周。周公旦以成王命興師伐殷,殺武庚祿父、管叔,放蔡叔,以武庚殷餘民封康叔為衛君,居河、淇閒故商墟。

周公旦懼康叔齒少,乃申告康叔曰:「必求殷之賢人君子長者,問其先殷所以興,所以亡,而務愛民。」告以紂所以亡者以淫於酒,酒之失,婦人是用,故紂之亂自此始。為梓材,示君子可法則。故謂之《康誥》、《酒誥》、《梓材》以命之。康叔之國,既以此命,能和集其民,民大說。

成王長,用事,舉康叔為周司寇,賜衛寶祭器,以章有德。

康叔卒,子康伯代立。康伯卒,子考伯立。考伯卒,子嗣伯立。嗣伯卒,子偼伯立。偼伯卒,子靖伯立。靖伯卒,子貞伯立。貞伯卒,子頃侯立。

頃侯厚賂周夷王,夷王命衛為侯。頃侯立十二年卒,子釐侯立。

釐侯十三年,周厲王出奔于彘,共和行政焉。二十八年,周宣王立。

四十二年,釐侯卒,太子共伯餘立為君。共伯弟和有寵於釐侯,多予之賂;和以其賂賂士,以襲攻共伯於墓上,共伯入釐侯羨自殺。衛人因葬之釐侯旁,謚曰共伯,而立和為衛侯,是為武公。

武公即位,修康叔之政,百姓和集。四十二年,犬戎殺周幽王,武公將兵往佐周平戎,甚有功,周平王命武公為公。五十五年,卒,子莊公揚立。

莊公五年,取齊女為夫人,好而無子。又取陳女為夫人,生子,蚤死。陳女女弟亦幸於莊公,而生子完。完母死,莊公令夫人齊女子之,立為太子。莊公有寵妾,生子州吁。十八年,州吁長,好兵,莊公使將。石碏諫莊公曰:「庶子好兵,使將,亂自此起。」不聽。二十三年,莊公卒,太子完立,是為桓公。

桓公二年,弟州吁驕奢,桓公絀之,州吁出奔。十三年,鄭伯弟段攻其兄,不勝,亡,而州吁求與之友。十六年,州吁收聚衛亡人以襲殺桓公,州吁自立為衛君。為鄭伯弟段欲伐鄭,請宋、陳、蔡與俱,三國皆許州吁。州吁新立,好兵,弒桓公,衛人皆不愛。石碏乃因桓公母家於陳,詳為善州吁。至鄭郊,石碏與陳侯共謀,使右宰丑進食,因殺州吁于濮,而迎桓公弟晉於邢而立之,是為宣公。

宣公七年,魯弒其君隱公。九年,宋督弒其君殤公,及孔父。十年,晉曲沃莊伯弒其君哀侯。

十八年,初,宣公愛夫人夷姜,夷姜生子伋,以為太子,而令右公子傅之。右公子為太子取齊女,未入室,而宣公見所欲為太子婦者好,說而自取之,更為太子取他女。宣公得齊女,生子壽、子朔,令左公子傅之。太子伋母死,宣公正夫人與朔共讒惡太子伋。宣公自以其奪太子妻也,心惡太子,欲廢之。及聞其惡,大怒,乃使太子伋於齊而令盜遮界上殺之,與太子白旄,而告界盜見持白旄者殺之。且行,子朔之兄壽,太子異母弟也,知朔之惡太子而君欲殺之,乃謂太子曰:「界盜見太子白旄,即殺太子,太子可毋行。」太子曰:「逆父命求生,不可。」遂行。壽見太子不止,乃盜其白旄而先馳至界。界盜見其驗,即殺之。壽已死,而太子伋又至,謂盜曰:「所當殺乃我也。」盜并殺太子伋,以報宣公。宣公乃以子朔為太子。十九年,宣公卒,太子朔立,是為惠公。

左右公子不平朔之立也,惠公四年,左右公子怨惠公之讒殺前太子伋而代立,乃作亂,攻惠公,立太子伋之弟黔牟為君,惠公奔齊。

衛君黔牟立八年,齊襄公率諸侯奉王命共伐衛,納衛惠公,誅左右公子。衛君黔牟奔于周,惠公復立。惠公立三年出亡,亡八年復入,與前通年凡十三年矣。

二十五年,惠公怨周之容舍黔牟,與燕伐周。周惠王奔溫,衛、燕立惠王弟穨為王。二十九年,鄭復納惠王。三十一年,惠公卒,子懿公赤立。

懿公即位,好鶴,淫樂奢侈。九年,翟伐衛,衛懿公欲發兵,兵或畔。大臣言曰:「君好鶴,鶴可令擊翟。」翟於是遂入,殺懿公。

懿公之立也,百姓大臣皆不服。自懿公父惠公朔之讒殺太子伋代立至於懿公,常欲敗之,卒滅惠公之後而更立黔牟之弟昭伯頑之子申為君,是為戴公。

戴公申元年卒。齊桓公以衛數亂,乃率諸侯伐翟,為衛筑楚丘,立戴公弟燬為衛君,是為文公。文公以亂故奔齊,齊人入之。

初,翟殺懿公也,衛人憐之,思復立宣公前死太子伋之後,伋子又死,而代伋死者子壽又無子。太子伋同母弟二人:其一曰黔牟,黔牟嘗代惠公為君,八年復去;其二曰昭伯。昭伯、黔牟皆已前死,故立昭伯子申為戴公。戴公卒,復立其弟燬為文公。

文公初立,輕賦平罪,身自勞,與百姓同苦,以收衛民。

十六年,晉公子重耳過,無禮。十七年,齊桓公卒。二十五年,文公卒,子成公鄭立。

成公三年,晉欲假道於衛救宋,成公不許。晉更從南河度,救宋。徵師於衛,衛大夫欲許,成公不肯。大夫元咺攻成公,成公出奔。晉文公重耳伐衛,分其地予宋,討前過無禮及不救宋患也。衛成公遂出奔陳。二歲,如周求入,與晉文公會。晉使人鴆衛成公,成公私於周主鴆,令薄,得不死。已而周為請晉文公,卒入之衛,而誅元咺,衛君瑕出奔。七年,晉文公卒。十二年,成公朝晉襄公。十四年,秦穆公卒。二十六年,齊邴歜弒其君懿公。三十五年,成公卒,子穆公遫立。

穆公二年,楚莊王伐陳,殺夏徵舒。三年,楚莊王圍鄭,鄭降,復釋之。十一年,孫良夫救魯伐齊,復得侵地。穆公卒,子定公臧立。定公十二年卒,子獻公衎立。

獻公十三年,公令師曹教宮妾鼓琴,妾不善,曹笞之。妾以幸惡曹於公,公亦笞曹三百。十八年,獻公戒孫文子、甯惠子食,皆往。日旰不召,而去射鴻於囿。二子從之,公不釋射服與之言。二子怒,如宿。孫文子子數侍公飲,使師曹歌巧言之卒章。師曹又怒公之嘗笞三百,乃歌之,欲以怒孫文子,報衛獻公。文子語蘧伯玉,伯玉曰:「臣不知也。」遂攻出獻公。獻公奔齊,齊置衛獻公於聚邑。孫文子、甯惠子共立定公弟秋為衛君,是為殤公。

殤公秋立,封孫文子林父於宿。十二年,甯喜與孫林父爭寵相惡,殤公使甯喜攻孫林父。林父奔晉,復求入故衛獻公。獻公在齊,齊景公聞之,與衛獻公如晉求入。晉為伐衛,誘與盟。衛殤公會晉平公,平公執殤公與甯喜而復入衛獻公。獻公亡在外十二年而入。

獻公後元年,誅甯喜。

三年,吳延陵季子使過衛,見蘧伯玉、史,曰:「衛多君子,其國無故。」過宿,孫林父為擊磬,曰:「不樂,音大悲,使衛亂乃此矣。」是年,獻公卒,子襄公惡立。

襄公六年,楚靈王會諸侯,襄公稱病不往。

九年,襄公卒。初,襄公有賤妾,幸之,有身,夢有人謂曰:「我康叔也,令若子必有衛,名而子曰『元』。」妾怪之,問孔成子。成子曰:「康叔者,衛祖也。」及生子,男也,以告襄公。襄公曰:「天所置也。」名之曰元。襄公夫人無子,於是乃立元為嗣,是為靈公。

四十二年春,靈公游于郊,令子郢仆。郢,靈公少子也,字子南。靈公怨太子出奔,謂郢曰:「我將立若為後。」郢對曰:「郢不足以辱社稷,君更圖之。」夏,靈公卒,夫人命子郢為太子,曰:「此靈公命也。」郢曰:「亡人太子蒯聵之子輒在也,不敢當。」於是衛乃以輒為君,是為出公。

六月乙酉,趙簡子欲入蒯聵,乃令陽虎詐命衛十餘人衰绖歸,簡子送蒯聵。衛人聞之,發兵擊蒯聵。蒯聵不得入,入宿而保,衛人亦罷兵。

出公輒四年,齊田乞弒其君孺子。八年,齊鮑子弒其君悼公。

孔子自陳入衛。九年,孔文子問兵於仲尼,仲尼不對。其後魯迎仲尼,仲尼反魯。

十二年,初,孔圉文子取太子蒯聵之姊,生悝。孔氏之豎渾良夫美好,孔文子卒,良夫通於悝母。太子在宿,悝母使良夫於太子。太子與良夫言曰:「茍能入我國,報子以乘軒,免子三死,毋所與。」與之盟,許以悝母為妻。閏月,良夫與太子入,舍孔氏之外圃。昏,二人蒙衣而乘,宦者羅御,如孔氏。孔氏之老欒甯問之,稱姻妾以告。遂入,適伯姬氏。既食,悝母杖戈而先,太子與五人介,輿猳從之。伯姬劫悝於廁,彊盟之,遂劫以登臺。欒甯將飲酒,炙未熟,聞亂,使告仲由。召護駕乘車,行爵食炙,奉出公輒奔魯。

仲由將入,遇子羔將出,曰:「門已閉矣。」子路曰:「吾姑至矣。」子羔曰:「不及,莫踐其難。」子路曰:「食焉不辟其難。」子羔遂出。子路入,及門,公孫敢闔門,曰:「毋入為也!」子路曰:「是公孫也?求利而逃其難。由不然,利其祿,必救其患。」有使者出,子路乃得入。曰:「太子焉用孔悝?雖殺之,必或繼之。」且曰:「太子無勇。若燔臺,必舍孔叔。」太子聞之,懼,下石乞、盂黶敵子路,以戈擊之,割纓。子路曰:「君子死,冠不免。」結纓而死。孔子聞衛亂,曰:「嗟乎!柴也其來乎?由也其死矣。」孔悝竟立太子蒯聵,是為莊公。

莊公蒯聵者,出公父也,居外,怨大夫莫迎立。元年即位,欲盡誅大臣,曰:「寡人居外久矣,子亦嘗聞之乎?」群臣欲作亂,乃止。

二年,魯孔丘卒。

三年,莊公上城,見戎州。曰:「戎虜何為是?」戎州病之。十月,戎州告趙簡子,簡子圍衛。十一月,莊公出奔,衛人立公子斑師為衛君。齊伐衛,虜斑師,更立公子起為衛君。

衛君起元年,衛石曼尃逐其君起,起奔齊。衛出公輒自齊復歸立。初,出公立十二年亡,亡在外四年復入。出公後元年,賞從亡者。立二十一年卒,出公季父黔攻出公子而自立,是為悼公。

悼公五年卒,子敬公弗立。敬公十九年卒,子昭公糾立。是時三晉彊,衛如小侯,屬之。

昭公六年,公子亹弒之代立,是為懷公。懷公十一年,公子穨弒懷公而代立,是為慎公。慎公父,公子適;適父,敬公也。慎公四十二年卒,子聲公訓立。聲公十一年卒,子成侯遫立。

成侯十一年,公孫鞅入秦。十六年,衛更貶號曰侯。

二十九年,成侯卒,子平侯立。平侯八年卒,子嗣君立。

嗣君五年,更貶號曰君,獨有濮陽。

四十二年卒,子懷君立。懷君三十一年,朝魏,魏囚殺懷君。魏更立嗣君弟,是為元君。元君為魏婿,故魏立之。元君十四年,秦拔魏東地,秦初置東郡,更徙衛野王縣,而并濮陽為東郡。二十五年,元君卒,子君角立。君角九年,秦并天下,立為始皇帝。二十一年,二世廢君角為庶人,衛絕祀。

太史公曰:余讀世家言,至於宣公之太子以婦見誅,弟壽爭死以相讓,此與晉太子申生不敢明驪姬之過同,俱惡傷父之志。然卒死亡,何其悲也!或父子相殺,兄弟相滅,亦獨何哉?


\end{pinyinscope}