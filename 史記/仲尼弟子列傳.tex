\article{仲尼弟子列傳}

\begin{pinyinscope}
孔子曰「受業身通者七十有七人」,皆異能之士也。德行:顏淵,閔子騫,冉伯牛,仲弓。政事:冉有,季路。言語:宰我,子貢。文學:子游,子夏。師也辟,參也魯,柴也愚,由也喭,回也屢空。賜不受命而貨殖焉,億則屢中。

孔子之所嚴事:於周則老子;於衛,蘧伯玉;於齊,晏平仲;於楚,老萊子;於鄭,子產;於魯,孟公綽。數稱臧文仲、柳下惠、銅鞮伯華、介山子然,孔子皆後之,不并世。

顏回者,魯人也,字子淵。少孔子三十歲。

顏淵問仁,孔子曰:「克己復禮,天下歸仁焉。」

孔子曰:「賢哉回也!一簞食,一瓢飲,在陋巷,人不堪其憂,回也不改其樂。」「回也如愚;退而省其私,亦足以發,回也不愚。」「用之則行,捨之則藏,唯我與爾有是夫!」

回年二十九,發盡白,蚤死。孔子哭之慟,曰:「自吾有回,門人益親。」

魯哀公問:「弟子孰為好學?」孔子對曰:「有顏回者好學,不遷怒,不貳過。不幸短命死矣,今也則亡。」

閔損字子騫。少孔子十五歲。

孔子曰:「孝哉閔子騫!人不閒於其父母昆弟之言。」不仕大夫,不食汙君之祿。「如有復我者,必在汶上矣。」

冉耕字伯牛。孔子以為有德行。

伯牛有惡疾,孔子往問之,自牖執其手,曰:「命也夫!斯人也而有斯疾,命也夫!」

冉雍字仲弓。

仲弓問政,孔子曰:「出門如見大賓,使民如承大祭。在邦無怨,在家無怨。」

孔子以仲弓為有德行,曰:「雍也可使南面。」

仲弓父,賤人。孔子曰:「犁牛之子騂且角,雖欲勿用,山川其舍諸?」

冉求字子有,少孔子二十九歲。為季氏宰。

季康子問孔子曰:「冉求仁乎?」曰:「千室之邑,百乘之家,求也可使治其賦。仁則吾不知也。」復問:「子路仁乎?」孔子對曰:「如求。」

求問曰:「聞斯行諸?」子曰:「行之。」子路問:「聞斯行諸?」子曰:「有父兄在,如之何其聞斯行之!」子華怪之,「敢問問同而答異?」孔子曰:「求也退,故進之。由也兼人,故退之。」

仲由字子路,卞人也。少孔子九歲。

子路性鄙,好勇力,志伉直,冠雄雞,佩豭豚,陵暴孔子。孔子設禮稍誘子路,子路後儒服委質,因門人請為弟子。

子路問政,孔子曰:「先之,勞之。」請益。曰:「無倦。」

子路問:「君子尚勇乎?」孔子曰:「義之為上。君子好勇而無義則亂,小人好勇而無義則盜。」

子路有聞,未之能行,唯恐有聞。

孔子曰:「片言可以折獄者,其由也與!」「由也好勇過我,無所取材。」「若由也,不得其死然。」「衣敝縕袍,與衣狐貉者立,而不恥者,其由也與!」「由也升堂矣,未入於室也。」

季康子問:「仲由仁乎?」孔子曰:「千乘之國可使治其賦,不知其仁。」

子路喜從游,遇長沮、桀溺、荷蓧丈人。

子路為季氏宰,季孫問曰:「子路可謂大臣與?」孔子曰:「可謂具臣矣。」

子路為蒲大夫,辭孔子。孔子曰:「蒲多壯士,又難治。然吾語汝:恭以敬,可以執勇;寬以正,可以比眾;恭正以靜,可以報上。」

初,衛靈公有寵姬曰南子。靈公太子蕢聵得過南子,懼誅出奔。及靈公卒而夫人欲立公子郢。郢不肯,曰:「亡人太子之子輒在。」於是衛立輒為君,是為出公。出公立十二年,其父蕢聵居外,不得入。子路為衛大夫孔悝之邑宰。蕢聵乃與孔悝作亂,謀入孔悝家,遂與其徒襲攻出公。出公奔魯,而蕢聵入立,是為莊公。方孔悝作亂,子路在外,聞之而馳往。遇子羔出衛城門,謂子路曰:「出公去矣,而門已閉,子可還矣,毋空受其禍。」子路曰:「食其食者不避其難。」子羔卒去。有使者入城,城門開,子路隨而入。造蕢聵,蕢聵與孔悝登臺。子路曰:「君焉用孔悝?請得而殺之。」蕢聵弗聽。於是子路欲燔臺,蕢聵懼,乃下石乞、壺黶攻子路,擊斷子路之纓。子路曰:「君子死而冠不免。」遂結纓而死。

孔子聞衛亂,曰:「嗟乎,由死矣!」已而果死。故孔子曰:「自吾得由,惡言不聞於耳。」是時子貢為魯使於齊。

宰予字子我。利口辯辭。既受業,問:「三年之喪不已久乎?君子三年不為禮,禮必壞;三年不為樂,樂必崩。舊穀既沒,新穀既升,鉆燧改火,期可已矣。」子曰:「於汝安乎?」曰:「安。」「汝安則為之。君子居喪,食旨不甘,聞樂不樂,故弗為也。」宰我出,子曰:「予之不仁也!子生三年然後免於父母之懷。夫三年之喪,天下之通義也。」

宰予晝寢。子曰:「朽木不可雕也,糞土之墻不可圬也。」

宰我問五帝之德,子曰:「予非其人也。」

宰我為臨菑大夫,與田常作亂,以夷其族,孔子恥之。

端木賜,衛人,字子貢。少孔子三十一歲。

子貢利口巧辭,孔子常黜其辯。問曰:「汝與回也孰愈?」對曰:「賜也何敢望回!回也聞一以知十,賜也聞一以知二。」

子貢既已受業,問曰:「賜何人也?」孔子曰:「汝器也。」曰:「何器也?」曰:「瑚璉也。」

陳子禽問子貢曰:「仲尼焉學?」子貢曰:「文武之道未墜於地,在人,賢者識其大者,不賢者識其小者,莫不有文武之道。夫子焉不學,而亦何常師之有!」又問曰:「孔子適是國必聞其政。求之與?抑與之與?」子貢曰:「夫子溫、良、恭、儉、讓以得之。夫子之求之也,其諸異乎人之求之也。」

子貢問曰:「富而無驕,貧而無諂,何如?」孔子曰:「可也;不如貧而樂道,富而好禮。」

田常欲作亂於齊,憚高、國、鮑、晏,故移其兵欲以伐魯。孔子聞之,謂門弟子曰:「夫魯,墳墓所處,父母之國,國危如此,二三子何為莫出?」子路請出,孔子止之。子張、子石請行,孔子弗許。子貢請行,孔子許之。

遂行,至齊,說田常曰:「君之伐魯過矣。夫魯,難伐之國,其城薄以卑,其地狹以泄,其君愚而不仁,大臣偽而無用,其士民又惡甲兵之事,此不可與戰。君不如伐吳。夫吳,城高以厚,地廣以深,甲堅以新,士選以飽,重器精兵盡在其中,又使明大夫守之,此易伐也。」田常忿然作色曰:「子之所難,人之所易;子之所易,人之所難:而以教常,何也?」子貢曰:「臣聞之,憂在內者攻彊,憂在外者攻弱。今君憂在內。吾聞君三封而三不成者,大臣有不聽者也。今君破魯以廣齊,戰勝以驕主,破國以尊臣,而君之功不與焉,則交日疏於主。是君上驕主心,下恣群臣,求以成大事,難矣。夫上驕則恣,臣驕則爭,是君上與主有卻,下與大臣交爭也。如此,則君之立於齊危矣。故曰不如伐吳。伐吳不勝,民人外死,大臣內空,是君上無彊臣之敵,下無民人之過,孤主制齊者唯君也。」田常曰:「善。雖然,吾兵業已加魯矣,去而之吳,大臣疑我,柰何?」子貢曰:「君按兵無伐,臣請往使吳王,令之救魯而伐齊,君因以兵迎之。」田常許之,使子貢南見吳王。

說曰:「臣聞之,王者不絕世,霸者無彊敵,千鈞之重加銖兩而移。今以萬乘之齊而私千乘之魯,與吳爭彊,竊為王危之。且夫救魯,顯名也;伐齊,大利也。以撫泗上諸侯,誅暴齊以服彊晉,利莫大焉。名存亡魯,實困彊齊。智者不疑也。」吳王曰:「善。雖然,吾嘗與越戰,棲之會稽。越王苦身養士,有報我心。子待我伐越而聽子。」子貢曰:「越之勁不過魯,吳之彊不過齊,王置齊而伐越,則齊已平魯矣。且王方以存亡繼絕為名,夫伐小越而畏彊齊,非勇也。夫勇者不避難,仁者不窮約,智者不失時,王者不絕世,以立其義。今存越示諸侯以仁,救魯伐齊,威加晉國,諸侯必相率而朝吳,霸業成矣。且王必惡越,臣請東見越王,令出兵以從,此實空越,名從諸侯以伐也。」吳王大說,乃使子貢之越。

越王除道郊迎,身御至舍而問曰:「此蠻夷之國,大夫何以儼然辱而臨之?」子貢曰:「今者吾說吳王以救魯伐齊,其志欲之而畏越,曰『待我伐越乃可』。如此,破越必矣。且夫無報人之志而令人疑之,拙也;有報人之志,使人知之,殆也;事未發而先聞,危也。三者舉事之大患。」句踐頓首再拜曰:「孤嘗不料力,乃與吳戰,困於會稽,痛入於骨髓,日夜焦脣乾舌,徒欲與吳王接踵而死,孤之願也。」遂問子貢。子貢曰:「吳王為人猛暴,群臣不堪;國家敝以數戰,士卒弗忍;百姓怨上,大臣內變;子胥以諫死,太宰嚭用事,順君之過以安其私:是殘國之治也。今王誠發士卒佐之徼其志,重寶以說其心,卑辭以尊其禮,其伐齊必也。彼戰不勝,王之福矣。戰勝,必以兵臨晉,臣請北見晉君,令共攻之,弱吳必矣。其銳兵盡於齊,重甲困於晉,而王制其敝,此滅吳必矣。」越王大說,許諾。送子貢金百鎰,劍一,良矛二。子貢不受,遂行。

報吳王曰:「臣敬以大王之言告越王,越王大恐,曰:『孤不幸,少失先人,內不自量,抵罪於吳,軍敗身辱,棲于會稽,國為虛莽,賴大王之賜,使得奉俎豆而修祭祀,死不敢忘,何謀之敢慮!』」後五日,越使大夫種頓首言於吳王曰:「東海役臣孤句踐使者臣種,敢修下吏問於左右。今竊聞大王將興大義,誅彊救弱,困暴齊而撫周室,請悉起境內士卒三千人,孤請自被堅執銳,以先受矢石。因越賤臣種奉先人藏器,甲二十領,鈇屈盧之矛,步光之劍,以賀軍吏。」吳王大說,以告子貢曰:「越王欲身從寡人伐齊,可乎?」子貢曰:「不可。夫空人之國,悉人之眾,又從其君,不義。君受其幣,許其師,而辭其君。」吳王許諾,乃謝越王。於是吳王乃遂發九郡兵伐齊。

子貢因去之晉,謂晉君曰:「臣聞之,慮不先定不可以應卒,兵不先辨不可以勝敵。今夫齊與吳將戰,彼戰而不勝,越亂之必矣;與齊戰而勝,必以其兵臨晉。」晉君大恐,曰:「為之柰何?」子貢曰:「修兵休卒以待之。」晉君許諾。

子貢去而之魯。吳王果與齊人戰於艾陵,大破齊師,獲七將軍之兵而不歸,果以兵臨晉,與晉人相遇黃池之上。吳晉爭彊。晉人擊之,大敗吳師。越王聞之,涉江襲吳,去城七里而軍。吳王聞之,去晉而歸,與越戰於五湖。三戰不勝,城門不守,越遂圍王宮,殺夫差而戮其相。破吳三年,東向而霸。

故子貢一出,存魯,亂齊,破吳,彊晉而霸越。子貢一使,使勢相破,十年之中,五國各有變。

子貢好廢舉,與時轉貨貲。喜揚人之美,不能匿人之過。常相魯衛,家累千金,卒終于齊。

言偃,吳人,字子游。少孔子四十五歲。

子游既已受業,為武城宰。孔子過,聞弦歌之聲。孔子莞爾而笑曰:「割雞焉用牛刀?」子游曰:「昔者偃聞諸夫子曰,君子學道則愛人,小人學道則易使。」孔子曰:「二三子,偃之言是也。前言戲之耳。」孔子以為子游習於文學。

卜商字子夏。少孔子四十四歲。

子夏問:「『巧笑倩兮,美目盼兮,素以為絢兮』,何謂也?」子曰:「繪事後素。」曰:「禮後乎?」孔子曰:「商始可與言詩已矣。」

子貢問:「師與商孰賢?」子曰:「師也過,商也不及。」「然則師愈與?」曰:「過猶不及。」

子謂子夏曰:「汝為君子儒,無為小人儒。」

孔子既沒,子夏居西河教授,為魏文侯師。其子死,哭之失明。

顓孫師,陳人,字子張。少孔子四十八歲。

子張問干祿,孔子曰:「多聞闕疑,慎言其餘,則寡尤;多見闕殆,慎行其餘,則寡悔。言寡尤,行寡悔,祿在其中矣。」

他日從在陳蔡閒,困,問行。孔子曰:「言忠信,行篤敬,雖蠻貊之國行也;言不忠信,行不篤敬,雖州裏行乎哉!立則見其參於前也,在輿則見其倚於衡,夫然後行。」子張書諸紳。

子張問:「士何如斯可謂之達矣?」孔子曰:「何哉,爾所謂達者?」子張對曰:「在國必聞,在家必聞。」孔子曰:「是聞也,非達也。夫達者,質直而好義,察言而觀色,慮以下人,在國及家必達。夫聞也者,色取仁而行違,居之不疑,在國及家必聞。」

曾參,南武城人,字子輿。少孔子四十六歲。

孔子以為能通孝道,故授之業。作《孝經》。死於魯。

澹臺滅明,武城人,字子羽。少孔子三十九歲。

狀貌甚惡。欲事孔子,孔子以為材薄。既已受業,退而修行,行不由徑,非公事不見卿大夫。

南游至江,從弟子三百人,設取予去就,名施乎諸侯。孔子聞之,曰:「吾以言取人,失之宰予;以貌取人,失之子羽。」

宓不齊字子賤。少孔子三十歲。

孔子謂子賤,「君子哉!魯無君子,斯焉取斯?」

子賤為單父宰,反命於孔子,曰:「此國有賢不齊者五人,教不齊所以治者。」孔子曰:「惜哉不齊所治者小,所治者大則庶幾矣。」

原憲字子思。

子思問恥。孔子曰:「國有道,穀。國無道,穀,恥也。」

子思曰:「克、伐、怨、欲不行焉,可以為仁乎?」孔子曰:「可以為難矣,仁則吾弗知也。」

孔子卒,原憲遂亡在草澤中。子貢相衛,而結駟連騎,排藜藿入窮閻,過謝原憲。憲攝敝衣冠見子貢。子貢恥之,曰:「夫子豈病乎?」原憲曰:「吾聞之,無財者謂之貧,學道而不能行者謂之病。若憲,貧也,非病也。」子貢慚,不懌而去,終身恥其言之過也。

公冶長,齊人,字子長。

孔子曰:「長可妻也,雖在累紲之中,非其罪也。」以其子妻之。

南宮括字子容。

問孔子曰:「羿善射,奡盪舟,俱不得其死然;禹稷躬稼而有天下?」孔子弗答。容出,孔子曰:「君子哉若人!上德哉若人!」「國有道,不廢;國無道,免於刑戮。」三復「白珪之玷」,以其兄之子妻之。

公皙哀字季次。

孔子曰:「天下無行,多為家臣,仕於都;唯季次未嘗仕。」

曾蒧字皙。侍孔子,孔子曰:「言爾志。」蒧曰:「春服既成,冠者五六人,童子六七人,浴乎沂,風乎舞雩,詠而歸。」孔子喟爾嘆曰:「吾與蒧也!」

顏無繇字路。路者,顏回父,父子嘗各異時事孔子。

顏回死,顏路貧,請孔子車以葬。孔子曰:「材不材,亦各言其子也。鯉也死,有棺而無槨,吾不徒行以為之槨,以吾從大夫之後,不可以徒行。」

商瞿,魯人,字子木。少孔子二十九歲。

孔子傳易於瞿,瞿傳楚人馯臂子弘,弘傳江東人矯子庸疵,疵傳燕人周子家豎,豎傳淳于人光子乘羽,羽傳齊人田子莊何,何傳東武人王子中同,同傳菑川人楊何。何元朔中以治易為漢中大夫。

高柴字子羔。少孔子三十歲。

子羔長不盈五尺,受業孔子,孔子以為愚。

子路使子羔為費郈宰,孔子曰:「賊夫人之子!」子路曰:「有民人焉,有社稷焉,何必讀書然後為學!」孔子曰:「是故惡夫佞者。」

漆雕開字子開。

孔子使開仕,對曰:「吾斯之未能信。」孔子說。

公伯繚字子周。

周訴子路於季孫,子服景伯以告孔子,曰:「夫子固有惑志,繚也,吾力猶能肆諸市朝。」孔子曰:「道之將行,命也;道之將廢,命也。公伯繚其如命何!」

司馬耕字子牛。

牛多言而躁。問仁於孔子,孔子曰:「仁者其言也讱。」曰:「其言也讱,斯可謂之仁乎?」子曰:「為之難,言之得無讱乎!」

問君子,子曰:「君子不憂不懼。」曰:「不憂不懼,斯可謂之君子乎?」子曰:「內省不疚,夫何憂何懼!」

樊須字子遲。少孔子三十六歲。

樊遲請學稼,孔子曰:「吾不如老農。」請學圃,曰:「吾不如老圃。」樊遲出,孔子曰:「小人哉樊須也!上好禮,則民莫敢不敬;上好義,則民莫敢不服;上好信,則民莫敢不用情。夫如是,則四方之民襁負其子而至矣,焉用稼!」

樊遲問仁,子曰:「愛人。」問智,曰:「知人。」

有若少孔子四十三歲。有若曰:「禮之用,和為貴,先王之道斯為美。小大由之,有所不行;知和而和,不以禮節之,亦不可行也。」「信近於義,言可復也;恭近於禮,遠恥辱也;因不失其親,亦可宗也。」

孔子既沒,弟子思慕,有若狀似孔子,弟子相與共立為師,師之如夫子時也。他日,弟子進問曰:「昔夫子當行,使弟子持雨具,已而果雨。弟子問曰:『夫子何以知之?』夫子曰:『《詩》不云乎?「月離于畢,俾滂沱矣。」昨暮月不宿畢乎?』他日,月宿畢,竟不雨。商瞿年長無子,其母為取室。孔子使之齊,瞿母請之。孔子曰:『無憂,瞿年四十後當有五丈夫子。』已而果然。問夫子何以知此?」有若默然無以應。弟子起曰:「有子避之,此非子之座也!」

公西赤字子華。少孔子四十二歲。

子華使於齊,冉有為其母請粟。孔子曰:「與之釜。」請益,曰:「與之庾。」冉子與之粟五秉。孔子曰:「赤之適齊也,乘肥馬,衣輕裘。吾聞君子周急不繼富。」

巫馬施字子旗。少孔子三十歲。

陳司敗問孔子曰:「魯昭公知禮乎?」孔子曰:「知禮。」退而揖巫馬旗曰:「吾聞君子不黨,君子亦黨乎?魯君娶吳女為夫人,命之為孟子。孟子姓姬,諱稱同姓,故謂之孟子。魯君而知禮,孰不知禮!」施以告孔子,孔子曰:「丘也幸,茍有過,人必知之。臣不可言君親之惡,為諱者,禮也。」

梁鱣字叔魚。少孔子二十九歲。

顏幸字子柳。少孔子四十六歲。

冉孺字子魯,少孔子五十歲。

曹卹字子循。少孔子五十歲。

伯虔字子析,少孔子五十歲。

公孫龍字子石。少孔子五十三歲。

自子石已右三十五人,顯有年名及受業見于書傳。其四十有二人,無年及不見書傳者紀于左:

冉季字子產。

公祖句茲字子之。

秦祖字子南。

漆雕哆字子斂。

顏高字子驕。

漆雕徒父。

壤駟赤字子徒。

商澤。

石作蜀字子明。

任不齊字選。

公良孺字子正。

后處字子裏。

秦冉字開。

公夏首字乘。

奚容箴字子皙。

公肩定字子中。

顏祖字襄。

鄡單字子家。

句井疆。

罕父黑字子索。

秦商字子丕。

申黨字周。

顏之仆字叔。

榮旂字子祈。

縣成字子祺。

左人郢字行。

燕伋字思。

鄭國字子徒。

秦非字子之。

施之常字子恒。

顏噲字子聲。

步叔乘字子車。

原亢籍。

樂欬字子聲。

廉絜字庸。

叔仲會字子期。

顏何字冉。

狄黑字皙。

邦巽字子斂。

孔忠。

公西輿如字子上。

公西葴字子上。

太史公曰:學者多稱七十子之徒,譽者或過其實,毀者或損其真,鈞之未睹厥容貌,則論言弟子籍,出孔氏古文近是。余以弟子名姓文字悉取論語弟子問并次為篇,疑者闕焉。


\end{pinyinscope}