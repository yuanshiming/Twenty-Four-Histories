\article{龜策列傳}

\begin{pinyinscope}
太史公曰:自古聖王將建國受命,興動事業,何嘗不寶卜筮以助善!唐虞以上,不可記已。自三代之興,各據禎祥。涂山之兆從而夏啟世,飛燕之卜順故殷興,百穀之筮吉故周王。王者決定諸疑,參以卜筮,斷以蓍龜,不易之道也。

蠻夷氐羌雖無君臣之序,亦有決疑之卜。或以金石,或以草木,國不同俗。然皆可以戰伐攻擊,推兵求勝,各信其神,以知來事。

略聞夏殷欲卜者,乃取蓍龜,已則棄去之,以為龜藏則不靈,蓍久則不神。至周室之卜官,常寶藏蓍龜;又其大小先後,各有所尚,要其歸等耳。或以為聖王遭事無不定,決疑無不見,其設稽神求問之道者,以為後世衰微,愚不師智,人各自安,化分為百室,道散而無垠,故推歸之至微,要絜於精神也。或以為昆蟲之所長,聖人不能與爭。其處吉凶,別然否,多中於人。至高祖時,因秦太卜官。天下始定,兵革未息。及孝惠享國日少,呂后女主,孝文、孝景因襲掌故,未遑講試,雖父子疇官,世世相傳,其精微深妙,多所遺失。至今上即位,博開藝能之路,悉延百端之學,通一伎之士咸得自效,絕倫超奇者為右,無所阿私,數年之閒,太卜大集。會上欲擊匈奴,西攘大宛,南收百越,卜筮至預見表象,先圖其利。及猛將推鋒執節,獲勝於彼,而蓍龜時日亦有力於此。上尤加意,賞賜至或數千萬。如丘子明之屬,富溢貴寵,傾於朝廷。至以卜筮射蠱道,巫蠱時或頗中。素有啊〃幀×因公行誅,恣意所傷,以破族滅門者,不可勝數。百僚蕩恐,皆曰龜策能言。后事覺姦窮,亦誅三族。

夫摓策定數,灼龜觀兆,變化無窮,是以擇賢而用占焉,可謂聖人重事者乎!周公卜三龜,而武王有瘳。紂為暴虐,而元龜不占。晉文將定襄王之位,卜得黃帝之兆,卒受彤弓之命。獻公貪驪姬之色,卜而兆有口象,其禍竟流五世。楚靈將背周室,卜而龜逆,終被乾谿之敗。兆應信誠於內,而時人明察見之於外,可不謂兩合者哉!君子謂夫輕卜筮,無神明者,悖;背人道,信禎祥者,鬼神不得其正。故書建稽疑,五謀而卜筮居其二,五占從其多,明有而不專之道也。

余至江南,觀其行事,問其長老,云龜千歲乃遊蓮葉之上,蓍百莖共一根。又其所生,獸無虎狼,草無毒螫。江傍家人常畜龜飲食之,以為能導引致氣,有益於助衰養老,豈不信哉!

褚先生曰:臣以通經術,受業博士,治春秋,以高第為郎,幸得宿衛,出入宮殿中十有餘年。竊好太史公傳。太史公之傳曰:「三王不同龜,四夷各異卜,然各以決吉凶,略闚其要,故作龜策列傳。」臣往來長安中,求龜策列傳不能得,故之大卜官,問掌故文學長老習事者,寫取龜策卜事,編于下方。

聞古五帝、三王發動舉事,必先決蓍龜。傳曰:「下有伏靈,上有兔絲;上有擣蓍,下有神龜。」所謂伏靈者,在兔絲之下,狀似飛鳥之形。新雨已,天清靜無風,以夜捎兔絲去之,既以篝燭此地燭之,火滅,即記其處,以新布四丈環置之,明即掘取之,入四尺至七尺,得矣,過七尺不可得。伏靈者,千歲松根也,食之不死。聞蓍生滿百莖者,其下必有神龜守之,其上常有青雲覆之。傳曰:「天下和平,王道得,而蓍莖長丈,其叢生滿百莖。」方今世取蓍者,不能中古法度,不能得滿百莖長丈者,取八十莖已上,蓍長八尺,即難得也。人民好用卦者,取滿六十莖已上,長滿六尺者,既可用矣。記曰:「能得名龜者,財物歸之,家必大富至千萬。」一曰「北斗龜」,二曰「南辰龜」,三曰「五星龜」,四曰「八風龜」,五曰「二十八宿龜」,六曰「日月龜」,七曰「九州龜」,八曰「玉龜」:凡八名龜。龜圖各有文在腹下,文云云者,此某之龜也。略記其大指,不寫其圖。取此龜不必滿尺二寸,民人得長七八寸,可寶矣。今夫珠玉寶器,雖有所深藏,必見其光,必出其神明,其此之謂乎!笔玉處於山而木潤,淵生珠而岸不枯者,潤澤之所加也。明月之珠出於江海,藏於蚌中,蚗龍伏之。王者得之,長有天下,四夷賓服。能得百莖蓍,并得其下龜以卜者,百言百當,足以決吉凶。

神龜出於江水中,廬江郡常歲時生龜長尺二寸者二十枚輸太卜官,太卜官因以吉日剔取其腹下甲。龜千歲乃滿尺二寸。王者發軍行將,必鉆龜廟堂之上,以決吉凶。今高廟中有龜室,藏內以為神寶。

傳曰:「取前足臑骨穿佩之,取龜置室西北隅懸之,以入深山大林中,不惑。」臣為郎時,見萬畢石朱方,傳曰:「有神龜在江南嘉林中。嘉林者,獸無虎狼,鳥無鴟梟,草無毒螫,野火不及,斧斤不至,是為嘉林。龜在其中,常巢於芳蓮之上。左脅書文曰:『甲子重光,得我者匹夫為人君,有土正,諸侯得我為帝王。』求之於白蛇蟠杅林中者,齋戒以待,譺然,狀如有人來告之,因以醮酒佗發,求之三宿而得。」由是觀之,豈不偉哉!笔龜可不敬與?

南方老人用龜支床足,行二十餘歲,老人死,移床,龜尚生不死。龜能行氣導引。問者曰:「龜至神若此,然太卜官得生龜,何為輒殺取其甲乎?」近世江上人有得名龜,畜置之,家因大富。與人議,欲遣去。人教殺之勿遣,遣之破人家。龜見夢曰:「送我水中,無殺吾也。」其家終殺之。殺之后,身死,家不利。人民與君王者異道。人民得名龜,其狀類不宜殺也。以往古故事言之,古明王聖主皆殺而用之。

宋元王時得龜,亦殺而用之。謹連其事於左方,令好事者觀擇其中焉。

宋元王二年,江使神龜使於河,至於泉陽,漁者豫且舉網得而囚之。置之籠中。夜半,龜來見夢於宋元王曰:「我為江使於河,而幕網當吾路。泉陽豫且得我,我不能去。身在患中,莫可告語。王有德義,故來告訴。」元王惕然而悟。乃召博士衛平而問之曰:「今寡人夢見一丈夫,延頸而長頭,衣玄繡之衣而乘輜車,來見夢於寡人曰:『我為江使於河,而幕網當吾路。泉陽豫且得我,我不能去。身在患中,莫可告語。王有德義,故來告訴。』是何物也?」衛平乃援式而起,仰天而視月之光,觀斗所指,定日處鄉。規矩為輔,副以權衡。四維已定,八卦相望。視其吉凶,介蟲先見。乃對元王曰:「今昔壬子,宿在牽牛。河水大會,鬼神相謀。漢正南北,江河固期,南風新至,江使先來。白雲壅漢,萬物盡留。斗柄指日,使者當囚。玄服而乘輜車,其名為龜。王急使人問而求之。」王曰:「善。」

於是王乃使人馳而往問泉陽令曰:「漁者幾何家?名誰為豫且?豫且得龜,見夢於王,王故使我求之。」泉陽令乃使吏案籍視圖,水上漁者五十五家,上流之廬,名為豫且。泉陽令曰:「諾。」乃與使者馳而問豫且曰:「今昔汝漁何得?」豫且曰:「夜半時舉網得龜。」使者曰:「今龜安在?」曰:「在籠中。」使者曰:「王知子得龜,故使我求之。」豫且曰:「諾。」即系龜而出之籠中,獻使者。

使者載行,出於泉陽之門。正晝無見,風雨晦冥。雲蓋其上,五采青黃;雷雨并起,風將而行。入於端門,見於東箱。身如流水,潤澤有光。望見元王,延頸而前,三步而止,縮頸而卻,復其故處。元王見而怪之,問衛平曰:「龜見寡人,延頸而前,以何望也?縮頸而復,是何當也?」衛平對曰:「龜在患中,而終昔囚,王有德義,使人活之。今延頸而前,以當謝也,縮頸而卻,欲亟去也。」元王曰:「善哉!神至如此乎,不可久留;趣駕送龜,勿令失期。」衛平對曰:「龜者是天下之寶也,先得此龜者為天子,且十言十當,十戰十勝。生於深淵,長於黃土。知天之道,明於上古。游三千歲,不出其域。安平靜正,動不用力。壽蔽天地,莫知其極。與物變化,四時變色。居而自匿,伏而不食。春倉夏黃,秋白冬黑。明於陰陽,審於刑德。先知利害,察於禍福,以言而當,以戰而勝,王能寶之,諸侯盡服。王勿遣也,以安社稷。」元王曰:「龜甚神靈,降于上天,陷於深淵。在患難中。以我為賢。德厚而忠信,故來告寡人。寡人若不遣也,是漁者也。漁者利其肉,寡人貪其力,下為不仁,上為無德。君臣無禮,何從有福?寡人不忍,柰何勿遣!」衛平對曰:「不然。臣聞盛德不報,重寄不歸;天與不受,天奪之寶。今龜周流天下,還復其所,上至蒼天,下薄泥涂。還遍九州,未嘗愧辱,無所稽留。今至泉陽,漁者辱而囚之。王雖遣之,江河必怒,務求報仇。自以為侵,因神與謀。淫雨不霽,水不可治。若為枯旱,風而揚埃,蝗蟲暴生,百姓失時。王行仁義,其罰必來。此無佗故,其祟在龜。後雖悔之,豈有及哉!王勿遣也。」

元王慨然而嘆曰:「夫逆人之使,絕人之謀,是不暴乎?取人之有,以自為寶,是不彊乎?寡人聞之,暴得者必暴亡,彊取者必后無功。桀紂暴彊,身死國亡。今我聽子,是無仁義之名而有暴彊之道。江河為湯武,我為桀紂。未見其利,恐離其咎。寡人狐疑,安事此寶,趣駕送龜,勿令久留。」

衛平對曰:「不然,王其無患。天地之閒,累石為山。高而不壞,地得為安。故云物或危而顧安,或輕而不可遷;人或忠信而不如誕謾,或醜惡而宜大官,或美好佳麗而為眾人患。非神聖人,莫能盡言。春秋冬夏,或暑或寒。寒暑不和,賊氣相奸。同歲異節,其時使然。故令春生夏長,秋收冬藏。或為仁義,或為暴彊。暴彊有鄉,仁義有時。萬物盡然,不可勝治。大王聽臣,臣請悉言之。天出五色,以辨白黑。地生五穀,以知善惡。人民莫知辨也,與禽獸相若。谷居而穴處,不知田作。天下禍亂,陰陽相錯。悤悤疾疾,通而不相擇。妖孽數見,傳為單薄。聖人別其生,使無相獲。禽獸有牝牡,置之山原;鳥有雌雄,布之林澤;有介之蟲,置之谿谷。故牧人民,為之城郭,內經閭術,外為阡陌。夫妻男女,賦之田宅,列其室屋。為之圖籍,別其名族。立官置吏,勸以爵祿。衣以桑麻,養以五穀。耕之耰之,鉏之耨之。口得所嗜,目得所美,身受其利。以是觀之,非彊不至。故曰田者不彊,囷倉不盈;商賈不彊,不得其贏;婦女不彊,布帛不精;官御不彊,其勢不成;大將不彊,卒不使令;侯王不彊,沒世無名。故云彊者,事之始也,分之理也,物之紀也。所求於彊,無不有也。王以為不然,王獨不聞玉櫝隻雉,出於昆山;明月之珠,出於四海;鐫石拌蚌,傳賣於市;聖人得之,以為大寶。大寶所在,乃為天子。今王自以為暴,不如拌蚌於海也;自以為彊,不過鐫石於昆山也。取者無咎,寶者無患。今龜使來抵網,而遭漁者得之,見夢自言,是國之寶也,王何憂焉。」

元王曰:「不然。寡人聞之,諫者福也,諛者賊也。人主聽諛,是愚惑也。雖然,禍不妄至,福不徒來。天地合氣,以生百財。陰陽有分,不離四時,十有二月,日至為期。聖人徹焉,身乃無災。明王用之,人莫敢欺。故云福之至也,人自生之;禍之至也,人自成之。禍與福同,刑與德雙。聖人察之,以知吉凶。桀紂之時,與天爭功,擁遏鬼神,使不得通。是固已無道矣,諛臣有眾。桀有諛臣,名曰趙梁。教為無道,勸以貪狼。系湯夏臺,殺關龍逢。左右恐死,偷諛於傍。國危於累卵,皆曰無傷。稱樂萬歲,或曰未央。蔽其耳目,與之詐狂。湯卒伐桀,身死國亡。聽其諛臣,身獨受殃。春秋著之,至今不忘。紂有諛臣,名為左彊。誇而目巧,教為象郎。將至於天,又有玉床。犀玉之器,象箸而羹。聖人剖其心,壯士斬其胻。箕子恐死,被髪佯狂。殺周太子歷,囚文王昌。投之石室,將以昔至明。陰兢活之,與之俱亡。入於周地,得太公望。興卒聚兵,與紂相攻。文王病死,載尸以行。太子發代將,號為武王。戰於牧野,破之華山之陽。紂不勝敗而還走,圍之象郎。自殺宣室,身死不葬。頭懸車軫,四馬曳行。寡人念其如此,腸如涫湯。是人皆富有天下而貴至天子,然而大傲。欲無猒時,舉事而喜高,貪很而驕。不用忠信,聽其諛臣,而為天下笑。今寡人之邦,居諸侯之閒,曾不如秋毫。舉事不當,又安亡逃!」

衛平對曰:「不然。河雖神賢,不如崑崙之山;江之源理,不如四海,而人尚奪取其寶,諸侯爭之,兵革為起。小國見亡,大國危殆,殺人父兄,虜人妻子,殘國滅廟,以爭此寶。戰攻分爭,是暴彊也。故云取之以暴彊而治以文理,無逆四時,必親賢士;與陰陽化,鬼神為使;通於天地,與之為友。諸侯賓服,民眾殷喜。邦家安寧,與世更始。湯武行之,乃取天子;春秋著之,以為經紀。王不自稱湯武,而自比桀紂。桀紂為暴彊也,固以為常。桀為瓦室,紂為象郎。徵絲灼之,務以費(民)[氓]。賦斂無度,殺戮無方。殺人六畜,以韋為囊。囊盛其血,與人縣而射之,與天帝爭彊。逆亂四時,先百鬼嘗。諫者輒死,諛者在傍。聖人伏匿,百姓莫行。天數枯旱,國多妖祥。螟蟲歲生,五穀不成。民不安其處,鬼神不享。飄風日起,正晝晦冥。日月并蝕,滅息無光。列星奔亂,皆絕紀綱。以是觀之,安得久長!雖無湯武,時固當亡。故湯伐桀,武王剋紂,其時使然。乃為天子,子孫續世;終身無咎,後世稱之,至今不已。是皆當時而行,見事而彊,乃能成其帝王。今龜,大寶也,為聖人使,傳之賢(士)[王]。不用手足,雷電將之;風雨送之,流水行之。侯王有德,乃得當之。今王有德而當此寶,恐不敢受;王若遣之,宋必有咎。後雖悔之,亦無及已。」

元王大悅而喜。於是元王向日而謝,再拜而受。擇日齋戒,甲乙最良。乃刑白雉,及與驪羊;以血灌龜,於壇中央。以刀剝之,身全不傷。脯酒禮之,橫其腹腸。荊支卜之,必制其創。理達於理,文相錯迎。使工占之,所言盡當。邦福重寶,聞于傍鄉。殺牛取帮,被鄭之桐。草木畢分,化為甲兵。戰勝攻取,莫如元王。元王之時,衛平相宋,宋國最彊,龜之力也。

故云神至能見夢於元王,而不能自出漁者之籠。身能十言盡當,不能通使於河,還報於江,賢能令人戰勝攻取,不能自解於刀鋒,免剝刺之患。聖能先知亟見,而不能令衛平無言。言事百全,至身而攣;當時不利,又焉事賢!賢者有恒常,士有適然。是故明有所不見,聽有所不聞;人雖賢,不能左畫方,右畫圓;日月之明,而時蔽於浮雲。羿名善射,不如雄渠、蜂門;禹名為辯智,而不能勝鬼神。地柱折,天故毋椽,又柰何責人於全?孔子聞之曰:「神龜知吉凶,而骨直空枯。日為德而君於天下,辱於三足之烏。月為刑而相佐,見食於蝦蟆。猬辱於鵲,騰蛇之神而殆於即且。竹外有節理,中直空虛;松柏為百木長,而守門閭。日辰不全,故有孤虛。黃金有疵,白玉有瑕。事有所疾,亦有所徐。物有所拘,亦有所據。罔有所數,亦有所疏。人有所貴,亦有所不如。何可而適乎?物安可全乎?天尚不全,故世為屋,不成三瓦而陳之,以應之天。天下有階,物不全乃生也。」

褚先生曰:漁者舉網而得神龜,龜自見夢宋元王,元王召博士衛平告以夢龜狀,平運式,定日月,分衡度,視吉凶,占龜與物色同,平諫王留神龜以為國重寶,美矣。古者筮必稱龜者,以其令名,所從來久矣。余述而為傳。

三月二月正月十二月十一月中關內高外下四月首仰足開肣開首俛大五月橫吉首俛大六月七月八月九月十月

卜禁曰:子亥戌不可以卜及殺龜。日中如食已卜。暮昏龜之徼也,不可以卜。庚辛可以殺,及以鉆之。常以月旦祓龜,先以清水澡之,以卵祓之,乃持龜而遂之,若常以為祖。人若已卜不中,皆祓之以卵,東向立,灼以荊若剛木,土卵指之者三,持龜以卵周環之,祝曰:「今日吉,謹以粱卵焍黃祓去玉靈之不祥。」玉靈必信以誠,知萬事之情,辯兆皆可占。不信不誠,則燒玉靈,揚其灰,以徵后龜。其卜必北向,龜甲必尺二寸。

卜先以造灼鉆,鉆中已,又灼龜首,各三;又復灼所鉆中曰正身,灼首曰正足,各三。即以造三周龜,祝曰:「假之玉靈夫子。夫子玉靈,荊灼而心,令而先知。而上行於天,下行於淵,諸靈數,莫如汝信。今日良日,行一良貞。某欲卜某,即得而喜,不得而悔。即得,發鄉我身長大,首足收人皆上偶。不得,發鄉我身挫折,中外不相應,首足滅去。」

靈龜卜祝曰:「假之靈龜,五巫五靈,不如神龜之靈,知人死,知人生。某身良貞,某欲求某物。即得也,頭見足發,內外相應;即不得也,頭仰足肣,內外自垂。可得占。」

卜占病者祝曰:「今某病困。死,首上開,內外交駭,身節折;不死,首仰足肣。」卜病者祟曰:「今病有祟無呈,無祟有呈。兆有中祟有內,外祟有外。」

卜系者出不出。不出,橫吉安;若出,足開首仰有外。

卜求財物,其所當得。得,首仰足開,內外相應;即不得,呈兆首仰足肣。

卜有賣若買臣妾馬牛。得之,首仰足開,內外相應;不得,首仰足肣,呈兆若橫吉安。

卜擊盜聚若干人,在某所,今某將卒若干人,往擊之。當勝,首仰足開身正,內自橋,外下;不勝,足肣首仰,身首內下外高。

卜求當行不行。行,首足開;不行,足肣首仰,若橫吉安,安不行。

卜往擊盜,當見不見。見,首仰足肣有外;不見,足開首仰。

卜往候盜,見不見。見,首仰足肣,肣勝有外;不見,足開首仰。

卜聞盜來不來。來,外高內下,足肣首仰;不來,足開首仰,若橫吉安,期之自次。

卜遷徙去官不去。去,足開有肣外首仰;不去,自去,即足肣,呈兆若橫吉安。

卜居官尚吉不。吉,呈兆身正,若橫吉安;不吉,身節折,首仰足開。

卜居室家吉不吉。吉,呈兆身正,若橫吉安;不吉,身節折,首仰足開。

卜歲中禾稼孰不孰。孰,首仰足開,內外自橋外自垂;不孰,足肣首仰有外。

卜歲中民疫不疫。疫,首仰足肣,身節有彊外;不疫,身正首仰足開。

卜歲中有兵無兵。無兵,呈兆若橫吉安;有兵,首仰足開,身作外彊情。

卜見貴人吉不吉。吉,足開首仰,身正,內自橋;不吉,首仰,身節折,足肣有外,若無漁。

卜請謁於人得不得。得,首仰足開,內自橋;不得,首仰足肣有外。

卜追亡人當得不得。得,首仰足肣,內外相應;不得,首仰足開,若橫吉安。

卜漁獵得不得。得,首仰足開,內外相應;不得,足肣首仰,若橫吉安。

卜行遇盜不遇。遇,首仰足開,身節折,外高內下;不遇,呈兆。

卜天雨不雨。雨,首仰有外,外高內下;不雨,首仰足開,若橫吉安。

卜天雨霽不霽。霽,呈兆足開首仰;不霽,橫吉。

命曰橫吉安。以占病,病甚者一日不死;不甚者卜日瘳,不死。系者重罪不出,輕罪環出;過一日不出,久毋傷也。求財物買臣妾馬牛,一日環得;過一日不得。行者不行。來者環至;過食時不至,不來。擊盜不行,行不遇;聞盜不來。徙官不徙。居官家室皆吉。歲稼不孰。民疾疫無疾。歲中無兵。見人行,不行不喜。請謁人不行不得。追亡人漁獵不得。行不遇盜。雨不雨。霽不霽。

命曰呈兆。病者不死。系者出。行者行。來者來。市買得。追亡人得,過一日不得。問行者不到。

命曰柱徹。卜病不死。系者出。行者行。來者來。市買不得。憂者毋憂。追亡人不得。

命曰首仰足肣有內無外。占病,病甚不死。系者解。求財物買臣妾馬牛不得。行者聞言不行。來者不來。聞盜不來。聞言不至。徒官聞言不徙。居官有憂。居家多災。歲稼中孰。民疾疫多病。歲中有兵,聞言不開。見貴人吉。請謁不行,行不得善言。追亡人不得。漁獵不得。行不遇盜。雨不雨甚。霽不霽。故其莫字皆為首備。問之曰,備者仰也,故定以為仰。此私記也。

命曰首仰足肣有內無外。占病,病甚不死。系者不出。求財買臣妾不得。行者不行。來者不來。擊盜不見。聞盜來,內自驚,不來。徙官不徙。居官家室吉。歲稼不孰。民疾疫有病甚。歲中無兵。見貴人吉。請謁追亡人不得。亡財物,財物不出得。漁獵不得。行不遇盜。雨不雨。霽不霽。凶。

命曰呈兆首仰足肣。以占病,不死。系者未出。求財物買臣妾馬牛不得。行不行。來不來。擊盜不相見。聞盜來不來。徙官不徙。居官久多憂。居家室不吉。歲稼不孰。民病疫。歲中毋兵。見貴人不吉。請謁不得。漁獵得少。行不遇盜。雨不雨。霽不霽。不吉。

命曰呈兆首仰足開。以占病,病甐死。系囚出。求財物買臣妾馬牛不得。行者行。來者來。擊盜不見盜。聞盜來不來。徙官徙。居官不久。居家室不吉。歲稼不孰。民疾疫有而少。歲中毋兵。見貴人不見吉。請謁追亡人漁獵不得。行遇盜。雨不雨。霽小吉。

命曰首仰足肣。以占病,不死。系者久,毋傷也。求財物買臣妾馬牛不得。行者不行。擊盜不行。來者來。聞盜來。徙官聞言不徙。居家室不吉。歲稼不孰。民疾疫少。歲中毋兵。見貴人得見。請謁追亡人漁獵不得。行遇盜。雨不雨。霽不霽。吉。

命曰首仰足開有內。以占病者,死。系者出。求財物買臣妾馬牛不得。行者行。來者來。擊盜行不見盜。聞盜來不來。徙官徙。居官不久。居家室不吉。歲孰。民疾疫有而少。歲中毋兵。見貴人不吉。請謁追亡人漁獵不得。行不遇盜。雨霽。霽小吉,不霽吉。

命曰橫吉內外自橋。以占病,卜日毋瘳死。系者毋罪出。求財物買臣妾馬牛得。行者行。來者來。擊盜合交等。聞盜來來。徙官徙。居家室吉。歲孰。民疫無疾。歲中無兵。見貴人請謁追亡人漁獵得。行遇盜。雨霽,雨霽大吉。

命曰橫吉內外自吉。以占病,病者死。系不出。求財物買臣妾馬牛追亡人漁獵不得。行者不來。擊盜不相見。聞盜不來。徙官徙。居官有憂。居家室見貴人請謁不吉。歲稼不孰。民疾疫。歲中無兵。行不遇盜。雨不雨。霽不霽。不吉。

命曰漁人。以占病者,病者甚,不死。系者出。求財物買臣妾馬牛擊盜請謁追亡人漁獵得。行者行來。聞盜來不來。徙官不徒。居家室吉。歲稼不孰。民疾疫。歲中毋兵。見貴人吉。行不遇盜。雨不雨。霽不霽。吉。

命曰首仰足肣內高外下。以占病,病者甚,不死。系者不出。求財物買臣妾馬牛追亡人漁獵得。行不行。來者來。擊盜勝。徙官不徙。居官有憂,無傷也。居家室多憂病。歲大孰。民疾疫。歲中有兵不至。見貴人請謁不吉。行遇盜。雨不雨。霽不霽。吉。

命曰橫吉上有仰下有柱。病久不死。系者不出。求財物買臣妾馬牛追亡人漁獵不得。行不行。來不來。擊盜不行,行不見。聞盜來不來。徙官不徙。居家室見貴人吉。歲大孰。民疾疫。歲中毋兵。行不遇盜。雨不雨。霽不霽。大吉。

命曰橫吉榆仰。以占病,不死。系者不出。求財物買臣妾馬牛至不得。行不行。來不來。擊盜不行,行不見。聞盜來不來。徙官不徙。居官家室見貴人吉。歲孰。歲中有疾疫,毋兵。請謁追亡人不得。漁獵至不得。行不得。行不遇盜。雨霽不霽。小吉。

命曰橫吉下有柱。以占病,病甚不環有瘳無死。系者出。求財物買臣妾馬牛請謁追亡人漁獵不得。行來不來。擊盜不合。聞盜來來。徙官居官吉,不久。居家室不吉。歲不孰。民毋疾疫。歲中毋兵。見貴人吉。行不遇盜。雨不雨。霽。小吉。

命曰載所。以占病,環有瘳無死。系者出。求財物買臣妾馬牛請謁追亡人漁獵得。行者行。來者來。擊盜相見不相合。聞盜來來。徙官徙。居家室憂。見貴人吉。歲孰。民毋疾疫。歲中毋兵。行不遇盜。雨不雨。霽霽。吉。

命曰根格。以占病者,不死。系久毋傷。求財物買臣妾馬牛請謁追亡人漁獵不得。行不行。來不來。擊盜盜行不合。聞盜不來。徙官不徙。居家室吉。歲稼中。民疾疫無死。見貴人不得見。行不遇盜。雨不雨。不吉。

命曰首仰足肣外高內下。卜有憂,無傷也。行者不來。病久死。求財物不得。見貴人者吉。

命曰外高內下。卜病不死,有祟。市買不得。居官家室不吉。行者不行。來者不來。系者久毋傷。吉。

命曰頭見足發有內外相應。以占病者,起。系者出。行者行。來者來。求財物得。吉。

命曰呈兆首仰足開。以占病,病甚死。系者出,有憂。求財物買臣妾馬牛請謁追亡人漁獵不得。行不行。來不來。擊盜不合。聞盜來來。徙官居官家室不吉。歲惡。民疾疫無死。歲中毋兵。見貴人不吉。行不遇盜。雨不雨。霽。不吉。

命曰呈兆首仰足開外高內下。以占病,不死,有外祟。系者出,有憂。求財物買臣妾馬牛,相見不會。行行。來聞言不來。擊盜勝。聞盜來不來。徙官居官家室見貴人不吉。歲中。民疾疫有兵。請謁追亡人漁獵不得。聞盜遇盜。雨不雨。霽。凶。

命曰首仰足肣身折內外相應。以占病,病甚不死。系者久不出。求財物買臣妾馬牛漁獵不得。行不行。來不來。擊盜有用勝。聞盜來來。徙官不徙。居官家室不吉。歲不孰。民疾疫。歲中。有兵不至。見貴人喜。請謁追亡人不得。遇盜凶。

命曰內格外垂。行者不行。來者不來。病者死。系者不出。求財物不得。見人不見。大吉。

命曰橫吉內外相應自橋榆仰上柱足肣。以占病,病甚不死。系久,不抵罪。求財物買臣妾馬牛請謁追亡人漁獵不得。行不行。來不來。居官家室見貴人吉。徙官不徙。歲不大孰。民疾疫有兵。有兵不會。行遇盜。聞言不見。雨不雨。霽霽。大吉。

命曰頭仰足肣內外自垂。卜憂病者甚,不死。居官不得居。行者行。來者不來。求財物不得。求人不得。吉。

命曰橫吉下有柱。卜來者來。卜日即不至,未來。卜病者過一日毋瘳死。行者不行。求財物不得。系者出。

命曰橫吉內外自舉。以占病者,久不死。系者久不出。求財物得而少。行者不行。來者不來。見貴人見。吉。

命曰內高外下疾輕足發。求財物不得。行者行。病者有瘳。系者不出。來者來。見貴人不見。吉。

命曰外格。求財物不得。行者不行。來者不來。系者不出。不吉。病者死。求財物不得。見貴人見。吉。

命曰內自舉外來正足發。[行]者行。來者來。求財物得。病者久不死。系者不出。見貴人見。吉。

此橫吉上柱外內自舉足肣。以卜有求得。病不死。系者毋傷,未出。行不行。來不來。見人不見。百事盡吉。

此橫吉上柱外內自舉柱足以作。以卜有求得。病死環起。系留毋傷,環出。行不行。來不來。見人不見。百事吉。可以舉兵。

此挺詐有外。以卜有求不得。病不死,數起。系禍罪。聞言毋傷。行不行。來不來。

此挺詐有內。以卜有求不得。病不死,數起。系留禍罪無傷出。行不行。來者不來。見人不見。

此挺詐內外自舉。以卜有求得。病不死。系毋罪。行行。來來。田賈市漁獵盡喜。

此狐狢。以卜有求不得。病死,難起。系留毋罪難出。可居宅。可娶婦嫁女。行不行。來不來。見人不見。有憂不憂。

此狐徹。以卜有求不得。病者死。系留有抵罪。行不行。來不來。見人不見。言語定。百事盡不吉。

此首俯足肣身節折。以卜有求不得。病者死。系留有罪。望行者不來。行行。來不來。見人不見。

此挺內外自垂。以卜有求不晦。病不死,難起。系留毋罪,難出。行不行。來不來。見人不見。不吉。

此橫吉榆仰首俯。以卜有求難得。病難起,不死。系難出,毋傷也。可居家室,以娶婦嫁女。

此橫吉上柱載正身節折內外自舉。以卜病者,卜日不死,其一日乃死。

此橫吉上柱足肣內自舉外自垂。以卜病者,卜日不死,其一日乃死。

(為人病)首俯足詐有外無內。病者占龜未已,急死。卜輕失大,一日不死。

首仰足肣。以卜有求不得。以系有罪。人言語恐之毋傷。行不行。見人不見。

大論曰:外者人也,內者自我也;外者女也,內者男也。首俛者憂。大者身也,小者枝也。大法,病者,足肣者生,足開者死。行者,足開至,足肣者不至。行者,足肣不行,足開行。有求,足開得,足肣者不得。系者,足肣不出,開出。其卜病也,足開而死者,內高而外下也。


\end{pinyinscope}