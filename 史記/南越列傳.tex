\article{南越列傳}

\begin{pinyinscope}
南越王尉佗者,真定人也,姓趙氏。秦時已并天下,略定楊越,置桂林、南海、象郡,以謫徙民,與越雜處十三歲。佗,秦時用為南海龍川令。至二世時,南海尉任囂病且死,召龍川令趙佗語曰:「聞陳勝等作亂,秦為無道,天下苦之,項羽、劉季、陳勝、吳廣等州郡各共興軍聚眾,虎爭天下,中國擾亂,未知所安,豪傑畔秦相立。南海僻遠,吾恐盜兵侵地至此,吾欲興兵絕新道,自備,待諸侯變,會病甚。且番禺負山險,阻南海,東西數千里,頗有中國人相輔,此亦一州之主也,可以立國。郡中長吏無足與言者,故召公告之。」即被佗書,行南海尉事。囂死,佗即移檄告橫浦、陽山、湟谿關曰:「盜兵且至,急絕道聚兵自守!」因稍以法誅秦所置長吏,以其黨為假守。秦已破滅,佗即擊并桂林、象郡,自立為南越武王。高帝已定天下,為中國勞苦,故釋佗弗誅。漢十一年,遣陸賈因立佗為南越王,與剖符通使,和集百越,毋為南邊患害,與長沙接境。

高后時,有司請禁南越關市鐵器。佗曰:「高帝立我,通使物,今高后聽讒臣,別異蠻夷,隔絕器物,此必長沙王計也,欲倚中國,擊滅南越而并王之,自為功也。」於是佗乃自尊號為南越武帝,發兵攻長沙邊邑,敗數縣而去焉。高后遣將軍隆慮侯灶往擊之。會暑溼,士卒大疫,兵不能踰嶺。歲餘,高后崩,即罷兵。佗因此以兵威邊,財物賂遺閩越、西甌、駱,役屬焉,東西萬餘里。乃乘黃屋左纛,稱制,與中國侔。

及孝文帝元年,初鎮撫天下,使告諸侯四夷從代來即位意,喻盛德焉。乃為佗親冢在真定,置守邑,歲時奉祀。召其從昆弟,尊官厚賜寵之。詔丞相陳平等舉可使南越者,平言好畤陸賈,先帝時習使南越。乃召賈以為太中大夫,往使。因讓佗自立為帝,曾無一介之使報者。陸賈至南越,王甚恐,為書謝,稱曰:「蠻夷大長老夫臣佗,前日高后隔異南越,竊疑長沙王讒臣,又遙聞高后盡誅佗宗族,掘燒先人冢,以故自棄,犯長沙邊境。且南方卑溼,蠻夷中閒,其東閩越千人眾號稱王,其西甌駱裸國亦稱王。老臣妄竊帝號,聊以自娛,豈敢以聞天王哉!」乃頓首謝,願長為藩臣,奉貢職。於是乃下令國中曰:「吾聞兩雄不俱立,兩賢不并世。皇帝,賢天子也。自今以後,去帝制黃屋左纛。」陸賈還報,孝文帝大說。遂至孝景時,稱臣,使人朝請。然南越其居國竊如故號名,其使天子,稱王朝命如諸侯。至建元四年卒。

佗孫胡為南越王。此時閩越王郢興兵擊南越邊邑,胡使人上書曰:「兩越俱為藩臣,毋得擅興兵相攻擊。今閩越興兵侵臣,臣不敢興兵,唯天子詔之。」於是天子多南越義,守職約,為興師,遣兩將軍往討閩越。兵未踰嶺,閩越王弟餘善殺郢以降,於是罷兵。

天子使莊助往諭意南越王,胡頓首曰:「天子乃為臣興兵討閩越,死無以報德!」遣太子嬰齊入宿衛。謂助曰:「國新被寇,使者行矣。胡方日夜裝入見天子。」助去後,其大臣諫胡曰:「漢興兵誅郢,亦行以驚動南越。且先王昔言,事天子期無失禮,要之不可以說好語入見。入見則不得復歸,亡國之勢也。」於是胡稱病,竟不入見。後十餘歲,胡實病甚,太子嬰齊請歸。胡薨,謚為文王。

嬰齊代立,即藏其先武帝璽。嬰齊其入宿衛在長安時,取邯鄲樛氏女,生子興。及即位,上書請立樛氏女為后,興為嗣。漢數使使者風諭嬰齊,嬰齊尚樂擅殺生自恣,懼入見要用漢法,比內諸侯,固稱病,遂不入見。遣子次公入宿衛。嬰齊薨,謚為明王。

太子興代立,其母為太后。太后自未為嬰齊姬時,嘗與霸陵人安國少季通。及嬰齊薨後,元鼎四年,漢使安國少季往諭王、王太后以入朝,比內諸侯;令辯士諫大夫終軍等宣其辭,勇士魏臣等輔其缺,衛尉路博德將兵屯桂陽,待使者。王年少,太後中國人也,嘗與安國少季通,其使復私焉。國人頗知之,多不附太后。太后恐亂起,亦欲倚漢威,數勸王及群臣求內屬。即因使者上書,請比內諸侯,三歲一朝,除邊關。於是天子許之,賜其丞相呂嘉銀印,及內史、中尉、太傅印,餘得自置。除其故黥劓刑,用漢法,比內諸侯。使者皆留填撫之。王、王太后飭治行裝重齎,為入朝具。

其相呂嘉年長矣,相三王,宗族官仕為長吏者七十餘人,男盡尚王女,女盡嫁王子兄弟宗室,及蒼梧秦王有連。其居國中甚重,越人信之,多為耳目者,得眾心愈於王。王之上書,數諫止王,王弗聽。有畔心,數稱病不見漢使者。使者皆注意嘉,勢未能誅。王、王太后亦恐嘉等先事發,乃置酒,介漢使者權,謀誅嘉等。使者皆東鄉,太后南鄉,王北鄉,相嘉、大臣皆西鄉,侍坐飲。嘉弟為將,將卒居宮外。酒行,太后謂嘉曰:「南越內屬,國之利也,而相君苦不便者,何也?」以激怒使者。使者狐疑相杖,遂莫敢發。嘉見耳目非是,即起而出。太后怒,欲鏦嘉以矛,王止太後。嘉遂出,分其弟兵就舍,稱病,不肯見王及使者。乃陰與大臣作亂。王素無意誅嘉,嘉知之,以故數月不發。太后有淫行,國人不附,欲獨誅嘉等,力又不能。

天子聞嘉不聽王,王、王太后弱孤不能制,使者怯無決。又以為王、王太后已附漢,獨呂嘉為亂,不足以興兵,欲使莊參以二千人往使。參曰:「以好往,數人足矣;以武往,二千人無足以為也。」辭不可,天子罷參也。郟壯士故濟北相韓千秋奮曰:「以區區之越,又有王、太后應,獨相呂嘉為害,願得勇士二百人,必斬嘉以報。」於是天子遣千秋與王太后弟樛樂將二千人往,入越境。呂嘉等乃遂反,下令國中曰:「王年少。太後,中國人也,又與使者亂,專欲內屬,盡持先王寶器入獻天子以自媚,多從人,行至長安,虜賣以為僮仆。取自脫一時之利,無顧趙氏社稷,為萬世慮計之意。」乃與其弟將卒攻殺王、太后及漢使者。遣人告蒼梧秦王及其諸郡縣,立明王長男越妻子術陽侯建德為王。而韓千秋兵入,破數小邑。其後越直開道給食,未至番禺四十里,越以兵擊千秋等,遂滅之。使人函封漢使者節置塞上,好為謾辭謝罪,發兵守要害處。於是天子曰:「韓千秋雖無成功,亦軍鋒之冠。」封其子延年為成安侯。樛樂,其姊為王太后,首願屬漢,封其子廣德為龍亢侯。乃下赦曰:「天子微,諸侯力政,譏臣不討賊。今呂嘉、建德等反,自立晏如,令罪人及江淮以南樓船十萬師往討之。」

元鼎五年秋,衛尉路博德為伏波將軍,出桂陽,下匯水;主爵都尉楊仆為樓船將軍,出豫章,下橫浦;故歸義越侯二人為戈船、下厲將軍,出零陵,或下離水,或柢蒼梧;使馳義侯因巴蜀罪人,發夜郎兵,下牂柯江:咸會番禺。

元鼎六年冬,樓船將軍將精卒先陷尋陜,破石門,得越船粟,因推而前,挫越鋒,以數萬人待伏波。伏波將軍將罪人,道遠,會期後,與樓船會乃有千餘人,遂俱進。樓船居前,至番禺。建德、嘉皆城守。樓船自擇便處,居東南面;伏波居西北面。會暮,樓船攻敗越人,縱火燒城。越素聞伏波名,日暮,不知其兵多少。伏波乃為營,遣使者招降者,賜印,復縱令相招。樓船力攻燒敵,反驅而入伏波營中。犁旦,城中皆降伏波。呂嘉、建德已夜與其屬數百人亡入海,以船西去。伏波又因問所得降者貴人,以知呂嘉所之,遣人追之。以其故校尉司馬蘇弘得建德,封為海常侯;越郎都稽得嘉,封為臨蔡侯。

蒼梧王趙光者,越王同姓,聞漢兵至,及越揭陽令定自定屬漢;越桂林監居翁諭甌駱屬漢:皆得為侯。戈船、下厲將軍兵及馳義侯所發夜郎兵未下,南越已平矣。遂為九郡。伏波將軍益封。樓船將軍兵以陷堅為將梁侯。自尉佗初王後,五世九十三歲而國亡焉。

太史公曰:尉佗之王,本由任囂。遭漢初定,列為諸侯。隆慮離溼疫,佗得以益驕。甌駱相攻,南越動搖。漢兵臨境,嬰齊入朝。其後亡國,徵自樛女;呂嘉小忠,令佗無後。樓船從欲,怠傲失惑;伏波困窮,智慮愈殖,因禍為福。成敗之轉,譬若糾墨。


\end{pinyinscope}