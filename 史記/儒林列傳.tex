\article{儒林列傳}

\begin{pinyinscope}
太史公曰:余讀功令,至於廣厲學官之路,未嘗不廢書而嘆也。曰:嗟乎!夫周室衰而關雎作,幽厲微而禮樂壞,諸侯恣行,政由彊國。故孔子閔王路廢而邪道興,於是論次詩書,修起禮樂。適齊聞韶,三月不知肉味。自衛返魯,然後樂正,雅頌各得其所。世以混濁莫能用,是以仲尼干七十餘君無所遇,曰「茍有用我者,期月而已矣」。西狩獲麟,曰「吾道窮矣」。故因史記作春秋,以當王法,其辭微而指博,後世學者多錄焉。

自孔子卒後,七十子之徒散游諸侯,大者為師傅卿相,小者友教士大夫,或隱而不見。故子路居衛,子張居陳,澹臺子羽居楚,子夏居西河,子貢終於齊。如田子方、段干木、吳起、禽滑釐之屬,皆受業於子夏之倫,為王者師。是時獨魏文侯好學。后陵遲以至于始皇,天下并爭於戰國,懦術既絀焉,然齊魯之閒,學者獨不廢也。於威、宣之際,孟子、荀卿之列,咸遵夫子之業而潤色之,以學顯於當世。

及至秦之季世,焚詩書,阬術士,六藝從此缺焉。陳涉之王也,而魯諸儒持孔氏之禮器往歸陳王。於是孔甲為陳涉博士,卒與涉俱死。陳涉起匹夫,驅瓦合適戍,旬月以王楚,不滿半歲竟滅亡,其事至微淺,然而縉紳先生之徒負孔子禮器往委質為臣者,何也?以秦焚其業,積怨而發憤于陳王也。

及高皇帝誅項籍,舉兵圍魯,魯中諸儒尚講誦習禮樂,弦歌之音不絕,豈非聖人之遺化,好禮樂之國哉?故孔子在陳曰:「歸與!歸與!吾黨之小子狂簡,斐然成章,不知所以裁之。」夫齊魯之閒於文學,自古以來,其天性也。故漢興,然後諸儒始得修其經藝,講習大射鄉飲之禮。叔孫通作漢禮儀,因為太常,諸生弟子共定者,咸為選首,於是喟然嘆興於學。然尚有干戈,平定四海,亦未暇遑庠序之事也。孝惠、呂后時,公卿皆武力有功之臣。孝文時頗徵用,然孝文帝本好刑名之言。及至孝景,不任儒者,而竇太后又好黃老之術,故諸博士具官待問,未有進者。

及今上即位,趙綰、王臧之屬明儒學,而上亦鄉之,於是招方正賢良文學之士。自是之后,言詩於魯則申培公,於齊則轅固生,於燕則韓太傅。言尚書自濟南伏生。言禮自魯高堂生。言易自菑川田生。言春秋於齊魯自胡毋生,於趙自董仲舒。及竇太后崩,武安侯田蚡為丞相,絀黃老、刑名百家之言,延文學儒者數百人,而公孫弘以春秋白衣為天子三公,封以平津侯。天下之學士靡然鄉風矣。

公孫弘為學官,悼道之郁滯,乃請曰:「丞相御史言:制曰『蓋聞導民以禮,風之以樂。婚姻者,居屋之大倫也。今禮廢樂崩,朕甚愍焉。故詳延天下方正博聞之士,咸登諸朝。其令禮官勸學,講議洽聞興禮,以為天下先。太常議,與博士弟子,崇鄉里之化,以廣賢材焉』。謹與太常臧、博士平等議曰:聞三代之道,鄉里有教,夏曰校,殷曰序,周曰庠。其勸善也,顯之朝廷;其懲惡也,加之刑罰。故教化之行也,建首善自京師始,由內及外。今陛下昭至德,開大明,配天地,本人倫,勸學修禮,崇化厲賢,以風四方,太平之原也。古者政教未洽,不備其禮,請因舊官而興焉。為博士官置弟子五十人,復其身。太常擇民年十八已上,儀狀端正者,補博士弟子。郡國縣道邑有好文學,敬長上,肅政教,順鄉里,出入不悖所聞者,令相長丞上屬所二千石,二千石謹察可者,當與計偕,詣太常,得受業如弟子。一歲皆輒試,能通一藝以上,補文學掌故缺;其高弟可以為郎中者,太常籍奏。即有秀才異等,輒以名聞。其不事學若下材及不能通一藝,輒罷之,而請諸不稱者罰。臣謹案詔書律令下者,明天人分際,通古今之義,文章爾雅,訓辭深厚,恩施甚美。小吏淺聞,不能究宣,無以明布諭下。治禮次治掌故,以文學禮義為官,遷留滯。請選擇其秩比二百石以上,及吏百石通一藝以上,補左右內史、大行卒史;比百石已下,補郡太守卒史:皆各二人,邊郡一人。先用誦多者,若不足,乃擇掌故補中二千石屬,文學掌故補郡屬,備員。請著功令。佗如律令。」制曰:「可。」自此以來,則公卿大夫士吏斌斌多文學之士矣。

申公者,魯人也。高祖過魯,申公以弟子從師入見高祖于魯南宮。呂太后時,申公游學長安,與劉郢同師。已而郢為楚王,令申公傅其太子戊。戊不好學,疾申公。及王郢卒,戊立為楚王,胥靡申公。申公恥之,歸魯,退居家教,終身不出門,復謝絕賓客,獨王命召之乃往。弟子自遠方至受業者百餘人。申公獨以詩經為訓以教,無傳(疑),疑者則闕不傳。

蘭陵王臧既受詩,以事孝景帝為太子少傅,免去。今上初即位,臧乃上書宿衛上,累遷,一歲中為郎中令。及代趙綰亦嘗受詩申公,綰為御史大夫。綰、臧請天子,欲立明堂以朝諸侯,不能就其事,乃言師申公。於是天子使使束帛加璧安車駟馬迎申公,弟子二人乘軺傳從。至,見天子。天子問治亂之事,申公時已八十餘,老,對曰:「為治者不在多言,顧力行何如耳。」是時天子方好文詞,見申公對,默然。然已招致,則以為太中大夫,舍魯邸,議明堂事。太皇竇太后好老子言,不說儒術,得趙綰、王臧之過以讓上,上因廢明堂事,盡下趙綰、王臧吏,後皆自殺。申公亦疾免以歸,數年卒。

弟子為博士者十餘人:孔安國至臨淮太守,周霸至膠西內史,夏寬至城陽內史,碭魯賜至東海太守,蘭陵繆生至長沙內史,徐偃為膠西中尉,鄒人闕門慶忌為膠東內史。其治官民皆有廉節,稱其好學。學官弟子行雖不備,而至於大夫、郎中、掌故以百數。言詩雖殊,多本於申公。

清河王太傅轅固生者,齊人也。以治詩,孝景時為博士。與黃生爭論景帝前。黃生曰:「湯武非受命,乃弒也。」轅固生曰:「不然。夫桀紂虐亂,天下之心皆歸湯武,湯武與天下之心而誅桀紂,桀紂之民不為之使而歸湯武,湯武不得已而立,非受命為何?」黃生曰:「冠雖敝,必加於首;履雖新,必關於足。何者,上下之分也。今桀紂雖失道,然君上也;湯武雖聖,臣下也。夫主有失行,臣下不能正言匡過以尊天子,反因過而誅之,代立踐南面,非弒而何也?」轅固生曰:「必若所云,是高帝代秦即天子之位,非邪?」於是景帝曰:「食肉不食馬肝,不為不知味;言學者無言湯武受命,不為愚。」遂罷。是後學者莫敢明受命放殺者。

竇太后好老子書,召轅固生問老子書。固曰:「此是家人言耳。」太后怒曰:「安得司空城旦書乎?」乃使固入圈刺豕。景帝知太后怒而固直言無罪,乃假固利兵,下圈刺豕,正中其心,一刺,豕應手而倒。太后默然,無以復罪,罷之。居頃之,景帝以固為廉直,拜為清河王太傅。久之,病免。

今上初即位,復以賢良徵固。諸諛儒多疾毀固,曰「固老」,罷歸之。時固已九十餘矣。固之徵也,薛人公孫弘亦徵,側目而視固。固曰:「公孫子,務正學以言,無曲學以阿世!」自是之後,齊言詩皆本轅固生也。諸齊人以詩顯貴,皆固之弟子也。

韓生者,燕人也。孝文帝時為博士,景帝時為常山王太傅。韓生推詩之意而為內外傳數萬言,其語頗與齊魯閒殊,然其歸一也。淮南賁生受之。自是之後,而燕趙閒言詩者由韓生。韓生孫商為今上博士。

伏生者,濟南人也。故為秦博士。孝文帝時,欲求能治尚書者,天下無有,乃聞伏生能治,欲召之。是時伏生年九十餘,老,不能行,於是乃詔太常使掌故晁錯往受之。秦時焚書,伏生壁藏之。其後兵大起,流亡,漢定,伏生求其書,亡數十篇,獨得二十九篇,即以教于齊魯之閒。學者由是頗能言尚書,諸山東大師無不涉尚書以教矣。

伏生教濟南張生及歐陽生,歐陽生教千乘兒寬。兒寬既通尚書,以文學應郡舉,詣博士受業,受業孔安國。兒寬貧無資用,常為弟子都養,及時時閒行傭賃,以給衣食。行常帶經,止息則誦習之。以試第次,補廷尉史。是時張湯方鄉學,以為奏讞掾,以古法議決疑大獄,而愛幸寬。寬為人溫良,有廉智,自持,而善著書、書奏,敏於文,口不能發明也。湯以為長者,數稱譽之。及湯為御史大夫,以兒寬為掾,薦之天子。天子見問,說之。張湯死后六年,兒寬位至御史大夫。九年而以官卒。寬在三公位,以和良承意從容得久,然無有所匡諫;於官,官屬易之,不為盡力。張生亦為博士。而伏生孫以治尚書徵,不能明也。

自此之後,魯周霸、孔安國,雒陽賈嘉,頗能言尚書事。孔氏有古文尚書,而安國以今文讀之,因以起其家。逸書得十餘篇,蓋尚書滋多於是矣。

諸學者多言禮,而魯高堂生最本。禮固自孔子時而其經不具,及至秦焚書,書散亡益多,於今獨有士禮,高堂生能言之。

而魯徐生善為容。孝文帝時,徐生以容為禮官大夫。傳子至孫延、徐襄。襄,其天姿善為容,不能通禮經;延頗能,未善也。襄以容為漢禮官大夫,至廣陵內史。延及徐氏弟子公戶滿意、桓生、單次,皆嘗為漢禮官大夫。而瑕丘蕭奮以禮為淮陽太守。是後能言禮為容者,由徐氏焉。

自魯商瞿受易孔子,孔子卒,商瞿傳易,六世至齊人田何,字子莊,而漢興。田何傳東武人王同子仲,子仲傳菑川人楊何。何以易,元光元年徵,官至中大夫。齊人即墨成以易至城陽相。廣川人孟但以易為太子門大夫。魯人周霸,莒人衡胡,臨菑人主父偃,皆以易至二千石。然要言易者本於楊何之家。

董仲舒,廣川人也。以治春秋,孝景時為博士。下帷講誦,弟子傳以久次相受業,或莫見其面,蓋三年董仲舒不觀於舍園,其精如此。進退容止,非禮不行,學士皆師尊之。今上即位,為江都相。以春秋災異之變推陰陽所以錯行,故求雨閉諸陽,縱諸陰,其止雨反是。行之一國,未嘗不得所欲。中廢為中大夫,居舍,著災異之記。是時遼東高廟災,主父偃疾之,取其書奏之天子。天子召諸生示其書,有刺譏。董仲舒弟子呂步舒不知其師書,以為下愚。於是下董仲舒吏,當死,詔赦之。於是董仲舒竟不敢復言災異。

董仲舒為人廉直。是時方外攘四夷,公孫弘治春秋不如董仲舒,而弘希世用事,位至公卿。董仲舒以弘為從諛。弘疾之,乃言上曰:「獨董仲舒可使相繆西王。」膠西王素聞董仲舒有行,亦善待之。董仲舒恐久獲罪,疾免居家。至卒,終不治產業,以修學著書為事。故漢興至于五世之閒,唯董仲舒名為明於春秋,其傳公羊氏也。

胡毋生,齊人也。孝景時為博士,以老歸教授。齊之言春秋者多受胡毋生,公孫弘亦頗受焉。

瑕丘江生為穀梁春秋。自公孫弘得用,嘗集比其義,卒用董仲舒。

仲舒弟子遂者:蘭陵褚大,廣川殷忠,溫呂步舒。褚大至梁相。步舒至長史,持節使決淮南獄,於諸侯擅專斷,不報,以春秋之義正之,天子皆以為是。弟子通者,至於命大夫;為郎、謁者、掌故者以百數。而董仲舒子及孫皆以學至大官。


\end{pinyinscope}