\article{越王句踐世家}

\begin{pinyinscope}
越王句踐,其先禹之苗裔,而夏后帝少康之庶子也。封於會稽,以奉守禹之祀。文身斷發,披草萊而邑焉。後二十餘世,至於允常。允常之時,與吳王闔廬戰而相怨伐。允常卒,子句踐立,是為越王。

元年,吳王闔廬聞允常死,乃興師伐越。越王句踐使死士挑戰,三行,至吳陳,呼而自剄。吳師觀之,越因襲擊吳師,吳師敗於檇李,射傷吳王闔廬。闔廬且死,告其子夫差曰:「必毋忘越。」

三年,句踐聞吳王夫差日夜勒兵,且以報越,越欲先吳未發往伐之。范蠡諫曰:「不可。臣聞兵者凶器也,戰者逆德也,爭者事之末也。陰謀逆德,好用凶器,試身於所末,上帝禁之,行者不利。」越王曰:「吾已決之矣。」遂興師。吳王聞之,悉發精兵擊越,敗之夫椒。越王乃以餘兵五千人保棲於會稽。吳王追而圍之。

越王謂范蠡曰:「以不聽子故至於此,為之柰何?」蠡對曰:「持滿者與天,定傾者與人,節事者以地。卑辭厚禮以遺之,不許,而身與之市。」句踐曰:「諾。」乃令大夫種行成於吳,膝行頓首曰:「君王亡臣句踐使陪臣種敢告下執事:句踐請為臣,妻為妾。」吳王將許之。子胥言於吳王曰:「天以越賜吳,勿許也。」種還,以報句踐。句踐欲殺妻子,燔寶器,觸戰以死。種止句踐曰:「夫吳太宰嚭貪,可誘以利,請閒行言之。」於是句踐以美女寶器令種閒獻吳太宰嚭。嚭受,乃見大夫種於吳王。種頓首言曰:「願大王赦句踐之罪,盡入其寶器。不幸不赦,句踐將盡殺其妻子,燔其寶器,悉五千人觸戰,必有當也。」嚭因說吳王曰:「越以服為臣,若將赦之,此國之利也。」吳王將許之。子胥進諫曰:「今不滅越,後必悔之。句踐賢君,種、蠡良臣,若反國,將為亂。」吳王弗聽,卒赦越,罷兵而歸。

句踐之困會稽也,喟然嘆曰:「吾終於此乎?」種曰:「湯系夏臺,文王囚羑里,晉重耳奔翟,齊小白奔莒,其卒王霸。由是觀之,何遽不為福乎?」

吳既赦越,越王句踐反國,乃苦身焦思,置膽於坐,坐臥即仰膽,飲食亦嘗膽也。曰:「女忘會稽之恥邪?」身自耕作,夫人自織,食不加肉,衣不重采,折節下賢人,厚遇賓客,振貧弔死,與百姓同其勞。欲使范蠡治國政,蠡對曰:「兵甲之事,種不如蠡;填撫國家,親附百姓,蠡不如種。」於是舉國政屬大夫種,而使范蠡與大夫柘稽行成,為質於吳。二歲而吳歸蠡。

句踐自會稽歸七年,拊循其士民,欲用以報吳。大夫逢同諫曰:「國新流亡,今乃復殷給,繕飾備利,吳必懼,懼則難必至。且鷙鳥之擊也,必匿其形。今夫吳兵加齊、晉,怨深於楚、越,名高天下,實害周室,德少而功多,必淫自矜。為越計,莫若結齊,親楚,附晉,以厚吳。吳之志廣,必輕戰。是我連其權,三國伐之,越承其弊,可克也。」句踐曰:「善。」

居二年,吳王將伐齊。子胥諫曰:「未可。臣聞句踐食不重味,與百姓同苦樂。此人不死,必為國患。吳有越,腹心之疾,齊與吳,疥癬也。願王釋齊先越。」吳王弗聽,遂伐齊,敗之艾陵,虜齊高、國以歸。讓子胥。子胥曰:「王毋喜!」王怒,子胥欲自殺,王聞而止之。越大夫種曰:「臣觀吳王政驕矣,請試嘗之貸粟,以卜其事。」請貸,吳王欲與,子胥諫勿與,王遂與之,越乃私喜。子胥言曰:「王不聽諫,後三年吳其墟乎!」太宰嚭聞之,乃數與子胥爭越議,因讒子胥曰:「伍員貌忠而實忍人,其父兄不顧,安能顧王?王前欲伐齊,員彊諫,已而有功,用是反怨王。王不備伍員,員必為亂。」與逢同共謀,讒之王。王始不從,乃使子胥於齊,聞其託子於鮑氏,王乃大怒,曰:「伍員果欺寡人!」役反,使人賜子胥屬鏤劍以自殺。子胥大笑曰:「我令而父霸,我又立若,若初欲分吳國半予我,我不受,已,今若反以讒誅我。嗟乎,嗟乎,一人固不能獨立!」報使者曰:「必取吾眼置吳東門,以觀越兵入也!」於是吳任嚭政。

居三年,句踐召范蠡曰:「吳已殺子胥,導諛者眾,可乎?」對曰:「未可。」

至明年春,吳王北會諸侯於黃池,吳國精兵從王,惟獨老弱與太子留守。句踐復問范蠡,蠡曰「可矣」。乃發習流二千人,教士四萬人,君子六千人,諸御千人,伐吳。吳師敗,遂殺吳太子。吳告急於王,王方會諸侯於黃池,懼天下聞之,乃祕之。吳王已盟黃池,乃使人厚禮以請成越。越自度亦未能滅吳,乃與吳平。

其後四年,越復伐吳。吳士民罷弊,輕銳盡死於齊、晉。而越大破吳,因而留圍之三年,吳師敗,越遂復棲吳王於姑蘇之山。吳王使公孫雄肉袒膝行而前,請成越王曰:「孤臣夫差敢布腹心,異日嘗得罪於會稽,夫差不敢逆命,得與君王成以歸。今君王舉玉趾而誅孤臣,孤臣惟命是聽,意者亦欲如會稽之赦孤臣之罪乎?」句踐不忍,欲許之。范蠡曰:「會稽之事,天以越賜吳,吳不取。今天以吳賜越,越其可逆天乎?且夫君王蚤朝晏罷,非為吳邪?謀之二十二年,一旦而棄之,可乎?且夫天與弗取,反受其咎。『伐柯者其則不遠』,君忘會稽之蚯?」句踐曰:「吾欲聽子言,吾不忍其使者。」范蠡乃鼓進兵,曰:「王已屬政於執事,使者去,不者且得罪。」吳使者泣而去。句踐憐之,乃使人謂吳王曰:「吾置王甬東,君百家。」吳王謝曰:「吾老矣,不能事君王!」遂自殺。乃蔽其面,曰:「吾無面以見子胥也!」越王乃葬吳王而誅太宰嚭。

句踐已平吳,乃以兵北渡淮,與齊、晉諸侯會於徐州,致貢於周。周元王使人賜句踐胙,命為伯。句踐已去,渡淮南,以淮上地與楚,歸吳所侵宋地於宋,與魯泗東方百里。當是時,越兵橫行於江、淮東,諸侯畢賀,號稱霸王。

范蠡遂去,自齊遺大夫種書曰:「蜚鳥盡,良弓藏;狡兔死,走狗烹。越王為人長頸鳥喙,可與共患難,不可與共樂。子何不去?」種見書,稱病不朝。人或讒種且作亂,越王乃賜種劍曰:「子教寡人伐吳七術,寡人用其三而敗吳,其四在子,子為我從先王試之。」種遂自殺。

句踐卒,子王鼫與立。王鼫與卒,子王不壽立。王不壽卒,子王翁立。王翁卒,子王翳立。王翳卒,子王之侯立。王之侯卒,子王無彊立。

王無彊時,越興師北伐齊,西伐楚,與中國爭彊。當楚威王之時,越北伐齊,齊威王使人說越王曰:「越不伐楚,大不王,小不伯。圖越之所為不伐楚者,為不得晉也。韓、魏固不攻楚。韓之攻楚,覆其軍,殺其將,則葉、陽翟危;魏亦覆其軍,殺其將,則陳、上蔡不安。故二晉之事越也,不至於覆軍殺將,馬汗之力不效。所重於得晉者何也?」越王曰:「所求於晉者,不至頓刃接兵,而況于攻城圍邑乎?願魏以聚大梁之下,願齊之試兵南陽莒地,以聚常、郯之境,則方城之外不南,淮、泗之閒不東,商、於、析、酈、宗胡之地,夏路以左,不足以備秦,江南、泗上不足以待越矣。則齊、秦、韓、魏得志於楚也,是二晉不戰分地,不耕而穫之。不此之為,而頓刃於河山之閒以為齊秦用,所待者如此其失計,柰何其以此王也!」齊使者曰:「幸也越之不亡也!吾不貴其用智之如目,見豪毛而不見其睫也。今王知晉之失計,而不自知越之過,是目論也。王所待於晉者,非有馬汗之力也,又非可與合軍連和也,將待之以分楚眾也。今楚眾已分,何待於晉?」越王曰:「柰何?」曰:「楚三大夫張九軍,北圍曲沃、於中,以至無假之關者三千七百里,景翠之軍北聚魯、齊、南陽,分有大此者乎?且王之所求者,鬬晉楚也;晉楚不鬬,越兵不起,是知二五而不知十也。此時不攻楚,臣以是知越大不王,小不伯。復讎、龐、長沙,楚之粟也;竟澤陵,楚之材也。越窺兵通無假之關,此四邑者不上貢事於郢矣。臣聞之,圖王不王,其敝可以伯。然而不伯者,王道失也。故願大王之轉攻楚也。」

於是越遂釋齊而伐楚。楚威王興兵而伐之,大敗越,殺王無彊,盡取笔吳地至浙江,北破齊於徐州。而越以此散,諸族子爭立,或為王,或為君,濱於江南海上,服朝於楚。

後七世,至閩君搖,佐諸侯平秦。漢高帝復以搖為越王,以奉越後。東越,閩君,皆其後也。

范蠡事越王句踐,既苦身力,與句踐深謀二十餘年,竟滅吳,報會稽之恥,北渡兵於淮以臨齊、晉,號令中國,以尊周室,句踐以霸,而范蠡稱上將軍。還反國,范蠡以為大名之下,難以久居,且句踐為人可與同患,難與處安,為書辭句踐曰:「臣聞主憂臣勞,主辱臣死。昔者君王辱於會稽,所以不死,為此事也。今既以雪恥,臣請從會稽之誅。」句踐曰:「孤將與子分國而有之。不然,將加誅于子。」范蠡曰:「君行令,臣行意。」乃裝其輕寶珠玉,自與其私徒屬乘舟浮海以行,終不反。於是句踐表會稽山以為范蠡奉邑。

范蠡浮海出齊,變姓名,自謂鴟夷子皮,耕于海畔,苦身戮力,父子治產。居無幾何,致產數十萬。齊人聞其賢,以為相。范蠡喟然嘆曰:「居家則致千金,居官則至卿相,此布衣之極也。久受尊名,不祥。」乃歸相印,盡散其財,以分與知友鄉黨,而懷其重寶,閒行以去,止于陶,以為此天下之中,交易有無之路通,為生可以致富矣。於是自謂陶朱公。復約要父子耕畜,廢居,候時轉物,逐什一之利。居無何,則致貲累巨萬。天下稱陶朱公。

朱公居陶,生少子。少子及壯,而朱公中男殺人,囚於楚。朱公曰:「殺人而死,職也。然吾聞千金之子不死於市。」告其少子往視之。乃裝黃金千溢,置褐器中,載以一牛車。且遣其少子,朱公長男固請欲行,朱公不聽。長男曰:「家有長子曰家督,今弟有罪,大人不遣,乃遺少弟,是吾不肖。」欲自殺。其母為言曰:「今遣少子,未必能生中子也,而先空亡長男,柰何?」朱公不得已而遣長子,為一封書遺故所善莊生。曰:「至則進千金于莊生所,聽其所為,慎無與爭事。」長男既行,亦自私齎數百金。

至楚,莊生家負郭,披藜藋到門,居甚貧。然長男發書進千金,如其父言。莊生曰:「可疾去矣,慎毋留!即弟出,勿問所以然。」長男既去,不過莊生而私留,以其私齎獻遺楚國貴人用事者。

莊生雖居窮閻,然以廉直聞於國,自楚王以下皆師尊之。及朱公進金,非有意受也,欲以成事後復歸之以為信耳。故金至,謂其婦曰:「此朱公之金。有如病不宿誡,後復歸,勿動。」而朱公長男不知其意,以為殊無短長也。

莊生閒時入見楚王,言「某星宿某,此則害於楚」。楚王素信莊生,曰:「今為柰何?」莊生曰:「獨以德為可以除之。」楚王曰:「生休矣,寡人將行之。」王乃使使者封三錢之府。楚貴人驚告朱公長男曰:「王且赦。」曰:「何以也?」曰:「每王且赦,常封三錢之府。昨暮王使使封之。」朱公長男以為赦,弟固當出也,重千金虛棄莊生,無所為也,乃復見莊生。莊生驚曰:「若不去邪?」長男曰:「固未也。初為事弟,弟今議自赦,故辭生去。」莊生知其意欲復得其金,曰:「若自入室取金。」長男即自入室取金持去,獨自歡幸。

莊生羞為兒子所賣,乃入見楚王曰:「臣前言某星事,王言欲以修德報之。今臣出,道路皆言陶之富人朱公之子殺人囚楚,其家多持金錢賂王左右,故王非能恤楚國而赦,乃以朱公子故也。」楚王大怒曰:「寡人雖不德耳,柰何以朱公之子故而施惠乎!」令論殺朱公子,明日遂下赦令。朱公長男竟持其弟喪歸。

至,其母及邑人盡哀之,唯朱公獨笑,曰:「吾固知必殺其弟也!彼非不愛其弟,顧有所不能忍者也。是少與我俱,見苦,為生難,故重棄財。至如少弟者,生而見我富,乘堅驅良逐狡兔,豈知財所從來,故輕棄之,非所惜吝。前日吾所為欲遣少子,固為其能棄財故也。而長者不能,故卒以殺其弟,事之理也,無足悲者。吾日夜固以望其喪之來也。」

故范蠡三徙,成名於天下,非茍去而已,所止必成名。卒老死于陶,故世傳曰陶朱公。

太史公曰:禹之功大矣,漸九川,定九州,至于今諸夏艾安。及苗裔句踐,苦身焦思,終滅彊吳,北觀兵中國,以尊周室,號稱霸王。句踐可不謂賢哉!蓋有禹之遺烈焉。范蠡三遷皆有榮名,名垂後世。臣主若此,欲毋顯得乎!


\end{pinyinscope}