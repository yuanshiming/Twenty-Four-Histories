\article{曹相國世家}

\begin{pinyinscope}
平陽侯曹參者,沛人也。秦時為沛獄掾,而蕭何為主吏,居縣為豪吏矣。

高祖為沛公而初起也,參以中涓從。將擊胡陵、方與,攻秦監公軍,大破之。東下薛,擊泗水守軍薛郭西。復攻胡陵,取之。徙守方與。方與反為魏,擊之。豐反為魏,攻之。賜爵七大夫。擊秦司馬夷軍碭東,破之,取碭、狐父、祁善置。又攻下邑以西,至虞,擊章邯車騎。攻爰戚及亢父,先登。遷為五大夫。北救阿,擊章邯軍,陷陳,追至濮陽。攻定陶,取臨濟。南救雍丘。擊李由軍,破之,殺李由,虜秦候一人。秦將章邯破殺項梁也,沛公與項羽引而東。楚懷王以沛公為碭郡長,將碭郡兵。於是乃封參為執帛,號曰建成君。遷為戚公,屬碭郡。

其後從攻東郡尉軍,破之成武南。擊王離軍成陽南,復攻之杠裏,大破之。追北,西至開封,擊趙賁軍,破之,圍趙賁開封城中。西擊將楊熊軍於曲遇,破之,虜秦司馬及御史各一人。遷為執珪。從攻陽武,下軒轅、緱氏,絕河津,還擊趙賁軍尸北,破之。從南攻犨,與南陽守齮戰陽城郭東,陷陳,取宛,虜齮,盡定南陽郡。從西攻武關、峣關,取之。前攻秦軍藍田南,又夜擊其北,秦軍大破,遂至咸陽,滅秦。

項羽至,以沛公為漢王。漢王封參為建成侯。從至漢中,遷為將軍。從還定三秦,初攻下辯、故道、雍、斄。擊章平軍於好畤南,破之,圍好畤,取壤鄉。擊三秦軍壤東及高櫟,破之。復圍章平,章平出好畤走。因擊趙賁、內史保軍,破之。東取咸陽,更名曰新城。參將兵守景陵二十日,三秦使章平等攻參,參出擊,大破之。賜食邑於寧秦。參以將軍引兵圍章邯於廢丘。以中尉從漢王出臨晉關。至河內,下修武,渡圍津,東擊龍且、項他定陶,破之。東取碭、蕭、彭城。擊項籍軍,漢軍大敗走。參以中尉圍取雍丘。王武反於[外]黃,程處反於燕,往擊,盡破之。柱天侯反於衍氏,又進破取衍氏。擊羽嬰於昆陽,追至葉。還攻武彊,因至滎陽。參自漢中為將軍中尉,從擊諸侯,及項羽敗,還至滎陽,凡二歲。

高祖(三)[二]年,拜為假左丞相,入屯兵關中。月餘,魏王豹反,以假左丞相別與韓信東攻魏將軍孫遬軍東張,大破之。因攻安邑,得魏將王襄。擊魏王於曲陽,追至武垣,生得魏王豹。取平陽,得魏王母妻子,盡定魏地,凡五十二城。賜食邑平陽。因從韓信擊趙相國夏說軍於鄔東,大破之,斬夏說。韓信與故常山王張耳引兵下井陘,擊成安君,而令參還圍趙別將戚將軍於鄔城中。戚將軍出走,追斬之。乃引兵詣敖倉漢王之所。韓信已破趙,為相國,東擊齊。參以右丞相屬韓信,攻破齊歷下軍,遂取臨菑。還定濟北郡,攻著、漯陰、平原、鬲、盧。已而從韓信擊龍且軍於上假密,大破之,斬龍且,虜其將軍周蘭。定齊,凡得七十餘縣。得故齊王田廣相田光,其守相許章,及故齊膠東將軍田既。韓信為齊王,引兵詣陳,與漢王共破項羽,而參留平齊未服者。

項籍已死,天下定,漢王為皇帝,韓信徙為楚王,齊為郡。參歸漢相印。高帝以長子肥為齊王,而以參為齊相國。以高祖六年賜爵列侯,與諸侯剖符,世世勿絕。食邑平陽萬六百三十戶,號曰平陽侯,除前所食邑。

以齊相國擊陳豨將張春軍,破之。黥布反,參以齊相國從悼惠王將兵車騎十二萬人,與高祖會擊黥布軍,大破之。南至蘄,還定竹邑、相、蕭、留。

參功:凡下二國,縣一百二十二;得王二人,相三人,將軍六人,大莫敖、郡守、司馬、候、御史各一人。

孝惠帝元年,除諸侯相國法,更以參為齊丞相。參之相齊,齊七十城。天下初定,悼惠王富於春秋,參盡召長老諸生,問所以安集百姓,如齊故[俗]諸儒以百數,言人人殊,參未知所定。聞膠西有蓋公,善治黃老言,使人厚幣請之。既見蓋公,蓋公為言治道貴清靜而民自定,推此類具言之。參於是避正堂,舍蓋公焉。其治要用黃老術,故相齊九年,齊國安集,大稱賢相。

惠帝二年,蕭何卒。參聞之,告舍人趣治行,「吾將入相」。居無何,使者果召參。參去,屬其後相曰:「以齊獄市為寄,慎勿擾也。」後相曰:「治無大於此者乎?」參曰:「不然。夫獄市者,所以并容也,今君擾之,姦人安所容也?吾是以先之。」

參始微時,與蕭何善;及為將相,有卻。至何且死,所推賢唯參。參代何為漢相國,舉事無所變更,一遵蕭何約束。

擇郡國吏木詘於文辭,重厚長者,即召除為丞相史。吏之言文刻深,欲務聲名者,輒斥去之。日夜飲醇酒。卿大夫已下吏及賓客見參不事事,來者皆欲有言。至者,參輒飲以醇酒,閒之,欲有所言,復飲之,醉而後去,終莫得開說,以為常。

相舍後園近吏舍,吏舍日飲歌呼。從吏惡之,無如之何,乃請參游園中,聞吏醉歌呼,從吏幸相國召按之。乃反取酒張坐飲,亦歌呼與相應和。

參見人之有細過,專掩匿覆蓋之,府中無事。

參子窋為中大夫。惠帝怪相國不治事,以為「豈少朕與」?乃謂窋曰:「若歸,試私從容問而父曰:『高帝新棄群臣,帝富於春秋,君為相,日飲,無所請事,何以憂天下乎?』然無言吾告若也。」窋既洗沐歸,窋侍,自從其所諫參。參怒,而笞窋二百,曰:「趣入侍,天下事非若所當言也。」至朝時,惠帝讓參曰:「與窋胡治乎?乃者我使諫君也。」參免冠謝曰:「陛下自察聖武孰與高帝?」上曰:「朕乃安敢望先帝乎!」曰:「陛下觀臣能孰與蕭何賢?」上曰:「君似不及也。」參曰:「陛下言之是也。且高帝與蕭何定天下,法令既明,今陛下垂拱,參等守職,遵而勿失,不亦可乎?」惠帝曰:「善。君休矣!」

參為漢相國,出入三年。卒,謚懿侯。子窋代侯。百姓歌之曰:「蕭何為法,顜若畫一;曹參代之,守而勿失。載其清凈,民以寧一。」

平陽侯窋,高后時為御史大夫。孝文帝立,免為侯。立二十九年卒,謚為靜侯。子奇代侯,立七年卒,謚為簡侯。子時代侯。時尚平陽公主,生子襄。時病癘,歸國。立二十三年卒,謚夷侯。子襄代侯。襄尚衛長公主,生子宗。立十六年卒,謚為共侯。子宗代侯。征和二年中,宗坐太子死,國除。

太史公曰:曹相國參攻城野戰之功所以能多若此者,以與淮陰侯俱。及信已滅,而列侯成功,唯獨參擅其名。參為漢相國,清靜極言合道。然百姓離秦之酷後,參與休息無為,故天下俱稱其美矣。


\end{pinyinscope}