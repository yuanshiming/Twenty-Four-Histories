\article{孝景本紀}

\begin{pinyinscope}
孝景皇帝者,孝文之中子也。母竇太后。孝文在代時,前后有三男,及竇太后得幸,前后死,及三子更死,故孝景得立。

元年四月乙卯,赦天下。乙巳,賜民爵一級。五月,除田半租,為孝文立太宗廟。令群臣無朝賀。匈奴入代,與約和親。

二年春,封故相國蕭何孫系為武陵侯。男子二十而得傅。四月壬午,孝文太后崩。廣川、長沙王皆之國。丞相申屠嘉卒。八月,以御史大夫開封陶青為丞相。彗星出東北。秋,衡山雨雹,大者五寸,深者二尺。熒惑逆行,守北辰。月出北辰閒。歲星逆行天廷中。置南陵及內史、祋祤為縣。

三年正月乙巳,赦天下。長星出西方。天火燔雒陽東宮大殿城室。吳王濞、楚王戊、趙王遂、膠西王卬、濟南王辟光、菑川王賢、膠東王雄渠反,發兵西鄉。天子為誅晁錯,遣袁盎諭告,不止,遂西圍梁。上乃遣大將軍竇嬰、太尉周亞夫將兵誅之。六月乙亥。赦亡軍及楚元王子藝等與謀反者。封大將軍竇嬰為魏其侯。立楚元王子平陸侯禮為楚王。立皇子端為膠西王,子勝為中山王。徙濟北王志為菑川王,淮陽王餘為魯王,汝南王非為江都王。齊王將廬、燕王嘉皆薨。

四年夏,立太子。立皇子徹為膠東王。六月甲戌,赦天下。後九月,更以(弋)[易]陽為陽陵。復置津關,用傳出入。冬,以趙國為邯鄲郡。

五年三月,作陽陵、渭橋。五月,募徙陽陵,予錢二十萬。江都大暴風從西方來,壞城十二丈。丁卯,封長公主子蟜為隆慮侯。徙廣川王為趙王。

六年春,封中尉(趙)綰為建陵侯,江都丞相嘉為建平侯,隴西太守渾邪為平曲侯,趙丞相嘉為江陵侯,故將軍布為鄃侯。梁楚二王皆薨。後九月,伐馳道樹,殖蘭池。

七年冬,廢栗太子為臨江王。(一)十(二)[一]月晦,日有食之。春,免徒隸作陽陵者。丞相青免。二月乙巳,以太尉條侯周亞夫為丞相。四月乙巳,立膠東王太后為皇后。丁巳,立膠東王為太子。名徹。

中元年,封故御史大夫周苛孫平為繩侯,故御史大夫周昌(子)[孫]左車為安陽侯,四月乙巳,赦天下,賜爵一級。除禁錮。地動。衡山、原都雨雹,大者尺八寸。

中二年二月,匈奴入燕,遂不和親。三月,召臨江王來。即死中尉府中。夏,立皇子越為廣川王,子寄為膠東王。封四侯。九月甲戌,日食。

中三年冬,罷諸侯御史中丞。春,匈奴王二人率其徒來降,皆封為列侯。

中四年三月,置德陽宮。大蝗。秋,赦徒作陽陵者。

中五年夏,立皇子舜為常山王。封十侯。六月丁巳,赦天下,賜爵一級。天下大潦。更命諸侯丞相曰相。秋,地動。

中六年二月己卯,行幸雍,郊見五帝。三月,雨雹。四月,梁孝王、城陽共王、汝南王皆薨。立梁孝王子明為濟川王,子彭離為濟東王,子定為山陽王,子不識為濟陰王。梁分為五。封四侯。更命廷尉為大理,將作少府為將作大匠,主爵中尉為都尉,長信詹事為長信少府,將行為大長秋,大行為行人,奉常為太常,典客為大行,治粟內史為大農。以大內為二千石,置左右內官,屬大內。七月辛亥,日食。八月,匈奴入上郡。

後元年冬,更命中大夫令為衛尉。三月丁酉,赦天下,賜爵一級,中二千石、諸侯相爵右庶長。四月,大酺。五月丙戌,地動,其蚤食時復動。上庸地動二十二日,壞城垣。七月乙巳,日食。丞相劉舍免。八月壬辰,以御史大夫綰為丞相,封建陵侯。

後二年正月,地一日三動。郅將軍擊匈奴。酺五日。令內史郡不得食馬粟,沒入縣官。令徒隸衣七緵布。止馬舂。為歲不登,禁天下食不造歲。省列侯遣之國。三月,匈奴入雁門。十月,租長陵田。大旱。衡山國、河東、雲中郡民疫。

後三年十月,日月皆(食)赤五日。十二月晦,雷。日如紫。五星逆行守太微。月貫天廷中。正月甲寅,皇太子冠。甲子,孝景皇帝崩。遺詔賜諸侯王以下至民為父後爵一級,天下戶百錢。出宮人歸其家,復無所與。太子即位,是為孝武皇帝。三月,封皇太后弟蚡為武安侯,弟勝為周陽侯。置陽陵。

太史公曰:漢興,孝文施大德,天下懷安,至孝景,不復憂異姓,而晁錯刻削諸侯,遂使七國俱起,合從而西鄉,以諸侯太盛,而錯為之不以漸也。及主父偃言之,而諸侯以弱,卒以安。安危之機,豈不以謀哉?


\end{pinyinscope}