\article{管晏列傳}

\begin{pinyinscope}
管仲夷吾者,潁上人也。少時常與鮑叔牙游,鮑叔知其賢。管仲貧困,常欺鮑叔,鮑叔終善遇之,不以為言。已而鮑叔事齊公子小白,管仲事公子糾。及小白立為桓公,公子糾死,管仲囚焉。鮑叔遂進管仲。管仲既用,任政於齊,齊桓公以霸,九合諸侯,一匡天下,管仲之謀也。

管仲曰:「吾始困時,嘗與鮑叔賈,分財利多自與,鮑叔不以我為貪,知我貧也。吾嘗為鮑叔謀事而更窮困,鮑叔不以我為愚,知時有利不利也。吾嘗三仕三見逐於君,鮑叔不以我為不肖,知我不遭時也。吾嘗三戰三走,鮑叔不以我怯,知我有老母也。公子糾敗,召忽死之,吾幽囚受辱,鮑叔不以我為無恥,知我不羞小睗而恥功名不顯于天下也。生我者父母,知我者鮑子也。」

鮑叔既進管仲,以身下之。子孫世祿於齊,有封邑者十餘世,常為名大夫。天下不多管仲之賢而多鮑叔能知人也。

管仲既任政相齊,以區區之齊在海濱,通貨積財,富國彊兵,與俗同好惡。故其稱曰:「倉廩實而知禮節,衣食足而知榮辱,上服度則六親固。四維不張,國乃滅亡。下令如流水之原,令順民心。」故論卑而易行。俗之所欲,因而予之;俗之所否,因而去之。

其為政也,善因禍而為福,轉敗而為功。貴輕重,慎權衡。桓公實怒少姬,南襲蔡,管仲因而伐楚,責包茅不入貢於周室。桓公實北征山戎,而管仲因而令燕修召公之政。於柯之會,桓公欲背曹沫之約,管仲因而信之,諸侯由是歸齊。故曰:「知與之為取,政之寶也。」

管仲富擬於公室,有三歸、反坫,齊人不以為侈。管仲卒,齊國遵其政,常彊於諸侯。後百餘年而有晏子焉。

晏平仲嬰者,萊之夷維人也。事齊靈公、莊公、景公,以節儉力行重於齊。既相齊,食不重肉,妾不衣帛。其在朝,君語及之,即危言;語不及之,即危行。國有道,即順命;無道,即衡命。以此三世顯名於諸侯。

越石父賢,在縲紲中。晏子出,遭之涂,解左驂贖之,載歸。弗謝,入閨。久之,越石父請絕。晏子懼然,攝衣冠謝曰:「嬰雖不仁,免子於緦何子求絕之速也?」石父曰:「不然。吾聞君子詘於不知己而信於知己者。方吾在縲紲中,彼不知我也。夫子既已感寤而贖我,是知己;知己而無禮,固不如在縲紲之中。」晏子於是延入為上客。

晏子為齊相,出,其御之妻從門閒而闚其夫。其夫為相御,擁大蓋,策駟馬,意氣揚揚甚自得也。既而歸,其妻請去。夫問其故。妻曰:「晏子長不滿六尺,身相齊國,名顯諸侯。今者妾觀其出,志念深矣,常有以自下者。今子長八尺,乃為人仆御,然子之意自以為足,妾是以求去也。」其後夫自抑損。晏子怪而問之,御以實對。晏子薦以為大夫。

太史公曰:吾讀管氏牧民、山高、乘馬、輕重、九府,及晏子春秋,詳哉其言之也。既見其著書,欲觀其行事,故次其傳。至其書,世多有之,是以不論,論其軼事。

管仲世所謂賢臣,然孔子小之。豈以為周道衰微,桓公既賢,而不勉之至王,乃稱霸哉?語曰「將順其美,匡救其惡,故上下能相親也」。豈管仲之謂乎?

方晏子伏莊公尸哭之,成禮然後去,豈所謂「見義不為無勇」者邪?至其諫說,犯君之顏,此所謂「進思盡忠,退思補過」者哉!假令晏子而在,余雖為之執鞭,所忻慕焉。


\end{pinyinscope}