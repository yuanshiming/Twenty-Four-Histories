\article{滑稽列傳}

\begin{pinyinscope}
孔子曰:「六藝於治一也。禮以節人,樂以發和,書以道事,詩以達意,易以神化,春秋以義。」太史公曰:天道恢恢,豈不大哉!談言微中,亦可以解紛。

淳于髡者,齊之贅婿也。長不滿七尺,滑稽多辯,數使諸侯,未嘗屈辱。齊威王之時喜隱,好為淫樂長夜之飲,沈湎不治,委政卿大夫。百官荒亂,諸侯并侵,國且危亡,在於旦暮,左右莫敢諫。淳于髡說之以隱曰:「國中有大鳥,止王之庭,三年不蜚又不鳴,不知此鳥何也?」王曰:「此鳥不飛則已,一飛沖天;不鳴則已,一鳴驚人。」於是乃朝諸縣令長七十二人,賞一人,誅一人,奮兵而出。諸侯振驚,皆還齊侵地。威行三十六年。語在田完世家中。

威王八年,楚大發兵加齊。齊王使淳于髡之趙請救兵,齎金百斤,車馬十駟。淳于髡仰天大笑,冠纓索絕。王曰:「先生少之乎?」髡曰:「何敢!」王曰:「笑豈有說乎?」髡曰:「今者臣從東方來,見道傍有禳田者,操一豚蹄,酒一盂,祝曰:『甌窶滿篝,汙邪滿車,五穀蕃熟,穰穰滿家。』臣見其所持者狹而所欲者奢,故笑之。」於是齊威王乃益齎黃金千溢,白璧十雙,車馬百駟。髡辭而行,至趙。趙王與之精兵十萬,革車千乘。楚聞之,夜引兵而去。

威王大說,置酒後宮,召髡賜之酒。問曰:「先生能飲幾何而醉?」對曰:「臣飲一斗亦醉,一石亦醉。」威王曰:「先生飲一斗而醉,惡能飲一石哉!其說可得聞乎?」髡曰:「賜酒大王之前,執法在傍,御史在後,髡恐懼俯伏而飲,不過一斗徑醉矣。若親有嚴客,髡帣韝鞠跽,待酒於前,時賜餘瀝,奉觴上壽,數起,飲不過二斗徑醉矣。若朋友交遊,久不相見,卒然相睹,歡然道故,私情相語,飲可五六斗徑醉矣。若乃州閭之會,男女雜坐,行酒稽留,六博投壺,相引為曹,握手無罰,目眙不禁,前有墮珥,后有遺簪,髡竊樂此,飲可八斗而醉二參。日暮酒闌,合尊促坐,男女同席,履舄交錯,杯盤狼藉,堂上燭滅,主人留髡而送客,羅襦襟解,微聞薌澤,當此之時,髡心最歡,能飲一石。故曰酒極則亂,樂極則悲;萬事盡然,言不可極,極之而衰。」以諷諫焉。齊王曰:「善。」乃罷長夜之飲,以髡為諸侯主客。宗室置酒,髡嘗在側。

其後百餘年,楚有優孟。

優孟,故楚之樂人也。長八尺,多辯,常以談笑諷諫。楚莊王之時,有所愛馬,衣以文繡,置之華屋之下,席以露床,啗以棗脯。馬病肥死,使群臣喪之,欲以棺槨大夫禮葬之。左右爭之,以為不可。王下令曰:「有敢以馬諫者,罪至死。」優孟聞之,入殿門。仰天大哭。王驚而問其故。優孟曰:「馬者王之所愛也,以楚國堂堂之大,何求不得,而以大夫禮葬之,薄,請以人君禮葬之。」王曰:「何如?」對曰:「臣請以彫玉為棺,文梓為槨,楩楓豫章為題湊,發甲卒為穿壙,老弱負土,齊趙陪位於前,韓魏翼衛其后,廟食太牢,奉以萬戶之邑。諸侯聞之,皆知大王賤人而貴馬也。」王曰:「寡人之過一至此乎!為之柰何?」優孟曰:「請為大王六畜葬之。以壟灶為槨,銅歷為棺,齎以薑棗,薦以木蘭,祭以糧稻,衣以火光,葬之於人腹腸。」於是王乃使以馬屬太官,無令天下久聞也。

楚相孫叔敖知其賢人也,善待之。病且死,屬其子曰:「我死,汝必貧困。若往見優孟,言我孫叔敖之子也。」居數年,其子窮困負薪,逢優孟,與言曰:「我,孫叔敖子也。父且死時,屬我貧困往見優孟。」優孟曰:「若無遠有所之。」即為孫叔敖衣冠,抵掌談語。歲餘,像孫叔敖,楚王及左右不能別也。莊王置酒,優孟前為壽。莊王大驚,以為孫叔敖復生也,欲以為相。優孟曰:「請歸與婦計之,三日而為相。」莊王許之。三日後,優孟復來。王曰:「婦言謂何?」孟曰:「婦言慎無為,楚相不足為也。如孫叔敖之為楚相,盡忠為廉以治楚,楚王得以霸。今死,其子無立錐之地,貧困負薪以自飲食。必如孫叔敖,不如自殺。」因歌曰:「山居耕田苦,難以得食。起而為吏,身貪鄙者餘財,不顧恥辱。身死家室富,又恐受賕枉法,為姦觸大罪,身死而家滅。貪吏安可為也!念為廉吏,奉法守職,竟死不敢為非。廉吏安可為也!楚相孫叔敖持廉至死,方今妻子窮困負薪而食,不足為也!」於是莊王謝優孟,乃召孫叔敖子,封之寢丘四百戶,以奉其祀。后十世不絕。此知可以言時矣。

其後二百餘年,秦有優旃。

優旃者,秦倡侏儒也。善為笑言,然合於大道,秦始皇時,置酒而天雨,陛楯者皆沾寒。優旃見而哀之,謂之曰:「汝欲休乎?」陛楯者皆曰:「幸甚。」優旃曰:「我即呼汝,汝疾應曰諾。」居有頃,殿上上壽呼萬歲。優旃臨檻大呼曰:「陛楯郎!」郎曰:「諾。」優旃曰:「汝雖長,何益,幸雨立。我雖短也,幸休居。」於是始皇使陛楯者得半相代。

始皇嘗議欲大苑囿,東至函谷關,西至雍、陳倉。優旃曰:「善。多縱禽獸於其中,寇從東方來,令麋鹿觸之足矣。」始皇以故輟止。

二世立,又欲漆其城。優旃曰:「善。主上雖無言,臣固將請之。漆城雖於百姓愁費,然佳哉!漆城蕩蕩,寇來不能上。即欲就之,易為漆耳,顧難為蔭室。」於是二世笑之,以其故止。居無何,二世殺死,優旃歸漢,數年而卒。

太史公曰:淳于髡仰天大笑,齊威王橫行。優孟搖頭而歌,負薪者以封。優旃臨檻疾呼,陛楯得以半更。豈不亦偉哉!

褚先生曰:臣幸得以經術為郎,而好讀外家傳語。竊不遜讓,復作故事滑稽之語六章,編之於左。可以覽觀揚意,以示後世好事者讀之,以游心駭耳,以附益上方太史公之三章。

武帝時有所幸倡郭舍人者,發言陳辭雖不合大道,然令人主和說。武帝少時,東武侯母常養帝,帝壯時,號之曰「大乳母」。率一月再朝。朝奏入,有詔使幸臣馬游卿以帛五十匹賜乳母,又奉飲糒飱養乳母。乳母上書曰:「某所有公田,願得假倩之。」帝曰:「乳母欲得之乎?」以賜乳母。乳母所言,未嘗不聽。有詔得令乳母乘車行馳道中。當此之時,公卿大臣皆敬重乳母。乳母家子孫奴從者橫暴長安中,當道掣頓人車馬,奪人衣服。聞於中,不忍致之法。有司請徙乳母家室,處之於邊。奏可。乳母當入至前,面見辭。乳母先見郭舍人,為下泣。舍人曰:「即入見辭去,疾步數還顧。」乳母如其言,謝去,疾步數還顧。郭舍人疾言罵之曰:「咄!老女子!何不疾行!陛下已壯矣,寧尚須汝乳而活邪?尚何還顧!」於是人主憐焉悲之,乃下詔止無徙乳母,罰謫譖之者。

武帝時,齊人有東方生名朔,以好古傳書,愛經術,多所博觀外家之語。朔初入長安,至公車上書,凡用三千奏牘。公車令兩人共持舉其書,僅然能勝之。人主從上方讀之,止,輒乙其處,讀之二月乃盡。詔拜以為郎,常在側侍中。數召至前談語,人主未嘗不說也。時詔賜之食於前。飯已,盡懷其餘肉持去,衣盡汙。數賜縑帛,檐揭而去。徒用所賜錢帛,取少婦於長安中好女。率取婦一歲所者即棄去,更取婦。所賜錢財盡索之於女子。人主左右諸郎半呼之「狂人」。人主聞之,曰:「令朔在事無為是行者,若等安能及之哉!」朔任其子為郎,又為侍謁者,常持節出使。朔行殿中,郎謂之曰:「人皆以先生為狂。」朔曰:「如朔等,所謂避世於朝廷閒者也。古之人,乃避世於深山中。」時坐席中,酒酣,據地歌曰:「陸沈於俗,避世金馬門。宮殿中可以避世全身,何必深山之中,蒿廬之下。」金馬門者,宦[者]署門也,門傍有銅馬,故謂之曰「金馬門」。

時會聚宮下博士諸先生與論議,共難之曰:「蘇秦、張儀一當萬乘之主,而都卿相之位,澤及後世。今子大夫修先王之術,慕聖人之義,諷誦詩書百家之言,不可勝數。著於竹帛,自以為海內無雙,即可謂博聞辯智矣。然悉力盡忠以事聖帝,曠日持久,積數十年,官不過侍郎,位不過執戟,意者尚有遺行邪?其故何也?」東方生曰:「是固非子所能備也。彼一時也,此一時也,豈可同哉!夫張儀、蘇秦之時,周室大壞,諸侯不朝,力政爭權,相禽以兵,并為十二國,未有雌雄,得士者彊,失士者亡,故說聽行通,身處尊位,澤及後世,子孫長榮。今非然也。聖帝在上,德流天下,諸侯賓服,威振四夷,連四海之外以為席,安於覆盂,天下平均,合為一家,動發舉事,猶如運之掌中。賢與不肖,何以異哉?方今以天下之大,士民之眾,竭精馳說,并進輻湊者,不可勝數。悉力慕義,困於衣食,或失門戶。使張儀、蘇秦與仆并生於今之世,曾不能得掌故,安敢望常侍侍郎乎!傳曰:『天下無害菑,雖有聖人,無所施其才;上下和同,雖有賢者,無所立功。』故曰時異則事異。雖然,安可以不務修身乎?《詩》曰:『鼓鐘于宮,聲聞于外。鶴鳴九皋,聲聞于天。』。茍能修身,何患不榮!太公躬行仁義七十二年,逢文王,得行其說,封於齊,七百歲而不絕。此士之所以日夜孜孜,修學行道,不敢止也。今世之處士,時雖不用,崛然獨立,塊然獨處,上觀許由,下察接輿,策同范蠡,忠合子胥,天下和平,與義相扶,寡偶少徒,固其常也。子何疑於余哉!」於是諸先生默然無以應也。

建章宮後閤重櫟中有物出焉,其狀似麋。以聞,武帝往臨視之。問左右群臣習事通經術者,莫能知。詔東方朔視之。朔曰:「臣知之,願賜美酒粱飯大飱臣,臣乃言。」詔曰:「可。」已又曰:「某所有公田魚池蒲葦數頃,陛下以賜臣,臣朔乃言。」詔曰:「可。」於是朔乃肯言,曰:「所謂騶牙者也。遠方當來歸義,而騶牙先見。其齒前后若一,齊等無牙,故謂之騶牙。」其後一歲所,匈奴混邪王果將十萬眾來降漢。乃復賜東方生錢財甚多。

至老,朔且死時,諫曰:「《詩》云『營營青蠅,止于蕃。愷悌君子,無信讒言。讒言罔極,交亂四國』。願陛下遠巧佞,退讒言。」帝曰:「今顧東方朔多善言?」怪之。居無幾何,朔果病死。傳曰:「鳥之將死,其鳴也哀;人之將死,其言也善。」此之謂也。

武帝時,大將軍衛青者,衛后兄也,封為長平侯。從軍擊匈奴,至余吾水上而還,斬首捕虜,有功來歸,詔賜金千斤。將軍出宮門,齊人東郭先生以方士待詔公車,當道遮衛將軍車,拜謁曰:「願白事。」將軍止車前,東郭先生旁車言曰:「王夫人新得幸於上,家貧。今將軍得金千斤,誠以其半賜王夫人之親,人主聞之必喜。此所謂奇策便計也。」衛將軍謝之曰:「先生幸告之以便計,請奉教。」於是衛將軍乃以五百金為王夫人之親壽。王夫人以聞武帝。帝曰:「大將軍不知為此。」問之安所受計策,對曰:「受之待詔者東郭先生。」詔召東郭先生,拜以為郡都尉。東郭先生久待詔公車,貧困饑寒,衣敝,履不完。行雪中,履有上無下,足盡踐地。道中人笑之,東郭先生應之曰:「誰能履行雪中,令人視之,其上履也,其履下處乃似人足者乎?」及其拜為二千石,佩青緺出宮門,行謝主人。故所以同官待詔者,等比祖道於都門外。榮華道路,立名當世。此所謂衣褐懷寶者也。當其貧困時,人莫省視;至其貴也,乃爭附之。諺曰:「相馬失之瘦,相士失之貧。」其此之謂邪?

王夫人病甚,人主至自往問之曰:「子當為王,欲安所置之?」對曰:「願居洛陽。」人主曰:「不可。洛陽有武庫、敖倉,當關口,天下咽喉。自先帝以來,傳不為置王。然關東國莫大於齊,可以為齊王。」王夫人以手擊頭,呼「幸甚」。王夫人死,號曰「齊王太后薨」。

昔者,齊王使淳于髡獻鵠於楚。出邑門,道飛其鵠,徒揭空籠,造詐成辭,往見楚王曰:「齊王使臣來獻鵠,過於水上,不忍鵠之渴,出而飲之,去我飛亡。吾欲刺腹絞頸而死。恐人之議吾王以鳥獸之故令士自傷殺也。鵠,毛物,多相類者,吾欲買而代之,是不信而欺吾王也。欲赴佗國奔亡,痛吾兩主使不通。故來服過,叩頭受罪大王。」楚王曰:「善,齊王有信士若此哉!」厚賜之,財倍鵠在也。

武帝時,徵北海太守詣行在所。有文學卒史王先生者,自請與太守俱,「吾有益於君」,君許之。諸府掾功曹白云:「王先生嗜酒,多言少實,恐不可與俱。」太守曰:「先生意欲行,不可逆。」遂與俱。行至宮下,待詔宮府門。王先生徒懷錢沽酒,與衛卒仆射飲,日醉,不視其太守。太守入跪拜。王先生謂戶郎曰:「幸為我呼吾君至門內遙語。」戶郎為呼太守。太守來,望見王先生。王先生曰:「天子即問君何以治北海令無盜賊,君對曰何哉?」對曰:「選擇賢材,各任之以其能,賞異等,罰不肖。」王先生曰:「對如是,是自譽自伐功,不可也。願君對言,非臣之力,盡陛下神靈威武所變化也。」太守曰:「諾。」召入,至于殿下,有詔問之曰:「何於治北海,令盜賊不起?」叩頭對言:「非臣之力,盡陛下神靈威武之所變化也。」武帝大笑,曰:「於呼!安得長者之語而稱之!安所受之?」對曰:「受之文學卒史。」帝曰:「今安在?」對曰:「在宮府門外。」有詔召拜王先生為水衡丞,以北海太守為水衡都尉。傳曰:「美言可以市,尊行可以加人。君子相送以言,小人相送以財。」

魏文侯時,西門豹為鄴令。豹往到鄴,會長老,問之民所疾苦。長老曰:「苦為河伯娶婦,以故貧。」豹問其故,對曰:「鄴三老、廷掾常歲賦斂百姓,收取其錢得數百萬,用其二三十萬為河伯娶婦,與祝巫共分其餘錢持歸。當其時,巫行視小家女好者,云是當為河伯婦,即娉取。洗沐之,為治新繒綺縠衣,閒居齋戒;為治齋宮河上,張緹絳帷,女居其中。為具牛酒飯食,行十餘日。共粉飾之,如嫁女床席,令女居其上,浮之河中。始浮,行數十里乃沒。其人家有好女者,恐大巫祝為河伯取之,以故多持女遠逃亡。以故城中益空無人,又困貧,所從來久遠矣。民人俗語曰『即不為河伯娶婦,水來漂沒,溺其人民』云。」西門豹曰:「至為河伯娶婦時,願三老、巫祝、父老送女河上,幸來告語之,吾亦往送女。」皆曰:「諾。」

至其時,西門豹往會之河上。三老、官屬、豪長者、裏父老皆會,以人民往觀之者三二千人。其巫,老女子也,已年七十。從弟子女十人所,皆衣繒單衣,立大巫后。西門豹曰:「呼河伯婦來,視其好醜。」即將女出帷中,來至前。豹視之,顧謂三老、巫祝、父老曰:「是女子不好,煩大巫嫗為入報河伯,得更求好女,后日送之。」即使吏卒共抱大巫嫗投之河中。有頃,曰:「巫嫗何久也?弟子趣之!」復以弟子一人投河中。有頃,曰:「弟子何久也?復使一人趣之!」復投一弟子河中。凡投三弟子。西門豹曰:「巫嫗弟子是女子也,不能白事,煩三老為入白之。」復投三老河中。西門豹簪筆磬折,向河立待良久。長老、吏傍觀者皆驚恐。西門豹顧曰:「巫嫗、三老不來還,柰之何?」欲復使廷掾與豪長者一人入趣之。皆叩頭,叩頭且破,額血流地,色如死灰。西門豹曰:「諾,且留待之須臾。」須臾,豹曰:「廷掾起矣。狀河伯留客之久,若皆罷去歸矣。」鄴吏民大驚恐,從是以後,不敢復言為河伯娶婦。

西門豹即發民鑿十二渠,引河水灌民田,田皆溉。當其時,民治渠少煩苦,不欲也。豹曰:「民可以樂成,不可與慮始。今父老子弟雖患苦我,然百歲後期令父老子孫思我言。」至今皆得水利,民人以給足富。十二渠經絕馳道,到漢之立,而長吏以為十二渠橋絕馳道,相比近,不可。欲合渠水,且至馳道合三渠為一橋。鄴民人父老不肯聽長吏,以為西門君所為也,賢君之法式不可更也。長吏終聽置之。故西門豹為鄴令,名聞天下,澤流後世,無絕已時,幾可謂非賢大夫哉!

傳曰:「子產治鄭,民不能欺;子賤治單父,民不忍欺;西門豹治鄴,民不敢欺。」三子之才能誰最賢哉?辨治者當能別之。


\end{pinyinscope}