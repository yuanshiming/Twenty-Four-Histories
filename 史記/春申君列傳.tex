\article{春申君列傳}

\begin{pinyinscope}
春申君者,楚人也,名歇,姓黃氏。游學博聞,事楚頃襄王。頃襄王以歇為辯,使於秦。秦昭王使白起攻韓、魏,敗之於華陽,禽魏將芒卯,韓、魏服而事秦。秦昭王方令白起與韓、魏共伐楚,未行,而楚使黃歇適至於秦,聞秦之計。當是之時,秦已前使白起攻楚,取巫、黔中之郡,拔鄢郢,東至竟陵,楚頃襄王東徙治於陳縣。黃歇見楚懷王之為秦所誘而入朝,遂見欺,留死於秦。頃襄王,其子也,秦輕之,恐壹舉兵而滅楚。歇乃上書說秦昭王曰:

天下莫彊於秦、楚。今聞大王欲伐楚,此猶兩虎相與鬬。兩虎相與鬬而駑犬受其獘,不如善楚。臣請言其說:臣聞物至則反,冬夏是也;致至則危,累棋是也。今大國之地,遍天下有其二垂,此從生民已來,萬乘之地未嘗有也。先帝文王、莊王之身,三世不妄接地於齊,以絕從親之要。今王使盛橋守事於韓,盛橋以其地入秦,是王不用甲,不信威,而得百里之地。王可謂能矣。王又舉甲而攻魏,杜大梁之門,舉河內,拔燕、酸棗、虛、桃,入邢,魏之兵雲翔而不敢捄。王之功亦多矣。王休甲息眾,二年而後復之;又并蒲、衍、首、垣,以臨仁、平丘,黃、濟陽嬰城而魏氏服;王又割濮磿之北,注齊秦之要,絕楚趙之脊,天下五合六聚而不敢救。王之威亦單矣。

王若能持功守威,絀攻取之心而肥仁義之地,使無後患,三王不足四,五伯不足六也。王若負人徒之眾,仗兵革之彊,乘毀魏之威,而欲以力臣天下之主,臣恐其有後患也。《詩》曰「靡不有初,鮮克有終」。《易》曰「狐涉水,濡其尾」。此言始之易,終之難也。何以知其然也?昔智氏見伐趙之利而不知榆次之禍,吳見伐齊之便而不知干隧之敗。此二國者,非無大功也,沒利於前而易患於後也。吳之信越也,從而伐齊,既勝齊人於艾陵,還為越王禽三渚之浦。智氏之信韓、魏也,從而伐趙,攻晉陽城,勝有日矣,韓、魏叛之,殺智伯瑤於鑿臺之下。今王妒楚之不毀也,而忘毀楚之彊韓、魏也,臣為王慮而不取也。

《詩》曰「大武遠宅而不涉」。從此觀之,楚國,援也;鄰國,敵也。《詩》云「趯趯毚免,還犬獲之。他人有心,余忖度之」。今王中道而信韓、魏之善王也,此正吳之信越也。臣聞之,敵不可假,時不可失。臣恐韓、魏卑辭除患而實欲欺大國也。何則?王無重世之德於韓、魏,而有累世之怨焉。夫韓、魏父子兄弟接踵而死於秦者將十世矣。本國殘,社稷壞,宗廟毀。刳腹絕腸,折頸摺頤,首身分離,暴骸骨於草澤,頭顱僵仆,相望於境,父子老弱系脰束手為群虜者相及於路。鬼神孤傷,無所血食。人民不聊生,族類離散,流亡為仆妾者,盈滿海內矣。故韓、魏之不亡,秦社稷之憂也,今王資之與攻楚,不亦過乎!

且王攻楚將惡出兵?王將借路於仇讎之韓、魏乎?兵出之日而王憂其不返也,是王以兵資於仇讎之韓、魏也。王若不借路於仇讎之韓、魏,必攻隨水右壤。隨水右壤,此皆廣川大水,山林谿谷,不食之地也,王雖有之,不為得地。是王有毀楚之名而無得地之實也。

且王攻楚之日,四國必悉起兵以應王。秦、楚之兵構而不離,魏氏將出而攻留、方與、铚、湖陵、碭、蕭、相,故宋必盡。齊人南面攻楚,泗上必舉。此皆平原四達,膏腴之地,而使獨攻。王破楚以肥韓、魏於中國而勁齊。韓、魏之彊,足以校於秦。齊南以泗水為境,東負海,北倚河,而無後患,天下之國莫彊於齊、魏,齊、魏得地葆利而詳事下吏,一年之後,為帝未能,其於禁王之為帝有餘矣。

夫以王壤土之博,人徒之眾,兵革之彊,壹舉事而樹怨於楚,遲令韓、魏歸帝重於齊,是王失計也。臣為王慮,莫若善楚。秦、楚合而為一以臨韓,韓必斂手。王施以東山之險,帶以曲河之利,韓必為關內之侯。若是而王以十萬戍鄭,梁氏寒心,許、鄢陵嬰城,而上蔡、召陵不往來也,如此而魏亦關內侯矣。王壹善楚,而關內兩萬乘之主注地於齊,齊右壤可拱手而取也。王之地一經兩海,要約天下,是燕、趙無齊、楚,齊、楚無燕、趙也。然後危動燕、趙,直搖齊、楚,此四國者不待痛而服矣。

昭王曰:「善。」於是乃止白起而謝韓、魏。發使賂楚,約為與國。

黃歇受約歸楚,楚使歇與太子完入質於秦,秦留之數年。楚頃襄王病,太子不得歸。而楚太子與秦相應侯善,於是黃歇乃說應侯曰:「相國誠善楚太子乎?」應侯曰:「然。」歇曰:「今楚王恐不起疾,秦不如歸其太子。太子得立,其事秦必重而德相國無窮,是親與國而得儲萬乘也。若不歸,則咸陽一布衣耳;楚更立太子,必不事秦。夫失與國而絕萬乘之和,非計也。願相國孰慮之。」應侯以聞秦王。秦王曰:「令楚太子之傅先往問楚王之疾,返而後圖之。」黃歇為楚太子計曰:「秦之留太子也,欲以求利也。今太子力未能有以利秦也,歇憂之甚。而陽文君子二人在中,王若卒大命,太子不在,陽文君子必立為後,太子不得奉宗廟矣。不如亡秦,與使者俱出;臣請止,以死當之。」楚太子因變衣服為楚使者御以出關,而黃歇守舍,常為謝病。度太子已遠,秦不能追,歇乃自言秦昭王曰:「楚太子已歸,出遠矣。歇當死,願賜死。」昭王大怒,欲聽其自殺也。應侯曰:「歇為人臣,出身以徇其主,太子立,必用歇,故不如無罪而歸之,以親楚。」秦因遣黃歇。

歇至楚三月,楚頃襄王卒,太子完立,是為考烈王。考烈王元年,以黃歇為相,封為春申君,賜淮北地十二縣。後十五歲,黃歇言之楚王曰:「淮北地邊齊,其事急,請以為郡便。」因并獻淮北十二縣。請封於江東。考烈王許之。春申君因城故吳墟,以自為都邑。

春申君既相楚,是時齊有孟嘗君,趙有平原君,魏有信陵君,方爭下士,招致賓客,以相傾奪,輔國持權。

春申君為楚相四年,秦破趙之長平軍四十餘萬。五年,圍邯鄲。邯鄲告急於楚,楚使春申君將兵往救之,秦兵亦去,春申君歸。春申君相楚八年,為楚北伐滅魯,以荀卿為蘭陵令。當是時,楚復彊。

趙平原君使人於春申君,春申君捨之於上舍。趙使欲夸楚,為瑁簪,刀劍室以珠玉飾之,請命春申君客。春申君客三千餘人,其上客皆躡珠履以見趙使,趙使大慚。

春申君相十四年,秦莊襄王立,以呂不韋為相,封為文信侯。取東周。

春申君相二十二年,諸侯患秦攻伐無已時,乃相與合從,西伐秦,而楚王為從長,春申君用事。至函谷關,秦出兵攻,諸侯兵皆敗走。楚考烈王以咎春申君,春申君以此益疏。

客有觀津人朱英,謂春申君曰:「人皆以楚為彊而君用之弱,其於英不然。先君時善秦二十年而不攻楚,何也?秦踰黽隘之塞而攻楚,不便;假道於兩周,背韓、魏而攻楚,不可。今則不然,魏旦暮亡,不能愛許、鄢陵,其許魏割以與秦。秦兵去陳百六十里,臣之所觀者,見秦、楚之日斗也。」楚於是去陳徙壽春;而秦徙衛野王,作置東郡。春申君由此就封於吳,行相事。

楚考烈王無子,春申君患之,求婦人宜子者進之,甚眾,卒無子。趙人李園持其女弟,欲進之楚王,聞其不宜子,恐久毋寵。李園求事春申君為舍人,已而謁歸,故失期。還謁,春申君問之狀,對曰:「齊王使使求臣之女弟,與其使者飲,故失期。」春申君曰:「娉入乎?」對曰:「未也。」春申君曰:「可得見乎?」曰:「可。」於是李園乃進其女弟,即幸於春申君。知其有身,李園乃與其女弟謀。園女弟承閒以說春申君曰:「楚王之貴幸君,雖兄弟不如也。今君相楚二十餘年,而王無子,即百歲後將更立兄弟,則楚更立君後,亦各貴其故所親,君又安得長有寵乎?非徒然也,君貴用事久,多失禮於王兄弟,兄弟誠立,禍且及身,何以保相印江東之封乎?今妾自知有身矣,而人莫知。妾幸君未久,誠以君之重而進妾於楚王,王必幸妾;妾賴天有子男,則是君之子為王也,楚國盡可得,孰與身臨不測之罪乎?」春申君大然之,乃出李園女弟,謹舍而言之楚王。楚王召入幸之,遂生子男,立為太子,以李園女弟為王后。楚王貴李園,園用事。

李園既入其女弟,立為王后,子為太子,恐春申君語泄而益驕,陰養死士,欲殺春申君以滅口,而國人頗有知之者。

春申君相二十五年,楚考烈王病。朱英謂春申君曰:「世有毋望之福,又有毋望之禍。今君處毋望之世,事毋望之主,安可以無毋望之人乎?」春申君曰:「何謂毋望之福?」曰:「君相楚二十餘年矣,雖名相國,實楚王也。今楚王病,旦暮且卒,而君相少主,因而代立當國,如伊尹、周公,王長而反政,不即遂南面稱孤而有楚國?此所謂毋望之福也。」春申君曰:「何謂毋望之禍?」曰:「李園不治國而君之仇也,不為兵而養死士之日久矣,楚王卒,李園必先入據權而殺君以滅口。此所謂毋望之禍也。」春申君曰:「何謂毋望之人?」對曰:「君置臣郎中,楚王卒,李園必先入,臣為君殺李園。此所謂毋望之人也。」春申君曰:「足下置之,李園,弱人也,仆又善之,且又何至此!」朱英知言不用,恐禍及身,乃亡去。

後十七日,楚考烈王卒,李園果先入,伏死士於棘門之內。春申君入棘門,園死士俠刺春申君,斬其頭,投之棘門外。於是遂使吏盡滅春申君之家。而李園女弟初幸春申君有身而入之王所生子者遂立,是為楚幽王。

是歲也,秦始皇帝立九年矣。嫪毐亦為亂於秦,覺,夷其三族,而呂不韋廢。

太史公曰:吾適楚,觀春申君故城,宮室盛矣哉!初,春申君之說秦昭王,及出身遣楚太子歸,何其智之明也!後制於李園,旄矣。語曰:「當斷不斷,反受其亂。」春申君失朱英之謂邪?


\end{pinyinscope}