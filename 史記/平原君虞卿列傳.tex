\article{平原君虞卿列傳}

\begin{pinyinscope}
平原君趙勝者,趙之諸公子也。諸子中勝最賢,喜賓客,賓客蓋至者數千人。平原君相趙惠文王及孝成王,三去相,三復位,封於東武城。

平原君家樓臨民家。民家有躄者,槃散行汲。平原君美人居樓上,臨見,大笑之。明日,躄者至平原君門,請曰:「臣聞君之喜士,士不遠千里而至者,以君能貴士而賤妾也。臣不幸有罷癃之病,而君之後宮臨而笑臣,臣願得笑臣者頭。」平原君笑應曰:「諾。」躄者去,平原君笑曰:「觀此豎子,乃欲以一笑之故殺吾美人,不亦甚乎!」終不殺。居歲餘,賓客門下舍人稍稍引去者過半。平原君怪之,曰:「勝所以待諸君者未嘗敢失禮,而去者何多也?」門下一人前對曰:「以君之不殺笑躄者,以君為愛色而賤士,士即去耳。」於是平原君乃斬笑躄者美人頭,自造門進躄者,因謝焉。其後門下乃復稍稍來。是時齊有孟嘗,魏有信陵,楚有春申,故爭相傾以待士。

秦之圍邯鄲,趙使平原君求救,合從於楚,約與食客門下有勇力文武備具者二十人偕。平原君曰:「使文能取勝,則善矣。文不能取勝,則歃血於華屋之下,必得定從而還。士不外索,取於食客門下足矣。」得十九人,餘無可取者,無以滿二十人。門下有毛遂者,前,自贊於平原君曰:「遂聞君將合從於楚,約與食客門下二十人偕,不外索。今少一人,願君即以遂備員而行矣。」平原君曰:「先生處勝之門下幾年於此矣?」毛遂曰:「三年於此矣。」平原君曰:「夫賢士之處世也,譬若錐之處囊中,其末立見。今先生處勝之門下三年於此矣,左右未有所稱誦,勝未有所聞,是先生無所有也。先生不能,先生留。」毛遂曰:「臣乃今日請處囊中耳。使遂蚤得處囊中,乃穎脫而出,非特其末見而已。」平原君竟與毛遂偕。十九人相與目笑之而未廢也。

毛遂比至楚,與十九人論議,十九人皆服。平原君與楚合從,言其利害,日出而言之,日中不決。十九人謂毛遂曰:「先生上。」毛遂按劍歷階而上,謂平原君曰:「從之利害,兩言而決耳。今日出而言從,日中不決,何也?」楚王謂平原君曰:「客何為者也?」平原君曰:「是勝之舍人也。」楚王叱曰:「胡不下!吾乃與而君言,汝何為者也!」毛遂按劍而前曰:「王之所以叱遂者,以楚國之眾也。今十步之內,王不得恃楚國之眾也,王之命縣於遂手。吾君在前,叱者何也?且遂聞湯以七十里之地王天下,文王以百里之壤而臣諸侯,豈其士卒眾多哉,誠能據其勢而奮其威。今楚地方五千里,持戟百萬,此霸王之資也。以楚之彊,天下弗能當。白起,小豎子耳,率數萬之眾,興師以與楚戰,一戰而舉鄢郢,再戰而燒夷陵,三戰而辱王之先人。此百世之怨而趙之所羞,而王弗知惡焉。合從者為楚,非為趙也。吾君在前,叱者何也?」楚王曰:「唯唯,誠若先生之言,謹奉社稷而以從。」毛遂曰:「從定乎?」楚王曰:「定矣。」毛遂謂楚王之左右曰:「取雞狗馬之血來。」毛遂奉銅槃而跪進之楚王曰:「王當歃血而定從,次者吾君,次者遂。」遂定從於殿上。毛遂左手持槃血而右手招十九人曰:「公相與歃此血於堂下。公等錄錄,所謂因人成事者也。」

平原君已定從而歸,歸至於趙,曰:「勝不敢復相士。勝相士多者千人,寡者百數,自以為不失天下之士,今乃於毛先生而失之也。毛先生一至楚,而使趙重於九鼎大呂。毛先生以三寸之舌,彊於百萬之師。勝不敢復相士。」遂以為上客。

平原君既返趙,楚使春申君將兵赴救趙,魏信陵君亦矯奪晉鄙軍往救趙,皆未至。秦急圍邯鄲,邯鄲急,且降,平原君甚患之。邯鄲傳舍吏子李同說平原君曰:「君不憂趙亡邪?」平原君曰:「趙亡則勝為虜,何為不憂乎?」李同曰:「邯鄲之民,炊骨易子而食,可謂急矣,而君之後宮以百數,婢妾被綺縠,餘粱肉,而民褐衣不完,糟糠不厭。民困兵盡,或剡木為矛矢,而君器物鐘磬自若。使秦破趙,君安得有此?使趙得全,君何患無有?今君誠能令夫人以下編於士卒之閒,分功而作,家之所有盡散以饗士,士方其危苦之時,易德耳。」於是平原君從之,得敢死之士三千人。李同遂與三千人赴秦軍,秦軍為之卻三十里。亦會楚、魏救至,秦兵遂罷,邯鄲復存。李同戰死,封其父為李侯。

虞卿欲以信陵君之存邯鄲為平原君請封。公孫龍聞之,夜駕見平原君曰:「龍聞虞卿欲以信陵君之存邯鄲為君請封,有之乎?」平原君曰:「然。」龍曰:「此甚不可。且王舉君而相趙者,非以君之智能為趙國無有也。割東武城而封君者,非以君為有功也,而以國人無勳,乃以君為親戚故也。君受相印不辭無能,割地不言無功者,亦自以為親戚故也。今信陵君存邯鄲而請封,是親戚受城而國人計功也。此甚不可。且虞卿操其兩權,事成,操右券以責;事不成,以虛名德君。君必勿聽也。」平原君遂不聽虞卿。

平原君以趙孝成王十五年卒。子孫代,後竟與趙俱亡。

平原君厚待公孫龍。公孫龍善為堅白之辯,及鄒衍過趙言至道,乃絀公孫龍。

虞卿者,游說之士也。躡蹻檐簦說趙孝成王。一見,賜黃金百鎰,白璧一雙;再見,為趙上卿,故號為虞卿。

秦趙戰於長平,趙不勝,亡一都尉。趙王召樓昌與虞卿曰:「軍戰不勝,尉復死,寡人使束甲而趨之,何如?」樓昌曰:「無益也,不如發重使為媾。」虞卿曰:「昌言媾者,以為不媾軍必破也。而制媾者在秦。且王之論秦也,欲破趙之軍乎,不邪?」王曰:「秦不遺餘力矣,必且欲破趙軍。」虞卿曰:「王聽臣,發使出重寶以附楚、魏,楚、魏欲得王之重寶,必內吾使。趙使入楚、魏,秦必疑天下之合從,且必恐。如此,則媾乃可為也。」趙王不聽,與平陽君為媾,發鄭朱入秦。秦內之。趙王召虞卿曰:「寡人使平陽君為媾於秦,秦已內鄭朱矣,卿之為奚如?」虞卿對曰:「王不得媾,軍必破矣。天下賀戰者皆在秦矣。鄭朱,貴人也,入秦,秦王與應侯必顯重以示天下。楚、魏以趙為媾,必不救王。秦知天下不救王,則媾不可得成也。」應侯果顯鄭朱以示天下賀戰勝者,終不肯媾。長平大敗,遂圍邯鄲,為天下笑。

秦既解邯鄲圍,而趙王入朝,使趙郝約事於秦,割六縣而媾。虞卿謂趙王曰:「秦之攻王也,倦而歸乎?王以其力尚能進,愛王而弗攻乎?」王曰:「秦之攻我也,不遺餘力矣,必以倦而歸也。」虞卿曰:「秦以其力攻其所不能取,倦而歸,王又以其力之所不能取以送之,是助秦自攻也。來年秦復攻王,王無救矣。」王以虞卿之言趙郝。趙郝曰:「虞卿誠能盡秦力之所至乎?誠知秦力之所不能進,此彈丸之地弗予,令秦來年復攻王,王得無割其內而媾乎?」王曰:「請聽子割,子能必使來年秦之不復攻我乎?」趙郝對曰:「此非臣之所敢任也。他日三晉之交於秦,相善也。今秦善韓、魏而攻王,王之所以事秦必不如韓、魏也。今臣為足下解負親之攻,開關通幣,齊交韓、魏,至來年而王獨取饱於秦,此王之所以事秦必在韓、魏之後也。此非臣之所敢任也。」

王以告虞卿。虞卿對曰:「郝言『不媾,來年秦復攻王,王得無割其內而媾乎』。今媾,郝又以不能必秦之不復攻也。今雖割六城,何益!來年復攻,又割其力之所不能取而媾,此自盡之術也,不如無媾。秦雖善攻,不能取六縣;趙雖不能守,終不失六城。秦倦而歸,兵必罷。我以六城收天下以攻罷秦,是我失之於天下而取償於秦也。吾國尚利,孰與坐而割地,自弱以彊秦哉?今郝曰『秦善韓、魏而攻趙者,必(以為韓魏不救趙也而王之軍必孤有以)王之事秦不如韓、魏也』,是使王歲以六城事秦也,即坐而城盡。來年秦復求割地,王將與之乎?弗與,是棄前功而挑秦禍也;與之,則無地而給之。語曰『彊者善攻,弱者不能守』。今坐而聽秦,秦兵不獘而多得地,是彊秦而弱趙也。以益彊之秦而割愈弱之趙,其計故不止矣。且王之地有盡而秦之求無已,以有盡之地而給無已之求,其勢必無趙矣。」

趙王計未定,樓緩從秦來,趙王與樓緩計之,曰:「予秦地(何)如毋予,孰吉?」緩辭讓曰:「此非臣之所能知也。」王曰:「雖然,試言公之私。」樓緩對曰:「王亦聞夫公甫文伯母乎?公甫文伯仕於魯,病死,女子為自殺於房中者二人。其母聞之,弗哭也。其相室曰:『焉有子死而弗哭者乎?』其母曰:『孔子,賢人也,逐於魯,而是人不隨也。今死而婦人為之自殺者二人,若是者必其於長者薄而於婦人厚也。』故從母言之,是為賢母;從妻言之,是必不免為妒妻。故其言一也,言者異則人心變矣。今臣新從秦來而言勿予,則非計也;言予之,恐王以臣為為秦也:故不敢對。使臣得為大王計,不如予之。」王曰:「諾。」

虞卿聞之,入見王曰:「此飾說也,王慎勿予!」樓緩聞之,往見王。王又以虞卿之言告樓緩。樓緩對曰:「不然。虞卿得其一,不得其二。夫秦趙構難而天下皆說,何也?曰『吾且因彊而乘弱矣』。今趙兵困於秦,天下之賀戰勝者則必盡在於秦矣。故不如亟割地為和,以疑天下而慰秦之心。不然,天下將因秦之(彊)怒,乘趙之獘,瓜分之。趙且亡,何秦之圖乎?故曰虞卿得其一,不得其二。願王以此決之,勿復計也。」

虞卿聞之,往見王曰:「危哉樓子之所以為秦者,是愈疑天下,而何慰秦之心哉?獨不言其示天下弱乎?且臣言勿予者,非固勿予而已也。秦索六城於王,而王以六城賂齊。齊,秦之深讎也,得王之六城,并力西擊秦,齊之聽王,不待辭之畢也。則是王失之於齊而取償於秦也。而齊、趙之深讎可以報矣,而示天下有能為也。王以此發聲,兵未窺於境,臣見秦之重賂至趙而反媾於王也。從秦為媾,韓、魏聞之,必盡重王;重王,必出重寶以先於王。則是王一舉而結三國之親,而與秦易道也。」趙王曰:「善。」則使虞卿東見齊王,與之謀秦。虞卿未返,秦使者已在趙矣。樓緩聞之,亡去。趙於是封虞卿以一城。

居頃之,而魏請為從。趙孝成王召虞卿謀。過平原君,平原君曰:「願卿之論從也。」虞卿入見王。王曰:「魏請為從。」對曰:「魏過。」王曰:「寡人固未之許。」對曰:「王過。」王曰:「魏請從,卿曰魏過,寡人未之許,又曰寡人過,然則從終不可乎?」對曰:「臣聞小國之與大國從事也,有利則大國受其福,有敗則小國受其禍。今魏以小國請其禍,而王以大國辭其福,臣故曰王過,魏亦過。竊以為從便。」王曰:「善。」乃合魏為從。

虞卿既以魏齊之故,不重萬戶侯卿相之印,與魏齊閒行,卒去趙,困於梁。魏齊已死,不得意,乃著書,上採春秋,下觀近世,曰節義、稱號、揣摩、政謀,凡八篇。以刺譏國家得失,世傳之曰虞氏春秋。

太史公曰:平原君,翩翩濁世之佳公子也,然未睹大體。鄙語曰「利令智昏」,平原君貪馮亭邪說,使趙陷長平兵四十餘萬眾,邯鄲幾亡。虞卿料事揣情,為趙畫策,何其工也!及不忍魏齊,卒困於大梁,庸夫且知其不可,況賢人乎?然虞卿非窮愁,亦不能著書以自見於後世云。


\end{pinyinscope}