\article{田敬仲完世家}

\begin{pinyinscope}
陳完者,陳厲公他之子也。完生,周太史過陳,陳厲公使卜完,卦得觀之否:「是為觀國之光,利用賓于王。此其代陳有國乎?不在此而在異國乎?非此其身也,在其子孫。若在異國,必姜姓。姜姓,四嶽之後。物莫能兩大,陳衰,此其昌乎?」

厲公者,陳文公少子也,其母蔡女。文公卒,厲公兄鮑立,是為桓公。桓公與他異母。及桓公病,蔡人為他殺桓公鮑及太子免而立他,為厲公。厲公既立,娶蔡女。蔡女淫於蔡人,數歸,厲公亦數如蔡。桓公之少子林怨厲公殺其父與兄,乃令蔡人誘厲公而殺之。林自立,是為莊公。故陳完不得立,為陳大夫。厲公之殺,以淫出國,故《春秋》曰「蔡人殺陳他」,罪之也。

莊公卒,立弟杵臼,是為宣公。宣公[二]十一年,殺其太子御寇。御寇與完相愛,恐禍及己,完故奔齊。齊桓公欲使為卿,辭曰:「羈旅之臣幸得免負檐,君之惠也,不敢當高位。」桓公使為工正。齊懿仲欲妻完,卜之,占曰:「是謂鳳皇于蜚,和鳴鏘鏘。有媯之後,將育于姜。五世其昌,并于正卿。八世之後,莫之與京。」卒妻完。完之奔齊,齊桓公立十四年矣。

完卒,謚為敬仲。仲生稚孟夷。敬仲之如齊,以陳字為田氏。

田稚孟夷生湣孟莊,田湣孟莊生文子須無。田文子事齊莊公。

晉之大夫欒逞作亂於晉,來奔齊,齊莊公厚客之。晏嬰與田文子諫,莊公弗聽。

文子卒,生桓子無宇。田桓子無宇有力,事齊莊公,甚有寵。

無宇卒,生武子開與釐子乞。田釐子乞事齊景公為大夫,其收賦稅於民以小斗受之,其(粟)[稟]予民以大斗,行陰德於民,而景公弗禁。由此田氏得齊眾心,宗族益彊,民思田氏。晏子數諫景公,景公弗聽。已而使於晉,與叔向私語曰:「齊國之政卒歸於田氏矣。」

晏嬰卒後,范、中行氏反晉。晉攻之急,范、中行請粟於齊。田乞欲為亂,樹黨於諸侯,乃說景公曰:「范、中行數有德於齊,齊不可不救。」齊使田乞救之而輸之粟。

景公太子死,後有寵姬曰芮子,生子荼。景公病,命其相國惠子與高昭子以子荼為太子。景公卒,兩相高、國立荼,是為晏孺子。而田乞不說,欲立景公他子陽生。陽生素與乞歡。晏孺子之立也,陽生奔魯。田乞偽事高昭子、國惠子者,每朝代參乘,言曰:「始諸大夫不欲立孺子。孺子既立,君相之,大夫皆自危,謀作亂。」又紿大夫曰:「高昭子可畏也,及未發先之。」諸大夫從之。田乞、鮑牧與大夫以兵入公室,攻高昭子。昭子聞之,與國惠子救公。公師敗。田乞之眾追國惠子,惠子奔莒,遂返殺高昭子。晏(孺子)[圉]奔魯。

田乞使人之魯,迎陽生。陽生至齊,匿田乞家。請諸大夫曰:「常之母有魚菽之祭,幸而來會飲。」會飲田氏。田乞盛陽生橐中,置坐中央。發橐,出陽生,曰:「此乃齊君矣。」大夫皆伏謁。將盟立之,田乞誣曰:「吾與鮑牧謀共立陽生也。」鮑牧怒曰:「大夫忘景公之命乎?」諸大夫欲悔,陽生乃頓首曰:「可則立之,不可則已。」鮑牧恐禍及己,乃復曰:「皆景公之子,何為不可!」遂立陽生於田乞之家,是為悼公。乃使人遷晏孺子於駘,而殺孺子荼。悼公既立,田乞為相,專齊政。

四年,田乞卒,子常代立,是為田成子。

鮑牧與齊悼公有郄,弒悼公。齊人共立其子壬,是為簡公。田常成子與監止俱為左右相,相簡公。田常心害監止,監止幸於簡公,權弗能去。於是田常復修釐子之政,以大斗出貸,以小斗收。齊人歌之曰:「嫗乎采芑,歸乎田成子!」齊大夫朝,御鞅諫簡公曰:「田、監不可并也,君其擇焉。」君弗聽。

子我者,監止之宗人也,常與田氏有卻。田氏疏族田豹事子我有寵。子我曰:「吾欲盡滅田氏適,以豹代田氏宗。」豹曰:「臣於田氏疏矣。」不聽。已而豹謂田氏曰:「子我將誅田氏,田氏弗先,禍及矣。」子我舍公宮,田常兄弟四人乘如公宮,欲殺子我。子我閉門。簡公與婦人飲檀臺,將欲擊田常。太史子餘曰:「田常非敢為亂,將除害。」簡公乃止。田常出,聞簡公怒,恐誅,將出亡。田子行曰:「需,事之賊也。」田常於是擊子我。子我率其徒攻田氏,不勝,出亡。田氏之徒追殺子我及監止。

簡公出奔,田氏之徒追執簡公于徐州。簡公曰:「蚤從御鞅之言,不及此難。」田氏之徒恐簡公復立而誅己,遂殺簡公。簡公立四年而殺。於是田常立簡公弟驁,是為平公。平公即位,田常為相。

田常既殺簡公,懼諸侯共誅己,乃盡歸魯、衛侵地,西約晉、韓、魏、趙氏,南通吳、越之使,修功行賞,親於百姓,以故齊復定。

田常言於齊平公曰:「德施人之所欲,君其行之;刑罰人之所惡,臣請行之。」行之五年,齊國之政皆歸田常。田常於是盡誅鮑、晏、監止及公族之彊者,而割齊自安平以東至瑯邪,自為封邑。封邑大於平公之所食。

田常乃選齊國中女子長七尺以上為後宮,後宮以百數,而使賓客舍人出入後宮者不禁。及田常卒,有七十餘男。

田常卒,子襄子盤代立,相齊。常謚為成子。

田襄子既相齊宣公,三晉殺知伯,分其地。襄子使其兄弟宗人盡為齊都邑大夫,與三晉通使,且以有齊國。

襄子卒,子莊子白立。田莊子相齊宣公。宣公四十三年,伐晉,毀黃城,圍陽狐。明年,伐魯、葛及安陵。明年,取魯之一城。

莊子卒,子太公和立。田太公相齊宣公。宣公四十八年,取魯之城。明年,宣公與鄭人會西城。伐衛,取毋丘。宣公五十一年卒,田會自廩丘反。

宣公卒,子康公貸立。貸立十四年,淫於酒婦人,不聽政。太公乃遷康公於海上,食一城,以奉其先祀。明年,魯敗齊平陸。

三年,太公與魏文侯會濁澤,求為諸侯。魏文侯乃使使言周天子及諸侯,請立齊相田和為諸侯。周天子許之。康公之十九年,田和立為齊侯,列於周室,紀元年。

齊侯太公和立二年,和卒,子桓公午立。桓公午五年,秦、魏攻韓,韓求救於齊。齊桓公召大臣而謀曰:「蚤救之孰與晚救之?」騶忌曰:「不若勿救。」段干朋曰:「不救,則韓且折而入於魏,不若救之。」田臣思曰:「過矣君之謀也!秦、魏攻韓、楚,趙必救之,是天以燕予齊也。」桓公曰:「善」。乃陰告韓使者而遣之。韓自以為得齊之救,因與秦、魏戰。楚、趙聞之,果起兵而救之。齊因起兵襲燕國,取桑丘。

六年,救衛。桓公卒,子威王因齊立。是歲,故齊康公卒,絕無後,奉邑皆入田氏。

齊威王元年,三晉因齊喪來伐我靈丘。三年,三晉滅晉後而分其地。六年,魯伐我,入陽關。晉伐我,至博陵。七年,衛伐我,取薛陵。九年,趙伐我,取甄。

威王初即位以來,不治,委政卿大夫,九年之閒,諸侯并伐,國人不治。於是威王召即墨大夫而語之曰:「自子之居即墨也,毀言日至。然吾使人視即墨,田野辟,民人給,官無留事,東方以寧。是子不事吾左右以求譽也。」封之萬家。召阿大夫語曰:「自子之守阿,譽言日聞。然使使視阿,田野不辟,民貧苦。昔日趙攻甄,子弗能救。衛取薛陵,子弗知。是子以幣厚吾左右以求譽也。」是日,烹阿大夫,及左右嘗譽者皆并烹之。遂起兵西擊趙、衛,敗魏於濁澤而圍惠王。惠王請獻觀以和解,趙人歸我長城。於是齊國震懼,人人不敢飾非,務盡其誠。齊國大治。諸侯聞之,莫敢致兵於齊二十餘年。

騶忌子以鼓琴見威王,威王說而捨之右室。須臾,王鼓琴,騶忌子推戶入曰:「善哉鼓琴!」王勃然不說,去琴按劍曰:「夫子見容未察,何以知其善也?」騶忌子曰:「夫大弦濁以春溫者,君也;小弦廉折以清者,相也;攫之深,醳之愉者,政令也;鈞諧以鳴,大小相益,回邪而不相害者,四時也:吾是以知其善也。」王曰:「善語音。」騶忌子曰:「何獨語音,夫治國家而弭人民皆在其中。」王又勃然不說曰:「若夫語五音之紀,信未有如夫子者也。若夫治國家而弭人民,又何為乎絲桐之閒?」騶忌子曰:「夫大弦濁以春溫者,君也;小弦廉折以清者,相也;攫之深而捨之愉者,政令也;鈞諧以鳴,大小相益,回邪而不相害者,四時也。夫復而不亂者,所以治昌也;連而徑者,所以存亡也:故曰琴音調而天下治。夫治國家而弭人民者,無若乎五音者。」王曰:「善。」

騶忌子見三月而受相印。淳于髡見之曰:「善說哉!髡有愚志,願陳諸前。」騶忌子曰:「謹受教。」淳于髡曰:「得全全昌,失全全亡。」騶忌子曰:「謹受令,請謹毋離前。」淳于髡曰:「狶膏棘軸,所以為滑也,然而不能運方穿。」騶忌子曰:「謹受令,請謹事左右。」淳于髡曰:「弓膠昔干,所以為合也,然而不能傅合疏罅。」騶忌子曰:「謹受令,請謹自附於萬民。」淳于髡曰:「狐裘雖敝,不可補以黃狗之皮。」騶忌子曰:「謹受令,請謹擇君子,毋雜小人其閒。」淳于髡曰:「大車不較,不能載其常任;琴瑟不較,不能成其五音。」騶忌子曰:「謹受令,請謹修法律而督姦吏。」淳于髡說畢,趨出,至門,而面其仆曰:「是人者,吾語之微言五,其應我若響之應聲,是人必封不久矣。」居朞,封以下邳,號曰成侯。

威王二十三年,與趙王會平陸。二十四年,與魏王會田於郊。魏王問曰:「王亦有寶乎?」威王曰:「無有。」梁王曰:「若寡人國小也,尚有徑寸之珠照車前後各十二乘者十枚,奈何以萬乘之國而無寶乎?」威王曰:「寡人之所以為寶與王異。吾臣有檀子者,使守南城,則楚人不敢為寇東取,泗上十二諸侯皆來朝。吾臣有子者,使守高唐,則趙人不敢東漁於河。吾吏有黔夫者,使守徐州,則燕人祭北門,趙人祭西門,徙而從者七千餘家。吾臣有種首者,使備盜賊,則道不拾遺。將以照千里,豈特十二乘哉!」梁惠王慚,不懌而去。

二十六年,魏惠王圍邯鄲,趙求救於齊。齊威王召大臣而謀曰:「救趙孰與勿救?」騶忌子曰:「不如勿救。」段干朋曰:「不救則不義,且不利。」威王曰:「何也?」對曰:「夫魏氏并邯鄲,其於齊何利哉?且夫救趙而軍其郊,是趙不伐而魏全也。故不如南攻襄陵以獘魏,邯鄲拔而乘魏之獘。」威王從其計。

其後成侯騶忌與田忌不善,公孫閱謂成侯忌曰:「公何不謀伐魏,田忌必將。戰勝有功,則公之謀中也;戰不勝,非前死則後北,而命在公矣。」於是成侯言威王,使田忌南攻襄陵。十月,邯鄲拔,齊因起兵擊魏,大敗之桂陵。于是齊最彊于諸侯,自稱為王,以令天下。

三十三年,殺其大夫牟辛。

三十五年,公孫閱又謂成侯忌曰:「公何不令人操十金卜於市,曰『我田忌之人也。吾三戰而三勝,聲威天下。欲為大事,亦吉乎不吉乎』?」卜者出,因令人捕為之卜者,驗其辭於王之所。田忌聞之,因率其徒襲攻臨淄,求成侯,不勝而奔。

三十六年,威王卒,子宣王辟彊立。

宣王元年,秦用商鞅。周致伯於秦孝公。

二年,魏伐趙。趙與韓親,共擊魏。趙不利,戰於南梁。宣王召田忌復故位。韓氏請救於齊。宣王召大臣而謀曰:「蚤救孰與晚救?」騶忌子曰:「不如勿救。」田忌曰:「弗救,則韓且折而入於魏,不如蚤救之。」孫子曰:「夫韓、魏之兵未獘而救之,是吾代韓受魏之兵,顧反聽命於韓也。且魏有破國之志,韓見亡,必東面而愬於齊矣。吾因深結韓之親而晚承魏之獘,則可重利而得尊名也。」宣王曰:「善。」乃陰告韓之使者而遣之。韓因恃齊,五戰不勝,而東委國於齊。齊因起兵,使田忌、田嬰將,孫子為(帥)[師],救韓、趙以擊魏,大敗之馬陵,殺其將龐涓,虜魏太子申。其後三晉之王皆因田嬰朝齊王於博望,盟而去。

七年,與魏王會平阿南。明年,復會甄。魏惠王卒。明年,與魏襄王會徐州,諸侯相王也。十年,楚圍我徐州。十一年,與魏伐趙,趙決河水灌齊、魏,兵罷。

十八年,宣王喜文學游說之士,自如騶衍、淳于髡、田駢、接子、慎到、環淵之徒七十六人,皆賜列第,為上大夫,不治而議論。是以齊稷下學士復盛,且數百千人。

十九年,宣王卒,子湣王地立。

湣王元年,秦使張儀與諸侯執政會于齧桑。三年,封田嬰於薛。四年,迎婦于秦。七年,與宋攻魏,敗之觀澤。

十二年,攻魏。楚圍雍氏,秦敗屈丐。蘇代謂田軫曰:「臣願有謁於公,其為事甚完,使楚利公,成為福,不成亦為福。今者臣立於門,客有言曰魏王謂韓馮、張儀曰:『煮棗將拔,齊兵又進,子來救寡人則可矣;不救寡人,寡人弗能拔。』此特轉辭也。秦、韓之兵毋東,旬餘,則魏氏轉韓從秦,秦逐張儀,交臂而事齊楚,此公之事成也。」田軫曰:「柰何使無東?」對曰:「韓馮之救魏之辭,必不謂韓王曰『馮以為魏』,必曰『馮將以秦韓之兵東卻齊宋,馮因摶三國之兵,乘屈丐之獘,南割於楚,故地必盡得之矣』。張儀救魏之辭,必不謂秦王曰『儀以為魏』,必曰『儀且以秦韓之兵東距齊宋,儀將摶三國之兵,乘屈丐之獘,南割於楚,名存亡國,實伐三川而歸,此王業也』。公令楚王與韓氏地,使秦制和,謂秦王曰『請與韓地,而王以施三川,韓氏之兵不用而得地於楚』。韓馮之東兵之辭且謂秦何?曰『秦兵不用而得三川,伐楚韓以窘魏,魏氏不敢東,是孤齊也』。張儀之東兵之辭且謂何?曰『秦韓欲地而兵有案,聲威發於魏,魏氏之欲不失齊楚者有資矣』。魏氏轉秦韓爭事齊楚,楚王欲而無與地,公令秦韓之兵不用而得地,有一大德也。秦韓之王劫於韓馮、張儀而東兵以徇服魏,公常執左券以責於秦韓,此其善於公而惡張子多資矣。」

十三年,秦惠王卒。二十三年,與秦擊敗楚於重丘。二十四年,秦使涇陽君質於齊。二十五年,歸涇陽君于秦。孟嘗君薛文入秦,即相秦。文亡去。二十六年,齊與韓魏共攻秦,至函谷軍焉。二十八年,秦與韓河外以和,兵罷。二十九年,趙殺其主父。齊佐趙滅中山。

三十六年,王為東帝,秦昭王為西帝。蘇代自燕來,入齊,見於章華東門。齊王曰:「嘻,善,子來!秦使魏冉致帝,子以為何如?」對曰:「王之問臣也卒,而患之所從來微,願王受之而勿備稱也。秦稱之,天下安之,王乃稱之,無後也。且讓爭帝名,無傷也。秦稱之,天下惡之,王因勿稱,以收天下,此大資也。且天下立兩帝,王以天下為尊齊乎?尊秦乎?」王曰:「尊秦。」曰:「釋帝,天下愛齊乎?愛秦乎?」王曰:「愛齊而憎秦。」曰:「兩帝立約伐趙,孰與伐桀宋之利?」王曰:「伐桀宋利。」對曰:「夫約鈞,然與秦為帝而天下獨尊秦而輕齊,釋帝則天下愛齊而憎秦,伐趙不如伐桀宋之利,故願王明釋帝以收天下,倍約賓秦,無爭重,而王以其閒舉宋。夫有宋,衛之陽地危;有濟西,趙之阿東國危;有淮北,楚之東國危;有陶、平陸,梁門不開。釋帝而貸之以伐桀宋之事,國重而名尊,燕楚所以形服,天下莫敢不聽,此湯武之舉也。敬秦以為名,而後使天下憎之,此所謂以卑為尊者也。願王孰慮之。」於是齊去帝復為王,秦亦去帝位。

三十八年,伐宋。秦昭王怒曰:「吾愛宋與愛新城、陽晉同。韓聶與吾友也,而攻吾所愛,何也?」蘇代為齊謂秦王曰:「韓聶之攻宋,所以為王也。齊彊,輔之以宋,楚魏必恐,恐必西事秦,是王不煩一兵,不傷一士,無事而割安邑也,此韓聶之所禱於王也。」秦王曰:「吾患齊之難知。一從一衡,其說何也?」對曰:「天下國令齊可知乎?齊以攻宋,其知事秦以萬乘之國自輔,不西事秦則宋治不安。中國白頭游敖之士皆積智欲離齊秦之交,伏式結軼西馳者,未有一人言善齊者也,伏式結軼東馳者,未有一人言善秦者也。何則?皆不欲齊秦之合也。何晉楚之智而齊秦之愚也!晉楚合必議齊秦,齊秦合必圖晉楚,請以此決事。」秦王曰:「諾。」於是齊遂伐宋,宋王出亡,死於溫。齊南割楚之淮北,西侵三晉,欲以并周室,為天子。泗上諸侯鄒魯之君皆稱臣,諸侯恐懼。

三十九年,秦來伐,拔我列城九。

四十年,燕、秦、楚、三晉合謀,各出銳師以伐,敗我濟西。王解而卻。燕將樂毅遂入臨淄,盡取齊之寶藏器。湣王出亡,之衛。衛君辟宮捨之,稱臣而共具。湣王不遜,衛人侵之。湣王去,走鄒、魯,有驕色,鄒、魯君弗內,遂走莒。楚使淖齒將兵救齊,因相齊湣王。淖齒遂殺湣王而與燕共分齊之侵地鹵器。

湣王之遇殺,其子法章變名姓為莒太史敫家庸。太史敫女奇法章狀貌,以為非恒人,憐而常竊衣食之,而與私通焉。淖齒既以去莒,莒中人及齊亡臣相聚求湣王子,欲立之。法章懼其誅己也,久之,乃敢自言「我湣王子也」。於是莒人共立法章,是為襄王。以保莒城而布告齊國中:「王已立在莒矣。」

襄王既立,立太史氏女為王后,是為君王后,生子建。太史敫曰:「女不取媒因自嫁,非吾種也,汙吾世。」終身不睹君王后。君王后賢,不以不睹故失人子之禮。

襄王在莒五年,田單以即墨攻破燕軍,迎襄王於莒,入臨菑。齊故地盡復屬齊。齊封田單為安平君。

十四年,秦擊我剛壽。十九年,襄王卒,子建立。

王建立六年,秦攻趙,齊楚救之。秦計曰:「齊楚救趙,親則退兵,不親遂攻之。」趙無食,請粟於齊,齊不聽。周子曰:「不如聽之以退秦兵,不聽則秦兵不卻,是秦之計中而齊楚之計過也。且趙之於齊楚,捍蔽也,猶齒之有脣也,脣亡則齒寒。今日亡趙,明日患及齊楚。且救趙之務,宜若奉漏甕沃焦釜也。夫救趙,高義也;卻秦兵,顯名也。義救亡國,威卻彊秦之兵,不務為此而務愛粟,為國計者過矣。」齊王弗聽。秦破趙於長平四十餘萬,遂圍邯鄲。

十六年,秦滅周。君王后卒。二十三年,秦置東郡。二十八年,王入朝秦,秦王政置酒咸陽。三十五年,秦滅韓。三十七年,秦滅趙。三十八年,燕使荊軻刺秦王,秦王覺,殺軻。明年,秦破燕,燕王亡走遼東。明年,秦滅魏,秦兵次於歷下。四十二年,秦滅楚。明年,虜代王嘉,滅燕王喜。

四十四年,秦兵擊齊。齊王聽相后勝計,不戰,以兵降秦。秦虜王建,遷之共。遂滅齊為郡。天下壹并於秦,秦王政立號為皇帝。始,君王后賢,事秦謹,與諸侯信,齊亦東邊海上,秦日夜攻三晉、燕、楚,五國各自救於秦,以故王建立四十餘年不受兵。君王后死,后勝相齊,多受秦閒金,多使賓客入秦,秦又多予金,客皆為反閒,勸王去從朝秦,不修攻戰之備,不助五國攻秦,秦以故得滅五國。五國已亡,秦兵卒入臨淄,民莫敢格者。王建遂降,遷於共。故齊人怨王建不蚤與諸侯合從攻秦,聽姦臣賓客以亡其國,歌之曰:「松耶柏耶?住建共者客耶?」疾建用客之不詳也。

太史公曰:蓋孔子晚而喜易。易之為術,幽明遠矣,非通人達才孰能注意焉!笔周太史之卦田敬仲完,占至十世之後;及完奔齊,懿仲卜之亦云。田乞及常所以比犯二君,專齊國之政,非必事勢之漸然也,蓋若遵厭兆祥云。


\end{pinyinscope}