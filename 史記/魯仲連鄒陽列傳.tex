\article{魯仲連鄒陽列傳}

\begin{pinyinscope}
魯仲連者,齊人也。好奇偉俶儻之畫策,而不肯仕宦任職,好持高節。游於趙。

趙孝成王時,而秦王使白起破趙長平之軍前後四十餘萬,秦兵遂東圍邯鄲。趙王恐,諸侯之救兵莫敢擊秦軍。魏安釐王使將軍晉鄙救趙,畏秦,止於蕩陰不進。魏王使客將軍新垣衍閒入邯鄲,因平原君謂趙王曰:「秦所為急圍趙者,前與齊湣王爭彊為帝,已而復歸帝;今齊[湣王]已益弱,方今唯秦雄天下,此非必貪邯鄲,其意欲復求為帝。趙誠發使尊秦昭王為帝,秦必喜,罷兵去。」平原君猶預未有所決。

此時魯仲連適游趙,會秦圍趙,聞魏將欲令趙尊秦為帝,乃見平原君曰:「事將柰何?」平原君曰:「勝也何敢言事!前亡四十萬之眾於外,今又內圍邯鄲而不能去。魏王使客將軍新垣衍令趙帝秦,今其人在是。勝也何敢言事!」魯仲連曰:「吾始以君為天下之賢公子也,吾乃今然後知君非天下之賢公子也。梁客新垣衍安在?吾請為君責而歸之。」平原君曰:「勝請為紹介而見之於先生。」平原君遂見新垣衍曰:「東國有魯仲連先生者,今其人在此,勝請為紹介,交之於將軍。」新垣衍曰:「吾聞魯仲連先生,齊國之高士也。衍人臣也,使事有職,吾不願見魯仲連先生。」平原君曰:「勝既已泄之矣。」新垣衍許諾。

魯連見新垣衍而無言。新垣衍曰:「吾視居此圍城之中者,皆有求於平原君者也;今吾觀先生之玉貌,非有求於平原君者也,曷為久居此圍城之中而不去?」魯仲連曰:「世以鮑焦為無從頌而死者,皆非也。眾人不知,則為一身。彼秦者,棄禮義而上首功之國也,權使其士,虜使其民。彼即肆然而為帝,過而為政於天下,則連有蹈東海而死耳,吾不忍為之民也。所為見將軍者,欲以助趙也。」

新垣衍曰:「先生助之將柰何?」魯連曰:「吾將使梁及燕助之,齊、楚則固助之矣。」新垣衍曰:「燕則吾請以從矣;若乃梁者,則吾乃梁人也,先生惡能使梁助之?」魯連曰:「梁未睹秦稱帝之害故耳。使梁睹秦稱帝之害,則必助趙矣。」

新垣衍曰:「秦稱帝之害何如?」魯連曰:「昔者齊威王嘗為仁義矣,率天下諸侯而朝周。周貧且微,諸侯莫朝,而齊獨朝之。居歲餘,周烈王崩,齊後往,周怒,赴於齊曰:『天崩地坼,天子下席。東藩之臣因齊後至,則斮。』齊威王勃然怒曰:『叱嗟,而母婢也!』卒為天下笑。故生則朝周,死則叱之,誠不忍其求也。彼天子固然,其無足怪。」

新垣衍曰:「先生獨不見夫仆乎?十人而從一人者,寧力不勝而智不若邪?畏之也。」魯仲連曰:「嗚呼!梁之比於秦若仆邪?」新垣衍曰:「然。」魯仲連曰:「吾將使秦王烹醢梁王。」新垣衍怏然不悅,曰:「噫嘻,亦太甚矣先生之言也!先生又惡能使秦王烹醢梁王?」魯仲魯曰:「固也,吾將言之。昔者九侯、鄂侯、文王,紂之三公也。九侯有子而好,獻之於紂,紂以為惡,醢九侯。鄂侯爭之彊,辯之疾,故脯鄂侯。文王聞之,喟然而嘆,故拘之牖里之庫百日,欲令之死。曷為與人俱稱王,卒就脯醢之地?齊湣王之魯,夷維子為執策而從,謂魯人曰:『子將何以待吾君?』魯人曰:『吾將以十太牢待子之君。』夷維子曰:『子安取禮而來[待]吾君?彼吾君者,天子也。天子巡狩,諸侯辟舍,納筦籥,攝衽抱機,視膳於堂下,天子已食,乃退而聽朝也。』魯人投其籥,不果納。不得入於魯,將之薛,假途於鄒。當是時,鄒君死,湣王欲入弔,夷維子謂鄒之孤曰:『天子弔,主人必將倍殯棺,設北面於南方,然後天子南面弔也。』鄒之群臣曰:『必若此,吾將伏劍而死。』固不敢入於鄒。鄒、魯之臣,生則不得事養,死則不得賻襚,然且欲行天子之禮於鄒、魯,鄒、魯之臣不果納。今秦萬乘之國也,梁亦萬乘之國也。俱據萬乘之國,各有稱王之名,睹其一戰而勝,欲從而帝之,是使三晉之大臣不如鄒、魯之仆妾也。且秦無已而帝,則且變易諸侯之大臣。彼將奪其所不肖而與其所賢,奪其所憎而與其所愛。彼又將使其子女讒妾為諸侯妃姬。處梁之宮。梁王安得晏然而已乎?而將軍又何以得故寵乎?」

於是新垣衍起,再拜謝曰:「始以先生為庸人,吾乃今日知先生為天下之士也。吾請出,不敢復言帝秦。」秦將聞之,為卻軍五十里。適會魏公子無忌奪晉鄙軍以救趙,擊秦軍,秦軍遂引而去。

於是平原君欲封魯連,魯連辭讓(使)者三,終不肯受。平原君乃置酒,酒酣起前,以千金為魯連壽。魯連笑曰:「所貴於天下之士者,為人排患釋難解紛亂而無取也。即有取者,是商賈之事也,而連不忍為也。」遂辭平原君而去,終身不復見。

其後二十餘年,燕將攻下聊城,聊城人或讒之燕,燕將懼誅,因保守聊城,不敢歸。齊田單攻聊城歲餘,士卒多死而聊城不下。魯連乃為書,約之矢以射城中,遺燕將。《書》曰:

吾聞之,智者不倍時而棄利,勇士不死而滅名,忠臣不先身而後君。今公行一朝之忿,不顧燕王之無臣,非忠也;殺身亡聊城,而威不信於齊,非勇也;功敗名滅,後世無稱焉,非智也。三者世主不臣,說士不載,故智者不再計,勇士不怯死。今死生榮辱,貴賤尊卑,此時不再至,願公詳計而無與俗同。

且楚攻齊之南陽,魏攻平陸,而齊無南面之心,以為亡南陽之害小,不如得濟北之利大,故定計審處之。今秦人下兵,魏不敢東面;衡秦之勢成,楚國之形危;齊棄南陽,斷右壤,定濟北,計猶且為之也。且夫齊之必決於聊城,公勿再計。今楚魏交退於齊,而燕救不至。以全齊之兵,無天下之規,與聊城共據期年之敝,則臣見公之不能得也。且燕國大亂,君臣失計,上下迷惑,栗腹以十萬之眾五折於外,以萬乘之國被圍於趙,壤削主困,為天下僇笑。國敝而禍多,民無所歸心。今公又以敝聊之民距全齊之兵,是墨翟之守也。食人炊骨,士無反外之心,是孫臏之兵也。能見於天下。雖然,為公計者,不如全車甲以報於燕。車甲全而歸燕,燕王必喜;身全而歸於國,士民如見父母,交游攘臂而議於世,功業可明。上輔孤主以制群臣,下養百姓以資說士,矯國更俗,功名可立也。亡意亦捐燕棄世,東游於齊乎?裂地定封,富比乎陶、衛,世世稱孤,與齊久存,又一計也。此兩計者,顯名厚實也,願公詳計而審處一焉。

且吾聞之,規小睗者不能成榮名,惡小恥者不能立大功。昔者管夷吾射桓公中其鉤,篡也;遺公子糾不能死,怯也;束縛桎梏,辱也。若此三行者,世主不臣而鄉里不通。鄉使管子幽囚而不出,身死而不反於齊,則亦名不免為辱人賤行矣。臧獲且羞與之同名矣,況世俗乎!笔管子不恥身在縲紲之中而恥天下之不治,不恥不死公子糾而恥威之不信於諸侯,故兼三行之過而為五霸首,名高天下而光燭鄰國。曹子為魯將,三戰三北,而亡地五百里。鄉使曹子計不反顧,議不還踵,刎頸而死,則亦名不免為敗軍禽將矣。曹子棄三北之恥,而退與魯君計。桓公朝天下,會諸侯,曹子以一劍之任,枝桓公之心於壇坫之上,顏色不變,辭氣不悖,三戰之所亡一朝而復之,天下震動,諸侯驚駭,威加吳、越。若此二士者,非不能成小廉而行小睗也,以為殺身亡軀,絕世滅後,功名不立,非智也。故去感忿之怨,立終身之名;棄忿悁之節,定累世之功。是以業與三王爭流,而名與天壤相獘也。願公擇一而行之。

燕將見魯連書,泣三日,猶豫不能自決。欲歸燕,已有隙,恐誅;欲降齊,所殺虜於齊甚眾,恐已降而後見辱。喟然嘆曰:「與人刃我,寧自刃。」乃自殺。聊城亂,田單遂屠聊城。歸而言魯連,欲爵之。魯連逃隱於海上,曰:「吾與富貴而詘於人,寧貧賤而輕世肆志焉。」

鄒陽者,齊人也。游於梁,與故吳人莊忌夫子、淮陰枚生之徒交。上書而介於羊勝、公孫詭之閒。勝等嫉鄒陽,惡之梁孝王。孝王怒,下之吏,將欲殺之。鄒陽客游,以讒見禽,恐死而負累,乃從獄中上書曰:

臣聞忠無不報,信不見疑,臣常以為然,徒虛語耳。昔者荊軻慕燕丹之義,白虹貫日,太子畏之;衛先生為秦畫長平之事,太白蝕昴,而昭王疑之。夫精變天地而信不喻兩主,豈不哀哉!今臣盡忠竭誠,畢議願知,左右不明,卒從吏訊,為世所疑,是使荊軻、衛先生復起,而燕、秦不悟也。願大王孰察之。

昔卞和獻寶,楚王刖之;李斯竭忠,胡亥極刑。是以箕子詳狂,接輿辟世,恐遭此患也。願大王孰察卞和、李斯之意,而後楚王、胡亥之聽,無使臣為箕子、接輿所笑。臣聞比干剖心,子胥鴟夷,臣始不信,乃今知之。願大王孰察,少加憐焉。

諺曰:「有白頭如新,傾蓋如故。」何則?知與不知也。故昔樊於期逃秦之燕,藉荊軻首以奉丹之事;王奢去齊之魏,臨城自剄以卻齊而存魏。夫王奢、樊於期非新於齊、秦而故於燕、魏也,所以去二國死兩君者,行合於志而慕義無窮也。是以蘇秦不信於天下,而為燕尾生;白圭戰亡六城,為魏取中山。何則?誠有以相知也。蘇秦相燕,燕人惡之於王,王按劍而怒,食以駃騠;白圭顯於中山,中山人惡之魏文侯,文侯投之以夜光之璧。何則?兩主二臣,剖心坼肝相信,豈移於浮辭哉!

故女無美惡,入宮見妒;士無賢不肖,入朝見嫉。昔者司馬喜髕腳於宋,卒相中山;范睢摺脅折齒於魏,卒為應侯。此二人者,皆信必然之畫,捐朋黨之私,挾孤獨之位,故不能自免於嫉妒之人也。是以申徒狄自沈於河,徐衍負石入海。不容於世,義不茍取,比周於朝,以移主上之心。故百里奚乞食於路,繆公委之以政;甯戚飯牛車下,而桓公任之以國。此二人者,豈借宦於朝,假譽於左右,然後二主用之哉?感於心,合於行,親於膠漆,昆弟不能離,豈惑於眾口哉?故偏聽生姦,獨任成亂。昔者魯聽季孫之說而逐孔子,宋信子罕之計而囚墨翟。夫以孔、墨之辯,不能自免於讒諛,而二國以危。何則?眾口鑠金,積毀銷骨也。是以秦用戎人由余而霸中國,齊用越人蒙而彊威、宣。此二國,豈拘於俗,牽於世,系阿偏之辭哉?公聽并觀,垂名當世。故意合則胡越為昆弟,由余、越人蒙是矣;不合,則骨肉出逐不收,朱、象、管、蔡是矣。今人主誠能用齊、秦之義,後宋、魯之聽,則五伯不足稱,三王易為也。

是以聖王覺寤,捐子之之心,而能不說於田常之賢;封比干之後,修孕婦之墓,故功業復就於天下。何則?欲善無厭也。夫晉文公親其讎,彊霸諸侯;齊桓公用其仇,而一匡天下。何則,慈仁慇勤,誠加於心,不可以虛辭借也。

至夫秦用商鞅之法,東弱韓、魏,兵彊天下,而卒車裂之;越用大夫種之謀,禽勁吳,霸中國,而卒誅其身。是以孫叔敖三去相而不悔,於陵子仲辭三公為人灌園。今人主誠能去驕傲之心,懷可報之意,披心腹,見情素,墮肝膽,施德厚,終與之窮達,無愛於士,則桀之狗可使吠堯,而蹠之客可使刺由;況因萬乘之權,假聖王之資乎?然則荊軻之湛七族,要離之燒妻子,豈足道哉!

臣聞明月之珠,夜光之璧,以闇投人於道路,人無不按劍相眄者。何則?無因而至前也。蟠木根柢,輪囷離詭,而為萬乘器者。何則?以左右先為之容也。故無因至前,雖出隨侯之珠,夜光之璧,猶結怨而不見德。故有人先談,則以枯木朽株樹功而不忘。今夫天下布衣窮居之士,身在貧賤,雖蒙堯、舜之術,挾伊、管之辯,懷龍逢、比干之意,欲盡忠當世之君,而素無根柢之容,雖竭精思,欲開忠信,輔人主之治,則人主必有按劍相眄之跡,是使布衣不得為枯木朽株之資也。

是以聖王制世御俗,獨化於陶鈞之上,而不牽於卑亂之語,不奪於眾多之口。故秦皇帝任中庶子蒙嘉之言,以信荊軻之說,而匕首竊發;周文王獵涇、渭,載呂尚而歸,以王天下。故秦信左右而殺,周用烏集而王。何則?以其能越攣拘之語,馳域外之議,獨觀於昭曠之道也。

今人主沈於諂諛之辭,牽於帷裳之制,使不羈之士與牛驥同皁,此鮑焦所以忿於世而不留富貴之樂也。

臣聞盛飾入朝者不以利汙義,砥厲名號者不以欲傷行,故縣名勝母而曾子不入,邑號朝歌而墨子回車。今欲使天下寥廓之士,攝於威重之權,主於位勢之貴,故回面汙行以事諂諛之人而求親近於左右,則士伏死堀穴巖(巖)[藪]之中耳,安肯有盡忠信而趨闕下者哉!書奏梁孝王,孝王使人出之,卒為上客。

太史公曰:魯連其指意雖不合大義,然余多其在布衣之位,蕩然肆志,不詘於諸侯,談說於當世,折卿相之權。鄒陽辭雖不遜,然其比物連類,有足悲者,亦可謂抗直不橈矣,吾是以附之列傳焉。


\end{pinyinscope}