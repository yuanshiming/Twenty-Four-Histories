\article{老子韓非列傳}

\begin{pinyinscope}
老子者,楚苦縣厲鄉曲仁里人也,姓李氏,名耳,字耼,周守藏室之史也。

孔子適周,將問禮於老子。老子曰:「子所言者,其人與骨皆已朽矣,獨其言在耳。且君子得其時則駕,不得其時則蓬累而行。吾聞之,良賈深藏若虛,君子盛德容貌若愚。去子之驕氣與多欲,態色與淫志,是皆無益於子之身。吾所以告子,若是而已。」孔子去,謂弟子曰:「鳥,吾知其能飛;魚,吾知其能游;獸,吾知其能走。走者可以為罔,游者可以為綸,飛者可以為矰。至於龍,吾不能知其乘風雲而上天。吾今日見老子,其猶龍邪!」

老子修道德,其學以自隱無名為務。居周久之,見周之衰,乃遂去。至關,關令尹喜曰:「子將隱矣,彊為我著書。」於是老子乃著書上下篇,言道德之意五千餘言而去,莫知其所終。

或曰:老萊子亦楚人也,著書十五篇,言道家之用,與孔子同時云。

蓋老子百有六十餘歲,或言二百餘歲,以其修道而養壽也。

自孔子死之後百二十九年,而史記周太史儋見秦獻公曰:「始秦與周合,合五百歲而離,離七十歲而霸王者出焉。」或曰儋即老子,或曰非也,世莫知其然否。老子,隱君子也。

老子之子名宗,宗為魏將,封於段干。宗子注,注子宮,宮玄孫假,假仕於漢孝文帝。而假之子解為膠西王卬太傅,因家于齊焉。

世之學老子者則絀儒學,儒學亦絀老子。「道不同不相為謀」,豈謂是邪?李耳無為自化,清靜自正。

莊子者,蒙人也,名周。周嘗為蒙漆園吏,與梁惠王、齊宣王同時。其學無所不闚,然其要本歸於老子之言。故其著書十餘萬言,大抵率寓言也。作漁父、盜跖、胠篋,以詆訿孔子之徒,以明老子之術。畏累虛、亢桑子之屬,皆空語無事實。然善屬書離辭,指事類情,用剽剝儒、墨,雖當世宿學不能自解免也。其言洸洋自恣以適己,故自王公大人不能器之。

楚威王聞莊周賢,使使厚幣迎之,許以為相。莊周笑謂楚使者曰:「千金,重利;卿相,尊位也。子獨不見郊祭之犧牛乎?養食之數歲,衣以文繡,以入大廟。當是之時,雖欲為孤豚,豈可得乎?子亟去,無污我。我寧游戲污瀆之中自快,無為有國者所羈,終身不仕,以快吾志焉。」

申不害者,京人也,故鄭之賤臣。學術以干韓昭侯,昭侯用為相。內修政教,外應諸侯,十五年。終申子之身,國治兵彊,無侵韓者。申子之學本於黃老而主刑名。著書二篇,號曰申子。

韓非者,韓之諸公子也。喜刑名法術之學,而其歸本於黃老。非為人口吃,不能道說,而善著書。與李斯俱事荀卿,斯自以為不如非。

非見韓之削弱,數以書諫韓王,韓王不能用。於是韓非疾治國不務修明其法制,執勢以御其臣下,富國彊兵而以求人任賢,反舉浮淫之蠹而加之於功實之上。以為儒者用文亂法,而俠者以武犯禁。寬則寵名譽之人,急則用介胄之士。今者所養非所用,所用非所養。悲廉直不容於邪枉之臣,觀往者得失之變,故作孤憤、五蠹、內外儲、說林、說難十餘萬言。

然韓非知說之難,為說難書甚具,終死於秦,不能自脫。

說難曰:

凡說之難,非吾知之有以說之難也;又非吾辯之難能明吾意之難也;又非吾敢橫失能盡之難也。凡說之難,在知所說之心,可以吾說當之。

所說出於為名高者也,而說之以厚利,則見下節而遇卑賤,必棄遠矣。所說出於厚利者也。而說之以名高,則見無心而遠事情,必不收矣。所說實為厚利而顯為名高者也,而說之以名高,則陽收其身而實疏之;若說之以厚利,則陰用其言而顯棄其身。此之不可不知也。

夫事以密成,語以泄敗。未必其身泄之也,而語及其所匿之事,如是者身危。貴人有過端,而說者明言善議以推其惡者,則身危。周澤未渥也而語極知,說行而有功則德亡,說不行而有敗則見疑,如是者身危。夫貴人得計而欲自以為功,說者與知焉,則身危。彼顯有所出事,迺自以為也故,說者與知焉,則身危。彊之以其所必不為,止之以其所不能已者,身危。故曰:與之論大人,則以為閒己;與之論細人,則以為粥權。論其所愛,則以為借資;論其所憎,則以為嘗己。徑省其辭,則不知而屈之;汎濫博文,則多而久之。順事陳意,則曰怯懦而不盡;慮事廣肆,則曰草野而倨侮。此說之難,不可不知也。

凡說之務,在知飾所說之所敬,而滅其所醜。彼自知其計,則毋以其失窮之;自勇其斷,則毋以其敵怒之;自多其力,則毋以其難概之。規異事與同計,譽異人與同行者,則以飾之無傷也。有與同失者,則明飾其無失也。大忠無所拂悟,辭言無所擊排,迺後申其辯知焉。此所以親近不疑,知盡之難也。得曠日彌久,而周澤既渥,深計而不疑,交爭而不罪,迺明計利害以致其功,直指是非以飾其身,以此相持,此說之成也。

伊尹為庖,百里奚為虜,皆所由干其上也。故此二子者,皆聖人也,猶不能無役身而涉世如此其汙也,則非能仕之所設也。

宋有富人,天雨牆壞。其子曰「不築且有盜」,其鄰人之父亦云,暮而果大亡其財,其家甚知其子而疑鄰人之父。昔者鄭武公欲伐胡,迺以其子妻之。因問群臣曰: 「吾欲用兵,誰可伐者?」關其思曰:「胡可伐。」迺戮關其思,曰:「胡,兄弟之國也,子言伐之,何也?」胡君聞之,以鄭為親己而不備鄭。鄭人襲胡,取之。此二說者,其知皆當矣,然而甚者為戮,薄者見疑。非知之難也,處知則難矣。

昔者彌子瑕見愛於衛君。衛國之法,竊駕君車者罪至刖。既而彌子之母病,人聞,往夜告之,彌子矯駕君車而出。君聞之而賢之曰:「孝哉,為母之故而犯刖罪!」與君游果園,彌子食桃而甘,不盡而奉君。君曰:「愛我哉,忘其口而念我!」及彌子色衰而愛弛,得罪於君。君曰:「是嘗矯駕吾車,又嘗食我以其餘桃。」故彌子之行未變於初也,前見賢而後獲罪者,愛憎之至變也。故有愛於主,則知當而加親;見憎於主,則罪當而加疏。故諫說之士不可不察愛憎之主而後說之矣。夫龍之為蟲也,可擾狎而騎也。然其喉下有逆鱗徑尺,人有嬰之,則必殺人。人主亦有逆鱗,說之者能無嬰人主之逆鱗,則幾矣。

人或傳其書至秦。秦王見孤憤、五蠹之書,曰:「嗟乎,寡人得見此人與之游,死不恨矣!」李斯曰:「此韓非之所著書也。」秦因急攻韓。韓王始不用非,及急,乃遣非使秦。秦王悅之,未信用。李斯、姚賈害之,毀之曰:「韓非,韓之諸公子也。今王欲并諸侯,非終為韓不為秦,此人之情也。今王不用,久留而歸之,此自遺患也,不如以過法誅之。」秦王以為然,下吏治非。李斯使人遺非藥,使自殺。韓非欲自陳,不得見。秦王後悔之,使人赦之,非已死矣。申子、韓子皆著書,傳於後世,學者多有。余獨悲韓子為說難而不能自脫耳。

太史公曰:老子所貴道,虛無,因應變化於無為,故著書辭稱微妙難識。莊子散道德,放論,要亦歸之自然。申子卑卑,施之於名實。韓子引繩墨,切事情,明是非,其極慘礉少恩。皆原於道德之意,而老子深遠矣。


\end{pinyinscope}