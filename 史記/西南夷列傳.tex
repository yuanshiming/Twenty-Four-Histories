\article{西南夷列傳}

\begin{pinyinscope}
西南夷君長以什數,夜郎最大;其西靡莫之屬以什數,滇最大;自滇以北君長以什數,邛都最大:此皆魋結,耕田,有邑聚。其外西自同師以東,北至楪榆,名為巂、昆明,皆編發,隨畜遷徙,毋常處,毋君長,地方可數千里。自冄以東北,君長以什數,徙、筰都最大;自筰以東北,君長以什數,冄駹最大。其俗或士箸,或移徙,在蜀之西。自冄駹以東北,君長以什數,白馬最大,皆氐類也。此皆巴蜀西南外蠻夷也。

始楚威王時,使將軍莊蹻將兵循江上,略巴、黔中以西。莊蹻者,故楚莊王苗裔也。蹻至滇池,地方三百里,旁平地,肥饒數千里,以兵威定屬楚。欲歸報,會秦擊奪楚巴、黔中郡,道塞不通,因還,以其眾王滇,變服,從其俗,以長之。秦時常頞略通五尺道,諸此國頗置吏焉。十餘歲,秦滅。及漢興,皆棄此國而開蜀故徼。巴蜀民或竊出商賈,取其筰馬、僰僮、髦牛,以此巴蜀殷富。

建元六年,大行王恢擊東越,東越殺王郢以報。恢因兵威使番陽令唐蒙風指曉南越。南越食蒙蜀枸醬,蒙問所從來,曰「道西北牂柯,牂柯江廣數里,出番禺城下」。蒙歸至長安,問蜀賈人,賈人曰:「獨蜀出枸醬,多持竊出市夜郎。夜郎者,臨牂柯江,江廣百餘步,足以行船。南越以財物役屬夜郎,西至同師,然亦不能臣使也。」蒙乃上書說上曰:「南越王黃屋左纛,地東西萬餘里,名為外臣,實一州主也。今以長沙、豫章往,水道多絕,難行。竊聞夜郎所有精兵,可得十餘萬,浮船牂柯江,出其不意,此制越一奇也。誠以漢之彊,巴蜀之饒,通夜郎道,為置吏,易甚。」上許之。乃拜蒙為郎中將,將千人,食重萬餘人,從巴蜀筰關入,遂見夜郎侯多同。蒙厚賜,喻以威德,約為置吏,使其子為令。夜郎旁小邑皆貪漢繒帛,以為漢道險,終不能有也,乃且聽蒙約。還報,乃以為犍為郡。發巴蜀卒治道,自僰道指牂柯江。蜀人司馬相如亦言西夷邛、筰可置郡。使相如以郎中將往喻,皆如南夷,為置一都尉,十餘縣,屬蜀。

當是時,巴蜀四郡通西南夷道,戍轉相馕。數歲,道不通,士罷餓離溼死者甚眾;西南夷又數反,發兵興擊,秏費無功。上患之,使公孫弘往視問焉。還對,言其不便。及弘為御史大夫,是時方筑朔方以據河逐胡,弘因數言西南夷害,可且罷,專力事匈奴。上罷西夷,獨置南夷夜郎兩縣一都尉,稍令犍為自葆就。

及元狩元年,博望侯張騫使大夏來,言居大夏時見蜀布、邛竹、杖,使問所從來,曰「從東南身毒國,可數千里,得蜀賈人市」。或聞邛西可二千里有身毒國。騫因盛言大夏在漢西南,慕中國,患匈奴隔其道,誠通蜀,身毒國道便近,有利無害。於是天子乃令王然于、柏始昌、呂越人等,使閒出西夷西,指求身毒國。至滇,滇王嘗羌乃留,為求道西十餘輩。歲餘,皆閉昆明,莫能通身毒國。

滇王與漢使者言曰:「漢孰與我大?」及夜郎侯亦然。以道不通故,各自以為一州主,不知漢廣大。使者還,因盛言滇大國,足事親附。天子注意焉。

及至南越反,上使馳義侯因犍為發南夷兵。且蘭君恐遠行,旁國虜其老弱,乃與其眾反,殺使者及犍為太守。漢乃發巴蜀罪人嘗擊南越者八校尉擊破之。會越已破,漢八校尉不下,即引兵還,行誅頭蘭。頭蘭,常隔滇道者也。已平頭蘭,遂平南夷為牂柯郡。夜郎侯始倚南越,南越已滅,會還誅反者,夜郎遂入朝。上以為夜郎王。

南越破後,及漢誅且蘭、邛君,并殺筰侯,冄駹皆振恐,請臣置吏。乃以邛都為越巂郡,筰都為沈犁郡,冄駹為汶山郡,廣漢西白馬為武都郡。

上使王然于以越破及誅南夷兵威風喻滇王入朝。滇王者,其眾數萬人,其旁東北有勞寖、靡莫,皆同姓相扶,未肯聽。勞寖、靡莫數侵犯使者吏卒。元封二年,天子發巴蜀兵擊滅勞寖、靡莫,以兵臨滇。滇王始首善,以故弗誅。滇王離難西南夷,舉國降,請置吏入朝。於是以為益州郡,賜滇王王印,復長其民。

西南夷君長以百數,獨夜郎、滇受王印。滇小邑,最寵焉。

太史公曰:楚之先豈有天祿哉?在周為文王師,封楚。及周之衰,地稱五千里。秦滅諸候,唯楚苗裔尚有滇王。漢誅西南夷,國多滅矣,唯滇復為寵王。然南夷之端,見枸醬番禺,大夏杖、邛竹。西夷后揃,剽分二方,卒為七郡。


\end{pinyinscope}