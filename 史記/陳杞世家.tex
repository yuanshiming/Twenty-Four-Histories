\article{陳杞世家}

\begin{pinyinscope}
陳胡公滿者,虞帝舜之後也。昔舜為庶人時,堯妻之二女,居于媯汭,其後因為氏姓,姓媯氏。舜已崩,傳禹天下,而舜子商均為封國。夏后之時,或失或續。至于周武王克殷紂,乃復求舜後,得媯滿,封之於陳,以奉帝舜祀,是為胡公。

胡公卒,子申公犀侯立。申公卒,弟相公皋羊立。相公卒,立申公子突,是為孝公。孝公卒,子慎公圉戎立。慎公當周厲王時。慎公卒,子幽公寧立。

幽公十二年,周厲王奔于彘。

二十三年,幽公卒,子釐公孝立。釐公六年,周宣王即位。三十六年,釐公卒,子武公靈立。武公十五年卒,子夷公說立。是歲,周幽王即位。夷公三年卒,弟平公燮立。平公七年,周幽王為犬戎所殺,周東徙。秦始列為諸侯。

二十三年,平公卒,子文公圉立。

文公元年,取蔡女,生子佗。十年,文公卒,長子桓公鮑立。

桓公二十三年,魯隱公初立。二十六年,衛殺其君州吁。三十三年,魯弒其君隱公。

三十八年正月甲戌己丑,桓公鮑卒。桓公弟佗,其母蔡女,故蔡人為佗殺五父及桓公太子免而立佗,是為厲公。桓公病而亂作,國人分散,故再赴。

厲公二年,生子敬仲完。周太史過陳,陳厲公使以《周易》筮之,卦得觀之否:「是為觀國之光,利用賓于王。此其代陳有國乎?不在此,其在異國?非此其身,在其子孫。若在異國,必姜姓。姜姓,太嶽之後。物莫能兩大,陳衰,此其昌乎?」

厲公取蔡女,蔡女與蔡人亂,厲公數如蔡淫。七年,厲公所殺桓公太子免之三弟,長曰躍,中曰林,少曰杵臼,共令蔡人誘厲公以好女,與蔡人共殺厲公而立躍,是為利公。利公者,桓公子也。利公立五月卒,立中弟林,是為莊公。莊公七年卒,少弟杵臼立,是為宣公。

宣公三年,楚武王卒,楚始彊。十七年,周惠王娶陳女為后。

二十一年,宣公後有嬖姬生子款,欲立之,乃殺其太子御寇。御寇素愛厲公子完,完懼禍及己,乃奔齊。齊桓公欲使陳完為卿,完曰:「羈旅之臣,幸得免負檐,君之惠也,不敢當高位。」桓公使為工正。齊懿仲欲妻陳敬仲,卜之,占曰:「是謂鳳皇于飛,和鳴鏘鏘。有媯之後,將育于姜。五世其昌,并于正卿。八世之後,莫之與京。」

三十七年,齊桓公伐蔡,蔡敗;南侵楚,至召陵,還過陳。陳大夫轅濤涂惡其過陳,詐齊令出東道。東道惡,桓公怒,執陳轅濤涂。是歲,晉獻公殺其太子申生。

四十五年,宣公卒,子款立,是為穆公。穆公五年,齊桓公卒。十六年,晉文公敗楚師于城濮。是歲,穆公卒,子共公朔立。共公六年,楚太子商臣弒其父成王代立,是為穆王。十一年,秦穆公卒。十八年,共公卒,子靈公平國立。

靈公元年,楚莊王即位。六年,楚伐陳。十年,陳及楚平。

十四年,靈公與其大夫孔寧、儀行父皆通於夏姬,衷其衣以戲於朝。泄冶諫曰:「君臣淫亂,民何效焉?」靈公以告二子,二子請殺泄冶,公弗禁,遂殺泄冶。十五年,靈公與二子飲於夏氏。公戲二子曰:「徵舒似汝。」二子曰:「亦似公。」徵舒怒。靈公罷酒出,徵舒伏弩廏門射殺靈公。孔寧、儀行父皆奔楚,靈公太子午奔晉。徵舒自立為陳侯。徵舒,故陳大夫也。夏姬,御叔之妻,舒之母也。

成公元年冬,楚莊王為夏徵舒殺靈公,率諸侯伐陳。謂陳曰:「無驚,吾誅徵舒而已。」已誅徵舒,因縣陳而有之,群臣畢賀。申叔時使於齊來還,獨不賀。莊王問其故,對曰:「鄙語有之,牽牛徑人田,田主奪之牛。徑則有罪矣,奪之牛,不亦甚乎?今王以徵舒為賊弒君,故徵兵諸侯,以義伐之,已而取之,以利其地,則後何以令於天下!是以不賀。」莊王曰:「善。」乃迎陳靈公太子午於晉而立之,復君陳如故,是為成公。孔子讀史記至楚復陳,曰:「賢哉楚莊王!輕千乘之國而重一言。」

[二十]八年,楚莊王卒。二十九年,陳倍楚盟。三十年,楚共王伐陳。是歲,成公卒,子哀公弱立。楚以陳喪,罷兵去。

哀公三年,楚圍陳,復釋之。二十八年,楚公子圍弒其君郟敖自立,為靈王。

三十四年,初,哀公娶鄭,長姬生悼太子師,少姬生偃。二嬖妾,長妾生留,少妾生勝。留有寵哀公,哀公屬之其弟司徒招。哀公病,三月,招殺悼太子,立留為太子。哀公怒,欲誅招,招發兵圍守哀公,哀公自經殺。招卒立留為陳君。四月,陳使使赴楚。楚靈王聞陳亂,乃殺陳使者,使公子棄疾發兵伐陳,陳君留奔鄭。九月,楚圍陳。十一月,滅陳。使棄疾為陳公。

招之殺悼太子也,太子之子名吳,出奔晉。晉平公問太史趙曰:「陳遂亡乎?」對曰:「陳,顓頊之族。陳氏得政於齊,乃卒亡。自幕至于瞽瞍,無違命。舜重之以明德。至於遂,世世守之。及胡公,周賜之姓,使祀虞帝。且盛德之後,必百世祀。虞之世未也,其在齊乎?」

楚靈王滅陳五歲,楚公子棄疾弒靈王代立,是為平王。平王初立,欲得和諸侯,乃求故陳悼太子師之子吳,立為陳侯,是為惠公。惠公立,探續哀公卒時年而為元,空籍五歲矣。

十年,陳火。十五年,吳王僚使公子光伐陳,取胡、沈而去。二十八年,吳王闔閭與子胥敗楚入郢。是年,惠公卒,子懷公柳立。

懷公元年,吳破楚,在郢,召陳侯。陳侯欲往,大夫曰:「吳新得意;楚王雖亡,與陳有故,不可倍。」懷公乃以疾謝吳。四年,吳復召懷公。懷公恐,如吳。吳怒其前不往,留之,因卒吳。陳乃立懷公之子越,是為湣公。

湣公六年,孔子適陳。吳王夫差伐陳,取三邑而去。十三年,吳復來伐陳,陳告急楚,楚昭王來救,軍於城父,吳師去。是年,楚昭王卒於城父。時孔子在陳。十五年,宋滅曹。十六年,吳王夫差伐齊,敗之艾陵,使人召陳侯。陳侯恐,如吳。楚伐陳。二十一年,齊田常弒其君簡公。二十三年,楚之白公勝殺令尹子西、子綦,襲惠王。葉公攻敗白公,白公自殺。

二十四年,楚惠王復國,以兵北伐,殺陳湣公,遂滅陳而有之。是歲,孔子卒。

杞東樓公者,夏后禹之後苗裔也。殷時或封或絕。周武王克殷紂,求禹之後,得東樓公,封之於杞,以奉夏后氏祀。

東樓公生西樓公,西樓公生題公,題公生謀娶公。謀娶公當周厲王時。謀娶公生武公。武公立四十七年卒,子靖公立。靖公二十三年卒,子共公立。共公八年卒,子德公立。德公十八年卒,弟桓公姑容立。桓公十七年卒,子孝公丐立。孝公十七年卒,弟文公益姑立。文公十四年卒,弟平公郁立。平公十八年卒,子悼公成立。悼公十二年卒,子隱公乞立。七月,隱公弟遂弒隱公自立,是為釐公。釐公十九年卒,子湣公維立。湣公十五年,楚惠王滅陳。十六年,湣公弟閼路弒湣公代立,是為哀公。哀公立十年卒,湣公子敕立,是為出公。出公十二年卒,子簡公春立。立一年,楚惠王之四十四年,滅杞。杞後陳亡三十四年。

杞小微,其事不足稱述。

舜之後,周武王封之陳,至楚惠王滅之,有世家言。禹之後,周武王封之杞,楚惠王滅之,有世家言。契之後為殷,殷有本紀言。殷破,周封其後於宋,齊湣王滅之,有世家言。后稷之後為周,秦昭王滅之,有本紀言。皋陶之後,或封英、六,楚穆王滅之,無譜。伯夷之後,至周武王復封於齊,曰太公望,陳氏滅之,有世家言。伯翳之後,至周平王時封為秦,項羽滅之,有本紀言。垂、益、夔、龍,其後不知所封,不見也。右十一人者,皆唐虞之際名有功德臣也;其五人之後皆至帝王,餘乃為顯諸侯。滕、薛、騶,夏、殷、周之閒封也,小,不足齒列,弗論也。

周武王時,侯伯尚千餘人。及幽、厲之後,諸侯力攻相并。江、黃、胡、沈之屬,不可勝數,故弗采著于傳(上)[云]。太史公曰:舜之德可謂至矣!禪位於夏,而後世血食者歷三代。及楚滅陳,而田常得政於齊,卒為建國,百世不絕,苗裔茲茲,有土者不乏焉。至禹,於周則杞,微甚,不足數也。楚惠王滅杞,其後越王句踐興。


\end{pinyinscope}