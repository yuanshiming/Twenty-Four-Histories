\article{司馬相如列傳}

\begin{pinyinscope}
司馬相如者,蜀郡成都人也,字長卿。少時好讀書,學擊劍,故其親名之曰犬子。相如既學,慕藺相如之為人,更名相如。以貲為郎,事孝景帝,為武騎常侍,非其好也。會景帝不好辭賦,是時梁孝王來朝,從游說之士齊人鄒陽、淮陰枚乘、吳莊忌夫子之徒,相如見而說之,因病免,客游梁。梁孝王令與諸生同舍,相如得與諸生游士居數歲,乃著子虛之賦。

會梁孝王卒,相如歸,而家貧,無以自業。素與臨邛令王吉相善,吉曰:「長卿久宦遊不遂,而來過我。」於是相如往,舍都亭。臨邛令繆為恭敬,日往朝相如。相如初尚見之,后稱病,使從者謝吉,吉愈益謹肅。臨邛中多富人,而卓王孫家僮八百人,程鄭亦數百人,二人乃相謂曰:「令有貴客,為具召之。」并召令。令既至,卓氏客以百數。至日中,謁司馬長卿,長卿謝病不能往,臨邛令不敢嘗食,自往迎相如。相如不得已,彊往,一坐盡傾。酒酣,臨邛令前奏琴曰:「竊聞長卿好之,願以自娛。」相如辭謝,為鼓一再行。是時卓王孫有女文君新寡,好音,故相如繆與令相重,而以琴心挑之。相如之臨邛,從車騎,雍容閒雅甚都;及飲卓氏,弄琴,文君竊從戶窺之,心悅而好之,恐不得當也。既罷,相如乃使人重賜文君侍者通殷勤。文君夜亡奔相如,相如乃與馳歸成都。家居徒四壁立。卓王孫大怒曰:「女至不材,我不忍殺,不分一錢也。」人或謂王孫,王孫終不聽。文君久之不樂,曰:「長卿第俱如臨邛,從昆弟假貸猶足為生,何至自苦如此!」相如與俱之臨邛,盡賣其車騎,買一酒舍酤酒,而令文君當鑪。相如身自著犢鼻褌,與保庸雜作,滌器於市中。卓王孫聞而恥之,為杜門不出。昆弟諸公更謂王孫曰:「有一男兩女,所不足者非財也。今文君已失身於司馬長卿,長卿故倦游,雖貧,其人材足依也,且又令客,獨柰何相辱如此!」卓王孫不得已,分予文君僮百人,錢百萬,及其嫁時衣被財物。文君乃與相如歸成都,買田宅,為富人。

居久之,蜀人楊得意為狗監,侍上。上讀子虛賦而善之,曰:「朕獨不得與此人同時哉!」得意曰:「臣邑人司馬相如自言為此賦。」上驚,乃召問相如。相如曰:「有是。然此乃諸侯之事,未足觀也。請為天子游獵賦,賦成奏之。」上許,令尚書給筆札。相如以「子虛」,虛言也,為楚稱;「烏有先生」者,烏有此事也,為齊難;「無是公」者,無是人也,明天子之義。故空藉此三人為辭,以推天子諸侯之苑囿。其卒章歸之於節儉,因以風諫。奏之天子,天子大說。其辭曰:

楚使子虛使於齊,齊王悉發境內之士,備車騎之眾,與使者出田。田罷,子虛過詫烏有先生,而無是公在焉。坐定,烏有先生問曰:「今日田樂乎?」子虛曰:「樂。」「獲多乎?」曰:「少。」「然則何樂?」曰:「仆樂齊王之欲夸仆以車騎之眾,而仆對以雲夢之事也。」曰:「可得聞乎?」

子虛曰:「可。王駕車千乘,選徒萬騎,田於海濱。列卒滿澤,罘罔彌山,揜兔轔鹿,射麋腳鱗。騖於鹽浦,割鮮染輪射中獲多,顧謂仆曰:『楚亦有平原廣澤游獵之地饒樂若此者乎?楚王之獵何與寡人?』仆下車對曰:『臣,楚國之鄙人也,幸得宿衛十有餘年,時從出游,游於後園,覽於有無,然猶未能遍覩也,又惡足以言其外澤者乎!』齊王曰:『雖然,略以子之所聞見而言之。』

「仆對曰:『唯唯。臣聞楚有七澤,嘗見其一,未覩其余也。臣之所見,蓋特其小小者耳,名曰雲夢。雲夢者,方九百里,其中有山焉。其山則盤紆岪郁,隆崇嵂崒;岑巖參差,日月蔽虧;交錯糾紛,上干青雲;罷池陂陁,下屬江河。其土則丹青赭堊,雌黃白坿,錫碧金銀,眾色炫燿,照爛龍鱗。其石則赤玉玫瑰,琳瑉琨珸,瑊玏玄厲,瑌石武夫。其東則有蕙圃衡蘭,芷若射干,穹窮昌蒲,江離麋蕪,諸蔗猼且。其南則有平原廣澤,登降陁靡,案衍壇曼,緣以大江,限以巫山。其高燥則生葴簛苞荔,薛莎青薠。其卑溼則生藏莨蒹葭,東薔雕胡,蓮藕菰蘆,蔄軒芋,眾物居之,不可勝圖。其西則有湧泉清池,激水推移;外發芙蓉蔆華,內隱鉅石白沙。其中則有神龜蛟鼉,瑁鱉黿。其北則有陰林巨樹,楩枏豫章,桂椒木蘭,蘗離朱楊,櫨梸梬栗,橘柚芬芳。其上則有赤猨蠷蝚,鹓雛孔鸞,騰遠射干。其下則有白虎玄豹,蟃蜒貙豻,兕象野犀,窮奇獌狿。

「『於是乃使專諸之倫,手格此獸。楚王乃駕馴駁之駟,乘雕玉之輿,靡魚須之橈旃,曳明月之珠旗,建干將之雄戟,左烏嗥之雕弓,右夏服之勁箭;陽子驂乘,纖阿為御;案節未舒,即陵狡獸,轔邛邛,蹵距虛,軼野馬而韢騊駼,乘遺風而射游騏;儵眒凄浰,雷動熛至,星流霆擊,弓不虛發,中必決眥,洞胸達腋,絕乎心系,獲若雨獸,揜草蔽地。於是楚王乃弭節裴回,翱翔容與,覽乎陰林,觀壯士之暴怒,與猛獸之恐懼,徼劇受詘,殫睹眾物之變態。

「『於是鄭女曼姬,被阿錫,揄紵縞,纖羅,垂霧縠;襞積褰縐,紆徐委曲,郁橈谿谷;衯衯裶裶,揚袘卹削,蜚纖垂髾;扶與猗靡,膳萃蔡,下摩蘭蕙,上拂羽蓋,錯翡翠之威蕤,繆繞玉綏;縹乎忽忽,若神僊之仿佛。

「『於是乃相與獠於蕙圃,媻珊勃窣上金隄,揜翡翠,射鵕鸃,微矰出,纖繳施,弋白鵠,連駕鵝,雙鶬下,玄鶴加。怠而後發,游於清池;浮文鹢,揚桂枻,張翠帷,建羽蓋,罔瑁,釣紫貝;摐金鼓,吹鳴籟,榜人歌,聲流喝,水蟲駭,波鴻沸,涌泉起,奔揚會,礧石相擊,硠硠礚礚,若雷霆之聲,聞乎數百里之外。

「『將息獠者,擊靈鼓,起烽燧,車案行,騎就隊,纚乎淫淫,班乎裔裔。於是楚王乃登陽雲之臺,泊乎無為,澹乎自持,勺藥之和具而後御之。不若大王終日馳騁而不下輿,脟割輪淬,自以為娛。臣竊觀之,齊殆不如。』於是王默然無以應仆也。」

烏有先生曰:「是何言之過也!足下不遠千里,來況齊國,王悉發境內之士,而備車騎之眾,以出田,乃欲力致獲,以娛左右也,何名為夸哉!問楚地之有無者,願聞大國之風烈,先生之餘論也。今足下不稱楚王之德厚,而盛推雲夢以為高,奢言淫樂而顯侈靡,竊為足下不取也。必若所言,固非楚國之美也。有而言之,是章君之惡;無而言之,是害足下之信。章君之惡而傷私義,二者無一可,而先生行之,必且輕於齊而累於楚矣。且齊東陼巨海,南有瑯邪,觀乎成山,射乎之罘,浮勃澥,游孟諸,邪與肅慎為鄰,右以湯谷為界,秋田乎青丘,傍偟乎海外,吞若雲夢者八九,其於胸中曾不蔕芥。若乃俶儻瑰偉,異方殊類,珍怪鳥獸,萬端鱗萃,充仞其中者,不可勝記,禹不能名,契不能計。然在諸侯之位,不敢言游戲之樂,苑囿之大;先生又見客,是以王辭而不復,何為無用應哉!」

無是公聽然而笑曰:「楚則失矣,齊亦未為得也。夫使諸侯納貢者,非為財幣,所以述職也;封疆畫界者,非為守御,所以禁淫也。今齊列為東藩,而外私肅慎,捐國踰限,越海而田,其於義故未可也。且二君之論,不務明君臣之義而正諸侯之禮,徒事爭游獵之樂,苑囿之大,欲以奢侈相勝,荒淫相越,此不可以揚名發譽,而適足以貶君自損也。且夫齊楚之事又焉足道邪!君未睹夫巨麗也,獨不聞天子之上林乎?

「左蒼梧,右西極,丹水更其南,紫淵徑其北;終始霸滻,出入涇渭;酆鄗潦潏,紆餘委蛇,經營乎其內。蕩蕩兮八川分流,相背而異態。東西南北,馳騖往來,出乎椒丘之闕,行乎洲淤之浦,徑乎桂林之中,過乎泱莽之野。汨乎渾流,順阿而下,赴隘陜之口。觸穹石,激堆埼,沸乎暴怒,洶涌滂撓,蜿灗膠戾,湢測泌瀄,橫流逆折,轉騰潎洌,澎濞沆瀣,穹隆雲撓,蜿灗膠戾,踰波趨浥,蒞蒞下瀨,批壧衝壅,奔揚滯沛,臨坻注壑,瀺灂霣墜,湛湛隱隱,砰磅訇礚,潏潏淈淈,湁潗鼎沸,馳波跳沫,汩急漂疾,悠遠長懷,寂漻無聲,肆乎永歸。然後灝溔潢漾,安翔徐徊,翯乎滈滈,東注大湖,衍溢陂池。於是乎蛟龍赤螭,鯁鰽螹離,鰅鳙鯱魠,禺禺鱋魶,揵鰭擢尾,振鱗奮翼,潛處于深巖;魚鱉讙聲,萬物眾夥,明月珠子,玓瓅江靡,蜀石黃碝,水玉磊砢,磷磷爛爛,采色澔旰,叢積乎其中。鴻鵠鷫鴇,鴐鵝鸀玉,鵁鶄鹮目,煩鶩鷛鸜,鷻鴜鵁鸕,群浮乎其上。汎淫泛濫,隨風澹淡,與波搖蕩,掩薄草渚,唼喋菁藻,咀嚼蔆藕。

「於是乎崇山巃嵸,崔巍嵯峨,深林鉅木,嶄巖嵾嵯,九嵏、嶻嶭,南山峨峨,巖陁甗锜,嶊崣崛崎,振谿通谷,蹇產溝瀆,谽呀豁閜,阜陵別島,崴磈嵔瘣,丘虛崛礨,隱轔郁礨,登降施靡,陂池貏豸,沇溶淫鬻,散渙夷陸,亭皋千里,靡不被筑。掩以綠蕙,被以江離,糅以蘪蕪,雜以流夷。專結縷,欑戾莎,揭車衡蘭,槁本射干,茈薑蘘荷,葴橙若蓀,鮮枝黃礫,蔣芧青薠,布濩閎澤,延曼太原,麗靡廣衍,應風披靡,吐芳揚烈,郁郁斐斐,眾香發越,肸蠁布寫,崦瞹苾勃。

「於是乎周覽泛觀,瞋盼軋沕,芒芒恍忽,視之無端,察之無崖。日出東沼,入於西陂。其南則隆冬生長,踴水躍波;獸則庸獏牦犛,沈牛麈麋,赤首圜題,窮奇象犀。其北則盛夏含凍裂地,涉冰揭河;獸則麒麟角端,騊駼橐駞,蛩蛩驒騱,駃騠驢騾。

「於是乎離宮別館,彌山跨谷,高廊四注,重坐曲閣,華榱璧璫,輦道纚屬,步櫩周流,長途中宿。夷嵏筑堂,纍臺增成,巖穾洞房,俛杳眇而無見,仰攀橑而捫天,奔星更於閨闥,宛虹拖於楯軒。青虯蚴蟉於東箱,象輿婉蟬於西清,靈圉燕於閒觀,偓佺之倫暴於南榮,醴泉涌於清室,通川過乎中庭。槃石裖崖,嵚巖倚傾,嵯峨磼礏,刻削崢嶸,玫瑰碧琳,珊瑚叢生,瑉玉旁唐,璸斒文鱗,赤瑕駁犖,雜臿其閒,垂綏琬琰,和氏出焉。

「於是乎盧橘夏孰,黃甘橙楱,枇杷橪柿,楟柰厚樸,梬棗楊梅,櫻桃蒲陶,隱夫郁棣,榙遝荔枝,羅乎后宮,列乎北園。貤丘陵,下平原,揚翠葉,杌紫莖,發紅華,秀朱榮,煌煌扈扈,照曜鉅野。沙棠櫟櫧,華氾檘櫨,留落胥餘,仁頻并閭,欃檀木蘭,豫章女貞,長千仞,大連抱,夸條直暢,實葉葰茂,攢立叢倚,連卷累佹,崔錯委骫,阬衡閜砢,垂條扶於,落英幡纚,紛容蕭蔘,旖旎從風,瀏蒞芔吸,蓋象金石之聲,管籥之音。柴池茈虒,旋環后宮,雜遝累輯,被山緣谷,循阪下隰,視之無端,究之無窮。

「於是玄猨素雌,蜼玃飛鸓,蛭蜩蠷蝚,螹胡豰蛫,棲息乎其閒;長嘯哀鳴,翩幡互經,夭蟜枝格,偃蹇杪顛。於是乎隃絕梁,騰殊榛,捷垂條,踔稀閒,牢落陸離,爛曼遠遷。

「若此輩者,數千百處。嬉游往來,宮宿館舍,庖廚不徙,後宮不移,百官備具。

「於是乎背秋涉冬,天子校獵。乘鏤象,六玉虯,拖蜺旌,靡雲旗,前皮軒,后道游;孫叔奉轡,衛公驂乘,扈從橫行,出乎四校之中。鼓嚴簿,縱獠者,江河為阹,泰山為櫓,車騎雷起,隱天動地,先後陸離,離散別追,淫淫裔裔,緣陵流澤,雲布雨施。」

「生貔豹,搏豺狼,手熊羆,足野羊,蒙鹖蘇,绔白虎,被豳文,跨野馬。陵三嵏之危,下磧歷之坻;俓陖赴險,越壑厲水。推蜚廉,弄解豸,格瑕蛤,鋋猛氏,罥騕褭,射封豕。箭不茍害,解脰陷腦;弓不虛發,應聲而倒。於是乎乘輿彌節裴回,翺翔往來,睨部曲之進退,覽將率之變態。然後浸潭促節,儵夐遠去,流離輕禽,蹴履狡獸,轊白鹿,捷狡兔,軼赤電,遺光燿,追怪物,出宇宙,彎繁弱,滿白羽,射游梟,櫟蜚虡,擇肉后發,先中命處,弦矢分,藝殪仆。

「然後揚節而上浮,陵驚風,歷駭飚,乘虛無,與神俱,轔玄鶴,亂昆雞。遒孔鸞,促鵔鸃,拂鹥鳥,捎鳳皇,捷鴛雛,掩焦明。

「道盡涂殫,迴車而還。招搖乎襄羊,降集乎北纮,率乎直指,闇乎反鄉。蹶(闕)[關],歷封巒,過鳷鵲,望露寒,下棠梨,息宜春,西馳宣曲,濯鹢牛首,登龍臺,掩細柳,觀士大夫之勤略,鈞獠者之所得獲。徒車之所轔轢,乘騎之所蹂若,人民之所蹈躤,與其窮極倦劇,驚憚慴伏,不被創刃而死者,佗佗籍籍,填阬滿谷,揜平彌澤。

「於是乎游戲懈怠,置酒乎昊天之臺,張樂乎轇輵之宇;撞千石之鐘,立萬石之鉅;建翠華之旗,樹靈鼉之鼓。奏陶唐氏之舞,聽葛天氏之歌,千人唱,萬人和,山陵為之震動,川谷為之蕩波。巴俞宋蔡,淮南于遮,文成顛歌,族舉遞奏,金鼓迭起,鏗鎗鐺鼞,洞心駭耳。荊吳鄭衛之聲,韶濩武象之樂,陰淫案衍之音,鄢郢繽紛,激楚結風,俳優侏儒,狄鞮之倡,所以娛耳目而樂心意者,麗靡爛漫於前,靡曼美色於後。

「若夫青琴宓妃之徒,絕殊離俗,姣冶嫻都,靚莊刻飭,便嬛綽約,柔橈嬛嬛,娬媚姌嫋;抴獨繭之褕袘,眇閻易以戌削,媥姺徶屑,與世殊服;芬香漚郁,酷烈淑郁;皓齒粲爛,宜笑旳皪;長眉連娟,微睇綿藐;色授魂與,心愉於側。

「於是酒中樂酣,天子芒然而思,似若有亡。曰:『嗟乎,此泰奢侈!朕以覽聽餘閒,無事棄日,順天道以殺伐,時休息於此,恐後世靡麗,遂往而不反,非所以為繼嗣創業垂統也。』於是乃解酒罷獵,而命有司曰:『地可以墾辟,悉為農郊,以贍萌隸;隤墻填塹,使山澤之民得至焉。實陂池而勿禁,虛宮觀而勿仞。發倉廩以振貧窮,補不足,恤寡,存孤獨。出德號,省刑罰,改制度,易服色,更正朔,與天下為始。』

「於是歷吉日以齊戒,襲朝衣,乘法駕,建華旗,鳴玉鸞,游乎六藝之囿,騖乎仁義之涂,覽觀春秋之林,射貍首,兼騶虞,弋玄鶴,建干戚,載雲睅,揜群雅,悲伐檀,樂樂胥,修容乎禮園,翺翔乎書圃,述易道,放怪獸,登明堂,坐清廟,恣群臣,奏得失,四海之內,靡不受獲。於斯之時,天下大說,向風而聽,隨流而化,喟然興道而遷義,刑錯而不用,德隆乎三皇,功羨於五帝。若此,故獵乃可喜也。

「若夫終日暴露馳騁,勞神苦形,罷車馬之用,抏士卒之精,費府庫之財,而無德厚之恩,務在獨樂,不顧眾庶,忘國家之政,而貪雉兔之獲,則仁者不由也。從此觀之,齊楚之事,豈不哀哉!地方不過千里,而囿居九百,是草木不得墾辟,而民無所食也。夫以諸侯之細,而樂萬乘之所侈,仆恐百姓之被其尤也。」

於是二子愀然改容,超若自失,逡巡避席曰:「鄙人固陋,不知忌諱,乃今日見教,謹聞命矣。」

賦奏,天子以為郎。無是公言天子上林廣大,山谷水泉萬物,乃子虛言楚雲夢所有甚眾,侈靡過其實,且非義理所尚,故刪取其要,歸正道而論之。

相如為郎數歲,會唐蒙使略通夜郎西僰中,發巴蜀吏卒千人,郡又多為發轉漕萬餘人,用興法誅其渠帥,巴蜀民大驚恐。上聞之,乃使相如責唐蒙,因喻告巴蜀民以非上意。檄曰:

告巴蜀太守:蠻夷自擅不討之日久矣,時侵犯邊境,勞士大夫。陛下即位,存撫天下,輯安中國。然後興師出兵,北征匈奴,單于怖駭,交臂受事,詘膝請和。康居西域,重譯請朝,稽首來享。移師東指,閩越相誅。右弔番禺,太子入朝。南夷之君,西僰之長,常效貢職,不敢怠墮,延頸舉踵,喁喁然皆爭歸義,欲為臣妾,道裏遼遠,山川阻深,不能自致。夫不順者已誅,而為善者未賞,故遣中郎將往賓之,發巴蜀士民各五百人,以奉幣帛,衛使者不然,靡有兵革之事,戰鬬之患。今聞其乃發軍興制,驚懼子弟,憂患長老,郡又擅為轉粟運輸,皆非陛下之意也。當行者或亡逃自賊殺,亦非人臣之節也。

夫邊郡之士,聞烽舉燧燔,皆攝弓而馳,荷兵而走,流汗相屬,唯恐居后,觸白刃,冒流矢,義不反顧,計不旋踵,人懷怒心,如報私讎。彼豈樂死惡生,非編列之民,而與巴蜀異主哉?計深慮遠,急國家之難,而樂盡人臣之道也。故有剖符之封,析珪而爵,位為通侯,居列東第,終則遺顯號於後世,傳土地於子孫,行事甚忠敬,居位甚安佚,名聲施於無窮,功烈著而不滅。是以賢人君子,肝腦涂中原,膏液潤野草而不辭也。今奉幣役至南夷,即自賊殺,或亡逃抵誅,身死無名,謚為至愚,恥及父母,為天下笑。人之度量相越,豈不遠哉!然此非獨行者之罪也,父兄之教不先,子弟之率不謹也;寡廉鮮恥,而俗不長厚也。其被刑戮,不亦宜乎!

陛下患使者有司之若彼,悼不肖愚民之如此,故遣信使曉喻百姓以發卒之事,因數之以不忠死亡之罪,讓三老孝弟以不教誨之過。方今田時,重煩百姓,已親見近縣,恐遠所谿谷山澤之民不遍聞,檄到,亟下縣道,使咸知陛下之意,唯毋忽也。

相如還報。唐蒙已略通夜郎,因通西南夷道,發巴、蜀、廣漢卒,作者數萬人。治道二歲,道不成,士卒多物故,費以巨萬計。蜀民及漢用事者多言其不便。是時邛筰之君長聞南夷與漢通,得賞賜多,多欲願為內臣妾,請吏,比南夷。天子問相如,相如曰:「邛、筰、冉、駹者近蜀,道亦易通,秦時嘗通為郡縣,至漢興而罷。今誠復通,為置郡縣,愈於南夷。」天子以為然,乃拜相如為中郎將,建節往使。副使王然于、壺充國、呂越人馳四乘之傳,因巴蜀吏幣物以賂西夷。至蜀,蜀太守以下郊迎,縣令負弩矢先驅,蜀人以為寵。於是卓王孫、臨邛諸公皆因門下獻牛酒以交驩。卓王孫喟然而嘆,自以得使女尚司馬長卿晚,而厚分與其女財,與男等同。司馬長卿便略定西夷,邛、筰、冉、駹、斯榆之君皆請為內臣。除邊關,關益斥,西至沬、若水,南至牂柯為徼,通零關道,橋孫水以通邛都。還報天子,天子大說。

相如使時,蜀長老多言通西南夷不為用,唯大臣亦以為然。相如欲諫,業已建之,不敢,乃著書,籍以蜀父老為辭,而己詰難之,以風天子,且因宣其使指,令百姓知天子之意。其辭曰:

漢興七十有八載,德茂存乎六世,威武紛紜,湛恩汪濊,群生澍濡,洋溢乎方外。於是乃命使西征,隨流而攘,風之所被,罔不披靡。因朝冉從冄,定筰存邛,略斯榆,舉苞滿,結軼還轅,東鄉將報,至于蜀都。

耆老大夫薦紳先生之徒二十有七人,儼然造焉。辭畢,因進曰:「蓋聞天子之於夷狄也,其義羈縻勿絕而已。今罷三郡之士,通夜郎之涂,三年於茲,而功不竟,士卒勞倦,萬民不贍,今又接以西夷,百姓力屈,恐不能卒業,此亦使者之累也,竊為左右患之。且夫邛、筰、西僰之與中國并也,歷年茲多,不可記已。仁者不以德來,彊者不以力并,意者其殆不可乎!今割齊民以附夷狄,弊所恃以事無用,鄙人固陋,不識所謂。」

使者曰:「烏謂此邪?必若所云,則是蜀不變服而巴不化俗也。余尚惡聞若說。然斯事體大,固非觀者之所覯也。余之行急,其詳不可得聞已,請為大夫粗陳其略。

「蓋世必有非常之人,然後有非常之事;有非常之事,然後有非常之功。非常者,固常[人]之所異也。故曰非常之原,黎民懼焉;及臻厥成,天下晏如也。

「昔者鴻水浡出,氾濫衍溢,民人登降移徙,陭區而不安。夏后氏戚之,乃堙鴻水,決江疏河,漉沈贍菑,東歸之於海,而天下永寧。當斯之勤,豈唯民哉。心煩於慮而身親其勞,躬胝無胈,膚不生毛。故休烈顯乎無窮,聲稱浹乎于茲。

「且夫賢君之踐位也。豈特委瑣握嚙,拘文牽俗,循誦習傳,當世取說云爾哉!必將崇論閎議,創業垂統,為萬世規。故馳騖乎兼容并包,而勤思乎參天貳地。且詩不云乎:『普天之下,莫非王土;率土之濱,莫非王臣。』是以六合之內,八方之外,浸潯衍溢,懷生之物有不浸潤於澤者,賢君恥之。今封疆之內,冠帶之倫,咸獲嘉祉,靡有闕遺矣。而夷狄殊俗之國,遼絕異黨之地,舟輿不通,人跡罕至,政教未加,流風猶微。內之則犯義侵禮於邊境,外之則邪行橫作,放弒其上。君臣易位,尊卑失序,父兄不辜,幼孤為奴,系纍號泣,內向而怨,曰『蓋聞中國有至仁焉,德洋而恩普,物靡不得其所,今獨曷為遺己』。舉踵思慕,若枯旱之望雨。盭夫為之垂涕,況乎上聖,又惡能已?故北出師以討彊胡,南馳使以誚勁越。四面風德,二方之君鱗集仰流,願得受號者以億計。故乃關沬、若,徼牂柯,鏤零山,梁孫原。創道德之涂,垂仁義之統。將博恩廣施,遠撫長駕,使疏逖不閉,阻深闇昧得耀乎光明,以偃甲兵於此,而息誅伐於彼。遐邇一體,中外提福,不亦康乎?夫拯民於沈溺,奉至尊之休德,反衰世之陵遲,繼周氏之絕業,斯乃天子之急務也。百姓雖勞,又惡可以已哉?

「且夫王事固未有不始於憂勤,而終於佚樂者也。然則受命之符,合在於此矣。方將增泰山之封,加梁父之事,鳴和鸞,揚樂頌,上咸五,下登三。觀者未睹指,聽者未聞音,猶鷦明已翔乎寥廓,而羅者猶視乎藪澤。悲夫!」

於是諸大夫芒然喪其所懷來而失厥所以進,喟然并稱曰:「允哉漢德,此鄙人之所願聞也。百姓雖怠,請以身先之。」敞罔靡徙,因遷延而辭避。

其後人有上書言相如使時受金,失官。居歲餘,復召為郎。

相如口吃而善著書。常有消渴疾。與卓氏婚,饒於財。其進仕宦,未嘗肯與公卿國家之事,稱病閒居,不慕官爵。常從上至長楊獵,是時天子方好自擊熊彘,馳逐野獸,相如上疏諫之。其辭曰:

臣聞物有同類而殊能者,故力稱烏獲,捷言慶忌,勇期賁、育。臣之愚,竊以為人誠有之,獸亦宜然。今陛下好陵阻險,射猛獸,卒然遇軼材之獸,駭不存之地,犯屬車之清塵,輿不及還轅,人不暇施巧,雖有烏獲、逢蒙之伎,力不得用,枯木朽株盡為害矣。是胡越起於轂下,而羌夷接軫也,豈不殆哉!雖萬全無患,然本非天子之所宜近也。

且夫清道而後行,中路而後馳,猶時有銜橛之變,而況涉乎蓬蒿,馳乎丘墳,前有利獸之樂而內無存變之意,其為禍也不亦難矣!夫輕萬乘之重不以為安,而樂出於萬有一危之涂以為娛,臣竊為陛下不取也。

蓋明者遠見於未萌而智者避危於無形,禍固多藏於隱微而發於人之所忽者也。故鄙諺曰「家累千金,坐不垂堂」。此言雖小,可以喻大。臣願陛下之留意幸察。

上善之。還過宜春宮,相如奏賦以哀二世行失也。其辭曰:

登陂阤之長阪兮,坌入曾宮之嵯峨。臨曲江之碕州兮,望南山之參差。巖巖深山之谾谾兮,通谷谹兮谽谺。汨淢噏習以永逝兮,注平皋之廣衍。觀眾樹之塕薆兮,覽竹林之榛榛。東馳土山兮,北揭石瀨。彌節容與兮,歷弔二世。持身不謹兮,亡國失埶。信讒不寤兮,宗廟滅絕。嗚呼哀哉!操行之不得兮,墳墓蕪穢而不修兮,魂無歸而不食。夐邈絕而不齊兮,彌久遠而愈佅。精罔閬而飛揚兮,拾九天而永逝。嗚呼哀哉!

相如拜為孝文園令。天子既美子虛之事,相如見上好僊道,因曰:「上林之事未足美也,尚有靡者。臣嘗為大人賦,未就,請具而奏之。」相如以為列僊之傳居山澤閒,形容甚臞,此非帝王之僊意也,乃遂就大人賦。其辭曰:

世有大人兮,在于中州。宅彌萬里兮,曾不足以少留。悲世俗之迫隘兮,朅輕舉而遠遊。垂絳幡之素蜺兮,載雲氣而上浮。建格澤之長竿兮,總光耀之采旄。垂旬始以為幓兮,抴彗星而為髾。掉指橋以偃蹇兮,又旖旎以招搖。攬欃槍以為旌兮,靡屈虹而為綢。紅杳渺以眩湣兮,猋風涌而雲浮。駕應龍象輿之蠖略逶麗兮,驂赤螭青虯之幽蟉蜿蜒。低卬夭蟜據以驕驁兮,詘折隆癋蠼以連卷。沛艾赳螑仡以佁儗兮,放散畔岸驤以孱顏。跮踱輵轄容以委麗兮,綢繆偃蹇怵毚以梁倚。糾蓼叫奡蹋以艐路兮,蔑蒙踴躍騰而狂趡。蒞颯卉翕熛至電過兮,煥然霧除,霍然雲消。

邪絕少陽而登太陰兮,與真人乎相求。互折窈窕以右轉兮,橫厲飛泉以正東。悉徵靈圉而選之兮,部乘眾神於瑤光。使五帝先導兮,反太一而從陵陽。左玄冥而右含雷兮,前陸離而後潏湟。廝征伯僑而役羨門兮,屬岐伯使尚方。祝融驚而蹕御兮,清雰氣而後行。屯余車其萬乘兮,綷雲蓋而樹華旗。使句芒其將行兮,吾欲往乎南嬉。

歷唐堯於崇山兮,過虞舜於九疑。紛湛湛其差錯兮,雜遝膠葛以方馳。騷擾沖蓯其相紛挐兮,滂濞泱軋灑以林離。鉆羅列聚叢以蘢茸兮,衍曼流爛壇以陸離。徑入雷室之砰磷郁律兮,洞出鬼谷之崫礨嵬壞。遍覽八纮而觀四荒兮,朅渡九江而越五河。經營炎火而浮弱水兮,杭絕浮渚而涉流沙。奄息總極氾濫水嬉兮,使靈媧鼓瑟而舞馮夷。時若薆薆將混濁兮,召屏翳誅風伯而刑雨師。西望崑崙之軋沕洸忽兮,直徑馳乎三危。排閶闔而入帝宮兮,載玉女而與之歸。舒閬風而搖集兮,亢烏騰而一止。低回陰山翔以紆曲兮,吾乃今目睹西王母皬然白首。載勝而穴處兮,亦幸有三足烏為之使。必長生若此而不死兮,雖濟萬世不足以喜。

回車朅來兮,絕道不周,會食幽都。呼吸沆瀣兮餐朝霞兮,杳芝英兮嘰瓊華。嬐侵潯而高縱兮,紛鴻涌而上厲。貫列缺之倒景兮,涉豐隆之滂沛。馳游道而修降兮,騖遺霧而遠逝。迫區中之隘陝兮,舒節出乎北垠。遺屯騎於玄闕兮,軼先驅於寒門。下崢嶸而無地兮,上寥廓而無天。視眩眠而無見兮,聽惝恍而無聞。乘虛無而上假兮,超無友而獨存。

相如既奏大人之頌,天子大說,飄飄有凌雲之氣,似游天地之閒意。

相如既病免,家居茂陵。天子曰:「司馬相如病甚,可往從悉取其書;若不然,後失之矣。」使所忠往,而相如已死,家無書。問其妻,對曰:「長卿固未嘗有書也。時時著書,人又取去,即空居。長卿未死時,為一卷書,曰有使者來求書,奏之。無他書。」其遺札書言封禪事,奏所忠。忠奏其書,天子異之。其書曰:

伊上古之初肇,自昊穹兮生民,歷撰列辟,以迄于秦。率邇者踵武,逖聽者風聲。紛綸葳蕤,堙滅而不稱者,不可勝數也。續昭夏,崇號謚,略可道者七十有二君。罔若淑而不昌,疇逆失而能存?

軒轅之前,遐哉邈乎,其詳不可得聞也。五三六經載籍之傳,維見可觀也。書曰「元首明哉,股肱良哉」。因斯以談,君莫盛於唐堯,臣莫賢於后稷。后稷創業於唐,公劉發跡於西戎,文王改制,爰周郅隆,大行越成,而後陵夷衰微,千載無聲,豈不善始善終哉。然無異端,慎所由於前,謹遺教於後耳。故軌跡夷易,易遵也;湛恩濛涌,易豐也;憲度著明,易則也;垂統理順,易繼也。是以業隆於繦褓而崇冠于二后。揆厥所元,終都攸卒,未有殊尤絕跡可考于今者也。然猶躡梁父,登泰山,建顯號,施尊名。大漢之德,逢涌原泉,沕潏漫衍,旁魄四塞,雲尃霧散,上暢九垓,下泝八埏。懷生之類霑濡浸潤,協氣橫流,武節飄逝,邇陜游原,迥闊泳沫,首惡湮沒,闇昧昭晢,昆蟲凱澤,回首面內。然後囿騶虞之珍群,徼麋鹿之怪獸,噵一莖六穗於庖,犧雙觡共抵之獸,獲周餘珍收龜于岐,招翠黃乘龍於沼。鬼神接靈圉,賓於閒館。奇物譎詭,俶儻窮變。欽哉,符瑞臻茲,猶以為薄,不敢道封禪。蓋周躍魚隕杭,休之以燎,微夫斯之為符也,以登介丘,不亦恧乎!進讓之道,其何爽與?

於是大司馬進曰:「陛下仁育群生,義征不憓,諸夏樂貢,百蠻執贄,德侔往初,功無與二,休烈浹洽,符瑞眾變,期應紹至,不特創見。意者泰山、梁父設壇場望幸,蓋號以況榮,上帝垂恩儲祉,將以薦成,陛下謙讓而弗發也。挈三神之驩,缺王道之儀,群臣恧焉。或謂且天為質闇,珍符固不可辭;若然辭之,是泰山靡記而梁父靡幾也。亦各并時而榮,咸濟世而屈,說者尚何稱於後,而云七十二君乎?夫修德以錫符,奉符以行事,不為進越。故聖王弗替,而修禮地祇,謁款天神,勒功中岳,以彰至尊,舒盛德,發號榮,受厚福,以浸黎民也。皇皇哉斯事!天下之壯觀,王者之丕業,不可貶也。願陛下全之。而後因雜薦紳先生之略術,使獲燿日月之末光絕炎,以展采錯事,猶兼正列其義,校飭厥文,作春秋一藝,將襲舊六為七,攄之無窮,俾萬世得激清流,揚微波,蜚英聲,騰茂實。前聖之所以永保鴻名而常為稱首者用此,宜命掌故悉奏其義而覽焉。」

於是天子沛然改容,曰:「愉乎,朕其試哉!」乃遷思回慮,總公卿之議,詢封禪之事,詩大澤之博,廣符瑞之富。乃作頌曰:

自我天覆,雲之油油。甘露時雨,厥壤可游。滋液滲漉,何生不育;嘉穀六穗,我穡曷蓄。

非唯雨之,又潤澤之;非唯濡之,氾尃濩之。萬物熙熙,懷而慕思。名山顯位,望君之來。君乎君乎,侯不邁哉!

般般之獸,樂我君囿;白質黑章,其儀可(嘉)[喜];旼旼睦睦,君子之能。蓋聞其聲,今觀其來。厥涂靡蹤,天瑞之徵。茲亦於舜,虞氏以興。

濯濯之麟,游彼靈畤。孟冬十月,君俎郊祀。馳我君輿,帝以享祉。三代之前,蓋未嘗有。

宛宛黃龍,興德而升;采色炫燿,熿炳煇煌。正陽顯見,覺寤黎烝。於傳載之,云受命所乘。

厥之有章,不必諄諄。依類託寓,諭以封巒。

披藝觀之,天人之際已交,上下相發允答。聖王之德,兢兢翼翼也。故曰「興必慮衰,安必思危」。是以湯武至尊嚴,不失肅祗;舜在假典,顧省厥遺:此之謂也。

司馬相如既卒五歲,天子始祭后土。八年而遂先禮中嶽,封于太山,至梁父禪肅然。

相如他所著,若遺平陵侯書、與五公子相難、草木書篇不采,采其尤著公卿者云。

太史公曰:春秋推見至隱,易本隱之以顯,大雅言王公大人而德逮黎庶,小雅譏小己之得失,其流及上。所以言雖外殊,其合德一也。相如雖多虛辭濫說,然其要歸引之節儉,此與詩之風諫何異。楊雄以為靡麗之賦,勸百風一,猶馳騁鄭衛之聲,曲終而奏雅,不已虧乎?余采其語可論者著于篇。


\end{pinyinscope}