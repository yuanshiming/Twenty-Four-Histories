\article{卷四十二列傳第二十三 王晏 蕭諶 蕭坦之 江祏}

\begin{pinyinscope}

 王晏,字士彥,琅邪臨沂人也。祖弘之,通直常侍。父普曜,秘書監。宋大明末,晏起家臨賀王國常侍,員外郎,巴陵王征北板參軍,安成王撫軍板刑獄,隨府轉車騎。晉熙王燮為郢州,晏為安西主簿。世祖為長史,與晏相遇。府轉鎮西,板晏記室諮議。



 沈攸之事難,鎮西職僚皆隨世祖鎮盆城。上時權勢雖重,而眾情猶有疑惑,晏便專心奉事,軍旅書翰皆委焉。性甚便僻,漸見親侍。乃留為上征虜撫軍
 府板諮議,領記室。從還都,遷領軍司馬,中軍從事中郎。常在上府,參議機密。建元初,轉太子中庶子。世祖在東宮,專斷朝事,多不聞啟,晏慮及罪,稱疾自疏。尋領射聲校尉,不拜。世祖即位,轉長兼侍中,意任如舊。



 永明元年,領步兵校尉,遷侍中祭酒,校尉如故。遭母喪,起為輔國將軍、司徒左長史。晏父普曜藉晏勢宦,多歷通官。晏尋遷左衛將軍,加給事中,未拜,而普曜卒,居喪有稱。起冠軍將軍、司徒左長史、濟陽太守,未拜,遷衛尉,將軍如故。四年,轉太子詹事,加散騎常侍。六年,轉丹陽尹,常侍如故。晏位任親重,朝夕進見,言論朝事,自豫章王嶷、尚書令王儉皆降意以接之,而晏每以疏漏被上呵責,連稱疾久之。上以晏須祿養,七年,轉為江州刺史。晏固辭不願出外,見許,留為吏部尚書,領太子右衛率。終以舊恩見寵。時王儉雖貴而疏,晏既領選,權行臺閣,與儉頗不平。儉卒,禮官
 議謚,上欲依王導謚為「文獻」,晏啟上曰:「導乃得此謚,但宋以來,不加素族。」出謂親人曰:「平頭憲事已行矣。」八年,改領右衛將軍,陳疾自解。上欲以高宗代晏領選,手敕問之。晏啟曰:「鸞清幹有餘,然不諳百氏,恐不可居此職。」上乃止。明年,遷侍中,領太子詹事,本州中正,又以疾辭。十年,改授散騎常侍、金紫光祿大夫,給親信二十人,中正如故。十一年,遷右僕射,領太孫右衛率。



 世祖崩,遺旨以尚書事付晏及徐孝嗣,令久於其職。鬱林即位,轉左僕射,中正如故。隆昌元年,加侍中。高宗謀廢立,晏便響應推奉。延興元年,轉尚書令,加後將軍,侍中、中正如故。封曲江縣侯,邑千戶。給鼓吹一部,甲仗五十人入殿。



 高宗與晏宴於東府,語及時事,晏抵掌曰:「公常言晏怯,今定何如?」建武元年,進號驃騎大將軍,給班劍二十人,侍中、令、中正如故。又加兵百人,領太子少傅,進爵為公,增邑為二千戶。以
 虜動,給兵千人。



 晏為人篤於親舊,為世祖所稱。至是自謂佐命惟新,言論常非薄世祖故事,眾始怪之。高宗雖以事際須晏,而心相疑斥,料簡世祖中詔,得與晏手敕三百餘紙,皆是論國家事,以此愈猜薄之。初即位,始安王遙光便勸誅晏,帝曰:「晏於我有勳,且未有罪。」遙光曰:「晏尚不能為武帝,安能為陛下。」帝默然變色。時帝常遣心腹左右陳世範等出塗巷採聽異言,由是以晏為事。晏輕淺無防慮,望開府,數呼相工自視,云當大貴。與賓客語,好屏人請間,上聞之,疑晏欲反,遂有誅晏之意。傖人鮮於文粲與晏子德元往來,密探朝旨,告晏有異志。世範等又啟上云:「晏謀因四年南郊,與世祖故舊主帥於道中竊發。」會虎犯郊壇,帝愈懼。未郊一日,敕停行。元會畢,乃召晏於華林省誅之。下詔曰:「晏閭閻凡伍,少無特操,階緣人乏,班齒官途。世祖在蕃,搜揚擢用,棄略疵瑕,遂升要重。而
 輕跳險銳,在貴彌著,猜忌反覆,觸情多端。故以兩宮所弗容,十手所共指。既內愧於心,外懼憲牘,掩跡陳痾,多歷年載。頻授蕃任,輒辭請不行,事似謙虛,情實詭伏。隆昌以來,運集艱難,匡贊之功,頗有心力。乃爵冠通侯,位登元輔,綢繆恩寄,朝莫均焉。谿壑可盈,無厭將及。視天畫地,遂懷異圖。廣求卜相,取信巫覡。論薦黨附,遍滿臺府。令大息德元淵藪亡命,同惡相濟,劍客成群。弟詡凶愚,遠相唇齒,信驛往來,密通要契。去歲之初,奉朝請鮮於文粲備告奸謀。朕以信必由中,義無與貳,推誠委任,覬能悛改。而長惡易流,構扇彌大,與北中郎司馬蕭毅、臺隊主劉明達等剋期竊發。以河東王鉉識用微弱,可為其主,得志之日,當守以虛器。



 明達諸辭列,炳然具存。昔漢後以反唇致討,魏臣以虯須為戮,況無君之心既彰,陵上之跡斯著!此而可容,誰寘刑辟!並可收付廷尉,肅明國典。」



 晏未敗數
 日,於北山廟答賽,夜還,晏既醉,部伍人亦飲酒。羽儀錯亂,前後十餘里中,不復相禁制。識者云「此勢不復久也」。



 晏子德元,有意尚。至車騎長史。德元初名湛,世祖謂晏曰:「劉湛、江湛,並不善終,此非佳名也。」晏乃改之。至是與弟晉安王友德和俱被誅。



 晏弟詡,永明中為少府卿。六年,敕位未登黃門郎,不得畜女妓。詡與射聲校尉陰玄智坐畜妓免官,禁錮十年。敕特原詡禁錮。後出為輔國將軍、始興內史。廣州刺史劉纘為奴所殺,詡率郡兵討之。延興元年,授詡持節廣州刺史。詡亦篤舊。



 晏誅,上又遣南中郎司馬蕭季敞襲詡殺之。



 蕭諶,字彥孚,南蘭陵蘭陵人也。祖道清,員外郎。父仙伯,桂陽王參軍。諶初為州從事,晉熙國侍郎,左常侍。諶於太祖為絕服族子,元徽末,世祖在郢州,欲知京邑消息,太祖遣諶就世祖宣傳謀計,留
 為腹心。升明中,為世祖中軍刑獄參軍,東莞太守。以勳勤封安復縣男,三百戶。建元初,為武陵王冠軍、臨川王前軍參軍,除尚書都官郎,建威將軍,臨川王鎮西中兵。



 世祖在東宮,諶領宿衛。太祖殺張景真,世祖令諶口啟乞景真命,太祖不悅,諶懼而退。世祖即位,出諶為大末令,未之縣,除步兵校尉,領射陽令,轉帶南濮陽太守,領御仗主。永明二年,為南蘭陵太守,建威將軍如故。復除步兵校尉,太守如故。世祖齋內兵仗悉付之,心膂密事,皆使參掌。除正員郎,轉左中郎將,後軍將軍,太守如故。世祖臥疾延昌殿,敕諶在左右宿直。上崩,遺敕諶領殿內事如舊。



 鬱林即位,深委信諶,諶每請急出宿,帝通夕不得寐,諶還乃安。轉衛軍司馬,兼衛尉,加輔國將軍。丁母憂,敕還復本任,守衛尉。高宗輔政,有所匡諫,帝既在後宮不出,唯遣諶及蕭坦之遙進,乃得聞達。諶回附高宗,勸行廢
 立,密召諸王典簽約語之,不許諸王外接人物。諶親要日久,眾皆憚而從之。鬱林被廢日,初聞外有變,猶密為手敕呼諶,其見信如此。諶性險進無計略,及廢帝日,領兵先入後宮,齋內仗身素隸服諶,莫有動者。



 海陵立,轉中領軍,進爵為公,二千戶。甲仗五十人。入直殿內,月十日還府。



 建武元年,轉領軍將軍,左將軍,南徐州刺史,給扶,進爵衡陽郡公,食邑三千戶。高宗初許事克用諶為揚州,及有此授,諶恚曰:「見炊飯熟,推以與人。」



 王晏聞之曰:「誰復為蕭諶作甌箸者。」諶恃勳重,干豫朝政,諸有選用,輒命議尚書使為申論。上新即位,遣左右要人於外聽察,具知諶言,深相疑阻。



 二年六月,上幸華林園,宴諶及尚書令王晏等數人盡歡。坐罷,留諶晚出,至華林閣,仗身執還入省,上遣左右莫智明數諶曰:「隆昌之際,非卿無有今日。今一門二州,兄弟三封,朝廷相報,政可極此。卿恆懷怨望,乃云炊
 飯已熟,合甑與人邪?今賜卿死。」諶謂智明曰:「天去人亦復不遠,我與至尊殺高、武諸王,是君傳語來去。我今死,還取卿。」於省殺之。至秋而智明死,見諶為祟。詔曰:「蕭諶擢自凡庸,識用輕險,因藉倖會,早預驅馳。永明之季,曲頒恩紀。鬱林昏悖,頗立誠效。寵靈優渥,期遇兼隆,內總戎柄,外暢蕃威,兄弟榮貴,震灼朝野。



 曾不感佩殊荷,少答萬一,自以勛高伊、霍,事均難賞,才冠當時,恥居物後。矯制王權,與奪由己。空懷疑懼,坐構嫌猜。覘候宮掖,希覬非望。蔽上罔下之心,誣君不臣之跡,固以彰暴民聽,喧聒遐邇。遂潛散金帛,招集不逞,交結禁衛,互為唇齒,密契戚邸,將肆姦逆。朕以其任寄既重,爵列河山,每加彌縫,弘以大信,庶能懷音,翻然悛改。而豺狼其性,凶謀滋甚。夫無將必戮,《陽秋》明義,況釁積禍盈,若斯之大。可收付廷尉,速正刑書。罪止元惡,餘無所問。」



 諶好左道,吳興沈文猷相諶
 云:「相不減高帝。」諶喜曰:「感卿意,無為人言也。」至是文猷伏誅。



 諶兄誕,字彥偉,初為殿中將軍。永明中為建康令,與秣陵令司馬迪之同乘行,車前導四卒,左丞沈昭略奏:「凡有鹵簿官,共乘不得兼列騶寺。請免誕等官。」



 詔贖論。延興元年,自輔國徐州為持節督司州刺史,將軍如故。明帝立,封安德侯,五百戶。進號冠軍。建武二年春,虜攻司州,誕盡力拒守,虜退,增封四百戶。徵左衛將軍。上欲殺諶,以誕在邊鎮拒虜,故未及行。虜退六旬,諶誅,遣黃門郎梁王為司州別駕,使誅誕,束身受戮,家口繫尚方。



 諶弟誄,與諶同豫廢立,為寧朔將軍、東莞太守,轉西中郎司馬。建武初,封西昌侯,千戶。轉太子左率。領軍解司州圍還,同伏誅。



 諶伯父仙民,官至太中大夫,卒。



 蕭坦之,南蘭陵蘭陵人也。祖道濟,太中大夫。父欣祖,有勳於世祖,至武進令。坦之與蕭諶同族。初為殿中將軍,累至世祖中軍板刑
 獄參軍。以宗族見驅使。



 除竟陵王鎮北征北參軍,東宮直閣,以勤直為世祖所知。除給事中,淮陵令,又除蘭陵令,給事中如故。尚書起部郎,司徒中兵參軍。世祖崩,坦之隨太孫文武度上臺,除射聲校尉,令如故。未拜,除正員郎、南魯郡太守。



 少帝以坦之世祖舊人,親信不離,得入內見皇后。帝於宮中及出後堂雜戲狡獪,坦之皆得在側。或值醉後裸袒,坦之輒扶持諫喻。見帝不可奉,乃改計附高宗,密為耳目。除晉安王征北諮議。隆昌元年,追錄坦之父勛,封臨汝縣男,食邑三百戶。



 徙征南諮議。



 高宗謀廢少帝,既與蕭諶及坦之定謀。帝腹心直閣將軍曹道剛疑外間有異,密有處分,諶未能發。始興內史蕭季敞、南陽太守蕭穎基遷都尉並應還都,諶欲待二蕭至,藉其勢力以舉事。高宗慮事變,以告坦之,坦之馳謂諶曰:「廢天子古來大事。比聞曹道剛、朱隆之等轉已猜疑。衛尉明日若不就事,
 無所復及。弟有百歲母,豈能坐聽禍敗,政應作餘計耳!」諶遑遽,明日遂廢帝,坦之力也。



 海陵即位,除黃門郎、兼衛尉卿,進爵伯,增邑為六百戶。建武元年,遷散騎常侍,右衛將軍,進爵侯,增邑為千五百戶。明年,虜動,假坦之節,督徐州征討軍事。虜圍鐘離,春斷淮洲,坦之擊破之。還加領太子中庶子,未拜,遷領軍將軍。



 永泰元年,為侍中、領軍。



 東昏立,為侍中、領軍將軍。永元元年,遭母喪,起復職,加右將軍,置府。



 江祏兄弟欲立始安王遙光,密謂坦之,坦之曰:「明帝取天下,已非次第,天下人至今不服。今若復作此事,恐四海瓦解。我其不敢言。」持喪還宅。宅在東府城東,遙光起事,遣人夜掩取坦之,坦之科頭著褌踰墻走,從東冶僦渡南渡,間道還臺,假節督眾軍討遙光,屯湘宮寺。事平,遷尚書右僕射,丹陽尹,右將軍如故。進爵公,增邑千戶。



 坦之肥黑無須,語聲嘶,時人號為「蕭啞」。剛狠專執,群
 小畏而憎之。遙光事平二十餘日,帝遣延明主帥黃文濟領兵圍坦之宅,殺之。子賞,秘書郎,亦伏誅。



 坦之從兄翼宗為海陵郡,將發。坦之謂文濟曰:「從兄海陵宅故應無他?」文濟曰:「海陵宅在何處?」坦之告之,文濟曰:「應得罪。」仍遣收之。檢家赤貧,唯有質錢貼子數百,還以啟帝,原死,繫尚方。



 和帝中興元年,追贈坦之中軍將軍、開府儀同三司。



 江祏,字弘業,濟陽考城人也。祖遵,寧朔參軍。父德鄰,司徒右長史。祏姑為景皇后,少為高宗所親,恩如兄弟。宋末解褐晉熙國常侍,太祖徐州西曹,員外郎,高宗冠軍參軍,帶灄陽令,竟陵王征北參軍,尚書水部郎。高宗為吳興,以祏為郡丞,加宣威將軍。廬陵王中軍功曹記室,安陸王左軍諮議,領錄事,帶京兆太守。除通直郎,補南徐州別駕。高宗輔政,委以心腹。隆昌元年,自正員郎補丹陽丞,
 中書郎。高宗為驃騎,鎮東府,以祏為諮議參軍,領南平昌太守,與蕭誄對直東府省內。



 時新立海陵,人情未服,高宗胛上有赤志,常祕不傳,祏勸帝出以示人。晉壽太守王洪範罷任還,上袒示之,曰:「人皆謂此是日月相。卿幸無泄言。」洪範曰:「公日月之相在軀,如何可隱。轉當言之公卿。」上大悅。會直後張伯、尹瓚等屢謀竊發,祏、誄憂虞無計,每夕輒托事外出。及入纂議定,加祏寧朔將軍。高宗為宣城王,太史密奏圖緯云「一號當得十四年」。祏入,帝喜以示祏曰:「得此復何所望。」及即位,遷守衛尉,將軍如故。封安陸縣侯,邑千戶。祏祖遵,以后父贈金紫光祿大夫;父德鄰,以帝舅亦贈光祿大夫。



 建武二年,遷右衛將軍,掌甲仗廉察。四年,轉太子詹事。祏以外戚親要,勢冠當時,遠致餉遺,或取諸王第名書好物。然家行甚睦,待子侄有恩意。



 上寢疾,永泰元年,轉祏為侍中、中書令,出入殿省。上崩,遺詔
 轉右僕射,祏弟衛尉祀為侍中,敬皇后弟劉暄為衛尉。東昏即位,參掌選事。高宗雖顧命群公,而意寄多在祏兄弟。至是更直殿內,動止關諮。永元元年,領太子詹事。劉暄遷散騎常侍,右衛將軍。祏兄弟與暄及始安王遙光、尚書令徐孝嗣、領軍蕭坦之六人,更日帖敕,時呼為「六貴」。



 帝稍欲行意,孝嗣不能奪,坦之雖時有異同,而祏堅意執制,帝深忿之。帝失德既彰,祏議欲立江夏王寶玄。劉暄初為寶玄郢州行事,執事過刻。有人獻馬,寶玄欲看之,暄曰:「馬何用看。」妃索煮肫,帳下諮暄,暄曰:「旦已煮鵝,不煩復此。」寶玄恚曰:「舅殊無《渭陽》之情。」暄聞之亦不悅。至是不同祏議,欲立建安王寶夤,密謀於遙光。遙光自以年長,屬當鼎命,微旨勸祏。祏弟祀以少主難保,勸祏立遙光。暄以遙光若立,己失元舅之望,不肯同。故祏遲疑久不決。遙光大怒,遣左右黃曇慶於清溪橋道中刺殺暄,曇慶
 見暄部伍人多,不敢發。事覺,暄告祏謀,帝處分收祏兄弟。祀時直在內殿,疑有異,遣信報祏曰:「劉暄似有異謀,今作何計?」祏曰:「政當靜以鎮之耳。」俄而召祏入見,停中書省。初,直齋袁文曠以王敬則勳當封,祏執不與。帝使文曠取祏,以刀環築其心曰:「復能奪我封否?」祏、祀同日見殺。



 祀字景昌,初為南郡王國常侍,歷高祖驃騎東閣祭酒,祕書丞,晉安王鎮北長史,南東海太守,行府、州事。治下有宣尼廟,久廢不修,祀更開掃構立。



 祀弟禧,居喪早卒。有子廞,字偉卿,年十二,聞收至,謂家人曰:「伯既如此,無心獨存。」赴井死。



 後帝於後堂騎馬致適,顧謂左右曰:「江祏若在,我當復能騎此不?」



 暄字士穆,出身南陽國常侍。遙光起事,以討暄為名。事平,暄遷領軍將軍,封平都縣侯,千戶。其年,又見殺。和帝中興元年,贈祏衛將軍,暄散騎常侍、撫軍將軍,並開府儀同三司,祀散騎常侍、太常卿。



 史臣曰:士死知己,蓋有生所共情,雖愚智之品有二,而逢迎之運唯一。夫懷可知之才,受知人之眄,無慚外物,此固天理,其猶藏在中心,銜恩念報。況乎義早蕃僚,道同遇合,逾越勝己,顧邁先流,棄子如遺,曾微舊德。使狗之喻,人致前譏,慚包疚心,我無其事。嗚呼!陸機所以賦《豪士》也。



 贊曰:王蕭提契,世祖基之。樂羊食子,里克無辭。
 江、劉後戚,明嗣是維。



 廢興異論,終用乖疑。



\end{pinyinscope}