\article{卷七本紀第七東昏侯}

\begin{pinyinscope}

 東昏侯寶卷,字智藏,高宗第二子也。本名明賢,高宗輔政後改焉。建武元年,立為皇太子。永泰元年七月,己酉,高宗崩,太子即位。八月,丁巳,詔雍州將士與虜戰死者,復除有差。又詔辨括選序,訪搜貧屈。



 庚申,鎮北將軍晉安王寶義進號征北大將軍、開府儀同三司。南中郎將建安王寶夤為郢州刺史。冬,十月,己未,詔刪省科律。十一月,戊子,立皇后褚氏,賜王公以下錢各有差。



 永元元年春,正月,戊寅,大赦,改元。詔研策秀才,考課百司。辛卯,車駕祀南郊。詔三品清資官以上應食祿者,有二親或祖父母年登七十,並給見錢。癸卯,以冠軍將軍南康王寶融為荊州刺史。



 二月,癸
 丑,以北中郎將邵陵王寶攸為南兗州刺史。是月,太尉陳顯達敗績於馬圈。夏,四月,己巳,立皇太子誦,大赦,賜民為父後爵一級。甲戌,以寧朔將軍柳惔為梁、南秦二州刺史。五月癸亥,以撫軍大將軍始安王遙光為開府儀同三司。六月,己酉,新除右衛將軍崔惠景為護軍將軍。癸亥,以始興內史范雲為廣州刺史。甲子,詔原雍州今年三調。秋,七月,丁亥,京師大水,死者眾,詔賜死者材器,并賑恤。八月,乙巳,蠲京邑遇水資財漂蕩者今年調稅。又詔為馬圈戰亡將士舉哀。丙午,揚州刺史始安王遙光據東府反。詔曲赦京邑,中外戒嚴。尚書令徐孝嗣以下屯衛宮城。遣領軍將軍蕭坦之率六軍討之。戊午,斬遙光,傳首。己未,以征北大將軍晉安王寶玄為南徐、兗二州刺史。己巳,尚書令徐孝嗣為司空,右衛將軍劉暄為領軍將軍。閏月,丙子,以江陵公寶覽為始安王。虜偽東徐州刺
 史沈陵降,以為北徐州刺史。九月,丁未,以輔國將軍裴叔業為兗州刺史,征虜長史張沖為豫州刺史。壬戌,以頻誅大臣,大赦天下。辛未,以太子詹事王瑩為中領軍。冬,十月,乙未,誅尚書令、新除司空徐孝嗣,右僕射、新除鎮軍將軍沈文季。乙巳,以始興內史顏翻為廣州刺史,征虜將軍沈陵為越州刺史。十一月,丙辰,太尉、江州刺史陳顯達舉兵於尋陽。乙丑,護軍將軍崔慧景加平南將軍、督眾軍南討事。丙寅,以冠軍將軍王鴻為徐州刺史。十二月,癸未,以前輔國將軍楊集始為秦州刺史。甲申,陳顯達至京師,宮城嚴警,六軍固守。乙酉,斬陳顯達,傳首。丁亥,以征虜將軍邵陵王寶攸為江州刺史。



 二年春,正月,壬子,以輔國將軍張沖為南兗州刺史。庚午,詔討豫州刺史裴叔業。二月,癸未,以黃門郎蕭寅為司州刺史。丙戌,以衛
 尉蕭懿為豫州刺史,征壽春。己丑,裴叔業病死,兄子植以壽春降虜。三月,癸卯,以輔國將軍張沖為司州刺史。乙卯,遣平西將軍崔慧景率眾軍伐壽春。丁未,以新除冠軍將軍張沖為南兗州刺史。崔慧景於廣陵舉兵襲京師。壬子,右衛將軍左興盛督京邑水步眾軍。南徐州刺史江夏王寶玄以京城納慧景。乙卯,遣中領軍王瑩率眾軍屯北籬門。壬戌,慧景至,瑩等敗績。甲子,慧景入京師,宮內據城拒守。豫州刺史蕭懿起義救援。夏四月,癸酉,慧景棄眾走,斬首。詔曲赦京邑、南徐兗二州。乙亥,以新除尚書右僕射蕭懿為尚書令。丙子,以晉熙王寶嵩為南徐州刺史。五月,乙巳,以虜偽豫州刺史王肅為豫州刺史。戊申,以桂陽王寶貞為中護軍。己酉,江夏王寶玄伏誅。壬子,大赦。乙丑,曲赦京邑、南徐兗二州。



 戊辰,以始安王寶覽為湘州刺史。六月,庚寅,車駕於樂游苑內會,如三元,京
 邑女人放觀。戊戌,以新除冠軍將軍張沖為郢州刺史,守五兵尚書陸慧曉為南兗州刺史。秋,七月,甲辰,以驃騎司馬張稷為北徐州刺史。八月,丁酉,以新除驃騎司馬陳伯之為豫州刺史。甲申夜,宮內火。冬,十月,己卯,害尚書令蕭懿。十一月,辛丑,以寧朔將軍張稷為南兗州刺史。甲寅,西中郎長史蕭穎胄起義兵於荊州。十二月,雍州刺史梁王起義兵於襄陽。戊寅,以冠軍長史劉繪為雍州刺史。



 三年春,正月,丙申朔,合朔時加寅漏上八刻,事畢,宮人於閱武堂元會,皇后正位,閹人行儀,帝戎服臨視。丁酉,以驃騎大將軍晉安王寶義為司徒,新除撫軍將軍建安王寶夤為車騎將軍、開府儀同三司。甲辰,以寧朔將軍王珍國為北徐州刺史。辛亥,車駕祀南郊,詔大赦天下,百官陳讜言。二月,丙寅,乾和殿西廂火。壬午,詔遣
 羽林兵征雍州,中外纂嚴。乙酉,以武烈將軍胡元進為廣州刺史。三月,己亥,以驃騎將軍沈徽孚為廣州刺史。甲辰,以輔國將軍張欣泰為雍州刺史。丁未,南康王寶融即皇帝位於江陵。癸丑,遣平西將軍陳伯之西征。六月,京邑雨水,遣中書舍人、二縣官長賑賜有差。蕭穎胄弟穎孚起兵廬陵。戊子,曲赦江州安成、廬陵二郡。秋,七月,癸巳,曲赦荊、雍二州。甲午,雍州刺史張欣泰、前南譙太守王靈秀率石頭文武奉建安王寶夤向臺,至杜姥宅,宮門閉,乃散走。己未,以征虜長史程茂為郢州刺史,驍騎將軍薛元嗣為雍州刺史。是日,元嗣以郢城降義師。八月,丁卯,以輔國將軍申胄監豫州事。辛巳,光祿大夫張瑰鎮石頭。辛未,以太子左率李居士總督西討諸軍事,屯新亭城。九月,甲辰,以居士為江州刺史,新除冠軍將軍王珍國為雍州刺史,車騎將軍建安王寶寅為荊州刺史。以輔國
 將軍申胄監郢州,龍驤將軍馬仙琕監豫州,驍騎將軍徐元稱監徐州。是日,義軍至南州,申胄軍二萬人於姑熟奔歸。戊申,以後軍參軍蕭璝為司州刺史,前輔國將軍魯休烈為益州刺史,輔國長史趙越嘗為梁、南秦二州刺史。丙辰,李居士與義軍戰於新亭,敗績。冬,十月,甲戌,王珍國與義軍戰於朱雀桁,敗績。戊寅,寧朔將軍徐元瑜以東府城降。青、冀二州刺史桓和入衛,屯東宮,己卯,以眾降。光祿大夫張瑰棄石頭還宮。於是閉宮城門自守。庚辰,以驍騎將軍胡虎牙為徐州刺史,左軍將軍徐智勇為益州刺史,游擊將軍牛平為梁、南秦二州刺史。李居士以新亭降,琅邪城主張木亦降。義師築長圍守宮城。



 十二月,丙寅,新除雍州刺史王珍國、侍中張稷率兵入殿廢帝,時年十九。



 帝在東宮便好弄,不喜書學,高宗亦不以為非,但勖以家人之行。令太子求一日再入朝,發詔不許,使三日
 一朝。嘗夜捕鼠達旦,以為笑樂。高宗臨崩,屬以後事,以隆昌為戒,曰:「作事不可在人後!」故委任群小,誅諸宰臣,無不如意。



 性重澀少言,不與朝士接,唯親信閹人及左右御刀應敕等,自江祏、始安王遙光誅後,漸便騎馬。日夜於後堂戲馬,與親近閹人倡伎鼓叫。常以五更就臥,至晡乃起。王侯節朔朝見,晡後方前,或際暗遣出。臺閣案奏,月數十日乃報,或不知所在。二年元會,食後方出,朝賀裁竟,便還殿西序寢。自巳至申,百僚陪位,皆僵仆菜色。比起就會,匆遽而罷。



 陳顯達事平,漸出遊走,所經道路,屏逐居民,從萬春門由東宮以東至於郊外,數十百里,皆空家盡室。巷陌懸幔為高障,置仗人防守,謂之「屏除」。或於市肆左側過親幸家,環回宛轉,周遍京邑。每三四更中,鼓聲四出,幡戟橫路,百姓喧走相隨,士庶莫辨。出輒不言定所,東西南北,無處不驅人。高障之內,設部伍羽儀。復
 有數部,皆奏鼓吹羌胡伎,鼓角橫吹。夜出晝反,火光照天。拜愛姬潘氏為貴妃,乘臥輿,帝騎馬從後。著織成褲褶,金薄帽,執七寶縛槊,戎服急裝,不變寒暑,陵冒雨雪,不避坑阱。馳騁渴乏,輒下馬解取腰邊蠡器酌水飲之,復上馬馳去。馬乘具用錦繡處,患為雨所沾濕,織雜彩珠為覆蒙,備諸雕巧。教黃門五六十人為騎客,又選無賴小人善走者為逐馬,左右五百人,常以自隨,奔走往來,略不暇息。置射雉場二百九十六處,翳中帷帳及步鄣,皆袷以綠紅錦,金銀鏤弩牙,瑇瑁帖箭。郊郭四民皆廢業,樵蘇路斷,吉凶失時;乳婦婚姻之家,移產寄室,或輿病棄尸,不得殯葬。有棄病人於青溪邊者,吏懼為監司所問,推置水中,泥覆其面,須臾便死,遂失骸骨。



 後宮遭火之後,更起仙華、神仙、玉壽諸殿,刻畫雕彩,青灊金口帶,麝香塗壁,錦幔珠簾,窮極綺麗。縶役工匠,自夜達曉,猶不副速,
 乃剔取諸寺佛剎殿藻井仙人騎獸以充足之。世祖興光樓上施青漆,世謂之「青樓」。帝曰:「武帝不巧,何不純用琉璃。」



 潘氏服御,極選珍寶。主衣庫舊物,不復周用,貴市民間金銀寶物,價皆數倍。



 虎魄釧一隻,直百七十萬。京邑酒租,皆折使輸金,以為金塗。猶不能足,下揚、南徐二州橋桁塘埭丁計功為直,斂取見錢,供太樂主衣雜費。由是所在塘瀆,多有隳廢。又訂出雉頭鶴氅白鷺縗。親幸小人因緣為奸利,課一輸十,郡縣無敢言者。



 三年夏,於閱武堂起芳樂苑。山石皆塗以五采;跨池水立紫閣諸樓觀,壁上畫男女私褻之像。種好樹美竹,天時盛暑,未及經日,便就萎枯;於是徵求民家,望樹便取,毀撤墻屋以移致之。朝栽暮拔,道路相繼,花藥雜草,亦復皆然。又於苑中立市,太官每旦進酒肉雜肴,使宮人屠酤。潘氏為市令,帝為市魁,執罰,爭者就潘氏決判。



 帝有膂力,能擔白虎幢。自
 製雜色錦伎衣,綴以金花玉鏡眾寶,逞諸意態。所寵群小黨與三十一人,黃門十人。初任新蔡人徐世檦為直閣驍騎將軍,凡有殺戮,皆其用命。殺徐孝嗣後,封為臨汝縣子。陳顯達事起,加輔國將軍。雖用護軍崔慧景為都督,而兵權實在世檦。及事平,世檦謂人曰:「五百人軍主,能平萬人都督。」



 世檦亦知帝昏縱,密謂其黨茹法珍、梅蟲兒曰:「何世天子無要人,但阿儂貨主惡耳。」法珍等爭權,以白帝。帝稍惡其兇強,以二年正月,遣禁兵殺之,世檦拒戰而死。自是法珍、蟲兒用事,並為外監,口稱詔敕;中書舍人王咺之與相唇齒,專掌文翰。其餘二十餘人,皆有勢力。崔慧景平後,法珍封餘干縣男,蟲兒封竟陵縣男。及義師起,江、郢二鎮已降,帝遊騁如舊,謂茹法珍曰:「須來至白門前,當一決。」義師至近郊,乃聚兵為固守之計。召王侯朝貴分置尚書都座及殿省。又信鬼神,崔慧景事時,拜
 蔣子文神為假黃皞、使持節、相國、太宰、大將軍、錄尚書、揚州牧、鐘山王。至是又尊為皇帝,迎神像及諸廟雜神皆入後堂,使所親巫朱光尚禱祀祈福。以冠軍將軍王珍國領三萬人據大桁,莫有鬥志,遣左右直長閹豎王寶孫督戰,呼為「王長子」。寶孫切罵諸將帥,直閣將軍席豪發憤突陣死。豪,驍將,既斃,眾軍於是土崩,軍人從朱雀觀上自投及赴淮死者無數。於是閉城自守,城內軍事委王珍國。兗州刺史張稷入衛京師。以稷為副,實甲猶七萬人。帝烏帽褲褶,備羽儀,登南掖門臨望。又虛設鎧馬齋仗千人,皆張弓拔白,出東掖門,稱蔣王出盪。素好鬥軍隊,初使宮人為軍,後乃用黃門。親自臨陳,詐被創,使人輿將去。



 至是於閱武堂設牙門軍頓,每夜嚴警。帝於殿內騎馬從鳳莊門入徽明門,馬被銀蓮葉具裝鎧,雜羽孔翠寄生,逐馬左右衛從,晝眠夜起如平常。聞外鼓叫聲,被大紅
 袍登景陽樓屋上望,弩幾中之。眾皆怠怨,不為致力。募兵出戰,出城門數十步,皆坐甲而歸。慮城外有伏兵,乃燒城傍諸府署,六門之內皆盪盡。城中閣道西掖門內,相聚為市,販死牛馬肉。帝初與群小計議,陳顯達一戰便敗,崔慧景圍城退走,謂義師遠來,不過旬日,亦應散去,敕太官辦樵米為百日糧而已。大桁敗後,眾情兇懼,法珍等恐人眾驚走,故閉城不復出軍。既而義師長圍既立,塹柵嚴固;然後出盪,屢戰不捷。帝尤惜金錢,不肯賞賜。法珍叩頭請之,帝曰:「賊來獨取我邪?



 何為就我求物!」後堂儲數百具榜,啟為城防;帝雲擬作殿,竟不與。又催御府細作三百人精仗,待圍解以擬屏除。金銀雕鏤雜物,倍急於常。王珍國、張稷懼禍及,率兵入殿,分軍又從西上閣入後宮斷之,御刀豐勇之為內應。是夜,帝在含德殿吹笙歌作《女兒子》。臥未熟,聞兵入,趨出北戶,欲還後宮。清曜閣
 已閉,閹人禁防黃泰平以刀傷其膝,仆地。顧曰:「奴反邪?」直後張齊斬首送梁王。



 宣德太后令曰:「皇室受終,祖宗齊聖,太祖高皇帝肇基駿命,膺錄受圖,世祖武皇帝系明下武,高宗明皇帝重隆景業,咸降年不永,宮車早晏。皇祚之重,允屬儲元;而稟質凶愚,發于稚齒。爰自保姆,迄至成童,忍戾昏頑,觸途必著。高宗留心正嫡,立嫡惟長,輔以群才,間以賢戚,內外維持,冀免多難,未及期稔,便逞屠戮。密戚近親,元勛良輔,覆族殲門,旬月相係。凡所任仗,盡慝窮奸,皆營伍屠販,容狀險醜,身秉朝權,手斷國命,誅戮無辜,納其財產,睚眥之間,屠覆比屋。身居元首,好是賤事,危冠短服,坐臥以之。晨出夜反,無復已極,驅斥氓庶,巷無居人。老細奔遑,置身無所。東邁西屏,北出南驅,負疾輿尸,填街塞陌。興築繕造,日夜不窮,晨構夕毀,朝穿暮塞。絡以隨珠,方斯巳陋;飾以璧榼,曾何足道!時暑赫曦,
 流金鑠石,移竹藝果,匪日伊夜,根未及植,葉已先枯,畚鍤紛紜,勤倦無已。散費國儲,專事浮飾,逼奪民財,自近及遠,兆庶恇患,流竄道路。府帑既竭,肆奪市道,工商裨販,行號道泣。屈此萬乘,躬事角抵,昂首翹肩,逞能橦木,觀者如堵,曾無怍容!芳樂、華林,並立闤闠,踞肆鼓刀,手銓輕重。干戈鼓噪,昏曉靡息,無戎而城,豈足云譬!至於居喪淫宴之愆,三年載弄之醜,反道違常之釁,牝雞晨鳴之慝,於事已細,故可得而略也。罄楚、越之竹,未足以言,校辛、癸之君,豈或能匹!征東將軍忠武奮發,投袂萬里,光奉明聖,翊成中興。乘勝席卷,掃清京邑,而群小靡識,嬰城自固,緩戮稽誅,倏彌旬月。宜速剿定,寧我邦家!可潛遣間介,密宣此旨,忠勇齊奮,遄加蕩撲,放斥昏兇,衛送外第。未亡人不幸,驟此百罹,感念存沒,心焉如割。奈何!奈何!」又令依漢海昏侯故事,追封東昏侯。茹法珍、梅蟲兒、王咺之
 等伏誅。豐勇之原死。



 史臣曰:漢宣帝時,南郡獲白虎,獲之者張武,言武張而猛服也。東昏侯亡德橫流,道歸拯亂,躬當翦戮,實啟太平。推閹豎之名字,亦天意也。



 贊曰:東昏慢道,匹癸方辛。乃隳典則,乃棄彞倫,玩習兵火,終用焚身。



\end{pinyinscope}