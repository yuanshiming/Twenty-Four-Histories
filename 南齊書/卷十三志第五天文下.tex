\article{卷十三志第五天文下}

\begin{pinyinscope}

 史臣曰:天文設象,宜
 備內外兩宮。但災之所躔,不必遍行景緯,五星精晷與二曜而為七,妖祥是主,曆數攸司,蓋有殊於列宿也。若北辰不移,據在杠軸,眾星動流,實繫天體,五星從伏,非關二義,故徐顯思以五星為非星,虞喜論之詳矣。



 五星相犯列宿雜災建元元年八月辛亥,太白犯軒轅大星。



 九月癸丑,太白從行於軫犯填星。



 二年六月丙子,太白晝見。



 四年二月丙戌,太白晝見在午上。



 六月辛卯,太白晝見午上。庚子,太白入東井,無所犯。七月己未,太白有光影。八月戊子,太白從軒轅犯女主星。甲辰,太白從行犯軒轅少民星。九月己卯,太白從行犯太微西蕃上將。辛酉,太白從行入太微,在右執法星西北一尺。戊辰,太白從行犯太微左執法。



 十二月壬子,太白從行犯填星,在氐度。丙辰,太白從行犯房北頭第一星。丁卯,太白犯楗閉星。



 永明元年六月己酉,太白行犯太微上將星。辛酉,太白行犯太微左執法。



 八月甲申,太白犯南斗第四星。九月乙酉,太白犯南斗第三星。壬辰,太白、熒惑合同在南斗度。十月丁卯,太白犯哭星。



 二年正月戊戌,太白晝見當午上。



 三月甲戌,太白從行入羽林。四月丙申,太白從行犯東井鉞星。六月戊辰,太白、熒惑合同在輿鬼度。己巳,太白從行輿鬼度犯歲星。



 三年四月丁未,太白晝見。癸亥,太白晝見當午上。



 五月戊子,太白犯少民星。八月丁巳,太白晝見當午上。



 十一月壬申,太白從行入氐。十二月己酉,太白填星合在箕度。



 四年九月壬辰,太白晝見當午。丙午,太白犯南斗。



 十一月庚子,太白入羽林,又犯天關。



 五年五月丁酉,太白晝見當午上。庚子,太白三犯畢左股第一星西南一尺。



 六月甲戌,太白犯東井北轅第三星,在西一尺。八月甲寅,太白從行入軒轅,在女主星東北一尺二寸,不為犯。戊辰,太白從在太微西蕃上將星西南五寸。辛巳,太白從在太微左執法星西北四寸。



 六年四月辛酉,太白從在熒惑北三寸,為犯,並在東井度。



 五月癸卯,太白晝見當午上。六月己巳,太白從在太微西蕃右執法星東南四寸,為犯。七月癸巳,太白在氐角星東北一尺,為犯。八月乙亥,太白從行在房南第二左股次將星西南一尺,為犯。閏八月甲午,太白晝見當
 午。



 十一月戊午,太白從在歲星西北四尺,同在尾度。又在熒惑東北六尺五寸,在心度,合宿。十二月壬寅,太白從行在填星西南二尺五寸斗度。



 七年二月辛巳,太白從行入羽林。



 十月癸酉,太白在歲星南,相去一尺六寸,從在箕度為合。



 十一月丁卯,太白從行入午林。



 八年正月丁未,太白晝見當午上。



 六月戊子,太白從行入東井。己丑,太白晝見當午。八月庚辰,太白從在軒轅女主星南七尺,為犯。九月丙申,太白從行在太微西蕃上將星西南一尺,為犯。丁未,太白從行入太微。辛酉,太白從行在進賢西五寸,為犯。十月乙亥,太白從行在亢南第二星西南一尺,為犯。甲申,太白從行入氐。



 十一
 月戊戌,太白從行在房北頭第二星東北一寸,又在楗閉星西南七寸,並為犯。又在熒惑西北二尺,為合宿。癸卯,太白從行在熒惑東北一尺,為犯。



 九年四月癸未,太白從歷,夕見西方,從疾參宿一度。比來多陰,至己丑開除,已見在日北,當西北維上,薄昏不見宿星,則為先歷而見。



 六月丙子,太白晝見當午上。七月辛卯,太白從行入太微,在西蕃上將星北四寸,為犯。九月乙亥,太白從行在南斗第四星北二寸,為犯。丁卯,太白在南斗第三星西一寸,為犯。



 十年二月甲辰,太白從行入羽林。



 五月辛巳,太白從行入東井,在軒轅西第一星東六寸,為犯。
 七月乙丑,太白從行在軒轅大星東八寸,為犯。



 十一年正月戊辰,太白從行在歲星西北六寸,為犯,在奎度。



 二月丁丑,太白從行東井北轅西頭第一星東北一尺,為犯。四月戊子,太白在五諸侯東第二星西北六寸,為犯。辛丑,太白從行入輿鬼,在東北星西南四寸,為犯。五月戊午,太白晝見當午,名為經天。癸亥,太白從行入軒轅大星北一尺二寸,無所犯。九月己酉,太白晝見當午上。十月丙戌,太白行在進賢星西南四寸,為犯。



 十一月戊戌,太白從行入氐。丁卯,太白從行在楗閉星西北六寸,為犯。



 十二月壬辰,太白從行在南斗第六星東南一尺,為犯。辛丑,太白
 從行在西建東星西南一尺,為犯。



 建元元年五月己未,熒惑犯太微西蕃上將,又犯東蕃上將。



 二年十月辛酉,熒惑守太微。



 四年六月戊子,熒惑從行入東井,無所犯。戊戌,熒惑在東井度,形色小而黃黑不明。丁丑,熒惑、太白同在東井度。



 七月甲戌,熒惑從行入輿鬼,犯積尸。十月癸未,熒惑從行犯太微西蕃上將星。



 丙戌,熒惑從入太微。



 十一月丙辰,熒惑從行在太微,犯右執法。



 永明元年正月己亥,熒惑逆犯上相。辛亥,熒惑守角。庚子,熒惑逆入太微。



 三月丁卯,熒惑守太白。六月戊申,熒惑從犯亢。己巳,熒惑從行犯氐東南星。



 七月戊寅,熒惑、填星同在氐度。丁亥,熒惑行犯房北頭第二星。八月乙丑,熒惑從行犯天江。甲戌,熒惑犯南斗第五星。



 十一月丙申,熒惑入羽林。



 二年八月庚午,熒惑犯太微西蕃上將。癸未,熒惑犯太微右執法。丁酉,熒惑犯太微右執法。



 十月庚申,熒惑犯進賢。



 十一月壬辰,熒惑犯亢南第二星。丙申,熒惑犯亢南星。十二月乙卯,熒惑入氐。



 三年二月乙卯,熒惑在房北頭第一星西北一尺,徘徊守房。



 四月戊戌,熒惑犯。六月乙亥,熒惑犯房。癸亥,熒惑犯天江南頭第二星。八月丁巳,熒惑犯南斗第五星。



 十一月丙戌,熒惑從行入羽林。



 四年八月戊辰,熒惑入太微。癸酉,熒惑犯太微右執法。戊子,熒惑在太微。



 九月戊申,熒惑犯歲星。己酉,熒惑犯歲星,芒角相接。十月丁丑,熒惑犯亢南頭第一星。



 十一月庚寅,熒惑犯氐西南星。十二月己未,熒惑犯房北頭第一星。庚申,熒惑入房北犯鉤鈐星。



 五年二月乙亥,熒惑、填星同在南斗度,為合宿。



 九月乙未,熒惑從行在哭星東,相去半寸。



 六年四月癸丑,熒惑伏在參度,去太白二尺五寸,辰星去太白五尺三寸,為合宿。甲戌,熒惑在辰星東南二尺五寸,俱從行,入東井曠中,無所犯。
 閏四月丁丑,熒惑從行在氐西南星北七寸,為犯。己卯,熒惑從行入氐,無所犯。乙巳,熒惑從行在房北頭第一上將右驂星南六寸,為犯。又在鉤鈐星西北五寸。



 十一月丙寅,熒惑從行在歲星西,相去四尺,同在尾度,為合宿。



 七年二月丙子,熒惑從行在填星西,相去二尺,同在牽牛度,為合宿。



 三月戊午,熒惑從在泣星西北七寸。戊辰,熒惑從行入羽林。八月戊戌,熒惑逆入羽林。九月乙丑,熒惑入羽林,成句己。



 八年四月丙申,熒惑從行入輿鬼,在西北星東南二寸,為犯。



 十月乙亥,熒惑入氐。



 十一月乙未,熒惑從入北落門,在第一星東南,去鉤鈐三
 寸,為犯。



 九年三月甲午,熒惑從在填星東七寸,在歲星南六寸,同在虛度,為犯,為合宿。



 四月癸亥,熒惑從行入羽林。閏七月辛酉,熒惑從行在畢左股星西北一寸,為犯。八月十四日,熒惑應伏在昴三度,前先歷在畢度,二十一日始逆行北轉,垂及玄冬熒惑囚死之時,而形色漸大於常。



 十年二月庚子,熒惑從入東井北轅西頭第一星西二寸,為犯。



 三月癸未,熒惑從行在輿鬼西北七寸,為犯。乙酉,熒惑從行入輿鬼。六月壬寅,熒惑從行入太微。



 十一年二月庚戌,熒惑從在填星西北六寸,為犯,同在營室。



 五月戊午,熒惑從行在歲星西南六寸,為犯,同在婁度。
 八月辛巳,熒惑從行入東井,在南轅西第一星東北一尺四寸。



 十一月丁巳,熒惑逆行在五諸侯東星北四寸,為犯。



 隆昌元年三月乙丑,熒惑從行入輿鬼西北星東一寸,為犯。癸酉,熒惑從行在輿鬼積尸星東北七寸,為犯。閏三月甲寅,熒惑從入軒轅。



 五月丁酉,熒惑從入太微,在右執法北二寸,為犯。



 建元四年正月己卯,歲星、太白俱從行,同在婁度為合宿。



 六月丁酉,歲星晝見。



 永明元年五月甲午,歲星入東井。



 七月壬午,歲星晝見。



 三年五月丙子,歲星與太白合。



 六月辛丑,歲星與辰星合。
 十月己巳,歲星從入太微。



 十一月甲子,歲星從入太微,犯右執法。



 四年閏二月丙辰,歲星犯太微上將。



 三月庚申,歲星犯太微上將。四月己未,歲星犯右執法。八月乙巳,歲星犯進賢,又與熒惑於軫度合宿。



 五年二月癸卯,歲星犯進賢。



 六月甲子,歲星晝見在軫度。十月己未,歲星從在氐西南星北七寸,又辰星從入氐,在歲星西四尺五寸,又太白從在辰星東,相去一尺,同在氐度,三星為合宿。



 十二月甲戌,歲星晝見。



 六年三月甲申,歲星逆行入氐宿。



 六月丙寅,歲星晝見在氐度。



 八年三月庚申,歲星守牽牛。



 九年二月壬午,歲星從在填星西七寸,同在虛度為合宿。閏七月辛酉,歲星在泣星北五寸,為犯,又守填星。



 九月辛卯,在泣星西一尺五寸,為合宿。



 永明元年六月,辰星從行入太微,在太白西北一尺。



 二年八月甲寅,辰星於翼犯太白。



 九年六月丙子,辰星隨太白於西方,在七星度,相去一尺四寸,為合宿。



 十一年九月丙辰,辰星依歷應夕見西方亢宿一度,至九月八日不見。



 隆昌元年正月丙戌,辰星見危度,在太白北一尺,為犯。



 建元三年十月癸丑,填星逆行守氐。



 四年七月戊辰,填星從行入氐。



 永明元年正月庚寅,填星守房心。



 三月甲子,填星逆行犯西咸星。



 二年二月戊辰,填星犯東咸星。



 四年十二月辛巳,填星犯建星。



 七年十二月戊辰,填星在須女度,又辰星從在填星西南一尺一寸,為合宿。



 八年三月庚申,填星守哭星。



 九年七月庚戌,填星逆在泣西星東北七寸,為犯。



 十月甲午,填星從行在泣星西北五寸,為犯。



 流
 星災建元元年十月癸酉,有流星大如三升塸,色白,尾長五丈,從南河東北二尺出,北行歷輿鬼西過,未至軒轅後星而沒;沒後餘中央,曲如車輪,俄頃化為白雲,久乃滅。流星自下而升,名曰飛星。



 三年十月丙午,有流星大如月,赤白色,尾長七丈,西北行入紫宮中,光照牆垣。



 四年正月辛未,有流星大如三升塸,赤色,從北極第二星北一尺出,北行一丈而沒。



 九月壬子,流星如鵝卵,從柳北出,入軒轅。又一枚如瓜大,出西行沒空中。



 永明元年六月己酉,有流星如二升碗,從紫宮出,南行沒氐。



 二年三月庚辰,有流星如二升碗,從天市中出,南行在心後。



 四年二月乙丑,有流星大如一升器。戊辰,有流星大如五升器。



 四月丁卯,有流星大如一升器,從南斗東北出,西行經斗入氐。六月丙戌,有流星大如鴨卵,從匏瓜南出,至虛而入。八月辛未,有流星大如三升塸,從觜星南出,西南行入天濛沒。



 十一月戊寅,有流星大如二升塸,白色,從亢東北出,行入天市。十二月丁巳,有流星大如三升碗,白色,從天市帝座出,東北行一丈而沒。



 五年六月辛未,有流星大如三升器,沒後有痕。



 九月丙申,有流星大如四升器,白色,有光照地。



 十二月甲子,西北有流星大如鴨卵,黃白色,尾長六尺,西南行一丈餘沒。



 六年三月癸酉,有流星大如鴨卵,赤色,無尾。四月丙辰,北面有流星大如二升器,白色,北行六尺而沒。



 七月癸巳,有流星大如鵝卵,白色,從匏瓜南出,西南行一丈沒空中。須臾,又有流星大如五升器,白色,從北河南出,東北行一丈三尺沒空中。十月戊寅,南面有流星,大如雞卵,赤色,在東南行沒,沒後如連珠。



 十二月壬寅,有流星大如鵝卵,黃白色,尾長三丈,有光,沒後有痕從梗河出,西行一丈許,沒空中。



 七年正月甲寅,有流星如五升器,白色,尾長四尺,從坐旗星出,西行入五車而過,沒空中。



 六月丁丑,流星大如二升器,黃赤色,有光,尾長六尺許,從亢南出,西行入翼中而沒,沒後如連珠。十月乙丑,有流星如三升器,赤黃色,尾長六尺,出紫宮內北極星,東南行三丈沒空中。壬辰,流星如三升器,白色,有光,從五車北
 出,行入紫宮,抵北極第一第二星而過,落空中,尾如連珠,仍有音響似雷。太史奏名曰「天狗」。



 八年四月癸巳,有流星如二升器,黃白色,有光,從心星南一尺許出,南行二丈沒,沒後如連珠。丁巳,流星如鵝卵,白色,長五丈許,從角星東北二尺出,西北行沒太微西蕃上將星間。



 六月癸未,有流星如鴨卵,赤色,從紫宮中出,西南行未至大角五尺許沒。七月戊申,有流星如五升器,赤白色,長七尺,東南行二丈,沒空中。十月乙亥,有流星如鵝卵,白色,從紫宮中出,西北行三丈許,沒空中。



 十一月乙未,有流星如鵝卵,赤白色,有光無尾,從氐北一丈出,南行入氐中沒。辛丑,流星如鵝卵,白色,從參伐出,南行一丈沒空
 中。又有一流星大如三升器,白色,從軫中出,東南行入婁中沒。



 九年五月庚子,有流星如雞子,白色無尾,從紫宮裏黃帝座星西二尺出,南行一丈沒空中。丁未,流星如李子,白色無尾,從奎東北大星東二尺出,東北行至天將軍而沒。戊申,流星如鵝卵,黃白色,尾長二丈,從箕星東一尺出,南行四丈沒。



 七月乙卯,西南有流星大如二升器,白色無尾,西南行一丈餘沒。戊午,有流星如二升器,黃白色,有光從天江星西出,東北經天入參中而沒,沒後如連珠。閏七月戊辰,流星如鵝卵,赤色,尾長二尺,從文昌西行入紫宮沒。己巳,西南有流星如二升器,白色,西南行一丈沒。九月戊子,有流星大如雞卵,白色,從少微星北頭出,東行入太微,抵帝座星而過,未至東蕃次相一尺沒,如散
 珠。



 十年正月甲戌,有流星如五升器,白色,從氐中出,東南行經房道過,從心星南二尺沒。



 三月癸未,有流星如雞卵,青白色,尾長四尺,從牽牛南八寸出,南行一丈沒空中。



 十一年二月壬寅,東北有流星如一升器,白色,無尾,北行三丈而沒。



 四月丙申,有流星如三升器,白色,有光,尾長一丈許,從箕星東北一尺出,行二丈許入斗度,沒空中,臨沒如連珠。五月壬申,有流星大如雞子,黃白色,從太微端門出,無所犯,西南行一丈許沒,沒後有痕。七月辛酉,有流星如雞子,赤色,無尾,從氐中出,西行一丈五尺沒空中。戊寅,有流星如雞卵,黃白色,從紫宮東蕃內出,東
 北行一丈五尺,至北極第五星西北四尺沒。九月乙酉,有流星如鴨卵,黃白色,從婁南一尺出,東行二丈沒。



 十二月己丑,西南有流星如三升器,黃赤色,無尾,西南行三丈許沒,散如遺火。



 永元三年夜,天開黃色明照,須臾有物,絳色,如小甕,漸漸大如倉廩,聲隆隆如雷,墜太湖中,野雉皆雊,世人呼為「木殃」。史臣案:《春秋緯》:「天狗如大奔星,有聲,望之如火,見則四方相射。」漢史云:「西北有三大星,如日狀,名曰天狗。天狗出則人相食。」《天官》云:「天狗狀如大奔星。」又云:「如大流星,色黃,有聲。其止地類狗所墜。望之如火光,炎炎沖天,其上銳,其下圓,如數頃田。見則流血千里,破軍殺將。」漢史又云:「昭明下為天狗,所下兵起血流。」昭明,星也。《洛書》云:「昭明見而霸者出。」《運斗樞》云:「昭明有芒角,兵徵也。」《河圖》云:「太白散為天狗。」漢
 史又云:「有星出,其狀赤白有光,即為天狗,其下小無足,所下國易政。」眾說不同,未詳孰是。推亂亡之運,此其必天狗乎?



 老人星建元元年十一月戊辰,老人星見南方丙上。八月癸卯,祠老人星。



 永明三年八月丁酉,老人星見南方丙上。



 六年八月壬戌,老人星見南方丙上。



 七年七月壬戌,老人星見南方丙上。



 九年閏七月戊寅,老人星見南方丙上。



 十年八月乙酉,老人星見。



 十一年九月丙寅,老人星見南方丙上。



 白虹雲氣建元四年二月辛卯,白虹貫日。



 永
 明十年七月癸酉,西方有白虹,須臾滅。



 十一年九月甲午,西方有白虹,南頭指申,北頭指戌上,久久消滅。



 建元四年二月辛卯,黑氣大小二枚,東至卯,西至酉,廣五丈,久久消滅。



 永明二年四月丁未,北斗第六第七星間有一白氣。



 四年正月辛未,黃白氣長丈五尺許,入太微。



 永明四年正月癸未,南面有陣雲一丈許。



 五年四月己巳,有雲色黑,廣五尺,東頭指丑,西頭指酉,並至地。



 十一月乙巳,東南有陣雲高一丈,北至卯,東南至巳,久久散沒。



 六年二月癸亥,東西有一梗云,半天,曲向西,蒼白色。



 三月庚辰,南面有梗云,黑色,廣六寸。



 七年十月辛未,有梗雲,蒼黑色,東頭至寅,西頭指酉,廣三尺,貫紫
 宮,久久消漫。



 八年十一月乙未,有梗云,黑色,六尺許,東頭至卯,西頭至酉,久久散漫。



 十二月庚辰,南面有陣雲,黑色,高一丈許,東頭至巳,西頭至未,久久散漫。



 十一年七月丙辰,東面有梗云,蒼白色,廣二尺三寸,南頭指巳至地,北頭指子至地,久久漸散漫。



 贊曰:陽精火鏡,陰靈水存。有稟有射,代為明昏。垂光滿蓋,列景周渾。具位臣輔,備象街門。災生霣薄,祟起飛奔。弗忘人懼,瑜瑕辯論。若任天道,灶亦多言。



\end{pinyinscope}