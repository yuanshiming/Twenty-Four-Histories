\article{卷三十列傳第十一 薛淵 戴僧靜 桓康(尹略) 焦度 曹虎}

\begin{pinyinscope}

 薛淵,河東汾陰人也。宋徐州刺史安都從子。本名道淵,避太祖偏諱改。安都以彭城降虜,親族皆入北。太祖鎮淮陰,淵遁來南,委身自結。果乾有氣力,太祖使領部曲,備衛帳內,從征伐。元徽末,以勳官至輔國將軍,右軍將軍,驍騎將軍、軍主,封竟陵侯。



 沈攸之難起,太祖入朝堂,豫章王嶷代守東府,使淵領軍屯司徒左府,分備京
 邑。袁粲據石頭,豫章王嶷夜登西門遙呼淵,淵驚起,率軍赴難,先至石頭焚門攻戰。事平,明旦眾軍還集杜姥宅,街路皆滿,宮門不開,太祖登南掖門樓處分眾軍各還本頓,至食後城門開,淵方得入見太祖,且喜且泣。太祖即位,增邑為二千五百戶。除淮陵太守,加寧朔將軍,驍騎將軍如故。尋為直閣將軍,冠軍將軍。仍轉太子左率。



 虜遣偽將薛道摽寇壽春,太祖以道摽淵之親近,敕齊郡太守劉懷慰曰:「聞道摽分明來,其兒婦並在都,與諸弟無復同生者,凡此類,無為不多方誤之,縱不全信,足使豺狼疑惑。」令為淵書與道摽示購之之意,虜得書,果追道摽,遣他將代之。



 世祖即位,遷左衛將軍。



 初,淵南奔,母索氏不得自拔,改嫁長安楊氏,淵私遣購贖,梁州刺史崔慧景報淵云:「索在界首,遣信拘引,已得拔難。」淵表求解職至界上迎之,見許。改授散騎常侍、征虜將軍。淵母南歸事竟
 無實。永明元年,淵上表解職送貂蟬。詔曰:「遠隔殊方,聲問難審。淵憂迫之深,固辭朝列。昔東關舊典,猶通婚宦;況母出有差,音息時至,依附前例,不容申許,便可斷表,速還章服。」淵以贖母既不得,又表陳解職,詔不許。後虜使至,上為淵致與母書。



 車駕幸安樂寺,淵從駕乘虜橋。先是敕羌虜橋不得入仗,為有司所奏,免官,見原。四年,出為持節、督徐州諸軍事、徐州刺史,將軍如故。明年遷右軍司馬,將軍如故,轉大司馬,濟陽太守,將軍如故。七年,為給事中、右衛將軍,以疾解職。歸家,不能乘車,去車腳,使人輿之而去,為有司所糾,見原。八年,為右將軍、大司馬,領軍討巴東王子響。子響軍主劉超之被捕急,以眠褥雜物十餘種賂淵自逃,淵匿之軍中,為有司所奏,詔原。十年,為散騎常侍,將軍如故。



 世祖崩,朝廷慮虜南寇,假淵節,軍主、本官如故。尋加驍騎將軍,假節、本官如故。隆昌元年,出
 為持節、督司州軍事、司州刺史,右將軍如故。延興元年,進號平北將軍,未拜,卒。明帝即位,方有詔賻錢五萬,布五百匹,克日舉哀。



 戴僧靜,會稽永興人也。祖飾,宋景平中,與富陽孫法先謀亂伏法,家口徙青州。僧靜少有膽力,便弓馬。事刺史沈文秀,俱沒虜。後將家屬叛還淮陰,太祖撫畜之,常在左右。僧靜於都載錦出,為歐陽戍所得,系兗州獄,太祖遣薛淵餉僧靜酒食,以刀子置魚腹中。僧靜與獄吏飲酒,既醉,以刀刻械,手自折鎖發屋而出。



 歸,太祖匿之齋內。以其家貧,年給穀千斛。虜圍角城,遣僧靜戰,蕩數捷,補帳內軍主。隨還京師,勛階至積射將軍、羽林監。



 沈攸之事起,太祖入朝堂,僧靜為軍主從。袁粲據石頭,太祖遣僧靜將腹心先至石頭。時蘇烈據倉城,僧靜射書與烈,夜縋入城。粲登城西南門,列燭火處分,臺軍至,射之,火乃滅,回登東門。其黨輔國將軍孫曇瓘驍勇
 善戰,每蕩一合,輒大殺傷,官軍死者百餘人。軍主王天生殊死拒戰,故得相持。自亥至丑,有流星赤色照地墜城中,僧靜率力攻倉門,身先士卒,眾潰,僧靜手斬粲,於是外軍燒門入。



 初,粲大明中與蕭惠開、周朗同車行,逢大桁開,駐車共語。惠開取鏡自照曰:「無年可仕。」朗執鏡良久曰:「視死如歸。」粲最後曰:「當至三公而不終也。」



 僧靜以功除前軍將軍,寧朔將軍。將士戰亡者,太祖為斂祭焉。



 升明二年,除游擊將軍。沈攸之平,論封諸將,以僧靜為興平縣侯,邑千戶。



 太祖即位,增邑千二百戶。除南濟陰太守,本官如故。除輔國將軍,改封建昌。建元二年,遷驍騎將軍,加員外常侍,轉太子左衛率。



 世祖踐阼,出為持節、督徐州諸軍事、冠軍將軍、北徐州刺史。買牛給貧民令耕種,甚得荒情。遷給事中、太子右率。尋加通直常侍。永明五年,隸護軍陳顯達,討荒賊桓天生於比陽。僧靜與平西司馬韓
 孟度、華山太守康元隆前進,未至比陽四十里,頓深橋。天生引虜步騎十萬奄至,僧靜合戰,大破之,殺獲萬計。天生退還比陽,僧靜進圍之。天生軍出城外,僧靜又擊破之。天生閉門不復出,僧靜力疲乃退。除征虜將軍、南中郎司馬、淮南太守。



 八年,巴東王子響殺僚佐,世祖召僧靜使領軍向江陵,僧靜面啟上曰:「巴東王年少,長史捉之太急,忿不思難故耳。天子兒過誤殺人,有何大罪!官忽遣軍西上,人情惶懼,無所不至,僧靜不敢奉敕。」上不答而心善之。徙為廬陵王中軍司馬、高平太守,將軍如故。九年,卒。詔曰:「僧靜志懷貞果,誠著艱難。剋殄西墉,勛彰運始。奄致殞喪,惻愴傷懷。賻錢五萬,布百匹。謚壯侯。」



 僧靜同郡餘姚人陳胤叔,本名承叔,避宣帝諱改。強辯果捷,便刀楯。初為左夾轂隊將。泰始初,隨太祖東討,遂歸身隨從征伐。小心慎事,以功見賞,封當陽縣子,官至太子左率。啟
 世祖以狖箭金禁用鐵多,不如鑄作。東冶令張候伯以鑄金禁鈍,不合用,事不行。永明三年,卒。



 桓康,北蘭陵承人也。勇果驍悍。宋大明中,隨太祖為軍容。從世祖在贛縣。



 泰始初,世祖起義,為郡所縶,眾皆散。康裝擔,一頭貯穆後,一頭貯文惠太子及竟陵王子良,自負置山中。與門客蕭欣祖、楊彖之、皋分喜、潛三奴、向思奴四十餘人相結,破郡獄出世祖。郡追兵急,康等死戰破之。隨世祖起義,摧堅陷陣,膂力絕人。所經村邑,恣行暴害。江南人畏之,以其名怖小兒,畫其形以辟瘧,無不立愈。見擢為世祖冠軍府參軍,除殿中將軍,武騎常侍,出補襄賁令。桂陽事起,康棄縣還都就太祖,會事平,除員外郎。



 元徽五年七月六日夜,少帝微行至領軍府,帝左右人曰:「一府人皆眠,何不緣牆入。」帝曰:「我今夕欲一處作適,待明日夜。」康與太祖所養健兒盧荒、向黑於
 門間聽得其語。明夕,王敬則將帝首至,扣門,康謂是變,與荒、黑曉下,拔白刃欲出。仍隨入宮。太祖鎮東府,除康武陵王中兵、寧朔將軍,帶蘭陵太守,常衛左右。



 太祖誅黃回,回時將為南兗州,部曲數千,欲收,恐為亂。召入東府,停外齋,使康將數十人數回罪,然後殺之。回初與屯騎校尉王宜與同石頭之謀,太祖隱其事,猶以重兵付回而配以腹心。宜與拳捷,善舞刀楯,回嘗使十餘人以水交灑,不能著。既慮宜與反己,乃先撤其軍將,宜與不與,回發怒,不從處分,擅斬之。諸將因此白太祖,以回握強兵,必遂反覆。康請獨往刺之,太祖曰:「卿等何疑甚,彼無能為也。」及回被召上車,愛妾見赤光冠其頭至足,苦捉留,回不肯止。時人為之語曰:「欲侜張,問桓康。」除後軍將軍,直閣將軍,南濮陽太守,寧朔如故。



 建元元年,封吳平縣伯,五百戶。轉輔國將軍,左軍將軍,遊擊將軍,太守如故。



 太祖謂
 康曰:「卿隨我日久,未得方伯,亦當未解我意,政欲與卿先共滅虜耳。」



 虜動,遣康行,假節。尋進冠軍將軍。三年春,於淮陽與虜戰,大破之,進兵攻陷虜樊諧城。太祖喜,敕康迎淮北義民,不剋。明年,以康為持節、督青冀二州東徐之東莞瑯邪二郡朐山戍北徐之東海漣口戍諸軍事、青冀二州刺史,冠軍如故。世祖即位,轉驍騎將軍,復前軍郡。其年,卒。詔曰:「康昔預南勛,義兼常懷,倍深惻愴。凶事所須,厚加料理。」年五十七。



 淮南人尹略,少伏事太祖,晚習騎射,以便捷見使為將。昇明中,為虎賁中郎、越騎校尉。建元初,封平固男,三百戶。永明八年,為游擊將軍,討巴東王子響,見害。贈輔國將軍、梁州刺史。



 焦度,字文續,南安氐人也。祖文珪,避難至襄陽。宋元嘉中,僑立天水郡略陽縣,乃屬焉。度以歸國,補北館客。孝武初,青州刺史顏師
 伯出鎮,臺差度領幢主送之。索虜寇青州,師伯遣度領軍與虜戰於沙溝杜梁,度身破陣,大捷。師伯板為己輔國府參軍。虜遣清水公拾賁敕文寇清口,度又領軍救援,刺虜騎將豹皮公墮馬,獲其具裝鎧霡,手殺數十人。師伯啟孝武稱度氣力弓馬並絕人,帝召還充左右。



 見度身形黑壯,謂師伯曰:「真健物也。」除西陽王撫軍長兼行參軍,補晉安王子勛夾轂隊主,隨鎮江州。



 子勛起兵,以度為龍驤將軍,領三千人為前鋒,屯赭圻。每與臺軍戰,常自排突,所向無不勝。事敗,逃宮亭湖中為寇賊。朝廷聞其勇,甚憂患之,使江州刺史王景文誘降度等,度將部曲出首,景文以為己鎮南參軍,尋領中直兵,厚待之。隨景文還都,常在府州內。景文被害夕,度大怒,勸景文拒命,景文不從。明帝不知也。以度武勇,補晉熙王燮防閣,除征虜鎧曹行參軍,隨鎮夏口。武陵王贊代燮為郢州,度仍留鎮,為
 贊前軍參軍。



 沈攸之事起,轉度中直兵,加寧朔將軍、軍主。太祖又遣使假度輔國將軍、屯騎校尉。攸之大眾至夏口,將直下都,留偏兵守郢城而已。度於城樓上肆言罵辱攸之,至自發露形體穢辱之,故攸之怒,改計攻城。度親力戰,攸之眾蒙楯將登,度令投以穢器,賊眾不能冒,至今呼此樓為「焦度樓」。事寧,度功居多,轉後軍將軍,封東昌縣子,東宮直閣將軍。為人朴澀,欲就太祖求州,比及見,意色甚變,竟不得一語。太祖以其不閑民事,竟不用。建元四年,乃除淮陵太守,本官如故。



 度見朝廷貴戚,說郢城事,宣露如初。好飲酒,醉輒暴怒。上常使人節之。年雖老,而氣力如故。尋除游擊將軍。永明元年,卒,年六十一。贈輔國將軍、梁秦二州刺史。



 子世榮,永明中為巴東王防閣。子響事,世榮避奔雍州,世祖嘉之,以為始興中兵參軍。



 曹虎,字士威,下邳下邳人也,本名虎頭。宋明帝末,為直廂。桂陽賊
 起,隨太祖出新亭壘出戰,先斬一級持還,由是識太祖。太祖為領軍,虎訴勳,補防殿隊主,直西齋。蒼梧廢,明日,虎欲出外避難,遇太祖在東中華門,問虎何之。虎因曰:「故欲仰覓明公耳。」仍留直衛。太祖鎮東府,以虎與戴僧靜各領白直三百人。



 累至屯騎校尉,帶南城令。豫平石頭,封羅江縣男,除前軍將軍。上受禪,增邑為四百戶。直閣將軍,領細仗主。尋除寧朔將軍、東莞太守。建元元年冬,虎啟乞改封侯官,尚書奏侯官戶數殷廣,乃改封監利縣。二年,除游擊將軍,本官如故。及彭、沛義民起,遣虎領六千人入渦。沈攸之橫吹一部,京邑之絕,虎啟以自隨。義民久不至,虎乃攻虜別營破之。將士貪取俘執,反為虜所敗,死亡二千人。



 世祖即位,除員外常侍,遷南中郎司馬,加寧朔將軍、南新蔡太守。永明元年徙為安成王征虜司馬,餘官如故。明年,江州蠻動,敕虎領兵戍尋陽,板輔國將軍,
 伐蠻軍主。又領尋陽相。尋除游擊將軍,輔國、軍主如故。世祖以虎頭名鄙,敕改之。



 六年四月,荒賊桓天生復引虜出據隔城,遣虎督數軍討之。虎令輔國將軍朱公恩領騎百匹及前行踏伏,值賊游軍,因合戰破之。遂進至隔城。賊黨拒守,虎引兵圍柵,絕其走路。須臾,候騎還報虜援已至,尋而天生率馬步萬餘人迎戰,虎奮擊大敗之,獲二千餘人。明日,遂攻隔城拔之,斬偽虎威將軍襄城太守帛烏祝,復殺二千餘人,賊棄平氏城退走。七年,遷冠軍將軍,驍騎如故。明年,遷太子左率,轉西陽王冠軍司馬、廣陵太守。上敕虎曰:「廣陵須心腹,非吾意可委者,不可得處此任。」隨郡王子隆代巴東王子響為荊州,備軍容西上,以虎為輔國將軍、鎮西司馬、南平內史。十一年,收雍州刺史王奐,敕領步騎數百,步道取襄陽。仍除持節、督梁南北秦沙四州諸軍事、西戎校尉、梁南秦二州刺史,將軍
 如故。尋進號征虜將軍。鬱林即位,進號前將軍。隆昌元年,遷督雍州郢州之竟陵司州之隨郡軍事、冠軍將軍、雍州刺史。建武元年,進號右將軍。二年,進督為監,進號平北將軍,爵為侯,增邑三百戶。



 四年,虜寇沔北,虎聚軍襄陽,與南陽太守房伯玉不協,不急赴救,末乃移頓樊城。虜主元宏遺虎書曰:「皇帝謝偽雍州刺史:神運兆中,皇居闡洛。化總元天,方融八表。而南有未賓之吳,治為兩主之隔。幽顯含嗟,人靈雍閼。且漢北江邊,密爾乾縣,故先動鳳駕,整我神邑。卿進無陳平歸漢之智,退闕關羽殉節之忠,嬰閉窮城,憂頓長沔,機勇兩缺,何其嗟哉!朕比乃欲造卿,逼冗未果,且還新都,饗厥六戎,入彼春月,遲遲揚旆,善脩爾略,以俟義臨。」虎使人答書曰:「自金精失道,皇居徙縣,喬木空存,茂草方鬱。七狄交侵,五胡代起,顧瞻中原,每用弔焉。知棄皋蘭,隨水瀍澗,伊川之象,爰在茲日。古
 人有云:『匪宅是卜,而鄰是卜。』樊、漢無幸,咫尺殊風,折膠入塞,乘秋犯邊,親屬窮於斬殺,士女困於虔劉。與彼蠢左,共為唇齒,仁義弗聞,苛暴先露。乃復改易氈裘,妄自尊大。我皇開運,光宅區夏,而式亂逋逃,棄同即異。每欲出車鞠旅,以征不庭,所冀干戚兩階,叛命來格,遂復遊魂不戢,乾沒孔熾。孤總連率,任屬方邵,組甲十萬,雄戟千群,以此戡難,何往不克。主上每矜率土,哀彼民黎,使不戰屈敵,兵無血刃。



 故部勒小戍,閉壁清野,抗威遵養,庶能懷音。若遂迷復,知進忘退,當金鉦戒路,雲旗北掃,長驅燕代,併羈名王,使少卿忽諸,頭曼不祀。兵交無遠,相為憫然。」



 永泰元年,遷給事中,右衛將軍,持節,隸都督陳顯達停襄陽伐虜。度支尚書崔慧景於鄧地大敗,虜追至沔北。元宏率十萬眾,從羽儀華蓋,圍樊城。虎閉門固守。虜去城數里立營頓,設氈屋,復再圍樊城,臨沔水,望襄陽岸乃去。虎遣
 軍主田安之等十餘軍出逐之,頗相傷殺。東昏即位,遷前將軍,鎮軍司馬。永元元年,始安王遙光反,虎領軍屯青溪中橋。事寧,轉散騎常侍、右衛將軍。



 虎形乾甚毅,善於誘納,日食荒客常數百人。晚節好貨賄,吝嗇,在雍州得見錢五千萬,伎女食醬菜,無重肴。每好風景,輒開庫拍張向之。帝疑虎舊將,兼利其財,新除未及拜,見殺,時年六十餘。和帝中興元年,追贈安北將軍、徐州刺史。



 史臣曰:解厄鴻門,資舞陽之氣;納降饗旅,仗虎侯之力。觀茲猛毅,藉以風威,未必投車挾輈,然後勝敵。故桓康之聲,所以震懾江蠡也。



 贊曰:薛辭親愛,歸身淮涘。戴類千秋,興言帝子。桓勇焦壯,爪牙之士。虎守西邊,功虧北鄙。



\end{pinyinscope}