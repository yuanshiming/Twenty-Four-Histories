\article{卷六本紀第六明帝}

\begin{pinyinscope}

 高宗明皇帝諱鸞,字景棲,始安貞王道生子也。小諱玄度。少孤,太祖撫育,恩過諸子。宋泰豫元年,為安吉令,有嚴能之名。補武陵王左常侍,不拜。元徽二年,為永世令。升明二年,為邵陵王安南記室參軍,未拜,仍遷寧朔將軍、淮南宣城二郡太守。尋進號輔國將軍。太祖踐阼,遷侍中,封西昌侯,邑千戶。建元二年,為持節、督郢州司州之義陽諸軍事、冠軍將軍、郢州刺史,進號征虜將軍。世祖即位,轉度支尚書,領右軍將軍。永明元年,遷侍中,領驍騎將軍。王子侯舊乘纏帷車,高宗獨乘下帷,儀從如素士。公事混撓,販食人擔火誤燒牛鼻,豫章王白世祖,世祖笑焉。轉為散騎常侍、左衛將
 軍,清道而行,上甚悅。二年,出為征虜將軍、吳興太守。四年,遷中領軍,常侍並如故。五年,為持節、監豫州郢州之西陽司州之汝南二郡軍事、右將軍、豫州刺史。七年,為尚書右僕射。八年,加領衛尉。十年,轉左僕射。十一年,領右衛將軍。世祖遺詔為侍中、尚書令,尋加鎮軍將軍,給班劍二十人。隆昌元年,即本號為大將軍,給鼓吹一部,親兵五百人。尋又加中書監、開府儀同三司。鬱林王廢,海陵王立,為使持節、都督揚南徐二州軍事、驃騎大將軍、錄尚書事、揚州刺史,開府如故,增班劍為三十人,封宣城郡公,二千戶。鎮東府城。給兵五千人,錢二百萬,布千匹。九江作難,假黃皞,事寧,表送之。尋加黃皞、都督中外諸軍事、太傅,領大將軍、揚州牧,增班劍為四十人,給幢絡三望車,前後部羽葆鼓吹,劍履上殿,入朝不趨,贊拜不名,置左右長史、司馬、從事中郎、掾、屬各四人,封宣城王,邑五千戶,持
 節、侍中、中書監、錄尚書並如故。未拜,太后令廢海陵王,以上入纂太祖為第三子,群臣三請,乃受命。



 建武元年冬,十月,癸亥,即皇帝位。詔曰:「皇齊受終建極,握鏡臨宸,神武重輝,欽明懿鑠,七百攸長,盤石斯固。而王度中蹇,天階薦阻,嗣命多違,蕃釁孔棘,宏圖景曆,將墜諸淵。宣德皇后遠鑒崇替,憲章舊典,疇咨臺揆,允定靈策,用集寶命於予一人。猥以虛薄,纘承大業,仰繫鴻丕,顧臨兆民,永懷先構,若履春冰,寅憂夕惕,罔識攸濟,思與萬國播此惟新。大赦天下,改元。宿衛身普轉一階,其餘文武,賜位二等。逋租宿責,換負官物,在建武元年以前,悉原除。



 劫賊餘口在臺府者,可悉原放。負釁流徙,並還本鄉。」太尉王敬則為大司馬,司空陳顯達為太尉,尚書令王晏加驃騎大將軍,中領軍蕭諶為領軍將軍、南徐州刺史,皇子寶義為揚州刺史,中護軍王
 玄邈為南兗州刺史,新除右將軍張瑰為右光祿大夫,平北將軍王廣之為江州刺史。乙丑,詔斷遠近上禮。丁卯,詔「自今雕文篆刻,歲時光新,可悉停省。蕃牧守宰,或有薦獻,事非任土,嚴加禁斷。」追贈安陸昭侯緬為安陸王。己巳,以安陸侯子寶晊為湘州刺史。詔曰:「頃守職之吏,多違舊典,存私害公,實興民蠹。今商旅稅石頭後渚及夫鹵借倩,一皆停息。所在凡厥公宜,可即符斷。主曹詳為其制,憲司明加聽察。」十一月,癸酉,以西中郎長史始安王遙光為揚州刺史,晉壽太守王洪範為青、冀二州刺史,尚書令王晏領太子少傅。甲戌,大司馬尋陽公王敬則等十三人進爵邑各有差。詔省新林苑,先是民地,悉以還主,原責本直。庚辰,立皇子寶義為晉安王,寶玄為江夏王,寶源為廬陵王,寶夤為建安王,寶融為隨郡王,寶攸為南平王。甲申,詔曰:「邑宰祿薄俸微,不足代耕,雖任土
 恆貢,亦為勞費,自今悉斷。」又詔「宣城國五品以上,悉與滿敘。自此以下,皆聽解遣。其欲仕,適所樂。」乙酉,追尊始安貞王為景皇,妃為懿后。



 丙戌,以輔國將軍聞喜公遙欣為荊州刺史,寧朔將軍豐城公遙昌為豫州刺史。丁亥,詔「細作中署、材官、車府,凡諸工,可悉開番假,遞令休息」。戊子,立皇太子寶卷,賜天下為父後者爵一級,孝子順孫、義夫節婦,普加甄賜明揚。表其衡閭,賚以束帛。己丑,詔「東宮肇建,遠近或有慶禮,可悉斷之。」壬辰,以新除征虜將軍江夏王寶玄為郢州刺史。永明中,御史中丞沈淵表百官年登七十,皆令致仕,並窮困私門。庚子,詔曰:「日者百司耆齒,許以自陳,東西二省,猶沾微俸,辭事私庭,榮祿兼謝,興言愛老,實有矜懷。自縉紳年及,可一遵永明七年以前銓敘之科。」上輔政所誅諸王,是月復屬籍,各封子為侯。



 十二月,壬子,詔曰:「上覽易遺,下情難達,是以
 甘棠見美,肺石流詠。自月一視黃辭,如有含枉不申、懷直未舉者,蒞民之司,並任厥失。」



 二年春,正月,辛未,詔「京師繫囚殊死,可降為五歲刑,三署見徒五歲以下,悉原散。王公以下,各舉所知。隨王公卿士,內外群僚,各舉朕違,肆心極諫。」



 索虜寇司、豫、徐、梁四州。壬申,遣鎮南將軍王廣之督司州征討,右衛將軍蕭坦之督徐州征討,尚書右僕射沈文季督豫州征討。己卯,詔京師二縣有毀發墳壟,隨宜脩理。又詔曰:「食惟民天,義高姬載,蠶實生本,教重軒經。前哲盛範,後王茂則,布令審端,咸必由之。朕肅扆巖廊,思弘風訓,深務八政,永鑒在勤,靜言日昃,無忘寢興。守宰親民之主,牧伯調俗之司,宜嚴課農桑,罔令遊惰,揆景肆力,必窮地利,固脩堤防,考校殿最。若耕蠶殊眾,具以名聞;游怠害業,即便列奏。主者詳為條格。」乙未,虜攻鐘離,徐州刺
 史蕭惠休破之。丙申,加太尉陳顯達使持節、都督西北征討諸軍事。丁酉,內外纂嚴。三月,戊申,詔「南徐州僑舊民丁,多充戎旅,蠲今年三課。」己未,司州刺史蕭誕與眾軍擊虜,破之。詔「雍、豫、司、南兗、徐五州遇寇之家,悉停今年稅調。其與虜交通,不問往罪。」丙寅,停青州麥租。虜自壽春退走。甲申,解嚴。



 夏,四月,己亥朔,詔「三百里內獄訟,同集京師,克日聽覽。此以外委州郡訊察。



 三署徒隸,原遣有差。」索虜圍漢中,梁州刺史蕭懿拒退之。己未,以新除黃門郎裴叔業為徐州刺史。五月,甲午,寢廟成,詔「監作長帥,可賜位一等,役身遣假一年,非役者蠲租同假限。」六月,壬戌,誅領軍將軍蕭諶、西陽王子明、南海王子罕、邵陵王子貞。乙丑,以右衛將軍蕭坦之為領軍將軍。秋,七月,辛未,以右將軍晉安王寶義為南徐州刺史。壬申,以冠軍將軍梁王為司州刺史。辛卯,以氐楊馥之為北秦州刺史、仇池
 公。八月,丁未,以右衛將軍廬陵王寶源為南兗州刺史。庚戌,以新除輔國將軍申希祖為兗州刺史。九月,己丑,改封南平王寶攸為邵陵王,蜀郡王子文為西陽王,廣漢王子峻為衡陽王,臨海王昭秀為巴陵王,永嘉王昭粲為桂陽王。冬,十一月,丁卯,詔曰:「軌世去奢,事殷哲后,訓物以儉,理鏡前王。朕屬流弊之末,襲澆浮之季,雖恭已弘化,刻意隆平,而禮讓未興,侈華猶競。永覽玄風,兢言集愧,思所以還淳改俗,反古移民。可罷東田,毀興光樓。」並詔水衡量省御乘。



 己卯,納皇太子妃褚氏,大赦。王公已下,班賜各有差。斷四方上禮。十二月,丁酉,詔曰:「舊國都邑,望之悵然。況乃自經南面,負扆宸居,或功濟當時,德覃一世,而塋壟欑穢,封樹不脩,豈直嗟深牧豎、悲甚信陵而已哉?



 昔中京淪覆,鼎玉東遷,晉元締構之始,簡文遺詠在民,而松門夷替,埏路榛蕪。



 雖年代殊往,撫事興懷。晉帝諸陵,
 悉加脩理,并增守衛。吳、晉陵二郡失稔之鄉,蠲三調有差。」



 三年春,正月,丁卯,以陰平王楊炅子崇祖為沙州刺史,封陰平王。北中郎將建安王寶夤為江州刺史。己巳,詔申明守長六周之制。乙酉,詔「去歲索虜寇邊,緣邊諸州郡將士有臨陣及疾病死亡者,並送還本土。」三月,壬午,詔「車府乘輿有金銀飾校者,皆剔除。」夏,四月,虜寇司州,戍兵擊破之。五月,己巳,以征虜將軍蕭懿為益州刺史,前軍將軍陰廣宗為梁、南秦二州刺史,前新除寧州刺史李慶宗為寧州刺史。秋,九月,辛酉,以冠軍將軍徐玄慶為兗州刺史。冬十月,以輔國將軍申希祖為司州刺史。閏十二月,戊寅,皇太子冠,賜王公以下帛各有差,為父後者賜爵一級。斷遠近上禮。又詔「今歲不須光新,可以見錢為百官供給。」



 四年春,正月,庚午,大赦。詔曰:「嘉肴停俎,定
 方旨於必甘;良玉在攻,表圭璋於既就。是以陶鈞萬品,務本為先;經緯九區,學斅為大。往因時康,崇建庠序,屯虞薦有,權從省廢,謳誦寂寥,倏移年稔,永言古昔,無忘旰昃。今華夏鳷安,要荒慕向,締修東序,實允適時。便可式依舊章,廣延國胄,弘敷景業,光被後昆。」壬寅,詔「民產子者,蠲其父母調役一年,又賜米十斛。新婚者,蠲夫役一年」。丙辰,尚書令王晏伏誅。二月,甲子,以左僕射徐孝嗣為尚書令,征虜將軍蕭季敞為廣州刺史。三月,乙未,右僕射沈文季領護軍將軍。秋,八月,追尊景皇所生王氏為恭太后。索虜寇沔北。冬,十月,又寇司州。甲戌,遣太子中庶子梁王、右軍司馬張稷討之。



 十一月,丙辰,以氐楊靈珍為北秦州刺史、仇池公、武都王。丁亥,詔「所在結課屋宅田桑,可詳減舊價。」十二月,甲子,以冠軍將軍裴叔業為豫州刺史,冠軍將軍徐玄慶為徐州刺史,寧朔將軍左興盛為兗州刺史。丁丑,遣度支
 尚書崔慧景率眾救雍州。



 永泰元年春,正月,癸未朔,大赦。逋租宿債在四年之前,皆悉原除。中軍大將軍徐孝嗣即本號,開府儀同三司。沔北諸郡為虜所侵,相繼敗沒。乙巳,遣太尉陳顯達持節救雍州。丁未,誅河東王鉉、臨賀王子岳、西陽王子文、衡陽王子峻、南康王子氏、永陽王子氏、湘東王子建、南郡王子夏、桂陽王昭粲、巴陵王昭秀。



 二月,癸丑,遣左衛將軍蕭惠休假節援壽陽。辛未,豫州刺史裴叔業擊虜於淮北,破之。辛巳,平西將軍蕭遙欣領雍州刺史。



 三月,丙午,蠲雍州遇虜之縣租布。戊申,詔曰:「仲尼明聖在躬,允光上哲,弘厥雅道,大訓生民,師範百王,軌儀千載。立人斯仰,忠孝攸出,玄功潛被,至德彌闡。雖反袂遐曠,而祧薦靡闕,時祭舊品,秩比諸侯。頃歲以來,祀典陵替,俎豆寂寥,牲奠莫舉,豈所以克昭盛烈,永隆風教者
 哉!可式循舊典,詳復祭秩,使牢餼備禮,欽饗兼申。」夏,四月,甲寅,改元,赦三署囚繫原除各有差。文武賜位二等。丙戌,以鎮軍將軍蕭坦之為侍中、中領軍。己未,立武陵昭王子子坦為衡陽王。丙寅,以西中郎長史劉暄為郢州刺史。丁卯,大司馬會稽太守王敬則舉兵反。



 五月,壬午,遣輔國將軍劉山陽率軍東討。乙酉,斬敬則,傳首。曲赦浙東、吳、晉陵七郡。以後軍長史蕭穎胄為南兗州刺史。丁酉,以北中郎將司馬元和為兗州刺史。秋,七月,以輔國將軍王珍國為青、冀二州刺史。癸卯,以太子中庶子梁王為雍州刺史,太尉陳顯達為江州刺史。己酉,帝崩於正福殿,年四十七。遺詔曰:「徐令可重申八命。中書監本官悉如故,沈文季可左僕射,常侍護軍如故,江祏可右僕射,江祀可侍中,劉暄可衛尉。軍政大事委陳太尉。內外眾事,無大小委徐孝嗣、遙光、坦之、江祏,其大事與沈文季、江祀、劉暄參懷。心
 膂之任可委劉悛、蕭惠休、崔惠景。」葬興安陵。



 帝明審有吏才,持法無所借。制御親幸,臣下肅清。驅使寒人不得用四幅傘,大存儉約。罷世祖所起新林苑,以地還百姓;廢文帝所起太子東田,斥賣之;永明中輿輦舟乘,悉剔取金銀還主衣庫。太官進御食,有裹蒸,帝曰:「我食此不盡,可四片破之,餘充晚食。」而世祖掖庭中宮殿服御,一無所改。性猜忌多慮,故亟行誅戮。潛信道術,用計數,出行幸,先占利害,南出則唱云西行,東遊則唱云北幸。簡於出入,竟不南郊。上初有疾,無輟聽覽,秘而不傳。及寢疾甚久,敕臺省府署文簿求白魚以為治,外始知之。身衣絳衣,服飾皆赤,以為厭勝。巫覡云:「後湖水頭經過宮內,致帝有疾。」帝乃自至太官行水溝。左右啟:「太官若無此水則不立。」帝決意塞之,欲南引淮流。會崩,事寢。



 史臣曰:高宗以支庶纂曆,據猶子而為論,一朝到此,誠非素心,遺
 寄所當,諒不獲免。夫戕夷之事,懷抱多端,或出自雄忍,或生乎畏懾。令同財之親,在我而先棄;進引之愛,量物其必違。疑怯既深,猜似外入,流涕行誅,非云義舉,事茍求安,能無內愧?既而自樹本根,枝胤孤弱,貽厥不昌,終覆宗社。若令壓紐之徵,必委天命,盤庚之祀,亦繼陽甲,杖運推公,夫何譏爾!



 贊曰:高宗傍起,宗國之慶。慕名儉德,垂文法令。兢兢小心,察察吏政。沔陽失土,南風不競。



\end{pinyinscope}