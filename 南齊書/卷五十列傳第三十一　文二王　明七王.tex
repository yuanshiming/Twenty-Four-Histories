\article{卷五十列傳第三十一 文二王 明七王}

\begin{pinyinscope}

 文惠太子四男:安皇后生鬱林王昭業;宮人許氏生海陵恭王昭文;陳氏生巴陵王昭秀;褚氏生桂陽王昭粲。



 巴陵王昭秀,字懷尚,太子第三子也。永明中封曲江公,千五百戶。十年,為寧朔將軍、濟陽太守。鬱林即位,封臨海郡王,二千戶。隆昌元年,為使持節、都督荊雍益寧梁南北秦七州軍事、西中郎將、荊州刺史。延興元年,徵為車騎將軍,衛京師,以永嘉王昭粲代之。



 明帝建武二年,通直常侍庾曇隆啟曰:「周定雒邑,天子置畿內之民;漢都咸陽,三輔為社稷之衛。中晉南遷,事移威弛,近郡名邦,多有
 國食。宋武創業,依擬古典,神州部內,不復別封。而孝武末年,分樹寵子,茍申私愛,有乖訓準。隆昌之元,特開母弟之貴,竊謂非古。聖明御宇,禮舊為先,畿內限斷,宜遵昔制,賜茅授土,一出外州。」詔付尚書詳議。其冬,改封昭秀為巴陵王。永泰元年見殺,年十六。



 桂陽王昭粲,太子第四子也。鬱林立,以皇弟封永嘉郡王,南徐州刺史。延興元年,出為使持節、都督荊雍益寧梁南北秦七州軍事、西中郎將、荊州刺史。明帝立,欲以聞喜公遙欣為荊州,轉昭粲為右將軍,中書令。建武二年,改封桂陽王。



 四年,遷太常,將軍如故。永泰元年見殺,年八歲。



 明帝十一男:敬皇后生東昏侯寶卷,江夏王寶玄,鄱陽王寶夤,和帝;殷貴嬪生巴陵隱王寶義,晉熙王寶嵩;袁貴妃生廬陵王寶源;管淑妃生邵陵王寶攸;許淑媛生桂陽王寶貞。餘皆早夭。



 巴陵隱王寶義,字智勇,明帝長子也。本名明基。建武元年,為持節、都督揚南徐州軍事、前將軍、揚州刺史。封晉安郡王,三千戶。寶義少有廢疾,不堪出人間,故止加除授,仍以始安王遙光代之。轉寶義為右將軍,領兵置佐,鎮石頭。二年,出為使持節、都督南徐州軍事、鎮北將軍、南徐州刺史。東昏即位,進征北大將軍,開府儀同三司,給仗。永元元年,給班劍二十人。始安王遙光誅,為都督揚南徐二州軍事、驃騎大將軍、揚州刺史,持節如故。東府被兵火,屋宇燒殘,帝方營宮殿,不暇修葺。寶義鎮西州。三年,進位司徒。和帝西臺建,以為侍中、司空,使持節、都督、刺史如故。梁王定京邑,宣德太后令以寶義為太尉,領司徒。詔云:「不言之化,形于自遠。」時人皆云此實錄也。梁受禪,封謝沐縣公,尋封巴陵郡王,奉齊後。天監中薨。



 江夏王寶玄,字智深,明帝第三子也。建武元年,為征虜將軍,領石
 頭戍事,封江夏郡王。仍出為持節、都督郢司二州軍事、西中郎將、郢州刺史。永泰元年,還為前將軍,領石頭戍事。未拜,東昏即位,進號鎮軍將軍。永元元年,又進車騎將軍,代晉安王寶義為使持節、都督南徐兗二州軍事、南徐兗二州刺史,將軍如故。



 寶玄娶尚書令徐孝嗣女為妃,孝嗣被誅離絕,少帝送少姬二人與之,寶玄恨望,密有異計。明年,崔慧景舉兵,還至廣陵,遣使奉寶玄為主。寶玄斬其使,因是發將吏防城。帝遣馬軍主戚平、外監黃林夫助鎮京口。慧景將渡江,寶玄密與相應,殺司馬孔矜、典簽呂承緒及平、林夫,開門納慧景。使長史沈佚之、諮議柳憕分部軍眾,乘八扛輿,手執絳麾幡,隨慧景至京師,住東城,百姓多往投集。慧景敗,收得朝野投寶玄及慧景軍名,帝令燒之,曰:「江夏尚爾,豈復可罪餘人。」寶玄逃奔數日乃出。帝召入後堂,以步鄣裹之,令群小數十人鳴
 鼓角馳繞其外,遣人謂寶玄曰:「汝近圍我亦如此。」少日乃殺之。



 廬陵王寶源,字智淵,明帝第五子也。建武元年,為北中郎將,鎮瑯邪城,封廬陵郡王。遷右將軍,領石頭戍事,乃出為使持節、都督南兗兗徐青冀五州軍事、後將軍、南兗州刺史。王敬則伏誅,徙寶源為都督會稽東陽臨海永嘉新安五郡軍事、會稽太守,將軍如故。永元元年,進號安東將軍。和帝即位,以為侍中、車騎將軍、開府儀同三司,都督、太守如故。未拜,中興二年薨。



 鄱陽王寶夤,字智亮,明帝第六子也。建武初,封建安郡王。二年,為北中郎將,鎮瑯邪城。明年,出為持節、都督江州軍事、南中郎將、江州刺史。東昏即位,為使持節、都督郢司二州軍事、征虜將軍、郢州刺史。尋進號前將軍。永元二年,徵為撫軍,領石頭戍事,未拜。三年,為車騎將軍、開府儀同三司,鎮石頭。其秋,雍州刺史張欣泰等謀
 起事於新亭,殺臺內諸主帥,事在《欣泰傳》。難作之日,前南譙太守王靈秀奔往石頭,率城內將吏見力去車腳載寶夤向臺城,百姓數千人皆空手隨後,京邑騷亂。寶夤至杜姥宅,日已欲暗,城門閉,城上人射之,眾棄寶夤逃走。寶夤逃亡三日,戎服詣草市尉,尉馳以啟帝,帝迎寶夤入宮問之。寶夤涕泣稱:「爾日不知何人逼使上車,仍將去,制不自由。」帝笑,乃復爵位。和帝立,西臺以寶夤為使持節、都督南徐兗二州軍事、衛將軍、南徐州刺史。少帝以為使持節、都督荊益寧雍梁南北秦七州軍事、荊州刺史,將軍如故。宣德太后臨朝,梁王為建安公,改封寶夤為鄱陽王。中興二年謀反,奔魏。



 邵陵王寶攸,字智宣,明帝第九子也。建武元年,封南平郡王。二年,改封。



 三年,為北中郎將,鎮琅邪城。永元元年,為持節、都督南北徐南兗青冀五州軍事、南兗州刺史,郎將如故。未拜,遷征虜將軍,領
 石頭戍事。丹陽尹,戍事如故。陳顯達事平,出為持節、督江州軍事、左將軍、江州刺史。以本號還京師,授中軍將軍,秘書監。中興二年謀反,宣德太后令賜死。



 晉熙王寶嵩,字智靖,明帝第十子也。永元二年,為冠軍將軍、丹陽尹。仍遷持節、都督南徐兗二州軍事、南徐州刺史,將軍如故。中興元年,和帝以為中書令。



 明年,謀反伏誅。



 桂陽王寶貞,明帝第十一子也。永元二年,為中護軍、北中郎將,領石頭戍事。



 中興二年謀反,伏誅。



 史臣曰:《春秋》書「鄭伯克段於鄢」,兄弟之恩離,君臣之義正。夫逆順有勢,況親兼一體,道窮數盡,或容觸啄。而寶玄自尋干戈,欣受家難。曾不悟執柯所指,跗萼相從,以此而圖萬全,未知其仿佛也。



 贊曰:文惠二王,於嗟夭殤。明子七國,終亦衰亡。



\end{pinyinscope}