\article{卷四十列傳第二十一 武十七王}

\begin{pinyinscope}

 武帝二十三男:穆皇后生文惠太子、竟陵文宣王子良;張淑妃生盧陵王子卿、魚復侯子響;周淑儀生安陸王子敬、建安王子真;阮淑媛生晉安王子懋、衡陽王子峻;王淑儀生隨郡王子隆;蔡婕妤生西陽王子明;樂容華生南海王子罕;傅充華生巴陵王子倫;謝昭儀生邵陵王子貞;江淑儀生臨賀王子岳;庾昭容生西陽王子文;荀昭華生南康王子琳;顏婕妤生永陽王子氏;宮人謝生湘東王子建;何充華生南郡王子夏;第六、十二、十五、二十二皇子早亡。子氏建武中繼衡陽元王後。



 竟陵文宣王子良,字雲英,世祖第二子也。初,沈攸之難,隨世祖在
 盆城,板寧朔將軍。仍為宋邵陵王左軍行參軍,轉主簿,安南記室參軍,邵陵王友,王名友。



 不廢此官。遷安南長史。升明三年,為使持節、都督會稽東陽臨海永嘉新安五郡、輔國將軍、會稽太守。



 宋世元嘉中,皆責成郡縣;孝武徵求急速,以郡縣遲緩,始遣臺使,自此公役勞擾。太祖踐阼,子良陳之曰:前臺使督逋切調,恆聞相望於道。及臣至郡,亦殊不疏。凡此輩使人,既非詳慎勤順,或貪險崎嶇,要求此役。朝辭禁門,情態即異;暮宿村縣,威福便行。但令朱鼓裁完,鈹槊微具,顧眄左右,叱吒自專。擿宗斷族,排輕斥重,脅遏津埭,恐喝傳郵。破崗水逆,商旅半引,逼令到下,先過己船。浙江風猛,公私畏渡,脫舫在前,驅令俱發。呵蹙行民,固其常理;侮折守宰,出變無窮。既瞻郭望境,便飛下嚴符,但稱行臺,未顯所督。先訶彊寺,卻攝群曹,開亭正榻,便振荊革。其次絳標寸紙,一日數至;徵村切里,
 俄刻十催。四鄉所召,莫辨枉直,孩老士庶,具令付獄。或尺布之逋,曲以當匹;百錢餘稅,且增為千。或誑應質作尚方,寄繫東冶,萬姓駭迫,人不自固。遂漂衣敗力,競致兼漿。值今夕酒諧肉飫,即許附申赦格;明日禮輕貨薄,便復不入恩科。筐貢微闕,總棰撻肆情,風塵毀謗,隨忿而發。及其蒜轉積,鵝慄漸盈,遠則分鬻他境,近則託貿吏民。反請郡邑,助民由申緩,回刺言臺,推信在所。如聞頃者令長守牧,離此每實,非復近歲。愚謂凡諸檢課,宜停遣使。密畿州郡,則指賜敕令,遙外鎮宰,明下條源。既各奉別旨,人競自罄。雖復臺使盈湊,會取正屬所辦,徒相疑僨,反更淹懈。凡預衣冠,荷恩盛世,多以闇緩貽愆,少為欺猾入罪。若類以宰牧乖政,則觸事難委,不容課逋上綱,偏覺非才。但賒促差降,各限一期,如乃事速應緩,自依違糾坐之。坐之之科,不必須重,但令必行,期在可肅。且兩裝之船,充擬千緒;
 三坊寡役,呼訂萬計。每一事之發,彌晨方辦,粗計近遠,率遣一部,職散人領,無減二十,舟船所資,皆復稱是。長江萬里,費固倍之。較略一年,脫得省者,息船優役,實為不少。兼折姦減竊,遠近暫安。



 封聞喜縣公,邑千五百戶。



 子良敦義愛古。郡民朱百年有至行,先卒,賜其妻米百斛,蠲一民給其薪蘇。



 郡閣下有虞翻舊床,罷任還,乃致以歸。後於西邸起古齋,多聚古人器服以充之。



 夏禹廟盛有禱祀,子良曰:「禹泣辜表仁,菲食旌約,服玩果粽,足以致誠。」使歲獻扇簟而已。



 建元二年,穆妃薨,去官。仍為征虜將軍、丹陽尹。開私倉賑屬縣貧民。明年,上表曰:「京尹雖居都邑,而境壤兼跨,廣袤周輪,幾將千里。縈原抱隰,其處甚多,舊遏古塘,非唯一所。而民貧業廢,地利久蕪。近啟遣五官殷濔、典簽劉僧瑗到諸縣循履,得丹陽、溧陽、永世等四縣解,並村耆辭列,堪墾之田,合計荒熟有八千五百五
 十四頃;修治塘遏,可用十一萬八千餘夫,一春就功,便可成立。」上納之。會遷官,事寢。



 是年,始制東宮官僚以下官敬子良。



 世祖即位,封竟陵郡王,邑二千戶。為使持節、都督南徐兗二州諸軍事、鎮北將軍、南徐州刺史。永明元年,徙為侍中、都督南兗兗徐青冀五州、征北將軍、南兗州刺史,持節如故。給油絡車。明年,入為護軍將軍,兼司徒,領兵置佐,侍中如故。鎮西州。三年,給鼓吹一部。四年,進號車騎將軍。



 子良少有清尚,禮才好士,居不疑之地,傾意賓客,天下才學皆游集焉。善立勝事,夏月客至,為設瓜飲及甘果,著之文教。士子文章及朝貴辭翰,皆發教撰錄。



 是時上新親政,水旱不時。子良密啟曰:臣思水潦成患,良田沃壤變為汙澤。農政告祥,因高肆務,播植既周,繼以旱虐。黔庶呼嗟,相視褫氣。夫國資於民,民資於食,匪食匪民,何以能政?臣每一念此,寢不便席。本始中,郡國大旱,
 宣帝下詔除民租。今聞所在逋餘尚多,守宰嚴期,兼夜課切,新稅力尚無從,故調於何取給?政當相驅為盜耳。愚謂逋租宜皆原除,少降停恩,微紓民命。



 自宋道無章,王風陵替,竊官假號,駢門連室。今左民所檢,動以萬數,漸漬之來,非復始適,一朝洗正,理致沸騰。小人之心,罔思前恩,董之以威,反怨後罰。獸窮則觸,事在匪輕。齊有天下日淺,恩洽未布,一方或饑,當加優養。愚謂自可依源削除,未宜便充猥役。且部曹檢校,誠存精密;令史奸黠,鮮不容情。情既有私,理或枉謬。耳目有限,群狡無極。變易是非,居然可見。詳而後取,於事未遲。



 明詔深矜獄圄,恩文累墜。今科網嚴重,稱為峻察。負罪離愆,充積牢戶。暑時鬱蒸,加以金鐵。聚憂之氣,足感天和。民之多怨,非國福矣。



 頃土木之務,甚為殷廣,雖役未及民,勤費已積。炎旱致災,或由於此。皇明載遠,書軌未一,緣淮帶江,數州地耳。以魏方
 漢,猶一郡之譬,以今比古,復為遠矣。何得不愛其民,緩其政,救其危,存其命哉?



 湘區奧密,蠻寇熾強,如聞南師未能挫戮。百姓齊民,積年塗炭,疽食侵淫,邊虞方重。交州夐絕一垂,實惟荒服,恃遠後賓,固亦恆事。自青德啟運,款關受職,置之度外,不足絓言。今縣軍遠伐,經途萬里,眾寡事殊,客主勢異,以逸待勞,全勝難必。又緣道調兵,以足軍力,民丁烏合,事乖習銳。廣州積歲無年,越州兵糧素乏,加以發借,必致恇擾。愚謂叔獻所請,不宜聽從;取亂侮亡,更俟後會。雖緩歲月,必有可禽之理,差息發動費役之勞。劉楷見甲以助湘中,威力既舉,蟻寇自服。



 詔折租布,二分取錢。子良又啟曰:臣一月入朝,六登玫陛,廣殿稠人,裁奉顏色,縱有所懷,豈敢自達。比天眚亟見,地孽亟臻,民下妖訛,好生噂沓。穀價雖和,比室饑嗛;縑纊雖賤,駢門裸質。臣一念此,每入心骨。三吳奧區,地惟河、輔,百度所
 資,罕不自出,宜在蠲優,使其全富。而守宰相繼,務在裒剋,圍桑品屋,以准貲課,致令斬樹發瓦,以充重賦,破民財產,要利一時。東郡使民,年無常限,在所相承,準令上直。每至州臺使命,切求懸急,應充猥役,必由窮困。乃有畏失嚴期,自殘軀命;亦有斬絕手足,以避徭役。生育弗起,殆為恆事。守長不務先富民而唯言益國,豈有民貧於下,而國富於上邪?



 又泉鑄歲遠,類多剪鑿,江東大錢,十不一在。公家所受,必須輪郭完全,遂買本一千,加子七百,猶求請無地,棰革相繼。尋完者為用,既不兼兩,回復遷貿,會非委積,徒令小民每嬰困苦。且錢帛相半,為制永久,或聞長宰須令輸直,進違舊科,退容姦利。



 八屬近縣,既在京畿,發借徵調,實煩他邑。民特尤貧,連年失稔,草衣藿食,稍有流亡。今農政就興,宜蒙賑給;若逋課未上,許以申原。兗豫二藩,雖曰舊鎮,往屬兵虞,累棄鄉土。密邇寇庭,下無安
 志。編草結庵,不違涼暑。扶準聚洛,靡有生向。俱稟人靈,獨絕溫飽,而賦斂多少,尚均沃實。謂凡在荒民,應加蠲減。



 又司市之要,自昔所難。頃來此役,不由才舉,並條其重貲,許以賈衒。前人增估求俠,後人加稅請代,如此輪回,終何紀極?兼復交關津要,共相唇齒,愚野未閑,必加陵誑,罪無大小,橫沒貲載。凡求試穀帛,類非廉謹,未解在事所以開容?



 夫獄訟惟平,畫一在制。雖恩家得罪,必宜申憲;鼎姓貽愆,最合從綱。若罰典惟加賤下,辟書必蠲世族,懼非先王立理之本。



 尚書列曹,上應乾象。如聞命議所出,先諮於都,都既下意,然後付郎,謹寫關行。愚謂郎官尤宜推擇。



 宋運告終,戎車屢駕,寄名軍牒,動竊數等。故非分充朝,資奉殷積。廣、越邦宰,梁、益郡邑,參差調補,實允事機。且此徒冗雜,罕遵王憲,嚴加廉視,隨違彈斥,一二年間,可減太半。



 五年,正位司徒,給班劍二十人,侍中如故。移居
 雞籠山邸,集學士抄《五經》、百家,依《皇覽》例為《四部要略》千卷。招致名僧,講語佛法,造經唄新聲。道俗之盛,江左未有也。



 世祖好射雉,子良諫曰:鑾舉亟動,天蹕屢巡,陵犯風煙,驅馳野澤。萬乘至重,一羽甚微。從甚微之懽,忽至重之誡。頃郊郛以外,科禁嚴重,匪直芻牧事罷,遂乃窀掩殆廢。且田月向登,桑時告至,士女呼嗟,易生噂議,棄民從欲,理未可安。曩時巡幸,必盡威防,領軍景先、詹事赤斧堅甲利兵,左右屯衛。今馳鶩外野,交侍疏闊,晨出晚還,頓遺清道,此實愚臣最所震迫。



 狡虜玩威,甫獲款關,二漢全富,猶加曲待。如聞使臣頻亦怨望,前會東宮,遂形言色。昔宋氏遣使,舊列階下,劉纘銜使,始登朝殿。今既反命,宜賜優禮。



 伏謂中堂雲構,實惟峻絕,簷陛深嚴,事隔涼暑,而別為一室,如或有疑。邊帶廣途,訛言孔熾,毀立之易,過於轉圓,若依舊制通敞,實允觀聽。



 頃市司驅扇,租估
 過刻,吹毛求瑕,廉察相繼,被以小罪,責以重備。愚謂宜敕有司,更詳優格。



 臣年方朝賢,齒未相及,以管窺天,猶知失得,廊廟之士,豈暗是非?未聞一人開一說為陛下憂國家,非但面從,亦畏威耳。臣若不啟,陛下於何聞之?



 先是六年,左衛、殿中將軍邯鄲超上書諫射雉,世祖為止。久之,超竟被誅。



 永明末,上將射雉。子良諫曰:忽聞外議,伏承當更射雉。臣下情震越,心懷憂悚,猶謂疑妄,事不必然。伏度陛下以信心明照,所以傾金寶於禪靈,仁愛廣洽,得使禽魚養命於江澤,豈惟國慶民歡,乃以翱翔治樂。夫衛生保命,人獸不殊;重軀愛體,彼我無異。故《禮》云:「聞其聲不食其肉,見其生不忍其死。」且萬乘之尊,降同匹夫之樂,夭殺無辜,傷仁害福之本。菩薩不殺,壽命得長。施物安樂,自無恐怖。不惱眾生,身無患苦。臣見功德有此果報,所以日夜劬勤,厲身奉法,實願聖躬康御若此。每至
 寢夢,脫有異見,不覺身心立就沄爛。陛下常日捨財修福,臣私心顒顒,尚恨其少,豈可今日有見此事?一損福業,追悔便難。臣此啟聞,私心實切。若是大事,不可易改,亦願陛下照臣此誠,曲垂三思;況此嬉游之間,非關當否,而動輒傷生,實可深慎!



 臣聞子孝奉君,臣忠事主,莫不靈祇通感,徵祥證登。臣近段仰啟,賜希受戒,天心洞遠,誠未達勝善之途,而聖恩遲疑,尚未垂履曲降尊極,豈可今月復隨此事?



 臣不隱心,即實上啟。



 雖不盡納,而深見寵愛。



 又與文惠太子同好釋氏,甚相友悌。子良敬信尤篤,數於邸園營齋戒,大集朝臣眾僧,至於賦食行水,或躬親其事,世頗以為失宰相體。勸人為善,未嘗厭倦,以此終致盛名。



 尋代王儉領國子祭酒,辭不拜。八年,給三望車。九年,京邑大水,吳興偏劇,子良開倉賑救,貧病不能立者於第北立廨收養,給衣及藥。十年,領尚書令。尋為使持節、都
 督揚州諸軍事、揚州刺史,本官如故。尋解尚書令,加中書監。



 文惠太子薨,世祖檢行東宮,見太子服禦羽儀,多過制度,上大怒。以子良與太子善,不啟聞,頗加嫌責。



 世祖不豫,詔子良甲仗入延昌殿侍醫藥。子良啟進沙門於殿戶前誦經,世祖為感夢見優曇缽華。子良按佛經宣旨使御府以銅為華,插御床四角。日夜在殿內,太孫間日入參承。世祖暴漸,內外惶懼,百僚皆已變服,物議疑立子良,俄頃而蘇,問太孫所在,因召東宮器甲皆入。遺詔使子良輔政,高宗知尚書事。子良素仁厚,不樂世務,乃推高宗。詔云:「事無大小,悉與鸞參懷。」子良所志也。



 太孫少養於子良妃袁氏,甚著慈愛,既懼前不得立,自此深忌子良。太行出太極殿,子良居中書省,帝使虎賁中郎將潘敞領二百人仗屯太極西階防之。成服後,諸王皆出,子良乞停至山陵,不許。進位太傅,增班劍為三十人,本官如
 故。解侍中。隆昌元年,加殊禮,劍履上殿,入朝不趨,贊拜不名。進督南徐州。其年疾篤,謂左右曰:「門外應有異。」遣人視,見淮中魚萬數,皆浮出水上向城門。尋薨,時年三十五。



 帝常慮子良有異志,及薨,甚悅,詔給東園溫明秘器,斂以袞冕之服。東府施喪位,大鴻臚持節監護,太官朝夕送祭。又詔曰:「褒崇明德,前王令典,追遠尊親,沿情所隆。故使持節、都督揚州諸軍事、中書監、太傅、領司徒、揚州刺史、竟陵王、新除督南徐州,體睿履正,神鑒淵邈。道冠民宗,具瞻允集。肇自弱齡,孝友光備。爰及贊契,協升景業。燮曜台陛,五教克宣。敷奏朝端,百揆惟穆。寄重先顧,任均負圖。諒以齊暉《二南》,同規往哲。方憑保佑,永翼雍熙。天不憖遺,奄焉薨逝。哀慕抽割,震于厥心。今龜謀襲吉,先遠戒期。宜崇嘉制,式弘風烈。可追崇假黃鉞、侍中、都督中外諸軍事、太宰、領大將軍、揚州牧,綠綟綬,備九服錫命之禮。
 使持節、中書監、王如故。給九旒鸞輅,黃屋左纛,轀輬車,前後部羽葆鼓吹,挽歌二部,虎賁班劍百人,葬禮依晉安平王孚故事。」



 初,豫章王嶷葬金牛山,文惠太子葬夾石,子良臨送,望祖硎山,悲感嘆曰:「北瞻吾叔,前望吾兄,死而有知,請葬茲地。」既薨,遂葬焉。



 所著內外文筆數十卷,雖無文採,多是勸戒。建武中,故吏范雲上表為子良立碑,事不行。子昭胄嗣。



 昭胄字景胤。泛涉有父風。永明八年,自竟陵王世子為寧朔將軍、會稽太守。



 鬱林初,為右衛將軍,未拜,遷侍中,領右軍將軍。建武三年,復為侍中,領驍騎將軍,轉散騎常侍,太常。以封境邊虜,永元元年,改封巴陵王。



 先是王敬則事起,南康侯子恪在吳郡,高宗慮有同異,召諸王侯入宮,晉安王寶義及江陵公寶覽等住中書省,高、武諸孫住西省,敕人各兩左右自隨,過此依軍法,孩抱者乳母隨
 入。其夜太醫煮藥,都水辦數十具棺材,須三更當悉殺之。子恪奔歸,二更達建陽門刺啟。時刻已至,而帝眠不起,中書舍人沈徽孚與帝所親左右單景雋共謀少留其事。須臾帝覺,景雋啟子恪已至,驚問曰:「未邪?」景雋具以事答。明日悉遣王侯還第。建武以來,高、武王侯居常震怖,朝不保夕,至是尤甚。



 及陳顯達起事,王侯復入宮,昭胄懲往時之懼,與弟永新侯昭穎逃奔江西,變形為道人。崔慧景舉兵,昭胄兄弟出投之。慧景事敗,昭胄兄弟首出投臺軍主胡松,各以王侯還第。不自安,謀為身計。子良故防閣桑偃為梅蟲兒軍副,結前巴西太守蕭寅,謀立昭胄。昭胄許事克用寅為尚書左僕射、護軍將軍。以寅有部曲,大事皆委之。時胡松領軍在新亭,寅遣人說之云:「須昏人出,寅等便率兵奉昭胄入臺,閉城號令。昏人必還就將軍,將軍但閉壘不應,則三公不足得也。」松又許諾。會東昏新起芳樂
 苑,月許日不復出游,偃等議募健兒百餘人從萬春門入突取之,昭胄以為不可。偃同黨王山沙慮事久無成,以事告御刀徐僧重。寅遣人殺山沙於路,吏於麝郤中得其事迹,昭胄兄弟與同黨皆伏誅。



 昭穎官至寧朔將軍、彭城太守。梁王定京邑,追贈昭胄散騎常侍、撫軍將軍,昭穎黃門郎。梁受禪,降封昭胄子同監利侯。



 廬陵王子卿,字雲長,世祖第三子也。建元元年,封臨汝縣公,千五百戶。兄弟四人同封。世祖即位,為持節、都督郢州司州之義陽軍事、冠軍將軍、郢州刺史。



 永明元年,徙都督荊湘益寧梁南北秦七州、安西將軍、荊州刺史,持節如故。始興王鑒為益州,子卿解督。



 子卿在鎮,營造服飾,多違制度。上敕之曰:「吾前後有敕,非復一兩過,道諸王不得作乖體格服飾,汝何意都不憶吾敕邪?忽作瑇瑁乘具,何意?已成不須壞,可速送下。純銀乘具,乃復可爾,何以作鐙亦
 是銀?可即壞之。忽用金薄裹箭腳,何意?亦速壞去。凡諸服章,自今不啟吾知復專輒作者,後有所聞,當復得痛杖。」



 又曰:「汝比在都,讀學不就,年轉成長。吾日冀汝美,勿得敕如風過耳,使吾失氣。」



 五年,入為侍中、撫軍將軍,未拜,仍為中護軍,侍中如故。六年,遷秘書監,領右衛將軍,尋遷中軍將軍,侍中並如故。十年,進號車騎將軍。俄遷使持節、都督南豫豫司三州軍事、驃騎將軍、南豫州刺史,侍中如故。子卿之鎮,道中戲部伍為水軍,上聞之,大怒,殺其典簽。遣宜都王鏗代之。子卿還第,至崩,不與相見。



 鬱林即位,復為侍中、驃騎將軍。隆昌元年,轉衛將軍、開府儀同三司,置兵佐。鄱陽王鏘見害,以子卿代為司徒,領兵置佐。尋復見殺,時年二十七。



 魚復侯子響,字云音,世祖第四子也。豫章王嶷無子,養子響,後有子,表留為嫡。世祖即位,為輔國將軍、南彭城臨淮二郡太守,見諸
 王不致敬。子響勇力絕人,關弓四斛力,數在園池中帖騎馳走竹樹下,身無虧傷。既出繼,車服異諸王,每入朝,輒忿怒,拳打車壁。世祖知之,令車服與皇子同。



 永明三年,遷右衛將軍。仍出為使持節、都督豫州郢州之西陽司州之汝南二郡軍事、冠軍將軍、豫州刺史。明年,進號右將軍。進督南豫州之歷陽、淮南、潁川、汝陽四郡。入為散騎常侍,右衛將軍。六年,有司奏:「子響體自聖明,出繼宗國。



 大司馬臣嶷昔未有胤,所以因心鞠養。陛下弘天倫之愛,臣嶷深猶子之恩,遂乃繼體扶疏,世祚垂改,茅蔣奄蔚,冢嗣莫移。誠欣惇睦之風,實虧立嫡之教。臣等參議,子響宜還本。」乃封巴東郡王,遷中護軍,常侍如故。尋出為江州刺史,常侍如故。



 七年,遷使持節、都督荊湘雍梁寧南北秦七州軍事、鎮軍將軍、荊州刺史。子響少好武,在西豫時,自選帶仗左右六十人,皆有膽幹。至鎮,數在內齋殺牛置酒,與
 之聚樂。令內人私作錦袍絳襖,欲餉蠻交易器仗。長史劉寅等連名密啟,上敕精檢。寅等懼,欲秘之。子響聞臺使至,不見敕,召寅及司馬席恭穆、諮議參軍江愈、殷曇粲、中兵參軍周彥、典簽吳修之、王賢宗、魏景淵於琴臺下詰問之。寅等無言。



 修之曰:「既以降敕旨,政應方便答塞。」景淵曰:「故應先檢校。」子響大怒,執寅等於後堂殺之。以啟無江愈名,欲釋之,而用命者已加戮。



 上聞之怒,遣衛尉胡諧之、游擊將軍尹略、中書舍人茹法亮領齋仗數百人,檢捕群小,敕:「子響若束首自歸,可全其性命。」諧之等至江津,築城燕尾洲,遣傳詔石伯兒入城慰勞。子響曰:「我不作賊,長史等見負,今政當受殺人罪耳。」



 乃殺牛具酒饌,餉臺軍。而諧之等疑畏,執錄其吏。子響怒,遣所養數十人收集府州器仗,令二千人從靈溪西渡,克明旦與臺軍對陣南岸。子響自與百餘人袍騎,
 將萬鈞弩三四張,宿江堤上,明日,凶黨與臺軍戰,子響於堤上放弩,亡命王沖天等蒙楯陵城,臺軍大敗,尹略死之,官軍引退。上又遣丹陽尹蕭順之領兵繼至,子響部下恐懼,各逃散。



 子響乃白服降,賜死。時年二十二。臨死,啟上曰:「劉寅等入齋檢杖,具如前啟。臣罪既山海,分甘斧鉞。奉敕遣胡諧之、茹法亮賜重勞,其等至,竟無宣旨,便建旗入津,對城南岸築城守。臣累遣書信喚法亮渡,乞白服相見,其永不肯,群小懼怖,遂致攻戰,此臣之罪也。臣此月二十五日束身投軍,希還天闕,停宅一月,臣自取盡,可使齊代無殺子之譏,臣免逆父之謗。既不遂心,今便命盡,臨啟哽塞,知復何陳。」



 有司奏絕子響屬籍,削爵土,收付廷尉法獄治罪。賜為蛸氏。諸所連坐,別下考論。贈劉寅侍中,席恭穆輔國將軍、益州刺史,江愈、殷曇粲黃門郎,周彥驍騎將軍。寅字景蕤,高平人也。有文義而學不閑世務。席恭穆,安定焉氏
 人,關隴豪族。



 上憐子響死,後游華林園,見猿對跳子鳴嘯,上留目久之,因嗚咽流涕。豫章王嶷上表曰:「臣聞將而必戮,炳自《春秋》,罄于甸人,著於《經禮》,猶懷不忍之言,尚有如倫之痛。豈不事因法往,情以恩留。故庶人蛸子響,識懷靡樹,見淪不逞,肆憤一朝,取陷兇德,遂使跡鄰非孝,事近無君,身膏草野,未雲塞釁。



 但韔矢倒戈,歸罪司戮,即理原心,亦既迷而知返。釁骨不收,辜魂莫赦,撫事惟往,載傷心目。昔閔榮伏痍,愴動墳園;思荊就闢,側懷丘墓。皆兩臣釁結於明時,二主議加於盛世,積代用之為美,歷史不以云非。伏顧一下天矜,爰詔蛸氏,使得安兆末郊,旋窆餘麓,微列葦韔之容,薄申封樹之禮。豈伊窮骸被德,實且天下歸仁。臣屬忝皇枝,偏留友睦,以臣繼別未安,子響言承出命,提攜鞠養,俯見成人,雖輟胤蕃條,歸體璇萼,循執之念不移,傅訓之憐何已。敢冒宸嚴,布此悲乞。」



 上
 不許。先是貶為魚復侯。



 安陸王子敬,字雲端,世祖第五子也。初封應城縣公。永明二年,出為持節、監南兗兗徐青冀五州、北中郎將、南兗州刺史。四年,進號右軍。明年,徙都督荊湘梁雍南北秦六州軍事、平西將軍、荊州刺史,持節如故。尋進號安西將軍。七年,徵侍中,護軍將軍。十年,轉散騎常侍、撫軍將軍、丹陽尹。十一年,進車騎將軍。



 尋給鼓吹一部。隆昌元年,遷使持節、都督南兗兗徐青冀五州、征北大將軍、南兗州刺史。延興元年,加侍中。高宗除諸蕃王,遣中護軍王玄邈徵九江,王廣之襲殺子敬,時年二十三。



 晉安王子懋,字雲昌,世祖第七子也。初封江陵公。永明三年,為持節、都督南豫豫司三州、南中郎將、南豫州剌史。魚復侯子響為豫州,子懋解督。四年,進號征虜將軍。南豫新置,力役寡少,加子懋領
 宣城太守。明年,為監南兗兗徐青冀五州軍事、後將軍、南兗州刺史,持節如故。六年,徙監湘州、平南將軍、湘州刺史。明年,加持節、都督。八年,進號鎮南將軍。撰《春秋例苑》三十卷奏之,世祖嘉之,敕付秘閣。九年,親府州事。十年,入為侍中,領右衛將軍。十一年,遷散騎常侍,中書監。未拜,仍為使持節、都督雍梁南北秦四州郢州之竟陵司州之隨郡軍事、征北將軍、雍州刺史,給鼓吹一部。豫章王喪服未畢,上以邊州須威望,許得奏之。



 鬱林即位,即本號為大將軍。子懋見幼主新立,密懷自全之計,令作部造器杖。



 陳顯達時為征虜,屯襄陽,欲脅取以為將帥。顯達密啟,高宗征顯達還。隆昌元年,遷子懋為都督江州剌史,留西楚部曲助鎮襄陽,單將白直俠轂自隨。顯達入別,子懋謂曰:「朝廷令身單身而反,身是天王,豈可過爾輕率。今猶欲將二三千人自隨,公意何如?」顯達曰:「殿下若不留部
 曲,便是大違敕旨,其事不輕。且此間人亦難可收用。」子懋默然,顯達因辭出便發去,子懋計未立,還鎮尋陽。



 延興元年,加侍中。聞鄱陽、隨郡二王見殺,欲起兵赴難。母阮在都,遣書欲密迎上,阮報其兄於瑤之為計,瑤之馳告高宗。於是纂嚴,遣平西將軍王廣之南北討,使軍主裴叔業與瑤之先襲尋陽,聲云為郢州行司馬。子懋知之,遣三百人守盆城。叔業溯流直下,至夜回下襲盆城。城局參軍樂賁開門納之。子懋率府州兵力,先已具船於稽亭渚,聞叔業得盆城,乃據州自衛。子懋部曲多雍土人,皆踴躍願奮,叔業畏之,遣于瑤之說子懋曰:「今還都,必無過憂,政當作散官,不失富貴也。」



 子懋既不出兵攻叔業,眾情稍沮。中兵參軍于琳之,瑤之兄也,說子懋重賂叔業,子懋使琳之往。琳之因說叔業請取子懋。叔業遣軍主徐玄慶將四百人隨琳之入州城,僚佐皆奔散,琳之從二
 百人拔刃入齋。子懋罵曰:「小人何忍行此事!」琳之以袖鄣面,使人害之。時年二十三。



 初,子懋鎮雍,世祖敕以邊略曰:「吾比連得諸處啟,所說不異,虜必無敢送死理,然為其備,不可暫懈。今秋犬羊輩越逸者,其亡滅之徵。吾今亦行密纂集,須有分明指的,便當有大處分。今普敕鎮守,並部偶民丁,有事即便應接運,已敕更遣,想行有至者,汝共諸人量覓,可使人數往南陽舞陰諸要處參覘。糧食最為根本,更不憂人仗,常行視驛亭馬,不可有廢闕。並約語諸州,當其堺皆爾,不如法,即問事。」又曰:「吾敕荊、郢二鎮各作五千人陣,本擬應接彼耳。賊若送死者,更即呼取之。已敕子真,魚繼宗、殷公愍至鎮,可以公愍為城主,三千人配之便足。



 汝可好以階級在意,勿得人求,或超五三階級。及文章詩筆,乃是佳事,然世務彌為根本,可常憶之。汝所啟仗,此悉是吾左右御仗也,云何得用之。品格不可
 乖,吾自當優量覓送。」先是啟求所好書,上又曰:「知汝常以書讀在心,足為深欣也。」



 賜子懋杜預手所定《左傳》及《古今善言》。



 隨郡王子隆,字雲興,世祖第八子也。有文才。初封枝江公。永明三年,為輔國將軍、南琅邪彭城二郡太守。明年,遷江州刺史,未拜,唐宇之賊平,遷為持節、督會稽東陽新安臨海永嘉五郡、東中郎將、會稽太守。遷長兼中書令。



 子隆娶尚書令王儉女為妃,上以子隆能屬文,謂儉曰:「我家東阿也。」儉曰:「東阿重出,實為皇家蕃屏。」未及拜,仍遷中護軍,轉侍中、左衛將軍。八年,代魚復侯子響為使持節、都督荊雍梁寧南北秦六州、鎮西將軍、荊州刺史,給鼓吹一部。其年,始興王鑒罷益州,進號督益州。九年,親府、州事。十一年,晉安王子懋為雍州,子隆復解督。鬱林立,進號征西將軍。隆昌元年,為侍中、撫軍將軍,領兵置佐。延興元年,轉中軍大將軍,侍中如故。



 子隆年
 二十一,而體過充壯,常服蘆茹丸以自銷損。高宗輔政,謀害諸王,世祖諸子中,子隆最以才貌見憚,故與鄱陽王鏘同夜先見殺。文集行於世。



 建安王子真,字雲仙,世祖第九子也。永明四年,為輔國將軍、南瑯邪彭城二郡太守。遷持節、督南豫司二州軍事、冠軍將軍、南豫州刺史,領宣城太守。進號南中郎將。六年,以府州稍實,表解領郡。七年,進號右將軍,遷丹陽尹,將軍如故。轉左衛將軍。七年,遷中護軍,仍出為持節、都督郢司二州軍事、平西將軍、郢州刺史。鬱林立,進號安西將軍。隆昌元年,為散騎常侍、護軍將軍。延興元年,轉鎮軍將軍,領兵置佐,常侍如故。其年見殺,年十九。



 西陽王子明,字雲光,世祖第十子也。永明元年,封武昌王。三年,失國璽,改封西陽。六年,為持節、都督南兗兗徐青冀五州軍事、冠軍
 將軍、南兗州刺史。



 八年,進號征虜將軍。十年,進左將軍,仍為督會稽東陽臨海永嘉新安五郡軍事、會稽太守。將軍如故。子明風姿明凈,士女觀者,咸嗟歎之。



 鬱林初,進號平東將軍。隆昌元年,為右將軍、中書令。延興元年,遷侍中,領驍騎將軍,右軍如故。建武元年,轉撫軍將軍,領兵置佐。二年,誅蕭諶,誣子明及弟子罕、子貞與諶同謀,見害。年十七。



 南海王子罕、字雲華,世祖第十一子也。永明六年,為北中郎將、南琅邪彭城二郡太守。上初以白下地帶江山,徙琅邪郡自金城治之,子罕始鎮此城。十年,為持節、都督南兗兗徐青冀五州軍事、征虜將軍、南兗州刺史。鬱林即位,進號後將軍。隆昌元年,遷散騎常侍、右衛將軍。建武元年,轉護軍將軍。二年,見殺。年十七。



 巴陵王子倫,字云宗,世祖第十三子也。永明七年,為持節、都督南
 豫司二州軍事、南中郎將、南豫州刺史。十年,遷北中郎將、南瑯邪彭城二郡太守。鬱林即位,以南彭城祿力優厚,奪子倫與中書舍人綦母珍之,更以南蘭陵代之。隆昌元年,遷散騎常侍、左將軍。延興元年,遣中書舍人茹法亮殺子倫,子倫正衣冠出受詔,曰:「鳥之將死,其鳴也哀;人之將死,其言也善。先朝昔滅劉氏,今日之事,理數固然。君是身家舊人,今銜此使,當由事不獲已。」法亮不敢答而退。年十六。



 邵陵王子貞,字雲松,世祖第十四子也。永明十年,為東中郎將、吳郡太守。



 鬱林即位,進號征虜將軍,還為後將軍。建武二年,見誅。年十五。



 臨賀王子岳,字雲嶠,世祖第十六子也。永明七年封。高宗誅世祖諸子,唯子岳及弟六人在後,世呼為七王。朔望入朝,上還後宮,輒
 歎息曰:「我及司徒諸兒子皆不長,高、武子孫日長大。」永泰元年,上疾甚,絕而復蘇。於是誅子岳等。



 延興建武中,凡三誅諸王,每一行事,高宗輒先燒香火,嗚咽涕泣,眾以此輒知其夜當相殺戮也。子岳死時,年十四。



 西陽王子文,字雲儒,世祖第十七子也。永明七年,封蜀郡王。建武中,改封西陽王。永泰元年,見殺。年十四。



 衡陽王子峻,字云嵩,世祖第十八子也。永明七年,封廣漢郡王。建武中,改封。永泰元年,見殺。年十四。



 南康王子琳,字雲璋,世祖第十九子也。母荀氏,盛寵。子琳鐘愛。永明七年,封宣城王。明年,上改南康公褚蓁以封子琳。永泰元年,見殺。年十四。



 湘東王子建,字云立,世祖第二十一子也。母謝氏,無寵,世祖度為
 尼。高宗即位,使還母。子建,永泰元年見殺,年十三。



 南郡王子夏,字雲廣,世祖第二十三子也。上春秋高,子夏最幼,寵愛過諸子。



 初,世祖夢金翅鳥下殿庭,搏食小龍無數,乃飛上天。永泰元年,子夏誅。年七歲。



 史臣曰:民之勞逸,隨所遭遇,習以成性,有識斯同。帝王子弟,生長尊貴,薪禽之道未知,富厚之圖已極。齠年稚齒,養器深宮,習趨拜之儀,受文句之學,坐躡搢紳,傍絕交友,情偽之事,不經耳目,憂懼之道,未涉胸衿。雖卓爾天悟,自得懷抱,孤寡為識,所陋猶多。朝出閫閨,暮司方岳,帝子臨州,親民尚小。年序次第,宜屏皇家,防驕剪逸,積代恆典,平允之情,操捶貽慮。故輔以上佐,簡自帝心,勞舊左右,用為主帥。州國府第,先令後行,飲食游居,動應聞啟。端拱守祿,遵承法度,張弛之要,莫敢厝言。行事執其權,典簽掣其肘,茍利之
 義未申,專違之咎已及。處地雖重,行己莫由,威不在身,恩未接下,倉卒一朝,艱難總集,望其釋位扶危,不可得矣。路溫舒云:「秦有十失,其一尚存。」斯宋氏之餘風,在齊而彌弊也。



 贊曰:武十七王,文宣令望,愛才悅古,仁信溫良,宗英是寄,遺惠未忘。廬陵犯色,安陸括囊。晉安早悟,隨郡雕章。建賀湘海,二陵二陽,幼
 蕃盛寵,南郡南康。



\end{pinyinscope}