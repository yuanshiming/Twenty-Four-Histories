\article{卷三十四列傳第十五 虞玩之 劉休 沈沖 庾杲之 王諶}

\begin{pinyinscope}

 虞玩之,字茂瑤,會稽餘姚人也。祖宗,晉庫部郎。父玫,通直常侍。玩之少閑刀筆,汎涉書史,解褐東海王行參軍,烏程令。路太后外親朱仁彌犯罪,依法錄治。太后怨訴孝武,坐免官。泰始中,除晉熙國郎中令,尚書起部郎,通直郎。元徽中,為右丞。時太祖參政,與玩之書曰:「張華為度支尚書,事不徒然。今漕藏有闕,吾賢居右丞,已覺金粟
 可積也。」玩之上表陳府庫錢帛,器械役力,所懸轉多,興用漸廣,慮不支歲月。朝議優報之。遷安成王車騎錄事,轉少府。



 太祖鎮東府,朝野致敬,玩之猶躡屐造席。太祖取屐視之,訛黑斜銳,斷,以芒接之。問曰:「卿此屐已幾載?」玩之曰:「初釋褐拜征北行佐買之,著已二十年,貧土竟不辦易。」太祖善之,引為驃騎諮議參軍。霸府初開,賓客輻湊,太祖留意簡接,玩之與樂安任遐,俱以應對有席上之美,齊名見遇。遐字景遠,好學,有義行,兼與太祖素遊,褚淵、王儉並見親愛。官至光祿大夫,永元初卒。



 玩之遷驍騎將軍,黃門郎,領本部中正。上患民間欺巧,及即位,敕玩之與驍騎將軍傅堅意檢定簿籍。建元二年,詔朝臣曰:「黃籍,民之大紀,國之治端。自頃氓俗巧偽,為日已久,至乃竊注爵位,盜易年月,增損三狀,貿襲萬端。或戶存而文書已絕,或人在而反托死叛,停私而云隸役,身強而
 稱六疾。編戶齊家,少不如此。皆政之巨蠹,教之深疵。比年雖卻籍改書,終無得實。若約之以刑,則民偽已遠;若綏之以德,則勝殘未易。卿諸賢並深明治體,可各獻嘉謀,以振澆化。又臺坊訪募,此制不近,優刻素定,閑劇有常。宋元嘉以前,茲役恆滿,大明以後,樂補稍絕。或緣寇難頻起,軍蔭易多,民庶從利,投坊者寡。然國經未變,朝紀恆存,相揆而言,隆替何速!此急病之洪源,晷景之切患,以何科算,革斯弊邪?」



 玩之上表曰:「宋元嘉二十七年八條取人,孝建元年書籍,眾巧之所始也。元嘉中,故光祿大夫傅隆,年出七十,猶手自書籍,躬加隱校。隆何必有石建之慎,高柔之勤,蓋以世屬休明,服道修身故耳。今陛下日旰忘食,未明求衣,詔逮幽愚,謹陳妄說。古之共治天下,唯良二千石,今慾求治取正,其在勤明令長。凡受籍,縣不加檢合,但封送州,州檢得實,方卻歸縣。吏貪其賂,民肆其奸,奸
 彌深而卻彌多,賂愈厚而答愈緩。自泰始三年至元徽四年,揚州等九郡四號黃籍,共卻七萬一千餘戶。於今十一年矣,而所正者猶未四萬。神州奧區,尚或如此,江、湘諸部,倍不可念。愚謂宜以元嘉二十七年籍為正。民惰法既久,今建元元年書籍,宜更立明科,一聽首悔,迷而不反,依制必戮。使官長審自檢校,必令明洗,然後上州,永以為正。若有虛昧,州縣同咎。今戶口多少,不減元嘉,而板籍頓闕,弊亦有以。



 自孝建已來,入勛者眾,其中操干戈衛社稷者,三分殆無一焉。勳簿所領而詐注辭籍,浮遊世要,非官長所拘錄,復為不少。尋蘇峻平後,庾亮就溫嶠求勳簿,而嶠不與,以為陶侃所上,多非實錄。尋物之懷私,無世不有,宋末落紐,此巧尤多。



 又將位既眾,舉恤為祿,實潤甚微,而人領數萬,如此二條,天下合役之身,已據其太半矣。又有改注籍狀。詐入仕流,昔為人役者,今反役
 人。又生不長髮,便謂為道人,填街溢巷,是處皆然。或抱子並居,竟不編戶,遷徙去來,公違土斷。屬役無滿,流亡不歸,寧喪終身,疾病長臥。法令必行,自然競反。又四鎮戍將,有名寡實,隨才部曲,無辨勇懦,署位借給,巫媼比肩,彌山滿海,皆是私役。行貨求位,其塗甚易,募役卑劇,何為投補?坊吏之所以盡,百里之所以單也。今但使募制明信,滿復有期,民無逕路,則坊可立表而盈矣。為治不患無制,患在不行,不患不行,患在不久。」



 上省玩之表,納之。乃別置板籍官,置令史,限人一日得數巧,以防懈怠。於是貨賂因緣,籍注雖正,猶強推卻,以充程限。至世祖永明八年,謫巧者戍緣淮各十年,百姓怨望。世祖乃詔曰:「夫簡貴賤,辨尊卑者,莫不取信於黃籍。豈有假器濫榮,竊服非分。故所以澄革虛妄,式允舊章。然釁起前代,過非近失,既往之愆,不足追咎。自宋昇明以前,皆聽復注。其有謫役邊
 疆,各許還本。此後有犯,嚴加翦治。」



 玩之以久宦衰疾,上表告退,曰:「臣聞負重致遠,力窮則困,竭誠事君,智盡必傾,理固然也。四十仕進,七十懸車,壯則驅馳,老宜休息。臣生於晉,長於宋,老於齊,世歷三代,朝市再易。臣以宋元嘉二十八年為王府行佐,於茲三十年矣。自頃以來,衰耗漸篤。為性不懶惰,而倦怠頓來。耳目本聰明,而聾矒轉積。



 腳不支身,喘不緒氣。景刻不推,朝晝不保。大功兄弟,四十有二人,通塞壽夭,唯臣獨存。朝露末光,寧堪長久!且知足不辱,臣已足矣。稟命饑寒,不求富貴,銅山由命,臣何恨焉,久甘之矣。直道事人,不免縲紲,屬遇聖明,知其非罪,臣之幸厚矣。授命於道消之晨,效節於百揆之日,臣忠之效也。降慶於文明之初,荷澤於天飛之運,臣命之偶也。不謀巧宦而位至九卿,德慚李陵而忝居門下。堯舜無窮,臣亦通矣。年過六十,不為夭矣。榮期之三樂,東平之
 一善,臣俱盡之矣。經昏踐亂,涉艱履危,仰聖德以求全,憑賢輔以申節,未嘗厭屈於勳權,畏溺於狐鼠,臣立身之本,於斯不虧。在其壯也,當官不讓;及其衰矣,豪露靡因。伏願慈臨,賜臣骸骨。非為希高慕古,愛好泉林,特以丁運孤貧,養禮多闕,風樹之感,夙自纏心。庶天假其辰,得二三年間,掃守丘墓,以此歸全,始終之報遂矣。」上省玩之表,許之。



 玩之於人物好臧否。宋末,王儉舉員外郎孔襜使虜,玩之言論不相饒,襜、儉並恨之。至是玩之東歸,儉不出送,朝廷無祖餞者。玩之歸家起大宅,數年卒。其後員外郎孔瑄就儉求會稽五官,儉方盥,投皂莢於地,曰:「卿鄉俗惡。虞玩之至死煩人。」



 孔襜字世遠,玩之同郡人,好典故學。與王儉至交。昇明中為齊臺尚書儀曹郎,太祖謂之曰:「卿儀曹才也。」儉為宰相,襜嘗謀議帷幕,每及選用,頗失鄉曲情。



 儉從容啟上曰:「臣有孔襜,猶陛下之有臣也。」永
 明中為太子家令,卒。時人呼孔襜、何憲為王儉三公。



 憲字子思,廬江人也。以強學見知。母鎮北長史王敷之女,聰明有訓識。憲為本州別駕。永明十年,使於虜中。



 劉休,字弘明,沛郡相人也。祖徽,正員郎。父超,九真太守。休初為駙馬都尉,奉朝請,宋明帝湘東國常侍。好學諳憶,不為帝所知。襲祖封南鄉侯。友人陳郡謝儼同丞相義宣反,休坐匿之,被繫尚方七年,孝武崩,乃得出。隨弟欽為羅縣。



 泰始初,諸州反,休筮明帝當勝,靜處不預異謀。數年,還投吳喜為輔師府錄事參軍。喜稱其才,進之明帝,得在左右。板桂陽王征北參軍。



 帝頗有好尚,尤嗜飲食。休多藝能,爰及鼎味,問無不解。後宮孕者,帝使筮其男女,無不如占。帝素肥,痿不能御內,諸王妓妾懷孕,使密獻入宮,生子之後,閉其母於幽房,前後十數。順帝,桂陽王休範子也。蒼梧王亦非帝子,陳
 太妃先為李道兒妾,故蒼梧微行,嘗自稱為李郎焉。帝憎婦人妒,尚書右丞榮彥遠以善棋見親,婦妒傷其面,帝曰:「我為卿治之,何如?」彥遠率爾應曰:「聽聖旨。」其夕,遂賜藥殺其妻。休妻王氏亦妒,帝聞之,賜休妾,敕與王氏二十杖。令休於宅後開小店,使王氏親賣掃帚皂莢以辱之。其見親如此。



 尋除員外郎,領輔國司馬、中書通事舍人,帶南城令。除尚書中兵郎,給事中,舍人、令如故。除安成王撫軍參軍,出為都水使者,南康相。休善言治體,而在郡無異績。還為正員郎,邵陵王南中郎錄事、建威將軍、新蔡太守。隨轉左軍府,加鎮蠻護軍,將軍、太守如故。遷諮議,司馬,進寧朔將軍,鎮蠻護軍、太守如故。



 徙尋陽太守,將軍、司馬如故。後遷長史。沈攸之難,世祖挾晉熙邵陵二王軍府鎮盆城,休承奉軍費,事寧,仍遷邵陵王安南長史,除黃門郎,寧朔將軍,前軍長史,齊臺散騎常侍。



 建元初,為御史中丞。頃之,休啟曰:「臣自塵榮南憲,星晷交春,謬聞弱
 奏,劾無空月。豈唯不能使蕃邦斂手,豪右屏氣,乃遣聽已暴之辜,替網觸羅之鳥。而猶以此,里失鄉黨之和,朝絕比肩之顧,覆背騰其喉唇,武人厲其觜吻。怨之所聚,勢難久堪;議之所裁,孰懷其允?臣竊尋宋世載祀六十,歷職斯任者五十有三,校其年月,不過盈歲。於臣叨濫,宜請骸骨。」上曰:「卿職當國司,以威裁為本,而忽憚世誚。卿便應辭之事始,何可獲惰晚節邪?」



 宋末,上造指南車,以休有思理,使與王僧虔對共監試。元嘉世,羊欣受子敬正隸法,世共宗之,右軍之體微古,不復見貴。休始好此法,至今此體大行。四年,出為豫章內史,加冠軍將軍。卒,年五十四。



 沈沖,字景綽,吳興武康人也。祖宣,新安太守。父懷文,廣陵太守。沖解褐衛尉五官,轉揚州主簿。宋大明中,懷文有文名,沖亦涉獵文義。轉西陽王撫軍法曹參軍,尋舉秀才,還為撫軍正佐,兼記室。
 及懷文得罪被繫,沖兄弟行謝,情哀貌苦,見者傷之。柳元景欲救懷文,言於帝曰:「沈懷文三子塗炭不可見,願陛下速正其罪。」帝竟殺之。元景為之歎息。沖兄弟以此知名。



 泰始初,以母老家貧,啟明帝得為永興令。遷巴陵王主簿,除尚書殿中郎。元徽中,出為晉安王安西記室參軍,還為司徒主簿,山陰令,轉司徒錄事參軍。世祖為江州,沖為征虜長史、尋陽太守,甚見委遇。世祖還都,使沖行府、州事。遷領軍長史。建元初,轉驃騎諮議參軍,領錄事,未及到任,轉黃門郎,仍遷太子中庶子。世祖在東宮,待以恩舊。及即位,轉御史中丞,侍中。冠軍廬陵王子卿為郢州,以沖為長史、輔國將軍、江夏內史,行府、州事。隨府轉為安西長史、南郡內史,行荊州府事,將軍如故。永明四年,徵為五兵尚書。



 沖與兄淡、淵名譽有優劣,世號為「腰鼓兄弟。」淡、淵並歷御史中丞,兄弟三人皆為司直,晉、宋未有也。中
 丞案裁之職,被憲者多結怨。淵永明中彈吳興太守袁彖,建武中彖從弟昂為中丞,到官數日,奏彈淵子繢父在僦白憲車,免官禁錮。沖母孔氏在東,鄰家失火,疑為人所焚爇,大呼曰:「我三兒皆作御史中丞,與人豈有善者!」



 世祖方欲任沖,沖西下至南州而卒。時年五十一。上甚惜之。喪還,詔曰:「沖喪柩至止,惻愴良深。以其昔在南蕃,特兼憫悼。」車駕出臨沖喪,詔曰:「沖貞詳閑理,志局淹正。誠著蕃朝,績彰出守。不幸早世,朕甚悼之。」追贈太常,謚曰恭子。


庾杲之,字景行,新野人也。祖深之,雍州刺史。父粲,司空參軍。杲之少而貞立,學涉文義。起家奉朝請,巴陵王征西參軍。郢州舉秀才,除晉熙王鎮西外兵參軍,世祖征虜府功曹,尚書駕部郎。清貧自業,食唯有韭
 \gezhu{
  艸殂}
 、抃韭、生韭雜菜,或戲之曰:「誰謂庾郎貧,食鮭常有二十七種。」言三九也。仍為世祖撫軍中軍記室,遷員外散騎常
 侍,正員郎,遷中書郎,領荊、湘二州中正。轉尚書左丞,常侍、領中正如故。出為王儉衛軍長史,時人呼入儉府為芙蓉池。儉謂人曰:「昔袁公作衛軍,欲用我為長史,雖不獲就,要是意向如此。今亦應須如我輩人也。」



 乃用杲之。遷黃門郎,兼御史中丞,尋即正。



 杲之風範和潤,善音吐。世祖令對虜使,兼侍中。上每歎其風器之美,王儉在座,曰:「杲之為蟬冕所照,更生風采。陛下故當與其即真。」帝意未用也。永明中,諸王年少,不得妄與人接,敕杲之與濟陽江淹五日一詣諸王,使申遊好。尋又遷廬陵王中軍長史,遷尚書吏部郎,參大選事。轉太子右衛率,加通直常侍。



 九年,卒。臨終上表曰:「臣昨夜及旦,更增氣疾,自省綿痼,頃刻危殆,無容復臥。任居隆顯,玷塵明世,乞解所忝,待終私庭。臣以凡庸,謬徼昌運,獎擢之厚,千載難逢。且年踰知命,志事榮顯,脩夭有分,無所厝言。若天鑒微誠,暫借餘曆,傾
 宗殞元,陳力無遠。仰違庭闕,伏枕鯁戀。送貂蟬及章。」詔不許。杲之歷在上府,以文學見遇。上造崇虛館,使為碑文。卒時年五十一,上甚惜之。謚曰貞子。



 時會稽孔廣,字淹源,亦美姿制。歷州治中,卒。



 王諶,字仲和,東海郯人也。祖萬慶,員外常侍。父元閔,護軍司馬。宋大明中,沈曇慶為徐州,辟諶為迎主簿,又為州迎從事,湘東王國常侍,鎮北行參軍,州、國、府主皆宋明帝也。除義陽王征北行參軍,又除度明帝衛軍府。諶有學義,累為帝蕃佐。及即位,除司徒參軍,帶薛令,兼中書舍人,見親遇,常在左右。諶見帝所行慘僻,屢諫不從,請退,坐此見怒,系尚方,少日出。尋除尚書殿中郎,徙記室參軍,正員郎,薛令如故。遷兼中書郎,晉平王驃騎板諮議,出為湘東太守,秩中二千石,未拜,坐公事免。復為桂陽王驃騎府諮議參軍,中書郎。



 明帝好圍棋,置圍棋州邑,以建安王休仁為圍棋州都
 大中正,諶與太子右率沈勃、尚書水部郎庾珪之、彭城丞王抗四人為小中正,朝請褚思莊、傅楚之為清定訪問。



 出為臨川內史,還為尚書左丞。尋以本官領東觀祭酒,即明帝所置總明觀也。



 遷黃門,轉正員常侍,輔國將軍,江夏王右軍長史,冠軍將軍。轉給事中,廷尉卿,未拜。建元中,武陵王曄為會稽,以諶為征虜長史行事,冠軍如故。永明初,遷豫章王太尉司馬,將軍如故。世祖與諶相遇於宋明之世,欲委任,為輔國將軍、晉安王南中郎長史、淮南太守,行府、州事。五年,除黃門郎,領驍騎將軍,遷太子中庶子,驍騎如故。諶貞正和謹,朝廷稱為善人,多與之厚。八年,轉冠軍將軍、長沙王車騎長史,徙廬陵王中軍長史,將軍如故。西陽王子明在南兗州,長史沈憲去職,上復徙諶為征虜長史,行南兗府、州事,將軍如故。諶少貧,嘗自紡績,及通貴後,每為人說之,世稱其志達。九年,卒。年六十九。



 史臣曰:鶉居鷇飲,裁樹司牧,板籍之起,尚未分民,所以愛字之義深,納隍之意重也。季世以後,務盡民力,量財品賦,以自奉養。下窮而上不恤,世澆而事愈變。故有竊名簿閥,忍賊肌膚,生濫死乖,趨避繩網。積虛累謬,已數十年,欺蔽相容,官民共有,為國之道,良宜矯革。若令優役輕徭,則斯詐自弭;明糾群吏,則茲偽不行。空閱舊文,徒成民幸。是以崔琰之譏魏武,謝安之論京師。斷民之難,豈直遠在周世哉?



 贊曰:玩之止足,為論未光。劉休善筮,安臥南湘。沖獲時譽,杲信珪璋。諶惟舊序,並用興王。



\end{pinyinscope}