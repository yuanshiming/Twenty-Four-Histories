\article{卷三十七列傳第十八 到捴 劉悛 虞悰 胡諧之}

\begin{pinyinscope}

 到捴,字茂謙,彭城武原人也。祖彥之,宋驃騎將軍。父仲度,驃騎從事中郎。



 捴襲爵建昌公。起家為太學博士,除奉車都尉,試守延陵令,非所樂,去官。除新安王北中郎行參軍,坐公事免。除新安王撫軍參軍,未拜,新安王子鸞被殺,仍除長兼尚書左民郎中。明帝立,欲收物情,以捴功臣後,擢為太子洗馬。除王景文安南諮議參軍。



 捴資籍豪富,厚自奉養,宅宇山池京師第一,妓妾姿藝,皆窮上品。
 才調流贍,善納交游,庖廚豐腆,多致賓客。愛妓陳玉珠,明帝遣求,不與,逼奪之,捴頗怨望。帝令有司誣奏捴罪,付廷尉,將殺之。捴入獄,數宿須鬢皆白。免死,系尚方,奪封與弟賁。捴由是屏斥聲玩,更以貶素自立。帝除捴為羊希恭寧朔府參軍,徙劉韞輔國、王景文鎮南參軍,並辭疾不就。尋板假明威將軍,仍除桂陽王征南參軍,轉通直郎,解職。帝崩後,弟賁表讓封還捴,朝議許之。遷司徒左西屬,又不拜。



 居家累年。



 弟遁,元徽中為寧遠將軍、輔國長史、南海太守,在廣州。升明元年,沈攸之反,刺史陳顯達起兵以應朝廷,遁以猶預見殺。遁家人在都,從野夜歸,見兩三人持堊刷其家門,須臾滅,明日而遁死問至。捴遑懼,詣太祖謝,即板為世祖中軍諮議參軍。建元初,遷司徒右長史,出為永嘉太守,為黃門郎,解職。



 世祖即位,遷太子中庶子,不拜。又除長沙王中軍長史,司徒左長史。宋世,上
 數遊會捴家,同從明帝射雉郊野,渴倦,捴為得早青瓜,與上對剖食之。上懷其舊德,意眄良厚。至是一歲三遷。



 永明元年,加輔國將軍,轉御史中丞。車駕幸丹陽郡宴飲,捴恃舊,酒後狎侮同列,言笑過度,為左丞庾杲之所糾,贖論。三年,復為司徒左長史,轉左衛將軍。



 隨王子隆帶彭城郡,捴問訊,不脩民敬,為有司所舉,免官。久之,白衣兼御史中丞。轉臨川王驃騎長史,司徒左長史,遷五兵尚書,出為輔國將軍、廬陵王中軍長史。母憂去官,服未終,八年,卒,年五十八。



 弟賁,初為衛尉主簿,奉車都尉。升明初,為中書郎,太祖驃騎諮議。建元中,為征虜司馬。卒。



 賁弟坦,解褐本州西曹。昇明二年,亦為太祖驃騎參軍。歷豫章王鎮西驃騎二府諮議。坦美鬚髯。與世祖、豫章王有舊。坦仍隨府轉司空太尉參軍。出為晉安內史,還又為大司馬諮議,中書郎,卒。



 劉悛,字士操,彭城安上里人也。彭城劉同出楚元王,分為三里,以別宋氏帝族。祖穎之,汝南新蔡二郡太守。父勔,司空。



 劉延孫為南徐州,初辟悛從事。隨父勔徵竟陵王誕於廣陵,以功拜駙馬都尉。



 轉宗愨寧蠻府主簿,建安王司徒騎兵參軍。復隨父勔征殷琰於壽春,於橫塘、死虎累戰皆勝。歷遷員外郎,太尉司徒二府參軍,代世祖為尚書庫部郎。遷振武將軍、蜀郡太守,未之任,復從父勔征討,假寧朔將軍,拜鄱陽縣侯世子。轉桂陽王征北中兵參軍,與世祖同直殿內,為明帝所親待,由是與世祖款好。



 遷通直散騎侍郎,出為安遠護軍、武陵內史。郡南江古堤,久廢不緝。悛脩治未畢,而江水忽至,百姓棄役奔走,悛親率厲之,於是乃立。漢壽人邵榮興六世同爨,表其門閭。悛強濟有世調,善於流俗。蠻王田僮在山中,年垂百餘歲,南譙王義宣為荊州,僮出謁。至是又出謁悛。明帝崩,
 表奔赴,敕帶郡還都。吏民送者數千人,悛人人執手,系以涕泣,百姓感之,贈送甚厚。



 仍除散騎侍郎。桂陽難,加寧朔將軍,助守石頭。父勔於大桁戰死,悛時疾病,扶伏路次,號哭求勔屍。勔尸項後傷缺,悛割髮補之。持喪墓側,冬月不衣絮。



 太祖代勔為領軍,素與勔善,書譬悛曰:「承至性毀瘵,轉之危慮,深以酸怛。終哀全生,先王明軌,豈有去縑纊,撤溫席,以此悲號,得終其孝性邪?當深顧往旨,少自抑勉。」



 建平王景素反,太祖總眾軍出頓玄武湖。悛初免喪,太祖欲使領支軍,召見悛兄弟,皆羸削改貌,於是乃止。除中書郎,行宋南陽八王事,轉南陽王南中郎司馬、長沙內史,行湘州事。未發,霸業初建,悛先致誠節。沈攸之事起,加輔國將軍。



 世祖鎮盆城,上表西討,求悛自代。世祖既不行,悛除黃門郎,行吳郡事。尋轉晉熙王撫軍中軍二府長史,行揚州事。出為持節、督廣州、廣州刺史,將軍如故。
 襲爵鄱陽縣侯。世祖自尋陽還,遇悛於舟渚間,歡宴敘舊,停十餘日乃下。遣文惠太子及竟陵王子良攝衣履,脩父友之敬。



 太祖受禪,國除。進號冠軍將軍。平西記室參軍夏侯恭叔上書,以柳元景中興功臣,劉勔殞身王事,宜存封爵。詔曰:「與運隆替,自古有之,朝議已定,不容復厝意也。」初,蒼梧廢,太祖集議中華門,見悛,謂之曰:「君昨直耶?」悛答曰:「僕昨乃正直,而言急在外。」至是上謂悛曰:「功名之際,人所不忘。卿昔於中華門答我,何其欲謝世事?」悛曰:「臣世受宋恩,門荷齊眷,非常之勛,非臣所及。進不遠怨前代,退不孤負聖明,敢不以實仰答。」遷太子中庶子,領越騎校尉。



 時世祖在東宮,每幸悛坊,閑言至夕,賜屏風帷帳。世祖即位,改領前軍將軍,中庶子如故。征北竟陵王子良帶南兗州,以悛為長史,加冠軍將軍、廣陵太守。轉持節、都督司州諸軍事、司州刺史,將軍如故。悛父勔討殷
 琰,平壽陽,無所犯害,百姓德之,為立碑祀。悛步道從壽陽之鎮,過勔碑,拜敬泣涕。初,義陽人夏伯宜殺剛陵戍主叛渡淮,虜以為義陽太守。悛設計購誘之,虜囗州刺史謝景殺伯宜兄弟、北襄城太守李榮公歸降。悛於州治下立學校,得古禮器銅罍、銅甑、山罍樽、銅豆鐘各二口,獻之。



 遷長兼侍中。車駕數幸悛宅。宅盛治山池,造甕牖。世祖著鹿皮冠,被悛菟皮衾,於牖中宴樂,以冠賜悛,至夜乃去。後悛從駕登蔣山,上數歎曰:「貧賤之交不可忘,糟糠之妻不下堂。」顧謂悛曰:「此況卿也。世言富貴好改其素情,吾雖有四海,今日與卿盡布衣之適。」悛起拜謝。遷冠軍將軍,司徒左長史。尋以本官行北兗州緣淮諸軍事。徙始興王前軍長史、平蠻校尉、蜀郡太守,將軍如故,行益州府、州事。郡尋改為內史。隨府轉安西。悛治事嚴辦,以是會旨。



 宋代太祖輔政,有意欲鑄錢,以禪讓之際,未及施行。
 建元四年,奉朝請孔覬上《鑄錢均貨議》,辭證甚博。其略以為「食貨相通,理勢自然。李悝曰:糴甚貴傷民,甚賤傷農。』民傷則離散,農傷則國貧。甚賤與甚貴,其傷一也。三吳國之關閫,比歲被水潦而糴不貴,是天下錢少。非穀穰賤,此不可不察也。鑄錢之弊,在輕重屢變。重錢患難用,而難用為累輕;輕錢弊盜鑄,而盜鑄為禍深。民所盜鑄,嚴法不禁者,由上鑄錢惜銅愛工也。惜銅愛工者,謂錢無用之器,以通交易,務欲令輕而數多,使省工而易成,不詳慮其為患也。自漢鑄五銖錢,至宋文帝,歷五百餘年,制度世有廢興,而不變五銖錢者,明其輕重可法,得貨之宜。以為宜開置泉府,方牧貢金,大興熔鑄。錢重五銖,一依漢法。府庫已實,國用有儲,乃量奉祿,薄賦稅,則家給民足。頃盜鑄新錢者,皆效作翦鑿,不鑄大錢也。摩澤淄染,始皆類故;交易之後,渝變還新。良民弗皆淄染,不復行矣。所
 鬻賣者,皆徒失其物。



 盜鑄者,復賤買新錢,淄染更用,反覆生詐,循環起奸,此明主尤所宜禁而不可長也。若官鑄已布於民,便嚴斷翦鑿:小輕破缺無周郭者,悉不得行;官錢細小者,稱合銖兩,銷以為大。利貧良之民,塞奸巧之路。錢貨既均,遠近若一,百姓樂業,市道無爭,衣食滋殖矣。」時議者多以錢貨轉少,宜更廣鑄,重其銖兩,以防民姦。



 太祖使諸州郡大市銅,會晏駕,事寢。



 永明八年,悛啟世祖曰:「南廣郡界蒙山下,有城名蒙城,可二頃地,有燒爐四所,高一丈,廣一丈五尺。從蒙城渡水南百許步,平地掘土深二尺,得銅。又有古掘銅坑,深二丈,並居宅處猶存。鄧通,南安人,漢文帝賜嚴道縣銅山鑄錢,今蒙山近青衣水南,青衣左側並是故秦之嚴道地。青衣縣又改名漢嘉。且蒙山去南安二百里,案此必是通所鑄。近喚蒙山獠出,云『甚可經略』。此議若立,潤利無極。」



 并獻蒙山銅一片,
 又銅石一片,平州鐵刀一口。上從之,遣使入蜀鑄錢,得千餘萬,功費多,乃止。



 悛仍代始興王鑑為持節、監益寧二州諸軍事、益州刺史,將軍如故。悛既藉舊恩,尤能悅附人主,承迎權貴。賓客閨房,供費奢廣。罷廣、司二州,傾資貢獻,家無留儲。在蜀作金浴盆,餘金物稱是。罷任,以本號還都,欲獻之,而世祖晏駕。



 鬱林新立,悛奉獻減少,鬱林知之,諷有司收悛付廷尉,將加誅戮。高宗啟救之,見原,禁錮終身。雖見廢黜,而賓客日至。悛婦弟王法顯同宋桂陽事,遂啟別居,終身不復見之。



 海陵王即位,以白衣除兼左民尚書,尋除正。高宗立,加領驍騎將軍,復故官,駙馬都尉。建武二年,虜主侵壽陽,詔悛以本官假節出鎮漅湖,遷散騎常侍、右衛將軍。虜寇既盛,悛又以本官出屯新亭。



 悛歷朝皆見恩遇。太祖為鄱陽王鏘納悛妹為妃,高宗又為晉安王寶義納悛
 女為妃,自此連姻帝室。王敬則反,悛出守瑯邪城,轉五兵尚書,領太子左衛率。未拜,明帝崩,東昏即位,改授散騎常侍,領驍騎將軍,尚書如故。衛送山陵,卒,年六十一。贈太常,常侍、都尉如故。謚曰敬。



 虞悰,字景豫,會稽餘姚人也。祖嘯父,晉左民尚書。父秀之,黃門郎。悰少而謹敕,有至性。秀之於都亡,悰東出奔喪,水漿不入口。州辟主簿,建平王參軍,尚書儀曹郎,太子洗馬,領軍長史,正員郎,累至州治中,別駕,黃門郎。



 初,世祖始從官,家尚貧薄。悰推國士之眷,數相分與;每行,必呼上同載。



 上甚德之。升明中,世祖為中軍,引悰為諮議參軍,遣吏部郎江謐持手書謂悰曰:「今因江吏郎有白,以君情顧,意欲相屈。」建元初,轉太子中庶子,遷後軍長史,領步兵校尉,鎮北長史、寧朔將軍、南東海太守。尋為豫章內史,將軍如故。悰治家富殖,奴婢無游手,雖在南土,而會稽
 海味無不畢致焉。遷輔國將軍、始興王長史、平蠻校尉、蜀郡太守。轉司徒司馬,將軍如故。



 悰善為滋味,和齊皆有方法。豫章王嶷盛饌享賓,謂悰曰:「今日肴羞,寧有所遺不?」悰曰:「恨無黃頷,何曾《食疏》所載也。」



 遷散騎常侍,太子右率。永明八年,大水,百官戎服救太廟,悰朱衣乘車鹵簿,於宣陽門外行馬內驅打人,為有司所奏,見原。



 上以悰布衣之舊,從容謂悰曰:「我當令卿復祖業。」轉侍中,朝廷咸驚其美拜。遷祠部尚書。世祖幸芳林園,就悰求扁米粣。悰獻粣及雜肴數十輿,太官鼎味不及也。上就悰求諸飲食方,悰秘不肯出。上醉後體不快,悰乃獻醒酒鯖鮓一方而已。出為冠軍將軍,車騎長史,轉度支尚書,領步兵校尉。



 鬱林立,改領右軍將軍,揚州大中正,兼大匠卿。起休安陵,於陵所受局下牛酒,坐免官。隆昌元年,以白衣領職。鬱林廢,悰竊嘆曰:「王、徐遂縛褲廢天子,天下豈有此理邪?」延
 興元年,復領右軍。明帝立,悰稱疾不陪位。帝使尚書令王晏齎廢立事示悰,以悰舊人,引參佐命。悰謂晏曰:「主上聖明,公卿戮力,寧假朽老以匡贊惟新乎?不敢聞命。」朝議欲糾之,僕射徐孝嗣曰:「此亦古之遺直。」



 眾議乃止。



 悰稱疾篤還東,上表曰:「臣族陋海區,身微稽土,猥屬興運,荷竊稠私,徒越星紀,終慚報答。衛養乖方,抱疾嬰固,寢瘵以來,倏踰旬朔,頻加醫治,曾未瘳損。惟此朽頓,理難振復,乞解所職,盡療餘辰。」詔賜假百日。轉給事中,光祿大夫,尋加正員常侍。永元元年,卒。時年六十五。



 悰性敦實,與人知識,必相存訪,親疏皆有終始,世以此稱之。



 從弟袤,矢志不仕。王敬則反,取袤監會稽郡,而軍事悉付寒人張靈寶,郡人攻郡殺靈寶,袤以不豫事得全。



 胡諧之,豫章南昌人也。祖廉之,治書侍御史。父翼之,州辟不就。諧之初辟州從事主簿,臨賀王國常侍,員外郎,撫軍行參軍,晉熙王
 安西中兵參軍,南梁郡太守。以器局見稱。徙邵陵王南中郎中兵,領汝南太守,不拜。除射聲校尉,州別駕。除左軍將軍,不拜,仍除邵陵王左軍諮議。



 世祖頓盆城,使諧之守尋陽城,及為江州,復以諧之為別駕,委以事任。文惠太子鎮襄陽,世祖以諧之心腹,出為北中郎征虜司馬、扶風太守,爵關內侯。在鎮毗贊,甚有心力。建元二年,還為給事中,驍騎將軍,本州中正,轉黃門郎,領羽林監。永明元年,轉守衛尉,中正如故。明年,加給事中。三年,遷散騎常侍,太子右率。五年,遷左衛將軍,加給事中,中正如故。



 諧之風形瑰潤,善自居處,兼以舊恩見遇,朝士多與交遊。六年,遷都官尚書。



 上欲遷諧之,嘗從容謂諧之曰:「江州有幾侍中邪?」諧之答曰:「近世唯有程道惠一人而已。」上曰:「當令有二。」後以語尚書令王儉,儉意更異,乃以為太子中庶子,領左衛率。



 諧之兄謨之亡,諧之上表曰:「臣私門罪釁,
 早備荼苦。兄弟三人,共相撫鞠,嬰孩抱疾,得及成人。長兄臣諶之,復早殞沒,與亡第二兄臣謨之銜戚家庭,得蒙訓長,情同極蔭。何圖一旦奄見棄放,吉兇分違,不獲臨奉,乞解所職。」詔不許。



 改衛尉,中庶子如故。



 八年,上遣諧之率禁兵討巴東王子響於江陵,兼長史行事。臺軍為子響所敗,有司奏免官,權行軍事如故。復為衛尉,領中庶子,本州中正。諧之有識計,每朝廷官缺及應遷代,密量上所用人,皆如其言,虞悰以此稱服之。十年,轉度支尚書,領衛尉。明年,卒,年五十一。贈右將軍、豫州刺史。謚曰肅。



 史臣曰:送錢贏兩,言此無忘。一笥之懷,報以都尉。千金可失,貴在人心。



 夫謹而信,泛愛眾,其為利也博矣。況乎先覺潛龍,結厚於布素?隨才致位,理固然也。



 贊曰:到藉豪華,晚懷虛素。虞生富厚,侈不違度。劉實朝交,胡乃蕃
 故,頡頏亮採,康衢騁步。



\end{pinyinscope}