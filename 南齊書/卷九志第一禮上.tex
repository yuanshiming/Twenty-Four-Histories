\article{卷九志第一禮上}

\begin{pinyinscope}

 禮儀繁博,與天地而為量。紀國立君,人倫攸始。三代遺文,略在經誥,蓋秦餘所亡逸也。漢初叔孫通制漢禮,而班固之志不載。及至東京,太尉胡廣撰《舊儀》,左中郎蔡邕造《獨斷》,應劭、蔡質咸綴識時事,而司馬彪之書不取。魏氏籍漢末大亂,舊章殄滅,侍中王粲、尚書衛覬集創朝儀,而魚豢、王沈、陳壽、孫盛並未詳也。吳則太史令丁孚拾遺漢事,蜀則孟光、許慈草建眾典。晉初司空荀摐因魏代前事,撰為《晉禮》,參考今古,更其節文,羊祜、任愷、庾峻、應貞並共刪集,成百六十五篇。後摯虞、傅咸纘續此製,未及成功,中原覆沒,今虞之《決疑注》是遺事也。江左僕射刁協、太常荀崧,補緝舊文,光祿
 大夫蔡謨又踵修輯朝故。宋初因循改革,事系群儒,其前史所詳,並不重述。永明二年,太子步兵校尉伏曼容表定禮樂。於是詔尚書令王儉制定新禮,立治禮樂學士及職局,置舊學四人,新學六人,正書令史各一人,乾一人,秘書省差能書弟子二人。因集前代,撰治五禮,吉、兇、賓、軍、嘉也。文多不載。若郊廟庠序之儀,冠婚喪紀之節,事有變革,宜錄時事者,備今志。其輿輅旗常,與往代同異者,更立別篇。



 建元元年七月,有司奏:「郊殷之禮,未詳郊在何年?復以何祖配郊?殷復在何時?未郊得先殷與不?明堂亦應與郊同年而祭不?若應祭者,復有配與無配?不祀者,堂殿職僚毀置云何?」八座丞郎通關博士議。曹郎中裴昭明、儀曹郎中孔逷議:「今年七月宜殷祠,來年正月宜南郊明堂,並祭而無配。」殿中郎司馬憲議:「南郊無配,饗祠如舊;明堂無配,宜應廢祀。其殷祠同用今年十月。」



 右僕射王儉議:「
 案《禮記·王制》,天子先祫後時祭,諸侯先時祭後祫。



 《春秋》魯僖二年祫,明年春禘,自此以後,五年再殷。《禮緯·稽命徵》曰:『三年一祫,五年一禘。』《經》《記》所論禘祫與時祭,其言詳矣,初不以先殷後郊為嫌。至於郊配之重,事由王迹,是故杜林議云『漢業特起,不因緣堯,宜以高帝配天』。魏高堂隆議以舜配天。蔣濟云『漢時奏議,謂堯已禪舜,不得為漢祖,舜亦已禪禹,不得為魏之祖。今宜以武皇帝配天』。晉、宋因循,即為前式。又案《禮》及《孝經援神契》並云:『明堂有五室。天子每月於其室聽朔布教,祭五帝之神,配以有功德之君。』《大戴禮記》曰:『明堂者,所以明諸侯尊卑也』。許慎《五經異義》曰:『布政之宮,故稱明堂。明堂,盛貌也。』《周官·匠人職》稱明堂有五室。鄭玄云:『周人明堂五室,帝一室也。』初不聞有文王之寢。《鄭志》趙商問云:『說者謂天子廟制如明堂,是為明堂即文廟邪?』鄭答曰:『明堂主祭上帝,以文王
 配耳,猶如郊天以後稷配也。』袁孝尼云:『明堂法天之宮,本祭天帝,而以文王配,配其父於天位則可,牽天帝而就人鬼,則非義也。』太元十三年,孫耆之議,稱『郊以祀天,故配之以后稷;明堂以祀帝,故配之以文王。由斯言之,郊為皇天之位,明堂即上帝之廟』。徐邈謂『配之為言,必有神主;郊為天壇,則堂非文廟』。《史記》云趙綰、王臧欲立明堂,于時亦未有郊配。漢又祀汾陰五畤,即是五帝之祭,亦未有郊配。『議者或謂南郊之日,已旅上帝,若又以無配而特祀明堂,則一日再祭,於義為黷。案,古者郊本不共日。蔡邕《獨斷》曰:『祠南郊。祀畢,次北郊,又次明堂、高廟、世祖廟,謂之五供。』馬融云:『郊天之祀,咸以夏正,五氣用事,有休有王,各以其時,兆於方郊,四時合歲,功作相成,亦以此月總旅明堂。』是則南郊、明堂各日之證也。近代從省,故與郊同日,猶無煩黷之疑。何者?其為祭雖同,所以致祭則異。
 孔晁云,言五帝佐天化育,故有從祀之禮,旅上帝是也。至於四郊明堂,則是本祀之所,譬猶功臣從饗,豈復廢其私廟?且明堂有配之時,南郊亦旅上帝,此則不疑於共日,今何故致嫌於同辰?



 又《禮記》『天子祭天地、四方、山川、五祀,歲遍』。《尚書·洛誥》『咸秩無文』。《詩》云『昭事上帝,聿懷多福』。據此諸義,則四方、山川,猶必享祀,五帝大神,義不可略。魏文帝黃初二年正月,郊天地明堂,明帝太和元年正月,以武皇帝配天,文皇帝配上帝,然則黃初中南郊、明堂,皆無配也。又郊日及牲色,異議紛然。《郊特牲》云:『郊之用辛,周之始郊也。』盧植云『辛之為言自新絜也。』鄭玄雲:『用辛日者,為人當齋戒自新絜也』。漢魏以來,或丁或己,而用辛常多。考之典據,辛日為允。《郊特牲》又云,郊牲幣宜以正色。繆襲據《祭法》,云天地絺犢,周家所尚;魏以建丑為正,牲宜尚白。《白虎通》云,三王祭天,一用夏正,所以然者,夏正
 得天之數也。魏用異朔,故牲色不同。今大齊受命,建寅創曆,郊廟用牲,一依晉、宋。謂宜以今年十月殷祀宗廟。自此以後,五年再殷。



 來年正月上辛,有事南郊。宜以共日,還祭明堂。又用次辛,饗祀北郊。而並無配。



 犧牲之色,率由舊章。」



 詔:「可。明堂可更詳」。



 有司又奏:「明堂尋禮無明文,唯以《孝經》為正。竊尋設祀之意,蓋為文王有配則祭,無配則止。愚謂既配上帝,則以帝為主。今雖無配,不應闕祀。徐邈近代碩儒,每所折衷,其云『郊為天壇,則堂非文廟』,此實明據。內外百司立議已定,如更詢訪,終無異說。傍儒依史,竭其管見。既聖旨惟疑,群下所未敢詳,廢置之宜,仰由天鑒。」詔「依舊」。



 建元四年,世祖即位。其秋,有司奏:「尋前代嗣位,或仍前郊年,或別更始,晉、宋以來,未有畫一。今年正月已郊,未審明年應南北二郊祀明堂與不?」依舊通關八座丞郎博士議。尚書令王儉議:「案秦為諸
 侯,雜祀諸畤,始皇並天下,未有定祠。漢高受命,因雍四畤而起北畤,始畤五帝,未定郊丘。文帝六年,新垣平議初起渭陽五帝廟。武帝初至雍郊見五畤,後常三歲一郊祠雍。元鼎四年,始立后土祠於汾陰,明年,立太一祠於甘泉,自是以後,二歲一郊,與雍更祠。成帝初即位,丞相匡衡於長安定南北郊。哀、平之際,又復甘泉、汾陰祠。平帝元始五年,王莽奏依匡衡議還復長安南北二郊。光武建武二年,定郊祀兆於洛陽。魏、晉因循,率由漢典,雖時或參差,而類多間歲。至於嗣位之君,參差不一,宜有定制。檢晉明帝太寧三年南郊,其年九月崩,成帝即位,明年改元即郊;簡文咸安二年南郊,其年七月崩,孝武即位,明年改元亦郊;宋元嘉三十年正月南郊,其年二月崩,孝武嗣位,明年改元亦郊。此則二代明例,差可依放。謂明年正月宜饗祀二郊,虞祭明堂,自茲厥後,依舊間歲。」尚
 書領國子祭酒張緒等十七人並同儉議。詔「可」。



 永明元年當南郊,而立春在郊後,世祖欲遷郊。尚書令王儉啟:「案《禮記·郊特牲》云:『郊之祭也,迎長日之至也,大報天而主日也。』《易說》『三王之郊,一用夏正』。盧植云:『夏正在冬至後,《傳》曰啟蟄而郊,此之謂也。』然則圜丘與郊各自行,不相害也。鄭玄云:『建寅之月,晝夜分而日長矣。』王肅曰:『周以冬祭天於圜丘,以正月又祭天以祈穀。』《祭法》稱『燔柴太壇』,則圜丘也。《春秋傳》云『啟蟄而郊,則祈穀也。謹尋《禮》、《傳》二文,各有其義,盧、王兩說,有若合符。中朝省二丘以並二郊,即今之郊禮,義在報天,事兼祈穀,既不全以祈農,何必俟夫啟蟄?史官唯見《傳》義,未達《禮》旨。又尋景平元年正月三日辛丑南郊,其月十一日立春;元嘉十六年正月六日辛未南郊,其月八日立春。此復是近世明例,不以先郊後春為嫌。若或以元日合朔為礙者,則
 晉成帝咸康元年正月一日加元服,二日親祠南郊。元服之重,百僚備列,雖在致齋,行之不疑。



 今齋內合朔,此即前準。若聖心過恭,寧在嚴潔,合朔之日,散官備防,非預齋之限者,於止車門外別立幔省,若日色有異,則列於省前。望實為允,謂無煩遷日。」



 從之。



 永明二年,祠部郎中蔡履議:「郊與明堂,本宜異日。漢東京《禮儀志》『南郊禮畢,次北郊、明堂、高廟、世祖廟,謂之五供』。蔡邕所據亦然。近世存省,故郊堂共日。來年郊祭,宜有定准。」



 太學博士王祐議:「來年正月上辛,宜祭南郊,次辛,有事明堂,後辛,饗祀北郊。」



 兼博士劉蔓議:「漢元鼎五年,以辛巳行事,自後郊日,略無違異。元封元年四月癸卯,登封泰山,坐明堂。五年甲子,以高祖配。漢家郊祀,非盡天子之縣,故祠祭之月,事有不同。後漢永平以來,明堂兆於國南,而郊以上丁,故供修三祀,得并在初月。雖郊有常日,明堂猶無定辰。何
 則?郊丁社甲,有說則從,經禮無文,難以意造,是以必算良辰,而不祭寅丑。且禮之奠祭,無同共者,唯漢以朝日合於報天爾。若依《漢書》五供,便應先祭北郊,然後明堂。則是地先天食,所未可也。」



 兼太常丞蔡仲熊議:「《鄭志》云『正月上辛,祀后稷於南郊,還於明堂,以文王配。』故宋氏創立明堂,郊還即祭,是用《鄭志》之說也。蓋為《志》者失,非玄意也。玄之言曰:『未審周明堂以何月,於《月令》則以季秋。』案玄注《月令》『季秋大饗帝』云『大饗,遍祭五帝』。又云『大饗於明堂,以文武配』。其時秋也,去啟蟄遠矣。又《周禮·大司樂》『凡大祭祀,宿縣』。尋宿縣之旨,以日出行事故也;若日暗而後行事,則無假預縣。果日出行事,何得方俟郊還?東京《禮儀志》不記祭之時日,而《志》云:『天郊夕牲之夜,夜漏未盡八刻進熟;明堂夕牲之夜,夜漏未盡七刻進熟。』尋明堂之在郊前一刻,而進獻奏樂,方待郊還。



 魏高堂隆表『九日南
 郊,十日北郊,十一日明堂,十二日宗廟』。案隆此言,是審於時定制,是則《周禮》、二漢及魏,皆不共日矣。《禮》以辛郊,《書》以丁祀,辛丁皆合,宜臨時詳擇。」



 太尉從事中郎顧憲之議:「《春秋傳》以正月上辛郊祀,《禮記》亦云郊之用辛,《尚書》獨云丁巳用牲于郊。先儒以為先甲三日辛,後甲三日丁,可以接事天神之日。後漢永平二年正月辛未,宗祀光武皇帝於明堂。辛既是常郊之日,郊又在明堂之前,無容不郊而堂,則理應郊堂。」



 司徒西閣祭酒梁王議:「《孝經》鄭玄注云『上帝亦天別名』。如鄭旨,帝與天亦言不殊。近代同辰,良亦有據。魏太和元年正月丁未,郊祀武皇帝以配天,宗祀文皇帝於明堂以配上帝,此則已行之前準。」



 驍騎將軍江淹議:「郊旅上天,堂祀五帝,非為一日再黷之謂,無俟厘革。」



 尚書陸澄議:「遺文餘事,存乎舊書,郊宗地近,勢可共日。不共者,義在必異也。元始五年正月六日辛未,
 郊高皇帝以配天,二十二日丁亥,宗祀孝文於明堂配上帝。永平二年正月辛未,宗祀五帝於明堂,光武皇帝配。章帝元和二年,巡狩岱宗,柴祭,翌日,祠五帝於明堂。柴山祠地,尚不共日,郊堂宜異,於例益明。



 陳忠《奏事》云『延光三年正月十三日南郊,十四日北郊,十五日明堂,十六日宗廟,十七日世祖廟』。仲遠五祀,紹統五供,與忠此奏,皆為相符。高堂隆表,二郊及明堂宗廟各一日,摯虞《新禮》議明堂南郊間三兆,禋天饗帝共日之證也。又上帝非天,昔人言之已詳。今明堂用日,宜依古在北郊後。漢唯南郊備大駕,自北郊以下,車駕十省其二。今祀明堂,不應大駕。」



 尚書令王儉議:「前漢各日,後漢亦不共辰,魏、晉故事,不辨同異,宋立明堂,唯據自郊徂宮之義,未達祀天旅帝之旨。何者?郊壇旅天,甫自詰朝,還祀明堂,便在日昃,雖致祭有由,而煩黷斯甚,異日之議,於理為弘。《春秋感精
 符》云『王者父天母地』,則北郊之祀,應在明堂之先。漢、魏北郊,亦皆親奉,晉泰寧有詔,未及遵遂。咸和八年,甫得營繕,太常顧和秉議親奉。康皇之世,已經遵用。宋氏因循,未遑厘革。今宜親祠北郊,明年正月上辛祠昊天,次辛瘞后土,後辛祀明堂,御並親奉。車服之儀,率遵漢制。南郊大駕,北郊、明堂降為法駕。袞冕之服,諸祠咸用。」詔「可」。



 建武二年,通直散騎常侍庾曇隆啟:「伏見南郊壇員兆外內,永明中起瓦屋,形制宏壯。檢案經史,無所準據。尋《周禮》,祭天於圜丘,取其因高之義,兆於南郊,就陽位也。故以高敞,貴在上昭天明,旁流氣物。自秦、漢以來,雖郊祀參差,而壇域中間,並無更立宮室。其意何也?政是質誠尊天,不自崇樹,兼事通曠,必務開遠。宋元嘉南郊,至時權作小陳帳以為退息,泰始薄加脩廣,永明初彌漸高麗,往
 年工匠遂啟立瓦屋。前代帝皇,豈於上天之祀而昧營構,所不為者,深有情意。《記》稱『掃地而祭,於其質也,器用陶匏,天地之性也』。故『至敬無文』,『以素為貴』。竊謂郊事宜擬休偃,不俟高大,以明謙恭肅敬之旨。庶或仰允太靈,俯愜群望。」詔「付外詳」。



 國子助教徐景嵩議:「伏尋《三禮》,天地兩祀,南北二郊,但明祭取犧牲,器用陶匏,不載人君偃處之儀。今棟瓦之構雖殊,俱非千載成例,宜務因循。」太學博士賀翽議:「《周禮》『王旅上帝,張氈案,設皇邸』。國有故而祭,亦曰旅。



 氈案,以氈為床於幄中,不聞郊所置宮宇。」兼左丞王摛議,掃地而祭於郊,謂無築室之議。並同曇隆。



 驍騎將軍虞炎議,以為「誠愨所施,止在一壇。漢之郊祀,饗帝甘泉,天子自竹宮望拜,息殿去壇場既遠,郊奉禮畢,旋幸於此。瓦殿之與帷宮,謂無簡格」。



 祠部郎李捴議:「《周禮》『凡祭祀張其旅幕,張尸次』。尸則有幄。仲師云『尸次,祭祀之
 尸所居更衣帳也』。凡祭之文,既不止於郊祀,立尸之言,理應關於宗廟。古則張幕,今也房省。宗廟旅幕,可變為棟宇;郊祀氈案,何為不轉製簷甍?」



 曇隆議不行。



 建武二年旱,有司議雩祭依明堂。祠部郎何佟之議曰:「《周禮·司巫》云:『若國大旱,則帥巫而舞雩。』鄭玄云:『雩,旱祭也。天子於上帝,諸侯以下於上公之神。』又《女巫》云『旱則舞雩』。鄭玄云:『使女巫舞旱祭,崇陰也。』鄭眾云:『求雨以女巫。』《禮記·月令》云:『命有司為民祈祀山川百原,乃大雩帝,用盛樂。乃命百縣雩祀百辟卿士有益於民者,以祈穀實。』鄭玄云:『陽氣盛而恆旱。山川百原,能興雲致雨者也。眾水所出為百原,必先祭其本。雩,籲嗟求雨之祭也。雩帝,謂為壇南郊之旁,祭五精之帝,配以先帝也。自鞀鼙至柷敔為盛樂,他雩用歌舞而已。百辟卿士,古者上公以下,謂勾龍、后稷之類也。《春秋傳》曰
 龍見而雩,雩之正當以四月。』王肅云:『大雩,求雨之祭也。傳曰龍見而雩,謂四月也。若五月六月大旱,亦用雩,《禮》於五月著雩義也。』晉永和中,中丞啟,雩制在國之南為壇,祈上帝百闢,舞童八列六十四人,歌《雲漢》詩,皆以孟夏,得雨報太牢。於時博士議,舊有壇,漢、魏各自討尋。《月令》云『命有司祈祀山川百原,乃大雩』。又云『乃命百縣雩祀百闢卿士』。則大雩所祭,唯應祭五精之帝而已。勾芒等五神,既是五帝之佐,依鄭玄說,宜配食於庭也。鄭玄云『雩壇在南郊壇之旁』,而不辨東西。尋地道尊右,雩壇方郊壇為輕,理應在左。



 宜於郊壇之東、營域之外築壇。既祭五帝,謂壇宜員。尋雩壇高廣,《禮》、《傳》無明文,案《覲禮》設方明之祀,為壇高四尺,用圭璋等六玉,禮天地四方之神,王者率諸侯親禮,為所以教尊尊也。雩祭五帝,粗可依放。謂今築壇宜崇四尺,其廣輪仍以四為度,徑四丈,周員十二丈而四
 階也。設五帝之位,各依其方,如在明堂之儀。皇齊以世祖配五精於明堂,今亦宜配饗於雩壇矣。古者,孟春郊祀祈嘉穀,孟夏雩祭祈甘雨,二祭雖殊,而所為者一。禮唯有冬至報天,初無得雨賽帝。今雖闕冬至之祭,而南郊兼祈報之禮,理不容別有賽答之事也。禮祀帝於郊,則所尚省費,周祭靈威仰若后稷,各用一牲;今祀五帝、世祖,亦宜各用一犢,斯外悉如南郊之禮也。武皇遏密未終,自可不奏盛樂。至於旱祭舞雩,蓋是籲嗟之義,既非存歡樂,謂此不涉嫌。其餘祝史稱辭,仰祈靈澤而已。禮舞雩乃使無闕,今之女巫,並不習歌舞,方就教試,恐不應速。依晉朝之議,使童子,或時取舍之宜也。司馬彪《禮儀志》云雩祀著皂衣,蓋是崇陰之義。今祭服皆緇,差無所革。其所歌之詩,及諸供須,輒勒主者申攝備辨。」從之。



 隆昌元年,有司奏,參議明堂,咸以世祖配。國子助教謝曇濟議:「案《
 祭法》禘郊祖宗,並列嚴祀。鄭玄注義,亦據兼饗。宜祖宗兩配,文、武雙祀。」助教徐景嵩、光祿大夫王逡之謂宜以世宗文皇帝配。祠部郎何佟之議:「周之文、武,尚推後稷以配天,謂文皇宜推世祖以配帝。雖事施於尊祖,亦義章於嚴父焉。」左僕射王晏議,以為「若用鄭玄祖宗通稱,則生有功德,沒垂尊稱,歷代配帝,何止於郊邪?今殷薦上帝,允屬世祖,百代不毀,其文廟乎!詔「可」。



 至永元二年,佟之又建議曰:「案《祭法》『有虞氏禘黃帝而郊嚳,祖顓頊而宗堯』。『周人禘嚳而郊稷,祖文王而宗武王』,鄭玄云『禘郊祖宗,謂祭祀以配食也。此禘謂祀昊天於圜丘也。祭上帝於南郊曰郊,祭五帝五神於明堂曰祖宗』,『郊祭一帝,而明堂祭五帝,小德配寡,大德配眾』。王肅云『祖宗是廟不毀之名』。



 果如肅言,殷有三祖三宗,並應不毀,何故止稱湯、契?且王者之後存焉,舜寧立堯、頊之廟,傳世祀之乎?漢文以高祖配
 泰畤,至武帝立明堂,復以高祖配食,一人兩配,有乖聖典。自漢明以來,未能反者。故明堂無兼配之祀。竊謂先皇宜列二帝於文祖,尊新廟為高宗,並世祖而泛配,以申聖主嚴父之義。先皇於武皇,倫則第為季,義則經為臣,設配饗之坐,應在世祖之下,並列,俱西向。」



 國子博士王摛議:「《孝經》『周公郊祀后稷以配天,宗祀文王於明堂以配上帝』。不云武王。又《周頌》『《思文》,後稷配天也』。『《我將》,祀文王於明堂也』。武王之文,唯《執競》云『祀武王』。此自周廟祭武王詩,彌知明堂無矣。」



 佟之又議:「《孝經》是周公居攝時禮,《祭法》是成王反位後所行。故《孝經》以文王為宗,《祭法》以文王為祖。又孝莫大於嚴父配天,則周公其人也。尋此旨,寧施成王乎?若《孝經》所說,審是成王所行,則為嚴祖,何得雲嚴父邪?



 且《思文》是周公祀后稷配天之樂歌,《我將》是祀文王配明堂之樂歌。若如摛議,則此二篇,皆應在復子明闢之後。請問
 周公祀后稷、文王,為何所歌?又《國語》云『周人禘嚳郊稷,祖文王,宗武王』。韋昭云『周公時,以文王為宗,其後更以文王為祖,武王為宗』。尋文王以文治而為祖,武王以武定而為宗,欲明文亦有大德,武亦有大功,故鄭注《祭法》云『祖宗通言耳』。是以《詩》云『昊天有成命,二後受之』。注云『二後,文王、武王也』。且明堂之祀,有單有合。故鄭云『四時迎氣於郊,祭一帝,還於明堂,因祭一帝,則以文王配』。明一賓不容兩主也。



 『享五帝於明堂,則泛配文、武』。泛之為言,無的之辭。其禮既盛,故祖宗並配。」



 參議以佟之為允。詔「可」。



 太祖為齊王,依舊立五廟。即位,立七廟,廣陵府君、太中府君、淮陰府君、即丘府君、太常府君、宣皇帝、昭皇后為七廟。建元二年,太祖親祀太廟六室,如儀,拜伏竟,次至昭後室前,儀注應倚立,上以為疑,欲使廟僚行事,又欲以諸王代祝令於昭后室前執爵。以問彭
 城丞劉瓛。瓛對謂:「若都不至昭後坐前,竊以為薄。廟僚即是代上執爵饋奠耳,祝令位卑,恐諸王無容代之。舊廟儀諸王得兼三公親事,謂此為便。」從之。及太子穆妃薨,卒哭,祔於太廟陰室。永明十一年,文惠太子薨,卒哭,祔於太廟陰室。太祖崩,毀廣陵府君。鬱林即位追尊文帝,又毀太中主,止淮陰府君。明帝立,復舊。及崩,祔廟,與世祖為兄弟,不為世數。



 史臣曰:先儒說宗廟之義,據高祖已下五世親盡,故親廟有四。周以后稷始祖,文、武二祧,所以云王立七廟也。禹無始祖,湯不先契,夏五殷六,其數如之。漢立宗廟,違經背古。匡衡、貢禹、蔡邕之徒,空有遷毀之議,亙年四百,竟無成典。



 魏氏之初,親廟止乎四葉,吳、蜀享祭,失禮已多。晉用王肅之談,以文、景為共世,上至征西,其實六也。尋其此意,非以兄弟為後,當以立主之義,可相容於七室。及楊
 元后崩,征西之廟不毀,則知不以元后為世數。廟有七室,數盈八主。江左賀循立議以後,弟不繼兄,故世必限七,主無定數。宋臺初立五廟,以臧後為世室。就禮而求,亦親廟四矣。義反會鄭,非謂從王。自此以來,因仍舊制。夫妻道合,非世葉相承,譬由下祭殤嫡,無關廟數,同之祖曾,義未可了。若據伊尹之言,必及七世,則子昭孫穆,不列婦人。若依鄭玄之說,廟有親稱,妻者言齊,豈或濫享?且閟宮之德,周七非數,楊元之祀,晉八無傷。今謂之七廟,而上唯六祀,使受命之君,流光之典不足。若謂太祖未登,則昭穆之數何繼?斯故禮官所宜詳也。



 宋泰豫元年,明帝崩。博士周洽議:「權制:諒闇之內,不親奉四時祠。」建元四年,尚書令王儉採晉中朝《諒暗議》奏曰:「權典既行,喪禮斯奪,事興漢世,而源由甚遠。殷宗諒間,非有服之稱,周王即吉,唯宴
 樂為譏。《春秋》之義,嗣君逾年即位,則預朝會聘享焉。《左氏》云『凡君即位,卿出並聘,踐修舊好』。



 又云『諸侯即位,小國聘焉,以繼好結信,謀事補闕,禮之大者』。至於諒暗之內而圖婚,三年未終而吉禘,齊歸之喪不廢搜,杞公之卒不徹樂,皆致譏貶,以明鑒戒。自斯而談,朝聘蒸嘗之典,卒哭而備行;婚禘搜樂之事,三載而後舉。通塞興廢,各有由然。又案《大戴禮記》及《孔子家語》並稱武王崩,成王嗣位,明年六月既葬,周公冠成王而朝於祖,以見諸侯,命祝雍作頌。襄十五年十一月『晉侯周卒』,十六年正月『葬晉悼公』。平公既即位,『改服修官,烝於曲沃』。《禮記·曾子問》『孔子曰,天子崩,國君薨,則祝取群廟之主而藏諸祖廟,禮也。卒哭成事,而後主各反其廟』。《春秋左氏傳》『凡君卒哭而祔,祔而後特祀於主,蒸嘗禘於廟』。先儒云『特祀於主者,特以喪禮奉新亡者主於寢,不同於吉。蒸嘗禘於廟者,卒哭成
 事,群廟之主,各反其廟。則四時之祭,皆即吉也。三年喪畢,吉禘於廟,躋群主以定新主也』。凡此諸義,皆著在經誥,昭乎方冊,所以晉、宋因循,同規前典,卒哭公除,親奉蒸嘗,率禮無違,因心允協。爰至泰豫元年,禮官立議,不宜親奉,乃引『三年之制自天子達』。又據《王制》稱『喪三年不祭,唯祭天地社稷,越紼而行事』。曾不知『自天子達』,本在至情,即葬釋除,事以權奪,委衰襲袞,孝享宜申;越紼之旨,事施未葬,卒哭之後,何紼可越?復依范宣之難杜預,譙周之論士祭,並非明據。晉武在喪,每欲存寧戚之懷,不全依諒暗之典;至於四時蒸嘗,蓋以哀疾未堪,非便頓改舊式。江左以來,通儒碩學所歷多矣,守而弗革,義豈徒然?又且即心而言,公卿大夫則負扆親臨,三元告始則朝會萬國,雖金石輟響,而簨泬充庭,情深於恆哀,而跡降於凡制,豈曰能安,國家故也。宗廟蒸嘗,孝敬所先,寧容吉
 事備行,斯典獨廢!就令必宜廢祭,則應三年永闕,乃復同之他故,有司攝禮,進退二三,彌乖典衷。謂宜依舊親奉。」從之。



 永明九年正月,詔太廟四時祭,薦宣帝面起餅、鴨;孝皇后筍、鴨卵、脯醬、炙白肉;高皇帝薦肉膾、俎羹;昭皇后茗、粣、炙魚:皆所嗜也。先是世祖夢太祖曰:「宋氏諸帝嘗在太廟,從我求食。可別為吾祠。」上乃敕豫章王妃庾氏四時還青溪宮舊宅,處內合堂,奉祠二帝二後,牲牢服章,用家人禮。



 史臣曰:漢氏之廟,遍在郡國,求祀已瀆,緣情又疏。重簷閟寢,不可兼建,故前儒抗議,謂之遷毀。光武入纂,南頓君已上四世,別祠舂陵。建武三年幸舂陵園廟是也。張衡《南都賦》曰「清廟肅以微微」。明帝至於章、和,每幸章陵,輒祠舊宅。建安末,魏氏立宗廟,皆在鄴都。魏文黃初二年,洛廟未成,親祠武帝於建始殿,用家人禮。世祖發
 漢明之夢,肇祀故宮,孝享既申,義合前典,亦一時之盛也。



 永明六年,太常丞何諲之議:「今祭有生魚一頭,乾魚五頭。《少牢饋食禮》雲『司士升魚臘膚魚,用鮒十有五』。上既云『臘』,下必是『鮮』,其數宜同。



 稱『膚』足知鱗革無毀。《記》云『槁魚曰商祭,鮮曰脡祭』。鄭注『商,量;脡,直也』。尋『商』旨裁截,『脡』義在全。賀循《祭義》猶用魚十五頭。今鮮頓刪約,槁皆全用。謂宜鮮、槁各二頭,槁微斷首尾,示存古義。」國子助教桑惠度議:「《記》稱尚玄酒而俎腥魚。玄酒不容多,鮮魚理宜約。乾魚五頭者,以其既加人功,可法於五味,以象酒之五齊也。今欲鮮、槁各雙,義無所法。」諲之議不行。



 十年,詔故太宰褚淵、故太尉王儉、故司空柳世隆、故驃騎大將軍王敬則、故鎮東大將軍陳顯達、故鎮東將軍李安民六人,配饗太祖廟庭。祠部郎何諲之議:「功臣配饗,累行宋世,檢其遺事,題列坐
 位,具書贈官爵謚及名,文不稱主,便是設板也。《白虎通》云『祭之有主,孝子以系心也』。揆斯而言,升配廟廷,不容有主。宋時板度,既不復存,今之所制,大小厚薄如尚書召板,為得其衷。」有司攝太廟舊人亦云見宋功臣配饗坐板,與尚書召板相似,事見《儀注》。



 十一年,右僕射王晏、吏部尚書徐孝嗣、侍中何胤奏:「故太子祔太廟,既無先準。檢宋元后故事,太尉行禮,太子拜伏與太尉俱。臣等參議,依擬前典。太常主廟位,太尉執禮祔,太孫拜伏,皆與之俱。正禮既畢,陰室之祭,太孫宜親自進奠。」詔「可」。



 建武二年,有司奏景懿後遷登新廟車服之儀。祠部郎何佟之議曰:「《周禮》王之六服,大裘為上,袞冕次之。五車,玉輅為上,金輅次之。皇后六服,禕衣為上,褕翟次之。首飾有三,副為上,編次之。五車,重翟為上,厭翟次之。上公有大裘玉輅,而上公夫人有副及禕衣,
 是以《祭統》云『夫人副禕立於東房』也。又鄭云『皇后六服,唯上公夫人亦有禕衣』。《詩》云『翟茀以朝』。鄭以翟茀為厭翟,侯伯夫人入廟所乘。今上公夫人副禕既同,則重翟或不殊矣。況景皇懿后禮崇九命,且晉朝太妃服章之禮,同於太后,宋代皇太妃唯無五牛旗為異。其外侍官則有侍中、散騎常侍、黃門侍郎、散騎侍郎各二人,分從前後部,同於王者,內職則有女尚書、女長御各二人,棨引同於太后。又魏朝之晉王,晉之宋王,並置百官,擬於天朝。至於晉文王終猶稱薨,而太上皇稱崩,則是禮加於王矣。故前議景皇后悉依近代皇太妃之儀,則侍衛陪乘並不得異,後乘重翟,亦謂非疑也。尋齊初移廟,宣皇神主乘金輅,皇帝親奉,亦乘金輅,先往行禮畢,仍從神主至新廟,今所宜依準也。」從之。



 永泰元年,有司議應廟見不。尚書令徐孝
 嗣議:「嗣君即位,並無廟見之文;蕃支纂業,乃有虔謁之禮。」左丞蕭琛議:「竊聞祗見厥祖,義著《商書》,朝於武宮,事光晉冊。豈有正位居尊,繼業承天,而不虔覲祖宗,格于太室?《毛詩·周頌》篇曰:『《烈文》,成王即政,諸侯助祭矣也。』鄭注云:『新王即政,必以朝享之禮祭於祖考,告嗣位也。』又篇曰『《閔予小子》,嗣王朝廟也』。鄭注雲:『嗣王者,謂成王也。除武王之喪,將始即政,朝於廟也。』則隆周令典,煥炳經記,體嫡居正,莫若成王。又二漢由太子而嗣位者,西京七主,東都四帝,其昭、成、哀、和、順五君,並皆謁廟,文存漢史;其惠、景、武、元、明、章六君,前史不載謁事,或是偶有闕文,理無異說。議者乃云先在儲宮,已經致敬,卒哭之後,即親奉時祭,則是廟見,故無別謁之禮。竊以為不然。儲后在宮,亦從郊祀,若謂前虔可兼後敬,開元之始,則無假復有配天之祭矣。若以親奉時祭,仍為廟見者,自漢及晉,支庶嗣位,並皆謁廟,既同有蒸嘗,何
 為獨修繁禮?且晉成帝咸和元年改號以謁廟,咸康元年加元服,又更謁。夫時非異主,猶不疑二禮相因,況位隔君臣,而追返以一謁兼敬。宜遠纂周、漢之盛範,近黜晉、宋之乖義,展誠一廟,駿奔萬國。」奏可。



 永明元年十二月,有司奏:「今月三日,臘祠太社稷。一日合朔,日蝕既在致齋內,未審於社祠無疑不?曹檢未有前準。」尚書令王儉議:「《禮記·曾子問》『天子嘗、禘、郊、社、五禮之祭,簠簋既陳』,唯大喪乃廢。至於當祭之日,火、日蝕則停。尋伐鼓用牲,由來尚矣,而簠簋初陳,問所不及。據此而言,致齋初日,仍值薄蝕,則不應廢祭。又初平四年,士孫瑞議以日蝕廢冠而不廢郊,朝議從之。



 王者父天親地,郊社不殊,此則前準,謂不宜廢。」詔「可」。



 永明十一年,兼祠部郎何佟之議:「案《禮記·郊特牲》:『社祭土而主
 陰氣也,君南向於北墉下,答陰之義也。』鄭玄云『答猶對也』。『北墉,社內北墻也』。



 王肅云:『陰氣北向,故君南向以答之。答之為言是相對之稱。』知古祭社,北向設位,齋官南向明矣。近代相承,帝社南向,太社及稷並東向,而齋官位在帝社壇北,西向,於神背後行禮;又名稷為稷社,甚乖禮意。乃未知失在何時,原此理當未久。竊以皇齊改物,禮樂惟新,中國之神,莫貴於社,若遂仍前謬,懼虧盛典。



 謂二社,語其義則殊,論其神則一,位並宜北向。稷若北向,則成相背。稷是百穀之總神,非陰氣之主,宜依先東向。齋官立社壇東北,南向立,東為上,諸執事西向立,南為上。稷依禮無兼稱,今若慾尊崇,正可名為太稷耳,豈得謂為稷社邪?



 臘祠太社日近,案奏事御,改定儀注。」



 儀曹稱治禮學士議曰:「《郊特牲》又云『君之南向,答陽也,臣之北向,答君也。』若以陽氣在
 南,則位應向北,陰氣在北,則位宜向南。今南北二郊,一限南向,皇帝黑瓚階東西向,故知壇墠無繫於陰陽,設位寧拘於南北?群神小祠,類皆限南面,薦饗之時,北向行禮,蓋欲申靈祇之尊,表求幽之義。魏世秦靜使社稷別營,稱自漢以來,相承南向。漢之於周,世代未遠,鄗上頹基,商丘餘樹,猶應尚存,迷方失位,未至於此,通儒達識,不以為非。庾蔚之昔已有此議,後徐爰、周景遠並不同,仍舊不改。」



 佟之議:「來難引君南向答陽,臣北向答君。敢問答之為言,為是相對?為是相背?相背則社位南向,君亦南向,可如來議。《郊特牲》云『臣之北向答君』,復是君背臣。今言君南臣北,向相稱答,則君南不得稱答矣。《記》何得云祭社君南向以答陰邪?社果同向,則君亦宜西向,何故在社南向?在郊西向邪?解則不然,《記》云,君之南向答陽,此明朝會之時,盛陽在南,故君南向對之,猶聖人南面而聽,向明而治之義耳,寧是祈祀天地之日乎?知
 祭社北向,君答故南向,祀天南向,君答宜北向矣。今皇帝黑瓚階東西向者,斯蓋始入之別位,非接對之時也。案《記》云『社所以神地之道也』。又云『社祭土而主陰氣』。又云『不用命,戮於社』。孔安國云『社主陰,陰主殺』。《傳》曰『日蝕,伐鼓於社』。杜預云『責群陰也』。社主陰氣之盛,故北向設位,以本其義耳。餘祀雖亦地祇之貴,而不主此義,故位向不同。不得見餘陰祀不北向,便謂社應南向也。案《周禮》祭社南向,君求幽,宜北向,而《記》云君南向,答陰之義,求幽之論不乖歟?魏權漢社,社稷同營共門,稷壇在社壇北,皆非古制。後移宮南,自當如禮。如靜此言,乃是顯漢社失周法,見漢世舊事。爾時祭社南向,未審出何史籍。就如議者,靜所言是祭社位向仍漢舊法,漢又襲周成規,因而不改者,則社稷三座,並應南向,今何改帝社南向,泰社及稷並東向邪?」



 治禮又難佟之,凡三往反。至建武二年,有司議:「
 治禮無的然顯據。」佟之議乃行。



 建武二年,祠部郎何佟之奏:「案《周禮·大宗伯》『以蒼璧禮天,黃琮禮地』。



 鄭玄又云『皆有牲幣,各放其器之色』。知禮天圜丘用玄犢,禮地方澤用黃牲矣。



 《牧人》云『凡陽祀用騂牲,陰祀用黝牲』。鄭玄云『騂,赤;黝,黑也。陽祀,祭天南郊及宗廟。陰祀,祭地北郊及社稷』。《祭法》云『燔柴於泰壇,祭天也。



 瘞埋於泰折,祭地也。用騂犢』。鄭云『地,陰祀,用黝牲,與天俱用犢,故連言之耳』。知此祭天地即南北郊矣。今南北兩郊同用玄牲,又明堂、宗廟、社稷俱用赤,有違昔典。又鄭玄云『祭五帝於明堂,勾芒等配食』。自晉以來,并圜丘於南郊,是以郊壇列五帝、勾芒等。今明堂祀五精,更闕五神之位,北郊祭地祗,而設重黎之坐,二三乖舛,懼虧盛則。」



 前軍長史劉繪議:「《語》云『犁牛之子騂且角,雖欲勿用,山川其舍諸』。



 未詳山川合為陰祀不?若在陰祀,則與
 黝乖矣。」



 佟之又議:「《周禮》以天地為大祀,四望為次祀,山川為小祀。周人尚赤,自四望以上牲色各依其方者,以其祀大,宜從本也。山川以下,牲色不見者,以其祀小,從所尚也。則《論》、《禮》二說,豈不合符?」參議為允。從之。



 永元元年,步兵校尉何佟之議曰:「蓋聞聖帝明王之治天下也,莫不尊奉天地,崇敬日月,故冬至祀天於圜丘,夏至祭地於方澤,春分朝日,秋分夕月,所以訓民事君之道,化下嚴上之義也。故禮云『王者必父天母地,兄日姊月』。《周禮·典瑞》云『王搢大圭,執鎮圭,藻藉五採五就以朝日』。馬融云『天子以春分朝日,秋分夕月』。《覲禮》『天子出,拜日於東門之外』。盧植云『朝日以立春之日也』。



 鄭玄云『端當為冕,朝日春分之時也』。《禮記·朝事議》云『天子冕而執鎮圭,尺有二寸,率諸侯朝日於東郊,所以教尊尊也』。故鄭
 知此端為冕也。《禮記·保傅》云『三代之禮,天子春朝朝日,秋暮夕月,所以明有敬也』。而不明所用之定辰。馬、鄭云用二分之時,盧植云用立春之日。佟之以為日者太陽之精,月者太陰之精。春分陽氣方永,秋分陰氣向長。天地至尊用其始,故祭以二至,日月禮次天地,故朝以二分,差有理據,則融、玄之言得其義矣。漢世則朝朝日,暮夕月。魏文帝詔曰:『《覲禮》天子拜日東門之外,反禮方明。《朝事議》曰天子冕而執鎮圭,率諸侯朝日於東郊。以此言之,蓋諸侯朝,天子祀方明,因率朝日也。漢改周法,群公無四朝之事,故不復朝於東郊,得禮之變矣。然旦夕常於殿下東向拜日,其禮太煩。今採周春分之禮,損漢日拜之儀,又無諸侯之事,無所出東郊,今正殿即亦朝會行禮之庭也。宜常以春分於正殿之庭拜日,其夕月文不分明。其議奏。』魏秘書監薛循請論云:『舊事朝日以春分,夕月以秋分。案《周禮》朝日無常日,
 鄭玄云用二分,故遂施行。秋分之夕,月多東潛,而西向拜之,背實遠矣。謂朝日宜用仲春之朔,夕月宜用仲秋之朔。』淳于睿駁之,引《禮記》云『祭日於東,祭月於西,以端其位』。《周禮》秋分夕月,並行於上世。西向拜月,雖如背實,亦猶月在天而祭之於坎,不復言背月也。佟之案《禮器》云『為朝夕必放於日月』。



 鄭玄云『日出東方,月出西方』;又云『大明生於東,月生於西,此陰陽之分,夫婦之位也』。鄭玄云『大明,日也』。知朝日東向,夕月西向,斯蓋各本其位之所在耳。猶如天子東西遊幸,朝堂之官及拜官者猶北向朝拜,寧得以背實為疑邪?佟之謂魏世所行,善得與奪之衷。晉初棄圜丘方澤,於兩郊二至輟禮,至於二分之朝,致替無義。江左草創,舊章多闕,宋氏因循,未能反古。竊惟皇齊應天御極,典教惟新,謂宜使盛典行之盛代,以春分朝於殿庭之西,東向而拜日,秋分於殿庭之東,西向而拜
 月,此即所謂必放日月以端其位之義也。使四方觀化者,莫不欣欣而頌美。



 旒藻之飾,蓋本天之至質也,朝日不得同昊天至質之禮,故玄冕三旒也。近代祀天,著袞十二旒,極文章之美,則是古今禮之變也。禮天朝日,既服宜有異,頃世天子小朝會,著絳紗袍、通天金博山冠,斯即今朝之服次袞冕者也。竊謂宜依此拜日月,甚得差降之宜也。佟之任非禮局,輕奏大典,實為侵官,伏追慚震。」從之。



 永明三年,有司奏:「來年正月二十五日丁亥,可祀先農,即日輿駕親耕。」



 宋元嘉、大明以來,並用立春後亥日,尚書令王儉以為亥日藉田,經記無文,通下詳議。



 兼太學博士劉蔓議:「《禮》,孟春之月,立春迎春,又於是月以元日祈穀,又擇元辰躬耕帝藉。盧植說禮通辰日,日,甲至癸也,辰,子至亥也。郊天,陽也,故以日。藉田,陰也,故以辰。
 陰禮卑後,必居其末,亥者辰之末,故《記》稱元辰,注曰吉亥。又據五行之說,木生於亥,以亥日祭先農,又其義也。」



 太常丞何諲之議:「鄭注云『元辰,蓋郊後吉亥也』。亥,水辰也,凡在墾稼,咸存灑潤。五行說十二辰為六合,寅與亥合,建寅月東耕,取月建與日辰合也。」



 國子助教桑惠度議:「尋鄭玄以亥為吉辰者,陽生於子,元起於亥,取陽之元以為生物,亥又為水,十月所建,百穀賴茲沾潤畢熟也。」



 助教周山文議:「盧植云『元,善也。郊天,陽也,故以日。藉田,陰也,故以辰』。蔡邕《月令章句》解元辰云『日,乾也。辰,支也。有事於天,用日。有事於地,用辰』。」



 助教何佟之議:「《少牢饋食禮》云『孝孫某,來日丁亥,用薦歲事於皇祖伯某』。注云『丁未必亥也,直舉一日以言之耳。禘太廟禮日用丁亥,若不丁亥,則用己亥、辛亥,茍有亥可也』。鄭又云『必用丁、己者,取其令名,自丁寧自變改,皆為謹敬』。如此,丁亥自是祭祀之日,
 不專施於先農。漢文用此日耕藉祠先農,故後王相承用之,非有別義。」



 殿中郎顧暠之議:「鄭玄稱先郊後吉辰,而不說必亥之由。盧植明子亥為辰,亦無常辰之證。漢世躬藉,肇發漢文,詔云『農,天下之本,其開藉田』。斯乃草創之令,末睹親載之吉也。昭帝癸亥耕於鉤盾弄田,明帝癸亥耕下邳,章帝乙亥耕定陶,又辛丑耕懷,魏之烈祖實書辛未,不繫一辰,徵於兩代矣。推晉之革魏,宋之因晉,政是服膺康成,非有異見者也。班固序亥位云『陰氣應亡射,該藏萬物,而雜陽閡種』。且亥既水辰,含育為性,播厥取吉,其在茲乎?固序丑位云『陰大旅,助黃鍾宣氣而牙物』。序未位云『陰氣受任,助蕤賓君主種物,使長大茂盛』。



 是漢朝迭選,魏室所遷,酌舊用丑,實兼有據。」參議奏用丁亥。詔「可」。



 建元四年正月,詔立國學,置學生百五十人。其有位樂入者五十
 人。生年十五以上,二十以還,取王公已下至三將、著作郎、廷尉正、太子舍人、領護諸府司馬諮議經除敕者、諸州別駕治中等見居官及罷散者子孫。悉取家去都二千里為限。太祖崩,乃止。



 永明三年正月,詔立學,創立堂宇,召公卿子弟下及員外郎之胤,凡置生二百人。其年秋中悉集。有司奏:「宋元嘉舊事,學生到,先釋奠先聖先師,禮又有釋菜,未詳今當行何禮?用何樂及禮器?」尚書令王儉議:「《周禮》『春入學,舍菜合舞』。《記》云『始教,皮弁祭菜,示敬道也』。又云『始入學,必祭先聖先師』。中朝以來,釋菜禮廢,今之所行,釋奠而已。金石俎豆,皆無明文。方之七廟則輕,比之五禮則重。陸納、車胤謂宣尼廟宜依亭侯之爵;範寧欲依周公之廟,用王者儀,範宣謂當其為師則不臣之,釋奠日,備帝王禮樂。此則車、陸失於過輕,二範傷於太重。喻希云『若至王者自設禮樂,則肆賞於至敬之所;
 若欲嘉美先師,則所況非備』。尋其此說,守附情理。皇朝屈尊弘教,待以師資,引同上公,即事惟允。元嘉立學,裴松之議應舞六佾,以郊樂未具,故權奏登歌。今金石已備,宜設軒縣之樂,六佾之舞,牲牢器用,悉依上公。」其冬,皇太子講《孝經》,親臨釋奠,車駕幸聽。



 建武四年正月,詔立學。永泰元年,東昏侯即位,尚書符依永明舊事廢學。領國子助教曹思文上表曰:「古之建國君民者,必教學為先,將以節其邪情而禁其流欲,故能化民裁俗,習與性成也。是以忠孝篤焉,信義成焉,禮讓行焉,尊教宗學,其致一也。是以成均煥於古典,虎門炳於前經。陛下體睿淳神,纘承鴻業,今制書既下,而廢學先聞,將恐觀國之光者,有以擬議也。若以國諱故宜廢,昔漢成立學,爰洎元始,百餘年中,未嘗暫廢,其間有國諱也。且晉武之崩,又其學猶存,斯皆先代不以國諱而廢學之明文也。永明以無
 太子故廢,斯非古典也。尋國之有學,本以興化致治也,天子於以諮謀焉,於以行禮焉。《記》雲『天子出征,受命於祖,受成於學。執有罪反,釋奠於學』。又云『食三老五更於太學,天子袒而割牲,執爵而酳,以教諸侯悌也』。於斯學,是天子有國之基,教也或以之。所言皆太學事也,今引太學不非證也。據臣所見,今之國學,即古之太學。晉初太學生三千人,既多猥雜,惠帝時欲辯其涇渭,故元康三年始立國子學,官品第五以上得入國學。



 天子去太學入國學,以行禮也。太子去太學入國學,以齒讓也。太學之與國學,斯是晉世殊其士庶,異其貴賤耳。然貴賤士庶,皆須教成,故國學太學兩存之也,非有太子故立也。然繫廢興於太子者,此永明之鉅失也。漢崇儒雅,幾致刑厝,而猶道謝三、五者,以其致教之術未篤也。古之教者,家有塾,黨有庠,術有序,國有學,以諷誦相
 摩。今學非唯不宜廢而已,乃宜更崇尚其道,望古作規,使郡縣有學,饗閭立教。請付尚書及二學詳議。」有司奏。從之。學竟不立。



 永明五年十月,有司奏:「南郡王昭業冠,求儀注未有前准。」尚書令王儉議:「皇孫冠事,歷代所無。禮雖有嫡子無嫡孫,然而地居正體,下及五世。今南郡王體自儲暉,實惟國裔,元服之典,宜異列蕃。案《士冠禮》『主人玄冠朝服,賓加其冠,贊者結纓』。鄭玄云『主人,冠者之父兄也』。尋其言父及兄,則明祖在,父不為主也。《大戴禮記·公冠篇》云公冠自為主,四加玄冕,以卿為賓。此則繼體之君及帝之庶子不得稱子者也。《小戴禮記·冠義》云『冠於阼,以著代也。醮於客位,三加彌尊,加有成也』。注稱『嫡子冠於阼,庶子冠於房』。《記》又云『古者重冠,故行之於廟,所以自卑而尊先祖也』。據此而言,彌與鄭注《儀禮》相會。是故中朝以來,太子冠則皇帝臨軒,司徒加冠,光
 祿贊冠。諸王則郎中加冠,中尉贊冠。今同於儲皇則重,依於諸王則輕。又《春秋》之義,『不以父命辭王父命』。《禮》『父在斯為子,君在斯為臣』。皇太子居臣子之節,無專用之道。南郡雖處蕃國,非支庶之列,宜稟天朝之命,微申冠阼之禮。晉武帝詔稱漢、魏遣使冠諸王,非古正典。此蓋謂庶子封王,合依公冠自主之義,至於國之長孫,遣使惟允。宜使太常持節加冠,大鴻臚為贊;醮酒之儀,亦歸二卿;祝醮之辭,附準經記,別更撰立,不依蕃國常體。國官陪位拜賀,自依舊章。其日內外二品清官以上,詣止車集賀,並詣東宮南門通箋。別日上禮,宮臣亦詣門稱賀,如上臺之儀。既冠之後,克日謁廟,以弘尊祖之義。此既大典,宜通關八座丞郎并下二學詳議。」僕射王奐等十四人議並同,並撰立贊冠、醮酒二辭。詔「可」。祝辭曰:「皇帝使給事中、太常、武安侯蕭惠基加南郡王冠。」祝曰:「筮日筮賓,肇加元服。棄爾幼
 志,從厥成德。親賢使能,克隆景福。」醮酒辭曰:「旨酒既清,嘉薦既盈。兄弟具在,淑慎儀形。永屆眉壽,於穆斯寧。」



 永明中,世祖以婚禮奢費,敕諸王納妃,上御及六宮依禮止棗慄腶脩,加以香澤花粉,其餘衣物皆停。唯公主降嬪,則止遺舅姑也。永泰元年,尚書令徐孝嗣議曰:「夫人倫之始,莫重冠婚,所以尊表成德,結歡兩姓。年代污隆,古今殊則,繁簡之儀,因時或異。三加廢於士庶,六禮限於天朝,雖因習未久,事難頓改,而大典之要,深宜損益。案《士冠禮》,三加畢,乃醴冠者,醴則唯一而已,故醴辭無二。若不醴,則每加輒醮以酒,故醮辭有三。王肅云『醴本古味,其禮重;酒用時味,其禮輕故也』。或醴或醮,二三之義,詳記於經文。今皇王冠畢,一酌而已,即可擬古設醴;而猶用醮辭,實為乖衷。尋婚禮實篚以四爵,加以合巹,既崇尚質之理,又象泮合之義。故
 三飯卒食,再酳用巹。先儒以禮成好合,事終於三,然後用巹合。儀注先酳巹,以再以三,有違旨趣。又《郊特牲》曰『三王作牢用陶匏』。



 言太古之時,無共牢之禮,三王作之,而用太古之器,重夫婦之始也。今雖以方樏示約,而彌乖昔典。又連巹以鎖,蓋出近俗。復別有牢燭,雕費採飾,亦虧曩制。



 方今聖政日隆,聲教惟穆,則古昔以敦風,存餼羊以愛禮,沿襲之規,有切治要,嘉禮實重,宜備舊章。謂自今王侯已下冠畢一酌醴,以遵古之義。醴即用舊文,於事為允。婚亦依古,以巹酌終酳之酒,並除金銀連鎖,自餘雜器,悉用埏陶。堂人執燭,足充鸑燎,牢燭華侈,亦宜停省。庶斫雕可期,移俗有漸。」參議並同。奏可。



 晉武太始二年,有司奏,故事皇后諱與帝諱俱下。詔曰:禮,內諱不出宮,近代諱之也。建元元年,太常上朝堂諱訓。僕射王儉議曰:「後諱依舊不立訓。禮,天
 子諸侯諱群祖,臣隸既有從敬之義,宜為太常府君諱。至於朝堂榜題,本施至極,既追尊所不及,禮降於在三,晉之京兆,宋之東安,不列榜題。孫毓議稱京兆列在正廟,臣下應諱,而不上榜。宋初博士司馬道敬議東安府君諱宜上榜,何承天執不同,即為明據。」其有人名地名犯太常府君及帝后諱者,皆改。宣帝諱同。二名不偏諱。所以改承明門為北掖,以榜有「之」字與「承」並。東宮承華門亦改為宣華云。



 漢末,蔡邕立漢《朝會志》,竟不就。秦人以十月旦為歲首,漢初習以大饗會,後用夏正,饗會猶未廢十月旦會也。東京以後,正旦夜漏未盡七刻,鳴鐘受賀,公侯以下執贄來庭,二千石以上升殿稱萬歲,然後作樂宴饗。張衡賦云「皇輿夙駕,登天光於扶桑」。然則雖云夙駕,必辨色而行事矣。魏武都鄴,正會文昌殿,用漢儀,又設百華燈。後魏文修洛陽宮室,權都許昌,宮殿狹小,元日於城南立氈殿,
 青帷以為門,設樂饗會。後還洛陽,依漢舊事。晉武帝初,更定朝會儀,夜漏未盡十刻,庭燎起火,群臣集。傅玄《朝會賦》云「華燈若乎火樹,熾百枝之煌煌」。



 此則因魏儀與庭燎並設也。漏未盡七刻,群臣入白賀,未盡五刻,就本位,至漏盡,皇帝出前殿,百官上賀,如漢儀。禮畢罷入,群臣坐,謂之辰賀。晝漏上三刻更出,百官奉壽酒,大饗作樂,謂之晝會。別置女樂三十人於黃帳外,奏《房中之歌》。



 江左多虞,不復晨賀,夜漏未盡十刻,開宣陽門,至平旦始開殿門;晝漏上五刻,皇帝乃出受賀。宋世至十刻乃受賀。其餘升降拜伏之儀,及置立后妃王公已下祠祀夕牲拜授吊祭,皆有儀注,文多不載。



 三月三日曲水會,古禊祭也。漢《禮儀志》云「季春月上巳,官民皆潔濯于東流水上,自洗濯祓除去宿疾為大潔」。不見東流為何水也。晉中朝雲,卿已下至於庶民,皆禊洛水之側,事見諸《禊賦》及《夏仲
 御傳》也。趙王倫篡位,三日,會天淵池誅張林。懷帝亦會天淵池賦詩。陸機云「天淵池南石溝,引御溝水,池西積石為禊堂。跨水,流杯飲酒」。亦不言曲水。元帝又詔罷三日弄具。今相承為百戲之具,雕弄技巧,增損無常。



 史臣曰:案禊與曲水,其義參差。舊言陽氣布暢,萬物訖出,姑洗潔之也。巳者祉也,言祈介祉也。一說,三月三日,清明之節,將修事於水側,禱祀以祈豐年。



 應劭云:「禊者,潔也,言自潔濯也。或云漢世有郭虞者,以三月上辰生二女,上巳又生一女,二日中頻生皆死,時俗以為大忌,民人每至其日,皆適東流水祈祓自潔濯,浮酌清流,後遂為曲水。」案高后祓霸上,馬融《梁冀西第賦》云「西北戌亥,玄石承輸。蝦蟆吐寫,庚辛之域」。即曲水之象也。今據禊為曲水事,應在永壽之前已有,祓除則不容在高后之後。祈農之說,於事
 為當。



 九月九日馬射。或說云,秋金之節,講武習射,像漢立秋之禮。



 史臣曰:案晉中朝元會,設臥騎、倒騎、顛騎,自東華門馳往神虎門,此亦角抵雜戲之流也。宋武為宋公,在彭城,九日出項羽戲馬臺,至今相承,以為舊準。



\end{pinyinscope}