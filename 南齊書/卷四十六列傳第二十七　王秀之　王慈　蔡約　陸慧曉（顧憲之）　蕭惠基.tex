\article{卷四十六列傳第二十七 王秀之 王慈 蔡約 陸慧曉(顧憲之) 蕭惠基}

\begin{pinyinscope}

 王秀之,字伯奮,瑯邪臨沂人也。祖裕,宋左光祿大夫、儀同三司。父瓚之,金紫光祿大夫。秀之幼時,裕愛其風采。起家著作佐郎,太子舍人。父卒,為庵舍於墓下持喪,服闋復職。吏部尚書褚淵見秀之正潔,欲與結婚,秀之不肯,以此頻轉為兩府外兵參軍。遷太子洗馬,司徒左西屬,桂陽王司空從事中郎。秀之知休範將反,辭疾不
 就。出為晉平太守。至郡期年,謂人曰:「此邦豐壤,祿俸常充。吾山資已足,豈可久留以妨賢路。」上表請代,時人謂「王晉平恐富求歸」。



 還為安成王驃騎諮議,轉中郎。又為太祖驃騎諮議。昇明二年,轉左軍長史、尋陽太守,隨府轉鎮西長史、南郡太守。府主豫章王嶷既封王,秀之遷為司馬、河東太守,辭郡不受。加寧朔將軍,改除黃門郎,未拜,仍遷豫章王驃騎長史。王於荊州立學,以秀之領儒林祭酒。遷寧朔將軍、南郡王司馬。復為黃門郎,領羽林監。



 遷長沙王中軍長史。世祖即位,為太子中庶子,吏部郎,出為義興太守,遷侍中祭酒,轉都官尚書。



 初,秀之祖裕性貞正,徐羨之、傅亮當朝,裕不與來往。及致仕隱吳興,與子瓚之書曰:「吾欲使汝處不競之地。」瓚之歷官至五兵尚書,未嘗詣一朝貴。江湛謂何偃曰:「王瓚之今便是朝隱。」及柳元景、顏師伯令僕貴要,瓚之竟不候之。



 至秀之為
 尚書,又不與令王儉款接。三世不事權貴,時人稱之。



 轉侍中,領射聲校尉。出為輔國將軍、隨王鎮西長史、南郡內史。州西曹茍平遺秀之交知書,秀之拒不答。平乃遺書曰:「僕聞居《謙》之位,既刊于《易》;傲不可長,《禮》明其文。是以信陵致夷門之義,燕丹收荊卿之節,皆以禮而然矣。



 丈夫處世,豈可寂漠恩榮,空為後代一丘土?足下業潤重光,聲居朝右,不修高世之績,將何隔於愚夫?僕耿介當年,不通群品,饑寒白首,望物嗟來。成人之美,《春秋》所善。薦我寸長,開君尺短,故推風期德,規於相益,實非碌碌有求於平原者也。僕與足下,同為四海國士。夫盛衰迭代,理之恆數。名位參差,運之通塞,豈品德權行為之者哉?第五之號,既無易於驃騎;西曹之名,復何推於長史?足下見答書題久之,以君若此非典,何宜施之於國士?如其循禮,禮無不答,謹以相還,亦何犯於逆鱗哉?君子處人以德不以位,相如
 不見屈於澠池,毛遂安受辱於郢門,造敵臨事,僕必先於二子。未知足下之貴,足下之威,孰若秦、楚兩王?僕以德為寶,足下以位為寶,各寶其寶,於此敬宜。常聞古人交絕,不泄惡言,僕謂之鄙。



 無以相貽,故薦貧者之贈。」平,潁川人。豫章王嶷為荊州時,平獻書令減損奢麗,豫章王優教酬答。尚書令王儉當世,平又與儉書曰:「足下建高世之名而不顯高世之迹,將何以書於齊史哉?」至是南郡綱紀啟隨王子隆請罪平,平上書自申。



 秀之尋徵侍中,領游擊將軍。未拜,仍為輔國將軍、吳興太守。秀之常云位至司徒左長史,可以止足矣。吳興郡隱業所在,心願為之。到郡脩治舊山,移置輜重。



 隆昌元年,卒官,年五十三。謚曰簡子。



 秀之宗人僧祐,太尉儉從祖兄也。父遠,光祿勳。宋世為之語曰:「王遠如屏風,屈曲從俗,能蔽風露。」而僧祐負氣不群,儉常候之,辭不相見。世祖數閱武,僧祐獻《講武賦》,
 儉借觀,僧祐不與。竟陵王子良聞僧祐善彈琴,於座取琴進之,不肯從命。永明末,為太子中舍人,在直屬疾,代人未至,僧祐委出,為有司所奏,贖論。官至黃門郎。時衛軍掾孔逭亦抗直,著《三吳決錄》,不傳。



 王慈,字伯寶,瑯邪臨沂人,司空僧虔子也。年八歲,外祖宋太宰江夏王義恭迎之內齋,施寶物恣聽所取,慈取素琴石研,義恭善之。少與從弟儉共書學。除秘書郎,太子舍人,安成王撫軍主簿,轉記室。遷秘書丞,司徒左西屬,右長史,試守新安太守,黃門郎,太子中庶子,領射聲校尉,安成王冠軍,豫章王司空長史,司徒左長史,兼侍中。出為輔國將軍、豫章內史,父憂去官。起為建武將軍、吳郡太守。遷寧朔將軍,大司馬長史,重除侍中,領步兵校尉。



 慈以朝堂諱榜,非古舊制,上表曰:「夫帝后之德,綢繆天地,君人之亮,蟬聯日月。
 至於名族不著,昭自方策,號謚聿宣,載伊篇籍。所以魏臣據中以建議,晉主依經以下詔。朝堂榜志,諱字懸露,義非綿古,事殷中世,空失資敬之情,徒乖嚴配之道。若乃式功鼎臣,贊庸元吏,或以勛崇,或由姓表。故孔悝見銘,謂標叔舅,子孟應圖,稱題霍氏。況以處一之重,列尊名以止仁;無二之貴,夤沖文而止敬。昔東平即世,孝章巡宮而灑泣;新野雲終,和熹見似而流涕。感循舊類,尚或深心;矧觀徽跡,能無惻隱?今扃禁嶔邃,動延車蓋,若使鑾駕紆覽,四時臨閱,豈不重增聖慮,用感宸衷?愚謂空標簡第,無益於匪躬;直述朝堂,寧虧於夕惕。



 伏惟陛下保合萬國,齊聖群生,當刪前基之弊軌,啟皇齊之孝則。」詔付外詳議。



 博士李捴議:「據《周禮》,凡有新令,必奮鐸以警眾,乃退以憲之於王宮。



 注『憲,表懸之也,」太常丞王僴之議:「尊極之名,宜率土同諱。目可得睹,口不可言。口不可言,則知之
 者絕,知之者絕,則犯觸必眾。」儀曹郎任昉議:「捴取證明之文,僴之即情惟允。直班諱之典,爰自漢世,降及有晉,歷代無爽。今之諱榜,兼明義訓,『邦』之字『國』,實為前事之徵。名諱之重,情敬斯極,故懸諸朝堂,搢紳所聚,將使起伏晨昏,不違耳目,禁避之道,昭然易從。此乃敬恭之深旨,何情典之或廢?尊稱霍氏,理例乖方。居下以名,故以不名為重;在上必諱,故以班諱為尊。因心則理無不安,即事則習行已久,謂宜式遵,無所創革。」



 慈議不行。



 慈患腳,世祖敕王晏曰:「慈在職未久,既有微疾,不堪朝,又不能騎馬,聽乘車在仗後。」江左來少例也。以疾從閑任,轉冠軍將軍、司徒左長史。慈妻劉秉女。子觀,尚世祖長女吳縣公主,脩婦禮,姑未嘗交答。江夏王鋒為南徐州,妃,慈女也,以慈為冠軍將軍、東海太守,加秩中二千石,行南徐州府事。還為冠軍將軍、廬陵王中軍長史,未拜,永明九年,卒。年四十一。



 謝超宗嘗謂慈曰:「卿書何當及虔公?」慈曰:「我之不得仰及,猶雞之不及鳳也。」時人以為名答。追贈太常,謚懿子。



 蔡約,字景捴,濟陽考城人也。祖廓,宋祠部尚書。父興宗,征西、儀同。約少尚宋孝武女安吉公主,拜駙馬都尉,秘書郎,不拜。順帝車騎驃騎行參軍,通直郎,不就。遷太祖司空東閣祭酒,太尉主簿。齊臺建,為世子中舍人,仍隨度東宮。



 轉鄱陽王友,竟陵王鎮北征北諮議,領記室,中書郎,司徒右長史,黃門郎,領本州中正。出為新安太守,復為黃門郎,領射聲校尉,通直常侍,領驍騎將軍,太子中庶子,領屯騎校尉。永明八年八月合朔,約脫武冠,解劍,於省眠,至下鼓不起,為有司所奏,贖論。太孫立,領校尉如故。



 出為宜都王冠軍長史、淮南太守,行府州事。世祖謂約曰:「今用卿為近蕃上佐,想副我所期。」約曰:「南豫密邇京師,不治自理。臣亦何人,爝火不息。」



 時諸王
 行事多相裁割,約在任,主佐之間穆如也。



 遷司徒左長史。高宗為錄尚書輔政,百僚屣履到席,約躡屐不改。帝謂江祏曰:「蔡氏故是禮度之門,故自可悅。」祏曰:「大將軍有揖客,復見於今。」建武元年,遷侍中。明年,遷西陽王撫軍長史,加冠軍將軍,徙廬陵王右軍長史,將軍如故。轉都官尚書,遷邵陵王師,加給事中,江夏王車騎長史,加征虜將軍,並不拜。



 好飲酒,夷淡不與世雜。遷太子詹事。永明元二年,卒。年四十四。贈太常。



 陸慧曉,字叔明,吳郡吳人也。祖萬載,侍中。父子真,元嘉中為海陵太守。



 時中書舍人秋當親幸,家在海陵,假還葬父,子真不與相聞。當請發民治橋,又以妨農不許。彭城王義康聞而賞焉。自臨海太守眼疾歸,為中散大夫,卒。



 慧曉清介正立,不雜交游。會稽內史同郡張暢見慧曉童幼,便嘉異之。張緒稱之曰:「江東裴、樂也。」初應州
 郡辟,舉秀才,衛尉史,歷諸府行參軍。以母老還家侍養,十餘年不仕。太祖輔政,除為尚書殿中郎。鄰族來相賀,慧曉舉酒曰:「陸慧曉年踰三十,婦父領選,始作尚書郎,卿輩乃復以為慶邪?」



 太祖表禁奢侈,慧曉撰答詔草,為太祖所賞,引為太傅東閣祭酒。建元初,仍遷太子洗馬。武陵王曄守會稽,上為精選僚吏,以慧曉為征虜功曹,與府參軍沛國劉璡同從述職。行至吳,璡謂人曰:「吾聞張融與陸慧曉並宅,其間有水,此水必有異味。」遂往,酌而飲之。廬江何點薦慧曉於豫章王嶷,補司空掾,加以恩禮。



 轉長沙王鎮軍諮議參軍。安陸侯緬為吳郡,復禮異慧曉,慧曉求補緬府諮議參軍。



 遷始興王前將軍安西諮議,領冠軍錄事參軍,轉司徒從事中郎,遷右長史。時陳郡謝朏為左長史,府公竟陵王子良謂王融曰:「我府二上佐,求之前世,誰可為比?」



 融曰:「兩賢同時,便是未有前例。」子良於
 西邸抄書,令慧曉參知其事。



 尋遷西陽王征虜、巴陵王後軍、臨汝公輔國三府長史,行府州事。復為西陽王左軍長史,領會稽郡丞,行郡事。隆昌元年,徙為晉熙王冠軍長史、江夏內史,行郢州事。



 慧曉歷輔五政,治身清肅,僚佐以下造詣,趣起送之。或謂慧曉曰:「長史貴重,不宜妄自謙屈。」答曰:「我性惡人無禮,不容不以禮處人。」未嘗卿士大夫,或問其故,慧曉曰:「貴人不可卿,而賤者可卿。人生何容立輕重於懷抱!」終身常呼人位。



 建武初,除西中郎長史,行事、內史如故。俄征黃門郎,未拜,遷吏部郎。尚書令王晏選門生補內外要局,慧曉為用數人而止,晏恨之。送女妓一人,欲與申好,慧曉不納。吏曹都令史歷政以來,諮執選事,慧曉任己獨行,未嘗與語。帝遣左右單景俊以事誚問,慧曉謂景俊曰:「六十之年,不復能諮都令史為吏部郎也。上若謂身不堪,便當拂衣而退。」帝甚憚之。後欲
 用為侍中,以形短小,乃止。出為輔國將軍、晉安王鎮北司馬、征北長史、東海太守,行府州事。入為五兵尚書,行揚州事。崔惠景事平,領右軍將軍,出監南徐州,少時,仍遷持節、督南兗兗徐青冀五州軍事、輔國將軍、南兗州刺史。至鎮俄爾,以疾歸,卒。年六十二。贈太常。



 同郡顧憲之,字士思,宋鎮南將軍凱之孫也。性尤清直。永明六年,為隨王東中郎長史、行會稽郡事。時西陵戍主杜元懿啟:「吳興無秋,會稽豐登,商旅往來,倍多常歲。西陵牛埭稅,官格日三千五百,元懿如即所見,日可一倍,盈縮相兼,略計年長百萬。浦陽南北津及柳浦四埭,乞為官領攝,一年格外長四百許萬。西陵戍前檢稅,無妨戍事,餘三埭自舉腹心。」世祖敕示會稽郡:「此詎是事宜?可訪察即啟。」憲之議曰:尋始立牛埭之意,非茍逼僦以納稅也,當以風濤迅險,人力不捷,屢致膠溺,濟急利物耳。既公私是樂,所以輸直
 無怨。京師航渡,即其例也。而後之監領者不達其本,各務己功,互生理外——或禁遏別道,或空稅江行,或撲船倍價,或力周而猶責,凡如此類,不經埭煩牛者上詳,被報格外十條,並蒙停寢。從來喧訴,始得暫弭。案吳興頻歲失稔,今茲尤饉,去乏從豐,良由饑棘。或征貨貿粒,還拯親累;或提攜老弱,陳力糊口。埭司責稅,依格弗降。舊格新減,尚未議登,格外加倍,將以何術?皇慈恤隱,振廩蠲調,而元懿幸災榷利,重增困瘼。人而不仁,古今共疾。且比見加格置市者前後相屬,非惟新加無贏,並皆舊格猶闕。愚恐元懿今啟,亦當不殊。若事不副言,懼貽譴詰,便百方侵苦,為公賈怨。元懿稟性苛刻,已彰往效,任以物土,譬以狼將羊,其所欲舉腹心,亦當虎而冠耳。書云「與其有聚斂之臣,寧有盜臣」。此言盜公為損蓋微,斂民所害乃大也。今雍熙在運,草木含澤,其非事宜,仰如聖旨。然掌斯任者,應
 簡廉平,廉則不竊於公,平則無害於民矣。愚又以便宜者,蓋謂便於公,宜於民也。竊見頃之言便宜者,非能於民力之外用天分地也,率皆即日不宜於民,方來不便於公。名與實反,有乖政體。凡如此等,誠宜深察。



 山陰一縣,課戶二萬,其民貲不滿三千者,殆將居半,刻又刻之,猶且三分餘一。凡有貲者,多是士人復除。其貧極者,悉皆露戶役民。三五屬官,蓋惟分定,百端輸調,又則常然。比眾局檢校,首尾尋續,橫相質累者,亦復不少。一人被攝,十人相追;一緒裁萌,千蘗互起。蠶事弛而農業廢,賤取庸而貴舉責,應公贍私,日不暇給,欲無為非,其可得乎?死且不憚,矧伊刑罰;身且不愛,何況妻子。是以前檢未窮,後巧復滋,網辟徒峻,猶不能悛。竊尋民之多偽,實由宋季軍旅繁興,役賦殷重,不堪勤劇,倚巧祈優,積習生常,遂迷忘反。四海之大,黎庶之眾,心用參差,難卒澄一。化宜以漸,不
 可疾責。誠存不擾,藏疾納汙,實增崇曠,務詳寬簡,則稍自歸淳。又被符簡,病前後年月久遠,具事不存,符旨既嚴,不敢暗信。



 縣簡送郡,郡簡呈使,殊形詭狀,千變萬源。聞者忽不經懷,見者實足傷駭。兼親屬里伍,流離道路,時轉寒涸,事方未已。其士人婦女,彌難厝衷。不簡則疑其有巧,欲簡復未知所安。愚謂此條,宜委縣簡保,舉其綱領,略其毛目,乃囊漏,不出貯中,庶嬰疾沈痼者,重荷生造之恩也。



 又永興、諸暨離唐宇之寇擾,公私殘燼,彌復特甚。儻值水旱,實不易念。俗諺云「會稽打鼓送恤,吳興步簷令史。」會稽舊稱沃壤,今猶若此;吳興本是脊土,事在可知。因循餘弊,誠宜改張。沿元懿今啟,敢陳管見。



 世祖並從之。由是深以方直見委。仍行南豫、南兗二州事,簽典咨事,未嘗與色,動遵法制。歷黃門郎,吏部郎。永元中,為豫章內史。



 蕭惠基,南蘭陵蘭陵人也。祖源之,宋前將軍。父思話,征西將軍、儀同三司。



 惠基幼以外戚見江夏王義恭,歎其詳審,以女結婚。解褐著作佐郎,征北行參軍,尚書水部,左民郎。出為湘東內史。除奉車都尉,撫軍車騎主簿。



 泰始初,兄益州刺史惠開拒命,明帝遣惠基奉使至蜀,宣旨慰勞。惠開降而益州土人反,引氐賊圍州城。惠基於外宣示朝廷威賞,於是氐人邵虎、郝天賜等斬賊帥馬興懷以降。還為太子中舍人。惠基西使千餘部曲並欲論功,惠基毀除勛簿,競無所用。或問其此意,惠基曰:「我若論其此勞,則驅馳無已,豈吾素懷之本邪?」



 出為武陵內史,中書黃門郎。惠基善隸書及弈棋,太祖與之情好相得,早相器遇。桂陽之役,惠基姊為休範妃,太祖謂之曰:「卿家桂陽遂復作賊。」太祖頓新亭壘,以惠基為軍副,惠基弟惠朗親為休範攻戰,惠基在城內了不自疑。出為豫章太守。還
 為吏部郎,遷長兼侍中。袁粲、劉秉起兵之夕,太祖以秉是惠基妹夫,時直在侍中省,遣王敬則觀其指趣,見惠基安靜不與秉相知,由是益加恩信。討沈攸之,加惠基輔國將軍,徙頓新亭。事寧,解軍號,領長水校尉。母憂去官。太祖即位,為征虜將軍,衛尉。惠基就職少時,累表陳解,見許。服闋,為征虜將軍、東陽太守,加秩中二千石。凡歷四郡,無所蓄聚。還為都官尚書,轉掌吏部。永明三年,以久疾徙為侍中,領驍騎將軍。尚書令王儉朝宗貴望,惠基同在禮閣,非公事不私覿焉。五年,遷太常,加給事中。



 自宋大明以來,聲伎所尚,多鄭衛淫俗,雅樂正聲鮮有好者。惠基解音律,尤好魏三祖曲及《相和歌》,每奏,輒賞悅不能已。當時能棋人琅邪王抗第一品,吳郡褚思莊、會稽夏赤松並第二品。赤松思速,善於大行;思莊思遲,巧於斗棋。宋文帝世,羊玄保為會稽太守,帝遣思莊入東與玄保戲,
 因製局圖,還於帝前覆之。



 太祖使思莊與王抗交賭,自食時至日暮,一局始竟。上倦,遣還省,至五更方決。



 抗睡於局後,思莊達曉不寐。世或云:「思莊所以品第致高,緣其用思深久,人不能對也。」抗、思莊並至給事中。永明中,敕抗品棋,竟陵王子良使惠基掌其事。



 初,思話先於曲阿起宅,有閑曠之致。惠基常謂所親曰:「須婚嫁畢,當歸老舊廬。」立身退素,朝廷稱為善士。明年卒,年五十九。追贈金紫光祿大夫。



 弟惠休,永明四年為廣州刺史,罷任,獻奉傾資。上敕中書舍人茹法亮曰:「可問蕭惠休。吾先使卿宣敕答其勿以私祿足充獻奉,今段殊覺其下情厚於前後人。



 問之,故當不復私邪?吾欲分受之也。」十一年,自輔國將軍、南海太守為徐州刺史。鬱林即位,進號冠軍將軍。建武二年,虜圍鐘離,惠休拒守。虜遣使仲長文真謂城中曰:「聖上方修文德,何故完城拒命?」參軍羊倫答曰:「獫狁孔
 熾,我是用急。」虜攻城,惠休拒戰破之。遷侍中,領步兵校尉,封建安縣子,五百戶。永元元年,徙吳興太守。徵為右僕射。吳興郡項羽神舊酷烈,世人云:「惠休事神謹,故得美遷。」二年,卒。贈金紫光祿大夫。



 惠休弟惠朗,善騎馬,同桂陽賊叛,太祖赦之,復加序用。永明九年為西陽王征虜長史,行南兗州事。典簽何益孫贓罪百萬,棄市,惠朗坐免官。



 史臣曰:長揖上宰,廷折公卿,古稱遺直,希之未過。若夫根孤地危,峻情不屈,則其道雖行,其身永廢。故多借路求容,遜辭自貶。高流世業,不待旁通,直轡揚鑣,莫能天閼。王秀之世守家風,不降節於權輔,美矣哉!



 贊曰:秀處邦朝,清心直己。伯寶世族,榮家為美。約守先業,觀進知止。慧曉貞亮,斯焉君子。惠基惠和,時之選士。



\end{pinyinscope}