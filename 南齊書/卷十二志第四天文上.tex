\article{卷十二志第四天文上}

\begin{pinyinscope}

 《易》曰:「聖人仰觀象於天,俯觀法於地。」天文之事,其來已久。太祖革命受終,膺集期運。



 宋升明三年,太史令將作匠陳文建陳天文,奏曰:「自孝建元年至升明三年,日蝕有十,虧上有七。占曰『有亡國失君之象。』一曰『國命絕,主危亡』。孝建元年至升明三年,太白經天五。占曰『天下革,民更王,異姓興』。孝建元年至升明三年,月犯房心四,太白犯房心五。占曰『其國有喪,宋當之』。孝建元年至永光元年,奔星出入紫宮有四。占曰『國去其君,有空國徙王』。大明二年至元徽四年,天再裂。占曰『陽不足,白虹貫日,人君惡之』。孝建二年至大明五年,月入太微;泰豫元年至昇明三年,月又入太微;孝建元年至
 元徽二年,太白入太微各八,熒惑入太微六。占曰『七耀行不軌道,危亡之象。貴人失權勢,主亦衰,當有王入為主』。孝建二年至升明二年,太白、熒惑經羽林各三。占曰『國殘更世』。孝建二年四月十三日,熒惑守南斗,成句己。占曰『天下易正更元』。孝建三年十二月一日,填星、熒惑、辰星合於南斗,占曰『改立王公』。大明二年十二月二十六日,太白犯填星于斗;六年十一月十五日,太白、填星合於危。占曰『天子失土』。景和元年十月八日,熒惑守太微,成句己。占曰『王者惡之,主命無期,有徙主,若主王,天下更紀』。泰始三年正月十七日,白氣見西南,東西半天,名曰長庚;六年九月二十七日,白氣又見東南,長二丈,並形狀長大,猛過彗星。占曰『除舊布新易主之象,遠期一紀』。至升明三年,一紀訖。泰始四年四月二十四日,太白犯填星于胃。占曰『主命惡之』。泰始七年六月十七日,太白、歲星、填星
 合於東井。



 占曰『改立王公』。元徽四年至昇明二年三月,日有頻食。占曰『社稷將亡,王者惡之』。元徽四年十月十日,填星守太微宮,逆從行歷四年。占曰『有亡君之戒,易世立王』。元徽五年七月一日,熒惑、太白、辰星合於翼。占曰『改立王公』。



 升明二年六月二十日,歲星守斗建。陰陽終始之門,大赦昇平之所起,律歷七政之本源,德星守之,天下更年,五禮更興,多暴貴者。昇明二年十月一日,熒惑守輿鬼;三年正月七日,熒惑守兩戒間,成句己。占曰『尊者失朝,必有亡國去王』。



 升明三年正月十八日,辰星孟效西方。占曰『天下更王』。升明三年四月,歲星在虛危,俳徊玄枵之野,則齊國有福厚,為受慶之符。」



 今所記三辰七曜之變,起建元訖於隆昌,以續宋史。建武世,太史奏事,明帝不欲使天變外傳,並秘而不出,自此闕焉。



 日蝕
 建元二年九月甲午朔,日蝕。



 三年七月己未朔,日蝕。



 永明元年十二月乙巳朔,日蝕。



 十年十二月癸未朔,加時在午之半度,到未初見日始蝕,虧起西北角,蝕十分之四,申時光色復還。



 隆昌元年五月甲戌合朔,巳時日蝕三分之一,午時光復還。



 月蝕建元四年七月戊辰,月在危宿蝕。



 永明二年四月丁巳,月在南斗宿蝕。



 三年十一月戊寅,月入東井曠中,因蝕三分之一。



 五年三月庚子,月在氐宿蝕。



 九月戊戌,月在胃宿蝕。



 六年九月癸巳,月蝕在婁宿九度,加時在寅之少弱,虧起東北角,蝕十五分之十一。十五日子時,蝕從東北始,至子時末都既,到丑時光色還復。



 七年八月丁亥,月在奎宿蝕。



 十月庚辰,月奄蝕熒惑。



 八年六月庚寅,月奄蝕畢左股第一星。



 十年十二月丁酉,月蝕在柳度,加時在酉之少弱,到亥時,月蝕起東角七分之二,至子時光色還復。



 永泰元年四月癸亥,月蝕,色赤如血。三日而大司馬王敬則舉兵,眾以為敬則祲烈所感。



 永元元年八月己未,月蝕盡,色皆赤。是夜,始安王遙光伏誅。



 史臣曰:日月代照,實重天行。上交下蝕,同度相掩。案舊說曰「日
 有五蝕」,謂起上下左右中央是也。交會舊術,日蝕不從東始,以月從其西,東行及日。於交中,交從外入內者,先會後交,虧西南角;先交後會,虧西北角。交從內出者,先會後交,虧西北角;先交後會,虧西南角。日正在交中者,則虧於西,故不嘗蝕東也。若日中有虧,名為黑子,不名為蝕也。漢尚書令黃香曰:「日蝕皆從西,月蝕皆從東,無上下中央者。」《春秋》魯桓三年日蝕,貫中下上竟黑。疑者以為日月正等,月何得小而見日中?鄭玄云:「月正掩日,日光從四邊出,故言從中起也。」



 王逸以為:「月若掩日,當蝕日西,月行既疾,須臾應過西崖既,復次食東崖。今察日蝕,西崖缺而光已復,過東崖而獨不掩。」逸之此意,實為巨疑。先儒難「月以望蝕,去日極遠,誰蝕月乎」?說者稱「日有暗氣,天有虛道,常與日衡相對。



 月行在虛道中,則為氣所弇,故月為蝕也。雖時加夜半,日月當子午,正隔於地,猶為暗氣所
 蝕,以天體大而地形小故也。暗虛之氣,如以鏡在日下,其光耀魄,乃見於陰中,常與日衡相對,故當星星亡,當月月蝕。」今問之曰:「星月同體,俱兆日耀,當月之蝕,星不必亡。若更有所當,星未嘗蝕。同稟異虧,其故何也?」



 答曰:「月為陰主,以當陽位,體敵勢交,自招盈損。星雖同類,而精景陋狹,小毀皆亡,無有受蝕之地,纖光可滿,亦不與弦望同形。」又難曰:「日之夜蝕,驗於夜星之亡;晝蝕既盡,晝星何故反不見?」答之曰:「夫言光有所沖,則有不衝之光矣;言有所當,亦有所不當矣。夜食度遠,與所當而同沒;晝食度近,由非沖而得明。」又問:「太白經天,實緣遠日。今度近更明,於何取喻?」答曰:「向論二蝕之體,周沖不同,經與不經,自由星遲疾。難蝕引經,恐未得也。」



 日光色建元四年十一月午時,日色赤黃無光,至暮,在箕宿。



 二年閏正月乙酉,日黃赤無光,至暮。



 永明五年十一月丁亥,日出高三竿,朱色赤黃,日暈,虹抱珥直背。



 建元元年十二月未時,日暈,匝黃白色,至申乃消散。



 永明二年正月丁酉,日交暈再重。



 三年二月丁卯,日有半暈,暈上生一珥。



 四年五月丙午,日暈再重,仍白虹貫日,在東井度。



 六年三月甲申,日於蘭雲中薄半暈,須臾過匝,日東南暈外有一直,並黃色。



 壬辰,日暈,須臾,日西北生虹貫日中。



 八年十一月己亥,日半暈,南面不匝;日東西帶暈,各生珥,長三尺,白色,珥各長十丈許,正沖日,久久消散,背因成重暈,並青絳色。



 九年正月甲午,日半暈,南面不匝;北帶暈生一抱,東西各生一珥;抱北又有半暈,抱珥並黃色;北又生白虹貫日,久久消散。



 建
 元元年六月甲申,日南北兩珥,西有抱,黃白色。



 永明二年十一月辛巳,日東北有一背。



 三年十一月庚寅,日西北有一背。



 四年正月辛巳,日南北各生一珥,又生一背。



 十二月辛未,日西北生一直,黃白色,戊寅,日北生一背,青絳色。



 五年八月己卯,日東南生一珥,並青絳色。



 六年二月丁巳,日東北生黃色,北有一珥,黃赤色,久久並散。庚申,日西有一背,赤青色,東西生一直,南北各生一珥,並黃白色。



 七年十月癸未,日東北生一背,青赤色,須臾消。



 八年六月戊寅,日於蒼白雲中南北各生一珥,青黃絳雜色,澤潤,並長三尺許,至巳午消。



 隆昌元年正月壬戌,日於蘭雲中暈,南北帶暈各生一直,同長
 一丈,須臾消。



 永元元年十二月乙酉,日中有三黑子。



 月暈犯建元四年十月庚寅,月暈五車及參頭。



 永明元年正月壬辰,是日至十五日,月三暈太微及熒惑。



 三月庚申至十三日,月三暈太微及熒惑。



 五年二月乙未,自九日至是日,月三暈太微。



 六年二月壬戌甲夜、十三日甲夜、十五日甲夜,月並暈太微。



 永明元年十一月己未,月南北各生一珥,又有一抱。



 月犯列星建元元年七月丁未,月犯心大星北一寸,丁卯,月入軒轅中犯第二星。



 十月丙申,月在心大星西北七寸。



 十一月壬戌,月在氐東南星五寸。十二月乙酉,月犯太微西蕃南頭第一星。庚寅,月行房道中,無所犯,癸巳,月入南斗魁中,無所犯。



 二年三月癸卯,月犯心大星,又犯後星。



 五月庚戌,月入南斗。七月己巳,月入南斗。



 三年二月癸巳,月犯太微上將。



 四年二月乙亥,月犯輿鬼西北星。丙子,月犯南斗魁第二星。辛未,月犯心大星,又犯後星。



 四月壬辰,月犯軒轅左民星。庚子,月犯箕東北星。五月丙寅,月犯心後星。



 戊寅,月掩昴西北星。六月乙未,月犯箕東北星。七月癸亥,月行南斗魁中,無所犯。庚辰,月犯軒轅女主。八
 月庚子,月犯昴西南星。壬寅,月犯五車東南星。壬申,月犯軒轅少民星。九月丁巳,月犯箕東北星。壬辰,月在營室度,入羽林中。



 二十日,月入輿鬼,犯積尸。



 十一月甲戌,月犯五車南星。十二月丁酉,月犯軒轅女主星,又掩女御。



 永明元年正月己亥,月犯心後星。



 三月乙未,月犯軒轅女主星。六月癸酉,月犯輿鬼西南星。八月乙丑,月犯南斗第四星,又犯輿鬼星。九月庚辰,月犯太白左蕃度。癸巳,月犯東井北轅西頭第一星。



 十二月丁卯,月犯心前星,又犯大星。己巳,月犯南斗第五星。



 二年二月甲子,月犯南斗第四星,又犯第三星。



 三月丁丑,月犯東井北轅北頭第一星。四月戊申,月犯軒轅右角。六月丙寅,月犯東井轅頭第一星。八月丙午,月掩心大星。戊申,月犯南斗第三星。戊子,月犯東井北轅西頭第一星。



 十一月庚辰,月犯昴星。丙戌,月犯軒轅左角。十二月壬戌,月犯心前星,又犯大星。



 三年二月己未,月犯南斗第五星。



 三月壬申,月在東井,無所犯。六月丙午,月掩心前星。八月丙辰,月犯東井北轅第二
 星。九月癸未,月犯東井南轅西頭第一星。



 四年正月癸酉,月入東井,無所犯。乙亥,月犯輿鬼。



 閏月辛亥,月犯房。二月丁卯,月犯東井鉞。三月乙未,月入東井,無所犯。



 七月辛亥,月犯東井。八月戊寅,月犯東井。九月辛卯,月與太白於尾合宿。丙午,月入東井。



 十一月辛丑,月入東井曠中。辛亥,月犯房北頭第二星。十二月己巳,月犯東井北轅東頭第二星。辛巳,月犯南斗第六星。



 五年正月丙午,月犯房鉤鈐。



 二月癸亥,月犯東井南轅西頭第二星。
 三月癸卯,月犯南斗第二星。六月乙丑,月犯南斗第六星,在南斗七寸。丙寅,月犯西建星北一尺。



 史臣曰:《月令》昏明中星,皆二十八宿。箕斗之間,微為疏闊。故仲春之與孟秋,建星再用,與宿度並列,亟經陵犯,災之所主,未有舊占。《石氏星經》云:「斗主爵祿,褒賢進士。故置建星以為輔。若犯建之異,不與鬥同。」則據文求義,亦宰相之占也。



 七月丁未,月行入東井曠中,無所犯。



 八月壬申,月在畢,犯左股第二星西北三寸。九月戊子,月在填星北二尺八寸,為合宿。十月戊寅,月入氐犯東南星西北一尺餘。



 十一月戊寅,月入氐。
 十二月戊午,月在東壁度,在熒惑北,相去二尺七寸,為合宿。甲子,月在東壁度東南九寸,為犯。癸酉,月在歲星南七寸,為犯。



 六年正月戊戌,月在角星南,相去三寸。



 二月丁卯,月在氐西南六寸。三月乙未,月入氐中,在歲星南一尺一寸,為合宿。四月癸丑,月犯東井南轅西頭第二星。壬戌,月在氐西南星東南五寸,為犯。



 漸入氐中,與歲星同在氐度,為合宿。癸亥,月行在房北頭第一星西南一尺,為犯。



 六月乙卯,月在角星東一寸,為犯。丁巳,月行入氐,無所犯。在歲星東三寸,為合宿。七月乙酉,月入房北頭第二次相星西北八寸,為犯。庚寅,月在牽牛中星南二寸,為犯。庚子,月行在畢左股第一星七寸,為犯,又進
 入畢。八月壬子,月行在歲星東二尺五寸,同在氐中,為合宿。九月庚辰,月在房北頭第一上相星東北一尺,為犯。又掩犯關楗閉星。丁酉,月行入東井。甲辰,月在左角星西北九寸,為犯。又在熒惑西南一尺六寸,為合宿。十月癸酉,月入氐中,在西南星東北三寸,為犯。閏月壬辰,月行入東井。



 十一月丙戌,月行入羽林中,無所犯。乙未,月行在東井南轅西頭第二星南一尺,為犯。丙寅,月在左角北八寸,為犯。辛未,月行在太白東北一尺五寸,同在箕度,為合宿。十二月甲申,月行在畢左股第二星北七寸,為犯。乙未,月行入氐西南星東北一尺,為犯。丙申,月在房北頭上相
 星北一尺,為犯。



 七年正月甲寅,月入東井曠中,無所犯。戊辰,月掩犯牽牛中星。



 二月辛巳,月掩犯東井北轅東頭第一星。三月庚申,月在歲星西北三尺,同在箕度,為合宿。四月乙酉,月入氐中,無所犯。丙戌,月犯房星北頭第一上相星北一尺,在楗閉西北四寸,為犯。六月乙酉,月犯牽牛中星。乙未,月入畢,在左股第二星東八寸,為犯。七月丁未,月入氐中,無所犯。戊申,在楗閉星東北一尺,為犯。八月甲戌,月入氐,在西南星東北一尺,為犯。庚寅,月在畢右股第一星東北一尺,為犯。九月丁巳,月掩犯畢右股第一星。庚申,月在東井北轅東頭第一星西北八寸,為犯。
 十月甲申,月行掩畢左股第三星。丁酉,月行在楗閉星西北八寸,為犯。



 十二月壬午,月在東井北轅東頭第一星北八寸,為犯。



 八年正月丁巳,月在亢南頭第二星南七寸,為犯。



 二月己巳,月行在畢右股第一星東北六寸,為犯。六月甲戌,月在亢南頭第二星西南七寸,為犯。八月乙亥,月在牽牛中星南九寸,為犯。辛卯,月在軒轅女御南八寸,為犯。九月辛酉,月在太微左執法星南四寸,為犯。十月壬午,月入東井曠中,無所犯。戊子,月在太微右執法星東南六寸,為犯。



 十一月戊戌,月行在填星北二尺二寸,為合宿。乙卯,月行在太
 微右執法星南二寸,為犯。十二月庚辰,月行在軒轅右角星南二寸,為犯。癸未,月掩犯太微右執法。



 九年正月辛丑,月在畢躔西星北六寸,為犯。庚申,月在歲星西北二尺五寸,同在須女度,為合宿。



 二月辛未,月入東井曠中,無所犯。壬申,月行東井北轅東頭第一星北九寸,為犯。三月丙申,月入畢,在左股第二星東北六寸,又掩大星。四月庚午,月在軒轅女御星南八寸,為犯。癸酉,月在太微東南頭上相星南八寸,為犯。癸未,月在歲星北,為犯,在危度。五月庚子,月行掩犯太微,在執法。丁未,月掩犯東建西星。七月癸巳,月在太白東五寸,為犯。乙未,月在太微東蕃南頭上相
 星西南五寸,為犯。壬寅,月掩犯東建星。癸卯,月在牽牛南星北五寸,為犯。乙巳,月在歲星北六寸,為犯。閏七月辛酉,月在軒轅女御星西南三寸,為犯。八月,月在軒轅左民星東八寸,為犯。九月乙丑,月掩牽牛南星。癸未,月入太微,在右執法東北四寸,為犯。甲申,月掩太微東蕃南頭上相星。十月甲午,月行在填星西北八寸,為犯,在虛度。戊申,月在軒轅女主星南四寸,掩女御,並為犯。辛亥,月入太微左執法東北七寸,為犯。



 十一月壬戌,月行掩犯歲星。己巳,月在畢右股大星東一寸,為犯。辛未,月在東井南轅西頭第二星南八寸,為犯。又入東井曠中。丙子,月行在軒轅左民星東北七寸,為犯。丁丑,月行在太微西蕃上
 將星南五寸,為犯。十二月庚寅,月行在歲星東南八寸,為犯。丙午,月掩犯太微東蕃南頭上相星。



 十年正月庚午,月在軒轅右角大民星南八寸,為犯。



 二月己亥,月行太微,在右掖門。甲辰,月行入氐中,掩犯東北星。壬子,月行入羽林。三月己卯,月行入羽林,在填星東北七寸,為犯。在危四度。四月甲午,月行入太微,在右掖門內。丙午,月行在危度,入羽林。五月己巳,月掩南斗第三星。甲戌,月行在危度,入羽林。六月戊子,月在張度,在熒惑星東三寸,為犯。



 己丑,月行入太微,在右掖門。丁酉,月掩西建星西。丁未,月行入畢,犯右股大赤星。七月甲戌,月行在畢躔星西北六寸,為犯。丁丑,月在東井北轅東
 頭第二星西南九寸,為犯。八月辛卯,月行西建星東一尺,又在東星西四寸,為犯。壬寅,月行在畢右股大赤星東北四寸,為犯。甲辰,月行入東井曠中,無所犯。戊申,月行在軒轅女主星西九寸,為犯。辛亥,月入太微,在左執法星北二尺七寸,為犯。



 九月癸亥,月行掩犯填星一寸,在危度。十月辛卯,月在危度,入羽林,無所犯。



 癸亥,月入東井曠中,無所犯。



 十一月甲子,月入畢,進右股大赤星西北五寸,為犯。壬申,月入太微,在右執法星東北一尺三寸,無所犯。丁丑,月入氐,無所犯。十二月甲午,月入東井曠中,又進北轅東頭第二星四寸,為犯。庚子,月入太微,在右執法星東北三尺,無所犯。



 十一年正月辛酉,月入東井曠中,無所犯。乙丑,月在軒轅女主星
 北八寸,為犯。壬申,月行在氐星東北九寸,為犯。



 二月甲午,月行入太微,在上將星東北一尺五寸,無所犯。壬寅,月行掩犯南斗第六星。癸卯,月掩犯西建中星,又掩東星。四月乙丑,月入太微,在右執法西北一尺四寸,無所犯。壬寅,月行在危度,入羽林,無所犯。五月丁巳,月行入太微左執法星北三尺,無所犯。甲子,月行在南斗第二星西七寸,為犯。乙丑,月掩犯西建中星。又犯東星六寸。六月辛丑,月行掩犯畢左股第三星。壬寅,月入畢。



 七月壬子,月入太微,在左執法東三尺,無所犯。丙辰,月行入氐,在東北星西南六寸,為犯。己未,月行南斗第六星南四寸,為犯。庚申,月行在西建星東南一寸,為犯。九月庚寅,月行在哭星西南六寸,為犯。壬辰,月行在營室度,入羽
 林,無所犯。丁酉,月入畢,在右股大赤星西北六寸,為犯。己亥,月入東井曠中,無所犯。乙巳,月行太微,當右掖門內,在屏星西南六寸,為犯。十月壬午,月行在東建中星九寸,為犯。



 十一月壬子,月在哭星南五寸,為犯。辛酉,月行在東井鉞星南八寸,又在東井南轅西頭第一星南五寸,並為犯。進入井中。丁卯,月入太微。壬申,月行入氐,無所犯。十二月辛巳,月入羽林,又入東井曠中,又入東井北轅西頭第二星南六寸,為犯。乙未,月入太微,在右執法星東北二尺,無所犯。乙亥,月入氐,無所犯。



 隆昌元年正月辛亥,月入畢,在左股第一星東南一尺,為犯。



 三月辛亥,月在東井北轅西頭第二星東七寸,為犯。甲申,月入太微,在
 屏星南九寸,為犯。六月乙丑,月入畢,在右股第一星東北五寸,為犯。又在歲星東南一尺,為犯。丁卯,月入東井南轅西頭第一星東北七寸,為犯。



 永元元年七月,月掩心中星。



\end{pinyinscope}