\article{卷十五 志第七 州郡下}

\begin{pinyinscope}

 荊巴郢司雍湘梁秦益寧荊州,漢靈帝中平末刺史王睿始治江陵,吳時西陵督鎮之。晉太康元年平吳,以為刺史治。愍帝建興元年,刺史周摐避杜弢賊奔建康,陶侃為刺史,治沌口。王敦治武昌。其後或還江陵,或在夏口。



 桓溫平蜀,治江陵。以臨沮西界,水陸紆險,行逕裁通,南通巴、巫,東南出州治,道帶蠻、蜒,田土肥美,立為汶陽郡,以處流民。屬氐陷襄陽,桓沖避居上明,頓陸遜樂鄉城上四十餘里,以田地肥良,可以為軍民資實,又接近三峽,無西疆之虞,故重戍江南,輕戍江北。苻堅敗後,復得襄陽。太元十四年,王忱還江陵。江陵去襄陽步道五
 百,勢同唇齒,無襄陽則江陵受敵,不立故也。自忱以來,不復動移。境域之內,含帶蠻、蜒,土地遼落,稱為殷曠。江左大鎮,莫過荊、揚。弘農郡陜縣,周世二伯總諸侯,周公主陜東,召公主陜西。故稱荊州為陜西也。領郡如左:


南郡江陵華容枝江臨沮編當陽南平郡孱陵作唐江安安南天門郡零陽澧陽臨澧漊中宜都郡夷道佷山夷陵宜昌
 南義陽郡平氏厥西河東郡聞喜松滋譙永安汶陽郡僮陽沮陽高安新興郡定襄新豐廣牧
 \gezhu{
  此下闕文}
 永寧郡長寧上黃武寧郡樂鄉長林。



 巴州,三峽險隘,山蠻寇賊,宋泰始三年,議立三巴校尉以鎮之。後省,升明二年,復置。建元二年,分荊州巴東、建平,益州巴郡為州,立刺史,而領巴東太守,又割涪陵郡屬。永明元年省,各還本屬焉。巴東郡魚復朐南浦聶陽巴渠新浦漢豐建平郡巫秭歸北井秦昌沙渠新鄉巴郡江州枳墊江臨江涪陵郡
 永寧郡長寧上黃武寧郡樂鄉長林。



 巴州,三峽險隘,山蠻寇賊,宋泰始三年,議立三巴校尉以鎮之。後省,升明二年,復置。建元二年,分荊州巴東、建平,益州巴郡為州,立刺史,而領巴東太守,又割涪陵郡屬。永明元年省,各還本屬焉。巴東郡魚復朐南浦聶陽巴渠新浦漢豐建平郡巫秭歸北井秦昌沙渠新鄉巴郡江州枳墊江臨江涪陵郡
 漢平涪陵漢玫郢州,鎮夏口,舊要害也。吳置督將為魯口屯,對魯山岸,因為名也。晉永嘉中,荊州刺史都督山簡自襄陽避賊奔夏口,庾翼為荊州,治夏口,並依地險也。太元中,荊州刺史桓沖移鎮上明,上表言:「氐賊送死之日,舊郢以北,堅壁相望,待以不戰。江州刺史桓嗣宜進屯夏口,據上下之中,於事為便。」義熙元年,冠軍將軍劉毅以為夏口二州之中,地居形要,控接湘川,邊帶溳、沔,請并州刺史劉道規鎮夏口。



 夏口城據黃鵠磯,世傳仙人子安乘黃鵠過此上也。邊江峻險,樓櫓高危,瞰臨沔、漢,應接司部,宋孝武置州於此,以分荊楚之勢。領郡如左:



 江夏郡沙陽蒲圻灄陽汝南沌陽惠懷
 竟陵郡竟陵雲杜霄城萇壽新市新陽武陵郡沅陵臨沅零陵辰陽酉陽沅南漢壽龍陽水舞陽黚陽巴陵郡下雋州陵巴陵監利武昌郡武昌鄂陽新義寧寄治鄂真陽《永明三年戶口簿》無西陽郡西陵蘄陽西陽孝寧期思《永明三年戶口簿》無義安左縣希水左縣東安左縣蘄水左縣
 齊興郡永明三年置綏懷齊康葺波綏平齊寧上蔡《永明三年戶口簿》無東䍧牱郡《永明三年戶口簿》云「新置,無屬縣」。



 宜南平陽西新市南新市西平陽東新市方城左郡城陽歸義北新陽郡西新陽安吉長寧義安左郡綏安南新陽左郡南新陽新興
 北新陽角陵新安北遂安左郡《永明三年簿》云「五縣皆缺」。



 東城綏化富城南城新安新平左郡平陽新市安城建安左郡霄城司州,鎮義陽。宋景平初,失河南地,元嘉末,僑立州於汝南縣瓠,尋罷。泰始中,立州於義陽郡。



 有三關之隘,北接陳、汝,控帶許、洛。自此以來,常為邊鎮。泰始既遷,領義陽,僑立汝南,領三郡。



 元徽四年,又領安陸、隨、安蠻三郡。領郡如左:



 南義陽郡孝昌平輿義昌平陽南安平春
 北義陽郡平陽義陽保城鄳鐘武環水隨郡隨永陽闕西安化安陸郡寄州治安陸應城新市新陽宣化汝南郡寄州治平輿北新息真陽安城南新息安陽臨汝汝南上蔡齊安郡齊安始安義城南安義昌義安淮南郡
 閣口平氏宋安左郡仰澤樂寧襄城安蠻左郡木蘭新化懷中聶陽南聶陽安蠻永寧左郡中曲陵曲陵孝懷安德東義陽左郡永寧革音威清永平東新安左郡第五南平林始平始安平林義昌固城新化西平
 新城左郡孝懷中曲南曲陵懷昌圍山左郡及刺章平北曲洛陽圍山曲陵建寧左郡建寧陽城北淮安左郡高邑南淮安左郡慕化柏源北隨安左郡濟山油潘
 東隨安左郡西隨高城牢山雍州,鎮襄陽,晉中朝荊州都督所治也。元帝以魏該為雍州,鎮酂城,襄陽別有重戍。庾翼為荊州,謀北伐,鎮襄陽。自永嘉亂,襄陽民戶流荒。咸康八年,尚書殷融言:「襄陽、石城,疆埸之地,對接荒寇。諸荒殘寄治郡縣,民戶寡少,可并合之。」朱序為雍州,於襄陽立僑郡縣,沒苻氐。氐敗,復還南,復用朱序。襄陽左右,田土肥良,桑梓野澤,處處而有。郗恢為雍州,于時舊民甚少,新戶稍多。宋元嘉中,割荊州五郡屬,遂為大鎮。疆蠻帶沔,阻以重山,北接宛、洛,平塗直至,跨對樊、沔,為鄢郢北門。



 部領蠻左,故別置蠻府焉。領郡如左:



 襄陽郡襄陽中廬已阜建昌
 南陽郡宛涅陽冠軍舞陰酈云陽許昌新野郡新野山都池陽穰交木惠懷始平郡武當武陽始平平陽廣平郡酂比陽廣平陰京兆郡鄧新豐杜魏扶風郡築陽郿汎陽
 馮翊郡鄀蓮勺高陸河南郡河南新城棘陽襄鄉河陰南天水郡略陽華陰西義成郡萬年義成建昌郡永興安寧華山郡藍田華山上黃南
 上洛郡建武中,此以下郡皆沒虜。



 上洛商北河南郡新蔡汝陰上蔡緱氏洛陽新安固始苞信弘農郡邯鄲圉盧氏順陽郡南鄉槐里清水丹水鄭順陽西汝南郡北上洛郡齊安郡齊康郡
 招義郡右五郡,不見屬縣。寧蠻府領郡如左:



 西新安郡新安汎陽安化南安義寧郡築義寧汎陽武當南陽南襄郡新安武昌建武武平北建武郡東萇秋霸北鄀高羅西萇秋平丘蔡陽郡
 樂安東蔡陽西蔡陽新化楊子新安永安郡東安樂新安西安樂勞泉安定郡思歸歸化皋亭新安士漢士頃懷化郡懷化編遂城精陽新化遂寧新陽武寧郡新安武寧懷寧新城永寧新陽郡東平林頭章新安朗城新市新陽武
 安西林義安郡郊鄉東里永明山都義寧西里義安南錫義清高安郡高安新集左義陽郡南襄城郡廣昌郡東襄城郡北襄城郡懷安郡北弘農郡
 西弘農郡析陽郡北義陽郡漢廣郡中襄城郡右十二郡沒虜。



 湘州,鎮長沙郡。湘川之奧,民豐土閑。晉永嘉元年,分荊州置,茍眺為刺史。此後三省,輒復置。



 元嘉十六年置,至今為舊鎮。南通嶺表,唇齒荊區。領郡如左:



 長沙郡臨湘羅湘陰醴陵瀏陽建寧吳昌桂陽郡
 郴臨武南平耒陽晉寧汝城零陵郡泉陵洮陽零陵祁陽觀陽永昌應陽衡陽郡湘西益陽湘鄉新康衡山營陽郡營道泠道營浦舂陵湘東郡茶陵新寧攸臨蒸重安陰山邵陵郡都梁邵陵高平武剛建興邵陽扶始興郡
 曲江桂陽仁化陽山令階含洭靈溪中宿湞陽始興臨賀郡臨賀馮乘富川封陽謝沐興安寧新開建撫寧始安郡本名始建,齊改。



 始安荔浦建陵左縣熙平永豐平樂齊熙郡梁州,鎮南鄭。魏景元四年平蜀所置也。晉永嘉元年,蜀賊沒漢中,刺史張光治魏興,三年,還漢中。



 建興元年,又為氐楊難敵所沒。桓溫平蜀,復舊土。後為譙縱所沒,縱平復舊。每失漢中,刺史輒鎮魏興。



 漢中為巴蜀捍蔽,故劉備得漢中,云「曹公雖來,無能為也」。是以
 蜀有難,漢中輒沒。雖時還復,而戶口殘耗。宋元嘉中,甄法護為氐所攻,失守。蕭思話復還漢中。後氐虜數相攻擊,關隴流民,多避難歸化,於是民戶稍實。州境與氐、胡相鄰,亦為威御之鎮。領郡如左:



 漢中郡南鄭城固沔陽西鄉西上庸魏興郡西城旬陽興晉廣昌南廣城《永元志》無廣城新興郡永元二年志無吉陽東關南新城郡房陵綏陽昌魏祁鄉閬陽樂平上庸郡
 上庸武陵齊安北巫上廉微陽新豐新安吉陽晉壽郡晉壽邵歡興安白水華陽郡宕渠華陽興宋嘉昌新巴郡新巴晉城晉安北巴西郡閬中安漢宋壽南國西國平周漢昌巴渠郡宣漢晉興始興巴渠東關始安下蒲
 懷安郡懷安義存宋熙郡興平宋安陽安元壽嘉昌《永元志》無白水郡晉壽新巴漢德益昌興安平周南上洛郡上洛商流民北豐陽渠陽義陽北上洛郡上洛商豐陽《永元志》無流民秬陽陽亭齊化西豐陽東鄴陽齊寧《永元志》無京兆新寧《永元志》無新附安康郡
 安康寧都南宕渠郡宕渠漢安宣漢宋康懷漢郡永豐綏成預德北陰平郡陰平平武南陰平郡陰平懷舊齊興郡齊興《永元志》無安昌《永元志》無鄖鄉錫安富略陽晉昌郡
 安晉宣漢吉陽萇壽東關新興延壽安樂東晉壽郡右一郡,縣邑事亡。



 弘農郡東昌魏郡略陽郡北梓潼郡廣長郡三水郡思安郡宋昌郡
 建寧郡南泉郡三巴郡江陵郡懷化郡歸寧郡東楗郡北宕渠郡宋康郡南漢郡南梓潼郡始寧郡江陽郡南部
 郡南安郡建安郡壽陽郡南陽郡宋寧郡歸化郡始安郡平南郡懷寧郡新興郡南平郡
 齊兆郡齊昌郡新化郡寧章郡鄰溪郡京兆郡義陽郡歸復郡安寧郡東宕渠郡宋安郡齊安郡
 凡四十五郡,荒或無民戶。



 秦州,晉武帝泰始五年置。舊土有秦之富,跨帶壟阪。太康省。惠帝元康七年復置。中原亂,沒胡。



 穆帝永和八年,胡偽秦州刺史王擢降,仍以為刺史,尋為苻健所破。十一年,桓溫以氐王楊國為秦州刺史,未有民土。至太元十四年,雍州刺史朱序始督秦州,則孝武所置也。寄治襄陽,未有刺史,是後雍州刺史常督之。隆安二年,郭銓始為梁、南秦州刺史,州寄治漢中。四年,桓玄督七州,但云秦州。元興元年,以苻堅子宏為北秦州刺史。自此荊州都督常督秦州,梁州常帶秦州刺史。義熙三年,以氐王楊國為北秦州刺史。十四年,置東秦州,劉義真為刺史。郭恭為梁州刺史,尹雅為秦州刺史。宋文帝為荊州都督,督秦州,又進督北秦州。州名雜出,省置不見。《永明郡國志》秦州寄治漢中南鄭,不曰南北。《元嘉計偕》亦云
 秦州,而荊州都督常督二秦,梁、南秦一刺史。是則《志》所載秦州為南秦,氐為北秦。領郡如左:



 武都郡下辯上祿陳倉略陽郡略陽臨漢安固郡安固南桓陵西扶風郡郿武功京兆郡杜藍田鄠
 南太原郡平陶始平郡始平槐里宋熙天水郡新陽河陽安定郡宋興朝那南安郡桓道中陶金城郡金城榆中臨洮襄馮
 翊郡蓮勺頻陽下邽萬年高陵隴西郡河關狄道首陽大夏仇池郡上辯倉泉白石夷安東寧郡西安北地南漢益州,鎮成都,起魏景元四年所治也。開拓夷荒,稍成郡縣,如漢之永昌,晉之雲山之類是也。蜀侯惲杜以來,四為偏據,故諸葛亮云「益州險塞,沃野天府」。劉頌亦謂「成都宜處親子弟,以為王國」。故立成都王穎,竟不之國。三峽險阻,蠻夷孔熾。西通芮芮河南,亦如漢
 武威張掖,為西域之道也。方面疆鎮,塗出萬里,晉世以處武臣。宋世亦以險遠,諸王不牧。泰始中,成都市橋忽生小洲,始康人邵碩有術數,見之曰:「洲生近市,當有貴王臨境。」永明二年,而始興王鎮為刺史。州土瑰富,西方之一都焉。領夷、齊諸郡如左:巴、涪陵二郡,見巴州:



 蜀郡成都郫牛鞞繁永昌廣漢郡雒什方新都郪伍城陽泉晉康郡江原臨邛樅陽晉樂漢嘉寧蜀郡廣漢升遷廣都墊江
 汶山郡都安齊基水晏官南陰平郡陰平綿竹南鄭南長樂東遂寧郡巴興小漢晉興德陽始康郡康晉談新成永寧郡欣平永安宜昌安興郡南漢建昌
 犍為郡僰道南安資中冶官武陽江陽郡江陽常安漢安綿水安固郡桓陵臨渭興固南苞清水沔陽南城固懷寧郡萬年西平懷道始平巴西郡閬中安漢西充國南充國漢昌平州益昌晉興東關梓潼郡涪
 梓潼漢德新興萬安西浦東江陽郡漢安安樂綿水南晉壽郡南晉壽白水南興西宕渠郡宕渠宣漢漢初東關天水郡西上邽冀宋興南新巴郡《永元志》,寄治陰平。



 新巴晉熙桓陵北陰平郡
 陰平南陽北桓陵扶風慎陽京兆綏歸新城郡下辯略陽漢陽安定扶風郡見《永元三年志》武江華陰茂陵南安郡見《永元三年志》南安華陽白水樂安桓道東宕渠獠郡宕渠平州漢初北部都尉越巂獠郡沈黎獠郡
 蠶陵令,無戶數。



 甘松獠郡始平獠郡齊開左郡齊通左郡右二左郡,建武三年置。



 寧州,鎮建寧郡,本益州南中,諸葛亮所謂不毛之地也。道遠土瘠,蠻夷眾多,齊民甚少。諸爨、氐強族,恃遠擅命,故數有土反之虞。領郡如左:



 建平郡同樂同瀨牧麻新興新定味同並萬安昆澤漏江談槁毋單存蠙
 南廣郡南廣常遷晉昌新興南朱提郡朱提漢陽堂狼南秦南褵褷郡且蘭萬壽毋斂晉樂綏寧丹南梁水郡梁水西隨毋掇勝休新豐建安驃封建寧郡新安永豐綏雲遂安麻雅臨江晉寧郡建伶連然滇池俞元穀昌秦臧雙柏雲南郡東
 古復西古復雲平邪龍西平郡西平暖江都陽西寧晉綏新城夜郎郡夜郎談柏談樂廣談東河陽郡東河陽楪榆西河陽郡比蘇建安成昌平蠻郡平蠻珣興
 古郡西中宛暖律高句町漏臥南興興寧郡青蛉弄棟西阿郡楪榆新豐遂平樂郡益寧安寧北朱提郡河陽義城宋昌郡江陽安上犍為
 永昌郡有名無民曰空荒不立永安永不建犍夏雍鄉西城博南益寧郡永明五年,刺史董仲舒啟置,領二縣,無民戶,自此已後皆然也。



 武陽綿水南犍為郡永明二年置西益郡江陽郡犍為郡永興郡永寧郡安寧
 郡右六郡,隆昌元年置。



 東朱提郡延興元年立安上郡建武三年,刺史郭安明啟置。



 贊曰:郡國既建,因州而剖。離過十三,合不踰九。分城列邑,名號殷阜。遷徙叛逆,代亡代有。



\end{pinyinscope}