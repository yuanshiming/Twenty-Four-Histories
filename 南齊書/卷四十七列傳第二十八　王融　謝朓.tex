\article{卷四十七列傳第二十八 王融 謝朓}

\begin{pinyinscope}

 王融,字元長,琅邪臨沂人也。』祖僧達,中書令,曾高並台輔。僧達答宋孝武云:「亡父亡祖,司徒司空。」父道琰,廬陵內史。母臨川太守謝惠宣女,惇敏婦人也。教融書學。融少而神明警惠,博涉有文才。舉秀才。晉安王南中郎板行參軍,坐公事免。竟陵王司徒板法曹行參軍,遷太子舍人。融以父官不通,弱年便欲紹興家業,啟世祖求自試曰:「臣聞春庚秋蟀,集候相悲,露木風榮,臨年共悅。



 夫唯動植且或有心,況在生靈而能無感?臣自奉望宮闕,沐浴恩私,拔迹庸虛,參名盛列,纓劍紫復,趨步丹墀,歲時歸來,誇榮邑里。然無勤而
 官,昔賢曾議;不任而祿,有識必譏。臣所用慷慨憤懣,不遑自晏。誠以深恩鮮報,聖主難逢,蒲柳先秋,光陰不待,貪及明時,展悉愚效,以酬陛下不世之仁。若微誠獲信,短才見序,文武吏法,唯所施用。夫君道含弘,臣術無隱,翁歸乃居中自見,充國曰『莫若老臣』。竊景前脩,敢蹈輕節,以冒不媒之鄙,式罄奉公之誠。抑又唐堯在上,不參二八,管夷吾恥之,臣亦恥之。願陛下裁覽。」遷秘書丞。



 從叔儉,初有儀同之授,融贈詩及書,儉甚奇憚之,笑謂人曰:「穰侯印詎便可解?」尋遷丹陽丞,中書郎。虜使遣求書,朝議欲不與。融上疏曰:臣側聞僉議,疑給虜書,如臣愚情,切有未喻。夫虜人面獸心,狼猛蜂毒,暴悖天經,虧違地義,逋竄燭幽,去來豳朔,綿周、漢而不悛,歷晉、宋其踰梗。豈有愛敬仁智,恭讓廉修,慚犬馬之馴心,同鷹虎之反目!設稿秣有儲,筋竿足用,必以草竊關燧,寇擾邊疆;寧容款塞卑辭,
 承衣請朔。陛下務存遵養,不時侮亡,許其膜拜之誠,納裘之贐,況復願同文軌?儻見款遣,思奉聲教;方致猜拒,將使舊邑遺逸,未知所寘,衰胡餘噍,或能自推。一令蔓草難鋤,涓流泛酌,豈直疥癢輕痾,容為心腹重患。



 抑孫武之言也,困則數罰,窘則多賞,先暴而後畏其眾者,虜之謂乎?前中原士庶,雖淪懾殊俗,至於婚葬之晨,猶巾祼為禮。而禁令苛刻,動加誅轘。于時獯粥初遷,犬羊尚結,即心徒怨,困懼成逃。自其將卒奔離,資峙銷闕,北畏勍蠕,西逼南胡,民背如崩,勢絕防斷。於是曲從物情,偽竊章服,歷年將絕,隱蔽無聞。



 既南向而泣者,日夜以覬;北顧而辭者,江淮相屬。兇謀歲窘,淺慮無方,於是稽顙郊門,問禮求樂。若來之以文德,賜之以副書,漢家軌儀,重臨畿輔,司隸傳節,復入關河,無待八百之師,不期十萬之眾,固其提漿佇俟,揮戈願倒,三秦大同,六漢一統。



 又虜前後奉使,不專漢人,必介以匈奴,備諸覘獲。且設官分職,彌見其情,抑退舊苗,扶任種戚。師保則后族馮晉國,總錄則邽姓直勒渴侯,臺鼎則丘頹、茍仁端,執政則目凌、鉗耳。至於東都羽儀,西京簪帶,崔孝伯、程虞虯久在著作,李元和、郭季祐上于中書,李思沖飾虜清官,游明根泛居顯職。今經典遠被,詩史北流,馮、李之徒,必欲遵尚;直勒等類,居致乖阻。何則?匈奴以氈騎為帷床,馳射為餱糧,冠方帽則犯沙陵雪,服左衽則風驤鳥逝。若衣以朱裳,戴之玄頍,節其揖讓,教以翔趨,必同艱桎梏,等懼冰淵,婆娑膋幰,困而不能前已。及夫春草水生,阻散馬之適,秋風木落,絕驅禽之歡,息沸唇於桑墟,別醍乳於冀俗,聽《韶雅》如龍聵,臨方丈若爰居,馮、李之徒,固得志矣,虜之凶族,其如病何?



 於是風土之思深,愎戾之情動,拂衣者連裾,抽鋒者比鏃,部落爭于下,酋渠危於上,我一舉而兼吞,卞莊
 之勢必也。且棘寶薦虞,晉疆彌盛,大鐘出智,宿氏以亡。



 帝略遠孚,無思不服,鑾光幸岱,匪暮斯朝。臣請收籍伊瀍,茲書復掌,猶取之內府,藏之外籝,於理有愜,即事何損。若狂言足採,請決敕施行。



 世祖答曰:「吾意不異卿。今所啟,比相見更委悉。」事竟不行。



 永明末,世祖欲北伐,使毛惠秀畫《漢武北伐圖》,使融掌其事。融好功名,因此上疏曰:臣聞情慉自中,事符則感,象構於始,機動斯彰。莊敬之道可宗,會揖讓其彌肅;勇烈之士足貴,應鼙鐸以增思。肇植生民,厥詳既緬,降及興運,維道有征,莫不有所因循而升皇業者也。若夫膏腴既稱,天乙知五方之富;皮幣已列,帝劉測四海之尊。異封禪之文,則升中之典攸鬯;歎輿地之圖,乃席卷之庸是立。



 伏惟陛下窮神盡聖,總極居中,偶化兩儀,均明二耀,拯玄綱於頹絕,反至道於澆淳,可謂區宇儀形,齊民先覺者也。臣亦遭逢,生此嘉運,鑿飲耕
 食,自幸唐年。而識用昏霾,經術疏淺,將捴且軸,豈蕨與薇。皇鑒燭幽,天高聽下,賞片言之或善,矜一物之失時,湔拂塵蒙,沾飾光價,拔足草廬,廁身朝序,復得拜賀歲時,瞻望日月,於臣心願,曾已畢矣。但千祀一逢,休明難再,思策金公駑,樂陳涓堨。竊習戰陣攻守之術,農桑牧藝之書,申、商、韓、墨之權,伊、周、孔、孟之道。常願待詔朱闕,俯對青蒲,請閑宴之私,談當世之務。位賤人微,徒深傾款。



 方今九服清怡,三靈和晏,木有附枝,輪無異轍,東鞮獻舞,南辮傳歌,羌、丱踰山,秦、屠越海,舌象玩委體之勤,輶譯厭瞻巡之數,固將開桂林於鳳山,創金城於西守。而蠢爾獯狄,敢仇大邦,假息關河,竊命函谷,淪故京之爽塏,變舊邑而荒涼,息反坫之儒衣,久伊川之被髮。北地殘氓,東都遺老,莫不茹泣吞悲,傾耳戴目,翹心仁政,延首王風。若試馳咫尺之書,具甄戎旅之卒,徇其墮城,納其降虜,可弗勞
 弦鏃,無待干戈。真皇王之兵,征而不戰者也。臣乞以執殳先邁,式道中原,澄澣渚之恆流,掃狼山之積霧,係單于之頸,屈左賢之膝,習呼韓之舊儀,拜鑾輿之巡幸。然後天移雲動,勒封岱宗,咸五登三,追蹤七十,百神肅警,萬國具僚,璯弁星離,玉帛雲聚,集三燭於蘭席,聆萬歲之禎聲,豈不盛哉!豈不韙哉!



 昔恆公志在伐莒,郭牙審其幽趣;魏后心存雲漢,德祖究其深言。臣愚昧,忖誠不足以知微,然伏揆聖心,規模弘遠,既圖載其事,必克就其功。臣不勝歡喜。



 圖成,上置琅邪城射堂壁上,遊幸輒觀視焉。



 九年,上幸芳林園,禊宴朝臣,使融為《曲水詩序》,文藻富麗,當世稱之。



 上以融才辯,十一年,使兼主客,接虜使房景高、宋弁。弁見融年少,問主客年幾?融曰:「五十之年,久逾其半。」因問:「在朝聞主客作《曲水詩序》。」



 景高又云:「在北聞主客此制,勝於顏延年,實願一見。」融乃示之。後日,宋弁於瑤
 池堂謂融曰:「昔觀相如《封禪》,以知漢武之德;今覽王生《詩序》,用見齊王之盛。」融曰:「皇家盛明,豈直比蹤漢武!更慚鄙制,無以遠匹相如。」上以虜獻馬不稱,使融問曰:「秦西冀北,實多駿驥,而魏主所獻良馬,乃駑駘之不若。求名檢事,殊為未孚。將旦旦信誓,有時而爽,駉駉之牧,不能復嗣?」宋弁曰:「不容虛偽之名,當是不習土地。」融曰:「周穆馬跡遍於天下,若騏驥之性,因地而遷,則造父之策,有時而躓。」弁曰:「王主客何為勤勤於千里?」融曰:「卿國既異其優劣,聊復相訪。若千里日至,聖上當駕鼓車。」弁曰:「向意既須,必不能駕鼓車也。」融曰:「買死馬之骨,亦以郭隗之故。」弁不能答。



 融自恃人地,三十內望為公輔。直中書省,夜歎曰:「鄧禹笑人。」行逢大行開,喧湫不得進。又歎曰:「車前無八騶卒,何得稱為丈夫!」



 朝廷討雍州刺史王奐,融復上疏曰:臣每覽史傳,見憂國忘家,捐生報德者,未曾不撫卷歎息,
 以為今古共情也。



 然或以片言微感,一餐小惠,參國士之眄,同布素之遊耳。豈有如臣,獨拔無聞之伍,過超非分之位,名器雙假,榮祿兩升,而宴安昃罷之晨,優游旰食之日。所以敢布丹愚,仰聞宸聽。



 今議者或以西夏為念,臣竊謂之不爾。其故何哉?陛下聖明,群臣悉力,順以制逆,上而御下,指開賞黜之言,微示生死之路,方域之人,皆相為敵。既兵威遠臨,人不自保,雖窮鳥必啄,固等命於梁鶉;困獸斯驚,終並懸於廚鹿。凱師勞飲,固不待晨。臣之寸心,獨有微願。



 自獫狁薦食,荒侮伊瀍,天道禍淫,危亡日至,母后內難,糧力外虛,謠言物情,屬當今會。若藉巫、漢之歸師,騁士卒之餘憤,取函谷如反掌,陵關塞若摧枯。



 但士非素蓄,無以即用,不教民戰,是實棄之。特希私集部曲,豫加習校。若蒙垂許,乞隸監省拘食人身,權備石頭防衛之數。臣少重名節,早習軍旅,若試而無績,伏受面欺之
 誅;用且有功,仰酬知人之哲。



 會虜動,竟陵王子良於東府募人,板融寧朔將軍、軍主。融文辭辯捷,尤善倉卒屬綴,有所造作,援筆可待。子良特相友好,情分殊常。晚節大習騎馬。才地既華,兼藉子良之勢,傾意賓客,勞問周款,文武翕習輻湊之。招集江西傖楚數百人,並有乾用。



 世祖疾篤暫絕,子良在殿內,太孫未入,融戎服絳衫,於中書省閣口斷東宮仗不得進,欲立子良。上既蘇,太孫入殿,朝事委高宗。融知子良不得立,乃釋服還省。歎曰:「公誤我。」鬱林深忿疾融,即位十餘日,收下廷尉獄,然後使中丞孔稚珪倚為奏曰:「融姿性剛險,立身浮競,動跡驚群,抗言異類。近塞外微塵,苦求將領,遂招納不逞,扇誘荒傖。狡弄聲勢,專行權利,反覆唇齒之間,傾動頰舌之內。威福自己,無所忌憚,誹謗朝政,歷毀王公。謂己才流,無所推下。事曝遠近,使融依源據答。」融辭曰:「囚實頑蔽,觸行多愆,
 但夙忝門素,得奉教君子。



 爰自總髮,迄將立年,州閭鄉黨,見許愚慎,朝廷衣冠,謂無釁咎。過蒙大行皇帝獎育之恩,又荷文皇帝識擢之重,司徒公賜預士林,安陸王曲垂眄接。既身被國慈,必欲以死自效,前後陳伐虜之計,亦仰簡先朝。今段犬羊乍擾,紀僧真奉宣先敕,賜語北邊動靜,令囚草撰符詔,于時即因啟聞,希侍鑾輿。及司徒宣敕招募,同例非一,實以戎事不小,不敢承教。續蒙軍號,賜使招集,銜敕而行,非敢虛扇。且格取亡叛,不限傖楚,『狡弄聲勢』,應有形跡;『專行權利』,又無贓賄;『反覆唇齒之間』,未審悉與誰言?『傾動頰舌之內』,不容都無主此。但聖主膺教,實所沐浴,自上《甘露頌》及《銀甕啟》、《三日詩序》、《接虜使語辭》,竭思稱揚,得非『誹謗』?且王公百司,唯賢是與,高下之敬,等秩有差,不敢踰濫,豈應『訾毀』?囚才分本劣,謬被策用,悚怍之情,夙宵兢惕,未嘗誇示里閭,彰曝遠邇,自循自省,並
 愧流言。良由緣淺寡虞,致貽囂謗。伏惟明皇臨宇,普天蒙澤,戊寅赦恩,輕重必宥。百日曠期,始蒙旬日,一介罪身,獨嬰憲劾。若事實有征,爰對有在,九死之日,無恨泉壤。」詔於獄賜死。時年二十七。臨死歎曰:「我若不為百歲老母,當吐一言。」融意欲指斥帝在東宮時過失也。



 融被收,朋友部曲參問北寺,相繼於道。融請救於子良,子良憂懼不敢救。融文集行於世。



 謝朓,字玄暉,陳郡陽夏人也。祖述,吳興太守。父緯,散騎侍郎。朓少好學,有美名,文章清麗。解褐豫章王太尉行參軍,歷隨王東中郎府,轉王儉衛軍東閣祭酒,太子舍人、隨王鎮西功曹,轉文學。



 子隆在荊州,好辭賦,數集僚友,朓以文才,尤被賞愛,流連晤對,不捨日夕。



 長史王秀之以朓年少相動,密以啟聞。世祖敕曰:「侍讀虞雲自宜恆應侍接。朓可還都。」朓道中為詩寄西府曰:「常恐鷹隼擊,秋菊委
 嚴霜。寄言罻羅者,寥廓已高翔。」遷新安王中軍記室。朓箋辭子隆曰:「朓聞潢汙之水,思朝宗而每竭;駑蹇之乘,希沃若而中疲。何則?皋壤搖落,對之惆悵;岐路東西,或以嗚悒。況乃服義徒擁,歸志莫從,邈若墜雨,飄似秋蒂。朓實庸流,行能無算,屬天地休明,山川受納,褒採一介,搜揚小善,舍耒場圃,奉筆菟園。東泛三江,西浮七澤,契闊戎旃,從容宴語。長裾日曳,後乘載脂,榮立府廷,恩加顏色。沐髮晞陽,未測涯涘;撫臆論報,早誓肌骨。不悟滄溟未運,波臣自蕩;渤澥方春,旅翮先謝。清切蕃房,寂寥舊蓽。輕舟反溯,弔影獨留,白雲在天,龍門不見。去德滋永,思德滋深。唯待青江可望,候歸艎於春渚;朱邸方開,效蓬心於秋實。如其簪履或存,衽席無改,雖復身填溝壑,猶望妻子知歸。攬涕告辭,悲來橫集。」



 尋以本官兼尚書殿中郎。隆昌初,敕朓接北使,朓自以口訥,啟讓不當,見許。



 高宗輔政,
 以朓為驃騎諮議,領記室,掌霸府文筆。又掌中書詔誥,除秘書丞,未拜,仍轉中書郎。出為宣城太守,以選復為中書郎。



 建武四年,出為晉安王鎮北諮議、南東海太守,行南徐州事。啟王敬則反謀,上甚嘉賞之。遷尚書吏部郎。朓上表三讓,中書疑朓官未及讓,以問祭酒沈約。



 約曰:「宋元嘉中,範曄讓吏部,朱修之讓黃門,蔡興宗讓中書,並三表詔答,具事宛然。近世小官不讓,遂成恆俗,恐此有乖讓意。王藍田、劉安西並貴重,初自不讓,今豈可慕此不讓邪?孫興公、孔覬並讓記室,今豈可三署皆讓邪?謝吏部今授超階,讓別有意,豈關官之大小?捴謙之美,本出人情,若大官必讓,便與詣闕章表不異。例既如此,謂都自非疑。」朓又啟讓,上優答不許。



 朓善草隸,長五言詩,沈約常云「二百年來無此詩也。」敬皇后遷祔山陵,朓撰哀策文,齊世莫有及者。



 東昏失德,江祏欲立江夏
 王寶玄,末更回惑,與弟祀密謂朓曰:「江夏年少輕脫,不堪負荷神器,不可復行廢立。始安年長入纂,不乖物望。非以此要富貴,政是求安國家耳。」遙光又遣親人劉渢密致意於朓,欲以為肺腑。朓自以受恩高宗,非渢所言,不肯答。少日,遙光以朓兼知衛尉事,朓懼見引,即以祏等謀告左興盛,興盛不敢發言。祏聞,以告遙光,遙光大怒,乃稱敕召朓,仍回車付廷尉,與徐孝嗣、祏、暄等連名啟誅朓曰:「謝朓資性險薄,大彰遠近。王敬則往構凶逆,微有誠效,自爾昇擢,超越倫伍。而溪壑無厭,著於觸事。比遂扇動內外,處處姦說,妄貶乘輿,竊論宮禁,間謗親賢,輕議朝宰,醜言異計,非可具聞。無君之心既著,共棄之誅宜及。臣等參議,宜下北里,肅正刑書。」詔:「公等啟事如此,朓資性輕險,久彰物議。直以彫蟲薄伎,見齒衣冠。昔在渚宮,構扇蕃邸,日夜縱諛,仰窺俯畫。及還京師,翻自宣露,江、漢無波,以為己功。素
 論於茲而盡,縉紳所以側目。去夏之事,頗有微誠,賞擢曲加,逾邁倫序,感悅未聞,陵競彌著。遂復矯構風塵,妄惑朱紫,詆貶朝政,疑間親賢。巧言利口,見醜前志。涓流纖孽,作戒遠圖。宜有少正之刑,以申去害之義。便可收付廷尉,肅明國典。」又使御史中丞範岫奏收朓,下獄死。時年三十六。



 朓初告王敬則,敬則女為朓妻,常懷刀欲報朓,朓不敢相見。及為吏部郎,沈昭略謂朓曰:「卿人地之美,無忝此職。但恨今日刑於寡妻。」朓臨敗嘆曰:「我不殺王公,王公由我而死。」



 史臣曰:晉世遷宅江表,人無北歸之計,英霸作輔,芟定中原,彌見金德之不競也。元嘉再略河南,師旅傾覆,自此以來,攻伐寢議。雖有戰爭,事存保境。王融生遇永明,軍國寧息,以文敏才華,不足進取,經略心旨,殷勤表奏。若使宮車未晏,有事邊關,融之報效,或不
 易限。夫經國體遠,許久為難,而立功立事,信居物右,其賈誼、終軍之流亞乎!



 贊曰:元長穎脫,拊翼將飛。時來運往,身沒志違。高宗始業,乃顧玄暉。逢昏屬亂,先蹈禍機。



\end{pinyinscope}