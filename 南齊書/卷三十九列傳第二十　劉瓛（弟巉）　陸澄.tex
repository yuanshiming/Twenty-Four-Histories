\article{卷三十九列傳第二十 劉瓛(弟巉) 陸澄}

\begin{pinyinscope}

 劉瓛,字子圭,沛國相人,晉丹陽尹惔六世孫也。祖弘之,給事中。父惠,治書御史。瓛初州辟祭酒主簿。宋大明四年,舉秀才;兄璲亦有名,先應州舉。至是別駕東海王元曾與瓛父惠書曰:「比歲賢子充秀,州閭可謂得人。」除奉朝請,不就。



 少篤學,博通《五經》。聚徒教授,常有數十人。丹陽尹袁粲於後堂夜集,巘在座,粲指庭中柳樹謂巘曰:「人謂此是劉尹時樹,每想高風;今復見卿清德,可謂不衰矣。」薦為秘書郎,不見用。除邵陵王郡主簿,安陸王國常侍,安成王撫軍行參軍,公事免。巘素無宦情,自此不復仕。除車騎行參軍,南彭城
 郡丞,尚書祠部郎,並不拜。袁粲誅,巘微服往哭,並致賻助。



 太祖踐阼,召巘入華林園談語,謂巘曰:「吾應天革命,物議以為何如?」巘對曰:「陛下誡前軌之失,加之以寬厚,雖危可安;若循其覆轍,雖安必危矣。」



 既出,帝顧謂司徒褚淵曰:「方直乃爾!學士故自過人。」敕巘使數入,而巘自非詔見,未嘗到宮門。上欲用巘為中書郎,使吏部尚書何戢喻旨。戢謂巘曰:「上意欲以鳳池相處,恨君資輕,可且就前除,少日當轉國子博士,便即後授。」巘曰:「平生無榮進意,今聞得中書郎而拜,豈本心哉!」後以母老闕養,重拜彭城郡丞,謂司徒褚淵曰:「自省無廊廟之才,所願唯保彭城丞耳。」上又以瓛兼總明觀祭酒,除豫章王驃騎記室參軍,丞如故,瓛終不就。武陵王曄為會稽太守,上欲令瓛為曄講,除會稽郡丞,學徒從之者轉眾。



 永明初,竟陵王子良請為征北司徒記室。瓛與張融、王
 思遠書曰:「奉教使恭召,會當停公事,但念生平素抱,有乖恩顧。吾性拙人間,不習仕進,昔嘗為行佐,便以不能及公事免黜,此皆眷者所共知也。量己審分,不敢期榮。夙嬰貧困,加以疏懶,衣裳容發,有足駭者。中以親老供養,褰裳徒步,脫爾逮今,二代一紀。先朝使其更自脩正,勉厲於階級之次,見其繿縷,或復賜以衣裳,袁、褚諸公咸加勸勵,終不能自反也。一不復為,安可重為哉?昔人有以冠一免不重加於首,每謂此得進止之儀。古者以賢制爵,或有秩滿而辭老,以庸制祿,或有身病而求歸者,永瞻前良,在己何若。又上下年尊,益不願居官次,廢晨昏也。先朝為此,曲申從許,故得連年不拜榮授,而帶帖薄祿。既習此歲久,又齒長疾侵,豈宜攝齋河間之聽,廁迹東平之僚?本無絕俗之操,亦非能偃蹇為高,此又諸賢所當深察者也。近奉初教,便自希得託跡於客遊之末,而固辭榮級,其故何
 耶?以古之王侯大人,或以此延四方之士,甚美者則有輻湊燕路,慕君王之義,驤鑣魏闕,高公子之仁,繼有追申、白而入楚,羨鄒枚而遊梁,吾非敢叨夫曩賢,庶欲從九九之遺蹤。既於聞道集泮不殊,而幸無職司拘礙,可得奉溫凊,展私計,志在此爾。」除步兵校尉,並不拜。



 瓛姿狀纖小,儒學冠於當時,京師士子貴遊莫不下席受業。性謙率通美,不以高名自居。遊詣故人,唯一門生持胡床隨後,主人未通,便坐問答。住在檀橋,瓦屋數間,上皆穿漏。學徒敬慕,不敢指斥,呼為青溪焉。竟陵王子良親往脩謁。七年,表世祖為瓛立館,以揚烈橋故主第給之,生徒皆賀。瓛曰:「室美為人災,此華宇豈吾宅邪?幸可詔作講堂,猶恐見害也。」未及徙居,遇病,子良遣從瓛學者彭城劉繪、順陽范縝將廚於瓛宅營齋。及卒,門人受學者並吊服臨送。時年五十六。



 瓛有至性,祖母病疽經年,手持膏藥,漬
 指為爛。母孔氏甚嚴明,謂親戚曰:「阿稱便是今世曾子。」阿稱,巘小名也。年四十餘,未有婚對。建元中,太祖與司徒褚淵為巘娶王氏女。王氏椓壁掛履,土落孔氏床上,孔氏不悅,巘即出其妻。



 及居父喪,不出廬,足為之屈,杖不能起。今上天監元年,下詔為巘立碑,謚曰貞簡先生。所著文集,皆是《禮》義,行於世。



 初,巘講《月令》畢,謂學生嚴植曰:「江左以來,陰陽律數之學廢矣。吾今講此,曾不得其仿佛。」時濟陽蔡仲熊禮學博聞,謂人曰:「凡鐘律在南,不容復得調平。昔五音金石,本在中土;今既來南,土氣偏陂,音律乖爽。」巘亦以為然。



 仲熊歷安西記室,尚書左丞。巘弟璡。



 璡字子璥。方軌正直。宋泰豫中,為明帝挽郎。舉秀才,建平王景素征北主簿,深見禮遇。邵陵王征虜安南行參軍。建元初,為武陵王曄冠軍征虜參軍。曄與僚佐飲,自割鵝炙。璡曰:「應刃落俎,膳夫之
 事,殿下親執鸞刀,下官未敢安席。」



 因起請退。與友人孔澈同舟入東,澈留目觀岸上女子,璡舉席自隔,不復同坐。豫章王太尉板行佐。兄瓛夜隔壁呼璡共語,璡不答,方下床著衣立,然後應。瓛問其久,璡曰:「向束帶未竟。」其立操如此。文惠太子召璡入侍東宮,每上事,輒削草。尋署中兵,兼記室參軍大司馬軍事。射聲校尉,卒官。



 陸澄,字彥淵,吳郡吳人也。祖邵,臨海太守。父瑗,州從事。澄少好學,博覽無所不知,行坐眠食,手不釋卷。起家太學博士,中軍衛軍府行佐,太宰參軍,補太常丞,郡主簿,北中郎行參軍。



 宋泰始初為尚書殿中郎,議皇后諱及下外,皆依舊稱姓。左丞徐爰案司馬孚議皇后不稱姓,《春秋》逆王后于齊。澄不引典據明,而以意立議,坐免官,白衣領職。郎官舊有坐杖,有名無實。澄在官積前後罰,一日并受千杖。轉通直郎,兼中書郎,尋轉兼左丞。



 泰始六年,詔
 皇太子朝賀服袞冕九章,澄與儀曹郎丘仲起議:「服冕以朝,實著經文。秦除六冕,漢明還備。魏晉以來,不欲令臣下服袞冕,故位公者加侍官。



 今皇太子禮絕群后,宜遵聖王盛典,革近代之制。」尋轉著作正員郎,兼官如故。



 除安成太守,轉劉韞撫軍長史,加綏遠將軍、襄陽太守,並不拜。仍轉劉秉後軍長史、東海太守。遷御史中丞。



 建元元年,驃騎諮議沈憲等坐家奴客為劫,子弟被劾,憲等晏然。左丞任遐奏澄不糾,請免澄官。澄上表自理曰:周稱舊章,漢言故事,爰自河雒,降逮淮海,朝之憲度,動尚先準。若乃任情違古,率意專造,豈謂酌諸故實,擇其茂典?



 案遐啟彈新除諮議參驃騎大將軍軍事沈憲、太子庶子沈曠并弟息,敕付建康,而憲被使,曠受假,俱無歸罪事狀。臣以不糾憲等為失。伏尋晉、宋左丞案奏,不乏於時,其及中丞者,從來殆無。王獻之習達朝章,近代之宗,其為左丞,彈司徒屬
 王蒙憚罰自解,屬疾遊行,初不及中丞。桓秘不奔山陵,左丞鄭襲不彈祕,直彈中丞孔欣時,又云別攝蘭臺檢校,此徑彈中丞之謂。唯左丞庾登之奏鎮北檀道濟北伐不進,致虎牢陷沒,蕃岳宰臣,引咎謝愆,而責帥之劾,曾莫奏聞,請收治道濟,免中丞何萬歲。夫山陵情敬之極,北伐專征之大,秘霸季之貴,道濟元勳之盛,所以咎及南司,事非常憲,然秘事猶非及中丞也。今若以此為例,恐人之貴賤,事之輕重,物有其倫,不可相方。



 左丞江奧彈段景文,又彈裴方明;左丞甄法崇彈蕭珍,又彈杜驥,又彈段國,又彈范文伯;左丞羊玄保又彈蕭汪;左丞殷景熙彈張仲仁;兼左丞何承天彈呂萬齡。



 並不歸罪,皆為重劾。凡茲十彈,差是憲、曠之比,悉無及中丞之議。左丞荀萬秋、劉藏、江謐彈王僧朗、王雲之、陶寶度,不及中丞,最是近例之明者。謐彈在今龕蒐之後,事行聖照。遠取十奏,近征
 二案,自宜依以為體,豈得舍而不遵?



 臣竊此人乏,謬奉國憲。今遐所糾,既行一時,若默而不言,則向為來準,後人被繩,方當追請,素餐之責,貽塵千載。所以備舉顯例,弘通國典,雖有愚心,不在微躬。請出臣表付外詳議。若所陳非謬,裁由天鑒。



 詔委外詳議。尚書令褚淵奏:「宋世左丞荀伯子彈彭城令張道欣等,坐界劫累發不禽,免道欣等官;中丞王準不糾,亦免官。左丞羊玄保彈豫州刺史管義之譙梁群盜,免義之官;中丞傅隆不糾,亦免隆官。左丞羊玄保又彈兗州刺史鄭從之濫上布及加課租綿,免從之官;中丞傅隆不糾,免隆官。左丞陸展彈建康令丘珍孫、丹陽尹孔山士劫發不禽,免珍孫、山士官;中丞何勖不糾,亦免勖官。左丞劉矇彈青州刺史劉道隆失火燒府庫,免道隆官;中丞蕭惠開不糾,免惠開官。左丞徐爰彈右衛將軍薛安都屬疾不直,免安都官;中丞張永結免。澄謏
 聞膚見,貽撓後昆,上掩皇明,下籠朝識,請以見事免澄所居官。」詔曰:「澄表據多謬,不足深劾,可白衣領職。」



 明年,轉給事中,秘書監,遷吏部。四年,復為秘書監,領國子博士。遷都官尚書。出為輔國將軍、鎮北鎮軍二府長史,廷尉,領驍騎將軍。永明元年,轉度支尚書。尋領國子博士。時國學置鄭王《易》,杜服《春秋》,何氏《公羊》,麋氏《穀梁》,鄭玄《孝經》。澄謂尚書令王儉曰:「《孝經》,小學之類,不宜列在帝典。」乃與儉書論之曰:《易》近取諸身,遠取諸物,彌天地之道,通萬物之情。自商瞿至田何,其間五傳。年未為遠,無訛雜之失;秦所不焚,無崩壞之弊。雖有異家之學,同以象數為宗。數百年後,乃有王弼。王濟云弼所悟者多,何必能頓廢前儒。若謂《易》道盡於王弼,方須大論,意者無乃仁智殊見。且《易》道無體不可以一體求,屢遷不可以一遷執也。晉太興四年,太常荀菘請置《周易》鄭玄注博士,行乎前
 代,於時政由王、庾,皆俊神清識,能言玄遠,舍輔嗣而用康成,豈其妄然。太元立王肅《易》,當以在玄、弼之間。元嘉建學之始,玄、弼兩立。逮顏延之為祭酒,黜鄭置王,意在貴玄,事成敗儒。今若不大弘儒風,則無所立學。眾經皆儒,惟《易》獨玄,玄不可棄,儒不可缺。謂宜並存,所以合無體之義。且弼於注經中已舉《繫辭》,故不復別注。今若專取弼《易》,則《繫》說無注。



 《左氏》太元取服虔,而兼取賈逵《經》,由服傳無《經》,雖在注中,而《傳》又有無《經》者故也。今留服而去賈,則《經》有所闕。案杜預注《傳》,王弼注《易》,俱是晚出,並貴後生。杜之異古,未如王之奪實,祖述前儒,特舉其違。又《釋例》之作,所弘惟深。



 《穀梁》太元舊有麋信注,顏益以范寧,麋猶如故。顏論閏分范注,當以同我者親。常謂《穀梁》劣,《公羊》為注者又不盡善。竟無及《公羊》之有何休,恐不足兩立。必謂范善,便當除麋。



 世有一《孝經》,題為鄭玄注,觀其用辭,不
 與注書相類。案玄自序所注眾書,亦無《孝經》。



 儉答曰:「《易》體微遠,實貫群籍,施、孟異聞,周、韓殊旨,豈可專據小王,便為該備?依舊存鄭,高同來說。元凱注《傳》,超邁前儒,若不列學官,其可廢矣。賈氏注《經》,世所罕習,《穀梁》小書,無俟兩注,存麋略范,率由舊式。凡此諸義,並同雅論。疑《孝經》非鄭所注,僕以此書明百行之首,實人倫所先,《七略》、《藝文》並陳之六藝,不與《蒼頡》《凡將》之流也。鄭注虛實,前代不嫌,意謂可安,仍舊立置。」



 儉自以博聞多識,讀書過澄。澄曰:「僕年少來無事,唯以讀書為業。且年已倍令君,令君少便鞅掌王務,雖復一覽便諳,然見卷軸未必多僕。」儉集學士何憲等盛自商略,澄待儉語畢,然後談所遺漏數百千條,皆儉所未睹,儉乃歎服。儉在尚書省,出巾箱幾案雜服飾,令學士隸事,事多者與之,人人各得一兩物;澄後來,更出諸人所不知事復各數條,並奪物將去。



 轉散騎常
 侍,秘書監,吳郡中正,光祿大夫。加給事中,中正如故。尋領國子祭酒。以竟陵王子良得古器,小口方腹而底平,可將七八升,以問澄,澄曰:「此名服匿,單于以與蘇武。」子良後詳視器底,有字仿佛可識,如澄所言。隆昌元年,以老疾,轉光祿大夫,加散騎常侍,未拜,卒。年七十。謚靖子。



 澄當世稱為碩學,讀《易》三年不解文義,欲撰《宋書》竟不成。王儉戲之曰:「陸公,書廚也。」家多墳籍,人所罕見。撰地理書及雜傳,死後乃出。



 澄弟鮮,得罪宋世,當死。澄於路見舍人王道隆,叩頭流血,以此見原。揚州主簿顧測以兩奴就鮮質錢,鮮死,子暉誣為賣券。澄為中丞,測與書相往反,後又箋與太守蕭糸面云:「澄欲遂子弟之非,未近義方之訓,此趨販所不為,況搢紳領袖,儒宗勝達乎?」測遂為澄所排抑,世以此少之。



 時東海王摛,亦史學博聞,歷尚書左丞。竟陵王子良校試諸學士,唯摛問無不對。永明中,天忽黃色
 照地,眾莫能解。摛雲是榮光。世祖大悅,用為永陽郡。



 史臣曰:儒風在世,立人之正道;聖哲微言,百代之通訓。洙泗既往,義乖七十;稷下橫論,屈服千人。自後專門之學興,命氏之儒起,石渠朋黨之事,白虎同異之說,《六經》五典,各信師言,嗣守章句,期乎勿失。西京儒士,莫有獨擅;東都學術,鄭賈先行。康成生炎漢之季,訓義優洽,一世孔門,褒成並軌,故老以為前脩,後生未之敢異。而王肅依經辯理,與碩相非,爰興《聖證》,據用《家語》,外戚之尊,多行晉代。江左儒門,參差互出,雖於時不絕,而罕復專家。晉世以玄言方道,宋氏以文章閒業,服膺典藝,斯風不純,二代以來,為教衰矣。建元肇運,戎警未夷,天子少為諸生,端拱以思儒業,載戢干戈,遽詔庠序。永明纂襲,克隆均校,王儉為輔,長於經禮,朝廷仰其風,胄子觀其則,由是家尋孔教,人誦儒書,執卷欣欣,此焉彌盛。建武繼立,
 因循舊緒,時不好文,輔相無術,學校雖設,前軌難追。劉瓛承馬、鄭之後,一時學徒以為師範。虎門初闢,法駕親臨,待問無五更之禮,充庭闕蒲輪之御,身終下秩,道義空存,斯故進賢之責也。其餘儒學之士,多在卑位,或隱世辭榮者,別見他篇云。



 贊曰:儒宗義肆,紛綸子圭。升堂受業,事越關西。璡居暗室,立操無攜。彥淵書史,疑問窮稽。



\end{pinyinscope}