\article{卷五十二列傳第三十三 文學}

\begin{pinyinscope}

 丘靈鞠檀超卞彬丘巨源王智深陸厥崔慰祖王逡之祖沖之
 賈淵丘靈鞠,吳興烏程人也。祖系,秘書監。靈鞠少好學,善屬文。與上計,仕郡為吏。州辟從事,詣領軍沈演之。演之曰:「身昔為州職,詣領軍謝晦,賓主坐處,政如今日,卿將來或復如此也。」舉秀才,為州主簿。累遷員外郎。



 宋孝武殷貴妃亡,靈鞠獻挽歌詩三首,雲「雲橫廣階暗,霜深高殿寒」。帝擿句嗟賞。除新安王北中郎參軍,出為剡烏程令,不得志。泰始初,坐東賊黨錮數年。



 褚淵為吳興,謂人曰:「此郡才士,唯有丘靈鞠及沈勃耳。」乃啟申之。明帝使著《大駕南討紀論》。久之,除太尉參軍,轉安北記室,帶扶風太守,不就。為尚書三公郎,建康令,轉通直郎,兼中書郎。



 昇明中,遷正員郎,領本郡中正,兼中書郎如故。時方禪讓,太祖使靈鞠參掌詔策。建元元年,轉中書郎,中正如故,敕知東宮手筆。尋又掌知國史。明年,出為鎮南長史、尋陽
 相,遷尚書左丞。世祖即位,轉通直常侍,尋領東觀祭酒。靈鞠曰:「人居官願數遷,使我終身為祭酒,不恨也。」



 永明二年,領驍騎將軍。靈鞠不樂武位,謂人曰:「我應還東掘顧榮塚。江南地方數千里,士子風流,皆出此中。顧榮忽引諸傖渡,妨我輩塗轍,死有餘罪。」



 改正員常侍。



 靈鞠好飲酒,臧否人物,在沈淵座見王儉詩,淵曰:「王令文章大進。」靈鞠曰:「何如我未進時?」此言達儉。靈鞠宋世文名甚盛,入齊頗減。蓬髮弛縱,無形儀,不治家業。王儉謂人曰:「丘公仕宦不進,才亦退矣。」遷長沙王車騎長史,太中大夫,卒。著《江左文章錄序》,起太興,訖元熙。文集行於世。



 檀超,字悅祖,高平金鄉人也。祖弘宗,宋南瑯邪太守。超少好文學,放誕任氣,解褐州西曹。嘗與別駕蕭惠開共事,不為之下。謂惠開曰:「我與卿俱起一老姥,何足相誇?」蕭太后,惠開之祖姑;長沙王道
 憐妃,超祖姑也。舉秀才。孝建初,坐事徙梁州,板宣威府參軍。孝武聞超有文章,敕還直東宮,除驃騎參軍、寧蠻主簿,鎮北諮議。超累佐蕃職,不得志,轉尚書度支郎,車騎功曹,桂陽內史。



 入為殿中郎,兼中書郎,零陵內史,征北驃騎記室,國子博士,兼左丞。



 超嗜酒,好言詠,舉止和靡,自比晉郗超為高平「二超」。謂人曰:「猶覺我為優也。」太祖賞愛之。遷驍騎將軍,常侍,司徒右長史。



 建元二年,初置史官,以超與驃騎記室江淹掌史職。上表立條例,開元紀號,不取宋年。封爵各詳本傳,無假年表。立十志:《律曆》、《禮樂》、《天文》、《五行》、《郊祀》、《刑法》、《藝文》依班固,《朝會》、《輿服》依蔡邕、司馬彪,《州郡》依徐爰。《百官》依范曄,合《州郡》。班固五星載《天文》,日蝕載《五行》;改日蝕入《天文志》。以建元為始。帝女體自皇宗,立傳以備甥舅之重,又立《處士》、《列女傳》。詔內外詳議。左僕射王儉議:「金粟之重,八政所先,食貨通則國富民實,宜
 加編錄,以崇務本。《朝會志》前史不書,蔡邕稱先師胡廣說《漢舊儀》,此乃伯喈一家之意,曲碎小儀,無煩錄。宜立《食貨》,省《朝會》。《洪範》九疇,一曰五行。五行之本,先乎水火之精,是為日月五行之宗也。今宜憲章前軌,無所改革。又立《帝女傳》,亦非淺識所安。若有高德異行,自當載在《列女》,若止於常美,則仍舊不書。」詔:「日月災隸《天文》,餘如儉議。」超史功未就,卒官。江淹撰成之,猶不備也。



 時豫章熊襄著《齊典》,上起十代。其序云:「《尚書·堯典》,謂之《虞書》,則附所述,故通謂之齊,名為《河洛金匱》。



 卞彬,字士蔚,濟陰冤句人也。祖嗣之,中領軍。父延之,有剛氣,為上虞令。



 彬才操不群,文多指刺。州辟西曹主簿,奉朝請,員外郎。宋元徽末,四貴輔政。



 彬謂太祖曰:「外間有童謠云:『可憐可念尸著服,孝子不在日代哭,列管暫鳴死滅族。』公頗聞不?」時王蘊居父憂,與袁粲同死,故云尸著服也。服者衣也,褚字邊衣也,孝除子,以
 日代者,謂褚淵也。列管,蕭也。彬退,太祖笑曰:「彬自作此。」齊臺初建,彬又曰:「誰謂宋遠,跂予望之。」太祖聞之,不加罪也。除右軍參軍。家貧,出為南康郡丞。



 彬頗飲酒,擯棄形骸。作《蚤虱賦序》曰:「余居貧,布衣十年不制。一袍之沴,有生所托,資其寒暑,無與易之。為人多病,起居甚疏,縈寢敗絮,不能自釋。



 兼攝性懈惰,懶事皮膚,澡刷不謹,浣沐失時,四體々,加以臭穢,故葦席蓬纓之間,蚤虱猥流。淫癢渭濩,無時恕肉,探揣護撮,日不替手。虱有諺言,朝生暮孫。若吾之虱者,無湯沐之慮,絕相弔之憂,宴聚乎久襟爛布之裳,服無改換,掐嚙不能加,脫略緩懶,復不勤於捕討,孫孫息息,三十五歲焉。」其略言皆實錄也。



 除南海王國郎中令,尚書比部郎,安吉令,車騎記室。彬性好飲酒,以瓠壺瓢勺杬皮為肴,著帛冠十二年不改易,以大瓠為火籠,什物多諸詭異,自稱「卞田居」,婦為「傅蠶室」。或諫曰:「卿都
 不持操,名器何由得升?」彬曰:「擲五木子,十擲輒鞬,豈復是擲子之拙。吾好擲,政極此耳。」永元中,為平越長史、綏建太守,卒官。



 彬又目禽獸云:「羊性淫而狠,豬性卑而率,鵝性頑而傲,狗性險而出。」皆指斥貴勢。其《蝦蟆賦》云:「紆青拖紫,名為蛤魚。」世謂比令僕也。又云:「科斗唯唯,群浮暗水。維朝繼夕,聿役如鬼。」比令史諮事也。文章傳於閭巷。



 永明中,瑯邪諸葛勖為國子生,作《雲中賦》,指祭酒以下,皆有形似之目。



 坐繫東冶,作《東冶徒賦》,世祖見,赦之。



 又有陳郡袁嘏,自重其文。謂人云:「我詩應須大材迮之,不爾飛去。」建武末,為諸暨令,被王敬則所殺。



 丘巨源,蘭陵蘭陵人也。宋初土斷屬丹陽,後屬蘭陵。巨源少舉丹陽郡孝廉,為宋孝武所知。大明五年,敕助徐爰撰國史。帝崩,江夏王義恭取為掌書記。明帝即位,使參詔誥,引在左右。自南臺御史為
 王景文鎮軍參軍,寧喪還家。



 元徽初,桂陽王休範在尋陽,以巨源有筆翰,遣船迎之,餉以錢物。巨源因太祖自啟,敕板起巨源使留京都。桂陽事起,使於中書省撰符檄,事平,除奉朝請。



 巨源望有封賞,既而不獲,乃與尚書令袁粲書曰:民信理推心,暗於量事,庶謂丹誠感達,賞報孱期;豈虞寂寥,忽焉三稔?議者必云筆記賤伎,非殺活所待;開勸小說,非否判所寄。然則先聲後實,軍國舊章,七德九功,將名當世。仰觀天緯,則右將而左相,俯察人序,則西武而東文,固非胥祝之倫伍,巫匠之流匹矣。



 去昔奇兵變起呼吸,雖兇渠即剿,而人情更迷。茅恬開城,千齡出叛,當此之時,心膂胡、越,奉迎新亭者,士庶填路,投名朱雀者,愚智空閨。人惑而民不惑,人畏而民不畏。其一可論也。



 臨機新亭,獨能抽刃斬賊者,唯有張敬兒;而中書省獨能奮筆弗顧者,唯有丘巨源。文武相方,誠有優劣,就其
 死亡以決成敗,當崩天之敵,抗不測之禍,請問海內,此膽何如?其二可論也。



 又爾時顛沛,普喚文士,黃門中書,靡不畢集,摛翰振藻,非為乏人,朝廷洪筆,何故假手凡賤?若以此賊強盛,勝負難測,群賢怯不染豪者,則民宜以勇獲賞;若云羽檄之難,必須筆傑,群賢推能見委者,則民宜以才賜列。其三可論也。



 竊見桂陽賊賞不赦之條凡二十五人,而李恒、鐘爽同在此例,戰敗後出,罪並釋然,而吳邁遠族誅之。罰則操筆大禍而操戈無害,論以賞科,則武人超越而文人埋沒,其四可論也。



 且邁遠置辭,無乃侵慢,民作符檄,肆言詈辱,放筆出手,即就齏粉。若使桂陽得志,民若不諲裂軍門,則應腰斬都市。嬰孩脯膾,伊可熟念。其五可論也。



 往年戎旅,萬有餘甲,十分之中,九分冗隸,可謂眾矣。攀龍附驎,翻焉雲翔。



 至若民狂夫,可謂寡矣。徒關敕旨,空然泥沈。詎其荷鷫塵末,皆是白起,操牘
 事始,必非魯連邪?民傎,國算迅足,馳烽旆之機,帝擇逸翰,赴罻羅之會。既能陵敵不殿,爭先無負,宜其微賜存在,少沾飲齕。遂乃棄之溝間,如蜉如蟻,擲之言外,如土如灰。絓隸帖戰,無拳無勇,並隨資峻級矣;凡豫臺內,不文不武,已坐拱清階矣。撫骸如此,瞻例如彼,既非草木,何能弭聲!



 巨源竟不被申。



 歷佐諸王府,轉羽林監。建元元年,為尚書主客郎,領軍司馬,越騎校尉。除武昌太守,拜竟,不樂江外行,世祖問之,巨源曰:「古人云:『寧飲建業水,不食武昌魚。』臣年已老,寧死於建業。」以為餘杭令。



 沈攸之事,太祖使巨源為尚書符荊州,巨源以此又望賞異,自此意常不滿。高宗為吳興,巨源作《秋胡詩》,有譏刺語,以事見殺。



 王智深,字雲才,琅邪臨沂人也。少從陳郡謝超宗學屬文。好飲酒,拙澀乏風儀。宋建平王景素為南徐州,作《觀法篇》,智深和之,見賞,
 辟為西曹書佐,貧無衣,未到職而景素敗。後解褐為州祭酒。太祖為鎮軍時,丘巨源薦之於太祖,板為府行參軍,除豫章王國常侍,遷太學博士,豫章王大司馬參軍,兼記室。



 世祖使太子家令沈約撰《宋書》,擬立《袁粲傳》,以審世祖。世祖曰:「袁粲自是宋家忠臣。」約又多載孝武、明帝諸鄙瀆事,上遣左右謂約曰:「孝武事跡不容頓爾。我昔經事宋明帝,卿可思諱惡之義。」於是多所省除。



 又敕智深撰《宋紀》,召見芙蓉堂,賜衣服,給宅。智深告貧於豫章王,王曰:「須卿書成,當相論以祿。」書成三十卷,世祖後召見智深於鸘明殿,令拜表奏上。



 表未奏而世祖崩。隆昌元年,敕索其書,智深遷為竟陵王司徒參軍,坐事免。江夏王鋒、衡陽王鈞並善待之。



 初,智深為司徒袁粲所接,及撰《宋紀》,意常依依。粲幼孤,祖母名其為愍孫,後慕荀粲,自改名,會稽賀喬譏之,智深於是著論。



 家貧無人事,嘗餓五日不得
 食,掘莧根食之。司空王僧虔及子志分與衣食。卒於家。



 先是陳郡袁炳,字叔明,有文學,亦為袁粲所知。著《晉書》未成,卒。



 潁川庾銑,善屬文,見賞豫章王,引至大司馬記室參軍,卒。



 陸厥,字韓卿,吳郡吳人,揚州別駕閑子也。厥少有風概,好屬文,五言詩體甚新奇。永明九年,詔百官舉士,同郡司徒左西掾顧暠之表薦焉。州舉秀才,王晏少傅主簿,遷後軍行參軍。



 永明末,盛為文章。吳興沈約、陳郡謝朓、琅邪王融以氣類相推轂。汝南周顒善識聲韻。約等文皆用宮商,以平上去入為四聲,以此制韻,不可增減,世呼為「永明體」。沈約《宋書·謝靈運傳》後又論宮商。厥與約書曰:范詹事《自序》:「性別宮商,識清濁,特能適輕重,濟艱難。古今文人,多不全了斯處,縱有會此者,不必從根本中來。」沈尚書亦云:「自靈均以來,此秘未睹。」或「暗與理合,匪由思至。張蔡曹王,曾無先覺,潘陸顏
 謝,去之彌遠。」



 大旨鈞使「宮羽相變,低昂舛節。若前有浮聲,則後須切響,一簡之內,音韻盡殊,兩句之中,輕重悉異。」辭既美矣,理又善焉。但觀歷代眾賢,似不都暗此處,而云「此秘未睹」,近於誣乎?



 案范云「不從根本中來」,尚書云「匪由思至」,斯可謂揣情謬於玄黃,擿句差其音律也。范又云「時有會此者」,尚書云「或暗與理合」,則美詠清謳,有辭章調韻者,雖有差謬,亦有會合,推此以往,可得而言。夫思有合離,前哲同所不免;文有開塞,即事不得無之。子建所以好人譏彈,士衡所以遺恨終篇。既曰遺恨,非盡美之作,理可詆訶。君子執其詆訶,便謂合理為暗。豈如指其合理而寄詆訶為遺恨邪?



 自魏文屬論,深以清濁為言,劉楨奏書,大明體勢之致,岨峿妥怗之談,操末續顛之說,興玄黃於律呂,比五色之相宣,茍此秘未睹,茲論為何所指邪?故愚謂前英已早識宮徵,但未屈曲指的,若今論
 所申。至於掩瑕藏疾,合少謬多,則臨淄所云「人之著述,不能無病」者也。非知之而不改,謂不改則不知,斯曹、陸又稱「竭情多悔,不可力彊」者也。今許以有病有悔為言,則必自知無悔無病之地;引其不了不合為暗,何獨誣其一合一了之明乎?意者亦質文時異,古今好殊,將急在情物,而緩於章句。情物,文之所急,美惡猶且相半;章句,意之所緩,故合少而謬多。義兼於斯,必非不知明矣。



 《長門》、《上林》,殆非一家之賦;《洛神》、《池鴈》,便成二體之作。



 孟堅精正,《詠史》無虧於東主;平子恢富,《羽獵》不累於憑虛。王粲《初征》,他文未能稱是;楊修敏捷,《暑賦》彌日不獻。率意寡尤,則事促乎一日;翳翳愈伏,而理賒於七步。一人之思,遲速天懸;一家之文,工拙壤隔。何獨宮商律呂,必責其如一邪?論者乃可言未窮其致,不得言曾無先覺也。



 約答曰:宮商之聲有五,文字之別累萬。以累萬之繁,配五聲之約,高
 下低昂,非思力所舉。又非止若斯而已也。十字之文,顛倒相配,字不過十,巧歷已不能盡,何況復過於此者乎?靈均以來,未經用之於懷抱,固無從得其仿佛矣。若斯之妙,而聖人不尚,何邪?此蓋曲折聲韻之巧無當於訓義,非聖哲立言之所急也。是以子云譬之「雕蟲篆刻」,云「壯夫不為」。



 自古辭人豈不知宮羽之殊,商徵之別?雖知五音之異,而其中參差變動,所昧實多,故鄙意所謂「此秘未睹」者也。以此而推,則知前世文士便未悟此處。



 若以文章之音韻,同弦管之聲曲,則美惡妍蚩,不得頓相乖反。譬由子野操曲,安得忽有闡緩失調之聲?以《洛神》比陳思他賦,有似異手之作。故知天機啟,則律呂自調;六情滯,則音律頓舛也。



 士衡雖云「炳若縟錦」,寧有濯色江波,其中復有一片是衛文之服?此則陸生之言,即復不盡者矣。韻與不韻,復有精粗,輪扁不能言,老夫亦不盡辨此。



 永元元年,始
 安王遙光反,厥父閑被誅,厥坐繫尚方。尋有赦令,厥恨父不及,感慟而卒,年二十八。文集行於世。



 會稽虞炎,永明中以文學與沈約俱為文惠太子所遇,意眄殊常。官至驍騎將軍。



 崔慰祖,字悅宗,清河東武城人也。父慶緒,永明中為梁州刺史。慰祖解褐奉朝請。父喪不食鹽,母曰:「汝既無兄弟,又未有子胤。毀不滅性,政當不進肴羞耳,如何絕鹽!吾今亦不食矣。」慰祖不得已從之。父梁州之資,家財千萬,散與宗族,漆器題為日字,日字之器,流乎遠近。料得父時假貰文疏,謂族子紘曰:「彼有,自當見還;彼無,吾何言哉!」悉火焚之。



 好學,聚書至萬卷,鄰里年少好事者來從假借,日數十帙,慰祖親自取與,未常為辭。



 為始安王撫軍墨曹行參軍,轉刑獄,兼記室。遙光好棋,數召慰祖對戲,慰祖輒辭拙,非朔望不見也。建武中,詔舉士,從兄慧景舉慰祖及平原劉孝標,並碩學。



 帝
 欲試以百里,慰祖辭不就。



 國子祭酒沈約、吏部郎謝朓嘗於吏部省中賓友俱集,各問慰祖地理中所不悉十餘事,慰祖口吃,無華辭,而酬據精悉,一座稱服之。朓歎曰:「假使班、馬復生,無以過此。」



 慰祖賣宅四十五萬,買者云:「寧有減不?」答曰:「誠慚韓伯休,何容二價。」



 買者又曰:「君但責四十六萬,一萬見與。」慰祖曰:「是即同君欺人,豈是我心乎?」



 少與侍中江祀款,及祀貴,常來候之,而慰祖不往也。與丹陽丞劉渢素善,遙光據東府反,慰祖在城內。城未潰一日,渢謂之曰:「卿有老母,宜其出矣。」命門者出之。慰祖詣闕自首,繫尚方,病卒。



 慰祖著《海岱志》,起太公迄西晉人物為四十卷,半未成。臨卒,與從弟緯書云「常欲更注遷、固二史,採《史》、《漢》所漏二百餘事,在廚簏,可檢寫之,以存大意。《海岱志》良未周悉,可寫數本,付護軍諸從事人一通,及友人任昉、徐夤、劉洋、裴揆。」又令「以棺親土,不須磚,勿設
 靈座」。時年三十五。



 王逡之,字宣約,瑯邪臨沂人也。父祖皆為郡守。逡之少禮學博聞。起家江夏王國常侍,大司馬行參軍,章安令,累至始安內史。不之官,除山陽王驃騎參軍,兼治書御史,安成國郎中,吳令。升明末,右僕射王儉重儒術,逡之以著作郎兼尚書左丞參定齊國儀禮。初,儉撰《古今喪服集記》,逡之難儉十一條。更撰《世行》五卷。轉國子博士。國學久廢,建元二年,逡之先上表立學,又兼著作,撰《永明起居注》。轉通直常侍,驍騎將軍,領博士、著作如故。出為寧朔將軍、南康相,太中、光祿大夫,加侍中。逡之率素,衣裘不浣,機案塵黑,年老,手不釋卷。建武二年卒。



 從弟圭之,有史學,撰《齊職儀》。永明九年,其子中軍參軍顥上啟曰:「臣亡父故長水校尉珪之,籍素為基,依儒習性。以宋元徽二年,被敕使纂集古設官歷代分職,凡在墳策,必盡
 詳究。是以等級掌司,咸加編錄。黜陟遷補,悉該研記。



 述章服之差,兼冠佩之飾。屬值啟運,軌度惟新。故太宰臣淵奉宣敕旨,使速洗正。



 刊定未畢,臣私門凶禍。不揆庸微,謹冒啟上。凡五十卷,謂之《齊職儀》。仰希永升天閣,長銘秘府。」詔付秘閣。



 祖沖之,字文遠,范陽薊人也。祖昌,宋大匠卿。父朔之,奉朝請。沖之少稽古,有機思。宋孝武使直華林學省,賜宅宇車服。解褐南徐州迎從事,公府參軍。



 宋元嘉中用何承天所制歷,比古十一家為密,沖之以為尚疏,乃更造新法。上表曰:臣博訪前墳,遠稽昔典,五帝躔次,三王交分,《春秋》朔氣,《紀年》薄蝕,談、遷載述,彪、固列志,魏世注歷,晉代《起居》,探異今古,觀要華戎。書契以降,二千餘稔,日月離會之徵,星度疏密之驗,專功耽思,咸可得而言也。加以親量圭尺,躬察儀漏,目盡毫釐,心窮籌策,考課推移,又曲備其詳矣。然而古歷
 疏舛,類不精密,群氏糾紛,莫審其會。尋何承天所上,意存改革,而置法簡略,今已乖遠。以臣校之,三睹厥謬,日月所在,差覺三度,二至晷景,幾失一日,五星見伏,至差四旬,留逆進退,或移兩宿。分至失實,則節閏非正;宿度違天,則伺察無準。



 臣生屬聖辰,詢逮在運,敢率愚瞽,更創新曆。謹立改易之意有二,設法之情有三。



 改易者一:以舊法一章,十九歲有七閏,閏數為多,經二百年輒差一日。節閏既移,則應改法,曆紀屢遷,實由此條。今改章法三百九十一年有一百四十四閏,令卻合周、漢,則將來永用,無復差動。其二:以《堯典》云「日短星昴,以正仲冬」。以此推之,唐世冬至日在今宿之左五十許度。漢代之初即用秦歷,冬至日在牽牛六度。漢武改立《太初曆》,冬至日在牛初。後漢四分法,冬至日在斗二十二。



 晉世姜岌以月蝕檢日,知冬至在斗十七。今參以中星,課以蝕望,冬至之日在斗十
 一。通而計之,未盈百載,所差二度。舊法並令冬至日有定處,天數既差,則七曜宿度,漸與舛訛。乖謬既著,輒應改易。僅合一時,莫能通遠。遷革不已,又由此條。今令冬至所在歲歲微差,卻檢漢注,並皆審密,將來久用,無煩屢改。



 又設法者,其一:以子為辰首,位在正北,爻應初九升氣之端,虛為北方列宿之中。元氣肇初,宜在此次。前儒虞喜,備論其義。今歷上元日度,發自虛一。其二:以日辰之號,甲子為先,歷法設元,應在此歲。而黃帝以來,世代所用,凡十一歷,上元之歲,莫值此名。今曆上元歲在甲子。其三:以上元之歲,曆中眾條,並應以此為始。而《景初曆》交會遲疾,元首有差。又承天法,日月五星,各自有元,交會遲疾,亦並置差,裁得朔氣合而已,條序紛錯,不及古意。今設法日月五緯交會遲疾,悉以上元歲首為始。群流共源,庶無乖誤。



 若夫測以定形,據以實效,懸象著明,尺表之驗
 可推,動氣幽微,寸管之候不忒。今臣所立,易以取信。但綜核始終,大存緩密,革新變舊,有約有繁。用約之條,理不自懼,用繁之意,顧非謬然。何者?夫紀閏參差,數各有分,分之為體,非不細密,臣是用深惜毫釐,以全求妙之準,不辭積累,以成永定之製,非為思而莫知,悟而弗改也。若所上萬一可採,伏願頒宣群司,賜垂詳究。



 事奏。孝武令朝士善曆者難之,不能屈。會帝崩,不施行。出為婁縣令,謁者僕射。



 初,宋武平關中得姚興指南車,有外形而無機巧,每行,使人於內轉之。昇明中,太祖輔政,使沖之追修古法。沖之改造銅機,圓轉不窮,而司方如一,馬均以來未有也。時有北人索馭飀者,亦云能造指南車,太祖使與沖之各造,使於樂遊苑對共校試,而頗有差僻,乃毀焚之。永明中,竟陵王子良好古,沖之造欹器獻之。



 文惠太子在東宮,見沖之曆法,啟世祖施行,文惠尋薨,事又寢。轉長水
 校尉,領本職。沖之造《安邊論》,欲開屯田,廣農殖。建武中,明帝使沖之巡行四方,興造大業,可以利百姓者,會連有軍事,事竟不行。



 沖之解鐘律,博塞當時獨絕,莫能對者。以諸葛亮有木牛流馬,乃造一器,不因風水,施機自運,不勞人力;又造千里船,於新亭江試之,日行百餘里。於樂遊苑造水碓磨,世祖親自臨視。又特善算。永元二年,沖之卒。年七十二。著《易》《老》《莊》義,釋《論語》《孝經》,注《九章》,造《綴述》數十篇。



 賈淵,字希鏡,平陽襄陵人也。祖弼之,晉員外郎。父匪之,驃騎參軍。世傳譜學。孝武世,青州人發古冢,銘云「青州世子,東海女郎」。帝問學士鮑照、徐爰、蘇寶生,並不能悉。淵對曰:「此是司馬越女,嫁茍晞兒。」檢訪果然。由是見遇。敕淵注郭子。泰始初,辟丹陽郡主簿,奉朝請,太學博士,安成王撫軍行參軍,出為丹徒令。昇明中,太祖嘉淵
 世學,取為驃騎參軍,武陵王國郎中令,補餘姚令。未行,仍為義興郡丞。永明初,轉尚書外兵郎,歷大司馬司徒府參軍。竟陵王子良使淵撰《見客譜》,出為句容令。



 先是譜學未有名家,淵祖弼之廣集百氏譜記,專心治業。晉太元中,朝廷給弼之令史書吏,撰定繕寫,藏秘閣及左民曹。淵父及淵三世傳學,凡十八州士族譜,合百帙七百餘卷,該究精悉,當世莫比。永明中,衛軍王儉抄次《百家譜》,與淵參懷撰定。



 建武初,淵遷長水校尉。荒傖人王泰寶買襲瑯邪譜,尚書令王晏以啟高宗,淵坐被求,當極法,子棲長謝罪,稽顙流血,朝廷哀之,免淵罪。數年,始安王遙光板撫軍諮議,不就,仍為北中郎參軍。中興元年,卒。年六十二。撰《氏族要狀》及《人名書》,並行於世。



 史臣曰:文章者,蓋情性之風標,神明之律呂也。蘊思含毫,游心內
 運,放言落紙,氣韻天成,莫不稟以生靈,遷乎愛嗜,機見殊門,賞悟紛雜。若子桓之品藻人才,仲治之區判文體,陸機辨於《文賦》,李充論於《翰林》,張視擿句褒貶,顏延圖寫情興,各任懷抱,共為權衡。屬文之道,事出神思,感召無象,變化不窮。



 俱五聲之音響,而出言異句;等萬物之情狀,而下筆殊形。吟詠規範,本之雅什,流分條散,各以言區。若陳思《代馬》群章,王粲《飛鸞》諸製,四言之美,前超後絕。少卿離辭,五言才骨,難與爭鶩。桂林湘水,平子之華篇,飛館玉池,魏文之麗篆,七言之作,非此誰先?卿、雲巨麗,升堂冠冕,張、左恢廓,登高不繼,賦貴披陳,未或加矣。顯宗之述傅毅,簡文之摛彥伯,分言制句,多得頌體。裴頠內侍,元規鳳池,子章以來,章表之選。孫綽之碑,嗣伯喈之後;謝莊之誄,起安仁之塵。顏延《楊瓚》,自比《馬督》,以多稱貴,歸莊為允。王褒《僮約》,束皙《發蒙》,滑稽之流,亦可奇瑋。五言之
 製,獨秀眾品。習玩為理,事久則瀆,在乎文章,彌患凡舊。若無新變,不能代雄。建安一體,《典論》短長互出;潘、陸齊名,機、岳之文永異。江左風味,盛道家之言:郭璞舉其靈變;許詢極其名理;仲文玄氣,猶不盡除;謝混情新,得名未盛。顏、謝並起,乃各擅奇,休、鮑後出,咸亦標世。朱藍共妍,不相祖述。今之文章,作者雖眾,總而為論,略有三體。一則啟心閑繹,托辭華曠,雖存巧綺,終致迂回。宜登公宴,本非準的。而疏慢闡緩,膏肓之病,典正可採,酷不入情。此體之源,出靈運而成也。次則緝事比類,非對不發,博物可嘉,職成拘制。或全借古語,用申今情,崎嶇牽引,直為偶說。唯睹事例,頓失精採。此則傅咸五經,應璩指事,雖不全似,可以類從。次則發唱驚挺,操調險急,雕藻淫艷,傾炫心魂。亦猶五色之有紅紫,八音之有鄭、衛。斯鮑照之遺烈也。三體之外,請試妄談。若夫委自天機,參之史傳,應思悱
 來,忽先構聚。



 言尚易了,文憎過意,吐石含金,滋潤婉切。雜以風謠,輕唇利吻,不雅不俗,獨中胸懷。輪扁斫輪,言之未盡,文人談士,罕或兼工。非唯識有不周,道實相妨。



 談家所習,理勝其辭,就此求文,終然翳奪。故兼之者鮮矣。



 贊曰:學亞生知,多識前仁。文成筆下,芬藻
 麗春。



\end{pinyinscope}