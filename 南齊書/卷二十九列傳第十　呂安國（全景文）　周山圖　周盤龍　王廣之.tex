\article{卷二十九列傳第十 呂安國(全景文) 周山圖 周盤龍 王廣之}

\begin{pinyinscope}

 呂安國,廣陵廣陵人也。宋大明末,安國以將領見任,隱重有幹局,為劉勔所稱。泰始二年,勔征殷琰於壽春,安國以建威將軍為勔軍副。眾軍擊破琰長史杜叔寶軍於橫塘,安國抄斷賊糧道,燒其運車,多所傷殺。琰眾奔退,勔遣安國追之,先至壽春。琰閉門自守,安國與輔國將軍垣閎屯據城南,於是眾軍繼至。安國勳第一,封彭澤縣男,未拜,明年,改封鐘武縣,加邑為四百戶。累至寧朔將軍、
 義陽太守。四年,又改封湘南縣男。虜陷汝南,司州失守,以安國為督司州諸軍事、寧朔將軍、司州刺史。六年,義陽立州治,仍領義陽太守。稍遷右軍將軍,假輔師將軍。元徽二年,為晉熙王征虜司馬,輔師將軍如故。轉游擊將軍。三年,出為持節、都督青兗冀三州緣淮前鋒諸軍事、輔師將軍、兗州刺史。明年,進號冠軍將軍,還為游擊將軍,加散騎常侍、征虜將軍。



 沈攸之事起,太祖以安國為湘州刺史,征虜將軍如故。先是王蘊罷州,南中郎將南陽王翽未之鎮,蘊寧朔長史庾佩玉權行州事,朝廷先遣南中郎將中兵參軍臨湘令韓幼宗領軍防州。沈攸之之難,二人各相疑阻,佩玉輒殺幼宗。平西將軍黃回至郢州,遣軍主任候伯行湘州事,又殺佩玉。候伯與回同衛將軍袁粲謀石頭事,回令候伯水軍乘舸往赴,會眾軍已至,不得入。太祖令安國至鎮,收候伯誅之。尋進號前將軍。建元元年,
 進爵,增邑六百戶。轉右衛將軍,加給事中。二年,虜寇邊,上遣安國出司州,安集民戶。詔曰:「郢、司之間,流雜繁廣,宜並加區判,定其隸屬。參詳兩州,事無專任,安國可暫往經理。」以本官使持節,總荊郢諸軍北討事,屯義陽西關。虜未至,安國移屯沔口以俟應接。改封湘鄉。



 世祖即位,授使持節、散騎常侍、平西將軍、司州刺史,領義陽太守。永明二年,徙都督南兗兗徐青冀五州諸軍事、平北將軍、南兗州刺史,仍為都督、湘州刺史。四年,湘川蠻動,安國督州兵討之。有疾,徵為光祿大夫,加散騎常侍。安國欣有文授,謂其子曰:「汝後勿作褲褶驅使,單衣猶恨不稱,當為朱衣官也。」上遣中書舍人茹法亮敕安國曰:「吾恒憂卿疾病,應有所須,勿致難也。」明年,遷都官尚書,領太子左率。六年,遷領軍將軍。安國累居將率,在朝以宿舊見遇。尋遷散騎常侍、金紫光祿大夫、兗州中正,給扶。上又敕茹法
 亮曰:「吾見呂安國疾狀,自不宜勞,且腳中既恆惡,扶人至吾前,於禮望殊成有虧,吾難敕之。其人甚諱病,卿可作私意向,其若好差不復須扶人,依例入,幸勿牽勉。」八年,卒,年六十四。贈使持節、鎮北將軍、南兗州刺史,常侍如故。給鼓吹一部。謚肅侯。



 時舊將帥又有吳郡全景文,字弘達。少有氣力,與沈攸之同載出都,到奔牛埭,於岸上息,有人相之:「君等皆方伯人,行當富貴也。」景文謂攸之曰:「富貴或可一人耳,今言皆然,此殆妄言也。」景文仍得將領為軍主。孝建初,為竟陵王驃騎行參軍,以功封漢水侯。除員外郎,積射將軍。泰始二年,為假節、寧朔將軍、冗從僕射、軍主。隨前將軍劉亮討破東賊於晉陵,除長水校尉,假輔國將軍。北討薛索兒於破釜,領水軍斷賊糧運。仍隨太祖於葛塚石梁,再戰皆有功。南賊相持未決,敕景文隸劉亮拒劉胡,攻圍力戰,身被數十創,除前軍將軍,封孝寧
 縣侯,邑六百戶。除寧朔將軍,游擊將軍,假輔師將軍,高平太守,鎮軍、安西二府司馬,驍騎將軍。元徽末,出為南豫州刺史、歷陽太守,輔國將軍如故。遷征虜將軍、南琅邪濟陰二郡太守、軍主,尋加散騎常侍。建元元年,以不預佐命,國除,授南瑯邪太守,常侍、將軍如故。遷光祿大夫,征虜將軍、臨川王征西司馬、南郡太守。



 還,累遷為給事中,光祿大夫。永明九年,卒。



 周山圖,字季寂,義興義鄉人也。少貧微,傭書自業。有氣幹,為吳郡晉陵防郡隊主。宋孝武伐太初,山圖豫勳,賜爵關中侯。兗州刺史沈僧榮鎮瑕丘,與山圖有舊,以為己建武府參軍。竟陵王誕據廣陵反,僧榮遣山圖領二百人詣沈慶之受節度,事平論勳,為中書舍人戴明寶所仰。泰始初,為殿中將軍。四方反叛,僕射王彧舉山圖將領,呼與語,甚悅,使領百舸為前驅。與軍主佼長生等攻破賊
 湖白、赭圻二城。除員外郎,加振武將軍。豫平濃湖,追賊至西陽還,明帝賞之,賜苑西宅一區。鎮軍將軍張永征薛安都於彭城,山圖領二千人迎運至武原,為虜騎所追,合戰,多所傷殺。虜圍轉急,山圖據城自固,然後更結陣死戰。突圍出,虜披靡不能禁。眾稱其勇,呼為「武原將」。及永軍大敗,山圖收散卒得千餘人,守下邳城。



 還,除給事中、冗從僕射、直閣將軍。



 山圖好酒多失,明帝數加怒誚,後遂自改。出為錢唐新城戍。是時豫州淮西地新沒虜,更於歷陽立鎮,五年,以山圖為龍驤將軍、歷陽令,領兵守城。



 初,臨海亡命田流自號「東海王」,逃竄會稽鄞縣邊海山谷中,立屯營,分布要害,官軍不能討。明帝遣直後聞人襲說降之,授流龍驤將軍,流受命,將黨與出,行達海鹽,放兵大掠而反。是冬,殺鄞令耿猷,東境大震。六年,敕山圖將兵東屯浹口,廣設購募。流為其副暨挐所殺,別帥杜連、梅
 洛生各擁眾自守。至明年,山圖分兵掩討,皆平之。



 豫章賊張鳳,聚眾康樂山,斷江劫抄。臺軍主李雙、蔡保數遣軍攻之,連年不禽。至是軍主毛寄生與鳳戰於豫章江,大敗。明帝復遣山圖討之。山圖至,先羸兵偃眾,遣幢主龐嗣厚遺鳳,要出會聚,聽以兵自衛,鳳信之。行至望蔡,山圖設伏兵於水側,擊斬鳳首,眾百餘人束首降。除寧朔將軍、漣口戍主。山圖遏漣水築西城,斷虜騎路,并以溉田。元徽三年,遷步兵校尉,加建武將軍。轉督高平下邳淮陽淮西四郡諸軍事、寧朔將軍、淮南太守。盜發桓溫塚,大獲寶物。客竊取以遺山圖,山圖不受,簿以還官。遷左中郎將。



 太祖輔政,山圖密啟曰:「沈攸之久有異圖,公宜深為之備。」太祖笑而納之。



 武陵王贊為郢州,太祖令山圖領兵衛送。世祖與晉熙王燮自郢下,以山圖為後防。



 攸之事起,世祖為西討都督,啟山圖為軍副。世祖留據盆城,眾議
 以盆城城小難固,不如還都。山圖曰:「今據中流,為四方勢援,大眾致力,川岳可為。城隍小事,不足難也。」世祖使城局參軍劉皆、陳淵委山圖以處分事。山圖斷取行旅船板,以造樓櫓,立水柵,旬日皆辦。世祖甚嘉之。授前軍將軍,加寧朔將軍,進號輔國將軍。攸之攻郢城,世祖令山圖量其形勢。山圖曰:「攸之見與鄰鄉,亟同征伐,悉其為人。性度險刻,無以結固士心。如頓兵堅城之下,適所以為離散之漸耳。」攸之既敗,平西將軍黃回乘輕舸從白服百餘人在軍前下緣流叫,盆城中恐,須臾知是回凱歸乃安。世祖謂山圖曰:「周公前言,可謂明於見事矣。」



 還都,太祖遣山圖領部曲鎮京城,鎮戍諸軍,悉受節度。遷游擊將軍,輔國如故。建元元年,封廣晉縣男,邑三百戶。出為假節、督兗青冀三州徐州東海朐山軍事、寧朔將軍、兗州刺史,百姓附之。二年進號輔國將軍。其秋,虜動,上策虜必不
 出淮陰,乃敕山圖曰:「知卿綏邊撫戎,甚有次第,應變算略,悉以相委。恐列醜未必能送死,卿丈夫無可藉手耳。」虜果寇朐山,為玄元度、盧紹之所破。虜於淮陽。是時淮北四州起義,上使山圖自淮入清,倍道應赴。敕山圖曰:「卿當盡相帥馭理,每存全重,天下事,唯同心力,山嶽可摧。然用兵當使背後無憂慮;若後冷然無橫來處,閉目痛打,無不摧碎。吾政應鑄金,待卿成勳耳。若不藉此平四州,非丈夫也。努力自運,勿令他人得上功。」會義眾已為虜所沒,山圖拔三百家還淮陰。表移東海郡治漣口,又於石鱉立陽平郡,皆見納。



 世祖踐阼,遷竟陵王鎮北司馬,帶南平昌太守,將軍如故。以盆城之舊,出入殿省,甚見親信。義鄉縣長風廟神姓鄧,先經為縣令,死遂發靈。山圖啟乞加神位輔國將軍。上答曰:「足狗肉便了事,何用階級為?」轉黃門郎,領羽林四廂直衛。



 山圖於新林立墅舍,晨
 夜往還。上謂之曰:「卿罷萬人都督,而輕行郊外。自今往墅,可以仗身自隨,以備不虞。」及疾,上手敕參問,遣醫給藥。永明元年,卒,年六十四。詔賜朝服一具,衣一襲。



 周盤龍,北蘭陵蘭陵人也。宋世土斷,屬東平郡。盤龍膽氣過人,尤便弓馬。



 泰始初,隨軍討赭圻賊,躬自鬥戰,陷陣先登。累至龍驤將軍,積射將軍,封晉安縣子,邑四百戶。元徽二年,桂陽賊起,盤龍時為冗從僕射、騎官主、領馬軍主,隨太祖頓新亭,與屯騎校尉黃回出城南,與賊對陣,尋引還城中,合力拒戰。事寧,除南東莞太守,加前軍將軍,稍至驍騎將軍。昇明元年,出為假節、督交廣二州軍事、征虜將軍、平越中郎將、廣州刺史,未之官,預平石頭。二年,沈攸之平,司州刺史姚道和懷貳被徵,以盤龍督司州軍事、司州刺史,假節、將軍如故。改封沌陽縣。太祖即位,進號右將軍。



 建元二年,虜寇
 壽春,以盤龍為軍主、假節,助豫州刺史垣崇祖決水漂漬。盤龍率輔國將軍張倪馬步軍於西澤中奮擊,殺傷數萬人,獲牛馬輜重。上聞之喜,詔曰:「醜虜送死,敢寇壽春,崇祖、盤龍正勒義勇,乘機電奮,水陸斬擊,填川蔽野。師不淹晨,西蕃剋定。斯實將率用命之功,文武爭伐之力。凡厥勳勤,宜時銓序,可符列上。盤龍愛妾杜氏,上送金釵鑷二十枚,手敕曰「餉周公阿杜」。轉太子左率。改授持節,軍主如故。



 明年,虜寇淮陽,圍角城。先是上遣軍主成買戍角城,謂人曰:「我今作角城戍,我兒當得一子。」或問其故。買曰:「角城與虜同岸,危險具多,我豈能使虜不敢南向?我若不沒虜,則應破虜。兒不作孝子,便當作世子也。」至虜圍買數重,上遣領軍將軍李安民為都督救之。敕盤龍曰:「角城漣口,賊始復進,西道便是無賊,卿可率馬步下淮陰就安民軍。鐘離船少,政可致衣仗數日糧,軍人扶淮步
 下也。」



 買與虜拒戰,手所傷殺無數,晨朝早起,手中忽見有數升血,其日遂戰死。盤龍子奉叔單馬率二百餘人陷陣,虜萬餘騎張左右翼圍繞之,一騎走還,報奉叔已沒。盤龍方食,棄箸,馳馬奮槊,直奔虜陣,自稱「周公來!」虜素畏盤龍驍名,即時披靡。時奉叔已大殺虜,得出在外,盤龍不知,乃沖東擊西,奔南突北,賊眾莫敢當。



 奉叔見其父久不出,復躍馬入陣。父子兩匹騎,縈攪數萬人,虜眾大敗。盤龍父子由是名播北國。形甚羸訥,而臨軍勇果,諸將莫逮。



 永明元年,遷征虜將軍、南瑯邪太守。三年,遷右衛將軍,加給事中。五年,轉大司馬,加征虜將軍、濟陽太守。世祖數講武,常令盤龍領馬軍,校騎騁槊。後以疾為光祿大夫。尋出為持節、都督兗州緣淮諸軍事、平北將軍、兗州刺史。進爵為侯。



 角城戍將張蒲與虜潛相構結,因大霧乘船入清中採樵,載虜二十餘人,藏伏惣下,直向城東門,防門不禁,仍登岸援
 白爭門。戍主皇甫仲賢率軍主孟靈寶等三十餘人於門拒戰,斬三人,賊眾被創赴水,而虜軍馬步至城外已三千餘人,阻塹不得進。淮陰軍主王僧慶等領五百人赴救,虜眾乃退。坐為有司所奏,詔白衣領職。八座尋奏復位。加領東平太守。



 盤龍表年老才弱,不可鎮邊,求解職,見許。還為散騎常侍、光祿大夫。世祖戲之曰:「卿著貂蟬,何如兜鍪?」盤龍曰:「此貂蟬從兜鍪中出耳。」十一年,病卒,年七十九。贈安北將軍、兗州刺史。



 子奉叔,勇力絕人,隨盤龍征討,所在為暴掠。世祖使領軍東討唐宇之,奉叔畏上威嚴,檢勒部下,不敢侵斥。為東宮直閣。鬱林在西州,奉叔密得自進。及即位,與直閣將軍曹道剛為心膂。道剛驍騎將軍,加冠軍將軍;奉叔游擊將軍,加輔國將軍:並監殿內直衛。少日,仍遷道剛為黃門郎,高宗固諫不納。奉叔善騎馬,帝從其學騎射,尤見親寵,得入後宮。尋加領淮陵太
 守、兗州中正。道剛加南濮陽太守。隆昌元年,除黃門郎,未拜,仍出為持節、都督青冀二州軍事、冠軍將軍、青州刺史。時帝謀誅宰輔,故出奉叔為外援,除道剛中軍司馬、青冀二州中正,本官如故。奉叔就帝求千戶侯,許之。高宗輔政,以為不可,封曲江縣男,三百戶,奉叔大怒,於眾中攘刀厲目,高宗說喻之,乃受。奉叔辭畢將之鎮,部伍已出。高宗慮其一出不可復制,與蕭諶謀,稱敕召奉叔於省內殺之,勇士數人拳擊久之乃死。



 啟帝云「奉叔慢朝廷」。帝不獲已,可其奏。高宗廢帝之日,道剛直閣省,蕭諶先入戶,若欲論事,兵人隨後奄進,以刀刺之,洞胸死,因進宮內廢帝。



 奉叔弟世雄,永元中為西江督護。陳顯達事後,世雄殺廣州刺史蕭季敞,稱季敞同逆,送首京師。廣州刺史顏翻討殺之。



 王廣之,字林之,沛郡相人也。少好弓馬,便捷有勇力。初為馬隊主。
 宋大明中,以功補本縣令,殿中,龍驤,強弩將軍,驃騎中兵,南譙太守。泰始初,除寧朔將軍、軍主,隸寧朔將軍劉懷珍征殷琰於壽春。琰將劉從築壘拒守,臺軍相拒移日。琰遣長史杜叔寶領五千人運車五百乘援從。懷珍遣廣之及軍主辛慶祖、黃回、千道連等要擊於橫塘。寶結營拒戰,廣之等肉薄攻營,自晡至日沒,大敗之,殺傷千餘人,遂退,燒其運車。從聞之,棄壘奔走。時合肥城反,官軍前後受敵,都督劉勔召諸軍主會議。廣之曰:「請得將軍所乘馬往平之。」勔以馬與廣之,廣之去三日,攻剋合肥賊。仍隨懷珍討淮北。



 時明帝遣青州刺史明僧暠北征至三城,為沈文秀所攻。廣之將步騎三千餘人,緣海救之,俱引退。廣之又進軍襲文秀所置長廣太守劉桃根,桃根棄城走。軍還,封安蠻縣子,三百戶。尋改蒲圻。除建威將軍、南陽太守,不之官。除越騎校尉、龍驤將軍、鐘離太守。遷為左
 軍將軍,加寧朔將軍、高平太守。又除游擊將軍,寧朔如故。加給事中,冠軍將軍。討宋建平,先登京口,改封寧都縣子,五百戶。太祖廢蒼梧,出廣之為假節、督徐州軍事、徐州刺史、鐘離太守,冠軍如故。



 沈攸之事起,廣之留京師,豫平石頭,仍從太祖頓新亭,進號征虜將軍。太祖誅黃回。回弟駟及從弟馬、兄子奴亡逸。太祖與廣之書曰:「黃回雖有微勛,而罪過轉不可容。近遂啟請御大小二輿為刺史服飾。吾乃不惜為其啟聞,政恐得輿,復求畫輪車。此外罪不可勝數,弟自悉之。今啟依法。」令廣之於江西搜捕駟等。建元元年,進爵為侯,食邑千戶。轉散騎常侍、左軍將軍。



 北虜動,明年,詔假廣之節,出淮上。廣之家在彭、沛,啟上求招誘鄉里部曲,北取彭城,上許之。以廣之為使持節、都督淮北軍事、平北將軍、徐州刺史。廣之引軍過淮,無所克獲,坐免官。尋除征虜將軍,加散騎常侍、太子右率。世
 祖即位,遷長沙王鎮軍司馬,南東海太守,司徒司馬,尋陽相,南新蔡太守,安陸王北中郎左軍司馬,廣陵太守,將軍如故。出為持節、都督徐州諸軍事、徐州刺史,將軍如故。還為光祿大夫、左將軍、司徒司馬。遷右衛將軍,轉散騎常侍,前將軍。世祖見廣之子珍國應堪大用,謂廣之曰:「卿可謂老蚌也。」廣之曰:「臣不敢辭。」



 上大笑。除游擊將軍,不拜。



 十一年,虜動,假廣之節,招募。隆昌元年,遷給事中、左衛將軍。時豫州刺史崔慧景密與虜通,有異志。延興元年,以廣之為持節、督豫州郢州之西陽司州之汝南二郡軍事、平西將軍、豫州刺史。預廢鬱林勳,增封三百戶。高宗誅害諸王,遣廣之徵安陸王子敬於江陽,給鼓吹一部。事平,仍改授使持節、散騎常侍、都督江州諸軍事、鎮南將軍、江州刺史。進封應城縣公,食邑二千戶。建武二年,虜圍司州,遣廣之持節督司州征討解圍。廣之未至百餘
 里,虜退,乃還。明年,遷侍中、鎮軍將軍,給扶。四年,卒。年七十三。追贈散騎常侍、車騎將軍,謚曰壯公。



 史臣曰:公侯捍城,守國之所資也。必須久習兵事,非一戰之力。安國等致效累朝,聲勤克舉,並識時變,咸知附托。盤龍驍勇,獨冠三軍,匈奴之憚飛將,曾不若也。壯矣哉!



 贊曰:安國舊將,協同遷社,同裨九江,翊從中夏。盤龍殺敵,洞開胡馬。廣之末年,旌旄驟把。



\end{pinyinscope}