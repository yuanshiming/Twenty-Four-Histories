\article{卷三十六列傳第十七 謝超宗 劉祥}

\begin{pinyinscope}

 謝超宗,
 陳郡陽夏人也。祖靈運,宋臨川內史。父鳳,元嘉中坐靈運事,同徙嶺南,早卒。超宗元嘉末得還。與慧休道人來往,好學,有文辭,盛得名譽。解褐奉朝請。新安王子鸞,孝武帝寵子,超宗以選補王國常侍。王母殷淑儀卒,超宗作誄奏之,帝大嗟賞,曰:「超宗殊有鳳毛,恐靈運復出。」轉新安王撫軍行參軍。



 泰始初,為建安王司徒參軍事,尚書殿中郎。三年,都令史駱宰議策秀才考格,五問並得為上,四、三為中,二為下,一不合與第。超宗議以為「片辭折獄,寸言挫眾,魯史褒貶,孔《論》興替,皆無俟繁而後秉裁。夫表事之淵,析理
 之會,豈必委牘方切治道。非患對不盡問,患以恆文弗奇。必使一通峻正,寧劣五通而常;與其俱奇,必使一亦宜採。」詔從宰議。



 遷司徒主簿,丹陽丞。建安王休仁引為司徒記室,正員郎,兼尚書左丞中郎。



 以直言忤僕射劉康,左遷通直常侍。太祖為領軍,數與超宗共屬文,愛其才翰。衛將軍袁粲聞之,謂太祖曰:「超宗開亮迥悟,善可與語。」取為長史、臨淮太守。



 粲既誅,太祖以超宗為義興太守。升明二年,坐公事免。詣東府門自通,其日風寒慘厲,太祖謂四座曰:「此客至,使人不衣自暖矣。」超宗既坐,飲酒數甌,辭氣橫出,太祖對之甚歡。板為驃騎諮議。及即位,轉黃門郎。



 有司奏撰立郊廟歌,敕司徒褚淵、侍中謝朏、散騎侍郎孔稚珪、太學博士王咺之、總明學士劉融、何法冏、何曇秀十人並作,超宗辭獨見用。



 為人仗才使酒,多所陵忽。在直省常醉,上召見,語及北方事,超宗曰:「虜動來二
 十年矣,佛出亦無如何!」以失儀出為南郡王中軍司馬。超宗怨望,謂人曰:「我今日政應為司驢。」為省司所奏,以怨望免官,禁錮十年。司徒褚淵送湘州刺史王僧虔,閣道壞,墜水;僕射王儉嘗牛驚,跣下車。超宗撫掌笑戲曰:「落水三公,墮車僕射。」前後言誚,稍布朝野。



 世祖即位,使掌國史,除竟陵王征北諮議參軍,領記室,愈不得志。超宗娶張敬兒女為子婦,上甚疑之。永明元年,敬兒誅,超宗謂丹陽尹李安民曰:「往年殺韓信,今年殺彭越,尹欲何計?」安民具啟之。上積懷超宗輕慢,使兼中丞袁彖奏曰:風聞征北諮議參軍謝超宗,根性浮險,率情躁薄,仕近聲權,務先諂狎。人裁疏黜,亟便詆賤;卒然面譽,旋而背毀。疑間台賢,每窮詭舌;訕貶朝政,必聲兇言。腹誹口謗,莫此之甚;不敬不諱,罕與為二。



 輒攝白從王永先到臺辨問「超宗有何罪過,詣諸貴皆有不遜言語,並依事列對」。



 永先列稱:「主
 人超宗恆行來詣諸貴要,每多觸忤,言語怨懟。與張敬兒周旋,許結姻好,自敬兒死後,惋歎忿慨。今月初詣李安民,語論『張敬兒不應死』。安民道:『敬兒書疏,墨迹炳然,卿何忽作此語?』其中多有不遜之言,小人不悉盡羅縷諳憶。」如其辭列,則與風聞符同。超宗罪自已彰,宜附常準。



 超宗少無士行,長習民慝。狂狡之跡,聯代所疾;迷慠之釁,累朝兼觸。刬容掃轍,久埋世表。屬聖明廣愛,忍禍舒慈,舍之憲外,許以改過。野心不悛,在宥方驕;才性無親,處恩彌戾。遂遘扇非端,空生怨懟,恣囂毒於京輔之門,揚兇悖於卿守之席。此而不翦,國章何寄?此而可貸,孰不可容?請以見事免超宗所居官,解領記室。輒勒外收付廷尉法獄治罪。超宗品第未入簡奏,臣輒奉白簡以聞。



 世祖雖可其奏,以彖言辭依違,大怒,使左丞王逡之奏曰:臣聞行父盡忠,無禮斯疾;農夫去草,見惡必耘。所以振纓稱良,
 登朝著績,未有尸位存私而能保其榮名者也。



 今月九日,治書侍御史臣司馬侃啟彈征北諮議參軍事謝超宗,稱「根性昏動,率心險放,悖議爽真,囂辭犯實,親朋忍聞,衣冠掩目,輒收付廷尉法獄治罪」。



 處劾雖重,文辭簡略,事入主書,被卻還外。其晚,兼御史中丞臣袁彖改奏白簡,始粗詳備。厥初隱衛,實彖之由。



 尋超宗植性險戾,稟行凶詖,豺狼野心,久暴遐邇。張敬兒潛圖反噬,罰未塞愆,而稱怨痛枉,形于言貌。協附姦邪,疑間勳烈,構扇異端,譏議時政,行路同忿,有心咸疾。而阿昧茍容,輕文略奏。又彈事舊體,品第不簡,而釁戾殊常者,皆命議親奏,以彰深愆。況超宗罪逾四凶,過窮南竹,雖下輒收,而文止黃案,沈浮互見,輕重相乖,此而不糾,憲綱將替。



 彖才識疏淺,質幹無聞,憑戚升榮,因慈荷任。不能克己厲情,少酬恩獎,撓法容非,用申私惠。何以糾正邦違,式明王度?臣等參議,
 請以見事免彖所居官,解兼御史中丞,輒攝曹依舊下禁止視事如故。



 治書侍御史臣司馬侃雖承稟有由,而初無疑執,亦合及咎。請杖督五十,奪勞百日。令史卑微,不足申盡,啟可奉行。



 侃奏彈之始,臣等並即經見加推糾,案入主書,方被卻檢,疏謬之愆,伏追震悚。



 詔曰:「超宗釁同大逆,罪不容誅。彖匿情欺國,愛朋罔主,事合極法,特原收治,免官如案,禁錮十年。」



 超宗下廷尉,一宿髮白皓首。詔徙越州,行至豫章,上敕豫章內史虞悰曰:「謝超宗令於彼賜自盡,勿傷其形骸。」



 明年,超宗門生王永先又告超宗子才卿死罪二十餘條。上疑其虛妄,以才卿付廷尉辯,以不實見原。永先於獄自盡。



 劉祥,字顯征,東莞莒人也。祖式之,吳郡太守。父敳,太宰從事中郎。祥宋世解褐為巴陵王征西行參軍,歷驃騎中軍二府,太祖太尉東閣祭酒,驃騎主簿。建元中,為冠軍征虜功曹,為府主武陵王曄
 所遇。除正員外。



 祥少好文學,性韻剛疏,輕言肆行,不避高下。司徒褚淵入朝,以腰扇鄣日,祥從側過,曰:「作如此舉止,羞面見人,扇鄣何益?」淵曰:「寒士不遜。」祥曰:「不能殺袁、劉,安得免寒士?」永明初,遷長沙王鎮軍,板諮議參軍,撰《宋書》,譏斥禪代,尚書令王儉密以啟聞,上銜而不問。歷鄱陽王征虜,豫章王大司馬諮議,臨川王驃騎從事中郎。



 祥兄整為廣州,卒官,祥就整妻求還資,事聞朝廷。於朝士多所貶忽。王奐為僕射,祥與奐子融同載,行至中堂,見路人驅驢,祥曰:「驢!汝好為之,如汝人才,皆已令僕。」著《連珠》十五首以寄其懷。辭曰:蓋聞興教之道,無尚必同;拯俗之方,理貴袪弊。故揖讓之禮,行乎堯舜之朝;干戈之功,盛於殷周之世。清風以長物成春,素霜以凋嚴戒節。



 蓋聞鼓篸懷音,待揚桴以振響;天地涵靈,資昏明以垂位。是以俊乂之臣,借湯、武而隆;英達之君,假伊、周而治。



 蓋聞懸
 饑在歲,式羨藜藿之飽;重炎灼體,不念狐白之溫。故才以偶時為劭;道以調俗為尊。



 蓋聞習數之功,假物可尋;探索之明,循時則缺。故班匠日往,繩墨之伎不衰;大道常存,機神之智永絕。



 蓋聞理定於心,不期俗賞;情貫於時,無悲世辱。故芬芳各性;不待汨渚之哀;明白為寶,無假荊南之哭。



 蓋聞百仞之臺,不挺陵霜之木;盈尺之泉,時降夜光之寶。故理有大而乖權;物有微而至道。



 蓋聞忠臣赴節,不必在朝;列士匡時,義存則乾。故包胥垂涕,不荷肉食之謀;王歜投身,不主廟堂之算。



 蓋聞智出乎身,理無或困;聲係於物,才有必窮。故陵波之羽,不能凈浪;盈岫之木,無以輟風。



 蓋聞良寶遇拙,則奇文不顯;達士逢讒,則英才滅耀。故墜葉垂蔭,明月為之隔輝;堂宇留光,蘭燈有時不照。



 蓋聞跡慕近方,必勢遺於遠大;情係驅馳,固理忘於肥遁。是以臨川之士,時結羨網之悲;負肆之氓,不抱
 屠龍之歎。



 蓋聞數之所隔,雖近則難;情之所符,雖遠則易。是以陟嘆流霜,時獲感天之誠;泣血從刑,而無悟主之智。



 蓋聞妙盡於識,神遠則遺;功接於人,情微則著。故鐘鼓在堂,萬夫傾耳;大道居身,有時不遇。



 蓋聞列草深岫,不改先冬之悴;植松澗底,無奪後凋之榮。故展禽三黜,而無下愚之譽;千秋一時,而無上智之聲。



 蓋聞希世之寶,違時則賤;偉俗之器,無聖必淪。故鳴玉黜於楚岫,章甫窮於越人。



 蓋聞聽絕於聰,非疾響所達;神閉於明,非盈光所燭。故破山之雷,不發聾夫之耳;朗夜之輝,不開矇叟之目。



 有以祥《連珠》啟上者,上令御史中丞任遐奏曰:「祥少而狡異,長不悛徙,請謁絕於私館,反唇彰於公庭,輕議乘輿,歷貶朝望,肆醜無避,縱言自若。厥兄浮櫬,天倫無一日之悲,南金弗獲,嫂侄致其輕絕,孤舟夐反,存沒相捐,遂令暴客掠奪骸柩,行路流嘆,有識傷心。攝祥門生孫狼
 兒列『祥頃來飲酒無度,言語闌逸。道說朝廷,亦有不遜之語,實不避左右,非可稱紙墨。兄整先為廣州,於職喪亡,去年啟求迎喪,還至大雷,聞祥與整妻孟爭計財物瞋忿,祥仍委前還,後未至鵲頭,其夜遭劫,內人並為凶人所淫略』。如所列與風聞符同。請免官付廷尉。」



 上別遣敕祥曰:「卿素無行檢,朝野所悉。輕棄骨肉,侮蔑兄嫂,此是卿家行不足,乃無關他人。卿才識所知,蓋何足論。位涉清途,於分非屈。何意輕肆口噦,詆目朝士,造席立言,必以貶裁為口實?冀卿年齒已大,能自感厲,日望悛革。如此所聞,轉更增甚,諠議朝廷,不避尊賤,肆口極辭,彰暴物聽。近見卿影《連珠》,寄意悖慢,彌不可長。卿不見謝超宗,其才地二三,故在卿前,事殆是百分不一。



 我當原卿性命,令卿萬里思愆。卿若能改革,當令卿得還。」獄鞫祥辭。祥對曰:「被問『少習狡異,長而不悛,頃來飲酒無度,輕議乘輿,歷貶
 朝望,每肆醜言,無避尊賤』,迂答奉旨。囚出身入官,二十餘年,沈悴草萊,無明天壤。皇運初基,便蒙抽擢,祭酒主簿,並皆先朝相府。聖明御宇,榮渥彌隆,諮議中郎,一年再澤。



 廣筵華宴,必參末列,朝半問訊,時奉天輝。囚雖頑愚,豈不識恩?有何怨望,敢生譏議?囚歷府以來,伏事四王:武陵功曹,凡涉二載;長沙諮議,故經少時;奉隸大司馬,並被恩拂,驃騎中郎,親職少日;臨川殿下不遺蟲蟻,賜參辭華。司徒殿下文德英明,四海傾屬。囚不涯卑遠,隨例問訊,時節拜覲,亦沾眄議。自餘令王,未被祗拜,既不經伏節,理無厚薄。敕旨制書,令有疑則啟。囚以天日懸遠,未敢塵穢。私之疑事,衛將軍臣儉,宰輔聖朝,令望當世,囚自斷才短,密以諮儉,儉為折衷,紙跡猶存。未解此理云何敢為『歷貶朝望』。云囚『輕議乘輿』,為向誰道?若向人道,則應有主甲,豈有事無仿佛,空見羅謗?囚性不耐酒,親知所悉,
 強進一升,便已迷醉。」其餘事事自申。乃徙廣州。祥至廣州,不得意,終日縱酒,少時病卒,年三十九。



 祥從祖兄彪,祥曾祖穆之正胤。建元初,降封南康縣公,虎賁中郎將。永明元年,坐廟墓不修削爵。後為羽林監。九年,又坐與亡弟母楊別居,不相料理,楊死不殯葬,崇聖寺尼慧首剃頭為尼,以五百錢為買棺材,以泥洹輿送葬劉墓。為有司所奏,事寢不出。



 史臣曰:魏文帝云「文人不護細行」,古今之所同也。由自知情深,在物無競,身名之外,一概可蔑。既徇斯道,其弊彌流,聲裁所加,取忤人世。向之所以貴身,翻成害已。故通人立訓,為之而不恃也。



 贊曰:超宗蘊文,祖構餘芬。劉祥慕異,言亦不群。違朝失典,流放南濆。



\end{pinyinscope}