\article{卷三十二列傳第十三 王琨 張岱 褚炫 何戢 王延之 阮韜}

\begin{pinyinscope}

 王琨,瑯邪臨沂人也。祖薈,晉衛將軍。父懌,不慧,侍婢生琨,名為昆侖。



 懌後娶南陽樂玄女,無子,改琨名,立以為嗣。琨少謹篤,為從伯司徒謐所愛。宋永初中,武帝以其娶桓脩女,除郎中,駙馬都尉,奉朝請。元嘉初,從兄侍中華有權寵,以門戶衰弱,待琨如親,數相稱
 薦。為尚書儀曹郎,州治中。累至左軍諮議,領錄事,出為宣城太守,司徒從事中郎,義興太守。歷任皆廉約。還為北中郎長史,黃門郎,寧朔將軍,東陽太守。孝建初,遷廷尉卿,竟陵王驃騎長史,加臨淮太守,轉吏部郎。吏曹選局,貴要多所屬請,琨自公卿下至士大夫,例為用兩門生。江夏王義恭嘗屬琨用二人,後復遣屬琨,答不許。



 出為持節、都督廣交二州軍事、建威將軍、平越將軍、平越中郎將、廣州刺史。



 南土沃實,在任者常致巨富,世云「廣州刺史但經城門一過,便得三千萬」也。琨無所取納,表獻祿俸之半。州鎮舊有鼓吹,又啟輸還。及罷任,孝武知其清,問還資多少?琨曰:「臣買宅百三十萬,餘物稱之。」帝悅其對。為廷尉,加給事中,轉寧朔將軍長史、歷陽內史。上以琨忠實,徙為寵子新安王東中郎長史,加輔國將軍,遷右衛將軍,度支尚書。出為永嘉王左軍、始安王征虜二府長史,
 加輔國將軍、廣陵太守,皆孝武諸子。泰始元年,遷度支尚書,尋加光祿大夫。



 初,從兄華孫長襲華爵為新建侯,嗜酒多愆失。琨上表曰:「臣門姪不休,從孫長是故左衛將軍嗣息,少資常猥,猶冀晚進。頃更昏酣,業身無檢。故衛將軍華忠肅奉國,善及世祀;而長負釁承對,將傾基緒。嗣小息佟閑立保退,不乖素風,如蒙拯立,則存亡荷榮,私祿更構。」出為冠軍將軍、吳郡太守,遷中領軍。坐在郡用朝舍錢三十六萬營餉二宮諸王及作絳襖奉獻軍用,左遷光祿大夫,尋加太常及金紫,加散騎常侍。廷尉虞龢議社稷合為一神,琨案舊糾駮。時龢深被親寵,朝廷多琨強正。



 明帝臨崩,出為督會稽東陽新安臨海永嘉五郡軍事、左軍將軍、會稽太守,常侍如故。坐誤竟囚,降號冠軍。元徽中,遷金紫光祿,弘訓太僕,常侍如故。本州中正,加特進。順帝即位,進右光祿大夫,常侍餘如故。順帝遜位,琨陪位及
 辭廟,皆流涕。



 太祖即位,領武陵王師,加侍中,給親信二十人。時王儉為宰相,屬琨用東海郡迎吏。琨謂信人曰:「語郎,三臺五省,皆是郎用人;外方小郡,當乞寒賤,省官何容復奪之。」遂不過其事。



 琨性既古慎,而儉嗇過甚,家人雜事,皆手自操執。公事朝會,必夙夜早起,簡閱衣裳,料數冠幘,如此數四,世以此笑之。尋解王師。



 建元四年,太祖崩,琨聞國諱,牛不在宅,去臺數里,遂步行入宮。朝士皆謂琨曰:「故宜待車,有損國望。」琨曰:「今日奔赴,皆應爾。」遂得病,卒。贈左光祿大夫,餘如故。年八十四。



 張岱,字景山,吳郡吳人也。祖敞,晉度支尚書。父茂度,宋金紫光祿大夫。



 岱少與兄太子中舍人寅、新安太守鏡、征北將軍永、弟廣州刺史辨俱知名,謂之張氏五龍。鏡少與光祿大夫顏延之鄰居,顏談議飲酒,喧呼不絕;而鏡靜翳無言聲。



 後延之於籬邊聞其與客
 語,取胡床坐聽,辭義清玄,延之心服,謂賓客曰:「彼有人焉。」由此不復酣叫。寅、鏡名最高,永、辨、岱不及也。



 郡舉岱上計掾,不行,州辟從事。累遷南平王右軍主簿,尚書水部郎。出補東遷令。時殷沖為吳興,謂人曰:「張東遷親貧須養,所以棲遲下邑。然名器方顯,終當大至。」隨王誕於會稽起義,以岱為建威將軍,輔國長史,行縣事。事平,為司徒左西曹。母年八十,籍注未滿,岱便去官從實還養,有司以岱違制,將欲糾舉。



 宋孝武曰:「觀過可以知仁,不須案也。」累遷撫軍諮議參軍,領山陰令,職事閑理。



 巴陵王休若為北徐州,未親政事,以岱為冠軍諮議參軍,領彭城太守,行府、州、國事。後臨海王為征虜廣州,豫章王為車騎揚州,晉安王為征虜南兗州,岱歷為三府諮議、三王行事,與典簽主帥共事,事舉而情得。或謂岱曰:「主王既幼,執事多門,而每能緝和公私,云何致此?」岱曰:「古人言一心可以
 事百君。我為政端平,待物以禮,悔吝之事,無由而及。明闇短長,更是才用之多少耳。」入為黃門郎,遷驃騎長史,領廣陵太守。新安王子鸞以盛寵為南徐州,割吳郡屬焉。高選佐史,孝武帝召岱謂之曰:「卿美效夙著,兼資宦已多。今欲用卿為子鸞別駕,總刺史之任,無謂小屈,終當大伸也。」帝崩,累遷吏部郎。



 明帝初,四方反,帝以岱堪乾舊才,除使持節、督西豫州諸軍事、輔國將軍、西豫州刺史。尋徙為冠軍將軍、北徐州刺史,都督北討諸軍事,並不之官。泰始末,為吳興太守。元徽中,遷使持節、督益寧二州軍事、冠軍將軍、益州刺史。數年,益土安其政。征侍中,領長水校尉,度支尚書,領左軍,遷吏部尚書。王儉為吏部郎,時專斷曹事,岱每相違執,及儉為宰相,以此頗不相善。



 兄子瑰、弟恕誅吳郡太守劉遐,太祖欲以恕為晉陵郡,岱曰:「恕未閑從政,美錦不宜濫裁。」太祖曰:「恕為人,我所悉。且
 又與瑰同勛,自應有賞。」岱曰:「若以家貧賜祿,此所不論,語功推事,臣門之恥。」尋加散騎常侍。建元元年,出為左將軍、吳郡太守。太祖知岱歷任清直,至郡未幾,手敕岱曰:「大邦任重,乃未欲回換,但總戎務殷,宜須望實,今用卿為護軍。」加給事中。岱拜竟,詔以家為府。陳疾,明年,遷金紫光祿大夫,領鄱陽王師。



 世祖即位,復以岱為散騎常侍、吳興太守,秩中二千石。岱晚節在吳興,更以寬恕著名。遷使持節、監南兗兗徐青冀五州諸軍事、後將軍、南兗州刺史,常侍如故。未拜,卒。年七十一。岱初作遺命,分張家財,封置箱中,家業張減,隨復改易,如此十數年。贈本官,謚貞子。



 褚炫,字彥緒,河南陽翟人也。祖秀之,宋太常。父法顯,鄱陽太守。兄炤,字彥宣,少秉高節,一目眇,官至國子博士,不拜。常非從兄淵身事二代,聞淵拜司徒,嘆曰:「使淵作中書郎而死,不當是一名士邪?
 名德不昌,遂令有期頤之壽。」



 炫少清簡,為從舅王景文所知。從兄淵謂人曰:「從弟廉勝獨立,乃十倍於我也。」



 宋義陽王昶為太常,板炫補五官,累遷太子舍人,撫軍車騎記室,正員郎。



 從宋明帝射雉,至日中,無所得。帝甚猜羞,召問侍臣曰:「吾旦來如皋,遂空行,可笑。」座者莫答。炫獨曰:「今節候雖適,而雲霧尚凝,故斯翬之禽,驕心未警。但得神駕游豫,群情便為載懽。」帝意解,乃於雉場置酒。遷中書侍郎,司徒右長史。



 昇明初,炫以清尚,與劉俁、謝朏、江斅入殿侍文義,號為「四友」。遷黃門郎,太祖驃騎長史,遷侍中,復為長史。齊臺建,復為侍中,領步兵校尉。以家貧,建元初,出補東陽太守,加秩中二千石。還,復為侍中,領步兵。凡三為侍中。出為竟陵王征北長史,加輔國將軍,尋徙為冠軍長史、江夏內史,將軍如故。



 永明元年,為吏部尚書。炫居身清立,非吊問不雜交
 游,論者以為美。及在選部,門庭蕭索,賓客罕至。出行,左右捧黃紙帽箱,風吹紙剝殆盡。罷江夏還,得錢十七萬,於石頭並分與親族,病無以市藥。表自陳解,改授散騎常侍,領安成王師。國學建,以本官領博士,未拜,卒,無以殯斂。時年四十一。贈太常,謚曰貞子。



 何戢,字慧景,廬江灊人也。祖尚之,宋司空。父偃,金紫光祿大夫,被遇於宋武。選戢尚山陰公主,拜駙馬都尉。解褐秘書郎,太子中舍人,司徒主簿,新安王文學,秘書丞,中書郎。



 景和世,山陰主就帝求吏部郎褚淵入內侍己,淵見拘逼,終不肯從,與戢同居止月餘日,由是特申情好。明帝立,遷司徒從事中郎,從建安王休仁征赭圻,板轉戢司馬,除黃門郎,出為宣威將軍、東陽太守,吏部郎。元徽初,褚淵參朝政,引戢為侍中,時年二十九。戢以年未三十,苦辭內侍,表疏屢上,時議許之。改授司徒左長史。



 太祖為領軍,與戢來往,數
 置歡宴。上好水引餅,戢令婦女躬自執事以設上焉。



 久之,復為侍中,遷安成王車騎長史,加輔國將軍、濟陰太守,行府、州事。出為吳郡太守,以疾歸。為侍中,祕書監,仍轉中書令,太祖相國左長史。建元元年,遷散騎常侍,太子詹事,尋改侍中,詹事如故。上欲轉戢領選,問尚書令褚淵,以戢資重,欲加常侍。淵曰:「宋世王球從侍中中書令單作吏部尚書,資與戢相似,頃選職方昔小輕,不容頓加常侍。聖旨每以蟬冕不宜過多,臣與王儉既已左珥,若復加戢,則八座便有三貂。若帖以驍、游亦為不少。」乃以戢為吏部尚書,加驍騎將軍。戢美容儀,動止與褚淵相慕,時人呼為「小褚公」。家業富盛,性又華侈,衣被服飾,極為奢麗。三年,出為左將軍、吳興太守。



 上頗好畫扇,宋孝武賜戢蟬雀扇,善畫者顧景秀所畫。時陸探微、顧彥先皆能畫,嘆其巧絕。戢因王晏獻之,上令晏厚酬其意。四年,卒。時年三十六。贈散騎常侍、撫軍,太
 守如故。謚懿子。女為鬱林王後,又贈侍中、光祿大夫。



 王延之,字希季,瑯邪臨沂人也。祖裕,宋左光祿儀同三司。父升之,都官尚書。延之出繼伯父秀才粲之。延之少而靜默,不交人事。州辟主簿,不就。舉秀才。



 除北中郎法曹行參軍。轉署外兵尚書外兵部,司空主簿,並不就。除中軍建平王主簿、記室,仍度司空、北中郎二府,轉秘書丞,西陽王撫軍諮議,州別駕,尋陽王冠軍、安陸王後軍司馬,加振武將軍,出為安遠護軍,武陵內史,不拜。宋明帝為衛軍,延之轉為長史,加宣威將軍。司徒建安王休仁征赭圻,轉延之為左長史,加寧朔將軍。



 延之清貧,居宇穿漏。褚淵往候之,見其如此,具啟明帝,帝即敕材官為起三間齋屋。遷侍中,領射聲校尉,未拜,出為吳郡太守。罷郡還,家產無所增益。除吏部尚書,侍中,領右軍,
 並不拜。復為吏部尚書,領驍騎將軍,出為後軍將軍、吳興太守。遷都督浙東五郡、會稽太守。轉侍中,秘書監,晉熙王師。遷中書令,師如故。未拜,轉右僕射。昇明二年,轉左僕射。



 宋德既衰,太祖輔政,朝野之情,人懷彼此。延之與尚書令王僧虔中立無所去就,時人為之語曰:「二王持平,不送不迎。」太祖以此善之。三年,出為使持節、都督江州豫州之新蔡晉熙二郡諸軍事、安南將軍、江州刺史。建元二年,進號鎮南將軍。



 延之與金紫光祿大夫阮韜,俱宋領軍劉湛外甥,並有早譽。湛甚愛之,曰:「韜後當為第一,延之為次也。」延之甚不平。每致餉下都,韜與朝士同例。太祖聞其如此,與延之書曰:「韜雲卿未嘗有別意,當緣劉家月旦故邪?」在州祿俸以外,一無所納,獨處齋內,吏民罕得見者。



 四年,遷中書令,右光祿大夫,本州大中正。轉左僕射,光祿、中正如故。尋領竟陵王師。永明二年,陳疾解職,
 世祖許之。轉特進,右光祿大夫,王師、中正如故。其年卒,年六十四。追贈散騎常侍,右光祿大夫、特進如故。謚簡子。



 延之家訓方嚴,不妄見子弟,雖節歲問訊,皆先克日。子倫之,見兒子亦然。



 永明中,為侍中。世祖幸瑯邪城,倫之與光祿大夫全景文等二十一人坐不參承,為有司所奏。詔倫之親為陪侍之職,而同外惰慢,免官,景文等贖論。建武中,至侍中,領前軍將軍,都官尚書,領游擊將軍,卒。



 阮韜,字長明,陳留人,晉金紫光祿大夫裕玄孫也。韜少歷清官,為南兗州別駕,刺史江夏王劉義恭逆求資費錢,韜曰:「此朝廷物。」執不與。



 宋孝武選侍中四人,並以風貌。王彧、謝莊為一雙,韜與何偃為一雙。常充兼假。泰始末,為征南江州長史。桂陽王休範在鎮,數出行遊,韜性方峙,未嘗隨從。



 至散騎常侍,金紫光祿大夫,領始興王師。永明二年,卒。



 史臣曰:內侍樞近,世為華選。金榼熲耀,朝之麗服,久忘儒藝,專授名家。



 加以簡擇少姿,簪貂冠冕,基蔭所通,後才先貌,事同謁者,以形骸為官,斯違舊矣。闢強之在漢朝,幼有妙察;仲宣之處魏國,見貶容陋。何戢之讓,雖未能深識前古之美,與夫尸官靦服者,何等級哉!



 贊曰:萬石祗慎,琨既為倫。五龍一氏,張亦繼荀。炫清褚族,戢遺何姻。延之居簡,名峻王臣。



\end{pinyinscope}