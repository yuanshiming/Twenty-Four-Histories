\article{卷二十六列傳第七 王敬則 陳顯達}

\begin{pinyinscope}

 王敬則陳顯達



 王敬則,晉陵南沙人也。母為女巫,生敬則而胞衣紫色,謂人曰:「此兒有鼓角相。」敬則年長,兩腋下生乳各長數寸。夢騎五色師子。年二十餘,善拍張。補刀戟左右。景和使敬則跳刀,高與白虎幢等,如此五六,接無不中。補俠轂隊主,領細鎧左右。與壽寂之同斃景和。明帝即位,以為直閣將軍。坐捉刀入殿啟事,擊尚方十餘日,乃復直閣。除奮武將軍,封重安縣子,邑三百五十戶。敬則少時於草中射獵,有蟲如鳥豆集其身,摘去乃脫,其處接流血。敬則惡之,詣道士卜,道士曰:「不須憂,此封侯之瑞也。」敬則聞之喜,故出聞自喜,故出都自效,至
 是如言。



 泰始初,以敬則為龍驤將軍、軍主,隨寧朔將軍劉懷珍征壽春,殷琰遣將劉從築四壘於死虎,懷珍遣敬則以千人繞後。直出橫塘,賊眾驚退。除奉朝請,出補東武暨陽令。



 敬則初出都,至陸主山下,宗侶十餘船同發,敬則船獨不進,乃令弟入水推之,見一烏漆棺。敬則曰:「爾非凡器。若是吉善,使船速進。吾富貴,當改葬爾。」船須臾去。敬則既入縣,收此棺葬之。



 軍荒之後,縣有一部劫逃紫山中為民患,敬則遣人致意劫帥,可悉出首,當相申論。治下廟神甚酷烈,百姓信之,敬則引神為誓,必不相負。劫帥既出,敬則於廟中設會,於座收縛,曰:「吾先啟神,若負誓,還神十牛。今不違誓。」即殺十牛解神,并斬諸劫,百姓悅之。遷員外郎。



 元徽二年,隨太祖拒桂陽賊於新亭,敬則與羽林監陳顯達、寧朔將軍高道慶乘舸於江中迎戰,大破賊水軍,焚其舟艦。事寧,帶南泰山太守,右俠轂主,轉
 越騎校尉,安成王車騎參軍。



 蒼梧王狂虐,左右不自保,敬則以太祖有威名,歸誠奉事。每下直,輒領府。夜著青衣,扶匐道路。為太祖聽察蒼梧去來。太祖命敬則於殿內伺機,未有定日。既而楊玉夫等危急殞帝,敬則時在家,玉夫將首投敬則,敬則馳詣太祖。太祖慮蒼梧所誑,不開門。敬則於門外大呼曰:「是敬則耳。」門猶不開。乃於墻投進其首,太祖縈水洗視,視竟,乃戎服出。



 敬則從入宮,至承明門,門郎疑非蒼梧還,敬則慮人覘見,以刀環塞窪孔,呼開門甚急。衛尉丞顏靈寶窺見太祖乘馬在外,竊謂親人曰:「今若不開內領軍,天下會是亂耳。」門開,敬則隨太祖入殿。明旦,四貴集議,敬則拔白刃在床側跳躍曰:「官應處分,誰敢作同異者!」昇明元年,遷員外散騎常侍、輔國將軍、驍騎將軍、領臨淮太守,增封為千三百戶,知殿內宿衛兵事。



 沈攸之事起,進敬則號冠軍將軍。太祖入
 守朝堂,袁粲起兵夕,領軍劉韞、直閣將軍卜伯興等於宮內相應,戒嚴將發。敬則開關掩襲,皆殺之。殿內竊發盡平,敬則之力也。遷右衛將軍,常侍如故。增封為二千五百戶,尋又加五百戶。又封敬則予元遷為東鄉侯,邑三百七十戶。齊臺建,為中領軍。



 太祖受禪,材官薦易太極殿柱,從帝欲避土,不肯出宮遜位。明日,當臨軒,帝又逃宮內。敬則將舉人迎帝,啟譬令出。帝拍敬則手曰:「必無過慮,當餉輔國十萬錢。」



 建元元年,出為使持節、散騎常侍、都督南兗兗徐青冀五州軍事、平北將軍、南兗州剌史,封尋陽郡公,邑三千戶。加敬則妻懷氏爵為尋陽國夫人。二年,進號安北將軍。虜寇淮、泗,敬則恐,委鎮還都,百姓皆驚散奔走,上以其功臣,不問,以為都官尚書、撫軍。



 尋遷使持節、散騎常侍、安東將軍、吳興太守。郡舊多剽掠,有十數歲小兒於路取遺物,殺之以徇,自此道不拾遺,郡無
 劫盜。又錄得一偷,召其親屬於前鞭之,令偷身長掃街路,久之乃令偷舉舊偷自代,諸偷恐為其所識,皆逃走,境內以清。出行,從市過,見屠肉桿,歡曰:「呈興昔無此桿,是我少時在此所作也。」



 遷護軍將軍,常侍如故,以家為府。三年,以改葬去職,詔贈敬則母尋陽公國太夫人。改授侍中、撫軍將軍。太祖遺詔敬則以本官領丹陽尹。尋遷為使持節、散騎常侍、都督會稽東陽臨海永嘉五郡軍事、鎮東將軍、會稽太守。永明二年,給鼓吹一部。



 會土邊帶湖海,發丁無士庶皆保塘投,敬則以功力有餘,悉評斂為錢,送臺庫以為便宜,上許之。竟陵王子良啟曰:伏尋三吳內地,國這關輔,百度所資。發庶凋流,日有困殆,蠶家罕獲,饑寒尤甚,富者稍增其饒,貧者轉鐘其弊,可為痛心,難以辭盡。頃錢貴物賤,殆欲兼倍,凡在觸類,莫不如茲。稼穡難劬,斛直數十,機杼勤苦,匹裁三百。所以然者,實亦有
 由。年常歲調,既有定期,僮恤所上,威是風陡,東間錢多剪鑿,鮮復完者,公家所受,必須員大,以兩代一,困於所貿,鞭捶質擊,益致無聊。



 臣昔忝會稽,粗閑物俗,塘古所上,本不入官。良由陂湖宜壅,橋路須通,均夫訂直,民自為用。若甲分毀壞,則年一修改;若乙限堅完,則終歲無役。今郡通課此直,悉以還臺,租賦之外,更生一調。致今塘路崩蕪,湖源泄散,害民損政,實此為劇。



 建元初,狡虜游魂,軍用殷廣。浙東五郡,丁稅一千,乃有質賣妻兒,以充此限,道路愁窮,不可聞見,所逋尚多,收上事絕,臣登具啟聞,即蒙蠲原。而此年租課,三分逋一,明知徒足擾民,實自弊國。愚謂塘丁一條,宜還復舊,在所逋恤,優量原除。凡應受錢,不限大小,仍令在所,折市布帛,若民有雜物,是軍國所須者,聽隨價準直,不必一應送錢,於分不虧其用,在私實荷其渥。



 昔晉氏初遷,江左草創,絹布所直,十倍於今,
 賦調多少,因時增減。永初中,官布一匹,直錢一千,而民間所輸,必為降落。今入官好布,匹堪百餘,其四民所送、猶依舊制。昔為刻上,今為刻下,氓庶空儉,豈不由之。



 救民拯弊,莫過減賦。時和歲稔,尚爾虛乏,儻值水旱,寧可熟念。且西京熾強,實基三輔,東都全固,實賴三河,歷代所同,古今一揆。石頭以外,裁足自供府州,方山以東,深關朝廷根本。夫股肱要重,不可不恤。宜蒙寬政,少加優養。略其目前小利,取其長久大益,無患民貲不股,國財不皂也。宗臣重寄,咸云利國,竊如愚管,未見可安。



 上不納。



 三年,進號征東將軍。宋廣州剌史王翼之子妾路氏,剛暴,數殺婢,翼之子法明告敬則,敬則付山陰獄殺之,路氏家訴,為有司所奏,山陰令劉岱坐棄市刑。敬則入朝,上謂敬則曰:「人命至重,
 是誰殺之?都不啟聞?」敬則曰:「是臣愚意。臣知何物科法,見背後有節,便言應得殺人。」劉岱亦引罪,上乃赦之。敬則免官,以公領郡。



 明年,遷侍中、中軍將軍。尋與王儉俱即本號開府儀同三司,儉既固讓,敬則亦不即受。七年,出為使持節、散騎常侍、都督豫州郢州之西陽司州之汝南二郡軍事、征西大將軍、豫州剌史,開府如故。進號驃騎。十一年,遷司空,常侍如故。世祖崩,遺詔改加侍中。高宗輔政,密有廢立意,隆昌元年,出敬則為使持節、都督會稽東陽臨海永嘉新安五郡軍事、會稽太守,本官如故。海陵王立,進位尉。



 敬則名位雖達,不以富貴自遇,危拱傍遑,略不嘗坐,接士庶皆吳語,而殷勤周悉。初為散騎使虜,於北館種揚柳,後員外郎虞長耀北使還,敬則問:「我昔種楊柳樹,今若大小?」長耀曰:「虜中以為甘棠。」敬則笑而不答。



 世祖御座賦詩,敬則執紙曰:「臣幾落此奴度內。」世祖問:「此何言?」
 敬則曰:「臣若知書,不過作尚書書都令史耳,那得今日?」敬則雖不大識書,而性甚警黠,臨州郡,令省事讀辭,下教判決,皆不失理。



 明帝即位,進大司馬,增邑千戶。臺使拜授日,雨大洪注,敬則文武皆失色,一客在傍曰:「公由來如此,昔拜丹陽吳興時亦然。」敬則大悅,曰:「我宿命應得雨。」乃列羽儀,備朝服,道引出聽事拜受,意猶不自得,吐舌久之,至事竟。



 帝既多殺害,敬則自以高、武舊臣,心懷憂恐。帝雖外厚其禮,而內相疑備,數訪問敬則飲食體乾堪宜,聞其衰老,且以居內地,故得少安。三年中,遣蕭坦之將齊伏五百人,行武進陵。敬則諸子在都,憂怖無計。上知之,遣敬則世子仲雄入東安慰之。仲雄善彈琴,當時新絕。江左有蔡邕焦尾琴,在主衣庫,上敕五日一給仲雄。仲雄於御前鼓琴作曰:「常歡負情儂,郎今果行許!」帝愈猜愧。



 永泰元年,帝疾,屢經危殆。以張環為平東將軍、
 吳郡太守,置兵佐,密防敬則。內外傳言當有異處分。敬則聞之,竊曰:「東今有誰?祗是欲平我耳!」諸子怖懼,第五子幼隆遣正員將軍徐嶽密以情告徐州行事謝眺為計,若同者,當往報敬則。眺執嶽馳啟之。敬則城局參軍徐庶家在京口,其子密以報庶,庶以告敬則五官王公林。公林,敬則族子,常所委信。公林權敬則急送啟賜兒死,單舟星夜還都。敬則令司馬張思祖草啟,既而曰:「若爾,諸郎在都,要應有信,且忍一夕。」其夜,呼僚佐文武樗蒲賭錢,謂眾曰:「卿諸人欲令我作何計?」莫敢先答。防閣丁興懷曰:「安祗應作耳。」敬則不作聲。明旦,召山陰令王詢、臺傳御史鐘離祖願,敬則橫刀坐,問詢等「發丁可得幾人?傅庫見有幾錢物?」詢答「縣丁卒不可上。」祖願稱「傳物多未輸入。」敬則怒,將出斬之。王公林又諫敬則曰:「官是事皆可悔,惟此事不可悔!官詎不更思!」敬則唾其面曰:「小子!我作
 事,何關汝小子!」乃起兵。



 上詔曰:「謝眺啟事騰徐岳列如右。五敬則稟質兇猾,本謝人網。直以寧季多艱,頗有力之用,驅獎所至,遂升榮顯。皇運肇基,預聞末議,功非匡國,賞實震主。爵冠執珪,身登衣袞,固以、作刺,縉紳側目。而溪谷易盈,鴟梟難改,猜心內駭,醜辭外布。永明之朝,履霜有漸,隆昌之世,堅冰將著,從容附會,朕有力焉。及景歷惟新,推誠盡禮,中使相望,軒冕成陰。迺嫌跡愈興,禍圖茲構,收合亡命,結黨聚群,外候邊警,內伺國隙。元遷兄弟,中萃淵藪,姦契潛通,將謀竊發。眺即姻家,嶽又邑子,取據匪他,昭然以信。方、邵之美未聞,韓、彭之畔已積。此而可空,孰寄刑典!便可即遣收掩,肅明國憲,大辟所加,其父子而已;凡諸詿誤,一人蕩滌。」收敬則子員久郎世雄、記室參軍季哲、太子洗馬幼隆、太子舍人少安等,於宅殺之。長子黃門郎元遷,為寧朔將軍,領千人於徐州擊
 虜,敕徐州刺史徐玄慶殺之。



 敬則招集配衣,二三日便發,欲動前中書令何胤還為尚書令,長史王弄璋、司馬張思祖止之。乃率實甲尤人過浙江。謂思祖曰:「應須作檄。」思祖曰:「公今自還朝,何用作此。」敬則乃止。



 朝廷遣輔國將軍司馬左興盛、後軍將軍直閣將軍崔恭祖、輔國將軍劉山陽、龍驤將軍直閣將軍馬軍主胡松三千餘人,築壘於曲阿長岡,右僕射沈文季為持節都督,屯湖頭,備京口路。



 鏂則以舊將舉事,百姓簷篙荷鍤隨逐之,十餘萬眾。至晉陵,南沙人范脩化殺縣令公上延孫以應之。敬則至武進陵口,慟哭乘肩舉而前。遇興盛、山旭二砦,盡力攻之。興盛使軍人遙告敬則曰:「公兒死已盡,公持許底作?」官軍不敵欲退,而圍不開,各死戰。胡松領馬軍突其後,白丁無器杖,皆驚散,敬則軍大敗。敬則索馬,再上不得上,興盛軍容袁文曠斬之,傅首。是時上疾已篤,敬則倉
 卒東起,朝廷震懼。東昏候在東宮,議欲叛,使人上屋望,見征虜亭失火,謂敬則至,急裝欲走。有告敬則者,敬則曰:「檀公三十六策,走上上計。汝父子唯應急走耳。」敬則之來,聲勢甚盛,裁少日而敗,時年七十餘。



 封左興盛新吳縣男,崔恭祖遂興縣男,劉山陽湘陰縣男,胡松沙陽縣男,各四百戶,賞平敬則也。又贈公上延孫為射聲校尉。



 陳顯達,南彭城人也。宋孝武世,為張永前軍幢主。景和中,以勞歷驅使。泰始初,以軍主隸徐州刺史劉懷珍北征,累至東海王板行參軍,員外郎。泰始四年,封彭澤縣子,邑三百戶。歷馬頭、義陽二郡太守,羽林監,濮陽太守。



 隸太祖討桂陽賊於新亭壘,劉勉大桁敗,賊進杜姥宅,及休範死,太祖欲還衛宮城,或諫太祖曰:「桂陽雖死,賊黨猶熾,人情難固,不可輕動。」太祖乃止。遣顯達率司空參軍高敬祖自查浦渡淮緣石頭北道入承明門,屯東堂。宮中恐動,得顯
 達乃至,乃稍定。顯達出杜姥宅,大戰破賊。矢中左眼,拔箭而鏃不出,地黃村潘嫗善禁,先以釘釘柱,嫗禹步作氣,釘即時出,乃禁顯達目中鏃出之。封豐城縣候,邑千戶。轉游擊將軍。



 尋為使持節、督廣交越三州湘州之廣興軍事、輔國將軍、平越中郎將、廣州刺史,進號冠軍。沈攸之事起,顯達遣軍援臺,長史到遁、司馬諸葛道謂顯達曰:「沈攸之擁眾百萬,勝負之勢未可知,如保境蓄眾,分遣信驛,密通彼此。」顯達於座手斬之,遣表疏歸心太祖。進使持節、左將軍。軍至巴丘,而沈攸之平。除散騎常侍、左衛將軍,轉前將軍、太祖太尉左司馬。齊臺建,為散騎常侍,左衛將軍,領衛尉。太祖即位,遷中護軍,增邑千六百戶,轉護軍將軍。顯達啟讓,上答曰:「朝廷爵人以序。卿忠發萬里,信誓如期,雖屠城殄國之勛,無以相加。此而不賞,典章何在。若必未宜爾,吾終不妄授。於卿數士,意同家人,豈止於
 群臣邪?過明,與王、李俱祗召也。」上即位後,御膳不宰牲,顯達上熊丞一盤,上即以充飯。



 建元二年,虜寇壽陽,淮南江北百姓搔動。上以顯達為使侍節、散騎常侍、都督南兗兗徐青冀五州諸軍事、平北將軍、南兗州刺史。之鎮,虜退。上敕顯達曰:「虜經破散後,當無復犯關理。但國家邊防,自應過存備豫。宋元嘉二十七年後,江夏王作南兗,處鎮盱眙,沈司空亦以孝建初鎮彼,政當以淮上要於廣陵耳。卿謂前代此處分云何?今僉議皆云卿應據彼地,吾未能決。乃當以擾動文武為勞。若是公計,不得憚之。」事竟不行。



 遷都督益寧二州軍事、安西將軍、益州刺史,領宋寧太守,持節、常侍如故。世祖即位,進號鎮西。益部山險,多不賓服。大度村獠,前後刺史不能制,顯達遣使責其租賧,獠帥曰:「兩眼刺史尚不敢調我!」遂殺其使。顯達分部將吏,聲將出獵,夜往襲之,男女無少長皆斬之。自此由
 夷震服。廣漢賊馬龍駒據郡反,顯達又討平之。



 永明二年,徵為侍中、護軍將軍。顯達累任在外,經太祖之憂,及見世祖,流涕悲咽,上亦泣,心甚嘉之。



 五年,荒人桓天生自稱桓玄宗族,與雍、司二州界蠻虜相扇動,據南陽故城。上遣顯達假節,率征虜將軍戴僧靜等水軍向宛、葉,雍、司眾軍受顯達節度。天生率虜眾萬餘入攻舞陰,舞陰戍主輔國將軍殷公愍擊殺其副張麒麟,天生被瘡退走。仍以顯達為使持節、散騎常侍、都督雍梁南北秦郢州之竟陵司州之隨郡軍事、鎮北將軍,領寧蠻校尉、雍州刺史。顯達進據舞陽城,遣僧靜等先進,與天生及虜再戰,大破之,官軍還。數月,天生復出攻舞陰,殷公愍破之,天生還竄荒中,遂城、平氏、白土三城賊稍稍降散。



 八年,進號征北將軍。其年,仍遷侍中、鎮軍將軍,尋加中領軍。出為使持節、散騎常侍、都督江州諸軍事、征南大將軍、江州刺
 史,給鼓吹一部。顯達廉厚有智計,自以人微位重,每遷官,常有愧懼之色。有子十餘人,誡之曰:「我本志不及此,汝等勿以富貴陵人!」家既豪富,諸子與王敬則諸兒,並精車牛,麗服飾。當世快年稱陳世子青,王三郎烏,呂文顯折角,江瞿曇白鼻。顯達謂其子曰:「塵尾扇是五謝家物,汝不鬚捉此自逐。」



 十一年秋,虜動,詔屯樊城。世祖遺詔,即本號開府儀同三司。隆昌元年,遷侍中、車騎將軍,開府如故,置兵佐。豫廢鬱林之勳,延興元年,為司空,進爵公,增邑千戶,甲仗五十人入殿。高宗即位,進太尉,侍中如故,改封鄱陽郡公,邑三千戶,加兵二百人,給油絡車。建武二年,虜攻徐、司,詔顯達出頓,往來新亭白下,以為聲勢。



 上欲悉除高、武諸孫,微言問顯達,答曰:「此等豈足介慮。」上乃止。顯達建武世心懷不安,深自貶匿,車乘朽故,導從鹵薄,皆用羸小,不過十數人。侍宴,酒後啟上曰:「臣年已老,富
 貴已足,唯少枕枕死,特就陛下乞之。」上失色曰:「公醉矣。」以年禮告退,不許。



 是時虜頻寇雍州,眾軍不捷,失沔北五郡。永泰元年,乃遣顯達北討。詔曰:「晉氏中微,宋德將謝,蕃臣外叛,要荒內侮,天未悔禍,左衽亂華,巢穴神州,逆移年載。朕嗣膺景業,踵武前王,靜言隆替,思又區夏。但多難甫夷,恩化肇治,興師擾眾,非政所先,用遠圖,權緩北略,冀戎夷知義,懷我好音。而凶醜剽狡,專事侵掠,驅扇異類,蟻聚西偏,乘彼自來之資,撫其天亡之會,軍無再駕,民不重勞,傳檄以定三秦,一麾而臣禹迹,在此舉矣。且中原士庶,久望皇威,乞師請援,結軌馳道。信不可失,時豈終朝。宜分命方嶽,因茲大號。侍中太尉顯達,可暫輟槐陰,指授群帥。」中外纂嚴。加顯達使持節,向襄陽。



 永元元年,顯達督平北將軍崔慧景眾軍四萬,圍南鄉界馬圈城,去襄陽三百里,攻之四十日,虜食盡,啖死人肉及樹皮,
 外圍既急,虜突走,斬獲千計。官軍競取城中絹,不復窮追。顯達入據其城,遣軍主莊丘黑進取南鄉縣故從陽郡治也。虜主元宏自領十餘萬騎奄至,顯達引軍渡水西據鷹子山築城,人情沮敗。虜兵甚急,軍主崔恭祖、胡松以鳥布幔盛顯達,數人簷之,逕道從分磧山均水口,臺軍緣道奔退,死者三萬餘人。左軍將張千虎死,追贈游擊將軍。顯達素有威名,著於蠻虜,至是大損喪焉。御史中丞范岫奏免顯達官,朝議優詔答曰:「昔衛、霍出塞,往往無功,馮、鄧入關,有時虧喪。況公規謨肅舉,期寄兼深、見可知難,無損威略,方振遠圖,廓清朔土,雖執憲有常,非所得議。」顯達表解職,不許,求降號,又不許。



 以顯達為都督江州軍事、江州刺史,鎮盆城,持節本官如故。初,王敬則事起,始安王遙光啟明帝慮顯達為變,欲追究軍還,事尋平,乃寢。顯達亦懷危怖。及東昏立,彌不樂還京師,得
 此授,甚喜。尋加領片南大將軍,給三望車。



 顯達聞京師大相殺戳,又知徐孝嗣等皆死,傳聞當遣兵襲江州,顯達懼禍,十一月十五日,舉兵。令長史庚弘遠、司馬虎龍與朝貴書曰:諸君足下:我太祖高皇帝睿哲自天,超人作聖,屬彼宋季,網紀自頓,應禪從民,構此基業。世祖武皇帝昭略通遠,克纂洪嗣,四關罷險,三河靜塵。鬱林海陵,頓孤負荷。明帝英聖,紹建中興。至乎後主,行悖三才,琴橫由席,繡積麻筵,淫犯先宮,穢興閨闥,皇陛為市之所,雕房起征戰之門。任非華尚,寵必寒廝。



 江僕射兄弟,忠言屬薦,正諫繁興,覆族之誅,於斯而至。故乃犴噬之刑,四剽於海路,家門之釁,一起於中都。蕭、劉二領軍,並升御座,共稟遺詔,宗戚之苦,諒不足談,之悲,何辜至此。徐司空歷葉忠榮,清簡流世,匡翼之功示著,傾宗之罰已彰。沈僕射年在懸車,將念機杖,歡歌園藪,絕影朝門,忽招陵上之罰,
 何萬古之傷哉。遂使紫臺之路,絕縉紳之儔;纓組之閣,罷金、張之胤。悲哉!蟬冕為賤寵之服。嗚呼!!皇陛列劫豎之坐。



 且天人同怨,乾象變錯,往歲三州流血,今者五地自動。昔漢池異色,胥王因之見廢;吳郡暫震,步生以為姦倖。況事隆於往怪,釁倍於前虐,此而未廢,孰不可興?



 王僕射、王領軍、崔護軍,中維簡正,逆念剖心。蕭衛尉、蔡詹事、沈左衛,各負良家,共傷時哈。先朝遺舊,志在名節,同列丹書,要同義舉。建安殿下季德沖遠,實允神器。昏明之舉,往聖流言。今忝役戎驅,丞請乞路。須京塵一靜,西迎大駕,歌舞太平,不亦隹哉!裴豫州宿遣誠言,久懷慷慨,計其勁兵,已登淮路;申司州志節堅明,分見迎合,總勒偏率,殿我而進;蕭雍州、房僧寄並已纂邁,旌鼓將及;南兗州司馬恭祖壯烈超群,嘉驛屢至,佇聽烽諜,共成唇齒;荊郢行事蕭、張二賢,莫不案劍餐風,橫戈待節;關幾蕃守之
 儔,孰非義侶。



 我太尉公禮道合聖,杖德脩文,神武橫於七伐,雄略震於九網。是乃從彼英序,還抗社稷。本欲鳴笳細錫,無勞戈刃。但忠黨有心,節義難遣。信次之間,森然十萬。飛咽於九派,列艦迷於三川,此蓋捧海澆螢,烈火消凍耳。吾子其擇善而從之,無令竹帛空為後人笑也。



 朝廷遣後軍將軍胡松、驍騎將軍李叔獻水軍據梁山;左衛將軍左興盛假節,加征虜將軍,督前鋒軍事,屯新亭;輔國將軍驍騎將軍徐世標領兵屯杜姥宅。顯達率眾數千人發尋陽,與胡松戰於採石,大破之,京邑震恐。十二月十三日,顯達至新林築城壘,左興盛率眾軍為拒戰之計。其夜,顯達多置屯火於巖側,潛軍渡取石頭北上襲宮城,遇風失曉,十四日平旦,數千人登落星崗,新亭軍望火,謂顯達猶在,既而奔歸赴救,屯城南。宮掖大駭,閉門守備。顯達馬從步軍數百人,於西州前與臺軍戰,再
 合,大勝,手殺數人,折,官軍繼至,顯達不能抗,退走至西州後鳥榜村,為騎官趙潭注刺落馬,斬之於離側,身湧湔離,似淳于伯之被刑也。時年七十二。顯達在江州,遇疾不治,尋而自差,意甚不悅。是冬連大雪,梟首於朱雀,百雪不集之。諸子皆伏誅。



 史臣曰:光武功臣所以能終其身名者,非唯不任職事,亦以繼奉明、章,心尊正嫡,君安乎上,臣習乎下。王、陳拔跡奮飛,則建元、永明之運;身極鼎將,則建武、永元之朝。勛非往時,位逾昔等,禮授雖重,情分不交。加以主猜政亂,危亡慮及,舉手捍頭,人思自免。干戈既用,誠淪犯上之跡,敵國起於同舟,況又疏於此者也?



 贊曰:糾糾敬則,臨難不惑。功成殿寢,誅我蝥賊。顯達孤根,應義南蕃。威揚寵盛,鼎食高門。王虧河、兗、陳挫襄、樊。



\end{pinyinscope}