\article{卷四十三列傳第二十四 江斆 何昌宇 謝抃 王思遠}

\begin{pinyinscope}

 江籞,字叔文,濟陽考城人也。祖湛,宋左光祿大夫、儀同三司。父恁,著作郎,為太子劭所殺。斆母,文帝女淮陽公主。幼以戚屬召見,孝武謂謝莊曰:「此小兒方當為名器。」少有美譽。桂陽王休範臨州,辟迎主簿,不就。尚孝武女臨汝公主,拜駙馬都尉。除著作郎,太子舍人,丹陽丞。時袁粲為尹,見斆歎曰:「風流不墜,政在江郎。」數與宴賞,留連日夜。遷安成王撫軍記室,秘書丞,中書郎。



 斆庶祖母王氏老疾,
 斆視膳嘗藥,七十餘日不解衣。及累居內官,每以侍養陳請,朝廷優其朝直。尋轉安成王驃騎從事中郎。初,湛娶褚秀之女,被遣,褚淵為衛軍,重斆為人,先通音意,引為長史。加寧朔將軍。順帝立,隨府轉司空長史,領臨淮太守,將軍如故。轉太尉從事中郎。



 齊臺建,為吏部郎。太祖即位,斆以祖母久疾連年,臺閣之職,永廢溫清,啟乞自解。初,宋明帝敕斆出繼從叔愻,為從祖淳後。於是僕射王儉啟:「禮無後小宗之文,近世緣情,皆由父祖之命,未有既孤之後,出繼宗族也。雖復臣子一揆,而義非天屬。江忠簡胤嗣所寄,唯籞一人,傍無眷屬,籞宜還本。若不欲江愻絕後,可以籞小兒繼愻為孫。」尚書參議,謂「間世立後,禮無其文。荀顗無子立孫,墜禮之始;何琦又立此論,義無所據。」於是籞還本家,詔使自量立後者。出為寧朔將軍、豫章內史,還,除太子中庶子,領驍騎將軍。未拜,門客通贓利,
 世祖遣信撿核,籞藏此客而躬自引咎,上甚有怪色。王儉從容啟上曰:「江籞若能治郡,此便是具美耳。」上意乃釋。



 永明初,仍為豫章王太尉諮議,領錄事,遷南郡王友,竟陵王司徒司馬。籞好文辭,圍棋第五品,為朝貴中最。遷侍中,領本州中正。司徒左長史,中正如故。



 五年,遷五兵尚書。明年,出為輔國將軍、東海太守,加秩中二千古,行南徐州事。



 七年,徙為侍中,領驍騎將軍,尋轉都官尚書,領驍騎將軍。王晏啟世祖曰:「江斆今重登禮閣,兼掌六軍,慈渥所覃,實有優忝。但語其事任,殆同閑輩。天旨既欲升其名位,愚謂以侍中領驍騎,望實清顯,有殊納言。」上曰:「斆常啟吾,為其鼻中惡。今既以何胤、王瑩還門下,故有此回換耳。」



 鬱林即位,遷掌吏部。隆昌元年,為侍中,領國子祭酒。鬱林廢,朝臣皆被召入宮,斆至雲龍門,託藥醉吐車中而去。明帝即位,改領秘書監,又改領晉安王師。



 建武二
 年,卒,年四十四。遺令儉約葬,不受賻贈。詔賻錢三萬,布百匹。子蒨啟遵斆令,讓不受。詔曰:「斆貽厥之訓,送終以儉,立言歸善,益有嘉傷,可從所請。」贈散騎常侍、太常,謚曰敬子。



 何昌宇,字儼望,廬江灊人也。祖叔度,吳郡太守。父佟之,太常。昌宇少而淹厚,為伯父司空尚之所遇。宋建安王休仁為揚州,闢昌宇州主簿。遷司徒行參軍,太傅五官,司徒東閣祭酒,尚書儀曹郎。建平王景素為征北南徐州,昌宇又為府主簿,以風素見重。母老求祿,出為湘東太守,加秩千石。為太祖驃騎功曹。昌宇在郡,景素被誅,昌宇痛之。至是啟太祖曰:伏尋故建平王,因心自遠,忠孝基性。徽和之譽,早布國言;勝素之情,夙洽民聽。世祖綢繆,太宗眷異,朝中貴人,野外賤士,雖聞見有殊,誰不悉斯事者?



 元徽之間,政關群小,構扇異端,共令傾覆。殷勤之非,古人所悼,況蒼梧將季,能無衒
 惑?一年之中,藉者再三,有必巔之危,無暫立之安,行路寒心,往來跼蹐。



 而王夷慮坦然,委之天命,惟謙惟敬,專誠奉國;閨無執戟之衛,門闕衣介之夫,此五尺童子所見,不假闊曲言也。一淪疑似,身名頓滅,冤結淵泉,酷貫穹昊。時經隆替,歲改三元,曠蕩之惠亟申,被枉之澤未流。俱沐溫光,獨酸霜露。



 明公鋪天地之施,散雲雨之潤,物無巨細,咸被慶渥。若今日不蒙照滌,則為萬代冤魂。昌宇非敢慕慷慨之士,激揚當世;實義切於心,痛入骨髓。瀝腸紓憤,仰希神照。辯明枉直,亮王素行,使還名帝籍,歸靈舊塋,死而不泯,豈忘德於黃壚!分軀碎首,不足上謝。



 又與司空褚淵書曰:天下之可哀者有數,而埋冤於黃泉者為甚焉。何者?百年之壽,同於朝露,揮忽去留,寧足道哉!政欲闔棺之日,不隕令名,竹帛傳芳烈,鐘石紀清英。是以昔賢甘心於死所者也。若懷忠抱義而負枉冥冥之下,時
 主未之矜,卿相不為言,良史濡翰,將被以惡名,豈不痛哉!豈不痛哉!



 竊尋故建平王,地屬親賢,德居宗望,道心惟沖,睿性天峻。散情風雲,不以塵務嬰衿;明發懷古,惟以琴書娛志。言忠孝,行惇慎,二公之所深鑒也。前者阮、楊連黨,構此紛紜,雖被明於朝貴,愈結怨於群醜。覘察繼蹤,疑防重著,小人在朝,詩史所嘆,少一句清識飲涕。王每永言終日,氣淚交橫。既推信以期物,故日去其備衛,朱門蕭條,示存典刑而已。求解徐州,以避北門要任;苦乞會稽,貪處東甌閑務。此並彰於事跡。與公道味相求,期心有素,方共經營家國,劬勞王室,何圖時不我與,契闊屯昏,忠誠弗亮,罹此百殃!



 歲朔亟流,已經四載。皇命惟新,人沾天澤,而幽然深酷,未蒙照明。封殯卑雜,窮魂莫寄,昭穆不序,松柏無行。事傷行路,痛結幽顯。吾等叩心泣血,實有望於聖時。公以德佐世,欲物得其所,豈可令建平王枉直
 不分邪?田叔不言梁事,袁絲諫止淮南,以兩國釁禍,尚回帝意,豈非親親之義,寧從敦厚?而今疑以未辨,為世大戮。若使王心跡得申,亦示海內理冤枉,明是非。夫存亡國,繼絕世,周漢之通典,有國之所急也。昔叔向之理,恃祁大夫而獲亮;戾太子之冤,資車丞相而見察。幽靈有知,豈不眷眷於明顧?碎首抽脅,自謂不殞。



 淵答曰:「追風古人,良以嘉歎。但事既昭晦,理有逆從。建平初阻,元徽未悖,專欲委咎阮、楊,彌所致疑。于時正亦謬參此機,若審如高論,其愧特深。」



 太祖嘉其義,轉為記室,遷司徒左西、太尉戶曹屬,中書郎,王儉衛軍長史。儉謂昌宇曰:「後任朝事者,非卿而誰?」



 永明元年,竟陵王子良表置文、學官,以昌宇為竟陵王文學,以清信相得,意好甚厚。轉揚州別駕,豫章王又善之。遷太子中庶子,出為臨川內史。除廬陵王中軍長史,未拜,復為太子中庶子,領屯騎校尉。遷吏部郎,轉
 侍中。



 臨海王昭秀為荊州,以昌宇為西中郎長史、輔國將軍、南郡太守,行荊州事。



 明帝遣徐玄慶西上害蕃鎮諸王,玄慶至荊州,欲以便宜從事。昌宇曰:「僕受朝廷意寄,翼輔外蕃,何容以殿下付君一介之使!若朝廷必須殿下還,當更聽後旨。」



 昭秀以此得還京師。



 建武二年,為侍中,領長水校尉,轉吏部尚書。復為侍中,領驍騎將軍。四年,卒。年五十一。贈太常,謚簡子。



 昌宇不雜交遊,通和泛愛,歷郡皆清白。士君子多稱之。



 謝鷿,字義潔,陳郡陽夏人也。祖弘微,宋太常。父莊,金紫光祿大夫。抃四兄揚、朏、顥、軿,世謂謝莊名兒為風、月、景、山、水。顥字仁悠,少簡靜。解褐秘書郎,累至太祖驃騎從事中郎。建元初,為吏部郎,至太尉從事中郎。永明初,高選友、學,以顥為竟陵王友。至北中郎長史。卒。



 抃年七歲,王彧見而異之,言於宋孝武。孝武召見於稠人廣眾
 之中,抃舉動閑詳,應對合旨,帝甚悅,詔尚公主,值景和敗,事寢。僕射褚淵聞抃年少,清正不惡,以女結婚,厚為資送。解褐車騎行參軍,遷秘書郎,司徒祭酒,丹陽丞,撫軍功曹。世祖為中軍,引為記室。



 齊臺建,遷太子中舍人。建元初,轉桂陽王友。以母老須養,出為安成內史。



 還為中書郎。衛軍王儉引為長史,雅相禮遇。除黃門郎,兼掌吏部。尋轉太子中庶子,領驍騎將軍,轉長史兼侍中。抃以晨昏有廢,固辭不受。世祖敕令速拜,別停朝直。



 遷司徒左長史,出為吳興太守。長城縣民盧道優家遭劫,誣同縣殷孝悌等四人為劫,抃收付縣獄考正。孝悌母駱詣登聞訴稱孝悌為道優所誹謗,橫劾為劫,一百七十三人連名保征,在所不為申理。鷿聞孝悌母訴,乃啟建康獄覆,道優理窮款首,依法斬刑。有司奏免抃官。抃又使典藥吏煮湯,失火,燒郡外齋南廂屋五間。又輒鞭除身,為有司所奏,
 詔並贖論,在郡稱為美績。母喪去官。服闋,為吏部尚書。



 高宗廢鬱林,領兵入殿,左右驚走報抃。抃與客圍棋,每下子,輒云「其當有意」。竟局,乃還齋臥,竟不問外事也。明帝即位,抃又屬疾不視事。後上宴會,功臣上酒,尚書令王晏等興席,抃獨不起,曰:「陛下受命,應天順民,王晏妄叨天功以為己力。」上大笑解之。座罷,晏呼抃共載還令省,欲相撫悅,抃又正色曰:「君巢窟在何處?」晏初得班劍,抃謂之曰:「身家太傅裁得六人。君亦何事一朝至此。」晏甚憚之。



 加領右軍將軍。兄朏在吳興,論啟公事稽晚,鷿輒代為啟,上見非其手跡,被問,見原。轉侍中,領太子中庶子,豫州中正。永泰元年,轉散騎常侍,太子詹事。



 其年卒,年四十五。贈金紫光祿大夫。謚簡子。



 初,兄朏為吳興,抃於征虜渚送別,朏指抃口曰:「此中唯宜飲酒。」抃建武之初專以長酣為事,與劉瑱、沈昭略以觴酌交飲,各至數斗。



 世祖嘗問
 王儉,當今誰能為五言詩?儉對曰:「謝朏得父膏腴,江淹有意。」



 上起禪靈寺,敕抃撰碑文。



 王思遠,琅邪臨沂人。尚書令晏從弟也。父羅雲,平西長史。思遠八歲,父卒,祖弘之及外祖新安太守羊敬元,並棲退高尚,故思遠少無仕心。宋建平王景素辟為南徐州主簿,深見禮遇。景素被誅,左右離散,思遠親視殯葬,手種松柏。與廬江何昌宇、沛郡劉璡上表理之,事感朝廷。景素女廢為庶人,思遠分衣食以相資贍,年長,為備笄總,訪求素對,傾家送遣。



 除晉熙王撫軍行參軍,安成王車騎參軍。建元初,為長沙王後軍主簿,尚書殿中郎,出補竟陵王征北記室參軍,府遷司徒,仍為錄事參軍。遷太子中舍人。文惠太子與竟陵王子良素好士,並蒙賞接。思遠求出為遠郡,除建安內史。長兄思玄卒,思遠友于甚至,表乞自解,不許。及詳日,又固陳,世祖乃
 許之。除中書郎,大司馬諮議。



 世祖詔舉士,竟陵王子良薦思遠及吳郡顧暠之、陳郡殷叡。邵陵王子貞為吳郡,世祖除思遠為吳郡丞,以本官行郡事,論者以為得人。以疾解職,還為司徒諮議參軍,領錄事,轉黃門郎。出為使持節、都督廣交越三州諸軍事、寧朔將軍、平越中郎將、廣州刺史。高宗輔政,不之任,仍遷御史中丞。臨海太守沈昭略贓私,思遠依事劾奏,高宗及思遠從兄晏、昭略叔父文季請止之,思遠不從,案事如故。



 建武中,遷吏部郎。思遠以從兄晏為尚書令,不欲並居內臺權要之職,上表固讓。曰:「近頻煩歸啟,實有微概。陛下矜遇之厚,古今罕儔。臣若孤恩,誰當戮力!既自誓輕軀命,不復以塵點為疑,正以臣與晏地惟密親,必不宜俱居顯要。慺慺丹赤,守之以死。臣實庸鄙,無足獎進。陛下甄拔之旨,要是許其一節。臣果不能以理自固,有乖則哲之明。犯冒之尤,誅責在己,
 謬賞之私,惟塵聖鑒。權其輕重,寧守褊心。且亦緣陛下以德禦下,故臣可得以禮進退。伏願思垂拯宥,不使零墜。今若祗膺所忝,三公不足為泰,犯忤之後,九泉未足為劇。而臣茍求刑戮,自棄富榮,愚夫不為,臣亦庶免。此心此志,可憐可矜。如其上命必行,請罪非理,聖恩方置之通塗,而臣固求擯壓,自愍自悼,不覺涕流。謹冒鈇鉞,悉心以請。窮則呼天,仰祈一照。」上知其意,乃改授司徒左長史。



 初,高宗廢立之際,思遠與晏閑言,謂晏曰:「兄荷世祖厚恩,今一旦贊人如此事,彼或可以權計相須,未知兄將來何以自立。若及此引決,猶可不失後名。」



 晏不納。及拜驃騎,集會子弟,謂思遠兄思微曰:「隆昌之末,阿戎勸吾自裁。若從其語,豈有今日?」思遠遽應曰:「如阿戎所見,猶未晚也。」及晏敗,故得無他。



 思遠清脩,立身簡潔。衣服床筵,窮治素凈。賓客來通,輒使人先密覘視,衣服垢穢,方便不前,形
 儀新楚,乃與促膝。雖然,既去之後,猶令二人交帚拂其坐處。上從祖弟季敞性甚豪縱,上心非之,謂季敞曰:「卿可數詣王思遠。」



 上既誅晏,遷之侍中,掌優策及起居注。永元二年,遷度支尚書。未拜,卒,年四十九。贈太常,謚貞子。



 思遠與顧暠之友善。暠之卒後家貧,思遠迎其兒子,經恤甚至。



 暠之,字士明,少孤,好學有義行。初舉秀才,歷官府閣。永明末,為太子中舍人,兼尚書左丞。隆昌初,為安西諮議,兼著作,與思遠並屬文章。建武初,以疾歸家,高宗手詔與思遠曰:「此人殊可惜。」就拜中散大夫。卒,年四十九。



 思微,永元中為江州長史,為陳伯之所殺。



 史臣曰:德成為上,藝成為下。觀夫二三子之治身,豈直清體雅業,取隆基構,行禮蹈義,可以勉物風規云。君子之居世,所謂美矣!



 贊曰:江纂世業,有聞時陂。何申舊主,辭出乎義。謝獻壽觴,載
 色載刺。思遠退食,沖心篤寄。



\end{pinyinscope}