\article{卷五十一列傳第三十二 裴叔業 崔慧景 張欣泰}

\begin{pinyinscope}

 裴叔業,河東聞喜人,晉冀州刺史徽後也。徽子游擊將軍黎,遇中朝亂,子孫沒涼州,仕於張氏。黎玄孫先福,義熙末還南,至滎陽太守。叔業父祖晚渡。少便弓馬,有武幹。宋元徽末,累官為羽林監,太祖驃騎行參軍。建元元年,除屯騎校尉。虜侵司豫二州,以叔業為軍主征討,本官如故。上初即位,群下各獻讜言。二年,叔業上疏曰:「成都沃壤,四塞為固,古稱一人守隘,萬夫趑趄。雍、齊亂於漢世,譙、李寇於晉代,成敗之跡,事載前史。頃世以來,綏馭乖術,地惟形勢,
 居之者異姓,國實武用,鎮之者無兵,致寇掠充斥,賧稅不斷。宜遣帝子之尊,臨撫巴蜀,總益、梁、南秦為三州刺史。率文武萬人,先啟岷漢,分遣郡戍,皆配精力,搜盪山源,糾虔奸蠹。威令既行,民夷必服。」除寧朔將軍,軍主如故。永明四年,累至右軍將軍,東中郎諮議參軍。



 高宗為豫州,叔業為右軍司馬,加建威將軍、軍主,領陳留太守。七年,為王敬則征西司馬,將軍、軍主如故。隨府轉驃騎。在壽春為佐數年。九年,為寧蠻長史、廣平太守。雍州刺史王奐事難,叔業率部曲於城內起義。上以其有乾用,仍留為晉安王征北諮議,領中兵,扶風太守,遷晉熙王冠軍司馬。延興元年,加寧朔將軍,司馬如故。叔業早與高宗接事,高宗輔政,厚任叔業以為心腹,使領軍掩襲諸蕃鎮,叔業盡心用命。



 建武二年,虜圍徐州,叔業以軍主隸右衛將軍蕭坦之救援。叔業攻虜淮柵外二城,克之,賊眾赴水死
 甚眾。除黃門侍郎。上以叔業有勳誠,封武昌縣伯,五百戶。



 仍為持節、督徐州軍事、冠軍將軍、徐州刺史。四年,虜主寇沔北,上令叔業援雍州。叔業啟:「北人不樂遠行,唯樂侵伐虜堺,則雍司之賊,自然分張,無勞動民向遠也。」上從之。叔業率軍攻虹城,獲男女四千餘人。徙督豫州、輔國將軍、豫州刺史,持節如故。



 永泰元年,叔業領東海太守孫令終、新昌太守劉思效、馬頭太守李僧護等五萬人圍渦陽,虜南兗州所鎮,去彭城百二十里。偽兗州刺史孟表固守拒戰,叔業攻圍之,積所斬級高五丈,以示城內。又遣軍主蕭璝、成寶真分攻龍亢戍,即虜馬頭郡也。虜閉城自守。偽徐州刺史廣陵王率二萬人、騎五千匹至龍亢,璝等拒戰不敵。



 叔業三萬餘人助之,數道攻虜。虜新至,營未立,於是大敗。廣陵王與數十騎走,官軍追獲其節。虜又遣偽將劉藻、高聰繼至,叔業率軍迎擊破之,再戰,斬
 首萬級,獲生口三千人,器仗驢馬絹布千萬計。虜主聞廣陵王敗,遣偽都督王肅、大將軍楊大眼步騎十八萬救渦陽,叔業見兵盛,夜委軍遁走。明日,官軍奔潰,虜追之,傷殺不可勝數,日暮乃止。叔業還保渦口,上遣使慰勞。



 高宗崩,叔業還鎮。少主即位,誅大臣,京師屢有變發。叔業登壽春城北望肥水,謂部下曰:「卿等欲富貴乎?我言富貴亦可辦耳。」永元元年,徙督南兗兗徐青冀五州軍事、南兗州刺史,將軍、持節如故。叔業見時方亂,不樂居近蕃,朝廷疑其欲反,叔業亦遣使參察京師消息,於是異論轉盛。叔業兄子植、揚並為直閣,殿內驅使。慮禍至,棄母奔壽陽,說叔業以朝廷必見掩襲。徐世檦等慮叔業外叛,遣其宗人中書舍人裴長穆宣旨,許停本任。叔業猶不自安,而植等說之不已,叔業憂懼,問計於梁王,梁王令遣家還都,自然無患。叔業乃遣子芬之等還質京師。明年,進
 號冠軍將軍。傳叔業反者不已,芬之愈懼,復奔壽春。於是發詔討叔業,遣護軍將軍崔慧景、征虜將軍豫州刺史蕭懿督水陸眾軍西討,頓軍小峴。叔業病困,植請救魏虜,送芬之為質。叔業尋卒,虜遣大將軍李醜、楊大眼二千餘騎入壽春。



 初,虜主元宏建武二年至壽春,其下勸攻城。宏曰:「不須攻,後當降也。」植等皆還洛陽。



 崔慧景,字君山,清河東武城人也。祖構,奉朝請。父系之,州別駕。慧景初為國子學生。宋泰始中,歷位至員外郎,稍遷長水校尉,寧朔將軍。太祖在淮陰,慧景與宗人祖思同時自結。太祖欲北渡廣陵,使慧景具船於陶家後渚,事雖不遂,以此見親。除前軍。沈攸之事平,仍出為武陵王安西司馬、河東太守,使防捍陜西。



 昇明三年,豫章王為荊州,慧景留為鎮西司馬,兼諮議,太守如故。太祖受禪,封樂安縣子,三百戶。豫章王遣慧景奉表稱慶還京師,太祖召見,加意勞
 接。轉平西府司馬、南郡內史。仍遷為南蠻長史,加輔國將軍,內史如故。先是蠻府置佐,資用甚輕,至是始重其選。



 建元元年,虜動,豫章王遣慧景三千人頓方城,為司州聲援。虜退,梁州賊李烏奴未平,以慧景為持節、都督梁南北秦沙四州軍事、西戎校尉、梁南秦二州刺史,將軍如故。敕荊州資給發遣,配以實甲千人,步道從襄陽之鎮。初,烏奴屢為官軍所破,走氐中,乘間出,擾動梁、漢,據關城。遣使詣荊州請降,豫章王不許。遣中兵參軍王圖南率益州軍從劍閣掩討,大摧破之,烏奴還保武興。慧景發漢中兵眾,進頓白馬。遣支軍與圖南腹背攻擊,烏奴大敗,遂奔於武興。



 世祖即位,進號冠軍將軍。在州蓄聚,多獲珍貨。永明三年,以本號還。遷黃門郎,領羽林監。明年,遷隨王東中郎司馬,加輔國將軍。出為持節、督司州軍事、冠軍將軍、司州刺史。母喪,詔起復本任。慧景每罷州,輒傾
 資獻奉,動數百萬,世祖以此嘉之。九年,以本號征還,轉太子左率,加通直常侍。明年,遷右衛將軍,加給事中。



 是時虜將南侵,上出慧景為持節、督豫州郢州之西陽司州之汝南二郡諸軍事、冠軍將軍、豫州刺史。鬱林即位,進號征虜將軍。慧景以少主新立,密與虜交通,朝廷疑懼。高宗輔政,遣梁王至壽春安慰之,慧景遣密啟送誠勸進,徵還,為散騎常侍,左衛將軍。建武二年,虜寇徐、豫,慧景以本官假節向鐘離,受王玄邈節度。



 尋加冠軍將軍。四年,遷度支尚書,領太子左率。



 冬,虜主攻沔北五郡,假慧景節,率眾二萬,騎千匹,向襄陽。雍州眾軍並受節度。永泰元年,慧景至襄陽,五郡已沒。加慧景平北將軍,置佐史,分軍助戍樊城。慧景頓渦口村,與太子中庶子梁王及軍主前寧州刺史董仲民、劉山陽、裴帟、傅法憲等五千餘人進行鄧城。前參騎還,稱虜軍且至。須臾,望數萬騎俱來,慧景據
 南門,梁王據北門,令諸軍上城上。時慧景等蓐食輕行,皆有饑懼之色。軍中北館客三人走投虜,具告之。虜偽都督中軍大將軍彭城王元勰分遣偽武衛將軍元蚪趣城東南,斷慧景歸路,偽司馬孟斌向城東,偽右衛將軍播正屯城北,交射城內。梁王欲出戰,慧景曰:「虜不夜圍人城,待日暮自當去也。」既而虜眾轉盛,慧景於南門拔軍,眾軍不相知,隨後奔退。虜軍從北門入,劉山陽與部曲數百人斷後死戰,虜遣鎧馬百餘匹突取山陽,山陽使射手射之,三人倒馬,手殺十餘人,不能禁,且戰且退。慧景南出過鬧溝,軍人蹈藉,橋皆斷壞,虜軍夾路射之,軍主傅法憲見殺,赴溝死者相枕。山陽取襖杖填溝,乘之得免。虜主率大眾追之,晡時,虜主至沔北,圍軍主劉山陽,山陽據城苦戰,至暮,虜乃退。眾軍恐懼,其夕皆下船還襄陽。



 東昏即位,改領右衛將軍,平北、假節如故。未拜。永元元年,
 遷護軍將軍,尋加侍中。陳顯達反,加慧景平南將軍,都督眾軍事,屯中堂。時輔國將軍徐世檦專勢號令,慧景備員而已。帝既誅戮將相,舊臣皆盡,慧景自以年宿位重,轉不自安。明年,裴叔業以壽春降虜,改授慧景平西將軍,假節、侍中、護軍如故,率軍水路征壽陽。軍頓白下,將發,帝長圍屏除出瑯邪城送之。帝戎服坐城樓上,召慧景單騎進圍內,無一人自隨者。裁交數言,拜辭而去。慧景既得出,甚喜。子覺為直閣將軍,慧景密與期。四月慧景至廣陵,覺便出奔。



 慧景過廣陵數十里,召會諸軍主曰:「吾荷三帝厚恩,當顧託之重。幼主昏狂,朝廷壞亂,危而不扶,責在今日。欲與諸君共建大功,以安宗社,何如?」眾皆響應。於是回軍還廣陵,司馬崔恭祖守廣陵城,開門納之。帝聞變,以征虜將軍右衛將軍左興盛假節,督京邑水陸眾軍。慧景停二日,便收眾濟江集京口。江夏王寶玄又
 為內應,合二鎮兵力,奉寶玄向京師。



 臺遣驍騎將軍張佛護、直閣將軍徐元稱、屯騎校尉姚景珍、西中郎參軍徐景智、游盪軍主董伯珍、騎官桓靈福等據竹里為數城。寶玄遣信謂佛護曰:「身自還朝,君何意苦相斷遏?」佛護答曰:「小人荷國重恩,使於此創立小戍。殿下還朝,但自直過,豈敢干斷。」遂射慧景軍,因合戰。慧景子覺及崔恭祖領前鋒,皆傖楚善戰;又輕行不爨食。以數舫緣江載酒肉為軍糧。每見臺軍城中煙火起,輒盡力攻擊,臺軍不復得食,以此饑困。元稱等議欲降,佛護不許。十二日,恭祖等復攻之,城陷,佛護單馬走,追得斬首,徐元稱降,餘軍主皆死。慧景至臨沂,令李玉之發橋斷路,慧景收殺之。



 臺遣中領軍王瑩都督眾軍,據湖頭築壘,上帶蔣山西巖,實甲數萬。慧景至查硎,竹塘人萬副兒善射獵,能捕虎,投慧景曰:「今平路皆為臺軍所斷,不可議進。



 唯宜從蔣山龍尾
 上,出其不意耳。」慧景從之,分遣千餘人魚貫緣山,自西巖夜下,鼓叫臨城中。臺軍驚恐,即時奔散。帝又遣右衛將軍左興盛率臺內三萬人拒慧景於北籬門,望風退走。慧景引軍入樂遊苑,恭祖率輕騎十餘匹突進北掖門,乃復出,宮門皆閉。慧景引眾圍之。於是東府、石頭、白下、新亭諸城皆潰。左興盛走,不得入宮,逃淮渚荻舫中,慧景擒殺之。宮中遣兵出盪,不剋。慧景燒蘭臺府署為戰場,守衛尉蕭暢屯南掖門處分城內,隨方應擊,眾心以此稍安。



 慧景稱宣德太后令,廢帝為吳王。時巴陵王昭胄先逃民間,出投慧景,慧景意更向之,故猶豫未知所立。竹里之捷,子覺與恭祖爭勳,慧景不能決。恭祖勸慧景射火箭燒北掖樓,慧景以大事垂定,後若更造,費用功力,不從其計。性好談義,兼解佛理,頓法輪寺,對客高談。恭祖深懷怨望。



 先是衛尉蕭懿為征虜將軍、豫州刺史,自歷陽步道
 征壽陽。帝遣密使告之,懿率軍主胡松、李居士等數千人自採石濟岸,頓越城,舉火,臺城中鼓叫稱慶。恭祖先勸慧景遣二千人斷西岸軍,令不得渡,慧景以城旦夕降,外救自然應散。至是恭祖請擊義師,又不許。乃遣子覺將精手數千人渡南岸。義師昧旦進戰,數合,士皆致死,覺大敗,赴淮死者二千餘人,覺單馬退,開桁阻淮。其夜,崔恭祖與驍將劉靈運詣城降,慧景眾情離壞,乃將腹心數人潛去,欲北渡江,城北諸軍不知,猶為拒戰。城內出盪,殺數百人。義軍渡北岸,慧景餘眾皆奔。慧景圍城凡十二日,軍旅散在京師,不為營壘。及走,眾於道稍散,單馬至蟹浦,為漁父所斬,以頭內鰍魚籃,擔送至京師,時年六十三。



 追贈張佛護為司州刺史,左興盛豫州刺史,並征虜將軍,徐景智、桓靈福屯騎校尉,董伯珍員外郎,李玉之給事中,其餘有差。



 恭祖者,慧景宗人,驍果便馬槊,氣力絕人,
 頻經軍陣,討王敬則,與左興盛軍容袁文曠爭敬則首,訴明帝曰:「恭祖禿馬絳衫,手刺倒賊,故文曠得斬其首。



 以死易勛,而見枉奪。若失此勛,要當刺殺左興盛。」帝以其勇健,使謂興盛曰:「何容令恭祖與文曠爭功。」遂封二百戶。慧景平後,恭祖系尚方,少時殺之。



 覺亡命為道人,見執伏法。臨刑與妹書曰:「捨逆旅,歸其家,以為大樂;況得從先君遊太清乎!古人有力扛周鼎,而有立錐之嘆,以此言死,亦復何傷!平生素心,士大夫皆知之矣。既不得附驥尾,安得施名於後世?慕古竹帛之事,今皆亡矣。」慧景妻女亦頗知佛義。



 覺弟偃,為始安內史,藏竄得免。和帝西臺立,以為寧朔將軍。中興元年,詣公車門上書曰:「臣竊惟太祖、高宗之孝子忠臣,而昏主之賊臣亂子者,江夏王與陛下,先臣與鎮軍是也。臣聞堯舜之心,常以天下為憂,而不以位為樂。彼孑然之舜,壟畝之人,猶尚若此;況祖業之
 重,家國之切?江夏既行之於前,陛下又蹈之於後,雖成敗異術,而所由同方也。陛下初登至尊,與天合符。天下纖介之屈,尚望陛下申之,絲髮之冤,尚望陛下理之,況先帝之子,陛下之兄,所行之道,即陛下所由哉?如此尚弗恤,其餘何幾哉?陛下德侔造化,仁育群生,雖在昆蟲草木,有不得其所者,覽而傷焉,而況乎友愛天至,孔懷之深!夫豈不懷,將以事割。此實左右不明,未之或詳。惟陛下公聽並觀,以詢之芻蕘。群臣有以臣言為不可,乞使臣廷辯之,則天人之意塞,四海之疑釋。必若不然,幸小民之無識耳。使其曉然知此,相聚而逃陛下,以責江夏之冤,朝廷將何以應之哉?若天聽沛然回光,發惻愴之詔,而使東牟朱虛東褒儀父之節,則荷戈之士,誰不盡死?愚戇之言,萬一上合,事乞留中。」



 事寢不報。偃又上疏曰:近冒陳江夏之冤,定承聖詔,已有褒贈,此臣狂疏之罪也。然臣所
 以諮問者,不得其實,罪在萬沒,無所復云。但愚心所恨,非敢以父子之親,骨肉之間,而僥幸曲陛下之法,傷至公之義。誠不曉聖朝所以然之意。若以狂主雖狂,而實是天子,江夏雖賢,實是人臣,先臣奉人臣逆人君,以為不可申明詔,得矣;然未審陛下亦是人臣不?而鎮軍亦復奉人臣逆人君,今之嚴兵勁卒,方指於象魏者,其故何哉?



 臣所不死,茍存視息,非有他故,所以待皇運之開泰,申冤魂之枉屈。今皇運既已開泰矣,而死於社稷盡忠,反以為賊,臣何用此生陛下世矣。



 臣聞王臣之節,竭智盡公以奉其上;居股肱之任者,申理冤滯,薦達群賢。凡此眾臣,夙興夜寐,心未嘗須臾之間而不在公。故萬物無不得其理,而頌聲作焉。



 臣謹案鎮軍將軍臣穎胄,宗室之親,股肱之重,身有伊、霍之功,荷陛下稷、旦之任。中領軍臣詳,受帷幄之寄,副宰相之尊。皆所以棟梁朝廷,社稷之臣,天
 下所當,遑遑匪懈,盡忠竭誠,欲使萬物得理,而頌聲大興者,豈復宜踰此哉?而同知先臣股肱江夏,匡濟王室,天命未遂,王亡與亡,而不為陛下瞥然一言。知而不言,是不忠之臣,不知而言,乃不智之臣,此而不知,將何所知?如以江夏心異先臣,受制臣力,則江夏同致死斃,聽可昏政淫刑,見殘無道。然江夏之異,以何為明,孔、呂二人,誰以為戮?手御麾幡,言輒任公,同心共志,心若膠漆,而以為異,臣竊惑焉。如以先臣遣使,江夏斬之,則征東之驛,何為見戮?陛下斬征東之使,實詐山陽;江夏違先臣之請,實謀孔矜。天命有歸,故事業不遂耳。夫唯聖人,乃知天命,守忠之臣,唯知盡死,安顧成敗。詔稱江夏遭時屯故,跡屈行令,內恕探情,無玷純節。今茲之旨,又何以處鎮軍哉?



 臣所言畢矣,乞就湯鑊。然臣雖萬沒,猶願陛下必申先臣。何則?惻愴而申之,則天下伏;不惻愴而申之,天下之人
 北面而事陛下者,徒以力屈耳。先臣之忠,有識所知,南史之筆,千載可期,亦何待陛下屈申而為褒貶。然小臣惓惓之愚,為陛下計耳。臣之所言,非孝於父,實忠於君。唯陛下熟察,少留心焉。



 臣頻觸宸嚴,而不彰露,所以每上封事者,非自為戇地,猶以《春秋》之義有隱諱之意也。臣雖淺薄,然今日之事,斬足斷頭,殘身滅形,何所不能?為陛下耳。



 臣聞生人之死,肉人之骨,有識之士,未為多感。公聽並觀,申人之冤,秉德任公,理人之屈,則普天之人,爭為之死。何則?理之所不可以已也。陛下若引臣冤,免臣兄之罪,收往失,發惻愴之詔,懷可報之意,則桀之犬實可吠堯,跖之客實可刺由,又何況由之犬,堯之客?臣非吝生,實為陛下重此名於天下。已成之基,可惜之寶,莫復是加。浸明浸昌,不可不循,浸微浸滅,不可不慎。惟陛下熟察,詳擇其衷。



 若陛下猶以為疑,鎮軍未之允決,乞下征東共
 詳可否。無以向隅之悲,而傷陛下滿堂之樂。何則?陛下昏主之弟,江夏亦昏主之弟;鎮軍受遺托之恩,先臣亦荷顧命之重。情節無異,所為皆同,殊者唯以成敗仰資聖朝耳。臣不勝愚忠,請使群臣廷辯者,臣乞專令一人,精賜本語,僥幸萬一,天聽昭然,則軻沈七族,離燔妻子,人以為難,臣豈不易!



 詔報曰:「具卿冤切之懷。卿門首義,而旌德未彰,亦追以慨然。今當顯加贈謚。」偃尋下獄死。



 張欣泰,字義亨,竟陵人也。父興世,宋左衛將軍。欣泰少有志節,不以武業自居,好隸書,讀子史。年十餘,詣吏部尚書褚淵,淵問之曰:「張郎弓馬多少?」



 欣泰答曰:「性怯畏馬,無力牽弓。」淵甚異之。闢州主簿,歷諸王府佐。元徽中,興世在家,擁雍州還資,見錢三千萬。蒼梧王自領人劫之,一夜垂盡,興世憂懼感病卒。欣泰兄欣華時任安成郡,欣泰悉封餘財以待之。



 建元初,歷官寧朔將軍,累除尚書都
 官郎。世祖與欣泰早經款遇,及即位,以為直閣將軍,領禁旅。除豫章王太尉參軍,出為安遠護軍、武陵內史。還復為直閣,步兵校尉,領羽林監。欣泰通涉雅俗,交結多是名素。下直輒遊園池,著鹿皮冠,衲衣錫杖,挾素琴。有以啟世祖者,世祖曰:「將家兒何敢作此舉止!」後從車駕出新林,敕欣泰甲仗廉察,欣泰停仗,於松樹下飲酒賦詩。制局監呂文度過見,啟世祖。世祖大怒,遣出外,數日,意稍釋,召還,謂之曰:「卿不樂為武職驅使,當處卿以清貫。」除正員郎。



 永明八年,出為鎮軍中兵參軍、南平內史。巴東王子響殺僚佐,上遣中庶子胡諧之西討,使欣泰為副。欣泰謂諧之曰:「今太歲在西南,逆歲行軍,兵家深忌,不可見戰,戰必見危。今段此行,勝既無名,負誠可恥。彼凶狡相聚,所以為其用者,或利賞逼威,無由自潰。若且頓軍夏口,宣示禍福,可不戰而擒也。」諧之不從,進屯江津,尹略等見
 殺。事平,欣泰徙為隨王子隆鎮西中兵,改領河東內史。



 子隆深相愛納,數與談宴,州府職局,多使關領,意遇與謝朓相次。典簽密以啟聞,世祖怒,召還都。屏居家巷,置宅南岡下,面接松山。欣泰負弩射雉,恣情閑放。



 眾伎雜藝,頗多閑解。



 明帝即位,為領軍長史,遷諮議參軍。上書陳便宜二十條,其一條言宜毀廢塔寺。帝並優詔報答。



 建武二年,虜圍鐘離城。欣泰為軍主,隨崔慧景救援。欣泰移虜廣陵侯曰:「聞攻鐘離是子之深策,可無謬哉!《兵法》云:『城有所不攻,地有所不爭。』豈不聞之乎?我國家舟舸百萬,覆江橫海,所以案甲於今不至,欲以邊城疲魏士卒。



 我且千里運糧,行留俱弊,一時霖雨,川谷涌溢,然後乘帆渡海,百萬齊進,子復奚以御之?乃令魏主以萬乘之重,攻此小城,是何謂歟?攻而不拔,誰之恥邪?假令能拔,子守之,我將連舟千里,舳艫相屬,西過壽陽,東接滄海,仗不再請,
 糧不更取,士卒偃臥,起而接戰,乃魚鱉不通,飛鳥斷絕,偏師淮左,其不能守,晈可知矣。如其不拔,吾將假法于魏之有司,以請子之過。若挫兵夷眾,攻不卒下,驅士填隍,拔而不能守,則魏朝名士,其當別有深致乎?吾所未能量。昔魏之太武佛狸,傾一國之眾,攻十雉之城,死亡太半,僅以身返。既智屈於金墉,亦雖拔而不守,皆算失所為,至今為笑。前鑒未遠,已忘之乎?和門邑邑,戲載往意。」



 虜既為徐州軍所挫,更欲於邵陽洲築城。慧景慮為大患。欣泰曰:「虜所以築城者,外示姱大,實懼我躡其後耳。今若說之以彼此各願罷兵,則其患自息。」慧景從之。遣欣泰至虜城下具述此意。及虜引退,而洲上餘兵萬人,求輸五百匹馬假道,慧景欲斷路攻之。欣泰說慧景曰:「歸師勿遏,古人畏之。死地之兵,不可輕也。勝之既不足為武,敗則徒喪前功。不如許之。」慧景乃聽虜過。時領軍蕭坦之亦援
 鐘離,還啟明帝曰:「邵陽洲有死賊萬人,慧景、欣泰放而不取。」帝以此皆不加賞。



 四年,出為永陽太守。永元初,還都。崔慧景圍城,欣泰入城內,領軍守備。



 事寧,除輔國將軍、廬陵王安東司馬。義師起,以欣泰為持節、督雍梁南北秦四州郢州之竟陵司州之隨郡軍事、雍州刺史,將軍如故。時少帝昏亂,人情咸伺事隙。



 欣泰與弟前始安內史欣時密謀結太子右率胡松、前南譙太守王靈秀、直閣將軍鴻選、含德主帥茍勵、直後劉靈運等十餘人,並同契會。



 帝遣中書舍人馮元嗣監軍救郢,茹法珍、梅蟲兒及太子右率李居士、制局監楊明泰等十餘人相送中興堂。欣泰等使人懷刀於座斫元嗣,頭墜果柈中,又斫明泰,破其腹,蟲兒傷刺數瘡,手指皆墮。居士逾牆得出,茹法珍亦散走還臺。靈秀仍往石頭迎建安王寶夤,率文武數百,唱警蹕,至杜姥宅。欣泰初聞事發,馳馬入宮,冀法珍等
 在外,城內處分,必盡見委,表裏相應,因行廢立。既而法珍得反,處分閉門上仗,不配欣泰兵,鴻選在殿內亦不敢發。城外眾尋散。少日事覺,詔收欣泰、胡松等,皆伏誅。



 欣泰少時有人相其當得三公,而年裁三十。後屋瓦墮傷額,又問相者,云「無復公相,年壽更增,亦可得方伯耳」。死時年四十六。



 史臣曰:崔慧景宿將老臣,憂危昏運,回董御之威,舉晉陽之甲,乘機用權,內襲少主,因樂亂之民,藉淮楚之剽,驍將授首,群帥委律,鼓鼙讙於宮寢,戈戟跱於城隍,陵埤負戶,士衰氣竭,屢發銅虎之兵,未有釋位之援,勢等易京,魚爛待盡。征虜將軍投袂以先國急,束馬旅師,橫江競濟,風驅電掃,制勝轉丸。越城之戰,旗獲蔽野,津行之捷,獻俘象魏。瞻塵望烽,窮壘重闢,戮帶定襄,曾未及此。盛矣哉,桓文異世也。



 贊曰:叔業外叛,淮肥失險。慧景倒戈,宮門晝掩。欣泰倉卒,霜刃不染。實起時昏,堅冰互漸。



\end{pinyinscope}