\article{卷五本紀第五海陵王}

\begin{pinyinscope}

 海陵恭王昭文,字季尚,文惠
 太子第二子也。永明四年,封臨汝公,邑千五百戶。初為輔國將軍、濟陽太守。十年,轉持節、督南豫州諸軍事、南豫州刺史,將軍如故。十一年,進號冠軍將軍。文惠太子薨,還都。鬱林王即位,為中軍將軍,領兵置佐。封新安王,邑二千戶。隆昌元年,為使持節、都督揚南徐二州諸軍事、揚州刺史,將軍如故。其年,鬱林王廢,尚書令西昌侯鸞議立昭文為帝。



 延興元年秋,七月,丁酉,即皇帝位。以尚書令、鎮軍大將軍、西昌侯鸞為驃騎大將軍、錄尚書事、揚州刺史、宣城郡公。詔曰:「太祖高皇帝英謀光大,受命作齊;世祖武皇帝宏猷冠世,繼暉下武;世宗文皇帝清明懿鑠,
 四海宅心:並德漏下泉,功昭上象,聲教所覃,無思不洽。洪基式固,景祚方融,而天步多阻,運鐘否剝。嗣君昏忍,暴戾滋多,棄侮天經,悖滅人紀。朝野重足,遐邇側視,民怨神恫,宗祧如綴。賴忠謨肅舉,霄漢廓清,俾三後之業,絕而更紐,七百之慶,危而復安。猥以沖人,入纂乾緒,載懷馭朽,若墜諸淵,思與黎元,共綏戩福。」大赦,改元。文武賜位二等。



 八月,甲辰,以新除衛尉蕭諶為中領軍,司空王敬則進位太尉,新除車騎大將軍陳顯達為司空,尚書左僕射王晏為尚書令,左衛將軍王廣之為豫州刺史,驃騎大將軍鄱陽王鏘為司徒。詔遣大使巡行風俗。丁未,詔曰:「新安國五品以上,悉與滿敘;自此以下,皆聽解遣。其欲仕者,適其所樂。」以驍騎將軍河東王鉉為南徐州刺史,西中郎將臨海王昭秀為車騎將軍,南徐州刺史永嘉王昭粲為荊州刺史。戊申,以輔國將軍王詡為廣州刺史,中
 書郎蕭遙欣為兗州刺史。庚戌,以車騎板行參軍李慶綜為寧州刺史。辛亥,以安西將軍王玄邈為中護軍,新除後軍司馬蕭誕為徐州刺史。壬子,以冠軍司馬臧靈智為交州刺史。乙卯,申明織成、金薄、彩花、錦繡履之禁。



 九月,癸酉,詔曰:「頃者以淮關徭戍,勤瘁於行役,故覃以榮階,薄酬厥勞。



 勛狀淹留,未集王府,非所以急舍爵之典,趣報功之旨。便可分遣使部,往彼銓用。」



 辛巳,以前九真太守宋慈明為交州刺史。癸未,誅新除司徒鄱陽王鏘、中軍大將軍隨郡王子隆。遣平西將軍王廣之誅南兗州刺史安陸王子敬。於是江州刺史晉安王子懋起兵,遣中護軍王玄邈討之。乙未,驃騎大將軍鸞假黃鋮,內外纂嚴。又誅湘州刺史南平王銳、郢州刺史晉熙王金求、南豫州刺史宜都王鏗。丁亥,以衛將軍廬陵王子卿為司徒,撫軍將軍桂陽王鑠為中軍將軍、開府儀同三司。冬,十月,癸巳,詔曰:「
 周設媒官,趣及時之制,漢務輕徭,在休息之典,所以布德弘教,寬俗阜民。朕君制八紘,志敷九德,而習俗之風,為弊未改,靜言多慍,無忘昏昃。督勸婚嫁,宜嚴更申明,必使禽幣以時,摽梅息怨。正廚諸役,舊出州郡,徵吏民以應其數,公獲二旬,私累數朔。又廣陵年常遞出千人以助淮戍,勞擾為煩,抑亦苞苴是育。今並可長停,別量所出。諸縣使村長路都防城直縣,為劇尤深,亦宜禁斷。」



 丁酉,解嚴。進驃騎大將軍、揚州刺史宣城公鸞為太傅,領大將軍、揚州牧,加殊禮,進爵為王。戊戌,誅新除中軍將軍桂陽王鑠、撫軍將軍衡陽王鈞、侍中秘書監江夏王鋒、鎮軍將軍建安王子真、左將軍巴陵王子倫。癸卯,以寧朔將軍蕭遙欣為豫州刺史,新除黃門郎蕭遙昌為郢州刺史,輔國將軍蕭誕為司州刺史。宣城王輔政,帝起居皆諮而後行。思食蒸魚菜,太官令答無錄公命,竟不與。
 辛亥,皇太后令曰:「司空、後將軍、丹陽尹、右僕射、中領軍、八座:夫明晦迭來,屯平代有,上靈所以眷命,億兆所以歸懷。自皇家淳耀,列聖繼軌,諸侯官方,百神受職。而殷憂時啟,多難薦臻,隆昌失德,特紊人鬼,非徒四海解體,乃亦九鼎將移。賴天縱英輔,大匡社稷,崩基重造,墜典再興。嗣主幼沖,庶政多昧,且早嬰尪疾,弗克負荷,所以宗正內侮,戚藩外叛,覘天視地,人各有心。雖三祖之德在民,而七廟之危行及。自非樹以長君,鎮以淵器,未允天人之望,寧息奸宄之謀!太傅宣城王胤體宣皇,鐘慈太祖,識冠生民,功高造物,符表夙著,謳頌有在,宜入承寶命,式寧宗祏。帝可降封海陵王,吾當歸老別館。昔宣帝中興漢室,簡文重延晉祀,庶我鴻基,於茲永固。言念家國,感慶載懷。」



 建武元年,詔「海陵王依漢東海王彊故事,給虎賁、旄頭、畫輪車,設鐘虡宮縣,供奉所須,每存隆厚。」十一月,稱王
 有疾,數遣御師占視,乃殞之。給溫明秘器,衣一襲,斂以袞冕之服。大鴻臚監護喪事。葬給轀輬車,九旒大輅,黃屋左纛,前後部羽葆鼓吹,挽歌二部,依東海王故事。謚曰恭王。年十五。



 史臣曰:郭璞稱永昌之名,有二日之象,而隆昌之號亦同焉。案漢中平六年,獻帝即位,便改元為光熹,張讓、段圭誅後,改元為昭寧,董卓輔政,改元為永漢,一歲四號也。晉惠帝太安二年,長沙王鳷事敗,成都王穎改元為永安;穎自鄴奪,河間王顒復改元為永興,一歲三號也。隆昌、延興、建武,亦三改年號。故知喪亂之軌迹,雖千載而必同矣。



 贊曰:穆穆海陵,因亡代興。不先不後,遭命是膺。



\end{pinyinscope}