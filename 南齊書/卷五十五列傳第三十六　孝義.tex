\article{卷五十五列傳第三十六 孝義}

\begin{pinyinscope}

 崔懷填公孫僧遠吳欣之韓係伯孫淡華寶韓靈敏封延伯吳達之
 王文殊朱謙之蕭睿明樂頤江泌杜棲陸絳子曰:「父子之道,天性也,君臣之義也。」人之含孝稟義,天生所同,淳薄因心,非俟學至。遲遇為用,不謝始庶之法;驕慢之性,多慚水菽之享。夫色養盡力,行義致身,甘心壟畝,不求聞達,斯即孟氏三樂之辭,仲由負米之歎也。通乎神明,理緣感召。情澆世薄,方表孝慈。故非內德者所以寄心,懷仁者所以標物矣。



 埋名韞節,鮮或昭著,
 紀夫事行,以列于篇。



 崔懷慎,清河東武城人也。父邪利,魯郡太守,宋元嘉中沒虜。懷慎與妻房氏篤愛,聞父陷沒,即日遣妻,布衣蔬食,如居喪禮。邪利後仕虜中書,戒懷慎不許如此,懷慎得書更號泣。懷填從叔模為滎陽太守,亦同沒虜,模子雖居處改節,而不廢婚宦。大明中,懷慎宗人冀州刺史元孫北使,虜問之曰:「崔邪利、模並力屈歸命,二家子侄,出處不同,義將安在?」元孫曰:「王尊驅驥,王陽回車,欲令忠孝並弘,臣子兩節。」泰始初,淮北陷沒,界上流奔者多有去就,懷慎因此入北。



 至桑乾,邪利時已卒,懷慎絕而後蘇。載喪還青州,徒跣冰雪,土氣寒酷,而手足不傷,時人以為孝感。喪畢,以弟在南,建元初又逃歸,而弟亦已亡。懷慎孤貧獨立,宗黨哀之,日斂給其升米。永明中卒。



 公孫僧遠,會稽剡人也。治父喪至孝,事母及伯父甚謹。年穀饑貴,僧遠省餐減食以供母及伯。弟亡,無以葬,身販貼與鄰里,供斂送之費。躬負土,手種松柏。



 兄姊未婚嫁,乃自賣為之成禮。名聞郡縣。太祖即位,遣兼散騎常侍虞炎等十二部使行天下,建元三年,表列僧遠等二十三人,詔並表門閭,蠲租稅。



 吳欣之,晉陵利城人也。宋元嘉末,弟尉之為武進縣戍,隨王誕起義,太初遣軍主華欽討之,吏民皆散,尉之獨留,見執將死。欣之詣欽乞代弟命,辭淚哀切,兄弟皆見原。建元三年,有詔蠲表。



 永明初,廣陵民章起之二息犯罪爭死,太守劉悛表以聞。



 韓係伯,襄陽人也。事父母謹孝。襄陽土俗,鄰居種桑樹於界上為志,係伯以桑枝蔭妨他地,遷界上開數尺,鄰畔隨復侵之,係伯輒更改種。久之,鄰人慚愧,還所侵地,躬往謝之。建
 元三年,蠲租稅,表門閭。以壽終。



 孫淡,太原人也。居長沙,事母孝。母疾,不眠食,以差為期。母哀之,後有疾,不使知也。豫章王領湘州,辟驃騎行參軍。建元三年,蠲租稅,表門閭。卒于家。



 華寶,晉陵無錫人也。父豪,義熙末戍長安,寶年八歲。臨別,謂寶曰:「須我還,當為汝上頭。」長安陷虜,豪歿。寶年至七十,不婚冠,或問之者,輒號慟彌日,不忍答也。



 同郡薛天生,母遭艱菜食,天生亦菜食,母未免喪而死,天生終身不食魚肉。



 與弟有恩義。



 又同郡劉懷胤與弟懷則,年十歲,遭父喪,不衣絮帛,不食鹽菜。建元三年,並表門閭。



 韓靈敏,會稽剡人也。早孤,與兄靈珍並有孝性。尋母又亡,家貧無以營凶,兄弟共種瓜半畝,朝採瓜子,暮已復生,以此遂辦葬事。靈珍亡,無子,妻卓氏守節不嫁,慮家人奪其志,未嘗告歸,靈敏事之
 如母。



 晉陵吳康之妻趙氏,父亡弟幼,值歲饑,母老病篤,趙詣鄉里自賣,言辭哀切,鄉里憐之,人人分升米相救,遂得免。及嫁康之,少時夫亡,家欲更嫁,誓死不貳。



 義興蔣雋之妻黃氏,夫亡不重嫁,逼之,欲赴水自殺,乃止。建元三年,詔蠲租賦,表門閭。



 永明元年,會稽永興吳翼之母丁氏,少喪夫,性仁愛。遭年荒,分衣食以貽里中饑餓者,鄰里求借,未嘗違。同里陳穰父母死,孤單無親戚,丁氏收養之,及長,為營婚娶。又同里王禮妻徐氏,荒年客死山陰,丁為買棺器,自往斂葬。元徽末,大雪,商旅斷行,村里比屋饑餓,丁自出鹽米,計口分賦。同里左僑家露四喪,無以葬,丁為辦塚槨。有三調不登者,代為輸送。丁長子婦王氏守寡執志不再醮。州郡上言,詔表門閭,蠲租稅。



 又廣陵徐靈禮妻遭火救兒,與兒俱焚死。太守劉悛以聞。



 又會稽人陳氏,有三女,無男。祖父母年八九十,老耄無所知,父篤
 癃病,母不安其室。值歲饑,三女相率於西湖採菱蓴,更日至市貨賣,未嘗虧怠。鄉里稱為義門,多欲取為婦,長女自傷煢獨,誓不肯行。祖父母尋相繼卒,三女自營殯葬,為庵舍墓側。



 又永興概中里王氏女,年五歲,得毒病,兩目皆盲。性至孝,年二十,父母死,臨屍一叫,眼皆血出,小妹娥舐其血,左目即開,時人稱為孝感。縣令何曇秀不以聞。



 又諸暨東洿里屠氏女,父失明,母痼疾,親戚相棄,鄉里不容。女移父母遠住羅,晝樵採,夜紡績,以供養。父母俱卒,親營殯葬,負土成墳。忽聞空中有聲云:「汝至性可重,山神欲相驅使。汝可為人治病,必得大富。」女謂是妖魅,弗敢從,遂得病。積時,鄰舍人有中溪蜮毒者,女試治之,自覺病便差,遂以巫道為人治疾,無不愈。家產日益,鄉里多欲娶之,以無兄弟,誓守墳墓不肯嫁,為山賊劫殺。縣令於琳之具言郡,太守王敬則不以聞。



 建武三年,吳興乘
 公濟妻姚氏生二男,而公濟及兄公願、乾伯並卒,各有一子欣之、天保,姚養育之,賣田宅為娶婦,自與二男寄止鄰家。明帝詔為其二子婚,表門閭,復徭役。



 吳郡范法恂妻褚氏,亦勤苦執婦業。宋升明中,孫曇瓘謀反亡命,褚謂其子僧簡曰:「孫越州先姑之姊子,與汝父親則從母兄弟,交則義重古人。逃竄脫不免,汝宜收之。」曇瓘尋伏法,褚氏令僧簡往斂葬。年七十餘,永明中卒。僧簡在都,聞病馳歸,未至而褚已卒,將殯,舉尸不起,尋而僧簡至焉。



 封延伯,字仲璉,渤海人也。有學行,不與世人交,事寡嫂甚謹。州辟主簿,舉秀才,不就。後乃仕。垣崇祖為豫州,啟太祖用為長史,帶梁郡太守。以疾自免,僑居東海,遂不至京師。三世同財,為北州所宗附。豫章王辟中兵,不就,卒。



 建元三年,大使巡行天下,義興陳玄子四世一百七十口同居。武陵郡邵榮興、文獻叔八世同居。東海徐
 生之、武陵范安祖、李聖伯、范道根五世同居。零陵譚弘寶、衡陽何弘、華陽陽黑頭疏從四世同居,並共衣食。詔表門閭,蠲租稅。又蜀郡王續祖、華陽郝道福並累世同爨。建武三年,明帝詔表門閭,蠲調役。



 吳達之,義興人也。嫂亡無以葬,自賣為十夫客以營塚槨。從祖弟敬伯夫妻荒年被略賣江北,達之有田十畝,貨以贖之,與之同財共宅。郡命為主簿,固以讓兄。



 又讓世業舊田與族弟,弟亦不受,田遂閑廢。建元三年,詔表門閭。



 河南辛普明僑居會稽,自少與兄同處一帳,兄亡,以帳施靈座,夏月多蚊,普明不以露寢見色。兄將葬,鄰人嘉其義,賻助甚多,普明初受,後皆反之。贈者甚怪,普明曰:「本以兄墓不周,故不逆來意。今何忍亡者餘物以為家財。」後遭母喪,幾至毀滅。揚州刺史豫章王辟為議曹從事。年五十卒。



 又有何伯
 璵,弟幼璵,俱厲節操。養孤兄子,及長為婚,推家業盡與之。安貧枯槁,誨人不倦,鄉里呼為人師。郡守下車,莫不修謁。永明十一年,伯璵卒。幼璵少好佛法,翦落長齋,持行精苦。梁初卒。兄弟年並八十餘。



 王文殊,吳興故鄣人也。父沒虜,文殊思慕泣血,蔬食山谷三十餘年。太守謝抃板為功曹,不就。永明十一年,太守孔琇之表曰:「文殊性挺五常,心符三教。



 以父沒獯庭,抱終身之痛,專席恒居,銜罔極之恤。服糸寧縞以經年,餌蔬菽以俟命,婚義滅於天情,官序空於素抱。儻降甄異之恩,榜其閭里」。鬱林詔榜門,改所居為「孝行里」。



 朱謙之,字處光,吳郡錢唐人也。父昭之,以學解稱於鄉里,謙之年數歲,所生母亡,昭之假葬田側,為族人朱幼方燎火所焚。同產姊密語之,謙之雖小,便哀戚如持喪。年長不婚娶。永明中,手刃殺幼方,詣獄自繫。縣令申靈勖表上,別駕孔稚圭、兼記室劉璡、司徒左西掾
 張融箋與刺史豫章王曰:「禮開報仇之典,以申孝義之情;法斷相殺之條,以表權時之制。謙之揮刃酬冤,既申私禮;繫頸就死,又明公法。今仍殺之,則成當世罪人;宥而活之,即為盛朝孝子。殺一罪人,未足弘憲;活一孝子,實廣風德。張緒、陸澄,是其鄉舊,應具來由。融等與謙之並不相識,區區短見,深有恨然。」豫章王言之世祖,時吳郡太守王慈、太常張緒、尚書陸澄並表論其事,世祖嘉其義,慮相復報,乃遣謙之隨曹虎西行。將發,幼方子惲於津陽門伺殺謙之,謙之之兄選之又刺殺惲,有司以聞。世祖曰:「此皆是義事,不可問。」悉赦之。吳興沈摐聞而歎曰:「弟死於孝,兄殉於義。孝友之節,萃此一門。」選之字處林,有志節,著《辯相論》。幼時顧歡見而異之,以女妻焉。官至江夏王參軍。



 蕭睿明,南蘭陵人。領軍將軍諶從祖兄弟也。父孝孫,左軍。睿明初
 仕員外殿中將軍,少有至性,奉親謹篤。母病躬禱,夕不假寐,及亡,不勝哀而卒。永明五年,世祖詔曰:「龍驤將軍、安西中兵參軍、松滋令蕭睿明,愛敬淳深,色養盡禮,喪過乎哀,遂致毀滅。雖未達聖教,而一至可愍。宜加榮命,以矜善人。可贈中書郎。」



 樂頤,字文德,南陽涅陽人。世居南郡。少而言行和謹,仕為京府參軍。父在郢州病亡,頤忽思父涕泣,因請假還,中路果得父凶問。頤便徒跣號咷,出陶家後渚,遇商人附載西上,水漿不入口數日。嘗遇病,與母隔壁,忍痛不言,嚙被至碎,恐母之哀己也。湘州刺史王僧虔引為主簿,以同僚非人,棄官去。吏部郎庾杲之嘗往候,頤為設食,枯魚菜菹而已。杲之曰:「我不能食此。」母聞之,自出常膳魚羹數種。杲之曰:「卿過於茅季偉,我非郭林宗。」仕至郢州治中,卒。



 弟預亦孝,父臨亡,執其手以托郢州行事王奐,預悲感悶絕,吐血
 數升,遂發病。官至驃騎錄事。隆昌末,預謂丹陽尹徐孝嗣曰:「外傳藉藉,似有伊周之事,君蒙武帝殊常之恩,荷託付之重,恐不得同人此舉。人笑褚公,至今齒冷。」孝嗣心甚納之。建武中為永世令,民懷其德。卒官。有一老嫗行擔斛蔌葉將詣市,聞預死,棄擔號泣。



 鴈門解仲恭,亦僑居南郡。家行敦睦,得纖豪財利,輒與兄弟平分。母病經時不差,入山採藥,遇一老父語之曰:「得丁公藤,病立愈。此藤近在前山際高樹垂下便是也。」忽然不見。仲恭如其言得之,治病,母即差。至今江陵人猶有識此藤者。



 江泌,字士清,濟陽考城人也。父亮之,員外郎。泌少貧,晝日斫屟,夜讀書,隨月光握卷升屋。性行仁義,衣弊,恐虱饑死,乃復取置衣中。數日間,終身無復虱。母亡後,以生闕供養,遇鮭不忍食。食菜不食心,以其有生意也。歷仕南中郎行參軍,所給募吏去役,得時病,莫有舍
 之者,吏扶杖投泌,泌親自隱恤,吏死,泌為買棺。無僮役,兄弟共輿埋之。領國子助教。乘牽車至染烏頭,見老翁步行,下車載之,躬自步去。世祖以為南康王子琳侍讀。建武中,明帝害諸王後,泌憂念子琳,詣志公道人問其禍福。志公覆香爐灰示之曰:「都盡,無所餘。」及子琳被害,泌往哭之,淚盡,繼之以血。親視殯葬,乃去。時廣漢王侍讀嚴桓之亦哭王盡哀。泌尋卒。泌族人兗州治中泌,黃門郎悆子也。與泌同名。世謂泌為「孝江泌」



 以別之。



 杜棲,字孟山,吳郡錢唐人,徵士京產子也。同郡張融與京產相友,每相造言論,棲常在側。融指棲曰:「昔陳太丘之召元方,方之為劣。以今方古,古人何貴。」



 棲出京師,從儒士劉瓛受學。善清言,能彈琴飲酒,名儒貴游多敬待之。中書郎周顒與京產書曰:「賢子學業清標,後來之秀。嗟愛之懷,豈知雲已。所謂人之英彥,若
 己有之也。」刺史豫章王聞其名,闢議曹從事,仍轉西曹佐。竟陵王子良數致禮接。國子祭酒何胤治禮,又重棲,以為學士,掌婚冠儀。以父老歸養,怡情壟畝。



 棲肥白長壯,及京產疾,旬日間便皮骨自支。京產亡,水漿不入口七日,晨夕不罷哭,不食鹽菜。每營買祭奠,身自看視,號泣不自持。朔望節歲,絕而復續,吐血數升。時何胤、謝朏並隱東山,遺書敦譬,誡以毀滅。至祥禫,暮夢見其父,慟哭而絕。初,胤兄點見棲歎曰:「卿風韻如此,雖獲嘉譽,不永年矣。」卒時年三十六。當世咸嗟惜焉。



 建武二年,剡縣有小兒,年八歲,與母俱得赤班病。母死,家人以小兒猶惡,不令其知。小兒疑之,問云:「母嘗數問我病,昨來覺聲羸,今不復聞,何謂也?」



 因自投下床,匍匐至母尸側,頓絕而死。鄉鄰告之縣令宗善才,求表廬,事竟不行。



 陸絳,字魏卿,吳郡人也。父閑,字遐業,有風概,與人交,不茍合。少為
 同郡張緒所知,仕至揚州別駕。明帝崩,閑謂所親曰:「宮車晏駕,百司將聽於冢宰。



 主王地重才弱,必不能振,難將至矣。」乃感心疾,不復預州事。刺史始安王遙光反,事敗,閑以綱佐被召至杜姥宅,尚書令徐孝嗣啟閑不預逆謀,未及報,徐世檦令殺之。絳時隨閑,抱閑頸乞代死,遂并見殺。



 史臣曰:澆風一起,人倫毀薄,抑引之教徒聞,珪璋之璞罕就。若令事長移忠,儻非行舉,姜桂辛酸,容遷本質。而旌閭變里,問餼存牢,不過鰥寡齊矜,力田等勸。其於扶獎名教,未為多也。



 贊曰:孝為行首,義實因心。白華秉節,寒木齊心。



\end{pinyinscope}