\article{卷五十四列傳第三十五 高逸}

\begin{pinyinscope}

 褚伯玉明僧紹顧歡臧榮緒何求劉虯庾易宗測杜京產
 沈飀士吳苞徐伯珍《易》有君子之道四焉,語默之謂也。故有入廟堂而不出,徇江湖而永歸,隱避紛紜,情迹萬品。若道義內足,希微兩亡,藏景窮巖,蔽名愚谷,解桎梏於仁義,示形神於天壤,則名教之外,別有風猷。故堯封有非聖之人,孔門謬雞黍之客。次則揭獨往之高節,重去就之虛名,激競違貪,與世為異。或慮全後悔,事歸知殆;或道有不申,行吟山澤。咸皆用宇宙而成心,借風雲以為戒。求志達道,未或非然;含貞養素,文以藝業。不然,與樵者之在山何殊別哉?故樊英就徵,不稱李固之望;馮恢下節,見陋張華之語。期之塵外,庶以弘多。若今十餘子者,仕不求聞,退不譏俗,全身幽履,服道儒門,斯逸民之
 軌操,故綴為《高逸篇》云爾。



 褚伯玉,字元璩,吳郡錢唐人也。高祖含,始平太守。父襜,征虜參軍。伯玉少有隱操,寡嗜欲。年十八,父為之婚,婦入前門,伯玉從後門出。遂往剡,居瀑布山。性耐寒暑,時人比之王仲都。在山三十餘年,隔絕人物。王僧達為吳郡,苦禮致之,伯玉不得已,停郡信宿,裁交數言而退。寧朔將軍丘珍孫與僧達書曰:「聞褚先生出居貴館,此子滅景雲棲,不事王侯,抗高木食,有年載矣。自非折節好賢,何以致之?昔文舉棲冶城,安道入昌門,於茲而三焉。夫卻粒之士,餐霞之人,乃可暫致,不宜久羈。君當思遂其高步,成其羽化。望其還策之日,暫紆清塵,亦願助為譬說。」僧達答曰:「褚先生從白雲遊舊矣。古之逸民,或留慮兒女,或使華陰成市。而此子索然,唯朋松石,介於孤峰絕嶺者積數十載。近故要其來此,冀慰日夜。比談討芝桂,借
 訪荔蘿,若已窺煙液,臨滄洲矣。知君欲見之,輒當申譬。」



 宋孝建二年,散騎常侍樂詢行風俗,表薦伯玉,加徵聘本州議曹從事,不就。



 太祖即位,手詔吳、會二郡,以禮迎遣,又辭疾。上不欲違其志,敕於剡白石山立太平館居之。建元元年卒,年八十六。常居一樓上,仍葬樓所。孔稚珪從其受道法,為於館側立碑。



 明僧紹,字承烈,平原鬲人也。祖玩,州治中。父略,給事中。僧紹宋元嘉中再舉秀才,明經有儒術。永光中,鎮北府辟功曹,並不就。隱長廣郡嶗山,聚徒立學。淮北沒虜,乃南渡江。明帝泰始六年,徵通直郎,不就。



 升明中太祖為太傅,教辟僧紹及顧歡、臧榮緒以旍幣之禮,徵為記室參軍,不至。僧紹弟慶符為青州,僧紹乏糧食,隨慶符之鬱洲,住弇榆山,棲雲精舍,欣玩水石,竟不一入州城。建元元年冬,詔曰:「朕側席思士,載懷塵外。齊郡明僧紹標志高棲,耽情墳素,
 幽貞之操宜加賁飾。」徵為正員外郎,稱疾不就。其後與崔祖思書曰:「明居士標意可重,吾前旨竟未達邪?小涼欲有講事,卿可至彼,具述吾意,令與慶符俱歸。」又曰:「不食周粟而食周薇,古猶發議。在今寧得息談邪?



 聊以為笑。」



 慶符罷任,僧紹隨歸,住江乘攝山。太祖謂慶符曰:「卿兄高尚其事,亦堯之外臣。朕雖不相接,有時通夢。」遺僧紹竹根如意,筍籜冠。僧紹聞沙門釋僧遠風德,往候定林寺,太祖欲出寺見之。僧遠問僧紹曰:「天子若來,居士若為相對?」



 僧紹曰:「山藪之人,政當鑿壞以遁。若辭不獲命,便當依戴公故事耳。」永明元年,世祖敕召僧紹,稱疾不肯見。詔徵國子博士,不就,卒。子元琳,字仲璋,亦傳家業。



 僧紹長兄僧胤,能玄言。宋世為冀州刺史。弟僧暠,亦好學,宋孝武見之,迎頌其名,時人以為榮。泰始初,為青州刺史。



 慶符,建元初為黃門。



 僧胤子惠照,元徽中為太祖平南主簿,從
 拒桂陽,累至驃騎中兵,與荀伯玉對領直。建元元年為巴州刺史,綏懷蠻蜒,上許為益州,未遷,卒。



 顧歡,字景怡,吳郡鹽官人也。祖赳,晉隆安末,避亂徙居。歡年六七歲書甲子,有簡三篇,歡析計,遂知六甲。家貧,父使驅田中雀,歡作《黃雀賦》而歸,雀食過半,父怒,欲撻之,見賦乃止。鄉中有學舍,歡貧無以受業,於舍壁後倚聽,無遺忘者。八歲,誦《孝經》、《詩》、《論》。及長,篤志好學。母年老,躬耕誦書,夜則燃糠自照。同郡顧顗之臨縣,見而異之,遣諸子與游,及孫憲之,並受經句。歡年二十餘,更從豫章雷次宗諮玄儒諸義。母亡,水漿不入口六七日,廬于墓次,遂隱遁不仕。於剡天台山開館聚徒,受業者常近百人。歡早孤,每讀《詩》至「哀哀父母」,輒執書慟泣,學者由是廢《蓼莪篇》不復講。



 太祖輔政,悅歡風教,徵為揚州主簿,遣中使迎歡。及踐阼,乃至。歡稱「山谷臣顧歡」,上
 表曰:「臣聞舉網提綱,振裘持領,綱領既理,毛目自張。然則道德,綱也;物勢,目也。上理其綱,則萬機時序;下張其目,則庶官不曠。是以湯、武得勢師道則祚延,秦、項忽道任勢則身戮。夫天門開闔,自古有之,四氣相新,絺裘代進。今火澤易位,三靈改憲,天樹明德,對時育物,搜揚仄陋,野無伏言。



 是以窮谷愚夫,敢露偏管,謹刪撰《老氏》,獻《治綱》一卷。伏願稽古百王,斟酌時用,不以芻蕘棄言,不以人微廢道,則率土之賜也,微臣之幸也。幸賜一疏,則上下交泰,雖不求民而民悅,不祈天而天應。應天悅民,則皇基固矣。臣志盡幽深,無與榮勢,自足雲霞,不須祿養。陛下既遠見尋求,敢不盡言。言既盡矣,請從此退。」



 是時員外郎劉思效表陳讜言曰:「宋自大明以來,漸見凋弊,征賦有增於往,天府尤貧於昔。兼軍警屢興,傷夷不復,戍役殘丁,儲元半菽,小民嗷嗷,無樂生之色。貴勢之流,貨室之族,車
 服伎樂,爭相奢麗,亭池第宅,競趣高華,至于山澤之人不敢採飲其水草。貧富相輝,捐源尚末。陛下宜發明詔,吐德音,布惠澤,禁邪偽,薄賦斂,省徭役,絕奇麗之賂,塞鄭、衛之倡,變歷運之化,應質文之用,不亦大哉!又彭、汴有鴟梟之巢,青丘為狐兔之窟,虐害踰紀,殘暴日滋。鬼泣舊泉,人悲故壤,童孺視編發而慚生,耆老看左衽而恥沒。陛下宜仰答天人引領之望,下吊氓黎傾首之勤,授鉞衛、霍之將,遺策蕭、張之師,萬道俱前,窮山蕩谷。此即恆山不足指而傾,渤海不足飲而竭,豈徒殘寇塵滅而已哉!」



 上詔曰:「朕夙旦惟夤,思弘治道,佇夢巖濱,垂精管庫,旰食縈懷,其勤至矣。吳郡顧歡、散騎郎劉思效,或至自丘園,或越在冗位,並能獻書金門,薦辭鳳闕,辨章治體,有協朕心。今出其表,外可詳擇所宜,以時敷奏。歡近已加旍賁,思效可付選銓序,以顯讜言。」歡東歸,上賜麈尾、素琴。



 永明
 元年,詔征歡為太學博士,同郡顧黯為散騎郎。黯字長孺,有隱操,與歡俱不就徵。



 歡晚節服食,不與人通。每旦出戶,山鳥集其掌取食。事黃老道,解陰陽書,為數術多效驗。初元嘉末,出都寄住東府,忽題柱云:「三十年二月二十一日。」



 因東歸。後太初弒逆,果是此年月。自知將終,賦詩言志云:「精氣因天行,遊魂隨物化。」剋死日,卒於剡山,身體柔軟,時年六十四。還葬舊墓,木連理出墓側,縣令江山圖表狀。世祖詔歡諸子撰歡《文議》三十卷。



 佛道二家,立教既異,學者互相非毀。歡著《夷夏論》曰:夫辯是與非,宜據聖典。尋二教之源,故兩標經句。道經云:「老子入關之天竺維衛國,國王夫人名曰凈妙,老子因其晝寢,乘日精入凈妙口中,後年四月八日夜半時,剖左腋而生,墜地即行七步,於是佛道興焉。」此出《玄妙內篇》。佛經云:「釋迦成佛,有塵劫之數。」出《法華無量壽》。或「為國師道士,儒林之宗,」



 出《瑞應本起》。



 歡論之曰:五帝、三皇,莫不有師。國師道士,無過老、莊,儒林之宗,孰出周、孔?若孔、老非佛,誰則當之?然二經所說,如合符契。道則佛也,佛則道也。



 其聖則符,其跡則反。或和光以明近,或曜靈以示遠。道濟天下,故無方而不入;智周萬物,故無物而不為。其入不同,其為必異。各成其性,不易其事。是以端委搢紳,諸華之容;剪髮曠衣,群夷之服。擎跽磬折,侯甸之恭;狐蹲狗踞,荒流之肅。棺殯槨葬,中夏之制;火焚水沈,西戎之俗。全形守禮,繼善之教;毀貌易性,絕惡之學。豈伊同人,爰及異物。鳥王獸長,往往是佛,無窮世界,聖人代興。或昭五典,或布三乘。在鳥而鳥鳴,在獸而獸吼;教華而華言,化夷而夷語耳。雖舟車均於致遠,而有川陸之節;佛道齊乎達化,而有夷夏之別。若謂其致既均,其法可換者,而車可涉川,舟可行陸乎?今以中夏之性,效西戎之法,既不全同,又不全異。下
 棄妻孥,上廢宗祀。嗜欲之物,皆以禮伸;孝敬之典,獨以法屈。悖禮犯順,曾莫之覺。弱喪忘歸,孰識其舊?且理之可貴者,道也;事之可賤者,俗也。



 舍華效夷,義將安取?若以道邪,道固符合矣;若以俗邪,俗則大乖矣。



 屢見刻舷沙門,守株道士,交諍小大,互相彈射。或域道以為兩,或混俗以為一。是牽異以為同,破同以為異。則乖爭之由,淆亂之本也。尋聖道雖同,而法有左右。始乎無端,終乎無末。泥洹仙化,各是一術。佛號正真,道稱正一。一歸無死,真會無生。在名則反,在實則合。但無生之教賒,無死之化切:切法可以進謙弱,賒法可以退夸強。佛教文而博,道教質而精:精非粗人所信,博非精人所能。



 佛言華而引,道言實而抑:抑則明者獨進,引則昧者競前。佛經繁而顯,道經簡而幽:幽則妙門難見,顯則正路易遵。此二法之辨也。



 聖匠無心,方圓有體,器既殊用,教亦異施。佛是破惡之方,
 道是興善之術。



 興善則自然為高,破惡則勇猛為貴。佛跡光大,宜以化物;道跡密微,利用為己。



 優劣之分,大略在茲。



 夫蹲夷之儀,婁羅之辯,各出彼俗,自相聆解。猶蟲嚾鳥聒,何足述效。



 歡雖同二法,而意黨道教。宋司徒袁粲托為道人通公駁之,其略曰:白日停光,恆星隱照,誕降之應,事在老先,似非入關,方炳斯瑞。



 又老、莊、周、孔,有可存者,依日末光,憑釋遺法,盜牛竊善,反以成蠹。



 檢究源流,終異吾黨之為道耳。



 西域之記,佛經之說,俗以膝行為禮,不慕蹲坐為恭,道以三繞為虔,不尚踞傲為肅。豈專戎土,爰亦茲方。襄童謁帝,膝行而進;趙王見周,三環而止。今佛法在華,乘者常安;戒善行交,蹈者恆通。文王造周,大伯創吳,革化戎夷,不因舊俗。豈若舟車,理無代用。佛法垂化,或因或革。清信之士,容衣不改;息心之人,服貌必變。變本從道,不遵彼俗,教風自殊,無患其亂。



 孔、老、釋迦,其人
 或同,觀方設教,其道必異。孔、老治世為本,釋氏出世為宗。發軫既殊,其歸亦異。符合之唱,自由臆說。



 又仙化以變形為上,泥洹以陶神為先。變形者白首還緇,而未能無死;陶神者使塵惑日損,湛然常存。泥洹之道,無死之地,乖詭若此,何謂其同?



 歡答曰:案道經之作,著自西周,佛經之來,始乎東漢,年踰八百,代懸數十。若謂黃老雖久,而濫在釋前,是呂尚盜陳恆之齊,劉季竊王莽之漢也。



 經云,戎氣強獷,乃復略人頰車邪?又夷俗長跽,法與華異,翹左跂右,全是蹲踞。故周公禁之於前,仲尼戒之於後。又舟以濟川,車以征陸。佛起於戎,豈非戎俗素惡邪?道出於華,豈非華風本善邪?今華風既變,惡同戎狄,佛來破之,良有以矣。佛道實貴,故戒業可遵;戎俗實賤,故言貌可棄。今諸華士女,民族弗革,而露首偏踞,濫用夷禮。云於翦落之徒,全是胡人,國有舊風,法不可變。



 又若觀風流教,其道
 必異,佛非東華之道,道非西戎之法,魚鳥異淵,永不相關,安得老、釋二教,交行八表?今佛既東流,道亦西邁,故知世有精麤,教有文質。然則道教執本以領末,佛教救末以存本。請問所異,歸在何許?若以翦落為異,則胥靡翦落矣。若以立像為異,則俗巫立像矣。此非所歸,歸在常住。常住之象,常道孰異?



 神仙有死,權便之說。神仙是大化之總稱,非窮妙之至名。至名無名,其有名者二十七品,仙變成真,真變成神,或謂之聖,各有九品,品極則入空寂,無為無名。若服食茹芝,延壽萬億,壽盡則死,藥極則枯,此修考之士,非神仙之流也。



 明僧紹《正二教論》以為:「佛明其宗,老全其生。守生者蔽,明宗者通。今道家稱長生不死,名補天曹,大乖老、莊立言本理。」



 文惠太子、竟陵王子良並好釋法。吳興孟景翼為道士,太子召入玄圃園。眾僧大會,子良使景翼禮佛,景翼不肯。子良送《十地經》與之。
 景翼造《正一論》,大略曰:「《寶積》云『佛以一音廣說法』。老子云『聖人抱一以為天下式』。



 『一』之為妙,空玄絕於有境,神化贍於無窮,為萬物而無為,處一數而無數,莫之能名,強號為一。在佛曰實相,在道曰玄牝。道之大象,即佛之法身。以不守之守守法身,以不執之執執大象。但物有八萬四千行,說有八萬四千法。法乃至於無數,行亦逮於無央。等級隨緣,須導歸一。歸一曰回向,向正即無邪。邪觀既遣,億善日新。三五四六,隨用而施。獨立不改,絕學無憂。曠劫諸聖,共遵斯一。老、釋未始於嘗分,迷者分之而未合。億善遍修,修遍成聖,雖十號千稱,終不能盡。



 終不能盡,豈可思議。」



 司徒從事中郎張融作《門律》云:「道之與佛,逗極無二。吾見道士與道人戰儒墨,道人與道士辨是非。昔有鴻飛天首,積遠難亮。越人以為鳧,楚人以為乙,人自楚越,鴻常一耳。」以示太子僕周顒。顒難之曰:「虛無法性,其
 寂雖同,位寂之方,其旨則別。論所謂『逗極無二』者,為逗極於虛無,當無二於法性耶?足下所宗之本一物為鴻乙耳。驅馳佛道,無免二末。未知高鑒緣何識本,輕而宗之,其有旨乎?」往復文多不載。



 歡口不辯,善於著筆。著《三名論》,甚工,鐘會《四本》之流也。又注王弼《易》二《繫》,學者傳之。



 始興人盧度,亦有道術。少隨張永北征。永敗,虜追急,阻淮水不得過。度心誓曰:「若得免死,從今不復殺生。」須臾見兩楯流來,接之得過。後隱居西昌三顧山,鳥獸隨之。夜有鹿觸其壁,度曰:「汝壞我壁。」鹿應聲去。屋前有池養魚,皆名呼之,魚次第來,取食乃去。逆知死年月,與親友別。永明末,以壽終。



 初,永明三年,徵驃騎參軍顧惠胤為司徒主簿。惠胤,宋鎮軍將軍覬之弟子也。



 閑居養志,不應徵闢。



 臧榮緒,東莞莒人也。祖奉先,建陵令,父庸民,國子助教。榮緒幼孤,
 躬自灌園,以供祭祀。母喪後,乃著《嫡寢論》,掃灑堂宇,置筵席,朔望輒拜薦,甘珍未嘗先食。純篤好學,括東西晉為一書,紀、錄、志、傳百一十卷。隱居京口教授。南徐州辟西曹,舉秀才,不就。



 太祖為揚州,徵榮緒為主簿,不到。司徒褚淵少時嘗命駕尋之,建元中啟太祖曰:「榮緒,朱方隱者。昔臧質在宋,以國戚出牧彭岱,引為行佐,非其所好,謝疾求免。蓬廬守志,漏濕是安,灌蔬終老。與友關康之沈深典素,追古著書,撰《晉史》十帙,贊論雖無逸才,亦足彌綸一代。臣歲時往京口,早與之遇。近報其取書,始方送出,庶得備錄渠閣,採異甄善。」上答曰:「公所道臧榮緒者,吾甚志之。其有史翰,欲令入天祿,甚佳。」



 榮緒惇愛《五經》,謂人曰:「昔呂尚奉丹書,武王致齋降位,李、釋教誡,並有禮敬之儀。」因甄明至道,乃著《拜五經序論》。常以宣尼生庚子日,陳《五經》拜之。自號「被褐先生。」又以飲酒亂德,言常為誡。永
 明六年卒,年七十四。



 初,榮緒與關康之俱隱在京口,世號為「二隱」。康之字伯愉,河東人。世居丹徒。以墳籍為務。四十年不出門。不應州府辟。宋太始中,徵通直郎,不就。晚以母老家貧,求為嶺南小縣。性清約,獨處一室,稀與妻子相見。不通賓客。弟子以業傳受。尤善《左氏春秋》。太祖為領軍,素好此學,送《春秋五經》,康之手自點定,并得論《禮記》十餘條。上甚悅,寶愛之。遺詔以經本入玄宮。宋末卒。



 何求,字子有,廬江灊人也。祖尚之,宋司空。父鑠,宜都太守。求元嘉末為宋文帝挽郎,解褐著作郎,中軍衛軍行佐,太子舍人,平南參軍,撫軍主簿,太子洗馬,丹陽、吳郡丞。清退無嗜欲。又除征北參軍事,司徒主簿,太子中舍人。泰始中妻亡,還吳葬舊墓。除中書郎,不拜。仍住吳,居波若寺,足不踰戶,人莫見其面。明帝崩,出奔國哀,除為司空從事中郎,不就。乃除永嘉太守。求時寄住南澗寺,不肯詣
 臺,乞於寺拜受,見許。一夜忽乘小船逃歸吳,隱虎丘山,復除黃門郎,不就。永明四年,世祖以為太中大夫,又不就。七年卒,年五十六。



 初,求母王氏為父所害,求兄弟以此無宦情。



 求弟點,少不仕。宋世徵為太子洗馬,不就。隱居東離門卞望之墓側。性率到,鮮狎人物。建元中,褚淵、王儉為宰相,點謂人曰:「我作《齊書》已竟,贊云:『淵既世族,儉亦國華。不賴舅氏,遑恤外家。』欲儉候之,知不可見,乃止。永明元年,徵中書郎。豫章王命駕造門,點從後門逃去。竟陵王子良聞之,曰:「豫章王尚不屈,非吾所議。」遺點嵇叔夜酒杯、徐景山酒鎗以通意。點常自得,遇酒便醉,交遊宴樂不隔也。永元中,京師頻有軍寇,點嘗結裳為褲,與崔慧景共論佛義,其語默之迹如此。



 點弟胤,有儒術,亦懷隱遁之志。所居宅名為小山。隆昌中為中書令,以皇后從叔見親寵。明帝即位,胤賣園宅,將遂本志。建武四年為散騎常侍、
 巴陵王師,聞吳興太守謝朏致仕,慮後之,於是奉表不待報而去,隱會稽山。上大怒,令有司奏彈胤,然發優詔焉。永元二年,徵散騎常侍,太常卿。



 劉虯,字靈預,南陽涅陽人也。舊族,徙居江陵。虯少而抗節好學,須得祿便隱。宋泰始中,仕至晉平王驃騎記室,當陽令。罷官歸家,靜處斷穀,餌術及胡麻。



 建元初,豫章王為荊州,教辟虯為別駕,與同郡宗測、新野庾易並遣書禮請,虯等各修箋答而不應辟命。永明三年,刺史廬陵王子卿表虯及同郡宗測、宗尚之、庾易、劉昭五人,請加蒲車束帛之命。詔徵為通直郎,不就。



 竟陵王子良致書通意。虯答曰:「虯四節臥病,三時營灌,暢餘陰於山澤,託暮情於魚鳥,寧非唐、虞重恩,周、邵宏施?虯進不研機入玄,無洙泗稷館之辯;退不凝心出累,非冢間樹下之節。遠澤既灑,仁規先著。謹收樵牧之嫌,
 敬加軾蛙之義。」



 虯精信釋氏,衣粗布衣,禮佛長齋。注《法華經》,自講佛義。以江陵西沙洲去人遠,乃徙居之。建武二年,詔徵國子博士,不就。其冬虯病,正晝有白雲徘徊簷戶之內,又有香氣及磬聲,其日卒。年五十八。



 劉昭與虯同宗,州辟祭酒從事不就,隱居山中。



 庾易,字幼簡,新野新野人也。徙居屬江陵。祖玫,巴郡太守。父道驥,安西參軍。易志性恬隱,不交外物。建元元年,刺史豫章王辟為驃騎參軍,不就。臨川王映臨州,獨重易,上表薦之,餉麥百斛。易謂使人曰:「民樵採麋鹿之伍,終其解毛之衣;馳騁日月之車,得保自耕之祿。於大王之恩,亦已深矣。」辭不受。永明三年,詔徵太子舍人,不就。以文義自樂。安西長史袁彖欽其風,通書致遺。易以連理機竹翹書格報之。建武二年,詔復徵為司徒主簿,不就。卒。



 宗測,字敬微,南陽人,宋徵士炳孫也。世居江陵。測少靜退,不樂人
 間。歎曰:「家貧親老,不擇官而仕,先哲以為美談,余竊有惑。誠不能潛感地金,冥致江鯉,但當用天道,分地利。孰能食人厚祿,憂人重事乎?」



 州舉秀才,主簿,不就。驃騎豫章王徵為參軍,測答府召云:「何為謬傷海鳥,橫斤山木?」母喪,身負土植松柏。豫章王復遣書請之,闢為參軍。測答曰:「性同鱗羽,愛止山壑,眷戀松筠,輕迷人路。縱宕巖流,有若狂者,忽不知老至。而今鬢已白,豈容課虛責有,限魚慕鳥哉?」永明三年,詔徵太子舍人,不就。



 欲游名山,乃寫祖炳所畫《尚子平圖》於壁上。測長子官在京師,知父此旨,便求祿還為南郡丞,付以家事。刺史安陸王子敬、長史劉寅以下皆贈送之,測無所受。齎《老子》《莊子》二書自隨。子孫拜辭悲泣,測長嘯不視,遂往廬山,止祖炳舊宅。



 魚復侯子響為江州,厚遣贈遺。測曰:「少有狂疾,尋山採藥,遠來至此。量腹而進松術,度形而衣薜蘿,淡然已足,豈容當此
 橫施!」子響命駕造之,測避不見。後子響不告而來,奄至所住,測不得已,巾褐對之,竟不交言,子響不悅而退。



 尚書令王儉餉測蒲褥。頃之,測送弟喪還西,仍留舊宅永業寺,絕賓友,唯與同志庾易、劉虯、宗人尚之等往來講說。刺史隨王子隆至鎮,遣別駕宗哲致勞問,測笑曰:「貴賤理隔,何以及此。」竟不答。建武二年,徵為司徒主簿,不就。卒。



 測善畫,自圖阮籍遇蘇門於行障上,坐臥對之。又畫永業佛影臺,皆為妙作。



 頗好音律,善《易》《老》,續皇甫謐《高士傳》三卷。又嘗遊衡山七嶺,著衡山、廬山記。



 尚之字敬文,亦好山澤。與劉虯俱以驃騎記室不仕。宋末,刺史武陵王辟贊府,豫章王辟別駕,並不就。永明中,與劉虯同徵為通直郎,和帝中興初,又徵為諮議,並不就。壽終。



 杜京產,字景齊,吳郡錢唐人。杜子恭玄孫也。祖運,為劉毅衛
 軍參軍。父道鞠,州從事,善彈棋,世傳五斗米道,至京產及子棲。京產少恬靜,閉意榮宦。頗涉文義,專修黃老。會稽孔覬,清剛有峻節,一見而為款交。郡召主簿,州闢從事,稱疾去。除奉朝請,不就。與同郡顧歡同契,始寧東山開舍授學。建元中,武陵王曄為會稽,太祖遣儒士劉瓛入東為曄講說,京產請瓛至山舍講書,傾資供待,子棲躬自屣履,為瓛生徒下食,其禮賢如此。孔稚珪、周顒、謝抃並致書以通殷勤。



 永明十年,稚珪及光祿大夫陸澄、祠部尚書虞悰、太子右率沈約、司徒右長史張融表薦京產曰:「竊見吳郡杜京產,潔靜為心,謙虛成性,通和發於天挺,敏達表於自然。學遍玄、儒,博通史、子,流連文藝,沈吟道奧。泰始之朝,掛冠辭世,遁捨家業,隱于太平。葺宇窮巖,採芝幽澗,耦耕自足,薪歌有餘。確爾不群,淡然寡慾,麻衣藿食,二十餘載。雖古之志士,何以加之。謂宜釋巾幽谷,結
 組登朝,則巖谷含懽,薜蘿起抃矣。」不報。建武初,徵員外散騎侍郎,京產曰:「莊生持釣,豈為白璧所回。」辭疾不就。年六十四,永元元年卒。



 會稽孔道征,守志業不仕,京產與之友善。



 永明中,會稽鐘山有人姓蔡,不知名。山中養鼠數十頭,呼來即來,遣去便去。



 言語狂易,時謂之「謫仙」。不知所終。



 沈飀士,字雲禎,吳興武康人也。祖膺期,晉太中大夫。飀士少好學,家貧,織簾誦書,口手不息。宋元嘉末,文帝令尚書僕射何尚之抄撰《五經》,訪舉學士,縣以飀士應選。尚之謂子偃曰:「山藪故有奇士也。」少時,飀士稱疾歸鄉,更不與人物通。養孤兄子,義著鄉曲。或勸飀士仕,答曰:「魚縣獸檻,天下一契,聖人玄悟,所以每履吉先。吾誠未能景行坐忘,何為不希企日損。」乃作《玄散賦》以絕世。太守孔山士辟,不應。宗人徐州刺史曇慶、侍中懷文、左率勃來候之,飀士未嘗
 答也。隱居餘干吳差山,講經教授,從學者數十百人,各營屋宇,依止其側。



 飀士重陸機《連珠》,每為諸生講之。



 征北張永為吳興,請飀士入郡。飀士聞郡後堂有好山水,乃往停數月。永欲請為功曹,使人致意。飀士曰:「明府德履沖素,留心山谷,民是以被褐負杖,忘其疲病。必欲飾渾沌以蛾眉,冠越客於文冕,走雖不敏,請附高節,有蹈東海而死爾。」



 永乃止。



 昇明末,太守王奐上表薦之,詔徵為奉朝請,不就。永明六年,吏部郎沈淵、中書郎沈約又表薦飀士義行,曰:「吳興沈飀士,英風夙挺,峻節早樹,貞粹稟於天然,綜博生乎篤習。家世孤貧,藜藿不給,懷書而耕,白首無倦,挾琴採薪,行歌不輟。長兄早卒,孤姪數四,攝尪鞠稚,吞苦推甘。年踰七十,業行無改。元嘉以來,聘召仍疊。玉質踰潔,霜操日嚴。若使聞政王庭,服道槐掖,必能孚朝規於邊鄙,播聖澤於荒垂。」詔又徵為太學博士;建
 武二年,徵著作郎;永元二年,徵太子舍人;並不就。



 飀士負薪汲水,并日而食,守操終老。篤學不倦,遭火,燒書數千卷,飀士年過八十,耳目猶聰明,手以反故抄寫,燈下細書,復成二三千卷,滿數十篋,時人以為養身靜嘿之所致也。著《周易兩繫》《莊子內篇訓》,注《易經》、《禮記》、《春秋》、《尚書》、《論語》、《孝經》、《喪服》、《老子要略》數十卷。以楊王孫、皇甫謐深達生死,而終禮矯偽,乃自作終制。年八十六,卒。



 同郡沈儼之,字士恭,徐州刺史曇慶子,亦不仕。徵太子洗馬,永明元年,徵中書郎。三年,又詔徵前南郡國常侍沈摐為著作郎,建武二年。徵太子舍人,永元二年,徵通直郎。摐字處默,宋領軍寅之兄孫也。



 吳苞,字天蓋,濮陽鄄城人也。儒學,善《三禮》及《老》、《莊》。宋泰始中,過江聚徒教學。冠黃葛巾,竹麈尾,蔬食二十餘年。隆昌元年,詔曰:「處士濮陽吳苞,棲志穹谷,秉操貞固,沈情味古,白首彌厲。徵太學博士。」
 不就。始安王遙光、右衛江祏於蔣山南為立館,自劉瓛卒後,學者咸歸之。以壽終。



 魯國孔嗣之,字敬伯。宋世與太祖俱為中書舍人,並非所好,自廬陵郡去官,隱居鐘山,朝廷以為太中大夫。建武三年卒。



 徐伯珍,字文楚,東陽太末人也。祖父並郡掾史。伯珍少孤貧,書竹葉及地學書。山水暴出,漂溺宅舍,村鄰皆奔走,伯珍累床而止,讀書不輟。叔父璠之與顏延之友善,還祛蒙山立精舍講授,伯珍往從學,積十年,究尋經史,游學者多依之。



 太守瑯邪王曇生、吳郡張淹並加禮辟,伯珍應召便退,如此者凡十二焉。徵士沈儼造膝談論,申以素交。吳郡顧歡擿出《尚書》滯義,伯珍訓答甚有條理,儒者宗之。



 好釋氏、老莊,兼明道術。歲常旱,伯珍筮之,如期雨澍。舉動有禮,過曲木之下,趨而避之。早喪妻,晚不復重娶,自比曾參。宅南九里
 有高山,班固謂之九巖山,後漢龍丘萇隱處也。山多龍鬚檉柏,望之五采,世呼為婦人巖。二年,伯珍移居之。門前生梓樹,一年便合抱;館東石壁夜忽有赤光洞照,俄爾而滅;白雀一雙棲其戶牖;論者以為隱德之感焉。永明二年,刺史豫章王闢議曹從事,不就。家甚貧窶,兄弟四人,皆白首相對,時人呼為「四皓」。建武四年卒,年八十四。受業生凡千餘人。



 同郡樓幼瑜,亦儒學。著《禮捃遺》三十卷。官至給事中。



 又同郡樓惠明,有道術。居金華山,禽獸毒螫者皆避之。宋明帝聞之,敕出住華林園,除奉朝請,固乞不受,求東歸。永明三年,忽乘輕舟向臨安縣,眾不知所以。尋而唐宇之賊破郡。文惠太子呼出住蔣山,又求歸,見許。世祖敕為立館。



 史臣曰:顧歡論夷夏,優老而劣釋。佛法者,理寂乎萬古,迹兆乎中世,淵源浩博,無始無邊,宇宙之所不知,數量之所不盡,盛乎哉!真
 大士之立言也。探機扣寂,有感必應,以大苞小,無細不容。若乃儒家之教,仁義禮樂,仁愛義宜,禮順樂和而已;今則慈悲為本,常樂為宗,施舍惟機,低舉成敬。儒家之教,憲章祖述,引古證今,於學易悟;今樹以前因,報以後果,業行交酬,連璅相襲。陰陽之教,占氣步景,授民以時,知其利害;今則耳眼洞達,心智他通,身為奎井,豈俟甘石。法家之教,出自刑理,禁姦止邪,明用賞罰;今則十惡所墜,五及無間,刀樹劍出,焦湯猛火,造受自貽,罔或差貳。墨家之教,遵上儉薄,磨踵滅頂,且猶非吝;今則膚同斷瓠,目如井星,授子捐妻,在鷹庇鴿。從橫之教,所貴權謀,天口連環,歸乎適變;今則一音萬解,無待戶說,四辯三會,咸得吾師。雜家之教,兼有儒墨;今則五時所宣,於何不盡。農家之教,播植耕耘,善相五事,以藝九穀;今則鬱單粳稻,已異閻浮,生天果報,自然飲食。道家之教,執一虛無,得性亡
 情,凝神勿擾;今則波若無照,萬法皆空,豈有道之可名,寧餘一之可得。道俗對校,真假將讎。釋理奧藏,無往而不有也。能善用之,即真是俗。九流之設,用藉世教,刑名道墨,乖心異旨,儒者不學,無傷為儒;佛理玄曠,實智妙有,一物不知,不成圓聖。若夫神道應現之力,感會變化之奇,不可思議,難用言象。而諸張米道,符水先驗,相傳師法,祖自伯陽。世情去就,有此二學,僧尼道士,矛盾相非。非唯重道,兼亦殉利。詳尋兩教,理歸一極。但跡有左右,故教成先後。廣略為言,自生優劣。道本虛無,非由學至,絕聖棄智,已成有為。有為之無,終非道本。若使本末同無,曾何等級。佛則不然,具縛為種,轉暗成明,梯愚入聖。途雖遠而可踐,業雖曠而有期。勸慕之道,物我無隔。而局情淺智,鮮能勝受。世途揆度,因果二門。雞鳴為善,未必餘慶;膾肉東陵,曾無厄禍。身才高妙,鬱滯而靡達;器思庸鹵,富厚
 以終生。忠反見遺,詭乃獲用。觀此而論,近無罪福,而業有不定,著自經文,三報開宗,斯疑頓曉。史臣服膺釋氏,深信冥緣,謂斯道之莫貴也。



 贊曰:含貞抱樸,
 履道敦學。惟茲潛隱,棄鱗養角。



\end{pinyinscope}