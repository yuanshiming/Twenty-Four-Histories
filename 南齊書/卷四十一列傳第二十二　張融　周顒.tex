\article{卷四十一列傳第二十二 張融 周顒}

\begin{pinyinscope}

 張融,字思光,吳郡吳人也。祖禕,晉琅邪王國郎中令。父暢,宋會稽太守。



 融年弱冠,道士同郡陸修靜以白鷺羽麈尾扇遣融,曰:「此既異物,以奉異人。」



 宋孝武聞融有早譽,解褐為新安王北中郎參軍。孝武起新安寺,傣佐多儭錢帛,融獨儭百錢。帝曰:「融殊貧,當序以佳祿。」出為封溪令。從叔永出後渚送之,曰:「似聞朝旨,汝尋當還。」融曰:「不患不還,政恐還而復去。」廣越嶂險,獠賊執融,將殺食之,融神色不動,方作洛生詠,賊異之而不害也。浮海至交州,於海中作《海賦》曰:蓋言之用也,情矣形乎,使天形寅內敷,情敷外寅者,言之
 業也。吾遠職荒官,將海得地,行關入浪,宿渚經波,傅懷樹觀,長滿朝夕,東西無里,南北如天,反覆懸烏,表裏菟色。壯哉水之奇也,奇哉水之壯也。故古人以之頌其所見,吾問翰而賦之焉。當其濟興絕感,豈覺人在我外。木生之作,君自君矣。



 分渾始地,判氣初天,作成萬物,為山為川。總川振會,導海飛門。爾其海之狀也,之相也,則窮區沒渚,萬里藏岸,控會河、濟,朝總江、漢。回混浩潰,巔倒發濤,浮天振遠,灌日飛高。摐粗江撞則八紘摧聵,鼓怒則九紐折裂。



 檜於活長風以舉波,漷音郭天地而為勢。水蟄音蟄澤于及涾音沓洽音合,來往相躭粗合。汩於湥音突溢於渤渤,頠紆壯石成窟,西衝虞淵之曲,既東振湯谷之阿。若木於是乎倒覆,折扶桑而為渣在牙。濩灤音藥水門音門渾,涫於官水和於和碨於磊雍,渤非勃淬音卒淪音侖澊音尊,蘭淺壟緌於拱。湍轉則日月似驚,浪動而星河如覆。既烈太山與昆侖相壓而共潰,又盛雷車震漢破天以折轂。



 淃
 於員漣涴於卵瀨于嫩,輾轉縱橫。揚珠起玉,流鏡飛明。是其回堆曲浦,欹關弱渚之形勢也。沙嶼相接,洲島相連。東西南北,如滿于天。梁禽楚獸,胡木漢草之所生焉。長風動路,深雲暗道之所經焉。苕苕蒂蒂,窅窅翳翳。



 晨烏宿音秀於東隅,落河浪其西界。茫沆於剛汴河,汩於突磈於磊漫無官桓。旁踞委岳,橫竦危巒。重彰岌岌,攢嶺聚立。嵂呂兀窟音窟暐呂今嶔欽,架石相陰。昚曈徒罪陀陀,橫出旁入。嵬嵬支罪磊磊,若相追而下及。峰勢縱橫,岫形參錯。或如前而未進,乍非遷而已卻。



 天抗暉於東曲,日倒麗於西阿。嶺集雪以懷鏡,嚴照春而自華。



 江洚許江水伯水伯許百,漈子曷嚴拍芬百嶺。觸山禋石,汙涘於各水寒音寒況於朗。碨於磊泱於朗水隈水阿音阿,流柴磹五感反瑀五窟。頓浪低波,矰苦降交苦交硄苦江,折嶺挫峰,旂浪硠音郎掊,崩山相祇苦合萬里藹藹,極路天外。電戰雷奔,倒地相磕。獸門象逸,魚路鯨奔。水遽龍魄,陸振虎魂。卻瞻無後,向望何前。長尋
 高眺,唯水與天。若乃山橫蹴浪,風倒摧波。磊若驚山竭嶺以竦石,鬱若飛煙奔雲以振霞。連瑤光而交採,接玉繩以通華。



 爾乎夜滿深霧,晝密長雲,高河滅景,萬里無文。山門幽暖,岫戶葐惸。九天相掩,王地交氛。汪汪橫橫音皇,沆沆於剛浩浩音害。淬粗貴潰大人之表,泱於朗蕩君子之外。風沫相排,日閉雲開。浪散波合,岳起山隤。



 若乃漉沙構白,熬波出素。積雪中春,飛霜暑路。爾其奇名出錄,詭物無書。



 高岸乳鳥,橫門產魚。則何心羅音羅庸音容鰭音詣,飛音非魜音人果音果骨音滑。哄日吐霞,吞河漱月。氣開地震,聲動天發。



 噴灑噦於月噫於戒,流雨而揚云。喬壯脊,架岳而飛墳。牴音挺動崩五山之勢,瞷矣簡睔矣煥七曜之文。蟕雋瑁蛑,綺貝繡螺。玄珠互彩,綠紫相華。遊風秋瀨,泳景登春。伏鱗漬彩,昇魵洗文。



 若乃春代秋緒,歲去冬歸。柔風麗景,晴雲積暉。起龍塗於靈步,翔螭道之神飛。浮微雲之如夢,落輕雨之依依。
 觸巧塗而感云紺遠,抵欒木以激揚。浪相礡傍各而起千狀,波獨湧乎驚萬容。蘋藻留映,荷芰提陰。扶容曼彩,秀遠華深。明藕移玉,清蓮代金。眄芬芳於遙渚,汎灼爍於長潯。浮艫雜軸,遊舶交艘。



 帷軒帳席,方遠連高。入驚波而箭絕,振排天之雄飆。越湯谷以逐景,渡虞淵以追月。偏萬里而無時,浹天地於揮忽。雕隼飛而未半,鯤龍趠貪教而不逮。舟人未及復其喘,已周流宇宙之外矣。



 陰鳥陽禽,春毛秋羽。遠翅風遊,高翮雲舉。翔歸棲去,連陰日路。瀾漲波渚,陶玄浴素。長紘四斷,平表九絕。雉翥成霞,鴻飛起雪。合聲鳴侶,並翰翻群。飛關溢繡,流浦照文。



 爾夫人微亮氣,小白如淋。涼空澄遠,層漢無陰。照天容於鮷渚,鏡河色於魦潯。括蓋餘以進廣,浸夏洲以洞深。形每驚而義維靜,跡有事而道無心。於是乎山海藏陰,雲塵入岫。天英偏華,日色盈秀。則若士神中,琴高道外。袖輕羽以衣風,逸玄
 裾於雲帶。筵秋月於源潮,帳春霞於秀瀨。曬蓬萊之靈岫,望方壺之妙闕。樹遏日以飛柯,嶺回峰以蹴月。空居無俗,素館何塵。穀門風道,林路雲真。



 若乃幽崖邑於夾曌倉夾,隈隩之窮,駿波虎浪之氣,激勢之所不攻。



 有卉有木,為灌為叢。路糅網雜,結葉相籠。通雲交拂,連韻共風。蕩洲礉去角岸,而千里若崩;衝崖沃島,其萬國如戰。振駿氣以擺雷,飛雄光以倒電。



 若夫增雲不氣,流風斂聲。瀾文復動,波色還驚。明月何遠,沙裏分星。至其積珍全遠,架寶諭深。瓊池玉壑,珠岫岑。合日開夜,舒月解陰。珊瑚開繢,琉璃竦華。丹文鏡色,雜照冰霞。洪洪潰潰,浴乾日月。淹漢星墟,滲河天界。風何本而自生,雲無從而空滅。籠麗色以拂煙,鏡懸暉以照雪。



 爾乃方員去我,混然落情。氣暄而濁,化靜自清。心無終故不滯,志不敗而無成。既覆舟而載舟,固以死而以生。弘芻狗於人獸,導至本以充形。雖萬物
 之日用,諒何緯其何經。道湛天初,機茂形外。亡有所以而有,非膠有於生末。亡無所以而無,信無心以入太。不動動是使山岳相崩,不聲聲故能天地交泰。行藏虛於用舍,應感亮於圓會。仁者見之謂之仁,達者見之謂之達。咶者幾於上善,吾信哉其為大矣。



 融文辭詭激,獨與眾異。後還京師,以示鎮軍將軍顧覬之,覬之曰:「卿此賦實超玄虛,但恨不道鹽耳。」融即求筆注之曰:「漉沙構白,熬波出素。積雪中春,飛霜暑路。」此四句,後所足也。



 覬之與融兄有恩好,覬之卒,融身負墳土。在南與交止太守卞展有舊,展於嶺南為人所殺,融挺身奔赴。



 舉秀才,對策中第,為尚書殿中郎,不就,為儀曹郎。泰始五年,明帝取荊、郢、湘、雍四州射手,叛者斬亡身及家長者,家口沒奚官。元徽初,郢州射手有叛者,融議家人家長罪所不及,亡身刑五年。



 尋請假奔叔父喪,道中罰幹錢敬道鞭杖五十,寄繫延
 陵獄。大明五年制,二品清官行僮幹杖,不得出十。為左丞孫緬所奏,免官。尋復位,攝祠、倉部二曹。領軍劉勔力戰死,祠曹議「上應哭勔不」,融議「宜哭」。於是始舉哀。倉曹又以「正月俗人所忌,太倉為可開不」,融議「不宜拘束小忌」。尋兼掌正廚。融見宰殺,回車徑去,自表解職。為安成王撫軍倉曹參軍,轉南陽王友。



 融父暢先為丞相長史,義宣事難,暢為王玄謨所錄,將殺之。玄謨子瞻為南陽王前軍長史,融啟求去官,不許。



 融家貧願祿,初與從叔征北將軍永書曰:「融昔稱幼學,早訓家風,雖則不敏,率以成性。布衣葦席,弱年所安;簞食瓢飲,不覺不樂。但世業清貧,民生多待,榛栗棗修,女贄既長,束帛禽鳥,男禮已大。勉身就官,十年七仕,不欲代耕,何至此事。昔求三吳一丞,雖屬舛錯;今聞南康缺守,願得為之。融不知階級,階級亦可不知,融政以求丞不得,所以求郡,求郡不得,亦可復求丞。」



 又
 與吏部尚書王僧虔書曰:「融,天地之逸民也。進不辨貴,退不知賤,兀然造化,忽如草木。實以家貧累積,孤寡傷心,八侄俱孤,二弟頗弱,撫之而感,古人以悲。豈能山海陋祿,申融情累。阮籍愛東平土風,融亦欣晉平閑外。」時議以融非治民才,竟不果。



 辟太祖太傅掾,歷驃騎豫章王司空諮議參軍,遷中書郎,非所好,乞為中散大夫,不許。融風止詭越,坐常危膝,行則曳步,翹身仰首,意制甚多。隨例同行,常稽遲不進。太祖素奇愛融,為太尉時,時與融款接,見融常笑曰:「此人不可無一,不可有二。」即位後,手詔賜融衣曰:「見卿衣服粗故,誠乃素懷有本;交爾藍縷,亦虧朝望。今送一通故衣,意謂雖故,乃勝新也。是吾所著,已令裁減稱卿之體。并履一量。」



 融與吏部尚書何戢善,往詣戢,誤通尚書劉澄。融下車入門,乃曰:「非是。」



 至戶外,望澄,又曰:「非是。」既造席,視澄曰:「都自非是。」乃去。其為異如此。



 又
 為長沙王鎮軍、竟陵王征北諮議,並領記室,司徒從事中郎。



 永明二年,總明觀講,敕朝臣集聽。融扶入就榻,私索酒飲之,難問既畢,乃長歎曰:「嗚呼!仲尼獨何人哉!」為御史中丞到捴所奏,免官,尋復。



 融形貌短醜,精神清澈。王敬則見融革帶垂寬,殆將至骼,謂之曰:「革帶太急。」融曰:「既非步吏,急帶何為?」



 融假東出,世祖問融住在何處?融答曰:「臣陸處無屋,舟居非水。」後日上以問融從兄緒,緒曰:「融近東出,未有居止,權牽小船於岸上住。」上大笑。虜中聞融名,上使融接北使李道固,就席,道固顧之而言曰:「張融是宋彭城長史張暢子不?」融嚬蹙久之,曰:「先君不幸,名達六夷。」豫章王大會賓僚,融食炙始行畢,行炙人便去,融欲求鹽蒜,口終不言,方搖食指,半日乃息。出入朝廷皆拭目驚觀之。八年,朝臣賀眾瑞公事,融扶入拜起,復為有司所奏,見原。遷司徒右長史。



 竟陵張欣時為諸暨令,坐罪當
 死。欣時父興世宋世討南譙王義宣,官軍欲殺融父暢,興世以袍覆暢而坐之,以此得免。興世卒,融著高履負土成墳。至是融啟竟陵王子良,乞代欣時死。子良答曰:「此乃是長史美事,恐朝有常典,不得如長史所懷。」遷黃門郎,太子中庶子,司徒左長史。



 融有孝義,忌月三旬不聽樂,事嫂甚謹。宋丞相義宣起事,父暢以不同將見殺,司馬竺超民諫免之。暢臨終謂諸子曰:「昔丞相事難,吾緣竺司馬得活,爾等必報其子弟。」後超民孫微冬月遭母喪,居貧,融往弔之,悉脫衣以為賻,披牛被而反。



 常以兄事微。豫章王嶷、竟陵王子良薨,自以身經佐吏,哭輒盡慟。



 建武四年,病卒。年五十四。遺令建白旌無旒,不設祭,令人捉麈尾登屋復魂,曰:「吾生平所善,自當凌雲一笑。」三千買棺,無製新衾。左手執《孝經》、《老子》,右手執小品《法華經》。妾二人,哀事畢,各遣還家。又曰:「以吾平生之風調,何至使婦人行哭
 失聲,不須暫停閨閣。」



 融玄義無師法,而神解過人,白黑談論,鮮能抗拒。永明中,遇疾,為《門律自序》曰:「吾文章之體,多為世人所驚,汝可師耳以心,不可使耳為心師也。夫文豈有常體,但以有體為常,政當使常有其體。丈夫當刪《詩》《書》,制禮樂,何至因循寄人籬下!且中代之文,道體闕變,尺寸相資,彌縫舊物。吾之文章,體亦何異,何嘗顛溫涼而錯寒暑,綜哀樂而橫歌哭哉?政以屬辭多出,比事不羈,不阡不陌,非途非路耳。然其傳音振逸,鳴節竦韻,或當未極,亦已極其所矣。汝若復別得體者,吾不拘也。吾義亦如文,造次乘我,顛沛非物。吾無師無友,不文不句,頗有孤神獨逸耳。義之為用,將使性入清波,塵洗猶沐。無得釣聲同利,舉價如高,俾是道場,險成軍路。吾昔嗜僧言,多肆法辯,此盡游乎言笑,而汝等無幸。」



 又云:「人生之口,正可論道說義,惟飲與食。此外如樹網焉。吾每以不爾為
 恨,爾曹當振綱也。」



 臨卒,又戒其子曰:「手澤存焉,父書不讀!況父音情,婉在其韻。吾意不然,別遺爾音。吾文體英絕,變而屢奇,既不能遠至漢魏,故無取嗟晉宋。豈吾天挺,蓋不隤家聲。汝若不看,父祖之意欲汝見也。可號哭而看之。」融自名集為《玉海》。



 司徒褚淵問《玉海》名,融答:「玉以比德,海崇上善。」文集數十卷行於世。



 張氏知名,前有敷、演、鏡、暢,後有充、融、卷、稷。



 周顒,字彥倫,汝南安城人。晉左光祿大夫顗七世孫也。祖虎頭,員外常侍。



 父恂,歸鄉相。



 顒少為族祖朗所知。解褐海陵國侍郎。益州刺史蕭惠開賞異顒,攜入蜀,為厲鋒將軍,帶肥鄉、成都二縣令。轉惠開輔國府參軍,將軍、令如故。仍為府主簿。



 常謂惠開性太險峻,每致諫,惠開不悅,答顒曰:「天險地險,王公設險,但問用險何如耳。」隨惠開還都。



 宋明帝頗好言理,以顒有辭義,引入殿內,親近宿直。
 帝所為慘毒之事,顒不敢顯諫,輒誦經中因緣罪福事,帝亦為之小止。轉安成王撫軍行參軍。元徽初,出為剡令,有恩惠,百姓思之。還歷邵陵王南中郎三府參軍。



 太祖輔政,引接顒。顒善尺牘,沈攸之送絕交書,太祖口授令顒裁答。轉齊臺殿中郎。



 建元初,為長沙王參軍,後軍參軍,山陰令。縣舊訂滂民,以供雜使。顒言之於太守聞喜公子良曰:「竊見滂民之困,困實極矣。役命有常,只應轉竭,蹙迫驅催,莫安其所。險者或竄避山湖,困者自經溝瀆爾。亦有摧臂斲手,茍自殘落,販傭貼子,權赴急難。每至滂使發動,遵赴常促,輒有柤杖被錄,稽顙階垂,泣涕告哀,不知所振。下官未嘗不臨食罷箸,當書偃筆,為之久之,愴不能已。交事不濟,不得不就加捶罰,見此辛酸,時不可過。山陰邦治,事倍餘城;然略聞諸縣,亦處處皆躓。唯上虞以百戶一滂,大為優足,過此列城,不無凋罄。宜應有以普
 救倒懸,設流開便,則轉患為功,得之何遠。」還為文惠太子中軍錄事參軍,隨府轉征北。



 文惠在東宮,顒還正員郎,始興王前軍諮議。直侍殿省,復見賞遇。



 顒音辭辯麗,出言不窮,宮商朱紫,發口成句。泛涉百家,長於佛理。著《三宗論》。立空假名,立不空假名。設不空假名難空假名,設空假名難不空假名。假名空難二宗,又立假名空。西涼州智林道人遺顒書曰:「此義旨趣似非始開,妙聲中絕六七十載。貧道年二十時,便得此義,竊每歡喜,無與共之。年少見長安耆老,多云關中高勝乃舊有此義,當法集盛時,能深得斯趣者,本無多人。過江東略是無一。貧道捉麈尾來四十餘年,東西講說,謬重一時,餘義頗見宗錄,唯有此塗白黑無一人得者,為之發病。非意此音猥來入耳,始是真實行道第一功德。」其論見重如此。



 顒於鍾山西立隱舍,休沐則歸之。轉太子僕,兼著作,撰起居注。遷中書
 郎,兼著作如故。常游侍東宮。少從外氏車騎將軍臧質家得衛恆散隸書法,學之甚工。



 文惠太子使顒書玄圃茅齋壁,國子祭酒何胤以倒薤書求就顒換之,顒笑而答曰:「天下有道,丘不與易也。」



 每賓友會同,顒虛席晤語,辭韻如流,聽者忘倦。兼善《老》、《易》,與張融相遇,輒以玄言相滯,彌日不解。清貧寡欲,終日長蔬食。雖有妻子,獨處山舍。



 衛將軍王儉謂顒曰:「卿山中何所食?」顒曰:「赤米白鹽,綠葵紫蓼。」文惠太子問顒:「菜食何味最勝?」顒曰:「春初早韭,秋末晚菘。」時何胤亦精信佛法,無妻妾。太子又問顒:「卿精進何如何胤?」顒曰:「三塗八難,共所未免。然各有其累。」太子曰:「所累伊何?」對曰:「周妻何肉。」其言辭應變,皆如此也。



 轉國子博士,兼著作如故。太學諸生慕其風,爭事華辯。後何胤言斷食生,猶欲食白魚、旦脯、糖蟹,以為非見生物。疑食蚶蠣,使學生議之。學生鍾岏曰:「旦之就脯,驟於屈伸;蟹
 之將糖,躁擾彌甚。仁人用意,深懷如怛。至於車螯蚶蠣,眉目內闕,慚渾沌之奇,礦殼外緘,非金人之慎。不悴不榮,曾草木之不若;無馨無臭,與瓦礫其何算。故宜長充庖廚,永為口實。」竟陵王子良見絜議,大怒。



 胤兄點,亦遁節清信。顒與書,勸令菜食。曰:「丈人之所以未極遐蹈,或在不近全菜邪?脫灑離析之討,鼎俎網罟之興,載之簡策,其來實遠,誰敢干議?觀聖人之設膳脩,仍復為之品節,蓋以茹毛飲血,與生民共始,縱而勿裁,將無厓畔。



 善為士者,豈不以恕己為懷?是以各靜封疆,罔相陵軼。況乃變之大者,莫過死生;生之所重,無踰性命。性命之於彼極切,滋味之在我可賒,而終身朝晡,資之以永歲,彼就冤殘,莫能自列,我業久長,吁哉可畏。且區區微卵,脆薄易矜,歂彼弱麑,顧步宜愍。觀其飲喙飛行,人應憐悼,況可心心撲褫,加復恣忍吞嚼。至乃野牧盛群,閉豢重圈,量肉揣毛,以俟枝
 剝,如土委地,僉謂常理,可為愴息,事豈一塗。若云三世理誣,則幸矣良快。如使此道果然,而受形未息,則一往一來,一生一死,輪回是常事。雜報如家,人天如客,遇客日鮮,在家日多,吾儕信業,未足長免,則傷心之慘,行亦自及。丈人於血氣之類,雖無身踐,至於晨鳧夜鯉,不能不取備屠門。財貝之一經盜手,猶為廉士所棄;生性之一啟鸞刀,寧復慈心所忍!



 騶虞雖饑,非自死之草不食,聞其風豈不使人多愧者!眾生之稟此形質,以畜肌跂,皆由其積壅癡迷,沈流莫反,報受穢濁,歷苦酸長,此甘與肥,皆無明之報聚也。



 何至復引此滋腴,自污腸胃。丈人得此有素,聊復寸言發起耳。」



 顒卒官時,會王儉講《孝經》未畢,舉曇濟自代,學者榮之。官為給事中。



 史臣曰:弘毅存容,至仁表貌,汲黯剛戇,崔琰聲姿,然後能不憚雄桀,亟成譏犯。張融標心托旨,全等塵外,吐納風雲,不論人物,而事
 君會友,敦義納忠,誕不越檢,常在名教。若夫奇偉之稱,則虞翻、陸績不得獨擅於前也。



 贊曰:思光矯矯,萬里千仞。升同應諧,黜同解擯。務在連衡,不謀銷印。彥倫辭辯,苦節清韻。白馬橫擒,雲梯獨振。



\end{pinyinscope}