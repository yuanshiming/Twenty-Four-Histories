\article{卷二十四列傳第五 柳世隆 張瑰}

\begin{pinyinscope}

 柳世隆,字彥緒,河東解人也。祖憑,馮翊太守。父叔宗,早卒。世隆少有風器。伯父元景,宋大明中為尚書令,獨賞愛之,異於諸子。言於孝武帝,得召見。



 帝曰:「三公一人,是將來事也。」海陵王休茂為雍州,辟世隆為迎主簿。除西陽王撫軍法曹行參軍,出為虎威將軍、上庸太守。帝謂元景曰:「卿昔以虎威之號為隨郡,今復以授世隆,使卿門世不絕公也。」元景為景和所殺,世隆以在遠得免。



 泰始初,諸州反叛,世隆以門禍獲申,事由明帝,乃據郡起兵,遣使應朝廷。



 弘農人劉僧驎亦聚眾應之。收合萬人,奄至襄陽萬山,為孔道存所
 破,眾皆奔散,僅以身免,逃藏民間,事平乃出。還為尚書儀曹郎,明帝嘉其義心,發詔擢為太子洗馬,出為寧遠將軍、巴西梓潼太守。還為越騎校尉,轉建平王鎮北諮議參軍,領南泰山太守,轉司馬、東海太守,入為通直散騎常侍。尋為晉熙王安西司馬,加寧朔將軍。時世祖為長史,與世隆相遇甚歡。



 太祖之謀渡廣陵也,令世祖率眾下,同會京邑,世隆與長流蕭景先等戒嚴待期,事不行。是時朝廷疑憚沈攸之,密為之防,府州器械,皆有素蓄。世祖將下都,劉懷珍白太祖曰:「夏口是兵沖要地,宜得其人。」太祖納之,與世祖書曰:「汝既入朝,當須文武兼資人與汝意合者,委以後事,世隆其人也。」世祖舉世隆自代。



 轉為武陵王前軍長史、江夏內史、行郢州事。



 升明元年冬,攸之反,遣輔國將軍中兵參軍孫同、寧朔將軍中兵參軍武寶、龍驤將軍騎兵參軍朱君拔、寧朔將軍沈惠真、龍驤將
 軍騎兵參軍王道起三萬人為前驅,又遣司馬冠軍劉攘兵領寧朔將軍外兵參軍公孫方平、龍驤將軍騎兵參軍朱靈真、沈僧敬、龍驤將軍高茂二萬人次之,又遣輔國將軍王靈秀、丁珍東、寧朔將軍中兵參軍王彌之、寧朔將軍外兵參軍楊景穆二千匹騎分兵出夏口,據魯山。攸之乘輕舸從數百人先大軍下住白螺洲,坐胡床以望其軍,有自驕色。既至郢,以郢城弱小不足攻,遣人告世隆曰:「被太后令,當暫還都。卿既相與奉國,想得此意。」世隆使人答曰:「東下之師,久承聲問。郢城小鎮,自守而已。」攸之將去,世隆遣軍於西渚挑戰,攸之果怒,令諸軍登岸燒郭邑,築長圍攻道,顧謂人曰:「以此攻城,何城不剋!」晝夜攻戰,世隆隨宜拒應,眾皆披卻。



 世祖初下,與世隆別,曰:「攸之一旦為變,焚夏口舟艦沿流而東,則坐守空城,不可制也。雖留攻城,不可卒拔。卿為其內,我為其外,乃
 無憂耳。」至是,世祖遣軍主桓敬、陳胤叔、茍元賓等八軍據西塞,令堅壁以待賊疲。慮世隆危急,遣腹心胡元直潛使入郢城通援軍消息,內外並喜。



 尚書符曰:沈攸之出自壟畝,寂寥累世,故司空沈公以從父宗蔭,愛之若子,羽翼吹噓,得升官次。景和昏悖,猜畏柱臣,而攸之凶忍,趣利樂禍,請銜詔旨,躬行反噬。



 又攸之與譚金、童泰壹等暴寵狂朝,並為心膂,同功共體,世號「三侯」,當時親暱,情過管、鮑。仰遭革運,兇黨懼戮,攸之反善圖全,用得自免。既殺從父,又虐良朋,雖呂布販君,酈寄賣友,方之斯人,未足為酷。



 泰始開闢,網漏吞舟,略其兇險,取其搏噬,故階亂獲全,因禍興福。



 攸之稟性空淺,躁而無謀。濃湖土崩,本非己力;彭城、下邳,望旗宵遁,再棄王師,久應肆法。值先帝宥其回溪之恥,冀有封崤之捷,故得幸會推遷,頻煩顯授,內端戎禁,外綏萬里。聖去鼎湖,遠頒顧命,托寄崇深,義感
 金石。而攸之始奉國諱,喜形於顏,普天同哀,己以為慶。



 累登蕃岳,自郢遷荊。晉熙王以皇弟代鎮,地尊望重,攸之斷割候迎,肆意陵略。料擇士馬,簡算器械,權撥精銳,並取自隨。郢城所留,十不遺一。專恣鹵奪,罔顧國典。



 踐荊已來,恆用奸數,既懷異志,興造無端。乃蹙迫群蠻,騷擾山谷,揚聲討伐,盡戶上丁;蟻聚郭邑,伺國衰盛,從來積年,求不解甲。遂四野百縣,路無男人,耕田載租,皆驅女弱。自古酷虐,未聞於此。



 昔歲桂陽內奰,宗廟阽危。攸之任官上流,兵彊地廣,勤王之舉,實宜悉行;裁遣羸弱,不滿三千,至郢州稟受節度,欲令判否之日,委罪晉熙。



 招誘劍客,羈絆行侶,竄叛入境,輒加擁護,逋亡出界,必遣窮追。



 視吏若讎,遇民如草,峻太半之賦,暴參夷之刑,鞭棰國士,全用虜法。一人逃亡,闔宗捕逮。皇朝赦令,初不遵奉,曠蕩之澤,長隔彼州,人懷怨望,十室而九。



 今乃舉兵內侮,姦回
 外熾,斯實惡熟罪成之辰,決癰潰疽之日。幕府過荷朝寄,義百常憤,董御元戎,龔行天罰。



 今遣新除使持節督郢州司州之義陽諸軍事平西將軍郢州刺史聞喜縣開國侯黃回、員外散騎常侍輔國將軍驍騎將軍重安縣開國子軍主王敬則、屯騎校尉長壽縣開國男軍主王宜與、屯騎校尉陳承叔、右軍將軍葛陽縣開國男彭文之、驃騎行參軍振武將軍邵宰,精甲二萬,沖其首旆。又遣散騎常侍游擊將軍湘南縣開國男呂安國、持節寧朔將軍越州刺史孫曇瓘、屯騎校尉寧朔將軍崔慧景、寧朔將軍左軍將軍新亭侯任候伯、龍驤將軍虎賁中郎將尹略、屯騎校尉南城令曹虎頭、輔國將軍驍騎將軍蕭順之、新除寧朔將軍游擊將軍下邳縣開國子垣崇祖等,舳艫二萬,駱驛繼邁。又遣屯騎校尉茍元賓、撫軍參軍郭文考、撫軍中兵參軍程隱俊、奉朝請諸襲光等,輕艓一萬,截其
 津要。驍騎將軍周盤龍、後將軍成買、輔國將軍王敕勤、屯騎校尉王洪範等,鐵騎五千,步道繼進,先據陸路,斷其走伏。持節、督雍梁二州郢州之竟陵司州之隨郡諸軍事、征虜將軍、寧蠻校尉、雍州刺史、襄陽縣開國侯、新除鎮軍將軍張敬兒,志節慷慨,卷甲樊、鄧,水步俱馳,破其巢窟。持節、督司州諸軍事、征虜將軍、司州刺史、領義陽太守、范陽縣侯姚道和,義烈梗概,投袂方隅,風馳電掩,襲其輜重。萬里建旍,四方飛旆,莫不總率眾師,雲翔雷動。人神同憤,遠邇並心。



 今皇上聖明,將相仁愛,約法三章,寬刑緩賦,年登歲阜,家給人足,上有惠民之澤,下無樂亂之心。攸之不識天時,妄圖大逆,舉無名之師,驅讎怨之眾,是以朝野審其易取,含識判其成禽。



 彼土士民,罹毒日久,今復相逼迫,投赴鋒刃。交戰之日,蘭艾難分,去就在機,望思先曉。無使一人迷疑,而九族就禍也。弘宥之典,有如皎
 日。



 郢城既不可攻,而平西將軍黃回軍至西陽,乘三層艦,作羌胡伎,溯流而進。



 攸之素失人情,本逼以威力,初發江陵,已有叛者,至是稍多。攸之日夕乘馬歷營撫慰,而去者不息。攸之大怒,召諸軍主曰:「我被太后令,建義下都,大事若剋,白紗帽共著耳;如其不振,朝廷自誅我百口,不關餘人。比軍人叛散,皆卿等不以為意。我亦不能問叛身,自今軍中有叛者,軍主任其罪。」於是一人叛,遣十人追,並去不反。莫敢發覺,咸有異計。劉攘兵射書與世隆許降,世隆開門納之。攘兵燒營而去,火起乃覺。攸之怒,銜鬚咀之。收攘兵兄子天賜、女婿張平慮斬之。軍旅大散。攸之渡魯山岸,猶有數十匹騎自隨。宣令軍中曰:「荊州城中大有錢,可相與還取,以為資糧。」郢城未有追軍,而散軍畏蠻抄,更相聚結,可二萬人,隨攸之,將至江陵,乃散。世隆乃遣軍副劉僧驎道追之。



 攸之已死,徵為侍中。仍遷
 尚書右僕射,封貞陽縣侯,邑二千戶。出為左將軍、吳郡太守,加秩中二千石。丁母憂。太祖踐阼,起為使持節、都督南豫司二州諸軍事、平南將軍、南豫州刺史,進爵為公。上手詔與司徒褚淵曰:「向見世隆毀瘠過甚,殆欲不可復識,非直使人惻然,實亦世珍國寶也。」淵答曰:「世隆至性純深,哀過乎禮。事陛下在危盡忠,喪親居憂,杖而後起,立人之本,二理同極。加榮增寵,足以厲俗敦風。」



 建元二年,進號安南將軍。是時虜寇壽陽,上敕世隆曰:「歷陽城大,恐不可卒治,正宜斷隔之,深為保固。處分百姓,若不將家守城,單身亦難可委信也。」



 尋又敕曰:「吾更歷陽外城,若有賊至,即勒百姓守之,故應勝割棄也。」垣崇祖既破虜,上欲罷並二豫,敕世隆曰:「比思江西蕭索,二豫兩辦為難。議者多雲省一足一於事為便。吾謂非乃乖謬。卿以為云何?可具以聞。」尋授後將軍、尚書右僕射,不拜。



 世隆性愛涉獵,
 啟太祖借秘閣書,上給二千卷。



 三年,出為使持節、督南兗兗徐青冀五州軍事、安北將軍、南兗州刺史。江北畏虜寇,搔動不安。上敕世隆曰:「比有北信,賊猶治兵在彭城,年已垂盡,或當未必送死。然豺狼不可以理推,為備或不可懈。彼郭既無關要,用宜開除,使去金城三十丈政佳耳。發民治之,無嫌。若作三千人食者,已有幾米?可指牒付信還。



 民間若有丁多而細口少者,悉令戍,非疑也。」又敕曰:「昨夜得北使啟,鐘離間賊已渡淮,既審送死,便當制加剿撲。卿好參候之,有急令諸小戍還鎮,不可賊至不覺也。賊既過淮,不容邇退散,要應有處送死者,定攻壽陽,吾當遣援軍也。」



 又遣軍助世隆,并給軍糧。虜退,上欲土斷江北,又敕世隆曰:「呂安國近在西,土斷郢、司二境上雜民,大佳,民始無驚恐。近又令垣豫州斷其州內,商得崇祖啟事,已行竟,近無雲云,殊稱前代舊意。卿視兗部中可
 行此事不?若無所擾,春便就手也。」其見親委如此。



 世祖即位,加散騎常侍。世隆善卜,別龜甲,價至一萬。永明建號,世隆題州齋壁曰「永明十一年」,謂典簽李當曰:「我不見也。」入為侍中、護軍將軍,遷尚書右僕射,領太子右率,雍州大中正,不拜,改授散騎常侍,尚書左僕射,中正如故。湘州蠻動,遣世隆以本官總督伐蠻眾軍,仍為使持節、都督湘州諸軍事、鎮南將軍、湘州刺史,常侍如故。世隆至鎮,以方略討平之。在州立邸治生,為中丞庾杲之所奏,詔原不問。復入為尚書左僕射,領衛尉,不拜。仍轉尚書令。



 世隆少立功名,晚專以談義自業。善彈琴,世稱柳公雙璅,為士品第一。常自云馬槊第一,清談第二,彈琴第三。在朝不干世務,垂簾鼓琴,風韻清遠,甚獲世譽。以疾遜位,改授侍中,衛將軍,不拜,轉左光祿大夫,侍中如故。



 九年,卒,時年五十。詔給東園秘器,朝服一具,衣一襲,錢一十萬,布
 三百匹,蠟三百斤。又詔曰:「故侍中左光祿大夫貞陽公世隆,秉德居業,才兼經緯。



 少播清微,長弘美譽。入參內禁,出贊西牧,專寄郢郊,剋挫巨猾,超越前勛,功著一代。及總任方州,民頌寬德,翼教崇闥,朝稱元正。忠謨嘉猷,簡於朕心,雅志素履,邈不可踰。將登鉉味,用燮鴻化,奄至薨殞,震慟良深。贈司空,班劍三十人,鼓吹一部,侍中如故。謚曰忠武。」上又敕吏部尚書王晏曰:「世隆雖抱疾積歲,志氣未衰,冀醫藥有效,痊差可期。不謂一旦便為異世,痛怛之深,此何可言。其昔在郢,誠心夙悃,全保一蕃,勛業克著。尋準契闊,增泣悲咽。卿同在情,亦當無已已耶!」



 世隆曉數術,於倪塘創墓,與賓客踐履,十往五往,常坐一處。及卒,墓正取其坐處焉。著《龜經秘要》二卷行於世。



 長子悅,早卒。



 張瑰,字祖逸,吳郡吳人也。祖裕,宋金紫光祿大夫。父永,右光祿大
 夫。曉音律,宋孝武問永以太極殿前鐘聲嘶,永答「鐘有銅滓」。乃扣鐘求其處,鑿而去之,聲遂清越。瑰解褐江夏王太尉行參軍,署外兵,隨府轉為太傅五官,為義恭所遇。遷太子舍人,中書郎,驃騎從事中郎,司徒右長史。初,永拒桂陽賊於白下,潰散,阮佃夫等欲加罪,太祖固申明之,瑰由此感恩自結。轉通直散騎常侍,驍騎將軍。遭父喪,還吳持服。



 升明元年,劉秉有異圖,弟遐為吳郡,潛相影響。因沈攸之事起,聚眾三千人,治攻具。太祖密遣殿中將軍卞白龍令瑰取遐。諸張世有豪氣,瑰宅中常有父時舊部曲數百。遐召瑰,瑰偽受旨,與叔恕領兵十八人入郡,與防郡隊主強弩將軍郭羅雲進中齋取遐,遐踰窗而走,瑰部曲顧憲子手斬之,郡內莫敢動者。獻捷,太祖以告領軍張沖,沖曰:「瑰以百口一擲,出手得盧矣。」即授輔國將軍、吳郡太守,封瑰義成縣侯,邑千戶。太祖故以嘉名
 錫之。除冠軍將軍、東海東莞二郡太守,不拜。



 建元元年,增邑二百戶。尋改封平都。遷侍中,加領步兵校尉。二年,遷都官尚書,領校尉如故。出為征虜將軍、吳興太守。三年,烏程令顧昌玄有罪,瑰坐不糾,免官。明年,為度支尚書。



 世祖即位,為冠軍將軍、鄱陽王北中郎長史、襄陽相、行雍州府州事,隨府轉征虜長史。四年,仍為持節、督雍梁南北秦四州郢州之竟陵司州之隨郡軍事、輔國將軍、雍州刺史,尋領寧蠻校尉。還為左民尚書,領右軍將軍,遷冠軍將軍、大司馬長史。十年,轉太常。自陳衰疾,願從閑養。明年,轉散騎常侍、光祿大夫。頃之,上欲復用瑰,乃以為後將軍、南東海太守,秩中二千石,行南徐州府州事,又行河東王國事。到官,復稱疾,還為散騎常侍、光祿大夫。



 鬱林即位,加金章紫綬。隆昌元年,給親信二十人。鬱林廢,朝臣到宮門參承高宗,瑰託腳疾不至。海陵立,加右將軍。
 高宗疑外蕃起兵,以瑰鎮石頭,督眾軍事。瑰見朝廷多難,遂恆臥疾。建武元年,轉給事中、光祿大夫,親信如故。月加給錢二萬。二年,虜盛,詔瑰以本官假節督廣陵諸軍事、行南兗州事,虜退乃還。



 瑰居室豪富,伎妾盈房,有子十餘人,常云「其中要應有好者」。建武末,屢啟高宗還吳,見許。優游自樂。或有譏瑰衰暮畜伎,瑰曰:「我少好音律,老而方解。平生嗜欲,無復一存,唯未能遣此處耳。」



 高宗疾甚,防疑大司馬王敬則,以瑰素著乾略,授平東將軍、吳郡太守,以為之備。及敬則反,瑰遣將吏三千人迎拒於松江,聞敬則軍鼓聲,一時散走,瑰棄郡逃民間。事平,瑰復還郡,為有司所奏,免官削爵。永元初,為光祿大夫。尋加前將軍,金章紫綬。三年,義師下,東昏假瑰節,戍石頭。義師至新亭,瑰棄城走還宮。梁初復為光祿。天監四年卒。



 史臣曰:文以附眾,武以立威,元帥之才,稱為國輔。沈攸之十年治兵,白首舉事,荊楚上流,方江東下。斯驅除之巨難,帝王之大敵。柳世隆勢居中夏,年淺位輕,首抗全師,孤城挑攻,臨埤授策,曾無汗馬。勍寇乖沮,力屈於高墉;亂轍爭先,降奔郢路。陸遜之破玄德,不是過也。及世道清寧,出牧內佐,體之以風素,居之以雅德,固興家之盛美也。



 贊曰:忠武匡贊,實號兼資。廟堂析理,高壘搴旗。游藝善術,安弦拂龜。義成祚土,功立帝基。



\end{pinyinscope}