\article{卷二本紀第二高帝下}

\begin{pinyinscope}

 建元元年夏,四月,甲午,上即皇帝位於南郊,設壇柴燎告天曰:「皇帝臣道成敢用玄牡,昭告皇皇后帝。宋帝陟鑒乾序,欽若明命,以命於道成。夫肇自生民,樹以司牧,所以闡極則天,開元創物,肆茲大道。天下惟公,命不于常。昔在虞、夏,受終上代,粵自漢、魏,揖讓中葉,咸炳諸典謨,載在方冊。水德既微,仍世多故,實賴道成匡拯之功,以弘濟于厥艱。大造顛墜,再構區宇,宣禮明刑,締仁緝義。晷緯凝象,川岳表靈,誕惟天人,罔弗和會。乃仰協歸運,景屬與能,用集大命於茲。辭德匪嗣,至于累仍,而群公卿士,庶尹御事,爰及黎獻,至于百戎,僉曰『皇天眷命,不可以固違,人神無托,不可以曠主』。畏天之威,
 敢不祗順鴻歷?敬簡元辰,虔奉皇符,升壇受禪,告類上帝,以永答民衷,式敷萬國。惟明靈是饗!」



 禮畢,大駕還宮,臨太極前殿。詔曰:「五德更紹,帝迹所以代昌;三正迭隆,王度所以改耀。世有質文,時或因革,其資元膺歷,經道振民,固以異術同揆,殊流共貫者矣。朕以寡昧,屬值艱季,推肆勤之誠,藉樂治之數,賢能悉心,士民致力,用獲拯溺龕暴,一匡天下。業未參古,功殆侔昔。宋氏以陵夷有徵,曆數攸及,思弘樂推,永鑒崇替,爰集天祿于朕躬。惟志菲薄,辭弗獲昭,遂欽從天人,式繇景命,祗月正于文祖,升禋鬯於上帝。猥以寡德,光宅四海,纂革代之蹤,托王公之上,若涉淵水,罔知所濟。寶祚初啟,洪慶惟新,思俾利澤,宣被億兆,可大赦天下。改昇明三年為建元元年。賜民爵二級,文武進位二等,鰥寡孤獨不能自存者穀人五斛。逋租宿債勿復收。有犯鄉論清議,贓污淫盜,一皆蕩滌,洗
 除先注,與之更始。長徒敕系之囚,特皆原遣。亡官失爵,禁錮奪勞,一依舊典。」



 封宋帝為汝陰王,築宮丹陽縣故治,行宋正朔,車旗服色,一如故事,上書不為表,答表不稱詔。降宋晉熙王燮為陰安公,江夏王躋為沙陽公,隨王棨為舞陰公,新興王嵩為定襄公,建安王禧為荔浦公,郡公主為縣君,縣公主為鄉君。詔曰:「繼世象賢,列代盛典,疇庸嗣美,前載令圖。宋氏通侯,乃宜隨運省替。但欽德懷義,尚表墳閭,況功濟區夏,道光民俗者哉?降差之典,宜遵往制。南康縣公華容縣公可為侯,萍鄉縣侯可為伯,減戶有差,以繼劉穆之、王弘、何無忌後。」



 以司空褚淵為司徒,吳郡太守柳世隆為南豫州刺史。詔曰:「宸運肇創,實命惟新,宜弘慶宥,廣敷蠲汰。劫賊餘口沒在臺府者,悉原放。諸負釁流徙,普聽還本土。」以齊國左衛將軍陳顯達為中護軍,中領軍王敬則為南兗州刺史,左衛將軍李安民為
 中領軍。戊戌,以荊州刺史嶷為尚書令、驃騎大將軍、開府儀同三司、揚州刺史,冠軍將軍映為荊州刺史,西中郎將晃為南徐州刺史,冠軍將軍垣崇祖為豫州刺史,驃騎司馬崔文仲為徐州刺史。



 斷四方上慶禮。己亥,詔曰:「自廬井毀制,農桑易業,鹽鐵妨民,貨鬻傷治,歷代成俗,流蠹歲滋。援拯遺弊,革末反本,使公不專利,氓無失業。二宮諸王,悉不得營立屯邸,封略山湖。太官池嵒,宮停稅入,優量省置。」



 庚子,詔「宋帝后蕃王諸陵,宜有守衛。」有司奏帝陵各置長一人,兵有差,王陵五人,妃嬪三人。



 五月,丙午,進河南王吐谷渾拾寅號驃騎大將軍。詔曰:「宸運革命,引爵改封,宋氏第秩,雖宜省替,其有預效屯夷、宣力齊業者,一仍本封,無所減降。有司奏留襄陽郡公張敬兒等六十二人,除廣興郡公沈曇亮等百二十二人。改《元嘉曆》為《建元曆》,木德盛卯終未,以正月卯祖,十二月未臘。丁
 未,詔曰:「設募取將,懸賞購士,蓋出權宜,非曰恆制。頃世艱險,浸以成俗,且長逋逸,開罪山湖。



 是為黥刑不辱,亡竄無咎。自今以後,可斷眾募。」壬子,詔封佐命文武功臣新除司徒褚淵等三十一人,進爵增戶各有差。乙卯,河南王吐谷渾拾寅奉表貢獻。丙辰,詔遣大使分行四方,遣兼散騎常侍十二人巡行。以交寧道遠,不遣使。己未,汝陰王薨,追謚為宋順帝,終禮依魏元、晉恭帝故事。辛酉,陰安公劉燮等伏誅。追封謚上兄道度為衡陽元王,道生為始安貞王。丙寅,追尊皇考曰宣皇帝,皇妣為孝皇后,妃為昭皇后。



 六月,辛未,詔「相國驃騎中軍三府職,可依資勞度二官,若職限已盈,所餘可賜滿。」壬申,以游擊將軍周山圖為兗州刺史。乙亥,詔曰:「宋末頻年戎寇,兼災疾凋損,或枯骸不收,毀櫬莫掩,宜速宣下埋藏營恤。若標題猶存,姓字可識,可即運載,致還本鄉。」有司奏遣外監典事四
 人,周行離門外三十五里為限。其餘班下州郡。無棺器標題者,屬所以臺錢供市。庚辰,七廟主備法駕即於太廟。詔「諸將及客,戮力艱難,盡勤直衛,其從還宮者,普賜位一階。」辛巳,罷荊州刺史。甲申,立皇太子賾。斷諸州郡禮慶。見刑入重者,降一等,並申前赦恩百日。



 立皇子嶷為豫章王,映為臨川王,晃為長沙王,曄為武陵王,暠為安成王,鏘為鄱陽王,鑠為桂陽王,鑒為廣陵王,皇孫長懋為南郡王。乙酉,葬宋順帝于遂寧陵。



 秋,七月,丁未,詔曰:「交止北景,獨隔書朔,斯乃前運方季,負海不朝,因迷遂往,歸款莫由。曲赦交州部內李叔獻一人即撫南士,文武詳才選用。並遣大使宣揚朝恩。」以試守武平太守行交州府事李叔獻為交州刺史。丙辰,以虜偽茄蘆鎮主陰平公楊廣香為沙州刺史。丁巳,詔「南蘭陵桑梓本鄉,長蠲租布;武進王業所基,復十年。」



 九月,辛丑,詔「二吳、義興三郡遭
 水,減今年田租。」乙巳,以新除尚書令、驃騎將軍豫章王嶷為荊、湘二州刺史,平西將軍臨川王映為揚州刺史。丙午,司空褚淵領尚書令。戊申,車駕幸宣武堂宴會,詔諸王公以下賦詩。



 冬,十月,丙子,立彭城劉胤為汝陰王,奉宋帝後。己卯,車駕殷祠太廟。辛巳,詔曰:「朕嬰綴世務,三十餘歲,險阻艱難,備嘗之矣。末路屯夷,戎車歲駕,誠藉時來之運,實資士民之力。宋元徽二年以來,諸從軍得官者,未悉蒙祿,可催速下訪,隨正即給。才堪餘任者,訪洗量序。若四州士庶,本鄉淪陷,簿籍不存,尋校無所,可聽州郡保押,從實除奏。荒遠闕中正者,特許據軍簿奏除。或戍捍邊役,末由旋反,聽於同軍各立五保,所隸有司,時為言列。」汝陰太妃王氏薨,追贈為宋恭皇后。



 十一月,庚子,以太子左衛率蕭景先為司州刺史。辛亥,立皇太子妃裴氏。甲申,封功臣驃騎長史江謐等十人爵戶各有差。



 二年春,正月,戊戌朔,大赦天下。以司空、尚書令褚淵為司徒,中軍將軍張敬兒為車騎將軍,中領軍李安民為領軍將軍,中護軍陳顯達為護軍將軍。辛丑,車駕親祠南郊。癸卯,詔索虜寇淮、泗,遣眾軍北伐,內外纂嚴。



 二月,丁卯,虜寇壽陽,豫州刺史垣崇祖破走之。置巴州。壬申,以三巴校尉明慧昭為巴州刺史。戊子,以寧蠻校尉蕭赤斧為雍州刺史,南蠻長史崔惠景為梁、南秦二州刺史。辛卯,詔西境獻捷,解嚴。癸巳,遣大使巡慰淮、肥。徐、豫邊民尤貧遘難者,刺史二千石量加賑恤。甲午,詔「江西北民避難流徙者,制遣還本,蠲今年租稅。單貧及孤老不能自存者,即聽番籍,郡縣押領。」三月,丁酉,以侍中西昌侯鸞為郢州刺史。戊戌,以護軍將軍陳顯達為南兗州刺史,吳郡太守張岱為中護軍。己亥,車駕幸樂遊苑宴,王公以下賦詩。辛丑,以征虜將軍崔祖思為青、冀二州刺史。夏,四
 月,丙寅,進高麗王樂浪公高璉號驃騎大將軍。



 五月,立六門都墻。六月,癸未,詔「昔歲水旱,曲赦丹陽、二吳、義興四郡遭水尤劇之縣,元年以前,三調未充,虛列已畢,官長局吏應共償備外,詳所除宥。」



 秋,七月,甲寅,以輔國將軍盧紹之為青、冀二州刺史。戊午,皇太子妃裴氏薨。



 閏月辛巳,遣領軍將軍李安民行淮、泗。庚寅,索虜攻朐山,青、冀二州刺史盧紹之等破走之。冬,十一月,戊子,以氐楊後起為秦州刺史。



 十二月,戊戌,以司空褚淵為司徒。乙巳,車駕幸中堂聽訟。壬子,以驃騎大將軍豫章王嶷為司空,揚州刺史、前將軍臨川王映為荊州刺史。



 三年春,正月,壬戌朔,詔王公卿士薦讜言。丙子,以平北將軍陳顯達為益州刺史,貞陽公柳世隆為南兗州刺史,皇子鋒為江夏王。領軍將軍李安民等破虜於淮陽。夏,四月,以寧朔將軍沈
 景德為廣州刺史。



 六月,壬子,大赦。逋租宿債,除減有差。秋七月,以冠軍將軍垣榮祖為徐州刺史。冬,十月,戊子,以河南王世子吐谷渾易度侯為西秦、河二州刺史,河南王。



 四年,春,正月,壬戌,詔曰:「夫膠庠之典,彞倫攸先,所以招振才端,啟發性緒,弘字黎氓,納之軌義,是故五禮之跡可傳,六樂之容不泯。朕自膺曆受圖,志闡經訓,且有司群僚,奏議咸集,蓋以戎車時警,文教未宣,思樂泮宮,永言多慨。今關燧無虞,時和歲稔,遠邇同風,華夷慕義。便可式遵前準,脩建敩學,精選儒官,廣延國胄。」以江州刺史王延之為右光祿大夫。癸亥,詔曰:「比歲申威西北,義勇爭先,殞氣寇場,命盡王事。戰亡蠲復,雖有恆典,主者遵用,每傷簡薄。建元以來戰亡,賞蠲租布二十年,雜役十年。其不得收屍,主軍保押,亦同此例。」以後將軍長沙王晃為護軍將軍,中軍將軍南郡王
 長懋為南徐州刺史,冠軍將軍安成王暠為江州刺史。



 二月,乙未,以冠軍將軍桓康為青、冀二州刺史。上不豫,庚戌,詔原京師囚繫有差,元年以前逋責皆原除。三月,庚申,召司徒褚淵、左僕射王儉詔曰:「吾本布衣素族,念不到此,因藉時來,遂隆大業。風道沾被,升平可期。遘疾彌留,至于大漸。公等奉太子如事吾,柔遠能邇,緝和內外,當令太子敦穆親戚,委任賢才,崇尚節儉,弘宣簡惠,則天下之理盡矣。死生有命,夫復何言!」壬戌,上崩於臨光殿,年五十六。四月,庚寅,上謚曰太祖高皇帝。奉梓宮於東府前渚升龍舟。



 丙午,窆武進泰安陵。



 上少沈深有大量,寬嚴清儉,喜怒無色。博涉經史,善屬文,工草隸書,弈棋第二品。雖經綸夷險,不廢素業。從諫察謀,以威重得眾。即位後,身不御精細之物,敕中書舍人桓景真曰:「主衣中似有玉介導,此制始自大明末,後泰始尤增其麗。留此置主衣,政是
 興長疾源,可即時打碎。凡復有可異物,皆宜隨例也。」後宮器物欄檻以銅為飾者,皆改用鐵,內殿施黃紗帳,宮人著紫皮履,華蓋除金花瓜,用鐵迴釘。每曰:「使我治天下十年,當使黃金與土同價。」欲以身率天下,移變風俗。上姓名骨體及期運歷數,並遠應圖讖數十百條,歷代所未有,臣下撰錄,上抑而不宣,盛矣。



 史臣曰:孫卿有言:「聖人之有天下,受之也,非取之也。」漢高神武駿聖,觀秦氏東遊,蓋是雅多大言,非始自知天命;光武聞少公之論讖,亦特一時之笑語;魏武初起義兵,所期「征西」之墓;晉宣不內迫曹爽,豈有定霸浮橋?宋氏崛起匹夫,兵由義立:咸皆一世推雄,卒開鼎祚。宋氏正位八君,卜年五紀,四絕長嫡,三稱中興,內難邊虞,兵革世動。太祖基命之初,武功潛用,泰始開運,大拯時艱,龍德在
 田,見猜雲雨之跡。及蒼梧暴虐,釁結朝野,百姓懍懍,命懸朝夕。權道既行,兼濟天下。元功振主,利器難以假人,群才戮力,實懷尺寸之望。豈其天厭水行,固已人希木德。歸功與能,事極乎此。雖至公於四海,而運實時來;無心於黃屋,而道隨物變。應而不為,此皇齊所以集大命也。



 贊曰:於皇太祖,有命自天,同度宇宙,合量山淵。宋德不紹,神器虛傳。寧亂以武,黜暴資賢。庸發西疆,功興北翰,偏師獨克,孤旅霆斷。援旆東夏,職司靜亂;指斧徐方,時惟伐叛;抗威京輦,坐清江漢。文藝在躬,芳塵淵塞。用下以才,鎮民以德。端己雄睟,君臨尊默。苞括四海,大造家國。



\end{pinyinscope}