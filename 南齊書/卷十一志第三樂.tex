\article{卷十一志第三樂}

\begin{pinyinscope}

 南郊樂舞歌辭,二漢同用,見《前漢志》,五郊互奏之。魏歌舞不見,疑是用漢辭也。晉武帝泰始二年,郊祀明堂,詔禮遵用周室肇稱殷祀之義,權用魏儀。後使傅玄造《祠天地五郊夕牲歌》詩一篇,《迎神歌》一篇。宋文帝使顏延之造《郊天夕牲》、《迎送神》、《饗神歌》詩三篇,是則宋初又仍晉也。建元二年,有司奏,郊廟雅樂歌辭舊使學士博士撰,搜簡採用,請敕外,凡肄學者普令制立。參議:「太廟登歌宜用司徒褚淵,餘悉用黃門郎謝超宗辭。」超宗所撰,多刪顏延之、謝莊辭以為新曲,備改樂名。永明二年,太子步兵校尉伏曼容上表,宜集英儒,刪纂雅樂。詔付外詳,竟不行。



 群臣出入,奏《肅咸之樂》:夤承寶命,嚴恭帝緒。奄受敷錫,升中拓宇。亙地稱皇,罄天作主。月域來寶,日際奉土。開元首正,禮交樂舉。六典聯事,九官列序。此下除四句。皆顏辭。



 牲出入,奏《引牲之樂》:皇乎敬矣,恭事上靈。昭教國祀,肅肅明明。有牲在滌,有潔在俎。以薦王衷,以答神祜。此上四句,顏辭。陟配在京,降德在民。奔精望夜,高燎佇晨。



 薦豆呈毛血,奏《嘉薦之樂》:我恭我享,惟孟之春。以孝以敬,立我蒸民。青壇奄靄,翠幕端凝。嘉俎重薦,兼籍再升。設業設虡,展容玉庭。肇禋配祀,克對上靈。此一篇增損
 謝辭。



 右夕牲歌,並重奏。



 迎神,奏《昭夏之樂》:惟聖饗帝,惟孝饗親。此下除二句。禮行宗祀,敬達郊禋。金枝中樹,廣樂四陳。此下除八句。月御案節,星驅扶輪。遙興遠駕,曜曜振振。告成大報,受釐元神。



 皇帝入壇東門,奏《永至之樂》:紫壇望靈,翠幕佇神。率天奉贄,罄地來賓。神貺並介,泯祗合祉。恭昭鑒享,肅光孝祀。威藹四靈,洞曜三光。皇德全被,大禮流昌。



 皇帝升壇,奏登歌辭:報惟事天,祭實尊靈。史正嘉兆,神宅崇禎。五畤昭鬯,六宗彞序。介丘望塵,皇軒肅舉。



 皇帝初獻,奏《文德宣烈之樂》:營泰畤,定天衷。思心緒,謀筮從。此下除二句。田燭置,雚火通。大孝昭,國禮融。此一句改,餘皆顏辭,此下又除二十二句。



 次奏《武德宣烈之樂》:功燭上宙,德耀中天。風移九域,禮飾八埏。四靈晨炳,五緯宵明。膺歷締運,道茂前聲。



 太祖高皇帝配饗,奏《高德宣烈之樂》。此章永明二年造奏。尚書令王儉辭。



 饗帝嚴親,則天光大。鋋弈前古,榮鏡無外。日月宣華,卿雲流靄。五漢同休,六幽咸泰。



 皇帝飲福酒,奏《嘉胙之樂》:鬯嘉禮,承休錫。盛德符景緯,昌華應帝策。聖藹耀昌基,融祉暉世曆。聲正涵月軌,書文騰日跡。
 寶瑞昭神圖,靈貺流瑞液。我皇崇暉祚,重芬冠往籍。



 送神,奏《昭夏之樂》:薦饗洽,禮樂該。神娛展,辰旆迴。洞雲路,拂璇階。紫分藹,青霄開。眷皇都,顧玉臺。留昌德,結聖懷。



 皇帝就燎位,奏《昭遠之樂》:天以德降,帝以禮報。牲樽俯陳,柴幣仰燎。事展司採,敬達瑄薌。煙贄青昊,震揚紫場。陳馨示策,肅志宗禋。禮非物備,福唯誠陳。



 皇帝還便殿,奏《休成之樂》。重奏。



 昭事上祀,饗薦具陳。回鑾轉翠,拂景翔宸。綴縣敷暢,鍾石昭融。羽炫深晷,籥曀行風。肆序輟度,肅禮停文。四金聳衛,六馭齊輪。



 ──右南郊歌辭北郊樂歌辭,案《周頌·昊天有成命》,郊祀天地也。是則周、漢以來,祭天地皆同辭矣。宋顏延之《饗地神辭》一篇,餘與南郊同。齊北郊,群臣入奏《肅咸樂》;牲入,奏《引牲》;薦豆毛血,奏《嘉薦》;皇帝入壇東門,奏《永至》;飲福酒,奏《嘉胙》;還便殿,奏《休成》:辭並與南郊同。迎送神《昭夏》登歌異。



 迎地神,奏《昭夏之樂》:詔禮崇營,敬饗玄畤。靈正丹帷,月肅紫墀。展薦登華,風縣凝鏘。神惟戾止,鬱葆遙莊。昭望歲芬,環遊辰太。穆哉尚禮,橫光秉藹。



 皇帝升壇登歌:佇靈敬享,禋肅彞文。縣動聲儀,薦潔牲芬。陰祇以貺,
 昭司式慶。九服熙度,六農祥正。



 皇帝初獻,奏《地德凱容之樂》:繕方丘,端國陰,掩珪晷,仰靈心。詔源委,遍丘林。此下除八句禮獻物,樂薦音。此下除二十二句餘皆顏辭。



 次奏《昭德凱容之樂》:慶圖浚邈,蘊祥秘瑤。伣天炳月,嬪光紫霄。邦化靈懋,閫則風調。儷德方儀,徽載以昭。



 送神,奏《昭夏之樂》:薦神升,享序楙。淹玉俎,停金奏。寶旆轉,旒駕旋。溢素景,鬱紫躔。靈心顧,留辰眷。洽外瀛,瑞中縣。



 瘞埋,奏《隸幽之樂》:后皇嘉慶,定祗玄畤。承帝休圖,祗敷靈祉。篚冪周序,
 軒朱凝會。牲幣芬壇,精明佇蓋。調川瑞昌,警岳祥泰。



 ──右北郊歌辭。



 明堂歌辭,祠五帝。漢郊祀歌皆四言,宋孝武使謝莊造辭,莊依五行數,木數用三,火數用七,土數用五,金數用九,水數用六。案《鴻範》五行,一曰水,二曰火,三曰木,四曰金,五曰土。《月令》木數八,火數七,土數五,金數九,水數六。蔡邕云:「東方有木三土五,故數八;南方有火二土五,故數七;西方有金四土五,故數九;北方有水一土五,故數六。」又納音數,一言得土,三言得火,五言得水,七言得金,九言得木。若依《鴻範》木數用三,則應水一火二金四也。



 若依《月令》金九水六,則應木八火七也。當以《鴻範》一二之數,言不成文,故有取舍,而使兩義並違,未詳以數立言為何依據也。《周頌·我將》祀文王,言皆四,其一句五,一句七。謝莊歌宋太祖亦無定句。建元初,詔黃門
 郎謝超宗造明堂夕牲等辭,並採用莊辭。建武二年,雩祭明堂,謝朓造辭,一依謝莊,唯世祖四言也。



 賓出入,奏《肅咸樂》歌辭二章:彞承孝典,恭事嚴聖。浹天奉贐,罄壤齊慶。司儀且序,羽容夙章。芬枝揚烈,黼構周張。助寶尊軒,酎珍充庭。璆縣凝會,琄朱佇聲。先期選禮,肅若有承。祗對靈祉,皇慶昭膺。



 尊事威儀,輝容昭序。迅恭明神,潔盛牲俎。肅肅嚴宮,藹藹崇基。皇靈降止,百祇具司。戒誠望夜,端烈承朝。依微昭旦,物色輕霄。



 《青帝歌》:參映夕,駟昭晨。靈乘震,司青春。雁將向,桐始蕤。和風舞,暄光遲。萌動達,萬品親。潤無際,澤無垠。



 《
 赤帝歌》:龍精初見大火中,朱光北至圭景同。帝在在離實司衡,雨水方降木堇榮。庶物盛長咸殷阜,恩澤四溟被九有。



 《黃帝歌》:履艮宅中宇,司繩總四方。裁化遍寒燠,布政司炎涼。此以下除八句。至分乘經晷,閉啟集恆度。帝暉緝萬有,皇靈澄國步。



 《白帝歌》:百川若鏡天地爽且明。雲沖氣舉盛德在素精。此下除四句。庶類收成歲功行欲寧。浹地奉渥罄宇承帝靈。



 《黑帝歌》:歲既暮日方馳。靈乘坎德司規。玄雲合晦鳥蹊。
 白雲繁亙天崖。此下除四句。



 晨晷促夕漏延。大陰極微陽宣。此下除二句皇帝還東壁,受福酒,奏《嘉胙樂》歌辭太廟同用:禮薦洽,福祚昌。聖皇膺嘉佑,帝業凝休祥。居極乘景運,宅德瑞中王。澄明臨四奧,精華延八饗。洞海同聲惠,澈宇麗乾光。靈慶纏世祉,鴻烈永無疆。



 送神,奏《昭夏樂》歌辭宋謝莊辭:蘊禮容,餘樂度。靈方留,景欲暮。開九重,肅五達。鳳參差,龍已秣。雲既動,河既梁。萬里照,四空香。神之車,歸清都。璇庭寂,玉殿虛。鴻化凝,孝風熾。顧靈心,結皇思。鴻慶遐鬯,嘉薦令芳。並帝明德,永祚深光增四句。



 牲出入,奏《引牲樂》歌詩:
 惟誠潔饗,維孝尊靈。敬芳黍稷,敬滌犧牲。騂繭在豢,載溢載豐。以承宗祀,以肅皇衷。蕭芳四舉,華火周傳。神鑒孔昭,嘉足參牷。



 薦豆呈毛血,奏《嘉薦樂》歌詩二章:肇禋戒祀,禮容咸舉。六典飾文,九司炤序。牲柔既昭,犧剛既陳。恭滌惟清,敬事惟神。加籩再御,兼俎兼薦。節動軒越,聲流金縣。



 奕奕閟幄,亹亹嚴闈。潔誠夕鑒,端服晨暉。聖靈戾止,翊我皇則。上綏四宇,下洋萬國。永言孝饗,孝饗有容。儐僚贊列,肅肅雍雍。



 ──右夕牲辭迎神,奏《昭夏樂》歌辭:地紐謐,乾樞回。華蓋動,紫微開。旌蔽日,車若云。駕六氣,乘煙
 煴。燁帝景,耀天邑。聖祖降,五雲集。此下除八句。懋粢盛,潔牲牷。百禮肅,群司虔。皇德遠,大孝昌。貫九幽,洞三光。神之安,解玉鑾。昌福至,萬宇歡。皆謝莊辭。



 皇帝升明堂。奏登歌辭:雍臺辯朔,澤宮選辰。挈火夕照,明水朝陳。六瑚賁室,八羽華庭。昭事先聖,懷濡上靈。肆夏式敬,升歌發德。永固洪基,以綏萬國。皆謝莊辭。



 初獻,奏《凱容宣烈樂》歌辭太廟同:釃醴具登,嘉俎咸薦。饗洽誠陳,禮周樂遍。祝辭罷祼,序容輟縣。蹕動端庭,鑾回嚴殿。神儀駐景,華漢高虛。八靈案衛,三祇解途。翠蓋澄耀,罼紘凝晨。玉虡息節,金輅懷音。戒誠達孝,颭心肅感。追馮皇鑒,思承淵範。神
 錫懋祉,四緯昭明。仰福帝徽,俯齊庶生。



 ──右祠明堂歌辭。建元、永明中奏。



 雩祭歌辭:清明暢,禮樂新。候龍景,選貞辰。陽律亢,陰晷伏。秏下土,薦璟稑。震儀警,王度乾。嗟云漢,望昊天。張盛樂,奏《雲舞》。集五精,延帝祖。雩有諷,勣有秩。珝鬯芬,圭瓚瑟。靈之來,帝閽開。車煜耀,吹徘徊。停龍犧,遍觀此。



 凍雨飛,祥風靡。壇可臨,奠可歆。對氓祉,鑒皇心。



 ──右迎神歌辭。依漢來郊歌三言。宋明堂迎神八解。



 浚哲維祖,長發其武。帝出自震,重光御宇。七德攸宣,九疇咸敘。靜難荊舒,凝威蠡浦。昧旦丕承,夕惕刑政。化
 一車書,德香粢盛。昭星夜景,非雲曉慶。衢室成陰,璧水如鏡。禮充玉帛,樂被管弦。於鑠在詠,陟配於天。自宮徂兆,靡愛牲牷。我將我享,永祚豐年。



 ──右歌世祖武皇帝依廟歌四言營翼日,鳥殷宵。凝冰泮,玄蟄昭。景陽陽,風習習。女夷歌,東皇集。奠春酒,秉青圭。命田祖,渥群黎。



 ──右歌青帝木生數三惟此夏德德恢臺,雨龍既御炎精來。火景方中南訛秩,靡草云黃含桃實。族雲蓊鬱溫風煽,興雨祁祁黍苗遍。



 ──右歌赤帝火成數七稟火自高明,毓金挺剛克。涼燠資成化,群方載厚德。陽季勾萌達,炎徂溽暑融。商暮百工止,歲極凌陰沖。皇流疏已清,原隰甸
 已平。咸言祚惟億,敦民保高京。



 ──右歌黃帝土成數五帝悅於兌執矩固司藏。百川收潦精景應徂商。嘉樹離披榆關命賓鳥。夜月如霜秋風方裊裊。商陰肅殺萬寶咸亦遒。勞哉望歲場功冀可收。



 ──右歌白帝金成數九白日短、玄夜深。招搖轉、移太陰。霜鍾鳴、冥陵起。星回天、月窮紀。聽嚴風、來不息。望玄雲、黝無色。曾冰洌、積羽幽。飛雪至、天山側。關梁閉、方不巡。合國吹、饗蠟賓。充微陽、究終始。百禮洽、萬祚臻。



 ──右歌黑帝水成數六敬如在,禮將周。神之駕,不少留。躡龍鑣,轉金蓋。紛
 上馳,雲之外。警七曜,詔八神。排閶闔,渡天津。有渰興,膚寸積。雨冥冥,又終夕。俾棲糧,惟萬箱。皇情暢,景命昌。



 ──右送神歌辭太廟樂歌辭,《周頌·清廟》一篇,漢《安世歌》十七章是也。



 永平三年,東平王蒼造光武廟登歌一章二十六句,其辭稱述功德。



 建安十八年,魏國初建,侍中王粲作登歌《安世詩》,說神靈鑒饗之意。明帝時,侍中繆襲奏:「《安世詩》本故漢時歌名,今詩所歌,非往詩之文。襲案《周禮》志云,《安世樂》猶周房中樂也。往昔議者,以房中歌后妃之德,宜改《安世》名《正始之樂》,後讀漢《安世歌》,亦說神來宴饗,無有后妃之言。思惟往者謂房中樂為后妃歌,恐失其意。方祭祀娛神,登歌先祖功德,下堂詠宴享,無事歌后妃之化也。」於是改《安世樂》曰《饗
 神歌》。散騎常侍王肅作宗廟詩頌十二篇,不入於樂。



 晉泰始中,傅玄造《廟夕牲昭夏》歌一篇,《迎送神肆夏》歌詩一篇,登歌七廟七篇。玄云:「登歌歌盛德之功烈,故廟異其文。至於饗神,猶《周頌》之《有瞽》及《雍》,但說祭饗神明禮樂之盛,七廟饗神皆用之。」夏侯湛又造宗廟歌十三篇。



 宋世王韶之造七廟登歌七篇。升明中,太祖為齊王,令司空褚淵造太廟登歌二章。建元初,詔黃門侍郎謝超宗造廟樂歌詩十六章。



 永明二年,尚書殿中曹奏:「太祖高皇帝廟神室奏《高德宣烈之舞》,未有歌詩,郊應須歌辭。穆皇后廟神室,亦未有歌辭。案傅玄云:『登歌廟異其文,饗神七室同辭。』此議為允。又尋漢世歌篇多少無定,皆稱事立文,並多八句,然後轉韻。時有兩三韻而轉,其例甚寡。張華、夏侯湛亦同前式。傅玄改韻頗數,更傷簡節之美。近世王韶之、顏延之並四韻乃轉,得賒促之中。顏延之、謝莊作
 三廟歌,皆各三章章八句,此於序述功業詳略為宜,今宜依之。郊配之日,改降尊作主,禮殊宗廟;穆后母儀之化,事異經綸。此二歌為一章八句,別奏事御奉行。」詔「可」。



 尚書令王儉造太廟二室及郊配辭。



 群臣出入,奏《肅咸樂》歌辭:潔誠颭孝,孝感煙霜。夤儀飾序,肅禮綿張。金華樹藻,肅哲騰光。殷殷升奏,嚴嚴階庠。匪椒匪玉,是降是將。懋分神衷,翊佑傳昌。



 牲出入,奏《引牲樂》歌辭:肇祀嚴靈,恭禮尊國。達敬敷典,結孝陳則。芬滌既肅,犧牷既整。聳誠流思,端儀選景。肆禮佇夜,綿樂望晨。
 崇席皇鑒,用饗明神。



 薦豆呈毛血,奏《嘉薦樂》歌辭:清思眑眑,紵寢微微。恭言載感,肅若有希。芬俎具陳,嘉薦兼列。凝馨煙颺,分照星晰。睿靈式降,協我帝道。上澄五緯,下陶八表。



 ──右夕牲歌辭迎神,奏《昭夏樂》歌辭:涓辰選氣,展禮恭祗。重闈月洞,層牖煙施。載虛玉鬯,載受金枝。天歌折饗,雲舞罄儀。神惟降止,泛景凝羲。帝華永藹,泯藻方摛。



 皇帝入廟北門,奏《永至樂》歌辭:戲繇惟則,姬經式序。九司聯事,八方承宇。鑾迾靜陳,縵樂具舉。凝旒若慕,傾璜載佇。振振璇衛,穆穆禮容。
 載藹皇步,式敷帝蹤。



 太祝祼地,奏登歌辭:清明既鬯,大孝乃熙。天儀睟愴,皇心儼思。既芬房豆,載潔牷牲。鬱祼升禮,鋗玉登聲。茂對幽嚴,式奉徽靈。以享以祀,惟感惟誠。



 皇祖廣陵丞府君神室奏《凱容樂》歌辭:國昭惟茂,帝穆惟崇。登祥緯遠,締世景融。紛綸睿緒,菴蔚王風。明進厥始,浚哲文終。



 皇祖太中大夫府君神室奏《凱容樂》歌辭:璇條夤蔚,瓊源浚照。懋矣皇烈,載挺明劭。永言敬思,式恭惟教。休途良鳷,榮光有耀。



 皇祖淮陰令府君神室奏《凱容樂》歌辭:
 嚴宗正典,崇饗肇禋。九章既飾,三清既陳。昭恭皇祖,承假徽神。貞祐伊協,卿藹是鄰。



 皇曾祖即丘令府君神室奏《凱容樂》歌辭:肅惟敬祀,潔事參薌。環袨像綴,緬密絲簧。明明烈祖,尚錫龍光。粵《雅》于姬,伊《頌》在商。



 皇祖太常卿府君神室奏《凱容樂》歌辭:神宮懋鄴,明寢昌基。德凝羽綴,道鬯容辭。假我帝緒,懿我皇維。昭大之載,國齊之祺。



 皇考宣皇神室奏《宣德凱容樂》歌辭:道紵期運,義開藏用。皇矣睿祖,至哉攸縱。循規烈照,襲矩重芬。德溢軒羲,道懋炎雲。



 昭皇后神室奏《凱容樂》歌辭:
 月靈誕慶,雲瑞開祥。道茂淵柔,德表徽章。粹訓宸中,儀形宙外。容蹈凝華,金羽傳藹。



 皇帝還東壁上福酒,奏《永祚樂》歌辭:構宸抗宇,合軫齊文。萬靈載溢,百禮以殷。朱弦繞風,翠羽停雲。桂樽既滌,瑤俎既薰。升薦惟誠,昭禮惟芬。降祉遙裔,集慶氤氳。



 送神,奏《肆夏樂》歌辭:禮既升,樂以愉。昭序溢,幽饗餘。人祗鬯,敬教敷。申光動,靈駕翔。芬九垓,鏡八鄉。福無屆,祚無疆。



 皇帝詣便殿,奏《休成樂》歌辭:睿孝式鬯,饗敬爰遍。諦容輟序,佾文靜縣。辰儀聳蹕,宵衛浮鑾。旒帟雲舒,翠華景摶。恭惟尚烈,休明再纏。
 國猷遠藹,昌圖聿宣。



 惟王建國,設廟凝靈。月薦流典,時祀暉經。瞻辰僾思,雨露追情。簡日筮晷,紵奠升文。金罍渟桂,沖幄舒薰。備僚肅列,駐景開雲。



 至饗攸極,睿孝惇禮。具物咸潔,聲香合體。氣昭扶幽,眇慕纏遠。迎絲驚促,迭佾留晚。聖衷踐候,節改增愴。妙感崇深,英徽彌亮。



 太祖高皇帝神室奏《高德宣烈樂》歌辭:悠悠草昧,穆穆經綸。乃文乃武,乃聖乃神。動龕危亂,靜比斯民。誕應休命,奄有八夤。握機肇運,光啟禹服。義滿天淵,禮昭地軸。澤靡不懷,威無不肅。戎夷竭歡,象來致福。偃風裁化,恆日敷祥。信星含曜,秬草流芳。
 七廟觀德,六樂宣章。惟先惟敬,是饗是將。



 穆皇後神室奏《穆德凱容之樂》歌辭:大姒嬪周,塗山儷禹。我后嗣徽,重規疊矩。肅肅紵宮,翔翔《雲舞》。有饗德馨,無絕終古。



 高宗明皇帝神室奏《明德凱容之樂》歌辭:多難固業,殷憂啟聖。帝宗纘武,惟時執競。起柳獻祥,百堵興詠。義雖祀夏,功符受命。遠無不懷,邇無不肅。其儀濟濟,其容穆穆。赫矣君臨,昭哉嗣服。允王維后,膺此多福。禮以昭事,樂以感靈。八簋陳室,六舞充庭。觀德在廟,象德在形。四海來祭,萬國咸寧。



 藉田歌辭,漢章帝元和元年,玄武司馬班固奏用《周頌·載芟》祠先農。晉傅玄作《祀先農先蠶夕牲歌詩》一篇八句,《迎送神》一篇,饗
 社稷、先農、先聖、先蠶歌詩三篇,前一篇十二句,中一篇十六句,後一篇十二句,辭皆敘田農事。胡道安《先農饗神詩》一篇,並八句。樂府相傳舊歌三章。永明四年藉田,詔驍騎將軍江淹造《藉田歌》。淹製二章,不依胡、傅,世祖口敕付太樂歌之。



 祀先農迎送神升歌:羽鑾從動,金駕時游。教騰義鏡,樂綴禮脩。率先丹耦,躬遵綠疇。靈之聖之,歲殷澤柔。



 饗神歌辭:瓊斝既飾,繡簋以陳。方燮嘉種,永毓宵民。



 元會大饗四廂樂歌辭,晉泰始五年太僕傅玄撰。正旦大會行禮歌詩四章,壽酒詩一章,食舉東西廂樂十三章,黃門郎張華作。上壽食舉行禮詩十八章,中書監荀勖、侍郎成公綏,言數各異。宋
 黃門郎王韶之造《肆夏》四章,行禮一章,上壽一章,登歌三章,食舉十章,前後舞歌一章。齊微改革,多仍舊辭。其前後舞二章新改。其臨軒樂,亦奏《肆夏·於鑠》四章。



 《肆夏樂》歌辭:於鑠我皇,體仁苞元。齊明日月,比景乾坤。陶甄百王,稽則黃軒。訐謨定命,辰告四蕃。



 右一曲,客入,四廂奏。



 將將蕃後,翼翼群僚。盛服待晨,明發來朝。饗以八珍,樂以《九韶》。仰祗天顏,厥猷孔昭。



 右一曲,皇帝當陽,四廂奏。皇帝入變服,四廂並奏前二曲。



 法章既設,初筵長舒。濟濟列辟,端委皇除。飲和無盈,威儀有餘。溫恭在位,敬終如初。



 九功既歌,六代惟時。
 被德在樂,宣道以詩。穆矣大和,品物咸熙。慶積自遠,告成在茲。



 右二曲,皇帝入變服,黃鐘太蔟二廂奏。



 大會行禮歌辭:大哉皇齊,長發其祥:祚隆姬夏,道邁虞唐。德之克明,休有烈光。配天作極,辰居四方。



 皇矣我後,聖德通靈:有命自天,誕授休禎。龍飛紫極,造我齊京;光宅宇宙,赫赫明明。



 右二曲,姑洗廂奏。



 上壽歌辭:獻壽爵,慶聖皇。靈祚窮二儀,休
 明等三光。



 右一曲,黃鐘廂奏。



 殿前登歌辭:明明齊國,緝熙皇道。則天垂化,光定天保。天保既定,肆覲萬方。禮繁樂富,穆穆皇皇。



 沔彼流水,朝宗天池。洋洋貢職,抑抑威儀。既習威儀,亦閑禮容。一人有則,作孚萬邦。



 烝哉我皇,實靈誕聖。履端惟始,對越休慶。如天斯崇,如日斯盛。介茲景福,永固洪命。



 右三曲,別用金石,太樂令跪奏。



 食舉歌辭:晨儀載煥,萬物咸睹。嘉慶三朝,禮樂備舉。元正肇始,典章徽明。萬方來賀,華夷充庭。多士盈九德,俯仰觀玉聲。恂恂俯仰,載爛其暉。鍾鼓震天區,禮容塞皇闈。思
 樂窮休慶,福履同所歸。



 五玉既獻,三帛是薦。爾公爾侯,鳴玉華殿。皇皇聖后,降禮南面。元首納嘉禮,萬邦同欽願。休哉休哉,君臣熙宴。建五旗,列四縣。樂有文,禮無倦。融王風,窮一變。



 禮至和,感陰陽。德無不柔,繫休祥。瑞徵辟,應嘉鍾。舞雲鳳,躍潛龍。景星見,甘露墜。木連理,禾同穗。玄化洽,仁澤敷。極禎瑞,窮靈符。



 懷荒遠,綏齊民。荷天祐,靡不賓。靡不賓,長世盛。昭明有融,繁嘉慶。繁嘉慶,熙帝載。含氣感和,蒼生欣戴。三靈協瑞,惟新皇代。



 王道四達,流仁德。窮理詠乾元,垂訓從帝則。靈化侔四時,幽誠通玄默。德澤被八紘,禮章軌萬國。



 皇猷緝,咸熙泰。禮儀煥帝庭,要荒服遐外。被髮襲纓冕,右衽回衿帶。天覆地載,澤流汪濊。聲教布濩,德光大。



 開元辰,畢來王。奉貢職,朝后皇。鳴珩佩,觀典章。樂王慶,悅徽芳。陶盛化,遊大康。惟昌明,永克昌。



 惟建元,德丕顯。齊七政,敷五典。彞倫序,洪化闡。



 王澤流,太平始。樹靈祗,恭明祀。介景祚,膺嘉祉。禮有容,樂有儀。金石陳,干羽施。邁《武》《濩》,均《咸池》。歌《南風》,德永稱。文明煥,頌聲興。



 王道純,德彌淑。寧八表,康九服。導禮讓,移風俗。移風俗,永克融。歌盛美,告成功。詠休烈,邈無窮。



 右黃鐘先奏《晨儀》篇,太蔟奏《五玉》篇,餘八篇二廂更奏之。



 《前舞》階步歌辭新辭:
 天挺聖哲,三方維綱。川岳伊寧,七耀重光。茂育萬物,眾庶咸康。道用潛通,仁施遐揚。德厚巛極,功高昊蒼。舞象盛容,德以歌章。八音既節,龍躍鳳翔。皇基永樹,二儀等長。



 《前舞凱容》歌詩舊辭:於赫景命,天鑒是臨。樂來伊陽,禮作惟陰。歌自德富,舞由功深。庭列宮縣,陛羅瑟琴。翿籥繁會,笙磬諧音。《簫韶》雖古,九奏在今。導志和聲,德音孔宣。



 光我帝基,協靈配乾。儀形六合,化穆自宣。如彼雲漢,為章于天。熙熙萬類,陶和當年。擊轅中韶,永世弗騫。



 《後舞》階步歌辭新辭:皇皇我后,紹業盛明。滌拂除穢,宇宙載清。允執中和,
 以蒞蒼生。玄化遠被,兆世軌形。何以崇德,乃作九成。妍步恂恂,雅曲芬馨。八風清鼓,應以祥禎。澤浩天下,功齊百靈。



 《後舞凱容》歌辭舊辭:假樂聖后,實天誕德。積美自中,王猷四塞。龍飛在天,儀形萬國。欽明惟神,臨朝淵默。不言之化,品物咸得。告成於天,銘勛是勒。翼翼厥猷,亹亹其仁。從命創制,因定和神。海外有截,九國無塵。冕旒司契,垂拱臨民。乃舞《凱容》,欽若天人。純嘏孔休,萬載彌新。



 《宣烈舞》,執干戚。郊廟奏,平冕,黑介幘,玄衣裳,白領袖、絳領袖中衣,絳合幅褲,絳襪。朝廷,則武冠,赤幘,生絳袍單衣,絹領袖,皂領袖中衣,虎文畫合幅褲,白布彩,皆黑韋緹。周《大武舞》,秦改為《五行》。漢高
 造《武德舞》,執干戚,象天下樂己除亂。按《禮》云「朱干玉戚,冕而舞《大武》」,是則漢放此舞而立也。魏文帝改《五行》還為《大武》,而《武德》曰《武頌舞》。明帝改造《武始舞》。晉世仍舊。傅玄六代舞歌有《武》辭,此《武舞》非一也。宋孝建初,朝議以《凱容舞》為《韶舞》,《宣烈舞》為《武舞》。據《韶》為言,《宣烈》即是古之《大武》,非《武德》也。今世諺呼為武王伐紂。其冠服,魏明帝世尚書所奏定《武始舞》服,晉、宋承用,齊初仍舊,不改宋舞名。其舞人冠服,見魏尚書奏,後代相承用之。



 《凱容舞》,執羽籥。郊廟,冠委貌,服如前。朝廷,進賢冠,黑介幘,生黃袍單衣,白合幅褲,餘如前。本舜《韶舞》,漢高改曰《文始》,魏復曰《大韶》。



 又造《咸熙》為《文舞》。晉傅玄六代舞有《虞韶舞》辭。宋以《凱容》繼《韶》為《文舞》。相承用魏咸熙冠服。



 《前舞》、《後舞》,晉泰始九年造。《正德大豫舞》,傅玄、張華各為歌辭。



 宋元
 嘉中,改《正德》為《前舞》,《大豫》為《後舞》。



 ──右朝會樂辭舞曲,皆古辭雅音,稱述功德,宴享所奏。傅玄歌辭云:「獲罪於天,北徙朔方,墳墓誰掃,超若流光。」如此十餘小曲,名為舞曲,疑非宴樂之辭。然舞曲總名起此矣。



 《明君》辭:明君創洪業,盛德在建元。受命君四海,聖皇應靈乾。五帝繼三皇,三皇世所歸。聖德應期運,天地不能違。仰之彌已高,猶天不可階。將復結繩化,靜拱天下齊。



 右一曲,漢章帝造《鼙舞歌》,云「關東有賢女」。魏明帝代漢曲云,「明明魏皇帝」。傅玄代魏曲作晉《洪業篇》云:「宣文創洪業,盛德存泰始。聖皇應靈符,受命君四海。」今前四句錯綜其辭,從「五
 帝」至「不可階」六句全玄辭,後二句本云「將復御龍氏,鳳皇在庭棲」,又改易焉。



 《聖主曲》辭:聖主受天命,應期則虞唐。升旒綜萬機,端扆馭八方。盈虛自然數,揖讓歸聖明。北化陵河塞,南威越滄溟。廣德齊七政,敷教騰三辰。萬宇必承慶,百福咸來臻。聖皇應福始,昌德洞祐先。



 《明君》辭:明君御四海,總鑒盡人靈。仰成恩已洽,竭忠身必榮。聖澤洞三靈,德教被八鄉。草木變柯葉,川岳洞嘉祥。愉樂盛明運,舞蹈升泰時。微霜永昌命,軌心長歡怡。



 《鐸舞》歌辭:
 黃《雲門》,唐《咸池》,虞《韶舞》,夏《夏》殷《濩》,列代有五。振鐸鳴金,延《大武》。清歌發唱,形為主。聲和八音,協律呂。身不虛動,手不徒舉。



 應節合度,周期序。時奏宮角,雜之以徵羽。樂以移風,禮相輔,安有出其所!



 ──右一曲,傅玄辭,以代魏《太和時》。「徵羽」下除「下厭眾目,上從鐘鼓」二句。



 《白鳩》辭:翩翩白鳩,再飛再鳴。懷我君德,來集君庭。



 ──右一曲,《舞敘》云:「《白符》或云《白符鳩舞》,出江南,吳人所造。



 其辭意言患孫皓虐政,慕政化也。其詩本云『平平白符,思我君惠,集我金堂』。



 言白者金行,符,合也,鳩亦合也。符鳩雖異,其義是同。」



 《
 濟濟》辭:暢飛暢舞,氣流芳。追念三五,大綺黃。



 ──右一曲,晉《濟濟舞歌》,六解,此是最後一解。



 《獨祿》辭:獨祿獨祿,水深泥濁。泥濁尚可,水深殺我!



 ──右一曲,晉《獨鹿舞歌》,六解,此是前一解。古辭《明君曲》後云:「勇安樂無慈,不問清與濁。清與無時濁,邪交與獨祿。」《伎祿》云:「求祿求祿,清白不濁。清白尚可,貪汙殺我!」晉歌為鹿字,古通用也。疑是風刺之辭。



 《碣石》辭:東臨礙石,以觀滄海。水河淡淡,山島竦峙。樹木叢生,百草豐茂。秋風蕭瑟,洪波湧起。日月之行,若出其
 中。星漢粲爛,若出其里。幸甚至哉!歌以言志。



 ──右一曲,魏武帝辭,晉以為《碣石舞歌》。詩四章,此是中一章。



 《淮南王》辭:淮南王,自言尊,百尺高樓與天連。我欲渡河河無梁,願作雙黃鵠,還故鄉。



 ──右一曲,晉《淮南王舞歌》。六解,前是第一,後是第五。



 《齊世昌》辭:齊世昌,四海安樂齊太平。人命長,當結久。千秋萬歲,皆老壽。



 ──右一曲,晉《杯槃歌》。十解,第三解云:「舞杯槃,何翩翩,舉坐翻覆壽萬年。」干寶云:「太康中有此舞。杯槃翻覆,至危之像。言晉世之士,茍貪飲食,智不及遠。」其第一解首句雲「晉
 世寧」,宋改為「宋世寧」。惡其杯槃翻覆,辭不復取。齊改為「齊世昌」。餘辭同後一。



 《公莫》辭:吾不見公莫時吾何嬰公來嬰姥時吾思君去時吾何零子以耶思君去時思來嬰吾云時母那何去吾。



 ──右一曲,晉《公莫舞歌》,二十章,無定句。前是第一解,後是第十九、二十解。雜有三句,並不可曉解。建武初,明帝奏樂至此曲,言是似《永明樂》,流涕憶世祖云。



 《白糸寧》辭:陽春白日風花香,趨步明月舞瑤裳。情發金石媚笙簧,羅袿徐轉紅袖揚。清歌流響繞鳳梁,如驚若思凝且翔。轉眄流精艷輝光,將流將引雙雁行。歡來何晚意何長,
 明君馭世永歌昌。



 ──右五曲,尚書令王儉造。《白糸寧歌》,周處《風土記》云:「吳黃龍中童謠云『行白者君追汝句驪馬』。後孫權征公孫淵,浮海乘舶,舶,白也。今歌和聲猶云『行白糸寧』焉。」



 《俳歌》辭:俳不言不語,呼俳噏所。俳適一起,狼率不止。生拔牛角,摩斷膚耳。馬無懸蹄,牛無上齒。駱駼無角,奮迅兩耳。



 ──右侏儒導舞人自歌之。古辭俳歌八曲,此是前一篇。二十二句,今侏儒所歌,擿取之也。



 角抵、像形、雜伎,歷代相承有也。其增損源起,事不可詳,大略漢世張衡《西京賦》是其始也。魏世則事見陳思王樂府《宴樂篇》,晉世則見傅玄《元正篇》、《朝會賦》。江左咸康中,罷紫鹿、
 跂行、鱉食、笮鼠、齊王卷衣、絕倒、五案等伎,中朝所無,見《起居注》,並莫知所由也。太元中,苻堅敗後,得關中簷橦胡伎,進太樂,今或有存亡,案此則可知矣。



 永明六年,赤城山雲霧開朗,見石橋瀑布,從來所罕睹也。山道士朱僧標以聞,上遣主書董仲民案視,以為神瑞。太樂令鄭義泰案孫興公賦造天台山伎,作莓苔、石橋、道士捫翠屏之狀,尋又省焉。



 皇齊啟運從瑤璣。靈鳳銜書集紫微。和樂既洽神所依。超商卷夏耀英輝。永世壽昌聲華飛。



 ──右《鳳皇銜書伎歌辭》,蓋魚龍之流也。元會日,侍中於殿前跪取其書。



 宋世辭云「大宋興隆膺靈符。鳳鳥感和銜素書。嘉樂之美通玄虛。惟新濟濟邁唐虞。



 巍巍蕩蕩道有餘」。齊初詔中書郎江淹改。



 《永平樂歌》者,竟陵王子良與諸文士造奏之。人為十曲。道人釋寶
 月辭頗美,上常被之管弦,而不列於樂官也。



 贊曰:綜採六代,和平八風。殷薦宴享,舞德歌功。



\end{pinyinscope}