\article{卷二十一列傳第二 文惠太子}

\begin{pinyinscope}

 文惠太子長懋,字雲喬,世祖長子也。世祖年未弱冠而生太子,為太祖所愛。



 姿容豐潤,小字白澤。宋元徽末,隨世祖在郢。世祖還鎮盆城拒沈攸之,使太子勞接將帥,親侍軍旅。除秘書郎,不拜。授輔國將軍,遷晉熙王撫軍主簿。事寧,世祖遣太子還都。太祖方創霸業,心存嫡嗣,謂太子曰:「汝還,吾事辦矣。」處之府東齋,令通文武賓客。敕荀伯玉曰:「我出行日,城中軍悉受長懋節度。我雖不行,內外直防及諸門甲兵,悉令長懋時時履行。」轉秘書丞,以與宣帝諱同,不就,改除中書郎,遷黃門侍郎,未拜。升明三年,太祖將受禪,世祖已還京師,以襄陽兵馬重鎮,不欲處他族,出太子為持節、
 都督雍梁二州、郢州之竟陵、司州之隨郡軍事、左中郎將、寧蠻校尉、雍州刺史。



 建元元年,封南郡王,邑二千戶。江左未有嫡皇孫封王,始自此也。進號征虜將軍。先是,梁州刺史范柏年誘降晉壽亡命李烏奴,討平氐賊楊城、蘇道熾等,頗著威名。沈攸之事起,柏年遣將陰廣宗領軍出魏興聲援京師,而候望形勢。事平,朝廷遣王玄邈代之。烏奴勸柏年據漢中不受命,柏年計未決,玄邈已至,柏年遲回魏興不肯下。太子慮其為變,乃遣說柏年,許啟為府長史,柏年乃進襄陽,因執誅之。柏年,梓潼人,徙居華陽,世為土豪,知名州里。宋泰始中,氐寇斷晉壽道,柏年以倉部郎假節領數百人慰勞通路,自益州道報命。除晉壽太守。討平氐賊,遂為梁州。柏年彊立,善言事,以應對為宋明帝所知。既被誅,巴西太守柳弘稱啟太祖,敕答曰:「柏年幸可不爾,為之恨恨!」時襄陽有盜發古塚者,
 相傳云是楚王塚,大獲寶物玉屐、玉屏風、竹簡書、青絲編。簡廣數分,長二尺,皮節如新。盜以把火自照,後人有得十餘簡,以示撫軍王僧虔,僧虔云是科斗書《考工記》,《周官》所闕文也。是時州遣按驗,頗得遺物,故有同異之論。會北虜南侵,上慮當出樊、沔。二年,徵為侍中、中軍將軍,置府,鎮石頭。



 穆妃薨,成服日,車駕出臨喪,朝議疑太子應出門迎。左僕射王儉曰:「尋《禮記·服問》『君所主夫人妻、太子、嫡婦』,言國君為此三人為主喪也。今鸞輿臨降,自以主喪而至,雖因事撫慰,義不在弔,南郡以下不應出門奉迎。但尊極所臨,禮有變革,權去杖絰,移立戶外,足表情敬,無煩止哭。皇太子既一宮之主,自應以車駕幸宮,依常奉候。既當成服之日,吉凶不容相干,宜以衰幘行事。望拜止哭,率由舊章。尊駕不以臨弔,奉迎則惟常體,求之情禮,如為可安。」解侍中。



 上以太子哀疾,不宜居石頭山障,移鎮西
 州。四年,遷使持節、都督南徐兗二州諸軍事、征北將軍、南徐州刺史。世祖即位,為皇太子。



 初,太祖好《左氏春秋》,太子承旨諷誦,以為口實。即正位東儲,善立名尚,禮接文士,畜養武人,皆親近左右,布在省闥。永明三年,於崇正殿講《孝經》,少傅王儉以擿句令太子僕周顒撰為義疏。



 五年冬,太子臨國學,親臨策試諸生,於坐問少傅王儉曰:「《曲禮》云『無不敬』。尋下之奉上,可以盡禮,上之接下,慈而非敬。今總同敬名,將不為昧?」



 儉曰:「鄭玄云『禮主於敬』,便當是尊卑所同。」太子曰:「若如來通,則忠惠可以一名,孝慈不須另稱。」儉曰:「尊卑號稱,不可悉同,愛敬之名,有時相次。



 忠惠之異,誠以聖旨,孝慈互舉,竊有徵據。《禮》云『不勝喪比於不慈不孝』,此則其義。」太子曰:「資敬奉君,資愛事親,兼此二塗,唯在一極。今乃移敬接下。豈復在三之義?」儉曰:「資敬奉君,必同至極,移敬逮下,不慢而已。」太子曰:「敬名雖同,
 深淺既異,而文無差別,彌復增疑。」儉曰:「繁文不可備設,略言深淺已見。《傳》云『不忘恭敬,民之主也」;《書》云『奉先思孝,接下思恭』。此又經典明文,互相起發。」太子問金紫光祿大夫張緒,緒曰:「愚謂恭敬是立身之本,尊卑所以並同。」太子曰:「敬雖立身之本,要非接下之稱。《尚書》云『惠鮮鰥寡』,何不言恭敬鰥寡邪?」緒曰:「今別言之,居然有恭惠之殊,總開記首,所以共同斯稱。」竟陵王子良曰:「禮者敬而已矣。自上及下,愚謂非嫌。」



 太子曰:「本不謂有嫌,正欲使言與事符,輕重有別耳。」臨川王映曰:「先舉必敬,以明大體,尊卑事數,備列後章,亦當不以總略而礙。」太子又以此義問諸學生,謝幾卿等十一人,並以筆對。



 太子問王儉曰:「《周易·乾卦》本施天位,而《說卦》云『帝出乎《震》』。



 《震》本非天,義豈相主?」儉曰:「《乾》健《震》動,天以運動為德,故言『帝出《震》』。」太子曰:「天以運動為德,君自體天居位,《震》雷為象,豈體天所出?」儉
 曰:「主器者莫若長子,故受之以《震》。萬物出乎《震》,故亦帝所與焉。」



 儉又諮太子曰:「《孝經》『仲尼居,曾子侍』。夫孝理弘深,大賢方盡其致,何故不授顏子,而寄曾生?」太子曰:「曾生雖德慚體二,而色養盡禮,去物尚近,接引非隔,弘宣規教,義在於此。」儉曰:「接引非隔,弘宣雖易,去聖轉遠,其事彌輕。既云『人能弘道』,將恐人輕道廢。」太子曰:「理既有在,不容以人廢言,而況中賢之才,弘上聖之教,寧有壅塞之嫌?」臨川王映諮曰:「孝為德本,常是所疑。德施萬善,孝由天性,自然之理,豈因積習?」太子曰:「不因積習而至,所以可為德本。」映曰:「率由斯至,不俟明德,大孝榮親,眾德光備,以此而言,豈得為本?」太子曰:「孝有深淺,德有小大,因其分而為本,何所稍疑?」



 太子以長年臨學,亦前代未有也。



 明年,上將訊丹陽所領囚,及南北二百里內獄,詔曰:「獄訟之重,政化所先。



 太子立年作貳,宜時詳覽,此訊事委以親決。」
 太子乃於玄圃園宣猷堂錄三署囚,原宥各有差。上晚年好遊宴,尚書曹事亦分送太子省視。



 太子與竟陵王子良俱好釋氏,立六疾館以養窮民。風韻甚和而性頗奢麗,宮內殿堂,皆雕飾精綺,過於上宮。開拓玄圃園,與臺城北塹等,其中樓觀塔宇,多聚奇石,妙極山水。慮上宮望見,乃傍門列修竹,內施高鄣,造游牆數百間,施諸機巧:宜須鄣蔽,須臾成立;若應毀撤,應手遷徙。善製珍玩之物,織孔雀毛為裘,光彩金翠,過於雉頭矣。以晉明帝為太子時立西池,乃啟世祖引前例,求東田起小苑,上許之。永明中,二宮兵力全實,太子使宮中將吏更番役築,宮城苑巷,制度之盛,觀者傾京師。



 上性雖嚴,多布耳目,太子所為,無敢啟者。後上幸豫章王宅,還過太子東田,見其彌亙華遠,莊麗極目,於是大怒,收監作主帥;太子懼,皆藏匿之,由是見責。



 太子素多疾,體又過壯,常在宮內,簡於遨
 遊。玩弄羽儀,多所僭疑,雖咫尺宮禁,而上終不知。



 十年,豫章王嶷薨,太子見上友于既至,造碑文奏之,未及鐫勒。十一年春正月,太子有疾,上自臨視,有憂色。疾篤,上表曰:「臣地屬元良,業微三善,光道樹風,於焉蓋闕,晨宵忷懼,有若臨淵。攝生舛和,構離痾疾,大漸惟幾,雇陰待謝。守器難永,視膳長違,仰戀慈顏,內懷感哽。竊惟死生定分,理不足悲,伏願割無已之悼,損既往之傷,寶衛聖躬,同休七百,臣雖沒九泉,無所遺恨。」時年三十六。



 太子年始過立,久在儲宮,得參政事;內外百司,咸謂旦暮繼體。及薨,朝野驚惋焉。上幸東宮,臨哭盡哀,詔斂以袞冕之服,謚曰文惠,葬崇安陵。世祖履行東宮,見太子服玩過制,大怒,敕有司隨事毀除,以東田殿堂為崇虛館。鬱林立,追尊為文帝,廟稱世宗。



 初,太子內懷惡明帝,密謂竟陵王子良曰:「我意色中殊不悅此人,當由其福德薄所致。」子良便苦
 救解。後明帝立,果大相誅害。



 史臣曰:上古之世,父不哭子。壽夭悠悠,尚嗟恆事。況夫正體東儲,方樹年德;重基累葉,載茂皇家;守器之君,已知耕稼,雖溫文具美,交弘盛跡,武運將終,先期夙殞,傳之幼少,以速顛危。推此而論,亦有冥數矣。



 贊曰:二象垂則,三星麗天。樹嫡惟長,義匪求賢。方為守器,植命不延。



\end{pinyinscope}