\article{卷二十八列傳第九 崔祖思 劉善明 蘇侃 垣榮祖}

\begin{pinyinscope}

 崔祖思,字敬元,清河東武城人,崔琰七世孫也。祖諲,宋冀州刺史。父僧護,州秀才。祖思少有志氣,好讀書史。初州避主簿,與刺史劉懷珍於堯廟祀神,廟有蘇侯像。懷珍曰:「堯聖人,而與雜神為列,欲去之,何如?」祖思曰:「蘇峻今日可謂四兇之五也。」懷珍遂令除諸雜神。



 太祖在淮陰,祖思聞風自結,為上輔國主簿,甚見親待,參豫謀議。除奉朝請,安成王撫軍行參軍,員外正員郎,冀州中正。宋朝初
 議封太祖為梁公,祖思啟太祖曰:「讖書云『金刀利刃齊刈之』。今宜稱齊,實應天命。」從之。轉為相國從事中郎,遷齊國內史。建元元年,轉長兼給事黃門侍郎。



 上初即位,祖思啟陳政事曰:「《禮誥》者,人倫之襟冕,帝王之樞柄。自古開物成務,必以教學為先。世不習學,民罔志義,悖競因斯而興,禍亂是焉而作。



 故篤俗昌治,莫先道教,不得以夷險革慮,儉泰移業。今無員之官,空受祿力。三載無考績之效,九年闕登黜之序。國儲以之虛匱,民力為之凋散。能否無章,涇渭混流。宜大廟之南,弘脩文序;司農以北,廣開武校。臺府州國,限外之職,問其所樂,依方課習,各盡其能。月供僮幹,如先充給。若有廢墮,遣還故郡。殊經奇藝,待以不次。士脩其業,必有異等,民識其利,能無勉勵?」



 又曰:「漢文集上書囊以為殿帷,身衣弋綈,以韋帶劍,慎夫人衣不曳地,惜中人十家之產,不為露臺。劉備取帳鉤銅鑄錢
 以充國用。魏武遣女,皂帳,婢十人;東阿婦以繡衣賜死,王景興以淅米見誚。宋武節儉過人,張妃房唯碧綃蚊幬,三齊穀席,五盞盤桃花米飯。殷仲文勸令畜伎,答云『我不解聲』。仲文曰『但畜自解』,又答『畏解,故不畜』。歷觀帝王,未嘗不以約素興,侈麗亡也。伏惟陛下,體唐城儉,踵虞為樸,寢殿則素木卑構,膳器則陶瓢充御。瓊簪玉箸,碎以為塵,珍裘繡服,焚之如草。斯實風高上代,民偃下世矣。然教信雖孚,氓染未革,宜加甄明,以速歸厚。詳察朝士,有柴車蓬館,高以殊等;雕墻華輪,卑其稱謂。馳禽荒色,長違清編,嗜音酣酒,守官不徙。物識義方,且懼且勸,則調風變俗,不俟終日。」



 又曰:「憲律之重,由來尚矣。故曹參去齊,唯以獄市為寄,餘無所言。路溫舒言『秦有十失,其一尚在,治獄之吏是也』。實宜清置廷尉,茂簡三官,寺丞獄主,彌重其選,研習律令,刪除繁苛。詔獄及兩縣,一月三訊,
 觀貌察情,欺枉必達。使明慎用刑,無忝大《易》;寧失不經,靡愧《周書》。漢來治律有家,子孫並世其業,聚徒講授,至數百人。故張、於二氏,潔譽文、宣之世;陳、郭兩族,流稱武、明之朝。決獄無冤,慶昌枝裔,槐袞相襲,蟬紫傳輝。今廷尉律生,乃令史門戶,族非咸、弘,庭缺于訓。刑之不措,抑此之由。如詳擇篤厚之土,使習律令,試簡有征,擢為廷尉僚屬。茍官世其家而不美其績,鮮矣;廢其職而欲善其事,未之有也。若劉累傳守其業,庖人不乏龍肝之饌,斷可知矣。」



 又曰:「樂者動天地,感鬼神,正情性,立人倫,其義大矣。按前漢編戶千萬,太樂伶官方八百二十九人,孔光等奏罷不合經法者四百四十一人,正樂定員,唯置三百八十八人。今戶口不能百萬,而太樂雅、鄭,元徽時校試千有餘人,後堂雜伎,不在其數,糜廢力役,傷敗風俗。今欲撥邪歸道,莫若罷雜伎,王庭唯置鐘虡、羽戚、登歌而已。如此,則官
 充給養,國反淳風矣。」



 又曰:「論儒者以德化為本,談法者以刻削為體。道教治世之粱肉,刑憲亂世之藥石。故以教化比雨露,名法方風霜。是以有恥且格,敬讓之樞紐;令行禁止,為國之關楗。然則天下治者,賞罰而已矣。賞不事豐,所病于不均;罰不在重,所困於不當。如令甲勛少,乙功多,賞甲而捨乙,天下必有不勸矣;丙罪重,丁眚輕,罰丁而赦丙,天下必有不悛矣。是賞罰空行,無當乎勸沮。將令見罰者寵習之臣,受賞者仇讎之士,戮一人而萬國懼,賞匹夫而四海悅。」



 又曰:「籍稅以厚國,國虛民貧;廣田以實廩,國富民贍。堯資用天之儲,實拯懷山之數;湯憑分地之積,以勝流金之運。近代魏置典農而中都足食,晉開汝、潁而汴河委儲。今將掃闢咸、華,題鏤龍漠,宜簡役敦農,開田廣稼。時罷山池之威禁,深抑豪右之兼擅,則兵民優贍,可以出師。」



 又曰:「古者左史記言,右史記事。
 故君舉必書,盡直筆而不污;上無妄動,知如絲之成綸。今者著作之官,起居而已;述事之徒,褒諛為體。世無董狐,書法必隱;時闕南史,直筆未聞。」



 又曰:「廢諫官,則聽納靡依。雖課勵朝僚,徵訪芻輿,莫若推舉質直,職思其憂。夫越任於事,在言為難,當官而行,處辭或易。物議既以無言望己,己亦當以吞默慚人。中丞雖謝咸、玄,未有全廢劾簡;廷尉誠非釋之,寧容都無訊牒!故知與其謬人,寧不廢職,目前之明效也。漢徵貢禹為諫大夫,矢言先策,夏侯勝狂直拘系,出補諷職,伐柯非遐,行之即善。」



 又曰:「天地無心,賦氣自均,寧得誕秀往古而獨寂寥一代!將在知與不知,用與不用耳。夫有賢而不知,知賢而不用,用賢而不委,委賢而不信,此四者,古今之通患也。今誠重郭隗而招劇辛,任鮑叔以求夷吾,則天下之士,不待召而自至矣。」上優詔報答。



 尋遷寧朔將軍、冠軍司馬,領齊郡太守、本官如故。是冬,
 虜動,遷冠軍將軍、軍主,屯淮上。二年,進號征虜將軍,軍主如故。仍遷假節、督青冀二州刺史,將軍如故。少時,卒。上嘆曰:「我方慾用祖思,不幸,可惜!」詔賻錢三萬,布五十匹。



 祖思宗人文仲,初辟州從事。泰始初,為薛安都平北主簿,拔難歸國。元徽初,從太祖於新亭拒桂陽賊,著誠效,除游擊將軍。沈攸之事起,助豫章王鎮東府,歷驃騎諮議,出為徐州刺史。建元初,封建陽縣子,三百戶。二年,虜攻鐘離,文仲擊破之。又遣軍主崔孝伯等過淮攻拔虜茬眉戍,殺戍主龍得侯及偽陽平太守郭杜羝、館陶令張德、濮陽令王明。時虜攻殺馬頭太守劉從,上曰:「破茬眉,足相補。」



 文仲又遣軍主陳靖攻虜竹邑戍主白仲都,又遣軍主崔延叔攻偽淮陽太守梁惡,並殺之。三年,淮北義民桓磊磈於抱犢固與虜戰,大破之。文仲馳啟,上敕曰:「北間起義者眾,深恐良會不再至,卿善獎沛中人,若能一時
 攘袂,當遣一佳將直入也。」



 文仲在政,為百姓所憚。除黃門郎,領越騎校尉,改封隨縣。嘗獻太祖纏鬚繩一枚,上為納受。永明元年,為太子左率,累至征虜將軍、冠軍司馬、汝陰太守。四年,卒。贈後將軍、徐州刺史。謚襄子。



 劉善明,平原人。鎮北將軍懷珍族弟也。父懷民,宋世為齊北海二郡太守。元嘉末,青州饑荒,人相食。善明家有積粟,躬食饘粥,開倉以救鄉里,多獲全濟,百姓呼其家田為「續命田」。



 少而靜處讀書,刺史杜驥聞名候之,辭不相見。年四十,刺史劉道隆辟為治中從事。父懷民謂善明曰:「我已知汝立身,復欲見汝立官也。」善明應辟。仍舉秀才。宋孝武見其對策強直,甚異之。



 泰始初,徐州刺史薛安都反,青州刺史沈文秀應之。時州治東陽城,善明家在郭內,不能自拔。伯父彌之詭說文秀求自效,文秀使領軍主張靈慶等五千援安都。



 彌之出門,密謂部曲曰:「
 始免禍坑矣。」行至下邳,起義背文秀。善明從伯懷恭為北海太守,據郡相應。善明密契收集門宗部曲,得三千人,夜斬關奔北海。族兄乘民又聚眾渤海以應朝廷。而彌之尋為薛安都所殺,明帝贈輔國將軍、青州刺史。



 以乘民為寧朔將軍、冀州刺史,善明為寧朔長史、北海太守,除尚書金部郎。乘民病卒,仍以善明為綏遠將軍、冀州刺史。文秀既降,除善明為屯騎校尉,出為海陵太守。郡境邊海,無樹木,善明課民種榆檟雜果,遂獲其利。還為後軍將軍、直閣。



 五年,青州沒虜,善明母陷北,虜移置桑乾。善明布衣蔬食,哀戚如持喪。明帝每見,為之歎息,時人稱之。轉寧朔將軍、巴西梓潼二郡太守。善明以母在虜中,不願西行,涕泣固請,見許。朝廷多哀善明心事。元徽初,遣北使,朝議令善明舉人,善明舉州鄉北平田惠紹使虜,贖得母還。



 幼主新立,群公秉政,
 善明獨結事太祖,委身歸誠。二年,出為輔國將軍、西海太守、行青冀二州刺史。至鎮,表請北伐,朝議不同。



 善明從弟僧副,與善明俱知名於州里。泰始初,虜暴淮北,僧副將部曲二千人東依海島;太祖在淮陰,壯其所為,召與相見,引為安成王撫軍參軍。蒼梧肆暴,太祖憂恐,常令僧副微行伺察聲論。使僧副密告善明及東海太守垣崇祖曰:「多人見勸北固廣陵,恐一旦動足,非為長算。今秋風行起,卿若能與垣東海微共動虜,則我諸計可立。」善明曰:「宋氏將亡,愚智所辨。故胡虜若動,反為公患。公神武世出,唯當靜以待之,因機奮發,功業自定。不可遠去根本,自貽猖蹶。」遣部曲健兒數十人隨僧副還詣領府,太祖納之。蒼梧廢,徵善明為冠軍將軍、太祖驃騎諮議、南東海太守、行南徐州事。



 沈攸之反,太祖深以為憂。善明獻計曰:「沈攸之控引八州,縱情蓄斂,收眾聚騎,營造舟仗,苞藏賊
 志,於焉十年。性既險躁,才非持重,而起逆累旬,遲回不進,豈應有所待也?一則暗於兵機,二則人情離怨,三則有掣肘之患,四則天奪其魄。本慮其剽勇,長於一戰,疑其輕速,掩襲未備。今六師齊備,諸侯同舉。昔謝晦失理,不鬥自潰;盧龍乖道,雖眾何施。且袁粲、劉秉,賊之根本,根本既滅,枝葉豈久?此是已籠之鳥耳。」事平,太祖召善明還都,謂之曰:「卿策沈攸之,雖復張良、陳平,適如此耳。」仍遷散騎常侍,領長水校尉,黃門郎,領後軍將軍、太尉右司馬。



 齊臺建,為右衛將軍,辭疾不拜。司空褚淵謂善明曰:「高尚之事,乃卿從來素意。今朝廷方相委待,詎得便學松、喬邪?」善明曰:「我本無宦情,既逢知己,所以戮力驅馳,願在申志。今天地廓清,朝盈濟濟,鄙懷既申,不敢昧於富貴矣。」



 太祖踐阼,以善明勳誠,欲與善明祿,召謂之曰:「淮南近畿,國之形勢,自非親賢,不使居之。卿為我臥治也!」代高宗
 為征虜將軍、淮南宣城二郡太守,遣使拜授,封新塗伯,邑五百戶。



 善明至郡,上表陳事曰:「周以三聖相資,再駕乃就;漢值海內無主,累敗方登;魏挾主行令,實逾二紀;晉廢立持權,遂歷四世。景祚攸集,如此之難者也。



 陛下凝輝自天,照湛神極,睿周萬品,道洽無垠。故能高嘯閑軒,鯨鯢自翦,垂拱雲帟,九服載晏,靡一戰之勞,無半辰之棘,苞池江海,籠苑嵩岱,神祇樂推,普天歸奉,二三年間,允膺寶命,胄臨皇歷,正位宸居。開闢以來,未有若斯之盛者也。夫常勝者無憂,恆成者好怠。故雖休勿休,姬旦作《誥》;安不忘危,尼父垂範。今皇運草創,萬化始基,乘宋季葉,政多澆苛,億北倒懸,仰齊蘇振。臣早蒙殊養,志輸肝血,徒有其誠,曾闕埃露。夙宵慚戰,如墜淵谷,不識忌諱,謹陳愚管,瞽言芻議,伏待斧皞。」所陳事凡十一條:其一以為「天地開創,人神慶仰,宜存問遠方,宣廣慈澤」;其二以為「京師
 浩大,遠近所歸,宜遣醫藥,問其疾苦,年九十以上及六疾不能自存者,隨宜量賜」;其三以為「宋氏赦令,蒙原者寡。愚謂今下赦書,宜令事實相副」;其四以為「匈奴未滅,劉昶猶存,秋風揚塵,容能送死,境上諸城,宜應嚴備,特簡雄略,以待事機,資實所須,皆宜豫辦」;其五以為「宜除宋氏大明泰始以來諸苛政細制,以崇簡易」;其六以為「凡諸土木之費,且可權停」;其七以為「帝子王姬,宜崇儉約」;其八以為「宜詔百官及府州郡縣,各貢讜言,以弘唐虞之美」;其九以為「忠貞孝悌,宜擢以殊階,清儉苦節,應授以民政」;其十以為「革命惟始,天地大慶,宜時擇才辨,北使匈奴」;其十一以為「交州險夐要荒之表,宋末政苛,遂至怨叛。今大化創始,宜懷以恩德,未應遠勞將士,搖動邊氓,且彼土所出,唯有珠寶,實非聖朝所須之急。討伐之事,謂宜且停」。



 又撰《賢聖雜語》奏之,托以諷諫。上答曰:「省所獻《雜語》,並
 列聖之明規,眾智之深軌。卿能憲章先範,纂鏤情識,忠款既昭,淵誠肅著,當以周旋,無忘聽覽也」。又諫起宣陽門;表陳宜明守宰賞罰;立學校,制齊禮;廣開賓館,以接荒民。上又答曰:「具卿忠讜之懷。夫賞罰以懲守宰,飾館以待遐荒,皆古之善政,吾所宜勉。更撰新禮,或非易制;國學之美,已敕公卿;宣陽門今敕停。寡德多闕,思復有聞」。



 善明身長七尺九寸,質素不好聲色,所居茅齋斧木而已,床榻几案,不加刬削。



 少與崔祖思友善,祖思出為青、冀二州,善明遺書曰:「昔時之游,於今邈矣。或攜手春林,或負杖秋澗,逐清風於林杪,追素月於園垂,如何故人,徂落殆盡。足下方擁旄北服,吾剖竹南甸,相去千里,間以江山,人生如寄,來會何時!嘗覽書史,數千年來,略在眼中矣。歷代參差,萬理同異。夫龍虎風雲之契,亂極必夷之幾,古今豈殊,此實一揆。日者沈攸之擁長蛇於外,粲、秉復為異
 識所推,唯有京鎮,創為聖基。遂乃擢吾為首佐,授吾以大郡,付吾關中,委吾留任。既不辦有抽劍兩城之用,橫槊搴旗之能,徒以挈瓶小智,名參佐命,常恐朝露一下,深恩不酬。



 憂深責重,轉不可據,還視生世,倍無次緒。藿羹布被,猶篤鄙好;惡色憎聲,暮齡尤甚。出蕃不與臺輔別,入國不與公卿遊,孤立天地之間,無猜無託,唯知奉主以忠,事親以孝,臨民以潔,居家以儉。足下今鳴笳舊鄉,衣繡故國,宋季荼毒之悲已蒙蘇泰,河朔倒懸之苦方須救拔。遣遊辯之士,為鄉導之使,輕裝啟行,經營舊壤,令泗上歸業,稷下還風,君欲誰讓邪?聊送諸心,敬申貧贈。」



 建元二年卒,年四十九,遺命薄殯。贈錢三萬,布五十匹。又詔曰:「善明忠誠夙亮,幹力兼宣,豫經夷險,勤績昭著。不幸殞喪,痛悼于懷。贈左將軍、豫州刺史,謚烈伯。」子滌嗣。善明家無遺儲,唯有書八千卷。太祖聞其清貧,賜滌家葛塘屯穀五
 百斛。



 善明從弟僧副,官至前將軍,封豐陽男,三百戶。永明四年,為巴西、梓潼二郡太守,卒。



 蘇侃,字休烈,武邑人也。祖護,本郡太守。父端,州治中。侃涉獵書傳,出身正員將軍,補長城令。薛安都反,引侃為其府參軍,使掌書記。安都降虜,侃自拔南歸。除積射將軍。遇太祖在淮上,便自委結。上鎮淮陰,以侃詳密,取為冠軍錄事參軍。是時張永、沈攸之敗後,新失淮北,始遣上北戍,不滿千人。每歲秋冬間,邊淮騷動,恆恐虜至。上廣遣偵候,安集荒餘,又營繕城府。上在兵中久,見疑於時,乃作《塞客吟》以喻志曰:「寶緯紊宗,神經越序。德晦河、晉,力宣江、楚。雲雷兆壯,天山繇武。直發指秦關,凝精越漢渚。秋風起,塞草衰,雕鴻思,邊馬悲。平原千里顧,但見轉蓬飛。星嚴海凈,月澈河明。清輝映幕,素液凝庭。



 金笳夜厲,羽轊晨征。斡晴潭而悵泗,枻松洲而悼情。蘭
 涵風而瀉艷,菊籠泉而散英。曲繞首燕之歎,吹軫絕越之聲。欷園琴之孤弄,想庭藿之餘馨。青關望斷,白日西斜。恬源靚霧,壟首輝霞。戒旋醿,躍還波,情綿綿而方遠,思裊裊而遂多。



 粵擊秦中之築,因為塞上之歌。歌曰:朝發兮江泉,日夕兮陵山。驚飆兮瀄汨,淮流兮潺湲。胡埃兮雲聚,楚旆兮星懸。愁墉兮思宇,惻愴兮何言。定寰中之逸鑒,審雕陵之迷泉。悟樊籠之或累,悵遐心以棲玄。」侃達上此旨,更自勤勵。委以府事,深見知待。



 元徽初,巴西人李承明作亂,太祖議遣侃銜使慰勞,還除羽林監,加建武將軍。



 桂陽之難,上復以侃為平南錄事,領軍主,從頓新亭,使分金銀賦賜諸將。事寧,除步兵校尉,出為綏虜將軍、山陽太守,清修有治理,百姓懷之。進號龍驤將軍,除前軍將軍。沈攸之事起,除侃游擊將軍,遷太祖驃騎咨議,領錄事,除黃門郎,復為太祖太尉咨議。



 侃事上既久,備悉起居,
 乃與丘巨源撰《蕭太尉記》,載上征伐之功。以功封新建縣侯,五百戶。齊臺建,為黃門郎,領射聲校尉,任以心膂。上即位,侃撰《聖皇瑞命記》一卷奏之。建元元年,卒,年五十三。上惜之甚至,追贈輔國將軍、梁南秦二州刺史,謚質侯。



 弟烈,字休文。初為東莞令,張永鎮軍中兵,累至山陽太守,寧朔將軍,游擊將軍。袁粲起事,太祖先遣烈助防城,仍隨諸將平石頭,封吉陽縣男。建元中,為假節、督巴州軍事、巴州刺史、巴東太守,寧朔將軍如故。永明中,至平西司馬、陳留太守,卒官。



 垣榮祖,字華先,下邳人,五兵尚書崇祖從父兄也。父諒之,宋北中郎府參軍。



 榮祖少學騎馬及射,或謂之曰:「武事可畏,何不學書?」榮祖曰:「昔曹操、曹丕上馬橫槊,下馬談論,此於天下可不負飲食矣。君輩無自全之伎,何異犬羊乎!」



 宋孝建中,州辟主簿,為後軍參軍。



 伯
 父豫州刺史護之子襲祖為淮陽太守,宋孝武以事徙之嶺南,護之不食而死。



 帝疾篤,又遣使殺襲祖。襲祖臨死,與榮祖書曰:「弟常勸我危行言遜,今果敗矣。」



 明帝初即位,四方反,除榮祖冗從僕射,遣還徐州說刺史薛安都曰:「天之所廢,誰能興之。使君今不同八百諸侯,如民所見,非計中也。」安都曰:「天命有在,今京都無百里地,莫論攻圍取勝,自可拍手笑殺。且我不欲負孝武。」榮祖曰:「孝武之行,足致餘殃。今雖天下雷同,正是速死,無能為也。」安都曰:「不知諸人云何,我不畏此。大蹄馬在近,急便作計。」榮祖被拘不得還,因收集部曲,為安都將領。假署冠軍將軍。安都引虜入彭城,榮祖攜家屬南奔朐山,虜遣騎追之不及。榮祖懼得罪,乃逃遁淮上。太祖在淮陰,榮祖歸附,上保持之。及明帝崩,太祖書送榮祖詣僕射褚淵,除寧朔將軍、東海太守。淵謂之曰:「蕭公稱卿幹略,故以此郡相處。」



 榮祖善彈,彈鳥毛盡而鳥不死。海鵠群翔,榮祖登城西樓彈之,無不折翅而下。



 除晉熙王征虜、安成王車騎中兵,左軍將軍。元徽末,太祖欲渡廣陵,榮祖諫曰:「領府去臺百步,公走,人豈不知?若單行輕騎,廣陵人一旦閉門不相受,公欲何之?公今動足下床,便恐即有扣臺門者,公事去矣。」及蒼梧廢,除寧朔將軍、淮南太守,進輔國將軍,除游擊將軍、太祖驃騎諮議,輔國將軍、西中郎司馬、汝陰太守,除冠軍將軍,給事中,驍騎將軍。豫佐命勳,封將樂縣子,三百戶,以其祖舊封封之。出為持節、督青冀二州刺史,冠軍如故。遷黃門郎。



 永明二年,為冠軍將軍、尋陽相、南新蔡太守。作大形棺材盛仗,使鄉人田天生、王道期載渡江北。監奴有罪,告之,有司奏免官削爵付東冶,案驗無實見原。



 為安陸王平西諮議,帶江陵令,仍遷司馬、河東內史。遷持節、督緣淮諸軍事、冠軍將軍、兗州刺史,領東平太守、
 兗州大中正。



 巴東王子響事,方鎮皆啟稱子響為逆,榮祖曰:「此非所宜言。政應云劉寅等孤負恩獎,逼迫巴東,使至於此。」時諸啟皆不得通,事平後,上乃省視,以榮祖為知言。九年,卒,年五十七。



 從父閎,宋孝建初,為威遠將軍、汝南新蔡太守,據梁山拒丞相義宣賊,以功封西都縣子。累遷龍驤將軍、司州刺史。義嘉事起,明帝使閎出守盱眙,領兵北討薛道標破之。封樂鄉縣男,三百戶。升明初,為散騎常侍,領長水校尉,與豫章王對直殿省,遷右衛將軍。太祖即位,以心誠封爵如舊,加給事中,領驍騎將軍。累遷金紫光祿大夫。年七十六,永明五年,卒,謚定。



 榮祖從弟歷生,亦為驍騎將軍。宋泰始初,薛安都反,以女婿裴祖隆為下邳太守,歷生時請假還北,謀殺祖隆,舉城應朝廷。事發奔走。歷官太子右率。性苛暴,好行鞭捶。與始安王遙光同反,伏誅。



 史臣曰:太祖作牧淮、兗,始基霸業,恩成北被,感動三齊。青、冀豪右,崔、劉望族,先睹人雄,希風結義。夫諫江都之略,似任光之言,雖議不獨興,理成合契,蓋帷幕之臣也。



 贊曰:淮鎮北州,獲在崔、劉。獻書上議,帝念忠謀。侃奉潛躍,皇瑞是鳩。



 垣方帶礪,削免虛尤。



\end{pinyinscope}