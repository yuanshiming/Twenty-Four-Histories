\article{卷五十三列傳第三十四 良政}

\begin{pinyinscope}

 傅琰虞願劉懷慰裴昭明沈憲李珪孔琇之太祖承宋氏奢縱,風移百城,輔立幼主,思振民瘼。為政未期,擢山陰令傅琰為益州刺史。乃捐華反樸,恭己南面,導民以躬,意存勿
 擾。以山陰大邑,獄訟繁滋,建元三年別置獄丞,與建康為比。永明繼運,垂心治術。杖威善斷,猶多漏網,長吏犯法,封刃行誅。郡縣居職,以三周為小滿。水旱之災,輒加賑恤。明帝自在布衣,曉達吏事,君臨意兆,專務刀筆,未嘗枉法申恩,守宰以之肅震。



 永明之世十許年中,百姓無雞鳴犬吠之警,都邑之盛,士女富逸,歌聲舞節,袨服華妝,桃花綠水之間,秋月春風之下,蓋以百數。及建武之興,虜難猋急,征役連歲,不遑啟居,軍國糜耗,從此衰矣。



 齊世善政著名表績無幾焉,位次遷升,非直止乎城邑。今取其清察有跡者,餘則隨以附焉。



 傅琰,字季珪,北地靈州人也。祖邵,員外郎。父僧祐,安東錄事參軍。琰美姿儀,解褐寧蠻參軍,本州主簿,寧蠻功曹。宋永光元年,補諸暨武康令,廣威將軍,除尚書左民郎,又為武康令,將軍如故。除吳
 興郡丞。泰始六年,遷山陰令。



 山陰,東土大縣,難為長官,僧祐在縣有稱,琰尤明察,又著能名。其年爵新亭侯。



 元徽初,遷尚書右丞。



 遭母喪,居南岸,鄰家失火,延燒琰屋,琰抱柩不動,鄰人競來赴救,乃得俱全。琰股髀之間,已被煙焰。服闋,除邵陵王左軍諮議,江夏王錄事參軍。



 太祖輔政,以山陰獄訟煩積,復以琰為山陰令。賣針賣糖老姥爭團絲,來詣琰,琰不辨核,縛團絲於柱鞭之,密視有鐵屑,乃罰賣糖者。二野父爭雞,琰各問「何以食雞」。一人云「粟」,一人云「豆」,乃破雞得粟,罪言豆者。縣內稱神明,無敢復為偷盜。琰父子並著奇績,江左鮮有。世云諸傅有《治縣譜》,子孫相傳,不以示人。



 昇明二年,太祖擢為假節、督益寧二州軍事、建威將軍、益州刺史、宋寧太守。



 建元元年,進號寧朔將軍。四年,徵驍騎將軍,黃門郎。永明二年,遷建威將軍、安陸王北中郎長史,改寧朔將軍。明年,徙廬陵王安西長
 史、南郡內史,行荊州事。



 五年,卒。琰喪西還,有詔出臨。



 臨淮劉玄明亦有吏能,為山陰令,大著名績。琰子翽問之,玄明曰:「我臨去當告卿。」將別,謂之曰:「作縣唯日食一升飰,而莫飲酒。」



 虞愿,字士恭,會稽餘姚人也。祖賚,給事中,監利侯。父望之,早卒。賚中庭橘樹冬熟,子孫競來取之,願年數歲,獨不取,賚及家人皆異之。元嘉末為國子生,再遷湘東王國常侍,轉潯陽王府墨曹參軍。明帝立,以愿儒吏學涉,兼蕃國舊恩,意遇甚厚。除太常丞,尚書祠部郎,通直散騎侍郎,領五郡中正,祠部郎如故。



 帝性猜忌,體肥憎風,夏月常著皮小衣,拜左右二人為司風令史,風起方面,輒先啟聞。星文災變,不信太史,不聽外奏,敕靈臺知星二人給愿,常直內省,有異先啟,以相檢察。



 帝以故宅起湘宮寺,費極奢侈。以孝武莊嚴剎七層,帝欲起十層,不可立,分為兩剎,各五層。新安太守巢尚
 之罷郡還,見帝,曰:「卿至湘宮寺未?我起此寺,是大功德。」愿在側曰:「陛下起此寺,皆是百姓賣兒貼婦錢,佛若有知,當悲哭哀愍。罪高佛圖,有何功德?」尚書令袁粲在坐,為之失色。帝乃怒,使人驅下殿,願徐去無異容。以舊恩,少日中,已復召入。



 帝好圍棋,甚拙,去格七八道,物議共欺為第三品。與第一品王抗圍棋,依品賭戲,抗每饒借之,曰:「皇帝飛棋,臣抗不能斷。」帝終不覺,以為信然,好之愈篤。願又曰:「堯以此教丹朱,非人主所宜好也。」雖數忤旨,而蒙賞賜猶異餘人。遷兼中書郎。



 帝寢疾,愿常侍醫藥。帝素能食,尤好逐夷,以銀缽盛蜜漬之,一食數缽。謂揚州刺史王景文曰:「此是奇味,卿頗足不?」景文曰:「臣夙好此物,貧素致之甚難。」帝甚悅。食逐夷積多,胸腹痞脹,氣將絕。左右啟飲數升酢酒,乃消。疾大困,一食汁滓猶至三升,水患積久,藥不復效。大漸日,正坐,呼道人,合掌便絕。願以侍疾
 久,轉正員郎。



 出為晉平太守,在郡不治生產。前政與民交關,質錄其兒婦,愿遣人於道奪取將還。在郡立學堂教授。郡舊出髯蛇膽,可為藥,有餉愿蛇者,願不忍殺,放二十里外山中,一夜蛇還床下。復送四十里外山,經宿,復還故處。願更令遠,乃不復歸,論者以為仁心所致也。海邊有越王石,常隱雲霧。相傳云「清廉太守乃得見」,愿往觀視,清徹無隱蔽。後瑯邪王秀之為郡,與朝士書曰:「此郡承虞公之後,善政猶存,遺風易遵,差得無事。」以母老解職,除後軍將軍。褚淵常詣願,不在,見其眠床上積塵埃,有書數帙。淵歎曰:「虞君之清,一至於此。」令人掃地拂床而去。



 遷中書郎,領東觀祭酒。兄季為上虞令,卒,愿從省步還家,不待詔便歸東。



 除驍騎將軍,遷廷尉,祭酒如故。愿嘗事宋明帝,齊初宋神主遷汝陰廟,愿拜辭流涕。建元元年卒,年五十四。願著《五經論問》,撰《會稽記》,文翰數十篇。



 劉
 懷慰,字彥泰,平原平原人也。祖奉伯,元嘉中為冠軍長史。父乘民,冀州刺史。懷慰初為桂陽王征北板行參軍。乘民死於義嘉事難,懷慰持喪,不食醯醬,冬月不絮衣。養孤弟妹,事寡叔母,皆有恩義。復除邵陵王南中郎參軍,廣德令,尚書駕部郎。懷慰宗從善明等為太祖心腹,懷慰亦豫焉。沈攸之有舊,令為書戒喻攸之,太祖省之稱善。除步兵校尉。



 齊國建,上欲置齊郡於京邑,議者以江右土沃,流民所歸,乃治瓜步,以懷慰為輔國將軍、齊郡太守。上謂懷慰曰:「齊邦是王業所基,吾方以為顯任。經理之事,一以委卿。」又手敕曰:「有文事者,必有武備。今賜卿玉環刀一口。」懷慰至郡,修治城郭,安集居民,墾廢田二百頃,決沈湖灌溉。不受禮謁,民有餉其新米一斛者,懷慰出所食麥飯示之,曰:「旦食有餘,幸不煩此。」因著《廉吏論》以達其意。太祖聞之,手敕褒賞。進督秦、沛二郡。妻子在都,賜米
 三百斛。兗州刺史柳世隆與懷慰書曰:「膠東流化,潁川致美,以今方古,曾何足云。」在郡二年,遷正員郎,領青冀二州中正。



 懷慰本名聞慰,世祖即位,以與舅氏名同,敕改之。出監東陽郡,為吏民所安。



 還兼安陸王北中郎司馬。永明九年卒,年四十五。明帝即位,謂僕射徐孝嗣曰:「劉懷慰若在,朝廷不憂無清吏也。」懷慰與濟陽江淹、陳郡袁彖善,亦著文翰。



 永明初,獻《皇德論》云。



 裴昭明,河東聞喜人,宋太中大夫松之孫也。父駰,南中郎參軍。昭明少傳儒史之業,泰始中,為太學博士。有司奏:「太子婚,納徵用玉璧虎皮,未詳何所準據。」昭明議:「禮納徵,儷皮為庭實,鹿皮也。晉太子納妃注『以虎皮二』。太元中,公主納徵,虎豹皮各一。豈其謂婚禮不詳。王公之差,故取虎豹文蔚以尊其事。虎豹雖文,而徵禮所不言;熊羆雖古,而婚禮所不及;珪璋雖美,或為用各異。



 今宜准的經
 誥。凡諸僻謬,一皆詳正。」於是有司參議,加圭璋,豹熊羆皮各二。



 元徽中,出為長沙郡丞,罷任,刺史王蘊謂之曰:「卿清貧,必無還資。湘中人士有須一禮之命者,我不愛也。」昭明曰:「下官忝為邦佐,不能光益上府,豈以鴻都之事仰累清風。」歷祠部通直郎。



 永明三年使虜,世祖謂之曰:「以卿有將命之才,使還,當以一郡相賞。」還為始安內史。郡民龔玄宣云神人與其玉印玉板書,不須筆,吹紙便成字,自稱「龔聖人」,以此惑眾。前後郡守敬事之,昭明付獄治罪。及還,甚貧罄。世祖曰:「裴昭明罷郡還,遂無宅。我不諳書,不知古人中誰比?」遷射聲校尉。九年,復遣北使。



 建武初為王玄邈安北長史、廣陵太守。明帝以其在事無所啟奏,代還,責之。



 昭明曰:「臣不欲競執關楗故耳。」昭明歷郡皆有勤績,常謂人曰:「人生何事須聚蓄,一身之外,亦復何須?子孫若不才,我聚彼散;若能自立,則不如一經。」



 故終身
 不治產業。中興二年卒。



 從祖弟顗,字彥齊。少有異操。泰始中於總明觀聽講,不讓劉秉席,秉用為參軍。昇明末,為奉朝請。齊臺建,世子裴妃須外戚譜,顗不與,遂分籍。太祖受禪,上表誹謗,掛冠去,伏誅。



 沈憲,字彥璋,吳興武康人也。祖說道,巴西梓潼二郡太守,父璞之,北中郎行參軍。憲初應州辟,為主簿。少有幹局,歷臨首、餘杭令,巴陵王府佐,帶襄令,除駕部郎。宋明帝與憲棋,謂憲曰:「卿,廣州刺史才也。」補烏程令,甚著政績。



 太守褚淵嘆之曰:「此人方員可施。」除通直郎,都水使者。長於吏事,居官有績。



 除正員郎,補吳令,尚書左丞。



 升明二年,西中郎將晃為豫州,太祖擢憲為晃長史,南梁太守,行州事。遷豫章王諮議,未拜,坐事免官。復除安成王冠軍、武陵王征虜參軍,遷少府卿。少府管掌市易,與民交關,有吏能者皆更此職。
 遷王儉鎮軍長史。



 武陵王曄為會稽,以憲為左軍司馬。太祖以山陰戶眾難治,欲分為兩縣。世祖啟曰:「縣豈不可治,但用不得其人耳。」乃以憲帶山陰令,政聲大著。孔稚珪請假東歸,謂人曰:「沈令料事特有天才。」加寧朔將軍。王敬則為會稽,憲仍留為鎮軍長史,令如故。



 遷為冠軍長史,行南豫州事,晉安王後軍長史、廣陵太守。西陽王子明代為南兗州,憲仍留為冠軍長史,太守如故,頻行州府事。永明八年,子明典簽劉道濟取府州五十人役自給,又役子明左右,及船仗贓私百萬,為有司所奏,世祖怒,賜道濟死。憲坐不糾,免官。尋復為長史、輔國將軍,以疾去官。除散騎常侍,未拜,卒。當世稱為良吏。



 憲同郡丘仲起,先是為晉平郡,清廉自立。褚淵嘆曰:「見可欲心能不亂,此楊公所以遺子孫也。」仲起字子震,少為憲從伯領軍寅之所知。宋元徽中,為太子領軍長史,官至廷尉。卒。



 李圭之,字孔璋,江夏鐘武人也。父祖皆為縣令。圭之少闢州從事。宋泰始初,蔡興宗為郢州,以圭之為安西府佐,委以職事,清治見知。遷鎮西中郎諮議,右軍將軍,兼都水使者。圭之歷職稱為清能,除游擊將軍,兼使者如故。轉兼少府,卒。



 先是,四年,滎陽毛惠素為少府卿,吏才強而治事清刻。敕市銅官碧青一千二百斤供御畫,用錢六十萬。有讒惠素納利者,世祖怒,敕尚書評賈,貴二十八萬餘,有司奏之,伏誅。死後家徒四壁,上甚悔恨。



 孔琇之,會稽山陰人也。祖季恭,光祿大夫,父靈運,著作郎。琇之初為國子生,舉孝廉。除衛軍行參軍,員外郎,尚書三公郎。出為烏程令,有吏能。還遷通直郎,補吳令。有小兒年十歲,偷刈鄰家稻一束,琇之付獄治罪。或諫之,琇之曰:「十歲便能為盜,長大何所不為?」縣中皆震肅。



 遷尚書左丞,又以職事知名。轉前軍將軍,兼少府。遷驍騎將軍,少府如故。



 出為寧朔將軍、高宗冠軍征虜長史、江夏內史。
 還為正員常侍,兼左民尚書、廷尉卿。出為臨海太守,在任清約,罷郡還,獻乾姜二十斤,世祖嫌少,及知琇之清,乃嘆息。除武陵王前軍長史,未拜,仍出為輔國將軍,監吳興郡,尋拜太守,治稱清嚴。



 高宗輔政,防制諸蕃,致密旨於上佐。隆昌元年,遷琇之為寧朔將軍、晉熙王冠軍長史,行郢州事,江夏內史。琇之辭,不許。未拜,卒。



 史臣曰:琴瑟不調,必解而更張也。魏晉為吏,稍與漢乖,苛猛之風雖衰,而仁愛之情亦減。局以峻法,限以常條,以必世之仁未及宣理,而期月之望已求治術。



 先公後私,在己未易;割民奉國,於物非難;期之救過,所利茍免。且目見可欲,嗜好方流,貪以敗官,取與違義,吏之不臧,罔非由此。擿奸辯偽,誠俟異識,垂名著績,唯有廉平。今世之治民,未有出於此也。



 贊曰:蒸蒸小民,吏職長親。棼亂須理,恤隱歸仁。枉直交瞀,寬猛代
 陳。伊何導物,貴在清身。



\end{pinyinscope}