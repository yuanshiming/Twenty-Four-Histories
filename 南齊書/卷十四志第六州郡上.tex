\article{卷十四志第六州郡上}

\begin{pinyinscope}

 揚南徐豫南豫南兗北兗北徐青冀江廣交越揚州,京輦神皋。漢、魏刺史鎮壽春,吳置持節督州牧八人,不見揚州都督所治。晉太康元年,吳平,刺史周浚始鎮江南。元帝為都督,渡江左,遂成帝畿,望實隆重。領郡如左:丹陽郡建康秣陵丹陽溧陽永世湖熟江寧句容會稽郡山陰永興上虞餘姚諸暨剡鄞始寧
 句章鄮吳郡吳婁海虞嘉興海鹽錢唐富陽鹽官新城建德壽昌桐廬吳興郡烏程武康餘杭東遷長城於潛臨安故鄣安吉原鄉東陽郡長山太末烏傷永康信安吳寧豐安定陽遂昌新安郡始新黟遂安歙海寧臨
 海郡章安臨海寧海始豐樂安永嘉郡永寧安固松陽橫陽樂成南徐州,鎮京口。吳置幽州牧,屯兵在焉。丹徒水道入通吳會,孫權初鎮之。《爾雅》曰:「絕高為京。」今京城因山為壘,望海臨江,緣江為境,似河內郡,內鎮優重。宋氏以來,桑梓帝宅,江左流寓,多出膏腴。領郡如左:南東海郡郯祝其襄賁利成西隰丹徒武進晉陵郡晉陵無錫延陵曲阿暨陽南沙海陽
 義興郡永明二年,割屬揚州,後復舊。



 陽羨臨津國山義鄉綏安南琅邪郡本治金城,永明徙治白下。



 臨沂江乘蘭陵承建武三年省譙建元二年,平陽郡流民在臨江郡者,立宣祚縣,尋改為譙。永明元年,省懷化一縣並屬。



 臨淮郡自此以下,郡無實土。



 海西射陽凌淮陰東陽淮浦建武二年省。



 淮陵郡司吾武陽建武三年,省泰山郡屬。甄城陽樂徐建武三年省
 南東莞郡東莞莒姑幕建武三年省。



 南清河郡南徐州領冀州東武城清河貝丘繹幕建武二年省。



 南彭城郡彭城武原傅陽蕃薛開陽洨僮下邳建武三年省呂建武四年省杼秋建武四年省北陵建武四年省南高平郡宋太始五年僑置,初寄治淮陰,復徙淮南當塗。二縣僑屬南豫,後屬南徐。



 金饗高平南濟陰郡城武單父城陽建武三年省。



 南濮陽郡廩丘東燕會鄄城建武三年,省濟陽郡度屬。榆次建武二年省南魯郡建武二年省魯樊西安建武
 二年省南平昌郡建武三年省安丘郡省,屬東莞。



 南泰山郡建武三年省南城郡省,度屬平昌,尋又省。廣平南濟陽郡建武三年省考城郡省,度屬魯,尋又省。



 豫州。晉元帝永昌元年,刺史祖約避胡賊,自譙還治壽春。壽春,淮南一都之會,地方千餘里,有陂田之饒。漢、魏以來揚州刺史所治,北拒淮水,《禹貢》云「淮海惟揚州」也。咸和四年,祖約以城降胡,復以庾亮為刺史,治蕪湖。蕪湖,浦水南入,亦為險奧。劉備謂孫權曰:「江東先有建業,次有蕪湖。」庾亮經略中原,以毛寶為刺史,治邾城,為胡所覆。荊州刺史庾翼領州,在武昌。諸郡失土荒民數千無佃業,
 翼表移西陽、新蔡二郡荒民就陂田於尋陽。穆帝永和五年,胡偽揚州刺史王浹以壽春降。而刺史或治歷陽,進馬頭及譙,不復歸舊鎮也。哀帝隆和元年,袁真還壽春。



 真為桓溫所滅,溫以子熙為刺史,戍歷陽。孝武寧康元年,桓沖移姑熟,以邊寇未靜,分割譙、梁二郡見民,置之浣川,立為南譙、梁郡。十二年,桓石虔還歷陽。庾準為刺史,表省諸權置,皆還如本。義熙二年,劉毅復鎮姑熟,上表曰:「忝任此州,地不為曠,西界荒餘,密邇寇虜,北垂蕭條,土氣彊獷,民不識義,唯戰是習。逋逃不逞,不謀日會。比年以來,無月不戰,實非空乏所能獨撫。請輔國將軍張暢領淮南、安豐、梁國三郡。」時豫州邊荒,至乃如此。十二年,劉義慶鎮壽春,後常為州治。撫接遐荒,捍禦疆埸。領郡如左:南汝陰郡建元二年罷南陳
 左郡二縣並慎汝陰宋安陽和城南頓陽夏宋丘《永元元年地志》無樊《永元志》無鄭《永元志》無東宋《永元志》無南陳左縣《永元志》無邊水《永元志》無晉熙郡新冶陰安懷寧南樓煩齊興太湖左縣潁川郡臨潁邵陵南許昌《永元志》無曲陽汝陽郡武津汝陽梁郡《永元元年地志》,南梁郡領睢陽、新汲、陳、蒙、崇義五縣。



 北譙梁蒙城父《永元志》屬南譙北陳郡陽夏西華萇平項
 陳留郡浚儀小黃雍丘南頓郡《永元元年地志》無和城南頓西南頓郡寄治州,《永元元年地志》無西南頓和城譙平鄉北梁郡《永元元年地志》無北蒙北陳西汝陰郡樓煩汝陰宋陳《永元志》無平豫《永元志》無固始《永元志》無新蔡《永元志》無汝南《永元志》無安城
 北譙郡寧陵譙蘄《永元志》屬南譙汝南郡《永元元年地志》無瞿陽安城上蔡北新蔡郡鮦陽新蔡固始苞信弋陽郡期思南新息弋陽上蔡平輿陳郡南陳萇平《永元志》無項《永元志》無西華《永元志》無陽夏《永元志》無安豐郡雩婁新化史水扶陽開化邊城松滋《永元志》屬北新蔡安豐
 光城左郡樂安光城茹田邊城郡《永元元年地志》無建寧郡陽城建寧齊昌郡陽塘保城齊昌永興右三郡,永明四年割郢州屬。



 南豫州。晉寧康元年,豫州刺史桓沖始鎮姑熟,後遷徙,見《晉書》。宋永初二年,分淮東為南豫州,治歷陽,而淮西為豫州。元嘉七年省並。大明元年復置,治姑熟。泰始二年治歷陽,三年治宣城,五年省。淮西沒虜,七年,復分淮東置南豫。建元二年,太祖以西豫吏民寡
 刻,分置兩州,損費甚多,省南豫。左僕射王儉啟:「愚意政以江西連接汝、潁,土曠民希。匈奴越逸,唯以壽春為阻。



 若使州任得才,虜動要有聲聞,豫設防禦,此則不俟南豫。假令或慮一失,醜羯之來,聲不先聞,胡馬倏至,壽陽嬰城固守不能斷其路,朝廷遺軍歷陽,已當不得先機。戎車初戒,每事草創,孰與方鎮常居,軍府素正。臨時配助,所益實少。安不忘危,古之善政。所以江左屢分南豫,意亦可求。如聞西豫力役尚復粗可,今得南譙等郡,民戶益薄,於其實益,復何足云。」太祖不從。永明二年,割揚州宣城、淮南,豫州歷陽、譙、廬江、臨江六郡,復置南豫州。四年,冠軍長史沈憲啟:「二豫分置,以桑堁子亭為斷。潁川、汝陽在南譙、歷陽界內,悉屬西豫,廬江居晉熙、汝陰之中,屬南豫。求以潁川、汝陽屬南豫,廬江還西豫。」七年,南豫州別駕殷瀰稱:「潁川、汝陽,荒殘來久,流民分散在譙、歷二境,多蒙復
 除,獲有郡名,租輸益微,府州絕無將吏,空受名領,終無實益。但寄治譙、歷,於方斷之宜,實應屬南豫。二豫亟經分置,廬江屬南豫,濱帶長江,與南譙接境,民黎租帛,從流送州,實為便利,遠逾西豫,非其所願,郡領灊舒及始新左縣,村竹產,府州採伐,為益不少。府州新創,異於舊藩。資役多闕,實希得廬江。請依昔分置。」尚書參議:「往年慮邊塵須實,故啟迴換。今淮、泗無虞,宜許所牒。」詔「可」。領郡如左:淮南郡于湖永明八年,省角城、高平、下邳三縣並。繁昌當塗浚遒定陵襄垣宣城郡廣德懷安宛陵廣陽石城臨城寧國宣城建元涇安吳歷陽郡
 歷陽龍亢雍丘南譙郡山桑蘄北許昌《永元志》無扶陽曲陽嘉平廬江郡舒建元二年為郡治灊始新和城《永元志》無西華《永元志》無呂亭左縣建元二年,割晉熙屬。譙建元二年,割南譙屬。



 臨江郡建元二年,罷並歷陽,後復置。



 烏江懷德酂南兗州,鎮廣陵,漢故王國。有江都浦水,魏文帝伐吳出此,見江濤盛壯,歎云:「天所以限南北也。」晉元帝過江,建興四年,揚聲北討,遣宣城公裒督徐、兗二州,鎮廣陵。其後或還江南,然立鎮自此始也。時百姓遭難,流移此境,流民多庇大姓以為客。元帝太興四年,詔
 以流民失籍,使條名上有司,為給客制度,而江北荒殘,不可檢實。明帝太寧三年,郗鑒為兗州,鎮廣陵,後還京口。是後兗州或治盱眙,或治山陽。桓玄以桓弘為青州,鎮廣陵。義熙二年,諸葛長民為青州,徙山陽。時鮮卑接境,長民表云:「此蕃十載釁故相襲,城池崩毀,荒舊散伏,邊疆諸戍,不聞雞犬。且犬羊侵暴,抄掠滋甚。」乃還鎮京口。晉末以廣陵控接三齊,故青、兗同鎮。宋永初元年,罷青并兗。三年,檀道濟始為南兗州,廣陵因此為州鎮。土甚平曠,刺史每以秋月多出海陵觀濤,與京口對岸,江之壯闊處也。永明元年,刺史柳世隆奏:「尚書符下土斷條格,并省僑郡縣。凡諸流寓,本無定憩,十家五落,各自星處。



 一縣之民,散在州境,西至淮畔,東屆海隅。今專罷僑邦,不省荒邑,雜居舛止,與先不異。離為區斷,無革游濫。謂應同省,隨堺并帖。若鄉屯里聚,二三百家,井甸可脩,區域易分者,
 別詳立。」於是濟陰郡六縣,下邳郡四縣,淮陽郡三縣,東莞郡四縣,以散居無實土,官長無廨舍,寄止民村,及州治立,見省,民戶帖屬。領郡如左:廣陵郡建元四年,罷北淮陽、北下邳、北濟陰、東莞四郡並。



 海陵廣陵高郵江都齊寧永明元年置海陵郡建陵寧海如皋臨江蒲濤臨澤齊昌永明元年置海安永明五年罷新郡,並此縣度屬。



 山陽郡東城山陽鹽城左鄉盱眙郡考城盱眙陽城直瀆長樂
 南沛郡沛蕭相北兗州,鎮淮陰。《地理志》云淮陰縣屬臨淮郡,《郡國志》屬下邳國,《晉太康地記》屬廣陵郡。



 穆帝永和中,北中郎將荀羨北討鮮卑,云「淮陰舊鎮,地形都要,水陸交通,易以觀釁。沃野有開殖之利,方舟運漕,無他屯阻。」乃營立城池。宋泰始二年失淮北,於此立州鎮。建元四年,移鎮盱眙,仍領盱眙郡。舊北對清泗,臨淮守險,有陽平石鱉,田稻豐饒。所領唯陽平一郡,永明七年,光祿大夫呂安國啟稱:「北兗州民戴尚伯六十人訴『舊壤幽隔,飄寓失所,今雖創置淮陰,而陽平一郡,州無實土,寄山陽境內。



 竊見司、徐、青三州,悉皆新立,並有實郡。東平既是望邦,衣冠所係。希於山陽、盱眙二界間割小戶置此郡,始招集荒落,使本壤族姓,有所歸依。』臣尋東平郡既是此
 州本領,臣賤族桑梓,願立此邦。」見許。領郡如左:陽平郡寄治山陽泰清永陽安宜豐國東平郡壽張割山陽官瀆以西三百戶置淮安割直瀆、破釜以東,淮陰鎮下流雜一百戶置。



 高平郡濟北郡泰山郡新平郡魯郡右荒。



 北徐州,鎮鐘離。《漢志》鍾離縣屬九江郡,《晉太康二年起居注》置淮
 南鐘離,未詳此前所省令。



 《晉地記》屬淮南郡。宋泰始末年屬南兗。元徽元年置州,割為州治,防鎮緣淮。永明元年,省北徐譙、梁、魏、陽平、彭城五郡。領郡如左:鍾離郡燕縣郡治朝歌虞永明元年,割馬頭屬。零永明元年,割馬頭屬。



 馬頭郡已吾永明元年,罷譙郡屬。二年,刺史戴僧靜又以濟縣並之。



 濟陰郡頓丘永明元年,罷定陶並。睢陵樂平永明元年,割鐘離屬。濟安永明元年,割鐘離屬。



 新昌郡頓丘穀熟尉氏沛郡
 相蕭沛青州,宋泰始初淮北沒虜,六年,始治鬱州上。鬱州在海中,周回數百里,島出白鹿,土有田疇魚鹽之利。劉善明為刺史,以海中易固,不峻城雉,乃累石為之,高可八九尺。後為齊郡治。建元初,徙齊郡治瓜步,以北海治齊郡故治,州治如舊。流荒之民,郡縣虛置,至於分居土著,蓋無幾焉。建元四年,移鎮朐山,後復舊。



 領郡如左:齊郡永明元年,罷秦郡並之,治瓜步。



 臨淄永明二年,省華城縣並齊安永明元年罷西安宿豫尉氏平虜昌國泰益都北海郡都昌宋鬱縣,建元改用漢名也。廣饒贛榆膠東劇下密平壽東莞琅邪二郡
 治朐山也即丘南東莞永明元年,以流戶置。北東莞冀州,宋元嘉九年分青州置。青州領齊、濟南、樂安、高密、平昌、北海、東萊、太原、長廣九郡,冀州領廣川、平原、清河、樂陵、魏郡、河間、頓丘、高陽、勃海九郡。泰始初,遇虜寇,並荒沒。今所存者,泰始之後更置立也。二州共一刺史。郡縣十無八九,但有名存,案《宋志》自知也。建元初,以東海郡屬冀州。全領一郡:北東海郡治連口襄賁僮下邳厚丘曲城江州,鎮尋陽,中流衿帶。晉元康元年,惠帝詔:「荊、揚二州,疆土曠遠。有司奏割揚州之豫章、鄱陽、廬陵、臨川、南康、建安、晉安為新州。新安、東陽、宣城舊豫章封內,豫章之東北,相去懸遠,可如故屬揚州。又割荊州之武昌、桂陽、安成并十郡,可因江水之名為江州,宜治
 豫章。」庾亮領刺史,都督六州,云以荊、江為本,校二州戶口,雖相去機事,實覺過半,江州實為根本。臨終表江州宜治尋陽,以州督豫州新蔡、西陽二郡,治湓城,接近東江諸郡,往來便易。其後庾翼又還豫章。義熙後,還尋陽。何無忌表:「竟陵去治遼遠,去江陵正三百里,荊州所立綏安郡民戶,參入此境,郡治常在夏口左右,欲資此郡助江濱戍防,以竟陵還荊州。又司州弘農、揚州松滋二郡,寄尋陽,人民雜居,宜並見督。」今九江在州鎮之北,彭蠡在其東也。領郡如左:尋陽郡。



 柴桑彭澤豫章郡南昌新淦艾建城建昌望蔡新
 吳永修吳平康樂豫章豐城臨川郡南城臨汝新建永城宜黃南豐東興安浦西豐廬陵郡石陽西昌東昌吉陽巴丘興平高昌陽豐遂興鄱陽郡鄱陽餘干葛陽樂安廣晉上饒安成郡平都新喻永新萍鄉宜陽廣興安復南康郡
 贛雩都南野寧都平固陂陽虔化永明八年,罷安遠縣並。南康南新蔡郡慎苞信陽唐左縣宋建安郡吳興建安將樂邵武建陽綏城沙村晉安郡候官羅江原豐晉安溫麻廣州,鎮南海。濱際海隅,委輸交部,雖民戶不多,而俚獠猥雜,皆樓居山險,不肯賓服。西南二江,川源深遠,別置督護,專征討之。卷握之資,富兼十世。尉他餘基,亦有霸跡,江左以其遼遠,蕃戚未有居者,唯宋隨王誕為刺史,領郡如左:南
 海郡番禺熙安博羅增城龍川懷化酉平綏寧新豐羅陽高要安遠河源東官郡懷安寶安海安欣樂海豐齊昌陸安興寧義安郡綏安海寧海陽義招潮陽程鄉新寧郡博林南興臨沇甘泉新成威平單牒龍潭城陽威化歸順初興撫納平鄉蒼梧郡廣信寧新封興撫寧遂城丁留懷熙猛陵
 廣寧蕩康僑寧思安高涼郡安寧羅州莫陽西鞏思平禽鄉平定永平郡夫寧安沂幹安盧平員鄉蘇平逋寧雷鄉開城毗平武林豐城晉康郡威城都城夫阮元溪安遂晉化永始端溪賓江熙寧樂城武定悅城文招義立新會郡盆允新夷封平初賓封樂義
 寧新熙永昌始康招集始成廣熙郡龍鄉羅平賓化寧鄉長化定昌永熙寶寧宋康郡廣化石門化隆遂度威覃單城開寧海鄰輿定綏定宋隆郡平興招興崇化建寧熙穆崇德海昌郡寧化招懷永建始化新建綏建郡新招四會化蒙化注化穆樂昌
 郡始昌樂山宋元義立安樂鬱林郡布山鬱平阿林建安始集龍平賓平新林綏寧中胄領方懷安歸化晉平威化桂林郡武熙騰溪潭平龍岡臨浦中留武豐程安威定潭中安遠安化龍定寧浦郡安廣簡陽平山寧浦興道吳安晉興郡晉興熙注桂林增翊安廣廣鬱晉城鬱陽齊樂郡
 希平觀寧臻安宋平綏南封陵齊康郡樂康齊建郡初寧永城齊熙郡交州,鎮交阯,在海漲島中。楊雄《箴》曰:「交州荒遰,水與天際。」外接南夷,寶貨所出。山海珍怪,莫與為比。民恃險遠,數好反叛。領郡如左:九真郡移風胥浦松原高安建初常樂津梧軍安吉龐武寧武平郡
 武定封溪平道武興根寧南移新昌郡范信嘉寧封山西道臨西吳定新道晉化九德郡九德咸驩浦陽南陵都洨越常西安日南郡西卷象林壽冷朱吾比景盧容無勞交阯郡龍編武寧望海句漏吳興西于朱緌南定曲易海平羸𨻻宋平郡昌國義懷綏寧
 宋壽郡建元二年,割越州屬。



 義昌郡永元二年,改沃屯置。



 越州,鎮臨漳郡,本合浦北界也。夷獠叢居,隱伏巖障,寇盜不賓,略無編戶。宋泰始中,西江督護陳伯紹獵北地,見二青牛驚走入草,使人逐之不得,乃志其處,云「此地當有奇祥」。啟立為越州。七年,始置百梁、隴蘇、永寧、安昌、富昌、南流六郡,割廣、交朱緌三郡屬。元徽二年,以伯紹為刺史,始立州鎮,穿山為城門,威服俚獠。土有瘴氣殺人。漢世交州刺史每暑月輒避處高,今交土調和,越瘴獨甚。刺史常事戎馬,唯以戰伐為務。



 臨漳郡漳平丹城勞石容城長石都并緩端合浦郡
 徐聞合浦朱盧新安晉始蕩昌朱豐宋豐宋廣永寧郡杜羅金安蒙廖簡留城百梁郡百梁始昌宋西安昌郡武桑龍淵石秋撫林南流郡方度北流郡永明六年立,無屬縣。



 龍蘇郡
 龍蘇富昌郡南立義立歸明高興郡宋和寧單高興威成夫羅南安歸安陳蓮高城新建思築郡鹽田郡杜同定川郡
 興昌隆川郡良國齊寧郡建元二年置,割鬱林之新邑、建初二縣並。



 開城建元二年置延海新邑建初越中郡馬門郡鐘吳田羅馬陵思寧封山郡安金吳春俚郡永明六年立,無屬縣。



 齊隆郡先屬交州,中改為囗囗,永泰元年,改為齊隆,還屬囗州。



\end{pinyinscope}