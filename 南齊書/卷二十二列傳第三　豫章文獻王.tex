\article{卷二十二列傳第三 豫章文獻王}

\begin{pinyinscope}

 豫章文獻王嶷,字宣儼,太祖第二子。寬仁弘雅,有大成之量,太祖特鐘愛焉。



 起家為太學博士、長城令,入為尚書左民郎、錢唐令。太祖破薛索兒,改封西陽,以先爵賜為晉壽縣侯。除通直散騎侍郎,以偏憂去官。桂陽之役,太祖出頓新亭壘,板嶷為寧朔將軍,領兵衛從。休範率士卒攻壘南,嶷執白虎幡督戰,屢摧卻之。事寧,遷中書郎。尋為安遠護軍、武陵內史。



 時沈攸之責賧,伐荊州界內諸蠻,遂及五溪,禁斷魚鹽。群蠻怒,酉溪蠻王田頭擬殺攸之使,攸之責賧千萬,頭擬輸五百萬,發氣死。其弟婁侯篡立,頭擬子田都走入獠中。於是蠻部大亂,抄掠平民,至郡城下。嶷遣隊主張莫兒率將吏
 擊破之。



 田都自獠中請立,而婁侯懼,亦歸附。嶷誅婁侯於郡獄,命田都繼其父,蠻眾乃安。



 入為宋順帝車騎諮議參軍、府掾,轉驃騎,仍遷從事中郎。詣司徒袁粲,粲謂人曰:「後來佳器也。」



 太祖在領軍府,嶷居青溪宅。蒼梧王夜中微行,欲掩襲宅內,嶷令左右舞刀戟於中庭,蒼梧從墻間窺見,以為有備,乃去。太祖帶南兗州,鎮軍府長史蕭順之在鎮,憂危既切,期渡江北起兵。嶷諫曰:「主上狂兇,人下不自保,單行道路,易以立功。外州起兵,鮮有克勝。物情疑惑,必先人受禍。今於此立計,萬不可失。」



 蒼梧王殞,太祖報嶷曰:「大事已判,汝明可早入。」順帝即位,轉侍中,總宮內直衛。



 沈攸之之難,太祖入朝堂,嶷出鎮東府,加冠軍將軍。袁粲舉兵夕,丹陽丞王遜告變,先至東府,嶷遣帳內軍主戴元孫二千人隨薛道淵等俱至石頭,焚門之功,元孫預焉。先是王蘊薦部曲六十人助為城防,實以為內應
 也。嶷知蘊懷貳,不給其仗,散處外省。及難作搜檢,皆已亡去。遷中領軍,加散騎常侍。上流平後,世祖自尋陽還,嶷出為使持節、都督江州豫州之新蔡晉熙二郡軍事、左將軍、江州刺史,常侍如故。給鼓吹一部。以定策功,改封永安縣公,千五百戶。仍徙都督荊湘、雍、益、梁、寧、南、北秦八州諸軍事、鎮西將軍、荊州刺史,持節、常侍如故。



 時太祖輔政,嶷務在省約,停府州儀迎物。初,沈攸之欲聚眾,開民相告,士庶坐執役者甚眾。嶷至鎮,一日遣三千餘人。見囚五歲刑以下不連臺者,皆原遣。



 以市稅重濫,更定樢格,以稅還民。禁諸市調及苗籍。二千石官長不得與人為市,諸曹吏聽分番假。百姓甚悅。禪讓之間,世祖欲速定大業,嶷依違其事,默無所言。



 建元元年,太祖即位,赦詔未至,嶷先下令蠲除部內昇明二年以前逋負。遷侍中,尚書令,都督揚、南徐二州諸軍事,驃騎大將軍,開府儀同三
 司,揚州刺史,持節如故。封豫章郡王,邑三千戶。僕射王儉箋曰:「舊楚蕭條,仍歲多故,荒民散亡,實須緝理。公臨蒞甫爾,英風惟穆,江、漢來蘇,八州慕義。自庾亮以來,荊楚無復如此美政。古人期月有成,而公旬日致治,豈不休哉!」



 會北虜動,上思為經略,乃詔曰:「神牧總司王畿,誠為治要;荊楚領馭遐遠,任寄弘隆。自頃公私凋盡,綏撫之宜,尤重恆日。」復以為都督荊、湘、雍、益、梁、寧、南、北秦八州諸軍事,南蠻校尉,荊、湘二州刺史,持節、侍中、將軍、開府如故。晉宋之際,刺史多不領南蠻,別以重人居之,至是有二府二州。荊州資費歲錢三千萬,布萬匹,米六萬斛,又以江、湘二州米十萬斛給鎮府;湘州資費歲七百萬,布三千匹,米五萬斛;南蠻資費歲三百萬,布萬匹,綿千斤,絹三百匹,米千斛,近代莫比也。尋給油絡俠望車。



 二年春,虜寇司、豫二州,嶷表遣南蠻司馬崔慧景北討,又分遣中兵參
 軍蕭惠朗援司州,屯西關。虜軍濟淮攻壽春,分騎當出隨、鄧,眾以為憂。嶷曰:「虜入春夏,非動眾時,令豫、司強守,遏其津要;彼見堅嚴,自當潰散,必不敢越二鎮而南也。」是時纂嚴,嶷以荊州鄰接蠻、蜑,慮其生心,令鎮內皆緩服。既而虜竟不出樊、鄧,於壽春敗走。尋給班劍二十人。



 其夏,於南蠻園東南開館立學,上表言狀。置生四十人,取舊族父祖位正佐臺郎,年二十五以下十五以上補之;置儒林參軍一人,文學祭酒一人,勸學從事二人,行釋菜禮。以穀過賤,聽民以米當口錢,優評斛一百。



 義陽劫帥張群亡命積年,鼓行為賊,義陽、武陵、天門、南平四郡界,被其殘破。沈攸之連討不能禽,乃首用之。攸之起事,群從下郢,於路先叛,結寨於三溪,依據深險。嶷遣中兵參軍虞欣祖為義陽太守,使降意誘納之,厚為禮遺,於坐斬首,其黨數百人皆散,四郡獲安。



 入為都督揚南徐二州諸軍事、中
 書監、司空、揚州刺史,持節、侍中如故。加兵置佐。以前軍臨川王映府文武配司空府。嶷以將還都,修治廨宇及路陌,東歸部曲不得齎府州物出城。發江津,士女觀送數千人,皆垂泣。嶷發江陵感疾,至京師未瘳,上深憂慮,為之大赦,三年六月壬子赦令是也。疾愈,上幸東府設金石樂,敕得乘輿至宮六門。



 太祖崩,嶷哀號,眼耳皆出血。世祖即位,進位太尉,置兵佐,解侍中,增班劍為三十人。建元中,世祖以事失旨,太祖頗有代嫡之意,而嶷事世祖恭悌盡禮,未嘗違忤顏色,故世祖友愛亦深。永明元年,領太子太傅,解中書監,餘如故。手啟上曰:「陛下以睿孝纂業,萬宇惟新,諸弟有序。臣屢荷隆愛,叨授台首,不敢固辭。俛仰祗寵,心魂如失。負重量力,古今同規。臣窮生如浮,質操空素,任居鼎右,已移氣序。自頃以來,宿疾稍纏,心慮恍惚,表於容狀。視此根候,常恐命不勝恩;加以星緯屢
 見災祥,雖修短有恆,能不耿介?比心欲從俗,啟解今職,但厝辭為鄙,或貽物誚,所以息意緘嘿,一委時運,而可復加寵榮,增其顛墜?且儲傅之重,實非恆選,遂使太子見臣必束帶,宮臣皆再拜,二三之宜,何以當此!陛下同生十餘,今唯臣而已,友于之愛,豈當獨臣鐘其隆遇!別奉啟事,仰祈恩照。



 臣近亦侍言太子,告意子良,具因王儉申啟,未知粗上聞未?福慶方隆,國祚永始,若天假臣年,得預人位,唯當請降貂榼,以飾微軀,永侍天顏,以惟畢世,此臣之願也。服之不衷,猶為身災,況寵爵乎!殊榮厚恩,必誓以命請。」上答曰:「事中恐不得從所陳。」



 宋氏以來,州郡秩俸及[雜]供給,多隨土所出,無有定準。嶷上表曰:「循革貴宜,損益資用,治在夙均,政由一典。伏尋郡縣長尉俸祿之制,雖有定科,而其餘資給,復由風俗。東北異源,西南各緒,習以為常,因而弗變。緩之則莫非通規,澄之則靡不入罪。
 殊非約法明章,先令後刑之謂也。臣謂宜使所在各條公用公田秩石迎送舊典之外,守宰相承,有何供調,尚書精加洗核,務令優衷。事在可通,隨宜開許,損公侵民,一皆止卻,明立定格,班下四方,永為恆制。」從之。



 嶷不參朝務,而言事密謀,多見信納。服闋,加侍中。二年,詔曰:「漢之梁孝,寵異列蕃,晉之文獻,秩殊恆序。況乃地侔前準,勛兼往式!雖天倫有本,而因事增情。宜廣田邑,用申恩禮。」增封為四千戶。



 宋元嘉世,諸王入齋閣,得白服裙帽見人主,唯出太極四廂,乃備朝服,自比以來,此事一斷。上與嶷同生,相友睦,宮內曲宴,許依元嘉。嶷固辭不奉敕,唯車駕幸第,乃白服烏紗帽以侍宴焉。啟自陳曰:「臣自還朝,便省儀刀、捉刀,左右十餘亦省,唯郊外遠行,或復暫有,入殿亦省。服身今所牽仗,二俠轂,二白直,共七八十人。事無大小,臣必欲上啟,伏度聖心脫未委曲,或有言其多少,不
 附事實,仰希即賜垂敕。」又啟:「揚州刺史舊有六白領合扇,二白拂,臣脫以為疑,不審此當云何?行園苑中乘輿,出籬門外乘輿鳴角,皆相仍如此,非止於帶神州者,未審此當云何?方有行來,不可失衷。」上答曰:「儀刀、捉刀,不應省也。俠轂、白直,乃可共百四五十以還正是耳。亦不曾聞人道此。吾自不使諸王無仗,況復汝耶?在私園苑中乘此非疑。郊外鳴角及合扇并拂,先乃有,不復施用,此來甚久。



 凡在鎮自異還京師,先廣州乃立鼓吹,交部遂有輦事,隨時而改,亦復有可得依舊者。汝若有疑,可與王儉諸人量衷,但令人臣之儀無失便行也。」



 又啟曰:「臣拙知自處,暗於疑訪,常見素姓扶詔或著布屩,不意為異。臣在西朝拜王,儀飾悉依宋武陵事例,有二鄣扇,仍此下都,脫不為疑;小兒奴子,並青布褲衫,臣齋中亦有一人,意謂外庶所服,不疑與羊車相類。曲荷慈旨,今悉改易。臣昔在
 邊鎮,不無羽衛,自歸朝以來,便相分遣,俠轂、白直,格置三百許人,臣頃所引,不過一百。常謂京師諸王不煩牽仗,若郊外遠行,此所不論。有仗者非臣一人,所以不容方幅啟省,又因王儉備宣下情。臣出入榮顯,禮容優泰,第宇華曠,事乖素約,雖宋之遺製,恩處有在,猶深非服之慚。威衛之請,仰希曲照。」



 上答曰:「傳詔臺家人耳,不足涉嫌。鄣扇,吾識及以來未見,故有敕耳。小兒奴子,本非嫌也。吾有所聞,豈容不敕汝知,令致物議耶?吾已有敕,汝一人不省俠轂,但牽之。吾昨不通仗事,儉已道,吾即令答,不煩有此啟。須間言,自更一二。」



 又啟曰:「違遠侍宴,將踰一紀,憂苦間之,始得開顏。近頻侍座,不勝悲喜。



 沾飲過量,實欲仰示恩狎,令自下知見,以杜游塵。陛下留恩子弟,此情何異,外物政自強生間節,聲其厚薄。伏度或未上簡。臣前在東田,承恩過醉,實思歎往秋之謗,故言啟至切,亦令
 群物聞之,伏願已照此心。前侍幸順之宅,臣依常乘車至仗後,監伺不能示臣可否,便互競啟聞,云臣車逼突黃屋麾旄,如欲相中。推此用意,亦何容易!仰賴慈明,即賜垂敕;不爾,臣終不知暗貽此累。比日禁斷整密,此自常理,外聲乃云起臣在華林,輒捉御刀,因此更嚴。度情推理,必不容爾,為復上啟知耳。但風塵易至,和會實難,伏願猶憶臣石頭所啟,無生間縫。此閑侍無次,略附茹亮口宣。臣由來華素,已具上簡,每欲存衷,意慮不周,或有乖常。且臣五十之年,為玩幾時,為此亦復不能以理內自制。北第舊邸,本自甚華,臣改修正而已,小小製置,已自仰簡。往歲收合得少雜材,並蒙賜故板,啟榮內許作小眠齋,始欲成就,皆補接為辦,無乖格製,要是檉柏之華,一時新凈。東府又有齋,亦為華屋。而臣頓有二處住止,下情竊所未安。訊訪東宮玄圃,乃有柏屋,製甚古拙,內中無此
 齋,臣乃欲壞取以奉太子,非但失之於前,且補接既多,不可見移,亦恐外物或為異論,不審可有垂許送東府齋理否?臣公家住止,率爾可安,臣之今啟,實無意識,亦無言者,太子亦不知臣有此屋,政以東宮無,而臣自處之,體不宜爾爾。所啟蒙允,臣便當敢成第屋,安之不疑。陛下若不照體臣心,便當永廢不修。臣自謂今啟非但是自處宜然,實為微臣往事,伏願必垂降許。伏見以諸王舉貨,屢降嚴旨,少拙營生,已應上簡。府州郡邸舍,非臣私有,今巨細所資,皆是公潤,臣私累不少,未知將來罷州之後,或當不能不試學營覓以自贍。連年惡疾餘,顧影單回,無事畜聚,唯逐手為樂耳。」上答曰:「茹亮今啟汝所懷及見別紙,汝勞疾亦復那得不動,何意為作煩長啟事!凡諸普敕,此意可尋,當不關汝一人也。宜有敕事,吾亦必道,頃見汝自更委悉,書不欲多及。屋事慎勿彊厝此意,白澤
 亦當不解何意爾。」



 三年,文惠太子講《孝經》畢,嶷求解太傅,不許。皇孫婚竟,又陳解,詔曰:「公惟德惟行,無所厝辭。且魯且衛,其誰與二?方式範當時,流聲史籍,豈容屢秉捴謙,以乖期寄。」嶷常慮盛滿,又因宮宴,求解揚州授竟陵王子良。上終不許,曰:「畢汝一世,無所多言。」世祖即位後,頻發詔拜陵,不果行。遣嶷拜陵,還過延陵季子廟,觀沸井,有水牛突部伍,直兵執牛推問,不許,取絹一匹橫繫牛角,放歸其家。為治存寬厚,故得朝野歡心。



 四年,唐宇之賊起,啟上曰:「此段小寇,出於兇愚,天網宏罩,理不足論。



 但聖明御世,幸可不爾,比藉聲聽,皆云有由而然。豈得不仰啟所懷,少陳心款?



 山海崇深,臣獲保安樂,公私情願,於此可見。齊有天下,歲月未久,澤沾萬民,其實未多,百姓猶險,懷惡者眾。陛下曲垂流愛,每存優旨。但頃小大士庶,每以小利奉公,不顧所損者大,擿籍檢工巧,督恤簡小塘,
 藏丁匿口,凡諸條制,實長怨府。此目前交利,非天下大計。一室之中,尚不可精,宇宙之內,何可周洗!公家何嘗不知民多欺巧,古今政以不可細碎,故不為此,實非乖理。但識理者百不有一,陛下弟兒大臣,猶不皆能伏理,況復天下悠悠萬品!怨積聚黨,兇迷相類,止於一處,何足不除?脫復多所,便成紜紜。久欲上啟,閑侍無因,謹陳愚管,伏願特留神思。」上答曰:「欺巧那可容!宋世混亂,以為是不?蚊蟻何足為憂,已為義勇所破,官軍昨至,今都應散滅。吾政恨其不辦大耳,亦何時無亡命邪!」後乃詔聽復籍注。五年,進位大司馬。八年,給皂輪車。尋加中書監,固讓。



 嶷身長七尺八寸,善持容範,文物衛從,禮冠百僚,每出入殿省,皆瞻望嚴肅。



 自以地位隆重,深懷退素,北宅舊有園田之美,乃盛修理之。七年,啟求還第,上令世子子廉代鎮東府。上數幸嶷第。宋長寧陵遂道出第前路,上曰:「我便是入
 他塚墓內尋人。」乃徙其表闕騏驎於東崗上。騏驎及闕,形勢甚巧,宋孝武於襄陽致之,後諸帝王陵皆模範而莫及也。永明末,車駕數游幸,唯嶷陪從。上出新林苑,同輦夜歸,至宮門,嶷下輦辭出,上曰:「今夜行,無使為尉司所呵也。」嶷對曰:「京輦之內,皆屬臣州,願陛下不垂過慮。」上大笑。上謀北伐,以虜所獻氈車賜嶷。每幸第清除,不復屏人。上敕外監曰:「我往大司馬第,是還家耳。」嶷妃庾氏常有疾,瘳,上幸嶷邸,後堂設金石樂,宮人畢至。每臨幸,輒極日盡歡。嶷謂上曰:「古來言願陛下壽偕南山,或稱萬歲,此殆近貌言。如臣所懷,實願陛下極壽百年亦足矣。」上曰:「百年復何可得,止得東西一百,於事亦濟。」



 十年,上封嶷諸子,舊例千戶,嶷欲五子俱封,啟減人五百戶。其年疾篤,表解職,不許,賜錢百萬營功德。嶷又啟曰:「臣自嬰今患,亟降天臨,醫走術官,泉開藏府,慈寵優渥,備極人臣。生年疾
 迫,遽陰無幾。願陛下審賢與善,極壽蒼旻,強德納和,為億兆御。臣命違昌數,奄奪恩憐,長辭明世,伏涕嗚咽。」薨,年四十九。其日,上再視疾,至薨,乃還宮。詔曰:「嶷明哲至親,勳高業始,德懋王朝,道光區縣,奄至薨逝,痛酷抽割,不能自勝,奈何奈何!今便臨哭。九命之禮,宜備其制。斂以袞冕之服,溫明秘器,命服一具,衣一襲,喪事一依漢東平王故事,大鴻臚持節護喪事,大官朝夕送奠。大司馬、太傅二府文武悉停過葬。」



 竟陵王子良啟上曰:「臣聞《春秋》所以稱王母弟者,以尊其所重故也。是以禮秩殊品,爵命崇異,在漢則梁王備出警入蹕之儀,在晉則齊王具殊服九命之贈。



 江左以來,尊親是闕,故致袞章之典,廢而不傳,實由人缺其位,非禮虧省。齊王故事,與今不殊,締構王業,功跡不異。凡有變革隨時之宜者,政緣恩情有輕重,德義有厚薄。若事籌前規,禮無異則。且梁、齊闕令終之美,猶饗
 褒贈之榮;況故大司馬仁和著於天性,孝悌終於立身,節義表於勤王,寬猛彰於御物,奉上無艱劬之貌,接下無毀傷之容!淡矣止於清貞,無喜慍之色;悠然棲於靜默,絕馳競之聲。



 《詩》云『靡不有初,鮮克有終』。夫終之者,理實為難,在於令行,無廢斯德。



 東平樂於小善,河間悅於詩書,勳績無聞,艱危不涉,尚致卓爾不群,英聲萬代;況今協贊皇基,經綸霸始,功業高顯,清譽逾彰,富貴隆重,廉潔彌峻,等古形今,孰類茲美!臣愚忖度,未有斯例!凡庶族同氣,愛睦尚少,豈有仰睹陛下垂友于之性若此者乎?共起布衣,俱登天貴;生平游處,何事不同?分甘均味,何珍不等?



 未常睹貌而天心不歡,見形而聖儀不悅。爰及臨危捨命,親瞻喘息,萬分之際,沒在聖目,號哭動乎天地,感慟驚乎鬼神,乃至撤膳移寢,坐泣遷旦,神儀損耗,隔宿改容,奉瞻聖顏,誰不悲悚!歷古所未聞,記籍所不載。既有若
 斯之大德,實不可見典服之贈不彰。如其脫致虧忘,追改為煩,不令千載之下,物有遺恨!其德不具美者,尚荷嘉隆之命;況事光先烈者,寧可缺茲盛典!臣恐有識之人,容致其議。



 且庶族近代桓溫、庾亮之類,亦降殊命,伏度天心,已當有在。」



 又詔曰:「寵章所以表德,禮秩所以紀功。慎終追遠,前王之盛策,累行疇庸,列代之通誥。故使持節、都督揚南徐二州諸軍事、大司馬、領太子太傅、揚州刺史,新除中書監豫章王嶷,體道秉哲,經仁緯義,挺清譽於弱齡,發韶風於早日,締綸霸業之初,翼贊皇基之始,孝睦著於鄉閭,忠諒彰乎邦邑。及秉德論道,總牧神甸,七教必荷,六府咸理。振風潤雨,無諐於時候;恤民拯物,有篤於矜懷。雍容廊廟之華,儀形列郡之觀,神凝自遠,具瞻允集。朕友于之深,情兼家國。方授以神圖,委諸廟勝,緝頌九弦,陪禪五岳,天不憖遺,奄焉薨逝。哀痛傷惜,震慟乎厥
 心。



 今先遠戒期,龜謀襲吉,宜加茂典,以協徽猷。可贈假黃皞、都督中外諸軍事、丞相、揚州牧,綠綟綬,具九服錫命之禮,侍中、大司馬、太傅、王如故。給九旒鸞輅,黃屋左纛,虎賁班劍百人,轀輬車,前後部羽葆鼓吹,葬送儀依東平王故事。」



 嶷臨終,召子子廉、子恪曰:「人生在世,本自非常,吾年已老,前路幾何。



 居今之地,非心期所及。性不貪聚,自幼所懷,政以汝兄弟累多,損吾暮志耳。無吾後,當共相勉厲,篤睦為先。才有優劣,位有通塞,運有富貧,此自然理,無足以相陵侮。若天道有靈,汝等各自修立,灼然之分無失也。勤學行,守基業,治閨庭,尚閑素,如此足無憂患。聖主儲皇及諸親賢,亦當不以吾沒易情也。三日施靈,唯香火、槃水、乾飯、酒脯、檳榔而已。朔望菜食一盤,加以甘果,此外悉省。葬後除靈,可施吾常所乘輿扇傘。朔望時節,席地香火、槃水、酒脯、乾飯、檳榔便足。雖才愧古人,意懷
 粗亦有在,不以遺財為累。主衣所餘,小弟未婚,諸妹未嫁,凡應此用,本自茫然,當稱力及時,率有為辦。事事甚多,不復甲乙。棺器及墓中,勿用餘物為後患也。朝服之外,唯下鐵鈽刀一口。作冢勿令深,一一依格,莫過度也。後堂樓可安佛,供養外國二僧,餘皆如舊。與汝游戲後堂船乘,吾所乘牛馬,送二宮及司徒,服飾衣裘,悉為功德。」子廉等號泣奉行。



 世祖哀痛特至,至冬乃舉樂宴朝臣,上虛欷流涕。諸王邸不得起樓臨瞰宮掖,上後登景陽,望見樓悲感,乃敕毀之。薨後,第庫無見錢,世祖敕貨雜物服飾得數百萬,起集善寺,月給第見錢百萬,至上崩乃省。



 嶷性泛愛,不樂聞人過失,左右有投書相告,置靴中,竟不視,取火焚之。齋庫失火,燒荊州還資,評直三千餘萬,主局各杖數十而已。群吏中南陽樂藹、彭城劉繪、吳郡張稷最被親禮。藹與竟陵王子良箋曰:「道德以可久傳聲,風流以
 浸遠隳稱。雖復青簡締芳,未若玉石之不朽;飛翰圖藻,豈伊雕篆之無沫!丞相沖粹表於天真,淵照殆乎機象。經邦緯民之範,體國成務之規,以業茂惟賢,功高則哲。



 神輝眇邈,睿算不追,感纏奉車,恨百留滯。下官夙稟名節,恩義軫慕,望遂結哀,輒欲率荊、江、湘三州僚吏,建碑壟首,庶徽猷有述,茂則方存。昔子香淳德,留銘江介,鉅平遺烈,墮淚漢南,況道尊前往,惠積聯綿者哉!下官今便反假,無由躬事刊斫,須至西州鳩集所資,托中書侍郎劉繪營辦。」



 藹又與右率沈約書曰:「夫道宣餘烈,竹帛有時先朽;德孚遺事,金石更非後亡。丞相獨秀生民,傍照日月。標勝丘園,素履穆於忠義;譽應華袞,功迹著於弼諧。無得而稱,理絕照載。若夫日用闃寂,雖無取於錙銖;歲功宏達,諒有寄於衡石。竊承貴州士民,或建碑表,俾我荊南,閱感無地。且作紀江、漢,道基分陜,衣冠禮樂,咸被後昆。若
 其望碑盡禮,我州之舊俗,傾罷肆,鄙土之遺風,庶幾弘烈或不泯墜。荊、江、湘三州策名不少,並欲各率毫厘,少申景慕。斯文之托,歷選惟疑,必待文蔚辭宗,德僉茂履,非高明而誰?豈能騁無愧之辭,酬式瞻之望!



 吾西州窮士,一介寂寥,恩周榮譽,澤遍衣食。永惟道蔭,日月就遠,緬尋遺烈,觸目崩心。常謂福齊南山,慶鐘仁壽。吾儕小人,貽塵帷蓋,豈圖一旦,遂投此請。」



 約答曰:「丞相風道弘曠,獨秀生民,凝猷盛烈,方軌伊、旦。憖遺之感,朝野同悲。承當刊石紀功,傳華千載,宜鬚盛述,實允來談。郭有道漢末之匹夫,非蔡伯喈不足以偶三絕,謝安石素族之台輔,時無麗藻,迄乃有碑無文。況文獻王冠冕彞倫,儀形宇內,自非一世辭宗,難或與比。約閭閈鄙人,名不入弟,欻酬今旨,便是以禮許人,聞命慚顏,已不覺汗之沾背也。」建武中,第二子子恪托約及太子詹事孔稚珪為文。



 子廉字景
 藹。初,嶷養魚復侯子響為世子,子廉封永新侯,千戶。子響還本,子廉為世子。除寧朔將軍、淮陵太守,太子中舍人,前軍將軍。善撫諸弟子。十一年卒,贈侍中,謚哀世子。



 第三子子操,泉陵侯。王侯出身官無定,準素姓三公長子一人為員外郎。建武中,子操解褐為給事中,自此齊末皆以為例。永泰元年,南康侯子恪為吳郡太守,避王敬則難奔歸,以子操為寧遠將軍、吳郡太守。永元中,為黃門郎。義師圍城,子操與弟宜陽侯子光卒於尚書都座。



 第四子子行,洮陽侯,早卒。



 子元琳嗣,今上受禪,詔曰:「褒隆往代,義炳彞則。朕當此樂推,思弘前典。



 豫章王元琳、故巴陵王昭胄子同,齊氏宗國,高、武嫡胤,宜祚井邑,以傳世祀。



 降新淦縣侯,五百戶。」



 史臣曰:楚元王高祖亞弟,無功漢世,東平憲王辭位永平,未及光武之業,梁孝惑於勝、詭,安平心隔晉運。蕃輔貴盛,地實高危,持
 滿戒盈,鮮能全德。豫章宰相之器,誠有天真,因心無矯,率由遠度,故能光贊二祖,內和九族,實同周氏之初,周公以來,則未知所匹也。



 贊曰:堂堂烈考,德邁前蹤。移忠以孝,植友惟恭。帝載初造,我王奮庸。邦家有闕,我王彌縫。道深日用,事緝民雍。愛傳餘祀,聲流景鐘。



\end{pinyinscope}