\article{卷十八志第十祥瑞}

\begin{pinyinscope}

 天符瑞令,遐哉邈矣。靈篇祕圖,固以蘊金匱而充石室,炳《契決》,陳《緯候》者,方策未書。啟覺天人之期,扶獎帝王之運。三五聖業,神明大寶,二謀協贊,罔不由茲。夫流火赤雀,實紀周祚;雕雲素靈,發祥漢氏;光武中興,皇符為盛;魏膺當塗之讖,晉有石瑞之文,史筆所詳,亦唯舊矣。齊氏受命,事殷前典。



 黃門郎蘇偘撰《聖皇瑞應記》,永明中庾溫撰《瑞應圖》,其餘眾品,史注所載。



 今詳錄去取,以為志云。



 《老子河洛讖》曰:「年曆七七水滅緒,風雲俱起龍麟舉。」宋水德王,義熙十四年,元熙二年,永初三年,景平一年,元嘉三十年,孝建三年,大明八年,永光一年,泰始七年,泰豫一年,元徽四年,升明三
 年,凡七十七年,故曰七七也。



 《易》曰:「雲從龍,風從虎。」關尹云:「龍不知其乘風雲而上天也。」



 讖又曰:「肅草成,道德懷書備出身,形法治吳出南京。」上即姓諱也。南京,南徐州治京口也。



 讖又曰:「墥堨河梁塞龍淵,消除水災洩山川。」墥堨河梁,為路也,路即道也。淵塞者,譬路成也。即太祖諱也。消水災,言除宋氏患難也。



 讖又曰:「上參南斗第一星,下立草屋為紫庭。神龍之岡梧桐生,鳳鳥舒翼翔且鳴。」南斗第一星,吳分也。草屋,蕭字也。又簫管之器,像鳳鳥翼也。



 讖又曰:「簫為二士,天下大樂。」二士,主字也。



 讖又曰:「天子何在草中宿。」宿,蕭也。



 《尚書中候儀明篇》曰:「仁人傑出,握表之象,曰角姓,合音之于。」蘇偘
 云:「蕭,角姓也。又八音之器有簫管也。」



 史臣曰:案晉光祿大夫何禎解「音之於」為曹字,謂魏氏也。王隱《晉書》云:「卯金音于,亦為魏也。」《候》書章句,本無銓序,二家所稱,即有前釋,未詳偘言為何推據。



 《孝經鉤命決》曰:「誰者起,視名將。」君者群也,理物為雄,憂劣相次以期興,將,太祖小諱也。征西將軍蕭思話見之曰:「此我家諱也。」



 王子年歌曰:「金刀治世後遂苦。帝王昏亂天神怒。災異屢見戒人主。三分二叛失州土。三王九江一在吳。餘悉稚小早少孤。一國二主天所驅。」金刀,劉也;三分二叛,宋明帝世也;三王九江者,孝武於九江興,晉安王子勛雖不終,亦稱大號,後世祖又於九江基霸迹,此三王也;一在吳,謂齊氏桑梓,亦寄治南吳也;一國二主,謂太祖符運潛興,為宋代驅除寇難。



 歌
 又曰:「三禾摻摻林茂孳,金刀利刃齊刈之。」刈,剪也。《詩》云:「實始剪商。」



 歌又曰:「欲知其姓草肅肅。谷中最細低頭熟。鱗身甲體永興福。」穀,道;熟,成,又諱也。太祖體有龍鱗,斑駁成文,始謂是黑歷,治之甚至而文愈明。伏羲亦鱗身也。



 《金雄記》曰:「鑠金作刀在龍里,占睡上人相須起。」又云:「當復有作肅入草。」蕭字也。《易》云:「聖人作之。」《記》又云:「草門可憐乃當悴,建號不成易運沸。」《詩》云不時,時也。不成,成也。建號,建元號也。易運,革命也。



 讖曰:「周文王受命,千五百歲,河雒出聖人,受命於己未,至丙子為十八周。



 旅布六郡東南隅,四國安定可久留。」案周滅殷後七百八十年,秦四十九年,漢四百二十五年,魏四十五年,晉百五十年,宋六十年,至建元元年,千五百九年也。



 武進縣彭山,舊塋在焉。
 其山岡阜相屬數百里,上有五色雲氣,有龍出焉。宋明帝惡之,遣相墓工高靈文占視,靈文先與世相祖善,還,詭答云:「不過方伯。」



 退謂世祖曰:「貴不可言。」帝意不已,遣人於墓左右校獵,以大鐵釘長五六尺釘墓四維,以為厭勝。太祖後改樹表柱,柱忽龍鳴,響震山谷,父老咸志之云。



 會稽剡縣刻石山,相傳為名,不知文字所在。昇明末,縣民襲祖行獵,忽見石上有文凡三處,苔生其上,字不可識。刊苔去之,大石文曰:「此齊者,黃公之化氣也。」立石文曰:「黃天星,姓蕭字某甲,得賢帥,天下太平。」小石文曰:「刻石者誰?會稽南山李斯刻秦望之封也。」



 益州齊後山,父老相傳,其名亦不知所起。升明三年,有沙門玄暢於山丘立精舍,其日,太祖受禪日也。



 嵩高山,升明三年四月,滎陽人尹午於山東南澗見天雨石,墜地
 石開,有璽在其中,方三寸。其文曰:「戊丁之人與道俱,肅然入草應天符。」又曰:「皇帝興運。」午奉璽詣雍州刺史蕭赤斧,赤斧表獻之。



 史臣案:昔大人見臨洮而銅人鑄,臨洮生董卓而銅人毀。有卓而世亂,世亂而卓亡,如有似也。晉末嵩高山出玉璧三十二,宋氏以為受命之詳。今此山出璽,而水德云謝,終始之徵,亦有類也。



 元徽四年,太祖從南郊,望氣者陳安寶見太祖身上黃紫氣屬天。安寶謂親人王洪範曰:「我少來未嘗見軍上有如此氣也。」



 太祖年十七,夢乘青龍西行逐日,日將薄山乃止,覺而恐懼。家人問占者,云「至貴之象也」。蘇偘云:「青,木色。日暮者,宋氏末運也。」



 泰始七年,明帝遣前淮南太守孫奉伯往淮陰監元會。奉伯與太祖同寢,夢上乘龍上天,於下捉龍腳不得。覺謂太祖曰:「兗州當大庇生民,弟不見也。」奉伯卒於宋。



 清河崔靈運為上府參軍,夢天帝謂己曰:「蕭道成是我第十九子,我去年已授其天子位。」自三皇五帝至齊,受命君凡十九人也。



 宋泰始中,童謠云「東城出天子」,故明帝殺建安王休仁。蘇偘云:「後順帝自東城即位,論者謂應之,乃是武進縣上所居東城裏也。」熊襄云:「上舊鄉有大道,相傳云秦始皇所經,呼為『天子路』,後遂為帝鄉焉。」案順帝實當援立,猶如晉之懷、愍,亦有徵符。齊運既無巡幸,路名或是秦舊,疑不能詳。



 世祖年十三,夢舉體生毛,髮生至足。又夢人指上所踐地曰「周文王之田」。



 又夢虛空中飛。又夢著孔雀羽衣。庾溫云:「雀,爵位也。」又夢鳳皇從天飛下青溪宅齋前,兩翅相去十餘丈,翼下有紫雲氣。及在襄陽,夢著桑屐行度太極殿階。



 庾溫云:「屐者,運應木也。」臣案,桑
 字為四十而二點,世祖年過此即帝位,謂著屐為木行也。屐有兩齒有聲,是為明兩之齒至四十二而行即真矣,及在郢州,夢人從天飛下,頭插筆來畫上衣兩邊,不言而去。庾溫釋云:「畫者,山、龍、華蟲也。」



 世祖宋元嘉十七年六月己未夜生,無火,婢吹灰而火自燃。



 世祖於南康郡內作伎,有弦無管,於是空中有篪聲,調節相應。



 世祖為廣興相,嶺下積旱,水涸,不通船,上部伍至,水忽暴長。庾溫云:「《易》利涉大川之義也。」



 世祖頓盆城,城內無水,欲鑿引江流,試掘井,得伏泉九處,皆湧出。



 建元元年四月,有司奏:「延陵令戴景度稱,所領季子廟,舊有湧井二所,廟祝列云舊井北忽聞金石聲,即掘,深三尺,得沸泉。其東忽有聲錚錚,又掘得泉,沸湧若浪。泉中得一銀木簡,長一尺,
 廣二寸,隱起文曰:『廬山道人張陵再拜謁詣起居。』簡木堅白而字色黃。」謹案《瑞應圖》:「浪井不鑿自成,王者清靜,則仙人主之。」《孔氏世錄》云:「葉精帝道,孔書明巧,當在張陵。」宋均注云:「張陵佐封禪。一雲陵,仙人也。」



 元徽三年,太祖在青溪宅,齋前池中忽揚波起浪,湧水如山,有金石響,須臾有青龍從池中出,左右皆見之。



 昇明元年,青龍見齊郡。



 建元四年,青龍見順陽郡清水縣平泉湖中。



 永明七年,黃龍見曲江縣黃池中,一宿二日。



 中興二年,山上雲障四塞,頃有玄黃五色如龍,長十餘丈,從西北升天。



 宋泰始末,武進舊塋有獸見,一角,羊頭,龍翼,馬足,父老咸見,莫之
 識也。



 永明十年,鄱陽郡獻一角獸,麟首,鹿形,龍鸞共色。《瑞應圖》云:「天子萬福允集,則一角獸至。」



 十一年,白象九頭見武昌。



 史臣曰:《記》云,升中于天,麟鳳至而龜龍格。則鳳皇巢乎阿閣,麒麟在乎郊藪,豈非馴之在庭,擾以成畜,其為瑞也如此!今觀魏、晉己來,世稱靈物不少,而亂多治少,史不絕書。故知來儀在沼,遠非前事,見而不至,未辨其為祥也。



 升明三年三月,白虎見歷陽龍亢縣新昌村。新昌村,嘉名也。《瑞應圖》云:「王者不暴白虎仁。」



 建元四年三月,白虎見安蠻虔化縣。



 中興二年二月,白虎見東平壽張安
 樂村。



 昇明二年,騶虞見安東縣五界山,師子頭,虎身,龍腳。《詩傳》云:「騶虞,義獸,白虎黑文,不食生物,至德則出。」



 昇明三年,太祖為齊王,白毛龜見東府城池中。



 建元二年,休安陵獲玄龜一頭。永明五年,武騎常侍唐潛上青毛神龜一頭。



 七年六月,彭城郡田中獲青毛龜一頭。



 八月,延陵縣前澤畔獲毫龜一枚。



 八年四月,長山縣王惠獲六目龜一頭,腹下有「萬歡」字,並有卦兆。



 六月,建城縣昌城田獲四目龜一頭,下有「萬齊」字。



 九年五月,長山縣獲神龜一頭,腹下有巽、兌卦。



 中興二年正月,邏將潘道蓋於山石穴中獲毛龜一頭。升明三年,世祖遣人詣宮亭湖廟還福,船泊渚,有白魚雙躍入船。



 永明五年,南豫州刺史建安王子真表獻金色魚一頭。



 建元元年八月,男子王約獲白雀一頭。九月,秣陵縣獲白雀一頭。



 二年四月,白雀集郢州府館。



 五月,白雀見會稽永興縣。



 永明元年五月,郢州丁坡屯獲白雀一頭。



 三年七月,安城王暠第獲白雀一頭。



 九月,南郡江陵縣獲白雀一頭。



 四年七月,白雀見臨汝縣。



 七年六月,鹽官縣獲白雀一頭。



 八年,天門臨澧縣獲白雀一頭。



 九年七月,吳郡錢塘縣獲白雀一頭。八月,豫州獲白雀一頭。



 十年五月,齊郡獲白雀一頭。



 建元元年五月,白烏見巴郡。



 永明四年三月,三足烏巢南安中陶縣庭。



 八年四月,陽羨縣獲白烏一頭。



 隆昌元年四月,陽羨縣獲白烏一頭。



 建元二年,江陵縣獲白鼠一頭。



 永明六年,白鼠見芳林園。



 十年九月,義陽郡獲白鼠一頭。



 永明四年,丹陽縣獲白兔一頭。



 升明元年六月,慶雲見益都。



 建元元年,世祖拜皇太子日,
 有慶雲在日邊。



 三年,華林園醴泉堂東忽有瑞雲,周圓十許丈,高下與景雲樓平,五色藻密,光彩映山,徘徊良久,行轉南行,過長船入華池。



 昇明二年,宣城臨成縣於藉山獲紫芝一枝。



 永明八年三月,陽城縣獲紫芝一株。



 隆昌元年正月,襄陽縣獲紫芝一莖。



 昇明二年四月,昌國縣徐萬年門下棠樹連理。



 九月,豫州萬歲澗廣數丈,有樹連理,隔澗騰枝相通,越壑跨水為一干。



 建元二年九月,有司奏上虞縣楓樹連理,兩根相去九尺,雙株均聳,去地九尺,合成一幹。



 故鄣縣楓樹連理,兩株相去七尺,大八圍,去地一丈,仍相合為樹,泯如一木。



 山陽縣界若邪村有一綍木,合為連理。



 淮陰縣建業寺梨樹連理。



 建康縣梨樹耀絪一本作耀攘五圍,連理六枝。



 永明元年五月,木連理生安成新喻縣。又生南梁陳縣。



 閏月,璇明殿外閣南槐樹連理。八月,鹽官縣內樂村木連理。



 二年七月,烏程縣陳文則家槿樹連理。



 七月,新冶縣槐慄二木合生異根連理,去地數尺,中央小開,上復為一。



 三年正月,安城縣榆樹二株連理。



 二月,安陽縣梓樹連理。九月,句陽縣之谷山槿樹連理,異根雙挺,共杪為一。



 十二月,永寧左郡渼木連理。四年二月,秣陵縣喬天明園中李樹連理生,高三尺五寸,兩枝別生,復高三尺,合為一干。



 五年正月,秣陵縣華僧秀園中四樹連理。



 六年四月,江寧縣北界賴鄉齊平里三成邏門外路東,太常蕭惠基園楥樹二株連理,其高相去二尺,南大北小,小者傾柯南附,合為一樹,枝葉繁茂,圓密如蓋。



 七年,江寧縣李樹二株連理,兩根相去一丈五尺。



 八年,巴陵郡樹連理四株。



 三月,武陵白沙戍槻木連理,相去五尺,俱高三尺,東西二枝,合而通柯。



 十二月,紫桑縣陶委天家樹連理。



 永明五年,山陰縣孔廣家園檉樹十二層。會稽太守隨王子隆獻之,種芳林園鳳光殿西。



 九年,秣陵縣鬥場裏安明寺有古樹,眾僧改架屋宇,伐以為薪,剖樹,木里自然有「法大德」三字。



 始興郡本無欓樹,調味有闕。世祖在郡,堂屋後忽生一株。



 昇明二年十月,甘露降建康縣。



 十一月,甘露降長山縣。十二月,甘露降彭山松樹,至九日止。



 建元元年九月,甘露降淮南郡桃石榴二樹。有司奏甘露降新汲縣王安世園樹。



 永明二年四月,甘露降南郡桐樹。



 四年二月,甘露降臨湘縣李樹。



 三月,甘露降南郡桐樹。四月,甘露降睢陽縣桃樹。



 五年四月,甘露降荊州府中閣外桐樹。



 六年,甘露降芳林園故山堂桐樹。



 九年八月,甘露降上定林寺佛堂庭,中天如雨,遍地如雪,其氣芳,其味甘,耀日舞風,至晡乃止。



 爾後頻降鐘山松樹,四十餘日乃止。



 十月,甘露降泰安陵樹。



 中興二年三月,甘露降茅山,彌漫數里。



 元徽四年三月,醴泉出昌國白鹿山,其味甚甘。



 永明元年正月,新蔡郡固始縣獲嘉禾,一莖五穗。



 八月,新蔡縣獲嘉禾,二莖九穗,一莖
 七穗。



 十一月,固始縣獲嘉禾,一莖九穗。



 二年八月,梁郡睢陽縣界野田中獲嘉禾,一莖二十三穗。



 五年九月,莒縣獲嘉禾一株。



 十年六月,海陵齊昌縣獲嘉禾,一莖六穗。



 十一年九月,睢陽縣田中獲嘉禾一株。



 昇明二年九月,建寧縣建昌村民採藥於萬歲山,忽聞澗中有異響,得銅鐘一枚,長二尺一寸,邊有古字。



 建元元年十月,涪陵郡蜑民田健所住巖間,常留雲氣,有聲響澈若龍吟,求之積歲,莫有見者。去四月二十七日,巖數里夜忽有雙光,至明往,獲古鐘一枚,又有一器名淳于,蜑人以為神物,奉祠之。



 永明四年四月,東昌縣山自比歲水以來,恆發異響,去二月十五日,有一巖褫落。縣民方元泰往視,於巖下得古鐘一枚。



 五年三月,豫寧縣長崗山獲神鐘一枚。



 九年十一月,寧蜀廣漢縣田所墾地入尺四寸,獲古鐘一枚,形高三尺八寸,圍四尺七寸,縣柄長一尺二寸,合高五尺,四面各九孔。更於陶所瓦間見有白光,窺尋無物,自後夜夜輒復有光。既經旬日,村民張慶宣瓦作屋,又於屋間見光照內外,慶宣疑之,以告孔休先,乃共發視,獲玉璽一鈕,璧方八分,上有鼻,文曰「帝真」。曲阿縣民黃慶宅左有園,園東南廣袤四丈。種菜,輒鮮異,雖加採拔,隨復更生。



 夜中恆有白光,皎質屬天,狀似縣絹,私疑非常。請師卜候,道士傅德占使掘之,深三尺,獲玉印一鈕,文曰「長承萬福」。



 永明二年正月,冠軍將軍周普孫於石頭北廂將堂見地有異光照城堞,往獲玉璽一鈕,方七分,文曰「明玄君。」十一月,虜國民齊詳歸入靈丘關,聞殷然有聲,仰視之,見山側有紫氣如云,眾鳥回翔其間。



 詳往氣所,獲璽方寸四分,獸鈕,文曰「坤
 維聖帝永昌」。送與虜太后師道人惠度,欲獻虜主。惠度睹其文,竊謂「當今衣冠正朔,在於齊國」。遂附道人惠藏送京師,因羽林監崔士亮獻之。三年七月,始興郡民龔玄宣云,去年二月,忽有一道人乞食,因探懷中出篆書真經一卷,六紙,又表北極一紙,又移付羅漢居士一紙,雲從兜率天宮下,使送上天子,因失道人所在。今年正月,玄宣又稱神人授皇帝璽,龜形,長五寸,廣二寸,厚二寸五分,上有「天地」字,中央「蕭」字,下「萬世」字。



 十年,蘭陵民齊伯生於六合山獲金璽一鈕,文曰「年予主」。



 世祖治盆城,得五尺刀一十口,永明年曆之數。



 升明三年,左里村人於宮亭湖得靫戟二枚,傍有古字,文遠不可
 識。



 泰始中,世祖於青溪宅得錢一枚,文有北斗七星雙節。又有人形帶劍。及治盆城,又得一大錢,文曰「太平百歲」。



 永明七年,齊興太守劉元寶治郡城,於塹中獲錢百萬,形極大,以獻臺為瑞,世祖班賜朝臣以下各有差。



 十年,齊安郡民王攝掘地得四文大錢一萬二千七百十枚,品制如一。



 建元元年,郢州監利縣天井湖水色忽澄清,出綿,百姓採以為纊。



 永明二年,護軍府門外桑樹一株,並有蠶絲綿被枝莖。



 史臣案:漢光武時有野蠶成繭,百姓得以成衣服。今則浮波幕樹,其亦此之類乎?



 永明八年,始興郡昌樂村獲白鳩一頭。



 二年,彭澤縣獲白雉一頭。



 七年,鬱林獲白雉一頭。



 十年,青州水臣液戍獲白雉一頭。



 五年,望蔡縣獲白鹿一頭。



 九年,臨湘獲白鹿一頭。



 六年,蒲濤縣亮野村獲白獐一頭。



 七年,荊州獲白獐一頭。



 八年,餘干縣獲白獐一頭。



 九年,義陽安昌縣獲白獐一頭。



 十年,司州清激戍獲白獐一頭。



 十一年,廣陵海陵縣獲白獐一頭。



 七年,越州獻白珠,自然作思惟佛像,長三寸。上起禪靈寺,置剎下。



 七年,吳郡太守江斅於錢塘縣獲蒼玉璧一枚以獻。



 七年,主書朱靈讓於浙江得靈石,十人舉乃起,在水深三尺而浮。世祖親投於天淵池試之,刻為佛像。



 二年,從陽丹水縣山下得古鼎一枚。



 三年,越州南高涼俚人海中網魚,獲銅獸一頭,銘曰「作寶鼎,齊臣萬年子孫承寶」。



 贊曰:天降地出,星見先吉。造物百品,詳之載述。



\end{pinyinscope}