\article{卷八本紀第八和帝}

\begin{pinyinscope}

 和帝諱寶融,字智昭,高宗第八子也。建武元年,封隨郡王,邑二千戶。三年,為冠軍將軍,領石頭戍軍事。永元元年,改封南康王,為持節,督荊、雍、益、寧、梁、南北秦七州軍事,西中郎將,荊州刺史。



 二年十一月,甲寅,長史蕭穎胄殺輔國將軍、巴西梓潼二郡太守劉山陽,奉梁王舉義。乙卯,教纂嚴。又教曰:「吾躬率晉陽,翦此凶孽,戎事方勤,宜覃澤惠。



 所領內繫囚見徒,罪無輕重,殊死已下,皆原遣。先有位署,即復本職。將吏轉一階。從征身有家口停鎮,給廩食。凡諸雜役見在諸軍帶甲之身,克定之後,悉免為民。其功效賞報,別有科條。」丙辰,以雍州刺史梁王為使持節、都督前鋒諸軍事、左將軍。丁
 巳,以蕭穎胄為右將軍、都督行留諸軍事。戊午,梁王上表勸進。十二月,乙亥,群僚勸進,並不許。壬辰,驍騎將軍夏侯亶自京師至江陵,稱宣德太后令:「西中郎將南康王宜纂承皇祚,光臨億兆。方俟清宮,未即大號,可且封宣城、南琅邪、南東海、東陽、臨海、新安、尋陽、南郡、竟陵、宜都十郡為宣城王,相國、荊州牧,加黃皞,置僚屬,選百官,西中郎府南康國並如故。須軍次近路,主者詳依舊典,法駕奉迎。」



 三年正月,乙巳,王受命,大赦,唯梅蟲兒、茹法珍等不在赦例。右將軍蕭穎胄為左長史,進號鎮軍將軍,梁王進號征東將軍。甲戌,以冠軍將軍楊公則為湘州刺史。甲寅,建牙於城南。二月,乙丑,以冠軍長史王茂先為江州刺史,冠軍將軍曹景宗為郢州刺史,右將軍邵陵王寶攸為荊州刺史。己巳,群僚上尊號,立宗廟及南北郊。甲申,梁王率大眾屯沔口,郢州刺史張沖拒守。三月,丁酉,張沖
 死,驃騎將軍薛元嗣等固城。



 中興元年春,三月,乙巳,即皇帝位,大赦,改元。文武賜位二等;鰥寡孤獨不能自存者穀,人五斛。即永元三年也。以相國左長史蕭穎胄為尚書令,晉安王寶義為司空,廬陵王寶源為車騎將軍、開府儀同三司,建安王寶夤為徐州刺史,散騎常侍夏侯詳為中領軍,領軍將軍蕭偉為雍州刺史。丙午,有司奏封庶人寶卷為零陽侯,詔不許。又奏為涪陵王,詔可。乙酉,尚書令蕭穎胄行荊州刺史,假梁王黃皞。



 壬子,以征虜將軍柳惔為益、寧二州刺史。己未,以冠軍將軍莊丘黑為梁、南秦二州刺史,冠軍將軍鄧元起為廣州刺史。夏,四月,戊辰,詔曰:「荊、雍義舉所基,實始王跡。君子勞心,細人盡力,宜加酬獎,副其乃誠。凡東討眾軍及諸向義之眾,可普復除。」



 五月,乙卯,車駕幸竹林寺禪房宴群臣。巴西太守魯休烈、巴東太守
 蕭惠訓子璝拒義軍。秋,七月,東軍主吳子陽十三軍救郢州,屯加湖。丁酉,征虜將軍王茂先擊破之。辛亥,以茂先為中護軍。丁卯,魯山城主孫樂祖以城降。己未,郢城主薛元嗣降。



 八月,丙子,平西將軍陳伯之降。乙卯,以伯之為江州刺史,子虎牙為徐州刺史。九月,乙未,詔梁王若定京邑,得以便宜從事。冬,十一月,乙未,以輔國將軍李元履為豫州刺中。壬寅,尚書令、鎮軍將軍蕭穎胄卒,以黃門郎蕭澹行荊州府州事。丁巳,蕭璝、魯休烈降。



 十二月,丙寅,建康城平。己巳,皇太后令以梁王為大司馬、錄尚書事、驃騎大將軍、揚州刺史,封建安郡公,依晉武陵王遵承制故事,百僚致敬。壬申,改封建安王寶寅鄱陽王。癸酉,以司徒、揚州刺史晉安王寶義為太尉,領司徒。甲戌,給大司馬錢二千萬,布絹各五千匹。乙酉,以輔國將軍蕭宏為中護軍。



 二年春,正月,戊戌,宣德太后臨朝,入居內殿。大司馬梁王解承制,致敬如先。己亥,以寧朔將軍蕭昺監南兗州。壬寅,以大司馬都督中外諸軍事,加殊禮。



 己酉,以大司馬長史王亮為守尚書令。甲寅,詔大司馬梁王進位相國,總百揆,揚州牧,封十郡為梁公,備九錫之禮,加遠遊冠,位在諸王上,加相國綠騑綬。己未,以新除右將軍曹景宗為郢州刺史。



 二月,壬戌,湘東王寶晊伏誅。戊辰,詔進梁公爵為梁王,增封十郡。三月,乙未,皇太后令給梁國錢五百萬,布五千匹,絹千匹。辛丑,鄱陽王寶寅奔虜,邵陵王寶攸、晉熙王寶嵩、桂陽王寶貞伏誅。甲午,命梁王冕十有二旒,建天子旌旗,出警入蹕,乘金根,駕六馬,備五時副車,置旄頭雲罕,樂舞八佾,設鐘歔宮懸。



 王子王女爵命一如舊儀。庚戌,以冠軍長史蕭秀為南徐州刺史,新除中領軍蔡道恭為司州刺史。車駕東歸至姑熟。丙辰,禪位梁王。丁
 巳,廬陵王寶源薨。夏,四月,辛酉,禪詔至,皇太后遜外宮。丁卯,梁王奉帝為巴陵王,宮於姑熟,行齊正朔,一如故事。戊辰,薨,年十五。追尊為齊和帝,葬恭安陵。



 史臣曰:夏以桀亡,殷隨紂滅,郊天改朔,理無延世。而皇符所集,重興西楚,神器暫來,雖有冥數,徽名大號,斯為幸矣。



 贊曰:和帝晚隆,掃難清宮。達機睹運,高頌永終。



\end{pinyinscope}