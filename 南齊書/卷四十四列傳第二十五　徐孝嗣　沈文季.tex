\article{卷四十四列傳第二十五 徐孝嗣 沈文季}

\begin{pinyinscope}

 徐孝嗣,字始昌,東海郯人也。祖湛之,宋司空;父聿之,著作郎:並為太初所殺。孝嗣在孕得免。幼而挺立,風儀端簡。八歲,襲爵枝江縣公,見宋孝武,升階流涕,迄于就席。帝甚愛之。尚康樂公主。泰始二年,西討解嚴,車駕還宮,孝嗣登殿不著韎,為治書御史蔡准所奏,罰金二兩。拜駙馬都尉,除著作郎,母喪去官。為司空太尉二府參軍,安成王文學。孝嗣姑適東莞劉舍,舍兄藏為尚書左丞,孝嗣往詣之。藏退語舍曰:「徐郎是令僕人,三十餘可知矣。汝宜善自結。」



 昇明中,遷太祖驃騎從事中郎,帶南彭城太守,隨府轉為太尉諮議
 參軍,太守如故。齊臺建,為世子庶子。建元初,國除,出為晉陵太守,還為太子中庶子,領長水校尉,未拜,為寧朔將軍、聞喜公子良征虜長史,遷尚書吏部郎,太子右衛率,轉長史。善趨步,閑容止,與太宰褚淵相埒。世祖深加待遇。尚書令王儉謂人曰:「徐孝嗣將來必為宰相。」轉充御史中丞。世祖問儉曰:「誰可繼卿者?」儉曰:「臣東都之日,其在徐孝嗣乎!」出為吳興太守,儉贈孝嗣四言詩曰:「方軌叔茂,追清彥輔。柔亦不茹,剛亦不吐。」時人以比蔡子尼之行狀也。在郡有能名。會王儉亡,上徵孝嗣為五兵尚書。



 其年,上敕儀曹令史陳淑、王景之、朱玄真、陳義民撰江左以來儀典,令諮受孝嗣。明年,遷太子詹事。從世祖幸方山。上曰:「朕經始此山之南,復為離宮之所。故應有邁靈丘。」靈丘山湖,新林苑也。孝嗣答曰:「繞黃山,款牛首,乃盛漢之事。今江南未廣,民亦勞止,願陛下少更留神。」上竟無所脩
 立。竟陵王子良甚善之。子良好佛法,使孝嗣及廬江何胤掌知齋講及眾僧。轉吏部尚書。尋加右軍將軍,轉領太子左衛率。臺閣事多以委之。



 世祖崩,遺詔轉右僕射。隆昌元年,遷散騎常侍、前將軍、丹陽尹。高宗謀廢鬱林,以告孝嗣,孝嗣奉旨無所釐贊。高宗入殿,孝嗣戎服隨後。鬱林既死,高宗須太后令,孝嗣於袖中出而奏之,高宗大悅。以廢立功,封枝江縣侯,食邑千戶。



 給鼓吹一部,甲仗五十人入殿。轉左僕射,常侍如故。明帝即位,加侍中、中軍大將軍。定策勳,進爵為公,增封二千戶。給班劍二十人,加兵百人。舊拜三公乃臨軒,至是帝特詔與陳顯達、王晏並臨軒拜授。



 北虜動,詔孝嗣假節頓新亭。時王晏為令,民情物望,不及孝嗣也。晏誅,轉尚書令,領本州中正,餘悉如故。孝嗣愛好文學,賞托清勝。器量弘雅,不以權勢自居,故見容建武之世。恭己自保,朝野以此稱之。



 初,孝嗣在
 率府,晝臥齋北壁下,夢兩童子遽云「移公床」。孝嗣驚起,聞壁有聲,行數步而壁崩壓床。建武四年,即本號開府儀同三司。孝嗣聞有詔,斂容謂左右曰:「吾德慚古人,位登袞職,將何以堪之。明君可以理奪,必當死請。若不獲命,正當角巾丘園,待罪家巷耳。」固讓不受。



 是時連年虜動,軍國虛乏。孝嗣表立屯田曰:「有國急務,兵食是同,一夫輟耕,於事彌切。故井陌疆里,長轂盛於周朝,屯田廣置,勝戈富於漢室。降此以還,詳略可見。但求之自古,為論則賒;即以當今,宜有要術。竊尋緣淮諸鎮,皆取給京師,費引既殷,漕運艱澀。聚糧待敵,每若苦不周,利害之基,莫此為急。臣比訪之故老及經彼宰守,淮南舊田,觸處極目,陂遏不脩,咸成茂草。平原陸地,彌望尤多。今邊備既嚴,戍卒增眾,遠資饋運,近廢良疇,士多饑色,可為嗟歎。愚欲使刺史二千石躬自履行,隨地墾闢。精尋灌溉之源,善商肥
 確之異。州郡縣戍主帥以下,悉分番附農。今水田雖晚,方事菽麥,菽麥二種,益是北土所宜,彼人便之,不減粳稻。開創之利,宜在及時。所啟允合,請即使至徐、兗、司、豫,爰及荊、雍,各當境規度,勿有所遺。別立主曹,專司其事。田器耕牛,臺詳所給。歲終言殿最,明其刑賞。此功克舉,庶有弘益。若緣邊足食,則江南自豐。權其所饒,略不可計。」事御見納。時帝已寢疾,兵事未已,竟不施行。



 帝疾甚,孝嗣入居禁中,臨崩受遺託,重申開府之命。加中書監。永元初輔政,自尚書下省出住宮城南宅,不得還家。帝失德稍彰,孝嗣不敢諫諍。及江示石見誅,內懷憂恐,然未嘗表色。始安王遙光反,眾情遑惑,見孝嗣入,宮內乃安。然群小用事,亦不能制也。進位司空,固讓。求解丹陽尹,不許。



 孝嗣文人,不顯同異,名位雖大,故得未及禍。虎賁中郎將許准有膽力,領軍隸孝嗣,陳說事機,勸行廢立。孝嗣遲疑久之,謂必
 無用干戈理,須少主出遊,閉城門召百僚集議廢之,雖有此懷,終不能決。群小亦稍憎孝嗣,勸帝召百僚集議,因誅之。冬,召孝嗣入華林省,遣茹法珍賜藥,孝嗣容色不異,少能飲酒,藥至斗餘,方卒。乃下詔曰:「周德方熙,三監迷叛,漢歷載昌,宰臣構戾,皆身膏斧鉞,族同煙燼。殷鑒上代,垂戒後昆。徐孝嗣憑藉世資,早蒙殊遇,階緣際會,遂登臺鉉。匡翼之誠無聞,諂黷之跡屢著。沈文季門世。
 此下缺沈文季,字仲達,吳興武康人。父慶之,宋司空。文季少以寬雅正直見知。孝建二年,起家主簿,徵秘書郎。以慶之勳重,大明五年,封文季為山陽縣五等伯。



 轉太子舍人,新安王北中郎主簿,西陽王撫軍功曹,江夏王太尉東曹掾,遷中書郎。



 慶之為景和所殺,兵仗圍宅,收捕諸子。文季長兄文叔謂文季曰:「我能死,爾能報。」遂自縊。文季揮刀馳馬去,收者不敢追,遂得免。



 明帝立,起文季為寧朔將軍,
 遷太子右衛率,建安王司徒司馬。赭圻平,為宣威將軍,廬江王太尉長史。出為寧朔將軍、征北司馬、廣陵太守。轉黃門郎,領長水校尉。明帝宴會朝臣,以南臺御史賀咸為柱下史,糾不醉者。文季不肯飲酒,被驅下殿。



 晉平王休祐為南徐州,帝問褚淵須幹事人為上佐,淵舉文季。轉寧朔將軍、驃騎長史、南東海太守。休祐被殺,雖用薨禮,僚佐多不敢至,文季獨往省墓展哀。



 出為臨海太守。元徽初,遷散騎常侍,領後軍將軍,轉秘書監。出為吳興太守。文季飲酒至五斗,妻王氏,王錫女,飲酒亦至三斗。文季與對飲竟日,而視事不廢。



 昇明元年,沈攸之反,太祖加文季為冠軍將軍,督吳興錢塘軍事。攸之先為景和銜使殺慶之。至是文季收殺攸之弟新安太守登之,誅其宗族。加持節,進號征虜將軍,改封略陽縣侯,邑千戶。明年,遷丹陽尹,將軍如故。



 齊國初建,為侍中,領秘書監。建元元年,
 轉太子右衛率,侍中如故。改封西豐縣侯,食邑千二百戶。



 文季風采棱岸,善於進止。司徒褚淵當世貴望,頗以門戶裁之,文季不為之屈。



 世祖在東宮,於玄圃宴會朝臣。文季數舉酒勸淵,淵甚不平,啟世祖曰:「沈文季謂淵經為其郡,數加淵酒。」文季曰:「惟桑與梓,必恭敬止。豈如明府亡國失土,不識枌榆。」遂言及虜動,淵曰:「陳顯達、沈文季當今將略,足委以邊事。」文季諱稱將門,因是發怒,啟世祖曰:「褚淵自謂是忠臣,未知身死之日,何面目見宋明帝?」世祖笑曰:「沈率醉也。」中丞劉休舉其事,見原。後豫章王北宅後堂集會,文季與淵並善琵琶,酒闌,淵取樂器為《明君曲》。文季便下席大唱曰:「沈文季不能作伎兒。」豫章王嶷又解之曰:「此故當不損仲容之德。」淵顏色無異,曲終而止。



 文季尋除征虜將軍,侍中如故,遷散騎常侍,左衛將軍,征虜如故。世祖即位,轉太子詹事,常侍如故。永明元
 年,出為左將軍、吳郡太守。三年,進號平東將軍。



 四年,遷會稽太守,將軍如故。



 是時連年檢籍,百姓怨望。富陽人唐宇之僑居桐廬,父祖相傳圖墓為業。宇之自云其家墓有王氣,山中得金印,轉相誑惑。三年冬,宇之聚黨四百人於新城水斷商旅,黨與分布近縣。新城令陸赤奮、桐廬令王天愍棄縣走。宇之向富陽,抄略人民,縣令何洵告魚浦子邏主從系公,發魚浦村男丁防縣。永興遣西陵戍主夏侯曇羨率將吏及戍左右埭界人起兵赴救。宇之遂陷富陽。會稽郡丞張思祖遣臺使孔矜、王萬歲、張繇等配以器仗將吏白丁,防衛永興等十屬。文季亦遣器仗將吏救援錢塘。



 宇之至錢塘,錢塘令劉彪、戍主聶僧貴遣隊主張玕於小山拒之,力不敵,戰敗。宇之進抑浦登岸,焚郭邑,彪棄縣走。文季又發吳、嘉興、海鹽、鹽官民丁救之。賊分兵出諸縣,鹽官令蕭元蔚、諸暨令陵琚之並逃走,
 餘杭令樂琰戰敗乃奔。是春,宇之於錢塘僭號,置太子,以新城戍為天子宮,縣廨為太子宮。弟紹之為揚州刺史。



 錢塘富人柯隆為尚書僕射、中書舍人,領太官令,獻鋌數千口為宇之作杖,加領尚方令。分遣其黨高道度徐寇東陽,東陽太守蕭崇之、長山令劉國重拒戰見害。崇之字茂敬,太祖族弟。至是臨難,貞正果烈。追贈冠軍將軍,太守如故。賊遂據郡。



 又遣偽會稽太守孫泓取山陰。時會稽太守王敬則朝正,故宇之謂乘虛可襲。泓至浦陽江,郡丞張思祖遣浹口戍主湯休武拒戰,大破之。上在樂遊苑,聞宇之賊,謂豫章王嶷曰:「宋明初,九州同反,鼠輩但作,看蕭公雷汝頭。」遣禁兵數千人,馬數百匹東討。賊眾烏合,畏馬。官軍至錢塘,一戰便散,禽斬宇之,進兵平諸郡縣。



 臺軍乘勝,百姓頗被抄奪。軍還,上聞之,收軍主前軍將軍陳天福棄市,左軍將軍中宿縣子劉明徹免官削爵
 付東冶。天福,上寵將也,既伏誅,內外莫不震肅。



 天福善馬槊,至今諸將法之。



 御史中丞徐孝嗣奏曰:「風聞山東群盜,剽掠列城,雖匪日而殄,要暫幹王略。



 郡縣闕攻守之宜,倉府多侵秏之弊,舉善懲惡,應有攸歸。吳郡所領鹽官令蕭元蔚、桐廬令王天愍、新城令陸赤奮等,縣為首劫破掠,並不經格戰,委職散走。元蔚、天愍還臺,赤奮不知所在。又錢塘令劉彪、富陽令何洵,乃率領吏民拒戰不敵,未委歸臺。餘建德、壽昌在劫斷上流,不知被劫掠不?吳興所領餘杭縣被劫破,令樂琰乃率吏民徑戰不敵,委走出都。會稽所領諸暨縣,為劫所破,令陵琚之不經格戰,委城奔走,不知所在。案元蔚等妄藉天私,作司近服,昧斯隱慝,職啟虔劉。會稽郡丞張思祖謬因承乏,總任是尸,涓誠芻效,終焉無紀。平東將軍吳郡太守文季、征虜將軍吳興太守西昌侯鸞,任屬關、河,威懷是寄。輒下
 禁止彪、琰、洵,思祖、文季視事如故,鸞等結贖論。」詔元蔚等免,思祖、鸞、文季原。



 文季固讓會稽之授,轉都官尚書,加散騎常侍。出為持節、督郢州司州之義陽諸軍事、左將軍、郢州刺史,還為散騎常侍,領軍將軍。世祖謂文季曰:「南士無僕射,多歷年所。」文季對曰:「南風不競,非復一日。」文季雖不學,發言必有辭採,當世稱其應對。尤善釐及彈棋,釐用五子。



 以疾遷金紫光祿大夫,加親信二十人,常侍如故。轉侍中,領太子詹事,遷中護軍,侍中如故。以家為府。隆昌元年,復為領軍將軍,侍中如故。豫廢鬱林,高宗欲以文季為江州,遣左右單景雋宣旨,文季口自陳讓,稱年老不願外出,因問右執法有人未,景雋還具言之。延興元年,遷尚書右僕射。



 明帝即位,加領太子詹事,增邑五百戶。尚書令王晏嘗戲文季為吳興僕射。文季答曰:「瑯邪執法,似不出卿門。」尋加散騎常侍,僕射如故。建武二年,虜寇壽
 春,豫州刺史豐城公遙昌嬰城固守,數遣輕兵相抄擊,明帝以為憂,詔文季領兵鎮壽春。文季入城,止游兵不聽出,洞開城門,嚴加備守,虜軍尋退,百姓無所傷損。增封為千九百戶。尋加護軍將軍,僕射、常侍如故。



 王敬則反,詔文季領兵屯湖頭,備京路。永元元年,轉侍中、左僕射,將軍如故。始安王遙光反,其夜,遣三百人於宅掩取文季,欲以為都督,而文季已還臺。



 明日,與尚書令徐孝嗣守衛宮城,戎服共坐南掖門上。時東昏已行殺戮,孝嗣深懷憂慮,欲與文季論世事,文季輒引以他辭,終不得及。事寧,加鎮軍將軍,置府。



 侍中、僕射如故。



 文季見世方昏亂,托以老疾,不豫朝機。兄子昭略謂文季曰:「阿父年六十為員外僕射,欲求自免,豈可得乎?」文季笑而不答。同孝嗣被害。其日先被召見,文季知敗,舉動如常,登車顧曰:「此行恐往而不反也。」於華林省死,時年五十八。朝野冤之。中
 興元年,贈侍中、司空,謚忠憲。



 兄子昭略,有剛氣。升明末為相國西曹掾,太祖賞之,及即位,謂王儉曰:「南士中有沈昭略,何職處之?」儉曰:「臣已有擬。」奏轉前軍將軍,上不欲違,可其奏。尋遷為中書郎。永明初,歷太尉大司馬從事中郎,驃騎司馬,黃門郎。南郡王友、學華選,以昭略為友,尋兼左丞。元年,出為臨海太守,御史中丞。累遷侍中,冠軍將軍,撫軍長史。永元元年,始安王遙光起兵東府,執昭略於城內。昭略潛自南出,濟淮還臺。至是與文季俱被召入華林省。茹法珍等進藥酒,昭略怒罵徐孝嗣曰:「廢昏立明,古今令典。宰相無才,致有今日。」以甌擲面破,曰「作破面鬼」。死時年四十餘。



 弟昭光,聞收至,家人勸逃去,昭光不忍捨母,遂見獲,殺之。中興元年,贈昭略太常,昭光廷尉。



 史臣曰:為邦之訓,食惟民天,足食足兵,民信之矣。屯田之略,實重
 戰守。



 若夫充國耕殖,用殄羌戎,韓浩、棗祇,亦建華夏置典農之官,興大佃之議。金城布險,峻壘綿疆,飛芻輓粒,事難支繼。一夫不耕,或鐘饑餒,緣邊戍卒,坐甲千群。故宜盡收地利,因兵務食。緩則躬耕,急則從戰。歲有餘糧,則紅食可待。前世達治,言之已詳。江左以來,不暇遠策,王旅外出,未嘗宿飽,四郊嬰守,懼等松芻。縣兵所救,經歲引日,凌風泙水,轉漕艱長。傾窖底之儲,盡倉敖之粟,流馬木牛,尚深前弊,田積之要,唯在江淮。郡國同興,遠不周急。故吳氏列戍南濱,屯農水右,魏世淮北大佃,而石橫開漕,皆輔車相資,易以待敵。孝嗣當蹙境之晨,薦希行之計,王無外略,民困首領,觀機而動,斯儀殆為空陳,惜矣!



 贊曰:文忠作相,器範先標。有容有業,可以立朝。豐城歷仕,音儀孔昭。為舟等溺,在運同消。



\end{pinyinscope}