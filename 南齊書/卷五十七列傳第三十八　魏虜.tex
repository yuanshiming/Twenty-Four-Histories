\article{卷五十七列傳第三十八 魏虜}

\begin{pinyinscope}

 魏虜,匈
 奴種也,姓托跋氏。晉永嘉六年,并州刺史劉琨為屠各胡劉聰所攻,索頭猗盧遣子曰利孫將兵救琨於太原,猗盧入居代郡,亦謂鮮卑。被髮左衽,故呼為索頭。猗盧孫什翼犍,字鬱律旃,後還陰山為單于,領匈奴諸部。太元元年,苻堅遣偽并州刺史苻洛伐犍,破龍庭,禽犍還長安,為立宅,教犍書學。分其部黨居雲中等四郡,諸部主帥歲終入朝,并得見犍,差稅諸部以給之。



 堅敗,子圭,字涉圭,隨舅慕容垂據中山,還領其部,後稍強盛。隆安元年,珪破慕容寶於中山,遂有并州,僭稱魏,年號天賜。追謚犍烈祖文平皇帝。珪死,謚道武皇帝。子木末立,年號太常,死,謚明元皇帝。子壽,字
 佛狸,代立,年號太平真君。宋元嘉中,偽太子晃與大臣崔氏、寇氏不睦,崔、寇譖之。玄高道人有道術,晃使祈福七日七夜,佛貍夢其祖父並怒,手刃向之曰:「汝何故信讒欲害太子!」佛狸驚覺,下偽詔曰:「王者大業,纂承為重,儲宮嗣紹,百王舊例。自今已往,事無巨細,必經太子,然後上聞。」晃後謀殺佛貍見殺。壽死,謚太武皇帝。



 立晃子浚,字烏雷直勤,年號和平。追謚晃景穆皇帝。浚死,謚文成皇帝。子弘,字萬民,立,年號天安。景和九年,偽太子宏生,改年為皇興。



 什翼珪始都平城,猶逐水草,無城郭,木末始土著居處。佛貍破梁州、黃龍,徙其居民,大築郭邑。截平城西為宮城,四角起樓,女墻,門不施屋,城又無塹。



 南門外立二土門,內立廟,開四門,各隨方色,凡五廟,一世一間,瓦屋。其西立太社。佛狸所居雲母等三殿,又立重屋,居其上。飲食廚名「阿真廚」,在西,皇后可孫恆出此廚求食。初,
 姚興以塞外虜赫連勃勃為安北將軍,領五部胡,屯大城,姚泓敗後,入長安。佛狸攻破勃勃子昌,娶勃勃女為皇后。義熙中,仇池公楊盛表云「索虜勃勃,匈奴正胤」是也。可孫昔妾媵之。殿西鎧仗庫屋四十餘間,殿北絲綿布絹庫土屋一十餘間。偽太子宮在城東,亦開四門,瓦屋,四角起樓。妃妾住皆土屋。婢使千餘人,織綾錦販賣,酤酒,養豬羊,牧牛馬,種菜逐利。太官八十餘窖,窖四千斛,半谷半米。又有懸食瓦屋數十間,置尚方作鐵及木。其袍衣,使宮內婢為之。偽太子別有倉庫。其郭城繞宮城南,悉築為坊,坊開巷。坊大者容四五百家,小者六七十家。每南坊搜檢,以備奸巧。城西南去白登山七里,於山邊別立父祖廟。城西有祠天壇,立四十九木人,長丈許,白幘、練裙、馬尾被,立壇上,常以四月四日殺牛馬祭祀,盛陳鹵簿,邊壇奔馳奏伎為樂。城西三里,刻石寫《五經》及其國記,於鄴取石
 虎文石屋基六十枚,皆長丈餘,以充用。



 國中呼內左右為「直真」,外左右為「烏矮真」,曹局文書吏為「比德真」,簷衣人為「樸大真」,帶仗人為「胡洛真」,通事人為「乞萬真」,守門人為「可薄真」,偽臺乘驛賤人為「拂竹真」,諸州乘驛人為「咸真」,殺人者為「契害真」,為主出受辭人為「折潰真」,貴人作食人為「附真」。三公貴人,通謂之「羊真」。



 佛狸置三公、太宰、尚書令、僕射、侍中,與太子共決國事。殿中尚書知殿內兵馬倉庫,樂部尚書知伎樂及角史伍柏,駕部尚書知牛馬驢騾,南部尚書知南邊州郡,北部尚書知北邊州郡。又有俟勤地何,比尚書;莫堤,比刺史;郁若,比二千石;受別官比諸侯。諸曹府有倉庫,悉置比官,皆使通虜、漢語,以為傳驛。蘭臺置中丞御史,知城內事。又置九豆和官,宮城三里內民戶籍不屬諸軍戍者,悉屬之。



 其車服,有大小輦,皆五層,下施四輪,三二百人牽之,四施絙索,備傾倒。



 軺車
 建龍旂,尚黑。妃后則施雜彩憲,無幢絡。太后出,則婦女著鎧騎馬近輦左右。虜主及后妃常行,乘銀鏤羊車,不施帷幔,皆偏坐垂腳轅中,在殿上亦跂據。



 正殿施流蘇帳,金博山,龍鳳朱漆畫屏風,織成幌。坐施氍毹褥。前施金香爐,琉璃缽,金碗,盛雜食器。設客長盤一尺,御饌圓盤廣一丈。為四輪車,元會日六七十人牽上殿。蠟日逐除,歲盡,城門磔雄雞,葦索桃梗,如漢儀。



 自佛狸至萬民,世增雕飾。正殿西築土臺,謂之白樓。萬民禪位後,常游觀其上。臺南又有伺星樓。正殿西又有祠屋,琉璃為瓦。宮門稍覆以屋,猶不知為重樓。



 並設削泥采,畫金剛力士。胡俗尚水,又規畫黑龍相盤繞,以為厭勝。



 泰始五年,萬民禪位子宏,自稱太上皇。宏立,號延興元年。至六年,萬民死,謚獻文皇帝。改號為承明元年,是歲元徽四年也。祖母馮氏,黃龍人,助治國事。



 初,佛狸母是漢人,為木末所殺,佛狸以
 乳母為太后。自此以來,太子立,輒誅其母。一云馮氏本江都人,佛狸元嘉二十七年南侵,略得馮氏,浚以為妾,獨得全焉。



 明年丁巳歲,改號太和。



 宋明帝末年,始與虜和好。元徽升明之世,虜使歲通。建元元年,偽太和三年也。宏聞太祖受禪,其冬,發眾遣丹陽王劉昶為太師,寇司、豫二州。明年,詔遣眾軍北討。宏遣大將郁豆眷、騕長命攻壽陽及鐘離,為豫州刺史垣崇祖、右將軍周盤龍、徐州刺史崔文仲等所破。宏又遣偽南部尚書托跋等向司州,分兵出兗、青界,十萬眾圍朐山,戍主玄元度嬰城固守。青冀二州刺史盧紹之遣子奐領兵助之。城中無食,紹之出頓州南石頭亭,隔海運糧柴供給城內。虜圍斷海道,緣岸攻城,會潮水大至,虜褭溺,元度出兵奮擊,大破之。臺遣軍主崔靈建、楊法持、房靈民萬餘人從淮入海,船艦至夜各舉兩火,虜眾望見,謂
 是南軍大至,一時奔退。



 初,元度自云臂上有封侯志,宋世以示世祖,時世祖在東宮,書與元度曰:「努力成臂上之相也。」虜退,上議加封爵,元度歸功於紹之,紹之又讓,故並見寢。上乃擢紹之為黃門郎。鬱州呼石頭亭為平虜亭。紹之字子緒,范陽人,自云盧諶玄孫。宋大明中,預攻廣陵,勳上,紹之拔迹自投,上以為州治中,受心腹之任。



 官至光祿大夫。永明八年卒。



 三年,領軍將軍李安民、左軍將軍孫文顯與虜軍戰於淮陽,大敗之。初,虜寇至,緣淮驅略,江北居民猶懲佛貍時事,皆驚走,不可禁止。乃於梁山置一軍,南置三軍,慈姥置一軍,洌州置二軍,三山置二軍,白沙洲置一軍,蔡州置五軍,長蘆置三軍,菰浦置二軍,徐浦置一軍,內外悉班階賞,以示威刑。



 偽昌黎王馮莎向司州,荒人桓天生說莎云:「諸蠻皆響應。」莎至,蠻竟不動。



 莎大怒,於淮邊獵而去。及壽春摧敗,朐山不拔,虜主出定州,大治道路。聲欲南行,
 不敢進。乃與偽梁郡王計曰:「兵出彭、泗間,無復鬥志,要當一兩戰得還歸。」



 既於淮陽被破,一時奔走。青、徐間赴義民,先是或抄虜運車,更相殺掠,往往得南歸者數千家。



 上未遑外略,以虜既摧破,且欲示以威懷,遣後軍參軍車僧朗北使。虜問僧朗曰:「齊輔宋日淺,何故便登天位?」僧朗曰:「虞、夏登庸,親當革禪;魏、晉匡輔,貽厥子孫。豈二聖促促於天位,兩賢謙虛以獨善?時宜各異,豈得一揆?茍曰事宜,故屈己應物。」虜又問:「齊主悉有何功業?」僧朗曰:「主上聖性寬仁,天識弘遠。少為宋文皇所器遇,入參禁旅。泰始之初,四方寇叛,東平劉子房、張淹,北討薛索兒,兼掌軍國,豫司顧命。宋桂陽、建平二王阻兵內侮,一麾殄滅。



 蒼梧王反道敗德,有過桀、紂,遠遵伊、霍,行廢立之事。袁粲、劉秉、沈攸之同惡相濟,又秉旄杖鉞,大定兇黨。戮力佐時,四十餘載,經綸夷險,十五六年,此功此德,可謂物無異
 議。」虜又問:「南國無復齊土,何故封齊?」僧朗曰:「營丘表海,實為大國。宋朝光啟土宇,謂是呂尚先封。今淮海之間,自有青、齊,非無地也。」又問:「蒼梧何故遂加斬戮?」僧朗曰:「蒼梧暴虐,書契未聞,武王斬紂,懸之黃鉞,共是所聞,何傷於義?」昇明中,北使殷靈誕、茍昭先在虜,聞太祖登極,靈誕謂虜典客曰:「宋魏通好,憂患是同。宋今滅亡,魏不相救,何用和親?」及虜寇豫州,靈誕因請為劉昶司馬,不獲。僧朗至北,虜置之靈誕下,僧朗立席言曰:「靈誕昔是宋使,今成齊民。實希魏主以禮見處。」靈誕交言,遂相忿詈,調虜曰:「使臣不能立節本朝,誠自慚恨。」劉昶賂客解奉君於會刺殺僧朗,虜即收奉君誅之,殯斂僧朗,送喪隨靈誕等南歸,厚加贈賻。世祖踐阼,昭先具以啟聞,靈誕下獄死,贈僧朗散騎侍郎。



 永明元年冬,遣驍騎將軍劉纘、前軍將軍張謨使虜。明年冬,虜使李道固報聘,世祖於玄
 武湖水步軍講武,登龍舟引見之。自此歲使往來,疆場無事。



 三年,初令鄰里黨各置一長,五家為鄰,五鄰為里,五里為黨。四年,造戶籍。



 分置州郡,雍州、涼州、秦州、沙州、涇州、華州、岐州、河州、西華州、寧州、陜州、洛州、荊州、郢州、北豫州、東荊州、南豫州、西兗州、東兗州、南徐州、東徐州、青州、齊州、濟州二十五州在河南;相州、懷州、汾州、東雍州、肆州、定州、瀛州、朔州、并州、冀州、幽州、平州、司州十三州在河北。凡分魏、晉舊司、豫、青、兗、冀、並、幽、秦、雍、涼十州地,及宋所失淮北為三十八州矣。



 明年,邊人桓天生作亂,虜遣步騎萬餘人助之,至比陽,為征虜將軍戴僧靜等所破。荒人胡丘生起義懸瓠,為虜所擊,戰敗南奔。偽安南將軍遼東公、平南將軍上谷公又攻舞陰,舞陰戍主輔國將軍殷公愍拒破之。六年,虜又遣眾助桓天生,與輔國將軍曹虎戰,大敗於隔城。至七年,遣使邢產、侯靈紹復通好。先
 是劉纘再使虜,太后馮氏悅而親之。馮氏有計略,作《皇誥》十八篇,偽左僕射李思沖稱史臣注解。是歲,馮氏死。八年,世祖還隔城所俘獲二千餘人。



 佛貍已來,稍僭華典,胡風國俗,雜相揉亂。宏知談義,解屬文,輕果有遠略。



 遊河北至比干墓,作《弔比干文》云:「脫非武發,封墓誰因?鳴呼介士,胡不我臣!」宏以己巳歲立圓丘、方澤,置三夫人、九嬪。平城南有乾水,出定襄界,流入海,去城五十里,世號為索干都。土氣寒凝,風砂恆起,六月雨雪。議遷都洛京。



 九年,遣使李道固、蔣少游報使。少游有機巧,密令觀京師宮殿楷式。清河崔元祖啟世祖曰「少游,臣之外甥,特有公輸之思。宋世陷虜,處以大匠之官。今為副使,必欲模範宮闕。豈可令氈鄉之鄙,取象天宮?臣謂且留少游,令使主反命。」



 世祖以非和通意,不許。少游,安樂人。虜宮室制度,皆從其出。



 初,佛貍討羯胡於長安,殺道人且盡。及元
 嘉南寇,獲道人,以鐵籠盛之。後佛貍感惡疾,自是敬畏佛教,立塔寺浮圖。宏父弘禪位後,黃冠素服,持戒誦經,居石窟寺。宏太和三年,道人法秀與茍兒王阿辱佩玉等謀反,事覺,囚法秀,加以籠頭鐵鎖,無故自解脫,虜穿其頸骨,使咒之曰:「若復有神,當令穿肉不入。」



 遂穿而殉之,三日乃死。偽咸陽王復欲盡殺道人,太后馮氏不許。宏尤精信,粗涉義理,宮殿內立浮圖。



 宏既經古洛,是歲下偽詔尚書思慎曰:「夫覆載垂化,必由四氣運其功,曦曜望舒,亦須五星助其暉。仰惟聖母,睿識自天,業高曠古,將稽詳典範,日新皇度。



 不圖罪逆招禍,奄丁窮罰,追惟罔極,永無逮及。思遵先旨,敕造明堂之樣。卿所制體含六合,事越中古,理圓義備,可軌之千載。信是應世之材,先固之器也。群臣瞻見模樣,莫不僉然欲速造,朕以寡昧,亦思造盛禮。卿可即於今歲停宮城之作,營建此構。興皇代之奇
 制,遠成先志,近副朕懷。」又詔公卿參定刑律。又詔罷臈前儺,唯年一儺。又詔:「季冬朝賀,典無成文,以褲褶事非禮敬之謂,若置寒朝服,徒成煩濁,自今罷小歲賀,歲初一賀。」又詔:「王爵非庶姓所僭,伯號是五等常秩。烈祖之胄,仍本王爵,其餘王皆為公,公轉為侯,侯即為伯,子男如舊。



 雖名易於本,而品不異昔。公第一品,侯第二品,伯第三品,子第四品,男第五品。



 十年,上遣司徒參軍蕭琛、范雲北使。宏西郊,即前祠天壇處也。宏與偽公卿從二十餘騎戎服繞壇,宏一周,公卿七匝,謂之蹋壇。明日,復戎服登壇祠天,宏又繞三匝,公卿七匝,謂之繞天。以繩相交絡,紐木枝棖,覆以青繒,形制平圓,下容百人坐,謂之為傘,一云「百子帳」也。於此下宴息。次祠廟及布政明堂,皆引朝廷使人觀視。每使至,宏親相應接,申以言義。甚重齊人,常謂其臣下曰:「江南多好臣。」偽侍臣李元凱對曰:「江南多好臣,
 歲一易主;江北無好臣,而百年一主。」宏大慚,出元凱為雍州長史,俄召復職。



 世祖初,治白下,謂人曰:「我欲以此城為上頓處。」後於石頭造露車三千乘,欲步道取彭城,形跡頗著。先是八年北使顏幼明、劉思敩反命,偽南部尚書李思沖曰:「二國之和,義在庇民。如聞南朝大造舟車,欲侵淮、泗,推心相期,何應如此?」幼明曰:「主上方弘大信於天下,不失臣妾。既與輯和,何容二三其德?疆場之言,差不足信。且朝廷若必赫怒,使守在外,亦不近相淮濆。」思沖曰:「我國之強,經略淮東,何患不蕩海東岳,政存於信誓耳。且和好既結,豈可復有不信?



 昔華元、子反,戰伐之際,尚能以誠相告,此意良慕也。」幼明曰:「卿未有子反之急,詎求登床之請?」



 是後宏亦欲南侵徐、豫,於淮、泗間大積馬芻。十一年,遣露布並上書,稱當南寇。世祖發揚、徐州民丁,廣設召募。北地人支酉聚數千人,於長安城北西山
 起義。遣使告梁州刺史陰智伯。秦州人王度人起義應酉,攻獲偽刺史劉藻,秦、雍間七州民皆響震,眾至十萬,各自保壁,望朝廷救其兵。宏遣弟偽河南王乾、尚書盧陽烏擊秦、雍義軍,幹大敗。酉迎戰,進至咸陽北濁谷,圍偽司空長洛王繆老生,合戰,又大破之,老生走還長安。梁州刺史陰智伯遣軍主席德仁、張弘林等數千人應接酉等,進向長安,所至皆靡。



 會世祖崩,宏聞關中危急,乃稱聞喪退師。太和十七年八月,使持節、安南大將軍、都督徐青齊三州諸軍事、南中郎將、徐州刺史、廣陵侯府長史、帶淮陽太守鹿樹生移齊兗州府長史府:「奉被行所尚書符騰詔:皇師電舉,搖旆南指,誓清江祲,志廓衡靄。以去月下旬,濟次河洛。前使人邢巒等至,審知彼有大艾。以《春秋》之義,聞喪寢伐。爰敕有司,輟鑾止軔,休馬華陽,戢戈嵩北。便肇經周制,光宅中區,永皇基於無窮,恢盛業
 乎萬祀。宸居重正,鴻化增新,四海承休,莫不銘慶。故以往示如律令。」并遣使弔國諱。遣偽大將楊大眼、張聰明等數萬人攻酉,酉、廣等並見殺。



 隆昌元年,遣司徒參軍劉敩、車騎參軍沈宏報使至北。宏稱字玄覽。其夏,虜平北將軍魯直清率眾降,以為督洛州軍事,領平戎校尉、征虜將軍、洛州刺史。是歲,宏徙都洛陽,改姓元氏。初,匈奴女名托跋,妻李陵,胡俗以母名為姓,故虜為李陵之後,虜甚諱之,有言其是陵後者,輒見殺,至是乃改姓焉。



 宏聞高宗踐阼非正,既新移都,兼欲大示威力,是冬,自率大眾分寇豫、徐、司、梁四州。遣偽荊州刺史薛真度、尚書郗祁阿婆出南陽,向沙堨,築壘開溝,為南陽太守房伯玉、新野太守劉思忌所破。



 建武二年春,高宗遣鎮南將軍王廣之出司州,右僕射沈文季出豫州,左衛將軍崔慧景出徐州。宏自率眾至壽陽,軍中有黑氈行殿,容二十人坐,輦邊皆三
 郎曷刺真,槊多白真毦,鐵騎為群,前後相接。步軍皆烏楯槊,綴接以黑蝦蟆幡。牛車及驢、駱駝載軍資妓女,三十許萬人。不攻城,登八公山,賦詩而去。別圍鐘離城,徐州刺史蕭惠休、輔國將軍申希祖拒守,出兵奮擊,宏眾敗,多赴淮死。乃分軍據邵陽州,柵斷水路,夾築二城。右衛將軍蕭坦之遣軍主裴叔業攻二城,拔之。惠休又募人出燒虜攻城車,虜力竭不能剋。



 王奐之誅,子肅奔虜,宏以為鎮南將軍、南豫州刺史。遣肅與劉昶號二十萬眾,圍義陽。司州刺史蕭誕拒戰,虜築圍塹柵三重,燒居民凈盡,并力攻城,城中負盾而立。王廣之都督救援,虜遣三萬餘人逆攻太子右率蕭季敞於下梁,季敞戰不利。



 司州城內告急,王廣之遣軍主黃門侍郎梁王間道先進,與太子右率蕭誄、輔國將軍徐玄慶、荊州軍主魯休烈據賢首山,出虜不備。城內見援軍至,蕭誕遣長史王伯瑜及軍主
 崔恭祖出攻虜柵,因風放火,梁王等眾軍自外擊之,昶、肅棄圍引退,追擊破之。



 輔國將軍桓和出西陰平,偽魯郡公郯城戍主帶莫樓、偽東海太守江道僧設伏路側,和與合戰,大敗之。青、徐民降者百餘家。青、冀二州刺史王洪範遣軍主崔延攻虜紀城,並拔之。宏先又遣偽尚書盧陽烏、華州刺史韋靈智攻赭陽城,北襄城太守成公期拒守。虜攻城百餘日,設以鉤沖,不捨晝夜,期所殺傷數千人。臺又遣軍主垣歷生、蔡道貴救援,陽烏等退,官軍追擊破之。夏,虜又攻司州櫟城二戍,戍主魏僧岷、朱僧起拒敗之。



 偽安南將軍、梁州刺史魏郡王元英十萬餘人通斜谷,寇南鄭。梁州刺史蕭懿遣軍主姜山安、趙超宗等數軍萬餘人,分據角弩、白馬、沮水拒戰,大敗。英進圍南鄭,土山沖車,晝夜不息。懿率東從兵二千餘人固守拒戰,隨手摧卻。英攻城自春至夏六十餘日不下,死傷甚眾,軍牛
 糧盡,搗曲為食,畜菜葉直千錢。懿先遣軍主韓嵩等征獠,回軍援州城,至黃牛川,為虜所破。懿遣氐人楊元秀還仇池,說氐起兵斷虜運道,氐即舉眾攻破虜歷城、崿蘭、駱谷、仇池、平洛、蘇勒六戍。偽尚書北梁州刺史辛黑末戰死。英遣軍副仇池公楊靈珍據泥公山,武興城主楊集始遣弟集朗與歸國氐楊馥之及義軍主徐曜甫迎戰於黃亙,大敗奔歸。時梁州土豪范凝、梁季群於家請英設會,伏兵欲殺英,事覺,英執季群殺之,凝竄走。英退保濁水,聞氐眾盛,與楊靈珍復俱退入斜谷,會天大雨,軍馬含漬,截竹煮米,於馬上持炬炊而食,英至下辨,靈珍弟婆羅阿卜珍反,襲擊,英眾散,射中英頰。偽陵江將軍悅楊生領鐵騎死戰救之,得免。梁、漢平。武都太守杜靈瑗、奮武將軍望法憘、寧朔將軍望法泰、州治中皇甫耽並拒虜戰死。追贈靈瑗、法憘羽林監,法泰積射將軍。



 時偽洛州刺
 史賈異寇甲口,為上洛太守李靜所破。三年,虜又攻司州櫟城,為戍主魏僧岷所拒破。秋,虜遣軍襲漣口,東海太守鄭延祉棄西城走,東城猶固守,臺遣冠軍將軍兗州刺史徐玄慶救援,虜引退,延祉伏罪。



 初,偽太后馮氏兄昌黎王馮莎二女,大馮美而有疾,為尼;小馮為宏皇后,生偽太子詢。後大馮疾差,宏納為昭儀。宏初徙都,詢意不樂,思歸桑乾。宏制衣冠與之,詢竊毀裂,解發為編服左衽。大馮有寵,日夜讒詢。宏出鄴城馬射,詢因是欲叛北歸,密選宮中御馬三千匹置河陰渚。皇后聞之,召執詢,馳使告宏,宏徙詢無鼻城,在河橋北二里,尋殺之,以庶人禮葬。立大馮為皇后,便立偽太子恪,是歲,偽太和二十年也。



 偽征北將軍恆州刺史鉅鹿公伏鹿孤賀鹿渾守桑乾,宏從叔平陽王安壽戍懷柵,在桑乾西北。渾非宏任用中國人,與偽定州刺史馮翊公目鄰、安樂公托跋阿乾兒謀
 立安壽,分據河北。期久不遂,安壽懼,告宏。殺渾等數百人,任安壽如故。



 先是偽荊州刺史薛真度、尚書卻祁阿婆為房伯玉所破,宏怒,以南陽小郡,誓取滅之。四年,自率軍向雍州。宏先至南陽,房伯玉嬰城拒守。宏從數萬騎,罩黃傘,去城一里。遣偽中書舍人公孫云謂伯玉曰:「我今蕩一六合,與先行異。先行冬去春還,不為停久;今誓不有所剋,終不還北,停此或三五年。卿此城是我六龍之首,無容不先攻取。遠一年,中不過百日,近不過一月,非為難殄。若不改迷,當斬卿首,梟之軍門。闔城無貳,幸可改禍為福。但卿有三罪,今令卿知。卿先事武帝,蒙在左右,不能盡節前主,而盡節今主,此是一罪。前歲遣偏師薛真度暫來此,卿遂破傷,此是二罪。武帝之胤悉被誅戮,初無報效,而反為今主盡節,違天害理,此是三罪。不可容恕,聽卿三思,勿令闔城受苦。」伯玉遣軍副樂稚柔答曰:「承
 欲見攻圍,期於必剋,卑微常人,得抗大威,真可謂獲其死所。先蒙武帝採拔,賜預左右,犬馬知恩,寧容無感。但隆昌、延興,昏悖違常,聖明纂業,家國不殊。



 此則進不負心,退不愧幽。前歲薛真度導誘邊氓,遂見陵突,既荷國恩,聊爾撲掃。



 回已而言,應略此責。」宏引軍向城南寺前頓止,從東南角溝橋上過,伯玉先遣勇士數人著斑衣虎頭帽,從伏竇下忽出,宏人馬驚退,殺數人,宏呼善射將原靈度射之,應弦而倒。宏乃過。宏時大舉南寇,偽咸陽王元憘、彭城王元勰、常侍王元嵩、寶掌王元麗、廣陵侯元燮、都督大將軍劉昶、王肅、楊大眼、奚康生、長孫稚等三十六軍,前後相繼,眾號百萬。其諸王軍朱色鼓,公侯綠色鼓,伯子男黑色鼓,並有鼙角,吹唇沸地。



 宏留偽咸陽王憘圍南陽,進向新野,新野太守劉思忌亦拒守。臺先遣軍主直閣將軍胡松助北襄城太守成公期守赭陽城,軍主鮑
 舉助西汝南、北義陽二郡太守黃瑤起戍舞陰城。宏攻圍新野城,戰鬥不息。遣人謂城中曰:「房伯玉已降,汝南為何獨自取糜碎?」思忌令人對曰:「城中兵食猶多,未暇從汝小虜語也。」雍州刺史曹虎遣軍至均口,不進。永泰元年,城陷,縛思忌,問之曰:「今欲降未?」思忌曰:「寧為南鬼,不為北臣。」乃死。贈冠軍將軍、梁州刺史。於是沔北大震,湖陽戍主蔡道福、赭陽城主成公期及軍主胡松、舞陰城主黃瑤起及軍主鮑舉、順陽太守席謙並棄城走。虜追軍獲瑤起,王肅募人臠食其肉。追贈冠軍將軍、兗州刺史。



 數日,房伯玉以城降。伯玉,清河人。既降,虜以為龍驤將軍,伯玉不肯受。高宗知其志,月給其子希哲錢五千,米二十斛。後伯玉就虜求南邊一郡,為馮翊太守。



 生子幼,便教其騎馬,常欲南歸。永元末,希哲入虜,伯玉大怒曰:「我力屈至此,不能死節,猶望汝在本朝以報國恩。我若從心,亦欲間
 關求反。汝何為失計?」遂卒虜中。



 虜得沔北五郡。宏自將二十萬騎破太子率崔慧景等於鄧城,進至樊城,臨沔水而去。還洛陽,聞太尉陳顯達經略五郡,圍馬圈,宏復率大眾南攻,破顯達而死。



 喪還,未至洛四百餘里,稱宏詔,徵偽太子恪會魯陽。恪至,勰以宏偽法服衣之,始發喪。至洛,乃宣布州郡,舉哀制服,謚孝文皇帝。



 是年,王肅為虜制官品百司,皆如中國。凡九品,品各有二。肅初奔虜,自說其家被誅事狀,宏為之垂涕。以第六妹偽彭城公主妻之。封肅平原郡公。為宅舍,以香塗壁。遂見信用。恪立,號景明元年,永元二年也。



 豫州刺史裴叔業以壽春降虜。先是偽東徐州刺史沈陵率部曲降。陵,吳興人,初以失志奔虜,大見任用,宏既死,故南歸,頻授徐、越二州刺史。時王肅為征南將軍、豫州都督。朝延既新失大鎮,荒人往來,詐雲肅欲歸國。少帝詔以肅為使持節、侍中、都督豫徐司
 三州、右將軍、豫州刺史,西豐公,邑二千戶。



 虜既得淮南,其夏,遣偽冠軍將軍南豫州刺史席法友攻北新蔡、安豐二郡太守胡景略於建安城,死者萬餘人,百餘日,朝廷無救,城陷,虜執景略以歸。其冬,虜又遣將桓道福攻隨郡太守崔士招,破之。



 後偽咸陽王憘以恪年少,與氐楊集始、楊靈祐、乞佛馬居及虜大將支虎、李伯尚等十餘人,請會鴻池陂,因恪出北芒獵,襲殺之。憘猶豫不能發,欲更剋日。



 馬居說憘曰:「殿下若不至北芒,便可回師據洛城,閉四門。天子聞之,必走向河北桑乾,仍斷河橋,為河南天子。隔河而治,此時不可失也。」憘又不從。靈祐疑憘反己,即馳告恪。憘聞事敗,欲走渡河,而天雨暗迷道,至孝義驛,恪已得洛城。遣弟廣平王領數百騎先入宮,知無變,乃還。遣直衛三郎兵討憘,執殺之。虜法,謀反者不得葬,棄尸北芒。王肅以疾卒。



 史臣曰:齊、虜分,江南為國歷三代矣。華夏分崩,舊京幅裂,觀釁阻兵,事興東晉。二庾藉元舅之盛,自許專征,元規臨邾城以覆師,稚恭至襄陽而反旆。褚裒以徐、兗勁卒,壹沒於鄒、魯。殷浩驅楊、豫之眾,大敗於山桑。桓溫弱冠雄姿,因平蜀之聲勢,步入咸關,野戰洛、鄴。既而鮮卑固於負海,羌、虜割有秦、代,自為敵國,情險勢分,宋武乘機,故能以次而行誅滅。及魏虜兼并,河南失境,兵馬土地,非復曩時。宋文雖得之知己,未能料敵,故師帥無功,每戰必殆。泰始以邊臣外叛,遂亡淮北,經略不振,乃議和親。太祖創命,未及圖遠。戎塵先起,侵暴方牧,淮、豫剋捷,青、海摧奔,以逸待勞,坐微百勝。自四州淪沒,民戀本朝,國祚惟新,歌奉威德,提戈荷甲,人自為斗,深壘結防,想望南旗。天子習知邊事,取亂而授兵律,若前師指日,遠掃臨、彭,而督將逗留,援接稽晚,向義之徒,傾巢盡室。既失事機,朝議
 北寢,偃武脩文,更思後會。永明之世,據已成之策,職問往來,關禁寧靜。疆埸之民,並安堵而息窺覦,百姓附農桑而不失業者,亦由此而已也。夫荊棘所生,用武之弊,寇戎一犯,傷痍難復,豈非此之驗乎?建武初運,獯雄南逼,豫、徐彊鎮,嬰高城,蓄士卒,不敢與之校武。胡馬蹈藉淮、肥,而常自戰其地。梯沖之害,鼓掠所亡,建元以來,未之前有。兼以穹廬華徙,即禮舊都,雍、司北部,親近許、洛,平塗數百,通驛車軌,漢世馳道,直抵章陵,鑣案所鶩,晨往暮返。虜懷兼弱之威,挾廣地之計,強兵大眾,親自凌殄,旍鼓彌年,矢石不息。朝規懦屈,莫能救御,故南陽覆壘,新野頹隍,民戶墾田,皆為狄保。雖分遣將卒,俱出淮南,未解沔北之危,已深渦陽之敗。征賦內盡,民命外殫,比屋騷然,不聊生矣。夫休頹之數,誠有天機,得失之迹,各歸人事,豈不由將率相臨,貪功昧賞,勝敗之急,不相救護?號令不明,固
 中國之所短也。



 贊曰:天立勍胡,竊有帝圖。即安諸夏,建號稱孤。齊民急病,并邑焚刳。



\end{pinyinscope}