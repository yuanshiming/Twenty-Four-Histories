\article{卷一本紀第一高帝上}

\begin{pinyinscope}

 太祖高皇帝諱道成,字紹伯,姓蕭氏,小諱鬥將,漢相國蕭何二十四世孫也。



 何子酂定侯延生侍中彪,彪生公府掾章,章生皓,皓生仰,仰生御史大夫望之,望之生光祿大夫育,育生御史中丞紹,紹生光祿勛閎,閎生濟陰太守闡,闡生吳郡太守永,永生中山相苞,苞生博士周,周生蛇丘長矯,矯生州從事逵,逵生孝廉休,休生廣陵府丞豹,豹生太中大夫裔,裔生淮陰令整,整生即丘令雋,雋生輔國參軍樂子,宋昇明二年九月贈太常,生皇考。



 蕭何居沛,侍中彪免官居東海蘭陵縣中都鄉中都里。晉元康元年,分東海為蘭陵郡。中朝亂,淮陰令整字公齊,過江居晉陵武進縣之東城里。寓
 居江左者,皆僑置本土,加以南名,於是為南蘭陵蘭陵人也。



 皇考諱承之,字嗣伯。少有大志,才力過人,宗人丹陽尹摹之、北兗州刺史源之並見知重。初為建威府參軍。義熙中,蜀賊譙縱初平,皇考遷揚武將軍、安固汶山二郡太守,善於綏撫。元嘉初,徙為武烈將軍、濟南太守。七年,右將軍到彥之北伐大敗,虜乘勝破青部諸郡國。別帥安平公乙旃眷寇濟南,皇考率數百人拒戰,退之。虜眾大集,皇考使偃兵開城門。眾諫曰:「賊眾我寡,何輕敵之甚!」皇考曰:「今日懸守窮城,事已危急,若復示弱,必為所屠,惟當見強待之耳。」虜疑有伏兵,遂引去。青州刺史蕭思話欲委鎮保險,皇考固諫不從,思話失據潰走。



 明年,征南大將軍檀道濟於壽張轉戰班師,滑臺陷沒,兗州刺史竺靈秀抵罪。



 宋文帝以皇考有全城之功,手書與都督長沙王義欣曰:「承之理民直亦不在武幹後,今擬為兗州刺史,
 檀征南詳之。」皇考與道濟無素故,事遂寢。遷輔國鎮北中兵參軍、員外郎。



 十年,蕭思話為梁州刺史,皇考為其橫野府司馬、漢中太守。氐帥楊難當寇漢川,梁州刺史甄法護棄城走,思話至襄陽不進。皇考輕軍前行,攻氐偽魏興太守薛健於黃金山,剋之。黃金山,張魯舊戍,南接漢川,北枕驛道,險固之極。健既潰散,皇考即據之。氐偽梁、秦二州刺史趙溫先據州城,聞皇考至,退據小城,薛健退屯下桃城,立柴營。皇考引軍與對壘,相去二里。健與偽馮翊太守蒲早子悉力出戰,皇考大破之。健等閉營自守不敢出,思話繼至,賊乃稍退。皇考進至峨公山,為左衛將軍、沙州刺史呂平大眾所圍積日,建武將軍蕭汪之、平西督護段虯等至,表裏奮擊,大破之。難當又遣息和領步騎萬餘人,夾漢水兩岸,援趙溫,攻逼皇考。



 相拒四十餘日。賊皆衣犀甲,刀箭不能傷。皇考命軍中斷槊長數尺,以
 大斧捶其後,賊不能當,乃焚營退。皇考追至南城,眾軍自後而進,連戰皆捷,梁州平。詔曰:「承之稟命先驅,蒙險深入,全軍屢剋,奮其忠果,可龍驤將軍。」隨府轉寧朔司馬,太守如故。入為太子屯騎校尉。文帝以平氐之勞,青州缺,將欲授用。彭城王義康秉政,皇考不附,乃轉為江夏王司徒中兵參軍、龍驤將軍、南泰山太守,封晉興縣五等男,邑三百四十戶。遷右軍將軍。元嘉二十四年殂,年六十四。梁土民思之,於峨公山立廟祭祀。昇明二年,贈散騎常侍、金紫光祿大夫。



 太祖以元嘉四年丁卯歲生。姿表英異,龍顙鐘聲,鱗文遍體。儒士雷次宗立學於雞籠山,太祖年十三,受業,治《禮》及《左氏春秋》。十七年,宋大將軍彭城王義康被黜,鎮豫章,皇考領兵防守,太祖舍業南行。十九年,竟陵蠻動,文帝遣太祖領偏軍討沔北蠻。二十一年,伐索虜,至丘檻山,並破走。二十三年,雍州刺史蕭思話鎮襄
 陽,啟太祖自隨,戍沔北,討樊、鄧諸山蠻,破其聚落。初為左軍中兵參軍。二十七年,索虜圍汝南戍主陳憲,臺遣寧朔將軍臧質、安蠻司馬劉康祖救之。文帝使太祖宣旨,授節度。聞虜主拓跋燾向彭城,質等回軍救援。至盱眙,太祖與質別軍主胡宗之等五軍,步騎數千人前驅。燾已潛過淮,卒相遇於莞山下。合戰敗績,緣淮奔退,宗之等皆陷沒。太祖還就質固守,為虜所攻圍,甚危急。事寧,還京師。



 二十九年,領偏軍征仇池。梁州西界舊有武興戍,晉隆安中沒屬氐;武興西北有蘭皋戍,去仇池二百里。太祖擊二壘,皆破之。遂從谷口入關,未至長安八十里,梁州刺史劉秀之遣司馬馬注助太祖攻談堤城,拔之,虜偽河間公奔走。虜救兵至,太祖軍力疲少,又聞文帝崩,乃燒城還南鄭。襲爵晉興縣五等男。孝建初,除江夏王大司馬參軍,隨府轉太宰,遷員外郎、直閣中書舍人、西陽王撫
 軍參軍、建康令。



 新安王子鸞有盛寵,簡選僚佐,為北中郎中兵參軍。陳太后憂,起為武烈將軍,復為建康令,中兵如故。景和世,除後軍將軍。值明帝立,為右軍將軍。



 時四方反叛,會稽太守尋陽王子房及東諸郡皆起兵。明帝加太祖輔國將軍,率眾東討。至晉陵,與賊前鋒將程捍、孫曇瓘等戰,一日破賊十二壘。分軍定諸縣,晉陵太守袁摽棄城走,東境諸城相繼奔散。



 徐州刺史薛安都反彭城,從子索兒寇淮陰,山陽太守程天祚舉城叛,徐州刺史申令孫又降,徵太祖討之。時太祖平東賊還,又將南討,出次新亭,前軍已發,而索兒自睢陵渡淮,馬步萬餘人,擊殺臺軍主孫耿,縱兵逼前軍張永營,告急。明帝聞賊渡,遽追太祖往救之,屯破釜。索兒向鐘離,永遣寧朔將軍王寬據盱眙,遏其歸路。索兒擊破臺軍主高道
 慶,走之於石鱉,將西歸。王寬與軍主任農夫先據白鵠澗,張永遣太祖馳督寬,索兒東要擊太祖,使不得前。太祖鼓行結陣,直入寬壘,索兒望見不敢發。經數日,索兒引軍頓石梁,太祖追之至葛塚,候騎還云賊至,太祖乃頓軍引管,分兩馬軍夾營外以待之。俄頃,賊馬步奄至,又推火車數道攻戰。



 相持移日,乃出輕兵攻賊西,使馬軍合擊其後,賊眾大敗,追奔獲其器仗。進屯石梁澗北。索兒夜遣千人來斫營,營中驚,太祖臥不起,宣令左右案部不得動,須臾賊散。太祖議欲於石梁西南高地築壘通南道,斷賊走路,索兒果來爭之。太祖率軍擊破之,賊馬自相踐藉死。索兒走向鐘離,太祖追至黯而還。除驍騎將軍,封西陽縣侯,邑六百戶。遷巴陵王衛軍司馬,隨鎮會稽。



 江州刺史晉安王子勛遣臨川內史張淹自鄱陽嶠道入三吳,臺軍主沈思仁與偽龍驤將軍任皇、鎮西參軍劉越緒各據險相守。明帝遣太祖領三千人討之。時朝廷器甲皆充
 南討,太祖軍容寡闕,乃編棕皮為馬具裝,析竹為寄生,夜舉火進軍。賊望見恐懼,未戰而走。還除桂陽王征北司馬、南東海太守、行南徐州事。



 初,明帝遣張永、沈攸之以眾喻降薛安都,謂太祖曰:「吾今因此北討,卿意以為何如?」太祖對曰:「安都才識不足,狡猾有餘。若長轡緩御,則必遣子入朝;今以兵逼之,彼將懼而為計,恐非國之利也。」帝曰:「眾軍猛銳,何往不克!卿每杖策,幸勿多言。」安都見兵至,果引索虜,永等敗於彭城。淮南孤弱,以太祖為假冠軍將軍、持節、都督北討前鋒諸軍事,鎮淮陰。



 泰始三年,沈攸之、吳喜北敗於睢口。諸城戍大小悉奔歸,虜遂進至淮北,圍角城,戍主賈法度力弱不敵。諸將勸太祖渡岸救之,太祖不許;遣軍主高道慶將數百張弩浮艦淮中,遙射城外虜;弩一發數百箭俱去,虜騎相引避之,乃命進戰,城圍即解。遷督南兗徐二州諸軍事、南兗州刺史,持節、假冠軍、督北討如故。五年,進督兗、青、冀三州。六年,除黃門侍郎,
 領越騎校尉,不拜。復授冠軍將軍。留本任。



 明帝常嫌太祖非人臣相,而民間流言,云「蕭道成當為天子」,明帝愈以為疑。



 遣冠軍將軍吳喜以三千人北使,令喜留軍破釜,自持銀壺酒封賜太祖。太祖戎衣出門迎,即酌飲之。喜還,帝意乃悅。七年,徵還京師;部下勸勿就徵,太祖曰:「諸卿暗於見事。主上自誅諸弟,為太子稚弱,作萬歲後計,何關佗族。惟應速發,事緩必見疑。今骨肉相害,自非靈長之運,禍難將興,方與卿等戮力耳。」拜散騎常侍、太子左衛率。時世祖以功當別封贛縣,太祖以一門二封,固辭不受,詔許之。



 加邑二百戶。明帝崩,遺詔為右衛將軍,領衛尉,加兵五百人。與尚書令袁粲、護軍褚淵、領軍劉勔共掌機事。又別領東北選事。尋解衛尉,加侍中,領石頭戍軍事。



 明帝誅戮蕃戚,江州刺史桂陽王休範以人凡獲全。及
 蒼梧王立,更有窺窬之望,密與左右閹人於後堂習馳馬,招聚士眾。元徽二年五月,舉兵於尋陽,收略官民,數日得士眾二萬人,騎五百匹。發盆口,悉乘商旅船艦。大雷戍主杜道欣、鵲頭戍主劉諐期告變,朝廷惶駭。太祖與護軍褚淵、征北張永、領軍劉勔、僕射劉秉、游擊將軍戴明寶、驍騎將軍阮佃夫、右軍將軍王道隆、中書舍人孫千齡、員外郎楊運長集中書省計議,莫有言者。太祖曰:「昔上流謀逆,皆因淹緩,至於覆敗。休範必遠懲前失,輕兵急下,乘我無備。今應變之術,不宜念遠,若偏師失律,則大沮眾心。宜頓新亭、白下,堅守宮掖、東府、石頭以待。賊千里孤軍,後無委積,求戰不得,自然瓦解。我請頓新亭以當其鋒;征北可以見甲守白下;中堂舊是置兵地,領軍宜屯宣陽門為諸軍節度;諸貴安坐殿中,右軍諸人不須競出。我自前驅,破賊必矣。」因索筆下議,並注同。



 中書舍人孫千齡與休範有密契,獨曰:「宜依舊遣軍據梁山、魯顯
 間,右衛若不出白下,則應進頓南州。」太祖正色曰:「賊今已近,梁山豈可得至!新亭既是兵沖,所以欲死報國耳。常日乃可屈曲相從,今不得也。」座起,太祖顧謂劉勔曰:「領軍已同鄙議,不可改易。」乃單車白服出新亭。加太祖使持節、都督征討諸軍、平南將軍,加鼓吹一部。



 治新亭城壘未畢,賊前軍已至。太祖方解衣高臥,以安眾心,乃索白虎幡,登西垣。使寧朔將軍高道慶、羽林監陳顯達、員外郎王敬則浮舸與賊水戰,自新林至赤岸,大破之,燒其船艦,死傷甚眾。賊步上新林,太祖馳使報劉勔,急開大小桁,撥淮中船舫,悉渡北岸。休范乘肩輿率眾至壘南,上遣寧朔將軍黃回、馬軍主周盤龍將步騎出壘對陣。休範分兵攻壘東,短兵接戰,自巳至午,眾皆失色。太祖曰:「賊雖多而亂,尋破也。」楊運長領三齊射手七百人,引強命中,故賊不得逼城。



 未時,張敬兒斬休範首。太祖遣隊主陳
 靈寶送首還臺,靈寶路中遇賊軍,埋首道側。



 臺軍不見休範首,愈疑懼。賊眾亦不知休範已死,別率杜黑蠡急攻壘東;司空主簿蕭惠朗數百人突入東門,叫噪至堂下,城上守門兵披退。太祖挺身上馬,率數百人出戰;賊皆推楯而前,相去數丈,分兵橫射。太祖引滿將發,左右將戴仲緒舉楯扞之,箭應手飲羽,傷百餘人。賊死戰不能當,乃卻。眾軍復得保城,與黑蠡拒戰,自晡達明旦,矢石不息。其夜大雨,鼓叫不復相聞,將士積日不得寢食,軍中馬夜驚,城內亂走,太祖秉燭正坐,厲聲呵止之,如此者數四。



 賊帥丁文豪設伏破臺軍於皂莢橋,直至朱雀桁,劉勔欲開桁,王道隆不從,勔及道隆並戰沒。初,勔高尚其意,託造園宅,名為「東山」,頗忽世務。太祖謂之曰:「將軍以顧命之重,任兼內外;主上春秋未幾,諸王並幼沖,上流聲議,遐邇所聞。此是將軍艱難之日,而將軍深尚從容,廢省羽
 翼,一朝事至,雖悔何追!」



 勔竟不納。賊進至杜姥宅,車騎典簽茅恬開東府納賊,冠軍將軍沈懷明於石頭奔散,張永潰於白下,宮內傳新亭亦陷。太后執蒼梧王手泣曰:「天下敗矣!」太祖遣軍主陳顯達、任農夫、張敬兒、周盤龍等,從石頭濟淮,間道從承明門入衛宮闕。



 休範即死,典簽許公與詐稱休範在新亭,士庶惶惑,詣壘投名者千數,太祖隨得輒燒之,乃列兵登城北,謂曰:「劉休範父子先昨皆已即戮,尸在南岡下。身是蕭平南,諸君善見觀。君等名皆已焚除,勿有懼也。」臺分遣眾軍擊杜姥宅、宣陽門諸賊,皆破平之。太祖振旅凱入,百姓緣道聚觀,曰:「全國家者此公也。」



 太祖與袁粲、褚淵、劉秉引咎解職,不許。遷散騎常侍、中領軍、都督南兗徐兗青冀五州軍事、鎮軍將軍、南兗州刺史,持節如故。進爵為公,增邑二千戶。太祖欲分其功,請益粲等戶,更日入直決事,號為「四貴」。秦時有太
 后、穰侯、涇陽、高陵君,稱為「四貴」,至是乃復有焉。四年,加太祖尚書左僕射,本官如故。



 休範平後,蒼梧王漸行兇暴。南徐州刺史建平王景素少有令譽,朝野歸心。景素亦潛為自全之計,布款誠於太祖,太祖拒而不納。七月,羽林監袁祗奔景素,便舉兵。太祖出屯玄武湖,遣眾軍北討,事平乃還。太祖威名既重,蒼梧王深相猜忌,幾加大禍。陳太妃罵之曰:「蕭道成有功於國,今若害之,後誰復為汝著力者?」



 乃止。



 太祖密謀廢立。五年七月戊子,帝微行出北湖,常單馬先走,羽儀禁衛隨後追之,於堤塘相蹈藉。左右張互兒馬墜湖,帝怒,取馬置光明亭前,自馳騎刺殺之,因共屠割,與左右作羌胡伎為樂。又於蠻岡賭跳。際夕乃還仁壽殿東阿氈屋中寢。



 語左右楊玉夫:「伺織女度,報我。」時殺害無常,人懷危懼。玉夫與其黨陳奉伯等二十五人同謀,於氈屋中取千牛刀殺蒼梧王,稱敕,使廂下奏
 伎,因將首出與王敬則,敬則送太祖。太祖夜從承明門乘常所騎赤馬入,殿內驚怖,即知蒼梧王死,咸稱萬歲。及太祖踐阼,號此馬為「龍驤將軍」,世謂為「龍驤赤」。



 明日,太祖戎服出殿庭槐樹下,召四貴集議。太祖謂劉秉曰:「丹陽國家重戚,今日之事,屬有所歸。」秉讓不當。太祖次讓袁粲,粲又不受。太祖乃下議,備法駕詣東城,迎立順帝。於是長刀遮粲、秉等,各失色而去。甲午,太祖移鎮東府,與袁粲、褚淵、劉秉各甲仗五十人入殿。丙申,進位侍中、司空、錄尚書事、驃騎大將軍,持節、都督、刺史如故,封竟陵郡公,邑五千戶,給油幢絡車,班劍三十人。太祖固辭上命,即驃騎大將軍、開府儀同三司。庚戌,進督南徐州刺史。封楊玉夫等二十五人爵邑各有差。十月戊辰,又進督豫、司二州。



 初,荊州刺史沈攸之與太祖於景和世同直殿省,申以歡好,以長女義興公主妻攸之第三子元和。攸
 之為郢州,值明帝晚運,陰有異圖,自郢州遷為荊州,聚斂兵力,將吏逃亡,輒討質鄰伍。養馬至二千餘匹,皆分賦戍邏將士,使耕田而食,廩財悉充倉儲。荊州作部歲送數千人仗,攸之割留,簿上供討四山蠻。裝治戰艦數百千艘,沈之靈溪裏,錢帛器械巨積,朝廷畏之。高道慶家在華容,假還過江陵。道慶素便馬,攸之與宴飲,於聽事前合馬槊,道慶槊中破攸之馬鞍,攸之怒,索刃槊,道慶馳馬而出。還都,說攸之反狀,請三千人襲之。朝議慮其事難濟,太祖又保持不許。太祖既廢立,遣攸之子司徒左長史元琰齎蒼梧王諸虐害器物示之,攸之未得即起兵,乃上表稱慶,并與太祖書推功。攸之有素書十數行,常韜在裲襠角,云是明帝與己約誓。十二月,遂舉兵。其妾崔氏、許氏諫攸之曰:「官年已老,那不為百口計!」攸之指裲襠角示之,稱太后令召己下都。京師恐懼。乙卯,太祖入居朝堂,命
 諸將西討,平西將軍黃回為都督前驅。



 前湘州刺史王蘊,太后兄子,少有膽力,以父楷名宦不達,欲以將途自奮。每撫刀曰:「龍淵、太阿,汝知我者。」叔父景文誡之曰:「阿答,汝滅我門戶!」



 蘊曰:「答與童烏貴賤覺異。」童烏,景文子絢小字;答,蘊小字也。蘊遭母喪罷任,還至巴陵,停舟一月,日與攸之密相交構。時攸之未便舉兵,蘊乃下達郢州。



 世祖為郢州長史,蘊期世祖出弔,因作亂據郢城,世祖知之,不出。蘊還至東府前,又期太祖出,太祖又不出弔,再計不行,外謀愈固。



 司徒袁粲、尚書令劉秉見太祖威權稍盛,慮不自安,與蘊及黃回等相結舉事,殿內宿衛主帥,無不協同。攸之反問初至,太祖往石頭與粲謀議,粲稱疾不相見。



 剋壬申夜起兵據石頭,劉秉恇怯,晡時,從丹陽郡載婦女入石頭,朝廷不知也。其夜,丹陽丞王遜告變,秉從弟領軍韞及直閣將軍卜伯興等嚴兵為內應。太祖命
 王敬則於宮內誅之。遣諸將攻石頭,王蘊將數百精手帶甲赴粲,城門已閉,官軍又至,乃散。眾軍攻石頭,斬粲。劉秉走雒簷湖,蘊逃鬥場,並擒斬之。粲位任雖重,無經世之略,疏放好酒。步屟白楊郊野間,道遇一士大夫,便呼與酣飲。明日,此人謂被知顧,到門求通,粲曰:「昨飲酒無偶,聊相要耳。」竟不與相見。嘗作五言詩云:「訪跡雖中宇,循寄乃滄州。」蓋其志也。劉秉少以宗室清謹見知。孝武世,秉弟遐坐通嫡母殷氏養女,殷亡舌中血出,眾疑行毒害,孝武使秉從弟祗諷秉啟證其事。秉曰:「行路之人,尚不應爾,今日乃可一門同盡,無容奉敕。」眾以此稱之,故為明帝所任。蒼梧廢,秉出集議,於路逢弟韞,韞開車迎問秉曰:「今日之事,固當歸兄邪?」秉曰:「吾等已讓領軍矣。」韞槌胸曰:「君肉中詎有血!」



 粲典簽莫嗣祖知粲謀,太祖召問嗣祖:「袁謀反,何不啟聞?」嗣祖曰:「事主義無二心,雖死不敢泄
 也。」蘊嬖人張承伯藏匿蘊。太祖並赦而用之。黃回頓新亭,聞石頭鼓噪,率兵來赴之,朱雀珝有戍軍,受節度,不聽夜過,會石頭已平,因稱救援。太祖知而不言,撫之愈厚,遣回西上,流涕告別。



 太祖屯閱武堂,馳結軍旅。閏月辛丑,詔假黃鉞,率大眾出屯新亭中興堂,治嚴築壘。教曰:「河南稱慈,諒由掩胔,廣漢流仁,實存殯朽。近袤製茲營,崇溝浚塹,古墟曩隧,時有湮移,深松茂草,或致刊薙。憑軒動懷,巡隍增愴。宜並為收改葬,并設薄祀。」



 二年正月,沈攸之攻郢城不剋,眾潰,自經死,傳首京邑。丙子,太祖旋鎮東府。二月癸未,進太祖太尉,增封三千戶,都督南徐、南兗、徐、兗、青、冀、司、豫、荊、雍、湘、郢、梁、益、廣、越十六州諸軍事。太祖解驃騎,辭都督,不許,乃表送黃鉞。三月己酉,增班劍為四十人、甲仗百人入殿。丙子,加羽葆鼓吹,餘並如故。



 辛卯,太祖誅鎮北將軍黃回。



 大明、
 泰始以來,相承奢侈,百姓成俗。太祖輔政,罷御府,省二尚方諸飾玩。



 至是,又上表禁民間華偽雜物:不得以金銀為箔,馬乘具不得金銀度,不得織成繡裙,道路不得著錦履,不得用紅色為幡蓋衣服,不得剪彩帛為雜花,不得以綾作雜服飾,不得作鹿行錦及局腳檉柏床、牙箱籠雜物、彩帛作屏鄣、錦緣薦席,不得私作器仗,不得以七寶飾樂器又諸雜漆物,不得以金銀為花獸,不得輒鑄金銅為像。



 皆須墨敕,凡十七條。其中宮及諸王服用,雖依舊例,亦請詳衷。



 九月丙午,進位假黃鉞、都督中外諸軍事、太傅、領揚州牧,劍履上殿,入朝不趨,贊拜不名。置左右長史、司馬、從事中郎、掾、屬各四人,使持節、太尉、驃騎大將軍、錄尚書、南徐州刺史如故。固辭,詔遣敦勸,乃受黃鉞,辭殊禮。甲寅,給三望車。



 三年正月,乙巳,太祖表蠲百姓逋負。丙辰,加前部
 羽葆鼓吹。丁巳,命太傅府依舊闢召。丁卯,給太祖甲仗五百人,出入殿省。甲午,重申前命,劍履上殿,入朝不趨,贊拜不名。三月甲辰,詔進位相國,總百揆,封十郡為齊公,備九錫之禮,加璽紱遠遊冠,位在諸侯王上,加相國綠綟綬,其驃騎大將軍、揚州牧、南徐州刺史如故。太祖三讓,公卿敦勸固請,乃受。甲寅,策相國齊公曰:天地變通,莫大乎炎涼;懸象著明,莫崇乎日月。嚴冬播氣,貞松之操自高;光景時昏,若華之映彌顯。是故英睿當亂而不移,忠賢臨危而盡節。自景和昏虐,王綱弛紊,太宗受命,紹開中興,運屬屯難,四郊多壘。蕭將軍震威華戎,實資義烈,康國濟民,於是乎在。朕以不造,夙罹閔凶。嗣君失德,書契未紀。威侮五行,虔劉九縣,神厭靈繹,海水群飛。彞器已塵,宗禋誰主?綴旒之殆,未足為譬,豈直《小宛》興刺,《黍離》作歌而已哉!天贊皇宋,實啟明宰,爰登寡昧,纂承大業,鴻緒再維,閎基重造,
 高勳至德,振古絕倫。昔保衡翼殷,博陸匡漢,方斯蔑如也。今將授公典禮,其敬聽朕命。



 乃者,袁鄧構禍,實繁有徒;子房不臣,稱兵協亂。跨蹈五湖,憑陵吳、越,浮祲虧辰,沈氛晦景,桴鼓振於王畿,鋒鏑交乎天邑。顧瞻宮掖,將成茂草,言念邦國,翦為仇讎。當此之時,人無固志。公投袂殉難,超然奮發,執金板而先馳,登寅車而戒路,軍政端嚴,卒乘輯睦,麾皞一臨,凶黨冰泮。此則霸業之基,勤王之始也。



 安都背叛,竊據徐方,敢率犬羊,陵虐淮滸;索兒愚悖,同惡相濟,天祚無象,背順歸逆;北鄙黔黎,奄墜塗炭,均人廢職,邊師告警。公受命宗祊,精貫朝日,擁節和門,氣踰霄漢,破釜之捷,斬馘蔽野,石梁之戰,禽其渠帥,保境全民,江陽即序。此又公之功也。



 張淹迷昧,弗顧本朝,爰自南區,志圖東夏,潛軍間入,竊覬不虞。于時江服未夷,皇塗薦阻。公忠誠慷慨,在險彌亮,深識九變,妙察五色,以寡制
 眾,所向風偃。朝廷無東顧之憂,閩越有來蘇之慶。此又公之功也。



 匈奴野心,侵掠疆場,前師失律,王旅崩撓,灑血成川,伏尸千里。醜羯人舟張,勢振彭、泗,乘勝長驅,窺覦京甸,冠帶之軌將湮,被發之容行及。公奉辭伐罪,戒旦晨征,兵車始交,氛祲時蕩,弔死撫傷,弘宣皇澤,俾我淮、肥,復沾盛化。此又公之功也。



 自茲厥後,獫狁孔熾,封豕長蛇,重窺上國。而世故相仍,師出日老。戰士無臨陣之心,戎卒有懷歸之思。是以下邳精甲,望風振恐,角城高壘,指日淪陷。公眷言王事,發憤忘食,躬擐甲胄,視險若夷。短兵纔接,巨猾鳥散,分疆畫界,開創青、兗。此又公之功也。



 泰始之末,入參禁旅,任兼軍國,事同顧命。桂陽負眾,輕問九鼎,裂冠毀冕,拔本塞源,入兵萬乘之國,頓戟象魏之下,烈火焚於王城,飛矢集乎君屋。機變倏忽,終古莫二,群後憂惶,元戎無主。公按劍凝神,則奇謀貫世;秉旄指麾,則懦
 夫成勇。曾不崇朝,新亭獻捷;信宿之間,宣陽底定。雲霧廓清,區宇康鳷。此又公之功也。



 皇室多難,釁起戚蕃。邗、晉、應、韓,翻為讎敵,建平失圖,興兵內侮。公又指授六師,義形乎色,役未踰旬,朱方寧晏。此又公之功也。



 蒼梧肆虐,諸夏麋沸,淫刑以逞,誰則無罪?火炎昆岡,玉石俱焚,黔首相悲,朝不謀夕。高祖之業已淪,文、明之軌誰嗣?公遠稽殷、漢之義,近遵魏、晉之典,猥以眇身,入奉宗祏,七廟清謐,九區反政。此又公之功也。



 袁粲無質,劉秉攜貳,韞、述相扇,成此亂階;醜圖潛構,危機竊發,據有石頭,志犯應、路。公神謀內運,霜鋒外舉,妖沴載澄,國塗悅穆。此又公之功也。



 沈攸之苞禍,歲月滋彰,蜂目豺聲,阻兵安忍。哀彼荊漢,獨為匪民,乃眷西顧,緬同異域。而經綸維始,九伐未申,長惡不悛,遂逞兇逆。驅合姦回,勢過虓虎,朝野憂疑,三軍沮氣。公秉皞出關,凝威江甸,正情與曒日同亮,明略與
 秋雲競爽。至義所感,人百其心,鼖鼓一麾,夏首寧謐,雲梯未舉,魯山克定。積年逋誅,一朝顯戮,沮浦安流,章臺順軌。此又公之功也。



 公有濟天下之勛,重之以明哲,道庇生民,志匡宇宙,戮力肆心,劬勞王室,自東徂西,靡有寧晏,險阻艱難,備嘗之矣。若乃締構宗稷之勤,造物資始之澤,雲布霧散,光被六幽,弼予一人,永清四海。是以秬草騰芳於郊園,景星垂暉於清漢,遐方款關而慕義,荒服重譯而來庭。往哉邈乎!無得而名焉。



 朕聞疇庸表德,前王盛典,崇樹侯伯,有國攸同。所以文命成功,玄珪顯錫;姬旦秉哲,曲阜啟蕃。或改玉以弘風,或胙土以宣化。禮絕常班,寵冠群辟,爰逮桓文,車服異數。惟公勛業超於先烈,而褒賞闕於舊章。古今之道,何其爽歟?靜言欽歎,良有缺然。



 今進授相國,以青州之齊郡,徐州之梁郡,南徐州之蘭陵、魯郡、瑯邪、東海、晉陵、義興,揚州之吳郡、會稽,凡十郡,
 封公為齊公。錫茲玄土,苴以白茅,定爾邦家,用建冢社。斯實尚父故蕃,世作盟主,紀綱侯甸,率由舊則。往者周、召建國,師保兼任,毛、畢執珪,入作卿士,內外之寵,同規在昔。今命使持節、兼太尉、侍中、中書監、司空、衛將軍、雩都縣開國侯淵授公相國印綬,齊公璽紱;持節、兼司空副、守尚書令僧虔授齊公茅土,金虎符第一至第五左,竹使符第一至第十左。相國位總百闢,秩逾三事,職以禮移,號隨事革。其以相國總百揆,去錄尚書之稱。送所假節、侍中貂蟬、中外都督太傅太尉印綬、竟陵公印策。其驃騎大將軍、揚州牧、南徐州刺史如故。又加公九錫,其敬聽後命:以公執禮弘律,儀刑區宇,遐邇一體,民無異業,是用錫公大輅、戎輅各一,玄牡二駟。公崇修南畝,所寶惟穀,王府充實,百姓繁阜,是用錫公袞冕之服,赤鋋副焉。公居身以謙,導物以義,熔鈞庶品,罔不和悅,是用錫公軒縣
 之樂,六佾之儛。公翼贊王猷,聲教遠洽,蠻夷竭歡,回首內附,是用錫公朱戶以居。公明鑒人倫,澄辨涇渭,官方與能,英鳷克舉,是用錫公納陛以登。公保佑皇朝,厲身化下,杜漸防萌,含生夤式,是用錫公虎賁之士三百人。公禦宄以刑,禦姦以德,君親無將,將而必誅,是用錫公鈇皞各一。公鳳舉四維,龍騫八表,威靈所振,異域同文,是用錫公彤弓一,彤矢百,枿弓十,枿矢千。公明發載懷,肅恭禋祀,孝敬之重,義感靈祗,是用錫公秬鬯一卣,珪瓚副焉。齊國置丞相以下,一遵舊式。往欽哉!其祗服朕命,經緯乾坤,宏亮洪業,茂昭爾大德,闡揚我高祖之休命。



 太祖三讓,公卿敦勸固請,乃受之。



 丁巳,下令赦國內殊死以下;今月十五日昧爽以前,一皆原赦;鰥寡孤獨不能自存者,賜穀五斛,府州所領,亦同蕩然。



 宋帝詔齊公十郡之外,隨宜除用。以齊國初建,給錢五百萬,布五千匹,絹五千匹。四月癸酉,
 詔進齊公爵為王,以豫州之南梁、陳郡、潁川、陳留,南兗州之盱眙、山陽、秦郡、廣陵、海陵、南沛十郡增封。使持節、司空、衛將軍褚淵奉策授璽紱,金虎符第一至第五左,竹使符第一至第十左,錫茲玄土,苴白茅,改立王社。相國、揚州牧、驃騎大將軍、南徐州刺史如故。丙戌,命齊王冕十有二旒,建天子旌旗,出警入蹕,乘金根車,駕六馬,備五時副車,置旄頭雲罕,樂儛八佾,設鐘虡宮縣。王世子為太子,王女王孫爵命一如舊儀。



 辛卯,宋帝禪位,下詔曰:惟德動天,玉衡所以載序;窮神知化,億兆所以歸心。用能經緯乾坤,彌綸宇宙,闡揚鴻烈,大庇生民。晦往明來,積代同軌,前王踵武,世必由之。



 宋德湮微,昏毀相襲。景和騁悖於前,元徽肆虐於後,三光再霾,七廟將墜。



 璇極委馭,含識知泯,我文、武之祚,眇焉如綴。靜惟此紊,夕惕疚心。



 相國齊王,天誕睿聖,河嶽炳靈,拯傾提危,澄氛靜亂,匡濟艱
 難,功均造物。



 宏謀霜照,秘算雲回,旌旆所臨,一麾必捷;英風所拂,無思不偃,表裏清夷,遐邇寧謐。既而光啟憲章,弘宣禮教,奸宄之類,睹隆威而隔情,慕善之儔,仰徽猶而增厲。道邁於重華,勛超乎文命,蕩蕩乎無得而稱焉。是以辮髮左衽之酋,款關請吏;木衣卉服之長,航海來庭。豈惟肅慎獻楛,越裳薦翬而已哉!故四奧載宅,六府克和;川陸效珍,禎祥鱗集;卿煙玉露,旦夕揚藻;嘉穟芝英,晷刻呈茂。革運斯炳,代終彌亮,負扆握樞,允歸明哲,固以獄訟去宋,謳歌適齊。



 昔金政既淪,水德締構,天之曆數,皎焉攸徵。朕雖寡昧,暗於大道,稽覽隆替,為日已久,敢忘列代遺則,人神至願乎?便遜位別宮,敬禪於齊,一依唐虞、魏晉故事。



 是日宋帝遜於東邸。備羽儀,乘畫輪車,出東掖門,問今日何不奏鼓吹,左右莫有答者。壬辰,策命齊王曰:伊太古初陳,萬物紛綸,開耀靈以鑒品物,立元後
 以馭蒸人。若夫容成、大庭之世,宓羲、五龍之辰,靡得而詳焉。自軒黃以降,墳素所紀,略可言者,莫崇乎堯舜。披金繩而握天鏡,開玉匣而總地維,德之休明,宸居靈極,期運有終,歸禪與能。所以大唐遜位,言勞然興歌,有虞揖讓,卿雲發採。亮符命之攸臻,坦至公以成務,懷生載懌,靈祗效祉,遺風餘烈,光被無垠。漢魏因循,弗敢失墜,爰逮晉氏,亦遵前儀。惟我祖宗英睿,勳格幽顯,從天人而齊七政,凝至德而撫四維。



 末葉不造,仍世多故,日蝕星隕,山淪川竭。



 惟王聖哲淵明,榮鏡宇宙,體望日之威,資就雲之澤,臨下以簡,御眾以寬,仁育群生,義征不譓,國塗薦阻,弘五慮而鳷寧,皇緒將湮,秉六術以匡濟。及至權臣內侮,蕃屏陵上,兵革雲翔,萬邦震駭,裁之以武風,綏之以文化,遐邇清夷,表里肅穆。戢琱戈而事黼黻,委旌門而恭儒館,聲化遠洎,荒服無塵,殊類同規,華戎一揆。是以五光來儀於
 軒庭,九穗含芳於郊牧。象緯昭澈,布新之符已顯;圖讖彪炳,受終之義既彰。靈祇乃眷,兆民引領。朕聞至道深微,惟人是弘,天命無常,惟德是與。所以仰鑒玄情,俯察群望,敬禪神器,授帝位於爾躬。四海困窮,天祿永終。於戲!王其允執厥中,儀刑前式,以副率土之欣望。命司裘而謁蒼昊,奏《雲門》而升圜丘。時膺大禮,永保洪業,豈不盛歟!



 再命璽書曰:皇帝敬問相國齊王。大道之行,與三代之英,朕雖暗昧,而有志焉。夫昏明相襲,晷景之恆度;春秋遞運,時歲之常序。求諸天數,猶且隆替,矧伊在人,能無終謝?是故勛華弘風於上葉,漢魏垂式於後昆。



 昔我高祖,欽明文思,振民育德,皇靈眷命,奄有四海。晚世多難,姦宄實繁,鼖鼓宵聞,元戎旦警,億兆夷人,啟處靡厝。加以嗣君荒怠,敷虐萬方,神鼎將遷,寶策無主,實賴英聖,匡濟艱危。惟王體天則地,舍弘光大,明並日月,惠均雲雨。國步斯
 梗,則棱威外發,王猷不造,則淵謨內昭。重構閩、吳,再寧淮、濟,靜九江之洪波,卷海沂之氛沴。放斥凶昧,存我宗祀,舊物惟新,三光改照。



 逮至寵臣裂冠,則裁以廟略;荊漢反噬,則震以雷霆。麾旆所臨,風行草靡;神算所指,龍舉雲屬。諸夏廓清,戎翟思韙,興文偃武,闡揚洪烈。明保沖昧,翱翔禮樂之場;撫柔黔首,咸躋仁壽之域。自霜路所墜,星辰所經,正朔不通,人跡罕至者,莫不逾山越海,北面稱蕃,款關重譯,脩其職貢。是以禎祥發採,左史載其奇;玄象垂文,保章審其度。鳳書表肆類之運,龍圖顯班瑞之期。重以珠衡日角,神資特挺,君人之義,在事必彰。《書》不云乎,「皇天無親,惟德是輔」。民心無常,惟惠之懷。神祇之眷如彼,蒼生之願如此。笙管變聲,鐘石改調。朕所以擁璇持衡,傾佇明哲。



 昔金德既淪,而傳祚於我有宋,歷數告終,實在茲日,亦以水德而傳於齊。式遵前典,廣詢群議,王公卿
 士,咸曰惟宜。今遣使持節、兼太保、侍中、中書監、司空、衛將軍、雩都縣侯淵,兼太尉、守尚書令僧虔奉皇帝璽綬,受終之禮,一依唐虞故事。王其允副幽明,時登元后,寵綏八表,以酬昊天之休命。



 太祖三辭,宋帝王公以下固請。兼太史令、將作匠陳文建奏符命曰:「六,亢位也。後漢自建武至建安二十五年,一百九十六年而禪魏;魏自黃初至咸熙二年,四十六年而禪晉;晉自太始至元熙二年,一百五十六年而禪宋;宋自永初元年至昇明三年,凡六十年。咸以六終六受。六,亢位也。驗往揆今,若斯昭著。敢以職任,備陳管穴。伏願順天時,應符瑞。」二朝百辟又固請。尚書右僕射王儉奏:「被宋詔遜位,臣等參議,宜剋日輿駕受禪,撰立儀注。」太祖乃許焉。



 史臣曰:案《太一九宮占》推漢高五年,太一在四宮,主人與客俱得吉,計先舉事者勝,是歲高祖破楚。晉元興二年,太一在七宮,太一
 為帝,天目為輔佐,迫脅太一,是年安帝為桓玄所逼出宮。大將在一宮,參相在三宮,格太一。經言,格者,已立政事,上下格之,不利有為,安居之世,不利舉動。元興三年,太一在七宮,宋武破桓玄。元嘉元年,太一在六宮,不利有為,徐、傅廢營陽王。七年,太一在八宮,關囚惡歲,大小將皆不得立,其年到彥之北伐,初勝後敗,客主俱不利。



 十八年,太一在二宮,客主俱不利,是歲氐楊難當寇梁、益,來年仇池破。十九年,大小將皆見關不立,兇,其年裴方明伐仇池,克百頃,明年失之。泰始元年,太一在二宮,為大小將奄擊之,其年景和廢。二年,太一在三宮,不利先起,主人勝,其年晉安王子勛反。元徽二年,太一在六宮,先起敗,是歲桂陽王休範反,並伏誅。



 四年,太一在七宮,先起者客,西北走,其年建平王景素敗。升明元年,太一在七宮,不利為客,安居之世,舉事為主人,應發為客,袁粲、沈攸之等
 反,伏誅。是歲太一在杜門,臨八宮,宋帝禪位,不利為客,安居之世,舉事為主人,禪代之應也。



\end{pinyinscope}