\article{卷五十六列傳第三十七 幸臣}

\begin{pinyinscope}

 紀僧真劉系宗茹法亮呂文顯呂文度有天象,必有人事焉。幸臣一星,列于帝座。經禮立教,亦著近臣之服。親倖之義,其來已久。爰自衰周,侯伯專命,桓、文霸主,至於戰國,寵用近習,不乏於時矣。漢文幸鄧通,雖錢遍天下,位止郎中。孝武韓嫣、霍去病,遂至侍中大司馬。迄於魏、晉,世任權重,才位稍爽,而
 信倖唯均。



 中書之職,舊掌機務。漢元以令僕用事,魏明以監令專權,及在中朝,猶為重寄。陳準歸任上司,荀勖恨於失職。《晉令》舍人位居九品,江左置通事郎,管司詔誥。其後郎還為侍郎,而舍人亦稱通事。元帝用瑯邪劉超,以謹慎居職。宋文世,秋當、周糾並出寒門。孝武以來,士庶雜選,如東海鮑照,以才學知名。又用魯郡巢尚之,江夏王義恭以為非選。帝遣尚書二十餘牒,宣敕論辯,義恭乃嘆曰:「人主誠知人。」及明帝世,胡母顥、阮佃夫之徒,專為佞倖矣。



 齊初亦用久勞,及以親信。關讞表啟,發署詔敕。頗涉辭翰者,亦為詔文,侍郎之局,復見侵矣。建武世,詔命殆不關中書,專出舍人。省內舍人四人,所直四省,其下有主書令史,舊用武官,宋改文吏,人數無員。莫非左右要密,天下文簿板籍,入副其省,萬機嚴秘,有如尚書外司,領武官,有制局監,領器仗兵役,亦用寒人被恩幸者。今立《倖
 臣篇》,以繼前史之末云。



 紀僧真,丹陽建康人也。僧真少隨逐征西將軍蕭思話及子惠開,皆被賞遇。惠開性苛,僧真以微過見罰,既而委任如舊。及罷益州還都,不得志,僧真事之愈謹。



 惠開臨終嘆曰:「紀僧真方當富貴,我不見也。」乃以僧真託劉秉、周顒。初,惠開在益州,土反,被圍危急,有道人謂之曰:「城圍尋解。檀越貴門後方大興,無憂外賊也。」惠開密謂僧真曰:「我子弟見在者,並無異才。政是道成耳。」僧真憶其言,乃請事太祖。隨從在淮陰,以閑書題,令答遠近書疏。自寒官歷至太祖冠軍府參軍、主簿。僧真夢蒿艾生滿江,驚而白之。太祖曰:「詩人採蕭,蕭即艾也。



 蕭生斷流,卿勿廣言。」其見親如此。



 元徽初,從太祖頓新亭,拒桂陽賊。蕭惠朗突入東門,僧真與左右共拒戰。賊退,太祖命僧真領親兵,游邏城中。事寧,除南臺御史、太祖領軍功曹。上將
 廢立,謀之袁粲、褚淵。僧真啟上曰:「今朝廷猖狂,人不自保,天下之望,不在袁、褚。



 明公豈得默己,坐受夷滅。存亡之機,仰希熟慮。」太祖納之。



 太祖欲度廣陵起兵,僧真又啟曰:「主上雖復狂釁,虐加萬民,而累世皇基,猶固盤石。今百口北度,何必得俱。縱得廣陵城,天子居深宮施號令,目明公為逆,何以避此?如其不勝,則應北走胡中,竊謂此非萬全策也。」上曰:「卿顧家,豈能逐我行耶。」僧真頓首稱無貳。昇明元年,除員外郎,帶東武城令。尋除給事中、邵陵王參軍。



 太祖坐東府高樓,望石頭城,僧真在側。上曰:「諸將勸我誅袁、劉,我意不欲便爾。」及沈攸之事起,從太祖入朝堂。石頭反夜,太祖遣眾軍掩討。宮城中望石頭火光及叫聲甚盛,人懷不測。僧真謂眾曰:「叫聲不絕,是必官軍所攻。火光起者,賊不容自燒其城,此必官軍勝也。」尋而啟石頭平。



 上出頓新亭,使僧真領千人在帳內。初,上在領
 軍府,令僧真學上手跡下名,至是報答書疏,皆付僧真,上觀之,笑曰:「我亦不復能別也。」初,上在淮陰治城,得古錫趺,大數尺,下有篆文,莫能識者。僧真曰:「何須辨此文字,此自久遠之物,九錫之徵也。」太祖曰:「卿勿妄言。」及上將拜齊公,已剋日,有楊祖之謀於臨軒作難。僧真更請上選吉辰,尋而祖之事覺。上曰:「無卿言,亦當致小狼狽,此亦何異呼沲之冰。」轉齊國中書舍人。



 建元初,帶東燕令,封新陽縣男,三百戶。轉羽林監,加建威將軍,遷尚書主客郎,太尉中兵參軍,令如故。復以本官兼中書舍人。太祖疾甚,令僧真典遺詔,永明元年,丁父喪,起為建威將軍,尋除南泰山太守,又為舍人,本官如故。領諸王第事。



 僧真容貌言吐,雅有士風。世祖嘗目送之,笑曰:「人何必計門戶,紀僧真常貴人所不及。」諸權要中,最被盼遇。除越騎校尉,餘官如故。出為建武將軍,建康令。還除左右郎將,泰山太守。
 加先驅使。尋除前軍將軍,遭母喪,開冢得五色兩頭蛇。世祖崩,僧真號泣思慕。明帝以僧真歷朝驅使,建武元年,除游擊將軍,兼司農,待之如舊。欲令僧真治郡,僧真啟進其弟僧猛為鎮蠻護軍、晉熙太守。永泰元年,除司農卿。明帝崩,掌山陵事。出為廬陵長內史,年五十五,卒。



 宋世道人楊法持,與太祖有舊。元徽末,宣傳密謀。升明中,以為僧正。建元初,罷道,為寧朔將軍,封州陵縣男,三百戶。二年,虜圍朐山,遣法持為軍主,領支軍救援。永明四年,坐役使將客,奪其鮭稟,削封。卒。



 劉系宗,丹陽人也。少便書畫,為宋竟陵王誕子景粹侍書。誕舉兵廣陵,城內皆死,敕沈慶之赦係宗,以為東宮侍書。泰始中為主書,以寒官累遷至勳品。元徽初為奉朝請,兼中書通事舍人,員外郎。封始興南亭侯,食邑三百七十戶。帶秣陵令。



 太祖廢蒼梧
 明旦,呼正直舍人虞整,醉不能起,系宗歡喜奉命。太祖曰:「今天地重開,是卿盡力之日。」使寫諸處分敕令及四方書疏。使主書十人書吏二十人配之,事皆稱旨。除羽林監,轉步兵校尉。仍除龍驤將軍,出為海鹽令。太祖即位,除龍驤將軍、建康令。永明元年,除寧朔將軍,令如故。尋轉右軍將軍、淮陵太守,兼中書通事舍人。母喪自解,起為寧朔將軍,復本職。



 四年,白賊唐宇之起,宿衛兵東討,遣系宗隨軍慰勞,遍至遭賊郡縣。百姓被驅逼者,悉無所問,還復民伍。係宗還,上曰:「此段有征無戰,以時平蕩,百姓安怗,甚快也。」賜係宗錢帛。上欲脩治白下城,難於動役。系宗啟謫役在東民丁隨宇之為逆者,上從之。後車駕講武,上履行白下城,曰:「劉系宗為國家得此一城。」



 永明中,虜使書常令係宗題答,秘書書局皆隸之。再為少府,遷游擊將軍、魯郡太守。鬱林即位,除驍騎將軍,仍除寧朔將軍、宣城太守。
 係宗久在朝省,閑於職事。明帝曰:「學士不堪治國,唯大讀書耳。一劉係宗足持如此輩五百人。」其重吏事如此。建武二年,卒官,年七十七。



 茹法亮,吳興武康人也。宋大明中出身為小史,歷齋乾扶侍。孝武末年作酒法,鞭罰過度,校獵江右,選白衣左右百八十人,皆面首富室,從至南州,得鞭者過半。



 法亮憂懼,因緣啟出家得為道人。明帝初罷道,結事阮佃夫,用為兗州刺史孟次陽典簽。累至太祖冠軍府行參軍。元徽初,除殿中將軍,為晉熙王郢州典簽,除長兼殿中御史。



 世祖鎮盆城,須舊驅使人,法亮求留為上江州典簽,除南臺御史,帶松滋令。



 法亮便辟解事,善於承奉,稍見委信。從還石頭。建元初,度東宮主書。除奉朝請,補東宮通事舍人。世祖即位,仍為中書通事舍人。除員外郎,帶南濟陰太守。永明元年,除龍驤將軍。
 明年,詔曰:「茹法亮近在盆城,頻使銜命,內宣朝旨,外慰三軍。義勇齊奮,人百其氣。險阻艱難,心力俱盡。宜沾茅土,以甄忠績。」封望蔡縣男,食邑三百戶。轉給事中,羽林監。七年,除臨淮太守,轉竟陵王司徒中兵參軍。



 巴東王子響於荊州殺僚佐,上遣軍西上,使法亮宣旨慰勞,安撫子響。法亮至江津,子響呼法亮,法亮疑畏不肯往。又求見傳詔,法亮又不遣。故子響怒,遣兵破尹略軍。事平,法亮至江陵,刑賞處分,皆稱敕斷決。軍還,上悔誅子響,法亮被責。少時,親任如舊。



 鬱林即位,除步兵校尉。延興元年,為前軍將軍。延昌殿為世祖陰室,藏諸御服。二少帝並居西殿,高宗即位住東齋,開陰室出世祖白紗帽防身刀,法亮歔欷流涕。除游擊將軍。高武舊人鮮有存者,法亮以主署文事,故不見疑,位任如故。永泰元年,王敬則事平,法亮復受敕宣慰。出法亮為大司農。中書勢利之職,
 法亮不樂去,固辭不受,既而代人已至,法亮垂涕而出。年六十四,卒官。



 呂文顯,臨海人也。初為宋孝武齋干直長。升明初為太祖錄尚書省事,累位至殿中侍御史,羽林監,帶蘭陵丞、令,龍驤將軍,秣陵令。封劉陽縣男。永明元年,除寧朔將軍,中書通事舍人,本官如故。文顯治事以刻核被知。三年,帶南清河太守。與茹法亮等迭出入為舍人。並見親幸。四方餉遺,歲各數百萬,並造大宅,聚山開池。五年,為建康令,轉長水校尉,歷帶南泰山、南譙太守,尋為司徒中兵參軍,淮南太守,直舍人省。累遷左中郎將,南東莞太守,右軍將軍。高宗輔政,以文顯守少府,見任使。歷建武、永元之世,尚書右丞,少府卿。卒。



 呂文度,會稽人,宋世為細作金銀庫吏,竹局匠。元徽中為射雉典事,隨監莫脩宗上郢。世祖鎮盆城拒沈攸之,文度仍留伏事,知
 軍隊雜役,以此見親。從還都,為石頭城監,仍度東宮。世祖即位,為制局監,位至員外郎,帶南濮陽太守。殿內軍隊及發遣外鎮人,悉關之,甚有要勢。故世傳越州嘗缺,上覓一直事人往越州,文度啟其所知費延宗合旨,上即以為刺史。永明中,敕親近不得輒有申薦,人士免官,寒人鞭一百。



 上性尊嚴,呂文顯嘗在殿側咳聲高,上使茹法亮訓詰之,以為不敬,故左右畏威承意,非所隸莫敢有言也。時茹法亮掌雜驅使簿,及宣通密敕;呂文顯掌穀帛事;其餘舍人無別任。虎賁中郎將潘敞掌監功作。上使造禪靈寺新成,車駕臨視,甚悅。



 敞喜,要呂文顯私登寺南門樓,上知之,繫敞上方,而出文顯為南譙郡,久之乃復。



 濟陽江瞿曇、吳興沈徽孚等,以士流舍人通事而已,無權利。徽孚粗有筆札。



 建武中文詔多其辭也。官至黃門郎。



 史臣曰:中世已來宰御天下,萬機碎密,不關外司,尚書八座五曹各有恒任,係以九卿六府,事存副職。咸皆冠冕搢紳,任疏人貴,伏奏之務既寢,趨走之勞亦息。關宣所寄,屬當有歸,通驛內外,切自音旨。若夫環纓斂笏,俯仰晨昏,瞻幄座而竦躬,陪蘭檻而高眄,探求恩色,習睹威顏,遷蘭變鮑,久而彌信,因城社之固,執開壅之機。長主君世,振裘持領,賞罰事殷,能不逾漏,宮省咳唾,義必先知。故能窺盈縮於望景,獲驪珠於龍睡。坐歸聲勢,臥震都鄙。賄賂日積,苞苴歲通。富擬公侯,威行州郡。制局小司,專典兵力,雲陛天居,互設蘭錡,羽林精卒,重屯廣衛。至於元戎啟轍,式候還麾,遮迾清道,神行案轡,督察來往,馳騖輦轂,驅役分部,親承幾案,領護所攝,示總成規。若徵兵動眾,大興民役,行留之儀,請托在手;斷割牢稟,賣弄文符,捕叛追亡,長戍遠謫;軍有千齡之壽,
 室無百年之鬼。害政傷民,於此為蠹。況乎主幼時昏,其為讒慝,亦何可勝紀也!



 贊曰:恩澤而侯,親幸為舊。便煩左右,既
 貴且富。



\end{pinyinscope}