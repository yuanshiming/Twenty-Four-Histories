\article{卷二十三列傳第四 褚淵(淵弟澄 徐嗣) 王儉}

\begin{pinyinscope}

 褚淵,字彥回,河南陽翟人也。祖秀之,宋太常。父湛之,驃騎將軍,尚宋武帝女始安哀公主。淵少有世譽,復尚文帝女南郡獻公主,姑侄二世相繼。拜駙馬都尉,除著作佐郎,太子舍人,太宰參軍,太子洗馬,秘書丞。湛之卒,淵推財與弟,唯取書數千卷。襲爵都鄉侯。歷中書郎,司徒右長史,吏部郎。宋明帝即位,加領太子屯騎校尉,不受。遷侍中,知東宮事。轉吏部尚書,尋領太子右衛率,固辭。



 司徒建安王休仁南討義嘉賊,屯鵲尾,遣淵詣軍,選將帥以下勛階得自專決。



 事平,加驍騎將軍。薛安都以徐州叛虜,頻寇淮、泗,遣淵慰勞北討
 眾軍。淵還啟帝言:「盱眙以西,戎備單寡,宜更配衣。汝陰、荊亭並已圍逼,安豐又已不守,壽春眾力,止足自保。若使游騎擾壽陽,則江外危迫。歷陽、瓜步、鐘離、義陽皆須實力重戍,選有乾用者處之。」



 帝在藩,與淵以風素相善。及即位,深相委寄,事皆見從。改封雩都縣伯,邑五百戶。轉侍中,領右衛將軍,尋遷散騎常侍,丹陽尹。出為吳興太守,常侍如故。



 增秩千石,固辭增秩。



 明帝疾甚,馳使召淵,付以後事。帝謀誅建安王休仁,淵固諫,不納。復為吏部尚書,領常侍、衛尉如故,不受,乃授右僕射,衛尉如故。淵以母年高羸疾,晨昏須養,固辭衛尉,不許。明帝崩,遺詔以為中書令、護軍將軍,加散騎常侍,與尚書令袁粲受顧命,輔幼主。淵同心共理庶事,當奢移之後,務弘儉約,百姓賴之。



 接引賓客,未嘗驕倦。王道隆、阮佃夫用事,奸賂公行,淵不能禁也。



 遭庶母郭氏喪,有至性,數日中,毀頓不可復識。
 期年不盥櫛,惟哭泣處乃見其本質焉。詔斷哭,禁吊客。葬畢,起為中軍將軍,本官如故。



 元徽二年,桂陽王休範反,淵與衛將軍袁粲入衛宮省,鎮集眾心。淵初為丹陽,與從弟炤同載出,道逢太祖,淵舉手指太祖車謂炤曰:「此非常人也。」出為吳興,太祖餉物別,淵又謂之曰:「此人材貌非常,將來不可測也。」及顧命之際,引太祖豫焉。太祖即平桂陽,遷中領軍,領南兗州,增戶邑。太祖固讓,與淵及衛軍袁粲書曰:「下官常人,志不及遠。隨運推斥,妄踐非涯,才輕任重,夙宵冰惕。近值國危,含氣同奮,況在下官,寧吝身命!履冒鋒炭,報效恆理,而褒嘉之典,偏見甄沐,貴登端戎,秩加爵土,瞻言霄衢,魂神震墜。下官奉上以誠,率性無矯,前後忝荷,未嘗固讓。至若今授,特深恇迫。實以銜恩先旨,義兼陵闕,識蔽防萌,宗戚構禍,引誚歸咎,既已靦顏,乃復乘災求幸,藉亂取貴,斯實國家之恥,非臣子所忍
 也。且榮不可濫,寵不可昧,乞蠲中候,請亭增邑,庶保止足,輸效淮湄。



 如使伐匈奴,凱歸反旆,以此受爵,不復固辭矣。」淵、粲答曰:「來告穎亮,敬挹無已。謙貶居心,深承非飾,此誠此旨,久著言外;況復造席舒衿,迂翰緒意,推情顧己,信足書紳。但今之所宜商榷,必以輕重相推。世惟多難,事屬雕弊,四維恇擾,邊氓未安,國家費廣,府藏須備,北狄侵邊,憂虞交切。宇內含識,尚為天下危心,相與共荷任寄若此,當可稍脩廉退不?求之懷抱,實謂不可。了其不可,理無固執。且勍寇窮凶,勢過原燎,釁逆倉卒,終古未聞。常時懼惑,當慮先定,結壘新亭,枕戈待敵:斷決之策,實有由然。鋒鏑初交,元惡送首,總律制奇,判於此舉。裂邑萬戶,登爵槐鼎,亦何足少酬勛勞,粗塞物聽!今以近侍禁旅,進升中候,乘平隨牒,取此非叨。濟、河昔所履牧,鎮軍秩不逾本,詳校階序,愧在未優,就加沖損,特虧朝制。奉職
 數載,同舟無幾,劉領軍峻節霜明,臨危不顧,音迹未晞,奄成今古。迷途失偶,慟不及悲。戎謨內寄,恆務倍急,秉操辭榮,將復誰委?誠惟軍柄所期。自增茂圭社,誓貫朝廷,匹夫里語,尚欲信厚,君令必行,逡巡何路!凡位居物首,功在眾先,進退之宜,當與眾共。茍殉獨善,何以處物!



 受不自私,彌見至公,表裏詳究,無而後可。想體殊常,深思然納。」太祖乃受命。



 其年,淵加尚書令、侍中,給班劍二十人,固讓令。三年,進爵為侯,增邑千戶。服闋,改授中書監,侍中、護軍如故,給鼓吹一部。明年,淵後嫡母吳郡公主薨,毀瘠如初,葬畢,詔攝職,固辭。又以期祭禮及,表解職,並不許。



 蒼梧酷暴稍甚,太祖與淵及袁粲言世事。粲曰:「主上幼年微過易改,伊、霍之事,非季代所行,縱使功成,亦終無全地。」淵默然,歸心太祖。及廢蒼梧,群公集議,袁粲、劉秉既不受任,淵曰:「非蕭公無以了此。」手取書授太祖。太祖曰:「相與不肯,我
 安得辭!」事乃定。順帝立,改號衛將軍、開府儀同三司,侍中如故。甲仗五十人入殿。沈攸之事起,袁粲懷貳,太祖召淵謀議。淵曰:「西夏釁難,事必無成。公當先備其內耳。」太祖密為其備。事平,進中書監、司空,本官如故。



 齊臺建,淵白太祖引何曾自魏司徒為晉丞相,求為齊官,太祖謙而不許。建元元年,進位司徒,侍中、中書監如故。封南康郡公,邑三千戶。淵固讓司徒。與僕射王儉書,欲依蔡謨事例。儉以非所宜言,勸淵受命,淵終不就。



 淵美儀貌,善容止,俯仰進退,咸有風則。每朝會,百僚遠國使莫不延首目送之。宋明帝嘗嘆曰:「褚淵能遲行緩步,便持此得宰相矣。」尋加尚書令,本官如故。二年,重申前命為司徒,又固讓。



 是年虜動,上欲發王公已下無官者為軍,淵諫以為無益實用,空致擾動,上乃止。朝廷機事,多與諮謀,每見從納,禮遇甚重。上大宴集,酒後謂群臣曰:「卿等並宋時公卿,亦當
 不言我應得天子。」王儉等未及答,淵斂板曰:「陛下不得言臣不早識龍顏。」上笑曰:「吾有愧文叔,知公為朱祜久矣。」



 淵涉獵談議,善彈琵琶。世祖在東宮,賜淵金鏤柄銀柱琵琶。性和雅有器度,不妄舉動。宅嘗失火,煙焰甚逼,左右驚擾,淵神色怡然,索輿來徐去。輕薄子頗以名節譏之,以淵眼多白精,謂之「白虹貫日」,言為宋氏亡徵也。



 太祖崩,遺詔以淵為錄尚書事。江左以來,無單拜錄者,有司疑立優策。尚書王儉議,以為:「見居本官,別拜錄,推理應有策書,而舊事不載。中朝以來,三公王侯,則優策並設,官品第二,策而不優。優者褒美,策者兼明委寄。尚書職居天官,政化之本,[故]尚書令品雖第三,拜必有策。錄尚書品秩不見,而總任彌重,前代多與本官同拜,故不別有策。即事緣情,不容均之凡僚,宜有策書,用申隆寄。既異王侯,不假優文。」從之。尋增淵班劍為三十人,五日一朝。頃之寢
 疾。



 上相星連有變,淵憂之,表遜位。又因王儉及侍中王晏口陳於世祖,世祖不許。又啟曰:「臣顧惟凡薄,福過災生,未能以正情自安,遠慚彥輔。既內懷耿介,便覺晷刻難推。叨職未久,首歲便嬰疾篤,爾來沈痼,頻經危殆,彌深憂震。陛下曲存遲回,或謂僉議同異,此出於留慈每過,愛欲其榮。臣年四十有八,叨忝若此,以疾陳遜,豈駭聽察!總錄之任,江左罕授,上鄰亞台,升降蓋微。今受祿弗辭,退絀斯願,於臣名器,非曰貶少。萬物耳目,皎然共見,寧足仰延聖慮,稍垂矜惜。



 臣若內飾廉譽,外循謙後,此則憲書行劾,刑網是肅。臣赤誠不能行,亦幽明所不宥。區區寸心,歸啟以實。自吝寸陰,實願萬倍堯世。昔王弘固請,乃於司徒為衛將軍,宋氏行之不疑,當時物無異議。以臣方之,曾何足說。伏願恢闡宏猷,賜開亭造,則臣死之日,猶生之年。」乃改授司空,領驃騎將軍,侍中、錄尚書如故。



 上遣侍
 中王晏、黃門郎王秀之問疾。薨,家無餘財,負債至數十萬。詔曰:「司徒奄至薨逝,痛怛慟懷,比雖尪瘵,便力出臨哭。給東園秘器,朝服一具,衣一襲,錢二十萬,布二百匹,蠟二百斤。」時司空掾屬以淵未拜,疑應為吏敬不?



 王儉議:「依《禮》,婦在塗,聞夫家喪,改服而入。今掾屬雖未服勤,而吏節稟於天朝,宜申禮敬。」司徒府史又以淵既解職,而未恭後授,府猶應上服以不?儉又議:「依中朝士孫德祖從樂陵遷為陳留,未入境,卒,樂陵郡吏依見君之服,陳留迎吏依娶女有吉日齊衰弔,司徒府宜依居官制服。」



 又詔曰:「夫褒德所以紀民,慎終所以歸厚。前王盛典,咸必由之。故侍中、司徒、錄尚書事、新除司空、領驃騎將軍、南康公淵,履道秉哲,鑒識弘曠。爰初弱齡,清風夙舉;登庸應務,具瞻允集。孝友著於家邦,忠貞彰於亮採。佐命先朝,經綸王化,契闊屯夷,綢繆終始。總錄機衡,四門惟穆,諒以同
 規往古,式範來今。



 謙光彌遠,屢陳降挹,權從高旨,用虧大猷。將登上列,永翼聲教。天不憖遺,奄焉薨逝。朕用震慟於厥心。其贈公太宰,侍中、錄尚書、公如故。給節,加羽葆鼓吹,增班劍為六十人。葬送之禮,悉依宋太保王弘故事。謚曰文簡。」



 先是庶姓三公轜車,未有定格。王儉議官品第一,皆加幢絡,自淵始也。又詔淵妻宋故巴西主埏隧暫啟,宜贈南康郡公夫人。



 長子賁,字蔚先。解褐秘書郎。升明中,為太祖太尉從事中郎,司徒右長史,太傅戶曹屬,黃門郎,領羽林監,齊世子中庶子,領翊軍校尉。建元初,仍為宮官,歷侍中。淵薨,服闋,見世祖,賁流涕不自勝。上甚嘉之,以為侍中,領步兵校尉,左民尚書、散騎常侍、秘書監,不拜。六年,上表稱疾,讓封與弟蓁。世以為賁恨淵失節於宋室,故不復仕。永明七年卒,詔賜錢三萬,布五十匹。



 蓁字茂緒。永明中,解褐為員外郎,出為義興太守。八年,改封巴東郡侯。明年,表讓封還賁子霽,詔許之。建武末,為太子詹事,度支尚書,領軍將軍。永元元年,卒,贈太常,謚穆。淵弟澄。



 澄字彥道。初,湛之尚始安公主,薨,納側室郭氏,生淵;後尚吳郡公主,生澄。淵事主孝謹,主愛之。湛之亡,主表淵為嫡。澄尚宋文帝女廬江公主,拜駙馬都尉。歷官清顯。善醫術。建元中,為吳郡太守,豫章王感疾,太祖召澄為治,立愈。尋遷左民尚書。淵薨,澄以錢萬一千就招提寺贖太祖所賜淵白貂坐褥,壞作裘及纓;又贖淵介幘犀導及淵常所乘黃牛。永明元年,為御史中丞袁彖所奏,免官禁錮,見原。遷侍中,領右軍將軍,以勤謹見知。其年卒。澄女為東昏皇后。永元元年,追贈金紫光祿大夫。



 時東陽徐嗣,妙醫術。有一傖父冷病積年,重茵累褥,床下設爐火,猶不差。



 嗣為作治,盛冬月,令傖
 父髁身坐石上,以百瓶水,從頭自灌。初與數十瓶,寒戰垂死,其子弟相守垂泣,嗣令滿數。得七八十瓶後,舉體出氣如雲蒸,嗣令床去被,明日,立能起行。云此大熱病也。又春月出南籬門戲,聞笪屋中有呻吟聲,嗣曰:「此病甚重,更二日不治,必死。」乃往視。一姥稱舉體痛,而處處有濆黑無數,嗣還煮升餘湯送令服之,姥服竟,痛愈甚,跳投床者無數,須臾,所處皆拔出長寸許,乃以膏塗諸瘡口,三日而復,云此名釘疽也。事驗甚多,過於澄矣。



 王儉,字仲寶,瑯琊臨沂人也。祖曇首,宋右光祿。父僧綽,金紫光祿大夫。



 儉生而僧綽遇害,為叔父僧虔所養。數歲,襲爵豫寧侯,拜受茅土,流涕嗚咽。幼有神彩,專心篤學,手不釋卷。丹陽尹袁粲聞其名,言之於明帝,尚陽羨公主,拜駙馬都尉。帝以儉嫡母武康公主同太初巫蠱事,不可以為婦姑,欲開塚離葬,儉因人自陳,密以死
 請,故事不行。解褐秘書郎,太子舍人,超遷秘書丞。上表求校墳籍,依《七略》撰《七志》四十卷,上表獻之,表辭甚典。又撰定《元徽四部書目》。



 母憂,服闋為司徒右長史。《晉令》,公府長史著朝服,宋大明以來著朱衣。



 儉上言宜復舊,時議不許。蒼梧暴虐,儉憂懼,告袁粲求出,引晉新安主婿王獻之為吳興例,補義興太守。還為黃門郎,轉吏部郎。升明二年,遷長兼侍中,以父終此職,固讓。



 儉察太祖雄異,先於領府衣裾,太祖為太尉,引為右長史,恩禮隆密,專見任用。轉左長史。及太傅之授,儉所唱也。少有宰相之志,物議咸相推許。時大典將行,儉為佐命,禮儀詔策,皆出於儉,褚淵唯為禪詔文,使儉參治之。齊臺建,遷右僕射,領吏部,時年二十八。太祖從容謂儉曰:「我今日以青溪為鴻溝。」對曰:「天應民順,庶無楚、漢之事。」建元元年,改封南昌縣公,食邑二千戶。明年,轉左僕射,領選如故。



 上壞宋明帝紫
 極殿,以材柱起宣陽門。儉與諸淵及叔父僧虔連名上表諫曰:「臣聞德者身之基,儉者德之輿。春臺將立,晉卿秉議;北宮肇構,漢臣盡規。彼二君者,或列國常侯,或守文中主,尚使諫諍在義即悅,況陛下聖哲應期,臣等職司隆重,敢藉前誥,竊乃有心!陛下登庸宰物,節省之教既詔;龍袞璇極,簡約之訓彌遠。乾華外構,采椽不斫,紫極故材,為宣陽門,臣等未譬也。夫移心疾於股肱,非良醫之美;畏影跡而馳鶩,豈靜處之方?且又三農在日,千軫咸事,輟望歲之勤,興土木之役,非所以宣昭大猷,光示遐邇。若以門居宮南,重陽所屬,年月稍久,漸就淪胥,自可隨宜修理而合度。改作之煩,於是乎息。所啟謬合,請付外施行。」上手詔酬納。



 宋世外六門設竹籬,是年初,有發白虎樽者,言「白門三重關,竹籬穿不完」。



 上感其言,改立都墻。儉又諫,上答曰:「吾欲令後世無以加也。」朝廷初基,制度草創,
 儉識舊事,問無不答。上嘆曰:「《詩》云『維嶽降神,生甫及申。』今亦天為我生儉也。」



 其年,儉固請解選,表曰:「臣遠尋終古,近察身事,邀恩幸藉,未見其倫。



 何者?子房之遇漢后,公達之逢魏君,史籍以為美談,君子稱其高義。二臣才堪王佐,理非曲私,兩主專仗威武,有傷寬裕,豈與庸流之人,憑含弘之澤者同年而語哉?預在有心,胡寧無感!如使傾宗殞元,有益塵露,猶當畢志驅馳,仰酬萬一,豈容稍在形飾,以徇常事!九流任要,風猷所先,玉石朱素,由斯而定。臣亦不謂文案之間都無微解,至於品裁臧否,特所未閑。雖存自勖,識不副意,兼竊而任,彼此俱壅,專情本官,庶幾仿佛。且前代掌選,未必具在代來,何為於今,非臣不可?傾心奉國,匪復退讓之與;預同休戚,寧俟位任為親。陛下若不以此理賜期,豈仰望於殊眷。頻冒嚴威,分甘尤戾。」見許。加侍中,固讓,復散騎常侍。



 上曲宴群臣數人,各
 使效伎藝。褚淵彈琵琶,王僧虔彈琴,沈文季歌《子夜》,張敬兒舞,王敬則拍張。儉曰:「臣無所解,唯知誦書。」因跪上前誦相如《封禪書》。上笑曰:「此盛德之事,吾何以堪之!」後上使陸澄誦《孝經》,自「仲尼居」而起。儉曰:「澄所謂博而寡要,臣請誦之。」乃誦《君子之事上》章。上曰:「善!張子布更覺非奇也。」尋以本官領太子詹事,加兵二百人。



 上崩,遺詔以儉為侍中、尚書令、鎮軍將軍。世祖即位,給班劍二十人。永明元年,進號衛軍將軍。參掌選事。二年,領國子祭酒、丹陽尹,本官如故。給鼓吹一部。三年,領國子祭酒。叔父僧虔亡,儉表解職,不許。又領太子少傅,本州中正,解丹陽尹。舊太子敬二傅同,至是朝議接少傅以賓友之禮。是歲,省總明觀,於儉宅開學士館,悉以四部書充儉家,又詔儉以家為府。四年,以本官領吏部。儉長禮學,諳究朝儀,每博議,證引先儒,罕有其例。八座丞郎,無能異者。令史諮
 事,賓客滿席,儉應接銓序,傍無留滯。十日一還學,監試諸生,巾卷在庭,劍衛令史儀容甚盛。作解散髻,斜插幘簪,朝野慕之,相與放效。儉常謂人曰:「江左風流宰相,唯有謝安。」蓋自比也。世祖深委仗之,土流選用,奏無不可。



 五年,即本號開府儀同三司,固讓。六年,重申前命。先是詔儉三日一還朝,尚書令史出外諮事;上以往來煩數,復詔儉還尚書下省,月聽十日出外。儉啟求解選,不許。七年,乃上表曰:「臣比年辭選,具簡天明,款言彰於侍接,丹誠布於朝野,物議不以為非,聖心未垂矜納。臣聞知慧不如明時,求之微躬,實允斯義。



 妄庸之人,沈浮無取,命偶休泰,遂踐康衢。秋葉辭條,不假風飆之力;太陽躋景,無俟螢爝之輝。晦往明來,五德遞運,聖不獨治,八元亮採。臣逢其時,而叨其位,常總端右,亟管銓衡,事涉兩朝,歲綿一紀。盛年已老,孫孺巾冠。人物徂遷,逝者將半。三考無聞,九流寂
 寞。能官之詠,輟響於當時;《大車》之刺,方興於來日。若夫珥貂衣兗之貴,四輔六教之華,誠知匪服,職務差簡,端揆雖重,猶可勉勵。至於品藻之任,尤懼其阻。夙宵罄竭,屢試無庸。歲月之久,近世罕比。非唯悔吝在身,故乃惟塵及國。方今多士盈朝,群才競爽,選眾而授,古亦何人。冒陳微翰,必希天照。至敬無文,不敢煩黷。」見許。改領中書監,參掌選事。其年疾,上親臨視。薨,年三十八。



 吏部尚書王晏啟及儉喪,上答曰:「儉年德富盛,志用方隆;豈意暴疾,不展救護,便為異世。奄忽如此,痛酷彌深!其契闊艱運,義重常懷,言尋悲切,不能自勝。痛矣奈何!往矣奈何!」詔衛軍文武及臺所兵仗可悉停待葬。又詔曰:「慎終追遠,列代通規,褒德紀勳,彌峻恆策。故侍中、中書令、太子少傅、領國子祭酒、衛軍將軍、開府儀同三司南昌公儉,體道秉哲,風宇淵曠。肇自弱齡,清猷自遠;登朝應務,民望斯屬。草
 昧皇基,協隆鼎祚。宏謨盛烈,載銘彞篆。及贊朕躬,徽績光茂。忠圖令範,造次必彰。四門允穆,百揆時序。宗臣之重,情寄兼常。方正位論道,永釐袞職,弼茲景化,以贊隆平;天不憖遺,奄焉薨逝,朕用震慟于厥心。可追贈太尉,侍中、中書監、公如故。給節,加羽葆鼓吹,增班劍為六十人。



 葬禮依故太宰文簡公褚淵故事。塚墓材官營辦。謚文憲公。」



 儉寡嗜慾,唯以經國為務,車服塵素,家無遺財。手筆典裁,為當時所重。少撰《古今喪服集記》並文集,並行於世。今上受禪,下詔為儉立碑,降爵為侯,千戶。



 儉弟遜,升明中為丹陽丞,告劉秉事,不蒙封賞。建元初為晉陵太守,有怨言。



 儉慮為禍,因褚淵啟聞。中丞陸澄依事舉奏。詔曰:「儉門世載德,竭誠佐命,特降刑書,宥遜以遠。」徙永嘉郡,道伏誅。



 史臣曰:褚淵、袁粲,俱受宋明帝顧托,粲既死節於宋氏,而淵逢興
 運,世之非責淵者眾矣。臣請論之:夫湯、武之跡,異乎堯、舜,伊、呂之心,亦非稷、契。



 降此風規,未足為證也。自金、張世族,袁、楊鼎貴,委質服義,皆由漢氏,膏腴見重,事起於斯。魏氏君臨,年祚短促,服褐前代,宦成後朝。晉氏登庸,與之從事,名雖魏臣,實為晉有,故主位雖改,臣任如初。自是世祿之盛,習為舊準,羽儀所隆,人懷羨慕,君臣之節,徒致虛名。貴仕素資,皆由門慶,平流進取,坐至公卿,則知殉國之感無因,保家之念宜切。市朝亟革,寵貴方來,陵闕雖殊,顧眄如一。中行、智伯,未有異遇。褚淵當泰始初運,清塗已顯,數年之間,不患無位,既以民望而見引,亦隨民望而去之。夫爵祿既輕,有國常選,恩非己獨,責人以死,斯故人主之所同謬,世情之過差也。



 贊曰:猗歟褚公,德素內充。民譽不爽,家稱克隆。從容佐世,貽議匪躬。文憲濟濟,輔相之體。稱述霸王,綱維典禮。期寄兩朝,綢繆宮陛。



\end{pinyinscope}