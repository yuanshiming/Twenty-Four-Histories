\article{卷三十五列傳第十六 高帝十二王}

\begin{pinyinscope}

 高帝十九男:昭皇后生武帝、豫章文獻王嶷;謝貴嬪生臨川獻王映、長沙威王晃;羅太妃生武陵昭王曄;任太妃生安成恭王暠;陸脩儀生鄱陽王鏘、晉熙王銶;袁脩容生桂陽王鑠;何太妃生始興簡王鑒、宜都王鏗;區貴人生衡陽王鈞;張淑妃生江夏王鋒、河東王鉉;李美人生南平王銳;第九、第十三、第十四、第十七皇子早亡。衡陽王鈞出繼元王後。



 臨川獻王映,字宣光,太祖第三子也。宋元徽四年,解褐著作佐郎,遷撫軍行參軍,南陽王文學。沈攸之事難,太祖時領南徐州,以映為寧朔將軍,鎮京口。事寧,除中軍諮議、從事中郎、輔國將軍、淮南
 宣城二郡太守,並不拜。仍為假節、督南兗兗徐青冀五州諸軍事、行南兗州刺史,將軍如故。尋除給事黃門侍郎,領前軍將軍,仍復為冠軍將軍、南兗州刺史,假節督,復為監軍,督五州如故。



 齊臺建,宋帝詔封映及弟晃、曄、暠、鏘、鑠、鑒並為開國縣公,各千五百戶,未及定土宇,而太祖踐阼。以映為使持節、都督荊湘雍益梁寧南北秦八州諸軍事、平西將軍、荊州刺史。封臨川王,食邑例二千戶。又領湘州刺史。豫章王嶷既留鎮陜西,映亦不行。改授散騎常侍、都督揚南徐二州諸軍事、前將軍、揚州刺史,持節如故。國家初創,映以年少臨神州,吏治聰敏,府州曹局皆重足以奉禁令,自宋彭城王義康以後未之有也。



 出為都督荊湘雍益梁巴寧南北秦九州諸軍事、鎮西將軍、荊州刺史,持節、常侍如故。給鼓吹一部。以國憂解散騎常侍,進號征西。永明元年入為侍中,驃騎將軍。二年,給
 油絡車。五年,即本號開府儀同三司。七年,薨。



 映善騎射,解聲律,工左右書左右射,應接賓客,風韻韶美,朝野莫不惋惜焉。



 時年三十二。詔賜東園秘器,朝服一具,衣一襲。贈司空。九子皆封侯。



 長子子晉,歷東陽、吳興二郡太守,秘書監,領後軍將軍。永元初,為侍中,遷左民尚書。坐從妹祖日不拜,為有司所奏,事留中,子晉遂不復拜。梁王定京邑,猶服侍中服。入梁為輔國將軍、高平太守。第二子子游,州陵侯。解褐員外郎,太子洗馬,歷琅邪、晉陵二郡太守,黃門侍郎。好音樂,解絲竹雜藝。梁初坐閨門淫穢及殺人,為有司所奏,請議禁錮。子晉謀反,兄弟並伏誅。



 長沙威王晃,字宣明,太祖第四子也。少有武力,為太祖所愛。宋世解褐秘書郎邵陵王友,不拜。升明二年,代兄映為寧朔將軍、淮南宣城二郡太守。初,沈攸之事起,晃便弓馬,多從武容,熏赫都街,時
 人為之語曰:「煥煥蕭四傘。」其年,遷為持節、監豫司二州郢州之西陽諸軍事、西中郎將、豫州刺史。



 太祖踐祚,晃欲陳政事,輒為典簽所裁,晃執殺之。上大怒,手詔賜杖。尋遷使持節、都督南徐兗二州諸軍事、後將軍、南徐州刺史。世祖為皇太子,拜武進陵,於曲阿後湖鬥隊,使晃御馬軍,上聞之,又不悅。入為侍中、護軍將軍,以國憂,解侍中,加中軍將軍。太祖臨崩,以晃屬世祖,處以輦轂近蕃,勿令遠出。永明元年,上遷南徐州刺史竟陵王子良為南兗州,以晃為使持節、都督南徐兗二州諸軍事、鎮軍將軍、南徐州刺史。入為散騎常侍,中書監。



 諸王在京都,唯置捉刀左右四十人。晃愛武飾,罷徐州還,私載數百人仗還都,為禁司所覺,投之江水。世祖禁諸王畜私仗,聞之大怒,將糾以法。豫章王嶷於御前稽首流涕曰:「晃罪誠不足宥。陛下當憶先朝念白象。」白象,晃小字也。上亦垂泣。太祖大漸時,
 誡世祖曰:「宋氏若不骨肉相圖,他族豈得乘其衰弊,汝深戒之。」故世祖終無異意。然晃亦不見親寵。當時論者以世祖優於魏文,減於漢明。



 尋加晃鎮軍將軍,轉丹陽尹,常侍、將軍如故。又為侍中、護軍將軍,鎮軍如故。尋進號車騎將軍,侍中如故。給油絡車,鼓吹一部。八年,薨,年三十一。賜東園秘器,朝服一具,衣一襲。即本號,贈開府儀同三司。



 世祖嘗幸鐘山,晃從駕,以馬槊刺道邊枯蘗,上令左右數人引之,銀纏皆卷聚,而槊不出。乃令晃復馳馬拔之,應手便去。每遠州獻駿馬,上輒令晃於華林中調試之。太祖常曰:「此我家任城也。」世祖緣此意,故謚曰威。



 武陵昭王曄,字宣照,太祖第五子也。母羅氏,從太祖在淮陰,以罪誅,曄年四歲,思慕不異成人,故每見愛。初除冠軍將軍,轉征虜將軍。曄剛穎俊出,工弈棋,與諸王共作短句,詩學謝靈運
 體,以呈上,報曰:「見汝二十字,諸兒作中最為優者。但康樂放蕩,作體不辨有首尾,安仁、士衡深可宗尚,顏延之抑其次也。」



 建元三年,出為持節、都督會稽東陽新安永嘉臨海五郡軍事、會稽太守,將軍如故。



 上遣儒士劉瓛往郡,為曄講《五經》。



 世祖即位,進號左將軍,入為中書令,將軍如故。轉散騎常侍,太常卿。又為中書令,遷祠部尚書,常侍並如故。



 曄無寵於世祖,未嘗處方岳,數以語言忤旨。世祖幸豫章王嶷東田宴諸王,獨不召曄。嶷曰:「風景殊美,今日甚憶武陵。」上乃呼之。曄善射,屢發命中,顧謂四坐曰:「手如何?」上神色甚怪。嶷曰:「阿五常日不爾,今可謂仰藉天威。」



 帝意乃釋。後於華林賭射,上敕曄疊破,凡放六箭,五破一皮,賜錢五萬。又於御席上舉酒勸曄,曄曰:「陛下嘗不以此處許臣。」上回面不答。



 久之,出為江州刺史,常侍如故。上以曄方出外鎮,求曄宅給諸皇子。曄曰:「先帝賜臣此宅,
 使臣歌哭有所。陛下欲以州易宅,臣請不以宅易州。」至鎮百餘日,典簽趙渥之啟曄得失,於是徵還為左民尚書。俄轉前將軍,太常卿,累不得志。



 冬節問訊,諸王皆出,曄獨後來,上已還便殿,聞曄至,引見問之。曄稱牛羸,不能取路。上敕車府給副御牛一頭。敕主客:「自今諸王來不隨例者,不得復為通。」



 以公事還過竟陵王子良宅,冬月道逢乞人,脫襦與之。子良見曄衣單,薦襦於曄。曄曰:「我與向人亦復何異!」尚書令王儉詣曄,曄留儉設食,柈中菘菜邑魚而已。又名後堂山為「首陽」,蓋怨貧薄也。



 尋為丹陽尹,常侍、將軍如故。始不復置行事,得自親政。轉侍中,護軍將軍。



 給油絡車。又給扶二人。世祖臨崩,遺詔為衛將軍,開府儀同三司,給鼓吹一部。



 大行在殯,竟陵王子良在殿內,太孫未立,眾論喧疑。曄眾中言曰:「若立長則應在我,立嫡則應在太孫。」鬱林即立,甚見憑賴。隆昌元年,年二十八,
 薨。



 賜東園秘器,朝服。贈司空,侍中如故。給節,班劍二十人。



 安成恭王暠,字宣曜,太祖第六子也。建元二年,除冠軍將軍,鎮石頭戍,領軍事。四年,出為使持節、督江州豫州之晉熙諸軍事、南中郎將、江州刺史。永明元年,進號征虜將軍。明年,為左衛將軍。尋遷侍中,領步兵校尉。轉中書令。五年,遷祠部尚書,領驍騎將軍。六年,出為南徐州刺史。九年,遷散騎常侍,秘書監,領石頭戍事。暠性清和多疾,其夏薨,年二十四。贈撫軍將軍,常侍如故。



 鄱陽王鏘,字宣韶,太祖第七子也。建元四年,世祖即位,以鏘為使持節、督雍梁南北秦四州郢州之竟陵司州之隨郡軍事、北中郎將、寧蠻校尉、雍州刺史。永明二年,進號征虜將軍。四年,為左衛將軍,遷侍中,領步兵校尉。七年,轉征虜將軍,丹陽尹。尋加散騎常侍,進號撫軍。出為江州刺史,常侍如故。九年,始親府、州事。加使持節、
 督江州諸軍事、安南將軍,置佐史,常侍如故。先是二年省江州府,至是乃復。十一年,為領軍,常侍如故。



 鏘和悌美令,有寵於世祖,領軍之授,齊室諸王所未為。鏘在官理事無壅,當時稱之。車駕遊幸,常甲仗衛從,恩待次豫章王嶷。其年,給油絡車。隆昌元年,轉尚書右僕射,常侍如故。俄遷侍中、驃騎將軍、開府儀同三司,領兵置佐。



 鏘雍容得物情,為鬱林王所依信。鬱林心疑高宗,諸王問訊,獨留鏘謂之曰:「公聞鸞於法身何如?」鏘曰:「臣鸞於宗戚最長,且受寄先帝。臣等年皆尚少,朝廷之乾,唯鸞一人,願陛下無以為慮。」鬱林退謂徐龍駒曰:「我欲與公共計取鸞,公既不同,我不能獨辦,且復小聽。」及鬱林廢,鏘竟不知。



 延興元年,進位司徒,侍中、驃騎如故。高宗鎮東府,權勢稍異,鏘每往,高宗常屣履至車迎鏘。語及家國,言淚俱下,鏘以此推信之。而宮臺內皆屬意於鏘,
 勸鏘入宮發兵輔政。制局監謝粲說鏘及隨王子隆曰:「殿下但乘油壁車入宮,出天子置朝堂,二王夾輔號令,粲等閉城門上仗,誰敢不同?東城人政共縛送蕭令耳。」



 子隆欲定計,鏘以上臺兵力既悉度東府,且慮事難捷,意甚猶豫。馬隊主劉巨,世祖時舊人,詣鏘請間,叩頭勸鏘立事。鏘命駕將入,復回還內與母陸太妃別,日暮不成行。數日,高宗遣二千人圍鏘宅害鏘,謝粲等皆見殺。鏘時年二十六。凡諸王被害,皆以夜遣兵圍宅,或斧關排墻叫噪而入,家財皆見封籍焉。



 桂陽王鑠,字宣朗,太祖第八子也。永明二年,出為南徐州刺史,鎮京口。歷代鎮府,鑠出蕃,始省軍府。四年,加散騎常侍。六年,遷中書令,度支尚書。七年,轉中書令,加散騎常侍。時鄱陽王鏘好文章,鑠好名理,時人稱為「鄱桂。」



 十年,遷太常,常侍
 如故。鑠清羸有冷疾,常枕臥。世祖臨視,賜床帳衾褥。隆昌元年,加前將軍。給油絡車,並給扶侍二人。海陵立,轉侍中、撫軍將軍,領兵置佐。



 鄱陽王見害,鑠遷中軍將軍,開府儀同三司。鑠不自安,至東府詣高宗還,謂左右曰:「向錄公見接殷勤,流連不能已,而貌有慚色,此必欲殺我。」三更中,兵至見害。時年二十五。



 始興簡王鑒,字宣徹,太祖第十子也。初封廣興王,後國隨郡改名。永明二年,世祖始以鑒為持節、都督益寧二州軍事、前將軍、益州刺史。廣漢什邡民段祖以錞于獻鑒,古禮器也。高三尺六寸六分,圍二尺四寸,圓如筒,銅色黑如漆,甚薄。



 上有銅馬,以繩縣馬,令去地尺餘,灌之以水,又以器盛水於下,以芒莖當心跪注錞于,以手振芒,則其聲如雷,清響良久乃絕。古所以節樂也。五年,鑒獻龍角一枚,長九尺三寸,色紅,有文。八年,進號安西將軍。



 明年,為散騎常侍,秘書監,領石頭戍事。上以與鑒久別,車駕幸石頭宴會賞賜。尋遷左衛將軍,未拜,遇疾。上為南康王子琳起青陽巷第新成,車駕與後宮幸第樂飲,其日鑒疾甚,上遣騎問疾相繼,為之詔止樂。薨,年二十一。遣贈中軍將軍,本官新除悉如故。



 江夏王鋒,字宣穎,太祖第十二子。永明五年,為輔國將軍,南彭城、平昌二郡太守。轉散騎常侍。七年,遷左衛將軍,仍轉侍中,領石頭戍事。九年,出為徐州刺史。鬱林即位,加散騎常侍。隆昌元年,入為侍中,領驍騎將軍,尋加秘書監。



 鋒好琴書,
 有武力。高宗殺諸王,鋒遺書誚責,左右不為通,高宗深憚之。不敢於第收鋒,使兼祠官於太廟,夜遣兵廟中收之。鋒出登車,兵人欲上車防勒,鋒以手擊卻數人,皆應時倒地,於是敢近者遂逼害之。時年二十。



 南平王銳,字宣毅,太祖第十五子也。永明七年,為散騎常侍,尋領驍騎將軍。



 明年,為左民尚書。朝直勤謹,未嘗屬疾,上嘉之。十年,出為持節、都督湘州諸軍事、南中郎將、湘州刺史,以此賞銳。鬱林即位,進號前將軍。



 延興元年,害諸王,遣裴叔業平尋陽,仍進湘州。銳防閣周伯玉勸銳拒叔業,而府州力弱不敢動,銳見害,年十九。伯玉下獄誅。



 宜都王鏗,字宣嚴,太祖第十六子也。初除遊擊將軍。永明十年,遷左民尚書。



 十一年,為持節、都督南豫司二州軍事、冠軍將軍、南豫州刺史,鎮姑熟。時有盜發晉大司馬桓溫女塚,得金蠶銀繭及珪璧等物。鏗使長史蔡約自往脩復,纖毫不犯。



 鬱林即位,進號征虜將軍。延興元年見害,年十八。



 晉熙王銶,字宣攸,太祖第十八子也。永明十一年,除驍騎將軍。隆昌元年,出為持節、督郢司二州軍事、冠軍將軍、郢州刺史。延興元年,進號征虜將軍。尋見害,年十六。



 河東王鉉,字宣胤,太祖第十九子也。隆昌元年,為驍騎將軍。出為徐州刺史,遷中書令。高宗誅諸王,以鉉年少才弱,故不加害。建武元年,轉為散騎常侍,鎮軍將軍,置兵佐。



 建武之世,高、武子孫憂危,鉉每朝見,常鞠躬俯僂,不敢平行直視。尋遷侍中、衛將軍。鉉年稍長。四年,誅王晏,以謀立鉉為名,免鉉官,以王還第,禁不得與外人交通。永泰元年,上疾暴甚,遂害鉉,時年十九。二子在孩抱,亦見殺。



 太祖諸王,鉉獨無後,眾竊冤之。乃使揚州刺史始安王遙光、臨川王子晉、竟陵王昭胄、太尉陳顯達、尚書令徐孝嗣、右僕射沈文季、尚書沈淵、沈約、王亮奏論鉉,帝答不許,再奏,乃從之。



 史臣曰:陳思王表云:「權之所存,雖疏必重;勢之所去,雖親必輕。」若夫六代之興亡,曹冏論之當矣。分珪命社,實寄宗城。就國之典,既隨世革,卿士入朝,作貴蕃輔。皇王託體,同稟尊極,仕無常資,秩有
 恆數,禮地兼隆,易生猜疑。



 世祖顧命,情深尊嫡,淵圖遠算,意在無遺。豈不以群王少弱,未更多難,高宗清謹,同起布衣,故韜末命於近親,寄重權於疏戚,子弟布列,外有強大之勢,疏親中立,可息覬覦之謀,表裏相維,足固家國。曾不慮機能運衡,權可制眾,宗族殲滅,一至於斯。曹植之言信之矣。



 贊曰:高十二王,始建封植。獻、昭機警,威、江才力。恭、簡恬和,鄱、桂清識。四王少盛,同規謹敕。



\end{pinyinscope}