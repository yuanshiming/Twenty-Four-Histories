\article{卷二十列傳第一 皇后}

\begin{pinyinscope}

 六
 宮位號,漢、
 魏以來,因襲增置,世不同矣。建元元年,有司奏置貴嬪、夫人、貴人為三夫人,修華、修儀、修容、淑妃、淑媛、淑儀、婕妤、容華、充華為九嬪,美人、中才人、才人為散職。永明元年,有司奏貴妃、淑妃並加金章紫綬,佩于窴玉。淑妃舊擬九棘,以淑為溫恭之稱,妃為亞後之名,進同貴妃,以比三司。



 夫人之號,不殊蕃國。降淑媛以比九卿。七年,復置昭容,位在九嬪。建元三年,太子宮置三內職,良娣比開國侯,保林比五等侯,才人比駙馬都尉。



 宣孝陳皇后,諱道正,臨淮東陽人,魏司徒陳矯後。父肇之,郡孝廉。後少家貧,勤織作。家人矜其勞,或止之,後終不改。嫁於
 宣帝,庶生衡陽元王道度、始安貞王道生,后生太祖。太祖年二歲,乳人乏乳,后夢人以兩甌麻粥與之,覺而乳大出,異而說之。宣帝從仕在外,后常留家治事教子孫。有相者謂后曰:「夫人有貴子而不見也。」后歎曰:「我三兒誰當應之!」呼太祖小字曰:「正應是汝耳。」



 宣帝殂後,后親自執勤,婢使有過誤,恕不問也。太祖雖從官,而家業本貧,為建康令時,高宗等冬月猶無縑纊,而奉膳甚厚。後每撤去兼肉,曰:「於我過足矣。」



 殂于縣舍,年七十三。昇明三年,追贈竟陵公國太夫人,蜜印,畫青綬,祠以太牢;建元元年,追尊孝皇后。贈外祖父肇之金紫光祿大夫,謚曰敬侯。后母胡氏為永昌縣靖君。



 高昭劉皇后,諱智容,廣陵人也。祖玄之,父壽之,並員外郎。後母桓氏夢吞玉勝生后,時有紫光滿室,以告壽之,壽之曰:「恨非是男。」桓曰:「雖女,亦足興家矣。」後母寢臥,家人常見上如有雲氣焉。年十
 餘歲,歸太祖,嚴正有禮法,家庭肅然。宋泰豫元年殂,年五十。歸葬宣帝墓側,今泰安陵也。門生王清與墓工始下鍤,有白兔跳起,尋之不得,及墳成,兔還棲其上。昇明二年,贈竟陵公國夫人;三年,贈齊國妃;印綬如太妃。建元元年,尊謚昭皇后。三年,贈后父金紫光祿大夫,母桓氏上虞都鄉君;壽之子興道司徒屬,文蔚豫章內史,義徽光祿大夫,義倫通直郎。



 武穆裴皇后,諱惠昭,河東聞喜人也。祖樸之,給事中。父璣之,左軍參軍。



 后少與豫章王妃庾氏為娣姒,庾氏勤女工,奉事太祖、昭后恭謹不倦,后不能及,故不為舅姑所重,世祖家好亦薄焉。性剛嚴,竟陵王子良妃袁氏布衣時有過,后加訓罰。升明三年,為齊世子妃。建元元年,為皇太子妃。三年,后薨。謚穆妃,葬休安陵。世祖即位,追尊皇后。贈璣之金紫光祿大夫,后母檀氏餘杭廣昌鄉元君。



 舊
 顯陽、昭陽殿,太后、皇后所居也。永明中無太后、皇后,羊貴嬪居昭陽殿西,范貴妃居昭陽殿東,寵姬荀昭華居鳳華柏殿。宮內御所居壽昌畫殿南閣,置白鷺鼓吹二部,乾光殿東西頭,置鐘磬兩廂,皆宴樂處也。上數遊幸諸苑囿,載宮人從後車。宮內深隱,不聞端門鼓漏聲,置鐘於景陽樓上,宮人聞鐘聲,早起裝飾。



 至今此鐘唯應五鼓及三鼓也。車駕數幸琅邪城,宮人常從,早發至湖北埭,雞始鳴。



 吳郡韓蘭英,婦人有文辭。宋孝武世,獻《中興賦》,被賞入宮。宋明帝世,用為宮中職僚。世祖以為博士,教六宮書學,以其年老多識,呼為「韓公」。



 文安王皇后,諱寶明,琅邪臨沂人也。祖韶之,吳興太守,父曄之,太宰祭酒。



 宋世,太祖為文惠太子納后,桂陽賊至,太祖在新亭,傳言已沒,宅復為人所抄掠,文惠太子、竟陵王子良奉
 穆后、庾妃及後挺身送后兄昺之家,事平乃出。建元元年,為南郡王妃。四年,為皇太子妃,無寵。太子為宮人製新麗衣裳及首飾,而後床帷陳設故舊,釵鑷十餘枚。永明十一年,為皇太孫太妃。鬱林即位,尊為皇太后,稱宣德宮。贈後父金紫光祿大夫,母桓氏豐安縣君。其年十二月,備法駕謁太廟,高宗即位,出居鄱陽王故第,為宣德宮。永元三年,梁王定京邑,迎后入宮稱制,至禪位。天監十一年,薨,年五十八。葬崇安陵。謚曰安后。兄晃義興太守。



 鬱林王何妃,名婧英,廬江灊人,撫軍將軍戢之女也。永明二年,納為南郡王妃。十一年,為皇太孫妃。鬱林王即位,為皇后。嫡母劉氏為高昌縣都鄉君,所生母宋氏,為餘杭廣昌鄉君。將拜,鏡在床無故墮地。其冬,與太后同日謁太廟。後稟性淫亂,為妃時,便與外人奸通。在後宮,復通帝左右楊氏之,與同寢處如伉儷。



 氏之又與帝
 相愛褻,故帝恣之。迎后親戚入宮,賞賜人百數十萬。以世祖耀靈殿處後家屬。帝被廢,後貶為王妃。



 海陵王王妃,名韶明,瑯邪臨沂人,太常慈女也。永明八年,納為臨汝公夫人。



 鬱林即位,為新安王妃。延興元年,為皇后。其年,降為海陵王妃。



 明敬劉皇后,諱惠端,彭城人,光祿大夫道弘孫也。太祖為高宗納之。建元三年,除西昌侯夫人。永明七年,卒,葬江乘縣張山。延興元年,贈宣城王妃;高宗即位,追尊為敬皇后。贈父通直郎景猷金紫光祿大夫,母王氏平陽鄉君。永泰元年,高宗崩,改葬,祔于興安陵。



 東昏褚皇后,名令璩,河南陽翟人,太常澄女也。建武二年,納為皇太子妃。



 明年,謁敬后廟。東昏即位,為皇后。帝寵潘妃,后不被遇。黃淑儀生太子誦,東昏廢,並為
 庶人。



 和帝王皇后,名蕣華,瑯邪臨沂人,太尉儉孫也。初為隨王妃。中興元年,為皇后。帝禪位,後降為妃。



 史臣曰:后妃之德,著自風謠,義起閨房,而道化天下。繰盆獻種,罔非耕織,佩管晨興,與子同事,可以光熙閫業,作儷公侯。孝、昭二後,並有賢明之訓,不得母臨萬國。寶命方昌,椒庭虛位,有婦人焉,空慕周興,禎符顯瑞,徒萃徽名。



 若使掖作同休,陰教遠燮,則馬、鄧風流,復存乎此。太祖創命,宮禁貶約,毀宋明之紫極,革前代之逾奢,衣不文繡,色無紅採,永巷貧空,有同素室。世祖嗣位,運藉休平,壽昌前興,鳳華晚構,香柏文檉,花梁繡柱,雕金鏤寶,頗用房帷,趙瑟《吳趨》,承閑奏曲,歲費傍恩,足使充牣,事由私蓄,無損國儲。高宗仗數矯情,外行儉陋,內奉宮業,曾莫云改。東昏喪道,侈風大扇,銷糜海內,以贍浮飾,哲婦傾城,同符殷、夏。嗚呼!所以垂戒於方來
 也。



 贊曰:宣武孝則,識有先知。高昭誕武,世載母儀。裴穆儲閫,位亦從隳。明敬典冊,配在宗枝。秋宮亦遽,軒景前虧。文安廢主,百憂已離。中興秉制,揖讓弘規。



\end{pinyinscope}