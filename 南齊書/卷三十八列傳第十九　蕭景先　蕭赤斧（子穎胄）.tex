\article{卷三十八列傳第十九 蕭景先 蕭赤斧(子穎胄)}

\begin{pinyinscope}

 蕭景先,南蘭陵蘭陵人,太祖從子也。祖爰之,員外郎。父敬宗,始興王國中軍。



 景先少遭父喪,有至性,太祖嘉之。及從官京邑,常相提攜。解褐為海陵王國上軍將軍,補建陵令,還為新安王國侍郎,桂陽國右常侍。太祖鎮淮陰,景先以本官領軍主自隨,防衛城內,委以心腹。除後軍行參軍,邛縣令,員外郎。與世祖款暱,世祖為廣興郡,啟太祖求景先同行,除世祖寧朔府司馬,自此常相隨逐。世祖為鎮西長史,以景先為鎮西長流參軍,除寧朔將軍,隨府轉撫軍中兵參軍,尋除諮議,領中兵如故。升明
 初,為世祖征虜府司馬,領新蔡太守,隨上鎮盆城。沈攸之事平,還都,除寧朔將軍,驍騎將軍,仍為世祖撫軍中軍二府司馬,兼左衛將軍。



 建元元年,遷太子左衛率,封新吳縣伯,邑五百戶。景先本名道先,乃改避上諱。



 出為持節、督司州軍事、寧朔將軍、司州刺史,領義陽太守。是冬,虜出淮、泗,增司部邊戍兵。義陽人謝天蓋與虜相構扇,景先言於督府,驃騎豫章王遣輔國將軍中兵參軍蕭惠朗二千人助景先。惠朗依山築城,斷塞關隘,討天蓋黨與。虜尋遣偽南部尚書頞跋屯汝南,洛州刺史昌黎王馮莎屯清丘。景先嚴備待敵。豫章王又遣寧朔將軍王僧炳、前軍將軍王應之、龍驤將軍莊明三千人屯義陽關外,為聲援。



 虜退,進號輔國將軍。



 景先啟稱上德化之美。上答曰:「風淪俗敗,二十餘年,以吾當之,豈得頓掃。



 幸得數載盡力救蒼生者,必有功於萬物也。治天下者,雖聖人猶須良佐,汝等各各自竭,不憂
 不治也。」



 世祖即位,徵為侍中,領左軍將軍,尋兼領軍將軍。景先事上盡心,故恩寵特密。初西還,上坐景陽樓召景先語故舊,唯豫章王一人在席而已。轉中領軍。車駕射雉郊外行游,景先常甲仗從,廉察左右。尋進爵為侯。領太子詹事,本官如故。



 遭母喪,詔超起為領軍將軍。遷征虜將軍、丹陽尹。五年,荒人桓天生引蠻虜於雍州界上,司部以北人情騷動。上以景先諳究司土,詔曰:「得雍州刺史張瑰啟事,蠻虜相扇,容或侵軼。蜂蠆有毒,宜時剿蕩。可遣征虜將軍丹陽尹景先總率步騎,直指義陽。可假節,司州諸軍皆受節度。」景先至鎮,屯軍城北,百姓乃安,牛酒來迎。



 軍未還,遇疾,遺言曰:「此度疾病,異於前後,自省必無起理。但夙荷深恩,今謬充戎寄,暗弱每事不稱,上慚慈旨。便長違聖世,悲哽不知所言。可為作啟事,上謝至尊,粗申愚心。毅雖成長,素闕訓範。貞等幼稚,未有所識。方以
 仰累聖明,非殘息所能陳謝。自丁荼毒以來,妓妾已多分張,所餘醜猥數人,皆不似事。可以明月、佛女、桂支、佛兒、玉女、美玉上臺,美滿、艷華奉東宮。私馬有二十餘匹,牛數頭,可簡好者十匹、牛二頭上臺,馬五匹、牛一頭奉東宮,大司馬、司徒各奉二匹,驃騎、鎮軍各奉一匹。應私仗器,亦悉輸臺。六親多未得料理,可隨宜溫恤,微申素意。所賜宅曠大,恐非毅等所居,須喪服竟,可輸還臺。劉家前宅,久聞其貨,可合率市之,直若短少,啟官乞足。三處田勤作,自足供衣食。力少,更隨宜買粗猥奴婢充使。不須餘營生。周旋部曲還都,理應分張,其久舊勞勤者,應料理,隨宜啟聞乞恩。」卒,時年五十。上傷惜之,詔曰:「西信適至,景先奄至喪逝,悲懷切割,自不勝任。今便舉哀。賻錢十萬。布二百匹。」景先喪還,詔曰:「故假節征虜將軍丹陽尹新吳侯景先,器懷開亮,幹局通敏。綢繆少
 長,義兼勛戚。誠著夷險,績茂所司。方升寵榮,用申任寄。奄至喪逝,悲痛良深。可贈侍中、征北將軍、南徐州刺史。給鼓吹一部。假節、侯如故。謚曰忠侯。」



 子毅,以勛戚子,少歷清官:太子舍人,洗馬,隨王友,永嘉太守,大司馬諮議參軍,南康太守,中書郎。建武初,為撫軍司馬,遷北中郎司馬。虜動,領軍守瑯邪城。毅性奢豪,好弓馬,為高宗所疑忌。王晏事敗,并陷誅之。遣軍圍宅,毅時會賓客奏伎,聞變,索刀未得,收人突進,挾持毅入與母別,出便殺之。



 蕭赤斧,南蘭陵人,太祖從祖弟也。祖隆子,衛軍錄事參軍。父始之,冠軍中兵參軍。赤斧歷官為奉朝請,以和謹為太祖所知。宋大明初,竟陵王誕反廣陵,赤斧為軍主,隸沈慶之。圍廣陵城,攻戰有勳,事寧,封永安亭侯,食邑三百七十戶。



 除車騎行參軍,出補晉陵令,員外郎,丹陽令,還除晉熙王撫軍中兵參軍,出為建威將軍、錢唐
 令。遷正員郎。赤斧治政為百姓所安,吏民請留之,時議見許,改除寧朔將軍。



 太祖輔政,以赤斧為輔國將軍、左軍會稽司馬,輔鎮東境。遷黃門郎,淮陵太守。順帝遜位,於丹陽故治立宮,上令赤斧輔送,至薨乃還。建元初,遷武陵王冠軍長史,驃騎司馬,南東海太守,輔國將軍並如故。遷長兼侍中,祖母喪去職。起為冠軍將軍、寧蠻校尉。出為持節、督雍梁南北秦四州郢州之竟陵司州之隨郡軍事、雍州刺史,本官如故。在州不營產利,勤於奉公。遷散騎常侍,左衛將軍。世祖親遇與蕭景先相比。封南豐縣伯,邑四百戶。遷給事中,太子詹事。



 赤斧夙患渴利,永明三年會,世祖使甲仗衛三廂,赤斧不敢辭,疾甚,數日卒,年五十六。家無儲積,無絹為衾,上聞之,愈加惋惜。詔賻錢五萬,上材一具,布百匹,蠟二百斤。追贈金紫光祿
 大夫。謚曰懿伯。子穎胄襲爵。



 穎胄字雲長,弘厚有父風。起家秘書郎。太祖謂赤斧曰:「穎胄輕朱被身,覺其趨進轉美,足慰人意。」遷太子舍人。遭父喪,感腳疾,數年然後能行。世祖有詔慰勉,賜醫藥。除竟陵王司徒外兵參軍,晉熙王文學。



 穎胄好文義,弟穎基好武勇。世祖登烽火樓,詔群臣賦詩。穎胄詩合旨,上謂穎胄曰:「卿文弟武,宗室便不乏才。」除明威將軍、安陸內史。遷中書郎。上以穎胄勛戚子弟,除左將軍,知殿內文武事,得入便殿。出為新安太守,吏民懷之。



 隆昌元年,永嘉王昭粲為南徐州,以穎胄為南東海太守,行南徐州事。轉持節、督青冀二州軍事、輔國將軍、青冀二州刺史。不行,除黃門郎,領四廂直。遷衛尉。



 高宗廢立,穎胄從容不為同異,乃引穎胄預功。建武二年,進爵侯,增邑為六百戶。賜穎胄以常所乘白牛俞牛。



 上慕儉約,欲鑄壞太官元日上壽銀酒槍,尚書令王晏等咸稱盛德。穎胄曰:「朝廷盛禮,莫過三
 元。此一器既是舊物,不足為侈。」帝不悅,後預曲宴,銀器滿席。穎胄曰:「陛下前欲壞酒槍,恐宜移在此器也。」帝甚有慚色。



 冠軍江夏王寶玄鎮石頭,以穎胄為長史,行石頭戍事。復為衛尉。出為冠軍將軍、廬陵王後軍長史、廣陵太守、行南兗州、府州事。是年虜動,揚聲當飲馬長江。



 帝懼,敕穎胄移居民入城,百姓驚恐,席卷欲南渡。穎胄以賊勢尚遠,不即施行,虜亦尋退。仍為持節、督南兗兗徐青冀五州諸軍事、輔國將軍、南兗州刺史。



 和帝為荊州,以穎胄為冠軍將軍、西中郎長史、南郡太守、行荊州府、州事。



 東昏侯誅戮群公,委任廝小,崔、陳敗後,方鎮各懷異計。永元二年十月,尚書令臨湘侯蕭懿及弟衛尉暢見害。先遣輔國將軍、巴西梓潼二郡太守劉山陽領三千兵受旨之官,就穎胄共襲雍州。雍州刺史梁王將起義兵,慮穎胄不識機變,遣使王天虎詣江陵,聲雲山陽西上,並襲
 荊、雍。書與穎胄,勸同義舉。穎胄意猶未決。初,山陽出南州,謂人曰:「朝廷以白虎幡追我,亦不復還矣。」席卷妓妾,盡室而行。



 至巴陵,遲回十餘日不進。梁王復遣天虎齎書與穎胄,陳設其略。是時或云山陽謀殺穎胄,以荊州同義舉,穎胄乃與梁王定契,斬王天虎首,送示山陽。發百姓車牛,聲云起步軍征襄陽。十一月十八日,山陽至江津,單車白服,從左右數十人,詣穎胄,穎胄使前汶陽太守劉孝慶、前永平太守劉熙曄、鎧曹參軍蕭文照、前建威將軍陳秀、輔國將軍孫末伏兵城內。山陽入門,即於車中亂斬之。副軍主李元履收餘眾歸附。遣使蔡道猷馳驛送山陽首於梁王,乃發教纂嚴,分部購募。東昏聞山陽死,發詔討荊、雍。贈山陽寧朔將軍、梁州刺史。



 穎胄有器局,既唱大事,虛心委己,眾情歸之。加穎胄右將軍,都督行留諸軍事,置佐史,本官如故。西中郎司馬夏侯詳加征虜
 將軍。遣寧朔將軍王法度向巴陵。



 穎胄獻錢二十萬,米千斛,鹽五百斛。諮議宗塞、別駕宗夬獻穀二千斛,牛二頭。



 換借富貲,以助軍費。長沙寺僧業富,沃鑄黃金為龍數千兩,埋土中,歷相傳付,稱為下方黃鐵,莫有見者,乃取此龍,以充軍實。



 十二月,移檄:西中郎府長史、都督行留諸軍事、右軍將軍、南郡太守、南豐縣開國侯蕭穎胄,司馬、征虜將軍、新興太守夏侯詳告京邑百官,諸州郡牧守:夫運不常夷,有時而陂;數無恆剝,否極則亨。昔商邑中微,彭、韋投袂;漢室方昏,虛、牟效節。故風聲永樹,卜世長久者也。昔我太祖高皇帝德範生民,功格天地,仰緯彤雲,俯臨紫極。世祖嗣興,增光前業,雲雨之所沾被,日月之所出入,莫不舉踵來王,交臂納貢。鬱林昏迷,顛覆厥序,俾我大齊之祚,翦焉將墜。



 高宗明皇帝建道德之盛軌,垂仁義之至蹤,紹二祖之鴻基,繼三五之絕業。昧旦丕顯,不明求
 衣,故奇士盈朝,異人輻湊。若乃經禮緯樂之文,定鼎作洛之制,非雲如醴之詳,白質黑章之瑞,諒以則天比大,無德稱焉。



 而嗣主不綱,窮肆陵暴,十愆畢行,三風咸襲。喪初而無哀貌,在戚而有喜容。



 酣酒嗜音,罔懲其侮;讒賊狂邪,是與比周。遂令親賢嬰荼毒之誅,宰輔受菹醢之戮。江僕射,蕭、劉領軍,徐司空,沈僕射,曹右衛,或外戚懿親,或皇室令德,或時宗民望,或國之虎臣,並勛彰中興,功比周、邵,秉鈞贊契,受遺先朝。咸以名重見疑,正直貽斃,害加黨族,虐及嬰孺。曾無《渭陽》追遠之情,不顧本枝殲落之痛。信必見疑,忠而獲罪,百姓業業,罔知攸暨。崔慧景內逼淫刑,外不堪命,驅土崩之民,為免死之計,倒戈回刃,還指宮闕。城無完守,人有異圖。賴蕭令君勳濟宗祏,業拯蒼氓,四海蒙一匡之德,億兆憑再造之功。江夏王拘迫威強,牽制巨力,迹屈當時,乃心可亮,竟不能內恕探情,顯
 加鴆毒。蕭令君自以親惟族長,任實宗臣,至誡苦言,朝夕獻入,讒丑交構,漸見疏疑,浸潤成災,奄離怨酷。用人之功,以寧社稷,刈人之身,以騁淫濫。



 台輔既誅,姦小競用。梅蟲兒、茹法珍妖忍愚戾,窮縱醜惡,販鬻主威,以為家勢,營惑嗣主,恣其妖虐。宮女千餘,裸服宣淫,孽臣數十,袒裼相逐。帳飲闤肆之間,宵遊街陌之上,提挈群豎,以為歡笑。



 劉山陽潛受兇旨,規肆狂逆,天誘其衷,即就梟翦。



 夫天生蒸民,樹之以君,使司牧之,勿使失性。豈有尊臨宇縣,毒遍黔首,絕親戚之恩,無君臣之義,功重者先誅,勛高者速斃。九族內離,四夷外叛,封境日蹙,戎馬交馳,帑藏既空,百姓已竭,不恤不憂,慢遊是好。民怨於下,天懲於上,故熒惑襲月,孽火燒宮,妖水表災,震蝕告沴。七廟阽危,三才莫紀,大懼我四海之命,永淪於地。



 南康殿下體自高宗,天挺英懿。食葉之徵,著於弱年,當璧之祥,兆乎綺歲。



 億
 兆顒顒,咸思戴奉。且勢居上游,任總連帥,家國之否,寧濟是當。莫府身備皇宗,忝荷顧託,憂深責重,誓清時難。今命冠軍將軍、西中郎諮議、領中直兵參軍、軍主楊公則,寧朔將軍、領中兵參軍、軍主王法度,冠軍將軍、諮議參軍、軍主龐颻,輔國將軍、諮議參軍、領別駕、軍主宗夬,輔國將軍、諮議參軍、軍主樂藹等,領勁卒三萬,陵波電邁,逕造秣陵。冠軍將軍、領諮議、中直兵參軍、軍主蔡道恭,輔國將軍、中直兵參軍、右軍府司馬、軍主席闡文,輔國將軍、中直兵參軍、軍主任漾之,寧朔將軍、中直兵參軍、軍主韓孝仁,寧朔將軍、中直兵參軍、軍主朱斌,中直兵參軍、軍主宗冰之,建威將軍、中直兵參軍、軍主朱景舒,寧朔將軍、中直兵參軍、軍主庾域,寧遠將軍、軍主庾略等,被甲二萬,直指建業。輔國將軍、武寧太守、軍主鄧元起,輔國將軍、前軍將軍、軍主王世興等,鐵騎一萬,分趨白下。



 征虜將
 軍、領司馬、新興太守夏侯詳,寧朔將軍、咨議參軍、軍主柳忱,寧朔將軍、領中兵參軍、軍主劉孝慶,建威將軍、軍主、江陵令江詮等,帥組甲五萬,駱驛繼發。



 雄劍高麾,則五星從流;長戟遠指,則雲虹變色。天地為之矞皇,山淵以之崩沸。莫府親貫甲胄,授律中權,董帥熊羆之士十有五萬,征鼓粉沓,雷動荊南。寧朔將軍、南康王友蕭穎達領虎旅三萬,抗威後拒。蕭雍州勳業蓋世,謀猷淵肅,既痛家禍,兼憤國難,泣血枕戈,誓雪怨酷,精卒十萬,已出漢川。張郢州節義慷慨,悉力齊奮。江州邵陵王、湘州張行事、王司州皆遠近懸契,不謀而同,並勒驍猛,指景風驅。舟艦魚麗,萬里蓋水,車騎雲屯,平原霧塞。以同心之士,伐倒戈之眾,盛德之師,救危亡之國,何征而不服,何誅而不克哉!



 今兵之所指,唯在梅蟲兒、茹法珍二人而已。諸君德載累世,勛著先朝,屬無妄之時,居道消之運,受迫群豎,
 念有危懼。大軍近次,當各思拔跡,來赴軍門。



 檄到之日,有能斬送蟲兒、法珍首者,封二千戶開國縣侯。若迷惑凶黨,敢拒軍鋒,刑茲無赦,戮及宗族。賞罰之信,有如曒日,江水在此,餘不食言。



 遣冠軍將軍楊公則向湘州。王法度不進軍,免官。公則進剋巴陵,仍向湘州。



 遣寧朔將軍劉坦行湘州事。



 穎胄遣人謂梁王曰:「時月未利,當須來年二月。今便進兵,恐非良策。」梁王曰:「今坐甲十萬,糧用自竭。況藉以義心,一時驍銳。且太白出西方,仗義而動,天時人謀,無有不利。昔武王伐紂,行逆太歲,豈復待年月邪?」穎胄乃從。



 遣西中郎參軍鄧元起率眾向夏口。



 三年正月,和帝為相國,穎胄領左長史,進號鎮軍將軍。於是始選用方伯。梁王屢表勸和帝即尊號,梁州刺史柳惔、竟陵太守曹景宗並勸進。穎胄使別駕宗寔撰定禮儀,上尊號,改元,於江陵立宗廟、南北郊,州府城門悉依建康宮,置尚
 書五省,以城南射堂為蘭臺,南郡太守為尹。建武中,荊州大風雨,龍入柏齋中,柱壁上有爪足處,刺史蕭遙欣恐畏,不敢居之。至是以為嘉祐殿。中興元年三月,穎胄為侍中、尚書令,假節、都督如故。尋領吏部尚書,監八州軍事,行荊州刺史,本官如故。左丞樂藹奏曰:「敕旨以軍旅務殷,且停朝直。竊謂匪懈于位,義昭夙興,國容舊典,不可頓闕。與兼右丞江詮等參議,八座丞郎以下宜五日一朝,有事郎坐侍下鼓,無事許從實還外。」奏可。



 梁王義師出沔口,郢州刺史張沖據城拒守。楊公則定湘州,行事張寶積送江陵,率軍會夏口。巴西太守魯休烈、巴東太守蕭惠訓遣子璝拒義師。穎胄遣汶陽太守劉孝慶進峽口,與巴東太守任漾之、宜都太守鄭法紹御之。時軍旅之際,人情未安,穎胄府長史張熾從絳衫左右三十餘人入千秋門,城內驚恐,疑有同異。御史中丞奏彈熾,詔以贖論。



 穎胄弟穎孚在京師,廬陵人脩靈祐竊將南上,於西昌縣山中聚兵二千人,襲郡,內史謝篹奔豫章。穎孚、靈祐據郡求援,穎胄遣寧朔將軍范僧簡入湘州南道援之。



 僧簡進剋安成,仍以為輔國將軍、安成內史。拜穎孚為冠軍將軍、廬陵內史。合二郡兵,出彭蠡口。



 東昏侯遣軍主彭盆、劉希祖三千人受江州刺史陳伯之節度,南討二郡義兵,仍進取湘州。南康太守王丹保郡應盆等。穎孚聞兵至,望風奔走。前內史謝篹復還郡。



 劉希祖至安成,攻戰七日,城陷,範僧簡見殺。希祖仍為安成內史。穎孚收散卒據西昌,謝篹又遣軍攻之,眾敗,奔湘州。以穎孚為督湘東衡陽零陵桂陽營陽五郡、湘東內史、假節、將軍如故。尋病卒。後脩靈祐又合餘眾攻篹,篹復敗走豫章,劉希祖亦以郡降。



 湘東內史王僧粲亦拒義,自稱平西將軍、湘州刺史,以南平鎮軍主周敷為長史,率前軍襲湘州,去
 州百餘里。楊公則長史劉坦守州城,遣軍主尹法略拒之,屢戰不勝。及聞建康城平,僧粲散走,乃斬之。南康太守王丹亦為郡人所殺。



 郢城降,義師眾軍東下。八月,魯休烈、蕭璝破汶陽太守劉孝慶等於峽口,巴東太守任漾之見殺,遂至上明,江陵大震。穎胄恐,馳告梁王曰:「劉孝慶為蕭璝所敗,宜遣楊公則還援根本。」梁王曰:「公則今溯流上荊,鞭長之義耳。蕭璝、魯休烈烏合之眾,尋自退散。政須荊州少時持重。良須兵力,兩弟在雍,指遣往征,不為難至。」穎胄乃追贈任漾之輔國將軍、梁州刺史。遣軍主蔡道恭假節屯上明拒蕭璝。



 時梁王已平郢、江二鎮。穎胄輔帝出居上流,有安重之勢。素能飲酒,啖白肉鱠至三升,既聞蕭璝等兵相持不決,憂慮感氣,十二月壬寅夜,卒。遺表曰:「臣疹患數日,不謂便至困篤,氣息綿微,待盡而已。臣雖庸薄,忝籍葭莩,過受先朝殊常之眷,循寵礪心,誓生
 以死。屬皇業中否,天地分崩,總率諸侯,翼奉明聖。



 賴社稷靈長,大明在運,故兵之所臨,無思不服。今四海垂平,干戈行戢,方希陪翠華,奉法駕,反東都,觀舊物。不幸遘疾,奄辭明世,懷此深恨,永結泉壤。竊惟王業至重,萬機甚大,登之實難,守之未易。陛下富於春秋,當遠尋祖宗創業艱難,殷鑒季末顛覆厥緒,思所以念始圖終,康此兆庶。征東大將軍臣衍,元勛上德,光贊天下,陛下垂拱仰成,則風流日化,臣雖萬沒,無所遺恨。」時年四十。和帝出臨哭。詔贈侍中、丞相,本官如故。前後部羽葆鼓吹,班劍三十人。轀輬車,黃屋左纛。



 梁王圍建康城,住在石頭,和帝密詔報穎胄凶問,秘不發喪。及城平,識者聞之,知天命之有在矣。



 梁天監元年,詔曰:「念功惟德,歷代所同,追遠懷人,彌與事篤。齊故侍中、丞相、尚書令穎胄,風格峻遠,器宇淵邵,清猷盛業,問望斯歸。締構義始,肇基王跡,契闊屯夷,載
 形心事。朕膺天改命,光宅區宇,望岱瞻河,永言增慟。可封巴東郡公,邑三千戶,本官如故。」喪還,今上車駕臨哭渚次。詔曰:「齊故侍中、丞相、尚書令穎胄葬送有期,前代所加殊禮,依晉王導、齊豫章王故事,可悉給。



 謚曰獻武。」範僧簡贈交州刺史。



 史臣曰:魏氏基於用武,夏侯諸曹,並以戚族而為將相。夫股肱為義,既有常然,肺腑之重,兼存宗寄。豐沛之間,貴人滿市,功臣所出,多在南陽。夫貞幹所以成務,非虛言也。



 贊曰:新吳事武,簡在帝心。南豐治政,迹顯亡衾。鎮軍茂績,機識弘深,荊南立王,向義漢陰。



\end{pinyinscope}