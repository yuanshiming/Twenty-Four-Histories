\article{卷三十三列傳第十四 王僧虔 張緒}

\begin{pinyinscope}

 王僧虔,琅邪臨沂人也。祖珣,晉司徒。伯父太保弘,宋元嘉世為宰輔。賓客疑所諱,弘曰:「身家諱與蘇子高同。」父曇首,右光祿大夫。曇首兄弟集會諸子孫,弘子僧達下地跳戲,僧虔年數歲,獨正坐採蠟燭珠為鳳凰。弘曰:「此兒終當為長者。」僧虔弱冠,弘厚,善隸書。宋文帝見其書素扇,歎曰:「非唯跡逾子敬,方當器雅過之。」除秘書郎,太子舍人。退默少交接,與袁淑、謝莊善。轉義陽王文學,太子洗馬,遷司徒左西屬。



 兄僧綽,為太初所害,親賓咸勸僧虔逃。僧虔涕泣曰:「吾兄奉國以忠貞,撫我以慈愛,今日之事,苦不見及耳。若同歸
 九泉,猶羽化也。」孝武初,出為武陵太守。兄子儉於中途得病,僧虔為廢寢食。同行客慰喻之。僧虔曰:「昔馬援處兒侄之間一情不異,鄧攸於弟子更逾所生,吾實懷其心,誠未異古。亡兄之胤,不宜忽諸。若此兒不救,便當回舟謝職,無復游宦之興矣。」還為中書郎,轉黃門郎,太子中庶子。



 孝武欲擅書名,僧虔不敢顯跡。大明世,常用拙筆書,以此見容。出為豫章王子尚撫軍長史,遷散騎常侍,復為新安王子鸞北中郎長史、南東海太守,行南徐州事,二蕃皆帝愛子也。尋遷豫章內史。入為侍中,遷御史中丞,領驍騎將軍。甲族由來多不居憲臺,王氏以分枝居烏衣者,位官微減,僧虔為此官,乃曰:「此是烏衣諸郎坐處,我亦可試為耳。」復為侍中,領屯騎校尉。泰始中,出為輔國將軍、吳興太守,秩中二千石。王獻之善書,為吳興郡,及僧虔工書,又為郡,論者稱之。



 徙為會稽太守,秩中二
 千石,將軍如故。中書舍人阮佃夫家在會稽,請假東歸。



 客勸僧虔以佃夫要倖,宜加禮接。僧虔曰:「我立身有素,豈能曲意此輩。彼若見惡,當拂衣去耳。」佃夫言於宋明帝,使御史中丞孫夐奏:「僧虔前蒞吳興,多有謬命,檢到郡至遷,凡用功曹五官主簿至二禮吏署三傳及度與弟子,合四百四十八人。又聽民何係先等一百十家為舊門。委州檢削。」坐免官。尋以白衣兼侍中,出監吳郡太守,遷使持節、都督湘州諸軍事、建武將軍、行湘州事,仍轉輔國將軍,湘州刺史。所在以寬惠著稱。巴峽流民多在湘土,僧虔表割益陽、羅、湘西三縣緣江民立湘陰縣,從之。



 元徽中,遷吏部尚書。高平檀珪罷沅南令,僧虔以為征北板行參軍,訴僧虔求祿不得,與僧虔書曰:「五常之始,文武為先,文則經緯天地,武則撥亂定國。僕一門雖謝文通,乃忝武達。群從姑叔,三媾帝室,祖兄二世,糜軀奉國,而致子侄餓死草
 壤。去冬今春,頻荷二敕,既無中人,屢見嗟奪。經涉五朔,踰歷四晦,書牘十二,接覲六七,遂不荷潤,反更曝鰓。九流繩平,自不宜獨苦一物,蟬腹龜腸,為日已久。饑虎能嚇,人遽與肉;餓麟不噬,誰為落毛?去冬乞豫章丞,為馬超所爭;今春蒙敕南昌縣,為史偃所奪。二子勛蔭人才,有何見勝?若以貧富相奪,則分受不如。身雖孤微,百世國士,姻媾位宦,亦不後物。尚書同堂姊為江夏王妃,檀珪同堂姑為南譙王妃;尚書婦是江夏王女,檀珪祖姑嬪長沙景王;尚書伯為江州,檀珪祖亦為江州;尚書從兄出身為後軍參軍,檀珪父釋褐亦為中軍參軍。僕於尚書,人地本懸,至於婚宦,不至殊絕。今通塞雖異,猶忝氣類,尚書何事乃爾見苦?泰始之初,八表同逆,一門二世,粉骨衛主,殊勛異績,已不能甄,常階舊途,復見侵抑。」僧虔報書
 曰:「征北板比歲處遇小優,殷主簿從此府入崇禮,何儀曹即代殷,亦不見訴為苦。足下積屈,一朝超升,政自小難。泰始初勤苦十年,自未見其賞,而頓就求稱,亦何可遂。吾與足下素無怨憾,何以相侵苦,直是意有佐佑耳。」



 珪又書曰:「昔荀公達漢之功臣,晉武帝方爵其玄孫;夏侯惇魏氏勳佐,金德初融,亦始就甄顯,方賞其孫,封樹近族。羊叔子以晉泰始中建策伐吳,至咸寧末,方加褒寵,封其兄子;卞望之以咸和初殞身國難,至興寧末,方崇禮秩,官其子孫;蜀郡主簿田混,黃初末死故君之難,咸康中方擢其子孫。似不以世代遠而被棄,年世疏而見遺。檀珪百罹六極,造化罕比,五喪停露,百口轉命,存亡披迫,本希小祿,無意階榮。自古以來有沐食侯,近代有王官。府佐非沐食之職,參軍非王官之謂。



 質非匏瓜,實羞空懸。殷、何二生,或是府主情味,或是朝廷意旨,豈與悠悠之人同口而語!使僕就此職,尚書能以郎見轉不?若使日得五升祿,則不
 恥執鞭。」僧虔乃用為安城郡丞。珪,宋安南將軍韶孫也。



 僧虔尋加散騎常侍,轉右僕射。升明元年,遷尚書僕射,尋轉中書令,左僕射。



 二年,為尚書令。僧虔好文史,解音律,以朝廷禮樂多違正典,民間競造新聲雜曲,時太祖輔政,僧虔上表曰:「夫懸鐘之器,以雅為用;凱容之禮,八佾為儀。今總章羽佾,音服舛異。又歌鐘一肆,克諧女樂,以歌為務,非雅器也。大明中,即以宮懸合和《鞞》、《拂》,節數雖會,慮乖《雅》體,將來知音,或譏聖世。若謂鐘舞已諧,重違成憲,更立歌鐘,不參舊例。四縣所奏,謹依《雅》條,即義沿理,如或可附。又今之《清商》,實由銅爵,三祖風流,遺音盈耳,京洛相高,江左彌貴。諒以金石干羽,事絕私室,桑濮鄭衛,訓隔紳冕,中庸和雅,莫復於斯。而情變聽移,稍復銷落,十數年間,亡者將半。自頃家競新哇,人尚謠俗,務在噍殺,不顧音紀,流宕無崖,未知所極,排斥正曲,崇長煩淫。士
 有等差,無故不可去樂,禮有攸序,長幼不可共聞。故喧醜之制,日盛於廛里;風味之響,獨盡於衣冠。宜命有司,務勤功課,緝理遺逸,迭相開曉,所經漏忘,悉加補綴。曲全者祿厚,藝妙者位優。利以動之,則人思刻厲。反本還源,庶可跂踵。」事見納。



 建元元年,轉侍中,撫軍將軍,丹陽尹。二年,進號左衛將軍,固讓不拜。改授左光祿大夫,侍中、尹如故。郡縣獄相承有上湯殺囚,僧虔上疏言之曰:「湯本以救疾,而實行冤暴,或以肆忿。若罪必入重,自有正刑;若去惡宜疾,則應先啟。



 豈有死生大命,而潛制下邑。愚謂治下囚病,必先刺郡,求職司與醫對共診驗;遠縣,家人省視,然後處理。可使死者不恨,生者無怨。」上納其言。



 僧虔留意雅樂,昇明中所奏,雖微有釐改,尚多遺失。是時上始欲通使,僧虔與兄子儉書曰:「古語云『中國失禮,問之四夷』。計樂亦如。苻堅敗後,東晉始備金石樂,故知不可全誣也。北國
 或有遺樂,誠未可便以補中夏之闕,且得知其存亡,亦一理也。但《鼓吹》舊有二十一曲,今所能者十一而已,意謂北使會有散役,得今樂署一人粗別同異者,充此使限。雖復延州難追,其得知所知,亦當不同。若謂有此理者,可得申吾意上聞否?試為思之。」事竟不行。



 太祖善書,及即位,篤好不已。與僧虔賭書畢,謂僧虔曰:「誰為第一?」僧虔曰:「臣書第一,陛下亦第一。」上笑曰:「卿可謂善自為謀矣。」示僧虔古迹十一帙,就求能書人名。僧虔得民間所有帙中所無者——吳大皇帝、景帝、歸命侯書,桓玄書,及王丞相導、領軍洽、中書令氏、張芝、索靖、衛伯儒、張翼十二卷奏之。又上羊欣所撰《能書人名》一卷。



 其年冬,遷持節、都督湘州諸軍事、征南將軍、湘州刺史,侍中如故。清簡無所欲,不營財產,百姓安之。世祖即位,僧虔以風疾欲陳解,會遷侍中、左光祿大夫、開府儀同三司。僧虔少時群從
 宗族並會,客有相之者云:「僧虔年位最高,仕當至公,餘人莫及也。」及授,僧虔謂兄子儉曰:「汝任重於朝,行當有八命之禮,我若復此授,則一門有二台司,實可畏懼。」乃固辭不拜,上優而許之。改授侍中、特進、左光祿大夫。客問僧虔固讓之意,僧虔曰:「君子所憂無德,不憂無寵。吾衣食周身,榮位已過,所慚庸薄無以報國,豈容更受高爵,方貽官謗邪!」兄子儉為朝宰,起長梁齋,制度小過,僧虔視之不悅,竟不入戶,儉即毀之。



 永明三年,薨。僧虔頗解星文,夜坐見豫章分野當有事故,時僧虔子慈為豫章內史,慮其有公事。少時,僧虔薨,慈棄郡奔赴。僧虔時年六十。追贈司空,侍中如故。謚簡穆。



 其論書曰:「宋文帝書,自云可比王子敬,時議者云『天然勝羊欣,功夫少於欣』。王平南廙,右軍叔,過江之前以為最。亡曾祖領軍書,右軍云『弟書遂不減吾』。變古制,今唯右軍、領軍;不爾,至今猶法鐘、張。亡從祖中書令
 書,子敬云『弟書如騎騾,駸駸恆欲度驊騮前』。庾征西翼書,少時與右軍齊名,右軍後進,庾猶不分,在荊州與都下人書云『小兒輩賤家雞,皆學逸少書,須吾下,當比之』。



 張翼,王右軍自書表,晉穆帝令翼寫題後答,右軍當時不別,久後方悟,云『小人幾欲亂真』。張芝、索靖、韋誕、鐘會、二衛並得名前代,無以辨其優劣,唯見其筆力驚異耳。張澄當時亦呼有意。郗愔章草亞於右軍。郗嘉賓草亞於二王,緊媚過其父。桓玄自謂右軍之流,論者以比孔琳之。謝安亦入能書錄,亦自重,為子敬書嵇康詩。羊欣書見重一時,親受子敬,行書尤善,正乃不稱名。孔琳之書天然放縱,極有筆力,規矩恐在羊欣後。丘道護與羊欣俱面受子敬,故當在欣後。范曄與蕭思話同師羊欣,後小叛,既失故步,為復小有意耳。蕭思話書,羊欣之影,風流趣好,殆當不減,筆力恨弱。謝綜書,其舅雲緊生起,是得賞也,恨少媚
 好。謝靈運乃不倫,遇其合時,亦得入流。賀道力書亞丘道護。庾昕學右軍,亦欲亂真矣。」又著《書賦》傳於世。



 第九子寂,字子玄,性迅動,好文章,讀《範滂傳》,未常不嘆挹。王融敗後,賓客多歸之。建武初,欲獻《中興頌》,兄志謂之曰:「汝膏梁年少,何患不達?



 不鎮之以靜,將恐貽譏。」寂乃止。初為秘書郎,卒,年二十一。



 僧虔宋世嘗有書誡子曰:知汝恨吾不許汝學,欲自悔厲,或以闔棺自欺,或更擇美業,且得有慨,亦慰窮生。但亟聞斯唱,未睹其實。請從先師聽言觀行,冀此不復虛身。吾未信汝,非徒然也。往年有意於史,取《三國志》聚置床頭,百日許,復從業就玄,自當小差於史,猶未近徬佛。曼倩有云:「談何容易。」見諸玄,志為之逸,腸為之抽,專一書,轉誦數十家注,自少至老,手不釋卷,尚未敢輕言。汝開《老子》卷頭五尺許,未知輔嗣何所道,平叔何所說,馬、鄭何所異,《指例》何所明,而便盛於麈尾,自呼談士,
 此最險事。設令袁令命汝言《易》,謝中書挑汝言《莊》,張吳興叩汝言《老》,端可復言未嘗看邪?談故如射,前人得破,後人應解,不解即輸賭矣。且論注百氏,荊州《八帙》,又《才性四本》、《聲無哀樂》,皆言家口實,如客至之有設也。汝皆未經拂耳瞥目,豈有庖廚不脩,而欲延大賓者哉?就如張衡思侔造化,郭象言類懸河,不自勞苦,何由至此?汝曾未窺其題目,未辨其指歸——六十四卦,未知何名;《莊子》眾篇,何者內外;《八帙》所載,凡有幾家;《四本》之稱,以何為長——而終日欺人,人亦不受汝欺也。由吾不學,無以為訓。



 然重華無嚴父,放勛無令子,亦各由己耳。汝輩竊議亦當云:「阿越不學,在天地間可嬉戲,何忽自課謫?幸及盛時逐歲暮,何必有所減?」汝見其一耳,不全爾也。



 設令吾學如馬、鄭,亦必甚勝;復倍不如今,亦必大減。致之有由,從身上來也。



 汝今壯年,自勤數倍許勝,劣及吾耳。世中比例舉眼是,汝足知此,不
 復具言。



 吾在世,雖乏德素,要復推排人間數十許年,故是一舊物,人或以比數汝等耳。



 即化之後,若自無調度,誰復知汝事者?舍中亦有少負令譽弱冠越超清級者,於時王家門中,優者則龍鳳,劣者猶虎豹,失蔭之後,豈龍虎之議?況吾不能為汝蔭,政應各自努力耳。或有身經三公,蔑爾無聞;布衣寒素,卿相屈體。或父子貴賤殊,兄弟聲名異。何也?體盡讀數百卷書耳。吾今悔無所及,欲以前車誡爾後乘也。汝年入立境,方應從官,兼有室累,牽役情性,何處復得下帷如王郎時邪?為可作世中學,取過一生耳。試復三思,勿諱吾言。猶捶撻志輩,冀脫萬一,未死之間,望有成就者,不知當有益否?各在爾身己切,豈復關吾邪?鬼唯知愛深松茂柏,寧知子弟毀譽事!因汝有感,故略敘胸懷。



 張緒,字思曼,吳郡吳人也。祖茂度,會稽太守。父寅,太子中舍人。緒
 少知名,清簡寡欲,叔父鏡謂人曰:「此兒,今之樂廣也。」州辟議曹從事,舉秀才。



 建平王護軍主簿,右軍法曹行參軍,司空主簿,撫軍、南中郎二府功曹,尚書倉部郎。都令史諮郡縣米事,緒蕭然直視,不以經懷。除巴陵王文學,太子洗馬,北中郎參軍,太子中舍人,本郡中正,車騎從事中郎,中書郎,州治中,黃門郎。



 宋明帝每見緒,輒嘆其清淡。轉太子中庶子,本州大中正,遷司徒左長史。吏部尚書袁粲言於帝曰:「臣觀張緒有正始遺風,宜為宮職。」復轉中庶子,領翊軍校尉,轉散騎常侍,領長水校尉,尋兼侍中,遷吏部郎,參掌大選。元徽初,東宮罷,選曹擬舍人王儉格外記室,緒以儉人地兼美,宜轉秘書丞,從之。緒又遷侍中,郎如故。



 緒忘情榮祿,朝野皆貴其風。嘗與客閑言,一生不解作諾。時袁粲、褚淵秉政,有人以緒言告粲、淵者,即出緒為吳郡太守,緒初不知也。遷為祠部尚書,復領中
 正,遷太常,加散騎常侍,尋領始安王師。昇明二年,遷太祖太傅長史,加征虜將軍。齊臺建,轉散騎常侍,世子詹事。建元元年,轉中書令,常侍如故。



 緒善言,素望甚重,太祖深加敬異。僕射王儉謂人曰:「北士中覓張緒,過江未有人,不知陳仲弓、黃叔度能過之不耳?」車駕幸莊嚴寺聽僧達道人講,座遠,不聞緒言,上難移緒,乃遷僧達以近之。尋加驍騎將軍。欲用緒為右僕射,以問王儉,儉曰:「南士由來少居此職。」褚淵在座,啟上曰:「儉年少,或不盡憶。江左用陸玩、顧和,皆南人也。」儉曰:「晉氏衰政,不可以為准則。」上乃止。四年,初立國學,以緒為太常卿,領國子祭酒,常侍、中正如故。緒既遷官,上以王延之代緒為中書令,時人以此選為得人,比晉朝之用王子敬、王季琰也。



 緒長於《周易》,言精理奧,見宗一時。常云何平叔所不解《易》中七事,諸卦中所有時義,是其一也。



 世祖即位,轉吏部尚書,祭酒
 如故。永明元年,遷金紫光祿大夫,領太常。明年,領南郡王師,加給事中,太常如故。三年,轉太子詹事,師、給事如故。緒每朝見,世祖目送之。謂王儉曰:「緒以位尊我,我以德貴緒也。」遷散騎常侍,金紫光祿大夫、師如故。給親信二十人。復領中正。長沙王晃屬選用吳興聞人邕為州議曹,緒以資籍不當,執不許。晃遣書佐固請之,緒正色謂晃信曰:「此是身家州鄉,殿下何得見逼!」七年,竟陵王子良領國子祭酒,世祖敕王晏曰:「吾欲令司徒辭祭酒以授張緒,物議以為雲何?」子良竟不拜,以緒領國子祭酒,光祿、師、中正如故。



 緒口不言利,有財輒散之。清言端坐,或竟日無食。門生見緒饑,為之辨餐,然未嘗求也。卒時年六十八。遺命作蘆葭轜車,靈上置杯水香火,不設祭。從弟融敬重緒,事之如親兄,齎酒於緒靈前酌飲,慟哭曰:「阿兄風流頓盡!」追贈散騎常侍、特進、金紫光祿大夫。謚簡子。



 子
 克,蒼梧世正員郎,險行見寵,坐廢錮。



 克弟允,永明中安西功曹,淫通殺人,伏法。



 允兄充,永明元年為武陵王友,坐書與尚書令王儉,辭旨激揚,為御史中丞到捴所奏,免官禁錮。論者以為有恨於儉也。



 案建元初,中詔序朝臣,欲以右僕射擬張岱。褚淵謂「得此過優,若別有忠誠,特進升引者,別是一理,仰由裁照。」詔「更量」。說者既異,今兩記焉。



 史臣曰:王僧虔有希聲之量,兼以藝業。戒盈守滿,屈己自容,方軌諸公,實平世之良相。張緒凝衿素氣,自然標格,搢紳端委,朝宗民望。夫如緒之風流者,豈不謂之名臣!



 贊曰:簡穆長者,其義恢恢;聲律草隸,燮理三臺。思曼廉靜,自絕風
 埃;游心爻系,物允清才。



\end{pinyinscope}