\article{卷四十九列傳第三十 王奐(從弟繢) 張沖}

\begin{pinyinscope}

 王奐,字彥孫,瑯邪臨沂人也。祖僧朗,宋左光祿、儀同。父粹,黃門郎。奐出繼從祖中書令球,故字彥孫。解褐著作佐郎,太子舍人,安陸王冠軍主簿,太子洗馬,本州別駕,中書郎,桂陽王司空諮議,黃門郎。元徽元年為晉熙王征虜長史、江夏內史,遷侍中,領步兵校尉。復出為晉熙王鎮西長史,加冠軍將軍、江夏武昌太守。征祠部尚書,轉掌吏部。



 升明初,遷冠軍將軍、丹陽尹。



 初,王晏父普曜為沈攸之長史,常慮攸之舉事,不得還。時奐為吏部,轉普曜為內職,晏深德之。及晏仕世祖府,奐從弟蘊反,世祖謂晏曰:「王奐宋家外戚,王
 蘊親同逆黨,既其群從,豈能無異意。我欲具以啟聞。」晏叩頭曰:「王奐脩謹,保無異志。晏父母在都,請以為質。」世祖乃止。



 出為吳興太守,秩中二千石,將軍如故。尋進號征虜將軍。建元元年,進號左將軍。明年,遷太常,領鄱陽王師,仍轉侍中,秘書監,領驍騎將軍。又遷征虜將軍、臨川王鎮西長史、領南蠻校尉、南郡內史。奐一歲三遷,上表固讓南蠻曰:「今天地初闢,萬物載新,刑蠻來威,巴濮不擾。但使邊民樂業,有司脩務,本府舊州,日就殷阜。臣昔遊西土,較見盈虛,兼日者戎燼之後,痍毀難復。雖復緝以善政,未及來蘇。今復割撤大府,制置偏校,崇望不足以助強,語實安能以相弊?



 且資力既分,職司增廣,眾勞務倍,文案滋煩。非獨臣見其難,竊以為國計非允。」



 見許。於是罷南蠻校尉官。進號前將軍。



 世祖即位,徵右僕射。仍轉使持節、監湘州軍事、前將軍、湘州刺史。永明二年,徙為散騎常侍、
 江州刺史。初省江州軍府。四年,遷右僕射,本州中正。奐無學術,以事幹見處。遷尚書僕射,中正如故。校籍郎王植屬吏部郎孔琇之以校籍令史俞公喜求進署,矯稱奐意,植坐免官。



 六年,遷散騎常侍,領軍將軍。奐欲請車駕幸府。上晚信佛法,御膳不宰牲。



 使王晏謂奐曰:「吾前去年為斷殺事,不復幸詣大臣已判,無容欻爾也。」王儉卒,上用奐為尚書令,以問王晏。晏位遇已重,與奐不能相推,答上曰:「柳世隆有重望,恐不宜在奐後。」乃轉為左僕射,加給事中,出為使持節、散騎常侍、都督雍梁南北秦四州郢州之竟陵司州之隨郡軍事、鎮北將軍、雍州刺史。上謂王晏曰:「奐於釋氏,實自專至。其在鎮或以此妨務,卿相見言次及之,勿道吾意也。」上以行北諸戍士卒多襤縷,送褲褶三千具,令奐分賦之。



 十一年,奐輒殺寧蠻長史劉興祖,上大怒,使御史中丞孔稚珪奏其事曰:雍州刺史王
 奐啟錄小府長史劉興祖,虛稱「興祖扇動山蠻,規生逆謀,誑言誹謗,言辭不遜」。敕使送興祖下都,奐慮所啟欺妄,於獄打殺興祖,詐啟稱自經死。



 止今體傷楗蒼,事暴聞聽。



 攝興祖門生劉倪到臺辨問,列「興祖與奐共事,不能相和。自去年朱公恩領軍征蠻失利,興祖啟聞,以啟呈奐,奐因此便相嫌恨。若云興祖有罪,便應事在民間;民間恬然,都無事跡。去十年九月十八日,奐使仗身三十人來,稱敕錄興祖付獄。



 安定郡蠻先在郡贓私,興祖既知其取與,即牒啟,奐不問。興祖後執錄,奐仍令蠻領仗身於獄守視。興祖未死之前,於獄以物畫漆柈子中出密報家,道無罪,令啟乞出都一辨,萬死無恨。」又云:「奐駐興祖嚴禁信使,欲作方便,殺以除口舌。」



 又云:「奐意乃可。奐第三息彪隨奐在州,凡事是非皆干豫,扇構密除興祖。」又云:「興祖家餉糜,中下藥,食兩口便覺,回乞獄子,食者皆大利。
 興祖大叫道:『糜中有藥!』近獄之家,無人不聞。」又云:「奐治著興祖日急,判無濟理。十一月二十一日,奐使獄吏來報興祖家,道興祖於獄自經死。尸出,家人共洗浴之,見興祖頸下有傷,肩胛烏,陰下破碎,實非興祖自經死。家人及門義共見,非是一人。」重攝檢雍州都留田文喜,列與倪符同狀。



 興祖在獄,嗛苦望下,既蒙降旨,欣願始遂,豈容於此,方復自經?敕以十九日至,興祖以二十一日死,推理檢跡,灼然矯假。尋敕使送下,奐輒拒詔,所謗諸條,悉出奐意。毀故丞相若陳顯達,誹訕朝事,莫此之深。彪私隨父之鎮,敢亂王法,罪並合窮戮。



 上遣中書舍人呂文顯、直閣將軍曹道剛領齋仗五百人收奐。敕鎮西司馬曹虎從江陵步道會襄陽。



 奐子彪素凶剽,奐不能制。女婿殷叡懼禍,謂奐曰:「曹、呂今來,既不見真敕,恐為奸變,政宜錄取,馳啟聞耳。」奐納之。彪輒令率州內得千餘人,開鎮庫,
 取仗,配衣甲,出南堂陳兵,閉門拒守。奐門生鄭羽叩頭啟奐,乞出城迎臺使。奐曰:「我不作賊,欲先遣啟自申。政恐曹、呂輩小人相陵藉,故且閉門自守耳。」



 彪遂出與虎軍戰,其黨范虎領二百人降臺軍,彪敗走歸。土人起義攻州西門,彪登門拒戰,卻之。奐司馬黃瑤起、寧蠻長史裴叔業於城內起兵攻奐。奐聞兵入,還內禮佛,未及起,軍人遂斬之。年五十九,執彪及弟爽、弼、殷叡,皆伏誅。



 詔曰:「逆賊王奐,險詖之性,自少及長。外飾廉勤,內懷兇慝,貽戾鄉伍,取棄衣冠。拔其文筆之用,擢以顯任,出牧樊阿,政刑弛亂。第三息彪矯弄威權,父子均勢。故寧蠻長史劉興祖忠於奉國,每事匡執,奐忿其異己,誣以訕謗,肆怒囚錄,然後奏聞。朕察奐愚詐,詔送興祖還都,乃懼奸謀發露,潛加殺害。欺罔既彰,中使辯核,遂授兵登陴,逆捍王命。天威電掃,義夫咸奮,曾未浹辰,罪人斯獲,方隅克殄,漢南肅清。
 自非犯官兼預同逆謀,為一時所驅逼者,悉無所問。」



 奐長子太子中庶子融,融弟司徒從事中郎琛,於都棄市。餘孫皆原宥。



 殷睿字文子,陳郡人,晉太常融七世孫也。宋元嘉末,祖元素坐染太初事誅。



 睿遺腹亦當從戮,外曾祖王僧朗啟孝武救之,得免。睿解文義,有口才,司徒褚淵甚重之,謂之曰:「諸殷自荊州以來,無出卿右者。」睿斂容答曰:「殷族衰悴,誠不如昔,若此旨為虛,故不足降;此旨為實,彌不可聞。」奐為雍州,啟睿為府長史。



 睿族父恒,字昭度,與睿同承融後。宋司空景仁孫也。恒及父道矜,並有古風,以是見蚩於世,其事非一。恆,宋泰始初為度支尚書,坐屬父疾及身疾多,為有司所奏。明帝詔曰:「殷道矜有生便病,比更無橫疾。恆因愚習惰,久妨清敘。左遷散騎常侍,領校尉。」恆歷官清顯,至金紫光祿大夫。建武中卒。



 奐弟伷女為長沙王晃妃,世祖詔曰:「奐自陷逆節,長沙王妃男
 女並長,且奐又出繼,前代或當有準,可特不離絕。」奐從弟繢。



 繢字叔素,宋車騎將軍景文子也。弱冠,為秘書郎,太子舍人,轉中書舍人。



 景文以此授超階,令繢經年乃受。景文封江安侯,繢襲其本爵,為始平縣五等男。



 遷秘書丞,司徒右長史。元徽末,除寧朔將軍、建平王征北長史、南東海太守,黃門郎,寧朔將軍、東陽太守。世祖為撫軍,吏部尚書張岱選繢為長史,呈選牒。太祖笑謂岱曰:「此可謂素望。」遷散騎常侍,驍騎將軍。出補義興太守,輒錄郡吏陳伯喜付陽羨獄,欲殺之。縣令孔逭不知何罪,不受繢教,為有司所奏,繢坐白衣領職。遷太子中庶子,領驍騎,轉長史兼侍中。世祖出射雉,繢信佛法,稱疾不從駕。轉左民尚書,以母老乞解職,改授寧朔將軍、大司馬長史、淮陵太守。出為宣城太守,秩中二千石。隆昌元年,遷輔國將軍、太傅長史,不拜。仍為冠軍將軍、豫章內史。進號征
 虜。又坐事免官。除冠軍將軍,司徒左長史,散騎常侍,隨王師。



 除征虜將軍,驃騎長史,遷散騎常侍,太常。永元元年卒,年五十三。謚靖子。



 繢女適安陸王子敬,世祖寵子。永明三年納妃,脩外舅姑之敬。世祖遣文惠太子相隨往繢家置酒設樂,公卿皆冠冕而至,當世榮之。



 張沖,字思約,吳郡吳人。父柬,通直郎。沖出繼從伯侍中景胤,小名查;父邵,小名梨。宋文帝戲景胤曰:「查何如梨?」景胤答曰:「梨是百果之宗,查何敢及。」



 沖亦少有至性,辟州主簿,隨從叔永為將帥,除綏遠將軍、盱眙太守。永征彭城,遇寒雪,軍人足脛凍斷者十七八,沖足指皆墮。除尚書駕部郎,桂陽王征南中兵,振威將軍。歷驃騎太尉南中郎參軍,不拜。遷征西從事中郎,通直郎,武陵王北中郎直兵參軍,長水校尉,除寧朔將軍,本官如故。遷左軍將軍,加寧朔將
 軍,輔國將軍。沖少從戎事,朝廷以幹力相待,故歷處軍校之官。出為馬頭太守,徙盱眙太守,輔國將軍如故。永明六年,遷西陽王冠軍司馬。八年,為假節、監青冀二州刺史事,將軍如故。沖父初卒,遺命曰:「祭我必以鄉土所產,無用牲物。」沖在鎮,四時還吳園中取果菜,流涕薦焉。仍轉刺史。



 鬱林即位,進號冠軍將軍。明帝即位,以晉壽太守王洪範代沖。除黃門郎,加征虜將軍。建武二年,虜寇淮泗,假沖節,都督青冀二州北討諸軍事,本官如故。



 虜并兵攻司州青徐,詔出軍分其兵勢。沖遣軍主桑系祖由渣口攻拔虜建陵、驛馬、厚丘三城,多所殺獲。又與洪軌範遣軍主崔季延襲虜紀城,據之。沖又遣軍主杜僧護攻拔虜虎坑、馮時、即丘三城,驅生口輜重還。至溘溝,虜救兵至,緣道要擊,僧護力戰,大破之。



 其年,遷廬陵王北中郎司馬、加冠軍將軍,未拜,豐城公遙昌為豫州,上慮寇難未已,徙沖
 為征虜長史、南梁郡太守。永泰元年,除江夏王前軍長史。東昏即位,出為建安王征虜長史、輔國將軍、江夏內史,行郢州府州事。永元元年,遷持節、督豫州軍事、豫州刺史,代裴叔業。竟不行。明年,遷督南兗兗徐青冀五州、輔國將軍、南兗州刺史,持節如故。會司州刺史申希祖卒,以沖為督司州軍事、冠軍將軍、司州刺史。裴叔業以壽春降虜,又遷沖為督南兗兗徐青冀五州、南兗州刺史,持節、將軍如故。並未拜。崔慧景事平,徵建安王寶夤還都,以沖為督郢司二州、郢州刺史,持節、將軍如故。一歲之中,頻授四州,至此受任。其冬,進征虜將軍。



 封定襄侯,食邑千戶。



 梁王義師起,東昏遣驍騎將軍薛元嗣、制局監暨榮伯領兵及糧運百四十餘船送沖,使拒西師。元嗣等懲劉山陽之敗,疑沖不敢進,停住夏口浦。聞義師將至,元嗣、榮伯相率入郢城。時竟陵太守房僧寄被代還至郢,東昏
 敕僧寄留守魯山,除驍騎將軍。僧寄謂沖曰:「臣雖未荷朝廷深恩,實蒙先帝厚澤。蔭其樹者不折其枝,實欲微立塵效。」沖深相許諾,共結盟誓。乃分部拒守,遣軍主孫樂祖數千人助僧寄據魯山岸立城壘。



 明年二月,梁王出沔口,圍魯山城。遣軍主曹景宗等過江攻郢城,未及盡濟,沖遣中兵參軍陳光靜等開門出擊,為義師所破,光靜戰死,沖固守不出。景宗於是據石橋浦,連軍相續,下至加湖。東昏遣軍主巴西梓潼二郡太守吳子陽、光子衿、李文釗、陳虎牙等十三軍援郢,至加湖不得進,乃築城舉烽,城內亦舉火應之。而內外各自保,不能相救。



 沖病死,元嗣、榮伯與沖子孜及長史江夏內史程茂固守。東昏詔贈沖散騎常侍、護軍將軍。假元嗣、子陽節。江水暴長,加湖城淹漬,義師乘高艦攻之,子陽等大敗散。魯山城乏糧,軍人於磯頭捕細魚供食,密治輕船,將奔夏口。梁王命偏
 軍斷其取路,防備越逸。房僧寄病死,孫樂祖窘,以城降。



 郢城被圍二百餘日,士庶病死者七八百家。魯山既敗,程茂及元嗣等議降,使孜為書與梁王。沖故吏青州治中房長瑜謂孜曰:「前使君忠貫昊天,操逾松竹。郎君但當端坐畫一,以荷析薪。若天運不與,幅巾待命,以下從使君。今若隨諸人之計,非唯郢州士女失高山之望,亦恐彼所不取也。」魯山陷後二日,元嗣等以郢城降。



 東昏以程茂為督郢司二州、輔國將軍、郢州刺史,元嗣為督雍梁南北秦四州郢州之竟陵司州之隨郡、冠軍將軍、雍州刺史,並持節。時郢魯二城已降,死者相積,竟無叛散。時以沖及房僧寄比臧洪之被圍也。贈僧寄益州刺史。



 時新蔡太守席謙,永明中為中書郎王融所薦。父恭穆,鎮西司馬,為魚復侯所害。至是謙鎮盆城,聞義師東下,曰:「我家世忠貞,殞死不二。」為陳伯之所殺。



 史臣曰:石碏棄子,弘滅親之戒;鮑永晚降,知事新之節。王奐誠在靡貳,跡允嚴科;張沖未達天心,守迷義運。致危之理異,為亡之事一也。



 贊曰:王居北牧,子未克家。終成乾紀,覆此胄華。張壘窮守,死如亂麻。為悟既晚,辯見方賒。



\end{pinyinscope}