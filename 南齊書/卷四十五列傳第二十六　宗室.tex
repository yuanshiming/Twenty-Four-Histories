\article{卷四十五列傳第二十六 宗室}

\begin{pinyinscope}

 衡陽元王道度始安貞王道生子遙光遙欣遙昌安陸昭王緬衡陽元王道度,太祖長兄也。與太祖俱受學雷次宗。宣帝問二兒學業,次宗答曰:「其兄外朗,其弟內潤,皆良璞也。」隨宣帝征伐,仕至安定太守,卒於宋世。



 建元二年,追加封謚。無子,太祖以第十一子鈞繼道度後。



 鈞字宣禮。永明四年為江州刺史,加散騎常侍。母區貴人卒,居喪盡禮。六年,遷為征虜將軍。八年,遷驍騎將軍,常侍如故,仍轉左衛將軍。鈞有好尚,為世祖所知。兄弟中意遇次鄱陽王
 鏘。十年,轉中書令,領石頭戍事。遷散騎常侍,秘書監,領驍騎如故。不拜。隆昌元年,改加侍中,給扶。海陵立,轉撫軍將軍,侍中如故。尋遇害,年二十二。



 明帝即位,以永陽王子氏仍本國,繼元王為孫。



 子氏,字雲璵,世祖第二十子也。永明七年,封義安王,後改永陽,永泰元年見害,年十四。復以武陵昭王曄第三子子坦奉元王後。



 始安貞王道生,字孝伯,太祖次兄也。宋世為奉朝請,卒。建元元年,追封謚。



 建武元年,追尊為景皇,妃江氏為后。立寢廟於御道西,陵曰脩安。生子鳳、高宗、安陸昭王緬。鳳字景慈,官至正員郎。卒於宋世,謚靖世子。明帝建武元年,贈侍中、驃騎將軍,開府儀同三司、始安靖王。改華林鳳莊門為望賢門,太極東堂書畫鳳鳥,題為神鳥,而改鸞鳥為神雀。子遙光嗣。



 遙光字元暉。生有躄疾,太祖謂不堪奉拜祭祀,欲封其弟,世祖諫,
 乃以遙光襲爵。初為員外郎,轉給事郎,太孫洗馬,轉中書郎,豫章內史,不拜。高宗輔政,遙光好天文候道,密懷規贊。隆昌元年,除驍騎將軍、冠軍將軍、南東海太守,行南徐州事;仍除南彭城太守,將軍如故;又除輔國將軍、吳興太守。高宗廢鬱林,又除冠軍將軍、南蠻校尉、西平中郎長史、南郡太守。一歲之內頻五除,並不拜。



 是時高宗欲即位,誅賞諸事唯遙光共謀議。



 建武元年,以為持節、都督揚南徐二州諸軍事、前將軍、揚州刺史。晉安王寶義為南徐州,遙光求解督,見許。二年,進號撫軍將軍,加散騎常侍,給通憲車鼓吹。遙光好吏事,稱為分明。頗多慘害。足疾不得同朝列,常乘輿自望賢門入。



 每與上久清閑,言畢,上索香火,明日必有所誅殺。上以親近單少,憎忌高、武子孫,欲並誅之,遙光計畫參議,當以次施行。永泰元年,即本位為大將軍,給油絡車。帝不豫,遙光數入侍疾,帝漸
 甚,河東王鉉等七王一夕見殺,遙光意也。



 帝崩,遺詔加遙光侍中、中書令,給扶。永元元年,給班劍二十人,即本號開府儀同三司。遙光既輔政,見少主即位,潛與江祏兄弟謀自樹立。弟遙欣在荊楚,擁兵居上流,密相影響。遙光當據東府號令,使遙欣便星速急下。潛謀將發,而遙欣病死。江祏被誅,東昏侯召遙光入殿,告以祏罪。遙光懼,還省便陽狂號哭,自此稱疾不復入臺。先是遙光行還入城,風飄儀傘出城外。



 遙光弟遙昌先卒壽春,豫州部曲皆歸遙光;及遙欣喪還葬武進,停東府前渚,荊州眾力送者甚盛。帝誅江祏後,慮遙光不自安,欲轉為司徒還第,召入喻旨。遙光慮見殺,八月十二日晡時,收集二州部曲,於東府門聚人眾,街陌頗怪其異,莫知指趣也。遙光召親人丹陽丞劉渢及諸傖楚,欲以討劉暄為名。夜遣數百人破東冶出囚,尚方取仗。又召驍騎將軍垣歷生,歷生隨
 信便至,勸遙光令率城內兵夜攻臺,輦籥燒城門,曰:「公但乘輿隨後,反掌可得。」遙光意疑不敢出。天稍曉,遙光戎服出聽事,停輿處分上仗登城行賞賜。歷生復勸出軍,遙光不肯,望臺內自有變。



 至日中,臺軍稍至,尚書符遙光曰:「逆順之數,皎然有征,干紀亂常,刑茲罔赦。蕭遙光宗室蚩庸,才行鄙薄,緹裙可望,天路何階。受遇自昔,恩加猶子,禮絕帝體,寵越皇季。旗章車服,窮千乘之尊;闉隍爽闓,踰百雉之制。及聖后在天,親受顧託,話言在耳,德音猶存,侮蔑天明,罔畏不義,無君之心,履霜有日,遂乃稱兵內犯,竊發京畿,自古巨釁,莫斯為甚。今便分命六師,弘宣九伐。皇上當親御戎軒,弘此廟略。信賞必罰,有如大江。」於是戒嚴,曲赦京邑。領軍蕭坦之屯湘宮寺,鎮軍司馬曹虎屯清溪大橋,太子右衛率左興盛屯東府東籬門。



 眾軍圍東城三面,燒司徒二府。遙光遣垣歷生從西門出戰,
 臺軍屢北,殺軍主桑天愛。初,遙光起兵,問諮議參軍蕭暢,暢正色拒折不從,十五日,暢與撫軍長史沈昭略潛自南出,濟淮還臺,人情大沮。十六日,垣歷生從南門出戰,因棄槊降曹虎軍,虎命斬之。遙光大怒,於床上自竦踴,使殺歷生兒。



 其晚,臺軍射火箭燒東北角樓,至夜城潰。遙光還小齋,帳中著衣帢坐,秉燭自照,令人反拒,齋閣皆重關。左右並踰屋散出。臺軍主劉國寶、時當伯等先入。



 遙光聞外兵至,吹滅火,扶匐下床。軍人排閣入,於暗中牽出斬首,時年三十二。



 遙光未敗一夕,城內皆夢群蛇緣城四出,各各共說之,咸以為異。臺軍入城,焚燒屋宇且盡。



 遙光府佐司馬端為掌書記,曹虎謂之曰:「君是賊非?」端曰:「僕荷始安厚恩,今死甘心。」虎不殺,執送還臺,徐世手剽殺之。劉渢遁走還家園,為人所殺。



 端,河內人。渢,南陽人,事繼母有孝行,弟溓事渢亦謹。



 詔斂葬遙光屍,原其諸子。追贈
 桑天愛輔國將軍、梁州刺史。以江陵公寶覽為始安王,奉靖王後。永元二年,為持節、督湘州、輔國將軍、湘州刺史。



 遙欣字重暉。宣帝兄西平太守奉之無後,以遙欣繼為曾孫。除秘書郎,太子舍人,巴陵王文學,中書郎。延興元年,高宗樹置,以遙欣為持節、督兗州緣淮軍事、寧朔將軍、兗州刺史。仍為督豫州郢州之西陽司州之汝南二郡、輔國將軍、豫州刺史,持節如故。未之任。建武元年,進號西中郎將,封聞喜縣公。遷使持節、都督荊雍益寧梁南北秦七州軍事、右將軍、荊州刺史。改封曲江公。高宗子弟弱小,晉安王寶義有廢疾,故以遙光為揚州居中,遙欣居陜西在外,權勢並在其門。遙欣好勇,聚畜武士,以為形援。四年,進號平西將軍。永泰元年,以雍州虜寇,詔遙欣以本官領刺史,寧蠻校尉,移鎮襄陽,虜退不行。永元元年卒,年三十一。贈侍中、司空,謚康公。葬用王禮。



 遙昌字季暉。解褐秘書郎,太孫舍人,給事中,秘書丞。延興元年,除黃門侍郎,未拜,仍為持節、督郢司二州軍事、寧朔將軍、郢州刺史。建武元年,進號冠軍將軍。封豐城縣公,千五百戶。未之鎮,徙督豫州郢州之西陽司州之汝南二郡軍事、征虜將軍、豫州刺史,持節如故。



 二年,虜主元宏寇壽春,遣使呼城內人。遙昌遣參軍崔慶遠、朱選之詣宏。慶遠曰:「旌蓋飄搖,遠涉淮、泗、風塵慘烈,無乃上勞?」宏曰:「六龍騰躍,倏忽千里,經途未遠,不足為勞。」慶遠曰:「川境既殊,遠勞軒駕。屈完有言:『不虞君之涉吾地也,何故?』」宏曰:「故當有故。卿欲使我含瑕依違,為欲指斥其事?」慶遠曰:「君包荒之德,本施北政,未承來議,無所含瑕。」宏曰:「朕本欲有言,會卿來問。齊主廢立,有其例不?」慶遠曰:「廢昏立明,古今同揆。中興克昌,豈唯一代?主上與先武帝,非唯昆季,有同魚水。武皇臨崩,托以後事。嗣孫荒迷,
 廢為鬱林,功臣固請,爰立明聖。上逼太后之嚴令,下迫群臣之稽顙,俯從億兆,踐登皇極。未審聖旨獨何疑怪?」宏曰:「聞卿此言,殊解我心。



 但哲婦傾城,何足可用。果如所言,武帝子弟今皆何在?」慶遠曰:「七王同惡,皆伏管、蔡之誅,其餘列蕃二十餘國,內升清階,外典方牧。哲婦之戒,古人所惑;然十亂盈朝,實唯文母。」宏曰:「如我所聞,靡有孑遺。卿言美而乖實。未之全信。」宏又曰:「雲羅所掩,六合宜一。故往年與齊武有書,言今日之事,書似未達齊主。命也。南使既反,情有愴然,朕亦休兵。此段猶是本意,不必專為問罪。



 若如卿言,便可釋然。」慶遠曰:「見可而進,知難而退,聖人奇兵。今旨欲憲章聖人,不失舊好,豈不善哉!」宏曰:「卿為欲朕和親?為欲不和?」慶遠曰:「和親則二國交歡,蒼生再賴;不和則二國交怨,蒼生塗炭。和與不和,裁由聖衷。」



 宏曰:「朕來為復游行鹽境,北去洛都,率爾便至。亦不攻城,亦不伐塢,卿
 勿以為慮。」



 宏設酒及羊炙雜果,又謂慶遠曰:「聽卿主克黜凶嗣,不違忠孝。何以不立近親,如周公輔成王,而茍欲自取?」慶遠答曰:「成王有亞聖之賢,故周公得輔而相之。今近蕃雖無悖德,未有成王之賢。霍光亦捨漢蕃親而遠立宣帝。」宏曰:「若爾,霍光向自立為君,當復得為忠臣不?」慶遠曰:「此非其類,乃可言宣帝立與不立義當云何。皇上豈得與霍光為匹?若爾,何以不言『武王伐紂,何意不立微子而輔之,茍貪天下?」宏大笑。明日引軍向城東,遣道登道人進城內施眾僧絹五百匹,慶遠、選之各褲褶絡帶。



 遙昌,永泰元年卒。上愛遙昌兄弟如子,甚痛惜之。贈車騎將軍、儀同三司。



 帝以問徐孝嗣,孝嗣曰:「豐城本資尚輕,贈以班台,如為小過。」帝曰:「卿乃欲存萬代準則,此我孤兄子,不得與計。」謚憲公。



 安陸昭王緬,字景業。善容止。初為秘書郎,宋邵陵王文學,中書郎。
 建元元年,封安陸侯,邑千戶。轉太子中庶子,遷侍中。世祖即位,遷五兵尚書,領前軍將軍,仍出為輔國將軍、吳郡太守,少時大著風績。竟陵王子良與緬書曰:「竊承下風,數十年來未有此政。」世祖嘉其能,轉持節、都督郢州司州之義陽軍事、冠軍將軍、郢州刺史。永明五年,還為侍中,領驍騎將軍,仍遷中領軍。明年,轉散騎常侍,太子詹事。出為會稽太守,常侍如故。遷使持節、都督雍梁南北秦四州荊州之竟陵司州之隨郡軍事、左將軍、寧蠻校尉、雍州刺史。緬留心辭訟,親自隱恤,劫抄度口,皆赦遣許以自新,再犯乃加誅,為百姓所畏愛。



 九年,卒。詔賻錢十萬,布二百匹。喪還,百姓緣沔水悲泣設祭,于峴山為立祠。贈侍中、衛將軍,持節、都督、刺史如故。給鼓吹一部。謚昭侯。年三十七。



 高宗少相友愛,時為僕射,領衛尉,表求解衛尉,私第展哀,詔不許。每臨緬靈,輒慟哭不成聲。建武元年,贈
 侍中、司徒、安陸王,邑二千戶。



 子寶晊嗣,為持節、督湘州軍事、輔國將軍、湘州刺史。弟寶覽為江陵公,寶宏汝南公,邑各千五百戶。二年,寶晊進號冠軍將軍。三年,寶宏改封宵城。永元元年,以安陸郡邊虜,寶晊改封湘東王,進號征虜將軍。二年,為左衛將軍。高宗兄弟一門皆尚吏事,寶晊粗好文章。義師下,寶晊在城內,東昏廢,寶晊望物情歸己,坐待法駕,既而城內送首詣梁王。宣德太后臨朝,以寶晊為太常。寶晊不自安,謀反,兄弟皆伏誅。



 史臣曰:太祖膺期御世,二昆夙殞,慶命傍流,追序蕃胙。安陸王緬以宗子戚屬,弱年進仕,典郡臨州,去有餘迹,遣愛在民。蓋因情而可感,學以從政,夫豈必然。



 贊曰:太祖二昆,追樹雙蕃。元託繼胤,貞興子孫。並用威福,自取亡存。安陸稱美,事表西魂。



\end{pinyinscope}