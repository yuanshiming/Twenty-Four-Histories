\article{卷十六 志第八 百官}

\begin{pinyinscope}

 建官設職,興自炎昊,方乎隆周之冊,表乎盛漢之書。存改回沿,備於歷代,先賢往學,以之雕篆者眾矣。若夫胡廣《舊儀》,事惟簡撮;應劭《官典》,殆無遺恨。王朗奏議,屬霸國之初基;陳矯增曹,由軍事而補闕。今則有《魏氏官儀》、魚豢《中外官》也。山濤以意辯人,不囗囗囗;荀勖欲去事煩,唯論并省。定制成文,本之《晉令》,後代承業,案為前準。肇域官品,區別階資,蔚宗選簿梗概,欽明階次詳悉,虞通、劉寅因荀氏之作,矯舊增新,今古相校。齊受宋禪,事遵常典,既有司存,無所偏廢。其餘散在史注,多已筌拾,覽者易知,不重述也。諸臺府郎令史職吏以下,具見長水校
 尉王圭之《職儀》。



 相國蕭、曹以來,為人臣極位。宋孝建用南譙王義宣。至齊不用人,以為贈,不列官。



 太宰。



 宋大明用江夏王義恭,以後無人。齊以為贈。



 太傅。



 太師、太保、太傅,周舊官。漢末,董卓為太師。晉惠帝初,衛瓘為太保。自後無太師,而太保為贈。齊唯置太傅。



 大司馬。



 大將軍。



 宋元嘉用彭城王義康,後無人。齊以為贈。



 太
 尉。



 司徒。



 司空。



 三公,舊為通官。司徒府領天下州郡名數戶口簿籍,雖無,常置左右長史、左西曹掾屬、主簿、祭酒、令史以下。晉世王導為司徒,右長史干寶撰立官府《職儀》已具。



 特進。



 位從公。



 諸開府儀同三司。



 驃騎將軍。



 車騎將軍。



 衛將軍。



 鎮軍將軍。



 中軍將軍。



 撫軍將軍。



 四征將軍。東、西、南、北。



 四鎮將軍。



 凡諸將軍加「大」字,位從公。開府儀同如公。凡公督府置佐:長史、司馬各一人,諮議參軍二人。



 諸曹有錄事,記室,戶曹,倉曹,中直兵,外兵,騎兵,長流賊曹,刑獄賊曹,城局,法曹,田曹,水曹,鎧曹,車曹,士曹,集曹,右戶,十八曹。城局曹以上署正參軍,法曹以下署行參軍,各一人。其行參軍無署者,為長兼員。其府佐史則從事中郎二人,倉曹掾、戶曹屬、東西閣祭酒各一人,主簿舍人御屬二人。加崇者,則左右長史四人,中郎掾屬並增數。其未及開府,則置府亦有佐史,其數
 有減。小府無長流,置禁防參軍。



 四安將軍。



 四平將軍。



 左、右、前、後將軍。



 征虜將軍。



 四中郎將。



 晉世荀羨、王胡之並居此官。宋、齊以來,唯處諸王,素族無為者。



 冠軍將軍。



 輔國將軍。



 寧朔將軍。



 寧遠將軍。



 龍驤將軍。



 凡諸小號,亦有置府者。



 太常。



 府置丞一人,五官、功曹、主簿,九府九史皆然。領官如左:



 博士,謂之太學博士;國子祭酒一人,博士二人,助教十人;建元四年,有司奏置國學,祭酒準諸曹尚書,博士準中書郎,助教準南臺御史。選經學為先。若其人難備,給事中以還明經者,以本位領。其下典學二人,三品,準太常主簿;戶曹、儀曹各二人,五品;白簿治禮吏八人,六品;保學醫二人;威儀二人。其夏,國諱廢學,有司奏省助教以下。永明三年,立學,尚書令王儉領祭酒。八年,國子博士何胤單為祭酒,
 疑所服,陸澄等皆不能據,遂以玄服臨試。月餘日,博議定,乃服朱衣。



 總明觀祭酒一人;右泰始六年,以國學廢,初置總明觀,玄、儒、文、史四科,科置學士各十人,正令史一人,書令史二人,乾一人,門吏一人,典觀吏二人。建元中,掌治五禮。永明三年,國學建,省。



 太廟令一人,丞一人;明堂令一人,丞一人;太祝令一人,丞一人;太史令
 一人,丞一人;廩犧令一人,丞一人;置令丞以下皆有職吏。



 太樂令一人,丞一人;諸陵令;永明末置,用二品三品勛。置主簿、戶曹各一人,六品保舉。



 光祿勳。



 府置丞一人。領官如左:



 左右光祿大夫;位從公,開府置佐史如公。



 光祿大夫;皆銀章青綬,詔加金章紫綬者,為金紫光祿大夫。樂安任遐為光祿,就王晏乞一片金,晏乃啟轉為金
 紫,不行。



 太中大夫;中散大夫。



 諸大夫官,皆處舊齒老年,重者加親信二十人。



 衛尉。



 府置丞一人。掌宮城管籥。張衡《西京賦》曰「衛尉八屯,警夜巡晝」。宮城諸卻敵樓上本施鼓,持夜者以應更唱,太祖以鼓多驚眠,改以鐵磬云。



 廷尉。



 府置丞一人,正一人,監一人,評一人,律博士一人。



 大司農。



 府置丞一人。領官如左:



 太倉令一人,丞一人;
 導官令一人,丞一人;籍田令一人,丞一人。



 少府。



 府置丞一人。領官如左:



 左右尚方令各一人,丞一人;鍛署丞一人;永明三年省,四年復置。



 御府令一人,丞一人;東冶令一人,丞一人;南冶令一人,丞一人;平準令一人,丞一人。



 上林令一人,丞一人。亦屬尚書殿中曹。



 將作大匠。



 太
 僕。



 大鴻臚。



 三卿不常置。將作掌宮廟土木。太僕掌郊禮執轡。鴻臚掌導護贊拜。有事權置兼官,畢乃省。



 乘黃令一人:



 掌五輅安車,大行凶器轀輬車;客館令:



 掌四方賓客。



 宣德衛尉、少府、太僕。



 鬱林王立,文安太后即尊號,以宮名置之。



 大長秋。



 鬱林立皇后置。



 錄尚書。



 尚書令。



 總領尚書臺二十曹,為內臺主。行遇諸王以下,皆禁駐。左右僕射分道。無令,左僕射為臺主,與令同。



 左僕射:



 領殿中主客二曹事,諸曹郊廟、園陵、車駕行幸、朝儀、臺內非違、文官舉補滿敘疾假事,其諸吉慶瑞應眾賀、災異賊發眾變、臨軒崇拜、改號格制、蒞官銓選,凡諸除署、功論、封爵、貶黜、八議、疑讞、通關案,則左僕射主,右僕射次經,維是黃案,左僕射右僕射署朱符見字,經都丞竟,右僕射橫畫成目,左僕射畫,令畫。右官闕,則以次並畫。若無左右,則直置僕射在其中間,總左右事。



 吏部尚書:



 領吏部、刪定、三公、比部四曹。



 度支尚書:



 領度支、金部、倉部、起部四曹。



 左民尚書:



 領左民、駕部二曹。



 都官尚書:



 領都官、水部、庫部、功論四曹。



 五兵尚書:



 領中兵、外兵二曹。



 祠部尚書:



 右僕射通職,不俱置。



 起部尚書:



 興立宮廟權置;事畢省。



 左丞一人:



 掌宗廟郊祠、吉慶瑞應、災異、立作格制、諸案彈、選用除置、吏補滿除遣注職。



 右丞一人:



 掌兵士百工補役死叛考代年老疾病解遣、其內外諸庫藏穀帛、刑罪創業諍訟、田地船乘、稟拘兵工死叛、考剔討補、差分百役、兵器諸營署人領、州郡租布、民戶移徙、州郡縣併帖、城邑民戶割屬、刺史二千石令長丞尉被收及免贈、文武諸犯削官事。白案,右丞上署,左丞次署。黃案,左丞上署,右丞次署。諸立格制及詳讞大事宗廟朝廷
 儀體,左丞上署,右丞次署。自令僕以下五尚書八座二十曹,各置郎中令史以下,又置都令史分領之。僕射掌朝軌,尚書掌讞奏,都丞任碎,在彈違諸曹緣常及外詳讞事。應須命議相值者,皆郎先立意,應奏黃案及關事,以立意官為議主。凡辭訴有漫命者,曹緣咨如舊。若命有咨,則以立意者為議主。



 武庫令一人;屬庫部。



 車府令一人,丞
 一人;屬駕部。



 公車令一人;大官令一人,丞一人;大醫令一人,丞一人;內外殿中監各一人;內外驊騮廄丞各一人;材官將軍一人,司馬一人;屬起部,亦屬領軍。



 侍中祭酒。高功者稱之。



 侍中。



 漢世為親近之職。魏、晉選用,稍增華重,而大意不異。宋文帝元嘉中,王華、王曇首、殷景仁等,並為侍中,情在親密,與帝接膝共語,貂拂帝手,拔貂置案上,語畢復手插之。孝武時,侍中何偃南郊陪乘,鑾輅過白門闕,偃將匐,帝乃接之曰:「朕乃陪卿。」齊世朝會,多以美姿容者
 兼官。永元三年,東昏南郊,不欲親朝士,以主璽陪乘,前代未嘗有也。侍中呼為門下。亦置令史。領官如左:



 給事黃門侍郎:



 亦管知詔令,世呼為小門下;散騎常侍,通直散騎常侍,員外散騎常侍:



 舊與侍中通官,其通直員外,用衰老人士,故其官漸替。宋大明雖華選比侍中,而人情久習,終不見重,尋復如初。



 散騎侍郎,通直散騎侍郎,員外散騎侍郎;給事中;奉朝請;駙馬都尉;集書省職,置正書、令史。朝散用衣冠之餘,人數
 猥積。永明中,奉朝請至六百餘人。



 中書監一人,令一人,侍郎四人,通事舍人無員。



 中書省職,置主書、令史、正書以下。



 秘書監一人,丞一人。郎。著作佐郎。



 晉秘書閣有令史,掌眾書,見《晉令》。令亦置令史、正書及弟子,皆典教書畫。



 御史中丞一人。



 晉江左中丞司隸分督百僚,傅咸所云「行馬內外」是也。今中丞則職無不察,專道而行,騶輻禁呵,加以聲色,武將相逢,輒致侵犯,若有鹵簿,至相驅擊。宋孝建二年制,中丞與尚書令分道,雖丞郎下朝相值,亦得斷之,餘內
 外眾官,皆受停駐。



 治書侍御史二人;侍御史十人。



 蘭臺置諸曹內外督令以下。



 謁者僕射一人。



 謁者十人。



 謁者臺,掌朝覲賓饗。



 領軍將軍、中領軍。



 護軍將軍、中護軍。



 凡為中,小輕,同一官也。諸為將軍官,皆敬領、護。諸王為將軍,道相逢,則領、護讓道。置長史、司馬、五官、功曹、主簿。



 左右二衛將軍。



 驍騎將軍。



 游擊將軍。



 晉世以來,謂領、護至驍、游為六軍。二衛置司馬、次官、功曹、主簿以下。



 左右二中郎將。



 前軍將軍,後軍將軍,左軍將軍,右軍將軍,號四軍。



 屯騎,步兵,射聲,越騎,長水:五校尉。



 虎賁中郎將。



 冗從僕射。



 羽林監。



 積射將軍。



 彊弩將軍。



 殿中將軍,員外殿中將軍。



 殿中司馬督。



 武衛將軍。



 武騎常侍。



 自二衛、四軍、五校已下,謂之「西省」,而散騎為「東省」。



 丹陽尹。



 位次九卿下。



 太子太傅。



 少傅:



 府置丞、功曹、五官、主簿;太子詹事;府置丞一人以下;太子率更令;
 太子家令:



 置丞;太子僕;太子門大夫;太子中庶子;太子中舍人;太子洗馬;太子舍人;太子左右衛率各一;太子翊軍步兵屯騎三校尉;太子旅賁中郎將一人;太子左右積弩將軍;
 太子殿中將軍、員外殿中將軍;太子倉官令;太子常從虎賁督。



 右東宮職僚。



 州牧、刺史。



 魏、晉世州牧隆重,刺史任重者為使持節都督,輕者為持節督。起漢順帝時,御史中丞馮赦討九江賊,督揚、徐二州軍事,而何、徐《宋志》云起魏武遣諸州將督軍,王珪之《職儀》云起光武,並非也。晉太康中,都督知軍事,刺史治民,各用人。惠帝末,乃并任,非要州則單為刺史。州朝置別駕、治中、議曹、文學祭酒、諸曹部從事史。



 護南蠻校尉。



 府置佐史。隸荊州。晉、宋末省。建元元年復置,三年省。延興元年置,建武省。



 護三巴校尉。



 宋置。建元二年改為刺史。



 寧蠻校尉。



 府亦置佐史,隸雍州。



 平蠻校尉。



 永明三年置,隸益州。



 鎮蠻校尉。



 隸寧州。



 護西戎校尉。



 護羌校尉。



 右四校尉,亦置四夷。



 平越中郎將。



 府置佐史,隸廣州。



 郡太守、內史。



 縣令、相。



 郡縣為國者,為內史、相。



 鎮蠻護軍。



 安遠護軍。



 晉世雜號,多為郡領之。



 諸王師、友、文學各一人。



 國官郎中令、中尉、大農為三卿,左右常侍、侍郎,上軍、中軍、下軍三軍,典書、典祠、學官、典衛四令,食官、廄牧長、謁者以下。公侯置郎中
 令一卿。



 贊曰:百司分置,惟皇命職。雲師鳥紀,各有其式。



\end{pinyinscope}