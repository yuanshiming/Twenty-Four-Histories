\article{卷四本紀第四鬱林王}

\begin{pinyinscope}

 鬱林王昭業,字元尚,文惠太子長子也。小名法身。世祖即位,封南郡王,二千戶。永明五年十一月戊子,冠於東宮崇政殿。其日小會,賜王公以下帛各有差,給昭業扶二人。七年,有司奏給班劍二十人,鼓吹一部,高選友、學。十一年,給皂輪三望車。詔高選國官。文惠太子薨,立昭業為皇太孫,居東宮。世祖崩,太孫即位。



 八月,壬午,詔稱先帝遺詔,以護軍將軍武陵王曄為衛將軍,征南大將軍陳顯達即本號,並開府儀同三司,尚書左僕射西昌侯鸞為尚書令。太孫詹事沈文季為護軍將軍。癸未,以司徒竟陵王子良為太傅。詔曰:「朕以寡薄,嗣膺寶政,對越靈命,欽若前圖,思所以敬守成規,拱
 揖群后。哀荒在日,有懵大猷,宜育德振民,光昭睿範。凡逋三調及眾責,在今年七月三十日前,悉同蠲除。其備償封籍貨鬻未售,亦皆還主。御府諸署池田邸冶,興廢沿事,本施一時,於今無用者,詳所罷省。



 公宜權禁,一以還民,關市征賦,務從優減。」丙戌,詔曰:「近北掠餘口,悉充軍實。刑故無小,罔或攸赦,撫辜興仁,事深睿範。宜從蕩宥,許以自新,可一同放遣,還復民籍。已賞賜者,亦皆為贖。」辛丑,詔曰:「往歲蠻虜協謀,志擾邊服,群帥授略,大殲兇醜。革城克捷,及舞陰固守,二處勞人,未有沾爵賞者,可分遣選部,往彼序用。」



 九月,癸丑,詔「東西二省府國,長老所積,財單祿寡,良以矜懷。選部可甄才品能,推校年月,邦守邑丞,隨宜量處,以貧為先。」辛酉,追尊文惠皇太子為世宗文皇帝。冬,十月,壬寅,尊皇太孫太妃為皇太后,立皇后何氏。



 十一月,辛亥,立臨汝公昭文為新安王,曲江公昭秀為
 臨海王,皇弟昭粲為永嘉王。



 隆昌元年春,正月,丁未,改元,大赦。加太傅、竟陵王子良殊禮,驍騎將軍、晉熙王銶為郢州刺史,丹陽尹、安陸王子敬為南兗州刺史,征北大將軍、晉安王子懋為江州刺史,臨海王昭秀為荊州刺史,永嘉王昭粲為南徐州刺史,征南大將軍陳顯達進號車騎大將軍,郢州刺史、建安王子真為護軍將軍。詔百僚極陳得失。又詔王公以下各舉所知。戊申,以護軍將軍沈文季為領軍將軍。己酉,以前將軍曹虎為雍州刺史,右衛將軍薛淵為司州刺史。庚戌,以寧朔將軍蕭懿為梁、南秦二州刺史,輔國長史申希祖為交州刺史。辛亥,車駕祠南郊。詔曰:「執耜暫忘,懸磬比室,秉機或惰,無褐終年。非怠非荒,雖由王道,不稂不莠,實賴民和。頃歲,多稼無爽,遺秉如積,而三登之美未臻,萬斯之基尚遠。且風土異宜,百民舛務,刑章
 治緒,未必同源。妨本害政,事非一揆,冕旒屬念,無忘夙興。可嚴下州郡,務滋耕殖,相畝闢疇,廣開地利,深樹國本,克阜民天。又詢訪獄市,博聽謠俗,傷風損化,各以條聞,主者詳為條格。」戊午,車駕拜景安陵。己巳,以新除黃門待郎周奉叔為青州刺史。



 二月,辛卯,車駕祠明堂。夏,四月,辛巳,衛將軍、開府儀同三司武陵王曄薨。戊子,太傅竟陵王子良薨。戊戌,以前沙州刺史楊炅為沙州刺史。丁酉,以驃騎將軍廬陵王子卿為衛將軍。尚書右僕射鄱陽王鏘為驃騎將軍,並開府儀同三司。



 閏月,乙丑,以南東海太守蕭穎胄為青、冀二州刺史。丁卯,鎮軍大將軍鸞即本號開府儀同三司。戊辰,以中軍將軍新安王昭文為揚州刺史。六月,丙寅,以黃門侍郎王思遠為廣州刺史。秋,七月,庚戌,以中書郎蕭遙欣為兗州刺史,東莞太守臧靈智為交州刺史。癸巳,皇太后令曰:「鎮軍、車騎、左僕射、前
 將軍、領軍、左衛、衛尉、八座:自我皇曆啟基,受終於宋,睿聖繼軌,三葉重光。太祖以神武創業,草昧區夏,武皇以英明提極,經緯天人。文帝以上哲之資,體元良之重,雖功未被物,而德已在民。三靈之眷方永,七百之基已固。嗣主特鐘沴氣,爰表弱齡,險戾著於綠車,愚固彰於崇正。狗馬是好,酒色方湎。所務唯鄙事,所疾唯善人。



 世祖慈愛曲深,每加容掩,冀年志稍改,立守神器。自入纂鴻業,長惡滋甚。居喪無一日之哀,縗絰為歡宴之服。昏酣長夜,萬機斯壅,發號施令,莫知所從。閹豎徐龍駒專總樞密,奉叔、珍之互執權柄,自以為任得其人,表裏緝穆,邁蕭、曹而愈信、布,倚太山而坐平原。於是恣情肆意,罔顧天顯,二帝姬嬪,並充寵御,二宮遺服,皆納玩府。內外混漫,男女無別。丹屏之北,為酤鬻之所,青蒲之上,開桑中之肆。又微服潛行,信次忘反,端委以朝虛位,交戟而守空宮積旬矣。宰
 輔忠賢,盡誠奉主,誅鋤群小,冀能悛革,曾無克己,更深怨憾。公卿股肱,以異己置戮,文武昭穆,以德譽見猜。放肆醜言,將行屠膾,社稷危殆,有過綴旒。昔太宗克光於漢世,簡文代興於晉氏,前事之不忘,後人之師也。鎮軍居正體道,家國是賴,伊霍之舉,實寄淵謨,便可詳依舊典,以禮廢黜。中軍將軍新安王,體自文皇,睿哲天秀,宜入嗣鴻業,永寧四海。外即以禮奉迎。未亡人屬此多難,投筆增慨。」



 昭業少美容止,好隸書,世祖敕皇孫手書不得妄出,以貴重之。進對音吐,甚有令譽。王侯五日一問訊,世祖常獨呼昭業至幄座,別加撫問,呼為法身,鐘愛甚重。文惠皇太子薨,昭業每臨哭,輒號咷不自勝,俄爾還內,歡笑極樂。在世祖喪,哭泣竟,入後宮,嘗列胡妓二部夾閣迎奏。為南郡王時,文惠太子禁其起居,節其用度,昭業謂豫章王妃庾氏曰:「阿婆,佛法言,有福德生帝王家。今日見作天
 王,便是大罪,左右主帥,動見拘執,不如作市邊屠酤富兒百倍矣。」及即位,極意賞賜,動百數十萬。每見錢,輒曰:「我昔時思汝一文不得,今得用汝未?」期年之間,世祖齋庫儲錢數億垂盡。開主衣庫與皇后寵姬觀之,給閹人豎子各數人,隨其所欲,恣意輦取;取諸寶器以相剖擊破碎之,以為笑樂。居嘗裸袒,著紅褌,雜採袒服。好鬥雞,密買雞至數千價。世祖御物甘草杖,宮人寸斷用之。毀世祖招婉殿,乞閹人徐龍駒為齋。龍駒尤親幸,為後閣舍人,日夜在六宮房內。昭業與文帝幸姬霍氏淫通,龍駒勸長留宮內,聲雲度霍氏為尼,以餘人代之。嘗以邪諂自進,每謂人曰:「古時亦有監作三公者。」皇后亦淫亂,齋閣通夜洞開,內外淆雜,無復分別。中書舍人綦母珍之、朱隆之,直閣將軍曹道剛、周奉叔,並為帝羽翼。高宗屢諫不納,先啟誅龍駒,次誅奉叔及珍之,帝並不能違。既而尼媼外
 入,頗傳異語,乃疑高宗有異志。中書令何胤以皇后從叔見親,使直殿省,嘗隨后呼胤為三父,與胤謀誅高宗,令胤受事,胤不敢當,依違杜諫,帝意復止。乃謀出高宗於西州,中敕用事,不復關諮。高宗慮變,定謀廢帝。二十二日壬辰,使蕭諶、坦之等於省誅曹道剛、朱隆之等,率兵自尚書入雲龍門,戎服加朱衣於上。比入門,三失履。王晏、徐孝嗣、蕭坦之、陳顯達、王廣之、沈文季係進。帝在壽昌殿,聞外有變,使閉內殿諸房閣,令閹人登興光樓望,還報云:「見一人戎服,從數百人,急裝,在西鐘樓下。」須臾,蕭諶領兵先入宮,截壽昌閣,帝走向愛姬徐氏房,拔劍自刺不中,以帛纏頸,輿接出延德殿。諶初入殿,宿衛將士皆操弓盾欲拒戰,諶謂之曰:「所取自有人,卿等不須動!」宿衛信之,及見帝出,各欲自奮,帝竟無一言。出西弄,殺之,時年二十一,輿尸出徐龍駒宅,殯葬以王禮。餘黨亦見
 誅。



 史臣曰:鬱林王風華外美,眾所同惑。伏情隱詐,難以貌求。立嫡以長,未知瑕釁,世祖之心,不變周道。既而愆鄙內作,兆自宮闈,雖為害未遠,足傾社稷。



 《春秋》書梁伯之過,言其自取亡也。



 贊曰:十愆有一,無國不失。鬱林負荷,棄禮亡律。



\end{pinyinscope}