\article{卷二十七列傳第八 劉懷珍 李安民 王玄載(弟玄邈)}

\begin{pinyinscope}

 劉懷珍,字道玉,平原人,漢膠東康王寄後也。祖昶,宋武帝平齊,以為青州治中,至員外常侍。伯父奉伯,宋世為陳南頓二郡太守。懷珍幼隨奉伯至壽陽,豫州刺史趙伯符出獵,百姓聚觀,懷珍獨避不視,奉伯異之,曰:「此兒方興吾宗。」



 本州闢主簿。元嘉二十八年,亡命司馬順則聚黨東陽,州遣懷珍將數千人掩討平之。



 宋文帝召問破賊事狀,懷珍讓功不肯當,親人怪問焉,懷珍曰:「昔國子尼恥陳河間之級,吾豈能論邦域之捷哉!」時人稱之。



 江夏王義恭出鎮盱
 眙,道遇懷珍,以應對見重,取為驃騎長兼墨曹行參軍。尋除振武將軍、長廣太守。孝建初,為義恭大司馬參軍、直閣將軍。懷珍北州舊姓,門附殷積,啟上門生千人充宿衛。孝武大驚,召取青、冀豪家私附得數千人,土人怨之。隨府轉太宰參軍。大明二年,虜圍泗口城,青州刺史顏師伯請援。孝武遣懷珍將步騎數千赴之,於麋溝湖與虜戰,破七城。拜建武將軍、樂陵河間二郡太守,賜爵廣晉縣侯。明年,懷珍啟求還,孝武答曰:「邊維須才,未宜陳請。」竟陵王誕反,郡豪民王弼勸懷珍應之,懷珍斬弼以聞。孝武大喜,除豫章王子尚車騎參軍,加龍驤將軍。



 泰始初,除寧朔將軍、東安東莞二郡太守,率龍驤將軍王敬則、姜產步騎五千討壽陽。廬江太守王仲子南奔,賊遣偽廬江太守劉道蔚五千人頓建武澗,築三城。



 懷珍遣軍主段僧愛等馬步三百餘人掩擊斬之。引軍至晉熙,偽太守閻
 湛拒守,劉子勛遣將王仲虯步卒萬人救之,懷珍遣馬步三千人襲擊仲虯,大破之於莫邪山,遂進壽陽。又遣王敬則破殷琰將劉從等四壘於橫塘死虎,懷珍等乘勝逐北,頓壽春長邏門。宋明帝嘉其功,除羽林監、屯騎校尉,將軍如故。懷珍請先平賊,辭讓不受。



 建安王休仁濃湖與賊相持,久未決。明帝召懷珍還,拜前將軍,加輔國將軍,領軍向青山助擊劉胡,事平,除游擊將軍,輔國將軍如故。



 青州刺史沈文秀拒命,明帝遣其弟文炳宣喻,使懷珍領馬步三千人隨文炳俱行。



 未至,薛安都引虜,徐、兗已沒,張永、沈攸之於彭城大敗。敕懷珍步從盱眙自淮陰濟淮救永等,而官軍為虜所逐,相繼奔歸,懷珍乃還。三年春,敕懷珍權鎮山陽。



 先是明帝遣青州刺史明僧暠北征,僧暠遣將於王城築壘,以逼沈文秀,塹壁未立,為文秀所破,仍進攻僧暠。帝使懷珍率龍驤將軍王廣之五百
 騎,步卒二千人沿海救援,至東海,而僧暠已退保東萊。懷珍進據朐城,眾心忷懼,或欲且保郁州。懷珍謂眾曰:「卿等傳文秀厚賂胡師,規為外援,察其徒黨,何能必就左衽。齊士庶見於名義積葉,聲介一馳,東萊可飛書而下,何容阻軍緩邁止於此邪?」遂進至黔陬。



 偽高密、平昌二郡太守潰走,懷珍達朝廷意,送致文炳,文秀終不從命,焚燒郭邑。



 百姓聞懷珍至,皆喜。偽長廣太守劉桃根領數千人戍不其城,懷珍引軍次洋水。眾皆曰:「文秀今游騎滿境內,宜堅壁伺隙。」懷珍曰:「今眾少糧單,我懸彼固,政宜簡精銳,掩其不備耳。」遣王廣之將百騎襲陷其城,桃根走。偽東萊太守鞠延僧數百人據城,劫留高麗獻使。懷珍又遣寧朔將軍明慶符與廣之擊降延僧,遣高麗使詣京師。文秀聞諸城皆敗,乃遣使張靈碩請降,懷珍乃還。



 其秋,虜遂侵齊,圍歷城、梁鄒二城,游騎至東陽,擾動百姓。冀州
 刺史崔道固、兗州刺史劉休賓告急。休賓,懷珍從弟也。朝廷以懷珍為使持節、都督徐兗二州軍事、輔國將軍、平胡中郎將、徐州刺史,封艾縣侯,邑四百戶,督水步四十餘軍赴救。二城既沒,乃止。改授寧朔將軍、竟陵太守,轉巴陵王征西司馬,領南義陽太守。建平王景素為荊州,仍徙右軍司馬,遷南郡太守,加寧朔將軍。明帝手詔懷珍曰:「卿性忠讜,平所仗賴。在彼與年少共事,不可深存受益。景素兒乃佳,但不能接物,頗亦墮事,卿每諫之。」懷珍奉旨。帝寢疾,又詔懷珍曰:「卿不應乃作景素佐,才舊所寄,今徵卿參二衛直。」會帝崩,乃為安成王撫軍司馬,領南高平太守。



 朝廷疑桂陽王休範,中書舍人王道隆宣旨,以懷珍為冠軍將軍、豫章太守。懷珍曰:「休範須有禍萌,安敢便發,若終為寇,必請奉律吞之。今者賜使,恐成猜迫。」固請不就,乃除黃門郎,領虎賁中郎將、青州大中正。桂陽反,加
 懷珍前將軍,守石頭。為使持節、督豫司二州郢州之西陽軍事、冠軍將軍、豫州刺史。建平王景素反,懷珍遣子靈哲領兵赴京師。升明元年,進號征虜將軍。



 沈攸之在荊楚,朝議疑惑,懷珍遣冗從僕射張護使郢,致誠於世祖,並陳計策。



 及攸之起兵,眾謂當沿流直下,懷珍謂僚佐曰:「攸之矜躁夙著,虐加楚服,必當阻兵中流,聲劫幼主。不敢長驅決勝明矣。」遣子靈哲領馬步數千人衛京師。攸之遣使許天保說結懷珍,懷珍斬之,送首於太祖。太祖送示攸之。進號左將軍,徙封中宿縣侯,增邑六百戶。攸之圍郢城,懷珍遣建寧太守張謨、游擊將軍裴仲穆統蠻漢軍萬人出西陽,破賊前鋒公孫方平軍數千人,收其器甲。進平南將軍,增督南豫、北徐二州,增邑為千戶。



 初,孝武世,太祖為舍人,懷珍為直閣,相遇早舊。懷珍假還青州,上有白驄馬,嚙人,不可騎,送與懷珍別。懷珍報上百匹絹。或
 謂懷珍曰:「蕭君此馬不中騎,是以與君耳。君報百匹,不亦多乎?」懷珍曰:「蕭君局量堂堂,寧應負人此絹。吾方欲以身名託之,豈計錢物多少。」太祖輔政,以懷珍內資未多,二年冬,徵為都官尚書,領前軍將軍,以第四子寧朔將軍晃代為豫州刺史。或疑懷珍不受代,太祖曰:「我布衣時,懷珍便推懷投款,況在今日,寧當有異?」晃發經日,而疑論不止。上乃遣軍主房靈民領百騎追送晃,謂靈民曰:「論者謂懷珍必有異同,我期之有素,必不應爾。卿是其鄉里,故遣卿行,非唯衛新,亦以迎故也。」懷珍還,仍授相國右司馬。建元元年,轉左衛將軍,加給事中,改霄城侯,增邑二百戶。明年,加散騎常侍。虜寇淮、肥,以本官加平西將軍,假節,西屯巢湖,為壽春勢援,虜退乃還。



 懷珍年老,以禁旅辛勤,求為閑職,轉光祿大夫,常侍如故。其冬,虜寇朐山,授使持節、安北將軍,本官如故,領兵救授。未至,事寧,解安
 北、持節。



 四年,疾篤,上表解職,上優詔答許,別量所授。其夏,卒,年六十三。遺言薄葬。世祖追贈散騎常侍、鎮北將軍、雍州刺史,謚曰敬侯。



 子靈哲,字文明。解褐王國常侍、行參軍,尚書直郎,齊臺步兵校尉。建元初,歷寧朔將軍,臨川王前軍諮議,廬陵內史,齊郡太守,前軍將軍。靈哲所生母嘗病,靈哲躬自祈禱,夢見黃衣老公曰:「可取南山竹筍食之,疾立可愈。」靈哲驚覺,如言而疾瘳。嫡母崔氏及兄子景煥,泰始中沒虜,靈哲為布衣,不聽樂。及懷珍卒,當襲爵,靈哲固辭以兄子在虜中,存亡未測,無容越當茅土,朝廷義之。靈哲傾產私贖嫡母及景煥,累年不能得。世祖哀之,令北使告虜主,虜主送以還南,襲懷珍封爵。靈哲永明初歷護軍長史,東中郎諮議,領中直兵,出為寧朔將軍、巴西梓潼二郡太守,西陽王左軍司馬。隆昌元年,卒,年四十九。



 李安民,蘭陵承人也。祖嶷,衛軍參軍。父欽之,殿中將軍,補薛令。安民隨父之縣,元嘉二十七年沒虜,率部曲自拔南歸。太初逆,使安民領支軍。降義師,板建威將軍,補魯爽左軍。及爽反,安民遁還京師,除領軍行參軍,遷左衛殿中將軍。大明中,虜侵徐、兗,以安民為建威府司馬、無鹽令,除殿中將軍,領軍討漢川互螫賊。



 晉安王子勛反,明帝除安民武衛將軍、領水軍,補建安王司徒城局參軍,擊赭圻、湖白、荻浦、獺窟,皆捷,除積射將軍、軍主。張興世據錢溪,糧盡,為賊所逼。安民率舟乘數百,越賊五城,送米與興世。偽軍主沈仲、王張引軍自貴口欲斷江,安民進軍合戰破之。又擊鵲尾、江城,皆有功。事平,明帝大會新亭,勞接諸軍主,樗蒲官賭,安民五擲皆盧,帝大驚,目安民曰:「卿面方如田,封侯狀也。」



 安民少時貧窶,有一人從門過,相之曰:「君後當大富貴,與天子交手共戲。」至是安民尋
 此人,不知所在。從張永、沈攸之討薛安都於彭城,軍敗,安民在後拒戰,還保下邳。除寧朔將軍,戍淮陽城。論貴口功,封邵武縣子,食邑四百戶。復隨吳喜、沈攸之擊虜,達睢口,戰敗,還保宿豫。淮北既沒,明帝敕留安民戍角城。



 除寧朔將軍、冗從僕射。戍泗口,領舟軍緣淮游防,至壽春。虜遣偽長社公連營十餘里寇汝陰,豫州刺史劉勔擊退之。虜荊亭戍主升乞奴棄城歸降,安民率水軍攻前,破荊亭,絕其津逕。遷寧朔將軍、冠軍司馬、廣陵太守、行南兗州事。



 太祖在淮陰,安民遙相結事,明帝以為疑,徙安民為劉韞冠軍司馬、寧朔將軍、京兆太守,又除寧朔將軍、司州刺史,領義陽太守,並不拜,重除本職,又不拜,改授寧朔將軍、山陽太守。泰始末,淮北民起義欲南歸,以安民督前鋒軍事,又請援接,不克,還。除越騎校尉,復為寧朔將軍、山陽太守。三巴擾亂,太守張澹棄涪城走,以安民假節、
 都督討蜀軍事、輔師將軍。五獠亂漢中,敕安民回軍至魏興,事寧,還至夏口。



 元徽初,除督司州軍事、司州刺史,領義陽太守,假節、將軍如故。別敕安民曰:「九江須防,邊備宜重,今有此授,以增鄢郢之勢,無所致辭也。」及桂陽王休範起事,安民出頓,遣軍援京師。征授左將軍,加給事中。建平王景素作難,冠軍黃回、游擊將軍高道慶、輔國將軍曹欣之等皆密遣致誠,而游擊將軍高道慶領眾出討,太祖慮其有變,使安民及南豫州刺史段佛榮行以防之。安民至京口,破景素軍於葛橋。景素誅,留安民行南徐州事。城局參軍王迥素為安民所親,盜絹二匹,安民流涕謂之曰:「我與卿契闊備嘗,今日犯王法,此乃卿負我也。」於軍門斬之,厚為斂祭,軍府皆震服。授冠軍將軍,驍衛將軍,不拜。轉征虜將軍、東中郎司馬、行會稽郡事。



 安民將東,太祖與別宴語,淹留日夜。安民密陳宋運將盡,曆數有歸。
 蒼梧縱虐,太祖憂迫無計,安民白太祖欲於東奉江夏王躋起兵,太祖不許,乃止。蒼梧廢,太祖征安民為使持節督北討軍事、冠軍將軍、南兗州刺史。沈攸之反,太祖召安民以本官鎮白下,治城隍,加征虜將軍。進軍西討,又進前將軍。行至盆城,沈攸之平,仍授督郢州司州之義陽諸軍事、郢州刺史,持節、將軍如故。升明三年,遷左衛將軍,領衛尉。太祖即位,為中領軍,封康樂侯,邑千戶。



 宋泰始以來,內外頻有賊寇,將帥已下,各募部曲,屯聚京師。安民上表陳之,以為:「自非淮北常備,其外餘軍,悉皆輸遣。若親近宜立隨身者,聽限人數。」



 上納之,故詔斷眾募。時王敬則以勳誠見親,至於家國密事,上唯與安民論議,謂安民曰:「署事有卿名,我便不復細覽也。」尋為領軍將軍。



 虜寇壽春,至馬頭。詔安民出征,加鼓吹一部。虜退,安民沿淮進壽春。先是宋世亡命王元初聚黨六合山,僭號,自雲
 垂手過膝。州部討不能擒,積十餘年。安民遣軍偵候,生禽元初,斬建康市。加散騎常侍。其年,虜又南侵,詔安民持節履行緣淮清泗諸戍屯軍。虜攻朐山、連口、角城,安民頓泗口,分軍應赴。三年,引水步軍入清,於淮陽與虜戰,破之。虜退,安民知有伏兵,乃遣族弟馬軍主長文二百騎為前驅,自與軍副周盤龍、崔文仲系其後,分軍隱林。及長文至宿豫,虜見眾少,數千騎遮之。長文且退且戰,引賊向大軍,安民率盤龍等趨兵至,合戰於孫溪渚戰父彎側,虜軍大敗,赴清水死不可勝數。虜遣其菟頭公送攻車材至布丘,左軍將軍孫文顯擊破走之,燒其車材。



 淮北四州聞太祖受命,咸欲南歸。至是徐州人桓摽之、兗州人徐猛子等,合義眾數萬砦險求援。太祖詔曰:「青徐泗州,義舉雲集。安民可長轡遐馭,指授群帥。」



 安民赴救留遲,虜急兵攻摽之等皆沒,上甚責之。



 太祖崩,遺詔加侍中。世
 祖即位,遷撫軍將軍、丹陽尹。永明二年,遷尚書左僕射,將軍如故。安民時屢啟密謀見賞,又善結尚書令王儉,故世傳儉啟有此授。



 尋上表以年疾求退,改授散騎常侍、金紫光祿大夫,將軍如故。四年,為安東將軍、吳興太守,常侍如故。卒官,年五十八。賻錢十萬,布百匹。



 吳興有項羽神護郡聽事,太守不得上。太守到郡,必須祀以軛下牛。安民奉佛法,不與神牛,著屐上聽事。又於聽上設八關齋。俄而牛死,葬廟側,今呼為「李公牛塚」。及安民卒,世以神為祟。詔曰:「安民歷位內外,庸績顯著。忠亮之誠,每簡朕心。敷政近畿,方申任寄。奄至殞喪,痛傷於懷。贈鎮東將軍,鼓吹一部,常侍、太守如故,謚曰肅侯。」



 王玄載,字彥休,下邳人也。祖宰,偽北地太守。父蕤,東莞太守。玄載解褐江夏王國侍郎、太宰行參軍。泰始初,為長水校尉。隨張永征
 彭城,臺軍大敗,玄載全軍據下邳城拒虜,假冠軍將軍。官軍新敗,人情恐駭,以玄載士望,板為徐州刺史、持節、監徐州豫州梁郡軍事、寧朔將軍、平胡中郎將,尋又領山陽、東海二郡太守。五年,督青、兗二州刺史,將軍、東海郡如故。七年,復為徐州,督徐兗二州、鐘離太守,將軍、郎將如故。遷左軍將軍。仍為寧朔將軍、歷陽太守,改持節、都督二豫、冠軍將軍、南豫州刺史,太守如故。遷撫軍司馬。出為持節、督梁南北秦三州軍事、冠軍將軍、西戎校尉、梁秦二州刺史。進號征虜將軍。尋徙督益寧二州、益州刺史、建寧太守,將軍、持節如故。



 沈攸之之難,玄載起義送誠,進號後軍將軍,封鄂縣子。征散騎常侍,領後軍,未拜,建元元年,為左民尚書,鄂縣子如故。會虜動,南兗州刺史王敬則奔京師,上遣玄載領廣陵,加平北將軍、假節、行南兗州事,本官如故。事寧,為光祿大夫、員外散騎常侍。永明四年,
 為持節監兗州緣淮諸軍事、平北將軍、兗州刺史。六年,卒,時年七十六。謚烈子。



 玄載夷雅好玄言,修士操,在梁益有清績,西州至今思之。



 從弟玄謨子瞻,宋明帝世為黃門郎,素輕世祖。世祖時在大床寢,瞻謂豫章王曰:「帳中物亦復隨人寢興。』世祖銜之,未嘗形色。建元元年,為冠軍將軍、永嘉太守,詣闕跪拜不如儀,為守寺所列。有司以啟世祖,世祖召瞻入東宮,仍送付廷尉殺之。遣左右口啟上曰:「父辱子死,王瞻傲慢朝廷,臣輒以收治。」太祖曰:「語郎,此何足計!」既聞瞻已死,乃默無言。



 瞻兄寬,宋世與瞻並為方伯,至是瞻雖坐事,而寬位待如舊也。寬泰始初為隨郡,值西方反,父玄謨在都,寬棄郡歸,明帝加賞,使隨張永討薛安都。寬辭以母猶存,在西為賊所執,請得西行。遂襲破隨郡,斬偽太守劉師念,拔其母。事平,明帝嘉之,使圖畫寬形。建元初,為散騎常侍、光祿大夫,領前軍將軍。
 永明元年,為太常。坐於宅殺牛,免官。後為光祿大夫。三年,卒。



 玄載弟玄邈,字彥遠。初為驃騎行軍參軍,太子左積弩將軍,射聲校尉。泰始初,遷輔國將軍、清河廣川二郡太守,幽州刺史。青州刺史沈文秀反,玄邈欲向朝廷,慮見掩襲,乃詣文秀求安軍頓。文秀令頓城外。玄邈即立營壘,至夜拔軍南奔赴義。比曉,文秀追不復及。明帝以為持節、都督青州、青州刺史,將軍如故。



 太祖鎮淮陰,為帝所疑,遣書結玄邈。玄邈長史房叔安勸玄邈不相答和。罷州還,太祖以經途令人要之,玄邈雖許,既而嚴軍直過,還都啟帝,稱太祖有異謀,太祖不恨也。昇明中,太祖引為驃騎司馬、冠軍將軍、太山太守,玄邈甚懼,而太祖待之如初。遷散騎常侍、驍騎將軍,冠軍如故。出為持節、都督梁南秦二州軍事、征虜將軍、西戎校尉、梁南秦二州刺史,兄弟同時為方伯。封河陽縣侯。建元元年,進號右將軍,
 侯如故。



 亡命李烏奴作亂梁部,陷白馬戍。玄邈率東從七八百人討之,不克,慮不自保,乃使人偽降烏奴,告之曰:「王使君兵眾羸弱,棄伎妾於城內,攜愛妾二人去已數日矣。」烏奴喜,輕兵襲州城,玄邈設伏擊破之,烏奴挺身走。太祖聞之,曰:「玄邈果不負吾意遇也。」還為征虜將軍、長沙王後軍司馬、南東海太守。遷都官尚書。



 世祖即位,轉右將軍、豫章王太尉司馬,出為冠軍將軍、臨川內史,秩中二千石。還為前軍司徒司馬、散騎常侍、太子右率。永明七年,為持節、都督兗州緣淮軍事、平北將軍、兗州刺史,未之任,轉大司馬,加後將軍。八年,轉太常,遷散騎常侍、右衛將軍,出為持節、監徐州軍事、平北將軍、徐州刺史。十一年,建康蓮華寺道人釋法智與州民周盤龍等作亂,四百人夜攻州城西門,登梯上城,射殺城局參軍唐潁,遂入城內。軍主耿虎、徐思慶、董文定等拒戰,至曉,玄邈率百
 餘人登城便門,奮擊,生擒法智、盤龍等。玄邈坐免官。鬱林即位,授撫軍將軍,遷使持節、安西將軍、歷陽南譙二郡太守。延興元年,加散騎常侍,尋轉中護軍。



 高宗使玄邈往江州殺晉安王子懋,玄邈苦辭不行,及遣王廣之往廣陵取安陸王子敬,玄邈不得已奉旨。給鼓吹置佐。建武元年,遷持節、都督南兗兗徐青冀五州軍事、平北將軍、南兗州刺史,轉護軍將軍,加散騎常侍。四年,卒,年七十二。



 贈安北將軍、雍州刺史。謚曰壯侯。



 同族王文和,宋鎮北大將軍仲德兄孫也。景和中,為義陽王昶征北府主簿。昶於彭城奔虜,部曲皆散,文和獨送至界上。昶謂之曰:「諸人皆去,卿有老母,何不去邪!」文和乃去。昇明中,為巴陵內史。沈攸之事起,文和斬其使,馳白世祖告變,棄郡奔郢城。永明中,歷青、冀、兗、益四州刺史,平北將軍。



 史臣曰:宋氏將季,離亂日兆,家懷逐鹿,人有異圖。故蕃岳阻兵
 之機,州郡觀釁之會。此數子皆宿將舊勛,與太祖比肩為方伯,年位高下,或為先輩,而薦誠君側,奉義萬里。以此知樂推之非妄,信民心之有歸。玄載兄弟門從,世秉誠烈,不為道家所忌,斯今之耿氏也。



 贊曰:霄城報馬,分義先推。靈哲守讓,方軌丁、韋。李佐東土,謀發天機。



 王為清政,其風不衰。玄邈簡朕,早背同歸。



\end{pinyinscope}