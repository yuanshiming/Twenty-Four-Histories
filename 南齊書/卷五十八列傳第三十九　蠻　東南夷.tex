\article{卷五十八列傳第三十九 蠻 東南夷}

\begin{pinyinscope}

 蠻,種類繁多,言語不一,咸依山谷,布荊、湘、雍、郢、司等五州界。宋世封西陽蠻梅蟲生為高山侯,田治生為威山侯,梅加羊為扞山侯。太祖即位,有司奏蠻封應在解例,參議以「戎夷疏爵,理章列代;酋豪世襲,事炳前葉。今宸曆改物,舊冊杓降,而梅生等保落奉政,事須繩總,恩命升贊,有異常品。謂宜存名以訓殊俗」。詔特留。以治生為輔國將軍、虎賁中郎,轉建寧郡太守,將軍、侯如故。



 建元二年,虜侵豫、司,蠻中傳虜已近,又聞官盡發民丁,南襄城蠻秦遠以郡縣無備,寇潼陽,縣令焦文度戰死。司州蠻引虜攻平昌戍,戍主茍元
 賓擊破之。秦遠又出破臨沮百方砦,殺略百餘人。北上黃蠻文勉德寇汶陽,太守戴元孫孤城力弱,慮不自保,棄戍歸江陵。荊州刺史豫章王遣中兵參軍劉伾緒領千人討勉德,至當陽,勉德請降,收其部落,使戍汶陽所治城子,令保持商旅,付其清通,遠遂逃竄。



 汶陽本臨沮西界,二百里中,水陸迂狹,魚貫而行,有數處不通騎,而水白田甚肥腴。桓溫時割以為郡。西北接梁州新城,東北接南襄城,南接巴、巫二邊,並山蠻凶盛,據險為寇賊。宋泰始以來,巴建蠻向宗頭反,刺史沈攸之斷其鹽米,連討不剋。晉太興三年,建平夷王向弘、向WA等詣臺求拜除,尚書郎張亮議「夷貊不可假以軍號」,元帝詔特以弘為折沖將軍、當平鄉侯,並親晉王,賜以朝服。宗頭其後也。太祖置巴州以威靜之。



 其武陵酉溪蠻田思飄寇抄,內史王文和討之,引軍深入,蠻自後斷其糧。豫章王遣中兵參軍莊
 明五百人將湘州鎮兵合千人救之,思飄與文和拒戰,中弩矢死,蠻眾以城降。



 永明初,向宗頭與黔陽蠻田豆渠等五千人為寇,巴東太守王圖南遣府司馬劉僧壽等斬山開道,攻其砦,宗頭夜燒砦退走。



 三年,湘川蠻陳雙、李答寇掠郡縣,刺史呂安國討之不克。四年,刺史柳世隆督眾征討,乃平。



 五年,雍、司州蠻與虜通,助荒人桓天生為亂。



 六年,除督護北遂安左郡太守田駟路為試守北遂安左郡太守,前寧朔將軍田驢王為試守新平左郡太守,皆郢州蠻也。



 九年,安隆內史王僧旭發民丁,遣寬城戍主萬民和助八百丁村蠻伐千二百丁村蠻,為蠻所敗,民和被傷,失馬及器仗,有司奏免官。



 西陽蠻田益宗,沈攸之時以功勞得將領,遂為臨川王防閣,叛投虜,虜以為東豫州刺史。建武三年,虜遣益宗攻司州龍城戍,為戍主朱僧起所破。



 蠻俗衣布徒跣,或椎髻,或剪髮。兵器以金
 銀為飾,虎皮衣楯,便弩射,皆暴悍好寇賊焉。



 東夷高麗國,西與魏虜接界。宋末,高麗王樂浪公高璉為使持節、散騎常侍、都督營平二州諸軍事、車騎大將軍、開府儀同三司。太祖建元元年,進號驃騎大將軍。三年,遣使貢獻,乘舶泛海,使驛常通,亦使魏虜,然彊盛不受制。



 虜置諸國使邸,齊使第一,高麗次之。永明七年,平南參軍顏幼明、冗從僕射劉思敩使虜。虜元會,與高麗使相次。幼明謂偽主客郎裴叔令曰:「我等銜命上華,來造卿國。所為抗敵,在乎一魏。自餘外夷,理不得望我鑣塵。況東夷小貊,臣屬朝廷,今日乃敢與我躡踵。」思籞謂偽南部尚書李思沖曰:「我聖朝處魏使,未嘗與小國列,卿亦應知。」思沖曰:「實如此。但主副不得升殿耳。此間坐起甚高,足以相報。」思敩曰:「李道固昔使,正以衣冠致隔耳。魏國必纓冕而至,豈容見黜。」幼明又謂虜主
 曰:「二國相亞,唯齊與魏。邊境小狄,敢躡臣蹤!」



 高麗俗服窮褲,冠折風一梁,謂之幘。知讀《五經》。使人在京師,中書郎王融戲之曰:「服之不衷,身之災也。頭上定是何物?」答曰:「此即古弁之遺像也。」



 高璉年百餘歲卒。隆昌元年,以高麗王樂浪公高雲為使持節、散騎常侍、都督營平二州諸軍事、征東大將軍、高麗王、樂浪公。建武三年,此下缺文報功勞勤,實存
 名烈。假行寧朔將軍臣姐瑾等四人,振竭忠效,攘除國難,志勇果毅,等威名將,可謂扞城,固蕃社稷,論功料勤,宜在甄顯。今依例輒假行職。伏願恩愍,聽除所假。寧朔將軍、面中王姐瑾,歷贊時務,武功并列,今假行冠軍將軍、都將軍、都漢王。建威將軍、八中侯餘古,弱冠輔佐,忠效夙著,今假行寧朔將軍、阿錯王。建威將軍餘歷,忠款有素,文武烈顯,今假行龍驤將軍、邁盧王。廣武將軍餘固,忠效時務,光宣國
 政,今假行建威將軍、弗斯侯。



 牟大又表曰:「臣所遣行建威將軍、廣陽太守、兼長史臣高達,行建威將軍、朝鮮太守、兼司馬臣楊茂,行宣威將軍、兼參軍臣會邁等三人,志行清亮,忠款夙著。往泰始中,比使宋朝,今任臣使,冒涉波險,尋其至效,宜在進爵,謹依先例,各假行職。且玄澤靈休,萬里所企,況親趾天庭,乃不蒙賴。伏願天監特愍除正。



 達邊效夙著,勤勞公務,今假行龍驤將軍、帶方太守。茂志行清壹,公務不廢,今假行建威將軍、廣陵太守。邁執志周密,屢致勤效,今假行廣武將軍、清河太守。」



 詔可,並賜軍號,除太守。為使持節、都督百濟諸軍事、鎮東大將軍。使兼竭者僕射孫副策命大襲亡祖父牟都為百濟王。曰:「於戲!惟爾世襲忠勤,誠著遐表,滄路肅澄,要貢無替。式循彞典,用纂顯命。往欽哉!其敬膺休業,可不慎歟!制詔行都督百濟諸軍事、鎮東大將軍百濟王牟大今以大襲
 祖父牟都為百濟王,即位章綬等玉銅虎竹符四。王其拜受,不亦休乎!」



 是歲,魏虜又發騎數十萬攻百濟,入其界,牟大遣將沙法名、贊首流、解禮昆、木干那率眾襲擊虜軍,大破之。建武二年,牟大遣使上表曰:「臣自昔受封,世被朝榮,忝荷節鉞,剋攘列辟。往姐瑾等並蒙光除,臣庶咸泰。去庚午年,獫狁弗悛,舉兵深逼。臣遣沙法名等領軍逆討,宵襲霆擊,匈梨張惶,崩若海蕩。乘奔追斬,僵尸丹野。由是摧其銳氣,鯨暴韜兇。今邦宇謐靜,實名等之略;尋其功勳,宜在褒顯。今假沙法名行征虜將軍、邁羅王,贊首流為行安國將軍、辟中王,解禮昆為行武威將軍、弗中侯,木干那前有軍功,又拔臺舫,為行廣威將軍、面中侯。伏願天恩特愍聽除。」又表曰:「臣所遣行龍驤將軍、樂浪太守兼長史臣慕遺,行建武將軍、城陽太守兼司馬臣王茂,兼參軍、行振武將軍、朝鮮太守臣張塞,行揚武將軍陳明,
 在官忘私,唯公是務,見危授命,蹈難弗顧。今任臣使,冒涉波險,盡其至誠。實宜進爵,各假行署。伏願聖朝特賜除正。」詔可,並賜軍號。



 加羅國,三韓種也。建元元年,國王荷知使來獻。詔曰:「量廣始登,遠夷洽化。加羅王荷知款關海外,奉贄東遐。可授輔國將軍、本國王。」



 倭國,在帶方東南大海島中,漢末以來,立女王。土俗已見前史。建元元年,進新除使持節、都督倭·新羅·任那·加羅·秦韓·慕韓六國諸軍事、安東大將軍、倭王武號為鎮東大將軍。



 南夷林邑國,在交州南,海行三千里,北連九德,秦時故林邑縣也。漢末稱王。



 晉太康五年始貢獻。宋永初元年,林邑王范楊邁初產,母夢人以金席藉之,光色奇麗。中國謂紫磨金,夷人謂之「楊邁」,故以為名。楊邁死,子咄立,慕其父,復改名楊邁。



 林邑有金山,金汁流出於浦。事尼乾道,鑄金銀人像,大十圍。元嘉二十二年,交州刺史
 檀和之伐林邑,楊邁欲輸金萬斤,銀十萬斤,銅三十萬斤,還日南地。大臣昪僧達諫,不聽。和之進兵破其北界犬戎區慄城,獲金寶無算,毀其金人,得黃金數萬斤,餘物稱是。和之後病死,見胡神為祟。孝建二年,始以林邑長史範龍跋為揚武將軍。



 楊邁子孫相傳為王,未有位號。夷人范當根純攻奪其國,篡立為王。永明九年,遣使貢獻金簟等物。詔曰:「林邑雖介在遐外,世服王化。當根純乃誠懇款到,率其僚職,遠績克宣,良有可嘉。宜沾爵號,以弘休澤。可持節、都督緣海諸軍事、安南將軍、林邑王。」範楊邁子孫範諸農率種人攻當根純,復得本國。十年,以諸農為持節、都督緣海諸軍事、安南將軍、林邑王。建武二年,進號鎮南將軍。永泰元年,諸農入朝,海中遭風溺死,以其子文款為假節、都督緣海軍事、安南將軍、林邑王。



 晉建興中,日南夷帥範稚奴文數商賈,見上國制度,教林邑王
 範逸起城池樓殿。



 王服天冠如佛冠,身被香纓絡。國人凶悍,習山川,善鬥。吹海蠡為角。人皆裸露。



 四時暄暖,無霜雪。貴女賤男,謂師君為婆羅門。群從相姻通,婦先遣娉求婿。女嫁者,迦藍衣橫幅合縫如井闌,首戴花寶。婆羅門牽婿與婦握手相付,咒願吉利。



 居喪剪髮,謂之孝。燔尸中野以為葬。遠界有靈鷲鳥,知人將死,集其家食死人肉盡,飛去,乃取骨燒灰投海中水葬。人色以黑為美,南方諸國皆然。區慄城建八尺表。日影度南八寸。



 自林邑西南三千餘里,至扶南。



 扶南國,在日南之南大海西蠻灣中,廣袤三千餘里,有大江水西流入海。其先有女人為王,名柳葉。又有激國人混填,夢神賜弓一張,教乘舶入海。混填晨起於神廟樹下得弓,即乘舶向扶南。柳葉見舶,率眾欲禦之。混填舉弓遙射,貫船一面通中人。柳葉怖,遂降。混
 填娶以為妻。惡其裸露形體,乃疊布貫其首。遂治其國,子孫相傳。至王槃況死,國人立其大將范師蔓。蔓病,姊子旃篡立,殺蔓子金生。



 十餘年,蔓少子長襲殺旃,以刃鑱旃腹曰:「汝昔殺我兄,今為父兄報汝。」旃大將范尋又殺長,國人立以為王,是吳、晉時也。晉、宋世通職貢。



 宋末,扶南王姓僑陳如,名朅耶跋摩,遣商貨至廣州。天竺道人那伽仙附載欲歸國,遭風至林邑,掠其財物皆盡。那伽仙間道得達扶南,具說中國有聖主受命。



 永明二年,朅耶跋摩遣天竺道人釋那伽仙上表稱扶南國王臣僑陳如朅耶跋摩叩頭啟曰:「天化撫育,感動靈祇,四氣調適。伏願聖主尊體起居康豫,皇太子萬福,六宮清休,諸王妃主、內外朝臣普同和睦,鄰境士庶萬國歸心,五穀豐熟,災害不生,土清民泰,一切安穩。臣及人民,國土豐樂,四氣調和,道俗濟濟,並蒙陛下光化所被,咸荷安泰。」又曰:「臣前遣
 使齎雜物行廣州貨易,天竺道人釋那伽仙於廣州因附臣舶欲來扶南,海中風漂到林邑,國王奪臣貨易,并那伽仙私財。具陳其從中國來此,仰序陛下聖德仁冶,詳議風化。佛法興顯,從僧殷集,法事日盛,王威嚴整,朝望國軌,慈愍蒼生,八方六合,莫不歸伏。如聽其所說,則化鄰諸天,非可為喻。臣聞之,下情踴悅,若暫奉見尊足,仰慕慈恩,澤流小國,天垂所感,率土之民,並得皆蒙恩祐。是以臣今遣此道人釋那伽仙為使,上表問訊奉貢,微獻呈臣等赤心,并別陳下情。但所獻輕陋,愧懼唯深。伏願天慈曲照,鑒其丹款,賜不垂責。」又曰:「臣有奴名鳩酬羅,委臣逸走,別在餘處,構結凶逆,遂破林邑,仍自立為王。永不恭從,違恩負義,叛主之愆,天不容載。伏尋林邑昔為檀和之所破,久已歸化。天威所被,四海彌伏,而今鳩酬羅守執奴凶,自專很彊。且林邑、扶南鄰界相接,親又是臣奴,猶
 尚逆去,朝廷遙遠,豈復遵奉。此國屬陛下,故謹具上啟。伏聞林邑頃年表獻簡絕,便欲永隔朝廷。豈有師子坐而安大鼠。伏願遣軍將伐兇逆,臣亦自效微誠,助朝廷剪撲,使邊海諸國,一時歸伏。陛下若欲別立餘人為彼王者,伏聽敕旨。脫未欲灼然興兵伐林邑者,伏願特賜敕在所,隨宜以少軍助臣,乘天之威,殄滅小賊,伐惡從善。平蕩之日,上表獻金五婆羅。今輕此使送臣丹誠,表所陳啟,不盡下情。謹附那伽仙并其伴口具啟聞。伏願愍所啟。并獻金鏤龍王坐像一軀,白檀像一軀,牙塔二軀,古具二雙,璢璃蘇鉝二口,瑇瑁檳榔柈一枚。」



 那伽仙詣京師,言其國俗事摩肸首羅天神,神常降於摩耽山。土氣恆暖,草木不落。其上書曰:「吉祥利世間,感攝於群生。所以其然者,天感化緣明。仙山名摩柷,吉樹敷嘉榮。摩寔首羅天,依此降尊靈。國土悉蒙祐,人民皆安寧。由斯恩被故,是以
 臣歸情。菩薩行忍慈,本跡起凡基。一發菩提心,二乘非所期,歷生積功業,六度行大悲。勇猛超劫數,財命舍無遺。生死不為厭,六道化有緣。具修於十地,遺果度人天。功業既已定,行滿登正覺。萬善智圓備,惠日照塵俗。眾生感緣應,隨機授法藥。佛化遍十方,無不蒙濟擢。皇帝聖弘道,興隆於三寶。垂心覽萬機,威恩振八表。國土及城邑,仁風化清皎。亦如釋提洹,眾天中最超。陛下臨萬民,四海共歸心。聖慈流無疆,被臣小國深。」詔報曰:「具摩肸降靈,流施彼土,雖殊俗異化,遙深欣讚。知鳩酬羅於彼背叛,竊據林邑,聚兇肆掠,殊宜剪討。



 彼雖介遐陬,舊修蕃貢,自宋季多難,海譯致壅,皇化惟新,習迷未革。朕方以文德來遠人,未欲便興干戈。王既敔列忠到,遠請軍威,今詔交部隨宜應接。伐叛柔服,實惟國典,勉立殊效,以副所期。那伽仙屢銜邊譯,頗悉中土闊狹,令其具宣。」



 上報以絳紫
 地黃碧綠紋綾各五匹。



 扶南人黠惠知巧,攻略傍邑不賓之民為奴婢,貨易金銀彩帛。大家男子截錦為橫幅,女為貫頭,貧者以布自蔽,鍛金環鏆銀食器。伐木起屋,國王居重閣,以木柵為城。海邊生大箬葉,長八九尺,編其葉以覆屋。人民亦為閣居。為船八九丈,廣裁六七尺,頭尾似魚。國王行乘象,婦人亦能乘象。鬥雞及犬希為樂。無牢獄,有訟者,則以金指環若雞子投沸湯中,令探之,又燒鎖令赤,著手上捧行七步,有罪者手皆燋爛,無罪者不傷。又令沒水,直者入即不沈,不直者即沈也。有甘蔗、諸蔗、安石榴及橘,多檳榔,鳥獸如中國。人性善,不便戰,常為林邑所侵擊,不得與交州通,故其使罕至。



 交州斗絕海島,控帶外國,故恃險數不賓。宋泰始初,刺史張牧卒,交趾人李長仁殺牧北來部曲,據交州叛。數年病死。從弟叔獻嗣事,號令未行,遣使求刺史。



 宋朝以南海太守沈煥為交
 州刺史,以叔獻為煥寧遠司馬、武平新昌二郡太守。叔獻得朝命,人情服從,遂發兵守險不納煥,煥停鬱林病卒。太祖建元元年,仍以叔獻為交州刺史,就安慰之。叔獻受命,既而斷割外國,貢獻寡少。世祖欲討之,永明三年,以司農劉楷為交州刺史,發南康、廬陵、始興郡兵征交州。叔獻聞之,遣使願更申數年,獻十二隊純銀兜鍪及孔雀毦,世祖不許。叔獻懼為楷所襲,間道自湘川還朝。



 六年,以始興太守房法乘代楷。法乘至鎮,屬疾不理事,專好讀書。長史伏登之因此擅權,改易將吏,不令法乘知。錄事房季文白之,法乘大怒,系登之於獄。



 十餘日,登之厚賂法乘妹夫崔景叔得出,將部曲襲州執法乘,謂之曰:「使君既有疾,不宜勞。」囚之別室。法乘無事,復就登之求書讀,登之曰:「使君靜處猶恐動疾,豈可看書。」遂不與。乃啟法乘心疾動,不任視事,世祖仍以登之為交州刺史。法乘還
 至嶺而卒。法乘,清河人。昇明中為太祖驃騎中兵,至左中郎將。性方簡,身長八尺三寸,行出人上,常自俯屈。青州刺史明慶符亦長與法乘等,朝廷唯此二人。



 史臣曰:書稱蠻夷猾夏,蓋總而為言矣。至於南夷雜種,分嶼建國,四方珍怪,莫此為先。藏山隱海,環寶溢目。商舶遠屆,委輸南州,故交、廣富實,牛刃積王府。充斥之事差微,聲教之道可被。若夫用德以懷遠,其在此乎?



 贊曰:司、雍分疆,荊及衡陽。參錯州部,地有蠻方。東夷海外,碣石、扶桑。



 南域憬遠,極泛溟滄。非要乃貢,並亦來王。



\end{pinyinscope}