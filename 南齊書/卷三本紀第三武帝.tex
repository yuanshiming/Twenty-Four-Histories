\article{卷三本紀第三武帝}

\begin{pinyinscope}

 世祖武皇帝諱賾,字宣遠,太祖長子也。小諱龍兒。生於建康青溪宅,其夜陳孝後、劉昭後同夢龍據屋上,故字上焉。初為尋陽國侍郎,辟州西曹書佐,出為贛令。江州刺史晉安王子勛反,上不從命,南康相沈肅之縶上於郡獄。族人蕭欣祖、門客桓康等破郡迎出上。肅之率將吏數百人追擊,上與左右拒戰,生獲肅之,斬首百餘級,遂率部曲百餘人舉義兵。始興相殷孚將萬兵赴子勛於尋陽,或勸上擊之,上以眾寡不敵,避屯揭陽山中,聚眾至三千人。子勛遣其將戴凱之為南康相,及軍主張宗之千餘人助之。上引兵向郡,擊凱之別軍主程超數百人於南康口,又進擊宗之,破斬
 之,遂圍郡城。凱之以數千人固守,上親率將士盡日攻之,城陷,凱之奔走,殺偽贛令陶沖之。上即據郡城,遣軍主張應期、鄧惠真三千人襲豫章。子勛遣軍主談秀之等七千人,與應期相拒於西昌,築營壘,交戰不能決。聞上將自下,秀之等退散。事平,徵為尚書庫部郎、征北中兵參軍、西陽縣子,帶南東莞太守、越騎校尉、正員郎、劉韞撫軍長史、襄陽太守。別封贛縣子,邑三百戶,固辭不受。



 轉寧朔將軍、廣興相。桂陽王休範反,上遣軍襲尋陽,至北嶠,事平,除晉熙王安西諮議,不拜,復還郡。轉司徒右長史、黃門郎。沈攸之在荊楚,宋朝密為之備。



 元徽四年,以上為晉熙王鎮西長史、江夏內史、行郢州事。順帝立,徵晉熙王燮為撫軍、揚州刺史,以上為左衛將軍,輔燮俱下。沈攸之事起,未得朝廷處分,上以中流可以待敵,即據盆口城為戰守之備。太祖聞之,喜曰:「此真我子也!」上表求西討,不許,乃
 遣偏軍援郢。平西將軍黃回等皆受上節度。加上冠軍將軍、持節。



 升明二年,事平,轉散騎常侍,都督江州、豫州之新蔡、晉熙二郡軍事,征虜將軍,江州刺史,持節如故。封聞喜縣侯,邑二千戶。其年,徵侍中、領軍將軍。給鼓吹一部。府置佐史。領石頭戍軍事。尋又加持節、督京畿諸軍事。三年,轉散騎常侍、尚書僕射、中軍大將軍、開府儀同三司,進爵為公,持節、都督、領軍如故。給班劍二十人。齊國建,為齊公世子,改加侍中、南豫州刺史,給油絡車,羽葆鼓吹,增班劍為四十人。以石頭為世子宮,官置二率以下,坊省服章,一如東宮。進爵王太子。太祖即位,為皇太子。



 建元四年三月,壬戌,太祖崩,上即位,大赦。征鎮州郡令長軍屯營部,各行喪三日,不得擅離任,都邑城守防備幢隊,一不得還。乙丑,稱先帝遺詔,以司徒褚淵錄尚書事,尚書左僕射王儉為尚書令,車騎將軍張敬兒為開府儀同
 三司。詔曰:「喪禮雖有定制,先旨每存簡約,內官可三日一還臨,外官間日一還臨。後有大喪皆如之。」丁卯,以右衛將軍呂安國為司州刺史。庚午,以司空豫章王嶷為太尉。



 癸酉,詔曰:「城直之制,歷代宜同,頃歲逋弛,遂以萬計。雖在憲宜懲,而原心可亮。積年逋城,可悉原蕩。自茲以後,申明舊科,有違糾裁。」庚辰,詔曰:「比歲未稔,貧窮不少,京師二岸,多有其弊。遣中書舍人優量賑恤。」夏,四月,丙午,以輔國將軍張倪為兗州刺史。辛卯,追尊穆妃為皇后。



 五月,乙丑,以丹陽尹聞喜公子良為南徐州刺史。甲戌,以新除左衛將軍垣崇祖為豫州刺史。癸未,詔曰:「頃水雨頻降,潮流薦滿,二岸居民,多所淹漬。遣中書舍人與兩縣官長優量賑恤。」六月,甲申,立皇太子長懋。詔申壬戌赦恩百日。



 乙酉,以鄱陽王鏘為雍州刺史,臨汝公子卿為郢州刺史。甲午,以寧朔將軍臧靈智為越州刺史。丙申,
 立皇太子妃王氏。進封聞喜公子良為竟陵王,臨汝公子卿為廬陵王,應城公子敬為安王,江陵公子懋為晉安王,枝江公子隆為隨郡王,皇子子真為建安王,皇孫昭業為南郡王。戊戌,詔曰:「水潦為患,星緯乖序。京都囚繫,可克日訊決;諸遠獄委刺史以時察判。建康、秣陵二縣貧民加賑賜,必令周悉。吳興、義興遭水縣,蠲除租調。」癸卯,以司徒褚淵為司空、驃騎將軍。秋,七月,庚申,以衛尉蕭順之為豫州刺史。壬戌,以冠軍將軍垣榮祖為青、冀二州刺史。



 八月,癸卯,司徒褚淵薨。九月,丁巳,以國哀故,罷國子學。己巳,以前軍將軍姜伯起為秦州刺史。辛未,以征南將軍王僧虔為左光祿大夫、開府儀同三司,尚書右僕射王奐為湘州刺史。冬,十二月,己丑,詔曰:「緣淮戍將,久處邊勞,三元行始,宜沾恩慶。可遣中書舍人宣旨臨會。後每歲皆如之。」庚子,以太子左衛率戴僧靜為徐州刺史。



 永明元年春,正月,辛亥,車駕祠南郊,大赦,改元。壬子,詔內外群僚各舉朕違,肆心規諫。又詔王公卿士,各舉所知,隨方登敘。詔曰:「經邦之寄,實資蒞民,守宰祿俸,蓋有恆準。往以邊虞告警,故沿時損益;今區宇寧晏,庶績咸熙,念勤簡能,宜加優獎。郡縣丞尉,可還田秩。」太尉豫章王嶷領太子太傅,護軍將軍長沙王晃為南徐州刺史,鎮北將軍竟陵王子良為南兗州刺史。庚申,以侍中蕭景先為中領軍。壬戌,立皇弟銳為南平王,鏗為宜都王,皇子子明為武昌王,子罕為南海王。甲子,為築青溪舊宮,詔槊仗瞻履。



 二月,辛巳,以征虜將軍楊炅為沙州刺史。辛丑,以隴西公宕昌王梁彌機為河、涼二州刺史,東羌王像舒彭為西涼州刺史。三月,癸丑,詔曰:「宋德將季,風軌陵遲,列宰庶邦,彌失其序,遷謝遄速,公私凋弊。泰運初基,草昧惟始,思述先範,永隆治根。蒞民之職,一以小滿為限。其有
 聲績剋舉,厚加甄異;理務無庸,隨時代黜。」丙辰,詔曰:「朕自丁荼毒,奄便周忌,瞻言負荷,若墜淵壑。而遠圖尚蔽,政刑未理,星緯失序,陰陽愆度。思播先澤,兼酬天眚,可申辛亥赦恩五十日,以期訖為始。京師囚繫,悉皆原宥。三署軍徒,優量降遣。都邑鰥寡尤貧,詳加賑恤。」戊寅,詔「四方見囚,罪無輕重,及劫賊餘口長徒敕系,悉原赦。逋負督贓,建元四年三月以前,皆特除。」夏,四月,壬午,詔曰:「魏矜袁紹,恩洽丘墓;晉亮兩王,榮覃餘裔。二代弘義,前載美談。袁粲、劉秉與先朝同獎宋室,沈攸之於景和之世,特有乃心,雖末節不終,而始誠可錄。歲月彌往,宜特優降。



 粲、秉前年改葬塋兆,未脩材槨,可為經理,令粗足周禮。攸之及其諸子喪柩在西者,可符荊州送反舊墓,在所為營葬事。」五月,丁酉,車騎將軍張敬兒伏誅。



 六月,丙寅,詔「凡坐事應覆治者,在建元四年三月已前,皆原宥。」秋,七月,戊戌,
 新除左光祿大夫王僧虔加特進。九月,己卯,以荊州刺史臨川王映為驃騎將軍,冠軍將軍廬陵王子卿為荊州刺史,吳郡太守安陸侯緬為郢州刺史。



 二年春,正月,乙亥,以司州刺史呂安國為南兗州刺史,征北將軍竟陵王子良為護軍將軍兼司徒,征北長史劉悛為司州刺史。丙子,以右光祿大夫王延之為特進。



 三月,乙亥,以吳興太守張岱為南兗州刺史,前將軍王奐為江州刺史,平北將軍呂安國為湘州刺史。戊寅,以少府趙景翼為廣州刺史。夏,四月,甲辰,詔「揚、南徐、南兗、徐、兗五州統內諸獄,并、豫、江三州府州見囚,江州尋陽、新蔡兩郡繫獄,並部送還臺,須候克日斷枉直。緣江遠郡及諸州,委刺史詳察訊。」己巳,以寧朔將軍程法勤為寧州刺史。



 六月,癸卯,車駕幸中堂聽訟。乙巳,以安陸王子敬為南兗州刺史。戊申,以黃門
 侍郎崔平仲為青、冀二州刺史。秋,七月,癸未,詔曰:「夫樂所自生,先哲垂誥,禮不忘本,積代同風。是以漢光遲回於南陽,魏文殷勤於譙國。青溪宮體天含暉,則地棲寶,光定靈源,允集符命。在昔期運初開,經綸方遠,繕築之勞,我則未暇。時流事往,永惟哽咽。朕以寡薄,嗣奉鴻基,思存締構,式表王跡。考星創制,揆日興功,子來告畢,規摹昭備。宜申釁落之禮,以暢感尉之懷,可克日小會。」甲申,立皇子子倫為巴陵王。八月,丙午,車駕幸舊宮小會,設金石樂,在位者賦詩。詔申「京師獄及三署見徒,量所降宥。領宮職司,詳賜幣帛」。戊申,車駕幸玄武湖講武。甲子,詔曰:「窆枯掩骼,義重前誥,恤老哀癃,實惟令典。



 朕永思民瘼,弗忘鑒寐。聲憓未敷,物多乖所。京師二縣,或有久墳毀發,可隨宜掩埋。遺骸未櫬,並加斂瘞。疾病窮困不能自存者,詳為條格,並加沾賚。」冬,十月,丁巳,以桂陽王鑠為南徐州刺史。



 十一月,丁亥,以始興王鑒為益州刺史。



 三年春,正月,丙辰,以大司農劉楷為交州刺史,安西咨議參軍崔慶緒為梁、南秦二州刺史。甲申,以晉安王子懋為南豫州刺史。辛卯,車駕祀南郊,大赦。都邑三百里內罪應入重者,降一等,餘依赦制。劾繫之身,降遣有差。賑恤二縣貧民。



 又詔曰:「《春秋國語》云『生民之有學斅,猶樹木之有枝葉。』果行育德,咸必由茲。在昔開運,光宅華夏,方弘典謨,克隆教思,命彼有司,崇建庠塾。甫就經始,仍離屯故,仰瞻徽猷,歲月彌遠。今遐邇一體,車軌同文,宜高選學官,廣延胄子。」又詔「守宰親民之要,刺史案部所先,宜嚴課農桑,相土揆時,必窮地利。



 若耕蠶殊眾,足厲浮墮者,所在即便列奏。其違方驕矜,佚事妨農,亦以名聞。將明賞罰,以勸勤怠。校核殿最,歲竟考課,以申黜陟。」二月,辛丑,車駕禮祠北郊。夏,四月,戊戌,以新除右衛將軍豫
 章王世子子響為豫州刺史,輔國將軍桓敬為兗州刺史。



 五月,乙未,詔曰:「氓俗凋弊,於茲永久,雖年穀時登,而歉乏比室。凡單丁之身及煢獨而秩養養孤者,並蠲今年田租。」是月,省總明觀。六月,庚戌,進河南王度易侯為車騎將軍。秋,七月,辛丑,詔「丹陽所領及餘二百里內見囚,同集京師;自此以外,委州郡決斷。」甲戌,左光祿大夫、開府儀同三司王僧虔薨。



 丁亥,以驃騎中兵參軍董仲舒為寧州刺史。



 八月,乙未,車駕幸中堂聽訟。丁巳,以行宕昌王梁彌頡為河、涼二州刺史。



 戊午,以尚書令王儉領太子少傅,太子詹事蕭順之為領軍將軍。冬,十月,壬戌,詔曰:「皇太子長懋講畢,當釋奠,王公以下可悉往觀禮。」十一月,乙丑,以冠軍將軍王文仲為青、冀二州刺史。



 十二月,丁酉,詔曰:「九穀之重,八材為末,是故潔粢豐盛,祝史無愧於辭,不籍千畝,周宣所以貽諫。昔期運初啟,庶政草昧,三推之典,
 我則未暇。朕嗣奉鴻基,思隆先軌,載耒躬親,率由舊式。可以開春發歲,敬簡元辰,鳴青鸞於東郊,冕朱紘而蒞事。仰薦宗禋,俯勖黔皂,將使囷庾內充,遺秉外牣。既富而教,茲焉攸在。」是夏,琅邪郡旱。百姓芟除枯苗,至秋擢穎大熟。



 四年春,正月,甲子,以南瑯邪、彭城二郡太守隨郡王子隆為江州刺史,征虜長史張瑰為雍州刺史,征虜將軍薛淵為徐州刺史,護軍將軍兼司徒竟陵王子良進號車騎將軍。富陽人唐宇之反,聚眾桐廬,破富陽、錢塘等縣,害東陽太守蕭崇之。



 遣宿衛兵出討,伏誅。丁酉,冠軍將軍、馬軍主陳天福坐討唐宇之燒掠百姓,棄市。



 辛卯,車駕幸中堂策秀才。



 閏月癸巳,立皇子子貞為邵陵王,皇孫昭文為臨汝公。丁未,以武都王楊集始為北秦州刺史。辛亥,車駕籍田。詔曰:「夫耕籍所以表敬,親載所以率民。朕景行前規,躬執良
 耜,千畛咸事,六稔可期,教義克宣,誠感兼暢。重以天符靈貺,歲月鱗萃,寶鼎開玉匣之祥,嘉禾發同穗之穎,甘露凝暉於坰牧,神爵騫翥於蘭囿。



 斯乃宗稷之慶,豈寡薄所臻!思俾休和,覃茲黔皂,見刑罪殊死以下,悉原宥。諸逋負在三年以前尤窮弊者,一皆蠲除。孝悌力田,詳授爵位,孤老貧窮,賜穀十石。



 凡欲附農而糧種闕乏者,並加給貸,務在優厚。」癸丑,以始興內史劉敕為廣州刺史。甲寅,以籍田禮畢,車駕幸閱武堂勞酒小會,詔賜王公以下在位者帛有差。戊午,車駕幸宣武堂講武。詔曰:「今親閱六師,少長有禮,領馭群帥,可量班賜。」



 二月,己未,立皇弟銶為晉熙王,鉉為河東王。庚寅,以光祿大夫王玄載為兗州刺史。



 三月,辛亥,國子講《孝經》,車駕幸學,賜國子祭酒、博士、助教絹各有差。



 夏,四月,丁亥,以尚書左僕射柳世隆為湘州刺史。臨沂縣麥不登,刈為馬芻,至夏更苗秀。五月,癸
 巳,詔「揚、南徐二州,今年戶租三分二取見布,一分取錢。



 來歲以後,遠近諸州輸錢處,並減布直,匹準四百,依舊折半,以為永制。」丙午,以吳興太守西昌侯鸞為中領軍。秋,八月,辛酉,以鎮南長史蕭惠休為廣州刺史。



 九月,甲寅,以征虜將軍王廣之為徐州刺史。冬,十二月,乙亥,以東中郎司馬崔惠景為司州刺史。



 五年春,正月,戊子,以太尉豫章王嶷為大司馬,車騎將軍竟陵王子良為司徒,驃騎將軍臨川王映、衛將軍王儉、中軍將軍王敬則並本號開府儀同三司,都官尚書沈文季為郢州刺史,左將軍安陸王子敬為荊州刺史,征虜將軍晉安王子懋為南兗州刺史,輔國將軍建安王子真為南豫州刺史。辛卯,詔曰:「朕昧爽丕顯,思康民瘼。



 雖年穀亟登,而饑饉代有。今履端肇運,陽和告始,宜協時休,覃茲黎庶。諸孤老貧病,並賜糧餼,遣使親賦,每存均普。」雍、司二州
 蠻虜屢動,丁酉,遣丹陽尹蕭景先出平陽,護軍將軍陳顯達出宛、葉。



 三月,戊子,車駕幸芳林園禊宴。丁未,以護軍將軍陳顯達為雍州刺史。夏,四月,車駕殷祠太廟。詔「系囚見徒四歲刑以下,悉原遣,五年減為三歲,京邑罪身應入重,降一等。」六月,辛酉,詔曰:「比霖雨過度,水潦洊溢,京師居民,多離其弊。遣中書舍人、二縣官長隨宜賑賜。」秋,七月,戊申,詔「丹陽屬縣建元四年以來至永明三年所逋田租,殊為不少。京甸之內,宜加優貸。其非中貲者,可悉原停。」八月,乙亥,詔「今夏雨水,吳興、義興二郡田農多傷,詳蠲租調。」



 九月,己丑,詔曰:「九日出商飆館登高宴群臣。」辛卯,車駕幸商飆館。館,上所立,在孫陵崗,世呼為「九日臺」者也。丙午,詔曰:「善為國者,使民無傷,而農益勸。是以十一而稅,周道克隆,開建常平,漢載惟穆。岱畎絲枲,浮汶來貢,杞梓皮革,必緣楚往。自水德將謝,喪亂彌多,師旅歲興,
 饑饉代有。貧室盡於課調,泉貝傾於絕域。軍國器用,動資四表,不因厥產,咸用九賦,雖有交貿之名,而無潤私之實。民咨塗炭,實此之由。昔在開運,星紀未周,餘弊尚重。農桑不殷於曩日,粟帛輕賤於當年。工商罕兼金之儲,匹夫多饑寒之患。良由圜法久廢,上弊稍寡。所謂民失其資,能無匱乎?凡下貧之家,可蠲三調二年。京師及四方出錢億萬,糴米穀絲綿之屬,其和價以優黔首。遠邦嘗市雜物,非土俗所產者,皆悉停之。必是歲賦攸宜,都邑所乏,可見直和市,勿使逋刻。」冬,十月,甲申,以中領軍西昌侯鸞為豫州刺史,侍中安陸侯緬為中領軍。初起新林苑。



 六年春,正月,壬午,以祠部尚書安成王暠為南徐州刺史。詔「二百里內獄同集京師,克日聽覽,自此以外,委州郡訊察。三署徒隸,詳所原釋。」三月,己亥,以豫章王世子子響為巴東王。癸卯,以光祿大
 夫周盤龍為行兗州刺史。



 五月甲午,以宕昌王梁彌承為河、涼二州刺史。六月,甲寅,以散騎常侍沈景德為徐州刺史。丙子,以始興太守房法乘為交州刺史。秋,七月,乙巳,都官尚書呂安國為領軍將軍。八月,乙卯,詔「吳興、義興水潦,被水之鄉,賜痼疾篤癃口二斛,老疾一斛,小口五斗。」九月,壬寅,車駕幸琅邪城講武,習水步軍。冬,十月,庚申,立冬,初臨太極殿讀時令。辛酉,以祠部尚書武陵王曄為江州刺史。



 閏月乙卯,詔曰:「北兗、北徐、豫、司、青、冀八州,邊接疆場,民多懸罄,原永明以前所逋租調。」辛卯,以尚書僕射王奐為領軍將軍。



 十一月,乙卯,以羽林監費延宗為越州刺史。庚申,以後將軍、晉安王子懋為湘州刺史,西陽王子明為南兗州刺史。



 七年春,正月,丙午,以中軍將軍王敬則為豫州刺史,中軍將軍陰智伯為梁、南秦二州刺史。戊申,詔曰:「雍州頻歲戎役,兼水
 旱為弊,原四年以前逋租。」



 辛亥,車駕祀南郊,大赦。京邑貧民,普加賑賜。又詔曰:「春頒秋斂,萬邦所以惟懷,柔遠能邇,兆民所以允殖。鄭渾宰邑,因姓立名,王濬剖符,戶口殷盛。今產子不育,雖炳常禁,比聞所在,猶或有之。誠復禮以貧殺,抑亦情由俗淡。宜節以嚴威,敦以惠澤。主者尋舊制,詳量附定,蠲恤之宜,務存優厚。」壬戌,驃騎將軍、開府儀同三司臨川王映薨。戊辰,詔曰:「諸大夫年秩隆重,祿力殊薄,豈所謂下車惟舊,趨橋敬老?可增俸,詳給見役。」二月,丙子,以左衛將軍、巴東王子響為中護軍。己丑,詔曰:「宣尼誕敷文德,峻極自天,發輝七代,陶鈞萬品,英風獨舉,素王誰匹!功隱於當年,道深於日月。感麟厭世,緬邈千祀,川竭谷虛,丘夷淵塞,非但洙泗湮淪,至乃饗嘗乏主。前王敬仰,崇脩寢廟,歲月亟流,鞠為茂草。今學敩興立,實稟洪規,撫事懷人,彌增欽屬。可改築宗祊,務在爽塏。量
 給祭秩,禮同諸侯。奉聖之爵,以時紹繼。」壬寅,以丹陽尹王晏為江州刺史。癸卯,以巴陵王子倫為豫州刺史。



 三月,丁未,以太子右衛率王玄邈為兗州刺史。庚戌,以中護軍、巴東王子響為江州刺史,中書令、隨郡王子隆為中護軍。甲寅,立皇子子岳為臨賀王,子峻為廣漢王,子琳為宣城王,子氏為義安王。夏,四月,戊寅,詔曰:「婚禮下達,人倫攸始,《周官》設媒氏之職,《國風》興及時之詠。四爵內陳,義不期侈,三鼎外列,事豈存奢!晚俗浮麗,歷茲永久,每思懲革,而民未知禁。乃聞同牢之費,華泰尤甚;膳羞方丈,有過王侯。富者扇其驕風,貧者恥躬不逮。或以供帳未具,動致推遷,年不再來,盛時忽往。宜為節文,頒之士庶。並可擬則公朝,方樏供設,合巹之禮無虧,寧儉之義斯在。如故有違,繩之以法。」五月,乙巳,尚書令、衛將軍、開府儀同三司王儉薨。甲子,以新除尚書左僕射柳世隆為尚書令。



 六月,
 丁亥,車駕幸瑯邪。秋,八月,庚子,以左衛將軍建安王子真為中護軍。



 冬,十月,己丑,詔曰:「三季澆浮,舊章陵替,吉凶奢靡,動違矩則。或裂錦繡以競車服之飾,塗金鏤石以窮塋域之麗。至班白不婚,露棺累葉,茍相誇衒,罔顧大典。可明為條制,嚴勒所在,悉使畫一。如復違犯,依事糾奏。」十二月,己亥,以中護軍、建安王子真為郢州刺史,江州刺史、巴東王子響為荊州刺史,前安西司馬垣榮祖為兗州刺史。



 八年,春,正月,庚子,征西大將軍王敬則進號驃騎大將軍,左將軍沈文季為領軍將軍,丹陽尹、鄱陽王鏘為江州刺史。詔放遣隔城虜俘,聽還本土。二月,壬辰,零陵王司馬藥師薨。夏,四月,戊辰,詔「公卿已下各舉所知,隨才授職。進得其人,受登賢之賞;薦非其才,獲濫舉之罰。」秋,七月,辛丑,以會稽太守安陸侯緬為雍州刺史。癸卯,詔曰:「陰
 陽舛和,緯象愆度,儲胤嬰患,淹歷旬晷。



 思仰祗天戒,俯紓民瘼,可大赦天下。」癸亥,詔「司、雍二州,比歲不稔,雍州八年以前、司州七年以前逋租悉原。汝南一郡復限更申五年。」八月,丙寅,詔「京邑霖雨既過,居民泛濫,遣中書舍人、二縣官長賑恤。」乙酉,以行河南王世子休留代為秦、河二州刺史。壬辰,以左衛將軍、隨郡王子隆為荊州刺史。巴東王子響有罪,遣丹陽尹蕭順之率軍討之,子響伏誅。冬,十月,丁丑,詔「吳興水淹過度,開所在倉賑賜。」癸巳,原建元以前逋租。



 十一月,乙卯,以建武將軍伏登之為交州刺史。十二月,乙丑,以振威將軍陳僧授為越州刺史。戊寅,詔「尚書丞郎職事繁劇,恤俸未優,可量增賜祿。」己卯,皇子子建為湘東王。癸巳,以監青冀二州軍、行刺史事張沖為青、冀二州刺史。



 九年春,正月,甲午,以侍中、江夏王鋒為南徐州刺史,冠軍將軍劉
 悛為益州刺史。辛丑,車駕祠南郊,詔「京師見囚繫,詳量原遣。」



 三月,乙卯,以南中郎司馬劉楷為司州刺史。辛丑,以太子左衛率劉纘為廣州刺史。夏,四月,乙亥,有司奏:「舊格一年兩過行陵,三月十五日曹郎以下小行,九月十五日司枯以下大行。今長停小行,唯二州一大行。」詔曰:「可。」六月,甲戌,以尚書左僕射王奐為雍州刺史。秋,九月,戊辰,車駕幸瑯邪城講武,觀者傾都,普頒酒肉。



 十年春,正月,戊午,詔「諸責負眾逋七年以前,悉原除。高貲不在例。孤老六疾,人穀五斛。內外有務眾官增祿俸。」以左民尚書、南平王銳為湘州刺史,司徒、竟陵王子良領尚書令,右衛將軍王玄邈為北徐州刺史,中軍將軍、廬陵王子卿進號車騎將軍,北中郎將、南海王子罕為兗州刺史,輔國將軍、臨汝公昭文為南豫州刺史,冠軍將軍王文和為北兗州刺史。二月,壬寅,鎮軍將軍陳顯達領中
 領軍。夏,四月,辛丑,大司馬豫章王嶷薨。



 五月,己巳,司徒、竟陵王子良為揚州刺史。秋,八月,丙申,以新城太守郭安明為寧州刺史。冬,十月,乙丑,車駕幸玄武湖講武。甲午,車駕殷祠太廟。十一月,戊午,詔曰:「頃者霖雨,樵糧稍貴,京邑居民,多離其弊。遣中書舍人、二縣官長賑賜。」



 十一年春,正月,癸丑,詔「京師見系囚,詳所原遣。」以驃騎大將軍王敬則為司空,江州刺史、鄱陽王鏘為領軍將軍,鎮軍大將軍陳顯達為江州刺史,右衛將軍崔慧景為豫州刺史。丙子,皇太子長懋薨。二月,壬午,以車騎將軍、廬陵王子卿為驃騎將軍、南豫州刺史,撫軍將軍、安陸王子敬進號車騎將軍。己丑,輔國將軍曹虎為梁、南秦二州刺史。癸卯,以新除中書監、晉安王子懋為雍州刺史。丙午,以冠軍將軍王文和為益州刺史。三月,乙亥,雍州刺史王奐伏
 誅。夏,四月,壬午,詔「東宮文武臣僚,可悉度為太孫官屬。」甲午,立皇太孫昭業、太孫妃何氏。詔「賜天下為父後者爵一級,孝子順孫義夫節婦粟帛各有差。」癸卯,以驍騎將軍劉靈哲為兗州刺史。五月,戊辰,詔曰:「水旱成災,穀稼傷弊,凡三調眾逋,可同申至秋登。京師二縣、朱方、姑熟,可權斷酒。」庚午,以輔國將軍蕭惠休為徐州刺史。丙子,以左民尚書、宜都王鏗為南豫州刺史。六月,壬午,詔「霖雨既過,遣中書舍人、二縣官長賑賜京邑居民。」秋,七月,丁巳,詔曰:「頃風水為災,二岸居民多離其患,加以貧病六疾,孤老稚弱,彌足矜念。遣中書舍人履行沾恤。」又詔曰:「水旱為災,實傷農稼。江淮之間,倉廩既虛,遂草竊充斥,互相侵奪,依阻山湖,成此逋逃。曲赦南兗、兗、豫、司、徐五州,南豫州之歷陽、譙、臨江、廬江四郡,三調眾逋宿債,並同原除。其緣淮及青、冀新附僑民,復除已訖,更申五年。」是月,上不
 豫,徙御延昌殿,乘輿始登階,而殿屋鳴吒,上惡之。虜侵邊,戊辰,遣江州刺史陳顯達鎮雍州樊城。上慮朝野憂惶,乃力疾召樂府奏正聲伎。戊寅,大漸。詔曰:「始終大期,賢聖不免,吾行年六十,亦復何恨。但皇業艱難,萬機事重,不能無遺慮耳。



 太孫進德日茂,社稷有寄。子良善相毗輔,思弘治道;內外眾事,無大小悉與鸞參懷,共下意。尚書中是職務根本,悉委王晏、徐孝嗣。軍旅捍邊之略,委王敬則、陳顯達、王廣之、王玄邈、沈文季、張瑰、薛淵等。百辟庶僚,各奉爾職,謹事太孫,勿有懈怠。知復何言。」又詔曰:「我識滅之後,身上著夏衣,畫天衣,純烏犀導,應諸器悉不得用寶物及織成等,唯裝復裌衣各一通。常所服身刀長短二口鐵環者,隨我入梓宮。祭敬之典,本在因心,東鄰殺牛,不如西家禴祭。我靈上慎勿以牲為祭,唯設餅、茶飲、乾飯、酒脯而已。天下貴賤,咸同此制。未山陵前,朔望設
 菜食。陵墓萬世所宅,意嘗恨休安陵未稱,今可用東三處地最東邊以葬我,名為景安陵。喪禮每存省約,不須煩民。百官停六時入臨,朔望祖日可依舊。諸主六宮,並不須從山陵。內殿鳳華、壽昌、耀靈三處,是吾所治制。夫貴有天下,富兼四海,宴處寢息,不容乃陋,謂此為奢儉之中,慎勿壞去。顯陽殿玉像諸佛及供養,具如別牒,可盡心禮拜供養之。應有功德事,可專在中。自今公私皆不得出家為道,及起立塔寺,以宅為精舍,並嚴斷之。唯年六十,必有道心,聽朝賢選序,已有別詔。諸小小賜乞,及閣內處分,亦有別牒。內外禁衛勞舊主帥左右,悉付蕭諶優量驅使之,勿負吾遺意也。」是日上崩,年五十四。



 上剛毅有斷,為治總大體,以富國為先。頗不喜游宴、雕綺之事,言常恨之,未能頓遣。臨崩又詔「凡諸遊費,宜從休息。自今遠近薦獻,務存節儉,不得出界營求,相高奢麗。金粟繒纊,弊民
 已多,珠玉玩好,傷工尤重,嚴加禁絕,不得有違準繩。」九月,丙寅,葬景安陵。



 史臣曰:世祖南面嗣業,功參寶命,雖為繼體,事實艱難。御袞垂旒,深存政典,文武授任,不革舊章。明罰厚恩,皆由上出,義兼長遠,莫不肅然。外表無塵,內朝多豫,機事平理,職貢有恆,府藏內充,民鮮勞役。宮室苑囿,未足以傷財,安樂延年,眾庶所同幸。若夫割愛懷抱,同彼甸人,太祖群昭,位後諸穆。昔漢武留情晚悟,追恨戾園,魏文侯克中山,不以封弟,英賢心跡,臣所未詳也。



 贊曰:武帝丕顯,徽號止戈。韶嶺歇祲,彭派澄波。威承景歷,肅御
 金科。北懷戎款,南獻夷歌。市朝晏逸,中外寧和。



\end{pinyinscope}