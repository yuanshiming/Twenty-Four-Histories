\article{卷三十一列傳第十二 江謐 荀伯玉}

\begin{pinyinscope}

 江謐,字令和,濟陽考城人也。祖秉之,臨海太守,宋世清吏。父徽,尚書都官郎,吳令,為太初所殺。謐系尚方,孝武平京邑,乃得出。解褐奉朝請,輔國行參軍,於湖令,強濟稱職。宋明帝為南豫州,謐傾身奉之,為帝所親待。即位,以為驃騎參軍。弟蒙貌醜,帝常召見狎侮之。謐轉尚書度支郎,俄遷右丞兼比部郎。



 泰始四年,江夏王義恭第十五女卒,年十九,未笄。禮官議從成人服,諸王服大功。左丞孫夐重奏:《禮記》女子十五而笄,鄭云應年許嫁者也。其未許嫁者,則二十而笄。射慈云十九猶為殤。禮官違越經典,於禮無據。」博士太常
 以下結免贖論;謐坐杖督五十,奪勞百日,謐又奏:「夐先不研辨,混同謬議。準以事例,亦宜及咎。」夐又結免贖論。詔「可」。



 出為建平王景素冠軍長史、長沙內史,行湘州事。政治苛刻。僧遵道人與謐情款,隨謐蒞郡,犯小事,餓系郡獄,僧遵裂三衣食之,既盡而死。為有司所奏,徵還。明帝崩,遇赦得免。為正員郎,右軍將軍。



 太祖領南兗州,謐為鎮軍長史、廣陵太守,入為游擊將軍。性流俗,善趨勢利。



 元徽末,朝野咸屬意建平王景素,謐深自委結,景素事敗,僅得免禍。蒼梧王廢後,物情尚懷疑惑,謐獨竭誠歸事太祖,以本官領尚書左丞。昇明元年,遷黃門侍郎,左丞如故。沈攸之事起,議加太祖黃皞,謐所建也。事平,遷吏部郎,稍被親待。



 遷太尉諮議,領錄事參軍。齊臺建,為右衛將軍。建元元年,遷侍中。出為臨川王平西長史、冠軍
 將軍、長沙內史、行湘州留事,先遣之鎮,既而驃騎豫章王嶷領湘州,以謐為長史,將軍、內史、知州留事如故。封永新縣伯,四百戶。三年,為左民尚書。諸皇子出閣用文武主帥,皆以委謐。尋敕曰:「江謐寒士,誠當不得競等華儕。然甚有才幹,堪為委遇,可遷掌吏部。」



 謐才長刀筆,所在事辦。太祖崩,謐稱疾不入,眾頗疑其怨不豫顧命也。世祖即位,謐又不遷官,以此怨望。時世祖不豫,謐詣豫章王嶷請間曰:「至尊非起疾,東宮又非才,公今欲作何計?」世祖知之,出謐為征虜將軍、鎮北長史、南東海太守。未發,上使御史中丞沈沖奏謐前後罪曰:「謐少懷輕躁,長習諂薄,交無義合,行必利動。特以奕世更局,見擢宋朝,而阿諛內外,貨路公行,咎盈憲簡,戾彰朝聽,輿金輦寶,取容近習。以沈攸之地勝兵強,終當得志,委心托身,歲暮相結;以劉景素親屬望重,物應樂推,獻誠薦子,窺窬非望。時艱網漏,得全首領。太祖匡飭天地,方弘遠圖,薄其難洗之瑕,許其革音之
 效,加以非分之寵,推以不次之榮,列跡勛良,比肩朝德。以往者微勤,刀筆小用,賞廁河山,任忝出入。輕險之性,在貴彌彰;貪昧之情,雖富無滿。重蒞湘部,顯行斷盜;及居銓衡,肆意受納。



 連席同乘,皆詖黷舊侶;密筵閑宴,必貨賄常客。理合升進者,以為己惠;事宜貶退者,並稱中旨。謂販鬻威權,姦自不露,欺主罔上,謗議可掩。先帝寢疾彌留,人神憂震。謐托病私舍,曾無變容。國諱經旬,甫暫入殿,參訪遺詔,覘忖時旨。



 以身列朝流,宜蒙兼帶,先顧不逮,舊位無加,遂崇飾惡言,肆醜縱悖,譏誹朝政,訕毀皇猷,遍蚩忠賢,歷詆臺相。至於蕃岳入授,列代恆規,勳戚出撫,前王彞則,而謐妄發樞機,坐構囂論。復敢貶謗儲后,不顧辭端,毀折宗王,每窮舌杪。皆云誥誓乖禮,崇樹失宜,仰指天,俯畫地,希幸災故,以申積憤。犯上之跡既彰,反噬之情已著。請免官削爵土,收送廷尉獄治罪。」詔賜死,時年五
 十二。



 子介,建武中,為吳令,治亦深切。民間榜死人髑髏為謐首,介棄官而去。



 荀伯玉,字弄璋,廣陵人也。祖永,南譙太守。父闡之,給事中。伯玉少為柳元景撫軍板行參軍,南徐州祭酒,晉安王子勛鎮軍行參軍。泰始初,子勛舉事,伯玉友人孫沖為將帥,伯玉隸其驅使,封新亭侯。事敗,伯玉還都賣卜自業。建平王景素聞而招之,伯玉不往。



 太祖鎮淮陰,伯玉歸身結事,為太祖冠軍刑獄參軍。太祖為明帝所疑,及徵為黃門郎,深懷憂慮。伯玉勸太祖遣數十騎入虜界,安置標榜,於是虜游騎數百履行界上,太祖以聞,猶懼不得留,令伯玉卜,伯玉斷卦不成行,而明帝詔果復太祖本任,由是見親待。從太祖還都,除奉朝請。令伯玉看宅,知家事。世祖罷廣興還,立別宅,遣人於大宅掘樹數株,伯玉不與,馳以聞。太祖曰:「卿執之是也。」轉太祖平
 南府,晉熙王府參軍。太祖為南兗州,伯玉轉為上鎮軍中兵參軍,帶廣陵令。



 除羽林監,不拜。



 初,太祖在淮南,伯玉假還廣陵,夢上廣陵城南樓上,有二青衣小兒語伯玉云:「草中肅,九五相追逐。」伯玉視城下人頭上皆有草。泰始七年,伯玉又夢太祖乘船在廣陵北渚,見上兩掖下有翅不舒。伯玉問何當舒,上曰:「卻後三年。」伯玉夢中自謂是咒師,向上唾咒之,凡六咒,有六龍出,兩掖下翅皆舒,還而復斂。元徽二年而太祖破桂陽,威名大震;五年而廢蒼梧。太祖謂伯玉曰:「卿時乘之夢,今且效矣。」



 升明初,仍為太祖驃騎中兵參軍,除步兵校尉,不拜。仍帶濟陽太守,中兵如故。霸業既建,伯玉忠勤盡心,常衛左右。加前軍將軍。隨太祖太尉府轉中兵,將軍、太守如故。建元元年,封南豐縣子,四百戶。轉輔國將軍,武陵王征虜司馬,太守如故。徙為安成王冠軍司馬,轉豫章王司空諮議,太守
 如故。



 世祖在東宮,專斷用事,頗不如法。任左右張景真,使領東宮主衣食官穀帛,賞賜什物,皆御所服用。景真於南澗寺捨身齋,有元徽紫皮褲褶,餘物稱是。於樂遊設會,伎人皆著御衣。又度絲錦與昆侖舶營貨,輒使傳令防送過南州津。世祖拜陵還,景真白服乘畫舴艋,坐胡床,觀者咸疑是太子。內外祗畏,莫敢有言。伯玉謂親人曰:「太子所為,官終不知,豈得顧死蔽官耳目!我不啟聞,誰應啟者?」



 因世祖拜陵後密啟之。上大怒,檢校東宮。世祖還至方山,日暮將泊。豫章王於東府乘飛䴏東迎,具白上怒之意。世祖夜歸,上亦停門籥待之,二更盡,方入宮。



 上明日遣文惠太子、聞喜公子良宣敕,以景真罪狀示世祖。稱太子令,收景真殺之。



 世祖憂懼,稱疾月餘日。上怒不解。晝臥太陽殿,王敬則直入,叩頭啟上曰:「官有天下日淺,太子無事被責,人情恐懼,願官往東宮解釋之。」太祖乃幸宮,
 召諸王以下於玄圃園為家宴,致醉乃還。



 上嘉伯玉盡心,愈見親信,軍國密事,多委使之。時人為之語曰:「十敕五令,不如荀伯玉命。」世祖深怨伯玉。上臨崩,指伯玉謂世祖曰:「此人事我忠,我身後,人必為其作口過,汝勿信也。可令往東宮長侍白澤,小卻以南兗州處之。」



 伯玉遭父憂,除冠軍將軍、南濮陽太守,未拜,除黃門郎,本官如故。世祖轉為豫章王太尉諮議,太守如故。俄遷散騎常侍,太守如故。伯玉憂懼無計,上聞之,以其與垣崇祖善,慮相扇為亂,加意撫之,伯玉乃安。永明元年,垣崇祖誅,伯玉並伏法。



 初,善相墓者見伯玉家墓,謂其父曰:「當出暴貴而不久也。」伯玉後聞之,曰:「朝聞道,夕死可矣。」死時年五十。



 史臣曰:君老不事太子,義烈之遺訓也。欲夫專心所奉,在節無貳,雖人子之親,尚宜自別,則偏黨為論,豈或傍啟!察江、荀之行也,雖
 異術而同亡。以古道而居今世,難乎免矣。



 贊曰:謐口禍門,荀言亟盡。時清主異,並合同殞。



\end{pinyinscope}