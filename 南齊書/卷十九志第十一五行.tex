\article{卷十九志第十一五行}

\begin{pinyinscope}

 《木傳》曰:「東方。《易經》,地上之木為《觀》。故木於人,威儀容貌也。木者,春生氣之始,農之本也。無奪農時,使民歲不過三日,行什一之稅,無貪欲之謀,則木氣從。如人君失威儀,逆木行,田獵馳騁,不反宮室,飲食沈湎,不顧禮制,出入無度,多發繇役,以奪民時,作為姦詐,以奪民財,則木失其性矣。蓋以工匠之為輪矢者多傷敗,故曰木不曲直。」



 宋泰豫元年,京師祗垣寺皂莢樹枯死。昇明末,忽更生花葉。《京房易傳》曰:「樹枯冬生,不出二年,國喪,君子亡。」其占同。宋氏禪位。



 建元元年,朱爵𦨵華表柱生枝葉。



 建元初,李子生毛。



 二年,武陵沅頭都尉治有桑樹,方冬生葉。《京房易傳》曰:「木冬生花,天下有喪。」其占同。後二年,宮車晏駕。



 四年,巴州城西古樓腳柏柱數百年,忽生花。



 永明六年,石子崗柏木長二尺四寸,廣四寸半,化為石。時車駕數游幸,應本傳「木失其性」也。



 永明中,大𦨵一舶無故自沉,艚中無水。



 隆昌元年,廬陵王子卿齋屋梁柱際無故出血。



 建武初,始安王遙光治廟,截東安寺屋以直廟垣,截梁,水出如淚。



 《貌傳》曰:「失威儀之制,怠慢驕恣,謂之狂,則不肅矣。下不敬,則上無威。天下既不敬,又肆其驕恣,肆之則不從。夫不敬其君,不從其政,則陰氣勝,故曰厥罰常雨。」



 永明八年四月,己巳起陰雨,晝或暫晴,夜時見星月,連雨積霖,至
 十七日乃止。



 十一年四月辛巳朔,去三月戊寅起,而其間暫時晴,從四月一日又陰雨,晝或見日,夜乍見月,回復陰雨,至七月乃止。



 永泰元年十二月二十九日雨,至永元元年五月二十一日乃晴。京房《易》曰:「冬雨,天下饑。春雨,有小兵。」時虜寇雍州,餘應本傳。



 《傳》曰:「大雨雪,猶庶徵之常雨也,然有甚焉。雨,陰。大雨雪者,陰之畜積甚也。一曰與大水同象,曰攻為雪耳。」



 建元二年閏月己丑,雨雪。



 三年十一月,雨雪,或陰或晦,八十餘日,至四年二月乃止。



 《傳》曰:「雷於天地為長子,以其首長萬物,與之出入。故雷出萬物出,雷入萬物入。夫雷者,人君之象,入則除害,出則興利。雷之微氣以正月出,其有聲者以二月出,以八月入,其餘微者以九月入。



 冬三
 月雷無出者;若是陽不閉陰,則出涉危難而害萬物也。」



 建元元年十月壬午,夜電光,因雷鳴。



 十一月庚戌,電光,有頃雷鳴,久而止。



 永明五年正月戊申,夜西北雷聲。



 六年十月甲申,夜陰細雨,始聞雷鳴於西北上。



 七年正月甲子,夜陰,雷鳴西南坤宮,隆隆一聲而止。



 八年正月庚戌,夜雷起坎宮水門,其音隆隆,一聲而止。



 九年二月丙子,西北有電光,因聞雷聲隆隆,仍續十聲而止。



 十年二月庚戌,夜南方有電光,因聞雷聲隆隆相續,丁亥止。



 十月庚子,電雷起西北。



 十一月丁丑,西南有光,因聞雷聲隱隱,再聲而止。西南坤宮。十二月甲申,陰雨,有電光,因聞西南及西北上雷鳴,頻續三聲。
 丙申,夜聞西北上雷頻續二聲。辛亥,雷雨。



 《傳》曰:「雨雹,君臣之象也。陽之氣專為雹,陰之氣專為霰。陽專而陰脅之,陰盛而陽薄之。雹者,陰薄陽之象也。霰者,陽脅陰之符也。《春秋》不書霰者,猶月蝕也。」



 建元四年五月戊午朔,雹。



 永明元年九月乙丑,雹落大如蒜子,須臾乃止。



 十一年四月辛亥,雹落大如蒜子,須臾滅。



 《貌傳》又曰:「上失節而狂,下怠慢而不敬,上下失道,輕法侵制,不顧君上,因以薦饑。貌氣毀,故有雞禍。」「一曰水歲雞多死及為怪,亦是也。上下不相信,大臣姦宄,民為寇盜,故曰厥極惡。」



 「一曰民多被刑,或形貌醜惡,風俗狂慢,變節易度,則為輕剽奇怪之服,故曰時則有
 服妖。」



 永明中,宮內服用射獵錦文,為騎射兵戈之象。至建武初,虜大為寇。



 永明中,蕭諶開博風帽後裙之制,為破後帽。世祖崩後,諶建廢立,誅滅諸王。



 永明末,民間製倚勸帽。及海陵廢,明帝之立,勸進之事,倚立可待也。



 建武中,帽裙覆頂;東昏時,以為裙應在下,而今在上,不詳,斷之。群下反上之象也。



 永元中,東昏侯自造遊宴之服,綴以花採錦繡,難得詳也。群小又造四種帽,帽因勢為名:一曰山鵲歸林者,《詩》云「《鵲巢》,夫人之德」,東昏寵嬖淫亂,故鵲歸其林藪;二曰兔子度坑,天意言天下將有逐
 兔之事也;三曰反縛黃離嘍,黃口小鳥也,反縛,面縛之應也;四曰鳳皇度三橋,鳳皇者嘉瑞,三橋,梁王宅處也。



 《貌傳》又曰:「危亂端見,則天地之異生。木者青,故曰青眚,為惡祥。凡貌傷者,金沴木,木沴金,衡氣相通。」



 延興元年,海陵王初立,文惠太子冢上有物如人,長數丈,青色,直上天,有聲如雷。



 火,南方,揚光輝,出炎𤓀為明者也。人君向明而治,蓋取其象。以知人為分,讒佞既遠,群賢在位,則為明而火氣從矣。人君疑惑,棄法律,不誅讒邪,則讒口行,內間骨肉,外疏忠臣,至殺世子,逐功臣,以妾為妻,則火失其性,上災宗廟,下災府榭,內熯本朝,外熯闕觀,雖興師眾,不能救也。



 永明三年正月,甲夜西北有野火,光上生精。西北有四,東北有一,並長七八尺,黃赤色。



 三月庚午,丙夜北面有野火,光上生精,長六尺;戊夜又有一
 枚,長五尺,並黃赤色。



 四年正月丁亥,夜有火精三處。



 閏月丁巳,夜有火精四所。



 十二月辛酉,夜東南有野火精二枚。



 五年十二月丙寅,夜西北有野火,火上生精,一枚,長三尺,黃白色。



 六年十一月戊申,夜西南及北三面有野火,火生精,九枚,並長二尺,黃赤色。



 九年二月丙寅,甲夜北面有野火,火上生精,二枚,西北又一枚,並長三尺,須臾消。



 永元二年八月,宮內火,燒西齋芃儀殿及昭陽、顯陽等殿,北至華
 林牆,西及秘閣北,屋三千餘間。



 《京房易傳》曰:「君不思道,厥妖火燒宮。」秘閣與《春秋》宣榭火同,天意若曰,既無紀綱,何用典文為也!二年冬,京師民間相驚云當行火災,南岸人家往往於籬間得布火纏者,云公家以此禳之。



 三年正月,豫章郡天火燒三千餘家。京房《易》占曰:「天火下燒民屋,是謂亂治殺兵作。」是年,臺軍與義師偏眾相攻於南江諸郡。



 三年二月,乾和殿西廂火,燒屋三十間。是時西齋既火,帝徙居東齋,高宗所住殿也。與燒宮占同。



 《傳》又曰:「犯上者不誅,則草犯霜而不死。或殺不以時,事在殺生失柄,故曰草妖也。」一曰:「草妖者,失眾之象也。」



 永元中,御刀黃文濟家齋前種昌蒲,忽生花,光影照壁,成五采,其兒見之,餘人不見也。少時,文濟被殺。劉
 歆《視傳》有羽蟲之孽,謂雞禍也。班固案《易》雞屬《巽》,今以羽蟲之孽類是也,依歆說附《視傳》云。



 建武二年,有大鳥集建安,形如水犢子。其年,郡大水。



 三年,大鳥集東陽郡,太守沈約表云:「鳥身備五採,赤色居多。」案《樂緯葉圖征》云:「焦明鳥質赤,至則水之感也。」



 永明二年四月,烏巢內殿東鴟尾。



 三年,大鳥集會稽上虞。其年,縣大水。



 《傳》曰:「維水沴火。」又曰:「赤眚赤祥。」



 建武四年,王晏子德元所居帷屏,無故有血灑之,少日而散。晏尋被誅。



 《思心傳》曰:「心者,土之象也。思心不睿,其過在瞀亂失紀。風於陽則為君,於陰則為大臣之象,專恣而氣盛,故罰常風。心為五事主,猶土為五行主也。」一曰:「陰陽相薄,偏氣陽多為風,其甚也常風。陰氣多
 者,陰而不雨,其甚也常陰。」一曰:「風宵起而晝晦,以應常陰同象也。」



 建元元年十一月庚戌,風夜暴起,雲雷合冥,從戌亥上來。



 四年十一月甲寅,酉時風起小駃,至二更雪落,風轉浪津。



 永明四年二月丙寅,巳時風迅急。



 十一月己丑,戌時風迅急,從西北戌亥上來。



 五年五月乙酉,子時風迅急,從西北戌亥上來。



 七年正月丁卯,陽徵陰賊之日,時加子,風起迅急,從北方子丑上來,暴疾浪津,寅時止。



 八年六月乙酉,時加子,風起迅急,暴疾浪津,發屋折木,塵沙,從西南未上來,因雷雨,須臾,風微雨止。



 九年七月甲寅,陽羽廉貞之日,時加亥,風起迅急,從東方來,暴疾彭勃浪津,至乙卯陰賊時漸微,名羽動羽。



 九月乙丑,時加未,雷,驟雨,風起迅急,暴疾浪津,從西北戌上來。



 十月壬辰,陽羽姦邪之日,時加丑,風起從北方子丑上來,暴疾浪津,迅急,塵埃,五日寅時漸微,名羽動宮。



 十年正月辛巳,陽商寬大之日,時加寅,風從西北上來,暴疾浪津,迅急,揚沙折木,酉時止。



 二月甲辰,陽徵姦邪之日,時加辰,風起迅急,從西北亥上來,暴疾彭勃浪津,至酉時止。



 三月丁酉,陽徵廉貞之日,時加未,風從北方子丑上來,迅急,暴疾浪津,戌時止。



 七月庚申,陰角貪狼之日,時加午,風從東北丑上來,迅急浪津,至辛酉巳時漸微。



 十一年二月庚寅,陽角廉貞之日,時加亥,風從西北亥上來,迅疾
 浪津,醜時漸微,為角動角。



 七月甲寅,陽羽廉貞之日,時加巳,風從東北寅上來,迅疾浪津,發屋折木,戊夜漸微,為羽動徵。己巳,陽角寬大之日,時加未,風從戌上來,暴疾,良久止,為角動商及宮。



 凡時無專恣,疑是陰陽相薄。



 建武元年三月乙酉,未時風起,浪津暴急,從北方上來,應本傳瞀亂。



 建武二年、三年、四年,每秋七月、八月,輒大風,三吳尤甚,發屋折木,殺人。京房占:「獄吏暴,風害人。」時帝嚴刻。



 永元元年七月十二日,大風,京師十圍樹及官府居民屋皆拔倒,應本傳。



 《
 傳》又曰:「山之於地,君之象也。山崩者,君權損,京陵易處,世將變也。陵轉為澤,貴將為賤也。」



 建元二年夏,廬陵石陽縣長溪水衝激山麓崩,長六七丈,下得柱千餘口,皆十圍,長者一丈,短者八九尺,頭題有古文字,不可識。江淹以問王儉,儉云:「江東不閑隸書,此秦漢時柱也。」後年宮車晏駕,世變之象也。



 永明二年秋,始興曲江縣山崩,壅底溪水成陂。京房占:「山崩,人主惡之。」



 《傳》又曰:「雷電所擊,蓋所感也。皆思心有尤之所致也。」



 建元二年閏六月丙戌,戊夜震電。



 四年五月五日,雷雹暗都,雷震於樂遊安昌殿,電火焚蕩盡。



 永明八年四月六日,雷震會稽山陰恆山保林寺,剎上四破,電火
 燒塔,下佛面窗戶不異也。



 永明中,雷震東宮南門,無所傷毀,殺食官一人。



 十一年三月,震於東齋,棟崩。左右密欲治繕,竟陵王子良曰:「此豈可治!留之志吾過,且旌天之愛我也。」明年,子良薨。



 《傳》又曰:「土氣亂者,木金水火亂之。」



 建武二年二月丁巳,地震。



 永元元年七月,地日夜十八震。



 九月十九日,地五震。



 金者,西方,萬物既成,殺氣之始也。其於王事,兵戎戰伐之道也。王者興師動眾,建立旗鼓,仗旄把鉞,以誅殘賊,止暴亂,殺伐應義,則金氣從。工冶鑄化,革形成器也。人君樂侵陵,好攻戰,貪城邑,輕百姓之命,人民不安,內外騷動,則金失其性。蓋冶鑄不化,冰滯固堅,
 故曰金不從革,又曰維木沴金。建武四年,明帝出舊宮送豫章王第二女綏安主降嬪,還上輦,輦上金翅無故自折落地。



 《言傳》曰:「言《易》之道,西方曰《兌》,為口。人君過差無度,刑法不一,斂從其重,或有師旅,炕陽之節,若動眾勞民,是言不從。人君既失眾,政令不從,孤陽持治,下畏君之重刑,陽氣勝則旱象至,故曰厥罰常陽也。」建元三年,大旱,時有虜寇。



 永民三年,大旱,明年,唐宇之起。



 建武二年,大旱,時虜寇方盛,皆動眾之應也。



 《言傳》曰:「下既悲苦君上之行,又畏嚴刑而不敢正言,則必先發於歌謠。歌謠,口事也。口氣逆則惡言,或有怪謠焉。」



 宋泰始既失彭城,江南始傳種消梨,先時所無,百姓爭欲種植。識者曰:「當有姓蕭而來者。」



 十餘年,齊受禪。元徽中,童謠曰:「襄陽白銅蹄,郎殺荊州兒。」後沈攸之反,雍州刺史張敬兒襲江陵,殺沈攸之子元琰等。



 永明元年元日,有小人發白虎樽,既醉,與筆札,不知所道,直云「憶高帝」。敕原其罪。



 世祖起青溪舊宮,時人反之曰:「舊宮者,窮廄也。」及上崩後,宮人出居之。



 永明初,百姓歌曰:「白馬向城啼,欲得城邊草。」後句間云「陶郎來」。白者金色,馬者兵事。



 三年,妖賊唐宇之起,言唐來勞也。



 世祖起禪靈寺初成,百姓縱觀。或曰:「禪者授也,靈非美名,所授必不得其人。」後太孫立,見廢也。



 永明中,宮內坐起御食之外,皆為客食。世祖以客非家人名,改呼為別食,時人以為分別之象。



 少時,上晏駕。



 文惠太子在東宮,作「兩頭纖纖」詩,後句云「磊磊落落玉山崩」,自此長王宰相相繼薨徂,二宮晏駕。



 文惠太子作七言詩,後句輒云「愁和諦」。後果有和帝禪位。



 永明中,虜中童謠云:「黑水流北,赤火入齊。」尋而京師人家忽生火,赤於常火,熱小微,貴賤爭取以治病。法以此火灸桃板七炷,七日皆差。敕禁之,不能斷。京師有病癭者,以火灸數日而差。



 鄰人笑曰:「病偶自差,豈火能為。」此人便覺頤間癢,明日癭還如故。後梁以火德興。



 文惠太子起東田,時人反云:「後必有癲童。」果由太孫失位。



 齊宋以來,民間語云:「擾攘建武上。」明帝初,誅害蕃戚,京師危
 駭。



 永元元年,童謠曰:「洋洋千里流,流翣東城頭。烏馬烏皮褲,三更相告訴。腳跛不得起,誤殺老姥子。」千里流者,江祏也。東城,遙光也。遙光夜舉事,垣歷生者烏皮褲褶往奔之。跛腳,亦遙光。老姥子,孝字之象,徐孝嗣也。永元中,童謠云:「野豬雖嗃嗃,馬子空閭渠。不知龍與虎,飲食江南墟。七九六十三,廣莫人無餘。烏集傳舍頭,今汝得寬休。但看三八後,摧折景陽樓。」識者解云「陳顯達屬豬,崔慧景屬馬」,非也。東昏侯屬豬,馬子未詳,梁王屬龍,蕭穎胄屬虎。崔慧景攻臺,頓廣莫門死,時年六十三。烏集傳舍,即所謂「瞻烏爰止,于誰之屋」。三八二十四,起建元元年,至中興二年,二十四年也。摧折景陽樓,亦高臺傾之意也。言天下將去,乃得休息也。



 齊、宋之際,民間語云「和起」,言以和顏而為變起也。後和帝立。



 崔慧景圍臺城,有一五色幡,飛翔在雲中,半日乃不見,眾皆驚怪,相謂曰:「幡者,事尋當翻覆也。」數日而慧景敗。



 《言傳》曰:「言氣傷則民多口舌,故有口舌之痾。金者白,故有日眚,若有白為惡祥。」



 宋昇明二年,飆風起建康縣南塘里,吹帛一匹入雲,風止,下御路。紀僧真啟太祖當宋氏禪者,其有匹夫居之。



 水,北方,冬藏萬物,氣至陰也,宗廟祭祀之象。死者精神放越不反,故為之廟以收其散,為之貌以收其魂神,而孝子得盡禮焉。敬之至,則神歆之,此則至陰之氣從,則水氣從溝瀆隨而流去,不為民害矣。人君不禱祀,簡宗廟,廢祭祀,逆天時,則霧水暴出,川水逆溢,壤邑軼鄉,沉溺民人,故曰水不潤下。



 建元二年,吳、吳興、義興三郡大水。



 二年夏,丹陽、吳二郡大水。



 四年,大水。



 永明五年夏,吳興、義興水雨傷稼。



 六年,吳興、義興二郡大水。



 建武二年冬,吳、晉陵二郡水雨傷稼。



 永元元年七月,濤入石頭,漂殺緣淮居民。應本傳。



 荊州城內有沙池,常漏水。蕭穎胄為長史,水乃不漏,及穎胄亡,乃復竭。



 《傳》曰:「極陰氣動,故有魚孽。魚孽者,常寒罰之符也。」



 永明九年,鹽官縣石浦有海魚乘潮來,水退不得去,長三十餘丈,黑色無鱗,未死,有聲如牛。土人呼為海燕,取其肉食之。



 永元元年四月,有大魚十二頭入會稽上虞江,大者近二十餘丈,小者十餘丈,一入山陰稱蒲,一入永興江,皆暍岸側,百姓取食
 之。



 《聽傳》曰:「不聰之象見,則妖生於耳,以類相動,故曰有鼓妖也。」一曰,聲屬鼓妖。



 永明元年十一月癸卯夜,天東北有聲,至戊夜。



 《傳》曰:「皇之不極,是謂不建,其咎在霿亂失聽,故厥咎霿。思心之咎亦霧。天者,正萬物之始,王者,正萬事之始,失中則害天氣,類相動也。天者轉於下而運於上,雲者起於山而彌於天,天氣動則其象應,故厥罰常陰。王者失中,臣下盛強,而蔽君明,則雲陰亦眾多而蔽天光也。



 建元四年十月丙午,日入後土霧勃勃如火煙。



 永明二年十一月己亥,四面土霧入人眼鼻,至辛丑止。



 二年十一月丙子,日出後及日入後,四面土霧勃勃如火煙。



 六年十一月庚戌,丙夜土霧竟天,昏塞濃厚,至六日未時小開,到
 甲夜後仍濃密,勃勃如火煙,辛慘,入人眼鼻。



 八年十月壬申,夜土霧竟天,濃厚勃勃如火煙,氣入人眼鼻,至九日辰時開除。



 九年十月丙辰,晝夜恆昏霧勃勃如火煙,其氣辛慘,入人眼鼻,兼日色赤黃,至四日甲夜開除。



 十年正月辛酉,酉初四面土霧勃勃如火煙,其氣辛慘,入人眼鼻。



 《傳》曰:「《易》曰『乾為馬』。逆天氣,馬多死,故曰有馬禍。」一曰,馬者,兵象也。將有寇戎之事,故馬為怪。



 建武四年,王晏出至草市,馬驚走,鼓步從車而歸,十餘日,晏誅。



 建武中,南岸有一蘭馬,走逐路上女子,女子窘急,走入人家床下避之,馬終不置,發床食女子股腳間肉都盡。禁司以聞,敕殺此馬,是後頻有寇賊。



 京房《易傳》曰:「生子二胸以上,民謀其主。三手以上,臣謀其主。二口已上,國見驚以兵。三耳已上,是謂多聽,國事無定。二鼻以上,國主久病。三足三臂已上,天下有兵。」其類甚多,蓋以象占之。



 永明五年,吳興東遷民吳休之家女人雙生二兒,胸以下齊以上合。



 京房《易傳》曰:「野獸入邑,其邑大虛。」又曰:「野獸無故入邑朝廷門及宮府中者,邑逆且虛。」



 永明中,南海王子罕為南兗州刺史,有獐入廣陵城,投井而死,又有象至廣陵,是後刺史安王子敬於鎮被害。



 建武四年春,當郊治圜丘,宿設已畢,夜虎攫傷人。



 建武中,有鹿入景皇寢廟,皆為上崩及禪代也。凡無占者,皆為不應本傳。



 贊曰:木怪夔魍,火為水妃。土產載物,金作明威。形聲異跡,影響同歸。皆由象應,莫不類推。



\end{pinyinscope}