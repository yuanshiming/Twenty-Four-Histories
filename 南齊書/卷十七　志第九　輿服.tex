\article{卷十七 志第九 輿服}

\begin{pinyinscope}

 昔三皇
 乘祗車出谷口,夏氏以奚仲為車正,殷有瑞車,山車垂句是也。《周禮》匠人為輿,以象天地。



 漢武天漢四年,朝諸侯甘泉宮,定輿服制,班於天下。光武建武十三年,得公孫述葆車,輿輦始具。蔡邕創立此志,馬彪勒成漢典,晉摯虞治禮,亦議五輅制度。江左之始,車服多闕,但有金戎,省充庭之儀。



 太興中,太子臨學,無高蓋車,元帝詔乘安車。元、明時,屬車唯九乘。永和中,石虎死後,舊工人奔叛歸國,稍造車輿。太元中,苻堅敗後,又得偽車輦,於是屬車增為十二乘。義熙中,宋武平關、洛,得姚興偽車輦。宋大明改修輦輅,妙盡時華,始備偽氐,復設充庭之制。永明中,更增藻飾,盛於前矣。
 案《周禮》以檢《漢志》,名器不同,晉、宋改革,稍與世異,今記時事而已。



 玉輅,漢金根也。漆畫輪,金塗縱容後路受福輠。兩廂上望板前優遊,通緣金塗鏤釭,碧絞罽,鑿鏤金薄帖。兩廂外織成衣,兩廂裡上施金塗鏤面釘,玳瑁帖。望板廂上金薄帖,金博山,登仙紐,松精。優游上,和鸞鳥立花趺銜鈴,銀帶玳瑁筒瓦,金塗鏤鷿,刀格,織成手匡金花鈿錦衣。優游下,隱膝,里施金塗鏤面釘,織成衣。優游橫前,施玳瑁帖,金塗花釘。優游前,金塗倒龍,後梢鑿銀玳瑁龜甲,金塗花沓。望板,金塗受福望龍諸校飾。抗及諸末,皆螭龍首。龍形板,在車前,銀帶花獸,金塗受福,緣裏邊,鏤鷿玳瑁織成衣。里,金塗鏤面花釘。外,金塗博山、闢邪虎、鳳皇銜花諸校飾。斗蓋,金塗鏤釭,二十八爪支子花,黃錦鬥衣,復碧絹柒布緣油頂,絳系絡,織成顏芚赭舌孔雀毛復錦,緣絞隨陰,懸珠蚌佩,金塗鈴,云朱結,仙人綬,雜色真孔雀眊。一轅,漆畫車衡,銀花帶,衡上金塗博山,四和鸞鳥立花趺銜鈴,所謂「鸞鳥立衡」也。又龍首銜軛,叉髦插翟尾,上下花沓,絳綠系的,望繩八枚。旂十二旒,畫升龍,竿首金塗龍銜火焰幡,真眊。棨戟,織成衣,金塗沓駐及受福,金塗雁鏤釭。漆案立床,在車中,錦復黃絞,為案立衣。錦復黃絞鄣泥。八幅,長九尺,緣紅錦芚帶,織成花芚的。



 五輅,江左相承駕四馬,左右騑
 為六。施絳系游御繩,其重轂貳轄飛軨幡,用赤油金,有紫真褲。左纛,置左騑馬軛上。金筼金加冠,狀如玉華「形」,在馬髦上。方釳,鐵廣數寸,有三孔,插翟尾其中。



 繁纓,金塗紫皮,紫真眊,橫在馬膺前。鏤錫,刻金為馬面當顱。皆如古制。世祖永明初,加玉輅為重蓋,又作麒麟頭,採畫,以馬首戴之。竟陵王子良啟曰:「臣聞車旗有章,載自前史,器必依禮,服無舛法。凡蓋員象天,軫方法地,上無二天之儀,下設兩蓋之飾,求之志錄,恐為乖衷。又假為麟首,加乎馬頭,事不師古,鮮或可施。」建武中,明帝乃省重蓋等。



 金輅。制度校飾如玉輅,而稍減少,亦以金塗。



 象輅。如金輅而制飾又減。



 木輅。制飾如象輅而尤減。



 革輅,如大輅。



 建大麾。赤旗也。首施大焰幡。



 宋升明三年,錫齊王大輅、戎輅各一。乘黃五輅,無大輅、戎輅。左丞王逡之議:「大輅,殷之祭車,故不登周輅之名,而《明堂位》云『大輅,殷輅也』。注云『大輅,木輅也』。



 《月令》『中央土,乘大輅』。注云『殷輅也』。《禮器》『大輅繁纓一就』。注云『大輅,殷之祭天車也』。《周禮》五路,玉路、金路、象路、革路、木路。則周之木路,殷之大輅也。周革路建大白,以即戎,此則
 戎路也。意謂國之大事,在祀與戎,故錫以殷祭天之車,與周之即戎之路。祀則以殷,戎必以周者,明郊天義遠,建前代之禮,即戎事近,故以今世之制。《明堂位》云『魯君孟春乘大路,載十有二旒日月之章,祀帝於郊』。天子以大輅以錫諸侯,良有以也。今木路,即大路也。」太尉左長史王儉議,宜用金輅九旒。時乘黃無副,借用五輅,大朝臨軒,權列三輅。



 玉、金輅,建碧旂。象木輅,建赤旂。永明初,太子步兵校尉伏曼容議,以為:「齊德尚青,五路五牛及五色幡旗,並宜以先青為次。軍容戎事之所乘,犧牲繭握之所薦,並宜悉依尚色。三代服色,以姓音為尚,漢不識音,故還尚其行運之色。今既無善律,則大齊所尚,亦宜依漢道。若有善吹律者,便應還取姓尚。」



 太子僕周顒議:「三代姓音,古無前記,裁音配尚,起自曼容。則是曼容善識姓聲,不復方假吹
 律。何故能識遠代之宮商而更迷皇朝之律呂,而云當今無知吹律以定所尚,宜附漢以從闕邪?皇朝本以行運為所尚,非關不定於音氏。如此,設有善律之知音,不宜遵聲以為尚。」散騎常侍劉朗之等十五人並議駁之,事不行。



 皇太子象輅。校飾如御,抃九旒降龍。



 皇太后皇后重翟車,金塗校具,白地人馬錦帖,廂隱膝後戶,白牙的帖,金塗面釘,漆畫輪,鐵鐺,金塗縱容後路輠,師子轄、抗簷皆施金塗螭頭及神龍雀等諸飾。軛衡上施金博山,又有金塗長角巴首。蓋,金塗,爪支子花二十八,青油俠碧絹黃絞蓋,漆布里。紫顏芚,黃絞紫絞隨陰,碧毛。外上施絳紫系絡。碧旂九旒,棨戟。宋元嘉《東宮儀記》雲中宮僕御重翟金根車,未詳得稱為金根也。



 皇太子妃厭翟車。如重翟,飾微減。



 指南車。四周廂上施屋,指南人衣裙襦天衣,在廂中。上四角皆施龍子竿,縣雜色真孔雀眊,烏布皂復幔,漆畫輪,駕牛,皆銅校飾。



 記里鼓車。制如
 指南,上施華蓋子,糸禁衣漆畫,鼓機皆在內。



 輦車,如犢車,竹蓬。廂外鑿鏤金薄,碧紗衣,織成芚,錦衣。廂裡及仰「頂」隱膝後戶,金塗鏤面,玳瑁帖,金塗松精,登仙花紐,綠四緣,四望紗萌子,上下前後眉,鏤鍱。轅枕長角龍,白牙蘭,玳瑁金塗校飾。漆鄣塵板在蘭前,金銀花獸玃天龍師子鏤面,榆花細指子摩尼炎,金龍虎。扶轅,銀口帶,龍板頭。



 龍轅軛上,金鳳皇鈴璅銀口帶,星後梢,玳瑁帖,金塗香沓,銀星花獸幔竿杖,金塗龍牽,縱橫長簹,背花香柒兆床副。自輦以下,二宮御車,皆綠油幢,絳系絡。御所乘,雙棟。其公主則碧油幢雲。《司馬法》曰「夏后氏輦曰金車,殷曰胡奴車,周曰輜車」,皆輦也。《漢書。叔孫通傳》云「皇帝輦出房」,成帝輦過後宮,此朝宴並用也。《輿服志》云「輦車具金銀丹青採祇雕畫蒲陶之文,乘人以行」。信陽侯陰就見井丹,左右人進輦,是為臣下亦得乘之。晉武帝給安平獻王孚雲母輦。晉中朝又有香衣輦,江左唯御所乘。



 臥輦。校飾如坐輦,不甚服用。



 漆畫輪車,金塗校飾如輦,微有減降。金塗鐺,縱容後輠師子副也。御為群公舉哀臨哭所乘。皇后、太子妃亦乘之。漆畫牽車,小形如輿車,金塗縱容後路師子輠,鐵鐺,錦衣。廂裏隱膝後戶,牙蘭,轅枕梢,幰竿戍棟梁,皆金塗校飾。御及皇太子所乘,即古之羊車也。晉泰始中,中護軍羊琇乘羊車,為司隸校尉劉毅所
 奏。武帝詔曰:「羊車雖無制,非素者所服,免官。」《衛玠傳》云:「總角乘羊車,市人聚觀。」今不駕羊,猶呼牽此車者為羊車云。



 輿車,形如軺車,漆畫,金校飾錦衣。兩廂後戶隱膝牙蘭,皆玳瑁帖,刀格,鏤面花釘。幰竿戍校棟梁。



 下施八㭎,金塗沓,兆床副。人舉之。一曰小輿,小行幸乘之。皇太子亦得於宮內乘之。衣書十二乘,資榆轂輪,箕子壁,綠油衣,廂外綠紗萌,油幢絡,通幰,竿刺代棟梁,柮檽真形龍牽,支子花。轅後伏神抗、承泥、沓,金塗校具。古副車之象也。今亦曰五時副車。



 青萌車,是謂手翕幔車。



 油絡畫安車,公主、王妃、三公特進夫人所乘。漢制,皇后、貴人紫罽軿車。晉皇后乘雲母油畫安車,駕六,以兩轅安車駕五為副。公主畫安車駕六,以兩轅安車駕三為副。公主畫安車駕三,三夫人青交絡安車駕三,皆以紫絳罽軿車駕三為副。九嬪世婦軿車駕二,王公妃特進夫人皂交絡為副。漢賤軺車而貴軿車,晉賤輜軿而貴
 軺車,皆行禮所乘。



 黃屋車,建碧旗九旒,九旒,鸞輅也。漢《輿服志》云:「金根車,蓋黃繒為里,謂之黃屋。」今金、玉輅皆以黃地錦,唯此車以黃繒。皆金塗校具,黃隱隨陰,青毛羽,二十八爪支子花,絳系絡。九命上公所乘。



 青蓋安車,朱轓漆班輪,駕一,左右騑,通幰車為副,諸王禮行所乘。凡車有轓者謂之軒。皂蓋安車,朱轓漆班輪,駕一,通幰牛車為副,三公禮行所乘。



 安車,黑耳皂蓋馬車,朱轓,駕一,牛車為副,國公列侯禮行所乘。



 馬車,駕一,九卿、領、護、二衛、驍游、四軍、五校從郊陵所乘。



 晉制,三公下至九卿,又各安車黑耳一乘,公駕三,特進駕二,卿駕一,復各軺車施黑耳後戶皂輪一乘。



 油絡軺車,尚書令、僕射、中書監、令、尚書、侍中、常侍、中黃門、中書、散騎侍郎,皆駕一牛,朝直所乘。晉制,尚書令施黑耳後戶皂輪,僕射、
 中書監、令直施後戶皂輪,尚書無後戶,皆漆輪轂,今猶然。



 安車,赤屏,駕一;又輅車,施後戶,為副,太子二傅禮行所乘。



 四望車,通幰,油幢絡,班漆輪轂。亦曰皂輪,以加禮貴臣。晉武詔給魏舒、陽燧四望小車。



 三望車,制度如四望。或謂之夾望,亦以加禮貴臣。次四望。



 油幢絡車,制似三望而減。王公加禮者之為常乘,次三望。



 平乘車,竹箕子壁仰,資榆為輪,通幰,竿刺代棟梁,柮檽真形龍牽,金塗支子花紐,轅頭後梢沓伏神承泥。庶人亦然,但不通幰。三公諸王所乘。自四望至平乘,皆銅校飾。



 轀輬車,四輪,飾如金根。四角龍首,施組銜璧,垂五採,析羽葆流蘇,前後雲氣錯畫帷裳,以素為池而黼黻。駕四白駱馬,太僕執轡。貴臣薨,亦如之,羽飾駕御,微有減降。



 《虞書》曰:「予欲觀古人之象,日、月、星辰、山、龍、華蟲,作繢;宗彞、藻、火、粉米、黼、黻,絺繡,以採章施於五色。」天子服備日、月以下,公山、龍以下,侯伯華蟲以下,子男藻、火以下,卿大夫粉米以下。天子六冕,王
 後六服,著在《周官》。公侯以下,咸有名則,佩玉組綬,並具禮文,後代沿革,見《漢志》《晉服制令》,其冠十三品,見蔡邕《獨斷》,並不復具詳。宋明帝泰始四年,更制五輅,議修五冕,朝會饗獵,各有所服,事見《宋注》。舊相承三公以下冕七旒,青玉珠,卿大夫以下五旒,黑玉珠。永明六年,太常丞何諲之議,案《周禮》命數,改三公八旒,卿六旒。尚書令王儉議,依漢三公服,山、龍九章,卿華蟲七章。從之。



 平冕,黑介幘,今謂平天冠。皂表,朱緣裏,廣七尺,長尺二寸,垂珠十二旒,以朱組為纓,如其綬色。



 衣皂上絳下,裳前三幅,後四幅。衣畫而裳繡,為日、月、星辰、山、龍、華蟲、藻、火、粉米、黼、黻十二章。素帶廣四寸,朱裏,以朱綠裨飾其側,要中以朱,垂以綠,垂三尺。中衣,以絳緣其領袖,赤皮韍,絳褲襪,赤鋋抃,郊廟臨朝所服也。漢世,冕用白玉珠為旒。魏明帝好婦人飾,改以珊瑚珠。晉初仍舊,後乃改。



 江左以美
 玉難得,遂用琫珠,世謂之白琁珠。



 袞衣,漢世出陳留襄邑所織。宋末用繡及織成。建武中,明帝以織成重,乃採畫為之,加飾金銀薄,世亦謂為天衣。



 史臣曰:黼黻之設,經緯為用,故五色六章十二衣還相為質也。歷代龍袞,織以成文,今體不勝衣,變易舊法,豈致美黻冕之謂乎!



 通天冠,黑介幘,金博山顏,絳紗袍,皂緣中衣,乘輿常朝所服。舊用駮犀簪導,東昏改用玉。其朝服,臣下皆同。



 黑介幘,單衣,無定色,乘輿拜陵所服。其白帢單衣,謂之素服,以舉哀臨喪。



 遠游冠,太子諸王所冠。太子朱纓,翠羽緌珠節。諸王玄纓,公侯皆同。



 平冕,各以組為纓,王公八旒,衣山、龍九章,卿七旒,衣華蟲七章,並
 助祭所服。皆畫皂絳繒為之。



 進賢冠,諸開國公、侯,鄉、亭侯,卿,大夫,尚書,關內侯,二千石,博士,中書郎,丞、郎,秘書監、丞、郎,太子中舍人、洗馬、舍人,諸府長史,卿,尹、丞,下至六百石令長小吏,以三梁、二梁、一梁為差,事見《晉令》。



 武冠,侍臣加貂蟬,餘軍校武職、黃門、散騎、太子中庶子、二率、朝散、都尉,皆冠之。唯武騎虎賁服文衣,插雉尾於武冠上。



 史臣曰:應劭《漢官》釋附蟬,及司馬彪志並不見侍中與常侍有異,唯言左右珥貂而已。案項氏說雲「漢侍中蟬,刻為蟬像,常侍但為榼而不蟬」,未詳何代所改也。



 法冠,廷尉等諸執法者冠之。



 高山冠,謁者冠之。



 樊噲冠,殿門衛士冠之。



 黑
 介幘冠,文冠;平幘冠,武冠。尚書令、僕射、尚書納言幘,後飾為異。



 童子空頂幘,施假髻,貴賤同服。



 救日蝕,文武官皆免冠,著赤介幘對朝服。赤幘,示威武也。



 褲褶,車駕親戎、中外纂嚴所服。黑冠,帽綴紫褾,以絡帶代鞶帶。中官紫褾,外官絳褾。其纂嚴戎服不綴褾,行留悉同。校獵巡幸,從官戎服革帶鞶帶,文官不纓,武官脫冠。



 袿衣屬大衣,謂之禕衣,皇后謁廟所服。公主會見大首髻,其燕服則施嚴雜寶為佩瑞。袿衣屬用繡為衣,裳加五色,鎖金銀校飾。



 綬,乘輿黃赤綬,黃赤縹綠紺五採。太子朱綬,諸王纁朱綬,皆赤黃縹紺四采。妃亦同。相國綠綟綬,三采,綠紫紺。郡公玄朱。侯伯青朱,子男素朱,皆三采。公世子紫,侯世子青,鄉、亭、關內侯墨綬,皆二采。



 郡國太守、內史青,尚書令、僕、中書監、令、秘書監皆黑,丞皆黃,諸府
 丞亦黃。皇后與乘輿同赤,貴嬪、夫人、貴人紫,王太妃,長公主、封君亦紫綬,六宮青綬青白紅,郡公、侯夫人青綬。



 乘輿傳國璽,秦璽也。晉中原亂,沒胡。江左初無之,北方人呼晉家為「白板天子」。冉閔敗,璽還南。



 別有行信等六璽,皆金為之,亦秦、漢之制也。皇后金璽,太子諸王金璽,皆龜鈕。公侯五等金章,公世子金印,侯銀印,貴嬪、夫人金章,公主、王太妃、封君金印,六宮以下公侯太夫人夫人銀印。其公、將軍金章,光祿大夫、卿、尹、太子傅、諸領護將軍、中郎將、校尉、郡國太守內史、四品五品將軍,皆銀章,尚書令、僕、中書監、令、秘書監丞、太子二率,諸府長史、卿、尹、丞、尉、中丞、都水使者、諸州刺史,皆銅印。



 三臺五省二品文官,皆簪白筆。王公五等及武官不簪,加內侍乃
 簪。



 百官執手板,尚書令、僕、尚書,手板頭復有白筆,以紫皮裹之,名曰「笏」。漢末仲長統謂百司皆宜執之。其肩上紫袷囊,名曰「契囊」,世呼為「紫荷」。



 佩玉,自乘輿以下,與晉、宋制同。建元四年,制王公侯卿尹珠水精,其餘用牙蚌。太官宰人服離支衣,後定。



 贊曰:文物煌煌,儀品穆穆。分別禮數,莫過輿服。



\end{pinyinscope}