\article{卷四十八列傳第二十九 袁彖 孔稚珪 劉繪}

\begin{pinyinscope}

 袁彖,字偉才,陳郡陽夏人也。祖洵,吳郡太守。父覬,武陵太守。彖少有風氣,好屬文及玄言。舉秀才,歷諸王府參軍,不就。覬臨終與兄顗書曰:「史公才識可嘉,足懋先基矣。」史公,彖之小字也。服未闋,顗在雍州起事見誅,宋明帝投顗尸江中,不聽斂葬。彖與舊奴一人,微服潛行求尸,四十餘日乃得,密瘞石頭後崗,身自負土。懷其文集,未嘗離身。明帝崩後,乃改葬顗。從叔司徒粲、外舅征西將軍蔡興宗並器之。



 除安成王征虜參軍,主簿,尚書殿中郎,出為廬陵內
 史,豫州治中,太祖太傅相國主簿,秘書丞。議駁國史,檀超以《天文志》紀緯序位度,《五行志》載當時詳沴,二篇所記,事用相懸,日蝕為災,宜居《五行》。超欲立處士傳。彖曰:「夫事關業用,方得列其名行。今棲遁之士,排斥皇王,陵轢將相,此偏介之行,不可長風移俗,故遷書未傳,班史莫編。一介之善,無緣頓略,宜列其姓業,附出他篇。」



 遷始興王友,固辭。太祖使吏部尚書何戢宣旨令就。遷中書郎,兼太子中庶子。



 又以中書兼御史中丞。轉黃門郎,兼中丞如故。坐彈謝超宗簡奏依違,免官。尋補安西諮議、南平內史。除黃門,未拜,仍轉長史、南郡內史,行荊州事。還為太子中庶子。本州大中正。出為冠軍將軍、監吳興郡事。



 彖性剛,嘗以微言忤世祖,又與王晏不協。世祖在便殿,用金柄刀子治瓜,晏在側曰:「外間有金刀之言,恐不宜用此物。」世祖愕然。窮問所以。晏曰:「袁彖為臣說之。」上銜怒良久,彖
 到郡,坐過用祿錢,免官付東冶。世祖游孫陵,望東冶,曰:「中有一好貴囚。」數日,專駕與朝巨幸冶,履行庫藏,因宴飲,賜囚徒酒肉,敕見彖與語,明日釋之。尋白衣行南徐州事,司徒諮議,衛軍長史,遷侍中。



 彖形體充腴,有異於眾。每從車駕射雉在郊野,數人推扶乃能徒步。幼而母卒,養於伯母王氏,事之如親。閨門中甚有孝義。隆昌元年,卒。年四十八。謚靖子。



 孔稚珪,字德璋,會稽山陰人也。祖道隆,位侍中。父靈產,泰始中罷晉安太守。有隱遁之懷,於禹井山立館,事道精篤,吉日於靜屋四向朝拜,涕泗滂沲。東出過錢塘北郭,輒於舟中遙拜杜子恭墓,自此至都,東向坐,不敢背側。元徽中,為中散、太中大夫。頗解星文,好術數。太祖輔政,沈攸之起兵,靈產密白太祖曰:「攸之兵眾雖彊,以天時冥數而觀,無能為也。」太祖驗其言,擢遷光祿大夫。以簏盛靈
 產上靈臺,令其占候。餉靈產白羽扇、素隱幾,曰:「君性好古,故遺君古物。」



 稚珪少學涉,有美譽。太守王僧虔見而重之,引為主簿。州舉秀才。解褐宋安成王車騎法曹行參軍,轉尚書殿中郎。太祖為驃騎,以稚珪有文翰,取為記室參軍,與江淹對掌辭筆。遷正員郎,中書郎,尚書左丞。父憂去官,與兄仲智還居父山舍。



 仲智妾李氏驕妒無禮,稚珪白太守王敬則殺之。服闋,為司徒從事中郎,州治中,別駕,從事史,本郡中正。



 永明七年,轉驍騎將軍,復領左丞。遷黃門郎,左丞如故。轉太子中庶子,廷尉。江左相承用晉世張、杜律二十卷,世祖留心法令,數訊囚徒,詔獄官詳正舊注。



 先是七年,尚書刪定郎王植撰定律章表奏之,曰:「臣尋《晉律》,文簡辭約,旨通大綱,事之所質,取斷難釋。張斐、杜預同注一章,而生殺永殊。自晉泰始以來,唯斟酌參用。是則吏挾威福之勢,民懷不對之怨,所以溫舒獻
 辭於失政,絳侯忼慨而興嘆。皇運革祚,道冠前王,陛下紹興,光開帝業。下車之痛,每惻上仁,滿堂之悲,有矜聖思。爰發德音,刪正刑律,敕臣集定張杜二注。謹礪愚蒙,盡思詳撰,削其煩害,錄其允衷。取張注七百三十一條,杜注七百九十一條。或二家兩釋,於義乃備者,又取一百七條。其注相同者,取一百三條。集為一書。凡一千五百三十二條,為二十卷。請付外詳校,擿其違謬。」從之。於是公卿八座參議,考正舊注。



 有輕重處,竟陵王子良下意,多使從輕。其中朝議不能斷者,制旨平決。至九年,稚珪上表曰:臣聞匠萬物者以繩墨為正,馭大國者以法理為本。是以古之聖王,臨朝思理,遠防邪萌,深杜姦漸,莫不資法理以成化,明刑賞以樹功者也。伏惟陛下躡曆登皇,乘圖踐帝,天地更築,日月再張,五禮裂而復縫,六樂穨而爰緝。乃發德音,下明詔,降恤刑之文,申慎罰之典,敕臣與公
 卿八座共刪注律。謹奉聖旨,諮審司徒臣子良,稟受成規,創立條緒。使兼監臣宋躬、兼平臣王植等抄撰同異,定其去取。



 詳議八座,裁正大司馬臣嶷。其中洪疑大議,眾論相背者,聖照玄覽,斷自天筆。



 始就成立《律文》二十卷,《錄敘》一卷,凡二十一卷。今以奏聞,請付外施用,宣下四海。



 臣又聞老子、仲尼曰:「古之聽獄者,求所以生之;今之聽獄者,求所以殺之。」



 「與其殺不辜,寧失有罪。」是則斷獄之職,自古所難矣。今律文雖定,必須用之;用失其平,不異無律。律書精細,文約例廣,疑似相傾,故誤相亂,一乖其綱,枉濫橫起。法吏無解,既多謬僻,監司不習,無以相斷,則法書徒明於帙里,冤魂猶結於獄中。今府州郡縣千有餘獄,如令一獄歲枉一人,則一年之中,枉死千餘矣。



 冤毒之死,上干和氣,聖明所急,不可不防。致此之由,又非但律吏之咎,列邑之宰亦亂其經。或以軍勛餘力,或以勞吏暮齒,
 獷情濁氣,忍並生靈,昏心狠態,吞剝氓物,虐理殘其命,曲文被其罪,冤積之興,復緣斯發。獄吏雖良,不能為用。



 使於公哭于邊城,孝婦冤於遐外。陛下雖欲宥之,其已血濺九泉矣。



 尋古之名流,多有法學。故釋之、定國,聲光漢臺;元常、文惠,績映魏閣。



 今之士子,莫肯為業,縱有習者,世議所輕。良由空勤永歲,不逢一朝之賞,積學當年,終為閭伍所蚩。將恐此書永墜下走之手矣。今若弘其爵賞,開其勸慕,課業宦流,班習胄子;拔其精究,使處內局,簡其才良,以居外仕;方岳咸選其能,邑長並擢其術:則皋繇之謨,指掌可致;杜鄭之業,鬱焉何遠!然後姦邪無所逃其刑,惡吏不能藏其詐,如身手之相驅,若絃栝之相接矣。



 臣以疏短,謬司大理。陛下發自聖衷,憂矜刑網,御廷奉訓,遠照民瘼。臣謹仰述天官,伏奏雲陛。所奏繆允者,宜寫律上,國學置律學助教,依《五經》例,國子生有欲讀者,策試上
 過高第,即便擢用,使處法職,以勸士流。



 詔報從納,事竟不施行。



 轉御史中丞,遷驃騎長史,輔國將軍。建武初,遷冠軍將軍、平西長史、南郡太守。稚珪以虜連歲南侵,征役不息,百姓死傷。乃上表曰:匈奴為患,自古而然,雖三代智勇,兩漢權奇,算略之要,二塗而已。一則鐵馬風馳,奮威沙漠;二則輕車出使,通驛虜庭。榷而言之,優劣可睹。今之議者,咸以丈夫之氣恥居物下,況我天威,寧可先屈?吳、楚勁猛,帶甲百萬,截彼鯨鯢,何往不碎?請和示弱,非國計也。臣以為戎狄獸性,本非人倫,鴟鳴狼踞,不足喜怒,蜂目蠆尾,何關美惡。唯宜勝之以深權,制之以遠罝,弘之以大度,處之以蝥賊。豈足肆天下之忿,捐蒼生之命,發雷電之怒,爭蟲鳥之氣!百戰百勝,不足稱雄,橫尸千里,無益上國。而蟻聚蠶攢,窮誅不盡,馬足毛群,難與競逐。漢高橫威海表,窘迫長圍;孝文國富刑清,事屈陵辱;宣帝撫
 納安靜,朔馬不驚;光武卑辭厚禮,寒山無靄。是兩京四主,英濟中區,輸寶貨以結和,遣宗女以通好,長轡遠馭,子孫是賴,豈不欲戰,惜民命也。唯漢武藉五世之資,承六合之富,驕心奢志,大事匈奴。遂連兵積歲,轉戰千里,長驅瀚海,飲馬龍城,雖斬獲名王,屠走凶羯,而漢之器甲十亡其九。故衛霍出關,千隊不反,貳師入漠,百旅頓降,李廣敗於前鋒,李陵沒於後陣,其餘奔北,不可勝數。遂使國儲空懸,戶口減半。好戰之功,其利安在?戰不及和,相去何若?



 自西朝不綱,東晉遷鼎,群胡沸亂,羌狄交橫,荊棘攢於陵廟,豺虎咆於宮闈,山淵反覆,黔首塗地,逼迫崩騰,開闢未有。是時得失,略不稍陳。近至元嘉,多年無事,末路不量,復挑強敵。遂乃連城覆徙,虜馬飲江,青、徐之際,草木為人耳。建元之初,胡塵犯塞;永明之始,復結通和,十餘年間,邊候且息。



 陛下張天造曆,駕日登皇,聲雷宇宙,勢
 壓河岳。而封豕殘魂,未屠劍首,長蛇餘喘,偷窺外甸,烽亭不靜,五載於斯。昔歲蟻壞,瘺食樊、漢,今茲蟲毒,浸淫未已。興師十萬,日費千金,五歲之費,寧可貲計。陛下何惜匹馬之驛,百金之賂,數行之詔,誘此凶頑,使河塞息肩,關境全命,蓄甲養民,以觀彼弊?我策若行,則為不世之福;若不從命,不過如戰失一隊耳。或云「遣使不受,則為辱命」。



 夫以天下為量者,不計細恥;以四海為任者,寧顧小節?一城之沒,尚不足惜;一使不反,曾何取慚?且我以權取貴,得我略行,何嫌其恥?所謂尺蠖之屈,以求伸也。臣不言遣使必得和,自有可和之理;猶如欲戰不必勝,而有可勝之機耳。今宜早發大軍,廣張兵勢,徵犀甲於岷峨,命樓船於浦海。使自青徂豫,候騎星羅,沿江入漢,雲陣萬里。據險要以奪其魂,斷糧道以折其膽,多設疑兵,使精銷而計亂,固列金湯,使神茹而慮屈。然後發衷詔,馳輕驛,辯辭重幣,陳列吉凶。北虜頑
 而愛奇,貪而好貨,畏我之威,喜我之賂,畏威喜賂,願和必矣。陛下用臣之啟,行臣之計,何憂玉門之下,而無款塞之胡哉?



 彼之言戰既殷勤,臣之言和亦慊闊。伏願察兩塗之利害,檢二事之多少,聖照玄省,灼然可斷。所表謬奏,希下之朝省,使同博議。臣謬荷殊恩,奉佐侯岳,敢肆瞽直,伏奏千里。



 帝不納。徵侍中,不行,留本任。



 稚珪風韻清疏,好文詠,飲酒七八斗。與外兄張融情趣相得,又與琅邪王思遠、廬江何點、點弟胤並款交。不樂世務,居宅盛營山水,憑幾獨酌,傍無雜事。門庭之內,草萊不剪,中有蛙鳴,或問之曰:「欲為陳蕃乎?」稚珪笑曰:「我以此當兩部鼓吹,何必期效仲舉。」



 永元元年,為都官尚書,遷太子詹事,加散騎常侍。三年,稚珪疾,東昏屏除,以床輿走,因此疾甚,遂卒。年五十五。贈金紫光祿大夫。



 劉繪,字士章,彭城人,太常悛弟也。父勔,宋末權貴,門多人客,使繪
 與之共語,應接流暢。勔喜曰:「汝後若束帶立朝,可與賓客言矣。」解褐著作郎,太祖太尉行參軍。太祖見而嘆曰:「劉公為不亡也。」



 豫章王嶷為江州,以繪為左軍主簿,隨鎮江陵,轉鎮西外兵曹參軍,驃騎主簿。



 繪聰警有文義,善隸書,數被賞召,進對華敏,僚吏之中,見遇莫及。瑯邪王詡為功曹,以吏能自進。嶷謂僚佐曰:「吾雖不能得應嗣陳蕃,然閣下自有二驥也。」



 復為司空記室錄事,轉太子洗馬,大司馬諮議,領錄事。時豫章王嶷與文惠太子以年秩不同,物論謂宮、府有疑,繪苦求外出,為南康相。郡事之暇,專意講說。上左右陳洪請假南還,問繪在郡何似?既而間之曰:「南康是三州喉舌,應須治幹。



 豈可以年少講學處之邪?」徵還為安陸王護軍司馬,轉中書郎,掌詔誥。敕助國子祭酒何胤撰治禮儀。



 永明末,京邑人士盛為文章談義,皆湊竟陵王西邸。繪為後進領袖,機悟多能。



 時張融、
 周顒並有言工,融音旨緩韻,顒辭致綺捷,繪之言吐,又頓挫有風氣。時人為之語曰:「劉繪貼宅,別開一門。」言在二家之中也。



 魚復侯子響誅後,豫章王嶷欲求葬之,召繪言其事,使為表。繪求紙筆,須臾便成。嶷惟足八字,云「提攜鞠養,俯見成人。」乃歎曰:「禰衡何以過此。」後北虜使來,繪以辭辯,敕接虜使。事畢,當撰《語辭》。繪謂人曰:「無論潤色未易,但得我語亦難矣。」



 事兄悛恭謹,與人語,呼為「使君」。隆昌中,悛坐罪將見誅,繪伏闕請代兄死,高宗輔政,救解之。引為鎮軍長史,轉黃門郎。高宗為驃騎,以繪為輔國將軍,諮議,領錄事,典筆翰。高宗即位,遷太子中庶子,出為寧朔將軍、撫軍長史。



 安陸王寶晊為湘州,以繪為冠軍長史、長沙內史,行湘州事,將軍如故。寶晊妃,悛女也。寶晊愛其侍婢,繪奪取,具以啟聞,寶晊以為恨,與繪不協。



 遭母喪去官。有至性,持喪墓下三年,食粗糲。服闋,為寧朔將軍、
 晉安王征北長史、南東海太守,行南徐州事。繪雖豪俠,常惡武事,雅善博射,未嘗跨馬。



 兄悛之亡,朝議贈平北將軍、雍州刺史,詔書已出,繪請尚書令徐孝嗣改之。



 及梁王義師起,朝廷以繪為持節、督雍梁南北秦四州郢州之竟陵司州之隨郡諸軍事、輔國將軍、領寧蠻校尉、雍州刺史。固讓不就。眾以朝廷昏亂,為之寒心,繪終不受,東昏改用張欣泰。繪轉建安王車騎長史,行府國事。義師圍城,南兗州刺史張稷總城內軍事,與會情款異常,將謀廢立,閑語累夜。東昏殞,城內遣繪及國子博士范雲等送首詣梁王於石頭,轉大司馬從事中郎。中興二年,卒。年四十五。



 繪撰《能書人名》,自云善飛白,言論之際,頗好矜詡。



 弟瑱,字士溫。好文章,飲酒奢逸,不吝財物。滎陽毛惠遠善畫馬,瑱善畫婦人,世並為第一。官至吏部郎。先繪
 卒。



 史臣曰:刑禮相望,勸戒之道。淺識言治,莫辯後先,故宰世之堤防,御民之羈絆。端簡為政,貴在畫一,輕重屢易,手足無從。律令之本,文約旨曠,據典行罰,各用情求。舒慘之意既殊,寬猛之利亦異,辭有出沒,義生增損。舊尹之事,政非一途,後主所是,即為成用。張弛代積,稍至遷訛。故刑開二門,法有兩路,刀筆之態深,舞弄之風起。承喜怒之機隙,挾千金之奸利,剪韭復生,寧失有罪,抱木牢戶,未必非冤。下吏上司,文簿從事,辯聲察色,莫用矜府,申枉理讞,急不在躬,案法隨科,幸無咎悔。至於郡縣親民,百務萌始,以情矜過,曾不待獄,以律定罪,無細非衍。蓋由網密憲煩,文理相背。夫懲恥難窮,盜賊長有,欲求猛勝,事在或然,掃墓高門,為利孰遠。故永明定律,多用優寬,治物不患仁心,見累於弘厚;為令貴在必行,而惡其舛雜也。



 贊曰:袁徇厥戚,猶子為情。稚珪夷遠,奏諫罷兵。士章機悟,立行砥名。



\end{pinyinscope}