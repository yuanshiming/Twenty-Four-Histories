\article{卷十志第二禮下}

\begin{pinyinscope}

 建元四年,高帝山陵,昭皇后應遷祔。祠部疑有祖祭及遣啟諸奠九飯之儀不。



 左僕射王儉議:「奠如大斂。賀循云『從墓之墓皆設奠,如將葬廟朝之禮』。范寧云『將窆而奠』。雖不稱為祖,而不得無祭。」從之。有司又奏:「昭皇后神主在廟,今遷祔葬,廟有虞以安神,神既已處廟,改葬出靈,豈應虞祭?鄭注改葬云『從廟之廟,禮宜同從墓之墓』。事何容異!前代謂應無虞。」左僕射王儉議:「範寧云『葬必有魂車』。若不為其歸,神將安舍?世中改葬,即墓所施靈設祭,何得不祭而毀耶?賀循云『既窆,設奠於墓,以終其事』。雖非正虞,亦粗相似。



 晉
 氏修復五陵,宋朝敬後改葬,皆有虞。今設虞非疑。」從之。



 建元二年,皇太子妃薨,前宮臣疑所服。左僕射王儉議:「《禮記·文王世子》『父在斯為子,君在斯為臣。』且漢魏以來,宮僚充備,臣隸之節,具體在三。昔庾翼妻喪,王允、滕弘謂府吏宜有小君之服,況臣節之重邪?宜依禮為舊君妻齊衰三月,居官之身,並合屬假,朝晡臨哭悉系東宮。今臣之未從官在遠者,於居官之所,屬寧二日半,仍行喪成服,遣箋表,不得奔赴。」從之。



 太子妃斬草乘黃,議建銘旌。僕射王儉議:「禮,既塗棺,祝取銘置於殯東,大斂畢,便應建於西階之東。」



 宋大明二年,太子妃斃,建九旒。有司又議:「斬草日建旒與不?若建旒,應幾旒?及畫龍升降雲何?又用幾翣?僕射王儉議:「旒本是命服,無關於兇事。今公卿以下,平存不能備禮,故在兇乃建耳。東宮秩同上公九命之儀,妃與儲君一體,義不容異,無緣未同常例,別立
 兇旒。大明舊事,是不經詳議,率爾便行耳。今宜考以禮典,不得效尤從失。吉部伍自有桁輅,凶部別有銘旌,若復立旒,復置何處?



 翣自用八。」從之。



 有司奏:「大明故事,太子妃玄宮中有石志。參議墓銘不出禮典。近宋元嘉中,顏延作王球石志。素族無碑策,故以紀德。自爾以來,王公以下,咸共遵用。儲妃之重,禮殊恆列,既有哀策,謂不須石志。」從之。



 有司奏:「穆妃卒哭後,靈還在道,遇朔望,當須設祭不?」王儉議:「既虞卒哭,祭之於廟,本是祭序昭穆耳,未全同卒吉四時之祭也,所以有朔望殷事。蕃國不行權制,宋江夏王妃卒哭以後,朔望設祭。帝室既以卒哭除喪,無緣方有朔望之祭。靈筵雖未升廟堂,而舫中即成行廟,猶如桓玄及宋高祖長沙、臨川二國,並有移廟之禮。豈
 復謂靈筵在途,便設殷事耶?推此而言,朔望不復俟祭。宋懿后時舊事不及此,益可知時議。」從之。



 建元三年,有司奏:「皇太子穆妃以去年七月薨,其年閏九月。未審當月數閏?



 為應以閏附正月?若用月數數閏者,南郡王兄弟便應以此四月晦小祥,至於祥月,不為有疑不?」左僕射王儉議:「三百六旬,尚書明義,文公納幣,春秋致譏。



 《穀梁》云「積分而成月」。《公羊》云「天無是月」。雖然,左氏謂告朔為得禮。



 是故先儒咸謂三年期喪,歲數沒閏,大功以下,月數數閏。夫閏者,蓋是年之餘日,而月之異朔,所以吳商云「含閏以正期,允協情理」。今杖期之喪,雖以十一月而小祥,至於祥縞,必須周歲。凡厭屈之禮,要取象正服。祥縞相去二月,厭降小祥,亦以則之。又且求之名義,則小祥本以年限,考於倫例,則相去必應二朔。今以厭屈而先祥,不得謂此事之非期,事既同
 條,情無異貫,沒閏之理,固在言先。設令祥在此晦,則去縞三月,依附準例,益復為礙。謂應須五月晦乃祥。此國之大典,宜共精詳。並通關八座丞郎,研盡同異。」



 尚書令褚淵難儉議曰:「厭屈之典,由所尊奪情,故祥縞備制,而年月不申。



 今以十一月而祥,從期可知。既計以月數,則應數閏以成典。若猶含之,何以異於縞制?疑者正以祥之當閏,月數相縣。積分餘閏,歷象所弘。計月者數閏,故有餘月,計年者苞含,故致盈積。稱理從制,有何不可?」



 儉又答淵難曰:「含閏之義,通儒所難。但祥本應期,屈而不遂。語事則名體具存,論哀則情無以異。跡雖數月,義實計年,閏是年之歸餘,故宜總而苞之。期而兩祥,緣尊故屈,祥則沒閏,象年所申,屈申兼著,二途具舉。經記之旨,其在茲乎?如使五月小祥,六月乃閏,則祥之去縞,事成二月,是為十一月以象前期,二朔以放後歲,名有區域,不得相參。魯
 襄二十八年『十二月乙未楚子卒』。唯書上月,初不言閏,此又附上之明義也。鄭、射、王、賀唯雲期則沒閏,初不復區別杖期之中祥,將謂不俟言矣。成休甫云『大祥後禫,有閏別數之』,明杖期之祥,不得方於浸縞之末。即恩如彼,就例如此。」淵又據舊義難儉十餘問,儉隨事解釋。



 祠部郎中王圭之議,謂「喪以閏施,功衰以下小祥值閏,則略而不言。今雖厭屈,祥名猶存,異於餘服。計月為數,屈追慕之心,以遠為邇。日既餘分,月非正朔,含而全制,於情唯允。僕射儉議,理據詳博,謹所附同。今司徒淵始雖疑難,再經往反,未同儉議。依舊八座丞郎通共博議為允。以來五月晦小祥,其祥禫自依常限。奏御,班下內外。」詔「可」。



 皇太子穆妃服,尚書左丞兼著作郎王逡問左僕射王儉:「中軍南郡王小祥,應待聞喜不?穆妃七月二十四日薨,聞喜公八月發
 哀,計十一月之限,應在六月。南郡王為當同取六月,則大祥復申一月,應用八月,非復正月,在存親之義,若各自為祥,廬堊相間,玄素雜糅,未審當有此疑不?」儉曰:「送往有已,復生有節,罔極非服制所申,祥縞明示終之斷。相待之義,經記無聞。世人多以廬室衰麻,不宜有異,故相去一二月者,或申以俱除。此所謂任情徑行,未達禮旨。昔撰《喪記》,已嘗言之。遠還之人,自有為而未祭,在家之子,立何辭以不變?禮有除喪而歸者,此則經記之遺文,不待之明據。假使應待,則相去彌年,亦宜必待,乃為衰絰永服以窮生,吉蠲長絕於宗廟,斯不可矣。茍曰非宜,則旬月之間,亦不容申。何者?



 禮有倫序,義無徒設。今遠則不待,近必相須,禮例既乖,即心無取。若疑兄弟同居,吉兇舛雜,則古有異宮之義。設無異宮,則遠還之子,自應開立別門,以終喪事。靈筵祭奠,隨在家之人,再期而毀。所以然者,《奔
 喪禮》云『為位不奠』,鄭玄云『以其精神不存乎此也』。聞哀不時,實緣在遠。為位不奠,益有可安。此自有為而然,不關嫡庶。庶子在家,亦不待嫡矣。而況儲妃正體王室,中軍長嫡之重,天朝又行權制,進退彌復非疑。謂不應相待。中軍祥縞之日,聞喜致哀而已,不受弔慰。及至忌辰變除,昆弟亦宜相就寫情而不對客。此國之大典,宜通關八座丞郎,共盡同異,然後奏御。」司徒褚淵等二十人並同儉議為允,請以為永制。詔「可」。



 建元三年,太子穆妃薨,南郡王聞喜公國臣疑制君母服。儉又議:「《禮》『庶人為國君齊衰』。先儒云『庶人在官,若府史之屬是也』。又諸侯之大夫妻為大人服繐衰七月,以此輕微疏遠,故不得盡禮。今皇孫自是蕃國之王公,太子穆妃是天朝之嫡婦。宮臣得申小君之禮,國官豈敢為夫
 人之敬?當單衣白帢素帶哭於中門外,每臨輒入,與宮官同。」



 永明十一年,文惠太子薨,右僕射王晏等奏:「案《喪服經》『為君之父、長子,同齊衰期』。今至尊既不行三年之典,止服期製,群臣應降一等,便應大功。



 九月功衰,是兄弟之服,不可以服尊。臣等參議,謂宜重其衰裳。減其月數,同服齊衰三月。至於太孫三年既申,南郡國臣,宜備齊衰期服。臨汝、曲江既非正嫡,不得禰先儲,二公國臣,並不得服。」詔依所議。



 又奏:「案《喪服經》雖有『妾為君之長子從君而服』。二漢以來,此禮久廢,請因循前準,不復追行。」詔曰:「既久廢,停便。」



 又奏:「伏尋御服文惠太子期內不奏樂,諸王雖本服期,而儲皇正體宗廟,服者一同,釋服,奏樂、姻娶,便應並通。竊謂二等誠俱是嘉禮,輕重有異:娶婦思嗣,事非全吉,三日不樂,禮有明文。宋世期喪
 降在大功者,婚禮廢樂,以申私戚,通以前典。」詔「依議」。



 又奏:「案禮,詳除皆先於今夕易服,明旦乃設祭。尋比世服臨然後改服,與禮為乖。今東宮公除日,若依例,皇太孫服臨方易服。臣等參議,謂先哭臨竟而後祭之。應公除者,皆於府第變服,而後入臨,行奉慰之禮。」詔「可」。



 建武二年,朝會,時世祖遏密未終,朝議疑作樂不。祠部郎何佟之議:「昔舜受終文祖,義非胤堯,及放勛徂落,遏密三祀。近代晉康帝繼成帝,於時亦不作樂。



 懷帝永嘉元年,惠帝喪制未終,於時江充議云,古帝王相承,雖世及有異,而輕重同禮。」從之。



 建武二年正月,有司以世宗文皇帝今二年正月二十四日再忌日,二十九日大祥,三月二十九日祥禫,至尊及群臣泄哀之儀,應定準。下二學八座丞郎,博士陶韶以為「名立義生,自古之制。文帝
 正號祖宗,式序昭穆,祥忌禫日,皇帝宜服祭服,出太極洩哀。百僚亦祭服陪位」。太常丞李捴議曰:「尋尊號既追,重服宜正,但已從權制,故苴杖不說。至於鉆燧既同,天地亦變,容得無感乎?且晉景獻皇后崩,群臣備小君之服。追尊之後,無違後典,追尊之帝,固宜同帝禮矣。雖臣子一例,而禮隨時異。至尊龍飛中興,事非嗣武,理無深衣之變。但王者體國,亦應吊服出正殿舉哀,百寮致慟,一如常儀。」給事中領國子助教謝曇濟議:「夫喪禮一制,限節兩分。虞祔追亡之情,小祥抑存之禮,斯盡至愛可申,極痛宜屈耳。文皇帝雖君德早凝,民化未洽,追崇尊極,實緣於性。今言臣則無實,論己則事虛。聖上馭宇,更奉天眷,祗禮七廟,非從三後,周忌祥禫,無所依設。」太學博士崔愝同陶韶議,太常沈淡同李撝議,國子博士劉警等同謝曇濟議。



 祠部郎何佟之議曰:「《春秋》之旨,臣子繼君親,雖恩
 義有殊,而其禮則一,所以敦資敬之情,篤方喪之義。主上雖仰嗣高皇,嘗經北面,方今聖曆御宇,垂訓無窮,在三之恩,理不容替。竊謂世宗祥忌,至尊宜吊服升殿,群臣同致哀感,事畢,百官詣宣德宮拜表,仍致哀陵園,以弘追遠之慕。」尚事令王晏等十九人同佟之議。詔「可。」



 海陵王薨,百官會哀。時纂嚴,朝議疑戎服臨會。祠部郎何佟之議:「羔裘玄冠不以吊,理不容以兵服臨喪。宋泰始二年,孝武大祥之日,於時百寮入臨,皆於宮門變戎服,著衣𢂿,入臨畢,出外,還襲戎衣。」從之。



 贊曰:姬制孔作,訓範百王。三千有數,四維是張。損益彞典,廢舉憲章。戎祀軍國,社廟郊庠。冠婚朝會,服紀兇喪。存為盛德,戒在先亡。



\end{pinyinscope}