\article{卷五十九列傳第四十 芮芮虜 河南 氐 羌}

\begin{pinyinscope}

 芮芮虜,塞外雜胡也,編髮左衽。晉世什翼圭入塞內後,芮芮逐水草,盡有匈奴故庭,威服西域。土氣早寒,所居為穹廬氈帳。刻木記事,不識文書。馬畜丁肥,種眾殷盛。常與魏虜為仇敵。



 宋世其國相希利垔解星算數術,通胡、漢語,常言南方當有姓名齊者,其人當興。昇明二年,太祖輔政,遣驍騎將軍王洪軌使芮芮,剋期共伐魏虜。建元元年八月,芮芮主發三十萬騎南侵,去平城七百里,魏虜拒守不敢戰,芮芮主於燕然山下縱獵而歸。上初踐阼,不遑出師。二年、三年,芮芮主頻遣使貢獻貂皮雜物。與上書欲伐魏虜,謂上「足
 下」,自稱「吾」。獻師子皮褲褶,皮如虎皮,色白毛短。



 時有賈胡在蜀見之,云此非師子皮,乃扶拔皮也。



 國相邢基祇羅回奉表曰:夫四象稟政,二儀改度,而萬物生焉。斯蓋虧盈迭襲,歷數自然也。昔晉室將終,楚桓竊命,實賴宋武匡濟之功,故能扶衰定傾,休否以泰。祚流九葉,而國嗣不繼。今皇天降禍於上,宋室猜亂於下。臣雖荒遠,粗窺圖書,數難以來,星文改度,房心受變,虛危納祉,宋滅齊昌,此其驗也。水運遘屯,木德應運,子年垂刈,劉穆之記,崏嶺有不衽之山,京房讖云:「卯金十六,草肅應王。」歷觀圖緯,休征非一,皆云慶鐘蕭氏,代宋者齊。會有使力法度及囗此國使反,採訪聖德,彌驗天縱之姿。故能挾隆皇祚,光權定之業;翼亮天功,濟悖主之難。樹勳京師,威振海外。仗義之功,侔蹤湯、武。冥績既著,寶命因歸,受終之歷,歸於有道。況夫帝無常族,有德必昌,時來之數,唯靈是與。陛下承
 乾啟之機,因乘龍之運。計應符革祚,久已踐極,荒裔傾戴,莫不引領。設未龍飛,不宜沖挹,上違天人之心,下乖黎庶之望。



 皇芮承緒,肇自二儀,拓土載民,地越滄海,百代一族,大業天固。雖吳漢殊域,義同唇齒,方欲剋期中原,龔行天罰。治兵繕甲,俟時大舉。振霜戈於并、代,鳴和鈴於秦、趙,掃殄凶醜,梟剪元惡。然後皇輿遷幸,光復中華,永敦鄰好,侔蹤齊、魯。使四海有奉,蒼生咸賴,荒餘歸仰,豈不盛哉!



 永明元年,王洪軌還京師,經途三萬餘里。洪軌,齊郡臨淄人,為太祖所親信,建武中為青冀二州刺史,私占丁侵虜界,奔敗結氣卒。



 芮芮王求醫工等物,世祖詔報曰:「知須醫及織成錦工、指南車、漏刻、並非所愛。南方治疾,與北土不同。織成錦工,並女人,不堪涉遠。指南車、漏刻、此雖有其器,工匠久不復存,不副為誤。」



 自芮芮居匈奴故庭,十年,丁零胡又南攻芮芮,得其故地。芮芮稍南徙,魏
 虜主元宏以其侵逼,遣偽平元王駕鹿渾、龍驤將軍楊延數十萬騎伐芮芮,大寒雪,人馬死者眾。先是益州刺史劉悛遣使江景玄使丁零,宣國威德。道經鄯善、于闐,鄯善為丁零所破,人民散盡。於闐尤信佛法。丁零僭稱天子,勞接景玄使,反命。



 芮芮常由河南道而抵益州。



 河南,匈奴種也。漢建武中,匈奴奴婢亡匿在涼州界雜種數千人,虜名奴婢為貲,一謂之「貲虜」。鮮卑慕容廆庶兄吐谷渾為氐王。在益州西北,亙數千里。其南界龍涸城,去成都千餘里。大戍有四,一在清水川,一在赤水,一在澆河,一在吐屈真川,皆子弟所治。其王治慕駕川。多畜,逐水草,無城郭。後稍為宮屋,而人民猶以氈廬百子帳為行屋。地常風寒,人行平沙中,沙礫飛起,行迹皆滅。肥地則有雀鼠同穴,生黃紫花;瘦地輒有瘴氣,使人斷氣,牛馬得之,疲
 汗不能行。宋初始受爵命,至宋末,河南王吐谷渾拾寅為使持節、散騎常侍、都督西秦河沙三州諸軍事、車騎大將軍、開府儀同三司、領護羌校尉、西秦河二州刺史。



 建元元年,太祖即本官進號驃騎大將軍。宋世遣武衛將軍王世武使河南,是歲隨拾寅使來獻。詔答曰:「皇帝敬問使持節、散騎常侍、都督西秦河沙三州諸軍事、車騎大將軍、開府儀同三司、領護羌校尉、西秦河二州刺史、新除驃騎大將軍、河南王:寶命革授,爰集朕躬,猥當大業,祗惕兼懷,聞之增感。王世武至,得元徽五年五月二十一日表,夏中濕熱,想比平安。又卿乃誠遙著,保寧遐疆。今詔升徽號,以酬忠款。遣王世武銜命拜授。又仍使王世武等往芮芮,想即資遣,使得時達。



 又奏所上馬等物悉至,今往別牒錦絳紫碧綠黃青等紋各十匹。」



 拾寅子易度侯好星文,嘗求星書,朝議不給。寅卒,三年,以河南王世子吐
 谷渾易度侯為使持節、都督西秦河沙三州諸軍事、鎮西將軍、領護羌校尉、西秦河二州刺史、河南王。永明三年,詔曰:「易度侯守職西蕃,綏懷允緝,忠績兼舉,朕有嘉焉。可進號車騎大將軍。」遣給事中丘冠先使河南道,并送芮芮使。至六年乃還。得玉長三尺二寸,厚一尺一寸。



 易度侯卒,八年,立其世子休留茂為使持節、督西秦河沙三州諸軍事、鎮西將軍、領護羌校尉、西秦河二州刺史。復遣振武將軍丘冠先拜授,并行弔禮。冠先至河南,休留茂逼令先拜,冠先厲色不肯,休留茂恥其國人,執冠先於絕巖上推墮深谷而死。冠先字道玄,吳興人,晉吏部郎傑六世孫也。上初遣冠先,示尚書令王儉,儉答上曰:「此人不啻堪行。」乃再銜命。及死,世祖敕其子雄曰:「卿父受使河南,秉忠守死,不辱王命,我甚賞惜。喪屍絕域,不可復尋,於卿後宦塗無妨,甚有高比。」賜錢十萬,布三十匹。



 氐
 楊氏,與苻氐同出略陽。漢世居仇池,地號百頃,建安中有百頃氐王是也。



 晉世有楊茂苾,後轉彊盛,事見前史。仇池四方壁立,自然有樓櫓卻敵狀,高並數丈。有二十二道可攀緣而升,東西二門,盤道可七里,上有岡阜泉源。氐於上平地立宮室果園倉庫,無貴賤皆為板屋土牆,所治處名洛谷。



 宋元嘉十九年,龍驤將軍裴方明等伐氐,克仇池,後為魏虜所攻,失地。氐王楊難當從兄子文德聚眾茄蘆,宋世加以爵位。文德死,從弟僧嗣、文慶傳代之。難當族弟廣香先奔虜,元徽中,為虜攻殺文慶,以為陰平公、茄蘆鎮主。文慶從弟文弘為白水太守,屯武興,朝議以為輔國將軍、北秦州刺史、武都王、仇池公。



 太祖即位,欲綏懷異俗。建元元年,詔曰:「昔絕國入贄,美稱前冊,殊俗內款,聲流往記。偽虜茄蘆鎮主、陰平郡公楊廣香,怨結同族,釁起親黨,當宋之世,遂舉地降敵。茄蘆失守,華陽
 暫驚,近單使先馳,宣揚皇威,廣香等追其遠世之誠,仰我惟新之化,肉袒請附,復地千里,氐羌雜種,咸同歸順。宜時領納,厚加優恤。



 廣香翻迷反正,可特量所授。部曲酋豪,隨名酬賞。」以廣香為督沙州諸軍事、平羌校尉、沙州刺史。尋進號征虜將軍。



 梁州刺史范柏年被誅,其親將李烏奴懼奔叛,文弘納之。烏奴率亡命千餘人攻梁州,為刺史王玄邈所破,復走還氐中。荊州刺史豫章王嶷遣兵討烏奴,檄梁州能斬送烏奴首,賞本郡,烏奴田宅事業悉賜之。與廣香書曰:夫廢興無謬,逆順有恆,古今共貫。賢愚同察。梁州刺史范柏年懷挾詭態,首鼠兩端,既已被伐,盤桓稽命。遂潛遣李烏奴叛。楊文弘扇誘邊疆荒雜。柏年今已梟禽,烏奴頻被摧破,計其餘燼,行自消夷。今遣參軍行晉壽太守王道寶、參軍事行北巴西新巴二郡太守任湜之、行宕渠太守王安會領銳卒三千,遄塗風邁,浮川
 電掩。又命輔國將軍三巴校尉明惠照、巴郡太守魯休烈、南巴西太守柳弘稱、益州刺史傅琰,並簡徒競鶩,選甲爭馳。雍州水步,行次魏興,並山東僑舊,會于南鄭。



 或汎舟墊江,或飛旌劍道,腹背飆騰,表裏震擊。



 文弘容納叛戾,專為淵藪,外侮皇威,內凌國族。君弈世忠款,深識理順,想即起義,應接大軍,共為掎角,討滅烏奴,剋建忠勤,茂立誠節。沈攸之資十年之積,權百旅之眾,師出境而城潰,兵未戰而自屠,朝廷無遺鏃之費,士民靡傷痍之弊。況蕞爾小豎,方之蔑如,其取殲殄,豈延漏刻!忝以寡昧,分陜司蕃,清氛蕩穢,諒惟任職。此府器械山積,戈旗林聳,士卒剽勁,蓄銳積威,除難剿寇,豈俟徵集!但以剪伐萌菌,弗勞洪斧,撲彼蚊蚋,無假多力。皇上聖哲應期,恩澤廣被,罪止首惡,餘無所問。賞罰之科,具寫如別。



 使道寶步出魏興,分軍溯墊江,俱會晉壽。太祖以文弘背叛,進廣香為
 持節、都督西秦州刺史。廣香子北部鎮將軍郡事炅為征虜將軍、武都太守。以難當正胤楊後起為持節、寧朔將軍、平羌校尉、北秦州刺史、武都王,鎮武興,即文弘從兄子也。



 三年,文弘歸降,復以為征西將軍、北秦州刺史。先是廣香病死,氐眾半奔文弘,半詣梁州刺史崔慧景。文弘遣從子後起進據白水。白水居晉壽上流,西接涪界,東帶益路,北連陰平、茄蘆,為形勝之地。晉壽太守楊公則啟經略之宜,上答曰:「文弘罪不可恕,事中政應且加恩耳。卿若能襲破白水,必加厚賞。」



 世祖即位,進後起號冠軍將軍。永明元年,以征虜將軍炅為沙州刺史、陰平王,將軍如故。二年,八座奏後起勤彰款塞,忠著邊城。進號征虜將軍。四年,後起卒,詔曰:「後起奄至殞逝,惻愴於懷。綏御邊服,宜詳其選。行輔國將軍、北秦州刺史、武都王楊集始,幹局沈亮,乃心忠款,必能緝境寧民、宣揚
 聲教。可持節、輔國將軍、北秦州刺史、平羌校尉、武都王。」後起弟後明為龍驤將軍、白水太守。



 集始弟集朗為寧朔將軍。五年,有司奏集始驅狐剪棘,仰化邊服。母以子貴,宜加榮寵。除集始母姜氏為太夫人,假銀印。九年,八座奏楊炅嗣勤西牧,馳款內昭,宜增戎章,用輝遐外。進號前將軍。



 十年,集始反,率氐、蜀雜眾寇漢川,梁州刺史陰智伯遣軍主寧朔將軍桓盧奴、梁季群、宋囗、王士隆等千餘人拒之,不利,退保白馬。賊眾萬餘人縱兵火攻其城柵,盧奴拒守死戰。智伯又遣軍主陰仲昌等馬步數千人救援。至白馬城東千溪橋,相去數里,集始等悉力攻之,官軍內外奮擊,集始大敗,十八營一時潰走,殺獲數千人。集始奔入虜界。



 隆昌元年,以前將軍楊炅為使持節、督沙州諸軍事、平西將軍、平羌校尉、沙州刺史。



 集始入武興,以城降虜,氐人苻幼孫起義攻之。



 建武二年,氐、虜寇漢中。梁州刺史蕭懿遣前
 氐王楊後起弟子元秀收合義兵,氐眾響應,斷虜運道。虜亦遣偽南梁州刺史仇池公楊靈珍據泥功山以相拒格,元秀病死,苻幼孫領其眾。高宗詔曰:「仇池公楊元秀,氐王苗胤,乃心忠勇,醜虜凶逼,血誠彌厲,宣播朝威,招誘戎種,萬里齊契,響然歸從。誠效顯著,實有可嘉。



 不幸殞喪,淒愴於懷。夫死事加恩,《陽秋》明義。宜追覃榮典,以弘勸獎。贈仇池公。持歸國。」



 氐楊馥之聚義眾屯沮水關,城白馬北。集始遣弟集朗率兵迎拒州軍於黃亙,戰大敗。集始走下辯,馥之據武興。虜軍尋退。馥之留弟昌之守武興,自引兵據仇池。



 詔曰:「氐王楊馥之,世纂忠義,率厲部曲,樹績邊城,克殄姦醜。復內稟朝律,外撫戎荒,款心式昭,朕甚嘉之。以為持節、督北秦雍二州諸軍事、輔國將軍、平羌校尉、北秦州刺史、仇池公。」



 沙州刺史楊炅進號安西將軍。三年,炅死,以炅子崇祖為假節、督沙州軍事、征虜將軍、
 平羌校尉、沙州刺史、陰平王。



 四年,偽南梁州刺史楊靈珍與二弟婆羅、阿卜珍率部曲三萬餘人舉城歸附,送母及子雙健、阿皮於南鄭為質。梁州刺史陰廣宗遣中兵參軍猷王思考率眾救援,為虜所得,婆羅、阿卜珍戰死。靈珍攻集始於武興,殺其二弟集同、集眾。集始窮急,請降。以靈珍為持節、督隴右軍事、征虜將軍、北梁州刺史、仇池公、武都王。永元二年,復以集始為使持節、督秦雍二州軍事、輔國將軍、平羌校尉、北秦州刺史。



 靈珍後為虜所殺。



 自虜陷仇池以後,或得或失。宋以仇池為郡,故以氐封焉。



 宕昌,羌種也。各有酋豪,領部眾汧、隴間。宋末,宕昌王梁彌機為使持節、督河涼二州、安西將軍、東羌校尉、河涼二州刺史、隴西公。建元元年,太祖進號鎮西將軍。又征虜將軍、西涼州刺史羌王像舒彭亦進為持節、平西將軍。後叛降虜。



 永明元年,八座奏前
 使持節、都督河涼二州軍事、鎮西將軍,東羌校尉、河涼二州刺史、隴西公、宕昌王梁彌機,前使持節、平北將軍、西涼州刺史、羌王像舒彭,並著勤西垂,寧安邊境,可復先官爵。詔又可以隴右都帥羌王劉洛羊為輔國將軍。



 機卒。三年,詔曰:「行宕昌王梁彌頡,忠款內附,著績西服,宜加爵命,式隆蕃屏。可使持節、督河涼二州諸軍事、安西將軍、東羌校尉、河涼二州刺史、隴西公、宕昌王。」頡卒。六年,以行宕昌王梁彌承為使持節、督河涼二州諸軍事、安西將軍、東羌校尉、河涼二州刺史、宕昌王。使求軍儀及伎雜書,詔報曰:「知須軍儀等九種,並非所愛。但軍器種甚多,致之未易。內伎不堪涉遠。秘閣圖書,例不外出。《五經集注》、《論語》今特敕賜王各一部。」俗重虎皮,以之送死,國中以為貨。



 史臣曰:氐、胡獷盛,乘運迭起,秦、趙僭差,相系覆滅,餘類蠢蠢,
 被西疆而奄北際。芮芮地窮幽都,戎馬天隔。氐楊密邇華、夷,分民接境,侵犯漢、漾,浸逼狼狐,疆埸之心,窺望威德,梁部多難,於斯為梗。殘羌遺種,際運肇昌,盡隴憑河,遠通南驛,據國稱蕃,並受職命。晉氏衰敗,中朝淪覆,滅餘四夷,庶雪戎禍,授以兵杖,升進軍麾,後代因仍,貪廣聲教,綏外懷遠,先名後實。貿易有無,世開邊利,羽毛齒革,無損於我。若夫九種之事,有囗囗至於此也。



 贊曰:芮芮、河南,同出胡種。稱王僭帝,擅強專統。氐、羌孽餘,散出河、隴。來賓往叛,放命承宗。



 附錄《南齊書》序《南齊書》,八紀,十一志,四十列傳,合五十九篇,梁蕭子顯撰。始江淹已為十志,沈約又為《齊紀》,而子顯自表武帝,別為此書。臣等因校正其訛謬,而敘其篇目,曰:將以是非得失、興壞理亂之故而為法戒,則必得其所托,而後能傳於久,此史之所以作也。然而所托不得其人,則或失其意,或亂其實,或析理之不通,或設辭之不善,故雖有殊功韙德非常之跡,將暗而不章,鬱而不發,而檮杌嵬瑣、奸回兇慝之形,可幸而掩也。



 嘗試論之,古之所謂良史者,其明必足以周萬事之理,其道必足以適天下之用,其智必足以通難知之意,其文必足以發難顯之情,然後其任可得而稱也。何以知其然邪?昔者,唐虞有神明之性,有微妙之德,使由之者不能知,知之者不能名。以為治天下之本,號令之所布,法度之所
 設,其言至約,其體至備,以為冶天下之具。



 而為二《典》者,推而明之,所記者,豈獨其跡邪?並與其深微之意而傳之。小大精粗,無不盡也;本末先後,無不白也。使誦其說者,如出乎其時;求其指者,如即乎其人。是可不謂明足以周萬事之理,道足以適天下之用,智足以通難知之意,文足以發難顯之情者乎?則方是之時,豈特任政者皆天下之士哉?蓋執簡操筆而隨者,亦皆聖人之徒也。兩漢以來為史者,去之遠矣。司馬遷從五帝三王既歿數千載之後,秦火之餘,因散絕殘脫之經,以及傳記百家之說,區區掇拾,以集著其善惡之跡,興廢之端,又創己意以為本紀、世家、八書、列傳之文,斯亦可謂奇矣。然而蔽害天下之聖法,是非顛倒而採摭謬亂者,亦豈少哉!是豈可不謂明不足以周萬事之理,道不足以適天下之用,智不足以通難知之意,文不足以發難顯之情者乎?



 夫自三代以後為史者如遷之文,亦不可不謂俊偉拔出之材、非常之士也,然顧以謂明不足以周萬事之理,道不足以適天下之用,智不足以通難知之意,文不足以發難顯之情者,何哉?蓋聖賢之高致,遷固有不能純達其情而見之於後者矣,故不得而與之也。遷之得失如此,況其他邪?至於宋、齊、梁、陳、後魏、後周之書,蓋無以議為也。



 子顯之於斯文,喜自馳騁,其更改破析刻彫藻繢之變尤多,而其文益下,豈夫材固不可以強而有邪?數世之史既然,故其事跡曖味,雖有隨世以就功名之君,相與合謀之臣,未有赫然得傾動天下之耳目,播天下之口者也。而一時偷奪傾危悖理反義之人,亦幸而不暴著於世,豈非所托不得其人故邪?可不惜哉!蓋史者所以明夫治天下之道也,故為之者亦必天下之材,然後其任可得而稱也。豈可忽哉!豈可忽哉!



 臣恂、臣寶臣、臣穆、臣藻、巨洙、臣覺、臣彥若、臣鞏謹敘目錄昧死上。



\end{pinyinscope}