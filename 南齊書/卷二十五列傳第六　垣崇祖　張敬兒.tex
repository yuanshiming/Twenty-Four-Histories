\article{卷二十五列傳第六 垣崇祖 張敬兒}

\begin{pinyinscope}

 垣崇祖,字敬遠,下邳人也。族姓豪強,石虎世,自略陽徙之於鄴。曾祖敞,為慕容德偽吏部尚書。祖苗,宋武征廣固,率部曲歸降,仍家下邳,官至龍驤將軍、汝南新蔡太守。父詢之,積射將軍,宋孝武世死事,贈冀州刺史。



 崇祖年十四,有幹略,伯父豫州刺史護之謂門宗曰:「此兒必大成吾門,汝等不及也。」刺史劉道隆辟為主簿,厚遇之。除新安王國上將軍。景和世,道隆求出為梁州,啟轉崇祖為義陽王征北行參軍,與道隆同行,使還下邳召募。



 明帝立,道隆被誅。薛安都反,明帝遣張永、沈攸之北討,安都使將裴祖隆、李世雄據
 下邳。祖隆引崇祖共拒戰,會青州援軍主劉彌之背逆歸降,祖隆士眾沮敗,崇祖與親近數十人夜救祖隆,與俱走還彭城。虜既陷徐州,崇祖仍為虜將游兵瑯邪間不復歸,虜不能制。密遣人於彭城迎母,欲南奔,事覺,虜執其母為質。崇祖妹夫皇甫肅兄婦,薛安都之女,故虜信之。肅仍將家屬及崇祖母奔朐山,崇祖因將部曲據之,遣使歸命。太祖在淮陰,板為朐山戍主,送其母還京師,明帝納之。



 朐山邊海孤險,人情未安。崇祖常浮舟舸於水側,有急得以入海。軍將得罪亡叛,具以告虜。虜偽城都將東徐州刺史成固公始得青州,聞叛者說,遣步騎二萬襲崇祖,屯洛要,去朐山城二十里。崇祖出送客未歸,城中驚恐,皆下船欲去。



 崇祖還,謂腹心曰:「賊比擬來,本非大舉,政是承信一說,易遣誑之。今若得百餘人還,事必濟矣。但人情一駭,不可斂集。卿等可急去此二里外大叫而來,
 唱『艾塘義人已得破虜,須戍軍速往,相助逐退』。」船中人果喜,爭上岸。崇祖引入據城,遣羸弱入島,令人持兩炬火登山鼓叫。虜參騎謂其軍備甚盛,乃退。崇祖啟明帝曰:「淮北士民,力屈胡虜,南向之心,日夜以冀。崇祖父伯並為淮北州郡,門族布在北邊,百姓所信,一朝嘯吒,事功可立。第名位尚輕,不足威眾,乞假名號,以示遠近。」明帝以為輔國將軍、北琅邪蘭陵二郡太守。亡命司馬從之謀襲郡,崇祖討捕斬之。數陳計算,欲剋復淮北。時虜聲當寇淮南,明帝以問崇祖。崇祖因啟:「宜以輕兵深入,出其不意,進可立不世之勳,退可絕其窺窬之患。」帝許之。



 崇祖將數百人入虜界七百里,據南城,固蒙山,扇動郡縣。虜率大眾攻之,其別將梁湛母在虜,虜執其母,使湛告部曲曰:「大軍已去,獨住何為!」於是眾情離阻,一時奔退。崇祖謂左右曰:「今若俱退,必不獲免。」乃住後力戰,大敗追者而歸。



 以久勞,封
 下邳縣子。泰豫元年,行徐州事,徙戍龍沮,在朐山南。崇祖啟斷水注平地,以絕虜馬。帝以問劉懷珍,云可立。崇祖率將吏塞之,未成。虜主謂偽彭城鎮將平陽公曰:「龍沮若立,國之恥也,以死爭之。」數萬騎掩至。崇祖馬槊陷陣不能抗,乃築城自守。會天雨十餘日,虜乃退。龍沮竟不立。歷盱眙、平陽、東海三郡太守,將軍如故。轉邵陵王南中郎司馬,復為東海太守。



 初,崇祖遇太祖於淮陰,太祖以其武勇,善待之。崇祖謂皇甫肅曰:「此真吾君也!吾今逢主矣,所謂千載一時。」遂密布誠節。元徽末,太祖憂慮,令崇祖受旨即以家口託皇甫肅,勒數百人將入虜界,更聽後旨。會蒼梧廢,太祖召崇祖領部曲還都,除游擊將軍。沈攸之事平,以崇祖為持節、督兗青冀三州諸軍事,累遷冠軍將軍、兗州刺史。太祖踐阼,謂崇祖曰:「我新有天下,夷虜不識運命,必當動其蟻眾,以送劉昶為辭。賊之
 所沖,必在壽春。能制此寇,非卿莫可。」徙為使持節、監豫司二州諸軍事、豫州刺史,將軍如故。封望蔡縣侯,七百戶。



 建元二年,虜遣偽梁王鬱豆眷及劉昶馬步號二十萬,寇壽春。崇祖召文武議曰:「賊眾我寡,當用奇以制之。當修外城以待敵,城既廣闊,非水不固,今欲堰肥水卻淹為三面之險,諸君意如何?」眾曰:「昔佛貍侵境,宋南平王士卒完盛,以郭大難守,退保內城。今日之事,十倍於前。古來相承,不築肥堰,皆以地形不便,積水無用故也。若必行之,恐非事宜。」崇祖曰:「卿見其一,不識其二。若捨外城,賊必據之,外修樓櫓,內築長圍,四周無礙,表裏受敵,此坐自為擒。守郭築堰,是吾不諫之策也。」乃於城西北立堰塞肥水,堰北起小城,周為深塹,使數千人守之。崇祖謂長史封延伯曰:「虜食而少慮,必悉力攻小城,圖破此堰。見塹狹城小,謂一往可克,當以蟻附攻之。放水一激,急逾三
 峽,事窮奔透,自然沈溺。



 此豈非小勞而大利邪?」虜眾由西道集堰南,分軍東路肉薄攻小城。崇祖著白紗帽,肩輿上城,手自轉式。至日晡時,決小史埭。水勢奔下,虜攻城之眾,漂墜塹中,人馬溺死數千人,眾皆退走。初,崇祖在淮陰見上,便自比韓信、白起,咸不信,唯上獨許之,崇祖再拜奉旨。及破虜啟至,上謂朝臣曰:「崇祖許為我制虜,果如其言。其恒自擬韓、白,今真其人也。」進為都督號平西將軍,增封為千五百戶。



 崇祖聞陳顯達李安民皆增給軍儀,啟上求鼓吹橫吹。上敕曰:「韓、白何可不與眾異!」給鼓吹一部。



 崇祖慮虜復寇淮北,啟徙下蔡戍於淮東。其冬,虜果欲攻下蔡,既聞內徙,乃揚聲平除故城。眾疑虜當於故城立戍,崇祖曰:「下蔡去鎮咫尺,虜豈敢置戍;實欲除此故城。政恐奔走殺之不盡耳。」虜軍果夷掘下蔡城,崇祖自率眾渡淮與戰,大破之,追奔數十里,殺獲千計。上遣使入關參虜消
 息還,敕崇祖曰:「卿視吾是守江東而已邪?所少者食,卿但努力營田,自然平殄殘醜。」敕崇祖修治芍陂田。



 世祖即位,徵為散騎常侍、左衛將軍。俄詔留本任,加號安西。仍遷五兵尚書,領驍騎將軍。初,豫章王有盛寵,世祖在東宮,崇祖不自附結。及破虜,詔使還朝,與共密議。世祖疑之,曲加禮待,酒後謂崇祖曰:「世間流言,我已豁諸懷抱,自今已後,富貴見付也。」崇祖拜謝。崇祖去後,上復遣荀伯玉口敕,以邊事受旨夜發,不得辭東宮。世祖以崇祖心誠不實,銜之。太祖崩,慮崇祖為異,便令內轉。



 永明元年四月九日,詔曰:「垣崇祖兇詬險躁,少無行業。昔因軍國多虞,採其一夫之用。大運光啟,頻煩升擢,溪壑靡厭,浸以彌廣。去歲在西,連謀境外,無君之心,已彰遐邇。特加遵養,庶或悛革。而猜貳滋甚,志興亂階,隨與荀伯玉驅合不逞,窺窬非覬,構扇邊荒,互為表裏。寧朔將軍孫景育究悉姦
 計,具以啟聞。除惡務本,刑茲罔赦。便可收掩,肅明憲辟。」死時年四十四。子惠隆,徙番禺卒。



 張敬兒,南陽冠軍人也。本名茍兒,宋明帝以其名鄙,改焉。父醜,為郡將軍,官至節府參軍。敬兒年少便弓馬,有膽氣,好射虎,發無不中。南陽新野風俗出騎射,而敬兒尤多膂力,求入隊為曲阿戍驛將,州差補府將,還為郡馬隊副,轉隊主。



 稍官寧蠻府行參軍。隨同郡人劉胡領軍伐襄陽諸山蠻,深入險阻,所向皆破。又擊湖陽蠻,官軍引退,蠻賊追者數千人,敬兒單馬在後,沖突賊軍,數十合,殺數十人,箭中左腋,賊不能抗。平西將軍山陽王休祐鎮壽陽,求善騎射人。敬兒自占見寵,為長兼行參軍,領白直隊。泰始初,除寧朔將軍,隨府轉參驃騎軍事,署中兵。



 領軍討義嘉賊,與劉胡相拒於鵲尾州。啟明帝乞本郡,事平,為南陽太守,將軍如故。初,王玄謨為雍州,土斷敬
 兒家屬舞陰,敬兒至郡,復還冠軍。三年,薛安都子柏令、環龍等竊據順陽、廣平,略義成、扶風界,刺史巴陵王休若遣敬兒及新野太守劉攘兵攻討,合戰,破走之。徙為順陽太守,將軍如故。南陽蠻動,復以敬兒為南陽太守。遭母喪還家。朝廷疑桂陽王休範,密為之備,乃起敬兒為寧朔將軍、越騎校尉。



 桂陽事起,隸太祖頓新亭。賊矢石既交,休範白服乘輿往勞樓下,城中望見其左右人兵不多,敬兒與黃回白太祖曰:「桂陽所在,備防寡闕,若詐降而取之,此必可擒也。」太祖曰:「卿若能辦事,當以本州相賞。」敬兒相與出城南,放仗走,大呼稱降。休範喜,召至輿側,回陽致太祖密意,休範信之。回目敬兒,敬兒奪取休範防身刀,斬休範首,休範左右數百人皆驚散,敬兒馳馬持首歸新亭。除驍騎將軍,加輔國將軍。太祖以敬兒人位既輕,不欲便使為襄陽重鎮。敬兒求之不已,乃微動太祖曰:「沈攸之在荊州,公知其欲何年作?不出敬兒以
 防之,恐非公之利也。」



 太祖笑而無言,乃以敬兒為持節、督雍梁二州郢司二郡軍事、雍州刺史,將軍如故,封襄陽縣侯,二千戶。部伍泊沔口,敬兒乘舴艋過江,詣晉熙王燮。中江遇風船覆,左右丁壯者各泅走,餘二小吏沒艙下,叫呼「官」,敬兒兩掖挾之,隨船覆仰,常得在水上,如此翻覆行數十里,方得迎接。失所持節,更給之。



 沈攸之聞敬兒上,遣人伺覘。見雍州迎軍儀甚盛,慮見掩襲,密自防備。敬兒至鎮,厚結攸之,信饋不絕。得其事跡,密白太祖。攸之得太祖書翰,論選用方伯密事,輒以示敬兒,以為反間,敬兒終無二心。元徽末,襄陽大水,平地數丈,百姓資財皆漂沒,襄陽虛耗。太祖與攸之書,令賑貸之,攸之竟不歷意。敬兒與攸之司馬劉攘兵情款,及蒼梧廢,敬兒疑攸之當因此起兵,密以問攘兵,攘兵無所言,寄敬兒馬鐙一雙,敬兒乃為之備。昇明元年冬,攸之反,遣使報敬兒,敬
 兒勞接周至,為設酒食,謂之曰:「沈公那忽使君來,君殊可命。」乃列仗於廳事前斬之,集部曲偵攸之下,當襲江陵。



 時攸之遺太祖書曰:吾聞魚相忘於江湖,人相忘於道術,彼我可謂通之矣。大明之中,謬奉聖主,忝同侍衛,情存契闊,義著斷金,乃分帛而衣,等糧而食。值景和昏暴,心爛形燋,若斯之苦,寧可言盡。吾自分碎首於閣下,足下亦懼滅族於舍人。爾時磐石之心既固,義無貳計,蹙迫時難,相引求全。天道矜善,此理不空,結姻之始,實關於厚。



 及明帝龍飛,諸人皆為鬼矣。吾與足下,得蒙大造,親過夙眷,遇若代臣,錄其心迹,復忝驅使,臨崩之日,吾豫在遺托,加榮授寵,恩深位高。雖復情謝古人,粗識忠節,誓心仰報,期之必死。此誠志竟未申遂,先帝登遐,微願永奪。自爾已來,與足下言面殆絕,非唯分張形跡自然至此,脫枉一告,未常不對紙流涕,豈願相誚於今哉?茍有所懷,不
 容不白。



 初得賢子賾疏,雲得家信,雲足下有廢立之事。安國寧民,此功巍巍,非吾等常人所能信也。俄奉皇太后假令,雲足下潛構深略,獨斷懷抱,一何能壯。但冠雖弊,不可承足,蓋共尊高故耳。足下交結左右,親行殺逆,以免身患,卿當謂龍逢、比干癡人耳。凡廢立大事,不可廣謀,但袁、褚遺寄,劉又國之近戚,數臣地籍實為膏腴,人位並居時望,若此不與議,復誰可得共披心胸者哉?昏明改易,自古有之,豈獨大宋中屯邪?前代盛典,煥盈篇史,請為足下言之。



 群公共議,宜啟太后,奉令而行,當以王禮出第。足下乃可不通大理,要聽君子之言,豈可罔滅天理,一何若茲?《孝經》云「資於事父以事君」。縱為宗社大計,不爾,寧不識有君親之意邪?乃復慮以家危,啖以爵賞,小人無狀,遂行弒害。



 吾雖寡識,竊從古比,豈有為臣而有近日之事邪?使一旦荼毒,身首分離,生自可恨,死者何罪?且
 有登齋之賞,此科出於何文?凡在臣隸,誰不惋駭!華夷扣心,行路泣血。乃至不殯,使流蟲在戶,自古以來,此例有幾?衛國微小,故有弘演,不圖我宋,獨無其人。撫膺惆悵,不能自己。足下與向之殺者何異?人情易反,還成嗟悲,為子君者,無乃難乎!蹊田之譬,豈復有異?管仲有言,君善未嘗不諫。



 足下諫諍不聞,甘崔杼之罪,何惡逆之苦!昔太甲還位,伊不自疑。昌邑之過,不可稱數,霍光荷托,尚共議於朝班,然後廢之。由有湯沐之施,論者不以劫主為名。



 桓溫之心,未忘於篡,海西失道,人倫頓盡,廢之以公,猶禮處之。當溫彊盛,誰能相抗,尚畏懼於形跡,四海不愜,未嘗有樂推之者。伊尹、霍光,名高於臣節,桓氏亦得免於脅奪,凡是諸事,布於書策,若此易曉,豈待指掌!卿常言比跡夷、叔,如何一旦行過桀、跖邪?



 聖明啟運,蒼生重造,普天率土,誰不歌抃!實是披心擊節、奉公忘私之日,而卿
 大收宮妓,劫奪天藏,器械金寶,必充私室,移易朝舊,布置私黨,被甲入殿,內外宮閣管籥,悉關家人。吾不知子孟、孔明遺訓如此?王、謝、陶、庾行此舉止?



 且朱方帝鄉,非親不授,足下非國戚也,一旦專縱自樹,云是兒守臺城,父居東府,一家兩錄,何以異此?知卿防固重復,猜畏萬端,言以御遠,實為防內。若德允物望,夷貊猶可推心共處;如其失理乖道,金城湯池無所用也。文長以戈戟自衛,何解滅亡。吳起有云「義禮不脩,舟中之人皆讎也。」足下既無伍員之痛,茍懷貪惏而有賊宋之心,吾寧捐申包之節邪?



 聞求忠臣者必出孝子之門,卿忠孝於斯盡矣。今竊天府金帛以行姦惠,盜國權爵以結人情,且授非其理,合我則賞,此事已復不可恆用,用之既訖,恐非忠策。



 且受者不感,識者不知,不能遏姦折謀,誠節慨惋。隔硋數千,無因自對,不能知復何情顏,當與足下敘平生舊款?吾聞前
 哲絕交,不出惡言,但此自陳名節於胸心,因告別於千載。放筆增嘆,公私潸淚,想不深怪往言。然天下耳目,豈伊可誣!抑亦當自知投杖無疆,為必先及。



 太祖出頓新亭,報攸之書曰:辱足下誚書,交道不終,為恥已足。欲下便來,何故多罔君子?吾結髮入仕,豈期遠大,蓋感子路之言,每不擇官而宦。逮文帝之世,初被聖明鑒賞;及孝武之朝,復蒙英主顧眄。因此感激,未能自反。及與足下斂袂定交,款著分好,何嘗不勸慕古人國士之心,務重前良忠貞之節?至於契闊杯酒,殷勤攜袖,薦女成姻,志相然諾,義信之篤,誰與間之!又乃景和陵虐,事切憂畏,明帝正位,運同休顯,啟臆論心,安危豈貳!元徽之季,聽高道慶邪言,欲相討伐,發威施敕,已行外內。



 于時臣子鉗口,道路以目。吾以分交義重,患難宜均,犯陵白刃,以相任保。悖主手敕,今封送相示。豈不畏威,念周旋之義耳。推此陰惠,何
 愧懷抱,不云足下猥含禍詖。前遣王思文所牒朝事,蓋情等家國,共詳衷否,虛心小大,必以先輸。問張雍州遷代之日,將欲誰擬?本是逆論來事,非欲代張,乃封此示張,激使見怒。



 若張惑一言,果興怨恨,事負雅素,君子所不可為,況張之奉國,忠亮有本,情之見與,意契不貳邪?又張雍州啟事,稱彼中蠻動,兼民遭水患,敕令足下思經拯之計。吾亦有白,論國如家,布情而往,每思虛達。事之相接,恆必猜離。反謂無故遣信,此乃覘察。平諒之襟,動則相阻,傷負心期,自誰作故?先時足下遣信,尋盟敦舊,厲以篤終,吾止附還白,申罄情本,契然遠要,方固金石。今日舉錯,定是誰恧久言邪?



 元徽末德,勢亡禋祀,足下備聞,無待亟述。太后惟憂,式遵前誥,興毀之略,事屬鄙躬。黜昏樹明,實惟前則,寧宗靜國,何愧前修?廢立有章,足下所允,冠弊之譏,將以何語?封為郡王,寧為失禮?景和無名,方之
 不愈乎?龍逢自匹夫之美,伊、霍則社稷之臣,同異相乘,非吾所受也。登齋有賞,壽寂已蒙之於前;同謀獲功,明皇亦行之於昔。此則接踵成事,誰敢異之!謂其大收宮女,劫奪天藏,器械金寶,必充私室。必若虛設市虎,亦可不翅此言;若以此詐民,天下豈患無眼?



 心茍無瑕,非所耿介。甲仗之授,事既舊典,豈見有任鎮邦家,勳經定主,而可得出入輕單,不資寵衛!斯之患慮,豈直身憂。祗奉此恩,職惟事理。朱方之牧,公卿僉意,吾亦謂微勳之次,無忝一州。且魏、晉舊事,帝鄉蕃職,何嘗豫州必曹,司州必馬?折膠受柱,在體非愧。袁粲據石頭,足下無不可;吾之守東府,來告便謂非。動容見疾,頻笑入戾,乃如是乎!



 袁粲、劉秉,受遇深重,家國既安,不思撫鎮,遂與足下表裏潛規,據城之夜,豈顧社稷。幸天未長亂,宗廟有靈,即與褚衛軍協謀義斷,以時殄滅。想足下聞之,悵然孤沮。小兒忝侍中,代
 來之澤,遇直上臺,便呼一家兩錄。發不擇言,良以太甚。吾之方寸,古列共言,乃以陶、庾往賢,大見譏責,足下自省,詎得以此見貽邪?比縱夷、叔,論吾則可,行過桀、跖,無乃近誣哉!



 謂吾不朝,此則良誨,朝之與否,想更問之。足下受先帝之恩施,擁戎西州,鼎湖之日,率土載奔,而宴安中流,酣飲自若,即懷狼望,陵侮皇朝。晉熙殿下以皇弟代鎮,而斷割候迎,罔蔑宗子,驅略士馬,悉以西上,郢中所遺,僅餘劣弱。



 昔征茅不入,猶動義師;況荊州物產,雍、裛、交、梁之會,自足下為牧,薦獻何品?良馬勁卒,彼中不無,良皮美罽,商賂所聚,前後貢奉,多少何如?唯聞太官時納飲食耳。桂陽之難,坐觀成敗,自以雍容漢南,西伯可擬。賴原即夭世,非望亦消。又招集逋亡,斷遏行侶。治舟試艦,恆以朝廷為旗的;秣馬按劍,常願天下有風塵。為人臣者,固若是邪?至乃不遵制書,敕下如空,國思莫行,命令擁隔,
 詔除郡縣,輒自板代,罷官去職,禁還京師。凶人出境,無不千里尋躡,而反募臺將,來必厚加給賞。太妃遣使市馬,齎寶往蜀,足下悉皆斷折,以為私財,此皆遠邇共聞,暴於視聽。主上睿明當璧,宇縣同慶,絕域奉贄,萬國通書,而盤桓百日,始有單騎,事存送往,於此可徵。不朝如此,誰應受誚?反以見呵,非所反側。今乃勒兵以窺象館,長戟以指魏闕,不亦為忠臣孝子之所痛心疾首邪?賢子元琰獲免虎口,及凌波西邁,吾所發遣。猶推素懷,不畏嗤嗤。



 足下尚復滅君臣之紀,況吾布衣之交乎?遂事不諫,既往難咎。今六師西向,為足下憂之。



 敬兒告變使至,太祖大喜,進號鎮軍將軍,加散騎常侍,改為都督,給鼓吹一部。攸之於郢城敗走,其子元琰與兼長史江乂、別駕傅宣等守江陵城。敬兒軍至白水,元琰聞城外鶴唳,謂是叫聲,心懼欲走。其夜,乂、宣開門出奔,
 城潰,元琰奔寵州,見殺。百姓既相抄敓,敬兒至江陵,誅攸之親黨,沒入其財物數十萬,悉以入私。攸之於湯渚村自經死,居民送首荊州,敬兒使盾擎之,蓋以青傘,徇諸市郭,乃送京師。進號往西將軍,爵為公,增邑為四千戶。



 敬兒於襄陽城西起宅,聚財貨。又欲移羊叔子墮淚碑,於其處立臺,綱紀諫曰:「羊太傅遺德,不宜遷動。」敬兒曰:「太傅是誰?我不識也。」敬兒弟恭兒,不肯出官,常居上保村中,與居民不異。敬兒呼納之甚厚,恭兒月一出視敬兒,輒復去。恭兒本名豬兒,隨敬兒改名也。



 初,敬兒既斬沈攸之,使報隨郡太守劉道宗,聚眾得千餘人,立營頓。司州刺史姚道和不殺攸之使,密令道宗罷軍。及攸之圍郢,道和遣軍頓堇城為郢援,事平,依例蒙爵賞。敬兒具以啟聞。建元元年,太祖令有司奏道和罪,誅之。道和字敬邕,羌主姚興孫也。父萬壽,偽鎮東大將軍,降宋武帝,卒於散騎
 侍郎。道和出身為孝武安北行佐,有世名,頗讀書史。常誑人云:「祖天子,父天子,身經作皇太子。」



 元徽中為游擊將軍,隨太祖新亭破桂陽賊有功,為撫軍司馬,出為司州,疑怯無斷,故及於誅。



 三年,徵敬兒為護軍將軍,常侍如故。敬兒武將,不習朝儀,聞當內遷,乃於密室中屏人學揖讓答對,空中俯仰,如此竟日,妾侍竊窺笑焉。太祖即位,授侍中,中軍將軍。以敬兒秩窮五等,一仍前封。建元二年,遷散騎常侍,車騎將軍,置佐史。太祖崩,敬兒於家竊泣曰:「官家大老天子,可惜!太子年少,向我所不及也。」



 遺詔加敬兒開府儀同三司,將拜,謂其妓妾曰:「我拜後,應開黃閣。」因口自為鼓聲。既拜,王敬則戲之,呼為褚淵。敬兒曰:「我馬上所得,終不能作華林閣勛也。」敬則甚恨。



 敬兒始不識書,晚既為方伯,乃習學讀《孝經》、《論語》。於新林慈姥廟為妾乞兒祝神,自稱三公。然而意知滿足,初得鼓吹,羞便
 奏之。



 初娶前妻毛氏,生子道文。後娶尚氏,尚氏有美色,敬兒棄前妻而納之。尚氏猶居襄陽宅不自隨,敬兒慮不復外出,乃迎家口悉下至都。啟世祖,不蒙勞問,敬兒心疑。及垣崇祖死,愈恐懼,妻謂敬兒曰:「昔時夢手熱如火,而君得南陽郡。



 元徽中,夢半身熱,而君得本州。今復夢舉體熱矣。」有閹人聞其言,說之。事達世祖。敬兒又遣使與蠻中交關,世祖疑其有異志。永明元年,敕朝臣華林八關齋,於坐收敬兒。敬兒左右雷仲顯知有變,抱敬兒而泣。敬兒脫冠貂投地曰:「用此物誤我。」少日,伏誅。詔曰:「敬兒蠢茲邊裔,昏迷不脩。屬值宋季多難,頗獲野戰之力。拔迹行伍,超登非分。而愚躁無已,矜伐滋深。往蒞本州,久苞異志。在昔含弘,庶能懲革。位班三槐,秩窮五等,懷音靡聞,姦回屢構。去歲迄今,嫌貳滋甚。鎮東將軍敬則、丹陽尹安民每侍接之日,陳其兇狡,必圖反噬。朕猶謂恩義所感,本質
 可移。頃者已來,釁戾遂著,自以子弟在西,足動殊俗,招扇群蠻,規擾樊、夏。假托妖巫,用相震惑,妄設徵祥,潛圖問鼎。履霜於開運之辰,堅冰於嗣業之世,此而可忍,孰不可容!天道禍淫,逆謀顯露。建康民湯天獲商行入蠻,備睹奸計,信驛書翰,證驗炳明。便可收掩,式正刑闢;同黨所及,特皆原宥。」



 子道文,武陵內史,道暢,征虜功曹,道固弟道休,並伏誅,少子道慶,見宥。後數年,上與豫章王嶷三日曲水內宴,舴艋船流至御坐前覆沒,上由是言及敬兒,悔殺之。



 恭兒官至員外郎。在襄陽聞敬兒敗,將數十騎走入蠻中,收捕不得。後首出,上原其罪。



 史臣曰:平世武臣,立身有術,若非愚以取信,則宜智以自免。心跡無阻,乃見優容。崇祖恨結東朝,敬兒情疑鳥盡,嗣運方初,委骨嚴憲。若情非發憤,事無感激,功名之間,不足為也。



 贊曰:崇祖為將,志懷馳逐。規搔淮部,立勛豫牧。敬兒蒞雍,深心防楚。豈不劬勞,實興師旅。烹犬藏弓,同歸異緒。



\end{pinyinscope}