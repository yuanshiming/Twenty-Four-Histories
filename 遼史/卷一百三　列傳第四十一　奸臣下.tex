\article{卷一百三 列傳第四十一 奸臣下}

\begin{pinyinscope}

 ○蕭余里也耶律合魯蕭得裏特蕭訛都斡蕭達魯古耶律塔不也蕭圖古辭



 蕭余里也,字訛都碗,國舅阿剌次子。便佞滑稽,善女工。重熙間,以外戚進。清寧初,補祗候郎君,尚鄭國公主,拜駙馬都尉,累遷南面林牙。以父阿剌為蕭革所譖,出余里也為奉先軍節度使。十年冬,召為北面林牙。咸雍中,會有告余里也與族人術哲謀害耶律乙辛,按無狀,出為寧遠軍節度使。自後余里也揣乙辛意,傾心事之,薦
 為國舅詳穩。大康初,封遼西郡王。時乙辛擅恣,凡不附己者出之,乃引余里也為北府宰相,兼知契丹行宮都部署事。及乙辛謀構皇太子,余里也多助成之,遂知北院樞密事,賜推誠協贊功臣。以女侄妻乙辛子綏也,恃勢橫肆,至有無君之語,朝野側目。帝出乙辛知南院大王事,坐與乙辛黨,以天平軍節度使歸第。尋拜西北路招討使。以母憂去官,卒。



 耶律合魯,字胡都堇,六院舍利古直之後。柔佞喜茍合。仕清寧初。時乙辛引用群小,合魯附之,遂見委任,俄擢南面林牙。乙辛譖皇太子,殺忠直,合魯多預其謀。弟
 吾也亦黨乙辛,時號「二賊」。乙辛薦為北院大王,卒。吾也亦至南院大王。



 蕭得裏特,遙輦窪可汗宮分人。善阿意順色。清寧初,乙辛用事,甚見引用,累遷北面林牙、同知北院宣徽使事。及皇太子廢,遣得裏特監送上京。得裏特促其行,不令下車,起居飲食數加陵侮,至則築圜堵囚之。大康中,遷西南招討使,歷順義軍節度使,轉國舅詳穩。壽隆五年,坐怨望,以老免死,闔門籍興聖宮,貶西北統軍司,卒。二子:得末、訛裏,乾統間以父與乙辛謀,伏誅。



 蕭訛都斡,國舅少父房之後。咸雍中,補牌印郎君。大康
 三年,樞密使乙辛陰懷逆謀,乃令護衛太保耶律查剌誣告耶律撒剌等廢立事。詔按無狀,皆補外。頃之,訛都斡希乙辛意,欲實其事,與耶律塔不也等入闕誣首:「耶律撒剌等謀害乙辛,欲立皇太子事,臣亦預謀。今不自言,恐事泄連坐。」帝果怒,徙皇太子於上京。訛都斡尚皇女趙國公主,為駙馬都尉。後與乙辛議不合,銜之,復以車服僭擬人主,被誅。訛都斡臨刑,語人曰:「前告耶律撒刺事,皆乙辛教我。恐事彰,殺我以滅口耳!」



 蕭達魯古,遙輦嘲古可汗宮分人。性奸險。清寧間,乙辛為樞密使,竊權用事,陰懷逆謀。達魯古比附之,遂見獎
 拔,稍遷至旗鼓拽剌詳穩。乙辛欲害太子,以達魯古凶果可使,遣與近侍直長撒把詣上京,同留守蕭撻得夜引力士至囚室,紿以有赦,召太子出,殺之,函其首以歸,詐云疾薨。以達魯古為國舅詳穩。達魯古恐殺太子事白,出入常佩刀,有急召即欲自殺。乾統間,詔樞密使耶律阿思大索乙辛黨人,達魯古以賂獲免。後以疾卒。



 耶律塔不也,仲父房之後。以善擊鞠幸於上,凡馳騁,鞠不離杖。咸雍初,補祗候郎君。與耶律乙辛善,故內外畏之。及太子被譖,按無跡,塔不也附乙辛,欲實其誣,與訛都斡等密奏:「太子謀亂事本實,臣不首,恐事覺連坐。」帝
 信之,廢太子。改延慶宮副使。壽隆元年,為行宮都部署。天祚嗣位,以塔不也黨乙辛,出為特免部節度使。及樞密使耶律阿思大索乙辛舊黨,塔不也以賂獲免。徙敵烈部節度使,復為敦睦宮使。天慶元年,出為西北路招討使。以疾卒。



 蕭圖古辭,字何寧,楮特部人。仕重熙中,以能稱,累遷左中丞。清寧初,歷北面林牙,改北院樞密副使。辨敏,善伺顏色,應對合上意。皇太后嘗曰:「有大事,非耶律化哥、蕭圖古辭不能決。」眷遇日隆。知北院樞密使事。六年,出知黃龍府。八年,拜南府宰相。頃之,為北院樞密使,詔許便
 宜從事。為人奸佞有餘,好聚斂,專愎,變更法度。為樞密數月,所薦引多為重元黨與,由是免為庶人。後沒入興聖宮,卒。



 論曰:舜流共工,孔子誅少正卯,治奸之法嚴矣。後世不是之察,反以為忠而信任之,不至於流毒宗社而未已。道宗之於乙辛是也。當其留仁先,討重元,若真為國計者;不知包藏禍心,待時而發耳。一旦專權,又得孝傑、燕哥、十三為之腹心,故肆惡而無忌憚。始誣皇后,又殺太子及其妃,其禍之酷,良可悲哉。嗚呼!君子所親,莫皇后、太子若也。奸臣殺之而不知,群臣言之而不悟。一時忠
 讜,廢戮幾盡。雖黑山親見官屬之盛,僅削一字王號,至私藏甲兵然後誅之。籲!乙辛之罪,固非一死可謝天下,抑亦道宗不明無斷有以養成之也。如蕭余里也輩,忘君黨惡,以饕富貴,雖幸而死諸牖下,其得免於遺臭之辱哉!



\end{pinyinscope}