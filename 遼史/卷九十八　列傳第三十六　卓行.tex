\article{卷九十八 列傳第三十六 卓行}

\begin{pinyinscope}

 蕭札刺耶律官奴蕭蒲離不遼之共國任事,耶律、蕭二族而已。二族之中,有退然自足,不淫於富貴,不詘於聲利,可以振頹風,激薄俗,亦足嘉尚者,得三人焉。作《卓行傳》。



 蕭札刺,字虛輦,北府宰相排押之弟。性介特,不事生業。



 保寧間,以戚屬進,累遷寧遠軍節度使。秩滿裏居,澹泊自適。統和末,召為南京馬步軍都指揮使。以疾求退,不
 聽,遷夷離畢。又以疾辭,許之。遂入頡山,杜門不出。上嘉其志,不復徵,札刺自是家於頡山。親友或過之,終日言不及世務。



 凡宴游相邀,亦不拒。一歲山居過半,與世俗不偶。耶律資忠重之,目日頡山老人。卒。



 耶律官奴,字奚隱,林牙斡魯之孫。沉厚多學,詳於本朝世系。嗜酒好佚。初,徵為宿直將軍。重熙九年,以疾去官。上以官奴屬尊,欲成其志,乃許自擇一路節度使。官奴辭曰:「臣愚鈍,不任官使。」加歸義軍節度使,輒請致政。



 官奴與歐里部人蕭哇友善,哇謂官奴曰:「仕不能致主澤民,成大功烈,何屑屑為也!吾與若居林下,以枕簟自隨,
 觴詠自樂,雖不官,無慊焉。」官奴然之。時稱「二逸」。乾統間,官奴卒。



 蕭蒲離不,字桵懶,魏國王惠之四世孫。父母蚤喪,鞠於祖父兀古匿。性孝悌。年十三,兀古匿卒,自以早失怙恃,復遭祖喪,哀毀逾禮,族里嘉嘆。嘗謂人曰:「我於親不得終養,今誰為訓者?茍不自勉,何以報鞠育恩!」自是力學,於文藝無不精。



 乾統間,以兀古匿之故召之,不應。常與親識游獵山水,奉養無長物僕隸,欣欣如也。或曰:「公胡不念以嗣先世功名?」



 答曰:「自度不足以繼先業,年逾強仕,安能益主庇民!」



 累徵,皆以疾辭。



 晚年,謝絕人事,卜居
 抹古山,屏遠葷茹,潛心佛書,延有道者談論彌日。人間所得何如,但曰:「有深樂!惟覺六鑿不相攘,餘無知者。」一日,易服,無疾而逝。



 論曰:「隱,固未易為也,而亦未可輕以與人。若札刺謝職不談時務,官奴兩辭節鎮,蒲離不召而不赴,雖未足謂之隱;然在當時能知內外之分,甘於肥遁,不猶愈於求富貴利達而為妻妾羞者哉!故稱卓行可也。」



\end{pinyinscope}