\article{卷三十八志第八 地理志二 東京道}

\begin{pinyinscope}

 東京遼陽府,本朝鮮之地。周武王釋箕子囚,去之朝鮮,因以封之。作八條之教,尚禮義,富農桑,外戶不閉,人不為盜。傳四十餘世。燕屬真番、朝鮮,始置吏、築障。秦屬遼東外徼。漢初,燕人滿王故空地。武帝元封三年,定朝鮮為真番、臨屯、樂浪、玄菟四郡。後漢出入青、幽二州,遼東、
 玄菟二郡,沿革不常。漢末為公孫度所據,傳子康;孫淵,自稱燕王,建元紹漢,魏滅之。晉陷高麗,後歸慕容垂;子寶,以勾麗王安為平州牧居之。元魏太武遣使至其所居平壤城,遼東京本此。



 唐高宗平高麗,於此置安東都護府;後為渤海大氏所有。大氏始保挹婁之東牟山。武後萬歲通天中,為契丹盡忠所逼,有乞乞仲象者,度遼水自固,武后封為震國公。傳子祚榮,建都邑,自稱震王,並吞海北,地方五千里,兵數十萬。中宗賜所都曰忽汗州,封渤海郡王。十有二世至彞震,僭號改元,擬建宮闕,有五京、十五府、六十二州,為遼東盛國。忽汗州即故平
 壤城也,號中京顯德府。太祖建國,攻渤海,拔忽汗城,俘其王大諲撰,以為東丹王國,立太子圖欲為人皇王以主之。神冊四年,葺遼陽故城,以渤海、漢戶建東平郡,為防禦州。天顯三年,遷東丹國民居之,升為南京。城名天福,高三丈,有樓櫓,幅員三十里。八門:東曰迎陽,東南曰韶陽,南曰龍原,西南曰顯德,西曰大順,西北曰大遼,北曰懷遠,東北曰安遠。宮城在東北隅,高三丈,具敵樓,南為三門,壯以樓觀,四隅有角樓,相去各二里。宮墻北有讓國皇帝御容殿。大內建二殿,不置宮嬪,唯以內省使副、判官守之。《大東丹國新建南京碑銘》,在宮門之南。外
 城謂之漢城,分南北市,中為看樓;晨集南市,夕集北市。街西有金德寺;大悲寺;附馬寺,鐵幡竿在焉;趙頭陀寺;留守衛;戶部司;軍巡院,歸化營軍千餘人,河、朔亡命,皆籍於此。東至北烏魯虎克四百里,南至海邊鐵山八百六十里,西至望平縣海口三百六十里,北至挹婁縣、範河二百七十里。東、西、南三面抱海。遼河出東北山口為範河,西南流為大口,入於海;東梁河自東山西流,與渾河合為小口,會遼河入於海,又名太子河,亦曰大梁水;渾河在東梁、範河之間;沙河出東南山西北流,徑蓋州入於海。有蒲河;清河;浿水,亦曰泥河,又曰蓒芋濼,水多
 蓒芋之草;駐蹕山,唐太宗征高麗,駐蹕其顛數曰,勒石紀功焉,俗稱手山,山顛平石之上有掌指之狀,泉出其中,取之不竭。又有明王山、白石山——亦曰橫山。天顯十三年,改南京為東京,府曰遼陽。



 戶四萬六百因。轄州、府、軍、城八十七。統縣九:遼陽縣。本渤海國金德縣也。漢壩水縣,高麗改為勾麗縣,渤海為常樂縣。戶一千五百。



 仙鄉縣。本漢遼隊縣,渤海為永豐縣八神仙傳》云:「仙人白仲理能煉神丹,點黃金,以救百姓。」戶一千五百。



 鶴野縣。本漢居就縣地,渤海為雞山縣。昔丁令成家
 此,去家千年,化鶴來歸,集於華表柱,以咮畫表云:「有鳥有鳥丁令威,去家千年今來歸;城郭雖是人民非,何不學仙塚累累。」戶一千二百。



 析木縣。本漢望平縣地,渤海為花山縣。戶一千。



 紫蒙縣。本漢鏤芳縣地。後拂涅國置東平府,領蒙州紫蒙縣。後徙遼城,並入黃嶺縣。渤海後為紫蒙縣。戶一千。



 興遼縣。本漢平郭縣地,渤海改為長寧縣。唐元和中,渤海王大仁秀南定新羅,北略諸部,開置群邑,遂定今名。戶一千。



 肅慎縣。以渤海戶置。



 歸仁縣。



 順化縣。



 開州,鎮國軍,節度。本濊貌地,高麗為慶州,渤海為東京龍原府。有宮殿。都督慶、鹽、穆、賀因州事。故縣六:曰龍原、永安、烏山、壁谷、熊山、白楊,皆廢。疊石為城,周圍二十里。唐薛仁貴徵高麗,與其大將溫沙門戰熊山,擒善射者於石城,即此。太祖平渤海,徙其民於大部落,城遂廢。聖宗代高麗還,周覽城基,復加完葺。開泰三年,遷雙、韓二州千餘戶實之,號開封府開遠軍,節度;更名鎮國軍。隸東京留守,兵事屬東京統軍司。統州三、縣一。



 開遠縣。本柵城地,高麗為龍原縣,渤海因之,遼初廢。



 聖宗東討,復置以軍額。民戶一千。



 鹽州。本渤海龍河郡,故縣四:海陽、接海、格川、龍河,皆廢。戶三百。隸開州。相去一百四十里。



 穆州,保和軍,刺史。本渤海會農郡,故縣四:會農、水歧、順化、美縣,皆廢。戶三百。隸開州。東北至開州一百二十里。統縣一:會農縣。



 賀州,刺史。本渤海吉理郡,故縣四:洪賀、送減、吉理、石山,皆廢。戶三百。隸開州。



 定州,保寧軍。高麗置州,故縣一,曰定東。聖宗統和十三年升軍,遷遼西民實之。隸東京留守司。統縣一:定東縣。高麗所置,遼徒遼西民居之。戶八百。



 保州,宣義軍,節度。高麗置州,故縣一,曰來遠。聖宗以高麗王詢擅立,問罪不服,統和末,高麗降,開泰三年取其保、定二州,於此置榷場。隸東京統軍司。統州、軍二,縣一:來遠縣。初徙遼西諸縣民實之,又徙奚、漢兵七百防戍焉。



 戶一千。



 宣州,定遠軍,刺史。開泰三年徒漢戶置。隸保州。



 懷化軍,下,刺史。開泰三年置。隸保州。



 辰州,奉國軍,節度。本高麗蓋牟城。唐太宗會李世勑攻破蓋牟城,即此。渤海改為蓋州,又改辰州,以辰韓得名。井邑駢列,最為沖會。遼徙其民於祖州。初曰長平軍。戶二千。



 隸東京留守司。統縣一:建安縣。



 盧州,玄德軍,刺史。本渤海杉盧郡,故縣五:山陽、杉盧、漢陽、白巖、霜巖,皆廢。戶三百。在東京一百三十里。



 兵事屬南女直湯河司。統縣一:熊岳縣。西至海一十五里,傍海有熊岳山。



 來運城。本熟女直地。統和中伐高麗,以燕軍驍猛,置兩
 指揮,建城防戍。兵事屬東京統軍司。



 鐵州,建武軍,刺史。本漢安市縣,高麗為安市城。唐太宗攻之不下,薛仁貴白衣登城,即此。渤海置州,故縣四:位城、河端、蒼山、龍珍,皆廢。戶一千。在京西南六十里。統縣一:湯池縣。興州,中興軍,節度。本漢海冥縣地。渤海置州,故縣三:盛吉、蒜山、鐵山,皆廢。戶二百。在京西南三百里。



 湯州。本漢襄平縣地。渤海置州,故縣五:靈峰、常豐、白石、均谷、嘉利,皆廢。戶五百。在京西北一百里。



 崇州,隆安軍,刺史。本漢長岑縣地。渤海置州,故縣三:崇山、溈水、綠城,皆廢。戶五百。在京東北一百五十里。統縣一:崇信縣。



 海州,南海軍,節度。本沃沮國地。高麗為沙卑城,唐李世勑嘗攻焉。渤海號南京南海府。疊石為城,幅員九里,都督沃、晴、椒三州。故縣六:沃沮、鷲巖、龍山、海濱、升平、靈泉,皆廢。太平中,大延琳叛,南海城堅守,經歲不下,別部酋長皆被擒,乃降。因盡徙其人於上京,置遷遼縣,移澤州民來實之。戶一千五百。統州二、
 縣一:臨溟縣。



 耀州,刺史。本渤海椒州;故縣五,椒山,貉嶺、澌泉、尖山、巖淵,皆廢。戶七百。隸海州。東北至海州二百里。統縣一:巖淵縣。東界新羅,故平壤城在縣西南。東北至海州一百二十里。



 嬪州,柔遠軍,刺史。本渤海晴州,故縣五:天晴、神陽、蓮池、狼山、仙巖,皆廢。戶五百。隸海州。東南至海州一百二十里。



 淥州,鴨淥軍,節度。本高麗故國,渤海號西京鴨淥府。



 城
 高三丈,廣輪二十里,都督神、桓、豐、五四州事。故縣三:神鹿、神化、劍門,皆廢。大延琳叛,遷餘黨於上京,置易俗縣居之。在者戶二千。隸東京留守詞。統州四、縣二:弘聞縣。



 神鄉縣。



 桓州。高麗中都城,故縣三:桓都、神鄉、洪水,皆廢。



 高麗王於此創立宮闕,國人謂之新國。五世孫釗,晉康帝建元初為慕容皝所敗,宮室焚蕩。戶七百。隸淥州。在西南二百里。



 豐州。渤海置盤安郡,故縣四:安豐、渤恪、隰壤、硤石,皆
 廢。戶三百。隸淥州。在東北二百一十里。



 正州。本沸流王故地,國為公孫康所並。渤海置沸流都。



 有沸流水。戶五百。隸淥州。在西北三百八十里。統縣一:東那縣。本漢東耐縣地。在州西七十里。



 慕州。本渤海安遠府地,故縣二:慕化、崇平,久廢。戶二百。隸淥州。在西北二百里。



 顯州,奉先軍,上,節度。本渤海顯德府地。世宗置,以奉顯陵。顯陵者,東丹人皇王墓也。人皇王性好讀者,不喜射獵,購書數萬卷,置醫巫閭山絕頂,築堂曰望海。山南去
 海一百三十里。大同元年,世宗親護人皇王靈駕歸自汴京。以大皇王愛醫巫閭山水奇秀,因葬焉。山形掩抱六重,於其中作影殿,制度宏麗。州在山東南,遷東京三百餘戶以實之。應歷元年,穆宗葬世宗於顯陵西山,仍禁樵採。有十三山,有沙河。隸長寧、積慶二宮,兵事屬東京都部署司。統州三、縣三:奉先縣。本漢無慮縣,即醫巫閭,幽州鎮山。世宗析遼東長樂縣民以為陵戶,隸長寧宮。



 山東縣。本漢望平縣。穆宗割渤海永豐縣民為陵戶,隸積慶宮。
 歸義縣。初置顯州,渤海民自來助役,世宗嘉憫,因籍其人戶置縣,隸長寧宮。



 嘉州,嘉平軍,下,刺史。隸顯州。遼西州,阜城軍,中,刺史。本漢遼西郡地,世宗置州,隸長寧宮,屬顯州。統縣一:長慶縣。統和八年,以諸宮提轄司大戶置。



 康州,下,刺史。世宗遷渤海率賓府大戶置,屬顯州。初隸長寧宮,後屬積慶宮。統縣一:率賓縣。本渤海率賓府地。



 宗州,下,刺史。在遼東石熊山,耶律隆運以所俘漢民置。



 聖宗立為州,隸文忠王府。王薨,屬提轄司。統縣一:熊山縣。本渤海縣地。



 乾州,廣德軍,上,節度。本漢無慮縣地。聖宗統和三年置,以奉景宗乾陵。有凝神殿。隸崇德宮,兵事屬東京都部署司。統州一、縣四:奉陵縣。本漢無慮縣地。括諸落帳戶,助營山陵。



 延昌縣。析延昌宮戶置。



 靈山縣。本渤海靈峰縣地。



 司農縣。本渤海麓郡縣,並麓波、雲川二縣大焉。



 海北州,廣化軍,中,刺史。世宗以所俘漢戶置。地在闖
 山之西,南海之北。初隸宣州,後屬乾州。統縣一:開義縣。



 貴德州,寧遠軍,下,節度。本漢襄平縣地,漢公孫度所據。太宗時察割以所俘漢民置。後以弒逆誅,沒入焉。聖宗建貴德軍,後更名。有陀河、大寶山。隸崇德宮,兵事屬東京部署司。統縣二:貴德縣。本漢襄平縣,渤海為崇山縣。



 奉德縣。本渤海緣城縣地,嘗置奉德州。



 沈州,昭德軍,中,節度。本挹婁國地。渤海建沈州,故縣九,皆廢。太宗置興遼軍,後更名。初隸永興宮,後屬敦睦宮,兵
 事隸東京都部署司。統州一、縣二:樂郊縣。太祖俘薊州三河民,建三河縣,後更名。



 靈源縣。太祖俘薊州吏民,建漁陽縣,後更名。



 巖州,白巖軍,下,刺史。本渤海白巖城,太宗撥屬沈州。



 初隸長寧宮,後屬敦睦宮。統縣一:白巖縣。渤海置。



 集州,懷眾軍,下,刺史。古陴離郡地,漢屬險瀆縣,高麗為霜巖縣,渤海置州。統縣一:奉集縣。渤海置。



 廣州,防禦。漢屬襄平縣,高麗為當山縣,渤海為鐵利郡。



 太祖遷渤海人居之,建鐵利州。統和八年省。開泰七年以漢戶置。統縣一:昌義縣。



 遼州,始平軍,下,節度。本拂涅國城,渤海為東平府。



 唐太宗親征高麗,李世勑拔遼城;高宗詔程振、蘇定方討高麗,至新城,大破之;皆此地也。太祖伐渤海,先破東平府,遷民實之。故東平府都督伊、蒙、陀、黑、北五州,共領縣十八,皆廢。太祖改為州,軍日東平,太宗更為始平軍。有遼河、羊腸河、錐子河、蛇山、狼山、黑山、巾子山。隸長寧宮,兵事屬北女直兵司馬。統州一、縣二:
 遼濱縣。



 安定縣。



 祺州,聖軍,下,刺史。本渤海蒙州地。太祖以檀州俘於此建檀州,後更名。隸弘義宮,兵事屬北女直兵馬司。統縣一:慶雲縣。太祖俘密雲民,於此建密雲縣;後更名。



 遂州,刺史。本渤海美州地,採訪使耶律頗德以部下漢民置。穆宗時,頗德嗣絕,沒入焉。隸延昌宮。統縣一:山河縣。本渤海縣,並黑川、麓川二縣置。



 通州,安遠軍,節度。本扶餘國王城,渤海號扶餘城。太祖
 改龍州,聖宗更今名。保寧七年,以黃龍府叛人燕頗餘黨千餘戶置,升節度。統縣四:通遠縣。本渤海扶餘縣,並布多縣置。



 安遠縣。本渤海顯義縣,並鵲州縣置。



 歸仁縣。本渤海強帥縣,並新安縣置。



 漁谷縣。本渤海縣。



 韓州,東平軍,下,刺史。本稿離國舊治柳河縣。高麗置鄚頡府,都督鄚、頡二州。渤海因之。今廢。太宗置三河、榆河二州。聖宗並二州置。隸延昌宮,兵事屬北女直兵馬司。統縣
 一:柳河縣。本渤海粵喜縣地,並萬安縣置。



 雙州,保安軍,下,節度。本挹婁故地。渤海置安定郡,久廢。漚裏僧王從太宗南征,以俘鎮、定二州之民建城置州。



 察割弒逆誅,沒入焉。故隸延昌宮,後屬崇德宮,兵事隸北女直兵馬司。統縣一:雙城縣。本渤海安夷縣地。



 銀州,富國軍,下,刺史。本渤海富州,太祖以銀冶更名。



 隸弘義宮,兵事屬北女直兵馬司。統縣三:延津縣。本渤海富壽縣,境有延津故城,更名。



 新興縣。本故越喜國地,渤海置銀冶,嘗置銀州。



 永平縣。本渤海優富縣地,太祖以俘戶置。舊有永平寨。



 同州,鎮安軍,下,節度。本漢襄平縣地,渤海為東平寨。



 太祖置州,軍曰鎮東,後更名。隸彰愍宮,兵事屬北女直兵馬司。統州一,未詳;縣二:東平縣。本漢襄平縣地。產鐵,撥戶三百採煉,隨征賦輸。



 永昌縣。本高麗永寧縣地。



 咸州,安東軍,下,節度。本高麗銅山縣地,渤海置銅山郡。地在漢候城縣北,渤海龍泉府南。地多山險,寇盜以為
 淵藪,乃招平、營等州客戶數百,建城居之。初號郝裏太保城,開泰八年置州。兵事屬北女直兵馬司。統縣一:咸平縣。唐安東都護,天寶中治營、平二州間,即此。太祖滅渤海,復置安東軍。開泰中置縣。



 信州,彰聖軍,下,節度。本越喜故城。渤海置懷遠府,今廢。聖宗以地鄰高麗,開泰初置州,以所俘漢民實之。兵事屬黃龍府都部署司。統州三,未詳;縣二:武昌縣。本渤海懷福縣地,析平川提轄司及豹山縣一千戶隸之。定武縣。本渤海豹山縣地,析平川提轄司並乳水縣
 人戶置。



 初名定功縣。



 賓州,懷化軍,節度。本渤海城。統和十七年,遷兀惹戶,置刺史于鴨子、混同二水之間,後升。兵事隸黃龍府都部署司。



 龍州,黃龍府。本渤海扶餘府。太祖平渤海還,至此崩,有黃龍見,更名。保寧七年,軍將燕頗叛,府廢。開泰九年,遷城於東北,以宗州、檀州漢戶一千復置。統州五、縣三:黃龍縣。本渤海長平縣,並富利、佐慕、肅慎置。



 遷民縣。本渤海水寧縣,並豐水、扶羅置。



 永平縣。渤海置。



 益州,觀察。屬黃龍府。統縣一:靜遠縣。



 安遠州,懷義軍,刺史。屬黃龍府。



 威州,武寧軍,刺史。屬黃龍府。清州,建寧軍,刺史。屬黃龍府。



 雍州,刺史。屬黃龍府。



 湖州,興利軍,刺史。渤海置。兵事隸東京統軍司。統縣一:長慶縣。



 渤州,清化軍,刺史。渤海置。兵事隸東京統軍司。統縣一:貢珍縣。渤海置。



 郢州,彰聖軍,刺史。渤海置。兵事隸北女直兵馬司。統縣一:延慶縣。



 銅州,廣利軍,刺史。渤海置。兵事隸北兵馬司。統縣一:析木縣。本漢望平縣地,渤海為花山縣。初隸東京,後來屬。



 洓州,刺史。渤海置。兵事隸南兵馬司。



 率賓府,刺史。故率賓國地。



 定理府,刺史。故挹婁國地。



 鐵利府,刺史。故鐵利國也。



 安定府。



 長嶺府。



 鎮海府,防禦。兵事隸南文直湯河司。統縣一:平南縣。



 冀州,防禦。聖宗建,升永安軍。



 東州。以渤海戶置。



 尚州。以渤海戶置。



 吉州,福昌軍,刺史。



 麓州,下,刺史。渤海置。荊州,刺史。



 懿州,寧昌軍,節度。太平三年越國公主以媵臣戶置。初日慶懿軍,更曰廣順軍,隸上京。清寧七年宜懿皇后進入,改今名。統縣二。



 寧昌縣。本平陽縣。



 順安縣。



 媵州,昌水軍,刺史。



 順化城,向義軍,下,刺史。開泰三年以漢戶置。兵事隸東京統軍司。



 寧州,觀察。統和二十九年伐高麗,以渤海降戶置。兵事隸東京統軍司。統縣一:
 新安縣。



 衍州,安廣軍,防禦。以漢戶置。初刺史,後升軍。兵事屬東京統軍司。統縣一:宜豐縣。



 連州,德昌軍,刺史。以漢戶置。兵事屬東京統軍司。統縣一;安民縣。



 歸州,觀察。太祖平渤海,以降戶置,後廢。統和二十九年伐高麗,以所俘渤海戶復置。兵事屬南文直湯河司。統縣一:
 歸勝縣。



 蘇州,安復軍,節度。本高麗南蘇,興宗置州。兵事屬南文直湯河司。統縣二:來蘇縣。



 懷化縣。



 復州,懷德軍,節度。興宗置。兵事屬南文直湯河司。統縣二:永寧縣。



 德勝縣。



 肅州,信陵軍,刺史。重熙十年州民亡入女直,取之復置。



 兵事隸北女直兵馬司。統縣一:清安縣。



 安州,刺史。兵事隸北女直兵馬司。



 榮州。



 率州。



 荷州。



 源州。



 渤海州。



 寧江州,混同軍,觀察。清寧中置。初防禦,後升。兵事屬東北統軍司。統縣一:
 混同縣。



 河川,德化軍。置軍器坊。



 祥州,瑞聖軍,節度。興宗以鐵驪戶置。兵事隸黃龍府都部署司。統縣一:懷德縣。



\end{pinyinscope}