\article{卷三十四志第四 兵衛志上}

\begin{pinyinscope}

 軒轅氏合符東海,邑於涿鹿之阿,遷徙往來無常處,以兵為營衛。飛狐以北,無慮以東,西暨流沙,四戰之地,聖人猶不免於兵衛,地勢然耳。



 遼國左都遼海,右邑涿鹿,兵力莫強焉。其在隋世,依紇臣水而居,分為十部。兵多者三千,少者千餘。順寒暑,逐水草畜牧。侵伐則十部相與議,興兵致役,合契而後動。獵則部得自行。至唐,大賀
 氏勝兵四萬三千人,分為八部。大賀氏中衰,僅存五部。有耶律雅里者,分五部為大,立二府以總之,析三耶律氏為七,二審密氏為五,凡二十部。刻木為契,政令大行。遜不有國,乃立遙輦氏代大賀氏,兵力益振,即太祖六世祖也。



 及太祖會李克用於雲中,以兵三十萬,盛矣。



 遙輦耶瀾可汗十年,歲在辛酉,太祖授鉞專征,破室韋、于厥、奚三國,俘獲廬帳,不可勝紀。十月,授大迭烈府夷離堇,明賞罰,繕甲兵,休息民庶,滋蕃群牧,務在戢兵。十一年,總兵四十萬伐代北,克郡縣九,俘九萬五千口。十二年,德祖討奚,俘七千戶。十五年,遙輦可汗卒,遺命遜位
 於太祖。太祖即位五年,討西奚、東奚,悉平之,盡有奚、霫之眾。



 六年春,親征幽州,東西旌旗相望,互數百里。所經郡縣,望風皆下,俘獲甚眾,振旅而還。秋,親征背陰國,俘獲數萬計。



 神冊元年,親征突厥、吐渾、黨項、小蕃、沙陀諸部,俘戶一萬五千六百;攻振武,乘勝而東,攻蔚、新、武、媯、儒五州,俘獲不可勝紀,斬不從命者萬四千七百級。盡有代北、河曲、陰山之眾,遂取山北八軍。四年,親征於骨里國,俘獲一萬四千二百口。五年,徵黨項,俘獲二千六百口;攻天德軍,拔十有二柵,徒其民。六年,出居庸關,分兵掠擅、順等州,安遠軍、三河、良鄉、望都、潞、滿城、遂城等
 縣,俘其民徙內地;皇太子略定州,俘獲甚眾。天贊元年,以戶口滋繁,糾轄疏遠,分北大濃兀為二部,立兩節度以統之。三年,西征黨項等國,俘獲不可勝紀。四年,又親征渤海。天顯元年,滅渤海國,地方五千里,兵數十萬,五京、十五府、六十二州,盡有其眾,契丹益大。



 會同初,太宗滅唐立晉,晉獻燕、代十六州,民眾兵強,莫之能御矣。



 兵制遼國兵制,凡民年十五以上,五十以下,隸兵籍。每正軍一名,馬三匹,打草穀、守營鋪家丁各一人。人鐵甲九事,馬韉轡,馬甲皮鐵,視其力;弓四,箭四百,長短槍、(金骨)(金朵)、斧
 鉞、小旗、錘錐、火刀百、馬盂、粆一斗、粆袋、搭鉚傘備一,縻馬繩二百尺,皆自備。人馬不給糧草,日遣打草穀騎四出抄掠以供之。鑄金魚符,調發軍馬。其捉馬及傳命有銀牌二百。軍所舍,有遠探攔子馬,以夜聽人馬之聲。



 凡舉兵,帝率蕃漢文武臣僚,以青牛白馬祭告天地、日神,惟不拜月,分命近臣告太祖以下諸陵及木葉山神,乃沼諸道徵兵。惟南、北、奚王,東京渤海兵馬,燕京統軍兵馬,雖奉詔,未敢發兵,必以聞。上遣大將持金魚符,合,然後行。始聞詔,攢戶丁,推戶力,核籍齊眾以待。自十將以上,次第點集軍馬、器仗。符至,兵馬本司自領,使者不得
 與。唯再共點軍馬訖,又以上聞。量兵馬多少,再命使充軍主,與本司互相監督。又請引五方旗鼓,然後皇帝親點將校。又選勛戚大臣,充行營兵馬都統、副都統、都監各一人。又選諸軍兵馬尤精銳者三萬人為護駕軍,又選驍勇三千人為先鋒軍,又先剽悍百人之上為遠探攔子軍,以上各有將領。又於諸軍每部,量眾寡,抽十人或五人,合為一隊,別立將領,以備勾取兵馬,騰遞公事。



 其南伐點兵,多在幽州北千里鴛鴦泊。及行,並取居庸關、曹王峪、白馬口、古北口、安達馬口、松亭關、榆關等路。將至平川、幽州境,又遣使分道催發,不得久駐,恐踐禾
 稼。出兵不過九月,還師不過十二月。在路不得見僧尼、喪服之人。



 皇帝親征,留親王一人在幽州,權知軍國大事。既入南界,分為三路,廣信軍、雄州、霸州各一。駕必由中道,兵馬都統、護駕等軍皆從。各路軍馬遇縣鎮,即時攻擊。若大州軍。必先料其虛實、可攻次第而後進後兵。沿途民居、圓囿、桑柘,必夷伐焚蕩。至宋北京,三路兵皆會,以議攻取。乃退亦然。三路軍馬前後左右有先鋒。遠探攔子馬各十數人,在先鋒前後二十餘里,全副衣甲,夜中每行十里或五里少駐,下馬側聽無有人馬之聲。有則擒之;力不可敵,飛報先鋒,齊力攻擊。如有大軍,走報
 主帥。敵中虛實,動必知之。軍行當道州城,防守堅固,不可攻擊,引兵過之。恐敵人出城邀阻,及圍射鼓噪,詐為攻擊。敵方閉城固守,前路無阻,引兵進,分兵抄截,使隨處州城隔絕不通,孤立無援。所過大小州城,至夜,恐城中出兵突擊,及與鄰州計會軍馬,甲夜,每夜以騎兵百人去城門左右百餘步,被甲執兵,立馬以待。乓出,力不能加,馳還勾集眾兵與戰。左右官道、斜徑、山路、河津,夜中並遣兵巡守。



 其打草穀家丁,備衣甲持兵,旋團為隊,必先斫伐園林,然後驅掠老幼,運土木填壕塹;攻城之際,必使先登,矢石擂木並下,止傷老幼。又於本國州縣
 起漢人鄉兵萬人,隨軍專伐園林,填道路。御寨及諸營壘,唯用桑柘梨慄。軍退,縱火焚之。敵軍既陣,料其陣勢小大,山川形勢,往回道路,救援捷徑,漕運所出,各有以制之。然後於陣四面,列騎為隊,每隊五、七百人,十隊為一道,十道當一面,各有主帥。最先一隊走馬大噪,沖突敵陣。得利,則諸隊齊進;若未利,引退,第二隊繼之。退者,息馬飲水粆。諸道皆然。更退迭進,敵陣不動,亦不力戰。歷二三日,待其困憊,又令打草穀家下馬施變雙帚,因風疾她,揚塵敵陣,更互往來。中既饑疲,目不相視,可以取勝。若陣南獲勝,陣北失利,主將在中,無以知之,則以本
 國四方山川為號,聲以相聞,得相救應。



 若帝不親征,重臣統兵不下十五萬眾,三路征還,北京會兵,進以九月,退以十二月,行事次第皆如之。若春以正月,秋以九月,不命都統,止遣騎兵六萬,不許深入,不攻城池,不伐林木;但於界外三百里內,耗蕩生聚,不令種養而已。



 軍人南界,步騎車帳不循阡陌。三道將領各一人,率攔子馬各萬騎,支散游弈百十里外,更迭覘邏。及暮,以吹角為號,眾即頓舍,環繞禦帳。自近及運,折木稍屈,為弓子鋪,不設槍管塹柵之備。



 每軍行,鼓三伐,不問晝夜,大眾齊發。未遇大敵,不乘戰馬;俟近敵師,乘新羈馬,蹄有餘力。
 成列不戰,退則乘之。



 多伏兵斷糧道,冒夜舉火,上風曳柴。饋餉自賚,散而復聚。



 善戰,能寒。此兵之所以強也。



\end{pinyinscope}