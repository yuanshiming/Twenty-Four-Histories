\article{卷五十八志第二十七 儀衛志四儀仗}

\begin{pinyinscope}

 帝王處則重門擊柝,出則以師兵為營衛,勞人動眾,豈得已哉。天下大患生於大欲,不得不遠慮深防耳。智英勇傑、魁臣雄藩於是乎在,寓武備於文物之中,此儀仗所由設也。



 金吾、黃麾六軍之仗,遼受之晉,晉受之後唐,後唐受之梁、唐,其來也有自。耶律儼、陳大任舊《志》有未備者,兼考之《遼朝雜禮》云。



 國仗王通氏言,舜歲遍四岳,民不告勞,營衛省、徵求寡耳。



 遼太祖匹馬一麾,斥地萬里,經營四方,末嘗寧居,所至樂從,用此道也。太宗兼制中國,秦皇、漢武之儀文日至,後嗣因之。



 旄頭豹尾,馳驅五京之間,終歲勤動,轍這相尋。民勞財匱,此之故歟。



 遼自大賀氏摩會受唐鼓纛之賜,是為國仗。其制甚簡,太宗伐唐、晉以前,所用皆是物也。著於篇首,以見艱創業之主,豈必厚衛其身雲。



 十二神纛,十二旗,
 十二鼓,曲柄華蓋,直柄華蓋。



 遙輦末主遺制,迎十二神纛、天子旗鼓置太祖帳前。諸弟剌哥等叛,勻德實縱火焚行宮,皇后命曷古魯救之,止得天子旗鼓。太宗即位,置旗鼓、神纛於殿前。聖宗以輕車儀衛拜帝山。



 渤海仗天顯四年,太宗幸遼陽府,人皇王備乘輿羽衛以迎。乾亨五年,聖宗東巡,東京留具儀衛迎車駕。此故渤海
 儀衛也。



 漢仗大賀失活入朝於唐,娑固兄弟繼之,尚主封王,飫觀上國。



 開元東封,邵固扈從,又覽太平之盛。自是朝貢歲至於唐。遼始祖涅里立遙輦氏,世為國相,目見耳聞,歆企帝王之容渾有年矣。遙輦致鼓纛於太祖帳前,會何足以副其雄心霸氣之所睥睨哉。闕後交梁聘唐,不憚勞勛。至於太宗,立晉以要冊禮,入汴而收法物,然後累世之所願欲者,一舉而得之。太原擅命,力非不敵,席卷法物,先致中京,蹤棄山河,不少顧慮,志可知矣。於是秦、漢
 以來帝王文物盡筆記於遼;周、宋按圖更制,乃非故物。遼之所重,此其大端,故特著焉。



 太宗會同元年,晉使馮道備車輅法物,上皇太后冊禮;劉邈、盧重備禮,上皇帝尊號。



 三年,上在薊州觀《導駕儀衛圖》,遂備法駕幸燕,御元和殿行入閣禮。



 六年,備法駕幸燕,迎導御元和殿。



 大同元年正月朔,備法駕至汴,上御崇元殿,受文武百僚朝賀。自是日以為常。二月朔,上御崇元殿,備禮受朝賀。
 三月,將幸中京鎮陽,詔收鹵簿法物,委所司押領先往。未幾鎮陽入漢,鹵簿法物隨世宗歸於上京。四月,皇太弟李胡遣使問軍事,上報曰,朝會起居如禮。是月,太宗崩,世宗即位,鹵簿法物備而不御。



 穆宗應歷元年,詔朝會依嗣聖皇帝故事,用漢禮。



 景宗乾亨五年二月,神樞升轀輬車,具鹵簿儀衛。六月,聖宗至上京,留守具法駕迎導。



 聖宗統和元年,車駕還上京,迎導儀衛如式。



 三年,駕幸上京,留守具儀衛奉迎。



 四年,燕京留守具儀衛導駕入京,上御元和殿,百僚朝
 賀。



 是後,儀衛常事,史不復書。



 鹵簿儀仗數馬匹步行擎執二千四百一十二人,坐馬擎執二百七十五人,坐馬樂人二百七十三人,步行教坊人七十一人,御馬牽攏五十二人,御馬二十六匹,官僚馬牽攏官六十六人,坐馬掛甲人五百九十八人,步行掛甲人百六十人,金甲二人,神輿十二人,長壽仙一人,諸職官等三百五人,內侍一人,引稍押衙二人,赤縣令一人,府牧一人,府吏二人,少尹一人,司錄一人,功曹一人,太常少卿一人,太常丞一人,太常博士一人,司徒一人,太僕卿
 一人,鴻臚卿一人,大理卿一人,御史大夫一人,侍御史二人,殿中侍御史二人,監察御史一人,兵部尚書一人,兵部侍郎一人,兵部郎中一人,兵部員外郎一人,符寶郎一人,左右諸衛將軍三十五人,左右諸折沖二十一人,左右諸果毅二十八人,尚乘奉御二人,排仗承直二人,左右夾騎二人,都頭六人,主帥一十四人教坊司差,押纛二人,左右金吾四人,虞候次飛一十六人,鼓吹令二人,漏刻生二人,押當官一人,司天監一人,令史一人,司辰一人,統軍六人,千牛備身二人,左右半勛二人,左右郎將四人,左右拾遺二人,左右補闕二人,起居舍人一人,左
 右諫議大夫二人,給事中書舍二人,左右散騎常侍二人,門下侍郎二人,中書侍郎二人,鳴鞭二人內侍內差,侍中一人,中書令一人,監門校尉二人,排列官二人,武衛隊正一人,隨駕諸司供奉官三十人,三班供奉官六十人,通事舍人四人,御史中丞二人,乘黃丞二人,都尉一人,太僕卿一人,步行太卜令一人。職官乘馬三百四匹,進馬四匹,駕車馬二十八匹。人之數凡四千二百三十有九,馬之數凡千五百二十。



 得諸本朝太常卿徐世隆家藏《遼朝雜禮》者如是。至於儀注之詳,不敢傳會云。



\end{pinyinscope}