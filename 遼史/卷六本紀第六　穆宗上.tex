\article{卷六本紀第六 穆宗上}

\begin{pinyinscope}

 穆宗孝安敬正皇帝,諱璟,小字述律。太宗皇帝長子,母曰靖安皇后蕭氏。會同一二年,封壽安王。



 天祿五年秋九月癸亥,世宗遇害。逆臣察割等伏誅。丁卯,即皇帝位,群臣上尊號曰天順皇帝,改元應厲。戊辰,如南京。



 是月,遣劉承訓告哀於漢。



 冬十一月,漢、周、南詔各遣使來吊。乙亥,詔朝會依嗣聖皇帝故事,用漢禮。



 十二月甲辰,漢遣使
 獻弓矢、鞍馬。壬子,鐵驪、鼻骨德皆來貢。



 二年春正月戊午朔,南遣使奉蠟丸書,及進犀兕甲萬屬。



 壬戌,太尉急古質謀逆,伏誅。



 二月癸卯,女直來貢。



 三月癸亥,南唐遣使奉蠟丸書。丁卯,復遣使來貢。甲申,以耶律撻烈為南院大王。



 夏四月丙戌朔,日有食之。己亥,鐵驪進鷹鶻。



 五月丙辰朔,視朝。壬午,南唐遣使來貢。



 六月壬辰,國舅政事令蕭眉古得、宣政殿學士李浣等謀南奔,事覺,詔暴其罪。乙未,祭天地。壬寅,漢為周所侵,遣使求援,命中臺省右相高模翰赴之。丁未,命乳媼之史曷魯世為阿束石烈夷離堇。



 秋七月乙亥,政事令婁
 國、林牙敵烈、侍中神都、郎君海裡等謀亂就執。



 八月己丑,眉古得、婁國等伏誅,杖李浣而釋之。



 九月甲寅朔,雲州進嘉禾四莖,二穗。戊午,詔以先平察割日,用白黑羊、玄酒祭天,歲以為常。壬戌,獵炭山。癸天。



 庚辰,敵烈部來貢。



 冬十月甲申朔,漢遣使進葡萄酒。甲午,司徒老古等獻白雉。戊申,回鶻及轄戞斯皆遣使來貢。



 十一月癸丑朔,視朝。己巳,地震。己卯,日南至,始用舊制行拜日禮朔州民黑兔。



 十二月癸未朔,高模翰及漢兵圍晉州。辛卯,以生日,飯僧,釋系囚。甲辰,獵於近郊。祀天地。辛亥,明王安端薨。



 三年春閏正月壬午朔,漢以高模翰卻周軍,遣使來謝。



 二月辛亥朔,詔用嗣聖皇帝舊璽。甲子,太保敵烈修易州城,鎮州以兵來挑戰,卻之。



 三月庚辰朔,南唐遣使來貢,因附書於漢,詔達之。庚寅,如慶州擊鞠。丁酉,漢遣使進球衣及馬。庚子,觀漁於神德湖。



 夏四月庚申,鐵驪來貢。



 五月壬寅,漢遣使言石晉樹先帝《聖德神功碑》為周人所毀,請再刻,許之。



 六月丁卯,應天皇太后崩。



 秋七月,不視朝。



 八月壬子,以生日、釋囚。己未,漢遣使求援。三河烏古、吐番、吐谷渾、鼻骨備皆遣使來貢。九月庚子,漢遣使貢藥。



 冬十月己酉,命太師唐骨德治大行皇太后園
 陵。李胡子宛、郎君嵇干、敵烈謀反,事覺,辭逮太平王罨撒葛、林牙華割、郎君新羅等,皆執之。



 十一月辛丑,謚皇太后曰貞烈,葬祖陵。漢遣使來會。



 是冬,駐蹕奉聖州,以南京水,詔免今歲租。



 四年春正月戊寅,回鶻來貢。己丑,華割、嵇乾等伏誅,宛及罨撒葛皆釋之。是月,周主威殂,養子王柴榮嗣立。



 二月丙午朔,周攻漢,命政事令耶律敵祿援之。丙辰,漢遣使進茶。幸南京。



 夏五月乙亥,忻、代二州叛漢,遣南院大王撻烈助敵祿討之。丁酉,撻烈敗周將符彥卿於忻口。



 六月癸亥,撻烈獻所獲。



 秋七月乙酉,漢民有為遼軍
 誤掠者,遣使來請,詔悉歸之。



 九月丙申,漢為周人所侵,遣使來告。



 冬十一月,彰國軍節度使蕭敵烈、太保許從贇奏忻、代二州捷。十二月辛酉朔,謁祖陵。庚午,漢遣使來貢。



 是冬,駐蹕杏堝。



 五年春正月辛未朔,鼻骨德來貢。



 二月庚子朔,日有食之。庚申,流遣使請上尊號,不許。



 壬戌,如哀潭。



 夏四月己酉,周侵漢,漢遣使求援。癸丑,命郎君蕭海瓈世為北府宰相。



 秋九月庚辰,漢主有疾,遣使來告。



 冬十月壬申,女直來貢。丁亥,謁太宗廟。庚寅,南唐遣使來貢。十一月乙未朔,漢主崇殂,子承鈞遣使來告,且求嗣立;遣使吊祭,
 遂封冊之。



 十二月乙丑朔,謁太祖廟。辛巳,漢遣使業議軍事。



 六年夏五月丁酉,謁懷陵。



 六月甲子,漢遣使來議軍事。



 秋七月,不視朝。



 九月戊午,謁祖陵。



 冬十一月壬寅,鼻骨德來貢。



 十二月己未朔,謁太祖廟。



 七年春正月庚子,鼻骨德來貢。



 二月辛酉,南唐遣使奉蠟丸書。辛未,駐蹕潢河。



 夏四月戊午朔,還上京。初,女巫肖古上延年藥方,當用男子膽和之。不數年,殺人甚多。是,覺其妄。辛巳,射殺之。



 五月辛卯,漢遣使來貢。



 六月丙辰朔,周遣使來聘。南唐遣使來貢。



 八月己未,周遣使來
 聘。



 是秋,不聽政。



 冬十月庚申,獵於七鷹山。



 十二月丁巳,詔大臣曰:「有罪者,法當刑。朕或肆怒,濫及無辜,卿等切諫,無或面從。」辛巳,還上京。



 八年春二月乙丑,駐蹕潢河。



 夏四月甲寅,南京留守蕭思溫攻下沿邊州縣,遣人勞之。



 五月,周陷束城縣。



 六月辛未,蕭思溫請益兵,乞賀幸燕。



 秋七月,獵於拽刺山。迄於九月,射鹿諸山,不視朝。



 冬十一月辛酉,漢遣使來告周復來侵。乙丑,使再至。



 十二月庚辰,又至。九年春正月戊辰,駐蹕潢河。



 夏四月丙戌,周來侵。戊戌,以南京留守蕭思溫為兵馬都總管擊之。是月,周拔益
 津,瓦橋、淤口三關。



 五月乙巳,陷瀛、莫二州。癸亥,如南京。辛未,周兵退。



 六月乙亥朔,視朝。戊寅,復容城縣。庚申,西幸,如懷州。是月,周主榮殂,子宗訓立。



 秋七月,發南京軍戌範陽。



 冬十二月戊寅,還上京。庚辰,王子敵烈、前宣徽使海思及蕭達干等謀反,事覺鞫之。辛巳,祀天地、祖考,告逆黨事敗。丙申,召群臣議時政。



 十年春正月,周殿前都點檢趙匡胤廢周立,建國號宋。



 夏五月乙巳,謁懷陵。壬子,漢以潞州歸附來告。丙寅,至自懷陵。



 六月庚申,漢以宋兵圍石州來告,遣大同軍節度使阿刺率四部往援,詔蕭思溫以三部兵助之。



 秋
 七月己亥朔,宋兵陷石州,潞州復叛,漢使來告。辛酉,政中令耶律壽遠、太保楚阿不等謀反,伏誅。以酒脯祠天地於黑山。八月,如秋山,幸懷州。庚午,以鎮茵石狻猊擊殺近待古哥。



 冬十月丙子,李胡子喜隱謀反,辭連李胡,下獄死。



 十一月,思獄中上書,陳便宜。



 十一年春時事月丙寅,釋喜隱。辛亥,司徒烏里只子迭刺哥誣告其父謀反,復詐乘傳及殺行人,以其父請,杖而釋之。



 三月丙辰,蕭思溫奏老人星見,乞行赦宥。



 閏月甲子朔,如潢河。



 夏四月癸巳朔,日有食之。是月,射鹿,不視朝。五月乙亥,司天王白、李正等進歷。



 六月甲午,赦。



 冬十一月,歲
 星犯月。



 十二年春正月甲戌,夜觀燈。



 二月己丑朔,以御史大夫蕭護思為北院樞密使,賜對衣、鞍馬。夏五月庚午,以旱,命左右以水相沃,頃之,果雨。



 六月甲午,祠木葉山及潢河。



 秋,如黑山,赤山射鹿。



 十三年春正月,自丁巳,晝夜酣飲者九日。丙寅,宋欲城益津關,命南京留守高勛、統軍使崔廷勛以兵擾之。癸酉,殺獸人海裏。



 二月庚寅,漢遣使來告,欲巡邊徼,乞張聲援。壬辰,如潢河。癸巳,觀群臣射,賜物有差。乙巳,老人星見。



 三月癸丑朔,殺鹿人里吉,梟其首以示掌鹿者。



 夏四月壬寅,獵於潢河。



 五月壬戌,視斡朗改國所進花鹿生麛。



 六月癸未,近侍傷獐,杖殺之。甲申,殺獐人霞馬。壬辰,詔諸路錄囚。



 秋七月辛亥朔,漢以宋侵來告。乙丑,薦時羞於廟。



 八月甲申,以生日,縱五坊鷹鶻。戊戌,幸近山,呼鹿射之。旬有七日返。



 九月庚戌朔,以青牛白以祭天地。飲於野次,終夕乃罷。



 辛亥,以酒脯祭天地,復終夜酣飲。



 冬十月丙申,漢以宋侵來告。



 十一月庚午,漢以宋侵來告。



 十二月戊子,射野鹿,賜虞人物有差。庚寅,殺彘人曷主。



\end{pinyinscope}