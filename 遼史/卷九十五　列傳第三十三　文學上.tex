\article{卷九十五 列傳第三十三 文學上}

\begin{pinyinscope}

 蕭韓家奴李汗
 遼起松漠,太祖以兵經略方內,禮文之事固所未遑。及太宗入汴,取晉圖書、禮器而北,然後制度漸以修舉。至景、聖間,則科目聿興,士有由下僚擢升侍從,駸駸崇儒之美。但其風氣剛勁,三面鄰敵,歲時以搜獮為務,而典章文物視古猶闕。



 然二百年之業,非數君子為之綜理,則後世惡所考述哉。作《文學傳》。



 蕭韓家奴,字休堅,涅剌部人,中書令安搏之孫。少好學,弱冠入南山讀書,博鑒經史,通遼、漢文字。統和十四年始仕。



 家有一牛,不任驅策,其奴得善價鬻之。韓家奴曰:「利己誤人,非吾所欲。」乃歸直取牛。二十八年,為右通進,典南京慄園。重熙初,同知三司使事。四年,遷天成軍節度使,徙彰愍宮使。帝與語,才之,命為詩友。嘗從容問曰:「卿居外有異聞乎?」韓家奴對曰:「臣惟知炒慄:小者熟,則大者必生;大者熟,則小者必焦。使大小均熟,始為盡美。不知其他。」蓋嘗掌慄園,故托慄以諷諫。帝大笑。詔作《四時逸樂賦》,帝稱善。



 時詔天下言治道之要,制問:「徭役不
 加於舊,征伐辦不常有,年穀既登,帑廩既實,而民重困,豈為吏者慢、為民者惰歟?今之徭役何者最重?何者尤苦?何所蠲省則為便益?



 補役之法何可以復?盜賊之害何可以止?」韓家奴對曰:臣伏見比年以來,高麗未賓,阻卜猶強,戰守之備,誠不容已。乃者,選富民防邊,自備糧糗。道路修阻,動淹歲月;比至屯所,費已過半;只牛單轂,鮮有還者。其無丁之家,倍直傭僦,人憚其勞,半途亡竄,故戍卒之食多不能給。求假於人,則十倍其息,至有鬻子割田,不能償者。或逋役不歸,在軍物故,則復補以少壯。其鴨綠江之東,戍役大率如此。況渤海、女直、高麗合
 從連衡,不時征討。富者從軍,貧者偵候。



 加之水旱,菽粟不登,民以日困。蓋勢使之然也。



 方今最重之役,無過西戍。如無西戍,雖遇兇年,困弊不至於此。若能徙西戍稍近,則往來不勞,民無深患。議者謂徙之非便:一則損威名,二則召侵侮,三則棄耕牧之地。臣謂不然。阻卜諸部,自來有之。曩對此至臚朐河,南至邊境,人多散居,無所統一,惟往來抄掠。及太祖西征,至於流沙,阻卜望風悉降,西域諸國皆願入貢。因遷種落,內置三部,以益吾國,不營城邑,不置戍兵,阻卜累世不敢為寇。統和間,皇太妃出師西域,拓土既遠,降附亦眾。自後一部或叛,鄰部
 討之,使同力相制,正得馭遠人之道。及城可敦,開境數千里,西北之民,徭役日增,生業日殫。警急既不能救,叛服亦復不恆。



 空有廣地之名,而無得地之實。若領土不已,漸至虛耗,其患有不勝言者。況邊情不可深信,亦不可頓絕。得不為益,舍不為損。國家大敵,惟在南方。今雖連和,難保他日。若南方有變,屯戍遼邈,卒難赴援。我進則敵退,我還則敵來,不可不慮也。方今太平已久,正可恩結諸部,釋罪而歸地,內徙戍兵以增堡障,外明約束以正疆界。每部各置酋長,歲修職貢。叛則討之,服則撫之。諸部既安,必不生釁。如是,則臣雖不能保其久而無
 變,知其必不深入侵掠也。或云,棄地則損威。殊不知殫費竭財,以貪無用之地,使彼小部抗衡大國,萬一有敗,損威豈淺?或又云,沃壤不可遽棄。臣以為土雖沃,民不能久居,一旦敵來,則不免內徙,豈可指為吾土而惜之?



 夫幣稟雖隨部而有,此特周急部民一偏之惠,不能均濟天下。如欲均濟天下,則當知民困之由,而窒其隙。節盤游,簡驛傳,薄賦斂,戒奢侈。期以數年,則困者可蘇,貧者可富矣。



 蓋民者國之本,兵者國之衛。兵不調則曠軍役,調之則損國本。



 且諸部皆有補役之法。昔補役始行,居者、行者類皆富實,故累世從戍,易為更代。近歲邊虞
 數起,民多匱乏,既不任役事,隨補隨缺。茍無上戶,則中戶當之。曠日彌年,其窮益甚,所以取代為艱也。非惟補役如此,在邊戍兵亦然。譬如一功之土,豈能填尋丈之壑!欲為長久之便,莫若使遠戍疲兵還於故鄉,薄其徭役,使人人給足,則補役之道可以復故也。



 臣又聞,自背有國家者,不能無盜。比年以來,群黎凋弊,利於剽竊,良民往往化為兇暴。甚者殺人無忌,至有亡命山澤,基亂首禍。所謂民以困窮,皆為盜賊者,誠如聖慮。今欲芟夷本根,願陛下輕徭省役,使民務農。衣食既足,安習教化,而重犯法,則民趨禮義,刑罰罕用矣。臣聞唐太宗問群
 臣治盜之方,皆曰:「嚴刑峻法。」太宗笑曰:「寇盜所以滋者,由賦無度,民不聊生。今朕內省嗜欲,外罷游幸,使海內安靜,則寇盜自止。」由此觀之,寇盜多寡,皆由衣食豐儉,徭役重輕耳。今宣徙可敦城於近地,與西南副都部署烏古敵烈、隗烏古等部聲援相接。罷黑嶺二軍,並開、保州,皆隸東京;益東北戍軍及南京總管兵。增修壁壘,候尉相望,繕完樓櫓,浚治城隍,以為邊防。此方今之急務也,願陛下裁之。



 擢翰林都林牙,兼修國史。仍詔諭之曰:「文章之職,國之光華,非才不用。以卿文學,為時大儒,是用授卿以翰林之職。朕之起居,悉以實錄。」自是日見
 親信,每入侍,賜坐。



 遇勝日,帝與飲酒賦詩,以相醻酢,君臣相得無比。韓家奴知無不言,雖諧謔不忘規諷。



 十三年春,上疏曰:「臣聞先世遙輦可汗窪之後,國祚中絕;自夷離堇雅里立阻午,大位始定。然上世俗樸,未有尊稱。



 臣以為三皇禮文末備,正與遙輦氏同。後世之君以禮樂治天下,而崇本追遠之義興焉。近者唐高祖創立先廟,尊四世為帝。背我太祖代遙輦即位,乃制文字,修禮法,建天皇帝名號,制宮室以示威服,興利除害,混一海內。厥後累聖相承,自夷離堇湖烈以下,大號未加,天皇帝之考夷離堇的魯猶以名呼。臣以為宜依唐典,迫崇
 四祖為皇帝,則陛下弘業有光,墜典復舉矣。」



 疏奏,帝納之,始行追冊玄、德二祖之禮。



 韓家奴每見帝獵,末嘗不諫。會有司奏獵秋山,熊虎傷死數十人,韓家奴書於冊。帝見,命去之。韓家奴既出,復書。



 他日,帝見之曰:「史筆當如是。」帝問韓家奴:「我國家創業以來,孰為賢主?」韓家奴以穆宗對。帝怪之曰:「穆宗嗜酒,喜怒不常,視人猶草芥,卿何謂賢?」韓家奴對曰:「穆宗雖暴虐,省徭輕賦,人樂其生。終穆之世,無罪被戮,未有過今日秋山傷死者。臣故以穆宗為賢。」帝默然。



 詔與耶律庶成錄遙輦可汗至重熙以來事跡,集為二十卷,進之。十五年,復詔曰:「古之治
 天下者,明禮義,正法度。我朝之興,世有明德,雖中外響化,然禮書末作,無以示後世。



 卿可與庶成酌酌古準今,制為禮典。事或有疑,與北、商院同議。」韓家奴既被詔,博考經籍,自天子達於庶人,情文制度可行於世,不繆於古者,撰成三卷,進之。又詔譯諸書,韓家奴欲帝知古今成敗,譯《通歷》、《貞觀政要》、《五代史》。



 時帝以其老,不任朝謁,拜歸德軍節度使。以善治聞。帝遣使問勞,韓家奴表謝。召修國史,卒,年七十二。有《六義集》十二卷行於世。



 李汗,初仕晉,為中書舍人。晉亡歸遼,當太宗崩、世宗立,恟不定,汗與高勛等十餘人羈留南京。久之,從歸上
 京,授翰林學士。



 穆宗即位,累遷工部侍郎。時汗兄濤在汴為翰林學士,密遣人召汗。汗得書,托求醫南京,易服夜出,欲遁歸汴。至涿,為徼巡者所得,送之南京,下吏。汗伺獄吏熟寢,以衣帶自經;不死,防之愈嚴。械赴上京,自投潢河中流,為鐵索牽掣,又不死。及抵上京,帝欲殺之。時高勛已為樞密使,救止之。屢言於上曰:「汗本非負恩,以母年八十,急於省觀致罪。且汗富於文學,方今少有倫比,若留掌詞命,可以增光國體。」帝怒稍解,仍令禁錮於奉國寺,凡六年,艱苦萬狀。



 會上欲建《太宗功德碑》,高勛奏曰:「非李汗無可秉筆者。」詔從之。文成以進,上悅,釋
 囚。尋加禮部尚書,宣政殿學士,卒。



 論曰:「統和、重熙之間,務修文治,而韓家奴對策,落落累數百言,概可施諸行事,亦遼之晁、賈哉。李汗雖以詞章見稱,而其進退不足論矣。」



\end{pinyinscope}