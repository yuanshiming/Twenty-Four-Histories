\article{卷五十六志第二十五 儀衛志二國服}

\begin{pinyinscope}

 上古之人,網罟禽獸,食肉衣皮,以儷鹿韋掩前後,謂之靴。然後夏葛、冬裘之制興焉。周公陳王業,《七月》之詩,至於一日于貉,三月條桑,八月載績,公私之用由是出矣。



 契丹轉居薦草之間,去邃古之風猶未遠也。太祖仲父述瀾,以遙輦氏於越之官,占居潢河沃壤,始置城邑,為
 樹藝、桑麻、組織之教,有遼王業之隆,其亦肇跡於此乎!太祖帝北方,太宗制中國,紫銀之鼠,羅綺之篚,麇載而至。纖麗毳,被土綢木。於是定衣冠之制,北班國制,南班漢制,各從其便焉。



 詳國服以著厥始去。



 祭服:遼國以祭山為大禮,服飾尤盛。



 大祀,皇帝服金文金冠,白綾袍,紅帶,懸魚,三山紅垂。



 飾犀玉刀錯,絡縫烏靴。



 小祀,皇帝硬帽,紅克絲龜文袍。皇后戴紅帕,服絡縫紅袍,懸玉佩,雙同心帕,絡縫烏靴。



 臣僚、命婦服飾,各從本部旗幟之色。



 朝服:太祖丙寅歲即皇帝位,朝服衷早,以備非常。其後行瑟瑟禮、大射柳,即此服。聖宗統和元年冊承天皇太后,給三品以上用漢法服,三品以下用大射柳之服。



 皇帝服實里薛袞冠,絡縫紅袍,垂飾犀玉帶錯,絡縫靴,謂之國服袞冕。太宗更以錦袍、金帶。



 臣僚戴氈冠,金花為飾,或加珠玉翠毛,額後垂金花,織成夾帶,中貯發一總。中紗冠,制如烏紗帽,無簷,不姤雙耳。



 額前綴金花,上結紫帶,末未綴珠。服紫窄袍,系傕羶帶,以黃紅色條裹革用之,用金玉、水晶、靛石綴飾,謂之「盤紫」。



 太宗更以錦袍、金帶。會同元
 年,群臣高年有爵秩者,皆賜之。



 公服:謂之「展裹」,著紫。興宗重熙二十二年,詔八房巾幘。道宗清寧元年,詔非勛戚之後及夷離堇副使並承應有職事人,不帶巾。



 皇帝紫皂幅巾,紫窄袍,玉束帶,或衣紅襖;臣僚亦幅巾,紫衣。常服:《宰相中謝儀》,帝常服。《高麗使入見儀》,臣僚便衣,謂之「說服力裹」。綠花窄袍,中單多紅綠色。貴者披貂裘,以紫黑色為貴,青次之。又有銀鼠,尤潔白。賤者貂毛、
 羊、鼠、沙狐裘。



 田獵服:皇帝幅巾,擐甲戎裝,以貂鼠或鵝項、鴨頭為瑴腰。蕃漢諸司使以上並戎裝,衣皆左衽,黑綠色。



 吊服:及祖叛弟剌哥等降,素服受之。



 素服,乘赭白馬。



 漢服:漢服黃帝始制冕冠章服,後王以祀以祭以享。夏收、殷喢、周弁以朝,冠端以居,所以別尊卑、辨儀物也。厥後唐以冕冠、青衣為祭服,通天、絳袍為朝服,平巾幘、袍酆為常服。
 大同元年正月朔,及宗皇帝入晉,備法駕,受文武百官賀於汴京崇元殿,自是日以為常。是年北歸,唐、晉文物,遼則用之。左右採訂,摭其常用者存諸篇。



 祭服:終遼之世,郊丘不建,大裘冕服不書。



 袞冕,祭祀宗廟、遺上將出征、飲至、踐阼、加元服、納後若元日受朝則服之。金飾,垂白珠十二旒,以組為纓,色如其綬,疏纊充耳,玉簪導。玄衣、瑨裳十二章:八章在衣,日、月、星、龍、華蟲、火、山、宗彞;四章在裳,藻、粉米、黼、黻。衣褾領,為升龍織成文,各為六等。龍、山以下,每章一行,行十二,白紗中單,黼領,青褾襈裾,
 黼革帶、大帶,劍佩綬,閤加金飾。《元日朝會儀》,皇帝服袞冕。



 朝服:乾亨五年,聖宗冊承天太后,給三品以上法服。《雜禮》,冊承天太后儀,侍中就席,解劍脫履。重熙五年尊號冊禮,皇帝服龍袞,北南臣僚並朝服,蓋遼制。會同中,太后、北面臣僚國服;皇帝、南面臣僚漢服。乾亨以後,在禮雖北面三品以上亦用漢服;重熙以後,大禮並漢服矣。常朝仍遵會同之制。皇帝通天冠,諸祭還及冬至、朔日受朝、臨軒拜王公、元會、冬會服這。冠加金博山,附蟬十二,首施珠翠。
 黑介幘,發纓以砲,玉若犀簪導。絳紗袍,白紗中單,鄀領,朱棨裾,白裙襦,絳蔽膝,白假帶方心曲領。其革帶佩劍綬,釦閤。若未加元服,則雙童髻,空頂,黑介幘,雙玉導,加寶飾。《元日上壽儀》,皇帝服通天冠,絳紗袍。



 皇太子遠游冠,謁廟還宮、元日、冬至、朔日入朝服之。



 三梁冠,加金附蟬九,首施珠翠。黑介幘發纓翠砲,犀簪導。



 絳紗袍,白紗中單,皂領閤,棨裾,白裙襦,白假帶方心曲領,絳紗蔽膝。其革帶劍佩綬,釦寫與上同,後改用白釦、墨閤。



 未冠,則雙單髻,空橫,黑介
 幘,雙玉導,加寶飾。《冊皇太子儀》,皇太子冠遠游,服絳紗袍。



 親王遠游冠,陪祭、朝饗、拜表、大事服之。冠三梁,加金附蟬。黑介幘,青砲導。絳紗單衣,白紗中單,皂領,棨裾,白裙襦。革帶鉤馳騁,假帶曲領方心,絳紗蔽膝,釦閤,劍佩綬。二品以上同。



 諸王遠游冠,三梁,黑介幘,青砲。



 三品以上進賢冠,三梁,寶飾。



 五品以上進賢冠,二梁,金飾。



 九品以上進賢冠,一梁,無飾。



 七品以上去劍佩綬。



 八品以下同公服。



 公服:《勘箭儀》,閣使公服,系履。遼國嘗用公服矣。



 皇帝翼善冠,朔視朝用之。離黃袍,九環帶,白練裙襦,六合靴。



 皇太子遠游冠,五日常朝、元日、乖至受朝服。絳紗單衣,白裙襦,革帶金鉤馳騁,假帶方心,紛崚囊,白釦,烏皮履。



 一品以下、五品以上,冠幘纓,簪導,謁見東宮及餘公事服之。絳紗單衣,白裙襦,帶鉤馳騁,假帶方心,釦履,
 紛崚囊。



 六品以下,冠幘纓,簪導,去紛崚囊,餘並同。



 常服:遼國謂之「穿執」。起居禮,臣僚穿執。言穿靴、執笏也。



 皇帝柘黃袍衫,折上頭巾,九環帶,六合靴,起自宇文氏。



 唐太宗貞觀已後,非元日、冬至受朝及大祭祀,皆常服而已。



 皇太子進德冠,九琪,金飾,絳紗單衣,白裙襦,白釦,烏皮履。



 五品以上,襆頭,亦曰折上巾,紫袍,牙笏,金玉帶。文官佩手巾、算袋、刀子、礪石、金魚袋、
 烏皮六合靴。



 六品以下,襆頭,緋衣,木笏,銀帶,銀魚袋佩,靴同。



 八品九品,襆頭,綠袍,蟻石帶,靴同。



\end{pinyinscope}