\article{卷七十一 列傳第九}

\begin{pinyinscope}

 室昉耶律賢適女裏郭襲耶律阿沒里室昉,字夢奇,南京人。幼謹厚篤學,不出外戶者二十年,雖里人莫識。其精如此。



 會同初,登進士第,為盧龍巡捕官。太宗人汴受冊禮,詔昉知制誥,總禮儀事。天祿中,為南京留守判官。應歷間,累遷翰林學士,出入禁闥十餘
 年。保寧間,兼政事舍人,數延問古今治亂得失,奏對稱旨。上多昉有理劇才,改南京副留守,決訟平允,人皆便之。遷工部尚書,尋改樞密副使,參知政事。



 頃之,拜樞密使,兼北府宰相,加同政事門下平章事。乾亨初,監修國史。



 統和元年,告老,不許。進《尚書無逸篇》以諫,太后聞而嘉獎。二年秋,詔修諸嶺路,昉發民夫二十萬,一日畢功。



 是時,昉與韓德讓、耶律斜軫相友善,同心輔政,整析蠹弊,知無不言,務在息民薄賦,以故法度修明,朝無異議。



 八年,復請致政。詔入朝免拜,賜幾杖,太后遣閣門使李從訓持詔勞問,令常居南京,封鄭國公。初,晉國公主建
 佛寺於南京,上許賜額。昉奏曰:「詔書悉罪無名寺院。今以主請賜額,不惟違前詔,恐此風愈熾。」上從之。表進所撰《實錄》二十卷,手詔褒之,加政事令,賜帛六百匹。



 九年,薦韓德讓自代,不從。上以勸年老苦寒,賜貂皮衾褥,許乘輦入朝。病劇,遣翰林學士張乾就第授中京留守,加尚父。卒,年七十五。上磋悼,輟朝二日,贈尚書令。遺言戒厚葬。恐人譽過情,自志其墓。



 耶律賢適,字阿古真,於越魯不古之子。嗜學有大志,滑稽玩世,人莫之知。惟於越屋質器之,嘗謂人曰:「是人當國,天下幸甚。」



 應歷中,朝臣多以言獲譴,賢適樂於靜退,
 游獵自娛,與親朋言不及時事。會討烏古還,擢右皮室詳穩。景宗在藩邸,常與韓匡嗣、女裡等游,言或刺譏,賢適勸以宜早疏絕,由是穆宗終不見疑,賢適之力也。



 景宗立,以功加檢校太保,尋遙授寧江軍節度使,賜推忠協力功臣。時帝初踐阼,多疑諸王或萌非望,陰以賢適為腹心,加特進同中書門下平章事。保寧二年秋,拜北院樞密使,兼侍中,賜保節功臣。三年,為西北路兵馬都部署。賢適忠介盧敏,推誠待人,雖燕息不忘政務。以故百司首職,罔敢偷惰,累年滯獄悉決之。



 大丞相高勛、契丹行宮都部署女里席寵放恣,及帝姨母、保母勢薰灼。
 一時納賂請謁,門若賈區。賢適患之,言於帝,不報,以病解職,又不允,令鑄手印行事。乾亨初,疾篤,得請。明年,封西平郡王,薨,年五十三。子觀音,大同軍節度使。



 女里,字涅烈袞,逸其氏族,補積慶宮人。應歷初,為習馬小底,以母憂去。一日至雅伯山,見一巨人,惶懼走。巨人止之曰:「勿懼,我地只也。葬爾母於斯,當速詣闕,必貴。」



 女里從之,累遷馬群侍中。



 時景宗在藩邸,以女裡出自本宮,待遇殊厚,女里亦傾心結納。及穆宗遇弒,女里奔赴景宗。是夜,集禁兵五百以衛。



 既即位,以翼戴功,加政事令、契丹行宮都部署,賞齎甚渥,尋加守太尉。北漢主劉
 繼元聞女里為上信任,遇其生日必致禮。



 女裡素貪,同列蕭阿不底亦好賄,二人相善。人有氈裘為耳子所著者,或戲曰:「若遇女里、阿不底,必盡取之!」傳以為笑。其貪猥如此。



 保寧末,坐私藏甲五百屬,有司方按詰,女裡袖中又得弒樞密院使蕭思溫賊書,賜死。



 女里善識馬,嘗行郊野,見數馬跡,指其一曰:「此奇駿也!」以己馬易之,果然。



 郭襲,不知何郡人。性端介,識治體。久淹外調。景宗即位,召見,對稱旨,知可任以事,拜南院樞密使,尋加兼政事令。



 以帝數游獵,襲上書諫曰:「昔唐高祖好獵,薊世長言
 不滿十旬未足為樂,高祖即日罷,史稱其美。伏念聖祖創業艱難,修德布政,宵肝不懈。穆宗逞無厭之欲,不恤國事,天下愁怨。



 陛下繼統、海內翕然望中興之治。十餘年間,征伐未已,而寇賊未弭;年穀雖登,而瘡痍未復。五宜戒懼修省,以懷永圖。



 側聞您意游獵,甚於往日。萬一有銜橛之變,搏噬之虞,悔將何及?況南有強敵伺隙而動,聞之得無生心乎?伏望陛下節從禽酣飲之樂,為生靈社稷計,則有無疆之休。」上覽而稱善,賜協贊功臣,拜武定軍節度使,卒。耶律阿沒里,字蒲鄰,遙輦嘲古可汗之四世孫。幼聰敏。



 保寧中、為南院宣徽使。統和初,皇太后稱制,與耶律斜軫參預國論,為都統。以征高麗功,遷北院宣徽使,加政事令。



 四年春,宋將曹彬、米信等侵燕,上親征,阿沒里為都監,屢破敵軍。十二年,行在多盜,阿沒里立禁捕法,盜始息。



 先是,叛逆之家,兄弟不知情者亦連坐。阿沒里諫曰:「夫兄弟雖日同胞,賦性各異,一行逆謀,雖不與知,輒坐以法,是刑及無罪也。自今,雖同居兄弟,不知情者免連坐。」太后嘉納,著為令。致仕,卒。



 阿沒里性好聚斂,每從征所掠人口,聚而建城,請為豐州,就以家奴閻貴為刺史,時議鄙之。子賢哥,左夷離畢。



 論曰:「景宗之世,人望中興,豈其勤心庶績而然,蓋承穆宗熒虐之餘,為善易見;亦由群臣多賢,左右弼諧之力也。



 室昉進《無逸》之篇,郭襲陳諫獵之疏,阿沒裡請免同氣之坐,所謂仁人之言,其利薄哉。賢過忠介,亦近世之名臣。女裡貪猥,後人所當取鑒者也。」



\end{pinyinscope}