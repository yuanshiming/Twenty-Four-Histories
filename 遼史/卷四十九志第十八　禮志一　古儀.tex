\article{卷四十九志第十八 禮志一 古儀}

\begin{pinyinscope}

 理自天設,情由人生。以理制情,而禮樂之用行焉。林豺梁獺,是生郊譔;窪尊燔黍,是生燕饗;堉滊瓦棺,是生喪葬;儷皮緇布,是生婚冠。皇造帝秩,三王彌文。一文一質,蓋本於忠。變通革弊,與時宜之。唯聖人為能通其意。執理者膠瑟聚訟,不適人情;徇情者稊稗綿蕝,,不中天理。泰、漢而降,君子無取焉。



 遼本朝鮮故壤,箕子八條之教,
 流風遺俗,蓋有存者。自其上世,緣情制宜,隱然有尚質之風。遙輦胡刺可汗制祭山儀,蘇可汗制瑟瑟儀,阻午可汗制柴冊、再生儀。其情樸,其用儉。



 敬天恤災,施惠本孝,出於悃忱,殆有得於膠瑟聚訟之表者。



 太古之上,椎輪五禮,何以異茲。太宗克晉,稍用漢禮。



 今國史院有金陳大任《遼禮儀志》,皆其國俗之故,又有《遼朝雜禮》,漢儀為多。別得宣文閣所藏耶律儼《志》,視大任為加詳。存其略,著於篇。



 吉儀祭山儀:設天神、地祗位於木葉山,東鄉;中立君樹,前植
 群樹,以像朝班;又偶植二樹,以為神門。皇帝、皇后至,夷離畢具禮儀。牲用赭白馬、玄牛、赤白羊,皆牡。僕臣曰旗鼓拽刺,殺牲,體割,懸之君樹。太巫以酒酹牲。禮官曰敵烈麻都,奏「儀辦」。皇帝服金文金冠,白綾袍,絳帶,懸魚,三山絳垂,飾犀玉刀錯,絡縫烏靴。皇后御絳溗,絡縫紅袍,懸玉佩,雙結帕,絡縫烏靴。皇帝、皇后御鞍馬。群臣在南,命婦在北,服從各部旗幟之色以從。皇帝、皇后至君樹前下馬,升南壇御塌坐。群臣、命婦分班,以次入就位;合班,拜訖,復位。皇帝、皇后詣天神、地祗位,致奠;閣門使讀祝訖,復位坐。北府宰相及惕
 隱以次致奠於君樹,遍及群樹。樂作。群臣、命婦退。皇帝率孟父、仲父、季父之族,三匝神門樹;餘族七匝。皇帝、皇后再拜,在位者皆再拜。上香,再拜如初。



 皇帝、皇后升壇,御龍女方茵坐。再聲警,詣祭東所,群臣、命婦從,班列如初。巫衣白衣,惕隱以素巾拜而冠之。巫三致辭。每致辭,皇帝皇后一拜,在位者皆一拜。皇帝、皇后各舉酒二爵,肉二器,再奠。大臣、命婦右持酒,左持肉各一器,少後立,一奠。命惕隱東向擲之。皇帝、皇后六拜,在位者皆六拜。皇帝、皇后復位,坐。命中丞奉茶果、餅餌各二器,奠於天神、地祗位。執事郎君二十人
 持福酒、胙肉,詣皇帝、皇后前。太巫奠酹訖,皇帝、皇后再拜,在位者皆再拜。皇帝、皇后一拜,飲福,受胙,復位,坐。在位者以次飲。皇帝、皇后率群臣復班位,再拜。聲蹕,一拜。退。



 太宗幸幽州大悲閣,遷白衣觀音像,建廟木葉山,尊為家神。於拜山儀過樹之後,增「詣菩薩堂儀」一節,然後拜神,非胡刺可汗之故也。興宗先有事於菩薩堂及木葉山遼河神,然後行拜山儀,冠服、節文多所變更,後因以為常。神主樹木,懸牲告辦,班位奠祝,致嘏飲福,往往暗合於禮。天理人情,放諸
 四海而準,信矣夫。興宗更制,不能正以經術,無以大過於昔,故不載。



 瑟瑟儀:若旱,擇吉日行瑟瑟儀以祈雨。前期,置百柱天棚。及期,皇帝致奠於先帝御容,乃射柳。皇帝再射,親王、宰執以次各一射。中柳者質志柳者冠服,不中者以冠服質之。



 不勝者進飲於勝者,然後各歸其冠服。又翼日,植柳天棚之東南,巫以酒醴、黍稗薦植柳,祝之。皇帝、皇后祭東方畢,子弟射柳。皇族、國舅、群臣與禮者,賜物有差。既三日雨,則賜敵麻都馬四匹、衣四襲;否則以水沃之。



 道宗清寧元年,皇帝射柳訖,詣風師壇,再拜。



 柴冊儀:擇吉日。前期,置柴冊殿及壇。壇之制,厚積薪,以木為三級壇。置其上。席百尺氈,龍女方茵。又置再生母後搜索之室。皇帝入再生室,行再生儀畢,八部之叟前導後扈,左右扶翼皇帝冊殿之東北隅。拜日畢,乘馬,選外戚之老者御。



 皇帝疾馳,僕,御者、從者以氈覆之。皇帝詣高阜地,大臣、諸部帥列儀仗,遙望以拜。皇帝蹤使敕曰:「先帝升遐,有伯叔父兄在,當選賢者。沖人不德,何以為謀?」群臣對曰:「臣等以先帝厚恩,陛下明德,咸願盡心,敢有他圖。」皇帝令曰:「必從汝等所
 願,我將信明賞罰。爾有功,陟而任之;爾有罪,黜而棄之。若聽朕命,則當漠之。」僉曰:「唯帝命是從。」



 皇帝於所識之地,封土石以志之。遂行。拜先帝御容,宴饗群臣。翼曰,皇帝出冊殿,護衛太保扶翼升壇。奉七廟神主置龍女方茵。北、南府宰相率群臣圜立,各舉氈邊,贊祝訖,樞密使奉玉寶、正冊入。有司讀冊訖,樞密使稱尊號以迸,群臣三稱:萬歲」,皆拜。宰相、北南院大王、諸部帥迸赭、白羊各一群。皇帝更衣,拜諸帝御容。遂宴群臣,賜齎各有差。拜曰儀:皇帝升露臺,設褥,向日再拜,上香。門使通,閣使
 或副、應拜臣僚殿左右階陪位,再拜。皇帝升坐。奏牘訖,北班起居畢,時相己下通名再拜,不出班,奏「聖躬萬福」,又再拜,備抵候。宣徽己下橫班同。諸司、閣門、北面先奏事;餘同。教坊與臣僚同。



 告廟儀:至日,臣僚昧爽朝服,詣太祖廟。次引臣僚,合班,先見御容,再拜畢,引班首左上,至褥位,再拜。贊上香,揖欄內上香畢,復褥位,再拜。各祗候立定。左右舉告廟祝版,於御容前跪捧。中書舍人俯跪,讀訖,俯興,退。引班首左下,復位,又再拜。分引上殿,次第進酒三。分班引出。



 謁廟儀:至日昧爽,南北臣僚各具朝服,赴廟。車駕至,臣僚於門外依位序立,望駕鞠躬。班首不出班,奏「聖躬萬福。」



 舍人贊各祗候畢,皇帝降車,分引南北臣僚左右入,至丹墀褥位,合班定,皇帝升露臺褥位。宣徽贊皇帝再拜,殿上下臣僚陪位皆再拜。上香畢,退,復位,再拜。分引臣僚左右上殿位立;進御容酒依常禮。若即退,再拜。舍人贊「好去」,引退。禮畢。



 告廟、謁廟,皆曰拜容。以先帝、先後生辰及忌辰行禮,自太宗始也。其後正旦、皇帝生辰、諸節辰皆行之。若忌辰及車駕行幸,亦嘗遣使行禮。凡瑟瑟、柴
 冊、再生、納後則親行之。凡柴冊、親征則告;幸諸京則謁。四時有薦新。



 孟冬朔拜陵儀,有司設酒饌於山陵。皇帝、皇后駕至,敵烈麻都奏「儀辦」。閣門使贊皇帝、皇后詣位四拜訖,巫贊祝燔胙及時服,酹酒薦牲。大巨、命婦以次燔胙,四拜。皇帝、皇后率群臣、命婦,循諸陵各三匝。還宮。翼曰,群臣入謝。



 爇節儀:皇帝即位,凡征伐叛國俘掠人民,或臣下進獻人口,或犯罪沒官戶,皇帝親覽閑田,建州縣以居之,設官治其事。及帝崩,所置人戶、府庫、錢粟,穹廬中置
 小氈殿,帝及后妃皆鑄金像納焉。節辰、忌曰、朔望,皆致祭於穹廬之前。



 又築土為臺,高丈餘,置大盤於上,祭酒食撒於其中,焚之,國俗謂之爇節。



 歲除儀:初夕,敕使及夷離畢率執事郎君至殿前,以鹽及羊膏置爐中燎之。巫及大巫以次贊祝火神訖,閣門使贊皇帝面火再拜。



 初,皇帝皆親拜,至道宗始命夷離畢拜之。



\end{pinyinscope}