\article{卷二十四本紀第二十四 道宗四}

\begin{pinyinscope}

 五年春正月壬申,如混同江。癸酉,賜宰相耶律孝傑名仁傑。乙亥,如山榆澱。



 三月辛未,以宰相仁傑獲頭鵝,加侍中。壬辰,以北院樞密使魏王耶律乙辛知南院大王事,加於越,知北院樞密使事耶律霂為北院樞密使,北院樞密副使耶律特裏底知北院樞密使事,左夷離畢耶律世遷同知北院樞密使事。



 夏四月己未,如納葛濼。



 五月丁亥,謁慶陵。以契丹行宮都部署耶律燕哥為南府宰相,北面林牙耶律永寧為夷離畢,同知南院樞密使事蕭撻不也及前副點檢、駙馬都尉蕭酬斡並封蘭陵郡王。



 六月辛亥,阻卜來貢。丁巳,以北府宰相、邁西郡王蕭餘里也為西北路招討使。己未,遣使錄囚。是月,放進士劉瓘等百一十三人。



 秋七月己卯,獵夾山。



 八月庚申,命有司撰《太宗神功碑》,立於南京。



 九月己卯,詔諸路毋禁僧徒開壇。壬午,禁扈從擾民。



 冬十月戊戌,夏國遣使來貢。己亥,駐蹕獨盧金。壬子,詔惟皇子仍一字王,餘並削降。丁巳,振平州貧民。己未,以趙王楊績為遼西
 郡王,魏王耶律乙辛降封混同郡王,吳王蕭韓家奴蘭陵郡王,致仕。



 十一月丁丑,召沙門守通開壇於內殿。癸未,復南京流民差役三年,被火之家免租稅一年。



 十二月丙午,彗星犯尾。乙卯,幸西京。戊午,行再生禮,赦雜犯死罪以下。



 六月春正月癸酉,如鴛鴦濼。辛卯,耶律乙辛出知興中府事。



 三月庚寅,封皇孫延禧為梁王,忠順軍節度使耶律頗德雨院大王,耶律仲禧南院樞密使,戶部使陳毅參知政事。



 夏四月乙卯,獵炭山。



 五月壬申,免平州復業民租賦一年。康寅,以旱,禱雨,命左右以水相沃,俄而雨
 降。



 六月戊戌,駐蹕納葛濼。戊申,以度支使王績參知政事。



 庚戌,女直遣使來貢。



 秋七月戊辰,觀市。癸未,為皇孫梁王延禧設旗鼓拽刺六人衛護之。甲申,獵沙嶺。



 九月壬寅,祠木葉山。乙酉,駐蹕藕絲澱。



 冬十月己未朔,省同知廣德軍節度使事,命奉先軍節度使兼巡警乾、顯二州。丁卯,耶律仁傑出為武定軍節度使。庚午,參知政事劉詵致仕。癸酉,以陳毅為漢人行官都部署,王績同知樞密院事。辛巳,回鶻遣使來貢。



 十一月己丑朔,日有食之。癸卯,召群臣議政。



 十二月甲子,以耶律特裏底為孟父敞穩。乙丑,以蕭撻不也為北府宰相,耶律世遷知北
 院樞密使事,耶律慎思同知北院樞密使事。庚午,免西京流民租賦一年。甲戌,減民賦。丁亥,豫行五旦禮。戊子,如混同江。



 七年春正月戊申,五國部長來貢。甲寅,女直貢良馬。



 二月甲子,如魚兒添。



 夏五月壬子,駐蹕嶺西。癸丑,有司奏永清、武清、固安三縣蝗。甲寅,以蕭撻不也兼殿前都點檢,蕭酬斡為漢人行宮都部署兼知樞密院事。



 六月甲子,詔月祭觀德殿,歲寒食,諸帝在時生辰及忌日,詣景宗御容殿致奠。丙寅,阻卜餘古赧來貢。丁卯,以翰林學士王言敷參知政事,封北院宣徽使石篤漆水郡王。



 秋
 七月戊子,如秋山。丙申,謁慶陵。



 八月丁卯,射鹿赤山,加國場使涅葛為靜江軍節度使。



 九月戊子,次懷州,命皇后謁懷陵。辛卯,次祖州,命皇后謁祖陵。乙巳,駐蹕藕絲澱。



 冬十月戊辰,以惕隱王九為南院大王,夷離畢莢抄只為彰國軍節度使。



 十一月乙酉,詔歲出官錢,振諸宮分及邊戍貧戶。丁亥,幸附馬都尉蕭酬斡第,方飲,宰相梁穎諫曰:「天子不可飲人臣家。」上即還宮。己亥,高麗遣使來貢。辛亥,除絹帛尺度狹短之令。



 十二月丁卯,武定軍節度使耶律仁傑以罪削爵為民。辛未,知興中府事耶律乙辛以罪囚於來州。



 八年春正月甲申,如混同江。丁酉,鐵驪、五國諸長各貢方物。二月戊午,如山榆澱。辛酉,詔北、南院官,凡給驛者,必先奏聞。貢新及奏獄訟,方許馳驛,餘並禁之。己巳,夏國獲宋將張天一,遣使來獻。壬申,以耶律頗德為南府宰相兼知北院樞密使,燕哥為惕隱,蕭撻不也兼知契丹行宮都部署事。三月庚戌,黃龍府女直部長術乃率部民內附,予官,賜印綬。是月,詔行櫃黍所定升斗。



 夏四月壬戌,以耶律世遷為上京留守。



 六月辛亥朔,駐蹕納葛濼。丙辰,夏國遣使來貢。丁巳,以耶律頗德為北院樞密使,耶律巢哥南府宰相,劉箱商院樞密使,蕭撻不也
 兼知北院樞密使事,王績漢人行宮都部署,蕭酬斡國舅詳穩。乙丑,阻卜長來貢。丙子,以耶律慎思知右夷離畢事。秋七月甲午,如秋山。南京霖雨,沙河溢永清、歸義、新城、安次、武清、香河六縣,傷稼。



 九月庚寅,謁慶陵。丁未,駐蹕藕絲澱。大風雪,牛馬多死,賜扈從官以下衣馬有差。



 冬十月乙卯,詔化哥傅導梁王延禧,加金吾衛大將軍。丙子,謁乾陵。



 十一月壬午,以乙室大王蕭何葛為南院宣徽使,權知奚六部大王事圖趕為本部大王。



 十二月癸丑,烏古敵烈統軍使耶律馬五為北院大王。庚申,降皇后為惠妃,出居乾陵。



 九年春五月辛巳,如春水。



 夏四月丙午朔,大雪,平地丈餘,馬死者十六、七。



 五月,如黑嶺。



 六月己未,駐蹕散水原。甲子,以耶律阿思為契丹行宮都部署,耶律慎思北院樞密副使。庚午,詔諸路檢括脫戶,罪至死者,原之。



 閏月丁丑,以漢人行宮副部署可漢奴為南院大王。戊寅,追謚庶人浚為昭懷太子。丁亥,阻卜來貢。己丑,以知興中府事邢熙年為漢人行宮都部署,漢人行宮部署王績為南院樞密副使。



 秋七月乙巳,獵馬尾山。丁巳,謁慶陵。癸亥,禁外官郡內貸錢取息及使者館於民家。



 八月,高麗王徽薨。



 九月癸卯朔,日有食之。乙酉,射熊於白石
 山,加圍場使涅葛左金吾衛大將軍。己巳,以高麗王徽子三韓國公勛權知國事。辛未,五國部長來貢。壬申,召北、南樞密院官議政事。



 冬十月丁丑,謁觀德殿。己卯,甫院樞密使劉筠薨。壬辰,混同郡王耶律乙辛謀亡入宋,伏誅。



 十一月丙午,進封梁王延禧為燕國王,大赦。以南院宣微使蕭何葛為南府宰相,三司使王經參知政事兼知樞密事。甲寅,詔僧善知仇校高麗所進佛經,頒行之。已未,定諸令史、譯史遷敘等級。



 十二月丁亥,以邢熙年知南院密使事。辛卯,以王言敷為漢人行宮都部署。高麗三韓國公王勛薨。



 是年,御前放進士李君裕等
 五十一人。



 十年春五月辛丑朔,如春水。丙午,復建南京奉福寺浮圖。



 戊辰,如山榆澱。



 二月庚午朔,萌古國遣使來聘。



 三月戊申,遠萌古國遣使來聘。丁巳,命知制誥王師儒、牌印郎君耶律固傅導燕國王延禧。



 夏四月丁丑,女直貢良馬。



 五月壬戌,駐蹕散水原。乙丑,阻卜來貢。丙寅,降國舅詳穩班位在敞穩之下。



 六月壬辰,禁毀銅錢為器。



 秋七月甲辰,如黑嶺。



 九月癸亥,駐蹕藕絲澱。冬十二月乙未,改慶州大安軍曰興平。是月,改明年為大安,赫雜犯死罪以下。



 大安元年春正月丁酉,如混同江。癸卯,王績知南院樞密使事,邢熙年為中京留守,戊申,以樞密直學士杜公謂參知政事。庚戌,五國酋長來貢良馬。



 二月辛未,如山榆澱。



 夏四月乙酉,宋主(王貢)殂,子煦嗣位,使來告哀。辛卯,酉幸。



 六月戊辰,駐蹕拖古烈。壬申,以王績為甫府宰相,蕭撻不也兼知南院樞密使事。丁丑,遣使吊祭於宋。戊寅,宋遣王真、甄祐等饋其先帝遺物。



 秋七月乙巳,遣使賀宋主即位。戊午,獵於赤山。



 八月丁卯,幸慶州。戊辰,謁慶陵。



 冬十月癸亥,駐蹕好草澱。戊辰,夏國王李秉常遣使報其母梁氏哀。甲申,以蕭撻不也為南院樞密使。



 十
 一月乙未,詔:「比者,外官因譽進秩,久而不調,民被其害。今後皆以資給遷轉。」丁酉,以南女直詳穩蕭袍里為北府宰相。辛亥,史臣進太祖以下七帝《實錄》。丙辰,遣使冊三韓國公王勛弟運為高麗國王。己未,詔僧尼無故不得赴闕。



 十二月甲戌,宋遣蔡卞來謝吊祭。



 二年春正月辛卯,如混同江。己酉,五國諸部長來貢。癸丑,召權翰林學士趙孝嚴、知制誥王師儒等講《五經》大義。



 二月癸酉,駐蹕山榆澱。是月,太白犯歲金。



 三月乙酉,女直貢戶馬。



 夏四月戊戌,北幸。癸丑,遣使加統軍使蕭訛都斡太子太保,碑將老古金吾衛大將軍,蕭雅哥靜
 江軍節度使,耶律燕奴右監門衛大將軍,仍賜資諸軍士。五月丁己朔,以牧馬蕃息歲至百萬,賞群牧官,以次進階。



 乙亥,駐蹕納葛濼。戊寅,宰相梁穎出知興中府事。是月,放進士張穀等二十六人。



 六月丁亥朔,以左夷離畢耶律坦為惕隱,知樞密院事耶律斡特刺兼知左夷離畢事。丙申,阻卜來朝。癸卯,遣使按諸路獄。甲辰,以同知南京留守事耶律那也知右夷離畢事。乙巳,阻卜酋長餘古赦及愛的來朝,詔燕國王延禧相結為友。戊申,以契丹行宮都部署耶律阿思兼知北院大王事。壬子,高墩以下、縣令、錄事兄弟及子,悉許敘用。



 秋七月西巳,
 惠妃母燕國夫人削古厭魅梁王事覺,伏誅,子蘭陵郡王蕭酬斡除名,置邊郡,仍隸興聖宮。戊午,獵沙嶺。



 甲子,賜興聖、積慶二宮貧民錢。乙酉,出粟振遼州貧民。



 九月庚午,還上京。壬申,發粟振上京、中京貧民。丙子,謁二儀、五鸞二殿。乙卯,出太祖、太宗所御鎧仗示燕國王延禧,諭以創業征伐之難,辛巳,召南府宰相議國政。



 冬十月乙酉朔,以樞密副使竇景庸知樞密院事。丙戌,五國部長來貢。丁亥,以夏國王李秉常薨,遣使詔其子乾順知國事。十一月甲戌,為燕國王禧行再生禮,曲赦上京囚。戊寅,高麗遣使謝封冊。癸未,
 出粟振乾、顯、成、懿四州貧民。



 十二月辛卯,以蘭陵郡王蕭撻不也為南院樞密使。己亥,夏國王李乾順遣使上其父遣物。



\end{pinyinscope}