\article{卷七十六 列傳第十四}

\begin{pinyinscope}

 耶律沙耶律抹
 只蕭乾侄討古耶律善補耶律海裡耶律沙,字安隱。其先嘗相遙輦氏。應歷間,累官南府宰相。景宗即位,總領南面邊事。保寧間,宋攻河東,沙將兵救之,有功,加守太保。



 乾亨初,宋復北侵,沙將兵由間道至白馬嶺,阻大澗遇敵。



 沙與諸將欲待後軍至而戰,冀
 王敵烈、監軍耶律抹只等以為急擊之便,沙不能奪。敵烈等以先鋒渡澗,未半,為宋人所擊,兵潰。敵烈及其子蛙哥、沙之子德里、令穩都敏、詳穩唐筈等五將俱沒。會北院大王耶律斜軫兵至,萬矢俱發,敵軍始退。



 沙將趨太原,會漢駙馬都尉盧俊來奔,言太原已陷,遂勒兵還。宋乘銳侵燕,沙與戰於高梁河,稍卻;遇耶律休哥及斜軫等邀擊,敗宋軍。宋主宵遁,至涿州,徽服乘驢車,間道而走。上以功釋前過。



 是年,復從韓匡嗣伐宋,敗績,帝欲誅之,以皇後營救得免。睿智皇后稱制,召賜幾杖,以優其老。復從伐宋,敗劉廷讓、李敬源之軍,賜齎優渥。統和
 六年卒。



 耶律抹只,字留隱,仲父隋國王之後。初以皇族入侍。景宗即位,為林牙,以干給稱。保寧間,遷樞密副使。



 乾亨元年春,宋攻河東,商府宰相耶律沙為都統,將兵往援,抹只監其軍。及白馬嶺敗,僅以身免。宋乘銳攻燕,將奚兵翊休哥擊敗之。上以功釋前過。是年冬,從都統韓匡嗣伐宋,戰於滿城,為宋將所給,諸軍奔潰;獨抹只部伍不亂,徐整破鼓而歸。璽書褒諭,改南海軍節度使。乾亨二年,拜樞密副使。



 統和初,為東京留守。宋將曹彬、米信等侵邊,抹只引兵至南京,先繕守禦備。及車駕臨幸,抹只與耶律休哥
 逆戰於涿之東,克之,遷開遠軍節度使。



 故事,州民歲輸稅,鬥粟折錢五,抹只表請折錢六,部民便之。統和末卒。



 蕭干,小字項烈,字婆典,北府宰相敵魯之子。性質直。



 初,察割之亂,其黨胡古只與乾善,使人召之。乾曰:「吾豈能從逆臣!」縛其人送壽安王。賊平,上嘉其忠,拜群牧都林牙。復以伐烏古功,遷北府宰相,改突呂不部節度使。



 乾亨初,宋伐河東,乘勝侵燕,詔幹拒之,戰於高梁河。



 耶律沙退走,干與耶律休哥等並力戰敗之,上手敕慰勞。自是每征伐必參決軍事。加政事令。二年,宋兵圍瓦橋,夜
 襲我營,干及耶律勻骨戰卻之。



 時皇后以父呼幹。及後為皇太后稱制,乾數條奏便宜,多見聽用。統和四年卒。侄討古。



 討古,字括寧,性忠簡。



 應歷初,始入侍。會冀王敵烈、宣徽使海思謀反,討古與耶律阿列密告於上,上嘉其忠,詔尚樸謹公主。保寧末,為南京統軍使。



 乾亨初,宋侵燕,討古與北院大王奚底拒之,不克,軍潰。



 討古等不敢復戰,退屯清河。帝聞其敗,遣使責之曰:「卿等不嚴偵候,用兵無法,遇敵即敗,奚以將為!」討古懼。頃之,援兵至,討古奮力以敗宋軍。上釋其罪,降為而京侍衛親軍都指揮使。
 四年卒。



 耶律善補,字瑤升,孟父楚國王之後。純謹有才智。



 景宗即位,授千牛衛大將軍,遷大同軍節度使。及伐宋,韓匡嗣與耶律沙將兵由東路進,善補以南京統軍使由西路進。



 善補聞匡嗣失利,斂兵還。乾亨未,與宋軍戰於滿城,為伏兵所圍,斜軫救之獲免。以失備,大杖決之。



 統和初,為惕隱。會宋來侵,善補為都元帥逆之,不敢戰,故嶺西州郡多陷,罷惕隱。以其叔安端有匡輔世宗功,上愍之,徵善補為南府宰相,遷南院大王。



 會再舉伐宋,欲攻魏府,召眾集議。將士以魏城無備,皆言可攻。善補曰:「攻
 固易,然城大叵量,若克其城,士卒貪俘掠,勢必不可遏。且傍多巨鎮,各出援兵,內有重敵,何以當之?」上乃止。



 善補性懦,守靜。凡征討,憚攻戰,急還,以故戰多不利。



 年七十四卒。



 耶律海裏,字留隱,令隱拔裡得之長子。察割之亂,其母的魯與焉。遣人召海裏,海裏拒之。亂平,的魯以子故獲免。



 海裏儉素,不喜聲利,以射獵自娛。雖居閑,人敬之若貴官然。保寧初,拜彰國軍節度使,遷惕隱。秩滿,稱疾不仕。久之,復為南院大王。及曹彬、米信等來侵,海裡有卻敵功,賜資忠保義匡國功臣。



 帝屢親征,海裏在南院十
 餘年,鎮以寬靜,戶口增給,時議重之。封漆水郡王,遷上京留守,薨。詔以家貧給葬具。



 論曰:「當高梁、朔州之捷,偏神之將如沙與抹只,既因休哥、斜軫類見其功,所謂失之東隅,收之桑榆。若蕭干、海裏拒察割之招,討古告誨思之變,則不止有戰功而已。其視善補畏懦,豈不優哉。」



\end{pinyinscope}