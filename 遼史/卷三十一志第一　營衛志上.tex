\article{卷三十一志第一 營衛志上}

\begin{pinyinscope}

 上古之世,草衣木食,巢居穴處,熙熙于於,不求不爭。



 愛自炎帝政衰,蚩尤作亂,始制干戈,以毒天下。軒轅氏作,戮之琢鹿之阿。處則象吻於宮,行則懸旄於纛,以為天下萬世戒。於是師兵營衛,不得不設矣。



 冀州以南,歷洪水之變,夏後始制城郭。其人土著而居綏服之中,外奮武衛,內揆文教。守在四邊。營衛之設,以備非常而已。並、
 營以北,勁風多寒,隨陽遷徒,歲無寧居,曠土萬里,寇賊奸宄乘隙而作。營衛之設,以為常然。其勢然也。



 有遼始大,設制尤密。居有宮衛,渭之斡魯朵;出有行營,謂之捺缽;分鎮邊圍,謂之部族。有事則以攻戰為務,閑暇則以畋漁為生。無曰不營,無在不衛。立國規模,莫重於此。作《營衛志》宮衛遼國之法:天子踐位置宮衛,分州縣,析部族,設官府,籍戶口,備兵馬。崩則扈從後妃宮帳,以奉陵寢。有調發,則丁壯從戎事,老弱居守。



 太祖曰弘義宮,應天皇后曰長
 寧宮,太宗曰永興宮,世宗曰積慶宮,穆宗曰延昌宮,景宗曰彰愍宮,承天太后曰崇德宮,聖宗曰興聖宮,興宗曰延慶宮,道宗曰太和宮,天柞曰永昌宮。



 文章文皇太弟有敦睦宮,丞相耶律隆運有文忠王府。凡州三十八,縣十,提轄司四十一,石烈二十三,瓦裏七十四,抹里九十八,得里二,閘撒十九。為正戶八萬,蕃漢轉戶十二萬三幹,共二十萬三千戶。



 算斡魯朵,太祖置。國語心腹曰「算」,宮曰「斡魯朵」。



 是為弘義宮。以心腹之衛置,益以渤海俘,錦州戶。其斡魯朵在臨潢府,陵寢在祖州東南二十里。正戶八干,藩漢轉戶七千,出騎軍六千。



 州五:錦、祖、嚴、祺、銀。



 縣一:富義。



 提轄司四:南京、酉京、奉聖州、平州。



 石烈二,曰須,曰速魯。



 瓦裏四:曰合不,曰撻撒,曰慢押,曰虎池。



 抹里四,曰膻,曰預墩,曰鶻突,曰糾里闡。



 得里二:曰述壘北,曰述壘南。



 國阿孽斡魯朵,太宗置。收國曰「國阿輦」。是為永興宮,初名孤穩斡魯朵。以太祖平渤海俘戶,東京、懷州提轄司及雲州懷仁縣、澤州灤河縣等戶置。其斡魯朵在游古
 河側,陵寢在懷州南三十里。正戶三千,蕃漢轉戶七千,出騎軍五千。



 州四:懷、黔、開、來。



 縣二:保和、灤河。



 提轄司四:南京、西京、奉聖州、平州。



 石烈一:北女古。



 瓦裏四:曰抹,曰母,曰合李只,曰述壘。



 抹里十三:曰述壘軫,曰大隔蔑,曰小隔蔑,曰母,曰歸化不術,曰唐括,曰吐谷,曰百爾瓜忒,曰合魯不只,曰移馬不只,曰膻,曰清帶,曰速穩。



 闡撤七:曰伯德部,曰守獄,曰穴骨只,曰合不頻尼,曰虎里狘,曰耶裏只挾室,曰僧隱令公。



 耶魯碗斡魯朵,世宗置。興盛曰。耶魯碗「。是為積慶宮。



 以文獻皇帝衛從及太祖俘戶,及雲州提轄司,並高、宜等州戶置。其斡魯朵在土河東,陵寢在長寧宮北。正戶五干,蕃漢轉戶八千,出騎軍八千。



 州三:康、顯、宜。



 縣一,山東。



 提轄司四。



 石烈一:分臘。



 瓦裏八:曰撻撤,曰合不,曰吸烈,曰通里,曰潭馬,曰槊不,曰耶裏直,曰耶魯兀也。



 抹里十:曰紇斯直,曰蠻葛,曰厥里,曰潭馬忒,曰出懶,曰速忽魯碗,曰牒裡得,曰閻馬,曰迭裡特,曰女古。



 蒲速碗斡魯朵,應天皇太后置。興隆曰「蒲速碗」。是為長寧宮。以遼州及海濱縣等戶置。其斡魯朵在高州,陵寢在龍化州東一百里。世宗分屬讓國皇帝宮院。正戶七千,蕃漢轉戶六千,出騎軍五干。



 州四:遼、儀坤、遼西、顯。



 縣三:奉先、歸義、定霸。



 提轄司四。



 石烈一,北女古。



 瓦裏六:曰潭馬,曰合不,曰達撤,曰慢押,曰耶裏只,曰渾只。



 抹里十三:曰渾得移鄰稍瓦只,曰合四卑臘因鐵里卑稍只,曰奪羅果只,曰拿葛只,曰合里只,曰婆渾昆母溫,曰阿魯埃得本,曰東廝里門,曰西廝里門,曰東钁里,曰西钁里,曰牒得只,曰滅母鄰母。



 奪里本斡魯朵,穆宗置。是為延昌宮。討平曰「奪里本」。



 以國阿輦斡魯朵戶及阻卜俘戶,中京提轄司、南京制置司、咸、信、韓等州戶置。其斡魯朵在糾雅里山南,陵寢在京商。正戶一千,蕃漢轉戶三千,出騎軍二千。



 州二:遂、韓。



 提轄司三,中京、南京、平州。



 石烈一:曰須。



 瓦裏四:曰抹骨古等,曰兀沒,曰潭馬,曰合里直。



 抹里四:曰抹骨登兀沒滅,曰土木直移鄰,曰息
 州決里,曰莫瑰奪石。



 監母斡魯朵,景宗置。是為彰愍宮。遺留曰「監母」。以章肅皇帝侍衛及武安州戶置。其斡魯朵在合魯河,陵寢在祖州南。正戶八千,蕃漢轉戶一萬,出騎軍一萬。



 州四:永、龍化、降聖、同。



 縣二:行唐、阜俗。



 提轄司四。



 石烈二:曰監母,曰南女古。



 瓦裏七:曰潭馬,曰奚烈,曰埃合里直,曰蠻雅葛,曰特末,曰烏也,曰滅合里直。



 抹里十一:曰尼母曷烈因稍瓦直,曰察改因麻
 得不,曰移失鄰斡直,曰辛古不直,曰撒改真,曰牙葛直,曰虎狘阿裏鄰,曰潑昆,曰潭馬,曰閘臘,曰楚兀真果鄰。



 孤穩斡魯朵,承天太后置。是為崇德宮。玉曰「孤隱」。



 以乾、顯、雙三州戶置。其斡魯朵在土河東,陵袝景宗皇帝。正戶六千,蕃漢轉戶一萬,出騎軍一萬。



 州四:乾、川、雙、貴德。



 縣一,潞上京。



 提轄司三:南京、西京、奉聖州。



 石烈三:曰钁里。曰滂,曰迭裡特女古。



 瓦裏七:曰達撒,曰耶裏,曰合不,曰歇不,曰合里直,曰慢押,曰耶裏直。



 抹里十一:曰阿里廝直述壘,曰預篤溫稍瓦直,曰潭馬,曰賃預篤溫一臘,曰牙葛直,曰牒得直,曰虎溫,曰孤溫,曰撒裏僧,曰阿里葛斯過鄰;曰鐵里乖穩钁里。



 閘撒五:曰合不直迷裏幾頻你,曰牒耳葛太保果直,曰爪里阿本果直,曰憎隱令公果直,曰老昆令公果直。



 女古斡魯朵,聖宗置。是為興聖宮。金曰「女古」。以國阿輦、
 耶魯碗、蒲速碗三斡魯朵戶置。其斡魯朵在女混活直,陵寢在慶州南安。正戶一萬,蕃漢轉戶二萬,出騎軍五千。



 州五:慶、隔、烏上京、烏東京、霸。



 提轄司四。



 石烈四:曰毫兀真女姑,曰拿幾真女室,曰女特裡特,曰女古滂。



 瓦裏六:曰女古,曰蒲速碗,曰鶻篤,曰乙抵,曰翁,曰埃也。抹里九:曰乙辛不只,曰鐵乖溫,曰埃合里只,曰
 嘲瑰,曰合魯山血古只,曰奪忒排登血古只,曰勞骨,曰虛沙,曰土鄰。



 閘撒五:曰達鄰頻你,曰和裡懶你,曰爪阿不厥真,曰粘獨裏僧,曰袍達夫人撅只。



 窩胯碗斡魯朵,興宗置。是為延慶宮。孽息曰「窩篤碗」。以諸斡魯朵及饒州戶置。其斡魯朵在高州西,陵寢在上京慶州。



 正戶七千,番漢轉戶一萬,出騎軍一萬。



 州三:饒、長春、泰。



 提轄司四。



 石烈二:曰窩篤碗、曰鶻篤骨。



 瓦裏六:曰窩篤碗,曰廝把,曰廝阿,曰糾里,曰得裏,曰歐烈。



 抹裡六:曰歐裏本,曰燕斯,曰緬四,曰乙僧,曰北得裏,曰南得里。



 阿思斡魯朵,道宗置。是為太和宮。寬大曰「阿思」。以諸斡魯朵御前承應人及興中府戶置。其斡魯朵在好水濼,陵寢在上京慶州。正戶一萬,蕃漢轉戶二萬,出騎軍一萬五千。



 石烈二,曰阿廝,曰耶魯。



 瓦裏八:曰阿廝,曰耶魯,曰得裏,曰糾里,曰撒不,
 曰鶻篤,曰蒲速斡,曰曷烈。



 抹里七:曰恩州得裏,曰斡奢得裏,曰歐裏本,曰特滿,曰查刺土鄰,曰糾里,曰阿里廝迷裏。



 阿魯碗斡魯朵,天祚皇帝置。是為永昌宮。輔佑曰「阿魯碗」。以諸斡魯朵御前承應人,春、宣州戶置。正戶八千,蕃漢轉戶一萬,出騎軍一萬。



 石烈二:曰阿魯碗,曰榆魯碗。



 瓦裏八:曰阿魯斡,曰合里也,曰鶻突,曰敵刺,曰謀魯斡,曰糾里,曰奪里剌,曰特末也。



 抹里八:曰蒲速碗,曰移替,曰斡篤碗,曰特滿,曰
 謀魯碗,曰移典,曰悅,曰勃得本。



 孝文皇太弟敦睦宮,謂之赤實得本斡魯朵。孝曰「赤實得本」。文獻皇帝承應人及渤海俘,建、沈、巖三州戶置。陵寢在祖州西南三十里。正戶三千,蕃漢轉戶五千,出騎軍五千。



 州三:建、沈、巖。



 提轄司一:南京。



 石烈二:曰嘲,曰與敦。



 瓦裏六:曰乙辛,曰得裏,曰奚烈直,曰大潭馬,曰小潭馬,曰與墩。



 抹里二:曰潭馬抹乖,曰柳實。



 閘撤二:曰聶裏頻你,曰打裏頻你。



 大丞相晉國王耶律隆運,本韓氏,名德讓。以功賜國姓,出宮籍,隸橫帳季父房。贈尚書令,溢文忠。無子,以皇族魏王貼不子耶魯為嗣,早卒;天祚皇帝又以皇子敖魯斡繼之。官給葬具,建廟乾陵側。擬諸宮例,建文忠王府。正戶五千,蕃漢轉戶八千,出騎軍一萬。



 州一。



 提轄司六:上京、中京、南京、酉京、奉聖州、平州。



 著帳郎君
 著帳郎君,初,遙輦痕德堇可汗以蒲古只等三族害於越釋魯,籍沒家屬人瓦里。淳欽皇后宥之,以為著帳郎君。世宗悉免。後族、戚、世官犯罪者沒入。



 著帳戶著帳戶,本諸斡魯朵析出,及諸罪沒入者。凡承應小底、司藏、鷹坊、湯藥、尚飲、盥漱、尚膳、尚衣、裁造等役,及宮中、親王抵從、伶官之屬,皆充之。



 凡諸宮衛人丁四十萬八千,騎軍十萬一千。著帳釋宥、沒入,隨時增損,無常額。



\end{pinyinscope}