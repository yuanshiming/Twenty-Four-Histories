\article{卷七十七 列傳第十五}

\begin{pinyinscope}

 蕭撻凜蕭觀音奴耶律題子耶律諧理耶律奴瓜蕭柳高勛奚和朔奴蕭塔列葛耶律撒合蕭撻凜,字駝寧,思溫之再從侄。父術魯列,善相馬,應歷間為馬群侍中。



 撻凜幼敦厚,有才略,通天文。保寧初,為
 宿直官,累任韺劇。統和四年,宋楊繼業率兵由代州來侵,攻陷城邑。撻凜以諸軍副部署,從樞密使耶律斜軫敗之,擒繼業於朔州。六年秋,改南院都監,從駕南征,攻沙堆,力戰被創,太后嘗親臨視。明年,加右監門衛上將軍、檢校太師,遙授彰德節度使。



 十一年,與東京留守蕭恆德伐高麗,破之。高麗稱臣奉貢。



 十二年,夏人梗邊,皇太妃受命總烏古及永興宮分軍討之,撻凜為阻卜都詳穩。凡軍中號令,太妃並委撻凜。師還,以功加兼侍中,封蘭陵郡王。十五年,敵烈部人殺詳穩而叛,遁於西北荒,撻凜將輕騎逐之,因討阻卜之未服者,諸蕃歲貢
 方物充於國,自後往來若一家焉。上賜詩嘉獎,仍命林牙耶律昭作賦,以述其功。撻凜以諸部叛服不常,上表乞建三城以絕邊患,從之。俄召為商京統軍使。



 二十年,復伐宋,擒其將王先知,破其軍於遂城,下祁州,上手詔獎諭。進至擅淵,宋主軍於城煌間,未接戰,撻凜按視地形,取宋之羊觀、鹽堆、鳧雁,中伏弩卒。明日,轊車至,太后哭之慟,輟朝五日。子慥古,南京統軍使。



 蕭觀音奴,字耶寧,奚王搭紇之孫。統和十二年,為右祗候郎君班詳穩,遷奚六部大王。先是,棒秩外,給獐鹿百數,皆取於民,觀音奴奏罷之。



 及伐宋,與蕭撻凜為先鋒,
 降祁州,下德清軍,上加優賞。



 同知南院事,卒。



 耶律題子,字勝隱,北府宰相兀里之孫。善射,工畫。保寧間,為御盞郎君。九年,奉使於漢,具言兩國通好長久之計,其主繼元深加禮重。



 統和二年,將兵與西邊詳穩耶律速撒討陀羅斤,大破之。



 四年,宋將楊繼業陷山西域邑,題子從北院樞密使耶律斜軫擊之,敗賀令圖於定安,授西南面招討都監。宋兵守蔚州急,召外援,題子聞之,夜伏兵道傍。黎明,宋兵果來,過未半而擊之;城中軍出,斜軫復邀之。兩軍俱潰,奔飛狐,地隘不得進,殺傷甚眾。賀令圖復集敗卒來襲蔚州,題子逆戰,破之,應州守
 將自遁。進圍寰州,冒矢石登城,宋軍大潰。當斜軫擒繼業於朔州,題子功居多。



 是年冬,復與蕭撻凜由東路擊宋,俘獲甚眾。後聞宋兵屯易州,率兵逆之,至易境而卒。



 初,題子破令圖,宋將有因傷而僕,題子繪其狀以示宋人,咸嗟神妙。



 耶律諧理,字烏古鄰,突舉部人。統和四年,宋將楊繼業來攻山西,諧理從耶律斜軫擊之,常居先鋒,偵候有功。是歲,伐宋,宋人拒於滹沱河,諧理率精騎便道先濟,獲其將康保威,以功詔世預節度使選。



 太平元年,稍遷本部節度使。六年,從蕭惠攻甘州,不克。



 會阻卜攻圍三克
 軍,諧理與都監耶律涅魯古往救,至可敦城西南,遇敵,不能陣,中流矢卒。



 耶律奴爪,字延寧,太祖異母弟南府宰相蘇之孫。有膂力,善調鷹隼。



 統和四年,宋楊繼業來侵,奴瓜為黃皮室糾都監,擊敗之,盡復所陷城邑。軍還,加諸衛小將軍。及伐宋,有功,遷黃皮室詳穩。六年,再舉,將先鋒軍,取宋游兵於定州,為東京統軍使,加金紫崇祿大夫。從奚王和朔奴伐兀惹,以戰失利,削金紫崇祿階。



 十九年,拜南府宰相。二十一年,復伐宋,擒其將王繼忠於望都,俘殺甚眾,以功加同政事門下平章事。二十六年,為遼興軍節
 度使,尋復為南府宰相。開泰初,加尚父,卒。



 蕭柳,字徒門,淳欽皇后弟阿古只五世孫。幼養於伯父排押之家,多知,能文,膂力絕人。



 統和中,叔父恆德臨終,薦其才,詔入侍衛。十七年,南伐,宋將範庭召列方陣而待。時皇弟隆慶為先鋒,問諸將佐誰敢當者,柳曰:「若得駿馬,則願為之先。」隆慶授以甲騎。



 柳攬轡,謂諸將曰:「陣若動,諸君急攻。」遂馳而前,敵少卻。隆慶席勢攻之,南軍遂亂。柳中流矢,裹創而戰,眾皆披靡。時排押留守東京,奉柳為四軍兵馬都指揮使。



 明年,為北女直詳穩,政濟寬猛,部民畏愛。遷東路統軍使。秩滿,百姓願留復任,許之。
 從伐高麗,遇大蛇當路,前驅者請避;柳曰:「壯士安懼此!」拔劍斷蛇。師還,致仕。



 柳好滑稽,雖君臣燕飲,詼諧無所忌,時人比之俳優。臨終,謂人曰:「吾少有致君志,不能直遂,故以諧進。冀萬有一補,俳優名何避!」頃之,被寢衣而坐,呼曰:「吾去矣!」



 言訖而逝。耶律觀音奴集柳所著詩千篇,目曰《歲寒集》。



 高勛,字鼎臣,晉北平王信韜之子。性通敏。仕晉為闔門使。會同九年,與杜重威來降。太宗入汴,授四方館使。好結權貴,能服勤大臣,多推譽之。



 天祿間,為樞密使,總漢軍事。五年,劉崇遣使來求封冊,詔勛冊崇為大漢神武
 皇帝。應歷初,封趙王,出為上京留守,尋移南京。會宋欲城益津,勛上書請假巡徼以擾之,帝然其奏,宋遂不果城。十七年,宋略地益津關,勛擊敗之,知南院樞密事。景宗即位,以定策功,進王秦。



 保寧中,以甫京郊內多隙地,請疏畦種稻,帝欲從之。林牙耶律昆宣言於朝曰:「高勛此奏,必有異志。果令種稻,引水為畦,設以京叛,官軍何自而入?」帝疑之,不納。尋遷南院樞密使。以毒藥饋駙馬都尉蕭啜里,事覺,流銅州。尋又謀害尚書令蕭思溫,詔獄誅之,沒其產,皆賜思溫家。



 奚和朔奴,字籌寧,奚可汗之裔。保寧中,為奚六部長。統
 和初,皇太后稱制,以耶律休哥領南邊事,和朔奴為南面行軍副部署。四年,宋曹彬、米信等來侵,和朔奴與休哥破宋兵於燕南,手詔褒美。軍還,怙權撾無罪人李浩至死,上以其功釋之。六年冬,南征,將本部軍由別道進擊敵軍於狼山,俘獲甚眾。



 八年,上表曰:「臣竊見太宗之時,奚六部二宰相、二常袞,誥命大常袞班在尊長左右,副常袞總知尊長五房族屬,二宰相匡輔酋長,建明善事。今宰相職如故,二常袞別無所掌,乞依舊制。」從之。



 十三年秋,遷都部署,伐兀惹。駐於鐵驪,秣馬數月,進至兀惹城。利其俘掠,請降不許,令急攻之。城中大恐,皆殊死戰。和
 朔奴知不能克,從副部署蕭恆德議,掠地東南,循高麗北界而還。以地遠糧絕,士馬死傷,詔降封爵,卒。子烏也,郎君班詳穩。



 蕭塔列葛,字雄隱,五院部人。八世祖只魯,遙輦氏時嘗為虞人。唐安祿山來改,只魯戰於照山之陽,敗之。以功為北府宰相,世預其選。



 塔列葛仕開泰間,累遷西南面招討使。重熙十一年,使西夏,諭伐宋事,約元昊出別道以會。十二年,改右夷離畢、同知南京留守,轉左夷雖畢,俄授東京留守,以世選為北府宰相,卒。



 耶律撒合,字率懶,乙室部人,南府宰相歐禮斯子。天祿
 間始仕。應歷中,拜乙室大王,兼知兵馬事。



 乾亨初,宋來侵,詔以本部兵守南京,與北院大王奚底、統軍蕭討古等逆戰,奚底等敗走,獨撒合全軍還。上諭之曰:「拒敵當如此。卿勉之,無憂不富貴。」加守太保。統和間卒。



 論曰:「遼在統和間,數舉兵伐宋,諸將如耶律諧理、奴瓜、蕭柳等俱有降城擒將之功。最後,以蕭撻凜為統軍,直祗澶淵。將與宋戰,撻凜中弩,我兵失倚,和議始定。或者天厭其亂,使南北之民休息者耶!」



\end{pinyinscope}