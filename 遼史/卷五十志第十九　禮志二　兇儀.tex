\article{卷五十志第十九 禮志二 兇儀}

\begin{pinyinscope}

 喪葬儀「聖宗崩,興宗哭臨於節塗殿。大行之夕四鼓終,皇帝率群臣入,柩前三致奠。奉柩出殿之西北門,就轀輬車,藉以素裀。巫者襏除之。詰旦,發引,至祭所,凡五致奠。太巫祈禳。皇族、外戚、大臣、諸京官以次致祭。乃以衣、弓矢、鞍勒、圖畫、馬駝、儀衛等物皆播之。至山
 陵,葬畢,上哀冊。



 皇帝御幄,命改火,面火致奠,三拜。又東向,再拜天地訖,乘馬,率送葬者邊神門之木乃下,東向又再拜。翼日詰旦,率群臣、命婦詣山陵,行初奠之禮。升御容殿,受遺賜。又翼日,再奠如初。興宗崩,首宗親擇地以葬。道宗崩,菆塗於游仙殿,有司奉喪服。天柞皇帝問禮於總知翰林院事耶律固,始服斬衰;皇族、外戚、使相、矮墩官及郎君服如之;餘官及丞應人皆白枲衣巾以入,哭臨。惕隱、三父房、南府宰相、遙輦常袞、九奚首郎君、夷離畢、國舅詳穩、十閘撒郎君、南院大王、郎君,各以次薦奠,進鞍馬、衣襲、犀玉帶等
 物,表列其數。讀訖,焚表。諸國所賻器服,親王、諸京留守奠祭、進賻物亦如之。



 先帝小歛前一日,皇帝喪服上香,奠酒,哭臨。其夜,北院樞密使、契丹行宮都部署入,小歛。翼日,遣北院樞密副使、林牙,以所賵器服,置之幽宮。靈柩升車,親王推之,至食羖之次。蓋遼國舊俗,於此刑羖羊以祭。皇族、外戚、諸京州官以次致祭。至葬所,靈柩降車,就溮,皇帝免喪服,步引至長福岡。是夕,皇帝入陵寢,授遺物於皇族、外戚及諸大臣,乃出。



 命以先帝寢幄,過於陵前神門之木。帝不親往,遣近侍冠服赴之。初奠,皇帝、皇后率皇族、外戚、使相、節
 度使、夫人以上命婦皆拜祭,循陵三匝而降。再奠,如初。辭陵而還。



 上謚冊儀:先一日,於菆塗殿西廊設御幄並臣僚幕次。太樂令展宮懸於殿庭,協律郎設舉麾位。至日,北、南面臣僚朝服,昧爽赴菆塗殿。先置冊、寶案於西廊下。閣使引皇帝至御幄,服寬衣皂帶。臣僚班齊,分班引入,向殿合班立定。引冊案上殿至褥位,寶案次之,設於西階。閣使引皇帝自西階升殿。



 初行,樂作;至位立,樂止。宣徽使揖皇帝鞠躬再拜,陪位者皆再拜。翰林使執臺牫以進,皇帝再拜。引至神座前,跪,奠三,樂作;
 進奠訖,復位,樂止。又再拜,陪位者皆再拜。引皇帝於神座前,北面立。捧冊函者去蓋,進前跪。冊案退,置殿西壁下。引讀冊者進前,俯伏跪,自通全銜臣讀溢冊。讀訖,俯伏興,復位。捧冊函者置於案上,捧寶函者進前跪,讀寶官通銜跪讀訖,引皇帝至褥位再拜,陪位者皆再拜。禮畢,引皇帝歸御幄。初行,樂作;至御幄,樂止。引臣僚分班出。若皇太后奠酒,依常儀。



 忌辰儀:先一日,奏忌辰榜子,預寫名紙。大紙一幅,用陰面後第三行書:文武百僚宰臣某以下謹詣西上閣門進名奉慰。」至日,應拜大小臣僚並皂衣、皂鞓帶,四
 鼓至時,於幕次前,在京於僧寺,班齊,依位望閥敘立。直日舍人跪右,執名紙在前,班首以下皆再拜。引退。名紙於宣徽便面付內侍奏聞。



 宋使祭奠吊慰儀:太皇太后至菆塗殿,服喪服。太后於北間南面垂簾坐,皇帝於南間北面坐。宋使至幕次,宣賜素服、皂帶。更衣訖,引南北臣僚大班,立定。可矮墩以下,並上殿依位立。先引祭奠使副捧祭文南洞門入,殿上下臣僚並舉哀,至丹墀立定。西上閣門使自南階下,受祭文,上殿啟封,置於香案,哭止。祭奠禮物列殿前。引使副南階上殿,至褥位立,揖,再拜。引大
 使近前上香,退,再拜。大使近前跪,捧臺牫,進奠酒三,教坊奏樂,退,再拜。揖中書二舍人跪捧祭文,引大使近前俯伏跪,讀訖,舉哀。引使副下殿立定,哭止。禮物擔床出畢,引使副近南,面北立。勾吊慰使副南洞門入。四使同見大行皇帝靈,再拜。引出,歸幕次。皇太后別殿坐,服喪服。先引北南面臣僚並於殿上下依位立,吊慰使副捧書匣右入,當殿立。閣門使右下殿受書匣,上殿奏「封全」。開讀訖,引使副南階上殿,傳達吊慰訖,退,下殿立。引禮物擔床過畢,引使副近南,北面立。勾祭奠使副入。四使同見,鞠躬,再拜。



 不出班,奏「聖
 躬萬福」,再拜。出班,謝面天顏,又再拜,立定。宣微傳聖旨撫問,就位謝,再拜。引出,歸幕次。皇帝禦南殿,服喪服。使副入見,如見皇太后儀,加謝遠接、撫問、湯藥,再拜。次宣賜使副並從人,祭奠使副別賜讀祭文例物。



 即日就館賜宴。高麗、夏國奉吊、進賻等使禮,略如之。道宗崩,天祚皇帝問禮於耶律固。宋國遣使吊及致祭、歸賵,皇帝喪服,禦游仙之北別殿。使入門,皇帝哭。使者詣柩前上香,讀祭文訖,又哭。有司讀遺詔,慟哭。使者出,少頃,復入,陳賻賵於柩前,皇帝入臨哭。退,更衣,禦游仙殿南之幄殿。



 使者入見且辭,敕有司賜宴
 於館。



 宋使告哀儀:皇帝素冠服,臣僚皂袍、皂鞋帶。宋使奉書右入,開墀內立。西上閣門使右階下殿,受書匣;上殿,欄內鞠躬,奏「封全」。開封,於殿西案授宰相讀訖,皇帝舉哀。



 舍人引使者右階上,欄內俯跪,附奏起居訖,俯興,立。皇帝宣問「南朝皇帝聖躬萬福」。使者跪奏「來時皇帝聖躬萬福」,起,退。舍人引使者右階下殿,於丹墀西,面東鞠躬。通事舍人通使者名某只候見,再拜。不出班,奏「聖躬萬福」,再拜。出班,謝面天顏,再拜。又出班,謝遠接、撫問、湯藥,再拜。贊祗候,引出,就幕次,宣賜衣物。
 引從人入,通名拜,奏「聖躬萬福」,出就幕,賜衣,如使者之儀。又引使者入,面殿鞠躬,贊謝恩。再贊「有敕賜宴」,再拜。贊祗候,出就幕次宴。引從人謝恩,拜敕賜宴,皆如初。宴畢,歸館。



 宋使進遺留禮物儀:百官昧爽朝服,殿前班立。宋遺留使、告登位使副入內門,館伴副使引謝登位使就幕次坐。館伴大使與遺留使副奉書入,至西上閣門外氈位立。閣使受書匣,置殿西階下案。引進使引遺留物於西上閣門入,即於廊下橫門出。



 皇帝升殿坐。宣徽使押殿前班起居畢,引宰臣押文武班起居,引中
 書令西階上殿,奏宋使見榜子。契丹臣僚起居,控鶴官起居。遺留使副西上閣門入,面殿立。舍人引使副西階上殿,附奏起居訖,引西階下殿,於丹墀東,西面鞠躬,通名奏「聖躬萬福」,如告哀使之儀。謝面天顏,謝遠接、撫問、湯藥。引遺留便從人見亦如之。次引告登位使副奉書匣,於東上閣門入,面殿立。閣使東階下殿,受書匣。中書令讀訖,舍人引使副東階上殿,附奏起居。引下殿,南面立。告登位禮物入,即於廊下橫門出。退,西面鞠躬,附奏起居,謝面天顏、遠接等,皆如遺留使之儀。宣賜遺留、登位兩使副並從人衣物,如告
 哀使。



 應坐臣僚皆上殿就立,分引兩使副等於兩廊立。皇帝間使副「沖涉不易」,丹墀內五拜。各引上殿祗候位立。大臣進酒,皇帝飲酒。契丹通,漢人贊,殿上臣僚皆拜,稱「萬歲」。贊各就坐,行酒殽、茶膳,饅頭畢,從人出水飯畢,臣僚皆起。契丹通,漢人贊,皆再拜,稱「萬歲」。各祗候。獨引宋使副下殿謝,五拜。引出。控鶴官門祗候,報閣門無事,供奉官卷班出。



 高麗、夏國告終儀:先期,於行宮左右下御帳,設使客幕次於東南。至日,北面臣僚各常服,其餘臣僚並朝服,入朝。



 使者至幕次,有司以嗣子表狀先呈樞密院,準
 備奏呈。先引北面臣僚並矮墩已上近禦帳,相對立,其餘臣僚依班位序立。引告終人使右入,至丹墀,面殿立。引右上,立;揖少前,拜,跪奏訖,宣問。若嗣子已立,恭身受聖旨。奏訖,復位。嗣子朱立,不宣問。引右下丹墀,面北鞠躬。通班畢,引面殿再拜。



 不出班,奏「聖躬萬福」,再拜。出班,謝面天顏,復位,再拜。出班,謝遠接,復位,再拜。贊祗候,退就幕次。再入,依前面北鞠躬,通辭,再拜;敘戀闕,再拜。贊「好去」。禮畢。



\end{pinyinscope}