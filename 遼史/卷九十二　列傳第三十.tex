\article{卷九十二 列傳第三十}

\begin{pinyinscope}

 耶律棠古蕭得裏底蕭酬斡耶律章奴耶律術者耶律棠古,字蒲速宛,六院郎君葛剌之後。



 大康中,補本班郎君,累遷至大將軍。性坦率,好別白黑,人有不善,必盡言無隱,時號「強棠古」。在朝數論宰相得失,由是久不得調,後出為西北戍長。



 乾統三年,蕭得裏底為西北路招討使,以後族慢侮僚吏。



 棠古不屈,乃罷之。棠古訟之
 朝,不省。天慶初,烏古敵烈叛,召拜烏古部節度使。至部,諭降之。遂出私財及發富民積,以振其困乏,部民大悅,加鎮國上將軍。會蕭得裏底以都統率兵與金人戰敗績,棠古請以軍法論。且曰:「臣雖老,願為國破敵。」不納。



 保大元年,乞致仕。明年,天祚出奔,棠古謁於倒塌嶺,為上流涕,上慰止之,復拜烏古部節度使。及至部,敵烈以五千人來攻,棠古率家奴擊破之,加太子太傅。年七十二卒。蕭得裏底,字乣鄰,晉王孝先之孫。父撒缽,歷官使相。



 得裏底短而僂,外謹內倨。大康中,補只候郎君,稍遷興聖
 宮副使,兼同知中丞司事。大安中,燕王妃生子,得裏底以妃叔故,歷寧遠軍節度使、長寧宮使。壽隆二年,監討達裡得、拔思母二部,多俘而還,改同知商京留守事。



 乾統元年,為北面林牙、同知北院樞密事,受詔與北院樞密使耶律阿思治乙辛餘黨。阿思納賄,多出其罪;得裏底不能制,亦附會之。



 四年,知北院樞密事。夏王李乾順為宋所攻,遣使請和解,詔得裏底與南院樞密使牛溫舒使宋平之。宋既許,得裏底受書之日,乃曰:「始奉命取要約歸,不見書辭,豈敢徙還。」遂對宋主發函而讀。既還,朝議為是。天慶三年,加守司徒,封蘭陵郡王。



 女直初起,
 廷臣多欲乘其未備,舉兵往討;得裏底獨沮之,以至敗衄。天祚以得裏底不合人望,出為西南面招討使。八年,召為北院樞密使,寵任彌筆。是時,諸路大亂,飛章告急者絡繹而至,得裏底不即上聞,有功者亦無甄別。由是將校怨怒,人無鬥志。



 保大二年,金兵至嶺東。會耶律撒八、習騎撒跋等謀立晉王敖盧斡事洩,上召得裏底議曰:「反者必以此兒為名,若不除去,何以獲安。」得裏底唯唯,竟無一言申理。王既死,人心益離。金兵逾嶺,天祚率衛兵西遁。元妃蕭氏,得裏底之侄,謂得裏底曰:「爾任國政,致君至此,何以生為!」得裏底但謝罪,不能對。明日,天
 祚怒,逐得裏底與其子麼撒。



 得裏底既去,為耶律高山奴送金兵。得裏底伺守者怠,脫身亡歸,復為耶律九斤所得,送乏耶律淳。時淳已僭號,得裏底自知不免,詭曰:「吾不能事僭竊之君!」不食數日,卒。子麼撒,為金兵所殺。



 蕭酬斡,字訛裏本,國舅少父房之後。祖阿剌,終採訪使。



 父別里剌,以後父封趙王。



 酬斡貌雄偉,性和易。年十四,尚越國公主,拜駙馬都尉,為只候郎君班詳穩。年十八,封蘭陵郡王。時帝欲立皇孫為嗣,恐無以解天下疑,出酬斡為國舅詳穩,降皇后為惠妃,遷於乾州。初酬斡母
 入朝,擅取驛馬,至是覺,奪其封號;復與妹魯姐為巫蠱,伏誅。詔酬斡與公主離婚,籍興聖宮,流烏古敵烈部。



 天慶中,以妹復尊為太皇太妃,召酬斡為南女直詳穩,遷征東副統軍。時廣州渤海作亂,乃與駙馬都尉蕭韓家奴襲其不備,平之,復敗敵將侯槩於川州。是歲,東京叛,遇敵來擊,師潰;獨酬斡率麾下數人力戰,歿於陣,追贈龍虎衛上將軍。



 耶律章奴,字特末衍,季父房之後。父查剌,養高不仕。



 章奴明敏善談論。大安中,補牌印郎君。乾統元年,累遷右中丞,兼領牌印宿直事。年六,以直宿不謹,降知內客省
 事。



 天慶四年,授東北路統軍副使。五年,改同知咸州路兵馬事。



 及天祚親征女直,蕭胡篤為先鋒都統,章奴為都監。大軍渡鴨子河,章奴與魏國王淳妻兄蕭敵里及其甥蕭延留等謀立淳,誘將座三百餘人亡歸。既而天祚為女直所敗,章奴乃遣敵里、延留以廢立事馳報淳。淳猶豫未決。會行宮使者乙信持天祚御札至,備言章奴叛命,淳對使者號哭,即斬敵里、延留首以獻天祚。章奴見淳不從,誘草寇數百攻掠上京,取府庫財物。至祖州,率僚屬告太祖廟云:「我大遼基業,由太祖百戰而成。今天下土崩,竊見興宗皇帝孫魏國王淳道德隆厚,能
 理世安民,臣等欲立以主社稷。會淳適好草甸,大事未遂。邇來天祚惟耽樂是從,不恤萬機;強敵肆侮,師徒敗績。加以盜賊蜂起,邦國危於累卵。臣等忝預族屬,世蒙恩渥,上欲安九廟之靈,下欲救萬民之命,乃有此舉。實出至誠,冀累聖垂祐。」西至慶州,復把諸廟,仍述所以舉兵之意,移檄州縣、諸陵官僚,士卒稍稍屬心。



 時饒州渤海及侯槩等相繼來應,眾至數萬,趨廣平澱。其黨耶律女古等暴橫不法,劫掠婦女財畜。章奴度不能制,內懷悔恨;又攻上京不克,北走降虜。順國女直阿鶻產率兵追敗之,殺其將耶律彌裏直,擒貴族二百餘人,其妻
 子配役繡院,或散諸近侍為婢;餘得脫者皆遁去。章奴詐為使者,欲奔女直,為邏者所獲,縛送行在,伏誅。



 耶律術者,字能典,於越蒲古只之後,魁偉雄辨。乾統初,補祗候郎君。六年,因柴冊,加觀察使。天慶五年,受詔監都統耶律斡里朵戰。及敗,左遷銀州刺史,徙咸州乣將。



 嘗與耶律章奴謀立魏國王淳。及聞章奴自鴨子河亡去,即引上數人往會之。道為游兵所執,送行在所。上問曰:「予何負卿而反?」術者對曰:「臣誠無憾。但以天下大亂,已非遼有,小人滿朝,賢臣竄斥,誠不忍見天皇帝艱難之業一旦土崩。



 臣所以痛入骨髓而有此舉,非為身
 計。」後數日,復問,術者厲聲數上過惡,陳社稷危亡之本,遂殺之。



 論曰:「遼末同事之臣,其善惡何相遠也!棠古骨鯁不屈權要,兩鎮烏古,恩威並著。酬斡平亂渤海,又以討叛力戰而死,忠可尚矣。得裏底縱女直而不計,寢變告而不聞。其蔽主聰明,為國階亂,莫斯之甚也。章奴、術者乘時多艱,潛謀廢立,將求寵幸,以犯大逆,其得免於天下之戮哉!」



\end{pinyinscope}