\article{卷五十三志第二十二 禮志六 嘉儀下}

\begin{pinyinscope}

 皇太后生辰朝賀儀:至日,臣僚入朝,國使至幕,班齊,如常儀。皇太后升殿坐,皇帝東面側坐。契丹舍人殿上通名,契丹、漢人臣僚,宋使副綴翰林學士班,東西兩洞門入,合班稱賀,班首上殿祝壽,分班引出,皆如正旦之儀。教坊起居,七拜,契丹、漢人臣僚入,進酒,皆如
 五旦之儀,唯宣答稱「聖旨」。皇帝降御座,進奉皇太后生辰禮物。過畢,皇帝殿上再拜,殿下臣僚皆再拜。皇帝升御座。引臣僚分班出,引中書令、北大王西階上殿,奏契丹臣僚進奉。次漢人臣僚並諸道進奉。控鶴官置擔床,起居,四拜畢;引進使鞠躬,通文武百僚某官某以下、高麗、夏國、諸道進奉。宣微使殿上贊進奉各付所司。控鶴官聲喏。擔床過畢,契丹、漢人臣僚以次謝,五拜。



 贊各祗候,引出。教坊、諸道進奉使謝如之。契丹臣僚謝宣宴,引上殿就位立,漢人臣僚並宋使副東洞門入,面西謝宣宴,如正旦儀。贊各上殿祗候。
 臣僚、使副上殿就位立,亦如之。監盞、教坊上殿,從人入東廊立,皆如之。御床入,皇帝初進酒,臣僚就位陪拜。皇太后飲酒,殿上應坐、侍立臣僚皆拜,稱「萬歲」。贊各祗候,立。皇太后卒飲,手賜皇帝酒。皇帝跪,卒飲,退就褥位,再拜,臣僚皆陪拜。若皇帝親賜使相、臣僚、宋使副酒,皆立飲。皇帝升坐,贊應坐臣僚並使副皆拜,稱「萬歲」。贊各就坐。行方醖朵殿臣僚酒,如正旦儀。一進酒,兩廊從人拜,稱「萬歲」,各就坐。親王進酒,如正旦儀。若皇太后手賜親王酒,跪飲訖,退露臺上,五拜。贊祗候。殿上三進酒,行餅茶訖,教坊跪,致語,揖臣僚、使
 副、廊下從人皆立。口號絕,贊拜亦如之。行茶、行肴膳,皆如之。大饌入,行粥碗。殿上七進酒,使相、臣僚樂曲終,揖廊下從人起,拜,稱「萬歲」。「各好去」,承受官引兩門出。曲破,揖臣僚、使副起,鞠躬。贊拜,皆拜,稱「萬歲」。贊各祗候,引臣僚、使副下殿。契丹臣僚謝宴畢,出。漢人臣僚、使副舞蹈,五拜畢,贊「各好去」。出洞門畢,報閣門無事,皇太后、皇帝起。



 應聖節,宋遣使來賀生辰、正旦,始制此儀,故詳見《賓儀》。



 凡五拜:拜,興。再拜,興。跪,搢笏,三舞蹈,三叩頭,出笏,
 就拜,興。拜,興。再拜,興。其就拜,亦曰俯伏興。



 《賓儀》,臣僚皆日坐,於此儀曰高醖,與方醖別。



 皇帝生辰朝賀儀:臣僚、國使班齊,皇帝升殿坐。臣僚、使副入,合班稱賀。合班出。皆如皇太后生辰儀。中書令、北大王奏諸道進奉表目。教坊起居,七拜。臣僚東西門入,合班再拜。贊進酒,班首上殿進酒。宣徽使宜答,群臣謝宣諭,分班。奏樂,皇帝卒飲,合班。班首下殿,分班出。皆如正旦之儀。進奉皆如皇太后生辰儀。皇帝詣皇太后殿,近上皇族、外戚、大臣並從,奉迎太皇即皇帝殿坐。皇太后御小輦,皇帝輦側步從,臣僚分行
 序引,宣徽使、諸司、閣門攢隊前引。教坊動樂,控鶴起居,四拜。引駕臣僚並於山樓南方立候。皇太后入閣,揖使副並臣僚入幕次。皇太后升殿坐,皇帝東方側坐。



 引契丹、漢人臣僚、使副兩洞門入,合班,起居,舞蹈,五拜。



 贊各祗候,面殿立。皇帝降御坐,殿上立,進皇太后生辰物。



 過畢,皇帝殿上再拜,殿上下臣僚皆拜。皇帝升御座,引臣僚分班出。契丹臣僚入,謝宣宴。漢人臣僚、使副入,通名謝宣宴,上殿就位。不應坐臣僚出,從人入,皆如儀。御床入,皇帝初進皇太后酒,皇太后賜皇帝酒,皆如皇太后生辰儀。贊各就坐,行酒。宣飲
 盡,就位謝如儀。殿上一進酒畢,從人入就位如儀。親王進酒,行餅茶,教坊致語如儀。行茶、行肴膳如儀。七進酒,使相樂曲終,從人起。曲破,臣僚、使副起。餘皆如正旦之儀。



 皇后生辰儀:臣僚昧爽朝。皇帝、皇后大帳前拜日,契丹、漢人臣僚陪拜。皇帝升殿坐,皇后再拜,臣僚殿下合班陪拜。



 皇帝賜皇后生辰禮物,皇后殿上謝,再拜,臣僚皆拜。契丹舍人通名,契丹、漢人臣僚以次入賀。盞入,舍人贊,舞蹈,五拜,起居不表「聖躬萬福」。贊再拜。班首上殿拜跪,自通全銜祝壽訖,引下殿,復位,鞠躬。贊
 舞蹈,五拜。贊各祗候。



 引宰臣一員上殿,奏百僚諸道進表目。教坊起居,七拜,不賀。



 控鶴官起居,四拜。諸道押衙附奏起居,賜宴,共八拜。契丹、漢人合班,進壽酒,舞蹈,五拜。引大臣一員上殿,欄外褥位搢笏,執臺盞進酒,皇帝、皇后受盞。退,復褥位。授臺出笏,欄內拜跪,自通全銜祝壽「臣等謹進千萬歲壽酒」訖,引下殿,復位,舞蹈,五拜,鞠躬。宣徽使奏宣答如儀,引上殿,搢笏執臺。皇帝、皇后飲。殿下臣僚分班,教坊奏樂,皆拜,稱「萬歲」。卒飲,皇帝、皇後授盞。引下殿,舞蹈,五拜。贊各祗候,引出。臣僚進奉如儀,宣宴如儀。教坊、監盞、臣僚上
 殿祗候如儀。皇后進皇帝酒,殿上贊拜,侍臣僚皆拜。皇帝受盞,皆拜。皇后坐,契丹舍人、漢人閣使殿上贊拜,皆拜,稱「萬歲」。贊各就坐。大臣進皇帝、皇後酒,行酒如儀。酒三行,行慄,行膳。又進皇帝、皇後酒。酒再行,大饌入,行粥。



 教坊致語,臣僚皆起立。口號絕,贊拜,稱「萬歲」,引下殿謝宴,引出,皆如常儀。



 進士接見儀:其日,舉人從時相至御帳側,通名榜子與時相榜子同奏訖,時相朝見如常儀。畢,揖進士第一名以下丹墀內面殿鞠躬,通名,四拜。贊各祗候,皆退。若有進文字者,不退,奉卷平立。閣門奏受,跪左膝授
 訖,直起退。禮畢。



 進士賜等甲敕儀:臣僚起居畢,讀卷官奏訖,於左方依等甲唱姓名序立,閣門交收敕牒。閣使奏引至丹墀,依等甲序立。



 閣使稱「有敕」,再拜,鞠躬。舍人宣敕「各依等甲賜卿敕牒一道,想宜知悉」,揖拜。各跪左膝,受敕訖,鞠躬,皆再拜。



 各祗候,分引左右相向侍立。候奏事畢,引兩階上殿,就位,齊聲喏,賜坐。酒三行,起,聲喏如初。退揖出。禮畢。牌印郎君行酒,閣使勸飲。



 進士賜章服儀:皇帝御殿,臣僚公服引進士入,東方面西,再拜,揖就丹墀位,面殿鞠躬。閣使稱「有敕」,再拜,鞠
 躬。



 舍人宣敕「各依等甲賜卿敕牒一道,兼賜章服,想宜知悉」,揖再拜。跪受敕訖,再拜。退,引至章服所,更衣訖,揖復丹墀位,鞠躬。贊謝恩,舞蹈,五拜。各祗候,殿東亭內序立。



 聲喏,坐。賜宴,簪花。宣閣使一員、閣門三人或二人勸飲終日。禮畢。



 宰相中謝儀:皇帝常服升殿坐,諸班起居如常儀。應坐臣僚上殿,其餘臣僚殿下東西侍立,皆如宋使初見之儀。引中謝官左入,至丹墀面西立。舍人當殿鞠躬,通新受具官姓名祗候中謝。宣徽殿上索通班舍人就贊禮位,贊某官至。宣微贊通班舍人二人對立,揖
 中謝官鞠躬。贊就拜位,舍人二人引面殿鞠躬。贊拜,中謝官舞蹈,五拜,不出班,奏「聖躬萬福」。贊再拜。揖出班跪,敘官,致詞訖,俯伏興,復位。贊拜,舞蹈,五拜。又出班,中謝致詞如初儀,共十有七拜。贊祗候,引右階上殿,就位。揖應坐臣僚聲喏坐。供奉官行酒,傳宣飲盡。



 臣僚搢笏,執盞起,位後立飲;置盞,出笏。贊拜,臣僚皆再拜。贊各坐,搢笏,執盞,授供奉官盞。酒三行,揖應坐僚聲喏立。引中謝官右階下殿,至丹墀,面殿鞠躬。贊拜,舞蹈,五拜,引右出。臣僚皆出。本相、樞密使同,餘官不升殿,賜酒,不帶節度使不通班,止通名,七拜。眾
 謝,班首一人出班中謝。拜表儀:其日,先於東上閣門陳設氈位,分引南北臣僚、諸國使副於氈位合班。通事舍人二人舁表案,置班首前,揖鞠躬,再拜,平身。中書舍人立案側,班首跪,搢笏,興,捧表,跪左膝,以表授中書舍人。出笏,就拜,興,再拜。中書舍人復置表案上。通事舍人舁表案於東上閣門入,卷班,分引出。



 禮畢。元日,皇帝不御坐行此儀,餘應上表有故皆仿此。



 賀生皇子儀:其日,奉先帝御容,設正殿,皇帝御八角殿升坐。聲警畢,北南宣微使殿階上左右立,北南臣僚
 金冠盛服,合班入。班首二人捧表立,讀表官先於左階上側立。二宣徽使東西階下殿受表,捧表者跪左膝授訖,就拜,興,再拜。各祗候。二宣徽使俱左階上授讀表官,讀訖,揖臣僚鞠躬。引北面班首左階上殿,欄內稱賀訖,引左階下殿,復位,舞蹈,五拜。



 禮畢。賀祥瑞儀:聲警,北南臣僚金冠盛服,合班立。班首二人各奉表賀,北南宣徽使左階下殿受表,上殿授讀表大臣。讀訖,揖殿下臣僚鞠躬,五拜畢,鞠躬。引班首二人左階上殿,欄內拜跪稱賀,致詞訖,引左階下殿,復位,五拜畢,鞠躬。宣答、聽制訖,再拜,鞠躬。謝宣諭,五拜
 畢,各祗候,分班侍立。



 禮畢,兩府奏事如常。



 乾統六年,木葉山瑞雲見,始行此儀。天慶元年,天雨穀,謝宣諭後,趙王進酒,教坊動樂,臣僚酒一行。禮畢,奏事。



 賀平難儀:皇帝、皇后升殿坐,北商臣僚並命婦合班,五拜。揖班首二人出班,俯跪,搢笏,執表,舁案近前。閣使受表,置案上,皆再拜。通事舍人二人舁案,左階上殿,置露臺上。讀表官受,入讀表。對御讀訖,臣僚殿下五拜,鞠躬。引班首二人左右階上殿,欄內並立。先引北面班首少前,跪致詞訖,退復褥位。次引南面班首亦
 如之。畢,分引左右階下殿,復位,五拜,鞠躬。宣徽稱「有敕」,再拜,宣答「內難已平,與公等內外同慶」。謝宣諭,五拜。卷班。臣僚從皇帝,命婦從皇后,詣皇太后殿,見先帝御容,陪位,皆再拜。皇太后正坐,稱賀,共十拜,並引上殿,賜宴如儀。



 平難之儀,道宗清寧九年,太叔重元謀逆,仁懿太后親率衛士與逆黨戰。事平,因制此儀。



 正旦朝賀儀:臣僚並諸國使昧爽入朝,奏「班齊」。皇帝升殿坐,契丹舍人殿上通訖,引契丹臣僚東洞門入,引漢人臣僚並諸國使西洞門入。合班,舞蹈,五拜,鞠躬,
 平身。引親王東階上殿,欄內褥位俯伏跪,自通全銜臣某等禍壽訖,伏興,退,引東階下殿,復位,舞蹈,五拜畢,鞠躬。宣徽使殿上鞠躬,奏「臣宣答」,稱「有敕」,班首以下聽制訖,再拜,鞠躬。宣徽傳宣云:「履新之慶,與公等同之。」舍人贊謝宣諭,拜,舞蹈,五拜。贊各祗候,分班引出,引班首西階上殿,奏表目訖,教坊起居,賀,十二拜,畢,贊各祗候。引契丹、漢1人臣僚並諸國使東西洞門入,合班,再拜。贊進酒,引親王東階上殿,就欄內褥位,搢笏,執臺盞,進酒訖,退,復褥位。



 置臺,出笏。少前俯跪,自通全銜臣某等謹進千萬歲壽酒。俯伏興,退,復褥
 位,與殿下臣僚皆再拜,鞠躬。侯宣徽使殿上鞠躬,奏「臣宣答」,稱「有敕」,親王以下再拜如初儀。傳宣云:「飲公等壽酒,與公等內外同慶。」舍人贊謝宣諭如初。



 贊各祗候,親王搢笏,執臺,殿下臣僚分班。皇帝飲酒,教坊奏樂,殿上下臣僚皆拜,稱「萬歲」。贊各祗候。樂止,教坊再拜。皇帝卒飲,親王進受盞,復褥位,置臺盞,出笏。揖臣僚合班,引親王東階下殿,復位,鞠躬,再拜。贊各祗候,分班引出。皇帝起,詣皇太后殿,臣僚並諸國使皆從。皇太后升殿,皇帝東方側坐。引契丹、漢人臣僚並諸國使兩洞門入,合班稱賀,進酒,皆如皇帝之儀。畢,引
 出。教坊入,起居、進酒亦如之。皇太后宣答稱「聖旨」。契丹班謝宣宴,上殿就位立。漢人臣僚並諸國使東洞門入,丹墀東方,面西鞠躬。舍人鞠躬,通文武百僚宰臣某已下謝宣宴,再拜;出班致詞訖,退復位,舞蹈,五拜。贊各上殿祗候,引宰臣以下並諸國使副,方裀朵殿臣僚,西階上殿就位立。不應坐臣僚並於西洞門出。



 二人監琖,教坊再拜。贊各上階,下殿謝宴,如皇太后生辰儀。



 冬至朝賀儀:臣僚班齊,如正旦儀。皇帝、皇后拜日,臣僚陪位再拜。皇帝、皇后升殿坐,契丹舍人通,臣僚入,合
 班,親王祝壽,宣答,皆如正旦之儀。謝訖,舞蹈,五拜,鞠躬。



 出班奏「聖躬萬福」;復位,再拜,鞠躬。班首出班,俯伏跪,祝壽訖,伏興,舞蹈,五拜,鞠躬。贊各祗候。分班,不出,合班。御床入,再拜,鞠躬。贊進酒。臣僚平身。引親王左階上殿,就欄內褥位,搢笏,執臺琖,進酒。皇帝、皇后受琖訖,退就褥位,置臺,出笏,俯伏跪。少前,自通全銜臣某等謹進2千萬歲壽酒。俯伏興,退,復褥位,再拜,鞠躬。殿下臣僚皆再拜,鞠躬。宣答如五旦儀。親王搢笏,執臺,分班。皇帝、皇后飲酒,奏樂;殿上下臣僚皆拜,稱「萬歲壽」,樂止。教坊再拜,臣僚合班。親王進受琖,至褥位,
 置臺琖,出笏,引左階下殿。御床出。親王復丹墀位,再拜,鞠躬。贊祗候。分班引出。班首右階上殿奏表目進奉。諸道進奉,教坊進奉過訖,贊進奉收。班首舞蹈,五拜,鞠躬。贊備祗候。班首出,臣僚復入,合班謝,舞蹈,五拜,鞠躬。贊各祗候。分班引出。聲警,皇帝、皇后起,赴北殿。皇太后於御容殿,與皇帝、皇后率臣僚再拜。皇太后上香,畢再拜。贊各祗候。可矮墩以上上殿。皇太后三進御容酒,陪位皆拜。皇太后升殿坐。皇帝就露臺上褥位,親王押北南臣僚班丹墀內立。皇帝再拜,臣僚皆拜,鞠躬。皇帝欄內跪,祝皇太后壽訖,復位,再拜。
 凡拜,皆稱「萬歲」。贊各祗候。臣僚不出,皇帝、皇后側座,親王進酒,臣僚陪拜,皇太后宜答,皆如正旦之儀。臣僚分班,不出,班首右階上殿奏表目,合班謝宣宴,上殿就位如儀。御床入。皇帝進皇太后酒如初,各就座行酒,宣飲盡,如皇太后生辰之儀。



 皇后進酒,如皇帝之儀。三進酒,行茶,教坊致語,行殽膳,大饌,七進酒。曲破,臣僚起,御床出,謝宴,皆如皇太后生辰儀。立春儀:皇帝出就內殿,拜先帝御容,北南臣僚丹墀內合班,再拜。可矮墩以上入殿,賜坐。帝進御容灑,陪位並侍立皆再拜。一進酒,臣僚下殿,左右相向立。皇帝
 戴幡勝,等第賜幡勝。臣僚簪畢,皇帝於土牛前上香,三奠酒,不拜。教坊動樂,侍儀使跪進彩杖。皇帝鞭土牛,可矮墩以上北南臣僚丹墀內合班,跪左膝,受彩杖,直起,再拜。贊各祗候。司辰報春至,鞭土牛三匝。矮墩鞭止,引節度使以上上殿,撒穀豆,擊土牛。撒穀豆,許眾奪之。臣僚依位坐,酒兩行,春盤入。



 酒三行畢,行茶。皆起。禮畢。



 重午儀,至日,臣僚昧爽赴御帳,皇帝系長壽彩縷升車坐,引北南臣僚合班,如丹墀之儀。所司各賜壽縷,揖臣僚跪受,再拜。引退,從駕至膳所,酒三行。若賜宴,臨
 時聽敕。



 重九儀:北南臣僚旦赴御帳。從駕至圍場,賜茶。皇帝就坐,引臣僚御前班立,所司各賜菊花酒,跪受,再拜。酒三行,揖起。藏鬮儀:至日,北南臣僚常服入朝,皇帝御天祥殿,臣僚依位賜坐。契丹南面,漢人北面,分朋行鬮。或五或七籌,賜膳。入食畢,皆起。頃之,復生行鬮如初。晚賜茶,三籌或五籌,罷教坊承應。若帝得鬮,臣僚進酒訖,以次賜酒。



 大康十年十二月二十二日,始行是儀。是日不御
 朝。



 歲時雜儀:正旦,國俗以糯飯和白羊髓為餅,丸之若拳,每帳賜四十九枚。戊夜,各於帳內窗中擲丸於外。數偶,動樂,飲宴。數奇,令巫十有二人鳴鈴,執箭,繞帳歌呼,帳內爆鹽壚中,燒地拍鼠,謂之驚鬼,居七日乃出。國語謂正旦為「乃捏咿兒」。「乃」,正也;「捏咿兒」,旦也。



 立春,婦人進春書,刻青繒為幟,像龍御之;或為蟾蜍,書幟曰「宜春」。



 人日,凡正月之日,一雞、二狗、三豕、四羊、五馬、六牛,七日為人。其占,晴為祥,陰為災。俗煎餅食於庭中,謂之「薰
 天」。



 二月一日為中和節,國舅族蕭氏設宴,以延國族耶律氏,歲以為常。國語是日為「𢘉里尀」。「𢘉里」,請也;「尀」,時也。𢘉,讀若狎;尀,讀若頗。



 二月八日為悉達太子生辰,京府及諸州雕木為像,儀仗百戲導從,循城為樂。悉達太子者,西域凈梵王子,姓瞿曇氏,名釋迦牟尼。以其覺性,稱之曰「佛」。



 三月三日為上巳,國俗,刻木為兔,分朋走馬射之;先中者勝,負朋下馬列跪進酒,勝朋馬上飲之。國語謂是日為「陶里樺」。「陶里」,兔也;「樺」,射也。



 五月重五日,午時,彩艾葉和綿著衣,七事以奉天子,北南臣僚各賜三事,君臣宴樂,渤海膳夫進艾糕。以五彩絲為索纏臂,謂之「合歡結」。又以彩絲宛轉為人形簪之,謂之「長命縷」。國語謂是日為「討賽咿兒。」。「討」,五;「賽咿兒」,月也。



 夏至之日,俗謂之「朝節」。婦人進彩扇,以粉脂相贈遺。



 六月十有八日,國俗,耶得氏設宴,以延國舅族蕭氏,亦謂之「𢘉里尀」。七月十三日,夜,天子於宮西三十里卓帳宿焉。前期,備酒饌。翼日,諸軍部落從者皆動蕃樂,飲宴至暮,乃歸
 行宮,謂之「迎節」。十五日中元,動漢樂,大宴。十六日昧爽,復往西方,隨行諸軍部落大噪三,謂之「送節」。國語謂之「賽咿兒奢」。「奢」,好也。



 八月八日,國俗,屠白犬,於寢帳前七步瘞之,露其喙。



 後七日中秋,移寢帳於其上。國語謂之「捏褐耐」。「捏褐」,犬也;「耐」,首也。



 九月重九日,天子率群臣部族射虎,少者為負,罰重九宴。



 射畢,擇高地卓帳,賜蕃、漢臣僚飲菊花酒。兔肝為謩,鹿舌為醬,又研茱萸酒,灑門戶以偲禳。國語謂是日為「必里遲離」,九月九日也。



 歲十月,五京進紙造小衣甲、槍刀、器械萬副。十五日,天子與群臣望祭木葉山,用國字書狀,並焚之。國語謂之「戴辣」。「戴」燒也;「辣」,甲也。冬至日,國俗,屠白羊、白馬、白雁,各取血和酒,天子望拜黑山。黑山在境北,俗謂國人魂魄,其神司之,猶申國之岱宗云。每歲是日,五京進紙造人馬萬餘事,祭山而焚之。俗甚嚴畏,非祭不敢近山。



 臘辰日,天子率北南臣僚並戎服,戊夜坐朝,作樂飲酒,等第賜甲仗、羊馬。國語謂是日為「炒伍珼伓」。「炒伍珼」,戰也。
 再生儀:凡十有二歲,皇帝本命前一年季冬之月,擇吉日。



 前期,禁門北除地置再生室、母後室、先帝神主輿。在再生室東南,倒植三岐木。其日,以童子及產醫嫗置室中。一婦人執酒,一叟持矢慅,立於室外。有司請神主降輿,致奠。奠訖,皇帝出寢殿,詣再生室。群臣奉迎,再拜。皇帝入室,釋服、跣。以童子從,三過岐木之下。每過,產醫嫗致詞,拂拭帝躬。



 童子過岐木七,皇帝臥木側,叟擊慅曰:「生男矣。」太巫誷皇帝首,興,群臣稱賀,再拜。產醫嫗受酒於執酒婦以進,太巫奉襁褓、彩結等物贊祝之,預選七叟,各立御名系於彩,皆跪進。皇
 帝選嘉名受之,賜物。再拜,退。群臣皆進襁褓、彩結等物。皇帝拜先帝諸御容,遂宴群臣。



 善哉,阻午可汗之垂訓後嗣也。孺子無不慕其親者,嗜欲深而愛淺,妻子具而孝衰。人人皆然,而況天子乎。再生之儀,歲一周星,使天子一行是禮,以起其孝心。夫體之也真,則其思之也切,孺子之慕,將有油然發於中心者,感發之妙,非言語文字之所能及。善哉,阻午可汗之垂訓後嗣也。始之以三過岐木,母氏劬勞能無念乎。終之以拜先帝御容,敬承宗廟宜何如哉。《詩》曰:「無念爾祖,聿修厥德。」



\end{pinyinscope}