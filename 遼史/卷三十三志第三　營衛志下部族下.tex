\article{卷三十三志第三 營衛志下部族下}

\begin{pinyinscope}

 遼
 起松漠,經營撫納,竟有唐、晉帝王之器,典章文物施及潢海之區,作史者尚可以故俗語耶?舊史有《部族志》,歷代之所無也。古者,巡守於方岳,五服之君各述其職,遼之部族實似之。故以部族置宮衛、行營之後云。



 遼內四部族:
 遙輦九帳族。



 橫帳三父房族。



 國舅帳拔里、乙室已族。



 國舅別部。



 太祖二十部,二國舅升帳分,止十八部。



 五院部。其先曰益古,凡六營。阻午可汗時,與弟撒裏本領之,曰迭刺部。傳至太祖,以夷離堇即位。天贊元年,以強大難制,析五石烈為五院,六爪為六院,各置夷離堇。會同元年,更夷離堇為大王。部隸北府,以鎮南境。大王及都監春夏居五院部之側,秋冬居羊門甸。
 石烈四:大蔑孤石烈。



 小蔑孤石烈。甌昆石烈。太宗會同二年,以烏古之地水草豐美,命居之。



 三年,益以海勒水之地為農田。



 乙習本石烈。會同二年,命以烏古之地。



 六院部。隸北府,以鎮南境。其大王及都監春夏居泰德泉之北,秋冬居獨盧金。石烈四:轄懶石烈。



 阿速石烈。



 斡納撥石烈。



 斡納阿刺石烈。會同二年,命居烏古。三年,益以海勒水地。



 乙室部。其先曰撤裏本,阻午可汗之世,與其兄益古分營而領之,曰乙室部。會同二年,更夷離堇為大王。隸南府,其大王及卻督監鎮駐西南之境,司徒居鴛鴦泊,閘撤狘居車抽山。



 石烈二:阿裡答石烈。



 欲主石烈。



 品部。其先曰拿女,阻午可汗以其營為部。太祖更諸
 部夷離堇為令穩。統和中,又改節度使。隸北府,屬西北路招討司,司徒居太子墳。凡戍軍隸節度使,留後戶隸司徒。石烈二:北哲裡只石烈。



 南轄懶石烈。



 楮特部。其先曰窪,阻午可以其營為部。隸南府,節度使屬西北路招討司,司徒居柏坡山及鏵山之側。石烈二:北石烈。



 南石烈。



 烏
 隗部。其先曰撤裡卜,與其兄涅勒同營,阻午可汗析為二:撤裡卜為烏隗部,涅勒為涅刺部。俱隸北府,烏隗部節度使屬東北路招討司,司徒居徐母山、郝星河之側。石烈二:北石烈。



 南石烈。



 涅刺部。其先曰涅勒,阻午可汗分其營為部。節度使屬西南路招討司,居黑山北,司徒居郝里河側。石烈二:北塌里石烈。



 南察里石烈。



 突呂不部。其先曰塔古里,領三營。阻午可汗命分其一與弟航斡為突舉部;塔古裡得其二,更為突呂不部。隸北府,節度使屬西北路招討司,司徒居長春州西。石烈二:北托不石烈。



 南須石烈。



 突舉部。其行曰航斡,阻午可汗分營置部。隸南府,戍於隗烏古部,司徒居冗泉側。石烈二:北石烈。



 南石烈。



 奚王府六部五帳分。其先曰時瑟,事東遙里十帳部主哲里。



 後逐哲里,自立為奚王。卒,弟吐勒斯立。遙輦鮮質可汗討之,俘其拒敵者七百戶,摭其降者。以時瑟鄰睦之故,止俘部曲之半,餘悉留焉。奚勢由是衰矣。初為五部:曰遙里,曰伯德,曰奧裏,曰梅只,曰楚里。太祖盡降之,號五部奚。天贊二年,有東扒時廝胡損者,恃險堅壁於箭笴山以拒命,椰榆日:「大軍何能為,我當飲墮瑰門下矣!」太祖滅之,以奚府給役戶,並括諸部隱丁,收合流散,置墜瑰部,因墜瑰門之語為名,
 遂號六部奚。命勃魯恩主之,仍號奚王。太宗即位,置宰相、常兗各二員。聖宗合奧里、梅只、墮瑰三部為一;特置二克部以足六部之數。奚王和朔奴討兀惹,敗績,籍六部隸北府。突呂不室韋部。本名大、小二黃室韋戶。太祖為達馬狘沙裏,以計降之,乃置為二部。隸北府,節度使屬東北路統軍司,戍泰州東北。



 涅刺拿古部。與突呂不室韋部同。節度使戍泰州東。



 迭刺迭達部。本鮮質可汗所俘奚七百戶,太祖即位,以為十四石烈,置為部。隸南府,節度使屬西南路招
 討司,戍黑山北,部民居慶州南。



 乙室奧隗部。神冊六年,太祖以所俘奚戶置。隸南府,節度使屆東北路兵馬司。



 楮特奧槐部。太祖以奚戶置。隸南府,節度使屬東京都部署司。品達魯虢部。太祖以所俘達魯虢部置。隸南府,節度使屬西南路招討司,戍黑山北。



 烏古涅刺部。亦曰涅離部。太祖取於骨里廣大千,神冊六年,析為烏古涅刺及圖魯二部。俱隸北府,節度使屬西南路招討司。
 圖魯部。節度使屬東北路統軍司。



 已上太祖以遙輦氏舊部族分置者凡十部,增置者八。



 聖宗三十四部:撤里葛部。奚有三營:曰撤里葛,日竊爪,曰耨碗爪。太祖伐奚,乞降,願為著帳子弟,籍於官分,皆設夷離堇。聖宗各置為部,改設節度使,皆隸南府,以備畋獵之役。居澤州東。



 窈爪部。與撤里葛部同。居潭州南。



 耨碗爪部。節度使屈東京都部署司。



 訛僕括部。與撤里葛三部同。居望雲縣東。



 特裡特勉部。初於八部各析二十戶以戍奚。偵候落馬河及速魯河側,置二十詳穩。聖宗以戶口蕃息,置為部,設節度使。隸南府,戍倒塌嶺,居橐駝岡。



 稍瓦部。初,取諸宮及橫帳大族奴隸置稍瓦石烈,「稍瓦」,鷹坊也,居遼水東,掌羅捕飛鳥。聖宗以戶口蕃息置部。節度使屈東京都部署司。



 曷術部。初,取諸官及橫帳大族奴隸置曷術石烈,「曷術」,鐵也,以冶於海濱柳濕河、三黜古斯、手山。聖宗以戶口蕃息置部。屬東京都部署司。



 遙里部。居潭、利二州間。石烈三:撤里必石烈。



 北石烈。



 帖魯石烈。伯德部。松山、平州之間,太師、太保居中京西。石烈六:啜勒石烈。



 速古石烈。



 腆你石烈。



 迭里石烈。



 旭特石烈。



 悅里石烈。



 楚里部。居潭州北。



 奧里部。統和十二年,以與梅只、墜瑰三部民籍數寡,合為一部。並上三部,本屬奚王府,聖宗分置。



 南克部。



 北克部。統和十二年,以奚府二克分置二部。



 隗衍突厥部。聖宗析四闢沙、四頗憊戶置,以鎮東北女直之境。開泰九年,節度使奏請置石烈。隸北府,屬黃龍府都部署司。奧衍突厥部。與隗衍突厥同。
 涅刺越兀部。以涅刺室韋戶置。隸北府,節度使屬西南面招討司,戍黑山北。



 奧衍女直部。聖宗以女直戶置。隸北府,節度使屬西北路招討司,戍鎮州境。自此至河西部,皆俘獲諸國之民。初隸諸官,戶口蕃息置部。訖於五國,皆有節度使。



 乙典女直部。聖宗以女直戶置。隸南府,居高州北。



 斡突碗烏古部。聖宗以烏古戶置。隸南府,節度使屬西南面招討司,戍黑山北。



 迭魯敵烈部。聖宗以敵烈戶置。隸北府,節度使屬烏古敵烈統軍司。



 室韋部。聖宗以室韋戶置。隸北府,節度使屬西北路招討司。



 術哲達魯虢部。聖宗以達魯虢戶置。隸北府,節度使屬東北路統軍司,戍境內,居境外。



 梅古悉部。聖宗以唐古戶置。隸北府,節度使屬西南面招討司。頡的部。聖宗以唐古戶置。隸北府,節度使屬西南面招討司。



 北敵烈部。聖宗以敵烈戶置。戍隗烏古部。



 匿訖唐古部。聖宗置。隸北府,節度使屬西南面招討司。



 北唐古部。聖宗以唐古戶置。隸北府,節度使屬黃龍府都部署司,戍府南。



 南唐古部。聖宗置。隸北府。



 鶴刺唐古部。與南唐古同。節度使屬西南面招討司。



 河西部。聖宗置。隸北府,節度使屬東北路統軍司。



 薛特部。開泰四年,以回鶻戶置。隸北府,居慈仁縣北。



 伯斯鼻骨德部。本鼻骨德戶。初隸諸宮,聖宗以戶口蕃息置部。隸北府,節度使屈東北路統軍司,戍境內,居境外。



 達馬鼻骨德部。聖宗以鼻骨德戶置。隸南府,節度使
 屬東北路統軍司。



 五國部。剖阿里國、盆奴里國、奧裏米國、越里篤國、越里吉國,聖宗時來附,命居本土,以鎮東北境,屬黃龍府都部署司。重熙六年,以越國吉國人尚海等訴酋帥渾敞貪污,罷五國酋帥,設節度使以領之。



 已上聖宗以舊部族置者十六人,增置十八。



 遼國外十部:烏古部。



 敵烈八部。



 隗古部。



 回跋部。



 嵒母部。



 吾禿婉部。



 迭刺葛部。



 回鶻部。



 長白山部。



 蒲盧毛朵部。



 右十部不能成國,附庸於遼,時叛時服,備有職貢,猶唐人之有羈縻州也。



\end{pinyinscope}