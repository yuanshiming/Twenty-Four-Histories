\article{卷八十一 列傳第十九}

\begin{pinyinscope}

 耶律庶成弟庶箴箴子蒲魯楊皙耶律韓留楊佶耶律和尚耶律庶成,字喜隱,小字陳六,季父房之後。父吳九,檢校太師。



 庶成幼好學,書過目不忘。善遼、漢文字,於詩尤工。重熙初,補牌印郎君,累遷樞密直學士。與蕭韓家奴各進《四時逸樂賦》,帝嗟賞。初,契丹醫人鮮知切脈審藥,上命庶成譯方脈書行之,自是人皆通習,雖諸部族亦知醫事。時人禁中,參決疑議。偕林牙蕭
 韓家奴等撰《實錄》及《禮書》。與樞密副使蕭德修定法令,上詔庶成曰:「方今法令輕重不倫。法令者,為政所先,人命所系,不可不慎。卿其審度輕重,從宜修定。」庶成參酌古今,刊正訛謬,成書以進。帝覽而善之。



 庶成方進用,為妻胡篤所誣,以罪奪官,絀為「庶耶律」。



 使吐蕃凡十二年,清寧間始歸。帝知其誣,詔復本族,仍遷所奪官,卒。庶成嘗為林牙,夢善卜者胡呂古卜曰:「官止林牙,因妻得罪。」及置於理,法當離婚。胡篤適有娠,至期不產而死。



 剖視之,其子以手抱心,識者謂誣夫之報。有詩文行於世。弟庶箴。庶箴,字陳甫,善屬文。重熙中,為本族將軍。咸雍元年,同知東京留守事,俄徙
 烏衍突厥部節度使。九年,知薊州事。



 明年,遷都林牙。上表乞廣本國姓氏曰:「我朝創業以來,法制修明;惟姓止分為二,耶律與蕭而已。始太祖制契丹大字,取諸部鄉里之名,續作一篇,著於卷末。臣請推廣之,使諸部各立姓氏,庶男女婚媾有合典禮。」帝以舊制不可遽厘,不聽。



 大康二年,出耶律乙辛為中京留守,庶箴與耶律孟簡表賀。



 頃之,乙辛復為樞密使,專權恣虐。庶箴私見乙辛泣曰:「前抗表,非庶箴之願也。」乙辛信其言,乃得自安。聞者鄙之。



 八年,致仕,卒。子蒲魯。



 蒲魯,字乃展。幼聰悟好學,甫七歲,能誦契丹大字。習漢文,未十年,博通經籍。



 重熙中,舉進士第。主文以國制無契丹試進士之條,聞於上,以庶箴擅令子就科目,鞭之二百。尋命蒲魯為牌印郎君。



 應
 詔賦詩,立成以進。帝嘉賞,顧左右曰:「文才如此,必不能武事。」蒲魯奏曰:「臣自蒙義方,兼習騎射,在流輩中亦可周旋。」帝未之信。會從獵,三矢中三兔,帝奇之,轉通進。



 是時,父庶箴嘗寄《戒諭詩》,蒲魯答以賦,眾稱其典雅。



 寵遇漸隆。清寧初卒。



 楊皙,字昌時,安次大。幼通《五經》大義。聖宗聞其穎悟,詔試詩,授秘書省校書郎。太平十一年,擢進士乙科,為著作佐郎。重熙十二年,累遷樞密都承旨,權度支使。登對稱旨,進樞密副使。歷長寧軍節度使,山西路轉運使,知興中府。清寧初,入知南院樞密使,與姚景行同總朝政。請行柴冊禮。封趙國公。以足疾,復知興中府。咸雍初,徙封齊,召賜同德功臣、尚書左僕射,兼中書令,拜樞密使,改
 封晉,給宰相、樞密使兩廳兼從,封趙王。



 屢請歸政,益賜保節功臣,致仕。大康五年,例改遼西郡王,薨。



 耶律韓留,字速寧,仲父隋國王之後。有明識,篤行義,舉止嚴重,工為詩。



 統和間,召攝御院通進。開泰三年,稍遷烏古敵烈部都監,俄知詳穩事。敵烈部叛,將宮分軍,從樞密使耶律世良討平之,加千牛衛大將軍。



 重熙元年,累遷至同知上京留守,改奚六部禿裏太尉。性不茍合,為樞密使蕭解里所忌。
 上欲召用韓留,解裏言目病不能視,議遂寢。四年,召為北面林牙。帝曰:「朕早欲用卿,聞有疾,故待之至今。」韓留對曰:「臣昔有目疾,才數月耳;然亦不至於昏。第臣駑拙,不能事權貴,是以不獲早睹天顏。



 非陛下聖察,則愚臣豈有今日耶!」詔進《述懷詩》,上嘉嘆。



 方將大用,卒。



 楊佶,字正叔,南京人。幼穎悟異常,讀書自能成句,識者奇之。弱冠,聲名籍甚。



 統和二十四年,
 舉進士第一,歷校書郎、大理正。開泰六年,轉儀曹郎,典掌書命,加諫議大夫。出知易州,治尚清簡,徵發期會必信。入為大理少卿。累遷翰林學士,文章號得體。八年,燕地饑疫,民多流殍,以佶同知南京留守事,發倉廩,振乏絕,貧民鬻子者計傭而出之。宋遣梅詢賀千齡節,詔佶迎送,多唱酬,詢每見稱賞。復為翰林學士。



 重熙元年,升翰林學士承旨。丁母憂,起復工部尚書。歷忠順軍節度使,朔、武等州觀察、處置使,天德軍節度使,加特進檢校太師、同中書門下平章事,復拜參知政事,兼知南院樞密使。



 十五年,出為武定軍節度使。境內亢旱,苗稼將槁。視事之夕,雨澤沾足。百姓歌曰:「何以薊我?上天降雨。誰共撫我?楊公為主。」桑陽水失故道,多為民害,
 乃以己俸創長橋,人不病涉。及被召,郡民攀轅泣送。上御清涼殿宴勞之,即日除吏部尚書,兼門下侍郎、同中書門下平章事。上曰:「卿今日何減呂望之遇文王!」佶對曰:「呂望比臣遭有十年之晚。」



 上悅。其居相位,以進賢為己任,事總大綱,責成百司,人人樂為之用。



 三請致政,許之,月給錢粟兼隸,四時遣使存問。卒。有《登瀛集》行於世。



 耶律和尚,字特抹,系出季父房。善滑稽。



 重熙初,補祗候郎君。時帝篤於親親,凡三父之皆序父兄行第,於和尚尤狎愛。然每侍宴飲,雖恢諧,未嘗有一言之過,由是上益重之。歷積慶、永興宮使,累遷至同知南院宣徽使事、南面林牙。十六年,出為懷化軍節度使,俄召為御史大夫。二十三年,因大冊,加天平軍節度使、檢校太師,徙中京路按問使,卒。



 和尚雅有美行,數以財恤親友,人皆愛重。然嗜酒不事事,以故不獲柄用。或以為言,答曰:「吾非不知,顧人生如風燈石火,不飲將何為?」晚年沈湎尤甚,人稱為「酒仙」云。



 論曰:「庶成定法令,治民者不容高下其手。庶箴雖嘗表請廣姓氏,以秩典禮;其隨勢俯仰,則有愧於其子蒲魯矣。楊皙為上寵遇,迭封王爵,而功業不少概見。然得愛民治國之要,其楊佶哉。」



\end{pinyinscope}