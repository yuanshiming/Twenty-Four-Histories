\article{卷十一本紀第十一 聖宗二}

\begin{pinyinscope}

 聖宗二四年春正月甲戌,觀漁土河。林牙耶律謀魯姑、彰德軍節度使蕭闥覽上東征俘獲,賜詔獎諭。丙子,樞使耶律斜軫、林雅勤德等上討女直所獲生口十餘萬、馬二十餘萬及諸物。



 己卯,朝皇太后。決滯訟。壬午,樞密使斜軫、林雅勤德、謀魯姑、節度使闥覽、統軍使室羅、侍中抹只、奚王府監軍迪烈與安吉等克女直還煙,遣近侍泥
 里吉詔旌其功,仍執手撫諭,賜酒果勞之。甲午,幸長濼。



 二月壬寅,以四番都軍李繼忠為檢校司徒、上柱國。癸卯,西夏李繼造叛宋來降,以為定難軍節度使、銀夏綏宥等州觀察鼾等使、特進檢校太師、都督夏州諸軍事。西番酋帥瓦泥乞移為保大軍節度使、鄜坊等州觀察處置等使。甲寅,耶律斜軫、蕭闥覽、謀魯姑等族師來朝,行飲至之禮,賞齎有差。丙寅,行次裊里井。



 三月甲戌,於越休哥奏宋遣曹彬、崔彥進、米信由雄州道,田重進飛道,潘美、楊繼業雁門道來侵,岐溝、涿州、固安、新城皆陷。詔宣徽使蒲領馳赴燕南,與休哥議軍事;分遣
 使者徵諸療兵益休哥以擊之;復遣東京留守耶律抹只以大軍繼進,賜劍專殺。乙亥,以親征告陵廟、山川。丙子,統軍使耶律頗德敗宋軍於固安,休哥絕其糧餉,擒將吏,獲馬牛、器仗甚眾。



 庚辰,寰州刺趙彥章以城叛,附於宋。辛巳,宋兵入涿州。順義軍節度副使趙希贊以朔州叛,附於宋。時上與皇太后駐兵駝較口,良徵兵馬以為應援。壬午,詔林牙勤德以兵守平州之海岸以備宋。仍報平州節度使迪裡姑,若勤德未至,遣人趣行;馬乏則括民馬;鎧甲闕,則取於顯州之甲坊。癸未,遼軍與宋田重進戰於飛狐,不利,冀州防禦使大鵬翼、康
 州刺史馬贇、馬軍指揮使何萬通陷焉。丁亥,以北院樞使耶律斜軫為山西兵馬都統,以北宣微使蒲領為南征都統,以副於越休哥。彰國軍節度使艾正、觀察判官宋雄以應州叛,附於宋。庚寅,遣飛龍使亞刺、文班吏亞達哥閱馬以給先發諸軍,詔駙馬都尉蕭繼遠領之。辛卯,武定軍馬步軍都指揮使、郢州防禦使呂行德、副都指揮使張繼從、馬軍都指揮使劉知進等以飛狐叛,附於宋。



 癸巳,賜林牙謀魯姑旗鼓四、劍一,率禁軍之驍銳者南助休哥。



 丙申,步軍都軍使穆超以靈丘叛,附於宋。詔遣使賜樞密使斜軫密旨及彰國軍節度使
 杓贇印以趣征討。



 夏四月己亥朔,次南京北郊。庚子,惕隱瑤升、西南面招討使韓德威以捷。辛丑,宋潘美陷雲州。壬寅,遣抹只、謀魯姑、勤德等領偏師以助休哥,仍賜旗鼓、杓贇印撫諭將校。癸卯,休哥復以捷報,上以酒脯祭天地,率群臣賀於皇太后。詔勤德還軍。丙午,頗備上所獲鎧仗數。戊申,監軍、宣徽使蒲領奏敵軍引退,而奚王籌寧、北大王蒲奴寧、統軍使頗德等以兵追躡,皆勝之。遺敞史勤德持詔褒美,及詔侍中抹只統諸軍赴行在所。頻不部節度使和盧睹、黃皮室詳穩解裏等各上所獲兵甲。又詔兩部突騎蔚州,以助闥覽。橫帳郎
 君老君奴率諸郎君巡徼居庸之北。將軍化哥統平州兵馬,橫帳郎君奴哥為黃皮室都監,郎君謁里為北府都監,各以步兵赴蔚州以助斜軫。庚戌,以斜軫為諸路兵馬都統,闥覽兵馬副部署,迪子都監,以代善補、韓德威。癸丑,以艾正、趙希贊及應州、朔州節度副使、奚軍小校隘離轄、渤海小校貫海等叛入於宋,籍其家屬,分賜有功將校。宋將曹彬、米信北渡拒馬河,與於越休哥對壘,挑戰,南北列營長六七里。時上次涿州東五十里。甲寅,詔於越休哥、奚王籌寧、宣徽使蒲領、南、北二王等嚴備水道,無使敵兵得潛至汲州。乙卯,休哥等敗宋軍,獻
 所獲器甲、貨財,賜詔褒美。蔚州左右都押衙李存璋、許彥欽等殺節度使蕭啜里,執監城使、銅州節度使耿紹忠,以城叛,附於宋。丙辰,復涿州,告天地。戊午上次沙姑河之北澱,召林雅勤德議軍事。諸將校所俘獲來上。奚王籌寧、南、北二王率所部將校來朝。以近侍粘米里所進自落鴇祭天地。己未,休哥、蒲領來朝,詔三司給軍前夏衣布。庚申,上朝皇太后。辛酉,大軍次固安。壬戌,圍固安城,統軍使頗德先登,城遂破,大縱俘獲。居民先被俘者,命以官物贖之。甲子,賞攻城將士有差。



 五月庚午,遼師與曹彬、米信戰於岐溝關,大敗之,追至拒馬河,
 溺死者不可勝紀;餘眾奔高陽,又為遼師沖擊,死者數萬,棄戈甲若丘陵。輓漕數萬人匿岐溝空城中,圍之。壬申,以皇太后生辰,縱還。癸酉,班師,還次新城。休哥、蒲領奏宋兵奔逃者皆殺之。甲戌,以軍捷,遣使分諭諸路京鎮。丁丑,詔諸將校,論功行賞,無有不實。己卯,次固安南,以青牛白馬祭天地。庚辰,以所俘宋人射鬼箭。詔遣詳穩排亞率弘義宮兵及南、北皮室、郎君、拽刺四軍赴應、朔二州界,與惕隱瑤升、招討韓備威等同御宋兵在山西之退者。辛巳,以瑤升軍赴山西。壬午,還次南京。癸未,休哥、籌寧、蒲奴寧進俘獲。



 斜軫遣判官蒲姑奏復蔚
 州,斬首二萬餘級,乘勝攻下靈丘、飛狐,賜蒲姑酒及銀器。丙戌,御元和殿,大宴從軍將校,封休哥為宋國王,加蒲領、籌寧、蒲奴寧及諸有功將校爵賞有差。



 丁亥,發南京,詔休哥備器甲,儲粟,待秋大舉南征。戊子,斜軫奏宋軍復圍蔚州,擊破之。詔以兵授瑤升、韓德威等。壬辰,以宋兵至平州,瑤升、韓德威不盡追殺,降詔詰責。仍諭,據城未降乾,必盡掩殺,無使遁逃。癸巳,以軍前降卒分賜扈從。乙未,賞頗備諸將校士卒。



 六月戊戌朔,詔韓德威赴闕,加統軍使頗德檢校太師。甲辰,詔南京留守休哥遣炮手西助斜軫。乙巳,以夷離畢侄里古部送輜重行
 宮,暑行日五十里,人馬疲乏,遣使讓之。丁未,度居庸關。壬子,南京留守奏百姓歲輸三司鹽鐵錢,折絹不如直,詔增之。甲寅,斜軫奏復寰州。乙卯,皇太妃、諸王、公主迎上嶺表,設御幄道傍,置景宗御容,率從臣進酒,陳俘獲於前,遂大宴。戊午,幸涼陘。以所俘分賜皇族及乳母。己未,聞所遣宣諭回鶻、核列哿國度里、亞裡等為術不姑邀留,詔速撒賜術不姑貨幣,諭以朝廷來遠之意,使乾由是乃得行。癸亥,以節度使韓毗哥、翰林學士邢抱樸等充雲州宣諭招撫使。丙寅,以太尉王八所俘生口分賜趙妃及於越迪輦乙里婉。



 秋七月丙子,樞密使斜軫
 遣侍御涅裏底、乾勤哥奏復朔州,擒宋將楊繼業,及上所獲將校印綬、誥敕,賜涅裏底等酒及銀器。辛巳,以捷告天地。以宋歸命者二百四十人分賜從臣。又以殺敵多,詔上京開龍寺建偉事一月,飯僧萬人。辛卯,斜軫奏:大軍至蔚州,營於州左。得諜報,敵兵且至,乃設伏以待。



 敵至,縱兵逆擊,追奔逐北,至飛孤口。遂乘勝鼓行而西,入寰州,殺守城吏卒千餘人。宋將楊繼業初以驍勇自負,號楊無敵,北據雲、朔數州。至是,引兵南出朔州三十里,至狼牙村,惡其名,不進,左右固請,乃行。遇斜軫,伏四起,中流矢,墮馬被擒。瘡發不食,三日死。遂函其首以獻。
 詔詳穩轄麥室傳其首於越休哥,以示諸軍,仍以朔州之捷宣諭南京、平州將吏。自是宋守雲、應諸州者,聞繼來死,皆棄城遁。



 八月丁酉朔,置先離闥覽官六員,領於骨里、女直、迪烈於等諸療人之隸宮籍者。以北大王蒲奴寧為山後五州都管。乙巳,韓備讓奏宋兵所掠州郡,其逃民禾稼,宜募人收獲,以其半給收者,從之。乙卯,斜軫還自軍,獻俘。己未,用室昉、韓德讓言,復山西今年租賦。詔第山西諸將校功過而賞罰之。



 乙室帳宰相安寧以功過相當,追告身一通;諦居部節度使佛奴笞五十。惕隱瑤升、拽刺贇烈、朔州節度使慎思、應州節度使骨只、
 雲州節度使化哥、軍校李元迪、蔚州節度佛留、都監崔其、劉繼琛,皆以聞敵逃遁奪官;贇烈仍配隸本貫;領國舅軍王六笞五十。壬戌,以斜軫我部將校前破女直,後有宋捷,第功加賞。癸亥,加斜軫守太保。



 九月丙寅朔,皇太妃以上納後,進衣物、駝馬,以助會親頒賜。甲戌,次黑河,以笪九登高於高水南阜,祭天。賜從臣命婦菊花酒。丁丑,次河陽北。戊寅,內外命婦進會親禮物。



 辛巳,納皇后蕭氏。丙戌,次儒州,以大軍將南征,詔遣皮室詳穩乞的、郎君拽刺先赴本軍繕甲兵。己丑,召北大王蒲奴寧赴行在所。甲午,皇太后行再生禮。



 冬十月丙申朔,黨
 項、阻卜遣使來貢。丁酉,皇太后復行再行禮,為帝祭神祈福。己亥,以乙室王帳郎君吳留為御史大夫。政事令室昉奏山西四州自宋兵後,人民轉徙,盜賊充斥,乞下有司禁止。命新州節度使蒲打里選人分道巡檢。北大王帳郎君曷葛只裏言本府王蒲奴寧十七罪,詔橫帳太保核國底鞫之。蒲奴寧伏其罪十一,笞二十釋之。曷葛只裏亦伏誣告六事,命詳酌罪之。知事勤德連,杖一百,免官。甲辰,出居庸關。乙巳,詔諸京鎮相次軍行,諸細務權停理問。庚戌,分遣拽刺沿邊偵候。辛亥,命皇族廬帳駐東京延芳澱。壬子,詔以敕榜村於越休哥,以南
 征諭拒馬河南六州。乙卯,幸南京。戊午,以南院大王留寧言,復南院部民今年租賦。壬戌,以銀鼠、青鼠及諸物賜京官、僧道、耆老。甲子,上與大臣分朋擊鞠。



 十一月丙寅朔,黨項來貢。庚午,以政事令韓德讓守司徒,壬申,以古北、松亭、榆關征稅不法,臻阻商旅,遣使鞫之。



 女十請以兵從征,許之。癸酉,御正殿,大勞南將校。丙子,南伐,次狹底堝,皇太后親閱輜重兵甲。丁丑,以休哥為先鋒都統。戊寅,日南至,上率從臣祭酒景宗御容。辛巳,詔以北大王蒲奴寧居奉聖州,山西五州公事,並聽與節度使蒲打里共裁決之。癸未,祭日月,為附馬都尉勤德祈
 福。乙酉,置諸部監,勒所部各守營伍,毋相錯雜。丙戌,遣謀魯姑、蕭繼遠沿邊巡徼。以所獲宋卒射鬼箭。丁亥,以青牛白馬祭天地。辛卯,次白佛塔川,獲自落馴狐,以為吉徵,祭天地。良馬都尉蕭繼遠、林牙謀魯姑、太尉林八等固守封疆,毋漏間諜。軍中無故不得馳馬,及縱諸軍殘南境桑果。壬辰,至唐興縣。時宋軍屯滹沱橋北,選將亂射之,橋不能守,進焚其橋。癸巳,涉沙河,休哥來議事。北皮室詳穩排亞獻所獲宋諜二人。上賜衣物,令還招諭泰州。楮特部節度使盧補古、都監耶律盼與宋戰於泰州,不利。甲午,祭麃鹿神。以盧補古臨陳遁逃,奪告
 身一通判官、都監各杖之。郎君拽刺雙骨裏遇宋先鋒於望都,擒其士卒九人,獲甲馬十一,賜酒及銀器。乙未,以盧補古等罪詔諭諸軍。以御盞郎君化哥權楮特部節度使,橫帳郎君佛留為都監,代盧補古。權領國舅軍桃畏請置二校領散卒,詔以郎君世音、頗德等充。命彰德軍節度使蕭闥覽、將軍迪子略地東路。詔休哥、排亞等議軍事。十二月己亥,休哥敗宋軍於望都,遣人獻俘。任寅,營於滹沱北,詔休哥以騎兵絕宋兵,毋令人邢州;命太師王六謹偵候。癸卯,小校曷主遇宋輜重,引兵殺獲甚眾,並焚其芻粟。



 甲辰,詔南大王與休哥合勢進
 討,宰相安寧領迪離部及三克軍殿。上率大軍與宋將劉廷讓、李敬源戰於莫州,敗之。乙巳,擒宋將賀令圖、楊重進等;國舅詳穩撻烈哥、宮使蕭打里死之。



 丙午,詔休哥以下入內殿,賜酒勞之。丁未,築京觀。復以南京禁軍擊楊圍城,守將以城降。詔禁侵掠。己酉營神榆村,詔上楊圍城粟麥、兵甲之數。辛亥,以黑白二牲祭天地。癸丑,拔馮母鎮,大縱俘掠。丙辰,邢州降。丁巳,拔深州,以不即降,誅襯將以下,縱兵大掠。李繼遷引五百騎窊塞,願婚大國,永作藩輔,詔以王子帳節度使耶律襄之女汀封義成公主下嫁,賜馬三千疋。



\end{pinyinscope}