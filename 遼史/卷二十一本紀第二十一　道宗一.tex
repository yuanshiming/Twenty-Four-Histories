\article{卷二十一本紀第二十一 道宗一}

\begin{pinyinscope}

 道宗孝文皇帝,諱洪基,字涅鄰,小字查刺。興宗皇帝長子,母日仁懿皇后蕭氏。六歲封梁王,重熙十一年進封燕國,總領中丞司事。明年,總北南院樞密使事,加尚書令,進封燕趙國王。二十一年為天下兵馬大元帥,知惕隱事,預朝政。帝性沉靜、嚴毅,每朝,興宗為之斂容。



 二十四年秋八月己丑,興宗崩,即皇帝位於柩前,哀慟不聽政。
 辛卯,百僚上表固請,許之。詔日:「朕以菲德,托居士民之上,第恐智識有不及,群下有水信;賦斂妄興,黨罰不中;上恩不能及下,下情不能達上。凡爾士庶,直言無諱。可則擇用,否則不以為愆。卿等其體朕意。」壬辰,以皇太弟重元為皇太叔,免漢拜,不名。癸巳,遣使報哀於宋及夏、高麗。



 甲午,遣重元安撫南京軍民。戊戌,以遺詔,命西並路招討使西平郡王蕭阿刺為北府宰相,仍權知南院樞密使事,北府宰相蕭虛烈為武定軍節度使。辛丑,改元清寧,大赦。



 九月戊午,詔常所幸圍場外毋禁。庚申,詔除護衛士,餘不得佩刃入宮;非勛戚後及夷離堇、副使、
 承應諸職事人不得冠巾。壬戌,詔夷離堇及副使之族並民如賤,不得服駝尼、水獺裘,刀柄、兔鶻、鞍勒、珮子不許用犀玉、骨突犀;惟大將軍不禁。乙丑,賜內外臣僚爵賞有差。庚午,尊皇太后為太皇太后。辛未,遣左夷離畢蕭謨魯、翰林學土韓運以先帝遺物遺宋。癸酉,遣使以即位報宋。丙子,尊皇后為皇太后,宴菆塗殿。以上京留守宿國王陳留為南京留守。壬午,遣使賜高麗、夏國先帝遺物。



 冬十月丁亥,有司請以帝生日為天安節,從之。以吳王仁先同知南京留守事,陳王塗孛特為南府宰相,進封吳王。壬寅,以順義軍節度使十神奴為南院大
 王。



 十一月甲子,葬興宗皇帝於慶陵。宋及高麗遣使來會。名其山日永興。丙寅,以南院大王侯古為中京留守,北府宰相西平郡王蕭阿刺進封韓王。壬申,次懷州。有事於太宗、穆宗廟。



 甲戌,謁祖陵。戊寅,冬至,有事於太祖、景宗、興宗廟,不受群臣賀。



 十二月丙戌,詔左夷離畢曰:「朕以眇沖,獲嗣大位,夙夜憂懼,恐弗克任。欲聞直言,以匡其失。今己數月,末見所以副朕委任股肱耳目之意。其令內外百官,比秩滿,各言一事。



 仍轉諭所部,無貴賤老幼,皆得直言無諱。」戊子,應聖節,上太皇太后壽,宴群臣、命婦,冊妃蕭氏為皇后。進封皇弟越王和魯斡為魯
 國王,許王阿璉為陳國王,楚王涅魯古徒封吳王。



 辛卯,詔部署院,事有機密即奏,其投謗訕書,輒受及讀者並棄市,癸巳,皇族十公悖母,伏誅。甲午,以樞密副使姚景行為參知政事,翰林學士吳湛為樞密副使,參知政事、同知樞密院事韓紹文為上京留守。丙申,宋遣歐陽修等來賀即位。戊戌,詔設學養士,頒《五經》傳疏,置博士、助教各一員。癸卯,以知琢州楊績參知政中兼同知樞密院事。庚戌,以聖宗在時生辰,赦上京囚。



 是年,御清涼殿放進士張孝傑等四十四人。



 二年春正月丙辰,詔州郡官及僚屬決囚,如諸部族例。
 己巳,詔二女古部與世預宰相、節度使之選者免皮室軍。是月,幸魚兒濼。



 二月乙酉,以左夷離畢蕭謨魯知西南面招討都監事。乙巳,以興宗在時生辰,宴群臣,命備賦詩。



 三月丁巳,應聖節,曲赦百里內因。己卯,禦制《放鷹賦》賜群臣,諭任臣之意。



 閏月己亥,始行東京所鑄錢。乙巳,南京獄空,進留守以下官。夏四月甲子,詔日:「方夏,長養鳥獸孳育之時,不得縱火於郊。&quot;五月戊戌,竭慶陵。甲辰,有事於興宗廟。



 六月丁巳,詔宰相舉才能之士。戊午,命有司籍軍補邊戍。



 辛酉,阻卜酋長來朝,貢方物。丁卯,高麗遣使來貢。辛未,罷史官預聞朝議,俾問宰相而後
 書。乙亥,中京蝗蝻為災。丙子,詔強盜得實者,聽諸路決之。丁丑,南院樞密使趙國王查葛為上京留守,同知南京留守事吳王仁先為南院樞密使。乙酉,遣使分道平賦稅,繕戎器,勸農桑,禁盜賊。



 八月辛未,如秋山。



 九月庚子,幸中京,祭聖宗、興宗於會安殿。



 冬十月丙子,如中會川。



 十一月戊戌,知左夷離畢事耶律劃里夷離畢,北院大王耶律仙童知黃龍府事。甲辰,文武百僚上尊號曰天佑皇帝,後日懿德皇后,大赦。乙巳,以皇太叔重元為天下兵馬大元帥,徙封趙國王查葛為魏國王、魯國王和魯斡為宋國王、陳國王阿璉為秦國王,吳王涅魯
 古進封楚國王,百官進遷有差。



 十二月戊申朔,以韓王蕭阿刺為北院樞密使,東京留守宿國王陳留北府宰相,宋國王和魯斡上京留守,秦國王阿璉知中丞司事。甲寅,上皇太后尊號日慈懿仁和文惠孝敬廣愛宗天皇太后。三年春正月庚辰,如鴨子河。丙戌,置倒塌嶺節度使。乙未,五國部長來貢方物。



 二月己未,如大魚濼,三月辛巳,以楚國王涅魯古為武定軍節度使。



 夏四月丙辰,帶暑永安山。



 五月己亥,如慶陵,獻酎於金殿、同天殿。



 六月辛未,以魏國王查葛為惕隱,同知樞密院事蕭唐古南府
 宰相,魏國王貼不東京留守。



 秋七月甲申,南京地震,赦其境內。乙酉,如秋山。



 八月辛亥,帝以《君臣同志華夷同風詩》進皇太后。



 九月庚子,幸中會川。



 冬十月己酉,謁祖陵。庚申,謁讓國皇帝及世宗廟。辛酉,奠酣於玉殿。



 十一月丙子,以左夷離畢蕭謨魯為契丹行宮都部署。庚子,高麗遣使來貢。



 十二月庚戌,禁職官於部內假貸貿易。戊辰,太皇太后不豫,曲赦行在五百里內因。己巳,太皇太后崩。



 四年春正月壬申朔,遣使報哀於宋、夏。如鴨子河鉤魚。



 癸酉,宋遣使奉宋主繪像來。丁亥,知易州事耶律頗得
 秩滿,部民乞留,許之。



 二月丙午,詔夷離畢:諸路鞠死罪,獄雖具,仍令別州縣覆按,無冤,然後決之;稱冤者,即具奏。庚戌,如魚兒濼。



 三月戊寅、募天德、鎮武、東勝等處勇捷者,籍為軍。甲午;肆赦。



 夏四月甲辰,褐慶陵。丁卯,宋遣使吊祭。



 五月庚午朔,上大行太皇太后尊謚日欽哀皇后。癸酉,葬慶陵。夏國、高麗遣使來會。乙酉,如水安山清暑。



 六月乙丑,以北院樞密使鄭王蕭革為南院樞密使,徙封楚王,南院樞密使吳王仁先為北院樞密使。



 秋七月辛巳,制諸掌內藏庫官盜兩貫以上者,許奴婢告。



 壬午,獵於黑嶺。



 冬十月戊戌朔,以同知東京留守事侯古
 為南院大王,保安軍節度使奚底為奚六部大王。



 十一月癸酉,行再生有柴冊禮,宴群臣於八方陂。庚辰,御清風殿受大冊禮。大赦。以吳王仁先為南京兵馬副元帥,徙封隋王。壬午,謁太祖及諸帝宮。丙戌,柯木葉山。禁造玉器。



 十二月辛丑,弛駝尼、水獺裘之禁。乙巳,許士庶畜鷹。



 辛亥,南院樞密使楚王蕭革復為北院樞密使。



 閏月己巳,賜皇太叔重元金券。



 是歲:皇子浚生。



 五年春,如春州。



 夏六月甲子朔,駐蹕納葛濼。己丑,以南院樞密使蕭阿速為北府宰相,樞密副使耶律乙辛南院樞密使,惕隱查葛遼興軍節度使,魯王謝家奴武定
 軍節度使,東京留守吳王貼不西京留守。



 秋七月丁酉,以烏古敵烈詳穩蕭謨魯為左夷離畢。



 冬十月壬子朔,幸南京,祭興宗於嘉寧殿。



 十一月,禁獵。



 十二月壬戌,以北院林牙奚馬六為右夷離畢,參知政事吳湛以弟洵冒入仕籍,削爵為民。



 是年,上御百福殿,放進士梁援等百一十五人。



 六年春,如鴛鴦濼。



 夏五月戊子朔,監修國史耶律白請編次御制詩賦,仍命白為序。己酉,駐蹕納葛濼。



 六月戊午朔,以東北路女直詳穩高家奴為惕隱。壬戌,遣使錄囚。丙寅,中京置國子監,命以時祭先聖先師。癸未,以隋
 王仁先為北院大王,賜禦制誥。



 冬十月甲子,駐蹕藕絲澱。



 七年春三月庚戌,如春州。以耶律乙辛知北院樞密使事。



 夏四月辛未,禁吏民畜海東青鶻。



 五月丙戌,清暑永安山。丙午,謁慶陵。辛亥,殺東京留守陳王蕭阿刺。



 六月壬子朔,日有食之。甲子,以蕭謨魯為順義軍節度使。



 丁卯,幸弘義、水興、崇德三宮致祭,射柳,賜宴,賞賚有差。



 戊辰,行再生禮,復命群臣分朋射柳。丁丑,以楚國王涅魯古知南院樞密使事。



 秋九月丁丑,駐蹕藕絲澱。



 冬十二月壬午,以知黃龍府事耶律阿里只為南院大王。



\end{pinyinscope}