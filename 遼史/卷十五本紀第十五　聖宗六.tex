\article{卷十五本紀第十五 聖宗六}

\begin{pinyinscope}

 二十
 八年春正月辛亥朔,不受賀。甲寅,如乾陵。癸酉,奉安大行皇太后梓宮於乾州菆塗殿。



 二月丙戌,宋遣王隨、王儒等來吊祭。己亥,高麗遣魏守愚等來祭。是月,遣左龍虎衛上將軍蕭合卓饋大行皇太后遺物於宋,仍遣臨海軍節度使蕭虛列、左領軍衛上將軍張崇濟謝宋吊祭。三月癸卯,上大行皇太后謚為聖神宣獻皇后。
 是月,宋、高麗遣使來會葬。



 夏四月甲子,葬太后於乾陵。賜大丞相耶律德昌名曰隆運。



 庚午,賜宅及陪葬地。



 五月己卯朔,如中京。辛卯,清暑七金山。乙巳,西北路招討使蕭圖玉奏伐甘州回鶻,破蕭州,盡俘其民。詔修土隗口故城以實之。丙午,高麗西京留守庚肇弒其主誦,擅立誦從兄詢,詔諸道繕甲兵,以備東征。



 秋八月戊申,振平州饑民。辛亥,幸中京。丙寅,謁顯、乾二陵。丁卯,自將伐高麗,遣使報宋。以皇弟楚國王隆祐留守京師,北府宰相、駙馬都尉蕭排押為都統,北面林牙僧奴為都監。九月乙酉,遣使冊西平王李德昭為夏國王。辛卯,遣樞密
 直學士高正、引進使韓把宣問高麗王詢。



 冬十月丙午朔,女直進良馬萬匹,乞從征高麗,許之。王詢遣使奉表乞罷師,不許。



 十一月乙酉,大軍渡鴨綠江,庚肇拒戰,敗之,退保銅州。



 丙戌,肇復出,右皮室詳穩耶律敵魯擒肇及副將李立,追殺數十里,獲所棄糧餉、鎧仗。戊子,銅、霍、貴、寧等州皆降。



 排押至奴吉達嶺,遇敵兵,戰敗之。辛卯,王詢遣使上表請朝,許之。禁軍士俘掠。以政事舍人馬保佑為開京留守。安州團練使王八為副留守。遣太子太師乙凜將騎兵一千,送保佑等赴京。



 壬辰,守將卓思正殺遼使者韓喜孫等十人,領兵出拒,保佑等還。遣乙
 凜領兵擊之。思正遂奔西京。圍之五日不克。駐蹕城西。高麗禮部郎中渤海陀失來降。庚子,遣排押、盆奴等攻開京,遇高麗兵,敗之。王詢棄城遁去,遂焚開京,至清江,還。



 二十九年春正月乙亥朔,班師,所降諸城復叛。至貴州南峻嶺谷,大雨連日,馬駝皆疲,甲仗多遺棄,霽乃得渡。己丑,次鴨祿江。庚寅,皇后及皇弟楚國王隆祐迎於來運城。壬辰,詔罷諸軍。己亥,次東京。



 二月己酉,謁乾、顯二陵。戊午,所俘高麗人分置諸陵廟,余賜內戚、大臣。



 三月己卯,大丞相晉國王耶律隆運薨。庚辰,皇弟楚國王隆
 祐權知北院樞密使事,樞密直學士高正為北院樞密副使。庚寅,南京、平州水,振之。己亥,以北院大王耶律室魯為北院樞密使,封韓王,北院郎君耶律世良為北院大王,前三司使劉慎行參知政事兼知南院樞密使事。



 夏四月,清暑老古堝。五月甲戌朔,詔己奏之事送所司附《日歷》。又詔帳族有罪,黥墨依諸部人例。乙未,以劉慎行為南院樞密使,南府宰相邢抱質知南院樞密使事。



 六月庚戌,升蔚州、利州為觀察使。乙卯,韓王耶律室魯薨。丙辰,以南院大王化哥為北院樞密使。丁巳,詔西北路招討使、駙馬都尉蕭圖玉安撫西鄙。置阻卜諸部節
 度使。



 是秋,獵於平地松林。



 冬十月庚子朔,駐蹕廣平澱。甲寅,贈大丞相晉國王耶律隆運尚書令,謚文忠。



 十一月庚午朔,幸顯州。



 十二月庚子朔,復如廣平澱。癸丑,以知南院樞密使事邢抱質年老,詔乘小車入朝。是月,置歸、寧二州。



 是年,御試,放高承顏等二人及第。



 開泰元年春正月己巳朔,宋遣趙湘、符成翰來賀。癸未,長白山三十部女直酋長來貢,乞授爵秩。甲申,駐蹕王子院。



 丙戌,望祠木葉山。丁亥,女直太保蒲捻等來朝。戊子,獵於賣曷魯林。庚寅,祠木葉山。辛卯,曷蘇館大王曷裡喜來朝。



 二月壬子,駐蹕瑞鹿原。



 三月甲戌,以蔚州為
 觀察,不隸武定軍。乙亥,如葦濼。



 丁丑,詔封皇女八人為郡主。乙酉,詔卜日行拜山、大射柳之禮,命北宰相、駙馬、蘭陵郡王蕭寧。樞密使、司空邢抱質督有司具儀物。丁亥,皇弟楚國王隆祐徙封齊國王,留守東京。



 夏四月庚子,高麗遣蔡忠順來,乞稱臣如舊,詔王詢親朝。



 壬寅,夏遣使進良馬。己酉,祀風伯。辛酉,以前孟父房敞穩蕭佛奴為左夷離畢。



 五月戊辰朔,還上京。詔裴玄感、邢祥知禮部貢舉,放進士史簡等十九人及第。以駙馬蕭紹宗為鄭州防禦使。乙亥,以邢抱質為大同軍節度使。



 六月,駐蹕上京。



 秋七月丙子,以耶律遂貞為遼興軍節度使,
 遂正北院宣徽使,張昭瑩南院宣徽使,耶律受益上京副留守,寇卿彰德軍節度使。命耶律釋身奴、李操充賀宋生辰國信使副,蕭涅袞、齊泰賀宋正旦使副。進士庚文昭、張素臣、郎玄達坐論知貢舉裴玄感、邢祥私曲,秘書省正字李萬上書,辭涉恕訕,皆杖而徒之,萬役陷河治。



 八月丙申朔,鐵驪那沙等送兀惹百餘戶至賓州,賜絲絹。



 是日,那沙乞賜佛像、儒書,詔賜《護國仁王佛像》一,《易》、《詩》、《書》、《春秋》、《禮記》各一部。己未,高麗王詢遣田拱之奉表稱病不能朝,詔復取六州地。是月,齊國王隆祐亮,輟朝五日。



 冬十月辛亥,如中京。



 閏月丁卯,贈隆祐守太師,謚
 仁孝。



 十一月甲午朔,文武百官加上尊號曰弘文宣武尊道至德崇仁廣孝聰睿昭聖神贊天輔皇帝。大赦,改元開泰。改幽都府為析津府,薊北縣為析津縣,幽都縣為宛平縣,覃恩中外。己亥,賜夏國使、東頭供奉官曹文斌、呂文貴、竇珪祐、守榮、武元正等爵有差。癸卯,前遼州錄事張庭美六世同居,儀坤州劉興澈四世同居,各給復三年。甲辰,西北招討使蕭圖玉奏七部太師阿裏底因其部民之怨,殺本部節度使霸暗並屬其家以叛,阻卜執阿裏底以獻,而沿邊諸部皆叛。



 十二月丙寅,奉遷南京諸帝石像於中京觀德殿,景宗及宣獻皇后於上京
 五鸞殿。壬申,振奉聖州饑民。庚辰,賜皇弟秦晉國王隆慶鐵券。癸未,劉晨言殿中高可垣、中京留守推官李可舉治獄明允,詔超遷之。甲申,詔諸道水災饑民質男文者,起來年正月,日計傭錢十文,償價傭盡,遣還其家。歸州言其居民本新羅所遷,未習文字,請設學以教之。詔允所請。貴德、龍化、儀坤、雙、遼同、祖七州,至是有詔始征商。己丑,詔諸鎮建宣敕樓。



 二月春正月癸巳朔,以裴玄感為翰林承旨,邢祥給事中,石用中翰林學士,呂德推樞密直學士,張儉政事舍人,邢抱質加開府儀同三司、守司空兼侍中,王繼忠中
 京留守、檢校太師,戶部侍郎劉涇加工部尚書,駙馬蕭紹宗加檢校太師,耶律控溫加政事令,封幽王。丁未,如瑞鹿原。北樞密使耶律化哥封豳王。以馬氏為麗儀,耿氏淑儀,尚寢白氏昭儀,尚服李氏順儀,尚功艾氏芳儀,尚儀孫氏和儀。己未,錄囚。烏古、敵烈叛,右皮室詳穩延壽率兵討之。是月,達旦國兵圍鎮州,州軍堅守,尋引去。



 二月丙子,詔以麥務川為象雷縣,女河川為神水縣,羅家軍為閭山縣,出於川為富庶縣,習家砦為龍山縣,阿覽峪為勸農縣,松山川為松山縣,金甸子為金原縣。壬午,遣北院樞密副使高正按察諸道獄。



 三月壬辰朔,
 化哥以西北路略平,留兵戍鎮州,赴行在。



 夏四月甲子,拜日。詔從上京請,以韓斌所括贍國、撻魯河、奉、豪等州戶二萬五千四百有奇,置長霸、興仁、保和等十縣。丙子,如緬山。



 五月辛卯朔,復命化哥等西討。



 六月辛酉朔,遣中丞耶律資忠使高麗,取六州舊地。



 秋七月壬辰,烏古、敵烈皆復故疆。乙未,西南招討使、政事令斜軫奏,黨項諸部叛者皆遁黃河北模赧山,其不叛者易黨、烏迷兩部因據其地,今復西遷,詰之則曰逐水草。不早圖之,後恐為患。又聞前後叛者多投西夏,西夏不納。詔遣使再間西遷之意,若歸故地,則可就加撫諭。使不報,上怒,欲
 代之。遂詔李德昭:「今黨項叛,我欲西伐,爾當東擊,毋失掎角之勢。」仍命諸軍各市肥馬。丁酉,以惕隱耶律滌洌為南府宰相,太尉五哥為惕隱。癸卯,鉤魚曲溝。戊申,詔以敦睦宮子錢振貧民。己酉,化哥等破阻卜酋長烏八之眾。丁卯,封皇子宗訓大內惕隱。



 八月壬戌,遣引進使李延弘賜夏國王李德昭及義成公主車馬。己丑,耶律資忠使高麗還。



 冬十月己未朔,畋麃井之北。命耶律阿營等使宋賀生辰。



 辛酉,駐蹕長濼。丙寅,詳穩張馬留獻女直人知高麗事者。上問之。曰:「臣三年前為高麗所虜,為郎官,故知之。自開東京馬行七日,有大砦,廣如開京。
 旁州所貢珍異,皆積於此。



 勝、羅等州之南,亦有二大砦,所積如之。若大軍行由前路,取曷蘇館女直北,直渡鴨綠江,並大河而上,至郭州與大路會,高麗可取而有也。」上納之。



 十一月甲午,錄囚。癸丑,樞密使豳王化哥以西征有罪,削其官封,出為大同軍節度使。



 十二月甲子,以北院大王耶律世良為北院樞密使,封歧王。



 以宰臣劉晟監修國史,牛璘為彰國軍節度使,蕭孝穆為西北路招討使。



 放進士鮮於茂昭等六人及第。



 三年春正月己丑,錄囚。阻卜酋長烏八來朝,封為王。乙未,如渾河。丁酉,女直及鐵驪各遣使來貢。是夕,彗星見
 西方。丙午,畋演河濱。壬子,帝及皇后獵瑞鹿原。



 二月戊午,詔增樞密使以下月俸。甲子,遣上京副留守耶律資忠復使高麗取六州舊地。三月庚子,遣耶律世良城招州。戊申,南京、奉聖、平、蔚、雲、應、朔等州置轉運使。



 夏四月戊午,詔南京管內毋淹刑獄,以妨農務。癸亥,烏古叛。乙亥,沙州回鶻曹順遣使來貢。丙子,以西北路招討都監蕭孝穆為北府宰相。



 五月乙酉,朔,清暑緬山。



 六月乙亥,合拔里、乙室二國舅為一帳,以乙室夷離畢蕭敵烈為詳穩以總之。甲申,封皇侄胡都古為廣平郡王。



 是夏,詔國舅詳穩蕭敵烈、東京留守耶律團石等討高麗,造浮
 梁於鴨綠江,城保、宣義、定遠等州。



 秋七月乙酉朔,如平地松林。壬辰,詔政事省、樞密院,酒間授官釋罪,毋即奉行,明日覆奏。



 八月甲寅朔,幸沙嶺。



 九月丁酉,八部敵烈殺其詳穩稍瓦,皆叛,詔南府宰相耶律吾刺葛招撫之。辛亥,釋敵烈數人,令招諭其眾。壬子,耶律世良遣使獻敵烈俘。



 冬十月甲寅朔,幸中京。丙子,以旗鼓拽刺詳穩題裡姑為奚六部大王。



 放進士張用行等三十一人及第出身。



 四年春正月乙酉,如瑞鹿原。丙戌,詔耶律世良再伐迪烈得。戊子,命詳穩拔姑瀦水瑞鹿原,以備春搜。丁酉,獵
 馬蘭澱。壬寅,東征。東京留守善寧、平章涅里袞奏,已總大軍及女直諸部兵分道進討,遂遣使費密詔軍前。



 二月壬子朔,如薩堤凍。於闐國來貢。



 夏四月癸丑,以林牙建福為北院大王。甲寅,蕭敵烈等代高麗還。丙辰,曷蘇館部請括女直王殊只你戶舊無籍者,會其不入賦役,從之。樞密使貫寧奏大破八部迪烈得,詔侍御撒刺獎諭,代行執手之禮。丙寅,耶律世良等上破阻卜俘獲數。戊辰,駐蹕沿柳湖。己巳,女直遣使來貢。壬申,耶律世良討烏古,破之。甲戌,遣使賞有功將校。世良討迪烈得至清泥堝。



 時於厥既平,朝廷議內徙其眾,於厥安土重遷,
 遂叛。世良懲創,既破迪烈得,輒殲其丁壯。勒兵渡曷刺河,進擊餘黨,斥候不謹,其將勃括聚兵稠林中,擊遼軍不備。遼軍小卻,結陣河曲。勃括是夜來襲。翌日,遼後軍至,勃括誘於厥之眾皆遁,世良追之,軍至險厄。勃括方阻險少休,遼軍偵知其所,世良不亟掩之,勃括輕騎遁去。獲其輜重及所誘於厥之眾,並遷迪烈得所獲轄麥里部民,城臚朐河上以居之。是月,蕭楊哥尚南平郡主。



 五月辛巳,命北府宰相劉晟為都統,樞密使耶律世良為副,殿前都點檢蕭屈烈為都監以伐高麗。晟先攜家置邊郡,致緩師期,追還之。以世良、屈烈總兵進討。以耶
 律德政為遼興軍節度使,蕭年骨烈天城軍節度使。李仲舉卒,詔賻恤其家。



 六月庚戌,上拜日如禮。以麻都骨世勛,易衣馬為好。以上京留守耶律八哥為北院樞密副使。



 秋七月,上又拜日,遂幸秋山。



 自八月射鹿至於九月,復自癸丑至於辛酉,連獵於有柏、碎石、太保、響應、松山諸山。丁卯,與夷離畢、兵部尚書蕭榮寧定力交契,以重君臣之好。丙子,以旗鼓拽刺詳穩題裡姑為六部奚王。



 冬十月,駐蹕撻刺割濼。



 十一月庚申,詔汰東京僧,及命上京、中京洎諸宮選精兵五萬五千人以備東征。



 十二月,南巡海徼。還,幸顯州。



 五年春正月丁未,北幸。庚戌,耶律世良、蕭屈烈與高麗戰於郭州西,破之,斬首數萬級,盡獲其輜重。乙卯,師次南海軍,耶律世良薨於軍。癸酉,駐蹕雪林。



 二月已卯,阻卜長來朝。辛巳,如薩堤濼。庚寅,以前東京統軍使耶律韓留為右夷離畢。戊戌,皇子宗真生。



 三月乙卯,鼻骨德長撒保特、賽刺等來貢。辛酉,諸道獄空,詔進階賜物。丙寅,以前北院大王耶律敬溫為阿扎割只。



 辛未,黨項魁可來降。



 夏四月乙亥,振招州民。戊寅,以左夷離畢蕭合卓為北院樞密使,曷魯寧為副使。庚辰,清暑孤樹澱。



 五月甲子,尚書蕭姬隱坐出使後期,削其官。丁卯,以耿元
 吉為戶部使。



 六月,以政事舍人吳克昌按察霸州刑獄。丁丑,回鶻獻孔雀。



 秋七月甲辰,獵於赤山。



 八且丙子,幸懷州,有事於諸陵。戊寅,還上京。



 九月癸卯,皇弟南京留守秦晉國王隆慶來朝,上親出迎勞至實德山。因同獵於松山。乙丑,駐蹕否堝。



 冬十月甲午;封秦晉國王隆慶長子查割中山郡王,次子遂哥樂安郡王。



 十一月辛丑朔,以參知政事馬保忠同知樞密院事、監修國史。丁巳,以北面林牙蕭隈窪為國舅詳穩。



 十二月乙酉,秦晉國王隆慶還,至北安薨,訃聞,上為哀慟,輟朝七日。丁酉,宋遣張遜、王承德來賀千齡節。



 是歲,放進士孫傑等因十八
 人及第。



 六年春正月癸卯,如錐子河。



 二月甲戌,以公主賽哥殺無罪婢,駙馬蕭圖玉不能齊家,降公主為縣主,削圖玉同平章事。丁丑,詔國舅帳詳穩蕭隗窪將本部兵東征高麗,其國舅司事以都監攝之。庚辰,以南面林牙涅合為南院大王。



 三月乙巳,如顯州,葬秦晉國王隆慶。有事於顯、乾二陵。



 追冊隆慶為太弟。



 夏四月辛卯,封隆慶少子謝家奴為長沙郡王,以樞密使漆水郡王耶律制心權知諸行宮都部署事。壬辰,禁命婦再醮。丙申,如涼陘。



 五月戊戌朔,命樞密使蕭合卓為都統,漢人行宮都部署
 王繼忠為副,殿前都點檢蕭屈烈為都監以伐高麗。翌日,賜合卓劍,俾得專殺。丙午,錄囚。己酉,設四帳都詳穩。甲寅,以南京統軍使蕭惠為右夷離畢。乙卯,祠木葉山、潢河。乙丑,駐蹕九層臺。



 六月戊辰朔,德妃蕭氏賜死,葬兔兒山西。後數日,大風起塚上,晝暝,大雷電而雨不止者逾月。是月,南京諸縣蝗。



 秋七月辛亥,如秋山。遣禮部尚書劉京、翰林學士吳叔達、知制誥仇正己、起居舍人程翥、吏部員外郎南承顏、禮部員外郎王景運分路按察刑獄。辛酉,以西南路招討請,置寧仁縣於勝州。九月庚子,還上京,以皇子屬思生,大赦。丁未,以駙馬蕭璉、節
 度使化哥、知制誥仇正己、楊佶充賀宋生辰正旦使副。



 乙卯,蕭合卓等攻高麗興化軍不克,還師。



 冬十月丁卯,南京路饑,輓雲、應、朔、弘等州粟振之。



 辛未,獵鏵子河。庚寅,駐蹕達離山。



 十一月乙卯,建州節度使石匡弼卒。



 十二月丁卯,上輕騎還上京。戊子,宋遣李行簡、張信來賀千齡節。翌日,宋馮元,張綸來賀正旦。



\end{pinyinscope}