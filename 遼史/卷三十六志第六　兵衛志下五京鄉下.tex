\article{卷三十六志第六 兵衛志下五京鄉下}

\begin{pinyinscope}

 遼建五
 京:臨演,契丹故壤;遼陽,漢之遼東,為渤海故國;中京,漢遼西地,自唐以來契丹有之。三京丁籍可紀者二十二萬六千一百,蕃漢轉戶為多。析津、大同,故漢地,籍丁八十萬六千七百。契丹本戶多隸宮帳、部族,其餘蕃漢戶丁分隸者,皆丁與焉。



 太祖建皇都於臨潢府。太宗定晉,晉主石敬瑭來獻十六城,乃定四京,改皇都為上京。有丁一十六萬七千二百。



 臨潢府:臨潢縣丁七千。



 長泰縣丁八千。



 保和縣丁六千。



 定霸縣丁六千。



 宣化縣丁四千。



 潞縣丁六千。



 易俗縣丁一千五百。



 遷遼縣丁一千五百。



 祖州:長霸縣丁四千。



 咸寧縣丁二千。



 越王城丁二千。



 懷州:扶餘縣丁三千。



 顯理縣丁二千。



 慶州
 玄寧縣丁一萬二千。



 泰州興國縣丁一千四百。



 長春州長春縣丁四千。



 烏州愛民縣丁二千。



 永州:長寧縣丁九千。



 義豐縣丁三千。



 慈仁縣丁八百。



 儀坤州廣義縣丁五千。



 龍化州龍化縣丁二千。



 降聖州永安縣丁一千五百。



 饒州:長樂縣丁八千。



 臨河縣丁二千。



 安民縣丁二千。



 頭下:徽州丁二萬。



 成州丁大千。



 懿州丁八千。



 渭州丁二千。



 原州丁一千。壕州丁一萬二千。



 福州丁五百。



 橫州丁四百。



 鳳州丁一千。



 遂州丁丁千。



 豐州丁一千。



 順州丁二千。



 閭州丁二千。



 松山州丁一千。



 豫州丁一千。



 寧州丁六百。



 東京,本渤海,以其地建南京遼陽府。統縣六,轄軍、府、州、城二十六,有丁四萬一千四百。天顯十三年,太宗改為東京。



 遼陽府:遼陽縣丁三千。



 仙鄉縣丁三千。



 鶴野縣丁二千四百。



 析木縣丁二千。



 紫蒙縣丁二千。



 興遼縣丁二千。



 開州開遠縣丁二千。



 鹽州丁五百。



 穆州丁五百。



 賀州丁五百。



 定州定東縣丁一千六百。



 保州來遠縣丁二千。辰州丁四千。



 盧州丁五百。



 鐵州丁二千。



 興州丁三百。



 湯州丁七百。



 崇州丁一千。



 海州丁三千。



 耀州丁一千二百。



 嬪州丁七百。



 淥州丁四千。



 桓州丁一千。



 豐州丁五百。



 正州丁七百。



 慕州丁三百。



 南京析津府,統縣十一,轄軍、府、州、城九,有丁五十六萬六千。



 析津府:析津縣丁四萬。



 宛平縣丁四萬四千。



 昌平縣丁一萬四千。



 良鄉縣丁一萬四千。



 潞縣丁一萬一千。



 安次縣丁二萬四千。



 武清縣丁二萬。



 永清縣丁一萬。



 香河縣丁一萬四千。



 玉河縣丁二千。漷陰縣丁一萬。



 順州懷柔縣丁一萬。



 檀州:蜜雲縣丁一萬。



 行唐縣丁六千。



 涿州:
 範陽縣丁二萬。



 固安縣丁二萬。



 新城縣丁二萬。



 歸義縣丁八萬。



 易州:易縣丁五萬。



 淶水縣丁五萬四千。



 容城縣丁一萬。



 薊州:漁陽縣丁八千。



 三河縣丁六千。



 玉田縣丁六千。



 平州:盧龍縣丁一萬四千。



 安喜縣丁一萬。



 望都縣丁六千。



 灤州:義豐縣丁八千。



 馬城縣丁六千。



 石城縣丁六千。



 營州廣寧縣丁六千。景州遵化縣丁六千。



 西京大同府,統縣七,轄軍、府、州、城十七,有丁三十二萬二千七百。



 大同府:大同縣丁二萬。



 雲中縣丁二萬。



 天成縣丁一萬。



 長青縣丁八千。



 奉義縣丁六千。



 懷仁縣丁六千。



 懷安縣丁六千。



 弘州:永寧縣丁二萬。



 順聖縣丁六千。



 德州宣德縣丁六千。



 豐州:
 富民縣丁二千四百。



 振武縣鄉兵三百。



 奉聖州:永興縣丁一萬六千。



 礬山縣丁六千。



 龍門縣丁八千;望雲縣丁二千。



 歸化州文德縣丁二萬。



 可汗州
 懷來縣丁六千。



 儒州縉山縣丁一萬。



 蔚州:靈仙縣丁四萬。



 定安縣丁二萬。



 飛狐縣丁一萬。



 靈丘縣丁六千。



 廣陵縣丁六千。



 應州:
 金城縣丁一萬六千。



 渾源縣丁一萬。



 河陰縣丁六千。



 朔州:都陽縣丁八千。



 寧遠縣丁四千。



 馬邑縣丁六千。



 金肅軍防秋兵一千。



 武州神武縣丁一萬。



 河清軍防秋兵一千。



 聖宗統和二十三年,城七金山,建人定府,號中京。統縣九,轄軍、府、州、城二十三。草創未定,丁籍莫考,可見者一縣:高州三韓縣丁一萬。



 大約五京民丁可見者,一百一十萬七千三百為鄉兵。



 屬國軍遼屬國可紀者五十有九,朝貢無常。有事則道使徵兵,或下詔專征;不從者討之。助軍眾寡,各從其便,無常額。
 又有鐵不得國者,興宗重熙十七年乞以兵助攻夏國,詔丁許。吐谷渾。鐵驪。靺鞨。



 兀惹。



 黑車子室韋。



 西奚。



 東部奚。



 烏馬山奚。



 斜離底。



 突厥。



 黨項。



 小蕃。



 沙陀。



 阻卜。



 烏古。



 素昆那。



 胡母思山蕃。



 波斯。



 大食。



 甘州回鶻。



 新羅。



 烏孫。



 敦煌。



 賃烈。



 要裏。



 回鶻。



 轄戛斯。



 吐蕃。



 黃室韋。小黃室韋。



 大黃室韋。



 阿薩蘭回鶻。



 于闐。



 師子。



 北女直。



 河西黨項。



 南京女直。



 沙州敦煌。



 曷蘇館。



 沙州回鵲。



 查只底。



 蒲盧毛朵。



 蒲奴里。



 大蕃。



 高昌。



 回拔。



 頗里。



 達裏底。



 拔思母。



 敵烈。



 粘八葛。



 梅裏急。



 耶睹刮。



 鼻骨德。



 和州回鶻。



 斡朗改。高麗。



 西夏。



 女直。



 遼之為國,鄰於梁、唐、晉、漢、周、宋。晉以恩故,始則父子一家,終則寇仇相攻,梁、唐、周、隱然一敵國;宋惟太宗征北漢,遼丁能救,餘多敗衄,縱得亦丁償失。良由石晉獻土,中國失五關這固然也。高麗小邦,屢喪遼兵。非以險阻足恃故歟。西夏彈丸之地,南敗宋,東抗遼。雖西北士馬雄勁,元昊、諒祚智勇過人,能使黨項、阻卜制肘大國,蓋亦襟山帶河,有以助其勢耳。雖然,宋久失地利,而舊《志》言兵,唯以敵宋為務。逾三關,聚議北京,猶丁敢輕進。豈不以大河在前,三鎮在後,臨事好謀之審,不容不然歟。



 二帳。十二宮一府、五京,有兵一百六十四萬二千八百。



 宮丁、大首領、諸部族,中京、頭下等州,屬國之眾,皆丁與焉。不輕用之,所以長世。



 邊境戍兵又得《高麗大遼事跡》,載東境戍兵,以備高麗、女直等國,見其守國規模,布置簡要,舉一可知三邊矣。



 東京至鴨淥西北峰為界:黃龍府正兵五千。



 咸州正兵一千。



 東京沿女直界至鴨淥江:
 軍堡凡七十,各守軍二十人,計正兵一千四百。



 來運城宣義軍營八,太子營正兵三百。



 大營正兵六百。



 蒲州營正兵二百。新營正兵五百。



 加陀營正兵三百。



 王海城正兵三百。



 柳白營正兵四百。



 沃野營正兵一千。



 神虎軍城正兵一萬。大康十年置。



 右一府、一州、二城、七十堡、八營,計正兵二萬二千。



\end{pinyinscope}