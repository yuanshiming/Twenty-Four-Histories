\article{卷四十八志第十七下 百官志四 南面京官}

\begin{pinyinscope}

 遼有
 五京。上京為皇都,凡朝官、京官皆有之;餘四京隨宜設官,為制不一。大抵西京多邊防官,南京、中京多財賦官。



 五京並置者,列陳之;特置者,分列於後。



 三京宰相府職名總目:
 左相。



 右相。



 左平章政事。



 右平章政事。



 東京宰相府。聖宗統和元年,詔三京左右相、左右平章事。



 中京宰相府。



 南京宰相府。



 諸京內省客省職名總目:某京某省便。



 某京某省副使。耶律蒲奴,開泰末為上京內容省
 副使。



 上京內省司。



 東京內省司。《地理志》,東京大內不置宮嬪,唯以內省使、副、判官守之。



 五京諸使職名總目:某京某使。王棠,重熙中為上京鹽鐵使。



 知某京某使事。張孝傑,清寧間知戶部使事。



 某京某副使。劉伸,重熙中為三司副使。



 同知某京某使事。道宗大康三年見撻不也同知度支使事。



 某京某判官。聖宗太平九年見戶部使判官。



 上京鹽鐵使司。



 東京戶部使司。



 中京度支使司。



 南京三司使司。



 南京轉運使司。亦曰燕京轉運使司。



 西京計司。



 五京留守司兼府尹職名總目:某京留守行某府尹事。聖宗統和元年見上京留守、行臨潢尹事吳王稍。



 某京副留守。天祚天慶六年見東京副留走高清臣。



 知某京留守事。蕭惠,開泰二年知東京留守事。



 某府少尹。聖宗太平四年見臨潢少尹鄭弘節。



 同知某京留守事。太平八年見中京同知耶律野。



 同簽某京留守事。蕭滴冽,太平六年同簽南京留事。



 某京留守判官。室肪,天祿中為南京留守判官。



 某京留守推官。聖宗開泰元年見中京留守推官李可舉。



 上京留守司。



 東京留守司。



 中京留守司。太宗大同元年命趙延壽為中京留守,治鎮州。



 聖宗統和十二年命室功為中京留守,治大定府。南京留守司。太宗天顯三年升東平郡為南京,治遼陽。十三年以幽州為南京,治析津。聖宗開泰元年改幽都府為析津府。



 西京留守司。



 五京都總管府職名總目:
 某京都總管、知某府事。



 同知某府事。聖宗太平五年見同知中京事蕭堯袞。上京都總管府。



 東京都總管府。



 中京都部管府。



 南京都總管府。



 西京都總管府。



 五京都虞候司職名總目:都虞候。



 上京都虞候司。



 東京都虞候司。



 南京都虞候司。



 西京都虞候司。



 中京都虞候司。



 五京警巡院職名總目:某京警巡使。



 某京警巡副使。



 上京警巡院。



 東京警巡院。



 中京警巡院。



 南京警巡院。



 西京警巡院。



 五京處置使司職名總目:某京處置使。



 上京處置司。



 東京處置司。



 中京處置司。



 西京處置司。



 南京處置司。



 五京學職名總目:道宗清寧五年,詔設學養士,頒經及傳疏,置博士、助教各一員。



 博士。



 助教。



 上京學。上京別有國子監,見朝官。



 東京學。



 中京學。中京別有國子監,與朝官同;南京學。亦曰南京太學,太宗置。聖宗統和十三年,賜水磑莊一區。



 西京學。



 已上五京官。



 上京城隍使司。亦曰上京皇城使。



 上京城隍使。韓德讓,景宗時為上京皇城使。



 東京渤海承奉官。聖宗開泰八年耶律八哥奏,渤海承奉班宜設官以統之,因置。



 渤海丞奉都知押班。



 遼陽大都督府。太宗會同二年置。



 遼陽大都督。會同二年,都督曷魯泊等關防遼陽東都。



 東京安撫使司。



 東京安撫使。



 東京軍巡院。《地理志》,東京有歸化營軍千餘人,籍河朔亡命於此,置軍巡院。



 東京軍巡使。



 中京文思院。



 中京文思使。馬人望父佺為中京文思使。



 中京路按問使司。



 中京路按問使。耶律和尚,重熙二十四年為中京路按問使。



 中京巡邏使司。



 中京巡邏使。耶律古昱,開泰間為中京巡邏使。



 中京大內都部署司。



 中京大內都部署。聖宗開泰元年見中京大內都部署。



 中京大內副部署。



 南京宣徽院。



 南京宣徽使。道宗壽隆元年見宣徽使耶律特末。



 知南京宣徽院使事。



 知南京宣徽院事。



 南京宣徽副使。



 同知南京宣徽院事。



 南京處置使司。聖宗開泰元年見秦王隆慶為燕京管內處置使。



 燕京管內處置使。



 南京侍衛親軍馬步軍都指揮使司。



 南京侍衛親軍馬步軍都指揮使。蕭討古,乾亨初為南京侍衛親軍都指揮使。



 南京馬步副指揮使。



 南京侍衛親軍馬軍都指揮使司。



 南京馬軍都指揮使。



 南京馬軍副指揮使。南京侍衛親軍步軍都指揮便司。



 南京步軍都指揮使。



 南京步軍副指揮使。



 南京慄園司。



 典南京慄園。



 雲州宣諭招撫使司。



 雲州管內宣諭招撫使二員。統和四年見韓毗哥、邢抱樸為雲州管內置諭招撫使。



 南面大蕃府官
 黃龍府。



 知黃龍府事。興宗重熙十三年見知黃龍府事耶律歐里斯。



 同知黃龍府事。



 黃龍府判官。



 黃龍府侍衛親軍馬步軍都指揮使。



 黃龍府侍衛親軍都指揮使。



 黃龍府侍衛親軍副指揮使。



 黃龍府侍衛馬軍都指揮使。



 黃龍府侍衛步軍都指揮使。



 黃龍府侍衛馬軍副指揮使。



 黃龍府侍衛步軍副指揮使。



 黃龍府學。



 博士。



 助教。



 興中府。



 知興中府事。咸雍元年見知興中府事楊績。



 同知興中府事。



 興中府判官。興中府學。



 博士。



 助教。



 南面方州官遼東、西,燕、秦、漢、唐已置郡縣,設官職矣。高麗、渤海因之。至遼,五京列峙,包括燕、代,悉為畿甸。二百餘年,城郭相望,田野益闢。冠以節度,承以觀察、防禦、團練等使,分以刺史、縣令,大略採用唐制。其間宗室、外戚、大臣之家築城賜額,謂之「頭下州軍」;唯節度使朝廷命之,後往往皆歸王府。不能州者謂之軍,不能縣者謂之城,不能城者謂之堡。其設官則未詳云。



 節度使職名總目:某州某軍節度使。



 某州某軍節度副使。



 同知節度使事。耶律玦,重熙中同知遼興軍節度使事。



 行軍司馬。



 軍事判官。



 掌書記。劉伸,重熙五年為彰武軍節度使掌
 書記。衙官:某馬步軍都指揮使司。



 都指揮使。



 副指揮使。



 某馬軍指揮使司。



 指揮使。



 副指揮使。



 某步軍指揮使司。



 指揮使。



 副指揮使。上京道:
 懷州奉陵軍節度使司。



 慶州玄寧軍節度使司。



 泰州德昌軍節度使司。



 長春州韶陽軍節度使司。



 儀坤州啟聖軍節度使司。



 龍化州興國軍節度使司。



 饒州匡義軍節度使司。



 徽州宣德軍節度使司。



 成州長慶軍節度使司。



 懿州廣順軍節度使司。



 渭州高陽軍節度使司。



 鎮州建安軍節度使司。東京道:開州鎮國軍節度使司。



 保州宣義軍節度使司。



 辰州奉國軍節度使司。



 興州中興軍節度使司。



 海州南海軍節度使司。



 淥州鴨淥軍節度使司。



 顯州奉先軍節度使司。



 乾州廣德軍節度使司。



 貴德州寧遠軍節度使司。



 沈州昭德軍節度使司。



 遼州始平軍節度使司。



 通州安遠軍節度使司。



 雙州保安軍節度使司。



 同州鎮安軍節度使司。



 咸州安東軍節度使司。信州彰聖軍節度使司。



 賓州懷化軍節度使司。



 懿州寧昌軍節度使司。



 蘇州安復軍節度使司。



 復州懷德軍節度使司祥州瑞聖軍切度使司。中京道:成州興府軍節度使司。



 興中府彰武軍節度使司。



 宜州崇義軍節度使司。



 錦州臨海軍節度使司。



 川州長寧軍節度使司。



 建州保靜軍節度使司。



 來州歸德軍節度使司。



 南京道:幽州盧龍軍節度使司。



 平州遼興軍節度使司。西京道:雲中大同軍節度使司。



 雲內州開遠軍節度使司。



 奉聖州武定軍節度使司。



 蔚州忠順軍節度使司。



 應州彰國軍節度使司。



 朔州順義軍節度使司。



 觀察使職名總目:某州軍觀察使。



 某州軍觀察副使。



 某州軍觀察判官。王鼎,清寧五年為易州觀察判官。



 州學。博士。



 助教。
 中京道:高州觀察使司。



 武安州觀察使司。



 利州觀察使司。東京道:益州觀察使司。



 寧州觀察使司。



 歸州觀察使司。



 寧江州混同軍觀察使司。上京道:
 本州永昌軍觀察使司。



 靜州觀察使司。



 團練使司職名總目:某州團練使。



 某州團練副使。



 某州團練判官。



 州學。



 博士。



 助教。東京道:
 安州團練使。



 防禦使司職名總目:某州防禦使。



 某州防禦副使。



 某州防禦判官。



 州學。



 博士。



 助教。東京道:廣州防禦使司。
 鎮海府防禦使司。



 冀州防禦使司。



 衍州安廣軍防禦使司。



 州刺史職名總目:某州刺史。



 某州同知州事。耶律獨擷,重熙中同知金肅軍事。



 某州錄事參軍。世宗天祿五年,詔州錄事參軍委政事省差注。



 州學。



 博士。



 助教。



 上京道五州:烏、降聖、維、防、招。



 東京道三十七州:穆、賀、盧、鐵、崇、耀、嬪、遼西、康、宗、海北、巖、集、祺、遂、韓、銀、安遠、威、清、雍、湖、渤、郢、銅、涑、率賓、定理、鐵利、吉、麓、荊、勝、順化、連、肅、烏。



 中京道十三州:恩、惠、榆、澤、北安、潭、松山、安德、黔、嚴、隰、遷、潤。



 南京道八州:順、檀、涿、易、薊、景、灤、營、西京道八州:弘、德、寧邊、歸化、可汗、儒、武、東勝。



 縣職名總目:
 某縣令。



 某縣丞。



 某縣主簿。世宗天祿五年,詔縣主簿委政事備注。



 某縣尉。



 縣學。大公鼎為良鄉縣尹,建孔子廟。



 博士。助教。



 五京諸州屬縣,見《地理志》。縣有驛遞、馬牛、旗鼓、鄉正、廳隸、倉司等役。有破產不能給者,良民患之。馬人望設法,使民出錢免役,官自募人,倉司給使以公使充,人以為
 便。



 南面分司官平理庶獄,採摭民隱,漢、唐以來,賢主以為恤民之令典。



 官不常設,有詔,則選材望官為之。



 分決諸道滯獄使。聖宗統和九年,命邢抱樸等五員,又命馬守瑛等三員,分決諸道滯獄。



 按察諸道刑獄使。開泰五年遣劉涇等分路按察刑獄。



 採訪使。太宗會同三年命於骨鄰為採訪使。



 南面財賦官遼國以畜牧、田漁為稼穡,財賦之官,初甚簡易。自涅裡
 教耕織、而後鹽鐵諸利曰以滋殖,既得燕、代,益富饒矣。



 諸錢帛司職名總目:某州錢帛都點檢。大公鼎為長春州錢帛都提點。



 長春路錢帛司。興宗重熙二十二年置。



 遼西路錢帛司。



 平川路錢帛司。



 轉運司職名總目:某轉運使某轉運副使。



 同知某轉運使。



 某轉運判官。



 山西路都轉運使司。楊晰,興宗重熙二十年為山西轉運使。



 奉聖州轉運使司。聖宗開泰三年置。蔚州轉運使司。



 應州轉運使司。



 朔州轉運使司。



 保州轉運使司。已上並開泰三年置。



 西山轉運使。聖宗太平三年見西山轉運使郎玄化。



 南面軍官《
 傳》曰:雖楚有材,晉實用之。」遼自太祖以來,攻掠五代、宋境,得其人,則就用之,東、北二鄙,以農以工,有事則從軍政。計之善者也。



 點檢司職名總目:某都點檢。穆宗應歷十六年見殿前都點檢耶律夷刺葛。



 某副點檢。聖宗太平六年見副點檢耶律野。



 同知某都點檢。道宗清寧九年見同知點檢司事耶律撻不也。



 點檢司。



 殿前都點檢司。



 點檢侍衛親軍馬步司。



 諸指揮使司職名總目:某軍都指揮使。聖宗統和二年見侍衛親軍都指揮使韓倬。



 某軍副指揮使。



 某軍都監。



 某軍都指揮使司。



 某軍副指揮使司。



 並同前。



 侍衛親軍馬步軍都指揮使司。



 侍衛親軍馬軍都指揮使司。



 侍衛親軍步軍都指揮使司。侍衛控鶴兵馬都指揮使司。



 侍衛漢軍兵馬都指揮使司。



 四軍兵馬都指揮使司。



 歸聖軍兵馬都指揮使司。聖宗統和五年,以宋降軍置七指揮署,左右廂,凡四十二員。七年,隸總管府。



 歸聖軍左廂兵馬都指揮使司。



 歸聖軍右廂兵馬都指揮使司。



 第一左廂兵馬都指揮使司。



 第一右廂兵馬都指揮使司。



 第二左廂兵馬都指揮使司。



 第二右廂兵馬都指揮使司。



 第三左廂兵馬都指揮使司。



 第三右廂兵馬都指揮使司。



 第四左廂兵馬都指揮使司。



 第四右廂兵馬都指揮使司。



 第五左廂兵馬都指揮使司。



 第五右廂兵馬都指揮使司。



 第六左廂兵馬都指揮使司。



 第六右廂兵馬都指揮使司。



 第七左廂兵馬都指揮使司。



 第七右廂兵馬都指揮使司。



 宣力軍都指揮使司。



 四捷軍都指揮使司。



 天聖軍都指揮使司。



 漢軍都指揮使司。



 諸軍都團練使職名總目:某軍都團練使。趙思溫,太祖神冊二年為漢軍都
 團練使。某軍團練副使。



 某軍團練判官。



 漢軍都團練使司。



 諸軍兵馬都總管府職名總目:某兵馬都總管。聖宗太平四年見兵馬都總管。



 某兵馬副總管。



 同知某兵馬事。



 某兵馬判官。



 兵馬都總管府。



 歸聖軍兵馬都總管府。



 南面邊防官三皇、五帝寬柔之化,澤及漢、唐。好生惡殺,習與性成。



 雖五代極亂,習於戰鬥者才幾人耳。宋以文勝,然遼之邊防猶重於南面,直以其地大民眾故耳。卒之親仁善鄰,桴鼓不鳴幾二百年。此遼之所以為美也歟。



 易州飛狐招安使司。聖宗統和二十三年改安撫使司。



 易州飛狐兵馬司。道示咸雍四年改易州安撫司。



 易州飛狐招撫司。



 西南面招安使司。耶律合住,景宗保寧初為西南面招
 安使。



 巡檢使司。耶律合住,景宗保守中為巡檢使。



 五州都總管府。耶律速撒,穆宗應歷初為義、霸、詳、順、聖五州都總管。



 山後五州都管司。聖宗統和四年見蒲奴寧為山後五州都管。



 五州制置使司。聖宗開泰九年見霸,建、宜、泉、錦五州制置使。



 三州處置使司。韓德樞,太宗時內平、灤、營三州處置使。霸州處置使司。統和二十七年廢。



\end{pinyinscope}