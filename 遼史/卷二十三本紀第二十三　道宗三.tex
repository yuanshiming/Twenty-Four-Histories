\article{卷二十三本紀第二十三 道宗三}

\begin{pinyinscope}

 八年春正月癸未,烏古敵烈部詳穩耶律巢等奏克北邊捷。



 以戰多殺人,飯僧南京、中京。甲申,如魚兒濼。壬寅,昏霧連日。二月丙辰,北、南樞密院言無事可陳。壬戌,以討北部功,烏古敵烈部說穩耶律巢知北院大王事,都監蕭阿魯帶烏古敵烈部詳穩,加左監門衛上將軍。戊辰,歲饑,免武安州租稅,振恩,蔚、順、惠等州民。



 三月癸卯,
 有司奏春、泰、寧江三州三千人願為僧尼,受具足戒,許之。



 夏四月壬子,振義、饒二州民。丁巳,駐蹕塔裏舍。已卯,清暑拖古烈。



 王月壬午,晉王仁先薨。



 立月甲寫作,振易州貧民。乙未,振中京。甲子,振興中府。甲戌,封北府宰相楊績為趙王,樞密副使耶律觀參知政事兼知南院樞密使事。丁丑,高麗遣吏來貢。



 秋七月已卯,慶州靳文高八世同居,詔賜爵。丙申,振饒州饑民。丁酉,幸黑嶺。丁未,以御書《華嚴經五頌》出示群臣。



 閏月辛未,射熊於羖羊山。



 八月庚辰,混同郡王侯古薨,遣使致祭。



 九月甲子,駐蹕藉絲澱。



 冬十月己丑,參知政事耶律觀矯制營私第,
 降為庶人。癸巳,回鶻來貢。



 十一月庚戌,免祖州稅。丙辰,大雪,許民樵採禁地。丁卯,賜延昌宮貧戶錢。



 十二月戊辰,漢人行宮都部署耶律仲禧封韓國公,樞密副使、參知政事趙微出為武定軍節度使,樞密副使柴德滋參知政事。漢人行宮副部署耶律大悲奴升都部署,同知南院樞密使事蕭韓家奴知左夷離畢事。丁丑,以坤寧節,大赦。庚寅,賜高麗佛經一藏。



 九年春天月丁未,如雙濼。



 夏四月壬辰,如旺國崖。



 秋七月甲辰,獵大熊山。戊申,烏古敵烈統軍言,八石烈敵烈人殺其節度使以叛。己酉,詔隗烏古部軍分道擊之。丙
 寅,南京奏歸義、淶水兩縣蝗飛入宋境,餘為蜂所食。



 八月丙申,以耶律仲禧為南院樞密使。



 九癸卯,駐蹕獨盧金。



 冬十月,幸陰山,遂如西京。



 十一月戊午,詔行幸之地免租一年。甲子,南院大王合理只致仕。



 十二月辛未,以知北字樞密使事耶律宜新為中京留守,南院宣徽使耶撒刺為南院大王。壬辰,高麗,夏國並遣使來貢。



 十年春正月乙卯,如鴛鴦濼。



 二月癸未,蠲來州復業民租賦。戊子,陰卜來貢。三月甲子,如拖古烈,以耶律巢為北院大王。



 夏四月,旱。辛未,以奚人達魯三世同居,賜官旌之。



 五月丙寅,錄囚。



 六月戊辰,親出題試進土。壬申,詔
 臣庶言得失。丙子,御永定殿,策賢良。



 秋七月丙辰,如秋山。癸亥,謁慶陵。



 九月庚戌,幸東京。謁二儀、五鸞殿。癸亥,祠木葉山。



 冬十月丁卯,駐蹕藉絲澱。丁丑,詔有司頒行《史記》、《漢書》。



 十一月戊午,高麗遣使來貢。



 十二月辛巳,改明年為康,大赫。



 大康元年春正月乙未,如混同江。壬寅,振雲州饑。



 二月丁卯,祥州火,遣使恤災。乙酉,駐蹕大魚濼。丁亥,以獲鵝,加鷹坊使耶律楊六為工部尚書。



 三月乙巳,命皇太子寫佛書。



 夏四月丙子,振平州饑。乙酉,如犢山。



 閏月丙午,振平、灣二州饑。庚戌,皇孫延禧生。



 五月甲子,賜妃之親
 及東宮僚屬爵有差。



 六月癸巳,以興聖宮使奚謝家奴知奚六部大王事。戊戌,知三司使事韓操以錢穀增羨,授三司使。癸卯,遣使按問諸路囚。以惕隱大悲奴為始平軍節度使,參知政事柴德滋武定軍節度使。乙卯,葉蕃來貢。丙辰,詔皇太子總領朝政,仍戒諭之。



 以武定軍節度使趙徽為南府宰相,樞密副使楊遵勖參知政事。



 秋七月辛酉朔,獵平地松林。丙寅,振南京貧民。



 八月庚寅朔,日有食之。



 九月乙亥,駐蹕藉絲澱。已卯,以南京饑,免租稅一年,仍出錢粟振之。冬十月,西北路酋長遐搭、雛搭、雙古等來降。



 十一月辛酉,皇后被誣,賜死;殺伶人
 趙惟一、高長命,並籍其家屬。



 十二月乙丑朔,以南京統軍使耶律蕊奴為惕隱,漢人行宮都部署耶律霂樞密副使,同知東京留守事蕭鐸刺夷離畢。庚寅,賜張孝傑國姓。壬辰,以西京留守蕭燕說為左夷離畢。



 二年春正月已未,如春水。庚辰,駐蹕雙濼。



 二月戊子,振黃龍府饑。癸丑,南京路饑,免租稅一年。



 三月辛酉,皇太后崩。壬戌,遣殿前副點檢耶律轄古報哀於宋。癸亥,遣使報哀於高麗、夏國。丁卯,大赦。戊寅,以皇太后遺物遣使遣宋、夏。



 夏六月乙酉朔,上大行皇太后尊謚曰仁懿皇后。戊子,宋及高麗、夏國各遣使吊祭、甲午,葬仁懿皇后
 於慶陵。乙亥,駐蹕拖古烈。壬寅,出北院樞使魏王耶律乙辛為中京留守。丁未,冊皇后蕭氏,對其父祗候郎君鱉裏刺為趙王,叔西北路招討使餘里也遼西郡王,兄漢人行宮都部署、駙馬尉霞抹柳城郡王,參知政事楊遵勖知南院樞密使事,北院樞密副使蕭速撒知北院樞密使事,漢人宮副部署劉詵參知政事。己酉,南府宰相趙徽致仕。



 秋七月戊辰,如秋山。癸酉,柳城郡王霞抹薨。



 八月庚寅,獵,則麛失其母,憫之,不射。



 九月戊午,以南京蝗,免明年租稅。己卯,駐蹕藕絲澱。



 冬十月戊戌,召中京留守魏王耶律乙辛復為北院樞密使。



 十一月
 甲戌,上欲觀《起居注》,修注郎不擷及忽突堇等不進,各杖二百,罷之,流林牙蕭巖壽於烏隗部。是月,南京地震,民舍多壞。



 十二月乙丑,以左夷離畢蕭撻不也為南京統軍使。三年春正月癸丑,如混同江。乙卯,省諸道春貢金帛,及停周歲所輸尚方銀。



 二月壬午朔,東北路統軍使蕭稀家奴加尚父,封吳王。甲申,詔北院樞密使魏王耶律乙辛同母兄大奴、同母弟阿思世預北、南院樞密之選,其異母諸弟世預夷離堇之選。己丑,如魚兒濼。辛卯,中京饑,罷巡幸。



 夏四月乙酉,泛舟黑龍江。



 五月丙辰,玉田、安
 次蝝傷稼。癸亥,日中有黑子。己巳,駐蹕犢山。乙亥,北院樞密使耶律乙辛奏,右護衛太保查刺等告知北院樞密使事蕭速撒等八人謀立皇太子,上以無狀,不治,出速撒等三人補外,護衛撒撥等六人各鞭百餘,徙於邊。丙子,以西北路招討使遼西郡王蕭餘里也為北府宰相,兼知契丹行宮都部署事。戊寅,詔告謀逆事者,重加官賞。



 六月己卯朔,耶律乙辛令牌印郎君蕭訛都斡誣首嘗預速撒等謀,籍其姓名以告。即命乙辛及耶律仲禧、蕭餘里也、耶律孝傑、楊遵勖、燕哥、抄只、蕭十三等鞠治,杖皇太子,囚之宮中。辛巳,殺宿直官敵裏刺等三人。壬午,
 殺宣徽使撻不也等二人。癸未,殺始平軍節度使撒刺等十人,又遣使殺上京留守速撒,及已徒護衛撒撥等六人。乙酉,殺耶律撻不也及其弟陳留。丙戌,廢皇太子為庶人,囚之上京。己丑,回鶻來貢。



 殺東京留守同知耶律回裡不。辛卯,殺速撒等諸子,籍其家。



 戊申,遣使按五京諸道獄。



 秋七月辛亥,護衛太保查刺加鎮國大將軍,預突呂不部節度使之選,室韋查刺及蕭寶神奴、謀魯古並加左衛大將軍,牌印郎君訛都斡尚皇女趙國公主,授附馬都尉、始平軍節度使,抵候郎君耶律撻不也及蕭圖古辭並加監門衛上將軍。壬子,知北院樞密副
 使蕭韓家奴為漢人行宮都部署。乙丑,如秋山。丁丑,渴慶陵。



 八月庚寅,漢人行宮都部署蕭韓家奴薨。辛丑,謁慶陵。



 九月癸亥,玉田貢嘉禾。壬申,修乾陵廟。



 冬十月辛丑,駐蹕藕絲澱。



 十一月,北院樞密使耶律乙辛遣其私人盜殺庶人浚於上京。



 閏十二月戊午,以北府宰相遼西郡王蕭餘里也知北院樞密使事,左夷離畢耶律燕哥為契丹行宮都部署。丙寅,預行正旦禮。



 是歲,南京大熟。



 四年春正月庚辰,如春水。甲午,振東京饑。



 二月乙丑,駐蹕掃獲野。戊辰,以東路統軍使耶律王九為惕隱。夏四
 月辛亥,高麗遣使乞賜鴨綠江以東地,不許。



 五月丙戌,駐蹕散水原。



 六月甲寅,阻卜諸酋長進良馬。



 秋七月甲戌,諸路奏飯僧尼三十六萬。



 八月癸卯,詔有司決滯獄。



 九月乙未,駐蹕藕絲澱。庚子,五國部長來貢。



 冬十月癸卯,以參知政事劉伸為保靜軍節度使。



 十一月丁亥,禁士庶服用錦綺、日月、山龍之文。己丑,回鶻遣使來貢。庚寅,南院樞密使耶律仲禧為廣德軍節度使。



 辛卯,錦州民張寶四世同居,命諸子三班院祗候。



 十二月丁卯,以北院樞密副使耶律霂知北院樞密使事。



\end{pinyinscope}