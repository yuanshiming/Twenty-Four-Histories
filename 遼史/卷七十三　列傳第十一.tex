\article{卷七十三 列傳第十一}

\begin{pinyinscope}

 耶律室魯歐里思王繼忠蕭孝忠陳昭袞蕭合卓耶律室備,字乙辛隱,六院部人。魁岸,美容儀。聖宗同年生,帝愛之。甫冠,補祗候郎君。未幾,為宿直官。



 及出師伐宋,為隊帥,從南府宰相耶律奴瓜、統軍使蕭撻覽略地趙、魏,有功,加檢校太師,為北院大王。攻拔通利軍。



 宋和
 議成,特進門下平章事,賜推誠竭節保義功臣。



 以本部俸羊多闕,部人空乏,請和贏老之羊及皮毛,歲易南中絹,彼此利之。拜北院樞密使,封翰王。自韓德讓知北院,職多廢曠,室魯拜命之日,朝野相慶。



 從上獵松林,至沙嶺卒,年四十四,贈守司徒、政事令。



 二子:十神奴、歐里斯。十神奴,南院大王。



 歐里思,字留隱,少有大志。未冠,補只候郎君。



 開泰初,為本部司徒。秩滿閑居,徵為郎君班詳穩。遷右皮室詳穩,將本部兵,從東平王蕭排押伐高麗,至茶、陀二河,戰不利。歐里思獨全軍還,帝嘉賞。終西南面招討使。



 王繼忠,不知何郡人。仕宋為鄆州刺史、殿前都虞候。



 統和二十一年,宋遣繼忠屯定之望都以輕騎覘我軍,遇南府宰相耶律奴瓜等,獲之。太后知其賢,授戶部使,以康默記族女女之。繼忠亦自激昂,事必盡力。宋以繼忠先朝舊臣,每遣使,必有附賜,聖宗許受之。



 二十二年,宋使來聘,遣繼忠弧矢、鞭策及求和札子,有曰:「自臨大位,愛養黎元。豈欲窮兵,惟思息戰。每敕邊事,嚴諭守臣。至於北界人民,不令小有侵擾,眾所具悉,爾亦備知。向以知雄州何承矩已布此懇,自後杳無所聞。汝可密言,如許通和,即當別使往請。」詔繼忠與宋使相見,仍許講和。



 以繼忠家無奴隸,賜宮戶三十,加左武衛上將軍,攝中京留守。



 開泰五年,為漢人行宮都部署,封瑯邪郡王。六年,進楚王,賜國姓。上嘗燕飲,議以蕭合卓為北院樞密使,繼忠曰:「合卓雖有刀筆才,暗於大體。蕭敵烈才行兼備,可任。」上不納,竟用合卓。及遣合卓伐高麗,繼忠為行軍副部署,攻興化鎮,月餘不下。師還,上謂明於知人,拜樞密使。



 太平三年致仕,卒。子懷玉,仕至防禦使。



 蕭孝忠,字撒板,小字圖古斯,志糠慨。開泰中,補祗候郎君,尚越國公主,拜駙馬都尉,累遷殿前都點檢。太平中,擢北府宰相。



 重熙七年,為東京留守。時禁渤海人擊球,
 孝忠言:「東京最為重鎮,無從禽之地,若非球馬,何以習武?且天子以四海為家,何分彼此?宜弛其禁。」從之。



 十二年,入朝,封楚王,拜北院樞密使。國制,以契丹、漢人分北、南院樞密治之,孝忠奏曰:「一國二樞密,風俗所以不同。若並為一,天下幸甚。」事未及行,薨。追封楚國王。



 帝素服哭臨,赦死囚數人,為孝忠薦福。葬日,親臨,賜官戶守家。子阿速,終南院樞密使。



 陳昭袞,小字王九,雲州人。工譯鞮,勇而善射。統和中,補祗候郎君,為奚拽刺詳穩,累遷敦睦宮保,兼掌圍場事。



 開泰五年秋,大獵,帝射虎,以馬馳太速,矢不及發。虎
 怒,奮勢將犯蹕。左右闢易,昭袞舍馬,捉虎兩耳騎之。虎駭,且逸。上命衛士追射,昭袞大呼止之。虎雖鐵山,昭袞終不墮地。伺便,拔佩刀弒之。輦至上前,慰勞良久。即日設燕,悉以席上金銀器賜之,特加節鉞,遷圍場都太師,賜國姓,命張儉、呂德懋賦以美之。



 遷歸義軍節度使,同知上京留守,歷西甫面招討都監,卒。



 蕭合卓,字合魯隱,突呂不部人。始為本部吏。統和初,以謹恪,補南院侍郎。十八年,北院樞密使韓德讓舉合卓為中丞,以太后遺物使宋。還,遷北院樞密副使。開泰三年,為左夷離畢。



 合卓久居近職,明習典故,善占對。以是
 尤被寵渥,升北院樞密使。時議以為無完行,不可大用;南院樞密使王繼忠侍宴,又諷其短。帝頗不悅。六年,遣合卓伐高麗,還,時求進者多附之;然其服食、僕馬不加於舊。帝知其廉,以族屬女妻其子,詔許親友饋獻,豪貴奔趨於門。



 太平五年,有疾,帝欲臨視,合卓辭曰:「臣無狀,猥蒙重任。今形容毀瘁,恐陛下見而動心。」帝從之。會北府宰相蕭樸問疾,合卓執其手曰:「吾死,君必為樞密使,慎勿舉勝己者。」樸出而鄙之。是日卒。子烏古,終本部節度使。



 論曰:「統和諸臣,名昭王室者多矣。室魯拜樞密使,朝野
 相慶,必有得民心者。繼忠既不能死國,雖通南北之和,有知人之鑒,奚足尚哉!孝忠、昭袞,皆有可稱者。合卓臨終,教蕭樸毋舉勝己者任樞密,其誤國之罪大矣!」



\end{pinyinscope}