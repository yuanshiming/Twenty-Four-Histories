\article{卷四十七志第十七上 百官志三 南面朝官}

\begin{pinyinscope}

 契丹國自唐太宗置都督、刺史,武後加以五封,玄宗置經略使,始有唐官爵矣。其後習聞河北藩鎮受唐官名,於是太師、太保、司徒、司空施於部族。太祖因之。大同元年,世宗始置北院樞密使。明年,世宗以高勛為南院樞密。則樞密之設,蓋自太宗入汴始矣。天祿四年,建政事
 省。於是南面官僚可得而書。



 其始,漢人樞密院兼尚書省,吏、兵刑有承旨,戶、工有主事,中書省兼禮部,別有戶部使司。以營州之地加幽、冀之年,用是適足矣。



 中葉彌文,耶律楊六為太傅,知有三師矣。忽古質為太尉,知有三公矣。於斡古得為常侍,劉涇為禮部尚書,知有門下、尚書省矣。庫部、虞部、倉部員外出使,則知備郎官列宿之員。



 室昉監修,則知國史有院。程翥舍人,則知起居有注。邢抱樸承旨,五言敷學士,則知有翰林內制。張干政事舍人,則知有中書外制。大理、司農有卿,國子、少府有監,九卿、列監見矣。金吾、千牛有大將,十六列互見矣。太
 子上有師保,下有府率,東宮備官也。節度、觀察、防禦、團練、刺史,咸在方州,如唐制也。



 凡唐官可考見者,列具於篇;無徵者不書。



 南面朝官遼有北面朝官矣,既得燕、代十有六州,乃用唐制,復設南面三省、六部、臺、院、寺、監、諸衛、東宮之官。誠有志帝王之盛制,亦以招徠中國之人也。



 三師府。本名三公,漢以巫相、太尉、御史大夫為三公,故稱三師。



 太師。穆宗應歷三年見太師唐骨德。



 太傅。太宗會同元年命馮通守太傅。



 太保。會同元年劉邈守太保。



 少師。《耶律資忠傳》見少師蕭把哥。



 少傅。



 少保。



 掌印。耶律乙辛,重熙中掌太保印。



 三公府。先漢丞相、太尉、御史大夫,後漢更名大司徒、大司馬、大司空,唐太尉、司徒、司空,又名三司。



 太尉。太宗天顯十一年見太尉趙思溫。



 司徒。世宗天祿元年見司徒劃設。



 司空。聖宗統和三十年見司空邢抱質。



 漢人樞密院。本兵部之職,在周為大司馬,漢為太尉。唐季宦官用事,內置樞密院,後改用士人。晉天福中廢,開遠元年復置。太祖初有漢兒司,韓知古總知漢兒司事。太宗入汴,因晉置樞密院,掌漢人兵馬之政,初兼尚書省。



 樞密使。太宗大同元年見樞密使李崧。知樞密使事。



 知樞密院事。



 樞密副使。楊遵勖,咸雍中為樞密副使。



 同知樞密院事。聖宗太平六年見同知樞密院事
 耶律迷離己。



 知樞密院副使事。楊晰,興宗重熙十二年知樞密院副使事。



 樞密直學士。聖宗統和二年見樞密直學士郭嘏。



 樞密都承旨。聖宗開泰九年見樞密都承旨韓紹芳。



 樞密副承旨。楊遵勛,重熙中為樞密副承旨。



 吏房承旨。



 兵刑房承旨。



 戶房主事。



 廳房即工部主事。



 中書省。初名政事省。太祖置官,世宗天祿四年建政事省,興宗得熙十三年改中書省。



 中書令。韓延徽,太祖時為政事令,韓知古,天顯初為中書令;會同五年又見政事令趙延壽。



 大丞相。太宗大同元年見大丞相趙延壽。



 左丞相,聖宗太平四年見左丞相張儉。



 右丞相。聖宗開泰元年見右丞相馬保忠。



 知中書省事。蕭孝友,興宗重熙十年知中書省事。



 中書侍郎。韓資讓,壽隆初為中書侍郎。



 同中書門下平章事。太祖加王鬱同政事門下平章事,太宗大同元年見平章事張礪。



 參知政事。聖宗統和十二年見參知政事邢抱樸。



 堂後宮。太平二年見堂後宮張克恭。



 主事。守當官。並見耶律儼《建官制度》。



 令史。耶律儼,道宗咸雍三年為中書省令史。



 中書舍人院。



 中書舍人。室昉,景宗保寧間為政事舍人;道宗咸雍三年見中書舍人馬鉉。



 右諫院。



 右諫議大夫。聖宗統和七年見諫議大夫馬得臣。



 右補闕右拾遺。劉景,穆宗應歷初為右拾遺。



 門下省。



 侍中。趙思忠,太宗會同中為侍中。



 常侍。興宗重熙十四年見常侍斡古得。



 散騎常侍。馬人望,天祚乾統中為左散騎常侍。



 給事中。聖宗統和二年見給事中郭嘏。



 門下侍郎。楊晰,清寧初為門下侍邯。



 起居舍人院。



 起居舍人。聖宗開泰五年見起居舍人程翥。



 知起居注。耶律敵烈,重熙末知起居注。



 起居郎。杜防,開泰中為起居郎。



 左諫院。



 左諫議大夫。



 左補闕。



 左拾遺。統和三年見左拾遺劉景。



 通事舍人院。



 通事舍人。統和七年見通事舍人李琬。



 符寶司。



 符寶郎。耶律玦,重熙初為符寶郎。東上閣門司。太宗會同元年置。



 東上閣門使。《韓延徽傳》見東上閣門使鄭延豐。



 東上閣門副便。



 西上閣門司。



 西上閣門使。統和二十一年見西上閣門使丁振。



 西上閣門副使。



 東頭承奉班。



 東頭承奉官。韓德讓,景宗時為東頭承奉官。



 西頭承奉班。



 西頭承奉官。



 通進司。



 左通過。



 右通過。耶律瑤質,景宗時為右通進。



 登聞鼓院。



 登聞鼓使。



 匭院。



 知匭院使。太平三年見知匭院事杜防。



 誥院。



 誥院給事。耶律鐸斡,重熙末為誥院給事。



 尚書省。太祖嘗置左右尚書。



 尚書令。蕭思溫,景宗保寧初為尚書令。



 左僕射。太祖初康默記為左尚書,三年見左僕射韓知古。



 右僕射。太宗會同元年見右僕射烈束。



 左丞。武白為尚書左丞。



 右丞。



 左司郎中。



 右司郎中。左司員外郎。



 右司員外郎。



 六部職名總目:某部。



 某部尚書。聖宗開泰元年見吏部尚書劉績。



 某部侍郎。王觀,興宗重熙中為兵部侍郎;李浣,穆宗朝累遷工部侍郎。



 某部郎中。劉輝,道宗大安末為禮部郎中。



 某部員外郎。開泰五年見禮部員外郎王景運。



 某部郎中。聖宗統和九年見虞部郎中崔。諸曹郎官未詳。



 御史臺。太宗會同元年置。



 御史大夫。會同九年見御史大夫耶律解里。



 御史中丞。



 侍御。重熙七年見南面侍御壯骨里。



 殿中司。



 殿中。聖宗開泰元年見殿中高可恆。



 殿中丞。



 尚舍局。見《遼朝雜禮》。



 奉御。



 尚乘局奉御。



 尚輦局奉御。



 尚食局奉御。



 尚衣局奉御。



 翰林院。掌天子文翰之事。



 翰林都林牙。興宗重熙十三年見翰林都林牙耶律庶成。



 南面林牙。耶律磨魯古,聖宗統和初為南面林牙。



 翰林學士。太宗大同元年見和凝為翰林學士。翰林學士承旨。《趙延壽傳》見翰林學士承旨張礪。



 翰林祭酒。韓德崇,景宗保寧初為翰林祭酒。



 知制誥。室昉。室昉,太宗入汴,詔知制誥。



 翰林畫院。



 翰林畫待詔。聖宗開泰七年見翰林畫待詔陳升。



 翰林醫官。天祚保大二年見提舉翰林醫官李奭國史院。



 監修國史。聖宗統和九年見監修國史室昉。



 史館學士。景宗保寧八年見史館學士。



 史館修撰。劉輝,大安末為史館修撰。



 修國史。耶律塊,重熙初修國史。



 宣政殿。



 宣政殿學士。穆宗應歷元年見宣政殿學士李浣。



 觀書殿。



 觀書殿學士。王鼎,壽隆初為觀書殿學士。



 昭文館。



 昭文館直學士。楊遵勖子晦為昭文館直學士。



 崇文館。



 崇文館大學士。韓延徽,太祖時為崇文館大學士。



 乾文閣。



 乾文閣學士。王觀,道宗咸雍五年為乾文閣學士。



 宣徽院。太宗會同元年置。



 宣徽使。



 知宣徽院事。馬得臣,統和初知宣徽院事。



 宣徽副使。



 同知宣徽使事。



 同知宣徽院事。



 內省。內省使。聖宗太平九年初見內省使
 內省副使。



 內藏庫。



 內藏庫提點。道宗清寧元年見內藏庫提點耶律烏骨。



 內侍省。



 黃門令。



 內謁者。



 內侍省押班。



 內侍左廂押班。



 內侍右廂押班。



 契丹、漢兒、渤海內侍都知。



 左丞宣使。



 右丞宣使。



 內庫。



 都提點內庫。



 尚衣庫。



 尚衣庫使。



 湯藥局。



 都提點、勾當湯藥。



 內侍省官,並見《王繼恩》、《趙
 安仁傳》。



 客省。太宗會同元年置。



 都客省。興宗重熙十年見都客省回鶻重哥。



 客省便。會同五年見客省使耶律化哥。



 左客省使。蕭護思,應歷初為左客省使。



 有客省使。



 客省副使。



 四方館。四方館使。高勛,太宗入汴為四方館使。



 四方館副使。道宗咸雍五年,詔四方館副使止以契丹人充。



 引進司。



 引進使。聖宗統和二十八年見引進使韓杞。



 點簽司。



 同簽點簽司事。興宗重熙六年見同簽點簽司事耶律圓寧。



 禮信司。



 勾當禮信司。興宗重熙七年見勾當禮信司骨欲。



 禮賓使司。



 禮賓使。大公鼎曾祖忠為禮賓使。



 寺官職名總目:某卿。興宗景福元年見崇祿卿李可封。



 某少卿。耶律儼子處貞為太常少卿。



 某丞。



 某主簿。



 太常寺。有博士、贊引、太祝、奉禮郎、協律郎。



 諸署職名總目:某署令。



 某署丞。



 太樂署。



 鼓吹署。



 法物庫。《遼朝雜禮》有法物庫所掌圖籍。



 法物庫使。



 法物庫副使。



 崇祿寺。本光祿寺,避太宗諱改。



 衛尉寺。



 宗正寺。職在大惕隱司。太僕寺。有乘黃署。



 大理寺。有提點大理寺,有大理正,聖宗統和十二年置。



 鴻臚寺。



 司農專。



 諸監職名總目:某太監。興宗景福元年見少府監馬憚。



 某少監。興宗重熙十七年見將作少監王企。



 某監丞。



 某監主簿。



 秘書監。有秘書郎,秘書郎正字。



 著作局。



 著作郎。



 著作佐郎。楊晰,聖宗太平十一年為著作佐
 郎。



 校書郎。楊桔,統和中為校書郎。



 正字。開泰元年見正字李萬。



 司天監。有太史令,有司歷,靈臺郎,挈壺正,五官正,丞,主簿,五官靈臺郎、保章正、司歷、監候、挈壺正、司辰,刻漏博士,典鐘,典鼓。



 國子監。上京國子監,太祖置。



 祭酒。



 司業。



 監丞。
 主簿。



 國子學。



 博士。武白為上京國子博士。



 助教。



 太府監。



 少府監。將作監。



 都水監。



 已上文官。



 諸衛職名總目:
 各衛。



 大將軍。聖宗開泰七年見皇子宗簡右衛大將軍。



 上將軍。王繼忠,統和二十二年加左武衛上將軍。



 將軍。聖宗太平四年見千牛衛將軍蕭順。



 折沖都尉。



 果毅都尉。



 親衛。



 勛衛。



 翊衛。



 左右衛。



 左右驍衛。



 左右武衛左右威衛。



 左右領軍衛。



 左右金吾衛。



 左右監門衛。



 左右千牛衛。



 左右羽林軍。



 左右龍虎軍。



 左右神武軍。



 左右神策軍。



 左右神威軍。



 已上武官。東宮三師府。凡東宮官多見《遼朝雜禮》。



 太子太師。太宗大同元年見太子太師李崧。



 太子太傅。世宗天祿五年見太子太傅趙瑩。



 太子太保。大同元年見太子太保趙瑩。



 太子少師。聖宗太平十一年見太子少師蕭從順。



 太子少傅。耶律合里,重熙中為太子少傅。



 太子少保。大同元年見太子少保馮玉。



 太子賓客院。



 太子賓客。



 太子詹事院。



 太子詹事。



 少詹事。



 詹事丞。



 詹事主簿。



 太子司直司。



 太子司直。



 左春坊。



 太子左庶子。



 太子中允。聖宗太平五年見太子中允馮若谷。



 太子司議郎。



 太子左諭德。



 太子左贊善大夫。



 文學館。



 祟文館學士。



 崇文館直學士。



 太子校書郎。聖宗太平五年見太子校書郎韓灤。



 司經局。
 太子洗馬。劉輝,大安末為太子洗馬。



 太子文學。



 太子校書郎。聖宗太平五年見太子校書郎張昱。



 太子正字。



 典設局。



 典設郎。



 宮門局。



 宮門郎。



 右春坊。



 太子右庶子。



 太子中舍人。



 太子舍人。



 太子右諭德。



 右贊善大夫。



 太子通事舍人。



 太子家令寺。



 太子家令。



 丞。



 主薄。



 太子率更寺。



 太子率更令。



 丞。



 主簿。



 太子僕寺。



 太子僕。



 丞。



 主簿。太子率府職名總目:某率。興宗重熙十四年見率府率習羅。



 太子左右衛率府。



 太子左右司禦率府。



 太子左右清道率府。



 太子左右監門率府。



 太子左右內率府。



 已上東宮官。



 王傅府。



 王傅。蕭惟信,重熙十五年為燕趙王傅。



 親王內史府。



 內史。道宗大康三年見內史吳家奴。



 長史。



 參軍。



 諸王文學館。



 諸王教授。姚景行,重熙中為燕趙國王教授。



 諸王伴讀。聖宗太平八年,長沙郡正宗允等奏選諸王伴讀。



 已上諸王府官。



 南面宮官漢兒行宮都部署院。亦曰南面行宮都部署司。聖宗開泰九年改左僕射。



 漢兒行宮都部署。開泰七年見漢兒行宮都部署
 石用中。



 漢兒行宮副部署。興宗重熙十五年見漢兒行宮副部署耶律敵烈。知南面諸行宮副部署。重熙十年見知南面諸行宮副部署耶律褭里。



 同知漢兒行宮都部署事。道宗大康三年見同知漢兒行宮都部署事蕭撻不也。



 同簽部署司事。耶律儼,大康中為同簽部署司事。



 都部署判官。耶律儼,咸雍中為都部署判官。



 十二宮南面行宮都部署司職名總目:
 某宮漢人行宮都部署。



 某官南面副都部署。



 某宮同知漢人都部署。



 弘義宮。



 永興宮。



 積慶宮。



 長寧宮。



 延昌宮。



 彰愍宮。



 崇德宮。



 興聖宮。



 延慶宮。



 太和宮。



 永昌宮。



 敦睦宮。



\end{pinyinscope}