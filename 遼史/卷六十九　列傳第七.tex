\article{卷六十九 列傳第七}

\begin{pinyinscope}

 耶律屋質耶律吼何魯不耶律安搏耶律窪耶律頹昱耶律撻烈耶律屋質,字敵輦,系出孟父房。姿簡靜,有器識,重然諾。遇事造次,處之從容,人莫能測。博學,知天文。



 會同間,為惕隱。太宗崩,諸大臣立世宗,太后聞之,怒甚,遣皇子李胡以兵逆擊,遇安端、劉哥等於泰德泉,敗歸。



 李胡盡執
 世宗臣僚家屬,謂守者曰:「我戰不克,先殪此曹!」



 人皆恟恟相謂曰:「若果戰,則是父子兄弟相夷矣!」軍次潢河橫渡,隔岸相拒。



 時屋質從太后,世宗以屋質善籌,欲行間,乃設事奉書,以試太后。太后得書,以示屋質。屋質讀竟,言曰:「太后佐太祖定天下,故臣願竭死力。若太后見疑,臣雖欲盡忠,得乎?



 為今之計,莫若以言和解,事必有成;否即宜速戰,以決勝負。然人心一搖,國禍不淺,惟太后裁察。」太后曰:「我若疑卿,安肯以書示汝?」屋質對曰:「李胡、永康王皆太祖子孫,神器非移他族,何不可之有?太后宜思長策,與永康王和議。」太后曰:「誰可遣者?」對曰:「太后
 不疑臣,臣請往。萬一永康王見聽,廟社之福。」太后乃遣屋質授書於帝。



 帝遣宣徽使耶律海思復書,辭多不遜。屋質諫曰:「書意如此,國家之優未艾也。能釋怨以安社稷,則臣以為莫若和好。」



 帝曰:「彼眾烏合,安能敵我?」屋質曰:「即不敵,奈骨肉何!況未知孰勝?借曰幸勝,諸臣之族執於李胡者無噍類矣。以此計之,惟和為善。」左右聞者失色。帝良久,問曰:「若何而和?」屋質對曰:「與太后相見,各紓忿恚,和之不難;不然,決戰非晚。」帝然之,遂遣海思詣太后約和。往返數日,議乃定。



 始相見,怨言交讓,殊無和意。太后謂屋質曰:「汝當為我畫之。」屋質進曰:「太后與大
 王若能釋怨、臣乃敢進說。」



 太后曰:「汝第言之。」屋質借謁者籌執之,謂太后曰:「昔人皇正在,何故立嗣聖?」太后曰:「立嗣聖者,太祖遺旨。」



 又曰:「大王何故擅立,不稟尊親?」帝曰:「人皇王當立而不立,所以去之。」屋質正色曰:「人皇王舍父母之國而奔唐,子道當如是耶?大王見太后,不少遜謝,惟怨是尋。太后牽於偏愛,托先帝遺命,妄授神器。如此何敢望和,當速交戰!」



 擲籌而退。太后泣曰:「向太祖遭諸弟亂,天下茶毒,瘡痍末復,庸可再乎!」乃索籌一。帝曰:「父不為而子為,又誰咎也。」亦取籌而執。左右感激,大慟。



 太后復謂屋質曰:「議既定,神器竟誰歸?」屋質曰:「太后
 若授水康王,順天合一,復何疑?」李胡厲聲曰:「我在,兀欲安得立!」屋質曰:「禮有世嫡,不傳諸弟。昔嗣聖之立,尚以為非,況公暴戾殘忍,人多怨言。萬口一辭,願立永康王,不可奪也。」太后顧李胡曰:「汝亦聞此言乎?汝實自為之!」



 乃許立永康。



 帝謂屋質曰:「汝與朕屬尤近,何反助太后?」屋質對曰:「臣以社稷至重,不可輕付,故如是耳。」上喜其忠。



 天祿二年,耶律天德、蕭翰謀反下獄,惕隱劉哥及其弟盆都結天德等為亂。耶律石刺潛告屋質,屋質遽引入見,白其事。



 劉哥等不服,事遂寢。未幾,劉哥邀駕觀樗蒲,捧觴上壽,袖刃而進。帝覺,命執之,親詰其事、劉哥自
 誓,帝復不問。屋質奏曰:「當使劉哥與石刺對狀,不可輒恕。」帝曰:「卿為朕鞫之。」屋質率劍士往訊之,天德等伏罪,誅天德,杖翰,遷劉哥,以盆都使轄戛斯國。



 三年,表列泰寧王察割陰謀事,上不聽。五年,為右皮室詳穩。秋,上祭讓國皇帝於行宮,與群臣皆醉,察割弒帝。屋質聞有言「衣紫者不可失」,乃易衣而出,亟遣人召諸王,及喻禁衛長皮室等同力討賊。時壽安王歸帳,屋質遣弟沖迎之。



 王至,尚猶豫。屋質曰:「大王嗣聖子,賊若得之,必不容。



 群臣將誰事,社稷將誰賴?萬一落賊手,悔將何及?」王始悟。



 諸將聞屋質出,相繼而至。遲明整兵,出賊不意,圍之,遂
 誅察割。亂既平,穆宗即位,謂屋質曰:「朕之性命,實出卿手;」



 命知國事,以逆黨財產盡賜之,屋質固辭。應歷五年,為北院大王,總山西事。



 保寧初,宋圍太原,以屋質率兵往援,至白馬嶺,遣勁卒夜出間道,疾馳駐太原西,鳴鼓舉火。宋兵以為大軍至,懼而宵遁。以功加於越。四年,漢劉繼元遣使來貢,致幣於屋質,屋質以聞,帝命受之。五年五月薨,年五十七。帝痛悼,輟朝三日。後道宗詔上京立祠祭享,樹碑以紀其功云。



 耶律吼,字葛魯,六院部夷離堇蒲古只之後。端愨好施,不事生產。太宗特加倚任。會同六年,為南院大王,蒞事
 清簡,人不敢以年少易之。



 時晉主石重貴表不稱臣,辭多踞慢,吼言晉罪不可不伐。及帝親征,以所部兵從。既入汴,諸將皆取內幣珍異,吼獨取馬鎧,帝嘉之。



 及帝崩於欒城,無遺詔,軍中憂懼不知所為。吼詣北院大王耶律窪議曰:「天位不可一日曠。若請於太后,則必屬李胡。



 李胡暴戾殘忍,詎能子民。必欲厭人望,則當立永康王。」窪然之。會耶律安搏來,意與吼合,遂定議立永康王,是為世宗。



 頃之,以功加採訪使,賜以寶貨。吼辭曰:「臣位已高,敢復求富!臣從弟的琭諸子坐事籍沒,陛下哀而出之,則臣受賜多矣!」上曰:「吼舍重賞,以族人為請,其賢遠
 甚。」許之,仍賜宮戶五十。時有取當世名流作《七賢傳》者,吼與其一。天祿三年卒,年三十九。子何魯不。



 何魯不,字斜寧,嘗與耶律屋質平察割亂。穆宗以其父吼首議立世宗,故不顯用。晚年為本族敞史。



 及景宗即位,以平察割功,授昭德軍節度使,為北院大王。



 時黃龍府軍將燕頗弒守臣以叛,何魯不討之,破於鴨綠江。坐不親追擊,以至失賊,杖之。乾亨間卒。



 耶律安搏,曾祖巖木,玄祖之長子;祖楚不備,為本部夷離堇。父迭里,幼多疾,時太祖為撻馬獨沙裏,常加撫育。神冊六年,為惕隱,從太祖將龍軍討阻卜、黨項有功。天
 贊三年,為南院夷離堇,徵渤海,攻忽汗城,俘斬甚眾。太祖崩,淳欽皇后稱制,欲以大元帥嗣位。迭里建言,帝位宜先嫡長;今東丹王赴朝,當立。由是忤旨。以黨附東丹王,詔下獄,訊鞫,加以炮烙。不伏,拭之,籍其家。安搏自幼若成人,居父喪,哀毀過禮,見者傷之。太宗屢加慰諭,嘗曰:「此兒必為令器。」既長,寡言笑,重然諾,動遵繩矩,事母至孝。以父死非罪,未葬,不預宴樂。世宗在藩邸,尤加憐恤,安搏密自結納。



 太宗伐晉還,至欒城崩,諸將欲立世宗,以李胡及壽安王在朝,猶豫未決。時安搏直宿衛,世宗密召問計。安搏曰:「大王聰安寬恕,人皇王之嫡長;先
 帝雖有壽安,天下屈意多在大王。今若不斷,後悔無及。」會有自京師來者,安搏詐以李胡死傳報軍中,皆以為信。於是安搏詣北、南二大王計之。北院大王窪聞而遽曰:「吾二人方議此事。先帝嘗欲以永康王為儲貳,今日之事有我輩在,孰敢不從!但恐不白太后而立,為國家啟釁。」安搏對曰:「大王既知先帝欲以永康王為儲副,況永康王賢明,人心樂附。今天下甫定,稍緩則大事去矣。若白太后,必立李胡。且李胡殘暴,行路共知,果嗣位,如社稷何?」南院大王吼曰:「此言是也。吾計決矣!」乃整軍,召諸將奉世宗即位於太宗柩前。



 帝立,以安搏為腹心,
 總知宿衛。是歲,約和於黃河橫渡。



 太后問安搏曰:「吾與汝有何隙?」安搏以父死為對,太后默然。及置北院樞密使,上命安搏為之,賜奴婢百口,寵任無比,事皆取決焉。然性太寬,事循茍簡,豪猾縱恣不能制。天祿末,察割兵犯御幄,又不能討,由是中外短之。



 穆宗即位,以立世宗之故,不復委用。應歷三年,或誣安搏與齊王罨撒葛謀亂,系獄薨。侄撒給,左皮室詳穩。



 耶律窪,字敵輦,隋國王釋魯孫,商院夷離堇綰思子。少有器識,人以公輔期之。



 太祖時,雖未官,常任以事。太宗即位,為惕隱。天顯末,帝援河東,窪為先鐸,敗張敬達軍
 於太原北。會同中,遷北院大王。及伐晉,復為先鐸,與梁漢璋戰於瀛州,敗之。



 太宗崩於欒城,南方州郡多叛,士馬困乏,軍中不知所為。



 窪與耶律吼定策立世宗,乃令諸將曰:「大行上賓,神器無主,永康王人皇王之嫡長,天人所屬,當立;有不從者,以軍法從事。」諸將皆曰:「諾。」世宗即位,賜宮戶五十,拜於越。



 卒,年五十四。



 耶律頹顯,字團寧,孟父楚國王之後。父末掇,嘗為夷離堇。



 頹晃性端直。會同中,領九石烈部,政濟寬猛。世宗即位,為惕隱。天祿三年,兼政事令,封漆水郡王。



 及穆宗立,以匡贊功,嘗許以本部大王。後將葬世宗,頹顯懇言於
 帝曰:「臣蒙先帝厚恩,未能報;幸及大葬,臣請陪位。」帝由是不悅,寢其議。薨。



 耶律撻烈,字涅魯袞,六院部郎君古直之後。沉厚多智,有任重才。年四十未仕。



 會同間,為邊部令穩。應歷初,升南院大王,均賦役,勸耕稼,部人化之,戶口豐殖。時周人侵漢,以撻烈都統西南道軍援之。周已下太原數城,漢人不敢戰。及聞撻烈兵至,周主遣郭從義、尚鈞等率精騎拒於忻口。撻烈擊敗之,獲其將史彥超,周軍遁歸,復所陷城邑,漢主詣撻烈謝。及漢主殂,宋師來伐,上命撻烈為行軍都統,發諸道兵救之。既出雁門,宋諜知而
 退。



 保寧元年,加兼政事令,致政。乾亨初,召之。上見須發皓然,精力猶健,問以政事,厚禮之。以疾薨,年七十九。撻烈凡用兵,賞罰信明,得士卒心。河東單弱,不為周、宋所並者,撻烈有力焉。在治所不修邊幅,百姓無稱,年穀屢稔。時耶律屋質居北院,撻烈居南院,俱有政跡,朝議以為「富民大王」云。



 贊曰:「立嗣以嫡,禮也。太宗崩,非安搏、吼、窪謀而克斷,策立世宗,非屋質直而能諫,杜太后之私,折李胡之暴,在成橫渡之約,則亂將誰定?四臣者,庶幾《春秋》首止之功
 哉。」



\end{pinyinscope}