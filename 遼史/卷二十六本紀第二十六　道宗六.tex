\article{卷二十六本紀第二十六 道宗六}

\begin{pinyinscope}

 壽隆元年春正月己亥,如混同江。庚戌,西南面招討司奏拔思母來侵,蕭阿魯帶等擊破之。乙卯,振奉聖州貧民。



 二月戊辰,賜左、右二皮室貧民錢。癸酉,高麗遣使來貢。



 乙亥,駐蹕魚兒濼。



 三月丙午,賜東北路貧民絹。



 夏四月丁卯,斡特刺奏討耶睹刮捷。乙亥,女直遣使來貢。



 庚寅,錄西北路有功將士。



 五月乙未朔,左夷離畢耶律吐
 朵為惕隱,南京宣徽使耶律特末為北院大王。癸卯,贈陣亡者官。丁巳,駐蹕特禮嶺。



 六月己巳,以知翼六部大王事回裡不為本部大王,權參知政事趙孝嚴為漢人行宮都部署,圍場都管撒八以討阻卜功,加鎮國大將軍。癸巳,阻卜長禿裏底及圖木葛來貢。



 秋七月庚子,阻卜長猛達斯等來貢。癸卯,獵沙嶺。癸丑,頗里八部來附,進方物。甲寅,斡特刺奏磨古斯捷。



 九月甲寅,祠木葉山。丙辰,詔西京炮人、弩人教西北路漢軍。冬十月甲子,駐蹕藕絲澱。甲戌,以北面林牙耶律大悲奴為右夷離畢。癸未,以參知正事王師儒為樞密副使,漢人行官都部
 署趙孝嚴參知政事。壬辰,錄討阻卜有功將士。



 十一月丙申,女直遣使進馬。己亥,以都統斡特刺為西北路招討使,封漆水郡王。甲辰,夏國進貝多葉佛經。庚申,高麗王昱疾,命其叔顒權知國事。



 十二月癸亥朔,以知北樞密使事耶律阿思為北院樞密使。



 是年,放進士陳衡甫等百三十人。



 二年春正月甲午,如春水。癸卯,西南面招討司討拔思母,破之。乙卯,駐蹕瑟尼思。辛酉,市牛給烏古、敵烈、隈烏古部貧民。



 二月癸亥,振達麻里別古部。



 夏四月己卯,振西北邊軍。



 六月辛酉,駐蹕撒里乃。



 秋七月甲午,阻卜來
 貢。丙午,獵赤山。



 八月乙丑,頗里八部進馬。



 九月丙午,徙烏古敵烈部於烏納水,以扼北邊之衛。



 冬十月戊辰,駐蹕藕絲澱。庚辰,高麗遣使來貢。



 十二月己未,斡特刺討梅裏急,破之。壬戌,南府宰相耶律鋒魯斡致仕。癸亥,蕭撻不也為北府宰相,耶律大悲奴殿前都點檢。乙亥,夏國獻宋俘。



 三年春正月丁亥,如春水。壬寅,烏古部節度使耶律陳家奴以功加尚書右僕射。癸卯,駐蹕雙山。



 二月丙辰朔,南京水,遣使振之。



 閏水丙午,阻卜長猛撒葛、粘八葛長禿骨撒、梅裏急長忽魯八等請復舊地,貢方物,從之。



 三月辛
 酉,燕國王延禧生子。癸亥,賜名撻魯。妃之父長哥遷左監門衛上將軍,仍賜官屬錢。是春,高麗王昱薨。



 夏四月,南府宰相趙廷睦出知興中府事,參知政事牛溫舒兼同知樞密院事。



 五月癸亥,斡特刺討阻卜破之。己巳,駐蹕撒里乃。



 六月甲申,詔罷諸路馳陽貢新。丙戌,詔每冬駐蹕之所,宰相以下構宅,毋役其民。辛丑,夏人來告宋城要地,遣使之宋,諭興夏和。庚戌,以契丹行宮都部署耶律吾也為南院大王。



 秋七月壬子朔,獵黑嶺。



 八月己亥,蒲盧毛朵部長率其民來歸。乙巳,彗星見西方。



 九月壬申,駐蹕藕絲澱。丁丑,以武定軍節度使梁援為漢人
 行宮都部署。戊寅,斡特刺奏討梅裏急捷。己卯,五國部長來貢。冬十月庚戌,以西北路招討使斡特刺為南府宰相。



 十一月乙卯,蒲盧毛朵部來貢。戊午,以安車召醫巫閭山僧志達。己未,以中京留守韓資讓知樞密院事,同知南院樞密使事蕭藥師奴知右夷離畢。丁丑,西北路統軍司奏討梅裏急捷。



 四年春正月壬子,如魚兒濼。己巳,徙阻卜等貧民於山前。



 辛未,宋遣使來饋錦綺。



 三月庚午,幸春州。丙子,有司奏黃河清。



 夏四月辛丑,以雨,罷獵。



 五月癸酉,那也奏北邊捷。甲戌,駐蹕撒里乃。



 六月戊寅朔,夏國為宋所攻,遣
 使求援。丁亥,以遼興軍節度使涅里為惕隱,前知惕隱事耶律郭三為南京統軍使。甲午,以參知政事牛溫舒兼知中京留守事。



 秋七月戊午,如黑嶺。



 冬十月乙亥朔,駐蹕藕絲澱。己卯,以南府宰相斡特刺兼契丹行宮都部署,以傅導燕國王延禧。十一月乙巳朔,知右夷離畢事蕭藥師奴、樞密直學士耶律儼使宋,颯與夏和。辛酉,夏復遣使求援。



 十二月壬辰,為燕國王延禧行再生禮,曲赦三百里內囚。



 五年春五月乙巳,如魚兒濼。己酉,詔夏國王李乾順伐拔思母等部。



 夏五月壬戌,藥師奴等使宋回,奏宋罷兵。
 癸亥,謁乾陵。



 戊辰,以南府宰相斡特刺兼西北路招討使,禁軍都統。己巳,駐蹕沿柳湖。



 六月甲申,以奚六部大王回離保為契丹行宮都部署,知右夷離畢事蕭藥師奴南面林牙,兼知契丹行宮都部署事。乙未,五國部長來朝。戊戌,阻卜來貢。己亥,以興聖宮使耶律郝家奴為右夷離畢。



 秋七月壬寅朔,惕德長禿的等來貢。辛亥,如大牢古山。



 閏九月丙子,駐蹕獨盧金。



 冬十月己亥朔,高麗王顒遣使乞封冊,丁巳,斡特刺奏討耶睹刮捷。丙寅,以同知南京留守事蕭得裏底知北院樞密使事。



 丁卯,宋遣郭知章、曹平來聘。戊辰,振遼州饑,仍免租賦一年。



 十一月甲戌,振南、北二亂。乙酉,夏國以宋罷兵,遣使來謝。十二月甲子,以參知政事趙孝嚴為漢人行宮都部署,漢人行宮都部署梁援為遼興軍節度使。



 六年春正月癸酉,南院大王耶律吾也薨。壬午,以太師致仕禿開起為奚六部大王。丁亥,如春水。辛卯,斡特刺執磨古斯來獻。丙申,詔問民疾苦。



 二月丁未,以烏古部節度使陳家奴為南院大王。乙酉,磔磨古斯於市。癸丑,出絹賜五院貧民。辛酉,宋遣使告宋主照殖,弟佶嗣位,即日遣使吊祭。



 三月甲申,弛朔州山林之禁。



 夏四月丁酉朔,日有食之。癸卯,如炭山。



 五月壬午,烏古部討茶扎
 刺,破之。乙酉,漢人行宮都部署趙孝嚴蔡。丙戌,駐蹕納葛濼。辛卯,宋遣使饋先帝遺物。



 乙未,以東京留守何魯掃古為惕隱,南院宣徽使蕭常哥為漢人行宮都部署。



 六月庚子,遣使賀宋主。辛丑,以有司案牌牘書宋帝「嗣位」為「登寶位」,詔奪宰相鄭顓以下官,出顓知興中府事。



 韓資讓為崇義軍節度使,御史中水韓君義為廣順軍節度使。癸丑,阻卜長來貢。戊午,遣使決五京滯獄。己未,以遼興軍節度使梁援為樞密副使。



 秋七月庚午,如沙嶺。壬申,耶睹刮諸部寇西北路。



 八月,斡特刺以兵擊敗之,使來獻捷。



 九月癸未,望祠木葉山。戊子,駐蹕藕絲澱。



 冬
 十月壬寅,以樞密副使王師儒監修國史。癸卯,五國諸部長來貢。甲寅,以平州饑,復其租賦一年。



 十一月壬申,以天德軍民田世榮三世同居,詔官之,令一子三班院祗候,丙子,召醫巫閭山僧志達設壇於內殿。戊子,夏國王李乾順遣使請尚公主。



 十二月乙未,女直遣使來貢。己亥,以知右夷離畢事郝家奴為北面林牙。辛亥,詔燕國王延禧擬注大將軍以下官。庚申,鐵驪來貢。宋遣使來謝。帝不豫。



 是歲,封高麗王顒長子俁為三韓國公。放進士康秉儉等八十七人。



 七年春正月壬戌朔,力疾御清風殿受百官及諸國使
 賀。是夜。白氣如練,自天而降。黑雲起於西北,疾飛有聲。北有青赤黑白氣,相雜而落。癸亥,如混同江。甲戌,上崩於行宮,年七十。遺詔燕國王延禧嗣位。



 六月庚子,上尊謚仁聖大教皇帝,廟號道宗。



 贊曰:道宗初即位,求直言,訪治道,勸農興學,救災恤患,粲然可觀。及夫謗訕之令既行,告計之賞日重。群邪並興,讒巧競進。賊及骨肉,皇基寢危。眾正淪胥,諸部反側。甲兵之用無寧歲矣。一歲而飯僧三十六萬,一日而祝發三千。徙勤小惠,茂計大本。尚足興論治哉!



\end{pinyinscope}