\article{卷十四本紀第十四 聖宗五}

\begin{pinyinscope}

 十
 六年春正月乙丑,如長濼。



 二月庚子,夏國遣使來貢。丙午,以監門衛上將軍耶律喜羅為中臺省左相。



 三月甲子,女直遣使來貢。乙亥,鼻骨德酋長來貢。



 夏四月癸卯,振崇德宮所隸州縣民之被水者。丁未,罷民輸官俸,給自內帑,己酉,祈雨。乙卯,如木葉山。



 五月甲子,祭白馬神。丁卯,祠木葉山,告來歲南伐。庚辰,鐵驪來貢。乙酉,還
 上京。婦人年逾九十者賜物。



 六月戊子朔,致奠於祖、懷二陵。是月,清暑炭山。



 秋七月丁巳朔,錄囚,聽政。



 八月丁亥朔,東幸。



 九月丁巳朔,駐蹕得勝口。



 冬十一月,遣使冊高麗國王誦。



 十二月丙戌朔,宋國王休哥薨,輟朝五日。進封皇弟恆王隆慶為梁國王、南京留守,鄭王隆祐為吳國王。



 是年,放進士楊又玄等二人。



 十七年春正月乙卯朔,如長春宮。夏四月,如炭山清暑。



 六月,兀惹烏昭慶來。



 秋七月,以伐宋詔諭諸道。



 九月庚辰朔,幸南京。己亥,南伐。癸卯,射鬼箭。北院樞密使魏王耶律斜軫薨,以韓德讓兼知北院樞密使事。



 冬十月癸
 酉,攻遂城,不克。遣蕭繼遠攻狼山鎮百砦,破之。次瀛州,與宋軍戰,擒其將庚昭裔、宋順,獲兵仗、器甲無算。進攻樂壽縣,拔之。次遂城,敵眾臨水以拒,縱騎兵突之,殺戮殆盡。



 是年,放進士初錫等四人及第。



 十八年春正月,還次南京,賞有功將士,罰不用命者。詔諸軍各還本道。



 二月,幸延芳澱。



 夏四月己未,駐蹕於清泉澱。



 五月丁酉,清暑炭山。



 六月,阻卜叛酞鶻碾之弟鐵刺不率部眾來附,鶻碾無所歸,遂降,詔誅之。



 秋七月,駐蹕於湯泉。



 九月乙亥朔,駐蹕黑河。



 冬十一月甲戌朔,授西平王李繼遷子德昭朔方軍節度使。



 十二月,回鶻來
 貢。



 是年,放進士南承保等三人及第。



 十九年春正月辛巳,以祗候郎君班詳穩觀音為奚六部大王。甲申,回鶻進梵僧名醫。



 三月乙亥,夏國遣李文貴來貢。乙酉,西南面招討司奏黨項捷。壬辰,皇后蕭氏以罪降為貴妃。賜大丞相韓德讓名德昌。



 夏四月乙巳,幸吳國王隆祐第視疾。丙午,問安皇太后。五月癸酉,清暑炭山。丙戌,冊蕭氏為齊天皇后。庚寅,以千拽刺詳穩耶律王奴為乙室大王。辛卯,以青牛白馬祭天地。



 六月乙巳,以所俘宋將庚昭裔為昭順軍節度使。戊午,夏國奏下宋恆、環、慶等三州,賜詔褒之。



 秋七月丙戌,以東京
 統軍使耶律奴瓜為南府宰相。



 八月庚戌,達盧骨部來貢。



 九月己巳朔,問安皇太后,戊子,駐蹕昌平。庚寅,西南面招討司奏討吐谷渾捷。辛卯,幸南京。



 冬十月乙亥,南伐。壬寅,次鹽溝。徙封吳國王隆祐為楚國王,留守京師。丁未,梁國王隆慶統先鋒軍以進。辛亥,射鬼箭。壬子,以青牛白馬祭天地。甲寅,遼軍與宋兵戰於遂城,敗之。庚申,以黑白羊祭天地。丙寅,次滿城,以泥淖班師。



 十一月庚午,射鬼箭。丙子,宋兵出淤口、益津關來侵,偵候謀窪、虞人招古擊敗之。己卯,觀漁儒門添。



 閏月己酉,鼻骨德來貢。己未,減關市稅。



 十二月庚辰,免南京、平州租稅。



 二十年春正月庚子,如延芳澱。癸丑,東方五色虹見。詔安撫西南面向化諸部。甲寅,夏國遣使貢馬、駝。辛酉,女直宰相夷離底來貢。



 二月丁丑,女直遣其子來朝。高麗遣使賀伐宋捷。



 三月甲寅,遣北府宰相蕭繼遠等南伐。壬戌,駐蹕鴛鴦濼。



 夏四月丙寅朔,文班太保達裏底敗宋兵於梁門。甲戌,南京統軍使蕭撻凜破宋軍於泰州。乙酉,南征將校獻俘,賜爵賞有差。戊子,鐵驪遣使來貢。



 五月乙卯,幸炭山清暑。



 六月,夏國遣劉仁勖來告下靈州。



 秋七月甲午朔,日有食之。丁酉,以邢抱樸為南院樞密使。辛丑,高遣遣使來貢本國《地裡圖》。



 九月癸巳朔,謁
 顯陵,告南伐捷。



 冬十月癸亥朔,至自顯陵。



 十二月,奚王府五帳六節度獻七金山土河川地,賜金幣。



 是歲,南京、平川麥秀兩歧。放進士邢祥等六人及第。



 二十一年春正月,如鴛鴦濼。



 三月壬辰,詔修《日歷》官毋書細事。甲午,朝皇太后。



 戊午,鐵驪來貢。



 夏四月乙丑,女直遣使來貢。戊辰,兀惹、渤海、奧裏米、越里篤、越里吉等五部遣使來貢。是月,耶律奴瓜、蕭撻凜獲宋將王繼忠於望都。



 五月庚寅朔,清暑炭山。丁巳,西平王李繼遷薨,其子德昭遣使來告。



 六月己卯,贈繼遷尚書令,遣西上閣門使丁振吊慰。辛巳,黨項來貢。乙酉,阻卜鐵刺里率
 諸部來降。是月,修可敦城。



 秋七月庚戌,阻卜、烏古來貢。甲寅,以奚王府監軍耶律室魯為南院大王。



 木月乙酉,阻卜鐵刺裏來朝。丙戌,朝皇太后。



 九月己亥,夏國李德昭遣使來謝吊贈。癸丑,幸文河湯泉,改其名曰松林。



 冬十月丁巳朔,駐蹕七渡河。戊辰,以楚國王隆祐為西南面招討使。



 十一月壬辰,故於越耶律休哥之子道士奴、高九等謀叛,伏誅。丙申,通括南院部民。



 十二月癸未,罷三京諸道貢。



 二十二年春正月丁亥,如鴛鴦濼。



 二月乙卯朔,女直遣使來貢。丙寅,商院樞密使邢抱樸薨,輟朝三日。



 三月己
 丑,罷番部賀千齡節及冬至、重五貢。乙未,西夏李德昭遣使上繼遷遺物。



 夏四月丁卯,朝皇太后。



 五月,清暑炭山。



 六月戊午,以可敦城為鎮州,軍曰建安。



 秋七月甲申,遣使封夏國李德昭為西平王。丁亥,兀惹、蒲奴里、剖阿里、越里篤、奧裏米等部來貢。



 八月丙辰,黨項來貢。庚申,阻卜酋鐵刺裏來朝。戊辰,鐵刺裏求婚,不許。丙子,駐蹕犬牙山。



 九月己丑,以南伐諭高麗。丙午,幸南京。女直遣使獻所獲烏昭慶妻子。丁未,致祭於太宗皇帝廟。以北院大王磨魯古、太尉老君奴監北、南王府兵。庚戌,命楚國王隆祐留寧京師。



 閏月己未,南伐。癸亥,次固安。以所
 獲諜者射鬼箭。甲子,以青牛白馬祭天地。丙寅,遼師與宋兵戰於唐興,大破之。



 丁卯,蕭撻凜與宋軍戰於遂城,敗之。庚午,軍於望都。



 冬十月乙酉,以黑白羊祭天地。丙戌,攻瀛州,不克。甲午,下祁州,齎降兵。以酒脯祭天地。己酉,西平王李德昭遣使謝封冊。



 十一月癸亥,馬軍都指揮使耶律課裡遇宋兵於詔州,擊退之。甲子,東京守蕭排押獲宋魏府官吏田逢吉、郭守榮、常顯、劉綽等以獻。丁卯,南院大王善補奏宋遣人遺王繼忠弓矢,密請求和。詔繼忠與便會,許和。庚午,攻破德清軍。壬申,次澶淵。蕭撻凜中伏弩死。乙亥,攻破通利軍。丁丑,宋遣崇儀
 副使曹利用請和,即遣飛龍使韓(木巳)持書報聘。



 十二月庚辰朔,日有食之,既。癸未,宋復遣曹利用來,以無還地之意,遣監門衛大將軍姚東之持書往報。戊子,宋遣李繼昌請和,以太后為叔母,願歲輸銀十萬兩,絹二十萬匹。



 許之,即遣閣門使丁振持書報聘。己丑,詔諸軍解嚴。是月,班師。皇太后賜大丞相齊王韓德昌姓耶律,徙王晉。



 是年,放進士李可封等三人。



 二十三年春正月戊午,還次南京。庚申,大饗將卒,爵賞有差。二月丙戌,復置榷場於振武軍。丁巳,夏國遣使告下宋青城。辛酉,朝皇太后。以惕隱化哥為南院大王,行軍
 都監老君奴為惕隱。乙丑,振黨項部。丁卯,回鶻來貢。丁丑,改易州飛狐招安使為安撫使。



 夏四月丙戌,女直及阿薩蘭回鶻各遣使來貢。乙未,鐵驪來貢。己亥,黨項來侵。



 五月戊申朔,宋遣孫僅等來賀皇太后生辰。乙卯,以金帛賜陣亡將士家。丙寅,高麗以與宋和,遣使來賀。



 六月壬辰,清暑炭山。甲午,阻卜酋鐵刺里遣使賀與宋和。



 己亥,達旦國九部遣使來聘。



 秋七月癸丑,問安皇太后。戊午,黨項來貢。辛酉,以青牛白馬祭天地。壬戌,烏古來貢。丁卯,女直遣使來貢。阿薩蘭回鶻遣使來請先留使者,皆遣之。



 九月甲戌,遣太尉阿里、太傅楊六賀宋主
 生辰。



 冬十月丙子朔,鼻骨德來貢。戊子,朝皇太后。甲午,駐蹕七渡河。癸卯,宋歲幣始至,後為常。



 十一月戊申,上遣太保合住、頒給使韓筍,太后遣太師盆奴、政事舍人高正使宋賀正旦。辛亥,觀漁桑乾河。丁巳,詔大承相耶律德昌出宮籍,屬於橫帳。



 十二月丙申,宋遣周漸等來賀千齡節。丁酉,復遣張若谷等來賀正旦。二十四年春正月,如鴛鴦濼。



 夏五月壬寅朔,幸炭山清暑。幽皇太妃胡輦於懷州,囚夫人夷懶於南京,餘黨皆生痤之。



 秋七月辛丑朔,南幸。



 八月丙戌,改南京宮宣教門為元和,外三門為南端,左掖門為萬春,右掖門為千
 秋。是月,沙州敦煌王曹壽遣使進大食國馬及美玉,以對衣、銀器等物賜之。



 九月,幸南京。



 冬十月庚午朔,帝率群臣上皇太后尊號曰睿德神略應運啟化承天皇太后,群臣上皇帝尊號曰至德廣孝昭聖天輔皇帝。大赦。



 是年,放進士楊佶等二十三人及第。



 二十五年春正月,建中京。



 二月,如鴛鴦濼。



 夏四月,清暑炭山。



 六月,賜皇太妃胡輦死於幽所。



 秋七月壬申,西平王李德昭母薨,遣使吊祭。甲戌,遣使起復。九月,西北路招討使蕭圖玉討阻卜,破之。



 冬十月丙申,駐蹕中京。



 十二月己酉,振饒州饑民。



 二十六年春二月,如長濼。



 夏四月辛卯朔,祠木葉山。



 五月庚申朔,還上京。丙寅,高麗進龍須草席。己巳,遣使賀中京成。庚午,致祭祖、懷二陵。辛未,駐蹕懷州。



 秋七月,增太祖、太宗、讓國皇帝、世宗謚,仍謚皇太弟李胡曰欽順皇帝。冬十月戊子朔,幸中京。



 十二月,蕭圖玉奏討甘州回鶻,降其王耶刺裏,撫慰而還。



 是年,放進士史克忠等一十三人。



 二十七年春正月,鉤魚土河。獵於瑞鹿原。



 夏四月丙戌朔,駐蹕中京,營建宮室。庚戌,廢霸州處置司。



 秋七月甲寅朔,霖雨,潢、土、斡刺、陰涼四河皆溢,漂沒民合。



 八月甲
 申,北幸。



 冬十一月壬子朔,行柴冊禮。



 十二月乙酉,南幸。皇太后不豫。戊子,肆赦。辛卯,皇太后崩於行宮。壬辰,遣使報哀於宋、夏、高麗。戊申,如中京。己酉,詔免賀千齡節。



 是歲,御前引試劉二宜等三人。



\end{pinyinscope}