\article{卷一百四 列傳第四十二 逆臣上}

\begin{pinyinscope}

 耶律轄底子迭
 裡特耶律察割耶律婁國耶律重元子涅魯古耶律滑哥《易》曰:「天尊地卑,乾坤定矣;卑高以陳,貴賤位矣。」



 貴賤位而後君臣之分定,君臣之分定而後天地和,天地和而後萬化成。五帝三王之治,用此道也。三代而降,臣弒其君者有之,子弒其父者有之。孔子作《春秋》以寓王法,誅死者於前,懼生者於後,其慮深遠矣。歐陽修作《唐書》,創《逆臣傳》,蓋亦《春秋》之意也。



 遼判逆之臣二十有二,跡其事
 則又有甚焉者,然豈一朝一夕之故哉。列於《傳》,所以公天下之貶,以示夫戒云。



 轄底,字涅烈兗,肅祖孫夷離堇怗刺之子。幼黠而辯,時險佞者多附之。



 遙輦痕德堇可汗時,異母史罨古只為迭刺部夷離堇。故事,為夷離堇者,得行再生禮。罨古只方就帳易服,轄底遂取紅袍、貂蟬冠,乘白馬而出。乃令黨人大呼曰:「夷離堇出矣!」眾皆羅拜,因行柴冊禮,自立為夷離堇。與於越耶律釋魯同知國政。及釋魯遇害,轄底懼人圖己,挈其二子迭裡特、朔刮奔渤海,偽為失明。後因球馬之會,與二子奪良馬奔歸國。益為奸惡,常以
 巧辭獲免。



 太祖將即位,讓轄度,轄底曰:「皇帝聖人,由天所命,臣豈敢當!」太祖命於於越。及自將代西南諸部,轄底誘刺葛等亂,不從者殺之。車駕還至赤水城,轄底懼,與刺葛俱北走,至榆河為追兵所獲。太祖問曰:「朕初即位,嘗以國讓,叔父辭之;今反欲立吾弟,何也?」轄底對曰:「始臣不知天子之貴,及陛下即位,衛從甚嚴,與凡庶不同。臣嘗奏事心動,始有窺覦之意。度陛下英琥,必不可取;諸弟懦弱,得則易圖也。



 事若成,豈容諸弟乎。」太祖謂諸弟曰:「汝輩乃從斯人之言耶!」迭刺曰:「謀大事者,須用如此人;事成,亦必去之。」



 轄底不復對。囚數月,縊殺之。



 將
 刑,太祖謂曰:「叔父罪當死,朕不敢赦。事有便國者,宜悉言之。」轄底曰:「迭刺部人眾勢強,故多為亂,宜分為二,以弱其勢。」子迭裡特。



 迭裡特,字海鄰。有膂力,善馳射,馬躓不僕。尤神於醫,視人疾,若隔紗睹物,莫不悉見。



 太祖在潛,已加眷遇,及即位,拜迭刺部夷離堇。太祖嘗思鹿醢解醒,以山林所有,問能取者。迭裡特曰:「臣能得之。」



 乘內廄馬逐鹿,射其一。欲復射,馬跌而斃。迭裡特躍而前,弓猶不弛,復獲其一。帝歡甚曰:「吾弟萬人敵!」會帝患心痛,召迭裡特視之。迭裡特曰:「膏肓有瘀血如彈丸,然藥不能及,必針而後愈。」
 帝從之。嘔心瘀血,痛止。



 帝以其親,每加賜賚;然知其為人,未嘗任以職。後從刺葛亂,與其父轄底俱縊殺之。



 察割,字歐辛,明王安端之子。善騎射,貌恭而心狡,人以為懦。太祖曰:「此兇頑,非儒也。」其父安端嘗使奏事,太祖謂近侍曰:「此子目若風駝,面有反相。朕若在,無令入門。



 世宗即位於鎮陽,字端聞之,欲持兩端。察割曰:「太弟忌刻,若果立,豈容我輩!永康王寬厚,且與劉哥相善,宜往與計。」安端即與劉哥謀歸世宗。及和議成,以功封泰寧王。



 會安端為西南面大詳穩,察割佯為父惡,陰遣人白於帝,即召之。既至上前,泣訴不勝哀,帝憫之,使領女
 石烈軍。出入禁中,數被恩遇。帝每出獵,察割托手疾,不操弓矢,但執練錘馳走。屢以家之細事聞於上,上以為誠。



 察割以諸族屬雜處,不克以逞,漸徙廬帳迫於行宮。右皮室詳穩耶律屋質察其奸邪,表列其狀。帝不信,以表示察割。



 察割稱屋質疾己,哽咽流涕。帝曰:「朕固知無此,何至泣耶!」



 察割時出怨言,屋質曰:「汝雖無是心,因我過疑汝,勿為非義可也。」他日屋質又請於帝,帝曰:「察割舍父事我,可保無他。」屋質曰:「察割於父既不孝,於君安能忠!」帝不納。



 天祿五年七月,帝幸太液谷,留飲三日,察割謀亂不果。



 帝伐周,至詳古山,太后與帝祭文獻皇帝
 於行宮,群臣皆醉。



 察割歸見壽安王,邀與語,王弗從。察割以謀告耶律盆都,盆都從之。是夕,同率兵入弒太后及帝,因僭位號。百官不從者,執其家屬。至夜,閱內府物,見碼碯碗,曰:「此希世寶,今為我有!」詫於其妻。妻曰:「壽安王、屋質在,吾屬無噍類,此物何益!」察割曰:「壽安年幼,屋質不過引數奴,詰旦來朝,固不足憂。」其黨矧斯報壽安、屋質以兵圍於外,察割尋遣人弒皇后於柩前,倉惶出陣。壽安遣人諭曰:「汝等既行弒逆,復將若何?」有夷離堇劃者委兵歸壽安王,餘眾望之,徐徐而往。察割知其不濟,乃擊群官家屬,執弓矢脅曰:「無過殺此曹爾!」叱令速
 出。時林牙耶律敵獵亦在擊中,進曰:「不有所廢,壽安王何以興。籍此為辭,猶可以免。」察割曰:「誠如公言,誠當使者?」敵獵請與罨撒葛同往說之,察割從其計。壽安王復令敵獵誘察割,臠殺之。諸子皆伏誅。



 婁國,字勉辛,文獻皇帝之子。天祿五年,遙授武定軍節度使。及察割作亂,穆宗與屋質從林牙敵獵計,誘而出之,婁國手刃察割。改南京留守。



 穆宗沉湎,不恤政事,婁國有覬覦之心,誘敵獵及群不逞謀逆。事覺,按問不服。帝曰:「朕為壽安王時,卿數以此事說我,今日豈有虛乎?」婁國不能對。及餘黨盡服,遂縊於可汗州西谷,詔有司
 擇絕後之地以葬。



 重元,小字孛吉只,聖宗次子。材勇絕人,眉目秀朗,寡言笑,人望而畏。



 太平三年,封秦國王。聖宗崩,欽哀皇后稱制,密謀立重元。重元以所謀白於上,上益重之,封為皇太弟。歷北院樞密使、南京留守、知元帥府事。重元處戎職,未嘗離輦下。先是契丹人犯法,例須漢人禁勘,受枉者多。重元奏請五京各置契丹警巡使,詔從之,賜以金券誓書。道宗即位,冊為皇太叔,免拜不名,為天下兵馬大元帥,復賜金券、四頂帽、二色袍,尊寵所未有。



 清寧九年,車駕獵濼水,以其子涅魯古素謀,與同黨陳國王陳
 六、知北院樞密事蕭胡睹等凡四百餘人,誘脅弩手軍陣於帷宮外。將戰,其黨多悔過效順,各自奔潰。重元既知失計,北走大漠,嘆曰:「涅魯古使我至此!」遂自殺。



 先是重元將舉兵,帳前雨赤如血,識者謂敗亡之兆。子涅魯古。涅魯古,小字耶魯綰,性陰狠。興宗一見,謂曰:「此子目有反相。」



 重熙十一年,封安定郡王。十七年,進王楚,為惕隱。清寧三年,出為武定軍節度使。七年,知南院樞密使事,說其父重元詐病,俟車駕臨問,因行弒逆。



 九年秋獵,帝用耶律良之計,遣人急召涅魯古。涅魯古以事洩,遽擁
 兵犯行宮。南院樞密使許王仁先等率宿衛士討之。



 涅魯古躍馬突出,為近侍詳穩渤海阿廝、護衛蘇射殺之。



 滑哥,字斯懶,隋國王釋魯之子。性陰險。初烝其父妾,懼事彰,與克蕭臺哂等共害其父,歸咎臺哂,滑哥獲免。



 太祖即位務廣恩施,雖知滑哥兇逆,姑示含忍,授以惕隱。



 六年,滑哥預諸遞之亂。事平,群臣議其罪,皆謂滑哥不可釋,於是與其子痕只俱陵遲而死,敕軍士恣取其產。帝曰:「滑哥不畏上天,反君弒父,其惡不可言。諸弟作亂,皆此人教之也。」



\end{pinyinscope}