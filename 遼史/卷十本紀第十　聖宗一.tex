\article{卷十本紀第十 聖宗一}

\begin{pinyinscope}

 聖宗文武大孝宣皇帝,諱隆緒,小字文殊奴。景宗皇帝長子,母曰睿智皇后蕭氏。帝幼喜書翰,十歲能詩。既長,精射法,曉音律,好繪畫。



 乾亨二年,封梁王。



 四年秋九月壬子,景宗崩。癸丑,即皇帝位於柩前,時年十二。皇后奉遣詔攝政,詔諭諸道。



 冬十月己未朔,帝始臨朝。辛酉,群臣上尊號曰昭聖皇帝,尊皇后為皇太后,大赦。以南院大
 王勃古哲總領山西諸州事,北院大王、於越休哥為南面行軍都統,奚王和朔奴副之,同政事門下平章事蕭道寧領本部軍駐南京。乙丑,如顯州。



 十一月甲午,置乾州。



 十二月戊午朔,耶律速撒討阻卜。辛酉,南京留守荊王道隱奏宋遣使獻犀帶請和,詔以無書卻之。甲子,撻刺干乃萬十醉言宮掖事,法當死,杖而釋之。辛未,西南面招討使秦王韓匡嗣薨。癸酉,奉大行皇帝宮于菆塗殿。庚辰,省置中臺省官。



 統和元年春正月戊午朔,以大行在殯,不受朝。乙丑,奉遺詔,召先帝庶兄質睦於菆塗殿前,復封寧王。加宰相
 室昉、宣徽使普領等恩。丙寅,荊王道隱有疾,詔遣使存問。是日,皇太后幸其邸視疾。戊辰,以烏隈烏骨里部療節度使耶律章瓦同政事門下平章事。甲戌,荊王道隱薨,輟朝三日,追封晉王,遣使撫慰其家。丙子,以於越休哥為南京留守,仍賜南面行營總管印綬,總邊事。渤海撻馬解里以受先帝厚恩,乞殉葬,詔不許,賜物以旌之。戊寅,遣使賜於越休哥及奚王籌寧、統軍使頗德等湯藥。命懇篤持送休哥不車膀,以諭燕民。辛巳,速撒獻阻僕俘。午,涿州刺史安吉奏宋築城河北,詔留守於越休哥撓之,勿令就功。趙妃及公主胡骨典、奚王籌寧、宰相
 安寧、北大王普奴宇、惕隱屈烈、吳王稍、宇王只沒與橫帳、國舅、契丹、漢官等並進助山陵費。癸未,齊國公主率內外命婦進物如之。甲申,西南面招討使直德威奏黨十五部侵邊,以兵擊破之。乙酉,以速撒破阻卜,下詔褒美;仍諭與大漢討黨項諸部。



 丁亥,樞密使兼政事令室昉以年老請解兼職,詔不允。



 二月戊子朔,禁所在官吏軍民不得無故聚眾私語及冒禁夜行,違者坐之。己丑,南京奏,聞宋多聚糧邊境及宋主將如臺山,詔休哥嚴為之備。甲午,葬景宗皇帝於乾陵,以近幸朗、掌飲伶人撻魯為殉。上與皇太后因為書附上大行。丙申,皇太
 后詣陵置奠,命繪近臣於御容殿,賜山陵工人物有差。庚子,以先帝遺物賜皇族及近臣。辛丑,南京統軍使耶律善補奏宋邊七十餘村來附,詔撫存之。乙巳,以御容殿為玉殿,酒穀為聖穀。速撒奏討黨項捷,遣使慰勞。戊申,以惕隱化哥為北院大王,解領為南府宰相。辛亥,幸聖山,遂謁三陵。甲寅,以皇女長壽公主下嫁國舅宰相蕭婆項之子吳留。



 三月戊午,天德軍節度使頹刺父子戰歿,以其弟涅離襲爵。



 己未,次獨山。遣使賞西南面有功將士。辛酉,以大父帳太尉耶律曷魯宇為惕隱。甲子,駐蹕遼河之平澱。辛巳,以國舅、同平章事蕭道寧為遼
 興節度使,仍賜號忠亮佐理功臣。壬午,以青牛白馬祭天地。



 夏四月丙戌朔,幸東京。以樞密副使耶律末只兼侍中,為東京留守。庚寅,謁太祖廟。癸巳,詔賜物命婦寡居者。丙申,南幸。辛丑,謁三陵,以東京所進物分賜陵寢官吏。復詔賜西南路招討使大漢劍,不用命者得專殺。壬寅,致享於凝神殿。



 癸卯,謁乾陵。乙巳,遣人以酒脯祭平章耶律河陽墓。庚戌,幸夫人烏骨里第,謁太祖御容,禮畢,幸公主胡古典第飲,賜與甚厚。壬子,大臣以在詬預政,宜有尊號,請下有司詳定冊禮。詔樞密院諭沿邊節將,至行禮日,止遣子弟奉表稱賀,恐失邊備。樞密
 請詔北府司徒頗德譯南京所進律文,從之。遂如徽州。如耶律慶朗為信州節度使。



 五月丙辰朔,國舅、政事門下閏章事蕭道寧以皇太后慶壽,請歸父母家行禮,而齊國公主及命婦、群臣各進物。設寓,賜國舅帳耆年物有差。壬戌,西南路招討請益兵討西突厥諸部,詔北王府耶律蒲奴寧以敵畢、迭烈二部兵赴之。癸亥以於越休哥在南院過用吏人,詔南大王毋相循襲。庚午,耶律善補招亡入宋者,得千餘戶歸國,詔令撫慰。辛未,次永州,祭王子藥師奴墓。乙亥,詔近臣議皇太后上尊號冊禮,樞密使韓德度以後漢太后監朝故事草定上之。丙
 子,以青牛白馬祭天地。戊寅,幸木葉山。西南路招討使大漢奏,近遣拽刺跋刺哥諭黨項諸部,來者甚眾,下詔褒美。



 六月乙酉朔,詔有司,冊皇太后日給三品以上法服,三品以下用大射柳之服。西南路招討使使奏黨項酋長執夷離堇子隈引等乞內附,詔撫納之,仍察其誠偽,謹邊備。丙戌,還上京。



 己丑,有司奏,同政事門下平章事、駙馬都尉盧俊與公主不協,詔離之,遂出俊為興國軍節度使。辛卯,有事於太廟。甲午,上率群臣上皇太后尊號曰承天皇太后,群臣上皇帝尊號曰天輔皇帝,大赦,改元統和。丁未,覃恩中外,文武官各進爵一級。



 以樞密
 副使耶律斜軫守司徒。



 秋七月甲寅朔,皇太后聽政。乙卯,上親錄囚。五子司徒婁國坐稱疾不山陵,笞二十。辛酉,行再生禮癸酉,臨潢尹哀袞進飲饌。上與諸王分朋擊鞠。丙子,韓德威遣詳穩轄馬上破黨項俘獲數,並送夷離堇之子來獻。辛巳,賞西南面有功將士。



 八月戊子,上西巡。己丑,謁祖陵。辛卯,皇太后祭楚國王蕭思溫墓。癸巳,上與皇太后謁懷陵,遂幸懷州。甲午,上與斜軫於太后前易弓矢鞍馬,約以為友。己亥,獵赤山,遣使薦熊肪、鹿脯於乾陵之神殿。以政事令孫楨無子,詔國舅小翁帳郎君桃隈為之後。乙巳,詔於越休哥提點元
 城。壬子,韓德威表請伐黨項之復叛者,詔許之;仍發別部兵數千以助之。



 九月癸丑朔,以東京、平州旱、蝗,詔振之。乙卯,謁永興、長寧、郭睦三宮。丙辰,南京留守奏,秋霖豁稼,請權停關征,以通山西糶易,從之。庚申,謁宣簡皇帝廟。辛酉,幸祖州,謁祖陵。壬戌,還上京。辛未,有司請以帝生日為千齡節,從之。皇太后言故於越屋只有傅導功,宜錄其子孫;遂命其子泮泱為林牙。丙子,如老翁川。



 冬十月癸未朔,天奏老人星見。戊子,以公主淑哥下嫁國舅詳穩照姑。癸巳,速撒奏敵烈部及叛蕃來降,悉復故地。乙未,以燕京留守於霸占休哥言,每歲諸節度使
 貢獻,如契丹官例,止進鞍馬,從之。丁酉,以吳王稍為上京留守,行臨潢尹事。上將征高麗,親閱東京留守耶律末只所部馬。丙午,命宣微使兼侍中蒲領、林牙肯德等將兵東討,賜旗鼓及銀符。



 十一月壬子朔,觀漁撻馬濼。癸丑,應州奏,獲宋諜者,言宋除道五臺山,將入靈丘界。詔諜者及居停人並磔於市,庚辰,上與皇太后祭乾陵,下詔諭三京左右相、左右平章事、副留守判官、諸道節度使判官、諸軍事判官、錄事參軍等,當執公方,毋得阿順。諸縣令佐如遇州官及朝使非理徵求,毋或畏徇。恆加採聽,以為殿最。民間有父母在,別籍異居者,聽鄰
 里覺察,坐之。有孝於父母,三世同居者,旌其門閭。



 十二月壬竿朔,謁凝神殿,遣使分祭諸陵,賜守殿官屬酒。



 是日,幸顯州。丁亥,以顯州歲貢綾錦分賜左右。甲午,東幸。



 己亥,皇太後觀漁於玉盆灣。辛丑,觀漁於浚淵。甲辰,敕諸刑闢已結正決遣而有冤者,聽詣臺訴。是夕,然萬魚燈於雙溪。



 戊申,千齡節,祭日月,禮畢,百僚稱賀。



 二年春正月甲子,如長濼。



 二月癸巳,國舅帳彰德軍節度使蕭闥覽來朝。甲午,賜將軍耶律敵不春衣、束帶。丙申,東路行軍、宣微使耶律蒲寧奏討女直捷,遣使執手獎諭。庚子,朝皇太后,太后因從觀獵於鐃樂川。乙巳,五國
 烏隈於厥節度使耶律隗塵以所轄諸部難治,乞賜詔給劍,便宜行事,從之。丙午,上與諸王大臣較射。丁未,韓德威以征黨項回,遂襲河東,獻所俘,賜詔褒美。



 三月乙卯,劃離部請今後詳穩止從本部選授為宜,上曰:「諸部官惟在得人,豈得定以所部為限。」不允。贈故同平章事趙延煦兼侍中。



 夏四月丁亥,宣徽使、同平章事耶律普寧、都監蕭勤德獻征女直捷,授普寧兼政事令,勤德神武衛大將軍,各賜金器諸物。庚寅,皇太后臨決滯獄。辛卯,祭風件。壬辰,以宣徽南院劉承規為承德軍節度使,崇德宮都部署、保義軍節度使張德筠為宣徽北院
 使。



 五月乙卯,祠木葉山。丁丑,駐蹕沿柳湖。



 六月巳卯朔,皇太后決獄,至月終。秋七月癸丑,皇太后行再生禮。



 八月辛卯,東京留守兼侍中耶律末只奏,女直術不直、賽裡等八族乞舉眾內附,詔納之。



 九月戊申朔,駐蹕土河。辛未,以景宗忌日,詔諸道京鎮遣官行香飯僧。



 冬十月丁丑朔,以歸化州刺史耶律普寧為彰德軍節度使,右武衛大將軍韓倬為彰國軍節度使兼侍衛親軍兵馬都指揮使。



 十一月壬子,以樞密直學士、%給事中鄭嘏為儒州刺史。



 是月,速撒等討阻卜,殺其酋長撻刺幹。



 十二月辛丑,以翰林學士承旨馬得臣為宣政殿學士,耶
 律頗德南京統軍使,耶律瑤升大內惕隱,大仁東京中臺省右平章事。



 三年春正月丙竿朔,如長濼。丁巳,以翰林學士邢抱樸為尚書、禮部侍郎、知制誥,左拾遺知制誥劉景、吏部郎中知制誥牛藏用並政事舍人。



 二月丙子朔,以牛藏用知樞密直學士。



 三月乙巳朔,樞密奏契丹諸役戶多困乏,請以富戶代之。



 上因閱諸療籍,涅刺、烏隈二部戶少而役重,並量免之。



 夏四月乙亥朔,祠木葉山。壬午,以鳳州刺史趙匡符為保靜軍節度使。癸未,以左監門衛大將軍王庭勖為奉先軍節度使,彰武軍節度使韓德凝
 為崇義軍節度使。



 五月壬子,還上京。癸酉,以國舅蕭道寧同平章事、知瀋州軍州事。



 六月甲戌朔,如柏坡。皇太后親決滯獄。乙亥,以歸義軍節度使王希嚴為興國軍節度使。



 秋七月甲辰朔,詔諸道繕甲兵,以備東征高麗。甲寅,東幸。甲子,遣郎君班哀賜秦王韓匡嗣葬物。丙寅,駐蹕土河。以暴漲,命造船橋,明日乘步輦出聽政。老人星見。丁卯,遣使閱東京諸軍兵器及東征道路。以平章事蕭道寧為昭德軍節度使,武定軍節度使、守司空兼政事令郭襲為天平軍節度使,大同軍節度使、守太子太師兼政事令劉延構為義成軍節度使,贈尚父秦王韓
 匡嗣尚書令。



 八月癸酉朔,以遼澤沮洳,罷征高麗。命樞密使耶律斜軫為都統,駙馬都尉蕭懇德為監軍,以兵討女直。丁丑,次稿城。



 庚辰,至顯州,謁凝神殿。辛巳,幸乾州,觀新宮。癸未,謁乾陵。甲申,命南、北面臣僚分巡山陵林木,及令乾、顯二州上所部里社之數。丙戌,北皮室詳穩進勇敢士七人。戊子,故南院大王諧領已里婉妻蕭氏奏夫死不能葬,詔有司助之。庚寅,東征都統所奏路尚陷濘,未可進討,詔俟澤涸深入。癸巳,皇太后謁顯陵。庚大,謁乾陵。辛丑,西幸。



 閏九月癸酉,命邢抱樸勾檢顯陵。丙子,行次海上。庚辰,重九,駱駝山登高,賜群臣菊花
 酒。辛巳,詔諭東征將師,乘水涸進討。丙申,女直宰相術不裏來貢。戊戌,駐蹕東古山。



 巳亥,速撒奏術不姑諸部至近澱,夷離堇易魯姑請行俘掠,上曰:「諸部於國無惡,保故俘掠,徒生事耳。」不允。



 冬十一月甲戌,詔吳王領秦王韓匡嗣葬祭事。丁丑,詔以東北路兵馬監軍妻婆底裡存撫邊民。戊寅,賜公主胡骨典葬夫金帛、工匠。辛卯,以韓德讓兼政事令。癸巳,禁行在市易布帛不中尺度者。丙申,東征女直,都統蕭闥覽、菩薩奴以行軍所經地里、物產來上。



\end{pinyinscope}