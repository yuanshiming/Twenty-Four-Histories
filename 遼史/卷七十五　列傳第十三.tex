\article{卷七十五 列傳第十三}

\begin{pinyinscope}

 耶律休哥孫馬哥耶律斜軫耶律奚低耶律學古弟烏不呂耶律休哥,字遜寧。祖釋魯,隋國王。父綰思,南院夷離堇。休哥少有公輔器。初烏古、室韋二部叛,休哥從北府宰相蕭幹討之。應歷末,為惕隱。



 乾亨元年,宋侵燕,北院大王奚底、統軍使蕭討古等敗績,南京被圍。帝命休哥代奚底,將五院軍往救。遇大敵於高梁河,與耶律斜軫分
 左右翼,擊敗之。追殺三十餘里,斬首萬餘級,休哥被三創。明旦,宋主遁去,休哥以創不能騎,輕車追至涿州,不及而還。



 是年冬,上命韓匡嗣、耶律沙伐宋,以報圍城之役。休哥率本部兵從匡嗣等戰於滿城。翌日將復戰,宋人請降,匡嗣信之。休哥曰:「彼眾整而銳,必不肯屈,乃誘我耳。宜嚴兵以待。」匡嗣不聽。休哥引兵憑高而視,須臾南兵大至,鼓噪疾馳。匡嗣倉卒不知所為,士卒棄旗鼓而走,遂敗績。休哥整兵進擊,敵乃卻。詔總南面戍兵,為北院大王。明年,車駕親征,圍瓦橋關。宋兵來救,守將張師突圍出。



 帝親督戰,休哥斬師,餘眾退走入城。宋陣於水南。
 將戰,帝以休哥馬介獨黃,慮為敵所識,乃賜玄甲、白馬易之。休哥率精騎渡水,擊敗之,追至莫州。橫尸滿道,(革義)矢懼罄,生獲數將以獻。帝悅,賜御馬、金盂,勞之曰:「爾勇過於名,若人人如卿,何優不克?」師還,拜於越。



 聖宗即位,太后稱制,令休哥總南面軍務,以便宜從事。



 休哥均戍兵,立更休法,勸農桑,修武備,邊境大治。統和四年,宋復來侵,其將範密、楊繼業出雲州;曹彬、米信出雄、易,取歧溝、涿州,陷固安,置屯。時北南院、奚部兵未至,休哥力寡,不敢出戰。夜以輕騎出兩軍間,殺其單弱以脅餘眾;晝則以精銳張其勢,使彼勞於防禦,以疲其力。又設伏林
 莽,絕其糧道。曹彬等以糧運不繼,退保白溝。月餘,復至。休哥以輕兵薄之,伺彼蓐食,擊其離伍單出者,且戰且卻。由是甫軍自救不暇,結方陣,塹地兩邊而行。軍渴乏井,漉淖而飲,凡四日始達於涿。聞太後軍至,彬等冒雨而遁。太后益以銳卒,追及之。彼力窮,環糧車自衛,休哥圍之。至夜,彬、信以數騎亡去,餘眾悉潰。追至易州東,聞宋師尚有數萬,瀕沙河而爨,促兵往擊之。宋師望塵奔竄,墮岸相蹂死者過半,沙河為之不流。太后旋旆,休哥收宋尸為京觀。封宋國王。



 又上言,可乘宋弱,略地至河為界。書奏,不納。及太后南征,休哥為先鋒,敗宋兵於望
 都。時宋將劉廷讓以數萬騎並海而出,約與李敬源合兵,聲言取燕。休哥聞之,先以兵扼其要地。會太後軍至,接戰,殺敬源,廷讓走瀛州。七年,宋遣劉廷讓等乘暑潦來攻易州,諸將憚之;獨休哥率銳卒逆擊於沙河之北,殺傷數萬,獲輜重不可計,獻於朝。太后嘉其功,詔免拜、不名。自是宋不敢北向。時宋人欲止兒啼,乃曰:「於越至矣!」



 休哥以燕民疲弊,省賦役,恤孤寡,戒戍兵無犯宋境,雖馬牛逸於北者悉還之。遠近向化,邊鄙以安。十六年,薨。是夕,雨木冰。聖宗詔立祠南京。



 休哥智略宏遠,料敵如神。每戰勝,讓功諸將,故士卒樂為之用。身更百戰,未
 嘗殺一無辜。二子:高八,官至節度使;高十,終於越。孫馬哥。



 馬哥,字訛特懶。興宗時,以散職入見。上問:「卿奉佛乎?」對曰:「臣每旦誦太祖、太宗及先臣遺訓,未暇奉佛。」



 帝悅。清寧中,遷唐古部節度使。咸雍中,累遷匡義軍節度使。



 大康初,致仕,卒。



 耶律斜軫,字韓隱,於越曷魯之孫。性明敏,不事生產。



 保寧元年,樞密使蕭思溫薦斜軫有經國才,上曰:「朕知之,第佚蕩,豈可羈屈?」對曰:「外雖佚蕩,中未可量。」



 乃召問以時政,占對剴切,帝器重之。妻以皇后之侄,命節制西南
 面諸軍,仍援河東。改南院大王。



 乾亨初,宋再攻河東,從耶律沙至白馬嶺遇敵,沙等戰不利;斜軫赴之,令麾下萬矢齊發,敵氣褫而退。是年秋,宋下河東,乘勝襲燕,北院大王耶律奚底與蕭討古逆戰,敗績,退屯清河北。斜軫取奚底等青幟軍於得勝口以誘敵,敵果爭赴。



 斜軫出其後,奮擊敗之。及高梁之戰,與耶律休哥分左右翼夾擊,大敗宋軍。



 統和初,皇太后稱制,益見委任,為北院樞密使。會宋將曹彬、米信出雄、易,楊繼業出代州。太后親帥師救燕,以斜軫為山西路兵馬都統。繼業陷山西諸郡,各以兵守,自屯代州。斜軫至安定,遇賀令圖軍,擊
 破之,追至五臺,斬首數萬級。



 明日,至蔚州,敵不敢出,斜軫書帛射城上,諭以招慰意。陰聞宋軍來救,令都監耶律題子夜伏兵險厄,俟敵至而發。城守者見救至,突出。斜軫擊其背,二軍俱潰,追至飛狐,斬首二萬餘級,遂取蔚州。賀令圖、潘美復以兵來,斜軫逆於飛狐,擊敗之。宋軍在渾源、應州者,皆棄城走。斜軫聞繼業出兵,令蕭撻凜伏兵於路。明旦,繼業兵至,斜軫擁眾為戰勢。繼業麾幟而前,斜軫佯退。伏兵發,斜軫進攻,繼業敗走,至狼牙村,眾軍皆潰。繼業為流矢所中,被擒。斜軫責曰:「汝與我國角勝三十餘年,今日何面目相見!」繼業但稱死罪而
 已。初,繼業在宋以驍勇聞,人號楊無敵,首建梗邊之策。至狼牙村,心惡之,欲避不可得。既擒,三日死。



 斜軫歸闕,以功加守太保。從太后南伐,卒於軍。太后親為哀臨,仍給葬具。庶子狗兒,官至小將軍。



 耶律奚低,孟父楚國王之後。便弓馬,勇於攻戰。景宗時,多任以軍事。



 統和四年,為右皮室詳隱。時宋將楊繼業陷山西郡縣,奚低從樞密使斜軫討之。凡戰必以身先,矢無虛發。繼業敗於朔州之南,匿深林中。奚低望袍影而射,繼業墮馬。先是,軍令須生擒繼業,奚低以故不能為功。



 後太后南伐,屢有戰績。以病卒。



 耶律學古,字乙辛隱,於越窪之庶孫。穎悟好學,工譯鞮及詩。保寧中,補御盞郎君。



 乾亨元年,宋既下河東,乘勝侵燕,學古受詔往援。始至京,宋敗耶律英底、蕭討古等,勢益張,圍城三周,穴地而進,城中民懷二心。學古以計安反側,隨宜備御,晝夜不少懈。適有敵三百餘人夜登城,學古戰卻之。會援軍至,圍遂解。學古開門列陣,四面鳴鼓,居民大呼,聲震天地。旋有高梁之捷。



 以功遙授保靜軍節度使,為南京馬步軍都指揮使。



 二年,伐宋,乞將漢軍,從之,改彰國軍節度使。時南境未靜,民思休息,學古禁寇掠以安之。會宋將潘美率兵分道來侵,學古以
 軍少,虛張旗幟,雜丁黃為疑兵。是夜,適獨虎峪舉烽火,遣人偵視,見敵俘掠村野,擊之,悉獲所掠物,擒其將領。自是學古與潘美各守邊約,無相侵軼,民獲安業。以功為惕隱,卒。弟烏不呂。



 烏不呂,字留隱。嚴重,有膂力,善屬文。統和中伐宋,屢任以軍事。



 嘗與爻直不相能,因曰:「爾奴才,何所知?」爻直訟於北院樞密使韓德讓。德讓怒,間曰:「爾安得此奴耶?」烏不呂對曰:「三父異籍時亦易得。」德讓笑而釋之。



 後從蕭恆德伐蒲盧毛朵部,以功為東路統軍都監。及德讓為大丞相,薦其材可任統軍使,太后曰:「烏不呂嘗不遜於
 卿,何善而薦嚴德讓奏曰:「臣恭相位,於臣猶不屈,況於其餘。



 以此知可用。若任使之,必能鎮撫諸蕃。」太后從之,加金紫崇祿大夫、檢校太尉。



 而弟國留以罪亡,烏不呂及其母俱下吏。恐禍及母,陰使人召國留,給曰:「太后知事之誣,汝第來勿畏。」國留至,沒有司,坐誅。其後,退歸田里,以疾卒。



 論曰:「宋乘下太原之銳,以師圍燕,繼遣曹彬、楊繼業等分道來伐。是兩役也,遼亦炭皮乎殆哉!休哥奮擊於高梁,敵兵奔潰;斜軫擒繼業於朔州,旋復故地。宋自是不復深入,社稷固而邊境寧,雖配古名將,無愧矣。然非學
 古之在甫南安其反側,則二將之功,蓋亦難致。故曰,國以人重,信哉。」



\end{pinyinscope}