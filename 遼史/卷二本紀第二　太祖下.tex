\article{卷二本紀第二 太祖下}

\begin{pinyinscope}

 四年春正月丙申,射虎東山。



 二月丙寅,修遼陽故城,以漢民、渤海戶實之,改為東平郡,置防禦使。



 夏五月庚辰,至自東平郡。



 秋八月丁酉,謁孔子廟,命皇后、皇太子分謁寺觀。



 九月,征烏古部,道聞皇太后不豫,一日馳六百里還,侍太后,病間,復還軍中。



 冬十月丙午,次烏古部,天大風雪,兵不能進,上禱於天,俄頃而霽。命皇太子將先
 鋒軍進擊,破之,俘獲生口萬四千二百,牛馬、車乘、廬帳、器物二十餘萬。自是舉部來附。



 五年春正月乙丑,始制契丹大字。



 夏五月丙寅,吳越王復遣滕彥休貢犀角、珊瑚,授官以遣。



 庚辰,有龍見於拽刺山陽水上,上射獲之,藏其骨內府。



 閏六月丁卯,以皇弟蘇為惕隱,康默記為夷離畢。



 秋八月己未朔,黨項諸部叛。辛未,上親征。



 九月己丑朔,梁遣郎公遠來聘。壬寅,大字成,詔頒行之。



 皇太子率迭刺部夷離堇污里軫等略地雲內、天德。



 冬十月辛未,攻天德。癸酉,節度使宋瑤降,賜弓矢、鞍馬、旗鼓,更其軍曰應天。甲戌,班師。宋瑤復
 叛。丙子,拔其城,擒宋瑤,俘其家屬,徙其民於陰山南。



 十二月己未,師還。



 六年春正月丙午,以皇弟蘇為南府宰相,迭里為惕隱。南府宰相,自諸弟構亂,府之名族多四罹其禍,故其位久虛,以鋤得部轄得里、只裏古攝之。府中數請擇任宗室,上以舊制不可輒變;請不已,乃告於宗廟而後授之。宗室為南府宰相自此始。



 夏五月丙戌朔,詔定法律,正班爵。丙申,詔畫前代直臣像為《招諫圖》,及詔長吏四孟月詢民利病。



 六月乙卯朔,日有食之。



 冬十月癸丑朔,晉新州防禦使王鬱以所部山北兵馬內附。



 丙子,上率大軍
 入居庸關。



 十一月癸卯,下古北口。丁未,分兵略檀、順、安遠、三河、良鄉、望都、潞、滿城、遂城等十餘城,俘其民徙內地。



 十二月癸丑,王鬱率其眾來朝,上呼鬱為子,賞賚甚厚,而徙其眾於潢水之南。庚申,皇太子率王鬱略地定州,康默記攻長蘆。唐義武軍節度使王笮直養子都囚其父,自稱留後。癸亥,圍涿州,有白免緣壘而上,是日破其郛。癸酉,刺史李嗣弼以城降。乙亥,存勖至定州,王都迎謁馬前。存勖引兵趨望都,遇我軍禿餒五千騎,圍之,存勖力戰數四,不解。李嗣昭領三百騎來救,我軍少卻,存勖乃得出,大戰,我軍不利,引歸。存勖至幽州,遣二百
 騎躡我軍後,我軍反擊,悉擒之。己卯,還次檀州,幽人來襲,擊走之,擒其裨將。詔徙檀、順民於東平、瀋州。



 天贊元年春二月庚申,復徇幽、薊地。癸酉,詔改元,赦軍前殊死以下。夏四月甲寅,攻薊州。戊年,拔之,擒刺史胡瓊,以盧國用、涅魯古典軍民事。壬戌,大饗軍士。癸亥,李存勖圍鎮州,張文禮求援,命郎君迭烈、將軍康末怛往擊,敗之,殺其將李嗣昭。辛未,攻石城縣,拔之。



 五月丁未,張文禮卒,其子處瑾遣人奉表來謝。



 六月,遣鷹軍擊西南諸部,以所獲賜貧民。



 冬十月甲子,以蕭霞的為北府宰相。分迭刺部為二院:斜涅赤為北院夷離堇,綰思為
 南院夷離堇。詔分北大濃兀為二部,立兩節度使以統之。



 十一月壬寅,命皇子堯骨為天下兵馬大元帥,略地薊北。



 二年春正月丙申,大元帥堯骨克平州,獲刺史趙思溫、裨將張崇。



 二月,如平州。甲子,以平州為盧龍軍。置節度使。



 三月戊寅。軍於箭笴山,討叛奚胡損,獲之,射以鬼箭。



 誅其黨三百人,沉之狗河。置奚墮瑰部,以勃魯恩權總其事。



 夏四月己酉,梁遣使來聘,吳越王遣使來貢。癸丑,命堯骨攻幽州,迭刺部夷離堇覿烈徇山西地。庚申,堯骨軍幽州東,節度使符存審遣人出戰,敗之。擒其將裴
 信父子。



 閏月庚辰,堯骨抵鎮州。壬午,拔曲陽。丙戌,下北平。



 是月,晉王李存勖即皇帝位,國號唐。



 五月戊午,堯骨師還。癸亥,大饗軍士,賞賚有差。



 六月辛丑,波斯國來貢。



 秋七月,前北府宰相蕭職阿古只及王鬱徇地燕、趙。



 冬十月辛未朔,日有食之。己卯,唐兵滅梁。



 三年春正月,遣兵略地燕南。



 夏五月丙午,以惕隱迭里為南院夷離堇。是月,徙薊州民實遼州地。渤海殺其刺史張秀實而掠其民。六月乙酉,召皇后、皇太子、大元帥及二宰相、諸部頭等詔曰:「上天降監,惠及丞民。聖主明王,萬載一遇。朕既上承天命,下統群生,每有征行,皆奉
 天意。是以機謀在己,取舍如神,國令既行,人情大附。舛論歸正,遐邇無愆。可謂大含溟海,安納泰山矣。自我國之經營,為群方之父母。憲章斯在,胤嗣何憂?升降有期,去來在我。良籌聖會,自有契於天人;眾國群王,豈可化其凡骨?三年之後,歲在丙戌,時值初秋,必有歸處。然未終兩事,豈負親誠?日月非遙,戒嚴是速。」



 聞詔者皆驚懼,莫識其意。是日,大舉征吐渾、黨項、阻卜等部。詔皇太子監國,大元帥堯骨從行。



 秋七月辛亥,曷刺等擊素昆那山東部族,破之。



 八月乙酉,至烏孤山,以鵝祭天。甲午,次古單于國,登阿里典壓得斯山,以麃鹿祭。



 九月丙申朔,
 次古加鶻城,勒石紀功。庚子,拜日於蹛林。



 丙午,遣騎攻阻卜。南府宰相蘇、南院夷離堇迭裡略地西南。



 乙卯,蘇等獻俘。丁已,鑿金河水,取烏山石,輦致潢河、木葉山,以示山川朝海宗獄之意。癸亥,大食國來貢。甲子,詔礱闢遏可汗故碑,以契丹、突厥、漢字紀其功。是月,破胡母思山諸蕃部,次業得思山,以赤牛青馬祭天地。回鶻霸里遣使來貢。冬十月丙寅朔,獵寓樂山,獲野獸數千,以充軍食。丁卯,軍於霸離思山。遣兵逾流沙,拔浮圖城,盡取西鄙諸部。



 十一月乙未朔,獲甘州回鶻都督畢離遏,因遣使諭其主烏母主可汗。射虎於烏刺邪里山,抵霸室
 山。六百餘里且行且獵,日有鮮食,軍士皆給。



 四年春正月壬寅,以捷報皇后、皇太子。



 二月丙寅,大元帥堯骨略黨項。丁卯,皇后遣康末怛問起居,進御服、酒膳。乙亥,蕭阿古只略燕、趙還,進牙旗兵仗。



 辛卯,堯骨獻黨項俘。



 三月丙申,饗軍於水精山。



 夏四月甲子,南攻小蕃,下之。皇后、皇太子迎謁於札里河。癸酉,回鶻烏母主可法遣使貢謝。



 五月甲寅,清暑室韋北陘。



 秋九月癸巳,至自西征。



 冬十月丁卯,唐以滅梁來告,即遣使報聘。庚辰,日本國來貢。辛巳,高麗國來貢。



 十一月丁酉,幸安國寺,飯僧,赦京師囚,縱王坊鷹鶻。



 己酉,新羅國來貢。



 十二
 月乙亥,詔曰:「所謂兩事,一事已畢,惟渤海世急未雪,豈宜安駐。」乃舉兵親征渤海大諲譔。皇后、皇太子、大元帥堯骨皆從。



 閏月壬辰,祠木葉山。壬寅以青牛白馬祭天地於烏山。己酉,次撒葛山,射鬼箭。丁巳,次商嶺,夜圍扶餘府。



 天顯元年春正月己未,白氣貫日。庚申,拔扶餘城,誅其守將。丙寅,命惕隱安端、前北府宰相蕭阿古只等將萬騎為先鋒,遇諲譔老相兵,破之。皇太子、大元帥堯骨、南府宰相蘇、北院夷離堇斜涅赤,南院夷離堇迭裡是夜圍忽汗城。己巳,諲譔請降。庚午,駐軍於忽汗城南。辛未,
 諲譔素服,稿索牽羊,率僚屬三百餘人出降。上優禮而釋之。甲戌,詔諭渤海郡縣。



 丙子。遣近待康末怛等十三人入城索兵器,為邏卒所害。丁丑,諲譔復叛,攻其城,破之。駕幸城中,諲譔請罪馬前。詔以兵衛諲譔及族屬以出。祭告天地,復還軍中。



 二月庚寅,安邊、鄚頡、南海、定理等府及諸道節度、刺史來朝,慰勞遣之。以所獲器幣諸物賜將士。壬辰,以青牛白馬祭天地。大赦,改元天顯。以平渤海遣使報唐。甲午,復幸忽汗城,閱府庫物,賜從臣有差。以奚部長勃魯恩、王鬱自回鶻、新羅、吐蕃、黨項、室韋、沙陀、烏古等從征有功,優加賞齎。丙午,改渤海國為
 東丹,忽汗城為天福。冊皇太子倍為人皇王以主之。以皇弟迭刺為左大相,渤海老相為右大相,渤海司徒大素賢為左次相,耶律羽之為右次相。赦其國內殊死以下。丁未,高麗、濊貊、鐵驪、鞨霫來貢。



 三月戊午,遣夷離畢康默記、左付射韓延徽攻長嶺府。甲子,祭天。丁卯,幸人皇王宮。己巳,安邊、鄚頡、定理三府叛,遣安端討之。丁丑,三府平。壬午,安端獻俘,誅安邊府叛帥二人。癸未,宴東丹國僚佐,頒賜有差。甲申,幸天福城。



 乙酉,班師,以大諲譔舉族行。



 夏四月丁亥朔,次傘子山。辛卯,人皇王率東丹國僚屬辭。



 是月,唐養子李嗣源反,郭存謙弒其主存
 勖,嗣源遂即位。



 五月辛酉,南海、定理二府復叛,大元帥堯骨討之。



 六月丁酉,二府平。丙午,次慎州,唐遣姚坤以國哀來告。



 秋七月丙辰,鐵州刺史衛鈞反。乙丑,堯骨攻拔鐵州。庚午,東丹國左大相迭刺卒。辛未,衛送大諲譔於皇都西,築城以居之。賜諲譔曰烏魯古,妻曰阿里只。盧龍行軍司馬張崇叛,奔唐。甲戌,次扶餘府,上不豫。是夕,大星隕於幄前。辛巳平旦,子城上見黃龍繚繞,可長一里,光耀奪目,入於行宮,有紫黑氣蔽天,逾日乃散。是日,上崩,年五十五。天贊三年上所謂「丙戌秋初,必有歸處」,至是乃驗。壬午,皇后稱制,權決軍國事。



 八月辛卯,
 康默記等攻下長嶺府。甲午,皇后奉梓宮西還。



 壬寅,堯骨討平諸州,奔赴行在。乙巳,人皇王倍繼至。



 九月壬戌,南府宰相蘇薨。丁卯,梓宮至皇都,權殯於子城西北。己巳,上謚升天皇帝,廟號太祖。



 冬十月,盧龍軍節度使盧國用叛,奔於唐。



 十一月丙寅,殺南夷離堇耶律迭里、郎君耶律匹魯等。



 二年八月丁酉,葬太祖皇帝於祖陵,置祖州天城軍節度使以奉陵寢。統和二十六年七月,進謚大聖大明天皇帝。重熙二十一年九月,加謚大聖大明神烈天皇帝。太祖所崩行宮在扶餘城西南雨河之間,後建升天殿於此,而以扶餘為黃龍府云。



 贊曰:遼之先,出自炎帝,世為審吉國,其可知者蓋自奇首云。奇首生都庵山,徙潢河之濱,傳至雅里,始立制度,置官屬,刻木為契,穴地為牢。讓阻午而不肯自立。雅裏生毗牒。



 與黃室韋挑戰,矢貫數札,是為懿祖。懿祖生勻德實,始教民稼穡,善畜牧,國以殷富,是為玄祖。玄祖生撒刺的,仁民愛物,始置錢冶,教民鼓鑄,是為德祖,即太祖之父也。世為契丹遙輦氏之夷離堇,執其政柄。德祖之弟述瀾,北征於厥、室韋,南略易、定、奚、霫,始興板築,置城邑,教民種桑麻,習織組,
 已有廣土眾民之志。而太祖受可汗之禪,遂建國。東征西討,如折枯拉朽。東自海,西至於流沙,北絕大漠,信威萬里,歷年二百,豈一日之故哉。周公誅管、蔡,人未有能非之者。刺葛、安端之亂,太祖既貸其死而復用之,非人君之度乎?舊史扶餘之變,亦異矣夫!



\end{pinyinscope}