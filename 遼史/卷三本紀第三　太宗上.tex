\article{卷三本紀第三 太宗上}

\begin{pinyinscope}

 太
 宗孝武惠文皇帝,諱備光,字德謹,小字堯骨。太祖第二子,母淳欽皇后蕭氏。唐天復二年生,神光異常,獵者獲白鹿、白鷹,人以為瑞。及長,貌嚴重而性寬仁,軍國之務多所取決。天贊元年,授天下兵馬大元帥,事詔統六軍南徇地。明年,下平州,獲趙思溫、張崇。回破箭笴山胡遜奚,諸部悉降。復以兵掠鎮、定,所至皆堅壁不敢戰。師
 次幽州,符存審拒於州南,縱兵邀擊,大破之,擒裨將裴信等數十人。及從太祖破於闕里諸部,定河堧黨項,下山西諸鎮,取回鶻單于城,東平渤海,破達盧古部,東西萬里,所向皆有功。



 天顯元年七月,太祖崩,皇后攝軍國事。



 明年秋,治祖陵畢。冬十一月壬戌,人皇王倍率群臣請於後曰:「皇子大元帥勛望,中外攸屬,宜承大統。」後從之。



 是日即皇帝位。癸亥,謁太祖廟。丙寅,行柴冊禮。戊辰,還都。壬申,御宣政殿,群臣上尊號曰嗣聖皇帝。大赦。有司請改元,不許。十二月庚辰,尊皇太后為太皇太后,皇后為應天皇太后,立妃蕭氏為皇后。禮畢,閱近侍班局。
 辛已,諸道將帥辭歸鎮。己丑,祀天地。庚寅,遣使諭諸國。辛卯,閱群於近效。戊戌,女直遣使來貢。壬寅,謁太祖廟。甲辰,閱旗鼓、客省諸局官屬。丁未,詔選遙輦氏九帳子弟可任官者。



 三年春正月己酉,閱北克兵籍。庚戌,閱南克兵籍。丁巳,閱皮室、拽刺、墨離三軍。己未,黃龍府羅涅河女直、達盧古來貢。庚午,以王鬱為興國軍節度使,守中書令。



 二月,幸長濼。已亥,惕隱涅里袞進白狼。辛丑,達盧古來貢。三月乙卯,東蒐。癸亥,獵羖劷山。乙丑,獵松山。唐義武軍節度使王都遣人以定州來歸。唐主出師討之,使來乞援,
 命奚禿里鐵刺往救之。



 四月戊寅,東巡。己卯,祭麃鹿神。丁亥,於獵所縱公私取羽毛革木之材。甲午,取箭材赤山。丙申,獵三山。鐵刺敗唐將王晏球於定州。唐兵大集,鐵刺請益師。辛丑,命惕隱涅里袞、都統查刺赴之。



 五月丙午,建天膳堂。獵索刺山。戊申,至自獵。丁卯,命林牙突呂不討烏古部。己巳,女直來貢。



 六月己卯,行瑟瑟禮。



 秋七月丁未,突呂不獻討烏古捷。壬子,王都奏唐兵破定州,鐵刺死之,涅里袞、查刺等數十人被執。上以出師非時,甚悔之,厚賜戰歿將校之家。庚午,有事於太祖廟。



 八月丙子,突厥來貢。庚辰,詔建《應天皇太后誕聖碑》於儀
 坤州。



 九月己卯,突呂不遣人獻討烏古俘。癸未,詔分賜群臣。



 己丑,幸人皇王倍第。庚寅,遣人使唐。辛卯,再幸人皇王第。



 癸巳,有司請以上生日為天授節,皇太后生日為永寧節。



 冬十月癸卯朔,以永寧節,上率群臣上壽於延和宮。己酉,謁太祖廟。唐遣使遺玉笛。甲子,天授節,上御五鸞殿受群臣及諸國使賀。



 十一月丙子,鼻骨德平貢。辛丑,自將伐唐。



 十二月癸卯,祭天地。庚戌,聞唐主復遣使平聘,上問左右,皆曰:「唐數遣使來,實畏威也。未可輕舉,觀釁而動可也。」上然之。甲寅,次杏堝,唐使至,遂班師。時人皇王在皇都,詔遣耶律羽之遷東丹民以實東
 平。其民或亡入新羅、女直,因詔困乏不能遷者,許上國富民給贍而隸屬之。升東平郡為南京。



 四年春正月壬申朔,宴群臣及諸國使,觀俳優角抵允。己卯,如瓜堝。



 二月庚戌,閱遙輦氏戶籍。



 三月甲午,望祀群神。



 夏四月辛亥,至自瓜堝。壬子,謁太祖廟。癸丑,謁太祖行宮。甲寅,幸天城軍,謁祖陵。辛酉,人皇王倍來朝。癸亥,錄囚。五月癸酉,謁二儀殿,宴群臣。女直來貢。戊子,射柳於太祖行宮。癸巳,行瑟瑟禮。



 六月丙午,突呂不獻烏古俘。戊申,分賜將士。己酉,西巡。己未,選輕騎數千獵近山。癸亥,駐蹕涼陘。



 秋七月庚辰,觀市,曲赦系囚。甲午,祠
 太祖而東。



 八月辛丑,至自涼陘,謁太祖廟。癸卯,幸人皇王第。己酉,謁太祖廟。



 九月庚午,如南京。戊寅,祠木葉山。己卯,行再生禮。



 癸巳,至南京。



 冬十月壬寅,幸人皇王第,宴群臣。甲辰,幸諸營,閱軍籍。庚戌,以雲中郡縣未下,大閱六軍。甲子,詔皇弟李胡帥師趣雲中討郡縣之未附者。



 十一月丙寅朔,以出師告天地。丁卯,餞皇弟李胡於西郊。



 壬申,命大內惕隱告出師於太祖行宮。甲申,觀漁三叉口。



 十二月戊申,女直來貢。戊午,至自南京。



 五年春正月庚午,皇弟李胡拔寰州捷至。甲午,朝皇太后。



 二月己亥,詔修南京。癸卯,李胡還自雲中,朝於行在。



 丙午,以先所俘渤海戶賜李胡。丙辰,上與人皇王朝皇太后。



 太后以皆工書,命書於前以觀之。辛酉,召群臣議軍國事。



 三月丙寅,朝皇太后。丁卯,皇弟李胡請赦宗室舍利郎君以罪系獄者,詔從之。己巳,幸皇叔安端第。辛未,人皇王獻白紵。乙亥,冊皇弟李胡為壽昌皇太弟,兼天下兵馬大元帥。



 壬午,以龍化州節度使劉居言同中書門下平章事。乙酉,宴人皇王僚屬便殿。庚寅,駕發南京。



 夏四月乙未,詔人皇王先赴祖陵謁太祖廟。丙辰,會祖陵。



 人皇王歸國。



 五月戊辰,詔修哀潭離宮。乙酉,謁太祖廟。



 六月己亥,射柳於行在。乙卯,如沿柳湖。丁巳,拜太祖御
 容於明殿。己未,敵烈德產貢。



 秋七月壬申,烏石來貢。戊子,薦時果於太祖廟。



 八月丁酉,以大聖皇帝、皇后宴寢之所號日月宮,因建《日月碑》。丙午,如九層臺。



 九月己卯,詔舍利普寧撫慰人皇王。庚辰,詔置人皇王儀衛。丁亥,至自九層臺,謁太祖廟。



 冬十月戊戌,遣使賜人丘王胙。癸卯,建《太祖聖功碑》於如迂正集會堝。甲辰,人皇王進玉笛。



 十一月戊寅,東丹奏人皇王浮海過唐。



 六年春正月甲子。西南邊將以慕化轄戞期國人來。乙丑,敵烈德來貢。丁卯,如南京。



 三月辛未,召大臣議軍國事。丁亥,人皇王倍妃蕭氏率其國僚屬來見。



 夏四月己
 酉,唐遣使來聘。是月置中臺省於南京。



 五月乙丑,祠木葉山。乙亥,至自南京。壬午,謁太祖陵。



 閏月庚寅,射柳於近郊。



 六月壬申,如涼陘。壬午,烏古來貢。



 秋七月丁亥,女直來貢。己酉,命將校以兵南略。壬子,薦時果於太祖廟。東幸。



 八月庚申,皇子述律生,告太祖廟。辛巳,鼻骨德平貢。



 九月甲午,詔修京城。



 冬十月丁丑,鐵驪來貢。



 十一月乙酉,唐遣使來聘。



 十二月甲寅朔,祭太祖廟。丙辰,遣人以詔賜唐盧龍軍節度使趙德鈞。



 七年春正月壬辰,征西將軍課里遣拽刺鐸括奏軍事。己亥,唐遣使來聘。癸卯,遣人使唐。戊申,祠木葉山。



 二月
 壬申,拽刺迪德使吳越還,吳越王遣使從,獻寶器。



 復遣使持幣往報之。三月己丑,林牙迪離畢指斥乘輿,囚之。



 丁未,遣使諸國。戊申,以率群臣朝於皇太后。



 夏四月甲戌,唐遣使來聘,致人皇王倍書。己卯,女直來貢。



 五月壬午朔,幸祖州,謁太祖陵。



 六月戊辰,禦制《太祖國碑》。戊寅,烏古、敵烈德來貢。



 庚辰,觀角抵戲。



 秋七月辛巳朔,賜中外官吏物有差。癸未,賜高年布帛。



 丙戌,召群臣耆老議政。壬辰,唐遣使遣紅牙笙。癸已,使復至,懼報定歸之役也。壬寅,唐盧龍軍節度使趙德鈞遣人進時果。丁未,薦新於太祖廟。



 八月壬戌,捕鵝於油柳湖,風雨暴至,舟
 覆,溺死者六十餘人,命存恤其家,識以為戒。戊辰,林牙迪離畢逸囚,復獲而鞫之,知其事本誣構,釋之。



 九月庚子,阻卜來貢。



 冬十月乙卯,唐遣使來聘。己巳,遣使雲中。



 十一月丁亥,遣使存問獲里國。丁未,阻卜貢海東青鶻三十連。十二月辛亥,以叛人泥離袞家口分賜群臣。丁巳,西狩,駐蹕平地松林。



 八年春正月戊子,女直來貢。庚子,命皇太弟李胡、左威衛上將軍撒割率兵伐黨項。癸卯,上親餞之。



 二月辛亥,吐谷渾、阻卜來貢。乙卯,克實魯使唐琿,以附獻物分賜群臣。



 三月辛卯,皇太弟討黨項勝還,宴勞之。丙申,唐遣
 使請罷征黨項兵,上以戰捷及黨項已聽命報之。



 夏四月戊午,黨項來貢。



 五月己丑,獵獨牛山,惕隱迪輦所乘內廄騮馬斃,因賜名其山曰騮山。戊戌,如沿柳湖。



 六月甲寅,阻來貢。甲子,回鶻阿薩蘭來貢。



 秋七月戊寅,行納後禮。癸未,皇子提離古生。丁亥,鐵驪、女直、卜來貢。



 冬十月乙巳,阻卜來貢。丙午,至自沿柳湖。辛亥,唐遣使來聘。己未,遣拔刺使唐。辛未,烏古吐魯沒來貢。



 十一月辛丑,太皇太后崩,遣使告哀於唐及人皇王倍。是月,唐主嗣源殂,子從厚立。十二月丁卯,黨項來貢。



 九年春正癸酉,漁於土河。丙申,黨項貢駝、鹿。己亥,南
 京進白獐。



 閏月戊午,唐遣使告哀,即日遣使吊祭。壬戌,東幸。女直來貢。



 二月壬申,祠木葉山。戊寅,葬太皇太后於德陵。前二日,發喪于菆殿,上且衰服以送。後追謚宣簡皇后,詔建碑於陵。



 三月癸卯,女直來貢。



 夏四月,唐李從珂弒其主自立。人皇王倍自唐上書請討。



 五月甲辰,如沿柳湖。癸丑,女直來貢。大星書隕。



 六月己巳朔,鼻骨德來貢。辛未,唐李從厚謝吊祭所遣使初至闕。



 秋八月壬午,自將南伐。乙酉,拽刺解裡手接飛雁,上異之,因以祭天地。



 九月庚子,西南星隕如雨。乙卯,次雲州。丁巳,拔河陰。



 冬十月丁亥,略地靈丘,父老進牛酒犒師。



 十一
 月辛丑,圍武州之陽城。壬寅,陽城降。癸卯,窪只城降,括所俘丁壯籍於軍。



 十二月壬辰,皇子阿缽撒葛裏生,皇后不豫。是月駐蹕百湖之西南。



 十年春正月戊申,皇后崩於行在。



 二月戊寅,百僚請加追謚,不許。辛巳,宰相涅里袞謀南奔,事覺,執之。



 三月戊午,黨項來貢。



 夏四月,吐谷渾酋長退欲德率眾內附。丙戌,皇太后父族及母前夫之族二帳並為國舅,以蕭緬思為尚父領之。己丑,錄囚。五月甲午朔,始制服行後喪。丙午,葬於奉陵。上自制文,謚曰彰德皇后。癸丑,以舍利王庭鶚為龍化州節度使。



 六月乙丑,吐渾來貢。辛未,幸
 品不里澱。



 秋七月乙卯,獵南赤山。



 冬十一月丙午,幸弘福寺為皇后飯僧,見觀音畫像,乃大聖皇帝、應天皇后及人皇王所施,顧左右曰:「昔與父母兄弟聚觀於此,歲時未幾,今我獨來!」悲嘆不已。乃自制文題於壁,以極追感之意。讀者悲之。



 十二月庚辰,如金瓶濼,遣拽刺化哥、窟魯里、阿魯掃姑等捉生敵境。



 十一年春正月,鉤魚於土河。庚申,如潢河。



 三月庚寅朔,女直來貢。



 夏四月庚申,謁祖陵。戊辰,還都,謁太祖廟。辛未,燕民之復業者陳汴州事宜。癸酉,女直諸部來貢。癸未,賜回鶻使衣有差。



 五月戊戌,清暑沿柳湖。



 六月戊午
 朔,鼻骨德來貢。乙酉,吐谷渾來貢。



 秋七月辛卯,烏古來貢。壬辰,蒲割(寧頁)公主率三河烏古來朝。丙申,唐河東節度使石敬瑭為其主所討,遣趙瑩因西南路招討盧不姑求救,上白太后曰:「李從珂弒君自立,神人共怒,宜行天討。」時趙德鈞亦遣使至,河東復遣桑維翰來告急,遂許興師。



 八月己未,遣蕭轄裡報河東師期,丙寅,吐谷渾來貢。庚午,自將以援敬瑭。



 九月癸巳,有飛鴜自墜而死,南府夷離堇曷魯恩得之以獻。



 卜之,吉。上曰:「此從珂自滅之兆也!」丁酉,入雁門。戊戌,次忻州,祀天地。己亥,次太原。庚子,遣使諭敬瑭曰:「朕興師遠來,當即與卿破賊。」會
 唐將高行周、符彥卿以兵來拒,遂勒兵陳於太原。及戰,佯為之卻。唐將張敬達、楊光遠又陣於西,未成列,以兵薄之。面蠔周、彥卿為伏兵所斷,首尾不相救。敬達、光遠大敗,棄仗如山,斬首數萬級。敬達走保晉安寨,夷離堇的魯與戰,死之。敬瑭率屬來見,上執手撫慰之。癸卯,圍晉安。甲辰,以的魯子徒離骨嗣為夷離堇,仍以父字為名,以旌其忠。南宰相鶻離底、奚監軍寅你已、將軍陪阿臨陳退懦,上召切責之。



 冬十月甲子,封敬瑭為晉王,幸其府。敬瑭與妻李率其親屬捧觴上壽。初圍晉安,分遣精兵守其要害,以絕援兵之咱。



 而李從珂遣趙延壽
 以兵二萬屯圍柏谷,範延廣以兵二萬噸遼州,幽州趙德鈞以所部兵萬餘由上黨趨延壽軍,合勢進擊。知此有備,皆逗留不進,從珂遂將精騎三萬出次河陽,親督諸軍。



 然知其不救,但日酣飲悲歌而已。丁卯,召敬瑭至行在所,賜坐。上從容語之曰:「吾三千里舉兵而來,一戰而勝,殆天意也。觀汝雄偉弘大,宜受茲南土,世為我藩輔。」遂命有司設壇晉陽,備禮冊命。



 十一月丁酉,冊敬瑭為大晉皇帝。自戊戌至戊申,候騎雨奏南有兵至,復奏西有兵至。命惕隱迪輦窪拒之。唐將張敬達在圍八十餘日,內外隔絕,軍儲殆盡,至濯馬糞、屑木以飼馬,馬饑至自相
 啖其鬃尾,死則以充食。光遠等勸敬達出降,敬達曰:「吾有死而已。爾欲降,寧斬吾首以降。」



 閏月甲子,楊光遠、安審琦殺敬達以降。上聞敬達至死不變,謂左右曰:「凡為人臣,當如此也!」命以禮葬。所降軍士及馬五千匹以賜晉帝。丙寅,祀天地以告成功。庚午,佧射蕭酷古只奏趙德鈞等諸援兵將遁,詔夜發兵追擊。德鈞等軍皆投戈棄甲,自知蹂踐,擠於川谷者不可勝紀。仍命皇太子馳輕騎險要,追及步兵萬餘,悉降之。辛未,兵度團柏谷,以酒肴祀天地。俄追及德鈞父子,乃率眾降。次潞州,召諸將議,皆請班師,從之。命南宰相解領、鶻離底、奚監軍
 寅你已、將軍陪阿先還。壬申,惕隱窪、林牙迪離畢來獻俘。晉帝辭歸,上與宴飲。酒酣,執手約為父子。以白貂裘一、廄馬二十、戰馬千二百餞之。命迪離畢將五千騎送入洛。臨別,謂之曰:「朕留此,候亂定乃還耳。」辛巳,晉帝至河陽,李從珂窮蹙,召人皇王倍同死,不從,遣人殺之,乃舉族自焚。詔收其士率戰歿者瘞之汾水上,以為京觀。晉命桑維翰為文,紀上功德。



 十二月乙酉朔,遣近待撻魯存瓿晉帝。丙戌,以晉安所獲分賜將校。戊子,遣使馳奏皇太后,及報諸道師還。庚寅,發太原。辛卯,聞晉帝入洛,遣郎君解裏德撫問。壬辰,次細河,閱降將趙德鈞父
 子兵馬。戊戌,次雁門,以沙太保所部兵分隸諸將。庚戌,幸應州。癸丑,唐大同、彰國、振武三節度使迎見,留之不遣。



 十二年春正月丙辰,次堆子口。唐大同軍節度判官吳巒閉城拒命,遣崔廷勛圍其城。庚申,上親征,至城一,巒降。辛酉,射鬼箭於雲州北。壬戌,祀天地。癸亥,遣國舅安端發奚西部民各還本土。丙寅,皇太后遣侍衛實魯趣行,是夕,率輕騎先進。丁丑,皇子述律迎謁於灤河,告國太祖行宮。戊寅,朝於皇太后,進珍玩為壽。



 二月丁亥,以軍前所獲俘叛入幽州者皆斬之。壬寅,詔諸部休
 養士卒。癸卯,晉遣唐所掠郎君刺哥、文班吏蕭嗀裡還朝。



 三月庚申,晉遣使來貢。丁卯,晉天雄軍節茺使範延廣潛遣人請內附,不納。己巳,遣郎君的烈古、梅裏迭烈使晉。壬午,晉使及諸國使來見。



 夏四月甲申,地震。幸平地松林,觀潢水源。五月甲寅,幸頻蹕澱。壬申,震開皇殿。



 六月甲申,晉遣戶部尚書聶延祚等請上尊號,及歸雁門以北與幽、薊之地,仍歲貢帛三十萬足,詔不許。庚戌,侍中列率言,範延廣叛晉,引兵南向。



 秋七月辛亥朔,詔諸部治兵甲。癸丑,幸懷州,謁奉陵。



 甲子,晉遣使來告範延廣反。庚午,遣耶律哀古裡使晉議軍畫。



 八月癸未,晉
 遣使復請上尊號,不許。庚寅,晉及太原劉知遠、南唐李嚈各遣使來貢。庚子,晉遣使以都汴及範延廣降來告。九月壬子,鼻骨德來貢。庚申,遣直里古使晉及南唐。癸亥,術不姑、女直來貢。辛未,遣使高麗。鐵驪。癸酉,回鶻來貢。冬十月庚辰朔,皇太后永寧節,晉及回鶻、敦煌諸國皆遣使來賀。壬午,詔回鶻使胡離只、阿刺保,問其風俗。丁亥,諸國使還,就遣蒲裏骨皮室胡末裡使其國。



 十一月己未,遣使求醫於晉。丁卯,鐵驪來貢。



 十二月甲申,東幸,祀木葉山。己丑,醫來。



\end{pinyinscope}