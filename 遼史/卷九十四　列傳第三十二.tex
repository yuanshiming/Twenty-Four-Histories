\article{卷九十四 列傳第三十二}

\begin{pinyinscope}

 蕭奉先李處溫張琳耶律餘睹蕭奉先,天祚元妃之兄也。外寬內忌。因元妃為上眷倚,累官樞密使,封蘭陵郡王。



 天慶二年,上幸混同江鉤魚。故事,生女直酋長在千里內者皆朝行在。適頭魚宴,上使諸酋次第歌舞為樂,至阿骨打,但端立直視,辭以不能。再三旨諭,不從。上密謂奉先曰:「阿骨打跋扈若此!可托以邊事誅之。」奉先曰:「彼粗人,不知禮義,且無大過,殺
 之傷向化心。設有異志,蕞爾小國,亦何能為!」上乃止。



 四年,阿骨打起兵犯寧江州,東北路統軍使蕭撻不也戰失利。上命奉先弟嗣先為都統,將番、漢兵往討,屯出河店。女直乃潛渡混同江,乘我師未備來襲。嗣先敗績,軍將往往遁去。



 奉先懼弟被誅,乃奏「東征潰軍逃罪,所至劫掠,若不肆赦,將嘯聚為患」。從之。嗣先詣闕待罪,止免官而已。由是士無鬥志,遇敵輒潰,郡縣所失日多。



 初,奉先誣耶律餘睹結附馬蕭昱謀立其甥晉王,事覺,殺昱。余睹在軍中聞之懼,奔女直。保大二年,余睹為女直監軍,引兵奄至,上憂甚。奉先曰:「余睹乃王子班之苗裔,此
 來實無亡遼心,欲立晉王耳。若以社稷計,不惜一子,誅之,可不戰而退。」遂賜晉王死。中外莫不流涕,人心益解體。



 當女直之兵未至也,奉先逢迎天祚,言:「女直雖能攻我上京,終不能遠離巢穴。」而一旦越三千里直搗雲中,計無所出,惟請播遷夾山。天祚方悟,顧謂奉先曰:「汝父子誤我至此,殺之何益!汝去,毋從我行。恐軍心忿怒,禍必及我。」



 奉先父子慟哭而去,為左右執送女直兵。女直兵斬其長子昂,送奉先及次子昱於其國主。通遇我兵,奪歸,天祚並賜死。



 李處溫,析津人。伯父儼,大康初為將作少監,累官參知
 政事,封漆水郡王,雅與北樞密使蕭奉先友舊。執政十餘年,善逢迎取媚,天祚及寵任之。儼卒,奉先薦處溫為相,處溫因奉先有援己力,傾心阿附,以固權位,而貪污尤甚,凡所接引,類多小人。



 保大初,金人陷中京,諸將莫能支。天祚懼,奔夾山,兵勢日迫。處溫與族弟處能、子奭,外假怨軍聲援,結都統蕭乾謀立魏國王淳,召番、漢官屬詣魏王府勸進。魏國王將出,奭乃持赭袍衣之,令百官拜舞稱賀。魏王固辭不得,遂稱天錫皇帝。以處溫守太尉,處能直樞密院,奭為少府少監,左企弓以下及親舊與其事者,賜官有差。



 會魏國王病,自知不起,密授處
 溫番漢馬步軍都元帥,意將屬以後事。及病亟,蕭乾等矯詔南面宰執入議,獨處溫稱疾不至,陰聚勇士為備,給云奉密旨防他變。魏國王卒,蕭乾擁契丹兵,宣言當立王妃蕭氏為太后,權主軍國事,眾無敢異者。



 乾以後命,召處溫至,時方多難,未欲即誅,但追毀元帥札子。



 處能懼及禍,落發為僧。尋有永清人傅遵說隨郭藥師入燕,被擒,具言處溫嘗遺易州富民趙履仁書達宋將童貫,欲挾蕭後納土歸宋。後執處溫問之,處溫曰:「臣父子於宣宗有定策功,宜世蒙有容,可使因讒獲罪?」後曰:「向使魏國王如周公,則終享親賢之名於後世。誤王者皆
 汝父子,何功之有!」並數其前罪惡。處溫無以對,乃賜死,奭亦伏誅。



 張琳,瀋州人。幼有大志。壽隆末,為秘書中允。天祚即位,累遷戶部使。頃之,擢南府宰相。



 初,天祚之敗於女直也,意謂蕭奉先不知兵,乃召琳付以東征事。琳以舊制,凡軍國大計,漢人不與,辭之。上不允,琳奏曰:「前日之敗,失於輕舉。若用漢兵二十萬分道進討,無不克者。」上許其半,仍詔中京、上京、長春、遼西四路計戶產出軍。時有起至二百軍者,生業蕩散,民甚苦之。四路軍甫集,尋復遁去。



 及中京陷,天祚幸雲中,留琳與李處溫佐魏國王淳
 守南京。



 處溫父子召琳,欲立淳為帝,琳曰:「王雖帝胄,初無上命;攝政則可,即真則不可。」處溫曰:「今日之事,天人所與,豈可易也!」琳雖有難色,亦勉從之。淳既稱帝,諸將咸居權要,琳獨守太師,十日一朝,平章軍國大事。陽以元老尊之,實則不使與政。琳由是鬱悒而卒。



 耶律餘睹,一名餘都姑,國族之近者也。慷慨尚氣義。保大初,歷官副都統。



 其妻天祚文妃之妹;文妃生晉王,最賢,國人皆屬望。時蕭奉先之妹亦為天祚元妃,生秦王。奉先恐秦王不得立,深忌余睹,將潛圖之。適耶律撻葛里之妻會余睹之妻子軍中,奉先諷人誣余睹結駙馬
 蕭昱、撻葛里,謀立晉王,尊天祚為太上皇。



 事覺,殺昱及撻葛里妻,賜文妃死。余睹在軍中聞之,懼不能自明被誅,即引兵千餘,並骨肉軍帳叛歸女直。



 會大霖雨,道途留阻。天祚遣知奚王府蕭遐賣、北宰相蕭德恭、大常袞耶律諦裡姑、歸州觀察使蕭和尚奴、四軍太師蕭千追捕甚急。至閭山,及之。諸將議曰:「蕭奉先恃寵,蔑害官兵。余睹乃宗室雄才,素不肯為其下。若擒之,則他日吾輩皆余睹矣。不如縱之。」還,給雲追襲不及。



 余睹既入女直,為其國前鋒,引婁室孛堇兵攻陷州郡,不測而至。天祚聞之大驚,知不能敵,率衛兵入夾山。



 余睹在女直為監
 軍,久不調,意不自安,乃假游獵,遁西夏。夏人問:「汝來有兵幾何?」余睹以二三百對,夏人不納,卒。



 論曰:「遼之亡也,雖孽降自天,亦柄國之臣有以誤之也。



 當天慶而後,政歸後族。奉先沮天祚防微之計,陷晉王非罪之誅,夾山之禍已見於此矣。處溫逼魏王以僭號,結宋將以賣國,跡其奸佞,如出一軌。嗚呼!天祚之所倚毗者若此,國欲不亡,得乎?張琳娖娖守位,余睹反覆自困,則又何足議哉!」



\end{pinyinscope}