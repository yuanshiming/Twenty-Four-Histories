\article{卷六十志第二十九 食貨志下}

\begin{pinyinscope}

 征商之法,則自太祖置羊城於炭山北,起榷務以通諸道市易。太宗得燕,置南京,城北有市,百物山待,命有司治其徵;餘四京及它州縣貨產懋遷之地,置亦如之。東平郡城中置看樓,分南、北市,禺中交易市北,午漏下交易市南。雄州、高昌、渤海亦立互市,以通南宋、西北諸部、高麗之貨、故女直以金、帛、布、蜜、蠟諸藥材及鐵離、靺鞨、
 於闕等部以蛤珠、青鼠、貂鼠、膠魚之皮、牛羊駝馬、毳罽等物,來易於遼者,道路繈屬。聖宗統和初燕京留守司言,民艱食,請馳居庸開稅,以通山西糴易。又令有司諭諸行宮,布帛短狹不中尺度者,不鬻於市。明年,詔以南、北府市場人少,宜率當部車百乘赴集。一奇峰路以通易州貿易。二十三年,振武軍及保州並置榷場。時北院大王耶律室魯以俸羊多闕,部人貧乏,請以羸老之羊及皮毛易南中之絹,上下為便。至天祚之亂,賦斂既重,交易法壞,財日匱而民日困矣。



 鹽筴之法,則自太祖以所得漢民數多,即八部中分古漢城別為一部治之。城
 在炭山南,有鹽池之利,即後魏滑鹽縣也,八部皆取食之。及徵幽、薊還,次於鶴剌濼,命取鹽給軍。自後濼中鹽益多,上下足用。會同初,太宗有大造於晉,晉獻十六州地,而瀛、莫在焉,始得河間煮海之利,置榷鹽院於香河縣,於是燕、雲迤北暫食滄鹽。一時產鹽之地如渤海、鎮城、海陽、豐州、陽洛城、廣濟湖等處,五京計司各以其地領之。



 其煎取之制,歲出之額,不可得而詳矣。



 坑治,則自太祖始並室韋,其地產銅、鐵、金、銀,其人善作銅、鐵器。又有曷術部者多鐵;「曷術」,國語鐵也。部置三治:曰柳濕河,曰三黜古斯,曰手山。神冊初,平渤海,得廣州,本渤海鐵
 利府,改曰鐵利州,地亦多鐵。東平縣本漢襄平縣故地,產鐵礦,置採煉者三百戶,隨賦供納。以諸坑冶多在國東,故東京置戶部司,長春州置錢帛司。太祖征幽、薊,師還,次山麓,得銀、鐵礦,命置冶。聖宗太平間,於黃河北陰山及潦河之源,各得金、銀礦,興治採煉。自此以訖天祚,國家皆賴其利。



 鼓鑄之法,先代撒剌的為夷離堇,以土產多銅,始造錢幣。



 太祖其子,襲而用之,遂致富強,以開帝業。太宗置五冶太師,以總四方錢鐵。石敬瑭窒籽乇咚員婦焜怠>白諞躍汕蛔閿謨茫賈嘈慮昧韃肌Jプ讜浯蟀採劍×跏毓
 饉厙⒅釵寮撲荊鷩嬤角戮苫ビ謾S墑槍抑蒎樸蛑小K醞澈統瞿誆厙湍暇┲罹盡?┲校畹潰斗Π儺眨械渲誓信鷩苽蚣⼵找允模徽劬。垢改浮C克甏呵錚怨僨琪轄浚皇ざ啵識┧燎迥惺加謾J鞘保盥凡壞沒跬苑浪街紙羧牖傖劍ㄒ嫜弦印5雷謚潰興牡齲涸幌逃海淮罌擔淮蟀玻皇俾。砸蚋腦酌F淙⼢謾㈩轡匏縕肌?br>第詔楊遵勖征啟部司逋戶舊錢,得四十餘萬繈,拜樞密直學士;劉伸為戶部使,歲入羨餘錢三十萬繈,抉南院樞密使;其以災沴,出錢以振貧乏及諸宮
 分邊戍人戶。是時,雖未有貫朽不可較之積,亦可謂富矣。至其末年,經費浩穰,鼓鑄仍舊,國用不給。雖以海云佛寺千萬之助,受而不拒,尋禁民錢不得出境。



 天祚之世,更鑄乾統、天慶二等新錢,而上下窮困,府庫無餘積。



 始太祖為迭烈府夷離堇也,懲遙輦氏單弱,於是撫諸部,明賞罰,不妄征討,因民之利而利之,群牧蓄息,上下給足。



 及即位,伐河東,下代北郡縣,獲牛、羊、駝、馬十餘萬。樞密使耶律斜軫計女直,復獲馬二十餘萬,分牧水草便地,數歲所增不勝算。當時,括富人馬,不加多,賜大、小鶻軍萬餘疋,不加少,蓋畜牧有法然也。咸雍五年,蕭陶
 隗為馬群太保,上書猶言群牧名存實亡,上下相欺,宜括實數以為定籍。闕後東丹國歲貢千疋,女直萬疋,直不古等國三百疋,阻卜及吾獨婉、惕德各二萬疋,西夏、室韋各三百疋,越里篤、剖阿里、奧裏米、薄奴里、鐵驪等諸部三百疋;人禁朔州路羊馬入宋,吐渾、黨項馬鬻於夏。以故群牧滋繁,數至百有餘萬,諸司牧官以次進階。自太祖及興宗垂二百年,群牧之盛如一日。天祚初年,馬猶有數萬群,每群不下千疋。祖宗舊制,常選南征馬數萬疋,牧於雄、霸、清、滄間,以備燕、雲緩急;復選數萬,給四時游畋;餘則分地以牧。法至善也。至末年,累與金戰,番
 漢戰馬損十六七,雖增價數倍,竟無所買,乃冒法賣官馬從軍。諸君牧私賣日多,畋獵亦不足用,遂為金所敗。棄眾播遷,以訖於亡。松漠以北舊馬,皆為大石林牙所有。



 遼之食貨其可見者如是耳。至於鄰國歲幣,諸屬國見貢土宜,雖累朝軍國經費多所仰給,然非本國所出,況名數已見《本紀》,茲不復載。



 夫冀北宜馬,海濱宜鹽,無以議為。遼地半沙磧,三時多寒,春秋耕獲及其時,黍稢高下因其地,蓋不得與中土同矣。



 然而遼自初年,農穀充羨,振饑恤難,用不少靳,旁及鄰國,沛然有餘,果何道而致其利歟?此無化,勸課得人,規措有法故也。世之論
 錢幣者,恆患其重滯之難致,鼓鑄之弗給也,於是楮幣權宜之法興焉。西北之通舟楫,比之東南,十才一二。遼之方盛,貨泉流衍,國用以殷,給戍賞片,賜與億萬,未聞有所謂楮幣也,又何道而致其便歟?此無他,舊儲新鑄,並聽民用故也。



 孟子曰:「周於利者,兇年不能殺。」人力茍至,一夫猶足以勝時災,況為國乎。以是知善謀國者,有道以制天時、地利之宜,無往而不遂其志。食莫大於谷,貨莫大於錢,特志二者,以表遼初用事之臣,亦善裕其國者矣。



\end{pinyinscope}