\article{卷二十本紀第二十 興宗三}

\begin{pinyinscope}

 十六年春正月己卯,如混同江。



 二月庚申,如魚兒濼。辛酉,禁群臣遇宴樂奏請私事。詔世選之官,從各部著舊擇材能者用之。



 三月丁亥,如黑水濼。癸巳,遣使審決雙州囚。壬寅,大雪。



 夏四月乙巳朔,皇太后不豫,上馳往視疾。丙午,皇太后愈,復如黑水濼。丁卯,肆赦六月戊申,清暑永安山。丁巳,阻卜大王屯禿古斯來朝,獻方物。戊午,
 詔士庶言事。秋七月辛卯,幸慶州。自是月至於九月,日射獵於楚不溝霞列、系輪、石塔諸山。



 冬十月辛亥,幸中京謁祖廟。丙辰,定公主行婦禮於舅姑儀。庚午,鐵驪仙門來朝,以始人貢,加右監門衛大將軍。



 十一月戊寅,祠木葉山,幸中京,朝皇太后。壬辰,禁漏洩宮中事。



 十二月辛丑朔,女直遣使來貢。辛亥,謁太祖廟,觀《太宗收晉圖》。癸丑,問安皇太后。乙卯,以太后愈,雜犯死罪減一等論,徒以下免。庚申,南府宰相杜防、韓紹榮奏事有誤,各以大杖決之。出防為武定軍節度使。壬戌,高麗遣使來貢。



 十七年春正月丁亥,如春水。



 閏月癸丑,射虎於候里吉。



 二月辛巳,振瑤穩、嘲穩部。是月,詔士庶言國家利便,不得及己事;奴婢所見,許白其主,不得自陳。夏國王李元昊薨,其子諒祚遣使來告,即遣永興宮使耶律褭里、右護衛太保耶律興老、將作少監王全慰奠。



 三月癸卯,以同知南京留守事蕭塔烈葛為左夷離畢,知右夷離畢事唐古為右夷離畢。丙午,夏國李諒祚遣使上其父元昊遺物。丁卯,鐵不得國使來,乞以本部軍助攻夏國,不許。



 夏四月辛未,武定軍節度使杜防復為南府宰相。丙子,高麗遣使來貢。甲申,蒲盧毛朵部大王蒲輦以造舟
 人來獻。



 六月庚辰,阻卜獻馬、駝二萬。辛卯,長白山太師柴葛、回跋部太師撒刺都來貢方物。



 秋七月丁未,於越摩梅欲之子不葛一及婆離八部夷離堇虎(番去)等內附,甲寅,錄囚,減雜犯死罪。



 八月丙戌,復南京貧戶租稅。戊子,以殿前都點檢耶律義先為行軍都部署,忠順軍節度使夏行美副部署,東北面詳穩耶律術者為監軍,伐蒲奴裡酋陶得里。



 冬十月甲申,南院大王耶律韓八薨。甲午,駐蹕獨盧金。



 十一月乙未朔,遣使括馬。丁巳,賜皇太弟重元金券。封皇子和魯斡為越王,阿璉許王,忠順軍節度使謝家奴陳王,西京留守貼不漢王,惕隱旅墳
 遼西郡王,行宮都部署別古得柳城郡王,奉陵軍節度使侯古饒樂郡王,安定郡王涅魯古進封楚王。



 十八年春正月甲午朔,日有食之。戊戌,留夏國賀正使不遣。己亥,遣北院樞密副使蕭惟信以伐夏告宋。辛丑,錄囚。



 丙午,如鴛鴦濼。丙辰,獵霸特山。耶律義先奏蒲奴裡捷。二月庚辰,幸燕趙國王洪基其帳視疾。乙酉,耶律義先等陶得里以獻。



 三月乙巳,高昌國遣使來貢。壬子,以洪基疾愈,赦雜犯死罪以下。丁巳,烏古遣使送款。



 夏四月癸酉,以南府宰相耶律高十為商京統軍使。



 五月甲辰,五國酋長各率其部來附。庚戌,回跋部長兀迭臺扎
 等來朝。戊午,五國節度使耶律仙童以降烏古叛人,授左監門衛上將軍。



 六月壬戌朔,以韓國王蕭惠為河南道行軍都統,趙王蕭孝友、漢王貼不副之。乙丑,錄囚。丙寅,行十二神纛禮。己巳,宋以遼師伐夏,遣錢逸致贐禮。庚辰,阻卜來貢馬、駝、珍玩。



 辛巳,夏國使來貢,留之不遣。丁亥,行再生禮。



 秋七月戊戌,親征。



 八月辛酉朔,渡河。夏人遁,乃還。



 九月丁未,蕭惠等為夏人所敗。



 冬十月,北道行軍都統耶律敵魯古率阻卜諸軍至賀蘭山。



 獲李元昊妻及其官僚家屬,遇夏大三千來戰,殪之;烏古敵烈部都詳穩蕭慈氏奴、南克耶律斡里死焉。



 十二月戊寅,
 慶陵林木火。己卯,錄囚。有弟從兄為強盜者,兄弟俱無子,特原其弟,十九年春正月庚寅,僧惠鑒加檢校太尉。庚子,耶律敵魯古復封漆水郡王,諸將校及阻卜等部酋長各進爵有差。贈蕭慈氏奴同中書門下平章事。辛丑,遣使問罪於夏國。壬寅,如魚兒濼。二月丁亥,夏將窪普、猥貨、乙靈紀等來攻金肅城,南面林牙耶律高家奴等破之。窪普被創遁去,殺猥貨、乙靈紀。



 三月戊戌,殿前都點檢蕭迭裡得與夏戰於三角川,敗之。癸卯,命西南招討使蕭浦奴、北院大王宜新、林牙蕭撒抹等帥師伐夏,以行宮都
 部署別古得監戰。甲辰,遣同知北院樞密使蕭革按軍邊城,以為聲援。己酉,駐蹕息雞澱。丙辰,幸殿前都點檢蕭迭裡得、駙馬都尉蕭胡睹帳視疾。



 夏四月丙寅,如魚兒濼。壬申,蒲盧毛朵部惕隱信篤來貢。



 甲申,高麗遣使來貢。



 五月己丑,如涼陘。癸巳,蕭蒲奴等人夏境,不與敵遇,縱軍俘掠而還。丁酉,夏國窪普來降。己亥,遠夷拔思母部遣使來貢。



 六月丙辰朔,置倒塌嶺都監。丙寅,謁慶陵。庚午,幸慶州,謁大安殿。壬申,詔醫卜、屠販、奴隸及倍父母或犯事逃亡者,不得舉進士。回跋、曷蘇館、蒲盧毛朵部各遣使貢馬。



 甲戌,宋遣使來賀伐夏捷,高麗使俱
 至。辛巳,御金鑾殿試進土。



 秋七月壬辰,駐蹕括里浦碗。癸巳,以燕趙國王洪基北南樞密院。乙未,阻卜長豁得刺弟斡得來朝,加太尉遣之。戊戌,錄囚。戊申,以左夷離畢蕭唐古為北院樞密副使。壬子,獵候里吉。八月丁卯,阻卜酋長喘只葛拔里斯來朝。



 九月壬寅,夏入侵邊,敵魯古遣六院軍將海裏擊敗之。



 冬十月庚午,還上京。辛未,夏國王李諒祚母遣使乞依舊稱藩。使還,詔諭別遣信臣詣閥,當徐思之。壬申,釋臨潢府役徒。甲戌,如中會川。



 十一月甲午,阻卜酋長豁得刺遣使來貢。庚戌,錄囚。王子。出南府宰相韓知白為武定軍節度使,樞密副
 使楊績長寧軍節度使,翰林學士王綱澤州刺史,張宥徽州刺史,知制誥周白海北州刺史。閏月乙卯,以漢王貼不為中京留守。辛未,以同知北院樞密使事蕭革為南院樞密使,南院大王耶律仁先北樞密使事,封宋王。十二月丁亥,北府宰相、趙王蕭孝友出為東京留守,東京留守蕭塔列葛為北府宰相,南院樞密使、潞王葛為南院大王。



 庚戌,韓國王蕭惠徙封魏王,致仕。壬子,夏國李諒祚遣使上表,乞依舊臣屬。



 二十年春正月戊戌,駐蹕混同江。



 二月甲申,遣前北院都監蕭友括等使夏國,索黨項叛戶。



 己丑,如蒼耳濼。甲
 辰,吐蕃遣使來貢。



 三月壬子朔,幸黑水。



 夏五月癸丑,蕭友括等使夏還,李諒祚母表乞如黨權進馬、駝、牛、羊等物。己巳,夏國遣使求唐隆鎮及乞罷所建城邑,以詔答之。



 六月丙戌,詔以所獲李元昊妻及前後所俘夏人,安置蘇州。



 以伐夏所獲物遣使遺宋。



 秋七月,如秋山。



 九月,詔更定條制。駐蹕中會川。



 冬十月己卯朔,括諸道軍籍。



 十一月庚申,以惕隱都監蕭謨魯為左夷離畢。甲子,命東京留守司總領戶部、內省事。丁卯,罷中丞記錄職官過犯,令承旨總之。



 十二月乙酉,以皇太后行再生禮,肆赦。



 二十一年春正月辛亥,如混同江。



 二月,如魚兒濼。



 夏四月癸未,以國舅詳穩蕭阿刺為西北路招討使,封西平郡王。六月丙子,駐蹕永安山。



 秋七月甲辰朔,召北府宰相蕭塔烈葛、南府宰相漢王貼不、南院樞密使蕭革、知北院樞密使事仁失等,賜坐,論古今治道。



 戊申,祀天地。己酉,詔北、南樞密院,日再奏事。壬子,追尊太祖之祖為簡獻皇帝,廟號玄祖,祖妣為簡獻皇后;太祖之考為宣簡皇帝,廟號德祖,妣為宣簡皇后。追封太祖伯父夷離堇嚴木為蜀國王,於越釋魯為隋國王。以燕趙國王洪基為天下兵馬大元帥、知惕隱事,賜詔諭之。癸亥,近侍
 小底盧寶偽學御畫,免死,配役終身。甲子,如秋山。戊辰,謁慶陵。以南院樞密使蕭革為北院樞密使,封吳王。辛未,如慶州。壬申,追封太祖弟寅底石為許國王。



 八月戊子,太尉烏者薨,詔配享聖宗廟。九月乙卯,平州進白兔。已末,謁懷陵。庚申,追上嗣聖皇帝、天順皇帝尊溢,及更溢彰德皇后日請安。癸亥,溢齊天皇后日仁德皇后。甲子,謁祖陵。增太祖謚大聖大明神烈天皇帝,更謚貞烈皇后日淳欽,恭順皇帝日章肅,後蕭氏溢日和敬。



 冬十月戊寅,駐蹕中會川。丁亥,夏國李諒祚遣使乞弛邊備,即遣蕭友括奉詔諭之。戊子,幸顯、懿二州。甲午,遼興軍
 節度使蕭虛烈封鄭王,南院大王、潞王查葛為南院樞密使,進封越國王。戊戌,射虎於南撒葛柏。辛丑,謁乾陵。



 十一月壬寅朔,增謚文獻皇帝為文獻欽義皇帝,及謚二後日端順,日柔貞。復更謚世宗孝烈皇后為懷節。丁未,增孝成皇帝謚日孝成康靖皇帝,更謚聖神宣獻皇后為睿智。甲子,次中會川。回鶻阿薩蘭遣使貢名馬、文豹。丙寅,錄囚。



 十二月戊戌,以北府宰相塔烈葛為南京統軍使,鄭王虛烈北府宰相,契丹行宮都部署耶律義先惕隱。釋役徒限年者。



 二十二年春正月乙巳,如混同江。二月丙子,回鶻阿薩
 蘭為鄰國所侵,遣使求援。庚辰,如春水。三月癸亥,李諒祚以賜詔許降,遣使來謝。丙寅,如黑水濼。



 夏四月戊子,獵鶴澱。



 五月壬寅,詔內地州縣植果。



 六月壬申,駐蹕胡呂山。癸未,高麗遣使來貢。



 秋七月己酉,阻卜大王屯禿古斯率諸部長獻馬、駝。庚申,如黑嶺。



 閏月庚午,烏古來貢。癸巳,長春州置錢帛司。



 九月壬辰,夏國李諒祚遣使進降表。甲午,遣南面林牙高家奴等奉詔撫諭。



 冬十月丙申朔,日有食之。



 十一月辛卯,詔諸職事官以禮受代及以罪去者置籍,歲申樞密院。



 十二月丙申朔,詔回鶻部副使以契丹人充。庚子,應聖節,曲赦徒以下罪。壬子,
 詔大臣日:「朕與宋主約為兄弟,歡好歲久,欲見其繪像,可諭來使。」



 二十三年春正月乙巳,如混同江。癸酉,獵雙子澱。戊子,夏國遣使貢方物。壬辰,如春水。甲午,獵盤直坡。



 三月丁亥,幸皇太弟重元帳。



 夏四月癸卯,高麗遣使來貢。癸丑,獵合只忽裏。



 五月己巳,李諒祚乞進馬、駝,詔歲貢之。庚寅,駐蹕永安山。壬辰,夏國遣使來貢。



 六月丙申,如慶州。己亥,謁慶陵。壬寅,高麗王徽請官其子,詔加檢校太尉。辛亥,吐蕃遣使來貢。



 秋七月己巳,夏國李諒祚遣使來求婚。甲戌,如秋山。己卯,詔八房族巾幘。



 九月庚寅,獵,遇
 三虎,縱火獲之。



 冬十月丁酉,駐蹕中京。戊戌,幸新建秘書監。辛丑,有事於祖廟。癸丑,以開泰寺鑄銀佛像,曲赦在京囚。丙辰,李諒祚遣使進誓表。



 十一月乙丑,阻卜部長來貢。壬申,帝率群臣上皇太后尊號日仁慈聖善欽孝廣德安靜貞純懿和寬厚崇覺儀天皇太后,大赦。內外官進級有差。癸未,錄囚。甲申,群臣上皇帝尊號日欽天奉道祐世興歷武定文成聖神仁孝皇帝,冊皇后蕭氏曰貞懿慈和文惠孝敬廣愛崇聖皇后。



 十二月丙申,如中會川。



 二十四年春正月癸亥,如混同江。戊辰,朝皇太后。辛巳,
 宋遣使來賀,饋馴象。



 二月己丑朔,召宋使釣魚、賦詩。癸巳,如長春河。甲寅,夏國遣使來賀。



 三月癸亥,皇太弟重元生子,曲赦行在及長春、鎮北二州徒以下罪。



 夏五月,駐蹕南崖。



 秋七月壬午,如秋山。次雨崖之北峪,不豫。



 八月丁亥,疾大漸,召燕趙國王洪基,諭以治國之要。戊子,大赦,縱五坊鷹鶻,焚鉤魚之具。己丑,帝崩於行宮,年四十。遺詔燕趙國王洪基嗣位。清寧元年十月庚子,上尊謚為神聖孝章皇帝,廟號興宗。



 贊曰:興宗即位年十有六矣,不能先尊母後而尊其,以致臨朝專政,賊殺不辜,又不能以禮幾諫,使齊天死
 於斌逆,有虧王者之孝,惜哉!若夫大行在殯,飲酒博鞠,疊見簡書。及其謁遺像而哀慟,受宋吊而衰絰,所為若出二人。何為其然歟?至於感富弼之言而申南宋之好,許諒祚之盟而罷西夏之兵,邊鄙不聳,政治內修,親策進士,大修條制,下至士庶,得陳便宜,則求治之志切矣,於時左右大臣,曾不聞一賢之進,一事之諫,欲庶幾古帝王之風,其可得乎?雖然,聖宗而下,可謂賢君矣。



\end{pinyinscope}