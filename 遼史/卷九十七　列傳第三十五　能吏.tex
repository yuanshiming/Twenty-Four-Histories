\article{卷九十七 列傳第三十五 能吏}

\begin{pinyinscope}

 大公鼎蕭文馬人望耶律鐸魯斡楊遵勖王棠
 漢以璽書賜二千石,唐疏刺史、縣令於屏,以示獎率,故二史有《循吏》、《良吏》之傳。



 遼自太祖創業,太宗撫有燕、薊,任賢使能之道亦略備矣。



 然惟朝廷參置國官,吏州縣者多遵唐制。歷世既久,選舉益嚴。



 時又分遣重臣巡行境內,察賢否而進退之。是以治民、理財、決獄、弭盜,各有其人。考其德政,雖未足以與諸循、良之列,抑亦可謂能
 吏矣。作《能吏傳》。



 大公鼎,渤海人,先世籍遼陽率賓縣。統和間,徙遼東豪右以實中京,因家於大定。曾祖忠,禮賓使。父信,興中主簿。



 公鼎幼莊願,長而好學。咸雍十年,登進士第,調沈州觀察判官。時遼東雨水傷稼,北樞密院大發瀕河丁壯以完堤防。



 有司承令峻急,公鼎獨曰:「邊障甫寧,大興役事,非利國便農之道。」乃疏奏其事。朝廷從之,罷役,水亦不為災。瀕河千里,人莫不悅。改良鄉令,省徭役,務農桑,建孔子廟學,部民服化。累遷興國軍節度副使。



 時有隸鷹坊者,以羅畢為名,擾害田里。歲久,民不堪。



 公鼎言於
 上,即命禁戢。會公鼎造朝,大臣諭上嘉納之意,公鼎曰:「一郡獲安,誠為大幸;他郡如此者眾,願均其賜於天下。」從之。徙長春州錢帛都提點。車駕如春水,貴主例為假貸,公鼎曰:「豈可輟官用,徇人情?」拒之。頗聞怨詈語,曰:「此吾職,不敢廢也。」俄拜大理卿,多所平反。



 天祚即位,歷長寧軍節度使、南京副留守,改東京戶部使。



 時盜殺留守蕭保先,始利其財,因而倡亂。民亦互生猜忌,家自為斗。公鼎單騎行郡,陳以禍福,眾皆投兵而拜曰:「是不欺我,敢弗聽命。」安輯如故。拜中京留守,賜貞亮功臣,乘傳赴官。時盜賊充斥,有遇公鼎於路者,即叩馬乞自新。公鼎
 給以符約,俾還業,聞者接踵而至。不旬日,境內清肅。天祚聞之,加賜保節功臣。時人心反側,公鼎慮生變,請布恩惠以安之,為之肆赦。



 公鼎累表乞歸,不許。會奴賊張撒八率無賴嘯聚,公鼎欲擊而勢有不能。嘆曰:「吾欲謝事久矣。為世故所牽,不幸至此,豈命也夫!」因憂憤成疾。保大元年卒,年七十九。



 子昌齡,左承制;昌嗣,洺州刺史;昌朝,鎮寧軍節度。



 蕭文,字國華,外戚之賢者也。父直善,安州防御史。



 文篤志力學,甚慍不形。大康初,掌秦越國王中丞司事,以才幹稱。尋知北面貼黃。王邦彥子爭蔭,數歲不能定,有司
 以聞。上命文詰之,立決。車駕將還宮,承詔閱習儀衛,雖執事林林,指顧如一。遷同知奉國軍節度使,歷國舅都監。



 壽隆末,知易州,兼西南面安撫使。高陽土沃民富,吏其邑者,每黷於貨,民甚苦之。文始至,悉去舊弊,務農桑,崇禮教,民皆化之。時大旱,百姓憂甚,文禱之輒雨。屬縣又蝗,議捕除之,文曰:「蝗,天災,捕之何益!」但反躬自責,蝗盡飛去;遺者亦不食苗,散在草莽,為烏鵲所食。會霪雨不止,文復隨禱而齊。是歲,大熟。朝廷以文可大用,遷唐古部節度使,高陽勒石頌之。後不知所終。



 馬人望,字儼叔,高祖胤卿,為石晉青州刺史,太宗兵至,
 堅守不降。城破被執,太宗義而釋之,徙其族於醫巫閭山,因家焉。曾祖廷煦,南京留守。祖淵,中京副留守。父詮,中京文思使。



 人望穎悟。幼孤,長以才學稱。咸雍中,第進士,為松山縣令。歲運澤州官炭,獨役松山,人望請於中京留守蕭吐渾均役他邑。吐渾怒,下吏,擊幾百日;復引詰之,人望不屈。蕭喜曰:「君為民如此,後必大用。」以事聞於朝,悉從所請。



 徙知涿州新城縣。縣與宋接境,驛道所從出。人望治不擾,吏民畏愛。近臣有聘宋還者,帝間以外事,多薦之,擢中京度支司鹽鐵判官。轉南京三司度支判官,公私兼裕。遷警巡使。



 京城獄訟填委,人望處決,
 無一冤者。會檢括戶口,末兩旬而畢。同知留守蕭保先怪而問之,人望曰:「民產若括之無遺,他日必長厚斂之弊,大率十得六七足矣。」保先謝曰:「公慮遠,吾不及也。」



 先是,樞密使乙辛竊弄威柄,卒害太子。及天祚嗣位,將報父仇,選人望與蕭報恩究其事。人望平心以處,所活甚眾。



 改上京副留守。會劇賊趙鐘哥犯闕,劫宮女、御物,人望率眾捕之。右臂中矢,炷以艾,力疾馳逐,賊棄所掠而遁。人望令關津譏察行旅,悉獲其盜。尋擢樞密都承旨。



 宰相耶律儼惡人望與己異,遷南京諸宮提轄制置。歲中,為保靜軍節度使。有二吏兇暴,民畏如虎。人望假以
 辭色,陰令發其事,黥配之。是歲諸處饑乏,惟人望所治粒食不闕,路不鳴桴。遙授彰義軍節度使。遷中京度支使,始至,府廩皆空;視事半歲,積粟十五萬斛,錢二十萬繦。徙左散騎常侍,累遷樞密直學士。



 未幾,拜參知政事,判南京三司使事。時錢粟出納之弊,惟燕為甚。人望以縑帛為通歷,凡庫物出入,皆使別籍,名曰「臨庫」。奸人黠吏莫得軒輊,乃以年老揚言道路。朝論不察,改南院宣徽使,以示優老。逾年,天祚手書「宣馬宣徽」四字詔之。既至,諭曰:「以卿為老,誤聽也。」遂拜南院樞密使。



 人不敢干以私,用人必公議所當與者。如曹勇義、虞仲文嘗為奸
 人所擠,人望推薦,皆為名臣。當時民所甚患者,驛遞、馬牛、旗鼓、鄉正、廳隸、倉司之役,至破產不能給。人望使民出錢,官自募役,時以為便。久之請老,以守司徒、兼侍中致仕。卒,談曰文獻。



 人望有操守,喜怒不形,未嘗附麗求進。初除執政,家人賀之。人望愀然曰:「得勿喜,失勿憂。抗之甚高,擠之必酷。」



 其畏慎如此。



 耶律鐸魯斡,字乙辛隱,季父房之後。廉約重義。



 重熙末,給事誥院。咸雍中,累遷同知南京留守事。被召,以部民懇留,乃賜誤褒獎。大康初,改西南面招討使,為北面林牙,遷左夷離畢。大安五年,拜商府宰相。壽隆初,致仕,卒。



 鐸魯斡所至有聲,吏民畏愛。及退居鄉里,子普古為烏古部節度使,遣人來迎。既至,見積委甚富。謂普古曰:「辭親入仕,當以裕國安民為事。枉道欺君,以茍貨利,非吾志也。」



 命駕而歸。普古後為盜所殺。楊遵勖,字益誡,涿州範陽人。重熙十九年登進士第,調儒州軍事判官,累遷樞密院副承旨。



 咸雍三年,為宋國賀正使;還,遷都承旨。天下之事,叢於樞府,簿書填委。遵勖一目五行俱下,剖決如流,敷奏詳敏。



 上嘉之。奉詔徵戶部逋錢,得四十餘萬緡,拜樞密直學士,改樞密副使。大康初,參知政事,徙知樞密院事,兼門下侍郎、平章事,
 拜南府宰相。耶律乙辛誣皇太子,詔遵勖與燕哥按其事,遵勖不敢證言,時議短之。尋拜北府宰相。



 大安中暴卒,年五十六。贈守司空,謚康懿。子晦,終昭文館直學士。



 王棠,涿州新城人。博古,善屬文。重熙十五年擢進士。



 鄉貢、禮部、廷試對皆第一。



 累遷上京鹽鐵使。或誣以賄,無狀,釋之。遷東京戶部使。



 大康二年,遼東饑,民多死,請賑恤,從之。三年,入為樞密副使,拜南府宰相。大安末,卒。



 棠練達朝政,臨事不怠,在政府修明法度,有聲。



 論曰:「孟子謂『民為貴,社稷次之』,司牧者當如何以盡心。公鼎奏罷完堤役以息民,拒公讓假貸以守法,單騎行
 郡,化盜為良,庶幾召、杜之美。文知易州,雨暘應禱,蝗不為災。



 人望為民不避囚擊,判度文,公私兼裕,亦卓乎未易及已。鐸魯斡吏畏民愛,楊遵勖決事如流,真能吏哉。」



\end{pinyinscope}