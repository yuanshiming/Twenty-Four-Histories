\article{史卷六十六 列傳第四}

\begin{pinyinscope}

 耶律敵刺蕭痕篤康默記孫延壽韓延徽子德樞孫紹勛紹芳資讓韓知古子匡嗣孫德源德凝耶律敵刺,字合魯隱,遙輦鮮質可汗之子。太祖踐昨,與敞穩海裏同心輔政。太祖知其忠實,命掌禮儀,且諉以軍事。



 後以平內亂功,代轄里為奚六部吐里,卒。



 敵刺善騎射,頗好禮文。



 蕭痕篤,字兒里軫,迭刺部人。其先相遙輦氏。



 痕篤少糠慨,以才能自任。早隸太祖帳下,數從征討。既踐陣,除北府宰相。痕篤事親孝,為政尚寬簡。



 康默記,本名照。少為薊州衙校,太祖侵薊州得之,愛其材,隸麾下。一切蕃、漢相涉事,屬默記折衷之,悉合上意。



 時諸部新附,文法未備,默記推析律意,論決重輕,不差毫厘。罹禁綱者,人人自以為不冤。頃之,拜左尚書。神冊三年,始建都,默記董役,人咸勸趨,百日而訖事。五年,為皇都夷離畢。會太祖出師居庸關,命默記將漢軍進逼長蘆水寨,俘馘甚眾。



 天贊四年,親征渤海,默記與韓知
 古從。後大諲撰叛,命諸將攻之。默記分薄東門,率驍勇先登。既拔,與韓延徽下長嶺府。軍還,已下城邑多叛,默記與阿古只平之。



 既破回跋城,歸營太祖山陵畢,卒。佐命功臣其一也。



 孫延壽,字胤昌,少倜儻,調其所親:「大丈夫為將,當效節邊垂,馬革裹尸。」景宗特授幹牛衛大將軍。宋人攻商京,諸將既成列,延壽獨奮擊陣前,敵遂大潰。以功遙授保大軍節度使。乾亨三年卒。



 韓延徽,字藏明,幽州安次人。父夢殷,累官薊、儒、順三州刺史。延徽少英,燕帥劉仁恭奇之,召為幽都府文學、平州錄事參軍,同馮道祗候院,授幽州觀察度支使。



 後守
 光為帥,延徽來聘,太祖怒其不屈,留之。述律後諫曰:「彼秉節弗撓,賢者也,奈何困辱之?」太祖召與語,合上意,立命參軍事。攻黨項、室韋,服諸部落,延徽之籌居多。



 乃請樹城郭,介市里,以居漢人之降者。又為定配偶,教墾藝,以生養之。以故逃亡者少。



 居久之,慨然杯其鄉里,賦詩見意,遂亡歸唐。已而與他將王緘有隙,懼及難,乃省親幽州,匿故人王德明舍。德明問所適,延徽曰:「吾將復走契丹。」德明不以為然。延徽笑曰:「彼失我,如失左右手,其見我必喜。」既至,太祖問故。



 延徽曰:「忘親非孝,棄君非忠。臣雖挺身逃,臣心在陛下。臣是以復來。」上大悅,賜名曰
 匣列。「匣列。」遼言復來也。



 即命為守政事令、崇文館大學士,中外事悉令參決。



 天贊四年,從征渤海,大諲撰乞降。既而復叛,與諸將破其城,以功拜左僕射。又與康默記攻長嶺府,拔之。師還,太祖崩,哀動左右。



 太宗朝,封魯國公,仍為政事令。使晉還,改南京三司使。



 世宗朝,遷南府宰相,建政事省,設張理具,稱盡力吏。



 天祿五年六月,河東使請行冊禮,帝詔延徽定其制,延徽奏一遵太宗冊晉帝禮,從之。



 應歷中,致仕。子德樞鎮東平,詔許每歲東歸省。九年卒,年七十八。上聞震悼,贈尚書令,葬幽州之魯郭,世為崇文令公。



 初,延徽南奔,太祖夢白鶴自帳中
 出;比還,復入帳中。



 詰旦,謂侍臣曰:「延徽至矣。」已而果然。太祖初元,庶事草創,凡營都邑,建宮殿,正君臣,定名分,法度井井,延徽力也。為佐命功臣之一。子德樞。



 德樞年甫十五,太宗見之,謂延徽曰:「是兒卿家之福,朕國之寶,真英物也!」未冠,守左羽林大將軍,遷特進太尉。



 時漢人降與轉徙者,多寓東平。丁歲災,饑饉疾厲。德樞請往撫字之,授遼興軍節度使。下車整紛剔蠹,恩煦信孚,勸農桑,興教化,期月民獲薊息。



 入為南院宣徽使,遙授天平軍節度使,平、灤、營三州管內觀察處置等使,門下平章事。已而加開府儀同三司、行侍中,封趙國公。保寧元年
 卒。孫紹勛、紹芳。



 紹勛,仕至東京戶部使。會大延琳叛,被執,辭不屈,賊以鋸解之,憤罵至死。



 紹芳,重熙間參知政事,加兼侍中。時廷議征李元昊,力諫不聽,出為廣德軍節度使。聞敗,嘔血卒。



 孫資讓,壽隆初拜中書侍郎、平章事。會宋徽宗嗣位,遣使來報,有司按籍,有「登寶位」文,坐是出為崇義軍節度使。



 改鎮遼興,卒。



 韓知古,薊州玉田人,善謀有識量。太祖平薊時,知古六歲,為淳欽皇后兄欲穩所得。後來嬪,知古從焉,未得省見。



 久之,負其有,怏怏不得志,挺身逃庸保,以供資用。



 其子匡嗣得親近太祖,因間言。太祖召見與語,賢之,命參
 謀義。神冊初,遙授彰武軍節度使。久之,信任益篤,總知漢兒司事,兼主諸國禮儀。時儀法疏闊,知古援據故典,參酌國俗,與漢儀雜就之,使國人易知而行。



 頃之,拜莊僕射,與康默記將漢軍征渤海有功,遷中書令。



 天顯中卒,為佐命功臣之一。子匡嗣。



 匡嗣以善醫,直長樂宮,皇後視之猶子。應歷十年,為太祖廟詳穩。後宋王甚隱謀叛,辭引匡嗣,上置不問。



 初,景宗在藩邸,善匡嗣。即位,拜上京留守。頃之,王燕,改南京留守。保寧末,以留守攝樞密使。



 時耶律虎古使宋還,言宋人必取河東,合先事以為備。匡嗣低之曰:「寧有是!」已
 而宋人果取太原,乘勝通燕。匡嗣與商府宰相沙、惕隱休哥侵宋,軍於滿城,方陣,宋人請降。



 匡嗣欲納之,休哥曰:「彼軍氣甚銳,疑誘我也。可整頓士卒以御。」匡嗣不聽。俄而宋軍鼓噪薄我,眾蹙踐,塵起漲天。



 匡嗣倉卒諭諸將;無當其鋒。眾既奔,遇伏兵扼要路,匡嗣棄旗鼓遁,其眾走易州山,獨休哥收所棄兵械,全軍還。



 帝怒匡嗣,數之曰:「爾違眾謀,深入敵境,爾罪一也;號令不肅,行伍不整,爾罪二也;棄我師旅,挺身鼠竄,爾罪三也;偵候失機,守禦弗備,爾罪四也;捐棄旗鼓,損威辱國,爾罪五也。」促令誅之。皇后引諸內戚徐為開解,上重違其請。



 良久,威
 稍霽,乃杖而免之。



 既而遙授晉昌軍節度使。乾亨三年,改西南面招討使,卒。



 睿智皇后聞之,遣使臨吊,賄贈甚厚,後追贈尚書令。五子:德源,德讓——後賜名隆運,德威,德崇,德凝。德源、德凝附傳,餘各有傳。



 德源,性愚而貪,早侍景宗邸。及即位,列近侍。保寧間,官崇義、興國二軍節度使,加檢校太師。以賄名,德讓貽書諫之,終不俊。以故論者少之。後加同政事不下平章事,遙攝保寧軍節度使。乾亨初卒。



 德凝,謙遜廉謹。保寧中,遷護軍司徒。開泰中,累遷護衛太保、都官使、崇義軍節度使。移鎮廣德,秩滿,部民請留,
 從之。改西南面招討使,黨項隆益答叛,平之。遷大同軍節度使,卒於官。



 子郭三,終天德軍節度使。孫高家奴,終南院宣徽使;高十,終遼興軍節度使。



\end{pinyinscope}