\article{卷七十八 列傳第十六}

\begin{pinyinscope}

 耶律合住劉景劉六符耶律褭履牛溫舒杜防蕭和尚弟特末耶律合里只耶律頗的耶律合住,字粘袞,太祖弟迭刺之孫。幼不好弄,臨事明敏,善談論。



 初以近族入侍,每從征伐有功。保寧初,加右
 龍虎衛上將軍。以宋師屢梗南邊,拜涿州刺史,西南兵馬都監、招安、巡檢等使,賜推忠奉國功臣。



 合住久任邊防,雖有克獲功,然務鎮靜,不妄生事以邀近功。鄰壤敬畏,屬部乂安。宋數遣人結歡,冀達和意,合住表聞其事,帝許議和,安邊懷敵,多有力焉。拜左金吾衛上將軍。



 秩滿,遙懾鎮國軍節度使,卒。



 合住智而有文,曉暢戎政。鎮範陽時,嘗領數騎徑詣雄州北門,與郡將立馬陳兩國利害,及周師侵邊本末。辭氣慷慨,左右壯之。自是,邊境數年無事。識者以謂合住一言,賢於數十萬兵。



 劉景,字可大,河間人。四世祖怦,即朱滔之甥,唐右僕射、
 盧龍軍節度使。父守敬,南京副留守。



 景資端厚,好學能文。燕王趙延壽闢為幽都府文學。應歷初,遷右拾遺、知制誥,為翰林學士。九年,周人侵燕,留守蕭思溫上急變,帝欲俟秋出師,景諒曰:「河北三關已陷於敵,今復侵燕,安可坐視!」上不聽。會父憂去。未幾,起復舊職。



 一日,召草赦;既成,留數月不出。景奏曰:「唐制,赦書日行五百里,今稽期弗發,非也。」上亦不報。



 景宗即位,以景忠實,擢禮部侍郎,遷尚書、宣政殿學士。



 上方欲倚用,乃書其勿曰:「劉景可為宰相。」頃之,為南京副留守。時留守韓匡嗣因扈從北上,景與其子德讓共理京事。



 俄召為戶部使,歷武
 定、開遠二軍節度使。



 統和六年致仕,加兼侍中。卒,年六十七。贈太子太師。



 子慎行,孫一德、二玄、三嘏、四端、五常、六符,皆具《六行傳》。



 劉六符,父慎行,由膳部員外郎累遷至北府宰相、監修國史。時上多即宴飲行誅賞,慎行諫曰:「以喜怒加威福,恐末當。」帝悟,諭政府「自今宴飲有刑賞事,翌日稟行」。為都統,伐高麗,以失軍期下吏,議貴乃免,出為彰武軍節度使。



 賜保節功臣。子六人:一德、二玄、三嘏、四端、五常、六符。



 德早世。玄終上京留守。常歷三司使、武定軍節度使。嘏、端、符皆第進士。嘏、端俱尚主,為駙馬都尉。三嘏獻聖
 宗《一矢斃雙鹿賦》,上喜其贍麗。與公主不諧,奔宋;歸,殺之。四端以衛尉少卿使宋賀生辰,方宴,大張女樂,竟席不顧,人憚其產。還,拜樞密直學士。



 六符有志操,能文。重熙初,遷政事舍人,擢翰林學士。



 十一年,與宣徽使蕭特末使宋索十縣地;還,為漢人行宮副部署。會宋遣使增歲幣以易十縣,復與耶律仁先使宋,定「進貢」



 名,宋難之。六符曰:「本朝兵強將勇,海內共知,人人願從事於宋。若恣其俘獲以飽所欲,與『進貢』字孰多?況大兵駐燕,萬一南進,何以御之!顧小節,忘大患,悔將何及!」



 宋乃從之,歲幣稱「貢」。六符還,加同中書門下平章事。及宋幣至,命六
 符為三司使以受之。



 六符與參知政事杜防有隙,防以六符嘗受宋賂,白其事,出為長寧軍節度使,俄召為三司使。



 道宗即位,將行大冊禮,北院樞密使蕭革曰:「行大禮備儀物,必擇廣地,莫若黃川。」六符曰:「不然。禮儀國之大體,帝王之樂不奏於野。今中京四方之極,朝覲各得其所,宜中京行之。」上從其議。尋以疾卒。



 耶律褭履,字海鄰,六院夷離堇蒲古只之後。風神爽秀,工於畫。



 重熙間,累遷同知點檢司事。駙馬都尉蕭胡睹為夏人所執,奉詔索之,三返以歸,轉永興宮使、右祗候郎君班詳穩。褭履將娶秦晉長公主孫,其母與公主婢
 有隙,謂褭履曰:「能去婢,乃許爾婚。」褭履以計殺之,婚成。事覺,有司以大闢論。褭履善畫,寫聖宗真以獻,得減,坐長流邊戍。復以寫真,召拜問知南院宣徽事。使宋賀正,寫宋主客以歸。



 清寧間,復使宋。宋主賜宴,瓶花隔面,未得其真。陛辭,僅一視,及境,以像示餞者,駭其神妙。聞重元亂,不即勤王。賊平入賀,帝責讓之。宴酣,顧褭履曰:「重元事成,卿必得為上客!」褭履大慚。咸雍中,加太子太師,卒。



 牛溫舒,範陽人。剛正,尚節義,有遠器。



 咸雍中,擢進土第,滯小官。大安初,累遷戶部使,轉給事中、知三司使事。國、
 民兼足,上以為能,加戶部侍郎,改三司使。壽隆中,拜參知政事,兼同知樞密院事,攝中京留守。



 部民詣闕請真拜,從之。召為三司使。



 乾統初,復參知政事,知南院樞密使事。五年,夏為宋所攻,來請和解。溫舒與蕭得裏底使宋。方大燕,優人為道士裝,索土泥藥爐。優曰:「土少不能和。」溫舒遽起,以手藉土懷之。宋主間其故,溫舒對曰:「臣奉天子威命來和,若不從,則當卷土收去。」宋人大驚,遂許夏和。還,加中書令,卒。



 杜防,涿州歸義縣人。開泰五年,擢進士甲科,累遷起居郎、知制浩,人以為有宰相器。太平中,遷政事舍人,拜樞
 密副使。重熙九年,夏人侵宋。宋遣郭稹來告,請與夏和,上命防使夏解之。如約罷兵,各歸侵地,拜參知政事。韓紹芳、劉六符忌之,防待以誠。十二年,紹芳等罷,愈見信任。十三年,拜南府宰相。十五年,防生子,帝幸其第,賜名王門奴。以進奏有誤,出為武定軍節度使。十七年,復召為南府宰相。二十一年秋,祭仁德皇后,詔儒臣賦詩,防為冠,賜金帶。



 道宗諒陰,為大行皇帝山陵使。清寧二年,上諭防曰:「朕以卿年老嗜酒,不欲煩以劇務。朝廷之事,總綱而已。」頃之,拜右丞相,加尚父,卒。上歡悼不已,賵賻加等,官給葬具,贈中書令,謚曰元肅。子公謂,終南府宰相。
 蕭和尚,字洪寧,國舅大父房之後。忠直,多智略。



 開泰初,補御盞郎君,尋為內史、太醫等局都林牙。使宋賀正,將宴,典儀者告,班節度使下。和尚曰:「班次如此,是不以大國之使相禮。且以錦服為貺,如待蕃部。若果如是,吾不預宴。」宋臣不能對,易以紫服,位視執政,使禮始定。」



 八年秋,為唐古部節度使,卒。弟特末。



 特末,字何寧。為人機辨任氣。



 太平中,累遷安東軍節度使,有能稱。十一年,召為左祗候郎君班詳穩。未幾,遷左夷離畢。重熙十年,累遷北院宣徽使。明年,與劉六符使宋,索十縣故地,宋請增銀、絹十萬兩、疋以易之。歸,稱旨,加同政
 事門下平章事。詔城西南渾底甸。



 還,復為北院宣徽使,卒。



 耶律合里只,字特滿,六院夷離堇蒲古只之後。



 重熙中,累遷酉南面招討都監。充宋國生辰使,館於白溝驛。宋宴勞,優者嘲蕭惠河西之敗。合里只曰:「勝負兵家常事。我嗣聖皇帝俘石重貴,至今興中有石家寨。惠之一敗,何足較哉?」宋人慚服。帝聞之曰:「優伶失辭,何可傷兩國交好!」鞭二百,免官。



 清寧初,起為懷化軍節度使。七年,入為北院大王,封豳國公。歷遼興軍節度使、東北路詳穩,加兼侍中。致仕,卒。



 合里只明達勤恪,懷柔有道。置諸賓
 館及西邊營田,皆自合里只發之。



 耶律頗的,字撒版,季父房奴瓜之孫。孤介寡合。重熙初,補牌印郎君。清寧初,稍遷知易州。去官,部民請留,許之。咸雍八年,改彰國軍節度使。上獵大牢古山,頗的謁於行宮。帝間邊事,對曰:「自應州南境至天池,皆我耕牧之地。



 清寧間,邊將不謹,為宋所侵,烽堠內移,似非所宜。」道宗然之。拜北面林牙。後遣人使宋,得其侵地,命頗的往定疆界。



 還,拜南院宣徽使。



 大康四年,遷忠順軍節度使,尋為南院大王,改同知南京留守事,召拜南府宰相,賜貞良功臣,封吳國公,為北院樞密使。廉謹奉公,知無不
 為。大安中致仕,卒。子霞抹,北院樞密副使。



 論曰:「耶律合住安邊講好,養兵息民,其慮深遠矣。六符啟釁邀功,豈國家之利哉?牛、杜、頗的、合里只輩銜命出使,幸不辱命。褭裏殺人婢以求婚,身負罪釁,畫其主客,以冀免死,亦可醜也。」



\end{pinyinscope}