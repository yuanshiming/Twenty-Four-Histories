\article{卷七十二 列傳第十}

\begin{pinyinscope}

 張儉邢抱樸馬得臣蕭樸耶律八哥張儉,宛平人,性端愨,不事外飾。



 統和十四年,舉進士第一,調雲州幕官。故事,車駕經行,長吏當有所獻。聖宗獵雲中,節度使進曰:「臣境無他產,惟幕僚張儉,一代之寶,願以為獻。」先是,上夢四人侍側,賜食人二口,至聞儉名,
 始悟。召見,睿止樸野;訪及世務,占奏三十餘事。由此顧遇特異,踐歷清畢,號稱明千。



 開泰中,累遷同知樞密院事。太平五年,出為武定軍節度使,移鎮大同。六年,入為南院樞密使。帝方眷倚,參知政事吳叔達與儉不相能,帝怒,出叔達為康州刺史,拜儉左丞相,封韓王。帝不豫,受遺詔輔立太子,是為興宗,賜貞亮弘靖保義守節耆德功臣,拜太師、中書令,加尚父,徙王陳。



 重熙五年,帝幸禮部貢院及親試進士,皆儉發之。進見不名,賜詩褒美。儉衣唯紬帛,食不重味,月俸有餘,周給親舊。



 方冬,奏事便殿,帝見衣袍弊惡,密令近侍以火夾穿孔記之,屢見
 不易。帝問其故,儉對曰:「臣服此袍已三十年。」時尚奢靡,故以此徽諷喻之。上憐其清貧,令恣取內府物,儉奉詔持布三端而出,益見獎重。儉弟五人,上欲俱賜進士第,固辭。



 有司獲盜八人,既戮之,乃獲正賊。家人訴冤,儉三乞申理。



 上勃然曰:「卿欲朕償命耶!」儉曰:「八家老稚無告,少加存恤,使得收葬,足慰存沒矣。」乃從之。儉在相位二十餘年,裨益為多。



 致政歸第,會宋書辭不如禮,上將親征。幸儉第,尚食先往具饌,卻之;進葵羹乾飯,帝食之美。徐問以策,儉極陳利害,且曰:「第遣一使間之,何必遠勞車駕?」上悅而止。復即其第賜宴,器玩悉與之。二十二年薨,
 年九十一,敕葬宛平縣。



 邢抱樸,應州人,刑部郎中簡之子也。抱樸性穎悟,好學博古。保寧初,為政事舍人、知制誥,累遷翰林學士,加禮部侍郎。統和四年,山西州縣被兵,命抱樸鎮撫之,民始安,加戶部尚書。遷翰林學士承旨,與室昉同修《實錄》。決商京滯獄還,優詔褒美。十年,拜參知政事。以樞密使韓德讓薦,按察諸道守令能否而黜陟之,大協人望。尋以母憂去官,詔起視事。



 表乞終制,不從;宰相密諭上意,乃視事。人以孝稱。及耶律休哥留守南京,又多滯獄,復詔抱樸平決之,人無冤者。改南院樞密院,卒,贈侍中。



 初,抱
 樸與弟抱質受經於母陳氏,皆以儒術顯,抱質亦官至侍中,時人榮之。



 馬得臣,南京人,好學博古,善屬文,尤長於詩。



 保寧間,累遷政事舍人、翰林學士,常預朝議,以正直稱。乾亨初,宋師屢犯邊,命為南京副留守,復拜翰林學士承旨。



 聖宗即位,皇太后稱制,兼侍讀學士。上閱唐高祖、太宗、玄宗三《紀》,得臣乃錄其行事可法者進之。及扈從伐宋,進言降不可弒,亡不可追,二三其德者別議。詔從之。俄兼諫議大夫,知宣徽院事。



 時上擊鞠無度,上書諫曰:臣竊觀房玄齡、杜如晦,隋季書生,向不遇太宗,安能為一代名
 相?臣雖不才,陛下在東宮,幸列侍從,今又得侍聖讀,未有裨補聖明。陛下嘗間臣以貞觀、開元之事,臣請略陳之。



 臣聞庸太宗侍太上皇宴罷,則挽輦至內殿;玄宗與兄弟歡飲,盡家人禮。陛下嗣祖考之祚,躬侍太后,可謂至孝。臣更望定省之餘,睦六親,加愛敬,則陛下親親之通,比隆二帝矣。



 臣又聞二帝耽玩經史,數引公卿講學,至於日昃。故當時天下翁然響風,以隆文治。今陛下游心典籍,分解章句,臣願研究經理,深造而罵行之,二帝之治不難致矣。



 臣又聞太宗射豕,唐儉諫之;玄宗臂鷹,韓休言之;二帝莫不樂從。今陛下以球馬為樂,愚臣思
 之,有不宜者三,故不避斧鉞言之。竊以君臣同戲,不免分爭,君得臣愧,彼負此喜,一不宜。躍馬揮杖,縱械馳驚,不顧上十之分,爭先取勝,失人臣禮,二不宜。輕萬乘之尊,圖一時之樂,萬一有銜勒之失,其如社稷、太后何?三不宜。儻陛下不以臣言為迂,少賜省覽,天下之福,群臣之願也。



 書奏,帝嘉歡良久。未幾卒,贈太子太保,詔有司給葬。



 蕭樸,字延寧,國舅少父房之族。父勞古,以善屬文,為聖宗詩友。樸幼如老成人。及長,博學多智。



 開泰初,補牌印郎君,為南院承旨,權知轉運事,尋改南面林牙。帝問以
 政,樸具陳百姓疾苦,國用豐耗,帝悅曰:「吾得人矣!」擢左夷離畢。時蕭合卓為樞密使,樸知部署院事,以酒廢事,出為興國軍節度使,俄召為南面林牙。太平三年,守太子太傅。明年,拜北府宰相,遷北院樞密使。時太平日久,帝留心翰墨,始畫譜牒以別嫡庶,由是爭訟紛起。樸有吏才,能知人主意,敷奏稱旨,朝議多取決之。封蘭陵郡王,進王恆,加中書令。及大延琳叛,詔安撫東京,以便宜從事。



 興宗即位,皇太后稱制,國事一委弟孝先。方仁德皇后以馮家奴所誣被害,樸屢言其冤,不報。每念至此,為之嘔血。



 重熙初,改王韓,拜東京留守。及遷太后於慶
 州,樸徙王楚,升南院樞密使。四年,王魏。薨,年五十,贈齊王。子鐸刺,國舅詳穩。



 耶律八哥,字烏古鄰,五院部人。幼聰慧,書一覽輒成誦。



 統和中,以世業為本部吏。未幾,升閘撒狘,尋轉樞密院侍御。會宋將曹彬、米信侵燕,八哥以扈從有功,擢上京留守。



 開泰四年,召為北院樞密副使。頃之,留守東京。七年,上命東平王蕭排押帥師伐高麗,八哥為都監,至開京,大掠而還。濟茶、陀二河,高麗追兵至。諸將皆欲使敵渡兩河擊之,獨八哥以為不可,曰:「敵若渡兩河,必殊死戰,乃危道也;不若擊於兩河之間。」排押從之,戰敗績。



 明
 年,還東京,奏渤海承奉官宜有以統領之,上從其言,置都知押班。後以茶、陀之敗,削使相,降西北路都監,卒。



 論曰:「張儉名符帝夢,遂結主知。服弊袍不易,志敦薄俗。功著兩朝,世稱賢相,非過也。邢抱樸甄別守令,大愜人望。兩決滯獄,民無冤濫。馬得臣引盛唐之治以諫其君。蕭樸痛皇后之誣,至於嘔血。四人者,皆以明經致位,忠葵若此,宜矣。聖宗得人,於斯為盛。」



\end{pinyinscope}