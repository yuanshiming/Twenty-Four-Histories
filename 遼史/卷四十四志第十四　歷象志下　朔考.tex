\article{卷四十四志第十四 歷象志下 朔考}

\begin{pinyinscope}

 古者太史掌正歲年以敘事,國史以事系日,以日、月、時系年。時月不正,則敘事不一。故二史合為一官,頒歷授時,必大一統。



 遼、漢、周、宋,俱行夏時,各自為歷。國史閏朔,頗有異同。遼初用《乙未元歷》,本何承天《元嘉歷》法;後用《大明歷》,本祖沖之《甲子元歷》法。承天日食晦朏,一章必
 七閏;沖之日食必朔,或四年一閏。用《乙未歷》,漢、周多同;用《大明歷》,則間與宋異。國史敘事,甲子不殊,閏朔多異,以此故也。耶律嚴《紀》以《大明》法追正《乙未》月朔,又與陳大任《紀》時或牴牾。稽古君子,往往惑之。



 用《五代》《職方考》志契丹州軍例,作《朔考》。法殊日「異」;傳訛曰「誤」;遼史不書國,儼、十大任偏見並見各名;他史以國冠朔。並見注於後。



 宋元豐元年十二月,詔司天監考遼及高麗、日本國歷與《奉元歷》同異。遼己未歲氣朔與《宣明歷》合,日本戊午歲與遼歷相近,高麗戊午年朔與《奉元歷》合,氣有不同。戊午,遼大康四年;己未,五年也。當遼、宋之世,二國司天國相參考矣。高麗所迸《大遼事跡》,載諸王冊文,頗見月朔,因附入。



 象孟子有言:「天之高也,星辰之運也,茍求其故,千歲之日至可坐而致。」甚哉!聖人之用心,可謂廣大精微,至矣盡矣。



 日有晷景,日有明魄,鬥有建除,星有昏旦。觀天之變
 而制器以修之,八尺之表,六尺之簡,百刻之漏,日月星辰示諸掌上。運行既察,度分既審,於是像天圜以顯運行,置地櫃以驗出入,渾象是作。天道之常,尋尺之中可以俯窺,陶唐之象是矣。設三儀以明度分,管一衡以正辰極,渾儀是作。天文之變,六合之表可以仰觀,有虞之璣是矣。體莫固於金,用莫利於水。範金走水,不出戶而知天道,此聖人之所以為聖也。



 歷代儀象表漏,各具於志。太宗大同元年,得晉歷象、刻漏、渾象。後唐清泰二年己稱損析不可施用,其至中京者概可知矣。古之煉銅,黑黃白青之氣盡,然後用之,故可施於久遠。



 唐沙門一
 行鑄渾天儀,時稱精妙,未幾銅鐵漸澀,不能自轉,置不復用。金質為精,水性不行,況移之冱寒之地乎?



 刻漏晉天福三年造。《周官》摯壺氏懸壺必爨之以火。地雖冱寒,蓋可施也。



 官星吉者官星萬餘名。遭秦焚滅圖籍,世秘不傳。漢收散亡,得甘德、石申、巫咸三家圖經。經緯合千餘官,僅存什一。分為三垣、四宮、二十八宿,樞以二極,建以北斗,緯以五星,日月代明,貴而太一。賤逮屎糠。占決之用,亦云備矣。
 司馬遷《天官書》既以具錄,後世保章守候,無出三家官星之外者。



 天象昭垂,歷代不易,而漢、晉、隨、唐之書累志天文,近於衍矣。且天象機樣,律格有禁,書於勝國之史,詿誤學者,不宜書。其日食、星變、風雲、震雪之祥,具載《帝紀》,不復書。



\end{pinyinscope}