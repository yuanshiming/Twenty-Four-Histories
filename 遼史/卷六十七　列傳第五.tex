\article{卷六十七 列傳第五}

\begin{pinyinscope}

 耶律覿烈弟羽之耶律鐸臻弟古突品不王鬱耶律圖魯窘耶律覿烈,字兀里軫,六院部蒲古只夷離堇之後。父偶思,亦為夷離堇。



 初,太祖為於越時,覿烈以謹願寬恕見器使。既即位,兄曷魯典宿衛,以故賣覿烈入侍帷幄,與聞政事。神冊三年,曷魯薨,命覿烈為迭刺部夷離堇,屬以南方事。會討黨項,皇太子為先鋒,覿烈副之。軍至天德、
 雲內,分道並進。覿烈率遍師渡河力戰,斬獲甚眾。



 天贊初,析迭刺部為北、南院,南夷離堇。時大元帥率師由古北口略燕地,覿烈徇山西,所至城堡皆下,太祖嘉其功,錫齎甚厚。從伐渤海,拔扶餘城,留覿烈與寅底石守之。



 天顯二年,留守南京。十年卒,年五十六。弟羽之。



 羽之,小字兀里,字寅底曬。幼豪爽不群,長嗜學,通諸部語。太祖經營之初,多預軍謀。天顯元年,渤海平,立皇太子為東丹王,以羽之為中臺省右次相。時人心未安,左大相迭刺不逾月薨,羽之荏事勤恪,威信並行。



 太宗即位,上表曰:「我大聖天皇始有東土,擇賢輔以撫斯民,不
 以臣愚而任之。國家利害,敢不以聞。渤海昔畏南朝,阻險自衛,居忽汗城。今去上京遼邈,既不為用,又不罷戍,果何為哉?先帝因彼離心,乘釁而動,故不戰而克。天授人興,彼一時也。遺種浸以蕃息,今居遠境,恐為後患。梁水之地乃其故鄉,地衍土沃,有木鐵鹽魚之利。乘其微弱,徙還其民,萬世長策也。彼得故鄉,又獲木鐵鹽魚之饒,必安居樂業。然後選徒以翼吾左,突厥、黨項、室韋夾輔吾右,可以坐制南邦,混一天下,成聖祖未集之功,貼後世無疆之福。」表奏,帝嘉納之。是歲,詔徙東丹國民於梁水,時稱其善。



 人皇王奔唐,羽之鎮撫國人,一切如故。
 以功加守太傅,遷中臺省左相。會同初,以冊禮赴闕,加特進。表奏左次相渤海薊貪墨不法事,卒。子和里,終東京留守。



 耶律鐸臻,字敵輦,六院部人。祖蒲古只,遙輦氏時再為本部夷離堇。耶律狼德等既害玄祖,暴橫益肆。蒲古只以計誘其黨,悉誅夷之。



 鐸臻幼有志節,太祖為於越,常居左右。後即位,梁人遣使求轅軸材,太祖難之。鐸臻曰:「梁名求材,實覘吾輕重。



 宜答曰:『材之所生,必深山窮谷,有神司之,須白鼻赤驢禱詞,然後可伐。』如此,則其語自塞矣。」已而果然。



 天贊三年,將伐渤海,鐸臻諫曰:「陛下先
 事渤海,則西夏必躡吾後。請先西討,庶無後顧憂。」太祖從之。及淳欽皇后稱制,惡鐸臻,囚之,誓曰:「鐵鎖朽,當釋汝!」既而召之,使者欲去鎖,鐸臻辭曰:「鐵未朽,可釋乎?」後聞,嘉嘆,趣召釋之。天顯二年卒。弟古、突呂不。



 古,字涅刺昆,初名霞馬葛。太祖為於越,嘗從略地山右。



 會李克用於雲州,古侍,克用異之曰:「是兒骨相非常,不宜使在左右。」以故太祖頗忌之。時方西討,諸弟亂作,聞變,太祖間古與否,曰無。喜曰:「吾無忠矣!」趣召古議。古陳殄滅之策,後皆如言,以故錫齎甚厚。



 神冊未,南伐,以古佐右皮室詳穩老古,與唐兵戰於雲碧店。老古中流矢,
 傷甚,太祖疑古陰害之。古知上意,跪曰:「陛下疑臣恥居老古麾下耶?及今老古在,請遣使問之。」太祖使問老古,對曰:「臣於古無可疑者。」上意乃釋。老古卒,遂以古為右皮室詳穩。



 既卒,太祖謂左右曰:「古死,猶長松自倒,非吾伐之也。」



 突呂不,字鐸袞,幼聰敏嗜學。事太祖見器重。及制契丹大字,突呂不贊成為多。未幾,為文班林牙,領國子博士、知制浩。明年,受詔撰決獄法。



 太祖略燕,詔與皇太子及王鬱攻定州。師還至順州,幽州馬步軍指揮使王千率眾來襲,突呂不射其馬躓,擒之。天贊二年,皇子堯骨為
 大元帥,突呂不為副,既克平州,進軍燕、趙,攻下曲陽、北平。至易州,易人來拒,逾濠而陣。李景章出降,言城中人無鬥志。大元帥將修攻具,突呂不諫曰:「我師遠來,人馬疲憊,勢不可久留。」乃止。軍還,大元帥以其謀聞,太祖大悅,賜齎優渥。



 車駕西征,突呂不與大元帥為先鐸,伐黨項有功,太祖犒師水精山。大元帥東歸,突呂不留屯西南部,復討黨項,多獲而還。太祖東伐,大諲撰降而復叛,攻之,突呂不先登。渤海平,承詔銘太祖功德於永興殿壁。班師,已下州郡往往復叛,突呂不從大元帥攻破之。



 淳欽皇后稱制,有飛語中傷者,後怒,突呂不懼而亡。太
 宗知其無罪,召還。天顯三年,討烏古部,俘獲甚眾。伐唐,以突呂不為左翼,攻唐軍霞沙寨,降之。十一年,送晉主石敬瑭入洛。及大冊,突呂不總禮儀事,加特進檢校太尉。會同五年卒。王鬱,京兆萬年人,庸義武軍節度使處直之孽子。伯父處存鎮義武,卒,三軍推其子郜襲,處直為都知兵馬使。光化三年,梁王朱全忠攻定州,郜遣處直拒於沙河。兵敗,入城逐郜,郜奔太原。亂兵推處直為留後,遣人請事梁王。梁與晉王克用絕好,表處直為義武軍節度使。



 初郜之亡也,鬱從之。晉王克用妻以女,用為新州防禦使。



 處直料晉必討張文禮,鎮亡,則定不獨存,益自疑。陰使鬱北導契丹入塞以牽晉兵,且許為嗣。鬱自奔晉,常恐失父心,得使,大喜。神冊六年,奉表送歟,舉室來降,太祖以為養子。



 未幾,鬱兄都囚父,自為留後,帝遣鬱從皇太子討之。至定州,都堅壁不出,掠居民而還。



 明年,從皇太子攻鎮州,遇唐兵於定州,破之。天贊二年秋,鬱及阿古只略地燕、趙,攻下磁窯務。從太祖平渤海,戰有功,加同政事門下平章事,改崇義軍節度使。



 太祖崩,鬱與妻會葬,其妻泣訴於淳欽皇后,求歸鄉國,許之。鬱奏曰:「臣本唐主之婿,主已被弒,此行夫妻豈能相保。願常侍太后。」
 後喜曰:「漢人中,惟王郎最忠孝。」以太祖嘗與李克用約為兄弟故也。尋加政事令。還宜州,卒。



 耶律圖魯窘,字阿魯隱,肅祖子洽窅之孫,勇而有謀略。



 太宗立晉之役,其父敵魯古為五院夷離堇,歿於兵,帝即以其職授圖魯窘。會同元年,改北院大王,嘗屏左右與議大事,占對合上意。



 從討石重貴,杜重威擁十萬餘眾拒浮沱橋,力戰數日,不得進。帝曰:「兩軍爭渡,人馬疲矣,計安出?」諸將請緩師,為後圖,帝然之。圖魯窘厲色進曰:「臣愚竊以為陛下樂於安逸,則謹守四境可也;既欲擴大疆宇,出師遠攻,詎能無厪聖慮。若中路而止,適為
 賊利,則必陷南京,夷屬邑。若此,則爭戰未已,吾民無奠枕之期矣。且彼步我騎,何慮不克。況漢人足力弱而行緩,如選輕銳騎先絕其餉道,則事蔑不濟矣。」



 帝嘉曰:「國強則其人賢,海巨則其魚大。」於是塞其餉道,數出師以牽撓其勢,重威果降如育。以功獲賜甚厚。明年春,卒軍中。



 論曰:「神冊初元,將相大臣拔起風塵之中,翼扶王運,以任職取名者,固一時之材;亦由太祖推誠御下,不任獨斷,用能總攬群策而為之用歟!其投天隙而列功庸,至有心腹、耳目、手足之喻,豈偶然哉!討黨項,走敵魯,平刺
 葛,定渤海,功亦偉矣。若黔記治獄不冤,頗德持論不撓,延徽立經陳紀,紹勛簫節而死,圖魯窘料敵制勝,豈器博者無近用,道長者其功遠歟?稱為佐命固宜。」



\end{pinyinscope}