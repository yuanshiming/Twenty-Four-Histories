\article{卷六十八 列傳第六}

\begin{pinyinscope}

 耶律解里耶律拔裡得耶律朔古耶律魯不古趙延壽高模翰趙思溫耶律漚里思張礪耶律解裏,字潑單,突呂不部人。世為小吏。解裏早隸太宗麾下,擢為軍校。天顯間,唐攻定州,既陷,解里為唐兵
 所獲;晉高祖立,始歸國。太宗貰其罪,拜御史大夫。



 會同九年伐晉,師次游沱河,奪中渡橋,降其將杜重威。



 上命解裏與降將張顏澤率騎兵三千疾趨河南,所至無敢當其鋒。



 既入汴,解裏等遷晉主重貴於開封府。彥澤恣殺掠,亂宮掖,解裏不能禁,百姓騷然,莫不怨憤。車駕至京,數彥澤罪,斬於市,汴人大悅;解裏亦被詰責,尋釋之。



 天祿間,加守太子太傅。應歷初,置本部令穩,解裏世其職,卒。



 耶律拔裡得,字孩鄰,太祖弟刺葛之子。太宗即位,以親愛見任。



 會同七年,討石重貴,拔裡得進圍德州,下之,擒
 刺史師居璠等二十七人。九年,再舉兵,次滹沱河,降杜重威,戰功居多。太宗入汴,以功授安國軍節度使,總領河北道事。師還,州郡往往叛,以應劉知遠,拔裡得不能守而歸。



 世宗即位,遷中京留守,卒。



 耶律朔古,字彌骨頂,橫帳孟父之後。幼為太祖所養。既冠,為右皮室詳穩。從伐渤海,戰有功。



 天顯七年,授三河烏古部都詳穩。平易近民,民安之,以故久其任。會同間,為惕隱。時晉主石重貴渝盟,帝親征,晉將杜重威擁眾拒滹沱。月餘,帝由他渡沱。朔古與趙延壽據中渡橋,重威兵卻,遂降。是歲,人汴。



 世宗即位,朔古奉大宗喪歸上
 京,佐皇太后出師,坐是免官,卒。



 耶律魯不古,字信寧,太祖從侄也。初,太祖制契丹國字,魯不古以贊成功,授林牙、監修國史。



 後率偏師,為西南邊大詳穩,從伐黨項有功。會河東節度使石敬瑭為其主所討,遣人求援,魯不古導送於朝,如其請。



 帝親率師往援,魯不古從擊唐將張敬達於太原北,敗之。會同初,從討黨項,俘獲最諸將,師還。



 天祿中,拜於越。六年,為北院大王。終年五十五。



 趙延壽,本姓劉,恆山人。父邟,令蓨。梁開平初,滄州節度使劉守文陷蓨,其稗將趙德鈞獲延壽,養以為子。



 少美
 容貌,好書史。唐明宗先以女妻之,及即位,對其文為興平公主,拜延壽馳駙都尉、樞密使。明宗子從榮恃權跋扈,內外莫不震懾,延壽求補外避之,出為宣武軍節度使。清泰初,加魯國公,復為樞密使,鎮許州。石敬瑭發兵太原,唐遣張敬達往討。會敬達敗保晉安寨,延壽與德鈞往救,聞晉安已破,走團柏峪。太宗追及,延壽與其父懼降。



 明年,德鈞卒,以延壽為幽州節度使,封燕王;及改幽州為南京,遷留守,總山南事。天顯未,以延壽妻在晉,詔取之以歸。自是益自激昂圖報。



 會同初,帝幸其第,加政事令。六年冬,晉人背盟,帝親征,延壽為先鋒,下貝州,授魏、
 博等州節度使,封魏王。敗晉軍於南樂,獲其將賽項羽。軍元城,晉將李守貞、高行周率兵來逆,破之。至頓丘,會大霖雨,帝欲班師。延壽諫曰:「晉軍屯河濱,不敢出戰,若徑入澶州,奪其橋,則晉不足平。」



 上然之。適晉軍先歸渡州,高行周至析城,延壽將輕兵逆戰;上親督騎士突其陣,敵遂潰。師還,留延壽徇貝、冀、深三州。



 八年,再伐晉,晉主遣延壽族人趙行實以書來招。時晉人堅壁不出,延壽給曰:「我陷虜久,寧忘父母之邦。若以軍逆,我即歸。」晉人以為然,遣杜重威率兵迎之。延壽至滹沱河,據中渡橋,與晉軍力戰,手殺其將王清,兩軍相拒。太宗潛由他
 渡濟,留延壽與耶律朔古據橋,敵不能奪,屢敗之,杜重威掃厥眾降。上喜,賜延壽龍鳳赭袍,且曰:「漢兵皆爾所有,爾宜親往撫慰。」延壽至營,杜重威、李守貞迎謁馬首。



 後太宗克汴,延壽因李崧求為皇太子,上曰:「吾於魏王雖割肌肉亦不惜,但皇太子須天子之子得為,魏王豈得為也?」



 蓋上嘗許滅晉後,以中原帝延壽,以故摧堅破敵,延壽常以身先。至是以崧達意,上命遷延壽秩。翰林學士承旨張礪進擬中京留守、大丞相、錄尚書事、都督中外諸軍事;上塗「錄尚書事、都督中外諸軍事」。世宗即位,以翊戴功,授樞密使。天祿二年薨。



 高模翰,一名松,渤海人。有膂力,善騎射,好談兵。初,太祖平渤海,模翰避地高麗,王妻以女。因罪亡歸。坐使酒殺人下獄,太祖知其才,貰之。



 天顯十一年七月,唐遣張敬達、楊光遠帥師五十萬攻太原,勢銳甚。石敬瑭遣人求救,太宗許之。九月,徵兵出雁門,模翰與敬達軍接戰,敗之,太原圍解。敬瑭夜出謁帝,約為父子。



 帝召模翰等賜以酒饌,親饗士卒,士氣益振。翌日,復戰,又敗之。敬達鼠竄晉安寨,模翰獻俘於帝。會敬瑭自立為晉帝,光遠斬敬達以降,諸州悉下。上諭模翰曰:「朕自起兵,百餘戰,卿功第一,雖古名將無以加。」乃授上將軍。會同元年,冊禮
 告成,宴百官及諸國使於二儀殿。帝指模翰曰:「此國之勇將,朕統一天下,斯人之力也。」群臣皆稱萬歲。



 及晉叛盟,出師南伐。模翰為統軍副使,與僧遏前驅,拔赤城,破德、貝諸寨。是冬,兼總左右鐵鷂子軍,下關南城邑數十。三月,來虎官楊覃赴乾寧軍,為滄州節度使田武名所圍,模翰與趙延壽聚議往救。俄有光自模翰目中出,索繞旗矛,焰焰如流星久之。模翰喜曰:「此天贊之祥!」遂進兵,殺獲甚眾。以功加侍中。略地鹽山,破饒安,晉人震怖,不敢接戰。



 加太傅。



 晉以魏府節度使杜重威領兵三十萬來拒,模翰謂左右曰:「軍法在正不在多。以多陵少,不
 義必敗。其晉之謂乎!」詰旦,以麾下三百人逆戰,殺其先鐸梁漢璋,餘兵敗走。手詔褒美,比漢之李陵。頃之,杜重威等復至滹沱河,帝召模翰問計。



 上善其言曰:「諸將莫及此。」乃令模翰守中渡橋。及戰,復敗之,上曰:「朕憑高觀兩軍之勢,顧卿英銳無敵,如鷹逐難兔。當圖形麟閣,爵貤後裔。」已而杜重威等降。車駕入汴,加特進檢校太師,封悊郡開國公,賜璽書、劍器。為汴州巡檢使,平淚汜諸山土賊,遷鎮中京。



 天祿二年,加開府儀同三司,賜對衣、鞍勒、名馬。應歷初,召為中臺省右相。至東京,父老歡迎曰:「公起戎行,致身富貴,為鄉里榮,相如、賈臣輩不足過
 也。」九年正月,遷左相,卒。



 趙思溫,字文美,盧龍人。少果銳,膂力兼人,隸燕帥劉仁恭幕。李存勖問罪於燕,思溫統遍師拒之。流矢中目,裂裳漬血,戰猶不已。為存勖將周德威所擒,存勖壯而釋其縛。久之,日見信用。與梁戰於莘縣,以驍勇聞,授平州刺史,兼平、營、薊三州都指揮使。



 神冊二年,太祖遣大將經略燕地,思溫來降。及伐渤海,以思溫為漢軍都團練使,力戰拔扶餘城。身被數創,太祖親為調藥。太宗即位,以功擢檢校太保、保靜軍節度使。天顯十一年,唐兵攻太原,石敬瑭遣使求救,上命思溫自嵐、憲間出兵援之。



 既罷兵,改南京留守、盧龍軍節度使、管內觀察處置等使、開府儀同三司,兼侍中,賜協謀靜亂翊聖功臣,尋改臨海軍節度使。



 會同初,從耶律牒(蟲葛)使晉行冊禮,還,加檢校太師。



 二年,有星殞於庭,卒。上遣使賻祭,贈太師、魏國公。子延照、延靖,官至使相。」



 耶律漚里思,六院夷離堇蒲古只之後。負勇略,每戰被重錯,揮鐵槊,所向披靡。



 會同間,伐晉,上至河而獵,適海東青鶻搏雉,晉人隔水以鴿引去。上顧左右曰:「誰為我得此人?」漚里思請內廄馬,濟河擒之,並殺救者數人還。上大悅,優加賞齎。



 既而晉將杜重威逆於望都,據水勒
 戰。漚里思介馬突陣,餘軍繼之。被圍,眾言陣薄處可出,漚里思曰:「恐彼有他備。」



 竟引兵沖堅而出;回視眾所指,皆大塹也。其料敵多此類。



 是年,總領敵烈皮室軍,坐私免部曲,奪官,卒。



 張礪,磁州人,初仕唐為掌書記,遷翰林學士。會石敬瑭起兵,唐主以礪為招討判官,從趙德鈞援張敬達於河東。及敬達敗,礪入契丹。



 後太宗見礪剛直,有文彩,擢翰林學士。礪臨事必盡言,無所避,上益重之。未幾,謀亡歸,為追騎所獲。上責曰:「妝何故亡?」礪對曰:「臣不習北方土俗、飲食、居處,意常鬱鬱,以是亡耳。」上顧通事高彥英曰:「
 朕嘗戒汝善遇此人,何乃使失所而?礪去,可再得耶?」遂杖顏英而謝礪。



 會同初,升翰林承旨,兼吏部尚書,從太宗伐晉。入汴,諸將蕭翰、耶律郎五、麻答輩肆殺掠,礪奏曰:「今大遼始得中國,宜以中國人治之,不可專用國人及左右近習。茍政令乖失,則人心不服,雖得之亦將失之。」上不聽。改右僕射,兼門下侍郎、平章事。



 頃之,車駕北還,至欒城崩。時礪在恆州,蕭翰與麻答以兵圍其第。礪方臥病,出見之。翰數之曰:「妝何故於先帝言國人不可為節度使?我以國舅之親,有征伐功,先帝留我守汴,以為宣武軍節度使,汝獨以為不可。又譖我與解裏好
 掠人財物子女。今必殺汝!」趣令鎖之。礪抗聲曰:「此國家大體,安危所系,吾實言之。欲殺即殺,奚以鎖為?」麻答以礪大臣,不可專殺,乃救止之。是夕,礪恚憤卒。



 論曰:「初,晉因遼之兵而得天下,故兼臣禮而父事之,割地以為壽,輸帛以為貢。未久也,而會同之師次滹沱矣。豈群帥貪功默武而致然歟?抑所謂信不由衷也哉?



 模翰以功名自終,可謂良將。若延壽之勛雖著,至於凱覦儲位,謬矣。利令智昏,固無足議。若乃成末釁以雋虧功,如解里者,何譏焉!」



\end{pinyinscope}