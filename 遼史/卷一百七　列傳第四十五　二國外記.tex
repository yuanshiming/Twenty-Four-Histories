\article{卷一百七 列傳第四十五 二國外記}

\begin{pinyinscope}

 高麗西夏高麗自有國以來,傳次久近,人民土田,歷代各有其志,然高麗與遼相為終始二百餘年。



 自太祖皇帝神冊間,高麗遣命名進寶劍。天贊三年,來貢。



 太宗天顯二年,來貢。會同二年,受晉上尊號冊,遣使往報。



 聖宗統和三年秋七月,詔諸通各完戎器,以備東征高麗。



 八月,以遼澤沮洳,罷師。十年,以東京留守蕭恆德伐高麗。



 十一年,王治
 遣樸良柔奉表請罪,詔取女直國鴨綠江東數百里地賜之。十二年,入貢。三月,王治遣使請所俘生口,詔續還之,仍遣使撫諭。十二月,王治進妓樂,詔卻之。十三年,治遣李周楨來貢,又進鷹。十月,遣李知白奉貢。十一月,遣使冊治為王。遣童子十人來學本國語。十四年,王治表乞為婚姻,以東京留守駙馬蕭恆德女下嫁之。六月,遣使來問起居。自是,至者無時。



 十五年,韓彥敬來納聘幣,吊駙馬蕭恆德妻越國公主薨。十一月,治薨,其侄誦遣王同穎來告。十二月,遣使致祭,詔其侄誦權知國事。十六年,遣使冊誦為王。二十年,誦遣使賀伐宋之捷。七月,
 來貢本國《地裡圖》。二十二年,以南伐事詔諭之。二十三年,高麗聞與宋和,遣使來賀。二十六年,進龍須草席,及賀中京城。二十七年,承天皇太后崩,遣使報以國哀。二十八年,誦遣魏守愚等來祭。三月,使來會葬。



 五月,高麗西京留守康肇弒其主誦,擅立誦眾兄詢。八月,聖宗自特伐高麗,報宋,遣引進使韓杞宣問詢。詢奉表乞罷師,不許。十一月,大軍渡鴨綠江,康肇拒戰於銅州,敗之。肇復出,右皮室詳穩耶律敵魯擒肇等,追奔數十里,獲所棄糧餉、鎧仗,銅、霍、貴、寧等州皆降。詢上表請朝,許之,禁軍士俘掠。以政事舍人馬保祐為開京留守,安州團練
 使王八為副留守。太子太師乙凜將騎兵一千,送保祐等赴京。宋將卓思正殺我使者韓喜孫等十人,領兵出拒,保祐等復還。乙凜領兵擊之,思正遂奔西京,圍之五日,不克,駐蹕於城西佛寺。高麗禮部郎中渤海陀失來降。遣排押、盆奴攻開京,遇敵於京西,敗之。



 詢棄城遁走,遂焚開京,至清江而還。二十九年正月,班師,所降諸城復叛。至貫州南嶺谷,大雨連日,霽乃得渡,馬駝皆疲乏,甲仗多遺棄。次鴨綠江,以所俘人分置諸陵廟,余賜內戚、大臣。



 開泰元年,詢遣蔡忠順來乞稱臣如舊,詔詢親朝。八月,遣田拱之奉表,稱病不能朝。詔復取六州之
 地。二年,耶律資忠使高麗取地,未幾還。三年,資忠復使,如前索地。五月,詔國舅詳穩蕭敵烈、東京留守耶律團石等造浮梁於鴨綠江,城保、宣義、定遠等州。四年,命北府宰相劉慎行為都統,樞密使耶律世良為副,殿前都點檢蕭虛烈為都監。慎行挈家邊上,致緩師期,追還之;以世良、虛烈總兵伐高麗。五年,世良等與高麗戰於郭州西,破之。六年,樞密使蕭合卓為都統,漢人行宮都部署王繼忠為副,殿前都點檢蕭虛烈為都監進討。蕭合卓攻興化軍不克,師還。七年,詔東平郡王蕭排押為都統,蕭虛烈副統,東京留守耶律八哥為都監,復伐高
 麗。十二月,蕭排押與戰於茶、陀二河之間,我軍不利,天雲、右皮室二軍沒溺者眾,天云軍詳穩海裏、遙輦帳詳穩阿果達、客省酌古、渤海詳穩高清明等皆沒於陣。八年,詔數排押討高麗罪,釋之。



 加有功將校,益封戰沒將校之妻,錄其子弟。以南皮室軍校有功,賜衣物銀絹有差,出金帛賜肴里、涅哥二奚軍。八月,遣郎君曷不呂等率諸部兵,會大軍同討高麗。詢遣使來乞貢方物。



 九年,資忠還,以詢降表進,釋詢罪。



 太平元年,詢薨,遣使來報嗣位,即遣使冊王欽為王。九年,賜欽物。十一年,聖宗崩,遣使告哀。七月,使來慰奠。



 興宗重熙七年,來貢。十二年三
 月,以加上尊號,來賀。



 十三年,遣使來貢。十四年三月,又來貢。十五年,入貢。八月,王欽薨,遣使來告。十六年,來貢。明年,又來貢。十九年,復貢。六月,遣使來賀伐夏之捷。二十二年,入貢。二十三年四月,王徽請官其子,詔加檢校太尉。



 興宗崩,道宗即位,清寧元年八月,遣使報國哀,以先帝遺留物賜之。十一月,使來會葬。二年、三年,皆來貢。四年春,遣使報太皇太后哀。五月,使來會葬。咸雍七年、八年,來貢。十二月,以佛經一藏賜徽。九年、十年,來貢。大康二年三月,皇太后崩,遣使報哀。六月,使來吊祭。四年,王徽乞賜鴨綠江以東地,不許。九年八月,王徽薨,以徽子
 三韓國公勛權知國事。十二月,勛薨。大安元年,冊勛子運為國王。



 二年,遣使來謝封冊。三年,來貢。四年三月,免歲貢。五年、六年,連貢。九年,賜王運羊。十年,運薨,子昱遣使來告,即賄贈。壽隆元年,來貢。十一月,王昱病,命其子顒權知國事。二年,來貢。三年三月,王昱薨。五年,王顒乞封冊。六年,封顒為三韓國公。



 七年,道宗崩,天祚即位,改為乾統元年,報道宗哀,使來慰奠。十二月,遣使來賀。五年,三韓國公顒薨,子俁遣使來告。八年,封侯為三韓國公,贈其父顒為國王。十二月,遣使來謝。九年,來貢。天慶二年,王俁母薨,來告,遣使致祭,起復。三年,遣使來謝致
 祭,又來謝起復。十年,乞兵於高麗以御金,而金人責之。至是遼國亡矣。



 西夏西夏,本魏拓跋氏後,其地則赫連國也。遠祖思恭,唐季受賜姓曰李,涉五代至宋,世有其地。至李繼遷始大,據夏、銀、綏、宥、靜五州,緣境七鎮,共東西二十五驛,南北十餘驛。子德明,曉佛書,通法律,嘗觀《太一金鑒訣》、《野戰歌》,制番書十二卷,又制字若符篆。



 其俗,衣白窄衫,氈冠,冠後垂紅結綬。自號嵬名,設官分文武。其冠用金縷帖,間起雲,銀紙帖,緋衣,金塗銀帶,佩蹀躞、解錐、短刀、弓矢,穿
 靴,禿發,耳重環,紫旋襴六襲。出入乘馬,張青蓋,以二旗前引,從者百餘騎。民庶衣青綠。革樂之五音為一音,裁禮之九拜為三拜。凡出兵先卜,有四:一灸勃焦,以艾灼羊胛骨;二擗弄,擗竹於地求數,若揲蓍然;三咒羊,其夜牽羊,焚香禱之,又焚穀火於野,次晨屠羊,腸胃通則吉,羊心有血則敗;四矢擊弦,聽其聲,知勝負及敵至之期。病者不用醫藥,召巫者送鬼,西夏語以巫為「廝」



 也;或遷他室,謂之「閃病,喜報仇,有喪則不伐人,負甲葉於背識之。仇解,用雞豬犬血和酒,貯於髑髏中飲之,乃誓曰:「若復報仇,穀麥不收,男女禿癩,六畜死,蛇入帳。」



 有力小
 不能復仇者,集壯婦,享以牛羊酒食,趨仇家縱火,焚其廬舍。俗曰敵女兵不祥,輒避去。訴於官,官擇舌辯氣直之人為和斷官,聽其屈直。殺人者,納命價錢百二十千。



 土產大麥、華豆、青稞、床子、古子蔓、咸地蓬實、芙蓉苗、小蕪荑、席雞草子、地黃葉、登廂草、沙蔥、野韭、拒灰條、白篙、咸地松實。



 民年十五為丁。有二丁者,取一為正軍。負提雜使一人為抄,四丁為兩抄。餘人得射它丁,皆習戰鬥。正軍馬駝各一,每家自置一帳。團練使上,帳、弓、矢各一,馬五百疋,橐駝一,旗鼓五,槍、劍、棍棓、粆袋、雨氈、渾脫、鍬、钁、箭牌、鐵笊籬各一;刺史以下,人各一駝,箭三百,毛幕
 一;餘兵三人共一幕。有炮手二百人,號「潑喜。」勇健者號「撞令郎」。齎糧不過一旬。晝則舉煙、揚塵,夜則篝火為候。若獲人馬,射之,號曰殺鬼招魂。或射草縛人。出軍用單日,避晦日。多立虛寨,設伏兵。衣重甲,乘善馬,以鐵騎為前鋒,用鉤索絞聯,雖死馬上不落。



 其民俗勇悍,衣冠、騎乘、土產品物、子侄傳國,亦略知其大概耳。



 初,西夏臣宋有年,賜姓曰趙;迨遼聖宗統和四年,繼遷叛宋,始來附遼,授特進檢校太師、都督夏州諸軍事,遂復姓李。十月,遣使來貢。六年,入貢。七年,來貢,以王子帳耶律襄之女封義成公主,下嫁繼遷。八年正月,來謝。三月,又來貢。九
 月,繼遷遣使獻宋俘。十月,以敗宋軍來告。十二月,下宋鱗、雩等州,來告,遣使封繼遷為夏國王。九年二月,遣使告伐宋之捷。四月,遣李知自來謝封冊。七月,復綏、銀二州,來告。十月,繼遷以宋所授敕命,遣使來上。是月,定難軍節度使李繼捧來附,授開府儀同三司、檢校太師,兼侍中,封西平王,仍賜推忠效順啟聖定難功臣。十二月,繼遷潛附於宋,遣韓德威持詔諭之。十年二月,韓德威還,奏繼遷托故不出,至靈州俘掠以還。西夏遣來奏德威俘掠,賜招撫諭。十月,來貢。十二年,入貢。十三年,敗宋師,遣使來告。十四年,又來貢。十五年三月,以破宋兵
 來告,封繼遷為西平王。六月,遣使來謝封冊。十六年,來貢。十八年,授繼遷子德明朔方軍節度使。十九年,遣李文冀來貢。六月,奏下宋恆、環、慶三州,賜詔褒美。二十年,遣使來進馬、駝。六月,遣劉仁勖來告下靈州。二十一年,繼遷死,其子德昭遣使來告。六月,贈繼遷尚書令,遣西上閣門使丁振吊慰。八月,德昭遣使來謝吊贈。二十年三月,德昭遣使上繼遷遺留物。七月,封德昭為西平王。十月,遣使來謝封冊。二十三年,下宋青城,來告。二十五年,德昭母薨,遣使吊祭,起復。二十七年,承天皇太后崩,遣使報哀於夏。二十八年,遣使冊德昭為夏國王。開
 泰元年,德昭遣使進良馬。二年,遣引進使李延弘賜夏國王李德昭及義成公主車馬。太平元年,來貢。十一年,聖宗崩,報哀於夏,德昭遣使來進賻幣。



 興宗即位,以興平公主下嫁李元昊,以元吳為駙馬都尉。



 重熙元年,夏國遣使來賀。李德昭薨,冊其子夏國公元昊為王。



 二年,來貢。十二月,禁夏國使沿路私市金鐵。七年,來貢。



 李元昊與興平公主不諧,公主薨,遣北院承旨耶律庶成持詔問之。九年,宋遣郭禎以伐夏來報。十年,夏國獻所俘宋將及生口。十一年,遣使問宋興師伐夏之由。十二月,禁吐渾鬻馬於夏,沿邊築障塞以防之。十二年工月,遣
 同知析津府事耶律敵烈、樞密都承旨王惟吉諭夏國與宋和。二月,元昊以加上尊號,遣使來賀。耶律敵烈等使夏國還,奏元昊罷兵,遣命名報宋。四月,夏國遣命名進馬、駝。七月,元昊上表請伐宋,不從。



 十月,夏人侵黨項,遣延昌宮使高家奴讓之。十三年四月,黨項及山西部族節度使屈烈以五部叛入西夏,詔徵諸道兵討之。



 六月,阻卜酋長烏八遣其子執元昊所遣求援使搒邑改來。八月,夏使對不以情,羈之。使復來,詢事宜不實對,笞之。十月,元昊上表謝罪,欲收集叛黨以獻,從之;進方物,命北院樞密副使蕭革迓之。元昊親率黨項三部來降,詰其納叛背盟,元昊伏罪。初,夏人執胡睹,至是,請以被執者來歸。
 詔所留夏使還其國。十二月,胡睹來歸,又遣使來貢。



 十七年,元昊薨,其子諒祚遣使來告,上其父遺留物。鐵不得國乞以本部軍助攻夏國,不許。十八年,復議伐夏,留其賀正使不遣,遣北院樞密副使蕭惟信以伐夏告宋。六月,夏國遣使來貢,留之。七月,親征。八月,渡河,夏人遁。九月,蕭惠為夏人所敗。十月,招討使耶律敵古率阻卜軍至賀蘭山,獲元昊妻及其官屬。遇其軍三千來拒,殪之;詳穩蕭慈氏奴、南克耶律斡里歿於陣。十九年正月,遣使問罪於夏。夏將窪普等攻金肅城,耶律高家奴等破之,窪普被創遁去,殺猥貨乙靈紀。三月,殿前都點
 檢蕭迭裡得與夏軍戰於三角川,敗之。招討使蕭蒲奴、北院大王宜新等帥師伐夏,都部署別古得為監戰。



 五月,蕭蒲奴等入夏境,不遇敵,縱軍俘掠而還。夏國窪普來降。十月,李諒祚母遣使乞依舊稱臣。十二月,諒祚上表如母訓。二十年二月,遣使索黨項叛戶。五月,蕭爻括使夏回,進諒祚母表:乞代黨項權進馬駝牛羊等物;又求唐隆鎮,仍乞罷所建城邑。以詔答之。六月,獲元昊妻,及俘到夏人置於薊州。



 二十一年十月,諒祚遣使乞弛邊備,遣爻括賚詔諭之。二十二年七月,諒祚進降表,遣林牙高家奴賚詔撫諭。二十三年正月,貢方物。五月,乞
 進馬、駝,詔歲貢之。七月,諒祚遣使求婚。十月,進誓表。二十四年,興宗崩,遣命報哀於夏。



 道宗即位,清寧元年,遣使來賀。九月,以先帝遺物賜夏。



 四年四月,遣使會葬。九年正月,禁民鬻銅于夏。咸雍元年五月,來貢。三年十一月,遣使進回鶻僧、金佛、《梵覺經》。十二月,諒祚薨。四年二月,諒祚子秉常遣使報哀,即遣使吊祭。



 秉常上其父遺物。十月,冊秉常為夏國王。十二月,來貢。五年七月,遣使來謝封冊。閏十一月,秉常乞賜印綬。九年,遣使來貢。大康二年正月,仁懿皇后崩,遣使報哀於夏,以皇太后遺物賜之。遣使來吊祭。五年,來貢。八年二月,遣使以所獲宋
 將張天益來獻。大安元年十月,秉常遣使報其母哀。二年十月,秉常薨,遣使詔其子乾順知國事。十二月,李乾順遣使上其父秉常遺物。四年七月,冊乾順為夏國王。五年六月,遣使來謝封冊。八年六月,夏為宋所侵,遣使乞援。壽隆三年六月,以宋人置壁壘於要地,遣使來告。四年六月,求援。十一月,遣樞密直學士耶律伊使宋,諷與夏和。夏復遣使來求援。



 五年正月,詔乾順伐拔思母等部。十一月,夏以宋人罷兵,遣使來謝。六年十一月,遣使請尚公主。七年,道宗崩,遣使告哀於夏。遣使來慰奠。



 天祚即位,乾統元年,夏遣使來賀。二年,復請尚公主。



 又以
 為宋所侵,遣李造福、田若水來求援。三年,復遣使請尚公主。十月,使復來求援。四年、五年,李造福等至,乞援。



 以族女南仙封成安公主下嫁乾順。六年正月,遣牛溫舒使宋,令歸所侵夏地。六月,遣李造福來謝。八年,乾順以成安公主生子,遣使來告。九年,以宋不歸地來告。十年,遣李造福等來貢。天慶三年六月,來貢。保大二年,天祚播遷,乾順率兵來掇,為金師所敗,乾順請臨其國。六月,遣使冊乾順為夏國皇帝,而天祚被執歸金矣。論曰:「高麗、西夏之事遼,雖嘗請婚下嫁,烏足以得其固志哉?三韓接壤,反覆易知;涼州負遠,納叛侵疆,乘隙輒
 動;貢使方往,事釁隨生。興師問罪,屢煩親征。取勝固多,敗亦貽悔。昔吳趙咨對魏之言曰:「大國有征伐之兵,小國有備御之固。豈其然乎!先王柔遠,以德而不以力,尚矣。遼亡,求援二國,雖能出師,豈金敵哉。」



\end{pinyinscope}