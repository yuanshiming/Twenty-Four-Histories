\article{卷四本紀第四 太宗下}

\begin{pinyinscope}

 會同元年春正月戊申朔,晉及諸國遣使來賀。晉使且言已命和凝撰《聖德神功碑》。戊辰,遣人使晉。



 二月壬午,室韋進白麃。戊子,鐵驪來貢。丁酉,獵松山。



 戊戌,幸遼河東。丙申,上思人皇王,遣惕隱率宗室以下祭其行宮。丁未,詔增晉使所經供億戶。



 三月壬戌,將東幸,三克言農務方興,請減輜重,促還朝,從之。丙寅,女直來貢。癸酉,東幸。



 夏四月戊寅朔,如南京。甲申,女直來貢。乙酉,幸溫泉。



 己丑,還宮,朝於皇太后。
 丁酉,女直貢弓矢。己亥,西南邊大詳穩耶律魯不古奏黨項捷。



 五月甲寅,晉復遣使請上尊號,從之。



 六月丙子朔,吐谷渾及女直來貢。辛卯,南唐來貢。癸巳,詔建日月四時堂,圖寫古帝王事於兩廡。



 秋七月癸亥,遣使賜晉馬。丁卯,遣鶻離底使晉,梅裏了古使南唐。戊辰,遣中臺省右相耶律述蘭迭烈哥使晉,臨海軍節度使趙思溫副之,冊晉帝為英武明義皇帝。



 八月戊子,女直來貢。庚子,吐谷渾、烏孫、靺鞨皆來貢。九月庚戌,黑車子室韋貢名馬。邊臣奏晉遣守司空馮道、左散騎常侍韋勛來上皇太后尊號,左僕射劉煦、右諫議大夫盧重上皇帝尊號,遂遣臨軍寅你已充接伴。壬子,詔群臣及高年,凡授大臣爵秩,皆賜錦袍、金帶、白馬、金飾鞍勒,著於令。



 冬十月甲戌朔,遣郎君迪裡等撫問晉使。壬寅,晉遣使來謝冊禮。是日,復有使進獨峰駝及名馬。



 十一月甲辰朔,命南北宰相及夷離堇就館賜晉使馮道以下宴。丙午,上御開皇殿,召見晉使。壬子,皇太后御開皇殿,馮道、韋勛冊上尊號曰睿文神武法天啟運明德章信至道廣敬昭孝嗣聖皇帝。大赦,改元會同。是月,晉復遣趙瑩奉表來賀,以幽、薊、瀛、莫、涿、檀、順、媯、儒、新、武、雲、應、朔、寰、蔚十六州並圖籍來獻。於是塘以皇都為上京,府曰臨潢。升幽州為南京,南京為東京。改新州為奉聖州,武州為歸化州。升北、南二院及乙室夷離堇為王,以主簿為令,令為刺史,刺史為節度使,二部梯裡已為司徒,達刺乾為副使,麻都不為縣令,縣達刺乾為馬步。置宣徵、閣門使,控鶴、客省、御史大夫、中丞、侍御、判官、文班牙署、諸宮院世燭,馬群、遙輦世燭,南北府、國舅
 帳郎君為敞史,諸部宰相、節度使帳為司空,二室韋
 闥林為僕射,鷹坊、監冶等局長為詳穩。



 十二月戊戌,遣同括、阿缽等使晉,制加晉馮道守太傅,劉煦守太保,餘官各有差。



 二年春正月乙巳,以受晉冊,遣使報南唐、高麗。丁未,御開皇殿,宴晉使馮道以下,賜物有差。戊申,晉遣金吾衛大將軍馬從斌、考功郎中劉知新來貢珍幣,命分賜群臣。丙辰,晉遣使謝免沿邊四州錢幣。



 二月戊寅,宴諸王及節茺使來賀受冊禮者,仍命皇太子惕隱迪輦餞之。癸巳,謁太祖廟,賜在京吏民物,及內外群臣官賞有差。丁酉,加兼侍中、左金吾衛上將軍王鄑檢校太尉。



 三月,畋於哀潭之側。戊申,女直來貢。丁巳,封
 皇子述律為壽安王,罨撒葛為太平王。己巳,大齎百姓。



 夏四月乙亥,幸木葉山。癸巳,東京路奏狼食人。



 五月乙巳,禁南京鬻牝羊出境。思奴古多里等坐盜官物,籍其家。南唐遣使來貢。丁未,以所貢物賜群臣。戊申,回鶻單于使人乞授官,詔第加刺史、縣令。



 六月丁丑,雨雪。是夏,駐蹕頻蹕澱。



 秋七月戊申,晉遣使進犀帶。庚戌,吐谷渾來貢。乙卯,敞史阿缽坐奉使失職,命笞之。



 閏月癸未,乙室大王坐賦調均,以木劍背撻而釋之;並罷南、北府民上供,及宰相、節度諸賦役非舊制者。乙酉,遣的烈賜晉烏古良馬。己丑,以南王府二刺史貪蠹,各杖一百,仍系虞候帳,備射鬼箭;選群臣為民所愛者代之。



 八月乙丑,晉遣使貢歲,奏輸戌、亥二歲金幣於燕京。



 九月甲戌,阻卜阿離底來貢。己卯,遣使使晉。



 冬十月丁未,
 上以烏古部水草肥美,詔北、南院徙三石烈戶居之。



 十一月丁亥,鐵驪、敦煌並遣使來貢。



 十二月庚子,鉤魚於土河。甲子,回鶻使者傔人有以刃相擊者,詔付其使處之。



 三年春正月戊子,吳越王遣使來貢。庚寅,人皇王妃來朝。



 回鶻使乞觀諸國使朝見禮,從之。壬辰,遺陪謁、阿缽使晉致生辰禮。
 晉以並、鎮、忻、代之吐谷渾來歸。



 二月己亥,奚王勞骨寧率六節度使朝貢。庚子,烏古遣使獻伏鹿國俘,賜其療夷離堇旗鼓以旌其功。壬寅,女直來貢。



 辛亥,墨離鶻末裡使回鶻阿薩蘭還,賜對衣勞之。乙卯,鴨淥江女直遣使來覲。



 三
 月戊辰,遣使使晉,報幸南京。己巳,如南京。辛未,命惕隱耶律涅離骨德率萬騎先驅。壬申,次石嶺,以奚王勞骨寧監軍寅你已朝謁不時,切責之。丙子,魯不姑上黨項俘獲數。



 癸未,獵水門,獲白鹿。庚寅,詔扈從擾民者從軍律。甲午,幸薊州。乙未,晉及南唐各遣使來覲。



 夏四月庚子,至燕,備法駕,入自拱辰門,御元和殿,行入閣禮。壬寅,遣人使晉,乙已,幸留守趙延壽別墅。丙午,晉遣宣徽使楊端、王眺等來問起居。王子,御便殿,宴晉及諸國使。丙辰,晉遣使進茶藥。壬戌,御昭慶殿,宴南京群臣。



 癸
 亥,晉遣使賀端午,以所進節物賜群臣。乙丑,南唐進白龜。



 五月庚午,以端竿宴群臣及諸國使,命回鶻、敦煌二使作本俗舞,俾諸使觀之。庚辰,晉遣使進弓矢。甲申,遣皇子天德及檢校司徒邸用和使。戊子,閱騎兵於南郊。



 六月乙未朔,東京宰相耶律羽之言渤海相大素賢不法,詔僚佐部
 民舉有才德者代之。丙申,閱步卒於南郊。庚子,晉及轄刺骨只遣使來見。壬寅,駕發燕京,命中書令蕭僧隱部諸道軍於長坐營。癸丑,次奉聖州。甲寅,勞軍士。



 秋七月己巳,獵猾底烈山。癸酉,朝於皇太后。丙子,從皇太后視人皇王妃疾。戊寅,人皇王妃蕭氏薨。己卯,以安重榮據鎮州叛晉,詔征南將軍柳嚴邊備。丙戌,徙人皇王行宮於其妃薨所。辛卯,晉遣使請行南郊禮,許之。



 八月己
 亥,詔
 東
 丹吏民為其王倍妃蕭氏服。庚子,阻卜來貢。壬寅,遣使南唐。乙巳,阻卜、黑車子室韋、賃烈等國來貢。南唐遣使求青氈帳,賜之。戊申,以安端私城為白川州。



 辛亥,鼻骨德使乞賜爵,以其國相授之。甲寅,阻卜來貢。乙卯,置白川州官屬。丙辰,詔以於諧里河、臚朐河之近地,給賜南院歐堇突呂、
 乙斯勃、北院溫納何刺三石烈人為農田。



 九月庚午,侍中崔窮古言:「晉主聞陛下數游獵,意請節之。」上曰:「騰之畋獵,非徒從樂,所以練習武事也。」乃詔諭之。壬午,邊將奏破吐谷渾,擒其長;詔止誅其首惡及其丁半,餘並釋之。丙戌,晉遣使貢名馬。戊子,女直及吳越王遣使來貢。



 冬十月辛丑,遣克郎使吳越,略姑使南唐。庚申,晉遣使貢布,及請親祠南嶽,從之。



 十一月己巳,南唐遣使奉蠟刃書言晉密事。丁丑,詔有司教民播種紡績。除姊亡妹績之法。



 十二月壬辰朔,率百僚謁太祖行宮。甲午,燔柴,禮畢,祠於神帳。丙申,遣使使晉。丙辰,詔契丹人授漢官
 者從漢儀,聽與漢人婚姻。丁巳,詔燕京皇城西南堞建涼殿。



 是冬,駐蹕於傘澱。



 四年春正月壬戌,以乙室、品卑、突軌三部鰥寡不能自存者,為之配。丙子,南唐遣使來貢。庚辰,涅刺、烏隗部獻黨項俘獲數。己丑,詔定徵黨項功。



 二月丙申,皇太子獲白獐。甲辰,晉遣使進香藥。丙子,鐵驪來貢。丁巳,詔有司編《邕祖奇首可汗事跡》。己未,晉遣楊彥詢來貢,且言我發重榮跋扈狀,遂留不遣。是月,晉鎮州安重榮執遼使者拽刺。



 三月,特授回鶻使闊里於越,並賜旌旗、弓劍、衣馬,余賜有差。癸酉,晉以許祀南郊,遣使來謝,進黃
 金進鎰。



 夏四月己卯,晉遣使進櫻桃。



 五月庚辰,吐谷常夷離堇蘇等叛入晉。遣牒蠟往諭晉及太原守臣。



 六月辛卯,振武軍節度副使趙崇逐其節度使耶律畫裏,以朔州叛,附晉。丙午,命室徽使哀古只赴朔州,以兵圍其城,有晉使至,請開壁,即勿聽,驛送闕下。



 秋七月癸亥,南唐遣使奉蠟丸書。丙寅,哀古只奏請遣使至朔令降,守者猶堅壁弗納。且言晉有貢物,命即以所貢物賜攻城將校。己巳,有司奏神纛車有蜂巢成蜜,史占之,吉。壬申,晉遣使進水晶硯。



 八月癸巳,南唐奉蠟丸書。庚子,晉遣使進犀弓、竹矢。



 吳越王遣使奉蠟丸書。



 九月壬申,有星
 孛於晉分。丁丑,幸歸化州。



 冬十月辛丑,有司奏燕、薊大熟。癸卯,吳越王遣使來貢。



 十一月丙寅,晉以討安重榮來告。庚午,吐谷渾請降,遣使撫諭。阻卜來貢,以其物賜左右。丙子,鴨淥江女直來貢。



 壬午,以永寧、天授二節及正旦、重午、冬至、臘並受賀,著令。



 十二月戊子,晉遣使來告山南節度使安從進反。詔以便宜討之。庚寅,南唐遣使奉蠟丸書。戊戌,晉遣王升鸞來貢。戊申,晉以敗安重榮來告,遂遣楊彥詢歸。辛亥,晉遣使乞罷戌兵,詔惕隱朔古班師。甲寅,攻拔朔州,遣控鶴指揮使諧里勞軍。時哀古戰歿城下,上怒,命誅城中丁壯,仍以叛民上戶
 三十為哀古只部曲。



 五年春正月丙辰朔,上在歸化州,御行殿受群臣朝。以諸道貢物進太后及賜宗室百僚。戊午,詔求直言,北王府郎君耶律海思應詔,召對稱旨,特授宣徽使。詔政事令僧隱等以契丹戶分屯南邊。戊辰,晉函安重榮首來獻。上數欲親討重榮,至是乃止。癸酉,遣使使晉,是月,晉以朔州平,遣使來賀,遂遣客省使耶律化哥使晉並致生辰禮。



 二月壬辰,上將南幸,以諸路有未平者,召太子及群臣議,皆曰:「今襄、鎮、朔三州雖已平,然吐渾為安重榮所誘,猶未歸命,宜發兵討之,以警諸部。」上曰:「正與
 聯合。」遂詔以明王隈恩代於越信恩為西南路招討使以討這。且諭明王宜先練習邊事,而後之官。甲午,如南京。遣使使晉索吐谷渾叛者。乙未,鼻骨德來貢。



 三月乙卯朔,晉遣齊州防禦使宋暉業、翰林茶酒使張言來問起居。



 閏月,駐蹕陽門。



 夏四月甲寅朔,鐵驪來貢,以其物分賜群臣。丙子,晉遣使進射柳鞍馬。



 五月五日戊子,禁屠宰。



 六月癸丑朔,晉齊王重貴遣來貢。丁巳,徒睹古、素撒來貢。乙丑,晉主敬瑭殂,子重貴立。戊辰,晉遣使告哀,輟朝七日。庚午,遣使往晉吊祭。丁丑,聞皇太后不豫,上馳入侍,湯藥改親嘗,仍告太祖廟,幸菩薩堂,飯僧五
 萬人。七月乃愈。



 秋七月庚寅,晉遣金吾衛大將軍言,判四方館事朱崇節來謝,書稱「孫」,不稱「臣」,遣客省使喬榮讓之。景延廣答曰:「先帝則聖朝所立,今主則我國自冊。為鄰為孫則可,奉表稱臣則不可。」榮還,具奏之,上始有南伐之意。辛卯,阻卜,鼻骨德、烏古來貢。將軍闥德里、蒲骨等率降將轄德至闕,並獻所獲,丁未,晉遣使以祖母哀來告。



 八月辛酉,女直、阻卜、烏古各貢方物。甲子,晉復襄州。



 戊辰,詔河東節度使劉知遠送叛臣烏古指揮使由燕京赴闕。癸酉,遣天城軍節茺使蕭拜石吊祭於晉。



 九月壬辰,遣使賀晉帝嗣位。



 冬十月己巳,徵諸道兵。
 遣將軍密骨德伐黨項。



 十一月乙未,武定軍奏松生棗。十二月癸亥,晉遣使謝。



 是冬,駐蹕赤城。



 六年春二月乙卯,晉遣使進先帝遺物。辛酉,晉遣使請居汴,從之。



 三月己卯朔,吳越王遣使來貢。甲申,梅裏喘引來歸。戊子,南唐遣使奉蠟丸書。丁未,晉至汴,遣使來謝。



 夏四月戊申朔,日有食之。



 五月己亥,遣使如晉致生辰禮。



 六月丁未朔,鐵驪來貢。己未,奚鋤骨裏療進白麝。辛酉,莫州進白鵲。晉遣使貢金。



 秋八月丁未朔,晉復貢金。己未,如奉聖州。晉遣其子延煦來朝。



 冬十一月辛卯,上京留守耶律迪輦得晉諜,知有二心。甲辰,鐵驪來貢。



 十二月丁未,如南京,議伐晉。命趙延壽、趙延昭、安端、解裏等同滄、恆、易、定分道而進,大軍繼之。



 是歲,楊彥昭請移鎮奈汐及新鎮,從之。



 七年春天上月甲戌朔,趙延壽、延昭率前鋒五萬騎次任丘。



 丙子,安端入雁門,圍忻、代。己卯,趙延壽圍貝州,其軍校邵珂開南門納遼兵,太守吳巒投井死。己丑,次元城,授延壽魏、博等州節度使,封魏王,率所部屯南樂。丙申,遣兵攻黎陽,晉張彥澤來拒。辛丑,晉遣使來修舊好,詔割河北諸州,及遣桑維翰、景延廣來議。



 二月甲辰朔,攻博州,刺史周儒以城降。晉平盧軍節茺使場光遠密道遼
 師自馬家口濟河。晉將景延廣命石斌守麻家口,白再榮守馬家口。未幾,周儒引遼軍麻答營於河東,攻鄆州北津,以應光遠。晉遣李守貞、皇甫遇、梁漢璋、薛懷讓將兵萬人,緣河水陸俱進。遼軍圍晉別將於戚城,晉主自將救之,遼師解云。守貞等至馬家口,麻答遣步卒萬人築營壘,騎兵萬人守於外,餘兵屯河西。渡未已,晉兵薄之,遼軍不利。



 三月癸酉朔,趙延壽言:「晉諸軍沿河置柵,皆畏層不敢戰。若率大兵直抵澶淵,據其橋梁,晉必可取。」是日,晉兵駐澶淵,其前軍高行周在戚城。乃命延壽、延昭以數萬騎出行周右,上以精兵出其左。戰至暮,上
 復以勁騎突其中軍,晉軍不能戰。會有諜者言晉軍東面數少,沿河城柵不固,乃急擊其東偏,眾皆奔潰。縱兵追及,遂大敗之。壬午,留趙延昭守貝州,徙所俘戶於內地。



 夏四月癸丑,還次南京。辛未,如涼陘。



 五月癸酉,耶律拔裡得奏破德州,擒刺史尹居及將吏二十七人。六月甲辰,黑車子室韋來貢。乙巳,維沒里、要裏等國來貢。



 秋七月己卯,晉楊光遠遣人奉蠟丸書。辛卯,晉遣張暉奉表乞和,留暉不遣。



 八月辛酉,回鶻遣使請婚,不許。是月,晉鎮州兵來襲飛狐,大同軍節度使耶律孔阿戰敗之。



 九月庚午朔,北幸。



 冬十月丁未,鼻骨德來貢。壬戌,天授
 節,諸國進賀,惟晉不至。



 十一月壬申,詔徵諸道兵,以閏月朔會溫榆河北。



 十二月癸卯,南伐。甲子,次古北口。



 閏月己巳朔,閱諸道兵於溫榆河。己卯,圍恆州,下其九縣。



 八年春正月庚子,分兵攻邢、洺、磁三州,殺掠殆盡。入鄴都境。張從恩、馬全節、安審琦兵悉陳於相州安陽水之南。



 皇甫遇與濮州刺史慕容彥超將兵千騎來覘遼軍。至鄴都,遇遼軍數萬,且戰縣卻,至榆林店。遼軍繼至,遇與彥超力戰百餘合,遇馬斃,步戰,審琦引騎兵逾水以救,遼軍乃還。



 二月,圍魏,晉將杜重威率兵來救。戊子,晉將折從阮陷勝州。三月戊戌,師拔祁州,殺其刺史沈斌。
 庚戌,杜重威、李守貞攻泰州。戊子,趙延壽率前鋒薄泰城。己未,重威、守貞引兵南遁,追至陽城,大敗之。復以步卒為方陣來拒,與虞二十餘合。壬戌,復搏戰十餘里。癸亥,圍晉兵於白團衛村。晉兵下鹿角為營。是夕大風。至曙,命鐵鷂軍下馬,拔其鹿角,奮短兵入擊。順風縱火揚塵,以助其。晉軍大呼曰:「都討保不用兵,令士卒徒死!」諸將皆奮出戰。張彥澤、藥元福、皇甫遇出兵大戰,諸將繼至,遼軍卻數百步。風益甚,晝晦如夜。



 符彥卿以萬騎橫擊遼軍,率步卒並進,遼軍不利。上乘奚車退十餘里,晉追兵急,獲一橐駝乘之乃歸。晉兵退保定州。



 夏四
 月甲申,還次南京,杖戰不力者各數百。庚寅,宴將士於元和殿。癸巳,如涼陘。



 六月戊辰,回鶻來貢。辛未,吐谷常、鼻骨德皆來貢。辛巳,黑車子室韋來貢。丁亥,趙延壽奏晉兵襲高陽,戌將擊走之。



 秋七月乙卯,獵平地松林。晉遣孟守中奉表請和,仍以前事答之。



 八月己巳,詔侍衛蕭素撒閱群牧於北陘。



 九月壬寅,次赤山,宴從臣,問軍國要務,對曰:「軍國之務,愛民為本。民富則兵足,兵足則國強。」上以為然。辛酉,還上京。冬十月辛未,祠木葉山。



 十一月戊戌,女直、鐵驪來貢。



 十二癸亥朔,朝謁太祖行宮。乙丑,雲州節度使耶律孔阿獲晉諜者。戊辰,臘,賜諸
 國貢使衣馬。



 九的正月庚子,回鶻來貢。丁未,女直來貢。



 二月戊辰,骨德奏軍籍。



 三月巳亥,吐谷渾遣軍校恤烈獻生口千戶,授恤烈檢校司空。



 夏四月辛酉朔,吐谷渾白可久來附。是月,如涼陘。王月庚戌,晉易州戌將孫方簡請內附。



 六月戊子,謁祖陵,更閟神殿為長思。秋七月辛亥,詔徵諸道兵,敢傷禾稼者,以軍法論。癸丑,女直來貢。乙卯,以阻卜酋長曷刺為本部夷離堇。



 八月丙寅,烏古來貢。是月,自鈄南伐。



 九月壬辰,閱諸道兵於漁陽西棗林澱。是月,趙延壽與晉張彥澤戰於定州,敗之。



 冬十一月戊子
 朔,進圍鎮州。丙申,先遣候騎報晉兵至,遣精兵斷河橋,晉兵退保武強。南院大王迪輦、將軍高模翰分兵由瀛州間道以進,杜重威遣貝州節度使梁漢璋率眾來拒。與戰,大敗之,殺梁漢璋。杜重威、張彥澤引兵據中渡橋,趙延壽以步座前擊,高彥溫以騎兵乘之,追奔逐北,殭尸數萬斬其將王清,宋彥筠墮水死。重威等退保中渡寨。義武軍節度使李殷以城降,遂進兵,夾滹而營,去中渡寨三里,分兵圍之。夜則列騎環守,晝則出兵抄掠,復命大內惕隱耶律朔骨里及趙延壽分兵圍守。自將騎卒夜渡河出其後,攻下欒城,降騎卒數千。



 分遣將士
 據其要害。下令軍中預備軍食,三日不得舉煙火,但獲晉人,即黥而縱之。諸饋運見者皆棄而走。於是晉兵內外隔絕,食盡勢窮。



 十二月丙寅,杜重威、李守貞、張彥澤等率所部二十萬眾來降。上擁數萬騎,臨犬阜,立馬以受之。授重威守太傅、鄴都留守,守貞天平軍節度使,餘各領舊職。分降卒之半付重威,半以隸趙延壽。命御史大夫解里、監軍傅桂兒、張彥澤持詔入汴,諭晉帝母李氏,以安其意,且召桑維翰、景延廣先來。留騎兵行只守魏,自率大軍而南。壬申,解裏等至汴,晉帝重貴素服拜命,輿母李氏奉表請罪。初,重貴絕和好,維翰數諫止之,
 不從,至是彥澤釘維翰,紿方自經死。詔收葬之,復其田園等宅,仍厚恤其家。甲戌,彥澤遷重貴及其母若妻於開封府署,以控鶴指揮使李榮督兵衛之。壬午,次赤岡。重貴舉族出封丘門,稿索牽羊以待。上不忍臨視,命改館封禪寺。晉百官縞衣紗帽,俯伏待罪。上曰:「其主負恩,其臣何罪,」命領職如故,即授安叔千金吾衛上將軍。叔千出班獨立,上曰:「汝邢州之請,朕所不忘。」乃加鎮國軍節度使,蓋在邢嘗密請內附也。將軍康祥執景延廣來獻,詔以牙籌數其罪,凡八,縶送都,道自殺。



 大同元年春正月丁亥朔,備法駕入汴,御崇元殿受百
 官賀。



 戊子,以樞密副使劉敏權知開封府,殺秦繼旻、李彥紳及鄭州防禦使楊承勛,以其北信為平盧軍節度使,襲父爵。初,楊光遠在青州求內附,其子承勛不聽,殺其判丘濤及北承祚等自歸於晉,故誅之。己丑,以張彥譯擅徙笪貴開封,殺桑維翰,縱兵大掠,不道,斬於市。晉人臠食之。辛卯,降重貴為崇祿大夫、檢校太尉,封負義侯。癸巳,以張礪為平章事,晉李崧為樞密使,馮道為太傅,和凝為翰林學士,趙瑩為太子太保,劉煦守太保,馮玉為太子少保。癸卯,遣趙瑩、馮玉、李彥韜將三百騎送負義侯及其母李氏、太妃安氏、妻馮氏、弟重睿、子
 延煦、延寶等於黃龍府安置。仍以其宮女五十人、內宦三人、東西班五十人、醫官一人、控鶴四人、皰丁七人、茶酒司三人、儀鸞三人、健卒十人從之。



 二月丁巳朔,建國號大遼大赦,改元大同。升鎮州為中京。



 以趙延壽為大丞相兼政事令、樞密使、中京留守,中外官僚將士爵賞有差。辛未,河東節度使北平王劉知遠自立為帝,國號漢。詔以耿崇美為昭義軍節度使,高唐英為昭德軍節度使,崔廷勛為河陽軍節度使,分據要地。



 三月丙戌朔,以蕭翰為宣武軍節度使,賜將吏爵賞有差。



 壬寅,晉諸司僚吏、嬪御、宦寺、方技、百工、圖籍、歷象、石經、銅人、明堂
 刻漏、太常樂譜、諸宮縣、齒簿、法物及鎧仗,悉送上京。磁州帥梁暉以相州降漢,己酉,命高唐英討之。



 夏四月丙辰朔,發自汴州,以馮道、李崧、和凝、李斡、徐臺符、張礪等從行。次赤岡,夜有聲如雷,直御幄,大星復隕於旗鼓前。乙丑,濟黎陽渡,顧謂侍臣曰:「朕此行有三失:縱兵掠芻粟,一也;括民私財,二也;不遽遣諸節度還鎮,三也。」皇太弟遣使問軍前事,上報曰:「初以兵二十萬降杜重威、張彥澤,下鎮州。及入汴,視其官屬具員者省之,當其才者任之。司屬雖存,官吏廢墮,猶雛飛之後,徒有空巢。久經離亂,一至於北。所在盜賊屯結,土功不息,饋餉非時,
 民不堪命。河東尚未歸命,西路酋帥亦相黨附,夙夜以思,制之之術,惟推心庶僚、和協軍情、撫綏百姓三乾而已。今所歸順凡七十六處,得戶一百九萬百一十八。非汴州炎熱,水土難居,止得一年,太平可指掌而致。且改鎮州為中京,巡幸。欲伐河東,姑俟別圖。其概如此。」戊辰,次高邑,不豫。丁丑,崩於欒城,年四十六。是歲九月壬子朔,葬於鳳山,陵曰懷陵,廟號太宗。統和二十六年七月,上尊謚孝武皇帝。重熙二十一年九月,增謚孝武惠文皇帝。



 贊曰:太宗甫定多方,遠近向化。建國號,備典章,至於厘
 庶政,閱名實,隸錄囚徒,教耕織,配鰥寡。求直言之士,得郎君海思即擢宣微。嘉唐張敬達忠於其君,卒以禮葬。輟游豫而納三克之請。憫士卒而下休養之令。親征晉國,重貴面縛。



 斯可謂威德兼私,英略間見者矣。入汴之後,無幾微之驕,有「三失」之訓。《傳》稱鄭伯之善處勝,《書》進《秦》《誓》之能悔過,太宗蓋兼有之,其卓矣乎!



\end{pinyinscope}