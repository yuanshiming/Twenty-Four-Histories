\article{卷五十九志第二十八 食貨志上}

\begin{pinyinscope}

 契丹舊俗,其富以馬,其強以兵。縱馬於野,馳兵於民。



 有事而戰,闊騎介夫,卯命辰集。馬逐水草,人仰湩酪,挽強射生,以給日用,糗糧芻茭,道在是矣。以是制勝,所向無前。



 及其有國,內建宗廟朝廷,外置郡縣牧守,制度日增,經費日廣,上下相師,服御浸盛,而食貨之用斯為急矣。於是五京及長、遼西、平州置鹽鐵、轉運、度支、錢帛諸
 司,以掌出納。



 其制數差等雖不可悉,而大要散見舊史。若農穀、租賦、鹽鐵、貿易、坑治、泉幣、群牧,逐類採摭,緝而為篇,以存一代食貨之略。



 初,皇祖勻德實為大迭烈府離堇,喜稼穡,善畜牧,相地利以教民耕。仲父述瀾為於越,飭國人樹桑麻,習組織。太祖平諸弟之亂,弭兵輕賦,專意於農。嘗以戶口滋繁,湩轄疏遠,分北大濃兀為二部,程以樹藝,諸部效之。



 太宗會同初,將東獵,三克奏減輜重,疾趨北山取物,以備國用,無害農務。尋詔有司勸農桑,教紡績。以烏古之地水草豐美,使甌昆石烈居之,益以海勒水之善地為農田。三年。



 如以諧里河、臚朐
 河近地,賜南院歐堇突呂、乙斯勃、北院溫納河剌三石烈人,以事耕種。八年,駐蹕赤山,宴從臣,問軍國要務。左右對曰:「軍國之務,愛民為本。民富則兵足,兵足則國強。」上深然之。是年,詔徵諸道兵,仍戒也有傷禾稼者以軍法論。



 應歷間,雲州進嘉禾,時謂重農所召。保寧七年,漢有宋兵,使來乞糧,詔賜慄二十萬斛助之。非經費有餘,其能若是?



 聖宗乾亨五年詔曰:「五稼不登,開帑藏而代民稅;螟蝗為災,罷徭役以恤饑貧。」統和三年,帝嘗過槁城,見乙室奧隗部下女人迪輦等黍過熟未獲,遣人助刈。太師韓德讓言,兵後逋民棄業,禾稼棲畝,募人獲之,以半給獲
 者。政事令室昉亦言,山西諸州給軍興,民力凋敝,田穀多躪於邊兵,請復今年租。六年,霜旱,霜旱,災民鎧,詔三司,舊以稅錢折粟,估價不實,其增以利民。又徙吉避寨居民三百戶理直氣壯檀、順、薊三州,擇沃壤,給牛、種穀。十三年,詔諸道置義倉。



 歲秋,社民隨所獲,戶出粟詩倉,社司籍其目。歲儉,發以振民。十五年,詔免南京舊欠義倉粟,仍禁諸軍官非時略牧妨農。



 開泰元年,詔曰:「朕惟百姓徭般煩重,則多給工價;年穀不登,發倉以貸;田園蕪廢者,則給牛、種以助之。」太平初幸燕,燕民以年豐進土產珍異。上禮高年,惠鰥寡,賜酺連日。



 九年,燕地饑,戶部副使
 王嘉請造船,募習海漕者,移遼東粟餉燕,議者稱道險不便而寢。



 興宗即位,遺使閱諸道禾稼。是年。通括戶口,詔曰:「朕於早歲,習知稼穡。力辦者廣務耕耘,罕聞輸納;家食者全虧種植,多至流亡。宜通檢括,普逐均平」。禁諸職官不得擅造酒糜穀;有婚祭者,有司給文字始聽。



 道宗初年,西北雨穀三十里,春州斗粟六錢。時西蕃多叛,上欲為守禦計,命耶律唐古督耕稼以給西軍。唐古率眾田臚朐河側,歲登上熟。移屯鎮州,凡十四稔,積粟數十萬斛,每斗不過數錢。以馬人望前為南京度支判官,公私兼裕,檢括戶口,用法平恕,及遷中京度支使。視事
 半見,積粟十五萬斛,擢左散騎常侍。遼之農穀至是為盛。而東京如咸、信、蘇、復、辰、海、同、銀、烏、遂、春、泰等五十餘城內,沿邊諸州,各有和糴倉,依祖宗法,出陳易新,許民自願假貸,收息二分。所在無虛二三十萬碩,雖累兵興,未嘗用乏。迨天慶間,金兵大入,盡為所有。會天祚播遷,耶律敵烈等逼立梁王雅里,令群牧人戶運鹽濼粟,人戶侵耗,議籍其產以償。雅裏自定其直:粟一車一羊,三車一牛,五車一馬,八車一駝。從者曰:「今一羊易粟二斗,尚不可得,此直太輕。」雅裡早:「民有則我有。若今盡償,眾何以堪?」事雖無及,然使天未絕遼,斯言亦足以收人
 心矣。



 夫賦稅之制,自大祖任延徽,始制國用。太宗籍五京戶以定賦稅,戶西之數無所於考。聖宗乾亨間,以上京「雲為戶」



 訾具實饒,善避徭役,遺害貧民,遂勒各戶,凡子錢到本,悉送歸官,與民均差。統和中,耶律昭言,西北之眾,每歲農時,一夫偵候,一夫治公田,二夫給糺官之役。當時沿邊各置屯田戍兵,易田積穀以給軍餉。故太平七年詔,諸屯田在官斛粟不得擅貸,在屯者力耕公田,不輸稅賦,此公田制也。餘民應募,或治閑田,或治私田,則計畝出粟以賦公上。統和十五年,募民灤河曠地,十年始租,此在官閑田制也。又詔山前後未納稅戶,
 並於密雲,燕樂兩縣,占田置業入稅,此私田制也。各部大臣從上征伐,俘掠人戶,自置郛郭,為頭下軍州。凡市井之賦,各歸頭下,惟酒稅赴納上京,此分頭下軍州賦為二等也。



 先是,遼東新附地不榷酤,而鹽曲之禁亦弛。馮延休、韓紹勛相繼商利,欲與燕地平山例如繩約,其民病之,遂起大延琳之亂。連年詔復其租,民始安靖。南京歲納三司鹽鐵錢折絹,大同歲納三司稅錢折粟。開遠軍故事,民歲輸稅,鬥粟折五錢,耶律抹只守郡,表請折六錢,亦皆利民善政也。



\end{pinyinscope}