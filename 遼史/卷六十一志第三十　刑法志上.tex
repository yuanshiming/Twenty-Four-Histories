\article{卷六十一志第三十 刑法志上}

\begin{pinyinscope}

 刑也者,始於兵而終於禮者也。鴻荒之代,生民有兵,如蜂有螫,自衛而已。蚩尤惟始作亂,斯民鴟義,奸宄並作,刑之用豈能已乎?帝堯清問下民,乃命三後恤功於民,伯夷降典,折民惟刑。故曰刑也者,始於兵而終於禮者也。先王順天地四時以建六卿。秋,刑官也,象時之成物焉。秋傳氣於夏,變色於春,推可知也。



 遼以用武立國,禁
 暴戢奸,莫先於刑。國初制法,有出於五服、三就之外者,兵之勢方張,禮之用未遑也。及阻午可汗可知宗室雅里之賢,命為夷離堇以掌刑闢,豈非士師之官,非賢者不可為乎。太祖、太宗經理疆土,擐甲之士歲無寧居,威克闕愛,理勢然也。子孫相繼,其法互有輕重;中間能審權宜,終之以禮者,惟景、聖二宗為優耳。



 然其制刑之凡有四:曰死,曰流,曰徒,曰杖。死刑有絞、斬、凌遲之屬,又有籍沒之法。流刑量罪輕重,置之邊城部族之地,遠則投諸境外,又遠則罰使絕域。徒刑一曰終身,二曰五年,三曰一年半;終身者決五百,其次遞減百;又有黥刺之法。杖
 刑自五十至三百,凡仗五十以上者,以沙袋決之;又有木劍、大棒、鐵骨朵之法。木劍、大棒之數三,自十五至三十;鐵骨朵之數,或五、或七。有重罪者,將決以沙袋,先於椎骨之上及四周擊之。拷訊之具,有粗、細杖及鞭、烙法。粗杖之數二十;細杖之數三,自三十至於六十。鞭、烙之數,凡烙三十者鞭三百,烙五十者鞭五百。被告諸事應伏而不服者,以此訊之。品官公事誤犯,民年七十以上、十五以下犯罪者,聽以贖論。贖銅之數,杖一百者,輸錢千。亦有八議、八縱之法。



 籍沒之法,始自太祖為撻馬狘沙裏時,奉痕德堇可汁命,按於越釋重遇害事,以其首
 惡家屬沒入瓦里。及淳欽皇后時析出,以為著帳郎君,至世宗詔免之。其後內外戚屬及世官之家,犯反逆等罪,復沒入焉;餘入沒為著帳永;其沒入宮分、分賜臣下者亦有之。木劍、大棒者,太宗時制。木劍面平北隆,大臣犯重罪,欲寬宥則擊之。沙袋者,穆宗時制,其制用熟皮合縫之,長六寸,廣二寸,柄一尺許。徒刑之數詳於生熙制,杖刑以下之數詳於咸雍制;其餘非常用而無定式者,不可殫紀。



 太祖初年,庶事草創,犯罪者量輕重決之。春後治諸弟逆黨,權宜立法。親王從逆,不磬諸甸人,或投高崖殺之;淫亂不軌者,五車轘殺之;逆父終者視
 此;訕詈犯上者,以熟鐵錐椿其口殺之。從坐者,量罪輕重杖決。杖有二:大者重錢五百,小者三百。又為梟磔、生瘞射鬼箭、炮擲,支解之刑。歸於重法,閑民使不為變耳。歲癸酉,下詔曰:「朕自北征以來,四方獄訟,積滯頗多。今休戰息民,群臣其副朕意,詳決之,無或冤枉。」及命北府宰相蕭敵魯等分道疏決。有遼鐵恤之意,昉見於此。神冊六年,克定諸夷,上謂侍臣曰:「凡國家庶務,巨細各殊,若憲度不明,則何以為治,群下亦何由知禁?」乃詔大臣定治契丹及諸夷之法,漢人則斷以《律令》,仍置釧院以達民冤。至太宗時,治渤海人一依漢法,餘無改焉。會同
 四年,皇族舍利郎君謀毒通事解裏等,已中者二人,使重杖之,及其妻流於厥拔離弭河,族造藥者。



 世宗天祿二年,天德、蕭翰、劉哥,及其弟盆都等謀反,天德伏誅,杖翰,流劉哥,遺盆都使轄戛斯國。夫四人之罪均而刑異。遼之世,同罪異論者蓋多。



 穆宗應歷十二年,國舅帳郎君蕭延之奴海裏強陵拽剌禿里未及之女,以法無文,加之宮刑,仍付禿里以為奴。因著為令。



 十六年,諭有司:「自先朝行幸頓次,必高立標識以禁行者。



 比聞楚古輩,故低置其標深草中,利人誤入,因之取財。自今有復然者,以死論。」然帝嗜酒及獵,不恤政事,五坊、掌獸、近侍、
 奉膳、掌酒人等,以獐鹿、野豕、鶻雉之屬亡失傷斃,及私歸逃亡,在造逾期,召不時至,或以奏對少不如意,或以飲食細故,或因犯者遷怒無辜,輒加炮烙鐵梳之刑。甚者至於無算。或以手刃剌之,斬擊射燎,斷手足,爛肩股,折腰脛,劃口碎齒,棄尸於野。且使築封於其地,死者至百有俠人。京師置百尺牢以處系囚。蓋其即位未久,惑女巫肖古之言,取人膽合延年藥,故殺人頗眾。後悟其詐,以鳴鏑叢射、騎踐殺之。



 及海里之死,為長夜之飲,五坊、掌獸人等及左右給事誅戮者,相繼不絕。雖嘗悔其因怒濫刑,諭大臣切諫;在廷畏懦,鮮能匡救,雖諫又不
 能聽。當其將殺壽哥、念古,殿前都點檢耶律夷臘葛諫曰:「壽哥等斃所掌雉,畏罪而亡,法不應死。」帝怒,斬壽哥等,支解之。命有司盡取鹿人之在系者凡六十五人,斬所犯重者四十四人,餘悉痛杖之。中有欲置死者,賴五子必攝等諫得免。已而怒頗德飼鹿不時,致傷而斃,遂殺之。季年,暴虐益甚,嘗謂太尉化葛曰:「朕醉中有處決不當者,醒當覆奏。」徒能言之,意無悛意,故及於難。雖云虐止褻御,上不及大臣,下不及百姓,然刑法之制,豈人主快情縱意之具邪。



 景宗在潛,已鑒其失。及即位,以宿衛失職,斬殿前都點檢耶律夷臘葛。趙王喜隱自囚所
 擅去械鎖,求見自辯,語之曰:「枉直未分,焉有出獄自辯之理?」命復縶之。既而躬錄囚待,盡召而釋之。保寧三年,以穆宗廢鐘院,窮民有冤者無所訴,故詔復之,仍使鑄釧,紀詔其上,道所以廢置之意。吳王稍為媽所告,有司請鞫,帝曰:「朕知其誣,若按問,恐餘人效之。」命斬以徇。五年,近侍實魯里誤觸神纛,法應死,杖而釋之。庶幾寬猛相濟。然緩於討賊,應歷逆黨至是始獲而誅焉,議者以此少之。



 聖宗沖年嗣位,睿智皇后稱制,留心聽斷,嘗勸帝宜寬法律。帝壯,益習國事,銳意於治。當時更定法令凡十數事,多合人心,其用刑又能詳慎。先是,契丹及漢
 人相毆致死,其法輕重不均,至是一等科之。統和十二年,詔契丹人犯十惡,亦斷以《律》。舊法,死囚尸市三日,至是一宿即聽收瘞。二十四年,詔主非犯謀反大逆及流死罪者,其奴婢無得告首;若奴婢犯罪至死,聽送有司,其主無得擅殺。二十九年,以舊法,宰相、節度使世選之家子孫犯罪,徒杖如齊民,惟免黥面,詔自今但犯罪當黥,即準法同科。開泰八年,以竊盜贓滿十貫,為首者處死,其法太重,故增至二十五貫,其首處死,從者決流。嘗敕諸處刑獄有冤,不能申雪者,聽詣御史臺陳訴,委官覆問。往時大理寺獄訟,凡開圖片奏者,以翰林學士、給事
 中、政事舍人詳決;至是始置少卿及正主之。猶慮其未盡,而親為錄囚。數遣使詣諸道審決冤滯,如邢抱樸之屬,所至,人自以為無冤。



 五院部民有自壞鎧甲者,其長佛奴杖殺之,上怒其用法太峻,詔奪官。吏以故不敢酷。撻剌干乃方寸在醉言宮掖事,法當死,特貰其罪。五院部民偶遺火,延及木葉山兆域,亦當死,杖而釋之,因著為法。至於敵八哥始竊薊州王令謙家財,及覺,以刃刺令謙,幸不死。有司擬以盜論,止加杖罪。又那母古犯竊盜者十有三次,皆以情不可恕,論棄市。因詔自今三犯竊盜者,黥額、徒三年;四則黥面、徒五年;至於五則處死。
 若是者,重輕適宜,足以示訓。近侍劉哥、烏古斯嘗從齊王妻而逃,以赦,後會千齡節出首,乃詔諸近侍、護衛集視而腰斬之。於是國無幸民,綱紀修舉,吏多奉職,人重犯法。故統和中,南京及易、平二州以獄空聞。至開泰五年,諸道皆獄空,有刑措之風焉。



 故事,樞密使非國家重務,未嘗親決,凡獄訟惟夷離畢主之。及蕭合卓、蕭樸相繼為樞密使,專尚吏才,始自聽訟。時人轉相效習,以狡智相高,風俗自此衰矣。故太平六年下詔曰:「朕以國家有契丹、漢人,故以南、北二院分治之,蓋欲去貪枉,除煩擾也;若貴賤異法,則怨必生。夫小民犯罪,必不能動有
 司以達於,惟內族、外戚多恃恩行賄,以圖茍免,如是則法廢矣。自今貴戚以事被告,不以事之大小,並令所在官司按問,具申北、南院覆問得實以聞;其不按輒申,及受請托為奏言者,以本犯人罪罪之。」七年,詔中外大臣曰:「《制條》中有遺闕及輕重失中者,其條上之,議增改焉。」。



\end{pinyinscope}