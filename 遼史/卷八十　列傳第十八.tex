\article{卷八十 列傳第十八}

\begin{pinyinscope}

 蕭敵烈弟拔刺耶律盆奴蕭排押弟恆德德子匹敵耶律資忠耶律瑤質耶律弘古高正耶律的琭大康乂蕭敵烈,字涅魯袞,宰相撻烈四世孫。識度弘遠,為鄉里推重。始為牛群敞史。帝聞其賢,召入侍,遷國舅詳穩。



 統
 和尋十八年,帝謂群臣曰:「高麗康肇弒其君誦,立誦族兄詢而相之,大逆之。宜發兵間其罪。」群臣皆曰可。敵烈諫曰:「國家連年征討,士卒撫敞。況陛下在諒陰;年穀不登,創痍末復。島夷小國,城壘完固。勝不為武;萬於失利,恐貼後悔。不如遣一介之使,往問其故。彼若伏罪則已;不然,俟服除歲豐,舉兵未晚。」時令己下,言雖不行,識者韙之。



 明年,同知左夷離畢事。改右夷離畢。開泰初,率兵巡西邊。時夷離堇部下閘撒狘撲里、失室、勃葛率部民遁,敵烈追擒之,令復業,遷國舅詳穩。從樞密使耶律世良伐高麗。還,加同政事門下平章事,拜上京留守。



 敵烈
 為人寬厚,達政體,廷臣皆謂有王佐才。漢人行宮都部署王繼忠薦其材可為樞密使,帝疑其黨而止。為中京留守,卒。族子忽古,有傳。弟拔刺。



 拔刺,字別勒隱。多智,善騎射。



 開泰間,以兄為右夷離畢,始補郎君,累遷奚六部禿裏太尉。太平末,大延琳叛,拔刺將北、南院兵往討,遇於蒲水,南院兵少卻。至手山,復與賊遇。拔刺刀易兩院旗幟,鼓勇力戰,破之。上聞,以手詔褒獎,賜內廄馬。



 重熙中,遷四捷軍詳穩,謝事歸鄉里。數歲,起為昭德軍節度使,尋改國舅詳穩,卒。



 耶律盆奴,字胡獨堇,惕隱涅魯古之孫。景宗時,為馬古
 部詳穩,政尚嚴急,民苦之。有司以聞,詔曰:「盆奴任方面寄,以細故究間,恐損威望,」尋遷馬群太保。



 統和十六年,隱實燕軍之不任事者,汰之。二十八年,駕征高麗,盆奴為先鋒。至銅州,高麗將康肇分兵為三以抗我軍:一營於州西,據三水之會,肇居其中;一營近州之山;一附城而營。盆奴率耶律弘古擊破三水營,擒肇,李玄蘊等軍望風潰。



 會大軍至,斬三萬餘級,追至開京,破敵於西嶺。高麗王詢聞邊城不守,遁去。



 盆奴入開京,焚其王宮,乃撫慰其民人。上嘉其功,遷北院大王,薨。



 蕭排押,字韓隱,國舅少父房之後。多智略,能騎射。



 統和
 初,為左皮室詳穩,討阻卜有功。四年,破宋將曹彬、米信兵於望都。凡軍事有疑,每預參決。尋總永興宮分糾及舍利、拽刺、二皮室等軍,與樞密使耶律斜軫收復山西所陷城邑。



 是冬,攻宋,隸先鋒;圍滿城,率所部先登,拔之,改南京統軍使。尚衛國公主,拜駙馬都尉,加同政事門下平章事。



 十三年,歷北、南院宣徽使。條上時政得失,及賦役法,上喜納焉。十五年,加政事令,遷東京留守。二十二年,復攻宋,將渤海軍,下德清軍。後蕭撻凜卒,專任南面事。宋和議成,為北府宰相,聖宗征高麗,將兵由北道進,至開京西嶺,破敵兵,斬數千級。高麗王詢懼,奔平州。
 排押入開京,大掠而還。帝嘉之,封蘭陵郡王。開泰二年,以宰相知西南面招討使。五年,進王東平。排押為政寬裕而善斷,諸部畏愛,民以殷富,時議多之。



 七年,再伐高麗,至開京,敵奔潰,縱兵俘掠而還。渡茶、陀二河,敵夾射,排押委甲仗走,坐是免官。



 太平三年,復王豳,薨。弟恆德。



 恆德,字遜寧。有膽略而善謀。



 統和元年,尚越國公主,拜駙馬都尉,遷南面林牙。從宣徽使耶律阿沒里征高麗還,改北面林牙。會宋將曹彬、米信侵燕,耶律休哥與恆德議軍事,多見信用,為東京留守。



 六年,上攻宋,圍沙堆,恆德獨當一面。城上矢石如雨,恆德意氣自若,督將士
 奪其陴。城陷,中流矢,太后親臨視,賜藥。攻長城口,復先登,太后益多其功。時高麗未附,恆德受詔,率兵拔其邊城。王治懼,上表請降。



 十二年八月,賜啟聖竭力功臣。從都部署和朔奴討兀惹,未戰,兀惹請降。恆德利其俘獲,不許。兀惹死戰,城不能拔。



 和朔奴議欲引退,恆德曰:「以彼倔強,吾奉招來討,無功而還,諸部謂我何!若深多獲,猶勝徒返。」和朔奴不得己,進擊東南諸部,至高麗北鄙。比還,道遠糧絕,士馬死傷者眾,坐是削功臣號。



 十四年,為行軍都部署,伐蒲盧毛朵部。還,公主疾,太后遣宮人賢釋侍之,恆德私焉。公主恚而薨,太后怒,賜死。



 後追
 封蘭陵郡王。子匹敵。



 匹敵,字蘇隱,一名昌裔。生未月,父母俱死,育於禁掖。



 既長,尚秦晉王公主,拜駙馬都尉,為殿前副點檢。統和八年,改北面林牙。太平四年,遷殿前都點檢,出為國舅詳穩。



 九年,渤海大延琳叛,劫掠鄰部,與南京留守蕭孝穆往討。孝穆欲全城降,乃築重城圍之,數月,城中人陰採納款,遂擒延琳,東京平,以功封蘭陵郡王。



 十一年,聖宗不豫。先是,欽哀與仁德皇后有隙,以匹敵嘗為後所愛,忌之。時護衛馮家奴上變,誣后弟浞卜與匹敵謀逆,以皇后攝政,徐議當立者。公主竊聞其謀,謂匹敵曰:「爾將無
 罪被戮。與其死,何若奔女直國以全其生!」匹敵曰:「朝廷詎肯以飛語害忠良。寧死弗適他國。」及欽哀攝政,殺之。



 耶律資忠,字沃衍,小字札刺,系出仲父房。



 兄國留善屬文,聖宗重之。時妻弟之妻阿古與奴通,將奔女直國,國留追及奴,殺之,阿古自經。阿古母有寵於太后,事聞,太后怒,將殺之。帝度不能救,遣人訣別,問以後事,國留謝曰:「陛下憫臣無辜,恩漏九泉,死且不朽!」既死,人多冤之。在獄著《兔賦》、《寤寐歌》,為世所稱。



 資忠博學,工辭章,年四十未仕。聖宗知其賢,召補宿衛。



 數問以古今治亂,資忠對無隱。開泰中,授中丞,眷遇日隆。



 初,高麗內屬、取女直
 六部地以賜。至是,貢獻不時至,詔資忠往問故。高麗無歸地意。由是權貴數短於上,出為上京副留守。三年,再使高麗,留弗遣。資忠每懷君親,輒有著述,號《西亭集》。帝與群臣宴,時一記憶曰:「資忠亦有此樂乎?」



 九年,高麗上表謝罪,始送資忠還。帝郊迎,同載以歸,命大臣宴勞,留禁中數日。謂曰:「朕將屈卿為樞密,何如?」



 資忠對曰:「臣不才,不敢奉詔。」乃以為林牙,知惕隱事。



 初,資忠在高麗也,弟昭為著帳郎君,坐罪沒家產。至是,乃復橫帳,且還舊產,詔以外戚女妻之。



 是時,樞密使蕭合卓、少師蕭把哥有寵,資忠不肯俯附,詆之。帝怒,奪官。數歲,出知來遠城
 事,歷保安、昭德二軍節度使。



 聖宗崩,表請會葬。既至,伏梓宮大慟曰:「臣幸遇聖明,橫被構譖,不獲盡犬馬報。」氣絕而蘇,興宗命醫治疾。久之,言國舅侍中無憂國心,陛下不當復用唐景福舊號,於是用事者惡之,遣歸鎮,卒。弟昭,有傳。



 耶律瑤質,字拔里堇,積慶宮人。父侯古,室韋部節度使。



 瑤質篤學廉介,有經世志。統和十年,累遷至積慶宮使。



 聖宗嘗諭瑤質曰:「聞卿正直,是以進用。國有利害,爾言宜無所隱。」由是所陳多見嘉納。



 上征高麗,破康肇軍於銅州,瑤質之力為多,王詢乞降,群臣議皆謂宜納。瑤質
 曰:「王詢始一戰而敗,遽求納款;此詐耳;納之,恐墮其奸計。待其勢窮力屈,納之未晚。」已而詢果遁,清野無所獲。其眾阻險而壘,攻之不下,瑤質以計降之。擢拜四蕃部詳穩。



 時招討使耶律頗的為總管,瑤質恥居其下,上表曰:「臣先朝舊臣,今既垂老,乞還新命,覬得常侍左右。」帝曰:「朕不使汝久處是任。」且命無隸招討,得專奏事到部。戢暴懷善,政績顯著。卒於官。



 耶律弘古,字盆訥隱,遙輦鮮質可汗之後。



 統和初,嘗以軍事任為拽刺詳穩,尋徙南京統軍使。十三年,徇地南鄙,克敵於四岳橋,斬首百餘級。攻宋,以戰功遷東京留
 守,封楚國公。後伐高麗,副先鋒耶律盆奴,擒康肇於銅州。三十年,西北部叛,從南府宰相耶律奴瓜討之。及典禁軍,號令整肅,諸部多降。尋遷侍中,卒。



 高正,不知何郡人。統和初,舉進士第,累遷樞密直學士。



 上將伐高麗,遣正先往諭意。及還,遷右僕射。時高麗王詢表請入覲,上許之,遣正率騎兵干人迓之。館於路,為高麗將卓思正所圍。正以勢不可敵,與麾下壯士突圍出,士卒死傷者眾。上悔輕發,釋其罪。



 明年,遷工部侍郎,為北院樞密副使。開泰五年卒。



 耶律的琭,字耶寧,仲父房之後。翌兵事,為左皮室詳穩。



 統和二十八年,伐高麗,的琭率本部軍與盆奴等擒康肇、李玄蘊於銅州。帝壯之曰:「以卿英才,為國戮力,真吾家千里駒也!」乃賜御馬及細鎧。



 明年,為北院大王,出為馬古敵烈部都詳穩。年七十二卒。



 大康乂,渤海人。開泰問,累審南府宰相,出知黃龍府,善綏撫,東部懷服。榆裏底乃部長伯陰與榆烈比來駙,送於朝。



 且言蒲盧毛朵界多渤海人,乞取之。詔從其請。康乂領兵至大石河駝準城,掠數百戶以歸。未幾卒。論曰:「高句驪弒其君誦而立詢,遼興問罪之師,宜其算簞壺漿以迎,除舍以待;而乃乘險旅拒,俾智者竭其謀,
 勇者窮其力。雖得其要領,而顓顓獨居一海之中自若也。豈服人者以德而不以力歟?況乎殘毀其宮室,系累其民人,所謂以燕伐燕也歟?嗚呼!朱崖之棄,捐之之力也,敵烈之諫有焉。」



\end{pinyinscope}