\article{卷三十本紀第三十 天祚皇帝四}

\begin{pinyinscope}

 五年春正月辛巳,黨項小斛祿遣人請臨其地。戊子,趨天德,過沙漠,金兵忽至。上徙步出走,近待進珠帽,卻之,乘張仁貴馬得脫,至天德。己丑,遇雪,無禦寒具,術者以貂裘帽進;途次絕糧,術者進(麥少)與棗;欲憩,術者即跪坐,倚之假寐。術者輩惟嚙冰雪以濟饑。過天德。至夜,將宿民家,給曰偵騎,其家知之,乃叩馬首,跪而大慟,潛宿其
 家。居數日,嘉其忠,遙授以節度使,遂趨黨項。以小斛祿為西南南招討使,總知軍事,仍賜其子及諸校爵賞有差。



 二月,至應州新城東六十里,為金人完顏婁室等所獲。



 八月癸卯,至金。丙午,降封海濱王。以疾終,年五十有四,在位二十四年。金皇統元年二月,改封豫王。五年,葬於廣寧府閭陽縣乾陵傍。



 耶律淳者,世號為北遼。淳小字涅里,興宗第四孫,南京留守、宋魏王和魯斡之子。清寧初,太后鞠育之。既長,篤好文學。昭懷太子得罪,上欲以淳為嗣。上怒耶律白斯不,知與淳善,出淳為彰聖等軍節度使。



 天祚即位,進王鄭。乾統二年,加越王。六年,拜
 南腐宰相,首議制兩府禮儀。上喜,徙王魏。其父和魯斡薨,即以淳襲父守南京。冬夏入朝,寵冠諸王。



 天慶五年,東征,都監章奴濟鴨子河,懷淳子阿撒等三百餘人亡歸,先遣敵裡等以廢立之謀報淳,淳斬敵裡首以獻,進封秦晉國,拜都元帥,賜金券,免漢拜禮,不名。許自擇將士,乃募燕、雲精兵。東至錦州,隊長武朝彥作亂,劫淳。淳匿而免,收朝彥誅之。會金兵至,聚兵戰於阿里軫斗,敗績,則亡卒數千人拒之。淳入潮,釋其罪,詔南京刻石紀功。



 保大二年,天祚入夾山,奚王加礅保、林牙耶律大石等引唐靈武故事,議欲立淳。淳不從,官屬勸進曰:「主
 上蒙塵,中原擾攘,若不王,百姓何歸?宜熟計之。」遂即位。百官上號天錫皇帝,改保大二年為建福元年,大赦。放進士李寶信等一十九人,遙降天祚為湘陰王。以燕、雲、平、上京、中京、遼西六路,淳主之;沙漠以北、南北路兩都招討府、諸蕃部族等,仍隸天祚。自此遼國分矣。封其妻普賢女為德妃,以回離保知北院樞密使,軍旅之事悉委大石。又遣使報宋,免歲幣,結好。宋人發兵問罪,擊敗之。尋遣使奉表於金,乞為附庸。



 事未決,淳病死,年六十。百官偽溢曰孝章皇帝,廟號宣宗,葬燕西香山永安陵。



 遺命遙立秦王定以存社稷,德妃為皇太后,稱制,
 改建福為德興元年,放進士李球等百八人。時宋兵來攻,戰敗之,由是人心大悅,兵勢日振。宰相李純等潛納宋兵,居民內應,抱關者被殺甚眾。翌日,攻內東門,衛兵力戰,宋軍大潰,逾城而走,死者相藉。五表於金,求立秦王,不從。而金兵大至,德妃奔天德軍,見天祚。天祚怒,誅德妃,降淳庶人,除其屬籍。



 耶律雅里者,天祚皇帝第二子也,字撒鸞。七歲,欲立為皇太子,別置禁衛,封梁王。



 保大三年,金師圍青塚寨,雅里在軍中。太保特母哥挾之出走,問道行至陰山。聞天祚失利趨雲內,雅里馳赴。時扈從者千餘人,多於天祚。天祚慮特母哥生變,欲誅之。
 責經不能全救諸王,將訊之。仗劍召雅裡問曰:「特母哥教汝何為?」



 雅裡對曰:「無他言。」乃釋之。



 天祚渡河奔夏,隊帥耶律敵列等劫雅里北走。至沙嶺,見蛇橫道而過,識者以為不祥。後三日,群僚共立雅里為主。雅裡遂即位,改元神歷,命士庶上便宜。



 雅里性寬大,惡誅殺。獲亡者,笞之而已。有自歸者,即官之。因謂左右曰「欲附來歸;不附則去。何須威副耶?」每取唐《貞觀政要》及林牙資忠所作《治國詩》,令待從讀之。



 烏古部節度使糾哲、迭裂部統軍撻不也、都監突里不等各率其眾來附。自是諸部繼至。而雅裏日漸荒怠,好擊鞠。特母哥切諫,乃不復出。以
 耶律敵列為樞密使,特母哥副之。敵裂劾西北路招討使蕭糾里熒惑眾心,志有不臣,與其子麻涅並誅之。



 以遙設為招討使,與諸部戰,數敗,杖免官。



 從行有疲困者,輒振給之。直長保德諫曰:「今國家空虛,賜賚若此,將何以相給耶?」雅里怒曰:「昔畋於福山,卿誣豬官,今復有此言。若無諸部,我將何取?」不納。初,令群牧運鹽濼倉粟,而民盜之,議籍以償。雅里乃自為直:每粟一車,償一羊;三車一牛;五車一馬;八車一駝。左右曰:「今一羊易粟二斗且不可得,乃償一車!」雅里曰:「民有則我有。



 若令盡償,民何堪?」



 後豬查刺山,一日而射黃羊四十,狼二十一,因致
 疾,卒,年三十。



 耶律大石者,世號為西遼。大石字重德,太祖八代孫也。通遼、漢字,善騎射,登天慶五年進士第,擢翰林應奉,尋升承旨。遼以斡林為林牙,故樂大石林牙。歷泰、祥二州刺史,遼興軍節度使。



 保大二年,金兵日逼,天祚播越,與諸大臣立秦晉王淳為帝。淳死,立其妻蕭德妃為太后,以守燕。及金兵至,蕭德妃歸天祚。天祚怒誅德妃而責大石曰:「我在,汝何敢立淳?」



 對曰:「陛下以全國之勢,不能一拒敵,棄國遠遁,使黎民塗炭。即立十淳,皆太祖子孫,豈不勝乞命於他人耶。」上無以答,賜酒食,赦其罪。



 大石不自安,遂殺蕭乙薛、坡裡括,自立為王,離
 鐵騎二百宵遁。北行三日,過黑水,見白達達詳穩床古兒。床古兒獻馬四百,駝二十,羊若干。西至可敦城,駐北庭都護府,會威武、崇德、會蕃、新、大林、柴河、駝等七州及大黃室韋、敵刺、王紀刺、茶赤刺、也喜、鼻古德、尼刺、達刺乖、達密里、密兒紀、合主、烏古里、阻卜、普速完、唐古、忽母思、奚的、纍而畢十八部王眾,諭曰:「我祖宗艱難創業,歷世九主,歷年二百。金以臣屬,逼我國家,殘我黎庶,屠翦我州邑,使我天祚皇帝蒙塵於外,日夜闖心疾首。我今伏義而西,欲借力諸蕃,翦我仇敵,復我疆宇。惟爾眾亦有軫我國家,憂我社稷,思兵救君父,濟生民於難者乎?」
 遂得精兵萬餘,置官吏,立排甲,具器仗。



 明年二月甲午,以青牛白馬祭天地、祖宗、整旅而西。先遺書回鶻王畢勒哥曰:「或我太祖皇帝北征,過卜古罕城,即遣使至甘州,詔爾祖烏母主曰:『汝思故國耶,朕即為汝復之;汝不能返耶,朕則有之。在朕,猶在爾也。』爾祖即表謝,以為遷國於此,十有餘世,軍民皆安土重遷,不能復返矣。是與爾國非一日之好也。今我將西至大食,假道炙國,其物致疑。」畢勒哥得書,即迎至邸,大宴三日。臨行,獻馬六百,駝百,羊三千,願質子孫為附庸,送至境外。所過,敵者勝之,降者安之。兵行萬里,歸者數國,獲駝、馬、牛、羊、財物,不
 可勝計。軍勢日盛,銳氣日倍。



 至尋思乾,西域諸國舉兵十萬,號忽兒珊,來拒戰。兩軍相望二里許。諭將士曰:「彼軍雖多而無謀,攻之,則首尾不救,我師必勝。」遣六院司大王蕭斡裏刺、招討副使耶律松山等將兵二千五百攻其右;樞密副使蕭刺阿不、招討使耶律術薛等將兵二千五百攻其左;自以眾攻其中。三軍俱進,忽兒珊大敗,殭尸數十里。駐軍尋思乾凡九十日,回回國王來降,貢方物。



 又西至起兒漫,文武百官冊立大石為帝,以甲辰歲二月五日即位,年三十八,號葛兒罕。復上漢尊號曰天祐皇帝,改元延慶。追謚祖父為嗣元皇帝,祖母為
 宣義皇后,冊元妃蕭氏為昭德皇后。因謂百官曰:「朕與卿等行三萬里,跋涉沙漠,風夜艱勤。賴祖宗之福,卿等之力,冒登大位。爾祖爾父宜加恤典,共享尊榮。」自蕭斡里剌等四十九人祖父,封爵有差。



 延慶三年,班師東歸,馬行二十日,得善地,遂建都城,號虎思斡耳朵,改延慶為康國元年。三月,以六院司大王蕭斡裏刺為兵馬都元帥,敵刺部前同知樞密院事蕭查刺阿不副之,茶赤刺部禿魯耶律燕山為都部署,護衛耶律鐵哥為都監,率七萬騎東征。以青牛白馬祭天,樹旗以誓於眾曰:「我大遼自太祖、太宗艱難而成帝業,其後嗣君耽樂無厭,
 不恤國政,盜賊蜂起,天下土崩。朕率爾眾,遠至朔漠,其復大業,以光中興。



 此非朕與爾世居之地。」申命元帥斡裏刺曰:「今當其往,倍賞必罰,與士卒同甘苦,擇善水草以立迸量敵而進,毋自取禍敗也。」行萬餘里無所得,牛馬多死,勒兵而還。大石曰:「皇天弗順,數也!」康國十年歿,在位二十年,廟號德宗。



 子夷列年幼,遺命皇后權國。後名塔不煙,號感主皇后,稱制,改元咸清,在位七拉。子夷列即位,改元紹興。籍民十八歲以上,得八萬四千五百戶。在位十三年歿,廟號仁宗。



 子幼,遣詔以妹普速完權國,稱制,改元崇福,號承天太后。後與駙馬蕭朵魯不弟
 樸古只沙裏通,出駙馬為東平王,羅織殺之。駙馬父斡裏刺以兵圍其宮,射殺普速無及樸古只沙裏。



 普速完在位十四年。



 仁宗次子直魯古即位,改元天禧,在位三十四年。時秋出豬,乃蠻王屈出律以伏兵八千擒之,而據其位。遂襲遼衣冠,尊直魯古為太上皇,皇后為皇太后,朝夕問起居,以待終焉。



 直魯古死,遼絕。



 耶律淳在天祚之世,歷王大國,受賜金券,贊拜不名。一時恩遇,無與為此。當天祚播越,以都元帥留守南京。獨不可奪大義以激燕民及諸大臣,興勤王之師,東拒金而迎天祚乎?



 乃自取之,是篡也。況忍王天祚哉?



 大石既帝淳而王天
 祚矣,復歸天祚。天祚責以大義,乃自立為王而去之。幸藉祖宗餘威遺智,建號萬里之外。雖寡母弱子,更繼迭承,歲九十年,亦可謂難矣。



 然淳與雅里、大石之立,皆在天祚之世。有君而復君之。



 其可乎哉?諸葛武侯為獻帝發喪,而後立先主為帝者,不可同年語矣。故著以為戒云。



 贊曰:遼起朔野,兵甲之盛,鼓行霵外,席卷河朔,樹晉植漢,何其壯歟?太祖、太宗乘百戰之勢,輯新造之邦,英謀睿略,可謂遠矣。雖以世宗中才,穆宗殘暴,連遘弒逆,而神器不搖。蓋由祖宗威令猶足以震疊其國人也。



 聖宗
 以來,內修政治,外拓疆宇。既而申固鄰好,四境乂安。維持二百餘年之基,有自來矣。



 降臻天祚,既丁末運,又觖人望,崇信奸回,自椓國本,群下離心。金兵一集,內難先作,廢立之謀,叛亡之跡,相繼蜂起。馴致土崩瓦解,不可復支,良可哀也!耶律與蕭,世為甥舅,義同休戚。奉先挾私滅公,首禍構難,一致於斯,天祚窮蹙,始悟奉先誤己,不幾晚乎!



 淳、雅里所謂名不正,言不順,事不成者也。大石茍延,彼善於此,亦幾何哉?



\end{pinyinscope}