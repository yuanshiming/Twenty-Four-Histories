\article{卷一百八 列傳第四十六 國語解}

\begin{pinyinscope}

 史
 自遷、固,以迄《晉》、《唐》,其為書雄深浩博,讀者未能盡曉。於是裴駰、顏師古、李賢、何超、董沖諸儒,訓詁音釋,然後制度、名物、方言、奇字,可以一覽而周知。其有助於後學多矣。



 遼之初興,與奚、室韋密邇,土俗言語大概近俚。至太祖、太宗,奄有朔方,其治雖參用漢法,而先世奇首、遙輦之制尚多存者。子孫相繼,亦尊守而不易。故史之所
 載,官制、宮衛、部族、地理,率以國語為之稱號。不有注釋以辨之,則世何從而知,後何從而考哉。今即本史參互研究,撰次《遼國語解》以附其後,庶幾讀者無齟齬之患云。



\end{pinyinscope}