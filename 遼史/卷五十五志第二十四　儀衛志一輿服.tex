\article{卷五十五志第二十四 儀衛志一輿服}

\begin{pinyinscope}

 遼太祖奮自朔方,及宗繼志述事,以成其業。於是舉渤海,立敬瑭,破重貴,盡致周、秦、兩漢、隋、唐文物之遺餘而居有之。路車法物以隆等威,金符玉璽以布號令。是以傳至九主二百餘年,豈獨以兵革之利,士馬之強哉。文謂之義,武謂之衛,足以成一代之規模矣。



 考遼所有輿服、符璽、儀仗,作《儀衛志》。



 輿服自黃帝而降,輿服這制,其來遠矣。禹乘四載作小車,商人得桑根之瑞為大輅,周人加金玉,象飾益備。秦取六國儀物,而分別其用,先王之制,置而弗御。至漢中葉,銳意稽古,然禮文之事,名存實亡,蓋得一於千百焉。唐之車輅因周、隋遺法,損益可知。而祭服皆青,朝服皆絳,常服用宇文制,以紫、緋、綠、碧分品秩。五代頗以常服代朝服。遼國自太宗入晉之後,皇帝與南班漢官用漢服;太后與北班契丹臣僚用國服,其漢服即五人晉之遺制也。



 考之載籍之可徵者,著《輿服篇》,冠諸《儀衛》之首。



 國輿契丹故俗,便於鞍馬。隨水草遷徙,則有氈車,任載有大車,婦人乘馬,亦有小車,貴富者加之華飾。禁制疏闊,貴適用而已。帝後加隆,勢固然也。輯其可知著於篇。



 大輿,《柴冊再生儀》載神主見之。



 輿,《臘儀》見皇帝、皇后升輿、降輿。



 總纛車,駕以御駝。《祭山儀》見皇太后升總纛車。



 車,《納后儀》見皇后就車。



 青甉車,二螭頭、蓋部皆飾以銀,駕用駝,公主下嫁以賜之。古者王姬下嫁,車服不系其夫,下王後一等。
 此其遺意歟。



 送終車,車樓純飾以錦,螭頭似銀,下縣鐸,後垂大氈,駕以牛。上載羊一,謂之祭羊,以擬送終之用。亦賜公主。



 椅,《冊皇太后儀》,皇帝乘椅,自便殿輿至西便門。



 鞍馬,《祭山儀》,皇帝乘馬,侍皇太后行。《臘儀》,皇帝降殭,祭東畢,乘馬入獵圍。《瑟瑟儀》,俱乘馬東行,群臣在南,命婦在北。



 漢輿及宗皇帝會同元年,晉使馮道劉顒等備車輅法物,上
 皇帝、皇太后尊號冊禮。自此天子車服昉見於遼。太平中行漢冊禮,乘黃令陳車略,尚輦奉御陳輿輦。盛唐輦輅,盡在遼遷矣。



 五輅:《周官》典輅有五輅。秦亡之後,漢創制。



 玉輅,祀天、祭地、享宗廟、朝賀、納後用之。青質,玉飾,黃屋,左纛。十二鑾在衡,二鈴在軾。龍輈左建旗,十二斿,皆畫升龍,長曳地。駕蒼龍,金葼,鏤錫,鞶纓十二就。



 遼國《勘箭儀》,皇帝乘玉輅至內門。聖宗開泰十年,上升玉輅自內三門入萬壽殿,進七廟御容酒。



 金輅,饗射,祀還、飲至用之。赤質,金飾,餘如玉輅,色從
 其質。駕赤騮。象輅,行道用之。黃質,象飾,餘如金輅。駕白翰。



 革輅,巡狩、武事用之。白質,革鞔。駕白翰。



 木輅,田獵用之。黑質,漆飾。駕黑駱。



 車:制小於輅,小事乘之。



 耕根車,耕藉用之。青質,蓋三重,餘如玉輅。



 安車,一名進賢車,臨幸用之。金飾,重輿,曲壁,八鑾在衡,紫油纁朱裹甉,朱絲絡網。駕赤騮,朱霵纓。四望車,一名明遠車,拜陵,臨吊則用之。金飾,青油纁朱裹通甉。駕牛,餘同安車。



 涼車,赤質,省方、罷獵用之。赤質,金塗,銀裝。五彩龍鳳織,藤油壁,緋條,蓮座。駕以橐駝。



 輦:用人挽,本宮中所乘。唐高宗始制七輦。《周官》巾車有輦,以人組挽之。玉平冊禮,皇帝御輦。



 大鳳輦,赤質,頂有金鳳,壁畫雲氣金翅。前有軾,下有構欄。絡帶皆繡雲鳳,銀梯。主輦八十人。



 大芳輦。



 仙游輦。



 小輦,《永壽節儀》,皇太后乘小輦。



 芳亭輦,黑質,幕屋緋欄,皆繡雲鳳。朱綠夾窗,花板紅
 網,兩簾四竿,銀飾梯。主輦百廿人。



 大玉輦。



 小玉輦。



 逍遙輦,常用之。棕屋,赤質,金塗,銀裝,紅條。輦官十二人,春夏緋衫,秋冬素錦服。



 平頭輦,常行用之。制如逍遙,無屋。冊承天皇太后儀,皇太后乘平頭輦。



 步輦,聖宗統和三年,駐蹕土河,乘步輦聽政。羊車,古輦車。赤質,兩壁龜文,鳳翅,緋甉,絡帶、門簾皆繡瑞羊,畫輪。駕以牛,隋易果下馬。童子十八人,服
 繡。



 瑞羊輓之。



 腰輿,前後長竿各二,金銀螭頭,緋繡鳳襴,上施錦褥,別設小床。奉輿十六人。



 小輿,赤質,青頂,曲柄,緋繡絡帶。制如鳳輦而小,上有御座。奉輿二十四人。



 皇太子車輅:金輅,從祀享、正冬大朝、納妃用之。《冊皇太子儀》,乘黃令陳金輅,皇太子升、降金輅。



 軺車,五日常朝、享宮臣、出入行道用之。金飾,紫甉朱
 裹。駕一馬。



 四望車,吊臨用之。金飾,紫油纁通甉。駕一馬。



\end{pinyinscope}