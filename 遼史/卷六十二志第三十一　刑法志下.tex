\article{卷六十二志第三十一 刑法志下}

\begin{pinyinscope}

 興宗即位,欽哀皇后始得志,昆弟專權。馮家奴等希欽哀意,誣蕭浞卜等謀反,連及嫡後仁德皇后。浞卜等十餘人與仁德姻援坐罪者四十餘輩,皆被大闢,仍籍其家。幽仁德於上京,既而遣人弒之。迫殞非命,中外切憤。欽哀後謀廢立,遷於慶州。及奉迎以歸,頗復預事,其酷虐不得逞矣。然興宗好名,喜變更,又溺浮屠法,務行小
 惠,數降赦宥,釋死囚甚眾。



 重熙元年,詔職事官公罪聽贖,私罪各從本法;子弟及家人受賕,不知情者,止坐犯人。先是,南京三司銷錢作器皿三斤,持錢出南京十貫,及盜遺火家物五貫者處死;至是,銅逾三白,持錢及所盜物二十貫以上處死。二年,有司奏:「元年詔曰,犯重罪徒終身者,加以捶楚,而又黥面。是犯一罪而具三刑,宜免黥。其職事官及宰相、節度使世選之家子孫,犯奸罪至徒者,未審黥否?」上諭曰:「犯罪而悔過自新者,亦有可用之人,一黥其面,終身為辱,朕甚憫焉。」後犯終身徒者,止刺頸。奴婢犯逃,若盜其主物,主無得擅黥其面,刺臂
 及頸者聽。犯竊盜者,初刺右臂,再刺在,三刺頸之右,四刺左,至於五則處死。五年,《新定條制》成,詔有司凡朝日執之,仍頒行諸道。蓋纂修太祖以來法令,參以古制。其刑有死、流、杖及三等之徒,而五凡、五百四十七條。



 時有群牧人竊易官印以馬與人者,法當死,帝曰:「一馬殺二人,不亦甚乎?」減死論。又有兄弟犯強盜當死,以弟從兄,且俱無子,特原其弟。至於枉法受賕,詐敕走遞,偽學御書,盜外國貢物者,例皆免死。郡王貼不家奴彌里吉告其主言涉怨望,鞫之無驗,當反坐,以欽哀皇後里言,竟不加罪,亦不斷付其主,僅籍沒焉。寧遠軍節度使蕭白
 強掠烏古敵烈都詳穩敵魯之女為妻,亦以後言免死,杖而奪其官。梅裏狗丹使酒殺人而逃,會永壽節出首,特赦其罪。皇妹秦國公主生日,帝幸其第,伶人張隋,本宋所遣汋者,大臣覺之以聞。召詰,款伏,乃遽釋之。後詔諸職官私取官物者,以正盜論。諸帳郎君等於禁地射鹿,決杖三百,不征償;小將軍決二百已下;至百姓犯者決三百。聖宗之風替矣。



 道宗清寧元年,詔諸宮都部署曰:「凡有機密事,即可面奏;餘所訴事,以法施行。有投誹訕之書,其受及讀者皆棄市。」



 二年,命諸郡長吏和諸部例,與僚同決罪囚,無致枉死獄中。



 下詔曰:「先時諸路
 死刑皆待決於朝,故獄訟留滯;自今凡強盜得實者,聽即決之。」四年,復詔左夷離畢曰:「比詔外路死型,聽所在官司即決。然恐未能悉其情,或有枉者。自今雖已款伏,仍令附近官司覆司。無冤然後決之,有冤者即具以聞。」



 咸雍元年,詔獄囚無家者,給以糧。六年,帝以契丹、漢人風俗不同,國法不可異施,於是命惕隱蘇、樞密使乙辛等更定《條制》。凡合於《律令》者,具載之;其不合者,別存之。



 時校定官即重熙舊制,更竊盜贓二十五貫處死一條,增至五十貫處死;又刪其重復者二條,為五百四十五條;取《律》一百七十三條,又創增七十一條,凡七百八十
 九條,增重編者至千餘條。皆分類列。以大康間所定,復以《律》及《條例》參校,續增三十六條。其後因事續校,至大安三年止,又增六十七條。



 條約既繁,典者不能遍習,愚民莫知所避,犯法者眾,吏得因緣為奸。故五年詔曰:「法者所以示民信,而致國治。簡易如天地,不忒如四時,使民可避而不可犯。比命有司纂修刑法,然不能明體朕意,多作條目,以罔民於罪,朕甚不取。自今復用舊法,餘悉除之。」



 然自大康元年,北院樞密使耶律乙辛等用事。宮婢單登等誣告宣懿皇后,乙辛以聞,即詔乙辛劾狀,因實其事。上怒,族伶人趙惟一,斬高長命,皆籍其家,仍
 賜皇后自盡。三年,乙辛又與其黨謀構昭懷太子,陰令右護衛太保耶律查剌,告知樞密院事蕭速撒等八人謀立皇太子。詔按無狀,出速撒、達不也外補,流護衛撒撥等六人。詔告首謀逆者,重加官賞;否則悉行誅戮。乙辛教牌印郎君蕭訛都斡自首「臣嘗預速撒等謀」,因籍姓名以告。帝信之,以乙辛等鞫按,至杖皇太子,囚之宮中別室,殺撻不也、撒剌等三十五人,又殺速撒等諸子;其幼稚及婦女、奴婢、家產,皆籍沒之,或分賜群臣。燕哥等詐為太子爰書以聞,上大怒,廢太子,徙上京,乙辛尋遣人殺於囚所。帝猶不寤,朝遷上下,無復紀律。



 天祚乾
 統元年,凡大康三年預乙辛所害者悉復官爵,籍沒者出之,流放者還鄉里。至二年,始發乙辛等,剖棺戮尸,誅其子孫,餘黨子孫減死,徙邊,其家屬奴婢皆分賜被害之家。



 如耶律撻不也、蕭達魯古等,黨人之尤兇狡者,皆以賂免。至於覆軍失城者,第免官而已。行軍將軍耶律涅里三人有禁地射鹿之罪,皆棄市。其職官諸局人有過者,鐫降決斷之外,悉從軍賞罰無章,怨讟日起;劇盜相挻,叛亡接踵。天祚大恐,益務繩以嚴酷,由是投崖、炮擲、釘割、臠殺之型復興焉。或有分尸五京,甚者至取其心獻祖廟。雖由天祚救患無策,流為殘忍,亦由祖
 宗有以啟之也。遼之先代,作法尚嚴。使春子孫皆有君人之量,知所自擇,猶非祖宗貽謀之道;不幸一有昏暴者,少引以藉口,何所不至。然遼之季世,與其先代用刑同,而興亡異者何歟?蓋創業之君,旋立於法未定之前,民猶未敢測也;亡國之主,施之於法既定之後,民復何所賴焉。此其所為異也。



 傳曰:「新國輕典。」豈獨權事宜而已乎?



 天祚末年,游畋無度,頗有倦勤意。諸子惟文妃所生敖盧斡最賢。蕭奉先乃元妃兄,深忌之。會文妃之女兄適耶律撻曷里,女弟適耶律餘睹,奉先乃誣告余睹等謀立晉王,尊天祚為太上皇。遂戮撻曷里及其妻,賜
 文妃自盡,敖盧斡以不與謀得免。及天祚西狩奉聖州,又以耶律撒八等欲劫立敖盧斡,遂誅撒八,盡其黨與。敖盧斡以有人望,即日賜死。當時從行百官、諸局承應人及軍士聞者,皆流涕。



 蓋自興宗時,遽起大獄,仁德皇后戕於幽所,遼政始衰。



 道宗殺宣懿皇店,遷昭懷太子,太子尋被害。天祚知其父之冤,而已亦幾殆,至是又自殺其子敖盧斡。傳曰:「於所厚者薄,無所不薄矣。」遼二百餘年,骨肉屢相殘滅。天祚荒暴尤甚,遂至於亡。噫!



\end{pinyinscope}