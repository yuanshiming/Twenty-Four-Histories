\article{卷八十五 列傳第二十三}

\begin{pinyinscope}

 蕭惠慈氏奴蕭迂魯鐸盧斡蕭圖玉耶律鐸軫蕭惠,字伯仁,小字脫古思,淳欽皇后弟阿古只五世孫。



 初以中宮親,為國舅詳穩。從伯父排押征高麗,至奴古達北嶺,高麗阻險以拒,惠力戰,破之。及攻開京,以軍律整肅聞,授契丹行宮都部署。開泰二年,改南京統軍使。未幾,為右夷離畢,加同中書門下平章事。朝議以遼東
 重地,非勛戚不能鎮撫,乃命惠知東京留守事。改西北路招討使,封魏國公。



 太平六年,討回鶻阿薩蘭部,徵兵諸路,獨阻卜酋長直剌後期,立斬以徇。進至甘州,攻圍三日,不克而還。時直剌之子聚兵來襲,阻卜酋長烏八密以告,惠未之信。會西阻卜叛,襲三克軍,都監涅魯古、突舉部節度使諧理、阿不呂等將兵三千來救,遇敵於可敦城西南。諧理、阿不呂戰歿,士卒潰散。



 惠倉卒列陣,敵出不意攻我營。眾請乘時奮擊,惠以我軍疲敝,未可用,弗聽。烏八請以夜所營,惠又不許。阻卜歸,惠乃設伏兵擊之。前鋒始交,敵敗走。惠為招討累年,屢遭侵掠,士
 馬疲困。七年,左遷南京侍衛親軍馬步軍都指揮使,尋遷商京統軍使。



 興宗即位,知興中府,歷順義軍節度使、東京留守、西南面招討使,加開府儀同三司、檢校太師,兼侍中,封鄭王,賜推誠協謀竭節功臣。重熙六年,復為契丹行宮都部署,加守太師,徙王趙。拜雨院樞密使,更王齊。



 是時帝欲一天下,謀取三關,集群臣議。惠曰:「兩國強弱,聖慮所悉。宋人西征有年,師老民疲,陛下親率六軍臨之,其勝必矣。」蕭孝穆曰:「我先朝與宋和好,無罪伐之,其曲在我;況勝敗未可逆料。願陛下熟察。」帝從惠言,乃遭使索宋十城,會諸軍於燕。惠與太弟帥師壓宋境,
 宋人重失十城,增歲幣請和。惠以首事功,進王韓。十二年,兼北府宰相,同知元帥府事,又為北樞密使。



 十三年,夏國李元昊誘山南黨項諸部,帝親征。元昊懼,請降。惠曰:「元昊忘奕世恩,萌奸計,車駕親臨,不盡歸所掠。天誘其衷,使彼來迎。天與不圖,後悔何及?」帝從之。



 詰旦,進軍。夏人列拒馬於河西,蔽盾以立,惠擊敗之。元昊走,惠麾先鋒及右翼邀之。夏人千餘潰圍出,我師逆擊,大風忽起,飛沙瞇目,軍亂,夏人乘之,蹂踐而死者不可勝計。詔班師。十七年,尚帝姊秦晉國長公主,拜附馬都尉。明年,帝復征夏國。惠自河南進,戰艦糧船綿互數百里。既入
 敵境,偵候不遠,銷甲載於車,軍士不得乘馬。諸將咸請備不虞,惠曰:「諒祚必自迎車駕,何暇及我?無故設備,徒自弊耳。」數日,我軍未營。候者報夏師至,惠方詰妄言罪,諒祚軍從阪而下。



 惠與麾下不及甲而走。追者射惠,幾不免,軍士死傷尤眾。師還,以惠子慈氏奴效於陣,詔釋其罪。



 十九年,請老,詔賜肩輿入朝,策杖上殿。辭章再上,乃許之,封魏國王。詔冬夏赴行在,參決疑議。既歸,遣賜湯藥及他錫齎不絕。每生日,輒賜詩以示尊寵。清寧二年薨,年七十四,遺命家人薄葬。訃聞,輟朝三日。



 惠性寬厚,自奉儉薄。興宗使惠恣取珍物,惠曰:「臣以戚屬據要
 地,祿足養廉,奴婢千餘,不為闕乏。陛下猶有所賜,貧於臣者何以待之。」帝以為然。故為將,雖數敗衄,不之罪也。



 弟虛列,武定軍節度使。二子:慈氏奴,兀古匿。兀古匿終北府宰相。



 慈氏奴,字寧隱。太平初,以戚屬補祗候郎君。上愛其勤慎,升閘撒狘,加右監門衛上將軍。



 西邊有警,授西北路招討都監,領保大軍節度使。政濟恩威,諸部悅附。入為殿前副點檢,歷烏古敵烈部詳穩。征李諒祚,為統軍都監,與西北路招討使敵魯古率蕃部諸軍由北路趨涼州,獲諒祚親屬。夏人扼險以拒,慈氏奴中流矢卒,年五十一,贈中書門下平章事。



 蕭迂魯,字胡突堇,五院部人。父約質,歷官節度使。



 迂魯重熙間為牌印郎君。清寧九年,國家既平重元之亂,其黨郭九等亡,詔迂魯追捕,獲之,遷護衛太保。咸雍元年,使宋議邊事,稱旨,知殿前副點檢事。



 五年,阻卜叛,為行軍都監,擊敗之,俘獲甚眾。初軍出,止給五月糧,過期糧乏,士卒往往叛歸。迂魯坐失計,免官,降戍西北部。末行,會北部兵起,迂魯將烏古敵烈兵擊敗之,每戰以身先,由是釋前罪,命總知烏古敵烈部。



 九年,敵烈叛,都監耶律獨迭以兵少不戰,屯臚朐河。敵烈合邊人掠居民,迂魯率精騎四百力戰,敗之,盡獲其輜重。



 繼聞酋長合術
 三千餘騎掠附近部落,縱兵躡其後,連戰二日,斬數千級,盡得被掠人畜而還。值敵烈黨五百餘騎劫捕鷹戶,逆擊走之,俘斬甚眾,自是敵烈勢沮。



 時,敵烈方為邊患,而阻卜相繼寇掠,邊人以故疲弊。朝廷以地遠,不能時益援軍,而使疆圉帖然者,皆迂魯力也。帝嘉其功,拜左皮室詳穩。



 會宋求天池之地,詔迂魯兼統兩皮室軍屯太牢古山以備之。大康初,阻卜叛,遷西北招討都監,從都統耶律趙三征討有功,改南京統軍都監、黃皮室詳穩。未幾,遷東北路統軍都監,卒。弟鐸盧斡。



 鋒盧斡,字撒板。幼警悟異常兒。三歲失母、哭盡哀,見者
 傷之。及長,魁偉沉毅,好學,善屬文,有才幹。年三十始仕,為朝野推重,給事北院知聖旨事。



 大康二年,乙辛再入樞府,鐸盧斡素與蕭巖壽善,誣以罪,謫戍西北部。坐皇太子事,特恩減死,仍錮終身。在戍十餘年,太子事稍直,始得歸鄉里,屏居謝人事。一日臨流,聞雉鳴,三復孔子「時哉」語,作古詩三章見志。當時名士稱其高情雅韻,不減古人。



 壽隆六年卒,年六十一。乾統初,贈彰義軍節度使。蕭圖玉,字兀衍,北府宰相海瓈之子。



 統和初,皇太后稱制,以戚屬入侍,尋為烏古部都監。討速母縷等部有功,
 遷烏古部節度使。十九年,總領西北路軍事。



 後以本路兵伐甘州,降其酋長牙懶。既而牙懶復叛,命討之,克肅州,盡遷其民於土隗口故城。師還,詔尚金鄉公主,拜附馬都尉,加同政事令門下平章事。



 上言曰:「阻卜今已服化,宜各分部,治以節度使。」上從之。自後,節度使往往非材,部民怨而思叛。開泰元年十一月,石烈太師阿裏底弒其節度使,西奔窩魯朵城,蓋古所謂龍庭單于城也。已而,阻卜復叛,圍圖玉於可敦城,勢甚張。圖玉使諸軍齊射卻之,屯於窩魯朵城。明年,北院樞密使耶律化哥引兵來救,圖玉遣人誘諸部皆降。帝以圖玉始雖失計,後
 得人心,釋之,仍領諸部。請益軍,詔讓之曰:「叛者既服,兵安用益?且前日之役,死傷甚眾,若從汝謀,邊事何時而息。」



 遂止。會公主坐弒家婢,降封郡主,圖玉罷使相。尋起為烏古敵烈部詳穩。以老代,還卒。子雙古,南京統軍使。孫訛篤斡,尚三韓郡王合魯之女骨浴公主,終烏古敵烈部統軍使,以善戰名於世。



 耶律鐸軫,字敵輦,積慶宮人。仕統和間。性疏簡,不顧小節,人初以是短之,後侵宋,分總羸師以從。及戰,取徘帛被介胄以自標顯,馳突出入敵陣,格殺甚眾。太后望見喜,召謂之曰:「卿戮力如此,何患不濟!」厚賞之。由是多以
 軍事屬任。俄授東北詳穩。開泰二年,進討阻卜,克之。



 重熙間,歷東北路統軍使、天德軍節度使。十七年,城西邊,命鐸軫相地及造戰艦,因成樓船百三十艘。上置兵,下立馬,規制堅壯,稱旨。及西征,詔鐸軫率兵由別道進,會於河濱。敵兵阻河而陣,帝御戰艦絕河擊之,大捷而歸,親賜卮酒。



 仍問所欲,鐸軫對曰:「臣幸被聖恩,得效駑力,萬死不能報國,又將何求?」帝愈重之,手書鐸軫衣裙曰:「勤國忠君,舉世無雙。」卒於官,年七十。子低烈,歷觀察、節度使。



 論曰:「初,遼之謀復三關也,蕭惠贊伐宋之舉,而宋人增
 幣請和。狃於一勝,移師西夏,而勇智俱廢,敗潰隨之。豈非貪小利,迷遠圖而然。況所得不償所亡,利果安在哉?同時諸將撫綏邊因,若迂魯忠勤不伐,鐸魯斡高情雅韻,鐸軫雖廉不逮蕭惠,而無邀功啟釁之罪,亦庶乎君子之風矣。」



\end{pinyinscope}