\article{卷六十四列傳第二 宗室}

\begin{pinyinscope}

 義宗倍子平王隆先晉王道隱章肅皇帝李胡子宋王喜隱順宗浚晉王敖盧斡義宗,名倍,小字圖欲,太祖長子,母淳欽皇后蕭氏。幼聰敏好學,外寬內摯。神冊元年春,立為皇太子。



 時太祖問侍臣曰:「受命之君,當事天敬神。有大功德者,朕欲祀之,
 何先?」皆以佛對。太祖曰:「佛非中國教。」倍曰:「孔子大聖,萬世所尊,宜先。」太祖大悅,即建孔子廟,詔皇太子春秋釋奠。



 嘗從征烏古、黨項,為先鋒都統,及經略燕地。太祖西征,留倍守京師,因陳取渤海計。天顯元年,從征渤海。拔扶餘城,上欲括戶口,倍諫曰:「今始得地而料民,民必不安。若乘破竹之勢,徑造忽汗城,克之必矣。」太祖從之。倍與大元帥德光為前鋒,夜圍忽汗城,大諲撰窮蹙,請降。尋復叛,太祖破之。改其國曰東丹,名其城曰天福,以倍為人皇王主之。仍賜天子冠服,建元甘霹,稱制,置左右大次四相及百官,一用漢法。歲貢布十五萬端,馬千匹。
 上諭曰:「此地瀕海,非可久居,留汝撫治,以見朕愛民之心。」駕將還,倍作歌以獻。陛辭,太祖曰:「得汝治東土,吾復何憂。」倍號泣而出。遂如儀坤州。



 未幾,諸部多叛,大元帥討平之。太祖訃至,倍即日奔赴山陵。倍知皇太后意欲立德光,乃謂公卿曰:「大元帥功德及人神,中外攸屬,宜主社稷。」乃與群臣請於太后而讓位焉。



 於是大元帥即皇帝位,是為太宗。



 太宗即立,見疑,以東平為南京,徙倍居之,盡遷其民。



 又置衛士陰伺動稱。倍既歸國,命王繼遠撰《建南京碑》,起書樓於西宮,作《樂田園詩》。唐明宗聞之,遣人跨海持書密召倍。倍因畋海上。使再至,倍謂左
 右日:「我以天下讓主上,今反見疑;不如適他國,以成吳太伯之名。」立木海上,刻詩曰:「小山壓大山,大山全無力。羞見故鄉人,從此投外國。」



 攜高美人,載書浮海而去。



 唐以天子儀衛迎倍,倍坐船殿,眾官陪列上壽。至汴,見明宗。明宗以莊宗後夏氏妻之,賜姓東丹,名之曰慕華。改瑞州為懷化軍,拜懷化軍節度使、瑞慎等州觀察使。復賜姓李,名贊華。移鎮滑州,遙領虔州節度使。倍雖在異國,常思其親,問安之使不絕。



 後明宗養子從珂弒其君自立,倍密報太宗曰:「從珂弒君,盍討之。」及太宗立石敬塘為晉王,加兵於洛。從珂欲自焚,召倍與俱,倍不從,遣
 壯士李彥紳害之,時年三十八。有一僧為收瘞之。敬瑭入洛,喪服臨哭,以王禮權厝。後太宗改葬千醫巫閭山,謚曰文武元皇王。世宗即位,謚讓國皇帝,陵曰顯陵。統和中,更謚文獻。重熙二十年,增謚文獻欽義皇帝,廟號義宗,及謚二后曰端順,曰柔貞。倍初市書至萬卷,藏於醫巫閭絕頂之望海堂。通陰陽,知音律,精醫藥、砭爇之術。工遼、漢文章,嘗譯《陰符經》。



 善畫本國人物,如《射騎》、《獵雪騎》、《千鹿圖》,皆入宋秘府。然性刻急好殺,婢妾微過,常加刲灼。夏氏懼而求削發為尼。五子:長世宗,次婁國、稍、隆先、道隱,各有傳。



 平王隆先,字團隱,母大氏。



 景宗即位,始封平王。未幾,兼政事令,留守東京。薄賦稅,省刑獄,恤鰥寡,數薦賢能之士。後與統軍耶律室魯同討高麗有功,還薨,葬醫巫閭山之道隱谷。



 平王為人聰明,博學能詩,有《閬苑集》行於世。



 保寧之季,其子陳哥與渤海官屬謀殺其父,舉兵作亂,上命轘裂於市。



 晉王道隱,字留隱,母高氏。



 道隱生於唐,人皇王遭李從珂之害,時年尚幼,洛陽僧匿而養之,因名道隱。太宗滅唐,還京,詔賜外羅山地居焉。性沉靜,有文武才,時人稱之。



 景宗即位,封蜀王,為上京留守。乾亨元年,遷守南京,
 號令嚴肅,民獲安業。居數年,徙封荊王。統和初,病薨,追封晉王。



 論曰:「自古新造之國,一傳而太子讓,豈易得哉?遼之義宗,可謂盛矣!然讓而見疑,豈不兆於建元稱制之際乎?斯則一時君臣昧於禮制之過也。



 「束書浮海,寄跡他國,思親不忘,問安不絕,其心甚有足諒者焉。觀其始慕泰伯之賢而為遠適之謀,終疾陳恆之惡而有請討之舉,志趣之卓,蓋巳見於早歲先祀孔子之言歟。善不令終,天道難潔,得非性卞嗜殺之所致也!「雖然,終遼之代,賢聖繼統,皆其子孫。至德之報,昭然在茲矣。」



 章肅皇帝,小字李胡,一名洪古,字奚隱,太祖第三子,母淳欽皇后蕭氏。



 少勇悍多力,而性殘酷,小怒輒黥人面,或投水火中。太祖嘗觀諸子寢,李胡縮項臥內,曰:「是必在諸子下。」又嘗大寒,命三子採薪。太宗不擇而取,最先至;人皇王取其乾者束而歸,後至;李胡取少而棄多,既至,袖手而立。太祖曰:「長巧而次成,少不及矣。」而母篤愛李胡。



 天顯五年,遣徇地代北,攻寰州,多俘而還,遂立為皇太弟,兼天下兵馬大元帥。太宗親征,常留守京師。世宗即位鎮陽,太后怒,遣李胡將兵擊之,至泰德泉,為安瑞、留哥所敗。



 太后與世宗隔潢河而陣,各言舉兵意。耶
 律屋質入諫太后曰:「主上已立,宜許之。」時李胡在側,作色曰:「我在,兀欲安得立?」屋質曰:「奈公酷暴失人心何!」太后顧李胡曰:「昔我與太祖愛汝異於諸子,諺云:『偏憐之子不保業,難得之妃不主家。』我非不欲立妝,汝自不能矣。」及會議,世宗使解劍而言。和約既定,趨上京。會有告李胡與太后謀廢立者,徒李胡祖州,禁其出入。



 穆宗時,其子喜隱謀反,辭逮李胡,囚之,死獄中,年五十,葬玉峰山西谷。統和中,追謚欽順皇帝。重熙二十一年,更謚章肅,後曰和敬。二子:宋王喜隱、衛王宛。



 喜隱,字元德,雄偉善騎射,封趙王。應歷中,謀反,事覺,上
 臨問有狀,以親釋之。未幾,復反,下獄。景宗即位,聞有赦,自去其械而朝。上怒曰:「汝罪人,何得擅離禁所。」



 詔誅守者,復置於獄。及改元保寧,乃宥之,妻以皇后之姊,復爵,王宋。



 喜隱輕僄無恆,小得志即驕。上嘗召,不時至,怒而鞭之,由是憤怨謀亂。



 貶而復召,適見上與劉繼元書,辭意卑遜,諫曰:「本朝於漢為祖,書旨如此,恐虧國體。」帝尋改之。授西南面招討使,命之河東索吐蕃戶,稍見進用。復誘群小謀叛,上命械其手足,築園土囚祖州。宋降座二百餘人欲劫立喜隱,以城堅不得入,立其子留禮壽,上京留守除室擒之。留禮壽伏誅,賜喜隱死。
 論曰:「李胡殘醋驕盈,太祖知其不才而不能教,太后不知其惡而溺愛之。初以屋質之言定立世宗,而復謀廢立,子孫繼以逆誅,並及其身,可哀也已。



 「夫自太祖之世,刺葛、安瑞首倡禍亂,太祖既不之誅,又復用之,固為有君人之量。然惟太祖之才足以駕馭,庶乎其可也。李胡而下,宗王反側,無代無之,遼之內難,與國始終。



 厥後嗣君,雖嚴法以繩之,卒不可止。烏乎,創業垂統之主,所以貽厥孫謀者,可不審歟!」



 順宗,名浚,小字耶魯斡,通宗長子,母宣懿皇后蕭氏。



 幼而能言,好學知書。道宗嘗曰:「此子聰慧,殆天授歟!」



 六歲,
 封梁王。明年,從上獵,矢連發三中。上顧左右曰:「朕祖宗以來,騎射絕人,威震天下。是兒雖幼,不墜其風。」



 後遇十鹿,射獲其九。帝喜,設宴。八歲,立為皇太子。大康元年,兼領北南樞密院事。



 及母後被害,太子有憂色。耶律乙辛為北院樞密使,常不自安。會護衛蕭忽古謀害乙辛,事覺,下獄。副點檢蕭十三謂乙辛曰:「臣民心屬太子,公非閥閱,一日若立,吾輩措身何地!」乃與同知北院宣徽事蕭特裹特謀構陷太子,陰令右護衛太保耶律查刺誣告都宮使耶律撒刺、知院蕭速撒、護衛蕭忽古謀廢立。詔按無跡,不治。



 乙辛復令牌印郎君蕭訛都斡等言:「查
 刺前告非妄,臣實與謀,欲殺耶律乙辛等,然後立太子。臣若不言,恐事發連坐。」



 帝信之,幽太子於別室,以耶律燕哥鞫按。太子具陳枉狀曰:「吾為儲副,尚何所求。公當為我辨之。」燕哥乃乙辛之黨,易其言為款伏。上大怒,廢太子為庶人。將出,曰:「我何罪至是!」十三叱登車,遣衛士闔其扉。攫於上京,囚園堵中。



 乙辛尋遣達魯古、撒八往害之,太子年方二十,上京留守蕭撻得給以疾薨聞。上哀之,命有司葬龍門山。欲召其妃,乙辛陰遣人殺之。



 帝後知其冤,悔恨無及,謚曰昭懷太子,以天子禮改葬玉峰山。乾統初,追尊大孝順聖皇帝,廟號順宗,妃蕭氏貞
 順皇后。一子,延禧,即天祚皇帝。



 論曰:「道宗知太子之賢,而不能辨乙辛之詐,竟絕父子之親,為萬世惜。乙辛知為一身之計,不知有君臣之義,豈復知有太子乎!奸邪之臣亂人家國如此,可不戒哉!可不戒哉!」



 晉王,小字敖盧斡,天祚皇帝長子,母曰文妃蕭氏。



 甫髫齔,馳馬善射。出為大示相耶律隆運後,封晉王。性樂道人善,而矜人不能。時宮中見讀書者輒斥。敖盧斡嘗入寢殿,見小底茶刺閱書,因取觀。會諸王至,陰袖而歸之,曰:「勿令他人見也。」一時號稱長者。



 及長,積有人望,內外
 歸心。保大元年,南軍都統耶律餘睹與其母文妃密謀立之,事覺,余睹降金,文妃伏誅,敖盧斡實不與謀,免。二年,耶律撒八等復謀立,不克。上知敖盧斡得人心,不忍加誅,令縊殺之。或勸之亡,敖盧斡曰:「安忍為蕞爾之軀,而失臣子之大節。」遂就死。聞者傷之。



 論曰:「天祚不君,臣下謀立其子,適以殺之。敖盧斡重君父之命,不亡而死,申生其恭矣乎!」



\end{pinyinscope}