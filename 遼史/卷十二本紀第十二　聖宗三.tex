\article{卷十二本紀第十二 聖宗三}

\begin{pinyinscope}

 五年春正月乙丑,破束城縣,縱兵大掠。丁卯,次文安,遣人諭降,不聽,遂擊破之。盡殺其不壯,俘其老幼。戊寅,上還南京。己卯,御元和殿,大齎將士。壬辰,如華林、天柱。



 二月取午朔,至自天柱。



 三月癸亥朔,幸長春宮,賞花釣魚,以牡丹遍賜近臣,觀宴累日,丁丑,以蹄居部下拽刺解裏偵候有功,命入御盞郎君班只候。



 夏四月癸巳朔,幸
 南京。丁酉,上率百僚冊上皇太后尊號曰睿德神略應運啟化承天皇太后;禮畢,群臣上皇帝尊號曰至德廣孝昭聖天輔皇帝。戊戌,詔有司條上勛舊,等第加恩。癸丑,清暑冰井。



 六月壬辰朔,召大臣決庶政。丙申,以耶律蘇為遙郡刺史。



 秋七月戊辰,涅刺部節度使撒葛里有惠政,民請留,從之。



 是月,獵平地松林。



 九月丙戌,幸南京。是冬止焉。



 六年春正月庚申,如華林、天柱。



 二月丁未,奚王籌寧殺無罪人李浩,所司議貴,請貸其罪。令出錢贍浩家,從之。甲寅,大同軍節度使、同平章政事劉京致仕。三月己未,休哥
 奏宋事宜,上親覽之。丙寅,以司天趙宗德、齊泰、王守平、邵祺、閻梅從征四載,言天象數有征,賜物有差。癸未,李繼遷遣使來貢。



 夏四月乙未,幸南京。丁酉,胡里室橫突韓德讓墮馬,皇太后怒,殺之。戊戌,幸宋國王休哥第。



 五月癸亥,南府宰相耶律沙薨。



 閏月丙戌朔,奉聖州言太祖所建金鈴閣壞,乞加修繕。詔以南征,恐重勞百姓,待軍還治之。壬寅,阿薩蘭回鶻來貢。



 甲寅,烏隈於厥部以歲貢貌鼠、青鼠皮非土產,皆於他處貿易以獻,乞改貢。詔自今上進牛馬。



 六月癸亥,黨項太保阿刺恍來朝,貢方物。乙丑,諭諸道兵馬備南征攻城器具。乙酉,夷
 離堇阿魯勃送沙州節度使曹恭順還,授於越。



 秋七月丙戌,觀市。己亥,遣南面招討使韓德威討河、涅諸蕃違命者。賜休哥、排亞部諸軍戰馬。己酉,駐蹕於洛河。



 壬子,加韓德威開府儀同三司兼政事令、門下平章事,東京留守兼侍中、漆水郡王耶律抹只為大同軍節度使。癸丑,排亞請增置涿州驛傳。



 八月丙辰,以青牛自馬祭天地。戊午,休哥與排亞、裊里易捉生,將至易州,遇宋兵,殺其指揮使而還。庚申,幸黎園溫湯。癸亥,以將伐宋,遣使祭木葉山。丁丑,瀕海女直遣使速魯裡來朝。西北路管押詳穩速撒哥以伐折立、助里二部,上所俘獲。東路林
 牙蕭勤德及統軍石老以擊敗文直兵,獻俘。大同軍節度使耶律抹只奏今歲霜旱乏食,乞僧價折粟,以利貧民。



 詔從之。濱誨女直遣廝魯裡來修土貢。九月丙申,化哥與木不姑春古裡來貢。休哥遣詳穩意德里獻所獲宋諜者。丁酉,皇太后幸韓德讓帳,厚加賞齎,命眾臣分朋雙陸以盡歡。戊戌,幸南京。己亥,有事於太宗皇帝廟。



 以唐元德為奉陵軍節度使。癸卯,祭旗鼓南伐。庚戌,次涿州,射帛書諭城中降,不聽。



 冬十月乙卯,縱兵四面攻之,地破乃降,因撫諭其眾。駙馬蕭勤德、太師閱覽皆中流矢。勤德載帝車中以歸。聞宋軍退,遣斜軫、排亞等追擊,大敗
 之。戊午,攻沙堆驛,破之。己巳,以黑白羊祭天地。庚午,以宋降軍分置七指揮,號歸聖軍。壬申,行軍參謀、宣政殿學士馬得臣言諭降宋軍,恐終不為用,請並放還。詔不允。丙子,籌寧奏破狼山捷。辛巳,復奏敗宋兵於益津關。癸未,進軍長城口,宋定州守將李興以兵來拒,休哥擊敗之,追奔五六里。



 十一月甲申朔,上以將攻長城口,詔諸軍備攻具。庚寅,駐長城口,督大軍四面進攻。士潰圍,委城遁,斜軫招之。不降;上與韓德讓邀擊之,殺獲殆盡,獲者分隸燕軍。辛卯,攻滿城,圍之。甲午,拔其城,軍士開北門遁,上使諭其將領,乃率眾降。戊戌,攻下祁州,
 縱兵大掠。己亥,拔新樂。庚子,破小狼山砦,丁未,宋軍千人出益津關。國舅郎君桃委、詳穩十哥擊走之,殺副將一人。己酉,休哥獻黃皮室穩徊地莫州所獲馬二十匹,士卒二十人。命賜降者衣帶,便隸燕京。辛亥,西路又送降卒二百餘人,給寒者裘衣。以馬得臣權宣徽院事。



 十二月甲寅朔,賜皮室詳穩乞得、禿骨裏戰馬。橫帳郎君達打里劫掠,命杖之。丙辰,敗於沙河。休哥獻奚詳穩耶魯所獲宋諜。丁巳,遣北宰相蕭繼遠等往規安平。侍衛馬軍司奏攻祁州、新樂,都頭劉贊等三十人有功,乞加恩賞。是月,大軍駐宋境。是歲,詔開貢舉,放高舉一人及第。



 七年春正月癸未朔,班師。戊子,宋雞壁砦守將郭榮率眾來降,詔屯南京。庚寅,次長城口。三卒出管劫掠,笞以徇眾,以所獲物分賜左右。壬辰,李繼遷與兄繼捧有怨,乞與通好,上知其非誠,不許。癸巳,諭諸軍趣易州。己亥,禁部從代民桑樣。癸卯,攻易州,宋兵出遂城來援,遣鐵林軍擊之,擒其指揮使王人。甲辰,大軍齊進,破易州,降刺史劉墀,守陴士卒南遁,上帥師邀之,無敢出者。即以馬質為刺史,趙質為兵馬都監,遷易州軍民於燕京。以東京騎將夏貞顯之子仙壽先登,授高州刺史。乙巳,幸易州,御五花樓,撫諭士庶。丙午,以青牛白民祭天地。詔
 諭三京諸道。戊申,次淶水,謁景宗皇帝廟。詔遣涿州刺史耶律守雄護送易州降人八百,還隸本貫。己酉,次歧溝,射鬼箭。辛亥,還次南京,六軍解嚴。



 二月壬子朔,上御元和殿受百官賀。詔雞壁砦民二百戶徙居檀、順、薊三州。甲寅,回鶻、于闐、師子等國來貢。乙卯,大饗軍士,爵賞有差。樞密使韓德讓封楚國王,駙馬都尉蕭寧遠同政事門下平章事。是日,幸長春宮。甲子,詔南征所俘有親屬分隸諸帳者,給官錢贖之,使相從。乙丑,賞南征女直軍,使東還。丙寅,禁舉人匿名飛書,謗訕朝廷。癸酉,吐番、黨項來貢。甲戌,雲州租賦請止輸水道,從之。丙子,以女
 直活骨德為本部相。分遣巫覡祭名山大川。丁丑,皇子佛寶奴生。



 戊寅,阿薩蘭、于闐、轄烈並遣使來貢。



 三月壬午朔,遣使祭木葉山。禁白牧傷禾稼。宋進士十七人挈家來歸,命有司考其中第者,補國學官。余授縣主簿、尉。



 李繼遷遣使來貢。丁亥,詔知易州趙質收戰亡士卒骸骨,築京觀。戊子,賜於越宋國王紅珠筋線,命入內神帳行再生禮,皇太后賜物甚厚。以雞壁砦民成廷朗等八戶隸飛狐。己丑,詔免雲州通賦。乙室王貫寧擊鞠,為所部郎君高四縱馬突死,詔訊高四罪。丙申,詔開奇峰路通易州市。戊戌,以王子帳耶律襄之女封義成公主,下
 嫁李繼遷。



 是春,駐蹕延芳澱。



 夏四月甲寅,還京。乙卯,國舅太師蕭閱覽為子排亞請尚皇女延壽公主,許之。丙辰,謁太宗皇帝廟。以御史大夫烏骨領乙室大王。己未,幸延壽寺飯僧。甲子,諫議大夫馬得臣以上好擊球,上疏切諫:「臣伏見陛下聽朝之暇,以擊球為樂。



 臣思此事有三不宜:上下分朋,君臣爭勝,君得臣奪,君輸臣喜,一不宜也;往來交錯,前後遮約,爭心競起,禮容全廢,若貪月杖,誤拂天衣,臣既失儀,君又難責,二不宜也;輕萬乘之貴,逐廣場之娛,地雖平,至為堅確,馬雖良,亦有驚蹶,或因奔擊,失其控御,聖體寧無虧損?太后豈不驚懼?
 三不宜也。臣望陛下念繼承之重,止危險之戲。」疏奏,大嘉納之。



 丁卯,吐渾還金、回鶻安進、吐蕃獨朵等自未來歸,皆賜衣帶。



 皇太后謁奇首可汗廟。丙子,以舍利軍耶律杳為常袞。己卯,駐蹕儒州龍泉。



 五月庚辰朔,遣宣徽使蒲領等率兵分道備宋。以遙輦副使控骨離為舍利拽刺詳穩。辛巳,祭風伯於儒州白馬村。休哥引軍至滿城,招降卒七百餘人,遣使來獻。詔隸東京。辛卯,獵桑乾河。壬辰,燕京奏宋兵至邊,時暑未敢與戰,且駐易州,俟彼動則進擊、退則班師。從之。



 六月庚戌朔,以太師柘母迎合,撾之二十。辛酉,詔燕樂、密雲二縣荒地許民耕種,免
 賦役十年。甲戌,宣政殿學士馬得臣卒,詔贈太子少保,賜錢十萬,粟百石。乙亥,詔出諸畜賜邊部貧民。是月,休哥、排亞破宋兵於泰州。



 秋七月乙酉,御含涼殿視朝。丙戌,以中丞耶律核麥哥權夷離畢,橫帳郎君耶律延壽為御史大夫。癸巳,遣兵南征。甲午,以迪離畢、涅刺、烏濊三部各四人益東北路夫人婆里德,仍給印綬。丁酉,勞南征將士。是日,帝與太后謁景宗皇帝廟。



 八月庚午,放進士高正等二人及第。



 冬十月,禁置網捕兔。



 十一月甲申,於闐張文賓進內丹書。



 十二月甲寅,鉤魚於沈子添。癸亥,獵於好草嶺。



\end{pinyinscope}