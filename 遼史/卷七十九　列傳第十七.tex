\article{卷七十九 列傳第十七}

\begin{pinyinscope}

 蕭孝穆子撒八孝先孝友蕭蒲奴耶律蒲古夏行美蕭孝穆,小字胡獨堇,游欽皇后弟阿古只五世孫。父陶瑰,為國舅詳穩。



 孝穆廉謹有禮法。統和二十八年,累遷西北路招討都監。



 開泰元年,遙授建雄軍節度使?加檢校太保。是年術烈等變,孝穆擊走之。冬,進軍可敦城。阻卜結五群牧長查刺、阿睹等,謀中外相應,孝穆悉誅之,
 乃嚴備御以待,餘黨遂潰。以功遷九水諸部安撫使。尋拜北府宰相,賜忠穆熙霸功臣,檢校太師,同政事門下平章事。八年,還京師。



 太平二年,知樞密院事,充漢人行宮都部署。三年,封燕王、南京留守、兵馬都總管。九年,大延琳以東京叛,孝穆為都統討之,戰於蒲水。中軍稍卻,副部署蕭匹敵、都監蕭蒲奴以兩冀夾擊,賊潰,追敗之於手山北。延琳走入城,深溝自衛。孝穆圍之,築重城,起樓櫓,使內外不相通,城中撤屋以爨。



 其將楊詳世等擒延琳以降,遼東悉平。改東京留守,賜佐國功臣。為政務寬簡,撫納流徙,其民安之。



 興宗即位,徙王秦,尋復為南
 京留守。重熙六年,進封吳國王,拜北院樞密使。八年,表請籍天下戶口以均徭役,又陳諸部及舍利軍利害。從之。由是政賦稍平,眾悅。九年,徙王楚。時天下無事,戶口蕃息,上富於春秋,每言及周取十縣,慨然有南伐之志。群臣多順旨。孝穆諫曰:「昔太祖南伐,終以無功。嗣聖皇帝僕唐立晉,後以重貴叛,長驅入汴;鑾馭始旋,反來侵軼。自後連兵二十餘年,僅福和好,蒸民樂業,南北相通。今國家比之曩日,雖曰富強,然勛臣、宿將往往物故。



 且宋人無罪,陛下不宜棄先帝盟約。」時上意己決,書奏不報。



 以年老乞骸骨,不許。十二年,復為北院樞密使,更正
 齊,薨。



 追贈大丞相、晉國王,謚曰貞。



 孝穆雖椒房親,位高益畏。太后有賜,輒辭不受。妻子無驕色。與人交,始終如一。所薦拔皆忠直士。嘗語人曰:「樞密選賢而用,何事不濟?若自親煩碎,則大事凝滯矣。」自蕭合卓以吏才進,其後轉效,不知大體。嘆曰:「不能移風易俗,偷安爵位,臣子之道若是乎。」時稱為「國寶臣」,目所著文曰《寶老集》。二子阿刺、撒八,弟孝先、孝忠、孝友,各有《傳》。



 撒八,字周隱。七歲,以戚屬加左右千牛衛大將軍。重熙初,補祗候郎君;性廉介,風姿爽朗,善球馬、馳射。帝每燕飲,喜諧謔。



 撒八雖承寵顧,常以禮自持,時人稱之。以柴
 冊禮恩,加檢校太傅、永興宮使,總領左右護衛,同知點檢司事。尚魏國公主,拜駙馬都尉,為北院宣徽使,仍總知朝廷禮儀。重熙末,出為西北路招討使、武寧郡王。居官以治稱。



 清寧初薨,年三十九,追封齊王。



 孝先,字延寧,小字海裏。統和十八年,補祗候郎君。尚南陽公主,拜駙馬都尉。



 開泰五年,為國舅詳穩。將兵城東鄙。還,為南京統軍使。



 太平三年,為漢人行宮都部署,尋加太子太傅。五年,遷上京留守。以母老求侍,復為國舅詳穩。改東京留守。會大延琳反,被圍數月,穴地而出。延琳平,留守上京。十一年,帝不豫,欽哀召孝先總禁衛事。



 興宗諒陰,欽哀弒仁德皇后,孝先與蕭浞卜、蕭匹敵等謀居多。及欽哀攝政,遙授天平軍節度使,加守司徒,兼政事令。



 重熙初,封楚王,為北院樞密使。孝先以椒房親,為太后所重。



 在樞府,好惡自恣,權傾人主,朝多側目。三年,太后與孝先謀廢立事,帝知之,勒衛兵出宮,召孝先至,諭以廢太后意。



 孝先震懾不能對。遷太后於慶州。孝先恆鬱鬱不樂。四年,徙王晉。後為南京留守,卒,謚忠肅。



 孝友,字撻不衍,小字陳留。開泰初,以戚屬為小將軍。



 太平元年,以大冊,加左武衛大將軍、檢校太保,賜名孝友。



 重熙元年,累遷西北路招討使,封蘭陵郡王。八年,進王
 陳。先是,蕭惠為招討使,專以威制西羌,諸夷多叛。孝友下車,厚加綏撫,每入貢,輒增其賜物,羌人以安。久之,浸成姑息,諸夷桀驁之風遂熾,議者譏其過中。



 十年,加政事令,賜效節宣庸定遠功臣,更正吳。後以葬兄孝穆、孝忠,還京師,拜南院樞密使,加賜翊聖協穆保義功臣,進王趙,拜中書令。丁母憂,起復北府宰相,出知東京留守。會伐夏,孝發與樞密使蕭惠失利河南,帝欲誅之,太后救免。復為東京留守,徙王燕,改上京留守,更王秦。清寧初,加尚父。頃之,復留守東京。明年,復為北府宰相。帝親制誥詞以褒寵之。以柴冊恩,遙授洛京留守,益賜純德
 功臣,致仕,進封豐國王。



 坐子胡睹首與重元亂,伏誅,年七十三。胡睹在《逆臣傳》。



 蕭蒲奴,字留隱,奚王楚不寧之後,幼孤貧,傭於醫家牧牛。傷人稼,數遭笞辱。醫者嘗見蒲奴熟寐,有蛇繞身,異之。



 教以讀書,聰敏嗜學。不數年,涉獵經史,習騎射。既冠,意氣豪邁。



 開泰間,選充護衛,稍進用。俄坐罪黥流烏古部。久之,召還,累任劇,遷奚六部大王,治有聲。



 太平九年,大延琳據東京叛,蒲奴為都監,將右翼軍,遇戰蒲水。中軍少卻。蒲奴與左翼軍夾攻之。先據高麗、女直要沖,使不得求援,叉敗賊於手山。延琳走入城。蒲奴不介馬
 而馳,追殺餘賊。已而大軍圍東京,蒲奴討諸叛邑,平吼山賊,延琳堅守不敢出。既被擒,蒲奴以功加兼侍中。



 重熙六年,改北阻卜副部署,再授奚六部大王。十五年,為西南面招討使,西征夏國。蒲奴以兵二千據河橋,聚巨艦數十艘,仍作大鉤,人莫測。戰之日,布舟於河,綿互三十餘里。



 遣人伺上流,有浮物輒取之。大軍既失利,蒲奴未知,適有人木順流而下,勢將壞浮梁,斷歸路,操舟者爭的致之,橋得不壞。



 明年,復西征,懸兵深入,大掠而還,復為奚六部大王。



 致仕,卒。



 耶律蒲古,字提隱,太祖弟蘇之四世孫。以武勇稱。統和
 初,為涿州刺史,從伐高麗有功。開泰末,為上京內容省副使。



 太平二年,城鴨綠江,蒲古守之,在鎮有治績。五年,改廣德軍節度使,尋遷東京統軍使。蒞政遷肅,諸部懾服。九年,大延琳叛,以書結保州。夏行美執其人送蒲古,蒲古入據保州,延琳氣沮。以功拜惕隱。



 十一年,為子鐵驪所弒。



 夏行美,渤海人。太平九年,大延琳叛,時行美總渤海軍於保州。延琳使人說欲與俱叛,行美執送統軍耶律蒲古,又誘賊黨百人殺之。延琳謀沮,乃嬰城自守,數月而破。以功加同政事門下平章事,錫齎甚厚。明年,擢忠順
 軍節度使。



 重熙十七年,遷副部署,從點檢耶律義先討蒲奴裡,獲其酋陶得里以歸。致仕,卒。上思其功,遣使祭於家。



 論曰:「不有君子,其能國乎?方其擒延琳,定遼東,一時諸將之功偉矣。宜其撫劍抵掌,賈餘勇以威天下也。蕭孝穆之諫南侵,其意防何其弘遠歟,是豈瞋目語難者所能知哉!至論移風俗為治之本,親煩碎為失大臣體,又何其深切著明也。



 為『國寶臣』,宜矣。孝先預弒仁德之謀,猶依城社以逃熏灌,為國巨蠹,雖功何議焉。」



\end{pinyinscope}