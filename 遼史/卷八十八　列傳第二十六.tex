\article{卷八十八 列傳第二十六}

\begin{pinyinscope}

 耶律仁先子撻不也耶律良蕭韓家奴蕭德蕭惟信蕭樂音奴耶律敵烈姚景行耶律阿思耶律仁先,字乣鄰,小宇查剌,孟父房之後。父瑰引,南府宰相,封燕王。



 仁先魁偉爽秀,有智略。重熙三年,補護衛。帝與論政,才之。仁先以不世遇,言無所隱。授宿直將軍,累遷殿前副點檢,改鶴剌唐古部節度使,俄召為北面
 林牙。



 十一年,升北院樞密副使。時宋請增歲幣銀絹以償十縣地產,仁先與劉符使宋,仍讀書「貢」。宋難之。仁先曰:「曩者石晉報德本朝,割地以獻,周人攘而取之,是非利害,灼然可見。」宋無辭以對。乃定議增銀、絹十萬兩、匹,仍稱「貢」。



 既還,同知南京留守事。



 十三年,代夏,留仁先鎮邊。未幾,召為契丹行宮都部署,奏復王子班郎君及諸宮雜役。十六年,遷北院大王,奏今兩院戶口殷庶,乞免他部助役,從之。十八年,再舉伐夏,仁先與皇太弟重元為前鋒。蕭惠失利於河南,帝猶欲進兵,仁先力諫,乃止。後知北院樞密使,遷東京留守。女直恃險,侵掠不止,
 仁先乞開山通道以控制之,邊民安業。封吳王。



 清寧初,為南院樞密使。以耶律化哥譖,出為南京兵馬副元帥,守太尉,更王隋。六年,復為北院大王,民歡迎數百里,如見父兄。時此、商院樞密官涅魯古、蕭胡睹等忌之,請以仁先為西北路招討使。耶律乙辛奏曰:「仁先舊臣,德冠一時,不宜補外。」復拜南院樞密使,更王許。



 九年七月,上獵太子山,耶律良奏重元謀逆,帝召仁先語之。仁先曰:「此曹兇狠,臣固疑之久矣。」帝趣仁先捕之。



 仁先出,且曰:「陛下宜謹為之備!」未及介馬,重元犯帷宮。



 帝欲幸北、南院,仁先曰:「陛下若舍扈從而行,賊必躡其後;且南、北大王心
 未可知。」仁先子撻不也曰:「聖意豈可違乎?」



 仁先怒,擊其首。帝悟,悉委仁先以討賊事。乃環車為營,拆行馬,作兵仗,率官屬近侍三十餘騎陣柢枑外。及交戰,賊眾多降。涅魯古中矢墮馬,擒之,重元被傷而退。仁先以五院部蕭塔剌所居最近,亟召之,分遣人集諸軍。黎明,重元率奚人二千犯行宮,蕭塔剌兵適至。仁先料賊不能久,俟其氣沮攻之。乃背營而陣,乘便奮擊,賊眾奔潰,追殺二十餘里,重元與數騎遁去。帝執仁先手曰:「平亂皆卿之功也。」加尚父,進封宋王,為北院樞密使,親制文以褒之,詔畫《濼河戰圖》以旌其功。咸雍元年,加於越,改封遼
 王,與耶律乙辛共知北院樞密事。乙辛恃寵不法,仁先抑之,由是見忌,出為南京留守,改王晉。恤孤煢,禁奸慝,宋聞風震服。議者以為自於越休哥之後,惟仁先一人而已。阻卜塔裡幹叛命,仁先為西北路招討使,賜鷹紐印及劍。



 上諭曰:「卿去朝廷遠,每俟奏行,恐失機會,可便宜從事。」



 仁先嚴斥候,扼敵沖,懷柔服從,庶事整飭。塔裡乾復來寇,仁先逆擊,追殺八十餘里。大軍繼至,又敗之。別部把裡斯、禿沒等來救,見其屢挫,不敢戰而降。北邊遂安。



 八年卒,年六十,遺命家人薄葬。弟義先、信先,俱有傳。



 子撻不也。



 撻不也,字胡獨堇。清寧二年,補祗候郎君,累遷永興宮使。以平重元之亂,避授正軍節度使,賜定亂功臣,同知殿前點檢司事。歷高陽、臨海二軍節度使、左皮室詳穩。



 大康六年,授西北路招討使,率諸部尊長入朝,加兼侍中。



 自蕭敵祿為招討之後,朝廷務姑息,多擇柔願者用之,諸部漸至跋扈。撻不也含容尤甚,邊防益廢,尋改西南面招討使。



 阻卜酋長磨古斯來侵,西北路招討使何魯掃古戰不利,招撻不也代之。磨古斯之為酋長,由撻不也所薦,至是遣人誘致之。磨古斯紿降,撻不也逆於鎮州西南沙磧間,禁士卒無得妄動。敵至,裨將耶律
 綰斯、徐烈見其勢銳,不及戰而走,遂被害,年五十八。贈兼侍中,謚曰貞憫。



 撻不也少謹願,後為族嫠婦所惑,出其妻,終以無子。人以此譏之。



 耶律良,字習捻,小字蘇,著帳郎君之後。生於乾州,讀書醫巫閭山。學既博,將入南山肄業,友人止之曰:「爾無僕御,驅馳千里,縱聞見過人,年亦垂暮。今若即仕,已有餘地。」



 良曰:「窮通,命也,非爾所知。」不聽,留數年而歸。



 重熙中,補寢殿小底,尋為燕趙國王近侍。以有貧,詔乘廄馬。遷修起居注。會獵秋山,良進《秋游賦》,上嘉之。



 清寧中,上幸鴨子河,作《捕魚賦》。由是寵遇稍隆,遷知制誥,兼知部署
 司事。奏請編禦制詩文,目曰《清寧集》;上命良詩為《慶會集》,親制其序。頃之,為敦睦宮使,兼權知皇太后宮諸局事。



 良聞重元與子涅魯古謀亂,以帝篤於親愛,不敢遽奏,密言於皇太后。太后托疾,召帝白其事。帝謂良曰:「汝欲間我骨肉耶?」良奏曰:「臣若妄言,甘伏斧鑕。陛下不早備,恐墮賊計。如召涅魯古不來,可卜其事。」帝從其言。使者及門,涅魯古意欲害之,羈於帳下。使者以佩刀斷帟而出,馳至行宮以狀聞。帝始信。亂平,以功遷漢人行宮都部署。



 咸雍初,同知南院樞密使事,為惕隱,出知中京留守事。



 未幾卒,帝嗟悼,遣重臣賻祭,給葬具,追封遼西
 郡王,謚曰忠成。蕭韓家奴,字括寧,奚長渤魯恩之後。性孝友。太平中,補祗候郎君,累遷敦睦宮使。伐夏,為左翼都監,遷北面林牙。



 俄為南院副部署,賜玉帶,改奚六部大王。治有聲。



 清寧初,封韓國公,歷南京統軍使、北院宣徽使,封蘭陵郡王。九年,上獵太子山,聞重元亂,馳詣行在。帝倉卒欲避於北、南大王院,與耶律仁先執轡固諫,乃止。明旦,重元復誘奚獵夫來。韓家奴獨出諭之曰:「汝曹去順效逆,徒取族滅。



 何若悔過,轉禍為福!」。獵夫投仗首服。以功殿前都點檢,封荊王,賜資忠保義奉國竭貞平亂功臣。



 咸
 雍二年,遷西南面招討使。大康初,徙王吳,賜白海東青鶻。皇太子為乙辛誣構,幽於上京。韓家奴上書力言其冤,不報。四年,復為西南面招討使。例削一字王爵,改王蘭陵,薨。子楊九,終右祗候郎君班詳穩,贈同中書門下平章事。



 蕭德,字特末隱,楮特部人。性和易,篤學好禮法。太平中,領牌印、直宿,累遷北院樞密副使,敷奏詳明,多稱上旨。詔與林牙耶律庶成修《律令》,改契丹行宮都部署,賜宮戶十有五。清寧元年,遷同知北院樞密使,封魯國公。上以德為先朝眷遇,拜南府宰相。五年,轉南京統軍使。九
 年,復為南府宰相。重元之亂,推鋒力戰,斬涅魯古首以獻,論功封漢王。



 咸雍初,以告老歸,優詔不許。久之,加尚父,致仕。卒,年七十二。



 蕭惟信,字耶寧,楮特部人。五世祖霞賴,南府宰相。曾祖烏古,中書令。祖阿古只,知平州。



 父高八,多智數,博覽古今。開泰初,為北院承旨,稍遷右夷離畢,以幹敏稱,拜南府宰相。累遷倒塌嶺節度使,知興中府,復為右夷離畢。陵青誘眾作亂,事覺,高八按之,止誅首惡,餘並釋之。歸奏,稱旨。



 惟信資沉毅,篤志於學,能辨論。重熙初始仕,累遷左中丞。十五年,徙燕趙國王傅,帝諭之曰:「燕趙左右
 多面識,不聞忠言,浸以成性。汝當以道規誨,使知君父之義。有不可處王邸者,以名聞。」惟信輔導以禮。十七年,遷北院樞密副使,坐事免官。尋復職,兼北面林牙。



 清寧九年,重元作亂,犯灤河行宮,惟信從耶律仁先破之,賜竭忠定亂功臣。歷南京留守、左右夷離畢,復為北院樞密副使。大康中,以老乞骸骨,不聽。樞密使耶律乙辛譖廢太子,中外知其冤,無敢言者,惟信數廷爭,不得復。告老,加守司徒,卒。



 蕭樂音奴,字婆丹,奚六部敞穩突呂不六世孫。



 父拔剌,三歲居父母喪,毀瘠過甚,養於家奴奚列阿不。



 重熙初,
 興宗獵奚山,過拔剌所居,奚列阿不言於近臣,拔剌得見上。年甫十歲,氣象如成人。帝悅之,錫賚甚厚。既長,有遠志,不樂仕進,隱於奚王嶺之插合谷。上以其名家,又有時譽,就拜舍利軍詳穩。



 樂音奴貌偉言辨,通遼、漢文字,善騎射擊鞠,所交皆一時名士。年四十,始為護衛。平重元之亂,以功遷護衛太保,改本部南克,俄為旗鼓拽剌詳隱。監障海東青鶻,獲白花者十三,賜榾柮犀並玉吐鶻。拜五蕃部節度使,卒。子陽阿,有傳。



 耶律敵烈,字撒懶,採訪使吼五世孫。寬厚,好學,工文詞。重熙末,補牌印郎君,兼知起居注。



 清寧元年,稍遷同知
 永州事,禁盜有功,改北面林牙承旨。



 九年,重元作亂。敵烈赴援,力戰平之,遙授臨海軍節度使。



 十年,徙武安州觀察使。咸雍元年,累遷長寧宮使。撿括戶部司乾州錢帛逋負,立出納經畫法,公私便之。大康四年,為南院大王。秩滿,部民請留,同知南京留守事。有疾,上命乘傳赴闕,遣太醫視之。遷上京留守。



 大安中,改塌母城節度使。以疾致仕,加兼侍中,賜一品俸。八年卒。



 姚景行,始名景禧。祖漢英,本周將,應歷初來聘,用敵國禮,帝怒,留之,隸漢人宮分。及景行既貿,始出籍,貫興中縣。景行博學。重熙五年,擢進士乙科,為將作監,改燕趙
 國王教授。不數年,至翰林學士,樞密副使,參知政事。性敦厚廉直,人望歸之。



 道宗即位,多被顧問,為北府宰相。九年秋,告歸,道聞重元亂,收集行旅得三百餘騎勤王。比至,賊已平。帝嘉其忠,賜以逆人財產。咸雍元年,出為武定軍節度使。明年,驛召拜南院樞密使。上從容問治道,引入內殿,出御書及太子書示之,賜什器車仗。帝有意伐宋,召景竹問曰:「宋人好生邊事,如何?」對曰:「自聖宗皇帝以威德懷遠,宋修職貢,迨今幾六十年。若以細故用兵,恐違先帝成約。」上然其言而止。



 致仕,不逾月復舊職。丁家艱,起復,兼中書令。上間古今儒士優劣,占對稱
 旨,知興中府,改朔方軍節度使。大康初,徙鎮遼興。以上京多滯獄,命為留守,不數月,以獄空聞。



 累乞致政,不從。復請,許之,加守太師。卒,遣使吊祭,追封柳城郡王,謚文憲。壽隆五年,詔為立祠。



 耶律阿思,字撒班。清寧初,補祗候郎君。以善射,掌獵事,進渤海近侍詳穩。



 重元之亂,與護衛蘇射殺涅魯古,賜號靖亂功臣,徙契丹行宮都部署。大安初,為北院大王,封漆水郡王。壽隆元年,為北院樞密使,監修國史。



 道宗崩,受顧命,加於越。錄乙辛黨人,罪重者當籍其家,阿思受賂,多所寬貰。蕭合魯嘗言當修邊備,阿思力沮其事,或
 譏其以金賣國。



 後以風疾失音,致仕,加尚父,封趙王。薨,年八十,追封齊國王。



 論曰:「灤河之變,重元擁兵行幄,微仁先等,道宗其危乎!當其止幸北、南院,召塔剌兵以靖大難,功宜居首。良以反謀白太后,韓家奴以逆順降奚人,德與阿思殺涅魯古,皆有討賊之力焉。仁先齊名休哥,動德兼備,此其一節歟。」



\end{pinyinscope}