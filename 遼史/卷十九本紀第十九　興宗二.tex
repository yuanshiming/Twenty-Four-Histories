\article{卷十九本紀第十九 興宗二}

\begin{pinyinscope}

 十年春正月辛亥朔,宋遣梁適、張從一、富弼、趙日宣來賀。甲子,處處遣吳育、馮戴來賀永壽節。



 二月庚辰朔,詔蒲盧毛朵部歸曷蘇館戶之沒入者使復業。



 甲申,北樞密院言,南、北二王府及諸部節度侍衛祗候郎君,皆出族帳,既免與民戍邊,其祗候事,請亦得以部曲代行。詔從其請。



 夏四月,詔罷修鴨綠江浮梁及漢兵屯戍之役。又
 以東京留守蕭撒大言,馳東京擊鞠之禁。



 六月戊寅朔,以蕭寧、耶律坦、崔禹稱、馬世良、耶律仁先、劉六符充賀宋生辰使副;耶律庶成、趙成、耶律烈、張旦充來歲賀宋正旦使副。



 秋七月壬戌,詔諸職官以官物者,以正盜論。諸敢以先朝已斷事相告言者,罪之。諸帳郎君等於禁地射鹿,決三百,不征償、小將軍決二百以下,及百姓犯者,罪同郎君論。



 八月丙戌,以醫者鄧延貞治詳穩蕭留寧疾驗,贈其父母官以獎之。



 九月辛亥,朝皇太后。國舅留寧薨。庚申,持太後射獲熊,上進酒為壽。癸亥,上獵馬孟山,草木蒙密,恐獵者誤射傷人,命耶律迪姑各書姓
 名於矢以志之。丙寅,夏國獻宋俘。以石硬砦太保郭三避虎不射,免官。



 冬十月丙戌,詔東京留守蕭孝忠察官吏有廉幹清強者,具以名聞。庚寅,以女直太師臺押為曷蘇館都大王。辛卯,以皇子胡盧斡裏生,北宰相、駙馬撒八寧迎上至其第宴飲,上命衛十與漢人角抵為樂。壬辰,復飲皇太后殿,以皇子生,肆赦。



 夕,復引公主、駙馬及內族大臣入寢殿劇飲。甲午,幸中京。



 庚子,以駙馬都尉蕭忽列為國舅詳穩。



 十一月丙辰,回鶻遣使來貢。



 十二月丙子朔,宋遣劉沆、王整來賀應聖節。乙未,置撻木不姑酋長。以胡撻刺為平章事。上聞宋設關河,治壕塹,
 恐為邊患;與南、北樞密昊國王蕭孝穆、趙國王蕭貫寧謀取宋舊割關南十縣地,遂遣蕭英、劉六符使宋。庚寅,宋遣張沔、侯宗亮、薛申、待其浚、施昌言、潘永照來賀永壽節及來歲正旦。



 以宣政殿學士楊佶為吏部尚書、判順義軍節度使事。丁酉,議伐宋,詔諭諸道。



 十一年春正月戊申,奉迎皇太后於內殿。庚戌,遣南院宣微使蕭特末、翰林學士劉六符使宋,取晉陽及瓦橋以南十縣地;且間興師伐夏及沿邊疏浚水澤,增益兵戍之故。



 二月壬寅,如鴛鴦濼。



 夏四月甲戌朔,頒南征賞罰令。



 六月乙亥,宋遣富弼、張茂實奉書來聘,以書答之。壬
 午,御含涼殿,放進士王實等六十四人。禁氈、銀鬻入宋。



 秋七月壬寅朔,詔盜易官馬者滅死論。外路官勤瘁正直者,考滿代;不治事者即易之。



 八月丙申,宋復遣富弼、張茂實奉書來聘,乞增歲幣銀絹,以書答之。九月壬寅,遣北院樞密副使耶律仁先、漢人行宮副部署劉六符使宋約和。是時,富弼為上言,大意謂遼與宋和,坐獲歲幣,則利在國家,臣下無與;與宋交兵,則利在臣下,害在國家。上感其言,和好始定。



 閏月癸未,耶律仁先遣大報,宋歲增銀、絹十萬兩、匹,文書稱「貢」,送至白溝;帝喜,宴群臣於昭慶殿。是日,振恤三父族之貧者。辛卯,仁先、劉六
 符還,進宋國誓書。



 冬十一月丁亥,群臣加上尊號日聰文聖武英略神功睿哲仁孝皇帝,冊皇后蕭氏曰貞懿宣慈崇聖皇后。大赦。梁王洪基進封燕國王。



 十二月癸卯,朝皇太后。甲辰,封皇太弟重元子涅魯古為安定郡王。己酉,以宣獻皇后忌日,上與皇太后素服,飯僧於延壽、憫忠、三學三寺。辛亥,詔蠲預備伐宋諸部租稅一年。



 王子,以吐渾、黨項多鬻馬夏國,詔謹邊防。己未,宋遣賀正旦及永壽節使居邸,帝微服往觀。丁卯,禁喪葬殺牛馬及藏珍寶。



 十二年春正月辛未,遣同知析津府事耶律敵烈、樞密
 院都承旨王惟吉諭夏國與宋和。壬申,以昊國王蕭孝穆為南院樞密使,北府宰相蕭孝忠北院樞密使,封楚王,韓國王蕭惠北府宰相、同知元帥府事,韓八南院大王,耶律侯哂東京留守,北院樞密副使耶律仁先同知東京留守事,北面林牙蕭革北院樞密副使。甲戌,如武清寨葦澱。



 二月壬寅,禁關南漢民弓矢。己酉,復國以加上尊號,遣使來賀。甲寅,耶律敵烈等使夏國還,奏元昊罷兵,即遣使報宋。



 三月辛卯,幸南京。壬辰,高麗國以加上尊號,遣使來賀。



 夏四月己亥,置回跋部詳穩、都監。庚子,夏國遣使進馬、駝。



 五月辛卯,斡魯、蒲盧毛朵部二便
 來貢失期,宥而遣還。



 乙未,詔復定禮制。是月,幸山西。



 六月丙午,詔世選宰相、節度使族屬及身為節度使之家,許葬用銀器;仍禁殺牲以祭。庚戌,詔漢人宮分戶絕,恆產以親族繼之。辛亥,阻卜大王屯禿古斯弟太尉撤葛裡來朝。丙辰,回鶻遣使來貢。甲子,以南院樞密使昊國王蕭孝穆為北院樞密使,徒封齊國王。



 秋七月丙寅朔,北院樞密使蕭孝忠薨,特釋系囚。庚寅,夏國遣使上表,請伐宋,不從。



 八月丙申,謁慶陵。辛丑,燕國王洪基加尚書令,知北南院樞密使事,進封燕趙國王。戊午,以前西北路招討使蕭塔烈葛為右夷離畢。庚申,於越耶律洪
 古薨。甲子,阻卜來貢。



 九月壬申,朝皇太后,謁望仙殿。壬午,謁懷陵。



 冬十月丁酉,駐蹕中會川。己亥,北院樞密使蕭孝穆薨,追贈大丞相、晉國王。庚子,詔諸路上重囚,遣官詳讞。辛亥,參知政事韓紹芳為廣德軍節度使,三司使劉六符長寧軍節度使。壬子,以夏入侵黨項,遣延昌宮使高家奴讓之。甲子,北府宰相蕭惠為北院樞密使,豳王遂哥為惕隱,惕隱敵魯古封漆水郡王、西北路招討使,樞密副使蕭阿刺同知北院宣微事。出飛龍馬廄,分賜群臣。



 十一月丁丑,追封楚王蕭孝忠為楚國王。丁亥,以上京歲儉,復其民租稅。癸巳,朝皇太后。



 十二月戊
 申,改政事省為中書省。



 十三年春正月甲子朔,朝皇太后。戊辰,如混同江。辛未,獵兀魯館岡。



 二月康戌,如魚兒濼。丙辰,以參知政事杜防為南府宰相。三月丁亥,高麗遣使來貢。以宣政殿學士楊佶參知政事。



 是月,置契丹警巡院。



 夏四月己酉,遣東京留守耶律侯哂、知黃龍府事耶律歐里斯將兵攻蒲盧毛朵部。甲寅,南院大王耶律高十奏黨項等部叛附夏國。丙辰,酉南面招討都監羅漢奴、詳穩斡魯等奏,山西部族節度使屈烈以五部叛入西夏,乞南、北府兵援送實威塞州戶。詔富者遣行,餘留屯田天德軍。



 五
 月壬戌朔,羅漢奴奏所發部兵與黨項戰不利,招討使蕭普達、四捷軍詳穩張佛奴歿於陣。李元昊來援叛黨。戊辰,詔徵諸道兵會酉南邊以討元昊。



 六月甲午,阻卜酋長烏八遣其子執元昊所遣求援使窳邑改來,乞以兵助戰,從之。駐蹕永安山。以將伐夏,遣延昌宮使耶律高家奴告宋。丙申,詔前南院大王耶律穀欲、翰林都林牙耶律庶成等編集國朝上世以來事跡。丙午,高麗遣使來貢。丁未,錄囚。



 秋七月辛酉,香河縣民李宜兒以左道惑眾,伏誅。庚午,行再生禮。庚辰,夏國遣使來朝。



 八月乙未,以夏使對不以情,羈之。丁巳,夏國復遣使來,詢以
 事宜,又不以實對,苔之。



 九月戊辰,宋以親征夏國,遣余靖致贐禮。壬申,會大軍於九十九泉,以皇太弟重元、北院樞密使韓國王蕭惠將先鋒兵西征。冬十月庚寅,祭天地。丙申,獲黨項偵人,射鬼箭。丁酉,李元昊上表謝罪。己亥,元昊遣使來奏,欲收叛黨以獻,從之。



 辛亥,元昊遣使來進方物,詔北院樞密副使蕭革迓之。壬子,軍於河曲。革言元昊親率黨項三部來,詔革詰其納叛背盟,元昊伏罪,賜酒,許以自新,遣之。召群臣議,皆以大軍既集,宜加討伐,癸丑,督數路兵掩襲,殺數千人,駙馬都尉蕭胡睹為夏人所執。丁巳,元昊遣使以先被執者來歸,詔
 所留夏使亦還其國。



 十一月辛酉,賜有功將校有差。甲子,班師。丁卯,改雲州為西京。辛巳,朝皇太后。



 十二月己丑,幸西京。戊戌,以北院樞密副使耶律敵烈為右夷離畢。己亥,高麗遣使來貢。戊申,蕭胡睹自夏來歸。



 十四年春正月庚申,以侍中蕭虛烈為南院統軍使,封遼西郡王。庚午,如鴛鴦濼。壬午,以金吾衛大將軍敵魯古為乙室大王。甲申,夏國遣使進鶻。以常侍斡古得戰毆,命其子習羅為師。二月庚子。朝皇太后。駐蹕撒刺濼。



 三月己卯,宋以伐夏師還,遣使來賀。



 四月辛亥,高麗遣使來貢。



 閏五月癸丑,清暑永安山。



 六月丁卯,謁慶陵。己卯,阻卜大
 王屯禿古斯率諸酋長來朝。庚辰,夏國遣使來貢。辛巳,以西南面招討使蕭普達戰毆,贈同中書門下平章事。



 秋七月戊申,駐蹕中會川。



 冬十月甲子,望祀木葉山。



 十一月壬午朔,回鶻阿薩蘭遣使來貢。甲辰,以同知北院宣徽事蕭阿刺為北府宰相。



 十二月癸丑,觀漢軍習炮射擊刺。癸亥,決滯獄。



 十五年春正月乙酉,如混同江。禁契丹以奴嬸鬻與漢人。



 二月乙卯,如長春河。丙寅,蒲盧毛朵界曷懶河戶來附,詔撫之。



 三月甲申,朝皇太后。乙酉,以應聖節,減死罪,釋徒以下。辛卯,朝皇太后。丁酉,高麗遣使來貢。詔諸道
 歲具獄訟以聞。夏四月辛亥朔,禁五京吏民擊鞠。戊午,罷遙輦帳戍軍。



 壬戌,以北女直詳穩蕭高六為莫六部大王。甲子,清暑永安山。



 甲戌,蒲盧毛朵曷懶河百八十戶來附。



 六月癸丑,以西京留守耶律馬六為漢人行宮都部署,參知政事楊佶出為武定軍節度使。戊辰,御清涼殿,放進士王棠等六十八人。甲戌,西北路招討使耶律敵魯古坐贓免官。



 秋七月乙酉,豳王遂哥薨。戊子,觀獲。乙未,以北院宣徽使旅墳為左夷離畢,前南府宰相耶律喜孫東北路詳穩。丙申,籍諸路軍。丁酉,如秋山。辛丑,禁扈從踐民田。丁未,以女直部長遮母率眾來附,加太
 師。



 八月癸丑,高麗王欽薨,遣使來告。



 九月甲辰,禁以罝網捕狐兔。



 冬十月己酉,駐蹕中會川。



 十一月丁亥,以南院樞密使蕭孝友為北府宰相,契丹行宮都部署耶律仁先南院大王,北府宰相蕭革同知北院樞密使事,知夷離畢事耶律信先漢人行宮都部署,左夷離畢旅墳惕隱,漢人行宮都部署耶律敵烈左夷離畢。己亥,渤海部以契丹戶例通括軍馬。乙巳,振南京貧民。



 十二月壬申,曲赦徒以下罪。是日為聖宗在時生辰。



\end{pinyinscope}