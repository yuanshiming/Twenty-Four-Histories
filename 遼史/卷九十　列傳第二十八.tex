\article{卷九十 列傳第二十八}

\begin{pinyinscope}

 蕭兀納耶律儼劉伸耶律胡呂蕭兀納,一名撻不也,字特免,六院部人。其先嘗為西南面拽剌。



 兀納魁偉簡重,善騎射。清寧初,兄圖獨以事入見,帝問族人可用者,圖獨以兀納對,補祗候郎君。遷近侍敞史,護衛太保。大康初,為北院宣徽使。時乙辛已害太子,因言宋魏國王和魯斡之子淳可為儲嗣。群臣莫敢言,唯兀納及夷離畢蕭陶隗諫曰:「舍嫡不立,是以國
 與人也。」帝猶豫不決。五年,帝出獵,乙辛請留皇孫,帝欲從之。兀納奏曰:「竊聞車駕出游,將留皇孫,茍保護非人,恐有他變。果留,臣請侍左右。」帝乃悟,命皇孫從行。由此,始疑乙辛。



 頃之,同知南院樞密使事,出乙辛、淳等。帝嘉其忠,封蘭陵郡王,人謂近於古社稷臣,授殿前都點檢。上謂王師儒、耶律固等曰:「兀納忠純,雖狄仁傑輔唐,屋質立穆宗,無以過也。卿等宜達燕王知之。」自是,令兀納輔導燕王,益見優寵。大安初,詔尚越國公主,兀納固辭。改南院樞密使,奏請掾史宜以歲月遷敘,從之。壽隆元年,拜北府宰相。



 初,天祚在潛邸,兀納數以直言忤旨。及
 嗣位,出為遼興軍節度使,守太傅。以佛殿小底王華誣兀納借內附犀角,詔鞫之。兀納奏曰:「臣在先朝,詔許日取帑錢十萬為私費,臣未嘗妄取一錢,肯借犀角乎!」天祚愈怒,奪太傅官,降寧邊州刺史,尋改臨海軍節度使。



 兀納上書曰:「自蕭海裏亡入女直,彼有輕朝廷心,宜益兵以備不虞。」不報,天慶元年,知黃龍府事,改東北路統軍使,復上書曰:「臣治與女直接境,觀其所為,其志非小。宜先其未發,舉兵圖之。」章數上,皆不聽。及金兵來侵,戰於寧江州,其孫移敵蹇死之,兀納退走入城。留官屬守御,自以三百騎渡混同江而西,城遂陷。後與蕭敵里拒
 金兵於長濼,以軍敗免官。五年,天祚親征,兀納殿,復敗績。後數日乃與百官人見,授上京留守。六年,耶律章奴叛,來攻京城,兀納發府庫以齎士卒,諭以逆順,完城池,以死拒戰。章奴無所得而去。以功授副元帥,尋為契丹都宮使。



 天祚以兀納先朝重臣,有定策勛,每延問以政,兀納對甚切。上雖優容,終不能用。以疾卒,年七十。



 耶律儼,字若思,析津人。本姓李氏。



 父仲禧,重熙中始仕。清寧初,同知南院宣徽使事。四年,城鴨子、混同二水間,拜北院宣徽使。咸雍初,坐誤奏事,出為榆州刺史。俄詔復舊職,遷漢人行宮都部署。六年,賜國姓,封韓國公,改
 南院樞密使。時樞臣乙辛等誣陷皇太子,詔仲禧偕乙辛鞫之,蔓引無辜,未嘗雪正。乙辛薦仲禧可任,拜廣德軍節度使,復為南院樞密使,卒,謚欽惠。



 儼儀觀秀整,好學,有詩名,登咸雍進士第。守著作佐郎,補中書省令史,以勤敏稱。大康初,歷都部署判官、將作少監。



 後兩府奏事,論群臣優劣,唯稱儼才俊。改少府少監,知大理正,賜紫。六年,遷大理少卿,奏讞詳平。明年,升大理卿。



 丁父憂,奪服,同簽部署司事。



 大安初,為景州刺史。繩胥徒,禁豪猾,撫老恤貧,末數月,善政流播,郡人刻石頌德。二年,改御史中丞,詔按上京滯獄,多所平反。同知宣徽院事,提
 點大理寺。六年冬,改山西路都轉運使。刮剔垢弊,奏定課額,益州縣俸給,事皆施行。



 壽隆初,授樞密直學士。以母憂去官,尋召復舊職。宋攻夏,李乾順遣使求和解,帝命儼如宋平之,拜參知政事。六年,駕幸鴛鴦濼,召至內殿,訪以政事。



 帝晚年倦勤,用人不能自擇,令各擲骰子,以採勝者官之。



 儼嘗得勝採,上曰:「上相之徵也!」遷知樞密院事,賜經邦佐運功臣,封越國公。修《皇朝實錄》七十卷。



 帝大漸,儼與北院樞密使阿思同受顧命。乾統三年,徙封秦國。六年,封漆水郡王。天慶中,以疾,命乘小車入朝。疾甚,遣太醫視之。薨,贈尚父,謚曰忠懿。



 儼素廉潔,一
 芥不取於人。經籍一覽成誦。又善伺人主意。



 妻邢氏有美色,常出入禁中,儼教之曰:「慎勿失上意!」由是權寵益固。三子:處貞,太常少卿;處廉,同知中京留守事;處能,少府少監。



 劉伸,字濟時,宛平人。少穎悟,長以辭翰聞。重熙五年,登進士第,歷彰武軍節度使掌書記、大理正。因奏獄,上適與近臣語,不顧,伸進曰:「臣聞自古帝王必重民命,願陛下省臣之奏。」上大驚異,擢樞密都承旨,權中京副留守。



 詔徙富民以實春、泰二州,伸以為不可,奏罷之。遷大理少卿,人以不冤。升大理卿,改西京副留守。以父憂,終制,
 為三司副使,加諫議大夫,提點大理寺。以伸明法而恕,案冤獄全活者眾,徙南京副留守。俄改崇義軍節度使,政務簡靜,民用不擾,致烏、鵲同巢之異,優詔褒之。改戶部使,歲入羨餘錢三十萬緡,拜南院樞密副使。



 道宗嘗謂大臣曰:「今之忠直,耶律搒、劉伸而已!」宰相楊績賀其得人,拜參知政事。上諭之曰:「卿勿憚宰相!」



 對此院樞密使乙辛勢焰方熾,伸奏曰:「臣於乙辛尚不畏,何宰相之畏!」乙辛銜之,相異排詆,出為保靜軍節度使。上終欲大用,加守太子太保,遷上京留守。乙辛以事徙鎮雄武,復以崇義軍節度使致仕。



 適燕、薊民饑,仰與政趙徽,韓
 造日濟以糜粥,所活不勝算。大安二年卒,上震悼,賻贈加等。



 耶律胡呂,字蘇撒,弘義宮分人。其先欲穩,佐太祖有功,為迭烈部夷離堇。父楊五,左監門衛大將軍。



 胡呂性謙謹,於人無適莫。重熙末,補寢殿小底。以善職,屢更華要,遷千牛衛大將軍。大安中,北阻卜酋磨魯斯叛,為招討都監,與耶律那也率精騎二千討平之,以功為漢人行宮副部署,兼知太和宮事。致仕,加同中書門下平章事,卒。



 論曰:「兀納當道宗昏惑之會,擁佑皇孫,使乙辛奸計不
 獲復逞,而遼祚以續。比之屋質立穆宗,非溢美也。儼以俊才蒞政,所至有能譽;纂述遼史,具一代治亂,亦云勤矣。但其固寵,不能以禮正家,惜哉。劉伸三為大理,民無冤抑;一登戶部,上下兼裕,至與耶律玦並稱忠直,不亦宜乎。」



\end{pinyinscope}