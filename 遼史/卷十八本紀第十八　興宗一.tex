\article{卷十八本紀第十八 興宗一}

\begin{pinyinscope}

 興宗神聖孝章皇帝,諱宗真,字夷不堇,小字只骨。聖宗長子,母曰欽哀皇后蕭氏。上始生,齊天皇后取養之。幼而聰明,長而魅偉,龍顏日角,豁達大度。善騎射,好儒術,通音律。三歲封梁王,太平元年冊為皇太子,十年六月判北南院樞密使事。



 十一年夏六月己卯,聖宗崩,即皇帝位於柩前。壬午,尊母元妃蕭氏為皇太后。甲申,遣使告
 哀於宋及夏、高麗。是年,御宣政殿放進士劉貞等五十七人。辛卯,大赦,改元景福。乙未,奉大行皇帝梓宮,殯於永安山太平殿。辛丑,皇太后賜附馬蕭鉏不里、蕭匹敵死,圍場都太師女直著骨里、右祗候郎君詳穩蕭延留等七人皆棄市,籍其家,遷齊天皇后於上京。



 秋七月丙午朔,皇太后率皇族大臨於太平殿。高麗遣使吊慰。上召晉王蕭普古等飲博,夜分乃罷。丁未,擊鞠。戊申,以耶律韓八為左夷離畢,特末里為左祗候郎君詳穩,橫帳郎君樂古權右祗候郎君詳穩。已酉,以耶律鄭留為於厥迪烈都詳穩,高八為右皮室詳穩。庚戌,振薊州饑民。
 癸丑,詔寫大行皇帝御容。甲寅,錄囚。以觀察姚居信為上將軍。建慶州於慶陵之南,徙民實之,充奉陵邑。乙卯,以比歲豐稔,罷給東京統軍司糧。丁巳,上謁大行皇帝御容,哀慟久之。因詔寫北府宰相蕭孝先,南府宰相蕭孝穆像於御容殿。以蕭阿姑軫為東京留守。



 丁卯,謁太平殿,焚先帝所御弓矢。幸晉王普古第視疾。辛未,錄囚。壬申,上謁神主帳,時奧隈蕭氏始入宮,亦命拜之。



 八月壬午,遷大行皇帝梓宮于菆塗殿。



 九月戊申,躬視慶陵。庚戌,問安於皇太后。辛亥,宋遣王隨、曹儀致祭,王瑠、許懷信、梅詢、張綸來慰兩宮,範諷、孫繼業賀即位,孔道輔、
 魏昭文賀皇太后冊禮。戊午,傑弧矢、鞍勒于菆塗殿。庚申,夏國遣使來慰。庚午,以宋使吊祭,喪服臨菆塗殿。甲戌,遣御史中丞耶律翥、司農卿張確、詳穩耶律勵、四方館使高維翰謝宋吊慰。



 冬十月戊寅,宰臣呂德懋薨。癸未,殺鉏不里黨彌勒奴、觀音奴等。丙戌,遣工部尚書高德順、崇祿卿李可封致先帝遺物於宋;以右領軍衛上將軍耶律遜、少府監馬憚充皇太后謝宋使;右監門衛上將軍耶律元載、引進使魏永充皇帝謝宋使。丁酉,夏國遣使來賻。戊戌,以蕭革、趙為果、恥律鬱、馬保業充來歲賀宋正旦使副。



 閏月辛亥,謁菆塗殿,閱玄宮閟器。有司
 請以生辰為永壽節,皇太后生辰為應聖節,從之。辛酉,閱新造鎧甲。丁卯,振黃龍府饑民。



 十一月壬辰,上率百僚奠於菆塗殿。出大行皇帝服御、玩好焚之,縱五坊鷹鶻。甲午葬文武大孝宣皇帝於慶陵。乙未,祭天地。問安皇太后。丙申,謁慶陵,以遺物賜群臣,名其山曰慶雲,殿曰望仙。



 十二月癸丑,到自慶陵。皇太后聽政,帝不親庶務,群臣表請,不從。



 是歲,以興平公主下嫁夏國王李德昭子元昊,以元昊為夏國公、駙馬都尉。



 重熙元年春正月壬申朔,皇太后御正殿,受帝與群臣朝。



 宋遣任布、王遵範、陳琰、王克善來賀。乙亥,宋遣鄭向、
 郭遵範來賀永壽節。丁丑,如雪林。



 二月,大搜。



 三月壬申朔,尚父、漆水郡王敵烈復為惕隱。



 是春,皇太后誣齊天皇后以罪,遣人即上京行弒。後請具浴以就死,許之,有頃,後崩。



 夏四月乙巳,清暑別輦斗。



 秋七月,豬平地松林。以蕭達溥、王英秀、蕭麓、張素羽充來歲賀宋正旦生辰使。



 八月丙午,駐蹕刺河源。皇子洪基生。



 冬十月已酉,幸中京。



 十一月已卯,帝率群臣上皇太后尊號曰法天應運仁德章聖皇太后;群臣上皇帝尊號曰文武仁聖昭孝皇帝。大赦,改元重熙。癸未,守遣劉隨、王德本來賀應聖節。以楊佶為翰林承旨。



 丙戌,夏國遣使來賀。辛卯,五
 國酋長來貢。夏國王李德昭薨,冊其子夏國公元昊為夏國王。



 十二月庚戌,宋遣胥偃、王從益、崔暨、張懷志來賀來歲正旦;又遣楊日嚴、王克纂來賀永壽節。以北大王耶律求翰同平章事。



 是年,放進士劉師貞等五十七人。



 二年春正月庚辰,東幸。乙酉,夏國遣使來貢。壬辰,女直詳穩臺押率所部來貢。宋遣曹琮來告母後劉氏哀,章得象、安繼昌來饋母後遺物。即遣興聖宮使耶律壽寧、給事中知制誥李奎充祭奠使;天德軍節度使耶律卿寧、大理卿和道亨、河西軍節度使耶律嵩、引進使馬世
 卿充兩宮吊慰使。秋七月甲子朔,以耶律寔、高升、耶律迪、王惟允充兩宮賀宋生辰使副,以耶律師古、劉五常充賀宋來歲正旦使副。



 八月丁酉,幸溫泉宮。乙卯,遣使閱諸路未稼。



 冬十一月甲申,宋遣劉寶,符忠、李昭述、張茂實等來謝慰奠。十二月乙未,宋遣丁度、王繼凝來賀應聖節。巳酉,禁夏國使沿路私市金、鐵。甲寅,宋遣章頻、李懿、王沖睦、張緯、李紘、李繼一來賀永壽節及來歲正旦。庚申,以北府宰相蕭孝先為樞密使。



 三年春正月丁卯,宋使章頻卒,詔有司賻贈,命近侍護喪以歸。辛卯,如春水。



 二月壬辰朔,以北院樞密使蕭普古
 為東京留守。戊申,耶律大師奴有侍襁褓恩,詔入屬籍。



 夏四月甲寅,振耶迷只部。



 五月庚申朔,清暑沿柳湖。是月,皇太后還政於上,躬守慶陵。六月已亥,以蕭普古為南院樞密使。



 秋七月戊子朔,上始親政,以耶律庶徵、劉六符、耶律睦、薄可久充賀宋來歲正旦使副。壬辰,如秋山。



 冬十月已未,駐蹕中會川。



 十二月,宋遣段少連、杜仁贊來賀來歲正旦、楊偕、李守忠來賀永壽節。



 四年春正月庚寅,如耶迷只里。



 三月乙酉朔,立皇后蕭氏。



 夏四月甲寅朔,如涼陘。



 五月庚子,清暑散水源。



 六月癸丑朔,皇子賓信奴生。以耶律信、呂士宗、蕭兗、郭揆充
 賀宋生辰及來歲正旦使副。



 秋七月壬午朔,獵於黑嶺。



 九月已酉,駐蹕長寧澱。



 冬十月,如王子城。



 十一月壬午,改南京總管府為元帥府。乙酉,行柴冊禮於白嶺,大赦。加尚父耶律信寧、政事令耶律求翰耆宿贊翊功臣。



 十二月癸丑,詔諸軍炮、弩、弓、劍手以時閱習。庚申,宋遣鄭戩、柴貽範、楊日華、張士禹來賀永壽節及正旦。



 五年春正月甲申,如魚兒濼。樞密使蕭延寧請改國舅乙室小功帳敞史為將軍,從之。



 夏四月庚申,以潞王查葛為南府宰相,崇德宮使耶律馬惕隱。甲子,幸後弟蕭無曲第,曲水泛觴賦詩。丁卯,頒新定條制,己巳,上與
 大臣分朋擊鞠。



 五月甲午,南幸。丁未,如胡土白山清暑。庚申,幸北院大王高十行帳拜奧,賜銀絹。



 壬戌,詔修南京宮闕府署。



 秋七月辛丑,錄囚。耶律把八誣其弟韓哥謀殺已,有司奏當反坐。臨刑,其弟泣訴:「臣惟一兄,乞貸其死。」上憫而從之。九月癸巳,獵黃花山,獲熊三十六,賞獵人有差。



 冬十月丁未,幸南京。辛亥,曲赦析津府境內囚。壬子,御元和殿,以《日射三十六熊賦》、《幸燕詩》試進士於廷;賜馮立、趙徽四十九人進士第。以馮立為右補闕,趙徽以下皆為太子中佶,賜緋衣、銀魚、遂大宴。御試進士自此始。宋遣宋郊、王世文來賀永壽節。甲子,宰臣張
 儉等請幸禮部貢院,歡飲到暮而罷,賜物有差。以耶律祥、張素民、耶律甫、王澤充賀宋生辰正旦使副。六年春正月丁丑,西幸。



 三月戊寅,以秦王蕭孝穆為北院樞密使,徙封昊王;晉王蕭孝先為南京留守。



 夏四月,獵野狐嶺。



 閏月,獵龍門縣西山。



 五月已酉,清暑炭山。以耶律韓八為北院大王。蕭把哥左夷離畢,王子郎君詳穩鼻姑得林牙,簽北面事耶律涅哥同簽點檢司。甲寅,錄囚。以南大王耶律信寧故匿重囚及待婢贓污,命撻以劍脊而奪其官;都監坐阿附及侍婢罪,皆論死,詔貸之。



 丙辰,以耶律信寧為西南路招使。庚申,出飛龍廄
 馬,賜皇太弟重元及北、南面侍臣有差。癸亥,以上京留守耶律胡睹袞為南大王,平章事蕭查刺寧上京留守,侍中管寧行宮都部署,耶律蒲奴寧烏古迪烈得都詳穩。甲子,以上京留守耶律洪古為北院大王。



 六月壬申朔,以善寧為殿有都點檢。護衛太保耶律合住兼長寧宮使,蕭阿刺里、耶律烏理斡、耶律和尚、蕭韓家奴、蕭特里、蕭求翰為各宮都部署。上酒酣賦詩,昊國王蕭孝穆、北宰相蕭撒八等皆屬和,夜中乃罷。已卯,祀天地。癸未,賜南院大王耶律胡睹兗命,上親為制誥詞,並賜詩以寵之。丙申,以北院大王侯哂為南京統軍使。



 秋七月辛
 丑朔,以北、南樞密院獄空,賞賚有差。壬寅,以皇太弟重元生子,賜詩及寶玩器物,曲赦死罪以下。癸卯,如秋山。



 八月已卯,北樞密院言越棘部民苦其酋帥坤長不法,多流亡;詔罷越棘等五國酋帥,以契丹節度使一員領之。



 冬十月癸酉,駐蹕石寶岡。



 十一月已亥朔,阻卜酋長來貢。辛亥,以契丹行宮都部署蕭惠為南院樞密使。壬子,以管寧為南院樞密使。蕭掃古諸行宮都部署,耶律褭裡知南面行宮副部署,蕭阿刺里左祗候郎君詳穩,耶律曷主右祗候郎君詳穩。庚申,幸晉國公主行帳視疾。



 封皇子洪基為梁王。



 十二月,以楊佶為忠順軍節度
 使。遣耶律斡、秦鑒耶律德、崔繼芳賀宋生辰及正旦。



 七年春正月戊戌朔,宋遣高若訥、夏元正、謝絳、張茂實來賀正旦及永壽節。辛丑,如混同江。



 二月庚午,如春州。乙亥,駐蹕東川。丁丑,高麗遣使來貢。壬午,幸五坊閱鷹鶻。乙酉,遣使慶州問安皇太后。



 三月戊戌朔,幸皇太弟重元行帳。壬寅,如蒲河澱。辛亥,夏國遣使來貢。甲寅,錄囚。



 夏四月己巳,以興平公主薨,遣北院承旨耶律庶成持詔問夏國王李元昊,公主生與元昊不睦,沒,詰其故。己卯,獵白馬堝。甲申,射免新澱井,乙未,獵金山,遣楊家進鹿尾茸於大安宮。



 六月乙亥,御清涼殿試進士,賜邢
 彭年以下五十五人第。



 秋七月甲辰,錄囚。乙巳,阻卜酋長屯禿古斯來朝。戊申,如黑嶺。



 九月丁未,駐蹕平澱。



 冬十月甲子朔,渡遼河。丙寅,駐蹕白馬澱。壬申,錄囚。



 十一月癸巳朔。以耶律元方、張泥、韓至德、蕭傅充駕宋生辰正旦使副。辛丑,問安皇太后,進珍玩。庚申,錄囚。



 十二月,召善擊鞠者數十大於東京。令與近臣角勝,上臨觀之。己巳,以皇太弟重元判北南院樞密使事,北府宰相撒八寧再任兼知東京留守事,耶律應穩南府宰相,查割折大內惕隱,乙室己帳蕭翰乾州節度使,劉六符參知政事,王子帳冠哥王子郎君詳穩,鉏窘大王平川節度
 使,宰臣張克恭守司空,宰臣韓紹芳加侍中,惕隱耶律馬六北院宣徽使,傅父耶律喜孫南府宰相,癸未,守遣王舉正、張士禹來賀永壽節。甲申,命日進酒於大安宮,致薦慶陵。丁亥,錄囚。非故殺者減科。南面待御壯骨裏詐取女直貢物,罪死;上以有吏能。默而流之。



 八年春正月壬辰朔,宋遣韓畸、王從益來賀。丙申,如混同江觀漁。戊戌,振品部。庚戌,叉魚於率沒里河。丁巳,禁朔州鬻羊於宋。



 二月丙子,駐蹕長春河。



 夏六月乙丑,詔括戶口。



 秋七月丁巳,謁慶陵,致奠於望仙殿;迎皇太后至顯州,謁園陵,還京。



 冬十月,駐蹕東京。



 十一月甲午,詔
 有言北院處事失平,擊鐘及邀駕告者,悉以奏聞。戊戌,朝皇太后,召僧論佛法。戊申,皇太后行再生禮,大赦。己酉,城長春。



 閏十二月壬辰,禮昊國王蕭孝穆疾。宋遣龐籍、杜贊來賀永壽節。



 九年春正月丙辰朔,上進酒於皇太后宮,御正殿。宋遣王拱辰、彭再恩來賀。庚申,如鴨子河。



 二月,駐蹕魚兒濼。



 三月辛未,以應聖節,大赦。



 五月乙卯朔,清暑永安山。



 六月,射柳祈雨。



 秋七月癸酉,宋遣郭禎以伐夏來報,遣樞密使杜防報聘。



 丁丑,如秋山。



 冬十月癸未朔,駐蹕中會川。十一月甲子,女直侵邊,發黃龍府鐵驪軍拒之。宋遣
 蘇伸、向傳範來賀應聖節。



 十二月庚寅,以北大王府布猥帳郎君自言先世與國聯姻,許置敞史,命本帳蕭胡睹為之。辛卯,以所得女直戶置肅州。



 以蕭迪、劉三嘏、耶律元方、王惟吉、耶律庶忠、孫女昭、蕭紹簡、秦德昌充賀宋生辰及來歲正旦使副。沼諸犯法者,不得為官吏。諸職官非婚祭,不得沉酗廢事。有治民安邊之略者,悉具
 以聞。



\end{pinyinscope}