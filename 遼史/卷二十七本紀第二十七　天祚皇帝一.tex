\article{卷二十七本紀第二十七 天祚皇帝一}

\begin{pinyinscope}

 天祚皇帝,諱延禧,字延寧,小字阿果。道宗之孫,父順宗大孝順聖皇帝,母貞順皇后蕭氏。大康元年生。六歲封梁王,加守太尉,兼中書令。後三年,進封燕國王。大安七年,總北南院樞密使事,加尚書令。為天下兵馬大元帥。



 壽隆七年正月甲戌,道宗崩,奉遺詔即皇帝位於樞前。群臣上尊號曰天祚皇帝。



 二月壬辰朔,改元乾統,大赦。
 詔為耶律乙辛所誣陷者,復其官爵,籍沒者出之,流放者還之。乙未,遣使告哀於宋及西夏、高麗。乙巳,以北府宰相蕭兀納為遼興軍節度使,加守太傅。三月丁卯,詔有司以張孝傑家屬分賜群臣。甲戌,召僧法頤放戒於內庭。



 夏四月,旱。



 六月庚寅朔,如慶州。甲午,宋遣王潛等來吊祭。丙申,高麗、夏國各遣使慰奠。戊戌,以南府宰相斡特刺兼南院樞密使。庚子,追謚懿德皇后為宣懿皇后。壬寅,以宋魏國王和魯斡為天下兵馬大元帥。乙巳,以北平郡王淳進封鄭王。丁未,北院樞密使耶律阿思加於越。辛亥,葬仁聖大孝文皇帝、宣懿皇后於慶陵。



 秋
 七月癸亥,阻卜、鐵驪來貢。



 八月甲寅,謁慶陵。



 九月壬申,謁懷陵。乙亥,駐蹕藕絲澱。



 冬十月壬辰,謁乾陵。甲辰,上皇考昭懷太子謚曰大孝順聖皇帝,廟號順宗,皇妣曰貞順皇后。



 十二月戊子,以樞密副使張琳知樞密院事,翰林學士張奉珪參知政事兼同知樞密院事。癸己,宋遣黃實來賀即位。丁酉,高丙、夏國並遣使來賀。乙巳,詔先朝已行事,不得陳告。



 初,以楊割為生女直部節度使。其俗呼為太師。是歲楊割死。傳於兄之子烏雅束,束死,其弟阿骨打襲。



 二年春五月,如鴨子河。



 二月辛卯,如春州。



 三月,大寒,冰
 復合。



 夏四月辛亥,詔誅乙辛黨,徙其子孫於邊;發乙辛、得裡特之墓,剖棺。戮尸;以其家屬分賜被殺之家。



 五月乙丑,斡特刺獻耶睹刮等部捷。



 六月壬辰,以雨罷獵,駐蹕散水原。丙午,夏國王李乾順復遣使請尚公主。丁未,南院大王陳家奴致仕。壬子,李乾順為宋所攻,遣李造福、田若水求援。



 閏月庚申,策賢良。壬申,降惠妃為庶人。



 秋七月,獵黑嶺,以霖雨,給獵人馬。阻卜來侵,斡特刺等戰敗之。



 冬十月乙卯,蕭海裏叛,劫乾州武軍器甲。命北面林牙郝家奴捕之,蕭海裏亡入陪術水阿典部。丙寅,以南府宰相耶律斡特刺為北院樞密使,參知政事牛
 溫舒知南院樞密使事。十一月乙未,郝家奴以不獲蕭海裏,兔官。壬寅,以上京留守耶律慎思為北院樞密副使。有司請以帝生日為天興節。



 三年春正月辛巳朔,如混同江。女直函蕭海裡首,遣使來獻。戊申,如春州。



 二月庚午,以武清縣大水,弛其陂澤之禁。



 夏五月戊子,以獵人多亡,嚴立科禁。乙巳,清暑赤勒嶺。



 丙午,謁慶陵。



 六月辛酉,夏國王李乾順復遣使請尚公主。



 秋七月,中京雨雹,傷稼。



 冬十月甲辰,如中京。己未,吐蕾遣使來貢。庚申,夏國復遣使求援。己巳,有事於觀德殿。



 十一月丙申,文武百官加上尊號曰惠文智武
 聖孝天祚皇帝,大赦,以宋魏國王和魯斡為皇太叔,梁王撻魯進封燕國王,鄭王淳為東京留守,進封越國王,百官各進一階。丁酉,以惕隱耶律何魯掃古為南院大王。戊戌,以受尊號,告廟。乙巳,謁太祖廟,追尊太祖之高祖曰昭烈皇帝,廟號肅祖,批曰昭烈皇后;曾祖曰莊敬皇帝,廟號懿祖,妣曰莊敬皇后。召監修國史耶律伊纂太祖諸帝《實錄》。



 十二月戊申,如藕絲澱。



 是年,放進士馬恭回等百三人。



 四年春正月戊子,幸魚兒濼。壬寅,獵木嶺。癸卯,燕國王撻魯薨。



 二月丁丑,鼻骨德遣使來貢。



 夏六月甲辰,駐蹕
 旺國崖。甲寅,夏國遣李造福、田若水求援。癸亥,吐蕃遣使來貢。



 秋七月,南京蝗。庚辰,獵南山。癸未,以西北路招討使蕭得裏底、北院樞密副使耶律慎思並知北院樞密使事。辛卯,以同知南院樞密使事蕭敵里為西北路招討使。



 冬十月己酉,鳳凰見於淳陰。己未,幸南京。



 十一月乙亥,御迎月樓,賜貧民錢。



 十二月辛丑,以張琳為南府宰相。



 五年春正月乙亥,夏國遣李造福等來求援,且乞伐宋。庚寅,以遼興軍節度使蕭常哥為北府宰相。丁酉,遣樞密直學士高端禮等颯宋罷伐夏兵。



 二月癸卯,微行,視
 民疾苦。丙午,幸鴛鴦濼。



 三月壬申,以族女南仙封成安公主,下嫁夏國王李乾順。



 夏四月甲申,射虎炭山。



 五月癸卯,清暑南崖。壬子,宋遣曾孝廣,王戩報聘。



 六月甲戌,夏國遣使來謝,及貢方物。乙丑,幸候里吉。



 秋七月,謁慶陵。



 九月辛亥,駐蹕藕絲澱。乙卯,謁乾陵。



 冬十一月戊戌,禁商賈之家應進士舉。丙辰,高麗三韓國公王顒薨。子俁遣使來告。



 十二月己巳,夏國復遣李造福、田若水求援。癸酉,宋遣林洙來議興夏約和。



 六年春正月辛丑,遣知北院樞密使事蕭得裏底、知南院樞密使事牛溫舒使宋,諷歸所侵夏地。



 夏五月,清暑散
 水原。



 六月辛巳,夏國遣李造福等來謝。



 秋七月癸巳,阻卜來貢。甲午,如黑嶺。庚子,獵鹿角山。



 冬十月乙亥,宋與夏通好,遣劉正符、曹穆來告。庚辰,以皇太叔、南京留守和魯斡兼惕隱,東京留守、越國王淳為南府宰相。



 十一月乙未,以謝家奴為南院大王。馬奴為奚主部大王。丙申,行柴冊禮。戊戌,大赦。以和魯斡為義和仁聖皇太叔,越國王淳進封魏國王,封皇子敖盧斡為晉王,習泥烈為饒樂郡王。己亥,謁太祖廟。甲辰,祠木葉山。



 十二月己巳,封耶律伊為漆水郡王,餘官進爵有差。



 七年春正月,鉤魚于鴨子河。



 二月,駐蹕大魚濼。



 夏六月,
 次散水原。



 秋七月,如黑嶺。



 冬十月,謁乾陵,獵醫巫閭山是年,放進士李石等百人。



 八年春正月,如春州。



 夏四月丙申,封高麗王俁為三韓國公,贈其父顒為高麗國王。



 五月,清暑散水原。



 六月壬辰,西北路招討使蕭敵里率諸蕃來朝。丙申,射柳祈雨。壬寅,夏國王李乾順以成安公主生子,遣使來告。丁未,如黑嶺。



 秋七月戊辰,以雨罷獵。



 冬十二月己卯,高麗遣使來謝。



 九年春正月丙午朔,如鴨子河。



 二月,如春州。



 三月戊午,夏國以宋不歸地,遣使來告。



 夏四月壬午,五國部來貢。



 六月乙亥,清暑特禮嶺。



 秋七月,隕霜,傷稼。甲寅,獵於候里吉。



 八月丁酉,雪,罷獵。



 冬十月癸酉,望祠木葉山。丁丑,詔免今年租稅。十二月甲申,高麗遣使來貢。



 是年,放進士劉楨等九十人。



 十年春五月辛丑,預行立春禮。如鴨子河。



 二月庚午朔,駐蹕大魚濼。



 夏四月丙子,五國部長來貢。丙戌,預行再生禮。癸巳,獵於北山。



 六月甲戌,清暑玉丘。癸未,夏國遣李造福等來貢。甲午,阻卜來貢。



 秋七月辛丑,謁慶陵。



 閏月辛亥,謁懷陵。己未,謁祖陵。壬戌,皇太叔和魯斡薨。



 九月甲戌,免重九節禮。



 冬十月,駐蹕藕絲澱。



 十二月己酉,
 改明年元。



 是歲,大饑。



 天慶元年春正月,鉤魚于鴨子河。



 二月,如春州。



 三月乙亥,五國部長來貢。



 夏五月,清暑散水原。



 秋七月,獵。



 冬十月,駐蹕藕絲澱。



 二年春正月己未朔,如鴨子河。丁丑,五國部長來貢。



 二月丁酉,如春州,幸混同江鉤魚,界外生女直酋長在千里內者,以故事皆來朝。適遇:「頭魚宴」,酒半酣,上臨軒,命諸酋次第起舞;獨阿骨打辭以不能。諭之再三,終不從。他日,上密謂樞密使蕭奉先曰:「前日之燕,阿骨打意氣雄豪,顧視不常,可托以邊事誅之。否則,必貽後患。」奉先
 曰:「犥人不知禮義,無大過而殺之,恐傷向化之心。假有異志,又何能為?」其弟吳乞買、粘罕、胡舍等嘗從獵,能呼鹿,刺虎,搏熊。上喜,輒加官爵。



 夏六月庚寅,清暑南崖。甲午,和州回鶴來貢。戊戌,成安公主來朝。甲辰,阻卜來貢。



 秋七月乙丑,獵南山。



 九月己未,射獲熊,燕群臣,上新御琵琶。初,阿骨打混同江宴歸,疑上知其異志,遂上知其異志,遂稱兵,先並旁近部族。女直趙三、阿鶻產拒之,阿骨打虜其家屬。二人走訴咸州,詳穩司送北樞密院。樞密使蕭奉先作常事以聞上,仍送咸州詰責,欲使自新。後數召,阿骨打竟稱疾不至。



 冬十月辛亥,高麗三韓國公王侯之母死,
 來告,即遣使致祭,起復。是月,駐蹕奉聖州。



 十一月乙卯,幸南京。丁卯,謁太祖廟。



 是年,放進士韓昉等七十七人。



 三年春正月丙寅,賜南京貧民錢。丁卯,如大魚濼。甲戌,禁僧尼破戒。丙子,獵狗牙出,大寒,獵人多死。



 三月,籍諸道戶,徙大牢古山圍場地居民於別土。阿骨打一日率五百騎突至咸州,吏民大驚。翌日,赴詳穩司,與趙三等面折庭下。阿骨打不屈,送所司問狀。一夕遁去。遣人訴於上,謂詳穩司欲見殺,故不敢留。自是召不復至。



 夏閏四月,李弘以左道聚眾為亂,支解。分示五京。



 六月乙卯,斡朗改國遣使來貢良犬。丙辰,夏國遣使來貢。



 秋七月,
 幸秋山。



 九月,駐蹕藕絲澱。



 十一月甲午,以三司使虞融知南院樞密使事。西南面招討使蕭樂古為南府宰相。十二月庚戌,高麗遣使謝致祭。癸丑,回鶻遣使來貢。甲寅,以樞密直學士馬人望參知政事。丙辰,知樞密院事耶律伊蓖。癸亥,高麗遣使來謝起復。



 四年春正月,如春州。初,女直起兵,以紇石烈部人阿疏不從,遣其部撒改討之。阿疏弟狄故保來告,詔諭使勿討,不聽,阿疏來奔。至是女直遣使來索,不發。



 夏五月,清暑散水原。



 秋七月,女直復遣使取阿疏,不發,乃遣待御阿息保問境上多建城堡之故。女直以慢語答曰:「若還
 阿疏,朝貢如故;不然,城未能已。」遂發渾河北諸軍,益東北路統軍司。阿骨打乃與弟粘罕、胡舍等謀,以銀術割、移烈、婁室、闍母等為帥,集女真諸部兵,擒遼障鷹官。及攻寧江州。東北路統軍可以聞。時上在慶州射鹿,聞之略不介意,遣海州刺史高仙壽統渤海軍應援。蕭撻不也遇女直,戰於寧江東,敗績。



 冬十月壬寅朔,以守司空蕭嗣先為東北路都統,靜江軍節度使蕭撻不也為副,發契丹奚軍三千人,中京禁兵及土豪二千人,別選諸路武勇二千餘人,以虞候崔公義為都押官,控鶴指揮邢穎為副,引軍屯出河店。兩軍對壘,女直軍潛渡混同江,掩
 擊遼眾。蕭嗣先軍演,崔公義、邢穎、耶律佛留、蕭葛十等死之,其獵兔者十有七人。蕭奉先懼其弟嗣先獲罪,輒奏東征軍所至劫掠,若不肆赦,恐聚為患。上從之,嗣先但免官而己。



 諸軍相謂曰:「戰則有死而無功,退則有生而無罪。」故士無鬥志,望風奔潰。



 十一月壬辰,都統蕭敵裡等營於斡粼濼東,又為女直所襲,士卒死者甚眾。甲午,蕭敵里亦坐免官。辛丑,以西北路招討使耶律斡里朵為行軍都統,副點檢蕭乙薛、同知南院樞密使事耶律章奴副之。十二月,咸、賓、祥三州有鐵驪、兀惹皆叛入女直。乙薛往援賓州,南軍諸將實婁、特烈等往援咸
 州,並為女直所敗。



\end{pinyinscope}