\article{卷五十二志第二十一 禮志五 嘉儀上}

\begin{pinyinscope}

 皇帝受冊儀:前期一日,尚舍奉御設幄於正殿北墉下,南面設御坐;奉禮郎設官僚、客使幕次於東西朝堂;太樂令設宮懸於殿庭,舉麾位在殿第二重西階上,東向;乘黃令陳車輅;尚輦奉御陳輿輦;尚舍奉御設解劍席於東西階。設文官六品已上位橫街南,東方
 西向;武官五品已上位橫街南,西方東向。皆北上重行,每等異位。將士各勒所部六軍仗屯諸門。金吾仗、黃麾仗陳於殿庭。至日,押冊官引冊自西便門入,置冊案西階上。



 通事舍人引侍從班人,就位。侍中東階上,解劍履,上殿,外俯伏跪,奏「中嚴」;下殿,劍履,復位立。閣使西階上殿,欄外跪請木契;面殿鞠躬,奏「奉敕喚仗。」殿中監、少監、殿中丞等押金吾四色仗入,位臣僚後。協律郎入,就舉麾位,符寶郎詣閣奉迎。通事舍人引文官四品至六品、武官三品至五品,就門外位。皇帝御輦至宣德門。宣徽使押內諸司班起居,引皇
 帝至閣,服袞冕。侍中東階下,解劍履,上殿,版奏外辦。



 太賞博士引太常卿,太常卿引帝。內諸司出,協律郎舉麾,太樂令令撞黃鐘之鐘,左五鐘皆應,工人鼓枲,樂作;皇帝即御坐,宣徽使贊扇合,樂上;贊簾卷,扇開。符寶郎奉寶進,左右金吾報平安。通事舍人引文官三品武官二品已上入門。樂作;就相向位畢,床上。通事舍人引侍從班、南班文官三品、武官二品已上合班,北向。東班西上,西班東上,起居,七拜。分班,各復位,通事舍人引押冊官押冊自西階下,至丹墀,當殿置香案冊案。置冊訖,樂作;就位,樂止。捧冊官近後,東西
 相對立,舍人引侍從班並南班合班,北向如初。贊再拜,在位者皆再拜;舞蹈,五拜。分班,各復位如初。捧冊官就西階下解劍席,解劍履,捧冊西階上殿,樂作;置冊御坐前,東西立,北向。捧冊官西墉下立,北上,樂止。讀冊官出班,當殿立,贊再拜,三呼「萬歲」。就西階下解劍席,解劍履,西階上殿,欄內立,當御坐前。侍中取冊,捧冊官捧冊匣至讀冊官前跪,相對捧冊。讀冊官俯伏跪;讀訖,俯伏興。捧冊官跪左膝,以冊授侍中。侍中受冊,以冊授執事者,降自西階,劍履訖,復當殿位。贊再拜,三呼「萬歲」,復分班位。舍人引侍從班、南班合班,
 北向如初。贊拜,在位者皆拜;舞蹈、鞠躬如初。



 通事舍人引班首西階下,解劍履。上殿,樂作;就欄內位,樂止。俯伏跪,通全銜臣某等致詞稱賀訖,俯伏興。降西階下,帶劍,納閤,樂作;復位,樂止。贊拜,在位者皆再拜,舞蹈,五拜,鞠躬。侍中臨軒西向,稱「有制」,皆再拜。侍中宣答訖,贊皆再拜,舞蹈,五拜,分班各復位。三品已上出,樂作;出門畢,床上。侍中當御坐俯伏跪,通全銜奏「禮畢」,俯伏興。退,東階下殿,帶劍,納履,復位。宣徽使贊扇合,下簾。



 太常博士、太常卿引皇帝起,樂作;至閣,樂止。舍人引文官四品、武官三品以下出門外,分班立;次
 引侍從班出,次兵部、吏部出。次金吾出,次起居郎、舍人出,次殿中監、少監押金吾細仗出,仍位臣僚後。次東西上閣門使於丹墀內鞠躬,奏衙內無事,卷班出。閣門使丹墀內鞠躬,揖「奉敕放仗」。出,門外文武班中間立,喚承受官。承受官聲喏,至閣使後,鞠躬,揖。閣使鞠躬。稱「奉敕放仗。」承受聲喏,鞠躬,揖,平身立,引聲「奉敕放仗」。聲絕,趨退。文武合班,再拜。舍人一員攝詞令官,殿前鞠躬,揖,稱「奉敕放黃麾仗」,出,放金吾仗亦如之,翼日,文武臣僚入問聖躬。



 太平元年,行此儀,大略遵唐、晉舊儀。又有《上契丹
 冊儀》,以阻午可汗柴冊禮合唐禮雜就之。又有《上漢冊儀》,與此儀大同小異,加以《上寶儀》。



 冊皇太后儀:前期,陳設於元和殿如皇帝受冊之儀。至日,皇帝御弘政殿。冊入,侍從班入,門外金吾列仗,文武分班。



 侍中解劍,奏「中嚴」。宣徽使請木契、喚仗皆如之,樂工入,閣使門外文武班中間立,喚承受官。聲喏,趨至閣使後立。閣使鞠躬,揖,稱「奉敕喚仗」。承受官鞠躬,聲喏,揖,引聲「奉敕喚仗」。文武合班,再拜。殿中監押仗入,文武班入,亦如之。宣微使押內諸司供奉官天橋班候。皇太后御紫宸殿,乘平頭輦,童子、女童隊樂
 引。至金鑾門,閣使奏內諸司起居訖,贊引駕,自下先行至元和殿,皇太后入西北隅閣內更衣。



 侍中解劍,上殿奏外辦。宣徽受版入奏。侍中降,復位。協律郎舉麾,樂作。太樂令、太常卿導引皇太后升坐。宣徽使贊扇合,簾卷,扇開,樂止。符寶郎奉寶置皇太后坐右。左右金吾大將軍對揖,鞠躬,奏「軍國內外平安」。東上閣門副使引丞相東門入,西上閣門副使引親王西門入,通事舍人引文武班入,如儀,樂作;至位,樂止。文武班趨進,相向再拜,退復位。



 東西上閣門使、宣徽使自弘政殿引皇帝御肩輿至西便門下。引入門,樂作;至
 殿前位,樂止。宣徽使贊皇帝拜,問皇太后「聖躬萬福」,拜。皇帝御西閣坐,合班起居如儀。北府宰相押冊,中書、樞密令史八人舁冊,東西上閣門使引冊,宣徽使引皇帝送冊,樂作;至殿前置冊位,樂止。宣微使贊皇帝再拜,稱「萬歲」,群臣陪位,揖。翰林學士四人、大將軍四人舁冊。



 皇帝捧冊行,三舉武,授冊。舁之西階上殿,樂作。置太后坐前,樂止。皇帝冊西面東立。舍人引丞相當殿再拜,三呼「萬歲」,解劍,西階上殿,樂作;至讀冊位,樂止。俯伏跪讀冊訖,俯伏三呼「萬歲」,復班位。宣徽使引皇帝下殿,樂作;至殿前位,樂止。皇帝拜,舞蹈,拜
 訖,引皇帝西階上殿。至皇太后坐前位,俯跪;致詞訖,俯伏興。引西階下,至殿前位,拜,舞蹈,拜,鞠躬。侍中臨軒,宣太后答稱「有制」,皇帝再拜。宣訖,引皇帝上殿,樂作;至西閣,樂止。丞相、親王、侍從文武合班,贊拜,舞蹈,三呼「萬歲」如儀。丞相上賀,侍中宣答如儀。丞相以下出,舉樂;出門,樂止。侍中奏「禮畢」,宣徽索扇,扇合,下簾。皇太后起,舉樂;入閣,樂止。



 文武官出門外分班侍從。兵部,吏部起居,金吾仗出,如儀。



 閣使奏「放仗」,皆如皇帝受冊之儀。



 冊皇后儀:至日,北南臣僚、內外命婦詣端拱殿幕次。皇
 后至閣,侍中奏「中嚴」,引命婦班入,就東西相向位立。皇帝臨軒,命使發冊。使副押冊至端拱殿門外幕次。侍中奏外辦。



 所司承旨索扇,扇上,舉麾,樂作;皇后出閣升坐,扇開,簾卷,偃麾,樂止。引命婦合班面殿起居,八拜。皇后降坐,樂作;至殿下褥位,樂止。引冊入,置皇后褥位前。侍中傳宣,皇后四拜,命婦陪位皆拜。引讀冊官至皇后褥位前,俯伏跪讀訖,皇后四拜,陪位者皆拜。引皇后升殿,使臣引冊,置皇后坐前冊案,退,西向侍立。命婦當殿稱賀,四拜。引班首東階上殿,致詞訖東階下殿,復位,四拜。侍中奏宣答稱「有教旨」,四拜。
 宣答訖,四拜。班首上殿進酒,皇后賜押冊使副等酒訖,侍中奏「禮畢」。承旨索扇,樂作,皇后起;入閣,樂止。分引命婦等東西門出。



 冊皇太子儀,前期一日,設幄坐於宣慶殿,設文武官幕次於朝堂,並殿庭板位,太樂令陳宮縣,皆如皇帝受冊儀。守宮設皇太子次於朝堂北,西向;乘黃令陳金輅朝堂門外,西向;皇太子儀仗、加蕭、鼓吹等陳宣慶門外;典儀設皇太子板位於殿橫街南,近東北向;設文武官五品以上位於樂縣東西;餘官如常儀。至日,門下侍郎奉冊,中書侍郎奉寶綬,各置於案。令史二
 人絳服,對舉案立。寶案在橫街北西向,冊案在北。門下侍郎、中書侍郎並立案後。侍中板奏「中嚴」。皇太子遠游冠,絳紗袍,秉堦出。太子舍人引入,就板位北面殿立。東宮官三師以下皆從,立皇太子東南,西向。太子入門,樂作;至位,樂止。典儀贊皇太子再拜,在位者皆再拜。中書令立太子東北,西向,門下侍郎引冊案,中書侍郎取冊,進授中書令,通復位,傳宣官稱「有制」,皇太子再拜。傳宣訖,再拜。中書令跪讀冊訖,俯伏興。皇太子再拜,受冊,退授左庶子。中書侍郎取寶,進授中書令。皇太子進受寶,退授左庶子。中書令以下退,
 復位。舁案者以案退。典儀贊再拜,皇太子拜,在位者皆再拜。



 太子舍人引皇太子退,樂作;出門,樂止。侍中奏「禮畢」。



 皇太子升金輅,左庶子以下夾侍,儀仗、鼓吹等並列宣慶門外,三師、三少諸宮臣於金輅前後導從,鳴鐃而行,還東宮。宮庭先設仗衛如式,至宮門,鐃止。皇太子降金輅,舍人引入就位坐,文武官臣序班稱賀。禮畢。



 冊王妃公主儀:至日,押冊使副並讀冊等官押冊東便門入,持節前導至殿。冊案置橫街北少東。引使副等面殿立而鞠躬。



 侍中臨軒稱「有制」,皆再拜,鞠躬。宣制
 訖,舞蹈,五拜,引冊於宣慶門出。使副等押領儀仗、冊案,赴各私第廳前,向闕陳列。設傳宣受冊拜褥,冊案置褥左,去冪蓋。使副案右序立。受冊者就位立,傳宣稱「有制」,再拜。宣制畢,舁冊人舉冊匣於褥前跪捧,引讀冊者與受冊者皆俯伏跪,讀訖,皆俯伏興。受冊者謝恩,國王五拜,王妃、公主四拜。若冊禮同日,先上皇太后冊寶,次臨軒同制,遣使冊皇后、諸王妃主,次冊皇太子。



 皇帝納后之儀,擇吉日。至日,後族畢集。詰旦,後出私舍,坐於堂。皇帝遣使及媒者,以牲酒饔餼至門。執事者
 以告,使及媒者入謁,再拜,平身立。少頃,拜,進酒於皇后,次及後之父母、宗族、兄弟。酒遍,再拜。納幣,致詞,再拜訖,後族皆坐。惕隱夫人四拜,請就車。後辭父母、伯叔父母、兄,各四拜;宗族長者,皆再拜。皇后升車,父母飲後酒,致戒詞,遍及使者、媒者、送者。發軔,伯叔父母、兄飲後酒如初。教坊遮道贊祝,後命賜以物。後族追拜,進酒,遂行。將至宮門,宰相傳敕,賜皇後酒,遍及送者。既至,惕隱率皇族奉迎,再拜。皇后車至便殿東南七十步止,惕隱夫人請降車。負銀罌,捧滕,履黃道行。後一人張羔裘若襲之,前一婦人捧鏡卻行。



 置鞍於
 通,後過其上。乃詣神主室三拜,南北向各一拜,酹酒。



 向謁者一拜。起居訖,再拜。次詣舅姑御容拜,奠酒。選皇族諸婦宜子孫者,再拜之,授以罌、滕。又詣諸帝御容拜,奠酒。



 神賜襲衣、珠玉、珮飾,拜受服之。后姊若妹、陪拜者各賜物。



 皇族迎者、後族送者遍賜酒,皆相偶飲訖,後坐別殿,送後者退食於次。媒者傳旨命送後者列於殿北。俟皇帝即御坐,選皇族尊者一人當奧坐,主婚禮。命執事者往來致辭於後族,引後族之長率送後者升,當御坐,皆再拜;又一拜,少進,附奏送後之詞;退復位,再拜。後族之長及送後者向當奧者三
 拜,南北向各一拜,向謁者一拜。後族之長跪問「聖躬萬福」,再拜;復奏送後之詞,又再拜。當奧者與媒者行酒三周,命送後者再拜,皆坐,終宴。翌日,皇帝晨興,詣先帝御容拜,奠酒訖,復御殿,宴後族及群臣,皇族、後族偶飲如初,百戲、角抵、戲馬較勝以為樂。又翌日,皇帝御殿,賜后族及贐送後者,各有差。受賜者再拜,進酒,再拜。皇帝御別殿,有司進皇后服飾之籍。酒五行,送後者辭訖,皇族獻后族禮物;後族以禮物謝當奧者。禮畢。



 公主下嫁儀:選公主諸父一人為婚主,凡當奧者、媒者
 致詞之儀,自納幣至禮成,大略如納后儀。擇吉日,詰旦,媒者趣尚主之家詣宮。侯皇帝、皇后御便殿,率其族入見。進酒訖,命皇族與尚主之族相偶飲。翼日,尚主之家以公主及婿率其族入見。致宴於皇帝、皇后。獻贐送者禮物訖,朝辭。賜公主青幰車二,螭頭、蓋部皆飾以銀,駕駝;送終車一,車樓純錦,銀螭,懸鐸,後垂大氈,駕牛,載羊一,謂之祭羊,擬送終之具,至覆尸儀物咸在。賜其婿朝服、四時襲衣、鞍馬,凡所須無不備。選皇族一人,送至其家。



 親王女封公主者婚儀:仿此,以親疏為差降。



\end{pinyinscope}