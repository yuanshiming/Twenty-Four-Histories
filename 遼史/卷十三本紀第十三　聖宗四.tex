\article{卷十三本紀第十三 聖宗四}

\begin{pinyinscope}

 八年春正月辛巳,如臺湖。庚寅,詔決滯獄。庚子,如沈子濼。二月丁未朔,於闐、回鶻各遣使來貢。壬申,女直遣使來貢。



 三月丁丑,李繼遷遣使來貢。庚辰,太白、熒惑斗,凡十有五次。乙酉,城杏堝,以宋俘實之。辛丑,置宜州。



 夏四月丙午朔,嚴州刺史李壽英有惠政,民請留,從之。



 庚戌,女直遣使來貢。庚午,以歲旱,諸部艱食,振之。



 五月戊子,
 以宋降卒分隸諸軍。庚寅,女直宰相阿海來貢,封順化王。丙申。清暑胡土白山。詔括民田。



 六月丙午,以北面林牙磨魯古為北院大王。阿薩蘭回鵲於越、達刺幹各遣使來貢。甲寅,月掩天駟第一星。丙辰,女直遣使來貢。



 秋七丹庚辰,改南京熊軍為神軍。詔東京路諸宮分提轄司,分置定霸、保和、宣化三縣,白川州置洪理,儀坤州置廣義,遼西州置長慶,乾州置安德各一縣。省遂、媯、松、饒、寧、海、瑞、玉、鐵里、奉德等十州,及玉田、遼豐、松山、弘遠、懷清、雲龍、平譯、平山等八縣,以其民分隸他郡。



 八月乙卯,以黑白羊祭天地。



 九月乙亥,北女直四部請內附。壬辰,
 李繼遷獻宋俘。



 冬十月丙午,以大敗宋軍,復遣使來告。己酉,阻卜等遣使來貢。是月,駐蹕大王川。



 十一月庚寅,以吐谷渾民饑,振之。酉,太白晝見。



 十二月癸卯,李繼遷下宋麟、鄜等州,遣使來告,女直遣使來貢。庚戌,遣使封李繼遷為夏國王。癸丑,回鶻來貢。



 是歲,放鄭云從等二人及第。



 九年春正月甲戌,女直遣使來貢。丙子,詔禁私度僧尼。



 庚辰,如臺湖。乙酉,樞密使、監修國史室昉等進《實錄》,賜物有差。戊子,選宋降卒五百置為宣力軍。辛卯,詔免三京諸道租賦,仍罷括田。



 二月丙午,夏國遣使告伐宋捷。
 丁未,以涿州刺史耶律王六為惕隱。甲子,建威寇、振化、來運三城,屯戍卒。



 閏月辛未朔,日有食之。壬申,遣翰林承旨邢抱樸、三司使李嗣、給事中劉京、政事舍人張乾、南京副留守吳浩分決諸道滯獄。



 三月庚子朔,振室韋、烏古諸部。戊申,復遣庫部員外郎馬守琪、倉部員外郎祁正、虞部員外郎崔祐,薊北縣令崔簡等分決諸道滯獄。甲子,幸南京。



 夏四月甲戌,回鶻來貢。乙亥,夏國王李繼遷遣杜白來謝封冊。丙戌,清暑炭山。



 五月己未,以秦王韓匡嗣私城為全州。



 六月丁亥,突厥來貢。是月,南京霖雨傷稼。



 秋七月癸卯,通括戶口。乙巳,詔諸道舉才行、
 察貪酷、撫高年、禁奢僭,有歿於王事者官其子孫。己未,夏國以復綏、銀二州,遣使來告。



 八月癸酉,銅州嘉禾生,東京甘露降。戊寅,女直進喚鹿人。壬午,東京進三足烏。



 九月庚子,鼻骨德來貢。己酉,駐蹕廟城。南京地震。



 冬十月丁卯,阿薩蘭回鶻來貢。壬申,夏國王李繼遷遣使來上宋所授敕命。丁丑,定難軍節度使李繼捧來附,授推忠效順啟聖定難功臣、開府儀同三司、檢校太師兼侍中,封西平王。



 十一月己亥,以青牛白馬祭天地。



 十二月,夏國王李繼遷潛附於宋,遣詔討使韓德威持詔諭之。



 是歲,放進士石用中一人及第。



 十年春正月丁酉,禁喪葬禮殺馬,及藏甲胄、金銀、器玩。



 丙午,如臺湖。



 二月乙丑朔,日有食之。韓德威奏李繼遷稱故不出,至靈州俘掠以還。壬申,兀惹來貢。壬午,免雲州租賦。庚寅,夏國以韓德威俘掠,遣使來奏,賜詔安慰。辛卯,給復雲州流民。



 三月甲辰,鐵驪來貢。丙辰,如炭山。



 夏四月乙丑,以臺湖為望幸里。庚寅,命群臣較射。



 五月癸巳,朔州流民給復三年。



 七月辛酉,鐵驪來貢。



 八月癸亥,觀稼,仍遣使分閱苗稼。



 九月癸卯,幸五臺山金河寺飯僧。



 冬十月壬申,夏國王遣使來貢。戊寅,鐵驪來貢。



 十一月壬辰,回鵲來貢。



 十二月庚辰,獵儒州東川。拜天。是
 月,以東京留守蕭恆德等伐高麗。



 十一年春正月壬寅,回鶻來貢。丙午,出內帑錢賜南京統軍司軍。高麗王治遣樸良柔奉表請罪,詔取女直鴨祿江東數百里地賜之。



 二月癸亥,霸州民妻王氏以妖惑眾,伏誅。



 夏四月,幸炭山清暑。



 六月,大雨。



 秋七月己丑,桑乾、羊河溢居庸關西,害禾稼殆盡,奉聖、南京居民廬舍多墊溺者。



 八月,如秋山。



 冬十月甲申朔,駐蹕蒲瑰阪。



 是年,放進士王熙載等二人及第。



 十二年春正月癸丑朔,漷陰鎮水,漂溺三十餘村,詔疏舊渠。甲寅,以同政事門下平章事耶律碩老為惕隱。詔
 復行在五十里內租。乙卯,幸延芳澱。戊午,鑉宜州賦調。庚申,郎君耶律鼻舍等謀叛,伏誅。壬戌,以南院大王耶律景為上京留守,封漆水郡王。霸州民李在宥年百三十有三,賜束帛、錦袍、銀帶,月給羊酒,仍復其家。



 二月甲申,免南京被水戶租賦。己丑,高麗來貢。甲午,免諸部歲輸羊及關征。庚子,回鵲來貢。



 三月丁巳,高麗遣使請所俘人育,詔贖還。戊午,幸南京。



 丙寅,遣使撫諭高麗。己巳,涿州木連理。壬電,如長春宮觀牡丹。是月,復置南京統軍都監。



 夏四月辛卯,幸南京。壬辰,樞密直學士劉恕為南院樞密副使。戊戌,以景宗石像成,幸延壽寺飯僧。



 五
 月甲寅,詔北皮室軍老不任事者免役。戊午,如炭山清暑。庚辰,武定軍節度使韓德沖秩滿,其民請留,從之。



 六月辛巳朔,詔州縣長吏有才能無過者,減一資考任之。



 癸未,可汗州刺史賈俊進新歷。庚子,錄囚。甲辰,詔龍、鳳兩軍老疾者代之。是月,太白、歲星相犯。



 秋七月辛亥朔,日有食之。甲寅,遣使視諸道禾稼。辛酉,南院樞密使室昉為中京留守,加尚父。丙寅,女直遣使來貢。



 戊辰,觀獲。庚午,詔契丹人犯十惡者依漢律。己卯,以翰林承旨邢抱樸參知政事。



 八月庚辰朔,詔皇太妃領西北路烏古等部兵及永興宮分軍,撫定西邊;以蕭撻凜督其軍事。
 乙酉,宋遣使求和,不許。



 戊子,以國舅帳克蕭徒骨為夷離畢。乙未,下詔戒諭中外官吏。



 丁酉,錄囚,雜犯死罪以下釋之。



 九月壬子,室韋、黨項、吐谷渾等來貢。辛酉,宋復遣使求和,不許。壬戌,行拜奧禮。癸酉,阻卜等來貢。



 冬十月乙酉,獵可汗州之西山。乙巳,詔定均稅法。丁未,大理寺置少卿及正。



 十一月戊申朔,行再生禮。鐵驪來貢。詔諸部所俘宋大有官吏儒生抱器能者,諸道軍有勇健者,具以名聞。庚戌,詔郡邑貢明經、茂材異等。甲寅,詔南京決滯獄。己未,官宋俘衛德升等六人。



 十二月戊寅朔,日有食之。詔並奚王府奧理、墮隗、梅只三部為一,其二
 克各分為部,以足六部之數。甲申,賜南京統軍司貧戶耕牛。戊子,高麗進妓樂,卻之。庚寅,禁游食民。



 癸巳,女直以宋人浮海賂本國及兀惹叛來告。丁未,幸南京。



 是年,放進士呂德懋等二人及第。



 十三年春正月壬子,幸延芳澱。甲寅,置廣靈縣。丁巳,增泰州、遂城等縣賦。庚申,詔諸道勸農。癸亥,長寧軍節度使蕭解裏秩滿,民請留,從之。庚午,如長春宮。



 二月丁丑朔,女直遣使來貢。甲辰,高麗遣李周楨來貢。



 三月癸丑,夏國遣使來貢。戊辰,武清縣百餘人入宋境剽掠,命誅之,還其所獲人畜財物。



 夏四月己卯,參知政事邢抱樸以
 母憂去官,起復。丙戌,詔諸道民戶應歷以來脅從為部曲者,仍籍州縣。甲午,如炭山清暑。五月壬子,高麗進鷹。乙亥,北、南、乙室三府請括富民馬以備軍需,不許,給以官馬。



 六月丙子朔,啟聖軍節度使劉繼琛秩滿,民請留,從之。



 丁丑,詔減前歲括田租賦。甲申,以宣徽使阿沒里私城為豐州。



 丙戌,詔許昌平、懷柔等縣諸人請業荒地。



 秋七月乙巳朔,女直遣使來貢。丁巳,兀惹烏昭度、渤海燕頗等侵鐵驪,遣奚王和朔奴等討之。壬戌,詔蔚、朔等州龍衛、威勝軍更戍。



 八月丙子,夏國遣使進馬。壬辰,詔修山譯祠宇、先哲廟貌,以時祀之。



 九月戊午,以南京太
 學生員浸多,特賜水磑莊一區。丁卯,奉安景宗及皇太后石像於延芳澱。



 冬十月乙亥,置義倉。辛巳,回鶻來貢。甲申,高麗遣李知白來貢。戊子,兀惹歸款,詔諭之。庚子,鼻骨德來貢。



 十一月乙巳,阿薩蘭回鵲遣使來貢。辛酉,遣使冊王治為高麗國王。戊辰,高麗遣童子十人來學本國語。



 十二月己卯,鐵驪遣使來貢鷹、馬。辛巳,夏國以敗宋人遣使來告。



 是年放進士王用極等二人。



 十四年春正月己酉,漁於潞河。丁巳,蠲三京及諸州稅賦。



 丙寅,夏國遣使來貢。庚午,以宣徽使阿沒里家奴閻貴為豐州刺史。二月庚寅,回鶻遣使來貢。三月壬寅,高
 麗王治表乞為婚,許以東京留守、駙馬蕭恆德文嫁之。庚戌,高麗復遣童子十人來學本國語。甲寅,韓德威奏討黨項捷。甲子,詔安集朔州流民。



 夏四月甲戌,東邊諸糾各置都監。庚寅,如炭山清暑。己亥,鑿大安山,取劉守光所藏錢。是月,奚王和朔奴、東京留守蕭恆德等王人以討兀惹不克,削官。改諸部令穩為節度使。



 五月癸卯,詔參知政事邢抱樸決南京滯獄。庚戌,朔州威勝軍一百七人叛入宋。



 六月辛未,如炭山清暑。鐵驪來貢。乙酉,回鶻來貢。己丑,高麗遣使來回起居。後至無時。



 秋七月戊午,回鶻等來貢。



 閏月丁丑,五院部進穴地所得金馬。



 冬十月丙辰,命劉遂教南京神武軍士劍法,賜袍帶錦幣。



 戊午,烏昭度乞內附。



 十一月甲戌,詔諸軍官毋非時畋獵妨農。乙酉,奉安景宗及太后石像於乾州。是月,回鶻阿薩蘭遣使為子求婚,不許。



 十二月甲寅。以南京道新定稅法太重,減之。甲子,撻凜誘叛酋阿魯敦等六十人斬之,封蘭陵郡王。幸南京。



 是年,放進士張儉等三人。



 十五年春正月庚午,幸延芳澱。丙子,以河西黨項叛,詔韓德威討之。庚辰,詔諸道勸民種樹,癸未,兀惹長武周來降。



 戊子,女直遣使來貢。己丑,詔南京決滯囚。乙未,免流民稅。



 二月丙申朔,如長春宮。戊戌,勸品部富民出錢
 以贍貧民。



 庚子,徙梁門、遂城、泰州、北平民於內地。丙午,夏國遣使來貢。甲寅,問安皇太后,丙辰,韓德威奏破黨項捷。丁巳,詔品部曠地令民耕種。



 三月乙丑朔,黨項來貢。戊辰,募民耕灤州荒地,免其租賦十年。己巳,夏國破宋兵,遣使來告。己卯,封夏國王李繼遷為西平王。壬午,通括宮分人戶,免南京逋稅及義倉粟。甲申,河西黨項乞內附。庚寅,兀惹烏昭度以地遠,乞歲時免進鷹、馬、貉皮,詔以生辰、正旦貢如舊,餘免。癸巳,宋主炅殂,子恆嗣位。甲午,皇太妃獻西邊捷。



 夏四月乙未朔,罷奚五部歲貢麇。戊戌,錄囚。壬寅,發義倉粟振南京諸縣民。丙午,廣
 德軍節度使韓德凝有善政,秩滿,其民請留,從之。己酉,幸南京。丁巳,致奠於太宗皇帝廟。己未,如炭山清暑。



 五月甲子朔,日有食之。己巳,詔平川決滯獄。是月,敵烈八部殺詳穩以叛,蕭撻凜追擊,獲部族之半。



 六月丙申,鐵驪來貢。壬子,夏國遣使來謝封冊。



 秋七月戊辰,黨項來貢。辛未,禁吐谷渾別部鬻馬於宋。



 丙子,高麗遣韓彥敬奉幣吊越國公主之喪。辛卯,詔南京疾決獄訟。八月丁酉,獵於平地松林,皇太后誡曰:「前聖有言:欲不可縱。吾兒為天下主,馳騁田獵,萬一有銜撅之變,適遺予憂。其深戒之!」



 九月丙寅,罷東邊戍卒,庚午,幸饒州,致奠太祖
 廟。戊子,蕭撻凜奏討阻卜捷。



 冬十月壬辰朔,駐蹕駝山,罷奚王諸部貢物。乙未,賜宿衛時服。丁酉,禁諸山寺毋濫度僧尼。戊戌,弛東京道魚濼之禁。戊申,以上京獄訟繁冗,詰其主者,辛酉,錄囚。



 十一月壬戌朔,錄囚。丙戌,幸顯州。戊子,謁顯陵。庚寅,謁乾陵。是月,高麗王治薨,侄誦遣王同穎來告。



 十二月乙巳,鉤魚土河。己酉,駐蹕駝山。壬子,夏國遣使來貢。甲寅,遣使祭高麗王治,詔其侄權知國事。丙辰,錄囚。



 是年,放進士陳鼎等二人。



\end{pinyinscope}