\article{卷六十五 列傳第三}

\begin{pinyinscope}

 耶律曷魯蕭敵魯阿古只耶委斜涅赤老古頗德耶律欲穩耶律海裡耶律曷魯,字控溫,一字洪隱,迭刺部人。祖匣馬葛,簡憲皇帝兄。父偶思,遙輦時為本部夷離堇,曷魯其長子也。



 性質厚。在髫鬢,與太祖游,從父釋魯奇之曰:「興我家者,必二兒也。」太祖既長,相與易裘馬為好,然曷魯事太祖
 彌謹。會滑哥弒其父釋魯,太祖顧曷魯曰:「滑哥弒父,料我必不能容,將反噬我。今彼歸罪臺曬為解,我姑與之。是賊吾不忘也!」自是,曷魯常佩刀從太祖,以備不虞。



 居久之,曷魯父偶思病,召曷魯曰:「阿保機神略天授,汝率諸弟赤心事之。」幾而太祖來問疾,偶思執其手曰:「爾命世奇才。吾兒蜀魯者,他日可委以事,吾已諭之矣。」既而以諸子屬之。



 太祖為撻馬狘沙裏,參預部族事,曷魯領數騎召小黃室韋來附。太祖素有大志,而知曷魯賢,軍國事非曷魯議不行。會討越兀與烏古部,曷魯為前鋒,戰有功。



 及太祖為迭刺部夷離堇,討英部,其長術裏通
 險而壘,攻莫能下,命曷魯持一笴往諭之。既入,為所執。乃說奚曰:「契丹與奚言語相通,實一國也。我夷離堇于奚豈有輘轢之心哉?漢人殺我祖奚首,夷離堇怨次骨,日夜思報漢人。顧力單弱,使我求援於奚,傳矢以示信耳。夷離堇受命於天,撫下以德,故能有此眾也。今奚殺我,違天背德,不祥莫大焉。且兵連禍結,當自此始,豈爾國之利乎!」述里感其言,乃降。



 太祖為於越,秉國政,欲命曷魯為迭刺部夷離堇。辭曰:「賊在君側,未敢遠去。」太塚討黑車子室韋,幽州劉仁恭遣養子霸率眾來救。曷魯伏兵桃山,俟霸眾過半而要之;與太祖合擊,斬獲甚
 眾,遂降室韋。太祖會李克用於雲州,時曷魯侍,克用顧而壯之曰:「偉男子為誰?」太祖曰:「吾族曷魯也。」



 會遙輦痕德堇可汗歿,群臣奉遺命請立太祖。太祖辭曰:「昔吾祖夷離堇雅里嘗以不當立而辭,今若等復為是言,何歟?」



 曷魯進曰:「曩吾祖之辭,遺命弗及,符瑞未見,第為國人所推戴耳。今先君言猶在耳,天人所與,若合符契。天不可逆,人不可拂,而君命不可違也。」太祖曰:「遺命田然,汝焉知天道?」曷魯曰:「聞於越之生也,神光屬天,異香盈幄,夢受神誨,龍錫金佩。天道無私,必應有德。我國削弱,齮齕於鄰部日久,以故生聖人以興起之。可汗知天意,故
 有是命。



 且遙輦九營棋布,非無可立者;小大臣民屬心於越,天也。昔者於越伯父釋魯嘗曰:『吾猶蛇,兒猶龍也。』天時人事,幾不可失。」太祖猶未許。是夜,獨召曷魯責曰:「眾以遺命迫我。汝不明吾心,而亦俯隨耶?」曷魯曰:「在昔夷離堇雅裡雖推戴者眾,辭之,而立阻午為可汗。相傳十餘世,君臣之分亂,紀綱之統隳。委質他國,若綴斿然。羽檄逢午,民疲奔命。



 興王之運,實在今日。應天順人,以答顧命,不可失也。」太祖乃許。明日,即皇帝位,命曷魯總軍國事。



 時制度未講,國用未充,扈從未備;用諸弟刺葛等往往凱非望。太祖官行營始置腹心郡,選諸部豪健
 二千餘充之,以曷魯及蕭敵魯總焉。已而諸弟之亂作,太祖命曷魯總領軍事,討平之,以功為迭刺部夷離堇。時民更兵焚剽,日以損敝,曷魯撫輯有方,畜牧益滋,民用富庶。乃討烏古部,破之。自是震懾,不敢復叛。乃請制朝儀、建元,率百官上尊號。太祖既備禮受冊,拜謁備為阿魯敦於越。「阿魯敦」者,遼言盛名也。



 後太祖伐西南諸夷,數為前鋒。神冊二年,從逼幽州,與唐節度使周德威拒戰可汗州西,敗其軍,遂圍幽州,未下。太祖以時暑班師,留曷備與盧國用守之。俄而救兵繼至,曷魯等以軍少無援,退。



 三年七月,皇都既成,燕群臣以落之。曷魯是
 日得疾薨,年四十七。既葬,賜名其阡宴答,山曰於越峪,詔立石紀功。



 清寧間,命立祠上京。



 初,曷魯病革,太祖臨視,問所欲言。曷魯曰:「陛下聖德寬仁,群生咸遂,帝業隆興。臣既蒙寵遇,雖瞑目無憾。惟析迭刺部議未決,願亟行之。」及薨,太祖流涕曰:「斯人若登三五載,吾謀蔑不濟矣!」



 後太祖二十一功臣,各有所擬,以曷魯為心雲。子惕刺、撒刺,俱不仕。



 論曰:「葛魯以肺腑之親,任帷幄之寄,言如蓍龜,謀成戰性,可謂算無遺策矣。其君臣相得之誠,庶吳漢之於光武歟?



 夫信其所可信,智也,太祖有焉。故曰,惟聖知聖,惟
 賢知賢,斯近之矣。」蕭敵魯,字敵輦,其母為德祖女弟,而淳欽皇后又其女兄也。五世祖曰胡母裡,遙輦氏時嘗使唐,唐留之幽州。一夕,折關遁歸國,由是世為決獄官。



 敵魯性寬厚,膂力絕人,習軍旅事。太祖潛藩,日侍左右,凡征討必與行陣。既即位,敵魯與弟阿古只、耶律釋魯、耶律曷備偕總宿衛。拜敵魯北府宰相,世其官。



 太祖征奚及討劉守光,敵備略地海濱,殺獲甚眾。頃之,刺葛等作亂,潰而北走。敵魯率輕騎追之,兼晝夜行。至榆河,敗其黨,獲刺葛以獻。太祖喜之,錫齎甚渥。後討西南夷,功居諸將先。神冊三
 年十二月卒。



 敵魯有膽略,聞敵所在即馳赴,親冒矢石,前後戰未嘗少衄,必勝乃止。以故在太祖功臣列,喻以手云。弟阿古只。



 阿古只,字撒本。少卓越,自放不羈。長驍勇善射,臨敵敢前。每射甲楯輒洞貴。太祖為於越時,以材勇充任使。既即位,與敵魯總腹心部。刺葛之亂也,淳欽皇後軍黑山,阻險自固。太祖方經略奚地,命阿古只統百騎往衛之。逆黨迭裡特、耶律滑哥索憚其勇略,相戒曰:「是不可犯也!」刺葛既北走,與敵備追擒於榆河。



 神冊初元,討西南夷有功;徇山西諸郡縣,又下之,敗周德威軍。三年,以功
 拜北府宰相,世其職。天贊初,與王鬱略地燕、趙,破磁羔鎮。太祖西征,悉誘以南面邊事。



 攻渤海,破扶餘城,獨將騎兵五百,敗老相軍三萬。渤海既平,改東丹國。頃之,已降郡縣復叛,盜賊蜂起。阿古只與康默記討之,所向披靡。會賊游騎七千自鴨綠府來援,勢張甚。



 阿古只帥魔下精銳,直犯其鋒,一戰克之,斬馘三千餘,遂進軍破回跋城。以病卒。功臣中喻阿古只為耳云。子安團,官至右皮室詳穩。



 耶律斜涅赤,字撒刺,六院部舍利裹古直之族。始字鋒怨,早隸太祖幕下,嘗有疾,賜樽酒飲而愈,遼言酒樽曰「
 撒刺」,故詔易字焉。



 太詛即位,掌腹心部。天贊初,分迭刺部為北、南院,斜涅赤為北院夷離堇。帝西征至流沙,威聲大振,諸夷潰散,乃命斜涅赤撫集之。



 及討渤海,破扶餘城,斜涅赤從太子大元帥率眾夜圍忽汗城,大僄掇降。已而復叛,命諸將分地攻之。詰且,斜涅赤感勵士伍,敢噪登陴,敵震懾,莫敢御,遂破之。



 天顯中卒,年七十,居佐命功臣之一。侄老古、頗德。



 老古,字撒懶,其母淳欽皇后姊也。老古幼養宮掖,既長,沉毅有勇略,隸太祖帳下。



 既即位,屢有戰功。刺葛之亂也,欲乘我不備為掩襲計,紿降。太祖將納之,命老古、耶
 律欲穩嚴號令,勒士卒,控轡以防其變。逆黨知有備,懼而遁。以功授右皮室詳穩,典宿衛。



 太祖侵燕、趙,遇唐兵雲碧店,老古恃勇輕敵,直犯其鋒。



 戰久之,被數創,歸營而卒。太祖深悼惜之,佐命功臣其一也。



 頗德,字兀古鄰。弱冠事太祖。天顯初,為左皮室詳穩,典宿衛,遷南院夷離堇,治有聲。



 石敬瑭破張敬達軍於太原北,時頗德勒兵為援,敬達遁。



 敬瑭追至晉安寨圍之,頗德領輕騎襲潞州,塞其餉道。詔諸將懼,殺敬達以降。會同初,改迭刺部夷離堇為大王,即拜頗德,既而加採訪使。舊制,肅祖以下宗室稱院,德祖宗室號三父房,稱
 橫帳,百官子弟及籍沒人稱著帳。耶律斜的言,橫帳班列,不可與北、南院並。太宗詔在廷議,皆曰然,乃詔橫帳班列居上。頗德奏曰:「臣伏見官制,北、南院大王品在惕隱上。今橫帳始圖爵位之高,願與北、商院參任;茲又恥與同列。夫橫帳與諸族皆臣也,班列奚以異?」帝乃諭百官曰:「朕所不知,卿等不宜面從。」詔仍舊制。其強直不撓如此。



 頗德狀貌秀偉,初太祖見之曰:「是子風骨異常兒,必為國器。」後果然。卒年四十九。



 耶律欲穩,字轄刺干,突呂不部人。



 祖臺押,遙輦時為北邊拽刺。簡獻皇后與諸子之罹難也,嘗倚之以免。太祖
 思其功不忘,又多欲隱嚴重,有濟世志,乃命典司近部,以遏諸族窺覬之想。



 欲穩既見器重,益感奮恩報。太祖始置宮分以自衛,欲穩率門客首附宮籍。帝益嘉其忠,詔以臺押配享廟廷。及平刺葛等亂,以功遷奚迭刺部夷離堇。從征渤海有功。天顯初卒。



 後諸帝以太祖之與欲穩也為故,往往取其子孫為友。官分中稱「八房」,皆其後也。弟霞裏,終奚六部禿里。



 耶律海裏,字涅刺昆,遙輦昭古可汗衣裔。



 太祖傳位,海裡與有力焉。初受命,屬籍比局萌覬覦,而遙輦故族尤觖望。海裏多先帝知人之明,而索服太祖威德,獨歸心
 焉。以故太祖托為耳目,數從征討。既清內亂,始置遙輦敞穩,命海裏領之。



 天顯初,徵渤海,海裏將遙輦幻,破忽汗城。師般,卒。



\end{pinyinscope}