\article{卷三十九志第九 地理志三 中京道}

\begin{pinyinscope}

 中京道大定府,虞為營州,夏屬冀州,周在幽州之分。秦郡天下,是為遼西。漢為新安平縣。漢末步奚居之,幅員千里,多大山深谷,阻險足以自固。魏武北征,縱兵大戰,降者二十餘萬,去之松漠。其後拓拔氏乘遼建牙於此,當饒樂河水之南,溫渝河水之北。唐太宗伐高麗,駐蹕於
 此。部帥蘇文從征有功。



 奚長可度率眾內附,力量饒樂都督府。咸通以後,契丹始大,奚族不敢復抗。太祖建國,舉族臣屬。聖宗嘗過七金山土河之濱,南望雲氣,有郛郭樓閥之狀,因議建都。擇良工於燕、薊,董役二歲,邦郭、宮掖、樓閣、府庫、市肆、廊廡,擬神都之制。統和二十四年,五帳院迸故奚王牙帳地。二十五年,城之,實以漢戶,號曰中京,府曰人定。



 皇城中有祖廟,景宗、承天皇后御容殿。城池湫濕,多鑿井洩之,人以為便。大同驛以待宋使,朝天館待新羅使,來賓館待夏使。有七金山、馬盂山、雙山、松山、土河。



 統州十、縣九:
 大定縣。白鞓故地。以諸國俘戶居之。長興縣。本漢賓從縣。以諸部人居之。



 富庶縣。本漢新安平地。開泰二年析京民置。



 勸農縣。本漢賓從縣地。開泰二年析京民置。



 文定縣。開泰二年析京民置。



 升平縣。開泰二年析京民置。



 歸化縣。本漢柳城縣地。



 神水縣。本漢徒河縣地。開泰二年置。



 金源縣。本唐青山縣境。開泰二年析京民置。



 恩州,懷德軍,下,刺史。本漢新安平縣地。太宗建州。



 開
 泰中,以渤海戶實之。初隸永興宮,後屬中京。統縣一:恩化縣。開泰中渤海人戶置。



 惠州,惠和軍,中,刺史。本唐歸義州地。太祖俘漢民數百戶免邈山下,創城居之,置州。屬中京。統縣一:惠和縣。聖宗遷上京惠州民,托諸宮院落帳戶置。



 高州,觀察。唐信州之地。萬歲通天元年,以契丹室活部置。開泰中,聖宗代高麗,以俘戶置高州。有平頂山、灤河。



 屬中京。統縣一:三韓縣。辰韓為扶餘,弁韓為新羅,馬韓為高麗。開泰中,聖宗伐高麗,俘三國之遺人置縣。戶五千。



 武安州,觀察。唐沃州地。太祖俘漢民居木葉山下,因建城以遷之,號杏堝新城。復以遼西戶益之;更曰新州。統和八年改今名。初刺史,後升。有黃柏嶺、裊羅水、個沒里水。屬中京。統縣一:沃野縣。



 利州,中,觀察。本中京阜俗縣。統和二十六年置刺史州,開泰元年升。屬中京。統縣一:阜俗縣。唐末,契丹漸熾,役使奚人,遷居琵琶川。統和四年置縣。初隸彰愍宮,更隸中京。後置州,仍屬中京。



 榆州,高平軍,下,刺史。本漢臨渝縣地,後隸有北平驪城縣。唐載初二年,析慎州置黎州,處靺鞨部落,後為奚人所據。太宗南征,橫帳解里以所俘鎮州民置州。開泰中沒人。屬中京。統縣二:和眾縣。本新黎縣地。



 永和縣。本漢昌城縣地。統和二十二年置。



 澤州,廣濟軍,下,刺史。本漢土垠縣地。太祖俘蔚州民,立寨居之,採煉陷河銀冶。隸中京留守司。開泰中置澤州。有松亭關、神山、九宮嶺、石子嶺、灤河、撒河。屬中京。統縣二:
 神山縣。神山在西南。



 灤河縣。本漢徐無縣地。屬永興宮。



 北安州,興化軍,上,刺史。本漢女祁縣地,屬上谷郡。



 晉為馮跋所據。唐為奚王府西省地。聖宗以漢戶置北安州。屬中京。統縣一:興化縣。本漢且居縣地。



 潭州,廣潤軍,下,刺史。本中京之龍山縣,開泰中置州,仍屬中京。統縣一:龍山縣。本漢交黎縣地。開泰二年以習家寨置。



 松山州,勝安軍,下,刺史。開泰中置。統和八年省,復置。
 屬中京。統縣一:松山縣。本漢文成縣地。邊松漠,商賈會沖。開泰二年置縣。有松山川。



 宋王曾《上契丹事》曰:出燕京北門,至望京館。五十里至順州。七十里至檀州,漸入山。五十里至金溝館。將至館,川原平曠,謂之金溝澱。自此入山,詰曲登陟,無復裡喉,但以馬行記日,約其里數。九十里至古北口,兩傍峻崖,僅客車軌。又度德勝嶺,盤道數層,俗名思鄉嶺,八十里至新館。過雕窠嶺、偏槍嶺,四十里至臥如來館。過烏灤河,東有灤州,又過摸斗嶺,一名渡雲嶺,芹菜嶺,七十里至柳河
 館。松亭嶺甚險峻,七十里至打造部落館。東南行五十里至牛山館。八十里至鹿兒峽館。過暇蟆嶺,九十里至鐵漿館。過石子嶺,自此漸出山,七十里至富穀館。八十里至通天館。二十里至中京大定府。城垣卑小,方圓才四里許。門但重屋,無築纛之制。南門曰朱夏,門內通步廊,多坊門。又有市樓四:曰天方、大衢、通韅、望闕。次至大同館。其門正北日陽德、閶闔。城內西雨隅岡上有寺。城南有園圃,宴射之所。自過古北口,居人草庵板屋,耕種,但無桑柘;所種皆從壟上,虞吹沙所壅。山中長松鬱然,深谷中時見畜牧牛馬橐駝,多青羊黃家。



 成州,興府軍,節度。晉國長公主以媵戶置,軍曰長慶,隸上京。復改軍名。統縣一:同昌縣。



 興中府。本霸州彰武軍,節度。古孤竹國。漢柳城縣地。



 慕容邈以柳城之北,龍山之商,福德之地,乃築龍城,構宮廟,改柳城為龍城縣,遂遷都,號曰和龍宮。慕容垂復居焉,後為馮跋所滅。元魏取為遼西邵。隨平高保寧,置營州。煬帝廢州置柳城郡。唐武德初,改營州總管府,尋為都督府。萬歲通天中,陷李萬榮。神龍初,移府幽州。開元四年復置柳城。八年西徒漁陽。十年還柳城。後為奚所
 據。太祖平奚及俘燕民,將建城,命韓知方擇其處。乃完葺柳城,號霸州彰武軍,節度。



 統和中,制置建、霸、宜、錦、白川等五州。尋落制置,隸積慶宮。後屬興聖宮。重熙十年升興中府。有大華山、小華山、香高山、麝香崖——天授皇帝刻石在焉、駐龍峪、神射泉、小靈河。統州二、縣四:興中縣。本漢柳城縣地。太祖掠漢民居此,建霸城縣。重熙中置府,更名。



 營丘縣。析霸城置。



 象雷縣。開泰二年以麥務川置。初隸中京,後屬。



 閭山縣。本漢且慮縣。開泰二年以羅家軍置。隸中京,
 後屬。



 安德州,化平軍,下,刺史。以霸州安德縣置,來屬。統縣一:安德縣。統和八年析霸城東南龍山徒河境戶置。初隸乾州,更屬霸州,置州來屬。



 黔州,阜昌軍,下,刺史。本漢遼西郡地。太祖平渤海,以所俘戶居之,隸黑水河提轄司。安帝置州,析宜、霸二州漢戶益之,初隸永興宮,更隸中京,後置府,來屬。統縣一:盛吉縣。太祖平渤海,俘興州盛吉縣民來居,因置
 縣。



 宜州,崇義軍,上,節度。本遼纍縣地。東丹王每秋畋於此。興宗以定州俘戶建州。有墳山,松柏連互百餘里,禁礁採;凌河,累石為堤。隸積慶宮。統縣二:弘政縣。世宗以定州俘戶置。民工織紝,多技巧。



 聞義縣。世宗置。初隸海北州,後來屬。



 錦州,臨海軍,中,節度。本漢遼東無慮縣。慕容皝置西樂縣。太祖以漢俘建州。有大胡僧山、小胡僧山、大查牙山、小查牙出、淘河島。隸弘義宮。統州一、縣二:永樂縣。



 安昌縣。



 巖州,保肅軍,下,刺史。本漢海陽縣地。太祖平渤海,遷漢戶雜居興州境,聖宗於此建城焉。隸弘義宮,來屬。統縣一:興城縣。



 川州,長寧軍,中,節度。本唐青山州地。太祖弟明王安端置。會同三年,沼為白川州。安端子察割以大逆誅,沒入,省曰川州。初隸崇德宮,統和中屬文忠王府。統縣三:弘理縣。統和八年以諸宮提轄司戶置。



 咸康縣。



 宜民縣。統和中置。



 建州,保靜軍,上,節度。唐武德中,置昌樂縣。太祖完葺故壘,置州。漢乾元年,故石晉太后詣世宗,求於漢城側耕墾自贍。許於建州南四十時給地五十頃,營構房室,創立宗廟。州在靈河之南,屢遭水害,聖宗遷於河北唐崇州故城。初名武寧軍,隸永興宮,後屬敦睦宮。統縣二:永霸縣。



 永康縣。本唐昌黎縣地。



 來州,歸德軍,下,節度。聖宗以文直五部歲饑來歸,置州居之。初刺史,後升。隸永興宮。有三州山、六州山、五脂山。
 統州二、縣一:來賓縣。本唐來遠縣地。



 隰州,平海軍,下,刺史。慕容皝置集寧縣。聖宗括帳戶遷信州,大雪不能進,建城於此,置焉。隸興聖宮,來屬。統縣一:海濱縣。本漢縣。瀕海,地多堿鹵,置鹽場於此。遷州,興善軍,下,刺史。本漢陽樂縣地。聖宗平大延琳,遷歸州民置,來屬。有箭牱山。統縣一:遷民縣。



 潤州,海陽軍,下,刺史。聖宗平大延琳,遼寧州之民居
 此,置州。統縣一:海陽縣。本漢陽樂縣地,遷潤州,本東京城內渤海民戶,因叛移於此。



\end{pinyinscope}