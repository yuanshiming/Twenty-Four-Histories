\article{卷四十三志第十三 歷象志中 閏考}

\begin{pinyinscope}

 月度不足,是生朔虛;天行有餘,是為氣盈。盈虛相懸,歲月乃牂。積牂而差,寒暑互易,百穀不成,庶政不明。聖人驗以斗柄,準以歲星,爰立閏法,信治百官。是故閏正而月正,月月正而歲正。歲月既正,頒令考績,無有不時。國史正歲年以敘事,莫重於此。



 遼始征歷梁、唐。入晉之後,奄有帝制,《乙未》、《大明》,歷法再變。穆宗應歷六
 年,周用顯《德欽天歷》;十年,
 宋用建隆《應天歷》。景宗乾亨四年,宋用《乾元歷》。聖宗統和十九年,宋用《儀天歷》;太平元年,宋用《崇天歷》。道宗清寧十年,宋用《明天歷》;大康元年,宋用《奉元歷》;大安七年,宋用《觀天歷》。天祚皇帝乾統六年,樣用《紀元歷》。五代歷三變,宋凡人變,遼終始再變。歷法不齊,故定朔置閏,時有不同,覽者惑焉。作《閏考》。



\end{pinyinscope}