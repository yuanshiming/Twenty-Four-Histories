\article{卷五十一志第二十 禮志三軍儀 禮志四}

\begin{pinyinscope}

 賓儀軍儀皇帝親征儀,常以秋冬,應敵制變或無時。將出師,必先告廟。乃立三神主祭之:曰先帝,曰道路,曰軍旅。刑青牛白馬以祭天地。其祭,常依獨樹;無獨樹;即所舍而行之。或皇帝服介胄,祭諸先帝宮廟,乃閱兵。將行,牝牡齗各一為颸祭。



 將臨敵,結馬尾,祈拜天地而後入。
 下城克敵,祭天地,牲以白黑羊。班師,以所獄牡馬、牛各一祭天地。出師以死囚,還師以一諜者,植柱縛其上,於所向之方亂射之,矢集如玹,謂之「射鬼箭」。



 臘儀:臘,十二月辰日。前期一日,詔司獵官選獵地。其日,皇帝、皇后焚香拜日畢,設圍,命獵夫張左右翼。司獵官奏成列,皇帝、皇后升輿,敵烈麻都以酒二尊、盤飧奉進,北南院大王以下進馬及衣。皇帝降輿,祭東畢,乘馬入圍中。皇太子、親王北群官進酒,分兩翼而行。皇帝始獲兔,群臣進酒上壽,各賜以酒。至中食之次,親王、大臣各進所獲。及酒訖,賜群臣飲,還宮。應歷元
 年冬,漢遣使來賀,自是遂以為常儀。



 統和中,罷之。



 出軍儀:制見《兵志》。



 賓儀常朝起居儀,昧爽,臣僚朝服入朝,各依慕次。內侍奏「班齊」。先引京官班於三門外,當直舍人放起居,再拜,各祗候。次依兩府以下文武官,於丹墀內面殿立,豎班諸司並供奉官,於東西道外相向立定。當直閣使副贊放起居,再拜,各祗候。退還幕次,公服。帝升殿坐,兩府並京官丹墀內聲喏,各祗候。教坊司同北班起居
 畢,奏事。



 燕京嘉寧殿,西京同文殿。朝服,鐻頭、袍笏;公服,紫衫,帽。



 正座儀:皇帝升殿坐,警聲絕。契丹、漢人殿前班畢,各依位侍立。次教坊班畢,卷退。京官班入拜畢,揖於右橫街西,依位班立。次武班入拜畢,依位立。文班入拜畢,依位立。北班入,起居畢,於左橫街東,序班立。次兩府班入,鞠躬,通宰臣某官已下起居,拜畢,引上殿奏事。



 已上六班起居,並七拜。內有不帶節度使,班首止通名,亦七拜。卷班,與常朝同。直院有旨入文班。留
 守司、三司、統軍司、制置司謂之京官;都部署司、宮使、副宮使,都承以下令史,北面主事以下隨駕諸司為武官;館、閣、大理寺,堂後以下,御史臺,隨駕閑員、令史、司天臺、翰林、醫官院為文官。天慶二年冬,教坊並服袍。



 臣僚接見儀:皇帝御座,奏見榜子畢,臣僚左入,鞠躬。



 通文武百僚宰臣某官以下祗候見。引面殿鞠躬,起居,凡七拜。



 引班首出班,謝面天顏,復位。舞蹈,五拜,鞠躬。宣答問制,再拜。宣訖,謝宣諭,五拜。各祗候畢,可矮墩以上引近前,問「聖躬萬福」。傳宣問「跋涉不易」,鞠躬。引
 班舍人贊各祗候畢,引右上,準備宣問。其餘臣僚並於右侍立。



 宣答云:「卿等久居鄉邑,來奉乘輿。時屬霜寒——或云炎蒸,諒多勞止。卿各平安好。想宜知悉。」



 問聖體儀:皇帝行幸,車駕至捺缽,坐御帳。臣僚公服,問「聖躬萬福」。贊再拜,各祗候。奏事。宣徽以下常服,教坊與臣僚同。



 保大元年夏,特旨通名再拜,不稱宰臣。



 車駕還京儀:前期一日,宣徽以下橫班,諸司、閣門並公服,於宿帳祗候。至日詰旦,皇帝乘玉輅,閣門宜諭軍
 民訖,導駕。時相以下進至內門,閣副勘箭畢,通事舍人鞠躬,奏「臣宣放仗」。禮畢。



 勘箭儀:皇帝乘玉輅,至內門。北南臣僚於輅前對班立。



 勘箭官執雌箭,門中立。東上閣門使詣車前,執雄箭左車左立,勾勘箭官進。勘箭官揖進,至車約五步,面車立。閣使言「受箭行勘」。勘箭官拜跪,受箭;舉手勘訖,鞠躬,奏「內外勘同」。閣使言「準敕行勘」。勘箭官平立,退至門中舊位立,當胸執箭,贊「軍將門仗官近前」。門仗官應聲開門,舉聲兩邊齊出,並列左右,立。勘箭官舉右手贊「呈箭」,次贊「內出喚仗禦箭一雙,準敕付左金
 吾仗行勘」。贊「合不合」,應「合、合、合」贊「同不同」,應「同、同、同」訖。勘箭官再進,依位立,鞠躬,自通全銜臣某對御勘箭同,退門中立。



 贊「其箭謹付閣門使進入。」事畢,其箭授閣使,轉付宣徽。



 宋使見皇太后儀:宋使賀生辰、正旦。至日,臣僚昧爽入朝,使者至幕次。臣僚班齊,皇太后御殿坐。宣徽使押殿前班起居畢,卷班。次契丹臣僚班起居畢,引應坐臣僚上殿,就位立;其餘臣僚不應坐者,退於東面侍立。漢人臣僚東洞門入,面西鞠躬。舍人鞠躬,通某以下起居,凡七拜畢;贊各祗候。



 引應坐臣僚上殿,就位
 立。中書令、大王西階上殿,奏宋使並從人榜子訖,就位立。其餘臣僚不應坐者,退於西面侍立。次引宋使副六人於東洞門入,丹墀內面齊立。閣使自東階下,受書匣,使人捧書者皆跪,閣使搢笏立,受書匣。自東階上殿,欄內鞠躬,奏「封全」訖,授樞密開封。宰臣對皇太后讀訖,引使副六人東階上殿,欄內立。使者揖生辰節大使少前,使者俯伏跪,附起居訖,起,復位立。次引賀皇太后正旦大使,附起居,如前儀。皇太后宣問「南朝皇帝聖躬萬福」,舍人揖生辰大使並皇太后正旦大使少前,皆跪,唯生辰大使奏「來時聖躬萬
 福」,皆俯伏,興。引東階下殿,丹墀內面殿齊立。引進使引禮物於西洞門入,殿前置擔床。控鶴官起居,四拜,擔床於東便門出畢,揖使副退於東方,西面,皆鞠躬。舍人鞠躬,通南朝國信使某官以下祗候見,舞蹈,五拜畢;不出班,奏「聖躬萬福」,再拜;揖班首出班,謝面天顏訖,復位,舞蹈,五拜畢,贊各上殿祗候,引各使副西階上殿就位。勾從人兩洞門入,面殿鞠躬,通名,贊拜,起居,四拜畢,贊各祗候,分班引兩洞門出。若宣問使副「跋涉不易」,引西階下殿,丹墀內舞蹈,五拜畢,贊各上殿祗候,引西階上殿,就位立。契丹舍人、漢人閣
 使齊贊拜,應坐臣僚並使副皆拜,稱「萬歲」。



 贊各就坐,行湯、行茶。供過人出殿門,揖臣僚並使副起,鞠躬。契丹舍人、漢人閣使齊贊,皆拜,稱「萬歲」。贊各祗候。



 先引宋使副西階下殿,西洞門出,次揖臣僚出畢,報閣門無事。



 皇太后起。



 宋使見皇帝儀:宋使賀生辰、正旦。至日,臣僚昧爽入朝,使者至幕次。奏「班齊」,聲警,皇帝升殿坐。宣徽使押殿前班起居畢,卷班出。契丹臣僚班起居畢,引應坐臣僚上殿,就位立;其餘臣僚不應坐者,並退於北面侍立。次引漢人臣僚北洞門入,面殿鞠躬。舍人鞠躬,通
 某官某以下起居,皆七拜畢,引應坐臣僚上殿,就位立。引首相南階上殿,奏宋使並從人榜子,就位立。臣僚並退於南面侍立。教坊入,起居畢,引南使副北洞門入,丹墀內面殿立。閣使北階下殿,受書匣,使人捧書匣者跪,閣使搢笏立,受於北階。上殿,欄內鞠躬,奏「封全」訖,授樞密開封。宰相對皇帝讀訖,舍人引使副北階上殿,欄內立。揖生辰大使少前,俯伏跪,附起居。俯伏興,復位立。



 大使俯伏跪,奏訖,俯伏興,退;引北階下殿,揖使副北方,南面鞠躬。舍人鞠躬,通南朝國信使某官某以下祗候見,起居,七拜畢;揖班首出班,謝
 面天顏,舞蹈,五拜畢;出班,謝遠接、御筵、撫問、湯藥,舞蹈,五拜畢,贊各祗候。引出,歸幕次。閣使傳宣賜對衣、金帶。勾從人以下入見。舍人贊班首姓名以下,再拜;不出班,奏「聖躬萬福」,贊再拜,稱「萬歲」。贊各祗候。引出。舍人傳宣賜衣。使副並從人服賜衣畢,舍人引使副入,丹墀內面殿鞠躬。舍人贊謝恩,拜,舞蹈,五拜畢,贊上殿祗候。引使副南階上殿,就位立。勾從人入,贊謝恩,拜,稱「萬歲」。贊「有敕賜宴」,再拜,稱「萬歲」。



 贊各祗候。承受官引北廊下立。御床入,大臣進酒,皇帝飲酒。



 契丹合人、漢人閣使齊贊拜,應坐並侍立臣僚皆拜,稱「萬
 歲」。



 贊各祗候。卒飲,贊拜,應坐臣僚皆拜,稱「萬歲」。贊各就坐行酒,親王、使相、使副共樂曲。若宣令飲盡,並起立飲訖。放盞,就位謝。贊拜,並隨拜,稱「萬歲」。贊各就坐。



 次行方茵地坐臣僚等官酒。若宣令飲盡,賀謝如初。殿上酒一行畢,贊廊下從人拜,稱「萬歲」。贊各就坐。若傳宣令飲盡,並拜,稱「萬歲」。贊各就坐。殿上酒三行,行茶、行希、行膳。酒五行,候曲終,揖廊下從人起,贊拜,稱「萬歲」。贊各祗候,引出。曲破,臣僚並使副並起,鞠躬。贊拜,應會臣僚並使副皆拜,稱「萬歲」。贊各祗候。引使副南階下殿,丹墀內舞蹈,五拜畢,贊備祗候。引出。次引
 眾臣僚下殿出畢,報閣門無事。皇帝起,聲蹕。



 曲宴宋使儀,昧爽,臣僚入朝,宋使至幕次。皇帝升殿,殿前、教坊、契丹文武班,皆如初見之儀。宋使副綴翰林學士班,東洞門入,面西鞠躬。舍人鞠躬,通文武百僚臣某以下起居,七拜。謝宣召赴宴,致詞訖,舞蹈,五拜畢,贊各上殿祗候。舍人引大臣、使相、臣僚、使副及方茵朵殿應坐臣僚並於西階上殿,就位立;其餘不應坐臣僚並於西洞門出。勾從人入,起居,謝賜宴,兩廊立,如初見之儀。二人監盞,教坊再拜,贊各上殿祗候。入御床,大臣進酒。舍人、閣使贊拜、行酒,皆如初見之
 儀。次行方茵朵殿臣僚酒,傳宣飲盡,如常儀。殿上酒一行畢,兩廊從人行酒如初。殿上行餅茶畢,教坊致語,揖臣僚、使副並廊下從人皆起立,候口號絕,揖臣僚等皆鞠躬。



 贊拜,殿上應坐並侍立臣僚皆拜,稱「萬歲」。贊各就坐。次贊廊下從人拜,亦如之。歇宴,揖臣僚起立,御床出,皇帝起,入閣。引臣僚東西階下殿,還幕次內賜花。承受官引從人出,賜花,亦如之。簪花畢,引從人復兩廊位立。次引臣僚、使副兩洞門入,復殿上位立。皇帝出閣,復生。御床入,揖應坐臣僚、使副及侍立臣僚鞠躬。贊拜,稱「萬歲」,贊各就坐,贊兩廊從人,亦
 如之。行單茶,行酒,行膳,行果。殿上酒九行,使相樂曲。聲絕,揖兩廊從人起,贊拜,稱「萬歲」,贊「各好去」,承受引出。曲破,殿上臣僚、使副皆起立,贊拜,稱「萬歲」。贊備祗候。引臣僚使副東西階下殿。契丹班謝宴出,漢人並使副班謝宴,舞蹈,五拜畢,贊「各好去。」引出畢,報閣門無事。後帝起。



 賀生辰正旦宋使朝辭太后儀:臣僚、使副班齊,如曲宴儀。



 皇太后升殿坐,殿前契丹文武起居、上殿畢。宰臣奏宋使副、從人朝辭榜子畢,就位立。舍人引使副北洞門入,面南鞠躬。



 舍人鞠躬,通南朝國信使某官某
 以下祗候辭,再拜;不出班,奏「聖躬萬福」,再拜;出班,戀闕,致詞訖,又再拜。贊各上殿祗候。舍人引南階上殿,就位立。引從人,贊姓名,再拜;奏「聖躬萬福」,再拜,稱「萬歲」。贊「各好去」,引出。



 殿上揖應坐臣僚並使副就位鞠躬。贊拜,稱「萬歲」。贊各就坐。行湯、行茶畢,揖臣僚並南使起立,與應坐臣僚鞠躬。贊拜,稱「萬歲」。贊各祗候,立。引使副六人于欄內拜跪,受書匣畢,直起立,揖少前,鞠躬,受傳答語訖。退。於北階下殿,丹墀內面殿鞠躬。舍人贊「各好去」,引出。臣僚出。



 賀生辰正旦宋使朝辭皇帝儀:臣僚入朝如常儀,宋使
 至幕次。於外賜從人衣物。皇帝升殿,宜徽、契丹文武班起居、上殿,如曲宴儀。中書令奏宋使副並從人朝辭榜子畢,臣僚並於南面侍立。教坊起居畢,舍人引使副六人北洞門入,丹墀北方,面南鞠躬。舍人鞠躬,通商朝國信使某官某以下祗候辭,再拜;起居,戀闕,如辭皇太后儀。贊各祗候,平身立。揖使副鞠躬。



 宣徽贊「有敕」,使副再拜,鞠躬,平身立。宣徽使贊「各賜卿對衣、金帶、匹段、弓箭、鞍馬等,想宜知悉」,使副平身立。揖大使三人少前,俯伏跪,搢笏,閣門使授別錄賜物。過畢,俯起,復位立。揖副使三人受賜,亦如之。贊謝恩,舞
 蹈,五拜。贊上殿祗候,舍人引使副商階上殿,就位立。引從人,贊謝恩,再拜,起居,再拜,贊賜宴,再拜;皆稱「萬歲」。



 贊各祗候,承受引兩廊立。御床入,皇帝飲酒,舍人、閣使贊臣僚、使副拜,稱「萬歲」,皆如曲宴。應坐臣僚拜,稱「萬歲」。就坐、行酒、樂曲,方茵、兩廊皆如之;行肴、行茶、行膳亦如之。行饅頭畢,從人起,如登位使之儀。曲破,臣僚、使副皆起立,拜,稱「萬歲」,如辭太后之儀。使副下殿,舞蹈,五拜。贊各上殿祗候,引北階上殿,欄內立。揖生辰、正旦大使二人少前,齊跪,受書畢,起立,揖磬折受起居畢,退。



 引北階下殿,丹墀內並鞠躬。舍人贊「各
 好去」,引南洞門出。



 次引殿上臣僚南北洞門出畢,報閣門無事。



 高麗使入見儀:臣僚常服,起居,應上殿臣僚殿上序立。



 閣門奏榜子,引高麗使副面殿立。引上露臺拜跪,附奏起居訖,拜,起立。閣門傳宣「王詢安否」使副皆跪,大使奏「臣等來時詢安」。引下殿,面殿立。進奉物入,列置殿前。控鶴官起居畢,引進使鞠躬,通高麗國王詢進奉。宣徽使殿上贊進奉赴庫,馬出,擔床出畢,引使副退,面西鞠躬。舍人鞠躬,通高麗國謝恩進奉使某官某以下祗候見,舞蹈,五拜。不出班,奏「聖躬萬福」,再拜。
 出班,謝面天顏,五拜。出班,謝遠接、湯藥,五拜。贊各祗候。使副私獻入,列置殿前。控鶴官起居,引進使鞠躬,通高麗國謝恩進奉某官某以下進奉。宣徽使殿上贊如初。引使副西階上殿序立。皇帝不入御床,臣僚伴酒。契丹舍人通,漢人閣使贊,再拜,稱「萬歲」,各就坐。酒三行,肴膳二味。若宣令飲盡,就位拜,稱「萬歲」,贊各就坐。肴膳不贊,起,再拜,稱「萬歲」。引下鍛,舞蹈,五拜。贊各祗候。引出,於幕次內別差使臣伴宴。起,宣賜衣物訖,遙謝,五拜畢,歸館。



 曲宴高麗使儀:臣僚入朝,班齊,皇帝升殿。宣徽、教坊、控
 鶴、文武班起居,皆如常儀;謝宣宴,如宋使儀。贊各上殿祗候。契丹臣僚謝宣宴。勾高麗使入,面南鞠躬。舍人鞠躬,通高麗國謝恩進奉使某官某以下起居,謝宣宴,共十二拜。贊各上殿祗候,臣僚、使副就位立。大臣進酒,契丹舍人通,漢人閣使贊,上殿臣僚皆拜。贊各祗候,進酒。大臣復位立,贊應坐臣僚拜,贊各就坐行酒。若宣令飲盡,贊再拜,贊各就坐。



 教坊致語,臣僚皆起立。口號絕,贊再拜,贊各就坐。凡拜,皆稱「萬歲」。曲破,臣僚起,下殿。契丹臣僚謝宴,中書令以下謝宴畢,引使副謝,七拜。贊「各好去」。控鶴官門外祗候,報閣門
 無事。供奉官卷班出。來日問聖體。



 高麗使朝辭儀:臣僚起居、上殿如常儀。閣門奏高麗使朝辭榜子。起居、戀闕,如宋使之儀。贊各上殿抵候,引西階上殿立。契丹舍人贊拜,稱「萬歲」。贊各就坐,中書令以下伴酒三行,肴膳二味,皆如初見之儀。既謝,贊「有敕宴」,五拜。贊「各好去」,引出,於幕次內別差使臣伴宴。畢,賜衣物,跪受,遙謝,五拜。歸館。



 西夏國進奉使朝見儀:臣僚常朝畢,引使者左入,至丹墀,面殿立。引使者上露臺立。揖少前,拜跪,附奏起居訖,俯興,復位,閣使宣問「某安否」,鞠躬聽旨,跪奏「某安」。
 俯伏興,退,復位。引左下,至丹墀,面殿立。禮物右入左出,畢,閣使鞠躬,通某國進奉使姓名候見,共一十七拜。贊祗候,平立。有私獻,過畢,揖使者鞠躬,贊「進奉收訖」。贊祗候,引左上殿,就位立。臣僚、使者齊聲喏。酒三行,引使左下,至丹墀謝宴,五拜。畢,贊「有敕宴」,五拜。祗候,引右出。



 禮畢。於外賜宴,客省伴宴,仍賜衣物。



 西夏使朝辭儀:常朝畢,引使者左入,通某國某使祗候辭,再拜。不出班,起居,再拜。出班,戀闕、致詞,復再拜。賜衣物,謝恩如常儀。若賜宴,五拜。畢,贊,「好去」右出。



\end{pinyinscope}