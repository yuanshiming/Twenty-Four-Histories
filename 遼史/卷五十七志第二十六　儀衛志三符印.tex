\article{卷五十七志第二十六 儀衛志三符印}

\begin{pinyinscope}

 遙輦氏之世,受印於回鶻。至耶瀾可汗請印於唐,武宗始賜「奉國契丹印」。太祖神冊元年,梁幽州刺史來歸,詔賜印綬。是時,太祖受位遙輦十年矣。會同九年,太宗伐晉,末帝表上傳國寶一、金印三,天子符瑞於是歸遼。



 傳國寶,秦始皇作,用藍玉,螭紐,六面,其正文「受命於
 天,既壽永昌」,魚鳥篆,子嬰以上漢離祖。王莽篡漢,平皇后投璽殿階,螭角微玷。獻帝失之,孫堅得於井中,傳至孫權,以歸於魏。魏文帝隸刻肩際曰「大魏受漢傳國之寶」。唐更名「受命寶」。晉亡歸遼。自三國以來,僭偽諸國往往模擬私制,歷代府庫所藏不一,莫辨真偽。聖宗開泰十年,馳驛取石晉所上玉璽於中京。興宗重熙七年,以《有傳國寶者為正統賦》試進士。天祚保大二年,遺傳國璽於桑乾河。



 玉印,太宗破晉北歸,得於汴宮,藏隨駕庫。穆宗應歷二年,詔用太宗舊寶。



 御前寶,金鑄,文曰「御前之寶」,以印臣僚宣命。



 詔書寶,文曰「書詔之寶」,凡書詔批答用之。



 契丹寶,受契丹冊儀,符寶郎捧寶置御坐東。金印三,晉帝所上,其文未詳。



 皇太后寶,制未詳。天顯二年,應天皇太后稱制,群臣上璽綬。冊承天皇太后儀,符寶郎奉寶置皇太后坐右。



 皇后印,文曰「皇后教印」。



 皇太子寶,未詳其制。重熙九年冊皇太子儀,中書令授皇太子寶。



 印
 吏部印,文曰「吏部之印」,銀鑄,以印文官制誥。



 兵部印,文曰「兵部之印」,銀鑄,以印軍職制誥。



 契丹樞密院、契丹諸行軍部署、漢人樞密院、中書省、漢人諸行宮都部署印,並銀鑄。文不過六字以上,以銀朱為色。



 南北王以下內外百司印,並銅鑄,以黃丹為色,諸稅務以赤石為色。



 杓紵,鷙鳥之總名,以為印紐,取疾速之義。行軍詔賜將帥用之。道宗賜耶律仁先鷹紐印,即此。



 符契
 自大賀氏八部用兵,則合契而動,不過刻木為俛合。太祖受命,易以金魚。



 金魚符七枚,黃金鑄,長六寸,各有字號,每全左右判合之。有事,以左半先授守將,使者執右半,大小、長短、字號合同,然後發兵。事訖,歸於內府。



 銀牌二百面,長尺,刻以國字,文曰「宜速」,又曰「敕走馬牌」。國有重事,皇帝以牌親授使者,手札給驛馬若干。



 驛馬闕,取它馬代。法,晝夜馬七百里,其次五百里。所至如天子親臨,須索更易,無敢違者。使回,皇帝親受之,手封牌印郎君收掌。



 木契,正面為陽,背面為陰,閣門喚仗則用之。朝賀之禮,宣微使請陽面木契下殿,至於殿門,以契授西上閣門使云:「授契行勘。」勘契官聲喏,跪受契,舉手勘契同,俯、興,鞠躬,奏「內外勘契同」。閣門使云:「準敕勘契,行勘。」勘契官執陰面木契聲喏。勘契官云:「內出喚仗木契一隻,準敕付左右金吾仗行勘。」勘契官云「合不合」,門仗官云「合」,凡再。勘契官云「同不同」,門仗官云「同」,亦再。勘契官近前鞠躬,奏:「勘官左金吾引駕仗、勾畫都知某官某,對御勘同。」平身,少退近後,右手舉契云:「其契謹付
 閣門使進入。」閣門使引聲喏,門仗官下聲喏。勘契官跪以契授,閣門使上殿納契,宣微使受契。閣門使下殿,奉敕喚仗。



 木箭,內箭為雄,外箭為雌,皇帝行幸則用之。還宮,勘箭官執雌箭,東上閣門使執雄箭,如勘契之儀,詳具《禮儀志》。



\end{pinyinscope}