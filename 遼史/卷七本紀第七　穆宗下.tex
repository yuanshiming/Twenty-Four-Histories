\article{卷七本紀第七 穆宗下}

\begin{pinyinscope}

 十四年春正月戊寅朔,奉安神纛。戊戌,漢以宋將來襲,馳告。二月壬子,詔西南面招討使撻烈進兵援漢。癸亥,如潢河。



 戊辰,支解鹿人沒答、海裡等七人於野,封土識其地。己巳,如老林東濼。壬申,漢以敗宋兵石州來告。



 夏四月本漢以擊退宋軍,遣使來謝。是月,黃龍府某露降。



 五月,射舐堿鹿於白鷹山,至於浹旬。



 六月丙午朔,獵
 於玉山,竟月忘返。



 秋七月壬辰,以酒脯祀黑山。



 八月乙巳,如磑子嶺,呼鹿射之,獲鹿四,賜虞人女環等物有差。丁未,還呂。戊申,以生日值天赦,不受賀,曲赦京師囚。乙卯,錄囚。



 九月,黃室韋叛。



 冬十月丙午,近侍烏古者進石錯,賜白金二百五十兩。丙辰,以掌鹿矧思代斡里為閘撒狘,賜金帶、金盞,銀二百兩。



 所隸死罪以下得專之。



 十一月壬午,日南至,宴飲達旦。自是晝寢夜飲。殺近侍小六於禁中。



 十二月丙午,以黑兔祭神。烏古叛,掠民財畜。詳穩僧隱與戰,敗績,僧隱及乙實等死之。



 十五年春正月乙卯,以樞密使雅里斯為行軍都統,虎
 軍詳穩楚思為行軍都監,益以突呂不部軍三百,合諸部兵討之。烏古夷離堇子勃勒底獨不叛,詔褒之。是月,老人星見。



 二月壬寅朔,日有食之。上東幸。甲寅,以獲鴨,除鷹坊刺面、腰斬之刑,復其徭役。是月,烏古殺其長窣離底,餘眾降,復叛。



 三月癸酉,近侍東兒進匕箸不時,手刃刺之。丁丑,大黃室韋酋長寅尼吉叛。癸未,王坊人四十戶叛入烏古。癸已,虞人沙刺迭偵鵝失期,加炮洛、鐵梳之刑而死。



 夏四月乙巳,小黃室韋叛,雅里斯、楚思等擊之,為室韋所敗,遣使詰之。乙卯,以禿裡代雅里斯為都統,以女古為監軍,率輕騎進討,仍令撻馬尋吉裡持
 詔招諭。



 五月壬申,尋吉裏奏,諭之不從。雅里斯以撻凜、蘇才群牧兵追至柴河,與戰不利。甲申,庫古只奏室韋長寅吉亡入敵烈。



 六月辛亥,俞魯古獻良馬,賜銀二千兩。以近侍忽刺比馬至先以聞,賜銀千兩。是月,敵烈來降。



 秋七月甲戌,雅里斯奏烏古至河德濼,遣夷離堇畫裏、夷離畢常思擊之。丁丑,烏古掠上京北榆林峪,遣林牙蕭幹討之。



 庚辰,雅里斯等與烏古戰,不利。



 冬十月丁未,常思與烏古戰,敗之。



 十二月甲辰,以近侍喜哥私歸,殺其妻。丁未,殺近侍隨魯。駐蹕黑山平澱。



 十六年春正月丁卯朔,被酒,不受賀。甲申,微行市中,賜
 酒家銀絹。乙酉,殺近侍白海及家僕衫福、押刺葛、樞密使門吏老古、撻馬失魯。



 三月己巳,東幸。庚午獲鴨,甲申獲鵝,皆飲達旦。



 五月甲申,以歲旱,泛舟於池禱雨;不雨,舍舟立水中而禱,俄頃乃雨。



 六月丙申,以白海死非其罪,賜其家銀絹。



 秋七月壬午,諭有司:凡行幸之所,必高立標識,令民勿犯,違以死論。



 八月丁酉,漢遣使貢金器、鎧甲。



 閏月乙丑,觀野鹿入馴鹿群,立馬飲至哺。



 九月庚子,以重九宴飲,夜以繼日,至壬子乃罷。己未,殺狼人哀里。



 冬十月庚辰,漢主有母喪,遣使賻吊。



 十二月甲子,幸酒人拔刺哥家,復幸殿前都點檢耶律夷臘葛第,宴飲連
 日。賜金盂、細錦及孕馬百疋,左右授官者甚眾。



 戊辰,漢遣使來貢。



 是冬,駐蹕黑山平澱。



 十七年春正月庚寅朔,林牙蕭干、郎君耶律賢適討烏古還,帝執其手,賜卮酒,授賢適右皮室詳穩。雅里斯、楚思、霞裏三人賜醨酒以辱之。乙卯,夷離畢骨欲獻烏古俘。



 二月甲子,高勛奏宋將城益津關,請以偏師擾之,上從之。



 夏四月戊辰,殺鷹人敵魯。丙子,射柳祈雨,復以水沃群臣。



 五月辛卯,殺鹿人札葛。壬辰,北府宰相蕭海瓈薨,輟朝,罷重五宴。



 六月己未,支解雉人壽哥、念古、殺鹿人四十四人。



 是夏,駐蹕哀潭。秋八月辛酉,生日,以政事
 令阿不底病亟,不受賀。



 九月自丙戌朔,獵於黑山、赤山,至於月終。



 冬十月乙丑,殺酒人粹你。



 十一月辛卯,殺近侍廷壽。壬辰,殺豕人阿不札、曷魯、術里者、涅里括。庚子,司天臺奏月當食不虧,上以為祥,歡飲達旦。壬寅,殺鹿人唐果、直哥、撒刺。



 十二月辛未,手殺饔人海裏,復臠之。



 是冬,駐蹕黑河平澱。



 十八年春正月乙酉朔,宴於宮中,不受賀。己亥,觀燈於市。以銀百兩市酒,命群臣亦市酒,縱飲三夕。



 二月乙卯,幸五坊使霞實里家,宴飲達旦。



 三月甲申朔,如潢河。乙酉,獲駕鵝,祭天地。造大酒器,刻為鹿文,名曰「鹿砙」,貯酒以
 祭天。庚戌,殺鶻人胡特魯、近侍化葛及監囚海裏,仍剉海里之尸。



 夏四月癸丑朔,殺彘屯奴。己未,為殿前都點檢夷臘葛置神帳,曲赦京幾囚。甲戌,撻烈於雕窠中得牝犬來進。



 是夏,清暑哀潭。



 秋七月辛丑,漢主承鈞殂,子繼元立,來告,遣使吊祭。



 九月戊子,
 殺詳穩八刺、拽刺痕篤等四人。己丑,登小賒天地。戊戌,知宋欲襲河東,諭西南面都統、南院大王撻烈豫為之備。己亥,獵熊,以喚鹿人鋪姑並掖庭戶賜夷臘葛。甲辰,以夷臘葛兼政事令,仍以黑山東抹真之地數十里賜之,以女環為近侍,女直詳穩戞陽為本部夷離堇。



 是秋,獵於西京諸山。



 冬十月辛亥朔,宋圍太原,詔撻烈為兵馬總管,發諸道兵救之。十一月癸卯,冬至,被酒,不受賀。十二月丁丑,殺酒人搭烈葛。



 是冬,駐蹕黑山東川。



 十九年春正月己卯朔,宴宮中,不受賀。己丑,立春,被酒,命殿前都點檢夷臘葛代行擊土牛禮。甲午,與群臣為
 葉格戲。戊戌,醉中驟加左右官。乙巳,詔太尉化哥曰:「朕醉中處事有乖,無得曲從。酒解,可覆奏。」自立春飲至月終,不聽政。二月甲寅,漢劉元嗣立,遣使乞封冊。辛酉,遣韓知範冊為皇帝。癸亥,殺前導末及益刺,剉其尸,棄之。甲子,漢遣使進白麃。己巳,如懷州,獵獲熊,歡飲方醉,馳還行宮。是夜,近侍小哥、盥人花哥、庖人辛古等六人反,帝遇弒,年三十九。廟號穆宗。後附葬懷陵。重熙二十一年,謚曰孝安敬正皇帝。贊曰:穆宗在位十八年,知女巫妖妄見誅,諭臣下濫刑切諫,非不明也。而荒耽於酒,畋獵無厭。偵鵝失期,加炮
 烙鐵梳之刑;獲鴨甚歡,除鷹坊刺面之令。賞罰無章,朝政不視,而嗜殺不已。變起肘腋,宜哉!



\end{pinyinscope}