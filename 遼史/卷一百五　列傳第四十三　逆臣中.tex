\article{卷一百五 列傳第四十三 逆臣中}

\begin{pinyinscope}

 蕭翰耶律牒蠟耶律朗耶律劉哥弟盆都耶律海思耶律敵獵蕭革蕭翰,一名敵烈,字寒真,宰相敵魯之子。



 天贊初,唐兵圍鎮州,節度使張文禮遣使告急。翰受詔與康末怛往救,克之,殺其將李嗣昭,拔石城。會同初,領漢軍侍衛。八年,伐晉,敗晉將杜重威,追至望都。翰奏曰:「可令軍下馬而射。」帝從其言,軍士步進。敵人持短兵猝至,我軍失利。帝悔之曰:「此吾用言之過至此!」及從駕入汴,為宣武軍節
 度使。



 會帝崩欒城,世宗即位。翰聞之,委事於李從敏,徑趨行在。是年秋,世宗與皇太后相拒於潢河橫渡,和議未定。太后間翰曰:「汝何怨而叛?」對曰:「臣母無罪,太后殺之,以此不能無憾。」初耶律屋質以附太后被囚,翰聞而快之,即囚所謂曰:「汝嘗言我輩不及,今在狴犴,何也?」對曰:「第願公不至如此!」翰默然。



 天祿二年,尚帝妹阿不里。後與天德謀反,下獄。復結惕隱劉哥及其弟盆都亂,耶律石刺告屋質,屋質遽入奏之,翰等不伏。帝不欲發其事。屋質固凈以為不可,乃詔屋質鞫按。翰伏辜,帝竟釋之。復與公主以書結明王安端反,屋質得其書以奏,翰
 伏誅。



 牒蠟,字述蘭,六院夷離堇浦古只之後。



 天顯中,為中臺省右相。會同元年,與趙思溫持節冊晉帝。



 及我師伐晉,到滹沱河,降晉將杜重威,牒蠟功居多。大同元年,平相州之叛,斬首數萬級。



 世宗即位,遣使馳報,仍命牒蠟執偏將術者以來。其使誤入術者營,術者得詔,反誘牒蠟,執送太后。牒蠟亡歸世宗。



 和約既成,封燕王,為南京留守。



 天祿五年,察割弒逆,牒蠟方醉,其妻扶入察割之幕,因從之。明旦,壽安王討亂,凡脅從者皆棄兵降;牒蠟不降,陵遲而死。妻子皆誅。



 朗,字歐新,季父房罨古只之孫。性輕佻,多力,人呼為「虎斯」。天顯間以材勇進,每戰輒克,由是得名。



 會同九年,太宗大汴,命知澶淵,控扼河渡。天祿元年,燕、趙已南皆應劉知遠,朗與汴守蕭翰棄城歸闕。先是,朗祖罨古只為其弟轄底詐取夷離堇,自是族中無任六院職事者;世宗不悉其事,以朗為六院大王。



 及察割作亂,遭人報朗曰:「事成矣!」朗遣詳穩蕭胡里以所部軍往,命曰:「當持兩端,助其勝者。」穆宗即位,伏誅,籍其家屬。



 劉哥,字明隱,太祖弟寅底石之子。幼驕狠,好陵侮人,長益兇狡。太宗惡之,使守邊徼,累遷酉擊邊大詳穩。



 會同
 十年,叔父安端從帝伐晉,以病先歸,與劉哥鄰居。



 世宗立於軍中,安端議所往,劉哥首建附世宗之策,以本部兵助之。時太后命皇太弟李胡率兵而商,劉哥、安端遇於泰德泉。



 既接戰,安端墜馬。王子天德馳至,欲以槍刺之。劉哥以身衛安端,射天德,貫甲不及膚。安端得馬復戰,太弟兵敗。劉哥與安端朝於行在。及和議成,太后間劉哥曰:「汝何怨而叛?」



 對曰:「臣父無罪,太后殺之,以此怨耳。」事平,以功為惕隱。



 天祿中,與其弟盆都、王子天德、侍衛蕭翰謀反,耶律石刺發其事,劉哥以飾辭免。後謂帝博,欲因進酒弒逆,帝覺之,不果,被囚。一日,召劉哥,鎖項
 以博。帝問:「妝實反耶?」



 劉哥誓曰:「臣若有反心,必生千頂疽死!」遂貰之。耶律屋質固凈,以為罪在不赦。上命屋質按之,具服。詔免死,流烏古部,果以千頂疽死。弟盆都。



 盆都,殘忍多為,膚若蛇皮。天祿初,以族屬為皮室詳穩。



 二年,與兄劉哥謀反,免死,使於轄戛斯國。既還,復預察割之亂,陵遲而死。



 異母弟二人:化葛里、奚寨。應歷初,無職任,以族子,甚見優禮。三年,或告化葛里、奚蹇與衛王宛謀逆,下獄,飾辭獲免。四年春,復謀反,伏誅。



 海思,字鐸袞,隋國王釋魯之庶子。機警口辯。



 會同五年,詔求直言。時海思年十八,衣羊裘,乘牛詣闕。



 有司問曰:「
 汝何故來?」對曰:「應詔言事。茍不以貧稚見遺,亦可備直言之選。」有司以聞。會帝將出獵,使謂曰:「俟吾還則見之。」海思曰:「臣以陛下急於求賢,是以來耳;今反繪於獵,請從此歸。」帝聞,即召見賜坐,問以治道。命明王安端與耶律頗德試之,數日,安端等奏曰:「海思之材,臣等所不及。」帝召海思問曰:「與汝言者何如人也?」對曰:「安端言無收檢,若空車走峻阪;頗德如著靴行曠野射鴇。」



 帝大笑。擢宣徽使,屢任以事。帝知其貧,以金器賜之,海思即散於親友。後從帝伐晉有功。



 世宗即位於軍中,皇太后以兵逆於潢河橫渡。太后遣耶律屋質責世宗自立。屋質至
 帝前,諭旨不屈;世宗遣海思對,亦不遜,且命之曰:「妝見屋質勿懼!」海思見太后還,不稱旨。



 既和,領太后諸局事。



 穆宗即位,與冀王敵烈謀反,死獄中。



 敵獵,字烏輦,六院夷離堇術不魯之子。少多詐。



 世宗即位,為群牧都林牙。察割謀亂,官僚多被囚系。及壽安王與耶律屋質率兵來討,諸黨以次引去。察割度事不成,即詣囚所,持弓矢脅曰:「悉殺此曹!」敵獵進曰:「殺何益於事?竊料屋質將立壽安王,故為此舉,且壽安未必知。若遣人藉此為辭,庶可免。」察割曰:「如公言。誰可使者?」敵獵曰:「大王若不疑,敵獵請與罨撒葛同往說之。」察割遣之。



 壽安王用敵獵計,誘殺察割,凡被脅之人無一被害者,皆敵猜之力。亂既平,帝喜賞,然未顯用。敵獵失望,居常怏怏,結群不逞,陰杯不軌。應歷二年,與其黨謀立婁國,事覺,陵遲死。



 蕭革,小字滑哥,字胡突堇,國舅房林牙和尚之子。警悟多智數。太平初,累遷官職事。游近習間,以諛悅相比暱,為流輩所稱,由是名達於上。重熙初,拜北面林牙。十二年,為北院樞密副使。帝嘗與近臣宴,謂革曰:「朕知卿才,故自拔擢,卿宜勉力!」革曰:「臣不才,誤蒙聖知,無以報萬一;惟竭愚忠,安敢怠?」



 明年,拜北府宰相。十五年,改同知北
 院樞密事。革怙寵專權,同僚具位而已。時夷離畢耶律義先知革奸佞,因侍燕,言革所短,用之將敗事。帝不聽。一日,上令義先對革巡擲,義先酒酣曰:「臣備位大臣,縱不能進忠去佞,安能與賊博乎!」革銜之,佯言曰:「公相謔,不既甚乎!」義先詬詈不已。帝怒,皇後解之曰:「義先酒狂,醒可治也。」翌日,上詔革謂曰:「義先無禮,可痛繩之。」革曰:「義先之才,豈逃聖鑾!然天下皆知忠直。今以酒過為罪,恐弗人望。」帝以革犯而不校,眷遇益厚。其矯情媚上多此類。拜南院樞密使,詔班諸王上,封吳王。改知北院,進王鄭,兼中書令。帝大漸,詔革曰:「大位不可一日曠,朕若
 弗寤,宜即令燕趙國王嗣位。」



 清寧元年,復為南院樞密使,更王楚。復徙北院,與國舅蕭阿刺同掌朝政。革多私撓,阿刺每裁正之,由是有隙,出阿刺為東京留守。會南郊,阿刺以例赴闕,帝訪群臣以時務,阿刺陳利病,言甚激切。革伺帝意不悅,因諧曰:「阿刺恃寵,有慢上心,非臣子禮。」帝大怒,縊阿刺於殿下。



 後上知革奸計,寵遇漸衰。八年,致仕,封鄭國王。九年秋,革以其子為重元婿,革預其謀,陵遲殺之。



\end{pinyinscope}