\article{卷二十五本紀第二十五 道宗五}

\begin{pinyinscope}

 三年春正月乙卯,如魚兒濼。甲戌,出錢粟振南京貧民,仍復其租賦。己卯,大雪。



 三月丙戌,發粟振中京饑。甲辰,以民多流散,除安泊逃戶徵償法。



 三月乙卯,高麗遣使來貢。己未,免錦州貧民租一年。甲戌,免上京貧民租如錦州。庚辰,女直貢良馬。



 夏四月戊子,賜中京貧民帛,及免諸路貢輸之半。丙申,賜隈烏古部貧民帛。庚子,如涼
 陘。甲辰,南府宰相王績薨。



 乙巳,詔出戶部司粟,振諸路流民及義州之饑。



 五月庚申,海雲寺進濟民錢千萬。



 秋七月丙辰,獵黑嶺。丁巳,出雜帛賜興聖宮貧民。庚午,大雨,罷猜。丁丑,秦越國王阿鏈薨。



 九月乙亥,駐蹕匣魯金。



 冬十月庚辰,以參知政事王經為三司使。壬辰,罷節度使已下官進珍玩。癸卯,追封秦越國王阿鏈為秦魏國王。



 十一月甲寅,以惕隱耶律坦同知南京留守事,遼興軍節度使耶律王九為南府宰相。十二月乙卯朔,以樞密直學士呂嗣立參知政事。



 四年春正月庚戌,如混同江。甲寅,太白晝見。甲子,五國
 部長來貢。庚午,兔上京通逃及貧戶稅賦。甲戌,以上京、南京饑,許良人自鬻。丁丑,曲赦西京役徒。



 二月己丑,如魚兒濼。甲午,曲赦春州役徒,終身著皆五歲免。己亥,如春州。赦泰州役徒。



 三月乙丑,免高麗歲貢。己巳,振上京及平、錦、來三州饑。



 夏四月己卯。振蘇、吉、復、淥、鐵五州貧民、並免其租稅。甲申,振慶州貧民。乙酉,減諸路常貢服御物。丁酉,立入粟補官法。癸卯,西幸。召樞密直學士耶律伊講《尚書洪範》。



 五月辛亥,命燕國王延禧寫《尚書五子之歌》。乙卯,振祖州貧民。丁巳,詔免役徒,終身著五歲免之。己未,振春州貧民。丙寅,禁挾私引水犯田。



 六月庚
 辰,駐蹕散水原。丁亥,命燕國王延禧知中丞司事,以同知南院樞密使事耶律聶裡知右夷離畢,知右夷離畢事耶律那也同知南院樞密使事。庚寅,北院樞密使耶律頗德致仕。



 秋七月戊申,曲赦奉聖州役徒。丙辰,遣使冊李乾順為夏國王。庚申,如秋山。己巳,禁錢出境。



 八月庚辰,有司奏宛平、永清蝗為飛鳥所食。庚寅,謁慶陵。



 冬十月丁丑,獵遼水之濱。己卯,駐蹕藕絲澱。癸未,免百姓所貸官粟。己丑,知北院樞密使事耶律阿思封漆水郡王。



 癸巳,以乙室大王耶律敵烈知西北路招討使事,權知西北路招討使事蕭朽哥知乙室大王事。壬寅,詔諸
 部長官親鞠獄訟。



 十一月庚申,興中府民張化法以父兄犯盜當死,請代,皆免。十二月戊寅,南府宰相耶律王九致仕。癸未,以孟父敞穩耶律慎思為中京留守。



 閏十二月癸卯朔,預行正且禮。丙午,如混同江。



 五年春正月癸未,如魚兒濼。甲午,高麗遣使來貢。



 三月癸酉,詔析津、大定二府精選舉人以聞,仍詔諭學者,當窮經明道。



 夏四月甲辰,以知奚六部大王事涅葛為本部大王。壬子,獵北山。甲子,霖雨,罷獵。



 五月丁亥,駐蹕赤勒嶺。己丑,以阻卜磨古斯為諸部長。



 癸巳,回鶻遣使貢良馬。己亥,以同知南院樞密使事耶律那也知右夷離
 畢事,左祗候郎君班詳穩耶律涅裡知北院大王事。



 六月甲寅,夏國遣使來謝封冊。壬戌,以參知政事王言敷為樞密副使,前樞密副使賈士勛參知政事,兼同知樞密院事。



 秋七月庚午,獵沙嶺。



 九月辛卯,遣使遺宋鹿脯。壬辰,駐蹕藕絲澱。



 冬十月乙巳,以新定法令太煩,復行舊法。庚申,以遼興軍節度使何葛為乙室大王。



 十一月丁卯朔,燕國王延禧生子,大赦,妮之族屬進爵有差。



 六年春五月,如混同江。



 二月辛丑,駐蹕雙山。



 三月辛未,女直遣使來貢。



 夏四月丁酉,東北路統軍司設掌法官。庚子,以同知南院樞密使事耶律吐朵知左夷離畢事。



 五月壬辰,駐蹕散水原。



 六月甲寅,遣使決五京囚。



 秋七月丙子,如黑嶺。冬十月丁酉,駐蹕藕絲澱。



 十一月壬戌,高麗遭使來貢。已巳,以南府宰相竇景庸為武定軍節度使。



 是年,放進士文充等七十二人。



 七年春正月壬戌,如混同江。



 二月己亥,駐蹕魚兒濼。壬寅,詔給渭州貧民耕牛、布絹。



 三月丙戌,駐蹕黑龍江。



 夏四月丙辰,以漢人行宮副部署耶律穀欲知乙室大王事。



 五月己未朔,日有食之。



 六月甲午,駐蹕赤勒嶺。己亥,倒塌嶺人進古鼎,有文曰:「萬歲永為實用。」辛丑,回鶻遣使貢方物。癸卯,以權知東京留守蕭陶隗為丹行宮
 都部署。丁未,端拱殿門災。



 秋七月戊午朔,回鶻遣使來貢異物,不納,厚賜遭之。



 八月庚寅,以霖雨,罷獵。壬寅,幸慶州,謁慶陵。



 九月丙申,還上京。己亥,日本國遣鄭元、鄭心及僧應範等二十八人來貢。



 冬十月辛己,命燕國王延禧為天下兵馬大元帥,總北南院樞密使事。



 十一月庚子,如藕絲澱。甲子,望杞木葉山。



 八年春正月乙酉,如山榆澱。乙未,阻卜諸長來降。



 三月己亥,駐蹕撻里舍澱。丁未,曲赦中京、蔚州役徒。



 夏四月乙卯,阻卜長來貢。丁丑,獵西山。惕德酋長胡裡只來附。



 五月甲辰,駐蹕赤勒嶺。



 六月乙丑,夏國為宋侵,遣使乞
 援。



 秋七月丁亥,獵沙嶺。



 九月乙巳,駐蹕藕絲澱。丁未,日本國遣使來貢。冬十月康戌朔,遣使遺宋鹿脯。丙辰,振酉北路饑。辛酉,阻卜磨古斯殺金吾吐古斯以叛,遣奚六部禿里耶律郭三發諸蕃部兵討之。壬申,南府宰相王經薨。戊寅,以左夷離畢耶律涅里為彰聖軍節度使。



 十一月戊子,以樞密副使王是敦兼知樞密院事,權參知政事韓資讓參知政事,漢人行宮都部署奚回離保知英六部大王事。丁酉,以通州潦水害稼,遣使振之。戊申,北院大王合魯薨。



 是年,放進士冠尊文等五十三人。



 九年春正月庚辰,如混同江。



 二月,磨古斯來侵。



 三月,西
 北路招討使耶律阿魯掃古追磨古斯還,都監蕭張九遇賊,與戰不利。二室韋、拽刺、北王府、特滿群牧、宮分等軍多隱沒。



 夏四月乙卯,興中府甘露降,遣使祠佛飯僧。癸酉,獵西山。



 六月丁未朔,駐蹕散水原。庚申,以遼興軍節度使榮哥為南院大王,知左夷離畢事耶律吐朵為左夷離畢。



 秋七月辛卯,如黑嶺。壬寅,遣使賜賜高麗羊。



 九月癸卯,振西北路貧民。



 冬十月庚戌,有司奏磨古斯詣西北路招討使耶律撻不也偽降,既而乘虛來襲,撻不也死之。阻卜烏古札叛,達裏底、拔思母並寇倒塌嶺。壬子,遣使籍諸路兵。癸丑,以南院大王特末同知南京留
 守事,命鄭家奴率兵往援倒塌嶺。甲寅,駐蹕藉絲澱,以左夷離畢耶律禿朵、圍場都管撒八並為西北路行軍都監。乙卯,詔以馬三千給烏古部。丙辰,有司奏阻卜長轄底掠酉路群牧。丁巳,振西北路貧民。己未,燕國王延禧生子,肆赦,妮之族屬並進級。壬戌,以樞密直學士趙廷睦參知政事兼同知南院樞密使事。癸亥,烏古敵烈統軍使蕭朽哥奏討阻卜等部捷。甲子,宋遣使告其母後曹氏哀,即遣使吊祭。己巳,詔廣積貯,以備水旱。



 十一月辛巳,特抹等奏討阻卜捷。



 十二月丙辰,宋遣使以母後遺留物來饋。



 十年春正月,如春水。癸未,惕德來貢。戊子,烏古扎等來降,達裏底、拔思母二部來侵,四捷軍都監特抹死之。



 二月甲辰,以破阻卜,賞有功者。丙午,西南面招討司奏討拔思母捷。癸丑,排稚、僕里、同葛、虎骨、僕果等來降。



 達裏底來侵。



 三月壬申朔,日有食之。山北路副部署蕭阿魯帶奏討達裏底捷。夏四月壬寅朔,惕德萌得斯、老古得等各率所部來附,詔復舊地。甲辰,駐蹕春州北平澱。丙午,烏古部節度使耶律陳家奴奏討茶扎刺捷。庚戌,以知北院樞密使事耶律斡特刺為都統,夷離畢耶律禿朵為副統,龍虎衛上將軍耶律胡呂都監,討磨古斯,遣
 積慶宮使蕭糺里監戰。辛亥,朽哥奏頗里八部來侵。



 擊破之。己巳,除玉田、密雲流民租賦一年。



 閏月庚子,賜西北路貧民錢。達裏底、拔思母二部來降。



 五月甲辰,駐蹕赤勒嶺。甲寅,括馬。戊午,西北路招討司奏敵烈等部來侵,統軍司出兵與戰,不利,招討司以兵擊破之,敦睦宮太師耶律愛奴及其子死之。辛酉,以知國舅詳穩事蕭阿烈同領西北路行軍事。



 六月辛未,宋遣使來謝吊祭。乙酉,烏古敵烈統軍使朽哥有罪,除名。丙戌,和烈葛等部來聘。癸巳,惕德來貢。乙亥,禁邊民與蓄部為婚。是夏,高麗國王運薨,子昱遣使來告,即遣使賻贈。



 秋七月庚
 子朔,獵赤山。是月,阻卜等寇倒塌嶺,盡掠西路群牧馬去,東北路統軍使耶律石柳以兵追及,盡獲所掠而還。



 九月乙未,以南院大王特末為南院樞密使。甲子,敵烈諸酋來降,釋其罪。是月,斡特刺破磨古斯。



 冬十月丙子,駐蹕藉絲澱。壬午,山北路副部署蕭阿魯帶以討達裏底功,加左金吾衛上將軍,癸巳,西北路統軍司獲阻卜拍撒葛、蒲魯等來獻。



 十一月乙巳,惕德銅刮、阻卜的烈等來降。達裏底及拔思母等復來侵,山北副部署阿魯帶擊敗之。



 十二月癸酉,三河縣民孫賓及其妻皆百歲,復其家。甲戌,以參知政事趙廷睦兼同知樞密院事,
 樞密副使王師儒參知政兼同知樞密院事。己卯,詔錄西北路有功將士及戰歿者,贈。乙酋,詔改明年元,減雜犯死罪以下,仍除貧民租賦。戊子,西北路統軍司奏討磨古斯捷。



\end{pinyinscope}