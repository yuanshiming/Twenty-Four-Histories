\article{卷六十三列傳第一 後妃}

\begin{pinyinscope}

 肅祖昭烈皇后蕭氏懿祖莊敬皇后蕭氏玄祖簡獻皇后蕭氏德祖宣簡皇后蕭氏太祖淳欽皇后述律氏
 太宗靖安皇后蕭氏世宗懷節皇后蕭氏世宗妃甄氏穆宗皇后蕭氏景宗睿智皇后蕭氏聖宗仁德皇后蕭氏聖宗欽哀皇后蕭氏興宗仁懿皇后蕭氏興宗貴妃蕭氏道宗宣懿皇后蕭氏道宗惠妃蕭氏
 天祚皇后蕭氏天祚德妃蕭氏天祚文妃蕭氏天祚元妃蕭氏《書》始嬪虞,《詩》興《關睢》。國史記載,往往自家而國,以立天下之本。然尊卑之分,不可易也。司馬遷列呂后於《紀》;班固因之,而傳元后於外戚之後;範曄登后妃於《帝紀》。天子紀年以敘事謂之《紀》,後曷為而紀之?自晉史列諸後以首《傳》,隋、唐以來,莫之能易也。



 遼因突厥,稱皇后曰「可敦」,國語謂之「忒俚塞」,尊稱曰「褥斡麼」,蓋以配后土而母
 之云。大祖稱帝,尊祖母曰太皇太后,母曰皇太后,嬪曰皇后。等以徽稱,加以美號,質於隋、唐,文於故俗。後族唯乙室、拔里氏,而世任其國事。



 太祖慕漢高皇帝,故耶律兼稱劉氏;以乙室、拔裡比蕭相國,遂為蕭氏。



 耶律儼、陳大任《遼史後妃傳》,大同小異,酌取其當著於篇。肅祖昭烈皇后蕭氏,小字卓真。歸肅祖,生四子,見《皇子表》。乾統三年,追尊昭烈皇后。



 懿祖莊敬皇后蕭氏,小字牙裏辛。肅祖嘗過其家曰:「同姓可結交,異姓可結婚。」知為蕭氏,為懿祖聘焉。生男女七人。乾統三年,追尊莊敬皇后。



 玄祖簡獻皇后蕭氏,小字月里朵。玄祖為狠德所害,後嫠居,恐不免,命四子往依鄰家耶律臺押,乃獲安。太祖生,後以骨相異常,懼有陰圖害者,鞠之別帳。重熙二十一年,追尊簡獻太后。德祖宣簡皇后蕭氏,小字嚴母斤。遙輦氏宰相剔刺之女。



 男、女六人,太祖長子也。天顯八年崩,袝德陵。重熙二十一年,追尊宣簡皇后。



 太祖淳欽皇后述律氏,諱平,小字月理朵。其先回鶻人糯思,生魏寧舍利,魏寧生慎思梅里,慎思生婆姑梅里,婆姑娶勻德恝王女,生後於契丹右大部。婆姑名月碗,
 仕遙輦氏為阿扎割只。



 後簡重果斷,有雄略。嘗至遼、土二河之會,有女子乘青牛車,倉卒避路,忽不見。未幾,童謠曰:「青牛嫗,曾避路。」



 蓋諺謂地祗為青牛嫗云。



 太祖即位,群臣上尊號曰地皇后。神冊元年,大冊,加號應天大明地皇后。行兵御眾,後嘗與謀。太祖嘗渡磧擊黨項,黃頭、臭泊二室韋乘虛襲之;後知,勒兵以待,奮擊,大破之,名震諸夷。



 時晉王李存勖欲結援,以叔母事後。幽州劉守光遣韓延徽求援,不拜,太祖怒,留之,使牧馬。後曰:「守節不屈,賢者也。宜禮用之。」太祖乃召延徽與語,大悅,以為謀主。吳主李嚈獻猛火油,以水沃之愈熾。太祖選三
 萬騎以攻幽州。後曰:「豈有試油而攻人國者?」指帳前樹曰:「無皮可以生乎?」



 太祖曰:「不可。」後曰:「幽州之有土有民,亦猶是耳。吾以三千騎掠其四野,不過數年,困而歸我矣,何必為此?萬一不勝,為中國笑,吾部落不亦解體乎!」其平渤海,後與有謀。



 太祖崩,後稱制,攝軍國事。及葬,欲以身殉,親戚百官力諫,因斷右腕納於柩。太宗即位,尊為皇太后。會同初,上尊號曰廣德至仁昭烈崇簡應天皇太后。初,太祖嘗謂太宗必興我家,後欲令皇太子倍避之,太祖冊倍為東丹王。太祖崩,太宗立,東丹王避之唐。太后常屬意於少子李胡。太宗崩,世宗即位於鎮陽,太
 后怒,遣李胡以兵逆擊。李胡敗,太后親率師遇於潢河之橫渡。賴耶律屋質諫,罷兵。遷太后於祖州。



 應歷三年崩,年七十五,袝祖陵,謚曰貞烈。重熙二十一年,更今謚。



 太宗靖安皇后蕭氏,小字溫,淳欽皇后弟室魯之女。帝為大元帥,納為妃,生穆宗。及即位,立為皇后。性聰慧潔素,尤被寵顧,雖軍旅、田獵必與。天顯十年崩,謚彰德,葬奉陵。



 重熙二十一年,更今謚。



 世宗懷節皇后蕭氏,小字撒葛只,淳欽皇后弟阿古只之女。



 帝為永康王,納之,生景宗。天祿末,立為皇后。明年秋,生萌古公主。在蓐,察割作亂,弒太后及帝。後乘步輦,
 直詣察割,謂畢收殮。明日遇害。謚曰孝烈皇后。重熙二十一年,更今謚。世宗妃甄氏,後唐宮人,有姿色。帝從太宗南征得之,寵遇甚厚,生寧王只沒。及即位,立為皇后。嚴明端重,風神閑雅。內治有法,莫干以私。劉知遠、郭威稱帝,世宗承強盛之資,奄奄歲時。後與參帷幄,密贊大謀,不果用。察割作亂,遇害。景宗立,葬二后於醫巫閭山,建廟陵寢側。



 穆宗皇后蕭氏,父知璠,內供奉翰林承旨。後生,有雲氣馥鬱久之。幼有儀則。帝居藩,納為妃。及正位中宮,性柔婉,不能規正。無子。



 景宗睿智皇后蕭氏,諱綽,小字燕燕,北府宰相思溫女。



 早慧。思溫嘗觀諸女掃地,惟後潔除,喜曰:「此女必能成家!」



 帝即位,選為貴妃。尋冊為皇后,生聖宗。



 景宗崩,尊為皇太后,攝國政。後泣曰:「母寡子弱,族屬雄強,邊防未靖,奈何?」耶律斜珍、錦德讓進曰:「信任臣等,何慮之有!」於是,後與斜珍、德讓參決大政,委於越休哥以南邊事。統和元年,上尊號曰承天皇太后。二十四年,加上尊號曰睿德神略應運啟化承天皇太后。二十七年崩,謚曰聖神宣獻皇后。重熙二十一年,更今謚;後明達治道,聞善必從,故群臣咸竭其忠。習知軍政,澶淵之役,親御戎車,指
 麾三軍,賞罰信明,將士用命。聖宗稱遼盛主,後教訓為多。



 聖宗仁德皇后蕭氏,小字菩薩哥,睿智皇后弟隗因之女。



 年十二,美而才,選入掖庭。統和十九年,冊為齊天皇后。



 嘗以草莛為殿式,密付有司,令造清風、天祥、八方三殿。



 既成,益寵異。所乘車置龍首鴟尾,飾以黃金。又造九龍輅、諸子車,以白金為浮圖,各有巧思。夏秋從行山谷間,花木如繡,車服相錯,人望之以為神仙。



 生皇子二,皆早卒。開泰五年,宮人耨斤生興宗,後養為子。帝大漸,耨斤管後曰:「老物寵亦有既耶?」左右扶後出。



 帝崩,耨斤自
 立為皇太后,是為欽哀皇后。護衛馮家奴、喜孫等希旨,誣告北府宰相蕭泥卜、國舅蕭匹敵謀逆。詔令鞫治,連及後。興宗聞之曰:「皇后侍先帝四十年,撫育眇躬,當為太后;今不果,反罪之,可乎?」欽哀曰:「此人若在,恐為後患。」帝曰:「皇后無子而老,雖在,無能為也。」欽哀不從,遷後於上京。



 車駕春搜,欽哀慮帝杯鞠育恩,馳遣人加害。使至,後曰:「我實無辜,天下共知。卿待我浴,而後就死,可乎?」使者退。比反,後已崩,年五十。是日,若有見後於木葉山陰者,乘青蓋車,衛從甚嚴。



 追尊仁德皇后。與欽哀並袝慶
 陵。



 聖宗欽哀皇后蕭氏,小字耨斤,淳欽皇后弟阿古只五世孫。



 黝面,狠視。母嘗夢金柱擎天,諸子欲上不能;後後至,與僕從皆升,異之。



 久之,入宮。嘗指承天太后榻,獲金雞,吞之,膚色光澤勝常。太后驚異曰:「是必有奇子!」已而生興宗。仁德皇后無子,取而養之如己出。後以興宗待仁德皇後遵,不悅。聖宗崩,令馮家奴等誣仁德皇后與蕭浞卜、蕭匹敵等謀亂,徒上京,害之。自立為皇太后,攝政,以生辰為應聖節。



 重熙元年,尊為仁慈聖善欽孝廣德安靖貞純寬厚崇覺儀天皇太后。三年,後陰召諸弟議,欲立少子重元,重元以所謀白帝。帝收太后符璽,遷
 於慶州七括官。六年秋,帝悔之,親馭奉迎,侍養益孝謹。後常不怪。帝崩,殊無戚容。見崇聖皇后悲泣如禮,謂曰:「汝年尚幼,何哀痛如是!」



 清寧初,尊為太皇太后。崩,謚曰欽哀皇后。



 後初攝政,追封曾祖為蘭陵郡王,父為齊國王,諸弟皆王之,雖漢五侯無以過。



 興宗仁懿皇后蕭氏,小字撻里,欽哀皇后弟孝穆之長女。



 性寬睿,姿貌端麗。帝即位,入宮,生道宗。重熙四年,立為皇后。二十三年,號貞懿慈和文惠孝敬慶愛崇聖皇后。



 通宗即位,尊為皇太后。清寧二年,上尊號曰慈懿仁和文惠孝敬廣愛宗天皇太后。九年秋,敦睦官使耶律
 良以重元與其子涅魯古反狀密告太后,乃言於帝。帝疑之,太后曰:「此社傻大事,宜早為計。」帝始戒嚴。及戰,太后親督衛士,破逆黨。大康二年崩,謚仁懿皇后。



 仁慈淑謹,中外感德。凡正旦、生辰諸國貢幣,悉賜貧瘠。



 嘗夢重元日:「臣骨在太子山北,不勝寒璠。」寤即命屋之,慈憫類此。



 興宗貴妃蕭氏,小字三妒,駙馬都尉匹里之女。選入東宮。



 帝即位,立為皇后。重熙初,以罪降貴妃。



 通宗宣懿皇后蕭氏,小字觀音,欽哀皇后弟樞密使惠之女。



 姿容冠絕,工詩,善談論。自制歌詞,尤善琵琶。重熙
 中,帝王燕趙,納為妃。清寧初,立為懿德皇后。



 皇太叔重元妻,以艷冶自矜,後見之,戒曰:「為貴家婦,何必如此!」



 後生太子浚,有專房寵。好音樂,伶官趙惟一得侍左右。



 太康初,宮婢單登、教坊朱頂鶴誣后與惟一私,樞密使耶律乙辛以聞。詔乙辛與張孝傑劾狀,因而實之。族誅惟一,賜后自盡,歸其尸於家。



 乾統初,追謚宣懿皇后,合葬慶陵。



 道宗惠妃蕭氏,小字坦思,附馬都尉霞抹之妹。大康二年,乙辛譽之,選入掖庭,立為皇后。



 居數歲,未見皇嗣。後妹斡特懶先嫁乙辛子綏也,後以宜子言於帝,離婚,納
 宮申。八年,皇孫延禧封粱王,降為惠妃,徙乾陵;斡特懶還其家。頃之,其母燕國夫人厭魁梁王,伏誅。



 貶妃為庶人,幽於宜州,諸弟沒入興聖宮。



 天慶六年,召還,封太皇太記。後二年,奔黑頂山,卒,葬太子山。



 天祚皇后蕭氏,小字奪里懶,宰相繼先五世孫。大安三年入宮。明年,封燕國王妃。乾統初,冊為皇后。性閑淑,有儀則。兄弟奉先、保先等緣後寵柄任。女直亂,從天祚西狩,以疾崩。天祚德妃蕭氏,小字師姑,北府宰相常哥之女。壽隆二年入官,封燕國妃,生子撻魯。乾統三年,改德妃,以柴冊
 禮,封撻魯為燕國王,加妃號贊翼。王薨,以哀戚卒。



 天祚文妃蕭氏,小字瑟瑟,國舅大父房之女。乾統初,帝幸耶律撻葛第,見而悅之,匿宮中數月。皇太叔和魯斡勸帝以禮選納,三年冬,立為文妃。生蜀國公主、晉王敖盧斡,尤被寵幸。以柴冊,加號承翼。



 善歌詩。女直亂作,日見侵迫。帝畋游不恤,忠臣多被疏斥。妃作歌諷諫,其詞曰:「勿嗟塞上兮暗紅塵,勿傷多難兮畏夷人;不如塞奸邪之路兮,選取賢臣。直須臥薪嘗膽兮,激壯士之捐身;可以朝清漠北兮,夕枕燕、雲。」又歌曰:「丞相來朝兮劍佩鳴,千官側目兮寂無聲。養成外患兮嗟何及!禍盡忠臣
 合罰不明。親戚並居兮藩屏位,私門潛畜兮爪牙兵。可憐往代兮秦天子,猶向宮中兮望太平。」天祚見而銜之。



 播遷以來,郡縣所失幾半,上頗有倦勤之意。諸皇子敖盧斡最賢,素有人望。元后兄蕭奉先深忌之,誣南軍都統余觀謀立晉王,以妃與聞,賜死。



 天祚元妃蕭氏,小字貴哥,燕國妃之妹。年十七,冊為元妮。性沉靜。嘗晝寢,近侍盜貂恟,妃覺而不言,宮掖稱其寬厚。從天祚四狩,以疾薨。



 論曰:「遼以鞍馬為家,後妃往往長於射御,軍旅田獵,未嘗不從。如應天之奮擊室韋,承天之御戎澶淵,仁懿之
 親破重元,古所未有,亦其俗也。



 靖安無毀無譽;齊天巧思,乃奢侈之漸;宣懿度曲知音,豈致誣蔑之階乎?文妃能歌詩諷諫,而謂謀私其子,非矣。若簡憲之艱危保孤,懷節之從容就義,雖烈丈夫何以過之。欽哀狠傑,賊殺嫡後,而興宗不能防閑其母,惜哉!」



\end{pinyinscope}