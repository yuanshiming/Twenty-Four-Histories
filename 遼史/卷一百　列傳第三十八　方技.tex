\article{卷一百 列傳第三十八 方技}

\begin{pinyinscope}

 直魯古王白魏璘耶律敵魯耶律乙不哥孔子稱「小道必有可觀」,醫卜是已。醫以濟夭札,卜以決猶豫,皆有補於國,有惠於民。前史錄而不遺,故傳。



 直魯古,吐谷渾人。初,太祖破吐谷渾,一騎士棄橐,反射不中而去。及追兵開橐視之,中得一嬰兒,即直魯古也。因所俘者問其故,乃知射囊者,嬰之父也。世善醫,雖馬上視疾,亦知標本。意不欲子為人所得,欲殺之耳。



 由是
 進於太祖,淳欽皇后收養之。長亦能醫,專事針灸。



 太宗時,以太醫給侍。嘗撰《脈訣》、《針灸書》,行於世。年九十卒。



 王白,冀州人,明天文,善卜筮,晉司天少監,太宗入汴得之。應歷十九年,王子只沒以事下獄,其母求卜,白曰:「此人當王,未能殺也,毋過憂!」景宗即位,釋其罪,封寧王,竟如其言。凡決禍福多此類。



 保寧中,歷彰武、興國二軍節度使。撰《百中歌》行於世。



 魏璘,不知何郡人,以卜名世,太宗得於汴。



 天祿元年,上命馳馬較遲疾,以為勝負。問王白及璘孰勝?



 白奏曰:「赤者勝。」璘曰:「臣所見,騣馬當勝。」既馳,竟如璘言。上異而問
 之,白曰:「今日火王,故知赤者勝。」



 璘曰:「不然,火雖王,而上有煙。以煙察之,青者必勝。」



 上嘉之。五年,察割謀逆,私卜於璘。璘始卜,謂曰:「大王之數,得一日矣,宜慎之!」及亂,果敗。應歷中,周兵犯燕,上以勝敗問璘。璘曰:「周姓柴也,燕分火也。柴入火,必焚。」



 其言果驗。



 璘嘗為太平王罨撒葛卜僭立事,上聞之,免死,流烏古部。



 一日,節度使召璘,適有獻雙鯉者,戲曰:「君卜此魚何時得食?」璘良久答曰:「公與僕不出今日,有不測禍,奚暇食魚?」



 亟命烹之。未及食,寇至,俱遇害。



 耶律敵魯,字撒不碗。其先本五院之族,始置宮分,隸焉。



 敵魯精於醫,察形色即知病原。雖不診候,有十全功。統和初,為大丞相韓德讓所薦,官至節度使。



 初,樞密使耶律斜軫妻有沉痾,易數醫不能治。敵魯視之曰:「心有蓄熱,非藥石所及,當以意療。因其聵,聒之使狂,用洩其毒則可。」於是令大擊鉦鼓於前。翌日果狂,叫呼怒罵,力極而止,遂愈。治法多此類,人莫能測。年八十卒。



 耶律乙不哥,字習捻,六院郎君褭古直之後。幼好學,尤長於卜筮,不樂仕進。



 嘗為人擇葬地曰:「後三日,有牛乘人逐牛過者,即啟土。」



 至期,果一人負乳犢,引牸牛而過。其人曰:「所謂『牛乘人』者,此也。」遂啟土。既葬,吉兇盡如其
 言。又為失鷹者占曰:「鷹在汝家東北三十里濼西榆上。」往求之,果得。當時占候無不驗。



 論曰:「方持,術者也。茍精其業而不畔於道,君子必取焉。直魯古、王白、耶律敵魯無大得失,錄之宣矣。魏璘為察割卜謀逆,為罨撒葛卜僭立,罪在不貰;雖有寸長,亦奚足取哉。存而弗削,為來者戒。」



\end{pinyinscope}