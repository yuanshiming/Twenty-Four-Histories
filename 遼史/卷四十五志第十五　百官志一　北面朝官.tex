\article{卷四十五志第十五 百官志一 北面朝官}

\begin{pinyinscope}

 官生於職,職沿於事,而名加之。後世沿名,不究其實。



 吏部一太宰也,為大司徒,為尚書,為中書,為門下。兵部一司馬也,為大司馬,為太尉,為樞密使。沿古官名,分今之職事以配之,於是先王統理天下之法,如治絲而棼,名實淆矣。



 契丹舊俗,事簡職專,官制樸實,不以名亂之,其興也勃焉。太祖神冊六年,詔正班爵。至於太宗,兼制中
 國,官分南、北,以國制治契丹,以漢制待漢人。國制簡樸,漢制由沿名之風固存也。遼國官制,分北、南院。北面治宮帳、部族、屬國之政,南面治漢人州縣、租賦、軍馬之事。因俗而治,得其宜矣。



 初,太祖分迭刺夷離堇為北、南二大王,謂之北、南院。



 宰相、樞密、宣徽、林牙,下至郎君、護衛,皆分北、南,其實所治皆北面之事。語遼官制者不可不辨。



 凡遼朝官,北樞密視兵部、南樞密視吏部,北、南二王視戶部,夷離畢視刑部,宣徽視工部,敵烈麻都視禮部,北、南府宰相總之。惕隱治宗族,林牙修文告,於越坐而論議以象公師。朝遷之上,事簡職專,此遼所以興也。



 北面朝官契丹北樞密院。掌兵機、武銓、群牧之政,凡契丹軍馬皆屬焉。以其牙帳居大內帳殿之北,故名北院。元好問報謂「北衙不理民」是也。



 北院樞密使。



 知北院樞密使事。



 知樞密院事。



 北院樞密副使。



 知北院樞密副使事。



 同知北院樞密使事。



 簽書北樞密院事。



 北院都承旨。



 北院副承旨。



 北院林牙。



 知北院貼黃。



 給事北院知聖旨頭子事。



 掌北院頭子。



 北樞密院敞史。



 北院郎君。



 北樞密院通事。



 北院椽史。



 北樞密院中丞司。



 北南樞密院點檢中丞司事。



 總知中丞司事。



 北院左中丞北院右中丞。



 同知中丞司事。



 北院侍御。



 契丹南樞密院。掌文銓、部族、丁賦之政,凡契丹人民皆屬焉。以其牙帳居大內之南,故名南院。元好問所謂「南衙不主兵」是也。



 南院樞密使。



 知南院樞密使事。



 知南院樞密事。



 南院樞密副使。



 知南院樞密副使事。



 同知南院樞密使事。



 簽書南樞密院事。



 南院都承旨。



 南院副承旨。



 南院林牙。



 知雨院貼黃。



 給事南院知聖旨頭子事。



 掌南院頭子。



 南樞密院敞史。



 南院郎君。



 南樞密院通事。



 南院掾史。



 南樞密院中丞司。



 北南樞密院點檢中丞司事。



 總知中丞司事。



 南院左中丞。南院有中丞。



 同知中丞司事。



 南院侍御。



 北宰相府。掌佐理軍國之大政,皇族四帳世預其選。



 北府左宰相。



 北府右宰相。



 總知軍國事。



 知國事。



 南宰相府。掌佐理軍國之大政,國舅五帳世預其選。



 南府左宰相。



 南府右宰相。



 總知軍國事。



 知國事。



 北大王院。分掌部族軍民之政。



 北院大王。初名迭刺部夷離堇,太祖分北、南院,太宗會同元年改夷離堇為大王。知北院大王事。



 北院太師。



 北院太保。



 北院司徒。



 北院司空。



 北院郎君。



 北院都統軍司。掌北院從軍之政令北院統軍使。



 北院副統軍使。



 北院統軍都監。



 北院詳穩司。掌北院部族軍馬之政令。



 北院詳穩。
 北院都監。



 北院將軍。



 北院小將軍北院都部署司。掌北院部族軍民之事。



 北院都部署北院副部署。



 南大王院。分掌部族軍民之政。



 南院大王。



 知南院大王事。



 南院太師。



 南院太保。天慶八年,省南院太保。



 南院司徒。



 南院司空。



 南院郎君。



 南院都統軍司。掌南院從軍之政令。



 南院統軍使。



 南院副統軍使。



 南院統軍都監。



 南院詳穩司。掌南院部族軍馬之政令。



 南院詳穩。



 南院都監。



 南院將軍。



 南院小將軍。



 南院都部署司。掌南院部族軍民之事。



 南院都部署。



 南院副部署。



 宣徽北院。太宗會同元年置,掌北院御前抵應之事。北院宣徽使。



 知北院宣徽事。



 北院宣徽副使。



 同知北院宣徽事。



 宣徽南院。會同元年置,掌南院御前祗應之事。



 南院宣徽使。



 知南院宣徽事。



 南院宣徽副使。



 同知南院宣徽事。



 大於越府。無職掌,班百僚之上,非有大功德者不授,遼國尊官,猶南面之有三公。太祖以遙輦氏於越受禪。終遼之世,以於越得重名者三人:耶律曷魯、屋質、仁先,謂之三於越。



 大於越。



 大惕隱司。太祖置,掌皇族之政教。興宗重熙二十一年,耶律義先拜惕隱,戒族人曰:「國家三父房最為貴族,凡天下風化之所自出,不孝不義,雖小不可為。」其妻晉國長公主之女,每見中表,必具禮服。義先以身率先,國族化之。遼國設官之實,於此可見。太祖有國,首設此官,其後百官擇人,必先宗姓。



 惕隱。亦日梯裡已。



 知惕隱司事。



 惕隱都監。



 夷離畢院。掌刑獄。



 夷離畢。



 左夷離畢。



 右夷離畢。



 知左夷離畢事。知右夷離畢事。



 敞史。



 選底。掌獄。



 大林牙院。掌文翰之事。



 北面都林牙。



 北面林牙承旨。



 北面林牙。



 左林牙。



 右林牙。



 敵烈麻都司。掌禮儀。



 敵烈麻都。



 總知朝廷禮儀。



 總禮儀事。



 文班司。所掌未詳。



 文班太保。



 文班林牙。



 文班牙署。



 文班吏。



 阿札割只。所掌未詳。遙輦故宮,後並樞密院。



 阿札割只。



 北面御帳官三皇聖人也,當淳樸之世,重門擊柝,猶嚴於待暴客。遼之先世,未有城郭、溝池、宮室之固,氈車為營,硬寨為宮,御帳之官不得不謹。出於貴戚為侍衛,著帳為近侍,北南部族為護衛,武臣為宿衛,親軍為禁衛,百官番宿為宿直。奉宸以司供御,三班以肅會朝,硬寨以嚴晨夜。法
 制可謂嚴密矣。考其凡如左。侍衛司。掌御帳親衛之事。



 侍衛太師。



 侍衛太保。



 侍衛司徒。



 侍衛司空侍衛。



 近侍局。



 近侍直長。



 近侍。



 近侍小底。



 近侍詳穩司。



 近侍詳穩近侍都監。



 近侍將軍。



 近侍小將軍。



 北護衛府。掌北院護衛之事。皇太后宮有左右護衛。



 北護衛太師。



 北護衛太保。



 北護衛司徒。



 總領左右護衛司總領左右護衛。



 左護衛司。



 左護衛太保。



 左護衛。



 右護衛司。



 石護衛太保。



 右護衛。南護衛府。掌南院護衛之事。



 南護衛太師。



 南護衛太保。



 南護衛司徒。



 總領左有護衛司。



 總領左右護衛。



 左護衛司。



 左護衛太保。



 左護衛。



 右護衛司。



 右護衛太保。



 右護衛。



 奉宸司。掌供奉震御之事。



 官名未詳。



 奉宸。



 三班院。掌左、右、寄班之事。



 左班都知。



 右班都知。



 寄班都知。



 三班院祗候。



 宿衛司。專掌宿衛之事。



 總宿衛事。亦曰典宿衛事。



 總知宿衛事。



 同掌宿衛事。



 宿衛官。



 禁衛局。



 總禁衛事。禁衛長。



 宿直司。掌輪直官員宿直之事。皇太后宮有宿直衛。



 宿直詳穩。



 宿直都監。



 宿直將軍。



 宿直小將軍。



 宿直官。



 宿直護衛。



 硬寨司。掌禁圍槍寨、下鋪、傳鈴之事。



 硬寨太保。



 皇太子惕隱司。掌皇太子宮帳之事。



 皇太子惕隱。



 北面著帳官。



 古者刑人不在君側。叛逆家屬沒為著帳,執事禁衛,可為寒心。此遼世所以多變起肘掖歟。



 著帳郎君院。遙輦痕德堇可汗以蒲古只等三族害於越室魯,家屬沒入瓦里。應天皇太后知國政,析出之,以為著帳郎君、娘子,每加矜恤。世宗悉免之。其後內族、外戚及世官之家犯罪者,皆沒入瓦里。人戶益眾,因復故名。皇太后、皇太妃帳,皆有著帳諸局。



 著帳郎君節度使。



 著帳郎君司徒。



 祗候郎君班詳穩司。



 祗候郎君班詳穩。



 祗候郎君直長。



 祗候郎君閘撒狘。



 祗候郎君。祗候郎君拽刺。



 左祗候郎君班詳穩司。



 左祗候郎君班詳穩。



 左祗候郎君直長。



 左祗候郎君閘撒狘。



 左祗候郎君。



 左祗候郎君拽刺。



 右祗候郎君班詳穩司。



 右祗候郎君班詳穩。



 右祗候郎君直長。



 右祗候郎君閘撒狘。



 右祗候郎君。



 右祗郎君拽刺。



 筆硯局。



 筆硯祗候郎君。



 筆硯吏。



 牌印局牌印郎君。



 裀褥局。



 裀褥郎君。



 燈燭局。



 燈燭郎君。



 床幔局。



 床幔郎君。



 殿幄局。



 殿幄郎君。



 車輿局。車輿郎君。



 御盞局。



 御盞郎君。



 本班局。



 本班郎君。



 皇太后祗應司。



 領皇太后諸局事。



 知皇太后宮諸司事。



 皇太妃祗應司。



 皇后祗應司。



 近位祗應司。



 皇太子祗應司。



 親王祗應司。



 著帳戶司。本諸斡魯朵戶析出,及諸色人犯罪沒人。凡御帳、皇太后、皇太妃、皇后、皇太子、近位、親王祗從、伶官,皆充其役。



 著帳節度使。



 著帳殿中。



 承應小底局。



 筆硯小底。



 寢殿小底。



 佛殿小底。



 司藏小底。



 習馬小底。



 鷹坊小底。



 湯藥小底。



 尚飲小底。盥漱小底。



 尚膳小底。



 尚衣小底。



 裁造小底。



 北面皇族帳官肅祖長子洽窅之族在五院司;叔子葛刺、季子洽禮及懿祖仲子帖刺、季子褭古直之簇旨在六院司。此五房者,謂之二院皇族。玄祖伯子麻魯無後,次子巖木之後日孟父房;叔子釋魯曰仲父房;季子力德祖,德祖之元子是為太祖天皇帝,謂之橫帳;次曰刺葛,曰迭刺,曰寅底石,曰安端,曰蘇,皆曰季父房。此一帳三房,謂之四帳皇族。二院治之以北、南二王,四帳治之以大內惕隱,皆統於大惕隱司。



 大內惕隱司。掌皇族四帳之政教。



 大內惕隱。



 知大內惕隱事。



 大內惕隱都監。



 大橫帳常袞司。掌太祖皇帝後九帳皇族之事。



 橫帳常袞。亦曰橫帳敞穩。



 橫帳太師。



 橫帳太保。



 橫帳司空。



 橫帳郎君。



 橫帳知事。



 孟父族帳常袞司。掌蜀國王巖木房族之事。



 仲父族帳常袞司。掌隋國王釋魯房族之事。



 季父族帳常袞司。掌德祖皇帝三房族之事。



 四帳都詳穩司。掌四帳軍馬之事。都詳穩。



 都監。



 將軍。本名敞史。



 小將軍橫帳詳穩司。



 孟父帳詳穩司。



 仲父帳詳穩司。



 季父帳詳穩司。



 舍利司。掌皇族之軍政。



 舍利詳穩。



 舍利都監。



 舍利將軍。



 舍利小將軍。



 舍利。



 梅里。



 親王國。官制未詳。



 王府近侍。



 王府祗候。



 大東丹國中臺省。太祖天顯元年置,乾亨四年聖宗省。



 左大相。



 右大相。



 左次相。



 右次相。



 王子院。掌王子各帳之事。



 王子太師。



 王子太保。



 王子司徒。王子司空。



 王子班郎君。



 咐馬都尉府。掌公主帳宅之事。



 馳馬都尉。



 北面諸帳官遼太祖有帝王之度者三:代遙輦氏,尊九帳於御營之上,一也;滅渤海國,存其族帳,亞於遙輦,二也;並奚王之眾,撫其帳部,擬於國族,三也。有英雄之智者三:任國舅以耦皇族,崇乙室以抗奚王,列二院以制遙輦是已。觀
 北面諸帳官,可以見之矣。



 遙輦九帳大常袞司。掌遙袞窪可汗、阻午可汗、胡刺可汗、蘇可汗、鮮質可汗、昭古可汗、耶瀾可汗、巴刺可汗、痕德堇可汗九世宮分之事。太祖受位於遙輦,以九帳居皇族一帳之上,設常袞司以奉之,有司不與焉。凡遼十二宮、五京,皆太祖以來征討所得,非受之於遙輦也。其待先世之厚,蔑以加矣。遼俗東向而尚左,御帳東向,遙輦九帳南向,皇族三父帳北向。



 東西為經,南北為緯,故謂御營為橫帳雲。



 大常袞。亦日敞穩。



 遙輦太師。



 遙輦太保遙輦太尉。



 遙輦司徒。



 遙輦司空。



 遙輦侍中。一作世燭。太宗會同元年置。



 敞史。



 知事。



 遙輦帳節度使司。節度使。



 節度副使。



 遙輦糺穩司。



 遙輦糺詳穩。



 遙輦糺都監。



 遙輦糺將軍。



 遙輦糺小將軍。



 遙輦克。官名未詳。



 大國舅司。掌國舅乙室已、拔里二帳之事。太宗天顯十年,合皇太后二帳為國舅司;聖宗開泰三年,又並乙室己、拔里二司為一帳。



 乙室已國舅大翁帳常袞。一件敞穩。



 乙室己國舅小翁帳常袞。



 拔里國舅大父帳常袞。



 拔里國舅少父帳常袞。



 國舅太師。



 國舅太保。



 國舅太尉。



 國舅司徒。



 國舅司空。



 敞史。太宗會同元年,改郎君為敞史。



 知事。



 國舅乙室已大翁帳詳穩司。



 國舅詳穩。



 國舅都監。



 國舅本族將軍。



 國舅本族小將軍。興宗重熙五年,樞密院奏,國舅乙室己小翁帳敞史,準大橫帳及國舅二父帳,改為將軍。



 國舅乙室己小翁帳詳穩司。



 國舅拔里大父帳詳穩司。



 國舅拔里少父帳詳穩司。



 國舅夷離畢司國舅夷離畢。



 國舅左夷離畢。



 國舅右夷離畢。



 敞史。



 國舅帳克。



 國舅別部。世宗置。



 官制未詳。



 國舅別部敞史。聖宗太平八年,見國舅別部敞史蕭塔葛。



 渤海帳司。官制末詳。



 渤海宰相。



 渤海太保。



 渤海撻馬。



 渤海近侍詳穩司;奚王府。



 乙室王府。並見《部族官》。



 北面宮官遼建諸宮斡魯朵,部族、蕃戶,統以北面宮官。具如左。



 諸行宮都部署院。總契丹漢人諸行宮之事。



 諸行宮都部署。



 知行宮諸部署司事。



 諸行宮副部署。



 諸行宮判官。契丹行宮都部署司。總行在行軍諸斡魯朵之政令。



 契丹行宮都部署。



 知契丹行宮都部署事。



 契丹行宮副部署。



 契丹行宮判官。



 行宮諸部署司。掌行在諸宮之政令。



 行宮都部署。



 行宮副部署。



 行宮部署判官。



 十二宮職名總目:某宮。



 某宮使。



 某宮副使。



 某宮太師。



 某宮太保。



 某宮侍中。太宗會同元年置,亦曰世燭。



 某宮都部署司。掌本宮契丹軍民之事。



 某宮都部署。



 某宮副部署。



 某宮判官。



 某宮提轄司。官制未詳。



 某宮馬群司。



 侍中。



 敞史。



 某石烈。石烈,縣也。



 夷離堇。本名彌里馬特本,改辛衷,會同元年升。



 麻普。本名達刺干,會同元年改。牙書。會同元年置。



 某瓦里。內族、外戚、世官犯罪,沒入瓦里。



 抹鶻。



 某抹里。



 閘撒狘。



 某得里。官名未詳。



 太祖弘義宮。



 太宗永興宮。



 世宗積慶宮。



 應天皇太后長寧宮。



 穆宗延昌宮。



 景宗彰愍宮。



 承天皇太后崇德宮。



 聖宗興聖宮。



 興宗延慶宮道宗太和宮。



 天祚永昌宮。



 孝文皇太弟敦睦宮。



 文忠王府。



 己上十二宮一府,部署、提轄、石烈、瓦里、抹里、
 得裡等,並見《營衛志》。



 押行宮輜重夷離畢司。掌諸宮巡幸扈從輜重之事。



 夷離畢。



 敞史。



\end{pinyinscope}