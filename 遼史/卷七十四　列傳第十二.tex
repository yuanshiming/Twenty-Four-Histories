\article{卷七十四 列傳第十二}

\begin{pinyinscope}

 耶律隆運弟德威德威孫滌魯耶律制心耶律勃古哲蕭陽阿武白蕭常哥耶律虎古磨魯古耶律隆運,本姓韓,名德讓,西南面招討使匡嗣之子也。



 統和十九年,賜名德昌;二十二年,賜姓耶律;二十八年,復賜名隆運。重厚有智略,明治體,喜建功立事。



 侍景宗,
 以謹飭聞,加東頭承奉官,補樞密院通事,轉上京皇城使,遙授彰德軍節度使,代其父匡嗣為上京留守,權知京事,甚有聲。尋復代父守南京,進人榮之。宋兵取河東,侵燕,五院鞮詳穩奚底、統軍蕭討古等敗歸,宋兵圍城,招脅甚急,人懷二心。詔隆運登城,日夜守御。援軍至,圍解。及戰高梁河,宋兵敗走,隆運邀擊,又破之。以功拜遼興軍節度使,徵為南院樞密使。



 景宗疾大漸,與耶律斜軫俱受顧命,立梁王為帝,皇后為皇太后,稱制,隆運總宿衛事,太后益寵任之。統和元年,加開府儀同三司,兼政事令。四年,宋遣曹彬、米信將十萬眾來侵,隆運從太后
 出師敗之,加守司空,封楚國公。師還,與北府宰相室昉共執國政。上言山西四州數被兵,加以歲饑,宜輕稅賦以來流民,從之。六年,太後觀擊鞠,胡里室突隆運墜馬,命立斬之。詔率師伐宋、圍沙堆,敵乘夜來襲,隆運嚴軍以待,敗走之,封楚王。九年,復言燕人挾奸,茍免賦役,貴族因為囊橐,可遣北院宣徽使趙智戒諭,從之。



 十一年,丁母憂,詔強起之。明年,室昉致政,以隆運代為北府宰相,仍領樞密使,監修國史,賜興化功臣。十二年六月,奏三京諸鞫獄官吏,多因請托,曲加寬貸,或妄行搒掠,乞行禁止。上可其奏。又表請任賢去邪,太后喜曰:「進賢輔政,真
 大臣之職。」憂加賜齎,服闋,加守太保、兼政事令。



 會北院樞密使耶律斜軫薨,隆運兼之。久之,拜大丞相,進王齊,總二樞府事。以南京、平州歲不登,奏免百姓農器錢,及請平諸郡商賈價,並從之。



 二十二年,從太后南征,及河,許宋成而還。徙王晉,賜姓,出宮籍,隸橫帳季父房後,乃改賜今名,位親王上,賜田宅及陪葬地。



 從伐高麗還,得末疾,帝與後臨視醫藥。薨,年七十一。



 贈尚書令,謚文忠,官給葬具,建廟乾陵側。無子。清寧三年,以魏王貼不子耶魯為嗣。天祚立,以皇子敖盧翰繼之。弟德威,侄制
 心。



 德威,性剛介,善馳射。保寧初,歷上京皇城使,儒州防禦使,改北院宣徽使。乾亨末,丁父喪,強起復職,權西南招討使。統和初,黨項寇邊,一戰卻之。賜劍許便宜行事,領突呂不、迭刺二紅軍。以討平稍古葛功,真授招討使。夏州李繼遷叛宋內附,德威請納之。既得繼遷,諸夷皆從,爾書褒獎。與惕隱耶律善補敗宋將楊繼業,加開府儀同三司、政事門下平章事。未幾,以山西域邑多陷,奪兵柄。李繼遷受賂,潛杯二心,奉詔率兵往諭,繼遷托以西征不出,德威至靈州俘掠而還。



 年五十五卒,贈兼侍中。子雱金,終彰國軍節度使。二孫:謝十、滌魯。謝十終惕隱。



 滌魯,字遵寧。幼養宮中,授小將軍。



 重熙初,歷北院宣徽使、右林牙、副點檢,拜惕隱,改西北路招討使,封漆水郡王,請減軍籍三千二百八十人。後以私取回鶻使者獺毛裘,及私取阻卜貢物,事覺,決大杖,削爵免官。俄起為北院宣徽使。十九年,改烏古敵烈部都詳穩,尋為東北路詳穩,封混同郡王。



 清寧初,徙王鄧,擢拜南府宰相。以年老乞骸骨,更王漢。



 大康中薨,年八十。



 滌魯神情秀徹,聖宗子視之,興宗待以兄禮,雖貴愈謙。



 初為都點檢,扈從獵黑嶺,獲熊。上因樂飲,謂滌魯曰:「汝有求平?」對曰:「臣富貴逾分,不敢他望。惟臣叔先朝優遇,身歿之後,不肖
 子坐罪籍沒,四時之薦享,諸孫中得赦一人以主祭,臣願畢矣。」詔免籍,復其產。子燕五,官至南京步軍都指揮使。



 制心,小字可汗奴。父德祟,善醫,視人形色,輒決其病,累官至武定軍節度使。



 制心善調鷹隼。統和中,為歸化州刺史。開泰中,拜上京留守,進漢人行宮都部署,封漆水郡王。以皇后外弟,恩遇日隆。樞密副使蕭合卓用事,制心奏合卓寡識度,無行檢,上默然。每內宴歡洽,輒避之。皇后不悅曰:「汝不樂耶?」制心對曰:「寵貴鮮能長保,以為是憂耳!」



 太平中,歷中京留守、惕隱、南京留守,徙王燕,遷
 商院大王。或勸制心奉佛,對曰:「吾不知佛法,惟心無私,則近之矣。」一日,沫浴更衣而臥,家人聞絲竹之聲,怪而入視,則已逝矣。年五十三。贈政事令,追封陳王。



 守上京時,酒禁方嚴,有捕獲私醖者,一飲而盡,笑而不詰。卒之日,部民若哀父母。



 耶律勃古哲,字蒲奴隱,六院夷離堇蒲古只之後。勇悍,善治生。保寧中,為天德軍節度使,歷南京侍衛馬步軍都指揮使。以討平黨項羌阿理撒米、僕里鱉米,遷南院大王。



 聖宗即位,太后稱制,會群臣議軍國事,勃古哲上疏陳便宜數事,稱旨,即日兼領山西路諸州事。統和四
 年,宋將曹彬等侵燕,勃古哲擊之甚力,賜輸忠保節致主功臣,總知山西五州。



 會有告勃古哲曲法虐民者,按之有狀,以大杖決之。八年,為南京統軍使,卒。子爻裏,官至詳穩。



 蕭陽阿,字稍隱。端毅簡嚴,識遼、漢字,通天文、相法。



 父卒,自五蕃部親挽喪車至奚王嶺,人稱其孝。



 年十九,為本班郎君。歷鐵林、鐵鷂、大鷹三軍詳穩。乾統元年,由烏古敵烈部屯田太保為易州刺史。幸臣劉彥良嘗以事至州,怙寵恣橫,為陽阿所沮。彥良歸,妄加毀訾,尋遣人代陽阿。州民千餘詣闕請留,即日授武安州觀察使。歷烏
 古涅里、順義、彰信等軍節度使,權知東北路統軍使事。



 聞耶律狼不、鐸魯斡等叛,獨引麾下三十餘人追捕之,身被二創,生擒十餘人,送之行在。坐不獲首惡,免官。未幾,權南京留守,卒。



 武白,不知何郡人。為宋國子博士,差知相州,至通利軍,為我軍所俘。詔授上京國子博士。改臨潢縣令,遷廣德軍節度副使。先是,有訟宰相劉慎行與子婦姚氏私者,有司出其罪。聖宗詔白鞫之,白正其事。使高麗還,權中京留守。時慎行諸子皆處權要,以白斷百姓分籍事不直,坐左遷。



 未幾近尚書左丞,知樞密事,拜遼興軍節度
 使。致仕,卒。



 蕭常哥,字胡獨堇,國舅之族。祖約直,同政事門下平章事;父實老,累官節度使。



 常哥魁偉寡言,年三十餘,始為祗候郎君。歷本族將軍、松山州刺史。壽隆二年,以女為燕王妃,拜永興官使。及妃生子,為南院宣徽使,尋改漢人行官都部署。乾統初,加太子太師,為國舅詳穩。二年,改遼與軍節度使,召為北府宰相,以柴冊禮,加兼侍中。



 天慶元年,致仕,卒,謚曰欽肅。



 耶律虎古,字海鄰,六院夷離堇覿烈之孫。少穎悟,重然諾。



 保寧初,補御琖郎君。十年,使宋還,以宋取河東之意
 聞於上。燕王韓匡嗣曰:「何以知之?」虎古曰:「諸僭號之國,宋皆並收,惟河東未下。今宋講武習戰,意必在漢。」匡嗣力沮,乃止。明年,宋果伐漢。帝以虎古能料事,器之,乃曰:「吾與匡嗣慮不及此。」授涿州刺史。



 統和初,皇太后稱制,召赴京師。與韓德讓以事相件,健讓怒,取護衛所戎仗擊其腦,卒。子磨魯古。磨魯古,字遙隱,有智識,善射。



 統和初,拜南面林牙。四年,宋侵燕,太后親征。磨魯古為前鐸,手中流矢,拔而復進。太后既至,磨魯古以創不能戰,與北府宰相蕭繼先巡邏境上。累遷北院大王。



 六年,伐宋為先鋒,與耶律奴瓜
 破其將李忠吉於定州。以疾卒於軍。



 論曰:「德讓在統和間,位兼將相,其克敵制勝,進賢輔國,功業茂矣。至賜姓名,王齊、晉,抑有寵於太后而致然歟?



 宗族如德威平黨項,滌魯完宗祀,制心不茍合,家聲益振,豈無所自哉!若勃古之忠,陽阿之孝,武白之直,亦彬彬乎一代之良臣矣。」



\end{pinyinscope}