\article{卷二十二本紀第二十二 道宗二}

\begin{pinyinscope}

 八年春正月癸丑,如鴨子河。



 二月,駐蹕納葛濼。



 三月戊申朔,楚王蕭革致仕,進封鄭國王。



 夏五月,吾獨婉惕隱屯禿葛等乞歲貢馬、駝,許之。



 六月丙子朔,駐蹕拖古烈。辛丑,以右夷離畢奚馬六為奚六部大王。是月,御清涼殿放進士王鼎等九十三人。



 秋七月甲子,射熊於外室刺。



 冬十月甲戌朔,駐蹕獨盧金。



 十二月庚辰,以知北院
 樞密使事蕭圖古辭為北院樞密使。



 癸未,幸酉京。戊子,以皇太后行再生禮,曲赦西京囚。



 九年春正月辛亥,幸鴛鴦濼。辛未,禁民鬻銅。



 三月辛未,宋主禎殂,以侄曙為子嗣位。



 夏五月丙午,以隋王仁先為南院樞密使,徙封許王。是月,清暑曷裏夕狘。



 秋七月丙辰,如太子山。戊午,皇太叔重元與其子楚國王涅魯古及陳國王陳六、同知北院樞密使事蕭胡睹、衛王貼不、林牙涅刺搏古、統軍使蕭迭裡得、駙馬都尉參及弟術者、圖骨、旗鼓拽刺詳穩耶律郭九、文班太保奚叔、內藏提點烏骨、護衛左太保敵不古、按答、副官使韓家奴、寶
 神奴等凡四百人,誘脅彎手軍犯行宮。時南院樞密使許王仁先、知北樞密院事趙王耶律乙辛、南府宰相蕭唐古、北院宣徽使蕭韓家奴、北院樞密副使蕭惟信、敦睦宮使耶律良等率宿衛士卒數千人御之。涅魯古躍馬突出,將戰,為近侍詳穩渤海阿廝、護衛蘇射殺之。己未,族逆黨家。庚申,重元亡入大漠,自殺。辛酉,詔諭諸道。



 壬戌,以仁先為北院樞密使,進封宋王,加尚父,耶律乙辛南院樞密使,蕭韓家奴殿前都點檢,封荊王。蕭惟信、耶律馮家奴並加太子太傅。宿衛官蕭乙辛、回鵑海鄰、褭里,耶律撻不也、阿斯、宮分人急裡哥、霞抹、乙辛、只魯並
 加上將軍。諸護衛及士卒、皰夫、彎手、傘子等三百餘人,各授官有差。耶律良密告重元變,命籍橫帳夷離堇房,為漢人行宮都部署。癸亥,貼不訴為重元等所脅,詔削爵為民,流鎮州。戊胡,以黑白羊祭天。



 八月庚午朔,遣使安撫南京吏民。癸酉,以永興宮使耶律塔不也有定亂功,為同知點檢司事。



 冬十月戊辰朔,幸興王寺。庚午,以六院部太保耶律合術知院大王事。是月,駐蹕藕絲澱。



 十一月辛丑,以甫院宣徽使蕭九哥為北府宰相。己未,追封故富春郡王耶律義先為許王。



 是歲,封皇子浚為梁王。



 十年春正月己亥,北幸。



 二月,禁南京民決水種粳稻。



 秋七月壬申,詔決諸路囚。辛巳,禁僧尼私詣行在,妄述禍福取財物。



 九月壬寅,幸懷州,謁太宗、穆宗廟。冬十月壬辰朔,駐蹕中京。戊午,禁民私刊印文字。



 十一月甲子,定吏民衣服之制。辛未,禁六齋日屠殺。丁丑,詔求乾文閣所閥經籍,命儒臣校仇。庚辰,以彰國軍節度使韓謝十為惕隱。詔南京不得私造御用彩緞,私貨鐵,及非時餘酒。命南京三司,每歲春秋以官錢饗將士。



 十二月癸巳,以北院大王蕭兀古匿為契丹行宮都部署。



 是歲,甫京、西京大熟。



 咸雍元年春正月辛酉朔,文武百僚加上尊號曰聖女神武全功大略廣智總仁睿孝天祐皇帝。改元,大赦。冊梁王浚為皇太子,內外官賜級有差。甲子,如魚兒濼。庚寅,詔諸遇正旦、重午、冬至,別表賀東宮。



 三月丁亥,以知興中府事楊績知樞密院事。



 夏四月辛卯,以知樞密院事張嗣復疾,改知興中府事。庚子,清暑拖古烈。



 五月辛巳,夏國遣使來貢。



 秋七月丙子,以皇太後射獲熊,賞齎百官有差。



 八月丙申,客星犯天廟,詔諸路備資賊,嚴火禁。



 九月乙亥,駐蹕藕系澱。丁丑,左夷離畢慥古為孟父敞穩。



 冬十月丁亥朔,幸醫巫閭山。己亥,皇太後射獲虎,
 大宴群臣,令備賦詩。



 十一月壬戌,有星如斗,逆行,隱隱有聲十二月甲午,以遼王仁先為南京留守,徒封晉王。辛亥,以南京留守蕭惟信為左夷離畢。壬子,熒惑與月並行,自衛至午。



 二年春正月丁巳,如鴨子河。宋賀正使王嚴卒,以禮送還。



 癸未,幸山榆澱。



 二月甲午,詔武定軍節度使姚景行,問以治道,拜南院樞密使。三月辛巳,以東北路詳穩耶律韓福奴為北院大王。壬午,彗星見於西方。



 夏四月,霖雨。



 五月乙亥,駐蹕拖古烈。辛巳,以戶部使劉詵為樞密副使。



 六月丙戌,回鶻來貢。甲辰,阻卜來貢。



 秋七月癸丑
 朔,以西北路招討使蕭術者為北府宰相,左夷離畢蕭惟信南院樞密使,同知南院樞密使事耶律白惕隱。丙辰,南院樞密使姚景行致仕。庚申,錄囚。辛酉,景行復前職。丁卯,如藕絲澱。以歲旱,遣使振山後貧民。



 九月壬子朔,日有食之。以參知政事韓孚為樞密副使。



 冬十二月壬午,以知樞密院事楊績為南院樞密使,樞密副使劉詵參知政事。戊子,僧守志加守司徒。丁酉,以西京留守合術為而院大王。辛丑,以蕭術者為武定軍節度使。



 是年,御永安殿放進士張臻等百一人。



 三年春正月辛亥,如鴨子河。甲子,御安流殿鉤魚。



 三月
 癸亥,宋主曙殂,子(王貢)嗣位,遣使告哀;即遣右護衛太保蕭撻不也、翰林學士陳覺等吊祭。



 閏月丁亥,扈駕軍營火,賜錢、粟及馬有差。辛卯,駐蹕春州北澱。乙巳,以蕭兀古匿為北府宰相。



 夏五月壬辰,駐蹕納葛濼。壬寅,賜隨駕官諸工人馬。



 六月戊申,有司奏新城縣民楊從謀反,偽署官吏。上曰:「小人無知,此兒戲爾。」獨流其首惡,餘釋之。庚戌,宋遣使饋其先帝遺物。辛亥,宋以即位,遣陳襄來報,即遣知黃龍府事蕭圖古辭、中書舍人馬鉉往賀。壬戌,商府宰相韓王蕭唐古致仕。壬申,以廣德軍節度使耶律蕊奴為商府宰相,度文使趙徽參知政事。秋七
 月辛丑,熒惑晝見,凡三十五日。



 九月戊戌,詔給諸路囚糧。癸卯,幸南京。



 冬十一月壬辰,夏國遣使進回鶻僧、金佛《梵覺經》。



 十二月丁未,以參知政事劉詵為樞密副使,東北路詳穩高八南院大王,樞密直學士張孝傑參知政事。己酉,以張孝傑同知樞密院事。丁巳,行再生禮,赦死罪以下。是月,夏國王李諒祚薨。



 是歲,南京旱、蝗。



 四年春正月甲戌朔,日有食之。丙子,如鴛鴦濼。辛巳,改易州兵馬使為安撫使。丁亥,獵炭山。辛卯,遣使振西京饑民。



 二月甲辰朔,詔元帥府募軍。壬子,夏國王李諒祚子秉常遣使告哀。癸丑,頒行《御制華嚴經贊》。丁卯,北行。



 三月丙子,遣使夏國吊祭。甲申,振應州饑民。乙酉,詔南京除軍行地,餘皆得種稻。庚寅,振朔州饑民。乙未,夏國李秉常遣使獻其父諒祚遺物。



 夏四月戊午,阿薩蘭回鶻遣使來貢。



 五月丙戌,駐蹕拖古烈。



 六月壬子,西北路雨穀,方三十里。丙寅,以北院林牙耶律趙三為北院大王,右夷離畢蕭素颯中京留守。



 秋七月壬申,置烏古敵烈部都統軍司。丙子,獵黑嶺。是月,南京霖雨,地震。



 九月己亥,駐踩藕絲澱。



 冬十月辛亥,曲赦南京徒罪以下囚。永清、武清、安次、固安、新城、歸義、容城諸縣水,復一歲租。戊辰,冊李秉常為夏國王。



 十二月辛亥,夏國遣使來貢。
 五年春正月,阻卜叛,以晉王仁先為西北路招討使,領禁軍討之。



 夏六月己亥,駐蹕拖古烈。丙午,吐蕃遣使來貢。壬戌,以南院樞密使蕭惟信知北院樞密使事。



 秋七月乙丑朔,日有食之。戊辰,夏國遣使來謝封冊。癸未,詔禁皇族恃勢侵漁細民。



 八月,謁慶陵。



 九月戊辰,仁先遣人奏阻卜捷。



 冬十月乙亥,駐蹕藕絲澱。



 十一月丁卯,詔四方館副使止以契丹人充。丁丑,五國剖阿里部叛,命蕭素颯討之。



 閏月戊申,夏國王李秉常遣使乞賜印綬。己未,僧志福加守司徒。



 十二月甲子,行皇太子再生禮,減諸路徒以下罪一等。乙丑,詔百官廷議國政。甲戌,五
 國來降,仍獻方物。



 六年春正月甲午,如千鵝凍。



 二月丙寅,阻卜來朝,貢方物。



 夏四月癸未,西北路招討司以所降阻卜酋長至行在。



 五月甲辰,清暑拖古烈。甲寅,設賢良科,詔應是科者,先以所業十萬言進。



 六月辛巳,阻卜來朝。乙酉,以惕隱耶律白為中京留守。



 是月,御永安殿放進士趙廷睦等百三十八人。



 秋七月辛亥,獵於合魯聶特。



 八月丙子,耶律白薨,追封遼西郡王。



 九月康戌,幸藕絲澱。甲寅,以馬希白詩才敏妙,十吏書不能給,召試之。



 冬十月丁卯,五國部長來朝。壬申,西北路招討司擒阻卜酋長來獻。



 十
 一月乙卯,禁鬻生熟鐵於回鶻、阻卜等界。



 十二月戊午,加圓釋、法鈞二僧並守司空。乙未,以坤字節,赦死罪以下。辛酉,禁漢人捕獵。



 七年春正月戊子,如鴨子河。



 二月乙丑,女直進馬。丙寅,以南院樞密使姚景行知興中府事。三月己酉,以討五國功,加知黃龍府事蒲延、懷化軍節度使高元紀、易州觀察使高正並千牛衛上將軍,五國節度使蕭陶蘇斡、寧江州防禦使大榮並靜江軍節度使。幸黑水。



 夏四月癸酉,如納葛濼。乙亥,禁布帛短狹不中尺度者。



 六月乙卯,吐蕃來貢。癸未,南院大王高八致仕。



 秋七月甲申朔,
 以東北路詳穩合里只為南院大王,西南面招討使抬得奴為莢六部大王。己丑,遣使按問五京囚。庚子,如藕絲澱。



 八月辛巳,置佛骨於招仙浮圖,罷獵,禁屠殺。



 冬十月己卯,如醫巫閭山。壬戌,以南府宰相耶律蕊奴為南京統軍使。戊辰,謁乾陵。庚辰,詔百官廷議軍國事。



 十一月戊子,免南京流民租。乙丑,振饒州饑民。丙午,高麗遣使來貢。



 十二月壬子,以契丹行宮都部署耶律胡睹知北院樞密使事,知北院樞密使事蕭惟信為南府宰相,兼契丹行宮都部署。



 丁巳,漢人行宮都郡署李仲禧、北院宣微使劉霂、樞密副使王觀、都承旨楊興工各賜國
 姓。戊寅,回鴿來貢。



 是歲,春州斗粟六錢。



\end{pinyinscope}