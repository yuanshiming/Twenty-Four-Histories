\article{卷四十志第十 地理志四 南京道}

\begin{pinyinscope}

 商京析津府,本古冀州之地。高陽氏謂之幽陵,陶唐曰幽都,有虞析為幽州。商並幽於冀。周分並為幽。《職方》,東北幽州,山鎮醫巫閭,澤藪畦養,川河、泲,浸葘、時。其利魚、鹽,其畜馬、牛、豕,其穀黍、稷、稻。武王封太保奭於燕。



 泰以其地為漁陽、上谷、石北平、遼西、遼東正郡。漢為燕國。



 歷
 封臧荼、盧綰、劉建、劉澤、劉旦,嘗置涿郡廣陽國。後漢為廣平國廣陽郡;或合於上谷,復置幽州。後周置燕及範陽郡,隋為幽州總管。唐置大都督府,改範陽節度使。安祿山、史思明、李懷仙、朱滔、劉怦、劉濟相繼割據。劉總歸唐。至張仲武、張九仲,以王得民。劉仁恭父子憎爭,遂入五代。自唐而晉,高祖以遼有援立之勞,割幽州等十六州以獻。太宗升為南京,又曰燕京。



 城方三十六里,崇三丈,衡廣一丈五尺。敵樓、戰櫓具。



 八門:東曰安東、迎春,南曰開陽、丹鳳,西曰顯西、清晉,北曰通天、拱辰。大內在西南隅。皇城內有景宗、聖宗御容殿二,東曰宣和,南曰大
 內。內門曰宣教,改元和;外三門曰南端、左掖、右掖。左掖改萬春,右掖改千秋。門有樓閣,球場在其南,東為永平館。皇城西門曰顯西,設而不開;北曰子北。



 西域巔有涼殿,東北隅有燕角樓。坊市、廨舍、寺觀,蓋不勝書。其外,有居庸、松亭、榆林之關,古北之口,桑乾河、高梁河、石子河、大安山、燕山——中有瑤嶼。府曰幽郡,軍號盧龍,開泰元年落軍額。



 統州六、縣十一:析津縣。本晉薊縣,改薊北縣,開泰元年更今名。以燕分野旅寅為析木之津,故名。戶二萬。



 宛平縣。本晉幽都縣,開泰元年改今名。戶二萬二干。



 昌平縣。本漢軍都縣,後漢屬廣陽郡,晉屬燕國,元魏置東燕州、平昌郡及昌平縣。郡廢,縣隸幽州。在京北九十里。



 戶七千。



 良鄉縣。燕為中都縣,漢改良鄉縣,舊屬涿郡,北齊天保七年省入薊縣,武平六年復置。唐聖歷元年改固市鎮,神龍元年復為良鄉縣,劉守光徙治此。在京南六十里。戶七千。



 潞縣。本漢舊縣,屬漁陽郡。唐武德二年置元州,貞觀元年州廢,復為縣。有潞水。在京東六十里。廣六千。



 宋次縣。本漢舊縣,屈漁陽郡。唐武德四年徒置東南
 五十里石梁城,貞觀八年又徙今縣西五里常道城,開元二十三年又徙耿就橋行市南。在京南一百二十里。戶一萬二千。



 永清縣。本漢益昌縣,隨置通澤縣,唐置武隆縣,改會昌,天寶初為永清縣。在京南一百五十里。戶五千。



 武清縣。前漢雍奴縣,屬漁陽郡。《水經注》,雍奴者,藪澤之名,四面有水曰雍,不流曰奴。唐天寶初改武清。在東京商一百五十里。戶一萬。



 香河縣。本武清孫村。遼於新倉置榷鹽院,居民聚集,因分武清、三河、潞三縣戶置。在京東南一百二十里。
 戶七千。



 玉河縣。本泉山地。劉仁恭於大安山創宮觀,師煉丹羽化之術於方士王若訥,因割薊縣分置,以供給之。在京西四十里。



 戶一千。



 漷陰縣。本漢泉山之霍村鎮。遼每季春,弋獵於延芳澱,居民成邑,就城故漷陰鎮,後改為縣。在京東南九十里。延若澱方數百里,春時鵝鶩所聚,夏秋多菱芡。國主春獵,衛士皆衣墨綠,各持連錘、鷹食、刺鵝錐,列水次,相去五七步。上風擊鼓,驚鵝稍離水面。國主親放海東青鶻擒之。鵝墜,恐鶻力不勝,在列者以佩錐
 刺鵝,急取其腦飼鴨。得頭鵝者,例賞銀絹。國主、皇族、群臣各有分地。戶五千。



 宋王曾《上契丹事》曰:自雄州白溝驕渡河,四十里至新城縣,古督亢亭之地。又七十里至涿州。北渡範水、劉李河,六十里至良鄉縣。渡盧溝河,六十里至幽州,號燕京。子城就羅郭西南為之。正南曰啟夏門,內有元和殿,東門曰宣和。城中坊閈皆有樓。有閔忠寺,本唐太宗為征遼陣亡將士所造;又有開泰寺,魏王耶律漢寧造。皆遣朝使游觀。南門外有於越王廨,為宴集之所。門外永平館,舊名碣石館,謂和後易之。南即桑乾河。



 順州,歸化軍,中,刺史。秦上谷,漢範陽,北齊歸德郡境。隨開皇中,粟末靺鞨與高麗戰不勝,厥稽部長突地稽率八郡勝兵數千人,自扶餘城西北舉落內附,置順州以處之。唐武德初改燕州,會昌中改歸順州,唐末仍為順州。有溫渝河;白遂河;曹王山,曹操嘗駐軍於此;黍谷山,鄒衍吹律之地,南有齊長城。城東北有華林、天柱二莊,遼建涼殿,春賞花,夏納涼。初軍曰歸寧,後更名。統縣一:懷柔縣。唐貞觀六年置,治五柳城,改順義縣。開元四年置松漠府彈汗州。天寶元年改歸化郡。乾元
 元年復今名。戶五千。



 檀州,武威軍,下,刺史。本燕漁陽郡地,漢為白檀縣。



 《魏書》,曹公歷白檀,破烏龍於柳城。《續漢書》,自檀在右北平。元魏創密雲郡,兼置安州。後周改為元州。隋開皇十八年割燕樂、密雲二縣置檀州。唐天寶元年改密雲郡,乾元元年復為檀州。遼加今軍號。有桑溪、鮑丘山、桃花山、螺山。統縣二:密雲縣。本漢白檀縣,後漢以居憕奚。元魏置密雲郡,領白檀、要陽、密雲三縣。高齊廢郡及二縣,來屬。戶五千。
 行唐縣。本定州行唐縣。太祖掠定州,破行唐,盡驅其民,北至檀州,擇曠土居之,凡置十寨,仍名行唐縣。隸彰愍宮。戶三千。



 涿州,永泰軍,上,刺史。漢高祖六年分燕置涿郡,魏文帝改範陽郡,晉為範陽國,元魏復為郡。隋開皇二年罷郡,屬幽州,大業三年以幽州為涿郡。唐武德元年郡廢,為涿縣,七年改範陽縣,大歷四年置涿州。石晉以歸太宗。有大房山、六聘山、涿水、樓桑河、橫溝河、禮遜河、祁溝河。統縣四:範陽縣。本漢涿縣。唐武德中,改範陽縣。有涿水、範
 水。



 戶一萬。



 固安縣。本漢方城縣,先屬廣陽國。隋開皇九年,自易州淶水縣移置,屬幽州,取漢故安縣名。唐武德四年屬北義州,徙治章信堡。貞觀二年義州廢,移今治,復屬幽州。在州東南九十里。戶一萬。



 新城縣。本漢新昌縣。唐大歷四年忻田安縣置,後省。後唐天成四年復析範陽縣置。在州南六十里。戶一萬。



 歸義縣。本漢易縣地。齊並入緀縣。唐武德五年置北義州,州廢,復置縣來屬。民居在巨馬河南,僑治
 新城。戶四千。



 易州,高陽軍,上,刺史。漢為易、故安二縣地。隨置易州,隨末為上谷郡。唐武德四年復易州,天寶元年仍上谷郡。



 乾元元年又改易州。五代隸定州節度使。會同九年孫方簡以其地來附。應歷九年為周世宗所取,後屬宋。統和七年攻克之,升高陽軍。有易水、淶水、狼山、太寧山、白馬山。統縣三:易縣。本漢縣,故城在今縣東南六十里。齊天保七年省。



 隋開皇十六年,於故安城西北隅置縣,即今縣治也。戶二萬五千。



 淶
 水縣。本漢道縣,今縣北一里故道城是也。元魏移於故城南,日口今縣置。周大象二年省。隋開皇十八年改淶水縣。



 在州東四十里。有淶水。戶二萬七千。



 容城縣。本漢縣,先屬涿郡,故城在雄州西南。唐武德五年屬北義州。貞觀元年還本屬。聖歷二年改全忠縣。天寶元年復名容城縣。在州東八十里。戶民皆居巨馬河南,僑治涿州新城縣。戶五千。



 薊州,尚武軍,上,刺史。秦漁陽、右北平二郡地。隋開皇中徒治玄州總管府,場帝改漁陽郡。唐武德元年廢入幽州,開元十八年分立薊州。統縣三:
 漁陽縣。本漢縣,屬漁陽郡。晉省,復置。元魏省。唐屬幽州,開元十八年置薊州。有鮑丘水。戶四千。



 三河縣。本漢臨胸縣地,唐開元四年忻潞州置。戶三千。



 玉田縣。本春秋無終子國。漢置無終縣,屬右北平郡。魏屬漁陽郡治,省,唐武德二年復置。貞觀初省,乾封中復置。



 萬歲通天元年更名玉田,屬營州。開元四年還屬幽州。八年屬營州。十一年又屬幽州。十八年來屬。《搜神記》:「雍伯,洛陽人,性孝,父母沒,葬無終山。山高八十里,上無水,雍伯置飲。人就有
 飲者,與石一斗,種生玉,因名玉田。」戶三千。



 景州,清安軍,下,刺史。本薊州遵化縣,重熙中置。戶三千。遵化縣,本唐平川買馬監,為縣來屬。



 平川,遼興軍,上,節度。商為孤竹國,春秋山戎國。秦為遼西、有北平二郡地,漢因之。漢末,公孫度據有,傳子康、孫淵,入魏。隨開皇中改平州。大業初夏為郡。唐武德初改州,天寶元年仍北平郡。後唐復為平州。太祖天贊二年取之,以定州俘戶錯置其地。統州二、縣三:盧龍縣。本肥如國。春秋晉滅肥,肥子奔燕,受封於此。



 漢、晉屬遼西郡。元魏為郡治,兼立平川。北齊屬北平郡。隋開皇中,省肥如,入新昌。十八年改新昌曰盧龍。唐為平州,後因之。戶七千。



 安喜縣。本漢令支縣地,久廢。太祖以定州安喜縣俘戶置。



 在州東北六十里。戶五千。



 望郡縣。本漢海陽縣,久廢。太祖以定州望都縣俘戶置。



 有海陽山。縣在州商三十里。戶三千。



 灤州,永安軍,中,刺史。本古黃洛城。灤河環繞,在盧龍山南。齊桓公伐山戎,見山神俞鬼,即此。秦為右北平。漢為石城縣,後名海陽縣,漢水為公孫度所有。晉以
 後屈遼西。



 石晉割地,在平川之境。太祖以俘戶置。灤州負山帶河,為朔漢形勝之地。有扶蘇泉,甚甘美,秦太子扶蘇北築長城嘗駐此;臨榆山,峰巒崛起,高千餘仞,下臨渝河。統縣三:義豐縣。本黃洛故城。黃洛水北出盧龍山,南流入於濡水。



 漢屬遼西郡,久廢。唐季入契丹,世宗置縣。戶四千。



 馬城縣。本盧龍縣地。唐開元二十八年析置縣,以通水運。



 東北有千金冶,東有茂鄉鎮。遼割隸灤州。在州西南四十里。



 戶三千。
 石城縣。漢置,屬石北平郡,久廢。唐貞觀中於此置臨渝縣,萬歲通天元年改右城縣,在灤州南三十里,唐儀鳳石刻在焉。今縣又在其南五十里,遼徙置以就鹽官。戶三千。營州,鄰海軍,下,剌史。本商孤竹國。秦屬遼西郡。漢為昌黎郡。



 前燕慕容醊徙都於此。元魏立營州,領昌黎、建德、遼東、樂浪、冀陽、營丘六部。後周為高寶寧所據。隋開皇置州,大業改遼西郡。唐武德元年改營州,萬歲通天元年始入契丹。聖歷二年僑治漁陽。開元五年還治柳城。天寶元年改曰柳城郡。後唐復為營州。
 太祖以居定州俘戶。統縣一:%廣寧縣。漢柳城縣,屬遼西郡。東北與奚、契丹接境。萬歲通天元年,入契丹李萬榮。神龍元年移幽州界。開元四年復舊地。遼改今名。戶三千。



\end{pinyinscope}