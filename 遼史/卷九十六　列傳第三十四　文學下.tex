\article{卷九十六 列傳第三十四 文學下}

\begin{pinyinscope}

 王鼎耶律昭劉輝耶律孟簡耶律穀欲王鼎,字虛中,涿州人。幼好學,居太寧山數年,博通經史。時為唐俊有文名燕、薊間,適上巳,與同志祓禊水濱,酌酒賦詩。鼎偶造席,唐俊見鼎樸野,置下坐。欲以詩困之,先出所作索賦,鼎援筆立成。唐俊驚其敏妙,因與定交。



 清寧五年,擢進士第。調易州觀察判官,改淶水縣令,累遷翰林學士。當代典章多出其手。上書言治道十事,帝
 以鼎達政體,事多咨訪。鼎正直不阿,人有過,必面詆之。



 壽隆初,升觀書殿學士。一日宴主第,醉與客忤,怨上不知己,坐是下吏。狀聞,上大怒,杖黥奪官,流鎮州。居數歲,有赦,鼎獨不免。會守臣召鼎為賀表,因以詩胎使者,有「誰知天雨露,獨不到孤寒」之句。上聞而憐之,即召還,復其職。



 乾統六年卒。



 鼎宰縣時,憩於庭,俄有暴風舉臥榻空中。鼎無懼色,但覺枕榻俱高,乃曰:「吾中朝端士,耶無干正,可徐置之。」



 須臾,榻發故處,風遂止。



 耶律昭,字述寧,博學,善屬文。統和中,坐兄國留事,流西北部。



 會蕭撻凜為西北路招討使,愛之,奏免其役,禮致
 門下。



 欲召用,以疾辭。撻凜問曰:「今軍旅甫罷,三邊宴然,惟阻卜伺隙而動。討之,則路遠難至;縱之,則邊民被掠;增戍兵,則饋餉不給;欲茍一時之安,不能終保無變。計將安出?」昭以書答曰:竊聞治得其要,則仇敵為一家;先其術,則部曲為行路。



 夫西北諸部,每當農時,一夫為偵候,一夫治公田,二夫給扎官之役,大率四丁無一室處。芻牧之事,仰給妻孥。一遭寇掠,貧窮立至。春夏賑恤,吏多雜以糠粃,重以掊克,不過數月,又得告困。且畜牧者,富國之本。有司防其隱沒,聚之一所,不得各就水草便地。兼以通亡戍卒,隨時補調,不習風土,故日疾月損,馴
 到耗竭。



 為今之計,莫若振窮薄賦,給以牛種,使遂耕稼。置游兵以防盜掠,頒俘獲以助伏臘,散畜牧以就便地。期以數年,富職可望。然後練簡精兵,以備行伍,何守之不固,何勸而不克哉?然必去其難制者,則餘種自畏。若舍大而謀小,避強而攻弱,非徒虛費財力,亦不足以威服其心。此二者,利害之機,不可不察。



 昭聞古之名將,安邊立功,在德不在眾。故謝玄以八千破苻堅百萬,休哥以五隊敗曹彬十萬。良由恩結士心,得其死力也。閣下膺非常之遇,專方面之寄,宜遠師古人,以就勛業。



 上觀乾象,下盡人謀;察地形之險易,料敵勢之虛實。慮無遺
 策,利施後世矣。撻凜然之。



 開泰中,獵於拔里堵山,為羯羊所觸,卒。



 劉輝,好學善屬文,疏簡有遠略。大康五年,第進士。



 大安未,為太子洗馬,上書言:「西邊諸番為患,士卒遠戍,中國之民疲於飛輓,非長久之策。為今之務,莫若城於鹽濼,實以漢戶,使耕田聚糧,以為西北之費。」言雖不行,識者韙之。



 壽隆二年,復上書曰:「宋歐陽修編《五代史》,附我朝於四夷,妄加貶訾。且宋人賴我朝寬大,許通和好,得盡兄弟之禮。今反令臣下妄意作史,恬不經意。臣請以趙氏初起事跡,詳附國史。」上嘉其言,遷禮部郎中。



 詔以賢
 良對策,輝言多中時病。擢史館修撰,卒。



 耶律孟簡,字復易,於越屋質之五世孫。父劉家奴,官至節度使。



 孟簡性穎悟。六歲,父晨出獵,俾賦《曉天星月詩》,孟簡應聲而成,父大奇之。既長,善屬文。大康初,樞密使耶律乙辛以奸險竊柄,出為中京留守,孟簡與耶律庶箴表賀。未幾,乙辛復舊職,銜之,謫巡磁窯關。時雖以讒見逐,不形辭色。



 遇林泉勝地,終日忘歸。明年,流保州。及聞皇太子被害,不勝哀痛,以詩傷之,作《放懷詩》二十首。自序云:「禽獸有哀樂之聲,螻蟻有動靜之形。在物猶然,況於人乎?然賢達哀樂,不在窮通、禍福之間。《易》曰:『樂天
 知命,故不憂。』是以顏淵簞瓢自得,此知命而樂者也。予雖流放,以道自安,又何疑耶?」



 大康中,始得歸鄉里。詣闕上表曰:「本朝之興,幾二百年,宜有國史以垂後世。」乃編耶律曷魯、屋質、休哥三人行事以進。上命置局編修。孟簡謂餘官曰:「史筆天下之大信,一言當否,百世從之。茍無明識,好惡徇情,則禍不測。故左氏、司馬遷、班固、範曄俱罹殃禍,可不慎歟!」



 乾統中,遷六院部太保。處事不拘文法,時多笑其迂。孟簡聞之曰:「上古之時,無簿書法令,而天下治。蓋簿書法令,適足以滋奸倖,非聖人致治之本。」改高州觀察使,修學校,招生徒。遷昭德軍節度使。以
 中京饑,詔與學士劉嗣昌減價糴粟。事未畢,卒。



 耶律穀欲,字休堅,六院部人。父阿古只,官至節度使。



 穀欲沖澹有禮法,工文章。統和中,為本部太保。開泰中,稍遷塌母城節度使。鞫霸州疑獄,稱旨,授啟聖軍節度使。太平中,復為本部太保。謝病歸,俄擢南院大王。嘆風俗日頹,請老,不許。



 興宗命為詩友,數問治要,多所匡建。奉詔與林牙耶律庶成、蕭韓家奴編遼國上世事跡及諸帝《實錄》,未成而卒,年九十。論曰:「孔子言:『誦《詩》三百,授之以政,不達。雖多,亦奚以為?』王鼎忠直達政,劉輝侍育宮,建言國計,昭陳邊防利害,
 皆洞達闓敏。孟簡疾乙辛奸邪,黜而不怨。孰謂文學之士,無益於治哉。」



\end{pinyinscope}