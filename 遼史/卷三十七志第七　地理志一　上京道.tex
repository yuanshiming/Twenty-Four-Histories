\article{卷三十七志第七 地理志一 上京道}

\begin{pinyinscope}

 帝堯畫天下為九州。舜以冀、青地天,分幽、並、營,為州十有二。幽州在渤、碣之間,並州北有代、朔,營州東暨遼海。其地負山帶海,其民執干戈,奮武衛,風氣剛勁,自古為用武之地。太祖以迭刺部之眾代遙輦氏,起臨潢,建皇都;東並渤海,得城邑之居百有三。太宗立晉,有幽、涿、檀、薊、順、營、平、蔚、朔、雲、應、新、媯、儒、武、寰十六州,於是割古幽、
 並、營之境而跨有之。東朝高麗,西臣夏國,南子石晉而兄弟趙宋,吳越、南唐航海輸貢。嘻,其盛矣!



 遼國其先曰契丹,本鮮卑之地,居遼澤中;去榆關一千一百三十里,去幽州又七百一十四里。南控黃龍,北帶潢水,冷陘屏右,遼河塹左。高原多榆柳,下隰饒蒲葦。當元魏時,有地數百里。至唐,大賀氏蠶食扶餘、室韋、奚、靺鞨之區,地方二千餘里。貞觀三年,以其地置玄州。尋置松漠都督府,建八部為州,各置刺史:達稽部曰峭落,紇便部曰彈汗州,獨活部曰無逢州,芬阿部曰羽陵州,突便部曰日連州,芮奚部曰徒河州,墜斤部曰萬丹州,伏部曰匹黎、
 赤山二州。以大賀氏窟哥為使持節十州軍事。分州建官,蓋昉於此。迨於五代,闢地東西三千里。遙輦氏更八部曰互利皆部、乙室活部、實活部、納尾部、頻沒部、內會雞部、集解部、奚嗢部,屬縣四十有一。每部設刺史,縣置令。太宗以皇都為上京,升幽州為南京,改南京為東京,聖宗城中京,興宗升雲州為西京,於是五京備焉。又以征伐俘戶建州襟耍之地,多因舊居名之;加以私奴置投下州。總京五,府六,州、軍、城百五十有六,縣二百有九,部族五十有二,屬國六十。東至於海,西至金山,暨於流沙,北至臚朐河,南至白溝,幅員萬里。



 上京道上京臨潢府,本漢遼東郡西安平之地。新莽曰北安平。太祖取天梯、蒙國、別魯等三山之勢於葦甸,射金齪箭以識之,謂之龍眉宮。神冊三年城之,名曰皇都。天顯十三年,更名上京,府曰臨潢。



 淶流河自西北南流,繞京三面,東入於曲江,其北東流為按出河。又有御河、沙河、黑河、潢河、鴨子河、他魯河、狼河、蒼耳河、輞子河、臚朐河、陰涼河、豬河、鴛鴦湖、興國惠民湖、廣濟湖、鹽濼、百狗濼、火神澱、馬盂山、兔兒山、野鵲山、鹽山、鑿山、松山、平地松林、大斧山、刊出、屈劣山、勒得山——唐所封大賀氏勒得王有墓存
 焉。



 戶三萬六千五百,轄軍、府、州、城二十五,統縣十:臨潢縣。太祖天贊初南攻燕、薊,以所俘大戶散居潢水之北,縣臨潢水,故以名。地宜種植。戶三千五百。



 長泰縣。本渤海國長平縣民,太祖伐大諲撰,先得是邑,遷其人於京西北,與漢民雜居。戶四千。



 定霸縣。本撫餘府強師縣民,太祖下扶餘,遷其人於京西,與漢人雜處,分地耕種。統和八年,以諸宮提轄司大戶置。隸長寧宮。戶二千。



 保和縣。本渤海國富利縣民,太祖破龍州,盡徒富利縣大散居京南。統和八年,以諸宮提轄司大戶置。隸
 彰愍宮。戶四千。



 潞縣。本幽州潞縣民,天贊元年,太祖破薊州,掠潞縣民,布於東京,與渤海人雜處。隸崇德宮。戶三千。



 易俗縣。本遼東渤海之民,太平九年,大延琳結構遼東夷叛,圍守經年,乃降,盡遷於京北,置縣居之。是年,又徒渤海叛人家屬置焉。戶一千。



 遷遼縣。本遼東諸縣渤海人,大延琳叛,擇其謀勇者置之左右。後以城降,戮之,徙其家於東東北,故名。戶一千。



 渤海縣。本東京人,因叛,徙置。



 興仁縣。開泰二年置。



 宣化縣。本遼東神化縣民,太祖破鴨淥府,盡徒其民居京之南。統和八年,以諸宮提轄司大戶置。隸彰愍宮。戶四千。



 上京,太祖創業之地。負山抱海,天險足以為固。地沃宜耕植,水草便畜牧。金齪一箭,二百年之基,壯矣。天顯元年,平渤海歸,乃展郛郭,建宮室,名以天贊。起三大殿:曰開皇、安德、五鸞。中有歷代帝王御容,每月朔望、節辰、忌日,在京文武百官並赴致祭。又於內城東南隅建天雄寺,奉安烈考宣簡皇帝遺像。是歲太祖崩,應天皇后於
 義節專斷腕,置太祖陵。



 即寺建斷腕樓,樹碑焉。太宗援立晉,遣宰相馮通、劉昫等恃節,具鹵簿、法服至此,冊上太宗及應天皇后尊號。太宗詔蕃部並依漢制,御開皇殿,闢承天門受禮,因改皇都為上京。



 城高二丈,不設敵樓,幅員二十七里。門,東曰迎春,曰雁兒;南曰順陽,曰南福;西曰金鳳,曰西雁兒。其北謂之皇城,高三丈,有樓櫓。門,東曰安東,南曰大順,西曰乾德,北曰拱辰。中有大內。內南門曰承天,有樓閣;東門曰東華,西曰西華。此通內出入之所。正南街東,留守詞衛,次鹽鐵司,次南門,龍寺街。南曰臨潢府,其側臨潢縣。縣西南崇孝寺,承天皇后
 建。寺西長泰縣,又西天長觀。西南國子監,臨北孔子廟,廟東節義寺。又西北安國寺,太宗所建。寺東齊天皇后故宅,宅東有元妃宅,即法天皇后所建也。其南貝聖尼寺,綾錦院、內省司、曲院,贍國、省司二倉,皆在大內西南,八作司與天雄專對。南城謂之漢城,南當橫街,各有樓對峙,下列井肆。東門之北潞縣,又東南興仁縣。南門之東回鶻營,回鶻商販留居上京,置營居之。西南同文驛,諸國信使居之。驛西南臨潢驛,以待夏國使。驛西福先寺。寺西宣化縣,西南定霸縣,縣西保和縣。西門之北易俗縣,縣東遷遼縣。



 周廣順中,胡嶠《記》曰:上京西樓,有邑屋市肆,交易無錢而用布。有綾錦諸工作、宦者、翰林、技術、教坊、角觝、儒、僧尼、道士。中國人並、汾、幽、薊為多。



 宋大中祥符九年,薛映《記》曰:上京者,中京正北八十里至松山館,七十里至崇信館,九十里至廣寧館,五十里至姚家寨館,五十里至咸寧館,三十里度潢水石橋,旁有饒州,唐於契丹嘗置饒樂州,今渤海人居之。五十里保和館,度黑水河,七十里宣化館,五十里長泰館。館西二十里有佛舍、民居,即祖州。又四十里至臨潢府。自過崇信館乃契丹舊境,其南奚地也。入西門,門曰金德,內有臨潢館。子城東門曰順
 陽。北行至景福門,又至承天門,內有昭德、宣政二殿與氈廬,皆東向。



 臨潢西北二百餘里號涼澱,在饅頭山南,避暑之處。多豐草,掘地文餘即有堅冰。



 祖州,天成軍,上,節度。本遼右八部世沒里地。太祖秋獵多於此,始置西樓。後因建城,號祖州。以高祖昭烈皇帝、曾祖莊敬皇帝、祖考簡獻皇帝、皇考宣簡皇帝所生之地,故名。



 城高二丈,無敵棚,幅員九里。門,東曰望京,南曰大夏,西曰液山,北曰興國。西北隅有內城。殿曰兩明,奉安祖考御容;曰二儀,以白金鑄太祖像;曰黑龍,曰清秘,各有太祖微時兵仗器物及服御皮毳之類,存之以示
 後嗣,使勿忘本。內南門曰興聖,凡三門,上有樓閣,東西有角樓。東為州廨及諸官廨舍,綾錦院,班院祗候蕃、漢、渤海三百人,供給內府取索。東南橫街,四隅有樓對峙,下連市肆。東長霸縣,西咸寧縣。有祖山,山有太祖天皇帝廟,御靴尚存。又有龍門、黎谷、液山、液泉、白馬、獨石、天梯之山。水則南沙河、西液泉。太祖陵鑿山為殿,曰明殿。殿南嶺有膳堂,以備時祭。門曰黑龍。東偏有聖蹤殿,立碑述太祖游獵之事。殿東有樓,立碑以紀太祖創業之功。皆在州西五里。天顯中太宗建,隸弘義宮。統縣二、城
 一:長霸縣。本龍州長平縣民,遷於此。戶二千。



 咸寧縣。本長寧縣。破遼陽,遷其民置。戶一千。



 越王城。太祖伯父於越王述魯西伐黨項、吐渾,俘其民放牧於此,因建城。在州東南二十里。戶一千。



 懷州,奉陵軍,上,節度。本唐歸誠州。太宗行帳放牧於此。天贊中,從太祖破扶餘城,下龍泉府,俘其人,築寨居之。



 會同中,掠燕、薊所俘亦置此。太宗崩,葬西山,曰懷陵。大同元年,世宗置州以奉焉。是年,有騎十餘,獵於祖州西五十里大山中,見太宗乘白馬,獨追白狐,射之,一發而斃;忽不見,但獲狐與失。是日,太宗崩於欒城。後於其地
 建廟,又於州之鳳凰門繪太宗馳騎貫狐之像。穆宗被害,葬懷陵側,建鳳凰殿以奉焉。有清涼殿,為行幸避暑之所。旨在州西二十里。



 隸永興宮。統縣二:扶餘縣。本龍泉府。太祖遷渤海扶餘縣降戶於此,世宗置縣。戶一千五百。



 顯理縣。本顯理府人,太祖伐渤海,俘其王大諲撰,遷民於此,世宗置縣。戶一千。



 慶州,玄寧軍,上,節度。本太保山黑河之地,巖谷險峻。



 穆宗建城,號黑河川,每歲來幸,射虎障鷹,軍國之事多委大臣,後遇弒於此。以地苦寒,統和八年,州廢。聖宗秋畋,
 愛其奇秀,建號慶州。遼國五代祖勃突,貌異常,有武略,力敵百人,眾推為玉。生於勃突山,因以名;沒,葬山下。在州二百里。慶雲山,本黑嶺也。聖宗駐蹕,愛羨曰:「吾萬歲後,當葬此。」興宗遵遺命,建永慶陵。有望仙殿、御容殿。置蕃、漢字陵三千戶,並隸大內都總管司。在州西二十里。有黑山、赤山、太保山、老翁嶺、饅頭山、興國湖、轄失濼、黑河。景幅元年復置,更隸興聖宮。統縣三:玄德縣。本黑山黑河之地。景福元年,括落帳人戶,從便居之。戶六千。



 孝安縣。



 富義縣。本義州,太宗遷渤海義州民於此。重熙元年降為義豐縣,後更名。隸弘義宮。



 泰州,德昌軍,節度。本契丹二十部族放牧之地。因黑鼠族累犯通化州,民不能御,遂移東南六百里來,建城居之,以近本族。黑鼠穴居,膚黑,吻銳,類鼠,故以名。州隸延慶宮,兵事屬東北統軍司。統縣二:樂康縣。倚郭。



 興國縣。本山前之民,因罪配遞至此,興宗置縣。戶七百。



 長春州,韶陽軍,下,節度。本鴨子河春獵之地。興宗重熙
 八年置。隸延慶宮,兵事隸東北統軍司。統縣一:長春縣。本混同江池。燕、薊犯罪者流配於此。戶二千。



 烏州,靜安軍,刺史。本烏丸之地,東胡之種也。遼北大王撥刺占為牧,建城,後宮收。隸興聖宮。有遼河、夜河、烏丸川、烏丸山。統縣一:愛民縣。撥刺王從軍南征,俘漢民置於此。戶一千。



 永州,永昌軍,觀察。承天皇太后所建。太祖於此置南樓。



 乾亨三年,置用於皇子韓八墓側。東潢河,南土河,二水合流,故號永州。冬月牙帳多駐此,謂之冬捺缽。有木葉山,上建契丹始祖廟,奇首可汗在南廟,可敦在北廟,繪
 塑二聖並八子神像。相傳有神人乘白馬,自馬盂山浮土河而東,有天女駕青牛車由平地松林泛潢河而下。至木葉山,二水合流,相遇為配偶,生八子。其後族屬漸盛,分為八部。每行軍及春秋時祭,必用白馬青牛,示不忘本云。興王寺,有白衣觀音像。太宗援石晉主中國,自潞州回,入幽州,幸大悲閣,指此像曰:「我夢神人令送石郎為中國帝,即此也。」因移木葉山,建廟,春秋告賽,尊為家神。興軍必告之,乃合符傳箭於諸部。又有高澱山、柳林澱,亦曰白馬澱。隸彰愍宮。統縣三:長寧縣。本顯德府縣名。太祖平渤海,遷其民於此。戶
 四千五百。



 義豐縣。本鐵利府義州。遼兵破之,遷其民於南樓之西北,仍名義州。重熙元年,廢州,改今縣。在州西北一百里。又嘗改富義縣,屬泰州。始末不可具考,今兩存之。戶一千五百。



 慈仁縣。太宗以皇子只撒古亡,置慈州墳西。重熙元年,州廢,改今縣。戶四百。



 儀坤州,啟聖軍,節度。本契丹右大部地。應天皇后建州。



 回鶻糯思居之,至四世孫容我梅里,生應天皇后述律氏,適太祖。太祖開拓四方,平渤海,後有力焉。俘掠有伎
 藝者多歸帳下,謂之屬珊。以所生之地置州。州建啟聖院,中為儀寧殿,太祖天皇帝、應天地皇后銀像在焉。隸長寧宮。統縣一:廣義縣。本回鶻部牧地。應天皇后以四征所俘居之,因建州縣。統和八年,以諸宮提轄司戶置來遠縣,十三年並入。戶二千五百。



 龍化州,興國軍,下,節度。本漢北安平縣地。契丹始祖奇首可汗居此,稱龍庭。太祖於此建東樓。唐天復二年,太祖為迭烈部夷離堇,破代北,遷其民,建城居之。明年,伐女直,俘數百戶實焉。天祐元年,增修東城,制度頗壯麗。
 十三年,太祖於城東金鈴岡受尊號曰大聖大明天皇帝,建元神冊。天顯元年,崩於東樓。太宗升節度。隸彰愍宮,兵事屬北路女直兵馬司。刺史州一,禾詳。統縣一:龍化縣。太祖東伐女直,南掠燕、薊,所俘建城置邑。戶一千。降聖州,開國軍,下,刺史。本大部落東樓之地。太祖春月行帳多駐此。應天皇后夢神人金冠素服,執兵仗,貌甚豐美,異獸十二隨之。中有黑兔躍入後懷,因而有娠,遂生太宗。時黑雲覆帳,火光照室,有聲如雷,諸部異之。穆宗建州。四面備三十里,禁樵採放牧。先屬延昌宮,後隸
 彰愍宮。統縣一:永安縣。本龍原府慶州縣名。太祖平渤海,破懷州之永安,遷其人置寨於此,建縣。戶八百。



 饒州,匡義軍,中,節度。本唐饒樂府地。貞觀中置松漠府。太祖完葺故壘。有潢河、長水濼、沒打河、青山、大福山、松山。隸延慶宮。統縣三:長樂縣。本遼城縣名。太祖伐渤海,遷其民,建縣居之。



 戶四千,內一千戶納鐵。



 臨河縣。本豐永縣大,太宗分兵伐渤海,遷於潢水之
 曲。戶一千。



 安民縣。太宗以渤海諸邑所俘雜置。戶一千。



 頭下軍州頭下軍州,皆諸王、外戚、大臣及諸部從征俘掠,或置生口,各團集建州縣以居之。橫帳諸王、國舅、公主許創立州城,自餘不得建城郭。朝廷賜州縣額。其節度使朝廷命之,刺史以下皆以本主部曲充焉。官位九品之下及井邑商賈之家,征稅各歸頭下;唯酒稅課納上京鹽鐵司。



 徽州,宣德軍,節度。景宗女秦晉大長公主所建。媵臣萬戶,在宜州之北二百里,因建州城。北至上京七百里。節度使以下,皆公主府署。戶一萬。



 成州,長慶軍,節度。聖宗文晉國長公主以上賜媵臣戶置。



 在宜州北一百六十里,因建州城。北至上京七百四十里。戶四千。



 懿州,廣順軍,節度。聖宗文燕國長公主以上賜媵臣戶置。



 在顯州東北二百里,因建州城。西北至上京八百里。戶四千。



 渭州,高陽軍,節度。駙馬都尉蕭昌裔建。尚秦國王隆慶女韓國長公主,以所賜媵臣建州城。顯州東北二百五十里。遼制,皇子嫡生者,其文與帝女同。戶一千。



 壕州。國舅宰相南征,俘掠漢民,居遼東西安平縣故地。



 在顯州東北二百二十里,西北至上京七百二十里。戶六千。原州。本遼東北安平縣地。顯州東北三百里。國舅金德俘掠漢民建城。西北至上京八百里。戶五百。



 福州。國舅蕭寧建。南征俘掠漢民,居北安平縣故地。在原州北二十里,西北至上京七百八十里。戶三百。



 橫州。國舅蕭克忠建。部下牧人居漢故遼陽縣地,因置州城。在遼州西北九十里,西北至上京七百二十里。有橫山。戶二百。鳳州:稿離國故地,渤海之安寧郡境,南王府五帳分地。



 在韓州北二百里,西北至上京九百里。戶四千。



 遂州。本高州地,商王府五帳放牧於此。在檀州西二百里,西北至上京一千里。戶五百。



 豐州。本遼澤大部落,遙輦氏僧隱牧地。北至上京三百五十里。戶五百。



 順州。本遼隊縣地。橫帳南王府俘掠燕、薊、順州之民,建城居之。在顯州東北一百二十里,西北至上京九百里。戶一千。



 閭州。羅吉王牧地,近醫巫閭山。在遼州西一百三十里,西北至上京九百五十里。戶一千。



 松山州。本遼澤大部落,橫帳普古王牧地。有松山。北至上京一百七十里。戶五百。



 豫州。橫帳陳王牧地。南至上京三百里。戶五百。



 寧州。本大賀氏勒得山,橫帳管寧王放牧地。在豫州東八十里,西南至上京三百五十里。戶三百。



 邊防城遼國西北邊界防邊城,因屯戍而立,務據形勝,不資丁賦。



 具列如左:靜州,觀察。本泰州之金山。天慶六年升。



 鎮州,建安軍,節度。本古可敦城。統和二十二年皇太妃
 奏置。選諸部族二萬餘騎充屯軍,專捍禦室韋、羽厥等國,凡有征討,不得抽移。渤海、女直、漢人配流之家七百餘戶,分居鎮、防、維三州。東南至上京三千餘里。



 維州,刺史。



 防州,刺史。河董城。本回鶻可敦城,語訛為河董城。久廢,遼人完之以防邊患。高州界女直常為盜,劫掠行旅,遷其族於此。東南至上京一千七百里。



 靜邊城。本契丹二十部族水草地。北鄰羽厥,每入為盜,建城,置兵千餘騎防之。東南至上京一千五百里。



 皮被河城。地控北邊,置兵五百於此防托。皮被河出回紇北,東南經羽厥,入臚朐河,沿河董城北,東流合沱漉河,入於海。南至上京一千五百里。



 招州,綏遠軍,刺史。開泰三年以女直戶置。隸西北路招討司。塔懶主城。大康九年置。在臚朐河。



\end{pinyinscope}