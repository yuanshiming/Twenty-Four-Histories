\article{卷四十一志第十一 地理志五 西京道}

\begin{pinyinscope}

 西京大同府,陶唐冀州之域。虞分並州。夏復屬冀州。周《職方》,正北曰並州。戰國屬趙,武靈王始置雲中郡。秦屬代王國,後為平城縣。魏屬新興郡。晉仍屬雁門。劉琨表封猗盧為代王,都平城。元魏通武於此遂建都邑。孝文帝改為司州牧,置代尹,遷都洛邑,改萬年,又置恆州。高
 齊文宣帝廢州為恆安鎮,今謂之東城,尋復恆州。周復恆安鎮,改朔州。隋仍為鎮。唐武德四年置北恆州,七年廢。貞觀十四年移雲中定襄縣於此。永淳元年默啜為民患,移民朔州。開元十八年置雲州。天寶元年改雲中郡。乾元元年曰雲州。乾符三年,大同軍節度使李國昌子克用為雲中守捉使,殺防禦使,據州以聞。僖宗赦克用,以國昌為大同軍防禦使,不受命。廣明元年,李琢攻國昌,國昌兵敗,與克用奔北地。黃巢入京師,詔發代北軍,尋赦國昌,使討賊。克用率三萬五千騎而南,收京師,功第一,國昌封隴西郡王。國昌卒,克用取雲州。既而
 所向失利,乃卑詞厚禮,與太祖會於雲州之東城,謀大舉兵攻梁,不果。克用於存勖滅梁,是為唐莊宗。同光三年,復以雲州為大同軍節度使。晉高祖代唐,以契丹有援立功,割山前、代北地為賂,大同來屬,因建西京。



 敵樓、棚櫓具。廣袤二十里。門,東曰迎春,南曰朝陽,西曰定西,北曰拱極。元魏宮垣占城之北面,雙閥尚在。遼既建都,用為重地,非親王不得主之。清寧八年建華嚴寺,奉安諸帝石像、銅像。又有天王寺、留守司衛,商曰西省。北門之東曰大同府,北門之西曰大同驛。初為大同軍節度,重熙十三年升為西京,府曰大同。



 統州二、縣七:
 大同縣。本大同川地。重熙十七年西夏犯邊,析雲中縣置。



 戶一萬。



 雲中縣。趙置。沿革與京府同。戶一萬。



 天成縣。本極塞之地。魏道武帝置廣牧縣,廟武德五年置定襄縣,遼析雲中置。在京北一百八十里。戶五千。



 長青縣。本白登臺地。冒頓單於縱精騎三十餘萬圍漢高帝於白登七日,即此。遼始置縣。有青陂。梁元帝《橫吹曲》云:「朝跋青陂,暮上白登。」在京東北一百一十
 里。戶四千。



 奉義縣。本漢陶林縣地。後廟武皇與太祖會此。遼析雲中置。戶三千。



 懷仁縣。本漢沙南縣。元魏葛榮亂,縣廢。隋開皇二年移雲內於此。大業二年置大利縣,置雲州,改屬定襄郡。隨末隋突厥。李克用敗赫連鐸,駐兵於此。遼改懷仁。在京南六十里。



 戶三千。



 懷安縣。本漢夷輿縣地。歷魏至隋,為突撅所據。唐克頡利,縣遂廢為懷荒鎮。高勛鎮燕,奏分歸化州文德縣置。初隸奉聖州,後來屬。在州西北二百八十里。廣三千。



 弘州,博寧軍,下,刺史。東魏靜帝置北靈丘縣。用初地陷突撅,開元中置橫野軍安邊縣,天寶亂廢,後為襄陰村。統和中,以寰州近邊,為宋將潘類所破,廢之仍於此置弘州,初軍曰永寧。有桑乾河、白道泉、白登山,亦曰火燒山,有火井。



 統縣二:永寧縣。戶一萬。



 順聖縣。本魏安塞軍,五代兵廢。商勛鎮幽州,奏景宗分永興縣置。初隸奉聖州。在州西北二百八十里。廣三千。



 德州,下,刺史。唐會昌中以西德店置德州。開泰八年
 以漢戶復置。有步落泉、金河山、野孤嶺、白道阪。縣一:宣德縣。本漢桐過縣地,屬雲中郡,後隸定襄郡,漢末廢。



 高齊置紫阿鎮。唐會昌中置縣。戶三千。



 豐州,天德軍,節度使。秦為上郡北境,漢屬五原郡。地磧鹵,少田疇。自晉永嘉之亂,屬赫連勃勃。後周置永豐鎮。



 隋開皇中升永豐縣,改豐州。大業七年為五原郡,義寧元年太守張遜奏改歸順邯。唐武德元年為豐州總管府。六年省,遷民於白馬縣,遂廢。貞觀四年分靈州境,置豐州都督府,領蕃戶。



 天寶初改九原郡。乾元元年復豐州,後入回鶻。會昌中克之,後唐改天德軍。太祖神冊五
 年攻下,更名應天軍,復為州。有大鹽灤、九十九泉、沒越灤、古磧口、青塚——即王昭君墓。



 兵事屬西南面招討司。統縣二:富民縣。本漢臨戎縣,遼改今名。戶一千二百。



 振武縣。本漢定襄郡盛樂縣。背負陰山,前帶黃河。元魏嘗都盛樂,即此。唐武德四年克突厥,建雲中都督府。麟德三年改單于大都督府。聖歷元年又改安北都督。開元七年割隸東受降城。八年置振武軍節度使。全昌五年為安北都護府。後唐莊宗以兄嗣本為振武節度使。太祖神冊元年,伐吐渾還,攻之,盡俘其
 民以東,唯存鄉兵三百人防戍。後更為縣。雲內州,開遠軍,下,節度。本中受降城地。遼初置代北雲朔招討司,改雲內州。清寧初升。有威塞軍、古可敦城、大同川、天安軍、永濟柵、安樂戍、拂雲堆。兵事屬西南面招討司。統縣二:柔服縣。



 寧人縣。



 天德軍,本中受降城。唐開元中廢橫塞軍,置天安軍於大同川。乾元中改天德軍,移永濟柵,今治是也。太祖平黨項,遂破天德,盡掠吏民以東。後置招討司,漸成井邑,
 乃以國族為天德軍節度使。有黃河、黑山峪、廬城、威塞軍、秦長城、唐長城;又有牟那山,鉗耳觜城在其北。



 寧邊州,鎮西軍,下,刺史。本唐隆鎮,遼置。兵事屬西南面招討司。



 秦聖州,武定軍,上,節度。本唐新州。後唐置團練使,總山後八軍,莊宗以弟存矩為之。軍亂,殺存矩於祁州,擁大將盧文進亡歸。太祖克新州,莊宗遣李嗣李源復取之。同光二年升威塞軍。石晉高祖割獻,太宗改升。有兩河會、溫泉、龍門山、涿鹿山。東南至南京三百里,西北至西京四百四十里。



 兵事屬西京都部署司。統州三、縣四:
 永興縣。本漢涿鹿縣地。黃帝興蚩尤戰於此。戶八千。



 礬山縣。本漢軍都縣。山出白綠礬,故名。有礬山、桑乾河。在州南六十里。戶三千。



 龍門縣。有龍門山,石壁對峙,高數百尺,望之若門。徼外諸河及沙漠潦水,皆於此趣海。雨則俄頃水逾十仞,晴則清淺可涉,實塞北控扼之沖要也。在州東北二百八十里。戶四千。



 望雲縣。本望雲川地。景守於此建潛邸,因而成井肆。穆宗崩,景宗入紹國統,號御莊。後置望雲縣,直隸彰愍宮,附庸於此。在州東北二百六十里。戶一千。



 歸化州,雄武軍,上,刺史。本漢下洛縣。元魏改文德縣。



 唐升武州,僖宗改毅州。後庸太宗復武州,明宗又為毅州,潞王仍為武州。晉高祖割獻於遼,改今名。有桑乾河;會河川;愛陽川;炭山,又謂之陘頭,有涼殿,承天皇后納涼於此,山東北三十里有新涼殿,景宗納涼於此,唯松柵數陘而已;斷雲嶺,極高峻,故名。州西北至西京四百五十里。統縣一:文德縣。本漢女祁縣地。元魏置。戶一萬。



 可汗州,清平軍,下,刺史。本漢潘縣,元魏廢。北齊置北燕郡,改懷戎縣。隋廢郡,屬涿郡。唐武德中復置北燕
 州,縣仍舊。貞觀八年改媯州。五代時,奚王去諸以數千帳徙媯州,自別為西奚,號可汗州;太祖因之。有媯泉在城中,相傳舜嬪二女於此。又有溫泉、版泉、磨並山、雞鳴山、喬山、歷山。



 統縣一:懷來縣。本懷戎縣,太祖改。戶三千。



 儒州,縉陽軍,中,刺史。唐置。後唐同光二年隸新州。



 太宗改奉聖州,仍屬。有南溪河、沽河、宋王峪、桃峪口。統縣一:縉山縣。本漢廣寧縣地。唐天寶中割媯川縣置。戶五千。



 蔚州,忠順軍,上,節度。周《職方》,並州川曰漚夷,在州境飛狐縣。趙襄子滅代;武靈王置代郡;項羽徙趙歇為代王;歇還趙,立陳餘王代;漢韓信斬餘,復置代郡;文帝初封代;皆此地。周宣帝始置蔚州,隋開皇中廢。唐武德四年復置。至德二年改興唐縣。乾元元年仍舊。大中後,朱邪執宜為刺史,有功,賜姓名李國昌。子克用乞為留後,僖宗不許。廣明初,攻敗國昌,代北無備,太祖來攻,克之,俘掠居民而去。石晉獻地,升忠順軍,後更武安軍。統和四年入宋,尋復之,降刺史,隸奉聖州,升觀察,復忠順軍節度。兵事屬西京都部署司。



 統縣五:
 靈仙縣。唐置興唐縣,梁改隆化縣,後唐同光初夏置,晉改今名。戶二萬。



 定安縣。本漢東安陽縣地,久廢。後唐太祖伐劉仁恭,次蔚州,晨霧晦冥,占,不利深入,會雷電大作,燕軍解去,即此。遼置定安縣。西北至州六十里。戶一萬。



 飛狐縣。後周大象二年置廣昌縣於五龍城,即此。隨仁壽元年改名飛狐。相傳有狐於紫荊嶺食五粒松子,成飛仙,故云。



 西北至州一百四十里。戶五千。



 靈丘縣。漢置。後漢省。東魏復置,屬靈丘郡。隋開皇中罷郡來屬。大業初改隸代州。唐武德六年仍舊。東北
 至州一百八十里。戶三千。



 廣陵縣。本漢延陵縣。隨唐為鎮州。後唐同光初分興唐縣置。石晉割屬遼。東南至州四十里。戶三千。應州,彰國軍,上,節度。唐武德中置金城縣,後改應州。後唐明宗,州人也。



 天成元年升彰國軍節度,興唐軍、寰州隸焉。遼因之。北龍首山,南雁門。兵事屬西京都部署司。統縣三。



 金城縣。本漢陰館縣地,漢末廢為陰館城。隨大業末陷突厥。唐始置金城縣,遼因之。戶八千。



 渾源縣。唐置。有渾源川。在州東南一百五十里。戶五
 千。



 河陰縣。本漢陰館縣地。初隸朔州,清寧中來屬。戶三千。



 朔州,順義軍,下,節度。本漢馬邑縣地。元魏孝文帝始置朔州,在今州北三百八十里定襄故城。葛榮亂,廢。高齊天保六年復置,在今州南四十七里新城。八年徙馬邑,即今城。



 武成帝置北道行臺。周武帝置朔州總管府。隨大業三年改馬邑郡。唐武德四年復朔州。遼升順義軍節度。兵事屬西京都部署司。統州一、縣三。



 鄯陽縣。本漢定襄縣地。建安中置新興郡。元魏置桑
 乾郡。



 高齊置招遠縣,郡仍舊。隨開皇三年罷郡,隸朔州。大業元年初名都陽縣,遼因之。戶四千。



 寧遠縣。齊天保六年,於朔州西置招遠縣。唐乾元元年改今名,遼因之。有寧遠鎮。



 東至朔州八十里。戶二千。



 馬邑縣。漢置,屬雁門郡。唐開元五年,析鄯陽縣東三十里置大同軍,倚郭置馬邑縣。商至朔州四十里。戶三千。



 武州,宣威軍,下,刺史。趙惠王置武川塞。魏置神武縣。



 唐末置武州。後唐改毅州。重熙九年復武州,號宣成軍。
 統縣一:神武縣。魏置。晉改新城。後唐太祖生神武川之新城,即此。初隸朔州,後置州,並寧遠為一縣來屬。戶五千。



 東勝州,武興軍,下,刺史。隋開皇七年置勝州。大業五年改榆林郡。唐貞觀五年於南河地置決勝州,故謂此為東勝州。



 天寶七年又為榆林郡。乾元元年復為勝州。太祖神冊元年破振武軍,勝州之民皆趨河東,州廢。晉割代北來獻,復置。兵事屬西南面招討司。統縣二:榆林縣。



 河濱縣。



 金肅州。重熙十二年伐西夏置。割燕民三百戶,防秋軍一千實之。屬西南面招討司。



 河清軍。西夏歸遼,開直路以趨上京。重熙十二年建城,號河清軍。徙民五百戶,防秋兵一千人實之。屬西南面招討司。



\end{pinyinscope}