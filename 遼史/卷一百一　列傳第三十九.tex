\article{卷一百一 列傳第三十九}

\begin{pinyinscope}

 伶官羅衣輕宦官王繼恩趙安仁伶,官之微者也。《五代史》列鏡新磨於《傳》,是必有所取矣。遼之伶官當時固多,然能因詼諧示諫,以消未形之亂,惟羅衣輕耳。孔子曰:「君子不以人廢言。」是宜傳。



 羅衣輕,不知其鄉里。滑稽通變,一時諧謔,多所規諷。



 興宗敗於李元昊也,單騎突出,幾不得脫。先是,元昊獲遼人,輒劓其鼻,有奔北者,惟恐追及。故羅衣輕止之曰:「且
 觀鼻在否?」上怒,以毳索擊帳後,將殺之。太子笑曰:「打諢底不是黃幡綽!」羅衣輕應聲曰:「行兵底亦不是唐太宗!」上聞而釋之。



 上嘗與太弟重元狎呢,宴酣,許以千秋萬歲後傳位。重元喜甚,驕縱不法。又因雙陸,賭以居民城邑。帝屢不竟,前後已償數城。重元既恃梁孝王之寵,又多鄭叔段之過,朝臣無敢言者,道路以目。一日復博,羅衣輕指其局曰:「雙陸休癡,和你都輸去也!」帝始悟,不復戲。清寧間,以疾卒。



 《周禮》,寺人掌中門之禁。至巷伯詩列於《雅》,勃貂功著於
 晉,雖忠於所事,而非其職矣。漢、唐中世,竊權蠹政,有不忍言者,是皆寵遇之過。遼宦者二人,其賢不肖皆可為後世鑒,故傳焉。



 王繼恩,棣州人。睿智皇后南征,繼恩被俘。



 初,皇后以公私所獲十歲已下兒容貌可觀者近百人,載赴涼陘,並使閹為豎,繼恩在焉。聰慧,通書及遼語。擢內謁者、內侍左廂押班。聖宗親政,累遷尚衣庫使、左承宣、臨門衛大將軍、靈州觀察使、內庫都提點。



 繼恩好清談,不喜權利,每得賜賚,市書至萬卷,載以自隨,誦讀不倦。每宋使來聘,繼恩多充宣賜使。後不知所終。



 趙安仁,字小喜,深州樂壽人,自幼被俘。



 統和中,為黃門令、秦晉國王府祗候。王薨,授內侍省押班、御院通進。開泰八年,與李勝哥謀奔南土,為游兵所擒。



 初,仁德皇后與欽哀有隙,欽哀密令安仁伺皇後動靜,無不知者。仁德皇后威權既重,安仁懼禍,復謀亡歸。仁德欲誅之,欽哀以言營救。聖宗曰:「小喜言父母兄弟俱在南朝,每一念,神魂隕越。今為思親,冒死而亡,亦孝子用心,實可憐憫。」



 赦之。重熙初,欽哀攝政,欲廢帝,立少子重元。帝與安仁謀遷太后慶州守陵,授安仁左承宣、監門衛大將軍,充契丹漢人渤海內侍都知,兼都提點。會上思太后,親
 馭奉迎,太后責曰:「汝負萬死,我嘗營救。不望汝報,何為離間我母子耶!」安仁無答。後不知所終。



 論曰:「名器所以礪天下,非賢而有功則不可授,況宦者乎。繼恩為內謁者,安仁為黃門令,似矣;何至溺於私愛,而授以觀察使、大將軍耶?《易》曰:『負且乘,致寇至。』此安仁所以不克有終,繼恩幸而免歟?」



\end{pinyinscope}