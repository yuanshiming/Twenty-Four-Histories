\article{卷一本紀第一 太祖上}

\begin{pinyinscope}

 太祖大聖大明神烈天皇帝,姓耶律氏,諱億,字阿保機,小字啜裡只,契丹迭刺霞瀨益石烈鄉耶律彌里人。德祖皇帝長子,母曰宣簡皇后蕭氏。唐咸通十三年生。初,母夢日墮懷中,有娠。及生,室有神光異香,體如三歲兒,即能匍匐。祖母簡獻皇后異之,鞠為己子。常匿於別幕,塗其面,不令他人見。



 三月能行;啐而能言,知未然事。自謂左右若有神人翼衛。雖齠齔,言必及世務。時伯父當國,疑輒咨焉。既長,身長九尺,豐上銳下,目光射人,關弓三百斤。為撻馬狘沙裏。時小黃室韋不附,太祖以計降之。伐越兀及烏古、六奚、比沙狘諸部,克之。國人號阿主沙裏。



 唐天復元年,歲辛酉,痕德堇可汗立,以太祖為本部夷離堇,專征討,連破室韋、于厥及奚帥轄刺哥,俘獲甚眾。冬十月,授大迭烈府夷離堇。



 明年秋七月,以兵四十萬伐河東代北,攻下九郡,獲生口九萬五千,駝、馬、牛、羊不可勝紀。九月,城龍化州於潢河之南,始建開教寺。



 明年春,伐女直,下之。獲其戶三百。九月,復攻下河東
 懷遠等軍。冬十月,引軍略至薊北,俘獲以還。先是德祖俘奚七千戶,徙饒樂之清河,至是創為奚迭刺部,分十三縣。遂拜太祖於越、總知軍國事。



 明年歲甲子,三月,廣龍化州之東城。九月,討黑車子室韋,唐盧龍軍節充使劉仁恭發兵數萬,遣養子趙霸來拒。霸至武州,太祖諜知之,伏勁兵桃山下。遣室韋人牟里詐稱其長酋所遣,約霸兵會平原。既至,四面伏發,擒霸,殲其眾,乘勝大破室韋。



 明年七月,復討黑車子室韋。唐河東節度使李克用遣通事康令德乞盟。冬十月,太祖以騎兵七萬會克用於雲州,宴酣,克用借兵以報劉仁恭木瓜澗之役,太
 祖許之。易袍馬,約為兄弟。及進兵擊仁恭,拔數州,盡徙其民以歸。



 明年二月,復擊劉仁恭。還,襲山北奚,破之。汴州朱全忠遣人浮海奉書幣、衣帶、珍玩來聘。十一月,遣偏師討奚、霫諸部及東北女直之未附者,悉破降之。十二月,痕德堇可汗殂,群臣奉遺命請立太祖。曷魯等勸進。太祖三讓,從之。



 元年春正月庚寅,命有司設壇於如迂王集會堝,燔柴告天,即皇帝位。尊母氏為皇太后,立皇后蕭氏。北宰相轄刺、南宰相耶律歐里思率群臣上尊號曰天皇帝,後曰地皇后。庚子,詔皇族承遙輦氏九帳為第十帳。



 二月戊午,以從弟迭慄底為迭烈府夷離堇。是月,徵黑車子室韋,降其八部。



 夏四月丁未朔,唐梁王朱全忠廢其主,壽弒之,自立為帝,國號梁,遣使來告。劉仁恭子守光囚其父,自稱幽州盧龍軍節度使。秋七月乙酉,其兄平州刺史守奇率其眾數千人來降,命置之平盧城。



 冬十月乙巳,討黑車子室韋,破之。



 二年春正月癸酉朔,御正殿受百官及諸國使朝。辛巳,始置惕隱,曲族屬,以皇北撒刺為之。河東李克用卒,子存勖襲,遣使吊慰。



 夏五月癸酉,詔撒刺討烏丸、黑車子室韋。



 秋八月壬子,幽州進合歡瓜。



 冬十月己亥朔,建明王
 樓。築長城於鎮東海口。遣輕兵取吐渾叛入室韋者。



 三年春正春月,幸遼東。



 二月丁酉朔,梁遣郎公遠來聘。



 三月,滄州節度使劉守文為弟守光所攻,遣人來乞兵討之。



 命皇弟舍利素、夷離堇蕭敵魯以兵會守文於北淖口。進至橫海軍近澱,一鼓破之,守光潰去。因名北淖口為會盟口。



 夏四月乙卯,詔左僕射韓知古建碑龍化州大廣寺以紀功德。



 五月甲申,置羊城於炭山之北以通市易。



 冬十月己巳,遣鷹軍討黑車子室韋,破之。西北嗢娘改部族進輓車人。



 四年秋七月子戊子朔,以後兄蕭敵魯為北府宰相。後族
 為相自此始。



 冬十月,烏馬山奚庫支及查刺底、鋤勃德等叛,討平之。



 五年春正月丙戌朔,日有食之。丙申,上親征西部奚。奚阻險,叛服不常,數招諭弗聽。是役所向輒下,遂分兵討東部奚,亦平之。於是盡有奚、霫之地。東際海,南暨白檀,西逾松漠,北抵潢水,凡五部,咸入版籍。



 三月,次濼河,刻石紀功。復略地薊州。



 夏四月壬申,遣人使梁。



 五月,皇弟刺葛、迭刺、寅底石、安端謀反。安端妻粘睦姑知之,以告,得實。上不忍加誅,乃與諸弟登山刑牲,告天地為誓而赦其罪。出刺葛為迭刺部夷離堇,封粘睦姑為晉國夫
 人。



 秋七月壬年朔,斜離底及諸蕃使來貢。



 八月甲子,劉守光僭號幽州,稱燕。



 冬十月戊午,置鐵冶。



 十一月壬午,遣人使梁。



 六年春正月,以化葛為惕隱。



 二月戊午,親征劉守光。



 三月,至自幽州。



 夏四月,梁郢王友珪弒父自立。



 秋七月丙午,親征術不姑,降之,俘獲以數萬計。命弟刺葛分兵攻平州。



 八月壬辰,上次恩德山。皇子李胡生。



 冬十月戊寅,刺葛破平州,還,復與迭刺、寅底石、安端等反。甲申,遣人使梁致祭。壬辰,還次北阿魯山,聞諸弟以兵阻道,引軍南趨十七濼。是日燔柴。翼日,次七渡河,諸弟各遣人謝
 罪。上猶矜憐,許以自親。



 是歲,以兵討兩冶,以所獲僧崇文等五十人歸西樓,建天雄寺以居之,以示天助雄武。



 七年春正月甲辰朔,以用兵免朝。晉王李存勖拔幽州,擒劉守光。甲寅,王師次赤水城,弟刺葛等乞降。上素服,乘赭白馬,以將軍耶律樂姑、轄刺僅阿缽為御,解兵器、肅侍衛以受之。因加慰諭。刺葛等引退,上復數遣使撫慰。



 二月甲戌朔,梁均王友貞討殺其兄史友珪,嗣立。



 三月癸丑,次蘆水,弟迭刺哥圖為奚王,與安端擁千餘騎而至,給稱入覲。上怒曰:「爾曹始謀逆亂,朕特恕之,使改過自新,尚爾反覆,將不利於朕。」遂拘之。以所部分隸諸軍。



 而刺葛引其眾至乙室堇澱,具天子旗鼓,將自立,皇太后陰遣人諭令避去。會弭姑乃、懷裡陽言車駕且至,其眾驚潰,掠居民北走,上以兵追之。刺葛遣其黨寅底石引兵徑趨行宮,焚其輜重、廬帳,縱兵大殺。皇后急遣蜀古魯救之,僅得天子旗鼓而已。其黨神速姑復劫西樓,焚明王樓。上至土河,秣馬休兵,若不為意。諸將請急追之,上曰:「俟其遠遁,人各懷土。懷土既切,其心必離,我軍乘之,破之必矣。」盡以先所獲資畜分賜將士,留夷離畢直里姑總政務。



 夏四月戊寅,北追刺葛。己卯,次彌里,問諸弟面木葉山射鬼箭厭禳,乃執叛人解里向彼,亦以
 其法厭之。至達里澱,選輕騎追及培只河,盡獲其黨輜重、生口。先遣室韋及吐渾酋長拔刺、迪裡姑等五人分兵伏其前路,命北宰相迪里古為先鋒進擊之。刺葛率兵逆戰,迪里古以輕兵薄之。其弟遏古只臨陣,射數十人斃,眾莫敢前。相拒至晡,眾乃潰。追至柴河,遂自焚其車乘廬帳而去。前遇拔刺、迪裡姑等伏發,合擊,遂大敗之。刺葛奔潰,遺其所奪神帳於路,上見而拜奠之。所獲生口盡縱歸本土。其黨庫古只、磨朵皆面縛請罪。師次札堵河,大雨暴漲。



 五月癸丑,遣北宰相迪輦率驍騎先渡。甲寅,奏擒刺葛、涅里袞阿缽於榆河,前北宰相蕭實
 魯、寅底石自剄不殊。遂以黑白羊祭天地。壬戌,刺葛、涅里袞阿缽詣行在,以稿索自縛,牽羊望拜。上還至大嶺。時大軍久出,輜重不相屬,士卒煮馬駒、採野菜以為食,孳畜道斃者十七八,物價十倍,器服資貨委棄於楚裏河,狼藉數百里,因更刺葛名暴里。丙寅,至庫里,以青牛白馬祭天地。以生口六百、馬二千三百分賜大、小鶻軍。



 六月辛巳,至榆嶺,以轄賴縣人掃古非法殘民,磔之。甲申,上登都庵山,撫其先奇首可汗遺跡,徘徊顧瞻而興歡焉。



 聞獄官涅離擅造大校,人不堪其苦,有至死者,命誅之。壬辰,次狼河,獲逆黨雅里、彌里,生埋之銅河南軌
 下。放所俘還,多為於骨里所掠。上怒,引輕騎馳擊。復遣驍將分道追襲,盡獲其眾並掠者。庚子,次阿敦濼,以養子涅里思附諸弟叛,以鬼箭射殺之。其餘黨六千,各以輕重論刑。於厥掠生口者三十餘人,變俾贖其罪,放歸本部。至石嶺西,詔收回軍乏食所棄兵仗,召北府兵驗而還之。以夷離堇涅里袞附諸弟為叛,不忍顯戮,命自投崖而死。



 秋八月己卯,幸龍眉宮,轘逆黨二十九人,以其妻女賜有功將校,所掠珍賓、孳畜還主,亡其本物者,命責償其家;不能償者,賜以其部曲。



 九月壬戌,上發自西樓。



 冬十月庚午,駐赤崖。戊寅,和州回鶻來貢。癸未,乙
 室府人迪里古、迷骨離部人特里以從逆誅。詔群臣分決滯訟,以韓知古錄其事,只裡姑掌捕亡。



 十一月,祠木葉山。還次昭烏山,省風俗,見高年,議朝政,定吉兇儀。



 十二月戊子,燔柴於蓮花濼。



 八年春正月甲辰,以曷魯為迭刺部夷離堇,忽烈為惕隱。



 於骨里部人特離敏執逆黨怖胡、亞裡只等十七人來獻,上親鞫之。辭多連宗及有脅從者,乃杖殺首惡怖胡,餘並原釋。於越率懶之子化哥屢奸謀,上每優容之,而反覆不悛,召父老群臣正其罪,並其子戮之,分其財以給衛士。有司所鞫逆黨三百人,獄既具,上以
 人命至重,死不復生,賜宴一日,隨其平生之好,使為之。酒酣,或歌、或舞、或戲射、角牴,各極其意。明日,乃以輕重刑。首惡刺葛,其次迭刺哥,上猶弟之,不忍置法,杖而釋之。以寅底石、安端性本庸弱,為刺葛所使,皆釋其罪。



 前於越赫底裡子解里、刺葛妻轄刺已實預逆謀,命皆絞殺之。寅底石妻涅離脅從,安端妻粘睦姑嘗有忠告,並免。因謂左右曰:「諸弟性雖敏黠,而蓄奸稔惡。嘗自矜有出人之智,安忍兇狠,谿壑可塞而貪黷無厭。求人之失,雖小而可恕,謂重如泰山;身行不義,雖入大惡,謂輕於鴻毛。暱比群小,謀及婦人,同惡相濟,以危國祚。雖欲
 不敗,其可得乎?北宰相實魯妻餘盧睹姑於國至親,一旦負朕,從於叛逆,未置之法而病死,此天誅也。解裏自幼與朕常同寢食,眷遇之厚,冠於宗屬,亦與其父背大恩而從不軌,茲可恕乎。」



 秋七月丙申朔,有司上諸帳族與謀逆者三百餘人罪狀,皆棄市。上嘆曰:「致人於死,豈朕所欲。若止負朕躬,尚可容貸。此曹恣行不道,殘害忠良,塗炭生民,剽掠財產。民間昔有萬馬,今皆徒小,有國以來所未嘗有。實不得已而誅之。」



 冬十月甲子朔,建開皇殿於明王樓基。



 九年春正月,烏古部叛,討平之。



 夏六月,幽州軍校齊行
 本舉其族及其部曲男女三千人請降,詔授檢校尚書、左僕射,賜名兀欲,給其廩食。數日亡去,幽帥周德威納之。及詔索之,德威語不遜,乃議南證。



 冬十月戊申,鉤魚于鴨淥江。新羅遣使貢方物,高麗遣使進寶劍,吳越王錢鏐遣滕彥休來貢。



 是歲,君基太一神數見,詔圖其像。



 神冊元年春二月丙戌朔,上在龍化州,迭烈部夷離堇耶律曷魯等率百僚請上尊號,三表乃允。丙申,群臣及諸屬國築壇州東,上尊號曰大聖大明天皇帝,後曰應天大明地皇后。大赦,建元神冊。初,闕地為壇,得金鈴,因名其地曰金鈴岡。壇側滿林曰冊聖林。



 三月丙辰,以迭
 烈部夷離堇曷魯為阿廬朵里於越,百僚進秩、頒賚有差,賜酺三日,立子倍為皇太子。夏四月乙酉朔,晉幽州節度使盧國用來降,以為幽州兵馬留後。甲辰,梁遣郎公遠來賀。



 六月庚寅,吳越王遣滕彥休來貢。



 秋七月壬申,親征突闕、吐渾、黨項、小蕃、沙陀諸部,皆平之。俘其酋長及其戶萬五千六百,鎧甲、兵仗、器服九十餘萬,寶貨、駝馬、牛羊不可勝算。



 八月,拔朔州,擒節度使李嗣本。勒石紀功於青塚南。



 冬十月癸未朔,乘勝而東。



 十一月,攻蔚、新、武、媯、儒五州,斬首萬四千七百餘級。自代北至河曲逾陰山,盡有其地。遂改武州為歸化州,媯州為可汗
 州,置西南面招討司,選有功者領之。其圍蔚州,敵樓無故自壞,眾軍大噪乘之,不逾時而破。時梁及吳越二使皆在焉,詔引環城觀之,因賜滕彥休名曰述呂。



 十二月,收山北八軍二年春二月,晉新州裨將戶文進殺節度使李存矩來降。進攻其城,刺史安金全遁,以文進部將劉殷為刺史。



 三月辛亥,攻幽州,節度使周德威以幽、並、鎮、定、魏五州之兵拒於居庸關之西,合戰於新州東,大破之,斬首三萬餘級,殺李嗣恩之子武八。以後弟阿骨只為統軍,實魯為先鋒,東出關略燕、趙,不遇敵而還。己未,於骨里叛,命
 室魯以兵討之。夏四月壬午,圍幽州,不克。



 六月乙巳,望城中有氣如煙火狀,上曰:「未可攻也。」



 以大暑霖潦,班師。留曷魯、盧國用守之。刺葛與其子賽保里叛入幽州。



 秋八月,李存勖遣李嗣源等救幽州,曷魯等以兵少而還。



 三年春正月丙申,以皇弟安端為大內惕隱,命攻雲州及西南諸部。



 二月,達旦國來聘。癸亥,城皇都,以禮部尚書康默記充版築使。梁遣使來聘。晉、吳越、渤海、高麗、回鶻、阻卜、黨項及幽、鎮、定、魏、潞等州各遣使來貢。



 夏四月乙巳,皇弟迭烈哥謀叛,事覺,知有罪當誅,預為營壙,而諸戚請免。上素惡其弟寅底石妻涅里袞,乃曰:「涅里袞
 能代其死,則從。」涅里袞自縊壙中,並以奴女古、叛人曷魯只生瘞其中。遂赦迭烈哥。



 五月乙亥,詔建孔子廟、佛寺、道觀。



 秋七月乙酉,於越曷魯薨,上震悼久之,輟朝三日,贈賻有加。冬十二月庚子朔,幸遼陽故城。辛丑,北府宰相蕭敵魯薨,戊午,以於越曷魯弟污里軫為迭烈部夷離堇,蕭阿古只為北府宰相。甲子,皇孫隈欲生。



\end{pinyinscope}