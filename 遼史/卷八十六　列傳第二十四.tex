\article{卷八十六 列傳第二十四}

\begin{pinyinscope}

 耶律化哥耶律斡臘耶律速撒蕭阿魯帶耶律那也耶律何魯掃古耶律世良耶律化哥,字弘隱,孟父楚國王之後。善騎射。



 乾亨初,為北院林牙。統和四年,南侵宋,化哥擒諜者,知敵由海路來襲,即先據平州要地。事平,拜上京留守,遷北院大王。十六年,復侵宋,為先鋒,破敵於遂城,以功遷南院大王,
 尋改北院樞密使。



 開泰元年,伐阻卜,阻卜棄輜重遁走,俘獲甚多。帝嘉之,封豳王。後邊吏奏,自化哥還闕,糧乏馬弱,勢不可守,上復遣化哥經略西境。化哥與邊將深入。聞蕃部逆命居翼只水,化哥徐以兵進。敵望風奔潰,獲羊馬及輜重。



 路由白拔烈,遇阿薩蘭回鶻,掠之。都監衰哀繼至,謂化哥曰:「君誤矣!此部實效順者。」化哥悉還所俘。諸蕃由此不附。上使按之,削王爵。以侍中遙領大同軍節度使,卒。耶律斡臘,字斯寧,奚迭剌部人。趫捷有力,善騎射。



 保寧初,補護衛。車駕臘頡山,適豪豬伏叢莽,帝射中,豬突出。
 御者托滿舍轡而避,廄人鶴骨翼之,斡臘復射而斃。



 帝嘉賞。及獵赤山,適奔鹿奮角突前,路隘不容避,垂犯蹕。



 斡臘以身當之,鹿觸而顛。帝謂曰:「朕因臘,兩瀕於危,賴卿以免,始見爾心。」遷護衛太保。



 從樞密恥律斜軫破宋將楊繼業軍於山西。統和十三年秋,為行軍都監,從都部署奚王和朔奴伐兀惹烏昭度,數月至其城。



 昭度請降。和朔奴利其俘掠,令四面急攻。昭度率眾死守,隨方捍禦。依傀堄虛構戰棚,誘我軍登陴,俄撤枝柱,登乾盡覆。



 和朔奴知不能下,欲退。蕭恆德謂師久無功,何以藉口,若深入大掠,猶勝空返。斡臘曰:「深入,恐所得不償
 所損。」恆德不從,略地東南,循高麗北鄙還。道遠糧絕,人馬多死。詔奪諸將官,惟斡臘以前議得免。



 尋加同政事門下平章事,為東京留守。開泰中卒。



 耶律速撒,字阿敏,性忠直簡毅,練武事。



 應歷初,為侍從,累遷突呂不部節度使。歷霸、濟、祥、順、聖五州都總管,俄為敦睦宮太師。保寧三年,改九部都詳穩。四年,伐黨項,屢立戰功,手詔勞之。



 統和初,皇太后稱制,西邊甫定,速撒務安集諸藩,利害輒具以聞,太后益信任之。凡臨戎,與士卒同甘苦,所獲均賜將校。賞順討逆,威信大振。在邊二十年卒。



 蕭阿魯帶,字乙辛隱,烏隗部人。父女古,仕至乣詳穩。



 阿魯帶少習騎射,曉兵法。清寧間始仕,累遷本部司徒。



 改烏古敵烈統軍都監。



 大安七年,遷山北副部署。九年,達理得、拔思母二部來侵,率兵擊卻之。達理得復劫牛羊去,阿魯帶引兵追及,盡獲所掠,斬渠帥數人。是冬,達理得等以三百餘人梗邊,復戰卻之,斬首二百餘級,加金吾衛上將軍,封蘭陵縣公。壽隆元年,第功,加同中書門下平章事,進爵郡公,改西北路招討使。



 乾統三年,坐留宋俘當遣還者為奴,免官。後被徵,以老疾致仕,卒。



 耶律那也,字移斯輦,夷離堇蒲古只之後。



 父斡,嘗為北
 克,從伐夏戰歿。季父趙三,始為宿直官,累遷至北面林牙。咸雍四年,拜北院大王,改西南面招討使。



 大康中,西北諸部擾邊,議欲往討,帝以為非趙三不可,遂拜西北路招討使,兼行軍都統,平之,以功復為北院大王。



 那也敦厚才敏。上以其父斡死王事,九歲加諸衛小將軍,為題裏司徙,尋召為宿直官。大康三年,為遙輦克。大安九年,為倒塌嶺節度使。明年冬,以北阻卜長磨古斯叛,與招討都監耶律胡呂率精騎二千往討,破之。那也薦胡呂為漢人行宮副部署。壽隆元年,復討達理得、拔思母等有功,賜詔褒美,改烏古敵烈部統軍使,邊境以寧。部民乞
 留,詔許再任。乾統六年,拜中京留守,改北院大王,薨。



 那也為人廉介,長於理民,每有鬥訟,親核曲直,不尚威嚴,常曰:「凡治人,本欲分別是非,何事迫脅以立名。」故所至以惠化稱。



 耶律何魯掃古,字烏古鄰,孟父房之後。



 重熙末,補祗候郎君。清寧初,加安州團練使。大康中,歷懷德軍節度使、奚六部禿裏太尉。詔與樞密官措畫東北邊事,改左護衛太保。侍上,言多率易,察無他腸,以故上優貸之。



 大安八年,知西北路招討使事。時邊部耶睹刮等來侵,何魯掃古誘北阻卜酋豪磨古斯攻之,俘獲甚眾,以功加左僕射。



 復討耶睹刮等,誤擊磨古斯,北阻卜由是叛命。遣都監張九討之,不克,二室韋與六院部、特滿群牧、宮分等軍俱陷於敵。



 何魯掃古不以實聞,坐是削官,決以大杖。



 壽隆間,累遷惕隱,兼侍中,賜保節功臣。道宗崩,與宰相耶律儼總山陵事。乾統中,致仕,卒。



 耶律世良,小字斡,六院部人。才敏給,練達國朝典故及世譜。上書與族弟敵烈爭嫡庶,帝始識之。



 時北院樞密使韓德讓病,帝問:「孰可代卿?」德讓曰:「世良可。」北院大王耶律室魯復問北院之選,德讓曰:「無出世良。」統和末,為北院大王。



 開泰初,因大冊禮,加檢校太尉、同政事門下
 平章事。時邊部拒命,詔北院樞密使耶律化哥將兵,以世良為都監,往御之。明年,化哥還,將罷兵。世良上書曰:「化哥以為無事而還,不思師老糧乏,敵人已去,焉能久守?若益兵,可克也。」



 帝即命化哥益兵,與世良追之。至安真河,大破而還。自是,邊境以寧。以功王岐,拜北院樞密使。



 三年,命選馬駝於烏古部。會敵烈部人夷剌殺其酋長稍瓦而叛,鄰部皆應,攻陷巨母古城。世良率兵壓境,遣人招之,降數部,各復故地。



 四年,伐高麗,為副部署。都統劉慎行逗留失期,執還京師,世良獨進兵。明年,至北都護府,破追兵於郭州。以暴疾卒。



 論曰:「大之懷小也以德,制之也以威。德不足懷,威不足制,而欲服人也難矣。化哥利俘獲,而諸蕃不附,何魯掃古誤擊磨古斯,而阻卜叛命,是皆喜於一旦之功,而不圖後日之患,庸何議焉。若斡臘之戒深入,速撒之務安集,亦鐵中之錚錚者邪?」



\end{pinyinscope}