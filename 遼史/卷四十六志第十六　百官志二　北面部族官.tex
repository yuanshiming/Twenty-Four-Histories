\article{卷四十六志第十六 百官志二 北面部族官}

\begin{pinyinscope}

 部族,詳見《營衛志》。設官之制具如左。



 部族職名總目:大部族。



 某部大王。本名夷離堇。



 某部左宰相。



 某部右宰相。



 某部太師。



 某部太保。



 某部太尉。



 某部司徒。本名惕隱。



 某部節度使司。



 某部節度使。



 某部節度副使。



 某部節度判官。



 某部族詳穩司。



 某部族詳穩。某部族都監。



 某部族將軍。



 某部族小將軍。



 某石烈。



 某石烈夷離堇。



 某石烈麻普。亦曰馬步,本名石烈達刺幹。



 甘石烈牙書。



 某彌里。彌里,鄉也。



 辛袞。本曰馬特本。



 小部族。



 某部族司徒府。



 某部族司徒。



 某部族司空。



 某部族節度使司。



 某部族詳穩司。



 某石烈。



 令穩。



 麻普。



 牙書。



 某彌里。



 辛袞五院部。有知五院事,在朝曰北大王院。



 六院部。有知六院事,在朝曰南大王院。



 乙室部。在朝曰乙室王府。有乙室府迪骨裏節度使司。



 奚六部。在朝曰奚王府。有二常袞,有二宰相,又有吐裏太尉,有奚六部漢軍詳穩,有奚拽刺詳穩,有先離撻覽官。



 已上四大王府,為大部族。
 品部。



 楮特部。



 烏隗部。



 突呂不部。



 突舉部。



 涅刺部。



 遙里部。



 伯德部。



 墮瑰部。



 楚里部。



 奧里部。



 南克部。



 北克部突呂不室韋部。



 涅刺拿古部。



 迭刺迭達部。



 乙室奧隗部。



 楮特奧隗部。



 品達魯虢部。



 烏古涅刺部。



 圖魯部。



 撒里葛部。



 窈爪部。



 耨碗爪部。



 訛僕括部。



 特裡特勉部。



 稍瓦部。曷術部。



 隗衍突厥部。



 奧衍突厥部。



 涅刺越兀部。



 奧衍女直部。



 乙典女直部。



 斡突碗烏古部。



 迭魯敵烈部。



 大黃室韋部。



 小黃室韋部。二黃室韋闥林,改為僕射。



 術哲達魯虢部;梅古悉部。



 頡的部。



 匿訖唐古部。



 北唐古部。



 南唐古部。



 鶴刺唐古部。



 河西部。



 北敵烈部。



 薛特部。



 伯斯鼻骨部。



 達馬鼻骨部。



 五國部。



 已上四十九節度,為小部族。



 北面坊場局冶牧廄等官
 遼始祖涅裡究心農工之事,太祖尤拳拳焉,畜牧畋漁固俗尚也。坊場牧廄,設官如左。諸坊職名總目:某坊使。



 某坊副使。



 某坊詳穩司。



 某坊詳穩。



 某坊都監。



 鷹坊。



 鐵坊。



 五坊。未詳。



 八坊。內有軍器坊,餘未詳。



 已上坊官。



 圍場。



 圍場都太師。



 圍場都管。



 圍場使。



 圍場副使。



 已上場官。



 局官職名總目:
 某局使。



 某局副使。



 客省局。



 器物局。



 太醫局。



 醫獸局。有四局都林牙。



 已上局官。



 五冶。未詳。



 太師。已上冶官。



 群牧職名總目:某路群牧使司。



 某群太保。



 某群侍中。



 某群敞史。



 總典群牧使司。



 總典群牧部籍使群牧都林牙。



 某群牧司。



 群牧使。



 群牧副使。



 西路群牧使司。



 倒塌嶺西路群牧使司。



 渾河北馬群司。



 漠南馬群司。



 漠北滑水馬群司。



 牛群司。



 己上群牧官。



 尚廄。



 尚廄使,
 尚廄副使。



 飛龍院。



 飛龍使。



 飛龍副使。



 總領內外廄馬司。



 總領內外廄馬。己上諸廄官。



 監鳥獸詳穩司職名總目:監某鳥獸詳穩。



 監某鳥獸都監。



 監某鳥。



 監某獸。



 監鹿詳穩司。



 監雉。



 已上監養鳥獸官。



 北面軍官遼宮帳、部族、京州、屬國,各自為軍,體統相承,分數秩然。雄長二百餘年,凡以此也。考其可知者如左。



 天下兵馬大元帥府。太子、親王總軍政。



 天下兵馬大元帥。



 副元帥。



 大元帥府。大臣總軍馬之政。



 大元帥。



 副元帥。



 都元帥府。大將總軍馬之事。



 兵馬都元帥。



 副元帥。



 同知元帥府事。



 便宜從事府。亦曰便宜行事。



 便宜從事。



 大詳穩司。



 大詳穩。



 都監。將軍。



 小將軍。



 軍校。



 隊帥。



 東都省。分掌軍馬之政。



 東都省太師。



 西都省。分掌軍馬之政。



 西都省太師。



 大將軍府。各統所治軍之政令。



 大將軍。



 上將軍。



 將軍。



 小將軍。



 護軍司。



 護軍司徒。



 衛軍司。



 衛軍司徒。



 諸路兵馬統署司。



 諸路兵馬都統署。



 諸路兵馬副統署。



 左皮室詳穩司。



 右皮室詳穩司。



 北皮室詳穩司。



 南皮室詳穩司。太宗選天下精甲三十萬為皮室軍。初,太祖以行營為宮,選諸部豪健千餘人,置為腹心部,耶律老古以功為右皮室詳穩。則皮
 室軍自太祖時己有,即腹心都是也。



 太宗增多至三十萬耳。黃皮室軍詳穩司。黃皮室,屬國名。



 屬珊軍詳穩司。應天皇太后置,軍二十萬。選蕃漢精兵,珍美如珊瑚,故名。



 舍利軍詳穩司。統皇族之從軍者,橫帳、三父房屬焉。



 北王府舍利軍詳穩司。五院皇族屬焉。



 南王府舍利軍詳穩司。六院皇族屬焉。



 禁軍都詳穩司。掌禁衛諸軍之事。



 各部族舍利司。掌備部族子弟之軍政。



 郎君軍詳穩司。掌著帳郎君之軍事。



 拽刺軍詳穩司。走卒謂之拽刺。



 旗鼓拽刺詳穩司。掌旗鼓之事。



 千拽刺詳穩司。



 猛拽刺詳穩司。



 墨離軍詳穩司。



 炮手軍詳穩司。掌飛炮之事。



 彎手軍詳穩司。掌強弩之事。



 鐵林軍詳穩司。



 大鷹軍詳穩司。



 鷹軍詳穩司。



 鶻軍詳穩司。大、小鶻軍,即二室韋軍號。



 鳳軍詳穩司。



 龍軍詳穩司。



 飛龍軍詳穩司。



 虎軍詳穩司。



 熊軍詳穩司。



 左鐵鷂子軍詳穩司。



 右鐵鷂子軍詳穩司。龍衛軍詳穩司
 威勝軍詳穩司。



 天云軍詳穩司。



 特滿軍詳穩司。



 敵烈軍詳穩司。



 敵烈皮室詳穩司。



 希里奚軍詳穩司。



 涅哥奚軍詳穩司。



 渤海軍詳穩司。



 女古烈詳穩司。



 奚王南克軍詳穩司。諸帳並有克官為長,餘同詳穩司。



 奚王北克軍詳穩司。



 國舅帳克軍。



 三克軍。



 頻必克軍。



 九克軍。



 十二行糺軍。諸糺並有司徒,餘同詳穩司。



 各官分糺軍。



 遙輦糺軍。



 各部族糺軍。



 群牧二糺軍。



 怨軍八營都詳穩司。天祚天慶六年,命秦晉王淳募遼東饑民,得二萬餘人,謂之怨軍。及淳僭位,改號常勝軍。



 前宜營。八營皆以所募州名為號。



 後宜營。



 前錦營。



 後錦營。乾營。



 顯營。



 乾顯大管。



 巖州營。



 北面邊防官遼境東接高麗,南與梁、唐、晉、漢、周、宋六代為勁敵,北鄰阻卜、術不姑,大國以十數;西制西夏、黨項、吐渾、回鶻等,強國以百數。居四戰之區,虎踞其間;莫敢與攖,制之有爾故而。觀於邊防之官,太祖、太宗之雄圖見矣。



 諸軍都虞候司。



 都虞候。



 奚王府。見《部族官》。



 大惕隱司。見《帳官》。



 大國舅司。



 大常袞司。



 五院司。見《部族官》。



 六院司。



 沓溫司。未詳。



 已上上京路諸司,控制諸奚。



 諸部署職名總目:某兵馬都部署。



 某兵馬副部署。



 某兵馬都監。



 某都部署判官。



 諸指揮使職名總目:某軍都指揮使。



 某軍副指揮使。某軍都監。



 諸統軍使職名總目:有都統軍使、副使、都監等官。



 東京兵馬都部署司。



 契丹、奚、漢、渤海四軍都指揮使司。



 契丹奚軍都指揮使司。



 漢軍都指揮使司。



 渤海軍都指揮使司。



 東京都統軍使司。



 東京都詳穩司。



 保州都統軍司。



 湯河詳穩司。亦曰南文直湯河司。



 杓窊司。未詳。



 金吾營。屬南面。



 銅州北兵馬指揮使司。



 淶州南兵馬指揮使司。



 己上遼陽路諸司,控扼高麗。



 黃龍府兵馬都部署司。一件都監署司。



 黃龍府鐵驪軍詳穩司。



 咸州兵馬詳穩司。有知咸州路兵馬事、同知咸州路兵馬事、威州誳將。



 東北路都統軍使司。有掌法官,道宗大安六年置。



 己上長春路諸司,控制東北諸國。



 南京都元帥府。本南京兵馬都總管府,興宗重熙四年改。



 有都元帥、大元帥。



 南京兵馬都總管府。屬南面。有兵馬都總管,有總領南
 面邊事,有總領南面軍務,有總領南面戍兵等官。南京馬步軍都指揮使司。屬南面。



 侍衛控鶴都指揮使司。屬南面。



 燕京禁軍詳穩司。



 南京都統軍司。又名燕京統軍司。聖宗統和十二年復置南京統軍都監。



 牛欄都統領司。



 都統領。



 副統領。



 距馬河戍長司。聖宗開泰七年,沿距馬河宋界東西七百餘里,特置戍長一員巡察。



 戍長。



 監軍寨統領司。



 石門統領司。



 南皮室軍詳穩司。



 北皮室軍詳穩司。



 猛拽刺詳穩司。



 管押平川甲馬司。



 管押平川甲馬。



 已上南京諸司,並隸元帥府,備御宋國。



 西南面安撫使司。



 西南面安撫使。



 西南面都招討司。太祖神冊元年置。亦曰西南路描討司。



 西南面招討使。



 西南邊大詳穩司。



 西南路詳穩司。



 西南面五押招討司。



 五押招討大將軍。西南路巡察司。又有西南巡邊官。



 西南路巡察將軍。



 西南面巡檢司。



 西南面巡檢。



 西南面同巡檢。



 西南面拽刺詳穩司。



 山北路都部署司。又有知山北道邊境事官。



 金肅軍都部署司。



 南王府。見《北面朝官》。



 北王府。



 乙室王府。



 山金司。一作山陰司。置在金山之北。



 已上西京諸司,控制西夏。



 西北路招討使司。有知西路招討事,有監軍。



 西北路管押詳穩司。



 西北路總領司。有總領西北路軍事官。



 領西北路十二班軍使司。



 契丹軍詳穩司。



 吐渾軍詳穩司。



 述律軍詳穩司。



 禁軍詳穩司。



 奚王府舍利軍詳穩司。



 大室韋軍詳穩司。



 小室韋軍詳穩司。



 北王府軍詳穩司。



 特滿軍詳穩司。



 群牧軍詳穩司。宮分軍詳穩司。



 西北路金吾軍。屬南面。



 西北路兵馬都部署司。



 西北路阻卜都部署司。



 西北路統軍司。



 西北路戍長司。



 西北路禁軍都統司。



 西北部鎮撫司。兼掌西北諸部軍民。有鎮撫西北部事官。



 西北路巡檢司。



 黑水河提轄司。在中京黔州置。



 已上西北路諸司,控制諸國。



 東北路兵馬詳穩司。亦曰東北面詳穩司。



 東北路監軍馬司。有東北路監軍馬使,有管押東北路軍馬事官。東北路女直詳穩司。



 北女直兵馬司。在東京遼州置。



 已上東北路諸司。



 東路兵馬都總管府。有東路兵馬都總管,有同知東路兵馬事官。東路都統軍使司。



 遙裡等十軍都詳穩司。



 遙里軍諸詳穩司。未詳。



 龍水諸夷安撫使。



 已上東路諸司。



 西南面節制司。有節制西南諸軍事。



 西南面都統軍司。



 已上西南邊諸司。山西兵馬都統軍司。



 西路招討使司。



 西邊大詳穩司。



 西蕃都軍所。聖宗統和四年置,授李繼沖。



 夏州管內蕃落使。聖宗統和四年置,授李繼遷。



 倒塌嶺節度使司。



 倒塌嶺統軍司。



 塌西節度使司。



 塌母城節度使司。



 已上西路諸司。



 北面行軍官遼行軍官,樞密、都統、部署之司,上下相維,先鋒、兩翼嚴重,中軍於遠探偵候為尤謹,臨陣委重於監戰。司存有常,秩然整暇,所以為制勝之道也。



 行樞密院。有左、右林牙,有參謀。



 行軍都統所。有監軍,有行軍諸部都監,有監戰。



 行軍都統。



 行軍副都統。



 行軍都監。



 行軍都押司。有都押官、副押官。



 行軍都部署司。



 先鋒使司先鋒都統所。



 左翼軍都統所。



 右翼軍都統所。



 中軍都統所。



 御營都統所。遠探軍。有小校,有拽刺。



 候騎。有偵候,有候人,有拽刺。



 東征行樞密院。



 東征都統所。亦曰東面行軍都統所,又曰東路行軍都統所。



 東征統軍司。



 東征先鋒使司。



 西征統軍司。



 南征都統所。亦曰南面行軍都統所。



 南征統軍司。



 南面行營總管府。



 南面行營都部署司。



 河南道行軍都統所。



 北道行軍都統所。



 東北面行軍都統所。



 西北面行軍都統所。



 西南面行軍都統所。



 北面屬國官。



 遼制,屬國、屬部官,大者擬王封,小者準部使。命其酋長與契丹人區別而用,恩戚兼制,得柔遠之道。考其可知者具如左。屬國職名總目:某國大王。



 某國於越。



 某國左相。



 某國右相。



 某國惕隱。亦曰司徒。



 某國太師。某國太保。



 某國司空。本名闥林。



 某國某部節度使司。



 某國某部節度使。



 某國某部節度副使。



 某國詳穩司。



 某國詳穩。



 某國都監。



 某國將軍。



 某國小將軍。



 大部職名:並同屬國。



 諸部職名:並同部族。



 女直國順化王府。景宗保寧九年,女直國來請宰相、夷離堇之職,以次授者二十一人。聖宗統和八年,封女直阿海為順化王,亦作阿改。天祚天慶二年有順國女直阿鶻產大王。



 北女直國大王府。



 南文直國大王府。



 昌蘇館路女直國大王府。亦曰合蘇袞部女直王,又曰合素女直王,又曰蘇館都大王。聖宗太平六年,曷蘇館諸部許建旗鼓。



 長白山女直國大王府。聖宗統和三十年,長白山三十部女直乞授爵秩。



 鴨綠江文直大王府。



 瀕海女直國大王府。



 阻卜國大王府。
 阻卜扎刺部節度使司。



 阻卜諸部節度使司。聖宗統和二十九年置。



 阻卜別部節度使司。



 西阻卜國大王府。



 北阻卜國大王府。



 西北阻卜國大王府乞粟河國大王府。



 城屈里國大王府。



 術不姑國大王府。亦曰述不妨。又有直不姑。



 阿薩蘭回鶻大王府。亦曰阿思懶王府。



 回鶻國單于府。



 興宗重熙二十二年,詔回鶻部副使以契丹人充。



 沙州回鶻忳煌郡王府。



 甘州回鶻大王府。



 高昌國大王府。



 黨項國大王府。



 西復國西平王府。



 高麗國王府。



 新羅國王府。



 曰本國王府。



 吐谷渾國王府。



 吐渾國王府。



 轄戛斯國王府。



 室韋國王府。



 黑車子室韋國王府。



 鐵驪國王府。



 靺鞨國王府。沙陀國王府。



 濊貊國王府。



 突厥國王府
 西突厥國王府。



 斡朗改國王府迪烈德國王府。亦曰敵烈,亦曰迭烈德。



 於厥國王府。



 越離睹國王府。亦曰斡離都。



 阿里國王府。



 襖里國王府。



 朱灰國王府。



 烏孫國王府。



 于闐國王府。



 獅子國王府。



 大食國王府。



 西蕃國王府。



 大蕃國王府。



 小蕃國王府。



 吐蕃國王府。



 阿撒里國王府。



 波刺國王府。



 惕德國王府。



 仙門國王府。



 鐵不得國王府。



 鼻國德國王府。



 轄刺國只國王府。



 賃烈國王府。獲里國王府。



 怕里國王府。



 噪溫國王府。



 阿缽頗得國王府。



 阿缽押國王府。



 拏沒里國王府。



 要里國王府。



 徒睹古國王府。亦曰徒魯古。



 素撒國王府。



 夷都袞國王府。



 婆都魯國王府。



 霸斯黑國王府。



 達離諫國王府。



 達盧古國王府。



 三河國王府。



 核列哿國王府。



 述律子國王府。



 殊保國王府。



 蒲暱國王府。



 烏里國王府。



 已上諸國。



 蒲盧毛朵部大王府。



 回跋部大三府。



 嵓母部大王府。



 黃龍府女直部大王府。道宗大康八年,賜官及印。



 吾禿婉部大王府。



 烏隈於厥部大王府。婆離八部大王府。



 於厥里部族大王府。太宗會同三年,賜旗鼓。



 已上大部。



 生女直部。



 直不姑部。



 狐山部。



 拔思母部。



 茶扎刺部。



 粘八葛部。



 耶睹刮部。



 耶迷只部。



 撻術不姑部。



 渤海部。



 西北渤海部。



 達裡得部。亦曰達離底。



 烏古部。



 隈烏古部。



 三河烏古部。



 烏隈烏骨里部。



 敵烈部。



 迪離畢部。



 涅刺部。



 烏濊部。已上三部,隸夫人婆底裏東北路管押司。



 鉏德部。



 諦居部。亦曰諦舉部。



 涅刺奧隗部。



 八石烈敵烈部。
 迭刺葛部。



 兀惹部。亦曰烏惹部。



 黨項部。



 隗衍黨項部。



 山南黨項部。



 北大濃兀部。



 南大濃兀部。



 九石烈部。



 嗢娘改部。



 鼻骨德部。



 退欲德部。



 涅古部遙思拈部。



 劃離部。聖宗統和元年,劃離部請今後詳穩於當部人內選授,不許。



 四部族部。



 四蕃部。



 三國部。



 素昆那山東部。



 胡母思山部。



 廬不姑部。



 照姑部。



 白可久部。



 俞魯古部。



 七火室韋部。



 黃皮室韋部。



 瑤穩部。嘲穩部。



 二女古部。



 蔑思乃部。



 麻達里別古部。



 梅裏急部。



 斡魯部。



 榆裏底乃部。



 率類部。



 五部番部。



 蒲奴里部。



 閘古胡里扒部。



 已上諸部。



\end{pinyinscope}