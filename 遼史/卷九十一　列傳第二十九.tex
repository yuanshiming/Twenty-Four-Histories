\article{卷九十一 列傳第二十九}

\begin{pinyinscope}

 蕭巖壽耶律撒剌蕭速撒耶律撻不也蕭撻不也蕭忽古耶律石柳蕭巖壽,室部人。性剛直,尚氣。仕重熙末。道宗即位,皇太后屢稱其賢,由是進用。



 上出獵較,巖壽典其事,未嘗高下於心,帝益重之。歷文班太保、同知樞密院事。咸雍四年,從耶律仁先伐阻卜,破之,有詔留屯,亡歸者眾,由
 是鐫兩官。十年,討敵烈部有功,為其部節度使。



 大康元年,同知南院宣徽使事,遷北面林牙。密奏乙辛以皇太子知國政,心不自安,與張孝傑數相過從,恐有陰謀,動搖太子。上悟,出乙辛為中京留守。會乙辛生日,上遣近臣耶律白斯本賜物為壽,乙辛因私屬自上:「臣見奸人在朝,陛下孤危。身雖在外,竊用寒心。」白斯本還,以聞。上遣人賜乙辛車,諭曰:「無慮弗用,行將召矣。」由是反疑巖壽,出為順義軍節度使。



 乙辛復入為樞密使,流巖壽於烏隗路,終身拘作。巖壽雖竄逐,恆以社稷為憂,時人為之語曰:「以狼牧羊,何能久長!」



 三年,乙辛誣巖壽與謀廢
 立事,執還殺之,年四十九。



 乾統間,贈同中書門下平章事,繪像宜福殿。巖壽廉直,面折廷諍,多與乙辛忤,故及於難。



 耶律撒剌,字董隱,南院大王磨魯古之孫。性忠直沉厚。



 清寧初,累遷西南面招討使,以治稱。咸雍九年,改北院大王。



 未幾,為契丹行宮都部署。



 大康二年,耶律乙辛為中京留守,詔百官延議,欲復召之,群臣無敢工言。撒剌獨奏曰:「蕭巖壽言乙辛有罪,不可為樞臣,故陛下出之;今復召,恐天下生疑。」進諫者三,不納,左右為之震驚。乙辛復為樞密使,見撒剌讓曰:「與君無憾,何獨異議?」撒剌
 曰:「此社稷計,何憾之有!」乙辛誣撒剌與速撒同謀廢立,詔按無跡,出為始平軍節度使。及蕭訛都斡誣首,竟遣使殺之。



 乾統間,追封漆水郡王,繪像宜福殿,仍追贈三子官爵。



 蕭速撒,字禿魯堇,突呂不部人。性沉毅。重熙間,累遷右護衛太保。蒲奴里叛,從耶律義先往討,執首亂陶得里以歸。



 清寧中,歷北面林牙、彰國軍節度使,入為北院樞密副使。咸雍十年,經略西南邊,撤宋堡障,戍以皮室軍,上嘉之。



 大康二年,知北院樞密使事。耶律乙辛權寵方盛,附麗者多至通顯;速撒未嘗造門。乙辛銜之,誣構速撒
 首謀廢立;按之無驗,出為上京留守。乙辛復令蕭訛都斡以前事誣告,上怒,不復加訊,遣使殺之。時方盛暑,尸諸原野,容色不變,烏鵲不敢近。



 乾統間,追封漆水郡王,繪像宜福殿。耶律撻不也,字撒班。系出季父房。父高家。仕至林牙,重熙間破夏人於金肅軍有功,優加賞賚。



 撻不也,清寧中補牌印郎君,累遷永興宮使。九年,平重元之亂,以功知點檢司事,賜平亂功臣,為懷德軍節度使。咸雍五年,遷遙輦克。



 大康三年,授北院宣徽使。耶律乙辛謀害太子,撻不也知其奸,欲殺乙辛及蕭特裡得、蕭十三等。乙辛
 知之,令其黨誣構撻不也與廢立事,殺之。



 乾統間,追封蘭陵郡五,繪像宜福殿。



 蕭撻不也,字斡裡端,國舅郡王高九之孫。性剛直。咸雍中,補祗候郎君。大康元年,為彰愍宮使,尚趙國公主,拜駙馬都尉。



 三年,改同知漢人行宮都部署。與北院宣徽使耶律撻不也善,乙辛嫉之,令人誣告謀廢立事。不勝沴掠,誣伏。上引問,昏聵不能自陳,遂見殺。



 乾統間,追封蘭陵郡王,繪像宜福殿。



 蕭忽古,字阿斯憐,性忠直,趫捷有力。甫冠,補禁軍。



 咸雍初,從招討使耶律趙三討番部之違命者。及請降,來介
 有能躍駝峰而上者,以儇捷相詫。趙三問左右誰能此,忽古被重鎧而出,手不及峰,一躍而上,使者大駭。趙三以女妻之。



 帝聞,召為護衛。



 對此院樞密使耶律乙辛以狡玦得幸,肆行兇暴。忽古伏於橋下。伺其過,欲殺之。俄以暴雨壞橋,不果。後又欲殺於獵所,為親友所沮。大康三年,復欲殺乙辛及蕭得裡特等,乙辛知而械擊之,考劾不服,流於邊。及太子廢徙於上京,召忽古至,殺之。



 乾統初,追贈龍虎衛上將軍。耶律石柳,字酬宛,六院部人。祖獨攧,南院大王。父安十,統軍副使。



 石柳性剛直,有經世志。始為牌印郎君。大康
 初,為夷離畢郎君。時樞密使耶律乙辛誣殺皇后,謀廢太子,斥忠賢,進奸黨,石柳惡其所為,乙辛覺之。太子既廢,以石附太子,流鎮州。天祚即位,召為御史中丞。時方治乙辛黨,有司不以為意。



 石柳上書曰:臣前為奸臣所陷,斥竄邊郡。幸蒙召用,不敢隱默。



 恩賞明則賢者勸,刑罰當則奸人消。二者既舉,天下不勞而治。臣見耶律乙辛身出寒微,位居樞要,竊權肆惡,不勝名狀。蔽先帝之明,誣陷順聖,構害忠讜,敗國罔上,自古所無。



 賴廟社之休,陛下獲纂成業,積年之冤,一旦洗雪。正陛下英斷,克成孝道之秋。如蕭得裹特實乙辛之黨,耶律合魯亦
 不為早辨,賴陛下之明,遂正其事。



 臣見陛下多疑,故有司顧望,不切推問。乙辛在先帝朝,權寵無比。先帝若以順考為實,則乙辛為功臣,陛下豈得立耶?



 先帝黜逐嬖後,詔陛下在左右,是亦悔前非也。陛下詎可忘父仇不報,寬逆黨不誅。今靈骨未獲,而求之不切。傳曰,聖人之德,無加於孝。昔唐德宗因亂失母,思慕悲傷,孝道益著。



 周公誅飛廉、惡來,天下大悅。今逆黨未除,大冤不報,上無以慰順考之靈,下無以釋天下之憤。怨氣上結,水旱為沴。



 臣願陛下下明詔,求順考之瘞所,盡收逆黨以正邦憲,快四方忠義之心,昭國家賞罰之用,然後致治之
 道可得而舉矣。



 謹別錄順聖升遐及乙辛等事,昧死以聞。書奏不報,聞者莫不嘆惋。乾統中,遙授靜江軍節度使,卒。子馬哥,同中書門下平章事。論曰:「《易》言『履霜,堅冰至』,謹始也。使通宗能從巖壽、撒剌之諫,後何得而誣,太子何得而廢哉?速撒、撻不也以忠言見殺,國欲無亂,得乎?石柳之書,亦幸出於乙辛既敗之後,獲行其說。有國家者,可不知人哉。」



\end{pinyinscope}