\article{卷五十四志第二十三 樂志}

\begin{pinyinscope}

 遼有國樂,有雅樂,有大樂,有散樂,有鐃歌、橫吹樂。



 舊史稱聖宗、興宗咸通音律,聲氣、歌辭、舞節,徵諸太常、儀鳳、教坊不可得。按《紀》、《志》、《遼朝雜禮》,參考史籍,定其可知者,以補一代之闕文。



 鳴呼!《咸》、《韶》、《夏》、《武》之樂,聲亡書逸,河間作《記》,史遷因以為《書》,寥乎希哉。遼之樂觀此足矣。



 國樂
 遼有國樂,猶先王之風;其諸國樂,猶諸侯之風。故志其略。



 正月朔日朝賀,用宮懸雅樂。元會,用大樂;曲破後,用散樂;角抵終之。是夜,皇帝燕飲,用國樂。



 七月十三日,皇帝出行宮三十里卓帳。十四日設宴,應從諸軍隨各部落動樂。十五日中元,大宴,用漢樂。



 春飛放杏堝,皇帝射獲頭鵝,薦廟燕飲,樂工數十人執小樂器侑酒。



 諸國樂太宗會同三年,晉宣徽使楊端、王慹等及諸國使朝見,
 皇帝御便殿賜宴。端、慹起進酒,作歌舞,上為舉觴極歡。



 會同三年端午日,百僚及諸國使稱賀,如式燕飲,命回鶻、敦煌二使作本國舞。



 天祚天慶二年,駕幸混同江,頭魚酒筵,半酣,上命諸酋長次第歌舞為樂。女直阿骨打端立直視,辭以不能。上謂蕭奉先曰:「阿骨打意氣雄豪,顧視不常,可托以邊事誅之。不然,恐貽後患。」奉先奏:「阿骨打無大過,殺之傷向化之意。蕞爾小國,又何能為。」



 雅樂
 自漢以後,相承雅樂,有古《頌》焉,有古《大雅》焉。



 遼闕郊廟禮,無頌樂。大同元年,太宗自汴將還,得晉太常樂譜、宮懸、樂架,委所司先赴中京。



 聖宗太平元年,尊號冊禮:設宮懸於殿庭,舉麾位在殿第三重西階之上,協律郎各入就舉麾位,太常博士引太常卿,太常卿引皇帝。將仗動,協律郎舉麾,太樂令令撞黃鐘之鐘,左右鐘皆應。工人舉帒,樂作;皇帝即御坐,扇合,樂止。王公入門,樂作;至位,樂止。通事舍人引押冊大臣,初動,樂作;置冊殿前香案訖,
 就位,樂止。舁冊官奉冊,初動,樂作;升殿,置冊御坐前,就西墉北上位,樂止。大臣上殿,樂作;至殿欄內位,樂止。大臣降殿階,樂作;復位,樂止。王公三品以上出,樂作;太常博士引太常卿,太常卿引皇帝降御坐入閣,床上。興宗重熙九年,上契丹冊,皇帝出,奏《隆安》之樂。



 聖宗統和元年,冊承天皇太后,設宮懸、剺鱉,太樂工、協律郎入。
 太后儀衛動,舉麾,《太和》樂作;太樂令、太常卿導引升御坐,簾卷,樂止。文武三品以上入,《舒和》樂作;至位,樂止。皇帝入門,《雍和》樂作;至殿前位,床上。宰相押冊,皇帝隨冊,樂作;至殿前置冊於案,樂止。翰林學士、大將軍舁冊,樂作;置御坐前,樂止。丞相上殿,樂作;至讀冊位,樂止。皇帝下殿,樂作;至位,樂止。太后宣答訖,樂作;皇帝至西閣,樂止。親王、丞相上殿,樂作;退班出,床上。
 下簾、樂作;皇太后入內,樂止。



 冊皇太子儀:太子初入門,《貞安》之樂作。



 冊禮樂工次第:四隅各置建鼓一覜,樂工各一人;宮懸每面九覜,每覜樂工一人;樂覜近北置帒、嚬各一,樂工各一人;樂虡內坐部樂工,左右各一百二人;樂覜西南武舞六十四人,執小旗二人;樂虡東南文舞六十四人,執小旗二人;協律郎二人;
 太樂令一人。



 唐《十二和樂》,遼初用之:《豫和》祀天神,《順和》祭地只,《永和》享宗廟,《肅和》登歌奠玉帛,《雍和》入俎接神,《壽和》酌獻飲神,《太和》節升降,《舒和》節出入,《
 昭和》舉酒,《休和》以飯,《正和》皇后受冊以行,《承和》太子以行。



 遼《十二安》樂:初,梁改唐《十二和樂》為《九慶》樂,後唐建唐宗廟,仍用《十二和》樂,晉改為《十二同》樂。《遼雜禮》:「天子出入,奏《隆安》;太子行,奏《貞安》。」則是遼嘗改樂名矣。餘十《安》樂名缺。



 遼雅樂歌辭,文闕不具;八音器數,大抵因唐之舊。



 八音:
 金祊、鐘。



 石球、磬。



 絲琴、瑟。



 竹龠、蕭、悢匏笙、竽。土塤。



 革鼓、鞀。



 木帒、嚬。



 十二律用周黍尺九寸管,空徑三分為本。通宗大康中,詔行嶮黍所定升斗,嘗定律矣。其法大抵用古律蔫。



 大樂自漢以來,因秦、楚之聲置樂府。至隋高祖詔求知音者,鄭譯得西域蘇祗婆七旦之聲,求合七音八十四調之說,由是雅俗之樂,皆此聲矣。用之朝廷,別於雅樂者,謂之大樂。



 晉高祖使馮道、劉琱冊應天太后、太宗皇帝,其聲器、工官與法駕,同歸於遼。



 聖宗統和元年,冊承天皇太后,童子弟子隊樂引太后輦至金鑾門。



 天祚皇帝天慶元年上壽儀,皇帝出東閣,鳴鞭,樂作;簾卷,扇開,樂止。
 太尉執臺,分班,太樂令舉麾,樂作;皇帝飲酒訖,樂止。應坐臣僚東西外殿,太樂令引堂上,樂升。大臣執臺,太樂令奏舉觴,登歌,樂作;飲訖,樂止。行臣僚酒遍,太樂令奏巡周,舉麾,樂作;飲訖,樂止。太常卿進御食,太樂令奏食遍,樂作;《文舞》入,三變,引出,樂止。次進酒,行臣僚酒,舉觴,巡周,樂作;飲訖,樂止。次進食,食遍,樂作;《武舞》入,三變,引出,樂止。扇合,簾下,鳴鞭,樂作;皇帝入西閣,樂止。



 大樂器:本唐太宗《七德》、《九功》之樂。武後毀唐宗廟,《七德》、《九功》樂舞遂亡,自後宗廟用隋《文》、《武》二舞。



 朝廷用高宗《景雲》樂代之,元會,第一奏《景雲》樂舞。杜佑《通典》已稱諸樂並亡,唯《景雲》樂舞僅存。唐末、五代板蕩之餘,在者希矣。遼國大樂,晉代所傳。《雜禮》雖見坐部樂工左右各一百二人,蓋亦以《景雲》遺工充坐部;其坐、立部樂,自唐已亡,可考者唯《景雲》四部樂舞而已。



 玉磬,方響,苓箋,
 築,臥箜篌,大箜篌,小箜篌,大琵琶,小琵琶,大五弦,小五弦,吹葉,大笙,
 小笙,觱篥,搏巢,簫,銅鈸,長笛,尺八笛,短笛。



 以上皆一人。



 毛員鼓,連鞀鼓,
 貝。以上皆二人,餘每器工一人。



 歌二人,舞二十人,分四部,《景雲》舞八人,《慶雲》樂舞四人,《破陣》樂舞四人,《承天》樂舞四人。



 大樂調:雅樂有七音,大樂亦有七聲,謂之七旦:一曰娑訢力,平聲;二曰雞識,長聲,三曰沙識,質直聲;四曰沙侯
 加濫,應聲;五曰沙臘,應和聲,六曰般贍,五聲;七曰俟利琥,斛牛聲。自隋以來,樂府取其聲,四旦二十八調整為大樂。



 娑訢力旦:正宮,高宮,中呂宮,通調宮,南呂宮,仙呂宮,黃鐘宮。



 雞識旦,越調,大食調,高大食調,雙調,小食調,歇指調,林鐘商調。沙識旦,大食角,
 高大食角,雙角,小食角,歇指角,林鐘角,越角。



 般涉旦:中呂調,正平調,高平調,
 仙呂調,黃鐘調,般涉調,高般涉調。



 右四旦二十八調,不用黍律,以琵琶弦葉之。皆從濁至清,迭更其聲,下益濁,上益清。七七四十九調,餘二十一調失其傳。蓋出《九部》樂之《龜茲部》云。



 大樂聲:各調之中,度曲協音,其聲凡十,曰:五、凡、工、尺、上、一、四、六、勾、合,近十二雅律,於律呂各闕其一,猶雅音之不及商也。



 散樂殷人作靡靡之樂,其聲往而不反,流為鄭、衛之聲。秦、漢之間,秦、楚聲作,鄭、衛浸亡。漢武帝以李延年典樂府,稱用西涼之聲。今之散樂,俳優、歌舞雜進,往往漢樂府之遺聲。晉天福三年,遣劉邈以伶官來歸,遼有散樂,蓋由此矣。遼冊皇后儀:呈百戲、角抵、戲馬以為樂。



 皇帝生辰樂次:酒一行觱篥起,歌。



 酒二行歌,手伎入。



 酒三行琵琶獨彈。



 餅、茶、致語。



 食入,雜劇進。



 酒四行闕。



 酒五行笙獨吹,鼓笛進。



 酒六行箏獨彈,築球。



 酒七行歌曲破,角抵。



 曲宴宋國使樂次:
 酒一行觱篥起,歌。



 酒二行歌。



 酒三行歌,手伎入。



 酒四行琵琶獨彈。



 餅、茶、致語。



 食入,雜劇進。



 酒五行闕。



 酒六行笙獨吹,合《法曲》。



 酒七行箏獨彈。



 酒八行歌,擊架樂。



 酒九行歌,角抵。



 散樂,以三音該三才之義,四聲調四時之氣,應十二管之數。截竹為四竅之笛,以葉音聲,而被之弦歌。三音:天音揚,地音抑,人音中,皆有聲無文。四時:春聲曰平,夏聲曰上,秋聲曰去,冬聲曰入。散樂器:觱篥、簫、笛、笙、琵琶、五弦、箜篌、箏、方響、杖鼓、第二鼓、第三鼓、腰鼓、大鼓、鞚、拍板。



 雜戲:自齊景公用倡優侏儒,至漢武帝設魚龍曼延之戲,後漢有繩舞、自刳之伎,杜佑以為多幻術,皆出西域。
 哇俚不經,故不具述。



 鼓吹樂鼓吹樂,一曰短簫鐃歌樂,自漢有之,謂之軍樂。《遼雜禮》,朝會設熊羆十二案,法駕有前後部鼓吹,百官鹵簿皆有鼓吹樂。



 前部,鼓吹令二人,扛鼓十二,金鉦十二,大鼓百二十,
 長鳴百二十,鐃十二,鼓十二,歌二十四,管二十四,蕭二十四,笳二十四。



 後部,鼓吹丞二人,大角百二十,
 羽葆十二,鼓十二,管二十四,簫二十四,饒十二,鼓十二,簫二十四,笳二十四。



 右前後鼓吹,行則導駕奏之,朝會則列仗,設而不奏。



 橫吹樂橫吹亦軍樂,與鼓吹分部而同用,皆屬鼓吹令。



 前部:大橫吹百二十,節鼓二,笛二十四,觱篥二十四,笳二十四,桃皮觱篥二十四,扛鼓十二,金鉦十二,
 小鼓百二十,中鳴百二十,羽葆十二,鼓十二,管二十四,蕭二十四,笳二十四。



 後部:小橫吹百二十四,笛二十四,
 蕭二十四,觱篥二十四,桃皮觱篥二十四。



 百官鼓吹,橫吹樂,自四品以上,各有增損,見《儀衛志》。



 自周衰,先王之樂浸以亡缺,《周南》變為《秦風》。始皇有天下,鄭、衛、秦、燕、趙、楚之聲迭進,而雅聲亡矣。漢、唐之盛,文事多西音,是為大樂、散樂;武事皆北音,是為鼓吹、橫吹樂。雅樂在者,其器雅,其音亦西雲。



\end{pinyinscope}