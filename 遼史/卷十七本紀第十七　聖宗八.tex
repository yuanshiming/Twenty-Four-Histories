\article{卷十七本紀第十七 聖宗八}

\begin{pinyinscope}

 五
 年春正月乙酉,如混同江。



 二月戊午,禁天下服用明金及金線綺;國親當服者,奏而後用。是月,如魚兒濼。



 三月壬辰,以左丞相張儉為武定軍節度使、同政事門下平章事,鄭弘節臨潢少尹,劉慎行遼興軍節度使,武定軍節度使蕭匹敵契丹行宮都部署,樞密副使楊又玄吏部尚書、參知政事兼樞密使。是月,如長春河魚兒濼,
 其水一夕有聲如雷,越沙岡四十里,別為一陂。



 夏五月,清暑永安山。以蕭從順為太子太師,吳叔達翰林學士,道十馮若穀加太子中允,耶律晨武定軍節度使,張儉彰信軍節度使,呂士宗禮部員外郎,李可舉順義軍節度使。



 秋七月,獵平地松林。



 九月,駐蹕南京。己亥,以蕭迪烈、李紹琪充賀宋太后生辰使副,耶律守寧、劉四端充賀宋主生辰使副。



 冬十月辛未,宋太后遣馮元宗、史方來賀順天節。



 十一月庚子,幸內果園宴,京民聚觀。求進士得七十二人,命賦詩,第其工拙,以張昱等一十四人為太子校書郎,韓欒等五十八人為崇文館校書郎,辛
 丑,以左祗候郎君詳穩蕭羅羅為右夷離畢。



 十二月丁巳,以漢人行宮都部署蕭孝先為上京留守,皇侄長沙郡王謝家奴匡義軍節度使,耶律仁舉興國軍節度使。甲子,蕭守寧為點檢侍衛親軍馬步軍。乙丑,北院樞密使蕭合卓薨。



 戊辰,以北府宰相蕭普古為北院樞密使。己巳,遣蕭諧、李琪充賀宋正旦使副。庚午,以參知政事劉京為順義軍節度使。乙亥,宋使李維、張綸來賀千齡節。



 是歲,燕民以年穀豐熟,車駕臨幸,爭以上物來獻。上禮高年,惠鰥寡,賜酺飲。至夕,六街燈火如晝,士庶嬉游,上亦微行觀之。丁丑,禁工匠不得銷毀金銀器。



 六年春正月己卯朔,宋遣徐奭、裴繼起、張若谷、崔準來賀。庚辰,如鴛鴦濼。



 二月己酉,以迷離己同知樞密院,黃翩為兵馬都部署,達骨只副之,赫石為都監,引軍城混同江、疏木河之間。黃龍府請建堡障三、烽臺十,詔以農隙築之。東京留守八哥奏黃翩領兵入女直界徇地,俘獲人、馬、牛、豕,不可勝計,得降戶二百七十,詔獎諭之。戊午,以耶律野為副點檢,以國舅帳蕭柳氏、徒魯骨領西北路十二班軍、奚王府舍利軍。己巳,南京水,遣使振之。庚午,詔黨項別部塌西設契丹節度使治之。



 三月戊寅朔,以大同軍節度使張儉入為南院樞密使、左丞相兼政
 事令,參知政事吳叔達責授將作少監,出為東州刺史。



 是月,阻卜來侵,西北路招討使蕭惠破之。



 夏四月丁未朔,以武定軍節度使耶律洪古為惕隱。戊申,蒲盧毛朵部多兀惹戶,詔索之。丙寅,如永安山。



 五月辛卯,以東京統軍使蕭慥古為契丹行宮都部署。癸卯,遣西北路招討使蕭惠將兵伐甘州回鶻。六月辛丑,詔凡官畜並印其左以識之。



 秋七月戊申,獵黑嶺。



 八月,蕭惠攻甘州不克,師還。自是阻卜諸部皆叛,遼軍與戰,皆為所敗,監軍涅裡姑、國舅帳太保曷不呂死之。詔遣惕隱耶律洪古、林牙化哥等將兵討之。



 九月,駐蹕遼河滸。



 冬十月丙子,
 曷蘇館諸部長來朝。庚辰,遣使問夏國五月與宋交戰之故。辛巳,以前南院大王直魯袞為烏古敵烈都詳穩。



 庚寅,以蕭孝順、蕭紹宗兼待中,駙馬蕭紹業平章政事,前南院大王胡睹堇同知上京留守,安哥通化州節度使。



 十一月乙丑,宋遣韓翼、田承說來賀順天節。戊辰,西北路招討司小校掃姑訴招討蕭惠三罪,詔都監奧骨禎按之。



 十二月庚辰,曷蘇館部乞建旗鼓,許之。辛巳,詔北南諸部廉察州縣及石烈、彌里之官,不治者罷之。詔大小職官有貪暴殘民者,立罷之,終身不錄;其不廉直,雖處重任,即代之;能清勤自持者,在卑位亦當薦拔;其
 內族受賂,事發,與常人所犯同科。戊戌,遣杜防、蕭蘊充賀宋生辰使副。庚子,駐蹕遼河。七年春正月壬寅朔,宋遣張保維、孫繼業、孔道輔、馬崇至來賀。如混同江。辛亥,以女直白縷為惕隱,蒲馬為嚴母部太師。甲寅,蒲盧毛朵部遣使來貢。



 夏四月乙未,獵黑嶺。



 五月,清暑永安山。西南路招討司奏陰山中產金銀,請置冶,從之。復遣使循遼河源求產金銀之礦。



 六月,禁諸屯田不得擅貨官粟。癸巳,詔蕭惠再討阻卜。



 秋七月己亥朔,詔更定法令。庚子,詔諭駙馬蕭鉏不、公主粘米袞:「爾於後有父母之尊,後或臨幸,祗謁先祖,祗拜空
 帳,失致敬之禮,今後可設像拜謁,」乙巳,詔輦路所經,旁三十步內不得耕種者,不在訴訟之限。



 九月,駐蹕遼河。



 冬十月丁卯朔,詔諸帳院庶孽,並從其母論貴賤。



 十一月,宋遣石中立、石貽孫來賀千齡節,王博文、王雙賀順天節。辛亥,以楊又玄、邢祥知貢舉。己未,匡義軍節度使中山郡王查葛、保寧軍節度使長沙郡王謝家奴,廣德軍節度使樂安郡王遂哥奏,各將之官,乞選伴讀書史,從之。癸亥,以三韓王欽為啟聖軍節度使,楊佶刑部侍郎、甲子,以左千牛衛上將軍耶律古昱為北院大王。



 十二月丁卯朔,遣耶律遂英、王永錫充賀宋太后生辰,蕭速
 撒、馬保永充賀正旦使副。癸酉,以金吾蕭高立為奚舍利軍詳穩。八年春正月己亥,如混同江。庚申,黨項侵邊,破之。甲子,詔州縣長吏勸農。



 二月戊子,燕京留守蕭孝穆乞於拒馬河接宋境上置戍長巡察,詔從之。



 三月,駐蹕長春河。



 夏五月,清暑永安山。



 六月,以韓寧、劉湘充賀宋太后生辰使副,吳克荷充賀夏國王李德昭生辰使。癸巳,權北院大王耶律鄭留奏,今歲十一月皇太子納妃,諸族備會親之帳。詔以豪盛者三十戶給其費。



 秋七月丁酉,以遙輦帳郎君陳哥為西北路巡檢,與蕭諧領同管二招
 討地。以南院大王耶律敵烈為上京留守。戊戌,獵平地松林。



 九月壬辰朔,以渤海宰相羅漢權東京統軍使。壬子,幸中京。北敵烈部節度使耶律延壽請視諸部,賜旗鼓,詔從之。癸丑,阻卜別部長胡懶來降。乙卯,阻卜長舂古來降。



 冬十月,宋遣唐肅、葛懷愍來賀順天節。樞密使、魏王耶律斜軫孫婦阿聒指斥乘與,其孫骨欲為之隱,事覺,乃並坐之,仍籍其家。詔燕城將士,若敵至,總管備城之東南,統軍守其西北,馬步軍備其野戰,統軍副使繕壁壘,課士卒,各練其事。



 十一月丙申,皇太子納妃蕭氏。以耶律求翰為北院大王。



 十二月辛酉朔,以遙輦
 太尉謝佛留為天云軍詳穩。壬申,以前北院大王耶律留寧為雙州節度使,庚筠崇德宮都部署,謝十永興宮都部署,旅墳宜州節度使。口庵遼州節度使,耶律野同知中京留守,耶律曷魯突愧為大將軍。丁丑,詔庶孽雖己為良,不得預世選。丁亥,宋遣寇瑊、庚德來賀千齡節,朱諫、曹英、張逸、劉永釗賀來歲兩宮正旦。詔兩國舅及南、北王府乃國之貴族,賤庶不得任本部官。



 是歲,放進士張宥等五十七人。



 九年春正月,至自中京。



 二月戊辰,遣使賜高麗王欽物。如斡凜河。



 夏五月,清暑永安山。



 六月戊子朔,以長沙郡王
 謝家奴為廣德軍節度使,樂安郡王遂哥匡義軍節度使,中山郡王查葛保定軍節度使,進封潞王,豫章王貼不長寧軍節度使。以耶律思忠、耶律荷、耶律皓、遙輦謝佛留、陳邈、韓紹一、韓知白、張震充賀宋兩宮生辰及來歲正旦使副。



 秋七月戊午朔,如黑嶺。



 八月己丑,東京舍利軍詳穩大延琳囚留守、駙馬都尉蕭孝先及南陽公主,殺戶部使韓紹勛、副使王嘉、四捷軍都指揮使蕭頗得,延琳遂僭位,號其國為興遼,年為天慶。初,東遼之地,自神冊來附,未有榷酤監曲之法,關市之徵亦甚寬弛。馮延休、韓紹勛相繼以燕地平山之法繩之,民不堪命。燕又
 仍歲大饑,戶部副使王嘉復獻計造船,使其民請海事者,漕粟以振燕民,水路艱險,多至覆沒。雖言不信,鞭楚搒掠,民怨思亂,故延琳乘之,首殺紹勛、嘉,以快其眾。延琳先事與副留守王道平謀,道平夜棄其家,逾城走,與延琳所遣召黃龍府黃翩者,俱至行在告變。上即徵諸道兵,以時進討。時國舅詳穩蕭匹敵治近延琳,先率本管及家兵據其要害,絕其西渡之計。渤海太保夏行美亦舊主兵,戍保州,延琳密馳書,使圖統師耶律蒲古。



 行美乃以實告,蒲古得書,遂殺渤海兵八百人,而斷其東路。



 延琳知黃龍、保州皆不附,遂分兵西取瀋州,其節度
 使蕭王六初至,其副張傑聲言欲降,故不急攻。及知其詐,而已有備,攻之不克而還。時南、北女直皆從延琳,高麗亦稽其貢。及諸道兵次第皆至,延琳嬰城固守。



 冬十月丙戌朔,以南京留守燕王蕭孝穆為都統,國舅詳穩蕭匹敵為副統,奚六部大王蕭蒲奴為都監以討之。



 十一月乙卯朔,如顯陵。丙寅,以瀋州節度副使張傑為節度使,其皇城進士張人紀、趙睦等二十二人入朝,試以詩賦,皆賜第,超授保州戍將夏行美平章事。壬申,以駙馬劉四端權知宣徽南院事。



 十二月丁未,守遣仇水、韓永錫來賀千齡節。命耶律育、吳克荷、蕭可觀、趙利用充
 賀宋生辰使副,耶律元古、崔閏、蕭昭古、竇振充來歲賀宋正旦使副。



 十年春正月乙卯朔,宋遣王夷簡、竇處約、張易、張士宜來賀。二月,幸龍化州。



 三月甲寅朔,詳穩蕭匹敵至自遼東,言都統蕭孝穆去城四面各五里許,築城堡以圍之。駙馬延寧與其妹穴地遁去,惟公主崔人在後,為守陴者覺而止。



 夏四月,如乾陵。以耶律行平為廣平軍節度使,夏行美為忠順軍節度使,李延弘知易州,蕭從順加太子太師。



 五月戊申,清暑柏坡。



 秋七月壬午,詔來歲行貢舉法。



 八月丙午,東京賊將楊詳世密送款,夜開南門
 納遼軍。擒延琳,渤海平。



 冬十月駐蹕長寧澱。



 十一月辛亥,南京留守燕王蕭孝穆以東征將士凱還,戎服見上,上大加宴勞。翌日,以孝穆為東平王、東京留守,國舅詳穩、駙馬都尉蕭匹敵封蘭陵郡王,奚王蒲奴加侍中;以權燕京留守兼侍中蕭惠為燕京統軍使,前統軍委窊大將軍、節度使,宰相兼樞密使馬保忠權知燕京留守,奚王府都監蕭阿古軫東京統軍使。詔渤海舊族有勛勞材力者敘用,餘分居來、隰、遷、潤等州。



 十二月乙巳,宋遣梅詢、王令傑來賀千齡節。漆水郡王耶律敵烈加尚父,烏古部節度使蕭普達為乙室部大王,尚書左僕射
 蕭琳為臨海軍節度使。



 十一年春正月己酉朔,如混同江。



 二月,如長春河。



 三月,上不豫。



 夏五月,大雨水,諸河橫流,皆失故道。



 六月丁丑朔,駐蹕大福河之北。己卯,帝崩於行宮,年六十一,在位四十九年。景福元年閏十月壬申,上尊謚曰文武大孝宣皇帝,廟號聖宗。



 贊曰:聖宗幼沖嗣位,政出慈闈。及宋人二道來攻,新御甲胄,一舉而復燕、雲,破信、彬,再舉而躪河、朔,不亦偉歟!既而侈心一後,佳兵不祥,東有茶、陀之敗,西有甘州之喪。此狃於常勝之過也。然其踐阼四十九年,理冤滯,舉
 才行,察貪殘,抑奢僭,錄死事之子孫,振諸部之急乏,責迎合不忠之罪,卻高麗文樂之歸。遼之諸帝,在位長久,今名無穹,其唯聖宗乎!



\end{pinyinscope}