\article{卷十六本紀第十六 聖宗七}

\begin{pinyinscope}

 七年春正月甲辰,如達離山。



 二月乙丑朔,拜日,如渾河。



 三月辛丑,命東北越里篤、剖阿里、奧裏米、蒲奴里、鐵驪等五部歲貢貂皮六萬五千,馬三百。丙午,烏古部節度使蕭普達討叛命敵烈,滅之。



 夏四月,拜日。丙寅,振川、饒二州饑。辛未,振中京貧乏。癸酉,禁匿名書。壬辰,以三司使呂德懋為樞密副使。



 閏月壬子,以蕭進忠為彰武軍
 節度使兼五州制置。戊午,吐蕃王並裡尊奏,凡朝貢,乞假道夏國,從之。



 五月丙寅,皇子宗真封梁王,宗元永清軍節度使,宗簡右衛大將軍,宗願左驍衛大將軍,宗偉右衛大將軍,皇侄宗範昭義軍節度使,宗熙鎮國軍節度使,宗亮繹州節度使,宗弼濮州觀察使,宗突曹州防禦使,宗顯、宗肅皆防禦使,以張儉守司徒兼政事令。



 六月丙申,品打魯瑰部節度使勃魯里至鼻灑河,遇微雨,忽天地晦冥,大風飄四十三人飛旋空中,良久乃墮數里外。勃魯里幸獲免。一酒壺在地乃不移。八月丙午,行大射柳之禮。庚申,以耶律留寧、吳守達使宋賀生辰,蕭高九、
 馬胎謀使宋賀正旦。加平章蕭弘義開府儀同三司、尚父兼政事令。



 秋七月甲子,詔翰林待詔陳升寫《南征得勝圖》於上京五鸞殿。丁卯,蒲奴里部來貢。



 九月庚申朔,蒲呢國使奏本國與烏里國封壤相接,數侵掠不寧,賜詔諭之。戊辰,詔內外官,因事受賕,事覺而稱子孫僕從者,禁之。庚午,錄囚。括馬給東征軍。是月,駐蹕土河川。



 冬十月,名中京新建二殿曰延慶,曰永安。壬寅,以順義軍節度使石用中為漢人行宮都部署。丙辰,詔以東平郡王蕭排押為都統,殿前都點檢蕭虛列為副統,東京守耶律八哥為都監伐高麗。仍諭高麗守吏,能率眾自歸
 者,厚賞;堅壁相拒者,追悔無及。



 十一月壬戌,以呂德懋知吏部尚書,楊又玄知詳覆院,劉晟為霸州節度使,北府宰相劉慎行為彰武軍節度使。庚辰,禁服用明金,縷金、貼金。戊子,幸中京。



 十二月丁酉,宋遣呂夷簡、曹璋來賀千齡節。是月,蕭排押等與高麗戰於茶、陀二河,遼軍失利,天雲、右皮室二軍沒溺者眾,遙輦帳詳穩阿果達、客省使酌古、渤海詳穩高清明、天云軍詳穩海裡等皆死之。



 放進士張克恭等三十七人及第。



 八年春正月,宋遣陳堯佐、張群來賀。壬戌,鐵驪來貢。



 建景宗廟於中京。封沙州節度使曹順為敦煌郡王,二月
 丁未,以前南院樞密使韓制心為中京留守,漢人行宮都部署王繼忠南院樞密使。丙辰,祭風伯。



 三月己未,以契丹弘義宮使赫石為興聖宮都部署,前遙恩拈部節度使控骨里積慶宮都部署,左祗候郎唱耶律罕因捷軍都監。乙亥,東平王蕭韓寧、東京留守耶律八哥、國舅平章事蕭排押、林牙要只等討高麗還,坐失律,數其罪而釋之。己卯,詔加征高麗有功渤海將校官。壬午,閱飛龍院馬。癸未,回跋部太師踏刺葛來貢。丙戌,置東京渤海承奉官都知押班。



 夏四月戊子朔,如緬山。



 五月壬申,以駙馬蕭克忠為長寧軍節度使。乙亥,遼寧州渤海戶於
 遼、土二河之間。己卯,曷蘇館惕隱阿不葛、宰相賽刺來貢。



 六月戊子,錄征高麗戰段將校子弟。己丑,以左夷離畢蕭解里為西南面招討使,御史大夫蕭要只為夷離畢。己亥,惕隱耶律合葛為南府宰相,南面林牙耶律韓留為惕隱。癸卯,弛大擺山猿嶺採木之禁。乙巳,以南皮室軍校等討高麗有功,賜金帛有差。



 秋七月己未,征高麗戰歿諸將,詔益對其妻。庚申,以東北路詳穩耶律獨迭為北院大王。辛酉,肴里、涅哥二奚軍征高麗有功,皆賜金帛。癸亥,詔阻卜依舊歲貢馬千七百,駝四百四十,豹鼠皮萬,青鼠皮二萬五千。戊辰,觀稼。己巳,回跋部太
 保麻門來貢。庚午,觀市,曲赦市中系囚。命解寧、馬翼充賀宋生辰使副。



 八月庚寅,遣郎君曷不呂籌率諸部兵會大軍討高麗。



 九月己巳,以石用中參知政事。宋遣崔遵度、正應昌來賀千齡節。壬申,錄囚。甲戌,復錄囚。庚辰,曷蘇館惕隱阿不割來貢。壬午,駐蹕土河川。



 冬十月乙酉,詔諸道,事無巨細,己斷者,每三月一次條奏。戊子,遣耶律繼崇、鄭玄瑕賀寧正旦。癸巳,詔橫帳三房不得與卑小帳族為婚;凡嫁娶,必奏而後行。癸卯,以前北院大王建福為阿扎割只。甲辰,改東路耗裏太保城為咸州,建節以領之。



 十一月甲寅,置雲州宣德縣。



 十二月辛卯,
 駐蹕中京。乙巳,以廣平郡正宗業為中京留守、大定尹,韓制心為惕隱。辛亥,高麗王詢遣使乞貢方物,詔納之。



 九年春天正月,宋遣劉平、張元普來賀。



 二月,如鴛鴦濼。



 五月庚午,耶律資忠使高麗還,王詢表請稱藩納貢,歸所留王人只刺里。只刺裏在高麗六年,忠節不屈,以為林牙。辛示,遣使釋王詢罪,並尤其請。癸酉,以耶律宗教檢校太傅,宗誨啟聖軍節度使,劉最太子太傅,仍賜保節功臣。



 秋七庚戌朔,日有食之,詔以近臣代拜救日。甲寅,遣使賜沙州回鶻敦煌郡王曹順衣物。以查刺、耿元吉、韓九、宋璋為來年賀宋生辰正旦使副。



 九月戊午,以
 駙馬蕭紹宗平章事。丁卯,文武百僚奉表上尊號,不許;表三上,乃從之。乙亥,沙州回鵲敦煌郡王曹順遣使來貢。括諸道漢民馬賜東征軍。以夷離畢延寧為兵馬副都部署,總兵東征。是月,駐蹕金瓶濼。宋遣宋綬、駱繼倫賀千齡節。冬十月戊寅朔,以涅里為奚王都監,突迭里為北王府舍利軍詳穩。郎君老使沙州還,詔釋宿累。國家舊使遠國,多用犯徒罪而有才略者,使還,即除其罪。戊子,西南招討奏黨項部有宋犀族輸貢不時,常有他意,宜以時遣使督之。詔曰:「邊鄙小族,歲有常貢。邊臣驕縱,徵斂無度,彼懷俱不能自達耳。



 第遣清慎官將,示以恩
 信,無或侵漁,自然效順。」復奏帝居、迭烈德部言節度使韓留有惠政,今當代,請留。上命進其治狀。



 辛丑,如中京。壬寅,大食國道使進象及方物,為子冊割請婚。十一月丁巳,以漆水郡王韓制心為南京留守、析津尹、兵馬都總管。己未,以夷離畢蕭孝順為南面諸行宮都部署,加左僕射。十二月丁亥,禁僧燃射煉指。戊子,詔中京建太祖廟,制度、祭器皆從古制。乙巳,招來年冬行大冊禮。



 放進士張仲舉等四十五人。



 太平元年春正月丁丑朔,宋使魯宗道、成吉來賀。如渾河。



 二月乙卯,幸拔河。壬戌,獵高柳林。



 三月戊戌,皇子勃
 只生。庚子,駙馬都尉蕭紹業建私城,賜名睦州,軍曰長慶。是月,太食國王復遣使請婚,封王子班郎君胡思裏女可老為公主,嫁之。



 夏四月戊申,東京留守奏,女直三十部酋長請各以其子詣闕關祗候。詔與其父俱來受約。乙卯,錄囚。丁卯,置來州。



 是月,清暑緬山。



 秋七月甲戌朔,賜從獵女直人秋衣。乙亥,遣骨裏取石晉所上玉璽於中京。阻卜來貢。辛巳,如沙嶺。是月,獵潢河。



 九月,幸中京。



 冬十月丁未,敵烈酋長頗自來貢馬、駝。戊申,錄囚。壬子,宋使李懿、王仲賓來賀千齡節,及蘇惟甫、周鼎賀來歲元正,即遣蕭善、程翥報聘。黨項長曷魯來貢。己未,以
 薩敏解里為都點檢,高六副點檢,耶律羅漢奴左皮室詳穩,嗓姑右皮室詳穩,聊了西北路金吾,耶律僧隱御史大夫,求哥駙馬都尉,蕭舂、骨裏並大將軍。庚申,幸通天觀,觀魚龍曼衍之戲。翌日,再幸。還,升玉輅,自內三門入萬壽殿,奠酒七廟御容,因宴宗室。



 十一月癸未,上御昭慶殿,文武百僚奉冊上尊號曰睿文英武遵道至德崇仁廣孝功成治定昭聖神贊天輔皇帝,大赦,改元太平,中外官進級有差。宋遣使來聘,夏、高麗遣使來貢。甲申,冊皇子梁正宗真為皇太子。



 二年春正月,如納水鉤魚。



 二月辛丑朔,駐蹕魚兒濼。



 三
 月甲戌,如長春州。丁丑,宋使薛貽廓來告宋主恆殂,子禎嗣位。遣都點檢耶律僧隱等充宋祭奠使副,林牙蕭日新、觀察馮延休充宋後吊慰使副。戊寅,遣金吾耶諧領、引進姚居信充宋主吊慰使副。戊子,為宋主飯三京僧。是月,地震,雲、應二州屋摧地陷,嵬白山裂數百步,泉湧成流。



 夏四月,如緬山清暑。



 五月乙亥,參知政事石用中薨。庚辰,鐵驪遣使獻兀惹十六戶。六月己未,宋遣使薛由等來饋其先帝遺物。



 秋七月己卯,以耶律信寧為奉陵軍節度使,高麗國參知政事王同顯靜海軍節度使,耶律遂忠長寧軍節度使,耿延毅昭德軍節度
 使,高守貞河西軍節度使。



 九月癸巳,遣尚書僧隱、韓格賀宋主即位。



 冬十月壬寅,遣堂後宮張克恭充賀夏國王李德昭生日使,耶律掃古、韓王充賀宋太后生日使副,耶律仙寧、史克忠充賀宋正旦使副。是月,駐蹕胡魯吉思澱。癸卯,賜宰臣呂德懋、參知政事吳叔達、樞密副使楊又玄、右丞相馬保忠錢物有差。



 辛亥,至上京,曲赦幾內囚。



 十一月丙戌,宋遣使來謝。



 十二月辛丑,高麗王詢薨,其子欽遣使來報,即命使冊欽為高麗國王。甲寅,宋遣劉燁、郭志言來賀千齡節。



 是年,放進士張漸等四十七人。
 三年春正月丙寅朔,如納水鉤魚。以僧隱為平章事。乙亥,以蕭臺德為南王府都監,林牙耶律信寧西北路招討都監。辛巳,賜越國公主私城之名曰懿州,軍曰慶懿。



 二月丙申,以丁振為武信軍節度使,改封蘭陵郡王,戊申,以東平郡王蕭排押為西南面都招討,進封豳王。



 夏四月,以耶律守寧為都點檢。



 五月,清暑緬山。



 六月戊申,以南院宣微使劉烴參知政事,蕭孝惠為副點檢,蕭孝恭東京統軍兼沿邊巡檢使。戊午,以蕭璉為左夷離畢,蕭琳為詳穩。



 秋七月戊寅,以南府宰相耶律合葛為上京留守,封漆水郡王。丙戌,以皇后生辰為順天節。丁亥,
 賜緬山名曰永安。是月,獵赤山。



 閏九月壬辰朔,以蕭伯達、韓紹雍充賀宋正旦使副,唐骨德、程昭文賀宋生辰使副。



 冬十月庚辰,宋遣薛奎、郭盛來賀順天節,王臻、慕容惟素賀千齡節。東征軍奏:「統帥諧領、常袞課奴率師自毛母國嶺入,林牙高九、裨將大匡逸等率師鼓山嶺入。閏月末至撻離河,不遇敵而還。以是月會於弘怕只嶺,駝、馬死者甚眾。」



 駐蹕遼河。



 十一月辛卯朔,以皇侄宗範為歸德軍節度使,北府宰相蕭孝穆南京留守,封燕王,南京留守韓制心南院大王、兵馬都總管,仇正燕京轉運使。



 十二月壬戌,以宗範為平章事,封三韓郡王,仇道衡中
 京副留守,馮延休順州刺史,郎玄化西山轉運使,趙其樞密直學士。丁卯,以蕭永為太子太師。己卯,封皇子重元秦國王。



 四年春正月庚寅朔,宋遣張傳、張士禹、程琳、丁保衡來賀。如鴨子河。



 二月己未朔,獵撻魯河。詔改鴨子河曰混同江,撻魯河曰長春河。



 三月戊子朔,千齡節,詔賜諸宮分耆老食。



 夏四月癸酉,以右承相馬保忠之子世弘使嶺表,至平地松林為盜所殺,特贈昭信軍節度使。



 五月,清暑永安山。



 六月己未,南院大王韓制心薨。戊辰,以鄭弘節為兵部郎中、劉慎行順義軍節度使。辛未,以燕王蕭孝穆子順
 為千牛衛將軍。甲戌,以中山郡王查哥為保靜軍節度使,樂安郡王遂哥廣德軍節度使,蕭解裏彰德軍節度使。庚辰,以遼興軍節度使周王胡都古為臨海軍節度使,漆水郡王敵烈南院大王。



 秋七月,如秋山。



 八月丙辰朔,以韓紹芳為樞密直學士,駙馬蕭匹敵都點檢。



 九月,以駙馬蕭紹宗為武定軍節度使,耶律宗福安國軍節度使。冬十月,駐蹕遼河。宋遣蔡齊、李用和來賀千齡節。



 十一月,追封南院大王韓制心為陳玉。



 十二月,以蕭從政為歸義軍節度使,康筠監門衛,充賀宋正旦使副。



 是年,放進士李炯等四十七人。



\end{pinyinscope}