\article{卷九十三 列傳第三十一}

\begin{pinyinscope}

 蕭陶蘇斡耶律阿息保蕭乙薛蕭胡篤蕭陶蘇斡,字乙辛隱,突呂不部人。四世祖因吉,發長五尺,時呼為「長發因吉」。祖裡拔,奧隗部節度使。



 陶蘇斡謹願,不妄交。伯父留哥坐事免官,聞重元亂,挈家赴行在。時陶蘇斡雖幼,已如成人,補筆硯小底。累遷祗候郎君,轉樞密院侍御。咸雍五年,遷崇德宮使。會有訴北南院
 聽訟不直者,事下,陶蘇斡悉改正之,為耶律阿思所忌。帝欲召用,輒為所沮。八年,歷漠北滑水馬群太保,數年不調,嘗曰:「用才未盡,不若閑。」乾統中,遷漠南馬群太保,以大風傷草,馬多死,鞭之三百,免官。九年,徙天齊殿宿衛。明年,穀價翔躍,宿衛士多不給,陶蘇斡出私廩周之,召同知南院樞密使事。



 天慶四年,為漢人行宮副部署。時金兵初起,攻陷寧江州。



 天祚召群臣議,陶蘇斡曰:「女直國雖小,其人勇而善射。自執我叛人蕭海裏,勢益張。我兵久不練,若遇強敵,稍有不利,諸部離心,不可制矣。為今之計,莫若大發諸道兵,以威壓之,庶可服也。」北院
 樞密使蕭得裏底曰:「如陶蘇斡之謀,徒示弱耳。但發滑水以北兵,足以拒之。」遂不用其計。



 數月間,邊兵屢北,人益不安。饒州渤海結構頭下城以叛,有步騎三萬餘,招之不下。陶蘇斡帥兵往討,擒其渠魁,斬首數千級,得所掠物,悉還其主。及耶律章奴叛,陶蘇斡與留守耶律大悲奴為守御。章奴既平,陶蘇斡請曰:「今邊兵懈弛,若清暑嶺西,則漢人嘯聚,民心益搖。臣愚以為宜罷此行。」



 不納。乃命陶蘇斡控扼東路,招集散卒。



 後以太子太傅致仕,卒。



 耶律阿息保,字特里典,五院部人。祖胡劣,太祖時徙居
 西北部,世為招討司吏。



 阿息保慷慨有大志,年十六,以才幹補內史。天慶初,轉樞密院侍御。金人起兵城境上,遣阿息保間之,金人曰:「若歸阿疏,敢不聽命。」阿息保具以聞。金兵陷寧江州,邊兵屢敗,遣阿息保與耶律章奴等齊書而東,冀以脅降。阿息保曰:「臣前使,依詔開諭,略無所屈。將還,謂臣曰:『若所請不遂,無相見。』今臣請獨往。」不聽。將行,別蕭得裏底曰:「不肖適異國,必無生還,願公善輔國家。」既至,阿息保見執。久乃遁歸。



 及天祚敗績,遷都巡捕使。六年,從阿疏討耶律章奴,加領軍衛大將軍。阿疏將兵而東,阿息保送至軍,乃還。天祚怒其專,鞭之
 三百。尋為奚六部禿裏太尉。後阿疏反,阿息保以遍師進擊,臨陣墜馬,被擒。因阿疏有舊得免。時阿疏頗好殺,阿息保謂曰:「欲舉大事,何以殺為!」由是全活著眾。會阿疏敗,乃還。以戰失利,囚中京數歲。



 保大二年,金兵至中京,始出獄。尋為敵烈皮室詳穩。是時,魏王淳僭號,屢遣人以書來招。阿息保封書以獻,因諫曰:「東兵甚銳,未可輕敵。」及右輦鐸之敗,天祚奔竄,召阿息保,不時至,疑有貳心,並怒為淳所招,殺之。



 初,阿息保知國將亡,前後諫甚切。及死以非罪,人尤惜之。



 蕭乙薛,字特免,國舅少父房之後。性謹願。壽隆間,累任
 劇官。



 天慶初,知國舅詳穩事,遷殿前副點檢。金兵起,為行軍副都統。以戰失利,罷職。六年,出為武定軍節度使,遷西京留守。明年,討劇賊董厖兒,戰易水西,大破之。以功為北府宰相,加左僕射,兼東北路都統。十年,金兵陷上京,詔兼上京留守、東北路統軍使。為政寬猛得宜,民之窮困者,輒加振恤,眾咸愛之。



 保大二年,金兵大至,乙薛軍潰,左遷西南面招討使。以部民流散,不赴。及天祚播遷,給侍從不闕,拜殿前都點檢。



 凡金兵所過,諸營敗卒復聚上京,遣乙薛為上京留守以安撫之。



 明年,盧彥倫以城叛,乙薛被執數月,以居官無過,得釋。



 後為耶律
 大石所殺。



 蕭胡篤,字合術隱。其先撒葛只,太祖時願隸宮分,遂為太和宮分人。



 曾祖敵魯,明醫。人有疾,觀其形色即知病所在。統和中,宰相韓德讓貴寵,敵魯希旨,言德讓宜賜國姓,籍橫帳,由是世預太醫選。子孫因之入官者眾。



 胡篤為人便玦,與物無件。清寧初,補近侍。大安元年,為彰愍宮太師。壽隆二年,轉永興宮太師。天慶初,累遷至殿前副點檢。五年,從天祚東征,為先鋒都統,臨事猶豫,凡隊伍皆以圍場名號之。進至剌離水,與金兵戰,敗,大軍亦卻。及討耶律章奴,以籍私奴為軍,遷知北院樞密使
 事,卒。



 胡篤長於騎射,見天祚好游畋,每言從禽之樂,以逢其意。



 天祚悅而從之。國政隳廢,自此始云。



 論曰:「甚矣,承平日久,上下狃於故常之可畏也!天慶之間,女直方熾,惟陶蘇斡明於料敵,善於忠諫;惜乎天祚痼蔽,不見信用。阿息保不死阿疏之難,乙薛甘忍盧彥倫之執,大節已失矣,他有所長,亦奚足取。胡篤以游畋逢迎天祚而隳國政,可勝罪哉。」



\end{pinyinscope}