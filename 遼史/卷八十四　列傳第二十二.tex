\article{卷八十四 列傳第二十二}

\begin{pinyinscope}

 蕭奪剌蕭普達耶律侯哂耶律古昱耶律獨攧蕭韓家蕭烏野蕭奪剌,字挩懶,遙輦窪可汗宮人。祖涅魯古,北院樞密副使。父撒抹,字胡獨堇,重熙初補祗候郎君,累遷北面林牙。



 十九年,從耶律宜新、蕭蒲奴伐夏,至蕭惠敗績之地,獲偵候者,知人煙聚落,多國人陷沒而不能還者,盡俘以歸,拜大父敞穩,知山北道邊境事。清寧初,歷西南面、
 西北路招討使,加同中書門下平章事,卒。



 奪剌體貌豐偉,騎射絕人。由祗候郎君升漢人行宮副部署。



 後為烏古敵烈統軍使,克敵有功,加龍虎衛上將軍,授西北路招討使。因陳北邊利害,請以本路諸部與倒塌嶺統軍司連兵屯戍。再表,不納。改東北路統軍使。



 乾統元年,以久練邊事,復為西北路如討使。北阻卜耶睹刮率鄰部來侵,奪剌逆擊,追奔數十里。二年,乘耶睹刮無備,以輕騎襲之,獲馬萬五千匹,牛羊稱是。先是,有詔方面無事,招討、副統軍、都監內一員入觀。是時同僚皆闕,奪剌以軍事付幕吏而朝,坐是免官。改西京留守,復為東北
 路統軍使。卒於官。



 蕭普達,字彈隱。統和初,為南院承旨。開泰六年,出為烏古部節度使。七年,敵烈部叛,討平之,徙烏古敵烈部都監。



 遣敵烈騎卒取北阻卜名馬以獻,賜詔褒獎。重熙初,改烏古敵烈部都詳穩,討諸蕃有功。



 普達深練邊事,能以悅使人。有所俘獲,悉散麾下,由是大得眾心。歷西南面招討使。黨項叛入西夏,普達討之,中流矢,歿於陣。帝聞,惜之,賻贈加厚。



 耶律侯哂,字禿寧,北院夷離堇蒲古只之後。祖查只,北院大王。父忽古,黃皮室詳穩。



 侯哂初為西南巡邊官,以
 廉潔稱,累遷南京統軍使,尋為北院大王。重熙十一年,黨項部人多叛入西夏,侯哂受詔,巡西邊沿河要地,多建城堡以鎮之,徙東京留守。十三年,與知府蕭歐里斯討蒲盧毛朵部有功,加兼侍中。致仕,卒。



 耶律古昱,字磨魯堇,北院林牙突呂不四世孫。有膂力,工馳射。



 開泰間,為烏古敵烈部都監。會部人叛,從樞密使耶律世良討平之,以功詔鎮撫西北部。教以種樹、畜牧,不數年,民多富實。中京盜起,命古昱為巡邏使,悉擒之。上親征渤海,將黃皮室軍,有破敵功,累遷御史中丞,尋授開遠軍節度使,徙鎮歸德。



 重熙二十一年,改天成軍節
 度使,卒於官,年七十,贈同中書門下平章事。二子:宜新,兀沒。



 宜新,重熙間從蕭惠討西夏。惠敗績,宜新一軍獨全,拜北院大王。兀沒,大康三年為漢人行官副部署。乙辛誣害太子,詞連兀沒,帝釋之。是秋,乙辛復奏與蕭楊九私議宮壺事,被害。



 乾統間,贈同中書門下平章事。



 耶律獨攧,字胡獨堇,太師古昱之子。



 重熙初,為左護衛,將禁兵從伐夏有功,授十二行乣司徙。



 再舉伐夏,獨攧括山西諸郡馬。還,遷拽剌詳穩。西南未平,命獨攧同知金肅軍事,夏人來侵,擊敗之,進涅剌奧隗部節度使。



 清寧元年,召為皇太后左護衛太保。四年,改寧遠軍節度
 使。東路饑,奏振之。歷五國、烏古部、遼興軍三鎮節度使,四捷軍詳穩。大康元年卒,追贈同中書門下平章事。子阿思,有傳。蕭韓家,國舅之族。性端簡,謹願,動循禮法。



 清寧中,為護衛太保。大康二年,遷知北院樞密副使。三年,經畫西南邊天池舊塹,立堡砦,正疆界,刻石而還,為漢人行宮都部署。是年秋獵,墮馬卒。



 蕭烏野,字草隱,其先出興聖宮分,觀察使塔裏直之孫也。



 性孝佛,尚禮法,雅為鄉黨所稱。



 重熙中,補護衛,興宗見其勤恪,遷護衛太保。清寧九年,佐耶律仁先平重元亂,以功加
 團練使。時敵烈部數為領部侵擾,民多困弊,命烏野為敵烈部節度使,恤困窮,省徭役,不數月,部人以安。尋以母老,歸養於家。母亡,尤極哀毀。服闋,歷官興聖、延慶二宮使,卒。



 論曰:「烏古敵烈,大部也,奪剌為統軍,克敵有功;普達居詳穩,悅以使人。西北,重鎮也,侯哂巡邊以廉稱;古昱鎮撫而民富;獨攧駐金肅而夏人不敢東獵。噫!部人內附,方面以寧,雖朝廷處置得宜,而諸將之力抑亦何可少哉。」



\end{pinyinscope}