\article{卷二十九本紀第二十九 天祚皇帝三}

\begin{pinyinscope}

 保大元年春正月丁酉朔,改元,肆赦。初,金人興兵,郡縣所失幾半。上有四子;長趙王,母趙昭容;次晉王,母文妃;次秦王、許王,皆元妃生。國人知晉王之賢,深所屬望。元妃之兄樞密使蕭奉先恐秦王不得立,潛圖之。文妃姊妹三人:長適耶律撻曷里,次文妃,次適余睹。一日,其姊若妹俱會軍前,奉先諷人誣駙馬蕭昱及余睹等謀立
 晉王,事覺,昱、撻曷裡等伏誅,文妃亦賜死;獨晉王未忍加罪。余睹在軍中,聞之大懼,即率千餘騎叛入金,上遺知奚王府事蕭遐買,北府宰相蕭德恭、太常兗耶律諦裡姑、歸州觀察使蕭和尚奴、四軍太師蕭干將所部兵追之,及諸閭山縣。諸將議曰:主上信蕭奉先言,奉先視吾輩蔑如也。余睹乃宗室豪俊,常不肯為奉先下。若擒其睹,他日吾覺皆余睹也不若縱之。」還,即紿曰:「追襲不及。」



 奉先既見余睹之亡,恐後日諸較亦叛,遂勸驟加爵賞,以結眾心。以蕭遐買為奚王,蕭德恭試中書門下平章事兼判上京留守事,耶律諦裡姑為龍虎衛上將軍,
 蕭和尚奴金吾衛上將軍,蕭乾鎮國大將軍二月,幸鴛鴦濼。夏五月,至曷里狘。



 秋七月,獵炭山。



 九月,至南京。



 冬十一月癸亥,以西京留守趙王習泥烈為惕隱。



 二年春正月乙亥,金克中京,進下澤州。上出居庸關,至鴛鴦濼。聞余睹引金人婁室勃堇奄至,蕭奉先曰:「余睹乃王子班之苗裔,此來欲立甥晉王耳,若為社稷計,不惜一子,明其罪誅之,可不戰而余睹自回矣。」上遂賜晉王死,素服三日,耶律撒八等皆伏誅。王素有人望,諸軍聞其死,無不流涕,由是人心解體。余睹引金人逼行宮,上率衛兵五千餘騎幸雲中,遺傳國璽於桑乾河。



 二月
 庚寅朔,日有食之,既。甲午,知北院大王事耶律馬哥、漢人行宮都部署蕭特末並為都統,太和宮使耶律補得副之,將兵屯鴛鴦濼。己亥,金師敗奚王霞末於北安州,遂降其城。



 三月辛酉,上聞金師將出嶺西,遂趨白水濼。乙丑,群牧使謨魯斡降金。丙寅,上至女古底倉。聞金兵將近,計不知所出。乘輕騎入夾山,方悟奉先之不忠。怒曰:「汝父子誤我至上,今欲於汝,何益於事!恐軍心忿怨,爾曹避敵茍安,禍必及我,其勿從行。」奉先下馬,哭拜而去。行未數里,左右執其父子,縛送金兵。金人斬其長子昂,以奉先及其次子昱械送金主。道遇遼軍,執以歸國,
 遂並賜死。逐樞密使蕭得裏底。



 召撻不也典禁衛。丁卯,以北院樞密副使蕭僧孝奴知北院樞密使事,同知北院樞密使事蕭查刺為左夷離畢。戊辰,同知殿前點檢事耶律高八率衛士降金。己巳,偵人蕭和尚、牌印郎君耶律哂斯為金師所獲。癸酉,以諸局百工多亡,凡扈從不限吏民,皆官之。初,詔留宰相張琳、李處溫與秦晉國王淳守燕。處溫聞上入夾山,數日命令不通,即與弟處能、子奭,外假怨軍,內結都蕭干,謀立淳。遂與諸大臣耶律大石、左企弓、虞仲文、曹勇義、康公弼集蕃漢百官、諸軍及父老數萬人詣淳府。處溫邀張琳至,白其事。琳
 曰:「攝政則可。」處溫曰:「天意人心已定,請立班耳。」處溫等請淳受禮,淳方出,李奭持赭袍被之,令百官拜舞山呼。淳驚駭,再三辭,不獲已而從之。以處溫守太尉,左企弓守司徒,曹勇義知樞密院事,虞仲文參知政事,張琳守太師,李處能直樞密院,李奭為少府少監、提舉翰林醫官,李爽、陳秘餘人曾與大計,並賜進士及第,授官有差。蕭乾為北樞密使,駙馬都尉蕭旦知樞密院事。改怨軍為常勝軍。於是肆赦,自稱天錫皇帝,改元建福,降封天祚為湘陰王。遂據有燕、雲、平及上京、遼西六路。天祚所有,沙漠已北,西南、西北路兩都招討府、諸蕃部族而
 已。



 夏四月辛卯,西南面招討使耶律佛頂降金,雲內、寧邊、東勝等州皆降。阿疏為金兵所擒。金已取西京,沙漠以南部族皆降。上遂循於訛莎烈。時北部謨葛失贐馬、駝、食羊。



 五月甲戌,都統馬哥收集散亡,會於漚里謹。丙子,以馬哥知北院樞密使事,兼都統。



 六月,淳寢疾,聞上傳檄天德、雲內、朔、武、應、蔚等州,合諸蕃精兵五萬騎,約以八月入燕;並遣人問勞,索衣裘、茗藥。淳甚驚,命南、北面大臣議。而李處溫、蕭乾等有迎秦拒湘之說,集蕃漢百官議之。從其議者,東立;惟南面行營都部署耶律寧西立。處溫等問故,寧曰:「天祚果能以諸蕃兵大舉奪燕,
 則是天數未盡,豈能拒之?否則,秦、湘,父子也,拒則皆拒。自古安有迎子而拒其父者?」處溫等相顧微笑,以寧扇亂軍心,欲殺之。淳欹枕長嘆曰:「彼忠臣也,焉可殺?



 天祚果來,吾有死耳,復何面目相見耶!」已而淳死,眾乃議立其妻蕭氏為皇太后,主軍國事。奉遺命,迎立天祚次子秦王定為帝。太后遂稱制,改元德興。處溫父子懼禍,南通童貫,欲挾蕭太后納土於宋,北通於金,欲為內應,外以援立太功自陳。蕭太后罵曰:「誤秦晉國王者,皆汝父子!」悉數其過數十,賜死,臠其子奭而磔之;籍其家,得錢七萬緡,金玉實器稱是,為宰相數月之間所取也。謨葛
 失以兵來援,為金人敗於洪灰水,擒其子陀古及其屬阿敵音,夏國援兵至,亦為金所敗。



 秋七月丁巳朔,敵烈部皮室叛,烏古部節度使耶律棠古討平之,加太子太保。乙丑,上京毛八十率二千戶降金。辛未,夏國遣曹價來問起居。



 八月戊戌,親遇金軍,戰於石輦驛,敗績,都統蕭特末及其侄撒古被執。辛丑,會軍於歡撻新查刺,金兵追之急,棄輜重以遁。



 九月,敵烈部叛,都統馬哥克之。



 冬十月,金兵攻蔚州,降。



 十一月乙丑,聞金兵至奉聖州,遂率衛兵屯於落昆髓。秦晉王淳妻蕭德妃五表於金,求立秦王,不許,以勁兵守居庸。



 及金兵臨並,厓石自崩,
 戍卒多壓死,不戰而潰。德妃出古北口,趨天德軍。



 十二月,知金主撫定南京,上遂由掃里關出居四部族詳穩之家。三年春正月丁巳,奚王回離保僭號,稱天復元年,命都統馬哥討之。甲子,初,張玨為遼興軍節度副使,民推玨領州事。



 秦晉王淳既死,蕭德妃遣時立愛知平州。玨知遼必亡,練兵蓄馬,籍丁壯為備。立愛至,玨弗納。金帥粘罕入燕,首問平州事於故參知政事康公弼。公弼曰:「玨狂妄寡謀,雖有鄉兵,彼何能為?示之不疑,圖之未晚。」金人招時立愛赴軍前,加玨臨誨軍節度使,仍知平州。既
 而又欲以精兵三千先下平州,擒張玨。公弼曰:「若加兵,是趣之叛也。」公弼請自往覘之。



 謂公弼曰:「遼之八路,七路已降;獨平州未解甲者,防蕭幹耳。」厚賂公弼而還。公弼復粘罕曰:「彼無足慮。」金人遂改平州為南京,加玨試中書問下平章事,判留守事,庚辰,宜、錦、乾、顯、成、川、豪、懿等州相繼皆降,上京盧彥倫叛,殺契丹人。



 二月乙酉朔,興中府降金。來州歸德軍節度使田顥、權隰州刺史杜師回、權遷州刺史高永昌、權潤州刺史張成,皆籍所管戶降金。丙戌,誅蕭德妃,降淳為庶人,盡釋其黨。癸巳,興中、宜州復城守。



 三月,駐蹕於雲內州南。



 夏四月甲申
 朔,以知北院樞密使事蕭僧孝奴為諸道大都督。



 丙申,金兵至居庸關,擒耶律大石。戊戌,金兵圍輜重於青塚,硬寨太保特母哥竊梁王雅里以遁,秦王、許王、諸妃、公主、從臣皆陷沒。庚子,梁宋大長公主特里亡歸。壬寅,金遣人來招。癸卯,答言請和。丙午,金兵送族屬輜重東行,乃遣兵邀戰於白水濼,趙王習泥烈、蕭道寧皆被執。上遣牌印郎君謀盧瓦送兔紐金印偽降。遂西遁云內。駙馬都尉乳奴詣金降。



 己酉,金復以書來招,答其書。壬子,金帥書來,不許請和。



 是月,特母哥摯雅里至,上怒不能盡救諸子,詰之。



 五月乙卯,夏國王李乾順遣使請臨其國。
 庚申,軍將耶律敵烈等夜劫梁王雅里奔西北部,立以為帝,改元神歷。辛酉,渡河,止於金肅軍北。回離保為眾所殺。



 六月,遣使冊李乾順為夏國皇帝。



 秋九月,耶律大石自金來歸。



 冬十月,復渡河東還,居突呂不部。梁王雅里及,耶律術烈繼之。十一月,術烈為眾所殺。



 四年春正月,上趨都統馬哥軍。金人來攻,棄營北遁,馬哥被執。謨葛失來迎,贐馬、駝、羊,又率部人防衛。時侍從乏糧數日,以衣易羊。至烏古敵烈部,以都黠檢蕭乙薛知北院樞密使事,封謨葛失為神於越王。特母哥降金。



 二月,耶律遙設等十人謀叛,伏誅。



 夏五月,金人既克燕,
 驅燕之大家東徙,以燕空城及涿、易、檀、順、景、薊州與宋以塞盟。左企弓、康公弼、曹勇義、虞仲文皆東遷。燕民流離道路,不勝其苦,入平州,言於留守張玨曰:「宰相左企弓不謀守燕,使吾民流離,無所安集。公今臨巨鎮,握強兵,盡忠於遼,必能使我復歸鄉土,人心亦惟公是望。」玨遂召諸將領議。皆曰:「聞天祚兵勢復振,出沒漠南。公若仗義勤王,奉迎天祚,以圖中興,先責左企弓等叛降之罪而誅之,盡歸燕民,使復其業,而以平州歸宋,則宋無不接納,平州遂為藩鎮矣。即後日金人加兵,內用平山之軍,外得宋為之援,又何懼焉!」玨曰:「此大事也,不可草
 草。



 翰林學士李石智而多謀,可召與議。」石至,其言與之合。乃遣張謙率五百餘騎,傳留守令,召宰相左企弓、曹勇義、樞密使虞仲文、參知政事康公弼至灤河西巖,遣議事官趙秘校往數十罪,曰:「天祚播遷夾山,不即奉迎,一也;勸皇叔秦晉王僭號,二也;詆訐君父,降封湖陰,三也,天祚遣知閣王有慶來議事而殺之,四也;檄書始至,有迎秦拒湘之議,五也;不謀守燕而降,六也,不顧大義,臣事於金,七也;根括燕財,取悅於金,八也;使燕人遷徒失業,九也;教金人發兵先下平州,十也。爾有十罪,所不容誅。」左企弓等無以對,皆縊殺之。仍稱保大三年,畫天
 祚象,朝夕謁,事必告而後行,稱遼官秩。六月,榜諭燕人復業,恆產為常勝軍所占者,悉還之。燕民既得歸,大悅。翰林學士李石更名安弼,僭故三司使高黨往燕山,說宋王安中曰:「平州帶甲萬餘,玨有文武材,可用為屏翰;不然,將為肘腋之患。」安中深然之,令安弼與黨詣宋。



 宋主詔帥臣王安中、詹度厚加安撫,與免三年常賦。玨聞之,自謂得計。



 秋七月,金人屯來州,闍母聞平州附宋,以二千騎問罪,先入營州。玨以精兵萬騎擊敗之。宋建平州為泰寧軍,以玨為節度使,以安弼、黨為徽猷閣待制,令宣撫司出銀絹數萬稿賞。



 玨喜,還迎。金人諜知,舉兵
 來襲,玨不得歸,奔燕。金人克三州,始來索玨,王安中諱之。索急,斬一人貌類者去。金人曰,非玨也,以兵來取。安中不得已,殺玨,函其首送金。天祚既得林牙耶律大石兵歸,又得陰山室韋謨葛失兵,自謂得天助,再謀出兵,復收燕、雲。大石林牙力諫曰:「自金人初陷長春、遼陽,則軍駕不幸廣平澱,而都中京;及陷上京,則都燕山;及陷中京,則幸雲中;自雲中而播遷夾山,向以全師不謀戰備,使舉國漢地皆為金有。國勢至此,而方求戰,非計也。



 當養兵待時而動,不可輕舉。」不從。大石遂殺乙薛及坡裡括,置北、南面官屬,自立為王,率所部西去。上遂率諸
 軍出夾山,下漁陽嶺,取天健、東勝、寧邊、雲內等州。南下武州,遇金人,戰于奄遏下水,復潰,直趨山陰。



 八月,國舅詳穩蕭撻不也、筆硯祗候察刺降金。是月,金主阿骨打死。



 九月,建州降金。



 冬十月,納突呂不部人訛哥之妻諳葛,以訛哥為本部節度使。昭古牙率眾降金。金攻興中府,降之。



 十一月,從行者舉兵亂,北護衛太保術者、舍利詳穩牙不裏等擊敗之。



 十二月,置二總管府。



\end{pinyinscope}