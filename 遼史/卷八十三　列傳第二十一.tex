\article{卷八十三 列傳第二十一}

\begin{pinyinscope}

 耶律韓八耶律唐古蕭術哲侄藥師奴耶律塊耶律僕里篤耶律韓八,字嘲隱,倜儻有大志,北院詳穩古之五世孫。



 太平中,游京師,寓行宮側,椎囊衣匹馬而已。帝微服出獵,見而問之曰:「汝為何人?」韓八初不識,漫應曰:「我北院部人韓八,來覓官耳。」帝與語,知有長才,陰識之。會北院奏南京疑獄久不決,帝召韓八馳驛審錄,舉朝皆驚。韓
 八量情處理,人無冤者。上嘉之。籍群牧馬,闕其二,同事者考尋不已;韓八略不加詰,即先馳奏,帝益信任。



 景福元年,為左夷離畢,徙北面林牙,眷遇優異。重熙六年,改北院大王,政務寬仁,復為左夷離畢。十二年,再為北院大王。入朝,帝從容謂曰:「卿守邊任重,當實府庫、振貧乏以報朕。」既受詔,愈竭忠謹,知無不言,便益為多。卒,年五十五。上聞,悼惜。死之日,篋無舊蓄,椸無新衣,遣使吊祭,給葬具。



 韓八平居不細務,喜慍不形。嘗失所乘馬,家僮以同色者代之,數月不覺。



 耶律唐古,字敵隱,於越屋質之庶子。廉謹,善屬文。



 統和
 二十四年,述屋質安民治盜之法以進,補小將軍,遷西南面巡檢,歷豪州刺史、唐古部詳穩。嚴立科條,禁奸民鬻馬於宋、夏界。因陳弭私販,安邊境之要。太后嘉之,詔邊郡遵行,著為令。



 朝議欲廣西南封域,黑山之西,綿互數千里,唐古言:「戍壘太遠,卒有警急,赴援不及,非良策也。」從之。西蕃來侵,詔議守禦計,命唐古勸督耕稼以給西軍,田於臚朐河側,是歲大熟。明年,移屯鎮州,凡十四稔,積粟數十萬斜,斗米數錢。重熙問,改隗衍黨項部節度使。先是,築可敦城以鎮西域。



 諸部縱民畜牧,反招寇掠。重熙四年,上疏曰:「自建可敦城已來,西蕃數為邊患,
 每煩遠戍。歲月既入,國力耗竭。不若復守故疆,省罷戍役。」不報。是年,致仕。乞勒其父屋質功於石,帝命耶律庶成制文,勒石上京崇孝寺。卒,年七十八。



 蕭術哲,字石魯隱,孝穆弟高九之子。以戚屬加監門衛上將軍。重熙十三年,將衛兵討李元昊有功,遷興聖宮使。蒲奴里部長陶得里叛,術哲為統軍都監,從都統耶律義先擊之,擒陶得里。術哲與義先不協,誣義先罪,免官。稍遷西南面招討都監,坐事下獄,以太后言,杖而釋之。



 清寧初,為國舅詳穩、西北路招討使,私取官粟三百斛,及代,留畜產,令主者鬻之以償。後族弟胡睹到部發
 其事,帝怒,決以大杖,免官。尋起為昭德軍節度使,徵為北院宣徽使。九年,上以術哲先為招討,威行諸部,復為西北路招討使。訓士卒,增器械,省追呼,嚴號令。人不敢犯,邊境晏然。十年,入朝,封柳城郡王。



 咸雍二年,拜北府宰相,為北院樞密使耶律乙辛所忌,誣術哲與護衛蕭忽古等謀害乙辛。詔獄無狀,罷相,出鎮順義軍。



 卒,追王晉、宋、梁三國。侄藥師奴。



 藥師奴,幼穎悟,謹禮法,補祗候郎君。



 大康中,為興聖宮使,累遷同知殿前點檢司事。上嘉其宿衛嚴肅,遷右夷離畢。夏王季乾順為宋所攻,求解,帝命藥師奴持節使
 宋,請罷兵通好,宋從之。拜南面林牙,改漢人行宮副部署。



 乾統初,出為安東軍節度使,卒。



 耶律玦,字吾展,遙輦鮮質可汗之後。



 重熙初,召修國史,補符實郎,累遷知北院副部署事。入見太后,後顧左右曰:「先皇謂玦必為偉人,果然。」除樞密副使,出為西南面招討都監,歷同簽南京留守事、南面林牙。



 皇弟秦國王為遼興軍節度使,以玦同知使事,多所匡正。十年,復為樞密副使。咸雍初,兼北院副部署。及秦國王為西京留守,請玦為佐,從之。歲中獄空者三,召為孟父房敞穩。



 玦不喜貨殖,帝知其貧,賜宮戶十。嘗謂宰相曰:「契丹忠王
 無如玦者,漢人則劉伸而已。然熟察之,玦優於伸。」先是,西北諸部久不能平,上遣玦問狀,執弛慢者痛繩之。以酒疾卒。耶律僕里篤,字燕隱,六院林牙突呂不也四世孫。



 開泰間,為本班郎君。有捕盜功,樞密使蕭樸薦之,遷率府率。太平中,同知南院宣徽事,累遷彰聖軍節度使。



 重熙十六年,知興中府,以獄空聞。十八年,伐夏,攝西南面招討使。十九年,夏入侵金肅軍,敗之,斬首萬餘級,加右武衛上將軍。時近邊群牧數被寇掠,遷倒塌嶺都監以治之,桴鼓不鳴。二十年,知金肅軍事。宰相趙惟節總領邊城橋道
 芻粟,請貳,帝命僕里篤副之,以稱職聞。



 清寧初,歷長寧、匡義二軍節度使,致仕。咸雍間卒。子阿固質,終倒嶺都監。



 論曰:「韓八因帝微行,才始見售。及任以事,落落知大體,不負上之知矣。唐古、術哲經略西北邊,勸農積慄,訓練士卒,敵人不敢犯。玦以忠直見稱於上,僕里篤以幹敏為宰相佐,在鎮俱以獄空聞。之數人者,豈特甲胄之士,抑亦李牧、程不識之亞歟。」



\end{pinyinscope}