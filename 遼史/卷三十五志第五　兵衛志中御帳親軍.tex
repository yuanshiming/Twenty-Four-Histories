\article{卷三十五志第五 兵衛志中御帳親軍}

\begin{pinyinscope}

 漢武帝多行幸之事,置期門、佽飛、羽林之目,天子始有親軍。唐太宗加親、動、翊、千牛之衛,布腹心之地,防衛密矣。遼太祖宗室盛強,分迭刺部為二,宮衛內虛,經營四方,未遑鳩集。皇后述律氏居守之際,摘蕃漢精銳為屬珊軍;太宗益選天下精甲,置諸爪牙為皮室軍。合騎五
 十萬,國威壯矣。



 大帳皮室軍。



 太宗置,凡三十萬騎。



 屬珊軍。



 地皇后置,二十萬騎。



 宮衛騎軍太祖以迭刺部受禪,分本部為五院、六院,統以皇族,而親衛缺然。乃立斡魯朵法,裂州縣,割戶丁,以強幹弱支。詒謀嗣績,世建宮衛。入則居守,出則扈從,葬則因以守陵。有兵事,則五京、二州各提轄司傳檄而集,不待調發
 州縣、部族,十萬騎軍己立具矣。恩意親洽,兵甲犀利,教練完習。簡天下精銳,聚之腹心之中。懷舊者歲深,增新者世盛。此軍制之良者也。弘義宮:正丁一萬六千,蕃漢轉丁一萬四千,騎軍六千。



 長寧宮:正丁一萬四千,蕃漢轉丁一萬二千,
 騎軍五千。



 永興宮,正丁六千,蕃漢轉丁一萬四千,騎軍五千。



 積慶宮:正丁一萬,蕃漢轉丁一萬六千,騎軍八千。



 延昌宮:
 正丁二千,蕃漢轉丁六千,騎軍二千。



 彰愍宮:正丁一萬六千,蕃漢轉丁二萬,騎軍一萬。



 崇德宮:正丁一萬二千,蕃漢轉丁二萬,
 騎軍一萬。



 興聖宮:正丁二萬,蕃漢轉丁四萬,騎軍五千。



 延慶宮:正丁一萬四千,蕃漢轉丁二萬,騎軍一萬。



 太和宮:
 正丁二萬,蕃漢轉丁四萬,騎軍一萬五千。



 永昌宮:正丁一萬四千,蕃漢轉丁二萬,騎軍一萬。



 敦睦宮:正丁六千,蕃漢轉丁一萬,
 騎軍五千。



 文忠王府:正丁一萬,蕃漢轉丁一萬六千,騎兵一萬。



 十二宮一府,自上京至南京總要之地,各置提轄司。重地每宮皆置,內地一二而已。太和、水昌二宮宜與興聖、延慶同,舊史不見提轄司,蓋闕文也。



 南京:弘義宮提轄司。



 長寧宮提轄司。



 永興宮提轄司。



 積慶宮提轄司。



 延昌宮提轄司。



 彰愍宮提轄司。



 崇德宮提轄司。



 興聖宮提轄司。



 延慶宮提轄司。



 敦睦宮提轄司。



 文忠王府提轄司。



 西京:弘義宮提轄司。



 長寧宮提轄司。



 永興宮提轄司。



 積慶宮提轄司。



 彰愍宮提轄司。



 崇德宮提轄司。



 延慶宮提轄司。



 文忠王府提轄司。



 奉聖州:
 弘義宮提轄司。



 長寧宮提轄司。



 永興宮提轄司。



 積慶宮提轄司。彰愍宮提轄司。



 祟德宮提轄司。



 興聖宮提轄司。



 延慶宮提轄司。



 文忠王府提轄司。



 平
 川:弘義宮提轄司。



 長寧宮提轄司。



 永興宮提轄司。



 積慶宮提轄司。



 延昌宮提轄司。



 彰愍宮提轄司。



 興聖宮提轄司。



 延慶宮提轄司。



 文忠王府提轄司。



 中京:
 延昌宮提轄司。



 文忠王府提轄司。



 上京:文忠王府提轄司。凡諸宮衛,丁四十萬八千,出騎軍十萬一千。大首領部族軍遼親王大臣,體國如家,征伐之際,往往置私甲以從正事。



 大者千餘騎,小者數百人,著籍皇府。國有戎政,量借三五千騎,常留餘兵為部族根本。



 太子軍。



 偉至軍。永康王軍。



 於越王軍。



 麻答軍。



 五押軍。



 眾部族軍眾部族分隸南北府,守衛四邊,各有司存,具如左。



 北府凡二十八部。



 侍從宮帳:奚王府部。



 鎮南境:五院部。



 六院部。



 東北路招討司:烏隗部。



 東北路統軍司:遙里部。



 伯德部。



 奧里部。



 南克部。



 北克部。



 圖盧部。



 術者達魯虢部。



 河西部。



 西北路招討司:突呂丁部。



 奧衍女直部。



 室韋部。西南路招討司:涅刺部。



 烏古涅刺部。



 涅刺越兀部。



 梅古悉部。



 頡的部。



 匿訖唐古部。



 鶴刺唐古部。



 黃龍府都部署司:隗衍突闕部。



 奧衍突闕部。



 北唐古部。



 五國部。



 烏古敵烈統軍司:迭魯敵烈部。



 戌隗烏古部:北敵烈部。



 南府凡一十六部。



 鎮駐西南境:乙室部。



 西南路招討司:品部。



 迭達迭刺部。



 品達魯虢部。



 乙典女直部。



 西北路招討司:楮特部。東北路統軍司:達馬鼻古德部。



 東北路女直兵馬司:
 乙室奧隗部。



 東京都部署司:楷特奧隗部。



 窈爪部。



 稍瓦部。



 曷術部。



 戍倒塌嶺:訛僕括部。



 屯駐本境:撒里葛部。



 南唐古部。



 薛特部。



\end{pinyinscope}