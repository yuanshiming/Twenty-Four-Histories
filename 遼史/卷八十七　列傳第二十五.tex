\article{卷八十七 列傳第二十五}

\begin{pinyinscope}

 耶律弘古耶律馬六蕭滴冽耶律適祿耶律陳家奴耶律特麼耶律仙童蕭素颯耶律大悲奴耶律弘古,字胡篤堇,樞密使化哥之弟。



 統和間,累遷順義軍節度使,入為北面林牙。太平元年,加同政事門下平章事,出為彰國軍節度使,兼山北道兵馬都部署,徙武定軍節度使。六年,拜惕隱。討阻卜有功。聖宗嘗刺臂
 血與弘古盟為友,禮遇尤異,拜南府宰相,改上京留守。



 重熙六年,遷南院大王,禦制誥辭以寵之。十二年,加於越。帝閔其勞,復授武定軍節度使,卒。訃聞,上哭曰:「惜哉善人!」喪至,親臨奠焉。



 耶律馬六,字揚隱,孟父楚國王之後。性寬和,善諧謔,親朋會遇,一坐盡傾。恬於榮利。



 與耶律弘古為刺血友,弘古為惕隱,薦補宿直官。重熙初,遷旗鼓拽剌詳穩。為人畏慎容物,或有面相陵折者,恬然若弗聞,不臧否世務。以故上益親狎。三年,遷崇德宮使,為惕隱,禦制誥辭以褒之。拜北院宣徽使,寵遇過宰輔,帝常以兄呼之。



 改遼
 興軍節度使,卒,年七十。子奴古達,終南京宣徽使。



 蕭滴冽,字圖寧,遙輦鮮質可汗宮人。



 重熙初,遙攝鎮國軍節度使。六年,奉詔使宋,傷足而跛,不告遂行,帝怒。及還,決以大杖,降同簽南京留守事。遇授靜江軍節度使,歷群牧都林牙,累遷右夷離畢。以才幹見任使。



 會車駕西征,元昊乞降,帝以前後反覆,遣滴冽往覘誠否。



 因為元昊陳述禍福,聽命乃還。拜北院樞密副使,出為中京留守。十九年,改西京留守,卒。



 耶律適祿,字撒懶。清寧初,為本班郎君,稍遷宿直官。



 乾統中,從伐阻卜有功,加奉宸。歷護衛太保,改弘義宮副
 使。時上京梟賊趙鐘哥跋扈自肆,適祿擒之,加泰州觀察使,為達魯虢部節度使。



 天慶中,知興中府,加金吾衛上將軍。為盜所殺。



 耶律陳家奴,字綿辛,懿祖弟葛剌之八世孫。



 重熙中,補牌印郎君。坐直日不至,降本班。會帝獵,陳家奴逐鹿圍內,鞭之二百。時耶律仁先薦陳家奴健捷比海東青鶻,授御盞郎君。歷鷹坊、尚廄、四方館副使,改徒魯古皮室詳隱。會太后生辰,進詩獻馴鹿,太后嘉獎,賜珠二琲,雜彩二百段。兄撒缽卒,陳家奴聞訃,不告而去。帝怒,鞭之。



 清寧初,累遷右夷離畢。適帝與燕國王射鹿俱中,王時
 年九歲,帝悅,陳家奴應制進詩。帝喜,解衣以賜。後皇太子廢,帝疑陳家奴黨附,罷之。



 時西北諸部寇邊,以陳家奴為烏古部節度使行軍都監,賜甲一屬、馬二匹,討諸部,擒其酋送於朝。偵候者見馬蹤,意寇至,陳家奴遣報元帥,耶律愛奴視之曰:「此野馬也!」將出獵,賊至,愛奴戰歿。有司詰按,陳家奴不伏,詔釋之。由是感激,每事竭力。後諸部復來侵,陳家奴率兵三往,皆克,邊境遂寧。



 以老告歸,不從。通宗崩,為山陵使,致仕。年八十卒。



 耶律特麼,季父房之後。重熙間,為北克,累遷六部禿裏太尉。大安四年,為倒塌嶺嶺節度使。頃之,為禁軍都監。是
 冬,討磨古斯,斬首二千餘級。十年,復討之。既捷,授南院宣徽使。壽隆元年,為北院大王。四年,知黃龍府事,薨。



 耶律仙童,仲父房之後。重熙初,為宿直官,累遷惕隱、都監。以寬厚稱。



 浦奴里叛,仙童為五國節度使,率師討之,擒其帥陶得里。



 又擊烏隗叛,降其眾,改彰國軍節度使,拜北院大王。清寧二年,知黃龍府事,遷侍衛親軍馬步軍都指揮,歷忠順、武定二軍節度使。致仕,封蔣國公。咸雍初,徙封許國,卒。



 蕭素颯,字特免,五院部人。重熙間始仕,累遷北院承旨,彰愍宮使。



 清寧初,歷左皮室詳穩、右夷離畢。咸雍五年,
 剖阿里部叛,素颯討降之,率其酋長來朝。帝嘉其功,徙北院林牙,改南院副部署,卒。



 子謀魯斡,字回璉,初補夷離畢郎君,遷文班太保。大康中,改南京統軍使,為右夷離畢。與樞密使耶律阿思論事不合,見忌,出為馬群太保。北部來侵,謀魯斡破之,以功遷同知烏古敵烈統軍,仍許便宜行事。



 後以讒毀,降領西北路戍軍,復為馬群太保,卒。



 耶律大悲奴,字休堅,王子班聶里古之後。大康中,歷永興延昌宮使、右皮室詳穩。會阻卜叛,奉詔招降之。壽隆二年,拜殿前都點檢。乾統初,歷上京留守、惕隱,復為都
 點檢,改西南面招討使。請老,不許。天慶中,留守上京,領北南樞密院點檢中丞諸司等事。以彰國軍節度使致仕,卒。



 大悲奴舉止馴雅,好禮儀,為時人所稱。



 論曰:「遼自神冊而降,席富強之勢,內修法度,外事征伐,一時將帥震揚威靈,風行電掃,討西夏,徵黨項,破阻卜,平敵烈。諸部震懾,聞鼙鼓而膽落股弁,斯可謂雄武之國矣。



 其戰勝攻取,必有奇謀秘計神變莫測者,將前史所載,未足以發之邪?抑天之所授,眾莫與爭而能然邪?



 雖然,兵者兇器,可戢而不可玩;爭者末節,可遏而不可召。此黃石公所謂柔能制剛,弱能制強也。又況乎仁者
 之無敵哉。遼之君臣智足守此,金人果能乘其敝而躡其後乎?是以於耶律弘古輩諸將,不能無慨然也。」



\end{pinyinscope}