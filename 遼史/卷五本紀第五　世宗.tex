\article{卷五本紀第五 世宗}

\begin{pinyinscope}

 世宗孝和莊憲皇帝,諱阮,小字兀欲。讓國皇帝長子,母柔貞皇后蕭氏。帝儀觀豐偉,內寬外嚴,善騎射,樂施予,人望歸之。太宗愛之如子。會同九年,從伐晉。



 大同元年春二月,封永康王。



 夏四月丁丑,太守崩於欒城。戊寅,梓宮次鎮陽,即皇帝位於樞前。甲申,次定州,命天德、朔古、解裏等護梓宮先赴上京。太后聞帝即位,遣太弟李胡率兵
 拒之。



 六月甲寅朔,次南京,五院夷離堇安端、詳穩劉哥遣人馳報,請為前鋒;至泰德泉,遇李胡軍,虞敗之。上遣郎君勤德等兩軍諭解。



 秋閏七月,次潢河,太后、李胡整兵拒於橫渡,相持數日。



 用屋質之謀,各罷兵趨上京。既而聞太后、李胡復有異謀,遷於祖州;誅司徒劃設及楚補里。



 八月壬竿朔,尊母蕭氏為皇太后,以太后族刺只撒古魯為國舅帳,立詳穩以總焉。以崇德宮戶分賜翼戴功臣,及北大王窪、南院大王吼各五士,安摶、楚補各百。的、鐵刺子孫先以非罪籍沒者歸之。癸未,始置北院樞密使,以安摶為之。九月壬子朔,葬嗣聖皇帝
 於懷陵。丁卯,行柴冊禮群臣上尊號曰天授皇帝。大赦,改大同元年為天祿元年。追謚皇考曰讓國皇帝。以安端主東丹國,封明王,察割為泰寧王,劉哥為惕隱,高勛為南院樞密使。



 二年春正月,天德、蕭翰、劉哥、盆都等謀反。誅天德,杖蕭翰,遷劉哥於邊,罰盆都使轄戞斯國。漢主劉知遠殂,子承祐立。



 夏四月庚辰朔,南唐遣李朗、王祚來慰且賀,兼奉蠟丸書,議攻漢。



 秋七月壬申,皇子賢生。



 冬十月壬午,南京留守魏王趙延壽薨,以中臺省右相牒(蟲葛)為南京留守,封燕王。



 十一月,駐蹕彰武南。



 三年春正月,蕭翰及公主阿不里謀反,翰伏誅,阿不里瘐死獄中。庚申,肆赦。內外官各進一階。



 夏六月戊寅,以敞史耶律胡離軫為北院大王。己卯,惕隱頹昱封漆水郡王。



 秋九月辛丑朔,召群臣議南伐。



 冬十月,遣諸將率兵攻下貝州高老鎮,徇地鄴都、南宮、堂陽,殺深州刺史史萬山,俘獲甚眾。



 四年春二月辛未,泰寧王察割來朝,留侍。是月,建政事省。



 三月戊戌朔,南唐遣趙延嗣、張福等來賀南征捷。



 秋九月乙丑朔,如山西。



 冬十月,自將南伐,攻下安平、內丘、束鹿等城,大獲而還。



 是歲,冊皇后蕭氏。
 五年春正月癸亥朔,如百泉湖。漢郭威弒其主自立,國號周,遣朱憲來告。即遣使致良馬。漢劉崇自立於太原。



 二月,周遣姚漢英、華昭胤來,以書矢抗禮,留漢英等。



 夏五月壬戌朔,太子太傅趙瑩薨,輟朝一日,命歸葬於汴。



 詔州縣錄事參軍、主簿,委政事省銓注。



 六月辛卯朔,劉崇為周所攻,遣使稱侄,乞授,且求封冊。



 即遣燕王牒(蟲葛)、樞密使高勛冊為大漢神武皇帝。南唐遣蔣洪來,乞舉兵應援。是夏,清暑百泉嶺。



 九月康申朔,自將南伐。壬戌,次歸化州祥古山。癸亥,祭讓國皇帝於行宮。群臣皆醉,察割反,帝遇弒,年三十四。



 應厲元年,葬於顯州西山,陵
 曰顯陵。二年,謚孝和皇帝,廟號世宗。統和二十六年七月,加謚孝和莊憲皇帝。



 贊曰:世宗,中才之主也。入繼大統,曾未三年,納唐丸書,即議南伐,既乏持重,宜乘周防,蓋有致禍之道矣。然而孝友寬慈,亦有君人之度焉。未及師還,變起沉湎,豈不可哀也哉!



\end{pinyinscope}