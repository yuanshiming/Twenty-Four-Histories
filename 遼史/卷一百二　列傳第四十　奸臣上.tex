\article{卷一百二 列傳第四十 奸臣上}

\begin{pinyinscope}

 耶律乙辛張孝傑耶律燕哥蕭十三《春秋》褒貶,善惡並書,示勸懲也。故遷、固傳佞幸、酷吏,歐陽修則並奸臣錄之,將俾為君者知所鑒,為臣者知所戒。此天地聖賢之心,國家安危之機,治亂之原也。遼自耶律乙辛而下,奸臣十人,其敗國皆足以為戒,故列於《傳》。



 耶律乙辛,字胡睹兗,五院部人。父迭刺,家貧,服用不給,
 部人號「窮迭刺」。



 初,乙辛母方娠,夜夢手搏羖羊,拔其角尾。既寤占之,術者曰:「此吉兆也。羊去角尾為王字,汝後有子當王。」及乙辛生,適在路,無水以浴,回車破轍,忽見湧泉。迭刺自以得子,欲酒以慶,聞酒香,於草棘間得二榼,因祭東焉。



 乙辛幼慧黠。嘗牧羊至日昃,迭刺視之,乙辛熟寢。迭刺觸之覺,乙辛怒曰:「何遽驚我!適夢人手執口月以食我,我已食月,啖日方半而覺,借不盡食之。」迭刺自是不令牧羊。及長,美風儀,外和內狡。重熙中,為文班吏,掌太保印,陪從入宮。皇后見乙辛詳雅如素宦,令補筆硯吏;帝亦愛之,累遷護衛太保。道宗即位,以乙辛
 先朝任使,賜漢人戶四十,同知點檢司事,常召決疑議,升北院同知,歷樞密副使。清寧五年,為南院樞密使,改知北院,封趙王。



 九年,耶律仁先為南院樞密使,時駙馬都尉蕭胡睹與重元黨,惡仁先在朝,奏曰:「仁先可任西北路招討使。」帝將從之。乙辛奏曰,「臣新參國政,未知治體。仁先乃先帝舊臣,不可遽離朝廷。」帝然之。重元亂平,拜北院樞密使,進王魏,賜匡時翊聖竭忠平亂功臣。咸雍五年,加守太師。詔四方有軍旅,許以便宜從事,勢震中外,門下饋賂不絕。凡阿順者蒙薦攤,忠直者被斥竄。



 大康元年,皇太子始預朝政,法度修明。乙辛不得逞,謀
 以事誣皇后。後既死,乙辛不自安,又欲害太子。乘間入奏曰:「帝與後如天地並位,中宮豈可曠?」盛稱其黨駙馬都尉蕭霞抹之妹美而賢。上信之,納於宮,尋冊為皇后。時護衛蕭忽古知乙辛奸狀,伏橋下,欲殺之。俄暴雨壞橋,謀不遂。林牙蕭巖壽密奏曰:「乙辛自皇太子預政,內懷疑懼,又與宰相張孝傑相附會。恐有異圖,不可使居要地。」出為中京留守。乙辛泣謂人曰:「乙辛無過,因讒見出。」其黨蕭霞抹輩以其言聞於上。上悔之。無何,出蕭巖壽為順義軍節度使,詔近臣議召乙辛事。北面官屬無敢言者,耶律撒刺曰,「初以蕭巖壽奏,出乙辛。若所言不
 當,宜坐以罪;若當,則不可復召。」累諫不從。乃復召為北院樞密使。



 時皇太子以母後之故,憂見顏色。乙辛黨欣躍相慶,讒謗沸騰,忠良之士斥逐殆盡。乙辛因蕭十三之言,夜召蕭得裹特謀構太子,令護衛太保耶律查刺誣告耶律撒刺等同謀立皇太子。詔按無跡而罷。又令牌印郎君蕭訛都斡詣上誣首:「耶律查刺前告耶律撒刺等事皆實,臣亦與其謀。本欲殺乙辛等而立太子。臣等若不言,恐事白連坐。」詔使鞫劾,乙辛迫令具伏。



 上怒,命誅撒刺及速撒等。乙辛恐帝疑,引數人庭詰,各令荷重校,繩擊其頸,不能出氣,人人不堪其酷,惟求速死。反
 奏曰:「別無異辭。」時方暑,尸不得瘞,以至地臭。乃囚皇太子於上京,監衛者皆其黨。尋遣蕭達魯古、撒把害太子。乙辛黨大喜,聚飲數日。上京留守蕭撻得以卒聞。上哀悼,欲召其妻,乙辛陰遣人殺之,以滅其口。



 五年正月,上將出獵,乙辛奏留皇孫,上欲從之。同知點檢蕭兀納諫曰:「陛下若從乙辛留皇孫,皇孫尚幼,左右無人,願留臣保護,以防不測。」遂與皇孫俱行。由是上始疑乙辛,頗知其奸。會北幸,將次黑山之平澱,上適見扈從官屬多隨乙辛後,惡之,出己辛知南院大王事。及例削一字王爵,改王混同,意稍自安。及赴闕入謝,帝即日遣還,改知興
 中府事。



 七年冬,坐以禁物鬻入外國,下有司議,法當死。乙辛黨耶律燕哥獨奏當入八議,得減死論,擊以鐵骨朵,幽於來州。



 後謀奔宋及私藏兵甲事覺,縊殺之。乾統二年,發塚,戮其尸。



 張孝傑,建州永霸縣人。家貧,好學。重熙二十四年,擢進士第一。



 清寧間,累遷樞密直學士。咸雍初,坐誤奏事,出為惠州刺史。俄召復舊職,兼知戶部司事。三年,參知政事,同知樞密院事,加工部侍郎。八年,封陳國公。上以孝傑勤幹,數問以事,為北府宰相。漢人貴幸無比。



 大康元年,賜國姓。明年秋獵,帝一日射鹿三十,燕從官。



 酒酣,命
 賦《雲上於天詩》,詔孝傑坐御榻旁。上誦《黍離》詩:「知我者謂我心憂,不知我者謂我何求。」孝傑奏曰:「今天下太平,陛下何憂?富有四海,陛下何求?」帝大悅。三年,群臣侍燕,上曰:「先帝用仁先、化葛,以賢智也。朕有孝傑、乙辛,不在仁先、化葛下,誠為得人。」歡飲至夜,乃罷。



 是年夏,乙辛譖皇太子,孝傑同力相濟。及乙辛受詔按皇太子黨人,誣害忠良,孝傑之謀居多。乙辛薦孝傑忠於社稷,帝謂孝傑可比狄仁傑,賜名仁傑,乃許放海東青鶴。六年,既出乙辛,上亦悟孝傑奸佞,尋出為武定軍節度使。坐私販廣濟湖鹽及擅改詔旨,削爵,貶安肅州,數年乃歸。大安
 中,死於鄉。乾統初,剖棺戮尸,以族產分賜臣下。



 孝傑久在相位,貪貨無厭,時與親戚會飲,嘗曰:「無百萬兩黃金,不足為宰相家。」初,孝傑及第,詣佛寺,忽迅風吹孝傑襆頭,與浮圖齊,墜地而碎。有老僧曰:「此人必驟貫,然亦不得其死。」竟如其言。



 耶律燕哥,字善寧,季父房之後。四世祖鐸穩,太祖異母弟。父曰豁里斯,官至太師。



 燕哥狡佞而敏。清寧間,為左護衛太保。太康初,轉北面林牙。初耶律乙辛自中京留守復為樞密使,以燕哥為耳目,凡聞見必以告。乙辛愛而薦之,帝亦以為賢,拜左夷離畢。及皇太子被誣,帝遣
 燕哥往訊之,太子謂燕哥曰:「帝惟我一子,今為儲嗣,復何求,敢為此事!公與我為昆弟行,當念無辜,達意於帝。」禱之甚懇。蕭十三聞之,謂燕哥曰:「宜以太子言,易為伏狀。」燕哥頷之,盡如所教以奏。及太子被逐,乙辛殺害忠良,多燕哥之謀,為契丹行官都部署。五年夏,拜南府宰相,遷惕隱。



 大安三年,為西京留守,致仕。壽隆初,以疾卒。



 蕭十三,蔑古乃部人。父鐸魯斡,歷官節度使。十三辨黠,善揣摩人意。清寧間,以年勞遷護衛太保。大康初,耶律乙辛復入樞府,益橫恣。時十三出入乙辛家,以朝臣不附者輒使出之,十三由宿衛遷殿前副點檢。



 三年夏,護
 衛蕭忽古等謀殺乙辛,事覺下獄。十三謂乙辛曰:「今太子猶在,臣民屬心。大王素無根柢之助,復有誣皇后之怨。若太子立,王置身何地?宜熟計之。」乙辛曰:「吾憂此久矣!」是夜,召蕭得裹特謀所以構太子事。十三計既行,尋遷殿前都點檢,兼同知樞密院事。復令蕭訛都斡等誣首耶律查刺前告耶律撒刺等事皆實,詔究其事,太子不服。別遣夷離畢耶律燕哥問太子,太子具陳所以見誣之狀。十三聞之,謂燕哥曰:「如此奏,則大事去矣!當易其辭為伏款。」燕哥入,如十三言奏之。上大怒,廢太子。太子將出,曰:「我何罪至是!」十三叱令登車,遣衛卒闔車門。
 是年,遷北院樞密副使,復陳陰害太子計,乙辛從之。



 及乙辛出知南院大王事,亦出十三為保州統軍使,卒。乾統間,剖棺戮尸。二子:的裡得、念經,皆伏誅。



\end{pinyinscope}