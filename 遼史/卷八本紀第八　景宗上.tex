\article{卷八本紀第八 景宗上}

\begin{pinyinscope}

 景宗孝成康靖皇帝,諱賢,字賢寧,小字明扆。世宗皇帝第二子,母曰懷節皇后蕭氏。察割之亂,帝甫四歲。穆宗即位,養永興宮,既長,穆宗酗酒怠政。帝一日與韓匡嗣語及時事,耶律賢適止這。帝悟,不復言。



 應因十九年春二月戊辰,入見,穆宗曰:「吾兒已成人,可付以政。」己巳,穆宗遇弒,帝率飛龍使女里、侍中蕭思溫、南院樞密使高勛
 率甲騎千人馳赴。黎明,至行在,哭之慟。群臣改進,遂即皇帝位於柩前。百官上尊號曰天贊皇帝,大赦,改元保寧。以殿前都點檢耶律夷臘、右皮室詳穩蕭烏里只宿衛不嚴,斬之。



 三月丙戌,入上京,以蕭思溫為北院樞密使。太平王罨撒葛亡入沙沱。己丑,夷離畢粘木袞以陰附罨撒葛伏誅。癸巳,罨撒葛入朝,甲午,以北院樞密使蕭思溫兼北府宰相。己亥,南院樞密使高勛封秦王。



 夏四月戊申朔,進封太平王罨撒葛為齊王,改封趙王喜隱為宋王,封隆先為平王,稍為吳王,道隱為蜀王,必攝為越王,敵烈為冀王,宛為衛王。五月戊寅,立貴妃蕭氏
 為皇后。丙申朔,射柳祈雨。有司請以帝生日為天清節,從之。壬寅,漢遣李匡弼、劉繼文、李元素等來賀。



 冬十月,東幸哀潭。



 十一月甲辰朔,行柴冊禮,祠木葉山,駐蹕鶴穀。乙巳,蕭思溫封魏王,北院大王屋質加於越。



 二年春正月丁未,如潢河。



 夏四月,幸東京,致奠於讓國皇帝及世宗廟。



 五月癸丑,西幸。乙卯,次盤道嶺,盜殺北樞密使蕭思溫。



 六月,還上京。



 秋七月,以右皮室詳穩賢適為北院樞密使。



 九月辛丑,得國舅蕭海只及海裏殺蕭思溫狀,皆伏誅,流其弟神觀於黃龍府。



 冬十二月庚午,漢遣使來貢。



 三年春正月甲寅,右夷離畢奚底遣人獻敵烈俘,詔賜有功將士。庚申,置登聞鼓院。辛酉,南京統軍使魏國公韓匡美封鄴王。二月癸酉,東幸。壬午,遣鐸遏使阿薩蘭回鶻。己丑,以青牛白馬祭天地。



 三月丁未,以飛龍使女里為契丹行宮都部署。



 夏四月丁卯,世宗妃啜里及蒲哥厭魅,賜死。己卯,祠木葉山,行再生禮。丙戌,至自東幸,戊子,蕭神睹伏誅。



 六月丙子,漢遣使問起居。自是繼月而至。丁丑,回鶻遣使來貢。



 秋七月辛丑,以北院樞密使賢適為西北路招討使。



 八月甲戌,如秋山。辛卯,祭皇兄吼墓,追冊為皇太子,謚莊聖。九月乙巳,賜傅父侍中達
 里迭、太保楚補、太保婆兒、保母回室、押雅等戶口、牛羊有差。又以潛邸給使者為撻馬部,置官掌之。壬子,幸歸化州。甲寅,如南京。



 冬十月己巳,以黑白羊祀神。癸未,漢遣使來貢。丙戌,鼻骨德、吐谷渾來貢。



 十一月庚子,臚朐河於越延尼裡等率戶四百五十來附,乞隸宮籍。詔留其戶,分隸敦睦、積慶、永興三宮,優賜遣之。



 十二月癸酉,以青牛白馬祭天地。己丑,皇子隆緒生。



 是冬,駐蹕金川。



 四年春二月癸亥,漢以皇子生,遣使來賀。



 閏月戊申,齊王罨撒葛薨。



 三月庚申朔,追冊為皇太叔。



 夏四月庚寅朔,追封蕭思溫為楚國王。



 是夏,駐蹕冰井。



 秋七月,如雲
 州。丁丑,鼻骨行車。



 冬十月丁亥朔,如南京。



 十二月甲午,詔內外官上封事。



 五年春正月甲子,惕隱休哥伐黨項,破之,以俘獲之數來上。漢遣使來貢。庚午,御五鳳樓觀登。



 二月丁亥,近侍實魯里誤觸神纛,法論死,杖釋之。壬辰,越王必攝獻黨項俘獲之數。戊申,以青牛白馬癸天地。辛亥,幸新城。



 三月乙卯朔,復幸新城。追封皇后祖胡母里為韓王,贈伯胡魯古兼政事令,尼古只兼侍中。



 夏四月丙申,白氣晝見。



 五月癸亥,於越屋質薨,輟朝三日。辛未,女直侵邊,殺都監達里迭、拽刺斡里魯,驅掠邊民牛馬。己卯,阿薩蘭
 回鶻來貢。六月庚寅,女直宰相及夷離堇來朝。丙申,漢遣人以宋事來告。秋七月庚辰,以保大軍節度使耶律斜裏底為中臺省左相。



 是月,駐蹕燕子城。



 九月壬子,鼻骨德部長曷魯撻覽來貢。



 冬十月丁酉,如南京。



 十一月辛亥朔,始獲應歷逆黨近侍小哥、花哥、辛古等,誅之。十二月戊戌,漢將改元,遣使稟命。是月,如歸化州。



 六年春正月癸未,幸南京。



 三月,宋遣使請和,以涿州刺史耶律昌術加侍中與宋議和。



 夏四月,宋王喜隱坐謀反廢。



 秋七月丁未朔,閣門使酌古加檢校太尉兼御史大夫,男海裏以告喜隱事,遙授隴州防禦使。庚申,獵於
 平地松林。



 冬十月乙亥朔,還上京。



 十二月戊子,以沙門昭敏為三京諸道僧尼都總管,加兼侍中。



 七年春正月戌朔,宋遣使來賀。壬寅,望祠木葉山。



 二月癸亥,漢雁門節度使劉繼文來朝,貢方物。丙寅,以青牛白馬祭天地。



 三月壬午,耶律速撒等獻黨項俘,分賜群臣。



 夏四月,遣郎君矧思使宋。己酉,祠木葉山。辛亥,射柳祈雨。如頻蹕澱清暑。



 五月丙戌,祭神姑。



 秋七月,黃龍府衛將燕頗殺都監張琚以叛,遣敞史耶律曷里必討之。九月,敗燕頗於治河,遣其弟安摶追之。燕頗走保匹惹城,安摶乃還,以餘黨千餘戶城通州。



 是秋,至自頻蹕
 澱。



 冬十月,鉤魚土河。



 八年春正月癸酉,宋遣使來聘。



 二月壬寅,諭史館學士,書皇后言亦稱「朕」暨「予」,著為定式。



 三月辛未,遣五使廉問四方鰥寡孤獨及貧乏失職者,振之。



 夏六月,以西南面招討使耶斜軫為北院大王。



 秋七月丙寅朔,寧王只沒妻安只伏誅,只沒、高勛等除名。



 辛未,宋遣使來賀天清節。



 八月癸卯,漢遣使言天清節設無遮會,飯僧祝厘。丁未,如秋山。己酉,漢以宋事來告。是月,女直侵貴德州東境。



 九月己巳,謁懷陵。辛未,東京統軍使察鄰、詳穩涸奏女直襲歸州五寨,剽掠而去。乙亥,鼻骨德來貢。壬
 午,漢為宋人所,遣使求援,命南府宰相耶律沙、冀王敵烈赴之。戊子,漢以宋師壓境,遣駙馬都尉盧俊來告。



 冬十月辛丑,漢以遼師退宋軍來謝。



 十一月丙子,宋主匡胤殂,其弟靈自立,遣使來告。辛卯,遣郎君王六、撻馬涅木古等使宋吊慰。



 十二月壬寅,遣蕭只古、馬哲賀宋即位。丁未,漢以宋軍復至、掠其軍儲來告,且乞賜糧為助。戊午,詔南京復禮部貢院。是月,轄戞斯國遣使來



\end{pinyinscope}