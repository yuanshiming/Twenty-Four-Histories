\article{卷九十九 列傳第三十七 列女}

\begin{pinyinscope}

 刑簡妻陳氏取
 律氏常哥耶律奴妻蕭氏耶律術者妻蕭氏耶律中妻蕭氏男女居室,人之大倫。與其得列女,不若得賢女。天下而有列女之名,非幸也。《詩》贊衛共姜,《春秋》褒宋伯姬,蓋不得已,所以重人倫之變也。遼據北方,風化視中土為疏。終遼之世,得賢女二,列女三,以見人心之天理有不與世道存亡者。



 邢簡妻陳氏,營州人。父陘,五代時累官司徒。



 陳氏甫笄,
 涉通經義,凡覽詩賦,輒能誦,尤好吟詠,時以女秀才名之。年二十,歸於簡。孝舅姑,閨門和睦,親黨推重。有六子,陳氏親教以經。後二子抱樸、抱質皆以賢,位宰相。統和十二年卒。睿智皇后聞之,嗟悼,贈魯國夫人,刻石以表其行。及遷袝,遣使以祭。論者謂貞靜柔順,婦道母儀始終無慊云。



 耶律氏,太師適魯之妹,小字常哥。幼爽秀,有成人風。



 及長,操行修潔,自誓不嫁。能詩文,不茍作。讀《通歷》,見前人得失,歷能品藻。



 咸雍間,作文以述時政。其略曰:「君以民為體,民以君為心。人主當任忠賢,人臣當任忠賢,人臣當去比周;則政
 化平,陰陽順。欲懷遠,則崇恩尚德;欲強國,則輕徭薄賦。四端五典為治教之本,六府三事實生民之命。淫侈可以為戒,勤儉可以為師。錯枉則人不敢詐,顯忠則人不敢欺。勿泥空門,崇飾土木;勿事邊鄙,妄費金帛。滿當思溢,安必慮危。刑罰當罪,則民勸善。不寶遠物,則賢者至。建萬世磐石之業,制諸部強橫之心。欲率下,則先正身;欲治遠,則始朝廷。」上稱善。時樞密使耶律乙辛愛其才,屢求詩,常哥遺以回文。乙辛知其諷己,銜之。大康三年,皇太子坐事,乙辛誣以罪,按無跡,獲免。會兄適魯謫鎮州,常哥與俱,常布衣疏食。人間曰:「何自苦如此?」對曰:「皇
 儲無罪道廢,我輩豈可美食安寢。」



 及太子被害,不勝哀痛。年七十,卒於家。



 耶律奴妻蕭氏,小字意辛,國舅駙馬都尉陶蘇斡之女。母胡獨公主。



 意辛美姿容,年二十,始適奴。事親睦族,以孝謹聞。嘗與娣姒會,爭言厭魅以取夫寵,意辛曰:「厭魅不若禮法。」



 眾問其故,意辛曰:「修己以潔,奉長以敬,事夫以柔,撫下以寬,毋使君子見其輕易,此之為禮法,自然取重於夫。以厭魅獲寵,獨不愧於心乎!」聞者大慚。



 初,奴與樞密使乙辛有隙。及皇太子廢,被誣奪爵,沒入興聖宮,流烏古部。上以意辛公主之女,欲使絕婚。意辛辭曰:「
 陛下以妄葭莩之親,使免流竄,實天地之恩。然夫婦之義,生死以之。妄自笄年從奴,一旦臨難,頓爾乖離,背綱常之道,於禽獸何異?幸陛下哀憐,與奴俱行,妥即死無恨!」帝感其言,從之。



 意辛久在貶所,親執役事,雖勞無難色。事夫禮敬,有加於舊。壽隆中,上書乞子孫為著帳郎君。帝嘉其節,召舉家還。



 子國隱,乾統間始仕。保大中,意辛在臨演,謂諸子曰:「吾度盧彥倫必叛,汝輩速避,我當死之。」賊至,遇害。



 耶律術者妻蕭氏,小字訛裏本,國舅孛堇之女。性端愨,有容色,自幼與他女異。年十八,歸術者。謹裕貞婉,娣姒
 推尊之。及居術者喪,極哀毀。既葬,謂所親曰:「夫婦之道,如陰陽表裏。無陽則陰不能立,無表則裹無所附。妾今不幸失所天,且生必有死,理之自然。術者早歲登朝,有才不壽。天禍妾身,罹此酷罰,復何依恃。儻死者可見,則從;不可見,則當與俱。」侍婢慰勉,竟無回意,自刃而卒。



 耶律中妻蕭氏,小字挼蘭,韓國王惠之四世孫。聰慧謹願。



 年二十歸於中,事夫敬順,親戚咸譽其德。中嘗謂曰:「汝可粗知書,以前貞淑為鑒。」遂發心誦習,多涉古今。



 天慶中,為賊所執,潛置刃於履,誓曰:「人欲污我者,即死之。」至夜,賊遁而免。久之,帝召中為五院都監,中謂妻曰:「吾
 本無宦情,今不能免。我當以死報國,汝能從我乎?」



 挼蘭對曰:「謹奉教。」及金兵徇地嶺西,盡徙其民,中守節死。挼蘭悲戚不形於外,人怪之。俄躍馬突出,至中死所自殺。



 論曰:「陳氏以經教二子,並為賢相,耶律氏自潔不嫁,居閨閫之內而不忘忠其君,非賢而能之乎。三蕭氏之節,雖烈丈夫有不能者矣。」



\end{pinyinscope}