\article{卷七十 列傳第八}

\begin{pinyinscope}

 耶律
 夷臘葛蕭海琢蕭護思蕭思溫蕭繼先耶律夷臘葛,字薊散,本官分人檢校太師合魯之子。



 應歷初,以父任入侍。數歲,始為殿前都點檢。時上新即位,疑諸王有異志,引夷臘葛為布衣交,一切機密事必與之謀,遷寄班都知,賜宮戶。



 時上酗酒,數以細故弒人。有
 監雉看因傷雉而亡,獲之欲誅,夷臘葛諫曰:「是罪不應死。」帝竟弒之,以尸付夷臘葛曰:「收汝故人!」夷臘葛終不為止。復有監鹿詳穩亡一鹿,下獄當死,夷臘葛又諫曰:「人命至重,豈可為一獸弒之?」



 良久,得免。



 遼法,騼歧角者,惟天子得射。會秋獵,善為鹿鳴者呼一騼至,命夷臘葛射,應弦而踣。上大悅,賜金、銀各百兩,名馬百疋,及黑山東抹真之地。



 後穆宗被弒,坐守衛不嚴,被誅。蕭海瓈,字寅的哂,其先遙輦氏時為本部夷離堇;父塔列,天顯間為本部令穩。



 海瓈貌魁偉,膂力過人。天祿間,娶明王安端女藹因翁主。



 應歷初,察割亂,藹因連坐,繼
 娶嘲瑰翁主。上以近戚,嘉其勤篤,命預北府宰相選。頃之,總知軍國事。



 時諸王多坐反逆,海瓈為人廉謹,達政體,每被命按獄,多得其情,人無冤者,由是知名。漢主劉承鈞每遣使入貢,必別致幣物,詔許受之。年五十卒,帝愍悼,輟朝二日。



 蕭護思,字延寧,世為北院吏,累遷御史中丞,總典群牧部籍。應歷初,遷左客省便。未幾,拜御史大夫。時諸王多坐事系獄,上以護思有才幹,詔窮治,稱旨,改北院樞密使,仍命世預宰相選。護思辭曰:「臣子孫賢否未知,得一客省使足矣。」



 從之。上晚歲酗酒,用刑多濫,護思居要地,
 齪齪自保,未嘗一言匡救,議者以是少之。年五十七卒。



 蕭思溫,小字寅古,宰相敵魯之族弟忽沒里之子。通書史。



 太宗時為奚禿里主尉,尚燕國公主,為群牧都林牙。思溫在軍中,握齱修邊幅,僚佐皆言非將帥才。尋為南京留守。



 初,周人攻揚州,上遣思溫躡其後,憚暑不敢進,拔緣邊數城而還。後周師來侵,圍馮母鎮,勢甚張。思溫請益兵,帝報曰:「敵來,則與統軍司並兵拒之;敵去,則務農作,勿勞士馬。」會敵入束城,我軍退渡滹沱而屯。思溫勒兵徐行,周軍數日不動。思溫與諸將議曰:「敵眾而銳,戰不利則有後患。



 不如頓兵以老其師,躡而擊之,可以
 必勝。」諸將從之。遂與統軍司兵會,飾他說請濟師。周人引退,思溫亦還。



 己而,周主復北侵,與其將傅元卿、李崇進等分道並進,圍瀛州,陷益津、瓦橋、淤口三關,垂迫固安。思溫不知計所出,但云車駕旦夕至;麾下士奮躍請戰,不從。已而,陷易、瀛、莫等州,京畿人皆震駭,往往遁入西山。思溫以邊防失利,恐朝廷罪己,表請親征。會周主榮以病婦,思溫退至益津,偽言不知所在。遇步卒二千餘人來拒,敗之。是年,聞周喪,燕民始安,乃班師。



 時穆宗湎酒嗜弒,思溫以密戚預政,無所匡輔,士論不與。



 十九年,春搜,上射熊而中,思溫與夷離畢牙里斯等進酒上
 壽,帝醉還宮。是夜,為庖人斯奴古等所弒。思溫與南院樞密使高勛、飛龍使女裡等立景宗。



 保寧初,為北院樞密使,兼北府宰相,仍命世預其選。上冊思溫女為後,加尚書令,封魏王。從帝獵閭山,為賊所害。



 蕭繼先,字楊隱,小字留只哥。幼穎悟,叔思溫命為子,睿智皇后尤愛之。乾亨初,尚齊國公主,拜駙馬都尉。



 統和四年,宋人來侵,繼先率邏騎逆境上,多所俘獲,上嘉之,拜北府宰相。自是出師,繼先必將本府兵先從。拔狼山百壘,從破宋軍應州,上南征取通利軍,戰稱捷力。及親征高麗,以繼先年老,留守上京。卒,年五十八。



 繼先雖處
 富貴,尚儉素,所至以善治稱,故將兵攻戰,未嘗失利,名重戚里。



 論曰:「嗚呼!人君之過,莫大於弒無辜。湯之伐桀也,數其罪曰『並告無辜於上下神祗』;武王之伐紂也,數其罪曰『無辜籲天』;堯之伐苗民也,呂侯追數其罪曰『弒戮無辜』。跡是言之,夷臘葛之諫,凜凜庶幾古君子之風矣。



 「雖然,善諫者不課於已然。蓋必先得於心術之徽,如察脈者,先其病而治之,則易為功。穆宗沉湎失德,蓋其資富強之勢以自肆久矣。使群臣於造次動作之際,此諫彼凈,提而警之,以防其甚,則亦詛至是哉。於以知護思、思溫
 處位優重,耽祿取容,真鄙夫矣!若海瓈之折獄,繼先之善治,可謂任職臣歟。」



\end{pinyinscope}