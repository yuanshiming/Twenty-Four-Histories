\article{卷一百六 列傳第四十四 逆臣下}

\begin{pinyinscope}

 蕭
 胡睹蕭迭裡得古迭耶律撒刺竹奚回離保蕭特烈蕭胡睹,字乙辛。口吃,視斜,發卷,伯父章穆見之曰:「是兒狀貌,族中未嘗有。」及壯,魁梧架傲,好揚人惡。



 重熙中,為只候郎君。俄遷興聖宮使,尚秦國長公主,授駙馬都尉。以不諧離婚,復尚齊國公主,為北面林牙。



 清寧中,歷北、商院樞密副使,代族兄術哲為西北路招討使。時蕭革與蕭阿刺俱為樞密使,不協,革以術哲為阿刺所愛,嫉
 之。術哲受代赴閥,先嘗借官粟,留直而去。胡睹希革意,發其事,術哲因得罪。



 胡睹又欲要權,歲時獻遺珍玩、畜產於革,二人相愛過於兄弟。胡睹族弟敵烈為北克薦國舅詳穩蕭胡篤於胡睹,胡睹見其辨給壯勇,傾心交結。每遇休沐,言論終日,人皆怪之。會胡睹同知北院樞密事,奏胡篤及敵烈可用,帝以敵烈為旗鼓拽刺詳穩,胡篤為宿直官。及革構陷其兄阿刺,胡篤陰為之助,時人醜之。



 耶律乙辛知北院樞密事,胡睹位在乙辛下,意怏怏不平。



 初,胡睹嘗與重元子涅魯古謀逆,欲其速發。會車駕獵太子山,遂與涅魯古脅弩手軍犯行宮。既戰,
 涅魯古中流矢而斃,眾皆逃散。時同黨耶律撒刺竹適在圍場,聞亂,率獵夫來援。其黨謂胡睹等曰:「我軍甚眾,乘其無備,中夜決戰,事冀有成;若至明日,其誰從我?」胡睹曰:「倉卒中,黑白不辨。若內外軍相應,則吾事去矣。黎明而發,何遲之有!」重元聽胡睹之計,令四面巡警待旦。是夜,同黨立重元僭位號,胡睹自為樞密使。



 明日戰敗,胡睹被創,單騎遁走,至十七濼,投水死。五子,同日誅之。



 蕭迭裡得,字胡睹堇,國舅少父房之後。父雙古,尚鈿匿公主,仕至國舅詳穩。



 迭裡得幼警敏不羈,好射獵。太平中,以外戚補祗候郎君,歷延昌宮使、殿前副點檢。重熙
 十三年伐夏,迭裡得將偏師首入敵境,多所俘掠,遷都點檢,改烏古敵烈部都詳穩。十八年,再舉西伐,迭裡得奏:「軍馬器械之事,務在選將,夏人豈為難制。但嚴設斥堠,不用掩襲計,何慮不勝?」帝曰:「卿其速行,無後軍期。」既而迭裡得失利還,復為都點檢。十九年,夏人來侵金肅軍,上遣迭裡得率輕兵督戰,至河南三角川,斬堠者八人,擒觀察使,以功命知漢人行官都部署事,出為酉甫面招討使。



 族弟黃八家奴告其主私議宮掖事,迭裡得寢之。事覺,決大杖,削爵為民。清寧中,上以所坐事非迭裡得所犯,起為南京統軍使。至是,從重元子涅魯古等
 亂,敗走被擒,伏誅。古迭,本宮分人,不知姓氏。好戲押,不喜繩檢。膂力過人,善擊鞠。



 重熙初,為護衛,歷宿直官。十三年,西征,以古迭為先鋒,夏人伏兵掩之,古迭力戰,麾下士多及,乃單騎突出。遇夏王李元吳來圍,勢甚急。古迭馳射,應弦輒僕;躍馬直擊中堅,夏兵不能當,晡乃還營。改興聖宮太保。



 清寧九年,從重元、涅魯古亂,與扈從兵戰,敗而遁,追擒之,陵遲而死。



 撒刺竹,孟父房滌例之孫。性兇暴。



 清寧中,累遷宣徽使,改殿前都點檢,首與重元謀亂。會帝獵灤河,重元恐事
 洩,與扈從軍倉卒而戰。其子涅魯古既死,同黨潰散。撒刺竹適在畋所,聞亂,劫獵夫以援。既至,知涅魯古已死,大悔恨之,謂曰:「我輩惟有死戰,胡為若兒戲,自取殞滅?今行官無備,乘夜劫之,大事可濟。若俟明旦,彼將有備,安知我眾不攜貳。一失機會,悔將無及。」重元、蕭胡睹等曰:「今夕但可四面圍之,勿令外軍得入,彼何能備!」



 不從。遲明,投仗而走,撒刺竹戰死。



 奚回離保,一名翰,字挼懶,奚王忒鄰之後。善騎射,趫捷而勇,與其兄鱉裏刺齊名。



 大安中,車駕幸中京,補護衛,稍遷鐵鷂軍詳穩。天慶間,徙北女直詳穩,兼知威州路
 兵馬事,改東京統軍。既而諸蕃入寇,悉破之,遷奚六部大王,兼總知東路兵馬事。



 保大二年,金兵至,天祚播遷,回離保率吏民立秦晉國王淳為帝。淳為署回了保知北院樞密事,兼諸軍都統,屢敗宋兵。淳死,其妻普賢女攝事。是年,金兵由居庸關入,回離保知北院,即箭笴山自立,號奚國皇帝,改元天復,設奚、漢、渤海三樞密院,改東、西節度使為二王,分司建官。



 時奚人巴輒、韓家奴等引兵擊附近契丹部落,劫掠人畜,群情大駭。會回離保為郭藥師所敗,一軍離心,其黨耶律阿古哲與其甥乙室八斤等殺之,偽立凡八月。



 蕭特烈,字訛都碗,遙輦窪可汗宮分人。乾統中,入宿衛,出為順義軍節度使。天慶四年,同知咸州路兵馬事。五年,以兵敗奪節度使。



 保大元年,遷隗古部節度使。及天祚在山西集群牧兵,特烈為副統軍。聞金兵將至,特烈諭士卒以君臣之義,死戰於石輦鐸。金兵不戰,特烈伺間欲攻之。天祚喜甚,召嬪御諸子登高同觀,將詫之。金兵望日月旗,知天祚在其下,以勁兵直趨奮擊,無敢當者,天祚遁走。特烈所至,招集散亡,尋為中軍都統,復敗於梯已山。



 天祚決意渡河奔夏,從臣切諫不聽,人情惶懼不知所為。



 特烈陰謂耶律兀直曰:「事勢如此,億兆離
 心,正我輩郊節之秋。不早為計,奈社棱何!」遂共劫梁王雅里,奔西北諸部,偽立為帝,特烈自為樞密使。



 雅里卒,欲擇可立者。會耶律兀直言術烈才德純備,兼興宗之孫,眾皆曰可,遂僭立焉,特烈偽職如故。未三旬,與術烈俱為亂兵所殺。



 論曰:「遼之秉國鈞,握兵柄,節制諸部帳,非宗室外戚不使,豈不以為帝王久長萬世之計哉。及夫肆叛逆,致亂亡,皆是人也。有國家者,可不深戒矣乎!」



\end{pinyinscope}