\article{卷四十二志第十二 歷象志上 歷}

\begin{pinyinscope}

 遼以幽、營立國,禮樂制度規模日完,授歷頒朔二百餘年。



 今奉詔修遼史,體與宋、金似,其《大明歷》不可少也。歷書法禁不可得,求《大明》歷元,得祖沖之法於外史,沖之之法,遼歷之所從出也歟?國朝亦嘗因之。以沖之法算,而至於遼更歷之年,以起元數,是蓋遼《大明歷》。遼歷因是固可補,然弗之補,史貴闕艾也。外史紀其法,同天存
 其職,遼史志是足矣。作《歷象志》。



 歷大同元年,太宗皇帝自晉汗京收百司僚屬伎術歷象,遷於中京,遼始有歷。先是,梁、唐仍用唐景福崇玄歷。晉天福四年,司天監馬重績奏上《乙未元歷》,號《調元歷》,太宗所收於汴是也。穆宗應歷十一年,司天王白、李玉等進歷,蓋《乙未元歷》也。聖宗統和十二年,可汗州刺史賈俊進新歷,則《大明歷》也。高麗所志《大遼古今錄》稱統和十二年始頒正朔改歷,驗矣。《大明歷》本宋祖沖之法,具見沈約《宋書》。具如左。



 宋武帝大明六年,祖沖之上《甲子元歷》法,未及施用,因名《大明歷》。



 上元甲子至宋大明七年癸卯,五萬一千九百三十九年算外。



 元法:五十九萬二千三百六十五。



 紀法:三萬九千四百九十一。



 章歲,三百九十一。



 章月:四千八百三十六。



 章閏:一百四十四。



 閏法:十二。



 月法:十一萬六千三百二十
 一。



 日法:三千九百三十九。



 餘數:二十萬七千四十四。



 歲餘:九千五百八十九。



 沒分:三百六十萬五千九百五十一。



 沒法:五萬一千七百六十一。



 周天:一千四百四十二萬四千六百六十四。



 虛分:萬四百四十九。



 行分法:二十三。



 小分法:一千七百一十七。



 通周:七十二萬六千八百一十。



 會周:七十一萬七千七百七十七。



 通法:二萬六千三百七十七。



 差率:三十九。



 推朔術:置入上元年數算外,以章月乘之,滿章多為積月,不盡為閏餘。閏作二百四十七以上,其年有閏。以月法乘積月,滿日法為積日,不盡為小餘。六旬去積日,不盡為大餘。大餘命以甲子,算外,所求年天王十一月朔也。小餘千八百四十九以上,其月大。



 求次月:
 加大餘二十九,小餘二千九十。小餘滿日法從大餘,大餘滿六旬去之,命如前,次月朔也。



 求弦望:加朔大餘七,小餘千五百七,小分一。小分滿四處小餘,小餘滿日法從大餘,命如前,上弦日也。又加得望,又加得下弦,又加得後月朔也。



 推閏術:以閏餘減章歲,餘滿閏法得一月,命以天王,算外,閏所在也。閏有進退,以無中氣為正。



 推二十四氣:
 置入上元年數算外,以餘數乘之,滿紀法為積日,不盡為小餘。六旬去積日,不盡為大餘。大餘命以甲子,算外,天王十一月冬至日也。



 求次氣:加大餘十五,小餘八千六百二十六,小分五。小分滿六從小餘,小餘滿紀法從大餘,命如前,次氣日也。



 求土王用事:加冬至大餘二十七,小餘萬五千五百二十八,季冬土用事日也。又加大餘九十一,小餘萬二千二百七十,次土用事日也。



 推沒術:以九十乘冬至小餘,以減沒分,滿沒法為日,不盡為日餘,命日以冬至,算外,沒日也。



 求次沒:加日六十九,日餘三萬四千四百四十二,餘滿沒法從日,次沒日也。日餘盡為滅。



 推日所在度術:以紀法乘朔積日為度實,周天去之,餘滿紀法為積度,不盡為度餘。命以虛一,次宿除之,算外,天正十一月朔夜半日所在度也。



 求次月:大月加度三十,小月加度二十九,入虛去度分。



 求行分:以小分法除度餘,所得為行分,不盡為小分,小分滿法從行分,行分滿法從度。



 求次日:加一度。入虛去行分六,小分百四十七。



 推月所在度術:以朔小餘乘百二十四為度餘,又以朔小餘乘八百六十為微分,微分滿月法從度餘,度餘滿紀法力度。以
 減朔夜半日所在,則月所在度。



 求次月:大月加度三十五,度餘三萬一千八百三十四,微分七萬七千九百六十七,小月加度二十二,度餘萬七千二百六十一,微分六萬三千七百三十六,入虛去度也。



 遲疾歷:推入遲疾歷術:以通法乘朔積日為通實,通周去之,餘滿通法
 為日,不盡為日餘。命日算外,天王十一月朔夜半入歷日也。



 求次月:大月加二日,小月加一日,日餘皆萬一千七百四十六。歷滿二十七日,日餘萬四千六百三十一,則去之。



 本次日:加一日。



 求日所在定度:以
 夜半入歷日餘乘損益率,以損益盈縮積分,如差率而一,所得滿紀法力度,不盡為度餘,以盈加縮減平行度及餘為定度。



 益之或滿法,損之或不足,以紀法進退。求度行分如上法。求次日,如所入遲疾加之。虛去分,如上法。



 陰陽歷:推入陰陽歷術:置通寶以會周去之,不滿交數三十五萬八千八百八十八半為朔入陽歷分,各去之,為朔入陰歷分,各滿通法得一日,不盡為日餘。命日算外,天王十一月朔夜半入歷日也。



 求次月:大月加二日,小月加一日,日餘皆二萬七百七十九。
 歷滿十三日,日餘萬五千九百八十七半,則去之。陽竟入陰,陰竟入陽。求次日:加一日。



 求朔望差:以二千二十九乘朔小餘,滿三百三為日餘,不盡倍之為小分,則朔差數也。加一十四日,日餘二萬一百八十六,小分百二十五。小分滿六百六從日餘,日餘滿通法為日,即望差數也。



 又加之,後月朔也。



 求合朔月食:置朔望夜半入陰陽歷及餘,有年者去之,置小分三
 百三,以差數加之。小分滿六百六從日餘,日餘滿通法從日,日滿一歷去之。命日算外,則朔望加時入歷也。朔望加時入歷一日,日餘四千一百九十八,小分四百二十八以下,十二日,日餘萬一千七百八十八,小分四百八十一以上,朔則交會,望則月食。



 求合朔月食定大小餘:令差數日餘加夜半入遲疾歷餘,日餘滿通法從日,則朔望加時入歷也。以入歷餘乘損益率,以損益盈縮積分,如差法而一,以盈減縮加本朔望小餘為定小餘。益之或滿法,損之或不足,以日法進退日。



 求合朔月食加時:以十二乘定小餘,滿日法得一辰,命以子,算外,加時所在辰也。有餘者四之,滿日法得一為少,二為半,三為太。又有餘者三之,滿日法得一為強,以強並少為少強,並半為半強,並太為太強。得二者為少弱,以並少為半弱,並半為太弱,並入為一辰弱,以前辰名之。



 求月去日道度:置入陰陽餘乘損益率,如通法而一,以損益數為定。定數十二而一為度。不盡因而一,為少、半、太。又不盡者三而一,一為強,二為少弱,則月去日道數也。陽歷在
 表,陰歷在裏。



 測景漏刻中星數:求昏明中星:各以度數如夜半日所在,則中星度。



 推五星術:木率:千五百七十五萬三千八十二。



 火率:三千八十萬四千一百九十六。



 土率:午四百九十三萬三百五十四。



 金率:二千三百六萬一十四。



 水
 率:四百五十七萬六千二百四。推五星術:置度實各以率去之,餘以減率,其餘,如紀法而一,為入歲日,不盡為日餘,命以天王朔,算外,星合日。



 求星合度。



 以入歲日及餘從天正朔日積度及餘,滿紀法從度,滿三百六十餘度分則去之,命以虛一,算外,星合所在度也。



 求星見日:以術伏日及餘加星合日及餘,餘滿紀法從日,命如前,見日也。求星見度:以術伏度及餘加星合度及餘,餘滿紀法從度,入虛去度分,命如前,星見度也。



 行五星法:以小分法除度餘,所得為行分,不盡為小分,及日加所行分,滿法從度,留者因前,逆則減之、伏不盡度。從行入虛,去行分六,小分百四十七,逆行出虛,則加之。



 木星:初與日合,伏,十六日,日餘萬七千八百三十二,行二度,度餘三萬七千五百四,晨見東方。從用行四分,百一十二日行十九度十一分。留,二十八日。逆,日行三分,八十六日退十一度五分。叉留二十八日。從,日行四分,百一十二日,夕伏西方,日度餘如初。一終三百九十八日,日餘三萬五千六百六十四,行三十三度,度餘二萬五千
 二百一十五。



 火星:初與日合,伏,七十二日,日餘六百人,行五十五度,度餘二萬大千八百六十五,晨見東方。從,疾,日行十七分,九十二日行六十八度。小遲,日行十四分,九十二日行五十六度。



 大遲,日行九分,九十二日行三十六度。留,十日。逆,日行六分,六十四日退十六度十六分。又留,十日。從,遲,日行九分。九十二日。小疾,日行十四分,九十二日。大疾,日行十七分,九十二日。夕伏西方,日度餘如初。一終七百八十日,日餘千二百一十六,行四百一十四度,度餘三萬二百五十八,除一周,定行四十九度,度餘萬九千八百九。



 土星:初與日合,伏,十七日,日餘千三百七十八,行一度,度餘萬九
 千三百三十三,晨見東方,行順,日行二分,八十四日行七度七分。留,三十三日。行逆,日行一分,百一十日退四度十八分。又留,三十三日。從,日行二分,八十四日,夕伏西方,日度餘如初。
 一終三
 百七十八日,日餘二千七百五十六,行十二度,度餘三萬一千七百九十八。



 金星:初與日合,伏,三十九日,日命三萬大千一百二十
 六,行四十九度,度餘三萬大千一百二十六,夕見西方。從,疾,日行一度五分,九十二日行百十二度。小遲,日行一度四分,九十二日行百八度。大遲,日行十七分,四十五日行三十三度六分。留,九日。遲,日行十六分,九日退大度六分,夕伏西方。



 伏五日,退五度,而與日合。又五日退五度,而晨見東方。逆,日行十六分,九日。留,九日。從,遲,日行十七分,四十五日。小疾,日行一度四分,九十二日。大疾,日行一度五分,九十二日。晨伏東方,日度餘如初。一終五百八十三日,日餘三萬大千七百六十一,行星如之。除一周,定
 行二百十八度,度餘二萬大千三百一十三。合二百九十一日,日餘二萬大千一百二十六,行星
 亦如之。水星:初與日合,伏,十四日,日餘三萬七千一百一十五,行三十度,度餘三萬七千一百一十五,少見西方。從,疾,日行一度六分,二十三日行二十九度。遲,日行二十分,八日行大度二十二分。留,二日。遲,日行十一分,二日退二十二分,夕伏西方。伏八日,退八度,而與日合。又八日退八度,晨見東方。逆,日行十一分,二日。留,二
 日。從,遲,日行二十分,八日。疾,日行一度大分,二十三日。晨伏東方,日度餘如初。



 一終百一十五日,日餘三萬四千七百三十九,行星如之。一合五十七日,日餘三萬七千一百一十五,行星亦如之。



 上元之歲,歲在甲子,天王甲子朔夜半冬至,日月五星聚於虛度之初,陰陽遲疾並自此始。



 梁武帝天監三年,沖之子暅上疏,論何承天歷乖謬不可用。



 九年正月,詔用祖沖之所造《甲子元歷》頒朔。陳氏因梁,亦用祖沖之歷。至遼,聖宗以賈俊所進新歷,因宋《大明》舊號行之。金曰《重修大明歷》。傳至皇元
 亦曰《重修大明歷》。及改《授時歷》,別立司天藍存肄之,每歲甲子冬至重修其法。



 書在太史院,禁莫得聞。



\end{pinyinscope}