\article{卷三十二志第二 營衛志中行營}

\begin{pinyinscope}

 《周官》土圭之法:曰東,景夕多風;曰北,景長多寒。天地之間,風氣異宜,人生其間,各適其便。王者因三才而節制之。長城以南,多雨多暑,其人耕稼以食,桑麻以衣,宮室以居,城郭以治。大漠之間,多寒多風,畜牧畋漁以食,皮毛以衣,轉徒隨時,車馬為家。此天時地利所以限南北
 也。遼國盡有大漠,浸包長城之境,因宜為治。秋冬違寒,春夏避暑,隨水草就畋漁,歲以為常。四時各有行在之所,謂之「捺缽」。



 春捺缽:曰鴨子河濼。皇帝正月上旬起牙帳,約六十曰方至。天鵝未至,卓帳冰上,鑿冰取魚。冰泮,乃縱鷹鶻捕鵝雁。晨出暮歸,從事弋獵。鴨子河凍東西二十里,南北三十里,在長春州東北三十五里,四面皆沙鍋,多榆柳否林。皇帝每至,侍御皆服墨綠色衣,各備連錘一柄,鷹食一器,刺鵝錐一枚,於澇周圍相去各五七步
 排立。皇帝冠巾,衣時服,系玉束帶,於上風望之。有鵝之處舉旗,探騎馳報,遠泊鳴鼓。鵝驚騰起,左右圍騎皆舉幟麾之。五坊擎進海東青鵝,拜授皇帝放之。鵝擒鵝墜,勢力不加,排立近者,舉錐刺鵝,取腦以飼鴿。救鶴人例賞銀絹。皇帝得頭鵝,薦廟,群臣備獻酒果,舉樂。更相酬酷,致賀語,皆插鵝毛於首以為樂。賜從人酒,遍散其毛。弋獵綱鉤,春盡乃還。



 夏捺缽:無常所,多在吐兒山。道宗每歲先幸黑山,拜聖宗、興宗陵,賞金蓮,乃幸子河避暑。吐兒山在黑山東北三
 百里,近饅頭山。黑山在慶州北十三里,上有池,池中有金蓮。子河在吐兒山東北三百里。懷州西山有清涼殿,亦為行幸避暑之所。四月中旬起牙帳,卜吉地為納涼所,五月末旬、六月上旬至。居五旬。與北、南臣僚議國事,暇曰游獵。七月申旬乃去。



 秋捺缽:曰伏虎林。七月中旬自納涼處起牙帳,入山射鹿及虎。林在永州西北五十里。嘗有虎據林,傷害居民畜牧。景宗領數騎獵焉,虎伏草際,戰慄不敢仰視,上舍之,因號伏澆林。每歲車駕至,皇族而下分布濼水側。
 伺夜將半,鹿飲鹽水,令獵人吹角效鹿鳴,既集而射之。謂之「抵堿鹿」,又名「呼鹿」。



 冬捺缽:曰廣平澱。在永州東南三十里,本名白馬澱。東西二十餘里,南北十餘里。地甚坦夷,四望皆沙磧,木多榆柳。其地饒沙,冬月稍暖,牙帳多於此坐冬,與北、南大臣會議國事,時出校獵講武,兼受南宋及諸國禮貢。皇帝牙帳以槍為硬寨,用毛繩連系。每槍下黑氈傘一,以庇衛士風雪。槍外小氈帳一層,每帳五人,各執兵仗為禁圍。南有省方殿,殿北約二里曰壽寧殿,皆
 木柱竹榱,以氈為蓋,彩繪韜柱,錦為壁衣,加徘繡額。



 又以黃布繡龍為地障,窗、槅皆以氈為之,傅以黃油絹。基高尺餘,兩廂廊廡亦以氈蓋,無門戶。省方殿北有鹿皮帳,帳次北有八方公用殿。壽寧殿北有長春帳,衛以硬寨。宮用契丹兵四千人,每曰輪番千人祗直。禁圍外卓槍為寨,夜則拔槍移卓御寢帳。周圍拒馬,外設鋪,傳鈴宿衛。



 每歲四時,周而復始。



 皇帝四時巡守,契丹大小內外臣僚並應役次人,及漢人宣徽院所管百司皆從。漢人樞密院、中書省唯摘宰相一員,樞密院都副承旨二員,令史十人,中書令史一
 人,御史臺、大理寺選摘一人扈從。每歲正月上旬,車駕啟行。宰相以下,還於中京居守,行遣漢人一切公事。除拜官僚,止行堂帖權差,俟會議行在所,取旨、出給誥敕。文官縣令、錄事以下更不奏聞,聽中書銓選;武官須奏聞。



 五月,納涼行在所,南、北臣僚會議。十月,坐冬行在所,亦如之。



 部族上部落曰部,氏族曰族。契丹故俗,分地而居,合族而處。



 有族而部者,五院、六院之類是也;有部而族者,奚王、室韋之類是也;有部而不族者,特裡特勉、稍瓦、曷術之類是
 也;有族而不部者,遙輦九帳、皇族三父房是也。



 奇首八部為高麗、蠕蠕所侵,僅以萬口附於元魏。生聚未幾,北齊見侵,掠男女十萬餘口。繼為突撅所逼,寄處高麗,不過萬家。部落離散,非復古八部矣。別部有臣附突厥者,內附於隋者,依紇臣水而居。部落漸眾,分為十部,有地遼西五百餘里。唐世大賀氏仍為八部,而松漠、玄州別出,亦十部也。



 遙輦氏承萬榮、可突於散敗之餘,更為八部;然遙輦、迭刺別出,又十部也。阻午可汗析為二十部,契丹始大。至於遼太祖,析九帳、三房之族,更列二十部。聖宗之世,分置十有六,增置十有八,並舊為五十四部;
 內有拔里、乙室已國舅族,外有附庸十部,盛矣!其氏族可知者,略具《皇族》、《外戚》二表。餘五院、六院、乙室部止見益古、撤裏本,涅刺、烏古部止見撒里隊、涅勒,突呂不、突舉部止見塔古里、航斡,皆兄弟也。奚王府部時瑟、哲里,則臣主也。品部有拿女,楮特部有窪。其餘世系名字,皆漫無所考矣。



 舊(志)曰:「契丹之初,草居野次,靡有定所。至涅里始制部族,各有分地。太祖之興,以迭刺部強熾,析為五院、六院。奚六部以下,多因俘降而置。勝兵甲者即著軍籍,分隸諸路詳穩、統軍、招討司。番居內地者,歲時田牧平莽間。邊防糾戶,生生之資,仰給畜牧,績毛飲湩,
 以為衣食。各安舊風,狃習勞事,不見紛華異物而遷。故家給人足,戎備整完。卒之虎視四方,強朝弱附,東逾蟠木,西越流沙,莫不率服。部族實為之爪牙雲。」



 古八部;悉萬丹部。



 何大何部。



 伏弗鬱部。



 習陵部。



 日連部。



 匹潔部。



 黎部。



 吐六於部。



 契丹之先,曰奇首可汗,生八子。其後族屬漸盛,分為八部,居松漠之間。今永州木葉山有契丹始祖廟,奇首可汗、可敦並八子像在焉。潢河之西,土河之北,奇首可汁故壤也。



 隨契丹十部:元魏末,莫弗賀勿於畏高麗、蠕蠕侵通。率車三千乘、眾萬口內附,乃去奇首可汗故壤,居白狼水東。北齊文宣帝自平州三道來侵,掠男女十餘萬口,分置諸
 州。又為突厥所逼,以萬家寄處高麗境內。隋開皇四年,諸莫弗賀悉眾款塞,聽居白狼故地。又別部寄處高麗者曰出伏等,率眾內附,詔置獨奚那頡之北。又別部臣附突厥者四千餘戶來降,詔給糧遣還,固辭不去,部落漸眾,徒逐水草,依紇臣水而居。在遼西正北二百里,其地東西互五百里,南北三百里。分為十部,逸其名。



 唐大賀氏八部:達稽部,峭落州。



 紀便部,彈汗州。



 獨活部,無逢州。



 芬問部,羽陵州。



 突便部,日連州。



 芮奚部,徒河州。



 墜斤部,萬丹州。



 伏部,州二:匹黎、赤山。



 唐太宗置玄州,以契丹大帥據曲為刺史。又置松漠都督府,以窟哥為都督,分八部,並玄州為十州。則十部在其中矣。



 遙輦氏八部:
 旦利皆部。



 乙室活部。



 實活部。



 納尾部。



 頻沒部。



 納會雞部。



 集解部。



 奚嗢部。當唐開元、天寶間,大賀氏既微,遼始祖涅里立迪輦祖里為阻午可汗。時契丹因萬榮之敗,部落凋散,即
 故有族眾分為八部。涅里所統迭刺部自為別部,不與其列。並遙輦、迭刺亦十部也。



 遙輦阻午可汗二十部:耶律七部。



 審密五部。



 八部。



 涅里相阻午可汗,分三耶律為七,二審密為五,並前八部二十部。三耶律:一曰大賀,二曰遙輦,三曰世里,即皇族也。



 二審密:一曰乙窒已,二曰拔里,即國舅也。其分郡皆末詳;可知者曰迭刺,曰乙室,曰品,曰楮
 特,曰烏隗,曰突呂不,曰捏刺,曰突舉,又有右大部、左大部,凡十,逸其二。大賀、遙輦析為六,而世里合為一,茲所以迭刺部終遙輦之世,強不可制云。



\end{pinyinscope}