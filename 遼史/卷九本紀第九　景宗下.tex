\article{卷九本紀第九 景宗下}

\begin{pinyinscope}

 發
 御
 春正月丙寅,女直遣使來貢。



 二月庚子,宋遣使致其先帝遣物。甲寅,以青牛白馬祭天地。



 三月癸亥,耶律沙、敵烈獻援漢之役所獲宋俘。戊振,詔以粟二十萬斛助漢。



 夏五月庚午,漢遣使來謝,且以宋事來告。己丑,女直二十一人來請宰相、夷離堇之職,以次授之。



 六月丙辰,以宋王喜隱為西南面招討使。



 秋七月庚申朔,回鶻遣
 使來貢。甲子,宋遣使來聘。壬申,漢以宋侵來告。丙子,遣使助漢戰馬。



 八月,漢遣使進葡萄酒。



 冬十月甲子,耶律沙以黨項降酋可醜、買友來見,賜詔撫諭。丁卯,以可醜為司徒,買友為太保,各賜物遣之。壬申,女直遣使來貢。乙酉,漢復遣使以宋事來告。



 十一月丁亥朔,司天奏日當食不虧。戊戌,吐欲渾叛入太原者四百餘戶,索而還之。,癸卯,祠木葉山。乙巳,遣太保迭烈割等使宋。乙卯,漢復遣使以宋事來告。十二月戊辰,獵於近郊,以所獲祭天。



 十年春正月癸丑,如長濼。



 二月庚午,阿薩蘭回鶻來貢。



 三月庚寅,祭顯陵。



 夏四月丁卯,西幸。己巳,女直遣使來貢。



 五月癸卯,賜女里死,遣人誅高勛等。



 六月己未,駐蹕沿柳湖。



 秋七月庚戌,享太祖廟。



 九月癸未朔,平王隆先子陳哥謀害其父,車裂以徇。



 是冬,駐蹕金川。



 乾亨元年春正月乙酉,遣撻馬長壽使宋,問興師伐劉繼元之故。丙申,長壽還,言「河東逆命,所當問罪。若北朝不援,和約如舊;不然則戰」。



 二月丁卯,漢以宋兵壓境,遣使乞援。詔南府宰相耶律沙為都統、冀王敵烈為監軍赴之;又命南院大王斜軫以所部從,樞密副使抹只督之。



 三月辛巳,速撒遣人以別療化哥等降,納之。丙戌,漢
 遣使謝撫諭軍民,詔北大王奚底、乙室王撒合等以兵戌燕。己丑,漢復告宋兵入境,詔左千牛衛大將軍韓侼、大同軍節度使耶律善補以本路兵南援。辛卯,女直遣使來貢。丁酉,耶律沙等與宋戰於白馬嶺,不利。冀王敵烈及突呂不部節度使都敏、黃皮室詳穩唐筈皆死之,士卒死傷甚眾。



 夏四月辛亥,漢以行軍事宜來奏,盧俊自代州馳狀告急。



 辛酉,敵烈來貢。



 五月己卯朔,宋兵至河東,漢與戰,不利,劉文,盧俊來奔。



 劉月,劉繼元降宋,漢亡。甲子,封劉繼文為彭城郡王,盧俊同政事門下平章事。宋主來侵。丁卯,北院大王奚底、統軍使蕭討古、乙
 室王撒合擊之。戰於沙河,失利。己巳,宋主圍南京。丁丑,詔諭耶律沙及奚底、討古等軍中事宜。



 秋七月癸未,沙等及宋兵戰於高梁河,少卻;休哥、斜軫橫擊,大敗之。宋主僅以身免,至涿州,竊乘驢車遁去。甲申,擊宋餘軍,所殺甚眾,獲兵仗、器甲、符印、糧饋、貨幣不可勝計。辛丑,耶律沙遣人上俘獲,以權知南京留守事韓德讓、權南京馬步軍都指揮使耶律學古、知三司事劉弘皆能安人心,捍城池,並賜詔褒獎。



 八月壬子,阻卜惕隱曷魯、夷離堇阿里睹等來朝。乙丑,耶律沙等獻俘。丙寅,以白馬之役責沙、抹只,復以走宋主功釋之;奚底遇敵而退,以劍
 背擊之;撒合雖卻,部伍不亂,宥之;冀王敵烈麾下先遁者斬之,都監以下杖之。壬申,宴沙、抹只等將校,賜物有差。



 九月己卯,燕王韓匡嗣為都統,南府宰相耶律沙為監軍,惕隱休哥、南院大王斜軫、權奚王抹只等只率所部兵南伐;仍命大同軍節度使善補領山西兵分道以進。



 冬十月乙,韓匡嗣與宋兵戰於滿城,敗績。辛未,太保矧思與宋兵戰於火山,敗之。乙亥,詔數韓匡嗣五罪,赦之。



 十一月戊寅,宴賞休哥及有功將校。乙未,南院樞密使兼政事令郭襲上書諫畋獵,嘉納之。辛丑,冬至,赦,改元乾亨。



 十二月乙卯,燕王匡嗣遙授晉昌軍節度
 使,降封秦王。壬戌,蜀王道隱南京留守,從封荊王。



 是冬,駐蹕南京。



 二年春正月丙子朔,封皇子隆緒為梁王,隆慶為恆王。丁亥,以惕隱休哥為北院大前樞密使賢適封西平郡王。



 二月戊辰,如清河。三月丁亥,西南面招討副使耶律王六、太尉化哥遣人獻黨項俘。閏月庚午,有鴇飛止禦帳,獲以祭天。



 夏四月庚辰,祈雨。戊子,清暑燕子城。



 五月,雷火陵松。



 六月己亥,喜隱復謀反,囚於祖州。



 秋七月戊午,王六等獻黨項俘。



 八月戊戌,東幸。



 冬十月辛未朔,命巫者祠天地及兵神。辛巳,將南,祭旗鼓。癸未,次
 南京。丁亥,獲敵人,射鬼箭。庚寅,次固安,以青牛白馬祭天地。己亥,圍瓦橋關。



 十一月庚子朔,宋兵夜襲營,突呂不部節度使蕭干及四捷軍詳穩耶律痕德戰卻之。壬寅,休哥敗宋兵於瓦橋東,守將張師引兵出戰,休哥奮擊,敗之。戊申,宋兵陳於水南,休哥涉水擊破之。追至莫州,殺傷甚眾。己酉,宋兵復來,擊之殆盡。



 丙辰,班師。乙丑,還次南京。



 十二月庚午朔,休哥拜於越。大饗軍士。



 三年春二月丙子,東幸。己丑,復幸南京。



 三月乙卯,皇子韓八卒。辛酉,葬潢、土二河之間,置永州。以秦王韓匡嗣為西南面招討使。



 夏五月丙午,以京漢軍亂,動立喜隱不
 克,偽立其子留禮壽,上京留這除室擒之。



 秋七月甲子,留禮壽伏誅。



 冬十月,如蒲瑰坡。



 十一月辛亥,加除室同政事門下平章事。是月,以南院樞密使郭襲為武定軍節度使。



 十二月,以遼興軍節度使韓德讓為南院樞密使。四年春正月己亥,如華林、天柱。



 三月乙未,清明。與諸王大臣較射,宴飲。



 夏四月,自將南伐。至滿城,戰不昨,守太尉奚瓦里中流矢死。統軍使善補為伏兵所圍,樞密使斜軫救免,詔以失備杖之。



 五月,班師,清暑燕子城。



 秋七月壬辰,遣使賜喜隱死。



 八月,如西京。



 九月良子,幸雲州。
 甲辰,獵於祥古山,帝不豫。壬子,次焦山,崩於行在。年三十五,在位十三年。遣詔梁王隆緒嗣位,軍國大事聽皇后命。統和元年正月壬戌,上尊謚孝成皇帝,廟號景宗。重熙二十一年,加謚孝成康靖皇帝。



 贊曰:遼興六十餘年,神冊、會同之間,日不暇給;天祿、應歷之君,不令其終;保寧而來,人人望治。以景宗之資,任人不疑,信賞必罰,若可與有為也。而竭國之力以助河東,破軍殺將,無救滅亡。雖一取償於宋,得不償失。知匡嗣之罪,數而不罰;善郭襲之諫,納而不用;沙門昭敏以左道亂德,龐以侍中。不亦感乎!



\end{pinyinscope}