\article{卷二十八本紀第二十八 天祚皇帝二}

\begin{pinyinscope}

 五年
 春正月,下詔親征,遣僧家奴持書約和,斥阿骨打名。



 阿骨打遣賽刺復書,若歸叛人阿疏,遷黃龍府於別地,然後議之。都統耶律斡里朵等與女直兵戰於達魯古城,敗績。



 二月,饒州渤海古欲等反,自稱大王。



 三月,以蕭謝佛留等討之。遣耶律張家奴等六人齊書使女直,斥其主名,冀以速降。



 夏四月癸丑,蕭謝佛留等為渤海
 古欲所敗,以南面副部署蕭陶蘇斡為都統,赴之。



 五月,陶蘇斡及古欲戰,敗績。張家奴等以阿骨打書來,復遣之往。



 六月亥朔,清暑特禮嶺。壬子,張家奴等還,阿骨打復書,亦斥名諭之使降。癸丑,以親征諭諸道。丙辰,陶蘇斡招獲古欲等。癸亥,以惕隱耶律末里為北院大王。是月,遣蕭辭刺使女直,以書辭不屈見留。



 秋七月辛未,宋遣使致助軍銀絹。丙子,獵於嶺東。是月,都統斡里朵等與女直戰於白馬凍,敗績。



 八月甲子,罷獵,趨軍中。以斡里朵等軍敗,免官。丙寅,以圍場使阿不為中軍都統,耶律張家奴為都監,率番、漢兵十萬;蕭奉先充御營都
 統,諸行營都部署耶律章奴為副,以精兵二萬為先鋒。餘分五部為正軍,貴族子弟千人為硬軍,扈從百司為護衛軍,北出駱駝口;以都點檢蕭胡睹姑為都統,樞密直學士柴誼為副,將漢步騎三萬,南出於江州。自長春州分道而進,發數月糧,期必滅女直。



 九月丁卯朔,女直軍隱黃龍府。己巳,知北院樞密使蕭得裏底出為西南面招討使。辭刺還,女直復遣賽刺以書來報,若歸我叛人阿疏等,即當班師。上親往。粘罕、兀術等以書來上,陽為卑哀之辭,實欲求戰。書上,上怒,下詔有「女直作過,大軍翦除」之語。女直主聚眾,剺面仰天偽哭曰:「始與汝等
 起兵,蓋苦契丹殘忍,欲自立國。今主上新徵,奈何?非人死戰,莫能當也。不殺我一族,汝等迎降,轉禍為福。」諸軍皆曰;「事已到此,惟命是從。」乙巳,耶律章奴反,奔上京,謀迎立魏國王淳。上遣附馬蕭晃領兵詣廣平澱護後妃,行宮小底乙信持書馳報魏國王。時章奴先遣王妃親弟蕭諦里以所謀說魏國王。王曰:「此非細事,主上自有諸王當立,北、南面大臣不來,而汝言及此,何也?」密令左右拘之。有頃,乙信等喪御札至,備言章奴等欲廢立事。魏國王立斬蕭諦裡等首以獻,單騎間道詣廣平澱待罪。上遇之如初。章奴知和魏國王不聽,率麾下掠慶、饒、
 懷、祖等州,結渤海群盜,眾至數萬,趨廣平澱犯行宮。順國女直阿鶻產以三百騎一戰而勝,擒其貴族二百餘人,並斬首以徇。其妻子配役繡院,或散諸近傳為婢,餘得脫者皆奔女直。章奴詐為使者,欲奔女直,為邏者所獲,縛送行在,腰斬於市,剖其心以獻祖廟,支解以徇五路。



 冬十一月,遣馳馬蕭特末、林牙蕭察刺等將騎兵五萬、步卒四十萬、新軍七十萬至駝門。十二月乙巳,耶律張家奴叛。戊申,親戰於護步答岡,敗績,盡亡其輜重。己未,錦州刺史耶律術者叛應張家奴。庚申,北面林牙耶律馬哥討張家奴。癸亥,以北院宣徽使蕭韓家奴知北
 院樞密使事,南院宣徽使蕭特末為漢人行官都部署。



 六年春五月丙寅朔,東京夜有惡少年十餘人,乘酒執刃,逾垣入留守府,問留守蕭保先所在:「今軍變,請為備。」蕭保先出,刺殺之。戶部使大公鼎聞亂,即攝留守事,與副留守高清明集奚、漢兵千人,盡捕其眾,斬之,撫定其民。東京故渤海地,太祖力戰二十餘年乃得之。而蕭保先嚴酷,渤海苦之,故有是變。其裨將渤海高永昌僭號,稱隆基元年。遣蕭乙薛、高興順招之,不從。



 閏月己亥,遣蕭韓家奴、張琳討之。戊午,貴德州守將耶律餘睹以廣州渤海叛附永昌,我師擊敗之。



 二月戊辰,侍禦司徒撻
 不也等討張家奴,戰於祖州,敗績。



 乙酉,遣漢人行宮都部暑蕭特末諸將討張家奴。戊子,張家奴誘饒州渤海及中京賊侯概等萬餘人,攻陷高州。



 三月,東面行軍副統酬斡等擒侯概於川州。



 夏四月戊辰,親征張家奴。癸酉,敗之。甲戌,誅叛黨,饒州渤海平。丙子,賞平賊將士有差;面蕭韓家奴、張琳等復為賊所敗。



 五月,清暑散水原。女直軍攻下沈州,復陷東京,擒高永昌。東京州縣族人痕李、鋒刺、吳十、撻不也、道刺、酬斡等十三人皆降女直。



 六月乙丑,籍諸路兵,有雜畜十頭以上者皆從軍。庚辰,魏國王淳進封秦晉國王,為都元帥,上京留守蕭撻
 不也為契丹行官都部署兼副元帥。丁亥,知北院樞密使事蕭韓家奴為上京留守。秋七月,獵秋山。春州渤海二千餘戶叛,東北路統軍使勒兵追及,盡俘以還。



 八月,烏古部叛,遣中丞耶律撻不也等招之。



 九月丙午,謁懷陵。



 冬十月丁卯,以張琳軍敗,奪官。庚辰,烏古部來降。



 十一月,東面行軍副統馬哥等攻曷蘇館,敗績。



 十二月乙亥,封庶人蕭氏為太皇太妃。辛巳,削副統耶律馬哥官。



 七年春正月甲寅,減廄馬粟,分給諸局。是月,女直軍攻春州,東北面諸軍不戰自潰,女古、皮室四部及勃海人皆降,復下泰州。



 二月,淶水縣賊董龐兒聚眾萬餘,西京
 留守蕭乙薛、南京統軍都監查刺與戰於易水,破之。



 三月,龐兒黨復聚,乙薛復擊破立於奉聖州。



 夏五月庚,東北面行軍諸將涅里、合魯、涅哥、虛古等棄市。乙巳,諸圍場隙地,縱百姓樵採。



 六月辛巳,以同知樞密院事餘里也為北院大王。



 秋七月癸卯,豬秋山。



 八月丙寅,獵狘斯那裏山,命都元帥秦晉王赴沿邊,會四路兵馬防秋。



 九月,上自燕至陰涼河,置怨軍八營,募自宜州者曰前宜、後宜,自錦州者曰前錦、後錦,自乾自顯者曰乾曰顯,又有乾顯大營、巖州營,凡二萬八千餘人,屯衛州蒺藜者。丁酉,獵輞子山。



 冬十月乙卯朔,至中京。



 十二月丙寅,
 都元帥秦晉國王淳遇女直軍,戰於蒺藜山,敗績。女直復拔顯州旁近州郡。庚午,下詔自責。癸酉,遣夷離畢查刺與大公鼎諸路募兵。丁丑,以西京留守蕭乙薛為北府宰相,東北路行軍都統奚霞末知奚六部大王事。



 是歲,女直阿骨打用鐵州楊樸策,即皇帝位,建元天輔,國號金。楊樸又言,自古英雄開國或受禪,必先求大國封冊,遂遣使議和,以求封冊。



 八年春五月,幸鴛鴦濼。丁亥,遣耶律奴哥等使金議和。



 庚寅,保安軍節度使張崇以雙州二百戶降金。東路諸州盜賊蜂起,掠民自隨以充食。



 二月,耶律奴哥還自金,
 金主復書曰:「能以兄事朕,歲貢方物,歸我上、中京、興中府三路州縣;以親王、公主、駙馬、大臣子孫為質;還我行人及元給信符,並宋、夏、高麗往復書詔、表牒,則可以如約。」



 三月甲午,復遣奴哥使會。



 夏四月辛酉,以西南面招討使蕭得裏底為北院樞密使。



 五月壬午朔,奴哥以書來,約不逾此月見報。戊戌,復遣奴哥使金,要以酌中之議。是月,至納葛濼。賊安生兒、張高兒聚眾二十萬,耶律馬哥等斬生兒於龍化州,高兒亡入懿州,與霍六哥相全。金主遣胡突兗與奴哥持書,報如前約。



 六月丁卯,遣奴哥等齏宋、夏、高麗書詔、表牒至金。霍六哥陷海北州,
 趣義州,軍帥回離保等擊敗之。通、祺、雙、遼四川之民八百餘戶降於金。



 秋七月,獵秋山。金復遣胡突兗來,免取質子及上京,興中府所屬州郡,裁減歲幣之數,「如能以兄事朕,冊用漢儀,可以如約」。



 八月庚午,遣奴哥、突迭使金,議冊禮。



 九月,突迭見留,遣奴哥還,謂之曰:「言如不從,勿復遣使。」閏月丙寅,遣奴哥復使金,而蕭寶、訛裡等十五人各率戶降於金。



 冬十月,奴哥、突迭持金書來。龍化州張應古等四人率眾降金。十一月,副元帥蕭撻不也薨。



 十二月甲申,議定冊禮,遣奴哥使金。寧昌軍節度使劉宏以懿州戶三千降金。時山前諸路大饑,乾、顯、宜、錦、
 興中等路,鬥粟直數縑,民削榆皮食之,既而人相食。



 是年,放進士王翬等百三人。



 九年春正月,金遣烏林答贊謨持書來迎冊。



 三月,至鴛鴦濼。賊張撒八誘中京射糧軍,僭號,南面軍帥餘睹擒撒八。



 三月丁未朔,遣知右夷離畢事蕭習泥烈等冊金主為東懷國皇帝。己酉,烏林答贊謨、奴哥等先以書報。



 夏五月,阻卜補疏只等叛,執招討使耶律斡里朵,都監蕭斜裏得死之。



 秋七月,獵南山。金復遣烏林答贊謨來,責冊文無「兄事」



 之語,不言「大金」而云「東懷」,乃小邦懷其德之義;及冊文有「渠材」二字,語涉輕侮;若「遙芬多戰」等
 語,皆非善意,殊乖體式。如依前書所定,然後可從。楊詢卿、羅子韋率眾降金。



 八月,以趙王習泥烈為西京留守。



 九月,至西京。復遣習泥烈、楊立忠先持冊槁使金。



 冬十月甲戌朔,耶律陳圖奴等二十餘人謀反,伏誅。是月,遣使送烏林答贊謨持書以還。



 十年春二月,幸鴛鴦濼。金復遣烏林答贊謨持書及冊文副本以來,仍責乞兵於高麗。三月己酉,民有群馬者,十取其一,給東路軍。庚申,以金人所定「大聖」二字,與先世稱號同,復遣習泥烈往議。金主怒,遂絕之。



 夏四月,獵胡土白山,聞金師再舉,耶律白斯不等選精兵三千以
 濟遼師。



 五月,金主親攻上京,克外乳,留守撻不也率從出降。



 六月乙酉,以北府宰相蕭乙薛為上京留守、知鹽鐵內省兩司、東北統軍司事。



 秋,獵沙嶺。



 冬,復至西京。



\end{pinyinscope}