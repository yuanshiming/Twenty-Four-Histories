\article{卷八十九 列傳第二十七}

\begin{pinyinscope}

 耶律斡特剌孩里竇景庸耶律引吉楊績趙徽王觀耶律喜孫耶律斡特剌,字乙辛隱,許國王寅底石六世孫。少不喜官祿,年四十一,始補本班郎君。時樞密使耶律乙辛擅權,讒害忠良,斡特剌恐禍及,深自抑畏。



 大康中,為宿直官,歷左、右護衛太保。大安元年,升燕王傅,徙左夷離畢。
 四年,改北院樞密副使。帝賜詩褒之,遷知北院樞密使事,賜翼聖佐義功臣。北阻卜酋長磨古斯叛,斡特剌率兵進討。會天大雪,敗磨古斯四別部,斬首千餘級,拜西北路招討,封漆水郡王,加賜宣力守正功臣。尋拜南府宰相。



 復討閘古胡里扒部,破之,召為契丹行宮都部署。



 先是,北、南府有訟,各州府得就按之;比歲,非奉樞密檄,不得鞫問,以故訟者稽留。斡特剌奏請如舊,從之。壽隆五年,復為西北路招討使,討耶睹刮部,俘斬甚眾,獲馬、駝、牛、羊各數萬。明年,擒磨古斯,加守太保,賜奉國匡化功臣。



 乾統初,乞致仕,不許,止罷招討。復兼南院樞密使,封混同
 郡王。遷北院樞密使,加守太師,賜推誠贊治功臣。致仕,薨,謚曰敬肅。



 孩里,字胡輦,回鶻人。其先在太祖時來貢,願留,因任用之。孩里,重熙間歷近侍長。清寧九年,討重元之亂有功,加金吾衛上將軍,賜平亂功臣。累遷殿前都點檢,以宿衛嚴肅稱。



 大康初,加守太子太保。二年,加同中書門下平章事。三年,改同知南院宣徽使事。會耶律乙辛出守中京,孩里入賀;及議復召,陳其不可。後乙辛再入樞府,出孩里為廣利軍節度使。



 及皇太子被誣,孩裏當連坐,有詔勿問。大安初,歷品達魯虢部節度使。壽隆五年,有
 疾,自言吾數已盡,卻醫藥,卒,年七十七。



 孩里信浮圖。清寧初,從上獵,墮馬,憤而復蘇。言始見二人引至一城,宮室宏敞,有衣絳袍人坐殿上,左右列侍,導孩裡升階。持牘者示之曰:「本取大腹骨欲,誤執汝。」牘上書「官至使相,壽七十七」。須臾還,擠之大壑而寤。道宗聞之,命書其事。後皆驗。



 竇景庸,中京人,中書令振之子。聰敏好學。清寧中,第進士,授秘書省校書郎,累遷少府少監。



 咸雍六年,授樞密直學士,尋知漢人行宮副部署事。大安初,遷南院樞密副使,監修國史,知樞密院事,賜同德功臣,封陳國公。有
 疾,表請致仕;不從,加太子太保,授武定軍節度使。審決冤滯,輕重得宜,以獄空聞。



 七年,拜中京留守。九年薨,謚曰肅憲。子瑜,三司副使。



 耶律引吉,字阿括,品部人。父雙古,鎮西邊二十餘年,治尚嚴肅,不殖貨利,時多稱之。



 引吉寅畏好義。以蔭補官,累遷東京副留守、北樞密院侍御。時肅革、蕭圖古辭等以玦見任,鬻爵納賄;引吉以直道處其間,無所阿唯。改客省使。時朝廷遣使括三京隱戶不得,以引吉代之,得數千餘戶。



 時昭懷太子知北南院事,選引吉為輔導。樞密使乙辛將傾太子,惡引吉在側,奏出之,為群牧林牙。
 大康元年,乙辛請賜牧地,引吉奏曰:「今牧地褊愜,畜不蕃息,豈可分賜臣下。」



 帝乃止。乙辛由是益嫉之,除懷德軍節度使,徙漠北猾水馬群太保,卒。



 楊績,良鄉人。太平十一年進士及第,累遷南院樞密副使。



 與杜防、韓知白等擅給進士堂貼,降長寧軍節度使,徙知琢州。



 清寧初,拜參知政事,兼同知樞密院事,為南府宰相。九年,聞重元亂,與姚景行勤王,上嘉之。十年,知興中府。成雍初,入知樞密院事。二年,乞致仕,不許,拜南院樞密使。



 帝以績舊臣,特詔燕見,論古今治亂,人臣邪正。帝曰:「方今群臣忠直,耶律搒、劉伸而已;然伸不及搒
 之剛介。」



 績拜駕曰:「何代無賢,世亂則獨善其身,主聖則兼濟天下。



 陛下銖分邪正,升黜分明,天下幸甚。」累表告歸,不許,封趙王。大康中,以例改王遼西。致仕,加守太保,薨。子貴忠,知興中府。



 趙徽,南京人。重熙五年,擢甲科,累遷大理正。



 清寧二年,銅州人妄毀三教,徽按鞫之,以狀聞,稱旨。



 歷煩劇,有能名。累遷翰林學士承旨。咸雍初,為度支使。三年,拜參知政事。出為武定軍節度使,及代,軍民請留。



 後同知樞密院事,兼南府宰相、門下侍郎、平章事。致仕,卒。追贈中書令,謚文憲。



 王觀,南京人。博學有才辯。重熙七年,中進士乙科。



 興宗崩,充夏國報哀使;還,除給事中。咸雍初,遷翰林學士。五年,兼乾文閣學士。七年,改南院樞密副使,賜國姓,參加政事,兼知南院樞密事。



 坐矯制修私第,削爵為民,卒。



 耶律喜孫,字盈穩,永興宮分人。興宗在青宮,嘗居左右輔導。聖宗大漸,喜孫與馮家奴告仁德皇后同宰相蕭浞卜等謀逆事。及欽哀為皇太后稱制,喜孫憂見龐任。



 重熙中,其子涅哥為近侍,坐事伏誅。帝以喜孫有翼戴功,且悼其子罪死,欲世其官,喜孫無所出之部,因見馬印文有品部號,使隸其部,拜南府宰相。尋出為東北路
 詳穩,卒。



 論曰:「孩里、引吉之為臣也,當乙辛擅權、蕭革貪默之日,雖與同官,而能以正自處,不少阿唯,其過人遠矣!傳曰:『歲寒知松柏之後凋。』二子有焉。若斡特剌之戰功,竇景庸之讞獄,楊績之忠告,亦賢矣夫。」



\end{pinyinscope}