\article{列傳第二十}

\begin{pinyinscope}

 ○郭藥師子安國耶律塗山烏延胡里改烏延吾里補蕭恭完顏習不主紇石烈胡剌耶律恕郭企忠烏孫訛論顏盞門都僕散渾坦鄭建充烏古論三合
 移剌溫蕭仲恭子拱蕭仲宣高松海陵諸子



 郭藥師,渤海鐵州人也。遼國募遼東人為兵,使報怨于女直,號曰「怨軍」,藥師為其渠帥。斡魯古攻顯州,敗藥師於城下。遼帝亡保天德,耶律捏里自立,改「怨軍」為「常勝軍」,擢藥師諸衛上將軍。捏里死,其妻蕭妃稱制,藥師以涿、易二州歸于宋。藥師以宋兵六千人奄至燕京,甄五
 臣以五千人奪迎春門,皆入城。蕭妃令閉城門與宋兵巷戰。藥師大敗,失馬步走,踰城以免。宋人猶厚賞之。



 太祖割燕山六州與宋人,宋使藥師副王安中守燕山。及安中不能庇張覺而殺之,函其首以與宗望,藥師深尤宋人,而無自固之志矣。宗望軍至三河,藥師等拒戰於白河。兵敗,藥師乃降。宗望遂取燕山。太宗以藥師為燕京留守,給以金牌,賜姓完顏氏。從宗望伐宋,凡宋事虛實,藥師盡知之。宗望能以懸軍深入,駐兵汴城下,約質納幣,割地全勝以歸者,藥師能測宋人之情,中其肯綮故也。及兩鎮不受約束,命諸將討之,藥師破順安軍營,
 殺三千餘人。海陵即位,詔賜諸姓者皆復本姓,故藥師子安國仍姓郭氏。



 郭安國,藥師子也。累遷奉國上將軍、南京副留守。貞元三年,南京大內火,海陵使右司郎中梁金求、同知安武軍節度事王全按問失火狀。留守馮長寧、都轉運使左瀛各杖一百,除名。安國及留守判官大良順各杖八十,削三官。火起處勾當官南京兵馬都指揮使吳濬杖一百五十,除名。失火位押宿兵吏十三人並斬。諭之曰:「朕非以宮闕壯麗也。自即位以來,欲巡省河南,汝等不知防慎,致外方姦細,燒延殆盡。本欲處爾等死罪,特以舊人
 寬貸之。押宿人兵法當處死,疑此輩容隱姦細,故皆斬也。」



 安國性輕躁,本無方略。海陵將伐宋,以安國將家子,擢拜兵部尚書,改刑部尚書。軍興,領武捷軍都總管,與武勝、武平軍為前鋒。海陵授諸將方略,安國前奏曰:「趙構聞王師至,其勢必逃竄。臣等不以遠近,追之獲而後已,但置之何地?」海陵大喜曰:「卿言是也。得構即置之寺觀,嚴兵守之。」及聞世宗即位,海陵謀北還,更置浙西道兵馬都統制府,以完顏元宜為都統制,安國副之。及海陵遇弒,眾惡安國所為,與李通輩皆殺之。



 贊曰:郭藥師者,遼之餘孽,宋之厲階,金之功臣也。以一
 臣之身而為三國之禍福,如是其不侔也。魏公叔痤勸其君殺衛鞅,豈無所見歟!



 耶律塗山,系出遙輦氏,在遼世為顯族。塗山仕至金吾衛大將軍、遙里相溫。遼帝奔天德,塗山以所部降,宗翰承制授尚書,為西北路招討使。宗翰伐宋,塗山率本部為先鋒。至汾州,遇宋將折家軍,請濟師併力破之。從攻太原、隆德府,從入汴,克洛陽。及從婁室平陜右。天會七年,授太子少保。十年,遷尚書左僕射。致仕,卒,年九十一。正隆例贈特進、郜國公。



 烏延胡里改,曷懶路星顯水人也。後授愛也窟謀克,因
 家焉。從闍母圍平州,有功。及伐宋,圍汴,五謀克與宋兵萬人遇於城南,胡里改先馳擊敗之,元帥府遂賞良馬一匹。天會五年,攻宗城縣,敵棄城走恩州,胡里改追殺千餘人,獲車四百兩。帥府賞牛三十頭、馬一匹。七年,討泰山群盜,平之,毀其營柵。兗州群寇三千餘保據山險,胡里改復破之。賞牛二十二頭、馬四匹。八年,攻廬州,至柘皋鎮,胡里改領甲士三十為前鋒,執宋所遣持書與劉四廂錡者七人。復以先鋒軍攻和州,比至含山縣五里,獲甲士二人,乃知宋三將將兵且至。胡里改伏其軍,遂獲姚觀察。帥府賞馬二匹。九年,定陜右,胡里改以所部遇
 敵千人,敗之,生擒甲士一人,盡得敵之虛實。又從蒲魯渾徇地熙秦,敗敵兵二千於秦州,賞馬一匹。宋人屯襄陽府,監軍按補遣胡里改領四猛安往攻之。宋兵三千已渡江,方營壁壘,乘其未就,突戰破之。梁王宗弼復河南,將攻陳州,遣胡里改以甲士三十捕偵候人。至蔡州西,遇兵八十餘,戰敗之,獲南頓縣令。及攻陳州,夜將四更,忽聞敵開門潰走,胡里改亟領二謀克軍追及之,而猛安突葛速亦領軍繼至,大敗之。皇統二年,遷定遠大將軍。八年,授臨洮少尹,兼熙秦路兵馬副都總管。九年,改同知京兆尹,兼本路兵馬都總管。天德改同知平陽
 尹,兼河東南路兵馬都總管。貞元三年,改同知曷懶路總管。大定四年,授胡里改節度使。七年,改歸德軍節度使。十年,移鎮顯德。卒官,年六十九。十九年,詔授其子五十六武功將軍,世襲本路婆朵火河謀克。



 烏延吾里補,曷懶路禪嶺人也。徙大名路。天會中,從其父達吉補隸元帥右監軍麾下。撻懶以事赴闕,以達吉補自隨。吾里補領其父謀克,從大軍攻滄州。方夷濠隍,城中兵來拒,吾里補以本部擊卻之。王師下青州,力戰有功,獲馬百匹以獻,降獲賊黨甚眾。青州戍將覿吉補以萊州兵眾,請濟於帥府。吾里補將十二謀克兵往救
 之。遂降其四營,拔其一營,得戶四千。又敗賊兵五萬於恩州,攻破其營,降戶五萬,獲牛畜萬餘。將至臨清縣,遇敵兵三千,又敗之,俘獲甚眾,生擒賊首以獻。帥府嘉其功,以奴婢百、牛三十賞之。時覿吉補敗于恩州之境,吾里補復以兵四千往救之,破敵萬餘。宋兵十萬在單父間,總管宗室移剌屋選步卒一萬、騎兵四千往討之。吾里補領其親管謀克以從,遇敵先登,力戰有功。大軍經略密州,吾里補將兵二千為前鋒,遇敵萬人於高密,遂敗其眾,追至城下,殺戮殆盡,獲馬牛三千餘。吾里補與孛太欲敗賊王義軍十餘萬于州南。是夜,賊兵數千來
 襲營,吾里補以兵橫擊走之。後從大軍攻楚、揚、通、泰等州。天眷二年,襲其父世襲猛安,授寧遠大將軍。皇統七年,益以親管謀克。天德三年,除同知歸德尹。正隆初,為唐古部族節度使。大定二年,為保大軍節度使。是歲改鎮通遠。是時,宋軍十萬餘入河、隴,據險要,攻郡邑。元帥左都監合喜奏益兵。詔益兵七千,遣吾里補與彰化軍節度使宗室璋等七人偕往,以備任使。進階龍虎衛上將軍。卒于軍中。



 蕭恭,字敬之,乃烈奚王之後也。父翊,天輔間歸朝,從攻興中,遂以為興中尹。師還,以恭為質子。宗望伐宋,翊當
 領建、興、成、川、懿五州兵為萬戶,軍帥以恭材勇,使代其父行,時年二十三。至中山,宋兵出戰,恭先以所部擊敗之。經山東,及渡淮,襲康王,皆在軍中。師還,帥府承制授德州防禦使,奚人之屯濱、棣間者皆隸焉。改棣州防禦使。皇統間,改同知橫海軍節度使。丁父憂,起復為太原少尹。用廉,遷同知中京留守事。累遷兵部侍郎,授世襲謀克。坐問禁中起居狀,決杖,奪一官。貞元二年,為同知大興尹。歲餘,遷兵部尚書,為宋國生日使。以母憂去官,起復為侍衛親軍馬步軍都指揮使。正隆四年,遷光祿大夫,復為兵部尚書。是歲,經畫夏國邊界,還過臨潼,失所佩
 金牌。至太原,憂恚成疾。時已具其事驛聞於朝,海陵復命給之,仍遣諭恭曰:「汝失信牌,亦猶不謹。朕方俟汝,欲有委使,乃稱疾耶?必以去日身佩信牌,歸則無以為辭,欲朕先知耳。」使至,恭已疾篤,稽顙受命,俄頃而卒。海陵方遣使與其子護衛九哥馳視,乃戒府官使善護之。至保州,已聞訃矣,海陵深悼惜之。命九哥護喪以還,所過州府設奠。喪至都,命百官致祭。親臨奠,賻贈甚厚,並賜廄馬一。謂九哥曰:「爾父銜命,卒於道途,甚可悼惜。朕乘此馬十年,今賜汝父,可常控至柩前。既葬,汝則乘之。」



 完顏習不主,年十六,從伐宋,攻下懷仁縣,功居最。從睿
 宗經略陜西,以兵七百人入丹州諸山,遇盜三千,擊敗之。又破賊四千,生擒其將帥。出隴州,以兵四百敗敵數千。宋兵七千來取鞏州,復擊走之。又以五千兵敗吳玠之眾三萬,白塔口遇敵五千,復敗之。別降定遠等寨。皇統二年,授同知臨洮尹,以憂去官。未期,以舊職起復,改孟州防禦使,遷臨洮尹。復以罪罷。正隆三年,起為京兆尹,改河南尹。卒,年五十八。



 紇石烈胡剌,晦發川唵敦河人,徙西北路。識契丹字,為帥府小吏。梁王宗弼復陜西,久不通問。睿宗在燕京,遣胡剌往候之。是時,宗弼自鳳翔攻和尚原,使胡剌視彼
 中地形,修道築城。天會十二年,往濱州密訪南邊事體,及觀劉豫治齊狀,盡得其虛實。睿宗甚嘉之。皇統初,從宗弼渡淮,及下廬、和二州,大破張浚、韓世忠等軍。遣胡剌馳奏,賞以金盂、重彩五端、絹五匹。七年,授同知景州軍州事,以廉,加忠武校尉。天德初,以監察御史分司行臺,歷同知濟州防禦使事,入為監察御史。秩滿再任。大定二年,遷刑部員外郎,與御史大夫白彥敬往西北部族市馬。累轉泗州防禦使,三遷蒲與路節度使,移寧昌軍,卒。



 耶律恕,字忠厚,本名耨里,遼橫帳秦王之族也。為人謹願有志,喜讀書,通契丹大小字。與耶律高八來歸。婁室
 問高八曰:「與爾同來者,誰可任用治軍旅事?」高八封曰:「耨里可。」婁室與宗翰伐宋,恕隸前鋒,取和尚原,攻仙人關,特為睿宗所知,再除太原、真定少尹。撒離喝辟署陜西參謀,委以軍務,遷行臺兵部侍郎,再遷尚書左司郎中。海陵為平章政事,謂恕曰:「君亦有黨乎?」恕正色曰:「窮則獨善其身,達則兼善天下。不以其道得之,非恕之志也,何朋黨之有!」海陵徐曰:「前言戲之耳。」久之,為沁南軍節度使,遷行臺工部尚書。行臺罷,改安國軍節度使,為參知政事。以疾求解,為興中尹,入為太子少保。正隆元年致仕。封廣平郡王。薨,年六十九。二年,例贈銀青光祿
 大夫。



 郭企忠,字元弼,唐汾陽王子儀之後。郭氏自子儀至承勳,皆節鎮北方。唐季,承勳入於遼,子孫繼為天德軍節度使,至昌金降為副使。企忠幼孤,事母孝謹。年十三,居母喪,哀毀如成人。服除,襲父官,加左散騎常侍。天輔中,大軍至雲中,遣耶律坦招撫諸部。企忠來降。軍帥命同勾當天德軍節度使事,徙所部居于韓州。及見太祖,問知其家世,禮遇優厚,以白鷹賜之。天會三年,伐宋,領西南諸部番、漢軍兵,為猛安,從破雁門,屯兵,加桂州管內觀察留後,鎮代州。明年,賊楊麻胡等聚眾數千于五臺,
 企忠與同知州事迪里討平之。遷知汾州事。是時汾州初下,居民多為軍士掠去,城邑蕭然。企忠詣帥府力請,願聽其親舊贖還。帥府從之。未幾,完實如故。石州賊閻先生眾數萬至城下,僚屬慮有內變,請為備。企忠曰:「吾於汾人有德,保無他。」乃率吏民城守。會援至,合擊,破之。六年,改靜江軍節度留後,遷天德軍節度使、汴京步軍都指揮使,累遷金吾衛上將軍。秩滿,權沁州刺史。到官歲餘,卒,年六十八。



 烏孫訛論,善騎射,襲父撒改謀克,從蒙刮攻東京及廣寧,擊北京山賊,皆有功。蕭霸哲來攻恩州,訛論以六十
 騎偵之。逮夜,遇敵數百騎,掩擊之,生獲三人,知霸哲眾九萬且至,故蒙刮得以為備,遂破霸哲。宗望伐宋,已至汴,訛論破尉氏、中牟援兵,取其城。久之,以兵百五十人破敵一千於滄州西。明年,再伐宋。蒙刮戍開州,訛論以騎四百守河,復敗千餘人,斬首七百餘。宗弼渡淮,阿里先具舟于江上,聞王善兵扼其前。宗弼使訛論濟師敗王善於和州北。李成以兵七萬據烏江,訛論帥二千人直前敗之。宗弼遂渡江至江寧。十五年,沂州竇防禦叛。訛論敗之,獲竇防禦。錄前後功,授猛安,加昭武大將軍。宗弼再取河南,訛論以五十騎敗楊家賊五百於徐州
 東。以功受賞,不可勝計。天德二年,除唐州刺史,移淄州,遷石壘部族節度使。行至北京,病卒。



 顏盞門都,隆州帕里乾山人也。身長,美須髯。天會間,從其兄羊艾在軍中。方取汴京,其兄戰歿,遂擐甲代其兄充軍。睿宗定陜右,以門都為蒲輦,隸監軍杲親管萬戶,攻饒風關。至坊州,杲欲與總管蒲魯虎會於鳳翔,遣門都領六十騎先往期會。及還,備得地形險阨,賞銀五十兩。其後梁王宗弼駐軍山東,遣人詣陜西,特召門都至。令齎廢齊及安撫百姓詔書,往諭監軍宗室杲。門都既還,宗弼賞以良馬銀絹。事畢,復遣從杲。天眷初,叛將定
 國軍節度使李世輔偽邀杲至私署,以獻甲為名,遂以兵劫執而去。門都突出,以告押軍猛安完顏撻懶,同率兵追及,首出與戰,杲由此得脫,以功遷明威將軍。復從杲招復陜西,進至鳳翔。齊國初廢,諸路多反復不一。杲授門都牌札,令往撫定。門都所至,多張甲兵,從者安之,違者討之,帖然無復叛者,杲甚嘉之。皇統初,遷廣威將軍。四年,授同知通遠軍節度使事,改知保安軍事。天德三年,為丹州刺史兼知軍事。正隆初,為寧州刺史。大定初,宋將吳璘等以軍數十萬人據秦、隴,元帥府承制以門都為勇烈軍都總管,領軍討之。宋人保據德順。都監
 合喜遣武威軍副都總管夾谷查剌,會宗室璋,議征討之策。璋與門都曰:「須都監親至,敵必退矣。」合喜領軍四萬來赴,遂復德順州。明年,秦、隴平,以功遷金吾衛上將軍,授通遠軍節度使。五年,改慶陽尹,兼本路兵馬都總管,卒于官。十九年,錄功,以子六哥世襲本路曷懶兀主猛安敵骨論窟申謀克,授武功將軍。



 門都性忠厚謹愨,安置營壁,尤能慎密。有敵忽來,雖矢石至前,泰然自若,迺號令士卒如平時,由是人益安附,而功易成焉。



 僕散渾坦,蒲與路挾懣人也。身長七尺,勇健有力,善騎射。年十六,從其父胡沒速征伐。初授脩武校尉,為宗弼
 扎也。天眷二年,與宋岳飛相拒。渾坦領六十騎,深入覘伺,至鄢陵,敗宋護糧餉軍七百餘人,多所俘獲。皇統九年,除慈州刺史,再遷利涉軍節度使,授世襲濟州和術海鸞猛安涉里斡設謀克。貞元初,以憂去官。起復舊職,歷泰寧、永定軍,改咸平尹。海陵殺渾坦弟樞密使忽土,召渾坦至南京。既見,沈思久之,謂之曰:「汝有功舊,不因忽土得官,以此致罪,甚可矜憫。」遂釋之。改興平軍節度使。世宗即位,以為廣寧尹。窩斡反,為行軍都統,與曷懶路總管徒單克寧俱在左翼,敗窩斡於長濼。改臨潢尹。賊平,賜金帛。改曷懶路兵馬都總管。徙顯德軍、慶陽尹。
 致仕。大定十二年,上思舊功,起為利涉軍節度使,復以金紫光祿大夫致仕。卒,年七十二。



 渾坦歷一十七官,未嘗為佐貳。性沈厚有識,雖未嘗學問,明於聽斷,所至有治聲云。



 鄭建充,字仲實,其先京兆人,占籍鄜州。仕宋,累官知延安府事。天會七年來降,仍知延安府,屯兵三千。宋劉光烈兵八萬來攻建充,相距四十餘日。攻益急,建充遣人會斜喝軍,夾擊破之,俘其裨將賀貴。遷節制司統制軍馬。改京兆府路兵馬都監。敗宋曲端於彭原。高昌宗據延安,為宋守,建充擊之,盡復城邑。復知延安軍府事。齊
 國建,累遷博州團練使,知寧州。齊國廢,朝廷以地賜宋,為宋環慶路經略安撫副使,仍知寧州。天眷復取陜西,仍以為經略安撫使,知慶陽。從破甘谷城,改平涼尹。是時營建南京宮室,大發河東、陜西材木,浮河而下,經砥柱之險,筏工多沉溺,有司不敢以聞,乃誣以逃亡,錮其家。建充白其事,請至砥柱解筏,順流散下,令善游者下流接出之,而錮者得釋。正隆軍興,括筋角造軍器,百姓往往椎牛取之,或生拔取其角,牛有泣下者。建充白其事於朝。



 建充性剛暴,常畜猘犬十數,奴僕有罪既笞,已復嗾犬嚙之,骨肉都盡。雖謙遜下士,於敵己上一無所
 屈。省部文移有不應法度,輒置之坐下,或即毀裂,由是在位者銜之。軍胥李換竊用公帑,自度不得免,乃誣建充藏甲欲反,更再鞫,皆無狀。方奏上,攝事者素與建充有隙,恐其得釋,使吏持文書紿建充曰:「朝省有命,奈何?」建充曰:「惟汝所為。」是夜,死於獄中。長子愬亦死焉。



 烏古論三合,曷懶路愛也窟河人,後徙真定。睿宗為右副元帥,聞三合勇略,選充扎也。後從宗弼征伐,補曲院都監。未幾,從伐宋。與宋兵遇於潁州,三合先登破之。皇統元年,領漢軍千戶,帥府再以軍四千隸焉。除同知鄭州防禦使事,再遷太子少詹事。大定六年,改洺州防禦
 使。上曰:「卿昔事睿宗,積勞苦。逮事朕,輔佐太子,宣力多矣。今典名郡,所以勞卿也。」遷永定軍節度使,歷臨潢、鳳翔尹,陜西路統軍使,東平尹。節制州郡,躬行儉約,政先寬簡,邊庭久寧,人民獲安。召為簽書樞密院事。卒。



 十八年,世宗追錄三合舊勞,授其子大興河北西路愛也窟河世襲猛安阿里門河謀克,階武功將軍。



 移剌溫本名阿撒,遼橫帳人,工契丹小字。睿宗為左副元帥伐宋,溫從大抃渡江,辟江寧府都巡檢。江寧、太平初下,宋遣諜人扇構百姓,應者數萬人。溫擒其諜者,遂不敢竊發。宗弼嘉之,賜銀千兩、重綵百端、絹二百匹。宗
 弼每出征伐,未嘗不在行間。除同知河北西路轉運使事。會宗弼巡邊,溫從軍,不之官。宗弼入朝,熙宗宴群臣,宗弼欲有奏請,已被酒失次,溫掖而出宮。明日,熙宗謂宗弼曰:「阿撒事叔甚謹,不可去左右。」由是宗弼益親信之。嘗謂女婿紇石烈志寧曰:「汝可效阿撒之為人也,可以幾古人矣。」未幾,除同知中京路都轉運使事,累遷左諫議大夫兼脩起居注。正隆伐宋,以本官為濟州路行軍萬戶,從至揚州。軍還,除同知宣徽院事。世宗御饌不適口,召溫嘗之。奏曰:「味非不美也,蓋南北邊事未息,聖慮有所在耳。」上意遂釋。歷永定、震武、崇義節度使,移臨
 海軍。州治近水,秋雨,水潦暴至城下,城頗決,百姓惶駭,不知所為。溫躬督役夫繕完之,雖臨不測,無所避。僚屬或止溫,溫曰:「為政疵癘,水泛溢為災,守臣之罪。當以此身為百姓謝,雖死不恨。」移鎮武定,歲旱且蝗,溫割指,以血瀝酒中,禱而酹之。既而雨霑足,有群鴉啄蝗且盡,由是歲熟,人以為至誠之感云。以老致仕,卒。



 贊曰:軍旅之事,鋒鏑在前,不計其死。耳屬金鼓,目屬旌旗,心屬號令,此行列之任也。自收國用兵,至於大定和宋以前,用命之士,雖細必錄,所以明功也。



 蕭仲恭,本名術里者。祖撻不也,仕遼為樞密使,守司徒,
 封蘭陵郡王。父特末,為中書令,守司空,尚主。仲恭性恭謹,動有禮節,能被甲超橐駝。遼故事,宗戚子弟別為一班,號「孩兒班」,仲恭嘗為班使,歷宮使、本班詳穩。遼帝西奔天德,仲恭為護衛太保,兼領軍事。至霍里底泊,大軍奄至,倉卒走。仲恭母馬乏,不能進,謂仲恭兄弟曰:「汝等盡節國家,無以我為也。」仲恭母,遼道宗季女也。遼主傷之,命弟仲宣留侍其母。仲恭從而西。時大雪,寒甚,遼主乏食,仲恭進衣並進乾Я。遼主困,仲恭伏冰雪中,遼主藉之以憩。凡六日,乃至天德,始得食。後與遼主俱獲,太宗以仲恭忠於其主,特加禮待。天會四年,仲恭使宋。且
 還,宋人意仲恭、耶律餘睹皆有亡國之戚,而余睹為監軍,有兵權,可誘而用之,乃以蠟丸書令仲恭致之餘睹,使為內應。仲恭素忠信,無反覆志,但恐宋人留不遣,遂陽許。還見宗望,即以蠟丸書獻之。宗望察仲恭無他,薄罰之。於是再舉伐宋,執二帝以歸。累遷右宣徽使,改都點檢。宗磐與宗乾爭辯於熙宗前,宗磐拔刀向宗乾,仲恭呵之乃止。既而宗磐以反罪誅,仲恭衛禁有備,以功加銀青光祿大夫,遷尚書右丞。皇統初,封蘭陵郡王,授世襲猛安,進拜平章政事,同監修國史,封濟王。詔葬遼豫王於廣寧,仲恭請往會葬,熙宗義而許之。改行臺左
 丞相。居無何,入為尚書右丞相,拜太傅,領三省事,封曹王。天德二年,封越國王,除燕京留守。海陵親為書,以玉山子賜之。是歲,薨,年六十一。謚貞簡。正隆例降王爵,改儀同三司、鄭國公。子拱。



 拱本名迪輦阿不,初為蘭子山猛安。海陵為宰相,徼取人譽,薦大臣子以為達官,遂以拱為禮部侍郎。耶律彌勒,拱妻女弟也,海陵將納為妃,使拱自汴取之。還過燕,是時仲恭為燕京留守,見彌勒身形不類處子,竊憂之,曰:「上多猜嫌,拱其及禍矣。」拱去不數日,仲恭卒。拱至上京,聞訃,以本官起復,佩信牌,往燕京治葬事。未行,彌勒
 入宮,果如仲恭所相度,即遣出宮。夜半召拱至禁中,詰問無狀。海陵終疑之,乃罷拱禮部侍郎,奪其信牌。拱待命,踰年不報,歸蘭子山治猛安事。是時,蕭恭、張九坐語禁中事得罪,拱至蘭子山,與客會語及之。有阿納與拱有隙,乃誣拱言張九無罪被誅,語涉怨謗。海陵遣使鞫之,戒使者曰:「此子狂妄,宜有此語,不然彼中安得知此事。」使者不復問拱,但榜掠其左驗,使如告語證之,拱遂見殺。



 仲宣,本名野里補,仲恭母弟。聰敏好學,沉厚少言。五歲,遙授郡刺史,累加太子少師,為本班詳穩。從天祚西,為
 護衛太保左右班詳穩。至石輦鐸,遼主留仲宣侍母,遂與其母皆見獲。太宗嘉之,且謂仲宣能知遼國故事,命權宣徽使,從睿宗伐康王。師還,家居者久之。皇統二年,特授鎮國上將軍,歷順義、永定、昭義、武寧四鎮節度使。為政平易,小吏不敢為姦。賄賂禁絕,奴婢入郡,人莫識其面。朔、潞百姓皆為立祠刻石頌之。正隆二年,卒,年六十四。



 高松,本名檀朵,澄州析木人。年十九,從軍為蒲輦,有力善戰,宗弼聞其名,召置左右,從破汴京及和尚原,累官咸平總管府判官。世宗即位,充管押東京路渤海萬戶。
 兵部尚書可喜謀反,前同知延安尹李老僧曰:「我與萬戶高松謀之,必從我矣。」眾曰:「若得此軍,舉事易矣。」老僧往見松,說松曰:「君有功舊人,至今不得大官,何也?」松曰:「我一縣令也,每念聖恩,累世不能報,尚敢有望乎!」老僧遂不敢言。可喜、布輝、阿瑣知事不可成,遂上變,共捕斡論赴有司。松從征窩斡,以功遷咸平少尹,四遷崇義軍節度使。卒,年七十四。



 贊曰:忠信行己,豈不大哉!蕭仲恭盡心故主,而富貴福澤嚮之,與宗室舊臣等矣。仲恭廷叱宗磐而朝廷尊,高松誼遏李老僧而社稷安,皆有古烈丈夫之風焉。



 海陵后徒單氏生太子光英,元妃大氏生崇王元壽,柔妃唐括氏生宿王矧思阿補,才人南氏生滕王廣陽。



 光英本名阿魯補,徒單后所生。是時燕京轉運使趙襲慶多男,故又名曰趙六。養于同判大宗正方之家,故崇德大夫沈璋妻張氏嘗為光英保母,於是贈璋銀青光祿大夫,賜宗正方錢千萬。



 天德四年二月,立光英為皇太子。是月,安置太祖畫像于武德殿,盡召國初嘗從太祖破寧江州有功者,得百七十六人,並加宣武將軍,賜酒帛。其中有忽里罕者,解其衣進光英曰:「臣今年百歲矣,有子十人。願太子壽考多男子與小臣等。」海陵使光
 英受其衣,海陵即以所服并佩刀賜忽里罕,答其厚意。後以「英」字與「鷹隼」字聲相近,改「鷹坊」為「馴鷙坊」。國號有「英國」又有「應國」,遂改「英國」為「壽國」,「應國」為「杞國」。宋亦改「光州」為「蔣州」,「光山縣」為「期思縣」,「光化軍」為「通化軍」云。



 太醫院保全郎李中、保和大夫薛遵義俱以醫藥侍光英,李中超換宣武將軍、太子左衛副率,薛遵義丁憂,起復宣武將軍、太子右衛副率。光英襁褓時,養於宗正方家,其後養于永寧宮及徒單斜也家。貞元元年,詔朝官,京官五品以下奉引自通天門入,居于東宮。



 正隆元年三月二十七日,光英生日,宴百官于神龍殿,賜京師大酺
 一日。四年八月,光英射鴉,獲之。海陵大喜,命薦原廟,賜光英馬一匹,黃金三斤,班賜從者有差。正隆六年,海陵行幸南京,次安肅州。光英獲二兔,遣使薦于山陵。居數日,復獲麞兔,從官皆稱賀。賜光英名馬弓矢,復遣使薦于山陵。六月,海陵至南京,群臣迎謁,海陵與徒單后、光英共載而入。



 海陵嘗言:「俟太子年十八,以天下付之。朕當日遊宴於宮掖苑囿中以自娛樂。」光英頗警悟,海陵謂侍臣曰:「上智不學而能,中性未有不由學而成者。太子宜擇碩德宿學之士,使輔導之,庶知古今,防過失。詩文小技,何必作耶。至於騎射之事,亦不可不習,恐其懦
 柔也。」及將親征,后與光英挽衣號慟,海陵亦泣下曰:「吾行歸矣。」



 後誦《孝經》。一日,忽謂人曰:「《經》言三千之罪,莫大於不孝,何為不孝?」對者曰:「今民家子博弈飲酒,不養父母,皆不孝也。」光英默然良久,曰:「此豈足為不孝耶!」蓋指言海陵弒母事。



 及伐宋,光英居守,以陀滿訛里也為太子少師兼河南路統軍使,以衛護之。完顏元宜軍變,海陵遇害,都督府移文訛里也,殺光英於汴京,死時年十二。後與海陵俱葬于大房山諸王墓次。



 訛里也,咸平路窟吐忽河人,襲其父忽土猛安。除邳州刺史,三遷昌武軍節度使、歸德尹、南京留守、河南路統軍使、太子少師。
 大定二年,遷元帥右都監。宋人陷陳、蔡,訛里也師久無功,已而兵敗于宋,解職。俄起為京兆尹。世宗謂之曰:「卿為河南統軍,門多私謁,百姓惡之。其後經略陳、蔡,不惟無功,且復致敗。以汝舊勞,故復用汝。京兆地近南邊,宜善理之。」大定三年,卒。



 元壽,天德元年封崇王。三年,薨。



 矧思阿補,正隆元年四月生。小底東勝家保養之,賜東勝錢千萬,仍為起第。五月已酉,彌月,封其母唐括氏為柔妃,賜京師貧者五千人錢,人錢二百。二年,矧思阿補生日,海陵與永壽太后及皇后、太子光英幸東勝家。三
 年正月五日,矧思阿補薨。海陵殺太醫副使謝友正、醫者安宗義及其乳母,杖東勝一百,除名。明日,追封矧思阿補為宿王,葬大房山。



 諫議大夫楊伯雄入直禁中,因與同直者相語,伯雄曰:「宿王之死,蓋養於宮外,供護雖謹,不若父母膝下。豈國家風俗素尚如此。」或以此言告海陵。海陵大怒,謂伯雄曰:「爾臣子也,君父所為,豈得言風俗。宮禁中事,豈爾當言。朕或體中不佳,間不視朝,只是少得人幾拜耳。而庶事皆奏決便殿,縱有死刑不即論決,蓋使囚者得緩其死。至於除授宣敕雖復稽緩,有何利害。朕每當閒暇,頗閱教坊聲樂,聊以自娛。《書》云:『內
 作色荒,外作禽荒,酣酒嗜音,峻宇雕墻,有一於此,未或不亡。』此戒人君不恤國事溺於此者耳。如我雖使聲樂喧動天地,宰相敢有濫與人官而吏敢有受賕者乎。外間敢有竊議者乎。爾諫官也,有可言之事,當公言之。言而不從,朕之非也。而乃私議,可乎?」伯雄對曰:「陛下至德明聖,固無竊議者。愚臣失言,罪當萬死,惟陛下哀憐。」海陵曰:「本欲殺汝,今只杖汝二百。」既決杖至四十,使近臣傳詔諭伯雄曰:「以爾籓邸有舊,今特釋之。」



 滕王廣陽,母南氏,本大抃家婢,隨元妃大氏入宮,海陵幸之,及有娠,即命為殿直。正隆二年九月二十六日,生
 廣陽。十月滿月,海陵分施在京貧民,凡用錢千貫。三年二月,封南氏為才人。七月,封廣陽為滕王。九月,薨。



 贊曰:海陵伐宋,光英居守,使陀滿訛里也以宮師兼統軍之任,計至悉也,豈料死其手乎。荀首有言:「不以人子,吾子其可得耶?」海陵睨人之子不翅魚肉,而獨己子之謀安,不可得矣。



\end{pinyinscope}