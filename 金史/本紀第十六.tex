\article{本紀第十六}

\begin{pinyinscope}

 宣宗下



 四年春正月壬辰朔,詔免朝。丙申,金安軍節度使行元帥府事古里甲古倫除名。丁酉,大元兵下好義堡,霍州刺史移剌阿里合等死之。詔贈官有差。庚戌,宋步騎十餘萬圍鄧州,聞援軍至,夜焚營去,招撫副使術虎移剌答追及之,奪其俘還。壬子,晝晦,有頃大雷電,雨以風。癸丑,戶部侍郎張師魯上書,請遣騎兵數千,及春,淮、蜀並
 進、以撓宋。丙辰,以武仙遙領中京留守,進官一階。



 三月辛丑,議遷睢州,治書侍御史蒲魯虎奉詔相視京東城池,還言勿遷便,乃止。癸卯,長春節,詔免朝。乙巳,林州元帥惟良擒叛人單仲、李俊,誅之,降其黨盧廣。己酉,以吏部尚書李復亨參知政事,南京兵馬使術甲賽也行懷、孟帥府事。辛亥,進平章政事高汝礪為尚書右丞相,監修國史,封壽國公。參知政事李復亨兼修國史。平章政事、陜西行尚書省胥鼎進封溫國公,致仕。壬子,紅襖賊於忙兒襲據海州,經略使完顏陳兒以兵擊敗忙兒,復取之。甲寅,木星犯鬼宿積尸氣。



 夏四月庚申朔,詔御史
 中丞完顏伯嘉提控防城事。癸亥,安武軍節度使柴茂破紅襖賊于棗強。祁州經略使段增順破叛賊甄全于唐縣。夏人犯邊,元帥石盞合喜破之。乙丑,以彰德、衛、輝、滑、濬諸州隸河南路轉運司。以河南路轉運司為都轉運,視中都,增置官吏。戊辰,禘于太廟。大元遣趙瑞以兵攻孟州。提控魯德、王安復大名府。以參知政事把胡魯權尚書右丞、左副元師,元帥左都監承立為右監軍權參知政事,同行尚書省元帥府于京兆。庚辰,東平元帥府總領提控蒲察山兒破紅襖賊於聊城。壬年,命六部檢法以法狀親白部官,聽其面議,大理寺如之。



 五月壬
 辰,定二品至三品立功遷官格。癸巳,紅襖賊寇樂陵、鹽山,橫海軍節度使王福連擊敗之,張聚來寇,又敗之。甲午,上擊鞠于臨武殿。丙申,以時暑,免常朝,四日一奏事。丁酉,諭工部暑月停工役。癸卯,大元兵徇庾州。丙辰,大元兵徇兗州,泰定軍節度使兀顏畏可死之。



 六月丙寅,遣人招張柔。丁卯,詔減監察御史四員。戊辰,山東民僑居者募壯士五百人,益東莒公燕寧軍。月犯土星。己巳,太白晝見于張,百八十有四日乃伏。甲戌,制諸倉場庫院巡護軍,受提舉倉場司及監支納官彈壓。京畿不雨,敕有司閱獄,雜犯死罪以下皆釋之。丁丑,大元遣楊在攻
 下大名,又攻開州及東明、長垣等縣。己卯,祈雨。庚辰,宋人方子忻來歸,有司處之鄭州。上曰:「吾民奔宋者,彼例衣食之。彼來歸者,不善視之,或復逃歸,漏泄機事。」命增子忻廩給,有司優遇之。元帥右監軍、權參知政事承立上封事。



 秋七月辛卯,宋人及紅襖賊犯河朔,諸郡皆降,獨滄州經略使王福固守。會益都賊張林來攻,福乃叛降林,帥府請討之。是日,雨。癸丑,林州行元帥府遣總領嚴祿等討紅襖賊于彰德府,生擒偽安撫使王九。詔參知政事李復亨為慰使,御史丞完顏伯嘉副之,循行郡縣勸農。以烏古論仲端等使大元。



 八月戊午朔,嚴
 實、成江、王贇據濟南,山東招撫高居實遣人招嚴實于青崖砦,獲其款以聞。李全犯東平府,監軍王庭玉敗之,擒其偽安化軍節度使張林。庚申,高陽公張甫請增兵守冀州。上諭樞密,潁州民渡淮為宋軍者凡十村,可追索主者,懲一二以試其餘。庚午,敕掌兵官不聽舉縣令。夏人陷會州,刺史烏古論世顯降。甲戌,陜西行省報龕谷敗夏人之捷。乙亥,上諭宰臣,河南水菑,唐、鄧尤甚。其被菑州縣,已除其租。餘順成之方,止責正供,和糴、雜徵並免。仍自今歲九月始,停周歲桑皮故紙折輸。流民佃荒田者如上優免。丙子,陜西行省與夏人議和。戊寅,定
 選補親軍法。己卯,罷葭州招撫司。壬午,陜西路行省承裔報定西州之捷。丙戌,以隨路諸軍戶徙河南、京東、西、南路,各設檢察使、副。恒山公武仙降大元。



 九月戊子,詔遣官于河南、陜西選親軍。辛卯,進《章宗實錄》。戊戌,大元木華里屯軍真定。置總領元帥府于歸德,以壽州、陳留兩鎮兵屬之。庚子,夏人入定西州。壬寅,宋人屯皂郊堡,行軍提控完顏益都擊敗之。大元遣塔忽等來。癸卯,夏人來侵。甲辰,滕州招捕提控夏義勇討紅襖賊,敗之。乙巳,詔參知政事李復亨提控芻糧事。己酉,夏人陷西寧州,尚書省都事僕散奴失不坐誅,駙馬都尉徒單壽春
 奪官一階,杖六十。癸丑,更定安泊逃亡出征軍人罪及捕獲賞格。甲寅,宋人出秦州,及夏人來侵。丙辰,鞏州行元帥府石盞合喜報定西州之捷。冬十月壬戌,大元遣蒙古塔忽、訛里剌等來。己卯,陜西東路行省報綏德州之捷。泗州元帥府言,紅襖賊一月四入寇,掠人畜而去。庚辰,上擊鞠于臨武殿。辛巳,授紅襖賊時青滕陽公、本處兵馬總領、元帥兼宣撫。癸未,京西山寨各設守禦使、副,令本路府總之。諭陜西行省圖復會州。上擊鞠于臨武殿。



 十一月丁亥朔,免越王永功朔望朝參。易水公靖安民為其下所殺。戊子,黃陵堈經略使烏古論石
 虎等以戰陣失律,伏誅。壬辰,木星晝見于翼,積六十有七日伏,夜又犯靈臺北第一星。甲午,河南水,遣官勸課。更浮山縣名忠孝。戊戌,詔復衛紹王王爵,仍加開府議同三司。壬寅,山東東路軍戶徙許州,命行東平總管府治之,判官一人分司臨潁。乙巳,詔柴茂權元帥左都監,蓋仁貴攝右都監,同行元帥府于真定。是月,大元木華里國王以兵圍東平。十二月甲戌,祈雪。禮部郎中權左司諫抹撚胡魯剌上封事。戊寅,詔軍官許月擊鞠者三次,以習武事。庚辰,獵,享于太廟。乙酉,鎮南軍節度使溫迪罕思敬上書言錢幣、稅賦二事。



 五年春正月丙戌朔,免朝。丁亥,世宗忌日,謁奠于啟慶宮。戊子,括南京諸州逋戶舊耕官田,給軍戶。壬辰,議禦西夏及征南事。諭皇太子以東平禦敵方略。甲午,諭樞密院,南伐事重,當詳議其便。撰故衛王事跡,如海陵庶人例。丁酉,大元兵攻天井關。戊戌,宋人襲泗州西城,提控王祿死之。辛丑,太白晝見于牛,二百三十有二日伏。乙巳,詔諸道兵集蔡州,己酉,伐宋。庚戌,山東行省報東平之捷。



 二月丙辰朔,置招撫司于單州。曲赦東平府。庚申,下詔伐宋。以內族惟弼權同簽樞密院事,行院于中京;斡勒合打權元帥府右都監,行元帥府于蔡、息;納合
 降福權簽樞密院事,行院于宿州;孛術魯達阿權元帥右都監,完顏訛論副右都監,行元帥府於唐、鄧。戊辰,罷懷州行元帥府,復置招撫司,與孟州經略司並受中京行樞密院節制。辛未,僕散安貞以元帥出息州,破宋人乾凈居山寺,拔黃土關。癸酉,以旱災,曲赦河南路。丙子,禁京城兵器。元帥紇石烈牙葉塔破宋兵,復泗州。進逼濠州,至渦口,乏糧而還西城。癸未,以旱災,詔中外。



 三月丙戌朔,上御仁安殿,祈雨,仍望祭于北郊。庚寅,宋人圍唐、鄧,行元帥府事完顏訛論力戰卻之。前鄧州千戶孛術魯毛良虎自拔歸國,訛論便宜遷其官三階,授同知唐
 州事,乞正授以示信,從之。乙未,罷河南路行三司。丙申,參知政事徒單思忠進尚書右丞、兼修國史,以太子詹事僕散毅夫為參知政事。諭宰臣曰:「今奉御、奉職多不留心采訪外事。聞章宗時近侍人秩滿,以所采事定升降。今亦宜預為攷核之法,以激勸之。」戊戌,長春節,免朝。己亥,夏因叛人竇趙兒之招,入據來羌城,孛術魯合住以重賞誘脅從人為內應,督兵急攻城,拔之。省試經義進士,考官於常額外多放喬松等十餘人。有司奏請駁放,上已允,尋復遣諭松等曰:「汝等中選而復黜,不能無動于心。方今久旱,恐傷和氣,今特恩放汝矣。」庚子,賜林
 州行元帥府經歷官康琚進士及第。琚以武階乞赴廷試,故有是命。丙午,以旱築壇祀雷雨師。壬子,雨。



 四月己未,山東行省蒙古綱言:「東莒公燕寧戰敗而死。寧所居天勝砦據險,寧亡,眾無所歸,變在朝夕。權署其提控孫邦佐為招撫使,黃摑兀也為總領,以撫其眾。」遣使請命,敕有司議之。辛酉,禱雨于太廟。丙寅,僕散安貞破宋黃、蘄等州。壬申,俘宋宗室男女七十餘口獻於京師。癸酉,詔親軍中武舉第而授職需次者,仍執舊役,廩給循常,闕至發遣。辛巳,監察御史劉從益以彈劾失當,奪官一階,罷之。詔定進士中下甲及監官散階至明威者舉充
 縣令法。



 五月甲申朔,日有食之。戊戌,宋人據楚丘,官軍復之。庚子,納蘭記僧伏誅,告人趙銳升職四等。壬寅,陜西元帥完顏賽不遣使來獻晉安、平陽之捷,方議其賞,御史烏古論胡魯劾其縱將士鹵掠,不副主上除亂救民之意,乞正其罪。上以賽不有功,詔勿問,賞議亦寢。癸卯,唐州守將訛論為元帥賽不猶子,與宋人戰唐州境上,為宋人所敗,死者七百餘人,匿之而以捷聞。御史納蘭發其事。上以賽不故,亦不之罪,而以是意諭之。乃稱納蘭敢言,錄其功付有司,秩滿考最。癸丑,東平內徙,命蒙古綱行省于邳州,王庭玉行帥府于黃陵堈。



 六月甲
 寅朔,尚書省奏駙馬都尉安貞反狀,上閱奏慮其不實,謂平章政事英王守純曰:「國家誅一大臣,必合天下後世公議。其令覆按之。」乙丑,遣使諭晉陽公郭文振、上黨公完顏開各守疆土,同心濟難,毋以細故啟釁端,誤國事。戊寅,僕散安貞坐謀反,并其三子皆伏誅。己卯,越王永功薨。庚辰,輟朝。壬午,上親奠于殯所。秋七月己亥,義勇軍叛,據碭山縣。庚子,詔增給徐州、清口等處戍兵衣糧。己酉,碭山賊夜襲永城縣,行軍副總領高琬敗之,命蒙古綱併力討捕。辛亥,單州招撫劉瓊乞移河南糧濟其軍,詔給之。



 八月壬子朔,罷黃陵堈招撫司。上諭尚書
 省,碭山叛軍家屬囚歸德,旬餘不給糧,恐傷其生。宰臣奏,已給之矣。又諭樞密,河北艱食,民欲南來者日益多,速令渡之,毋致殍死。癸亥,林、懷帥府邀擊紅襖賊于伏恩村,敗之。甲子,詔南征潰軍復歸而能力戰者,依出界立功格賞之。乙丑,宋人掠沈丘,殺縣令。甲戌,命有司除逋戶負租,毋徵見戶。



 九月甲申,以京東歲饑多盜,遣御史大夫紇石烈胡失門為宣慰使,往撫安之。更定監察御史違犯的決法。丁亥,詔州府及軍官捕盜慢職,四品以下宣慰使決之,三品以下上奏裁。戊子,增授隰州招撫使軒成官,改受陜西省節制。乙巳,崇進、駙馬都尉定國公
 徒單公弼薨。庚戌,歲星犯左執法。右丞相高汝礪表乞致仕,詔溫留之。冬十月癸丑,進汝礪官榮祿大夫。命僕散毅夫行尚書省于京東,督諸軍芻糧。乙卯,太醫侯濟、張子英治皇孫疾,用藥瞑眩,皇孫不能任,遂不療,罪當死。上曰:「濟等所犯誠宜死,然在諸叔及弟兄之子,便應準法行之,以朕孫故殺人,所不忍也。」命杖七十,除名。尚書省言:「司、縣官貪暴不法,部民逃亡,既有決罰,他縣停匿亦宜定罪。隨處土民久困徭役,客戶販鬻坐獲厚利,官無所斂,亦宜稍及客戶,以寬土民。行院帥府幕職,雖無部眾,亦嘗贊畫戎功,而推賞止進官一階,宜聽主將
 保奏,第功行賞。」上皆從其請。戊午,遣親軍討河南群盜。辛酉,大元兵攻綏德州。壬戌,夏人復侵龕谷。甲子,敕監察所彈事,同列不可預聞,著為令。丁卯,夏人犯定西、積石之境。戊寅,分京畿戍卒萬二千,河中民兵八千,以許州元帥紇石烈鶴壽將之,屯潼關西。



 十一月癸未,陜西東路行省報安塞堡敗夏人之捷。甲申,諭太府減損食品。庚寅,募民興南陽水田。壬辰,太子、親王、百官表賀安塞堡之捷,卻之。乙未,夏人攻龕谷。宋人攻蘄縣。紅襖賊掠宿州。辛丑,詔蠲徐、邳、宿、泗等州逋租,官民有能墾闢閒田,除來年科征。歸德、亳、壽、潁停閣逋戶租外,仍蠲三
 之一。逋戶田廬有司募民承業,禁其毀損,以俟來復。蒲城縣民李文秀等謀反,伏誅。壬寅,宋人焚潁州,執防禦判官而去。是日,相國寺火。大元兵攻延安。



 十二月辛亥朔,以大元兵下潼關、京兆,詔省院議之。壬子,罷避舉縣令法。丁巳,禮部侍郎烏古孫仲端、翰林待制安庭珍使北還,各遷一階。庚申,罷河南義軍。丁卯,詔罷新簽民軍,減樞密院掌兵官及京城戍兵,仍諭行院帥府,毋擅增設補簽。辛未,罷行總管府及招討統軍檢察等司。定宋人來歸賞格及詐誘徵防軍人逃亡罪法。癸酉,元帥合達買住及其將士以延安功特賞賚之,仍下詔獎諭。



 閏
 月辛巳朔,大元兵徇鄜州,保大軍節度使完顏六斤、權元帥右都監紇石烈鶴壽、右都監蒲察婁室、遙授金安軍節度使女奚烈資祿皆死之。乙酉,提控術甲咬住破沈丘賊于陳瓦。丙戌,頒詔撫諭河南土寇。戊子,熒惑犯軒轅。己丑,孫瑀及捕盜官吾古出招降泰和縣賊二千人,詔斬其首惡,餘皆釋之。同知保靜軍節度使郭澍以徵糧失期,誣殺平民,坐誅。辛卯,官軍復葭州。癸巳,通遠軍節度使孛術魯合住削官。甲午,月犯熒惑。丙申,紅襖賊入蒙城縣,縣官失其符印,軍民死者甚眾,賊大掠而去。戊戌,鎮星晝見于軫。己亥,發兵捕京東盜。太白晝
 見于室。壬寅,發上林署粟賑貧民。陳、亳等州,鹿邑、城父諸縣,盜蜂起,趣樞府遣官討之。捕盜軍所過殘民,遣御史一人按視。軍所獲牛,有司以官錢收贖。戊申,詔定招捕土寇官賞格。己酉,更造「興定寶泉」,每一貫當「通寶」四百貫。



 元光年春正月庚戌朔,免朝。辛亥,世宗忌辰,謁奠于啟慶宮。元帥惟弼破紅襖賊於張騫店。壬子,遣官墾種京東、西、南三路水田。甲寅,禁非邊關急速事無馳傳,有濫乘者州縣徑白省部,四方館從御史臺,外路從分按御史治之。詔陜西西路行省徙京兆者,兵退還治平涼。
 坊州刺史把移失剌以棄城,伏誅。鄭州防禦使裴滿羊哥、同知防禦使古里甲石倫除名。平西節度使把古咬住奪官一階。丁卯,詔撫諭京東百姓。



 二月壬午,詔徙中京、唐、鄧、商、虢、許、陜等州屯軍及諸軍家屬赴京兆、同、華就糧屯。乙酉,陜西西路行省請以厚賞募河西蕃部族寺僧,圖復大通城,命行省樞密院籌之。癸巳,上諭宰臣,宋人以重兵攻平輿、褒信,我師力戰卻之,又偵知其事狀之詳。若俟帥府上功推賞,豈急於勸獎之道?其遣清望官,齎空名宣敕,核實給之。乙未,詔諭河南、陜西。大元兵屯葭州。壬寅,權定行省、樞府、元帥府輒杖左右司、
 經歷司官罪法。甲辰,上念鄜延被兵,又延安受圍,嘗發民粟給軍。詔除延安、鄜、坊、丹、葭、綏德稅租,仍令有司償其粟直,不足者許補官。戊申,恒州軍變,萬戶呼延棫等千餘人殺掠城中,焚廬舍而去。己酉,遣元帥左監軍訛可行元帥府事,節制三路軍馬伐宋,同簽樞密院事時全行院事,副之。



 三月辛酉,宋人掠確山縣之劉村。丙寅,歲星犯太微左執法。戊辰,樞密院委差官賈天安上書言利害。壬申,尚書右丞徒單思忠以病馬輸官,冒取高價,御史劾之,有司以監主自盜論死,上顧惜大體,降授陳州防禦使。癸酉,提控李師林敗夏人于永木嶺。郭文
 振表,近得俘者言,南北合兵將攻河南、陜西。詔樞密備禦。夏四月辛巳,以金吾衛上將軍、勸農使訛可簽樞密院事。置大司農司,設大司農卿、少卿、丞,京東、西、南三路置行司,並兼採訪事。壬午,大元兵攻陵川縣。丁酉,林懷路行元帥府事惟良削官兩階,罷之。更定辟舉縣令之法,而復行之。戊戌,籍丁憂待闕、追殿等官,備防秋。丁未,行樞密院報淮南之捷。



 五月戊申朔,大元兵屯隰、吉、翼等州。壬戌,訛可、時全軍大敗。甲子,訛可以敗績當死,上面數而責之,勉其後效命,朘官兩階。丁卯,召致政胥鼎等赴省議利害。壬申,時全伏誅。



 六月戊寅朔,造舟運陜
 西糧,由大慶關渡抵湖城。癸未,大赦。陳州防禦使呂子羽坐乏軍興,自盡。制諸監官及八品以下職事,丁憂、待闕、任滿、遙授者,試補侍衛親軍。命各路司農司設捕盜方略。丁酉,紅襖賊掠柳子鎮,驅百姓及驛馬而去,提控張瑀追擊,奪所掠還。偽監軍王二據黎陽縣,提控王泉討之,復其城。秋七月庚戌,大元將按察鄔以其眾屯晉安、冀州之境。丙辰,上黨公完顏開復澤州。己未,歸德行樞密院王庭玉報曹州破紅襖賊之捷。庚申,定監當官選法。河北群盜犯封丘、開封界,令樞密院禦捕。甲子,京東總帥紇石烈牙吾塔請自今行院帥府幕職,有過得自
 決之。不允。戊辰,紅襖賊襲徐州之十八里砦,又襲古城、桃園,官軍破之。乙亥,太白晝見經天,與日爭光。



 八月丁丑,定西征將士官賞有差。己卯,彗星見西方。甲申,增定藏匿逃亡親軍罪及告捕賞格。積石州蕃族叛附于夏,鞏州提控尼旁古三郎討之,獲羊千口,進尚膳,詔卻之。以彗星見,改元,大赦。諭旨宰臣曰:「赦書已頒,時刻之間,人命所係。其令將命者速往,計期而至。」以大司農把胡魯為參知政事。癸巳,河間公移剌眾家奴、高陽公張甫兵復河間府,是日,報捷者始達。上以道途梗塞,報者艱虞,命厚賞之。夏人入德順。壬寅,祈雨。



 九月丙午朔,以左
 右警巡使兼彈壓。諭陜西行省備邊。壬子,牙吾塔請以兵由壽州渡淮,搗宋人巢穴,不從。乙卯,議經略淮南。己巳,宋人掠遂平縣之石砦店,復侵南陽,唐州提控夾谷九住敗之。冬十月丁丑,夏人掠德順之神林堡。壬午,宋張惠攻零子鎮,為斡魯朵所敗,虜其裨將二人。河中府萬戶孫仲威執其安撫使阿不罕胡魯剌據城叛,陜西行省遣將討平之。癸未,復曹州。甲申,上獵于近郊,詔免百官送迎,且勿令治道,以勞百姓。庚寅,徙彰德招撫使杜先軍於衛州。乙未,大元兵下榮州之胡壁堡及臨晉。庚子,詔所司巡護避兵民資產。甲辰,以京兆官民避兵
 南山者多至百萬,詔兼同知府事完顏霆等安撫其眾。



 十一月丁未,大元兵徇同州,定國軍節度使李復亨、同知定國軍節度使訛可皆自盡。甲寅,京東總帥牙吾塔報臨淮破宋兵之捷。戊辰,大元蒙古蒲花攻鳳翔府。



 十二月乙亥朔,上謂皇太子曰:「吾嘗夜思天下事,必索燭以記,明而即行,汝亦當然。」以河中治中侯小叔權元帥府右都監,許便宜行事。乙酉,遷同知平陽府事史詠龍虎衛上將軍,賜號「守節忠臣」,權行平陽公府事。丁亥,疊州總管青宜可卒,特命其子角襲職。詔諭近侍局曰:「奉御、奉職皆少年,不知書。朕憶曩時置說書人,日為講
 論自古君臣父子之教,使知所以事上者,其復置。」己丑,蘭州提控唐括昉敗夏人于質孤堡。大元以大軍攻鳳翔。



 二年春正月甲辰朔,詔免朝賀。乙巳,世宗忌日,謁奠于啟慶宮。右丞相汝礪乞致政,上面諭使留。大元兵下河中府,權元帥右都監侯小叔復之。壬子,壽州防禦使完顏乃剌奪官四階。甲寅,上諭宰臣曰:「向有人言便宜事,卿等屢奏乞作中旨行之。帝王從諫足矣,豈可掠人之美以為己出哉!」戊午,四方館瘸驢以罪罷,宰臣請以散地羈縻之,上曰:「此輩豪傑,正須誠待,若以術制,適
 使自疑。但不畀軍政,外補何害?」授瘸驢恒州刺史。又謂:「鬻爵恩例有丁憂官得起復者,是教人以不孝也,何為著此令哉?」丁卯,大元兵復下河中府。



 二月甲戌朔,皇后生辰,詔免賀禮。己卯,丞相汝礪朝會,免拜,設榻殿下,久立賜休。壬午,詔「軍官犯罪,舊制更不可任用,今多故之秋,人才難得,朕欲除大罪外,徒刑追配有武藝善掌兵者,量才復用。其令尚書省議以聞」。丁亥,大赦。己亥,鳳翔圍解。石盞合喜加金紫光祿大夫,升左監軍,特授大名府谷忽申猛安,完顏仲元加光祿大夫,升右監軍,特授河北東路洮委必剌猛安,各賜金鞶帶有差。



 三月甲辰
 朔,宋人襲汝陽。壬子,誡諭平章英王守純崇飲。癸丑,以河中府推官籍阿外權元帥右都監,代領侯小叔軍。甲寅,上謂宰臣:「人有才堪任事,其心不下者,終不足貴。」丞相汝礪對曰:「其心不正而濟之以才,所謂虎而翼者也,雖古聖人亦未易知。」上以為然。丙辰,長春節,免朝。以戶部尚書石盞畏忻為參知政事,兼修國史。辛酉,禁茶。壬戌,詔以鳳翔戰功及頒賞等級遍諭諸郡。甲子,以完顏伯嘉權參知政事,行尚書省于河中府。辛未,詔職官犯罪非死罪除名,遇赦幸免,有才幹者中外並用。夏四月癸酉朔,復霍州汾西縣,詔給空名宣敕,遷賞將士之有
 功者。丙子,設京兆南山安撫司。丁丑,故鳳翔萬戶完顏醜和以死節贈懷遠大將軍,授刺史職。其父恕除以功例賞外,遷兩官,升職二等。己卯,遣官閱河南帥府見兵,籍閑官豪右親丁及遼東、河北客戶為軍。庚子,募西山獵戶為軍。



 五月癸卯朔,始造「元光重寶」。丙午,復河中府及榮州,遣人持檄招前恒山公武仙。乙卯,權平陽公史詠復霍州及洪洞縣。丁巳,始造「元光珍貨」,同銀行用。戊午,以檄招東平嚴實。己未,參知政事毅夫言:「脅從人號『忠孝軍』,而置沿淮者所為多不法,請防閑之。」上曰:「人心無常,顧馭之何如耳。馭之有術,遠方猶且聽命,況此輩
 乎!不然,雖左右亦難防閑。正在廓開大度而已。若是而不能致太平者,命也。」庚申,簽河南路寄居官民充軍。辛酉,徙晉陽公郭文振兵于孟州。甲子,徙權平陽公史詠兵於解州、河中府。



 六月乙亥,京東總帥牙吾塔報淮南之捷。丁亥,罷行省所置監察御史兼彈壓之職。戊子,議遣人招李全、嚴實、張林。甲午,詔罷河中行省,置元帥府。辛丑,遙授靜難軍節度使顏盞蝦麻等以保鳳翔功進官。秋七月壬寅朔,夏人犯積石州,羌界寺族多陷沒,惟桑逋寺僧看逋、昭逋、廝沒及答那寺僧奔鞠等拒而不從。詔賞諸僧鈐轄正將等官,而給以廩祿。乙巳,遣兵守衛解州
 鹽池。庚戌,以空名宣敕遷賞諸部降人。壬子,除市易用銀及銀與寶泉私相易之禁。癸丑,敕諸御史曰:「瑣細事非人主所宜詰,然凡涉姦弊,靡不有關國政者。比聞朝官及承應人月給俸糧,多雜糠土,有司所收曷嘗有是物哉。至于出納斗斛,亦小大不一,此皆理所不容者,而臺官初不問。事事須朕言之,安用汝曹也!」乙卯,丹鳳門壞。丁巳,陰坡族之骨鞠門等叛歸夏,元帥夾谷瑞發兵討之,以捷聞。御史中丞師安石言制敵二事。戊午,宰臣方對次,有司奏前奉御溫敦太平卒。上大駭曰:「朕屢欲授太平一職,每以事阻,今僅授之未數日而亡,豈非天
 耶!」因謂宰臣曰:「海陵時有護衛二人私語,一曰富貴在天,一曰由君所賜。海陵竊聞之,詔授言由君所得以五品職,意謂誠由己也,而其人以疾竟不及授。章宗秋獵,聞平章張萬公薨,歎曰:『朕乃將拜萬公丞相,而遂不起,命也。』」乙丑,詔籍陜西路僑居官民為軍。



 八月辛未朔,邳州從宜經略使納合六哥等率都統金山顏俊以沂州百餘人,晨入省署,殺行尚書省蒙古綱,據州反。壬申,詔賞京兆路官軍保全南山諸谷之功,以所全人數多寡為等第,千人以上官一階,三千人以上兩階,五千人以上三階,仍陞職一等,能以力戰護之者又增一階,戰
 沒者就以贈之。甲戌,遣官持空名宣敕,諭以重賞招納合六哥,拒命,即命牙吾塔合行院兵討滅之。乙亥,火星入鬼宿中,掩積尸氣。乙酉,詔能捕獲反賊六哥者,除見定官外,仍與世襲謀克。丙戌,遣官分行蔡、息、陳、亳、唐、鄧、裕諸州,洎司農司州縣吏同議,凡民丁相聚立砦避兵,與各巡檢軍相依者,五十戶以上置砦長一員,百戶增副一員,仍先遷一官,能安民弭盜勸農者論功注授。



 九月庚子朔,日有食之。宋人入壽州,女奚烈蒲乃力戰卻之。壬寅,樞密院奏提控術甲剉只罕破宋人之功。甲辰,宋人攻南陽。丙午,牙塔報桃園、淮陽之捷,並以納合
 六哥結構李全之狀來告。戊申,降人孫邦佐自李全軍中歸,遙授知東平府兼山東西路兵馬都總管。官軍與宋人力戰于胡陂而卻之,提控術虎春兒為所殺。癸丑,納合六哥所署偽都統烏古論賽漢、夾谷留住等來歸。己未,贈術虎春兒銀青榮祿大夫。丙寅,扎也胡魯等拔邳州南城。丁卯,權御史中丞帥安石等劾英王守純不實,付有司鞠治,尋詔免罪,而猶責論之。冬十月癸酉,徙晉陽公郭文振等兵于衛州。乙亥,制行樞密院及元帥府,農隙之月分番巡徼校獵,月不過三次。丁丑,上獵于近郊。己卯,祫于太廟。壬午,火星犯靈臺。乙酉,上獵于近
 郊。辛卯,詔石壕店、澠池、永寧縣各屯兵千人。壬辰,滕州人時明謀反,伏誅。戊戌,唐、鄧行元帥報淮南之捷。



 十一月己亥,紅襖賊偽監軍徐福等來降。詔進牙吾塔官一階,賜金幣有差。辛丑,總帥牙吾塔報邳州之捷,函叛人六哥首以獻。開封縣境有虎咥人,詔親軍百人射殺之,賞射獲者銀二十兩,而以內府藥賜傷者。丙午,邳州紅襖賊三千來降,初擬置諸陳、許之間,上以為「若輩雖降,家屬尚在河朔,餘黨必殺之,所得者寡而被害者眾,亦復安忍?不若命使撫諭,加以官賞而遣之還。果忠於我,雖處河朔豈負我耶?且餘眾感恩,將有效順者矣」。戊午,
 以上黨公完顏開之請,諭開及郭文振、史詠、王遇、張道、盧芝等各與所鄰帥府相視可耕土田,及瀕河北岸之地,分界而種之,以給軍餉。辛酉,鞏州行元帥府報會州破夏人之捷。



 十二月己巳朔,徙沿淮巡檢邊軍于內地。癸酉,以空名宣命金銀符給完顏開賞功。辛巳,詔延安土人充司縣官義軍使者選人代之,量免其民差稅。邳州民丁死戰陣者各贈官一階。歸德、徐、邳、宿、泗、永、亳、潁、壽等州復業及新地民,免差稅二年,見戶一年,嘗供給邳州者復免一年之半,睢州、陳留、杞縣免三之一。



 丁亥,上不豫,免朝。戊子,皇太子率百官及王妃、公主入問起
 居。己丑,復入問起居。庚寅,上崩于寧德殿,壽六十有一。上疾大漸,暮夜,近臣皆出,惟前朝資明夫人鄭氏年老侍側,上知其可托,詔之曰:「速召皇太子主後事。」言絕而崩。夫人秘之。是夜,皇后及遣妃龐氏問安寢閣。龐氏陰狡機慧,常以其子守純年長不得立,心鞅鞅。夫人恐其為變,即紿之曰:「上方更衣,后妃可少休他室。」伺其入,遽鑰之,急召大臣,傳遺詔立皇太子,始啟戶出后妃,發喪。皇太子方入宮,英王守純已先入,皇太子知之,分遣樞密院官及東宮親衛軍官移剌蒲阿集軍三萬餘於東華門街。部署即定,命護衛四人監守純於近侍局,乃即
 皇帝位於柩前。壬辰,宣遺詔。是日,詔赦中外。明年



 正月戊戌朔,改元正大,謚大行曰繼天興統述道勤仁英武聖孝皇帝,廟號宣宗。三月庚申,葬德陵。



 贊曰:宣宗當金源末運,雖乏撥亂反正之材,而有勵精圖治之志。迹其勤政憂民,中興之業蓋可期也,然而卒無成功者何哉?良由性本猜忌,崇信翽御,獎用吏胥,苛刻成風,舉措失當故也。執中元惡,此豈可相者乎,顧乃懷其援立之私,自除廉陛之分,悖禮甚矣。高琪之誅執中,雖云除惡,律以《春秋》之法,豈逃趙鞅晉陽之責?既不能罪而遂相之,失之又失者也。遷汴之後,北顧大元之
 朝日益隆盛,智識之士孰不先知?方且狃於餘威,牽制群議,南開宋釁,西啟夏侮,兵力既分,功不補患。曾未數年,昔也日闢國百里,今也日蹙國里,其能濟乎?再遷遂至失國,豈不重可歎哉!



\end{pinyinscope}