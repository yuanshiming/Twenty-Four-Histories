\article{列傳第二十四}

\begin{pinyinscope}

 ○李石子獻可完顏福壽獨吉義烏延蒲離黑烏延蒲轄奴烏延查剌李師雄尼龐古鈔兀孛術魯定方夾谷胡剌蒲察斡論夾谷查剌



 李石,字子堅,遼陽人,貞懿皇后弟也。先世仕遼,為宰相。高祖仙壽,嘗脫遼主之舅於難,遼帝賜仙壽遼陽及湯池地千頃,他物稱是,常以李舅目之。父雛訛只,桂州觀
 察使,高永昌據東京,率眾攻之,不勝而死。石敦厚寡言,而器識過人。天會二年,授世襲謀克,為行軍猛安。睿宗為右副元帥,引置軍中,屬之宗弼。八年,除禮賓副使,轉洛苑副使。天眷元年,置行臺省於汴,石為汴京都巡檢使,歷大名少尹、汴京馬軍副都指揮使,累官景州刺史。海陵營建燕京宮室,石護役皇城端門。海陵遷都燕京,石隨例入見。海陵指石曰:「此非葛王之舅乎?」葛王,謂世宗也。未幾,除興中少尹。石知海陵忌宗室,頗歉前日之言,秩滿,託疾還鄉里。世宗留守東京,禦契丹括里,石留東京巡察城中。海陵使副留守高存福伺察世宗動靜,
 知軍李蒲速越知存福謀,以告世宗,石因勸世宗先除存福,然後舉事,世宗從之。大定元年,以定策功為戶部尚書。無何,拜參知政事。



 阿瑣殺同知中都留守蒲察沙離只,遣使奉表東京,而群臣多勸世宗幸上京者。石奏曰:「正隆遠在江、淮,寇盜蜂起,萬姓引領東向,宜因此時直赴中都,據腹心以號令天下,萬世之業也。惟陛下無牽於眾惑。」上意遂決,即日啟行。世宗納石女後宮,生鄭王永蹈、衛紹王永濟,是為元妃李氏。



 三年,戶部尚書梁金求上言:「大定以前,官吏士卒俸粟支帖真偽相雜,請一切停罷。」石買革去舊貼,下倉支粟,倉司不敢違,以新粟
 與之。上聞其事,以問梁金求。梁金求對不以實。上命尚書左丞翟永固鞫之。梁金求削官四階,降知火山軍,石罷為御史大夫。久之,封道國公。



 六年,上幸西京,石與少詹事烏古論三合守衛中都宮闕。詔曰:「京師巡禦,不可不嚴。近都猛安內選士二千人巡警,仍給口豢芻粟。」謂宰臣曰:「府庫錢幣非徒聚貨也,若軍士貧弱,百姓困乏,所費雖多,豈可已哉?」故事,凡行幸,留守中都官每十日表問起居。上以使傳頻煩,命二十日一進表。七年,拜司徒,兼太子太師,御史大夫如故。賜第一區。



 安化軍節度使徒單子溫,平章政事合喜之侄也,贓濫不法,石即劾奏之。方
 石奏事,宰相下殿立,俟良久。既退,宰相或問石奏事何久,石正色曰:「正為天下姦污未盡誅耳。」聞者悚然。一日,上謂石曰:「御史分別庶官邪、正。卿等惟劾有罪,而未嘗舉善也,宜令監察分路刺舉善惡以聞。」



 石司憲既久,年浸高。御史臺奏,事有在制前斷定,乞依新條改斷者。上曰:「若在制前行者,豈可改也。」上御香閣,召中丞移剌道謂之曰:「李石耄矣,汝等宜盡心。向所奏事甚不當,豈涉於私乎?」他日,又謂石曰:「卿近累奏皆常事,臣下善惡邪正,無語及之。卿年老矣,不能久居此,若能舉一二善事,亦不負此職也。」十年,進拜太尉、尚書令。詔曰:「太后兄弟
 惟卿一人,故命領尚書事。軍國大事,涉於利害,識其可否,細事不煩卿也。」進封平原郡王。



 平章政事完顏守道奏事,石神色不懌。世宗察之,謂石曰:「守道所奏,既非私事,卿當共議可否。在上位者所見有不可,順而從之,在下位者所見雖當,則遽不從乎?豈可以與己相違而蓄怒哉。如此則下位者誰敢復言?」石對曰:「不敢。」上曰:「朕欲於京府節鎮運司長佐三員內任文臣一員,尚未得人。」石奏曰:「資考未至,不敢擬。」上曰:「近觀節度轉運副使中才能者有之。海陵時,省令史不用進士,故少尹節度轉運副使中乏人。大定以來,用進士,亦頗有人矣,節度轉
 運副使中有廉能者具以名聞,朕將用之。朝官不歷外任,無以見其才,外官不歷隨朝,無以進其才,中外更試,庶可得人。」他日,上復問曰:「外任五品職事多闕,何也?」石對曰:「資考少有及者。」上曰:「茍有賢能,當不次用之。」對不稱旨,上表乞骸骨,以太保致仕,進封廣平郡王。十六年,薨。上輟朝臨弔,哭之慟,賻錢萬貫,官給葬事。少府監張僅言監護,親王、宰相以下郊送,謚襄簡。



 石以勳戚,久處腹心之寄,內廷獻替,外罕得聞。觀其劾奏徒單子溫退答宰臣之問,氣岸宜有不能堪者。時論得失半之,亦豈以是耶?舊史載其少貧,貞懿后周之,不受,曰:「國家方急
 用人,正宜自勉,何患乎貧。」后感泣曰:「汝茍能此,吾復何憂。」及中年,以冒粟見斥,眾譏貪鄙,如出二人。史又稱其未貴,人有慢之者,及為相,其人以事見石,惶恐。石曰:「吾豈念舊惡者。」待之彌厚。能為長者言如是,又與他日氣岸迥殊。



 山東、河南軍民交惡,爭田不絕。有司謂兵為國根本,姑宜假借。石持不可,曰:「兵民一也,孰輕孰重?國家所恃以立者,紀綱耳,紀綱不明,故下敢輕冒。惟當明其疆理,示以法禁,使之無爭,是為長久之術。」趣有司按問,自是軍民之爭遂息。北京民曹貴謀反,大理議廷中,謂貴等陰謀久不能發,在法「詞理不能動眾,威力不足率
 人」,罪止論斬。石是之。又議從坐,久不能決。石曰:「罪疑惟輕。」入,詳奏其狀,上從之,緣坐皆免死。北鄙歲警,朝廷欲發民穿深塹以禦之。石與丞相紇石烈良弼皆曰:「不可。古築長城備北,徒耗民力,無益於事。北俗無定居,出沒不常,惟當以德柔之。若徒深塹,必當置戍,而塞北多風沙,曾未期年,塹已平矣。不可疲中國有用之力,為此無益。」議遂寢。是皆足稱云。



 世宗在位幾三十年,尚書令凡四人:張浩以舊官,完顏守道以功,徒單克寧以顧命,石以定策,他無及者。明昌五年,配享世宗廟廷。子獻可、逵可。



 獻可字仲和,大定十年,中進士第。世宗喜曰:「太后家有子孫舉進士,甚盛事也。」累官戶部員外郎,坐事降清水令,召為大興少尹,遷戶部侍郎,累遷山東提刑使。卒。衛紹王即位,以元舅贈特進,追封道國公。子道安,擢符寶郎。



 完顏福壽,曷速館人也。父合住,國初來歸,授猛安。天眷二年,福壽襲父合住職,授定遠大將軍,累加金吾衛上將軍。海陵省併猛安謀克,遂停封。正隆末,海陵伐宋,福壽領婁室、臺答藹二猛安由山東道進至泰安。既受甲,福壽乃誘將校北還,而高忠建、盧萬家奴等亦各率眾
 萬餘俱歸東京,欲共立世宗。至遼口,世宗遣徒單思忠、府吏張謀魯瓦等來迎,察其去就。思忠等以數騎馳入軍中,見福壽等問曰:「將軍何為至此?」福壽等向南指海陵而言曰:「此人失道,不能保天下。國公乃太祖皇帝親孫,我輩欲推戴為主,以此來耳。」諸軍皆東向拜,呼萬歲。為書以授思忠。於是督諸軍渡遼水,徑至東京城下,即諭軍士擐甲入衛宮城,殺高存福等。明日,與諸將及東京吏民從婆速路兵馬都總管完顏謀衍勸進。世宗即位,以福壽為元帥右監軍,賜以銀幣御馬。



 初,謀衍之至也,大會諸軍,以福壽之軍居左,高忠建軍居右。忠建曰:「
 何以我軍為右軍?」謀衍曰:「樹置在我,爾曷敢言!」福壽曰:「始建大事,左右軍高下何足爭也。」遂讓忠建為左軍。世宗聞而賢之。未幾,從完顏謀衍討白彥敬、紇石烈志寧於北京。是冬,上聞臨潢尹兼元帥左都監吾扎忽等與窩斡戰不利,命福壽將兵進討。已敗賊,俘獲生口萬計。世宗以紇石烈志寧代之,召還,授興平軍節度使,復其世襲猛安,尋領濟州路諸軍事。大定三年,卒。



 獨吉義,本名鶻魯補,曷速館人也。徙居遼陽之阿米吉山。祖回海,父秘剌。改國二年,曷速館來附,秘剌領戶三百,遂為謀克。秘剌長子照屋,次子忽史與義同母。秘剌
 死,忽史欲承謀克。義曰:「長兄雖異母,不可奪也。」忽史乃以謀克歸照屋,人咸義之。義以質子至上京。善女直、契丹字,為管勾御前文字。天會十五年,擢右監門衛大將軍,除寧化州刺史。察廉,遷迭剌部族節度使、復州防禦使,改卓魯部族節度使、河南路統軍都監,為武勝軍節度使。邊郡妄稱寇至,統軍司徙居民於汴,義獨不聽,日與官屬擊球游宴。統軍司使人責之,義曰:「太師梁王南伐淮南,死者未葬,亡者未復,彼豈敢先發?此城中有榷場,若自動,彼將謂我無人。」既而果無事,統軍謝之,請以沿邊唐州等處諸軍猛安皆隸于義。貞元元年,改唐古
 部族節度使,為彰化軍,改利涉軍節度使。是時,海陵伐宋,諸軍往往逃歸,而世宗在東京得眾心。都統白彥敬自北京使人陰結義,欲與共圖世宗。頃之,世宗即位,義即日來歸,具陳所以與彥敬密謀者。世宗嘉其不欺,以為參知政事。



 上謂義曰:「正隆率諸道兵伐宋,若反IM北指,則計將安出?」義曰:「正隆多行無道,殺其嫡母,阻兵虐眾,必將自斃。陛下太祖之孫,即位此其時也。」上曰:「卿何以知之?」義曰:「陛下此舉若太早,則正隆未渡淮,太遲則窩斡必太熾。今正隆已渡淮,窩斡未至太盛,將士在南,家屬皆在此,惟早幸中都為便。」上嘉納之。次榛子嶺,世宗
 聞海陵死于軍中,謂義曰:「信如卿所料。」大定二年,罷為益都尹,兼本路兵馬都總管,賜金五十兩、銀五百兩。三年,以疾致仕。四年,薨于家,年七十一。



 子和尚,大定初,除應奉翰林文字,佩金牌。陀滿訛里也子撒曷輦充護衛,司吏王得兒加保義校尉,皆佩銀牌。持詔書宣諭中都以南州郡,及往南京諭太傅張浩。中道聞海陵遇害,南京及都督府皆奉表賀,乃止。和尚為奉使,擅廢置州縣官,輒行殺戮,詔尚書省鞫治之。十九年,詔以義孫引壽為斜魯答阿世襲謀克。義性辯給,善談論,服玩不尚奢侈,食不兼味云。



 贊曰:章宗嘗問群臣:「世宗初起東京,大臣為誰?」完顏守貞對曰:「止有李石一人。」章宗歎曰:「茍如此,信有天命也。」完顏謀衍部署諸軍,高忠建爭長,完顏福壽讓忠建而己下之,其功多矣。當是時,獨吉義最先至,諸將尚未肯附。由是言之,果天也,非人力也。



 烏延蒲離黑,速頻路哲特猛安人,改屬合懶路。祖思列,預平烏春、窩謀罕之亂,及伐遼、宋,皆有功,追授猛安,贈銀青光祿大夫。父國也,襲猛安。蒲離黑從太祖伐遼,勇聞軍中。天眷三年,襲猛安,授寧遠大將軍,累官武寧軍節度使,遷京兆尹。海陵伐宋,行武威軍都總管。軍還,為
 順義軍節度使。徒單合喜定秦、隴,蒲離黑統完顏習尼列、顏盞門都兵救德順州,改延安、平涼尹。致仕,封任國公。大定十九年卒。



 烏延蒲轄奴,速頻路星顯河人也,後改隸曷懶路。父忽撒渾,天輔初,追授猛安,親管謀克。蒲轄奴身長有力,多智略,襲其父猛安謀克,階寧遠大將軍。天德二年,授陳州防禦使。貞元元年,改昌武軍節度使,以善綏撫,再任。海陵南征,改歸德尹,為神策軍都總管。當屯濟州,比至山東,盜已據其城。蒲轄奴領十餘騎往覘之,忽為其眾所圍。乃與軍士皆下馬,立而射之,殺百餘人。賊眾敗走,
 迤邐襲之,至暮而還。明日,攻破其城,號令士卒,毋害居民,郡中獲安。民感其惠,為立祠以祭。大定二年,為慶陽尹。元帥左都監徒單合喜奏宋軍十萬餘據險阻,剽掠郡邑,請益師。詔益兵七千,與舊兵合為二萬。遣蒲轄奴與延安尹高景山等分領其軍以往。卒于軍,年六十一。子查剌。



 烏延查剌,銀青光祿大夫蒲轄奴子也。力兼數人,勇果無敵。正隆六年伐宋,諸猛安謀克兵皆行,州縣無備。契丹括里陷韓州,圍信州,遠近震駭。查剌道出咸平,遂率本部亟還信州,與戰敗之。已而賊復整兵環攻,且登其
 城,查剌下巨木壓之,殺賊甚眾,括里乃解去。查剌左右手持兩大鐵簡,簡重數十斤,人號為「鐵簡萬戶」。追及括里于韓州東八里許,賊方就平野為陣,查剌身率銳士,以鐵簡左右揮擊之。無不僵仆。賊不能成列,乃易馬督軍復擊之。賊眾大敗,遂走,東京、咸平、隆州民復帖然。



 世宗即位,查剌謁見,充護衛,為驍騎副都指揮使,領萬戶。擊窩斡,戰於花道。大軍未集,查剌在左翼,領六百騎與賊戰,殺賊三千餘人。宗亨、蒲察世傑七謀克戰不利,世傑走查剌軍,賊合圍攻之。查剌圜拒而戰,宗敘軍來援,賊乃引去。西過裊嶺,追及於陷泉。賊先犯右翼,查剌迎
 擊之,賊退走。窩斡募人刺之,偽護衛阿不沙身長有力,奮大刀自後斫查剌,查剌回顧,以簡背擊阿不沙,折其右臂。與紇石烈志寧軍合擊,賊遂大敗。



 窩斡平,以為宿直將軍,賜銀三百兩、重彩二十端。丁父憂,以本官起復,襲其父猛安,除蔡州防禦使,改宿州,遷昌武軍節度使,徙鎮邠州。為賀宋歲元使,射淮上柳樹,矢入其樹飲羽。宋人素聞其名,甚異之。改鳳翔尹,入為右副點檢,出為興中尹,改婆速路總管。高麗憚其威名,凡以事至婆速路者,望見而跪之。二十五年,為興平軍節度使,卒官。



 查剌貞愨寡言,平居極和易,及臨戰奮勇,見者無不辟易,
 雖重圍萬眾,出入若無人之境云。



 李師雄,字伯威,雁門人也。有材力,喜談兵,慕古之英雄,故名師雄。宋宣和中以騎射登科,累官大名、清平尉。王師至大名,師雄與府僚出降,攝本路兵馬都監。齊國建,以為大總管府先鋒都統制,知淄州。齊廢,為汴京馬軍都虞候,歷知寧海軍、曹州刺史。皇統二年,為武勝軍節度使。正隆末,為河州防禦使。宋將吳璘軍攻秦、隴,會師雄以事就逮臨洮,宋兵至城下,州人乘城拒守,謀欲出降,師雄止之。宋將權儀鞭馬方上浮橋,師雄射之,墜于橋下,遂擒權儀,宋師退。後從元帥左監軍徒單合喜以
 兵攻河州,有功。未幾,以疾歸汴,卒。



 尼龐古鈔兀,曷速館人。初為大抃扎也,補元帥府通事。宋將韓世忠率軍數萬圍邳州,鈔兀將輕騎數百與偵人數輩間道往救之,敗敵兵六千。翌日,宋兵復圍下邳,鈔兀復敗之。宋人攻濟州,奪戰艦略盡。是時,鈔兀往宿州,分蒲魯虎軍,還至大河,與敵遇,力戰敗之,盡復戰艦。王師復河南,宋別將由胡陵夜襲孛堇布輝營,士卒盡沒。鈔兀從東平總管併力戰,卻之。元帥府賞以銀幣。鈔兀勇敢,善伺敵虛實,以此屢捷。帥府承制加忠顯校尉,為蕃部禿里,賜錢萬貫、幣帛三百匹、衣一襲、馬二匹。將
 之官,河間尹大抃白于元帥,請留鈔兀以給邊事,許之。復賜錢萬貫、銀二百五十兩、重彩三百端、馬三匹。錄功,授慶陽少尹。



 海陵將伐宋,而契丹反,召入諭之曰:「汝久在邊陲,屢立戰功。昨遣樞密使僕散忽土、留守石抹懷忠等討契丹,師久無功,已置諸法。今命汝與都統白彥敬、副統紇石烈志寧進討。」因賜具裝廄馬四疋。鈔兀與彥敬等至北京,未能進。會世宗即位遼陽,鈔兀迎謁,遷輔國上將軍,與都統吾札忽、副統渾坦討窩斡。鈔兀行至窊歷,與窩斡遇,左軍小卻,鈔兀挺槍馳入其陣,手殺二十餘人,賊乃退。元帥僕散忠義自花道追之,鈔兀以前
 鋒追及于陷泉,遂大敗之。事平,遷西北路招討使,改東北路。



 鈔兀與完顏思敬有隙,思敬為北京留守,奉詔至招討司,鈔兀不出餞。世宗聞之,遣使切責之曰:「卿本大抃扎也,起身細微。受國厚恩,累歷重任,乃以私憾,不餞詔使。當內省自訟,後勿復爾。朕不能再三曲恕汝也。」既而思敬為平章政事,東北路招討使鈔兀以私取諸部進馬,事覺被逮,將赴京師。鈔兀為人尚氣,次海濱縣,慨然曰:「吾豈能為思敬辱哉!」遂縊而死。十九年,詔以鈔兀舊功,授其子和尚世襲布輝猛安徒胡眼謀克。



 孛術魯定方,本名阿海,內吉河人也。材勇絕倫。海陵素
 聞其名。天德初,召授武義將軍,充護衛。數月,轉十人長,遷宿直將軍,賜予甚厚。尋為殿前右衛將軍,又三月,擢殿前右副點檢,世襲猛安,改左副點檢。出為河南尹,改彰德軍節度使。海陵南伐,定方為神勇軍都總管。大定二年,宋人陷汝州,河南統軍使宗尹遣定方將兵四千往取之。汝州東南及北面皆山林險阻,不可以騎軍戰。是時,宋兵由鴉路出沒,定方至襄城,得敵虛實,遂牒諭汝州屬縣曰:「我率許州戍兵十二萬徑取汝州,爾等可備糧草二十萬,使人揚言欲據要路絕宋兵往來。」既而定方引兵趨鴉路,宋人聞之,果棄城遁去。定方至魯山
 境,知宋兵已去,遂遣輕騎二百追至布褲叉,擊敗之,遂復汝州。授鳳翔尹。宋人阻邊,以本職行河南道軍馬副統,率步騎六萬,將由壽州進軍,次亳州。宋李世輔陷宿州,定方從左副元帥志寧戰於城下。時天大暑,定方督戰,馳突敵陣中,出入數四,渴甚,因出陣下馬取水,為人所害,年四十四。上聞而閔之,詔有司致祭,賻銀五百兩、重綵二十端,贈金紫光祿大夫。



 夾谷胡剌,上京宋葛屯猛安人。初在左副元帥撻懶帳下,有戰功,授武德將軍,襲其父謀克。正隆末,山東盜起,胡剌為行軍猛安討賊,遇賊千五百人於徐州南,敗之。
 山東路統軍司選諸軍八百人作十謀克,胡剌將之,與驍騎軍皆隸點檢司。行至淮南,海陵遣以騎兵三百二十往揚州,敗宋兵千五百人於宣化鎮。僕散忠義伐宋,胡剌領萬戶由泗州進戰,遇敵於宿州,歿于陣,贈鎮國上將軍。



 蒲察斡論,上京益速河人,徙臨潢。祖忽土華,父馬孫,俱贈金紫光祿大夫。斡論剛毅有技能。天輔初,以功臣子充護衛,遷左衛將軍、定武軍節度使,召為右副都點檢。天德初,授世襲臨潢府路曷呂斜魯猛安,改東平尹,賜錢千萬,累除河南尹。海陵伐宋,以本官為右領軍都監。
 大定二年,仍為河南尹,兼河南路都統軍使。宋以萬人據壽安縣,嵩州刺史石抹突剌、押軍萬戶徒單賽補以騎兵三百巡邏,遇於縣東,請師於斡論。斡論使猛安完顏鶻沙虎率七百人助之。宋兵多,突剌使士卒下馬,跪而射之。宋兵不能當,走入縣城。突剌進逼之,宋人棄城去,追及于鐵索口,復大敗之,遂復壽安。改北京留守、大定尹,卒官。



 夾谷查剌,隆州失撒古河人也。祖不剌速,國初授世襲曷懶兀主猛安、曷懶路總管。父謝奴,官至工部尚書。查剌狀貌魁偉,善女直、契丹書。天德初,以功臣子充護衛。
 二年,授武義將軍。未幾,擢符寶郎,凡再考,出為濼州刺史,改知平定軍事。海陵南征,為武威軍副都總管。軍還,大定二年,授景州刺史,遷同知京兆尹。時彰化軍節度使宗室璋等與宋將吳璘相拒於德順州,元帥左都監徒單合喜遣查剌與諸將議破敵策。璋等議曰:「我兵雖屢勝,而敵兵不退者,知我軍少故也。須都監親至,方可破敵。」於是合喜領兵四萬至,遂下德順州。入為殿前右衛將軍,襲父猛安,改左衛將軍,遷右副點檢。有疾,丞相良弼視之,謂所親曰:「此人國器也。他人有疾,吾未嘗往焉。」九年,出為東北路招討使兼德昌軍節度使,乃賜金
 帶。到官,治有勤績,邊境以安。其斷獄公平,道不拾遺。遷臨潢尹兼本路兵馬都總管,蕃部畏服。改西北路招討使。上遣使宣諭曰:「今諸部初附,命汝撫綏,當使治聲達於朕聽。」大定十二年卒。



 查剌性忠實,內明敏,每論大事,超越倫輩。太師勖嘗曰:「查剌不學而知,方之古人,如此者鮮矣。」



 贊曰:陷泉之捷,震電燁燁。符離之克,我勢攸赫。隴、坻手雹手暴,淮、濄鉤鈲,成矣。故列敘諸將之功焉。



\end{pinyinscope}