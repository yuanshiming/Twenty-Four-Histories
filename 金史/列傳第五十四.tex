\article{列傳第五十四}

\begin{pinyinscope}

 ○徒單兀典石盞女魯歡蒲察官奴內族承立一名慶山奴



 徒單兀典,不知其所始,累官為武勝軍節度使,駐鄧州。尋遷中京留守,知金昌府事,駐洛陽。鄧及洛陽兀典皆城之,且招亡命千人,號「熊虎軍」,以剽掠南鄙為事。宋人亦時時報復,邊民為之搔動。兀典資性深刻,而以大自居,好設耳目,凡諸將官屬下及民家細事,令親暱日報
 之,務為不可欺。正大間,以兵部尚書權參知政事,行省事於徐州。自恃得君,論議之際,不少假貸。同列皆畏之。



 天興元年正月,朝廷聞大兵入饒風,移兀典行省閿鄉,以備潼關。徒單百家為關陜總帥,便宜行事。百家馳入陜,榜州民云:「淮南透漏軍馬,慮其道由潼關,勢不能守,縣鎮遷入大城,糧斛輜重聚之陜州,近山者入山寨避兵。」會阿里合傳旨召兀典入援,兀典遂與潼關總帥納合合閏、秦藍總帥都點檢完顏重喜、安平都尉苗秀、蕩寇都尉術甲某、振武都尉張翼及虎威、鷹揚、葭州劉趙二帥,軍十有一萬、騎五千,盡撒秦藍諸隘之備。從虢入
 陜。同、華、閿鄉一帶軍糧數十萬斛,備關船二百餘艘,皆順流東下。俄聞大兵近,糧皆不及載,船悉空下。復盡起州民,連靈寶、硤石倉粟,游騎至,殺掠不勝計。又遣陜州觀察副使兼規措轉運副使抹捻速也以船八十往運潼關、閿鄉糧,行及靈寶北河夾灘。義軍張信、侯三集壯士三百餘,保老幼,立水柵。北將忽魯罕只乘淺攻之不能克,遇速也船至即降,大兵得此船遂破侯、張,殺戮殆盡。



 是時陜州同知內族探春願從行省征進,兀典授以帥職,聽招在城民充軍。探春厚擬官賞。數日無一人,乃以兀典命招之,得壯士八百。宣差趙三三名偉,亦依探
 春招募,偉人所知識,不二日得軍八百餘,號「破敵軍」。兀典忌偉得眾,欲挾詐坑之。完顏素蘭時為同華安撫使,力諫乃止。尋以偉權興寶軍節度使,兼行元帥府事,領軍三百,屯金雞堡。大兵即知潼關焚棄,長驅至陜。賀都喜不待命出城迎戰,馬蹶幾為所獲,兀典易以一馬,遂下令不復令一人出,大兵亦去。自此潼關諸渡船筏俱盡,偉亦無船可渡矣。



 初,兀典發閿鄉,拜天,賞軍,人白金三兩,將校有差。州之庫藏,軍資器械,為之一空。期日進發,已而不行,日造銀器及兵幕牌印,陜州及鹽司牌亦奪取之。又欲卻州民財物以資軍,素蘭諫之而止。二月
 戊午,乃行。有李先生者諫曰:「方今大兵俱在河南,河北空虛,相公可先取衛州,出其不意。彼知我軍在北,必分兵北渡,京師即得少寬,相公入援亦易為矣。」兀典大怒,以為泄軍機,斬之於市,遂行。軍士各以老幼自隨。州中亦有關中、河中遷避商賈老幼,亦倚兵力從行,婦女皆嫁士卒,軍中亦有強娶奪者。是日,軍出兩東門及南門,不遵洛陽路,乃由州西南徑入大山冰雪中。葭州劉、趙兩帥即日叛去,大兵以數百騎遙躡其後。明日,張翼軍叛往朱陽,入鹿盧關,大兵追及降之。山路積雪,晝日凍釋,泥淖及脛,隨軍婦女棄擲幼稚,哀號盈路。軍至鐵嶺,
 大兵潛召洛陽大軍從西三縣過盧氏,所至燒官民廬舍積聚,慮為金軍所據,又反守鐵嶺,以斷歸路。金兵知必死,皆有鬥志,然已數日不食,行二百里許,困憊不支,頗亦散走。於是完顏重喜先降,大軍斬於馬前。鄭倜劫苗英降,英不從,殺之,攜其首以降,於是士卒大潰。兀典、合閏提數十騎走山間,追騎禽得,皆殺之。先是,兀典嘗為鄧州節度使,世襲謀克黃摑三合時為宣差都總領,與兀典親厚,故決計入鄧。是役也,安平、蕩寇、鷹揚、振威諸都尉,及西安、金雞等軍,脫走者百才一二。



 二月,素蘭竄歸,有報徒單百家言「行省至」,百家欲出迎,父老遮馬
 前哀訴云:「行省復來,吾州碎矣,願無出迎。」百家曉之曰:「前日兀典,欲劫此州,為素蘭力勸而止,此行省非兀典,乃素蘭也。」父老乃聽百家出城。陜州自軍出。日有逃還者,百家皆撫納之,所得及萬人。百家又募收所棄甲仗。若獲二副,即以一與之,其一官出直買之,由是軍稍振。



 五月,總帥副點檢顏盞領軍復立商州總帥。華州人王某立虢州,權刺史。七月,制旨召百家入援,以權西安軍節度使、行元帥事阿不罕奴十剌為金安軍節度使、關陜總帥。



 九月,鞏昌知府元帥完顏忽斜虎入陜州,詔拜參知政事,行尚書省事。以河中總帥府經歷李獻能
 充左右司員外郎。獻能字欽叔,貞祐三年進士。復立山寨,安撫軍民。十月朔,制旨召忽斜虎赴南陽留山寺,以阿不罕奴十剌權參知政事,行省。



 時趙偉為河解元帥,屯金雞堡,軍務隸陜省,行省月給糧以贍其軍。明年五月,麥熟,省劄令偉計置兵食,權罷月給。十月,偉軍食又盡,屢白陜省,云無糧可給,偉私謂其軍言:「我與李員外郎有隙,坐視我軍飢餓,不為存恤。」於是自往永寧勸喻,偉頗為小民所信,往往獻糧,或導其發藏。南縣把隘軍提控以偉橫恣言於行省,行省遣趙提控者權元帥,守永寧元村寨,偉還金雞。



 十一月冬至,大兵已攻破元村
 寨,偉攻解州不能下,於是密遣總領王茂軍士三十人入陜州。匿菜圃中凡三四日,乘夜,王茂殺北城邏卒,舉號召偉軍八百渡河,入城劫殺阿不罕奴十剌、李獻能、提控蒲鮮某、總領來道安,因誣奏:「奴十剌等欲反,臣誅之矣。」朝廷知其冤而莫敢詰,就授偉元帥左監軍,兼西安軍節度使,行總帥府事。食盡。括粟,粟又盡,以明年三月降大兵。或謂偉軍餉不繼,以劫掠自資,一日詣李獻能,獻能靳之,曰:「從宜破敵不易。」由是憾之。乃乘奴十剌宴飲不設備,選死士二十八人,夜由後河灘踰城而上,取餅爐碎石擲屋瓦門扇為箭鏃聲。州人疑叛軍多,不
 敢動,遂開門納軍。殺行省以下官屬二十一人,獻能最為所恨,故被害尤酷。



 偉之變,絳州錄事張升字進之,大同人,戶工部令史出身,曾為漁陽簿,遷絳州錄事,謂知識者曰:「我本小人,受國家官祿,今日國家遭不幸,我不能從反賊。」言訖,赴水死,岸上數百人皆嗟惜之。



 及徒單百家鄭西之敗,單騎間道數百里入京。為上言兀典等鐵嶺敗狀。於是籍重喜、合閏、兀典家貲,暴兀典為罪首,榜通衢云。



 石盞女魯歡,本名十六。興定三年,以河南路統軍使為元帥右都監,行平涼元帥府事。先是,陜西行省胥鼎言:「
 平涼控制西垂,實為要地。都監女奚烈古里間材識凡庸,不閑軍務,且以入粟補官,遂得升用,握重兵,當方面,豈能服眾。防秋在邇,宜選才謀、有宿望、善將兵者代之。」故以命女魯歡。



 十一月,女魯歡上言:「鎮戎赤溝川,東西四十里,地無險阻,當夏人往來之衝,比屢侵突,金兵常不得利。明年春,當城鎮戎,彼必出兵來撓。乞於二三月間征傍郡兵,聲言防護,且令鄜、鞏各屯兵境上示進伐之勢,以制其肘。臣領平涼之眾,由鎮戎而入,攻其心腹。彼自救之不暇,安能及我。如此則鎮戎可城,而彼亦不敢來犯。又所在官軍多河北、山西失業之人,其家屬仰
 給縣官,每患不足。鎮戎土壤肥沃,又且平衍,臣裨將所統幾八千人,每以遷徙不常為病。若授以荒田,使耕且戰,則可以禦備一方,縣官省費而食亦足矣。其餘邊郡亦宜一體措置。」上嘉納焉。遷昌武軍節度使。



 元光二年九月,又言:「商洛重地,西控秦陜,東接河南,軍務繁密,宜選才幹之士為防禦使、攝帥職以鎮之。又舊來諸隘守禦之官,並從帥府辟置,其所辟者,多其親暱,殖產營私,專事漁獵,及當代去,又復保留,此最害之甚者。宜令樞府選舉,以革其弊。又州之戍兵艱於饋運,亦合依上屯田,以免轉輸之費。」又言:「每年防秋,諸隘守者不過數十
 人,餘眾盡屯保安、石門、大荊、洛南以為應援,中間相距遠至百里,倉猝豈能徵集。宜近隘築營。徙見兵居之,以待緩急。又南邊所設巡檢十員,兵率千人,此乃平時以詰姦細者,已有大軍。宜悉罷去。」朝廷略施行之。



 正大九年二月,以行樞密院事守歸德。乙丑,大元將忒木泬率真定、信安、大名、東平、益都諸軍來攻。是日,無雲而雷,有以《神武秘略》占之者,曰「其城無害」,人心稍安。適慶山奴潰軍亦至,城中得之,頗有鬥志。己巳,提控張定夜出斫營,發數砲而還。定平日好談兵,女魯歡令自募一軍,使為提控,小試而勝,上下遂恃以為可用。初患砲少,欲以
 泥或磚為之,議者恐為敵所輕,不復用。父老有言北門之西一菜圃中時得古砲,云是唐張巡所埋,掘之,得五千有奇,上有刻字或「大吉」字者。大兵晝夜攻城,駐營于南城外,其地勢稍高。相傳是安祿山將尹子奇於此攻巡、遠,得睢陽。時經歷冀禹錫及官屬王璧、李琦、傅瑜極力守禦,城得不拔。



 方大兵圍城,決鳳池大橋水以護城。都水官言,去歲河決敖游堌時,會以水平量之,其地與城中龍興塔平,果決此口,則無城矣。及大兵至,不得已遣招撫陳貴往決之,纔出門,為游騎所鈔,無一返者。三月壬午朔,攻城不能下,大軍中有獻決河之策者,主
 將從之。河既決,水從西北而下,至城西南,入故濉水道,城反以水為固。求獻策者欲殺之,而不知所在。四月,以女魯歡為總帥,佩金虎符。罷司農司,以其官蒲察世達為集慶軍節度使、行六部侍郎。溫特罕道僧歸德府同知,李無黨府判。五月,圍城稍緩,頗遷民出城就食。



 十二月,哀宗次黃陵岡,遣奉職術甲搭失不、奉職權奉御粘合斜烈來歸德徵糧。女魯歡遣侍郎世達,治中王元慶權郎中,儀封從宜完顏胡土權元帥,護送載糧千五百石。是月晦二更發船。二年正月,達蒲城東二十里。六軍給糧盡,因留船不聽歸,且命張布為幄,上遂用此舟以
 濟。



 及上來歸德,隨駕軍往往出城就糧,時城中止有馬用一軍,近七百人。用山西人,與李辛同鄉里,嘗為辛軍彈壓,在歸德權果毅都尉,車駕至,授以帥職。此軍外復有官奴忠孝軍四百五十人。河北潰軍至者皆縱遣之,故城中惟此兩軍。上時召用計事,而不及官奴,故官奴有異心。朝廷知兩人不協,恐生變。三月戊辰,制旨令宰相錫宴省中,和解之。是夜,用撤備,官奴以兵乘之為亂。明日,攻用軍,用敗走被殺,眾下城投水奪船而去者,斯須而盡。官奴在雙門,驅知府女魯歡至,言「汝自車駕到府,上供不給,好醬亦不與,汝罪何辭。」遂以一馬載之。
 令軍士擁至其家,檢其家雜醬凡二十甕,且出所有金具,然後殺之。即提兵入見,言「石盞女魯歡等反,臣殺之矣。」上不得已,就赦其罪,且暴女魯歡之惡。後其姪大安入蔡,上言求湔雪,上復其官,語在烏古論鎬傳。



 禾速嘉兀底代女魯歡為總帥,軍變,官奴無意害兀底,使二卒召之,道官奴有善意。兀底喜,各以金十星與之,同見官奴。二卒復恐受金事泄,亦殺之。



 初,河北潰軍至歸德,糧餉不給。朝庭命孛術魯阿海行總帥府事,以親軍武衛皆隸之。往宿州就食,軍士有不願者,誶語道中,朝廷聞之,使問其故。或言願入京或陳州,阿海請從其願,以券
 給之,軍心稍定。既而令求誶語者,阿海得四人,斬之國子監前,由是諸軍洶洶。二月庚子夜,劫府民武邦傑及蒲察咬住等凡九家,一軍遂散。數日,遂有官奴之變。



 蒲察官奴,少嘗為北兵所虜,往來河朔。後以姦事繫燕城獄,劫走夏津,殺回紇使者得鞍馬資貨,即自拔歸。朝廷以其種人,特恩收充忠孝軍萬戶。此軍月給甚憂,官奴日與群不逞博,為有司所劾。事聞,以其新自河朔來,未知法禁,詔勿問。



 移剌蒲阿攻平陽,官奴請行,論功第一,遷本軍提控,佩金符。三峰山之敗,走襄陽,說宋制使以取鄧州自效,制使信之,至與同燕飲。已而知汴城罷
 攻,復謀北歸。遣移剌留哥入鄧,說鄧帥粘合,稱欲劫南軍為北歸計。留哥以情告粘合,官奴繼以騎卒十餘入城議事,粘合欲就甕城中擒之。官奴知事泄,即馳還,見制得騎兵五百,掠鄧之邊面小城,獲牛羊數百,宋人不疑。官奴掩宋軍得馬三百,至鄧州城下,移書粘合辨理屈直,留馬於鄧而去。乃縛忠孝軍提控姬旺,詐為唐州太守,械送北行,隨營帳取供給,因得入汴。有言其出入南北軍,行數千里而不懾,其智略有可取者,宰相以為然,乃使權副都尉。未幾,提軍數百馳入北軍獵騎中,生挾一回紇而還。遂巡黃陵、八谷等處,劫牛羊糧資甚
 眾,尋轉正都尉。又以軍至黃陵,幾獲鎮州大將,於是中外皆以為可用,遂拜為元帥,統馬軍。



 天興元年十二月,從哀宗北渡。上次黃陵岡,平章白撒率諸將戰,官奴之功居多。及渡河朔,惟官奴一軍號令明肅,秋毫無犯。明年正月,上至歸德。知府石盞女魯歡以軍眾食寡,懼不能給,請於上,令河北潰軍至者就糧於徐、宿、陳三州,親衛軍亦遣出城就食,上不得已從之。乃召諭官奴曰:「女魯歡盡散衛兵,卿當小心。」



 是時,惟官奴忠孝軍四百五十人、馬用軍七百人留府中。用本果毅都尉,上至歸德始升為元帥,又嘗召之謀事,而不及官奴,故官奴始有
 圖用之志。是時,大元將忒木泬攻歸德。官奴既總兵柄,私與國用安謀,欲邀上幸海州。及近侍局直長阿勒根兀惹使用安迴,附奏帖,謂海州可就山東豪傑以圖恢復,且已具舟楫,可通遼東。上覽奏不從。又嘗請上北渡,再圖恢復,女魯歡沮之,自是有異心矣。且一軍倚外兵肆為剽掠,官奴不之禁。於是,左丞李蹊、左右司郎中張天綱、近侍局副使李大節俱為上言官奴有反狀。上竊憂之,以馬軍總領紇石烈阿里合、內族習顯陰察其動靜,與朝臣言及,則曰:「我從官奴微賤中起為大帥,何負而反耶?卿等勿過慮。」阿里合、習顯知官奴漸不能制,反
 泄上意。上亦懼官奴、馬用相圖,因以為亂,命宰執置酒和解之。用撤備。俄官奴乘隙率其軍攻用,用軍敗走。官奴亂殺軍民,以卒五十人守行宮。劫朝官皆聚於都水毛花輦宅,以兵監焉。驅參知政事石盞女魯歡至其家,悉出所有金具,然後殺之。乃遣都尉馬實被甲持刃劫直長把奴申於上前,上初握劍,見實,擲劍於地曰:「為我言於元帥,我左右止有此人,且留侍我。」實不敢迫,逡巡而退。凡殺朝官左丞李蹊已下三百餘人,軍將、禁衛、民庶死者三千。郎中完顏胡魯剌、都事冀禹錫赴水死。



 禹錫字京甫,龍山人。至寧元年進士,仕歷州郡有能聲。歸
 德受兵,禹錫為行院都事,經畫寧禦一府倚重。聞變,或勸以微服免,不從,見害。



 是日蒲暮,官奴提兵入見,言:「石盞女魯歡等反,臣殺之矣。」上不得已,赦其罪,以為樞密副使、權參知政事。



 初,官奴之母,自河北軍潰,北兵得之。至是,上乃命官奴因其母以計請和,故官奴密與忒木泬議和事,令阿里合往言,欲劫上以降。忒木泬信之,還其母,因定和計。官奴乃日往來講議,或乘舟中流會飲。其遣來使者二十餘輩,皆女直、契丹人,上密令官奴以金銀牌與之,勿令還營。因知王家寺大將所在,故官奴畫斫營之策。先是,忠孝軍都統張姓者,謂官奴決欲劫
 上北降,遂率本軍百五十人圍官奴之第,數之曰:「汝欲獻主上,我輩皆大朝不赦者,使安歸乎?」官奴懼,乃以其母出質,云:「汝等若以吾母自北中來,疑我與北有謀,即殺之。我不恨。」張意稍解,既以好語與之約曰:「果如參政所言,今後勿復言講和,北使至,即當殺之。」官奴曰:「殺亦可,不殺亦可,奏而殺之亦可。」張乃退,官奴即聚軍北草場,自言無反情,今勿復相疑也。遂畫斫營之策。



 五月五日,祭天。軍中陰備火槍戰具,率忠孝軍四百五十人,自南門登舟,由東而北,夜殺外提邏卒,遂至王家寺。上御北門,系舟待之。慮不勝則入徐州而遁。四更接戰,忠孝
 初小卻。再進,官奴以小船分軍五七十出柵外,腹背攻之。持火槍突入,北軍不能支,即大潰,溺水死者凡三千五百餘人,盡焚其柵而還。遂真拜官奴參知政事、兼左副元帥,仍以御馬賜之。



 槍制,以敕黃紙十六重為筒,長二尺許,實以柳炭、鐵滓、磁未、硫黃、砒霜之屬,以繩繫槍端。軍士各懸小鐵罐藏火,臨陣燒之,焰出槍前丈餘,藥盡而筒不損。蓋汴京被攻已嘗得用,今復用之。



 兵既退,官奴入亳州,留習顯總其軍。上御照碧堂,無一人敢奏對者,日悲泣云:「自古無不亡之國、不死之君,但恨我不知用人,故為此奴所囚耳。」於是,內局令宋乞奴與奉御
 吾古孫愛實、納蘭忔答、女奚烈完出密謀誅官奴。或言,官奴密令兀惹計構國用安,脅上傳位,恢復山東。事不成則獻上於宋,自贖反復之罪。官奴以己未往亳州,辛酉,召之還,不至。再召,乃以六月己卯還。上諭以幸蔡事,官奴憤憤而出,至於扼腕頓足,意趣叵測。上決意欲誅之,遂與內侍宋乞奴處置,令裴滿抄合召宰相議事,完出伏照碧堂門間。官奴進見,上呼參政,官奴即應。完出從後刺其肋,上亦拔劍斫之。官奴中創投階下以走,完出叱忔答、愛實追殺之。



 忠孝軍聞難,皆擐甲,完出請上親撫慰之。名呼李泰和,授以虎符,使往勞軍,因召范陳
 僧、王山兒、白進、阿里合。進先至,殺之堂下。阿里合中路覺其事,悔發之晚,為亂箭所射而死。乞奴、愛實、忔答皆授節度使、世襲千戶,完出兼殿前右衛將軍,范陳僧、王山兒忠孝軍元帥。於是,上御雙門,赦忠孝軍以安反側。除崔立不赦外,其餘常所不原者咸赦之。



 初,官奴解睢陽之圍,侍從官屬久苦飢窘,聞蔡州城池堅固、兵眾糧廣,咸勸上南幸。惟官奴以嘗從點檢內族斜烈過蔡,知其備禦不及睢陽,力爭以為不可,故號於眾曰:「敢言南遷者斬!」眾以官奴為無君,諷上早為計,會其變,遂以計誅之。後遣烏古論蒲鮮如蔡,還言其城池兵糧果不足
 恃,上已在道,無可奈何。及蔡受兵,始悔不用官奴之言,特詔尚書省月給其母妻糧,俾無失所。



 習顯既黨官奴,一日率忠孝軍劫官庫金四千兩。上命歸德治中溫特罕道僧、帥府經歷把奴申鞫問,顯伏罪下獄。官奴變,顯脫走,殺總領完顏長樂於宮門,殺道僧、奴申於其家,遂奔亳。及官奴伏誅,詔點檢阿勒根阿失答即亳州斬顯及忠孝軍首領數人。兀惹使用安未還,伺於中路,數其罪殺之。



 內族慶山奴,名承立,字獻甫,統軍使拐山之子,平章白撒之從弟也。為人儀觀甚偉,而內恇怯無所有。至寧初,
 宣宗自彰德赴闕,慶山奴迎見于臺城。宣宗喜,遣先還中都觀變。宣宗既即位,以承立為西京副留守,權近侍局直長,進官五階,賜錢五千貫,且詔曰:「汝雖授此職,姑留侍朕,遇闕赴之,仍給汝副留守祿。此朕特恩,宜知悉也。」貞祐初,遷武衛軍副都指揮使,兼提點近侍局。胡沙虎專權僭竊,嘗為宣宗言之,後胡沙虎伏誅,慶山奴愈見寵幸,以為殿前右副都點檢。三年,大元兵圍中都,詔以慶山奴為宣差便宜都提控,率所募兵往援。俄為元帥右都監,行帥府事,兼前職。四年,知慶陽府事,兼慶原路兵馬都總管,以所獲馬駝進,詔諭曰:「此皆軍士所得,
 即以與之可也,朕安用哉。後勿復進。」因令遍諭諸道帥府焉。



 興定元年正月,大元兵及夏人回經寧州,慶山奴以兵邀擊敗之,以功進元帥左都監,兼保大軍節度使,行帥府事於鄜州。二年五月,夏人率步騎三千由葭州入寇,慶山奴以兵逆之,戰于馬吉峰,殺百餘人,斬酋首二級,生擒數十人,獲馬三十餘疋。三年四月,夏人據通秦寨,慶山奴遣提控納合買住討之。夏人以步騎二萬逆戰,買住擊敗之,夏人由葭盧川遁去,凡斬首八百級。俄而復攻寨據之,慶山奴率兵與戰,斬首千級,復其寨。詔賜慶山奴金帶一,將士賞賚有差。四年四月,破夏
 兵於宥州,斬首千餘級,遂圍神堆府。慶山奴四面攻之,士卒方登陴,援兵大至,復擊走之。



 正大四年,李全據楚州,詔以慶山奴為元帥,同總帥完顏訛可將兵守盱眙,且令城守勿出戰。已而全軍盱眙界,二帥迎敵大敗,死者萬餘人,委棄資杖甚眾,時軍無見糧,轉輸不繼,民疲奔命,愁歎盈路。諸相不肯正言,樞密判官白華拜章乞斬之以謝天下,不報。降為定國軍節度使,又以受賂奪一官。



 八年正月,鳳翔破,兩行省徙京兆居民於河南,令慶山奴以行省守之。時京兆行省止有病卒八百、瘦馬二百,承立懼不能守,屢上奏請還。每奏一帖,附其兄白
 撒一書,令為地,朝廷不許。十月,慶山奴棄京兆還朝,留同知乾州軍州事、保義軍提控茍琪守之。慶山奴行至閿鄉,哀宗遣近侍裴滿七斤授以黃陵岡從宜,不聽入見。未幾,代徒單兀典行省事於徐州。九年正月,自徐引兵入援,選精銳一萬五千,與徐帥完顏兀論統之,將趨歸德。義勝軍總領侯進、杜政、張興等率所部三千人降大兵。慶山奴留睢州三日不敢進,聞大兵且至,懼此州不可守,退保歸德。二月,行次楊驛店,遇小乃泬軍。遂潰。兀論戰死,慶山奴馬躓被擒,惟元帥郭恩、都尉烏林答阿督率三百餘人走歸德。大兵以一馬載慶山奴,擁迫
 而行,道中見真定史帥,承立問曰:「君為誰?」史帥言:「我真定五路史萬戶也。」承立曰:「是天澤乎?」曰:「然。」曰:「吾國已殘破,公其以生靈為念。」及見大帥忒木泬,誘之使招京城,不從,又偃蹇不屈。左右以刀斫其足折,亦不降,即殺之。議者以承立累敗不能解其軍職,死有餘責,而能以死報國,亦足稱云。



 初,睢州刺史張文壽聞大兵將至,遷旁縣居民入城,大聚芻粟,然無固守意,日夜謀走以自便。既而,聞承立人援,即以州事付其僚佐,託以應援徐兵,夜啟關契家走歸德,慶山奴以為行部郎中,死楊驛。俄大兵圍睢州,以無主將,故殘破之甚也。



 兀論,丞相賽不
 之姪,元光間例以諸帥為總領,兀論以丞相故獨不罷。金朝防近族而用疏屬,故白撒、承立、兀論輩皆腹心倚之。



 贊曰:官奴素行反側,倏南倏北,若龍斷然。哀宗一旦倚為腹心,終為所制,照碧之處,何異幽囚,其事與梁武、侯景大同而小異。徒單兀典、慶山奴為將皆貪,宜數取敗。女魯歡無大失行,而死於官奴,哀宗猶暴其罪,冤哉。



\end{pinyinscope}