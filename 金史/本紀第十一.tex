\article{本紀第十一}

\begin{pinyinscope}

 章宗三



 三年春正月己亥朔,日有食之。辛丑,宋、夏遣使來賀。癸卯,諭有司:「凡館接伴并奉使者,毋以語言相勝,務存大體。奉使者亦必得其人乃可。」乙卯,詔罷講議所。丙辰,如城南春水。丁巳,併上京、東京兩路提刑司為一,提刑使、副兼安撫使、副,安撫專掌教習武事,毋令改其本俗。己未,以都南行宮名建春。甲子,至自春水。乙丑,宋主以祖
 母喪,遣使告哀。



 二月己巳朔,幸建春宮。辛巳,諭宰臣曰:「自今內外官有闕,有才能可任者,雖資歷未及,亦具以聞。雖親故,毋有所避。」以武衛軍都指揮使烏林答天益等為宋弔祭使。甲申,至自建春宮。丙戌,斜出內附。辛卯,平章政事粘割斡特剌薨。



 三月戊戌,以禮部尚書張暐為御史大夫。壬寅,復榷醋。甲寅,如萬寧宮。丁巳,敕隨處盜賊,毋以強為竊,以多為少,以有為無。嘯聚三十人以上奏聞,違者杖百。丙寅,高麗王王皓以弟晫權國事,遣使奉表來告。夏四月戊戌朔,諭有司:「宰相遇雨,可循殿廡出入。」丙申,諭御史臺曰:「隨朝大小官雖有才能,率多
 茍簡,朕甚惡之,其察舉以聞。提刑司所察廉能污濫官,皆當殿奏,餘事可轉以聞。」以侍御史孫俁為宣問高麗王王皓使。



 五月庚子,右宣徽使張汝方以漏泄廷議,削官兩階。壬寅,射柳、擊球,縱百姓觀。戊申,以客省使移剌郁為夏國生日使。甲子,參知政事楊伯通表乞致仕,不許。秋七月丙午,幸香山。己酉,如萬寧宮。甲寅,還宮。



 八月辛未,獵于近郊。癸酉,獵于香山。戊寅,如萬寧宮。庚辰,以護衛石和尚為押軍萬戶,率親軍八百人、武衛軍千六百人戍西北路。癸未,還宮。宋遣使來報謝。



 九月丙申朔,天壽節,宋、夏遣使來賀。以中都路都轉運使孫鐸等為
 賀宋生日使。乙巳,獵於近郊。庚戌,參知政事楊伯通再表乞致政,不許。戊午,木波進馬。冬十月庚午,獵于近郊。癸未,行樞密院言斜出等請開榷場於轄里裊,從之。丁亥,定官民存留見錢之數,設回易務,更立行用鈔法。



 十一月丁酉,樞密使兼平章政事襄至自軍,癸卯,以為尚書左丞相,監修國史。丁未,以太常卿楊庭筠等為賀宋正旦使。戊申,詔獎諭樞密副使夾谷衡以下將士。辛亥,定屬託法。定軍前官吏賞格。以邊事定,詔中外,減死罪,徒已下釋之。賜左丞相襄以下將士金幣有差。甲寅,冬獵。



 十二月甲子朔,獵於酸棗林。大風寒,罷獵,凍死者
 五百餘人。己巳,還都。丙戌,尚書右丞嵒罷。高麗權國事王晫遣使奉表來告。



 四年春正月癸巳朔,宋、高麗、夏遣使來賀。乙巳,尚書左董師中致仕。辛酉,監察御史姬端修以妄言下吏。尚書左丞相襄為司空,職如故。樞密副使夾谷衡為平章政事,封英國公。前知濟南府事張萬公起復為平章政事,封壽國公。楊伯通為尚書左丞。簽樞密院事完顏匡為尚書右丞。



 二月乙丑,如建春宮春水。己巳,還宮。庚午,御宣華門,觀迎佛。辛未,如建春宮。赦姬端修罪,令居家俟命。司空襄言,西南路招討使僕散揆治邊有功,召赴闕,以
 知興中府事紇石烈子仁代之。壬申,諭有司:「自三月一日為始,每旬三品至五品官各一人轉對,六品亦以次對。臺諫勿與,有應奏事,與轉對官相見,無面對者上章亦聽。」乙亥,還宮。戊寅,如建春宮。庚辰,上諭點檢司曰:「自蒲河至長河及細河以東,朕常所經行,官為和買其地,令百姓耕之,仍免其租稅。」甲申,還宮。乙酉,以西南路招討使僕散揆為參知政事。起姬端修為太學博士。如建春宮。戊子,還宮。



 三月丁酉,同判大睦親府事宗浩為樞密使,封崇國公。己亥,如建春宮。遣使冊王晫為高麗國王。戶部尚書孫鐸、郎中李仲略、國子祭酒趙忱始轉對
 香閣。丁未,敕尚書:「官員必須改除者議之,其月日淺毋數改易。」乙卯,尚書省奏減親軍武衛軍額及太學女直、漢兒生員,罷小學官及外路教授。詔學校仍舊,武衛軍額再議,餘報可。司空襄、右丞匡、參知政事揆請罷諸路提點刑獄,從之。戊午,雨雹。夏四月癸亥,改提刑司為按察使司。戊辰,如萬寧宮。壬申,左丞楊伯通致仕。御史大夫張暐以奏事不實,追一官,侍御史路鐸追兩官,俱罷之。姬端修杖七十,贖。壬午,英王從憲進封瀛王。詔同州、許州節度使罷兼陜西、河南副統軍。



 五月壬辰朔,以旱,下詔責躬,求直言,避正殿,減膳,審理冤獄,命奏事於
 泰和殿。戊戌,命有司望祭嶽瀆禱雨。己亥,應奉翰林文字陳載言四事:其一,邊民苦于寇掠;其二,農民困于軍須;其三,審決冤滯,一切從寬,茍縱有罪;其四,行省官員,例獲厚賞,而沿邊司縣,曾不霑及,此亦干和氣,致旱災之所由也。上是之。壬寅,以兵部郎中完顏撒里合為夏國生日使。戊申,宰臣以京畿雨,率百官請御正殿,復常膳。不從。尚書省奏上更定給發虎符制,著于令。庚戌,諭宰臣曰:「諸路旱,或關執政。今惟大興、宛平兩縣不雨,得非其守令之過歟?」司空襄、平章政事萬公、參知政事揆上表待罪。上以罪己答之,令各還職。詔頒銅杖式。壬子,
 祈雨于太廟。乙卯,更定軍功賞格。戊午,司空襄以下再請御正殿,復常膳。不從。庚申,平章政事夾谷衡薨。以宿直將軍徒單仲華為橫賜夏國使。



 六月丁卯,雨。司空襄以下復表請御正殿,復常膳。從之。甲戌,以雨足,命有司報謝于太廟。丁丑,右補闕楊庭秀言:「自轉對官外,復令隨朝八品以上、外路五品以上及出使外路有可言者,並許移檢院以聞。則時政得失,民間利病,可周知矣。」從之。己卯,以雨足,報祭社稷。辛巳,遣官報祀嶽瀆。癸未,奉職醜和尚進《浮漏水稱影儀簡儀圖》。命有司依式造之。丁亥,定宮中親戚非公事傳達語言、轉遞諸物及書簡
 出入者罪。



 七月甲辰,更定尚藥、儀鸞局學者格。辛亥,敕宣徽院官,天壽節凡致仕宰執悉召與宴。丙辰,以久雨,令大興府祈晴。



 八月己巳,獵于近郊。壬申,獵于香山。甲戌,以皇嗣未立,命有司祈于太廟。丁丑,獵于近郊。庚辰,還宮。



 九月庚寅朔,天壽節,宋、高麗、夏遣使來賀。己亥,如薊州秋山。己未,以知東平府事僕散琦等為賀宋生日使。冬十月丙寅,至自秋山。壬午,初定百官休假。甲申,初置審官院。



 十一月乙未,敕京、府、州、縣設普濟院,每歲十月至明年四月設粥,以食貧民。丙申,平章政事張萬公表乞致政,不許。庚戌,命有司祈雪。甲寅,定護衛改充奉
 御格。以知濟南府事范楫等為賀宋正旦使。



 十二月己未,除授文字初送審官院。辛酉,更定考試隨朝檢、知法條格。右補闕楊庭秀請類集太祖、太宗、世宗三朝聖訓,以時觀覽。從之,仍詔增熙宗為四朝。癸未,更定科舉法。增設國史院女直、漢人同修史各一人。定親軍及承應人退閑遷賞格。是月,淑妃李氏進封元妃。



 五年春正月戊子朔,宋、高麗、夏遣使來賀。乙未,以尚書省言:「會試取策論、詞賦、經義不得過六百人,合格者不及其數,則闕之。」丙申,如春水。庚子,命左右司五日一轉奏事。辛丑,諭點檢司:「車駕所至,仍令百姓市易。」庚戌,定
 猛安謀克軍前怠慢罷世襲制。



 二月辛未,至自春水。辛巳,有司奏:「應奉翰林文字溫迪罕天興與其兄直學士思齊同僚學士院,定撰制誥文字,合無乃避?」詔不須避,仍為定制。



 閏月癸卯,定進納粟補官之家存留弓箭制。丁未,上與宰臣論置相曰:「徒單鎰,朕志先定。賈鉉如何?」皆曰:「知延安府事孫即康可。」平章政事萬公亦曰:「即康及第,先鉉一榜。」上曰:「至此安問榜次,特以賈才可用耳!」尚書省奏:「右補闕楊庭秀言,乞令尚書省及第左右官一人,應入史事者編次日歷,或一月,或一季,封送史院。」上是其言,仍令送著作局潤色,付之。



 三月庚申,大睦親府進
 重修《玉牒》。平章政事張萬公乞致政,不許。壬戌,命有司禱雨。癸亥,雨。戶部尚書孫鐸、大理卿完顏撒剌、國子司業蒙括仁本召對便殿。丙寅,如萬寧宮。戊辰,定妻亡服內婚娶聽離制。親王、宰執、百官再請上尊號。不許。庚午,以知大興府事卜為御史大夫。丙子,尚書省奏:「擬同知商州事蒲察西京為濟南府判官。」上曰:「宰相豈可止徇人情,要當重惜名爵。此人不堪,朕常記之,止與七品足矣!」庚辰,以上京留守徒單鎰為平章政事,封濟國公。辛巳,定本國婚聘禮制。改山東東路舊皇城猛安名曰合里哥阿鄰。



 四月丙戌朔,文武百官再請上尊號。不許,丙
 午,尚書省進《律義》。



 五月乙卯朔,定猛安謀克鬥毆殺人遇赦免死罷世襲制。以雨足,遣使報祭社稷。丁巳,定策論進士及承廕人試弓箭格。戊午,敕來日重五拜天,服公裳者拜禮仍舊,諸便服者並用女直拜。己未,敕諸路按察司,糾察親民官以大杖箠人者。乙亥,親王、文武百官、六學各上表請上尊號。不許。庚辰,地震。詔定進納官有犯決斷法。



 六月乙巳,遣有司祈晴,望祭嶽瀆。



 七月乙卯朔,以晴,遣官望祭嶽鎮海瀆。癸亥,定居祖父母喪婚娶聽離法。初置蒲思衍群牧。辛未,平章政事萬公特則賜告兩月。甲戌,獵于近郊。



 八月壬辰,幸香山。乙未,至自香
 山。丁未,敕審官院奏事,其院官皆許升殿。戊申,更定鎮防軍犯徒配役法。



 九月甲寅朔,天壽節,宋、高麗遣使來賀。戊午,命樞密使宗浩、禮部尚書賈鉉佩金符行省山東等路括地。己未,尚書省奏:「西北路招討使獨吉思忠言,各路邊堡墻隍,西自坦舌,東至胡烈公,幾六百里,向以起築匆遽,並無女墻副隄。近令修完,計工七十五萬,止役戍軍,未嘗動民,今已畢功。」上賜詔獎諭。修《玉牒》成。定皇族收養異姓男為子者徒三年,姓同者減二等,立嫡違法者徒一年。癸亥,如薊州秋山。冬十月庚寅,至自秋山。庚子,風霾。宋遣使來告哀。辛丑,集百官于尚書省,問:「
 間者亢旱,近則久陰,豈政有錯謬而致然歟?」各以所見對。以禮部郎中劉公憲為高麗生日使。丁未,獵于近郊。以宿直將軍完顏觀音奴為夏國生日使。



 十一月癸丑朔,日有食之。乙卯,以國史院編修官呂卿雲為左補闕兼應奉翰林文字。審官院以資淺駁奏,上諭之曰:「明昌間,卿雲嘗上書言宮掖事,辭甚切直,皆他人不能言者,卿輩蓋不知也。臣下言事不令外人知,乃是謹密,正當顯用,卿宜悉之。」以工部尚書烏古論誼等為宋弔祭使。初定品官過闕則下制。己巳,宋復遣使來告哀。辛未,以殿前右副點檢紇石烈忠定為賀宋正旦使。



 十二月
 癸未朔,詔改明年為泰和元年。以河南路統軍使充等為宋弔祭使。乙未,定管軍官受所部財物輒放離役及令人代役法。辛丑,詔宮籍監戶,百姓自願以女為婚者聽。癸卯,定造作不如法,三年內有損壞者罪有差。



 泰和元年正月壬子朔,宋、高麗、夏遣使來賀。壬戌,宋遣使獻先帝遺留物。己巳,以太府監孫復言:「方今在仕者三萬七千餘員,而門廕補敘居三之二,諸司待闕,動至累年。蓋以補廕猥多,流品混淆,本未相舛,至於進納之人,既無勞績,又非科第,而亦廕及子孫,無所分別,欲流之清,必澄其源。」乃更定廕敘法而頒行之。尚書省奏:「今
 杖式輕細,民不知畏,請用大杖。」詔不許過五分。庚午,如長春宮春水。辛未,上以方春,禁殺含胎兔,犯者罪之,告者賞之。甲戌,初命文武官官職俱至三品者許贈其祖。



 二月壬辰,去造土茶律。丁未,至自春水。



 三月乙丑,夏國遣使來謝。壬申,幸天長觀。癸酉,如萬寧宮。乙亥,宋遣使來報謝。丁丑,更定鎮防千戶謀克放老入除格。辛巳,敕官司、私文字避始祖以下廟諱小字,犯者論如律。夏四月甲辰,詔諭契丹人戶,累經簽軍立功者,官賞恩例與女直人同,仍許養馬、為吏。



 五月甲寅,擊球于臨武殿,令都民縱觀。丙辰,樞密使宗浩罷。壬戌,幸玉泉山。戊寅,削
 尊長有罪卑幼追捕律。以直東上閣門劉頍為橫賜高麗使。



 六月己卯,幸香山。乙酉,平章政事張萬公乞致仕。不許。辛卯,祈雨于北郊。己亥,用尚書省言,申明舊制,猛安謀克戶每田四十畝,樹桑一畝。毀樹木者有禁,鬻地土者有刑。其田多污萊,人戶闕乏,並坐所臨長吏。按察司以時勸督,有故慢者量決罰之,仍減牛頭稅三之一。敕尚書省舉行風俗奢僭之禁。乙巳,初許諸科征鋪馬、黃河夫、軍須等錢,折納銀一半,願納錢鈔者聽。丁未,詔有司修蓮花漏。



 七月辛酉,禁放良人不得應諸科舉,子孫不在禁限。甲子,諭刑部官,凡上書人言及宰相者
 不得申省。乙丑,更定右選注縣令丞簿格。己巳,初禁廟諱同音字。



 八月庚辰,初命戶絕者田宅以三分之一付其女及女孫。戊子,特改授司空襄河間府路算注海世襲猛安。乙未,至自萬寧宮。丙申,宋遣使來報謝。壬寅,制猛安謀克並隸按察司,監察御史止按部糾舉,有罪則併坐監臨之官。詔推排西、北京、遼東三路人戶物力。



 九月戊申朔,天壽節,宋、高麗、夏遣使來賀。更定贍學養士法:生員,給民佃官田人六十畝,歲支粟三十石;國子生,人百八畝,歲給以所入,官為掌其數。以右宣徽使徒單懷忠等為賀宋生日使。甲寅,如秋山。丙子,至自秋山。冬
 十月乙酉,祫享于太廟。戊子,平章政事張萬公乞致仕,不許。壬辰,御史臺奏:「在制,按察司官比任終遣官考核,然後尚書省命官覆察之。今監察御史添設員多,宜分路巡行,每路女直、漢人各一人同往。」從之,仍敕分四路。丙申,御史大夫卞乞致仕,不許。戊戌,以武衛軍都揮指使司判官納合鉉為高麗生日使。壬寅,敕有司:「購遺書宜尚其價,以廣搜訪。藏書之家有珍惜不願送官者,官為謄寫。畢復還之,仍量給其直之半。」甲辰,以刑部員外郎完顏綱為夏國生日使。



 十一月庚戌,司空襄以下文武百官復請上尊號。不許。辛亥,敕尚書省:「凡役眾勞民之
 事,勿輕行之。」丁巳,諭工部曰:「比聞懷州有橙結實,官吏檢視,已嘗擾民,今復進柑,得無重擾民乎?其誡所司,遇有則進,無則已。」庚申,以殿前右衛將軍紇石烈七斤等為賀宋正旦使。



 十二月辛巳,敕改原廟春秋祭祀稱朝獻。司空襄以下復請上尊號。詔不允,仍斷來章。丁酉,司空襄等進《新定律令敕條格式》五十二卷。辛丑,詔頒行之。壬寅,獵于近郊。乙巳,初定廉能官升注格。



 二年春正月丁未朔,宋、高麗、夏遣使來賀。乙卯,始朝獻於衍慶宮。庚申,幸芳苑觀燈。癸酉,歸德軍節度副使韓琛以強市民布帛,削一官,罷之。甲戌,如建春宮。



 二月戊
 戌,初置內侍寄祿官。乙巳,還宮。



 三月甲寅,初置宮苑司都、同監各一人。甲子,蔡王從彞母充等太師卒,詔有司定喪禮葬儀,事載《從彞傅》。



 四月庚辰,幸升國長公主第問疾。己亥,定遷三品官格。復撲買河濼法。辛丑,諭御史臺,諸訴事于臺,當以實上聞,不得輒稱察知。癸卯,如萬寧宮。命有司祈雨。



 五月甲辰朔,日有食之。戊申,如泰和宮。辛亥,初薦新于太廟。壬戌,諭有司曰:「金井捺缽不過二三日留,朕之所止,一涼廈足矣。若加修治,徒費人力。其籓籬不急處,用圍幕可也。」甲子,更泰和宮曰慶寧,長樂川曰雲龍。己巳,敕御史臺,京師拜廟及巡幸所過
 州縣,止令灑掃,不得以黃土覆道,違者糾之。



 六月辛卯,諭尚書省,諸路禾稼及雨多寡,令州郡以聞。



 七月辛亥,有司奏還宮日請用黃麾仗。不許。乙卯,朝獻於衍慶宮。



 八月丙申,鳳凰見于磁州武安縣鼓山石聖臺。丁酉,還宮。皇太生。



 九月壬寅朔,天壽節,宋、高麗、夏遣使來賀。甲寅,以拱衛直都指揮使完顏瑭等為賀宋生日使,且戒之曰:「兩國和好久矣,不宜爭細故,傷大體。」癸亥,以皇子生,親謝南北郊。庚午,封皇子為葛王。冬十月戊寅,報謝于太廟及山陵。甲申,以鳳凰見,詔中外。丙戌,獵近郊。壬辰,遣尚輦局副使李仲元為高麗國生日使。以宿直
 將軍紇石烈毅為夏國生日使,瀛王府司馬獨吉溫為橫賜使。



 十一月甲辰,更定德運為土,臘用辰。以西京留守宗浩為樞密使。戊申,以更定德運,詔中外。庚申,初命外官三品到任進表稱謝。甲子,幸玉虛觀,遣使報謝于太清宮。



 十二月癸酉,以皇子晬日,放僧道戒牒三千。以武安軍節度使徒單公弼等為賀宋正旦使。戊寅,冬獵。庚辰,報謝于高禖。丁酉,還都。



 閏月庚戌,司空襄薨。癸丑,初命監察御史非特旨不許舉官。辛酉,遣使報謝于北嶽。定人戶物力隨時推收法。丁卯,遣使報謝於長白山。冬,無雪。



 三年春正月辛未朔,宋、高麗、夏遣使來賀。癸酉,遣官祈雪于北嶽。丁丑,朝獻于衍慶宮。己卯,以樞密使宗浩為尚書右丞相,右丞完顏匡為左丞,參知政事僕散揆為右丞,御史中丞孫即康、刑部尚書賈鉉並為參知政事。庚辰,如建春宮。



 二月癸丑,還宮。甲子,定諸職官省親拜墓給假例。



 三月壬申,平章政事張萬公致仕。庚辰,如萬寧宮。丁亥,定從人銅牌賣毀罪賞制。庚寅,定職官應遷三品格,刺史以上及隨朝資歷在刺史以上身故者,每半年一次敷奏。甲午,如玉泉山。丙申,以殿前都點檢僕散端為御史大夫。



 四月乙巳,禘于太廟。敕點檢司:「致仕
 官入宮,年高艱于步履者,並聽策杖,仍令舍人護衛扶之。」丁巳,敕有司祈雨,仍頒土龍法。己未,命吏部侍郎李炳、國子司業蒙括仁本、知登聞檢院喬宇等再詳定《儀禮》。庚申,諭省司:「宮中所用物,如民間難得,勿強市之。」癸亥,尚書省奏,遣官分路覆實御史所察事。



 五月壬申,以重五,拜天,射柳,上三發三中。四品以上官侍宴魚藻殿。以天氣方暑,命兵士甲者釋之。丙戌,以定律令、正土德、鳳凰來、皇嗣建,大赦。辛卯,皇子葛王薨。壬辰,定擅增減宮門鎖鑰罪。丙申,作太極宮。



 六月己亥,太白晝見。壬寅,詔選聰明方正之士為修起居注。又詰點檢司,諸親軍
 所設教授及授業人若干,其為教何法,通大義者幾人,各具以聞。戊申,定職官追贈法,惟嘗犯贓罪者不在追贈之列。壬戌,遣官行視中都田禾雨澤分數。



 七月壬申,朝獻于衍慶宮。乙亥,定大臣薨百官奉慰禮。庚辰,獵于近郊。丁亥,上諭宰臣:「凡奏事,朕欲徐思或如己者,若除授事,可俟三五日再奏,餘並二十日奏之。」



 八月丙辰,還宮。庚申,命編修官左容充宮教,賜銀、幣。



 九月丙寅朔,天壽節,宋、高麗、夏遣使來賀。壬申,以刑部尚書承暉等為賀宋生日使。戊子,以萬寧宮提舉司隸工部。壬辰,詔定千戶謀克受隨處捕盜官公移,盜急,不即以眾應之者
 罪有差。召右丞相宗浩還朝。冬十月戊戌,日將暮,赤如赭。己亥,大風。甲辰,申、酉間天大赤,夜將旦亦如之。壬子,右丞僕散揆至自北邊,丙辰,召至香閣慰勞之。以尚食局使師孝為高麗生日使。庚申,尚書左丞完顏匡等進《世宗實錄》。上降座,立受之。壬戌,以薊州刺史完顏太平為夏國生日使。奉御完顏阿魯帶以使宋還,言宋權臣韓侂胄市馬厲兵,將謀北侵。上怒,以為生事,笞之五十,出為彰德府判官。及淮平陷,乃擢為安國軍節度副使。丁卯,諭尚書省:「士庶陳言皆從所司以聞,自今可悉令詣闕,量與食直,仍給官居之。其言切直及繫利害重
 者,並三日內奏聞。」



 十一月辛未,以簽樞密院事獨吉思忠等為賀宋正旦使。丁丑,冬獵,以獲兔,薦山陵。甲午,詔監察等察事可二年一出。



 十二月庚子,諭宰臣曰:「賀正宋使且至,可令監察隨之,以為常。」壬寅,還都。己酉,賜天長觀額為太極宮。辛亥,詔諸親王、公主每歲寒食、十月朔聽朝謁興、裕二陵,忌辰亦如之。癸丑,詔遣監察御史分按諸路,所遣者女直人,即以漢人朝臣偕,所遣者漢人,即以女直朝臣偕。戊午,敕行宮名曰光春,其朝殿曰蘭皋,寢殿曰煇寧。



\end{pinyinscope}