\article{列傳第六十四}

\begin{pinyinscope}

 文藝下



 ○趙渢周昂王庭筠劉昂李經劉從益呂中孚張建附李純甫王鬱宋九嘉龐鑄李獻能王若虛王元節麻九疇李汾元德明子好問



 趙渢,字文孺,東平人。大定二十二年進士,仕至禮部郎中。性沖淡,學道有所得。尤工書,自號「黃山」。趙秉文云:「渢之正書體兼顏、蘇,行草備諸家體,其超放又似楊凝式,當處蘇、黃伯仲間。」黨懷英小篆,李陽冰以來鮮有及者,時人以渢配之,號曰「黨趙」。有《黃山集》行於世。



 周昂,字德卿,真定人。父伯祿字天錫,大定進士,仕至同知沁南軍節度使。昂年二十四擢第。調南和簿,有異政。遷良鄉令,入拜監察御史。路鐸以言事被斥,昂送以詩,語涉謗訕,坐停銓。久之,起為隆州都軍,以邊功復召為三司官。大安兵興,權行六部員外郎。



 其甥王若虛嘗學
 於昂,昂教之曰:「文章工於外而拙於內者,可以驚四筵而不可以適獨坐,可以取口稱而不可以得首肯。」又云:「文章以意為主,以言語為役,主強而役弱則無令不從。今人往往驕其所役,至跋扈難制,甚者反役其主,雖極辭語之工,而豈文之正哉。」



 昂孝友,喜名節,學術醇正,文筆高雅,諸儒皆師尊之。既歷臺省,為人所擠,竟坐詩得罪,謫東海上十數年。始入翰林,言事愈切。出佐三司非所好,從宗室承裕軍。承裕失利,跳走上谷,眾欲徑歸,昂獨不從,城陷,與其從子嗣明同死於難。嗣明字晦之。



 王庭筠,字子端,遼東人。生未期,視書識十七字。七歲學
 詩,十一歲賦全題。稍長,涿郡王翛一見,期以國士。登大定十六年進士第。調恩州軍事判官,臨政即有聲。郡民鄒四者謀為不軌,事覺,逮捕千餘人,而鄒四竄匿不能得。朝廷遣大理司直王仲軻治其獄,庭筠以計獲鄒四,分別詿誤,坐預謀者十二人而已。再調館陶主簿。



 明昌元年三月,章宗諭旨學士院曰:「王庭筠所試文,句太長,朕不喜此,亦恐四方傚之。」又謂平章張汝霖曰:「王庭筠文藝頗佳,然語句不健,其人才高,亦不難改也。」四月,召庭筠試館職,中選。御史臺言庭筠在館陶嘗犯贓罪,不當以館閣處之,遂罷。乃卜居彰德,買田隆慮,讀書黃華
 山寺,因以自號。是年十二月,上因語及學士,歎其乏材,參政守貞曰:「王庭筠其人也。」三年,召為應奉翰林文字,命與秘書郎張汝方品第法書、名畫,遂分入品者為五百五十卷。



 五年八月,上顧謂宰執曰:「應奉王庭筠,朕欲以詔誥委之,其人才亦豈易得。近黨懷英作《長白山冊文》,殊不工。聞文士多妒庭筠者,不論其文,顧以行止為訾。大抵讀書人多口頰,或相黨。昔東漢之士與宦官分朋,固無足怪。如唐牛僧孺、李德裕,宋司馬光、王安石,均為儒者,而互相排毀何耶。」遂遷庭筠為翰林修撰。



 承安元年正月,坐趙秉文上書事,削一官,杖六十,解職,語在
 秉文傳。二年,降授鄭州防禦判官。四年,起為應奉翰林文字。泰和元年,復為翰林修撰,扈從秋山,應制賦詩三十餘首,上甚嘉之。明年,卒,年四十有七。上素知其貧,詔有司賻錢八十萬以給喪事,求生平詩文藏之秘閣。又以御製詩賜其家,其引云:「王遵古,朕之故人也。乃子庭筠,復以才選直禁林者首尾十年,今茲云亡,玉堂、東觀,無復斯人矣。」



 庭筠儀觀秀偉,善談笑,外若簡貴,人初不敢與接。既見,和氣溢於顏間,殷勤慰藉如恐不及,少有可取極口稱道,他日雖百負不恨也。從游者如韓溫甫,路元亨、張進卿,李公度,其薦引者如趙秉文、馮璧、李純
 甫,皆一時名士,世以知人許之。為文能道所欲言,暮年詩律深嚴,七言長篇尤工險韻。有《藂辨》十卷,文集四十卷。書法學米元章,與趙渢、趙秉文俱以名家,庭筠尤善山水墨竹云。



 子曼慶,亦能詩並書,仕至行省右司郎中,自號「淡游」云。



 劉昂,字之昂,興州人。大定十九年進士。曾、高而下七世登科。昂天資警悟,律賦自成一家,作詩得晚唐體,尤工絕句。李純甫《故人外傳》云,昂早得仕,年三十三為尚書省掾,調平涼路轉運副使。時術士有言昂官止五品,昂不信。俄以母憂去職,連蹇十年,卜居洛陽,有終焉之志。
 有薦其才於章宗者,泰和初,自國子司業擢為左司郎中。會掌書大中與賈鉉漏言除授事,為言者所劾,獄辭連昂。章宗震怒。一時聞人如史肅、李著、王宇、宗室從郁皆譴逐之,鉉尋亦罷政。昂降上京留守判官,道卒,竟如術者之言。



 李經,字天英,錦州人。作詩極刻苦,喜出奇語,不蹈襲前人。李純甫見其詩曰:「真今世太白也。」由是名大震。再舉不第,拂衣去。南渡後,其鄉帥有表至朝廷,士大夫識之曰:「此天英筆也。」朝議以武功就命倅其州,後不知所終。



 劉從益,字雲卿,渾源人。其高祖捴,天會元年詞賦進士,
 子孫多由科第入仕。從益登大安元年進士第,累官監察御史,坐與當路辨曲直,得罪去。久之,起為葉縣令,修學勵俗,有古良吏風。葉自兵興,戶減三之一,田不毛者萬七千畝有奇,其歲入七萬石如故。從益請於大司農,為減一萬,民甚賴之,流亡歸者四千餘家。未幾,被召,百姓詣尚書省乞留,不聽。入授應奉翰林文字,踰月以疾卒,年四十四。葉人聞之,以端午罷酒為位而哭,且立石頌德,以致哀思。



 從益博學強記,精於經學。為文章長於詩,五言尤工,有《蓬門集》。



 子祁字京叔。為太學生。甚有文名。值金末喪亂,作《歸潛志》以紀金事,修《金史》多採用焉。



 呂中孚,字信臣,冀州南宮人。張建字吉甫,蒲城人。皆有詩名。中孚有《清漳集》。建明昌初授絳州教官,召為宮教、應奉翰林文字。以老請致仕,章宗愛其純素,不欲令去,授同知華州防禦使,仍賜詩以寵之。自號「蘭泉」,有集行於世。



 李純甫,字之純,弘州襄陰人。祖安上,嘗魁西京進士。父採,卒於益都府治中。純甫幼穎悟異常,初業詞賦,及讀《左氏春秋》,大愛之,遂更為經義學。擢承安二年經義進士。為文法莊周、列禦寇、左氏、《戰國策》,後進多宗之。又喜談兵,慨然有經世心。章宗南征,兩上疏策其勝負,上奇
 之,給送軍中,後多如所料。宰執愛其文,薦入翰林。及大元兵起,又上疏論時事,不報。宣宗遷汴,再入翰林。時丞相高琪擅威福柄,擢為左司都事,純甫審其必敗,以母老辭去。既而高琪誅,復入翰林,連知貢舉。正大末,坐取人踰新格,出倅坊州。未赴,改京兆府判官。卒於汴,年四十七。



 純甫為人聰敏,少自負其材,謂功名可俯拾,作《矮柏賦》,以諸葛孔明、王景略自期。由小官上萬方書,援宋為證,甚切,當路者以迂闊見抑。中年,度其道不行,益縱酒自放,無仕進意。得官未成考,旋即歸隱。日與禪僧士子游,以文酒為事,嘯歌袒裼出禮法外,或飲數月不醒。
 人有酒見招,不擇貴賤必往,往輒醉,雖沉醉亦未嘗廢著書。然晚年喜佛,力探其奧義。自類其文,凡論性理及關佛老二家者號「內稿」,其餘應物文字為「外稿」。又解《楞嚴》、《金剛經》、《老子》、《莊子》。又有《中庸集解》、《鳴道集解》,號「中國心學、西方文教」。數十萬言,以故為名教所貶云。



 王鬱,字飛伯,大興人。儀狀魁奇,目光如鶻。少居釣臺,閉門讀書,不接人事。久之,為文法柳宗元,閎肆奇古,動輒數千言。歌詩俊逸,效李白。嘗作《王子小傳》以自敘。天興初元,汴京被圍,上書言事,不報。四月,圍稍解,挺身突出,為兵士所得。其將遇之甚厚,鬱經行無機防,為其下所
 忌,見殺。臨終,懷中出書曰:「是吾平生著述,可傳付中州士大夫曰,王鬱死矣。」年三十餘。同時以詩鳴者,雷琯、侯冊、王元粹云。



 宋九嘉,字飛卿,夏津人。為人剛直豪邁,少遊太學,有能賦聲。長從李純甫讀書,為文有奇氣,與雷淵、李經相伯仲。中至寧元年進士第。歷藍田、高陵、扶風、三水四縣令,咸以能稱。入為翰林應奉。正大中,以疾去。沒于癸巳之難。



 龐鑄,字才卿,遼東人。少擢第,仕有聲。南渡後,為翰林待制,遷戶部侍郎。坐游貴戚家,出倅東平,改京兆路轉運
 使,卒。博學能文,工詩,造語奇健不凡,世多傳之。



 李獻能,字欽叔,河中人。先世有為金吾衛上將軍者,時號「李金吾家」。迨獻能昆弟皆以文學名,從兄獻卿、獻誠、從弟獻甫相繼擢第,故李氏有「四桂堂」。



 獻能苦學博覽,於文尤長於四六。貞祐三年,特賜詞賦進士,廷試第一人,宏詞優等。授應奉翰林文字。在翰苑凡十年,出為鄜州觀察判官。用薦者復為應奉,俄遷修撰。正大末,以鎮南軍節度副使充河中帥府經歷官。大元兵破河中,奔陜州,行省以雚左右司郎中,值趙三三軍變遇害,年四十三。



 獻能為人眇小而黑色,頗有髯。善談論,每敷說今
 古,聲鏗亮可聽。作詩有志於風雅,又刻意樂章。在翰院,應機敏捷號得體。趙秉文、李純甫嘗曰:「李獻能天生今世翰苑材。」故每薦之,不令出館。家故饒財,盡於貞祐之亂,在京師無以自資。其母素豪奢,厚於自奉,小不如意則必訶譴,人視之殆不堪憂,獻能處之自若也。時人以純孝稱之。嘗謂人云:「吾幼夢官至五品,壽不至五十。」後竟如其言。



 王若虛,字從之,槁城人也。幼穎悟,若夙昔在文字間者。擢承安二年經義進士。調鄜州錄事,歷管城、門山二縣令,皆有惠政,秩滿,老幼攀送,數日乃得行。用薦入為國
 史院編修官,遷應奉翰林文字。奉使夏國,還授同知泗州軍州事,留為著作佐郎。正大初,《宣宗實錄》成,遷平涼府判官。未幾,召為左司諫,後轉延州刺史,入為直學士。



 元興元年,哀宗走歸德。明年春,崔立變。群小附和,請為立建功德碑,翟奕以尚書省命召若虛為文。時奕輩恃勢作威,人或少忤,則讒構立見屠滅。若虛自分必死,私謂左右司員外郎元好問曰:「今召我作碑,不從則死。作之則名節掃地,不若死之為愈。雖然,我姑以理諭之。」乃謂奕輩曰:「丞相功德碑當指何事為言?」奕輩怒曰:「丞相以京城降,活生靈百萬,非功德乎?」曰;「學士代王言,
 功德碑謂之代王言可乎?且丞相既以城降,則朝官皆出其門,自古豈有門下人為主帥誦功德而可信乎後世哉?」奕輩不能奪,乃召太學生劉祁、麻革輩赴省,好問、張信之喻以立碑事,曰:「眾議屬二君,且已白鄭王矣,二君其無讓。」祁等固辭而別。數日,促迫不已,祁即為草定,以付好問,好問意未愜,乃自為之。既成,以示若虛,乃共刪定數字,然止直敘其事而已。後兵入城,不果立也。



 金亡,微服北歸鎮陽,與渾源劉郁東游泰山,至黃峴峰,憩萃美亭,顧謂同游曰:「汩沒塵土中一生,不意晚年乃造仙府,誠得終老此山,志願畢矣。」乃令子忠先歸,遣子恕
 前行視夷險,因垂足坐大石上,良久瞑目而逝,年七十。所著文章號《慵夫集》若干卷、《滹南遺老》若干卷、傳於世。



 王元節,字子元,弘州人也。祖山甫,遼戶部侍郎。父詡,海陵朝,左司員外郎。元節幼穎悟,雖家世貴顯,而從學甚謹。渾源劉捴愛其才俊,以女妻之,遂傳其賦學,登天德三年詞賦進士第。雅尚氣節,不能隨時俯仰,故仕不顯。及遷密州觀察判官,既罷,即逍遙鄉里,以詩酒自娛,號曰「遁齊」。年五十餘卒。有詩集行於世。



 弟元德,亦第進士。有能名於時,終南京路提刑使。



 孫國綱,字正之。業儒術,
 尤長吏事。為人端重樂易,或有忤者,略不與校,亦未嘗形于怒色。大安三年,試補尚書吏部掾,未幾,轉御史臺令史。宣宗聞其材幹,興定三年特召為近侍,奉職承應,甚見寵遇,勒留凡三考,出為同知申州事。無何,召為筆硯直長,擢監察御史,秩滿,敕留再任,蓋知其材器故也。開興元年,關陜完顏總帥屯河中府,與大元軍戰敗績,哀宗遣國綱乘上廄馬,徑詣河中問敗軍之由,還至中途,值大兵見殺,時年四十四。



 麻九疇,字知幾,易州人。三歲識字。七歲能草書,作大字有及數尺者,一時目為神童。章宗召見,問:「汝入宮殿中,
 亦懼怯否?」對曰:「君臣,父子也。子寧懼父耶?」上大奇之。弱冠入太學,有文名。南渡後,寓居郾、蔡間,入遂平西山,始以古學自力。博通《五經》,於《易》、《春秋》為尤長。興定末,試開封府,詞賦第二,經義第一。再試南省,復然。聲譽大振,雖婦人小兒皆知其名。及廷試,以誤絀,士論惜之。已而隱居不為科舉計。正大初,門人王說、王采苓俱中第,上以其年幼,怪而問之。乃知嘗師九疇。平章政事侯摯、翰林學士趙秉文連章薦之,特賜盧亞榜進士第。以病,未拜官告歸。再授太常寺太祝,權博士,俄遷應奉翰林文字。九疇性資野逸,高蹇自便,與人交,一語不相入則逕去
 不返顧。自度終不能與世合,頃之,復謝病去。居郾城,天興元年,大元兵入河南,挈家走確山,為兵士所得,驅至廣平,病死,年五十。



 九疇初因經義學《易》,後喜邵堯夫《皇極書》,因學算數,又喜卜筮、射覆之術。晚更喜醫,與名醫張子和游,盡傳其學,且為潤色其所著書。為文精密奇健,詩尤工致。後以避謗忌,持戒不作。明昌以來,稱神童者五人,太原常添壽四歲能作詩,劉滋、劉微、張漢臣後皆無稱,獨知幾能自樹立,耆舊如趙秉文,以征君目之而不名。



 李汾,字長源,太原平晉人。為人尚氣,跌宕不羈。性褊躁,
 觸之輒怒,以是多為人所惡。喜讀史。工詩,雄健有法。避亂入關,京兆尹子容愛其材,招致門下。留二年去,之涇州,竭左丞張行信,一見即以上客禮之。元光間,游大梁,舉進士不中,用薦為史館書寫。書寫,特抄書小史耳,凡編修官得日錄,纂述即定,以稿授書寫,書寫錄潔本呈翰長。汾既為之,殊不自聊。時趙秉文為學士,雷淵、李獻能皆在院,刊修之際,汾在旁正襟危坐,讀太史公、左丘明一篇,或數百言,音吐洪暢,旁若無人。既畢,顧四坐漫為一語云「看」。秉筆諸人積不平,而雷、李尤切齒,乃以嫚罵官長訟于有司,然時論亦有不直雷、李者。尋罷入關。
 明年來京師,上書言時事,不合,去客唐、鄧間。恒山公武仙署行尚書省講議官。既而仙與參知政事完顏思烈相異同,頗謀自安,懼汾言論,欲除之。汾覺,遁泌陽,仙令總帥王德追獲之,鎖養馬平,絕食而死,年未四十。



 汾平生詩甚多,不自收集,世所傳者十二三而已。



 元德明,系出拓拔魏,太原秀容人。自幼嗜讀書,口不言世俗鄙事,樂易無畦畛,布衣蔬食處之自若,家人不敢以生理累之。累舉不第,放浪山水間,餘酒賦詩以自適。年四十八卒。有《東巖集》三卷。子好問,最知名。



 好問字裕之。七歲能詩。年十有四,從陵川郝晉卿學,不事舉業,淹
 貫經傳百家,六年而業成。下太行,渡大河,為《箕山》、《琴臺》等詩。禮部趙秉文見之,以為近代無此作也。於是名震京師。中興定五年第,歷內鄉令。正大中,為南陽令。天興初,擢尚書省掾,頃之,除左司都事,轉行尚書省左司員外郎。金亡,不仕。



 為文有繩尺,備眾體。其詩奇崛而絕雕劌,巧縟而謝綺麗。五言高古沈鬱。七言樂府不用古題,特出新意。歌謠慷慨,挾幽、并之氣。其長短句,揄揚新聲,以寫恩怨者又數百篇。兵後,故老皆盡,好問蔚為一代宗工,四方碑板銘志,盡趨其門。其所著文章詩若干卷、《杜詩學》一卷、《東坡詩雅》三卷、《錦禨》一卷、《詩文自警》十卷。



 晚年尤以著作自任,以金源氏有天下,典章法度幾及漢、唐,國亡史作,己所當任。時金國實錄在順天張萬戶家,乃言於張,願為撰述,既而為樂夔所沮而止。好問曰:「不可令一代之跡泯而不傳。」乃構亭於家,著述其上,因名曰「野史」。凡金源君臣遺言往行,采摭所聞,有所得輒以寸紙細字為記錄,至百餘萬言。今所傳者有《中州集》及《壬辰雜編》若干卷。年六十八卒。纂修《金史》,多本其所著云。



 贊曰:韓昉、吳激,楚材而晉用之,亦足為一代之文矣。蔡珪、馬定國之該博,胡礪、楊伯仁之敏贍,鄭子聃、麻九疇
 之英俊,王鬱、宋九嘉之邁往。三李卓犖,純甫知道,汾任氣,獻能尤以純孝見稱。王庭筠、黨懷英、元好問自足知名異代。王競、劉從益、王若虛之吏治,文不掩其所長。蔡松年在文藝中,爵位之最重者,道金人言利,興黨獄,殺田玨,文不能掩其所短者歟?事繼母有至行,其死家無餘貲,有足取云。



\end{pinyinscope}