\article{列傳第七十三}

\begin{pinyinscope}

 外國下



 ○高麗



 高麗國王,王楷。其地鴨綠江以東,曷懶路以南,東南皆至于海。自遼時,歲時遣使修貢,事具《遼史》。



 唐初,靺鞨有粟末、黑水兩部,皆臣屬于高麗。唐滅高麗,粟末保東牟山漸彊大,號渤海,姓大氏,有文物禮樂。至唐末稍衰,自後不復有聞。金伐遼,渤海來歸,蓋其遺裔也。黑水靺鞨
 居古肅慎地,有山曰白山,蓋長白山,金國之所起焉。女直雖舊屬高麗,不復相通者久矣。及金滅遼,高麗以事遼舊禮稱臣于金。



 初,有醫者善治疾,本高麗人,不知其始自何而來,亦不著其姓名,居女直之完顏部。穆宗時戚屬有疾,此醫者診視之,穆宗謂醫者曰:「汝能使此人病愈,則吾遣人送汝歸汝鄉國。」醫者曰:「諾。」其人疾果愈,穆宗乃以初約歸之。乙離骨嶺僕散部胡石來勃堇居高麗、女直之兩間,穆宗使族人叟阿招之,因使叟阿送醫者,歸之高麗境上。醫者歸至高麗,因謂高麗人,女直居黑水部者部族日彊,兵益精悍,年穀屢稔。高麗王聞
 之。乃通使于女直。既而,胡石來來歸,遂率乙離骨嶺東諸部皆內附。



 穆宗十年癸未,阿疏自遼使其徒達紀來說曷懶甸人,曷懶甸人執之。穆宗以達紀送高麗,謂高麗王曰:「前此為亂於汝鄙者,皆此輩也。」及破蕭海里,使斡魯罕往高麗報捷,高麗亦使使來賀。未幾,復使斜葛與斡魯罕往聘,高麗王曰:「斜葛,女直之族弟也,其禮有加矣。」乃以一大銀盤為謝。



 厥後,曷懶甸諸部盡欲來附,高麗聞之不欲使來附,恐近於己而不利也,使人邀止之。斜葛在高麗及往來曷懶道中,具知其事,遂使石適歡往納曷懶甸人。未行而穆宗沒,康宗嗣,遣石適歡以
 星顯統門之兵往至乙離骨嶺,益募兵趨活涅水,徇地曷懶甸,收叛亡七城。高麗使人來告曰:「事有當議者。」曷懶甸官屬使斜勒詳穩、冶剌保詳穩往,石適歡亦使盃魯往,高麗執冶剌保等,而遣盃魯曰:「無與爾事。」於是,五水之民皆附於高麗,團練使陷者十四人。



 二年甲申,高麗來攻,石適歡大破之,殺獲甚眾,追入其境,焚略其戍守而還。四月,高麗復來攻,石適歡以五百人禦於闢登水,復大破之,追入闢登水,逐其殘眾踰境。於是,高麗王曰:「告邊釁者皆官屬祥丹、傍都里、昔畢罕輩也。」十四團練、六路使人在高麗者,皆歸之,遣使來請和。遂使斜葛
 經正疆界,至乙離骨水、曷懶甸活禰水,留之兩月。斜葛不能聽訟,每一事輒至枝蔓,民頗苦之。康宗召斜葛還,而遣石適歡往。石適歡立幕府于三潺水,其嘗陰與高麗往來為亂階者,即正其罪,餘無所問。康宗以為能。



 四年丙戌,高麗使使黑歡方石來賀嗣位,康宗使盃魯報聘,且尋前約,取亡命之民,高麗許之。曰:「使使至境上受之。」康宗以為信然,使完顏部阿聒、烏林答部勝昆往境上受之。康宗畋於馬紀嶺乙隻村以待之。阿聒、勝昆至境上,高麗遣人殺之,而出兵曷懶甸,築九城。



 康宗歸,眾咸曰:「不可舉兵也,恐遼人將以罪我。」太祖獨曰:「若不舉
 兵,豈止失曷懶甸,諸部皆非吾有也。」康宗以為然,乃使斡塞將兵伐之,大破高麗兵。六月,高麗率眾來戰,斡塞敗之,進圍其城。七月,高麗復請和,康宗曰:「事若酌中,則與之和。」高麗許歸亡入之民,罷九城之戍,復所侵故地,遂與之和。



 收國元年九月,太祖已克黃龍府,命加古撒喝攻保州。保州近高麗,遼侵高麗置保州。至是,命撒喝取之,久不下,撒喝請濟師,且言高麗王將遣使來。太祖使納合烏蠢以百騎益之,詔撒喝曰:「汝領偏師,屢破重敵,多所俘獲,及聞胡沙數戰有功,朕甚嘉之。若保州未下,但守邊戍,吾已克黃龍府,聞遼主且至,俟破大敵復
 益汝兵。所言高麗遣使事,未知果否,至則護送以來。邊境之事,慎之毋忽。」十一月,係遼女直麻懣太彎等十五人皆降,攻開州取之,盡降保州諸部女直。太祖以撒喝為保州路都統。



 太祖已破走遼主軍,撒喝破合主、順化二城,復請濟師攻保州,使斡魯以甲士千人往。二年閏月,高麗遣使來賀捷,且曰:「保州本吾舊地,願以見還。」太祖謂使者曰:「爾其自取之。」詔撒喝、烏蠢等曰:「若高麗來取保州,益以胡剌古、習顯等軍備之,或欲合兵,無得輒往,但謹守邊戍。」及撒喝、阿實賚等攻保州,遼守將遁去,而高麗兵已在城中。既而,高麗國王使蒲馬請保州,詔
 諭高麗王曰:「保州近爾邊境,聽爾自取,今乃勤我師徒,破敵城下。且蒲馬止是口陳,俟有表請,即當別議。」



 天輔二年十二月,詔諭高麗國王曰:「朕始興師伐遼,已嘗布告,賴皇天助順,屢敗敵兵,北自上京,南至于海,其間京府州縣部族人民悉皆撫定。今遣孛堇術孛報諭,仍賜馬一匹,至可領也。」



 三年,高麗增築長城三尺,邊吏發兵止之,弗從,報曰:「修補舊城。」曷懶甸孛堇胡剌古、習顯以聞,詔曰:「毋得侵軼生事,但慎固營壘,廣布耳目而已。」



 四年,咸州路都統司以兵分屯於保州、畢里圍二城,請益兵,詔曰:「汝等分列屯戍,以固封守,甚善。高麗累世臣事
 于遼,或有交通,可常遣人偵伺。」



 使習顯以獲遼國州郡諭高麗,其國方誅亂者,使謂習顯曰:「此與先父國王之書。」習顯就館。凡誅戮官僚七十餘人,即依舊禮接見,而以表來賀,並貢方物。復以遼帝亡入夏國報之。



 高隨、斜野奉使高麗,至境上,接待之禮不遜,隨等不敢往,太宗曰:「高麗世臣於遼,當以事遼之禮事我,而我國有新喪,遼主未獲,勿遽彊之。」命高隨等還。天會二年,同知南路都統鶻實答奏,高麗納叛亡、增邊備,必有異圖。詔曰:「凡有通問,毋違常式。或來侵略,則整爾行列與之從事。敢先犯彼者,雖捷必罰。」詔闍母以甲士千人戍海島,以備
 之。



 四年,國王王楷遣使奉表稱籓,優詔答之。上使高伯淑、烏至忠使高麗,凡遣使往來當盡循遼舊,仍取保州路及邊地人口在彼界者,須盡數發還。敕伯淑曰:「若一一聽從,即以保州地賜之。」高伯淑至高麗,王楷附表謝,一依事遼舊制。八年,楷上表,乞免索保州亡入邊戶。是歲,高麗十人捕魚,大風飄其船抵海岸,曷蘇館人獲之,詔還其國。既而勖上表請不索保州亡入高麗戶口,太宗從之,自是保州封域始定。



 皇統二年,詔加楷開府儀同三司、上柱國。六年,楷薨,子晛嗣立。



 大定四年,鴨綠江堡戍頗被侵越焚毀。五年正月,世宗因正旦使朝辭,
 諭之曰:「邊境小小不虞,爾主使然邪,疆吏為之邪?若果疆吏為之,爾主亦當懲戒之也。」初,高麗使者別有私進禮物以為常,是歲萬春節,上以使者私進不應典禮,詔罷之。



 十年,王晛弟翼陽公皓廢晛自立。十月,賜生日使、大宗正丞颭至界上,高麗邊吏稱前王已讓位,不肯受使者。十一年三月,王皓以讓國來奏告,詔婆速路勿受,有司移文詳問。高麗告曰:「前王久病,昏耄不治,以母弟皓權攝國事。」上曰:「讓國大事也,何以不先陳請。」詔有司再詳問。高麗乃以王晛讓國表來,大略稱先臣楷遺訓傳位於弟,又言其子有罪不可立之意。上疑之,以問宰
 執,丞相良弼奏曰:「此不可信。晛止一子,往年生孫,嘗有表自陳生孫之喜,一也。皓嘗作亂,晛囚之,二也。今晛不遣使,皓乃遣使,三也。朝廷賜晛生日使,皓不轉達於晛,乃稱未敢奉受,四也。今皓篡兄誣於天子,安可忍也。」右丞孟浩曰:「當詢彼國士民,果皆推服,即當遣使封冊。」上曰:「封一國之君詢於民眾,此與除拜猛安謀克何異。」乃卻其使者,而以詔書詳問王晛,吏部侍郎靖為宣問王晛使。



 皓實篡國,囚晛於海島。靖至高麗,皓稱王晛已避位出居他所,病加無損,不能就位拜命,往復險遠,非使者所宜往。靖竟不得見晛,乃以詔授皓,轉取晛表附
 奏,其言與前表大概相同。靖還,上問大臣,皆曰:「晛表如此,可遂封之。」丞相良弼、平章政事守道曰:「待皓祈請未晚也。」十二月,皓遣其禮部侍郎張翼明等請封。十二年三月,遂賜封冊。皓生日在正月十九日,是歲十二月將盡,未及遣使,有司請至來歲舉行焉。



 十五年,高麗西京留守趙位寵叛皓,遣徐彥等九十六人上表曰:「前王本非避讓,大將軍鄭沖夫、郎將李義方實弒之。臣位寵請以慈悲嶺以西至鴨綠江四十餘城內屬,請兵助援。」上曰:「王皓已加封冊,位寵輒敢稱兵為亂,且欲納土,朕懷撫萬邦,豈助叛臣為虐。」詔執徐彥等送高麗。頃之,
 王皓定趙位寵之亂,遣使奏謝,自位寵之亂,皓所遣生日回謝、橫賜回謝、賀正旦、進奉、萬春節等使,皆阻不通,至是,皓并奏之。詔答其意,其合遣人使令節次入朝。



 十七年,賀正旦禮物,玉帶乃石似玉者,有司請移問,上曰:「彼小國無能識者,誤以為玉耳,不必移問。」乃止。十二月,有司奏高麗下節押馬官順成例外將帶甲三過界,上以使人所坐罪重,但令發還本國而已。二十三年,皓母任氏薨,皓乞免賜生日及賀謝等事,詔從之。



 章宗即位,詔使至界上頗稽滯,詔移問,高麗遜謝。明昌三年,下節金挺回至平州撫寧縣,毆死當驛人何添兒,有司請「凡
 人使往還,乞量設兵衛。」參知政事張萬公曰:「可於宿頓之地巡護之。」上可其奏。詔自今接送伴使副,失關防者當坐。故事,賀正旦使十二月二十九日入見,明昌六年十二月己卯立春,詔於前二日丁丑入見云。



 承安二年,皓表自陳衰病,以國讓其弟晫。晫權國事。是歲,皓廢,晫嗣立。



 泰和四年正月乙丑朔,高麗傔人以小佩刀割梨廡下巡廊,奉職見而糾之,詔館伴官自今前期移文禁止。是歲,王晫薨,子韺嗣立。



 泰和七年正月,是時用兵伐宋,夏亦有故,獨高麗遣正旦使,詔不賜曲宴。及天壽節,夏、高麗使者皆在,有司奏:「大定初,宋未請和,夏、高麗使
 者賜曲宴,今請依大定故事。」詔從之。



 至寧元年八月,王祦薨,嗣子未行起復。九月,宣宗即位,邊吏奏:「高麗牒稱,嗣子未起復,不可以凶服迎吉詔,又不可以草土名銜署表。」禮官議:「人臣不以私恩廢公義,宜權用吉服迎詔,署表用權國事名銜。俟高麗告哀使至闕,然後遣使致祭、慰問及行封冊。」制可。



 明年,宣宗遷汴,遼東道路不通,興定三年,遼東行省奏高麗復有奉表朝貢之意,宰臣奏:「可令行省受其表章,其朝貢之禮俟他日徐議。」宣宗以為然,乃遣使撫諭高麗,終以道路不通,未遑迎迓,詔行省且羈縻勿絕其好,然自是不復通問矣。



 贊曰:金人本出鞨靺之附於高麗者,始通好為鄰國,既而為君臣,貞祐以後道路不通,僅一再見而已。入聖朝猶子孫相傳自為治,故不復備論,論其與金事相涉者
 焉。



 金國語解



 今文《尚書》辭多奇澀,蓋亦當世之方言也。《金史》所載本國之語,得諸重譯,而可解者何可闕焉。若其臣僚之小字,或以賤,或以疾,猶有古人尚質之風,不可文也。國姓為某,漢姓
 為某,後魏孝文以來已有之矣。存諸篇終,以備考索。



 ○官
 稱



 都勃極烈,總治官名,猶漢雲冢宰。



 諳版勃極烈,官之尊且貴者。



 國論勃極烈,尊禮優崇得自由者。



 胡魯勃極烈,統領官之稱。



 移賚勃極烈,
 位第三曰「移賚」。



 阿買勃極烈,治城邑者。



 乙室勃極烈,迎邪之官。



 札失哈勃極烈,守官署之稱。



 昃勃極烈,陰
 陽之官。



 迭勃極烈,倅貳之職。



 猛安,千夫長。謀克,百夫長也。



 諸颭「詳穩」,邊戍之官。



 諸「移里堇」,部落墟砦之首領。



 詳穩、移裏堇,本遼語,金
 人因之而稍異同焉。



 禿里,掌部落詞訟,察非違者。



 烏魯古,牧圉之官。



 斡里朵,官府治事之所。



 人事



 孛論出,胚胎之名。



 阿胡迭,長子。骨赧,季也。蒲陽溫,曰幼子。



 益都,次第之通稱。第九曰「烏也」,
 十六曰「女魯歡」。



 按答海,客之通稱。



 山只昆,舍人也。



 散亦孛,奇男子。



 散答,老人之稱也。



 什古乃,瘠人。



 撒合輦,黧黑之名。



 保活里,侏儒。



 阿里孫,貌不揚也。



 阿徒罕,採薪之子。



 答不也,耘田者。



 阿土古,善採捕者。阿里喜,圍獵也。



 拔里速,角牴戲者。



 阿離合懣,臂鷹鶻者。



 胡
 魯剌,戶長。阿合,人奴也。



 兀術,曰頭。粘罕,心也。畏可,牙,又曰吾亦可。



 盤里合,將指。



 三合,人之靨也。



 牙吾塔,瘍瘡。



 蒲剌都,目赤而盲也。



 石哥里,溲疾。



 謾
 都謌,癡騃之謂。



 謀良虎,無賴之名。皆不美之稱也。



 與人同受福曰「忽都」。以力助人曰「阿息保」。



 辭不失,酒醒也。



 奴申,和睦之義。



 訛出虎,寬容之名也。



 賽里,安樂。



 迪古乃,來也。



 撒八,迅速之義。



 烏
 古出,方言曰再休,猶言再不復也。



 凡事之先者曰「石倫」。以物與人已然曰「阿里白」。



 吾里補,畜積之名。



 習失,猶人云常川也。



 凡市物已得曰「兀帶」,取以名子者,猶言貨取如物然也。



 ○物象



 兀典,明星。



 阿鄰,山。太神,高也。
 山之上銳者曰「哈丹」,坡陀曰「阿懶」,大而峻曰「斜魯」。



 忒鄰,海也。沙忽帶,舟也。



 生鐵曰「斡論」,釜曰「闍母」,刃曰「斜烈」。



 婆盧火者槌也。



 金曰「桉春」。



 銀術可,珠也。



 布囊曰「蒲盧渾」,盆曰「阿里虎」,罐曰「活女」。



 烏烈,草廩也。



 沙剌,衣襟也。



 活臘胡,色之赤者也。



 胡剌,灶突。



 物類



 桓端,松。
 阿虎里,松子。孰輦,蓮也。



 活離罕,羔。合喜,犬子。訛古乃,犬之有文者。



 斜哥,貂鼠。



 蒲阿,山雞。窩謀罕,鳥卵也。



 姓氏



 完顏,漢姓曰王。烏古論曰
 商。紇石烈曰高。徒單曰杜。女奚烈曰郎。兀顏曰硃。蒲察曰
 李。顏盞曰張。溫迪罕曰溫。石抹曰蕭。奧屯曰曹。孛術魯曰
 魯。移剌曰劉。斡勒曰石。納剌曰康。夾谷曰仝。裴滿曰
 麻。尼忙古曰魚。斡準曰趙。阿典曰雷。阿里侃曰何。溫敦曰
 空。吾魯曰惠。抹顏曰孟。都烈曰強。散答曰駱。呵不哈曰
 田。烏林答曰蔡。僕散曰林。術虎曰董。古里甲曰汪。



 其後氏族或因人變易,難以遍舉,姑載其可知者云。



 金國語解終。



\end{pinyinscope}