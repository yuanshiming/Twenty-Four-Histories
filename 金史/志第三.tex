\article{志第三}

\begin{pinyinscope}

 歷下



 ○步月離第五



 轉終分:一十四萬四千一百一十,秒六千六十六。



 轉終日:二十七日,餘二千九百,秒六千六十六。



 轉中日:一十三日,餘四千六十五,秒三千三十三。



 朔差日:一,餘五千一百四,秒三千九百三十四。



 象策:七日,餘二千一分,二十二秒半。



 秒母:
 一萬。



 上弦:九十一度,三十一分,四十二秒。



 望:一百八十二度,六十二分,八十四秒。



 下弦:二百七十三度,九十四分,二十六秒。



 月平行度:十三度,三十六分,八十七秒半。



 分、秒母:一百。



 七日:初數,四千六百四十八。末數,五百八十二。



 十四日:初數,四千六十五。末數,一千一百六十五。



 二十一日:初數,三千四百八十三。末數,一千七百四十七。



 二十八日:初數,二千九百一。末數,二千三百二十九。



 求經朔弦望入轉



 置天正朔積分,以轉終分及秒去之,不盡,如日法而一,為日,不滿為餘秒,即天正十一月經朔入轉日及餘秒。以象策累加之,去命如前,即得弦、望經日加時入轉日及餘秒。徑求次朔入轉。以朔差加之。



 轉定分及積度朓棵率



 表略



 求朔弦望入轉朓棵定數



 置入轉小餘,以其日算外,損益率乘之,如日法而一,所得,以損益積為定數。其四七日下餘,如初數以下,初率乘之,初數而一,以損益朓棵積為定數。如初數以上,初數減之,餘乘末率,末數而一,便為朓棵定數。



 求朔弦望定日



 置經朔、弦、望小餘,朓減朒加入氣入轉朓棵定
 數,滿與不足,進退大餘,命甲子算外,各得定朔、弦、望日辰及餘。定朔前幹名與後幹名同者,其月大;不同者,其月小。月內無中氣者為閏。視定朔小餘:秋分後,在日法四分之三以上者,進一日。春分後,定朔日出分與春分日出分相減之餘,三約之,用減四分之三,定朔小餘及此數以上者,亦進一日。或有交,虧初在日入前者,不進之。



 定弦、望小餘在日出分以下者,退一
 日。望或有交,虧初在日出前者,小餘雖在日出後,亦退之。如十七日
 望者,
 又視定朔小餘在四分之三以下之數,春分後用減定之數。與定望小餘在日出分以上之數相較之;朔少望多者,望不退,而朔猶進之。望少朔多者,朔不進,而望猶退之。日月之行,有盈有縮,遲疾加減之數,或有四大三小;若隨常理,當察其時早晚,隨所近而進退之,使不過三大二小。



 求定朔弦望中積



 置定朔、弦、望大小餘與經朔、弦、望大小餘相減之餘,以加減經朔、弦、望入氣日餘,經朔、弦、望少即加之,多即減之。即為定朔、弦、望入氣。以加其氣中積,即為定朔、弦、望中積。其餘以日法退除為分秒。



 求定朔弦望加時日度



 置定朔、弦、望約餘,以所入氣日損益率乘,盈縮損益。萬約之,
 以損益其下盈縮積,乃盈加縮減定朔弦望中積;又以冬至加時日躔黃道宿度加之,依宿次去之,即得定朔、弦、望加時日所在度及分秒。又置定朔、弦、望約餘,副置之。以乘其日盈縮之損益率,萬約之,應益者盈加縮減,應損者盈減縮加其副,滿百為分,分滿百為度,以加其日夜半日度,命之,各得其日加時日躔黃道宿次。若先於歷注定每日夜半日度,即為妙也。



 求定朔弦望加時月度



 凡合朔加時日月同度,其定朔加時黃道日度,即為定朔加時黃道月度。弦、望各以弦、望度加定弦、望加時黃道
 日度,依宿次去之,即得定朔、弦、望加時黃道月度及分秒。



 求夜半午中入轉



 置經朔入轉,以經朔小餘減之,為經朔夜半入轉。又經朔小餘與半法相減之餘,以加減經朔加時入轉,經朔少,如半法加之;多,如半法減之。為經朔午中入轉。若定朔大餘有進退者,亦加減轉入,否則因經為定。每月累加一日,滿終日及餘秒去命如前,各得每日夜半、午中入轉。求夜半,因定朔夜半入轉累加之。求午中,因定朔午中入轉累加之。求加時入轉者,如求加時入氣術。



 求加時及夜半月度



 置其日入轉算外轉定分,以定朔、弦、望小餘乘之,如日法而一,為加時轉分。分滿百為度。減定朔、弦、望加時月度,為夜半月度。以所得轉定分累加之,即得每日夜半月度。或朔至弦、望,或至後朔,皆可累加之。然近則差少,遠則差多。置所求前後夜半相距月度為行度,計其相距入轉積度,與行度相減,餘以相距日數除為日差,行度多以日差加每日轉定分,行度少以日差減每日轉定分,然後用之可中。或欲速求,用此數,欲究其故,宜用後術。



 求晨昏月度



 置其日晨分,乘其日算外轉定分,日法而一,為晨轉分。用減定分,餘為昏轉分。又以朔、弦、望定小餘、乘轉定分,日法而一,為加時分。以減晨、昏轉分,為前;不足,覆減
 之,為後。乃前加後減加時月度,即晨昏月所在宿度及分秒。



 求朔弦望晨昏定程



 各以其朔昏定月,減上弦昏定月,餘為朔後昏定程。以上弦昏定月,減望昏定,餘為上弦後昏定程。以望晨定月,減下弦晨定月,餘為望後晨定程。以下弦晨定月,減後朔晨定月,餘為下弦後晨定程。



 求每日轉定度



 累計每程相距日下轉積度,與晨昏定程相減,餘以相距日數除之,為日差,定程多加之,定程少減之。以加減每日轉定分,
 為轉定度。因朔、弦、望晨昏月,每日累加之,滿宿次去之,為每日晨昏月度及分秒。凡注歷:朔日以後注昏月,望後一日注晨月。古歷有九道月度,其數雖繁,亦難削去,具其術如後。



 求平交日辰



 置交終日及餘秒,以其月經朔加時入交汎日及餘秒減之,為平交入其月經朔加時後日及餘秒。以加其月經朔大小餘,其大餘命甲子算外,即平交日辰及餘秒。求次交者,以交終日及餘秒加之,大餘滿紀法去之,命如前,即次平交日辰及餘秒。



 求平交入轉朓棵定數



 置平交小餘,加其日夜半入轉餘,以乘其日損益率,日
 法而一,所得,以損益其下朓朒積,為定數。



 求正交日辰



 置平交小餘,以平交入轉朓棵定數,朓減朒加之,滿與不足,進退日辰,即正交日辰及餘秒。與定朔日辰相距,即所在月日。



 求經朔加時中積



 各以其月經朔加入氣日及餘,加其氣中積餘,其日命為度,其餘以日法退除為分秒,即其經朔加時中積度及分秒。



 求正交加時黃道月度



 置平交入經朔加時後算及餘秒,以日法通日,內餘,進二位,如三萬九千一百二十一分為度,不滿退除為分秒,以加其月經朔加時中積,然後以冬至加時黃道日度加而命之,即其得其月正交加時月離黃道宿度及分秒。如求次交者,以交終度及秒加而命之,即得所求。



 求黃道宿積度



 置正交時黃道宿全度,以正交加時月離黃道宿度及分秒減之,餘為距後度及分秒,以黃道宿度累加之,即各得正交後黃道宿積度及分秒。



 求黃道宿積度入初末限



 置黃道宿積度及分秒,滿交象度及分秒去之,如在半交象以下,為初限;以上者,以減交象度及分秒,餘為入末限。入交積度交象度並在交會術中。



 求月行九道宿度



 凡月行所交:冬入陰歷,夏入陽歷,月行青道。冬至夏至後,青道半交在春分之宿,當黃道東。立冬立夏後,青道半交在立春之宿,當黃道東南。至所衝之宿亦如之。冬入陽歷,夏入陰歷,月行白道。冬至夏至後,白道半交在秋分之宿,當黃道西。立冬立夏後,白道半交在立秋之宿,當黃道西北。至所衝之宿亦如之。春入陽歷,秋入陰歷,月行朱道。春分秋分後,硃道半交在夏至之宿,當黃道南。立春立秋後,朱道半交在立夏之宿,當黃道西南。至所衝之宿亦如之。春入陰歷,秋入陽歷,月行黑道。春分秋分後,黑道半交在
 冬至之宿當黃道北。立春立秋後,黑道半交在立冬之宿,當黃道東北。至所衝之宿亦如之。四序離為八節,至陰陽之所交,皆與黃道相會,故月行有九道。各以所入初末限度及分秒,減一百一度,餘以所入初末限度及分乘之,半而退位為分,分滿百為度,命為月道與黃道汎差。凡日以赤道內為陰,外為陽;月以黃道內為陰,外為陽。故月行正交,入夏至後宿度內為同名,入冬至後宿度內為異名。其在同名者,置月行與黃道汎差,九因八約之,為定差,半交後,正交前,以差減;正交後,半交前,以差加。此加減出入六度,正,如黃赤道相交同名之差,若較之漸異,則隨交所在,遷變不同也。仍以正交度距秋分度數,乘定差,如象限而一,
 所得為月道與赤道定差。前加者為減,減者為加。其中異名者,置月行與黃道汎差,七因八約之,為定差。半交後,以差加;正交後,半交前,以差減。此加減出入六度,異,如黃道赤道相交異名之差,較之漸同,則隨交所遷變不常。仍以正交度距春分度數,乘定差,如象限而一,所得為月道與赤道定差。前加者為減,減者為加。各加減黃道宿積度,為九道宿積度。以前宿九道積度減之,為其宿九道度及分。其分就近約為太半少。論春夏秋冬以四時日所在宿度為正。



 求正交加時月離九道宿度



 以正交加時黃道日度及分,減一百一度,餘以正交度
 及分乘之,半而退位為分,分滿百為度,命為月道與黃道泛差。其在同名者,置月行與黃道汎差。九因八約之,為定差,以加;仍以正交度距秋分度數,乘定差,如象限而一,所得為月道與赤道定差,以減,其在異名者,置月行與黃道汎差,七因八約之,為定差,以減;仍以正交度距春分度數,乘定差,如象限而一,所得為月道與赤道定差,以加。置正交加時黃道月度及分,以二差加減之,即為正交加時月離九道宿度及分。



 求定朔望加時月所在度



 置定朔加時日躔黃道宿次,凡合朔加時,月行潛在日
 下,與太陽同度,是為加時月離宿次。各以弦、望度及分秒,加其所當弦、望加時月躔黃道宿度,滿宿次去之,命如前,各得定朔、弦、望加時月所在黃道宿度及分秒。



 求定朔弦望加時九道月度



 各以朔、弦、望加時月離黃道宿度及分秒,加前宿正交後黃道積度,為定朔、弦、望加時正交後黃道積度。如前求九道積度,以前宿九道積度減之,餘為定朔、弦、望加時九道月離宿度及分秒。其合朔加時,若非正交,則日在黃道,月在九道,所入宿度,雖多少不同,考其兩極,若應繩準。故云:月行潛在日下,與太陽同度,即為加時九道月度。其求晨昏夜半月度,並依前術。



 ○
 步交會第六



 交終分:一十四萬二千三百一十九,秒九千三百六十八。



 交終日:二十七日,餘一千一百九分,秒九千三百六十八。



 交中日:十三,餘三千一百六十九,秋九千六百八十四。



 交朔日:二,餘一千六百六十五,秒六百三十二。



 交望日:十四,餘四千二,秒五千。



 秒母:一萬。



 交終:三百六十三度,七十九分,三十六秒。



 交中:一百八十一度,八十九分,六十八秒。



 交象:九十度,九十四分,八十四秒。



 半交象:四十五度,四十七分,四十二秒。



 日蝕
 既前限:二千四百。定法:二百四十八。



 日蝕既後限:三千一百。定法:三百二十。



 月蝕限:五千一百。



 月蝕既限:一千七百。定法:三百四十。



 分秒母:一百。



 求朔望入交



 置天正朔積分,以交終分去之,不盡,如日法而一,為日,不滿為餘,即天正十一月經朔加時入交汎日及餘秒。交朔加之,得次朔。交望加之,得次望。再加交望,亦得次朔。各為朔、望入交汎日及餘秒



 求定朔每日夜半入交



 各置入交汎日及餘秒,減去經朔、望小餘,即為定朔、望夜半入交汎日及餘秒。若定朔、望有進退者,亦進退交日,否則因經為定。大月加二日,小月加一日,餘皆加四千一百二十秒六百三十二,即次朔夜半入交。累加一日,滿交終日及餘秒去之,即每日夜半入交汎日及餘秒。



 求定朔望加時入交



 置經朔、望加時入交汎日及餘秒,以入氣入轉朓棵定數,朓減朒加之,即定朔加時入交汎日及餘秒。



 求定朔望加時入交積度及陰陽歷



 置定朔、望加時入交汎日,以日法通之,內餘,進二位,如三萬九千一百二十一而一為度,不滿退除為分秒,即定朔、望加時月行入交積度。以定朔、望加時入轉遲疾度,遲減疾加之,即月行之入交定積度。如交中度以下,入陽歷積度;以上,去之,餘為入陰歷積度。每日夜半,準此求之。



 求月去黃道度



 視月入陰陽歷積度及分,如交象以下,為少象;以上,覆減交中,餘為老象。置所入老少象度於上,列交象度於下,相減相乘,倍而退位為分,滿百為度,用減所入老少象度及分,餘又與交中度相減相乘,八因之,以百一十
 除為分,分滿百為度,即得月去黃道度。



 求朔望加時入交常日及定日



 朔望入交汎日,以入氣朓棵定數,朓減朒加之,為入交常日。



 又置入轉朓棵定數,進一位,一百二十七而一,所得朓減朒加入交常日,為入交定日及餘秒。



 求人交陰陽歷前後分



 視入交定日,如交中以下,為陽歷;以上,去之,為陰歷。如一日上下,以日法通日為分。為交後分。十三日上下,覆減交中,為交前分。



 求日月蝕其定餘



 置朔、望入氣入轉朓棵定數,同名相從,異名相消,以一千三百三十七乘之,定朔、望加時入轉算外轉定分除之,所得,以朓減朒加經朔、望小餘,為汎餘。



 日蝕:視汎餘如半法以下,為中前分;半法以上,去半法,為中後分。置中前後分,與半法相減相乘,倍之,萬約為分,曰時差。中前,以時差減汎餘為定餘,覆減半法,餘為午前分。中後,以時差加汎為定餘,減去半法,為午後分。



 月食:視汎餘在日入後、夜半前者,如日法四分之三以
 下,減去半法,為酉前分;四分之三以上,覆減日法,餘為酉後分,又視汎餘在夜半後、日出前者,如日法四分之一以下,為卯前分,四分之一以上,覆減半法,餘為卯後分。其卯酉前後分,自相乘。四因,退位,萬約為分,以加汎餘,為定餘。各置定餘,以發斂加時法求之,即得日月所蝕之辰刻。



 求日月食甚日行積度



 置定朔、望食甚大小餘,與經朔、望大小餘相減之餘,以加減經朔、望入氣日小餘,經朔、望日少加多減。即為食甚入氣。以加其氣中積,為食甚中積。又置食甚入氣小餘,以所入氣
 日損益率盈縮之損益乘之,日法而一,以損益其日盈縮積;盈加縮減食甚中積,即為食甚日行積度及分。



 求氣差



 置日食甚日行積度及分,滿中限去之,餘在象限以下,為初限;以上,覆減中限,為末限,皆有相乘,進二位,如四百七十八而一,所得,用減一千七百四十四,餘為氣差恆數。以午前後分乘之,半晝分除之,所得,以減恒數為定數。不及減,覆減之,為定數。應加者減之,減者加之。春分後,陽歷減,陰歷加;秋分後,陽歷加,陰歷減。春分前、秋分後各二日二千一百分為定氣,於此加減之。



 求刻差



 置日食甚日行積度及分,滿中限去之,餘與中限相減相乘,進二位,如四百七十八而一,所得,為刻差恒數。以午前後分乘之,日法四分之一除之,所得為定數。若在恒數以上者,倍恒數,以所得數減之為定數,依其加減。冬至後,午前陽加陰減,午後陽減陰加。夏至後,午前陽減陰加,午後陽加陰減。



 求日食去前後定分



 氣刻二差定數,同名相從,異名相消,為食差。依其加減去交前後分,為去交前後定分。視其前後定分,如在陽歷,即不食;如在陰歷,即有食之。如交前陰歷不及減,反減之,反減食差。為交後陽歷;交後陰歷不及減,反減之,為交
 前陽歷;即不食,交前陽歷不及減,反減之,為交後陰歷;交後陽歷,不及減,反減之,為交前陰歷;即日有食之。



 求日食分



 視去交前後定分,如二千四百以下,為既前分,以二百四十八除為大分。二千四百以上,覆減五千五百,不足減者不食。為既後分,以三百二十除為大分。不盡,退除為秒,即得日食之分秒。



 求月食分



 視去交前後分,不用氣刻差者。一千七百以下者,食既。以上,覆
 減五千一百,不足減者不食。餘以三百四十除為大分,不盡,退除為秒,即為月食之分秒也。去交分在既限以下,覆減既限,亦以三百四十除,為既內之大分。



 求日食定用分



 置日食之大分,與三十分相減相乘,又以二千四百五十乘之,如定朔入轉算外轉定分而一,所得,為定用分。減定餘,為初虧分。加定餘,為復圓分。各以發斂加時法求之,即得日食三限辰刻。



 求月食定用分



 置月食之大分,與三十分相減相乘,又以二千一百
 乘之,如定望入轉算外轉定分而一,所得,為定用分。加減定餘,為初虧、復圓分。各如發斂加時法求之,即得月食三限辰刻。



 月食既者,以既內大分與十五相減相乘,又以四千二百乘之,如定望入轉算外轉定分而一,所得,為既內分。用減定用分,為既外分。置月食餘減定用分,為初虧。因加既外分,為食既。又加既內分,為食甚。既定餘分也。再加既內分,為生光。復加既外分,為復圓。各以發斂加時法求之,既得月食五限辰刻。



 求月食入更點



 置食甚所入日晨分,倍之,五約為更法。又五約更法,為點法。乃置月食初末諸分,昏分以上減昏分,晨分以下加晨分。如不滿更法為初更。不滿點法為一點。依法以次求之,既各得更點數。



 求日食所起



 食在既前,初起西南,甚於正南,復於東南;食在既後,初起西北,甚於正北,復於東北。其食八分以上,皆起正西,復於正東。此據正午地而論之。



 求月食所起



 月在陽歷:初起東北,甚於正北,復於西北。月在陰歷:初
 起東南,甚於正南,復於西南。其食八分以上,皆起正東,復於正西。此亦據午地而論之



 求日食出入帶食所見分數



 各以食甚小餘,與日出入分相減,餘為帶食差,以乘所食之分,滿定用分而一,月食既者,以既內分減帶食差,餘乘所食分,如既外分而一。不及減者,為帶食既出入。以減所食分,即日月出入帶食所見之分。其食甚在晝,晨為漸進,昏為已退。食甚在夜,晨為已退,昏為漸進。



 求日月食甚宿次



 置日月食甚日行積度,望即更加半周天。以天正冬至加時黃道日度,加而命之,依黃道宿次去之,即各得日月食甚
 宿度及分。



 ○步五星第七



 木星



 周率:二百八萬六千一百四十二,五十四秒。



 歷率:二千二百六十五萬五百七。



 歷度法:六萬二千一十四。



 周日:三百九十八日,八十八分。



 歷度:三百六十五度,二十四分,八十二秒。



 歷中:一百八十二度,六十二分,四十一秒。



 歷策:一十五度,二十一分,八十七秒。



 伏見:一十三度。



 以下表格略



 火星



 周率:四百七萬九千四十一,秒九十七。



 歷率:三百五十九萬二千七百五十八,秒三十二。



 歷度法:九千八百三十六半。



 周日:七百七十九日,九十三分,一十六秒。



 歷度:三百六十五度,二十四分,七十六秒。



 歷中:一百八十二度,六十二
 分,三十八秒。



 歷策:一十五度,二十一分,八十六秒。



 伏見:一十九度。



 以下表格略



 土星



 周率:一百九十七萬七千四百一十二,秒四十六。



 歷率:五千六百二十二萬三千二百一十九。



 歷度法:一十五萬三千九百二十八。



 周日:三百七十八日,九分,三秒。



 歷度:三百六十五度,二十五分,六十六秒。



 歷中:一百八十二度,六十二分,八十三秒。



 歷策:一十五度,二十一分,九十秒。



 伏見:一十七度。



 以下表格略



 金星



 周率:三百五萬三千八百四,秒二十三。



 歷率:一百九十萬二百四十,秒一十一。



 歷度法:五千
 二百三十。



 周日:五百八十三日,九十分,一十四秒。



 合日:二百九十一日,九十五分,七秒。



 歷度:三百六十五度,二十四分,六十八秒。



 歷中:一百八十二度,六十二分,三十四秒。



 歷策:一十五度,二十一分,八十六秒。



 伏見:一十度半。



 以下表格略



 水星



 周率:六十萬六千三十一,秒八十四。



 歷率:一百九十一萬二百四十二,秒三十五。



 歷度法:五千二百三十。



 周日:一百一十五日,八十七分,六十秒。



 合日:五十七日,九十三分,八十秒。



 歷度:三百六十五度,二十四分,七十一秒。



 歷中:一百八十二度,六十二分,三十五秒半。



 歷策:一十五度,二十一分,八
 十六秒。



 晨伏夕見:一十四度。



 夕伏晨見:一十九度。



 以下表格略



 求五星天正冬至後平合及諸段中積中星



 置通積分,各以其星周率去之。不盡,為前合分。覆減周率,餘為後合分。如日法而一,不滿退除為分秒,即其星天正冬至後平合中積、中星。命為日,曰中積。命為度,曰中星。以段日累加中積,即為諸段中積。以平度累加中星,經退減之,即為諸段中星。



 求五星平合及諸段入歷



 置前通積分,各加其星後合分,以歷率去之,不盡,各以其星歷度法除為度,不滿退為分秒,即為其星平合入
 歷度及分秒。以諸段限度累加之,即得諸段入歷。



 求五星平合及諸盈縮差



 各置其星其段入歷度及分秒,如在歷中以下,為在盈;以上,減去歷中,餘為在縮。以其星歷策除之為策數,不盡為入策度及分,命策數算外,以其策數下損益率乘之,如歷策而一為分,以損益其下盈縮積度,即為其星其段盈縮定差。



 求五星平合及諸段定積



 各置其星其段中積,以其盈縮定差盈加減之。即其段定積日及分。以加天正冬至大餘及約分,滿紀法六
 十去之,不盡,即為定日及加時分秒。不滿命甲子算外,即得日辰。



 求五星及諸段所在日月



 各置其段定積日及分,以加天閏日及分,滿朔策及約分除之為月數,不盡,為入月已來日數及分。其月數命天正十一月算外,即得其段入月經朔日數及分,以日辰相距為所在定朔月日。



 求五星平合及諸段加時定星



 各置中星,以盈縮定差盈加縮減之,金星倍之,水星三因之,然後加減。即為五星諸段定星。以加天正冬至加時黃道日度,依
 宿命之,即其星其段加時所在宿度及分秒。



 求五星諸段初日晨前夜半定星



 各以其段初行率,乘其段定積日下加時分,百約之,乃順減退加其日加時定星,即為其段初日晨前夜半定星所在宿度。



 求諸段日率度率



 各以其段日辰距後段日辰為日率。以其段夜半宿次與後段夜半宿次相減,餘為夜率。



 求諸段平行分



 各置其段度率及分秒,以其段日率除之,即其段平行
 度及分秒。



 求諸段總差日差



 以本段前後平行分相減,餘為其段汎差。假令求木星次疾*差,乃以順疾、順遲平行分相減,餘為次疾泛差。他皆仿此。倍而退位為增減差,加減其段平行分,為初末日行分。前多後少者,加為初,減為末。前少後多者,減為初,加為末。倍增減差為總差,以日率減一除之,為日差。



 求前後伏遲退段增減差



 前伏者,置後段初日行分,加其日差之半,為末日行分。後伏者,置前段末日行分,加其日差之半,為初日行分。以減伏段平行分,餘為增減差。前遲者,置前段末日行
 分,倍其日差減之,為初日行分。後遲者,置後段初日行分,倍其日差減之,為末日行分。以遲段平行分減之,餘為增減差。前後近留之遲段。



 木、火、土三星退行者,六因平行分,退一位,為增減差。



 金星前後伏退,三因平行分,半而退位,為增減差。前退者,置後段初日行分,以其日差減之,為末日行分,後退者,置前段末日行分,以其日差減之,為初日行分。以本段平行分減,餘為增減差。



 水星,半平行分為增減差,皆以增減差加減平行分,為初末日行分。前多後少,加初減末;前少後多,減初加末。又倍增減差為總差,
 以日率減一除之,為日差。



 求每日晨前夜半星行宿次



 各置其段初日行分,以日差累損益之後少則損之,後多則益之。為每日行度及分秒。乃順加退減之,滿宿次去之,即得每日晨前夜半星行宿次。視前段末日、後段初日行分相較之數,不過一二日差為妙。或多日差數倍,或顛倒不倫,當類會前後增減差稍損益之,使其有倫,然後用之。或前後平行俱多俱少,則平注之。或總差之秒,不盈一分,亦平注之。若有不倫而平注之得倫者,亦平注之。



 求五星平合及見伏入氣



 置定積,以氣策及約分除之,為氣數,不滿為入氣日及分秒,命天正冬至算外,即所求平合及伏見入氣日及
 分秒。



 求五星平合及見伏行差



 各以其段初日星行分與其太陽行分相減,餘為行差。若金在退行,水在退合者,相併為行差。如水星夕伏晨見者,直以太陽行分為行差。



 求五星定合見伏汎積



 木、火、土三星,各以平合晨疾夕伏定積,便為定合定見定伏汎積。金、水二星,置其段盈縮差,水星倍之。各以行差除之,為日,不滿退除為分秒。若在平合夕見晨伏者,盈減縮加;如在退合夕伏晨見者,盈加縮減。皆以加減定積,
 為定合定見定伏汎積。



 求五星定合定積定星



 木、火、土三星,各以平合行差除其日太陽盈縮差,為距合差日。以太陽盈縮差減之,為距合差度。日在盈歷,以差日差度減之。在縮,加之。加減其星定合汎積,為定合定積定星。



 金、水二星順合退合,各以平合退合行差除其日太陽盈縮差,為距合差日。順加退減太陽盈縮差,為距合差度。順在盈歷,以差日差度加之;在縮,減之。退在盈歷,以差日減之,差度加之;在縮,以差日加之,差度減之。皆以加減其星定合及再定合汎積,為定合再定
 合定積定星。以冬至大餘及約分,加定積,滿紀法去,命,即得定合日辰。以冬至加時黃道日度,加定星,滿宿次去之,即得定合所在宿次。其順退所在盈縮,太陽盈縮也。



 求木水土三星定見伏定積日



 各置其星定見伏汎積,晨加夕減象限日及分秒,半中限為象限,如中限以下,自相乘,以上,覆減歲周日及分秒,餘亦自相乘,滿七十五而一,所得,以其星伏見度乘之,十五除之,為差。其差如其段行差而一,為日,不滿退除為分秒。見加伏減汎積為定積。加命如前,即得日辰也。



 求金水二星定見伏定積日



 各
 以伏見日行差,除其日太陽盈縮差,為日。若晨伏夕見,日在盈歷,加之,在縮,減之。如夕伏晨見,日在盈歷,減之,在縮,加之。加減其星汎積為常積。視常積,如中限以下,為冬至後,以上,去之,餘為夏至後。其二至後,如象限以下,自相乘,以上,覆減中限,亦自相乘,各如法而一,為分。冬至後晨,夏至後夕,以一十八為法。冬至後夕,夏至後晨,以七十五為法。以伏見度乘之,十五除之,為差。差滿行差而一,為日,不滿退除為分秒。加減常積為定積。冬至後晨見夕伏,加之;夕見晨伏,減之。夏至後晨見夕伏,減之;夕見晨伏,加之也。加命如前,即得定見伏日辰。



 其水星,夕疾,在大暑氣初日至立冬氣九日三十五分以下者,不見。晨留,在
 大寒氣初日至立夏氣九日三十五分以下者,春不晨見,秋不夕見者,亦舊有之矣。



 渾象



 古之言天者有三家:一曰蓋天,二曰宣夜,三曰渾天。漢靈帝時,蔡邕於朔方上書,言「宣夜之學,絕無師法」;《周髀》術數具存,考驗天狀,多所違失;惟有渾天為近,最得其情,近世太史候臺銅儀是也。立八心體圓而具天地之形,以正黃道赤道之表裏,以行日月之度數,步五緯之遲速,察氣候之推遷,精微深妙,百代所不可廢者也。然傳歷久遠,製造者眾,測候占察,互有得失,張衡之制,謂之《
 靈憲》,史失其傳。魏、晉以來,官有其器,而無本書,故前志亦闕。吳中常侍王蕃云:「渾天儀者,羲和之舊器,謂之機衡。」積代相傳,沿革不一。宋太平興國中,蜀人張思訓首創其式,造之禁中,踰年而成,詔置文明殿東鼓樓下,曰「太平渾儀」。自思訓死,璣衡斷壞,無復知其法制者。景德中,歷官韓顯符依仿劉曜時、孔挺、晁崇之法,失之簡略。景祐中,冬官正舒易簡乃用唐梁令瓚、僧一行之法,頗為詳備,亦失之於密而難為用。元祐時,尚書右丞蘇頌與昭文館校理沈括奉敕詳定《渾儀法要》,遂奏舉吏部勾當官韓公廉通《九章勾股法》,常以推考天度與張
 衡、王蕃、僧一行、梁令瓚、張思訓法式,大綱可以尋究。若據算術考案象器,亦能成就,請置局差官製造。詔如所言。奏鄭州原武主簿王沇之,太史局官周日嚴、于太古、張促宣,同行監造。制度既成,詔置之集英殿,總謂之渾天儀。公廉交造儀時,先撰《九章勾股驗測渾天書》一卷,貯之禁中,今失其傳,故世無知者。



 舊制渾儀,規天矩地,機隱於內,上布經躔,次具日月五星行度,以察其寒暑進退,如張衡渾天、開元水運銅渾儀者,是也。久而不合,乖於施用。公廉之制則為輪三重:一曰六合儀,縱置地渾中,即天經環也,與地渾相結,其體不動;二曰三辰儀,
 置六合儀內;三曰四游儀,置三辰儀內。植四龍柱於地渾之下,又置鰲雲於六合儀下。四龍柱下設十字水趺,鑿溝道通水以平高下。別設天常單環於六合儀內,又設黃道赤道二單環,皆置三辰儀內,東西相交,隨天運轉,以驗列舍之行。又為四象環,附三辰儀,相結於天運環,黃赤道兩交為直距二縱置於四游儀內。北屬六合儀地渾之上,以正北極出地之度。南屬六合儀地渾之下,以正南極入地之度。此屬儀之大形也。直距內夾軒望筒一,於筒之半設關軸,附直距上,使運轉低昂,筒常指日,日體常在筒竅中,天西行一周,日東移一度,仍以
 窺測四方星度,皆斟酌李淳風、孔挺、韓顯符、舒易簡之制也。三辰儀上設天運環,以水運之。水運之法始於漢張衡,成於唐梁令瓚及僧一行,復於太平興國中張思訓,公廉今又變正其制,設天運環,下以天柱關軸之類上動渾儀,此新制也。



 舊制渾象,張衡所謂置密室中者,推步七曜之運,以度歷象昏明之候,校二十四氣,考晝夜刻漏,無出於渾象。《隋志》稱梁秘府中有宋元嘉中所造者,以木為之,其圓如丸,遍體布二十八宿、三家星色、黃赤道、天河等,別為橫規繞於外,上下半之,以象地也。開元中,詔僧一行與梁令瓚更造銅渾象,為圓天之象,
 上具列宿周天度數,注水激輪,令其自轉,一日一夜天轉一周,又別置日月五星循繞,絡在天外,令得運行。每天西轉一匝,日正東行一度,月行一十三度有奇,凡二十九轉而日月會,三百六十五轉而日行一匝。仍置木櫃以為地平,令象半在地上,半在地下,又立二木偶人於地平之前,置鐘鼓使木人自然撞擊以報辰刻,命之曰《水運渾天俯視圖》。既成,命置之武成殿。



 宋太史局舊無渾象,太平興國中,張思訓準開元之法,而上以蓋為紫宮,旁為周天度,而東西轉之,出新意也。



 公廉乃增損《隋志》制之,上列二十八宿周天度數,及紫微垣中外官
 星,以俯窺七政之運轉,納於六合儀天經地渾之內,同以木櫃載之。其中貫以樞軸,南北出渾象外,南長北短,地渾在木櫃面,橫置之,以象地。天經與地渾相結,縱置之,半在地上,半隱地下,以象天。其樞軸北貫天經上杠中,末與杠平,出櫃外三十五度稍弱,以象北極出地。南亦貫天經出下杠外,入櫃內三十五度少弱,以象南極入地。就赤道為牙距,四百七十八牙以銜天輪,隨機輪地轂正東西運轉,昏明中星既應其度,分至節氣亦驗應而不差。



 王蕃云:「渾象之法,地當在天內,其勢不便,故反觀其形,地為外郭,於已解者無異,詭狀殊體而合于
 理,可謂奇巧者也。」今地渾說在渾象外,蓋出于王蕃制也。其下則思訓舊制,有樞輪關軸,激水運動,以直神搖鈴扣鐘擊鼓,置時刻十二神司辰像於輪上,時初、正至,則執牌循環而出,報隨刻數以定晝夜長短。至冬水凝,運轉遲澀,則以水銀代之。



 今公廉所製,共置一臺,臺中有二隔,渾儀置其上,渾象置其中,激水運轉,樞機輪軸隱于下。內設晝夜時刻機輪五重;第一重曰天輪,以撥渾象赤道牙距;第二重曰撥牙輪,上安牙距,隨天柱中輪轉動,以運上下四輪;第三重曰時刻鐘鼓輪,上安時初、時正百刻撥牙,以扣鐘擊鼓搖鈴;第四重曰日時初
 正司辰輪,上安時初十二司辰、時正十二司辰;第五重曰報刻司辰輪,上安百刻司辰。以上五輪並貫於一軸,上以天束束之,下以鐵杵臼承之,前以木閣五層蔽之,稍增異其舊制矣。五輪之北,又側設樞輪,其輪以七十二輻為三十六洪,束以三輞,夾持受水三十六壺。轂中橫貫鐵樞軸一,南北出軸為地轂,運撥地輪。天柱中輪動,機輪動渾象,上動渾天儀。又樞輪左設天池、平水壺,平水壺受天池水,注入受水壺,以激樞輪。受水壺落入退水壺。由壺下北竅引水入升水下壺,以昇水下輪運水入昇水上壺,上壺內升水上輪及河車同轉上下輪,
 運水入天河,天河復流入天地,每一晝一夜周而復始。此公廉製渾儀、渾象二器而通三用,總而名之曰渾天儀。



 金既取汴,皆輦致于燕,天輪赤道牙距撥輪懸象鐘鼓司辰刻報天池水壺等器久皆棄毀,惟銅渾儀置之太史局候臺。但自汴至燕相去一千餘里,地勢高下不同,望筒中取極星稍差,移下四度纔得窺之。明昌六年秋八月,風雨大作,雷電震擊,龍起渾儀鰲雲水趺下,臺忽中裂而摧,渾儀仆落臺下,旋命有司營葺之,復置臺上。貞祐南渡,以渾儀熔鑄成物,不忍毀拆,若全體以運則艱於輦載,遂委而去。



 興定中,司天臺官以臺中不
 置渾儀及測候人數不足,言之於朝,宜鑄儀象,多補生員,庶得盡占考之實。宣宗召禮部尚書楊雲翼問之,雲翼對曰:「國家自來銅禁甚嚴,雖罄公私所有,恐不能給。今調度方殷,財用不足,實未可行。」他日,上又言之,於是止添測候之人數員,鑄儀之議遂寢。



 初,張行簡為禮部尚書提點司天監時,嘗制蓮花、星丸二漏以進,章宗命置蓮花漏二禁中,星丸漏遇車駕巡幸則用之。貞祐南渡,二漏皆遷於汴,汴亡廢毀,無所稽其制矣。



\end{pinyinscope}