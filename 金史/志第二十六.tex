\article{志第二十六}

\begin{pinyinscope}

 刑



 昔者先王因人之知畏而作刑,因人之知恥而作法。畏也、恥也,五性之良知,七情之大閑也。是故,刑以治已然,法以禁未然,畏以處小人,恥以遇君子。君子知恥,小人知畏,天下平矣!是故先王養其威而用之,畏可以教愛。慎其法而行之,恥可以立廉。愛以興仁,廉以興義,仁義興,刑法不幾於措乎?金初,法制簡易,無輕重貴賤之別,
 刑、贖並行,此可施諸新國,非經世久遠之規也。天會以來,漸從吏議,皇統頒制,兼用古律。厥後,正隆又有《續降制書》。大定有《權宜條理》,有《重修制條》。明昌之世,《律義》、《敕條》並修,品式當浸備。既而《泰和律義》成書,宜無遺憾。然國脈紓蹙,風俗醇樗,世道升降,君子觀一代之刑法,每有以先知焉。金法以杖折徒,累及二百,州縣立威,甚者置刃於杖,虐於肉刑。季年,君臣好用筐篋故習,由是以深文傅致為能吏,以慘酷辦事為長才。百司姦贓真犯,此可決也,而微過亦然。風紀之臣,失糾皆決。考滿,校其受決多寡以為殿最。原其立法初意,欲以同疏戚、壹小大,
 使之咸就繩約於律令之中,莫不齊手並足以聽公上之所為,蓋秦人強主威之意也。是以待宗室少恩,待大夫士少禮。終金之代,忍恥以就功名,雖一時名士有所不免。至於避辱遠引,罕聞其人。殊不知君子無恥而犯義,則小人無畏而犯刑矣。是故論者於教愛立廉之道,往往致太息之意焉。雖然,世宗臨御,法司奏讞,或去律援經,或揆義制法。近古人君聽斷,言幾於道,鮮有及之者。章宗、宣宗嘗親民事,當寧裁決,寬猛出入雖時或過中,迹其矜恕之多,猶有祖風焉。簡牘所存,可為龜鑑者,《本紀》、《刑志》詳略互見云。



 金國舊俗,輕罪笞以柳篸,殺人及盜劫者,擊其腦殺之,沒其家貲,以十之四入官,其六償主,併以家人為奴婢。其親屬欲以馬牛雜物贖者從之。或重罪亦聽自贖,然恐無辨於齊民,則劓、刵以為別。其獄則掘地深廣數丈為之。太宗雖承太祖無變舊風之訓,亦稍用遼、宋法。天會七年,詔凡竊盜,但得物徒三年,十貫以上徒五年,刺字充下軍,三十貫以上終身,仍以贓滿盡命刺字於面,五十貫以上死,徵償如舊制。熙宗天眷元年十月,禁親王以下佩刀入宮。衛禁之法,實自此始。三年,復取河南地,乃詔其民,約所用刑法皆從律文,罷獄卒酷毒刑
 具,以從寬恕。至皇統間,詔諸臣,以本朝舊制,兼採隋、唐之制,參遼、宋之法。類以成書,名曰《皇統制》,頒行中外。時制,杖罪至百,則臀、背分決。及海陵庶人以脊近心腹,遂禁之,雖主決奴牌,亦論以違制。又多變易舊制,至正隆間,著為《續降制書》,與《皇統制》並行焉。然二君任情用法,自有異於是者矣。及世宗即位,以正隆之亂,盜賊公行,兵甲未息,一時制旨多從時宜,遂集為《軍前權宜條理》。大定四年,尚書省奏:「大興民男子李十、婦人楊仙哥並以亂言當斬。」上曰:「愚民不識典法,有司亦未嘗丁寧誥戒,豈可遽加極刑。」以減死論。五年,命有司復加刪定《條
 理》,與前《制書》兼用。七年,左藏庫夜有盜殺都監郭良臣盜金珠,求盜不得。命點檢司治之,執其可疑者八人鞫之,掠三人死,五人誣伏。上疑之,命同知大興府事移剌道雜治。既而親軍百夫長阿思缽鬻金於市,事覺,伏誅。上聞之曰:「箠楚之下,何求不得,奈何鞫獄者不以情求之乎?」賜死者錢人二百貫,不死者五十貫。於是禁護衛百夫長、五十夫長非直日不得帶刀入宮。是歲,斷死囚二十人。八年,制品官犯賭博法,贓不滿五十貫者其法杖,聽贖。再犯者杖之。且曰:「杖者所以罰小人也。既為職官,當先廉恥,既無廉恥,故以小人之罰罰之。」九年,因御
 史臺奏獄事,上曰:「近聞法官或各執所見,或觀望宰執之意,自今制無正條者皆以律文為準。」復命杖至百者臀、背分受,如舊法。已而,上謂宰臣曰:「朕念罪人杖不分受,恐至深重,乃令復舊。今聞民間有不欲者,其令罷之。」十年,尚書省奏:「河中府張錦自言復父仇,法當死。」上曰:「彼復父仇,又自言之,烈士也。以減死論。」十一年,詔諭有司曰:「應司獄廨舍須近獄安置,囚禁之事常親提控,其獄卒必選年深而信實者輪直。」十二年,尚書省言:「內丘令蒲察臺補自科部內錢立德政碑,復有其餘錢二百餘貫,罪當除名。今遇赦當敘,仍免征贓。」上以貪偽,勿敘,
 且曰:「乞取之贓,若有赦原,予者何辜?自今可並追還其主,惟應入官者免征。」尚書省奏,盜有發塚者,上曰:「功臣墳墓亦有被發者,蓋無告捕之賞,故人無所畏。自今告得實者量與給賞。」故咸平尹石抹阿沒剌以贓死於獄,上謂:「其不尸諸市已為厚幸。貧窮而為盜賊,蓋不得已。三品職官以贓至死,愚亦甚矣!其諸子可皆除名。」先是,詔自今除名人子孫有在仕者並取奏裁。十三年,詔立春後、立秋前,及大祭祀,月朔、望,上、下弦,二十四氣,雨未晴,夜未明,休暇并禁屠宰日,皆不聽決死刑,惟強盜則不待秋後。十五年,詔有司曰:「朕惟人命至重,而在制竊
 盜贓至五十貫者處死,自今可令至八十貫者處死。」十七年,陳言者乞設提刑司,以糾諸路刑獄之失。尚書省議,以謂久恐滋弊。上乃命距京師數千里外懷冤上訴者,集其事以待選官就問。



 時濟南尹梁肅言,犯徒者當免杖。朝廷以為今法已輕於古,恐滋姦惡,不從。嘗詔宰臣,朝廷每歲再遣審錄官,本以為民伸冤滯也,而所遣多不盡心,但文具而已。審錄之官,非止理問重刑,凡訴訟案牘,皆當閱實是非,囚徒不應囚繫則當釋放,官吏之罪即以狀聞,失糾察者嚴加懲斷,不以贖論。又以監察御史體察東北路官吏,輒受訟牒,為不稱職,笞之五
 十。又謂宰臣曰:「比聞大理寺斷獄,雖無疑者亦經旬月,何耶?」參知政事移剌道對曰:「在法,決死囚不過七日,徒刑五日,杖罪三日。」上曰:「法有程限,而輒違之,弛慢也。」罷朝,御批送尚書省曰:「凡法寺斷重輕罪各有期限,法官但犯皆的決,豈敢有違。但以卿等所見不一,至於再三批送,其議定奏者書奏牘亦不下旬日,以致事多滯留,自今當勿復爾。」又曰:「故廣寧尹高楨為政尚猛,雖小過,有杖而殺之者。即罪至於死而情或可恕,猶當念之,況其小過者乎?人之性命安可輕哉!」上以正隆《續降制書》多任己意,傷於苛察。而與皇統之《制》並用,是非淆亂,莫
 知適從,姦吏因得上下其手。遂置局,命大理卿移剌綎總中外明法者共校正。乃以皇統、正隆之《制》及大定《軍前權宜條理》、後《續行條理》,倫其輕重,刪繁正失。制有闕者以律文足之。制、律俱闕及疑而不能決者,則取旨畫定。《軍前權宜條理》內有可以常行者亦為定法,餘未應者亦別為一部存之。參以近所定徒杖減半之法,凡校定千一百九十條,分為十二卷,以《大定重修制條》為名,詔頒行焉。



 二十年,上見有蹂踐禾稼者,謂宰相曰:「今後有踐民田者杖六十,盜人穀者杖八十,並償其直。」二十一年,尚書省奏:「鞏州民馬俊妻安姐與管卓姦,俊以斧
 擊殺之,罪當死。」上曰:「可減死一等,以戒敗風俗者。」二十二年,上謂宰臣曰:「凡尚書省送大理寺文字,一斷便可聞奏。如烏古論公說事,近取觀之,初送法寺如法裁斷,再送司直披詳,又送闔寺參詳,反覆三次,妄生情見,不得結絕。朕以國政不宜滯留,昨雖炙艾六百炷,未嘗一日不坐朝,欲使卿等知勤政也。自今可止一次送寺,闔寺披詳,荀有情見即具以聞,毋使滯留也。」二十三年,尚書省奏:「益都民范德年七十六,為劉祐毆殺。祐法當死,以祐父母年俱七十餘,家無侍丁,上請。」上曰:「范德與祐父母年相若,自當如父母相待,至毆殺之,難議末減,其
 論如法。」尚書省奏招討司官及禿里乞取本部財物制,上曰:「遠人止可矜恤,若進貢不闕,更以兵邀之,強取財物,與盜何異?且或因而生事,何可不懲。」又曰:「朕所行制條,皆臣下所奏行者,天下事多,人力有限,豈能一一盡之。必因一事奏聞,方知有所窒礙,隨即更定。今有聖旨、條理,復有制條,是使姦吏得以輕重也。」大興府民趙無事帶酒亂言,父千捕告,法當死。上曰:「為父不恤其子而告捕之,其正如此,人所甚難。可特減死一等。」武器署丞奕、直長骨赧坐受草畔子財,奕杖八十,骨赧笞二十,監察御史梁襄等坐失糾察罰俸一月。上曰:「監察,人君之
 耳目。事由朕發,何以監察為?」上以法寺斷獄,以漢字譯女直字,會法又復各出情見,妄生穿鑿,徒致稽緩,遂詔罷情見。二十五年二月,上以婦人在囚,輸作不便,而杖不分決,與殺無異,遂命免死輸作者,決杖二百而免輸作,以臀、背分決。時后族有犯罪者,尚書省引「八議」奏,上曰:「法者,公天下持平之器,若親者犯而從減,是使之恃此而橫恣也。昔漢文誅薄昭,有足取者。前二十年時,后族濟州節度使烏林達鈔兀嘗犯大辟,朕未嘗宥。今乃宥之,是開後世輕重出入之門也。」宰臣曰:「古所以議親,尊天子,別庶人也。」上曰:「外家自異於宗室,漢外戚權太
 重,至移國祚,朕所以不令諸王、公主有權也。夫有功於國,議勛可也。至若議賢,既曰賢矣,肯犯法乎?脫或緣坐,則固當減請也。」二十六年,遂奏定太子妃大功以上親,及與皇家無服者、及賢而犯私罪者,皆不入議。上謂宰臣曰:「法有倫而不倫者,其改定之。」監察御史陶鈞以攜妓遊北苑,歌飲池島間,迫近殿廷,提控官石玠聞而發之。鈞令其友閻恕屬玠得緩。既而事覺,法司奏,當徒二年半。詔以鈞耳目之官,攜妓入禁苑,無上下之分,杖六十,玠、恕皆坐之。二十八年,上以制條拘於舊律,間有難解之詞,命刪修明白,使人皆曉之。



 舊禁民不得收制書,
 恐滋告訐之弊,章宗大定二十九年,言事者乞許民藏之。平章張汝霖曰:「昔子產鑄刑書,叔向譏之者,蓋不欲預使民測其輕重也。今著不刊之典,使民曉然知之,猶江、河之易避而難犯,足以輔治,不禁為便。」以眾議多不欲,詔姑令仍舊禁之。



 明昌元年,上問宰臣曰:「今何不專用律文?」平章政事張汝霖曰:「前代律與令各有分,其有犯令,以律決之。今國家制、律混淆,固當分也。」遂置詳定所,命審定律、令。承安二年,制軍前受財法,一貫以下,徒二年,以上徒三年,十貫處死。符寶典書北京奴盜符寶局金牌,伏誅,仍除屬籍。按虎、阿虎帶失覺察,各杖七十。
 泰和二年,御史臺奏:「監察御史史肅言,《大定條理》:自二十年十一月四日以前,奴娶良人女為妻者,並準已娶為定,若夫亡,拘放從其主。離夫摘賣者令本主收贖,依舊與夫同聚。放良從良者即聽贖換,如未贖換間與夫所生男女並聽為良。而《泰和新格》復以夫亡服除準良人例,離夫摘賣及放夫為良者,並聽為良。若未出離再配與奴,或雜姦所生男女並許為良。如此不同,皆編格官妄為增減,以致隨處訴訟紛擾,是涉違枉。」敕付所司正之。初,詔凡條格入制文內者,分為別卷。復詔制與律文輕重不同,及律所無者,各校定以聞。如禁屠宰之類,
 當著于令也,慎之勿忽,律令一定,不可更矣。明昌三年七月,右司郎中孫鐸先以詳定所校《名例篇》進,既而諸篇皆成,復命中都路轉運使王寂、大理卿董師中等重校之。四年七月,上以諸路枷杖多不如法,平章政事守貞曰:「枷杖尺寸有制,提刑兩月一巡察,必不敢違法也。」五年正月,復令鉤校制、律,即付詳定所。時詳定官言:「若依重修制文為式,則條目增減,罪名輕重,當異於律。既定復與舊同頒,則使人惑而易為姦矣!臣等謂,用今制條,參酌時宜,準律文修定,歷採前代刑書宜於今者,以補遺闕,取《刑統》疏文以釋之,著為常法,名曰《明昌律義》。別編
 榷貨、邊部、權宜等事,集為《敕條》。」宰臣謂:「先所定令文尚有未完,俟皆通定,然後頒行。若律科舉人,則止習舊律。」遂以知大興府事尼厖古鑒、御史中丞董師中、翰林待制奧屯忠孝小字牙哥,提點司天臺張嗣、翰林修撰完顏撒剌、刑部員外郎李庭義、大理丞麻安上為校定官,大理卿閻公貞,戶部侍郎李敬義、工部郎中賈鉉為覆定官,重修新律焉。時奏獄而法官有獨出情見者,上曰:「或言法官不當出情見,故論者紛紛不已。朕謂情見非出於法外,但折衷以從法爾。」平章守貞曰:「是制自大定二十三年罷之。然律有起請諸條,是古亦許情見矣。」上曰:「科
 條有限,而人情無窮,情見亦豈可無也。」明昌五年,尚書省奏:「在制,《名例》內徒年之律,無決杖之文便不用杖。緣先謂流刑非今所宜,且代流役四年以上俱決杖,而徒三年以下難復不用。婦人比之男子雖差輕,亦當例減。」遂以徒二年以下者杖六十,二年以上杖七十,婦人犯者並決五十,著於《敕條》。



 承安三年,敕尚書省:「自今特旨事,如律令程式者,始可送部。自餘創行之事,但召部官赴省議之。」四年四月,尚書省請再覆定令文,上因敕宰臣曰:「凡事理明白者轉奏可也。文牘多者恐難遍覽,其三推情疑以聞。」五月,上以法不適平,常行杖樣,多不能
 用。遂定分寸,鑄銅為杖式,頒之天下。且曰:「若以笞杖太輕,恐情理有難恕者,訊杖可再議之。」五年五月,刑部員外郎馬復言:「外官尚苛刻者不遵銅杖式,輒用大杖,多致人死。」詔令按察司糾劾黜之。先嘗令諸死囚及除名罪,所委官相去二百里外,并犯徒以下逮及二十人以上者,並令其官就讞之。刑部員外郎完顏綱言:「自是制行,如上京最近之地往還不下三、二千里,如北京留守司亦動經數月,愈致稽留,未便。」詔復從舊,令委官追取鞫之。



 十二月,翰林修撰楊庭秀言:「州縣官往往以權勢自居,喜怒自任,聽訟之際,鮮克加審。但使譯人往來傳
 詞,罪之輕重,成於其口,貨賂公行,冤者至有三、二十年不能正者。」上遂命定立條約,違者按察司糾之。且謂宰臣曰:「長貳官委幕職及司吏推問獄囚,命申御史臺聞奏之制,當復舉行也。」又命編前後條制,書之於冊,以備將來考驗。



 泰和元年正月,尚書省奏,以見行銅杖式輕細,姦宄不畏,遂命有司量所犯用大杖,且禁不得過五分。



 十二月,所修律成,凡十有二篇:一曰《名例》,二曰《衛禁》,三曰《職制》,四曰《戶婚》五曰《廄庫》,六曰《擅興》,七曰《賊盜》,八曰《鬥訟》,九曰《詐偽》,十曰《雜律》,十一曰《捕亡》,十二曰《斷獄》。實《唐律》也,但加贖銅皆倍之,增徒至四年、五年為七,削
 不宜於時者四十七條,增時用之制百四十九條,因而略有所損益者二百八十有二條,餘百二十六條皆從其舊。又加以分其一為二、分其一為四者六條,凡五百六十三條,為三十卷,附注以明其事,疏義以釋其疑,名曰《泰和律義》。自《官品令》、《職員令》之下,曰《祠令》四十八條,《戶令》六十六條,《學令》十一條,《選舉令》八十三條,《封爵令》九條、《封贈令》十條,《宮衛令》十條,《軍防令》二十五條,《儀制令》二十三條,《衣服令》十條,《公式令》五十八條,《祿令》十七條,《倉庫令》七條,《廄牧令》十二條,《田令》十七條,《賦役令》二十三條,《關市令》十三條,《捕亡令》二十條,《賞令》二十五條,《
 醫疾令》五條,《假寧令》十四條,《獄官令》百有六條,《雜令》四十九條,《釋道令》十條,《營繕令》十三條,《河防令》十一條,《服制令》十一條,附以年月之制,曰《律令》二十卷。又定《制敕》九十五條,《榷貨》八十五條,《蕃部》三十九條,曰《新定敕條》三卷,《六部格式》三十卷。司空襄以進,詔以明年五月頒行之。



 貞祐三年,上謂宰臣:「自今監察官犯罪,其事關軍國利害者,並笞決之。」貞祐四年,詔:「凡監察失糾劾者,從本法論。外使入國私通本國事情,宿衛、近侍官、承應人出入親王、公主、宰執家,災傷乏食有司檢核不實致傷人命,轉運軍儲而有私載,考試舉人而防閑不嚴,其罰
 並決。在京犯至兩次者,臺官減監察一等治罪,論贖,餘止坐,專差任滿日議定。若任內曾以漏察被決,依格雖為稱職,止從平常,平常者從降罰。」興定元年八月,上謂宰臣曰:「律有八議,今言者或謂應議之人即當減等,何如?」宰臣對曰:「凡議者先條所坐及應議之狀以請,必議定然後奏裁也。」上然之,曰:「若不論輕重而輒減之,則貴戚皆將恃此以虐民,民何以堪。」



\end{pinyinscope}