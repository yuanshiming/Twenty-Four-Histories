\article{志第二十二}

\begin{pinyinscope}

 儀衛上



 ○常朝儀衛內外立仗常行儀衛行仗法駕黃麾仗



 金制,天子之儀衛,一曰立仗,二曰行仗。其衛士,曰護衛、曰親軍、曰弩手,曰控鶴、曰傘子、曰長行。立仗則有殿庭內仗、殿庭外仗,凡大禮、大朝會則用之。其朔望常朝,弩手百人分立兩階而已。行仗則有法駕、大駕、黃麾仗,凡行幸及郊廟祀享則用之。其非大禮遠出,則有常行儀衛、宮中導從焉。大抵模仿宋制,錯綜增損而用之。其宿
 衛則見《兵志》云。



 初,國制,凡朔望常朝日,殿下列衛士,簾下置甲兵。正隆元年,海陵去甲兵,惟存錦衣弩手百人,分立兩階。其儀,都副點檢,公服偏帶。常朝則展紫。左右衛將軍、宿直將軍,展紫,金束帶,各執玉、水晶及金飾骨朵。左右親衛,盤裹紫襖,塗金束帶,骨朵,佩兵械。供御弩手、傘子百人,並金花交腳襆頭,塗金銅鈒襯花束帶,骨朵。左右班執儀物內侍二十人,展紫,塗金束帶。朝參日,弩手、傘子直於殿門外,分兩面排立。司辰報時畢,皇帝御殿坐,鳴鞭,閣門報班齊。執擎儀物內侍分降殿階,南向立。點檢司起居,弩
 手、傘子於殿門外北面山呼聲喏,訖,即於殿門外東西相向排立。都點檢以次三員升殿,都點檢在東近南,左副又少南,右副在西,東向對立。左右衛將軍在殿下東西對立。省臣隨班起居畢,左右司侍郎從宰執奏事。殿中侍御史隨班起居畢,東西對立於左右衛將軍之北,少前。修起居注分殿陛東西對立於殿欄外副階下,以俟。奏事畢,皇帝還閣,侍衛者乃退。



 凡遇大禮、大朝會,則有內外立仗。熙宗皇統元年正月,上冊寶,立仗一千一百八十人。自是以後,至海陵時,俱用三千人。世宗大定七年,上冊寶,頗損其數,且以天德、
 貞元不設車輅,遂并去之。是後,或減至二千,或一千、或八百、或六百人。天德二年,海陵立后,發冊勤政殿,設黃麾細仗,用前六部,攝官七十一,擎執六百七十八人。受冊泰和殿,用後六部,攝官三十六,擎執三百二十二人。大定八年正月,冊皇太子於大安殿,用黃麾半仗二千二百六十五人,奉表于仁政殿用黃麾細仗一千四百二人。二十七年,冊皇太孫,亦如之。



 大定八年,黃麾半仗,攝官一百七十五人,擎執二千八十一人,編排職掌九人。



 殿庭內仗。以中心東西相向一重,并面北旗幟為中道。左行,自北西向排列。黃麾幡一首,執者三人。碧襴
 官一,大雉扇二。碧襴官一,中雉扇六,碧襴官一,小雉扇六。碧襴官一,朱團扇六。碧襴官一,睥睨四。碧襴官一,紅大傘一。碧襴官一,紫方傘二。碧襴官一,華蓋一。右行,東向列者,並同。面北,第一行,牙門旗八,共二十四人,分左右。留中道。第二行,監門校尉十二,分左右。第三行,長壽幢一,押旗大將軍一,居中,次東五方龍旗十五,次西五方鳳旗十五。第四行,自內而東,青龍旗五,紅龍旗二十。自內而西,青龍旗五,紅龍旗二十。第五行,同上,又君王萬歲旗一,五人居中。日旗一,五人在左。月旗一,五人在右。第六行,自內而東,天下太平旗、苣紋旗、日月合璧
 旗、苣紋旗、青龍旗、赤龍旗、河瀆旗、江瀆旗各一,旗五人,排仗通直官一,排仗大將一。未、午、巳、辰、卯、寅旗各一,青天王旗、白天王旗各一。自內而西,祥雲旗、五星連珠旗、祥雲旗、黃龍旗、白龍旗、黑龍旗、淮瀆旗、濟瀆旗各一,旗五人,通直官一,大將一。申、酉、戌、亥、子、丑旗各一、緋天王旗、皂天王旗各一。第七行,自內而東,孔雀旗一,五人。蒼烏旗、兕旗、犛牛旗、騼蜀旗,赤熊旗、白狼旗、金鸚鵡旗、馴犀旗、角端旗、狖鸃旗、騶牙旗、野馬旗、瑞麥旗、甘露旗各一,旗五人。自內而西者同。



 外仗。在門外。左邊,西向,自北排列。第一部,第一行,侍御史、大將軍、折衝都尉各一,主
 帥三。第二行,絳引幡五首十五人,龍頭竿四、弓矢五、揭鼓二、龍頭竿四、儀鍠斧五、龍頭竿四、朱刀盾五、龍頭竿四、綠刀盾五、龍頭竿四、小戟五。第三行,與第一行同。第四行,與第二行同。第二部、第三部、第四部、第五部以次而南,各為前後四行,其名數與第一部同,惟無絳引幡。右五部,東向排列,色數皆同。左第五行,從北,每大旗一,均用小紅龍旗二間之。角宿旗一,三人,均用二。亢宿旗一,三人,均用二。氐宿旗一,三人,均用二。房宿旗一,三人,均用二。心宿旗一,三人,均用二。尾宿旗一,三人,均用二。箕宿旗一,三人,均用二。斗宿旗一,三人,均用二。牛宿旗
 一,三人,均用二。女宿旗一,三人,龍旗并黃排襴旗各一。虛宿旗一,三人,紅、黃排襴旗二。危宿旗一,三人,紅、紫排襴旗二。室宿旗一,三人,黃紫排襴旗二。壁宿旗一,三人,紅,黃排襴旗二。重輪旗一,三人,紅、紫排襴旗二。左攝提旗一,三人,黃、紫排襴旗二。青龍旗一,三人,紅、黃排襴旗二。木星旗一,三人,紅、紫排襴旗二。火星旗一,三人,黃、紫排襴旗二。土星旗一,三人,紅、黃排襴旗二。金星旗一,三人,紅、紫排襴旗二。水星旗一,三人,吏兵并紫排襴旗各一。北岳旗一,三人,吏兵並龍君旗各一。東岳旗一,三人,龍君并黃熊旗各一。中岳旗一,三人,黃熊并赤豹旗各
 一。西岳旗一,三人,赤豹并力士旗各一。南岳旗一,三人,力士并虎君旗各一。朱雀旗一,三人,虎君并天馬旗各一。右第五行,從北。奎旗一,三人。婁旗一,三人。胃旗一,三人。昴旗一,三人。畢旗一,三人,觜旗一,三人。參旗一,三人。井旗一,三人。鬼旗一,三人。皆均用二旗如前。柳宿旗一,三人,紅龍并黃排襴旗各一。星宿旗一,三人,紅、黃排襴旗二。張宿旗一,三人,紅、紫排襴旗二。翼宿旗一,三人,紫、黃排襴旗二。軫宿旗一,三人,紅、黃排襴旗二。重輪旗一,三人,紅、紫排襴旗二。右攝提旗一,三人,紫、黃排襴旗二。白虎旗一,三人,紅、黃排襴旗二。東方神旗一,三人,紅、紫
 排襴旗二。南方神旗一,三人,黃、紫排襴旗二,中央神旗一,三人,紅龍排襴旗二。西方神旗一,三人,紅、紫排襴旗二。北方神旗一,三人,力士并紫排襴旗各一。風伯旗一,三人,力士并虎君旗各一。雨師旗一,三人,虎君并黃熊旗二。雷公旗一,三人,黃熊并赤豹旗二。電母旗一,三人,赤豹并吏兵旗二。北斗旗一,三人,吏兵并龍君旗二。玄武旗一,三人,龍君并。天馬旗二。三人執一旗者重立,二人各執小旗者亦重立。



 殿門外仗,亦從北,留中道。飛麟旗、駃騠旗、鸞旗、麟旗、馴象旗各二,共十人,從中分列為第一重。鶡雞旗、貔旗、玉馬旗、三角獸旗、黃鹿旗各二,
 共十人,次外分列為第二重。其次,第一部都尉三員,第二部至第五部俱二員,為第三重。又其次五部,各刀盾二十,為第四重。又其次五部,各弓矢二十,為第五重。左右同。



 黃麾細仗,攝官八十八人,擎執一千三百五人,編排職掌九人。



 內仗,中道左一行,自北西向排列。黃麾幡一首,執者三人。大雉扇六、中雉扇六、小雉扇六、朱團扇六、睥睨四、紅大傘一、紫方傘二、華蓋一,凡傘扇之上皆有碧襴官一。右行東向,排次同。面北,第一行,長壽幢一,居中。牙門旗八,共二十四人,分左右。第二行,君王萬歲
 旗五人,居中。日旗五人,監門校尉五人,在左。月旗五人,監門校尉五人,在右。第三行,五方龍旗十五在左,五方鳳旗十五在右。第四行,紅龍旗三十四,第五行,紅龍旗三十四,皆分左右。第六行,自內而東,太平、苣紋、合壁、苣紋、赤龍、青龍旗各一,旗五人,通直一人,大將一人。未、午、巳、辰、卯、寅旗各一,青天王旗、白天王旗各一。自內而西,祥雲、連珠、祥雲、黃龍。白龍,黑龍旗各一,旗五人,通直一人,大將一人。申、酉、戌、亥、子、丑旗各一,緋天王旗、皁天王旗各一。第七行,自內而東,河瀆、江瀆、兕、赤熊、馴犀、角端、狖鸃、綱子旗各一,旗五人。自內而西,淮瀆、濟瀆、兕、赤熊、
 馴犀、角端、狖鸃、綱子旗各一,旗五人。



 外仗,左邊西向,自北排列,第一行,五部,侍御史、大將軍、折衝都尉各一,主帥各二。第二行,第一部,絳引幡五首,十五人。龍頭竿四、弓矢五、揭鼓二、儀鍠斧五,龍頭竿四、弓矢五、朱刀盾五、綠刀盾五,龍頭竿四、儀鍠斧五、朱刀盾五、綠刀盾五,龍頭竿四、小戟五,龍頭竿四、小戟五。第二部至五部無絳引幡,餘色並同,以次相接而南。右五部東向,亦如之。左第三行,從北,角、亢、氐、房、心、尾、箕、斗、牛、女、虛、危、室、壁旗各一,旗三人。次重輪、左攝提、青龍旗各一,木、火、土、金、水星旗各一,北、東、中、南、西岳旗各一,旗三人。次紫排襴四、
 黃排襴四、紅排襴四、吏兵旗二、天馬旗一。右第三行,從北,奎、婁、胃、昴、畢、觜、參、井、鬼、柳、星、張、翼、軫旗各一,旗三人。次重輪、右揶提、白虎旗各一,東、南、中、西、北方神旗各一,風伯、雨師、雷公、電母、北斗旗各一,旗三人。次紫排襴四、黃排襴四、紅排襴四、吏兵旗二、天馬旗一。



 行仗。天子非祀享巡幸遠出,則用常行儀衛。弩手二百人,軍使五人,控鶴二百人,首領四人,俱服紅地藏根牡丹錦襖、金鳳花交腳襆頭、塗金銀束帶,控鶴或皂帽碧襖,各執金鍍銀蒜瓣骨朵。長行四百人,拳腳襆頭、紅錦四衣癸襖、塗金束帶,二人紫衫前導,無執物,余執列糸骨
 朵七十八、瓜八十八,鐙三十四,在控鶴前,金吾仗八十、金花大劍六十俱垂紅絨結子、儀鍠斧五十八,在控鶴後。其常朝、御殿、郊廟、臨幸,凡步輦出入則有近侍導從,執金鍍銀骨朵者二人,左右扇十人,拂子四人,香盒二人,香球二人,節二人,幢二人,盂一人,唾壺一人,凈巾一人,鐁鑼一人,水罐一人,交椅一人,斧一人,皇帝出閣則分立閣門之外,導引至殿,皇帝升座則降階以俟,入閣然後放仗。



 天眷三年,熙宗幸燕,始備法駕,凡用士卒萬四千五十六人,攝官在外。海陵遷都于燕,用黃麾仗萬三百四十
 八人。天德二年祀廟,用黃麾四千人。世宗即位,凡行幸祀享並用三千人,間亦不滿其數。大定十一年前,祀南郊、朝享太廟及至郊壇,用大駕七千人,此其大較也。



 天眷法駕人數。攝官六百九十九人:將軍、大將軍四十三人,折衝、果毅一百二十六人,校尉五十六人,郎將三十四人,帥兵官二百四十六人,統軍六人,都頭六人,千牛一人,旅帥二人,部轄指揮使二人,押纛二人,押衙四人,四色官四人,押旗二人,引駕官四人,進馬四人,押仗通直二人,押仗大將二人,碧襴一十六人,長史二人,鼓吹令二人,鼓吹丞二人,典事五人,太史令一人,太史正
 一人,司丞一人,府牧一人,刻漏生四人,縣令一人,御史大夫一人,僚佐一十人,進輅職掌二人,夾輅將軍二人,陪輅將軍二人,教馬官二人,四省局官八人,導駕官四十八人,抱駕頭官一人,執扇筤一人,尚輦奉御二人,殿中少監二人,供奉職官二人,令史四人,書令史四人,押仗二人,殿中侍御史二十四人。諸班直隊二千九百四十五人:鈞容直三百六人人員六,長行三百,執旗一百三十六人,引駕六十二人人員二,長行六十,駕頭天武官一十二人,執從物茶酒班一十一人,御龍直仗劍六人,天武把行門八人,殿前班擊鞭一十人,御龍直四十人人員二,長行三十八,骨
 朵直一百三十四人,部押二人,殿前班行門三十五人,捧日馬隊七百人,奉宸步隊七百人,天武骨朵大劍三百一十人人員一十,長行三百,東第四班三十一人人員一,長行三十,扇筤天武二十人,捧日隊從領人員一十七人,簇輦茶酒班三十一人人員一,長行三十,鈞容直三十一人,人員一,長行三十。招箭班三十三人人員三,長行三十,天武約襴三百一十人。人員一十,長行三百。車輅下駕士六百三十八人:玉輅下一百四十人控馬踏路四,駕士一百二十八,挾輅八,金輅下六十四人控馬踏路四,駕士六十,象輅下駕士四十人,革輅、木輅、耕根車駕士同上,革車二,共五十人,指南、記里車各三十人,輅車、鸞旗、皮軒車各
 十八人,黃鉞、豹尾車各十五人,屬車八,共八十人。輦輿下六百八十五人:小輿一,長行二十四人,逍遙一,共三十五人什將節級九,長行二十六,平輦下四十二人什將節級九,人員七,長行二十六,腰輿共一十九人人員一,什將虞候二,長行一十六,大輦下三百七十一人掌輦人員四,什將十二,長行三百五十五,分五番,芳亭輦一,長行六十人,御馬三十二,共百三十四人。控馬,天武官六十四。挾馬,騎御馬直長行六十四人。騎御馬直人員三,天武節級三人,押馬六人,象二十三人。擎執人、舁士共八千七百七十一人。鼓吹樂工九百九十四人。馬六千七十八匹。



 天德五年,海陵遷都於燕,用黃麾仗一萬八百二十三
 人攝官在內,騎三千九百六十九,分八節。



 第一節。中道,象二十三人。節級二人,銅鑼,七寶,俞石、銀鉤各一,鐵鉤二,小旗十五,並服花腳襆頭、青錦絡縫緋衣癸衫,金鍍銀雙鹿束帶第一引,七十二人:清道一,武弁、緋雲鶴袍、褲、革帶,執黑漆杖。幰弩一,赤平巾幘、緋辟邪衫、革帶、赤褲。誕馬二,控四人,赤平巾幘、緋繡寶相花衫、銀革帶,纓轡涼屜二副。軺車一,赤馬二,駕士十八人,武弁、緋繡雉大袖衫、白褲。馬,纓轡涼屜、銅面、包尾。縣令一員,朝服,坐軺車。僚佐四員,並朝服。控馬八人,錦帽、絡縫紫衫、大佩、銀帶。紫方傘一,黃抹額、寶相花衫、銀帶、大口褲。朱團扇一,曲蓋一,緋抹額、寶相花衫、革帶、褲。青衣二,青平巾、青衫、褲、革帶,執青竹杖。車輻棒二,赤平巾、緋白澤衫、革帶、赤褲。告止幡二,執者六人,緋抹額、寶相花衫、革帶,褲。傳教幡一,信幡一,各三人,並黃抹額、寶相花衫、革帶、大口褲。小戟
 十六。服同上。



 第二引,二百六十四人:清道二,幰弩一,誕馬四,控八人,服並如前。㧏鼓一,金鉦一,平巾幘、緋鸞衫、抹帶、褲、錦螣蛇。大鼓六,黃雷花衫、褲、抹額、抹帶。節一,幢一,麾一,夾槊二,角四,儀刀十,並平巾幘、緋繡寶相花衫,銀革帶、大口褲。革車一,赤馬四,駕士二十五人,武弁、獬豸大袖、勒帛、馬飾如前。府牧一員,朝服坐車。僚佐四員,控馬八人,服並如前。鐃鼓一,簫二,笳二,笛一,篳篥一,並平幘、緋寶相花衫、銀褐抹帶。大橫吹一,緋苣紋袍、褲、抹額、抹帶。青衣四,車輻棒四,紫方傘一,朱團扇四,曲蓋一,告止幡二,六人,傳教幡二,六人,信幡二,六人,小戟四十,服並如前。刀盾三十六,銀褐抹額、寶相花衫、銀革帶、褲。弓矢三十六,錦帽、青寶相花衫、銀革帶、褲。槊三十六。錦帽、紫寶相花袍、革帶、褲。



 朱雀旗隊三十四人:
 折衝都尉三人,平巾幘、紫辟邪衫、革帶、大口鍠、錦螣蛇、橫刀弓矢。皞槊二,平巾幘、緋繡寶相花衫。革帶、褲。朱雀旗一,五人,緋抹額、寶相花衫、革帶、大口褲、橫刀,引夾人加弓矢。弩六,弓矢六,槊十二。並平巾、緋寶相花衫、橫刀、革帶、褲。



 龍旗隊七十一人:大將軍一人,朝服。引旗四人,黃抹額、寶相花衫、革帶、大口褲。旗十二,風伯旗一、雨師旗一、雷公旗一、電母旗一、北斗旗一、五星旗五、左右攝提旗二,執、夾共六十人,皆五色寶相花衫、抹額、革帶、褲、橫刀,引夾人加弓矢,後凡執旗者並同。副竿二,錦帽、黃寶相花衫、革帶、褲。護旗四人。加黃抹額、弓矢。



 太僕三車八十一人:指南車,駕士三十人,武弁、緋霡繡孔雀大袖、銀褐帶、褲。記里鼓車,駕士三十人,獬豸大袖。鸞旗車,駕士十八人,瑞鷹大袖。駕車赤馬十二,執黑杖者三人。



 外仗。牙門旗隊二十
 八人:分左右。白澤旗二,執夾各五人,綠具裝冠、人馬甲、錦臂韝、橫刀,引夾加弓矢。金吾牙門旗第一門,牙門旗四,執夾十二人,青寶相花衫、抹額、革帶、大口褲、橫刀,引夾人加弓矢。監門校尉六人。長腳襆頭、緋抹額、獅子裲襠、銀帶、橫刀、弓矢、烏皮靴,後隊同。



 前部馬隊,第一隊七十人:折衝、果毅都尉二人,錦帽、緋辟邪袍、褲、革帶、橫刀、弓矢。角宿、亢宿、斗宿、牛宿旗四,旗各五人,並五色寶相花衫、抹額、革帶、橫刀,引夾加弓矢。弩六,弓矢十四,並錦帽、青寶相花衫、革帶、褲。槊二十八。緋色衫,餘同上。



 第二隊七十人:折衝、果毅都尉二人,白澤衫。氐宿、女宿、房宿、虛宿旗四,旗五人,弩六,弓矢十四,。槊二十八。服、執如前。



 第三隊七十人:折衝、果毅都尉二人,心宿、危宿、尾宿、室宿旗四,旗五人,弩六,弓矢十四,槊二十八。
 服、執如前。



 第二節。中道。金吾引駕騎二十人:折衝都尉二人,平巾幘、緋闢邪衫、革帶、褲、橫刀、弓矢。弩六,弓矢六。槊六。並平巾幘、緋寶相花裲襠、革帶、褲。前部鼓吹五百四十七人:鼓吹令二人,長腳襆頭、綠公服、角帶、絲鞭、烏皮靴。府吏四人,長腳襆頭,綠寬衫、角帶、黃絹半臂、烏靴。部轄指揮使一人,平巾幘、紫寶相花衫、革帶、錦螣蛇。主帥四十八人,分五項,平巾幘、緋鸞衫、革帶、褲、執儀刀。㧏鼓、金鉦各十二,平巾幘、緋鸞衫、銀褐抹帶、錦螣蛇。大鼓、長鳴各百二十,黃雷花衫、抹額、抹帶。鐃鼓十二,緋苣紋衫、抹額、抹帶。歌二十四,拱辰管二十四,簫二十四,笳二十四,服如鉦鼓,無螣蛇。大橫吹百二十。服如饒鼓。外仗。馬部第四隊六十人:分左右。折衝都尉二人,緋麟衫。箕宿、壁宿旗各一,
 旗五人,弩六,弓矢十四,槊二十人。服、執並如前隊。第五隊六十人:折衝都尉二人,奎宿、井宿旗各一,旗五人,弩六,弓矢十四,槊二十八。服、執並如前隊。第六隊六十人:折衝都尉二人,緋瑞鷹袍。婁宿、鬼宿旗各一,旗五人,弩六,弓矢十四,槊二十八。服、執並如前隊。第七隊六十人:折衝都尉二人,胃宿、柳宿旗各一,旗五人,弩六,弓矢十四,槊二十八。服、執並如前隊。第八隊六十人:折衝都尉二人,昴宿、星宿旗各一,旗五人,弩六,弓矢十四,槊二十八。服、執並如前隊。第九隊六十人:折衝都尉二人,赤豹袍。畢宿、張宿旗各一,旗五人,弩六,弓矢十四,槊二十八。服、執同前。第十隊七十人:折衝都尉二人,瑞馬袍。觜宿、
 翼宿、參宿、軫宿旗各一,旗五人,弩六,弓矢十四,槊二十八。服、執如前。步甲隊,第一、第二兩隊百一十人:領軍衛將軍二人,平巾幘、紫白澤袍、褲、帶、錦螣蛇、橫刀、弓矢。犦槊四,平巾幘、緋寶相花袍、大口褲。折衝都尉四人,服如將軍。鶡雞旗二,貔旗二,旗各五人,朱牟甲弓矢四十,朱牟甲刀盾四十。兜牟、甲身、披膊、錦臂韝,行縢、鞋襪、勒甲、革帶。



 第三節。中道,前部鼓吹第二,五百二十三人:侍御在外。節鼓二,笛二十四,簫二十四,篳篥二十四,笳二十四,桃皮篳篥二十四,黑平巾幘、緋對鸞衫、銀褐勒帛、大口褲。主帥二十六人,分四項,革帶、執儀刀、服如上,無勒帛。㧏鼓、金鉦各十二,黑平巾幘、緋繡對鸞衫、銀褐勒帛、大口褲、錦螣蛇。小鼓百二十,中鳴百二十,黃雷花袍、褲、抹額、抹帶。羽葆鼓十二,
 青苣紋袍、抹額、抹帶。歌二十四,拱辰管二十四,簫二十四,笳二十四,服如前色。侍御史二員,朝服。黃麾幡一,三人。武弁、緋寶相花衫、銀褐勒帛、大口褲,執者馬、褲者步。外仗。步甲,第三隊五十二人:折衝、果毅都尉二人,紫瑞馬袍。玉馬旗二,旗五人,青牟甲弓矢四十。服、執並同前隊。第四隊五十二人:折衝、果毅都尉二人,瑞鷹袍。三角獸旗二,旗五人,青牟甲刀盾四十。第五隊五十二人:折衝、果毅都尉二人,白澤袍。黃鹿旗二,旗五人,黑牟甲弓矢四十。第六隊五十二人:折衝、果毅都尉二人,服同。飛麟旗二,旗五人,黑牟甲刀盾四十。第七隊五十二人:折衝、果毅都尉二人,赤豹袍。駃騠旗二,旗五人,銀褐牟甲弓矢四十。第八隊五
 十二人:折衝、果毅都尉二人,服同。鸞旗二,旗五人,銀褐牟甲刀盾四十。第九隊五十二人:折衝、果毅都尉二人,瑞鷹袍。麟旗二,旗五人,黃牟甲弓矢四十。第十隊五十二人:折沖、果毅都尉二人,馴象旗二,旗五人,黃牟甲刀盾四十。服、執如前。金吾牙門旗第二門,牙門旗四,執夾十二人,監門校尉六人。服、執同第一門。左右屯衛將軍二人,平巾幘、紫飛麟袍、大口褲、錦螣蛇、革帶、橫刀、弓矢。絳引幡二十,執者六十人,武弁、緋繡寶相花衫、銀褐勒帛、大口褲。共八十人。



 第四節。中道,六軍儀仗二百五十二人:統軍六人,花腳襆頭、紫繡抹額、孔雀袍、革帶、橫刀、胡魯、器仗、珂馬。都頭六人,長腳襆頭、紫寶相花大袖、革帶、橫刀。神武軍旗二、羽林軍旗二、龍武軍旗二,旗各五人,執人錦
 帽,引夾人貼金帽。排襴旗四十八、吏兵旗四、力士旗四、赤豹旗四、黃熊旗四、龍君旗四、虎君旗四、掩尾天馬旗六,旗一人,錦帽、五色寶相花衫、革帶、錦臂韝。白簳槍九十,交腳襆頭、五色寶相花衫、抹額、革帶、汗褲。柯舒二十四,鐙杖十八。並貼金帽、五色寶相花衫、革帶。引駕龍墀旗隊六十五人:排仗通直二人,排仗大將二人,並長腳襆頭、紫公服、紅鞓帶、絲鞭、烏皮靴。天王旗四、十二辰旗各一、旗一人,並錦帽、五色寶相花衫、革帶、臂韝。天下太平旗一、五方龍旗五,旗五人,執人錦帽,引夾人貼金帽,服並如上,橫刀、弓矢。君王萬歲旗一、日月旗各一、旗五人。執人錦帽,引夾人貼金帽,服、執已見前例。御馬六十六人:馬十六匹,匹四人,控馬三十二人,貼金帽、紫寶相花衫、革帶。夾馬三十二人。皁帽、青錦襖、塗金銅束帶。廣武節級一人,錦帽,執黑杖,服同控馬。
 管押騎御馬直人員一人。皁帽、紅錦襖、塗金、銅束帶。中道隊三十二人:大將軍一人,朝服、絲鞭。日月合璧旗一、苣紋旗二、五星連珠旗一、祥雲旗二,旗各五人。服、執見前例。長壽幢一。平巾幘、緋寶相花衫、革帶、大口褲。金吾細杖一百人:青龍旗一、白虎旗一、五嶽神旗五、五方神旗五,旗各四人,並四色寶相花衫、青黃銀褐皁抹額、抹帶、橫刀,引夾如前。押旗二人,長腳襆頭、紫公服、紅鞓角帶、烏皮靴。五方龍旗各三、五方鳳旗各三,旗一人,並五色衫、抹額、革帶、橫刀。四瀆旗四,旗五人。並皁寶相花衫、抹額、革帶、橫刀、引夾如前。外仗。黃麾前第一部二百七十二人:殿中侍御史二人,朝服。左右屯衛大將軍二人,折衝都尉二人,平巾幘、紫飛麟袍、革帶、大口褲、錦螣蛇、橫刀、弓矢。主帥二十人,平巾幘、緋寶相花衫、革帶、褲、儀刀。
 龍頭竿一百,揭鼓六,儀鍠斧二十,小戟二十,弓矢四十,朱縢絡刀盾二十,槊二十,綠縢駱刀盾二十。並青寶相花衫、抹額、抹帶、行螣、鞋襪。第二部二百七十二人:殿中侍御史二人,左右領軍衛大將軍二人,折衝都尉二人,紫繡白澤袍。主帥二十人,龍頭竿一百,揭鼓六,儀鍠斧二十,小戟二十,弓矢四十,朱縢絡刀盾二十,槊二十,綠縢絡刀盾二十。服並緋。第三部二百七十二人:殿中侍御史二人,左右屯衛大將軍二人,折衝都尉二人,紫瑞鷹袍。主帥二十人,龍頭竿一百,揭鼓六,儀鍠斧二十,小戟二十,弓矢四十,朱縢絡刀盾二十,槊二十,綠縢絡刀盾二十。服並黃,餘同上部。



 第五節。中道,八寶香案共三百人:輿士九十六人,平巾幘、緋寶相花衫、大口褲、塗金銀束帶。燭籠三十二,大佩銀腰帶,服同輿士。行馬十六,服同燭籠。碧襴官十六人,弓腳襆頭、碧襴衫、塗金銅束帶、烏皮靴,後四人執長刀。符寶郎八人,長腳襆頭、綠公服、角帶、槐簡、步導。援寶三十二人,人員二人,武弁、紫寶相花衫、革帶、執黑漆杖。長行三十人,緋寶相花衫、執黑漆杖。香案八,輿士三十二人,服同燭籠、行馬。案後金吾仗六,方傘二,大雉扇四,服並同碧襴官。金吾仗十二人,四色官四人,長腳襆頭、綠公服、大口褲、金銅腰帶,前二人執槐簡,後二執金銅儀刀。押仗二人,長腳襆頭、紫公服、紅鞓帶、烏皮靴。金甲二人,披膊,兜牟、鉞斧、錦臂韝、勒甲絳。進馬四人。平巾幘,紫犀牛裲襠、革帶、褲、刀、矢弓。金吾引駕四十九人:千牛將軍一人,千牛十人,郎將二人,並緋繡抹額、紫犀牛裲襠、革帶、大口褲、橫
 刀、弓矢、珂馬,將軍平巾幘、無抹額,千牛郎將花腳襆頭,餘同。長史二人、長腳襆頭、綠公服、金銅腰帶、褲、烏皮靴。引駕官四人。長腳襆頭、紫公服、紅鞓帶、烏皮靴。中雉扇十二,大傘二,小雉扇四,華蓋二,香蹬一座,八人,火燎一,二人。武弁、緋寶相花大袖、革帶、大口褲。腰輿人員、什將三人,皁帽、紅錦襖、塗金銀束帶。人員執杖。長行十六人,拳腳襆頭、紅錦四衣癸襖、塗金銀腰帶。排列官二人,長腳襆頭,紫公服、紅鞓帶、烏皮靴。小輿二十四人,白鞓銀束帶,服同長行人。逍遙輦人員、什將共十六人,皁帽、塗金銀束帶、紅錦方勝練鵲。人員執黑漆杖。長行二十六人,紅地白獅錦襖、塗金銀帶、冠同。平輦人員、什將十六人,皁帽、紅錦團襖、塗金銀帶。輿輦共一百三人。諸班開道旗隊一百七十七人:開道旗一,鐵甲、兜牟、紅背子、劍、緋馬甲。皂纛旗十二,旗一人,黑漆鐵笠、皁皮人馬甲。引
 駕六十二人,皁帽、紅錦團襖、紅背子、鐵人馬甲、箭、兵械、骨朵。輔龍直一百二人。皁帽、紅背子、骨朵、鐵人馬甲。外仗。黃麾前第四部二百七十二人:殿中侍御史二人,左右武衛大將軍二人,折衝都尉二人,主帥二十人,龍頭竿一百,揭鼓六,儀鍠斧二十,小戟二十,弓矢四十,朱縢絡刀盾二十,槊二十,綠縢絡刀盾二十。黃寶相花衫,餘並如前第一部。第五部二百七十二人。除左右驍衛大將軍與都尉服赤豹袍,龍竿以下服銀褐花衫,餘名色並如前第二部。第六部二百七十二人。除將軍、都尉服瑞馬袍,龍竿以下服皁花衫,餘名色並如前第三部。



 第六節。中道,門旗隊一百二十三人:騎執門旗四十,五方色龍旗十,步執紅龍門旗六十,麋旗一,簇輦紅龍旗
 八,日月旗二,麟旗一,鳳旗一,旗皆一人。並鐵甲、兜牟、紅錦襖、紅背子,馬執者惟紅背子,步執門旗仍帶劍。金輅,皇太后乘之,公主侍坐,故在玉輅之前。駕士九十四人,赤平巾幘、緋繡對鳳大袖、緋抹額、赤褲、鞋襪。擊鞭內侍十人,皁帽、紅錦襖、塗金銀束帶。駕頭下,御床也。抱駕頭內侍一人,長腳襆頭、紫羅公服、塗金銀束帶。控馬二人,錦帽、錦絡縫寬衫、銀大佩腰帶。廣武官十二人,錦帽、白鞓銀束帶、襖。茶酒班執從物十一人,水罐二、香球二、唾盂一、廝羅一、手巾一、御椅三人、踏床一,皁帽、碧錦團襖、紅錦背子、塗金銀束帶。共百三十人。拱聖直,人員二人。長行三十八人。真珠頭巾、紅錦四衣癸襖、塗金銀束帶。導駕官四十二人,朝服。從人八十四,錦帽、紫絡縫寬衫、大佩銀腰帶。仗劍六人,皁帽、紅錦團襖、紅錦背子、鐵甲、弓矢、器械。廣武把行門八人,殿班把行
 門三十五人。服並如仗劍。玉輅,帝后同乘,太子陪坐。駕士百二十八人,服如金輅,惟用青色。千牛將軍一人,具裝,執長刀於輅右。左右點檢二人,披金甲。夾輅大將軍二人,陪輅將軍二人,並朝服。進輅職掌二人,長腳襆頭、紫寬衫、塗金銀腰帶。教馬官二人,長腳襆頭、緋抹額、紫寶相花衫、塗金銀腰帶。部押二人,皁帽、鐵甲、紅錦襖、執骨朵。挾輅八人,控踏路馬四人,馬二匹,銅面、包尾、涼屜,人服如駕士。共一百五十三人。龍翔馬隊二十隊,六百二十人,分左右,每隊人員三人,皁帽、鐵甲、紅錦襖、紅背子、弓矢、劍、骨朵、甲馬。殿侍二十八人。鐵甲、紅錦背子、弓矢、器械、甲馬。東第五班,金槍六隊,每隊旗三人、槍二十五人,內二十人佩弓矢。共一百六十八人。並裹鐵兜牟、金鎗。銀槍六隊,每隊旗三人、鎗二十五人,內
 二十人佩弓矢。共一百六十八人,並裹鐵笠,銀槍。東第四班,二隊,每隊旗三人、弩二十五人,共五十六人。鐵笠、兜牟。神勇步隊七百人:分左右作四重,每重人員十,皁帽、紅錦團襖、弓矢、器械、骨朵。長行六百六十人,並鐵兜牟、甲。內拱聖骨朵直一百六十四人,拱聖槍直一百六十四人,內執子旗者二人,余執槍。拱聖弓箭直一百六十六人,弓矢、器械、執骨朵。拱聖弩直一百六十六人。挾弩、胡魯。廣武骨朵大劍三百一十人:指揮使五人,紅錦襖、紅背子。都頭五人,紅襖、紅背子、并皁帽、塗金腰帶、骨朵。長行三百人。內一百人簇四金鵬錦帽、紫孔雀寬襖、白鞓銀束帶、骨朵,二百人金鍍銀花朱紅笠、緋對鳳寬襖、銀帶、執銀花大劍。導駕官四十二員,從者八十四人。服已見前。外仗。青龍白虎隊五十二人:果毅都尉二人,青
 龍旗一、白虎旗一,旗五人,弩六人,弓矢十四,槊二十。服已見前。



 第七節。中道,駕后輔龍直樂三十一人:拍板一,篳篥十五,笛十四,人員一人。長行三十人,樂器自備,並皁帽、紅錦襖、塗金束帶,並馬。人員執骨朵。扇筤二十五人:執筤官一人,控馬二人,服並如前例。紅龍扇二,長腳襆頭、紫公服、塗金銀束帶。廣武二十人:錦帽、繡寬襖、白鞓銀束帶、紫對鳳十領、緋對鳳十領。七寶輦輿士四十二人:什將、人員十六人,皁帽、紅錦團襖。長行二十六人。盤裹襆頭、紅錦四衣癸襖、塗金束帶。持鈒隊五十人:旅帥二人,服如都尉。重輪旗二,旗五人,服同前例。紅羅大傘二,大雉扇八,小雉扇八,紅羅繡華蓋一,武弁、緋寶相花衫、革帶、褲、錦螣蛇。朱團扇八,
 黃寶相花衫。真武幢一,皁寶相花衫。睥睨八,緋寶相花大袖。麾一,幢一。紫寶相花衫、銀褐抹帶。後部鼓吹三百三十七人:鼓吹丞二人,典士四人,部轄指揮使一人,主帥十八人,金鉦、㧏鼓各三,羽葆鼓十二,歌二十四,拱辰管二十四,簫二十四,笳二十四,節鼓二,鐃鼓十二,歌十六,簫二十四,笳二十四,小橫吹一百二十。青苣紋袍、抹額、抹帶,餘並與前同。金吾牙門旗第三門,牙門旗四,旗三人,監門校尉六人。服、執同第一門。黃麾後第一部二百七十二人,第二部二百七十二人,第三部二百七十二人,殿中侍御、衛大將軍、折衝都尉、龍頭竿以下名色,並如前三部。



 第八節。中道,後部鼓吹第二,百二十人:笛二十四,簫二十四,篳篥二十四,笳二十四,桃皮篳篥二十四。服並如前。屬車八,牛二十四,駕士八十人。武牟、緋繡雲鶴大袖、銀褐抹帶、大口褲。黃鉞車,赤馬二,駕士十五人。武牟、緋對鵝大袖、銀褐抹帶、大口褥。豹尾車,赤馬二,駕士十五人。武弁、緋立豹大袖、銀褐抹帶、大口褲。玄武隊六十一人:金吾折衝都尉一人,平巾幘、紫辟邪袍、革帶、褲、螣蛇、橫刀、弓矢。皞槊二,平巾幘、緋寶相花衫、革大帶。仙童旗一、玄武旗一、螣蛇旗一、神龜旗一、旗五人,服、執如前例。槊十九,弓矢十五,弩四。平巾幘、緋寶相花衫、革帶、褲。黃麾後第四部二百七十二人,第五部二百七十二人,第六部二百七十二人,攝官名數服色並如前第四、第五、第
 六部。絳引幡二十,執者六十人。並武弁、緋繡寶相花衫、銀褐抹帶、大口褲。諸從駕官並於仗後陪從,朝服不足者公服。凡應乘馬者,並同宋制。



\end{pinyinscope}