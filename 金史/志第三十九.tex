\article{志第三十九}

\begin{pinyinscope}

 百官四



 ○符制



 初,穆宗之前,諸部長各刻信牌,交互馳驛,訊事擾人。太祖獻議,自非穆宗之命,擅製牌號者置重法。自是,號令始一。收國二年九月,始製金牌,後又有銀牌、木牌之制。蓋金牌以授萬戶,銀牌以授猛安,木牌則謀克、蒲輦所佩者也。故國初與空名宣頭付軍帥,以為功賞。



 遞牌,即國初之信牌也。至皇統五年三月,復更造金銀牌,
 其制皆不傳。大定二十九年,製綠油紅字者,尚書省文字省遞用之。朱漆金字者,敕遞用之。並左右司掌之,有合遞文字,則牌送各部,付馬鋪轉遞,日行二百五十里。如臺部別奉聖旨文字,亦給如上制。



 虎符之制,承安元年製。以禮官言,漢與郡國守相為銅虎符,唐以銅魚符,起軍旅、易守長等用之。至是,斟酌漢、唐典故,其符用虎,並五左一右,左者留御前,以侍臣親密者掌之,其右付隨路統軍司、招討司長官主之,闕則次官主之。若發兵三百人以上及徵兵、召易本司長貳官,從尚書省奏請左第一符,近侍局以囊封付主奏者,尚書備錄聖旨,與
 符以函同封,用尚書省印記之,皆專使帶牌馳送至彼。主符者視其封,以右符勘合,然後奉行,若一有參差者,不敢承用。主者復用囊封貯左符,上用職印,具發兵狀與符以本司印封,即日還付使者,送尚書省以進,乃更其封,以付內掌之人。若復有事,左符以次出,周而復始,仍各置曆注付受日月。若盜賊急速不容先陳者,雖三百人以上,其掌兵官司亦許給付,隨即言上,詔即施行之。貞祐三年,更定樞密院用鹿符,宣撫司用魚符,統軍司用虎符。若發銀牌,若省付部及點檢司者,左右司用匣封印,驗封交受。若發於他處,並封題押,以匣貯之。



 ○
 印制



 太子之寶。大定二十二年,世宗幸上京。鑄「守國之寶」以授皇太子。二十八年,世宗不豫,以皇太孫攝政,鑄「攝政之寶」。貞祐三年十二月,以皇太子守緒控制樞密院,詔以金鑄「撫軍之寶」,如世宗時制,於啟稟之際用之。



 百官之印。天會六年,始詔給諸司,其前所帶印記無問有無新給,悉上送官,敢匿者國有常憲。至正隆元年,以內外官印新舊名及階品大小不一,有用遼、宋舊印及契丹字者,遂定制,命禮部更鑄焉。三師、三公、親王、尚書令並金印,方二寸,重八十兩,駝紐。一字王印,方一寸七分半,金鍍銀,重四十兩,鍍金三字。諸郡王印,方一寸六
 分半,金鍍銀,重三十五兩,鍍金三字。國公無印。一品印,方一寸六分半,金鍍銀,重三十五兩,鍍金三字。二品印,方一寸六分,金鍍銅,重二十六兩。東宮三師、宰執與郡王同。三品印,方一寸五分半,銅,重二十四兩。四品印,方一寸五分,銅,重二十兩。五品印,方一寸四分,銅,重二十兩。六品印,一寸三分,銅,重十六兩。七品印,一寸二分,銅,重十六兩。八品印,一寸一分半,銅,重十四兩。九品印,一寸一分,銅,重十四兩。凡朱記,方一寸,銅,重十四兩。



 天德二年行尚書省以其印小,遂命擬尚書省印小一等改鑄。大定二十四年二月,鑄行尚書省、御史臺、并左右三
 部印,以從幸上京。泰和元年八月,安國軍節度使高有鄰言:「本州所掌印三:曰『安國軍節度使之印』;曰『邢州觀察使印』,吏、戶、禮案用之;曰『邢州之印』,兵、刑、工案用之。以名實不正,乞改鑄。」宰臣奏謂:「節度使專行之事自當用節度使印,觀察使亦如之,其六曹提點所軍兵民訟,則當用本州印,著為定制。」上從之。泰和八年閏四月,敕殿前都點檢司,依總管府例鑄印,以「金」、「木」、「水」、「火」、「土」五字為號,如本司差人則給之。



 ○鐵券



 以鐵為之,狀如卷瓦。刻字畫襴,以金填之。外以御寶為合,半留內府,以賞殊功也。



 ○
 官誥



 親王,紅遍地雲氣翔鸞錦褾,金鸞五色羅十五幅,寶裝犀軸。一品,紅遍地雲鶴錦褾,金雲鶴五色羅十四幅,犀軸。二品、三品,紅遍地龜蓮錦褾,素五色綾十二幅,玳瑁軸。四品、五品,紅遍地水藻戲鱗錦褾,大白綾十幅,銀裏間鍍軸,元牙軸承安四年改之,大安二年復改為金縷角軸。六品、七品,紅遍地草錦褾,小白綾八幅,角軸,大安加銀縷。公主、王妃與親王同。郡主、縣主、夫人,紅遍地瑞蓮鸂水鶒錦褾,金蓮鸂水鶒五色羅十五幅。郡王夫人、國夫人,紅遍地芙蓉花錦褾,金花五色綾十二幅,玳瑁軸。縣君、孺人、鄉君,紅遍地雜花錦褾,素五色小綾十幅,
 銀裏間鍍軸。軸之制,如徑二寸餘大錢貫樞之,兩端復以犀象為鈿以轄之,可圓轉如輪。金格,一品,紅羅畫雲氣盤龍錦褾,金龍五色羅十七幅,寶裝玉軸。二品,翔鳳褾,金鳳羅十六幅,犀軸。三品、四品,盤鳳褾,金鳳羅十五幅。五品,翔鸞錦褾,金鸞羅十四幅。以上幅皆用五色羅,軸皆用犀。六品,御仙花錦褾,金花五色綾十二幅。七品、八品、九品,太平花錦褾,金花五色小綾十幅。軸皆用玳瑁。凡褾皆紅,幅皆五色。夫人以上制授,餘敕授,皆給本色錦囊。



 ○百官俸給



 正一品:三師,錢粟三百貫石,曲米麥各五十稱石,春衣羅五十匹,秋衣綾五十匹,春秋絹各二百匹,綿千兩。三公,錢粟二百五十貫石,曲米麥各四十稱石,春衣羅四十匹,秋衣綾四十匹,春秋絹各一百五十匹,綿七百兩。親王、尚書令,錢粟二百二十貫石,曲米麥各
 三十五稱石,春衣羅三十五匹,秋衣綾三十五匹,春秋絹各一百二十匹,綿六百兩。皇統二年,定制,皇兄弟及子封一字王者為親王,給二品俸,餘宗室封一字王者以三品俸給之。天德二年,以三師、宰臣以下有以一官而兼數職者,及有親王食其祿而復領他事者,前此並給以俸,今宜從一高,其兼職之俸並不重給。至大定二十六年,詔有一官而兼數職,其兼職得罪亦不能免,而無廩給可乎。遂以職務煩簡定為分數,給兼職之俸。從一品:左右丞相、都元帥、樞密使、郡王、開府儀同,錢粟二百貫石,曲米麥各三十稱石,春秋衣羅綾各三十匹,絹
 各一百匹,綿五百兩。平章政事,錢粟一百九十貫石,曲米麥各二十八稱石,春羅秋綾各二十五匹,絹各九十五匹,綿四百五十兩。大宗正,錢粟一百八十貫石,曲米麥各二十五稱石,羅綾同上,絹各九十匹,綿四百兩。



 正二品:東宮三師、副元帥、左右丞,錢粟一百五十貫石。曲米麥各二十二稱石,春羅秋綾各二十二匹,絹各八十匹,綿三百五十兩。從二品:錢粟一百四十貫石,曲米麥各二十稱石,春羅秋綾各二十匹,絹各七十五匹,綿三百兩。同判大宗正,錢粟一百二十貫石,曲米麥各十八稱石,春羅秋綾各十八匹,絹各七十匹,綿二百五十兩。



 正三品:錢粟七十貫石,曲米麥各十六稱石,春羅秋綾各十二匹,絹各五十五匹,綿二百兩。外官,錢粟一百貫石,曲米麥各十五稱石,絹各四十匹,綿二百兩,公田三十頃。統軍使、招討使、副使,錢粟八十貫石,曲米麥十三稱石,絹各三十五匹,綿百六十兩,公田二十五頃。都運、府尹,錢粟七十貫石,曲米麥十二稱石,絹各三十匹,綿百四十兩。天德二年,省奏:「職官公田歲入有數,前此百姓各隨公宇就輸,而吏或貪冒,多取以傷民。宜送之官倉,均定其數,與月俸隨給。」從三品:錢粟六十貫石,曲米麥各十四稱石,春秋衣羅綾各十匹,絹各五十匹,綿百
 八十兩。外官,錢粟六十貫石,曲米麥各十稱石,絹各二十五匹,綿一百二十兩,公田二十一頃。皇統元年二月,詔諸官、職俱至三品而致仕者,俸祿、傔人,各給其半。



 正四品:錢粟四十五貫石,曲米麥各十二稱石,春秋衣羅綾各八匹,絹各四十匹,綿一百五十兩。外官,錢粟四十五貫石。副統軍,錢粟五十貫石,絹各二十二匹,綿八十兩,職田十七頃。餘同下:曲米麥各八稱石,絹各二十匹,綿七十兩,公田十五頃,許帶酒三十瓶、鹽三石。從四品:錢粟四十貫石,曲麥米各十稱石,春秋羅綾各六匹,絹各三十匹,綿一百三十兩。外官,錢粟四十貫石,曲米麥
 各七稱石,絹各十八匹,綿六十兩,公田十四頃。猛安,錢粟四十八貫石,餘皆無。烏魯古使,同,無職田。大定二十年,詔猛安謀克俸給,令運司折支銀絹。省臣議:「若估粟折支,各路運司儲積多寡不均,宜令依舊支請牛頭稅粟。如遇兇年盡貸與民,其俸則於錢多路府支放,錢少則支銀絹亦未晚也。」從之。



 正五品:錢粟三十五貫石,曲米麥各八稱石,春秋衣羅綾各五匹,絹各二十五匹,綿一百兩。外官,刺史、知軍、鹽使、錢粟三十五貫石,曲米麥各六稱石,絹各十七匹,綿五十五兩,公田十三頃。餘官,錢粟三十貫石,曲米麥同上,絹各十六匹,綿五十兩,
 職田十頃。從五品:錢粟三十貫石,曲米麥六稱石,春秋羅綾各五匹,絹各二十匹,綿八十兩。外官,錢粟二十五貫石,曲米麥各四稱石,絹各十匹,綿四十兩,公田七頃。謀克,錢粟二十貫石,餘皆無。喬家部族都鈐轄,無職田。



 正六品:錢粟二十五貫石,麥五石,絹各十七匹,綿七十兩。外官與從六品,皆錢粟二十貫石,曲米麥三稱石,絹各八匹,綿三十兩,公田六頃。從六品:錢粟二十二貫石,麥五石,春秋絹各十五匹,綿六十兩。烏魯古副使,同,無職田。



 正七品:錢粟二十二貫石,麥四石,衣絹各一十二匹,綿五十五兩。外官,諸同知州軍、都轉運判、諸府推官、
 諸節度判、諸觀察判、諸京縣令、諸劇縣令、提舉南京京城、規措渠河官、諸都巡檢、諸酒曲鹽稅副、諸正將、錢粟一十八貫石,曲米麥各二稱石,春秋衣絹各七匹,綿二十五兩。諸司屬令、諸府軍都指揮,俸同上,無職田。潼關使,錢粟一十八貫石,曲米麥各一稱石,衣絹各六匹,綿三十兩,無職田。從七品:錢粟一十七貫石,麥四石,衣絹各一十匹,綿五十兩。外官、統軍司知事,錢粟一十七貫石,麥四石,衣絹各一十匹,綿五十兩。諸鎮軍都指揮使,錢粟一十八貫石,曲米麥各二稱石,衣絹各七匹,綿二十五兩。諸招討司勘事官、諸縣令、諸警巡副、京兆府竹監
 管勾、五品鹽使司判、諸部禿里、同提舉上京皇城司、同提舉南京京城所、黃河都巡河官、諸酒稅榷場使,錢粟一十七貫石,曲米麥各二稱石,衣絹各七匹,綿二十五兩,職田五頃。會安關使,諸知鎮城堡寨,錢粟一十五貫石,曲米麥各一稱石,衣絹各六匹,綿二十兩,職田四頃。



 正八品:朝官,錢粟一十五貫石,麥三石,衣絹各八匹,綿四十五兩。外官,市令、諸錄事、諸防禦判、赤縣丞、諸劇縣丞、崇福埽都巡河官、諸酒稅使、醋使、榷場副、諸都巡檢,錢粟一十五貫石,曲米麥各一稱石,衣絹各六匹,綿二十兩,職田四頃。烏魯古判官,俸同上,無職田。按察司知
 事、大興府知事、招討司知事、諸副都巡檢使,錢粟一十三貫石,曲米麥各一稱石,衣絹各六匹,綿二十兩,職田二頃。諸司屬丞,俸同上,無職田。諸節鎮以上司獄、諸副將,錢粟一十三貫石,衣絹各三匹,綿一十兩,職田二頃。南京京城所管勾、京府諸司使管勾、河橋諸關渡譏察官、同樂園管勾、南京皇城使、通州倉使,錢粟一十二貫石,衣絹各三匹,綿一十兩。節鎮諸司使、中運司柴炭場使,錢粟一十貫石,衣絹各二匹,綿八兩。從八品:朝官,錢粟一十三貫石,麥三石,衣絹各七匹,綿四十兩。外官,南京交鈔庫使、諸統軍按察司知法,錢粟一十三貫石,麥
 三石,衣絹各七匹,綿四十兩。諸州軍判官、諸京縣丞、諸次劇縣丞、諸三品鹽司判官、漕運司管勾,永豐廣備庫副使、左右別貯院木場使,錢粟一十三貫石,曲米麥各一稱石,衣絹各六匹,綿二十兩,職田三頃。諸麼忽、諸移里堇,錢粟一十三貫石,麥二石,衣絹各五匹,綿一十五兩,職田三頃。



 正九品:朝官,錢粟一十二貫石,麥二石,衣絹各六匹,綿三十五兩。外官,南京交鈔庫副,錢粟一十二貫石,麥二石,衣絹六匹,綿三十五兩。諸警巡判官,錢粟一十三貫石,曲米麥各一稱石,衣絹六匹,綿一十兩,職田三頃。諸縣丞、諸酒稅副使,錢粟一十二貫石,麥一
 石五斗,衣絹各五匹,綿一十七兩,職田三頃。市丞、諸司候、諸主簿、諸錄判、諸縣尉、散巡河官、黃河埽物料場官,錢粟一十二貫石,麥一石,衣絹各三匹,綿一十兩,職田二頃。管勾泗州排岸兼巡檢、副都巡檢、諸巡檢,俸例同上,並無麥及職田。諸鹽場管勾、左右別貯院木場副、永豐廣備庫判,錢粟一十二貫石,衣絹各三匹,綿一十兩,職田二頃。諸部將、隊將,錢粟一十二貫石,麥一石,衣絹各三匹,綿一十兩,職田二頃。店宅務管勾,錢粟一十二貫石,綿絹同上。京府諸司副、南京皇城副、通州倉副、同管勾河橋、諸副譏察,錢粟一十一貫石,衣絹各二匹,綿
 八兩。諸州軍司獄,錢粟一十一貫石,衣絹各二匹,綿八兩,職田二頃。節鎮諸司副、中運司柴炭場副,錢粟一十貫石,衣絹各二匹,綿八兩。從九品:朝官,錢粟一十貫石,麥二石,衣絹各五匹,綿三十兩。外官,諸教授,錢粟一十二貫石,麥一石,衣絹各三匹,綿一十兩,職田二頃。三品以上官司知法,錢粟一十貫石,麥一石,衣絹各三匹,綿一十兩。司候判官,錢粟一十貫石,衣絹各二匹,綿八兩,職田二頃。諸防次軍轄,俸同上,無職田。諸榷場同管勾、左右別貯院木場判,錢粟一十貫石,衣絹各三匹,綿六兩。諸京作院都監、通州倉判、五品以上官司知法,錢粟
 九貫石,衣絹各二匹,綿六兩。諸府作院都監、諸埽物料場都監,錢粟八貫石,衣絹各一匹,綿六兩。諸節鎮作院都監、諸司都監,錢粟八貫石,衣絹各二匹。諸司同監,錢粟七貫石,絹同上。陜西東路德順州世襲蕃巡檢,分例月支錢粟一十貫石,衣絹各二匹,綿一十兩。陜西西路原州世襲蕃巡檢,月支錢二貫三百九十文,米四石五斗,絹三匹。河東北路葭州等處世襲蕃巡檢,月支錢粟一十貫石,絹二匹,綿一十兩。



 宮闈歲給。太后、太妃宮,每歲各給錢二千萬,綵二百段,絹千匹,綿五千兩。諸妃,歲給錢千萬,彩百段,絹三百匹,
 綿三千兩。嬪以下,錢五百萬,彩五十段,絹二百匹,綿二千兩。貞元元年,妃、嬪、婕妤、美人、及供膳女侍、并仙韶、長春院供應人等,歲給錢帛各有差。



 凡內職,貞祐之制,正一品,歲錢八千貫,幣百段,絹五百匹,綿五千兩。正二品,歲錢六千貫,幣八十段,絹三百匹,綿四千兩。正三品,歲錢五千貫,幣六十段,絹二百匹,綿三千兩。正四品,歲錢四千貫,幣四十段,絹百五十匹,綿二千兩。正五品,尚宮夫人,歲錢二千貫,幣二十段,絹百匹,綿千兩。尚宮左右夫人至宮正夫人,錢千五百貫,幣十九段,絹九十匹,綿九百兩。寶華夫人以下至資明夫
 人,錢千貫,幣十八段,絹八十匹,綿八百兩。有大、小令人,大、小承御,大、小近侍,俸各異。正六品,尚儀御侍以下,錢五百貫,幣十六段,絹五十匹,綿二百兩。正七品,司正御侍以下,錢四百貫,幣十四段,絹四十匹,綿百五十兩。正八品,典儀御侍以下,錢三百貫,幣十二段,絹三十匹,綿百兩。正九品,掌儀御侍以下,錢二百五十貫,幣十段,絹二十六匹,綿百兩。



 百司承應俸給。省令史、譯史、錢粟一十貫石,絹四匹,綿四十兩。省通事、樞密令史譯史,錢粟十二貫石,絹三匹,綿三十兩。樞密通事、六部御史臺令譯史,錢粟一十貫石,衣絹三匹,綿三十兩。六部等通事、誥院令史、國史院
 書寫、隨府書表、親王府祗候郎君、典客署引接書表,錢粟八貫石,絹二匹,綿二十兩。走馬郎君、一品子孫十貫石,內祗八貫石,班祗七貫石,並絹二匹,綿二十兩。護衛長,支正六品俸。長行,從六品俸。符寶郎、奉御、東宮護衛長,錢粟十七貫石,絹八匹,綿四十兩。東宮護衛長行,十五貫石,絹四匹,綿四十兩。筆硯承奉、閣門祗候、侍衛親軍百戶,十二貫石,絹四匹,綿三十兩。妃護衛、奉職、符寶典書、東宮入殿小底,十貫石,絹三匹,綿三十兩,勒留則添二貫石。尚衣、奉御、捧案、擎執、奉輦、知把書畫、隨庫本把、左右藏庫本把、儀鸞局本把、尚輦局本把、妃奉事,八
 貫石,絹三匹,綿三十兩。侍衛親軍五十戶,九貫石,絹三匹,綿三十兩。未係班,絹三匹,綿二十兩。長行,七貫石,絹二匹,綿二十兩。弩傘什將,八貫石。傘子,五貫石。太醫長行,八貫石,正奉上太醫,十貫石。副奉上,同。隨位承應都監,未及十五歲者六貫石,從八品七貫石,從七品八貫石,從六品九貫石,從五品十貫石,從四品十二貫石,止掌文書者添支三貫石,牌子頭等添支二貫石。司天四科人,九品六貫石,八品七貫石,六品九貫石,五品十貫石,四品十二貫石,止教授管勾十貫石,學生錢三貫、米五斗。典客、書表,八貫石,絹二匹,綿二十兩。東宮筆硯,六
 貫石。尚廄獸醫,秘書監楷書,六貫石。秘書琴棋等待詔,七貫石。駝馬牛羊群子、擠酪人,皆三貫石。



 諸使司都監食直,二十萬貫以上六十貫,十萬貫已上五十貫,五萬貫已上四十貫,三萬貫已上三十貫,二萬貫已上二十五貫。諸院務監官食直,五千貫已上監官二十貫、同監十五貫,二千貫已上監官十五貫、同監十貫,一千貫已上監官十五貫,一千貫已下監官十貫。



 舊制,凡監臨使司、院務之商稅,增者有賞,虧者剋俸。大定九年,上以吏非祿無以養廉,於是止增虧分數為殿最,乃罷剋俸、給賞之制,而監官酬賞仍舊。二十年,詔十
 萬貫以上鹽酒等使,若虧額五厘,剋俸一分。奏隨處提點院務官賞格,其省除以上提點官、并運司親管院務,若能增者十分為率以六分入官,二分與提點所官、二分與監官充賞,若虧亦依此例剋俸,若能足數則全給。大定二十二年,定每月先支其半外,如不虧則全支,虧一分則剋其一分,補足貼支。隨路使司、院務并坊場,例多虧課。上曰:「若其實可減處,約量裁減,亦公私兩便也。」二十三年,以省除提控官、與運司置司處,虧課一分剋俸一分,其罰涉重。亦命先給月俸之半,餘半驗所虧分數剋罰補,公田則不在剋限。二十六年四月,奏定院務
 監官虧永陪償格。



 諸京府運司提刑司節鎮防刺等,漢人、女直、契丹司吏、譯史、通史、孔目官,八貫。押司官,七貫。前後行,六貫。諸防刺已上女直、契丹司吏、譯史、通事,不問千里內外,錢七貫,公田三頃。諸鹽使司都目,十四貫。司吏,六貫。諸巡院司縣司獄等司吏,有譯史,通事者同,錢五貫。凡諸吏人,月支大紙五十張,小紙五百張,筆二管、墨二錠。



 諸職官上任,不過初二日,罷任過初五日者,給當月俸。或受差及因公幹未能之官者,計程外聽給到任祿。若文牒未至,前官在任,及後官已到,前官差出,其祿兩支,
 職田皆給後官。凡職田,畝取粟三斗、草一稱。倉場隨月俸支俸,曲則隨直折價。諸親王授任者,祿從多,職田從職。朝官兼外者同。六十以上及未六十而病致仕者,給其祿半。承應及軍功初出職未歷致仕,雖未六十者亦給半祿。內外吏員及諸局分承應人,病告至百日則停給。除程給假者俸祿職田皆以半給,衣絹則全給。皇家袒免以上親戶別給。夫亡,妻亦同。若同居兄弟收充猛安謀克及歷任承應人者,不在給限。大功以上,錢粟一十三貫石,春秋衣絹各四匹。小功,粟一十貫石,春秋衣絹各三匹。緦麻、袒免,錢粟八貫石,春秋衣絹二匹。



 諸馳驛及長行馬,職官日給。謂奉宣省院臺部委差、或許差者,下文置所等官同。一品三貫文,二品二貫文,三品一貫五百文,四品一貫二百文,五品一貫文,六品八百文,七百六百文,八品九品四百文。有職事官日給,外路官往回口券,依上款給一品二貫五百文,二品一貫六百文,三品一貫二百文,四品一貫文,五品九百文,六品七百文,七品六百文,八品九品四百文。無職事官並驗前職日給,無前職者以應仕及待闕職事給之。四品一貫三百文,五品一貫二百文,六品九百文,七品七百文,八品九品五百文。隨朝吏員宣差及省部差委官踏逐者,引者亦同。及統軍司按察司書吏譯人、
 本局差委及隨逐者,日給錢各一百五十文。燕賜各部官僚以下,日給米糧分例,無草地處內,親王給馬二十五匹草料,親王米一石,宰執七斗,王府三斗,府尉二斗,員外郎、司馬各一斗六升,監察御史、尚書省都事、大理司直、六部主事各八升,檢、知法七升,省令、譯史六升,院臺令譯史,省通事各五升,院臺通事、六部令譯史通事、省祗候郎君、使庫都監各四升,誥院令史、樞密院移剌各三升,王府直府、王府及省知印直省、御史臺通引、王府教讀、王傅府尉等下司吏、外路通事、省醫工調角匠、招討司移剌各二升、寫誥諸祗候人本破人同、大程官院子
 酒匠柴火各一升,萬戶一斗六升,猛安八升,謀克四升,蒲輦二升,正軍阿里喜,旗鼓吹笛司吏各一升。諸外方進貢及回賜、并人使長行馬,每匹日給草一稱、粟一斗。宮中東宮同承應人因公差出,皆驗見請錢粟貫石、口給食料,若係本職者住程不在給限,其常破馬草料局分,如被差長行馬公幹本支草料,即聽驗日剋除,若特奉宣差勾當者,依本格。十八貫石以上九百文,十七貫石八百六十文,十五貫石以上五百四十文,七貫石以上四百六十文,六貫石四百二十文,五貫石三百八十文,四貫石三百三十文,三貫石二百八十文,二貫石二百
 三十文。



 諸試護衛親軍,聽自起發日為始,計程至都,比至試補,其間各日給口券,若揀退還家者,亦驗回程給之。未起閑住口數不在支限。其正收之後再揀退者,亦給人三口米糧錢一百文、馬二匹草料。諸簽軍赴鎮防處、及班祗充押遞橫差別路勾當千里以上者,沿路各日給米一升、馬一匹草料。無馬有驢者,各支依本格。車駕巡幸,顧工,馬夫三百文,步夫二百三十文,圍鵝夫,隨程乾辦人各二百文,傳遞果子夫一百五十文。車駕巡幸,若於私家內安置行宮者,約量給賜段匹。太廟神廚祠祭度勾當人、少府監隨色工匠、
 部役官受給官司吏,錢粟二貫石,春秋衣絹各一匹。



 諸局作匠人請俸,繡女都管錢粟五貫石,都繡頭錢粟四貫石,副繡頭三貫五百石,中等細繡人三貫石,次等細繡人二貫五百石,習學本把正辦人錢支次等之半,描繡五人錢粟三貫石,司吏二人三貫石。修內司,作頭五貫石,工匠四貫石,春秋衣絹各二匹。軍夫除錢糧外,日支錢五十,米一升半。百姓夫每日支錢一百、米一升半。國子監雕字匠人,作頭六貫石,副作頭四貫石,春秋衣絹各二匹。長行三貫石,射糧軍匠錢粟三貫石,春秋衣絹各二匹,習學給半。初習學匠錢六百,米六斗,春秋
 絹各一匹,布各一匹。民匠日支錢一百八十文。



 諸隨朝五品以下職事官身故,因公差出、及以理去任、未給解由者,身故同。驗品,從去鄉地里支給津遣錢。並受職事給之,下條承應人准此。若外路官員在任依理身故者,皆依上官品地里減半給之。若係五百里內不在給限,五百里外,五品一百貫,六品七品八十貫,八品九品六十貫。一千里外,五品一百二十貫,六品七品一百貫,八品九品八十貫。二千里外,五品一百七十貫,六品七品一百五十貫,八品九品一百貫。三千里外,五品二百五十貫,六品七品二百貫,八品九品一百五十貫。諸隨朝承應人身故應給津遣錢者,
 護衛東宮護衛同、奉御、符寶、都省樞密院御史臺令譯史同九品官,通事、宗正府六部令譯史、統軍司書史譯書、按察司書史,同。親軍減九品官五分之二,通事、隨朝書表、吏員、譯人、統軍司通事、守當官,按察司書吏、譯人,分治都水監典吏,同。及諸局分承應人武衛軍同減五分之三。天壽節設施老疾貧民錢數,在都七百貫宮籍監給,諸京二十五貫此以下並係省錢給,諸府二十貫文,諸節鎮一十五貫文,諸防刺州軍一十貫文,諸外縣五貫文。城寨系保鎮同。諸孤老幼疾人,各月給米二斗、錢五百文,春秋衣絹各一匹五歲以下三分給二,身死者給錢一貫埋殯。諸因災傷或遭賊驚卻饑荒去處,良民典顧、冒賣為驅,遇恩官贖為良分
 例若元價錢給,男子一十五貫文,婦人同,老幼各減半。六歲以下即聽出離,不在贖換之限。諸士庶陳言利害,若有可採,行之便於官民者,依驗等第給賞,上等銀絹三十兩匹,中等二十兩匹,下等一十兩匹,其陳數事,止從一支。若用大事應補官者,從吏部格。



 宣宗貞祐元年十二月,以糧儲不足,詔隨朝官、承應人俸,計口給之,餘依市直折之。諭旨省臣曰:「聞親軍俸,粟每石以麥六斗折之,所省能幾,而失眾心,令給本色。」二年八月,始給京府州縣及轉運司吏人月俸有差。舊制惟吏案孔目官有俸,餘止給食錢,故更定焉。三年,詔損宮中諸位歲給有差。監察御史田迥秀言:「國家調度,行
 纔數月,已後停滯,所患在支太多,收太少,若隨時裁損所支,而增其收,庶可久也。」因條五事:「一曰朝官及令譯史、諸司吏員、諸局承應人,太冗濫宜省併之。隨處屯軍皆設寄治官,徒費俸給,不若令有司兼總之。且沿河亭障各駐鄉兵,彼皆白徒,皆不可用,不若以此軍代之,以省其出。」四月,以調度不及,罷隨朝六品以下官及承應人從己人力輸傭錢。減修內司所役軍夫之半。經兵處,州、府、司吏減半,司、縣三分減一,其餘除開封府、南京轉運司外,例減三分之一。有祿官吏而不出境者,並罷給券,出境者給其半。興定二年正月,詔「:陜州等處司、縣官
 征稅不足,閣其俸給何以養廉,自今不復閣俸。」彰化軍節度使張行信言:「送宣之使,其視五品而上各有定數,後竟停罷。今軍官以上奉待使者有所饋獻,至六品以下亦不免如例,而莫能辦,則斂所部以與之,至有獲罪者。保舉縣尹,特增其俸,然法行至今,而關以西尚有未到任者,豈所舉少而不敷耶?宜廣選舉,以補其闕。且丞簿亦親民者也,而獨不增,安能禁其侵牟哉!」



\end{pinyinscope}