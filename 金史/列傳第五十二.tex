\article{列傳第五十二}

\begin{pinyinscope}

 ○白華斜卯愛實合周附石抹世勣



 白華,字文舉,庾州人。貞祐三年進士。初為應奉翰林文字。正大元年,累遷為樞密院經歷官。二年九月,武仙以真定來歸,朝廷方經理河北,宋將彭義斌乘之,遂由山東取邢、洺、磁等州。華上奏曰:「北兵有事河西,故我得少寬。今彭義斌招降河朔郡縣,駸駸及於真定,宜及此大舉,以除後患。」時院官不欲行,即遣華相視彰德,實擠之
 也,事竟不行。



 三年五月,宋人掠壽州,永州桃園軍失利,死者四百餘人。時夏全自楚州來奔。十一月庚申,集百官議和宋。上問全所以來,華奏:「全初在盱眙,從宋帥劉卓往楚州。州人訛言劉大帥來,欲屠城中北人耳。眾軍怒,殺卓以城來歸。全終不自安,跳走盱眙,盱眙不納,城下索妻孥,又不從,計無所出,乃狼狽而北,止求自免,無他慮也。」華因是為上所知。全至後,盱眙、楚州,王義深、張惠、范成進相繼以城降。詔改楚州為平淮府,以全為金源郡王、平淮府都總管,張惠臨淄郡王,義深東平郡王,成進膠西郡王。和宋議寢。四年,李全據楚州,眾皆謂盱眙
 不可守,上不從,乃以淮南王招全,全曰:「王義深、范成進皆我部曲而受王封,何以處我。」竟不至。



 是歲,慶山奴敗績于龜山。五年秋,增築歸德城,擬工數百萬,宰相奏遣華往相役,華見行院溫撒辛,語以民勞,朝廷愛養之意,減工三之一。溫撒,李辛賜姓也。



 六年,以華權樞密院判官。上召忠孝軍總領蒲察定住、經歷王仲澤、戶部郎中刁璧及華諭之曰:「李全據有楚州,睥睨山東,久必為患。今北事稍緩,合乘此隙令定住權監軍,率所統軍一千,別遣都尉司步軍萬人,以璧、仲澤為參謀,同往沂、海界招之,不從則以軍馬從事,卿等以為何如?」華對曰:「臣以
 為李全借大兵之勢,要宋人供給餽餉,特一猾寇耳。老狐穴塚,待夜而出,何足介懷。我所慮者北方之強耳。今北方有事,未暇南圖,一旦事定,必來攻矣。與我爭天下者此也,全何預焉。若北方事定,全將聽命不暇,設不自量,更有非望,天下之人寧不知逆順,其肯去順而從逆乎!為今計者,姑養士馬,以備北方。使全果有不軌之謀,亦當發於北朝息兵之日,當此則我易與矣。」上沉思良久曰:「卿等且退,容我更思。」明日,遣定住還屯尉氏。



 時陜西兵大勢已去,留脫或欒駐慶陽以擾河朔,且有攻河中之耗,而衛州帥府與恒山公府並立,慮一旦有警,節
 制不一,欲合二府為一,又恐其不和,命華往經畫之。初,華在院屢承面諭云:「汝為院官,不以軍馬責汝。汝辭辯,特以合喜、蒲阿皆武夫,一語不相入,便為齟齬,害事非細,今以汝調停之,或有乖忤,罪及汝矣。院中事當一一奏我,汝之職也。今衛州之委,亦前日調停之意。」



 國制,凡樞密院上下所倚任者名奏事官,其目有三,一曰承受聖旨,二曰奏事,三曰省院議事,皆以一人主之。承受聖旨者,凡院官奏事,或上處分,獨召奏事官付之,多至一二百言,或直傳上旨,辭多者即與近侍局官批寫。奏事者,謂事有區處當取奏裁者殿奏,其奏每嫌辭費,必欲
 言簡而意明,退而奉行,即立文字,謂之檢目。省院官殿上議事則默記之,議定歸院,亦立檢目呈覆。有疑則復稟,無則付掾史施行。其赴省議者,議既定,留奏事官與省左右司官同立奏草,圓覆諸相無異同,則右司奏上。此三者之外又有難者,曰備顧問,如軍馬糧草器械、軍帥部曲名數、與夫屯駐地里阨塞遠近之類,凡省院一切事務,顧問之際一不能應,輒以不用心被譴,其職為甚難,故以華處之。



 五月,以丞相賽不行尚書省事於關中,蒲阿率完顏陳和尚忠孝軍一千駐邠州,且令審觀北勢。如是兩月,上謂白華曰:「汝往邠州六日可往復否?」
 華自量日可馳三百,應之曰:「可。」上令密諭蒲阿纔候春首,當事慶陽。華如期而還。上一日顧謂華言:「我見汝從來凡語及征進,必有難色,今此一舉特銳於平時,何也?」華曰:「向日用兵,以南征及討李全之事梗之,不能專意北方,故以北向為難。今日異於平時,況事至於此,不得不一舉。大軍入界已三百餘里,若縱之令下秦川則何以救,終當一戰摧之。戰於近裏之平川,不若戰於近邊之險隘。」上亦以為然。



 七年正月,慶陽圍解,大軍還。白華上奏:「凡今之計,兵食為急。除密院已定忠孝軍及馬軍都尉司步軍足為一戰之資,此外應河南府州亦
 須簽揀防城軍,秋聚春放,依古務農講武之義,各令防本州府城,以今見在九十七萬,無致他日為資敵之用。」五月,華真授樞密判官,上遣近侍局副使七斤傳旨云:「朕用汝為院官,非責汝將兵對壘,第欲汝立軍中綱紀、發遣文移、和睦將帥、究察非違,至於軍伍之閱習、器仗之修整,皆汝所職。其悉力國家,以稱朕意。」



 八年,大軍自去歲入陜西,翱翔京兆、同、華之間,破南山砦柵六十餘所。已而攻鳳翔,金軍自閿鄉屯至澠池,兩行省晏然不動。宰相臺諫皆以樞院瞻望逗遛為言,京兆士庶橫議蜂起,以至諸相力奏上前。上曰:「合達、蒲阿必相度機會,
 可進而進耳。若督之使戰,終出勉強,恐無益而反害也。」因遣白華與右司郎中夾谷八里門道宰相百官所言,并問以「目今二月過半,有怠歸之形,諸軍何故不動?」且詔華等往復六日。華等既到同,諭兩行省以上意。合達言:「不見機會,見則動耳。」蒲阿曰:「彼軍絕無糧餉,使欲戰不得,欲留不能,將自敝矣。」合達對蒲阿及諸帥則言不可動,見士大夫則言可動,人謂合達近嘗得罪,又畏蒲阿方得君,不敢與抗,而亦言不可動。華等觀二相見北兵勢大皆有懼心,遂私問樊澤、定住、陳和尚以為何如,三人者皆曰:」他人言北兵疲困,故可攻,此言非也。大兵
 所在,豈可輕料?是真不敢動。」華等還,以二相及諸將意奏之,上曰:」我故知其怯不敢動矣。」即復遣華傳旨諭二相云:「鳳翔圍久,恐守者力不能支。行省當領軍出關,宿華陰界,次日及華陰,次日及華州,略與渭北軍交手。計大兵聞之必當奔赴,且以少紓鳳翔之急,我亦得為掣肘計耳。」二相迴奏領旨。華東還及中牟,已有兩行省納奏人追及,華取報密院副本讀之,言:「領旨提軍出關二十里至華陰界,與渭北軍交,是晚收軍入關。」華為之仰天浩歎曰:「事至於此,無如之何矣。」華至京,奏章已達,知所奏為徒然,不二三日鳳翔陷,兩行省遂棄京兆,與牙
 古塔起遷居民於河南,留慶山奴守之。



 夏五月,楊妙真以夫李全死於宋,構浮橋於楚州之北,就北帥梭魯胡吐乞師復仇。朝廷覘知之,以謂北軍果能渡淮,淮與河南跬步間耳,遣合達、蒲阿駐軍桃源界滶河口備之。兩行省乃約宋帥趙范,趙葵為夾攻之計。二趙亦遣人報聘,俱以議和為名,以張聲勢。二相屢以軍少為言,而省院難之,因上奏云:「向來附關屯駐半年,適還舊屯,喘不及息,又欲以暑月東行,實無可圖之事,徒自疲而已。況兼桃源、青口蚊虻湫濕之地,不便牧養,目今非征進時月,決不敢妄動。且我之所慮,特楚州浮梁耳。姑以計圖
 之,已遣提控王銳往視可否。」奏上,上遣白華以此傳諭二相,兼領王銳行。二相不悅。蒲阿遣水軍虹縣所屯王提控者以小船二十四隻,令華順河而下,必到八里莊城門為期,且曰:「此中望八里莊,如在雲間天上,省院端坐,徒事口吻,今樞判親來,可以相視可否,歸而奏之。」華力辭不獲,遂登舟。及淮與河合流處,纔及八里莊城門相直,城守者以白鷂大船五十溯流而上,占其上流以截華歸路。華幾不得還,昏黑得徑先歸,乃悟兩省怒朝省不益軍,謂皆華輩主之,故擠之險地耳。是夜二更後,八里莊次將遣人送款云:」早者主將出城開船,截大金
 歸路,某等商議,主將還即閉門不納,渠已奔去楚州,乞發軍馬接應。」二相即發兵騎、開船赴約,明旦入城安慰,又知楚州大軍已還河朔,宋將燒浮橋,二相附華納奏,上大喜。



 初,合達謀取宋淮陰。五月渡淮。淮陰主者胡路鈐往楚州計事於楊妙真,比還,提正官郭恩送款於金,胡還不納,慟哭而去。合達遂入淮陰,詔改歸州,以行省烏古論葉里哥守之,郭恩為元帥右都監。既而,宋人以銀絹五萬兩匹來贖盱眙龜山,宋使留館中,郭恩謀劫而取之,或報之于盱眙帥府,即以軍至,恩不果發。明日,宋將劉虎、湯孝信以船三十艘燒浮梁,因遣其將夏友
 諒來攻盱眙,未下。泗州總領完顏矢哥利館中銀絹,遂反。防禦使徒單塔剌聞變,扼罘山亭甬路,好謂之曰:「容我拜辭朝廷然後死。」遂取朝服望闕拜,慟良久,投亭下水死。矢哥遂以州歸楊妙真,總帥納合買住亦以盱眙降宋。



 九月,陜西行省防秋,時大兵在河中,睿宗已領兵入界,慶山奴報糧盡,將棄京兆而東。一日,白華奏,偵候得睿宗所領軍馬四萬,行營軍一萬,布置如此,「為今計者,與其就漢禦之,諸軍比到,可行半月,不若徑往河中。目今沿河屯守,一日可渡,如此中得利,襄、漢軍馬必當遲疑不進。在北為投機,在南為掣肘,臣以為如此便」。上
 曰:「此策汝畫之,為得之他人?」華曰:「臣愚見如此。」上平日銳於武事,聞華言若欣快者,然竟不行。



 未幾,合達自陜州進奏帖,亦為此事,上得奏甚喜。蒲阿時在洛陽,驛召之,蓋有意於此矣。蒲阿至,奏對之間不及此,止言大兵前鋒忒木泬統之,將出冷水谷口,且當先禦此軍。上曰:「朕不問此,只欲問河中可搗否。」蒲阿不獲已,始言睿宗所領兵騎雖多,計皆冗雜。大兵軍少而精,無非選鋒。金軍北渡,大兵必遣輜重屯於平陽之北,匿其選鋒百里之外,放我師渡,然後斷我歸路與我決戰,恐不得利。」上曰:「朕料汝如此,果然。更不須再論,且還陜州。」蒲阿曰:「合
 達樞密使所言,此間一面革撥恐亦未盡,乞召至同議可否。」上曰:「見得合達亦止此而已,往復遲滯,轉致誤事。」華奏合達必見機會,召至同議為便。副樞赤盞合喜亦奏蒲阿、白華之言為是。上乃從之。召合達至,上令先與密院議定,然後入見。既議,華執合達奏帖舉似再三,竟無一先發言者。移時,蒲阿言:「且勾當冷水谷一軍何如。」合達曰:「是矣。」遂入見。上問卿等所議若何,合達敷奏,其言甚多,大概言河中之事與前日上奏時勢不同,所奏亦不敢自主,議遂寢。二相還陜,量以軍馬出冷水谷,奉行故事而已。十二月,河中府破。



 九年,京城被攻。四月兵
 退,改元天興。是月十六日,併樞密院歸尚書省,以宰相兼院官,左右司首領官兼經歷官,惟平章白撒、副樞合喜、院判白華、權院判完顏忽魯剌退罷。忽魯剌有口辯,上愛幸之。朝議罪忽魯剌,而書生輩妒華得君,先嘗以語撼之,用是而罷。金制,樞密院雖主兵,而節制在尚書省。兵興以來,茲制漸改,凡是軍事,省官不得預,院官獨任專見,往往敗事。言者多以為將相權不當分,至是始併之。



 十二月朔,上遣近侍局提點曳剌粘古即白華所居,問事勢至於此,計將安出。華附奏:「今耕稼已廢,糧斛將盡,四外援兵皆不可指擬,車駕當出就外兵。可留皇兄
 荊王使之監國,任其裁處。聖主既出,遣使告語北朝,我出非他處收整軍馬,止以軍卒擅誅唐慶,和議從此斷絕,京師今付之荊王,乞我一二州以老耳。如此則太后皇族可存,正如《春秋》紀季入齊為附庸之事,聖主亦得少寬矣。」於是起華為右司郎中。初,親巡之計決,諸將皆預其議,將退,首領官張袞、聶天驥奏:「尚有舊人諳練軍務者,乃置而不用,今所用者,皆不見軍中事體,此為未盡。」上問未用者何人,皆曰院判白華,上頷之,故有是命。



 明日,召華諭之曰:「親巡之計已決,但所往群議未定,有言歸德四面皆水,可以自保者,或言可沿西山入鄧。或
 言設欲入鄧,大將速不泬今在汝州,不如取陳、蔡路轉往鄧下。卿以為如何?」華曰:「歸德城雖堅,久而食盡,坐以待斃,決不可往。欲往鄧下,既汝州有速不泬,斷不能往。以今日事勢,博徒所謂孤注者也。孤注云者,止有背城之戰。為今之計,當直赴汝州,與之一決,有楚則無漢,有漢則無楚。汝州戰不如半途戰,半途戰又不如出城戰,所以然者何?我軍食力猶在,馬則豆力猶在。若出京益遠,軍食日減,馬食野草,事益難矣。若我軍便得戰,存亡決此一舉,外則可以激三軍之氣,內則可以慰都人之心。或止為避遷之計,人心顧戀家業,未必毅然從行。可
 詳審之。」遂召諸相及首領官同議,禾速嘉兀地不、元帥豬兒、高顯、王義深俱主歸德之議,丞相賽不主鄧,議竟不能決。明日,制旨京城食盡,今擬親出,聚集軍士於大慶殿諭以此意,諭訖,諸帥將佐合辭奏曰:「聖主不可親出,止可命將,三軍欣然願為國家效死。」上猶豫,欲以官奴為馬軍帥,高顯為步軍帥,劉益副之,蓋採輿議也,而三人者亦欲奉命。權參政內族訛出大罵云:「汝輩把鋤不知高下,國家大事,敢易承邪!」眾默然,惟官奴曰:「若將相可了,何至使我輩。」事亦中止。



 明日,民間哄傳車駕欲奉皇太后及妃后往歸德,軍士家屬留後。目今食盡,坐
 視城中俱餓死矣。縱能至歸德,軍馬所費支吾復得幾許日。上聞之,召賽不、合周、訛出、烏古孫卜吉、完顏正夫議,餘人不預。移時方出,見首領官、丞相言,前日巡守之議已定,止為一白華都改卻,今往汝州就軍馬索戰去矣。遂擇日祭太廟誓師,擬以二十五之日啟行。是月晦,車駕至黃陵岡,復有北幸之議,語在《白撒傳》。



 天興二年正月朔,上次黃陵岡,就歸德餫船北渡,諸相共奏,京師及河南諸州聞上幸河北,恐生他變,可下詔安撫之。是時,在所父老僧道獻食,及牛酒犒軍者相屬,上親為拊慰,人人為之感泣。乃赦河朔,招集兵糧,赦文條畫十餘款,
 分道傳送。二日,或有云:「昨所發河南詔書,倘落大軍中,奈泄事機何。」上怒,委近侍局官傳旨,謂首領官張袞、白華、內族訛可當發詔時不為後慮,皆量決之。是時衛州軍兩日至蒲城,而大軍徐躡其後。十五日,宰相諸帥共議上前,郎中完顏胡魯剌秉筆書,某軍前鋒,某軍殿後,餘事皆有條畫。書畢,惟不言所往,華私問胡魯剌,託以不知。是晚,平章及諸帥還蒲城軍中。夜半,訛可、袞就華帳中呼華云:「上已登舟,君不知之耶?」華遂問其由,訛可云:「我昨日已知上欲與李左丞、完顏郎中先下歸德,令諸軍並北岸行,至鳳池渡河。今夜平章及禾速嘉、元帥
 官奴等來,言大軍在蒲城曾與金軍接戰,勢莫能支,遂擁主上登舟,軍資一切委棄,止令忠孝軍上船,馬悉留營中。計舟已行數里矣。」華又問:「公何不從往?」云:「昨日擬定首領官止令胡魯剌登舟,餘悉隨軍,用是不敢。」是夜,總帥百家領諸軍舟往鳳池,大軍覺之,兵遂潰。



 上在歸德。三月,崔立以汴京降,右宣徽提點近侍局移剌粘古謀之鄧,上不聽。時粘古之兄瑗為鄧州節度使、兼行樞密院事,其子與粘古之子並從駕為衛士。適朝廷將召鄧兵入援,粘古因與華謀同之鄧,且拉其二子以往,上覺之,獨命華行,而粘古改之徐州。華既至鄧,以事久不
 濟,淹留于館,遂若無意於世者。會瑗以鄧入宋,華亦從至襄陽,宋署為制乾,又改均州提督。後范用吉殺均之長吏。送款于北朝,遂因而北歸。士大夫以華夙儒貴顯,國危不能以義自處為貶云。



 用吉者,本姓孛術魯,名久住。初歸入宋,謁制置趙范,將以計動其心,故更姓名范用吉。趙怒其觸諱,斥之,用吉猶應對如故。趙良久方悟,且利其事與己符,遂擢置左右,凡所言動,略不加疑,遂易其姓曰花,使為太尉,改鎮均州。未幾,納款于北。後以家人誣以欲叛,為同列所害。



 贊曰:白華以儒者習吏事,以經生知兵,其所論建,屢中
 事機,然三軍敗衄之餘,士氣不作,其言果可行乎。從瑗歸宋,聲名掃地,則猶得列於金臣之傳者,援蜀譙周等例云。



 斜卯愛實,字正之,策論進士也。正大間,累官翰林直學士,兼左司郎中。天興元年正月,聞大兵將至,以點檢夾谷撒合為總帥,率步騎三萬巡河渡,命宿直將軍內族長樂權近侍局使,監其軍。行至封丘而還。入自梁門,樞密副使合喜遇之,笑語撒合曰:「吾言信矣,當為我作主人。」蓋世俗酬謝之意也。明日,大兵遂合,朝廷置而不問。於是愛實上言曰:「撒合統兵三萬,本欲乘大兵遠至,喘
 息未定而擊之。出京纔數十里,不逢一人騎,已畏縮不敢進。設遇大兵,其肯用命乎?乞斬二人以肅軍政。」不報。蓋合喜輩以京師倚此一軍為命,初不敢俾之出戰,特以外議哄然,故暫出以應之云。



 衛紹、鎬厲二王家屬,皆以兵防護,且設官提控,巡警之嚴過於獄犴。至是,衛紹宅二十年,鎬厲宅四十年。正大間,朝臣屢有言及者,不報。愛實乃上言曰:「二族衰微,無異匹庶,假欲為不善,孰與同惡?男女婚嫁,人之大欲,豈有幽囚終世,永無伉儷之望,在他人尚且不忍,況骨肉乎!」哀宗感其言,始聽自便。未幾,有青城之難。



 愛實憤時相非其人,嘗歷數曰:「平
 章白撒固權市恩,擊丸外百無一能。丞相賽不菽麥不分,更謂乏材,亦不至此人為相。參政兼樞密副使赤盞合喜麤暴,一馬軍之材止矣,乃令兼將相之權。右丞顏盞世魯居相位已七八年,碌碌無補,備員而已。患難之際,倚注此類,欲冀中興,難矣。」於是世魯罷相,賽不乞致仕,而白撒、合喜不恤也。



 是年四月,京城罷攻,大兵退。既而以害唐慶事,和議遂絕。於是再簽民兵為守禦備。八月,括京城粟,以轉運使完顏珠顆、張俊民、曳剌克忠等置局,以推舉為名,珠顆諭民曰:「汝等當從實推唱,果如一旦糧盡,令汝妻子作軍食,復能吝否?」既而罷括粟令,
 復以進獻取之。前御史大夫內族合周復冀進用,建言京城括粟可得百餘萬石。朝廷信之,命權參知政事,與左丞李蹊總其事。先令各家自實,壯者存石有三斗,幼者半之,仍書其數門首,敢有匿者以升斗論罪。京城三十六坊,各選深刻者主之,內族完顏久住尤酷暴。有寡婦二口,實豆六斗,內有蓬子約三升,久住笑曰:「吾得之矣。」執而以令於眾。婦泣訴曰:「妾夫死於兵,姑老不能為養,故雜蓬粃以自食耳,非敢以為軍儲也。且三升,六斗之餘。」不從,竟死杖下。京師聞之股栗,盡投其餘於糞溷中。或白於李蹊,蹊顰蹙曰:「白之參政。」其人即白合周,周
 曰:「人云『花又不損,蜜又得成』。予謂花不損,何由成蜜?且京師危急,今欲存社稷耶?存百姓耶?」當時皆莫敢言,愛實遂上奏,大概言:「罷括粟,則改虐政為仁政,散怨氣為和氣。」不報。



 時所括不能三萬斛,而京城益蕭然矣。自是之後,死者相枕,貧富束手待斃而已。上聞之,命出太倉米作粥以食餓者,愛實聞之嘆曰:「與其食之,寧如勿奪。」為奉御把奴所告。又近侍干預朝政,愛實上章諫曰:「今近侍權太重,將相大臣不敢與之相抗。自古僕御之臣不過供給指使而已,雖名僕臣,亦必選擇正人。今不論賢否,惟以世胄或吏員為之。夫給使令之材,使預社稷
 大計,此輩果何所知乎。」章既上,近侍數人泣訴上前曰:「愛實以臣等為奴隸,置至尊何地耶!」上益怒,送有司。近侍局副使李大節從容開釋,乃赦之,出為中京留守,後不知所終。



 合周者,一名永錫。貞祐中,為元帥左監軍,失援中都,宣宗削除官爵,杖之八十。已而復用。四年,以御史大夫權尚書右丞,總兵陜西。合周留澠池數日,進及京兆,而大兵已至,合周竟不出兵,遂失潼關。有司以敵至不出兵當斬,諸皇族百餘人上章救之,上曰:「向合周救中都,未至而軍潰,使宗廟山陵失守,罪當誅,朕特寬貸以全其命。尋復重職,今鎮陜西,所犯乃爾,國家大法,
 豈敢私耶!」遂再奪爵,免死除名。至是,為參知政事。性好作詩詞,語鄙俚,人采其語以為戲笑。因自草《括粟榜文》,有「雀無翅兒不飛,蛇無頭兒不行」等語,以「而」作「兒」,掾史知之,不敢易也。京城目之曰「雀兒參政」。哀宗用而不悟,竟致敗事。



 石抹世勣,字景略。幼勤學,為文有體裁。承安二年,以父元毅死王事,收充擎執。五年,登詞賦、經義兩科進士第。貞祐三年,累官為太常丞,預講議所事。時朝廷徙河北軍戶河南,宰職議給以田,世勣上言曰:「荒閑之田及牧馬地,其始耕墾,費力當倍,一歲斷不能熟。若奪民素蒔
 者與之,則民將失所,且啟不和之端。況軍戶率無耕牛,雖或有之,而廩給未敢遽減。彼既南來,所捐田宅為人所有,一旦北歸,能無爭奪?切謂宜令軍戶分人歸守本業,收其晚禾,至春復還為固守計。」會侍御史劉元規亦言給田不便,上大悟,乃罷之。未幾,遷同知金安軍節度使。興定二年,選為華州元帥府參議官。初,右都監完顏合達行帥府于楨州,嘗以前同知平涼府事卓魯回蒲乃速為參議,及移駐華州,陜西行省請復用蒲乃速,令世勣副之。上曰:「蒲乃速但能承奉人耳,餘無所長,非如世勣可任以事。華為要鎮,而輕用其人,或致敗事。」遂獨
 用世勣焉。尋入為尚書省左司郎中。元光元年,奪一官,解職。初,世勣任華州,有薦其深通錢穀者,復察不如所舉,未籍行止中。後主者舉覺,平章英王以世勣避都司之繁,私屬治籍吏冀改他職,奏下有司,故有是責。久之,起為禮部侍郎,轉司農,改太常卿。正大中,為禮部尚書,兼翰林侍講學士。



 天興元年冬,哀宗將北渡,世勣率朝官劉肅、田芝等二十人求見仁安殿。上問卿等欲何言,世勣曰:「臣等聞陛下欲親出,切謂此行不便。」上曰:「我不出,軍分為二,一軍守,一軍出戰。我出則軍合為一。」世勣曰:「陛下出則軍分為三,一守、一戰、一中軍護從,不若不
 出為愈也。」上曰:「卿等不知,我若得完顏仲德、恒山公武仙付之兵事,何勞我出。我豈不知今日將兵者,官奴統馬兵三百止矣,劉益將步兵五千止矣,欲不自將,得乎?」上又指御榻曰:「我此行豈復有還期,但恨我無罪亡國耳。我未嘗奢侈,未嘗信任小人。」世勣應聲曰:「陛下用小人則亦有之。」上曰:「小人謂誰?」世勣歷數曰:「移剌粘古、溫敦昌孫、兀撒惹、完顏長樂皆小人也。陛下不知為小人,所以用之。」肅與世勣復多有言,良久,君臣涕泣而別。初,肅等求見,本欲數此四人。至是,世勣獨言之,於是哀宗以世勣行。自蒲城至歸德。明年六月,走蔡州,次新蔡
 縣之姜寨。



 世勣子嵩,時為縣令,拜上於馬前,兵亂後父子始相見。上嘉之,授嵩應奉翰林文字,以便養親。蔡城破,父子俱死。嵩字企隆,興定二年經義進士。



 贊曰:愛實言衛、鎬家屬禁錮之虐,京城括粟之暴,近侍干政之橫;世勣言河北軍戶給田之不便,親出渡河之非計;皆藥石之言也。然金至斯時,病在膏肓間矣,倉扁何施焉。其為忠讜,則不可廢也。



\end{pinyinscope}