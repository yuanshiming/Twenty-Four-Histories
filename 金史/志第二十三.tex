\article{志第二十三}

\begin{pinyinscope}

 儀衛下



 ○大駕鹵簿



 世宗大定三年,袷享,用黃麾仗三千人。分四節:第一節,無縣令、府牧,即用黃麾前三部,次前部鼓吹,次金吾牙門旗,次駕頭,次引駕龍墀隊,次天王、十二辰等旗。第二節,黃麾第四、第五部,次君王萬歲日月旗,次御馬,內增控馬司圉、挾馬司圉各一十六人,次日
 月合璧、五星連珠等旗,次八寶,內增執黑杖傳喝一十八人在香案前,次七寶輦。第三節,黃麾後第一、第二部,次玉輅,次栲栳隊,次導駕門仗官。第四節,黃麾後第三、第四、第五部,次金輅,次牙門旗,次後部鼓吹。大定六年九月,西京還都,用黃麾仗二千五百四十二人攝官在內,騎七百六十二匹。分四節:第一節,攝官五十四人,執擎三百二人,樂工一百七十人。第二節,攝官三十二人,執擎三百七十六人。第三節,仗內攝官四十四人,導駕官四十二人。門仗官一百人,玉輅青馬八、駕士一百四十人,護駕栲隊五百人,執擎二百四十二人。
 第四節,攝官五十人,金輅赤馬八、駕士九十四人,控鶴二十二人,樂工八十四人,執擎二百九十人。是歲,上還自西京,有司備儀仗,皇太子乘綴輅,上疑其非禮,以問禮官,無能知者,上怒,皆責降之。明年,將冊皇太子,宰臣奏當備儀仗告廟,上曰:「前朕受尊號謁謝,但令朕用宋真宗故事,朝服乘馬,於禮甚輕,今皇太子乃用備禮何耶?」丞相良弼謝,上徐曰:「此文臣因循,不加意爾。」先是,凡行幸皆役民執仗,是後詔以軍士易之。



 大定十一年,將有事於南郊,朝享太廟,右丞石琚奏其禮,上曰:「前朝漢人祭天,惟務整肅儀仗,此自奉耳,非敬天也。朕謂祭
 天在誠,不在儀仗之盛也,其減半用之。」於是,遂增損黃麾仗為大駕鹵簿,凡用七千人攝官在內,分八節。



 第一節。第一引,七十人,縣令。第二引,二百六十四人,府牧。第三引,二百二十九人,御史大夫,名色與府牧同,頗損其數,而增行止旗一。



 第二節。金吾皁纛旗一十二人,朱雀隊三十四人,指南、記里鼓車皆五十二人,鸞旗車一十八人。前部鼓吹一百二十九人。清游隊七十二人:內白澤旗二,旗五人,綠具裝冠、綠皮甲勒皮、錦臂韝、橫刀,引夾加弓矢,綠皮馬甲、包尾全。折衝都尉二人,黑平巾幘、紫繡辟邪袍、革帶、銀褐大口褲、錦螣蛇、橫刀、弓矢。弩六、弓矢二十四,槊三十。並錦帽、青繡寶相花衫、革帶、銀褐大口褲。佽飛隊四十八人:內
 果毅都尉二,黑平巾幘、紫繡飛麟袍、革帶、銀褐大口褲、錦螣蛇,佩橫刀、弓矢。虞候佽飛三十人,鐵甲、兜牟、橫刀、弓矢、黑馬甲全。鐵甲佽飛一十六人。服、執如上。前部馬隊,第一隊六十四人,第二、第三隊皆六十人,第四、第五隊皆五十八人。殳叉仗五十四人:內帥兵官二人,黑平巾幘、緋寶相花衫、革帶、銀褐大口褲,執儀刀。殳二十六,叉二十六。五色寶相花衫、抹額、抹帶、行縢、鞋襪。行止旗一。緋繡寶相花衫、抹額、銀褐抹帶、大口褲。



 第三節。前部鼓吹第二,三百六十九人。前步甲隊,第一至第五隊皆四十二人。衙門旗二十人。黃麾前第一部一百五十人,第二部一百二十人。殳叉仗五十八人。行止旗一。



 第四節。黃麾幡三人,六軍儀仗二百二十六人,御馬三十
 三人,黃麾前第三至第五部皆一百二十人,青龍白虎隊五十二人,殳叉仗五十六人,行止旗一。



 第五節。八寶二百三十二人,平頭輦三十人,七寶輦四十二人。班劍、儀刀隊二百人:內將軍二人,折衝都尉二人,平巾幘、緋闢邪袍、革帶、銀褐大口褲、錦螣蛇,執儀刀。班劍、儀刀各九十八。並平巾幘、緋繡寶相花衫、革帶、銀褐大口褲、錦螣蛇、執儀刀。驍衛翊衛隊六十人:內供奉郎將二員,黑平巾幘、緋繡寶瑞馬袍、革帶、銀褐大口褲,執儀刀。鳳旗二,旗五人,服、執如前。弩、弓矢、槊皆一十六。服如班劍,橫刀。夾轂隊,第一隊九十二人:內折衝都尉二人,平巾幘、緋繡飛麟袍、革帶、銀褐大口褲,執儀刀。寶符旗二,旗五人,朱鍪甲刀盾八十。硃甲、錦臂韝、行縢、鞋襪。第二隊八十二人:內果毅
 都尉二人,白澤袍。飛黃旗二,旗五人,銀褐鍪甲刀盾七十。第三隊八十二人:內果毅都尉二人,赤豹袍。吉利旗二,旗五人,皁鍪甲刀盾七十。殳叉仗五十六人。行止旗一。



 第六節。馬步門旗隊一百人,駕頭一十五人,廣武官、茶酒班執從物者二十三人。御龍直四十人。紅錦團襖、鍍金束帶,內人員二皁帽,三十八人真珠頭巾。玉輅一百五十一人。栲栳隊五百人:內金槍隊一百二十六人,分左右,人員十八、並鐵甲、皁帽、紅錦背子,執小旗,馬甲,紅錦包尾。長行一百八人,鐵甲、兜牟、紅錦背子、錦臂韝,甲馬、紅錦包尾,執金槍。銀槍隊一百二十六人,人員十八、長行一百八人,服並如上,銀槍。弓箭直步隊一百二十四人,人員四、鐵甲、皁帽、紅錦團
 花戰袍、弓矢,執銀骨朵,馬甲全。長行一百二十人,鐵笠、紅錦團花戰袍、鐵甲、弓矢、骨朵。骨朵直步隊一百二十四人,人員四、長行一百二十人。服甲同上,無弓矢。金吾牙門旗二十人,黃麾後第一部一百五十人,第二部一百二十人,殳叉仗五十二人,行止旗一。



 第七節。扇筤二十五人,金輅九十四人。大安輦一百八十一人:內尚輦奉御二人,殿中少監二人,奉職官二人,並公服。令史四人,書令史四人,七人烏介幘、緋四衣癸素衫、銀褐抹帶、大口褲、皁靴,一人長腳襆頭、紫羅公服、角帶皁靴。掌輦四人,武弁、黃繡寶相花衫、銀褐抹帶、大口褲。人員十二,皁帽、紅錦團襖、銅束帶,內指揮使一人執銀骨朵。舁士一百五十一人。服同掌輦。御馬三十三人。持鈒隊三十九人。後部鼓吹一百六十人。
 黃麾後第三至第五部皆一百二十人。後步甲隊第一至第二隊皆四十二人。殳叉五十六人。行止旗一。



 第八節。後部鼓吹第二,一百四十人。象輅、革車、木輅皆五十人,進賢車二十六人,豹尾車一十八人,屬車八十人。玄武隊六十一人。後步甲隊第三至第五隊皆四十二人。金吾牙門旗二十人。後部馬隊第一隊七十六人,第二隊六十四人,第三隊六十人。殳叉仗六十人。行止旗一。後分行旗、止旗為二。以上名數與黃麾同者不重述。



 章宗明昌五年六月,尚書省奏:「大定六年,世宗自西京還都,採宋省方還京之儀,用黃麾仗二千人、及金玉輅、
 栲栳隊甲騎五百人、導駕官四十二員,自後遂不復用。今車駕幸景明宮,還都之日宜依用之。」制可。承安元年,省臣奏:「南郊大禮,大駕鹵簿當用人二萬一千二百一十八、馬八千一百九十八。世宗親行郊祀,仗用七千人。今擬大定制外量添甲卒三百,栲栳隊、執楇人二百四十八,通七千五百四十八人,仍分八節。」從之。泰和六年,上欲親行祫享,命有司計其役費,尚書省奏:「當用仗三千五百人,錢一萬餘貫,馬八百六十五匹。舊例,馬皆借取於民,親軍、班祗皆自備從事。今軍旅方興,官馬以備緩急,不可借用,民亦不可重擾,宜令有司攝事。」上詔再
 議之。八年四月,禘于太廟,依元年例,用黃麾仗三千人,屯門仗五百人。



 ○皇太后皇后鹵簿



 用唐、宋制,共二千八百四十人。清旅隊三十人,清游旗一,執一人、引二人、夾二人。並平巾幘、緋裲襠、大口褲、佩弓矢、橫刀、執槊、弩、騎。次金吾衛折衝都尉一人,平巾幘、紫槊襠、大口褲、錦螣蛇、弓矢、橫刀。皞槊二人,平巾幘、緋衫、大口褲,夾折衝。領四十騎:二十人執槊、四人弩、十六人橫刀。並平巾幘、緋裲襠、大口褲、橫刀、弓矢。次虞候佽飛二十八人。並平巾幘、緋裲襠、大口褲、弓矢、橫刀,騎夾道,分左右均布至黃麾仗。次內僕令一人,丞一人,依本品服,分左右。各書令史二人。平巾幘、緋衫、大口褲,騎從。次黃麾一,執一人,夾二人。武弁、硃衣、革帶,正道騎。次左右廂黃麾仗,
 廂各三行,行百人,從內第一行,短戟、五色氅,執者並黃地白花綦襖、帽、行縢、鞋、襪。次外第二行,戈、五色氅,執人並赤地黃花綦襖、帽、行縢鞋、襪。次外第三行,儀鍠、五色幡。並青地赤花綦襖、帽、鞋、行縢、鞋、襪。次左右領軍衛、左右威衛、左右武衛、左右驍衛、左右衛等,衛各三行,行二十人,分前、後。衛各主帥六人,唯左右領軍衛各三人,並平巾幘、緋裲襠、大口褲,領軍衛前後獅子文袍、帽、餘衛豹文袍、帽,各執鍮石裝長刀,騎領,分前、後。每衛各果毅都尉一人檢校。被繡袍,以上各一名步從。左右領軍衛有絳引幡,引前、掩後各三。執者六人,並平巾幘、緋衫、大口褲。次內謁者監四人,給事二人,內常侍二人,內侍少監二人。並騎,分左右。以上各有內給使一人,步從。次內給使百二十人。皆宮人,並平巾幘、緋衫、大口褲,分左右,在車後。次偏扇、
 團扇、方扇各二十四。分左右,以宮人執之,皆服間彩大袖裙襦、彩衣、革帶、履。次香蹬一,執擎內給使四人。平巾幘、緋裲襠、大口褲,在重翟車前。次重翟車,馬四,駕士二十四人。平巾幘、青衫、大口褲、鞋襪。次行障二,坐障二。分左右夾車,宮人執之。服同執扇。次內寺伯二人,領寺人六。分左右,平巾幘、緋裲襠、大口褲、執御刀,並騎,夾重翟車。次腰輿一,輿士八人,團雉扇二。夾輿。次大傘四,次大雉扇八。分左右,橫行為二重。次錦華蓋二。單行,正道。次小雉扇、朱團扇各十二。並橫行,分左右。次錦曲蓋二十四。橫行,為二重。次錦六柱八扇。分左右。自腰輿以下並內給使執之,服同前。次宮人車。次絳麾二。分左右,執各一人,武弁、朱衣、革帶、鞋襪。次後黃麾一,執一人,夾二人。並騎,武弁、朱衣、革帶,正道。次供奉宮人。在黃麾後。次厭翟車,馬四,駕士二十
 四人。次翟車、安車皆四馬,駕士各二十四人。次四望車、金根車,皆駕牛三,駕士各十二人。服同前。次左右廂牙門各二,每門執二人,夾四人。並赤綦襖、黃袍、帽。第一門在前黃麾前,第二門在後黃麾後。次左右領軍衛,每廂各一百五十人,執殳,並赤地黃花綦襖、帽、行縢、鞋襪。前與黃麾仗齊,後盡鹵簿。廂各主帥四人,檢校。平巾幘、緋衫、大口褲、被黃袍帽,執鍮石長刀,騎。其服豹文者二在內,服獅文者二,一引前,一護後。次左右領軍衛、折衝都尉各一人,檢校殳仗。以上各一人騎從。次後殳仗內正道置牙門一,每門監門校尉二人,皆平巾幘、緋裲襠、大口褲,執銀裝長刀,騎。每廂各巡檢校尉一人、往來檢校。服仗同前。前後部鼓吹,金鉦、㧏鼓、大鼓、長鳴、中鳴、鐃吹、羽葆、鼓吹、橫吹、節鼓、御馬並減
 大駕之半。



 是歲,重翟等六車改用圓方輅輦,及行障、坐障、錦六柱、宮人等車,其制度人數並見《輿服志》。天德二年,海陵立后,皇后乘龍飾肩輿,有司設二步障於殿之西階,設扇左右各十,傘一,此蓋殿庭導引之儀也。又設皇太后導從六十人,傘子不在數內,並服簇四盤雕團花紅錦襖、金花襆頭、塗金銀束帶。永壽、永寧宮導駕各三十人,傘子各二人,此亦常行之儀也。



 ○皇太子鹵簿



 受冊寶謝廟,凡大禮、大朝會則用之。有司奏當用唐、宋儀禮,詔止用千人。中道。清游隊二十四人:折衝都尉一人,白澤旗一,五人,弩四、弓六、槊八。並騎。清道
 直盪隊一十八人:折衝都尉二人,皞槊四,弓矢十二。並騎。誕馬四,控攏八人。正直旗隊三十三人:果毅都尉一人,重輪旗一、馴犀旗二、野馬旗一,馴象旗二,旗各五人,副竿二。並騎。細引隊一十四人:果毅都尉二人,弓矢六,槊六。槊與弓矢相間,並騎。前部鼓吹九十八人:並騎。府史二人,金鉦、㧏鼓各二,大鼓十二,長鳴八,鐃鼓二,簫六,笳六、帥兵官二、節鼓二、小鼓十二、中鳴八、桃皮篳篥四、歌四、拱辰管六、篳篥六、大橫吹十二、羽葆鼓二、帥兵官二。傘扇八:梅紅傘二,大雉扇四,中雉扇二。小輿一十八人。導引官一十二人:中允二人,諭德二人、庶子二人、詹事二人、太師一人、太傅一人、太保
 一人,少師一人在金輅後。並騎。親勳翊衛圍子隊七十四人:郎將二人。儀刀七十二。並騎。金輅七十人。三衛隊一十八人。執儀刀。厭角隊六十二人:郎將一人,祥雲旗一,五人,弩三,弓七,稍十五,並騎。又郎將一人,祥雲旗一,五人,弩三,弓七,槊十五。並騎。朱團扇一十六人:司禦率府校尉四人,騎。朱團扇三,紫曲蓋三,硃團扇三,紫曲蓋三。大角一十八。後部鼓吹五十四人:並騎。管轄指揮一人,金鉦、㧏鼓各一,鐃鼓二,簫六,歌六,篳篥六,節鼓一,主帥二人,笛六,笳四,拱辰管六,小橫吹十,主帥二人。後拒隊四十六人:果毅都尉一人,騎。三角獸旗一,五人,弩四,弓矢十六,槊二十。外仗。左行二
 百四人。牙門十六人:並騎。牙門旗一,三人,監門校尉三人,郎將一人,班劍九。前第一隊二十七人:司禦率府一人,果毅都尉一人,折衝都尉一人,主帥一人,並騎。絳引幡三首,九人,麟頭竿二,儀鍠斧二,弓矢二,麟頭竿二,儀鍠斧二,朱刀盾二,小戟二。第二、第三、第四、第五隊各一十四人。與第一部麟頭竿已下同。後第一隊四十七人:牙門旗一,三人,監門校尉三人,果毅都尉一人,主帥一人,絳引幡三,九人,鶡雞旗一,五人,槊四,弩三,槊四,弓矢三,槊四,弓矢三,朱刀盾二,小戟二。並騎。後第二隊二十九人:果毅都尉一人,綱子旗一,五人,槊五,弩三,槊五,弓矢三,槊三,弓矢四。並
 騎。後第三隊二十九人:果毅都尉一人,黃鹿旗一,五人,槊五,弩三,槊五,弓矢三,槊三,弓矢四。並騎。右行二百四人,排列同。



 太子常行儀衛,導從六十二人,傘子二人,並服梅紅繡羅雙盤鳳襖、金花襆頭、塗金銀束帶。凡從物鏾鑼、唾盂、水罐等事並用銀金飾。傘用梅紅羅、坐麒麟金浮圖。椅用金鍍銀圈、雙戲麒麟椅背,紅絨絳結。殿庭與宴,衣敦用繡羅間金盤鳳,卓衣則用繡羅獨角間金盤獸。東宮視事,朱髹飾椅,塗金銀獸銜、紅絨絳結,明金團花椅背,案衣則用素羅,色皆梅紅,蒙帕踏腳同。



 ○親王傔從



 引接十人,皁衫、盤裹、束帶、乘馬。牽攏官五十
 人,首領紫羅襖、素襆頭,執銀裹牙杖,傘子紫羅團荅繡芙蓉襖、間金花交腳襆頭,餘人紫羅四衣癸繡芙蓉襖、兩邊黃絹義襴,並用金鍍銀束帶,襆頭同。邀喝四人。傘用青表紫裏,金鍍銀浮圖。椅用銀裹圈背。水罐、鏾鑼、唾盂並用銀。郡王牽攏官三十人,未出宮者二十人。國公牽攏官二十人,未出宮者十四人。郡王引接六人,國公四人,未出宮者各減半。人從儀物並依一品職事官制。



 ○諸妃嬪導從



 四十人,襆頭、繡盤蕉紫衫、塗金束帶。妃用偏扇、方扇、團扇各十六,諸嬪各十四,皆宮人執,服雲腳紗帽、紫四衣癸衫、束帶、綠靴。大傘各一,傘子二人,就用本
 服錦襖襆帶。大長公主導從一十二人,皇妹皇女一十人,並服紫羅繡胸背葵花夾襖、盤裹、襆頭、大佩銀腰帶,牙杖各二。其諸宗室女,各以親疏差降之。傘制,皇太子三位妃皆青羅表紫裏、金浮圖,親王公主王妃金鍍銀浮圖、郡主縣主夫人銀浮圖,皆青表紫裏,諸臣下母妻各從其夫子勛封品級用傘。



 ○百官儀從



 正一品:三師、三公、尚書令,朱衣直省各十人,三公稱直府。牽攏官各六十人,並服紫衫帽、銀偏帶,內執藤棒二對、骨朵三對,牙杖三對,簇馬六人,傘子二人。交椅、水罐、鏾鑼、盂子、唾碗等事以次執之,服皁衫帽、塗金銅
 束帶。後凡執色人並同。邀喝四人。傘用青羅紫裏、銀浮圖。從一品:尚書左右丞相、平章政事、都元帥、樞密使,直省同,樞密稱直院,以班祗人充。牽攏官五十人,邀喝四人。判大宗正,引接十人、牽攏官四十人。大興尹,面前兩對,餘並同。以上交椅並用銀裹圈背、紫絲滌結。正二品:東宮三師、左右副元帥。尚書左右丞,直省八人,牽攏官四十人,邀喝三人,傘用朱浮圖。從二品:參知政事、樞密副使、御史大夫,直省同,御史臺稱通引,以皞使班祗人充。牽攏三十六人,邀喝數同。正三品:東宮三少、元帥左右監軍、殿前都點檢、六部尚書、諸京留守、宣徽、勸農使、翰林學士承旨等官,凡同品
 者,各引接六人,牽攏官二十人。以上交椅並用直背銀間妝、青絲滌結。諸京都轉運使、招討使、諸路提刑使、諸府尹兼本路兵馬都總管及留守,牽攏官五十人。外任,統軍使、都運、招討使、副使、諸府尹兼總管,牽攏官五四十五人,公使七十人。從三品:元帥左右都監、勸農副使、殿前副都點檢及御史中丞等官,凡同品者,各引接六人,牽攏官一十八人,內中丞引從則給緋衫。外任,運使、節度使、牽攏官四十人,諸節鎮、諸部族節度同,公使上鎮七十人、中鎮六十五人、下鎮六十人。以上外任官人從服色,除諸招討、總管、部族節度、群牧使自來無射糧軍
 人力者並仍舊外,留守、統軍、總管、都運、招討、府尹、轉運、節度使人力亦仍舊,其數雖多,俱不得過四十人,並服紫衫、銀帶,銀裹圈背交椅、銀水罐、鏾鑼、盂、碗、牙杖,內銀裹骨朵、大劍各兩封,及邀喝,唯運使無骨朵、大劍。正四品:左右諫議大夫、國子祭酒、六部侍郎等官,凡同品者,各引接八人,本破十二人。外任,留守同統軍都監、提刑副使,各牽攏官三十人。從四品:殿前左右衛將軍、諸猛安千戶、親王府尉、諸京同知轉運等官,凡同品者,各引接四人,本破十二人。外任,牽攏官三十五人,公使上防禦六十人、中防禦五十五人、下防禦五十人。正五
 品:尚書左右司郎中、翰林待制、太常少卿等官,凡同品者,各本破八人。外任,牽攏官三十人,公使上州五十人、中州四十五人、下州四十人。凡防禦、刺史、知軍、并京府統軍司、節鎮佐貳官人從,並服紫衫、角束帶,直背銀交椅、鏾鑼、盂子、唾碗、牙杖,傘用青表碧裏青浮圖。防禦、刺史、知軍仍用銀裹骨朵、大劍一對,邀喝,唯隨路副統軍則不邀喝。從五品:六部郎中、侍御史、大理少卿等官,凡同品者,本破七人,侍御引從則給緋衫。外任,本破十人。以上職事官並許張蓋。正六品:尚書左右司員外等官,凡同品者,本破六人。外任,本破九人。從六品:尚書六
 部員外等官,凡同品者,本破五人。外任,本破九人。正七品:殿中侍御史等官,凡同品者,本破四人。外任,本破七人。縣令,公使十人。都軍,公使六人。從七品:應奉翰林文字等官,凡同品者,本破四人。外任,本破六人。縣令,公使十人。正八品:大理評事等官,凡同品者,本破二人。外任,本破六人。從八品:太常太祝等官,凡同品者,本破二人。外任,本破五人。正九品:御藥都監等官,凡同品者,本破一人。外任,本破三人。從九品:隨殿位承應、同監等官,凡同品者,本破一人。外任,本破一人。尚書省樞密院令譯史通事、六部御史臺及統軍司通事、誥院令
 史、國史院書寫等職,各設本破一人。以上職官,人力從物不得僭越。其外任官,人從服執,以本處公用或贓罰錢置。



 凡內外官自親王以下,傔從各有名數差等,而朱衣直省不與。其賤者,一曰引接亦曰引從,內宮從四品以上設之。二曰牽攏官,內外正五品以上設之。三曰本破,內外正四品以下設之。四曰公使,外官正三品以下設之。五曰從己人力,外官正三品京都留守、大興府尹以下等官設之。本破如牽攏之職,公使從公家之事,從己執私家之役者也。五等皆以射糧軍充,其軍非驗物力以事攻
 討,特招募民年十七以上、三十以下魁偉壯健者收刺,以資糧給之,故曰射糧。其首領則有將節、承局、什將等名,而皆統於隨路都兵馬總管府焉。金之所以禮臣下、足任使者,其亦先代之遣法歟?



 外任官從己人力,諸京留守、大興府尹,五十人。統軍、都轉運、招討、按察使,諸路兵馬都總管,四十五人。轉運、節度使,四十人。提控、諸群牧、防禦使,三十五人。外任親王傅、同知留守、副統軍、按察副使、諸州剌史知軍事,三十人。同知都轉運使事、副招討、副留守、同知府尹兼總管,提舉漕運司,諸五品鹽使,二十五人。都轉運副使、按察司簽事、少尹、副總管、
 同知轉運節度使事,二十人。京都兵馬都指揮使,一十八人。轉運節度副使,十七人。兵馬都鈐轄,十五人。親王府尉、諸京留守總判官、同知防禦使事,十三人。警巡使、兵馬副都指揮、同提舉漕運司,正六品,鹽副使,從六品,酒曲鹽稅使、同知州軍事,一十人。統軍都轉運司京府總管散府等判官、京推官,九人。親王府司馬、招討判官,赤劇縣令,提舉上京皇城兵馬鈐轄,正七品,酒曲鹽稅副使、都轉運判官、府推官、節度觀察判官,八人。京縣次劇縣令、都巡檢使、正將、府軍都指揮使,七人。司屬令、親王府文學、招討司勘事官、諸縣令、警巡副使、知城堡寨
 鎮,從七品、鹽判、同提舉上京皇城、節鎮軍都指揮使、都巡河、同七品酒使、防禦判官,六人。市令、錄事、赤劇縣丞、副都巡檢使、副將、都巡檢、州軍判官,五人。統軍司知事、親王府記室參軍,司屬丞,正八品,酒使副、京縣次劇縣丞、諸司使,四人。大興府招討、按察司知事、京府運司節鎮司獄、管勾河橋關度譏察官,從八品,鹽判官、漕運司勾當官、警巡判官、諸縣丞、市丞、司候、主簿、錄事司判官、縣尉、副都巡檢、諸巡檢、巡河官,正九品,酒使、諸司副使,三人。鹽場管勾、防刺以下司獄、部隊將、同管勾河橋、副譏察、司候判官、教授、統軍按察司知法、軍轄、諸司都監、
 節鎮以上知法,二人。鹽場同管勾、防刺以下知法、諸司同監、統軍按察司書史、統軍司譯書通事,一人。婆速公使、從己人力,於附近東京澄州招募漢人百姓投充。謂非猛安謀克所管者。合懶、恤品、胡里改、蒲與路並於各管猛安謀克所管上中戶內輸差驅丁,依射糧軍例支給錢糧,周年一易。部羅火、土魯渾扎石合亦同。其諸颭及群牧官員,若猛安謀克應差本管戶民充人力者,並上中戶輪當。



 諸內外官有兼職各應得人從者,從多給,餘各驗品類差。諸親王引接、引從,在都兵馬司差,公主隨朝者從守部本破內差,外路者并所在州府就差。諸
 王府引從、相府牽攏官、引接,周年替代,自餘十月滿代,並以射糧軍充。諸隨朝六品以下職官、並諸局承應者,願令從己輸庸者聽,仍具姓名申部,本處官司周年內不得占使。諸職官之任、以理去官者,接送人力於從己人內給半,取接者皆於所在官司出給印券差取,送還者須到本所給券發還,如無驗者權閣支請,候會問別無逃亡將帶,然後放支。諸致仕官職俱至三品者,從己人力於願往處給半,不得輸庸。身故應送還者又減半給之,若年未六十而致仕及罷去者,則不給。



\end{pinyinscope}