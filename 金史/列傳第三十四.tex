\article{列傳第三十四}

\begin{pinyinscope}

 ○黃久約李晏李仲略李愈王賁許安仁梁襄路伯達



 黃久約,字彌大,東平須城人也。曾祖孝綽有隱德,號「潛山先生」。父勝,通判濟州。母劉氏,尚書右丞長言之妹,一夕夢鼠銜明珠,寤而久約生,歲實在子也。擢進士第,調鄆城主簿,三遷曹州軍事判官。有盜竊民財,訴者以為強,郡守欲傅以重辟。久約閱實,囚得免死。累擢禮部員
 外郎,兼翰林修撰,升待制,授磁州刺史。磁並山,素多盜,既獲而款伏者,審錄官或不時至,繫者多以杖殺,或死獄中。久約惻然曰:「民雖為盜,而不死于法可乎?」乃盡請讞之而後行。



 久之,復入翰林為直學士,尋授左諫議大夫,兼禮部侍郎,為賀宋生日副使。至臨安,適館伴使病,宋人議欲以副使代行使事,久約曰:「設副使亦病,又將使都轄、掌儀輩行禮乎?」竟令國信使獨前行,副使與館伴副使聯騎如故,乃終禮而還。道經宿、泗,見貢新枇杷子者,州縣調民夫遞進,還奏罷之。



 時以貧富不均,或欲令富民分貸貧者,下有司議,久約曰:「物之不齊,物之情
 也。貧富不均,亦理之常。若從或者言,適足以斂怨,非損有餘補不足之道。」章宗時領右丞相,韙其議。尋上章請老,詔諭之曰:「卿忠直敢言,匡益甚多,未可使去左右。」遷太常卿,仍兼諫職。



 時郡縣多闕官,久約言:「世豈乏材,閡於資格故也。明詔每責大臣以守格法而滯人材,乞斷自宸衷而力行之。」世宗曰:「此事宰相不屬意,而使諫臣言之歟?」即日授刺史者數人。久約又言,宜令親王以下職官遞相推舉,世宗曰:「薦舉人材,惟宰相當為耳,他官品雖高,豈能皆有知人之監?方今縣令最闕,宜令刺史以上舉可為縣令者,朕將察其實能而用之。」又謂久約
 曰:「近日察舉好官,皆是諸科監臨,全無進士,何也?豈薦舉之法已有姦弊,不可久行乎?」久約曰:「諸科中豈無廉能人,不因察舉有終身不至縣令者,此法未可廢也。」上曰:「爾舉孫必福是乎?」久約曰:「臣頃任磁州時,必福為武安丞,臣見其廉潔向公,無所顧避,所以保舉。不謂必福既任警巡使,處決凝滯。」上曰:「必福非獨遲緩,亦全不解事,所以罪不及保官者,幸其無贓汙耳。」久約無以對。必福五經出身,蓋諸科人,故上問及之。翌日侍朝,故事,宰相奏事則近臣退避,久約欲趨出,世宗止之,自是諫臣不避,以為常。



 章宗即位,久約以國富民貧、本輕末重、任
 人太雜、吏權太重、官鹽價高、坊場害民、與夫選左右、擇守令八事為獻,皆嘉納之。再乞致仕,不許,授橫海軍節度使以優佚之。明昌二年致仕,卒。久約雋朗敢言,性友弟,為文典贍,有外祖之風云。



 李晏,字致美,澤州高平人。性警敏,倜儻尚氣。皇統六年,登經義進士第。調岳陽丞。再轉遼陽府推官,歷中牟令。會海陵方營汴京,運木於河,晏領之。晏以經三門之險,前後失敗者眾,乃馳白行臺,以其木散投之水,使工取於下流,人皆便之。丁內艱,服除,召補尚書省令史。辭去,為衛州防禦判官。世宗素識其才名,尋召為應奉翰林
 文字,特令詣閣謝,上顧謂左右曰:「李晏精神如舊。」慰勞甚悉。時方議郊禮,命攝太常博士,俄而真授。為高麗讀冊官,五遷祕書少監,兼尚書禮部郎中,除西京副留守。世宗謂侍臣曰:「翰林舊人少,新進士類不學,至於詔赦冊命之文鮮有能者,可選外任有文章士為之。」左右舉晏,上曰:「李晏朕所自識。」於是召為翰林直學士,兼太常少卿。以母老乞歸養,授鄭州防禦使,未赴,母卒。起復為翰林直學士。



 世宗御後閣,召晏讀新進士所對策,至「縣令闕員取之何道」,上曰:「朕夙夜思此,未知所出。」晏對曰:「臣伏念久矣,但無路不敢言。今幸待罪侍從,得承大問,
 願竭所知。」上曰:「然則何如?」對曰:「國朝設科取士,始分南北兩選,北選百人,南選百五十人,合二百五十人。詞賦經義入仕之人既多,所以縣令未嘗闕員。其後南北通選,止設詞賦一科,每舉限取六七十人。入仕之人既少,縣令闕員,蓋由此也。」上以為然,詔後取人毋限以數。尋擢吏部侍郎,兼前職,諭旨曰:「卿性果敢,有激揚之意,故以授卿,宜加審慎,毋涉荒唐。」俄為中都路推排使,遷翰林侍講學士,兼御史中丞。



 會朝士以病謁告,世宗意其詐,謂晏曰:「卿素剛正,今某詐病,以宰相親故,畏而不糾歟?」晏跪對曰:「臣雖老,平生所恃者,誠與直爾。百官病告,
 監察當視。臣為中丞,官吏姦私則當言之。病而在告,此小事臣容有不知,其畏宰相何圖焉。」既出,世宗目送之,曰:「晏年老,氣猶未衰。」一日,御史臺奏請增監察員,上曰:「採察內外官吏,固係監察。然爾等有所聞知,亦當彈劾。況糾正非違,臺官職也,茍不能正其身,如正人何?」顧謂晏曰:「豳王年少未練,朕以臺事委卿,當一一用意。」



 初,錦州龍宮寺,遼主撥賜戶民俾輸稅于寺,歲久皆以為奴,有欲訴者害之島中。晏乃具奏:「在律,僧不殺生,況人命乎!遼以良民為二稅戶,此不道之甚也,今幸遇聖朝,乞盡釋為良。」世宗納其言,於是獲免者六百餘人。故同判
 大睦親府事謀衍家有民質券,積其息不能償,因沒為奴,屢訴有司不能直,至是,投匭自言。事下御史臺,晏檢擿案狀得其情,遂奏免之。尋為賀宋正旦國信副使。及世宗不豫,命宿禁中,一時詔冊,皆晏為之。



 章宗立,晏畫十事以上。一曰風俗奢心慄,宜定制度。二曰禁游手。三曰宜停鑄錢。四曰免上戶管庫。五曰太平宜興禮樂。六曰量輕租稅。七曰減鹽價。八曰免監官陪納虧欠。九曰有司尚茍且,乞申明經久遠圖。十曰禁網差密,宜尚寬大。又奏「乞委待制黨懷英、修撰張行簡更直進讀陳言文字,以廣視聽」。皆採納之。以年老乞致仕,改禮部尚書,兼
 翰林學士承旨。越二年,復申前請,授沁南軍節度使,久之,致仕。上念其先朝舊人,復起為昭義軍節度使。明昌六年,歸老,得疾,詔除其子左司員外郎仲略為澤州刺史,以便侍養。承安二年卒,年七十五,謚曰文簡。



 仲略,字簡之。聰敏力學,登大定十九年詞賦進士第,調代州五臺主簿。以母憂去,服闋,轉韓州軍事判官,遷澤州晉城令,補尚書省令史。除翰林修撰,兼太常博士。改授左司都事,為立夏國王讀冊官。還,權領左司。一日,奏事退,上顧謂侍臣曰:「仲略精神明健,如俊鶻脫帽。」又曰:「李仲略健吏也。」未幾,轉員外郎,以親病求侍,特授澤州
 刺史以便祿養。先是,晏領沁南軍節度使,澤於懷為支郡,父子相繼,鄉人榮之。以父喪免,起為戶部郎中。



 時上命六品以上官,十日以次轉對,乃進言曰:「凡救其末,不若正其本。所謂本者厚風俗,去冗食,養財用而已。厚風俗在乎立制度,禁奢心慄。去冗食在乎寵力農,抑游墮。養財用在乎廣儲蓄,時斂散。商賈不通難得之貨,工匠不作無用之器,則下知重本。下知重本,則末息矣。」又條陳制度之宜,上嘉納之。俄授翰林直學士,兼前職,因命充經義讀卷官。上問曰:「有司以謂經義不若詞賦,罷之何如?」仲略奏曰:「經乃聖人之書,明經所以適用,非詞賦比。
 乞自今以經義進士為考試官,庶得碩學之士。」上可其奏。改吏部郎中,遷侍郎,兼翼王傅,俄兼宛王傅。



 時知大興府事紇石烈執中坐贓,上命仲略鞫之,罪當削解。權要競言太重,上頗然之,仲略奏曰:「教化之行,自近者始。京師,四方之則也。郡縣守令無慮數百,此而不懲,何以勵後?況執中兇殘很愎,慢上虐下,豈可宥之。」上曰:「卿言是也。」未幾,授山東東西路按察使。尋以病訪醫京師,泰和五年卒。上聞之,歎曰:「此人於國家宣力多矣,何遽止是耶!」贈朝列大夫,謚曰襄獻。



 仲略性豪邁有父風,剛介特立,不阿權貴,臨事明敏無留滯,故所任以幹濟稱云。



 李愈,字景韓,絳之正平人。業儒術,中正隆五年詞賦進士第,調河南澠池主簿。察廉優等為平陽酒副使,遷冀氏令,累遷解州刺史。章宗即位,召授同知中都路都轉運使事,改同知濟南府。明昌二年,授曹王傅,兼同知定武軍節度使事。王奉命宴賜北部,愈從行,還過京師,表言:「諸部所貢之馬,止可委招討司受於界上,量給迴賜,務省費以廣邊儲。擬自臨潢至西夏沿邊創設重鎮十數,仍選猛安謀克勛臣子孫有材力者使居其職,田給於軍者許募漢人佃種,不必遠挽牛頭粟而兵自富強矣。」上覽其奏,謂宰臣曰:「愈一書生耳,其用心之忠如是。」
 以表下尚書省議。會愈遷同知西京留守,過闕復上言,以為「前表儻可採,乞斷自宸衷」,上納用焉。自是,命五年一宴賜,人以為便。改棣州防禦使。未幾,授大興府治中,上諭之曰:「卿資歷應得三品,以是員方闕而卿能幹,故用之,當知朕意。」北京提刑副使范楫、知歸德府事鄧儼各舉愈以自代,由是擢河南路提刑使。上言:「隨路提刑司乞留官一員,餘分部巡按。」又言:「本司見置許州,乞移治南京為便。」並從之。憲臺廉察,九路提刑司以愈為最。



 五年,入見,尚書省以聞,上問宰執有何議論,平章政事守貞曰:「李愈言河決事。」上曰:「愈嚮陳備禦北邊策。言甚
 荒唐。」守貞曰:「愈於見職甚幹。」上曰:「蓋以其敢為耳。」又曰:「李愈論河決事,謂宜遣大臣視護以慰人心,其言良是。」明年,改河平軍節度使。承安二年,徙順義軍,奏陳屯田利害,上遣使宣諭,仍降金牌俾領其事。四年,召為刑部尚書。先是,刑部尚書闕,上以愈為可用,令議之。或言愈病,上曰:「愈比陳言,有退地千里而爭言其功之語,卿等定惡此人多言耶。」特召用之。舊制,陳言者漏所言事於人,並行科罪,仍給告人賞。愈言:「此蓋所以防閑小人也。比年以來詔求直言,及命朝臣轉對,又許外路官言事,此皆聖言樂聞忠讜之意,請除去舊條以廣言路。」上嘉
 納焉。尋為賀宋正旦副使。



 泰和二年春,上將幸長樂川,愈切諫曰:「方今戍卒貧弱,百姓騷然,三叉尤近北陲,恒防外患。兼聞泰和宮在兩山間,地形狹隘,雨潦遄集,固不若北宮池臺之勝,優游閑適也。」上不從,夏四月,愈復諫曰:「北部侵我舊疆千有餘里,不謀雪恥,復欲北幸,一旦有警,臣恐丞相襄、樞密副使闍母等不足恃也。況皇嗣未立,群心無定,豈可遠事逸游哉。」上異其言。未幾,授河平軍節度使,改知河中府事,致仕。泰和六年卒,年七十二。謚曰清獻。自著《狂愚集》二十卷。



 王賁,字文孺,其先自臨潢移貫宛平。曾祖士方,正直敢
 言。遼道宗信樞密使耶律乙辛之讒殺其太子,世無敢白其冤者,士方擊義鐘以訴,遼主感悟,卒誅乙辛,厚賞士方,授承奉官。父中安,擢進士第,坐田玨黨事廢。世宗即位黨禁解,終沂州防禦使。



 賁性孝友,勤敏好學,第進士,由復州軍事判官補尚書省令史,擢右三部檢法司正。待御史賈鉉舉賁安靜有守,不尚奔競,政府亦言其廉素,善論議。擢河北東西、大名府路提刑判官,選授尚書省都事,以喪去。用薦者多,起復刑部員外郎、侍御史,累遷南京路按察使,卒。賁敦厚尚義,篤於親朋,不營產業,比歿,家甚窶,上聞憫惜之,贈朝列大夫,仍厚恤共家。



 弟質,字敬叔,登大定二十五年進士第,累官吏部主事,以才幹舉遷昭義軍節度副使。章宗問質臨事若何,張萬公對曰:「勝其兄賁。」章宗曰:「及其兄亦可矣。」後以禮部尚書致仕,終。



 許安仁,字子靜,獻州交河人。幼孤,能自刻苦讀書,善屬文。登大定七年進士第,調河間縣主簿。累遷太常博士,兼國史院編修官。章宗為皇太孫,安仁以講學被選東宮,轉左補闕、應奉翰林文字。上即位,改國子監丞,兼補闕,徙翰林修撰,同知制誥,兼職如故。侍御史賈鉉以安仁守道端愨,薦于朝。同知濟南府事路伯達繼上章稱
 其立己純正,宜加顯任,超授禮部郎中,兼左補闕。適朝議以流人實邊,安仁言:「昔漢有募民實邊之議,蓋度地營邑,制為田宅,使至者有所居,作者有所用,於是輕去故鄉而易於遷徙。如使被刑之徒寒餓困苦,無聊之心,靡所顧藉,與古之募民實塞不同,非所宜行。」上然之。明昌四年春,上將幸景明宮,安仁與同列諫曰:「昔漢、唐雖有甘泉、九成避暑之行,然皆去京師不遠。非如金蓮千里之外,鄰沙漠,隔關嶺,萬一有警,何以應變,此不可不慮也。」疏奏,遂罷幸。出為澤州刺史,作《無隱論》上之,凡十篇,曰本朝、曰情欲、曰養心、曰田獵、曰公道、曰養源、曰冗
 官、曰育材、曰限田、曰理財。在郡二年,徙同知河南府事,升汾陽軍節度使,致仕。泰和五年卒,年七十七,謚曰文簡。安仁質實無華,澹然有古君子風,故為時人所稱云。



 梁襄,字公贊,絳州人。少孤,養於叔父寧。性穎悟,日記千餘言。登大定三年進士第,調耀州同官主簿。三遷邠州淳化令,有善政。察廉,升慶陽府推官,召為薛王府掾。世宗將幸金蓮川,有司具辦,襄上疏極諫曰:



 金蓮川在重山之北,地積陰冷,五穀不殖,郡縣難建,蓋自古極邊荒棄之壤也。氣候殊異,中夏降霜,一日之間,寒暑交至,特與上京、中都不同,尤非聖躬將攝之所。凡奉養之具無
 不遠勞飛挽,越山踰險,其費數倍。至於頓舍之處,軍騎闐塞,主客不分,馬牛風逸以難收,臧獲逋逃而莫得,奪攘蹂躪,未易禁止。公卿百官衛士,富者車帳僅容,貧者穴居露處,輿臺皂隸,不免困踣,飢不得食,寒不得衣,一夫致疾,染及眾人,夭傷無辜,何異刃殺。此特細故耳,更有大於此者。



 臣聞高城峻池,深居邃禁,帝王之籓籬也,壯士健馬,堅甲利兵,帝王之爪牙也。今行宮之所,非有高殿廣宇城池之固,是廢其籓籬也。持甲常坐之馬,日暴雨蝕,臣知其必羸瘠矣。禦侮待用之軍,穴居野處,冷啖寒眼,臣知其必疲瘵矣。衛宮周廬才容數人,一旦霖
 潦積旬,衣甲弓刀霑濕柔脆,豈堪為用,是失其爪牙也。秋杪將歸,人已疲矣,馬已弱矣,裹糧已空,褚衣已弊,猶且遠幸松林,以從畋獵,行於不測之地,往來之間,動踰旬月,轉輸移徙之勞,更倍於前矣。



 以陛下神武善騎射,舉世莫及,若夫銜橛之變,猛摯之虞,姑置勿論。設於行獵之際,烈風暴至,麈埃漲天,宿霧四塞,跬步不辨,以致翠華有崤陵之避、襄城之迷,百官狼狽於道途,衛士參錯於隊伍,當此宸衷寧無戒悔。夫神龍不可以失所,人主不可以輕行,良謂此也。所次之宮,草略尤甚,殿宇周垣,唯用氈布。押宿之官、上番之士,終日驅馳,加之飢渴,
 已不勝倦。更使徹曙巡警,露坐不眠,精神有限,何以克堪。雖陛下悅以使人,勞而不怨,豈若不勞之為愈也。故君人者不可恃人無異謀,要在處己於無憂患之域也。



 燕都地處雄要,北倚山險,南壓區夏,若坐堂隍,俯視庭宇,本地所生,人馬勇勁,亡遼雖小,止以得燕故能控制南北,坐致宋幣。燕蓋京都之選首也。況今又有宮闕井邑之繁麗,倉府武庫之充實,百官家屬皆處其內,非同曩日之陪京也。居庸、古北、松亭、榆林等關,東西千里,山峻相連,近在都畿,易於據守,皇天本以限中外,開大金萬世之基而設也。奈何無事之日,越居草萊,輕不貲之
 聖躬,愛沙磧之微涼,忽祖宗之大業,此臣所惜也。又行幸所過,山徑阻修,林谷晻靄,上有縣崖,下多深壑,垂堂之戒,不可不思。



 臣聞漢、唐離宮,去長安纔百許里,然武帝幸甘泉,遂中江充之姦,太宗居九成,幾致結社之變。太康畋於洛汭,后羿拒河而失邦;魏帝拜陵近郊,司馬懿竊權而篡國。隋煬、海陵,雖惡德貫盈,人誰敢議?止以離棄宮闕,遠事巡征,其禍遂速,皆可為殷鑒也。臣嘗論之:安民濟眾,唐、虞猶難之。而今日之民,賴陛下之英武,無兵革之憂,賴陛下之聖明,無官吏之虐,賴陛下之寬仁,無刑罰之枉,賴陛下之節儉,無賦斂之繁,可謂能安
 濟矣。而遊畋納涼之樂,出於富貴之餘,靜而思動,非如衣食切身有不可去者,罷之至易耳。唐太宗將行關南,畏魏徵而停,漢文帝欲馳霸陵,袁盎諫而遽止。是陛下能行唐、虞之難行,而未能罷中主之易罷,臣所未諭也。



 且燕京之涼,非濟南之比,陛下牧濟南日,每遇炎蒸,不離府署,今九重之內,臺榭高明,宴安穆清,何暑得到。議者謂陛下北幸久矣,每歲隨駕大小,前歌後舞而歸,今茲再出,寧有遽不可乎。臣愚以為患生於不戒者多矣,西漢崇用外戚,而有王莽之禍,梁武好納叛降,而有侯景之變。今者累歲北幸,狃於無虞,往而不止,臣甚懼焉。
 夫事知其不可猶冒為之,則有後難必矣。



 議者又謂往年遼國之君,春水秋山,冬夏捺缽,舊人猶喜談之,以為真得快樂之趣,陛下效之耳。臣愚以謂三代之政今有不可行者,況遼之過舉哉。且本朝與遼室異,遼之基業根本,在山北之臨潢,臣知其所游,不過臨潢之旁,亦無重山之隔,冬猶處於燕京。契丹之人,以逐水草牧畜為業,穹廬為居,遷徙無常,又壤地褊小,儀物殊簡,輜重不多,然隔三五歲方能一行,非歲歲皆如此也。我本朝皇業,根本在山南之燕,豈可捨燕而之山北乎?上京之人,棟宇是居,不便遷徙。方今幅員萬里,惟奉一君,承平日
 久,制度殊異,文物增廣,輜重浩穰,隨駕生聚,殆逾於百萬。如何歲歲而行,以一身之樂,歲使百萬之人困於役、傷於財、不得其所,陛下其忍之歟?臣又聞,陛下於合圍之際,麋鹿充牣圍中,大而壯者,纔取數十以奉宗廟,餘皆縱之,不欲多殺。是陛下恩及於禽獸,而未及於隨駕眾多之臣庶也。



 議者謂,前世守文之主,生長深宮。畏見風日,彎弧上馬,皆所不能,志氣銷懦,筋力拘柔,臨難戰懼,束手就亡。陛下監其如此,不憚勤身,遠幸金蓮,至於松漠,名為坐夏打圍,實欲服勞講武。臣愚以為戰不可忘,畋獵不可廢,宴安鴆毒亦不可懷,然事貴適中,不可
 過當。今過防驕惰之患,先蹈萬有一危之途,何異無病而服藥也。況欲習武不必度關,涿、易、雄、保、順、薊之境地廣又平,且在邦域之中,獵田以時,誰曰不可?伏乞陛下發如綸之旨,回北轅之車,塞雞鳴之路,安處中都,不復北幸,則宗社無疆之休,天下莫大之願也。



 方今海內安治,朝廷尊嚴,聖人作事,固臣下將順之時,而臣以螻蟻之命,進危切之言,仰犯雷霆之威,陷於吏議,小則名位削除,大則身首分磔,其為身計,豈不愚謬。惟陛下深思博慮,不以人廢言,以宗廟天下為心,俯垂聽納,則小臣素願遂獲,雖死猶生,他非所覬望也。



 世宗納之,遂為罷
 行,仍諭輔臣曰:「梁襄諫朕毋幸金蓮川,朕以其言可取,故罷其行。然襄至謂隋煬帝以巡游敗國,不亦過乎。如煬帝者蓋由失道虐民,自取滅亡。民心既叛,雖不巡幸,國將安保?為人上者,但能盡君道,則雖時或巡幸,庸何傷乎?治亂無常,顧所行何如耳。豈必深處九重便謂無虞,巡游以時即兆禍亂者哉!」



 襄由是以直聲聞。擢禮部主事、太子司經。選為監察御史,坐失察宗室弈事,罰俸一月。世宗責之曰:「監察,人君耳目,風聲彈事可也。至朕親發其事,何以監察為?」轉中都路都轉運戶籍判官,未幾,遷通遠軍節度副使,以喪去。服闋,授安國軍節度副
 使,同知定武軍節度事,避父諱改震武軍。太常卿張暐、曹州刺史段鐸薦襄學問該博,練習典故,可任禮官。轉同知順義軍節度使事、東勝州刺史。坐簸揚俸粟責倉典使償,為按察司所劾,以贖論。歷隩州刺史,累遷保大軍節度使,卒。



 襄長於《春秋左氏傳》,至于地理、氏族,無不該貫。自蚤達至晚貴,膳服常淡薄,然議者譏其太儉云。



 贊曰:金起東海,始立國即設科取士,蓋亦知有文治也。漸摩培養,至大定間人材輩出,文義蔚然。加以世宗之聽納,人各盡其所能,論議書疏有可傳者。惜史無全文,僅存梁襄《諫北幸》一書,辭雖過繁而意亦切至,故備載
 之,以見當時君明臣直,不以言為忌。金之致治於斯為盛,嗚呼休哉。



 路伯達,字仲顯,冀州人也。性沉厚,有遠識,博學能詩,登正隆五年進士第,調諸城主簿。由泗州榷場使補尚書省掾,除興平軍節度副使,入為大理司直。大定二十四年,世宗將幸上京,伯達上書諫曰:「人君以四海為家,豈獨舊邦是思,空京師而事遠巡,非重慎之道也。」書奏,不報。閱歲,改秘書郎,兼太子司經。時章宗初嚮學,伯達以文行知名,選為侍讀,居無何以憂去。會安武軍節度使王克溫舉伯達行義,起為同知西京路轉運使事,召為尚
 書禮部員外郎,兼翰林修撰,敕與張行簡進讀陳言文字。



 先是,右丞相襄奏移賀天壽節於九月一日,伯達論列以其非時,平章政事張汝霖、右丞劉瑋及臺諫亦皆言其不可,下尚書省議,伯達曰:「上始即政,當行正、信之道,今易生辰非正,以紿四方非信。且賀非其時,是輕禮重物也。」因陳正名從諫之道。升尚書刑部郎中。上問群臣曰:「方今何道使民務本業、廣儲蓄?」伯達對曰:「布德流化,必自近始。請罷畿內採獵之禁,廣農郊以示敦本,輕幣重穀,去奢長儉,遵月令開籍田以率先天下,如是而農不勸、粟不廣者未之有也。」是時,採捕禁嚴,自京畿至
 真定、滄、冀,北及飛狐,數百里內皆為禁地,民有盜殺狐兔者有罪,故伯達及之。累遷刑部侍郎、太常卿,拜安國軍節度使,未幾,改鎮安武。



 嘗使宋回,獻所得金二百五十兩、銀一千兩以助邊,表乞致仕,未及上而卒。其妻傅氏言之,上嘉其誠,贈太中大夫,仍以金銀還之,傅泣請,弗許。傅以伯達嘗修冀州學,乃市信都、棗強田以贍學,有司具以聞,上賢之,賜號成德夫人。



 子鐸、鈞。鈞字和叔,登大定二十五年進士第,終萊州觀察判官。鐸最知名,別有傳。



 贊曰:金詘宋稱臣稱姪,受其歲幣,禮也。使聘於其國,燕
 享禮也,納其重賂其可乎哉?時人貪利忘禮,習以為常,莫有知其為非者。故去則云酬勞效,還則戶增物力,上下交征,惟利是事,此何誼耶?伯達獨能明其非禮,回獻所饋,齎志未畢,傅氏又能成之,及歸所獻,竟以買田贍學。婦人秉心之烈、制事之宜,乃能如是,士大夫溺於世俗之見者寧不愧哉。賜號成德,不亦宜乎。



\end{pinyinscope}