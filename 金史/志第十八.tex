\article{志第十八}

\begin{pinyinscope}

 禮十



 ○冊皇后儀



 天德二年十月九日,冊妃徒單氏為皇后。前一日,儀鸞司設座勤政殿,南向。設群臣次於朝堂。大樂令展宮縣於殿庭,設協律郎舉麾位於樂縣西北,東向。閣門設百官班位於庭,並如常朝之儀。又設典儀位於班位之東北,贊者二人在南少卻,俱西向。設冊使副位
 於殿門外之東,又設冊使副受命位於百官班前。又設冊寶幄次二於殿後東廂,俱南向。



 其日,諸衛勒所部,略列黃麾細仗於庭。符寶郎奉八寶置於左右。吏部侍郎奉冊,禮部侍郎奉寶匣,皆置於床,訖,出就門外班。大樂令、協律郎、樂工、典儀。贊者各入就位。群官等依時刻集朝堂,俱就次,各服朝服。侍中約刻板奏:「請中嚴。」通事舍人引群官入,就庭東西相向立,以北為上。又引冊使副立於東偏門,西向。門下侍郎引主節,奉節立於殿下東廊橫街北。中書令、中書侍郎帥舉捧冊官,奉冊床立於節南。侍中、門下侍郎帥舉捧寶官,奉寶床立於冊床之
 南,俱西面。侍中版奏:「外辦。」殿上索扇。協律郎舉麾,宮縣作。皇帝服通天冠、絳紗袍,出自東房,曲直華蓋、警蹕侍衛如常儀。即座,南向坐,簾捲,樂止。通事舍人引冊使副入,宮縣作。使副就受命位,侍中、中書令、門下侍郎、中書侍郎、舉捧官依舊西面立,群臣合班,橫行北面,如常朝之儀,立定。典儀曰:「再拜。」贊者承傳,班首已下群官在位者皆再拜。班首問起居,又再拜。閣門官引攝侍中出班承制,降詣使副東北,西向稱:「有制。」使副稍前,鞠躬再拜,攝侍中宣制曰:「命公等持節授后冊寶。」宣制訖,又俱再拜,侍中還班。門下侍郎引主節詣冊使所,主節以節授
 門下侍郎、門下侍郎執節西向授太尉,太尉受付主節,主節立於使副之左右。門下侍郎退還班位。中書侍郎引冊床,門下侍郎引寶床,立於冊使東北,西向,以次授與太尉,太尉皆捧受。冊床置於北,寶床置於南。侍中、中書令、禮儀使、舉捧冊寶官及舁床者,退於東西磚道之左右,相向立。門下侍郎、中書侍退還班位。典儀曰:「再拜。」贊者承傳,群官在位者皆再拜,訖,分班東西相向立。捧舉舁冊寶床者進,冊床先行,讀冊官次之,寶床次行,讀寶官次之。舉舁官各分左右,通事舍人引冊使隨之以行,持節者前導。太尉初行,宮縣樂作,出殿門,樂止。攝
 侍中出班升殿奏:「侍中臣言禮畢。」殿上索扇,簾降,宮縣作。降座,入自東房,樂止。通事舍人引群官在位者以次出。俟太尉、司徒復命,禮畢,還內。



 先是,有司預設太尉、司徒本品革車鹵簿於門外至殿門左右排列。俟使副出,鼓吹振作。禮儀使、舉捧官、執節者并抬舁人,以冊寶少駐於泰和門,太尉、司徒及讀冊寶官暫歸幕次。內侍閣門引入泰和殿,俟至殿下位,鼓吹止。有司預供張,泰和殿設皇后座於扆前,殿上垂簾。又設東西房於座之左右稍北。又設受冊位於殿庭西階之南,東向。又設內命婦次於殿之左右。大樂令設宮縣於庭,協律郎設舉麾
 位于殿上。又設冊寶次於門外。又設行事官次於門左右。又設外命婦次於門之內。其日,諸衛於殿門外略設黃麾細仗。有司設二步障於殿之西階。簾前設扇,左右各十。紅傘一,在西階欄子外。又設舉冊寶案位於使副之前,北向。又設宣徽使位於北廂,南向。司贊設內外命婦以下陪列位於殿庭磚道之左右,每等重行異位北向,內命婦在後。又設司贊位於東階東南,贊者二人在南少退,俱西向。質明,執事官大樂令等各就位。皇后常服,乘龍飾肩輿,至泰和殿後閣,近仗導衛如常儀。宣徽使奏:「中嚴。」冊使副入門,宮縣作,俟冊使庭中立,樂止。冊
 在北,寶在南,使副立於床後。禮儀使帥持節者立於前,舉捧冊寶官立於冊寶床左右,讀冊寶官各立於其後。宣徽使奏:「外辦。」內侍閣門官引后出後閣,宮縣作。簾捲,皇后降自西階,左右步障傘扇從,至階下,望勤政殿御閣所在立,樂止。冊使進,立於右,宣曰:「有制。」閣門使內侍贊:「再拜。」冊使宣曰:「制遣太尉臣某、司徒臣某,恭授后冊寶。」閣門使內侍贊:「再拜。」冊使少退。中書令、侍中及舉捧官率抬舁人奉冊寶以次進於前,宮縣作。冊寶床自東階升,並置於殿之前楹間,冊床在北,寶床在南,中留讀冊寶官立位,並去帕及蓋,抬舁人執之,退立於西朵殿。
 舉抬官分左右相向立,讀冊寶官各立於床之東,西向,立既定,樂止。閣門使內侍贊:「再拜。」捧謝表官以表授左立內侍,內侍以授后,受訖,以付右立內侍,內侍持表立於右。閣門使贊:「再拜。」訖,冊使退,宮縣作。持表內侍以表付閣門官,隨冊使行。冊使副至門,鼓吹振作如來儀,入西偏門,鼓吹止。冊使副至御閣所在,俯伏,跪奏:「太尉臣某、司徒臣某,奉制授冊寶,禮畢。」俯伏,興,退。持表閣門官進表,近侍接入,進讀,訖,退。



 初,冊使退,及門樂止。閣門內侍引后自西階升殿,宮縣作。傘扇止於簾外,退於左右朵殿前。步障止於階下,卷之。后於座前南向立,樂止。中
 書令詣冊床南立,北向,稱:「中書令臣某,謹讀冊。」讀畢,降自東階,立於欄外第一墀上,西向。次侍中詣寶床南立,北向,揖稱:「侍中臣某,讀寶。」讀畢降階,立於中書令之北,西向。內侍閣門引升座,宮縣作,坐定,樂止。舉捧官以次招抬舁人持帕蓋覆匣床,奉置殿之左右,冊床在東,寶床在西。置訖,舉捧官以次招抬降階,立於中書令、侍中之後,立定,合班北向,閣門贊:「再拜。」拜訖,降東階,退出殿門。其抬舁人置冊寶床於東西訖,各由朵殿下階,於侍中等班後直出殿門,以俟復入,抬舁入宮。受冊表謝訖,內侍跪奏:「禮畢。」閣門引內外命婦陪列者以次進,就北向位。
 班首初行,宮縣作,至位樂止。閣門曰:「再拜。」命婦皆再拜。閣門引班首自西階升,樂作,至階樂止,進當座前,北向躬致稱賀,訖,降自西階,樂作,至位樂止。閣門曰:「再拜。」舍人承傳,命婦等皆再拜。閣門使前承令,降自西階,詣命婦前西北,東向,稱:「有教旨。」命婦等皆拜,閣門使宣曰:「祗奉聖恩,授以冊寶,榮幸之至,競厲增深。所賀知。」舍人曰:「再拜。」命婦皆再拜,訖,內侍引內命婦還宮。班首初行,樂作,出門,樂止。內侍引外命婦出次。宣徽使奏稱:「禮畢。」降座,宮縣作,入東房,樂止。歸閣,宮縣作,至閣,樂止。更常服。內侍承教旨,宣外命婦入會,並如常儀。會畢,閣門引命外
 婦降階,橫班北向,舍人曰:「再拜。」訖,以次出。還宮,如來儀。中書門下侍郎復以引進司帥抬舁人進冊寶入內,付與都點檢司,退。別日,會群官,會妃主宗室等,賜酒,設食,簪花,教坊作樂,如內宴之儀。十一日,朝永壽、永寧兩宮。皇后既受冊,越二日,內侍設座於所御殿,南向。其日夙興,宣徽使版奏:「中嚴。」質明,諸侍衛宮人俱詣寢殿奉迎,宣徽使版奏:「外辦。」后首飾禕衣御車,內侍前導,降自西階以出,侍衛如常儀。至太后之裹門外,降車,障扇侍衛如常儀,入立於西廂,東向。將至,宣徽使版奏:「請中嚴。」既降車,宣徽使版奏:「外辦。」太后常服,宣徽使引升座,南
 向。宣徽使引后進,升自西階,北面再拜,進跪致謝詞。存撫賜酒食,並如家人之儀。禮畢,宣徽使贊:「再拜。」訖。宣徽使引降自西階以出。出門,宣徽使奏:「禮畢。」降座入宮。



 ○奉冊皇太后儀



 天德二年正月,詔有司:「擇日奉冊唐殷國妃、岐國太妃,仍別建宮名。合行典禮,禮官檢詳條具以聞。」其日質明,有司各具傘扇,侍衛如儀,及兵部約量差軍兵,并文武百官詣兩宮迎請,引導皇太后入內,並赴受冊殿,入御幄,侍衛如式。次奉冊太尉等俱以冊置於案,奉寶司徒等俱以寶置於案,皆盛以匣,覆以帕,詣別殿門外幄次。教坊提點率教坊入。侍衛官各就列。皇
 帝常服乘輿,至別殿後幄次。通事舍人引宣徽使版奏:「中嚴。」復位,少頃,又奏:「外辦。」幄簾捲,教坊樂作,扇合,兩宮皇太后出自後幄,並即御座,南向,扇開,樂止。分左右少退。通事舍人引文武百僚班左入,依品,重行西向,立定。通事舍人喝:「起居。」班依常朝例起居,七拜,訖,引文武百僚班分東西相向立。通事舍人、太常博士贊引,太常卿前導,押冊官押冊而行,奉冊太尉、讀冊中書令、舉冊官等以次從之。次押寶官押寶而行,奉寶司徒、讀寶侍中、舉寶官等以次從之。俱自正門入,教坊樂作,至殿庭西階下少東,北向,於褥位少置,樂止。冊北,寶南。通事舍人、太常
 博士贊引,太常卿前導,押冊官押冊升,樂作,奉冊太尉等從之,進至兩宮皇太后座前褥位,樂止。兩宮冊寶齊上,齊讀。舉冊官夾侍。奉冊太尉各搢笏,北向跪,俯伏,興,退立。讀冊中書令俱進,向冊前跪奏稱:「攝中書令具官臣某,謹讀冊。」舉冊官單跪對舉,中書令各搢笏,讀訖,執笏,俯伏,興,搢笏,捧冊興,於位東迴冊函北向,並進,跪置於御座前褥位。中書令舉冊官俱降,還位。奉冊太尉並降階,東向以俟。押寶官押寶升,樂作,奉寶司徒等從之,進至兩宮皇太后座前褥位,樂止。舉寶官夾侍。奉寶司徒各搢笏,北向跪,俯伏,興,退立。讀寶侍中俱進,當寶前跪奏稱:「攝侍中
 具官臣某,謹讀寶。」舉寶官單跪對舉,侍中各搢笏,讀訖,執笏,俯伏,興,搢笏,捧寶興,於位東迴寶函北向,並進,跪置於御座前褥位冊之南。通事舍人、太常博士贊引太尉、司徒以次應行事官俱降自西階,復本班序立。宣徽使一員詣皇帝御幄前,俯伏,跪奏:「臣某謹請皇帝詣兩宮皇太后前,行稱賀之禮。」俯伏,興。贊引皇帝再拜,又奏:「請北向跪。」皇帝賀曰:「嗣皇帝臣某言云云。」俯伏,興,又再拜,訖,又奏:「請皇帝少立。」內侍承旨退,西向稱:「兩宮皇太后旨云云。」皇帝再拜。宣徽使前引,皇帝歸幄,常服乘輿還內,侍衛如來儀。



 應階下文武百僚重行立定,通事舍人喝:「
 拜。」在位皆再拜。通事舍人引太師詣西階升,俯伏,跪奏稱:「文武百僚具官臣某等稽首言,皇太后殿下顯對冊儀,永安帝養。仰祈福壽,與天同休。」俯伏,興,降自西階,復位立定。通事舍人贊:「在位官皆再拜。」舞蹈,三稱「萬歲」,又再拜。宣徽使升自東階,取旨退,臨階西向稱:「兩宮皇太后旨。」通事舍人贊:「在位官皆再拜。」畢,宣曰:「公等忠敬盡心,推崇協力。膺茲令典,感愧良深。」宣訖,還位。通事舍人贊:「謝宣諭,拜。」在位官皆再拜,舞蹈,三稱「萬歲」,又再拜。通事舍人分引應北向官各分班東西立。宣徽使升自東階,奏稱:「具官臣等言,禮畢。」降還位。扇合,皇太后並興,教
 坊樂作,降座,還殿後幄次,扇開,樂止。通事舍人引宣徽使奏:「解嚴。」中書侍郎等各帥捧冊床官升殿,跪捧冊並置於床,次門下侍郎等各帥捧寶床官升殿,跪捧寶並置於床,訖,通事舍人引詣東上閣門,投進所司。文武百僚以次出。皇太后常服乘輿,各還本宮,引導如來儀。文武百僚詣東上閣門拜表賀皇帝,退。禮畢,各赴本宮,受內外命婦稱賀。所司預於殿內設皇太后御座,司賓引內外命婦於殿庭北向依序立。尚儀奏請,皇太后常服即座。司贊曰:「再拜。」命婦皆再拜。司賓引班首詣西階升,跪賀稱:「妾某氏等言,伏惟皇太后殿下,天資聖善,昭受
 鴻名,凡在照臨,不勝欣抃。」興,降階復位。司贊曰:「再拜。」內外命婦皆再拜,尚宮承旨,降自西階,於命婦之北東向立,司贊曰:「再拜。」在位者皆再拜,尚宮乃宣答曰:「膺茲典禮,感愧良深。」司贊曰:「再拜。」在位者皆再拜,退。赴別殿賀皇帝,亦如賀皇太后之儀,惟不致詞,不宣答。



 ○冊皇太子儀



 大定八年正月,冊皇太子,禮官擬奏,皇太子乘輿至翔龍門,東宮官導從,不乘馬。冊皇太子前三日,遣使同日奏告天地宗廟。冊前一日,宣徽院帥儀鸞司,設御座於大安殿當中,南向。設皇太子次於門外之東,西向。又設文武百僚應行事官,東宮官等次於門外
 之東、西廊。又設冊寶幄次於殿後東廂,俱南向。又設受冊位於殿庭橫階之南。工部官與監造冊寶官公服,自製造所導引冊寶床,由宣華門入,約宣徽院同進呈畢,赴幄次安置。大樂令帥其屬,展樂縣於庭。



 其日,兵部帥其屬,設黃麾仗於大安殿門之內外。其日質明,文武百僚應行事官並朝服入次。東宮官各朝服,自東宮乘馬導從,至左翔龍門外下馬,入就次。通事舍人分引百官入立班,東西相向。次引侍中、中書令、門下侍郎、中書侍郎及捧舁冊寶官,詣殿後幄次前立。少頃,奉冊寶出幄次,由大安殿東降,至庭中褥位,權置訖,奉引冊寶官立
 於其後。皇太子服遠遊冠、朱明衣出次,執圭,三師三少已下導從,立於門外。侍中奏:「中嚴。」符寶郎奉八寶由東西偏門分入,升置御座之左右。侍中奏:「外辦。」內侍承旨索扇,扇合,皇帝服通天冠、絳紗袍以出,曲直華蓋侍衛如常儀,鳴鞭,宮縣樂作。皇帝出自東序,即御座,爐煙升,扇開簾捲,樂止。典贊儀引皇太子入門,宮縣樂作,至位樂止。師、少已下從入,立於皇太子位東南,西向。典儀贊:「皇太子再拜。」搢圭,舞蹈,又再拜,奏:「聖躬萬福。」又再拜,引近東,西向立。師、少已下并奉引冊寶官等,各赴百官東班,樂作,至位樂止。通事舍人引百官俱橫班北向。典儀
 贊:「拜。」在位官皆再拜,搢笏,舞蹈,又再拜,起居,又再拜,畢,百官各還東西班。師、少已下并行事官各還立位。典贊儀引皇太子復受冊位,樂作,至位樂止。侍中承旨,稱:「有制。」皇太子已下應在位官皆再拜,躬身,侍中宣制曰:「冊某王為皇太子。」又再拜。通事舍人、太常博士引中書令詣讀冊位,中書侍郎引冊匣置於前,捧冊官西向跪捧,皇太子跪,讀畢,俯伏,興。皇太子再拜。中書令詣捧冊位,奉冊授皇太子,搢圭,跪受冊,以授右庶子,右庶子跪受,皇太子俯伏,興,右庶子以冊,興,置於床,中書令已下退復本班。次通事舍人、太常博士引侍中詣奉寶位,門下
 侍郎引寶盝立於其右,侍中奉寶授皇太子,搢圭,跪受,以授左庶子,左庶子跪受,皇太子俯伏,興,左庶子以寶興,置於床,侍中已下退復本班。典儀贊:「再拜。」畢,引皇太子退。初行,樂作,左右庶子帥其屬,舁冊寶床匣以出,出門,樂止。侍中奏:「禮畢。」內侍承旨索扇,扇合,簾降,鳴鞭,樂作,皇帝降座,入自西序還後閣,侍衛如來儀,扇開,樂止。侍中奏:「解嚴。」所司承旨,放仗衛以次出。皇太子入次,改服公服,還東宮,導從如來儀。



 冊後二日,兵部設黃麾仗於仁政殿門之內外,陳設並如大安殿之儀。百官服朝服。皇太子公服至次,改服遠遊冠、朱明衣。通事舍人引
 百官入至階下立班,東西相向。典贊儀引皇太子執圭出次,立於門外。侍中奏:「中嚴。」少頃,又奏:「外辦。」皇帝出自東序,即座,簾捲。通事舍人引百官俱橫班北向,典儀贊:「拜。」在位官皆再拜,搢笏,舞蹈,又再拜,起居,又再拜,訖,分班。皇太子捧表入,至拜表位立,俟閣門使將至,單跪捧表,閣門使接表,皇太子俯伏,興,典儀贊:「再拜。」搢圭,舞蹈,又再拜。俟讀表單,侍中承旨退稱:「有制。」典儀贊:「再拜。」興,躬身,侍中宣訖,典儀贊:「再拜。」搢圭,舞蹈,又再拜。引皇太子退。侍中奏:「禮畢。」扇合,鳴鞭,入西序,還後閣,侍衛如來儀。侍中奏:「解嚴。」放仗,百官以次出。後二日,百官奉表稱賀,
 如常儀。



 ○正旦生日皇太子受賀儀



 大定二年,世宗命有司議親王百官及妃主命婦見皇太子禮。有司按唐、宋舊儀,擬親王宗室賀皇太子,依冊畢受賀禮。然唐禮元正復有降階見伯叔、答群官再拜之文,又無妃主命婦見太子之禮。稽古令文,應致恭之官相見,或貴賤殊隔,或長幼親戚,任從私禮。自今若在東宮候皇太子,便服,則當從私禮接見。若三師以下,遇皇太子誕日,在御前,則候皇太子先進酒畢,百官望皇太子再拜,班首跪進酒,又再拜。若賜酒,即當殿跪飲畢,又再拜。以為定制,命班行之。



 十二月晦,皇太子奏狀曰:「按禮文,親王并一品宗室皆北面拜伏,臣但答揖而已。雖曰尊宗子,而在長幼惇敘之間,誠所未安。當時遽蒙頒降,未獲謙讓。明日元正,有司將舉此禮,伏望聖慈許臣答拜,庶敦親親友愛之義。」上從其請,命尚書省頒下所司。



 若皇太子生日,則公服,左上露臺欄子外。先再拜,二閣使齊揖入欄子內,拜跪,祝畢,就拜,興,復位,再拜,又再拜,棲臺進酒,退跪。候飲畢,接盞,復位,轉臺與執事者,再拜。宣徽使以酒進,皇帝親賜酒,接盞稍退跪飲,畢,宣徽使接盞,復位再拜。復揖入欄子內,跪搢笏,受賜物畢,出笏,興,復位,再拜,退更衣,入
 殿稍東,西向立。皇妃等進勸生日酒,皇太子跪,皇妃等亦跪,飲畢,各再拜。群官致賀,則其日質明,皆公服集於門外,少詹事奏:「請內嚴。」又奏:「外備。」典儀引升座,文武宮臣入,就庭下重行北向立,典儀曰:「再拜。」在位官皆再拜,班首稍前跪奏:「元正首祚。」生日則云:「慶誕令辰,伏惟皇太子殿下福壽千秋。」賀畢復位,典儀曰:「再拜。」宮臣皆再拜,坐受,分東西序立。次引東宮三師於殿上,三少於殿柱外,北向東上立。皇太子詣南向褥位,典儀曰:「再拜。」師、少皆再拜,班首同前稱賀,復位。執事者酌酒一卮,班首奉進,樂作,飲訖,樂止。回勸師、少畢,各復位。典儀贊師,少
 再拜,皇太子答拜。師、少出,皇太子就坐。次引親王入欄子內,一品宗室於欄子外,餘宗室序班庭下,拜致賀、進酒如上儀。皇太子答拜畢,就坐。復引隨朝三師三公宰執於殿上,三品以上職事官於露階上,四品以下於庭下,北向,每等重行以東為上,立。皇太子詣褥位。典儀曰:「再拜。」上下皆再拜,畢,班首少前致賀,復位,執事者酌酒一卮,班首奉進,樂作,飲畢,樂止。如有進獻如常儀。回勸三師三公,餘殿上群官則令執事者以盤行酒,飲畢,典儀曰:「再拜。」上下皆再拜,乃答拜,引群官以次出。少詹事跪奏:「禮畢。」自是歲賀為定制。



 ○
 皇太子與百官相見儀



 三師三公欄子內北向躬揖,班首稍前問侯,皇太子離位稍前,正南立,答揖。宰執及一品職事官扣欄子北向躬揖,答揖如前。二品職事官欄子外向南躬揖,皇太子起揖。三品職事官露階稍南躬揖,皇太子坐揖。四品以下職事官庭下躬揖,跪問候,皇太子坐受。太子太師、太傅、太保與隨朝三師同。東宮三少與隨朝二品同。詹事已下,並在庭下面北,每品重行以東為上,再拜,稍前問候,又再拜,皇太子坐受。大定二年所定也。七年,定制,皇太子赴朝,許與親王宰執相見,餘官宗室並迴避。後亦許與樞密使副、御史大夫、判宗
 正、東宮三師相見。九年,定制,凡皇太子出,於都門三里外設褥位,三公宰執以下公服重行立,皇太子便服,三公宰執以下鞠躬,班首致辭云:「青宮萬福。」再拜,皇太子答拜,退。迎、送皆同。



\end{pinyinscope}