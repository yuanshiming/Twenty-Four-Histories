\article{本紀第二}

\begin{pinyinscope}

 太祖



 太祖應乾興運昭德定功仁明莊孝大聖武元皇帝,諱旻,本諱阿骨打,世祖第二子也。母曰翼簡皇后拏懶氏。遼道宗時有五色雲氣屢出東方,大若二千斛囷倉之狀,司天孔致和竊謂人曰:「其下當生異人,建非常之事。天以象告,非人力所能為也。」咸雍四年戊申七月一日,太祖生。幼時與群兒戲,力兼數輩,舉止端重,世祖尤愛
 之。世祖與臘碚、麻產戰於野鵲水,世祖被四創,疾困,坐太祖於膝,循其髮而撫之,曰:「此兒長大,吾復何憂?」十歲,好弓矢。甫成童,即善射。一日,遼使坐府中,顧見太祖手持弓矢,使射群烏,連三發皆中。遼使矍然曰:「奇男子也!」太祖嘗宴紇石烈部活離罕家,散步門外,南望高阜,使眾射之,皆不能至。太祖一發過之,度所至踰三百二十步。宗室謾都訶最善射遠,其不及者猶百步也。天德三年,立射碑以識焉。



 世祖伐卜灰,太祖因辭不失請從行,世祖不許而心異之。烏春既死,窩謀罕請和。既請和,復來攻,遂圍其城。太祖年二十三,被短甲,免胄,不介馬,行
 圍號令諸軍。城中望而識之。壯士太峪乘駿馬持槍出城,馳刺太祖。太祖不及備,舅氏活臘胡馳出其間,擊太峪,槍折,刺中其馬,太峪僅得免。嘗與沙忽帶出營殺略,不令世祖知之。且還,敵以重兵追之。獨行隘巷中,失道,追者益急。值高岸與人等,馬一躍而過,追者乃還。世祖寢疾,太祖以事如遼統軍司。將行,世祖戒之曰:「汝速了此事,五月未半而歸,則我猶及見汝也。」太祖往見曷魯騷古統軍,既畢事,前世祖沒一日還至家。世祖見太祖來,所請事皆如志,喜甚,執太祖手,抱其頸而撫之,謂穆宗曰:「烏雅束柔善,惟此子足了契丹事。」穆宗亦雅重太
 祖,出入必俱。太祖遠出而歸,穆宗必親迓之。



 世祖已擒臘醅,麻產尚據直屋鎧水。肅宗使太祖先取麻產家屬,康宗至直屋鎧水圍之。太祖會軍,親獲麻產,獻馘於遼。遼命太祖為詳穩,仍命穆宗、辭不失、歡都皆為詳穩。久之。以偏師伐泥厖古部跋黑、播立開等,乃以達塗阿為鄉導,沿帥水夜行襲之,鹵其妻子。初,溫都部跋忒殺唐括部跋葛,穆宗命太祖伐之。太祖入辭,謂穆宗曰:「昨夕見赤祥,此行必克敵。」遂行。是歲大雪,寒甚。與烏古論部兵沿土溫水過末鄰鄉,追及跋忒於阿斯溫山北濼之間,殺之。軍還,穆宗親迓太祖於靄建村。



 撒改以都統伐
 留可,謾都訶合石土門伐敵庫德。撒改與將佐議,或欲先平邊地部落城堡,或欲徑攻留可城,議不能決,願得太祖至軍中。穆宗使太祖往,曰:「事必有可疑。軍之未發者止有甲士七十,盡以畀汝。」謾都訶在米里迷石罕城下,石土門未到,土人欲執謾都訶以與敵,使來告急,遇太祖於斜堆甸。太祖曰:「國兵盡在此矣。使敵先得志於謾都訶,後雖種誅之,何益也。」乃分甲士四十與之。太祖以三十人指撒改軍。道遇人曰:「敵已據盆搦嶺南路矣。」眾欲由沙偏嶺往,太祖曰:「汝等畏敵耶?」既度盆搦嶺,不見敵,已而聞敵乃守沙偏嶺以拒我。及至撒改軍,夜急
 攻之,遲明破其眾。是時,留可、塢塔皆在遼。既破留可,還攻塢塔城,城中人以城降。初,太祖過盆搦嶺,經塢塔城下,從騎有後者,塢塔城人攻而奪之釜。太祖駐馬呼謂之曰:「毋取我炊食器。」其人謾言曰:「公能來此,何憂不得食。」太祖以鞭指之曰:「吾破留可,即於汝乎取之。」至是,其人持釜而前曰:「奴輩誰敢毀祥穩之器也。」遣蒲家奴招詐都,詐都乃降,釋之。穆宗將伐蕭海里,募兵得千餘人。女直兵未嘗滿千,至是,太祖勇氣自倍,曰:「有此甲兵,何事不可圖也!」海里來戰,與遼兵合,因止遼人,自為戰。勃海留守以甲贈太祖,太祖亦不受。穆宗問何為不受?曰:「
 被彼甲而戰,戰勝則是因彼成功也。」穆宗末年,令諸部不得擅置信牌馳驛訊事,號令自此始一,皆自太祖啟之。



 康宗七年,歲不登,民多流莩,彊者轉而為盜。歡都等欲重其法,為盜者皆殺之。太祖曰:「以財殺人,不可!財者,人所致也。」遂減盜賊徵償法為征三倍。民間多逋負,賣妻子不能償,康宗與官屬會議,太祖在外庭以帛繫杖端,麾其眾,令曰:「今貧者不能自活,賣妻子以償債。骨肉之愛,人心所同。自今三年勿征,過三年徐圖之。」眾皆聽令,聞者感泣,自是遠近歸心焉。歲癸巳十月,康宗夢逐狼,屢發不能中,太祖前射中之。旦日,以所夢問僚佐,眾
 曰:「吉。兄不能得而弟得之之兆也。」是月,康宗即世,太祖襲位為都勃極烈。遼使阿息保來,曰:「何以不告喪?」太祖曰:「有喪不能弔,而乃以為罪乎?」他日,阿息保復來,徑騎至康宗殯所,閱賵馬,欲取之。太祖怒,將殺之,宗雄諫而止。既而遼命久不至。遼主好畋獵、淫酗,怠于政事,四方奏事,往往不見省。紇石烈阿疏既奔遼,穆宗取其城及其部眾,不能歸。遂與族弟銀術可、辭里罕陰結南江居人渾都僕速,欲與俱亡入高麗。事覺,太祖使夾古撒喝捕之,而銀術可、辭里罕先為遼戍所獲,渾都僕速已亡去,撒喝取其妻子而還。



 二年甲午六月,太祖至江西,遼
 使使來致襲節度之命。初,遼每歲遣使市名鷹海東青于海上,道出境內,使者貪縱,徵索無藝,公私厭苦之。康宗嘗以不遣阿疏為言,稍拒其使者。太祖嗣節度,亦遣蒲家奴往索阿疏,故常以此二者為言,終至于滅遼然後已。至是,復遣宗室習古迺、完顏銀術可往索阿疏。習古乃等還,具言遼主驕肆廢弛之狀。於是召官僚耆舊,以伐遼告之,使備衝要,建城堡,修戎器,以聽後命。遼統軍司聞之,使節度使捏哥來問狀,曰:「汝等有異志乎?修戰具,傷守備,將以誰禦?」太祖答之曰:「設險自守,又何問哉!」遼復遣阿息保來詰之。太祖謂之曰:「我小國也,事大
 國不敢廢禮。大國德澤不施,而逋逃是主,以此字小,能無望乎?若以阿疏與我,請事朝貢。茍不獲已,豈能束手受制也。」阿息保還,遼人始為備,命統軍蕭撻不野調諸軍於寧江州。太祖聞之,使僕聒剌復索阿疏,實觀其形勢。僕聒剌還言:「遼兵多,不知其數。」太祖曰:「彼初調兵,豈能遽集如此。」復遣胡沙保往,還言:「惟四院統軍司與寧江州軍及渤海八百人耳。」太祖曰:「果如吾言。」謂諸將佐曰:「遼人知我將舉兵,集諸路軍備我,我必先發制之,無為人制。」眾皆曰:「善。」乃入見宣靖皇后,告以伐遼事。后曰:「汝嗣父兄立邦家,見可則行。吾老矣,無貽我憂,汝必不
 至是也。」太祖感泣,奉觴為壽。即奉后率諸將出門,舉觴東向,以遼人荒肆,不歸阿疏,并己用兵之意,禱於皇天后土。酹畢,后命太祖正坐,與僚屬會酒,號令諸部。使婆盧火徵移懶路迪古乃兵,斡魯古、阿魯撫諭斡忽、急賽兩路係遼籍女直,實不迭往完睹路執遼障鷹官達魯古部副使辭列、寧江州渤海大家奴。於是達魯古部實里館來告曰:「聞舉兵伐遼,我部誰從?」太祖曰:「吾兵雖少,舊國也,與汝鄰境,固當從我。若畏遼人,自往就之。」



 九月,太祖進軍寧江州,次寥晦城。婆盧火徵兵後期,杖之,復遣督軍。諸路兵皆會于來流水,得二千五百人。致遼之
 罪,申告于天地曰:「世事遼國,恪修職貢,定烏春、窩謀罕之亂,破蕭海里之眾,有功不省,而侵侮是加。罪人阿疏,屢請不遣。今將問罪於遼,天地其鑒佑之。」遂命諸將傳挺而誓曰:「汝等同心盡力,有功者,奴婢部曲為良,庶人官之,先有官者敘進,輕重視功。茍違誓言,身死梃下,家屬無赦。」師次唐括帶斡甲之地,諸軍禳射,介而立,有光如烈火,起於人足及戈矛之上,人以為兵祥。明日,次扎只水,光見如初。將至遼界,先使宗幹督士卒夷塹。既度遇渤海軍攻我左翼七謀克,眾少卻,敵兵直犯中軍。斜也出戰,哲垤先驅。太祖曰:「戰不可易也。」遣宗乾止之。宗
 乾馳出斜也前,控止哲垤馬,斜也遂與俱還。敵人從之,耶律謝十墜馬,遼人前救。太祖射救者斃。併射謝十中之。有騎突前,又射之,徹扎洞胸。謝十拔箭走,追射之,中其背,飲矢之半,僨而死,獲所乘馬。宗乾與數騎陷遼軍中,太祖救之,免胄戰。或自傍射之,矢拂於顙。太祖顧見射者,一矢而斃。謂將士曰:「盡敵而止。」眾從之,勇氣自倍。敵大奔,相蹂踐死者十七八。撒改在別路,不及會戰,使人以戰勝告之,而以謝十馬賜之。撒改使其子宗翰、完顏希尹來賀,且稱帝,因勸進。太祖曰:「一戰而勝,遂稱大號,何示人淺也。」進軍寧江州,諸軍填塹攻城。寧江人自
 東門出,溫迪痕、阿徒罕邀擊,盡殪之。



 十月朔,克其城,獲防禦使大藥師奴,陰縱之,使招諭遼人。鐵驪部來送款。次來流城,以俘獲賜將士。召渤海梁福、斡答剌使之偽亡去,招諭其鄉人曰:「女直、渤海本同一家,我興師伐罪,不濫及無辜也。」使完顏婁室招諭係遼籍女直。師還,謁宣靖皇后,以所獲頒宗室耆老,以實里館貲產給將士。初命諸路以三百戶為謀克,十謀克為猛安。酬斡等撫定讒謀水女直。鱉古酋長胡蘇魯以城降。



 十一月,遼都統蕭糺里、副都統撻不野將步騎十萬會于鴨子河北。太祖自將擊之。未至鴨子河,既夜,太祖方就枕,若有扶
 其首者三,寤而起,曰:「神明警我也!」即鳴鼓舉燧而行。黎明及河,遼兵方壞凌道,選壯士十輩擊走之。大軍繼進,遂登岸。甲士三千七百,至者纔三之一。俄與敵遇于出河店,會大風起,塵埃蔽天,乘風勢擊之,遼兵潰。逐至斡論濼,殺獲首虜及車馬甲兵珍玩不可勝計,遍賜官屬將士,燕犒彌日。遼人嘗言女直兵若滿萬則不可敵,至是始滿萬云。斡魯古敗遼兵,斬其節度使撻不野。僕虺等攻賓州,拔之。兀惹雛鶻室來降。遼將赤狗兒戰于賓州,僕虺、渾黜敗之。鐵驪王回離保以所部降。吾睹補、蒲察復敗赤狗兒、蕭乙薛軍于祥州東。斡忽、急塞兩路降。斡
 魯古敗遼軍于咸州西,斬統軍實婁于陣。完顏婁室克咸州。



 是月,吳乞買、撒改、辭不失率宮屬諸將勸進,願以新歲元日恭上尊號,太祖不許。阿離合懣、蒲家奴、宗翰等進曰:「今大功已建,若不稱號,無以繫天下心。」太祖曰:「吾將思之。」



 收國元年正月壬申朔,群臣奉上尊號。是日,即皇帝位。上曰:「遼以賓鐵為號,取其堅也。賓鐵雖堅,終亦變壞,惟金不變不壞。金之色白,完顏部色尚白。」於是國號大金,改元收國。丙子,上自將攻黃龍府,進臨益州。州人走保黃龍,取其餘民以歸。遼遣都統耶律訛里朵、左副統蕭
 乙薛、右副統耶律張奴、都監蕭謝佛留,騎二十萬、步卒七萬戍邊。留婁室、銀術可守黃龍,上率兵趨達魯古城,次寧江州西。遼使僧家奴來議和,國書斥上名,且使為屬國。庚子,進師,有火光正圓,自空而墜。上曰:「此祥征,殆天助也!」酹白水而拜,將士莫不喜躍,進逼達魯古城。上登高望遼兵若連雲灌木狀,顧謂左右曰:「遼兵心貳而情怯。雖多不足畏!」遂趨高阜為陣。宗雄以右翼先馳遼左軍,左軍卻。左翼出其陣後,遼右軍皆力戰。婁室、銀術可衝其中堅。凡九陷陣,皆力戰而出。宗翰請以中軍助之。上使宗乾往為疑兵。宗雄已得利,擊遼右軍,遼兵遂
 敗。乘勝追躡,至其營,會日已暮,圍之。黎明,遼軍潰圍出,逐北至阿婁岡。遼步卒盡殪,得其耕具數千以給諸軍。是役也,遼人本欲屯田,且戰且守,故併其耕具獲之。



 二月,師還。三月辛未朔,獵于寥晦城。四月,遼耶律張奴以國書來。上以書辭慢侮,留其五人,獨遣張奴回報,書亦如之。五月庚午朔,避暑于近郊。甲戌,拜天射柳。故事,五月五日、七月十五日、九月九日拜天射柳,歲以為常。



 六月己亥朔,遼耶律張奴復以國書來,猶斥上名。上亦斥遼主名以復之,且諭之使降。七月戊辰,以弟吳乞買為諳班勃極烈,國相撒改為國論勃極烈。辭不失為阿買
 勃極烈,弟斜也為國論昊勃極烈,甲戌,遼使辭剌以書來,留之不遣。九百奚營來降。



 八月戊戌,上親征黃龍府。次混同江,無舟,上使一人道前,乘赭白馬徑涉,曰:「視吾鞭所指而行。」諸軍隨之,水及馬腹。後使舟人測其渡處,深不得其底。熙宗天眷二年,以黃龍府為濟州,軍曰利涉,蓋以太祖涉濟故也。



 九月,克黃龍府,遣辭剌還,遂班師。至江,徑渡如前。丁丑,至自黃龍府。己卯,黃龍見空中。癸巳,以國論勃極烈撒改為國論忽魯勃極烈,阿離合懣為國論乙室勃極烈。



 十一月,遼主聞取黃龍府,大懼,自將七十萬至駝門。附馬蕭特末、林牙蕭查剌等騎五
 萬、步四十萬至斡鄰濼。上自將禦之。



 十二月己亥,行次爻剌,會諸將議。皆曰:「遼兵號七十萬,其鋒未易當。吾軍遠來。人馬疲乏,宜駐於此,深溝高壘以待。」上從之。遣迪古乃、銀術可鎮達魯古。丁未,上以騎兵親候遼軍,獲督餉者,知遼主以張奴叛,西還二日矣。是日,上還至熟結濼,有光見于予端。戊申,諸將曰:「今遼主既還,可乘怠追擊之。」上曰「敵來不迎戰,去而追之,欲以此為勇邪?」眾皆悚愧,願自效。上復曰:「誠欲追敵,約齎以往,無事餫饋。若破敵,何求不得。」眾皆奮躍,追及遼主于護步答岡。是役也,兵止二萬。上曰:「彼眾我寡,兵不可分。視其中軍最堅,
 遼主必在焉。敗其中軍,可以得志。」使右翼先戰。兵數交,左翼合而攻之。遼兵大潰,我師馳之,橫出其中。遼師敗績,死者相屬百餘里。獲輿輦帟幄兵械軍資,他寶物馬牛不可勝計。是戰,斜也援矛殺數十人,阿離本被圍,溫迪罕迪忽迭以四謀克兵出之,完顏蒙刮身被數創,力戰不已,功皆論最。蕭特末等焚營遁去,遂班師。來谷撒喝取開州,婆盧火下特鄰城,辭里罕降。



 二年正月戊子,詔曰:「自破遼兵,四方來降者眾,宜加優恤。自今契丹、奚、漢、渤海、係遼籍女直、室韋、達魯古、兀惹、鐵驪諸部官民,己降或為軍所俘獲,逃遁而還者,勿以
 為罪。其酋長仍官之,且使從宜居處。」



 閏月,高永昌據東京,使撻不野來求援。高麗遣使來賀捷,且求保州。詔許自取之。二月己巳,詔曰:「比以歲凶,庶民艱食,多依附豪族,因為奴隸,及有犯法,徵償莫辦,折身為奴者,或私約立限,以人對贖,過期則為奴者,並聽以兩人贖一為良。若元約以一人贖者,即從元約。」四月乙丑,以斡魯統內外諸軍,與蒲察、迪古乃會咸州路都統斡魯古討高永昌。胡沙補等被害。五月,斡魯等敗永昌,撻不野擒永昌以獻,戮之於軍。東京州縣及南路係遼女直皆降。詔除遼法,省稅賦,置猛安謀克一如本朝之制。以斡魯為南
 路都統。迭勃極烈阿徒罕破遼兵六萬于照散城。九月己亥,上獵近郊。乙巳,南路都統斡魯來見於婆盧買水。始製金牌。十二月庚申朔,諳班勃極烈吳乞買及群臣上尊號曰大聖皇帝,改明年為天輔元年。



 天輔元年正月,開州叛,加古撒喝等討平之。國論昊勃極烈斜也以兵一萬取泰州。四月,遼秦晉國王耶律捏里來伐,迪古乃、婁室、婆盧火將兵二萬。會咸州路都統斡魯古擊之。五月丁巳,詔自收寧江州已後同姓為婚者,杖而離之。七月戊申,以完顏斡論知東京事。八月癸亥,高麗遣使來請保州。十二月甲子,斡魯古等敗耶律捍里兵于蒺
 藜山,拔顯州,乾、懿、豪、徽、成、川、惠等州皆降。



 是月,宋使登州防禦使馬政以國書來,其略曰:「日出之分,實生聖人。竊聞征遼,屢破勍敵。若克遼之後,五代時陷入契丹漢地,願畀下邑。」



 二年正月庚寅,遼雙州節度使張崇降。使散睹如宋報聘,書曰:「所請之地,今當與宋夾攻,得者有之。」



 二月癸丑朔,遼使耶律奴哥等來議和。辛酉,孛堇迪古乃、婁室來見。上以遼主近在中京,而敢輒來,皆杖之。劾里保、雙古等言,咸州都統斡魯古知遼主在中京而不進討,芻糧豐足而不以實聞,攻顯州時所獲生口財畜多自取。



 三
 月癸未朔,命闍哥代為都統而鞫治之,斡魯古坐降謀克。壬辰,遼使耶律奴哥以國書來。庚子,以婁室言黃龍府地僻且遠,宜重戍守,乃命合諸路謀克,以婁室為萬戶鎮之。



 四月辛巳,遼使以國書來。



 五月丙申,命胡突袞如遼。



 六月甲寅,詔有司禁民凌虐典雇良人,及倍取贖直者。甲戌,遼通、祺、雙、遼等州八百餘戶來歸,命分置諸部,擇膏腴之地處之。



 七月癸未,詔曰:「匹里水路完顏術里古、渤海大家奴等六謀克貧乏之民,昔嘗給以官糧,置之漁獵之地。今歷日已久,不知登耗,可具其數以聞。」胡突袞還自遼,耶律奴哥復以國書來。丙申,胡突袞如
 遼。遼戶二百來歸,處之泰州。詔遣阿里骨、李家奴、特裏底招諭未降者。仍詔達魯古部勃堇辭列:「凡降附新民,善為存撫。來者各令從便安居,給以官糧,毋輒動擾。」



 八月,胡突袞還自遼,耶律奴哥、突迭復以國書來。



 九月戊子,詔曰:「國書詔令,宜選善屬文者為之。其令所在訪求博學雄才之士。敦遣赴闕。」



 閏月庚戌朔,以降將霍石、韓慶和為千戶。九百奚部蕭寶、乙辛,北部訛里野,漢人王六兒、王伯龍,契丹特末、高從祐等,各率眾來降。遼耶律奴哥以國書來。



 十月癸未,以龍化州降者張應古、劉仲良為千戶。乙未,咸州都統司言,漢人李孝功、渤海二哥
 率眾來降。命各以所部為千戶。



 十二月甲辰,遣孛堇術孛以定遼地諭高麗。耶律奴哥以國書來。遼懿州節度使劉宏以戶三千并執遼候人來降,以為千戶。川州寇二萬已降復叛,紇古烈照里擊破之。



 三年正月甲寅,東京人為質者永吉等五人結眾叛。事覺,誅其首惡,餘皆杖百,沒入在行家屬資產之半。詔知東京事斡論,繼有犯者並如之。丙辰,詔鱉古孛堇酬斡曰:「胡魯古、迭八合二部來送款,若等先時不無交惡,自今毋相侵擾。」



 三月,耶律奴哥以國書來。



 四月丙子朔,日有食之。



 五月壬戌,詔咸州路都統司曰:「兵興以前,曷蘇
 館、回怕里與係遼籍、不係遼籍女直戶民,有犯罪流竄邊境或亡入于遼者,本皆吾民,遠在異境,朕甚憫之。今即議和,當行理索。可明諭諸路千戶、謀克,遍與詢訪其官稱、名氏、地里,具錄以上。」



 六月辛卯,遼遣太傳習泥烈等奉冊爾來,上擿冊文不合者數事復之。散睹還自宋。宋使馬政及其子宏來聘。散睹受宋團練使,上怒,杖而奪之。宋使還,復遣孛堇辭列、魯等如宋。



 七月辛亥,遼人楊詢卿、羅子韋各率眾來降,命各以所部為謀克。



 八月己丑,頒女直字。



 九月,以遼冊禮使失期,詔諸路軍過江屯駐。



 十一月,習泥烈等復以國書來。曷懶甸長城,高
 麗增築三尺。詔胡剌古、習顯慎固營壘。



 四年二月,辭列、曷魯還自宋。宋使趙良嗣、王暉來議燕京、西京地。



 三月甲辰,上謂群臣曰:「遼人屢敗,遣使求成,惟飾虛辭,以為緩師之計,當議進討。其令咸州路統軍司治軍旅、修器械,具數以聞。」辛酉,詔咸州路都統司曰:「朕以遼國和議無成,將以四月二十五日進師。」令斜葛留兵一千鎮守,闍母以餘兵來會于渾河。遼習泥烈以國書來。



 四月乙未,上自將伐遼。以遼使習泥烈、宋使趙良嗣等從行。



 五月甲辰,次渾河西,使宗雄先趨上京,遣降者馬乙持詔諭城中。壬子,至上京,詔官民曰:「遼主失
 道,上下同怨。朕興兵以來,所過城邑負固不服者即攻拔之,降者撫恤之,汝等必聞之矣。今爾國和好之事,反覆見欺,朕不欲天下生靈久罹塗炭,遂決策進討。比遣宗雄等相繼招諭,尚不聽從。今若攻之,則城破矣!重以弔伐之義,不欲殘民,故開示明詔,諭以禍福,其審圖之。」上京人恃禦備儲蓄為固守計。甲寅,亟命進攻。上謂習泥烈、趙良嗣等曰:「汝可觀吾用兵,以卜去就。」上親臨城,督將士諸軍鼓噪而進。自旦及巳,闍母以麾下先登,克其外城,留守撻不野以城降。趙良嗣等奉觴為壽,皆稱萬歲。是日,赦上京官民。詔諭遼副統余睹。壬戌,次沃黑
 河。宗乾率群臣諫曰:「地遠時暑,軍馬罷乏,若深入敵境,糧饋乏絕,恐有後艱。」上從之,乃班師,命分兵攻慶州。余睹襲闍母於遼河,完顏背答、烏塔等戰卻之,完顏特虎死焉。



 七月癸卯,上至自伐遼。



 九月,燭隈水部實里古達等殺孛堇酬斡、僕忽得以叛。



 十月戊辰朔,日有食之。戊寅,命斡魯分胡剌古、烏春之兵以討實里古達。



 十一月,東京留守司乞本京官民質子增數番代,上不許,曰:「諸質子已各受田廬,若復番代,則往來動搖,可並仍舊。」



 十二月,宋復使馬政來請西京之地。



 五年春正月,斡魯敗實里古達於合撻剌山,誅首惡四
 人,餘悉撫定。



 二月,遣昱及宗雄分諸路猛安謀克之民萬戶屯泰州,以婆盧火統之,賜耕牛五十。



 四月乙丑朔,宗翰請伐遼,詔諸路預戒軍事。



 五月,遼都統耶律餘睹等詣咸州降。閏月辛巳,國論胡魯勃極烈撒改薨。



 六月癸巳,余睹與其將吏來見。丙申,千戶胡離答坐擅署部人為蒲里衍,杖一百,罷之。庚子,詔諳版勃極烈吳乞買貳國政。以昊勃極烈斜也為忽魯勃極烈,蒲家奴為昊勃極烈,宗翰為移賚勃極烈。



 七月庚辰,詔咸州都統司曰:「自餘睹來,灼見遼國事宜,已決議親征,其治軍以俟師期。」尋以連雨罷親征。命昊勃極烈昱為都統,稱賚勃
 極烈宗翰副之,帥師而西。



 十二月辛丑,以忽魯勃極烈杲為內外諸軍都統,以昱、宗翰、宗乾、宗望、宗盤等副之。甲辰,詔曰:「遼政不綱,人神共棄。今欲中外一統,故命汝率大軍以行討伐。爾其慎重兵事,擇用善謀,賞罰必行,糧餉必繼,勿擾降服,勿縱俘掠,見可而進,無淹師期。事有從權,毋須申稟。」戊申,詔曰:「若克中京,所得禮樂儀仗圖書文籍,並先次津發赴闕。」



 六年正月癸酉,都統杲克高、恩、回紇三城。乙亥,取中京,遂下澤州。



 二月庚寅朔,日有食之。己亥,宗翰等敗遼奚王霞末於北安州,降。奚部西節度使訛里剌以本部降。
 壬寅,都統杲遣使來奏捷,并獻所獲貨寶。詔曰:「汝等提兵于外,克副所任,攻下城邑,撫安人民,朕甚嘉之。所言分遣將士招降山前諸部,計悉已撫定,續遣來報。山後若未可往,即營田牧馬,俟及秋成,乃圖大舉。更當熟議,見可則行。如欲益兵,具數來上,不可恃一戰之勝,輒有弛慢。新降附者當善撫存。宣諭將士,使知朕意。」宗翰駐北安,遣希尹等略地,獲遼護衛耶律習泥烈,知遼主獵鴛鴦濼,以其子晉王賢而有人望,惡而殺之,眾益離心。雖有西北、西南兩路兵馬,皆羸弱。遂遣耨碗溫都等報都統杲進兵襲之。



 三月,都統杲出青嶺,宗翰出瓢嶺,追
 遼主于鴛鴦濼。遼主奔西京。宗翰復追至白水濼,不及,獲其貨寶。己巳,至西京。壬申,西京降。希尹追遼主於乙室部,不及。乙亥,西京復叛。



 是月,遼秦晉國王耶律捏里即位于燕。



 四月辛卯,復取西京。壬辰,遣徒單吳甲、高慶裔如宋。戊戌,都統杲自西京趨白水濼,昊勃極烈昱襲毗室部于鐵呂川,為敵所敗。還會察剌兵,追至黃水北,大破之。耶律坦招徠西南諸部,西至夏,其招討使耶律佛頂降。金肅、西平二郡漢軍四千餘人叛去,耶律坦等襲取之。闍母、婁室招降天德、雲內、寧邊、東勝等州,獲阿疏而還。是時,山西城邑諸部雖降,人心未固,遼主保陰
 山,耶律捍里在燕京,都統杲遣宗望入奏,請上臨軍。



 五月辛酉,宗望來奏捷,百官入賀,賜宴歡甚。先是,獲遼樞密使得里底、節度使和尚、雅里斯、餘里野等,都統杲使阿鄰護送赴闕。得里底道亡,阿鄰坐誅。耶律捍里遣使請罷兵。戊寅,使楊勉以書諭捍里,使之降。謀葛失遣其子菹泥刮失貢方物。



 六月戊子朔,上親征遼,發自上京。諳班勃極烈吳乞買監國。辛亥,詔諭上京官民曰:「朕順天弔伐,已定三京,但以遼主未獲,兵不能已。今者親征,欲由上京路進,恐撫定新民,驚疑失業,已出自篤密呂。其先降後叛逃入險阻者,詔後出首,悉免其罪。若猶拒
 命,孥戮無赦。」



 是月,耶律捍里卒。斡魯、婁室敗夏人於野谷。



 七月甲子,詔諸將無得遠迎,以廢軍務。乙丑,上京漢人毛八十率二千餘戶降,因命領之。丙寅,以斡荅剌招降者眾,命領八千戶,以忽薛副之。壬午,希尹以阿疏見杖而釋之。



 八月己丑,次鴛鴦濼。都統杲率官屬來見。癸巳,上追遼主于大魚濼。昱、宗望追及遼主于石輦鐸,與戰,敗之,遼主遁。己亥,次居延北。辛丑,中京將完顏渾黜敗契丹、奚、漢六萬於高州,孛堇麻吉死之。得里得滿部降。昱、宗望追遼主于烏里質鐸,不及。



 九月庚申,次草濼。闍母平中京部族之先叛者,及招撫沿海郡縣。節度使
 耶律慎思領諸部入內地。乙丑,詔六部奚曰:「汝等既降復叛,扇誘眾心,罪在不赦。尚以歸附日淺,恐綏懷之道有所未孚,故復令招諭。若能速降,當釋其罪,官皆仍舊。」歸化州降。戊辰,次歸化州。甲戌,宗雄薨。丁丑,奉聖州降。



 十月丙戌朔,次奉聖州。詔曰;「朕屢敕將臣,安輯懷附,無或侵擾。然愚民無知,尚多逃匿山林,即欲加兵,深所不忍。今其逃散人民,罪無輕重,咸與矜免。有能率眾歸附者,授之世官。或奴婢先其主降,並釋為良。其布告之,使諭朕意。」蔚州降。庚寅,余睹等遣蔚州降臣翟昭彥、徐興、田慶來見。命昭彥、慶皆為刺史,興為團練使。詔曰:「比以
 幽、薊一方招之不服,今欲帥師以往,故先安撫山西諸部。汝等既已懷服,宜加撫存。官民未附已前,罪無輕重及係官逋負,皆與釋免,諸官各遷敘之。」丁酉,蔚州翟昭產、田慶殺知州事蕭觀寧等以叛。丙午,復降。



 十一月,詔諭燕京官民,王師所至,降者赦其罪,官皆仍舊。



 十二月,上伐燕京。宗望率兵七千先之,迪古乃出得勝口,銀術哥出居庸關,婁室為左翼,婆盧火為右翼,取居庸關。丁亥,次媯州。戊子,次居庸關。庚寅,遼統軍都監高六等來送款。上至燕京,入自南門,使銀術哥、婁室陣于城上,乃次于城南。遼知樞密院左企弓、虞仲文,樞密使曹勇義,
 副使張彥忠,參知政事康公弼,僉書劉彥宗奉表降。辛卯,遼百官詣軍門叩頭請罪,詔一切釋之。壬辰,上御德勝殿,群臣稱賀。甲午,命左企弓等撫定燕京諸州縣。詔西京官吏曰:「乃者師至燕都,已皆撫定。唯蕭妃與官屬數人遁去,已發兵追襲,或至彼路,可執以來。」黃龍府叛,宗輔討平之。



 七年正月丁巳,遼奚王回離保僭稱帝。甲子,遼平州節度使時立愛降。詔曲赦平州。又詔諳班勃極烈曰:「比遣昂徙諸部民人於嶺東,而昂悖戾,騷動煩擾,致多怨叛。其違命失眾,當置重典。若或有疑,禁錮以待。」庚午,詔中
 京都統斡論曰:「聞卿撫定人民,各安其業,朕甚嘉之。回離保聚徒逆命,汝宜計畫,無使滋蔓。」壬申,詔招諭回離保。癸酉,以時立愛言招撫諸部。己卯,宋使來議燕京、西京地。庚辰,宜、錦、乾、顯、成、川、豪、懿等州皆降。甲申,詔曰:「諸州部族歸附日淺,民心未寧。今農事將興,可遣分諭典兵之官,無縱軍士動擾人民,以廢農業。」



 二月乙酉朔,命撒八詔諭興中府,降之。遼來州節度使田顥、隰州刺史杜師回、遷州刺史高永福、潤州刺史張成皆降。壬辰,詔諳版勃極烈曰:「郡縣今皆撫定,有逃散未降者,已釋其罪,更宜招諭之。前後起遷戶民,去鄉未久,豈無懷土之
 心?可令所在有司,深加存恤,毋輒有騷動。衣食不足者,官賑貸之。」癸巳,詔曰:「頃因兵事未息,諸路關津絕其往來。今天下一家,若仍禁之,非所以便民也。自今顯、咸、東京等路往來,聽從其便。其間被虜及鬻身者,並許自贖為良。」仍令馳驛布告。興中、宜州復叛。宋使趙良嗣來,請加歲幣以代燕稅,及議畫疆與遣使賀正旦生辰、置榷場交易,並計議西京等事。癸卯,銀術哥、鐸剌如宋。乙巳,詔都統杲曰:「新附之民有材能者,可錄用之。」戊申,詔平州官與宋使同分割所與燕京六州之地。癸丑,大赦。



 是月,改平州為南京,以張覺為留守。



 三月甲寅朔,將誅昂,
 以習不失諫,杖之七十,仍拘泰州。戊午,都統杲等言耶律麻哲告余睹、吳十、鐸刺等謀叛,宜早圖之。上召余睹等,從容謂之曰:「朕得天下,皆我君臣同心同德以成大功,固非汝等之力。今聞汝等謀叛,若誠然耶,必須鞍馬甲胄器械之屬,當悉付汝,朕不食言。若再為我擒,無望免死。欲留事朕,無懷異志,吾不汝疑。」余睹等皆戰慄不能對。命杖鐸剌七十,餘並釋之。宋使盧益、趙良嗣、馬宏以國書來。



 四月丁亥,遣斡魯、宗望襲遼主于陰山。壬辰,復書于宋。師初入燕,遼兵復犯奉聖州,林牙大石壁龍門東二十五里。都統斡魯聞之,遣照立、婁室、馬和尚等率兵討
 之,生獲大石,悉降其眾。癸巳,詔曰:「自今軍事若皆中覆,不無留滯。應此路事務申都統司,餘皆取決樞密院。」契丹九斤取聚黨興中府作亂,擒之,九斤自殺。命習古乃、婆盧火監護長勝軍,及燕京豪族工匠,由松亭關徙之內地。己亥,次儒州。斡魯、宗望等襲遼權六院司喝離質于白水濼,獲之。其宗屬秦王、許王等十五人降。聞遼主留輜重青塚,以兵萬人往應州,遣照里、背荅、宗望、婁室、銀術哥等追襲之。宗望追及遼主,決戰,大敗之,獲其子趙王習泥烈及傳國璽。



 五月甲寅,南京留守張覺據城叛。丙寅,次野狐嶺。己巳,次落藜濼。斡魯等以趙王習泥烈、
 林牙大石、附馬乳奴等來獻,并上所獲國璽。宗雋以所俘遼主子秦王、許王、女奧野等來見。奚路都統撻懶攻速古、啜里、鐵尼所部十三巖,皆平之。又遣奚馬和尚攻下達魯古并五院司諸部,執其節度乙列。回離保為其下所殺。辛巳,詔諭南京官民。



 六月壬午朔,次鴛鴦濼。是日,闍母敗張覺于營州。丙申,上不豫,將還上京,命移賚勃極烈宗翰為都統,昊勃極烈昱、迭勃極烈斡魯副之,駐兵雲中,以備邊。己酉,次斡獨山驛,召諳班勃極烈吳乞買。



 七月辛酉,次牛山。宗翰還軍中。



 八月辛巳朔,日有食之。乙未,次渾河北。諳班勃極烈吳乞買率宗室百
 官上謁。戊申,上崩于部堵濼西行宮,年五十六。



 九月癸丑,梓宮至上京。乙卯,葬宮城西南,建寧神殿。丙辰,諳班勃極烈即皇帝位。天會三年三月,上尊謚曰武元皇帝,廟號太祖,立原廟于西京。天會十三年二月辛酉,改葬和陵,立《開天啟祚睿德神功之碑》於燕京城南嘗所駐蹕之地。皇統四年,改和陵曰睿陵。五年十月,增謚應乾興運昭德定功睿神莊孝仁明大聖武元皇帝。貞元三年十一月,改葬於大房山,仍號睿陵。



 贊曰:太祖英謨睿略,豁達大度,知人善任,人樂為用。世祖陰有取遼之志,是以兄弟相授,傳及康宗,遂及太祖。
 臨終以太祖屬穆宗,其素志蓋如是也。初定東京,即除去遼法,減省租稅,用本國制度。遼主播越,宋納歲幣,以幽、薊、武、朔等州與宋,而置南京於平州。宋人終不能守燕、代,卒之遼主見獲,宋主被執。雖功成於天會間,而規摹運為賓自此始。金有天下百十有九年,太祖數年之間算無遺策,兵無留行,底定大業,傳之子孫。鳴呼,雄哉!



\end{pinyinscope}