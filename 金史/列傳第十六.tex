\article{列傳第十六}

\begin{pinyinscope}

 ○劉彥宗劉萼劉筈劉仲誨劉頍時立愛韓企先子鐸



 劉彥宗,子魯開,大興宛平人。遠祖怦,唐盧龍節度使。石晉以幽、薊入遼,劉氏六世仕遼,相繼為宰相。父霄,至中京留守。彥宗擢進士乙科。天祚走天德,秦晉國王耶律捏里自立于燕,擢彥宗留守判官。蕭妃攝政,遷簽書樞密院事。太祖至居庸關,蕭妃自古北口遁去,都監高六
 送款于太祖。太祖奄至,駐蹕城南,彥宗與左企弓等奉表降。太祖一見,器遇之,俾復舊,遷左僕射,佩金牌。



 張覺為南京留守,太祖聞覺有異志,使彥宗、斜缽宣慰之。太祖至鴛鴦濼,不豫,還上京,留宗翰都統軍事,留彥宗佐之。及張覺敗奔于宋,眾推張敦固為都統,殺使者,乘城拒守,攻之不肯下。彥宗同中書門下平章事,知樞密院事,加侍中,佐宗望軍。宗望奏,方圖攻取,凡州縣之事委彥宗裁決之。



 天會二年,詔彥宗曰:「中京等兩路先多拒命,故遣使撫諭,貰其官民之罪,所犯在降附前者勿論。卿等選官與使者往諭之,使勤于稼穡。」未幾,大舉伐宋,
 彥宗畫十策,詔彥宗兼領漢軍都統。蔡靖以燕山降。詔彥宗凡燕京一品以下官皆承制注授,遂進兵伐宋。至汴,宋少帝割地納質,師還。宗望分將士屯安肅、雄、霸、廣信之境,留闍母、彥宗于燕京節制諸軍。明年,再伐宋,已圍汴京,彥宗謂宗翰、宗望曰:「蕭何入關,秋毫無犯,惟收圖籍。遼太宗入汴,載路車、法服、石經以歸,皆令則也。」二帥嘉納之,執二帝以歸。



 天會六年薨,年五十三,追封鄆王。正隆二年,例降封開府儀同三司。大定十五年,追封兗國公,謚英敏。子萼、筈。



 萼,彥宗季子也。遼末以蔭補閣門祗候。天輔七年,授禮
 賓使,累官德州防禦使。天德初,稍加擢用,歷左右宣徽使,拜參知政事,進尚書左丞,為沁南軍節度使,歷臨洮、太原尹。正隆南伐,為漢南道行營兵馬都統制。大定初,除興中尹,封任國公,歷順天、定武軍節度使、濟南尹。萼淫縱無行,所至貪墨狼籍。廉使劾之,詔遣大理少卿張九思就濟南鞫問。既就逮,不測所以,引刃自殺,不死。詔削官一階,罷歸田里,卒。子仲詢,天德三年,賜王彥潛榜及第。



 筈,彥宗次子。幼時以廕隸閣門,不就,去從學。遼末調兵,而筈在選中。遼兵敗,左右多散亡,乃選筈為扈從,授左
 承制。遼主西奔,蕭妃攝政,賜筈進士第,授尚書左司員外郎,寄班閣門。



 天輔七年,太祖取燕,筈從其父兄出降,遷尚書左司郎中。八年,授殿中少監。太祖崩,宋,夏遣使弔慰,凡館見禮儀皆筈詳定。遷衛尉少卿,授西上閣門使,仍從事元帥府。元帥府以便宜從事,凡約束廢置及四方號令多從筈之畫焉。



 天會二年,遷太常少卿、東上閣門使,從宗翰伐宋,圍太原。遷衛尉卿,權簽宣徽院事。四年,授左諫議大夫。秋,復南征,權中書省樞密院事。丁父憂,明年起復,直樞密院事加給事中。七年,為禮部侍郎。十年,改彰信軍節度使,權簽中書省樞密院事。



 天眷
 二年,改左宣徽使,熙宗幸燕,法駕儀仗筈討論者為多。皇統二年,充江南封冊使,假中書侍郎。既至臨安,而宋人榜其居曰「行宮」,筈曰:「未受命,而名行宮,非也。」請去榜而後行禮。宋人驚服其有識,欲厚賄說之,奉金珠三十餘萬,而筈不之顧,皆歎曰:「大國有人焉。」



 六年,為行臺尚書右丞相,兼判左宣徽使事,留京師。或請釐革河南官吏之濫雜者,筈曰:「廢齊用兵江表,求一切近效,其所用人不必皆以章程,故有不由科目而為大吏,不試弓馬而握兵柄者。今撫定未久,姑收人心,奈何為是紛更也。」遂仍其舊。



 七年,帥府議於館陶築三城,以為有警即令
 北軍入居之。筈曰:「今天下一家,孰為南北。設或有變,軍人入城,獨能安耶。當嚴武備以察姦,無示彼此之間也。」其後,竟從筈議。初,以河外三州賜夏人,或言秦之在夏者數千人,皆願來歸。諸將請約之,筈曰:「三小州不足為輕重,恐失朝廷大信。且秦人之在蜀者倍多於此,何獨捨彼而取此乎。」遂從筈議。陜西邊帥請完沿邊城郭以備南寇,筈曰:「我利車騎而不利城守。今城之,則勞民而結怨。況盟已定,豈可妄動。」遂罷之。



 九年八月,拜司空。九月,拜平章政事,封吳國公,行臺右丞相如故。天德元年,封滕王。二年,拜尚書右丞相兼中書令,進封鄭王。未幾,
 以疾求解政務,授燕京留守,進封曹王。居數月,乞致仕。筈自為宣徽使,以能得悼后意,致位宰相。海陵即位,意頗鄙之。及筈求致仕,詔略曰:「不為暗於臨事,不為諂於事君。未許告歸,姑從解職。」筈因慚懼而死,年五十八。子仲誨。



 仲誨字子忠。皇統初,以宰相子授忠勇校尉。九年,賜進士第,除應奉翰林文字。海陵嚴暴,臣下應對多失次。嘗以時政訪問在朝官,仲誨從容敷奏,無懼色,海陵稱賞之。貞元初,丁父憂,起復翰林修撰。大定二年,遷待制,尋兼修起居注、左補闕。



 三年,詔仲誨與左司員外郎蒲察
 蒲速越廉問所過州縣,仲誨等還奏狀,詔玉田縣令李方進一階,順州知法、權密雲縣事王宗永擢密雲縣尉,順州司候張璘、密雲縣尉石抹烏者皆免去。丁母憂,起復太子右諭德,遷翰林直學士、改棣州防禦使。厭次縣捕得強盜數十人,詣州欲以全獲希賞。仲誨疑其有冤,緩其獄。同僚曰:「縣境多盜,請置之法,以懲其餘。」仲誨乃擇老稚者先釋之。未幾,乃獲真盜。



 入為禮部侍郎兼左諭德,遷太子詹事兼左諫議大夫。上曰:「東宮官屬,尤當選用正人,如行檢不修及不稱位者,具以名聞。」又曰:「東宮講書或論議間,當以孝儉德行正身之事告之。」頃之,
 東宮請增牧人及張設什用,上謂仲誨曰:「太子生於富貴,每教之恭儉。朕服御未嘗妄有增益,卿以此意諭之。」改御史中丞。



 十四年,為宋國歲元使,宋主欲變親起接書之儀,遣館伴王抃來議,曲辨強說,欲要以必從。仲誨曰:「使臣奉命,遠來修好,固欲成禮,而信約所載,非使臣輒敢變更。公等宋國腹心,毋僥倖一時,失大國歡。」往復再三,竟用舊儀,親起接書成禮而還。



 復為太子詹事,遷吏部尚書,轉太子少師兼御史中丞。坐失糾舉大長公主事,與侍御史李瑜各削一階。仲誨前後為東宮官且十五年,多進規戒,顯宗特加禮敬。大定十九年,卒。



 仲誨
 立朝峻整,容色莊重,世宗嘗曰:「朕見劉仲誨嘗若將切諫者。」其以剛嚴見知如此。



 頍字元矩。以大臣子孫充閣門祗候,調莘縣令,召為承奉班都知,遷西上閣門副使兼宮苑令,累遷西上、東上閣門使。泰和二年,宋盱眙軍報:明年賀正旦使魯、楊明輝。及過界,副使乃王處久。入見,魯殿上不雙跪。詔就閣詰問先報名銜楊明輝不復報改王處久之故,及不雙跪者。魯對,拜時並雙跪,有足疾似單跪者。初,南苑有唐舊碑,書「貞元十年御史大夫劉怦葬」。上見之曰:「苑中不宜有墓。」頍家本怦後,詔賜頍錢三百貫改葬
 之。三遷右宣徽使。貞祐二年,轉左宣徽使。明年,致仕,遷一官。上曰:「卿舊人也,今朝廷多故,豈宜去位。朕自東宮薨後,思慮不周,俟稍寧息,即以上郡處卿。」頃之,起為知開封府。四年正月元日,攝左宣徽使。再請老,未半歲復起為御史中丞。詔安撫河南路,捕盜賊。坐與保靜軍節度使會飲,解職。起為太子詹事,遷太子少師。詹事院欲闢廣東宮周墻,頍請於皇太子曰:「師旅饑饉之際,何為興此役。」遂止。尋卒。



 時立愛,字昌壽,涿州新城人。父承謙,以財雄鄉里,歲饑發倉廩賑貧乏,假貸者與之折券。遼太康九年,中進士
 第,調泰州幕官。丁父憂,服除,調同知春州事。未逾年,遷雲內縣令,再除文德令。樞密院選為吏房副都承旨,轉都承旨。累遷御史中丞,剛正敢言,忤權貴。除燕京副留守,丁母憂,起復舊職,遷遼興軍節度使兼漢軍都統。



 太祖已定燕京,訪求得平州人韓詢持詔招諭平州。是時,奚王回離保在盧能嶺,立愛未敢即朝見,先使人來送款曰:「民情愚執,不即順從,願降寬恩,以慰反側。」詔曰:「朕親巡西土,底定全燕,號令所加,城邑皆下。爰嘉忠款,特示優恩,應在彼大小官員可皆充舊職,諸囚禁配隸並從釋免。」於是,遼帝尚在天德,平州雖降,民心未固。奚王
 回離保軍所在保聚,薊州已降復叛。民間流言謂:「金人所下城邑,始則存撫,後則俘掠。」時立愛雖開諭而不肯信,乃上表:「乞下明詔,遣官分行郡邑,宣諭德義。他日兵臨于宋,順則撫之,逆則討之,兵不勞而天下定矣。」上覽表嘉之,詔答曰:「卿始率吏民歸附,復條利害,悉合朕意,嘉歎不忘。山西部族緣遼主未獲,恐陰相連結,故遷處于嶺東。西京人民既無異望,皆按堵如故。或有將卒貪悍,冒犯紀律,輒掠降人者。已諭諸部及軍帥,約束兵士,秋毫有犯,必刑無赦。今遣斡羅阿里等為卿副貳,以撫斯民,其告諭所部,使知朕意。」



 其後,以平州為南京,用張
 覺為留守,時立愛遂去平州。而張覺遂因燕京人東徙,其眾怨望,覺遂叛入于宋。立愛既去平州歸鄉里,太祖以燕、薊與宋,新城入于宋。宋累詔立愛,立愛見宋政日壞,不肯起,戒其宗族不得求仕。



 及宗望再取燕山,立愛詣幕府上謁,拜同中書門下平章事,任其子侄數人。立愛從宗望軍數年,謀畫居多,封陳國公。表求解機務,不從。九年,為侍中、知樞密院事。久之,加中書令。天會十五年,致仕,加開府儀同三司、鄭國公。薨于家,年八十二。賻贈錢布繒帛有差。詔同簽書燕京樞密院事趙慶襲護喪事,葬用皆官給之。



 韓企先,燕京人。九世祖知古,仕遼為中書令,徙居柳城,世貴顯。乾統間,企先中進士第,回翔不振。都統杲定中京,擢樞密副都承旨,稍遷轉運使。宗翰為都統經略山西,表署西京留守。天會六年,劉彥宗薨,企先代之,同中書門下平章事、知樞密院事。七年,遷尚書左僕射兼侍中,封楚國公。



 初,太祖定燕京,始用漢官宰相賞左企弓等,置中書省、樞密院于廣寧府,而朝廷宰相自用女直官號。太宗初年,無所改更。及張敦固伏誅,移置中書、樞密于平州,蔡靖以燕山降,移置燕京,凡漢地選授調發租稅皆承制行之。故自時立愛、劉彥宗及企先輩,官為
 宰相,其職大抵如此。斜也、宗乾當國,勸太宗改女直舊制,用漢官制度。天會四年,始定官制,立尚書省以下諸司府寺。



 十二年,以企先為尚書右丞相,召至上京。入見,太宗甚驚異曰:「朕疇昔嘗夢此人,今果見之。」於是,方議禮制度,損益舊章。企先博通經史,知前代故事,或因或革,咸取折衷。企先為相,每欲為官擇人,專以培植獎勵後進為己責任。推轂士類,甄別人物,一時臺省多君子。彌縫闕漏,密謨顯諫,必咨於王。宗翰、宗乾雅敬重之,世稱賢相焉。皇統元年,封濮王。六年,薨,年六十五。正隆二年,例降封齊國公。大定八年,配享太宗廟廷。



 十年,司空
 李德固孫引慶求襲其祖猛安,世宗曰:「德固無功,其猛安且闕之。漢人宰相惟韓企先最賢,他不及也。」十一年,將圖功臣像于衍慶宮,上曰:「丞相企先,本朝典章制度多出斯人之手,至於關決大政,與大臣謀議,不使外人知之,由是無人能知其功。前後漢人宰相無能及者,置功臣畫像中,亦足以示勸後人。」十五年,謚簡懿。



 韓鐸字振文,企先次子也。皇統末,以大臣子授武義將軍。熙宗聞其有儒學,賜進士第,除宣徽判官。再遷刑部員外郎,海陵遣中使諭之曰:「郎官,高選也。汝勳賢之子,行已蒞官,能世其家,故以命汝。茍能夙夜在公,當不次
 擢用,雖公相可到。」鐸感奮,獄或有疑,據經議讞。海陵伐宋,改兵部員外郎。大定初,遷本部郎中,累官河州防禦使,求養親,解去。召為左諫議大夫,遷中都路都轉運使。頃之,上謂宰臣曰:「韓鐸年高,不任繁劇,且其母老矣,可與之便郡。」於是改順天軍節度使。卒。



 贊曰:太祖入燕,始用遼南、北面官僚制度。是故劉彥宗、時立愛規為施設,不見於朝廷之上。軍旅之暇,治官政,庀民事,務農積穀,內供京師,外給轉餉,此其功也。韓企先入相兩朝,幾二十年,成功著業,世宗稱其賢焉。



\end{pinyinscope}