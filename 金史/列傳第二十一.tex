\article{列傳第二十一}

\begin{pinyinscope}

 ○張通古張浩張汝霖張玄素張汝弼耶律安禮納合椿年祁宰



 張通古,字樂之,易州易縣人。讀書過目不忘,該綜經史,善屬文。遼天慶二年進士第,補樞密院令史。丁父憂,起復,懇辭不獲,因遁去,屏居興平。太祖定燕京,割以與宋。
 宋人欲收人望,召通古。通古辭謝,隱居易州太寧山下。宗望復燕京,侍中劉彥宗與通古素善,知其才,召為樞密院主奏,改兵刑房承旨。天會四年,初建尚書省,除工部侍郎,兼六部事。高慶裔設磨勘法,仕宦者多奪官,通古亦免去。遼王宗乾素知通古名,惜其才,遣人諭之使自理。通古不肯,曰:「多士皆去,而己何心,獨求用哉!」宗乾為論理之。除中京副留守,為詔諭江南使,宋主欲南面,使通古北面。通古曰:「大國之卿當小國之君。天子以河南、陜西賜之宋,宋約奉表稱臣,使者不可以北面。若欲貶損使者,使者不敢傳詔。」遂索馬欲北歸。宋主遽命設
 東西位,使者東面,宋主西面,受詔拜起皆如儀。使還,聞宋已置戍河南,謂送伴韓肖胄曰:「天子裂壤地益南國,南國當思圖報大恩。今輒置守戍,自取嫌疑,若興師問罪,將何以為辭?江左且不可保,況齊乎?」肖胄惶恐曰:「敬聞命矣。」即馳白宋主。宋主遽命罷戍。通古至上京,具以白宗乾,且曰:「及其部置未定,當議收復。」宗乾喜曰:「是吾志也。」即除參知行臺尚書省事。未幾,詔宗弼復取河南,通古請先行至汴諭之。比至汴,宋人已去矣。或謂通古曰:「宋人先退,詐也,今聞將自許、宿來襲我。」通古曰:「南人宣言來者,正所以走耳。」乃使人覘之,宋人果潰去。宗弼
 撫髀笑曰:「誰謂書生不能曉兵事哉?」



 河南卒孫進詐稱「皇弟按察大王」,謀作亂。是時海陵為相,內懷覬覦,欲先除熙宗弟胙王常勝,因孫進稱皇弟大王,遂指名為胙王以誣構之。熙宗自太子濟安薨後,繼嗣未定,深以為念。裴滿后多專制,不得肆意後宮,頗鬱鬱,因縱酒,往往迷惑妄怒,手刃殺人。及海陵中傷胙王,熙宗以為信然不疑,遣護衛特思就汴京鞫治。行臺知熙宗意在胙王,導引孫進連屬之。通古執其咎,極力辯止。及孫進引服,蓋假託名稱,將以惑眾,規取財物耳,實無其人也。特思奏狀,海陵譖之曰:「特思且將徼福於胙王。」熙宗益以海
 陵為信,遂殺胙王,并特思殺之。行臺諸人乃責通古曰:「為君所誤,今坐死矣。」通古曰:「以正獲罪死,賢於生。」海陵既殺胙王,不復緣害他人,由是坐止特思,行臺不坐。



 天德初,遷行臺左丞,進拜平章政事,封譚王,改封鄆王。以疾求解機務,不許。拜司徒,封沈王。海陵御下嚴厲,收威柄,親王大臣未嘗少假以顏色,惟見通古,必以禮貌。



 會磁州僧法寶欲去,張浩、張暉欲留之不可得,朝官又有欲留之者。海陵聞其事,詔三品以上官上殿,責之曰:「聞卿等每到寺,僧法寶正坐,卿等皆坐其側,朕甚不取。佛者本一小國王子,能輕舍富貴,自苦修行,由是成佛,今
 人崇敬。以希福利,皆妄也。況僧者,往往不第秀才,市井游食,生計不足,乃去為僧,較其貴賤,未可與簿尉抗禮。閭閻老婦,迫於死期,多歸信之。卿等位為宰輔,乃復效此,失大臣體。張司徒老成舊人,三教該通,足為儀表,何不師之?」召法寶謂之曰:「汝既為僧,去住在己,何乃使人知之?」法寶戰懼,不知所為。海陵曰:「汝為長老,當有定力,今乃畏死耶?」遂於朝堂杖之二百,張浩、張暉杖二十。



 正隆元年,以司徒致仕,進封曹王。是年,薨,年六十九。



 通古天資樂易,不為表襮,雖居宰相,自奉如寒素焉。子沉,天德三年,賜楊建中榜及第。



 張浩,字浩然,遼陽渤海人。本姓高,東明王之後。曾祖霸,仕遼而為張氏。天輔中,遼東平,浩以策干太祖,太祖以浩為承應御前文字。天會八年,賜進士及第,授秘書郎。太宗將幸東京,浩提點繕修大內,超遷衛尉卿,權簽宣徽院事,管勾御前文字,初定朝儀。求養親,去職。起為趙州刺史。官制行,以中大夫為大理卿。天眷二年,詳定內外儀式,歷戶、工、禮三部侍郎,遷禮部尚書。田玨黨事起,臺省一空,以浩行六部事。簿書叢委,決遣無留,人服其才。以疾求外,補除彰德軍節度使,遷燕京路都轉運使。俄改平陽尹。平陽多盜,臨汾男子夜掠人婦,浩捕得,榜
 殺之,盜遂衰息。近郊有淫祠,郡人頗事之。廟祝、田主爭香火之利,累年不決。浩撤其祠屋,投其像水中。強宗黠吏屏跡,莫敢犯者。郡中大治。乃繕葺堯帝祠,作擊壤遺風亭。



 海陵召為戶部尚書,拜參知政事。天德二年,丁母憂。起復參知政事,進拜尚書右丞。天德三年,廣燕京城,營建宮室。浩與燕京留守劉筈、大名尹盧彥倫監護工作,命浩就擬差除。既而暑月,工役多疾疫。詔發燕京五百里內醫者,使治療,官給藥物,全活多者與官,其次給賞,下者轉運司舉察以聞。



 貞元元年,海陵定都燕京,改燕京為中都,改析津府為大興府。浩進拜平章政事,賜
 金帶玉帶各一,賜宴于魚藻池。浩請凡四方之民欲居中都者,給復十年,以實京城,從之。拜尚書右丞相兼侍中,封潞王,賜其子汝霖進士及第。未幾,改封蜀王,進拜左丞相。正隆二年,改封魯國公。表乞致仕。海陵曰:「人君不明,諫不行,言不聽,則宰相求去。宰相老病不能任事則求去。卿於二者何居?」浩對曰:「臣羸病不堪任事,宰相非養病之地也,是以求去。」不許。



 海陵欲伐宋,將幸汴,而汴京大內失火,於是使浩與敬嗣暉營建南京宮室。浩從容奏曰:「往歲營治中都,天下樂然趨之。今民力未復,而重勞之,恐不似前時之易成也。」不聽。浩朝辭,海陵問
 用兵利害。浩不敢正諫,乃婉詞以對,欲以微止海陵用兵,奏曰:「臣觀天意,欲絕趙氏久矣。」海陵愕然曰:「何以知之?」對曰:「趙構無子,樹立疏屬,其勢必生變,可不煩用兵而服之。」海陵雖喜其言,而不能從也。浩至汴,海陵時時使宦者梁珫來視工役,凡一殿之成,費累巨萬。珫指曰:「某處不如法式。」輒撤之。浩不能抗而與之均禮。汴宮成,海陵自燕來遷居之。浩拜太傅、尚書令,進封秦國公。



 海陵至汴,累月不視朝,日治兵南伐,部署諸將。浩欲奏事,不得見。會海陵遣周福兒至浩家,浩附奏曰:「諸將皆新進少年,恐誤國事。宜求舊人練習兵者,以為千戶謀克。」
 而海陵部署已定,惡聞其言,乃杖之。海陵自將發汴京,皇后、太子居守。浩留治尚書省事。



 世宗即位于遼陽,揚州軍變,海陵遇害。都督府使使殺太子光英于南京。浩遣戶部員外郎完顏謀衍上賀表。明年二月,浩朝京師,入見。世宗謂曰:「朕思天位惟艱,夙夜惕懼,不遑寧處。卿國之元老,當戮力贊治,宜令後世稱揚德政,毋失委注之意也。」俄拜太師、尚書令,封南陽郡王。世宗曰:「卿在正隆時為首相,不能匡救,惡得無罪。營建兩宮,殫竭民力,汝亦嘗諫,故天下不以咎汝,惟怨正隆。而卿在省十餘年,練達政務,故復用卿為相,當自勉,毋負朕意。」浩頓首
 謝。居數日,世宗謂浩曰:「卿為尚書令,凡人材有可用者,當舉用之。」浩舉紇石烈志寧等,其後皆為名臣。



 浩有疾,在告者久之。遣左司郎中高衎及浩姪汝弼宣諭。浩力疾入對,即詔入朝毋拜,許設座殿陛之東,若有咨謀,然後進對。或體中不佳,不必日至省中,大政可就第裁決。浩雖受詔,然每以退為請。三年夏,復申前請。乃除判東京留守。疾不能赴任,因請致仕。



 初,近侍有欲罷科舉者,上曰:「吾見太師議之。」浩入見,上曰:「自古帝王有不用文學者乎?」浩對曰:「有。」曰:「誰歟?」浩曰:「秦始皇。」上顧左右曰:「豈可使我為始皇乎!」事遂寢。



 是歲,薨。上輟朝一日。詔左宣
 徽使趙興祥率百官致奠,賻銀千兩、重彩五十端、絹五百匹。謚曰文康。明昌五年,配享世宗廟廷。泰和元年,圖像衍慶宮。子汝為、汝霖、汝能、汝方、汝猷。



 汝霖字仲澤,少聰慧好學,浩嘗稱之曰:「吾家千里駒也。」貞元二年,賜呂忠翰榜下進士第,特授左補闕,擢大興縣令,再遷禮部員外郎、翰林待制。大定八年,除刑部郎中,召見於香閣,諭之曰:「卿以待制除郎中,勿以為降。朕以刑部闕漢官,故以授卿。且卿入仕未久,姑試其能耳。如職事修舉,當有陞擢。爾父太師以戶部尚書升諸相位,由崇德大夫躐遷金紫,卿所自見也。當既厥心,無忝
 乃父。」明年,授太子左諭德兼禮部郎中。



 先是,知登聞檢院王震改禮部郎中,世宗諭宰臣曰:「此除未允人望,禮官當選有學術士,如張汝霖者可也。」於是,命汝霖兼之而除震別職。擢刑部侍郎。以憂解,起復為太子詹事,遷太子少師兼御史中丞。世宗召謂曰:「卿嘗言,監察御史所察州縣官多因沽買以得名譽,良吏奉法不為表襮,必無所稱。朕意亦然。卿今為臺官,可革其弊。」尋改中都路都轉運使、太子少師兼禮部尚書,俄轉吏部,為御史大夫。



 時將陵主簿高德溫大收稅戶米,逮御史獄。汝霖具二法上。世宗責之曰:「朕以卿為公正,故登用之。德溫
 有人在宮掖,故朕頗詳其事。朕肯以宮掖之私撓法耶?不謂卿等顧徇如是。」汝霖跪謝。久之,上顧左諫議大夫楊伯仁曰:「臺官不正如此。」伯仁奏曰:「罪疑惟輕,故具二法上請,在陛下裁斷耳。且人材難得,與其材智而邪,不若用愚而正者。」上作色曰:「卿輩皆愚而不正者也。」未幾,復坐失出大興推官高公美罪,謫授棣州防禦使。頃之,復為太子少師兼禮部尚書。拜參知政事,太子少師如故。是日,汝霖兄汝弼亦進拜尚書左丞,時人榮之。



 後因朝奏日論事上前,世宗謂曰:「朕觀唐史,見太宗行事初甚厲精,晚年與群臣議多飾辭,朕不如是也。」又曰:「唐太
 宗,明天子也,晚年亦有過舉。朕雖不能比迹聖帝明王,然常思始終如一。今雖年高,敬慎之心無時或怠。」汝霖對曰:「古人有言,『靡不有初,鮮克有終』,有始有卒者,其惟聖人乎!魏徵所言守成難者,正謂此也。」上以為然。二十五年,章宗以原王判大興府事,上命汝霖但涓視事日且加輔導。尋坐擅支東宮諸皇孫食料,奪官一階。久之,遷尚書右丞。



 是時,世宗在位久,熟悉天下事,思得賢材與圖致治,而大臣皆依違茍且,無所薦達。一日,世宗召宰臣謂曰:「卿等職居輔相,曾無薦舉何也?且卿等老矣,殊無可以自代者乎?惟朕嘗言某人可用,然後從而言
 之。卿等既無所言,必待朕知而後進用,將復有幾?」因顧汝霖曰:「若右丞者,亦因右丞相言而知也。」汝霖對曰:「臣等茍有所知,豈敢不薦,但無人耳。」上曰:「春秋諸國分裂,土地偏小,皆稱有賢。今天下之大,豈無人才?但卿等不舉而已。今朕自勉,庶幾致治。他日子孫誰與共治乎?」汝霖等皆有慚色。二十八年,進拜平章政事,兼修國史,封芮國公。世宗不豫,與太尉徒單克寧、右丞相襄同受顧命。章宗即位。加銀青榮祿大夫,進封莘。



 先是,右丞相襄言:「熙宗聖節蓋七月七日,為係景宣忌辰,更用正月受外國賀。今天壽節在七月,雨水淫暴,外方人使赴闕,有
 礙行李,乞移他月為便。」汝霖言:「帝王之道當示信於天下。昔宋主構生日,亦係五月。是時,都在會寧,上國遣使賜禮,不聞有霖潦礙阻之說。今與宋構好日久,遽以暑雨為辭,示以不實。萬一雨水踰常,愆期到闕,猶愈更用別日。」參知政事劉瑋、御史大夫唐括貢、中丞李晏、刑部尚書兼右諫議大夫完顏守貞、修起居注完顏烏者、同知登聞檢院事孫鐸亦皆言其不可。帝初從之,既而竟用襄議。時帝在諒陰,初出獵,諫院聯章言心喪中未宜。其後冬獵,汝霖諫之。詔答曰:「卿能每事如此,朕復何憂。然時異事殊,難同古昔,如能斟酌得中,斯為當矣。」



 一日,
 帝謂宰臣曰:「今之用人,太拘資歷,如此何能得人?」汝霖奏曰:「不拘資格,所以待非常之材。」帝曰:「崔祐甫為相,未踰年薦八百人,豈皆非常材耶?」時有司言民間收藏制文,恐因而滋訟,乞禁之。汝霖謂:「王者之法,譬猶江、河,欲使易避而難犯。本朝法制,坦然明白,今已著為不刊之典,天下之人無不聞誦。若令私家收之,則人皆曉然不敢為非,亦助治之一端也。不禁為便。」詔從之。



 明昌元年三月,表乞致仕,不許。十二月,卒。時帝獵饒陽,訃聞,敕百官送葬,賻禮加厚,謚曰文襄。



 汝霖通敏習事,凡進言必揣上微意,及朋附多人為說,故言不忤而似忠也。初,章
 宗新即位,有司言改造殿庭諸陳設物,日用繡工一千二百人,二年畢事。帝以多費,意輟造。汝霖曰:「此非上服用,未為過侈。將來外國朝會,殿宇壯觀,亦國體也。」其後奢用浸廣,蓋汝霖有以導之云。



 張玄素,字子真,與浩同曾祖。祖祐,父匡,仕遼至節度使。玄素初以廕得官。高永昌據遼陽,玄素在其中。斡魯軍至,乃開門出降,特授世襲銅州猛安。天會間,歷西上閣門使、客省使、東宮計司。天眷元年,以靜江軍節度使知涿州,察廉最,進官一階。皇子魏王道濟遙領中京,以玄素為魏王府同提點,尋改鎮西軍節度使,遷東京路都
 轉運使,改興平軍節度使。正隆末年,天下盜起,玄素發民夫增築城郭,同僚諫止之,不聽。未幾,寇掠鄰郡,皆無備,而興平獨安。世宗即位,玄素來見于東京。玄素在東京,希海陵旨,言世宗嘗取在官黃糧,及摭其數事。至是來見,世宗一切不問。玄素與李石力言宜早幸燕京,上深然之。遷戶部尚書,出鎮定武,遂致仕。年八十四,卒。



 玄素厚而剛毅,人畏憚之。往往以片紙署字其上治瘧疾,輒愈,人皆異之。



 汝弼,字仲佐,父玄征,彰信軍節度使,玄素之兄也。汝弼初以父蔭補官。正隆二年,中進士第,調沈州樂郊縣主
 簿。玄徵妻高氏與世宗母貞懿皇后有屬,世宗納玄征女為次室,是為元妃。張氏生趙王允中。世宗即位于遼陽,汝弼與叔玄素俱往歸之,擢應奉翰林文字。



 世宗御翠巒閣,召左司郎中高衎及汝弼問曰:「近日除授,外議何如?宜以實奏,毋少隱也。有不可用者當改之。」衎、汝弼皆無以對。自皇統以來,內藏諸物費用無度,吏夤緣為奸,多亡失。汝弼與宮籍直長高公穆、入殿小底王添兒閱實之,以類為籍,作四庫以貯之。於是,內藏庫使王可道等皆杖一百,汝弼等各進階。頃之,兼修起居注,轉右司員外郎。母憂去官。起復吏部郎中,累遷吏部尚書,拜
 參知政事。



 詔徙女直猛安謀克于中都,給以近郊官地,皆脊薄。其腴田皆豪民久佃,遂專為己有。上出獵,猛安謀克人前訴所給地不可種藝,詔拘官田在民久佃者與之。因命汝弼議其事。請「條約立限,令百姓自陳。過限,許人首告,實者與賞。」上可其奏。仍遣同知中都轉運使張九思拘籍之。



 上問:「高麗、夏皆稱臣。使者至高麗,與王抗禮。夏王立受,使者拜,何也?」左丞襄對曰:「故遼與夏為甥舅,夏王以公主故,受使者拜。本朝與夏約和,用遼故禮,所以然耳。」汝弼曰:「誓書稱一遵遼國舊儀,今行之已四十年,不可改也。」上曰:「卿等言是也。」上聞尚書省除授
 小官多不稱職,召汝弼至香閣謂之曰:「他宰相年老,卿等宜盡心。」汝弼對曰:「材薄不足以副聖意耳。」進拜尚書右丞。於是,戶部糶官倉粟,汝弼請使暖湯院得糴之。上讓曰:「汝欲積陰德邪?何區區如此。」



 左丞相徒單克寧得解政務,為樞密使。是日,汝弼亦懷表乞致仕。上使人止之曰:「卿年未老,未可退也。」進左丞,與族弟參知政事汝霖同日拜,族里以為榮。有年未六十而乞致仕者,上不許。汝弼曰:「聖旨嘗許六十致仕。」上責之曰:「朕嘗許至六十者致仕,不許未六十者。且朕言六十致仕,是則可行,否則當言。卿等不言,皆此類也。」久之,坐擅增諸皇孫食
 料,與丞相守道、右丞粘割斡特剌、參政張汝霖各削官一階。上曰:「准法當解職,但示薄責耳。」汝弼在病告,上謂宰相曰:「汝弼久居執政,練習制度,頗能斟酌人材,而用心不正。」乃罷為廣寧尹,賜通犀帶。



 汝弼為相,不能正諫。上所欲為,則順而導之,所不欲為,則微言以觀其意。上責之,則婉辭以引過,終不忤之也。而上亦知之。且黷貨,以計取諸家名園甲第珍玩奇好,士論薄之。二十七年,薨。



 汝弼既與永中,甥舅,陰相為黨。章宗即位,汝弼妻高氏每以邪言怵永中,覬非望,畫永中母像,侍奉祈祝,使術者推算永中。有司鞫治,高氏伏誅。事連汝弼,上以事
 覺在汝弼死後,得免削奪。



 耶律安禮,本名納合,系出遙輦氏。幼孤,事母以孝聞。遼季,間關避難,未嘗一日怠溫凊。入朝,當路者重其行義,使主帥府文字,授左班殿直。天眷初,從元帥於山西。母喪,不克歸葬,主帥憐之,賻禮甚厚。安禮冒大暑,挽柩行千餘里,哀毀骨立,行路嗟歎。服除,由行臺吏、禮部主事累遷工部侍郎,改左司郎中。



 天德間,罷行臺尚書省,入為工部侍郎,累遷本部尚書。明年冬,為宋國歲元使。被詔鞫治韓王亨獄于廣寧。亨無反狀,安禮還奏。海陵怒,疑安禮梁王宗弼故吏,乃責安禮曰:「孛迭有三罪。其論
 阿里出虎有誓券不當死,既引伏。其謂不足進馬,及密遣刺客二者,安得無之?汝等來奏,欲測我喜怒以為輕重耳。」乃遣安禮再往,與李老僧同鞫之。老僧由是殺亨于獄。海陵猶謂安禮輒殺亨以絕滅事跡,親戚得以不坐。安禮之不附上刻下乃如此。



 改吏部尚書,護大房山諸陵工作。拜樞密副使,封譚國公,遷尚書右丞,進封郕國公,轉左丞。議降累朝功臣封爵,密諫伐江南,忤海陵意,罷為南京留守,封溫國公。安禮長於吏事,廉謹自將,從帥府再伐宋,寶貨人口一無所取。貴為執政,奴婢止數人,皆有契券,時議賢之。薨,年五十六。



 納合椿年,本名烏野。初置女直字,立學官於西京,椿年與諸部兒童俱入學,最號警悟。久之,選諸學生送京師,俾上京教授耶魯教之,椿年在選中。補尚書省令史,累官殿中侍御史,改監察御史。海陵為相,薦為右司員外郎,編定新制。海陵篡立,以為諫議大夫。椿年有酒失,海陵使之戒酒,遂終身不復飲。改秘書監,修起居注,授世襲猛安,為翰林學士兼御史中丞。貞元初,起上京諸猛安於中都、山東等路安置,以勞賜玉帶閑廄馬。奉遷山陵,還為都點檢。賜今名,拜參知政事。海陵謂椿年曰:「如卿吏材甚難得,復有如卿者乎?」椿年薦大理丞紇石烈
 婁室。海陵以婁室為右司員外郎。未旬日,海陵謂椿年曰:「吾試用婁室,果如卿言。惟賢知賢,信矣。」婁室後賜名良弼,有宰相才,世宗時,至左丞相,號賢相焉。



 正隆二年,椿年薨。海陵親臨哭之,追封特進、譚國公,謚忠辯,賻銀二千兩、彩百端、絹千匹、錢千萬。以長子參謀合為定遠大將軍,襲猛安,次子合答為忠武校尉。及歸葬,再賜錢百萬,仍給道路費。



 椿年有宰相才,好推挽士類,然頗營產業,為子孫慮。冒占西南路官田八百餘頃。大定中,括撿田土,百姓陳言官豪占據官地,貧民不得耕種。溫都思忠子長壽、椿年子猛安參謀合等三十餘家凡冒占
 三千餘頃。詔諸家除牛頭稅地各再給十頃,其餘盡賦貧民種佃。世頗以此譏椿年云。



 祁宰,字彥輔,江淮人。宋季以醫術補官。王師破汴得之,後隸太醫。累遷中奉大夫、太醫使。數被賞賚,常感激欲自效。海陵將伐宋,宰欲諫,不得見。會元妃有疾,召宰診視。既入見,即上疏諫,其略言:「國朝之初,祖宗以有道伐無道,曾不十年,蕩遼戡宋。當此之時,上有武元、文烈英武之君,下有宗翰、宗雄謀勇之臣,然猶不能混一區宇,舉江淮、巴蜀之地,以遺宋人。況今謀臣猛將,異於曩時。且宋人無罪,師出無名。加以大起徭役,營中都,建南京,
 繕治甲兵,調發軍旅,賦役煩重,民人怨嗟,此人事之不修也。間者晝星見於牛斗,熒惑伏於翼軫。巳歲自刑,害氣在揚州,太白未出,進兵者敗,此天時不順也。舟師水涸,舳艫不繼,而江湖島渚之間,騎士馳射,不可驅逐,此地利不便也。」言甚激切。海陵怒,命戮於市,籍其家產,天下哀之。綦戩,宰婿也,海陵疑奏疏戩為之。辭曰:「實不知也。」海陵猶杖戩。召禁中諸司局官至咸德門,諭以殺宰事。



 明年,世宗即位於遼東。四年,詔贈資政大夫,復其田宅。章宗即位,詔訪其子忠勇校尉、平定州酒監公史,擢尚藥局都監。泰和初,詔定功臣謚,尚書省掾李秉鈞上
 言:「事有宜緩而急,若輕而重者,名教是也。伏見故贈資政大夫祁宰以忠言被誅,慕義之士,盡傷厥心。世宗即位,贈之以官,陛下錄用其子,甚大惠也。雖武王比干之墓,孔子譽夷、齊之仁,何以異此。而有司拘文,以職非三品不在議謚之例,臣竊疑之。若職至三品方得請謚,當時居高官、食厚祿者,不為無人,皆畏罪淟涊,曾不敢申一喙,畫一策,以為社稷計。卒使立名死節之士,顧出於醫卜之流,亦可以少愧矣。臣以謂非常之人,當以非常之禮待之。乞詔有司特賜謚以旌其忠,斯亦助名教之一端也。」制曰:「可。」下太常,謚曰忠毅。



 贊曰:異哉,海陵之為君也,舞智御下而不恤焉。君子仕於朝,動必以禮,然後免於恥。張通古、耶律安禮位不及張浩,進退始終,其賢遠矣。浩無事不為,無役不從,為相最久,用之厚,遇之薄,豈亦自取之邪?海陵伐宋,浩、安禮位皆大臣,一以婉辭,一以密諫,賢於不諫而已。祁宰一醫流,獨能極諫,其後皆如所言。海陵戕之,足以成其百世之名耳。納合椿年援引善類,有君子風。其死適在宋兵未舉之前,然觀其好營產殖,亦未必忘身徇國之士也。祁宰卓乎不可及也夫!



\end{pinyinscope}