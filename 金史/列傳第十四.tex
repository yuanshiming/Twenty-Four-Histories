\article{列傳第十四}

\begin{pinyinscope}

 ○太宗諸子宗磐本名蒲魯虎宗固本名胡魯宗本本名阿魯蕭玉附杲本名斜也宗義本名孛吉宗乾本名斡本充本名神土懣子檀奴等永元本名元奴袞本名梧桐襄本名永慶袞本名蒲甲



 太宗子十四人:蒲魯虎、胡魯、斛魯補、阿魯帶、阿魯補、斛沙虎、阿鄰、阿魯、鶻懶、胡里甲、神土門、斛孛束、斡烈、鶻沙。



 宗磐本名蒲魯虎。天輔五年,都統杲取中京,宗磐與斡魯、宗翰宗乾皆為之副。天會十年,為國論忽魯勃極烈。熙宗即位,為尚書令,封宋國王。未幾,拜太師,與宗乾、宗翰並領三省事。



 熙宗優禮宗室,宗翰沒後,宗磐日益跋扈。嘗與宗乾爭論於上前,即上表求退。烏野奏曰:「陛下富於春秋,而大臣不協,恐非國家之福。」熙宗因為兩解。宗磐愈驕恣。其後於熙宗前持馬向宗乾,都點檢蕭仲恭呵止之。



 既而左副元帥撻懶、東京留守宗雋入朝,宗
 磐陰相黨與,而宗雋遂
 為右丞相,用事。撻懶屬尊,功多,先薦劉豫,立為齊帝,至是唱議以河南、陜西與宋,使稱臣。熙宗命群臣議,宗室大臣言其不可。宗磐、宗雋助之,卒以與宋。其後宗磐、宗雋、撻懶謀作亂,宗乾、希尹發其事,熙宗下詔誅之。坐與宴飲者,皆貶削決責有差。赦其弟斛魯補等九人,並赦撻懶,出為行臺左丞相。



 皇后生日,宰相諸王妃主命婦入賀。熙宗命去樂,曰:「宗磐等皆近屬,輒構逆謀,情不能樂也。」以黃金合及兩銀鼎獻明德宮太皇太后,並以金合、銀鼎賜宗乾、希尹焉。



 宗固本名胡魯。天會十五年為燕京留守,封豳王。宗雅本名斛魯補,封代王。宗偉本名阿魯補,封虞王。宗英本名斛沙虎,封滕王。宗懿本名阿鄰,封薛王。宗本本名阿魯,封厚王。鶻懶封翼王。宗美本名胡里甲,封豐王。神土門封鄆王。斛孛束封霍王。斡烈封蔡王。宗哲本名鶻沙,封畢王。皆天眷元年受封。宗順本名阿魯帶,天會二年薨,皇統五年贈金紫光祿大夫,後封徐王。



 宗磐既誅,熙宗使宗固子京往燕京慰諭宗固。既而翼王鶻懶復與行臺左丞相撻懶謀反伏誅。詔曰:「燕京留守豳王宗固等或謂當絕屬籍,朕所不忍。宗固等但不得稱皇叔,其母妻封號從而降者,審依舊典。」皇統二年,復封宗雅為代王。宗固為判大宗正,六年,為太保、右丞相兼中書令。是歲,薨。



 海陵在熙宗時,見太宗諸子勢彊,而宗磐尤跋扈,與鶻懶相繼皆以逆誅,心忌之。熙宗厚於宗室,禮遇不衰。海陵嘗與秉德、唐括辯私議,主上不宜寵遇太宗諸子太甚。及篡立,謁
 奠太廟。韓王亨素號材武,使攝右衛將軍,密諭之曰:「爾勿以此職為輕,朕疑太宗諸子太強,得卿衛左右,可無慮耳。」遂與秘書監蕭裕謀去宗本兄弟。太宗子孫於是焉盡,語在《宗本傳》中。



 宗本本名阿魯。皇統九年,為右丞相兼中書令,進太保,鄰三省事。海陵篡立,進太傅,領三省事。



 初,宗乾謀誅宗磐,故海陵心忌太宗諸子。熙宗時,海陵私議宗本等勢強,主上不宜優寵太甚。及篡立,猜忌益深,遂與秘書監蕭裕謀殺太宗諸子。誣以秉德出領行臺,與宗本別,因會飲,約內外相應。使尚書省令史蕭玉告宗本親謂玉言:「以汝於我故舊,必無它意,可布腹心事。鄰省臨行,言彼在外諭說軍民,無以外患為慮。若太傅為內應,何事不成。」又云:「長子鎖里虎當大貴,因是不令見主上。」宗本又言:「左丞相於我及我妃處,稱主上近日見之輒不喜,
 故心常恐懼,若太傅一日得大位,此心方安。」唐括辯謂宗本言:「內侍張彥善相,相太傅有天子分。」宗本答曰:「宗本有兄東京留守在,宗本何能為是。」時宗美言「太傅正是太宗主家子,秪太傅便合為北京留守。」卞臨行與宗本言「事不可遲」。宗本與玉言「大計只於日近圍場內予決」。宗本因以馬一匹、袍一領與玉,充表識物。玉恐圍場日近,身縻於外,不能親奏,遂以告祕書監蕭裕。裕具以聞。



 蕭玉出入宗本家,親信如家人。海陵既與蕭裕謀殺宗本、秉德,詔天下,恐天下以宗本、秉德輦皆懿親大臣,本無反狀,裕構成其事,而蕭玉與宗本厚,人所共知,使
 玉上變,庶可示信。於是使人召宗本等擊鞠,海陵先登樓,命左衛將軍徒單特思及蕭裕妹婿近侍局副使耶律闢離刺小底密伺宗本及判大宗正事宗美,至,即殺之。宗美本名胡里甲,臨死神色不變。



 宗本已死,蕭裕使人召蕭玉。是日,玉送客出城,醉酒,露髮披衣,以車載至裕弟點檢蕭祚家。逮日暮,玉酒醒,見軍士圍守之,意為人所累得罪,故至此。以頭觸屋壁,號咷曰:「臣未嘗犯罪,老母年七十,願哀憐之。」裕附耳告之曰:「主上以宗本諸人不可留,已誅之矣,欲加以反罪,令汝主告其事。今書汝告款已具,上即問汝,汝但言宗本輩反如狀,勿復異
 詞,恐禍及汝家也。」裕乃以巾服與玉,引見海陵。海陵問玉。玉言宗本反,具如裕所教。



 海陵遺使殺東京留守宗懿、北京留守卞。及遷益都尹畢王宗哲、平陽尹稟、左宣徽使京等,家屬分置別所,止聽各以奴婢五人自隨。既而使人要之於路,并其子男無少長皆殺之。而中京留守宗雅喜事佛,世稱「善大王」,海陵知其無能,將存之以奉太宗。後召至關,不數日,竟殺之。太宗子孫死者七十餘人,太宗後遂絕。卞本名可喜。稟本名胡離改。京,宗固子,本名胡石賚。



 蕭玉既如蕭裕教對海陵,海陵遂以宗本、秉德等罪詔天下,以玉上變實之。



 海陵使太府監完
 顏馮六籍宗本諸家,戒之曰:「珠玉金帛入於官,什器吾將分賜諸臣。」馮六以此不復拘籍什器,往往為人持去,馮六家童亦取其檀木屏風。少監劉景前為監丞時,太府監失火,案牘盡焚毀,數月方取諸司簿帳補之,監吏坐是稽緩,當得罪。景為吏,倒署年月。太倉都監焦子忠與景有舊,坐逋負,久不得調,景為盡力出之。久之,馮六與景就宮中相忿爭,馮六言景倒署年月及出焦子忠事。御史劾奏景,景黨誘馮六家奴發盜屏事。馮六自陳於尚書省。海陵使御史大夫趙資福、大理少卿許竑雜治。資福等奏馮六非自盜,又嘗自首。海陵素惡馮六與
 宗室游從,謂宰臣曰:「馮六嘗用所盜物,其自首不及此。法,盜宮中物者死,諸物已籍入官,與宮中物何異。」謂馮六曰:「太府掌宮中財賄,汝當防制姦欺,而自用盜物。」於是,馮六棄市,資福、竑坐鞫獄不盡,決杖有差。景亦伏受焦子忠賂金。海陵曰:「受金事無左驗,景倒署年月,以免吏罪,是不可恕。」遂殺之。



 大定二年,追封宗固魯王、宗雅曹王、宗順隋王、宗懿鄭王、宗美衛王、宗哲韓王、宗本潞王、神土門豳王、斛孛束瀋王、斡烈鄂王,胡里改、胡什賚、可喜並贈金吾衛上將軍,惟宗磐、阿魯補、斛沙虎、鶻懶四人不復加封。



 蕭玉,奚人。既從蕭裕誣宗本罪,海陵喜甚,自尚書省令史為禮部尚書加特進,賜錢二千萬、馬五百匹、牛五百頭、羊千口,數月為參知政事。丁母憂,以參政起復,俄授猛安,子尚公主。海陵謂玉曰:「朕始得天下,常患太宗諸子方強,賴社稷之靈,卿發其姦。朕無以報此功,使朕女為卿男婦,代朕事卿也。」賜第一區,分宗本家貲賜之。頃之,代張浩為尚書右丞,拜平章政事,進拜右丞相,封陳國公。



 文思署令閻拱與太子詹事張安妻坐姦事,獄具,不應訊而訊之。海陵怒,玉與左丞蔡松年、右丞耶律安禮、御史中丞馬諷決杖有差。玉等入謝罪。海陵曰:「為人
 臣以己意愛憎,妄作威福,使人畏之。如唐魏徵、狄仁傑、姚崇、宋璟,豈肯立威使人畏哉,楊國忠之徒乃立威使人畏耳。」顧謂左司郎中吾帶、右司郎中梁球曰:「往者德宗為相,蕭斛律為左司郎中,趙德恭為右司朗中,除吏議法,多用己意。汝等能不以己意愛憎為予奪輕重,不亦善乎。朕信任汝等,有過則決責之,亦非得已。古者大臣有罪,貶謫數千里外,往來疲於奔走,有死道路者。朕則不然,有過則杖之,已杖則任之如初。如有不可恕,或處之死,亦未可知。汝等自勉。」



 正隆三年,拜司徒,判大宗正事。五年,玉以司徒兼御史大夫。使參知政事李通諭
 旨曰:「判宗正之職固重,御史大夫尤難其人。朕將行幸南京,官吏多不法受賕,卿宜專糾劾,細務非所責也。御史大夫與宰執不相遠,朕至南京,徐當思之。」繼以司徒判大興尹,玉固辭司徒。海陵曰:「朕將南巡,京師地重,非大臣不能鎮撫,留卿居守,無為多讓。」海陵至南京,以玉為尚書左丞相,進封吳國公。



 海陵將伐宋,因賜群臣宴,顧謂玉曰:「卿嘗讀書否?」對曰:「亦嘗觀之。」中宴,海陵起,即召玉至內閤,因以《漢書》一冊示玉。既而擲之曰:「此非所問也,朕欲與卿議事。朕今欲伐江南,卿以為如何?」玉對曰:「不可。」海陵曰:朕視宋國猶掌握間耳,何為不可。」玉曰:「
 天以長江限南北,舟楫非我所長。苻堅百萬伐晉,不能以一騎渡,以是知其不可。」海陵怒,叱之使出。及張浩因周福兒附奏,海陵杖張浩,并杖玉。因謂群臣曰:「浩大臣,不面奏,因人達語,輕易如此。玉以苻堅比朕,朕欲斷其舌,釘而礫之,以玉有功,隱忍至今。大臣決責,痛及爾體,如在朕躬,有不能已者,汝等悉之。」



 及海陵自將發南京,玉與張浩留治省事。世宗即位,降奉國上將軍,放歸田里,奪所賜家產。久之,起為孟州防御使。世宗戒之曰:「昔海陵欲殺太宗子孫,借汝為證,遂被進用。朕思海陵肆虐,先殺宗本諸人,然後用汝質成其事,豈得專罪汝等。
 今復用汝,當思改過。若謂嘗居要地,以今日為不足,必罰無赦。」轉定海軍節度使,改太原尹,與少尹烏古論掃喝互訟不公事,各削一官,解職,尋卒。



 子德用。大定二十四年,尚書省奏玉子德用當升除,上曰:「海陵假口於玉以快其毒,玉子豈可升除邪。」



 贊曰:宗磐嘗從斜也取中京,不可謂無勞伐者,世祿鮮禮,自古有之,在國家善為保全之道耳。熙宗殺宗磐而存恤其母后,雖云矯情,猶畏物論。海陵造謀,殺宗本兄弟不遺餘力。太宗舉宋而有中原,金百世不遷之廟也,再傳而無噍類,於是太祖之美意無復幾微存者。春秋
 之世,宋公舍與夷而立其弟,禍延數世,害及五國,誠足為後世監乎。



 杲本名斜也,世祖第五子,太祖母弟。收國元年,太宗為諳班勃極烈,杲為國論吳勃極烈。天輔元年,杲以兵一萬攻泰州,下金山縣,女固、脾室四部及渤海人皆來降,遂克泰州。城中積粟轉致烏林野,賑先降諸部,因徙之內地。



 天輔五年,為忽魯勃極烈,都統內外諸軍,取中京實北京也,蒲家奴、宗翰、宗乾、宗磐副之,宗峻領合扎猛安,皆受金牌,耶律餘睹為鄉導。詔曰:「遼政不綱,人神共棄。今欲中外一統,故命汝率大軍,以行討伐。爾其慎重兵事,
 擇用善謀。賞罰必行,糧餉必繼。勿擾降服,勿縱俘掠。見可而進,無淹師期。事有從權,毋煩奏稟。」復詔曰:「若克中京,所得禮樂圖書文籍,並先次律發赴闕。」



 當是時,遼人守中京者,聞知師期,焚芻糧,欲徙居民遁去。奚王霞未則欲視我兵少則迎戰,若不敵則退保山西。杲知遼人無鬥志,乃委輜重,以輕兵擊之。六年正月,克高、恩回紇三城,進至中京。遼兵皆不戰而潰,遂克中京。獲馬一千二百、牛五百、駝一百七十、羊四萬七千、車三百五十兩。乃分兵屯守要害之地。駐兵中京,使使奏捷、獻俘。詔曰:「汝等提兵于外,克副所任,攻下城邑,撫安人民,朕甚嘉
 之。分遣將士招降山前諸部,計已撫定。山後若未可往,即營田牧,俟秋大舉,更當熟議,見可則行。如欲益兵,具數來上。無恃一戰之勝,輒自弛慢。善撫存降附,宣諭將士,使知朕意。」



 完顏歡都游兵出中京南,遇騎兵三十餘紿曰:「乞明旦來降於此。」杲信之,使溫迪痕阿里出、納合鈍恩、蒲察婆羅偎、諸甲拔剔鄰往迎之。奚王霞末兵圍阿里出等。遂據阪去馬,皆殊死戰,敗霞末兵,追殺至暮而還。是役,納合鈍恩功為多。



 宗翰降北安州,希尹獲遼護衛習泥烈,言遼主在鴛鴦濼畋獵,可襲取之。宗翰移書于杲,請進兵。使者再往,曰:「一失機會,事難圖矣。」杲意
 尚未決。宗乾勸杲當從宗翰策,杲乃約宗翰會奚王嶺。既會,始定議,杲出青嶺,宗翰出瓢嶺,期羊城濼會軍。時遼主在草濼,使宗翰與宗乾率精兵六千襲之。遼主西走,其都統馬哥趨搗里。宗翰遣撻懶以兵一千往擊之。撻懶請益兵於都統杲,而獲遼樞密使得里底父子。



 西京已降得叛,杲使招之不從,遂攻之。留守蕭察刺踰城降。四月,復取西京。杲率大軍趨白水濼,分遣諸將招撫未降州郡及諸部族。於是,遼秦晉國王耶律捏里自立于燕京。山西諸城雖降,而人心未固,杲遣宗望奏事,仍請上臨軍。耶律坦招西南招討司及所屬諸部,西至
 夏境皆降,耶律佛頂亦降于坦。金肅、西平二郡漢軍四千叛去,坦與阿沙兀野、撻不野簡料新降丁壯,迨夜襲之。詰旦,戰于河上,大敗其眾,皆委仗就擒。



 耶律捏里移書於杲請和,杲復書,責以不先稟命上國,輒稱大號,若能自歸,當以燕京留守處之。捏里復以書來,其略曰:「昨即位時,在兩國絕聘交兵之際。奚王與文武百官同心推戴,何暇請命。今諸軍已集,儻欲加兵,未能束手待斃也。昔我先世,未嘗殘害大金人民,寵以位呈,日益強大。今忘此施,欲絕我宗祀,於義何如也。儻蒙惠顧,則感戴恩德,何有窮已。」杲復書曰L:「閤下向為元帥,總統諸軍,任
 非不重,竟無尺寸之功。欲據一城,以抗國兵,不亦難乎。所任用者,前既不能死國,今誰肯為閤下用者。而云主辱臣死,欲恃此以成功,計亦疏矣。幕府奉詔,歸者官之,逆者討之。若執迷不從,期于殄滅而後已。」捏里乃遣使請于太祖。賜捏里詔曰:「汝,遼之所屬,位居將相,不能與國存亡,乃竊據孤城,僭稱大號,若不降附,將有後悔。」



 六月,上發京師,詔都統曰:「汝等欲朕親征,已於今月朔旦啟行。遼主今定何在?何計可以取之,其具以聞。」杲使馬和尚奉迎太祖於撻魯河。斡魯、婁室敗夏將李良輔,杲使完顏希尹等奏捷,且請徙西南招討司諸部于內地。
 希尹等見上于大濼西南,上嘉賞之。上至鴛鴦濼,杲上謁。上追遼主至回離畛川,南伐燕京,次奉聖州。詔曰:「自今諸訴訟書付都統杲決遣。若有大疑,即令聞奏。」太祖定燕京,還次鴛鴦濼,以宗翰為都統,杲從上還京師。



 太宗即位,杲為諳班勃極烈,與宗乾俱治國政。天會三年伐宋,杲領都元帥,居京師。宗翰、宗望分道進兵。四年,再伐宋,獲宋二主以歸。



 天會八年,薨。皇統三年,追封遼越國王。天德二年,酏享太祖廟廷。正隆例封遼王。大定十五年,謚曰智烈。子孛吉。



 宗義本名孛吉,斜也之第九子。天德間,為平章政事。



 海
 陵已殺太宗子孫,尤忌斜也諸子盛強,欲盡除宗室勛舊大臣。是時,左副元帥撒離喝在汴京與撻不野有隙,撻不野女為海陵妃,海陵陰使撻不野圖撒離喝。於是都元帥府令史遙設迎合風指,詐為撒離喝與其子宗安家書,宗女誤遺宮外,遙設因拾得之,以上變。其書契丹小字,其封題已開。其中白紙一幅,有白字隱約,狀若經水浸,致字畫可讀者,上有撒離喝手署及某王印。書辭云:「阿渾,汝安樂否。早晚到闕下。前者走馬來時,曾議論我教汝阿渾平章、謀里野阿渾等處覷事勢再通往來,緩急圖謀,知汝已嘗備細言之。謀里野阿渾所言
 煞是,只殺撻不野則南路無憂慮矣。」詳略互見《撒離喝傳》中。女直謂子「阿渾」。前「阿渾」謂撒離喝子,其子宗安。後「阿渾平章」指宗義,宗義本宗室子,猶有舊稱。以是殺宗義、謀里野,并殺宗安及太祖妃蕭氏、任王隈喝及魏王斡帶孫活里甲。遙設詐書無活里甲,海陵見其坦率善脩飾,惡之。大臣以無罪為請,海陵曰:「第殺之,無復言也。」殺斜也子孫百餘人,謀里野子孫二十餘人。謀里野,景祖孫,謾都訶次子。



 斜也有幼子阿虎里,其妻撻不野女,海陵妃大氏女兄。將殺阿虎里,使者不忍見其面,以衾覆而縊之,當其頤,久不死,及去被再縊之,海陵遣使赦其
 死,遂得免。後封為王,授世襲千戶。



 大定初,追復宗義官爵,贈特進。弟蒲馬、孛論出、阿魯、隈喝並贈龍虎衛上將軍。



 宗乾本名斡本,太祖庶長子。太祖伐遼,遼人來禦,遇於境上。使宗乾率眾先往填塹,士卒畢渡。渤海軍馳突而前,左翼七謀克少卻,遂犯中軍。杲輒出戰,太祖曰:「遇大敵不可易也。」使宗乾止杲。宗乾馳出杲前,控止導騎哲垤之馬,杲乃還。達魯古城之戰,宗乾以中軍為疑兵。太祖既攻下黃龍府,即欲取春州。遼主聞黃龍不守,大懼,即自將,籍宗戚豪右少年與四方勇士及能言兵者,皆
 隸軍中。宗乾勸太祖毋攻春州,休息士卒。太祖以為然,遂班師。



 宗乾得降人,言春、泰州無守備,可取。於是斜也取春、泰州,宗雄、宗乾等下金山縣。宗雄即以兵三千屬宗乾,招集未降諸部。宗乾擇土人之材幹者,以詔書諭之。於是女固、脾室四部及渤海人皆降。



 太祖克臨潢府,至沃黑河。宗乾諫曰:「地遠時暑,士罷馬乏,若深入敵境,糧餫不繼,恐有後艱。」上從之,遂班師。從都統杲取中京。宗翰自北安州移書于杲。是時,希尹獲遼人,知遼主在鴛鴦濼,可襲取之。杲不能決。宗翰使再至。宗乾謂杲曰:「移賚勃極烈灼見事機,再使來請,彼必不輕舉。且彼已
 發兵,不可中止,請從其策。」再三言之,杲乃報宗翰會奚王嶺。當時無宗乾,杲終無進兵意。既會軍於羊城濼,杲使宗乾與宗翰以精兵六千襲遼至五院司。遼主已遁去,與遼將耿守忠戰于西京城東四十里。守忠敗走。



 太宗即位,宗乾為國論勃極烈,與斜也同輔政。天會三年,獲遼主于應州西餘睹谷。始議禮制度,正官名,定服色,興庠序,設選舉,治歷明時,皆自宗乾啟之。四年,官制行,詔中外。



 十年,熙宗為諳班勃極烈,宗乾為國論左勃極烈。熙宗即位,拜太傅,與宗翰等並領三省事。天眷二年,進太師,封梁宋國王,入朝不拜,策杖上殿,仍以杖賜之。
 宗乾有足疾,詔設坐奏事。無何,監修國史。皇統元年,賜宗乾輦輿上殿,制詔不名。



 上幸燕京,宗乾從。有疾,上親臨問。自燕京還,至野狐嶺,宗乾疾亟不行,上親臨問,語及軍國事,上悲泣不已。明日,上及后同往視,后親與宗乾饋食,至暮而還。因赦罪囚,與宗乾禳疾。居數日,薨。上哭之慟,輟朝七日。大臣死輟朝,自宗乾始。上致祭,是日庚戌,太史奏戌亥不宜哭,上不聽曰:「朕幼沖時,太師有保傅之力,安得不哭。」哭之慟。上生日不舉樂。上還上京,幸其第視殯事。及喪至上京,上臨哭之。及葬,臨視之。



 海陵篡立,追謚憲古弘道文昭武烈章孝睿明皇帝,廟號
 德宗,以故第為興聖宮。大定二年,除去廟號,改謚明肅皇帝。及海陵廢為庶人,二十二年,皇太子允恭奏,略曰:「追惟熙宗世嫡統緒,海陵無道,弒帝自立,崇正昭穆,削其斷王,俾齒庶人之列。瘞之閑曠,不封不樹,既已申大義而明至公矣。海陵追崇其親,逆配於廟。今海陵既廢為庶人,而明肅猶竊帝尊之名,列廟祧之數。海陵大逆,正名定罪,明肅亦當緣坐。是時明肅已殂,不與於亂,臣以謂當削去明肅帝號,止從舊爵。或從太祖諸王有功例,加以官封,明詔中外,俾知大義。」書奏,世宗嘉納,下尚書省議。於是追削明肅帝號,封為皇伯、太師、遼王,謚忠
 烈,妻子諸孫皆從降。明昌四年,配享太祖廟廷。



 子棄、亮、兗、襄、袞。亮,是為海陵庶人。



 充本名神士懣。母李氏,徒單氏以為己子。熙宗初,加光祿大夫。天眷間,為汴京留守。皇統間,封淄國公,為吏部尚書,進封代王,遷同判大宗正事。九年,拜左丞相。是歲,薨。追封鄭王。大定二十二年,追降儀同三司、左丞相。子檀奴、元奴、耶補兒、阿里白。



 檀奴,為歸德軍節度使。阿里白,定遠大將軍、和魯忽土猛安忽鄰河謀克。海陵弒徒單氏,以充嘗為徒單養子,因并殺檀奴及阿里白。元奴、耶補兒逃歸於世宗。檀奴
 贈榮祿大夫,阿里白輔國上將軍。詔有司改葬。世宗時,元奴為宗正丞;耶補兒為鎮國上將軍,後為同知濟南尹事。



 永元字惇禮,本名元奴。幼聰敏,日誦千言。皇統元年,試宗室子作詩,永元中格。善《左氏春秋》,通其大義。天德初,授百女山世襲謀克。



 海陵伐宋,已渡淮,軍士多亡歸而契丹叛,由是疑宗室益甚。已殺永元弟檀奴、阿里白,永元與弟耶補兒逃匿得免。



 世宗即位于遼陽,與耶補兒俱來歸,上慰勞甚厚。授宗正丞,改符寶郎,為灤州刺史。授世襲猛安,乞以謀克與耶補兒,詔許之。轉棣州防禦
 使、泰寧軍節度使。



 張弘信通檢山東,專以多得民間物力為功,督責苛急。永元面責弘信曰:「朝廷以差調不均,立通檢法。今使者所至,以殘酷妄加農民田產,箠擊百姓有至死者。市肆賈販貿易有贏虧,田園屋宇利入有多寡,故官子孫閉門自守,使與商賈同處上役,豈立法本意哉。」弘信無以對。於是棣州賦稅得以實自占。遷震武軍節度使。



 大定六年,丁母憂,起復崇義軍節度使,徙順義軍。朔州西境多盜,而猾吏大姓蠹獄訟,[B139]亂賦役,永元剔其宿姦,百姓安之。坐賣馬與驛人取贏利,及浚州防禦使斡論坐縱孳畜踐民田,俱解職。頃之,永元起
 為保大軍節度使,歷昭義、絳陽、震武軍,遷濟南尹、北京副留守。



 燈國家婢醜底與咸平人化胡有姦,醜底於主印處紹取印署空紙與化胡,遂寫作永元、寧國生寧日時辰,誣告永元、寧國謀逆。詔有司鞫問,乃醜底意望為良,使化胡為之。上曰:「化胡與醜底有姦,造作惡言,誣害宗室,化胡斬,醜底處死。」改興中尹,為彰德軍節度使。卒官,年五十一。喪過中都,遣使致祭,賻銀三百兩、彩十端、絹百匹。



 永元歷典大籓,多知民間利害,所至稱治,相、棣、順義政跡尤著,其民並為立祠。



 兗本名梧桐。皇統七年,為左副點檢,轉都點檢。九年,為
 會寧牧,改左宣徽使。海陵篡立,兗使宋還,拜司徒兼都元帥,領三省事,進拜太尉。及殺太祖妃蕭氏,盡以其財產賜兗。罷都元帥府,立樞密院,兗為樞密使,太尉、領三省事如故。天德四年十二月晦,薨。明日,貞元元年元旦,海陵為兗輟朝,不受賀。宋、夏、高麗、回鶻賀正旦使,命有司受其貢獻。追進兗王爵。大定二十二年,追降特進。



 兗妻烏延氏,正隆六年坐與奴有姦,海陵殺之。其弟南京兵馬副都指揮使習泥烈私于族弟屋謀魯之妻,屋謀魯之奴謀欲執習泥烈,習泥烈乃殺其奴。海陵聞之,遂殺習泥烈。



 兗子阿合,大定中為符寶祗候,俄遷同知定
 武軍節度使。上曰:「汝歲秩未滿,朕念乃祖乃父為汝遷官,勿為不善,當盡心學之。」



 襄本名永慶,海陵母弟。為輔國上將軍。卒,天德二年,追封衛王,再贈司徒。大定二十二年,追降銀青光祿大夫。



 子和尚封應國公,賜名樂善。左宣徽使許霖之子知彰與和尚鬥爭,其母妃命家奴捽入凌辱之,使人曳霖至第毆詈之。明日,霖訴于朝。詔大興尹蕭玉、左丞良弼、權御史大夫張忠輔、左司員外郎王全雜治,妃杖一百,殺其家奴為首者,餘決杖有差。霖嘗跪於妃前,失大臣體,及所訴有妄,笞二十。



 大定間,家奴小僧月一妄言和尚
 熟寢之次有異徵,襄妃僧酷以為信然,召日者李端卜之。端云當為天子,司天張友直亦云當大貴。家奴李添壽上變。僧酷、和尚下吏驗問有狀,皆伏誅。上曰:「朕嘗痛海陵翦滅宗族。今和尚所為如此,欲貸其罪,則妖妄誤惑愚民者,便以為真,不可不滅。朕於此子,蓋不得已也。」傷閔者久之。



 袞本名蒲甲,亦作蒲家,桀驚強悍。海陵不喜其為人。初為輔國上將軍。天德初,加特進,封王,為吏部尚書,判大宗正事。坐語禁中起居狀,兵部侍郎蕭恭首問,護衛張九具言之。海陵親問。恭奪官解職,張九對不以實,特處死,
 袞與翰林學士承旨宗秀、護衛麻吉、小底王之章皆決杖有差。海陵自是愈忌之。未幾,授猛安。



 及遷中都,道中以蒲家為西京留守。西京兵馬完顏謨盧瓦與蒲家有舊,同在西京,遂相往來。蒲家嘗以玉帶遺之。蒲家稱謨盧瓦驍勇不減尉遲敬德。編修官圓福奴之妻與蒲家姻戚,圓福奴嘗戒蒲家曰:「大王名太彰著,宜少謙晦。」蒲家心知海陵忌之,嘗召日者問休咎。家奴喝里知海陵疑蒲家,乃上變告之,言與謨盧瓦等謀反,嘗召日者問天命。御史大夫高楨、刑部侍郎耶律慎須呂就西京鞫之,無狀。海陵怒,使使者往械蒲家等至中都,不復究
 問,斬之於市。謨盧瓦、圓福奴並日者皆凌遲處死。



 贊曰:金議禮制度,班爵祿,正刑法,治歷明時,行天子之事,成一代之典,杲、宗乾經始之功多矣。杲子宗義為海陵所殺,宗乾之後又不幸而有海陵,故其子孫之昌熾既鮮,而亦不免於僇辱焉。秦、漢而下,宗臣世家與國匹休者,何其少歟。君子於此,可以觀世變矣。



\end{pinyinscope}