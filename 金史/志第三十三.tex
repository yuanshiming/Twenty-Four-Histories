\article{志第三十三}

\begin{pinyinscope}

 選舉二



 ○文武選



 金制,文武選皆吏部統之。自從九品至從七品職事官,部擬。正七品以上,呈省以聽制授。凡進士則授文散官,謂之文資官。自餘皆武散官,謂之右職,又謂之右選。文資則進士為優,右職則軍功為優,皆循資,有升降定式而不可越。



 凡銓注,必取求仕官解由,撮所陳行績資歷之要為銓頭,以定其能否?其有犯公私罪贓污者,謂之
 犯選格,則雖遇恩而不得與。舊制,犯追一官以至追四官,皆解任周年,而復仕之。承安二年,定制,每追一官則殿一年,凡罷職會赦當敘者,及降殿當除者,皆具罪以聞,而後仕之。凡增課陞至六品者,任回復降。既廉升而再任覆察不同者,任回亦降。自進士、舉人、勞效、廕襲、恩例之外,入仕之途尚多,而所定之時不一。若牌印、護衛、令史之出職,則皇統時所定者也。檢法、知法、國史院書寫,則海陵庶人所置者也。若宗室將軍、宮中諸局承應人、宰相書表、太子護衛、妃護衛、王府祗候郎君、內侍、及宰相之子、並譯史、通事、省祗候郎君、親軍驍騎諸格,則
 定於世宗之時,及章宗所置之太常檢討、內侍寄祿官,皆仕進之門戶也。



 凡官資以三十月為考,職事官每任以三十月為滿,群牧使及管課官以三周歲為滿,防禦使以四十月、三品以上官則以五十月、轉運則以六十月為滿。司天、太醫、內侍官皆至四品止。凡外任循資官謂之常調,選為朝官謂之隨朝,隨朝則每考升職事一等,若以廉察而升者為廉升,授東北沿邊州郡而陞者為邊陞。凡院務監當差使則皆從九品。凡品官任都事、典事、主事、知事、及尚書省令史、覆實、架閣司管勾、直省直院局長副、檢法、知法、院務監當差使、及諸令史、譯
 史、掌書、書史、書吏、譯書、譯人、通事、并諸局分承應有出身者,皆為流外職。凡此之屬,或以尚書省差遣,或自本司判補,其出職或正班,雜班,則莫不有當歷之名職。既仕則必循升降之定式,雖或前後略有損益之殊,而定制則莫能渝焉。



 凡門廕之制,天眷中,一品至八品皆不限所廕之人。貞元二年,定廕敘法,一品至七品皆限以數,而削八品用廕之制。世宗大定四年五月,詔:「皇家袒免以上親,就蔭者依格引試,中選者勿令當儤使。」五年十月,制:「亡宋官當廕子孫者,並同亡遼官用蔭。」又曰:「教坊出身人,若任
 流內職者,與文武同用蔭。自餘有勤勞者,賞賜而已。昔正隆時常使教坊輩典城牧民,朕甚不取。」又更定冒廕及取蔭官罪賞格。七年五月,命司天臺官四品以上官改授文武資者,並聽如太醫例蔭。其制,凡正班,蔭亦正班;雜班,蔭雜班。明昌元年,以上封事者乞六品官添廕,吏部言:「天眷中,八品用廕,不限所廕之人。貞元中,七品用蔭,方限以數。當是時,文始於將仕,武始於進義,以上至七品儒林、忠顯,各七階,許廕一名。至六品承直、昭信,計九階,許廕二人。自大定十四年,文武官從下各增二階,其七品視舊為九階,亦蔭一人,至五品凡十七階,方
 廕二人,其五品至三品並無間越,唯六品不用蔭。乞依舊格,五品以上增蔭一名,六品廕子孫兄弟二人,七品仍舊為格。」時又以舊格雖有己子許廕兄弟姪,蓋所以崇孝悌也。而新格禁之,遂聽讓廕。舊制,司天、太醫、內侍、長行雖至四品。如非特恩換授文武官資者,不許用蔭,以本人見允承應,難使係班故也。泰和二年,定制,以年老六十以上退、與患疾及身故者,雖至止官,擬令係班,除存習本業者聽廕一名,止一子者則不須習即蔭。



 凡諸色出身文武官一品,蔭子孫至曾孫及兄弟姪孫六人,因門廕則五人。二品則子孫至曾孫及兄姪五人,
 因門廕則四人。三品子孫兄弟姪四人,因門廕則三人。四品、五品三人,因門廕則二人。六品二人,七品子孫兄弟一人,因門廕則六品、七品子孫兄弟一人。舊格,門蔭惟七品一人,餘皆加一人。明昌格,自五品而上皆增一人。凡進納官,舊格正班三品廕四人,雜班三人。正班武略子孫兄弟一人。雜班明威一人,懷遠以上二人,鎮國以上三人。司天、太醫遷至四品詔換文武官者,廕一人。



 凡進士所歷之階,及所循注之職。貞元元年,制南選,初除軍判、丞、簿從八品。次除防判、錄事正八品,三除下令從七品,。四中令、推官、節察判正七品,五六皆上令。從六品。北選,初軍
 判、簿、尉,二下令,三中令,四上令,已後並上令,通注節察判、推官。正隆元年格,上甲者初上簿軍判、丞、簿、尉,中甲者初中簿軍判、丞、簿、尉,下甲者初下簿軍判、丞、簿、尉。第二任皆中簿軍判、丞、簿、尉。三、四、五、六、七任皆縣令,回呈省。



 大定二年,詔文資官不得除縣尉。八定格,歷五年任令即呈省。十三年,制第二任權注下令。舊制,狀元授承德郎,以十四年官制,文武官皆從下添兩重,命狀元更授承務郎,次舊授儒林郎,更為承事郎。第二甲以下舊授從仕郎,更為將仕郎。十五年,敕狀元除應奉,兩考依例授六品。十八年,敕狀元行不顧名者與外除。十九年,命
 本貫察其行止美惡。二十一年,復命第三任注縣令。二十二年,敕進士授章服後,再試時務策一道,所謂策試者也。內才識可取者籍其名,歷任後察其政,若言行相副則升擢任使。是年九月,復詔令後及第人,策試中者初任即升之。二十三年格,進士,上甲,初錄事、防判,二下令,三中令。中甲,初中簿,二上簿,三下令。下甲,初下簿,二中簿,三下令。試中策者,上甲,初錄事、防判,二中令,三上令。中甲,初上簿,二下令,三中令。下甲,初中簿,二錄事、防判、三中令。又詔今後狀元授應奉,一年後所撰文字無過人者與外除。二十六年格,以相次合為令者減一資
 歷。二十六年格,三降兩降免一降,文資右職外官減最後,上令一任通五任回呈省,遂定格,上甲,初錄事、防判,二中令,三、四、五上令。中甲,初中簿,二下令,三中令,四、五上令,策試進士,初錄事、防判,二、三、四、五上令。其次,初上簿,二中令,三、四、五上令。又次,初中簿,二下令,三中令,四、五上令。下甲,初下簿,二下令,三中令,四,五上令。二十七年,制進士階至中大夫呈省。



 明昌二年,罷勘會狀元行止之制。七年格,縣令守闕各依舊格注授。泰和格,諸進士及第合授資任須歷遍乃呈省。雖未盡歷,官已至中大夫亦呈省。又諸詞賦、經義進士及第後,策試中選,合
 授資任歷遍呈省,仍每任升本等首銓選。貞祐三年,狀元授奉直大夫,上甲儒林郎,中甲以下授徵事郎。



 經義進士。皇統八年,就燕京擬注。六年,與詞賦第一人皆擬縣令,第二人當除察判,以無闕遂擬軍判。第二、第三甲隨各人住貫擬為軍判、丞、簿。舊制,《五經》及第未及十年與關內差使,已十年者與關外差使,四十年除下令。正隆三年,不授差使,至三十年則除縣令。大定二十八年始復設是科,每舉專主一經。



 女直進士。大定十三年,皆除教授。二十二年,上甲第二第三人初除上簿,中甲則除中簿,下甲則除下簿。大定二
 十五年,上甲甲首遷四重,餘各遷兩重。第二第三甲授隨路教授,三十月為一任,第二任注九品,第三、第四任注錄事、軍防判,第五任下令。尋復令第四任注縣令。二十六年,減一資歷注縣令。二十八年,添試論。後皆依漢人格。



 宏詞,上等遷兩官,次等遷一官,臨時取旨授之。恩榜,章宗大定二十九年,敕今後凡五次御簾進士,可一試而不黜落,止以文之高下定其次,謂之恩榜。女直人遷將仕,漢人登仕,初任教授,三十月任滿,依本格從九品注授。明昌元年,敕四舉終場,亦同五舉恩例,直赴御試。明
 昌五年,敕神童三次終場,同進士恩榜遷轉。兩次終場,全免差使。第六任與縣令,依本格遷官,如一次終場,初入仕則一除一差。其餘并依本門戶,仍使應二舉,然後入仕。每舉放四十人。凡恩例補蔭同進士者,謂大禮補致仕、遺表、陣亡等恩澤,補承襲錄用,並與國王并宗室女為婚者。正隆二年格,初下簿,二中簿,三上簿,四下令,五中令,六、七上令,回呈省。



 凡特賜同進士者,謂進粟、出使回、歿于王事之類,皆同雜班,補蔭亦以雜班。正隆元年格,初授下簿,二中簿,三縣丞、四軍判,五、六防判。七、八下令,九中令,十上令。尋復
 更初注下等軍判、丞、簿、尉,次注中等軍判、丞、簿、尉,第三注上等軍判、丞、簿、尉,四下令,五中令,六上令。



 律科、經童。正隆元年格,初授將仕郎,皆任司候,十年以上並一除一差,十年外則初任主簿,第二任司候,第三主簿,四主簿,五警判,六市丞,七諸縣丞,八次赤丞,九赤縣丞,十下縣令。十一中縣令,五任上縣令,呈省。三年制,律科及第及七年者與關內差使,七年外者與關外差。諸經及第人未十年者關內差,已十年關外差。律科四十年除下令。經童及第人視餘人復展十年,然後理算月日。大定十四年,以從下新增官階,遂定制,律科及第
 者授將仕佐郎。十六年特旨,以四十年除下令太遠,其以三十二年不犯贓罪者授下令。十七年,敕諸科人仕至下令者免差。二十年,省擬,無贓罪及廉察無惡者減作二十九年注下令,經童亦同此。二十六年,省擬,以相次當為縣令者減一資歷選注。敕命諸科人累任之餘月日至四十二月,準一除一差。又敕,舊格六任縣令呈省,遂減為五任。二十八年,減赤縣丞一任。明昌五年,制仕二十六年之上者,如該廉升則注縣令。六年,減諸縣丞、赤縣丞兩任後吏格,十年內擬注差使,十年外一除一差。若歷八任、或任至三十二年注下令,則免差須遍
 歷而後呈省。所歷之制,初、二下簿,三、四中簿,五、六、七上簿,犯選格者又歷上簿兩任,八、九則注下令,十中令,十一、十二上令。



 凡武舉,泰和三年格,上甲第一名遷忠勇校尉,第二、第三名遷忠翊校尉。中等遷修武校尉,收充親軍,不拘有無蔭,視舊格減一百月出職。下等遷敦武校尉,亦收充親軍,減五十月出職。承安元年格,第一名所歷之職,初都巡、副將,二下令,三中令,四、五上令。第二、第三名,初巡尉、部將,二上簿,三下令,四中令,五、六上令。餘人,初副巡、軍轄,二中簿,三下令,四中令,五、六上令。



 凡軍功有六:一曰川野見陣,最出當先,殺退敵軍。二曰攻打抗拒州縣山寨,奪得敵樓。三曰爭取船橋,越險先登。四曰遠探捕得喉舌。五曰險難之間,遠處報事情成功。六曰謀事得濟,越眾立功。皇統八年格,凡帶官一命昭信校尉正七品以上者,初除主簿及諸司副使正九品,二主簿及諸司使正八品,三下令從七品,四中令正七品,五上令,或通注鎮軍都指揮使正七品及正將。其官不至昭信及無官者,自初至三任通注丞、簿,四下令,五中令,六上令及知城寨從七品。章宗大定二十九年,遷至鎮國者取旨升除後。吏格之所定,女直人昭信校尉以上者,初下簿,二下
 令,三中令,四、五上令。女直一命遷至昭信校尉、餘人至昭信已上者,初下簿,二中簿,三下令,四中令,五、六上令。凡至宣武將軍以上者,初下令,二中令,二中令,三、四上令。



 凡勞效謂年老千戶、謀克也。大定五年,制河南、陜西統軍司,千戶十年以上擬從七品,三十年千戶、四十年以上之謀克從八品,二十年以上千戶、三十年以上謀克從九品,二十年以上謀克與正班、與差使,十年以上賞銀絹,皆以所歷千戶、謀克、蒲輦單月日通算。二十年,制以先曾充軍管押千戶、謀克、蒲輦二十年以上、六十五歲放罷者,視其強健者與差除、令係班,不則量加遷賞。後更定吏
 格,若一命遷宣武將軍以上,當授從七品職事者,初下令,二中令,三、四上令。官不至宣武,初授八品者授錄事,二赤劇丞,三下令,四中令,五、六上令。初授九品官者,初下簿,二中簿,三上簿,四下令,五中令,六、七上令。大定九年格,三虞候順德軍千戶四十年以上者與從八品,三十年千戶、四十年以上謀克從九品,二十年以上千戶、三十年以上謀克與正班,以下賞銀絹。大定十四年,定隨路軍官出職,以新制從下創添兩重,舊遷忠武校尉者今遷忠勇校尉。中都永固軍指揮使及隨路埽兵指揮使出職,舊遷敦武校尉者今遷進義校尉。武衛軍,大
 定十七年定制,其猛安曰都將,謀克曰中尉,蒲輦曰隊正。都將三十月遷一官,至昭信注九品職事。以隊正陞中尉。中尉升都將。



 省令史選之門有四:曰文資,曰女直進士,曰右職,曰宰執子,其出仕之制各異。



 文資者,舊惟聽左司官舉用,至熙宗皇統八年,省臣謂:「若止循舊例舉勾,久則善惡不分而多僥倖。」遂奏定制,自天眷二年及第榜次姓名,從上次第勾年至五十已上、官資自承直郎從六品至奉德大夫從五品,無公私過者,一闕勾二人試驗,可則收補,若皆可即籍名令還職待補。官至承直郎以上,一考得
 除正七品以上,從六品以下職事,兩考者除從六品以上、從五品以下。奉直大夫從六品以上,一考者除從六品以上。從五品以下,兩考者除從五品以上、正五品以下,節運同。



 正隆元年,罷是制,止於密院臺及六部吏人令史內選充。大定元年,世宗以胥吏既貪墨,委之外路幹事又不知大體,徒多擾動,至二年,罷吏人而復皇統選進士之制。承直郎以上者,一考正七品,除運判、節察判、軍刺同知。兩考者從六品,除京運判、總府判、防禦同知。奉直大夫已上,一考者從六品,除同前。兩考從五品,除節運副、京總管府留守司判官。七年,以散階官至五品
 亦勾充,不願者聽。十一年,以進士官至承直者眾,遂不論官資但以榜次勾補。二十七年,以外多闕官,論者以為資考所拘,難以升進,乃命不論官資,凡一考者與六品,次任降除正七品,第三任與六品,第四任升為從五品。兩考者與從五品,次任降除六品,第三、四任皆與從五品,五任升正五品。承安二年,以習學知除、刑房知案、及兵興時邊關令史,三十月除隨朝闕。泰和八年,以習學知除十五月以上,選充正知除。一考後理算資考。大安三年,以從榜次則各人所歷月日不齊,遂以吏部等差其所歷歲月多寡為次,收補知除,考滿則授隨朝職。



 貞祐五年,進士未歷任者,亦得充補,一考者除上縣令,再任上縣令升正七品,如已歷一任任丞簿者,舊制除六品,乃更為正七品,一任回降從七品,再任正七品升六品,如歷兩任丞簿者,一考舊除六品,乃更為正七品,一任回免降,復免正七品一任,即升六品。曾歷令一任者,依舊格六品,再任降除七品,還升從五品。興定二年,敕初任未滿及歷任者,考滿升等為從七品。初任未滿者爾兩任、未歷任者四任、回升正七品,兩任正七皆免回降。凡不依榜次勾取者同隨朝升除,俟榜次所及日聽再就補。興定五年,定進士令史與右職令史同格,考滿
 未應得從七者與正七品,回降從七一任。所勾諸府令史不及三考出職者除從七品,回降除八品。若一任應得從七品者除六品,回降正七品,若一任應得正七品者免降。



 女直進士令史,二十七年格,一考注正七品,兩考注正六品。二十八年,敕樞密院等處轉省者,並用進士。明昌元年,敕至三考者與漢人兩考者同除。明昌三年,罷契丹令史,其闕內增女直令史五人。五年,以與進士令史辛苦既同,資考難異,遂定與漢進士一考與從六品,兩考與從五品。



 宰執子弟省令史,大定十二年,制凡承廕
 者,呈省引見,除特恩任用外,並內奉班收,仍於國史院署書寫、太常署檢討、秘書監置校勘、尚書省準備差使,每三十月遷一重,百五十月出職。如承應一考以上,許試補省令譯史,則以百二十月出職,其已歷月日皆不紐折,如係終場舉人,即聽尚書省試補。十七年,定制,以三品職事官之子,試補樞密院令史。遂命吏部定制,宰執之子、并在省宗室郎君,如願就試令譯史,每年一就試,令譯史考試院試補外,緦麻袒免宗室郎君密院收補。大定二十八年,制以宗室第二從親并宰相之子,出職與六品外,宗室第三從親並執政之子,出職與正七
 品。其出職皆以百五十月,若見已轉省之餘人,則至兩考止與正七品。二十九年,四從親亦許試補。



\end{pinyinscope}