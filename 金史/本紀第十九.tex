\article{本紀第十九}

\begin{pinyinscope}

 世
 紀補



 景宣帝諱宗峻,本諱繩果,太祖第二子。母曰聖穆皇后唐括氏,太祖元妃。宗峻在諸子中最嫡。



 天輔五年,忽魯勃極烈杲都統諸軍取中京,帝別領合扎猛安,受金牌,既克中京,遂與杲俱襲遼主于鴛鴦濼。遼主走陰山,耿守忠救西京,帝與宗翰等擊走之。西京城南有浮圖,敵先據之,下射,士卒多傷。帝曰:「先取是,則西京可下。」既
 而攻浮圖,克下,遂下西京。太祖崩,帝與兄宗乾率宗室群臣立太宗。天會二年薨。



 熙宗即位,追上尊謚曰景宣皇帝,廟號徽宗。改葬興陵。海陵弒立,降熙宗為東昏王,降帝為豐王。世宗復熙宗廟謚,尊帝為景宣皇帝。子合剌、常勝、查剌。合剌是為熙宗。



 睿宗立德顯仁啟聖廣運文武簡肅皇帝諱宗堯,初諱宗輔,本諱訛里朵,大定上尊謚,追上今諱。魁偉尊嚴,人望而畏之。性寬恕,好施惠,尚誠實。太祖征伐四方,諸子皆總戎旅,帝常在帷幄。



 天輔六年,太祖親征,太宗居黃龍府,安福哥誘新降之民以叛,帝與烏古迺討平之。
 南路軍帥鶻實答以贓敗,帝往閱實之,咸稱平允。



 天會五年,宗望薨,帝為右副元帥,駐兵燕京。十一月,分遣諸將伐宋,帝發自河間,徇地淄、青。六年正月,宋馬括兵二十萬至樂安,帝率師擊破之。聞宋主在揚州,時東作方興,留大軍夾河屯田而還,軍山西。二月,移剌古破宋臺宗雋、宋忠軍五萬于大名,明日再破之,獲宗雋、忠而還。冀州人乘夜出兵襲照里營。照里擊敗之。宋主奉表請和,密書以誘契丹、漢人。詔伐宋。帝發自河北,降滑州,取開德府,攻大名府,克之,河北平。



 初,伐宋,河北、河東諸將議不決,或欲先定河北、或欲先平陜西,太宗兩用其策。而宗翰來
 會于濮,既平河北,遂取東平及徐州,盡得宋人江淮運致金幣在徐州官庫者,分給諸軍,而劉豫遂以濟南降。使拔高速等襲宋主于揚州,而宋主聞之,比拔高速至揚州,前夕已渡江矣。宋主乃貶去帝號,再以書來請存社稷,語在《宗翰傳》中,既而宗弼追宋主,宋渡江,入于杭州,復遁入海,宗弼乃還。



 於是,婁室所下陜西城邑輒叛,宗翰等曰:「前討宋,故分西師合于東軍,而陜西五路兵力雄勁,當併力攻取。今撻懶撫定江北,宗弼以精兵二萬先往洛陽。以八月往陜西,或使宗弼遂將以行,或宗輔、宗乾、希尹中以一人往。」上曰:「婁室往者所向輒辦,
 今專征陜西,豈倦于兵而自愛邪?卿等其戮力焉!」由是詔帝往。



 是時,宋張浚兵取陜西,帝至洛水治兵,張浚騎兵六萬,步卒十二萬壁富平。帝至富平,婁室為左翼,宗弼為右翼,兩軍並進,自日中至于昏暮,凡六合戰,破之。耀州、鳳翔府皆來降。遂下涇、渭二州。敗宋經略使劉倪軍于瓦亭,原州降。撒離喝破德順軍靜邊寨,宋涇原路統制張中孚、知鎮戎軍李彥琦以城降。宋秦鳳路都統制吳玠軍于隴州境上,招討都監馬五擊走之,降一縣而還。帝進兵降甘泉等三堡,取保川城,破宋熙河路副總管軍三萬,獲馬千餘,拔安西等二寨,熙州降。分遣左
 翼都統阿盧補、右翼都統守弼招撫城邑之未下者,遂得鞏、洮、河、樂、西寧、蘭、廓、積石等州,定遠、和政、甘峪、寧洮、安隴等城寨及鎮堡蕃漢營部四十餘,於是涇原、熙河兩路皆平。撒離喝降慶陽府,慕洧以環州降。既定陜西五路,乃選騎兵六千,使撒離喝列屯衝要。於是班師,與宗翰俱朝京師,立熙宗為諳版勃極烈,帝為左副元帥。



 十三年,行次媯州薨,年四十,陪葬睿陵,追封潞王,謚襄穆。皇統六年,進冀國王。正隆二年,追贈太師、上柱國,改封許王。世宗即位,追上尊謚立德顯仁啟聖廣運文武簡肅皇帝,廟號睿宗。二年,改葬于大房山,號景陵。



 顯宗體道弘仁英文睿德光孝皇帝,諱允恭,本諱胡土瓦,世宗第二子,母曰明德皇后烏林答氏。皇統六年丙寅歲生。體貌雄偉,孝友謹厚。



 大定元年十一月,世宗即位于東京。乙酉,封楚王,置官屬。十二月,從至中都。



 二年四月己卯,賜名允迪。五月壬寅,立為皇太子,世守謂之曰:「在禮貴嫡,所以立卿。卿友于兄弟,接百官以禮,勿以儲位生驕慢。日勉學問,非有召命,不須侍食。」帝上表謝。專心學問,與諸儒臣講議於承華殿。燕閑觀書,乙夜忘倦,翼日輒以疑字付儒臣校證。九月庚子,詔東宮三師對皇太子稱名,少師以降稱臣。十一月庚子,生辰,百官
 賀于承華殿,世宗賜以襲衣良馬,賜宴于仁政殿,皇族百官皆與。自後生辰,世宗或幸東宮,或宴內殿,歲以為常。十二月辛卯,奏曰:「東宮賀禮,親王及一品皇族皆北面拜伏,臣但答揖。伏望天慈聽臣答拜,庶惇親親友愛之道。」世宗從之,以為定制。



 世宗聞儒者鄭松賢,松先為同知博州防禦事致仕,起為左諭德,詔免朝參,令輔太子讀書。松以友諭自處,帝嘗顧松使取服帶,松對曰:「臣忝諭德,不敢奉命。」帝改容稱善,自是益加禮遇。每出獵獲鹿,輒分賜之。



 四年九月,納妃徒單氏,行親迎禮。故事,大賀鹵簿天子乘玉路,皇太子鹵簿乘金路。六年,世宗
 行自西京還都,禮官不知皇太子自有鹵簿金路,乃請太子就乘大駕綴路,行在天子之前。上疑其非禮,詳閱舊典,禮官始覺其誤。於是禮部郎中李邦直、員外郎李山削一階,太常少卿武之才、太常丞張子羽、博士張矩削兩階。



 頃之,禮官議受冊謁謝太廟,服常朝服,乘馬,世宗曰:「此與外宮禮上後謁諸神廟無異,海陵一時率意行之,何足為法?大冊與三歲祫享當用古禮為是。孔子曰:『禮與其奢也寧儉。』不當輕易如此。」又曰:「右丞蘇保衡雖漢人不通經史,參政石琚通經史而不言,前日禮官既已削奪,猶不懼邪?其具前代典禮以聞,朕將擇而處
 之。」久之,將授太子冊寶,儀注備儀仗告太廟。上曰:「朕受尊號謁謝,乃用故宋真宗故事,常朝服乘馬。皇太子乃用備禮,前後不稱,甚無謂也。」謂右丞相良弼、左丞守道曰:「此卿等不用心所致。」良弼等謝曰:「臣愚慮不及此。」上復曰:「此文臣因循故也。」是年十月甲申,祫享于太廟,行亞獻禮。



 七年,帝有疾,詔左丞守道侍湯藥,徙居瓊林苑臨芳殿調治。



 八年正月甲戌,改賜名允恭。庚辰,受皇太子冊寶,帝上表謝。



 九年五月,世宗命避暑于草濼,隋王惟功從行,其應從行者皆給道路費。帝奏曰:「遠去闕廷,獨就涼地,非臣子所安,願罷行。」世宗曰:「汝體羸弱,山後
 高涼,故命汝往。」丁丑,百官奉辭于都城之北,再拜,帝答拜。是月,百官承詔具箋問起居。



 六月,百官問起居如前。八月乙酉,至自草濼,百官迎謁于都城之北,如送儀。丙戌,入見,世宗曰:「吾兒相別經夏,極甚思憶也。」九月,詔皇太子供膳勿月支,歲給五千萬。



 十年八月,帝在承華殿經筵,太子太保壽王爽啟曰:「殿下頗未熟本朝語,何不屏去左右漢官,皆用女直人。」帝曰:「諭德、贊善及侍從官,曷敢輒去?」爽乃揖而退。帝曰:「宮官四員謂之諭德、贊善,義可見矣,而反欲去之,無學故也。」有使者自山東還,帝問民間何所苦,使者曰:「錢難最苦。官庫錢滿有露積者,
 而民間無錢,以此苦之。」帝曰:「貯之空室,雖多奚為。」謂戶部尚書張仲愈曰:「天子富藏天下,何必獨在府庫也。」因奏曰:「錢在府庫,何異銅礦在野。乞流轉,使公私俱利。」世宗嘉納,詔有司議行之。



 十年一月丁亥,有事於圓丘,帝行亞獻禮。



 十二年五月,世宗聞德州防禦使胡剌謀叛,因曰:「朕於親親之道未嘗不篤,而輒敢如此。」帝徐奏曰:「叔胡剌性荒縱,耽娛樂,而無子嗣,忽如此狂謀,望更閱實之。」十月己未,祫享于太廟,帝攝行祀事。



 十三年十月,承詔與趙王惟中、曹王惟功獵于保州、定州。十一月甲午,還京師。



 十四年四月乙亥,世宗御垂拱殿,帝及諸王侍
 側。世宗論及兄弟妻子之際,世宗曰:「婦言是聽而兄弟相違,甚哉。」帝對曰:「《思齊》之詩曰:『刑于寡妻,至于兄弟,以御于家邦。』臣等愚昧,願相勵而修之。」因引《棠棣》華萼相承,脊令急難之義,為文見意,以誡兄弟焉。



 十五年,世宗詔五品職事官謝見皇太子。



 十七年五月甲辰,侍宴于常武殿,典食令涅合進粥,帝將食,有蜘蛛在粥碗中,涅合恐懼失措,帝從容曰:「蜘蛛吐絲乘空,忽墮此中爾,豈汝罪哉。」十月己卯,祫享于太廟,攝行祀事。



 十九年四月戊申,有事于太廟,攝行祀事。丁巳,詹事烏林答愿入謝,帝命取襆頭腰帶,官屬請曰:「此見宰相師傅之禮也。」帝
 曰:「愿事陛下久,以此加敬耳。」皆曰:「非臣等所及。」十一月,改葬明德皇后於坤厚陵,帝徒行挽靈車。遇大風雪,左右進雨具,帝卻之。比至頓所,衣盡霑濕,觀者無不下淚。海陵雖貶黜為庶人,宗乾尚稱明肅皇帝,議者以為未盡,帝具表奏論。世宗嘉納之。於是宗乾削去帝號,降封遼王。



 二十四年,世宗將幸上京,詔帝守國,作「守國之寶」以授之。其遣使、祭享、五品以上官及事利害重者遣使馳奏,六品以下官、其餘常事,並聽裁決。每三日一次於集賢殿受尚書省啟事。京朝官遇朔望具朝服問侯。車駕在路,每二十日一遣使問起居。已達上京,每三十日
 一問起居。世宗曰:「今巡幸或能留一二年,以汝守國。譬之農家種田,商人營利,但能不墜父業。即為克家子也。」帝對曰:「臣在東宮二十餘年,過失甚多,陛下以明德皇后之故未嘗見責。臣誠愚昧,不克負荷,乞備扈從。」世宗曰:「凡人養子,皆望投老得力。朕留太尉、左右丞、參政輔汝,彼皆國家舊人,可與商議。且政事無難,但用心公正,無納讒邪,一月之後,政事自熟。」帝流涕堅辭,左右為之感動。三月,世宗如上京,帝守國留中都。初,帝在東宮,或攜中侍步于芳苑。中侍出入禁中,未嘗限阻。此輩見帝守國,各為得意,帝知之,謂諸中侍曰:「我向在東宮,不親
 國政,日與汝輩語話。今即守國,汝等有召命然後得入。」五月,世宗至上京,賜敕書曰:「朕以前月八日到遼陽,此月二日達上京,翌日祀慶元廟。省方觀民,古之制也。汝守國任重,夏暑方熾,益當自愛,無貽朕憂。」帝謂徒單克寧曰:「車駕巡幸,以國事見屬。刑名最重,人之死生繫焉。凡有可議,當盡至公。比主上還都,勿有廢事。」自是,凡啟稟刑名,帝自披閱,召都事委曲折正,移晷忘倦,或賜之食。近侍報瑤池位蓮開,當設宴。帝曰:「聖上東巡,命我守國,何敢宴遊廢事?採致數花足矣。」七月,遣子金源郡王麻達葛奉表問起居,請世宗還都。十一月壬寅,帝冬獵。
 辛亥,還都。



 二十五年正月乙酉朔,免群臣賀禮。帝自守國,深懷謙抑,宮臣不庭拜,啟事時不侍立,免朔望禮。京朝朔日當具公服問候,並停免。至是,群臣當賀,亦不肯受。甲寅,帝如春水。二月庚申,還都。丁卯,遣子金源郡王麻達葛奉表賀萬春節。四月,久不雨,帝親禱,即日霑足。



 六月甲寅,帝不豫。庚申,崩于承華殿。世宗自上京還,次天平山好水川,訃聞,為位臨奠于行宮之南,大慟者久之。親王、百官、皇族、命婦及侍衛皆會哭,世宗號泣還宮。比至中都,為位奠哭者凡七焉。世宗以豳王永成為中都留守,來護喪,遣滕王府長史再興、御院通進阿里剌
 來保護金源郡王,遣左宣徽使唐括鼎來致祭,詔妃徒單氏及諸皇孫喪服並如漢制。帝王儲位久,恩德在人者深,每日三時哭臨,侍衛軍士皆爭入臨,伏哭于承華殿下,聲殷如雷。中都百姓市門巷端為位慟哭。七月壬午朔,賜謚宣孝太子。九月庚寅,殯於南園熙春殿。己酉,世宗至自上京,未入國門,先至熙春殿致奠,慟哭久之。比葬,親臨者六。帝事世宗,凡巡幸西京、涼陘,及上陵、祭廟、謁衍慶宮,田獵觀稼,拜天射柳,未嘗去左右。上有事于圓丘,及親享于太廟,則行亞獻禮,不親祀則攝行祀事。國有大慶則率百官上表賀。正旦、萬春節則總班上
 壽。冬十月庚戌朔,宰相以下朝見于慶和殿,太尉完顏守道上壽,世宗追悼悽愴者久之。十一月甲申,靈駕發引,世宗路祭于都城之西。庚寅,葬于大房山。世宗欲加帝號,以問群臣,翰林修撰趙可對曰:「唐高宗追謚太子弘為孝敬皇帝。」左丞張汝弼曰:「此蓋出于武后。」遂止,乃建廟于衍慶宮後,祭和三獻,樂用登歌。



 二十六年,立子璟為皇太孫。二十九年,世宗崩。太孫即位,是為章宗。



 五月甲午,追謚體弘仁英文睿德光孝皇帝,廟號顯宗。丁酉,祔于太廟,陵曰裕陵。



 帝天性仁厚,不忍刑殺。梁檀兒盜金銀葉,憐其母老,李福興盜段匹,值坤厚陵禮成,
 家令本把盜銀器,值萬春節,皆委曲全活之。亡失物者,責其償而不加罪。聞四方饑饉,輒先奏,加賑贍。因田獵出巡,所過問民間疾苦。敬禮大臣,友愛兄弟。葬明德皇后於坤厚陵,諸妃皆祔,自磐寧宮發引,趙王惟中以其母挽車先發,令張黃蓋者前行,帝呼執蓋者不應,少府監張僅言欲奏其事,帝止之。嘗作《重光座銘》,及刻座右銘於小玉碑,并刻其碑陰,皆深有理致。最善射而不殫物,嘗奉詔拜陵,先獵,射一鹿獲之,即命罷獵,曰:「足奉祀事,焉用多殺?」好生蓋其天性云。



 贊曰:遼王杲取中京,宗翰、宗望皆從,景宣別領合扎猛
 安。合扎猛安者,太祖之猛安也。宗翰請立熙宗,宗乾不敢違,太宗不能拒,其義正,其理直矣。舊史稱睿宗寬恕好施惠,熙宗不終,海陵隕斃,自時厥後,得大位者皆其子孫,有以夫。顯宗孝友惇睦,在東宮二十五年,不聞有過。承意開導,四方陰受其賜。天下不假之年,惜哉!



\end{pinyinscope}