\article{列傳第六十一}

\begin{pinyinscope}

 忠義三



 ○徒單航完顏陳和尚楊沃衍烏古論黑漢陀滿胡土門姬汝作愛申馬肩龍附禹顯



 徒單航,一名張僧,駙馬樞密使某之子也。父號九駙馬,衛王有事北邊,改授都元帥,仍權平章,殊不允人望。張僧時為吏部侍郎,力勸其父請辭帥職,遂拜平章。至寧
 元年,胡沙虎弒逆,降航為安州刺史。會北兵大至城下,聲言「都城已失守,汝可速降。」航謂其民曰:「城守雖嚴,萬一攻破,汝輩無孑遺矣。我家兩世駙馬,受國厚恩,決不可降。汝輩計將安出?」其民曰:「太守不屈,我輩亦何忍降,願以死守。」航乃盡出家財以犒軍民,軍民皆盡力備禦。又五日,城危,航度不可支,謂其妻孥曰:「今事急矣,惟有死爾。」乃先縊其妻拏,謂其家人曰:「我死即撤屋焚之。」遂自縊死。城破,人猶力戰,曰:「太守既死,我輩不可獨降。」死者甚眾。



 完顏陳和尚,名彞,字良佐,亦以小字行,豐州人。系出蕭
 王諸孫。父乞哥,泰和南征,以功授同知階州軍事,及宋復階州,乞哥戰歿於嘉陵江。貞祐中,陳和尚年二十餘,為北兵所掠,大帥甚愛之,置帳下。時陳和尚母留豐州,從兄安平都尉斜烈事之甚謹。陳和尚在北歲餘,託以省母,乞還。大帥以卒監之至豐,乃與斜烈劫殺監卒。奪馬奉其母南奔,大兵覺,合騎追之,由他路得免。既而失馬,母老不能行,載以鹿角車,兄弟共挽,南渡河。宣宗奇之。



 斜烈以世官授都統,陳和尚試補護衛,未幾轉奉御。及斜烈行壽、泗元帥府事,奏陳和尚自隨,詔以充宣差提控,佩金符。斜烈辟太原王渥為經歷。渥字仲澤,文章
 論議與雷淵、李獻能相上下,故得師友之。陳和尚天資高明,雅好文史,自居禁衛日,人以秀才目之。至是,渥授以《孝經》、《小學》、《論語》、《春秋左氏傳》,略通其義。軍中無事,則窗下作牛毛細字,如寒苦之士,其視世味漠然。



 正大二年,斜烈落帥職,例為總領,屯方城。陳和尚隨以往,凡兄軍中事皆預知之。斜烈時在病,軍中李太和者,與方城鎮防軍葛宜翁相毆,訴於陳和尚,宜翁事不直,即量笞之。宜翁素兇悍,恥以理屈受杖,竟鬱鬱以死,留語其妻,必報陳和尚。妻訟陳和尚以私忿侵官,故殺其夫,訴於臺省,於近侍,積薪龍津橋南,約不得報,則自焚以謝其
 夫。以故陳和尚繫獄。議者疑陳和尚,狃於禁近,倚兵閫之重,必橫恣違法,當以大辟。奏上,久不能決。陳和尚聚書獄中讀之,凡十有八月。明年,斜烈病愈,詔提兵而西,入朝,哀宗怪其瘦甚,問:「卿寧以方城獄未決故耶?卿但行,吾今赦之矣。」以臺諫復有言,不敢赦。未幾,斜烈卒。上聞,始馳赦陳和尚,曰:「有司奏汝以私忿殺人。汝兄死,失吾一名將。今以汝兄故,曲法赦汝,天下必有議我者。他日,汝奮發立功名,國家得汝力,始以我為不妄赦矣。」陳和尚且泣且拜,悲動左右,不能出一言為謝。乃以白衣領紫微軍都統,踰年轉忠孝軍提控。



 五年,北兵入大昌
 原,平章合達問誰可為前鋒者,陳和尚出應命。先已沐浴易衣,若將就木然者,擐甲上馬不反顧。是日,以四百騎破八千眾,三軍之士踴躍思戰,蓋自軍興二十年始有此捷。奏功第一,手詔褒諭,授定遠大將軍、平涼府判官,世襲謀克。一日名動天下。



 忠孝一軍,皆回紇、乃滿、羌、渾及中原被俘避罪來歸者,鷙狠凌突,號難制。陳和尚御之有方,坐作進退皆中程式,所過州邑常料所給外秋毫無犯,街曲間不復喧雜,每戰則先登陷陣,疾若風雨,諸軍倚以為重。六年,有衛州之勝。八年,有倒回谷之勝。自刑徒不四五遷為禦侮中郎將。



 副樞移剌蒲阿無
 持重之略,嘗一日夜馳二百里趨小利,軍中莫敢諫止。陳和尚私謂同列曰:「副樞以大將軍為剽略之事,今日得生口三百,明日得牛羊一二千,士卒喘死者則不復計。國家數年所積,一旦必為是人破除盡矣。」或以告蒲阿,一日,置酒會諸將飲,酒行至陳和尚,蒲阿曰:「汝曾短長我,又謂國家兵力當由我盡壞,誠有否?」陳和尚飲畢,徐曰「有。」蒲阿見其無懼容,漫為好語云:「有過當面論,無後言也。」



 九年正月,三峰山之敗,走鈞州。城破,大兵入,即縱軍巷戰。陳和尚趨避隱處,殺掠稍定乃出,自言曰:「我金國大將,欲見白事。」兵士以數騎夾之,詣行帳前。問其
 姓名,曰:「我忠孝軍總領陳和尚也。大昌原之勝者我也,衛州之勝亦我也,倒回谷之勝亦我也。我死亂軍中,人將謂我負國家,今日明白死,天下必有知我者。」時欲其降,斫足脛折不為屈,豁口吻至耳,噀血而呼,至死不絕。大將義之,酹以馬湩,祝曰:「好男子,他日再生,當令我得之。」時年四十一。是年六月,詔贈鎮南軍節度使,塑像褒忠廟,勒石紀其忠烈。



 斜烈名鼎,字國器,畢里海世襲猛安。年二十,以善戰知名。自壽、泗元帥轉安平都尉,鎮商州,威望甚重,敬賢下士,有古賢將之風。初至商州,一日搜伏,於大竹林中得
 歐陽脩子孫,問而知之,併其族屬鄉里三千餘人皆縱遣之。



 楊沃衍,一名斡烈,賜姓兀林答,朔州靜邊官莊人,本屬唐括迪剌部族。少嘗為北邊屯田小吏,會大元兵入境,朝命徙唐括族內地,沃衍留不徙,率本部族願從者入保朔州南山茶杞溝,有眾數千,推沃衍為招撫使,號其溝曰府,故殘破鎮縣徒黨日集,官軍不能制。又與大兵戰,連獲小捷,及乏食,遂行剽劫。官軍捕之,拒戰不下,轉走寧、隩、武、朔、寧邊諸州,民以為病。朝廷遣人招之,沃衍即以眾來歸。時宣宗適南遷,次淇門,聞之甚喜,遂以為
 武州刺史。



 武州屢經殘毀,沃衍入州未幾,而大兵來攻,死戰二十七晝夜不能拔,乃退,時貞祐二年二月也。既而朝廷以武州終不可守,令沃衍遷其軍民駐岢嵐州,以武州功擢為本州防御使。俄升岢嵐為節鎮,以沃衍為節度使,仍詔諭曰:「卿於國盡忠,累有勞績。今特升三品,恩亦厚矣,其益勵忠勤,與宣撫司輯睦以安軍民。」沃衍自奉詔即以身許國,曰:「為人不死王事而死於家,非大丈夫也。」



 三年,奉旨屯涇、邠、隴三州,沃衍分其軍九千人為十翼五都統,親統者十之四。是冬,西夏四萬餘騎圍定西州,元帥右都監完顏賽不以沃衍提控軍事,率
 兵與夏人戰,斬首幾二千,生擒數十人,獲馬八百餘匹,器械稱是,餘悉遁去。詔陜西行省視功官賞之。



 興定元年春,上以沃衍累有戰功,賜今姓。未幾,遙授通遠軍節度使、兼鞏州管內觀察使。是冬,詔陜西行省伐宋,沃衍與元帥左都監內族白撒、通遠軍節度使溫迪罕婁室、同知通遠軍節度使事烏古論長壽、平西軍節度副使和速嘉兀迪將兵五千出鞏州鹽川,至故城逢夏兵三百,擊走之。又入西和州至岐山堡,遇兵六千凡三隊,遣軍分擊,逐北三十餘里,斬首四百級,生獲十人、馬二百匹、甲仗不勝計。尋復得散關。二年正月,捷報至,上大喜,
 詔遷沃衍官一階,遙授知臨洮府事。三年,武休關之捷,沃衍功居多,詔特遷一官。



 元光元年正月,遙授中京留守。六月,進拜元帥右監軍,仍世襲納古胡里愛必剌謀克。二年春,北兵游騎數百掠延安而南,沃衍率兵追之,戰於野豬嶺,獲四人而還。俄而,兵大至,駐德安寨,復擊走之。未幾,大兵攻鳳翔還,道出保安,沃衍遣提控完顏查剌破於石樓臺,前後獲馬二百、符印數十。詔有司論賞。初,聞野豬嶺有兵,沃衍約陀滿胡土門以步軍會戰。胡土門宿將,常輕沃衍,至是失期。沃衍戰還,會諸將欲斬胡土門,諸將哀請乃釋之。時大兵聲勢益振,陜西行
 省檄沃衍清野,不從,曰;「我若清野,明年民何所得食?」遂隔大澗持勢使民畢麥事。正大二年,進拜元帥左監軍,遙領中京留守。



 八年冬,平章合達、參政蒲阿由鄧州而西,沃衍自豐陽川遇於五朵山下,問禹山之戰如何,合達曰:「我軍雖勝,而大兵已散漫趨京師矣。」沃衍憤云:「平章、參政蒙國厚恩,握兵柄,失事機,不能戰禦,乃縱兵深入,尚何言耶!」



 三峰山之敗,沃衍走鈞州。其部曲白留奴、呆劉勝既降,請於大帥,願入鈞招沃衍。大帥質留奴,令勝入鈞見沃衍,道大帥意,降則當授大官。沃衍善言慰撫之,使前,拔劍斫之,曰:「我起身細微,蒙國大恩,汝欲以
 此污我耶!」遂遺語部曲後事,望汴京拜且哭曰:「無面目見朝廷,惟有一死耳。」即自縊。部曲舉火并所寓屋焚之,從死者十餘人。沃衍死時年五十二。



 初,大兵破西夏,長驅而至,關輔千里皆洶洶不安,雖智者亦無如之何。沃衍與其部將劉興哥者率兵往來邠、隴間,屢戰屢勝,故大軍猝不能東下。



 興哥,鳳翔虢縣人,起於群盜,人呼曰「熱劉」。後於清化戰死,大兵至酹酒以弔,西州耆老語之。至為泣下。



 烏古論黑漢,初以親軍入仕,嘗為唐、鄧元帥府把軍官。天興二年,唐州刺史內族斜魯病卒,鄧州總帥府以
 蒲察都尉權唐州事。宋軍兩來圍唐,又唐之糧多為鄧州所取,以故乏食。六月,遣萬戶夾谷定住入歸德,奏請軍糧,不報。七月,鎮防軍馮總領、甄改住為變,殺蒲察都尉。時朝廷道梗,帥府承制以黑漢權刺史行帥府事。



 既而鎮防軍有歸宋之謀,時裕州大成山聶都統一軍五百人在州,獨不欲歸宋,與鎮防軍為敵,鎮防不能勝,棄老幼奔棗陽,宋人以故知唐之虛實。會鄧帥移剌瑗以城叛歸于宋,遺書招黑漢,黑漢殺其使者不報。宋王安撫率兵攻唐,鄂司王太尉繼至,攻益急。黑漢聞哀宗遷蔡,遣人求救,上命權參政兀林答胡土將兵以往。宋人設伏,
 縱其半入城,邀擊之,胡土大敗,僅存三十騎以還。



 城中糧盡,人相食,黑漢殺其愛妾啖士,士爭殺其妻子。官屢聚議欲降,黑漢與聶都統執議益堅,馮總領乃私出城與王安撫會飲,約明日宋軍入城。馮歸,宋軍不得入,聶都統請馮議事,即坐中斬之。及其黨皆死。總領趙醜兒者初與馮同謀,內不自安,開西門納宋軍。黑漢率大成山軍巷戰,自辰至午,宋軍大敗而出,殺傷無數。宋人城下大呼趙醜兒,約併力殺大成山軍。大成軍敗,宋人獲黑漢,脅使降,黑漢不屈,為所殺。其得脫走者十餘人,總領移剌望軍、女奚烈軍、醜兒走蔡州,皆得遷賞,後俱死
 於甲午之難。



 陀滿胡土門,字子秀,策論進士也。累官翰林待制。貞祐二年,遷知中山府。三年,改知臨洮府、兼本路兵馬都總管。叛賊蘭州程陳僧等誘夏人入寇,圍臨洮凡半月,城中兵數千而粟且不支,眾皆危之。胡土門日為開諭逆順禍福,皆自奮。因捕其黨欲為內應者二十人,斬之,擲首城外。賊四面來攻,乃夜出襲賊壘,夏兵大亂,金軍乘之,遂大捷,夏人遁去。



 四年,知河中府事,權河東南路宣撫副使。十月,進元帥右監軍、兼前職。興定二年,為絳陽軍節度使、兼絳州管內觀察使。十月,遷元帥左監軍、行
 元帥府事、兼知晉安府、河東南路兵馬都總管。於是,修城池,繕甲兵,積芻糧,以備戰守。民不悅,行省胥鼎聞之,遺以書曰:「元帥始鎮河中,惠愛在民,移IM晉安,遠近忻仰。去歲兵入,平陽不守,河東保完者惟絳而已。蓋公坐籌制勝,威德素著,故不動聲氣以至無虞也。邇來傳聞,治政太剛,科徵太重,鼎切憂之。古人有言,御下不寬則人多懼禍,用人有疑則士不盡心。況大兵在邇,鄰境已虛,小人易動,誠不可不慮也。願公以謙虛待下,忠孝結人,明賞罰,平賦稅,上以分聖主宵旰之憂,下以為河東長城之託。」胡土門得書,懼民不從且或生變,乃上言:「臣
 本瑣材,猥膺重寄,方將治隍碑、積芻糧為捍禦之計,而小民難與慮始,以臣政令頗急,皆有怨言,遂貽行省之憂。自聞訓諭,措身無所,內自悛悔,外加寬撫,庶幾少慰眾心。而近以朝命分軍過河,則又喧言帥臣不益兵保守,而反助河南,將棄我也。人心如此,恐一旦遂生他變。向者李革在平陽,人不安之,而革隱忍不言,以至于敗。臣實拙繆,無以服人,敢以鼎書上聞,惟朝廷圖之。」朝廷以鼎言,遣吏部尚書守顏閭山代之。或曰,胡土門欲以計去晉安,乃大興役,恣為殺戮,務失民心,故鼎言及之。未幾,晉安失守,死者幾百萬人,遂失河東。



 三年八月,改
 太常卿、權簽樞密院事、知歸德府事。元光二年二月,坐上書不實,削一官。正大三年七月,復為臨洮府總管。四年五月,城破被執,誘之降不應,使之跪不從,以刀亂斫其膝脛,終不為屈,遂殺之。五年,詔贈中京留守,立像褒忠廟,錄用其子孫。其妻烏古論氏亦死節,有傳。



 姬汝作,字欽之,汝陽人,全州節度副使端脩之姪孫也。父懋,以蔭試部掾,轉尚書省令史。汝作讀書知義理,性豪宕,不拘細行,平日以才量稱。正大末,避兵崧山,保鄉鄰數百家,眾以長事之。後徙居交牙山砦,會近侍局使烏古論四和撫諭西山,以便宜授汝作北山招撫使,佩
 銀符,遂遷入汝州。



 初,汝州殘破之後,天興元年正月,同知宣徽院事張楷授防禦使,自汴率襄、郟縣土兵百餘人入青陽垛。時呼延實者領青陽砦事。實趙城人,本楊沃衍部曲,以戰功至寶昌軍節度使,閑居汝之西山。楷自揣不能服眾,乃以州事託實,尋往鄧州從恒山公武仙。後大元兵至,城破,殺數千人,乃許降,以張宣差者管州事。三月,鈞州潰軍柳千戶者入州,張逃去,柳遂據之。未幾,城復破。及汝作至,北兵雖去,但空城爾。汝作招集散亡,復立市井,北兵屢招之不從,數戰互有勝負。已而北兵復來攻,汝作親督士卒,以死拒之。兵退,間道納奏,
 哀宗宣諭:「此州無險固可恃,汝乃能為國用命,今授以同知汝州防禦使,便宜從事。」



 是時,此州南通鄧州,西接洛陽,東則汴京,使傳所出,供億三面,傳通音耗。然呼延實在青陽為總帥,忌汝作城守之功,不能相下,州事動為所制。實欲遷州入山,謂他日必為大兵所破。汝作以為「倉中糧尚多,四面潰軍日至,此輩經百死,激之皆可用,朝廷倚我守此州,總帥乃欲棄之,何心哉。」讒間既行,有相圖之隙,詳議官楊鵬釋之曰:「外難未解而顧私忿。」語甚諄切。實乃還山,鵬因勸汝作納奏,乞死守此州,以堅軍民之心。其冬,戰于襄、郟,得馬百餘,士氣頗振,遂以
 汝作為總帥,不復與實相關矣。



 天興二年六月,哀宗在蔡州,遣使徵兵入援。州人為邏騎所擾,農事盡廢,城中糧亦垂盡。是月,中京破,部曲私議有脣亡之懼,計以城降,懼汝作,不敢言,乃以遷州入山白之。汝作怒曰:「吾家父祖食祿百年,今朝延又以州事帥職委我,吾生為金民,死為金鬼。汝輩欲避於山,非欲降乎?有再言遷者吾必斬之。」



 八月,塔察將大兵攻蔡,經汝州。州人梁皋作亂,與故吏溫澤、王和七八人徑入州廨,汝作不為備,遂為所殺。時宣使石珪體究洛陽所以破及強伸死節事,以路阻,留汝州驛。梁皋既殺汝作,走告珪曰:「汝作私積糧
 斛,不恤軍民,眾怒殺之矣。皋不圖汝作官職,惟宣使裁之。」珪懼,乃以皋權汝州防御使、行帥府事。脫走入蔡,以皋殺汝作事聞。哀宗甚嗟惜之,遣近侍張天錫贈汝作昌武軍節度使,子孫世襲謀克,仍詔峴山帥呼延實、登封帥范真併力討皋。天錫避峴山遠,先約范真,真以麾下李某者往,以撫諭軍民為名。皋率軍士迎於東門,知朝廷圖己,陰為之備,李猶豫不敢發。皋館天錫于望崧樓,隱毒於食,天錫遂中毒而死。皋後為大元兵所殺。



 楊鵬字飛卿,能詩。



 愛申,逸其族與名,或曰一名忙哥。本虢縣鎮防軍,累功
 遷軍中總領。李文秀據秦州,宣宗詔鳳翔軍討之,軍圍秦州城。時愛申在軍中,有罪當死。宣宗問之樞帥,有知其名者奏此人將帥材,忠實可倚。宣宗命馳赦之,以為德順節度使、行元帥府事。正大四年春,大兵西來,擬以德順為坐夏之所,德順無軍,人甚危之。愛申識鳳翔馬肩龍舜卿者可與謀事,乃遺書招之,肩龍得書欲行,鳳翔總管禾速嘉國鑒以大兵方進,吾城可恃,德順決不可守,勸勿往。肩龍曰:「愛申平生未嘗識我,一見許為知己。我知德順不可守,往則必死,然以知己故。不得不為之死耳。」乃舉行橐付族父,明為死別,冒險而去。既至,不
 數日受圍,城中惟有義兵鄉軍八九千人,大兵舉天下之勢攻之。愛申假舜卿鳳翔總管府判官,守禦一與共之。凡攻百二十晝夜,力盡乃破,愛申以劍自剄,時年五十三。軍中募生致肩龍,而不知所終。臺諫有言當贈德順死事者官,以勸中外。詔各贈官,配食褒忠廟。



 肩龍字舜卿,宛平人。先世遼大族,有知興中府者,故人號興中馬氏。祖大中,金初登科,節度全、錦兩州。父成誼,明昌五年登科,仕為京兆府路統軍司判官。肩龍在太學有賦聲。宣宗初,有誣宗室從坦殺人,將置之死。人不敢言其冤,肩龍上書,大略謂:「從坦有將帥材,少出其右者,臣一
 介書生,無用於世,願代從坦死,留為天子將兵。」書奏,詔問:「汝與從坦交分厚歟?」肩龍對曰:「臣知有從坦,從坦未嘗識臣。從坦冤人,不敢言,臣以死保之。」宣宗感悟,赦從坦,授肩龍東平錄事,委行省試驗。宰相侯摯與語不契,留數月罷歸,將渡河,與排岸官紛競,搜篋中,得軍馬糧料名數及利害數事,疑其為姦人偵伺者,繫歸德獄根勘。適從坦至,立救出之。正大三年,客鳳翔,元帥愛申深器重之,至是,同死於難。



 禹顯,鴈門人。貞祐初,隸上黨公張開,累以戰功授義勝軍節度使、兼沁州招撫副使。元光二年四月,大帥達兒
 泬、按察兒攻河東,張開遣顯扼龍豬谷,夾攻敗之,擒元帥韓光國,獲輜重甲仗甚眾,追至祁縣而還,所歷州縣悉復之。顯將軍三百人,守襄垣,八年不遷。大帥嘗集河朔步騎數萬攻之,至於數四不能拔。既而,戰於玉女寨,大獲。開言於朝,權元帥右都監。正大六年冬十二月,軍內變,城破被擒。帥義之,不欲加害。初以鐵繩鈐之,既而密與舊部曲二十人遁去,聞上黨公軍復振,將往從之。大兵四向來追,顯適與負釜一兵相失,乞飯山寺中,僧走報焉,被執不屈死,時年四十一。



 秦州人張邦憲,字正叔,登正大中進士第,為永固令。天興二年,避兵徐州。卓
 翼率兵至城,邦憲被執,將驅之北,邦憲罵曰:「我進士也,誤蒙朝廷用為邑長,可從汝曹反耶!」遂遇害。



 劉全者,彭城民也。率鄉鄰數百避兵沫溝,推為砦主。北兵至徐,盡俘其老幼,全父亦在其中,北兵質之以招全,全縛其人送徐州,因竊其父以歸。徐帥益都嘉其忠,承制以為昭信校尉,遙領彭城縣尉。後遇國用安,怒其不附己,見殺。



\end{pinyinscope}