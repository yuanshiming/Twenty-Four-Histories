\article{列傳第十}

\begin{pinyinscope}

 ○婁室活女謀衍仲本名石古乃海里銀術可彀英本名撻懶麻吉子沃側拔離速習古迺



 婁室,字斡里衍,完顏部人。年二十一,代父白荅為七水諸部長。太祖克寧江州,使婁室招輸係遼籍女直,遂降移燉益海路太彎照撒等。敗遼兵于婆刺趕山。復敗遼
 兵,擒兩將軍。既而益改、捺末懶兩路皆降。進兵咸州,克之。諸部相繼來降,獲遼北女直係籍之戶。遼都統耶律訛里朵以二十餘萬眾來戍邊。太祖趨達魯古城,次寧江州西,召婁室。婁室見上于軍中。上見婁室馬多疲乏,以三百給之,使隸右翼宗翰軍,與銀術可縱兵衝其中堅,凡九陷陣,皆力戰而出。復與銀術可戍邊。



 及九百奚營等部來降,則與銀術可攻黃龍府,上使完顏渾黜、婆盧火、石古乃以兵四千助之,敗遼兵萬餘于白馬濼。宗雄等下金山縣,使婁室分兵二千,招沿山逃散之人。耶律捏里軍蒺藜山,斡魯古、婁室等破之,遂取顯州。太祖
 取黃龍府,婁室請曰:「黃龍一都會,且僻遠,茍有變,則鄰郡相扇而起。請以所部屯守。」太祖然之,仍合諸路謀克,命婁室為萬戶,守黃龍府。進都統,從杲取中京,與希尹等襲走迪六、和尚、雅里斯等,敗奚王霞末,降奚部西陳度訛里刺。遼主自鴛鴦濼西走,婁室等追至白水濼,獲其內庫寶物。數字遂與闍母攻破西京。復與闍母至天德、雲內、寧邊、東勝,其官吏皆降,獲阿疏。



 夏人救遼,兵次天德,婁室使突捻、補攧以騎二百為候兵,夏人敗之,幾盡。阿士罕復以二百騎往,遇伏兵,獨阿士罕脫歸。時久雨,諸將欲且休息,婁室曰:「彼再破吾騎兵,我若不復往,
 彼將以我怯,即來攻我矣。」乃選千騎,與習失、拔離速往。斡魯壯其言,從之。婁室遲明出陵野嶺,留拔離速以兵二百據險守之。獲生口問之,其帥李良輔也。將至野穀,登高望之。夏人恃眾而不整,方濟水為陣,乃使人報斡魯。婁室分軍為二,迭出迭入,進退轉戰三十里。過宜水,斡魯軍亦至,合擊敗之。



 遼都統大石犯奉聖州,壁龍門東二十五里,婁室、照里、馬和尚等以兵取之,生獲大石,其眾遂降。遼闢里刺守奉聖州,棄城遁去。後與宗望追遼帝,婁室,蒲宗以二十騎候敵,敗其軍三千人於三山,有千人將趨奉聖州,蒲察復敗之,擒其主帥而還。夏人屯
 兵於可敦館,宗翰遣婁室戍朔州,築城於霸德山西南二十里,遂破朔州西山兵二萬,擒其帥趙公直。其後復襲遼帝于餘都谷,獲之。賜鐵券,惟死罪乃笞之,餘罪不問。



 銀術可圍太原,宋統制劉臻救太原,率眾十萬出壽陽,婁室擊破之,繼敗宋兵數千於榆次。宋張灝軍出汾州,拔離速擊走之。灝復營文水,數室也突葛速、拔離速與戰,灝大敗。宗翰定太原,婁室取汾、石二州,及其屬縣溫泉、方山、離石,蒲察降壽陽,取平定軍及樂平,復招降遼州及榆社、遼山、和順諸縣。宗翰趨汴州,使婁室等自平陽道先趨河南,曰:「若至澤州,與賽里、婆盧火、習失遇,
 當與俱進。」習失之前軍三謀克,敗宋兵三千于襄垣,遇伏兵二千,又敗之。撒刺荅破天井關,復破步兵於孔子廟南,遂降河陽。婁室軍至,既渡河,遂薄西京。城中兵來拒戰,習失逆擊敗之,西京降。婁室取偃師,永安軍、鞏縣降。撒刺荅敗宋兵於汜水。於是,滎陽、滎澤、鄭州、中牟相次皆降。宗翰已與宗望會軍于汴,使婁室率師趨陜津,攻河東郡縣之未下者。阿離士罕敗敵于河上,撒按敗敵于陜城下,鶻沙虎降虢州守陴卒三百人,遂克陜府。習古迺、桑袞破陜之散卒于平陸西北。活女別破敵於平陸。婁室破蒲、解之軍二萬,盡覆之,安邑、解州皆降,遂
 克河中府,降絳、慈、隰、石等州。



 宗翰往洛陽,使婁室取陜西,敗宋將范致虛軍,下同、華二州,克京兆府,獲宋制置使傅亮,遂克鳳翔。阿鄰等破宋大兵於河中,斡魯破宋劉光烈軍於馮翊,訛特刺、桑袞敗敵於渭水,遂取下邽。宗翰會京輔伐康王,命婁室、蒲察專事陜西,以婆盧火、繩果監戰。繩果等遇敵於蒲城及同州,皆破之。婁室、蒲察克丹州,破臨真,進克延安府,遂降綏德軍及靜邊、懷遠等城寨十六,復破青澗城。宋安撫使折可求以麟、府、豐三州,及堡寨九,降于婁室。晉寧所部九寨皆降,而晉寧軍久不下,婁室欲去之,賽里不可,曰:「此與夏鄰,且生
 他變。」城中無井,日取河水以為飲,乃決渠於東,泄其水,城中遂困。李位、石乙啟郭門降,諸將率兵入城。守將徐徽言據子城,戰三日,眾潰,徽言出奔,獲之。使之拜,不聽,臨之以兵,不為動,縶之軍中。使先降者諭之使降,徽言大罵,與統制孫昂皆不屈,乃并殺之。遂降定安堡、渭平寨及鄜、坊二州。於是,婁室、婆盧火守延安,折可求屯綏德,蒲察還守蒲州。延安、鄜、坊州皆殘破,人民存者無幾,婁室置官府輯安之。別將斡論降建昌軍。京兆府叛,婁室復討平之。遂與阿盧補、謀里也至三原,訛哥金、阿骨欲擊淳化兵,敗之。婁室攻乾州,已築甬道,列駁具,而州
 降。遂進兵克邠州,軍于京兆。



 陜西城邑已降定者,輒復叛,於是睿宗以右副元帥,總陜西征伐。時婁室已有疾,睿宗與張浚戰于富平,宗弼左翼軍已卻,婁室以右翼力戰,軍勢復振,張浚軍遂敗。睿宗曰:「力疾鏖戰,以徇王事,遂破巨敵,雖古名將何以加也。」以所用犀玉金銀器,及甲胄,并馬七匹與之。



 天會八年,薨。十三年,贈泰寧軍節度使,兼侍中,加太子太師。皇統元年,贈開府儀同三司,追封莘王。以正隆例改贈金源郡王,配享太宗廟廷,謚壯義。子活女、謀衍、石古乃。



 活女,年十七從攻寧江州,力戰創甚,扶出陣間。太祖憑
 高望見,問之,知是婁室子,親撫慰賜藥,歎曰:「此兒他日必為名將。」其攻濟州,敗敵八千。與敵遇於信州,移刺本陷于陣,活女力戰出之,敵遂北。敗耶律佛頂等兵于瀋州。及宗翰以兵襲奚王霞末,活女以兵三百,敗敵二千。從攻乙室部,敗之,破其二營。迭刺部族叛,率二謀克突入,大破之。



 活女常從婁室圍太原,宋將種師中以兵十萬來援,活女擊敗之。大軍至河,無船,不得渡。婁室遣活女循水上下,活女率軍三百,自孟津而下,度其可渡,遂引軍以濟,大軍於是皆繼之。宋將郭京出兵數萬,趨婁室營,活女從旁奮擊,敵亂,遂破之。師還,破敵於平陸渡,
 得其船以濟。又以兵破敵於張店原。時屯留、太平、翼城皆有重敵,並破之。又分兵取陜西,蒲州降,留活女鎮之。攻鳳翔,活女先登。睿宗定陜西,活女為都統,進攻涇州,敗其兵。王開山以兵拒歸路,邀戰,再擊,再敗之,遂降京兆、鳳翔諸縣。



 婁室薨,襲合扎猛安,代為黃龍府路萬戶。天眷二年,為元帥右都監,遷左監軍。元帥府罷,改安化軍節度使。歷京兆尹,封廣平郡王,以正隆例,改封代國公,進封隋國公,謚貞濟。卒年六十一。



 謀衍,勇力過人,善用長矛突戰。天眷間,充牌印祗候,授顯武將軍,擢符寶郎。皇統四年,其兄活女襲濟州路萬
 戶,以親管奧吉猛安讓謀衍,朝廷從之,權濟州路萬戶。八年,為元右都監。天德三年,為順天軍節度使,歷河間、臨潢尹,數月改婆速路兵馬都總管。



 撒八反,謀衍往討之,是時世宗為東京留守,自將討括里還,遇謀衍于常安縣,盡以甲士付之。世宗還東京,完顏福壽、高忠建率所部南征軍,亡歸東京。謀衍亦率其軍來附,即以臣禮上謁,遂殺高存福、李彥隆等。謀衍、福壽、忠建及諸將吏民勸進,世宗即位,拜右副元帥。都統白彥敬,副統紇石烈志寧在北京,拒不受命,謀衍伐之,遇其眾于建州之境,皆不肯戰,彥敬、志寧遂降。


二年正月,謀衍率諸軍
 討窩斡,會兵於濟州,合甲士萬三千人,過泰州,至術虎崖,乃捨輜重,持數日糧,輕騎追之。是時窩斡新敗於泰州,將走濟州。謀衍兵至長濼南,獲其諜者,知敵將由別路邀糧運,遂分軍往迎之。敵吏者來降,謀衍用其計,因夜亟往邀敵輜重,忽大風,不能燧火,路暗莫相辨,比曉纔行三十餘里。將至敵營,將士少憩,謀衍率善射者數十騎,往覘之。而都統志寧、克寧等,已敗敵眾二萬於餘長濼,追殺甚眾,敵遂西遁。志寧軍先追及於霿
 \gezhu{
  松}
 河,急擊敗之。而謀衍貪鹵掠,不復追,以故敵得縱去,遂涉懿州界,陷靈山、同昌、惠和等縣,窺取北京,西敗三韓縣。
 惟克寧軍追躡,謀衍託馬弱,引還懿州。上聞之,下詔切責謀衍,以僕散忠義為右副元帥代之,紇石烈志寧為右監軍代完顏福壽。而謀衍子斜哥暴橫軍中,詔勒歸本貫。



 謀衍至京師,以為同判大宗正事,世宗責之曰:「朕以汝為將,汝不追賊,當正汝罪。以汝父婁室有大功,特免汝死。汝雖非宗室,而授此職,汝其勉之。」未幾,速頻路軍士術里古,告斜哥寄書與謀衍謀反,有司并上其書,世宗察其誣,詔鞫告者,術里古款伏,遂誅之。召謀衍謂之曰:「人有告卿子為反謀者,朕知卿必不為此,今告者果自服罪,宜悉此意。」



 初,窩斡方熾,上使溫迪罕阿魯帶,
 守古北口。及窩斡敗于陷泉,入于奚中,率諸奚攻古北口。阿魯帶因其妻生日,輒離軍六十里,賊眾聞之,來襲,殺傷士卒甚眾。阿魯帶坐除名。詔謀衍,蒲察烏里雅、蒲察通以兵三千,會舊屯兵,擊之。擒賊黨猛安合住。未幾,窩斡平,乃還。



 七年,出為北京留守,上御便殿,賜食,及御服衣帶佩刀,謂之曰:「以卿故老,欲以均勞逸,故授此職,卿其勉之。」改東京留守,封榮國公。大定十一年,薨,年六十四。



 謀衍性忠厚,善擊球射獵,時論以為雖智略不及其父,而勇敢肖之云。



 仲,本名石古乃。體貌魁偉,通女直、契丹、漢字。其兄斡魯
 為統軍,愛仲才,欲使通吏事,每視事,常在左右,遇事輒問之,應對如響,斡魯嘆曰:「此子必為令器。」皇統初,充護衛,授世襲謀克。天德元年,攝其兄活女濟州萬戶,部內稱治。除濱州刺史,以母憂去官。起復知積石軍事,轉同知河南尹。



 正隆六年,伐宋,為神勇軍副都總管。與大軍北還,除同知大興尹,將兵二千,益遵化屯軍。備契丹。遷西南路招討使,兼天德軍節度使,政尚忠信,決獄公平,蕃部不敢寇邊。召為左副都點檢,宿衛嚴謹,每事有規矩,後來者守其法,莫能易也。世宗常謂侍臣曰:「石古乃入直,朕寢益安。」



 五年,宋人請和,為姪國,不稱臣,仲為報
 問使。仲請與宋主相見禮儀,世宗曰:「宋主親起立接書,則授之。」及至宋,一一如禮。正隆用兵,宋人執商州刺史完顏守能以歸,至是,仲取守能與俱還,上嘉之。轉都點檢,兼侍衛親軍都指揮使,遷河南路統軍使,上曰:「卿在禁近,小心畏慎。河南控制江、淮,為國重地,卿益勉之。」賜廄馬、金帶、玉吐鶻。後有罪,解職。久之,起為西北路招討使,改北京留守,卒。



 海里,婁室族子。體貌豐偉,善用槊。婁室為黃龍府萬戶,海里從徙於孰吉訛母。從婁室追及遼主於朔州阿敦山,遼主從數十騎逸去,婁室遣海里及術得,往見遼主,
 諭之使降。遼主已窮蹙,待於阿敦山之東,婁室因獲之,賞海里金五十兩、銀五百兩、幣帛二百匹、綿三百兩。睿宗經略陜西,海里戰卻吳玠軍於涇、邠之南,尋遣脩棧道,宋人恐棧道成,以兵來拒,破其兵,賞銀百五十兩、奴婢十人。



 天眷元年,擢宿直將軍。與定宗磐、宗雋之亂,再遷廣威將軍,除都水使者。改西北路招討都監,歷復州、灤州刺史。耶盧椀群牧使,迭刺部族節度使,同知大興尹、兼中都路兵馬都總管,改武寧軍節度使,廣寧尹。卒,年六十二。



 銀術可,宗室子。太祖嗣位,使蒲家奴如遼取阿疏,事久
 不決,乃使習古迺、銀術可繼往。當是時,遼主荒于政,上下解體。銀術可等還,具以遼政事人情告太祖,且言遼國可伐之狀。太祖決意伐遼,蓋自銀術可等發之。



 太祖與耶律訛里朵戰於達魯古城,遼兵二十餘萬,銀術可、婁室率眾衝其中堅,凡九陷陣,輒戰而出,大敗遼軍。銀術可為謀克,遂與婁室戍邊,復與婁室、渾黜、婆盧火、石古乃等攻黃龍府,敗遼兵萬餘於白馬濼。太祖拒遼兵,銀術可守達魯古城。收國二年,分鴨撻、阿懶所遷謀克二千戶,以銀術可為謀克,屯寧江州。



 遼大冊使習泥烈遣回,約以七月半至,而盡九月習泥烈未來,上使諸軍過江
 屯駐。遼曳刺、麻答十三人,兵士八人縱火於渾河,以絕芻牧。銀術可獲之,乃知遼邊吏乙薛使之,太祖命釋之。從都統杲克中京,銀術可與習古乃、蒲察、胡巴魯率兵三千,擊奚王霞末於京西七十里,霞未棄兵遁。遼主西奔天德,銀術可以兵絕其後,遼主遂見獲。



 後從宗翰伐宋,圍太原,宗翰進兵至澤州,及宗翰還西京,太原未下,皆命銀術可留兵圍之。招討都監馬五破宋兵於文水。節度使耿守忠等敗宋黃迪兵於西都谷,所殺不可勝計。宋樊夔、施詵、高豐等軍來救太原,分據近部,銀術可與習失、盃魯、完速大破之。索里乙室,破宋兵於太谷。宋
 兵據太谷、祁縣,阿鶻懶、拔離速復取之。種師中出井陘,據榆次,救太原,銀術可使斡論擊之,破其軍。活女斬師中於殺熊嶺,進攻宋制置使姚古軍于隆州谷,大敗之。撒里土敗宋軍於回馬口,郭企忠殲宋軍於五臺。及宗翰定太原,與宗望會兵于汴,銀術可等攻汴城,克之。師還,銀術可降岢嵐、寧化等軍,攻嵐州拔之,招降火山軍。與希尹同賜鐵券。



 宗翰趨洛陽,賽里取汝州,銀術可取鄧州,殺其將李操等。薩謀魯入襄陽,拔離速入均州,馬五取房州,擒轉運使劉吉、鄧州通判王彬。拔離速破唐、蔡、陳三州,克潁昌府,沙古質別克舊潁昌。



 宗翰會伐康
 王,銀術可守太原。天會十年,為燕京留守。天會十三年,致仕,加保大軍節度使,同中書門下平章事,遷中書令,封蜀王。天眷三年,薨,年六十八。以正隆例贈金源郡王,配饗太宗廟廷。大定十五年,謚武襄,改配享太祖廟廷,子彀英。



 彀英,本名撻懶。幼警敏有志膽,初丱角,太祖見而奇之。年十六,父銀術可授以甲,使從伐遼,常為先鋒,授世襲謀克。



 宗翰自太原還西京,銀術可圍守之,彀英在行間,屢有功。宋兵數萬救太原,至南關,銀術可與弟拔離速、完顏婁室等擊之,當隘巷間,一卒揮刀向拔離速,彀英
 以刀斷其腕,一卒復從旁以槍刺之,彀英斷其槍,追殺之。拔太原,下河東諸州,攻汴京,皆有功。與都統馬五徇地漢上,至上蔡,以先鋒破孔家軍。睿宗攻開州,彀英先登,流矢中其口,睿宗親視之,創未愈,強起之,攻大名府。第功,宗弼第一,彀英次之。攻東平,彀英居最。



 拔離速襲宋康王于揚州,彀英為先鋒。拔離速追宋孟后於江南,彀英前行趨潭州。宋大兵在常武,彀英以選兵薄其城,敗千餘人。明日,城中出兵來戰,彀英以五百騎敗之,獲馬二百匹,遂攻常武。拔離速以諸軍為大陣,居其後,彀英以五百騎為小陣,當前行,即麾兵馳宋軍,宋軍亂,遂
 大敗之。拔離速觀其周旋,嘆賞之。



 其後河東郡縣多叛,彀英以先鋒攻絳州,克之。復攻沁州,飛砲擊其石脅,歸營中。諸軍攻沁州,三日不能下,別將骨赧強起彀英指麾士卒,遂克之。



 攝河東路都統,從左監軍移刺余睹招西北諸部。彀英將騎三千五百,平其九部,獲生口三千,馬牛羊十五萬。以先鋒破宋吳山軍,再戰再勝,遂恤宋兵於隘,死者不可勝計,宋兵遁去。



 宗弼再取和尚原,彀英以本部破宋五萬人,遂奪新叉口,宗弼留兵守之。是夜,大雪,道路皆冰,和尚原宋兵勢重不可徑取,宗弼用彀英微,入自傍近高山叢薄翳薈間,出其不意,遂取
 和尚原。



 彀英請速入大散關,自以本部為殿,以備伏兵。宗弼至仙人關,彀英先攻之,宗弼止之,彀英不止,宗弼以刀背擊其兜鍪,使之退,彀英曰:「敵氣巳沮,不乘此而取之,後必悔之。」已而果然。宗弼嘆曰:「既往不咎。」乃班師。彀英殿,且戰且卻,遂達秦中。



 齊國初廢,元帥右監軍撒離喝馳驛撫治諸郡,至同州,故齊觀察使李世輔出迎,陽墜馬稱折臂,歸。撒離喝入城,世輔詐使通判獻甲,以壯士十人,被甲上事,世輔自壁後突出,執撒離喝。彀英方索馬于外,變起倉卒,不得入。城門已閉,皆有兵衛,至東門,合荅雅領騎三十餘,與彀英遇,遂斬門者出。
 而世輔擁眾自西門出,彀英與合荅雅襲之,一進一退以綴世輔,使不得速。世輔慮救兵至,乃要撒離喝與之盟,勿使追之。留撒離喝於道側,彀英識其聲,與騎而歸。除安遠大將軍,攝太原尹,四境咸治,兼攝河東南、北兩路兵馬都總管。



 朝廷以河南、陜西與宋,已而復取之,師至耀州。宋人每旦出城,張旗閱隊,抵暮而還。道隘,騎不得逞。彀英請兵五百,薄暮先使五十人趨山巔,令之曰:「旦日視敵出,舉幟指其所向。」乃以餘兵伏山谷間。明日,城中人出閱如前,山巔旗舉,伏兵發,宋兵爭馳入城。彀英麾軍登城,拔宋幟,立金軍旗幟。宋兵後者望見之不
 敢入,遂降,城中人亦降。



 宋吳玠擁重兵據涇州,涇原以西多應之。元帥撒離喝欲退守京兆,俟河南、河東軍。彀英曰:「我退守,吳玠必取鳳翔、京兆、同、華,據潼關,吾屬無類矣。」撒離喝曰:「計將安出?」彀英曰:「事危矣,不如速戰。我軍陣涇之南原,宋兵必自西原來。彀英與斜補出各以選騎五百摧其兩翼,元帥當其中擊之,可以得志。」監軍拔離速曰:「二子當其左右,拔離速願當其中。元帥據岡阜,多張旗幟為疑兵,可以得志。」撒離喝從之。吳玠兵果自西原來,彀英、斜補出擊其左右,自旦至午,吳玠左右軍少退,拔離速當其前衝擊之,遂敗玠軍,僵尸枕藉,大
 澗皆滿。自此蜀人喪氣,不敢復出,關、陜遂定。



 歷行臺吏部工部侍郎,從宗弼巡邊,遷刑部尚書,轉元帥左都監。天德二年,遷右監軍。元帥府罷,改山西路統軍使,領西南、西北兩路招討兵馬,坐無功,降臨海軍節度使,歷平陽、太原尹。正隆末,為中都留守,兼西北面都統,討契丹撒八,駐軍歸化州。



 世宗即位於遼陽悻使彀英姪阿魯瓦持詔往歸化,命彀英為左副元帥,就遣使召陜西統軍徒單合喜,宣大定改元詔、赦于西南、西北招討司,河東、河北、山東諸路州鎮,調猛安軍屯京畿。阿魯瓦見彀英,彀英猶豫未決,士卒皆欲歸世宗,彀英不得已,乃受詔。
 以元帥令下諸路,亟泥馬槽二萬具,諸路聞之,以為大軍且至,然後遣人宣赦,所至皆聽命。



 大定元年十一月,彀英以軍至中都,同知留守璋請至府議事。彀英疑璋有謀,乃陽許諾,排節仗若將往者,遂率騎從出施仁門,駐兵通州。見世宗於三河。詔彀英以便宜規措河南、陜西、山東邊事。二年正月,至南京,遂復汝、潁、嵩等州縣,授世襲猛安。入拜平章政事,罷為東京留守,未行,改濟南尹。



 初,彀英宿將恃功,在南京頗瀆貨,不恤軍民。詔使問以邊事,彀英不答,謂詔使曰:「爾解何事,待我到闕奏陳。」及召入,竟無一語及邊事者。在相位多自專,己所欲輒
 自奏行之。除留守,輒忿忿不接賓客,雖近臣往亦不見。上怒,遂改濟南。上數之曰:「朕念卿父有大功于國,卿舊將亦有功,故改授此職,卿宜知之。若復不悛,非但不保官爵,身亦不能保也。」彀英頓首謝。



 久之,改平陽尹,致仕。起為西京留守,以母憂去官。尋以本官起復。俄復為東京,歷上京,詔曰:「上京王業所起,風俗日趨詭薄,宗室聚居,號為難治。卿元老大臣,眾所聽服,當正風俗,檢制宗室,持以大體。」十五年。致仕。



 久之,史臣上《太宗》、《睿宗實錄》,上曰:「當時舊人親見者,惟彀英在。」詔脩撰溫迪罕締達往北京就其家問之,多更定焉。



 十九年,薨,年七十四。最
 前後以功被賞者十有一,金為兩二百五十,銀為兩六千五百,絹為疋八百,綿為兩二千,馬三百十有四,牛羊六千五百,奴婢百三十人。



 麻吉,銀術可之母弟也。年十五,隸軍中,從破高麗兵,下寧江州,平係遼女直,克黃龍府,皆身先力戰,以功為謀克,繼領猛安。破奚兵千餘。自斡魯古攻下咸、信、瀋州及東京諸城,麻吉皆有功。都統杲取中京,與稍合、胡拾答別降楚里迪部,屯兵高州。以兵援蒙刮勃堇,大破敵兵,變敗恩州兵五萬人。討平遼人聚中京山谷者,降三千餘人。戰于高州境上,伏矢射之中目,遂卒。



 麻吉大小三
 十餘戰,所至皆捷。皇統中,贈銀青光祿大夫,謚毅敏。子沃側。



 沃側,年十七,隸軍中,從拔離速擊遼將馬五,敗之。麻吉死,領其職。宗望伐宋,至河上。宋兵屯于河外,以二舟來伺我師,乃遣沃側率勇士數輩,以一舟往迎之,盡俘以還。襲康王於江、淮間,沃側皆與焉。師還,駐東平。及廢齊,屯兵河北,招降旁近諸營,多獲畜產兵仗,軍帥嘉之,賞以甲馬。



 後攻陜西,為右翼都統,攻城破敵,皆與有功。師還,正授謀克。遷華州防禦使,屬關中歲饑,盜賊充斥,沃側募兵討平之,部以無事。郡人列狀丐留,不報。未幾,除
 迪列部族節度使,改迭刺部。用廉入為都水使者,秩滿,同知燕京留守事,為西北路招討使。



 撒八秩滿已數月,冒其俸祿,不即解去,沃側發其事。撒八反,沃側遇害。



 拔離速,銀術可弟。天輔六年,宗翰在北安州,將會斜也于奚王嶺,遼兵奄至古北口,使婆盧火、渾黜各領兵二百,擊之。渾黜請濟師,宗翰欲自往,希尹、婁室曰:「此易與耳,請以千人為公破之。」渾黜以騎士三十人前行,至古北口,遇其遊兵,逐入山谷,遼人以步騎萬餘迫戰,亡騎五人,渾黜退據關口。希尹、婁室至,拔離速、訛謀罕、胡實海推鋒奮擊,大破之,斬馘甚眾,盡獲甲胄輜重。希尹與
 撒里古獨、裴滿突捻敗其伏兵,殺千餘人,獲馬百餘匹。婁室拒夏人出陵野嶺,留拔離速以兵二百,據險守之。



 銀術可圍太原,近縣先已降,宋軍來救太原者復據太谷、祁縣,拔離速、阿鶻懶復取之。宋姚古軍隆州谷,拔離速敗之,張灝兵出汾州,又擊走之。天會四年,克太原,拔離速為管勾太原府路兵馬事,復與婁室敗宋兵于文水,遂從宗翰圍汴。與銀術可略地襄、鄧,入均州,還攻唐、蔡、陳三州,皆破之,克潁昌府。遂與泰欲、馬五襲宋康王于揚州,康王渡江入于建康。



 天會十五年,遷元帥左都監。宗弼再定河南,撒離喝經略陜西,至涇州,拔離速大
 破宋軍于渭州,渭州、德順軍皆降,陜西平。遷元帥左監軍,加金吾衛上將軍,卒,謚敏定。



 習古迺,亦書作實古迺。嘗與銀術可俱往遼國取阿疏,還言遼人可取之狀,太祖始決意伐遼矣。婆盧火取居庸關,蕭妃自古北口出奔,太祖使習古乃追之,不及。後為臨潢府軍帥,討平迭刺,其群官率眾降者,請使就領諸部。太宗賜以空名宣頭及銀牌,使以便宜授之。獲遼許王莎邏、駙馬都尉蕭乙辛。遼梁王雅里在紇里水自立,不知果在何處,至是始知之。於是,徙遼降人於泰州,時暑未可徙,習古乃請姑處之嶺西。及習古迺築新城
 於契丹周特城,詔置會平州。



 烏虎里部人迪烈、劃沙率部族降,朝廷以撻僕野為本部節度使,烏虎為都監。習古迺封還撻僕野等宣誥,以便宜加撻僕野散官,填空名告身授之,及錄上降附有勞故官八百九十三人,朝廷從之。於是,迪烈加防禦使,為本部節度使。劃沙加諸司使,為節度副使,知迪烈底部事。撻離答加左金吾衛上將軍,節度副使,知突鞠部事。阿枲加觀察使,為本部節度使。其餘遷授有差。以厖葛城地分賜烏虎里、迪烈底二部及契丹人,其未墾者聽任力占射。



 久之,領咸州煙火事。天會六年,完顏慎思所部及其餘未置猛安謀
 克戶口,命習古乃通閱具籍以上。天會十年,改南京路軍帥司為東南路都統司,習古乃為都統,移治東京,鎮高麗。



 贊曰:金啟疆土,斡魯、斡魯古方面功最先著,婆盧火、婁室最先封,泰州之邊圉,黃龍之沖要,寄亦重矣。若闍母之勤勞南路,婁室之經營陜西,銀術可之圍守太原,勞亦至矣。斡魯古之不治,闍母之敗,譴罰之亟,諸將慴焉。夫能以弱小終制強大,其效驗與。銀術可、習古乃觀人之國而知其可伐,古語云「國有八觀」,善矣夫。



\end{pinyinscope}