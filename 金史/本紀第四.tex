\article{本紀第四}

\begin{pinyinscope}

 熙宗



 熙宗弘基纘武莊靖孝成皇帝,諱亶,本諱合剌,太祖孫,景宣皇帝子。母蒲察氏。天輔三年己亥歲生。天會八年,諳班勃極烈杲薨,太宗意久未決。十年,左副元帥宗翰、右副元帥宗輔、左監軍完顏希尹入朝,與宗乾議曰:「諳班勃極烈虛位已久,今不早定,恐授非其人。合剌,先帝嫡孫,當立。」相與請於太宗者再三,乃從之。四月庚午,詔
 曰:「爾為太祖之嫡孫,故命爾為諳班勃極烈,其無自謂沖幼,狎于童戲,惟敬厥德。」諳班勃極烈者,太宗嘗居是官,及登大位,以命弟杲。杲薨,帝定議為儲嗣,故以是命焉。



 十三年正月己巳,太宗崩。庚午,即皇帝位。甲戌,詔中外。詔公私禁酒。癸酉,遣使告哀于齊、高麗、夏及報即位,仍詔齊自今稱臣勿稱子。二月乙巳,追謚太祖后唐括氏曰聖穆皇后,裴滿氏曰光懿皇后。追冊太祖妃僕散氏曰德妃,烏古論氏曰賢妃。辛酉,改葬太祖于和陵。



 三月己卯,齊、高麗使來弔祭。庚辰,謚大行皇帝曰文烈,廟號太宗。乙酉,葬太宗于和陵。甲午,以國論右勃極烈、都
 元帥宗翰為太保,領三省事,封晉國王。戊戌,詔諸國使賜宴。不舉樂。四月戊午,齊、高麗遣使賀即位。丙寅,昏德公趙佶薨,遣使致祭及賻贈。是月,甘露降于熊岳縣。



 五月甲申,左副元帥宗輔薨。九月壬申,追尊皇考豐王為景宣皇帝,廟號徽宗,皇妣蒲察氏為惠昭皇后。戊寅,尊太祖后紇石烈氏、太宗后唐括氏皆為太皇太后,詔中外。乙酉,改葬徽宗及惠昭后于興陵。



 十一月,以尚書令宋國王宗磐為太師。乙亥,初頒歷。己卯,以元帥左監完顏希尹為尚書左丞相兼侍中,太子少保高慶裔為左丞,平陽尹蕭慶為右丞。己丑,建天開殿于爻剌。十二
 月癸亥,始定齊、高麗、夏朝賀、賜宴、朝辭儀。以京西鹿囿賜農民。



 十四年正月己巳朔,上朝太皇太后于兩宮。齊、高麗、夏遣使來賀。癸酉,頒歷于高麗。丁丑,太皇太后紇石烈氏崩。乙酉,萬壽節,齊、高麗、夏遣使來賀。上本七月七日生,以同皇考忌日,改用正月十七日。二月癸卯,上尊謚曰欽憲皇后,葬睿陵。三月壬午,以太保宗翰、太師宗磐、太傅宗乾並領三省事。丁酉,高麗遣使來弔。八月丙辰,追尊九代祖以下曰皇帝、皇后,定始祖、景祖、世祖、太祖、太宗廟皆不祧。癸亥,詔齊國與本朝軍民訴訟相關者,文移署年,止用天會。十月甲寅,以吳激為高麗王
 生日使,蕭仲恭為齊劉豫回謝并生日正旦使。



 十五年正月癸亥朔,上朝太皇太后于明德宮。齊、高麗、夏遣使來賀。初用《大明歷》。己卯,萬壽節,齊、高麗、夏遣使來賀。六月庚戌,尚書左丞高慶裔、轉連使劉思有罪伏誅。七月辛巳,太保、領三省事、晉國王宗翰薨。丙戌夜,京師地震。封皇叔宗雋、宗固,叔祖暈皆為王。丁亥,汰兵興濫爵。十月乙卯,以元帥左監軍撻懶為左副元帥,封魯國王。宗弼右副元帥,封瀋王。知樞密院事兼侍中時立愛致仕。十一月丙午,廢齊國,降封劉豫為蜀王,詔中外。置行臺尚書省于汴。十二月戊辰,劉豫上表謝封爵。癸未,詔改
 明年為天眷元年。大赦。命韓昉、耶律紹文等編修國史。以勖為尚書左丞、同中書門下平章事。徙蜀王劉豫臨潢府。



 天眷元年正月戊子朔,上朝明德宮。高麗、夏遣使來賀。頒女直小字。封大司空昱為王。甲辰,萬壽節,高麗、夏遣使來賀。二月壬戌,上如爻剌春水。乙丑,幸天開殿。己巳,詔罷來流水、混同江護邏地,與民耕牧。三月庚寅,以禁苑隙地分給百姓。戊申,以韓昉為翰林學士。



 四月丁卯,命少府監盧彥倫營建宮室,止從儉素。壬午,朝享于天元殿。立裴滿氏為貴妃。五月己亥,詔以經義、詞賦兩科
 取士。



 六月戊午,上至自天開殿。秋七月辛卯,左副元帥撻懶、東京留守宗雋來朝。丁酉,按出滸河溢,壞廬舍,民多溺死。壬寅,左丞相希尹罷。八月甲寅朔,頒行官制。癸亥,回鶻遣使朝貢。己卯,以河南地與宋。以右司侍郎張通古等使江南。以京師為上京,府曰會寧,舊上京為北京。九月甲申朔,以奭為會寧牧,封鄧王。乙未,詔百官誥命,女直、契丹、漢人各用本字,渤海同漢人。丁酉,改燕京樞密院為行臺尚書省。戊戌,上朝明德宮。甲辰,以奕為平章政事。己酉,省燕中西三京、平州東、西等路州縣。辛亥,權行臺左丞相張孝純致仕。



 十月甲寅朔,以御前管
 勾契丹文字李德固為參知政事。丙寅,封叔宗強為紀王,宗敏邢王,太宗子斛魯補等十三人為王。己巳,始禁親王以下佩刀入宮。辛未,定封國制。癸酉,以東京留守宗雋為尚書左丞相兼侍中,封陳王。十一月丙辰,以康宗以上畫像工畢,奠獻于乾元殿。十二月癸亥,新宮成。甲戌,高麗遣使入貢。丁丑,立貴妃裴滿氏為皇后。



 二年正月壬午朔,高麗、夏遣使來賀。戊戌,萬壽節,高麗、夏遣使來賀。以左丞相宗雋為太保、領三省事,進封袞國王。興中尹完顏希尹復為尚書左丞相兼侍中。二月乙未,上如天開殿。三月丙辰,命百官詳定儀制。四月甲
 戌,百官朝參,初用朝服。己卯,宋遣使謝河南地。



 五月戊子,太白晝見。乙巳,上至自天開殿。六月己酉朔,初御冠服。辛亥,吳十謀反,伏誅。己未,上從容謂侍臣曰:「朕每閱《貞觀政要》,見其君臣議論,大可規法。」翰林學士韓昉對曰:「皆由太宗溫顏訪問,房、杜輩竭忠盡誠。其書雖簡,足以為法。」上曰:「太宗固一代賢君,明皇何如?」昉曰:「唐自太宗以來,惟明皇、憲宗可數。明皇所謂有始而無終者。初以艱危得位,用姚崇、宋璟,惟正是行,故能成開元之治。末年怠於萬機,委政李林甫,奸諛是用,以致天寶之亂。茍能慎終如始,則貞觀之風不難追矣。」上稱善。又曰:「周
 成王何如主?」昉對曰:「古之賢君。」上曰:「成王雖賢,亦周公輔佐之力。後世疑周公殺其兄,以朕觀之,為社稷大計,亦不當非也。」



 七月辛巳,宋國王宗磐、袞國王宗雋謀反,伏誅。丙戌,以右副元帥宗弼為都元帥,進封國王。丁亥,以誅宗磐等詔中外。己丑,以左副元帥撻懶為臺左丞相,杜充為行臺右丞相,蕭寶、耶律輝行臺平章政事。甲午,咸州祥穩沂王暈坐與宗磐謀反,伏誅。辛丑,以太傅、領三省事宗乾為太師,領三省如故,進封梁宋國王。



 八月辛亥,行臺左相撻懶、翼王鶻懶及活離胡土、撻懶子斡帶、烏達補謀反,伏誅。丁丑,太白晝見。



 九月戊
 寅朔,降封太宗諸子。大司空昱罷。丙申,初居新宮。立太祖原廟于慶元宮,壬寅,宋遣王倫等乞歸父喪及母韋氏等,拘倫不遣。以溫都思忠諸路廉問。十月癸酉,夏國使來告喪。十二月,豫國公昱薨。



 三年正月丁丑朔,高麗、夏遣使來賀。癸巳,萬壽節,高麗、夏遣使來賀。以都元帥宗弼領行臺尚書省事。四月乙巳朔,溫都思忠廉問諸路,得廉吏杜遵晦以下百二十四人,各進一階,貪吏張軫以下二十一人皆罷之。癸丑,蜀國公完顏銀術哥薨。丁卯,上如燕京。



 五月丙子,詔元帥府復取河南、陜西地。己卯,詔冊李仁孝為夏國王。命
 都元帥宗弼以兵自黎陽趨汴,右監軍撒離合出河中趨陜西。



 是月,河南平。六月,陜西平。上次涼陘大旱。使蕭彥讓、田決西京囚。秋七月癸卯朔,日有食之。乙卯,宗弼遣使奏河南、陜西捷。丁卯,詔文武官五品以上致仕,給俸祿之半,職三品者仍給傔人。



 八月辛巳,招撫諭陜西五路。壬午,初定公主、郡縣主及駙馬官品。九月壬寅朔,宗弼來朝。戊申,上至燕京。己酉,親饗太祖廟。庚申,宗弼還軍中。夏國遣使謝賻贈。癸亥,殺左丞相完顏希尹、右丞蕭慶及希尹子昭武大將軍把搭、符寶郎漫帶。戊辰,夏國遣使謝封冊。十一月癸丑,以孔子四十九代
 孫璠襲封衍聖公。癸亥,以都點檢蕭仲恭為尚書右丞,前西京留守昂為平章政事。甲子,行臺尚書右丞相杜充薨。十二月乙亥,都元帥宗弼上言宋將岳飛、張俊、韓世忠率眾渡江,詔命擊之。丁丑,地震。己亥,以元帥左監軍阿離補為左副元帥,右監軍撒離合為右副元帥。



 皇統元年正月辛丑朔,高麗、夏遣使來賀。庚戌,群臣上尊號曰崇天體道欽明文武聖德皇帝。初御袞冕。癸丑,謝太廟。大赦。改元。丁巳,萬壽節,高麗、夏遣使來賀。己未,初定命婦封號。夏國請置榷場,許之。己巳,封平章政事昂為漆水郡王。



 二月戊寅,詔諸致仕官職俱至三品者,
 俸祿人力各給其半。宗弼克廬州。乙酉,改封海濱王耶律延禧為豫王,昏德公趙佶為天水郡王,重昏侯趙桓為天水郡公。戊子,上親祭孔子廟,北面再拜。退謂侍臣曰:「朕幼年游佚,不知志學,歲月逾邁,深以為悔。孔子雖無位,其道可尊,使萬世景仰。大凡為善,不可不勉。」自是頗讀《尚書》、《論語》及《五代》、《遼史》諸書,或以夜繼焉。



 三月己未,上宴群臣于瑤池殿,適宗弼遣使奏捷,侍臣多進詩稱賀。帝覽之曰:「太平之世,當尚文物,自古致治,皆由是也。」四月丙子,以濟南尹韓昉參知政事。辛巳,宗弼請伐江南,從之。五月己酉,太師、領三省事、梁宋國王宗乾薨。庚戌,
 上親臨。日官奏,戌、亥不宜哭泣。上曰:「君臣之義,骨肉之親,豈可避之。」遂哭之慟,命輟朝七日。



 六月甲戌,詔都元帥宗弼與宰執同入奏事。庚寅,行臺平章政事耶律暉致仕。壬辰,有司請舉樂,上以宗乾新喪不允。甲午,衛王宗強薨,上親臨、輟朝如宗乾喪。七月癸卯,以景宣皇帝忌辰,命尚食徹肉。丙午,以宗弼為尚書左丞相兼侍中、都元帥、領行臺如故。己酉,宗弼還軍中。辛亥,參知政事耶律讓罷。



 九月戊申,上至自燕京。朝太皇太后于明德宮。詔賜鰥寡孤獨不能自存者,人絹二匹、絮三斤。是秋,蝗。都元帥宗弼伐宋,渡淮。以書讓宋,宋復書乞罷兵,宗
 弼以便宜畫淮為界。



 十一月己酉,高麗國賀受尊號。稽古殿火。十二月癸巳,夏國賀受尊號。天水郡公趙桓乞本品俸,詔賙濟之。左丞勖進先朝《實錄》三卷,上焚香立受之。



 二年正月乙未朔,高麗、夏遣使來賀。己亥,上獵于來流河。乙巳,命封高麗。丁未,上至自來流河。辛亥,萬壽節,高麗、夏遣使來賀。壬子,衍聖公孔璠薨,子拯襲。二月丁卯,上如天開殿。甲戌,賑熙河路,戊子,皇子濟安生。辛卯,宋使曹勛來許歲幣銀、絹二十五萬兩、匹,畫淮為界,世世子孫,永守誓言。改封蜀王劉豫為曹王。壬辰,以皇子生,
 赦中外。



 三月辛丑,還自天開殿。大雪。丙午,以宗弼為太傅。丙辰,遣左宣徽使劉筈以袞冕圭冊冊宋康王為帝。歸宋帝母韋氏及故妻邢氏、天水郡王并妻鄭氏喪于江南。戊午,立子濟安為皇太子。四月丙寅,以臣宋告中外。庚午,五雲樓、重明等殿成。五月癸巳朔,不視朝。上自去年荒於酒,與近臣飲,或繼以夜。宰相入諫,輒飲以酒,曰:「知卿等意,今既飲矣,明日當戒。」因復飲。乙卯,賜宋誓詔。辛酉,宴群臣於五雲樓,皆盡醉而罷。



 七月甲午,回鶻遣使來貢。北京、廣寧府蝗。丁酉,賜宗弼金券。八月丁卯,詔歸朱弁、張邵、洪皓于宋。辛未,復太宗子胡盧為王。賑陜西。



 九月壬辰,詔給天水郡王子、姪、婿,天水郡公子俸給。



 十一月甲寅,平章政事漆水郡王昂薨,追封鄆王。十二月乙丑,高麗王遣使謝封冊。庚午,宋遣使謝歸三喪及母韋氏。壬申,上獵于核耶呆米路。癸未,還宮。甲申,皇太子濟安薨。



 三年正月己丑朔,以皇太子喪不御正殿,群臣詣便殿稱賀。宋、高麗、夏使詣皇極殿遙賀。乙巳,萬壽節,如正旦儀。三月辛卯,以尚書左丞勖為平章政事,殿前都點檢宗憲尚書左丞。丁酉,太皇太后唐括氏崩。己酉,封子道濟為魏王。



 五月丁巳朔,京兆進瑞麥。癸亥,上致祭太
 皇太后。甲申,初立太廟、社稷。六月己酉,初置驍毅軍。七月丙寅,上致祭太皇太后。庚辰,太原路進獬豸并瑞麥。



 八月辛卯,詔給天水郡王孫及天水郡公婿俸祿。丙申,老人星見。乙巳,謚太皇太后曰欽仁皇后。戊申,葬恭陵。十二月癸未朔,日有食之。



 四年正月癸丑朔,宋、高麗、夏遣使來賀。甲寅,詔以去年宋幣賜始祖以下宗室。己未,以宋使王倫為平州轉運使,既受命,復辭,罪其反覆,誅之。乙丑,陜西進嘉禾十有二莖,莖皆七穗。己巳,萬壽節,宋、高麗、夏遣使來賀。乙亥,上祭欽仁皇后,哭盡哀。二月癸未,上如東京。丙申,次百
 泊河春水。丁酉,回鶻遣使來賀,以粘合韓奴報之。



 五月辛亥朔,次薰風殿。六月辛巳朔,日有食之。



 七月庚午,建原廟於東京。八月癸未,殺魏王道濟。九月乙酉,上如東京。壬子,畋于沙河,射虎獲之。乙卯,遣使祭遼主陵。辛酉,詔薰風殿二十里內及巡幸所過五里內,並復一歲。癸酉,行臺左丞相張孝純薨。十月壬辰,立借貸飢民酬賞格。甲辰,以河朔諸郡地震,詔復百姓一年,其壓死無人收葬者,官為斂藏之。陜西、蒲、解、汝、蔡等處因歲饑,流民典雇為奴婢者,官給絹贖為良,放還其鄉。十一月己酉,上獵于海島。十二月甲午,至東京。



 五年正月丁未朔,宋、高麗、夏遣使來賀。癸亥,萬壽節,宋、高麗、夏遣使來賀。二月乙未,次濟州春水。三月戊辰,次天開殿。五月戊午,初用御製小字。壬寅,以平章政事勖諫,上為止酒,仍布告廷臣。六月乙亥朔,日有食之。



 八月戊戌,發天開殿。九月庚申,至自東京。十月辛卯,增謚太祖。閏月戊寅,大名府進牛生麟。壬辰,懷州進嘉禾。十二月戊申,增謚始祖以下十帝及太宗、徽宗。丁巳,赦。



 六年正月辛未朔,宋、高麗、夏遣使來賀。壬申,封太祖諸孫為王。乙亥,畋于謀勒。甲申,還京師。丁亥,萬壽節,宋、高麗、夏遣使來賀。庚寅,以邊地賜夏國。壬辰,如春水。帝從
 禽,導騎誤入大澤中,帝馬陷,因步出,亦不罪導者。乙未,封偎喝為王。二月丙寅,右丞相韓企先薨。



 三月壬申,以阿離補為行臺右丞相。四月庚子朔,上至自春水。以同判大宗正事宗固為太保、右丞相兼中書令。戊午,行臺右丞相阿離補薨。五月壬申,高麗王楷薨。辛卯,以左宣徽使劉筈為行臺右丞相。



 六月乙巳,殺宇文虛中及高士談。乙丑,遣使弔祭高麗,并起復嗣王晛。九月戊辰朔,以許王破汴,睿宗平陜西,鄭王克遼及婁室、銀術可皆有大功,並為立碑。戊寅,曹王劉豫薨。是歲,遣粘割韓奴招耶律大石,被害。



 七年正月乙丑朔,宋、高麗、夏遣使來賀。辛巳,萬春節,宋、高麗、夏遣使來賀。癸未,以西京鹿囿為民田。丁亥,太白經天。三月戊寅,高麗遣使謝弔祭、起復。四月戊午,宴便殿。上醉酒,殺戶部尚書宗禮。六月丁酉,殺橫海軍節度使田、左司郎中奚毅、翰林待制邢具瞻及王植、高鳳廷、王傚、趙益興、龔夷鑒等。



 七月己巳,太白經天,曲赦畿內。九月,太保、右丞相宗固薨。以都元帥宗弼為太師、領三省事,都元帥、行臺尚書省事如故,平章政事勖為左丞相兼侍中,都點檢宗賢為右丞相兼中書令,行臺右丞相劉筈、右丞蕭仲恭為平章政事,李德固為尚書右
 丞,秘書監蕭肄為參知政事。十月壬子,平章行臺尚書省事奚寶薨。十一月癸酉,以工部侍郎僕散太彎為御史大夫。乙亥,兵部尚書秉德進三角羊。己卯,詔減常膳羊豕五之二。癸未,以尚書左丞宗憲為行臺平章政事,同判大宗正事亮為尚書左丞。十二月戊午,參知政事韓昉罷。兵部尚書秉德為參知政事。



 八年正月庚申朔,宋、高麗、夏遣使來賀。丙子,萬壽節,宋、高麗、夏遣使來賀。二月壬子,以哥魯葛波古等為橫賜高麗、夏國使。甲寅,以大理卿宗安等為高麗王晛封冊使。乙卯,上如天開殿。四月戊子朔,日有食之。辛丑,遣參
 知政事秉德等為廉察官吏。庚戌,至自天開殿。甲寅,《遼史》成。六月乙卯,平章政事蕭仲恭為行臺左丞相,左丞亮為平章政事,都點檢唐括辯為尚書左丞。高麗遣使謝封冊。



 七月乙亥,御史大夫僕散彎罷,以侍衛親軍都指揮使阿魯帶為御史大夫。戊寅,以尚書左丞唐括辯奉職不謹,杖之。八月戊戌,宗弼進《太祖實錄》,上焚香立受之。庚子,以尚書左丞相勖領行臺尚書省事,右丞相宗賢為太保、尚書左丞相。丙午,以行臺左丞相蕭仲恭為尚書右丞相。閏月庚申,宰臣以西林多鹿,請上獵。上恐害稼,不允。丙寅,太廟成。九月丙申,尚書左丞唐括
 辯罷。以左宣徽使稟為尚書左丞。十月辛酉,太師、領三省事、都元帥、越國王宗弼薨。



 十一月壬辰,太白經天。乙未,左丞相宗賢、左丞稟等言,州郡長吏當並用本國人。上曰:「四海之內,皆朕臣子,若分別待之,豈能致一。諺不云乎,『疑人勿使,使人勿疑』。自今本國及諸色人,量才通用之。」辛丑,以尚書左丞相宗賢為左副元帥,平章政事亮為尚書左丞相兼侍中,參知政事秉德為平章政事。庚戌,左副元帥宗賢復為太保,左丞相、左副元帥如故。十二月乙卯,以右丞相蕭仲恭為太傅、領三省事,左丞相亮為尚書右丞相。乙亥,以左丞相宗賢為太師、領三
 省事兼都元帥。



 九年正月甲申朔,宋、高麗、夏遣使來賀。戊戌,太師、領三省事、都元帥宗賢罷。領行臺尚書省事勖為太師、領三省事,同判大宗正事充為尚書左丞相,右丞相亮兼都元帥。庚子,萬壽節,宋、高麗、夏遣使來賀。壬寅,左丞相充薨。丙午,以右丞相亮為左丞相,判大宗正事宗本為尚書右丞相,左副元帥宗敏為都元帥,南京留守宗賢為左副元帥兼西京留守。己酉,宗賢復為太保、領三省事。



 二月甲寅,會寧牧唐括辯復為尚書左丞,尚書左丞稟為行臺平章政事。三月癸未朔,日有食之。辛丑,以司空
 宗本為尚書右丞相兼中書令,左丞相亮為太保、領三省事。



 四月壬申夜,大風雨,雷電震壞寢殿鴟尾,有火入上寢,燒幃幔,帝趨別殿避之。丁丑,有龍鬥於利州榆林河水上。大風壞民居、官舍,瓦木人畜皆飄揚十數里,死傷者數百人。五月戊子,以四月壬申、丁丑天變,肆赦。命翰林學士張鈞草詔,參知政事蕭肄擿其語以為誹謗,上怒,殺鈞。是日,曲赦上京囚。庚寅,出太保、領三省事亮領行臺尚書省事。戊申,武庫署令耶律八斤妄稱上言宿直將軍蕭榮與胙王元為黨,誅之。



 六月己未,以都元帥宗敏為太保、領三省事兼左副元帥,左丞相宗賢兼
 都元帥。八月庚申,以劉筈為司空,行臺右丞相如故。宰臣議徙遼陽、勃海之民於燕南,從之。侍從高壽星等當遷,訴於后,后以白上,上怒議者,杖平章政事秉德,殺左司郎中三合。九月丙申,以領行臺尚書省事亮復為平章政事。戊戌,以右丞相宗本為太保、領三省事,左副元帥宗敏領行臺尚書省事,平章政事秉德為尚書左丞相兼中書令,司空劉筈為平章政事。庚子,以御史大夫宗甫為參知政事。



 十月乙丑,殺北京留守胙王元及弟安武軍節度使查剌、左衛將軍特思。大赦。癸酉,以翰林學士京為御史大夫。十一月癸未,殺皇后裴滿氏。召胙
 王妃撒卯入宮。戊子,殺故鄧王子阿懶、達懶。癸巳,上獵於忽剌渾土溫。遣使殺德妃烏古論氏及夾谷氏、張氏。十二月己酉朔,上至自獵所。丙辰,殺妃裴滿氏於寢殿。而平章政事亮因群臣震恐,與所親駙馬唐括辯、寢殿小底大興國、護衛十人長忽土、阿里出虎等謀為亂。丁巳,以忽土、阿里出虎當內直,命省令史李老僧語興國。夜二鼓,興國竊符,矯詔開宮門,召辯等。亮懷刀與其妹夫特廝隨辯入至宮門,守者以辯駙馬,不疑,內之。及殿門,衛士覺,抽刃劫之,莫敢動。忽土、阿亮出虎至帝前,帝求榻上常所置佩刀,不知已為興國易置其處,忽土、阿
 里出虎遂進弒帝,亮復前手刃之,血濺滿其面與衣。帝崩,時年三十一。左丞相秉德等遂奉亮坐,羅拜呼萬歲,立以為帝。降帝為東昏王,葬于皇后裴滿氏墓中。貞元三年,改葬于大房山蓼香甸,諸王同兆域。大定初,追謚武靈皇帝,廟號閔宗,陵曰思陵。別立廟。十九年,升祔于太廟,增謚弘基纘武莊靖孝成皇帝。二十七年,改廟號熙宗。二十八年,以思陵狹小,改葬於峨眉谷,仍號思陵。詔中外。



 贊曰:熙宗之時,四方無事,敬禮宗室大臣,委以國政,其繼體守文之治,有足觀者。末年酗酒妄殺,人懷危懼。所
 謂前有讒而不見,後有賊而不知。馴致其禍,非一朝一夕故也。



\end{pinyinscope}