\article{列傳第三十六}

\begin{pinyinscope}

 ○完顏匡完顏綱完顏定奴



 完顏匡,本名撒速,始祖九世孫。事豳王允成,為其府教讀。大定十九年,章宗年十餘歲,顯宗命詹事烏林答願擇德行淳謹、才學該通者,使教章宗兄弟。閱月,愿啟顯宗曰:「豳王府教讀完顏撒速、徐王府教讀僕散訛可二人,可使教皇孫兄弟。」顯宗曰:「典教幼子,須用淳謹者。」已而召見于承華殿西便殿。顯宗問其年,對曰:「臣生之歲,
 海陵自上京遷中都,歲在壬申。」顯宗曰:「二十八歲爾,詹事乃云三十歲何也?」匡曰:「臣年止如此,詹事謂臣出入宮禁,故增其歲言之耳。」顯宗顧謂近臣曰:「篤實人也。」命擇日,使皇孫行師弟子禮。七月丁亥,宣宗、章宗皆就學,顯宗曰:「每日先教漢字,至申時漢字課畢,教女直小字,習國朝語。」因賜酒及彩幣。頃之,世宗詔匡、訛可俱充太子侍讀。



 寢殿小底駝滿九住問匡曰:「伯夷、叔齊何如人?」匡曰:「孔子稱夷、齊求仁得仁。」九住曰:「汝輩學古,惟前言是信。夷、齊輕去其親,不食周粟餓死首陽山,仁者固如是乎?」匡曰:「不然,古之賢者行其義也,行其道也。伯夷思
 成其父之志以去其國,叔齊不茍從父之志亦去其國。武王伐紂,夷、齊叩馬而諫。紂死,殷為周,夷、齊不食周粟,遂餓而死。正君臣之分,為天下後世慮至遠也,非仁人而能若是乎!」是時,世宗如春水,顯宗從,二人者馬上相語遂後。顯宗遲九住至,問曰:「何以後也?」九住以對,顯宗嘆曰:「不以女直文字譯經史,何以知此。主上立女直科舉,教以經史,乃能得其淵奧如此哉。」稱善者良久,謂九住曰:「《論語》『知之為知之,不知為不知,是知也』。汝不知不達,務辯口以難人。由是觀之,人之學、不學,豈不相遠哉。」顯宗嘗謂中侍局都監蒲察查刺曰:「入殿小底完顏訛
 出、侍讀完顏撒速,與我同族,汝知之乎?」對曰:「不知也。」顯宗曰:「撒速,始祖九世孫。訛出,保活里之世也。始祖兄弟皆非常人,汝何由知此。」



 顯宗命匡作《睿宗功德歌》,教章宗歌之,其詞曰:「我祖睿宗,厚有陰德。國祚有傳,儲嗣當立。滿朝疑懼,獨先啟策。徂征三秦,震驚來附。富平百萬,望風奔仆。靈恩光被,時雨春暘。神化周浹,春生冬藏。」蓋取宗翰與睿宗定策立熙宗,及平陜西大破張浚於富平也。二十三年三月萬春節,顯宗命章宗歌此詞侑觴,世宗愕然曰:「汝輩何因知此?」顯宗奏曰:「臣伏讀《睿宗皇帝實錄》,欲使兒子知創業之艱難,命侍讀撒速作歌教
 之。」世宗大喜,顧謂諸王侍臣曰:「朕念睿宗皇帝功德,恐子孫無由知,皇太子能追念作歌以教其子,嘉哉盛事,朕之樂豈有量哉。卿等亦當誦習,以不忘祖宗之功。」命章宗歌數四,酒行極歡,乙夜乃罷。



 二十五年,匡中禮部策論進士。是歲,世宗在上京,顯宗監國。三月甲辰,御試,前一日癸卯,讀卷官吏部侍郎李晏、棣州防禦使把內刺、國史院編修官夾谷衡、國子助教尼龐古鑑進稟,策題問「契敷五教,皋陶明五刑,是以刑措不用,比屋可封。今欲興教化,措刑罰,振紀綱,施之萬世,何術可致?」匡已試,明日入見,顯宗問對策云何,匡曰:「臣熟觀策問敷教、
 措刑兩事,不詳『振紀綱』一句,只作兩事對,策必不能中。」顯宗命匡誦所對策,終篇,曰:「是亦當中。」匡曰:「編修衡、助教鑒長於選校,必不能中。」已而匡果下第。顯宗惜之,謂侍臣曰:「我只欲問教化、刑罰兩事,乃添振紀綱一句,命刪去,李晏固執不可,今果誤人矣。」謂侍正石敦寺家奴、唐括曷答曰:「侍讀二十一年府試不中,我本不欲侍讀再試,恐傷其志,今乃下第,使人意不樂。」是歲初取止四十五人,顯宗命添五人,僕散訛可中在四十五人,後除書畫直長。匡與訛可俱為侍讀,匡被眷遇特異,顯宗謂匡曰:「汝無以訛可登第怏怏,但善教金源郡王,何官不
 可至哉。」是歲,顯宗薨,章宗判大興尹,封原王,拜右丞相,立為皇太孫。匡仍為太孫侍讀。二十八年,匡試詩賦,漏寫詩題下注字,不取,特賜及第,除中都路教授,侍讀如故。」



 章宗即位,除近侍局直長,歷本局副使、局使,提點太醫院,遷翰林直學士。使宋,上令權更名弼,以避宋祖諱,事載《本紀》。遷秘書監,仍兼太醫院、近侍局事,再兼大理少卿。遷簽書樞密院事,兼職如故。承安元年,行院于撫州。河北西路轉運使溫昉行六部事,主軍中餽餉,屈意事匡,以馬幣為獻,及私以官錢佐匡宴會費,監察御史姬端修劾之,上方委匡以邊事,遂寢其奏。三年,入奏邊
 事,居五日,還軍。尋入守尚書左丞,兼修國史,進《世宗實錄》。



 章宗立提刑司,專糾察黜陟,當時號為外臺,匡與司空襄,參政揆奏:「息民不如省官,聖朝舊無提刑司,皇統、大定間每數歲一遣使廉察,郡縣稱治。自立此官,冀達下情,今乃是非混淆,徒煩聖聽。自古無提點刑獄專薦舉之權者,若陛下不欲遽更,不宜使兼採訪廉能之任。歲遣監察體究,仍不時選使廉訪。」上從其議,於是監察體訪之使出矣。



 初,匡行院於撫州,障葛將攻邊境,會西南路通事黃摑按出使烏都碗部知其謀,奔告行院為之備,迎擊障葛,敗其兵。按出與八品職,遷四官。匡遷三
 官。匡奏乞以所遷三官讓其兄奉御賽一,上嘉其義,許之。改樞密副使,授世襲謀克。



 宋主相韓侂胄。侂胄嘗再為國使,頗知朝廷虛實。及為相,與蘇師旦倡議復仇,身執其咎,繕器械,增屯戍,初未敢公言征伐,乃使邊將小小寇鈔以嘗試朝廷。泰和五年正月,入確山界奪民馬。三月,焚平氏鎮,剽民財物,掠鄧州白亭巡檢家貲,持其印去。遂平縣獲宋人王俊,唐州獲宋諜者李忤,俊襄陽軍卒,忤建康人。俊言宋人於江州、鄂、岳屯大兵,貯甲仗,修戰艦,期以五月入寇。忤言侂胄謂大國西北用兵連年,公私困竭,可以得志,命修建康宮,勸宋主都建康節
 制諸道。河南統軍司奏請益兵為之備。詔平章政事僕散揆為河南宣撫使,籍諸道兵,括戰馬,臨洮、德順、秦、鞏各置弓手四千人。詔揆遺書宋人曰:「奈何興兵?」宋人辭曰:「盜賊也。邊臣不謹,今黜之矣。」



 宋人將啟邊釁,太常卿趙之傑、知大興府承暉、中丞孟鑄皆曰:「江南敗衄之餘,自救不暇,恐不敢敗盟。」匡曰:「彼置忠義保捷軍,取先世開寶、天禧紀元,豈忘中國者哉。」大理卿畏也曰:「宋兵攻圍城邑,動輒數千,不得為小寇。」上問參政思忠,思忠極言宋人敗盟有狀、與匡、畏也合,上以為然。及河南統軍使紇石烈子仁使宋還,奏宋主修敬有加,無他志。上問
 匡曰:「於卿何如?」匡曰:「子仁言是。」上愕然曰:「卿前議云何,今乃中變邪?」匡徐對曰:「子仁守疆圉,不妄生事,職也。《書》曰『有備無患』,在陛下宸斷耳。」於是罷河南宣撫司,僕散揆還朝。



 六年二月,宋人陷散關,取泗州、虹縣、靈璧。四月,復詔僕散揆行省事于汴,制諸軍。頃之,以匡為右副元帥。揆請匡先取光州,還軍懸瓠,與大軍合勢南下。匡奏:「僕散揆大軍渡淮,宋人聚兵襄、沔以窺唐、鄧,汴京留兵頗少,有掣肘之患,請出唐、鄧。」從之。遣前鋒都統烏古論慶壽以騎八千攻棗陽,遣左翼提控完顏江山以騎五千取光化,右翼都統烏古孫兀屯取神馬坡,皆克之。匡
 軍次白虎粒,都統完顏按帶取隨州,烏古論慶壽扼赤岸,斷襄、漢路。宋隨州將雷大尉遁去,遂克隨州。於是宋鄧城、樊城戍兵皆潰。賜詔獎諭,戒諸軍毋虜掠、焚壞城邑。匡進兵圍德安,分遣諸將徇下安陸、應城、雲夢、漢川、荊山等縣,副統蒲察攻宜城縣取之。十二月,敗宋兵二萬人於信陽之東,詔曰:「卿總師出疆屢捷,殄寇撫降,日闢土宇。彼恃漢、江以為險阻,箠馬而渡,如涉坦途,荊、楚削平,不為難事,雖天佑順,亦卿籌畫之效也。益宏遠圖,以副朕意。」匡進所獲女口百人。詔匡權尚書右丞,行省事,右副元帥如故。



 吳曦以蜀、漢內附,詔匡先取襄陽以
 屏蔽蜀、漢。完顏福海破宋援襄陽兵於白石峪,遂取穀城縣。僕散揆得疾,遂班師,至蔡,疾革,詔右丞相宗浩代之。七年二月,揆薨。匡久圍襄陽,士卒疲疫,會宗浩至汴,匡乃放軍朝京師,轉左副元帥,賜宴于天香殿,還軍許州。九月,宗浩薨,匡為平章政事,兼左副元帥,封定國公,代宗浩總諸軍,行省于汴京。



 初,僕散揆初至汴,既定河南諸盜,乃購得韓侂胄族人元靚,使行間於宋。元靚渡淮,宋督視江、淮兵馬事丘灊奏之宋主。是時,宋主、侂胄見兵屢敗以為憂,欲乞盟無以為請,得密奏,即命遣人護元靚北歸,因請議和。密使其屬劉祐送元靚申和議
 于揆,揆曰:「稱臣割地,獻首禍之臣,然後可。」宋主因密諭丘灊,使歸罪邊將以請焉。及宗浩代揆,方信孺至,宗浩以方信孺輕佻不可信,移書宋人,果欲請和當遣朱致和、吳琯、李大性、李璧來。侂胄得報大喜過望,乃召張巖于建康,罷為福建觀察使,歸罪蘇師旦,貶之嶺南。是時,李璧已為參政,不可遣。朱致知、吳琯已死,李大性知福州,道遠不能遽至。乃遣左司郎中王冉來,至濠州,匡使人責以稱臣等數事,柟以宋主、侂胄情實為請,依靖康二年正月請和故事,世為伯姪國,增歲幣為三十萬兩、匹、犒軍錢三百萬貫,蘇師旦等俟和議定當函首以獻。
 柟至汴,以侂胄書上元帥府,匡復詰之,柟懇請曰:「此事實出朝旨,非行人所專。」匡察其不妄,乃具奏。章宗詔匡移書宋人,當函侂胄首贖淮南地,改犒軍錢為銀三百萬兩。於是,宋吏部侍郎史彌遠定計殺韓柟胄,彌遠知國政,和好自此成矣。



 於是,廷議諸軍已取關隘不可與。王柟以宋參政錢象祖書來,略曰:



 竊惟昔者修好之初,蒙大金先皇帝許以畫淮為界。今大國遵先皇帝聖意,自盱眙至唐、鄧畫界仍舊,是先皇帝惠之於始,今皇帝全之于後也。然東南立國,吳、蜀相依,今川、陜關隘,大國若有之,則是撤蜀之門戶,不能保蜀,何以固吳?已增歲
 幣至三十萬,通謝為三百萬貫,以連歲師旅之餘,重以喪禍,豈易辦集。但邊隙既開和議,區區悔艾之實,不得不黽勉遵承。又蒙聖畫改輸銀三百萬兩,在本朝宜不敢固違,然傾國資財,竭民膏血,恐非大金皇帝棄過圖新,兼愛南北之意也。



 主上仁慈寬厚,謹守信誓,豈有意於用兵。止緣侂胄啟釁生事,迷國罔上,以至地斯。是以奮發英斷,大正國典,朋附之輩,誅斥靡貸。今大國欲使斬送侂胄,是未知其已死也。侂胄實本庸愚,怙權輕信,有誤國事,而致侂胄誤國者,蘇師旦也。師旦既貶,侂胄尚力庇之,囑方信孺妄言已死,近推究其事,師旦已行
 斬首。儻大國終惠川、陜關隘,所畫銀兩悉力祗備,師旦首函亦當傳送,以謝大國。



 本朝與大國通好以來,譬如一家叔姪,本自協和,不幸奴婢交鬥其間,遂成嫌間。一旦猶子翻然改悟,斥逐奴隸,引咎謝過,則前日之嫌便可銷釋,奚必較錙銖毫末,反傷骨肉之恩乎?惟吳、蜀相為首尾,關隘繫蜀安危,望敢備奏,始終主盟,使南北遂息肩之期,四方無兵革之患,不勝通國至願。



 是時,陜西宣撫司請增新得關隘戍兵萬人。王柟狀稟,如蒙歸川、陜關隘,韓侂胄首必當函送,遵上國之命。匡奏曰:「關隘之事,臣初亦惑之,今當增戍萬人,壁壘之役,餽餫之勞,
 費用必廣。祖宗所以取者,以關隘僅能自保耳,非有益於戰也。設能入寇,縱之平地,以鐵騎蹂之,無一得脫。彼哀祈不已者,以前日負固尚且摧覆,今遂失之,是無一日之安也。必謂兵力得之不可還賜,則漢上諸郡皆膏腴耕桑之地,棗陽、光化歸順之民數萬戶,較之陜右輕重可知,獨在陛下決之耳。」詔報曰:「侂胄渠魁,既請函首,宋之悔服,可謂誠矣。」匡乃遣王柟還,復書曰:「宋國負渝盟之罪,自陳悔艾,主上德度如天,不忍終絕,優示訓諭,許以更成,所以覆護鎮撫之恩,至深至厚。昨奉聖訓,如能斬送韓侂胄,徐議還淮南地。來書言韓侂胄已死,
 將以蘇師旦首易之,飾辭相紿如此。至于犒軍銀兩欲俟歸關隘然後祗備,是皆有咈聖訓。及王柟狀稟,如蒙歸還川、陜關隘,其韓侂胄首必當函送。聖訓令斬送侂胄首者,本欲易淮南地,陜西關隘不預焉。王柟所陳亦非元畫事理,不敢專決,具奏。奉旨『朕以生靈為念,已貰宋罪,關隘區區豈足深較,既能函送韓侂胄首,陜西關隘可以還賜』。今恩訓如此,其體大國寬仁矜恤曲從之意,追修誓書,齎遣通謝人使赴闕。」



 王柟之歸也,匡要以先送叛亡驅掠,然後割賜淮南、川、陜,及彼誓書草本有犯廟諱字及文義有不如體製者,諭令改之。宋人以叛亡
 驅掠散在州縣,一旦拘刷,未易聚集。今已四月,農事已晚,邊民連歲流離失所,扶攜道路,即望復業,過此農時,遂失一歲之望。歲幣犒軍物多,非旬月可辦。錢象祖復以書來,略曰:「竊見大金皇帝前日聖旨,如能斬送韓侂胄首,沿淮之地並依皇統、大定已畫為定。又睹今來聖旨,既能送侂胄首,陜西關隘可併還賜。以此仰見聖慈寬大,初無必待發遣驅掠官兵,然後退兵交界之語。誓書草本添改處,先次錄本齎呈,并將侂胄首函送,及管押納合、道僧、李全家口一併發還。欲望上體大金皇帝畫定聖旨,先賜行下沿邊及陜西所屬,候侂胄首至界
 上,即便抽回軍馬,歸還淮南及川、陜關隘地界。所有驅掠官兵,留之何益,見已從實刷勘,發還,其使人禮物歲幣等已起發至真、揚間,伺候嘉報,迤邐前去界首,以俟取接。」



 匡得錢象祖書,即具奏,詔報曰:「朕以生靈之故,已從所請,稱臣割地,尚且闊略,區區小節,何足深較。其侂胄、師旦首函及諸叛亡至濠州,即聽通謝人使入界,軍馬即當徹還,川、陜關隘,俟歲幣犒軍銀綱至下蔡,畫日割賜。」匡得詔書,即以諭宋人,使如詔書從事。



 泰和八年閏四月乙未,宋獻韓侂胄、蘇師旦首函至元帥府,匡遣平南撫軍上將軍紇石烈貞以,侂胄、師旦首函露布以聞。五
 月丁未,遣戶部尚書高汝礪、禮部尚書張行簡奏告天地,武衛軍都指揮使徒單鏞奏告太廟,御史中丞孟鑄告社稷。是日,上御應天門,立黃麾仗,受宋馘。尚書省奏露布,親王百官起居上表稱賀。獻馘廟社,以露布頒中外。竿侂胄、師旦首并二人畫像于通衢,百姓縱觀,然後漆其首,藏之軍器庫。丙辰,匡朝京師,進官兩階,賜玉帶、金一百兩、銀一千五百兩,重幣三十端。罷元帥府仍為樞密院。六月癸酉,宋通謝使許弈、吳衡等入見。癸未,以宋人請和詔天下。



 十一月丙辰,章宗崩,匡受遺詔,立衛紹王。其遺詔略曰:「皇叔衛王,承世宗之遺體,鐘厚慶於
 元妃,人望所歸,歷數斯在。今朕上體太祖皇帝傳授至公之意,付畀寶祚,即皇帝位於柩前。載惟禮經有嫡立嫡、無嫡立庶,今朕之內人見有娠者兩位,已詔皇帝,如其中有男當立為儲貳,如皆是男子,擇可立者立之。」丁巳,衛紹王即位。戊午,章宗內人范氏胎氣有損。大安元年四月,平章政事僕散端、左丞孫即康奏:「承御賈氏產期已出三月,有人告元妃李氏令賈氏詐稱有身。」詔元妃李氏、承御賈氏皆賜死。初,章宗大漸,匡與元妃俱受遺詔立衛王,匡欲專定策功,遂構殺李氏。數日,匡拜尚書令,封申王。大安元年十二月,薨。



 匡事顯宗,深被恩遇。
 自章宗幼年,侍講讀最親幸,致位將相,怙寵自用,官以賄成。承安中,撥賜家口地土,匡乃自占濟南、真定、代州上腴田,百姓舊業輒奪之,及限外自取。上聞其事,不以為罪,惟用安州邊吳泊舊放圍埸地、奉聖州在官閑田易之,以向自占者悉還百姓。宣宗嘗謂侍臣曰:「撒速往年嘗受人玉吐鶻,然後與之官,此豈宰相所為哉?」



 完顏綱,本名元奴,字正甫。明昌中,為奉御,累官左拾遺。詔三叉口置捺缽。綱上疏諫,疏中有云:「賊出沒其間」,詔尚書省詰問,所言不實,章宗以綱諫官,不之罪。遷刑部員外郎,綱言:「諸犯死罪除名移推相去二百里,并犯徒
 罪連逮二十人以上者並令就問,曾經所屬按察司審讞者移推別路,官亦依上就問。凡告移推之人皆已經本路按察審訖,即當移推別路。按察司部分廣闊,如上京路移推臨潢路,最近亦往復二三千里,北京留守司移推西北路招討司,最近亦須數月。乞依舊制,令移推官司追取其人歸問。」從之。



 故事,使夏國者夏人饋贈禮物,視書幾道以為多寡。泰和元年,綱為賜夏主生日使,章宗命齎三詔,左司員外郎孫椿年奏詔為一道,尋自陳首,上責宰臣曰:「椿年忽略,卿等奈何不奏也。」轉工部郎中,上言:「太府監官兼尚食局官,乞於少府監依此例,
 注能幹官一員兼儀鸞局官,儀鸞局官一員兼少府監官,相須檢治。」從之。四年,詔綱與喬宇、宋元吉編類陳言文字,綱等奏:「凡闕涉宮庭及大臣者摘進,其餘以省臺六部各為一類。」凡二十卷。遷同簽宣徽院事。



 六年,與宋連兵,陜西諸將頗相異同,以綱為蜀漢路安撫使、都大提舉兵馬事,與元帥府參決西事,調羌兵之未附者。於是,知鳳翔府事完顏昱、同知平涼府事蒲察秉鉉分駐鳳翔諸隘,通遠軍節度使承裕、秦州防禦使完顏璘屯成紀界,知臨洮府事石抹仲溫駐臨洮,同知臨洮府事術虎高琪、彰化軍節度副使把回海備鞏州諸鎮,乾州
 刺史完顏思忠扼六盤,陜西路都統副使斡勒牙刺、京兆府推官蒲察秉彞戍虢華、扼潼關蒲津,陜西都統完顏忠本名裊懶、同知京兆府事烏古論兗州守京兆要害,以鳳翔、臨洮路蕃漢弓箭手及緋翮翅軍散據邊陲。緋翮翅,軍名也。元帥右臨軍充右都監蒲察貞分總其事。



 宋吳曦以兵六千攻鹽川,鞏州戍將完顏王善、隊校僕散六斤、猛安龍延常擊走之,斬首二百級。七月,吳曦兵五萬由保坌、姑蘇等路寇秦州,承裕、璘以騎千餘擊之,曦兵大敗,追奔四十里。曦別兵萬人入來遠鎮,術虎高琪破之。



 青宜可者,吐蕃之種也。宋取河湟,夏取河西
 四郡,部落散處西鄙,其魯黎族帥曰冷京,據古疊州,有四十三族、十四城、三十餘萬戶,東鄰宕昌,北接臨洮、積石,南行十日至筍竹大山,蓋蠻境也。西行四十日至河外,俗不論道里,而以日計之云。冷京卒,子耳骨延嗣,宋不能制,縻以官爵。傳六世至青宜可,尤勁勇得眾,以宋政令不常,有改事中國之意。曹佛留為洮州刺史。佛留材武有智策,能結諸羌。青宜可畏慕佛留,以父呼之,請舉國內附。朝廷以宋有盟不許,厚賜金帛以撫之。明昌間,屬羌已彪殺郡佐反,是時綱為奉御,奉詔與曹佛留計事,因召青宜可會兵擊破已彪。曹佛留遷同知臨洮
 尹,兼洮州刺史。子普賢為洮州管內巡檢使。綱屢以事至洮,佛留每謂綱言青宜可願內屬,出其至情,綱輒奏之,上終不納。及綱部署陜西,上密敕經略西事。於是,曹佛留已死,普賢為懷羌巡檢使。綱至洮,馳召普賢攝同知洮州事。普賢傳箭入羌中,青宜可大喜,率諸部長,籍其境土人民,詣綱請內屬。綱奏其事,上以青宜可為疊州副都總管,加廣威將軍。詔青宜可曰:「卿統有部人,世為雄長,嚮風慕義,背偽歸朝,願效純誠,恒輸忠力,緬懷嘉矚,式厚褒旌。覽卿進上所受偽牌,朝廷之馭諸蕃固無此例,欲使卿有以鎮撫部族、增重觀望,是以特加改
 命,賜金牌一、銀牌二,到可祗承,服我新恩,永為籓衛。」曹普賢真授同知洮州事,綱遷拱衛直都指揮使,遷三階,安撫,都大提舉如故。以商州刺史烏古論兗州領、曹普賢押領、青宜可勾當。詔曰:「完顏綱,初行時汝未知朝廷有青宜可之事,獨言可以招撫,必獲其用,既而果來效順。今汝勿以青宜可兵勢重大,卑屈失體,亦勿以蕃部而藐視之。」



 九月,詔安慰陜西,略曰:「京兆、鳳翔、臨洮三路,應被宋兵逼脅,背國從偽,或沒落外境,若能自歸者,官吏依舊勾當,百姓各令復業,元拋地土依數給付。及受宋人旗榜結構等,或值驚擾因而避役逃亡,未發覺者,
 許令所在官司陳首,並行釋免,更不追究,軍前可用之人隨宜任使。限外不首,復罪如初。



 宋程松遣別將曲昌世襲方山原,自率兵數萬分道襲和尚原、西山寨、龍門等關。是日,大霧四塞,既又暴雨,和尚原、西山寨,龍門關戍兵不知宋兵來,松遂據之。蒲察貞遣行軍副統裴滿阿里、同知隴州事完顏孛論以兵千人伏方山原下,萬戶奧屯撒合門、美原縣令術虎合沓別將壯士五百,取間道潛登,出宋兵上,自高而下,宋兵大駭,伏兵合擊,遂破之。貞乃分遣術虎合沓、部將完顏出軍奴率兵千人出黃兒谷取和尚原,同知會州事女奚列南家、押軍猛
 安粘割撒改率兵千人出大寧谷取西山寨,貞自以兵七百由中路取龍門等關。程松已焚閣道,貞且修道且進兵。至小關,松將楊廷據險注射,貞不得前,令行軍副統裴滿阿里為疑兵,潛遣猛安胡信率甲士五十人繞出其後,反擊之,宋兵大亂,遂斬廷于陣。宋兵走二里關,復敗。宋將彭統領宋兵走龍門,追擊,大破之。合沓乘夜潛登和尚原絕頂,宋人驚以為神,皆散走,破其眾二千,生獲數十人。南家斬木開道以登西山,再與宋兵遇,皆敗之,遂盡復故地。



 宋吳曦將馮興、楊雄、李珪以步騎八千人入赤谷,將寇秦州。承裕、完顏璘、河州防禦使蒲察
 秉鉉逆擊,破之。宋步兵趨西山,騎兵走赤谷。承裕分兵躡宋步兵,宋步兵據山搏戰,部將唐括按答海率二百騎馳擊之,甲士蒙葛挺身先入其陣,眾乘之,宋步兵大潰,殺數百人,追者至皂郊城,斬首二千級。猛安把添奴追宋騎兵,殺千餘人,馮興僅以身免,楊雄、李珪皆為金軍所殺。十月,綱以蕃、漢步騎一萬出臨潭,充以關中兵一萬出陳倉,蒲察貞以岐、隴兵一萬出成紀,石抹仲溫以隴右步騎五千出鹽川,完顏璘以本部兵五千出來遠。



 初,吳玠、吳璘俱為宋大將,兄弟父子相繼守西土,得梁、益間士眾心。璘孫曦太尉、昭武軍節度使、成都潼川
 府夔利等州路宣撫副使,泰和六年出兵興元,有窺關、隴之志,誘募邊民為盜,遣諜以利餌鳳翔卒溫昌,結三虞候軍為內應。昌詣府上變。曦遣諸將出秦、隴間,與綱等諸軍相拒。上聞韓侂胄忌曦威名,可以間誘致之,梁、益居宋上游,可以得志于宋,封曦蜀國王,鑄印賜詔,詔綱經略之。其賜曦詔曰:



 宋自佶、桓失守,構竄江表,僭稱位號,偷生吳會,時則乃祖武安公玠捍禦兩川。洎武順王璘嗣有大勛,固宜世胙大帥,遂荒西土,長為籓輔,誓以河山,後裔縱有欒黶之汰,猶當十世宥之。然威略震主者身危,功蓋天下者不賞,自古如此,非止于今。



 卿家
 專制蜀漢,積有歲年,猜嫌既萌,進退維谷,代之而不受,召之而不赴,君臣之義,已同路人,譬之破桐之葉不可以復合,騎虎之勢不可以中下矣。此事流傳,稔於朕聽,每一思之,未嘗不當饋歎息,而卿猶偃然自安。且卿自視翼贊之功孰與岳飛?飛之威名戰功暴于南北,一旦見忌,遂被參夷之誅,可不畏哉。故智者順時而動,明者因機而發,與其負高世之勛見疑于人,惴惴然常懼不得保其首領,曷若順時因機,轉禍為福,建萬世不朽之業哉!



 今趙擴昏孱,受制強臣,比年以來,頓違誓約,增屯軍馬,招納叛亡。朕以生靈之故,未欲遽行討伐,姑遣有
 司移文,復因來使宣諭,而乃不顧道理,愈肆憑陵,虔劉我邊陲,攻剽我城邑。是以忠臣扼腕,義士痛心,家與為仇,人百其勇,失道至此,雖欲不亡得乎?朕已分命虎臣,臨江問罪,長驅並騖,飛渡有期,此正豪傑分功之秋也。



 卿以英偉之姿,處危疑之地,必能深識天命,洞見事機。若按兵閉境,不為異同,使我師併力巢穴,而無西顧之虞,則全蜀之地,卿所素有,當加封冊,一依皇統冊構故事。更能順流東下,助為掎角,則旌麾所指,盡以相付。天日在上,朕不食言。今送金寶一鈕,至可領也。



 綱次臨江被詔,進至水洛,訪得曦族人端,署為水洛城巡檢使,遣
 持詔間行諭曦。曦得詔意動,程松尚在興元,未敢發,詐稱杖殺端,以蔽匿其事。松兵既敗,曦乃遣掌管機宜文字姚圓與端奉表送款。綱遣前京北府錄事張仔會吳曦于興州之置口,曦言歸心朝廷無他,張仔請以告身為報,曦盡出以付之,仍獻階州。



 朝廷以曦初附,恃中國為援,欲先取襄陽以為蜀漢屏蔽,乃詔右副元帥匡先攻襄陽,詔略曰:「陜西一面雖下四州,吳曦之降朕所經略。自大軍出境,惟卿所部力戰為多,方之前人無所愧謝。今南伐之事責成卿等,區區俘獲不足羨慕,果能為國建功,豈止一身榮寵,後世子孫,亦保富貴。」匡得詔,乃移
 兵趨襄陽。十二月,曦遣果州團練使郭澄、仙人關使任辛奉表及蜀地圖志、吳氏譜牒來上。



 七年正月,召綱赴京師,以為陜西宣撫副使,進三階。還軍,吳曦遣郭澄進謝恩表、誓表、賀全蜀歸附三表,親王百官稱賀,朝廷以詔答之,并賜誓詔。郭澄朝辭,諭澄曰:「汝主效順,以全蜀歸附,朕甚嘉之。然立國日淺,恐宋兵侵軼,人心不安,凡有當行事務已委宣撫完顏綱移文計議。或有緊急,即差人就去講究。大定間,汝主嘗以事入覲,今亦多歲,朕嘉汝主之義,懷想不忘,欲得其繪像,如見其面。今已遣使封冊,俟回日附進。可以此意歸諭汝主。」詔以同知臨
 洮府事術虎高琪為封冊使,翰林直學士喬宇副之。詔高琪曰:「卿以邊面宣力,加之讀書,蜀人識卿威名,勿以財賄動心,失大國體。檢制隨去奉職,勿有違枉生事。」



 頃之,宋安丙殺吳曦。上聞曦死,遣使責綱,詔曰:「曦之降,自當進據仙人關,以制蜀命,且為曦重。既不據關,復撤兵,使丙無所憚,是宜有今日也。」於是,詔贈曦太師,命德順州刺史完顏思忠招魂葬於水洛縣。以曦族兄端之子為曦後。詔諭陜西軍士,略曰:「汝等爰自去冬,出疆用命,擐披甲胄,冒涉艱險,直取山外數州,比之他軍實有勤效。界外屯駐日久,負勞苦,恩賞未行,有司申奏不明,以
 致如此。朕已令增給賞物,以酬爾勞。惟是餘賊未殄,猶須經略。眷我師徒,久役未解,深懷憫念,寤寐弗忘。汝等益思體國之忠,奮敵愾之勇,協心畢力,建立功勳;高爵厚祿,朕所不吝。」



 宋人復陷階州、西和州,綱至鳳翔,詔徹五州之兵退保要害,五州之民願徙內地者厚撫集之。以近侍局直長為四川安慰使。蒲察貞撤黃牛戍,宋安丙乘之,連兵來襲,遂陷散關,鞏州鈐轄兀顏阿失死之。詔奪綱官一階,降兵部侍郎,權宣撫副使。遣戶部侍郎尼龐古懷忠按治綱以下將吏。懷忠未至陜西,綱、貞遣兵潛自昆谷西山養馬澗入,四面攻之,復取散關,斬宋
 將張統領、于團練。綱遣使奏捷,詔書獎諭,貞等釋不問。



 八年,宋獻韓侂胄、蘇師旦首,詔以陜西關隘還之,宋罷兵。綱還京師。是歲,章宗崩,衛紹王即位,除陜西路按察使,累官尚書左丞。至寧元年,綱行省事于縉山,徒單鎰使人謂綱曰:「高琪駐兵縉山甚得人心,士皆思奮,與其行省親往,不若益兵為便。」綱不聽。徒單鎰復使人止之曰:「高琪措畫已定,彼之功即行省之功。」綱不從。綱至縉山,遂大敗。



 胡沙虎斬關入中都,遷衛紹王於衛邸,命綱子安和作家書,使親信人召綱。綱至,囚之憫忠寺,明日,押至市中,使張霖卿數以失四川、敗縉山之事,殺之。



 貞
 祐四年,綱子權復州刺史安和上書訟父冤,略曰:「先臣綱在章宗時,招懷西羌青宜可等十八部族,取宋五州,吳曦以全蜀歸朝。胡沙虎無故見殺,奪其官爵。」詔下尚書省議:「謹按元年詔書云,胡沙虎屢害良將,正謂綱輩也。」乃追復尚書左丞。弟定奴。



 定奴與兄綱俱知名,充護衛,除平涼府判官,累官同知真定府。從平章政事僕散揆伐宋,加平南虎威將軍。兵罷,遷河南東路副統軍,三遷武勝軍節度使,入為右副點檢。大安二年,遷元帥右都監,救西京,改震武軍節度使。元帥奧屯襄敗績,定奴坐失期及不以軍敗實奏,降
 河州防禦使。遷鎮西軍節度使、河東北路按察轉運使。宣宗即位,改知歸德府。貞祐二年,改知河南府,兼河南副統軍。尋遷河南統軍使,兼昌武軍節度使。請內外五品以上舉能幹之士充河北州縣官。改簽樞密院事、殿前都點檢、兼侍衛親軍都指揮使。復為簽樞密院事、行院事兼知歸德府事,改兼武寧軍節度使,行院于徐州。召為刑部尚書、參知政事。興定三年,薨。



 贊曰:章宗伐宋之役,三易主帥,兵家所忌也。宋不知乘此以為功,猶曰有人焉?韓侂胄心彊智疏,蘇師旦謀淺任大,函首燕、薊,南北皆曰賊臣,何哉?完顏匡、完顏綱皆
 泰和終功之臣,然匡隳忠於大安,綱罔難於至寧,富貴之惑人,乃如此邪?



\end{pinyinscope}