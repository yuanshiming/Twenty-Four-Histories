\article{本紀第十三}

\begin{pinyinscope}

 衛紹王



 衛紹王諱永濟,小字興勝,更諱允濟,章宗時避顯宗諱,詔改「允」為「永」。世宗第七子,母曰元妃李氏。衛王長身,美髯須,天資儉約,不好華飾,大定十一年,封薛王。是歲,進封禭王。十七年,授世襲猛安。二十五年,加開府義同三司。二十六年,為秘書監。明年,轉刑部尚書。又明年,改殿前都點檢。二十九年,世宗崩,章宗即位,進封潞王。起復,
 判安武軍節度使。五月,至冀州,以到任表謝,賜詔優答。明昌二年,進封韓王。四年,改判興平軍。五年,改沁南軍。承安二年,改封衛王。三年,改昭義軍。泰和元年,改判彰德府事。五年,改判平陽府。初,章宗誅鄭王永蹈、趙王永中,久,頗悔之。七年,下詔追復舊封,仍賜謚。而永蹈無後,乃以衛王子按陳為鄭王後,賜衛王詔曰:「朕念鄭王自棄天常,以干國憲,槁瘞曠野,忽諸不祀。歷歲既久,深用愴然。親親之情。有懷難置。已詔追復舊爵,改葬如儀。稽考古禮,以卿之子按陳為鄭王後,謹其祭祀,卿其悉之。」已而改武定軍節度使。八年十一月,自武定軍入朝。是
 時,章宗已感嗽疾,衛王且辭行,而章宗意留之。宗章初年,雅愛諸王,置王傳府尉官以傳導德義。及永中、永蹈之誅,由是疏忌宗室,遂以王傅府尉檢制王家,苛部嚴密,門戶出入皆有籍。而衛王乃永蹈母弟,柔弱鮮智能,故章宗愛之。既無繼嗣,而諸叔兄弟多在,章宗皆不肯立,惟欲立衛王,故於辭行留之。無何,章宗大漸,元妃李氏、黃門李新喜、平章政事完顏匡定策。章宗崩,匡等傳遣詔,立衛王。衛王固讓,乃承詔舉哀,即皇帝位于柩前。明日,群臣朝見于大安殿。詔路府州縣為大行皇帝服七日。



 大安元年正月辛丑,飛星如火。起天市垣,有尾,跡若赤龍。壬戌,改元,大赦。立元妃徒單氏為皇后。



 二月乙丑朔,太白晝見,經天。壬辰,章宗內人范氏損其遣腹,以詔內外。初,章宗遣詔:「內人有娠者兩人,生男則立為儲貳。」至是平章政事僕散端等奏:「承御賈氏當以十一月免乳,今則已出三月。范氏產期合在正月,醫稱胎氣有損,用藥調治,脈息雖和,胎形已失。范氏願削髮為尼。」封皇子六人為王。



 三月甲辰,道陵禮成,大赦。詔曰:「自今於朕名不連續,及昶、詠等字,不須別改。」以平章政事僕散端為右丞相。



 四月庚辰,殺章宗元妃李氏及承御賈氏。以平
 章政事完顏匡為尚書令。



 五月,高麗賀即位。試宏詞科。



 七月,幸海王莊,臨奠魯國公主。



 八月,萬秋節,宋遣使來賀。



 九月,如大房山,謁奠睿陵、裕陵、道陵。百官表請建儲,不允。



 十月,歲星犯左執法。己卯,詔戒勵風俗。



 十一月,平陽地震,有聲如雷,自西北來。



 十二月,詔平陽地震,人戶三人死者免租稅一年,二人及傷者免一年,貧民死者給葬錢五千,傷者三千。尚書令申王完顏匡薨。右丞相僕散端為左丞相,進封兄郢王永功為譙王,御史大夫張行簡為太保。



 二年正月庚戌朔,日中有流星出,大如盆,其色碧,向西
 行,漸如車輪,尾長數丈,沒於濁中,至地復起,光散如火。



 二月,客星入紫征垣,光散為赤龍。地大震,有聲如雷。以禮部侍郎耿端義為參知政事。



 四月,校《大金儀禮》。北方有黑氣,如大道,東西亙天。徐、邳州河清五百餘里,以告宗廟社稷。



 五月,詔儒臣編《續資治通鑑》。



 六月,大旱。下詔罪已,振貧民闕食者。曲赦西京、太原兩路雜犯,死罪減一等,徒以下免。丙寅,地震。



 七月,地震。



 八月,地震。乙丑,立子胙王從恪為皇太子。萬秋節,宋遣使來賀。獵于近郊。夏人侵葭州。



 九月,地大震。乙未,詔求直言,招勇敢,撫流亡。庚子,遣使慰撫宣德行省軍士。丙午,京師戒嚴。上日
 出巡撫,百官請視朝,不允。辛亥,宣德行省罷。癸丑,詔撫諭中都、西京、清、滄被兵民戶。



 十一月,獵于近郊。中都大悲閣東渠內火自出,逾旬乃滅。閣南殺竿下石罅中火自出,人近之即滅,俄復出,如是者復旬日。中都火焮民居。



 十二月乙卯朔,日有食之。是歲大饑。禁百姓不得傳說邊事。



 三年正月乙酉朔,宋、高麗、夏遣使來賀。熒惑入氐中。



 二月,熒惑犯房宿。有大風從北來,發屋折木,通玄門重關折,東華門重關折。



 閏月,熒惑犯鍵閉星。



 三月,大悲閣災,延及民居。有黑氣起北方,廣長若大堤,內有三白氣貫
 之,如龍虎狀。括民間馬,令職官出馬有差。



 四月,我大元太祖法天啟運驛武皇帝來征。遣西北路招討使粘合合打乞和。平章政事獨吉千家奴,參知政事胡沙行省事備邊。西京留守紇石烈胡沙虎行樞密院事。參知事奧屯忠孝為尚書右丞。戶部尚書梁堂為參知政事。六月壬寅,更定軍前賞罰格。



 八月,詔獎諭行省官,慰撫軍士。千家奴、胡沙自撫州退軍,駐于宣平。河南大名路軍逃歸,下詔招撫之。



 九月,千家奴、胡沙敗績于會河堡,居庸關失守。禁男子不得輒出中都城門。大元前軍至中至都,中都戒嚴。參知政事梁堂鎮撫京城。



 十月,每夜初更
 正,東、西北天明如月初出,經月乃滅。熒惑犯壘壁陣。上京留守徒單鎰遣同知烏古孫兀屯將兵二萬衛中都。泰州刺史術虎高琪屯通玄門外。上巡撫諸軍。罷宣德行省。



 十一月,殺河南陳言人郝贊。以上京留守徒單鎰為右丞相。簽中都在城軍。紇石烈胡沙虎棄西京,走還京師,即以為右副元帥,權尚書左丞。是時,德興府、弘州、昌平、懷來、縉山、豐潤、密雲、撫寧、集寧,東過平、灤,南至清、滄,由臨潢過遼河,西南至忻、代,皆歸大元。初,徒單鎰請徒桓、昌、撫百姓入內地。上信梁堂議,以責鎰曰:「是自蹙境土也。」及大元已定三州,上悔之。至是,鎰復請置行省
 事于東京,備不虞。上不悅曰:「無故遣大臣,動搖人心。」未幾,東京不守,上乃大悔。右副元帥胡沙虎請兵二萬屯宣德,詔與三千人屯媯川。平章政事千家奴、參知政事胡沙坐覆全軍,千家奴除名,胡沙責授咸平路兵馬總管。萬戶頭屯古北口。



 十二月,簽陜西兩路漢軍五千人赴中都。太保張行簡、左丞相僕散端宿禁中議軍事。左丞相僕散端罷。



 崇慶元年正月己酉朔,改元,赦。宋、夏遣使來賀。右副元帥胡沙虎請退軍屯南口,詔數其罪,免之。三月,大旱,遺使冊李遵頊為夏國王。以御史大夫福興為參知政事。參知
 政事孟鑄為御史大夫。夏人犯葭州,延安路兵馬總管完顏奴婢稟之。五月,簽陜西勇敢軍二萬人,射糧軍一萬人,赴中都。括陜西馬。安武軍度使致仕賈鉉起復參知政事。參知政事福典為尚書左丞。詔賣空名敕牒。河東、陜西大饑,斗米錢數千,流莩滿野。以南京留守僕散端為河南、陜西安撫使,提控軍馬。



 七月,有風自東來,吹帛一段,高數十丈,飛動如龍形,墜於拱辰門。



 八月,萬秋節,以兵事不設宴。



 十月,曲赦西京、遼東、北京。



 十一月,賑河東南路、南京路、陜西東路、山東西路、衛州旱災。



 十二月,夏國王李遵頊謝封冊。



 至寧元年正月,賑河東陜西饑。



 二月,詔撫諭遼東。知大名府事烏古論誼謀不軌,伏誅。



 三月,太陰、太白與日並見,相去尺餘。



 五月,改元。詔諭咸平路契丹部人之嘯聚者。起胡沙虎復為右副元帥,領武衛軍三千人屯通玄門外。陜西大旱。



 六月,夏人犯保安州,殺刺史,犯慶陽府,殺同知府事。以戶部尚書胥鼎、刑部尚書王維翰為參知政事。



 八月,尚書左丞完顏元奴將兵備邊。詔軍官、軍士賜賚有差。大霧,晝晦。治中福海別將兵屯城北。辛卯,胡沙虎矯詔以誅反者,招福海執而殺之,奪其兵。壬辰,自通玄門入,殺知大興府徒單南平、刑部會郎徒單沒拈
 於廣陽門西。福海男符寶鄯陽、都統石古乃率眾拒戰,死之。胡沙虎叩東華門,遣人呼守直親軍百戶冬兒、五十戶蒲察六斤,不應。許以世襲猛安三品官職,亦不應。都點檢徒單渭河縋而出,護衛斜烈掊鎖啟門,胡沙虎以兵入宮,盡遂衛士,代以其黨,自稱監國都元帥。癸巳,逼上出宮。以素車載至故邸,以武衛軍二百人錮守之。尚宮左夫人鄭氏為內職,掌寶璽,聞難,端居璽所待變。胡沙虎遣黃門入收璽,鄭曰:「璽,天子所用,胡沙虎人臣,取將何為?」黃門曰:「今天時大變,主上猶且不保,況璽乎?御侍當思自脫計。」鄭厲聲罵曰:「若輩宮中近侍,恩遇尤
 隆,君難不以死報之,反為逆豎奪璽耶!我死可必,璽必不與。」遂瞑目不語。黃門出,胡沙虎卒取「宣命之寶」,偽除其黨醜奴為德州防禦使、烏古論奪剌順天軍節度使、提控宿直將軍徒單金壽永定軍節度使,及其餘黨凡數十人,皆遷宮。遂使宦者李思中害上於邸。誘奉御和尚作書急召其父左丞元奴議事,元奴以軍來,並其子皆殺之。



 九月甲辰,宣宗即位。丁未,詣邸臨奠,伏哭盡哀。敕以禮改葬。胡沙虎請廢為庶人,詔百官議于朝堂,議者三百餘人。太子少傅奧屯忠孝、侍讀學士蒲察思忠請從廢黜,戶部尚書武都、拾遣田庭芳等三十人請
 降為王侯。太子太保張行簡請用漢昌邑王、晉海西公故事,侍郎史完顏訛出等十人請降復王封。胡沙虎固執前議,宣宗不得已,乃降封東海郡侯。昭雪道陵元妃李氏、承御賈氏。



 十月辛亥,元帥右監軍術虎高琪殺胡沙虎于其第。胡沙虎者,紇石烈執中也。宣宗乃下詔削其官爵。贈石古乃順州刺史,鄯陽順天軍節度副使,凡從二人拒戰者,千戶賞錢五百貫,謀克三百貫,蒲輦散軍二百貫,各遷官兩階,戰沒者贈賞付其家。冬兒加龍虎衛上將軍,再遷宿進將軍。蒲察六斤加定遠大將軍、武衛軍鈐轄。石古乃子尚幼,給俸八貫石,敕有司,俟其年
 十五以聞。貞祐四年,詔追復衛王謚曰紹。



 贊曰:衛紹王政亂於內,兵敗於外,其滅亡已有徵矣。身弒國蹙,記注亡失,南遷後不復紀載。皇朝中統三年,翰林學士承旨王鶚有志論著,求大安、崇慶事不可得,采摭當時詔令,故金部令史竇祥年八十九,耳目聰明,能記憶舊事,從之得二十餘條。司天提點張正之寫災異十六條,張承旨家手本載舊事五條,金禮部尚書楊雲翼日錄四十條,陳老日錄三十條,藏在史館。條件雖多,重復者三之二。惟所載李妃、完顏匡定策,獨吉千家奴兵敗,紇石烈執中作難,及日食、星變、地震、氛昆,不相背
 盭。今校其重出,刪其繁雜。《章宗實錄》詳其前事,《宜宗實錄》詳其後事。又於金掌奏目女官大明居士王氏所紀,得資明夫人援璽一事,附著於篇,亦可以存其梗概云爾。



\end{pinyinscope}