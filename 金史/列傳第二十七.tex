\article{列傳第二十七}

\begin{pinyinscope}

 ○蘇保衡翟永固魏子平孟浩田玨附梁肅移剌綎移剌子敬



 蘇保衡,字宗尹,雲中天成人。父京,遼進士,為西京留守。宗翰兵至西京,京出降。久之,京病篤,以保衡屬宗翰。京死,宗翰薦之於朝。賜進士出身,補太子洗馬,調解州軍事判官。左監軍撒離喝駐軍陜西,辟幕府,參議軍事,累
 官同知興中尹。天德間,繕治中都,張浩舉保衡分督工役。改大興少尹,督諸陵工役。再遷工部尚書。海陵治兵伐宋,與徐文等造舟於通州,海陵獵近郊,因至通州視工作。兵興,保衡為浙東道水軍都統制,率舟師泛海,徑趨臨安。宋兵來襲,敗于海中,副統制鄭家死之。



 大定二年,召赴中都。是時,山東盜賊嘯聚,契丹攻掠臨潢等州郡,百姓困弊。詔保衡安撫山東,前太子少保高思廉安撫臨潢,發倉粟以賑之,無衣者賜以幣帛,或官粟有闕,則收糴以給之,無妻室者具姓名以聞。還除刑部尚書。與工部尚書宗永、兵部侍郎完顏餘里也,往河南、山東、
 陜西宣問屯田軍人,有曾破大敵及攻城野戰立功者,具姓名以聞。或以寡敵眾,或與敵相當能先登敗敵者,正軍及擐甲阿里喜補官一階,猛安謀克以功狀上尚書省,曾隨海陵軍至淮上破敵者亦準上遷賞。



 僕散忠義伐宋,保衡行戶部於關中,兼糾察,許以便宜,黜守令不法者十餘人。邠守傅慎微忤用事者,被讒構下獄且死,保衡力救之得免。入為太常卿,遷禮部尚書。三年,拜參知政事。宋人請和,詔保衡往南京,與僕散忠義斟酌事宜,行之。入奏,進右丞。四年,宋人請和,師還,保衡朝京師。初,宮女稱心縱火十六位,延燒諸殿,上以方用兵,國
 用不足,不復營繕。及宋和,詔保衡監護役事,遣少府監張仲愈取南京宮殿圖本。上聞之,謂保衡曰:「追仲愈還。民間將謂朕效正隆華侈也。」



 六年冬,有疾,求致仕,不許。遣敬嗣暉傳詔曰:「卿以忠直擢居執政,齒髮未衰,遽以小疾求退。善加攝養,以俟疾間視事。」未幾,薨,年五十五。世宗將放鷹近郊,聞之乃還,為輟朝,賻贈,命有司致祭。



 翟永固,字仲堅,中都良鄉人。太祖與宋約攻遼,事成以燕歸宋。宋人以經義兼策取士,永固中第一,授開德府儀曹參軍。金破宋,永固北歸。中天會六年詞賦科,授懷安丞,遷望雲令,補樞密院令史,辟左副元帥宗翰府掾。
 永固家貧,求外補,宗翰愛其能,不許,以錢三千貫周之,薦於朝,攝左司郎中。除定武軍節度副使,歷同知清州防禦使,入為工部員外郎。以母憂去官,起復禮部郎中,遷翰林直學士。



 海陵篡立,宋國賀正旦使至廣寧,海陵使使以廢立事諭宋使,遣還之。以侍衛親軍都指揮使完顏思恭為報諭宋使,永固為副,且令永固伺察宋人動靜。使還,改禮部侍郎。久之,分護燕京宮室役事,永固請寫《無逸圖》於殿壁,不納。俄遷太常卿,考試貞元二年進士,出《尊祖配天賦》題,海陵以為猜度己意,召永固問曰:「賦題不稱朕意。我祖在位時祭天拜乎?」對曰:「拜。」海陵
 曰:「豈有生則致拜,死而同體配食者乎?」對曰:「古有之,載在典禮。」海陵曰:「若桀、紂曾行,亦欲我行之乎?」於是永固、張景仁皆杖二十。而進士張汝霖賦第八韻有曰:「方今將行郊祀。」海陵詰之曰:「汝安知我郊祀乎?」亦杖之三十。頃之,永固遷禮部尚書,賜笏頭球文金帶。改永定軍節度使。正隆二年,例降二品以上官爵,永固階光祿大夫不降,以寵異之。遷翰林學士承旨,與直學士韓汝嘉俱召至內殿,問以將親伐宋事,永固對曰:「宋人事本朝無釁隙,伐之無名。縱使可伐,亦無煩親征,遣將帥可也。」由是大忤海陵意,永固即請致仕。正隆四年正月丁巳,海
 陵朝永壽宮,四品以上官賜宴,永固至殿門外,海陵即以致仕宣命授之,永固歸臥于家。大定二年,起拜尚書左丞,請依舊制廉察官吏,革正隆守令之汙,從之。明年,表乞致仕,詔不許。罷為真定尹,賜通犀帶。尚書省奏,永固自執政為真定尹,其傘蓋當用何制度,上曰:「用執政制度。」遂著為令。五年,懇乞致仕,許之。六年,薨。



 魏子平,字仲均,弘州人。登進士第,調五臺主簿,累除為尚書省令史,除大理丞,歷左司都事,同知中都轉運使事,太府監。正隆三年,為賀宋主生日副使。是時,海陵謀伐宋,子平使還,入見,海陵問江左事,且曰:「蘇州與大名孰
 優?」子平對曰:「江、湖地卑濕,夏服蕉葛,猶不堪暑,安得與大名比也。」海陵不悅。世宗即位,除戶部侍郎。大定二年,丞相僕散忠義伐宋,置元帥府於南京,子平掌饋運,給金牌一、銀牌六,糧道給辦。進戶部尚書。六年,復為賀宋主生日使,上曰:「使宋無再往者,卿昔年供河南軍儲有勞,用此優卿耳。」



 久之,拜參知政事。上問子平曰:「古者稅什一而民足,今百一而民不足,何也?」子平對曰:「什一取其公田之入,今無公田而稅其私田,為法不同。古有一易再易之田,中田一年荒而不種,下田二年荒而不種。今乃一切與上田均稅之,此民所以困也。」上又問曰:「
 戍卒逋亡物故,今按物力高者補之,可乎?」對曰:「富家子弟騃懦不可用,守戍歲時求索無厭,家產隨壞。若按物力多寡賦之,募材勇騎射之士,不足則調兵家子弟補之,庶幾官收實用,人無失職之患。」上從之。



 海州捕賊八十餘人,賊首海州人,其兄今為宋之軍官。上聞之,謂宰相曰:「宋之和好,恐不能久,其宿、泗間漢軍,以女直軍代之。」子平曰:「誓書稱沿邊州城,除自來合設置射糧軍數並巡尉外,更不得屯軍守戍。」上曰:「此更代之,非增戍也。」



 上曰:「前日令內任官六品以上,外任五品以上,並舉所知。未聞有舉之者,豈無其才,蓋知而不舉也。」子平曰:「請
 令當舉之官,每任須舉一人。」澤州刺史劉德裕、祁州刺史斜哥、滄州同知訛里也、易州同知訛里剌、楚丘縣令劉春哥以贓汙抵罪,上欲詔示中外,丞相守道以為不可,上以問子平曰:「卿意何如?」子平曰:「臣聞懲一戒百,陛下固宜行之。」上曰:「然。」遂降詔焉。



 宋人於襄陽漢江上造舟為浮梁三,南京統軍司聞而奏之,上問宰臣曰:「卿等度之,以為何如?」子平曰:「臣聞襄陽薪芻,皆於江北取之,殆為此也。」上曰:「朕與卿等治天下,當治其未然。及其有事,然後治之,則亦晚矣。」河南統軍使宗敘求入見奏邊事,上使修起居注粘割斡特剌就問狀。宗敘言:「得邊報
 及宋來歸者言,宋國調兵募民,運糧餉,完城郭,造戰船浮橋,兵馬移屯江北。自和議後即罷制置司,今復置矣。商、虢、海州皆有姦人出沒,此不可不備。嘗報樞密院,彼視以為文移,故欲入見言之。」斡特剌召凡言邊事者詰問,皆無實狀,行至境上,問知襄陽浮橋乃樵採之路,如子平策。還奏。詔凡妄說邊關兵事者徒二年,告人得實,賞錢五百貫。



 上問宰臣曰:「祭宗廟用牛。牛盡力稼穡有功於人,殺之何如?」子平對曰:「惟天地宗廟用之,所以異大祀之禮也。」



 十一年,罷為南京留守,未幾致仕。十五年,起為平陽尹,復致仕。二十六年,薨於家。



 孟浩,字浩然,濼州人。遼末年登進士第。天會三年,為樞密院令史,除平州觀察判官。天眷初,選入元帥府備任使,承制除歸德少尹,充行臺吏、禮部郎中,入為戶部員外郎、郎中。韓企先為相,拔擢一時賢能,皆置機要,浩與田玨皆在尚書省,玨為吏部侍郎,浩為左司員外郎。既典選,善銓量人物,分別賢否,所引用皆君子。而蔡松年、曹望之、許霖皆小人,求與玨相結,玨薄其為人拒之。松年,蔡靖子。靖將兵不能守燕山,終敗宋國,玨頗以此譏斥松年,松年初事宗弼於行臺省,以微巧得宗弼意,宗弼當國,引為刑部員外郎。望之為尚書省都事,霖為省
 令史。皆怨玨等,時時毀短之於宗弼,凡與玨善者皆指以為朋黨。韓企先疾病,宗弼往問之,是日,玨在企先所,聞宗弼至,知其惡己,乃自屏以避。宗弼曰:「丞相年老且疾病,誰可繼丞相者?」企先舉玨,而宗弼先入松年譖言,謂企先曰:「此輩可誅。」玨聞流汗浹背。企先薨,玨出為橫海軍節度使。選人龔夷鑒除名,值赦,赴吏部銓,得預覃恩。玨已除橫海,部吏以夷鑒白玨,玨乃倒用月日署之。許霖在省典覃恩,行臺省工部員外郎張子周素與玨有怨,以事至京師,微知夷鑒覃恩事,嗾許霖發之,樉以專擅朝政。詔獄鞫之,擬玨與奚毅、邢具瞻、王植、高鳳庭、
 王效、趙益興、龔夷鑒死,其妻子及所往來孟浩等三十四人皆徙海上,仍不以赦原。天下冤之。世宗在熙宗時,知田玨黨事皆松年等構成之。而浩等三十二人遇天德赦令還鄉里,多物故,惟浩與玨兄穀、王補、馮煦、王中安在。大定二年,召見,復官爵。浩為侍御史,穀為大理丞,補為工部員外郎,煦為兵部主事,中安知火山軍事,而浩尋復為右司員外郎。



 浩篤實,遇事輒言,無所隱。上嘉其忠,每對大臣稱之。有疾,求外補,除祁州刺史,致仕,歸。七年,起為御史中丞,而浩已年老,世宗以不次用之,再閱月,拜參知政事。故事,無自中丞拜執政者,浩辭曰:「不
 次之恩,非臣所敢當。」上曰:「卿自刺史致仕,除中丞,國家用人,豈拘階次?卿公正忠勤,雖年高猶可宣力數年,朕思之久矣。」浩頓首謝。



 世宗敕有司東宮涼樓增建殿位,浩諫曰:「皇太子義兼臣子,若所居與至尊宮室相侔,恐制度未宜,固宜示以儉德。」上曰:「善。」遂罷其役,因謂太子曰:「朕思漢文純儉,心常慕之,汝亦可以為則也。」未幾,皇太子生日,上宴群臣于東宮,以大玉杓、黃金五百兩賜丞相志寧,顧謂群臣曰:「卿等能立功,朕亦褒賞如此。」又曰:「參政孟浩公正敢言,自中丞為執政。卿等能如是,朕亦不次用之。」世宗嘗曰:「女直本尚純朴,今之風俗,日薄
 一日,朕甚憫焉。」浩對曰:「臣四十年前在會寧,當時風俗與今日不同,誠如聖訓。」上曰:「卿舊人,固知之。」上謂宰臣曰:「宋前廢帝呼其叔湘東王為『豬王』,食之以牢,納之泥中,以為戲笑。書于史策,所以勸善而懲惡也。海陵以近習掌記注,記注不明,當時行事,實錄不載,眾人共知之者求訪書之。」浩對曰:「良史直筆,君舉必書。帝王不自觀史,記注之臣乃得盡其直筆。」浩復奏曰:「歷古以來,不明賞罰而能治者,未之聞也。國家賞善罰惡,蓋亦多矣,而天下莫能知。乞自今凡賞功罰罪,皆具事狀頒告之,使君子知勸以遷善,小人知懼以自警。」從之。進尚書右丞,
 兼太子少傅。罷為真定尹,上曰:「卿年雖老,精神不衰,善治軍民,毋遽言退。」以通犀帶賜之。十三年,薨。



 田穀自大理丞累官同知中京留守,終于利涉軍節度使。



 二十九年,章宗詔尚書省曰:「故吏部侍郎田玨等皆中正之士,小人以朋黨陷之,由是得罪。世宗用孟浩為右丞,當時在者俱已用之,亡者未加追復,其議以聞。」張汝霖奏曰:「玨專權樹黨,先朝已正罪名,莫不稱當。今追贈官爵,恐無懲勸。」汝霖先朝大臣,嘗與顧命,上初即位,不肯輒逆其意,謂之曰:「卿既以為不可,姑置之。」蓋張浩與蔡松年友善,故汝霖猶擠之也。汝霖死後,章宗復詔尚書省曰:「
 蓋自田玨黨事之後,有官者以為戒,惟務茍且,習以成風。先帝知玨等無罪,錄用生存之人,有擢至宰執者,其次有為節度、防禦、刺史者。其死者猶未追復,子孫猶在編戶,朕甚憫焉。惟旌賢顯善,無間存沒,宜推先帝所以褒錄忠直之意,並加恩恤,以勵風俗。據田玨一起人除已敘用外,但未經任用身死,並與復舊官爵。其子孫當時已有官職,以父祖坐黨因而削除者,亦與追復。應合追復爵位人等子孫不及蔭敘者,亦皆量與恩例。」



 梁肅,字孟容,奉聖州人。自幼勤學,夏夜讀書,往往達旦,母葛氏常滅燭止之。天眷二年,擢進士第,調平遙縣主
 簿,遷望都、絳縣令。以廉,入為尚書省令史。除定海軍節度副使,改中都警巡使,遷山東西路轉運副使。營治汴宮,肅分護役事。攝大名少尹。正隆末,境內盜起,驅百姓平人陷賊中不能自辨者數千人,皆繫大名獄。肅到官,考驗得其情讞,出者十八九。大定二年,宛平趙植上書曰:「頃者,正隆任用閹寺,少府少監兼上林署令胡守忠因緣巧倖,規取民利。前薊州刺史完顏守道、前中都警巡使梁肅,勤恪清廉,願加進擢。」於是守忠落少監,守道自濱州刺史召為諫議大夫,肅中都轉運副使改大興少尹。



 肅上疏言:「方今用度不足,非但邊兵耗費而已。吏
 部以常調除漕司僚佐,皆年老資高者為之,類不稱職。臣謂凡軍功、進士諸科、門蔭人,知錢穀利害,能使國用饒足而不傷民者,許上書自言。就擇其可用,授以職事。每五年委吏部通校有無水旱屯兵,視其增耗而黜陟之。自漢武帝用桑弘羊始立榷酤法,民間粟麥歲為酒所耗者十常二三。宜禁天下酒曲,自京師及州郡官務,仍舊不得酤販出城。其縣鎮鄉村,權行停止。」不報。



 三年,坐捕蝗不如期,貶川州刺史,削官一階,解職。上御便殿,召左諫議大夫奚籲、翰林待制劉仲誨,秘書少監移剌子敬,訪問古今事。少間,籲從容請曰:「梁肅材可惜,解職
 太重。」上曰:「卿言是也。」乃除河北東路轉運副使。是時,窩斡亂後,兵食不足,詔肅措置沿邊兵食。移牒肇州、北京、廣寧鹽場,許民以米易鹽,兵民皆得其利。四年,通檢東平、大名兩路戶籍物力,稱其平允。他使者所至皆以苛刻增益為功,百姓訴苦之。朝廷敕諸路以東平、大名通檢為準,於是始定。



 七年,父憂去官。起復都水監。河決李固,詔肅視之,還奏:「決河水六分,舊河水四分。今障塞決河,復故道為一,再決而南則南京憂,再決而北則山東、河北皆可憂。不若止於李固南築隄,使兩河分流,以殺水勢便。」上從之。



 改大理卿。尚輦局本把石抹阿里哥與
 釘鉸匠陳外兒共盜宮中造車銀釘葉,肅以阿里哥監臨,當首坐。他寺官以陳外兒為首,抵死。上曰:「罪疑惟輕,各免死,徒五年,除名。」於時,東京久不治,上自擇肅為同知東京留守事。遷中都都轉運使,轉吏部尚書。上疏論臺諫,其大旨謂:「臺官自大夫至監察,諫官自大夫至拾遺,陛下宜親擇,不可委之宰相,恐樹私恩,塞言路也。」上嘉納之。復請奴婢不得服羅,上曰:「近已禁奴婢服明金矣,可漸行之。」肅舉同安主簿高旭,除平陽酒使,肅奏曰:「明君用人,必器使之。旭儒士,優於治民,若使坐列肆,榷酒酤,非所能也。臣愚以為諸道鹽鐵使依舊文武參注,
 其酒稅使副以右選三差俱最者為之。」上曰:「善。」改刑部尚書。



 宋主屢請免立受國書之儀,世宗不從。及大興尹璋為十四年正旦使,宋主使人就館奪其書,而重賂之。璋還,杖一百五十,除名。以肅為宋國詳問使,其書略曰:「盟書所載,止於帝加皇字,免奉表稱臣稱名再拜,量減歲幣,便用舊儀,親接國書。茲禮一定,於今十年。今知歲元國信使到彼,不依禮例引見,輒令迫取於館,姪國禮體當如是耶?往問其詳,宜以誠報。」肅至宋,宋主一一如約,立接國書。肅還,附書謝,其略曰:「姪宋皇帝謹再拜,致書于叔大金應天興祚欽文廣武仁德聖孝皇帝闕下。
 惟十載遵盟之久,無一毫成約之違,獨顧禮文,宜存折衷。矧辱函封之貺,尚循躬受之儀,既俯迫於輿情,嘗屢伸于誠請,因歲元之來使,遂商榷以從權。敢勞將命之還,先布鄙悰之懇,自余專使肅控請祈。」肅還至泗州,先遣都管趙王府長史駝滿蒲馬入奏。世宗大喜,欲以肅為執政,左丞相良弼曰:「梁肅可相,但使宋還即為之,宋人自此輕我矣。」上乃止。



 久之,為濟南尹,上疏曰:「刑罰世輕世重,自漢文除肉刑,罪至徒者帶鐐居役,歲滿釋之,家無兼丁者,加杖準徒。今取遼季之法,徒一年者杖一百,是一罪二刑也,刑罰之重,於斯為甚。今太平日久,當
 用中典,有司猶用重法,臣實痛之。自今徒罪之人,止居作,更不決杖,」不報。



 未幾,致仕,起復彰德軍節度使,召拜參知政事。上謂侍臣曰:「梁肅以治入異等,遂至大任,廉吏亦可以勸矣。」肅奏:「漢之羽林,皆通《孝經》。今之親軍,即漢之羽林也。臣乞每百戶賜《孝經》一部,使之教讀,庶知臣子之道,其出職也,可知政事。」上曰:「善,人之行,莫大於孝,亦由教而後能。」詔與護衛俱賜焉。復上奏曰:「方今斗米三百,人已困餓,以錢難得故也。計天下歲入二千萬貫以上,一歲之用餘千萬。院務坊場及百姓合納錢者,通減數百萬。院務坊場可折納穀帛,折支官兵俸給,使
 錢布散民間,稍稍易得。」上曰:「懸欠院務,許折納,可也。」



 肅上疏論生財舒用八事。一曰罷隨司通事;二曰罷酒稅司杓欄人;三曰天水郡王本族已無在者,其餘皆遠族,可罷養濟;四曰裁減隨司契丹吏員;五曰罷榷醋,以利與民;六曰量減鹽價,使私鹽不行,民不犯法;七曰隨路酒稅許折納諸物;八曰今歲大稔,乞廣糴粟麥,使錢貨流出。上曰:「趙氏養濟一事,乃國家美政,不可罷。其七事,宰相詳議以聞。」上又曰:「朕在位二十餘年,鑒海陵之失,屢有改作,亦不免有繆戾者,卿等悉心奏之。」肅論「正員官被差,權攝官有公罪,及正員還任,皆准去官勿論,往
 往其人茍且,不事其事。乞于縣令中留十人備差,無差正員官。」上曰:「自今權攝有公罪,正員雖還而本職未替者,勿以去官論之。」肅曰:「誠如聖旨。」肅與宰相奏事,既罷,肅跪而言曰:「四時畋獵,雖古禮,聖人亦以為戒。陛下春秋高,屬時嚴寒,馳聘於山林之間。法宮燕處,亦足怡神,願為宗社自重,天下之福也。」上曰:「朕諸子方壯,使之習武,故時一往爾。」



 同知震武軍節度使鄧秉鈞陳言四事,其一言外多闕官,及循資擬注不得人,上以問宰相張汝弼,曰:「循資格行已久,仍舊便。」肅曰:「不然。如亡遼固不足道,其用人之法有仕及四十年無敗事,即與節度使,
 豈必循資哉。」上曰:「仕四十年已衰老。察其政績,善者升之,後政再察之,善又升之,如此可以得人,亦無曠事。」肅曰:「誠如聖訓。」肅論盜賊不息,請無禁兵器。上曰:「所在有兵器,其利害如何?」肅曰:「他路則已,中都一路上農夫聽置之,似乎無害。」上曰:「朕將思之。」



 凡使宋者,宋人致禮物,大使金二百兩,銀二千兩,副使半之,幣帛雜物稱是。及推排物力,肅自以身為執政,昔嘗使宋,所得禮物多,當為庶民率先,乃自增物力六十餘貫,論者多之。



 二十三年,肅請老,上謂宰臣曰:「梁肅知無不言,正人也。卿等知而不言,朕實鄙之。雖然,肅老矣,宜從其請。」遂再致仕。詔
 以其子汝翼為閣門祗候。二十八年,薨。謚正憲。



 移剌綎,本名移敵列,契丹虞呂部人。通契丹、漢字,尚書省辟契丹令史,攝知除,擢右司都事。正隆南伐,兼領契丹、漢字兩司都事。大定二年,除真定少尹,入為侍御史。母憂去官。起復右司員外郎,累官陳州防禦使。左丞相紇石烈良弼致仕,上問:「誰可代卿者?」對曰:「陳州防禦使移剌綎,清幹忠正,臣不及也。」遂召為太府監。改刑部侍郎。



 十九年,以按出虎等八猛安,自河南徙置大名、東平之境。還為大理卿,被詔典領更定制條。初,皇統間,參酌隋、唐、遼、宋律令,以為皇統制條。海陵虐法,率意更改,或
 同罪異罰,或輕重不倫,或共條重出,或虛文贅意,吏不知適從,夤緣舞法。慥取皇統舊制及海陵續降,通類校定,通其窒礙,略其繁碎。有例該而條不載者,用例補之。特闕者用律增之。凡制律不該及疑不能參決者,取旨畫定。凡特旨處分,及權宜條例內有可常行者,收為永格。其餘未可削去者,別為一部。大凡一千一百九十條,為十二卷。書奏,詔頒行之,賜銀幣有差。頃之,摘徙山東猛安八謀克于河北東路,置之酬斡、青狗兒兩猛安舊居之地,詔無牛耕者買牛給之。攝御史大夫。數月,改御史中丞,兼同修國史,遷刑部尚書,改吏部尚書。尋改大
 興尹。駕幸上京,顯宗守國,使人諭之曰:「自大駕東巡,京尹所治甚善。我將有春水之行,當益勤乃事。」還以所獲鵝鴨賜之。有疾在告,遣官醫診視。復為刑部尚書。上還自上京,以為西京留守,改臨洮尹,卒。



 移剌子敬,字同文,本名屋骨朵魯,遼五院人。曾祖霸哥,同平章事。父拔魯,準備任使官。都統杲克中京,遼主西走,留拔魯督輜重,已而輜重被掠,拔魯乃自髡,逃於山林。子敬讀書好學,皇統間,特進移剌固修《遼史》,辟為掾屬。《遼史》成,除同知遼州事。舊本自有占地,歲入數百貫,州官歲取其課,地主以為例,未嘗請辯。子敬曰:「已有
 公田,何為更取民田。」竟不取。秩滿,郡人請留於行臺省,不許。天德三年,入為翰林修撰,遷禮部郎中。



 正隆元年,諸將巡邊,詔子敬監戰,軍帥以戰獲分將士,亦以遺子敬,子敬不受。及還,入見,海陵謂之曰:「汝家貧而不茍得,不受俘獲,朕甚嘉之。」凡同行官僚所取者,皆沒入於官。其後詔子敬宴賜諸部,諭之曰:「凡受進,例遣宰臣,以汝前能稱職,故特命汝。」使還,遷翰林待制。大定二年,以待制同修國史。是時,窩斡餘黨散居諸猛安謀克中,詔子敬往撫之,仍宣諭猛安謀克,及州縣漢人,無以前時用兵相殺傷,挾怨輒害契丹人。使還,改秘書少監,兼修起
 居注,修史如故。詔曰:「以汝博通古今,故以命汝。」常召入講論古今及時政利害,或至夜半。子敬有良馬,平章政事完顏元宜索之,子敬以元宜為相也,不與。至是,元宜乞致仕,罷為東京,子敬乃以此馬贐行,識者韙之。



 是時,僕散忠義伐宋,宋請和,而書式、疆界未定。子敬與秘書少監石抹頤、修起居注張汝弼侍便殿,上曰:「宋主求成,反覆無信,喜為誇大。」子敬對曰:「宋人自來浮辭相欺,來書言海陵敗于采石,大軍北歸,按兵不襲,俾全師而還。海陵未嘗敗于采石,其譎詐多此類也。回書宜言往者大軍若令渡江,宋國境土,必為我有。」上曰:「彼以詭詐,我
 以誠實,但當以理折之。」遷右諫議大夫,起居注如故。



 上幸西京,州縣官入見,猛安謀克不得隨班。子敬奏軍民一體,合令猛安謀克隨班入見,上嘉納之,於是責讓宣徽院。及端午朝會,詔依子敬奏行之。子敬言山後禁獵地太廣,有妨百姓耕墾,上用其言,遂以四外獵地與民。遷秘書監,諫議、起居如故。



 子敬舉同知宣徽院事移剌神獨斡、兵部侍郎移剌按答,太子少詹事烏古論三合自代,上不許。子敬與同簽宣徽院事移剌神獨斡侍,上曰:「亡遼不忘舊俗,朕以為是。海陵習學漢人風俗,是忘本也。若依國家舊風,四境可以無虞,此長久之計也。」世宗
 將如涼陘,子敬與右補闕粘割斡特剌、左拾遺楊伯仁奏曰:「車駕至曷里滸,西北招討司囿於行宮之內地矣。乞遷之於界上,以屏蔽環衛。」上曰:「善。」詔尚書省曰:「招討斜里虎可徙界上,治蕃部事。都監撒八仍於燕子城治猛安謀克事。」



 上與侍臣論古之人君賢否,子敬奏曰:「陛下凡與宰臣謀議,不可不令史官知之。」上曰:「卿言是也。」轉簽書樞密院事,同修國史,出為河中尹,請老。河中地熱,上恐子敬不耐暑,改興中尹。子敬女自懿州來興中省謁,遇盜途中,剽掠其行李且盡,既而還之,謝曰:「我輩初不知為府尹家也,尹有德于民,尚忍侵犯邪。」徙咸平、
 廣寧尹。二十一年,致仕,卒于家,年七十一。子敬嘗使宋,及受諸部進貢,所受禮物,皆散之親舊。及卒,家無餘財,其子質宅以營葬事。



 贊曰:金制,尚書令、左右丞相、平章政事,是謂宰相;左右丞、參知政事,是謂執政。大抵因唐官而稍異焉,因革不同,無足疑者。《書》曰:「元首明哉,股肱良哉,庶事康哉。」又曰:「元首叢脞哉,股肱惰哉,萬事隳哉。」宰相、執政,豈異道邪?蘇保衡、翟永固、魏子平、孟浩、梁肅皆當時之賢執政也。移剌綎、子敬有其才,適其時,而位不及者,亦命也夫。



\end{pinyinscope}