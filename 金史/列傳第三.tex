\article{列傳第三}

\begin{pinyinscope}

 始祖以下諸子



 ○斡魯輩魯謝庫德孫拔達謝夷保子盆納謝里忽烏古出跋黑崇成本名僕灰劾孫子浦家奴麻頗子謾都本謾都訶斡帶
 斡賽子宗永斡者孫璋昂本名吾都補子鄭家



 始祖明懿皇后生德帝烏魯,季曰斡魯,女曰注思版,皆福壽之語也。以六十後生子,異之,故皆以嘉名名之焉。



 德帝思皇后生安帝,季曰輩魯。輩魯與獻祖俱徙海姑水,置屋宇焉。



 輩魯之孫胡率。胡率之子劾者,與景祖長子韓國公劾者同名。韓國公前死,所謂肅宗納劾者之妻加古氏者是也。穆宗四年伐阿疏。阿疏走遼。遼使使來止伐阿疏軍。穆宗陽受遼帝約束,先歸國,留劾者守阿疏城。凡三年,卒攻破之。天會十五年贈特進。



 安帝節皇后生獻祖,次曰信德,次曰謝庫德,次曰謝夷保,次曰謝里忽。



 謝庫德之孫拔達,謝夷保之子盆納,皆佐世祖有功。盆納勇毅善射,當時有與同名者,嘗有貳志,目之曰「惡盆納」。天會十五年,拔達贈儀同三司,盆納贈開府儀同三司。在世祖時,歡都、冶訶及劾者、拔達、盆納五人者,不離左右,親若手足,元勛之最著者也。明昌五年皆配饗世祖廟廷。



 准德、束里保者,皆加古部人。申乃因、醜阿皆馳滿部人。富者粘沒罕,完顏部人。阿庫德、白達皆雅達瀾水完顏部勃董。此七人者,當攜離之際,能一心竭力輔
 戴者也。



 達紀、胡蘇皆術甲部勃董。勝昆、主保皆術虎部人。阿庫德,溫迪痕部人。此五人者,又其次者也。



 世祖初年,跋黑為變,烏春盛強,使人召阿庫德、白達。阿庫德曰:「吾不知其他,死生與太師共之。」太師,謂世祖也。白達大喜曰:「我心正如此耳。烏春兵來,堅壁自守,勿與戰可也。」達紀、胡蘇居琵里郭水,烏春兵出其間,不為變,終拒而不從。勝昆居胡不干村,其兄滓不乃勃堇,烏春止其家,而以兵圍勝昆。烏春解去,世祖殺滓不乃,勝昆請無孥戮,世祖從之。世祖破桓赧、散達,主保死焉。天會十五年,准德、申乃因、阿庫德、白達皆贈金紫光祿大夫。束里保、
 醜阿、富者粘沒罕、達紀、胡蘇、勝昆、主保、溫迪痕、阿庫德皆贈銀青光祿大夫,皆天會十五年追贈。



 又有胡論加古部勝昆勃堇、蟬春水烏延部富者郭赧,畏烏春彊,請世祖兵出其間,以為重也。世祖使斜列、躍盤將別軍過之。郭赧教科列取先在烏春軍中二十二人,烏春覺之,殺二人,得二十人。郭赧又以士人益斜列軍。穆宗他日嘉此功不能忘,以斜列之女守寧妻郭赧子胡里罕焉。



 婆多吐水裴滿部斡不勃堇附於世祖,桓赧焚之。斡不卒,世祖厚撫其家。因併錄之,以見立國之艱難云。



 謝里忽者,昭祖將定法制,諸父、國人不悅,已執昭祖,將
 殺之。謝里忽亟往,彎弓注矢,射於眾中,眾乃散去,昭祖得免。國俗,有被殺者,心使巫覡以詛祝殺之者,乃繫刃於杖端,與眾至其家,歌而詛之曰:「取爾一角指天、一角指地之牛,無名之馬,向之則華面,背之則白尾,橫視之則有左右翼者。」其聲哀切悽婉,若《蒿里》之音。既而以刃畫地,劫取畜產財物而還。其家一經詛祝,家道輒敗。



 及來流水烏薩扎部殺完顏部人,昭祖往烏薩扎部以國俗治之,大有所獲,頒之於諸父昆弟而不及謝里忽。謝里忽曰:「前日免汝於死者吾之力,往治烏薩扎部者吾之謀也。分不及我。何邪。」昭祖於是早起,自齎間金列鞢
 往饋之。時謝里忽猶未起,擁寢衣而問曰:「爾為誰?」昭祖曰:「石魯先擇此寶,而後頒及他人,敢私布之。」謝里忽既揚言,初不自安,至是乃大喜。列鞢者,腰佩也。



 獻祖恭靖皇后生昭祖,次曰朴都,次曰阿保寒,次曰敵酷,次曰敵古迺,次曰撒里輦,次曰撒葛周。



 昭祖威順皇后生景祖,次曰烏古出。次室達胡末,烏薩扎部人,生跋黑、僕里黑、斡裡安。次室高麗人,生胡失答。



 烏古出,初昭祖久無子,有巫者能道神語,甚驗,乃往禱焉。巫良久曰:「男子之魂至矣。此子厚有福德,子孫昌盛;可拜而受之。若生,則名之曰烏古迺。」是為景祖。又良久
 曰:「女子之魂至矣,可名曰五鵶忍。」又良久曰:「女子之兆復見,可名曰斡都拔。」又久之,復曰:「男子之兆復見,然性不馴良,長則殘忍,無親親之恩,必行非義,不可受也。」昭祖方念後嗣未立,乃曰:「雖不良,亦願受之。」巫者曰:「當名之曰烏古出。」既而生二男二女,其次弟先後皆如巫者之言,遂以巫所命名名之。



 景祖初立,烏古出酗酒,屢悖威順皇后。后曰:「巫言驗矣,悖亂之人終不可留。」遂與景祖謀而殺之。部人怒曰:「此子性如此,在國俗當主父母之業,奈何殺之?」欲殺景祖。后乃匿景祖,出謂眾曰:「為子而悖其母,率是而行,將焉用之?吾割愛而殺之,烏古迺
 不知也,汝輩寧殺我乎?」眾乃罷去。烏古出之子習不失,自有傳。



 跋黑及同母弟二人,自幼時每爭攘飲食,昭祖見而惡之,曰:「吾娶此妾而生子如此,後必為子孫之患。」世祖初立,跋黑果有異志,誘桓赧、散達、烏春、窩謀罕離間部屬,使貳於世祖。世祖患之,乃加意事之,使為勃堇而不令典兵。



 跋黑既陰與桓赧、烏春謀計,國人皆知之,而童謠有「欲征則附於跋黑,欲死則附於劾里缽、頗刺淑」之語。世祖亦以策探得兄弟部人向背。烏春、桓赧相次以兵來攻,世祖外禦強兵,而內畏跋黑之變。將行,聞跋黑食
 於其愛妾之父家,肉張咽而死,且喜且悲,乃迎尸而哭之。



 崇成,本名僕灰,泰州司屬司人,昭祖玄孫也。大定十八年收充奉職,改東宮入殿小底,轉護衛。二十五年,章宗為原王,充本府祗候郎君。明年,上為皇太孫,復為護衛。上即位,授河間府判官,以憂去職。起復為宿直將軍,累遷武衛軍都指揮使。泰和三年卒,賻贈有加。崇成謹飭有守,宿衛二十餘年,未嘗有過,故久侍密近云。



 景祖昭肅皇后生韓國公劾者,次世祖,次沂國公劾孫,次肅宗,次穆宗。次室注思灰,契丹人,生代國公劾真保。
 次室溫迪痕氏,名敵本,生虞國公麻頗、隋國公阿離合懣、鄭國公謾都訶。劾者、阿離合懣別有傳。



 劾孫。天會十四年大封宗室,劾孫追封王爵。正隆例降封鄭國公。



 子蒲家奴又名昱,嘗從太祖伐留可、塢塔。太祖使蒲家奴招詐都,詐都即降。康宗八年,係遼籍女直紇石烈部阿里保太彎阻兵,招納亡命,邊民多亡歸之。蒲家奴以偏師夜行書止,抵石勒水,襲擊破之,盡俘其孥而還。邊氓自此無復亡者。後與宗雄視泰州地土,太祖因徙萬家屯田于其地。



 天輔五年,蒲家奴為吳勃極烈,遂為都統,使襲遼帝,而以雨潦不果行。既而,忽魯勃
 極烈杲都統內外諸軍以取中京,蒲家奴等皆為之副。遼帝西走,都統杲使蒲家奴以兵一千助撻懶擊遼都統馬哥,與撻懶不相及,蒲家奴與賽里、斜野降其西北居延之眾。而降民稍復逃散,毗室部亦叛,遂率兵襲之。至鐵呂川,遇敵八千,遂力戰,兵敗。察刺以兵來會,追及敵兵於黃水,獲畜產甚眾。是役也,奧燉按打海被十一創,竟敗敵兵而還。軍于旺國崖西。



 賽里亦以兵會太祖,自草濼追遼帝,蒲家奴、宗望為前鋒,戒之曰:「彼若深溝高壘,未可與戰,即偵伺巡邏,勿令遁去,以俟大軍。若其無備,便可擊也。」上次胡離畛川,吳十、馬和尚至小魚濼,
 夜潛入遼主營,執新羅奴以還,送知遼帝所在。蒲家奴等晝夜兼行,追及于石輦鐸。我兵四千,至者才千人,遼兵圍之。余睹指遼帝麾蓋,騎兵馳之,遼帝遁去,兵遂潰,所殺甚眾。



 宗翰為西北西南兩路都統,蒲家奴、斡魯為之副。烏虎部叛,蒲家奴討平之。天會間,為司空,封王。天眷二年,宗磐等誅,辭及蒲家奴,詔奪司空。是年,薨。天德初,配享太祖廟廷。正隆二年,例封豫國公。



 麻頗,天會十五年封王,正隆例封虞國公。



 長子謾都本,孝友恭謹,多謀而善戰。年十五,隸軍中,從攻窩盧歡。及系遼女直胡失荅等為變,謾都本自為質,遂從胡失荅
 歸,中途以計殺守者而還。攻寧江州,敢黃龍府,破高永昌,取春、泰州,皆有功,多受賞賚,遂為謀克。討嶺東未服州郡。過土河東山,敗賊三千人。奚、契丹寇土河西,與猛安蒙葛、麻吉擊之。謾都本對敵之中,推鋒力戰,破其眾九萬人。奚眾萬餘保阿鄰甸,復擊敗之,降其旁近居人。復以五百騎破遼兵一千,生擒其將以歸。與闍母攻興中府,中流矢卒,年三十七。天眷中,贈金紫光祿大夫,謚英毅。



 謾都訶,屢從征伐,天會二年為阿捨勃極烈,參議國政,明年薨。天會十五年,大封宗室,追封王。正隆例封鄭國
 公,明昌五年,謚定濟。



 蠻睹,襲父麻頗猛安。蠻睹卒,子掃合襲。掃合卒,子撒合輦襲。撒合輦卒,子惟鎔襲。



 惟鎔本名沒烈,字子鑄,駢脅多力,喜周急人。至寧初,守楊文關有功,兼都統,護漕運。貞祐二年,佩金牌護親軍家屬遷汴,遙授同知祁州軍州事,充提控。貞祐三年,破紅襖賊於大沫堌,惟鎔入自北門,諸軍繼進,生獲劉二祖,功最。遷泰安軍節度副使,改遂王府尉幄都水少監、東平府治中。坐誤以刃傷同知府事紇石烈牙吾塔,當削降殿年,仍從軍自效。討花帽賊于曹、濟間,行省蒙古綱奏其功,復前職。遷邳州經略使,
 卒。子從傑襲猛安,累功遙授鎮南軍節度副使。



 世祖翼簡皇后生康宗,次太祖,次魏王斡帶,次太宗,次遼王斜也。次室徒單氏生衛王斡賽,次魯王斡者。次室僕散氏生漢王烏故乃。次室術虎氏生魯王闍母。次室術虎氏生沂王查刺。次室烏古論氏生鄆王昂。



 斡帶,年二十餘,撒改伐留可,斡帶與習不失、阿里合懣等俱為裨將。諸將議攻取,斡帶主攻城便。太祖將至軍,斡帶迎之,謂太祖曰:「留可城且下,忽惑他議。」太祖從之。至軍中,眾議乃決。斡帶急起治攻具。其夜進兵攻城,遲明破之。及二涅囊虎路、二蠢出路寇盜,斡帶盡平之。



 康
 宗二年甲申,蘇濱水諸部不聽命,康宗使斡帶等往治其事。行次活羅海川撒阿村,召諸部。諸部皆至,惟含國部斡豁勃堇不至。斡準部狄庫德勃堇、職德部廝故速勃堇亦皆遁去,遇塢塔於馬紀嶺,塢塔遂執二人以降。於是,使斡帶將後伐斡豁,募軍于蘇濱水,斡豁元聚固守,攻而拔之。進師北琴海闢登路,攻拔泓忒城,取畔者以歸。



 太祖於母弟中最愛斡帶。斡帶歸自泓忒城,太祖以事如寧江州,欲與斡帶偕行,斡帶曰:「兵役久勞,未及息也。」遂不果行。太祖還,晝寐于來流水傍,夢斡帶之場圃火,禾盡焚,不可撲滅,覺而深念之,以為憂。是時,斡帶
 已寢疾,太祖至,聞之,過家門不下馬,徑至斡帶所問疾。未幾薨,年三十四。太祖每哭之慟,謂人曰:「予強與之偕行,未必死也。」



 斡帶剛毅果斷,服用整肅,臨戰決策,有世祖風。世祖之世,軍旅之事多專任之。太祖平遼,歎曰:「恨斡帶之不及見也。」天會十五年,追封儀同三司、魏王,謚曰定肅。



 斡賽,穆宗初,斡準部族相鈔略,遣納根涅孛堇以其兵往治,納根涅擅募蘇濱水人為兵,不別的,輒攻略之。其人來告,穆宗使斡賽及冶訶往問狀。納根涅雖伏而不肯人賞所取,因遁去。冶訶等皆不欲追,斡賽督軍而進。至把
 忽嶺西毛密水,及之,大破其眾,納根涅死焉。斡賽撫定蘇濱水民部,執納根涅之母及其妻子而歸。穆宗曰:「斡賽年尚幼,已能集事,可嘉也。」康宗二年甲申,斡帶治蘇濱水諸部,斡賽、斡魯佐之,定諸部而還。



 久之,高麗殺行人阿聒、勝昆,而築九城於曷懶甸。斡賽將內外兵,劾古活你茁、蒲察狄古乃佐之。高麗兵數萬來拒,斡賽分兵為十隊,更出迭人,遂大破之。斡賽母和你隈疾篤,召還,以斡魯代之。未幾,斡賽復至軍,再破高麗軍,進圍其城。七月,高麗請和,盡歸前後亡命及所侵故地,退九城之戍,遂與之和。皇統五年,追封衛國王。



 宗永,本名挑撻,斡賽子。長身美髯,忠確勇毅。天眷初,以宗室子預誅宗磐,擢寧遠大將軍。皇統初,充牌印祗候。五年,出為趙州刺史,秩滿再任,轉興平軍節度使,改大名尹。貞元三年,復為興平軍節度使,歷昭德軍、臨洮、鳳翔尹。



 大定二年,入為工部尚書,與蘇保衡、完顏餘里也遷加伐宋士官賞。宋永性滯不習事,凡與土賊戰者一概加之。世宗久乃知之,謂宰相曰:「若一概追還,必生怨望。若因循不問,則爵賞濫矣。其與土賊戰者,有能以寡敵眾,一人敵三十人以上者,依已遷為定。」改同簽大宗正事、震武軍節度使,卒。



 斡者,天會十五年大封宗室,追封魯王,正隆例改封公。子神土懣,驃騎衛上將軍。



 子璋本名胡麻愈,多勇略,通女直、契丹、漢字。年十八,左副元帥撒離喝引在麾下。以事如京師,見梁王宗弼與語,宗弼悅之。皇統六年,父神土懣卒,宗弼奏璋可襲謀克,詔從之。天德三年,充牌印祗候,以罪免,奪其謀克,寓居中都。



 海陵伐宋,左衛將軍蒲察沙離只同知中都留守,佩金牌掌留府事。世宗即位于遼陽,璋勸沙離只歸世宗,沙離只不從。璋與守城軍官烏林荅石家奴、烏林荅愿、徒單三勝、蒲察蒲查等以兵晨入留守府,遂殺沙離只及判官漫捻撒離喝,推
 宗強子阿瑣為留守,璋行同知留守事。遣石家奴佩沙離只金牌與愿、蒲查、中都轉運使左淵子貽慶、大興少尹李天吉子磐奉表如東京,賀即位。世宗嘉之,以願、蒲查為武義將軍,充護衛。貽慶賜及弟,授從仕郎。磐充閤門祗候。就以璋為同知中都事。



 璋以殺沙離只自攝同知留守,世宗因而授之,心常不自安,遂與兵部尚書可喜謀,因世宗謁山陵作亂。大定二年,上謁山陵,璋等九人會于可喜家,說萬戶高松,不從。璋知事不成,乃與可喜共執斡論詣有司陳,上誅可喜、李惟忠等,以璋為彰化軍節度使。



 宋將吳璘出散關,據寶雞以西,詔璋赴元
 帥都監徒單合喜軍前任使。於是,宋人據原州,寧州刺史顏盞門都以兵四千攻之,不克。宋將姚良輔以兵十萬至原州,權副統完顏習尼列以千騎援門都兵,而姚良輔兵多,諸將皆不敢與戰。及璋至軍,會平涼、涇州、潘原、長武等戍兵,合二萬人。璋使押軍猛安石抹許里阿補以兵二千軍於城北,習尼列以兵三千軍於城西北十里麥子原,皆據高阜為陣。璋以本部兵陣於城西。姚良輔出自北嶺,先遣萬人攻許里阿補,自以軍九萬陣麥子原下,捍以劍盾、行馬,外列騎士,步卒居其中,敢死士鎖足行馬間,持大刀為拒,分為八陣,而別以騎二千
 襲璋軍。璋方出迎戰,習尼列來報曰:「宋之重兵皆在麥子原矣。」璋遣萬戶特里失烏也以押軍猛安奚慶喜、照撒兵二千援許里阿補,遣撒屋出、崔尹以兵二千益習尼列。許里阿補與宋人接戰,良久,敗之。宋兵在麥子原者最堅,習尼列與移刺補、奧屯撒屋出、崔尹、僕根撒屈出以兵五千沿壕為狀,餘兵皆捨馬步戰,擊其前行騎士,走之。於是,行馬以前衝以長槍,行馬以後射以勁弓。良輔兵稍挫,習尼列乘勝麾兵,撤其行馬,破其七陣。良輔復整兵出,習尼列少卻,而璋已破城下宋兵,與習尼列會。使僕根以伏兵擊良輔。習尼列亦整兵與戰,奮擊
 之,大破良輔軍,斬首萬餘級,墜壕死者不可勝數,鎖足行馬者盡殪之,獲甲二萬餘,器仗稱是。良輔亦中兩創脫去。遂圍原州,穴其西城,城圮,宋人宵遁。璋等入原州。宋戍軍在寶雞以西,聞之皆自散關遁去。



 京兆尹烏延蒲離黑、丹州刺史赤盞胡速魯改已去德順州,宋吳璘復據之,都監合喜以璋權都統,與習尼列將兵二萬救德順。璋率騎兵前行,與璘騎拴二萬戰於張義堡遂沙山下,敗之,追北四十餘里。璘軍遇隘不得前,斬首數十級。璋至德順,璘據城北險要為營,璋亦策營與璘相望,可三里許。兩軍遇於城東,凡五接戰,璘軍敗走,璋追至
 城下。璘軍已據城北岡阜,與其城上兵相應,以弩夾射璋軍。璋軍陽卻,城中出兵來追,璋反施與戰,大敗之。合喜遣統軍都監泥河以兵七千來會,與璘軍復戰,敗之。璘遣兵據東山堡,欲樹柵,璋與習尼列、泥河議曰:「敵若據東山堡,此城亦不可拔,宜急擊之。」於是璋先據要地,習尼列以兵逼東山堡,璘兵恃濠相拒,短兵接,璘兵退走,習尼列追擊之。璘城北營兵可六千人,登北岡來戰,璋之漢軍少卻,傷者二百人。璘遂焚璋軍攻城具,璋率移刺補猛安兵踰北岡擊走之。璘軍隔小塹射璋軍,移刺補少卻,習尼列望見北原火發,乃止攻東山堡,亟與
 將士來赴,引善射者先登,率劉安漢軍三百人擊敗之。璘軍皆走險,璘以軍三萬據險作三陣,皆環以劍盾、行馬。璋遣萬戶石抹迭勒由別路自後擊之,特里失烏也、移刺補以二千人當其前,以強弓射之,璘兵大敗,墮溝壑者甚眾。璋軍度澗追之,斬數千級而還。



 璘軍雖敗,猶恃其眾,都監合喜使武威軍副總管夾古查刺來問策。諸將皆曰:「吳璘恃險,不善野戰,我退軍平涼,彼必棄險就平地,然後可圖也。」璋曰:「不然。彼恃其眾,非特恃險也。昔人有言,『寧棄千軍,不棄寸地』,故退兵不如濟師。我退軍平涼,彼軍深入吾地,固壘以拒我,則如之何。」查刺還
 報,合喜於是親率四萬人赴之。吳璘詰旦乘陰霧晦冥分兵四道來襲,戰于城東,離而復合者數四。漢軍千戶李展率麾下兵先登奮擊之,璘軍陣動。璋乘勝踵擊,璘軍復敗,追至北岡,璘走險,璋急擊之,殺略殆盡。璘分半軍守秦州,合喜駐軍水洛城東,自六盤山至石山頭分兵守之,斷其餉道。璘乃引歸。



 宋經略使荊皋以步騎三萬自德順西去,璋以兵八千、習尼列以兵五千追擊之。習尼列兵乃出其前,還自赤觜,遇其前鋒,敗之于高赤崖下。復與其中軍戰,自日昃至暮,乃罷。荊皋乘夜來襲營,為退軍八十里。明日,習尼列追之。璋兵至上八節,宋
 兵據險為陣,璋捨馬步戰,地險不得接,相拒至曙。宋兵勸,璋乘之,追至甘谷城,習尼列兵亦至,宋兵宵遁,璋遂班師。習尼列追至伏羌城,不及而還。



 上使御史中丞達吉視諸軍功狀,達吉舊與璋有隙,故損其功。詔璋將士賞比諸軍半之,璋兼陜西路都統,進官一階。及元帥府上功,璋居多,詔達吉削官兩階,杖八十,解職。上復賞璋及將士如諸軍,以璋為西北路招討使。召為元帥左都監,兼安化軍節度使,賜以弓矢衣帶佩刀。改益都尹,左都監如故。



 宋人棄海州遁去,焚官民廬舍且盡。璋至海州,得所棄糧三萬六千餘石,安集其人,復其屯戍。五年,
 宋人約和,罷三路都統,復置陜西路統軍司,璋為統軍使。上曰:「監軍合喜年老,故授卿此職。邊境無事,且召卿矣。」以本官兼京兆尹。



 召為御史大夫。璋奏:「竊觀文武百官有相為朋黨者,今在臺自臣外無女直人,乞不限資考,量材奏擬。」上曰:「朋黨為誰,即糾治之。朕選女直人,未得其人,豈以資考為限,論其人材而已。」頃之,璋奏曰:「太祖武元皇帝受天明命,太宗皇帝奄定宋土,自古帝王之興,必稱受命,當製『大金受命之寶』,以明示萬世。」上曰:「卿言正合朕意。」乃遣使夏國市玉,十八年,受命寶成,奏告天地宗廟社稷,上御正殿。



 十三年,改大興尹,為賀宋正旦使。
 璋受命使宋,即行,上遣人馳諭璋曰:「宋人若不遵舊禮,慎勿付書。如不令卿等入見,即持書歸。若迫而取之,亦勿赴宴,其回書及禮物一切勿受。」璋至臨安,宋人請以太子接書,不從。宋人就館迫取書,璋與之,且赴宴,多受禮物。有司以聞,上怒,欲置之極刑。左丞相良弼奏曰:「璋為將,大破宋軍,宋人仇之久矣。將因此陷之死地,未可知也。今若殺璋,或者墮其計中耳。」上以為然,乃杖璋百五十,除名,副使客省使高翊杖百,沒入其所受禮物。



 後歲餘,上念璋有征伐功,起為景州刺史,遷武定軍節度使,授山東西路蒲底山拏兀魯河謀克,改臨洮
 尹。十九年,卒。



 鄆王昂,本名吾都補,世祖最幼子也。常從太祖征伐。天輔六年,昂與稍喝以兵四千監護諸部降人,處之嶺東,就以兵守臨潢府。昂不能撫御,降人苦之,多叛亡者。上聞之,使出里底戒諭昂。已過上京,諸部皆叛去,惟章愍宮、小室韋二部達內地。詔諳版勃極烈吳乞買曰:「比遣昂徙諸部,多致怨叛,稍喝駐兵不與討襲,致使降人復歸遼主,違命失眾,當置重法。若有所疑,則禁錮之,俟師還定議。」是時,太宗居守,辭不失副之,辭不失勸太宗因國慶可薄其罰,於是杖昂七十,拘之泰州,而殺稍喝。



 天
 會六年,權元帥左都監。十五年,為西京留守。天眷三年,為平章政事。皇統元年,封漆水郡王。二年,制詔昂暑銜帶「皇叔祖」字,封鄆王。是歲,薨。



 子鄭家、鶴壽。鶴壽累官耶魯瓦群牧使,死于契丹撒八之難,語在《忠義傳》。



 鄭家,皇統初,以宗室子授定遠大將軍,除磁州刺史。天德間,為右諫議大夫,累遷會寧尹、安化軍節度使,改益都尹。海陵伐宋,為浙東道副統制,與工部尚書蘇保衡以舟師自海道趨臨安,至松林島阻風,泊島間。詰旦,舟人望見敵舟,請為備。鄭家問:「去此幾何?」舟人曰:「以水路測之,且三百里。風迅,行即至矣。」鄭家不曉海路舟楫,不
 之信。有頃,敵果至,見我軍無備,即以火砲擲之。鄭家顧見左右舟中皆火發,度不得脫,赴水死,時年四十一。



\end{pinyinscope}