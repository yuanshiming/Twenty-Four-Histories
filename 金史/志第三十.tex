\article{志第三十}

\begin{pinyinscope}

 食貨四



 ○鹽



 金制,榷貨之目有十,曰酒、曲、茶、醋、香、礬、丹、錫、鐵,而鹽為稱首。貞元初,蔡松年為戶部尚書,始復鈔引法,設官置庫以造鈔、引。鈔,合鹽司簿之符。引,會司縣批繳之數。七年一釐革之。初,遼、金故地濱海多產鹽,上京、東北二路食肇州鹽,速頻路食海鹽,臨潢之北有大鹽濼,烏古里石壘部有鹽池,皆足以食境內之民,嘗徵其稅。及得
 中土,鹽場倍之,故設官立法加詳焉。然而增減不一,廢置無恒,亦隨時救弊而已。益都、濱州舊置兩鹽司,大定十三年四月,併為山東鹽司。二十一年滄州及山東各務增羨,冒禁鬻鹽,朝論慮其久或隳法,遂併為海豐鹽使司。十一月,又併遼東等路諸鹽場,為兩鹽司。大定二十五年,更狗濼為西京鹽司。是後惟置山東、滄、寶坻、莒、解、北京、西京七鹽司。



 山東、滄、寶坻斤三百為袋,袋二十有五為大套,鈔、引、公據三者俱備然後聽鬻。小套袋十,或五、或一,每套鈔一,引如袋之數。寶坻零鹽較其斤數,或六之三,或六之一,又為小鈔引給之,以便其鬻。解鹽
 斤二百有五十為一席,席五為套,鈔引則與陜西轉運司同鬻,其輸粟於陜西軍營者,許以公牒易鈔引。西京等場鹽以石計,大套之石五,小套之石三。北京大套之石四,小套之石一。遼東大套之石十,皆套一鈔,石一引。零鹽積十石,亦一鈔而十引。



 其行鹽之界,各視其地宜。山東、滄州之場九,行山東、河北、大名、河南、南京、歸德諸府路,及許、亳、陳、蔡、潁、宿、泗、曹、睢、鈞、單、壽諸州。莒之場十二,濤洛場行莒州,臨洪場行贛榆縣,獨木場行海州司候司、朐山、東海縣,板浦場行漣水、沐陽縣,信陽場行密州,之五場又與大鹽場通行沂、邳、徐、宿、泗、滕六州。西由
 場行萊州錄事司及招遠縣,衡村場行既墨、萊陽縣,之二場鈔引及半袋小鈔引,聽本州縣鬻之。寧海州五場皆鬻零鹽,不用引目。黃縣場行黃縣,巨風場行登州司候司、蓬萊縣,福山場行福山縣,是三場又通行旁縣棲霞。寧海州場行司候司、牟平縣,文登場行文登縣。寶坻鹽行中都路,平州副使於馬城縣置局貯錢。解鹽行河東南北路,陜西東、及南京河南府、陜、鄭、唐、鄧、嵩、汝諸州。西京、遼東鹽各行其地。北京宗、錦之末鹽,行本路及臨潢府、肇州、泰州之境,與接壤者亦預焉。



 世宗大定三年二月,定軍私煮鹽及盜官鹽之法,命猛安謀克巡捕。三
 年十一月,詔以銀牌給益都、濱、滄鹽使司。十一年正月,用西京鹽判宋俁言,更定狗濼鹽場作六品使司,以俁為使,順聖縣令白仲通為副,以是歲入錢為定額。四月,以烏古里石壘民饑,罷其鹽池稅。十二年十月,詔西北路招討司猛安所轄貧及富人奴婢,皆給食鹽。宰臣言:「去鹽濼遠者,所得不償道里之費。」遂命計口給直,富家奴婢二十口止。



 十三年二月,併榷永鹽為寶坻使司,罷平、灤鹽錢。滄州舊廢海阜鹽場,三月,州人李格請復置,詔遣使相視。有司謂:「是場興則損滄鹽之課,且食鹽戶仍舊,而鹽貨歲增,必徒多積而不能售。」遂寢其議。三月,
 大鹽濼設鹽稅官。復免烏古里石壘部鹽池之稅。二十一年八月,參知政事梁肅言:「寶坻及傍縣多闕食,可減鹽價增粟價,而以粟易鹽。」上命宰臣議,皆謂:「鹽非多食之物,若減價易粟,恐久而不售,以至虧課。今歲糧以七十餘萬石至通州,比又以恩、獻等六州粟百餘萬石繼至,足以賑之,不煩易也。」遂罷。十二月,罷平州椿配鹽課。二十三年七月,博興縣民李孜收日炙鹽,大理寺具私鹽及刮鹹土二法以上。宰臣謂非私鹽可比,張仲愈獨曰:「私鹽罪重,而犯者猶眾,不可縱也。」上曰:「刮鹼非煎,何以同私?」仲愈曰:「如此則渤海之人恣刮鹼而食,將侵官
 課矣。」力言不已,上乃以孜同刮鹼科罪。後犯則同私鹽法論。



 十一月,張邦基言:「寶坻鹽課,若每石收正課百五十斤,慮有風幹折耗。」遂令石加耗鹽二十二斤半,仍先一歲貸支償直,以優灶戶。



 二十四年七月,上在上京,謂丞相烏古論元忠等曰:「會寧尹蒲察通言,其地猛安謀克戶甚艱。舊速頻以東食海鹽。蒲與、胡里改等路食肇州鹽,初定額萬貫,今增至二萬七千。若罷鹽引,添灶戶,庶可易得。」元忠對曰:「已嘗遣使咸平府以東規畫矣。」上曰:「不須待此,宜亟為之。」通又言:「可罷上京酒務,聽民自造以輸稅。」上曰:「先灤州諸地亦嘗令民煮鹽,後以不便罷
 之,今豈可令民自沽耶?」二十五年十月,上還自上京,謂宰臣曰:「朕聞遼東,凡人家食鹽,但無引目者,既以私治罪。夫細民徐買食之,何由有引目。可止令散辦,或詢諸民,從其所欲。」因為之罷北京、遼東鹽使司。二十八年,尚書省論鹽事,上曰:「鹽使司雖辦官課,然素擾民。鹽官每出巡,而巡捕人往往私懷官鹽,所至求賄及酒食,稍不如意則以所懷誣以為私鹽。鹽司茍圖羨增,雖知其誣亦復加刑。宜令別設巡捕官,勿與鹽司關涉,庶革其弊。」五月,創巡捕使,山東、滄、寶坻各二員,解、西京各一員。山東則置於濰州、招遠縣,滄置於深州及寧津縣,寶坻
 置於易州及永濟縣,解置於澄城縣,西京置於兜答館,秩從六品,直隸省部,各給銀牌,取鹽使司弓手充巡捕人,且禁不得於人家搜索,若食鹽一斗以下不得究治,惟盜販煮則捕之,在三百里內者屬轉運司,外者即隨路府提點所治罪,盜課鹽者亦如之。



 章宗大定二十九年十月,上朝隆慶宮,諭有司曰:「比因獵,知百姓多有鹽禁獲罪者,民何以堪?朕欲令依平、灤、太原均辦例,令民自煎,其令百官議之。」十二月,戶部尚書鄭儼等謂:「若令民計口定課,民既輸乾辦錢,又必別市而食,是重費民財,而徒增煎販者之利也。且今之鹽價,蓋昔日錢幣
 易得之時所定,今日與向不同,況太平日久,戶口蕃息,食鹽歲課宜有羨增,而反無之,何哉?緣官估高,貧民利私鹽之賤,致虧官課爾。近已減寶坻、山東、滄鹽價斤為三十八文,乞更減去八文,歲不過減一百二十餘萬貫,官價既賤,所售必多,自有羨餘,亦不全失所減之數。況今府庫金銀約折錢萬萬貫有奇,設使鹽課不足,亦足補百有餘年之經用,若量入為出,必無不足之患。乞令平、灤乾辦鹽課亦宜減價,各路巡鹽弓手不得自專巡捕,庶革誣罔之弊。」禮部尚書李晏等曰:「所謂乾辦者,既非美名,又非良法。必欲杜絕私煮盜販之弊,莫若每斤
 減為二十五文,使公私價同,則私將自己。又巡鹽兵吏往往挾私鹽以誣人,可令與所屬司縣期會,方許巡捕,違者按察司罪之。」刑部尚書郭邦傑等則謂:「平、灤瀕海及太原鹵地可依舊幹辦,餘同儼議。」御史中丞移剌仲方則謂:「私煎盜販之徒,皆知禁而犯之者也。可選能吏充巡捕使,而不得入人家搜索。」同知大興府事王翛請每斤減為二十文,罷巡鹽官。左諫議大夫徒單鎰則以乾辦為便。宰臣奏:「以每斤官本十文,若減作二十五文,似為得中。巡鹽弓手可減三分之一,鹽官出巡須約所屬同往,不同獲者不坐。可自來歲五月一日行之。」上遂
 命寶坻、山東、滄鹽每斤減為三十文,已發鈔引未支者准新價足之,餘從所請。



 十二月,遂罷西京、解鹽巡捕使。時既詔罷干辦鹽錢,十二月以大理司直移剌九勝奴、廣寧推官宋扆議北京、遼東鹽司利病,遂復置北京、遼東鹽使司,北京路歲以十萬餘貫為額,遼東路以十三萬為額。罷西京及解州巡捕使。



 明昌元年七月,上封事者言河東北路乾辦鹽錢歲十萬貫太重,以故民多逃徙,乞緩其徵督。上命俟農隙遣使察之。十二月,定禁司縣擅科鹽制。二年五月,省臣以山東鹽課不足,蓋由鹽司官出巡不敢擅捕,必約所屬同往,人不畏故也。遂詔,
 自今如有盜販者,聽鹽司官輒捕。民私煮及藏匿,則約所屬搜索。巡尉弓兵非與鹽司相約,則不得擅入人家。三年六月,孫即康等同鹽司官議:「軍民犯私鹽,三百里內者鹽司按罪,遠者付提點所,皆徵捕獲之賞於販造者。猛安謀克部人煎販及盜者,所管官論贖,三犯杖之,能捕獲則免罪。又濱州渤海縣永和鎮去州遠,恐藏盜及私鹽,可改為永豐鎮與曹子山村,各創設巡檢,山東、寶坻、滄鹽司判官乞升為從七品,用進士。」上命猛安謀克杖者再議,餘皆從之。尚書省奏:「山東濱、益九場之鹽行於山東等六路,濤洛等五場止行於沂、邳、徐、宿、滕、泗
 六州,各有定課,方之九場,大課不同。若令與九場通比增虧。其五場官恃彼大課,恐不用力,轉生姦弊。」遂定令五場自為通比。舊法與鹽司使副通比,故至是始改焉。



 五年正月,八小場鹽官左蓽等,以課不能及額,繳進告敕。遂遣使按視十三場再定,除濤洛等五場係設管勾,可即日恢辦,乃以蓽所告八場,從大定二十六年制,自見管課,依新例永相比磨。戶部郎中李敬義等言:「八小場今新定課有減其半者,如使俱從新課,而舊課已辦入官,恐所減錢多,因而作弊,而所收錢數不復盡實附歷納官。」遂從明昌元年所定酒稅院務制,令即日收辦。



 十一月,以舊制猛安謀克犯私鹽酒曲者,轉運司按罪,遂更定軍民犯私鹽者皆令屬鹽司,私酒曲則屬轉運司,三百里外者則付提點所,若逮問犯人而所屬吝不遣者徒二年。



 十二月,尚書省議山東、滄州舊法每一斤錢四十一文,寶坻每一斤四十三文,自大定二十九年赦恩并特旨,減為三十文,計減百八十五萬四千餘貫。後以國用不充,遂奏定每一斤復加三文為三十三文。至承安三年十二月,尚書省奏:「鹽利至大,今天下戶口蕃息,食者倍於前,軍儲支引者亦甚多,況日用不可闕之物,豈以價之低昂而有多寡也。若不隨時取利,恐徒
 失之。」遂復定山東、寶坻、滄州三鹽司價每一斤加為四十二文。解州舊法每席五貫文,增為六貫四百文。遼東、北京舊法每石九百文,增為一貫五百文。西京煎鹽舊石二貫文,增為二貫八百文,撈鹽舊一貫五百文,增為二貫文,既增其價,復加其所鬻之數。七鹽司舊課歲入六百二十二萬六千六百三十六貫五百六十六文,至是增為一千七十七萬四千五百一十二貫一百三十七文二分。山東舊課歲入二百五十四萬七千三百三十六貫,增為四百三十三萬四千一百八十四貫四百文。滄州舊課歲入百五十三萬一千二百貫,增為二百
 七十六萬六千六百三十六貫。寶坻舊入八十八萬七千五百五十八貫六百文,增為一百三十四萬八千八百三十九貫。解州舊入八十一萬四千六百五十七貫五百文,增為一百三十二萬一千五百二十貫二百五十六文。遼東舊入十三萬一千五百七十二貫八百七十文,增為三十七萬六千九百七十貫二百五十六文。北京舊入二十一萬三千八百九十二貫五百文,增為三十四萬六千一百五十一貫六百一十七文二分。西京舊入十萬四百一十九貫六百九十六文,增為二十八萬二百六十四貫六百八文。



 四月,宰臣奏:「在法,猛安
 謀克有告私鹽而不捕者杖之,其部人有犯而失察者,以數多寡論罪。今乃有身犯之者,與犯私酒曲、殺牛者,皆世襲權貴之家,不可不禁。」遂定制徒年、杖數,不以贖論,不及徒者杖五十。



 八月,命山東、寶坻、滄州三鹽司,每春秋遣使督按察司及州縣巡察私鹽。



 泰和元年九月,省臣以滄、濱兩司鹽袋,歲買席百二十萬,皆取於民。清州北靖海縣新置滄鹽場,本故獵地,沮洳多蘆,宜弛其禁,令民時採而織之。



 十一月,陜西路轉運使高汝礪言:「舊制,捕告私鹽酒曲者,計斤給賞錢,皆徵于犯人。然鹽官獲之則充正課,巡捕官則不賞。巡捕軍則減常人之
 半,免役弓手又半之,是罪同而賞異也。乞以司縣巡捕官不賞之數,及巡捕弓手所減者,皆徵以入官,則罪賞均矣。」詔從之。三年二月,以解鹽司使治本州,以副治安邑。十一月,定進士授鹽使司官,以榜次及入仕先後擬注。



 四年六月,以七鹽使司課額七年一定為制,每斤增為四十四文,時桓州刺史張煒乞以鹽易米,詔省臣議之。



 六月,詔以山東、滄州鹽司自增新課之後,所虧歲積,蓋官既不為經畫,而管勾、監同與合干人互為姦弊,以致然也。即選才幹者代兩司使副,以進士及部令史、譯人、書史、譯史、律科、經童、諸局分出身之廉慎者為管勾,
 而罷其舊官。



 十月,西北路有犯花鹼禁者,欲同鹽禁罪,宰臣謂:「若比私鹽,則有不同。」詔定制,收鹼者杖八十,十斤加一等,罪止徒一年,賞同私礬例。五年六月,以山東、滄州兩鹽司侵課,遣戶部員外郎石鉉按視之,還言令兩司分辦為便。詔以周昂分河北東西路、大名府、恩州、南京、睢、陳、蔡、許、潁州隸滄鹽司,以山東東西路、開、濮州、歸德府、曹、單、亳、壽、泗州隸山東鹽司,各計口承課。十月,簽河北東西大名路按察司事張德輝言:「海壖人易得私鹽,故犯法者眾,可量戶口均配之。」尚書省命山東按察司議其利便,言:「萊、密等州比年不登,計口賣鹽所斂
 雖微,人以為重,恐致流亡。且私煮者皆無籍之人,豈以配買而不為哉!」遂定制,命與滄鹽司皆馳驛巡察境內。



 六年三月,右丞相內族宗浩、參知政事賈鉉言:「國家經費惟賴鹽課,今山東虧五十餘萬貫,蓋以私煮盜販者成黨,鹽司既不能捕,統軍司、按察司亦不為禁,若止論犯私鹽者之數,罰俸降職,彼將抑而不申,愈難制矣!宜立制,以各官在職時所增虧之實,令鹽司以達省部,以為升降。」遂詔諸統軍、招討司,京府州軍官,所部有犯者,兩次則奪半月俸,一歲五次則奏裁,巡捕官但犯則的決,令按察司御史察之。



 四月,從涿州剌史夾谷蒲乃言,
 以萊州民所納鹽錢聽輸絲綿銀鈔。七年九月,定西北京、遼東鹽使判官及諸場管勾,增虧升降格,凡文資官吏員,諸局署承應人、應驗資歷注者,增不及分者升本等首,一分減一資,二分減兩資,遷一官,四分減兩資,遷兩官,虧則視此為降。如任迴驗官注擬者,增不及分升本等首,一分減一資,二分減一資、遷一階,四分減兩資、遷兩階,虧者亦視此為降。



 十二月,尚書省以盧附翼所言,遂定制灶戶盜賣課鹽法,若應納鹽課外有餘,則盡以申官,若留者減盜一等。若刮鹼土煎食之,採黃穗草燒灰淋鹵,及以酵粥為酒者,杖八十。八年七月,宋克俊
 言:「鹽管勾自改注進士諸科人,而監官有失超陞縣令之階,以故怠而虧課,乞依舊為便。」有司以泰和四年改注時,選當時到部人截替,遂擬以秋季到部人注代。八年七月,詔沿淮諸榷場,聽官民以鹽市易。



 宣宗貞祐二年十月,戶部言:「陽武、延津、原武、滎澤、河陰諸縣饒鹼鹵,民私煎不能禁。」遂詔置場,設判官、管勾各一員,隸戶部。既而,御史臺奏:「諸縣皆為有力者奪之,而商販不行。」遂敕御史分行申明禁約。三年十二月,河東南路權宣撫副使烏古論慶壽言:「絳、解民多業販鹽,由大陽關以易陜、虢之粟,及還渡河,而官邀糴其八,其旅費之外所存
 幾何?而河南行部復自運以易粟於陜,以盡奪民利。比歲河東旱蝗,加以邀糴,物價踴貴,人民流亡,誠可閔也。乞罷邀糴,以紓其患。」四年七月,慶壽又言:「河中乏糧,既不能濟,而又邀糴以奪之。夫鹽乃官物,有司陸運至河,復以舟達京兆、鳳翔,以與商人貿易,艱得而甚勞。而陜西行部每石復邀糴二斗,是官物而自糴也。夫鹽乃官物,有司陸運至河,復以舟達京兆、鳳翔,以與商人貿易,艱得而甚勞。而陜西行部每石復邀糴二斗,是官物而自糴也。夫轉鹽易物,本濟河中,而陜西復強取之,非奪而何?乞彼此壹聽民便,則公私皆濟。」上從之。興定二年六月,以延安行六部員外郎盧進建言:「綏德之嗣武城、義合、克戎寨近河地多產鹽,請設鹽場管勾一員,歲獲十三萬餘斤,可輸
 錢二萬貫以佐軍。」三年,詔用其言,設官鬻鹽給邊用。四年,李復享言:「以河中西岸解鹽舊所易粟麥萬七千石充關東之用。」尋命解鹽不得通陜西,以北方有警,河禁方急也。元光二年內族訛可言,民運解鹽有助軍食,詔修石墻以固之。



 ○酒



 金榷酤因遼、宋舊制,天會三年始命榷官以周歲為滿。世宗大定三年,詔宗室私釀者,從轉運司鞫治。三年,省奏中都酒戶多逃,以故課額愈虧。上曰:「此官不嚴禁私釀所致也。」命設軍百人,隸兵馬司,同酒使副合千人巡察,雖權要家亦許搜索。奴婢犯禁,杖其主百。且令大
 興少尹招復酒戶。八年,更定酒使司課及五萬貫以上,鹽場不及五萬貫者,依舊例通注文武官,餘並右職有才能,累差不虧者為之。九年,大興縣官以廣陽鎮務虧課,而懼奪其俸,乃以酒散部民,使輸其稅。大理寺以財非入己,請以贖論。上曰:「雖非私贓,而貧民亦被其害,若止從贖,何以懲後。」特命解職。二十六年,省奏鹽鐵酒曲自定課後,增各有差。上曰:「朕頃在上京,酒味不嘉。朕欲如中都曲院取課,庶使民得美酒。朕日膳亦減省,嘗有一公主至,而無餘膳可與。朕欲日用五十羊何難哉!慮費用皆出於民,不忍為也。監臨官惟知利己,不知利何
 從來?若恢辦增羨者酬遷,虧者懲殿,仍更定併增併虧之課,無失元額。如橫班只虧者,與餘差一例降罰,庶有激勸。且如功酬合辦二萬貫,而止得萬七八千,難迭兩酬者,必止納萬貫,而輒以餘錢入己。今後可令見差使內不迭酬餘錢,與後差使內所增錢通算為酬,庶錢可入官。及監官食直,若不先與,何以責廉。今後及格限而至者,即用此法。」又奏罷杓欄人。二十七年,議以天下院務,依中都例,改收曲課,而聽民酤。戶部遣官詢問遼東來遠軍,南京路新息、虞城,西京路西京酒使司、白登縣、迭剌部族、天成縣七處,除稅課外,願自承課賣酒。上曰:「
 自昔監官多私官錢,若令百姓承辦,庶革此弊。其試行之。」



 明昌元年正月,更定新課,令即日收辦。中都曲使司,大定間,歲獲錢三十六萬一千五百貫,承安元年歲獲四十萬五千一百三十三貫。西京酒使司,大定間,歲獲錢五萬三千四百六十七貫五百八十八文,承安元年歲獲錢十萬七千八百九十三貫。七月,定中都曲使司以大定二十一年至明昌六年為界,通比均取一年之數為額。五年四月,省奏:「舊隨處酒稅務,所設杓欄人,以射糧軍歷過隨朝差役者充,大定二十六年罷去,其隨朝應役軍入,各給添支錢粟酬其勞。今擬將元收杓欄
 錢,以代添支,令各院務驗所收之數,百分中取三,隨課代輸,更不入比,歲約得錢三十餘萬,以佐國用。」泰和四年九月,省奏:「在都曲使司,自定課以來八年併增,宜依舊法,以八年通該課程,均其一年之數,仍取新增諸物一分稅錢併入,通為課額。以後之課,每五年一定其制。」又令隨處酒務,元額上通取三分作糟酵錢。六年,制院務賣酒數各有差,若數外賣、及將帶過數者,罪之。宣宗貞祐三年十二月,御史田迥秀言:「大定中,酒稅歲及十萬貫者,始設使司,其後二萬貫亦設,今河南使司亦五十餘員,虛費月廩,宜依大定之制。」元光元年,復設曲使
 司。



 ○醋稅



 自大定初,以國用不足,設官榷之,以助經用。至二十三年,以府庫充牣,遂罷之。章宗明昌五年,以有司所入不允所出,言事者請榷醋息,遂令設官榷之,其課額,俟當差官定之。後罷。承安三年三月,省臣以國用浩大,遂復榷之。五百貫以上設都監,千貫以上設同監一員。



 ○茶



 自宋人歲供之外,皆貿易於宋界之榷場。世宗大定十六年,以多私販,乃更定香茶罪賞格。章宗承安三年八月,以謂費國用而資敵,遂命設官製之。以尚書省令史承德郎劉成往河南視官造者,以不親嘗其味,但採
 民言謂為溫桑,實非茶也,還即白上。上以為不幹,杖七十,罷之。四年三月,於淄、密、寧海、蔡州各置一坊,造新茶,依南方例每斤為袋,直六百文。以商旅卒未販運,命山東、河北四路轉運司以各路戶口均其袋數,付各司縣鬻之。買引者,納錢及折物,各從其便。



 五月,以山東人戶造賣私茶,侵侔榷貨,遂定比煎私礬例,罪徒二年。



 泰和四年,上謂宰臣曰:「朕賞新茶,味雖不嘉,亦豈不可食也。比令近侍察之,乃知山東、河北四路悉椿配於人。既曰強民,宜抵以罪。此舉未知運司與縣官孰為之,所屬按察司亦當坐罪也。其閱實以聞。自今其令每袋價減三
 百文,至來年四月不售,雖腐敗無傷也。」五年春,罷造茶之坊。三月,上諭省臣曰:「今雖不造茶,其勿伐其樹,其地則恣民耕樵。」六年,河南茶樹槁者,命補植之。十一月,尚書省奏:「茶,飲食之餘,非必用之物。比歲下上競啜,農民尤甚,市井茶肆相屬。商旅多以絲絹易茶,歲費不下百萬,是以有用之物而易無用之物也。若不禁,恐耗財彌甚。」遂命七品以上官,其家方許食茶,仍不得賣及饋獻。不應留者,以斤兩立罪賞。七年,更定食茶制。八年七月,言事者以茶乃宋土草芽,而易中國絲錦錦絹有益之物,不可也。國家之鹽貨出於鹵水,歲取不竭,可令易茶。
 省臣以謂所易不廣,遂奏令兼以雜物博易。宣宗元光二年三月,省臣以國蹙財竭,奏曰:「金幣錢穀,世不可一日闕者也。茶本出於宋地,非飲食之急,而自昔商賈以金帛易之,是徒耗也。泰和間,嘗禁止之,後以宋人求和,乃罷。兵興以來,復舉行之,然犯者不少衰,而邊民又窺利,越境私易,恐因泄軍情,或盜賊入境。今河南、陜西凡五十餘郡,郡日食茶率二十袋,袋直銀二兩,是一歲之中妄費民銀三十餘萬也。奈何以吾有用之貨而資敵乎?」乃制親王,公主及見任五品以上官,素蓄者存之,禁不得賣、饋,餘人並禁之。犯者徒五年,告者賞寶泉一萬
 貫。



 ○諸征商



 海陵貞元元年五月,以都城隙地賜隨朝大小職官及護駕軍,七月,各徵錢有差。大定二年,制院務創虧及功酬格。八月,罷諸路關稅,止令譏察。三年,尚書省奏:「山東西路轉運司言,坊場河渡多逋欠。」詔如監臨制,以年歲遠近為差,蠲減。又以尚書工部令史劉行義言,定城郭出賃房稅之制。五年,以前此河濼罷設官,復召民射買,兩界之後,仍舊設官。二十年正月,定商稅法,金銀百分取一,諸物百分取三。章宗大定二十九年,戶部言天下河泊已許與民同利,其七處設官可罷之,委所
 屬禁豪強毋得擅其利。



 明昌元年正月,敕尚書省,定院務課商稅額,諸路使司院務千六百一十六外,比舊減九十四萬一千餘貫,遂罷坊場,免賃房稅。十月,尚書省奏:「今天下使司務,既減課額,而監官增虧既有升遷追殿之制,宜罷提點所給賞罰俸之制,但委提刑司,察提點官侵犯場務者,則論如制。」詔從之。二年,詔減南京出賃官房及地基錢。三年,諭提刑司,禁勢力家不得固山澤之利。又司竹監歲採入破竹五十萬竿,春秋兩次輸都水監,備河防,餘邊刀筍皮等賣錢三千貫,葦錢二千貫,為額。明昌五年,陳言者乞復舊置坊場,上不許,惟
 許增置院務,詔尚書省參酌定制,遂擬遼東、北京依舊許人分辦,中都等十一路差官按視,量添設院務于二十三處,自今歲九月一日立界,制可。大定間,中都稅使司歲獲十六萬四千四百四十餘貫,承安元年,歲獲二十一萬四千五百七十九貫。泰和六年五月,制院務課虧,令運司差監榷。



 ○金銀之稅



 大定三年,制金銀坑冶許民開採,二十分取一為稅。泰和四年,言事者以金銀百分中取一,諸物取三,今物價視舊為高,除金銀則額所不能盡該,自餘金銀可並添一分。詔從之。七年三月,戶部尚書高汝礪言:「
 舊制,小商貿易諸物收錢四分,而金銀乃重細之物,多出富有之家,復止三分,是為不倫,亦乞一例收之。」省臣議以為如此恐多匿隱。遂止從舊。



\end{pinyinscope}