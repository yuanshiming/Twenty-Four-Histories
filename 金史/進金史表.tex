\article{進金史表}

\begin{pinyinscope}

 開府儀同三司、上柱國、錄軍國重事、中書右丞相、監修國史、領經筵事、提調太醫院廣惠司事臣阿魯圖言:



 竊惟漢高帝入關,任蕭何而收秦籍;唐太宗即祚,命魏徵以作《隋書》。蓋歷數歸真主之朝,而簡編載前代之事,國可滅史不可滅,善吾師惡亦吾師。矧夫典故之源流,章程之沿革,不披往牒,曷蓄前聞。



 維此金源,起於海裔,以滿萬之眾,橫行天下,不十年之久,專制域中。其用兵也如縱燎而乘風,其得國也若置郵而傳令。及煟興於禮樂,乃煥有乎聲明。嘗循初而汔終,因考功而論德。非武元之英略,不足以開九帝之業,非大定之仁政,不足以固百年之基。天會有吞四
 海之勢,而未有壹四海之規;明昌能成一代之制,而亦能壞一代之法。海陵無道,自取覆敗;宣宗輕動,曷濟中興。迨夫浚郊多壘之秋,汝水飛煙之日,天人屬望,久有在矣;君臣守義,蓋足取焉。



 我太祖法天啟運聖武皇帝,以有名之師,而釋奕世之愾;以無敵之仁,而收兆民之心。勁卒搗居庸關,北拊其背,大軍出紫荊口,南搤其吭。指顧可成於雋功,操縱莫窺於廟算,懲彼取遼之暴,容其涉河以遷。太宗英文皇帝席卷雲、朔,而徇地並、營,囊括趙、代,而傳檄齊、魯,滅夏國以蹴秦、鞏,通宋人以逼河、淮。睿宗仁聖景襄皇帝冒萬險,出饒風,長驅平陸;戰三峰,乘大雪,遂定中原。



 太陽出而爝火熸,正音作而眾樂廢。爰及世祖聖德神功文武皇帝,恢弘至化,勞來遺黎。燕地定都,撤武靈之舊址,遼陽建省,撫肅慎之故墟。於時張柔歸金史於其先,王鶚輯金事於其後。是以纂修之命,見諸敷遺之謀,延祐申舉而未遑,天歷推行而弗竟。



 臣阿魯圖誠惶誠懼,頓首頓首,欽惟皇帝陛下緝熙聖學,紹述先猷,當邦家間暇之時,治經史討論之務。念彼泰和以來之事跡,涉我聖代初興之歲年。太祖受帝號於丙寅,先五載而硃鳳應,世皇毓聖質於乙亥,蚤一歲而黃河清。若此貞符,昭然成命。第以變故多而舊史闕,耆艾沒而新說訛,弗折衷於大朝,恐失真於他日。於是聖心獨斷,盛事力行,申命臣阿魯圖以中書右丞相、臣別兒怯不花以中書左丞相領三史事,臣脫脫以前中書右丞相仍都總裁,臣御史大夫帖睦爾
 達世、臣中書平章政事賀惟一、臣翰林學士承旨張起巖、臣翰林學士歐陽玄、臣治書侍御史李好文、臣禮部尚書王沂、臣崇文太監楊宗瑞為總裁官,臣江西湖東道肅政廉訪使沙剌班、臣江西湖東道肅政廉訪副使王理、臣翰林待制伯顏、臣國子博士費著,臣秘書監著作郎趙時敏,臣太常博士商企翁為史官,集眾技以貴成書,儜奏篇以覽近監。臣阿魯圖仰承隆委,俯竭微勞。紬石室之文,誠乏司馬遷之作,獻《金鏡》之錄,願攄張相國之忠。謹撰述本紀十九卷、志三十九卷、表四卷、列傳七十三卷、目錄二卷,裝潢成一百三十七帙,隨表以聞,上塵天覽,無任慚愧戰汗屏營之至。



 臣阿魯圖誠惶誠懼,頓首頓首謹言。



 至正四年十一月日,開府儀同三司、上柱國、錄軍國重事、中書右丞相、監修國史、領經筵事、提調太醫院廣惠司事臣阿魯圖上表。



\end{pinyinscope}