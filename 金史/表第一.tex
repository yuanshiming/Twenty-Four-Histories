\article{表第一}

\begin{pinyinscope}

 宗室表



 古者太史掌敘邦國之世次,辨其姓氏,別其昭穆,尚失。金人初起完毅十二部,其後皆以部為氏,史臣記錄有稱「宗室」者,有稱完顏者。稱完顏者。稱完顏者亦有二焉,有同姓完顏,蓋疏族,若石土門、迪古乃是也;有異姓完顏,蓋部人,若歡都是也。大定以前稱「宗室」,明昌以後避睿宗諱稱「內族」,其實一而已,書名不書氏,其制如此。宣宗詔宗室
 皆稱完顏,不復識別焉。大定、泰和之間,袒免以上親皆有屬籍,以敘授官,大功以上,薨卒輟朝,親親之道行焉。貞祐以後,譜牒散失,大概僅存,不可殫悉,今掇其可次第者著於篇。其上無所系、下無所承者,不能盡錄也。



 表略



 右諸宗室可譜者凡十一族,雖稱系出某帝,而不能世次,不譜於各帝之下,所
 以慎也。



\end{pinyinscope}