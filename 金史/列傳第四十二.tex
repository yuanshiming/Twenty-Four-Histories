\article{列傳第四十二}

\begin{pinyinscope}

 ○納坦謀嘉鄒谷高霖孟奎烏林答與郭俁溫迪罕達王擴移剌福僧奧屯忠孝蒲察思忠紇石烈胡失門完顏宇斡勒合打蒲察移剌都



 納坦謀嘉,上京路牙塔懶猛安人。初習策論進士,大定二十六年,選入東宮,教鄆王琮、瀛王瑰讀書。以終場舉
 人試補上京提刑司書史,以廉能著稱。承安元年,契丹陀鎖寇掠韓州、信州,提刑司問諸書史誰入奏者,皆難之,謀嘉請行。五年,特賜同進士出身,調東京教授、湯池主簿、太學助教。丁母憂,服闋,累除翰林修撰,兼修起居注、監察御史。貞祐初,遷吏部員外郎、翰林待制、侍御史。完顏宇舉謀嘉才行,志在匡國,可預軍政。充元帥府經歷官。中都被圍,食且盡,胥鼎奏:「京師官民能贍足貧民者,計所贍遷官,皆先給據。」謀嘉不受據而去。中都危急,謀嘉曰:「帥臣統數萬眾,不能出城一戰,何如自縛請降邪?」宣宗議遷都,謀嘉曰:「不可。河南地狹土薄,他日宋、夏
 交侵,河北非我有矣。當選諸王分鎮遼東,河南,中都不可去也。」不聽。頃之,除唐州刺史。入為太常少卿兼左拾遺,遷鄭州防禦使。改左諭德,轉少詹事,攝御史中丞,未幾,攝太子詹事。興定元年,潼關失守,遷河南統軍使兼昌武軍節度使,攝簽樞密院事,行院許州,汰去冗食軍士二千餘人。上書諫伐宋,不聽。三年,降潁州防禦使。有告宋人將襲潁州者,已而宋兵果至,謀嘉有備,乃引去。有司上功,不及告者,謀嘉請而賞之。四年,召為翰林侍講學士兼兵部侍郎,同修國史。五年卒。



 鄒谷,字應仲,密州諸城人。中大定十三年進士第,累官
 沈王府文學。尚書省奏擬大理司直,上曰:「司直爭論情法,折正疑難,谷非所長也。」宰臣曰:「谷有吏才,陜西、河南訪察及定課皆稱職。」上以谷為同知曹州軍州事。召為刑部主事,轉北京、臨潢提刑判官,入為大理寺丞。尚書省點差接送伴宋國使官,令史周昂具數員呈請。左司都事李炳乘醉見之,怒曰:「吾口舉兩人即是,安用許為?」命左右攬昂衣欲杖之,會左司官召昂去乃已,詈諸令史為奴畜。明日語權令史李秉鈞曰:「吾豈惟箠罵,汝進退去留,亦皆在我!」群吏將陳訴,會官劾奏,事下大理寺議,差接送伴官事當奏聞,炳謂口舉兩人,當科違制。谷
 曰:「口舉兩人,一時之言,當杖贖。攬昂衣欲加杖,當決三十。」上曰:「李炳讀書人,何乃至是?」宰臣對曰:「李炳疾惡,眾人不能容耳。」上曰:「炳誠過矣,告者未必是也。」乃從谷議。歷濟南、彰德府治中,吏部郎中,河東按察副使,沂州防禦使。歷定海、泰寧軍節度使。泰和六年,致仕。貞祐初卒。



 高霖,字子約,東平人。大定二十五年進士,調符離主簿。察廉,遷泗水令,再調安國軍節度判官。以父憂還鄉里,教授生徒,恒數百人。服除,為絳陽軍節度判官。用薦舉,召為國史院編修官。建言:「黃河所以為民害者,皆以河流有曲折,適逢隘狹,故致湍決。按《水經》當疏其厄塞,行
 所無事。今若開雞爪河以殺其勢,可免數埽之勞。凡捲埽工物,皆取於民,大為時病。乞並河隄廣樹榆柳,數年之後,隄岸既固,埽材亦便,民力漸省。」朝廷從之。遷應奉翰林文字兼前職,改監察御史。丁母憂,起復太常博士。改都水監丞,簽陜西路按察司事,體訪官員能否,仍赴闕待對。時南徵調發繁急,民稍稽滯,有司皆坐失誤軍期罪。霖言其枉,悉出之。授都水少監。大安初,為耀州刺史。三年,遷河北東路按察副使,改韓王傅,兼韓林直學士。崇慶初,改工部侍郎兼直學士。至寧元年八月,霖奉儲偫迎宣宗至新城,敕霖南迎諸妃。既至,賜錢千貫,遷
 官三階。貞祐二年,除河平軍節度使兼都水監。霖請城宜村為衛州以護北門,上從之。入為兵部尚書,知大興府事,俄權參知政事,與右丞相承暉行省于中都。尋改中都留守,兼本路兵馬都總管。平章政事抹捻盡忠棄中都南奔,霖與子義傑率其徒夜出,不能進,謂義傑曰:「汝可求生,吾死於此矣。」霖死,義傑伏群屍中以免。贈翰林學士承旨,令立碑鄉里,歲時致祭,訪其子孫錄用,謚文簡。



 孟奎,字元秀,遼陽人也。大定二十一年進士,調黎陽主簿。丁母憂,服闋,調淄州軍事判官,遷汲縣令。察廉,改定興令。補尚書省令史,從參知政事馬琪塞澶淵決河,改
 中都左警巡使。平章政事完顏守貞禮接士大夫在其門者,號「冷巖十俊」,奎其一也。改都轉運司支度判官、上京等路提刑判官。初,遼東契丹判餘里也嘗殺驛使大理司直,有契丹人同名者,有司輒繫之獄,奎按囚速頻路讞而出之,既而果獲其殺司直者。遷同知西京路轉運使事。置行樞密院于鎮寧,充宣差規措所官給軍用。改簽河東南北路按察司事、武州刺史。上言三事,其一曰親民之寄,「今吏部之選頗輕,使武夫計資而得,權歸胥吏。每縣宜參用士人,使紀綱其事。」未幾,改曹州刺史,再調同知中都路都轉運使事。旱,詔審錄中都路冤獄,
 多平反。大安初,除博州防禦使,凡屬縣事應赴州者,不得泊於逆旅,以防吏姦,人便之。改山東東西路安撫副使,遷北京、臨潢等路按察轉運使,以本官為行六部侍郎。劾奏監軍完顏訛出虛造功狀,訛出坐免官。詔以奎為宣差都提控。貞祐初,以疾卒,謚莊肅。



 烏林答與,本名合住,大名路納鄰必剌猛安人。充奉職、奉御、尚食局直長,兼頓舍。除監察御史,累官武勝軍節度使、北京按察轉運使、太子詹事、武衛軍都指揮使。貞祐二年,知東平府事,權宣撫副使。改西安軍節度使,入為兵部尚書。上言:「按察轉運司拘榷錢穀,糾彈非違,此
 平時之治法。今四方兵動,民心未定,軍士動見刻削,乞權罷按察及勸農使。」又曰:「東平屯兵萬餘,可運濱鹽易糧芻給之。」又曰:「潼關及黃河津要,將校皆出卒伍,類庸懦不可用。乞選材武者代之。」又曰:「兗、曹、濮、濬諸郡皆可屯重兵,敕州縣官勸民力穡,至於防秋,則清野保城。」下尚書省,竟不施行。新制科買軍器材物稽緩者並的決,與奏:「有司必督責趣辦,民將不堪,可量罰月俸。」從之。坐前在陜州市物虧直,降鄭州防禦使。尋召為拱衛直都指揮使,復為兵部尚書。興定三年,卒。



 郭俁,字伯有,澤州人。大定二十二年進士,調長子主簿、
 萊州觀察判官、萊陽縣令,補尚書省令史,知管差除。除大理司直。丁母憂,起復太常博士、左司都事。御史臺舉俁及前應奉翰林文字張楫、吏部主事王質、刑部主事抹捻居中、通事舍人完顏合住、弘文校理把掃合、吏部架閣管勾烏古論和尚、尚書省令史溫迪罕思敬皆才幹可用。詔各升一等,遷除俁平陽府治中、張楫國子博士、王質昭義軍節度副使、抹捻居中大理司直、完顏合住侍儀司令、把掃合同知弘文院事、烏古論和尚利涉軍節度副使、溫迪罕思敬同知定武軍節度事。久之,俁召為同知登聞鼓院兼秘書丞,遷禮部郎中、滕州刺史、
 同知真定府事。上言:「每季合注巡尉官,吏、刑兩部斟酌盜賊多寡處選注。」詔議行之。改中都、西京按察副使,遷國子祭酒。泰和六年,伐宋,充宣差山東安撫副使。七年,遷山東宣撫副使。大安元年,遷遼東按察轉運使,改中都路都轉運使、泰定軍節度使、陜西東路按察轉運使。貞祐三年,罷按察司,仍充本路轉運使,行六部尚書。改河北西路轉運使,致仕。元光二年,卒。



 溫迪罕達,字子達,本名謀古魯,蓋州按春猛安人。性敦厚,寡言笑。初舉進士,廷試搜閱官易達藐小,謂之曰:「汝欲求作官邪?」達曰:「取人以才學,不以年貌。」眾咸異之。明
 昌五年,中第,調固安主簿。以憂去官,服除,調信州判官。丞相襄辟行省幕府。改順州刺史,補尚書省令史,除南京警巡使。居父喪,是時伐宋兵興,起復,給事行尚書省。大安初,遷德興府判官,再遷監察御史。宣宗遷汴,以本職護送衛士妻子。復被詔運大名粟,由御河抵通州,事集,遷一官,轉戶部員外郎、左司郎中。遇繼母憂,起復太常少卿,充陜西元帥府經歷官。



 興定元年,召還,攝侍御史,上疏論伐宋,略曰:「天時向暑,士馬不利,宜俟秋涼,無不可者。」又曰:「遼東興王之地,移剌都不能守,走還南京。度今之勢,可令濮王守純行省蓋州,駐兵合思罕,以繫
 一方之心。昔祖宗封建諸王,錯峙相維,以定大業。今乃委諸疏外,非計也。」宣宗曰:「一子非所愛,但幼不更事,詎能辦此?」逾月,復上言:「天下輕重,係于宰相,邇來每令權攝,甚無謂也。今之將帥,謀者不能戰,戰者不能謀。今豈無其人,但用之未盡耳。」宣宗曰:「人才難知,故先試其稱否,卿何患焉。所謂用之未盡者為誰?」對曰:「陜西統軍使把胡魯忠直幹略,知延安府古里甲石倫深沉有謀,能得士心,雖有微過,不足以累大。」宰相高琪、高汝勵惡其言。俄充陜州行樞密院參議官。二年,召為戶部侍郎。改刑部,兼左司諫,同知集賢院。改大理卿,兼越王傅。尋遷
 河南統軍使、昌武軍節度使,行六部,攝同簽樞密院,行院許州。改集慶軍節度使。



 是時,東方薦饑,達上疏曰:「亳州戶舊六萬,今存者無十一,何以為州?且今調發數倍於舊,乞量為減免。」是歲大水,碭山下邑野無居民,轉運司方憂兵食,達謾聞二縣無主稻田且萬頃,收可數萬斛,即具奏。朝遷大駭,詔戶部尚書高夔佩虎符專治其事,所獲無幾,夔坐累抵罪。達自念失奏,因感愧發病,尋卒。



 王擴,字充之,中山永平人。明昌五年進士,調鄧州錄事,潤色律令文字。遷懷安令。猾吏張執中誣敗二令,擴到
 官,執中挈家避去。改徐州觀察判官,補尚書省令史,除同知德州防禦使事。被詔賑貸山東西路饑民,棣州尤甚,擴輒限數外給之。



 泰和伐宋,山東盜賊起,被安撫使張萬公牒提控督捕。擴行章丘道中,遇一男子舉止不常,捕訊,果歷城大盜也。眾以為有神。再遷監察御史,被詔詳讞冤獄。是時,凡鬥殺奏決者,章宗輒減死,由是中外斷獄,皆以出罪為賢。擴謂同輩曰:「生者既讞,地下之冤云何!」是時,置三司治財,擴上書曰:「大定間,曹望之為戶部,財用殷阜,亦存乎人而已。今三司職掌,皆戶部舊式,其官乃戶部之舊官,其吏亦戶部之舊吏,何愚於戶
 部而智於三司乎?」既而三司亦竟罷。張煒職辦西北路糧草者數年,失亡多,尚書省奏擴考按,會煒亦舉王謙自代,王謙發其奸蠹,擴按之無所假借。煒舊與擴厚,使人諉擴曰:「君不念同舍邪?」擴曰:「既奉詔,安得顧故人哉!」



 大安中,同知橫海軍節度事,簽河東北路按察事。貞祐二年,上書陳河東守禦策,大概謂:「分軍守隘,兵散而不成軍。聚之隘內,軍合則勢重。饋餉一途,以逸待勞,以主待客,此上策也。」又曰:「軍校猥眾,分例過優,萬戶一員,其費可給兵士三十人。本路三從宜,萬戶二百餘員,十羊九牧,類例可知。乞以千人為一軍,擇望重者一人萬戶,
 兩猛安、四謀克足以教閱約束矣,豈不簡易而省費哉。」又曰:「按察兼轉運,本欲假糾劾之權,以檢括錢穀。邇來軍興,糧道軍府得而制之。今太原、代、嵐三軍皆其州府長官,如令通掌資儲,則弊立革,按察之職舉矣。」又曰:「數免租稅,科糴益繁,民不為恩,徒增廩給,教練無法,軍不足用。」書奏,不見省。



 遷汴後,召為戶部侍郎,遷南京路轉運使。太府監奏羊瘦不可供御。宣宗召擴詰問。擴奏曰:「官無羊,皆取於民,今民心未安,宜崇節儉。廷議肥瘠紛紛,非所以示聖德也。」宣宗首肯之。平章政事高琪閱尚食物,謂擴曰:「聖主焦勞萬機,賴膳羞以安養,臣子宜盡
 心。」擴曰:「此自食監事,何勞宰相!」高琪默然,銜之。有司奪市人衣,以給往戍潼關軍士,京師大擾。擴白宰相,請三日造之。高琪怒不從。潼關已破,大元兵至近郊,遣擴行六部事,規辦潼關芻糧。偕戶部員外郎張好禮往商、虢,過中牟,不可進。高琪奏擴畏避,下吏論死。宣宗薄其責,削兩階,杖七十,張好禮削三階,杖六十。降為遙授隴州防禦使,行六部侍郎,規辦秦、鞏軍食。逾月,權陜西東路轉運使,行六部尚書。致仕。興定三年,卒,謚剛毅。擴博學多才,梗直不容物,以是不振於時云。



 移剌福僧,東北路烏連苦河猛安人。以廕補吏部令史,
 轉樞密院,調滕州軍事判官,歷甄官署直長、豳王府司馬、順義軍節度副使。部內世襲猛安木吞掠民婦女,藏之窟室,人頗聞之,無敢發其罪者。福僧請于節度使,願自效,既跡得其所在,率眾入索之,得婦女四十三人,木吞抵罪。徙橫海軍,轉同知開遠軍節度事,簽北京、臨潢按察事,興中治中,莫州刺史。上言:「沿邊軍官私役軍人,邊防不治,及擾動等事,按察司專一體究,各路宣差提控嚴勒禁治。」詔尚書省行之。



 大安初,改沃州,同知興中府事。福僧督民繕治城郭,浚濠為禦守備,百姓頗怨。頃之,兵果至,攻其北城。福僧戰其北,使備其西,薄暮果攻
 其西,以有備乃解去。尋改廣寧。崇慶元年秋,福僧被牒如鄰郡,大兵薄城,其子銅和尚率家奴拒戰,廣寧賴之以完。福僧還,悉放奴為良,終不言子之功,識者多之。未幾,充遼東宣撫副使。歲大饑,福僧出沿海倉粟,先賑其民,而後奏之,優詔獎諭。至寧元年,除鞏王傅兼吏部郎中。胡沙虎作難,福僧稱疾不出。宣宗封胡沙虎澤王,百官皆賀,福僧不往,胡沙虎欲摭而罪之。詔除福僧壽州防禦使。貞祐三年,遷山東西路按察轉運使。是歲按察司罷,仍充轉運使。久之,致仕。



 興定二年十一月庚辰,宣宗御登賢門,召致仕官,兵部尚書完顏蒲剌都、戶部尚
 書蕭貢、刑部尚書僕散偉、工部尚書奧屯扎里吉、翰林學士完顏孛迭、轉運使福僧、河東北路轉運使趙重福、沁南軍節度使豬奮、鎮南軍節度使石抹仲溫、泰定軍節度使李元輔、中衛尉完顏奴婢、原州刺史紇石烈孛吉賜食,訪問時政得失。福僧乃上書曰:「為今之計,惟先招徠颭人。選擇颭人舊有宿望雄辨者,諭以恩信,彼若內附,然後中都可復,遼東可通。今西北多虞,而南鄙不敢撤戍,芻糧調度,仰給河南,賦役頻繁,民力疲弊。宜開宋人講和之端,撫定河朔,養兵蓄銳,策之上也。」又曰:「山東殘破,群盜滿野,官軍既少,且無騎兵。若宋人資以糧
 餉,假以官爵,為患愈大。當選才幹官充宣差招捕,以恩賞諭使復業。募其壯悍為兵,亦致勝之一也。」又曰:「自承安用兵,軍中設監戰官,論議之間,動相矛盾,不懲其失,反以為法。若輩平居,皆選材勇自衛,一旦有急,驅疲懦出戰,寧不敗事?罷之為便。」書奏,朝廷略施用焉。元光元年卒。



 贊曰:宣宗急於求賢,而使小人間之;悅於直言,而使邪說亂之。貞祐、興定之間,豈無其人哉。雖故直言蔽於所惑,群才詘於見忌耳。自納坦謀嘉以下,可攷見焉。



 奧屯忠孝,字全道,本名牙哥,懿州胡土虎猛安人。幼孤,
 事母孝。中大定二十二年進士科,調蒲州司候,察廉,遷一官,除校書郎兼太子司經。三遷禮部員外郎。遷翰林待制,權戶部侍郎,佐參知政事胥持國治決河,以勞進一階。除河平軍節度使,兼都水監,遂疏七祖佛河及王村、周平、道口、雞爪、孫家港,復開東明、南陽岡、馬蹄、孫村諸河。忠孝常曰:「河之為患,不免勞民。復壘石為岸十餘里,民不勝其病矣。」改沁南軍,坐前在衛州勾集妨農軍借民錢不令償,由是貧富不相假貸,軍民不相安,降寧海州刺史。改滑州,歷同知南京留守,遷定國軍節度使,復為沁南軍。入為太子少傅兼禮部尚書。



 貞祐初,議降
 衛紹王,忠孝與蒲察思忠附胡沙虎議,語在思忠傳。頃之,拜參知政事。中都圍急,糧運道絕,詔忠孝搜括民間積粟,存兩月食用,悉令輸官,酬以銀鈔或僧道戒牒。是時,知大興府事胥鼎計畫軍食,奏許人納粟買官,鼎已籍者,忠孝再括之,令百姓兩輸,欲為己功。左諫議大夫張行信上疏論之曰:「民食止存兩月,而又奪之,使當絕食,不獨歸咎有司,而亦怨朝廷之不察也。」宣宗善行信言,命近臣與忠孝同審取焉。謂忠孝曰:「國家本欲得糧,今既得矣,姑從民便可也。」頃之,行信復奏曰:「參政奧屯忠孝平生矯偽不近人情,急於功名,詭異要譽,慘刻害
 物,忍而不恤。勾當河防,河朔居民不勝其病。軍負民錢,抑不令償。東海欲用胡沙虎,舉朝皆曰不可,忠孝獨力薦。及胡沙虎作難,忠孝自謂有功。詔議東海爵號,忠孝請籍沒其子孫,及論特末也則云不當籍沒,其偏黨不公如此。無事之時,猶不容一相非才,況今多故,乃使此人與政,如社稷何!」宣宗曰:「朕初即位,當以禮進退大臣,卿語其親知,諷之求去可也。」行信以語右司郎中把胡魯,把胡魯以宣宗意白忠孝,忠孝塤然不聽。頃之,罷為太子太保,出知濟南府事,改知中山府。尋薨,年七十,謚惠敏。



 蒲察思忠,本名畏也,隆安路合懶合兀主猛安人。大定二十五年進士,調文德、漷陰主簿,國子助教,應奉翰林文字,太學博士,累遷涿州刺史,吏部郎中,遷潞王傅。被詔與翰林侍讀學士張行簡討論武成王廟配等列,思忠奏曰:「伏見武成王廟配享諸將,不以世代為先。後按唐祀典,李靖、李勣居吳起、樂毅上。聖朝太祖以二千之眾,破百萬之師,太宗克宋,成此帝業,秦王宗翰、宋王宗望、婁室、谷神與前代之將,各以功德間列可也。」思忠論多矯飾,不盡錄,錄其頗有理者云。遷大理卿,兼左司諫,同修國史。



 泰和六年,平章政事僕散揆宣撫河南,詔以
 備禦攻守之法,集百官議於尚書省。廷臣尚多異議,思忠曰:「宋人攻圍城邑,動至數千,不得為小寇。但當選擇賢將,宜攻宜守,臨時制變,無不可者。」上以為然。頃之,遷翰林侍講學士兼左諫議大夫,大理卿、同修國史如故。再閱月,兼知審官院正職,外兼四職自思忠始。宋人請和。賜銀五十兩、重彩十端。丁母憂,起復侍講學士,兼諫議、修史、知審官院,轉侍讀,兼兵部侍郎。



 貞祐初,胡沙虎請廢衛紹王為庶人,思忠與奧屯忠孝阿附胡沙虎,曰:「竊人之財,猶謂之盜,況偷天位以私己乎!」宣宗不從。頃之,遷太子太保兼侍讀、修國史。二年春,享于太廟,思忠
 攝太尉,醉毆禮直官,御史臺劾奏,降秘書監兼同修國史。頃之,遷翰林學士同修國史,卒。



 紇石烈胡失門,上京路猛安人。明昌五年進士,累官補尚書省令史,除中都路支度判官,調河北東路都勾判官,累官翰林直學士、大理卿、右諫議大夫。興定二年,伐宋,充元帥左都監紇石烈牙吾塔參議官。牙吾塔至楚州,不待行省僕散安貞節制,輒進兵。宋人堅壁不出,野無所掠,軍士疲乏,餓死相望,直前至江而復。安貞劾奏之,牙吾塔坐不奉詔約,胡失門不矯正,特詔原之。改同知彰德府事。五遷吏部尚書。五年,拜御史大夫。元光元
 年,兼大司農。二年,薨,宣宗輟朝,百官致奠。



 完顏宇,本名訛出,西南路猛安人。大定二十八年進士,累調河東北路提刑司知事,改同知遼州軍州事,召為國史院編修官,遷應奉翰林文字、南京路轉運副使。丁父憂,起復太府監丞,改吏部員外郎。大安初,除知登聞檢院,累遷右司郎中、翰林待制,兼侍御史。貞祐初,議衛紹王事,語在《衛紹王紀》。



 中都圍急,詔於東華門置招賢所,內外士庶皆得言事,或不次除官,由是閭閻細民,往往炫鬻求售。王守信者,本一村夫,敢為大言,以諸葛亮為不知兵,宇薦于朝。詔署行軍都統,募市井無賴為兵,
 教閱進退跳擲,大概似童戲。其陣法大書「古今相對」四字於旗上,作黃布袍、緇巾、鑞牌各三十六事,牛頭響環六十四枚,欲以怖敵而走之,大率皆誕妄。因與其眾出城,殺百姓之樵採者以為功。賈耐兒者,本歧路小說人,俚語詼嘲以取衣食,製運糧車千兩。是時材木甚艱,所費浩大,觀者皆竊笑之。草澤李棟在衛紹王時嘗事司天監李天惠,依附天文,假託占卜,趨走貴臣,俱為司天官。棟嘗密奏白氣貫紫微,主京師兵亂,幸不貫徹,得不成禍。既而高琪殺胡沙虎,宣宗愈益信之。



 左諫議大夫張行信奏曰:「狂子庸流,猥蒙拔擢,參預機務,甚無謂也。
 司天之官,占見天象,據經陳奏,使人主飭已修政,轉禍為福。如有天象,乞令諸監官公同陳奏,所見或異,則各以狀聞,不宜偏聽也。」上召行信與宇面計守信事,復與近侍就決于高琪。高琪言守信不可用,上乃以行信之言為然。



 頃之,宇遷禮部侍郎,改東京副留守、隴州防禦使,遷安化軍節度使,兼山東路統軍副使。興定元年四月,詔宇以本官權元帥左都監,行元帥府事,和輯苗道潤、移剌鐵哥軍事,語在道潤傳。十二月,密州破,宇為亂軍所殺。



 斡勒合打,蓋州本得山猛安人。以蔭補官,充親軍,調山陰
 尉。縣當兵衝,合打率土豪官兵身先行陣。貞祐初,以功遷本縣令。縣升為忠州,合打充刺史。州被兵久,耕桑俱廢,詔徙其民於太和嶺南。合打遙授同知太原府事,仍領其眾。俄以本官遙授彰國軍節度使,權河東北路宣撫副使,督糧餉往代州。合打不欲行,因與宣撫使完顏伯嘉爭辨。合打恐伯嘉奏聞,乃先奏伯嘉辱己。御史臺廉得其事,未及奏,伯嘉、合打皆改遷。合打改武寧軍節度使。數月,召為勸農使。久之,為金安軍節度使。興定元年,復為勸農使,歷知河間府,權元帥右都監,行元帥府事,駐兵蔡、息間。權同簽樞密院事,守河清,改知歸德
 府事。合打屢守邊要,無他將略,雖未嘗敗北,亦無大功。元光元年,卒。



 蒲察移剌都,東京猛安人。父吾迭,太子太傅致仕。移剌都勇健多力,充護衛十人長,調同知秦州防禦使事、武衛軍鈐轄,以憂去官。起復武器署令。從軍,兵潰被執。貞祐二年,與降兵萬餘人俱脫歸。遷隆安府治中,賜銀百兩,重幣六端,遙授信州刺史。有功,遷蒲與路節度使兼同知上京留守事,進三階,改知隆安府事。逾年,充遼東、上京等路宣撫使兼左副元帥。再閱月,就拜尚書右丞。移剌都與上京行省蒲察五斤爭權,及賣隆安戰馬,擅
 造銀牌,睚眥殺人,已而矯稱宣召,棄隆安赴南京,宣宗皆釋不問。除知河南府事,俄改元帥左監軍,權左副元帥,充陜西行省參議官。無何,兼陜西路統軍使。興定二年四月,改簽樞密院事,權右副元帥,行樞密院於鄧州。御史臺奏移剌都在軍中,買沙覆道,盜用官銀,矯制收禁書,指斥鑾輿,使親軍守門,護衛押宿,擬前後衛仗,婢妾效內人妝飾等數事。詔吏部尚書阿不罕斜不失鞫之,坐是誅。



 贊曰:讀《金史》,至張行信論奧屯忠孝事,曰:嗟乎,宣宗之不足與有為也如此!夫進退宰執,豈無其道也哉!語其
 親知,諷之求去,豈禮邪?是故奧屯忠孝、蒲察思忠之黨比,紇石烈胡失門之疲眾,完顏宇之輕信誤國,斡勒合打之詆訟上官,於是曾不之罪,失政刑矣,豈小懲大誡之道哉!



\end{pinyinscope}