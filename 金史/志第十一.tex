\article{志第十一}

\begin{pinyinscope}

 禮三



 ○宗廟禘袷朝享時享儀



 金初無宗廟。天輔七年九月,太祖葬上京宮城之西南,建寧神殿於陵上,以時薦享。自是諸京皆立廟,惟在京師者則曰太廟。天會六年,以宋二帝見太祖廟者,是也。或因遼之故廟,安置御容,亦謂之廟。天眷三年,熙宗幸燕及受尊號,皆親享恭謝,是也。皇統三年,初立太廟,八年,太廟成,則上京之廟也。貞元初,海陵遷燕,乃增廣舊
 廟,奉遷祖宗神主于新都,三年十一月丁卯,奉安於太廟。正隆中,營建南京宮室,復立宗廟,南渡因之。其廟制,史不載,傳志雜記或可概見,今附之。



 汴京之廟,在宮南馳道之東。殿規,一屋四注,限其北為神室,其前為通廊。東西二十六楹,為間二十有五,每間為一室。廟端各虛一間為夾室,中二十三間為十一室。從西三間為一室,為始祖廟,祔德帝、安帝、獻祖、昭祖、景祖祧主五,餘皆兩間為一室。或曰:「惟第二、第二室兩間,餘止一間為一室,總十有七間。」世祖室祔肅宗,穆宗室祔康宗,餘皆無祔。每室門一、牖一、門在左,牖在右,皆南向。石室之龕於各室之西壁,東向。其始祖之龕
 六,南向者五、東向者一,其二其三俱二龕,餘皆一室一龕,總十八龕。祭日出主於北墉下,南向。禘祫則並出主,始祖東向,群主依昭穆南北相向,東西序列。室戶外之通廊,殿階二級,列陛三,前井亭二。外作重垣四繚,南東西皆有門。內垣之隅有樓,南門五闔,餘皆三。中垣之外東北,冊寶殿也,太常官一人季視其封緘,謂之點寶。內垣之南曰大次,東南為神庖。廟門翼兩廡,各二十有五楹,為齋郎執事之次。西南垣外,則廟署也。神門列戟各二十有四,植以木錡。戟下以板為掌形,畫二青龍,下垂五色帶長五尺,享前一日則縣戟上,祭畢藏之。



 室次。
 大定十二年,議建閔宗別廟,禮官援晉惠、懷、唐中宗、後唐莊宗升祔故事,若依此典,武靈皇帝無嗣亦合升祔。然中宗之祔,始則為虛室,終增至九室。惠、懷之祔乃遷豫章、穎川二廟、莊宗之祔乃祧懿祖一室。今太廟之制,除祧廟外,為七世十一室,如當升祔武靈,既須別祧一廟。《荀子》曰:「有天下者事七世。」若旁容兄弟,上毀祖考,則天子有不得事七世者矣。伏睹宗廟世次,自睿宗上至始祖,凡七世,別無可祧之廟。《晉史》云:「廟以容主為限,無拘常數。」東晉與唐皆用此制,遂增至十一室。康帝承統,以兄弟為一室,故不遷遠廟而祔成帝。唐以敬、文、武
 三宗同為一代,於太廟東間增置兩室,定為九代十一室。今太廟已滿此數,如用不拘常數之說,增至十二室,可也。然廟制已定,復議增展,其事甚重,又與睿宗皇帝祏室昭穆亦恐更改。《春秋》之義不以親親害尊尊,《漢志》云:「父子不並坐,而孫可從王父。」若武靈升祔,太廟增作十二室。依春秋尊尊之典,武靈當在十一室,禘袷合食。依孫從王父之典,當在太宗之下,而居昭位,又當稱宗。然前升祔睿宗已在第十一室,累遇袷享,睿宗在穆位,與太宗昭位相對,若更改祏室及昭穆序,非有司所敢輕議,宜取聖裁。十九年四月,禘祔閔宗,遂增展太廟為
 十二室。二十九年,世宗將祔廟,有司言:「太廟十二室,自始祖至熙宗雖係八世。然世宗與熙宗為兄弟,不相為後,用晉成帝故事,止係七世,若特升世宗、顯宗即係九世。」於是五月遂祧獻祖、昭祖,升祔世宗、明德皇后,顯宗于廟。



 貞祐二年,宣宗南遷,廟社諸祀並委中都,自抹拈盡忠棄城南奔,時謁之禮盡廢。四年,禮官言:「廟社國之大事,今主上駐蹕陪京,列聖神主已遷于此,宜重修太廟社稷,以奉歲時之祭。按中都廟制,自始祖至章宗凡十二室,而今廟室止十一,若增建恐難卒成。況時方多故,禮宜從變,今擬權祔肅宗主世祖室,始祖以下諸神
 主于隨室奉安。」主用慄,依唐制,皇統九年所定也。祏室,旁及上下皆石,門東向,以木為闔,髹以朱。室中有褥,奠主訖,帝主居左,覆以黃羅帕,后主居右,覆以紅羅帕。



 黼扆。以紙,木為筐,兩足如立屏狀。覆以紅羅三幅,繡金斧五十四,裹以紅絹,覆於屏上,其半無文者垂於其後。置北墉下,南向,前設几筵以坐神主。五席,各長五尺五寸,闊二尺五寸。莞筵,粉純。以藺為席,緣以紅羅,以白繡蕙文及雲氣之狀,復以紅絹裹之。每位二。繅席,畫純。以五色絨織青蒲為之,緣以紅羅,畫藻文及雲氣狀,亦以紅絹裹之。每位二,在莞上。次席,黼純。以輕筠為之,亦
 曰桃枝席,緣以紅綃,繡鐵色斧,裹以紅絹。每位二,在繅席上。虎席二,大者長同,惟闊增一尺。以虎皮為褥,有厓,以紅羅繡金色斧緣之。又有小虎皮褥,制同三席。時喧則用桃枝次席,時寒則去桃枝加虎皮褥。夏、秋享,則用桃枝次席。二冬,則去桃枝加小虎皮褥於繅席上。臘冬,則又添大虎皮褥二於繅席上,遷小虎皮褥二在大褥之上。曲几三足,直几二足,各長尺五寸,以丹漆之。帝主前設曲几,后設直几。



 禘祫



 大定十一年,尚書省奏禘祫之儀曰:「《禮緯》:『三年一祫,五年一禘。』唐開元中,太常議,禘祫之禮皆為殷祭,祫
 為合食祖廟,禘謂禘序尊卑。申先君逮下之慈,成群嗣奉親之孝。自異常享,有時行之。祭不欲數,數則黷。不欲疏,疏則怠。是以王者法諸天道,以制祀典,丞嘗象時,禘祫象閏。五歲再閏,天道大成,宗廟法之,再為殷祭。自周以後,並用此禮。自大定九年已行祫禮,若議禘祭,當於袷後十八月孟夏行禮。」詔以「三年冬祫、五年夏禘」為常禮。又言:「海陵時,每歲止以二月、十月遣使兩享,三年祫享。按唐禮四時各以孟月享于太廟,季冬又臘享,歲凡五享。若依海陵時歲止兩享,非天子之禮,宜從典禮歲五享。」從之。享日並出神主前廊,序列昭穆。應圖功臣配
 享廟廷,各配所事之廟,以位次為序。以太子為亞獻,親王為終獻,或並用親王。或以太尉為亞獻,光祿卿為終獻。其月則停時享。儀闕。



 朝享儀



 大定十一年十一月,郊祀前一日,朝享太廟。齋戒如親郊。享前三日,太廟令帥其屬,掃除廟之內外。點檢司於廟之前約度,設兵衛旗幟。尚舍於南神門之西設饌幔十一,南向,以西為上。殿中監帥尚舍,陳設大次殿。又設小次於阼階下,稍南,西向。又設皇帝拜褥位殿上,版位稍西。又設黃道褥於廟門之內外,自玉輅至升輦之所,又自大次至東神門。又設七祀位一於殿下橫
 街之北,西街之西,東向,配享功臣位於殿下道東,橫街之南,西向,北上。前二日,大樂令設宮縣之樂於庭中,四方各設編鐘三、編磬三。東方編鐘起北,編磬間之,東向。西方編磬起北,編鐘間之,西向。南方編磬起西,編鐘間之,北方編鐘起西,編磬間之,俱北向。設特磬、大鐘、穀鐘共十二,於編縣之內,各依辰位。樹路鼓、路鞀於北縣之內,道之左右。晉鼓一,在其後稍南。植建鼓、鞞鼓、應鼓於四隅,建鼓在中,鞞鼓在左,應鼓在右,置柷敔於縣內,柷一在道東,敔一在道西。立舞表於酂綴之間。設登歌之樂於殿上前楹間,金鐘一在東,太磬一在西,俱北向。柷
 一在金鐘北稍西,敔一在玉磬北稍東。搏拊二,一在柷北,一在敔北,東西相向。琴瑟在前。其匏竹者立於階間,重行北向。諸工人各位於縣後。前一日,太廟令開室,奉禮郎帥其屬,設神位於每室內北墉下。各設黼扆一、莞席一、繅席二、次席二、紫綾厚褥一、紫綾蒙褥一、曲几一、直几一。



 又設皇帝版位於殿東間門內,西向。又設飲福位於東序,西向。又設亞終獻位於殿下橫街之北稍東,西向。助祭親王宗室使相位在亞終獻之後,助祭宗室位在橫街之南,西向。奉瓚官、奉瓚盤官、進爵酒官、奉爵官等又在其南,奉匜槃巾篚官位於其後。七祀獻官位
 在奉爵官之南,助奠讀祝奉罍洗爵洗等官位於其後。司尊彝官位在七祀獻官之南,亞終獻司罍洗爵洗奉爵酒官等又在其南,並西向,北上。大禮使位於西階之西稍南,與亞終獻相對。太尉、司徒,助祭宰相位在大禮使之南,侍中、執政官又在其南,禮部尚書、太常卿、太僕卿、光祿卿、功臣獻官在西,舉冊、光祿丞、太常博士又在其西,功臣助奠罍洗爵洗等官位於功臣獻官之後。又設監祭御史位二於西階下,俱東向,北上。奉禮郎、太廟令、太官令、太祝、宮闈令、祝史位於亞獻終獻奉爵酒官之南,薦籩豆簠簋官、薦俎齋郎又在太祝、奉禮郎之南。
 太廟丞、太官丞各位於令後。協律郎位二,一於殿上前楹間,一於宮縣之西北,俱東向。大樂令於登歌樂縣之北,大司樂於宮縣之北,良醖令於酌尊所,俱北向。又設助祭文武群官位於橫街之南,東向北上。又設光祿卿陳牲位於東神門外橫街之東,西向,以南為上。設廩犧令位於牲西南,北向。諸太祝位於牲東,各當牲後,祝史各陪其後,俱西向。設禮部尚書省牲位於牲前稍北,又設御史位於禮部尚書之西,俱南向。禮部帥其屬,設祝冊案於室戶外之右。司尊彞帥其屬,設尊彞之位於室戶之左,每位斝彞一、黃彞一、犧尊二、象尊二、著尊二、山
 罍二,各加勺、冪、坫為酌尊。又設瓚槃爵坫於篚,置于始祖尊彞所。又設壺尊二、太尊二、山罍四,各有坫、冪,在殿下階間,北向西上,設而不酌。七祀功臣每位設壺尊二於座之左,皆加冪、坫於內,酌尊加勺,皆藉以席。奉禮郎設祭器,每位四簋在前,四簠次之,次以六醿,次以六鉶,籩豆為後。左十有二籩,右十有二豆,皆濯而陳之,藉以席。籩豆加以巾,蓋於內。籩一、豆二、簠一、簋一、并俎四,設於每室饌幔內。又設御洗二於東階之東。又設亞終獻罍洗於東橫街下東南,北向,罍在洗東,篚在洗西,南肆,實以巾。又設亞終獻爵洗於罍洗之西,罍在洗東,篚在
 洗西,南肆,實以巾、爵并坫。執巾罍巾篚各位於其後。



 享日丑前五刻,太常卿帥執事者,設燭於神位前及戶外。光祿卿帥其屬,入實籩豆。籩之實,魚鱐、糗餌、粉餈、乾棗、形鹽、鹿脯、榛實、乾裛、桃、菱、芡、慄,以序為次。豆之實,芹菹、筍菹、葵菹、菁菹、韭菹、酏食、魚醢、兔醢、豚拍、鹿臡、醓醢、糝食,以序為次。又鉶實以羹,加芼滑,登實以大羹,簠實以稻粱,簋實以黍稷,粱在稻前,稷在黍前。良醖令入實尊彝。斝彞、黃彞實以鬱鬯,犧尊、象尊、著尊實以玄酒外,皆實以酒用香藥酒,各加坫、勺、冪。殿下之尊罍,壺尊、太尊、山罍,內除山罍上尊實以玄酒外,皆實以酒,加冪、坫。太廟令
 帥其屬,設七祀功臣席褥於其次,每位各設莞席一、碧綃褥一,又各設版位於其座前,又籩豆簠簋各二、俎一。每位次各設壺尊二於神座之右,北向,玄酒在西。良醖令以法酒實尊如常,加勺、冪,置爵於尊下,加坫。光祿卿實饌。左二籩,慄在前,鹿脯次之。右二豆,菁菹在前,鹿臡次之。俎實以羊熟,簠簋實以黍稷。太廟令又設七祀燎柴,及開瘞坎於西神門外之北。太府監陳異寶、嘉瑞、伐國之寶,戶部陳諸州歲貢,金為前列,玉帛次之,餘為後,皆於宮縣之北,東西相向,各藉以席。凡祀神之物,當時所無者則以時物代之。



 省牲器:前一日未後,廟所禁
 行人。司尊彞、奉禮郎及執事者,升自西階以俟。少頃,諸太祝與廩犧令,以牲就位。禮直官、贊者引禮部尚書、光祿卿丞詣省牲位,立定。禮直官引禮部尚書,贊引者引御史,入就西階升,遍視滌濯。訖,執事者皆舉冪曰:「潔。」俱降,就省牲位,禮直官稍前曰:「告潔畢,請省牲。」次引禮部尚書侍郎稍前,省牲訖,退復位。次引光祿卿丞出班,巡牲一匝。光祿丞西向曰:「充。」曰:「備。」廩犧令帥諸太祝巡牲一匝,西向躬身曰:「腯。」禮直官稍前曰:「省牲畢,請就省饌位。」引禮部尚書以下各就位,立定。御史省饌具畢,禮直官贊:「省饌訖。」俱還齋所。光祿卿、丞及太祝、廩犧令以次
 牽牲詣廚,授太官令。禮直官引禮部尚書詣廚,省鼎鑊,視濯溉,訖,還齋所。晡後一刻,太官令帥宰人,執鸞刀割牲,祝史各取毛血,每座共實一豆,遂烹牲。祝史洗肝於鬱鬯,又取肝惣,每座共實一豆,俱還饌所。



 鑾駕出宮:前一日,有司設大駕鹵簿於應天門外,尚輦進玉輅於應天門內,南向。其日質明,侍臣直衛及導駕官,於致齋殿前,左右分班立俟。通事舍人引侍中俯伏,跪,奏:「請中嚴。」皇帝服通天冠、絳紗袍。少頃,侍中奏:「外辦。」皇帝出齋室,即御座,群官起居訖,尚輦進輿。侍中奏:「請皇帝升輿。」皇帝乘輿,侍衛警蹕如常儀。太僕卿先詣玉輅所,攝衣
 而升,正立執轡。導駕官前導,皇帝至應天門內玉輅所,侍中進當輿前,奏:「請皇帝降輿升輅。」皇帝升輅。太僕卿立授綏,導駕官分左右步導,以裏為上。門下侍郎進當輅前,奏:「請車駕進發。」奏訖,俯伏,興,退復位。侍衛儀物止於應天門內,車駕動,稱:「警蹕。」至應天門,門下侍郎奏:「請車駕少駐,敕侍臣上馬。」侍中奉旨退,稱曰:「制可。」門下侍郎退,傳制,稱:「侍臣上馬。」贊者承傳:「敕侍臣上馬。」導駕官分左右前導,門下侍郎奏:「請車駕進發。」車駕動,稱「警蹕」,不鳴鼓吹。將至太廟,禮直官、贊者各引享官,通事舍人分引從享群官、宗室子孫,於廟門外,立班奉迎。駕至廟
 門,迴輅南向,侍中於輅前奏稱:「侍中臣某言,請皇帝降輅,步入廟門。」皇帝降輅,導駕官前導,皇帝步入廟門,稍東。侍中奏:「請皇帝升輿。」尚輦奉輿,侍衛如常儀。皇帝乘輿至大次,侍中奏:「請皇帝降輿,入就大次。」皇帝入就次,簾降,傘扇侍衛如常儀。太常卿、太常博士各分立於大次左右。導駕官詣廟庭班位,立俟。



 晨稞:享日丑前五刻,諸享官及助祭官,各服其服。太廟令、良醖令帥其屬,入實尊罍。光祿卿、太官令、進饌者實籩豆簠簋,並徹去蓋冪。奉禮郎、贊者先入,就位。贊者引御史、太廟令、太祝、宮闈令、祝史與執事官等,各自東偏門入,就位。未明二
 刻,禮直官引太常寺官屬并太祝、宮闈令升殿,開始祖祏室。太祝、宮闈令捧出帝后神主,設於座。以次,逐室神主各設於內黼扆前,置定。贊者引御史、太廟令、宮闈令、太祝、祝史與太常官屬,於當階間,重行北向立。奉禮郎於殿上贊:「奉神主。」訖,奉禮曰:「再拜。」贊者承傳,御史以下皆再拜,訖,各就位。大樂令帥工人二舞入。就位。禮直官贊者各引享官,通事舍人分引助祭文武群官宗室入就位。符寶郎奉寶,陳於宮縣之北。皇帝入大次。



 少頃,侍中奏:「請中嚴。」皇帝服袞冕。侍中奏:「外辦。」太常卿俯伏,跪,奏稱:「太常卿臣某言,請皇帝行事。」俯伏,興。簾捲,皇帝出
 次。太常卿、太常博士前導,傘扇侍衛如常儀,大禮使後從。至東神門外,殿中監跪進鎮圭,太常卿奏:「請執圭。」皇帝執鎮圭。傘扇仗衛停於門外,近侍者從入。協律郎跪伏舉麾,興。工鼓柷,宮縣《昌寧之樂》作。至阼階下,偃麾,戛敔,樂止。升自阼階,登歌樂作,左右侍從量人數升至版位,西向立,樂止。前導官分左右侍立。太常卿前奏:「請再拜。」皇帝再拜。奉禮曰:「眾官再拜。」贊者承傳,凡在位者皆再拜。奉禮又贊:「諸執事者各就位。」禮直官、贊者分引執事者各就殿上下之位。太常卿奏:「請皇帝詣罍洗位。」登歌樂作,至阼階,樂止。降自阼階,宮縣樂作,至洗位,樂止。
 內侍跪取匜,興,沃水。又內侍跪取盤,興,承水。太常卿奏:「請搢鎮圭。」皇帝搢鎮圭,盥手,訖,內侍跪取巾於篚,興,以進。帨手,訖。奉瓚官以瓚跪進,皇帝受瓚,內侍奉匜,沃水,又內侍跪奉槃承水,洗瓚訖。內侍跪奉巾以進,皇帝拭瓚,訖,內侍奠槃匜,又奠巾於篚。奉瓚槃官以槃受瓚。太常卿奏:「請執鎮圭。」前導,皇帝升殿,宮縣樂作,至阼階下,樂止。皇帝升自阼階,登歌樂作,太常卿前導,詣始祖位酌尊所,樂止。奉瓚槃官以瓚蒞鬯,執尊者舉冪,侍中跪酌鬱鬯,訖,太常卿前導,入詣始祖室神位前,北向立。太常卿奏:「請搢鎮圭。」跪。奉瓚槃官西向跪,以瓚授奉瓚
 官,奉瓚西向以瓚跪進。太常卿奏:「請執瓚以鬯稞地。」皇帝執瓚以鬯稞地,訖,以瓚授奉瓚槃官,太常卿奏:「請執鎮圭。」俯伏,興,前導出戶外。太常卿奏:「請再拜。」皇帝再拜,太常卿前導詣次位,並如上儀。



 祼畢。太常卿奏:「請還版位。」登歌樂作,至版位西向立,樂止。太常卿奏:「請還小次。」前導皇帝行,登歌樂作,降自阼階,登歌樂止,宮縣樂作。將至小次,太常卿奏:「請釋鎮圭。」殿中監跪受鎮圭。皇帝入小次,簾降,樂止。少頃,宮縣奏《來寧之曲》,以黃鍾為宮,大呂為角,大簇為徵,應鍾為羽,作《仁豐道洽之舞》,九成止。黃鐘三奏,大呂、太簇、應鐘各再奏,送神通用《來寧之
 曲》。初,晨稞將畢,祝史各奉毛血及肝惣之豆,先於南神門外,齋郎奉爐炭蕭蒿黍稷,各立於肝惣之後。皇帝既晨稞畢,至樂作六成,皆入自正門,升自太階。諸太祝於階上各迎毛血肝稞,進奠於神座前。祝史立於尊所,齋郎奉爐置於室戶外之左,其蕭蒿黍稷各置於爐炭下。齋郎降自西階,諸太祝各取肝燔於爐,還尊所。



 進熟:皇帝升稞,太官令帥進饌者,奉陳於南神門外諸饌幔內,以西為上。禮直官引司徒出詣饌所,與薦俎齋郎奉俎,并薦籩豆簠簋官奉籩豆簠簋,禮直官、太官令引以序入自正門,宮縣《豐寧之樂》作。徹豆通用。至太階,樂止。祝史
 俱進徹毛血之豆,降自西階,以出。饌升,諸太祝迎於階上,各設於神位前。先薦牛,次薦羊,次薦豕及魚。禮直官引司徒以下,降自西階,復位。諸太祝各取蕭蒿黍稷擩於脂,燎於爐炭,訖,還尊所。贊者引舉冊官升自西階,詣始祖位之右,進取祝冊置在版位之西,置訖,於祝冊案近南立。太常卿跪奏:「請詣罍洗位。」簾捲,出次,宮縣樂作。殿中監跪進鎮圭,太常卿奏:「請執鎮圭。」前導,詣罍洗位,樂止。盥手,洗爵,並如晨稞之儀。盥洗訖,太常卿奏:「請執鎮圭。」前導,升殿,宮縣樂作,至阼階下,樂止。升自阼階,登歌樂作。太常卿前導,詣始祖位尊彝所,登歌樂作,至尊
 彞所,登歌樂止,宮縣奏《大元之樂》,文舞進。奉爵官以爵蒞尊,執尊者舉冪,侍中跪酌犧尊之泛齊,訖,太常卿前導,入詣始祖室神位前,北向立。太常卿奏:「請搢鎮圭。」跪。奉爵官以爵授進爵酒官。進爵酒官西向以爵跪進,太常卿奏:「請執爵三祭酒。」三祭酒於茅苴,訖,以爵授進爵酒官,進爵酒官以爵授奉爵官。太常卿奏:「請執鎮圭。」興。前導,出戶外,太常卿奏:「請少立。」樂止。舉冊官進舉祝冊,中書侍郎搢笏跪讀祝,舉祝官舉冊奠訖,先詣次位。太常卿奏:「請再拜。」再拜訖,太常卿前導,詣次位行禮,並如上儀。酌獻畢,太常卿前導還版位,登歌樂作,至位西向
 立定,樂止。太常卿奏:「請還小次。」登歌樂作,降自阼階,登歌樂止,宮縣樂作。將至小次,太常卿奏:「請釋鎮圭。」殿中監跪受鎮圭。入小次,簾降,樂止,文舞退,武舞進,宮縣奏《肅寧之樂》,作《功成治定之舞》,舞者立定,樂止。



 皇帝酌獻訖,將詣小次,禮直官引博士,博士引亞獻,詣盥洗位,北向立,搢圭,盥手,帨手,執圭。詣爵洗位,北向立,搢圭,洗爵、拭爵以授執事者,執圭。升自西階,詣始祖位尊彞所,西向立。宮縣樂作。執事者以爵授亞獻,亞獻搢圭,執爵,執尊者舉冪,太官令酌象尊之醴齊,訖,詣始祖神位前,搢圭,跪。執事者以爵授亞獻,亞獻執爵祭酒。三祭酒於
 茅苴,奠爵,執圭,俯伏,興,少退,再拜,訖,博士前導,亞獻詣次位行禮,並如上儀。禮畢,樂止。終獻除本服執笏外,餘如亞獻之儀。七祀功臣獻官行禮畢。太常卿跪奏:「請詣飲福位。」簾捲,出次,宮縣樂作。殿中監跪進鎮圭,太常卿奏:「請皇帝執鎮圭。」前導,至阼階下,樂止。升自阼階,登歌樂作,將至飲福位,樂止。



 初,皇帝既獻訖,太祝分神位前三牲肉,各取前腳第二骨加於俎,又以籩取黍稷飯共置一籩,又酌上尊福酒合置一尊。又禮直官引司徒升自西階,東行,立於阼階上前楹間,北向。皇帝既至飲福位,西向立。登歌《福寧之樂》作。太祝酌福酒於爵,以奉侍中,
 侍中受爵捧以立,太常卿奏:「請皇帝再拜。」訖,奏:「請搢圭。」跪,侍中以爵北向跪以進,太常卿奏:「請執爵。」三祭酒於沙池。又奏:「請啐酒。」皇帝啐酒,訖,以爵授侍中。太常卿奏:「請受胙。」太祝以黍稷飯籩授司徒,司徒跪奉進,皇帝受以授左右。太祝又以胙肉俎跪授司徒,司徒受俎訖跪進,皇帝受以授左右。禮直官引司徒退立,侍中再以爵酒跪進。太常卿奏:「請皇帝受爵飲福。」飲福訖,侍中受虛爵以興,以授太祝。太常卿奏:「請執圭。」俯伏,興。又奏:「請皇帝再拜。」再拜訖,樂止。太常卿前導,皇帝還版位,登歌樂作,俟至位,樂止。太祝各進徹籩豆,登歌《豐寧之樂》作,卒
 徹,樂止。奉禮曰:「賜胙行事,助祭官再拜。」贊者承傳,在位官皆再拜,宮縣《來寧之樂》作,一成止。太常卿奏:「禮畢。」前導,降自阼階,登歌樂止,宮縣樂作,出門,宮縣樂止,傘扇仗衛如常儀。太常卿奏:「請釋鎮圭。」殿中監跪受鎮圭,皇帝還大次。通事舍人、禮直官、贊者各引享官、宗室子孫及從享群官,以次出。及引導駕官東神門外大次前祗候,前導如常儀。贊者引御史已下俱復執事位,立定。奉禮曰:「再拜。」皆再拜。贊者引工人、舞人以次出。大禮使帥諸禮官、太廟令、太祝、官闈令,升納神主如常儀。禮畢,禮直官引大禮使已下降自西階,至橫街,再拜而退。其祝
 冊藏於匱。七祀功臣分奠,如祫享之儀。



 時享



 有司行事。前期,太常寺舉申禮部,關學士院司天堂臺,擇日。以其日報太常寺。前七日,受誓戒於尚書省。其日質明,禮直官設位版於都堂之下,依已定《誓戒圖》,禮直官引三獻官,并應行事執事官等,各就位,立定,贊:「揖。」在位官皆對揖,訖,禮直官以誓文奉初獻官,初獻官搢笏,讀誓文:「某月,某日,孟春,薦享太廟,各揚其職。不恭其事,國有常刑。」讀訖,執笏。七品以下官先退,餘官對拜訖乃退。散齋四日,治事如故,宿於正寢,唯不弔喪、問疾、作樂、判署刑殺文字決罰罪人及預穢惡。致齋,三日於本
 司,唯享事得行,其餘悉禁,一日於享所。已齋而闕者,通攝行事。前三日,兵部量設兵衛,列於廟之四門。前一日,禁斷行人。儀鸞司設饌幔十一所於南神門外西,南向。又設七祀司命、戶二位於橫街之北,道西,東向。又設群官齋宿次於廟門之東西舍。前二日,大樂局設登歌之樂於殿上。太廟令帥其屬,掃除廟殿門之內外,於室內鋪設神位於北墉下,當戶南向。設几於筵上,又設三獻官拜褥位二。一在室內,一在室外。學士院定撰祝文訖,計會通進司請御署,降付禮部,置於祝案。祠祭局濯溉祭器與尊彞訖,鋪設如儀。內太尊二、山罍二在室。犧尊五、象尊五、
 雞彞一、鳥彝一在室戶外之左,爐炭稍前。著尊二、犧尊二在殿上,象尊二、壺尊六在下。俱北向西上,加冪,皆設而不酌。并設獻官罍洗位。禮部設祝案於室戶外之右。禮直官設位版并省牲位,如式。前一日,諸太祝與廩犧令以牲就東神門外。司尊彝與禮直官及執事皆入,升自西階,以俟。禮直官引太常卿,贊者引御史,自西階升,遍視滌濯。執尊者舉冪告潔,訖,引降就省牲位。廩犧令少前,曰:「請省牲。」退復位。太常卿省牲,廩犧令及太祝巡牲告備,皆如郊社儀。既畢,太祝與廩犧令以次牽牲詣廚,授太官令。贊者引光祿卿詣廚,請省鼎鑊,申視滌溉。
 贊者引御史詣廚,省饌具,訖,與太常卿等各還齋所。太官令帥宰人以鸞刀割牲,祝史各取毛血,每室共實一豆,又取肝惣共實一豆,置饌所,遂烹牲。光祿卿帥其屬,入實祭器。良醖令人實尊彞。



 享日質明,百官各服其品服。禮直官、贊者先引御史、博士、太廟令、太官令、諸太祝、祝史、司尊彝與執罍篚官等,入自南門,當階間,北面西上,立定。奉禮曰:「再拜。」贊者承傳,皆再拜,訖,贊者引太祝與宮闈令,升自西階,詣始祖室,開祏室,太祝捧出帝主,宮闈令捧出后主,置於座。帝主在西,後主在東。贊者引太祝與宮闈令,降自西階,俱復位。奉禮曰:「再拜。」贊者承傳,在位官
 皆再拜,訖,俱各就執事位。大樂令帥工人入。禮直官、贊者分引三獻官與百官,俱自南東偏門入,至廟庭橫街上,三獻官當中,北向西上,應行事執事官并百官,依品,重行立。奉禮曰:「拜。」贊者承傳,應北向在位官皆再拜。其先拜者不拜。拜訖,贊者引三獻官詣廟殿東階下西向位,其餘行事執事官與百官,俱各就位。訖,禮直官詣初獻官前,稱:「請行事。」協律郎跪,俯伏,興,樂作。禮直官引初獻詣盥洗位,北向立定,樂止。搢笏,盥手,帨手,執笏。詣爵洗位,北向立,搢笏,洗瓚,拭瓚,以瓚授執事者,執笏,升殿,樂作。至始祖室尊彞所,西向立,樂止。執事者以瓚奉初獻官,初獻官
 搢笏,執瓚。執尊者舉冪,太官令酌鬱鬯,訖,初獻以瓚授執事者,執笏,詣始祖室神位前,樂作,北向立,搢笏,跪。執事者以瓚授初獻官。初獻官執瓚,以鬯稞地,訖,以瓚授執事者,執笏,俯伏,興,出戶外,北向,再拜,訖,樂止。每室行禮,並如上儀。禮直官引初獻降復位。初獻將升稞,祝史各奉毛血肝惣豆,及齋郎奉爐炭蕭蒿黍稷篚,各於饌幔內以俟。初獻晨稞訖,以次入正門,升自太階。諸太祝皆迎毛血肝惣豆於階上,俱入奠於神座前。齋郎所奉爐炭蕭蒿篚,皆置於室戶外之左,與祝史俱降自西階以出。諸太祝取肝惣,洗於鬱鬯,燔於爐炭,訖,還尊所。
 享日,有司設羊鼎十一、豕鼎十一於神廚,各在鑊右,初獻既升稞,光祿卿帥齋郎詣廚,以匕升羊於鑊,實于一鼎,肩、臂、臑、肫、胳、正脊一、橫脊一、長脅一、短脅一、代脅一,皆二骨以並。次升豕如羊,實於一鼎。每室羊豕各一鼎,皆設扃冪。齋郎對舉,入鑊,放饌幔前。齋郎抽扃,委于鼎右,除冪,光祿卿帥太官令,以匕升羊,載于一俎。肩臂臑在上端,肫胳在下端,脊脅在中。次升豕如羊,各載于一俎。每室羊豕各一俎。齋郎既以扃舉鼎先退,置於神廚,訖,復還饌幔所。禮直官引司徒出詣饌幔前,立以俟。光祿卿帥其屬,實籩以粉餈,實豆以糝食,實簠以粱,實簋
 以稷。俟初獻稞畢,復位,祝史俱進徹毛血之豆,降自西階以出。禮直官引司徒,帥薦籩豆簠簋官,奉俎齋郎,各奉籩豆簠簋羊豕俎,每室以序而進,立於南神門之外以俟,羊俎在前,豕俎次之,籩豆簠簋又次之。入自正門,樂作,升自太階,諸太祝迎引於階上,樂止。各設於神位前,訖,禮直官引司徒以下,降自西階,樂作,復位,樂止。諸太祝各取蕭蒿黍稷手需於脂,燔於爐炭,還尊所。



 禮直官引初獻詣罍洗位,樂作,至位,北向立,樂止,搢笏,盥手,帨手,執笏。詣爵洗位,北向立,搢笏,洗爵,拭爵,以爵授執事者,執笏,升殿,樂作,詣始祖室酌尊所,西向立,樂止。執事
 者以爵授初獻。初獻搢笏執爵,執事者舉冪,太官令酌犧尊之泛齊,訖,次詣第二室酌尊所,如上儀。詣始祖神位前,樂作,北向立,搢笏跪,執事者以爵授初獻,初獻執爵,三祭酒於茅苴,奠爵,執笏,俯伏,興,出室戶外,北向立,樂止。贊者引太祝詣室戶外,東向,搢笏,跪讀祝文。讀訖,執笏,興。次詣第二室。次詣每室行禮,並如上儀。初獻降階,樂作,復位,樂止。禮直官次引亞獻詣盥洗位,北向立,搢笏,盥手,帨手,執笏。詣爵洗位,北向立,搢笏,洗爵,拭爵以授執事官。執笏,升殿,詣始祖酌尊所,西向立,執事者以爵授亞獻。亞獻搢笏,執爵,執尊者舉冪,太官令酌
 象尊之醴齊,訖,次詣第二室酌尊所,如上儀。詣始祖神位前,樂作,北向立,搢笏,跪,執事者以爵授亞獻。亞獻執爵,三祭酒於茅苴,尊爵,執笏,俯伏,興,出戶外,北向再拜,訖,樂止。次詣每室行禮,並如上儀。降階,樂作,復位,樂止。禮直官次引終獻詣盥洗、及升殿行禮,並如亞獻之儀,降復位。次引太祝徹籩豆少移故處,樂作,卒徹,樂止。俱復位。禮直官曰:「賜胙。」贊者承傳曰:「賜胙,再拜。」在位者皆再拜。禮直官引太祝、宮闈令奉神主,太祝搢笏,納帝主於匱,奉入祏室,執笏,退復位。次引宮闈令納后主於匱,奉入祏室,並如上儀,退復位。禮直官、贊者引行事、執事官各
 就位,奉禮曰:「再拜。」贊者承傳,應在位官皆再拜。禮直官、贊者引百官次出,大樂令帥工人次出,太官令帥其屬,徹禮饌,次引監祭御史詣殿監視卒徹,訖,還齋所。太廟令闔戶以降。太常藏祝版於匱。光祿以胙奉進,監祭御史就位展視,光祿卿望闕再拜,乃退。其七祀,夏灶、中霤,秋門、厲,冬行,鋪設祭器,入實酒饌,俟終獻將升獻,獻官行禮,並讀祝文。每歲四孟月並臘五享,並如上儀。



\end{pinyinscope}