\article{列傳第四十九}

\begin{pinyinscope}

 ○古里甲石倫內族訛可撒合輦強伸烏林答胡土內族思烈紇石烈牙吾塔



 古里甲石倫,隆安人。以武舉登第。為人剛悍,頗自用,所在與人不合。宣宗以其勇善戰,每任用之。貞祐二年,累遷副提控、太原府判官,與從宜都提控、振武軍節度使完顏蒲刺都議拒守不合,措置乖方,敵因大入,幾不可
 禦。既乃交章論列,以自辨其無罪,上惡其不和,詔分統其兵。未幾,遷同知太原府事。奏請招集義軍,設置長校,各立等差。都統授正七品職,副統正八品,萬戶正九品,千戶正班任使,謀克雜班。仍三十人為一謀克,五謀克為一千戶,四千戶為一萬戶,四萬戶為一副統,兩副統為一都統,外設一總領提控。制可。



 四年,遷河東宣撫副使,上章言宣撫使烏古論禮不肯分兵禦敵,且所行多不法。詔禮罷職,石倫遷絳陽軍節度使,權經略使,尋知延安府事、兼鄜延路兵馬都總管。大元兵圍忻州,石倫率兵往援,以兵護其民入太原,所保軍民甚眾。興定元
 年七月,改河平軍節度、兼衛州管內觀察使,詔諭曰:「朕初謂汝勇果,為國盡力,故倚以濟事。尋聞汝嗜酒不法,而太原知府烏古論德升亦屢嘗為朕言之,然皆瑣屑,乃若不救汾州,豈細事哉?有司議罪如此,汝其悉之,益當戮力,以掩前過。」是年十一月,遷鎮西軍節度使、兼嵐州管內觀察使、行元帥府事。



 二年四月,石倫言:「去歲北兵破太原,游兵時入嵐州境,而官民將士悉力扞禦,卒能保守無虞。向者河東內郡,皆駐以精甲,實以資儲,視邊城尤為完富,然兵一至,相繼淪沒。嵐兵寡而食不足,惟其上下協同,表裏相應,遂獲安帖。當大軍初入,郡縣
 倉皇,非此帥府控制,則庾、管、保德、岢嵐、寧化皆不可知矣。今防秋不遠,乞朝廷量加旌賞,務令益盡心力,易以鎮守。」詔有功者各遷官一級,仍給降空名宣敕,令樞密院遣授之。



 三年二月,石倫奏:「向者并、汾既破,兵入內地,臣謂必攻平陽,平陽不守,將及潞州,其還當由龍州谷以入太原。故臣嘗請兵欲扼其歸路,朝廷不以為然,既而皆如臣所料。始敵入河東時,郡縣民皆攜老幼徙居山險,後雖太原失守,而眾卒不從,其意謂敵不久留,且望官軍復至也。今敵居半歲,遣步騎擾諸保聚,而官軍竟無至者,民其能久抗乎。夫太原,河東之要郡;平陽,陜
 西,河南之籓籬也。若敵兵久不去,居民盡從,屯兵積糧以固基本,而復擾吾郡縣未殘者,則邊城指日皆下矣。北路不守,則南路為邊,去陜西、河南益近,臣竊憂之,故復請兵以圖戰守。而樞府檄臣,并將權太原治中郭遹祖、義軍李天祿等萬餘人,就其糧五千石,會汾州權元帥右都監抹捻胡剌復太原。臣召遹祖,欲號令其眾,遹祖不從。尋得胡剌報曰:『嘗問軍數於遹祖,但稱天祿等言之,未嘗親閱。問糧,則曰散在數處。』蓋其情本欲視朝廷以己有兵糧,冀或見用,以取重職,不可指為實用也。雖然,臣已遣提控石盞吾里忻等領軍以往矣。但敵勢
 頗重,而往者皆新集白徒,絕無精銳,恐不能勝。乞於河南、陜西量分精兵,以增臣力,仍令陜西州郡近河東者給之資糧,更令南路諸軍綴敵之南,以分其勢,如此庶幾太原可復也。」詔陜西、河東行省分糧與之,請兵之事,以方伐宋不從。



 三月,石倫復上言曰:「頃者大兵破太原,招民耕稼,為久駐之基。臣以太原要鎮,所當必爭,遣提控石盞吾里忻引官兵義兵共圖收復。又以軍士有功者宜速賞之,故擬令吾里忻得注授九品之職,以是請於朝,而執政以為賞功罰罪皆須中覆。夫河東去京師甚遠,移報往返不暇數十日,官軍皆敗亡之餘,鋒銳略
 盡,而義兵亦不習行陣,無異烏合,以重賞誘之猶恐不為用,況有功而久不見報乎。夫眾不可用則不能退敵,敵不退則太原不可復,太原不可復則平陽之勢日危,而境土日蹙矣。今朝廷抑而不許,不過慮其濫賞耳。借使有濫賞之弊,其與失太原之害孰重?」於是詔從其請,自太原治中及他州從七品以下職、四品以下散官,並聽石倫遷調焉。



 是月,石倫復言:「日者遣軍潛搗敵壘,欲分石州兵五百權屯方山,剿殺土寇,且備嵐州,而同知蒲察桓端拒而不發。又召同知寧邊軍節度使姚里鴉鶻與之議兵,竟不聽命。近領兵將取太原,委石州刺史
 納合萬家權行六部,而辭以他故,幾誤軍糧。約武州刺史郭憲率所領併進,憲亦不至。臣猥當方面之任,而所統官屬並不稟從,乞朝廷嚴為懲誡,庶人知職分,易以責辦。」宰臣惡之,乃奏曰:「桓端、鴉鶻已經奏改,無復可議。石倫身兼行部,不自規畫,而使萬家往來應給,石州無人,恐亦有失。武州邊郡,正當兵衝,使憲率軍離城,敵或乘之,孰與守禦?萬家等不從,未為過也。」上以為然,因遣諭石倫曰:「卿嘗行院于歸德,衛州防備之事,非不素知,乃屢以步騎為請何耶?比授卿三品,且數免罪譴卿,嘗自誓以死報國,今所為如此,豈報國之道哉!意謂河南
 之眾必不可分,但圖他日得以藉口耳。卿果赤必為國,盡力經畫,亦足自效。萬家等若必懲戒,彼中誰復可使者?姑為容忍可也。」



 閏三月,石倫駐兵太原之西,俟諸道兵至進戰,聞脅從人頗有革心,上言于朝,乞降空名宣敕、金銀符,許便宜遷注,以招誘之。上從其請,並給付之,仍聽注五品以下官職。



 六月,保德州振威軍萬戶王章、弩軍萬戶齊鎮殺其刺史孛術魯銀術哥,仍滅其家,脅官吏軍民同狀白嵐州帥府,言銀術哥專恣慘酷,私造甲仗,將謀不軌。石倫密令同知州事把蒲剌都圖之,蒲剌都乃與兵吏置酒召章等欽,擒而族誅之。至是,朝廷
 命行省胥鼎量宜遷賞,仍令蒲剌都攝州事,撫安其眾焉。



 六月,遷金安軍節度使,行帥府事於葭州。時鄜州元帥內族承立慮夏人入寇,遣納合買住以兵駐葭州,石倫輒分留買住兵千八百人,令以餘兵屯綏德,而後奏之。有司論罪當絞,既而遇赦,乃止除名。元光元年,起為鄭州同知防禦使,與防禦使裴滿羊哥部內酤酒不償直,皆除名。三月,上諭元帥監軍內族訛可曰:「石倫今以罪廢,欲再起之,恐生物議,汝軍前得無用之乎。此人頗善戰,果可用便當遣去。古亦有白衣領職者,渠雖除名,何害也。」十月,大元兵圍青龍堡,詔以石倫權左都監,將
 兵會上黨公、晉陽公往援之。兵次彈平寨東三十里,敵兵梗道不得進,會青龍堡破,召還。既而復以罪免。



 正大八年,大兵入河南,州郡無不下者,朝議以權昌武軍節度使粘葛仝周不知兵事,起石倫代之。石倫初赴昌武,詔諭曰:「卿先朝宿將,甚有威望,故起拜是職。元帥蘇椿、武監軍皆曉兵事,今在昌武,宜與同議,勿復不睦失計也。」時北兵已至許,石倫赴鎮,幾為游騎所獲。數日,知兩省軍敗,潰軍踵來。有忠孝軍完顏副統入城,兩手皆折,血污滿身,州人憂怖不知所出。石倫遣歸順軍提控嵐州人高珪往斥候,珪因持在州軍馬糧草數目奔大元
 軍,仍告以城池深淺。俄大兵至城下,以鳳翔府韓壽孫持檄招降,言三峰敗狀。石倫、蘇椿不詰問即斬之市中。既而武監軍偏裨何魏輩開東門,內族按春開南門,夾谷太守開西門。大元軍入城,擒蘇椿,問以大名南奔之事,椿曰:「我本金朝人,無力故降,我歸國得為大官,何謂反耶!」大將怒其不屈,即殺之。石倫投廨後井中,仝周自縊州廨。武監軍者初不預開門之謀,何魏輩欲保全之,故言於大將曰:「監軍令我輩獻門。」然亦怒其不迎軍而降,亦殺之。



 仝周名暉,字子陽,策論進士,興定間為徐州行樞密院參議官,上章言:「惟名與器不可假人,自古帝
 王靡不為重。今之金銀牌,即古符節也,其上有太祖御畫,往年得佩者甚難,兵興以來授予頗濫,市井道路黃白相望,恐非所以示信於下也。乞寶惜之,有所甄別。」上以語宰臣,而丞相高琪等奏:「時方多難,急於用人,駕馭之方,此其一也,如故為便。」



 蘇椿,大名人,初守大名,歸順於大元,正大二年九月,自大名奔汴,詔置許州,至是,見殺。



 完顏訛可,內族也。時有兩訛可,皆護衛出身,一曰「草火訛可」,每得賊,好以草火燎之,一曰「板子訛可」,嘗誤以宮中牙牌報班齊者為板子,故時人各以是目之。



 正大八
 年九月,大兵攻河中。初,宣宗議遷都,朝臣謂可遷河中:「河中背負關陜五路,士馬全盛,南阻大河,可建行臺以為右翼。前有絳陽、平陽、太原三大鎮,敵兵不敢輕入。應三鎮郡縣之民皆聚之山寨,敵至則為晝攻夜劫之計。屯重軍中條,則行在有萬全之固矣。」主議者以河中在河朔,又無宮室,不及汴梁,議遂寢。宣宗既遷河南,三二年之後,詔元帥都監內族阿祿帶行帥府事。阿祿帶恇怯不能軍,竭民膏血為浚築之計。未幾,絳州破,阿祿帶益懼,馳奏河中孤城不可守。有旨親視,果不可守則棄之,無至資敵。阿祿帶遂棄河中,燒民戶官府,一二日而
 盡。尋有言河中重鎮,國家基本所在,棄之為失策,設為敵人所據,則大河之險我不得專恃矣。宣宗悔悟,繫阿祿帶同州獄,累命完復之,隨守隨破。至是,以內族兩訛可將兵三萬守之。大兵謀取宋武休關。未幾,鳳翔破,睿宗分騎兵三萬入散關,攻破鳳州,徑過華陽,屠洋州,攻武休關。開生山,截焦崖,出武休東南,遂圍興元。興元軍民散走,死於沙窩者數十萬。分軍而西,西軍由別路入沔州,取大安軍路開魚鱉山,撤屋為筏,渡嘉陵江入關堡,並江趨葭萌,略地至西水縣而還。東軍止屯興元、洋州之間,遂趨饒峰。宋人棄關不守,大兵乃得入。



 初,大兵
 期以明年正月合南北軍攻汴梁,故自將攻河中。河中告急,合打蒲阿遣王敢率步兵一萬救之。十二月,河中破。初,河中主將知大兵將至,懼軍力不足,截故城之半守之。及被攻,行帳命築松樓高二百尺,下瞰城中,土山地穴百道並進。至十一月,攻愈急。自王敢救軍至,軍士殊死鬥,日夜不休,西北樓櫓俱盡,白戰又半月,力盡乃陷。草訛可戰數十合始被擒,尋殺之。板訛可提敗卒三千奪船走,北兵追及,鼓噪北岸上,矢石如雨。數里之外有戰船橫截之,敗軍不得過,船中有齎火砲名「震天雷」者連發之,砲火明,見北船軍無幾人,力斫橫船開,得至
 潼關,遂入閿鄉。尋有詔赦將佐以下,責訛可以不能死,車載入陜州,決杖二百。識者以為河中城守不下,德順力竭而陷,非戰之罪,故訛可之死,人有冤之者。



 初,訛可以元帥右監軍、邠涇總帥、權參知知事,奉旨於邠、涇、鳳翔往來防秋。奉御六兒監戰,於訛可為孫行,而訛可動為所制,意頗不平,漸生猜隙。七年九月,召赴京師,改河中總帥,受京兆節制。此時六兒同赴召,謂訛可奉旨往來防秋,而乃畏怯避遠,正與朝旨相違,上意頗罪訛可。及河中陷,苦戰力盡,而北兵百倍臨之,人謂雖至不守猶可以自贖,竟杖而死,蓋六兒先入之言主之也。



 劉祁
 曰:「金人南渡之後,近侍之權尤重。蓋宣宗喜用其人以為耳目,伺察百官,故奉御輩採訪民間,號『行路御史』,或得一二事即入奏之,上因以責臺官漏泄,皆抵罪。又方面之柄雖委將帥,又差一奉御在軍中,號曰『監戰』,每臨機制變,多為所牽制,遇敵輒先奔,故師多喪敗。」哀宗因之不改,終至亡國。



 論曰:古里甲石倫善戰而好犯法,故見廢者屢,晚起為將,卒死於難。金運將終,又用數奇之李廣,其乏絕不亦宜乎。草訛可力戰而死,板訛可亦力戰,不死於陣而死於刑,論者以為有近侍先入之言。夫以褻御治軍,既掣
 之肘,又信其讒以殺人,金失政刑矣。唐之亡,坐以近侍監軍,金蹈其轍,哀哉。



 撒合輦,字安之,內族也。宣宗朝,累遷同簽樞密院事。元光二年十二月庚寅夜,宣宗病篤,英王盤都先入侍,哀宗後至,東華門已閉,聞英王在宮,遣樞密院官及東宮親衛軍總領移剌蒲阿勒兵東華門,都點檢駙馬都尉徒單合住奏中宮,得旨,領符鑰啟門。合住見上,上命撒合輦解合住刀佩之,哀宗遂入,明日即位,由是見親信。正大元年正月庚申,以輦同判大睦親府事,兼前職。刑部完顏素蘭言:「把胡魯策功第一,非超拜右丞相無以
 酬之。」然同功數人亦有不次之望,故胡魯之命中輟,輦猶升二品云



 四年,大元既滅西夏,進軍陜西。四月丙申,召尚書溫迪罕壽孫、中丞烏古孫卜吉、祭酒裴滿阿虎帶、直學士蒲察世達、右司諫陳規、監察烏古論四和完顏習顯、同判睦親府事撒合輦同議西事,上曰:「已諭合達盡力決一戰矣。」群臣多主和事,獨輦力破和議,語在《陳規傳》。



 八月,朝廷得清水之報,令有司罷防城及修城丁壯,凡軍需租調不急者權停。初,聞大兵自鳳翔入京兆,關中大震,以中丞卜吉、祭酒阿忽帶兼司農卿,簽民兵,督秋稅,令民入保為避遷計。當時議者以謂大兵未
 至而河南先亂,且曰:「御史監察城洛陽,治書供帳北使,中丞下兼司農簽軍督稅,臺政可知矣。」至是,上謂撒合輦曰:「諺云水深見長人。朝臣或欲我一戰,汝獨言當靜以待之,與朕意合,今日有太平之望,皆汝謀也。先帝嘗言汝可用,可謂知人矣。」



 未幾,右拾遺李大節、右司諫陳規言撒合輦諂佞納賄及不公事,奏帖留中不報。明惠皇后嘗傳旨戒曰:「汝諂事上,上之騎鞠皆汝所教。」尉忻亦極言之,上頗悟,出為中京留守、兼行樞密院事。初,宣宗改河南府為金昌府,號中京,又擬少室山頂為御營,命移剌粘合築之,至是撒合輦為留守。



 九年正月,北兵
 從河清徑渡,分兵至洛,出沒四十餘日。二月乙亥,立砲攻城。洛中初無軍,得三峰潰卒三四千人,與忠孝軍百餘守禦。時輦疽發于背,不能軍,同知溫迪罕斡朵羅主軍務,有大事則就輦稟之。三月甲申,忠孝軍百餘騎入使宅,強擁輦出奔,輦不得已從之,并以官屬及其子自隨,才出南裏城門,城上軍覺,閉之甕城中,矢石亂下,人馬多死傷。輦知不能出,仰呼求救,軍士知出奔非輦意,以繩引而上,送入其宅,不敢出。鎮撫官縛出奔之黨,欲殺之,已斬三人,輦親為乞命,得免。乙酉,斡朵羅齎金帛出北門,如前日巡城犒軍之狀,既出即沿城而西,直出
 外壕,城上人呼曰:「同知講和去矣。」軍士及將領隨而下者三四百人。少之,輦傳令云:「同知叛降,有再下城者斬。」凡斬三四人,乃定。丙戌夜,城東北角破,輦奪南門出不得,投濠水死。已而,大兵退,強伸復立帥府。



 強伸,本河中射糧軍子弟,貌極寢陋,而膂力過人。興定初,從華州副都統安寧復潼關,以勞任使,嘗監郃陽醋。後客洛下,選充官軍,戍陜鐵嶺,軍潰被虜,從都尉兀林答胡土竄歸中京。時中京已破,留守兼行樞密院使內族撒合輦死之,元帥任守真復立府事,以便宜署伸警巡使。後守真率部曲軍從行省思烈入援,鄭州之敗,守
 真死。天興元年八月,中京人推伸為府簽事,領所有軍二千五百人,傷殘老幼半之。甫三日,北兵圍之,東西北三面多樹大砲。伸括衣帛為幟,立之城上,率士卒赤身而戰,以壯士五十人往來救應,大叫,以「憨子軍」為號,其聲勢與萬眾無異。兵器已盡,以錢為鏃,得大兵一箭,截而為四,以筒鞭發之。又創遏砲,用不過數人,能發大石於百步外,所擊無不中。伸奔走四應,所至必捷。得二駝及所乘馬皆殺之,以犒軍士,人不過一啖,而得者如百金之賜。九月,大兵退百里外。閏月,復攻,兵數倍於前。又一月,不能拔。事聞,哀宗降詔褒諭,以伸為中京留守、元
 帥左都監、世襲謀克、行元帥府事。



 十月,參知政事內族思烈自南山領軍民十餘萬入洛,行省事。二年二月,伸建一堂於洛川驛之東,名曰「報恩」,刻詔文於石,願以死自效。三月,中使至,以伸便宜從事。



 是月,大兵自汴驅思烈之子於東門下,誘思烈降。思烈即命左右射之,既而知崔立之變,病不能語而死。總帥忽林答胡土代行省事,伸行總帥府事,月餘糧盡,軍民稍稍散去。



 五月,大兵復來,陣於洛南,伸陣水北。有韓帥者匹馬立水濱,招伸降,伸謂帥曰:「君獨非我家臣子耶?一日勤王,猶遺令名於世,君既不能,乃欲誘我降耶?我本一軍卒,今貴為留
 守,誓以死報國耳。」遂躍而射之。帥奔陣,率步卒數百奪橋,伸軍一旗手獨出拒之,殺數人,伸乃手解都統銀符與之佩,士卒氣復振。初,築戰壘於城外四隅,至五門內外皆有屏,謂之迷魂牆。大兵以五百騎迫之,伸率卒二百鼓噪而出,大兵退。



 六月,行省胡土率眾走南山,鷹揚都尉獻西門以降,伸知城不能守,率死士數十人突東門出,轉戰至偃師,力盡就執。載以一馬,擁迫而行。伸宛轉不肯進,強掖之,將見大帥塔察。及中京七里河,伸語不遜,兵卒相謂曰:「此人乖角如此,若見大帥,其能降乎?不若殺之。」因好語誘之曰:「汝能北面一屈膝,吾貸汝命。」
 伸不從,左右力持使北面,伸拗頭南向,遂殺之。



 烏林答胡土。正大九年正月戊子,北兵以河中一軍由洛陽東四十里白坡渡河。白坡故河清縣,河有石底,歲旱水不能尋丈。國初以三千騎由此路趨汴,是後縣廢為鎮,宣宗南遷,河防上下千里,常以此路為憂,每冬日命洛陽一軍戍之。河中破,有言此路可徒涉者,已而果然。北兵既渡,奪河陰官舟以濟諸軍。時胡土為破虜都尉,戍潼關,以去冬十二月被旨入援,至偃師,聞白坡徑渡之耗,直趨少室,夜至少林寺。時登封縣官民已遷太平頂御寨。明日,胡土使人紿縣官云:「吾軍中家屬輜重
 欲留此山,即率兵赴汴京。」因攝縣官下山,使之前導,一軍隨之而上。山既險固,糧亦充足,遂有久住之意。尋縱軍下山劫掠居民,甚於盜賊,旁近一二百里無不被害。胡土畏變,知而不禁,又所劫牛畜糧糗,亦分有之。



 七月,恆山公武仙、參政思烈兩行省軍,屯登封城南大林下,遣人約之入京。胡土百計不肯下,不得已,乃分其軍四千,與思烈俱東。八月三日,兩行省軍潰於中牟,胡土狼狽上山,殘卒三二十人外偏裨無一人至者。十二月,思烈自留山行省於中京,徵兵同保洛陽,又遷延不行。思烈以檄來,言:「若依前逗留,自有典憲,吾不汝容矣。」胡土
 懼,乃挈妻子及軍往中京,留其半山上以為巢穴。天興二年三月,思烈病卒,留語胡土代行省事。六月,敵勢益重,強伸方盡力戰禦,而胡土即領輕騎、挈妻子棄城南奔,遂失中京。



 初,胡土在太平頂既顧望不進,又懼人議己,乃出榜募人為救駕軍,云:「一旅之眾可以興復國家,諸人有能奮發許國捐軀者,豈不濟大事乎!」於是,不逞之徒隨募而出,得澤人緝麻觜、武錄事等二十餘人,促令赴京。行及盧店,即行劫,械至,杖之二百,人無不竊笑。既而走蔡州,上召見慰問,而心薄之。會宋人攻唐州,元帥烏古論黑漢屢遣人告急,即命胡土領忠孝軍百人,
 就征西山招撫烏古論換住、黃八兒等軍赴之。胡土率兵至唐,宋人斂避,縱其半入城,夾擊之,胡土大敗,僅存三十騎以還,換住死焉。



 既而以胡土為殿前都點檢,罷權參政。大兵圍蔡,分軍防守,胡土守西面。十一月,胡土之奴竊其金牌,夜縋城降,朝士喧播謂胡土縱之往,將有異志。胡土聞之,內不自安,乞解軍職。上慰之曰:「卿父子昆弟皆為帥臣,受恩不為不厚,顧肯降耶。且卿向在洛陽不即降,而千里遠來降於蔡,豈人情也哉。聞卿遇奴太察,且其衣食不常給之,此蓋往求溫飽耳,卿何慊焉。」因賜饌以安其心。初,胡土罷機政,頗有怨言,左右勸
 上誅之,上不聽。及令守西城,尤怏怏不樂,至是始感恩無他慮矣。



 尋以總帥孛術魯婁室與胡土皆權參政,婁室與右丞仲德同事,胡土防守如故,復以都尉承麟為東面元帥權總帥。先是,攻東城,婁室隨機備禦。二日移攻南城,烏古論鎬易之,砲擊城樓幾仆,右丞仲德率軍救援,乃罷攻。俄而四面受敵,仲德艱於獨援,遂薦承麟代婁室東面,而乞與婁室同救應。初,胡土失外城,頗慚恨,聲言力小不能令眾,仲德亦薦之,故有是命。蔡城破,投汝水死。



 贊曰:撒合輦本以佞進,烏林答胡土戰陣不武,付以孤
 城,望其捍禦大難,豈得為知人乎。強伸一射糧卒耳,及授以兵,乃能應變制勝,遠過二人,力盡乃斃,猶有烈丈夫之風焉。古人有言:「四郊多壘,拔士為將。」使金運未去,伸足以建功名矣夫。



 內族思烈,南陽郡王襄之子也。資性詳雅,頗知書史。自五六歲入宮充奉御,甚見寵幸,世號曰「自在奉御」。當宣宗入承大統,胡沙虎跋扈,思烈尚在髫齔,嘗涕泣跪抱帝膝致說曰:「願早誅權臣,以靖王室。」帝急顧左右掩其口。自是帝甚器重之。後由提點近侍局遷都點檢。天興元年,汴京被圍,哀宗以思烈權參知政事,行省事于鄧
 州。會武仙引兵入援,於是思烈率諸軍發自汝州,過密縣,遇大元兵,不用武仙阻澗之策,遂敗績于京水,語在《武仙傳》。中京留守、元帥左監軍任守真死之。上聞,罷思烈行省之職,以守中京。無何,大兵圍中京未能下,崔立遣人監思烈子於中京城下,招之使降。思烈不顧,令軍士射之。既而知崔立已以汴京歸順,病數日而死。初,思烈會武仙等軍入援,即與仙論議不同,仙以思烈方得君,每假借之。思烈謂仙本無入援意,特以朝廷遣一參政召兵,迫於不得已乃行耳。然仙知兵,頗以持重為事。思烈急於入京,不聽仙策,於是左右司員外郎王渥乃
 勸思烈曰:「武仙大小數百戰,經涉不為不多,兵事當共議。」思烈疑其與仙有謀,幾斬之,渥自以無愧於內,不懼也。已而思烈果敗,渥歿於陣。



 渥字仲澤,後名仲澤,太原人。性明俊不羈,博學善談論,工尺牘,字畫清美,有晉人風。少游太學,長於詞賦,登興定二年進士第。為時帥奧屯邦獻、完顏斜烈所知,故多在兵間。後辟寧陵令,有治蹟,入為尚書省令史。因使宋至揚州,應對敏給,宋人重之。及還,為太學助教,轉樞密院經歷官,俄遷右司都事,稍見信用。及思烈往鄧州,以渥為左右司員外郎,從行。



 贊曰:思烈夙惠,請誅權奸以立主威,有甘羅、辟疆之風,
 所謂「茂良不必父祖」者也。中京之圍,崔立脅其子使招之降,不顧而趣射之,何愧乎橋玄。至如不從武仙之言,以至於敗,此蓋時人因惜王仲澤之死而有是言,仙無入援之意則非誣也。



 紇石烈牙吾塔,一名志。本出親軍,性剛悍喜戰。貞祐間,僕散安貞為山東路宣撫使,以牙吾塔為軍中提控。是時,山東群盜蜂起,安貞遣牙吾塔破巨蒙等四堌,又破馬耳山砦,殺劉二祖賊黨四千餘人,降賊八千,虜其偽宣差程寬、招軍大使程福,又降脅從民三萬餘人。貞祐四年六月,積功累遷欄通渡經略使。十月,為元帥左都
 監。十二月,行山東西路兵馬都總管府事,兼武寧軍節度使、徐州管內觀察使。



 興定二年正月,宋兵萬餘攻泗州,牙吾塔赴援,至臨淮,遇宋人三百,掩殺殆盡。及泗州,宋兵八千圍甚急,督眾進戰,大破之,溺水死者甚眾,獲馬三百餘匹,俘五十餘人。又圍盱眙,宋人閉門堅守,不敢出。以騎兵分掠境內,而時遣羸卒薄城誘之。宋人出騎數百來拒,牙吾塔麾兵佯北,發伏擊之,斬首二百。宋人復出步騎八千來援,合擊敗之,殺一太尉,斬首三百。尋獲覘者,稱青平宋兵甚眾,將救盱眙。牙吾塔移兵赴之,宋兵步騎七千人突出,兵少卻,旋以輕騎扼其後。初
 逗留不與戰,縱之走東南,薄諸河,斬首千餘,溺死者無算,獲馬牛數百,甲仗以千計。師還,遇宋兵三千於連塘村,斬首千餘級,俘五十人,獲馬三十五疋,宣宗以其有功,賜金帶一。三年正月,敗宋人於濠州之香山村。二月,又敗之於滁州,斬首千級。拔小江寨,殺統制王大篷等,斬三萬,俘萬餘人。又拔輔嘉平山寨,斬首數千,俘五百餘人,獲馬牛數百,糧萬斛。三月,提控奧敦吾里不大敗宋人于上津縣,兵還至濠州,宋人以軍八千拒戰,牙吾塔迎擊敗之,獲馬百餘疋。



 五年正月,上以紅襖賊助宋為害,邊兵久勞苦,詔牙吾塔遺宋人書求戰,略曰:「宋與我
 國通好,百年於此,頃歲以來,納我叛亡,絕我貢幣,又遣紅襖賊乘間竊出,跳梁邊疆,使吾民不得休息。彼國若以此曹為足恃,請悉眾而來,一決勝負,果能當我之鋒,沿邊城邑當以相奉。度不能,即宜安分保境,何必狐號鼠竊、乘陰伺夜以為此態耶?且彼之將帥亦自受鉞總戎,而臨敵則望風遠遁,被攻則閉壘深藏,逮吾師還,然後現形耀影以示武。夫小民尚氣,女子有志者猶不爾也,切為彼國羞之。」



 先是,宋將時青襲破泗州西城。二月,牙吾塔將兵取之,宋兵拒守甚力,乃募死士以梯沖並進,大敗宋兵。時青乘城指麾,射中其目,遂拔眾南奔。乃
 陳兵橫絕走路擊之,宋兵大潰,遂復泗州西城。三月,復出兵宋境,以報其役,破團山、賈家等諸寨,進逼濠州。牙吾塔慮州人出拒,躬率勁兵逆之,遇邏騎二百於城東,擊殺過半。會偵者言前路芻糧甚艱,乃西掠定遠,由渦口而還。九月,又率兵渡淮,大破宋兵於團山,詔遷官升職有差。



 元光元年五月,以京東便宜總帥兼行戶、工部事,上因謂宰臣曰:「牙吾塔性剛,人皆畏之,委之行部,無不辦者。至於御下亦頗有術,提控有胡論出者,渠厚待之,常同器而食,其人感奮,遂以戰死。」英王守純曰:「凡為將帥,駕馭人材皆當如此。」上曰:「然。」未幾,宋人三千潛渡
 淮,至聊林,盡伐隄柳,塞汴水以斷吾糧道。牙吾塔遣精甲千餘破之,獲其舟及渡者七百人,汴流由是復通。



 二年四月,上言:「賞罰國之大信,帝王所以勸善而懲惡,其令一出,不可中變。向官軍戰歿者皆廩給其家,恩至厚也。臣近抵宿州,乃知例以楮幣折支,往往不給,至于失所。此殆有司出納之吝,不能奉行朝廷德意之過也。自今願支本色,令得贍濟。」以糧儲方艱,詔有司給其半。



 紅襖賊寇壽、潁,剽掠數日而去。牙吾塔聞之,率兵渡淮,偵知朱村、孝義村有賊各數百,分兵攻之,連破兩柵,及焚其村塢數十。還遇宋兵數百,陣淮南岸,擊殺其半,尋有
 兵千餘自東南來追,復大敗之。



 先是,納合六哥殺元帥蒙古綱,據邳州以叛。十月,牙吾塔圍之,焚其樓櫓,斬首百餘。於是,宋鈐轄高顯、統制侯進、正將陳榮等知不能守,共誅六哥,持其首縋城降。六哥既誅,眾猶拒守,方督兵進攻,宋總領劉斌、提控黃溫等縛首亂顏俊、戚誼、完顏乞哥,及梟提控金山八打首,遣其校馬俊、吳珪來獻。既而紅襖監軍徐福、統制王喜等亦遣其總領孫成、總押徐琦納款。劉斌等遂率軍民出降,牙吾塔入城,撫慰其眾,各使安集,又招獲紅襖統制十有五人,將官訓練百三十有九人。十一月,遣人來報,仍函六哥首以獻。宣
 宗大喜,進牙吾塔官一階,賜金三百兩、內府重幣十端,將士遷賞有差。



 正大三年十一月,北兵猝入西夏,攻中興府甚急。召陜西行省及陜州、靈寶二總帥訛可、牙吾塔議兵。又詔諭兩省曰:「倘邊方有警,內地可憂,若不早圖,恐成噬臍。旦夕事勢不同,隨機應變,若逐旋申奏,恐失事機,並從行省從宜規畫。」



 四年,牙吾塔復取平陽,獲馬三千。是歲,大兵既滅夏國,進攻陜西德順、秦州、清水等城、遂自鳳翔入京兆,關中大震。五年,圍慶陽。六年十月,上命陜省以羊酒及幣赴慶陽犒北帥,為緩師計。北中亦遣唐慶等往來議和,尋遣斡骨欒為小使,徑來行
 省。十二月,詔以牙吾塔與副樞蒲阿權簽樞密院事,內族訛可將兵救慶陽。七年正月,戰于大昌原,慶陽圍解。詔以牙吾塔為左副元帥,屯京兆。初,斡骨欒來,行省恐洩事機,因留之。蒲阿等既解慶陽之圍,志氣驕滿,乃遣還,謂使者曰:「我已準備軍馬,可戰鬥來。」語甚不遜,斡骨欒以此言上聞,太宗皇帝大怒,至應州,以九日拜天,即親統大兵入陜西。八年,遷居民於河南,棄京兆東還。五月,至閿鄉,得寒疾,汗不出,死。



 「塔」亦作「太」,亦曰「牙忽帶」,蓋女直語,無正字也。是歲九月,國信使內族乘慶自北使還,始知牙吾塔不遜激怒之語,且言慶等在旁心魄震
 蕩,殆不忍聞。當時以帥臣不知書,誤國乃爾。



 塔為人鷙狠狼戾,好結小人,不聽朝廷節制。嘗入朝,詣省堂,詆毀宰執,宰執亦不敢言,而上倚其鎮東方,亦優容之。尤不喜文士,僚屬有長裾者,輒以刀截去。又喜凌侮使者,凡朝廷遣使來,必以酒食困之。或辭以不飲,因併食不給,使餓而去。司農少卿張用章以行戶部過宿,塔飲以酒。張辭以寒疾,塔笑曰:「此易治耳。」趨左右持艾來,臥張於床,灸之數十。又以銀符佩妓,屢往州郡取賕,州將之妻皆遠迎迓,號「省差行首」,厚賄之。御史康錫上章劾之,且曰:「朝廷容之,適所以害之。欲保全其人,宜加裁制。」朝廷
 竟不治其罪,以屢敗宋兵,威震淮、泗。好用鼓椎擊人,世呼曰「盧鼓椎」,其名可以怖兒啼,大概如呼「麻胡」云。



 有子名阿里合,世目曰「小鼓椎」,嘗為元帥,從哀宗至歸德,與蒲察官奴作亂,伏誅。



 康錫,字伯祿,趙州人。至寧元年進士。正大初,由省掾拜御史,劾侯摯、師安石非相材,近侍局宗室撒合輦聲勢熏灼,請托公行,不可使在禁近,時論韙之。轉右司都事、京南路司農丞,為河中路治中。河中破,從時帥率兵南奔,濟河,船敗死。為人氣質重厚,公家之事知無不為,與雷淵、翼禹錫齊名。



 贊曰:金自胡沙虎、高琪用事,風俗一變,朝廷矯寬厚之
 政,好為苛察,然為之不果,反成姑息。將帥鄙儒雅之風,好為粗豪,然用非其宜,終至跋扈。牙吾塔戰勝攻取,威行江、淮,而矜暴不法,肆侮王人,此豈可制者乎?棄陜而歸,死於道途,殆其幸歟!其子效尤,竟陷大僇,君子乃知康錫之言不為過也。



\end{pinyinscope}