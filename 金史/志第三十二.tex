\article{志第三十二}

\begin{pinyinscope}

 選舉一



 ○進士諸科律科經童科制舉武舉試學士院官司天醫學試科



 自三代鄉舉里選之法廢,秦、漢以來各因一代之宜,以盡一時之才,茍足於用即已,故法度之不一,其來遠矣!在漢之世,雖有賢良方正諸科以取士,而推擇為吏,由是以致公卿,公卿子弟入備宿衛,因被寵遇,以位通顯。魏、晉而下互有因革,至於唐、宋,進士盛焉。當時士君子之進,不由是塗則自以為慊,此由時君之好尚,故人心
 之趣向然也。遼起唐季,頗用唐進士法取人,然仕於其國者,考其致身之所自,進士纔十之二三耳!金承遼後,凡事欲軼遼世,故進士科目兼採唐、宋之法而增損之。其及第出身,視前代特重,而法亦密焉。若夫以策論進士取其國人,而用女直文字以為程文,斯蓋就其所長以收其用,又欲行其國字,使人通習而不廢耳。終金之代,科目得人為盛。諸宮護衛、及省臺部譯史、令史、通事、仕進皆列於正班,斯則唐、宋以來之所無者,豈非因時制宜,而以漢法為依據者乎?金治純駁,議者於是每有別焉。宣宗南渡,吏習日盛,苛刻成風,殆亦多故之秋,急
 於事功,不免爾歟。自時厥後,仕進之歧既廣,僥倖之俗益熾,軍伍勞效,雜置令祿,門蔭右職,迭居朝著,科舉取士亦復泛濫,而金治衰矣!原其立經陳紀之初,所為升轉之格,考察之方,井井然有條而不紊,百有餘年才具不乏,豈非其效乎?奉詔作《金史》,志其《選舉》,因得而詳論之,司天、太醫、內侍等法歷代所有,附著于斯。鬻爵、進納,金季之弊莫甚焉,蓋由財用之不足而然也,特載《食貨志》。



 金設科皆因遼、宋制,有詞賦、經義、策試、律科、經童之制海陵天德三年,罷策試科。世宗大定十一年,創設女直
 進士科,初但試策,後增試論,所謂策論進士也。明昌初,又設制舉宏詞科,以待非常之士。故金取士之目有七焉。其試詞賦、經義、策論中選者,謂之進士。律科、經童中選者,曰舉人。凡養士之地曰國子監,始置於天德三年,後定制,詞賦、經義生百人,小學生百人,以宗室及外戚皇后大功以上親、諸功臣及三品以上官兄弟子孫,年十五以上者入學,不及十五者入小學。大定六年始置太學,初養士百六十人,後定五品以上官兄弟子孫百五十人,曾得府薦及終場人二百五十人,凡四百人。府學亦大定十六年置,凡十七處,共千人。初以嘗與廷試
 及宗室皇家袒免以上親、并得解舉人為之。後增州學,遂加以五品以上官、曾任隨朝六品官之兄弟子孫,餘官之兄弟子孫經府薦者,同境內舉人試補三之一,闕里廟宅子孫年十三以上不限數,經府薦及終場免試者不得過二十人。凡試補學生,太學則禮部主之,州府則以提舉學校學官主之,曾得府薦及終場舉人,皆免試。



 凡經,《易》則用王弼、韓康伯註,《書》用孔安國註,《詩》用毛萇註、鄭玄箋,《春秋左氏傳》用杜預註,《禮記》用孔穎達疏,《周禮》用鄭玄註、賈公彥疏,《論語》用何晏集註。邢昺疏,《孟子》用趙岐註,孫奭疏,《孝經》用唐玄宗註,《史記》用裴駰註,《
 前漢書》用顏師古註。《後漢書》用李賢註,《三國志》用裴松之註,及唐太宗《晉書》、沈約《宋書》,蕭子顯《齊書》、姚思廉《梁書》《陳書》、魏收《後魏書》、李百藥《北齊書》、令狐德棻《周書》、魏征《隋書》、新舊《唐書》、新舊《五代史》,《老子》用唐玄宗註疏,《荀子》用楊倞註,《揚子》用李軌、宋咸、柳宗元、吳秘注,皆自國子監印之,授諸學校。凡學生會課,三日作策論一道,又三日作賦及詩各一篇,三月一私試,以季月初先試賦,間一日試策論,中選者以上五名申部。遇旬休、節辰皆有假,病則給假,省親遠行則給程。犯學規者罰,不率教者黜。遭喪百日後求入學者,不得與釋奠禮。凡國子學
 生三年不能充貢,欲就諸局承應者,學官試,能粗通大小各一經者聽。



 章宗大定二十九年,上封事者乞興學校,推行三舍法,及鄉以八行貢春官,以設制舉宏詞。事下尚書省集百官議,戶部尚書鄧儼等謂:「三舍之法起於宋熙寧間,王安石罷詩賦,專尚經術。太學生初補外舍,無定員。由外升內舍,限二百人。由內升上舍,限百人。各治一經,每月考試,或特免解,或保舉補官。其法雖行,而多席勢力、尚趨走之弊,故蘇軾有『三舍既興,貨賂公行』之語,是以元祐間罷之,後雖復,而宣和三年竟廢。臣等謂立法貴乎可久,彼三舍之法委之學官選試,啟僥
 倖之門,不可為法。唐文皇養士至八千人,亡宋兩學五千人,今策論、詞賦、經義三科取士,而太學所養止百六十人,外京府或至十人,天下僅及千人。今若每州設學,專除教授,月加考試,每舉所取數多者賞其學官。月試定為三等籍之,一歲中頻在上等者優復之,不率教、行惡者黜之,庶幾得人之道也。又成周鄉舉里選法卒不可復,設科取士各隨其時。八行者乃亡宋取《周禮》之六行孝、友、睦、姻、任、恤,加之中、和為八也。凡人之行莫大於孝廉,今已有舉孝廉之法,及民有才能德行者令縣官薦之。今制,犯十惡姦盜者不得應試,亦六德六行之遺
 意也。夫制舉宏詞,蓋天子待非常之士,若設此科,不限進士,并選人試之,中選擢之臺閣,則人自勉矣。」上從其議。遂計州府戶口。增養士之數,於大定舊制京府十七處千人之外,置節鎮、防禦州學六十處,增養千人。各設教授一員,選五舉終場或進士年五十以上者為之。府學二十有四,學生九百五人。大興、開封、平陽、真定、東平府各六十人,太原、益都府各五十人,大定、河間、濟南、大名、京兆府各四十人,遼陽、彰德府各三十人,河中、慶陽、臨洮、河南府各二十五人,鳳翔、平涼、延安、咸平、廣寧、興中府各二十人。節鎮學三十九,共六百一十五人。絳、定、衛、懷、滄州各三十人,萊、密、潞、汾、冀、邢、兗州各二十五人,代、同、邠州各二十人,奉聖州十五人,餘二十三節鎮皆十人。防禦州學二十一,共二百三十五人。博、德、洺、
 棣、亳各十五人,餘十六州各十人。凡千八百人。



 女直學。自大定四年,以女直大小字譯經書頒行之。後擇猛安謀克內良家子弟為學生,諸路至三千人。九年,取其尤俊秀者百人至京師,以編修官溫迪罕締達教之。十三年,以策、詩取士,始設女直國子學,諸路設女直府學,以新進士為教授。國子學策論生百人,小學生百人。府州學二十二,中都、上京、胡里改、恤頻、合懶、蒲與、婆速、咸平、泰州、臨潢、北京、冀州、開州、豐州、西京、東京、蓋州、隆州、東平、益都、河南、陜西置之。凡取國子學生、府學生之制,皆與詞賦、經義生同。又定制,每謀克取二人,若宗室每二十戶內無願學者,
 則取有物力家子弟年十三以上、二十以下者充。凡會課,三日作策論一道,季月私試如漢生制。大定二十九年,敕凡京府鎮州諸學,各以女直、漢人進士長貳官提控其事,具入官銜。河南、陜西女直學,承安二年罷之,餘如舊。



 凡諸進士舉人,由鄉至府,由府至省,及殿廷,凡四試皆中選,則官之。至廷試五被黜,則賜之第,謂之恩例。又有特命及第者,謂之特恩。恩例者但考文之高下為第,而不復黜落。凡詞賦進士,試賦、詩、策論各一道。經義進士,試所治一經義、策論各一道。其設也,始於太宗天會元年十一月,時以急欲得漢士以撫輯新附,初無定數,亦
 無定期,故二年二月、八月凡再行焉。五年,以河北、河東初降,職員多闕,以遼、宋之制不同,詔南北各因其素所習之業取士,號為南北選。熙宗天眷元年五月,詔南北選各以經義、詞賦兩科取士。海陵庶人天德二年,始增殿試之制,而更定試期。三年,併南北選為一,罷經義、策試兩科,專以詞賦取士。貞元元年,定貢舉程試條理格法。正隆元年,命以《五經》、《三史》正文內出題,始定為三年一闢。



 大定四年,敕宰臣:「進士文優則取,勿限人數。」十八年,謂宰臣:「文士有偶中魁選,不問操履,而輒授翰苑之職。如趙承元,朕聞其無士行,果敗露。自今榜首,先訪察
 其鄉行,可取則授以應奉,否則從常調。」十九年,謂宰臣曰:「自來御試賦題,皆士人嘗擬作者。前朕自選一題,出人所不料,故中選者多名士,而庸才不及焉。是知題難則名儒亦擅場,題易則庸流易僥倖也。」平章政事唐括安禮奏曰:「臣前日言,士人不以策論為意者,正為此爾。宜各場通考,選文理俱優者。」上曰:「并答時務策,觀其議論,材自可見,卿等其議之。」二十年,謂宰臣曰:「朕嘗諭進士不當限數,則對以所取之外無合格文,故中選者少,豈非題難致然耶?若果多合格,而有司妄黜之,甚非理也。」又曰:「古者鄉舉有行者,授以官。今其考滿,察鄉曲實
 行出倫者擢之。」又曰:「舊不選策,今兼選矣。然自今府會兩試不須試策,已中策後,則試以制策,試學士院官。」二十二年,謂宰臣曰:「漢進士魁,例授應奉,若行不副名,不習制誥之文者,即與外除。」二十三年,謂宰臣曰:「漢進士,皇統間人材殆不復見,今應奉以授狀元,蓋循資爾。制誥文字,各以職事鋪敘,皆有定式,故易。至撰赦詔,則鮮有能者。」參知政事粘哥斡特剌對曰:「舊人已登第尚為學不輟,今人一及第輒廢而不學,故爾。」上於聽政之隙,召參知政事張汝霖、翰林直學士李晏讀新進士所對策,至縣令闕員取之何道?上曰:「朕夙夜思此,未知所出。」
 晏對曰:「臣竊念久矣!國朝設科,始分南北兩選,北選詞賦進士擢第一百五十人,經義五十人,南選百五十人,計三百五十人。嗣場,北選詞賦進士七十人,經義三十人,南選百五十人,計二百五十人。以入仕者多,故員不闕。其後南北通選,止設詞賦科,不過取六七十人,以入仕者少,故縣令員闕也。」上曰:「自今文理可採者取之,毋限以數。」二十八年,復經義科。



 章宗明昌元年正月,言事者謂:「舉人四試而鄉試似為虛設,固當罷去。其府會試乞十人取一人,可以群經出題,而註示本傳。」上是其言,詔免鄉試,府試以五人取一人,仍令有司議外路添考
 試院,及群經出題之制。有司言:「會試所取之數,舊止五百人,比以世宗敕中格者取,乞依此制行之。府試舊六處,中有地遠者,命特添三處,上京、咸平府路則試於遼陽,河東南北路則試於平陽,山東東路則試於益都。以《六經》、《十七史》、《孝經》、《論語》、《孟子》、及《荀》、《揚》、《老子》內出題,皆命於題下註其本傳。」又諭有司曰:「舉人程文所用故事,恐考試官或遽不能憶,誤失人材,可自注出處,注字之誤,不在塗注乙之數。」



 明昌二年,敕官或職至五品者,直赴御試。四年,平章政事守貞言:「國家官人之路,惟女直、漢人進士得人居多。諸司局承應,舊無出身,自大定後始敘使,
 至今鮮有可用者。近來放進士第數稍多,此舉更宜增取,若會試止以五百人為限,則廷試雖欲多取,不可得也。」上乃詔有司,會試毋限人數,文合格則取。



 六年,言事者謂:「學者率恃有司全注本傳以示之,故不勉讀書,乞減子史注本傳之制。又經義中選之文多膚淺,乞擇學官,及本科人充試官。」省臣謂:「若不與本傳,恐碩學者有偶忘之失,可令但知題意而已。」遂命擇前經義進士為眾所推者、才識優長者為學官,遇差考試官之際,則驗所治經參用。詞賦進士,題注本傳,不得過五十字。經義進士,御試第二場,試論日添試策一道。



 承安四年,上諭
 宰臣曰:「一場放二狀元,非是。後場廷試,令詞賦、經義通試時務策,止選一狀元,餘雖有明經、法律等科,止同諸科而已。」至宋王安石為相,作新經,始以經義取人。且詞賦、經義、人素所習之本業,策論則兼習者也。今捨本取兼習,恐不副陛下公選之意。」遂定御試同日各試本業,詞賦依舊,分立甲次,第一名為狀元,經義魁次之。恩例與詞賦第二人同,餘分為兩甲中下人,並在詞賦之下。五年,詔考試詞賦官各作程文一道,示為舉人之式,試後赴省藏之。時宰臣奏:「自大定二十五年以前,詞賦進士不過五百人,二十八年以不限人數,取至五百八十
 六人。先承聖訓合格則取,故承安二年取九百二十五人。兼今有四舉終場恩例,若會試取人數過多,則涉泛濫。」遂定策論、詞賦、經義人數,雖多不過六百人,少則聽其闕。時太常丞郭人傑轉對言,詞賦舉人,不得作別名兼試經義,及入學生精加試選,無至濫補。上敕宰臣曰:「近已奏定,後場詞賦經義同日試之。若府會試更不令兼試,恐試經義者少,是虛設此科也。別名之弊,則當禁之。補試入學生員,已有舊條,恐行之滅裂爾,宜嚴防閑。」張行簡轉對言:「擬作程文,本欲為考試之式,今會試考試官、御試讀卷官皆居顯職,擢第後離筆硯久,不復常
 習,今臨試擬作之文,稍有不工,徒起謗議。」詔罷之。



 泰和元年,平章政事徒單鎰病時文之弊,言:「諸生不窮經史,唯事末學,以致志行浮薄。可令進士試策日,自時務策外,更以疑難經旨相參為問,使發聖賢之微旨、古今之事變。」詔為永制。先嘗敕樂人不得舉進士,而奴免不良者則許之。尚書省奏:「舊稱工樂,謂配隸之色及倡優之家。今少府監工匠,太常大樂署樂工,皆民也,而不得與試。前代令諸選人身及祖、父曾經免為良者,雖在官不得居清貫及臨民,今反許試,誠玷清論。」詔遂定制,放良人不得應諸科舉,其子孫則許之。上又謂:「德行才能非
 進士科所能盡,可通行保舉之制。省奏:「在《周禮》,『大司徒以鄉三物教萬民而賓興之。」所謂萬民,農工商賈皆是也。前代立賢無方,如版築之士、鼓刀之叟,垂光簡策者不可勝舉。今草澤隱逸才行兼備者,令謀克及司縣舉,按察司具聞,以旌用之,既有已降令文矣。」上命復宣旨以申之。



 宣宗貞祐二年,御史臺言:「明年省試以中都、遼東、西北京等路道阻,宜於中都、南京兩處試之。」三年,諭宰臣曰:「國初設科,素號嚴密,今聞會試至於雜坐喧嘩,何以防弊?」命治考官及監察罪。興定二年,御史中丞把胡魯言:「國家數路收人,惟進士之選最為崇重,不求
 備數,惟務得賢。今場會試,策論進士不及二人取一人,詞賦、經義二人取一,前雖有聖訓,當依大定之制,中選即收,無問多寡,然大定間赴試者或至三千,取不過五百。泰和中,策論進士三人取一,詞賦、經義四人取一,向者貞祐初,詔免府試,赴會試者幾九千人。而取八百有奇,則是十之一而已。時已有依大定之制,亦何嘗二人取一哉!今考官泛濫如此,非所以為求賢也。宜於會試之前,奏請所取之數,使恩出于上可也。」詔集文資官議,卒從泰和之例。又謂宰臣曰:「從來廷試進士,日晡後即遣出宮,恐文思遲者不得盡其才,令待至暮時。」特賜經
 義進士王彪等十三人及第,上覽其程文,愛其辭藻,咨嘆久之。因怪學者益少,謂監試官左丞高汝礪曰:「養士學糧,歲稍豐熟即以本色給之,不然此科且廢矣!」五年,省試經義進士,考官於常格外多取十餘人,上命以特恩賜第。又命河北舉人今府試中選而為兵所阻者,免後舉府試。



 策論進士,選女直人之科也。始大定四年,世宗命頒行女直大小字所譯經書。每謀克選二人習之。尋欲興女直字學校,猛安謀克內多擇良家子為生,諸路至三千人。九年,選異等者百人,薦於京師,廩給之。命溫迪罕
 締達教以古書,作詩、策,後復試,得徒單鎰以下三十餘人。十一年,始議行策選之制,至十三年始定每場策一道,以五百字以上成,免鄉試府試,止赴會試御試。且詔京師女直國子學,諸路設女直府學,擬以新進士充教授,以教士民子弟之願學者。俟行之久學者眾,則同漢進士三年一試之制。乃就憫忠寺試徒單鎰等,其策曰:「賢生於世,世資於賢,世未嘗不生賢,賢未嘗不輔世。蓋世非無賢,惟用與否。若伊尹之佐成湯,傅說之輔高宗,呂望之遇文王,皆起耕築漁釣之間,而其功業卓然,後世不能企及者,蓋殷、周之君能用其人,盡其才也。本
 朝以神武定天下,聖上以文德綏海內,文武並用,言小善而必從,事小便而不棄,蓋取人之道盡矣!而尚憂賢能遺於草澤者,今欲盡得天下之賢用之,又俾賢者各盡其能,以何道而臻此乎?」憫忠寺舊有雙塔,進士入院之夜半,聞東塔上有聲如音樂,西入宮。考試官侍御史完顏蒲涅等曰:「文路始開而有此,得賢之祥也。」中選者得徒單鎰以下二十七人。十六年,命皇家兩從以上親及宰相子,直赴御試。皇家袒免以上親及執政官之子,直赴會試。至二十年,以徒單鎰等教授中外,其學大振。遂定制,今後以策、詩試三場,策用女直大字,詩用小
 字,程試之期皆依漢進士例。省臣奏:「漢人進士來年三月二十日鄉試,八月二十日府試,次年正月二十日會試,三月十二日御試。」敕以來年八月二十五日於中都、上京、咸平、東平府等路四處府試,餘從前例。上曰:「契丹文字年遠,觀其所撰詩,義理深微,當時何不立契丹進士科舉,今雖立女直字科,慮女直字創製日近,義理未如漢字深奧,恐為後人議論。」丞相守道曰:「漢文字恐初亦未必能如此。由歷代聖賢漸加修舉也。聖主天姿明哲,令譯經教天下,行之久亦可同漢人文章矣!」上曰:「其同漢人進士例。譯作程文,俾漢官覽之。」二十二年三月,
 策試女直進士。至四月癸丑,上謂宰臣曰:「女直進士試已久矣,何尚未考定?」參知政事斡特剌對曰:「以其譯付看故也。」上令速之。二十三年,上曰:「女直進士設科未久,若令積習精通,則能否自見矣。」二十八年,諭宰臣曰:「女直進士惟試以策,行之既久,人能預備,今若試以經義可乎?」宰臣對曰:「《五經》中《書》、《易》、《春秋》已譯之矣,俟譯《詩》、《禮》畢,試之可也。」上曰:「大經義理深奧,不加歲月不能貫通。今宜於經內姑試以論題,後當徐試經義也。」



 章宗大定二十九年,詔許諸人試策論進士舉。七月,省奏:「如詩、策、論俱作一日程試,恐力有不逮。詩、策作一日,論作一日,
 以詩、策合格為中選,而以論定其名次。上曰:「論乃新添,至第三舉時當通定去留。」明昌元年,猛安謀克願試進士者擬依餘人例,不可令直赴御試。」上曰:「是止許女直進士,毋令試漢進士也。」又定制,餘官第五品散階,令直赴會試,官職俱至五品,令直赴御試。承安二年,敕策論進士限丁習學。遂定制,內外官員、諸局分承應人、武衛軍、若猛安謀克女直及諸色人,戶止一丁者不許應試,兩丁者許一人,四丁二人,六丁以上止許三人。三次終場,不在驗丁之限。三年,定制,女直人以年四十五以下,試進士舉,於府試十日前,委佐貳官善射者試射。其制,
 以六十步立垛,去射者十五步對立兩竿,相去二十步,去地二丈,以繩橫約之。弓不限強弱,不計中否,以張弓巧便、發箭迅正者為熟閑。射十箭中兩箭,出繩下至垛者為中選。餘路委提刑司,在都委監察體究。如當赴會試御試者,大興府佐貳官試驗,三舉終場者免之。四年,禮部尚書賈鉉言:「策論進士程試弓箭,其兩舉終場及年十六以下未成丁者,若以弓箭退落,有失賢路。乞於及第後試之,中者別加任使,或升遷,否者降之。」省臣謂:「舊制三舉終場免試,今兩舉亦免之,未可。若以未成丁免試,必有妄匿年者,如果幼,使徐習未晚也。至於及第
 後試驗升降,則已有定格矣。」詔從舊制。在泰和格,復有以時務策參以故事,及疑難經旨為問之制。



 宣宗南遷,興定元年,制中都、西京等路,策論進士及武舉人權於南京、東平、婆速、上京四處府試。五年,上賜進士斡勒業德等二十八人及第。上覽程文,怪其數少,以問宰臣,對曰:「大定制隨處設學,諸謀克貢三人或二人為生員,贍以錢米。至泰和中,人例授地六十畝。所給既優,故學者多。今京師雖存府學,而月給通寶五十貫而已。若於諸路總管府、及有軍戶處置學養之,庶可加益。京師府學已設六十人,乞更增四十人。中京、亳州、京兆府並置學
 官於總府,以謀克內不隸軍籍者為學生,人畀地四十畝。漢學生在京者亦乞同此,餘州府仍舊制。」上從之。



 凡會試之數,大定二十五年,詞賦進士不得過五百人。二十八年,以不限人數,遂至五百八十六人。章宗令合格則取,故承安二年至九百二十五人。時以復加四舉終場者,數太濫,遂命取不得過六百人。泰和二年,上命定會試諸科取人之數,司空襄言:「試詞賦、經義者多,可五取一。策論絕少,可四取一。恩榜本以優老於場屋者。四舉受恩則太優,限以年則礙異材。可五舉則授恩。」平章徒單鎰等言:「大定二十五年至明昌初,率三四人取
 一。」平章張汝霖亦言:「五人取一,府試百人中纔得五耳。」遂定制,策論三人取一,詞賦、經義五人取一,五舉終場年四十五以上、四舉終場年五十以上者受恩。



 凡考試官,大定間,府試六處,各差詞賦試官三員,策論試官二員。明昌初,增為九處,路各差九員,大興府則十一員。承安四年,又增太原為十處。有司請省之,遂定策論進士女直經童千人以上差四員,五百人以上三員,不及五百二員。各以職官高者一人為考試官,餘為同考試官。詞賦進士與律科舉人共及三千以上五員,二千四員,不及二千三員。經義進士及經童舉人千人四
 員,五百以上三員,百人以上二員,不及百人以詞賦考官兼之。後又定制,策論試官,上京、咸平、東平各三員,北京、西京、益都各二員。律科,監試官一員,試律官二員,隸詞賦考試院。經童,試官一員,隸經義考試院,與會試同。其彌封并謄錄官、檢搜懷挾官、自餘修治試院、監押門官,並如會試之制。大定二十年,上以往歲多以遠地官考試不便,遂命差近者。



 凡會試,知貢舉官、同知貢舉官,詞賦則舊十員,承安五年為七員。經義則六員,承安五年省為四員。詮讀官二員。泰和三年,上以彌封官渫語於舉人,敕自今女直司
 則用右選漢人封,漢人司則以女直司封。宣宗貞祐三年,以會試賦題已曾出,而有犯格中選者,復以考官多取所親,不怒其不公,命究治之。



 凡御試,讀卷官,策論、詞賦進士各七員,經義五員,餘職事官各二員。制舉宏詞共三員。泰和七年,禮部尚書張行簡言:「舊例,讀卷官不避親,至有親人,或有不敢定其去留,或力加營護,而為同列所疑。若讀卷官不用與進士有親者,則讀卷之際得平心商確。」上遂命臨期多擬,其有親者汰之。



 凡府試策論進士,大定二十年定以中都、上京、咸平、東
 平四處。至明昌元年,添北京、西京、益都為七處,兼試女直經童。凡上京、合懶、速頻、胡里改、蒲與、東北招討司等路者,則赴會寧府試。咸平、隆州、婆速、東京、蓋州、懿州者,則赴咸平府試。中都、河北東西路者,則赴大興府試。西京並西南、西北二招討司者,則赴大同府試。北京、臨潢、宗州、興州、全州者,則赴大定府試。山東西、大名、南京者,則赴東平府試。山東東路則試於益都。凡詞賦、經義進士及律科、經童府試之處,大定間,大興、大定、大同、開封、東平、京兆凡六處。明昌初,增遼陽,平陽,益都為九處。承安四年復增太原為十。中都、河北則試於大興府,上京、
 東京、咸平府等路則試於遼陽府,餘各試於其境。



 凡鄉試之期,以三月二十日。府試之期,若策論進士則以八月二十日試策,間三日試詩。詞賦進士則以二十五日試賦及詩,又間三日試策論。經義進士又間詞賦後三日試經義,又三日試策。次律科,次經童,每場皆間三日試之。會試,則策論進士以正月二十日試策,皆以次間三日,同前。御試,則以三月二十日策論進士試策,二十三日試詩論,二十五日詞賦進士試賦詩論,而經義進士亦以是日試經義,二十七日乃試策論。若試日遇雨雪,則候晴日。御試唱名後,試策則稟奏,宏詞則作
 二日程試。舊制,試女直進士在再試漢進士後。大定二十九年以復設經義科,更定是制。



 凡監檢之制,大興府則差武衛軍。餘府則於附近猛安內差摘,平陽府則差順德軍。凡府會試,每四舉人則差一人,復以官一人彈壓。御試策進士則差弩手及隨局承應人,漢進士則差親軍,人各一名,皆用不識字者,以護衛十人。親軍百人長、五十人長各一人巡護。泰和元年,省臣奏:「搜檢之際雖當嚴切,然至於解髮袒衣,索及耳鼻,則過甚矣,豈待士之禮哉!故大定二十九年已嘗依前故事,使就沐浴,官置衣為之更之,既可防濫,且不
 虧禮。」上從其說,命行之。



 恩例。明昌元年,定制,省元直就御試,不中者許綴榜未。解元但免府試,四舉終場依五舉恩例,所試文卷惟犯御名廟諱、不成文理者則黜之,餘並以文之優劣為次。仍一日試三題,其五舉者止試賦詩,女直進士亦同此例。承安五年,敕進士四舉該恩,詞賦、經義當以各科為場數,不得通數。又恩榜人應授官者,監試官於試時具數以奏,特恩者授之。泰和三年,以經義會元與策論詞賦進士不同,若御試被黜則附榜末,為太優,若同恩例,又與四舉者不同。遂定制,依曾經府試解元免府試之
 例。會試下第,再舉直赴御試。



 律科進士,又稱為諸科,其法以律令內出題,府試十五題,每五人取一人。大定二十二年定制,會試每場十五題,三場共通三十六條以上,文理優、擬斷當、用字切者,為中選。臨時約取之,初無定數。其制始見於海陵庶人正隆元年,至章宗大定二十九年,有司言:「律科止知讀律,不知教化之源,可使通治《論語》、《孟子》,以涵養其氣度。」遂令自今舉後,復於《論語》、《孟子》內試小義一道,府會試別作一日引試,命經義試官出題,與本科通考定之。



 經童之制,凡士庶子年十三以下,能誦二大經、三小經,
 又誦《論語》諸子及五千字以上,府試十五題通十三以上,會試每場十五題,三場共通四十一以上,為中選。所貴在幼而誦多者,若年同,則以誦大經多者為最。初,天會八年時,太宗以東平童子劉天驥,七歲能誦《詩》、《書》、《易》、《禮》、《春秋左氏傳》及《論語》、《孟子》,上命教養之,然未有選舉之制也。熙宗即位之二年,詔闢貢舉,始備其列,取至百二十二人。天德間,廢之。章宗大定二十九年,上謂宰臣曰:「經童豈遽無人,其議復置。」明昌元年,益都府申:「童子劉住兒年十一歲,能詩賦,誦大小六經,所書行草頗有法,孝行夙成,乞依宋童子李淑賜出身,且加以恩詔。」
 召至內殿,試《鳳凰來儀》賦、《魚在藻》詩,又令賦《旱》詩,上嘉之,賜本科出身,給錢粟官舍,令肄業太學。明昌三年,平章政事完顏守貞言:「經童之科非古也,自唐諸道表薦,或取五人至十人。近代宋仁宗以為無補,罷之。本朝皇統間取五十人,因以為常,天德時復廢。聖主復置,取以百數,恐久積多,不勝銓擬,乞諭旨約省取之。」上曰:「若所誦皆及格,何如?」守貞曰:「視最幼而誦不訛者精選之,則人數亦不至多也。」復問參知政事胥持國,對曰:「所誦通否易見,豈容有濫。」上曰:「限以三十或四十人,若百人皆通,亦可復取其精者。」持國曰:「是科蓋資教之術耳。夫
 幼習其文,長玩其義,使之蒞政,人格出焉。如中選者,加之修習進士舉業,則所記皆得為用。臣謂可勿令遽登仕途,必習舉業,而後官使之可也。若能擢進士第,自同進士任用。如中府薦或會試,視其次數,優其等級。幾舉不得薦者,從本出身,似可以激勸而得人矣!」詔議行之。



 制舉有賢良方正、能直言極諫、博學宏材、達於從政等科,試無常期。上意欲行,即告天下。聽內外文武六品以下職官無公私過者,從內外五品以上官薦於所屬,詔試之。若草澤士,德行為鄉里所服者,則從府州薦之。凡試,則先投所業策論三十道於學士院,視其詞理優者,
 委官以群經子史內出題,一日試論三道,如可,則庭試策一道,不拘常務,取其無不通貫者,優等遷擢之。宏詞科試詔、誥、章、表、露布、檄書,則皆用四六;誡、諭、頌、箴、銘、序、記,則或依古今體,或參用四六。於每舉賜第後進士及在官六品以下無公私罪者,在外官薦之,令試策官出題就考,通試四題,分二等遷擢之。二科皆章宗明昌元年所創者也。



 武舉,嘗設於皇統時,其制則見於《泰和式》,有上中下三等。能挽一石力弓,以重七錢竹箭,百五十步立貼,十箭內,府試欲中一箭,省試中二箭,程試中三箭。又遠射二
 百二十步垛,三箭內一箭至者。又百五十步內,每五十步設高五寸、長八寸臥鹿二,能以七斗弓、二大鑿頭鐵箭馳射,府試則許射四反,省試三反,程試二反,皆能中二箭者。又百五十步內,每三十步,左右錯置高三尺木偶人戴五寸方板者四,以槍馳刺,府試則許馳三反,省試二反,程試三反,左右各刺落一板者。又依廕例問律一條,又問《孫》、《吳》書十條,能說五者,為上等。凡程試,若一有不中者,皆黜之。若射貼弓八斗,遠射二百一十步,射鹿弓六斗,《孫》、《吳》書十條通四,為中等。射貼弓七斗,遠射二百五步,射鹿弓五斗,《孫》、《吳》書十條通三,為下等。解律、
 刺板,皆欲同前。凡不知書者,雖上等為中,中則為下。凡試中中下,願再試者聽。舊制,就試上等不中,不許再試中下等。泰和元年,定制,不分舊等,但從所願,試中則以三等為次。二年,省奏:「武舉程式當與進士同時,今年八月府試,欲隨路設考試所,臨期差官,恐以創立未見應試人數,遂權令各處就考之。」宣宗貞祐三年,同進士例,賜敕命章服。時以隨處武舉入試者,自非見居職任及已用於軍前者,令郡縣盡遣詣京師,別為一軍,以備緩急。其被薦而未授官者,亦量材任之。元光二年,東京總帥紇石烈牙吾塔言:「武舉入仕,皆授巡尉軍轄,此曹雖
 善騎射,不歷行陣,不知軍旅,一旦臨敵,恐致敗事。乞盡括付軍前為長校,俟有功則升之。」宰臣奏:「國家設此科與進士等,而欲盡置軍中,非獎進人材之道。」遂籍丁憂、待闕、去職者付之。



 試學士院官。大定二十八年,敕設科取士為學士院官。禮部下太常,按唐典,初入學士院例先試,今若於進士已仕者,以隨朝六品,外路五品職事官薦,試制詔誥等文字三道,取文理優者充應奉。由是翰苑之選為精。明昌五年,以學士院撰文字人少,命尚書省訪有文采者勾取權試之。



 凡司天臺學生,女直二十六人,漢人五十人,聽官民家年十五以上,三十以下試補。又三年一次,選草澤人試補。其試之制,以《宣明歷》試推步,及《婚書》、《地理新書》試合婚、安葬,並《易》筮法,六壬課、三命五星之術。凡醫學十科,大興府學生三十人,餘京府二十人,散府節鎮十六人,防禦州十人,每月試疑難,以所對優劣加懲勸,三年一次試諸太醫,雖不系學生,亦聽試補。



\end{pinyinscope}