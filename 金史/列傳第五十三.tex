\article{列傳第五十三}

\begin{pinyinscope}

 ○完顏奴申崔立聶天驥赤盞尉忻



 完顏奴申,字正甫,素蘭之弟也。登策論進士第,仕歷清要。正大三年八月,由翰林直學士充益政院說書官。五年,轉吏部侍郎。監察御史烏古論石魯剌劾近侍張文壽、仁壽、李麟之受敵帥饋遺,詔奴申鞫問,得其姦狀,上曲赦其罪,皆斥去,朝論快之。九月,改侍講學士,以御史大夫奉使大元。至龍駒河,朝見太宗皇帝。十二月,還。明
 年六月,遷吏部尚書,復往。八年春,還。朝廷以勞拜參知政事。



 天興元年春,大兵駐鄭州海灘寺,遣使招哀宗降。復以奴申往乞和。不許,攻汴益急。汴受圍數月,倉庫匱乏,召武仙等入援不至,哀宗懼,以曹王訛可出質,請罷攻。冬十月,哀宗議親出捍禦,以奴申參知政事、兼樞密副使,完顏習捏阿不樞密副使、兼知開封府、權參知政事,總諸軍留守京師。又以翰林學士承旨烏古孫卜吉提控諸王府,同判大睦親府事兼都點檢內族合周管宮掖事,左副點檢完顏阿撒、右副點檢溫敦阿里副之,戶部尚書完顏珠顆兼裏城四面都總領,御史大夫裴
 滿阿虎帶兼鎮撫軍民都彈壓,諫議大夫近侍局使行省左右司郎中烏古孫奴申兼知宮省事。又以把撒合為外城東面元帥,術甲咬住南面元帥,崔立西面元帥,孛術魯買奴北面元帥。乙酉,除拜定,以京城付之。又以戶部侍郎刁璧為安撫副使,總招撫司,規運京外糧斛。設講議所,受陳言文字,以大理卿納合德輝、戶部尚書仲平、中京副留守愛失等總其事。



 十二月辛丑,上出京,服絳紗袍,乘馬導從如常儀。留守官及京城父老從至城外奉辭,有詔撫諭,仍以鞭揖之。速不泬聞上已出,復會兵圍汴。初,上以東面元帥李辛跋扈出怨言,罷為兵部侍
 郎,將出,密喻奴申等羈縶之。上既行,奴申等召辛,辛懼,謀欲出降,棄馬踰城而走。奴申等遣人追及之,斬於省門。汴民以上親出師,日聽捷報,且以二相持重,幸以無事。俄聞軍敗衛州,蒼黃走歸德,民大恐,以為不救。時汴京內外不通,米升銀二兩。百姓糧盡,殍者相望,縉紳士女多行乞於市,至有自食其妻子者,至於諸皮器物皆煮食之,貴家第宅、市樓肆館皆撤以爨。及歸德遣使迎兩宮,人情益不安,於是民間有立荊王監國以城歸順之議,而二相皆不知也。



 天興二年正月丙寅,省令史許安國詣講議所言:「古者有大疑,謀及卿士,謀及庶人。今
 事勢如此,可集百官及僧道士庶,問保社稷、活生靈之計。」左司都事元好問以安國之言白奴申,奴申曰:「此論甚佳,可與副樞議之。」副樞亦以安國之言為然。好問曰:「自車駕出京,今二十日許,又遣使迎兩宮。民間洶洶,皆謂國家欲棄京城,相公何以處之?」阿不曰:「吾二人惟有一死耳。」好問曰:「死不難,誠能安社稷、救生靈,死而可也。如其不然,徒欲一身飽五十紅衲軍,亦謂之死耶?」阿不款語曰:「今日惟吾二人,何言不可。」好問乃曰:「聞中外人言,欲立二王監國,以全兩宮與皇族耳。」阿不曰:「我知之矣,我知之矣。」即命召京城官民。明日皆聚省中,諭以事
 勢危急當如之何。有父老七人陳詞云云,二相命好問受其詞。白之奴申,顧曰:「亦為此事也。」且問副樞「此事謀議今幾日矣」?阿不屈指曰:「七日矣。」奴申曰:」歸德使未去,慎勿泄。」或曰是時外圍不解,如在陷阱,議者欲推立荊王以城出降,是亦《春秋》紀季入齊之義,況北兵中已有曹王也。眾憤二人無策,但曰死守而已。忽聞召京城士庶計事,奴申拱立無語,獨阿不反復申諭:「國家至此無可奈何,凡有可行當共議之」,且繼以涕泣。



 明日戊辰,西面元帥崔立與其黨孛術魯長哥、韓鐸、藥安國等為變,率甲卒二百橫刀入省中,拔劍指二相曰:「京城危困已
 極,二公坐視百姓餓死,恬不為慮,何也?」二相大駭,曰:「汝輩有事,當好議之,何遽如是。」立麾其黨先殺阿不,次殺奴申及左司郎中納合德輝等,餘見《崔立傳》。



 劉祁曰:「金自南渡之後,為宰執者往往無恢復之謀,臨事相習低言緩語,互相推讓,以為養相體。每有四方災異、民間疾苦,將奏必相謂曰:『恐聖主心困。』事至危處輒罷散,曰『俟再議』,已而復然。或有言當改革者,輒以生事抑之,故所用必擇軟熟無鋒芒易制者用之。每北兵壓境,則君臣相對泣下,或殿上發長吁而已。兵退,則大張具,會飲黃閣中矣。因循茍且,竟至亡國。又多取渾厚少文者置之
 台鼎,宣宗嘗責丞相僕散七斤『近來朝廷紀綱安在』?七斤不能對,退謂郎官曰:『上問紀綱安在,汝等自來何嘗使紀綱見我。』故正人君子多不見用,雖用亦未久而遽退也。」祁字京叔,渾源人。



 贊曰:劉京叔《歸潛志》與元裕之《壬辰雜編》二書雖微有異同,而金末喪亂之事猶有足徵者焉。哀宗北禦,以孤城弱卒託之奴申、阿不二人,可謂難矣。雖然,即墨有安平君,玉壁有韋孝寬,必有以處此。



 崔立,將陵人,少貧無行,嘗為寺僧負鈸鼓,乘兵亂從上黨公開為都統、提控,積階遙領太原知府。正大初,求入
 仕。為選曹所駮,每以不至三品為恨。圍城中授安平都尉。天興元年冬十二月,上親出師,授西面元帥。性淫姣,常思亂以快其欲。



 藥安國者,管州人,年二十餘,有勇力。嘗為嵐州招撫使,以罪繫開封獄,既出,貧無以為食。立將為變,潛結納之,安國健啖,日飽之以魚,遂與之謀。先以家置西城上,事不勝則挈以逃。日與都尉楊善入省中候動靜,布置已定,召善以早食,殺之。二年正月,遂帥甲卒二百,撞省門而入。二相聞變趨出,立拔劍曰:「京城危困,二公欲如何處之?」二相曰:「事當好議之。」立不顧,麾其黨張信之、孛術魯長哥出省,二相遂遇害。馳往東華
 門,道遇點檢溫屯阿里,見其衷甲,殺之。即諭百姓曰:「吾為二相閉門無謀,今殺之,為汝一城生靈請命。」眾皆稱快。是日,御史大夫裴滿阿忽帶、諫議大夫左右司郎中烏古孫奴申、左副點檢完顏阿散、奉御忙哥、講議蒲察琦、戶部尚書完顏珠顆皆死。



 立還省中,集百官議所立。立曰:「衛紹王太子從恪,其妹公主在北兵中,可立之。」乃遣其黨韓鐸以太后命往召從恪。須臾入,以太后誥命梁王監國。百官拜舞山呼,從恪受之,遂遣送二相所佩虎符詣速不泬納款。凡除拜皆以監國為辭。立自稱太師、軍馬都元帥、尚書令、鄭王,出入御乘輿,稱其妻為王
 妃,弟倚為平章政事,侃為殿前都點檢。其黨孛術魯長哥御史中丞,韓鐸都元帥兼知開封府事,折希顏、藥安國、張軍奴並元帥,師肅左右司郎中,賈良兵部郎中兼右司都事,內府之事皆主之。初,立假安國之勇以濟事,至是復忌之,聞安國納一都尉夫人,數其違約斬之。



 壬申,速不泬至青城,立服御衣,儀衛往見之。大帥喜,飲之酒,立以父事之。既還,悉燒京城樓櫓,火起,大帥大喜,始信其實降也。立託以軍前索隨駕官吏家屬,聚之省中,人自閱之,日亂數人猶若不足。又禁城中嫁娶,有以一女之故殺數人者。未幾,遷梁王及宗室近族皆置宮中,
 以腹心守之,限其出入。以荊王府為私第,取內府珍玩實之。二月乙酉,以天子袞冕后服上進。又括在城金銀,搜索薰灌,訊掠慘酷,百苦備至。郕國夫人及內侍高祐、京民李民望之屬,皆死杖下。溫屯衛尉親屬八人,不任楚毒,皆自盡。白撒夫人、右丞李蹊妻子皆被掠死。同惡相濟,視人如仇,期於必報而後已。人人竊相謂曰:「攻城之後七八日之中,諸門出葬者開封府計之凡百餘萬人,恨不早預其數而值此不幸也。」立時與其妻入宮,兩宮賜之不可勝計。立因諷太后作書陳天時人事,遣皇乳母招歸德。當時冒進之徒爭援劉齊故事以冀非分
 者,比肩接武。



 四月壬辰,立以兩宮、梁王、荊王及諸宗室皆赴青城,甲午北行,立妻王氏備仗衛送兩宮至開陽門。是日,宮車三十七兩,太后先,中宮次之,妃嬪又次之,宗族男女凡五百餘口,次取三教、醫流、工匠、繡女皆赴北。四月,北兵入城。立時在城外,兵先入其家,取其妻妾寶玉以出,立歸大慟,無如之何。



 李琦者,山西人,為都尉,在陳州與粘哥奴申同行省事,陳州變,入京,附崔立妹婿折希顏,娶夾谷元之妻,妻年二十餘,有姿色,立初拘隨駕官之家屬,妻輿病而往,得免。琦娶之後,有言其美者,立欲強之。琦每見立欲奪人妻,必差其夫遠出,一日
 差琦出京,琦以妻自隨,如是者再三,立遂欲殺琦。琦又數為折希顏所折辱,乃首建殺立之謀。李伯淵者,寶坻人,本安平都尉司千戶,美姿容,深沉有謀,每憤立不道,欲仗義殺之。李賤奴者,燕人,嘗以軍功遙領京兆府判,壬辰冬,車駕東狩,以都尉權東面元帥。立初反,以賤奴舊與敵體,頗貌敬之。數月之後,勢已固,遂視賤奴如部曲然。賤奴積不能平,數出怨言,至是與琦等合。三年六月甲午,傳近境有宋軍,伯淵等陽與立謀備禦之策。翌日晚,伯淵等燒外封丘門以警動立。是夜,立殊不安,一夕百臥起。比明,伯淵等身來約立視火,立從苑秀、折希
 顏數騎往,諭京城民十五以上、七十以下男子皆詣太廟街點集。既還,行及梳行街,伯淵欲送立還二王府,立辭數四,伯淵必欲親送,立不疑,倉卒中就馬上抱立。立顧曰:「汝欲殺我耶?」伯淵曰:「殺汝何傷。」即出匕首橫刺之,洞而中其手之抱立處,再刺之,立墜馬死。伏兵起,元帥黃摑三合殺苑秀。折希顏後至不知,見立墜馬,謂與人鬥,欲前解之,隨為軍士所斫,被創走梁門外,追斬之。伯淵繫立屍馬尾,至內前號于眾曰:「立殺害劫奪,烝淫暴虐,大逆不道,古今無有,當殺之不?」萬口齊應曰:「寸斬之未稱也。」乃梟立首,望承天門祭哀宗。伯淵以下軍民皆
 慟,或剖其心生啖之。以三尸挂闕前槐樹上,樹忽拔,人謂樹有靈,亦厭其為所汙。已而有告立匿宮中珍玩,遂籍其家,以其妻王花兒賜丞相鎮海帳下士。



 初,立之變也,前護衛蒲鮮石魯負祖宗御容五,走蔡。前御史中丞蒲察世達、西面元帥把撒合挈其家亦自拔歸蔡。七月己巳,以世達為尚書吏部侍郎,權行六部尚書。世達嘗為左司郎中,同簽樞密院事,充益政院官,皆稱上意。及上幸歸德,遣世達督陳糧運。陳變,世達亦與脅從,尋間道之汴,至是徒往行在,上念其舊,錄用之。左右司官因奏把撒合、石魯亦宜任用,上曰:「世達曲從,非出得已,然
 朕猶少降資級,以示薄罰。彼撒合掌軍一面,石魯宿衛九重,崔立之變,曾不聞發一矢,束手於人。今雖來歸,待以不死,足以示恩,又安得與世達等?撒合老矣,量用其子可也。石魯但當酬其負御容之勞。」未幾,以撒合為北門都尉,其子為本軍都統。石魯復充護衛。世達字正夫,泰和三年進士。



 論曰:崔立納款,使其封府庫、籍人民以俟大朝之命可也。乘時僭竊,大肆淫虐,徵索暴橫,輒以供備大軍為辭,逞欲由己,斂怨歸國,其為罪不容誅矣。而其志方且要求劉豫之事,我大朝豈肯效尤金人者乎!金俘人之主,
 帝人之臣,百年之後適啟崔立之狂謀,以成青城之烈禍。曾子曰:「戒之戒之,出乎爾者,反乎爾者也。」豈不信哉!



 聶天驥,字元吉,五臺人。至寧元年進士,調汝陰簿,歷睢州司候、封丘令。興定初,辟為尚書省令史。時胥吏擅威,士人往往附之,獨天驥不少假借,彼亦不能害也。尋授吏部主事,權監察御史。夏使賀正旦,互市於會同館,外戚有身貿易於其間者,天驥上章曰:「大官近利,失朝廷體,且取輕外方。」遂忤太后旨。出為同知汝州防禦使事,未赴,陜西行尚書省驛召,特旨遙領金安軍節度副使,兼行尚書省都事。未幾,人為右司員外郎,轉京兆治中,
 尋為衛州行尚書六部事。慶陽圍急,朝廷遣宿州總帥牙古塔救之,以天驥充經歷官。圍解,從別帥守邠,帥欲棄州而東,天驥力勸止之,不從,帥坐是被繫逮,天驥降京兆治中。尋有訟其冤者,即召為開封簽事,旬月復右司員外郎。丁母憂,未卒哭,奪哀復職。哀宗遷歸德,天驥留汴中。崔立變,天驥被創甚,臥一十餘日。其女舜英謁醫救療,天驥歎曰:「吾幸得死,兒女曹乃為謁醫,尚欲我活耶?」竟鬱鬱以死。舜英葬其父,明日亦自縊,有傳。



 天驥沉靜寡言,不妄交。起於田畝,能以雅道自將,踐歷臺省若素宦然,諸人多自以為不及也。



 赤盞尉忻,字大用,上京人。當襲其父謀克,不願就,中明昌五年策論進士第。後選為尚書省令史、吏部主事、監察御史,言「諸王駙馬至京師和買諸物,失朝廷體。」有詔禁止。遷鎮南軍節度副使、息州刺史。耕鞠場種禾,兩禾合穗,進於朝,特詔褒諭。改丹州,遷鄭州防禦使,權許州統軍使。丞相高汝礪嘗薦其才可任宰相。元光二年正月,召為戶部侍郎。未幾,權參知政事。二月,為戶部尚書,權職如故。三月,拜參知政事,兼修國史。詔諭近臣曰:「尉忻資稟純質,事可倚任,且其性孝,朕今相之,國家必有望,汝輩當效之也。」正大元年五月,拜尚書右丞。哀宗欲
 修宮室,尉忻極諫,至是臥薪嘗膽為言,上悚然從之。同判睦親府內族撒合輦交結中外,久在禁近。哀宗為太子,有定策功,由是頗惑其言,復倚信日深,臺諫每以為言。太后嘗戒敕曰:「上之騎鞠舉樂,皆汝教之,再犯必杖汝。」哀宗終不能去。尉忻諫曰:「撒合輦姦諛之最,日在天子左右,非社稷福。」上悔悟,出為中京留守,朝論快之。五年,致仕,居汴中,崔立之變明日,召家人付以後事,望睢陽慟哭,以弓弦自縊而死,時年六十三。一子名董七,沒於兵間。弟秉甫,字正之。



 贊曰:聶天驥素履清慎,赤盞尉忻天資忠諒,在治世皆
 足為良臣,不幸仕亂離之朝,以得死為願欲,哀哉!



\end{pinyinscope}