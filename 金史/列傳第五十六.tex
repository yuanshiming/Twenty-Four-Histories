\article{列傳第五十六}

\begin{pinyinscope}

 ○苗道潤王福移剌眾家奴武仙張甫靖安民郭文振胡天作張開燕寧



 苗道潤,貞祐初,為河北義軍隊長。宣宗遷汴,河北土人往往團結為兵,或為群盜。道潤有勇略,敢戰鬥,能得眾心。比戰有功,略定城邑,遣人詣南京求官封。宰相難其
 事,宣宗召河南轉運使王擴問曰:「卿有智慮,為朕決道潤事。今即以其眾使為將。肯終為我盡力乎?」擴對曰:「兼制天下者,以天下為度。道潤得眾,有功因而封之,使自為守,羈縻使之,策之上也。今不許,彼負其眾,何所不可為。」宣宗顧謂宰執曰:「王擴之言,實契朕心。」於是除道潤宣武將軍、同知順天軍節度使事。貞祐四年,復以功遷懷遠大將軍、同知中山府事。再閱月,復戰有功,遷驃騎上將軍、中都路經略使、兼知中山府事。頃之,加中都留守、兼經略使。道潤前後撫定五十餘城。



 興定元年,詔道潤恢復中都,以山東兵益之。道潤奏:「去年十一月,臣遣
 總領張子明招降蠡州獨吉七斤。近日,河北東路兵馬都總管移剌鐵哥移軍蠡州,襲破子明軍,殺數百人,子明亦被創。臣將提兵問罪,重以鐵哥自拔來歸,但備之而已。今欲復取都城,乞無罪鐵哥,直令受臣節制,庶可集事。」宣宗以問宰相,奏曰:「道潤、鐵哥不協,不可相統屬。」詔以完顏宇行元帥府事,督道潤復中都,和輯鐵哥軍。



 初,道潤與順天軍節度使李琛不相能,兩軍士兵因之相攻,琛遣兵攻滿城、完州,道潤軍拒戰,殺琛兄榮及弟明等。琛奏:「潞州提控烏林答吾典承道潤風指,日謀侵害。山東行省數諭道潤與臣通和,竟不見從,且殺臣兄
 榮、弟明等,恣橫如此,將為後患。」又奏:「乞令河北州府官不相統攝,並聽帥府節制。仍遣官增減諸路兵力,使權均勢敵無相吞併,則百姓安農畝矣。」道潤奏李琛以眾叛,陷滿城,攻完州。琛亦奏道潤叛。廷議以為兩人失和,故至于此,令山東行省樞密院諭琛:「行省在彼,自當俱聽節制,何待帥府。士兵本以義團結,且耕且戰。今乃聚之城寨,遂相併吞。百姓不安,皆由官長無所忌憚使之然也。嚴為約束,依時樹藝,無致生事。」有詔道潤與移剌鐵哥合兵撫定河北,令諸道兵互相應援。既而道潤與賈仝、賈瑀互相攻擊,詔道潤、賈仝、王福、武仙、賈瑀分畫
 各路元帥府控制之,彰德衛輝招撫司隸樞密院。賈瑀既與道潤相攻,已而詐為約和,道潤信之,遂伏兵刺殺道潤。朝廷不能問,一軍徬徨所依,提控靖安民乞權隸潞州行元帥府,聽其節制。時興定二年也。



 右丞侯摯乞以保、蠡、完三州隸真定,而蠡州舊受移剌眾家奴節制,一旦改隸真定,恐因而交爭。靖安民等願隸潞州,乃令河北行省審處之。經略副使張柔奏:「賈瑀攻易州寨,殺刺史馬信及其裨校,奪所佩金符而去。」頃之,張柔攻賈瑀殺之。道潤既死,靖安民代領其眾,是後乃封建矣。



 初,貞祐四年,右司諫術甲直敦乞封建河朔,詔尚書省
 議,事寢不行。興定三年,以太原不守,河北州縣不能自立,詔百官議所以為長久之利者。翰林學士承旨徙單鎬等十有六人以謂「制兵有三,一曰戰,二曰和,三曰守。今欲戰則兵力不足,欲和則彼不肯從,唯有守耳。河朔州郡既殘毀,不可一概守之,宜取願就遷徙者屯于河南、陜西,其不願者許自推其長,保聚險阻。」刑部侍郎奧屯胡撒合三人曰:「河北於河南有輔車之勢,蒲、解於陜西有襟喉之要,盡徙其民,是撤其籓籬也。宜令諸郡,選才幹眾所推服、能糾眾遷徙者,願之河南或晉安、河中及諸險隘,量給之食,授以曠土,盡力耕稼。置僑治之官,以
 撫循之。擇其壯者,教之戰陣。敕晉安、河中守臣檄石、嵐、汾、霍之兵,以謀恢復,莫大之便。」兵部尚書烏林答與等二十一人曰:「河朔諸州,親民掌兵之職,擇土人嘗居官、有材略者授之,急則走險,無事則耕種。」宣徽使移剌光祖等三人曰:「度太原之勢,雖暫失之,頃亦可復。當募土人威望服眾者,假以方面重權。能克復一道,即以本道總管授之。能捍州郡,即以長佐授之。必能各保一方,使百姓復業。」提點尚食局石抹穆請以高爵募民,大概同光祖議。宰臣欲置公府,宣宗意未決,御史中丞完顏伯嘉曰:「宋人以虛名致李全,遂有山東實地。茍能統眾守
 土,雖三公亦何惜焉。」宣宗曰:「他日事定,公府無乃多乎。」伯嘉曰:「若事定,以三公就節鎮何不可者。」宣宗意乃決。



 四年二月,封滄州經略使王福為滄海公,河間路招撫使移剌眾家奴為河間公,真定經略使武仙為恒山公,中都東路經略使張甫為高陽公,中都西路經略使靖安民為易水公,遼州從宜郭文振為晉陽公,平陽招撫使胡天作為平陽公,昭義軍節度使完顏開為上黨公,山東安撫副使燕寧為東莒公。九公皆兼宣撫使,階銀青榮祿大夫,賜號「宣力忠臣」,總帥本路兵馬,署置官吏,徵斂賦稅,賞罰號令得以便宜行之。仍賜詔曰:「乃者邊
 防不守,河朔失寧,卿等自總戎昭,備殫忠力,若能自效,朕復何憂。宜膺茅土之封,復賜忠臣之號。除已畫定所管州縣外,如能收得鄰近州縣者,亦聽管屬。」



 王福,本河北義軍,積戰功累遷同知橫海軍節度使事、滄州經略副使。興定元年,福遣提控張聚、王進復濱、棣二州,以聚攝棣州防禦使,進攝濱州刺史。久之,福與聚有隙,聚以棣州附於益都張林。



 興定三年九月,福上言:「滄州東濱滄海,西連真定,北備大兵,可謂要地。乞選重臣為經略使,得便宜從事,以鎮撫軍民。」朝廷以福初率義兵復滄州,招集殘兵,今有眾萬餘,器甲完具,自雄一
 方。與益都張林、棣州張聚皆為鄰境。今利津已不守,遼東道路艱阻,且其意本欲自為使,但託詞耳。因而授之,使招集濱、棣之人,通遼東音問,今若不許,宋人或以大軍迫脅,或以官爵招之,將貽後悔。」宣宗以為然,乃以福為本州經略使,仍令自擇副使。會福有戰功,遷遙授同知東平府事、權元帥右都監,經略節度如故。興定四年,封為滄海公,以清、滄、觀州,鹽山、無棣、樂陵、東光、寧津、吳橋、將陵、阜城、蓚縣隸焉。



 四月,紅襖賊李二太尉寇樂陵,棣州張聚來攻,福皆擊卻之。李二復寇鹽山,經略副使張文與戰,李二大敗,擒其統制二人,斬首二千級,獲馬
 三十匹。七月,宋人與紅襖賊入河北,福嬰城固守。益都張林、棣州張聚日來攻掠,滄州危蹙,福將南奔,為眾所止,遂納款於張林。東平元帥府請討福,乞益河南步卒七千、騎兵五百,滑、濬、衛州資助芻糧,先定賞格,以待有功。朝廷以防秋在近,河南兵不可往,東平兵少,不能獨成功,待至來年春,使東平帥府與高陽公併力討之,乃止。



 移剌眾家奴,積戰功,累官河間路招撫使,遙授開州刺史,權元帥右都監,賜姓完顏氏。興定四年,與張甫俱封。眾家奴封河間公,以獻、蠡、安、深州、河間、肅寧、安平、武強、
 饒陽、六家莊、郎山寨隸焉。興定末,所部州縣皆不可守。元光元年,移屯信安,本張甫境內。張甫因奏:「信安本臣北境,地當衝要,乞權改為府以重之。」詔改信安為鎮安府。是歲,與甫合兵,復取河間府及安、蠡、獻三州,與張甫皆遷金紫光祿大夫。二年,眾家奴及張甫同保鎮安,各當一面,別遣總領提控孫汝楫、楊壽、提控袁德、李成分保外垣,遂全鎮安。未幾,眾家奴奏:「鎮安距迎樂堌海口二百餘里,實遼東往來之衝。高陽公甫有海船在鎮安西北,可募人直抵遼東,以通中外之意。若賞不重不足以使人,今擬應募者特遷忠顯校尉,授八品職,仍賞寶
 泉五千貫。如官職已至忠顯八品以上者,遷兩官、升職一等,回日再遷兩官、升職二等。」詔從之。



 武仙,威州人。或曰嘗為道士,時人以此呼之。貞祐二年,仙率鄉兵保威州西山,附者日眾,詔仙權威州刺史。興定元年,破石海於真定,宣差招撫使惟宏請加官賞,真授威州刺史,兼真定府治中,權知真定府事。遷洺州防禦使、兼同知真定府事,遙授河平軍節度使。興定四年,遷知真定府事,兼經略使,遙領中京留守,權元帥右都監。無何,封恆山公,以中山、真定府,沃、冀、威、鎮寧、平定州,抱犢寨,欒城、南宮縣隸焉。同時九府,財富兵強恒山最
 盛。



 是歲,歸順于大元,副史天倪治真定。仙兄貴為安國軍節度使,史天祥擊之,貴亦歸順于大元。仙與史天倪俱治真定且六年,積不相能,懼天倪圖己,嘗欲南走。宣宗聞之,詔樞密院牒招之,仙得牒大喜。正大二年,仙賊殺史天倪,復以真定來降。天元大將笑乃泬討仙,仙走。閱月,乘夜復入真定,笑乃泬復擊之,仙乃奔汴京。五年,召見,哀宗使樞密判官白華導其禮儀,復封為恒山公,置府衛州。七年,仙圍上黨,已而大兵至,仙遁歸。未幾,衛州被圍,內外不通。詔平章政事合達、樞密副使蒲阿救之,徙仙兵屯胡嶺關,扼金州路。



 八年十一月,大元兵涉
 襄漢,合達、蒲阿駐鄧州,仙由荊子口會鄧州軍。天興元年正月丁酉,合達、蒲阿敗績於三峰山,仙從四十餘騎走密縣,趨御寨,都尉烏林答胡土不納,幾為追騎所得。乃舍騎,步登嵩山絕頂清涼寺,謂登封蘭若寨招撫使霍琢僧秀曰:「我豈敢入汴京。一旦有急,縛我獻大國矣。」遂走南陽留山,收潰軍得十萬人,屯留山及威遠寨。立官府,聚糧食,修器仗,兵勢稍振。



 三月,汴京被圍,哀宗以仙為參知政事、樞密副使,河南行省,詔與鄧州行省思烈合兵入救。八月,至密縣東,遇大元大將速不泬兵過之,仙即按軍眉山店,報思烈曰:「阻澗結營待仙至俱進,
 不然敗矣。」思烈急欲至汴,不聽,行至京水,大兵乘之,不戰而潰。仙亦令其軍散走,期會留山,仙至留山,潰軍至者益眾。哀宗罷思烈為中京留守,詔仙曰:「思烈不知兵,向使從卿阻澗之策,豈有敗哉。軍務一以付卿,日夕以待,戮力一心,以圖後舉。」十一月,遣刑部主事烏古論忽魯召仙,仙不欲行,乃上疏陳利害,請緩三月,生死入援。



 初,思烈至鄭州,承制授宣差總領黃摑三合五朵山一帶行元帥府事、兼行六部尚書。及仙還留山,惡三合權盛,改為征行元帥,屯比陽。三合怨仙奪其權,乃歸順于大元,大將速不泬署三合守裕州。三合乃詐以書約仙
 取裕州,可以得志。仙信之。三合乃報大元大將,遣兵夾擊,敗仙于柳河,仙跳走聖朵寨。



 初,沈丘尉曹政承制召兵西山,裕州防禦使李天祥不用命,政斬之以徇。仙至聖朵,謂政曰:「何故擅誅吾將?」政曰:「天祥違詔,逗遛不行,政用便宜斬之。」仙怒曰:「今日宣差來起軍,明日宣差來起軍,因此軍卒戰亡殆盡矣。自今選甚人來亦不聽,且教兒郎輩山中休息。」又曰:「天祥果有罪,待我來處置,汝何人,輒敢殺之!」政曰:「參政柳河失利,不知存亡,天祥違詔,何為不殺?」仙大怒,叱左右奪政所佩銀牌,令總領楊全械系之。會赦,猶囚之,及仙敗,始得釋,與楊全俱降宋。



 是時,哀宗走歸德,遣翰林修撰魏璠間道召仙。行至裕州,會仙敗于柳河,璠矯詔招集潰軍以待仙,仙疑璠圖已。二年正月,仙閱兵,選鋒尚十萬,璠曰:「主上旦夕西首望公,公不宜久留於此。」仙怒,幾殺璠。璠及忽魯剌還歸德,仙乃奏請誅璠,哀宗不聽,以璠為歸德元帥府經歷官。璠字邦彥,渾源人,貞祐二年進士云。



 仙部將董祐有戰功,詔賜虎符,仙畏其偪己,久不與佩。祐憾之,乃結官奴欲殺仙,猶豫未敢發。近侍局使完顏四和有謀敢斷,嘗徵兵鄧州,圉牧使移剌呆合有異志。六四和以計誅之。祐使謂四和曰:「仙終不肯入援,祐等位卑,力不能誅,惟
 君為國家圖之。」四和曰:「已殺呆合,復殺武仙,他日使者來,人誰肯信。」不從。仙知祐嘗有此謀,使祐使河北,其後竟殺之。



 三月,仙以聖朵軍食不足,徙軍鄧州,仰給于鄧州總帥移剌瑗。鄧州倉廩亦乏,乃分軍新野、順陽、淅川就食民家。遣講議官朱概、劉琢往襄陽,借糧於宋制置使史嵩之。琢、概持兩端,畏留,乃以情告史嵩之曰:「仙兵勢不復振矣。」且曰:「名為借糧,實欲納款,待將軍一諾耳。」嵩之以為實然,遣田俊持書報仙。四月,仙遣大理少卿張伯直取糧于襄陽,屯軍小江口以待之。嵩之聞張伯直至大喜,謂仙送款矣,發書乃謝狀也,大怒,留伯直不
 遣。



 仙自順陽入鄧州,移剌瑗畏逼,以女女仙,仙不疑,納之,乃還順陽。鄧州糧盡,瑗終疑仙。五月,瑗舉城降宋。嵩之益知仙軍虛實,使孟珙率兵五千襲仙軍于順陽。是時,仙令士卒刈麥供軍,未至二里許,始覺,仙率帳下百餘人迎擊之,孟珙不敢前。俄頃,軍士稍集,有五六百人,大敗珙軍。珙與數百人脫走,生擒其統制、統領數十人,獲馬千餘。至是,概、琢妄謂將納款於嵩之之語泄矣,仙皆誅之。



 移剌瑗本名粘合,字廷玉。世襲契丹猛安,累功鄧州便宜總帥。既至襄陽,使更姓名,稱歸正人劉介,具將校禮謁制置使。瑗大悔恨,明年三月,疽發背死。



 孟珙
 雖敗而去,仙懼宋兵復來,七月,徙淅川之石穴。是時,哀宗在蔡州,遣近侍兀顏責仙赴難,詔曰:「朕平日未嘗負卿,國家危難至此,忍擁兵自恃,坐待滅亡邪?」將士聞之,相視哽咽,皆願赴難與國同生死。仙懼眾心有變,乃殺馬牛,與將士三千人歃血盟誓,不負國家,眾乃大喜。無何,仙復謂眾曰:「蔡州道梗,吾兵食少,恐不能到。且蔡不可堅守,縱到亦無益。近遣人覘視宋金州,百姓據山為柵極險固,廣袤百里,積糧約三百萬石。今與汝曹共圖之,可不勞而下,留老弱守此寨以為根本,然後選勁勇趨蔡,迎上西幸,未晚也。」眾未及應,即令戒行李。取淅川
 溯流而上,山路險阻,霖雨旬日水湍悍,老幼溺死者不可勝數,糧食絕,軍士亡者八九。仙計無所出,八月,乃由荊子口東還,自內鄉將入聖朵寨,至峽石左右八疊秋林,聞總領楊全已降宋,留秋林十日乃遷大和。九月,至黑谷泊,進退失據,遂謀北走,行部尚書盧芝、侍郎石玠不從。



 芝字庭瑞,河東人,任子補官,以西安軍節度使行尚書。玠字子堅,河中人,崇慶二年進士,以汝州防禦使行侍郎。二人相與謀曰:「吾等知仙不恤國家久矣。諫之不從,去之未可,事至今日,正欠蔡州一死耳。假若不得到蔡州,死於道中,猶勝死於仙也。」既去,仙始覺,追玠殺
 之。芝走至南陽,為土賊所害。



 甲午,蔡州破。糧且盡,將士大怨,皆散去。仙無所歸,乃從十八人北渡河,又亡五人。五月,趨澤州,為澤之戍兵所殺。



 張甫,賜姓完顏氏。初歸順大元。涿州刺史李瘸驢招之,興定元年正月,甫與張進俱來降。東平行省蒙古綱承制除甫中都路經略使,進經略副使。二年,苗道潤死,河北行省侯摯承制以李瘸驢權道潤中都路經略使,甫與張柔為副。頃之,苗道潤之眾請以靖安民代道潤。是時,張柔、安民實分掌道潤部眾,朝廷乃以瘸驢為中都東路經略使,自雄、霸以東皆隸之。



 甫、進與永定軍節度
 使賈仝不協,以兵相攻,奪據仝地,取仝馬以遺經略使李瘸驢,瘸驢受之。朝廷怪瘸驢不能和輯州府,乃有向背,召瘸驢別與官職。召東平蒙古綱講睦甫與賈仝。綱遣同知安武軍王郁、博野令高常住往平之,輒留瘸驢不遣,因奏曰:「張甫本受瘸驢招降,情意厚善,今遣郁先與瘸驢議所以平之者然後可。況甫等不識禮義之人,瘸驢就徵則皆自疑,恐生他變,故不避專擅之罪。」詔從綱奏。未幾,賈仝復以兵捕甫部民,殺甫參議官邢畢。甫率兵攻之,賈仝敗走,遂自縊死。甫請符印以安輯部眾,詔與之。



 無何,李瘸驢歸順大元。甫為中都東路經略使、
 遙授同知彰德府事、權元帥右都監。三年,張進為中都南路經略使。甫奏:「真定兵衝,乞遣重臣與恒山公武仙併力守之。」不報。及真定不守,甫復奏:「權元帥右都監柴茂保冀州水寨,孤立無援,若不益兵,非臣之所知也。」



 四年,甫封高陽公,以雄、莫、霸州,高陽、信安、文安、大城、保定、靜海、寶坻、武清、安次縣隸焉。元光元年,移剌眾家奴不能守河間,甫居之信安。是歲,以功進金紫光祿大夫,始賜姓完顏。二年二月,張進亦遷元帥左監軍,賜姓完顏。



 靖安民,德興府永興縣人。貞祐初,充義軍,歷謀克、千戶、總領、萬戶、都統,皆隸苗道潤麾下。以功遙授定安縣令,
 遷涿州刺史,遙授順天軍節度使。充提控。興定元年,遙授安武軍節度使。興定二年,遷知德興府事、中都路總領招撫使。是歲,苗道潤死,安民代領其眾,行省承制以涿州刺史李瘸驢權中都路經略使。三年,詔瘸驢自雄、霸以東為中都東路經略使,自易州以西安民為中都西路經略使。西山義軍屯壘諸招撫皆隸焉。



 四年,遙授知德興府事,權元帥左監軍,行中都西路元帥府事。三月,安民上書曰:「苗道潤撫定州縣五十餘城,其功甚大,西京路經略使劉鐸嫉其功,反間賈瑀、李琛與道潤不協,轉相攻伐,竟以陰謀殺道潤。鐸令所部劉智元等掠
 鎮撫孫資孫、招撫楊德勝家人二十餘口,錮之山寨。若鐸常居此,恐致敗事。」劉鐸亦遣副使劉璋詣南京自訴,且言:「安民侵入飛狐之境,冒濫封拜,誘惑人心,強抑總領馮通等輸銀粟。索飛狐總領王彥暉,彈壓劉智元、杜貴,欲充偏裨。彥暉等拒之,輒殺貴而杖智元,竟驅彥暉而去。」又言:「經略職卑,以致從宜李柏山等日謀見害。乞許罷去。」廷議,劉鐸本行招誘逋亡,今乃與安民互相論列,以起爭端。苗道潤死,安民實代領其眾,彥暉等軍本隸道潤,當聽安民節制。乃召鐸還。頃之,封易水公,以涿、易、安肅、保州,君氏川、季鹿、三保河、北江、礬山寨、青白口、
 朝天寨,水谷、懽谷、車安寨隸焉。十月,安民出兵至礬山,復取簷車寨。



 大元兵圍安民所居山寨,守寨提控馬豹等以安民妻子及老弱出降,安民軍中聞之駭亂,眾議欲降以保妻子。安民及經歷官郝端不肯從,遂遇害。詔贈金紫光祿大夫。



 郭文振,字拯之,太原人。承安二年進士。累官遼州刺史。貞祐四年,昭義節度使必蘭阿魯帶請升遼州為節鎮,廷議遼州城郭人戶不稱節鎮,而文振有功當遷,乃以本官充宣差從宜都提控。興定元年,詔文振接應苗道潤,恢復中都,會道潤與賈仝相攻而止。



 文振治遼州,深
 得眾心。興定三年,遷遙授中都副留守,權元帥左都監,行河東北路元帥府事,刺史、從宜如故。文振招降太原東山二百餘村,遷老幼于山寨,得壯士七千,分駐營柵,防護秋獲。文振奏:「若秋高無兵,直取太原,河東可復。」優詔許之。十月,權元帥右都監、行元帥府事,與張開合堅、臺州兵復取太原。四年,詔升樂平縣為皋州,壽陽縣西張寨為晉州,從文振之請也。



 文振上疏曰:「揚子雲有言:『御得其道,則天下狙詐咸作使;御失其道,則天下狙詐咸作敵。』有天下者審所御而已。河朔自用兵之後,郡邑蕭然,並無官長,武夫悍卒因緣而起以為得志,僭越名
 位,瓜分角競,以相侵攘,雖有內除之官,亦不得領其職,所為不法,可勝言哉?乞行帥府擅請便宜,妄自誇張以尊大其權,包藏之心蓋可知也。朝廷因而撫之,假權傅授,至與各路帥府力侔勢均,不相統屬。陜西行省總為節制,相去遼遠,道路梗塞,卒難聞知。故飛揚跋扈,無所畏憚,鄰道相望,莫敢誰何。自平陽城破以來,河北不置行省,朝廷信臣不復往來布揚聲教,但令曳剌行報而已。所司勞以酒食,悅以貨財,借其聲,共欺朝廷。姦倖既行,遂至驕恣,變故之生,何所不有,此臣所以夙夜痛心而為之憂懼也。乞分遣公廉之官,遍詣訪察,庶知所
 在利害之實。伏見澤、潞等處芻糧猶廣,人民猶眾,地多險阻,乞選重臣復置行省,皆聽節制,上下相維,可臂指使之,則國勢日重,姦惡不萌矣。」是時,澤、潞已詔張開規劃,不能盡用文振之言,但令南京兵馬使術甲賽也行帥府於懷、孟而已。是歲,封晉陽公,河東北路皆隸焉。



 文振奏:「孟州每以豪猾不逞之人攝行州事,朝廷重於更代,就令主之。去年,伯德和攝刺史,提控伯德安殺之,奪其職。河東行省以陳景璠代安,安內不能平,因誣告景璠死罪,朝廷未及按問,安輒逐之。恥受臣節制,宣言于眾,待道路稍通,當隸恒山公節制。今真定已不守,安猶
 向慕不已。臣徵兵諸郡,安輒詭辭不遣。臣若興師,是自生一敵,非國家之便也。聞安有女,臣輒違律令為姪孫述娶之,安遂見許。臣非願與安為姻,為公家計,屑就之耳。自結親以來,安頗循率以從王事,法不當娶而輒娶之,敢以此罪為請。」宣宗嘉其意,遣近臣慰諭之。文振復奏:「武仙所統境土甚大,雖與林州元帥府共招撫之,乞更選本土州縣官,重其職任,同與安集,可使還定。」宣宗用其策。



 五年,文振奏:「臣所統嵐、管、庾、石、寧化、保德諸州,境土闊遠,不能周知利害,恐誤軍國大計。伏見葭州刺史古里甲蒲察智勇過人,深悉河東事勢,乞令行元帥
 府事,或為本路兵馬都總管,與臣分治。」詔文振就擇可者處之便地,仍受文振節制。



 上黨公張開以厚賞誘文振將士,頗有亡歸者。詔分遼、潞粟賑太原饑民,張開不與。文振奏其事,詔遣使慰諭之。文振復申前請,以葭州刺史古里甲蒲察分治嵐、管以西諸州,制可,仍令防秋後再度其宜。文振請分上黨粟以贍太原,詔文振與張開計度。頃之,詔以石州隸晉陽公府。


元光元年,林州行元帥府惟良得罪召還,文振奏:「近聞惟良召還,臣竊以為不可。惟良在林州五歲,政尚寬厚,大得民心,今茲被召,軍民遮路泣留。其去未幾,
 \gezhu{
  山義}
 尖之眾作亂,逐招撫使
 康瑭。乞遣惟良還林州為便。」不許。



 文振上書:「乞遣前平章政事胥鼎行省河北,諸公府、帥府並聽節制,詔諭百姓使知不忘遺黎之意,然後以河南、陜西精銳併力恢復。」不報。文振復奏:「河朔百姓引領南望,臣再四請於樞府,但以會合府兵為言。公府雖號分封,力實單弱,且不相統攝,所在被兵。朝廷不即遣兵復河北,人心將以為舉河朔而棄之,甚非計也。」文振大抵欲起胥鼎為行省,定河北,朝廷不能用。



 二年,詔文振應援史詠復河東。是歲,遼州不能守,徙其軍於孟州,以部將郝安等為文振副,護沿山諸寨。文振辭公府,詔不許。頃之,文振部將汾
 州招撫使王遇與孟州防禦使納蘭謀古魯不相能,復徙衛州,然亦不可以為軍,迄正大間,寓于衛而已。



 胡天作,字景山,管州人。初以鄉兵守禦本州,累功少中大夫、管州刺史。興定二年,遙授同知太原府事,刺史如故。是歲,平陽失守,改同知平陽府事。三年,復取平陽,天作言:「汾、潞皆置帥府,平陽大鎮,今稍完復,所管州縣,不下十萬戶,復業者相繼不絕,其過汾、潞遠甚,宜一體置之。」是時,晉安、嵐州皆有帥府,乃以天作充便宜招撫使、權元帥左都監。四年,封平陽公,以平陽、晉安府,隰、吉州隸焉。天作請以晉安府之翼城縣為翼州,以垣曲、絳縣
 隸焉。置平水縣于汾河之西,朝廷皆從之。



 初,軒成本隸程琢麾下,琢死,成率眾保隰州,以為同知隰州軍州事、兼提控軍馬。成增繕器甲,招納亡命,頗有他志。是時,隰州方用兵,未可制,天作請增置要害州縣,以分其勢。隰州之境蒲縣最居其衝,可改為州,隰川之仵城鎮可改為縣,選官守備。詔升蒲縣為蒲州,以大寧縣隸之,仵城鎮為仵城縣。天作守平陽凡四年,屢有功,詔錄其子定哥為奉職。



 元光元年十月,青龍堡危急,詔遣古里甲石倫會張開、郭文振兵救之,次彈平寨東三十里,不得進。知府事術虎忽失來、總領提控王和各以兵歸順,臨城
 索其妻子,兵民皆潰,執天作出。天作已歸順,詔誅忽失來子之南京者,命天作子定哥承應如故。天作已受大元官爵,佩虎符,招撫懷、孟之民,定哥聞之,乃自經死,贈信武將軍、同知睢州軍州事。詔張開、郭文振招天作,天作至濟源,欲脫走,先遣人奏表南京,大元大將惡其反復,遂誅之。



 天作死後,宣宗以同知平陽府事史詠權行平陽公府事,後封平陽公。平陽初破,詠父祚、母蕭氏藏於窟室,索出之,使祚招詠,祚乃自縊死,蕭氏逃歸。詠妻梗氏亦自死。宣宗贈祚榮祿大夫、京兆郡公,謚成忠。蕭氏封京兆郡太夫人,賜號歸義。梗氏贈京兆郡夫人,謚
 義烈。未幾,詠乞內徙,徙其軍于解州河中府。



 張開,賜姓完顏氏,景州人。至寧末,河北兵起,開團結鄉兵為固守,累功遙授同知清州防禦事,兼同知觀州事。貞祐四年,開率所部復取河間府及滄、獻二州十有三縣。開有宣撫司留付名宣敕二百道,奏乞從權署置,就任所復州縣舊官,闕者補之。詔遷同知觀州軍州事。開復清州,乞輸鹽易糧,詔與之糧。遷觀州刺史、權本州經略使。至是,始賜姓完顏氏。開奏乞許便宜,及論淇門、安陽、黎陽皆作堰塞水,河運不通,乞開發水道,不報。觀州糧盡,是歲秋,徙軍輝州,乞麥種三千石、驢騾三百或
 寶券二百貫,戶部不與。御史臺奏:「開自觀州轉戰來此,久著勞績,欲令其軍耕種以自給,有司計小費拒不與。乞斷自宸衷,與之麥種,若無牛可與,給以寶券。」制可。


是歲,潼關不守,被召入衛南京。興定元年,遙授澤州刺史。二年,遙授同知彰德府、兼總領提控。三年,充潞州招撫使。林州元帥府徙潞人實林州,既復遣還。開乞隸晉安元帥府,或與林州並置元帥府,各自為治。十月,開以權昭義軍節度使、遙授孟州防禦使、權元帥左都監、行元帥府事,與郭文振共復太原。四年,封上黨公,以澤、潞、沁州隸焉。五年,詔復以涉縣為崇州,從開請也。元光元年,
 復取高平縣及澤州。二年,大戰壺關,有功。既而潞州危急,開奏:「封建公府以固屏翰,今胡天作出平陽,郭文振南徙河東,公府獨臣與史詠而已。乞升澤、沁二州為節鎮,以重守禦。」詔以澤為忠昌軍,沁為義勝軍。林州
 \gezhu{
  山義}
 尖寨眾亂,逐招撫使康瑭,推杜仙為招撫使,開請以盧芝瑞為副,代領其眾。又奏:「比聞郭文振就食懷、孟,史詠徙解州,高倫遷葛伯寨,各自保守,民安所仰哉?臣領孤軍,內無儲峙,外無應援,臣不敢避失守之罪,恐益重朝廷之憂。」



 正大間,潞州不守,開居南京,部曲離散,名為舊公,與匹夫無異。天興初,起復,與劉益為西面元帥,領安平
 都尉紀綱軍五千攻衛州,敗績于白公廟。是時,哀宗走歸德,開與劉益謀收潰兵從衛,不果,遂與承裔西走,皆為民家所殺。



 初置公府,開與恒山公武仙最強。後駐兵馬武山,遣人間道請糧二萬石,用事者難之,止給二千石。公府將佐得報皆不敢白,開聞,置酒召諸將曰:「朝廷待某特厚,今日與諸君一醉。」諸將問故,曰:「頃以糧竭為請,祈二萬而得二千,是吾君相不以武仙輩待我也。」是時,郭文振處開西北,當兵之衝,民貧地瘠,開又不奉命以糧賑文振軍。文振窮竄,開勢愈孤,以至於敗。



 燕寧,初為莒州提控,守天勝寨,與益都田琢、東平蒙古
 綱相依為輔車之勢。山東雖殘破,猶倚三人為重。紅襖賊王公喜據注子堌,率眾襲據沂州。寧擊走之,遂復沂州,語在《田琢傳》。寧既屢破紅襖賊,招降胡七、胡八,引為腹心,賊中聞之多有欲降者。累官遙授同知安化軍節度使事、山東安撫副使。興定四年,封東莒公,益都府路皆隸焉。五年,與蒙古綱、王庭玉保全東平,以功遷金紫光祿大夫。還天勝,戰死。蒙古綱奏:「寧克盡忠孝,雖位居上公,祖考未有封爵,身沒之後老稚無所衣食,乞降異恩以勵節義之士。」詔贈故祖皋銀青榮祿大夫,祖母張氏范陽郡夫人,父希遷金紫光祿大夫,母彭氏、繼母許
 氏、妻霍氏皆為範陽郡夫人,族屬五十二人皆廩給之。



 自益都張林逐田涿,繼而寧死,蒙古綱勢孤,徙軍邳州,山東不復能守矣。



 贊曰:苗道潤死,中分其地,靖安民有其西之半,中分以東者其後張甫有之,然無北境矣。大凡九公封建,《宣宗實錄》所載如此。他書載滄海公張進、河間公移剌中哥、易水公張進、晉陽公郭棟,此必正大間繼封,如史詠繼胡天作者,然不可考矣。



\end{pinyinscope}