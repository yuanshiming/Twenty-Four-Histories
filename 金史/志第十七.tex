\article{志第十七}

\begin{pinyinscope}

 禮九



 ○國初即位儀



 收國元年春正月壬申朔,諸路官民耆老畢會,議創新儀,奉上即皇帝位。阿離合懣、宗翰乃陳耕具九,祝以闢土養民之意。復以良馬九隊,隊九匹,別為色,并介胄弓矢矛劍奉上。國號大金,建元收國。天會元年九月六日,皇弟諳版孛極烈即皇帝位。己未,告祀天地。丙寅,大赦,改元。



 ○
 受尊號儀



 皇統元年正月二日,太師宗乾率百僚上表,請上皇帝尊號,凡三請,詔允。七日,遣上京留守奭告天地社稷,析津尹宗強告太廟。十日,帝服袞冕御元和殿,宗乾率百僚恭奉冊禮。冊文云云,「臣等謹奉玉冊、玉寶,上尊號曰崇天體道欽明文武聖德皇帝」。是日,皇帝改服通天冠,宴二品以上官及高麗、夏國使。十二日,恭謝祖廟,還御宣和門,大赦,改元。



 大定七年,恭上皇帝尊號。前三日,遣使奏告天地宗廟社稷。前二日,諸司停奏刑罰文字。百官習儀於大安殿庭。兵部帥其屬,設黃麾仗於大安殿門之內外。宣徽院帥儀鸞司,於前一日設受
 冊寶壇於大安殿中間,又設御榻於壇上,又設冊寶幄次於大安殿門外,及設皇太子幕次於殿東廊,又設群官次于大安門外。大樂令與協律郎前一日設宮縣於殿庭,又設登歌樂架于殿上,立舞表于殿下。符寶郎其日俟文武群官入,奉八寶置于御座左右,候上冊寶訖,復舁寶還所司。其日質明,奉冊太尉、奉寶司徒、讀冊中書令、讀寶侍中以次應行事官,並集於尚書省,俟冊寶興,乘馬奉迎。冊寶至應天門,下馬由正門步導入,至大安殿門外,置冊寶於幄次。舁冊寶床弩手人等分立於左右。文武群官並朝服入次。攝太常卿與大樂令帥工
 人入就位,協律郎各就舉麾位。舁冊寶案官由西偏門先入,置案於殿東西間褥位,置訖,各退于西階冊寶位後。捧冊官,捧寶官、舁冊匣官、舁寶盝官由西偏門先入,至殿西階下冊寶褥位之西,東向立,俟閣門報。



 通事舍人引攝侍中版奏:「中嚴。」訖,典儀、贊者各就位。閣門官引文武百僚分左右入,於殿階下磚道之東西,相向立。符寶郎奉八寶由西偏門分入,升置殿上東西間相向訖,分左右立於寶後。通事舍人引攝侍中版奏:「外辦。」扇合,服袞冕以出,曲直甲蓋、侍衛警蹕如常儀。殿上鳴鞭,訖,殿下亦鳴鞭。初索扇,協律郎跪,俯伏,興,舉麾。工鼓柷,奏《
 乾寧之曲》。出自東房,即座,儀使副添香,爐煙升,扇開,簾捲。協律郎偃麾,戛敔,樂止。太常博士、通事舍人自冊寶幄次分引冊,太常卿前導,吏部待郎押冊而行,奉冊太尉、讀冊中書令、舉冊官於冊後以次從之。次太常博士、通事舍人二員分引寶,禮部侍郎押寶而行,奉寶司徒、讀寶侍中、舉寶官於寶後以次從之。由正門入,宮縣奏《歸美揚功之曲》。太常卿於冊床前導,至第一墀香案南,藉寶冊褥位上少置。太常卿與舉冊寶官退於冊寶稍西,東向立。應博士、舍人立於其後,舁冊寶床弩手、傘子官等又於其後,皆東向。太尉、司徒、中書令。侍中皆於冊
 後,面北以次立。吏部侍郎、禮部侍郎次立於其後。立定,樂止。閣門舍人分引東西兩班群官合班,轉北向立,中間少留班路。俟立定,太常博士、通事舍人四員分引太尉、司徒、中書令、侍中、吏部禮部侍郎以次各復本班,訖,博士、舍人退以俟。初引時,樂奏《歸美揚功之曲》,至位立定,樂止。典儀曰:「拜。」贊者承傳,太尉以下應在位官皆舞蹈,五拜。班首出班起居訖,又贊:「再拜。」如朝會常儀。



 太常博士、通事舍人四員再引太尉、司徒、中書令、侍中、吏禮部侍郎復進至冊寶所稍南,立定。舁冊寶床弩手,傘子官並進前,舉冊寶床興。太常博士、通事舍人二員分引
 冊,太常卿前導,吏部侍郎押冊而行,奉冊太尉、讀冊中書令、舉冊官於冊後以次從之。冊初行,樂奏《肅寧之曲》。次通事舍人、太常博士又二員分引寶,禮部侍郎押寶而行,奉寶司徒、讀寶侍中、舉寶官於寶後以次從之,詣西階下,至冊寶褥位少置冊北,寶南,樂止。舁冊寶床弩手、傘子官等退於後稍西,樂向立。



 捧冊官與舁冊官並進前,取冊匣升。太常博士、通事舍人分引冊,太常卿側身導冊先升,奉冊太尉、讀冊中書令、舉冊官、捧冊官於冊後以次從升。冊初行,樂奏《肅寧之曲》。進至殿上,博士舍人分左右於前楹立以俟,讀冊中書令於欄子外前楹稍
 西立以俟,舉冊官、捧冊官立於其後。奉冊太尉從升,至褥位,搢笏,少前跪置訖,執笏,俯伏,興、樂止,退於前楹稍西立以俟。太常博士立於後。太常卿少退東向立。舁冊官立於其後,皆東向。捧冊官先入,舉冊官次入,讀冊中書令又次入。捧冊官四員皆搢笏雙跪捧。舉冊官二員亦搢笏,兩邊單跪對舉。中書令執笏進,跪稱:「中書令臣某讀冊。」讀訖,俯伏,興。中書令俟冊興,先退。通事舍人引,降自東階,復本班。訖,太常卿降復寶床前,舁冊官並進,與捧冊官等取冊匣興,置於殿東間褥位案上,西向。捧舉冊官等降自東階,還本班。舁冊官亦退。太常博士引
 奉冊太尉降自西階,東向立以俟。次捧寶官與舁寶官俟讀冊中書令讀訖出,並進前,取寶盝升。太常博士、通事舍人分引寶,太常卿側身導寶,先升。奉寶司徒、讀寶侍中、舉寶官、捧寶官於寶後以次從升。寶初行,樂奏《肅寧之曲》,進至殿上,博士舍人俱退不升,並於前楹稍西立俟。讀寶侍中於欄子外前楹間稍西立以俟。舉寶官、捧寶官立於其後。奉寶司徒從升,至褥位,搢笏,少前跪置,訖,執笏,俯伏,興,樂止。司徒退於前楹西,立以俟。太常卿少退,東向立。舁寶官立於其後,皆東向。捧寶官先入,舉寶官次入,讀寶侍中又次入。捧寶官四員皆搢笏雙
 跪捧。舉寶官二員亦搢笏兩邊單跪對舉。侍中執笏進,跪稱:「侍中臣某讀寶。」讀訖,俯伏,興。侍中俟寶興先退,通事舍人引,降自西階,復本班,訖,舁寶官進前,與捧寶舉寶官等取寶盝興,置於殿之西間褥位案上,東向。捧寶舉寶等與太常卿俱降自西階,及吏部侍郎皆復本班。舁寶官亦退。太常博士引奉寶司徒次奉冊太尉,東向立定。



 博士舍人贊引太尉司徒進,詣第一墀香案南褥位立定,博士舍人稍退。典儀曰:「拜。」贊者承傳,在位官皆再拜,訖,博士舍人二員引太尉詣東階升,宮縣奏《純誠享上之曲》,至階,止。閣門使二員引太尉進至前,立定,
 樂止。閣門使揖贊太尉拜跪賀,殿下閣門揖百僚躬身,太尉稱「文武百僚具官臣等言」,致賀詞云云,俯伏,興,退至階上。博士舍人分引太尉降至東階,初降,宮縣作《肅寧之曲》,復香案南褥位立定,樂止。博士舍人少退。典儀曰:「拜。」贊者承傳,太尉、司徒及在位群官俱再拜舞蹈,三稱「萬歲」,又再拜。訖,通事舍人引攝侍中升自東階前楹間,躬承旨,退臨階西向,稱:「有制。」典儀曰:「拜。」贊者承傳,太尉、司徒及在位群官俱再拜,躬身宣詞云云,宣訖,通事舍人引侍中還位。典儀曰:「拜。」贊者承傳,階上下應在位群官俱再拜舞蹈,三稱「萬歲」,又再拜。訖,博士舍人分引
 太尉、司徒就百僚位。初引,宮縣作《肅寧之曲》,至位立定,樂止。閣門舍人分引應北面位群官,各分班東西相向立定。通事舍人引攝侍中并自東階,當前楹間,跪奏:「禮畢。」俯伏,興,引降還位。扇合,簾降。協律郎俯伏,興,舉麾,工鼓柷,奏《乾寧之曲》。降座,入自東房,還後閣,進膳,侍衛警蹕如儀。扇開,樂止。捧冊官帥舁冊床人,捧寶官帥舁寶床人,皆升殿取匣、盝,蓋訖,置於床前。引進司官前導,通事舍人贊引,詣東上閣門上進。通事舍人分引文武百僚等以次出,歸幕次,賜食,以俟上壽。上冊寶禮畢,有司供辦御床及與宴群官位,並如曲宴儀。



 攝太常卿大
 樂令帥工人入,並協律郎各就舉麾位,俟舍人報。通事舍人引三師以下文武百僚親王宗室等分左右入,至殿階下稍南,東西相向立。通事舍人先引攝侍中版奏:「中嚴。」少頃,又奏:「外辦。」扇合,鳴鞭。協律郎跪,俯伏,興,工鼓柷,宮縣奏《乾寧之曲》。服通天冠、絳紗袍,即座,簾捲。內侍贊:「扇開。」殿上下鳴鞭,戛敔,樂止。儀使副等添香,爐煙升。通事舍人引班首以下合班,樂奏《肅寧之曲》,至北向位,重行立定,中間少留班路。通事舍人引攝侍中詣東階升,至殿上少立。閣門舍人引禮部尚書出班前,北向俯伏,跪奏,稱:「禮部尚書臣某言,請允群臣上壽。」俯伏,興,躬
 身。通事舍人引攝侍中少退。舍人贊:「禮部尚書再拜。」訖,贊:「祗候。」復本班。內侍局進御床入。次良醖令於殿下橫階南酹酒,訖,典儀曰:「拜。」贊者承傳,在位官皆再拜,隨拜三稱「萬歲」,訖,平立。



 太常博士、通事舍人分引攝上公由東階升。初升,宮縣奏《肅寧之曲》。殿上,舍人少退,二閣使揖上公進,至進酒褥位,樂止。宣徽使以爵授上公,上公搢笏,受爵。詣榻前跪進。受爵訖,上公執槃授宣徽使,訖,二閣使揖上公入欄子內,贊:「拜。」跪。殿下,閣門揖百僚皆躬身。通事舍人揖攝侍中進,詣前楹間,躬承旨,退臨階西向稱:「有制。」典儀曰:「拜。」贊者承傳,上公及在位群官皆
 再拜,隨拜三稱「萬歲」,訖,躬身宣曰:「得公等壽酒,與公等內外同慶。」閣門舍人贊宣諭訖,上公與百僚皆舞蹈五拜,訖,閣門舍人引百僚分班東西序北向立。博士舍人再引上公自東階升,宮縣奏《肅寧之曲》,至進酒褥位,樂止。上公搢笏,宣徽使授上公槃,上公詣欄子內褥位,跪舉酒,宮縣奏《景命萬年之曲》,飲訖,樂止。上公進受虛爵訖,復褥位,以爵授宣徽使,訖,二閣使揖上公退,內侍局舁御床出。博士舍人並進前分引,降自東階,宮縣作《肅寧之曲》。閣門舍人分引東西兩班,隨上公俱復北向位,立定,樂止。典儀曰:「拜。」贊者承傳,在位官皆再拜,三稱「萬
 歲」,訖,平立。殿上,通事舍人揖攝侍中進,詣前楹間,躬承旨,退臨階西向,閣門官先揖,百僚躬身,侍中稱:「有制。」典儀曰:「拜。」贊者承傳,在位官皆再拜,訖,躬身宣曰:「延王公等升殿。」典儀曰:「拜。」贊者承傳,在位官皆再拜,訖,搢笏,舞蹈,又再拜,訖。太常博士、通事舍人引王公以下合赴宴群官,分左右升殿,不與宴群官分左右捲班出,宮縣奏《肅寧之曲》。百僚至殿上坐後立,樂止。內侍局進御床入。依尋常宴會,再進第一爵酒,登歌奏《聖德昭明之曲》,飲訖,樂止。執事者行官酒,宮縣作《肅寧之曲》,文舞入,觴行一周,樂止。尚食局進食,執事者設群官食,宮縣奏《保
 大定功之舞》,三成,止,出。又進第二爵酒,登歌奏《天贊堯齡之曲》,飲訖,樂止。執事者行群官酒,宮縣作《肅寧之曲》,武舞入,觴行一周,樂止。尚食局進食,執事者設群官食,宮縣奏《萬國來同之舞》,三成,止,出。又進第三爵酒,登歌奏《慶雲之曲》,飲訖,樂止。執事者行群官酒,宮縣作《肅寧之曲》,觴行一周,樂止。尚食局進食,執事者設群官食,宮縣奏《肅寧之曲》,食畢,樂止。閣門官分揖侍宴群官起,立於席後。通事舍人引攝侍中詣榻前,俯伏,興,跪奏:「侍中臣某言,禮畢。」俯伏,興。閣門舍人分引群官俱降東西階,內侍局舁御床出,宮縣作《肅寧之曲》,至北向位立定,樂
 止。典儀曰:「拜。」贊者承傳,在位官皆再拜,訖,搢笏,舞蹈,又再拜,訖,再分班東西序立。扇合,簾降,殿上下鳴鞭。協律郎俯伏,跪,舉麾,興,工鼓柷,奏《乾寧之曲》。降座,入自東房,還後閣,侍衛如來儀。內侍贊:「扇開。」戛敔,樂止。通事舍人引攝侍中版奏:「解嚴。」所司承旨放仗,在位群官皆再拜以次出。



 ○元日聖誕上壽儀



 皇帝升御座,鳴鞭、報時畢,殿前班小起居,各復侍立位。舍人引皇太子并臣僚使客合班入,至丹墀,舞蹈五拜,平立。閣使奏諸道表目,皇太子以下皆再拜。引皇太子升殿褥位,搢笏,捧盞盤,進酒,皇帝受
 置於案。皇太子退復褥位,轉盤與執事者,出笏,二閣使齊揖入欄子內,拜跪致詞云:「元正啟祚,品物咸新,恭惟皇帝陛下與天同休。」若聖節則云:「萬春令節,謹上壽卮,伏願皇帝陛下萬歲萬歲萬萬歲。」祝畢,拜,興,復褥位,同殿下群僚皆再拜。宣徽使稱:「有制。」在位皆再拜,宣答曰:「履新上壽,與卿等內外同慶。」聖節則曰:「得卿壽酒,與卿等內外同慶。」詞畢,舞蹈五拜,齊立。皇太子搢笏,執盤,臣僚分班,教坊奏樂。皇帝舉酒,殿上下侍立臣僚皆再拜。皇太子受虛盞,退立褥位,轉盤與執事者,出笏,左下殿,樂止,合班,在位臣僚皆再拜。分引與宴官上殿,次引宋
 國人從至丹墀,再拜,不出班奏:「聖躬萬福。」再拜,喝:「有敕賜酒食。」又再拜,各祗候,平立,引左廊立。次引高麗、夏人從,如上儀畢,分引左右廊立。御果床入,進酒。皇帝飲,則坐宴侍立臣皆再拜。進酒官接盞還位,坐宴官再拜,復坐。行酒,傳宣,立飲,訖,再拜,坐。次從人再拜,坐。三盞,致語,揖臣使并從人立。誦口號畢,坐宴侍立官皆再拜,坐,次從人再拜,坐。食入,七盞,曲將終,揖從人立,再拜畢,引出。聞曲時,揖臣使起,再拜,下殿。果床出。至丹墀,合班謝宴,舞蹈五拜,各祗候,分引出。大定六年正月,上御大安殿,受皇太子以下百官及外國使賀,賜宴,文武五品以
 上侍坐者有定員,為常制。十七年,詔以皇族袒免以上親,雖無官爵封邑,若與宴當有班次。禮官言:「按唐典,皇家周親視三品,大功親、小功尊屬視四品,小功親、緦麻尊屬視五品,緦麻袒免以上視六品。」上命以此制為班次。



 ○朝參常朝儀



 天眷二年五月,詳定常朝及朔,望儀,准前代制,以朔日、六日、十一日、十五日、二十一日、二十六日為六參日。後又定制,以朔、望日為朝參,餘日為常朝。凡朔、望朝參日,百官卯時至冪次,皇帝辰刻視朝,供御弩手、傘子直於殿門外,分兩面排立。司辰入殿報時畢,皇
 帝御殿坐,鳴鞭。閣門報班齊。執擎儀物內侍分降殿階兩傍,面南立。宿衛官自都點檢至左右親衛,祗應官自宣徽閣門祗候,先兩拜,班首少離位,奏:「聖躬萬福。」兩拜。弩手、傘子先於殿門外東西向排立,俟奏「聖躬萬福」時,即就位北面山呼聲喏,起居畢,即相向對立。擎御傘直立左班內侍上。都點檢以次升殿,副點檢在少南,東西相向立。左右衛在殿下,東西相向立。閣門乃引親王班,贊班首名以下再拜,訖,班首少離位,奏:「聖躬萬福。」歸位再拜畢,先退。次引文武百僚班首以下應合朝參官,并府運六品以上官,皆左入,至丹墀之東,西向鞠躬畢,閣
 門通唱,復引至丹墀。閣門贊班首名以下起居,舞蹈五拜,又再拜,畢,領省宰執升殿奏事。殿中侍御史對立於左右衛將軍之北少前,脩起居東西對立於殿欄子內副階下,餘退,右出。初,帝就坐,置寶匣於殿階上東南角。後定制,師傅起居畢,御案始東入,置定,捧案內侍東西分下,侍殿隅。直日主寶捧寶當殿叩欄奏:「封全。」符寶郎及當監印郎中各一員,監當手分令史用印,訖,主寶吏封授主寶,俟奏事畢進封,訖,內侍徹案。若常朝,則親王班退,引七品以上職事官,分左右班入丹墀,再拜,班首稍前起居畢,復位,再拜。宰執升殿,餘官分班退。



 大定
 二年五月,命臺臣定朝參禮。五品以上官職趨朝朝服,入局治事則展皂。自來朝參,除殿前班外,若遇朔望,自七品以上職事官皆赴。其餘朝日,五品以上職事官得赴,六品以下止於本司局治事。如左右司員外郎、侍御史、記注院等官職,雖不係五品,亦赴朝參。若拜詔,則但有職事并七品以上散官,皆赴。朝參,吏員、令譯史、通事、檢法各於本局待,官員朝退,赴局簽押文字,不得於宮給署押。七品以下流外職,遇朝日亦不合入宮。如左右司都事有須合取奏事,乃聽入宮。七品以上職事官,如遇使客朝辭見日,依朔望日,皆赴。若元日、聖節、拜詔、車
 駕出獵送迎、詣祖廟燒飯,但有職事并七品以上散官,皆赴。凡親王宗室已命官者年十六以上,皆隨班赴起居。大定五年,右諫議大夫移剌子敬言:「猛安謀克不得與州鎮官隨班入見,非軍民一體之意。」上是其言,責宣徽院令隨班入見。凡班首遇朝參,有故不赴,以次押班。



 凡五品以上及侍御史,尚書諸司郎中、太常丞、翰林修撰起居注、殿中侍御史、補闕、拾遺赴召,或假一月以上若除官出使之類,皆通班入見辭、謝,餘官於殿門外見。謝班皆舞蹈七拜,辭班四拜,門見謝、辭並再拜。



 ○肆赦儀



 大定七年正月十一日,上尊冊禮畢。十四月,應
 天門頒赦。十一年制同。前期,宣徽院使率其屬,陳設應天門之內外,設御座于應天門上,又更衣御幄於大安殿門外稍東,南向。閣門使設捧制書箱案於御座之左。少府監設雞竿於樓下之左,竿上置大盤,盤中置金雞,雞口銜絳幡,幡上金書「大赦天下」四字,卷而銜之。盤四面近邊安四大鐵鐶,盤底四面近邊懸四大朱索,以備四伎人攀緣。又設捧制書木鶴仙人一,以紅繩貫之,引以轆轤,置於御前欄幹上。又設承鶴畫臺於樓下正中,臺以弩手四人對舉。大樂署設宮縣於樓下,又設鼓一於宮縣之左稍北,東向。兵部立黃麾仗於門外。刑部、
 御史臺、大興府以囚徒集於左仗外。御史臺、閣門司設文武百官位於樓下,東西相向。又設典儀位於門下稍東,南向。宣徽院設承受制書案於畫臺之前。又設皇太子侍立褥位於門下稍東,西向。又設皇太子致賀褥位於百官班前。又設協律郎位於樓上前楹稍東,西向。尚書省委所司設宣制書位於百官班之北稍東,西向。司天臺設雞唱生於東闕樓之上。尚衣局備皇帝常服,如常日視朝之服。尚輦設輦於更衣御幄之前。躬謝禮畢,皇帝乘金輅入應天門,至幄次前,侍中俯伏,跪奏:「請降輅入幄。」俯伏,興。皇帝降輅入幄,簾降。少頃,侍中奏:「中嚴。」又少
 頃,俟典贊儀引皇太子就門下侍立位,通事舍人引群官就門下分班相向立,侍中奏:「外辦。」皇帝服常朝服,尚輦進輦,侍中奏:「請升輦。」傘扇侍衛如常儀,由左翔龍門踏道升應天門,至御座東,侍中奏:「請降輦升座。」宮縣樂作。所司索扇五十柄,扇合,皇帝臨軒即御座,樓下鳴鞭,簾捲扇開,執御傘者張於軒前以障日,樂止。東上閣門使捧制書置於箱,閣門舍人二員從,以俟引繩降木鶴仙人。通事舍人引文武群官合班北向立,宮縣樂作。凡分班、合班則樂作,立定即止。典儀曰:「再拜。」在位官皆再拜,訖,分班相向立。侍中詣御座前承旨,退,稍前南向,宣曰:「
 奉敕樹金雞。」通事舍人於門下稍前東向,宣曰:「奉敕樹金雞。」退復位。



 金雞初立,大樂署擊鼓,樹訖鼓止。竿木伎人四人,緣繩爭上竿,取雞所銜絳幡,展示訖,三呼「萬歲」。通事舍人引文武群官合班北向立。樓上乘鶴仙人捧制書,循繩而下至畫臺,閣使奉承置於案。閣門舍人四員舉案,又二員對捧制書,閣使引至班前,西向稱:「有制。」典儀曰:「拜。」在位官皆再拜,訖,以制書授尚書省長官,稍前搢笏,跪受,訖,以付右司官,右司官搢笏,跪受,訖,長官出笏,俯伏,興,退復位。右司官捧制書詣宣制位,都事對捧,右司官宣讀,至「咸赦除之」。所司帥獄吏引罪人詣班
 南,北向,躬稱:「脫枷。」訖,三呼「萬歲「,以罪人過。右司官宣制訖,西向,以制書授刑部官。跪受訖,以制書加於笏上,退以付其屬,歸本班。典儀曰:「拜。」在位官皆再拜,舞蹈,又再拜。典贊儀引皇太子至班前褥位立定,典儀曰:「拜。」皇太子以下群官皆再拜。典贊儀引皇太子稍前,俯伏,跪致詞,俯伏,興。典儀曰:「再拜。」皇太子以下群官皆再拜,搢笏,舞蹈,又再拜。侍中於御座前承旨,退臨軒宣曰:「有制。」典儀曰:「再拜。」皇太子以下群官皆再拜。侍中宣答,宣訖歸侍位,典儀曰:「再拜。」皇太子已下群官皆再拜,搢笏,舞蹈,又再拜,訖,典贊儀引皇太子至門下褥位,通事舍人引群
 官分班相向立。侍中詣御座前,俯伏,跪奏:「禮畢。」俯伏,興,退復位。所司索扇,宮縣樂作,扇合,簾降,皇帝降座,樂止。樓下鳴鞭,皇帝乘輦還內,傘扇侍衛如常儀。侍中奏:「解嚴。」通事舍人承敕,群臣各還次,將士各還本所。



 ○臣下拜赦詔儀



 宣赦日,於應天門外設香案,及設香輿於案前,又於東側設卓子,自皇太子宰臣以下序班定。閣門官於箱內捧赦書出門置於案。閣門官案東立,南向稱:「有敕。」贊皇太子宰臣百僚再拜,皇太子少前上香訖,復位,皆再拜。閣門官取赦書授尚書省都事,都事跪受,及尚書省令史二人齊捧,同升於卓子讀,在位官皆
 跪聽,讀訖,赦書置於案,都事復位。皇太子宰臣百僚以下再拜,搢笏,舞蹈,執笏,俯伏,興,再拜。拱衛直以下三稱「萬歲」,訖,退。其降諸書,禮亦准此,惟不稱「萬歲」。其外郡,尚書省差官送赦書到京府節鎮,先遣人報,長官即率僚屬吏從,備旗幟音樂彩輿香輿,詣五里外迎。見送赦書官,即於道側下馬,所差官亦下馬,取赦書置彩輿中,長官詣香輿前上香,訖,所差官上馬,在香輿後,長官以下皆上馬後從,鳴鉦鼓作樂導至公,從正門入,所差官下馬。執事者先設案并望闕褥位於庭中,香輿置於案之前,又設所差官褥位在案之側,又設卓子於案之東
 南。所差官取赦書置於案,彩輿退。所差官稱:「有敕。」長官以下皆再拜。長官少前上香,訖,退復位,又再拜。所差官取赦書授都目,都目跪受,及孔目官二員,三人齊捧赦書,同高幾上宣讀,在位官皆跪聽。讀訖,都目等復位。長官以下再拜,舞蹈,俯伏,興,再拜。公吏以下三稱「萬歲」。禮畢。明日,長官率僚屬,音樂送至郭外。



\end{pinyinscope}