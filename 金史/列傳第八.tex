\article{列傳第八}

\begin{pinyinscope}

 ○撒改宗憲本名阿懶習不失宗亨本名撻不也宗賢本名賽里石土門忠本名迪古乃習室思敬本名撒改



 撒改者,景祖孫,韓國公劾者之長子,世祖之兄子也。劾者於次最長。景祖方計定諸部,愛世祖膽勇材略。及諸子長,國俗當異宮居,而命劾者與世祖同邸,劾者專治
 家務,世祖主外事。世祖襲節度使,越劾孫而傳肅宗、穆宗,皆景祖志也。穆宗初襲位,念劾者長兄不得立,遂命撒改為國相。



 穆宗履藉父兄址業,鋤除彊梗不服己者,使撒改取馬紀嶺道攻阿疏,穆宗自將,期阿疏城下會軍。撒改行次阿不塞水,烏延部斜勒勃堇來謁,謂撒改曰:「聞國相將與太師會軍阿疏城下,此為深入必取之策,宜先撫定潺蠢、星顯之路,落其黨附,奪其民人,然後合軍未晚也。」撒改從之,攻鈍恩城,請濟師,穆宗與之,撒改遂攻下鈍恩城,而與穆宗來會阿疏城下。鈍恩在南,阿疏在北,穆宗初遣撒改分道,即會攻阿疏。聞其用斜
 勒計,先取鈍恩城,與初議不合,頗不然之。及遼使來止勿攻阿疏,然後深以先取鈍恩城為功也。及以國相都統討留可、詐都、塢塔等軍,而阿疏亡入於遼,終不敢歸,留可、詐都、塢塔、鈍恩皆降。



 康宗沒,太祖稱都勃極烈,與撒改分治諸部,匹脫水以北太祖統之,來流水人民撒改統之。明年甲午,嗣節度命方至。



 遼主荒於遊畋,政事怠廢,太祖知遼可伐,遂起兵。九月,與遼人戰于界上,獲謝十,太祖使告克于撒改,則以所獲謝十乘馬,撒改及將士皆歡呼曰:「義兵始至遼界,一戰面是勝,滅遼必自此始矣。」遣子宗翰及完顏希尹來賀捷,因勸進,太祖未之
 從也。十月,師克寧江州,破遼師十萬于鴨子河,師還。十二月,太宗及撒改、辭不失率諸將復勸進。收國元年正月朔,太祖即位,撒改行國相如故。伐遼之計決於迪古迺,贊成大計實自撒改啟之。撒改自以宗室近屬,且長房,繼肅宗為國相,既貴且重,故身任大計,贊成如此,諸人莫之或先也。



 太祖即位後,群臣奏事,撒改等前跪,上起,泣止之曰:「今日成功,皆諸君協輔之力,吾雖處大位,未易改舊俗也。」撒改等感激,再拜謝。凡臣下宴集,太祖嘗赴之,主人拜,上亦答拜。天輔後,始正君臣之禮焉。七月,,太宗為諳版勃極烈,撒改國論勃極烈,辭不失阿買
 勃極烈,杲國論昊勃極烈。勃極烈,女直之尊官也。太衣自正位號,凡半歲,未聞有封拜。太宗介弟優禮絕等,杲母弟之最幼者,撒改、辭不失以宗室,同封拜。九月,加國論胡魯勃極烈。天輔五年,薨。太祖往弔,乘白馬,剺額哭之慟。及葬,,復親臨之,賵以所御馬。



 撒改為人,敦厚多智,長於用人,家居純儉,好稼穡。自始為國相,能馴服諸部,訟獄得其情,當時有言:「不見國相,事何從決。」及舉兵伐遼,撒改每以宗臣為內外倚重,不以戰多為其功也。天會十五年,追封燕國王。正隆降封陳國公。大定三年,改贈金源郡王,配饗太祖廟廷,謚忠毅。十五年,詔圖像于
 衍慶宮。子宗翰、宗憲。宗翰別有傳。



 宗憲三名阿懶。頒行女直字書,年十六,選入學。太宗幸學,宗憲與諸生俱謁,宗憲進止恂雅,太宗召至前,令誦所習,語音清亮,善應對。侍臣奏曰:「此左副元帥宗翰弟也。」上嗟賞久之。兼通契丹、漢字。未冠,後宗翰伐宋,汴京破,眾人爭府庫取財物,宗憲獨載圖書以歸。朝廷議制度禮樂,往往因仍遼舊,宗憲曰:「方今奄有遼、宋,當遠引前古,因時制宜,成一代之法,何乃近取遼人制度哉。」希尹曰:「而意甚與我合。」由是器重之。



 撻懶、宗雋唱議以齊地與宋,宗憲廷爭折之,當時不用其言,其後宗弼復
 取河南、陜西地,如宗憲策。以捕宗磐、宗雋功。授昭武大將軍」修國史,累官尚書左丞。熙宗從容謂之曰:「響以河南、陜西地與宋人,卿以為不當與,今復取之,是猶用卿言也。卿識慮深遠,自今以往,其盡言無隱。」宗憲拜謝,遂攝門下侍郎。



 初,熙宗以疑似殺左丞相希尹,久之,察其無罪,深閔惜之,謂宗憲曰:「希尹有大功于國,無罪而死,朕將錄用其孫,如之何?」宗憲對曰:「陛下深念希尹,錄用其孫,幸甚。若不先明死者無罪,生者何由得仕。」上曰:「卿言是也。」即日復希尹官爵,用其孫守道為應奉翰林文字。皇統五年,將肆赦,議覃恩止及女直人,宗憲奏曰:「莫
 非王臣,慶幸豈可有間邪。」遂改其文,使均被焉。轉行臺平章政事。天德初,為中京留守、安武軍節度使。封河內郡王。改太原尹,進封鉅鹿郡王。正隆例奪王爵,再遷震武、武定軍節度使。



 世宗即位,遣使召之,詔曰:「叔若能來,宜速至此,若為紇石烈志寧、白彥敬所遏,亦不煩叔憂。」宗憲聞世宗即位,先已棄官來歸,與使者遇於中都,遂見上於小遼口,除中都留守,即遣赴任。詔與元帥完顏彀英同議軍事。明年,改西京留守。八月,改南京。僕散忠義自行臺朝京師,宗憲攝行臺尚書省事。召為太子太師,上謂宗憲曰:「卿年老舊人,更事多矣,皇太子年尚少,
 謹訓導之。」俄拜平章政,太子太師如故。詔以《太祖實錄》賜宗憲及平章政事完顏元宜、左丞紇石烈良弼、判秘書監溫王爽各一本。



 移刺高山奴前為寧州刺史,以貪污免,世宗以功臣子孫宗族中無顯仕者,以為秘書少監。是時,母喪未除,有司奏其事,宗憲曰:「高山奴傲狠貪墨,不可致之左右。」世宗曰:「朕以其父祖有功耳,既為人如此,豈可玷職位哉。」追還制命,因顧右丞蘇保衡、參政石琚曰:「此朕之過舉,不可不改,卿等當盡心以輔朕也。」有司言,諸路猛安謀克,怙其世襲多擾民,請同流官,以三十月為考。詔下尚書省議,宗憲乃上議曰:「昔太祖
 皇帝撫定天下,誓封功臣襲猛安謀克,今若改為遷調,非太祖約。臣謂凡猛安謀克,當明核善惡,進賢退不肖,有不職者,其弟姪中更擇賢者代之。」上從其議。進拜右丞相。大定六年,薨,年五十九。上輟朝,悼惜者久之,命百官致奠,賻銀一千五百兩、重彩五十端、絹五百匹。



 習不失本作辭不失,後定為習不失,昭祖之孫,烏骨出之次子也。初,昭祖久無繼嗣,與威順皇后徒單氏禱於巫,而生景祖及烏骨出。烏骨出長而酗酒,屢悖其母。昭祖沒,徒單氏與景祖謀而殺之。部人怒,欲害景祖,徒單氏自以為事,而景祖乃得免。



 習不失健捷,能左右射。世
 祖襲節度,肅宗與拒桓赧、散達,戰於斡魯紺出水,已再失利,世祖至軍,吏士無人色。世祖使習不失先陣於脫豁改原,而身出搏戰,敗其步軍。習不失自陣後奮擊之,敗其騎軍,所乘馬中九矢,不能馳,遂步趨而出。方戰,其外兄烏葛名善射,居敵騎中,將射,習不失熟視識之,呼曰:「此小兒,是汝一人之事乎,何為推鋒居前如此。」以弓弰擊馬首而去。是役也,習不失之功居多。桓赧、散達既敗,習不失馬棄陣中者亦自歸。



 世祖嘗疑術甲孛里篤或與烏春等為變,遣習不失單騎往觀,孛里篤與忽魯置酒樓上以飲之。習不失聞其私語暱暱,若將執己者,
 一躍下樓,傍出籓籬之外,棄馬而歸,其勇如此。盃乃約烏春舉兵,世祖至蘇素海甸與烏春遇,肅宗前戰,斜列、習不失佐之,束縕縱火,煙焰蔽天,大敗烏春,執盃乃以歸。太祖獲麻產,獻馘于遼,遼人賞功,穆宗、太祖、歡都、習不失皆為詳隱焉。後與阿里合懣、斡帶俱佐撒改攻留可城,下之。太祖伐遼,使領兵千人,夾侍左右。出河店之役,惟習不失之策與太祖合,卒破十萬之師,挫其軍鋒。遂與太宗、撒改等勸進。收國元年七月,與太宗、撒改、杲俱為勃極烈,習不失為阿買勃極烈云。



 天輔七年,太宗與習不失居守,鄆王昂違紀律失眾,法當死。於是,遼人
 以燕京降,宋人約歲幣。三月,世宗生。習不失謂太宗曰:「兄弟骨肉,以恩掩義,寧屈法以全之。今國家迭有大慶,可減昂以無死,若主上有責言,以我為說。」太宗然之,遂杖昂以聞。太祖每伐遼,輒命習不失與太宗居守,雖無方面功,而倚任與撒改比侔矣。是歲七月,薨會太祖班師道病,太宗奉迎謁見,恐太祖感動而疾轉甚,不敢以薨告。太祖輒問曰:「阿買勃極烈安在?」太宗紿對曰:「今即至矣。」正隆二年,贈開府儀同三司。追封曹國公。大定三年,進封金源郡王,配饗太祖廟廷,謚曰忠毅。



 子鶻沙虎,國初有功,天會間,為真定留守。子撻不也。



 宗亨本名撻不也,性忠謹。天眷初,以宗室子,充護衛。擒宗磐、宗雋有功,加忠勇校尉,遷昭信校尉、尚廄局直長。三年,升本局副使。丁父憂,時宗正官屬,例以材選,宗亨在選中,遂起復,為淑溫特宗室將軍。改會寧府少尹,歷登州刺史,改獻州刺史,為特滿群牧使、同知北京路轉運使,改澤州定國軍節度使。海陵庶人南伐,以本職領武揚軍都總管,過淮。



 世宗即位,以手詔賜宗亨,宗亨得詔,即入朝。大定二年,授右宣徽使,未幾,為北京路兵馬都統,以討契丹賊。右副元帥僕散忠義與窩斡遇于花道,宗亨與左翼萬戶蒲察世傑等,以七謀克
 軍與之戰,失利。及窩斡敗,其黨括里、扎八率眾南奔,宗亨追及之。扎八詐降,宗亨信之。扎八詭曰:「括里遁,願往邀。」宗亨聽其去。大縱軍士,取賊所棄囊橐人畜,多自有之。括里、扎八亡入于宋。坐是,降為寧州刺史。



 宗賢本名賽里,習不失之孫也。從都統杲取中京,襲遼帝于鴛鴦濼。宗翰使撻懶襲耶律馬哥,都統使蒲家奴及賽里等,以兵助之。蒲家奴使賽里、斜野、裴滿胡撻、達魯古廝列、耶律吳十等各率兵分行招諭,獲遼留守迪越家人輜重,并降群牧官木盧瓦,得馬甚多,使逐水草牧之。賽里等趨業迭,遂以偏師深入,敵邀擊之,撒合戰
 沒。蒲家奴至旺國崖西,賽里兵會之。累官至左副點檢。



 天眷二年,方捕宗雋,賽里坐會飲其家,奪官爵。未幾,復官。皇統四年,授世襲謀克,轉都點檢,封豳國公。拜平章政事。進拜右丞相,兼中書令。進拜太保、左丞相,監修國史。罷為左副元帥。無何,復為太保、左丞相,左副元帥如故。進太師,領三省事,兼都元帥,監修國史。出為南京留守,領行臺尚書省事。復為左副元帥,兼西京留守。再為太保,領三省事。復為左丞相,兼都元帥。



 賽里自護衛,未十年位兼將相,常感激,思自效以報朝廷。雖於悼后為母黨,后專政,大臣或因之以取進用,賽里未嘗附之。皇
 太子濟安薨,魏王道濟死,熙宗未有嗣子,賽里勸熙宗選後官以廣繼嗣,不少顧忌於后,后以此怨之。與海陵同在相位,未嘗少肯假借,海陵雖專而心憚賽里,外以屬尊加禮敬而內常忌之。海陵知悼后怨賽里,因與后共力排出之,賽里亦不以是少變。



 胙王常勝死,熙宗納其妻宮中,頃之,殺悼后及妃數人,將以常勝妻為后,未果也。及海陵弒熙宗,詭以熙宗將議立后,召諸王大臣,賽里聞召,以為信然,將入宮,謂人曰:「上必欲立常勝妻為后,我當力爭之。」及被執,猶以為熙宗將立常勝妻,而先殺之也,曰:「誰能為我言者,我死固不足惜,獨念主上
 左右無助耳。」遂遇害。



 石土門,漢字一作神徒門,耶懶路完顏部人,世為其部長。父直離海,始祖弟保活里四世孫,雖同宗屬,不相通問久矣。景祖時,直離海使部人邈孫來,請復通宗系。景祖留邈孫歲餘,厚其餼廩飲食,善遇之。及還,以幣帛數篚為贈,結其厚意。久之,耶懶歲饑,景祖與之馬牛,為助糴費,使世祖往致之。會世祖有疾,石土門日夕不離左右,世祖疾愈辭歸,與握手為別,約它日無相忘。石土門體貌魁偉,勇敢善戰,質直孝友,彊記辯捷,臨事果斷。



 世祖襲位,交好益深,鄰部不悅,遂合兵攻之。石土門使弟
 阿斯懣率二百人南下拒敵,敵兵千人,已出其東據高阜,石土門將五千人迎擊之。敵將斡里本者,勇土也,出挑戰,石土門射中其馬,斡里本反射,射中石土門腹,石土門拔箭,戰愈力。阿斯懣與勇士七人步戰,殺斡里本,諸部兵遂敗。石土門因招諭諸部,使附於世祖,世祖嘉之。後伐烏春、窩謀罕及鈍恩、狄庫德等,皆以所部從戰,有功。



 弟阿斯懣尋卒,及終喪,大會其族,太祖率官屬往焉,就以代遼之議訪之。方會祭,有飛鳥自燕而西,太祖射之,矢貫左翼而墜,石土門持至上前稱慶曰:「烏鳶人所甚惡,今射獲之,此吉兆也。」即以金版獻之,後以本部
 兵從擊高麗。及伐遼,功尤多。王師攻下西京,賜以金牌。其子蟬蠢從行,上語之曰:「吾妃之妹白散者在遼,俟其獲,當以為汝婦。」竟如其言。



 上之西征,諸將皆從,石土門乃率善射者三百人來衛京師,時太宗居守,喜其至,親出迎勞。繼聞黃龍府叛,與睿宗討平之,睿宗賜以奴婢五百人,師還,賞賚良渥。至是卒,年六十一。正隆二年,封金源郡王。子習失、思敬。



 完顏忠本名迪古乃,字阿思魁。石土門之弟。太祖器重之,將舉兵伐遼,而未決也,欲與,迪古乃計事,於是宗斡、宗乾、完顏希尹皆從。居數日,少間,太祖與迪古乃馮肩
 而語曰:「我此來豈徒然也,有謀於汝,汝為我決之。遼名為大國,其實空虛,主驕而士怯,戰陣無勇,可取也。吾欲舉兵,杖義而西,君以為如何?」迪古乃曰:「以主公英武,士眾樂為用。遼帝荒於畋獵,政令無常,易與也。」太祖然之。明年,太祖伐遼,使婆盧火來征兵,迪古乃以兵會師。收國元年十二月,上禦遼主兵,次爻刺,迪古乃與銀術哥守達魯古路。二年,與斡魯、蒲察會斡魯古,討高永昌,破其兵,東京降。遂與斡魯古等禦耶律捏里,敗之於蒺藜山,拔顯州,乾、惠等州降。



 天輔二年,與婁室俱入見,上曰:「遼主近在中京,而敢輒來,各杖之三十。」太祖駐軍草濼,
 迪古乃敢奉聖州,破其兵五千於雞鳴山,奉聖州降。太祖入燕京,迪古乃出德勝口,以代石土門為耶懶路都勃堇。天會二年,以耶懶地蒲斥鹵,遷其部於蘇濱水,仍以術實勒之田益之。



 熙宗即位,加太子太師。十四年,加保大軍節度使,同中書門下平章事,薨。天德二年,迪古乃配饗太祖廟廷。大定二年,追封金源郡王。



 習室。康宗時,高麗築九城于曷懶甸,習室從斡賽軍。太祖攻寧江州,習室推鋒力戰,授猛安。後從斜也克中京,襲遼主于鴛鴦濼,略定山囗,敗夏將李良輔兵,與婁室俱獲遼帝於余睹谷。



 宗翰伐宋,與銀術可圍守太原。明年,
 攻襄垣,下潞城,降西京,至汴。元帥府以懷、孟北阻太行,南瀕河,控制險要,使習室統十二猛安軍鎮撫之。」於是,殄平寇盜,招集流亡,四境以安。天會五年,薨。熙宗時,贈特進。大定間,謚威敏。



 世宗思太祖、太宗創業艱難,求當時群臣勳業最著者,圖像于衍慶宮:遼王斜也、金源郡王撒改、遼王宗乾、秦王宗翰、宋王宗望、梁王宗弼、金源郡王習不失、金源郡王斡魯、金源郡王希尹、金源郡王婁室、楚王宗雄、魯王闍母、金源郡王銀術可、隋國公阿離合懣、金源郡王完顏忠、豫國公蒲家奴、金源郡王撒離喝、兗國公劉彥宗、特進斡魯古、齊國公韓企先,並習
 室凡二十一人。



 初,海陵罷諸路萬戶,置蘇濱路節度使。世宗時,近臣奏請改蘇濱為耶懶節度使,不忘舊功。上曰:「蘇濱、耶懶二水相距千里,節度使治蘇濱,不必敗。石土門親管猛安於孫襲封者,可改為耶懶猛安,以示不忘其初。」



 思敬本名撒改,押懶河人,金源郡王神土懣之子,習失弟也。初名思恭,避顯宗諱改焉。體貌雄偉,美鬚髯,純直有材幹。年十一,從其父謁見太祖。太祖在納鄰澱,方獵,因詔從獵,射黃羊獲之,太祖賜以從馬。



 宗翰自太原伐宋,從其兄習室攻太原。宗翰取河南,思敬從完顏活
 女涉渡河,下洛陽、圍汴皆有功。師還,隸遼王宗乾麾下。太宗幸東京溫湯,思敬權護衛,押衛卒百人從行。領謀克。從征術虎麟有功,遂充護衛。天眷二年,以捕宗磐、宗雋功,遷顯武將軍。



 熙宗捕魚混同江,網索絕,曹國王宗敏乘醉,鞭馬入江,手引繫網大繩,沉於水中。熙宗呼左右救之,倉卒莫有應者,思敬躍入水,引宗敏出。熙宗稱嘆,賞賚甚厚。擢右衛將軍,襲押懶路萬戶,授世襲謀克。七年,召見,賜以襲衣、廄馬、錢萬貫。及歸,復遣使賜弓劍。是年,入為工部尚書,改殿前都點檢。無何,為吏部尚書。



 天德初,為報諭宋國使。宋人以舊例,請觀錢塘江潮,思
 敬不觀,曰:「我國東有巨海,而江水有大於錢塘者。」竟不往。使還,拜尚書右丞,罷為真定尹。用廉,封河內郡王,徙封鉅鹿。丁母憂,起復本官,改益都尹。正隆二年,例奪王爵,改慶陽尹。



 大定二年,授西南路招討使,封濟國公,兼天德軍節度使。俄為北路都統,佩金牌及銀牌二。西北路招討使唐括孛古底副之。將本路兵二千,會孛古底,視地形衝要,或于狗濼屯駐,伺契丹賊出沒之地,置守禦,遠斥候,賊至則戰,不以晝夜為限。詔孛古底曰:「爾兵少,思敬未至,不得先戰。」僕散忠義敗窩斡於陷泉,詔思敬選新馬三千,備追襲。窩斡入于奚中,思敬為元帥右
 都監,以舊領軍入奚地張哥宅,會大軍討之。敗偽節度特末也,獲二百餘人。賊降將稍合住與其黨神獨斡,執窩斡并其母徐輦、妻子弟姪家屬及金銀牌印詣思敬降。思敬獻俘于京師,賜金百兩、銀千兩、重綵四十端、玉帶、廄馬、名鷹。拜右副元帥,經略南邊,駐山東。罷為北京留守。復拜右副元帥,仍經略山東。



 初,猛安謀克屯田山東,各隨所受地土,散處州縣。世宗不欲猛安謀克與民戶難處,欲使相聚居之,遣戶部郎中完顏讓往元帥府議之。思敬與山東路總管徒單克寧議曰:「大軍方進伐宋,宜以家屬權寓州縣,量留軍眾以為備禦。俟邊事寧
 息,猛安謀克各使聚居,則軍民俱便。」還奏,上從之。其後遂以猛安謀克自為保聚,其田土與民田犬牙相入者,互易之。三年四月,召還京師,以為北京留守,賜金鞍、勒馬。七年,召為平章政事。先是,省併猛安謀克,及海陵時無功授猛、克者,皆罷之,失職者甚眾。思敬請量才用之,上從其請。



 思敬前為真定尹,其子取部民女為妾。至是,其兄乞離異,其妾畏思敬在相位,不敢去。詔還其家。



 九年,拜樞密使,上疏論五事:其一,女直人可依漢人以文理選試。其二,契丹人可分隸女直猛安。其三,鹽濼官可罷去。其四,與猛安同勾當副千戶官亦可罷。其五,親王
 府官屬以文資官擬注,教以女直語言文字。上皆從之。其後女直人試進士,夾谷衡、尼厖古鑑、徒單鎰、完顏匡輩,皆由此致宰相,實思敬啟之也。



 久之,上謂思敬曰:「朕欲修《熙宗實錄》,卿嘗為侍從,必能記其事跡。」對曰:「熙宗時,內外皆得人,風雨時,年穀豐,盜賊息,百姓安,此其大概也,何必餘事。」上大悅。世宗喜立事,故其微諫如此。大定十三年,薨。上輟朝,親臨喪,哭之慟。曰:「舊臣也。」賻贈加厚,葬禮悉從官給。



 孫吾侃術特,大定二十四年,除明威將軍,授速濱路寶鄰山猛安。



 贊曰:劾者讓國世祖,以開帝業。撒改治國家,定社稷,尊
 立太祖,深謀遠略,為一代宗臣,賢矣哉。習不失蓋前人之愆,著勛五世。《易》曰「有子考無咎」,其此之謂乎。始祖與季弟異部而處,子孫俱為強宗,而取遼之策,卒定於迪古乃,豈天道陰有以相之邪。



\end{pinyinscope}