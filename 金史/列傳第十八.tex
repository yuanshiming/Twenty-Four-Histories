\article{列傳第十八}

\begin{pinyinscope}

 ○熙宗二子斜卯阿里突合速烏延蒲盧渾赤盞暉大抃本名撻不野磐本名蒲速越阿離補子方



 熙宗諸子:悼平皇后生太子濟安,賢妃生魏王道濟。


濟安,皇統二年二月戊子生於天開殿。上年二十四始有皇子,喜甚,遣使馳報明德宮太皇太后。五日命名,大
 赦天下。三月甲寅,告天地宗廟。丁巳,剪
 \gezhu{
  髟}
 ,奏告天地宗廟。戊午,冊為皇太子。封皇后父太尉胡塔為王,賜人口、馬牛五百、駝五十、羊五千。隨朝職官並遷一資,皆有賜。已未,詔天下。十二月,濟安病劇,上與皇后幸佛寺焚香,流涕哀禱,曲赦五百里內罪囚。是夜,薨。謚英悼太子,葬興陵之側,上送至烏只黑水而還。命工塑其像于儲慶寺,上與皇后幸寺安置之。海陵毀上京宮室,寺亦隨毀。



 道濟,皇統三年,命為中京留守,以直學士阿懶為都提點,張玄素為同提點,左右輔導之。俄封魏王,封其母為賢妃。初居外,至是養之宮中。未幾,熙宗怒殺之。



 贊曰:國初制度未立,太宗、熙宗皆自諳班勃極烈即帝位。諳班勃極烈者,漢語云最尊官也。熙宗立濟安為皇太子,始正名位,定制度焉。



 斜卯阿里,父渾坦,穆宗時內附,數有戰功。阿里年十七,從其伯父胡麻谷討詐都,獲其弟沙里只。高麗築九城於曷懶甸,渾坦攻之,遇敵於木里門甸,力戰久之,阿里挺槍馳刺其將於陣中,敵遂潰。渾坦與石適歡合兵於徒門水,阿里首敗敵兵,取其二城。高麗入寇,以我兵屯守要害,不得進,乃還。阿里追及于曷懶水,高麗人爭走冰上,阿里乘之,殺略幾盡,遂合兵於石適歡。道遇敵兵
 五萬,擊走之。又與石適歡遇敵七萬,阿里先登,奮擊大敗之。石適歡曰:「汝一日之間,三破重敵,功豈可忘。」乃厚賜之。



 斡塞、烏睹本攻駝吉城,阿里鑿墉為門,日已暮,不可入,以兵守之,旦日遂取其城。烏睹本以被甲并乘馬賜之。從攻下寧江州,授猛安。又從攻信州、賓州,皆克之。遼人來攻孛堇忽沙里城,阿里率百餘騎救之。遼兵數萬,阿里兵少,乃令軍士裂衣多為旗幟,出山谷間,遼兵望見,遁去。



 蘇、復州叛,眾至十萬。旁近女直皆保於太尉胡沙家,築壘為固。敵圍之數重,守者糧芻俱盡,牛馬相食其鬃尾,人易子而食。夜,縋二人出,告急於阿里。阿里
 赴之,內外合擊之,破其眾於闢離密罕水上,剿殺幾盡,水為之不流。蒲離古胡什吉水、馬韓島凡十餘戰,破數十萬眾。契丹、奚人聚舟千艘,將入于海。阿里以二十七舟邀之,中流矢,臥舟中,中夜始蘇。敵船已入王家島,即夜取海路追及之,敵走險以拒,阿里以騎兵邀擊,再中流矢,力戰不退,竟破之,盡獲其舟。於是,蘇、復州、婆速路皆平。



 攻顯州,下靈山縣,取梁魚務,敗余睹兵,功皆最。後與散睹魯屯高州,契丹昭古牙、九斤合興中兵數萬攻胡里特寨,阿里以八謀克兵救之。胡里特先往,敗於城下。阿里指陣前緋衣者二十餘人曰:「此必賊酋也。」麾兵
 奮擊,皆殺之,餘眾大潰。來州、隰州兵圍胡里特城,聞阿里來救,即解圍去。



 闍母討張覺,有兵出樓峰口山谷間,阿里、散篤魯、忽盧補三猛安擊敗之。宗望代闍母討張覺,阿里再敗平州兵。及伐宋,阿里別擊宋兵,敗之。孟陽之役,阿里扼橋渡力戰。明年,再伐宋,至保州、中山,累破之。進圍真定,阿里與婁室、豁魯乘風縱火,焚其樓櫓,諸軍畢登,克其城。師至河上,粘割胡撒擊走宋人,扼河津,兵數千遂渡河。諸將分出大名境,阿里破敵四百盡殪,遂圍汴。汴中夜出兵來焚攻具,阿里與謀克常孫陽阿禦之,其眾大潰。還攻趙州,降之。



 天會六年,伐宋主,取陽穀、莘
 縣,敗海州兵八萬人,海州降。破賊船萬餘於梁山泊。招降滕陽、東平、泰山群盜。盜攻范縣,擊走之,獲船七百艘。宗弼攻下睢陽,與烏延蒲盧渾先以二千人往招壽春,具舟淮水上。時康民聚賈船四百與壽春相近,術列速以騎四百破康民,斬馘數千。與當海、大抃破賊十萬於淮南。比至江,連破宋兵,獲舟二百艘。宗弼至江寧,阿里、蒲盧渾別降廣德軍,先趣杭州。去杭十餘里,遇宋伏兵二千,取我前驅甲士三十人。阿里使諸軍去馬搏戰,伏兵敗,皆逼死於水。宗弼至餘杭,而宋主走明州,阿里與蒲盧渾以精騎四千襲之,破東關兵,濟曹娥江,敗宋兵
 於高橋鎮。至明州,頗失利。宋主已入于海,乃退軍餘姚。宗弼使當海濟師,遂下明州,執宋守臣趙伯諤,進至昌國縣。宋主自昌國走溫州,由海路追三百餘里,弗及。遂隳明州,與宗弼俱北歸。



 睿宗經略陜西,駐涇州,阿里先取渭州。睿宗趨熙河,阿里、斜喝、韓常三猛安為前軍。十二年,與高彪監護水運。宋以舟師阻亳州河路,擊敗之,追殺六十餘里,獲其將蕭通。破漣水水寨賊,盡得其大船,遂取漣水軍,招徠安輯之。天眷間,盜據石州,阿里討之。粘割胡撒與所部先登,遂克其城,石州平。



 宗弼再伐宋,阿里已老,督造戰船。宋稱臣,詔賜阿里錢千萬。自結
 髮從軍,大小數十戰,尤習舟楫,江、淮用兵,無役不從,時人以水星目之。為迭里部節度使,歷順義、泰寧軍,歸德、濟南尹。天德初,致仕,加特進,封王。正隆例封韓國公,召赴闕,命造戰船。以疾薨,年七十八,謚智敏。



 阿里性忠直,多智略。兄弟相友愛,家故饒財,以己猛安及財物盡與弟愛拔里。愛拔里不肯受,逃避歲餘,阿里終與之。



 突合速,宗室子,拿罕塞人。初隸萬戶石家奴麾下,嘗領偏師破雲中諸山寇盜。宗望攻平州,遣突合速討應州賊,平之,撫安其民而還。及伐宋,在宗翰軍,以八謀克破石嶺關屯兵數萬,殺戮幾盡。師至太原,祁縣降而復叛,
 突合速攻下之。進取文水縣,後從諸帥列屯汾州之境。宋河東軍帥郝仲連、張思正,陜西軍帥張關索及其統制馬忠,合兵數萬來援,皆敗之。



 宗翰南伐至潞還,太原猶未下,即留完顏銀術可總督諸軍,經略其地。於是,宋援兵大至,突合速從馬五、沃魯破宋兵四千於文水。聞宋將黃迪等以兵三十萬柵于縣之西山,復與耿守忠合兵九千擊之,殺八萬餘人,獲馬及資糧甚眾。宋制置使姚古率兵至隆州谷,突合速與拔離速以步騎萬餘禦之。種師中兵十萬據榆次,銀術可乃召突合速,使中分其兵而還,與活女等合兵八千擊敗之,斬師中於殺
 熊嶺。宋將張灝以兵十萬營于文水近郊,復與拔離速擊破之。潞州復叛,宋兵號十七萬,骨赧、突合速、拔離速皆被圍。突合速麾軍士,下馬力戰,遂潰圍而出。



 及再舉伐宋,宗翰命婁室率軍先趨汴。婁室至澤州,突合速、沃魯以五百騎為前驅,往招河陽。先據黃河津,宋兵萬餘背水陣,進擊敗之,皆擠于水,遂降河陽。汴京平,諸將西趣陜津,略定河東郡縣。突合速取憲州,遇其援軍,擊敗之,生擒其將。孛堇濃瑰術魯等攻保德,未下,突合速進兵助擊,梯衝並進,遂克其城。孛堇烏谷攻石州,屢敗,亡其三將,軍士歿者數百人。突合速謂烏谷曰:「敵皆步兵,
 吾不可以騎戰。」烏谷曰:「聞賊挾妖術,畫馬以系其足,疾甚奔馬,步戰豈可及之。」突合速笑曰:「豈有是耶?」乃令諸軍去馬戰,盡殪之。六年,宗輔駐師鄧州,突合速、馬五、拔離速西取均、房,遂下其城。攻唐、蔡、陳州及潁昌府,皆克之。



 天眷初,除彰德軍節度使。三年,為元帥左監軍。皇統八年,改濟南尹。天德間,封定國公,授世襲千戶。卒,年七十二。正隆二年,贈應國公。



 初,突合速以次室受封,次室子因得襲其猛安。及分財異居,次室子取奴婢千二百口,正室子得八百口。久之,正室子爭襲,連年不決,家貲費且盡,正室子奴婢存者二百口,次室子奴婢存者纔
 五六十口。世宗聞突合速諸子貧窘,以問近臣,具以爭襲之故為對,世宗曰:「次室子豈當受封邪?」遂以嫡妻長子襲。



 烏延蒲盧渾,曷懶路烏古敵昏山人。父孛古剌,龍虎衛上將軍。蒲盧渾膂力絕人,能挽強射二百七十步。與兄鶻沙虎俱以勇健隸闍母軍,居帳下。攻黃龍府,力戰有功。闍母敗于兔耳山,張覺復整兵來,諸將皆不敢戰。蒲盧渾登山望之,乃紿諸將曰:「敵軍少,急擊可破也。若入城,不可復制。」遂合戰,破之。



 郭藥師、蔡靖以燕京降,蒲盧渾率九十騎先伺察城中居民去就。遂將漢兵千,隸元
 顏蒙適攻真定。進攻贊皇,取之,獲人畜甲仗萬餘。汴城破,日已暮,宋人猶力戰,槍刺中蒲盧渾手,戰益力,遂敗宋軍,賜金五十兩。



 睿宗為右副元帥,已定關、陜,議取劍外諸州,遂拔和尚原。元帥府承制以蒲盧渾為河北西路兵馬都總管。及宋主在揚州,蒲盧渾與蒙適將萬騎襲之,宋主已渡江,破其餘兵。後與斜卯阿里俱從宗弼自淮西渡江取江寧。宗弼入杭州,宋主走明州,再走溫州,由海道追三百餘里,隳明州而歸,語在《阿里傳》。



 天眷二年,授鎮國上將軍,除安國軍,以疾去官。皇統六年,授世襲謀克,起為延安尹,賜尚衣一襲,尋致仕。海陵遷中
 都,起為歸德尹,就其家授之,賜銀牌、襲衣、玉吐鶻,馳驛之官。蒲盧渾留數十日,已違程,復聽致仕。召赴京師,至薊州,見海陵于獵所。明日,從獵,獲一狐。海陵曰:「卿年老,尚能馳逐擊獸,健捷如此。」賜以御服,封豳國公。除太子少師,進太子太保,改真定尹,入判大宗正事。



 頃之伐宋,以本官行右領軍副都督事。師次西采石,海陵欲渡江,蒲盧渾曰:「宋軍船高大,我船庳小,恐不可遽渡。」海陵怒曰:「汝昔從梁王追趙構於海島,皆大舟耶?今乃沮吾兵事!設不能遽渡江,不過有少損耳。爾年已七十,縱自愛,豈有不死理耶。明日當與奔睹先濟。」既而復止之,乃遣
 別將先渡江,舟小不可戰,遂失利,兩猛安及兵士二百餘人皆陷沒。海陵遇害,軍還。



 大定二年,至中都上謁,除東京留守。世宗召問年幾何,對曰:「臣今年七十三矣。」上曰:「卿宿將,久練兵事,年雖老,精神不衰。」因命到官,每旬月一視事。賜衣一襲,進階開府儀同三司,仍封豳國公。是歲,卒。十八年,孫扎虎遷廣威將軍,襲烏古敵昏山世襲猛安,并親管謀克。



 赤盞暉,字仲明,其先附於遼,居張皇堡,故嘗以張為氏。後家來州。暉體貌雄偉,慷慨有志略。少遊鄉校。遼季以破賊功,授禮賓副使,領來、隰、遷、潤四州屯兵。天輔六年
 降,仍命領其眾,從闍母定興中府義、錦等州。及破張覺,皆與有功,以粟萬五千石助軍,授洺州刺史。



 宗望初伐宋,孟陽之戰,敵之中軍徑薄宗望營,暉與諸將擊敗之,追殺至城下。訖師還,數立戰功。明年,再舉伐宋,攻下保州、真定,暉皆與焉。進圍汴,宋人夜出兵二萬焚我攻具,暉以二謀克兵擊走之。凡城中出兵拒戰,暉之所當,無不勝捷。



 既克宋還,從攻河間。敵將李成以雄、莫之兵來援,暉與所部迎擊,馬傷而墮,暉輒奮起步斗,竟敗成兵。是日,凡七戰皆勝,敵人多逼死濠隍間,暉兩臂亦數中流矢。賊將劉先生以兵二萬夜襲營,暉力戰達旦,賊始
 敗走,皆溺死于水。暉復傅城力戰,如是連月,諸軍四面合攻,遂克之。加桂州管內觀察使,因留撫河間。時居民皆為軍士所掠,老幼存者亡幾。暉下令軍中聽贖還之。未幾,皆按堵如故。



 從睿宗經略山東,既攻下青州,復從闍母攻濰州。暉督其裨校先登,而城中積芻茭乘風縱火發機石,暉率將士衝冒而下,力戰敗之。軍還,復以三十騎破敵于范橋。帥府承制加靜江軍節度使。進攻,城中砲出,幾中暉,拂其甲裳裂之。暉益奮攻,卒破其城。又從攻泗州,克之。還屯汶陽,破賊眾于梁山濼,獲舟千餘。移軍攻濟州,既敗敵兵,因傅城諭以禍福,乃舉城降。暉
 約束軍士,無秋毫犯,自是曹、單等州皆聞風而下。



 從攻壽春、歸德,及渡淮為先鋒,遇重敵于秀州、蘇州,皆擊敗之,遂至餘杭。通糧餉,治橋道,暉之力為多。乃還,載《資治通鑒》版以歸。大軍過江寧,徙其官民北渡,時暑多疾疫,老弱轉死道路,其知府陳邦光者訴于宗弼,怒將殺之,暉曰:「此義士也。」力營救之,竟得免。



 富平之戰,暉在右翼,遇濘而敗,睿宗念其前功,杖而釋之。師至熙河,暉別降諸寨將鈐轄及吐蕃酋長等,并民戶萬五千餘。蘭州叛,與訛魯補等攻下之,獲河州安撫使白常、熙河路副都總管劉維輔以獻。還攻慶陽,兩敗重敵,殺其將戴巢。師
 還,遷歸德軍節度使。



 宋州舊無學,暉為營建學舍,勸督生徒,肄業者復其身,人勸趨之。屬縣民家奴王夔者,嘗業進士,暉以錢五十萬贖之,使卒其業,夔後至顯官。密州吏龐乙卒於官,其孤貧,不克葬,暉為營治葬事,且資給其家。



 十三年,復從大軍渡淮。還鎮,丁母憂,尋以舊職起復。既廢齊,為安化軍節度使。天眷三年,復河南,宋人乘間陷海州,帥府以登、萊、沂、密四州委暉經畫,敵無敢窺其境者。為定海軍節度使,尋改濟南尹,累遷光祿大夫。俄以罪罷。久之,起為昌武軍節度使。天德二年,遷南京留守,尋改河南路統軍使,授世襲猛安,拜尚書右丞,
 封河內郡王。歲餘,拜平章政事,封戴王。正隆初,出為興平軍節度使。正隆降王爵,為樞密副使,封景國公。未幾,復為左丞,封濟國公。尋除大興尹,封榮國公。薨,年六十五。大定間謚曰武康。子師直,登進士第。



 大抃,本名撻不野,其先遼陽人,世仕遼有顯者。太祖伐遼,遼人徵兵遼陽,時抃年二十餘,在選中。遼兵敗,抃脫身走寧江。寧江破,抃越城而逃,為軍士所獲,太祖問其家世,因收養之。收國二年,為東京奚民謀克。是時,初破高永昌,東京旁郡邑未盡服屬,使抃伺察反側。有聞必達,太祖以為忠實,授猛安,兼同知東京留守事。



 取中、西
 兩京,隸闍母軍。遼軍二十萬來戰,吳王使抃以本部守營,抃堅請出戰,不許。或謂抃曰:「戰,危事,獨苦請,何也?」抃曰:「丈夫不得一決勝負,尚何為。茍臨戰不捷,雖死猶生也。」吳王聞而壯之,乃遣出戰。既合戰,闍母軍少卻,遼兵後躡之,抃麾本部兵橫擊,殺數百人,由是顯名軍中。



 天會三年,宗望伐宋,信德府居燕、汴之中,可駐軍以濟緩急,欲遂攻之,恐不能亟下,議未決。抃獨率本部兵,選善射者射其城樓,別以輕銳潛升於樓角之間,遂克其城。明年,軍至濬州,宋人已燒河橋,宗望下令,「軍中有能先濟者功為上」。抃捕得十餘舟,使勇悍者徑渡,擊其守者而奪
 其戍柵,由是大軍俱濟。



 八月,再伐宋,授萬戶,賜金牌。既破汴京,抃為河間路都統。已克河間,闍母怒其不早降,因縱軍大掠,抃諫止之,已掠者官為贖還。除河間尹,從攻襲慶府。先一日,抃命軍士預備畚鍤及薪,既傅城,諸將方經營攻具,未鳴鼓,抃軍有素備,遂先登。軍帥以抃未鳴鼓輒戰,不如軍令,請罪抃,朝廷釋弗問,仍例賞之。



 宗弼伐江南,濟淮,宋將時康民率兵十七萬來拒,抃率本部從擊,敗之。復以騎二千與當海擊敗淮南賊十萬,殺萬餘人,王善來降。將渡江,抃軍先渡,舟行去岸尚遠,宋列兵江口,抃視其水可涉,則麾兵捨舟趨岸疾擊之,
 宋兵走,大軍相繼而濟。俄遇杜充兵六萬於江寧之西,抃與鶻盧補擊走之。師還,抃留為揚州都統,經略淮、海、高郵之間。再為河間尹,兼總河北東路兵馬。



 十一年,入見,太宗賜坐,慰勞甚久,特遷太子太保,賜衣一襲、馬二匹及鞍轡鎧甲,改元帥右都監。齊國廢,抃守汴京。熙宗念抃久勞,降御書寵異之。天眷三年,罷漢、渤海千戶謀克,以抃舊臣,獨命依舊世襲千戶。是歲,拜元帥右監軍。



 宗弼再伐宋,宋人稱臣乞和,遂班師,抃獨留汴,行元帥府事。皇統三年,加開府儀同三司。八年,進左監軍。天德二年,改右副元帥,兼行臺左丞。遷平章行臺省事,進行
 臺右丞相,右副元帥如故。海陵疑左副元帥撒離喝,以為行臺左丞相,使抃伺察之,詔軍事不令撒離喝與聞。撒離喝不知海陵意旨,每與抃爭軍事不能得,遂與抃有隙。海陵竟殺撒離喝,召抃入朝,拜尚書右丞相,封神麓郡王。



 四年,請老,為東京留守。貞元三年,拜太傅,領三省事,累封漢國王。十二月,有疾,海陵幸其第問之。是歲,薨,年六十八。海陵親臨哭之,詔有司廢務三日,禁樂三日。其三日當賜三國使館燕,以不賜教坊樂,命左宣徽使敬嗣暉宣諭之。贈太師、晉國王,謚傑忠,遣使護喪歸葬。正隆奪王爵,贈太傅、梁國公。子磐。



 磐,本名蒲速越,以大臣子累官登州刺史,襲猛安。大定三年,除嵩州刺史,從僕散忠義伐宋有功。五年,召為符寶郎,遷拱衛直都指揮使。



 初,磐以伐宋功,進官一階,磐心少之,頗形于言。上聞之,下吏按問,杖一百五十,改左衛將軍。詔求良弓,磐多自取,及護衛入直者,輒以己意更代。護衛婁室告其事,詔點檢司詰問。磐有妹在宮中為寶林,磐屬內侍僧兒員思忠使言於寶林曰:「我無罪,問事者迫我,使自誣服。」寶林訴于上,上怒,杖僧兒一百,磐責隴州防禦使。上戒之曰:「汝在近密,執迷自用,朕以卿父之功,不忍廢棄,姑令補外,其思勉之。」改亳州防禦
 使,遷武寧軍節度使,坐事除名。起為韓州刺史。改祁州刺史,復坐事,削四官,解職。



 久之,尚書省奏「大磐以年當敘」,上曰:「剛暴之人,屢冒刑章,不可復用。太傅大抃,別無嫡嗣,其世襲猛安謀克,不可易也。」



 阿離補,宗室子,系出景祖。屢從征伐,滅遼舉宋皆有功。天會九年,睿宗經略陜西,阿離補為左翼都統,與右翼都統宗弼撫定鞏、洮、河、西寧、蘭、廓等州軍,來賓、定遠、和政、甘峪、寧洮、安隴等城寨,及鎮、堡、蕃、漢營部四十餘處,漢官軍民蕃部酋長甚眾,於是涇原、熙河兩路皆平。詔以兄猛安沙離質親管謀克之餘戶,以阿離補為世襲
 謀克。天會十二年,為元帥右都監。十五年,遷左監軍。天眷三年,從宗弼復河南,遷左副元帥。皇統三年,封譚國公。六年,為行臺左丞相,元帥如故。是歲,薨。



 大定間,大褒功臣,圖像衍慶宮。歡都死康宗時,不及與馳騖遼、宋之郊,然而異姓之臣莫先焉。故定衍慶亞次功臣:代國公歡都,金源郡王石土門,徐國公渾黜,鄭國公謾都訶,濮國公石古乃,濟國公蒲查,韓國公斜卯阿里,元帥左監軍拔離速,魯國公蒲察石家奴,銀青光祿大夫蒙適,隨國公活女,特進突合速,齊國公婆盧火,開府儀同三司烏延蒲盧渾,儀同三司阿魯補,鎮國上將軍烏林答泰
 欲,太師領三省事勖,太傅大抃,大興尹赤盞暉,金吾衛上將軍耶律馬五,驃騎衛上將軍韓常並阿離補咸著勛焉。子言、方,言別有傳。



 方以宗室子累官京兆少尹,遷陜西路統軍都監。方專事財賄,不恤軍旅,詔戒之曰:「卿宗室舊人,乃縱肆敗法,惟利是營,朕甚惡之。自今至於後日,萬一為之,必罰無赦。」大定三年,遷元帥右都監,轉元帥左監軍,改順天軍節度使,上曰:「卿本無功,歷顯仕,不能接僚友,往往交惡,在京兆貪鄙彰聞,至無謂也。朕念卿已過中年,必能悛改,慎勿復爾。」除西南路招討使,朝廷以兵部郎中高通
 為招討都監,以佐之。詔通曰:「卿到天德,毋以其官長曲從之也。簡閱沿邊士卒,毋用孱弱之人,毋以僕隸代役。女直舊風,凡酒食會聚,以騎射為樂。今則弈棋雙陸,宜悉禁止,令習騎射。從其居處之便,亦不可召集擾之。」久之,方坐強買部人馬二匹,削一階,解職,降耀州刺史。通亦坐贓除名。方後遷橫海軍節度使,人為同簽大宗正事,簽書樞密院事。



 初,阿魯補當授謀克,未封而薨,烏帶受之。烏帶死,兀答補襲之。兀答補死,烏也阿補當襲。是時,已降海陵為庶人,世宗以烏帶在熙宗逆黨中,其子孫不合受封,停封者久之,而阿離補功亦不可廢絕,特詔
 方襲之云。



 贊曰:斜卯阿里、突合速、烏延蒲盧渾、赤盞暉、大抃、阿離補等六人,皆收國以來所謂熊羆之士、不二心之臣也,其功有可錄者焉。



\end{pinyinscope}