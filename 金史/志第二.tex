\article{志第二}

\begin{pinyinscope}

 歷上



 昔者聖人因天道以授人時,釐百工以熙庶政,步推之法,其來尚矣。自漢太初迄于前宋,治歷者奚啻七十餘家,大概或百年或數十年,率一易焉。蓋日月五星盈縮進退,與夫天運,至不齊也,人方製器以求之,以俾其齊,積寡至多,不能無爽故爾。



 金有天下百餘年,歷惟一易。天會五年,司天楊級紹造《大明歷》,十五年春正月朔,始
 頒行之。其法,以三億八千三百七十六萬八千六百五十七為歷元,五千二百三十為日法。然其所本,不能詳究,或曰因宋《紀元歷》而增損之也。正隆戊寅三月辛酉朔,司天言日當食,而不食。大定癸巳五月壬辰朔,日食,甲午十一月甲申朔,日食,加時皆先天。丁酉九月丁酉朔,食乃後天。由是占候漸差,乃命司天監趙知微重修《大明歷》,十一年歷成。時翰林應奉耶律履亦造《乙未歷》。二十一年十一月望,太陰虧食,遂命尚書省委禮部員外郎任忠傑與司天歷官驗所食時刻分秒,比校知微、履及見行歷之親疏,以知微歷為親,遂用之。明昌初,司
 天又改進新歷,禮部郎中張行簡言:「請俟他日月食,覆校無差,然後用之。」事遂寢。是以終金之世,惟用知微歷,我朝初亦用之,後始改《授時歷》焉。今其書存乎太史,採而錄之,以為《歷志》。



 ○步氣朔第一



 演紀:上元甲子距今大定庚子,八千八百六十三萬九千六百五十六年。



 日法:五千二百三十分。



 歲實:一百九十一萬二百二十四分。



 通餘:二萬七千四百二十四分。



 朔實:一十五萬四千四百四十五分。



 通閏:五萬六千八百八十四分。



 歲策:三百六十五日,餘一千二百七十四分。



 朔策:二十九日,餘二千七百七十五分。



 氣策:一十五日,餘一千一百四十二分,六十秒。



 望策:一十四日,餘四千二分,四十五秒。



 象策:七日,餘二千一分,二十二秒半。



 沒限:四千八十七分,三十秒。



 朔虛分:二千四百五十五分。



 旬周:三十一萬三千八百分。



 紀法:六十。



 秒母:九十。



 求天正冬至



 置上元甲子以來積年,歲實乘之,為通積分。滿旬周去之,不盡以日法約之為日,不盈為餘,命甲子算外,即所求天正冬至日大小餘。



 求次氣



 置天正冬至大小餘,以氣策累加之,秒盈秒母從分,分滿日法從日,即得次氣日及餘秒。



 求天正經朔



 以朔實去通積分,不盡為閏餘,以減通積分為朔積分。滿旬周去之,不盡如日法而一為日,不盈為餘,即所求天正經大小餘也。



 求弦望及次朔



 置天正經朔大小餘,以象策累加之,即各得弦、望及次朔經日及餘秒也。



 求沒日



 置有沒之恆氣小餘,如沒限以上,為有沒之氣。以秒母乘之,內其秒,用減四十七萬七千五百五十六,餘滿六千八百五十六而一,所得併恆氣大餘,命為沒日。



 求滅日



 置有滅之朔小餘,經朔小餘不滿朔虛分者。六因之,如四百九十一而一,所得併經朔大餘,命為滅日。



 ○步卦候第二



 候策:五,餘三百八十,秒八十。



 卦策:六,餘四百五十七,秒六。



 貞策:三,餘二百二十八,秒四十六。



 秒母:九十。



 辰法:二千六百一十五。



 半辰法:一千三百七半。



 刻法:三百一十三,秒八十。



 辰刻:八,十百四分,秒六十。



 半辰刻:四,五十二分,秒三十。



 秒母:一百。



 求七十二候



 置中氣大小餘,命之為初候,以候策累加之,即次候及末候也。



 求六十四卦



 置中氣大小餘,命之為公卦;以卦策累加之,得辟卦;又加之,得侯內卦。以貞策加之,得節氣之初,為侯外卦;又
 以貞策加之,得大夫卦。又以卦策加之,為卿卦。



 求土王用事



 以貞策減四季中氣大小餘,即土王用事日也。



 求發斂



 置小餘,以六因之,如辰法而一為辰。如不盡,以刻法除之為刻。命子下算外,即得加時所在辰刻及分。如加半辰法,即命子刻初。



 表略



 ○步日躔第三



 周天分:
 一
 百九十一萬二百九十三分,五百三十秒。



 歲差:六十九,五百三十秒。秒母一萬。



 周天諶:三百六十五度,二十五分,六十八秒。



 象限:九十一,三十一分,九秒。



 二十四氣日積度及盈縮



 求每日盈縮朓朒



 各置其氣損益率,求盈縮之損益,求朒磕用朒磕之損益。六因,如象限而一,為氣中率。與後氣中率相減,為合差。半合差加減其氣中率,為初末泛率。至後:加初,減末。分後:減初,加末。又置合差,六因,如象限而一,為日,半之,加減初未泛率,為初末定率。至後:減初,加末。分後:加初,減末。以日差累加減其氣初末定率,為每日損益分。至後減,分後加。各以每日損益分加減氣下盈縮、朒磕,為每日盈縮、朒磕二分前一氣元後率相減為合差者,皆用前氣合差。



 求經朔弦望入氣



 置天正閏餘,以日法除為日,不滿為餘,如氣策以下,以減氣策,為入大雪氣。以上去之,餘亦減氣策,為入
 小雪氣。即得天正經朔入氣日及餘也。以象策累加之,滿氣策去之,即得弦、望入次氣日及餘。因加,後朔入氣日及餘也。



 求每日損益、盈縮、朒磕



 以日差益加減損加減其氣初損益率,為每日損益率。馴積損益其氣盈縮、肉磕積,
 為每日盈、朒磕積。



 求經朔弦望入氣朒磕定數



 各以所入恆氣小餘,以乘其日損益率,如日法而一,以所損益其下朒磕積為定數。



 赤道宿度



 斗二十五度牛七度少女十一度少虛九度少秒六十八危十五度半室十七度壁八度太



 右北方七宿九十四度秒六十八



 奎十六度半婁十二
 度胃十五度昴十一度少畢十七度少觜半度參十度半



 右西方七宿八十三度井三十三度少鬼二度半柳十三度太星六度太張十七度少翼十八度太軫十七度



 右南方七宿一百九度少角十二度亢九度少氐一十六
 度房五度太心六度少尾十九度少箕十度半



 右東方七宿七十九度



 求冬至赤道
 日度



 置通積分,以周天分去之,餘日法而一為度,不滿退除為分秒。以百為母。命起赤道虛宿七度外去之,至不滿宿,即所求年天正冬至加時日躔赤道度及分秒。



 求春分夏至秋分赤道日度



 置天正冬至加時赤道日度,累加象限,滿赤道宿次
 去之,即各得春分、夏至、秋分加時日在宿度及分秒。



 求四正赤道宿積度



 置四正赤道宿全度,以四正赤道日度及分減之,餘為距後度。以赤道宿度累加之,各得四正後赤道宿積度及分。



 求赤道宿積度入初末限



 視四正後赤道宿積度及分,在四十五度六十五分秒五十四半以下為入初限,以上者用減象限,餘為入末限。



 求二十宿黃道度



 以四正後赤道宿入初末限度及分,減一百一度,餘以初末限度及分乘之,進位,滿百為分,分滿百為度。至後以減、分後以加赤道宿積度,為其宿黃道積度。以前宿黃道積度減之其四正之宿,先加象限,然後前宿減之。為其宿
 黃道度及分。其分就約為太、半、少。



 黃道宿度



 斗二十三度牛七度女十一度虛九度少秒六十八危十六度室十八度少壁九度半



 右北方七宿九十四度六十八秒



 奎十七度太婁十二度太胃十五度半昴十一度畢十六度半觜半度參九度太



 右西方七宿八十三度太一百七十七、七十五、六十八



 井三十度半鬼二度半柳十三度少星六度太張十七度太翼二十度軫十八度半



 右南方七宿一百九度少
 二百八十七、六十八



 角十二度太亢九度太氐十六度少房五度太心六度尾十八度少箕九度半



 右東方七宿七十八度少三百六十五、二十五、六十八



 前黃道宿度,依今歷歲差所在算定。如上考往古,下驗將來,當據歲差,每移一度,依術推變當時宿度,然後可步七曜,知其所在。



 求天正冬至加時黃道日度



 以冬至加時赤道日度及分秒,減一百一度,餘以冬至赤度及分秒乘之,進位,滿百為分,分滿百為度。命曰黃赤道差。用減冬至加時赤道日度及分秒,即所求年天正冬至加時黃道日度及分秒。



 求二十四氣加時黃道日度



 置所求年冬至日躔黃赤道差,以次年
 黃赤道差減之,餘以所求氣數乘之,二十四而一,所得以加其氣中積及約分,又以其氣初日盈縮數盈加縮減之,用加冬至加時黃道日度,依宿次去之,即各得其氣加時黃道日躔宿度及分秒。如其年冬至加時赤道宿度空分秒在歲差以下者,即加前宿全度,然後求黃赤道差,餘依術算。



 求二十四氣每日晨前夜半黃道日度



 副置其氣小餘,以
 其氣初日損益率乘之,盈縮之損益。萬約之為分,應益者盈加縮減,應損者盈減縮加其副,日法除之為度,不滿退除為分秒,以減其氣加時黃道日度,即各得其氣初日晨前夜半黃道日度。每日加一度,以百約每日損益率,盈縮之損益。應益者盈加縮減,應損者盈減縮加,為每日晨前夜半黃道日度及分秒。



 求每日午中黃道日度



 置一萬分,以所入氣日盈縮損益率,應益者盈加縮減,應損者盈減縮加,皆加減損益率,餘半之,滿百為分,不滿為秒,以加其日清晨前夜半黃道日度,即其日午中日躔黃道宿度及分秒。



 求每日午中黃道積度



 以二至加時黃道日度,距至所求日午中黃道日度,為入二至後黃道積度及分秒。



 每日午中黃道入初末限



 視二至後黃道積度,在四十三度一十
 二分秒八十七以下為初限,以上用減象限,餘為入末限。其積度滿象限去之,為二分後黃道積度,在四十八度一十八分秒二十二以下為初限,以上用減象限,餘為入末限。



 求每日午中赤道日度



 以所
 求日午中黃道積度,入至後初限,分後末限,度及分秒,進三位,加二十萬二千五十少,開平方除之,所得減去四百四十九半,餘在初限者,直以二至赤道日度加而命之。在末限者,以減象限,餘以二分赤道日度加而命之。即每日午中赤道日度。以所求日午中黃道積度,入至後末限,分後初限,度及分秒,進三位,用減三十萬三千五十少,開平方除之,所得,以減五百五十半,其在初限者,以所減之餘,直以二分赤道日度加而命之。在末限
 者,以減象限,餘以二至赤道日度加而命之。即每日午中赤道日度。



 太陽黃道十二次入宮宿度,雨水,危十三度三十九分五十秒外,入衛分,陬訾之次,辰在亥。春分,奎二度三十五分八十五秒外,入魯分,降婁之次,辰在戌。穀雨,胃四度
 二十四分三十三秒外,入趙分,大梁之次,辰在酉。小滿,畢七度九十六分六秒外,入晉分,實沈之次,辰在申。夏至,井九度四十七分一十秒外,入秦分,鶉首之次,辰在未。大暑,柳四度九十五分一十六秒外,入周分,鶉火之次,辰在午。處暑,張十五度五十六分三十五秒外,入楚分,鶉尾之次,辰在巳。秋分,軫十度四十四分五秒外,入鄭分,壽星之次,辰在辰。霜降,氐一度七十七分七十七秒外,入宋分,大火
 之
 次,辰在卯。小雪,尾三度九十七分九十二秒外,入
 燕分,析木之次,辰在寅。冬至,斗四度三十六分六十六秒外,入吳越分,星紀之次,辰在丑。大寒,女二度九十一分九十一秒外,入齊分,玄枵之次,辰在子。



 求入宮時刻



 各置入宮宿度及分秒,以其日晨前夜半日度減之,相近一度之間者求之。餘以日法乘其分,其秒從於下,亦通乘之,為實;以其日太陽行分為法,實如法而一,所得,依發斂加
 時求之,即得其日太陽入宮時刻及分秒。



 ○步晷漏第四



 中限:一百八十二日,六十二分,一十八秒。



 冬至初限,夏至末限:六十二日,二十分。



 夏至初限,冬至末限:一百二十日,四十二分。



 冬至地中晷影常數:一丈二尺八寸三分。



 夏至地中晷影常數:一尺五寸六分。



 周法:一千四百二十八。



 內外法:一萬八百九十六。



 半法:二千六百一十五。



 日法:四分之三:三千九百二十二半。



 日法:四分之一:一千三百七半。



 昏明分:一百三十分,七十五秒。



 昏明刻:二刻,一百五十六分,九十秒。



 刻法:三百一十三分,八十秒。



 秒母:一百。



 求午中入氣中積



 置所求日大餘及半法,以所入之氣大小餘減之,為其日午中入氣。以加其氣中積,為其日午中中積。小餘以日法除為約分。



 求二至後午中入初末限



 置午中中積及分,如中限以下,為冬至後。以上去中限,為夏至後。其二至後,如在初限以下,為初限。以上覆減中限,餘為入末限也。



 求午中晷影定數



 視冬至後初限、夏至後末限,百通日,內分,自相乘,副置之。以一千四百五十除之,所得加五萬三百八十,折半限分併之;除其副為分。分滿十為寸,寸滿十為尺,用減冬至地中晷影常數,為所求晷影定數。視夏至後初限、冬至後末限,百通日,內分,自相乘為上位。下置入限分,
 以二百二十五乘,百約之,加一十九萬八千七十五為法。夏至前後半限以上者,減去半限,列於上位。下位置半限。各百通日,內分,先相減,後相乘。以七千七百除之,所得以加其法反除上位,為分,分滿十為寸,寸滿十為尺,用加夏至地中晷影常數,為所求晷影定數。



 求四方所在晷影



 各於其處測冬夏二至晷影,乃相減之餘,為其處二至晷差。亦以地中二至晷數相減,為地中二至晷差。其所求日在冬至後初限、夏至後末限者,如在半限以下,倍之;半限以上,覆減半限,餘亦倍之,併入限日,三因折半,以日為分,十為寸,以減地中二至晷差為法。置地中冬
 至晷影常數,以所求日地中晷影定數減之,餘以其年二至晷差乘之為實。實如法而一,所得,以減其處冬至晷數,即得其處其日晷影定數。所求日在夏至後初限、冬至後末限者,如在半限以下,倍之;半限以上,覆減半限,餘亦倍之,並入限日,三因四除,以日為分,十為寸,以加地中二至晷差為法。置所求日地中晷影定數,以地中夏至晷影常數減之,餘以其處二至晷差乘之為實。實如法而一,所得,以加其處夏至晷數,即得其處其日晷影定數。



 二十四氣陟降及日出分



 表略



 二分
 前後陟降率



 春分前三日太陽入赤道內,秋分後三日太陽出赤道外,故其陟降與他日不倫,今各別立數而用之。



 驚蟄,十二日,陟四六十七,一十六此為末率,於此用畢。其減差亦止於此。十三日,陟四四十一,六。十四日,陟四三十六,九十。十五日,
 陟四一。



 秋分,初日,降四三十八。一日,降四三十九。二日,降四五十七。三日降四六十八。此為初率,始用之。其加差亦始於此。



 求每日出入晨昏半晝分



 各以陟降初率,陟減降加其氣初日日出分,為一日下日出分。以增損差,仍加減加減差。增損陟降率,馴積而加減之,即為每日日出分。覆減日法,餘為日入分。以日出分日入分而半之,為半晝分。以昏明分減日出分為晨分,加日入分為昏分。



 求日出入辰刻



 置日出入分,以六因之,滿辰法而一,為辰數,不盡,刻法除之為刻數,不滿為分,命子正算外,即得所求。



 求晝夜刻



 置日出分,十二乘之,刻法而一,為刻,不滿為分,即為夜刻。覆減百刻,餘為晝刻。



 求更點率



 置晨分,四因,退位為更率。二因更率,退位為點率。



 求更點所在辰刻



 置更點率,以所求更點數因之,又六因,內加昏明分,滿辰法而一,為辰數。不盡,滿刻法除之為刻數,不滿為分,
 命其辰刻算外,即得所求。



 求四方所在漏刻



 各於所在下水漏,以定其處冬至或夏至夜刻,乃與五十刻相減,餘為至差刻。置所求日黃道去赤道內外度及分,以至差刻乘之,進一位,如二百三十九而一,為刻,不盡以刻法乘之,退除為分,內減外加五十刻,即所求日夜刻,以減百刻,餘為晝刻。其日出入辰刻及更點差率算等,並依術求之。



 求黃道內外度



 置日出分,如日法四分之一以上,去之,餘為外分。如日法四分之一以下,覆減之,餘為內分。置內外分,千乘之,如
 內外法而一,為度,不滿退除為分,即為黃道去赤道內外度。內減外加象限,即得內道去極度。



 求距中度及更差度



 置半法,以晨分減之,餘為距中分,百乘之,如周法而一,為距中度。用減一百八十三度一十二分八十四秒,餘四因退位,為每更差度。



 求昏明五更中星



 置距中度,以其日午中赤道日度加而命之,即昏中星所格宿次,因為初更中星。以更差度累加之,命赤道宿次去之,即得逐更及
 明中星。



\end{pinyinscope}