\article{本紀第十}

\begin{pinyinscope}

 章宗二



 四年春正月己巳朔,以皇太后喪,不受朝。辛未,以平章政事夾谷清臣為尚書右丞相,監修國史。丁丑,遣戶部侍郎李獻可等分路勸農事。癸未,尚書省奏大興府推官蘇德秀為禮部主事,上曰:「朕既嘗語卿,百官當使久於其職。彼方任理官,復改戶曹,尋又除禮部,人才豈能兼之?若久於其職,但中材勝於新人,事既經練,亦必有
 濟,後不可輕易改除。」上又言:「凡稱政有異迹者,謂其斷事軼才也。若止清廉,此乃本分,以貪汙者多,故顯其異耳。宰臣又言:『近言事者謂,方今孝弟廉恥道缺,乞正風俗。』此蓋官吏不能奉宣教化使然。今之察舉官吏者,多責近效,以乾辦為上,其有秉心寬厚,欲行德化者,輒謂之迂闊。故人人皆以教化為餘事,此孝弟所以廢也。若諭所司,官吏有能務行德化者,擢而用之,則教化可行,孝弟可興矣。今之所察舉,皆先才而後德。巧猾之徒,雖有臟污,一旦見用,猶為能吏,此廉恥所以喪也。若諭所司,察舉官吏,必審真偽,使有才無行者不以覬覦,非
 道求進者加之糾劾,則奔兢之俗息,而廉恥可興矣!」辛卯,賑河北諸路被水災者。癸巳,諭點檢司:「行宮外地及圍獵之處悉與民耕,雖禁地,聽民持農器出入。」丙申,東京路副使王勝進鷹,遣諭之曰:「汝職非輕,民間利害,官吏邪正,略不具聞,而乃以鷹進,此豈汝所職也!後毋復爾。」



 二月戊戌朔,如春水。始以春、秋二仲月上戊日祭社稷。癸丑,獵于姚村澱。癸亥,至自春水。丙寅,參知政事張萬公罷。



 三月戊辰朔,諸部提刑司入見,各問以職事,仍誡諭曰:「朕特設提刑司,本欲安民,于今五年,效猶未著。蓋多不識本職之體,而徒事細碎,以致州縣例皆畏宿
 而不敢行事。乃者山東民艱于食,嘗遣使賑濟,蓋卿等不職,故至於此。既往之失,其思悛改。」庚午,上將幸景明宮,御史中丞董師中等上書切諫,不報。壬申,章再上,補闕許安仁、拾遺路鐸皆諫,乃止。制定民習角牴,槍棒罪。以工部尚書胥持國為參知政事。丙子,特賜有司孔端甫及第,授小學教授,尋以年老,命食主簿半俸致仕。甲申,幸香山永安寺及玉泉山。甲午,定配享功臣。敕自今御史臺奏事,修起居注並令回避。夏四月丁酉朔,幸興陵崇妃第。是日,始舉樂。自己亥至癸卯,百官三表請上尊號,上曰:「祖宗古先有受尊號者,蓋有其德,故有其名。
 比年五穀不登,百姓流離,正當戒懼修身之日,豈得虛受榮名耶?」不許,仍斷來章。戊申,親禘于太廟。庚戌,如萬寧宮。辛亥,右丞相清臣率百官及耋艾等復請上尊號,學官劉璣亦率六學諸生趙楷等七百九十五人詣紫宸門請上尊號,如唐元和故事,不許。丁巳,賑河州饑。敕女直進士及第後,仍試以騎射,中選者升擢之。乙丑,減尚廄食穀馬。



 五月丙寅朔,曹王永升及諸王請上尊號,不許。以尚廄局使石抹貞為橫賜夏國使。己巳,上以群臣累上尊號不受,詔諭中外,徒罪以下遞降一等,杖以下原之。甲戌,觀稼于近郊。辛巳,諭左司:「遍諭諸路,令月
 具雨澤田禾分數以聞。」癸未,以久雨,鋋。



 六月癸丑,賜有司所舉德行才能之士安州崔秉仁、袞州翟駒、錦州齊文乙、大名孫可久、陳信仁、應州董戣並同進士出身。丙辰,以晴,致祭嶽鎮海瀆。壬戌,尚書右丞相夾谷清臣進封戴國公,西京留守完顏守貞為平章政事,封蕭國公。尚書右丞劉瑋薨。秋七月辛巳,南京路提刑司自許州遷治南京。己丑,制三品以上官有故者,若親、賢、勛、舊,尚書省即與聞奏,議加追贈。命以銀改鑄「禮信之寶」,仍塗以金。以同判大睦親府事襄為樞密使。以御史中丞董師中等為賀宋生日使。



 八月己亥,樞密使襄帥百僚再
 請上尊號,不許。是日,歲星、太白晝見。庚子,大赦。甲辰,至自萬寧宮。丁未,釋奠孔子廟,北面再拜。辛亥,國史院進《世宗實錄》,上服袍帶,御仁政殿,降座,立受之。



 九月甲子朔,天壽節,御大安殿,受親王百官及宋、高麗、夏使朝賀。戊辰,以參知政事來谷衡為尚書右丞,戶部尚書馬琪為參知政事。敕尚書省:「大定二十九年以後土庶言事,或係國家或邊關大利害已嘗施行者,可特補一官,有益於官民,量給以賞。」以西上閣門使大枿為夏國生日使。庚午,如山陵,次奉先縣。辛未,拜天于縣西。壬申,致奠諸陵。癸酉,如秋山。



 十一月庚午,右丞相清臣、參知政事
 持國上表丐閑,優詔不許。戊寅,以翰林直學士完顏匡等為賀宋正旦使,命匡權易名弼,以避宋諱。壬午,木冰。丙戌,詔諸職官以贓污不職被罪、以廉能獲升者,令隨路、京、府、州、縣列其姓名,揭之公署,以示勸懲。庚寅,夏國嗣子李純佑遣使來訃告。



 十二月甲午朔,夏國李純佑遣使奉故王仁孝遣表以進。諭大興府於暖湯院日給米五石,以贍貧者。戊戌,定武軍節度使鄭王永蹈以謀反,伏誅。己亥,諭有司,以鄭王財產分賜諸王,澤國公主財物分賜諸公主。甲辰,諸王府增置司馬一人。以紇石烈珵為高麗生日使,西上閣門使大枿等為夏國敕祭
 慰問使。庚戌,尚書省以科目近多得人,乞是舉增取進士。上然之,詔有司:「會試毋限人數。」甲寅,冊長白山之神為開天弘聖帝。丙辰,獵於近郊。是歲,大有年。邢、洺、深、冀及河北西路十六謀克之地,野蠶成繭。



 五年春正月癸朔,宋、高麗、夏遣使來賀。乙丑,昭容李氏進位淑妃。己巳,初用唐、宋典禮,皇后忌辰皆廢務。尚書省進區田法,詔相其地宜,務從民便。又言遣官劭農之擾,命提刑司禁止之。乙亥,以葉魯、谷神始製女直字,詔加封贈,依倉頡立廟盩厔例,祠於上京納里渾莊。歲時致祭,令其子孫拜奠,本路官一人及本千戶春秋二
 祭。辛巳,前中都路都轉運使王寂薦三舉終場人蔡州文商經明行修,足備顧問。前河北西路轉連使李揚言慶陽府進士李獎純德博學,鄉曲譽之。絳州李天祺、應州康晉侯屢赴廷試,皆有才德。上曰:「文商可令召之。李獎給主簿半俸終身,余賜同進士出身。」遣國子祭酒劉璣冊李純祐為夏國王。丁亥,幸城南別宮。



 二月丁酉,初定長吏勸課能否賞格。尚書省奏:「禮官言孝懿皇太后祥除已久,宜易隆慶宮為東宮,慈訓殿為承華殿。」從之。詔購求《崇文總目》內所闕書籍。戊戌,祭社稷,以宣獻皇后忌辰,用熙寧祀儀,樂縣而不作。甲辰,鄆王琮薨。己酉,
 宰臣請罷北邊屯駐軍馬,不允。癸丑,以齊河縣民張涓、濟陽縣王琛、河州李錡急義好施,詔復之終身,仍著于令。命宣徽使移剌敏、戶部主事赤盞實理哥相視北邊營屯,經畫長久之計。



 三月壬申,初定限錢禁。庚辰,初定日月風雨雷師常祀。戊子,置弘文院,譯寫經書。夏四月壬辰朔,幸北苑。庚子,詔各路所舉德行才能之士,涿州時琦、雲中劉摯、鄭州李升、恩州傅礪、濟南趙摯、興中田扈方六人,並特賜同進士出身。以文商為國子教授,特遷登仕郎。己酉,詔自今筐櫝床榻之飾毋以金玉。壬子,特賜翰林待制溫迪罕迪翰林學士承旨、中奉大夫。乙
 卯,幸景明宮,董師中、賈守謙、路鐸先後凡兩上封事切諫,不報。



 五月庚午,次烏十撒八。戊子,桓、撫二州旱,遣使禱于縉山。



 六月壬辰,如冰井。己亥,出獵。登胡土白山。酹酒再拜。曹王永升以下進酒。丙午,拜天,曲赦西北路,己未,如查沙秋山。是月,宋前主甗殂。



 七月戊辰,獵于豁赤火,一發貫雙鹿。是日,獲鹿二百二十二,賜扈從官有差。辛巳,次魯溫合失不。是日,上親射,獲黃羊四百七十一。乙酉,次冰井。丙戌,以天壽節,宴樞光殿,凡從官及承應人遇覃恩遷秩者,並受宣敕於殿前。時久雨初霽,有龍曳尾于殿前雲間。戊子,御膳羹中有發,上舉視而棄之,
 戒左右毋宣言。



 八月辛亥,至自景明宮。壬子,河決陽武故堤,灌封丘而東。丁巳,賜從幸山後親軍銀、絹有差。



 九月戊午朔,天壽節,宋、高麗、夏遣使來賀。壬戌,命增定捕盜官被殺賻錢及官賞格。甲子,都水監官王汝嘉等坐河決,各削官兩階,杖七十,罷之。乙丑,上御睿思殿,諸路提刑使入見。戊辰,初令民買撲隨處金、銀、銅冶。命參知政事馬琪往視河決,仍許便宜從事。壬申,宋主遣使來告哀。戊寅,以知大興府事尼厖古等為鑒宋國弔祭使。敕尚書省,集百官議備邊事。壬午,特推恩東宮舊人司經王伯溫等八人官有差。甲申,命上京等九路并諸抹
 及糺等處選軍三萬,俟來春調發,仍命諸路并北阻棨以六年夏會兵臨潢。冬十月庚寅,右丞相夾谷清臣等表請上尊號,不允。宋遺使獻遣留物。壬寅,右丞相清臣復清上尊號,國子祭酒劉璣亦率六學諸生上表陳請,不允。遣戶部員外郎何格賑河決被災人戶。庚戌,張汝弼妻高陀斡以謀逆,伏誅。壬子,尚書省奏,升提刑司所察廉官南皮縣令史肅以下十有二人,而大興主簿蒙括蠻都亦在選中,上知其人,曰:「蠻都澆浮人也,升之可乎?與其任澆浮,孰若用淳厚。況蠻都常才,才智過人猶不當用,恐敗風俗,況常才耶!其再察之。」



 閏月戊午朔,宋
 主遣使報即位。甲子,親王、百官各奉表請上尊號,不允。丙寅,以代國公歡都等五人配享祖廟廷。甲戌,以河東南、北提刑使王啟等為賀宋主即位使。乙亥,獵于近郊。戊寅,上問輔臣曰:「孔子廟諸處何如?」平章政事守貞曰:「諸縣見議建立。」上因曰:「僧徒修飾宇像甚嚴,道流次之,惟儒者於孔子廟最為滅裂。」守貞曰:「儒者不能長居學校,非若僧道久處寺觀。」上曰:「僧道以佛、老營利,故務在莊嚴閎侈,起人施利自多,所以為觀美也。」庚辰,參知政事馬琪自行省回,具奏河防利害,語載《琪傅》中。丙戌,以翰林待制奧屯忠孝權戶部侍郎,太府少監溫昉權
 工部侍郎,行戶、工部事,修治河防。以引進使完顏衷為夏國生日使。



 十一月癸巳,詔罷紫荊嶺所護圍場。庚子,以右宣徽使移剌敏等為賀宋正旦使。癸丑,太白晝見。



 十二月辛酉,平章政事完顏守貞罷。以知大興府事尼厖古鑑為參知政事,以戶中郎中李敬義為賜高麗生日使。丁卯,免被黃河水災今年秋稅。辛巳,敕減修內司備營造軍千人,都城所五百人。癸未,敕尚書省,自今獻靈芝嘉禾者賞。



 六年春正月丁亥朔,受宋、高麗、夏使朝賀。庚寅,太白晝見。辛卯,敕有司給天水郡公家屬田宅。壬辰,如春水。庚
 戌,罷陜西括地。辛亥,諭胥持國,河上役夫聚居,恐生疾疫,可廩醫護視之。乙卯,次御林。



 二月丁巳朔,敕有司:「行宮側及獵所有農者勿禁。」己未,始祭高禖。庚午,至自春水。丁丑,京師地震。大雨雹,晝晦,震應天門右鴟尾。癸未,宋遣使來報謝。



 三月丙戌朔,日有食之。甲午,以翰林直學士孛術魯子元兼右司諫,監察御史田仲禮為左拾遣,翰林修撰僕散訛可兼右拾遺,諭之曰:「國家設置諫官,非取虛名,蓋責實效,庶幾有所裨益。卿等皆朝廷選擢,置之諫職,如國家利害、官吏邪正,極言無隱。近路鐸左遷,本以他罪,卿等勿以被責,遂畏縮不言,其悉心戮
 力,毋得緘默。」丙申,如萬寧宮。戊戌,以北邊糧運,括群牧所、三招討司猛安謀克、隨糺及迭剌、唐古部諸抹、西京、太原官民駝五千充之,惟民以駝載為業者勿括。以銀五十萬兩、錢二十三萬六千九百貫以備支給。銀五萬兩、金盂二千八百兩、金牌百兩、銀盂八千兩、絹五萬匹、雜彩千端、衣四百四十六襲以備賞勞。庚子,以郡舉才行之士翟介然以下三人特賜進士及第,李貞固以下十五人同進士出身。夏四月癸亥,敕有司:「以增修曲阜宣聖廟工畢,賜衍聖公以下三獻法服及登歌樂一部,仍遣太常舊工往教孔氏子弟,以備祭禮。」甲子,以尚書
 左丞烏林答愿為平章政事,右丞夾谷衡為尚書左丞。丙子,幸玉泉山。戊寅,以修河防工畢,參知政事胥持國進官二階,翰林待制奧屯忠孝以下三十六人各一階,獲嘉令王維翰以下五十六人各賜銀弊有差。庚辰,以尚書右丞相來谷清臣為左丞相,監修國史,封密國公。樞密使襄為尚書右丞相,封任國公。參知政事胥持國為尚書右丞。壬午,賜宰臣手詔,以風俗不淳,官吏茍且,責之。



 五月丙戌,命減萬寧宮陳設九十四所。辛卯,以出師,遣禮部尚書張暐告于廟社。乙未,判平陽府事鎬王永中以罪賜死,并及二子,丁酉,詔中外。乙巳,詔諸路猛
 安謀克農隙講武,本路提刑司察其惰者罰之。庚戌,命左丞相來谷清臣行省于臨潢府。



 六月丙辰,右諫議大夫賈守謙、右拾遺僕散訛可坐鎬王永中事奏對不實,削官二階,罷之。御史中丞孫即康,右補闕蒙括胡剌、右拾遺田仲禮各罰金二十斤。丙寅,以樞密副使唐括貢為樞密使。以久雨,鋋。庚辰,太白經天。辛巳,左丞相清臣遣使來獻捷。



 七月丙申,幸曹王永升第。甲辰,始定文武官六貫石以上、承應人并及廕者、若在籍儒生章服制。



 八月己未,命袞州長官以曲阜新修廟告成于宣聖。癸亥,至自萬寧宮。己巳,以溫敦伯英言,命禮部令學官講
 經。辛未,以吏部尚書吳鼎樞等賀宋生日使。壬申,行省都事獨吉永中來報捷。乙亥,敕宮中承應人出職後三年內犯贓罪者,元舉官連坐,不在去官之限,著為令。辛巳,木波進馬。



 九月壬午朔,天壽節,宋、高麗、夏遣使來賀。甲申,冊靜寧山神為鎮安公,忽土白山神為瑞聖公。丙戌,知河間府事移剌仲方為御史大夫。辛卯,如秋山。以尚書左司郎中粘割胡上為夏國生日使。冬十月丙辰,至自秋山。丁巳,以歲幸春水、秋山,五日一進起居表,自今可十日一進。乙亥,命尚書左丞來谷衡行省于撫州,命選親軍、武衛軍各五百人以從,仍給錢五千萬。



 十
 一月戊子,左丞相夾谷清臣罷,右丞相襄代領行省事。丙申,以刑部尚書紇石烈貞等為賀宋正旦使。壬寅,初定猛安謀克鎮邊後放免者授官格。禁射糧軍,應役但成隊伍,不得持兵器及凡可以傷人者。甲辰,報敗敵於望雲。乙巳,以樞密使唐括貢、御史大夫移剌仲方、禮部尚書張暐等二十三人充計議官,凡軍事則議之。戊申,初定縣官增水田升除制。



 十二月乙卯,詔招撫北邊軍民。以知登聞檢院賈益為高麗生日使,戶部員外郎納蘭昉為橫賜使。戊午,禮部尚書張暐等進《大金儀禮》。丁卯,應奉翰林文字趙秉文上書論姦欺。乙亥,詔加五鎮
 四瀆王爵。庚辰,上幸後園閱軍器。是月,右丞相襄率駙馬都尉僕散揆等進軍大鹽濼,分兵攻取諸營。



 承安元年春正月辛巳朔,受宋、高麗、夏使朝賀。甲申,大鹽濼群牧使移剌睹等為廣吉剌部兵所取敗,死之。丁亥,國子學齊長張守愚上《平邊議》三篇,特授本學教授,仍以其議付史館。



 二月甲子,命有司祀高禖如新儀。丁卯,右丞相襄、左丞衡至自軍前。己巳,復命還軍。幸都南行宮春水。甲戌,至自行宮。是月,初造虎符發兵。



 三月丁酉,如萬寧宮。不雨,遣宮望祭嶽鎮海瀆于北郊。癸卯,敕尚書省:「刑獄雖已奏行,其間恐有疑枉,其再議以聞。人命
 至重,不可不慎也。」甲辰,遣參知政事尼厖古鑑祈雨於社稷。丁未,復遣使就祈雨于東嶽。夏四月辛亥,命尚書右丞胥持國祈雨于太廟。壬子,遣使審決冤獄。京城禁傘扇。戊午,初行區種法,民十五以上、六十以下有土田者,丁種一畝。乙丑,命御史大夫移剌仲方祈雨于社稷。壬申,命參知政事馬琪祈雨于太廟。甲戌,尚書省以趙承元言,請追上孝孝懿皇太后冊寶,然後行謚冊禮。禮官執奏尊皇太后已詔示中外,無追冊禮,從之。戊寅,上以久不雨,命禮部尚書張暐祈于北嶽。己卯,遣官望祭嶽鎮海瀆于北郊。



 五月庚辰朔,觀稼于近郊,因閱區田。乙酉,
 以久旱,徙市。庚寅,詔復市如常。壬辰,以尚藥局副使粘割忠為橫賜夏國使。乙未,參知政事尼厖古鑒薨。庚子,雨足。



 六月甲寅,上以百姓艱食,詔出倉粟十萬石減價以糶之。乙丑,平晉縣民利通家蠶自成綿段,長七尺一寸五分,闊四尺九寸,詔賜絹十匹。丁卯,敕自今長老、大師、大德不限年甲,長老、大師許度弟子三人,大德二人,戒僧年四十以上者度一人。其大定十五年附籍沙彌年六十以上並令戒,仍不許度弟子。尼、道士、女冠亦如之。御史大夫移剌仲方罷。庚午,幸環秀亭觀稼。癸酉,詔應禁軍器路分,步弓手擬於射糧軍內選之,馬弓手
 擬於猛安謀克軍戶餘丁內選之。其有為百姓害,從本州縣斷遣。無猛安戶,於二百里內屯駐軍餘丁內取之,依步弓手月給二貫石。



 七月庚辰,御紫宸殿,受諸王、百官賀,賜諸王、宰執酒。敕有司:「以酒萬尊置通衢,賜民縱飲。」乙酉,敕今後高麗、夏使入見數奏,令新設各國通事具公服與閣門使上殿監聽。命有司收瘞西北路陣亡骸骨。



 八月己酉,獵于近郊。癸丑,幸玉泉山。甲子,以郊祀日期詔中外。戊辰,至自萬寧宮。以陜西西北路轉運使董師中為御史大夫。癸酉,左丞衡丁父憂。



 九月丁丑朔,天壽節,宋、高、夏遣使來賀。幸天長觀。辛巳,以右丞相襄
 為左丞相,監修國史,封常山郡王。壬午,賜襄酒百尊。太白晝見。癸未,都人進酒三千一百瓶,詔以賜北邊軍吏。以吏部尚書張嗣等為賀宋生日使。癸巳,左丞衡起復。丁酉,知大興府卜、同知郭鑄以擅逮問宰臣,各笞四十。辛丑,西南路招討使僕散揆至自軍。乙巳,以國子監丞烏古論達吉不為夏國生日使。冬十月丙午朔,詔選親軍八百人戍撫州。庚戌,命左丞相襄行省于北京,簽書樞密院事完顏匡行院於撫州。丙辰,祫享于太廟。



 十一月戊子,參知政事馬琪罷。庚寅,特滿群牧契丹陀鎖、德壽反,泰州軍擊敗之。御史大夫董師中、北京留守裔並
 為參知政事。甲午,以陜西路統軍使崇道等為賀宋正旦使。丁酉,朝享于太廟。戊戌,有事于南郊,大赦,改元。己亥,曹王永升率親王、百官賀。癸卯,命有司祈雪,仍遣官祈於東嶽。



 十二月丙午,樞密使唐括貢率百官請上尊號,不允。乙酉,遣提點太醫近侍局使李仁惠勞賜北邊將士,授官者萬一千人,授賞者幾二萬人,凡用銀二十萬兩,絹五萬匹、錢三十二貫。庚戌,以同知登聞檢院阿不罕德剛為高麗國生日使,壬子,樞密使唐括貢復率百官請上尊號,不允。



 二年春正月乙亥朔,宋、高麗、夏遣使來賀。乙酉,敕職官
 犯贓私不得訴於同官。丁亥,如安州春水。丁酉,至自春水。辛丑,宋主以母后喪,遣使告哀。



 二月丁巳,敕自今職官犯贓,每削一官殿一年。是日,太白晝見,經天。是月,特命襲封衍聖公孔元措世襲兼曲阜令。



 三月己卯,親王、百官復請上尊號,不允。壬午,命尚書戶部侍郎溫昉佩金符,行六部尚書於撫州。庚寅,幸西園閱軍器。辛卯,始定保舉德行才能格。癸巳,平章政事烏林答愿罷。丁酉,樞密使唐括貢率百官請上尊號,不允。以參知政事裔代左丞相襄行省于北京。夏四月甲寅,如萬寧宮。丙辰,命有司祈雨,望祭嶽鎮海瀆于北郊。甲子,祈雨于社稷。
 尚書省奏:「比歲北邊調度頗多,請降僧道空名度牒紫褐師德號以助軍儲。」從之。癸酉,親王宣敕始用女直字。



 五月甲戌朔,諭宰臣曰:「比以軍須,隨路賦調。司縣不度緩急,促期徵斂,使民費及數倍,胥吏又乘之以侵暴。其令提刑司究察之。」丙子,集官吏于尚書省,詔諭之曰:「今紀綱不立,官吏弛慢,遷延茍簡,習以成弊。職官多以吉善求名,計得自安,國家何賴焉?至於徇情賣法,省部令史尤甚。尚書省其戒諭之。」丁丑,北京行省參知政事裔移駐臨潢府。庚辰,升撫州為鎮寧軍。以雨足,報祭于社稷。甲申,望祭嶽鎮海瀆于北郊。丁亥,左丞相襄詣臨潢
 府。己丑,皇子生,庚寅,詔中外,降死罪,釋徒以下。



 六月乙巳,命禮部尚書張暐報祀高禖。丙午,雨雹。戌申,以澄州刺史王遵古為翰林直學士,仍敕無與選述,入直則奏聞,或霖雨,免入直,以遵古年老,且嘗侍講讀也。庚戌,詔罷瑤光殿工作。甲寅,置全州盤安軍節度使,治安豐縣。乙卯,封皇子為壽王。



 閏月甲午,出西橫門觀稼。秋七月壬寅朔,幸天長觀,建普天大醮,禁屠宰七日,無奏刑,百司權停決罰。己未,命西上閣門使劉頍賜參知政事裔宴于行省。戊辰,天壽節,御紫宸殿受朝。



 八月庚辰,敕計議官所進奏帖,可直言利害,勿用浮辭。辛巳,以邊事未
 寧,詔集六品以上官於尚書省,問攻守之計。應中外臣僚不以職位高下,或有方略材武,或長於調度,各舉三五人以備選用,無有顧望不盡所懷,期五日封章以進。議者凡八十四人,言攻者五,守者四十六,且攻且守者三十三,召對睿思殿,論難久之。癸未,至自萬寧宮。丙戌,以左丞相襄為左副元帥,參知政事董師中尚書左丞,左宣徽使嵒尚書右丞,戶部尚書楊伯通參知政事。尚書左丞夾谷衡罷。右丞胥持國致仕。庚寅,參知政事裔罷。樞密使唐括貢致仕。壬辰,以左副元帥襄為樞密使兼平章政事。



 九月辛丑朔,天壽節,宋、高麗、夏遣使來賀。
 壬寅,遣官分詣上京、東京、北京、咸平、臨潢、西京等路招募漢軍,不足則簽補之。乙巳,以夏使朝辭,詔答許復保安、蘭州榷場。丁未,以知歸德府事完顏愈為賀宋生日使。癸丑,以上京留守粘割斡特剌為平章政事。辛酉,以樞密使兼平章政事襄,知大興府事胥持國為樞密副使、權參知政事,行省于北京。乙丑,始置軍器監,掌治戎器,班少府監下,設甲坊、利器二署隸焉。丁卯,分遣官於東、西、北京,河北等路,中都二節鎮,買牛五萬頭。冬十月庚午朔,初設講議所官十員,共議錢穀,以中都路轉運使孫鐸、戶部侍郎高汝礪等為之。庚辰,尚書省奏,高麗
 國牒報,其王以老疾,令母弟晫權國事。壬午,尚書省行推排。丁亥,皇子壽王薨。壬辰,詔獎諭西南路招討使僕散揆等有功將士。甲午,大雪,以米千石賜普濟院,令為粥以食貧民。丙申,以禮部員外郎蒙括仁本為夏國生日使。



 十一月甲辰,冬至,有事於南郊。乙巳,以薪貴,敕圍場地內無禁樵採。壬子,諭尚書省:「猛安謀克既不隸提刑司,宜令監察御史察其臧否。」庚申,北京留守裔以行省失職,杖一百,除名。右諫議大夫納蘭昉杖九十,削官二階,罷之。甲子,諭宰臣曰:「朕居九重,民間難以遍知,宰相不見賓客,何以得知民間利害。」



 十二月己巳朔,敕御
 史臺糾察諂佞趨走有實跡者。己卯,始鑄「承安寶貨」。癸未,遣戶部侍郎上官瑜體究西京逃亡,勸率沿邊軍民耕種,戶部郎中李敬義規措臨潢等路農務。乙酉,諭宰臣:「今後水潦旱蝗、盜賊竊發,命提刑司預為規畫。」戊子,諭西南路將士。庚寅,豫王永成進馬八十匹,賜詔獎諭,稱皇叔豫王而不名。



\end{pinyinscope}