\article{本紀第九}

\begin{pinyinscope}

 章宗一



 章宗憲天光運仁文義武神聖英孝皇帝,諱璟,小字麻達葛,顯宗嫡子也。母曰孝懿皇后徒單氏。大定八年,世宗幸金蓮川,秋七月丙戌,次冰井,上生。翌日,世宗幸東宮,宴飲歡甚,語顯宗曰:「祖宗積慶而有今日,社稷之福也。」又謂司徒李石、樞密使紇石烈志寧等曰:「朕子雖多,皇后止有太子一人。幸見嫡孫又生於麻達葛山,朕嘗
 喜其地衍而氣清,其以山名之。」群臣皆稱萬歲。十八年,封金源郡王。始習本朝語言小字,及漢字經書,以進士完顏匡、司經徐孝美等侍讀。二十四年,世宗東巡,顯宗守國,上奉表詣上京問安,仍請車駕還都,世宗嘉其意,賜敕書答諭。二十五年三月,萬春節,復奉表朝賀。六月,顯宗崩,世宗遣滕王府長史鳷、御院通進鷿來護視。十二月,進封原王,判大興府事。入以國語謝,世宗喜,且為之感動,謂宰臣曰:「朕當命諸王習本朝語,惟原王語甚習,朕甚嘉之。」諭旨曰:「朕固知汝年幼,服制中未可付以職,然政事亦須學,京輦之任,姑試爾才,其勉力。」二十六
 年四月,詔賜名璟。五月,拜尚書右丞相。世宗謂曰:「宮中有《輿地圖》,觀之可以具知天下遠近阨塞。」又謂宰臣曰:「朕所以置原王於近輔者,欲令親見朝廷議論,習知政事之體故也。」十一月,詔立為皇太孫,稱謝於慶和殿。世宗諭之曰:「爾年尚幼,以明德皇后嫡孫惟汝一人,試之以事,甚有可學之資。朕從正立汝為皇太孫,建立在朕,保守在汝,宜行正養德,勿近邪佞,事朕必盡忠孝,無失眾望,則惟汝嘉。」二十七年三月,世宗御大安殿,授皇太孫冊,赦中外。丁巳,謁謝太廟及山陵。始受百官箋賀。二十八年十二月乙亥,世宗不豫,詔攝政,聽授五品以下
 官。丁亥,受「攝政之寶」。



 二十九年春正月癸巳,世宗崩,即皇帝位于柩前。丙申,詔中外。賜丙外官覃恩兩重,三品已上者一重,免今歲租稅,并自來懸欠係官等錢,鰥寡孤獨人絹一匹、米兩石。己亥,遷大行皇帝梓宮于大安殿。癸卯,以皇太后命為令旨。甲辰,以大理卿王元德等報哀于宋、高麗、夏。乙卯,白虹貫日亙天。丁巳,參知政事宗浩罷。山東統軍裔以私過都城不赴哭臨,笞五十,降授彰化軍節度使。戊午,名皇太后宮曰仁壽,設衛尉等官。



 二月辛酉朔,日有食之。癸亥,始聽政。追尊皇考為皇帝,尊母為皇太后。甲子,命學士院進呈漢、唐便民事,及
 當今急務。乙丑,白虹亙天。敕登聞鼓院所以達冤枉,舊嘗鎖戶,其令開之。戊辰,更仁壽宮名隆慶。詔宮籍監戶舊係睿宗及大行皇帝、皇考之奴婢者,悉放為良。己巳,敕御史臺,自今監察令本臺辟舉,任內不稱職亦從奏罷。丁丑,增定百官俸。乙酉,詔有司稽考典故,許引用宋事。是月,宋主內禪,子惇嗣立。



 三月壬辰,朝于隆慶宮,是月凡五朝。己酉,詔以生辰為天壽節。癸丑,夏國遣使來弔。夏四月己巳,夏國遣使來祭。辛未,宋遣使來弔祭。乙酉,葬世宗光天興運文德武功聖明仁孝皇帝於興陵。戊子,朝于隆慶宮。



 五月庚寅朔,太白晝見。壬寅,宋主遣
 使來報嗣位。夏國遣使來賀即位。丙午,以祔廟禮成,大赦。丁未,地生白毛。庚戌,詔罷送宣錢,今後諸護衛考滿賜官錢二千貫。壬子,敕收錄功臣子孫,量材於分承應。戊午,朝于隆慶宮。以東北路招討使溫迪罕速可等為賀宋主即位使。河溢曹州。



 閏月庚申朔,封兄珣為豐王,琮鄆王,瑰瀛王,從彞沂王,弟從憲壽王,玠溫王。辛酉,制諸飢民賣身已贖放為良,復與奴生男女,並聽為良。丙寅,觀稼于近郊。庚午,以樞密副使唐括貢為御史大夫。壬申,封乳母孫氏蕭國夫人,姚氏莘國夫人。丙子,進封趙王永中漢王,曹王永功冀王,豳王永成吳王,虞王
 永升隨王,徐王永蹈衛王,滕王永濟潞王,薛王永德潘王。庚辰,宋遣使來賀即位。癸未,朝于隆慶宮。詔學士院,自今誥詞並用四六。乙酉,詔諸有出身承應人,係將來受親民之職,可命所屬諭使為學。其護衛、符寶、奉御、奉職,侍直近密,當選有德行學問之人為之教授。



 六月己丑朔,有司言:「律科舉人止知讀律,不知教化之原,必使通治《論語》、《孟子》,涵養器度。遇府、會試,委經義試官出題別試,與本科通定去留為宜。」從之。詔有司,請親王到任各給錢二十萬。辛卯,修起居注完顏烏者、同知登聞檢院孫鐸皆上書諫罷圍獵,上納其言。拾遺馬升上《儉德箴》。
 乙未,初置提刑司,分按九路,並兼勸農採訪事,屯田、鎮防諸軍皆屬焉。丁酉,幸慶壽寺。作瀘溝石橋。己亥,朝于隆慶官。甲辰,罷送赦禮物錢,朝于隆慶官。乙卯,高麗國王皓遣使來弔祭及會葬。敕有司移報宋、高麗、夏,天壽節於九月一日來賀。丁巳,命提刑官除後於便殿聽旨,每十月使副內一員入見議事,如止一員則令判官入見,其判官所掌煩劇可升同隨朝職任。秋七月辛酉,減民地稅十之一,河東南、北路十之二,下田十之三。甲子,朝于隆慶宮。乙丑,敕近侍官授外任三品、四品、賜金帶一,重幣有差。丁卯,以太尉、尚書令東平郡王徒單克寧
 為太傅,改封金源郡王。辛未,高麗遣使來賀即位。甲戌,奉皇太后幸壽安宮。辛巳,詔京、府、節鎮、防禦州設學養士。初設經童科。御史大夫唐括貢罷。禮部尚書移剌履為參知政事。以刑部尚書完顏守貞等為賀宋生日使。



 八月戊子朔,奉皇太后幸壽安宮。辛卯,敕有司,京、府、州、鎮設學校處,其長貳幕職內各以進士官提控其事,仍具入銜。壬辰,初定品官子孫試補令史格,及提刑部司所掌三十二條。左司諫敦安民上疏論三事:曰崇節儉,去嗜欲,廣學問。丁酉,如大房山。戊戌,謁奠諸陵。己亥,還都。庚子,朝于隆慶宮,是月凡三朝。壬寅,制提刑司設女直、
 契丹、漢兒知法各一人。甲辰,參知政事劉瑋罷。丙辰,宋、高麗、夏遣使來賀天壽節。



 九月戊午朔,天壽節,以世宗喪,不受朝。庚申,詔增守山陵為二十丁,給地十頃。壬戌,詔罷告捕亂言人賞。甲子,制諸盜賊聚集至十人,或騎五人以上,所屬移捕盜官捕之,仍遞言省部,三十人以上聞奏,違者杖百。是日,朝於隆慶宮。是月凡四朝。丁卯,制強族大姓不得與所屬官吏交往,違者有罪。戊辰,以隆慶宮衛尉把思忠為夏國生日使。庚午,以尚輦局使崇德為橫賜高麗使。丙子,獵于近郊。戊寅,監察御史焦旭劾奏太傅克寧、右丞相襄不應請車駕田獵,上曰:「此
 小事,不須治之。」乙酉,如大房山。冬十月丁亥朔,謁奠諸陵。己丑,還都。庚寅,朝于隆慶宮,是月凡四朝。辛卯,上顧謂宰臣曰:「翰林闕人。」平章政事汝霖對曰:「鳳翔治中郝俁可。」汝霖諫止田獵,詔答曰:「卿能每事如此,朕復何憂?然時異事殊,得中為當。」丙申,冬獵。己亥,次羅山。庚子,次玉田。辛丑,沁州、丹州進嘉禾。丁未,次寶坻。庚戌,中侍石抹阿古誤帶刀入禁門,罪應死,詔杖八十。癸丑,至自寶坻。



 十一月己未,朝于隆慶宮。辛酉,以右宣徽院使裴滿餘慶等為賀宋正旦使。癸亥,上謂宰臣曰:「今之用人,太拘資歷。循資之法,起於唐代,如此何以得人?」平章政事
 汝霖對曰:「不拘資格,所以待非常之材。」上曰:「崔祐甫為相,未踰年薦八百人,豈皆非常之材歟?」甲子,諭尚書省曰:「太傅年高,每趨朝而又赴省,恐不易。自今旬休外,四日一居休,庶得調攝。常事他相理問,惟大事白之可也。」戊辰,諭尚書省,自今五品以上官各舉所知,歲限所舉之數,如不舉者坐以蔽賢之罪。仍依唐制,內五品以上官到任即舉自代,並從提刑司採訪之。己巳,初制轉遞文字法。壬申,朝于隆慶宮。乙亥,命參知政事移剌履提控刊修《遼史》。丁丑,以西上閣門使移剌邴為高麗生日使。御史臺奏:「故事,臺官不得與人相見。蓋為親王、宰執、
 形勢之家,恐有私徇。然無以訪知民間利病、官吏善惡。」詔自今許與四品以下官相見,三品以上如故。辛巳,詔有司,今後諸處或有饑饉,令總管、節度使或提刑司先行賑貸或賑濟,然後言上。



 十二月丙戌朔,朝于隆慶宮,是月凡五朝。詔罷鑄錢。丁亥,密州進白雉。壬辰,諭有司,女直人及百姓不得用網捕野物,及不得放群雕枉害物命,亦恐女直人廢射也。戊戌,復置北京、遼東鹽使司,仍罷西京、解鹽巡捕使。以河東南、北路提刑司言,賑寧化、保德、嵐州饑,其流移復業,給復一年。是日,禁宮中上直官及承應人毋得飲酒。乙巳,祭奠興陵。壬子,諭臺臣曰:「提刑司
 所舉劾多小過,行則失大體,不行則恐有所沮,其以此意諭之。」甲寅,宋、高麗、夏遣使來賀正旦。是冬,無雪。



 明昌元年春正月丙辰朔,改元。以世宗喪,不受朝賀。上朝于隆慶宮,是月凡四朝。丁巳,制諸王任外路者許游獵五日,過此禁之,仍令戒約人從,毋擾民。辛酉,諭尚書省:「宰執所以總持國家,不得受人饋遺。或遇生辰,受所獻毋過萬錢。若緦大功以上親,及二品以上官,不禁。」壬戌,以知河中府事王蔚為尚書右丞,刑部尚書完顏守貞為參知政事。甲子,如大房山。乙丑,奠謁興陵、裕陵。丙寅,還都。戊辰,制禁自披剃為僧道者。敕外路求世宗御
 書。辛未,如近畿春水。己卯,如春水。



 二月丁卯朔,太白晝見。丙申,遣諭諸王,凡出獵毋越本境。壬寅,諭有司,寒食給假五日,著于令。甲辰,至自春水。朝于隆慶宮,是月凡四朝。癸丑,地生白毛。甲寅,如大房山。



 三月乙卯朔,謁奠興陵。丙辰,還都。朝于隆慶宮,是月凡六朝。己未,敕點檢司,諸試護衛人須身形及格,若功臣子孫善射出眾,雖不及格,亦令入見。癸亥,禮官言:「民或一產三男,內有才行可用者可令察舉,量材敘用。其驅婢所生,舊制官給錢百貫,以資乳哺,尚書省請更給錢四十貫,贖以為良。」制可。丙寅,有司言:「舊制,朝官六品以下從人輸庸者聽,五品
 以上不許輸庸,恐傷官體。其有官職俱至三品、年六十以上致仕者,人力給半,乞不分內外,願令輸庸者聽。」從之。己巳,擊球於西苑,百僚會觀。癸酉,詔內外五品以上,歲舉廉能官一員,不舉者坐蔽賢罪。乙亥,初設應制及宏詞科。丁丑,制內外官并諸局承應人,遇祖父母、父母忌日並給假一日。辛巳,詔修曲阜孔子廟學。壬午,如壽安宮。夏四月甲申朔,朝于隆慶宮,是月凡四朝。戊戌,如壽安宮。



 五月,不雨。乙卯,祈于北郊及太廟。朝于隆慶宮,是月凡三朝。丙辰,以鷹坊使移剌寧為橫賜夏國使。戊午,拜天於西苑。射柳、擊球,縱百姓觀。壬戌,祈雨于社稷。
 甲子,制省元及四舉終場人許該恩。己巳,復祈雨於太廟。庚午,置知登聞鼓院事一人。丙子,以祈雨,望祭嶽鎮海瀆于北郊。戊寅,命內外官五品以上,任內舉所知才能官一員以自代。壬午,以參政事移剌履為尚書右丞,御史中丞徒單鎰為參知政事,尚書右丞相襄罷。



 六月己丑,制定親王家人有犯,其長史府掾失覺察、故縱罪。壬辰,奉皇太后幸慶壽寺。甲辰,敕僧、道三年一試。秋七月己巳,以禮部尚書王翛等為賀宋生日使。庚午,朝於隆慶宮。丁丑,詔罷西北路蝦麻山市場。



 八月癸未朔,禁指託親王、公主奴隸占綱船、侵商旅及妄徵錢債。乙
 酉,詔設常平倉。丁亥,至自壽安宮。戊子,朝於隆慶宮,是月凡三朝。己丑,以判大睦親府事宗寧為平章政事。壬辰,幸玉泉山,即日還宮。癸巳,罷諸府鎮流泉務。選才幹之官為諸州刺史,皆召見諭戒之。戊戌,上諭宰臣曰:「何以使民棄末而務本,以廣儲蓄?」令集百官議。戶部尚書鄧儼等曰:「今風俗侈靡,宜定制度,辨上下,使服用居室,各有差等。抑昏喪過度之禮,禁追逐無名之費。用度有節,蓄積自廣矣!」右丞履、參知政事守貞、鎰曰:「凡人之情,見美則願,若不節以制度,將見奢侈無極,費用過多,民之貧乏,殆由此致。方今承平之際,正宜講究此事,為經
 久法。」上是履議。壬寅,敕麻吉以皇家袒免之親,特收充尚書省祗候郎君,仍為永制。丁未,獵于近郊。巳酉,宋、高麗、夏遣使來賀天壽節。



 九月壬子朔,天壽節,以世宗喪,不受朝。丙辰,為廉能進擢北海縣令張翱等十八人官。己未,以武衛軍副都指揮使烏林答謀甲為夏國生日使。庚申,朝于隆慶宮。壬戌,如秋山。冬十月丁亥,至自秋山。戊子,朝于隆慶宮。丙申,詔賜貴德州孝子翟單、遂州節婦張氏各絹十匹、粟二十石。戊戌,以有司言,登聞鼓院同記注院,勿有所隸。制民庶聘財為三等,上百貫,次五十貫,次二十貫。丁未,獵于近郊。



 十一月乙卯,朝于隆
 慶宮。是月凡五朝。以惑眾亂民,禁罷全真及五行毗盧。以僉書樞密院事把德固等為賀宋正旦使。丁巳,制諸職官讓蔭兄弟子侄者,從其所請。戊辰,召禮部尚書王翛、諫議大夫張暐詣殿門,諭之曰:「朝廷可行之事,汝諫官、禮官即當辯析。小民之言,有可採者朕尚從之,況卿等乎?自今所議毋但附合於尚書省。」辛未,以西上閣門使移剌撻不也為高麗生日使。丙子,冬獵。巳卯,次雄州。判真定府事吳王永成、判定武軍節度使隋王永升來朝。



 十二月壬午,免獵地今年稅。丁亥,次饒陽。己丑,平章政事張汝霖薨。丁酉,至自饒陽。甲辰,幸太傳徒單克寧
 第視疾。以克寧為太師、尚書令,封淄王,賜銀千五百兩,絹二千匹。乙巳,朝于隆慶宮。丙午,詔有司,正旦可先賀隆慶宮,然後進酒。丁未,宋、高麗、夏遣使來賀正旦。



 二年春正月庚戌朔,以世宗喪,不受朝。癸丑,諭有司,夏國使可令館內貿易一日。尚書省言,故事許貿易三日,從之。甲寅,始許宮中稱聖主。乙卯,皇太后不豫,自是日往侍疾,丙夜乃還。辛酉,皇太后崩。丙寅,以左副都點檢皞等報哀于宋、高麗、夏。庚午,太師、尚書令淄王徒單克寧薨。甲戌,百官表請聽政,不許。戊寅,詔賜陀括里部羊三萬口、重幣五百端、絹二千匹,以振其乏。吳王永成、隋
 王永升以聞國喪奔赴失期,罰其俸一月,其長史笞五十。己卯,在司言,漢王永中以疾失期,上諭使回。



 二月壬午,百官復聽政,不許。壬辰,上始視朝。敕親王及三品官之家,毋許僧尼道士出入。諭有司:「進士程文但合格者即取之,毋限人數。」丙申,以樞密副使夾谷清臣為尚書左丞。戊戌,更定奴誘良人法。丙午,初設王傳府尉官。



 三月丁巳,夏國遣使來弔。癸亥,敕有司,國號犯漢、遼、唐、宋等名不得封臣下。有司議:「以遼為恒,宋為汴,秦為鎬,晉為并,漢為益,梁為邵,齊為彭,殷為譙,唐為絳,吳為鄂,蜀為夔,陳為宛,隋為涇,虞為澤。」制可,丁卯,夏國遣使來
 祭。乙亥,高麗遣使來弔祭。丁丑,宋遣使來弔祭。



 四月戊寅朔,尚書省言:「齊民與屯田戶往往不睦,若令遞相婚姻,實國家長久安寧之計。」從之。乙酉,葬孝懿皇太后于裕陵。戊子,制諸部內災傷,主司應言而不言及妄言者杖七十,檢視不以實者罪如之,因而有傷人命者以違制論,致枉有徵免者坐贓論,妄告者戶長坐詐不以實罪,計贓重從詐匿不輸法。庚寅,禁民庶不得服純黃銀褐色,婦人勿禁,著為永制。辛卯,上幸壽安宮,諫議大夫張暐等上疏請止其行,不允。癸巳,諭有司:「自今女直字直譯為漢字,國史院專寫契丹字者罷之。」甲午,改封永
 中為并王,永功為魯王,永成袞王,永升曹王,永蹈鄭王,永濟韓王,永德豳王。戊戌,增太學博士助教員。己亥,學士院新進唐杜甫、韓愈、劉禹錫、杜牧、賈島、王建,宋王禹備、歐陽修、王安石、蘇軾、張耒、秦觀等集二十六部。庚子,改壽安宮名萬寧。壬寅,如萬寧宮。詔襲封衍聖公孔元措視四品秩。



 五月庚戌,敕自今四日一奏事,仍免朝。戊辰,詔諸郡邑文宣王廟、風雨師、社稷神壇隳廢者,復之。詔御史臺令史並以終場舉人充。



 六月戊子,平章政事宗寧薨。癸巳,禁稱本朝人及朝言語為「蕃」,違者杖之。丙午,尚書右丞移剌履薨。秋七月丁巳,以參知政事徒
 單鎰為尚書右丞,御史中丞夾谷衡為參知政事。己未,觀稼于近郊。己巳,禁職官元日、生辰受所屬獻遺,仍為永制。以同僉大睦親府事袞等為賀宋生日使。庚午,諭有司:「自今外路公主應赴闕,其駙馬都尉非奉旨,毋擅離職。」八月癸未,至自萬寧宮。己亥,敕山東、河北闕食等處,許納粟補官。諭有司:「自今親王所領,如有軍處,令佐貳總押軍事。」乙巳,宋、高麗、夏遣使來賀天壽節。



 九月丁未朔,天壽節,以皇太后喪,不受朝。甲寅,如大房山。乙卯,謁奠裕陵。丙辰,還都。丁巳,以西上閣門使白琬為夏國生日使。己未,定詐為制書未施行制。以尚書左丞夾谷
 清臣為平章政事,封芮國公,參知政事完顏守貞為尚書左丞,知大興府事張萬公為參知政事。庚申,如秋山。冬十月己丑,至自秋山。甲午,敕司獄毋得與府州司縣官筵宴遠往,違者罪之。禁以太一混元受籙私建庵室者。壬寅,以河北、山東旱,應雜犯及強盜已未發覺減死一等,釋徒以下。



 十一月丙午朔,制諸女直人不得以姓氏釋為漢字。甲寅,禁伶人不得以歷代帝王為戲,及稱萬歲,犯者以不應為事重法科。丁巳,以豳王傅宗璧等為賀宋正旦使。戊午,夏人殺我邊將阿魯帶。甲子,制投匿名書者,徒四年。丙寅,以近侍局副使完顏匡為高麗生
 日使。壬申,敕提刑司官自今每十五日一朝。



 十二月乙亥朔,敕三品致仕官所得傔從毋令輸庸。己卯,定鎮邊守將致盜賊罪。甲申,獵于近郊。乙酉,詔罷契丹字。己丑,尚書右丞徒單鎰罷。癸卯,宋、高麗、夏遣使來賀正旦。



 三年春正月乙巳朔,以皇太后喪,不受朝。丙辰,以孝懿皇后小祥,尚書省請依明昌元年世宗忌辰例,諸王陪位,服慘紫,去金玉之飾,百官不視事,禁音樂屠宰,從之。壬戌,如春水。



 二月甲戌朔,敕猛安謀克許於冬月率所屬戶畋獵二次,每出不得過十日。壬辰,至自春水。丁酉,獵于近郊。辛丑,詔追復田等官爵。



 閏月甲子,以山東
 路統軍使烏林答愿為御史大夫。



 三月乙亥,更定強盜徵贓、品官及諸人親獲強盜官賞制。辛巳,初設左右衛副將軍。癸未,瀘溝石橋成。幸熙春園。丁亥,如萬寧宮。辛卯,詔賜棣州孝子劉瑜、錦州孝子劉慶祐絹、粟,旌其門閭,復其身。上因問宰臣曰:「從來孝義之人曾官使者幾何?」左丞守貞對曰:「世宗時有劉政者嘗官之,然若輩多淳質不及事。」上曰:「豈必盡然。孝義之人素行已備,稍可用既當用之,後雖有希覬作偽者,然偽為孝義,猶不失為善。可檢勘前後所申孝義之人,如有可用者,可具以聞。」癸巳,尚書省奏:「言事者謂,釋道之流不拜父母親屬,
 敗壞風俗,莫此為甚。禮官言唐開元二年敕云:『聞道士、女冠、僧、尼不拜二親,是為子而忘其生,傲親而徇於末。自今以後並聽拜父母,其有喪紀輕重及尊屬禮數,一准常儀。』臣等以為宜依典故行之。」制可。左丞守貞言:「上嘗命臣問忻州陳毅上書所言事,其一極論守令之弊,臣面問所以救之之道,竟不能言。」上曰:「方今政欲知其弊也。彼雖無救弊之術,但能言其弊,亦足嘉矣。如毅言及隨處有司不能奉行條制,為人傭雇尚須出力,況食國家祿而乃如是,得無虧臣子之行乎?其令檢會前後所降條理舉行之。」是日,溫王玠薨。丁酉,命有司祈雨,望
 祀嶽鎮海瀆於北郊。



 四月壬寅朔,定宣聖廟春秋釋奠,三獻官以祭酒、司業、博士充,祝詞稱「皇帝謹遣」,及登歌改用太常樂工。其獻官并執事與享者並法服,陪位學官公服,學生儒服。尚書省奏:「提刑司察與涿州進士劉器博、博州進士張安行、河中府胡光謙,光謙年雖八十三,尚可任用。」敕劉器博、張安行特賜同進士出身,胡光謙召赴闕。甲辰,祈雨于社稷。丙午,罷天山北界外採銅。戊申,瀛王環薨。戊午,詔集百官議北邊開壕事。詔賜雲內孝子孟興絹十匹、粟二十石,賜同州貞婦師氏謚曰「節」。丙寅,以旱災,下詔責躬。丁卯,復以祈雨,望祀嶽鎮海瀆
 山川于北郊。戊辰,敕親王衣領用銀褐紫綠。遣御史中丞吳鼎樞等審決中都冤獄,外路委提刑司處決。左丞守貞以旱,上表乞解職,不允。參知政事衡、萬公皆入謝。上曰:「前詔所謂罷不急之役、省無名之費、議冗官、決滯獄四事,其速行之。」



 五月壬申朔,以尚書禮部員外郎孛術魯子元為橫賜高麗使。癸酉,罷北邊開壕之役。甲戌,祈雨于社稷。是日,雨。戊寅,出宮女百八十三人。尚書省奏:「近以山東、河北之饑,已委宣差所至安撫賑濟。」復遣右三部司正范文淵往視之。乙酉,以雨足,致祭社稷。戊子,百官賀雨足。尚書左丞完顏守貞罷。己丑,以雨足,
 望祀嶽鎮海瀆。



 六月癸卯,宰臣請罷提刑司,上曰:「諸路提刑司官止三十餘員,猶患不得其人,州郡三百餘處,其能盡得人乎?」弗許。甲寅,以久雨,命有司祈晴。丁巳,定提刑司條制。辛酉,詔定內外所司公事故疑申呈罪罰格。乙丑,以知大名府事劉瑋為尚書右丞。有司言:「河州災傷,民乏食,而租稅有未輸。」詔免之。諭戶部:「可預給百官冬季俸,令就倉以時直糶與貧民,秋成各以其貲糴之,其所得必多矣,而上下便之。其承應人不願者,聽。」秋七月戊寅,敕尚書省曰:「飢民如至遼東,恐難遽得食,必有飢死者。其令散糧官問其所欲居止,給以文書,命
 隨處官長計口分散,令富者出粟養之,限以兩月,其粟充秋稅之數。」己卯,祁州刺史頓長壽、安武軍節度副使胡剌坐賑濟不及四縣,各杖五十。癸未,詔增北邊軍千二百人,分置諸堡。丁亥,胡光謙至闕,命學士院以雜文試之,稱旨。上曰:「朕欲親問之。」辛卯,以殿前都點檢僕散端等為賀宋生日使。己亥,上謂宰臣曰:「聞諸王傅尉多苛細,舉動拘防,亦非朕意。是職之設,本欲輔導諸王,使歸之正,得其大體而已。」平章政事清臣曰:「請以聖意遍行之。」曰:「已諭之矣。」



 八月癸卯,敕諸職官老病不肯辭避,有司諭使休閑者,不在給俸之列,格前勿論。上以軍民
 不和、吏員姦弊,詔四品以下、六品以上集議於尚書省,各述所見以聞。甲辰,集三品以下、六品以上官,問以朝政得失及民間利害,令各書所對。丁未,以有司奏寧海州文登縣王震孝行,以嘗業進士,并試其文,特賜同進士出身,仍注教授一等職任。辛亥,至自萬寧宮。特賜胡光謙明昌二年進士第三甲及第,授將仕郎、太常寺奏禮郎。官制舊設是職,未嘗除人,以光謙德行才能,故特授之。己未,以烏林答愿為尚書左丞。辛酉,獵于近郊。乙丑,上謂宰臣曰:「朕欲任官,令久於其事。若今日作禮官,明日司錢穀,雖間有異材,然事能悉辦者鮮矣!」對曰:「使
 中材之人久於其職,事既熟,終亦得力。」上問太常卿張暐:「古有三恪,今何無之?」暐具典故以聞。丁亥,宋、高麗、夏遣使來賀天壽節。



 九月庚午朔,天壽節,以皇太后喪,不受朝。諭尚書省:「去歲山東、河北被災傷處所閣租稅及借貸錢粟,若便徵之,恐貧民未蘇,俟豐收日以分數帶徵可也。」又諭宰臣曰:「隨路提刑司舊止察老病不任職及不堪親民者,如得其實,即改除他路。若他路提刑司覆察得實,勿復注親民之職。卿等其議行之。」甲戌,以郊社署令唐括合達為夏國生日使。己卯,如秋山。免圍場經過人戶今歲夏秋租稅之半,曾當差役者復一年。冬
 十月壬寅,至自秋山。丙午,敕御史臺,提刑司自今保申廉能官,勿復有乞升品語。壬子,有司奏增修曲阜宣聖廟畢,敕:「黨懷英撰碑文,朕將親行釋奠之禮,其檢討典故以聞。」甲寅,敕置常平倉處,並令州、府官以本職提舉,縣官兼管勾其事,以所糴多寡約量升降,以為永制。賜河南路提刑司所舉逸民游總同進士出身,以年老不樂仕進,授登仕郎,給正八品半俸終身。戊午,諭尚書省訪求博物多知之士。癸亥,遣諭諸王府傅尉曰:「朕分命諸王出鎮,蓋欲政事之暇,安便優逸,有以自適耳。然慮其舉措之間或違於理,所以分置傅尉,使勸導彌縫,不
 入於過失而已。若公餘遊宴不至過度,亦復何害。今聞爾等或用意太過,凡王門細碎之事無妨公道者,一一干與,贊助之道,豈當如是?宜各思職分,事舉其中,無失禮體。仍就諭諸王,使知朕意。」丙寅,敕應保舉官及試中書判者委官覆察,言行相副者量與升除,隨朝及六品以上各隨所長用之。己巳,獵于近郊。



 十一月庚午朔,尚書省奏:「翰林侍講學士黨懷英舉孔子四十八代孫端甫,年德俱高,該通古學。濟南府舉魏汝翼有文章德誼,苦學三十餘年,已四舉終場。蔚州舉劉震亨學行俱優,嘗充舉首。益都府舉王樞博學善書,事親至孝。」敕魏汝
 翼特賜進士及第,劉震亨等同進士出身,並附王澤榜。孔端甫俟春暖召之。丙子,詔臣庶名犯古帝王而姓復同者禁之,周公、孔子之名亦令回避。戊寅,升相州為彰德府。以前右副都點檢溫敦忠等為賀宋正旦使。壬午,尚書省奏:「知河南府事程嶧乞進封父祖。」權尚書禮部郎中黨懷英言:「凡宰執改除外任長官,其佐官以下相見禮儀皆與他長官不同,其子亦得試補省令史。其子且爾,父祖封贈理當不同,合舉宰執一例封贈。」從之。甲申,改提刑司令為書史。丙申,以有司言:「河州定羌民張顯孝友力田,焚券已責,又獻粟千石以賑饑。棣州民
 榮楫賑米七百石、錢三百貫,冬月散柴薪三千束。皆別無希覬。」特各補兩官,仍正班敘。



 十二月癸卯,以東上閣門使張汝猷為高麗生日使。辛亥,諭有司祈雪。癸丑,獵於近郊。丙辰,有赤氣見於北方。丁巳,敕華州下邽縣置武定鎮倉,京兆櫟陽縣置粟邑鎮倉,許州舞陽縣置北舞渡倉,各設倉草都監一人,縣官兼領之。乙丑,定到任告致仕格。丁卯,宋、高麗、夏遣使來賀正旦。



\end{pinyinscope}