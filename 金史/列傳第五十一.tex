\article{列傳第五十一}

\begin{pinyinscope}

 ○完顏賽不白撒一名承裔赤盞合喜



 完顏賽不,始祖弟保活里之後也。狀貌魁偉,沉厚有大略。初補親衛軍,章宗時,選充護衛。明昌元年八月,由宿直將軍為寧化州刺史。未幾,遷武衛軍副都指揮使。泰和二年,轉胡里改路節度使。四年,升武衛軍都指揮使,尋為殿前左副都點檢。及平章僕散揆伐宋,為右翼都統。六年六月,宋將皇甫斌遣率步騎數萬由確山、褒信
 分路侵蔡,聞郭倬、李爽之敗,阻溱水不敢進。於是,揆遣賽不及副統尚廄局使蒲鮮萬奴、深州刺史完顏達吉不等以騎七千往擊之。會溱水漲,宋兵扼橋以拒,賽不等謀潛師夜出,達吉不以騎涉水出其右,萬奴等出其左。賽不度其軍畢渡,乃率副統阿魯帶以精兵直趨橋,宋兵不能遏,比明大潰。萬奴以兵斷真陽路,諸軍追擊至陳澤,斬首二萬級,獲戰馬雜畜千餘。兵還,進爵一級,賜金幣甚厚。



 貞祐初,拜同簽樞密院事。三年,遷知臨洮府事,兼陜西路副統軍。上召見諭曰:「卿向在西京,盡心為國,及治華州,亦嘗宣力,今始及三品。特升授汝此職
 者,以陜西安撫副使烏古論兗州不遵安撫使達吉不節制,多致敗事。今已責罰兗州,命卿副之。宜益務盡心,其或不然,復當別議行之。」八月,知鳳翔府事,兼本路兵馬都總管,俄為元帥右都監。四年四月,調兵拔宋木陡關。五月,夏人於來羌城界河修折橋,以兵守護,賽不遣兵焚之。八月,夏人寇結耶觜川,遣兵擊走之,尋又破其眾於車兒堡。



 興定元年二月,轉簽樞密院事。時上以宋歲幣不至,且復侵盜,詔賽不討之。四月,與宋人戰於信陽,斬首八千,生擒統制周光,獲馬數千、牛羊五百。又遇宋人於隴山、七里山等處,前後六戰,斬獲甚眾。尋遣兵
 渡淮,略中渡店,拔光山、羅山、定城等縣,破光州兩關,斬首萬餘,獲馬牛及布,分給將士。詔賜玉兔鶻一、內府重幣十端。



 七月,上章言:「京都天下之根本,其城池宜極高深,今外城雖堅,然周六十餘里,倉猝有警難於拒守。竊見城中有子城故基,宜於農隙築而新之,為國家久長之利。及凡河南、陜西州府,皆乞量修。」從之。



 二年正月,破宋人於鐵山及上石店、唐縣。四月,進兼西南等路招討使、西安軍節度使、陜州管內觀察使。奉詔攻棗陽,宋出兵三萬拒戰,稍誘擊之,宋兵敗走城,薄諸濠,殺及溺死者三千餘人,遂進兵圍之。宋騎兵千、步卒萬來援,逆戰
 復大敗之。七月,遷行山東西路兵馬都總管,兼武寧軍節度使。三年二月,奪宋白石關,殺其守者千餘人,獲鎧仗千計。三月,破宋兵於七口倉,又奪宋小鶻倉,獲糧九千石、兵仗三十餘萬。是月,復敗宋兵三千于石鶻崖。



 四年三月,奉詔出兵河北招降,晉安權府事皇甫珪、正平縣令席永堅率五千餘人來歸,得糧萬石。時河北所在義軍官民堅守堡寨,力戰破敵者眾。賽不上章言:「此類忠赤可嘉,若不旌酬無以激人心。乞朝廷量加官賞,萬一敵兵復來,將爭先效用矣。」上覽奏,召樞密官曰:「朕與卿等亦嘗有此議,以不見彼中事勢,故一聽帥臣規畫。
 今觀此奏,甚稱朕意,其令有司遷賞之。」是年四月,遷樞密副使。



 五年五月,奉詔引兵救河東,戰屢捷,復晉安、平陽二城。監察御史言其不能檢束士眾,縱之虜略,請正其罪。上以有功,詔勿問。元光二年五月,復河中。六月,詔諭宰臣曰:「樞密副使賽不本皇族,先世偶然脫遺。朕重其舊人,且久勞王家,已命睦親府附於屬籍矣。卿等宜知之。」正大元年五月,拜平章政事。未幾,轉尚書右丞相。雅與參知政事李蹊相得,及蹊以公罪出尹京洛,賽不數薦蹊,比唐魏徵,以故蹊得復相。三年,宣宗廟成,將禘祭,議配享功臣,論者紛紜。賽不為大禮使,因言:「丞相福
 興死王事,七斤謹守河南以迎大駕,功宜配享。」議遂定。



 四年,吏部郎中楊居仁上封事,言宰相宜擇人,上語大臣曰:「相府非其人,御史諫官當言,彼吏曹,何與于此。」尚書左丞顏盞世魯素嫉居仁,亦以為僭,賽不徐進曰:「天下有道,庶人猶得獻言,況在郎官。陛下有寬弘之德,故不應言者猶言。使其言可用則行之,不可用不必示臣下也。」上是之。居仁字行之,大興人。泰和三年進士。天興末時北渡,舉家投黃河死。



 五年,行尚書省于京兆,謂都事商衡曰:「古來宰相必用文人,以其知為相之道。賽不何所知,使居此位,吾恐他日史官書之,某時以某為相
 而國乃亡。」即促衡草表乞致仕。平章政事侯摯朴直無蘊藉,朝廷鄙之,天興元年兵事急,自致仕起為大司農,未幾復致仕,徐州行尚書省無敢行者,復拜摯平章政事。都堂會議,摯以國勢不支,因論數事,曰:「只是更無擘劃。」白撒怒曰:「平章出此言,國家何望耶!」意在置之不測。賽不顧謂白撒曰:「侯相言甚當。」白撒遂含憤而罷。



 時大元兵薄汴,白撒策後日講和或出質,必首相當行,力請賽不領省事,拜為左丞相,尋復致仕。是年冬,哀宗遷歸德,起復為右丞相,樞密使,兼左副元帥,封壽國公,扈從以行。河北兵潰,從至歸德,又請致仕。二年七月,復詔行
 尚書省事於徐州。既至,以州乏糧,遣郎中王萬慶會徐、宿、靈璧兵取源州,令元帥郭恩統之。九月,恩至源州城下,敗績而還。再命卓翼攻豐縣,破之。初,郭恩以敗為恥,託疾不行,乃密與河北諸叛將郭野驢輩謀歸國用安,執元帥商瑀父子、元帥左都監紇石烈善住,併殺之。又逐都尉斡轉留奴、泥龐古桓端、蒲察世謀、元帥右都監李居仁、員外郎常忠。自是,防城與守門者皆河北義軍,出入自恣。賽不先病疽,久不視事,重為賊黨所制,束手聽命而已。



 初,源、徐交攻,郭野驢者每辭疾不行,賽不遂授野驢徐州節度副使,兼防城都總領,實羈之也。野驢
 既見徐州空虛,乃約源州叛將麻琮內外相應。十月甲申,詰旦,襲破徐州。時蔡已被圍,徐州將士以朝命阻絕,且逼大兵,議出降。賽不弗從,恐被執,至是投河求死,流三十餘步不沒,軍士援出之。又五日,自縊于州第。麻琮乃遣人以州降大元。



 子按春,正大中充護衛,坐與宗室女姦,杖一百收係。居許州,大兵至許,按春開南門以降。從攻京師,曹王出質,朝臣及近衛有從出者,按春極口大罵,以至指斥。是冬,復自北中逃迴,詔令押入省,問事情,按春隨近侍登階作揮涕之狀。詔問丞相云:「按春自北中來,丞相好與問彼中息耗。」賽不附奏曰:「老臣不幸
 生此賊,事至今日,恨不手刃之,忍與對面語乎!」十二月,車駕東狩,留後二相下開封,擒捕斬之獄中。



 贊曰:賽不臨陣對壘既有將略,洎秉鈞衡,觀其救解楊居仁、侯摯等言,殊有相度,按春之事尤有古人之風焉。晚以老病,受制叛臣,致修匹夫匹婦之節,此猶大廈將傾,非一木之所能支也,悲夫!



 內族白撒,名承裔,末帝承麟之兄也,系出世祖諸孫。自幼為奉御。貞祐間,累官知臨洮府事、兼本路兵馬都總管。興定元年,為元帥左都監,行帥府事於鳳翔。是年,詔陜西行省伐宋,白撒出鞏州鹽川,遇宋兵於皂郊堡,敗
 之。又遇宋兵于天水軍,掩擊,宋兵大潰。二年四月,復敗宋兵,至雞公山,遂拔西和州,毀其諸隘營屯。遣合扎都統完顏習涅阿不率軍趨成州,宋帥羅參政、統制李大亨焚廬舍棄城遁,留千餘人城守,督兵赴之,逐克焉,獲糧七萬斛,錢數千萬。河池縣守將楊九鼎亦焚縣舍走保清野原。統制高千據黑谷關甚固,遣兵襲之,千遁去,獲糧二萬斛,器械稱是,因夷其險而還。三年,破虎頭關,敗宋兵于七盤子、雞冠關。褒城縣官民自焚城宇遁,因取其城。興元府提刑兼知府事趙希昔聞兵將至,率官民遁,於是白撒遂取興元,以駐兵焉。命提控張秀華馳
 視洋州,官民亦遁,又取其城。尋聞漢江之南三十里,宋兵二千據山而陣,遣提控唐括移失不擊走之。行省以捷聞,宣宗大悅,進白撒官一階。時朝議以蘭州當西夏之衝,久為敵據,將遣白撒復之,白撒奏曰:「臣近入宋境,略河池,下鳳州,破興元,抵洋州而還。經涉險阻數千里,士馬疲弊,未得少休,而欲重為是舉,甚非計也,不若息兵養士以備。」從之。



 未幾,權參知政事,行省事于平涼。四年,上言:「宋境山州宕昌東上拶一帶蕃族,昔嘗歸附,分處德順、鎮戎之間。其後有司不能存撫,相繼亡去。近聞復有歸心,然不招之亦無由自至。誠得其眾,可以助兵,
 寧謐一方。臣以同知通遠軍節度使事烏古論長壽及通遠軍節度副使溫敦永昌皆本蕃屬,且久鎮邊鄙,深得彼心,已命遣人招之。其所遣及諸來歸者,皆當甄獎,請預定賞格以待之。」上是其言。



 是年,夏兵三萬由高峰嶺入寇定西州,環城為柵,白撒遣刺史愛申阿失剌與行軍提控烏古論長壽、溫敦永昌出戰,大敗之,斬首千餘,獲馬仗甚眾。五年五月,白撒言:「近詔臣遣官諭諸蕃族以討西夏,臣即令臨洮路總管女奚烈古里間計約喬家丙令族首領以諭餘族。又別遣權左右司都事趙梅委差官遙授合河縣尉劉貞同往撫諭。未幾,梅、貞報
 溪哥城等處諸族,與先降族共願助兵七萬八千餘人,本國蕃族願助兵九千,若更以官軍繼為聲援,勝夏必矣。臣已令古里間將鞏州兵三萬,宜更擇勇略之臣副之。梅、貞等既悉事勢,當假以軍前之職。蕃僧納林心波亦招誘有功,乞遷官授職以獎勵之。」上皆從其請。



 元光元年二月,行省上言:「近與延安元帥完顏合達、納合買住議:河北郡縣俱已殘毀,陜西、河南亦經抄掠。比者西北二敵併攻鄜延,城邑隨陷,惟延安孤墉僅得保全。若今秋復至,必長驅而深入,雖京兆、鳳翔、慶陽、平涼已各益軍,而率皆步卒,且相去闊遠,卒難應援,倘關中諸鎮
 不支,則河南亦不安矣。今二敵遠去,西北少休,宜乘此隙徑取蜀、漢、實國家基業萬全之計。」詔樞密議之。



 先是,夏兵數十萬分寇龕谷、鄜延、大通諸城,上召白撒等授以方略,命發兵襲其浮橋,遂趨西涼。別遣將取大通城,出溪哥路,略夏地。白撒徐出鎮戎,合達出環州,以報三道之役。白撒馳至臨洮,遣總管女奚烈古里間、積石州刺史徒單牙武各攝帥職,率兵西入,遇夏兵千餘於踏南寺,擊走之。夏人據大通城,因圍之,分兵奪其橋,與守兵七千人戰,大敗之,幾殺其半,入河死者不可計,餘兵焚其橋西遁。乃還軍攻大通,克之,斬首三千,因招來諸
 寺族被脅僧俗人,皆按堵如故。以河梁既焚,塞外地寒少草,師遂還。



 十二月,行省言:「近有人自北來者,稱國王木華里悉兵沿渭而西,謀攻鳳翔,鳳翔既下乃圖京兆,京兆卒不可得,留兵守之,至春蹂踐二麥以困我。未幾,大兵果圍鳳翔,帥府遣人告急。臣以為二鎮脣齒也,鳳翔蹉跌,則京兆必危,而陜右大震矣。然平川廣野實騎兵馳騁之地,未可與之爭鋒。已遣提控羅桓將兵二千,循南山而進,伺隙攻其柵壘,以紓城圍。更乞發河南步騎以備潼關。」詔付尚書省樞密院議之。



 二年冬,哀宗即位,邊事益急。正大五年八月,召白撒還朝,拜尚書右丞,
 未幾,拜平章政事。白撒居西垂幾十年,當宋、夏之交,雖頗立微效,皆出諸將之力。然本恇怯無能,徒以儀體為事,性愎貪鄙,及入為相,專愎尤甚。嘗惡堂食不適口,每以家膳自隨,國家顛覆,初不恤也。



 九年正月,諸軍敗績於三峰山。大兵與白坡兵合,長驅趨汴。令史楊居仁請乘其遠至擊之,白撒不從,且陰怒之。遂遣完顏麻斤出、邵公茂等部民萬人,開短堤,決河水,以固京城。功未畢而騎兵奄至,麻斤出等皆被害,丁壯無二三百人得反者。壬辰,棄衛州,運守具入京。初,大兵破衛州,宣宗南遷,移州治於宜村渡,築新城於河北岸,去河不數步,惟北面
 受敵,而以石包之,歲屯重兵於此,大兵屢至不能近。至是,棄之,隨為大兵所據。



 甲午,修京城樓櫓。初,宣宗以京城闊遠難守,詔高琪築里城,公私力盡僅乃得成。至是,議所守。朝臣有言裏城決不可守,外城決不可棄。大兵先得外城,糧盡救絕,走一人不出。裏城或不測可用,於是決計守外城。時在城諸軍不滿四萬,京城周百二十里,人守一乳口尚不能遍,故議避遷之民充軍。又召在京軍官於上清宮,平日防城得功者如內族按出虎、大和兒、劉伯綱等皆隨召而出,截長補短假借而用,得百餘人。又集京東西沿河舊屯兩都尉及衛州已起義軍,
 通建威得四萬人,益以丁壯六萬,分置四城。每面別選一千,名「飛虎軍」,以專救應,然亦不能軍矣。



 三月,京城被攻,大臣分守四面。白撒主西南,受攻最急,樓櫓垂就輒摧,傳令取竹為護簾,所司馳入城大索,竟無所得,白撒怒欲斬之。員外郎張袞附所司耳語曰:「金多則濟矣,胡不即平章府求之。」所司懷金三百兩徑往,賂其家僮,果得之。



 已而兵退,朝廷議罷白撒,白撒不自安,乃謂令令史元好問曰:「我妨賢路久矣,得退是幸,為我撰乞致仕表。」頃之,上已遣使持招至其第,令致仕。既廢,軍士恨其不戰誤國,揚言欲殺之。白撒懼,一夕數遷,上以親軍二百
 陰為之衛。軍士無以泄其憤,遂相率毀其別墅而去。其黨元帥完顏斜捻阿不領本部軍戍汴,聞之徑詣其所,斬經其垣下者一人以鎮之。



 是時,速不泬等兵散屯河南,汴城糧且盡,累召援兵復無至者。冬十月,乃復起白撒為平章政事、權樞密使、兼右副元帥。於是,群臣為上畫出京計,以賽不為右丞相、樞密使、兼左副元帥,內族訛出右副元帥、兼樞密副使、權參知政事,李蹊兵部尚書、權尚書左丞,徒單百家元帥左監軍、行總帥府事。東面元帥高顯,副以果毅都尉粘合咬住兵五千。南面元帥完顏豬兒,副以建威都尉完顏斡論出兵五千。西面元
 帥劉益、上黨公張開,副以安平都尉紀綱軍五千。北面元帥內族婁室,副以振威都尉張閏軍五千。中翼都尉賀都善軍四千,隸總帥百家。都尉內族久住,副都尉王簡、總領王福胤神臂軍三千五百,左翼元帥內族小婁室親衛軍一千,右翼元帥完顏按出虎親衛軍一千,總領完顏長樂、副帥溫敦昌孫馬軍三百,郡王王義深馬軍一百五十,郡王范成進、總領蘇元孫圭軍三千,隸總帥百家。飛騎都尉兼合里合總領術虎只魯歡、總領夾谷得伯、颭軍田眾家奴等百人及諸臣下,發京師。



 十二月甲辰,車駕至黃陵岡,白撒先降大兵兩寨,得河朔
 降將,上赦之,授以印及金虎符。群臣議以河朔諸將前導,鼓行入開州,取大名、東平,豪傑當有響應者,破竹之勢成矣。溫敦昌孫曰:「太后、中宮皆在南京,北行萬一不如意,聖主孤身欲何所為?若往歸德,更五六月不能還京。不如先取衛州,還京為便。」白撒奏曰:「聖體不便鞍馬,且不可令大兵知上所在,今可駐歸德。臣等率降將往東平,俟諸軍至,可一鼓而下,因而經略河朔,且空河南之軍。」上以為然。時上已遣官奴將三百騎探漚麻岡未還,上將御船,賜白撒劍,得便宜從事,決東平之策。官奴還奏衛州,有糧可取?上召白撒問之,白撒曰:「京師且不
 能守,就得衛州,欲何為耶?以臣觀之,東平之策為便。」上主官奴之議。



 明年正月朔,次黃陵岡。是日,歸德守臣以糧糗三百餘船來餉,遂就其舟以濟南岸,未濟者萬人,大元將回古乃率四千騎追擊之,賀都喜揮一黃旗督戰,身中十六七箭,軍殊死鬥,得卒十餘人,大兵少卻。上遣送酒百壺勞之。須臾,北風大作,舟皆吹著南岸,諸兵復擊之,溺死者近千人,元帥豬兒、都尉紇石烈訛論等死之。建威都尉完顏訛論出降於大元。上於北岸望之震懼,率從官為豬兒等設祭,哭之,皆贈官,錄用其子姪,斬訛論出二弟以徇。



 遂命白撒攻衛州。上駐兵河上,留
 親衛軍三千護從,都尉高顯步軍一萬,元帥官奴忠孝軍一千,郡王范成進、王義深、上黨公張開、元帥劉益等軍總帥百家總之,各齎十日糧,聽承裔節制。發自蒲城,上時已遣賽不將馬軍北向矣,白撒以三十騎追及,謂賽不曰:「有旨,命我將馬軍。」賽不謂上曰:「北行議已決,不可中變。」上曰:「丞相當與平章和同。」完顏仲德持御馬銜苦諫曰:「存亡在此一舉,衛州決不可攻。」上麾之曰:「參政不知。」白撒遂攻衛州,兵至城下,御旗黃傘招之不下。其夜,北騎三千奄至,官奴、和速嘉兀地不、按出虎與之戰,北兵卻六十里。然自發蒲城,遷延八日始至衛,而猝無
 攻具,縛槍為雲梯。州人知不能攻,守益嚴。凡攻三日不克。及聞河南大兵濟自張家渡,至衛西南,遂班師。大兵踵其後,戰於白公廟,敗績,白撒等棄軍遁,劉益、張開皆為民家所殺。車駕還次蒲城東三十里,白撒使人密奏劉益一軍叛去。點檢抹捻兀典、總領溫敦昌孫時侍行帳中,請上登舟,上曰:「正當決戰,何遽退乎?」少頃,白撒至,倉皇言於上曰:「今軍已潰,大兵近在堤外,請聖主幸歸德。」上遂登舟,侍衛皆不知,巡警如故。時夜已四更矣,遂狼狽入歸德。



 白撒收潰兵大橋,得二萬餘人,懼不敢入。上聞,遣近侍局提點移剌粘古、紇石烈阿里合、護衛二
 人以舟往迎之。既至,不聽入見,并其子下獄。諸都尉司軍以白撒不戰而退,發憤出怨言。上乃暴其罪曰:」惟汝將士,明聽朕言:我初提大軍次黃陵岡得捷,白撒即奏宜渡河取衛州,可得糧十萬石,乘勝恢復河北。我從其計,令率諸軍攻衛。去蒲城二百餘里,白撒遷延八日方至,又不預備攻具,以致敗衄。白撒棄軍竄還蒲城,便言諸軍已潰,北兵勢大不可當,信從登舟,幾死于水。若當時知諸軍未嘗潰,只河北戰死,亦可垂名於後。今白撒已下獄,不復錄用,籍其家產以賜汝眾,其盡力國家,無效此人。」囚白撒七日而餓死,發其弟承麟、子狗兒徐州
 安置。當時議者,衛州之舉本自官奴,歸之白撒則亦過矣。



 初,瀕河居民聞官軍北渡,築坦塞戶,潛伏洞穴,及見官奴一軍號令明肅,撫勞周悉,所過無絲髮之犯,老幼婦子坦然相視,無復畏避。俄白撒輩縱軍四出,剽掠俘虜,挑掘焚炙,靡所不至。哭聲相接,屍骸盈野。都尉高祿謙、苗用秀輩仍掠人食之,而白撒誅斬在口,所過官吏殘虐不勝,一飯之費有數十金不能給者,公私皇皇,日皆徯大兵至矣。



 白撒目不知書,姦黠有餘,簿書政事,聞之即解,善談議,多知,接人則煦煦然,好貨殖,能捭闔中人主心,遂浸漬以取將相。既富貴,起第於汴之西城,規
 模擬宮掖,婢妾百數,皆衣金縷,奴隸月廩與列將等,猶以為未足也。上嘗遣中使責之曰:「卿汲汲於此,將無北歸意耶?」白撒終不悛,以及於禍。



 贊曰:白撒本非將才,恇怯誤國,徒能阿合以取富貴,性愎貪鄙,當此危亡,方謀封殖以自逸,此猶大廈將焚而燕雀不悟者歟!



 赤盞合喜,性剛愎,好自用,朝廷以其有才幹任之。宣宗時,累遷蘭州刺史、提控軍馬。貞祐四年十一月,夏人四萬餘騎圍定西,輦致攻具,將取其城。合喜及楊斡烈等率兵鏖戰走之,斬首二千級,俘數十人,獲馬八百餘匹,
 器械稱是,餘悉遁去。興定元年正月,以屢敗夏人,遙授同知臨洮府事,兼前職。是冬,陜西行省奉詔伐宋,合喜權行元帥府,駐來遠寨以張聲勢,既而獲捷。二年四月,宋兵數千侵臨洮,合喜擊走之,斬獲甚眾。三年四月,遷元帥左都監,行元帥府事於鞏州。



 四年四月,夏人犯邊,合喜討之,師次鹿兒原,遇復兵千人,遣提控烏古論世顯率偏師敗之,都統王定亦破其眾一千五百於新泉城。九月,夏人攻鞏州,合喜遣兵擊之,一日十餘戰,夏人退據南閑,遣精兵三萬傅城,又擊走之,生擒夏將劉打、甲玉等。訊知夏大將你思丁、兀名二人謀,以為鞏帥府
 所在,鞏既下則臨洮、積石、河、洮諸城不攻自破,故先及鞏,且構宋統制程信等將兵四萬來攻。合喜聞之,飭兵嚴備。俄而兵果至,合喜督兵搏戰,卻之,殺數千人。攻益急,將士殊死戰,殺傷者以萬計。夏人焚其攻具,拔柵而去。合喜已先伏甲要地邀之,復率眾躡其後,斬首甚眾。十月,以功遙授平西軍節度使。



 元光元年,大將萌古不花攻鳳翔,朝廷以主將完顏仲元孤軍不足守禦,命合喜將兵援之。二年二月,木華黎國王、斜里吉不花等及夏人步騎數十萬圍鳳翔,東自扶風、岐山,西連汧、隴,數百里間皆其營柵,攻城甚急,合喜盡力,僅能禦之。於是,
 合喜以同知臨洮府事顏盞蝦蟆戰尤力,遂以便宜升為通遠軍節度使,上嘉其功,許之。是歲,升簽樞密院事。哀宗即位,拜參知政事,權樞密副使。



 正大八年十一月,鄧州馳報大元兵破嶢峰關,由金州東下。報至時日已暮,省院官入奏,上曰:「事至於此,奈何?」上即位至是八年,從在東宮日立十三都尉,每尉不下萬人,彊壯矯捷,極為精練。步卒負擔器甲糧糗重至六七斗,一日夜行二百里。忠孝軍萬八千人,皆回紇、河西及中州人夜掠而逃歸者,人有從馬,以騎射選之乃得補。親衛、騎兵、武衛、護衛,遷外諸軍又二十餘萬。故頻年有大昌原、倒回谷
 之捷,士氣既振,遂有一戰之資。至是,院官同奏:「北軍冒萬里之險,歷二年之久,方入武休,其勞苦已極。為吾計者,以兵屯睢、鄭、昌武、歸德及京畿諸縣,以大將守洛陽、潼關、懷、孟等處,嚴兵備之。京師積糧數百萬斛,令河南州郡堅壁清野,百姓不能入城者聚保山砦。彼深入之師,欲攻不能,欲戰不得,師老食盡,不擊自歸矣。」上太息曰:「南渡二十年,所在之民破田宅、鬻妻子以養軍士。且諸軍無慮二十餘萬,今敵至不能迎戰,徒以自保,京城雖存,何以為國,天下其謂我何!」又曰:「存亡天命,惟不負民可也。」乃詔合達、蒲阿等屯軍襄、鄧。



 九年正月,兩省
 軍潰于三峰山,北兵進薄京師。三月庚子,議曹王出質。大兵北行,留速不泬攻城,攻具已辦,既有納質之請,即又云:「有我受命攻城,但曹王出則退,不然不罷也。」壬寅,曹王入辭,宴於宮中。癸卯,北兵立攻具,沿壕列木柵,以薪草填壕,頃刻平十餘步。主兵者以議和之故不敢與戰,但於城上坐視而已。



 城中喧哄,上聞之,從六七騎出端門至舟橋。時新雨淖,車駕忽出,人驚愕失措,但跪於道傍,亦有望而拜者,上自麾之曰:「勿拜,恐泥污汝衣。」倉皇中,市肆米豆狼藉於地,上敕衛士令各歸其家,老幼遮擁至有誤觸御衣者。少頃,宰相從官皆至,進笠不受,曰:「
 軍士暴露,我何用此為。」所過慰勞軍士,皆踴躍稱萬歲,臣等戰死無所恨,至有感泣者。西南軍士五六十輩聚而若有言者,上就問之,跪曰:「大兵芻土填壕,功已過半,平章傳令勿放一鏃,恐壞和事,想豈有計耶?」上顧謂其中長者云:「朕為生靈,稱臣進奉無不從順,止有一子,養來成長,今往作質子矣。汝等略忍,待曹王出,大兵不退,汝等死戰未晚。」復有拜泣者曰:「事急矣,聖主毋望和事。」乃傳旨城上放箭。西水門千戶劉壽控御馬仰視曰:「聖主無信賊臣,賊臣盡,大兵退矣。」衛士欲擊之,上止曰:「醉矣,勿問。」是日,曹王出詣軍前,大兵併力進攻。甲辰,上
 復出撫東門將士,太學生楊奐等前白事。上問何所欲言,曰:「臣等皆太學生,令執砲夫之役,恐非國家百年以來待士之意。」敕記姓名,即免其役。過南薰門,值被創者,親傅以藥,手酌卮酒以賜,且出內府金帛以待有功者。是日,大兵驅漢俘及婦女老幼負薪草填壕塹,城上箭鏃四下如雨,頃刻壕為之平。



 龍德宮造砲石,取宋太湖、靈璧假山為之,小大各有斤重,其圓如燈球之狀,有不如度者杖其工人。大兵用砲則不然,破大磑或碌碡為二三,皆用之。攢竹砲有至十三稍者,餘砲稱是。每城一角置砲百餘枝,更遞下上,晝夜不息,不數日,石幾與裏
 城平。而城上樓櫓皆故宮及芳華、玉谿所拆大木為之,合抱之木,隨擊而碎,以馬糞麥秸布其上,綱索旃褥固護之。其懸風板之外皆以牛皮為障,遂謂不可近。大兵以火炮擊之,隨即延爇不可撲救。父老所傳周世宗築京城,取虎牢土為之,堅密如鐵,受砲所擊唯凹而已。大兵壕外築城圍百五十里,城有乳口樓櫓,壕深丈許,闊亦如之,約三四十步置一鋪,鋪置百許人守之。



 初,白撒命築門外短牆,委曲狹隘容二三人得過,以防大兵奪門。及被攻,諸將請乘夜斫營,軍乃不能猝出,比出,已為北兵所覺。後又夜募死士千人,穴城由壕徑渡,燒其砲
 坐。城上懸紅紙燈為應,約燈起渡壕,又為圍者所覺。又放紙鳶,置文書其上,至北營則斷之,以誘被俘者。識者謂前日紙燈、今日紙鳶,宰相以此退敵難矣。右丞世魯命作《江水曲》,使城上之人靜夜唱之,蓋河朔先有此曲以寄謳吟之思,其謬計如此。



 合喜先以守鳳翔自誇,及令守西北隅,其地受攻最急,而合喜當之,語言失措,面無人色。軍士特以車駕數出慰勞,人自激昂,爭為效命耳。其守城之具有火砲名「震天雷」者,鐵罐盛藥,以火點之,砲起火發,其聲如雷,聞百里外,所爇圍半畝之上,火點著甲鐵皆透。大兵又為牛皮洞,直至城下,掘城為龕,
 間可容人,則城上不可奈何矣。人有獻策者,以鐵繩懸「震天雷」者,順城而下,至掘處火發,人與牛皮皆碎迸無迹。又飛火槍,注藥以火發之,輒前燒十餘步,人亦不敢近。大兵惟畏此二物云。



 四月罷攻。至是十六晝夜矣,內外死者以百萬計,大兵知不可下,乃謾為好語云:「兩國已講和,更相攻耶?」朝廷亦就應之。明日,遣戶部侍郎楊居仁出宜秋門以酒炙犒師,於是營幕稍稍外遷,遂退兵。



 壬戌,合喜以大兵退,議入賀。諸相皆不欲,獨合喜以守城為己功,持論甚力,呼令史元好問曰:「罷攻已三日而不入賀,何也?速召翰苑官作表。」好問以白諸相,權參
 政內族思烈曰:「城下之盟,諸侯以為恥,況以罷攻為可賀歟?」合喜怒曰:「社稷不亡,帝后免難,汝等不以為喜耶?」明日,近侍局直長張天任至省,好問私以賀議告之,天任曰:「人不知恥乃若是耶!」因謂諸相曰:「京城受兵,上深以為辱。聞百官欲入賀,誠有此否?」會學士趙秉文不肯撰表,議遂寢。



 是月,以尚書省兼樞密院事,合喜罷樞密。合喜既失兵柄,意殊不樂,欲銷院印,諸相謂院事仍在,印有用時,不宜毀。合喜怒,欲笞其掾。有投匿名書於御路云:「副樞合喜、總帥撒合、參政訛出皆國賊,朝廷不殺,眾軍亦須殺之,為國除害。」衛士以聞。撒合飲藥死,訛出
 稱疾不出,惟合喜坦然若無事者,上亦無所問,由是軍國之事盡決于合喜矣。



 初,大兵圍汴,司諫陳岢屢上封事言得失,切中時病。合喜大怒,召入省,呼其名責之曰:「子為『陳山可』耶?果如子言,能退大敵,我當世世與若為奴。」聞者無不竊笑。蓋不識「岢」字,至分為兩耳。



 天興元年七月,權參知政事思烈、恒山公武仙合軍自汝州入援,詔以合喜為樞密使,統京城軍萬五千應之,且命賽不為之助。八月己酉朔,駐於近郊,候益兵乃進屯中牟古城。凡三日,聞思烈軍潰,即夜棄輜重馳還。黎明至鄭門,聚軍乃入。言者謂:「合喜始則抗命不出,中則逗遛不進,
 終則棄軍先遁,委棄軍資不可勝計,不斬之無以謝天下。」上貸其死,免為庶人,既而籍其家以賜軍士。



 既廢,居汴中,常鞅鞅不樂。會大將速不泬遣人招之,合喜即治裝欲行,崔立邀至省酌酒餞送,且以白金二百兩為贐。明日,復詣省別立,方對語,適一人自歸德持文書至,發視之,乃行省傳哀宗語以諭合喜者,其言曰:「卿朕老臣,中間雖廢出,未嘗忘卿。今崔立已變,卿處舊人尚多,若能反正,與卿世襲公相。」立怒,叱左右繫之獄,是日斬之。



 論曰:合喜初年用兵西夏,屢著勞效,要亦諸將顏盞蝦蟆等功也。既當大任,遂自矜伐,汴城之役,舉措煩擾,質
 出兵退,即圖稱賀,此豈有體國之誠心者乎。中牟之潰,眾怒所歸,幸逭一死,猶懷異圖,卒殞猜疑,天蓋假手於崔立也。



\end{pinyinscope}