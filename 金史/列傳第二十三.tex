\article{列傳第二十三}

\begin{pinyinscope}

 世宗諸子



 ○永中永蹈永功子璹永德永成永升



 世宗昭德皇后生顯宗、趙王孰輦、越王斜魯。元妃張氏生鄗王允中、越王允功。元妃李氏生鄭王允蹈、衛紹王允濟、潞王允德。昭儀梁氏生豫王允成。才人石抹氏生夔王允升。孰輦、斜魯皆早卒。



 鎬王永中,本名實魯剌,又名萬僧。大定元年,封許王。五年,判大興尹。七年,進封越王。十一年,進封趙王。十三年,拜樞密使。十九年,子石古乃加光祿大夫。是歲,改葬明德皇后于坤厚陵,永中母元妃張氏陪葬。十一月庚申,自磐寧宮發引。永中以元妃柩先發,使執黃傘者前導。俄頃,皇后柩出磐寧宮,顯宗徒跣。少府監張僅言呼執黃傘者,不應。既葬,僅言欲奏其事,顯宗解之曰:「是何足校哉,或傘人誤耳。」僅言乃止。



 二十一年,改判大宗正事。永中不悅,顯宗勸之曰:「宗正之職,自親及疏,自近及遠,此親賢之任也。且皇子之貴,豈以官職閑劇為計邪?」永
 中乃喜。二十四年,世宗幸上京,顯宗居守,并留永中。顯宗先遣章宗、宣宗奉表問起居于上京,既而遣永中子光祿大夫石古乃奉表。世宗喜謂豫國公主曰:「皇太子孝德天成,先遣二子,繼遣此子,兄弟之際相友愛如此也。」



 二十五年六月,世宗在天平山好水川清暑,顯宗薨于中都,詔曹王永功視章宗,召永中赴行在。是歲,與章宗及永功等並加開府儀同三司。二十六年,復為樞密使。是歲,世宗賜諸孫名。石古乃曰瑜,神土門日璋,阿思懣曰,阿離合懣曰彖。二十七年,年十五以上,加奉國上將軍。章宗即位,起復判西京留守,進封漢王,與諸
 弟各賜金五百兩、銀五千兩、錢二千貫、重幣三百端、絹二千匹。再賜永中修公廨錢三百萬,特加石古乃銀青榮祿大夫,阿離合懣奉國上將軍。



 明昌二年正月辛酉,孝懿皇后崩。判真定府事吳王永成、判定武軍節度使隋王永升奔喪後期,各罰俸一月,杖其長史五十。永中適有寒疾,不能至。上怒,頗意諸王有輕慢心,遣使責永中曰:「已近公除,亦不須來。」二月丙戌,禫祭,永中始至,入臨。辛卯,始克行燒飯禮。壬辰,永中及諸王朝辭,賜遺留物,禮遇雖在,而嫌忌自此始矣。



 四月,進封并王。三年,判平陽府事,進封鎬王。初置王傅、府尉官,名為官屬,實檢
 制之也。府尉希望風旨,過為苛細。永中自以世宗長子,且老矣,動有掣制,情思不堪,殊鬱鬱,乃表乞閑居。詔不許。四年,鄭王永蹈以謀逆誅。增置諸王司馬一員,檢察門戶出入,球獵游宴皆有制限,家人出入皆有禁防。河東提刑判官把里海坐私謁永中,杖一百,解職。前近侍局副使裴滿可孫嘗受永中請託,為石古乃求除官,可孫已改同知西京留守,猶坐免。故尚書右丞張汝弼,永中母舅也。汝弼妻高陀斡自大定間畫永中母像,奉之甚謹,挾左道為永中求福,希覬非望。明昌五年,高陀斡坐詛祝誅。上疑事在永中,未有以發也。



 會鎬王傅尉奏
 永中第四子阿離合懣因防禁嚴密,語涉不道。詔同簽大睦親府事褲、御史中丞孫即康鞫問,并求得第二子神徒門所撰詞曲有不遜語。家奴德哥首永中嘗與侍妾瑞雪言:「我得天下,子為大王,以爾為妃。」詔遣官復按狀同。再遣禮部尚書張暐、兵部侍郎烏古論慶裔復之。上謂宰臣曰:「鎬王只以語言得罪,與永蹈罪異。」參知政事馬琪曰:「永中與永蹈罪狀雖異,人臣無將,則一也。」上曰:「大王何故輒出此言?」左丞相清臣曰:「素有妄想之心也。」詔以永中罪狀宣示百官雜議,五品以下附奏,四品以上入對便殿。皆曰:「請論如律。」惟宮籍監丞盧利用乞
 貸其死。詔賜永中死,神徒門、阿離合懣等皆棄市。敕有司用國公禮收葬永中,平陽府監護,官給葬具,妻子威州安置。泰和七年,詔復永中王爵,賜謚曰厲。敕石古乃於威州擇地,以禮改葬,歲時祭奠。貞祐二年,詔徙永中妻、子石古乃等鄭州安置。



 貞祐三年,太康縣人劉全嘗為盜,亡入衛真界,詭稱愛王。所謂愛王,指石古乃。石古乃實未嘗有王封,小人妄以此目之。劉全欲為亂,因假託以惑眾,誘王氏女為妻,且言其子方聚兵河北。東平人李寧居嵩山,有妖術。全同縣人時溫稱寧可論大事,乃使范元書偽號召之。寧至,推為國師,議僭立。事覺,全、
 溫、寧皆伏誅。



 貞祐四年,潼關破,徙永中子孫於南京。興定二年,亳州譙縣人孫學究私造妖言云:「愛王終當奮發,今匿跡民間,自號劉二。」衛真百姓王深等皆信以為誠然。有劉二者出而當之,遣歐榮輩結構逆黨,市兵仗,大署旌旗,謀心慄立。事覺,誅死者五十二人,緣坐者六十餘人。永中子孫禁錮,自明昌至於正大末,幾四十年。天興初,詔弛禁錮。未幾,南京亦不守云。



 鄭王永蹈,本名銀術可,初名石狗兒。大定十一年,封滕王,未期月進封徐王。二十五年,加開府儀同三司。二十六年,為大興尹。章宗即位,判彰德軍節度使,進封衛王。
 明昌二年,徙封鄭王。三年,改判定武軍。



 初,崔溫、郭諫、馬太初與永蹈家奴畢慶壽私說讖記災祥,畢慶壽以告永蹈:「郭諫頗能相人。」永蹈乃召郭諫相已及妻子。諫說永蹈曰:「大王相貌非常,王妃及二子皆大貴。」又曰:「大王,元妃長子,不與諸王比也。」永蹈召崔溫、馬太初論讖記天象。崔溫曰:「丑年有兵災,屬兔命者來年春當收兵得位。」郭諫曰:「昨見赤氣犯紫微,白虹貫月,皆注丑後寅前兵戈心慄亂事。」永蹈深信其說,乃陰結內侍鄭雨兒伺上起居,以崔溫為謀主,郭諫、馬太初往來游說。河南統軍使僕散揆尚永蹈妹韓國公主,永蹈謀取河南軍以為
 助,與妹澤國公主長樂謀,使駙馬都尉蒲剌睹致書于揆,且先請婚,以觀其意。揆拒不許結婚,使者不敢復言不軌事。永蹈家奴董壽諫永蹈,不聽。董壽以語同輩奴千家奴,上變。是時,永蹈在京師,詔平章政事完顏守貞、參知政事胥持國、戶部尚書楊伯通、知大興府事尼龐古鑑鞫問,連引甚眾,久不能決。上怒,召守貞等問狀。右丞相夾谷清臣奏曰:「事貴速絕,以安人心。」於是,賜永蹈及妃卞玉,二子按春、阿辛,公主長樂自盡。蒲剌睹、崔溫、郭諫、馬太初等皆伏誅。僕散揆雖不聞問,猶坐除名。董壽免死,隸監籍。千家奴賞錢二千貫,特遷五官雜班敘
 使。自是諸王制限防禁密矣。



 泰和七年,詔復王封,備禮改葬,賜謚曰剌,以衛王永濟子按辰為永蹈後,奉其祭祀。



 越王永功,本名宋葛,又名廣孫,貞元二年生。沉默寡言笑,勇健絕人,涉書史,好法書名畫。大定四年,封鄭王。七年,進封隋王。十一年,進封曹王。十五年,除刑部尚書。上曰:「侍郎張汝霖,汝外舅行也,可學為政。」十七年,授活活土世襲猛安。十八年,改大興尹。



 世宗幸金蓮川,始出中都,親軍二蒼頭縱馬食民田,詔永功:「蒼頭各杖一百。彈壓百戶二人失覺察,勒停。」上次望京澱,永功奏曰:「親軍
 人止一蒼頭、兩彈壓服勤,為日久矣。臣昧死違詔,量決蒼頭,使彈壓待罪,可使償其田直,惟陛下憐察。」上皆從之。



 老嫗與男婦憩道傍,婦與所私相從亡去,或告嫗曰:「向見年少婦人自水邊小徑去矣。」嫗告伍長蹤跡之。有男子私殺牛,手持血刃,望見伍長,意其捕己,即走避之。嫗與伍長疑是殺其婦也,捕送縣,不勝楚毒,遂誣服。問尸安在?詭曰:「棄之水中矣。」求之水中,果獲一尸,已半腐。縣吏以為是男子真殺若婦矣,即具獄上。永功疑之曰:「婦死幾何日,而尸遽半腐哉。」頃之,嫗得其婦於所私者。永功曰:「是男子偶以殺人就獄,其拷掠足以稱殺牛之
 科矣。」遂釋之而去。武清黃氏、望雲王氏豪猾不逞,永功發其罪,畿內肅然。



 二十三年,判東京留守。是月,改河間尹。閱月,改北京留守。居無何,上謂宰臣曰:「朕聞永功到北京為政無良,雖朕子,萬一敗露,法可廢乎。朕已戒敕永功,卿等可諭其長史,俾匡正之。」到北京凡七月,改東京留守。世宗幸上京,過東京,永功從。明年,上還至天平山好水川,皇太子薨。詔永功護喪事,尋拜御史大夫。章宗封原王,加開府儀同三司。趙王永中及永功兄弟皆加開府儀同三司。明年,判大宗正事。



 應州僧與永功有舊,將訴事于彰國軍節度使移剌胡剌,求永功手書與
 胡剌為地。胡剌得書,奏之。上謂宰臣曰:「永功以書囑事胡剌,此雖細微,不可不懲也。凡人小過不治,遂至大咎。有犯必懲,庶幾能改,是亦教也。」皆曰:「陛下用法無私,臣下敢不敬畏。」於是永功解職。未幾,復判大宗正事。



 章宗即位,除判平陽府事,進封冀王。永功之官,隨引醫人沈思存過制限,當解職。上曰:「朕知此事,當痛斷監奴及治府掾長史管轄府事者罪,仍著于令。」家奴王唐犯罪至徒,永功曲庇之。平陽治中高德裔失覺察,笞四十。於是永功改判濟南府。詔永功曰:「所坐雖細事,法令不得不如此。今已釋矣,後毋復然。濟南先帝舊治,風土甚好,可
 悉此意也。」改授山東西路把魯古世襲猛安。二年,判廣寧府事,進封魯王。明年,判彰德府事。承安元年,進封郢王。明年,判太原府事。泰和七年,改西京留守。八年,復判平陽府事。大安元年,進封譙王,判中山府事。明年,進封越王。



 宣宗即位,免常參。明年,從遷汴京。久之,詔永功每月朔一朝。興定四年,詔永功無朝。五年,有疾,賜御藥。疾革,賜尚醫診視,一日五遣使候問。是歲,薨。上哭之慟,謚曰忠簡。



 子福孫、壽孫、粘沒曷。大定二十六年,詔賜福孫名璐,壽孫名璹,粘沒曷名琳。是年,璐加奉國上將軍。章宗即位,加銀青榮祿大夫,封蕭國公。初為興陵崇妃養
 子,常居京師,奉朝請。泰和五年,卒。章宗輟朝,百官進名奉慰。



 璹本名壽孫,世宗賜名,字仲實,一字子瑜。資質簡重,博學有俊才,喜為詩,工真草書。大定二十七年,加奉國上將軍。明昌初,加銀青榮祿大夫。衛紹王時,加開府儀同三司。貞祐中,封胙國公。正大初,進封密國公。



 璹奉朝請四十年,日以講誦吟詠為事,時時潛與士大夫唱酬,然不敢明白往來。永功薨後,稍得出游,與文士趙秉文、楊雲翼、雷淵、元好問、李汾、王飛伯輩交善。初,宣宗南遷,諸王宗室顛沛奔走,璹乃盡載其家法書名畫,一帙不遺。
 居汴中,家人口多,俸入少,客至,貧不能具酒肴,蔬飯共食,焚香煮茗,盡出藏書,談大定、明昌以來故事,終日不聽客去,樂而不厭也。



 天興初,璹已臥疾,論及時事,嘆曰:「兵勢如此,不能支,止可以降。全完顏氏一族歸吾國中,使女直不滅則善矣,餘復何望。」是時,曹王出質,璹見哀宗於隆德殿。上問:「叔父欲何言?」璹奏曰:「聞訛可欲出議和。訛可年幼,不苦諳練,恐不能辦大事。臣請副之,或代其行。」上慰之曰:「南渡後,國家比承平時有何奉養,然叔父亦未嘗沾溉。無事則置之冷地,無所顧藉,緩急則置于不測,叔父盡忠固可,天下其謂朕何?叔父休矣。」於是
 君臣相顧泣下。未幾,以疾薨。年六十一。



 平生詩文甚多。自刪其詩,存三百首,樂府一百首,號《如庵小稿》。第五子守禧,字慶之,風神秀徹,璹特鐘愛,嘗曰:「平日所蓄書畫將以付斯子。」及汴城降,守禧病卒,年未三十。



 潞王永德,本名訛出。大定二十五年,與章宗及諸兄俱加開府儀同三司。二十七年,封薛王。明年,除秘書監。二十九年,進判秘書監,進封沈王。明昌元年,授山東東路把魯古必剌猛安。二年,進封豳王。五年,遷勸農使。承安二年,進封潞王。承安三年,再任勸農使。泰和元年,有司劾永德元日進酒後期,有詔勿問。衛紹王時,累遷太子
 太師。宣宗即位,改同判大睦親府事。興定五年,遷判大睦親府事。子斡論,賜名琰。



 豫王永成,本名鶴野,又曰婁室。母昭儀梁氏。永成風姿奇偉,博學,善屬文。世宗尤愛重之。大定七年,始封沈王,以太學博士王彥潛為府文學,永成師事之。十一年,進封豳。十五年,就外第。十六年,判秘書監。明年,授世襲山東東路把魯古猛安,判大睦親府事。既而改中都路胡土靄哥蠻猛安。二十年,改授翰林學士承旨。二十三年,判定武軍節度使事,尋改判廣寧府。二十五年,世宗幸上京,命留守中都,判吏部尚書,進開府儀同三司,為御
 史大夫。



 章宗即位,起復,進封吳,判真定府事。明昌元年,改山東西路盆買必剌猛安。明年,進封兗。坐率軍民圍獵,解職,奉表謝罪。上賜手詔曰:「卿親實肺腑,夙著忠純,侍顯考於春宮,曲盡友于之愛,洎沖人之繼統,愈明忠赤之心,艱難之中,多所裨益。朕心簡在,毫楮莫窮,用是起之苫塊之中,授以維城之任。自典籓服,歲月薦更,蕞爾趙邦,知驥足之難展,眇哉鎮府,固牛刀之莫施。方思驛召以赴朝,何意遽罹於國憲。偶因時獵,頗擾部民,法所不寬,憲臺聞上。朕尚含容累月,未忍即行,雖欲遂於私恩,竟莫違於公議,解卿前職,即乃世封。噫,祖宗立法,
 非一人之敢私;骨肉至親,豈千里而能間。以此退閑之小誡,欲成終始之洪恩。《經》云:『在上不驕,高而不危。』是以知節慎者修身之本,驕矜者敗德之源。朕每自勵,今以戒卿。昔東平樂善,能成不朽之名,梁孝奢淫,卒致憂疑之悔。前人所行,可為龜鑒。卿兼資文武,多藝多才,履道而行,何施不可。如能德業日新,無慮牽復之晚。朕素不工詞翰,臨文草草,直寫所懷,冀不以辭害意也。」未幾,授沁南軍節度使。三年,改判咸平府事,未赴,移判太原府事。上以永成誕日,親為詩以賜,有「美譽自應輝玉牒,忠誠不待啟金滕」之語,當世榮之。



 七年,改判平陽府事。承
 安改元,以覃恩進封豫。明年冬,進馬八十疋,以資守禦之備。上賜詔獎諭曰:「卿夙有雋望,時惟茂親,通達古今,砥礪忠義。方分憂於外服,來輸駿於上閑,欲助邊防,以增武備。惟盡心於體國,乃因物以見誠。載念懇勤,良深嘉獎。」五年,再任。俄召還,以疾不能入見。上親幸其第臨視。泰和四年,薨。訃聞,上為之震悼,賻贈甚厚,謚曰忠獻。



 永成自幼喜讀書,晚年所學益醇,每暇日引文士相與切磋,接之以禮,未嘗見驕色。自號曰「樂善居士」,有文集行於世云。



 夔王允升,改名永升,本名斜不出,一名鶴壽。大定十一
 年,封徐王,進封虞王。二十六年,加開府儀同三司。明年,判吏部尚書,授山東西路按必出虎必剌猛安。章宗即位,加恩宗室,徙封隋王,除定武軍節度使。明昌二年,改封曹王。久之,改封宛王。衛紹王即位,徙今封。貞祐元年九月,宣宗以允升年高,素羸疾,詔宮中聽扶杖。尋薨。既殯,燒飯,上親臨奠。



 贊曰:世宗保全宗室,無所不至,雖矯海陵之失,亦由天資仁厚而然也。其子永中、永蹈皆死章宗之手,其理蓋有不可詰者。章宗無後,則厥報不爽矣。



\end{pinyinscope}