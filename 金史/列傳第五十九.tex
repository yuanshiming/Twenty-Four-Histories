\article{列傳第五十九}

\begin{pinyinscope}

 忠義一



 ○胡沙補特虎僕忽得粘割韓奴曹珪溫迪罕蒲睹訛里也納蘭綽赤魏全鄯陽夾谷守中石抹元毅伯德梅和尚烏古孫兀屯高守約和速嘉安禮王維翰移剌古與涅
 宋扆烏古論榮祖烏古論仲溫九住李演劉德基王毅王晦齊鷹揚術甲法心高錫



 欒共子曰:「民生於三,事之如一,唯其所在,則致死焉。」公卿大夫居其位,食其祿,國家有難,在朝者死其官,守郡邑者死城郭,治軍旅者死行陣,市井草野之臣發憤而死,皆其所也。故死得其所,則所欲有甚於生者焉。金代褒死節之臣,既贈官爵,仍錄用其子孫。貞祐以來,其禮有加,立祠樹碑,歲時致祭,可謂至矣。聖元詔修遼、金、宋
 史,史臣議凡例,凡前代之忠於所事者,請書之無諱,朝廷從之。烏乎,仁哉聖元之為政也。司馬遷記豫讓對趙襄子之言曰:「人主不掩人之美,而忠臣有成名之義。」至哉斯言,聖元之為政,足為萬世訓矣。作《忠義傳》。



 胡沙補,完顏部人。年三十五從軍,頗見任用。太祖使僕刮剌往遼國請阿竦,實觀其形勢。僕刮剌還言遼兵不知其數,太祖疑之,使胡沙補往。還報曰:「遼兵調兵,尚未大集。」及見統軍,使其孫被甲立於傍,統軍曰:「人謂汝輩且反,故為備耳。」及行道中,遇渤海軍,渤海軍向胡沙補且笑且言曰:「聞女直欲為亂,汝輩是邪。」具以告太祖,又
 曰:「今舉大事不可後時,若俟河凍,則遼兵盛集來攻矣。乘其未集而早伐之,可以得志。」太祖深然之。及破寧江州,戰于達魯古城,皆有功,賜以旗鼓並御器械。高永昌請和,胡沙補往招之,取胡突古以歸。高永昌詐降于斡魯,斡魯使胡沙補、撒八往報。會高楨降,言永昌非真降者,斡魯乃進兵。永昌怒,遂殺胡沙補,撒八,皆支解之。胡沙補就執,神色自若,罵永昌曰:「汝叛君逆天,今日殺我,明日及汝矣。」罵不絕口,至死。年五十九。天會中,與撒八俱贈遙鎮節度使。



 特虎,雅撻瀾水人。軀幹雄偉,敢戰鬥。達魯古城之役,活
 女陷敵,特虎救出之。攻照散城,遼兵三千來拒,特虎先登,敗之。攻盧葛營,麻吉墮馬,特虎獨殺遼兵數輩,掖而出之。賞賚逾渥。自臨潢班師,至遼河,余睹來襲,婁室已引去,特虎獨殿,馬憊乃步斗,婁室與數騎來救,特虎止V毄唬骸拔乙砸凰籃吹1111校鵠矗惚形摶妗!彼烀粛于陣。皇統間,贈明威將軍。



 僕忽得,宗室子。初事國相撒改,伐蕭海里有功。與酬斡俱,招降燭偎水部族,酬斡為謀克,僕忽得領行軍千戶。從破黃龍府,戰于達魯古城,皆有功。寧江州渤海乙塞補叛,僕忽得追復之。天輔五年九月,酬斡、僕忽得往鱉
 古河籍軍馬,燭偎水部實里古達等七人殺酬斡、僕忽得,投其尸水中,俱年四十三。太祖悼惜,遣使弔賻加等。六年正月,斡魯伐實里古達于石里罕河,追及於合撻剌山,殺四人,撫定餘眾。詔斡魯求酬斡、僕忽得尸以葬。天眷中,贈酬斡奉國上將軍、僕忽得昭義大將軍。



 酬斡,亦宗室子也。年十五隸軍,從太祖伐遼,率濤溫路兵招撫三坦、石里很、跋苦三水鱉古城邑,皆降之。敗室韋五百於阿良葛城,獲其民眾。至是死焉。



 粘割韓奴,以護衛從宗弼征伐,賜鎧甲弓矢戰馬。初,太祖入居庸關,遼林牙耶律大石自古北口亡去,以其眾
 來襲奉聖州,壁于龍門東二十五里,婁室往取之,獲大石并降其眾。宗望襲遼主輜重於青塚,以大石為鄉導,詔曰:「遼趙王習泥烈、林牙大石、北王喝里質、節度使訛里刺、孛堇赤狗兒、招討迪六、祥穩六斤、同知海里及諸官民,並釋其罪。」復詔斡魯曰:「林牙大石雖非降附,其為鄉導有勞,可明諭之。」時天輔六年也。既而亡去,不知所往。



 天會二年,遼詳穩撻不野來降,言大石稱王於北方,署置南北面官僚,有戰馬萬匹,畜產甚眾。詔曰:「追襲遼主,必酌事宜而行。攻討大石,須俟報下。」三年,都M懲暄諐希尹言,聞夏人與耶律大石約曰:「大金既獲遼主,諸軍
 皆將1111歸矣,宜合兵以取山西諸部。」詔答曰:「夏人或與大石合謀為釁,不可不察,其嚴備之。」七年,泰州路都統婆盧火奏:「大石已得北部二營,恐後難制,且近群牧,宜列屯戍。」詔答曰:「以二營之故發兵,諸部必擾,當謹斥候而已。」八年,遣耶律餘睹、石家奴、拔離速追討大石,徵兵諸部,諸部不從,石家奴至兀納水而還。余睹報元帥府曰:「聞耶律大J阯諍蛵州之域,恐與夏人合,當遣1111使索之。」夏國報曰:「小國與和州壤地不相接,且不知大石所往也。」皇統四年,回紇遣使入貢,言大石與其國相鄰,大石已死。詔遣韓奴與其使俱往,因觀其國風俗,加武義將軍,
 奉使大石。韓奴去後不復聞問。



 大定中,回紇移習覽三人至西南招討司貿易,自言:「本國回紇鄒括番部,所居城名骨斯訛魯朵,俗無兵器,以田為業,所獲十分之一輸官。耆老相傳,先時契丹至不能拒,因臣之。契丹所居屯營,乘馬行自旦至日中始周匝。近歲契丹使其女婿阿本斯領兵五萬北攻葉不輦等部族,不克而還,至今相攻未已。」詔曰:「此人非隸朝廷番部,不須發遣,可於咸平府舊有回紇人中安置,毋令失所。」



 是歲,粘拔恩君長撒里雅寅特斯率康里部長孛古及戶三萬餘求內附,乞納前大石所降牌印,受朝廷牌印。詔西南招討司遣
 人慰問,且觀其意。禿里餘睹、通事阿魯帶至其國見撒里雅,具言願歸朝廷,乞降牌印,無他意也。因曰:「往年大國嘗遣粘割韓奴自和州往使大石,既入其境,大石方適野,與韓奴相遇,問韓奴何人敢不下馬,韓奴曰:『我上國使也,奉天子之命來招汝降,汝當下馬聽詔。』大石曰:『汝單使來,欲事口舌耶?』使人捽下,使韓奴跪,韓奴罵曰:『反賊,天子不忍於爾加兵,遣招汝。爾縱不能面縛請罪闕下,亦當盡敬天子之使,乃敢反加辱乎!』大石怒,乃殺之。此時大石林牙已死,子孫相繼,西方諸部仍以大石呼之。』」余睹、阿魯帶還奏,并奏韓奴事。世宗嘉韓奴忠節,
 贈昭毅大將軍,召其子永和縣商酒都監詳古、汝州巡檢婁室諭之曰:「汝父奉使萬里,不辱君命,能盡死節,朕甚閔之。」詳古為尚輦局直長,遷武義將軍,婁室為武器署直長。



 曹珪,徐州人。大定四年,州人江志作亂,珪子弼在賊黨中,珪謀誅志,并弼殺之。尚書省議,當補二官雜班敘。詔曰:「珪赤心為國,大義滅親,自古罕聞也。法雖如是,然未足以當其功,更進一官,正班用之。」



 溫迪罕蒲睹,為兀者群牧使。西北路契丹撒八等反,諸群牧皆應之。蒲睹聞亂作,選家奴材勇者數十人,給以
 兵仗,陰為之備。賊不得發,乃紿諸奴曰:「官閱兵器,願借兵仗以應閱。」諸奴以為實然,遂借與之。明旦,賊至,蒲睹無以禦之。賊執蒲睹而問之曰:「今欲反未?」蒲睹曰:「吾家世受國厚恩,子姪皆仕宦,不能從汝反而累吾族也。」賊怒,臠而殺之,子與孫皆與害。



 是時,迪斡群牧使徒單賽里、副使赤盞胡失答,耶魯瓦群牧使鶴壽,歐里不群牧使完顏術里骨、副使完顏辭不失,卜迪不部副使赤盞胡失賴,速木典颭詳穩加古買住,胡睹颭詳穩完顏速沒葛,轄木颭詳穩高彭祖等皆遇害。



 鶴壽,鄆王昂子,本名吾都不。五院部人老和尚率眾來招鶴壽與俱反,鶴壽
 曰:「吾宗室子,受國厚恩,寧殺我,不能與賊俱反。」遂與二子皆被殺。



 訛里也,契丹人。為尚廄局直長。大定初,招諭契丹,窩斡叱令訛里也跪見,訛里也不從,謂曰:「我朝廷使也,豈可屈節於汝。汝等早降可全性命,若大軍至,汝輩悔將何及。」窩斡怒曰:「汝本契丹人,而不我從,敢出是言。」遂害之。從行驍騎軍士閏孫、史大、習馬小底頗答皆被害。三年,贈訛里也宣武將軍,錄其子阿不沙為外帳小底。閏孫、史大皆贈修武校尉。頗答贈忠翊校尉。



 納蘭綽赤,咸平路伊改河猛安人。契丹括里使人招之,
 綽赤不從。括里兵且至,綽赤遂團結旁近村寨為兵,出家馬百餘匹給之,教以戰陣擊刺之法,相與拒括里于伊改渡口,由是賊眾月餘不得進。既而括里兵四萬人大至,綽赤拒戰,賊兵十倍,遂見執,臠而殺之。詔贈官兩階,二子皆得用廕。



 魏全,壽州人。泰和六年,宋李爽圍壽州,刺史徒單羲盡籍城中兵民及部曲廝役得三千餘人,隨機拒守堅甚。羲善撫御,得眾情,雖婦人皆樂為用。同知蒲烈古中流矢卒,羲益勵不衰,募人往斫爽營,全在選中,為爽兵所執。爽謂全曰:「若為我罵金主,免若死。」全至城下,反罵宋
 主,爽乃殺之,至死罵不絕口。



 僕散揆遣河南統軍判官乞住及買哥等以騎二千人救壽州,去壽州十餘里與爽兵遇,乞住分兩翼夾擊爽兵,大破之,斬首萬餘級,追奔至城下,拔其三柵,焚其浮梁。羲出兵應之,爽兵大潰,赴淮死者甚眾。爽與其副田林僅脫身去,餘兵脫者十之四。詔遷羲防禦使、乞住同知昌武軍節度使事、買哥河南路統軍判官。



 贈蒲烈古昭勇大將軍,官其子圖剌。



 贈全宣武將軍、蒙城縣令,封其妻為鄉君,賜在州官舍三間、錢百萬,俟其子年至十五歲收充八貫石正班局分承應,用所贈官蔭,仍以全死節送史館,鏤版頒諭天
 下。



 鄯陽,宗室子。為符寶祗候。完顏石古乃為護衛十人長。至寧元年八月,紇石烈執中作亂,入自通玄門。是日,變起倉猝,中外不知所為,鄯陽、石古乃往天王寺召大漢軍五百人赴難,與執中戰於東華門外。執中揚言曰:「大漢軍反矣,殺一人者賞銀一錠。」執中兵眾,大漢軍少,二人不勝而死。須臾,執中兵殺五百人殆盡。



 執中死,詔削官爵。詔曰:「宣武將軍、護衛十人長完顏石古乃,修武校尉、符寶祗候鄯陽,忠孝勇果,沒于王事。石古乃贈鎮國上將軍、順州刺史,鄯陽贈宣武將軍、順天軍節度副使。
 嘗從拒戰猛安賞錢五百貫、謀克三百貫、蒲輦散軍二百貫,各遷兩階。戰沒者,贈賞付其家。石古乃子尚幼,以八貫石俸給之,俟年十五以聞。



 夾谷守中,咸平人,本名阿土古。大定二十二年進士,歷清池、聞喜主簿,補尚書省令史,除刑部主事、監察御史、修起居注。轉禮部員外郎、大名治中,歷嵩琢、北京、臨洮路按察副使。以憂去官,起復同知曷懶路兵馬都總管府事,坐事謫韓州刺史,尋復同知平涼府事。大安二年,為秦州防禦使,遷通遠軍節度使。至寧末,移彰化軍,未行,夏兵數萬入鞏州。守中乘城備守,兵少不能支,城陷,
 官吏盡降,守中獨不屈。夏人壯之,且誘且脅,守中益堅,遂載而西。至平涼,要以招降府人,守中佯許,至城下即大呼曰:「外兵矢盡且遁矣,慎勿降。」夏人交刃殺之。



 興定元年,監察御史郭著按行秦中,得其事以聞。詔贈資善大夫、東京留守,仍收其子兀母為筆硯承奉。



 石抹元毅,本名神思,咸平府路酌赤烈猛安莎果歌仙謀克人也。以蔭補吏部令史。再調景州寧津令,有劇盜白晝恣劫為民害,元毅以術防捍,賊散去。入為大理知法,除同知亳州防禦使事,被省檄,錄陜右五路刑獄,無冤人。復委受宋歲幣,故事有私遺物,元毅一無所受。明
 昌初,驛召為大名等路提刑判官,以最遷汾陽軍節度副使。時石、嵐間賊黨嘯聚,肆行剽掠,朝廷命元毅捕之,賊畏而遁。元毅追襲,盡殪之,二境以安。遷同知武勝軍節度使事,別郡有殺人者,屢鞫不伏,元毅訊不數語,即具服。河東北路田多山阪磽瘠,大比時定為上賦,民力久困,朝廷命相地更賦,元毅以三壤法平之,民賴其利。改彰德府治中,尋以邊警授撫州刺史。會邊將失守,芻糧馬牛焚剽殆盡,元毅率吏卒三十餘人出州經畫軍餉,卒與敵遇。州倅暨從吏堅請還,元毅曰:「我輩責任邊守,遇敵而奔,其如百姓何?縱得自安,復何面目見朝廷乎!」
 遂執弓矢令眾。眾感其忠,爭為效死。元毅力戰,射無不中。敵去而復合,元毅氣愈厲,鏖戰久之,眾寡不敵,遂遇害,時年四十七。事聞,上深驚悼,贈信武將軍,召用其子世勣侍儀司承應。



 世勣後登進士第,奏名之日,上謂宰臣曰:「此神思子耶。」歎賞者久之。元毅性沈厚,武勇過人,每讀書見古人忠義事,未嘗不嗟歎賞慕,喜動顏色,故臨難能死所事云。



 伯德梅和尚,泰州人也。性鯁直,尚氣節。正隆五年,收充護衛,授曷魯碗群牧副使。未幾,復召為護衛十人長,改尚廄局副使,遷本局使,轉右衛將軍拱衛使。典尚廄者
 十餘年,積勞特遷官二階,除復州刺史。明昌初,為西北路副招討,收泰州防禦使,升武勝軍節度使。六年,移鎮崇義軍。時有事北邊,左丞相夾谷清臣行省于臨潢,檄為副統。會敵入臨潢,梅和尚暨護衛闢合土等領軍逆擊之。敵積陣以待,梅和尚直搗其陣,殺傷甚眾。敵知孤軍無繼,聚兵圍之。度不能免,乃下馬相背射,復殺百餘人,矢盡猶以弓提擊,為流矢所中死,闢合土等皆沒。



 上聞之震悼,詔贈龍虎衛上將軍,躐遷十階,特賜錢二十萬,命以禮葬之,特皆官給,以其子都奴為軍前猛安,中奴護喪,就差權同知臨潢府事李達可為敕祭使,同知
 德昌軍節度使事石抹和尚為敕葬使。承安五年,上諭尚書省曰:「梅和尚死王事,其子都奴從軍久有功,其議所以酬之。」乃命為典署丞。



 烏古孫兀屯,上京路人。大定末,襲猛安。明昌七年,以本兵充萬戶,備邊有功,除歸德軍節度副使,改盤安軍,察廉,遷同知速頻路節度使事。以憂去官,起復歸德府治中,遷唐州刺史。泰和六年四月,宋皇甫斌步騎萬人侵唐州,兀屯兵甚少,遣泌陽尉白散不、巡檢蒲閑各以五十人乘城拒守。兀屯見宋兵在城東北者可破,令軍事判官撒虎帶以精兵百人自西門出,繞出東北宋兵營
 後掩擊之,殺數十百人,宋兵大亂,迨夜乃遁去。五月,皇甫斌復以兵數萬來攻,行省遣泌陽副巡檢納合軍勝救唐州。兀屯出兵與軍勝合兵城東北,設伏兵以待之。乃分騎兵為三,一出一入以致宋兵。宋兵陷于淖,伏兵發,中衝宋兵為二,遂大潰。追奔至湖陽,斬首萬餘級,獲馬三百匹。宋別將以兵三千來襲,遇之竹林寺,殪之。納合軍勝手殺宋將,取其金帶印章以獻。詔遷兀屯同知河南府事,軍勝遷梁縣令,各進兩階。兀屯賞銀三百五十兩、重彩十端,為右副元帥完顏匡右翼都統。匡取棗陽,遣元屯襲神馬坡,宋兵五萬人夾水陣,以強弩拒岸,
 兀屯分兵奪其三橋,自辰至午連拔十三柵,遂取神馬坡。從攻襄,至漢江,兀屯亂流徑度。復進一階,號平南虎威將軍。宋人請和,遷河南副統軍。大安初,遷昌武軍節度使,副統軍如故。遷西南路招討使。兀屯御下嚴酷,軍士多亡,杖六十。除同知上京留守事。大安三年,將兵二萬入衛中都,遷元帥右都監、轉左都監,兼北京留守。有功,賜金吐鶻、重彩十端。遷元帥左監軍,留守如故。貞祐元年閏月,以兵入衛中都,詔以兵萬六千人守定興,軍敗,兀屯戰沒。



 高守約,字從簡,遼陽人。大定二十八年進士,累官觀州
 刺史。大元兵徇地河朔,郭邦獻已歸順,從至城下,呼守約曰:「從簡當計全家室。」守約弗顧,至再三,寧約厲聲曰:「吾不汝識也。」城破被執,使之跪,守約不屈,遂死。詔贈崇義軍節度使,謚忠敬。



 和速嘉安禮,字子敬,本名酌,大名路人。穎悟博學,淹貫經史。大定二十八年進士。至寧末,為泰安州刺史。貞祐初,山東被兵,郡縣望風而遁,或勸安禮去之,安禮曰:「我去,城誰與守,且避難負國家之恩乎?」乃團練繕完,為禦守計。已而大元兵至,戰旬日不能下,謂之曰:「此孤城耳,內無糧儲,外無兵援,不降無遺類矣。」安禮不聽。城破被
 執,初不識其為誰,或妄以酒監對,安禮曰:「我刺史也,何以諱為?」使之跪,安禮不屈,遂以戈撞其胸而殺之。詔贈泰定軍節度使,謚堅貞。



 王維翰,字之翰,利州龍山人。父庭,遼季率縣人保縣東山,後以眾降。維翰好學不倦,中大定二十八年進士。調貴德州軍事判官,察廉遷永霸令。縣豪欲嘗試維翰,設事陳訴,維翰窮竟之,遂伏其詐,杖殺之,健訟衰息。歷弘政、獲嘉令,佐胥持國治河決,有勞,遷一階。改北京轉運戶籍判官,補尚書省令史。除同知保靜軍節度使事,檢括戶籍,一郡稱平。屬縣有奴殺其主人者,誣主人弟殺
 之,刑部疑之。維翰審讞,乃微行物色之,得其狀,奴遂引服。改中都轉運副使,攝侍御史,奏事殿中,章宗曰:「佳御史。」就除侍御史。改左司員外郎,轉右司朗中。僕散揆伐宋,維翰行省左右司郎中。泰和七年,河南旱蝗,詔維翰體究田禾分數以聞。七月,雨,復詔維翰曰:「雨雖沾足,秋種過時,使多種蔬菜猶愈於荒萊也。蝗蝻遺子,如何可絕?」舊有蝗處來歲宜菽麥,諭百姓使知之。」



 八年,宋人受盟,還為右司郎中,進官一階。上問:「宋人請和復能背盟否?」維翰對曰:「宋主怠于政事,南兵佻弱,兩淮兵後千里蕭條,其臣懲韓侂胄、蘇師旦,無復敢執其咎者,不足憂
 也。唯北方當勞聖慮耳。」久之,遷大理卿、兼潞王傅,同知審官院事。新格,教坊樂工階至四品,換文武正資,服金紫。維翰奏:「伶優賤工,衣縉紳之服,非所以尊朝廷也。」從之。大安初,權右諫議大夫,三司欲稅間架,維翰諫不聽。轉御史中丞,無何,遷工部尚書、兼大理卿,改刑部尚書,拜參知政事。



 貞祐初,罷為定海軍節度使。是時,道路不通,維翰舟行遇盜,呼謂之曰:「爾輩本良民,因亂至此,財物不惜,勿恐吾家。」盜感其言而去。至鎮,無兵備,鄰郡皆望風奔潰,維翰謂吏民曰:「孤城不可守。此州阻山浮海,當有生地,無俱為魚肉也。」乃縱百姓避難。維翰率吏民
 願從者奔東北山,結營堡自守,力窮被執不肯降。妻姚氏亦不肯屈,與維翰俱死。詔贈中奉大夫,姚氏芮國夫人,謚貞潔。



 移剌古與涅,安化軍節度使。貞祐初,大元年兵取密州,古與涅率兵力戰,流矢連中其頸,既拔去復中其頰,死焉。貞祐三年,詔贈安遠大將軍、知益都府事。



 宋扆,中都宛平人也。正隆五年進士。歷辰州、寧化州軍事判官,曹王府記室參軍。陜西西路轉運都勾判官。補尚書省令史,除武定軍節度副使、中都右警巡使。時固安縣丞劉昭與部民裴原爭買鄰田,扆用昭屬,抑原使
 毋爭。御史臺劾奏,奪一官,解職,降廣寧府推官。改遼東路鹽使。丁父憂,起復吏部員外郎,歷薊、曹、景州刺史,同知中都路轉運使事,遷北京、臨潢等路按察使。改安國軍節度使、河東南路轉運使。御史劾其前任按察侵民舍不稱職,降沂州防禦使,移濬州,遷山東西路轉運使,改定海軍節度使。貞祐二年,改沁南軍,正月,大元兵至懷州,城破死焉。扆天資刻酷,所至不容物,以是蹭蹬於世云。



 烏古論榮祖,本名福興,河間人。明昌二年進士,歷官補尚書省令史,除都轉運司都勾判官,轉弘文校理,升中
 都總管府判官,察廉除震武軍節度副使、彰德府司馬,累遷戶部員外郎、寧海州刺史。貞祐二年城破,榮祖猶力戰,死之。贈安武軍節度使,賜謚毅勇。



 烏古論仲溫,本名胡剌,蓋州按春猛安人。大定二十五年進士,累官太學助教、應奉翰林文字、河東路提刑判官,改河北東路轉運副使。御史薦前任提刑稱職,遷同知順天軍節度使事,簽上京、東京等路按察司事,改提舉肇州漕運、兼同知武興軍節度使事、東勝州刺史。坐前在上京不稱職,降鎮寧軍節度副使。改滑州刺史、河東南路按察副使、壽州防禦使。貞祐初,遷鎮西軍節度使。
 是時,中都被圍,遂至太原,移書安撫使賈益謙,約以鄉兵救中都。因馳驛如平陽,將與益謙會於絳,不能進,抵平陽而還。仲溫嘗治平陽,吏民爭留之,仲溫曰:「平陽巨鎮,易為守禦,於私計得矣,如嵐州何。」遂還鎮。已而大元兵大至,城破,不屈而死。贈資德大夫、婆速路兵馬都總管,謚忠毅,歲時致祭。



 九住,宗室子,為武州刺史,唐括孛果速為軍事判官。貞祐二年十一月,大元兵取九住子姪抵城下,謂之曰:「山東、河北今皆降我,汝之家屬我亦得已,茍不速降,且殺之也。」九住曰:「當以死報國,遑恤家為。」無何,城破,力戰而
 死,孛果速亦不屈死焉。詔贈九住臨海軍節度使,加驃騎衛上將軍。孛果速建州刺史,加鎮國上將軍。仍令樹碑,歲時致祭。



 李演,字巨川,任城人。泰和六年進士第一,除應奉翰林文字。再丁父母憂,居鄉里,貞祐初,任城被兵,演墨衰為濟州刺史,畫守禦策。召集州人為兵。搏戰三日,眾皆市人不能戰,逃散。演被執,大將見其冠服非常,且知其名,問之曰:「汝非李應奉乎?」演答曰:「我是也。」使之跪,不肯,以好語撫之,亦不聽,許之官祿,演曰:「我書生也,本朝何負於我,而利人之官祿哉!」大將怒,擊折其脛,遂曳出殺之,
 時年三十餘。贈濟州刺史,詔有司為立碑云。



 劉德基,大興人。貞祐元年,特賜同進士出身。守官邊邑,夏兵攻城,德基坐事,積薪其傍,謂家人曰「城破即焚我。」及城破,其家人不忍縱火,遂被執。脅使跪降,德基不屈。同僚故人紿夏人曰:「此人素病狂,故敢如此。」德基曰:「為臣子當如此爾,吾豈狂耶?」夏人壯其義,乃繫諸獄,冀其改圖。已而召問,德基大罵,終不能從,曰:「吾豈茍生者哉!」遂害之。贈朝列大夫、同知通遠軍節度使事。



 王毅,大興人。經義進士,累官東明令。貞祐二年,東明圍急,毅率民兵願戰者數百人拒守。城破,毅猶率眾抗戰,
 力窮被執,與縣人王八等四人同驅之郭外。先殺二人,王八即前跪將降,毅以足踣之,厲聲曰:「忠臣不佐二主,汝乃降乎!」驅毅者以刃斫其脛,毅不屈而死。贈曹州刺史。



 王晦,字子明,澤州高平人。少負氣自渼,常慕張詠之為人,友妻與人有私,晦手刃殺之。中明昌二年進士,調長葛主簿,有能聲。察廉除遼東路轉運司都勾判官,提刑司舉其能,轉北京轉運戶籍判官。遷安陽令,累除簽陜西西路按察司事,改平涼治中。召為少府少監,遷戶部郎中。貞祐初,中都戒嚴,或舉晦有將帥才,俾募人自將,
 得死士萬餘統之。率所統衛送通州粟入中都,有功,遷霍王傅。以部兵守順州。通州圍急,晦攻牛欄山以解通州之圍。賜賚優渥,遷翰林侍讀學士,加勸農使。九月,順州受兵,晦有別部在滄、景,遣人突圍召之,眾皆踴躍思奮,而主者不肯發。王臻,晦之故部曲也,免胄出見,且拜曰:「事急矣,自苦何為,茍能相從,可不失富貴。」晦曰:「朝廷何負汝耶?」臻曰:「臻雖負國,不忍負公。」因泣下。晦叱曰:「吾年六十,致位三品,死則吾分,詎從汝耶。」將射之,臻掩泣而去。無何,將士縋城出降,晦被執,不肯降,遂就死。



 初,晦就執,謂其愛將牛斗曰:「若能死乎?」曰:「斗蒙公見知,安忍
 獨生。」併見殺。詔贈榮祿大夫、樞密副使,仍命有司立碑,歲時致祭。錄其子汝霖為筆硯承奉。



 齊鷹揚,淄州軍事判官。楊敏中,屯留縣尉致仕。張乞驢,淄州民。貞祐初,大元兵取淄州,鷹揚等募兵備禦,城破,率眾巷戰。鷹揚等三人創甚被執,欲降之,鷹揚伺守者稍怠,即起奪槊殺數人,與敏中、乞驢皆不屈以死。詔贈鷹揚嘉議大夫、淄州刺史,仍立廟于州,以時致祭。敏中贈昭勇大將軍、同知橫海軍節度使事。乞驢特贈宣武將軍、同知淄州軍州事。



 術甲法心,薊州猛安人。官至北京副留守。貞祐二年,為
 提控,與同知順州軍州事溫迪罕咬查剌俱守密雲縣。法心家屬在薊州,大元兵得之,以示法心曰:「若速降當以付汝,否則殺之。」法心曰:「吾事本朝受厚恩,戰則速戰,終不能降也,豈以家人死生為計耶。」城破,死于陣。咬查剌被執,亦不屈而死。



 盤安軍節度判官蒲察颭舍與雞澤縣令溫迪罕十方奴同守薊州,眾潰而出,颭舍、十方奴死之。



 詔贈法心開府儀同三司、樞密副使,封宿國公,咬查剌鎮國上將軍、順州刺史,颭舍金紫光祿大夫、薊州刺史,十方奴鎮國上將軍、薊州刺史。仍命樹碑,以時致祭



 高錫,字永之,德基子。以廕補官。積勞調淄州酒使,課最。遷平鄉令。察廉遷遼東路轉運支度判官、太倉使、法物庫使、兼尚林置直長、提舉都城所,歷北京、遼東轉運副使、同知南京路轉運使事。貞祐初,累遷河北東路按察轉運使。城破,遂自投城下而死。



\end{pinyinscope}