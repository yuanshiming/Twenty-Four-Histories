\article{志第二十}

\begin{pinyinscope}

 樂上



 《傳》曰:「王者功成作樂,治定制禮。」豈二帝三王之彌文哉!蓋有天下者,將一軌度、正民俗、合人神、和上下,舍禮樂何以焉。金初得宋,始有金石之樂,然而未盡其美也。及乎大定、明昌之際,日修月葺,粲然大備。其隸太常者,即郊廟、祀享、大宴、大朝會宮縣二舞是也。隸教坊者,則有鐃歌鼓吹,天子行幸鹵簿導引之樂也。有散樂。有渤海
 樂。有本國舊音,世宗嘗寫其意度為雅曲,史錄其一,其俚者弗載云。



 ○雅樂



 凡大祀、中祀、天子受冊寶、御樓肆赦、受外國使賀則用之。初,太宗取汴,得宋之儀章鐘磬樂虡,挈之以歸。皇統元年,熙宗加尊號,始就用宋樂,有司以鐘磬刻「晟」字者犯太宗諱,皆以黃紙封之。大定十四年,太常始議:「歷代之樂各自為名,今郊廟社稷所用宋樂器犯廟諱,宜皆刮去,更為製名。」於是,命禮部、學士院、太常寺撰名,乃取大樂與天地同和之義,名之曰「太和」。文、武二舞。皇統年間,定文舞曰《仁豐道洽之舞》,武舞曰《功成治定之
 舞》。《貞元儀》又改文舞曰《保大定功之舞》,武舞曰《萬國來同之舞》。大定十一年又有《四海會同之舞》,於是一代之制始備。



 明昌五年,詔用唐、宋故事,置所,講議禮樂。有司謂:「雅樂自周、漢以來止存大法,魏、晉而後更造律度,訖無定論。至後周保定中,得古玉斗於地中,以造尺律,其後牛弘以為不可,止用蘇綽鐵尺,至隋亦用之。唐興,因隋樂不改,及黃巢之亂,樂縣散失,太常博士殷盈孫以周法鑄穀鐘、編鐘,處士蕭承訓等校石磐,合而奏之。至周顯德以黍定律,議者謂比唐樂高五律。宋初亦用王朴所制樂,時和峴以周顯德律音近哀思,乃依西京銅
 望臬、石尺重造十二管,取聲下王朴一律。景祐初,李照取黍累尺成律,以其聲猶高,更用太府布帛尺,遂下太常樂三律。皇祐中,阮逸、胡瑗改造止下一律,或謂其聲弇鬱不和,依舊用王朴樂。元豐間,楊傑參用李照鐘磬加四清聲,下王朴樂二律,以為新樂。元祐間,范鎮又造新律,下李照樂一律,而未用。至崇寧間,魏漢津以范鎮知舊樂之高,無法以下之,乃以時君指節為尺,其所造鐘磐即今所用樂是也。然以王朴所制聲高,屢命改作,李照以太府尺制律,人習舊聽疑於太重。其後范鎮等論樂,復用李照所用太府尺、即周、隋所用鐵尺,牛弘等
 以謂近古合宜者也。今取見有樂,以唐初開元錢校其分寸亦同,則漢津所用指尺殆與周、隋、唐所用之尺同矣。漢津用李照、范鎮之說,而恥同之,故用時君指節為尺,使眾人不敢輕議。其尺雖為詭說,其制乃與古同,而清濁高下皆適中,非出於法數之外私意妄為者也。蓋今之鐘磬雖崇寧之所製,亦周、隋、唐之樂也。閱今所用樂律,聲調和平,無太高太下之失,可以久用。唯辰鐘、辰磬自昔數缺,宜補鑄辰鐘十五,辰磬二十一,通舊各為二十四虡。」上曰:「嘗觀宋人論樂,以為律主於人聲,不當泥於其器,要之在聲和而已。」於是,命禮部符下南京,取
 宋舊工,更鑄辰鐘十有二。又以舊鐘姑洗、夷則皆高五律,無射高二律,別鑄以補之,乃協。又琢辰磬各十有二,以其半少劣,擇其諧者而用之。初,正隆間,海陵營太廟于汴,貞祐南遷,宣宗修之,以祔諸帝神主。其地,故宋景靈宮之址也。掘其下,得編鐘十三,編磬八,皆刻「大晟」字。時朝廷多故,禮器散亡,竟亦不能備也,



 大定十一年,太常議:「按《唐會耍》舊制,南北郊宮縣用二十架,周、漢、魏、晉、宋、齊六朝及唐《開元》、宋《開寶禮》,其數皆同。《宋會耍》用三十六架,《五禮新儀》用四十八架,其數多,似乎太侈。今擬《太常因革禮》,天子宮縣之樂三十六虡,宗廟與殿庭同,
 郊丘則二十虡,宜用宮縣二十架,登歌編鐘、編磬各一虡。又按《周禮大司樂》:『凡樂,圜鐘為宮,黃鐘為角,太蔟為徵,姑洗為羽。雷鼓、雷鞀、孤竹之管、雲和之琴瑟、雲門之舞,冬日至於地上之圜丘奏之,若樂六變,則天神皆降,可得而禮矣。』六變,謂六成也。唐、宋因之。蓋圜鐘,夾鐘也,用為宮者以上應房、心,有天帝明堂之象也。宮聲三奏,角徵羽各一奏,合陽之奇數,欲神聽之也。凡樂起於陽,至少陰而止,圜鐘自卯至申其數有六,故六變而樂止,則天神皆降,可得而禮也。樂曲之名,唐以『和』,宋以『安』,本朝定樂曲以『寧』為名,今止有太廟祫享樂曲,而郊祀樂曲
 未備。皇統九年拜天用《乾寧之曲》,今圜丘降神固可就用。今太廟祫享,皇帝升降行止奏《昌寧之曲》,迎俎奏《豐寧之曲》,酌獻、舞出入奏《肅寧之曲》,飲福奏《福寧之曲》,宋《開寶禮》亦可就用。餘有郊祀曲名,皇帝入中濆、奠玉幣、迎俎、酌獻、舞出入樂曲,宜皆以『寧』字製名。」遂命學士院撰焉。皇帝入中濆奏《昌寧之曲》,降神、送神奏《乾寧之曲》,昊天上帝奏《洪寧之曲》,皇地祇奏《坤寧之曲》,配位奏《永寧之曲》,飲福奏《福寧之曲》,升降、望燎、出入大小次,並與入中濆同,餘載儀注及樂章。又命太常議文武二舞所當先後,太常議:「按唐、宋郊廟之禮,並先文後武,本朝自
 行禘祫之禮亦然。惟唐韋萬石建議謂先儒相傳,以揖讓得天下則先奏文,以征伐得天下則先奏武。當時雖從,尋復改之。其以《開元禮》先文後武為定。方丘如圜丘之儀,社稷則用登歌。」



 宗廟。皇帝入門,宮縣以無射宮,升殿,登歌以夾鐘,皆奏《昌寧之曲》。迎神、送神奏《來寧之曲》,九成。天德二年,晨稞畢,還小次,方奏迎神曲。大定十一年,朝享,奏依《開元》、《開寶禮》,至版位,即奏黃鐘宮三、大呂角二、太蔟徵二、應鐘羽二,曲詞皆同。進俎,奏《豐寧之曲》。酌獻,宮縣奏無射《大元之曲》。諸室之曲,德帝曰《大熙》,安帝曰《大安》,獻祖曰《大
 昭》,昭祖曰《大成》,景祖曰《大昌》,世祖曰《大武》,肅宗曰《大明》,穆宗曰《大章》,康宗曰《大康》,太祖曰《大定》,太宗曰《大惠》,熙宗曰《大同》,睿宗曰《大和》,昭德皇后廟曰《儀坤》,世宗曰《大鈞》,顯宗曰《大寧》,章宗曰《大隆》,宣宗曰《大慶》。皇帝還版位及亞終獻,皆奏無射宮《肅寧之曲》。飲福,登歌奏夾鐘宮《福寧之曲》。徹豆,奏《豐寧之曲》,皆用無射宮。大定十二年制,祫禘時享有司攝事,初獻盥洗,奏無射宮《肅寧之曲》。升階,登歌奏夾鐘宮《嘉寧之曲》。餘並與親享同。其別廟昭德皇后、宣孝太子所用,並載儀注、樂章。



 舊制,太廟、皇考廟樂工各三十九人。大定二十九年,升祔顯宗,有司
 以為:「宋之太廟、別廟,堂上樂各四十八人,今之樂工少十八人,擬令皇考廟舊樂工皆充兩廟堂上樂,以應前代九十六人之數。」尚書省議:「古樂工無定數。」遂奏太廟、別廟通以百人為定。明昌六年,創設宮縣,樂工一百五十六人。承安三年,敕:「祭廟用教坊奏古樂,非禮也。其自今召百姓材美者,給以食直,教閱以待用。」泰和元年,命宮縣樂工月給錢粟二貫石,遇正樂工闕,驗色收補。四年,尚書省奏:「宮縣樂工總用二百五十六人,而舊所設止百人,時或用之即以貼部教坊閱習。自明昌間,以渤海教坊兼習,而又創設九十二人。且宮縣之樂行大
 禮乃始用之,若其數復闕,但前期遣漢人教坊及大興府樂人習之,亦可備用。」遂詔罷創設者。宣宗南遷,祔諸帝主於汴京太廟。禮官言:「祔享禮畢,車駕還宮,至承天門外,百官奉迎,宮縣奏《采茨》。」以樂虡未備,遂止用教坊樂。哀宗遷蔡,天興二年七月丁巳,太祖、太宗及后妃御容至自汴京,奉安於乾元寺。左宣徽使溫敦七十五奏當用樂。上曰:「樂須太常,奈何?」七十五曰:「市有優樂,可假用之。」權左右司員外郎王鶚奏曰:「世俗之樂,豈可施于帝王之前?」遂止。



 樂舞名數。太廟登歌,鐘一虡,磬一虡,歌工四,籥二,塤二,
 篪二,笛二,巢笙二,和笙二,簫二,七星匏一,九耀匏一,閏餘匏一,搏拊二,柷一,敔一,麾一,一弦琴、三弦琴、五弦琴、七弦琴、九弦琴各二,瑟四。別廟登歌並同。親祠則用金鐘、玉磬,攝祭則用編鐘、編磬。宮縣樂三十六虡:編鐘十二虡,編磬十二虡,大鐘、穀鐘、特磬各四虡。建鼓、應鼓、鞞鼓各四,路鼓二,路鞀二,晉鼓一,巢笙、竽笙各十,簫十,籥十,篪十,笛十,塤八,一絃琴三,三絃、五絃、七絃、九弦琴各六,瑟十二,柷一,敔一,麾一。文舞所執籥、翟各六十四,武舞所執朱干、玉戚各六十四,引舞所執旌二,纛二,牙杖二,單鞀二,單鐸二,雙鐸二,金鐃二,金錞二,金鉦二,相鼓
 二,雅鼓二。有司攝祭,宮縣二十虡:編鐘四,編磬四,辰鐘十二。建鼓四,路鼓四,路鞀二,晉鼓一,巢笙、竽笙、簫、塤、篪、笛各八,一絃琴三,三絃、五絃、七弦、九絃琴各六,瑟八,柷、敔各一,麾一。登歌及二舞引舞所執與親祠同。



 皇帝受冊寶。前期,大樂令與協律郎設樂縣於殿廷。又設舉麾位二,一於殿西階,一於樂縣西北。又設登歌樂架於殿上。至日,侍中奏:「外辦。」宮縣樂作,皇帝乃出,即坐,樂止。奉寶入門,樂作,置褥位上,樂止。初引時宮縣樂作,至位立定,樂止。寶初行,樂作,至御前置訖,樂止。皇帝受寶訖,樂作,侍中奏:「稱賀。」樂止。皇太子升殿,登歌樂作,復
 位,樂止。侍中奏:「禮畢。」宮縣樂作,皇帝還幕次,樂止。



 御樓宣赦。前期,大樂署設宮縣於樓下,又設鼓一於宮縣之左。至日,金雞初立,大樂署擊鼓,立訖,鼓止。侍中奏:「外辦。」大樂令撞黃鐘之鐘,右五鐘皆應,《昌寧之樂》作,皇帝乃出。宣讀訖,百官舞蹈,禮畢,大樂令撞蕤賓之鐘,左五鐘皆應,《昌寧之樂》作,皇帝降座,樂止。凡皇帝出入升降及分班合班,皆樂作,坐、立定乃止。其冊命中宮、皇太子、太孫,受外國使賀。宴外國使,皆用宮縣。



 ○散樂



 元日、聖誕稱賀,曲宴外國使,則教坊奏之。其樂器名曲不傳。皇統二年宰臣奏:「自古並無伶人赴朝參之
 例,所有教坊人員只宜聽候宣喚,不合同百寮赴起居。」從之。章宗明昌二年十一月甲寅,禁伶人不得以歷代帝王為戲及稱萬歲者,以不應為事重法科。泰和初,有司又奏太常工人數少,即以渤海,漢人教坊及大興府樂人兼習以備用。



 ○鼓吹樂



 馬上樂也。天子鼓吹、橫吹各有前、後部,部又各分二節。金初用遼故物,其後雜用宋儀。海陵遷燕及大定十一年鹵簿,皆分鼓吹為四節,其他行幸惟用兩部而已。



 △前部第一:



 鼓吹令二人



 㧏鼓十二金鉦十二



 大鼓百二十長鳴百二十



 鐃鼓一十二歌二十四



 拱辰管二十四簫二十四



 笳二十四大橫吹一百二十



 △前部第二:



 節鼓二笛二十四



 簫二十四篳篥二十四



 笳二十四桃皮篳篥二十四



 㧏鼓十二金鉦十二



 小鼓百二十中鳴百二十



 羽葆鼓十二歌二十四



 拱辰管十四簫二十四



 △後部第一:



 鼓吹丞二人



 㧏鼓三金鉦三



 羽葆鼓十二歌二十四



 拱辰管二十四簫二十四



 笳二十四節鼓二



 鐃鼓十二歌十六



 簫二十四笳二十四



 小橫吹百二十



 △後部第二:



 笛二十四簫二十四



 篳篥二十四笳二十四



 桃皮篳篥二十四



 ○本朝樂曲



 世宗大定九年十一月庚申,皇太子生日,上宴于東宮,命奏新聲,謂大臣曰:「朕製此曲,名《君臣樂》,今天下無事,
 與卿等共之,不亦樂乎?」辭律不傳。十三年四月乙亥,上御睿思殿,命歌者歌女直詞,顧謂皇太子曰:「朕思先朝所行之事,未嘗暫忘,故時聽此詞,亦欲令汝輩知女直醇質之風。至於文字、語言或不通曉,是忘本也!」二十五年四月,幸上京,宴宗室於皇武殿,飲酒樂,上諭之曰:「今日甚欲成醉,此樂不易得也。昔漢高祖過故鄉,與父老歡飲,擊築而歌,令諸兒和之。彼起布衣,尚且如是,況我祖宗世有此土,今天下一統,朕巡幸至此,何不樂飲!」于時宗室婦女起舞,進酒畢,群臣故老起舞,上曰:「吾來故鄉數月矣,今迴期已近,未嘗有一人歌本曲者,汝曹來
 前,吾為汝歌。」乃命宗室子敘坐殿下者皆上殿,面聽上歌。曲道祖宗創業艱難,及所以繼述之意。上既自歌,至慨想祖宗音容如睹之語,悲感不復能成聲,歌畢,泣下數行。右丞相元忠暨群臣宗戚捧觴上壽,皆稱萬歲。於是諸老人更歌本曲,如私家相會,暢然歡洽。上復續調歌曲,留坐一更,極歡而罷。其辭曰:



 猗歟我祖,聖矣武元。誕膺明命,功光于天。拯溺救楚,深根固蒂。克開我後,傳福萬世。無河海陵,淫昏多罪。反易天道,荼毒海內。自昔肇基,至于繼體。積累之業,淪胥且墜。望戴所歸,不謀同意。宗廟至重,人心難拒。
 勉副樂推,肆予嗣緒。二十四年,兢業萬幾。億兆庶姓,懷保安綏。國家閒暇,廓然無事。乃眷上都,興帝之第。屬茲來游,惻然予思。風物減耗,殆非昔時。于鄉于里,皆非初始。雖非初始,朕自樂此。雖非昔時,朕無異視。瞻戀慨想,祖宗舊宇。屬屬音容,宛然如睹。童嬉孺慕,歷歷其處。壯歲經行,恍然如故。舊年從游,依俙如昨。歡誠契闊,旦暮之若。于嗟闊別兮,云胡不樂。



 郊祀樂歌



 皇帝入中濆,宮縣黃鐘宮《昌寧之曲》:凡步武同。



 袞服穆穆,臨于中濆。瞻言圜壇,皇皇后帝。禋祀肇稱,
 磬香維德。爰暨百神,於昭受職。



 降神,宮縣《乾寧之曲》、《仁豐道洽之舞》。圜鐘為宮,黃鐘為角,太蔟為徵,姑洗為羽。圜鐘三奏,黃鐘、太蔟、姑洗皆一奏,詞並同:



 我金之興,皇天錫羨。惟神之休,爰茲郊見。有玉其禮,有牲其薦。將受厥明,來寧來燕。



 皇帝盥洗,宮縣黃鐘宮《昌寧之曲》:



 因天事天,惇宗將禮。爰飭攸司,奉時罍洗。挹彼注茲,迺升壇陛。先事而虔,神勞豈弟。



 皇帝升壇,登歌大呂宮《昌寧之曲》:



 相在國南,崇崇其趾。烝哉皇王,維時蒞止。至誠通神,克禋克祀。於萬斯年,昊天其子。



 昊天上帝,奠玉幣,登歌大呂宮《洪寧之曲》:



 穆穆君王,有嚴有翼。珮環鏘然,圜壇是陟。嘉德升聞,馨非黍稷。高明降監,百神受職。



 皇地祇,《坤寧之曲》:



 肅敬明祇,躬行奠贄。其贄維何?黃琮制幣。從祀群靈,咸秩厥位。惟皇能饗,允集熙事。



 配位太祖皇帝,《永寧之曲》:



 肇舉明禋,皇天后土。皇祖武元,爰作神主。功昭耆定,
 歌以大呂。綏我思成,有秩斯祜。



 司徒迎俎,宮縣黃鐘宮《豐寧之曲》:



 穆穆皇皇,天子躬祀。群臣相之,罔不敬止。俎豆畢陳,物其嘉矣。馨香始升,明神燕喜。



 昊天上帝,酌獻,登歌大呂宮《嘉寧之曲》:



 郊禋展敬,昭事上靈。太尊在席,有醑斯馨。酌言獻之,靈其醉止。福祿來宜,以答明祀。



 皇地祇,《泰寧之曲》:



 袞服穆穆,臨彼泰折。於昭神宮,埋幣瘞血。爰稱匏爵,斟言薦潔。方輿常安,扶我帝業。



 配位太祖皇帝,《燕寧之曲》:、



 烝哉高后,肇迪丕基。功與天合,配天以推。薦時清旨,孔肅其儀。來寧來燕,福祿綏之。



 文舞退,武舞進,宮縣黃鐘宮《咸寧之曲》:



 奉祀郊丘,《雲門》變舞。進秉朱干,停揮翟羽。於昭睿文,復肖聖武。無疆維烈,天子受祜。



 亞終獻,宮縣黃鐘宮《咸寧之曲》、《功成治定之舞》:



 掃地南郊,天神以俟,於皇君王,克禋克祀。交於神明,玄酒陶器。誠心靖純,非貴食味。



 皇帝飲福,登歌大呂宮《福寧之曲》:



 所以承天,無過乎質。天其祐之,惟精惟一。泰尊爰挹,馨香薦德。惠我無疆,子孫千億。



 徹豆,登歌大呂宮《豐寧之曲》:



 大禮爰陳,為豆孔碩。肅肅其容,於顯百辟。皇靈降監,馨聞在德。明禋斯成,孚休罔極。



 送神,宮縣圜鐘宮《乾寧之曲》:



 赫赫上帝,臨監禋祀。居然來歆,昭答祖配。圜壇四成,神安其位。升歌贊送,天人悅喜。



 方丘樂歌



 迎神,《鎮寧之曲》。大鐘宮再奏,太蔟角再奏,姑洗徵再奏,
 南呂羽再奏,詞同:



 至哉坤儀,萬匯資生。稱物平施,流謙變盈。禮修泰折,祭極精誠。皇皇靈眷,永奠寰瀛。



 初獻盥洗,太蔟宮《肅寧之曲》:



 禮有五經,無先祭禮。即時伸虔,惟時盥洗。品物吉蠲,威儀濟濟。錫之純嘏,來歆愷悌。



 初獻升壇,應鐘宮《肅寧之曲》:



 無疆之德,至哉坤元。沉潛剛克,資生實蕃。方丘之儀,惟敬無文。神其來思,時歆薦殷。



 初獻奠玉幣,太蔟宮《億寧之曲》:



 禮行方澤,文物備舉。惟皇地祇。昭假來下。奠瘞玉帛,純誠內著。神保是享,陟降斯祜。



 司徒捧俎,太蔟宮《豐寧之曲》:



 四階秩儀,壇於方澤。昭事皇祇,即陰以墌。潔肆於祊,孔嘉且碩。神其福之,如幾如式。



 正位酌獻,太蔟宮《溥寧之曲》:



 蕩蕩坤德,物無不載。柔順利貞,含洪光大。籩豆既陳,金石斯在。四海永寧,福祿攸介。



 配位酌獻配太宗也,太蔟宮《保寧之曲》:



 詞闕。



 亞終獻升壇,太蔟宮《咸寧之曲》:



 卓彼嘉壇,奠玉方澤。百辟祇肅,八音純繹。祀事孔明,柔祇感格。



 徹豆,應鐘宮《豐寧之曲》:



 修理方丘,吉蠲是宜。籩豆靜嘉,登於有司。芬芬馨香,來享來儀。郊儀將終,聲歌徹之。



 送神,林鐘宮《鎮寧之曲》:



 因地方丘,濟濟多儀。樂成八變,靈祇格思。薦餘徹豆,神貺昭垂。億萬斯年,永祐丕基。



 詣望燎位,太蔟宮《肅寧之曲》。詞同升壇。



\end{pinyinscope}