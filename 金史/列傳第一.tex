\article{列傳第一}

\begin{pinyinscope}

 后妃上



 ○始祖明懿皇后德帝思皇后安帝節皇后獻祖恭靖皇后昭祖威順皇后景祖昭肅皇后世祖翼簡皇后肅宗靖宣皇后穆宗貞惠皇后康宗敬僖皇后太祖聖穆皇后太祖光懿皇后
 太祖欽憲皇后太祖宣獻皇后太祖崇妃蕭氏太宗欽仁皇后熙宗悼平皇后海陵嫡母徒單氏海陵母大氏海陵后徒單氏海陵諸嬖附



 古者天子娶后,三國來媵,皆有娣姪,凡十二女。諸侯一娶九女。所以正嫡妾,廣繼嗣,息妒忌,防淫慝,塞禍亂也。后亡,則媵為繼室,各以其敘。無三媵,則娣姪繼室,亦各以其敘。繼室者,治其內政,不敢正其位號。禮,廟無兩祔,不並尊也。魯成風始兩祔,宋國三媵,齊管氏三歸,《春秋》皆譏之。《周禮》內宰,其屬則內小臣、閽人、寺人次之,九嬪、
 世婦、女御、女祝、女史、典婦功、典絲、典枲、內司服又次之。《昏義》稱「后立六宮、三夫人、九嬪、二十七世婦、八十一御妻」,不與《春秋》、《周禮》合,後世因仍其說,後宮遂至數千。



 金代,后不娶庶族,甥舅之家有周姬、齊姜之義。國初諸妃皆無位號,熙宗始有貴妃、賢妃、德妃之號。海陵淫嬖,後宮浸多,元妃、姝妃、惠妃、貴妃、賢妃、宸妃、麗妃、淑妃、德妃、昭妃、溫妃、柔妃凡十二位。大定後宮簡少,明昌以後大備。



 內官制度:諸妃視正一品,比三夫人。昭儀、昭容、昭媛、修儀、修容、修媛、充儀、充容充媛視正二品,比九嬪。婕妤九人視正三品。美人九人視正四品,才人九人視正五
 品,比二十七世婦。寶林二十七人視正六品,御女二十七人視正七品,采女二十七人視正八品,比八十一御妻。又有尚宮、尚儀、尚服、尚食、尚寢、尚功,皆內官也。



 太祖嫡后聖穆生景宣,光懿生宗乾,有定策功,欽憲有保佑之功,故自熙宗時聖穆、光懿、欽憲皆祔。宣獻生睿宗,大定祔焉。故太祖廟祔四后,睿、世、顯、宣皆祔兩后,惟太宗、景宣、熙宗、章宗室祔一后。貞、慈、光獻、昭聖雖庶姓,皆以子貴。宣宗冊溫敦氏,乃賜姓,變古甚矣。故自初起至於國亡,列其世次,著其族里,可考鑒焉。其無與於世道者,置不錄。



 始祖明懿皇后,完顏部人。年六十餘嫁始祖。天會十五年追謚。



 德帝思皇后,不知何部人。天會十五年追謚。



 安帝節皇后,不知何部人。天會十五年追謚。



 獻祖恭靖皇后,不知何部人。天會十五年追謚。



 昭祖威順皇后徒單氏,諱烏古論都葛,活刺渾水敵魯鄉徒單部人。其父拔炭都魯海。后性剛毅,人莫敢以為室。獻祖將為昭祖娶婦,曰:「此子勇斷異常,柔弱之女不可以為配。」乃為昭祖娶焉。天會十五年追謚。



 景祖昭肅皇后,唐括氏,帥水隈鴉村唐括部人,諱多保
 真。父石批德撒骨只,巫者也。后有識度,在父母家好待賓客,父母出,則多置酒饌享鄰里,迨于行旅。景祖飲食過人,時人名之「活羅」,解在《景祖紀》。昭祖曰:「儉嗇之女吝惜酒食,不可以配。」烏古乃聞后性度如是,乃娶焉。



 遼使同乾來伐五國浦聶部,景祖使后與劾孫為質於拔乙門,而與同乾襲取之,遼主以景祖為節度使。



 后雖喜賓客,而自不飲酒。景祖與客飲,后專聽之。翌日,枚數其人所為,無一不中其啟肯。有醉而喧呶者,輒自歌以釋其忿爭。軍中有被笞罰者,每以酒食慰諭之。景祖行部,輒與偕行,政事獄訟皆與決焉。



 景祖沒後,世祖兄弟凡用
 兵,皆稟於后而後行,勝負皆有懲勸。農月,親課耕耘刈獲,遠則乘馬,近則策杖,勤於事者勉之,晏出早休者訓勵之。



 后往邑屯村,世祖、肅宗皆從。會桓赧、散達偕來,是時已有隙,被酒,語相侵不能平,遂舉刃相向。后起,兩執其手,謂桓赧、散達曰:「汝等皆吾夫時舊人,奈何一旦遽忘吾夫之恩,與小兒子輩忿爭乎。」因自作歌,桓赧、散達怒乃解。其後桓赧兄弟起兵來攻,當是時,肅宗先已再失利矣,世祖已退烏春兵,與桓赧戰于北隘甸。部人失束寬逃歸,袒甲而至,告曰:「軍敗矣。」后方憂懣,會康宗來報捷,后乃喜。既而桓赧、散達皆降。



 后不妒忌,闊略女
 工,能輯睦宗族,當時以為有丈夫之度雲。天會十五年追謚。



 世祖翼簡皇后,拿懶氏。大安元年癸酉歲卒。天會十五年追謚。



 肅宗靖宣皇后,蒲察氏。太祖將舉兵,入告于后。后曰:「汝邦家之長,見可則行。吾老矣,無貽我憂,汝亦必不至是。」太祖奉觴為壽,即奉后出門,驩酒禱天。后命太祖正坐,號令諸將。自是太祖每出師還,輒率諸將上謁,獻所俘獲。天會十五年追謚。



 穆宗貞惠皇后,烏古論氏。天會十五年追謚。



 康宗敬僖皇后,唐括氏。天會十五年追謚。



 太祖聖穆皇后,唐括氏。天會十三年追謚」。仍贈后父留速太尉、榮國公,祖迭胡本司徒、英國公,曾禱劾迺司空、溫國公。



 太祖光懿皇后,裴滿氏。天會十三年追謚。



 太祖欽憲皇后,紇石烈氏。天會十三年,尊為太皇太后,宮號慶元。十四年正月己巳朔,熙宗朝于慶元宮,然後御乾元殿,受群臣賀。是月丁丑,崩于慶元宮。二月癸卯,祔葬睿陵。



 太祖宣獻皇后,僕散氏,睿宗母也。天會十三年,追冊曰
 德妃。大定元年追謚。



 崇妃,蕭氏。熙宗時封貴妃。天德二年正月,封元妃。是月,尊封太妃。海陵母大氏事蕭氏甚謹。海陵篡立,尊大氏為皇太后,居永寧宮。每有宴集,太妃坐上坐,大氏執婦禮。海陵積不能平,及殺宗義等,誣太妃以隱惡,殺之,併殺所生子任王隈喝。



 大定十九年,詔改葬。大宗正丞宗安監護葬事,遣使致祭。上欲復太妃舊號,下禮官議。「前代稱太妃者皆以子貴。古者入廟稱『后』繫夫,在朝稱『太』繫子,與今蕭妃事不同,恐不得稱『太』,止當追封妃號。」詔從之,乃封崇妃云。



 太宗欽仁皇后,唐括氏。熙宗即位,與太祖欽憲皇后俱尊為太皇太后,號明德宮。贈后父阿魯束太尉、宋國公,祖寬匹司徒、英國公,曾祖阿魯瑣司空、溫國公。十四年正月己巳朔,上朝兩宮太后,然後御乾元殿受賀,自後歲以為常。皇統元年,上自燕京還京師,朝謁于明德宮。明年,上如天開殿,皇子生,使使馳報太后。太后至大開殿,上與皇后親迎之。三年,崩于明德宮。謚曰欽仁皇后,祔葬恭陵。



 熙宗悼平皇后,裴滿氏。熙宗即位,封貴妃。天眷元年,立為皇后。父忽達拜太尉,贈曾祖斜也司空,祖鶻沙司徒。
 皇統元年,熙宗受尊號,冊為慈明恭孝順德皇后。二年,太子濟安生。是歲,熙宗年二十四,喜甚,乃肆赦,告天地宗廟。彌月,冊為皇太子,未一歲薨。



 熙宗在位,宗翰、宗乾、宗弼相繼秉政,帝臨朝端默。雖初年國家多事,而廟算制勝,齊國就慶,宋人請臣,吏清政簡,百姓樂業。宗弼既沒,舊臣亦多物故,后干預政事,無所忌憚,朝官往往因之以取宰相。濟安薨後,數年繼嗣不立,后頗掣製熙宗。熙宗內不能平,因無聊,縱酒酗怒,手刃殺人。左丞相亮生日,上遣大興國以司馬光畫像、玉吐鶻、廄馬賜之,后亦附賜生日禮物。熙宗聞之,怒,遂杖興國而奪回所賜。
 海陵本懷覬覦,因之疑畏愈甚,蕭牆之變,從此萌矣。近侍高壽星隨例遷屯燕南,入訴於后,后激怒熙宗,殺左司郎中三合,杖平章政事秉德,而壽星竟得不遷。秉德、唐括辯之姦謀起焉,海陵乘之,以成逆亂之計。



 久之,熙宗積怒,遂殺后,而納胙王常勝妃撒卯入宮繼之。又殺德妃烏古論氏,妃夾谷氏、張氏、裴滿氏。明日,熙宗遇弒。海陵已弒熙宗,欲收人心,以后死無罪,降熙宗為東昏王,追謚后為悼皇后,封后父忽達為王。大定間,復熙宗帝號,加謚后為悼平皇后,祔葬思陵。



 海陵嫡母,徒單氏。宗乾之正室也。徒單無子,次室李氏
 生長子鄭王充,次室大氏生三子,長即海陵庶人也。徒單氏賢,遇下有恩意,大氏事之甚謹,相得歡甚。徒單雖養充為己子,充與海陵俱為熙宗宰相,充嗜酒,徒單常責怒之,尤愛海陵。海陵自以其母大氏與徒單嫡妾之分,心常不安。及弒熙宗,徒單與太祖妃蕭氏聞之,相顧愕然曰:「帝雖失道,人臣豈可至此。」徒單入宮見海陵,不曾賀,海陵銜之。



 天德二年正月,徒單與大氏俱尊為皇太后。徒單居東宮,號永壽宮,大氏居西宮,號永寧宮。天德二年,太后父蒲帶與大氏父俱贈太尉,封王。徒單太后生日,酒酣,大氏起為壽。徒單方與坐客語,大氏跽者
 久之。海陵怒而出。明日,召諸公主宗婦與太后語者皆杖之。大氏以為不可。海陵曰:「今日之事,豈能尚如前日邪。」自是嫌隙愈深。



 天德四年,海陵遷中都,獨留徒單於上京。徒單常憂懼,每中使至,必易衣以俟命。大氏在中都常思念徒單太后,謂海陵曰:「永壽宮待吾母子甚厚,慎毋相忘也。」十二月十四日,徒單氏生日,海陵使秘書監納合椿年往上京為太后上壽。貞元元年,大氏病篤,恨不得一見。臨終,謂海陵曰:「汝以我之故,不令永壽宮偕來中都。我死,必迎致之,事永壽宮當如事我。」



 三年,右丞相僕散師恭、大宗正丞胡拔魯往上京奉遷山陵,海
 陵因命永壽宮太后與俱來。繼使平章政事蕭玉迎祭祖宗梓宮於廣寧,海陵謂玉曰:「醫巫閭山多佳致。祭奠禮畢,可奏太后於山水佳處遊覽。」及至沙流河,海陵迎謁梓宮,遂謁見太后。海陵命左右約杖二束自隨,跪於太后前,謝罪曰:「亮不孝,久闕溫靖,願太后痛笞之。不然,且不安。」太后親扶起之,叱約杖者使去。太后曰:「今庶民有克家子,立百金之產,尚且愛之不忍笞。我有子如此,寧忍笞乎。」十月,太后至中都,海陵帥百官郊迎,入居壽康宮。是日,海陵及後宮、宰臣以下奉觴上壽,極歡而罷。



 海陵侍太后於宮中,外極恭順,太后坐起,自扶腋之,常
 從輿輦徒行,太后所御物或自執之。見者以為至孝,太后亦以為誠然。及謀伐宋,太后諫止之,海陵心中益不悅,每謁太后還,必忿怒,人不知其所以。



 及至汴京,太后居寧德宮。太后使侍婢高福娘問海陵起居,海陵幸之,因使伺太后動靜。凡太后動止,事無大小,福娘夫特末哥教福娘增飾其言以告海陵。及樞密使僕散師恭征契丹撒八,辭謁太后,太后與師恭語久之。大概言「國家世居上京,既徙中都,又自中都至汴,今又興兵涉江、淮伐宋,疲弊中國,我嘗諫止之,不見聽。契丹事復如此,奈何」。福娘以告海陵。海陵意謂太后以充為子,充四子皆
 成立,恐師恭將兵在外,太后或有異圖。乃召點檢大懷忠、翰林待制斡論、尚衣局使虎特末、武庫直長習失使殺太后于寧德宮,命護衛高福、辭勒、浦速斡以兵士四十人從,且戒之曰:「汝等見太后,但言有詔,令太后跪受,即擊殺之,匆令艱苦。太后同乳妹安特,多口必妄言,當令速死。」及指名太后左右數人,皆令殺之。太后方樗浦,大懷忠等至,令太后跪受詔。太后愕然,方下跪,虎特末從後擊之,仆而復起者再。高福等縊殺之,年五十三。并殺安特及郡君白散、阿魯瓦、叉察,乳母南撒,侍女阿斯、斡里保,寧德宮護衛溫迪罕查刺,直長王家奴、撒八,小
 底忽沙等。海陵命焚太后于宮中,棄其骨於水。並殺充之子檀奴、阿里白、元奴,耶補兒邇匿,歸于世宗。自軍中召師恭還,殺之。及殺阿斯子孫、撒八二子、忽沙二子。封高福娘為鄖國夫人,以特末哥為澤州刺史。海陵許福娘征南回以為妃,賜銀二千兩。敕戒特末哥:「無酗酒毆福娘,毆福娘必殺汝。」



 大定間,謚徒單氏曰哀皇后,自澤州械特末哥、福娘至中都誅之。其後貶海陵為庶人。宗斡去帝號,復封遼王,徒單氏降封遼王妃云。



 海陵母,大氏。天德二年正月,與徒單氏俱尊為皇太后。大氏居永寧宮。曾祖堅嗣贈司空,祖臣寶贈司徒,父昊
 天贈太尉、國公,兄興國奴贈開府儀同三司、衛國公。十一月,昊天進封為王。



 三年正月十六日,海陵生日,宴宗室百官於武德殿。大氏懽甚,飲盡醉。明日,海陵使中使奏曰:「太后春秋高,常日飲酒不過數杯,昨見飲酒沉醉。兒為天子,固可樂,若聖體不和,則子心不安,其樂安在。至樂在心,不在酒也。」及遷中都,永壽宮獨留上京,大氏常以為言。



 貞元元年四月,大氏有疾,詔以錢十萬貫求方藥。及病篤,遺言海陵,當善事永壽宮。戊寅,崩。詔尚書省:「應隨朝官至五月一日方治事。中都自四月十九日為始,禁樂一月。外路自詔書到日後,官司三日不治事,
 禁樂一月,聲鐘七晝夜。」



 貞元三年,大祥,海陵率後宮奠哭于菆宮。海陵將遷山陵于大房山,故大氏猶在菆宮也。九月,太祖、太宗、德宗梓宮至中都。尊謚曰慈憲皇后。海陵親行冊禮,與德宗合葬于大房山,升祔太廟。大定七年,降封海陵太妃,削去皇后謚號。及宗乾降帝號,封遼王,詔徒童單氏為妃,而大氏與順妃李氏、寧妃蕭氏、文妃徒單氏並追降為遼王夫人。



 廢帝海陵后,徒單氏。太師斜也之女。初為岐國妃,天德二年封為惠妃,九月,立為皇后。三年十一月二十一日,后生日,百僚稱賀於武德殿。久之,海陵後宮浸多,后寵
 頗衰,希得進見。沈璋妻張氏嘗為光英保母,耶律徹在北京與海陵游從,海陵使璋妻及徹妻侯氏入宮侍后。徹本名神涅,負官錢二千六百餘萬,海陵皆免之。正隆六年,海陵幸南京。六月癸亥,左丞相張浩率百官迎謁。海陵備法駕,乘玉輅,與后及太子光英共載而入。海陵伐宋,后與光英居守。海陵遇害,陀滿訛里也殺光英于汴。后至中都,居于海陵母大氏故宮。頃之,世宗憐其無依,詔歸父母家于上京,歲賜錢二千貫,奴婢皆給官廩。大定十年卒。



 海陵為人善飾詐,初為宰相,妾媵不過三數人。及踐大
 位,逞欲無厭,後宮諸妃十二位,又有昭儀至充媛九位,婕妤美人才人三位,殿直最下,其他不可舉數。初即位,封岐國妃徒單氏為惠妃,後為皇后。第二娘子大氏封貴妃,第三娘子蕭氏封昭容,耶律氏封脩容。其後貴妃大氏進封惠妃,貞元元年,進封姝妃,正隆二年,進封元妃。昭容蕭氏,天德二年,特封淑妃,貞元二年,進封宸妃。修容耶律氏,天德四年,進昭媛,貞元元年,進昭儀,三年,進封麗妃。即位之初,後宮止此三人,尊卑之敘,等威之辨,若有可觀者。及其侈心既萌淫肆蠱惑,不可復振矣。



 昭妃阿里虎,姓蒲察氏,駙馬都尉沒里野女。初嫁宗盤
 子阿虎迭。阿虎迭誅,再嫁宗室南家。南家死,是時南家父突葛速為元帥都監,在南京,海陵亦從梁王宗弼在南京,欲取阿里虎,突葛速不從,遂止。及篡位方三日,詔遣阿里虎歸父母家。閱兩月,以婚禮納之。數月,特封賢妃,再封昭妃。阿里虎嗜酒,海陵責讓之,不聽,由是寵衰。



 昭妃初嫁阿虎迭,生女重節。海陵與重節亂,阿里虎怒重節,批其頰,頗有詆訾之言。海陵聞之,愈不悅。阿里虎以衣服遺前夫之子,海陵將殺之,徒單后率諸妃嬪求哀,乃得免。



 凡諸妃位皆以侍女服男子衣冠,號「假廝兒」。有勝哥者,阿里虎與之同臥起,如夫婦。廚婢三娘以告
 海陵,海陵不以為過,惟戒阿里虎勿笞箠三娘。阿里虎榜殺之。海陵聞昭妃閤有死者,意度是三娘,曰:「若果爾,吾必殺阿里虎。」問之,果然。是月,光英生月,海陵私忌,不行戮。阿里虎聞海陵將殺之也,即不食,日焚香禱祝,冀脫死。逾月,阿里虎已委頓不知所為,海陵使人縊殺之,並殺侍婢擊三娘者。



 貴妃定哥,姓唐括氏。有容色。崇義節度使烏帶之妻。海陵舊嘗有私,侍婢貴哥與知之幔烏帶在鎮,每遇元會生辰,使家奴葛魯、葛溫詣闕上壽,定哥亦使貴哥候問海陵及兩宮太后起居。海陵因貴哥傳語定哥曰:「自古天
 子亦有兩后者,能殺汝夫以從我乎?」貴哥歸,具以海陵言告定哥。定哥曰:「少時醜惡,事已可恥。今兒女已成立,豈可為此。」海陵聞之,使謂定哥:「汝不忍殺汝夫,我將族滅汝家。」定哥大恐,乃以子烏荅補為辭,曰:「彼常侍其父,不得便。」海陵即召烏荅補為符寶祗候。定哥曰:「事不可止矣。」因烏帶醉酒,令葛溫、葛魯縊殺烏帶,天德四年七月也。海陵聞烏帶死,詐為哀傷。已葬烏帶,即納定哥宮中為娘子。貞元元年,封為貴妃,大愛幸,許以為后。每同輦遊瑤池,諸妃步從之。海陵嬖寵愈多,定哥希得見。一日獨居樓上,海陵與他妃同輦從樓下過,定哥望見,號
 呼求去,詛罵海陵,海陵陽為不聞而去。



 定哥自其夫時,與家奴閻乞兒通,嘗以衣服遺乞兒。及為貴妃,乞兒以妃家舊人,給事本位。定哥既怨海陵疏己,欲復與乞兒通。有比丘尼三人出入宮中,定哥使比丘尼向乞兒索所遺衣服以調之。乞兒識其意,笑曰:「妃今日富貴忘我耶。」定哥欲以計納乞兒宮中,恐閽者索之,乃令侍兒以大篋盛褻衣其中,遣人載之入宮。閽者索之,見篋中皆褻衣,固已悔懼。定哥使人詰責閽者曰:「我,天子妃。親體之衣,爾故玩視,何也?我且奏之。」閽者惶恐曰:「死罪。請後不敢。」定哥乃使人以篋盛乞兒載入宮中,閽者果不敢
 復索。乞兒入宮十餘日,使衣婦人衣,雜諸宮婢,抵暮遣出。貴哥以告海陵。定哥縊死,乞兒及比丘尼三人皆伏誅。封貴哥莘國夫人。



 初,海陵既使定哥殺其夫烏帶,使小底藥師奴傳旨定哥,告以納之之意。藥師奴知定哥與閻乞兒有姦,定哥以奴婢十八口賂藥師奴使無言與乞兒私事。定哥敗,杖藥師奴百五十。先是,藥師奴嘗盜玉帶當死,海陵釋其罪,逐去。及遷中都,復召為小底。及藥師奴既以匿定哥姦事被杖,後與秘書監文俱與靈壽縣主有姦,文杖二百除名,藥師奴當斬。海陵欲杖之,謂近臣曰:「藥師奴於朕有功,再杖之即死矣。」丞相李
 睹等執奏藥師奴於法不可恕,遂伏誅。海陵以葛溫、葛魯為護衛,葛溫累官常安縣令,葛魯累官襄城縣令,大定初,皆除名。



 麗妃石哥者,定哥之妹,祕書監文之妻也。海陵私之,欲納宮中。乃使文庶母按都瓜主文家。海陵謂按都瓜曰:「必出而婦,不然我將別有所行。」按都瓜以語文,文難之。按都瓜曰:「上謂別有所行,是欲殺汝也。豈以一妻殺其身乎。」文不得已,與石哥相持慟哭而訣。是時,海陵遷都至中京,遣石哥至中都,俱納之。海陵召文至便殿,使石哥穢談戲文以為笑。定哥死,遣石哥出宮。不數日復召
 入,封為修容。貞元三年,進昭儀。正隆元年,進封柔妃。二年,進麗妃。



 柔妃彌勒,姓耶律氏。天德二年,使禮部侍郎蕭拱取之於汴。過燕京,拱父仲恭為燕京留守,見彌勒身形非若處女者,歎曰:「上必以疑殺拱矣。」及入宮,果非處女,明日遣出宮。海陵心疑蕭拱,竟致之死。彌勒出宮數月,復召入,封為充媛,封其母張氏莘國夫人,伯母蘭陵郡君蕭氏為鞏國夫人。蕭拱妻擇特懶,彌勒女兄也。海陵既奪文妻石哥,卻以擇特懶妻文。既而詭以彌勒之召,召擇特懶入宮,亂之。其後,彌勒進封柔妃云。



 昭妃阿懶,海陵叔曹國王宗敏妻也。海陵殺宗敏而納阿懶宮中,貞元元年,封為昭妃。大臣奏「宗敏屬近尊行,不可」。乃令出宮。



 修儀高氏,秉德弟糺里妻也。海陵殺諸宗室,釋其婦女。宗本子莎魯刺妻、宗固子胡里刺妻、胡失來妻及糺里妻,皆欲納之宮中,諷宰相奏請行之。使徒單貞諷蕭裕曰:「朕嗣續未廣,此黨人婦女有朕中外親,納之宮中何如?」裕曰:「近殺宗室,中外異議紛紜,奈何復為此邪。」海陵曰:「吾固知裕不肯從。」乃使貞自以己意諷裕,必欲裕等請其事。貞謂裕曰:「上意已有所屬,公固止之,將成疾矣。」裕曰:「必不肯已,唯上擇焉。」貞曰:「必欲公
 等白之。」裕不得已,乃其奏,遂納之。未幾,封高氏為修儀,加其父高耶魯瓦輔國上將軍,母完顏氏封密國夫人。高氏以家事訴於海陵。海陵自熙宗時悻見悼后干政,心惡之,故自即位,不使母、后得預政事。於是,遣高氏還父母家。詔尚書省,凡后妃有請于宰相者,收其使以聞。



 昭媛察八,姓耶律氏。嘗許嫁奚人蕭堂古帶。海陵納之,封為昭媛。堂古帶為護衛,察人使侍女習捻以軟金鵪鶉袋數枚遺之。事覺。是時,堂古帶謁告在河間驛,召問之。堂古帶以實對,海陵釋其罪。海陵登寶昌門樓,以察八徇諸后妃,手刃擊之,墮門下死,并誅侍女習捻。



 壽寧縣主什古,宋王宗望女也。靜樂縣主蒲刺及習捻,梁王宗弼女也。師姑兒,宗雋女也。皆從姊妹。混同郡君莎里古真及其妹餘都,太傅宗本女也,再從姊妹。郕國夫人重節,宗磐女孫,再從兄之女。及母大氏表兄張定安妻奈刺忽、麗妃妹蒲魯胡只,皆有夫,唯什古喪夫。海陵無所忌恥,使高師姑、內哥、阿古等傳達言語,皆與之私。凡妃主宗婦嘗私之者,皆分屬諸妃,出入位下。奈刺忽出入元妃位,蒲魯胡只出入麗妃位,莎里古填、餘都出入貴妃位,什古、重節出入昭妃位,蒲刺、師姑兒出入淑妃位。海陵使內哥召什古。先於暖位小殿置琴阮其
 中,然後召之。什古已色衰,常譏其衰老以為笑。唯習捻、莎里古真最寵,恃勢笞決其夫。海陵使習捻夫稍喝押護衛直宿,莎里古真夫撒速近侍局直宿。謂撒速曰:「爾妻年少,遇爾直宿,不可令宿於家,常令宿於妃位。」每召入,必親伺候廊下,立久,則坐於高師姑膝上。高師姑曰:「天子何勞苦如此。」海陵曰:「我固以天子為易得耳。此等期會難得,乃可貴也。」每於臥內遍設地衣,惈逐以為戲。莎里古真在外為淫泆。海陵聞之大怒,謂莎里古真曰:「爾愛貴官,有貴如天子者乎。爾愛人才,有才兼文武似我者乎。爾愛娛樂,有豐富偉岸過於我者乎。」怒甚,氣咽
 不能言。少頃,乃撫慰之曰:「無謂我聞知,便爾慚恧。遇燕會,當行立自如,無為眾所測度也,恐致非笑。」後京屢召入焉。餘都,牌印松古刺妻也。海陵嘗曰:「餘都貌雖不揚,而肌膚潔白可愛。」蒲刺進封壽康公主,什古進封昭寧公主,莎里古真進封壽陽縣主,重節進封蓬萊縣主。重節即昭妃蒲察氏所生,蒲察怒重節與海陵淫,批其頰,海陵怒蒲察氏,終殺之者也。



 凡宮人在外有夫者,皆分番出入。海陵欲率意幸之,盡遣其夫往上京,婦人皆不聽出外。常令教坊番直禁中,每幸婦人,必使奏樂,撤其幃帳,或使人說淫穢語於其
 前。嘗幸室女不得遂,使元妃以手左右之。或妃嬪列坐,輒率意淫亂,使共觀。或令人效其形狀以為笑。凡坐中有嬪御,海陵必自擲一物於地,使近侍環視之,他視者殺。誡宮中給使男子,於妃嬪位舉首者刓其目。出入不得獨行,便旋,須四人偕往,所司執刀監護,不由路者斬之。日入後,下階砌行者死,告者賞錢二百萬。男女倉猝誤相觸,先聲言者賞三品官,後言者死,齊言者皆釋之。



 女使闢懶有夫在外,海陵封以縣君,欲幸之,惡其有娠,飲以麝香水,躬自揉拉其腹,欲墮其胎。闢懶乞哀,欲全性命,茍得乳免,當不舉。海陵不顧,竟墮其胎。



 蒲察阿虎
 迭女叉察,海陵姊慶宜公主所生,嫁秉德之弟特里。秉德誅,當連坐,太后使梧桐請于海陵,由是得免。海陵白太后欲納叉察。太后曰:「是兒始生,先帝親抱至吾家養之,至于成人。帝雖舅,猶父也,不可。」其後,嫁宗至安達海之子乙刺補。海陵數使人諷乙刺補出之,因而納之。叉察與完顏守誠有奸,守誠本名遏里來,事覺,海陵殺守誠,太后為叉察求哀,乃釋之。叉察家奴告叉察語涉不道,海陵自臨問,責叉察曰:「汝以守誠死詈我邪?」遂殺之。



 同判大宗正阿虎里妻浦速碗,元妃之妹,因入見元妃,海陵逼淫之。浦速碗自是不復入宮。



 世宗為濟南尹,海
 陵召夫人烏林荅氏。夫人謂世宗曰:「我不行,上必殺王。我當自勉,不以相累也。」夫人行至良鄉自殺,是以世宗在位二十九年,不復立后焉。



\end{pinyinscope}