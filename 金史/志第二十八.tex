\article{志第二十八}

\begin{pinyinscope}

 食貨二



 ○田制



 量田以營造尺,五尺為步,闊一步,長二百四十步為畝,百畝為頃。民田業各從其便,賣質於人無禁,但令隨地輸租而已。凡桑棗,民戶以多植為勤,少者必植其地十之三,猛安謀克戶少者必課種其地十之一,除枯補新,使之不闕。凡官地,猛安謀克及貧民請射者,寬鄉一丁百畝,狹鄉十畝,中男半之。請射荒地者,以最下第
 五等減半定租,八年始征之。作己業者以第七等減半為稅,七年始征之。自首冒佃比鄰地者,輸官租三分之二。佃黃河退灘者,次年納租。



 太宗天會九年五月,始分遣諸路勸農之使者。熙宗天眷十四年,罷來流、混同間護邏地,以予民耕牧,海陵正隆元年二月,遣刑部尚書紇石烈婁室等十一人,分行大興府、山東,真定府,拘括係官或荒閑牧地,及官民占射逃絕戶地,戍兵占佃宮籍監、外路官本業外增置土田,及大興府,平州路僧尼道士女冠等地,蓋以授所遷之猛安謀克戶,且令民請射,而官得其租也。



 世宗大定五年十二月,上以京畿兩猛
 安民戶不自耕墾,及伐桑棗為薪鬻之,命大興少尹完顏讓巡察。十年四月,禁侵耕圍場地。十一年,謂侍臣曰:「往歲,清暑山西,傍路皆禾稼,殆無牧地。嘗下令,使民五里外乃得耕墾。今聞其民以此去之他所,甚可矜憫。其令依舊耕種,毋致失業。凡害民之事患在不知,知之朕必不為。自今事有類此,卿等即告毋隱。」十三年,敕有司:「每歲遣官勸猛安謀克農事,恐有煩擾。自今止令各管職官勸督,弛慢者舉劾以聞。」十七年六月,刑州男子趙迪簡言:「隨路不附籍官田及河灘地,皆為豪強所占,而貧民土瘠稅重,乞遣官拘籍冒佃者,定立租課,復量減
 人戶稅數,庶得輕重均平。」詔付有司,將行而止。復以近都猛安謀克所給官地率皆薄瘠,豪民租佃官田歲久,往往冒為己業,令拘籍之。又謂省臣曰:「官地非民誰種,然女直人戶自鄉土三四千里移來,盡得薄地,若不拘刷良田給之,久必貪乏,其遣官察之。」又謂參知政事張汝弼曰:「先嘗遣問女直土地,皆云良田。及朕出獵,因問之,則謂自起移至此,不能種蒔,斫蘆為席,或斬芻以自給。卿等其議之。」省臣奏:「官地所以人多蔽匿盜耕者,由其罪輕故也。」乃更條約,立限令人自陳,過限則人能告者有賞。遣同知中都路轉運使張九思往拘籍之。十九
 年二月,上如春水,見民桑多為牧畜嚙毀,詔親王公主及勢要家,牧畜有犯民桑者,許所屬縣官立加懲斷。



 十二月謂宰臣曰:「亡遼時所撥地,與本朝元帥府,已曾拘籍矣。民或指射為無主地,租佃及新開荒為己業者可以拘括。其間播種歲久,若遽奪之,恐民失業。」因詔括地官張九思戒之。復謂宰臣曰:「朕聞括地事所行極不當,如皇后莊、太子務之類,止以名稱便為官地,百姓所執憑驗,一切不問,其相鄰冒占官地,復有幸免者。能使軍戶稍給,民不失業,乃朕之心也。」二十年四月,以行幸道隘,扈從人不便,詔戶部沿路頓捨側近官地,勿租與民
 耕種。又詔故太保阿里先於山東路撥地百四十頃,大定初又於中都路賜田百頃,命拘山東之地入官。五月,諭有司曰:「白石門至野狐嶺,其間澱濼多為民耕植者,而官民雜畜往來無牧放之所,可差官括元荒地及冒佃之數。」



 二十一年正月,上謂宰臣曰:「山東、大名等路猛安謀克戶之民,往往驕縱,不親稼穡,不令家人農作,盡令漢人佃蒔,取租而已。富家盡服紈綺,酒食遊宴,貧者爭慕效之,欲望家給人足,難矣!近已禁賣奴婢,約其兇吉之禮,更當委官閱實戶數,計口授地,必令自耕,力不贍者,方許佃於人。仍禁其農時飲酒。」又曰:「奚人六猛安,
 已徙居咸平、臨潢、泰州,其地肥沃,且精勤農務,各安其居。女直人徙居奚地者,菽粟得收獲否?」左丞守道對曰:「聞皆自耕,歲用亦足。」上曰:「彼地肥美,異於他處,惟附都民以水害稼者賑之。」三月,陳言者言,豪強之家多占奪田者。上曰:「前參政納合椿年占地八百頃,又聞山西田亦多為權要所占,有一家一口至三十頃者,以致小民無田可耕,徙居陰山之惡地,何以自存?其令占官地十頃以上者皆括籍入官,將均賜貧民。」省臣又奏:「椿年子猛安參謀合、故太師耨碗溫敦思忠孫長壽等,親屬計七十餘家,所占地三千餘頃。」上曰:「至秋,除牛頭地外,仍各給
 十頃,餘皆拘入官。山後招討司所括者,亦當同此也。」又謂宰臣曰:「山東路所括民田,已分給女直屯田人戶,復有籍官閑地,依元數還民,仍免租稅。」六月,上謂省臣曰:「近者大興府平、灤、薊、通、順等州,經水災之地,免今年稅租。不罹水災者姑停夏稅,俟稔歲征之。」時中都大水,而濱,棣等州及山後大熟,命修治懷來以南道路,以來糶者。又命都城減價以糶。又曰:「近遣使閱視秋稼,聞猛安謀克人惟酒是務,往往以田租人,而預借三二年租課者。或種而不耘,聽其荒蕪者。自今皆令閱實各戶人力,可耨幾頃畝,必使自耕耘之,其力果不及者方許租賃。
 如惰農飲酒,勸農謀克及本管猛安謀克并都管,各以等第科罪。收獲數多者,則亦以等第遷賞。」七月,上謂宰臣曰:「前徙宗室戶於河間,撥地處之,而不回納舊地,豈有兩地皆占之理?自今當以一處賜之。山東刷民田已分給女直屯田戶,復有餘地,當以還民而免是歲之租。」八月,尚書省奏山東所刷地數,上謂梁肅曰:「朕嘗以此問卿,卿不以言。此雖稱民地,然皆無明據,括為官地有何不可?」又曰:「黃河已移故道,梁山濼水退,地甚廣,已嘗遣使安置屯田。民昔嘗恣意種之,今官已籍其地,而民懼徵其租,逃者甚眾。若徵其租,而以冒佃不即出首罪
 論之,固宜。然若遽取之,恐致失所。可免其征,赦其罪,別以官地給之。」御史臺奏:「大名、濟州因刷梁山濼官地,或有以民地被刷者。」上復召宰臣曰:「雖曾經通檢納稅,而無明驗者,復當刷問。有公據者,雖付本人,仍須體問。」十月,復與張仲愈論冒占田事。



 二十二年,以附都猛安戶不自種,悉租與民,有一家百口壟無一苗者。上曰:「勸農官,何勸諭為也,其令治罪。」宰臣奏曰:「不自種而輒與人者,合科違例。」上曰:「太重,愚民安知。」遂從大興少尹王脩所奏,以不種者杖六十,謀克四十,受租百姓無罪。又命招復梁山濼流民,官給以田。時人戶有執契據指墳壟
 為驗者,亦拘在官,先委恩州刺史奚晦招之,復遣安肅州刺史張國基驗實給之,如已撥係猛安,則償以官田。上曰:「工部尚書張九思執強不通,向遣刷官田,凡犯秦、漢以來名稱,如長城、燕子城之類者,皆以為官田。此田百姓為己業不知幾百年矣,所見如此,何不通之甚也。」八月,以趙王永中等四王府冒占官田,罪其各府長史府掾,及安次,新城,宛平、昌平、永清、懷柔六縣官,皆罰贖有差。



 九月,遣刑部尚書移剌慥於山東路猛安內摘八謀克民,徙于河北東路酬斡、青狗兒兩猛安舊居之地,無牛者官給之。河間宗室未徙者令盡徙于平州,無力
 者官津發之,土薄者易以良田。先嘗令俟豐年則括籍官地,至是歲,省臣復以為奏,上曰:「本為新徙四猛安貧窮,須刷官田與之,若張仲愈等所擬條約太刻,但以民初無得地之由,自撫定後未嘗輸稅,妄通為己業者,刷之。如此,恐民苦之,可為酬直。且先令猛安謀克人戶,隨宜分處,計其丁壯牛具,合得土田實數,給之。不足,則以前所刷地二萬餘頃補之。復不足,則續當議。」時有落兀者與婆薩等爭懿州地六萬頃,以皆無據驗,遂沒入官。



 二十七年,隨處官豪之家多請占官地,轉與它人種佃,規取課利。命有司拘刷見數,以與貧難無地者,每丁授
 五十畝,庶不至失所,餘佃不盡者方許豪家驗丁租佃。章宗大定二十九年五月,擬再立限,令貧民請佃官地,緣今已過期,計已數足,其占而有餘者,若容告訐,恐滋姦弊。況續告漏通地,敕旨已革,今限外告者宜卻之,止付元佃。兼平陽一路地狹人稠,官地當盡數拘籍,驗丁以給貧民。上曰:「限外指告多佃官地者,卻之,當矣。如無主不顧承佃,方許諸人告請。其平陽路宜計丁限田,如一家三丁己業止三十畝,則更許存所佃官地一頃二十畝,餘者拘籍給付貧民可也。」七月,論旨尚書省曰:「唐、鄧、潁、蔡、宿、泗等處,水陸膏腴之地,若驗等級,量立歲租,
 寬其徵納之限,募民佃之,公私有益。今河南沿邊地多為豪民冒占,若民或流移至彼,就募令耕,不惟貧民有贍,亦增羨官租。其給丁壯者田及耕具,而免其租稅。」八月,尚書省奏:「河東地狹,稍凶荒則流亡相繼。竊謂河南地廣人稀,若令招集他路流民,量給閑田,則河東飢民減少,河南且無曠地矣。」上從所請。九月戊寅,又奏:「在制,諸人請佃官閑地者免五年租課,今乞免八年,則或多墾。」並從之。十一月,尚書省奏:「民驗丁佃河南荒閑官地者,如願作官地則免租八年,願為己業則免稅三年,並不許貿易典賣。若豪強及公吏輩有冒佃者,限兩月陳
 首,免罪而全給之,其稅則視其鄰地定之,以三分為率減一分,限外許諸人告詣給之。」制可。



 明昌元年二月,諭旨有司曰:「瀕水民地,已種蒔而為水浸者,可令以所近官田對給。」三月,敕:「當軍人所受田,止令自種,力不足者方許人承佃,亦止隨地所產納租,其自欲折錢輸納者從民所欲,不願承佃者毋強。」六月,尚書省奏:「近制以猛安謀克戶不務栽植桑果,已令每十畝須栽一畝,今乞再下各路提刑及所屬州縣,勸諭民戶,如有不栽及栽之不及十之三者,並以事怠慢輕重罪科之。」詔可。八月,敕:「隨處係官閑地,百姓已請佃者仍舊,未佃者以付屯
 田猛安謀克。」三年六月,尚書省奏:「南京、陜西路提刑司言,舊牧馬地久不分撥,以致軍民起訟,比差官往各路定之。凡民戶有憑驗己業,及宅井墳園,已改正給付,而其中復有官地者,亦驗數對易之矣。兩路牧地,南京路六萬三千五百二十餘頃,陜西路三萬五千六百八十餘頃。」五年,諭旨尚書省:「遼東等路女直、漢兒百姓,可並令量力為蠶桑。」二月,陳言人乞以長吏勸農立殿最,遂定制:「能勸農田者,每年謀克賞銀絹十兩匹,猛安倍之,縣官於本等升五人。三年不怠者猛安謀克遷一官,縣官陞一等。田荒及十之一者笞三十,分數加至徒一年。
 三年皆荒者,猛安謀克追一官,縣官以陞等法降之。」為永格。六年二月,詔罷括陜西之地。又陜西提刑司言:「本路戶民安水磨、油栿,所占步數在私地有稅,官田則有租,若更輸水利錢銀,是重併也,乞除之。」省臣奏:「水利錢銀以輔本路之用,未可除也,宜視實占地數,除稅租。」命他路視此為法。



 承安二年,遣戶部郎中上官瑜往西京并沿邊,勸舉軍民耕種。又差戶部郎中李敬義往臨潢等路規畫農事。舊令,軍人所授之地不得租賃與人,違者苗付地主。泰和四年九月定制,所撥地土十里內自種之數,每丁四十畝,續進丁同此,餘者許令便宜租賃
 及兩和分種,違者錢業還主。上聞六路括地時,其間屯田軍戶多冒名增口,以請官地,及包取民田,而民有空輸稅賦、虛抱物力者,應詔陳言人多論之。五年二月,尚書省奏:「若復遣官分往,追照案憑,訟言紛紛,何時已乎?」遂令虛抱稅石已輸送入官者,命於稅內每歲續剋之。泰和七年,募民種佃清河等處地,以其租分為諸春水處餌鵝鴨之食。八年八月,戶部尚書高汝礪言:「舊制,人戶請佃荒地者,以各路最下第五等減半定租,仍免八年輸納。若作己業,並依第七等稅錢減半,亦免三年輸納。自首冒佃比鄰田,定租三分納二。其請佃黃河退灘
 地者,次年納租。向者小民不為久計,比至納租之時多巧避匿,或復告退,蓋由元限太遠,請佃之初無人保識故爾。今請佃者可免三年,作己業者免一年,自首冒佃并請退灘地,並令當年輸租,以鄰首保識,為長制。」



 宣宗貞祐三年七月,以既徙河北軍戶於河南,議所以處之者。宰臣曰:「當指官田及牧地分畀之,已為民佃者則俟秋獲後,仍日給米一升,折以分鈔。」太常丞石抹世績曰:「荒田牧地耕闢費力。奪民素墾則民失所。況軍戶率無牛,宜令軍戶分人歸守本業,至春復還,為固守計。」上卒從宰臣議,將括之,侍御史劉元規上書曰:「伏見朝廷有
 括地之議,聞者無不駭愕。向者河北、山東已為此舉,民之塋墓井灶悉為有軍有,怨嗟爭訟至今未絕,若復行之,則將大失眾心,荒田不可耕,徒有得地之名,而無享利之實。縱得熟土,不能親耕,而復令民佃之,所得無幾,而使紛紛交病哉!」上大悟,罷之。



 八月,先以括地事未有定論,北方侵及河南,由是盡起諸路軍戶南來。共圖保守,而不能知所以得軍糧之術。眾議謂可分遣官聚耆老問之,其將益賦,或與軍田,二者孰便。參政汝礪言:「河南官民地相半,又多全佃官地之家,一旦奪之,何以自活?小民易動難安,一時避賦遂有舍田之言,及與人能勿
 悔乎,悔則忿心生矣!如山東撥地時,腴地盡入富家,瘠者乃付貧戶,無益於軍,而民有損。惟當倍益官租,以給軍食。復以係官荒田牧地量數與之,令其自耕,則民不失業,官不厲民矣!」從之。三年十月,高汝礪言:「河北軍戶徙居河南者幾百萬口,人日給米一升,歲費三百六十萬石,半以給直,猶支粟三百萬石。河南租地計二十四萬頃,歲租纔一百五十六萬,乞於經費之外倍徵以給之。」遂命右司諫馮開等五人分詣諸郡,就授以荒官田及牧地可耕者,人三十畝。



 十一月,又議以括荒田及牧馬地給軍。命尚書右丞高汝礪總之。汝礪還奏:「今頃畝之數較之
 舊籍甚少,復瘠惡不可耕,均以可耕者與之,人得無幾。又僻遠之處必徙居以就之,彼皆不能自耕,必以與人,又當取租於數百里之外。況今農田且不能盡闢,豈有餘力以耕叢薄交固、草根糾結之荒地哉!軍不可仰此得食也,審矣。今詢諸軍戶,皆曰:『得半糧猶足自養,得田不能耕,復罷其廩,將何所賴?』臣知初籍地之時,未嘗按閱其實,所以不如其數,不得其處也。若復考計州縣,必各妄承風旨,追呼究結以應命。不足其數,則妄指民田以充之,則所在騷然矣!今民之賦役三倍平時,飛挽轉輸,日不暇給,而復為此舉,何以堪之。且軍戶暫遷,行有還期,
 何為以此病民哉!病民而軍獲利,猶不可為,況無所利乎!惟陛下加察。」遂詔罷給田,但半給糧、半給實直焉。四年,復遣官括河南牧馬地,既籍其數,上命省院議所以給軍者,宰臣曰:「今軍戶當給糧者四十四萬八千餘口,計當口占六畝有奇,繼來者不與焉。但相去數百里者,豈能以六畝之故而遠來哉!兼月支口糧不可遽罷,臣等竊謂軍戶願佃者即當計口給之。自餘僻遠不願者,宜准近制,係官荒地許軍民耕闢例,令軍民得占蒔之。」院官曰:「牧馬地少,且久荒難耕,軍戶復乏農器,然不給之,則彼自支糧外,更無從得食,非蓄銳待敵之計。給之
 則亦未能遽減其糧,若得遲以歲月,俟頗成倫次,漸可以省官廩耳。今奪於有力者,即以授其無力者,恐無以耕。乞令司縣官勸率民戶,借牛破荒,至來春然後給之。司縣官能率民戶以助耕而無騷動者,量加官賞,庶幾有所激勸。」宰臣復曰:「若如所言,則司縣官貪慕官賞,必將抑配,以至擾民。今民家之牛,量地而畜之。況比年以來,農功甫畢則併力轉輸猶恐不及,豈有暇耕它人之田也。惟如臣等前奏為便。」詔再議之。乃擬民有能開牧馬地及官荒地作熟田者,以半給之為永業,半給軍戶。奏可。四年,省奏:「自古用兵,且耕且戰,是以兵食交足。今
 諸帥分兵不啻百萬,一充軍伍咸仰於官,至於婦子居家安坐待哺,蓋不知屯田為經久之計也。願下明詔,令諸帥府各以其軍耕耨,亦以逸待勞之策也。」詔從之。



 興定三年正月,尚書右丞領三司事侯摯言:「按河南軍民田總一百九十七萬頃有奇,見耕種者九十六萬餘頃,上田可收一石二斗,中田一石,下田八斗,十一取之,歲得九百六十萬石,自可優給歲支,且使貧富均,大小各得其所。臣在東平嘗試行二三年,民不疲而軍用足。」詔有司議行之。四年十月,移剌不言:「軍戶自徙於河南,數歲尚未給田,兼以移徙不常,莫得安居,故貧者甚眾。請
 括諸屯處官田,人給三十畝,仍不移屯它所,如此則軍戶可以得所,官糧可以漸省。」宰臣奏:「前此亦有言授地者,樞密院以謂俟事緩而行之。今河南罹水災,流亡者眾,所種麥不及五萬頃,殆減往年太半,歲所入殆不能足。若撥授之為永業,俟有獲即罷其家糧,亦省費之一端也。」上從之。又河南水災,逋戶太半,田野荒蕪,恐賦入少而國用乏,遂命唐、鄧、裕、察、息、壽、潁、亳及歸德府被水田,已燥者布種,未滲者種稻,復業之戶免本租及一切差發,能代耕者如之,有司擅科者以違制論,闕牛及食者率富者就貸。五年正月,京南行三司石抹斡魯言:「京
 南、東、西三路,屯軍老幼四十萬口,歲費糧百四十餘萬石,皆坐食民租,甚非善計。宜括逋戶舊耕田,南京一路舊墾田三十九萬八千五百餘頃,內官田民耕者九萬九千頃有奇。今饑民流離者太半,東、西、南路計亦如之,朝廷雖招使復業,民恐既復之後生計未定而賦斂隨之,往往匿而不出。若分給軍戶人三十畝,使之自耕,或召人佃種,可數歲之後畜積漸饒,官糧可罷。」令省臣議之,更不能行。



 ○租賦



 金制,官地輸租,私田輸稅。租之制不傳,大率分田之等為九而差次之。夏稅畝取三合,秋稅畝取五升,又
 納秸一束,束十有五斤。夏稅六月止八月,秋稅十月止十二月,為初、中、末三限,州三百里外,紓其期一月。屯田戶佃官地者,有司移猛安謀克督之。泰和五年,章宗諭宰臣曰:「十月民獲未畢,遽令納稅可乎?」改秋稅限十一月為初。中都、西京、北京、上京、遼東、臨潢、陜西地寒,稼穡遲熟,夏稅限以七月為初。凡輸送粟麥,三百里外石減五升,以上每三百里遞減五升。粟折秸百稱者,百里內減三稱,二百里減五稱,不及三百里減八稱,三百里及輸本色槁草,各減十稱。計民田園、邸舍、車乘、牧畜、種植之資,藏鏹之數,徵錢有差,謂之物力錢。遇差科,必按版
 籍,先及富者,勢均則以丁多寡定甲乙。有橫科,則視物力,循大至小均科。其或不可分摘者,率以次戶濟之。凡民之物力,所居之宅不預。猛安謀克戶、監戶、官戶所居外,自置民田宅,則預其數。墓田、學田,租稅、物力皆免。



 民愬水旱應免者,河南、山東、河東、大名、京兆、鳳翔、彰德部內支郡,夏田四月,秋田七月,餘路夏以五月,秋以八月,水田則通以八月為限,遇閏月則展期半月,限外愬者不理。非時之災則無限。損十之八者全免,七分免所損之數,六分則全徵。桑被災不能蠶,則免絲綿絹稅。諸路雨雪及禾稼收獲之數,月以捷步申戶部。凡敘使品官
 之家,並免雜役,驗物力所當輸者、止出雇錢。進納補官未至廕子孫、及凡有出身者、謂司吏、譯人等。出職帶官敘當身者、雜班敘使五品以下、及正品承應已帶散官未出職者,子孫與其同居兄弟,下逮終場舉人,係籍學生、醫學生,皆免一身之役。三代同居,已旌門則免差發,三年後免雜役。



 太宗天會元年,敕有司輕徭賦,勸稼穡。十年,以遼人士庶之族賦役等差不一,詔有司命悉均之。熙宗天眷五年十二月,詔免民戶殘欠租稅。皇統三年,蠲民稅之未足者。世宗大定二年五月,謂宰臣曰:「凡有徭役,均科強戶,不得抑配貧民。」有言以用度不足,奏預借河
 北東西路、中都租稅,上以國用雖乏,民力尤艱,遂不允。三年,以歲歉,詔免二年租稅。又詔曰:「朕比以元帥府從宜行事,今聞河南、陜西、山東、北京以東、及北邊州郡,調發甚多,而省部又與他州一例征取賦役,是重擾也。可憑元帥府已取者例,蠲除之。」五年,命有司:「凡罹蝗旱水溢之地,蠲其賦稅。」六年,以河北、山東水,免其租。八年十月,彰德軍節度使高昌福上書言稅租甚重,上諭翰林學士張景仁曰:「今租稅法比近代甚輕,而以為重,何也?」景仁曰:「今之稅斂殊輕,非稅斂則國用何從而出?」二年二月,尚書省奏:「天下倉廩貯粟二千七十九萬餘石。」上
 曰:「朕聞國無九年之蓄則國非其國,朕是以括天下之田以均其賦,歲取九百萬石,自經費七百萬石外,二百萬石又為水旱之所蠲免及賑貸之用,餘纔百萬石而已。朕廣蓄積,備饑謹也。小民以為稅重,小臣沽民譽,亦多議之。蓋不慮國家緩急之備也。」



 十二年正月,以水旱免中都、西京、南京、河北、河東、山東、陜西去年租稅。十三年,謂宰臣曰:「民間科差,計所免已過半矣。慮小民不能詳知,吏緣為姦,仍舊征取,其令所在揭榜諭之。」十月,敕州縣官不盡力催督稅租,以致逋懸者,可止其俸,使之徵足,然後給之。十六年正月,詔免去年被水旱路分租
 稅。十七年,上問宰臣曰:「遼東賦稅舊六萬餘石,通檢後幾二十萬。六萬時何以仰給,二十萬後所積幾何?」戶部契勘,謂:「先以官吏數少故能給,今官吏兵卒及孤老數多,以此費大。」上曰:「當察其實,毋令妄費也。」十七年三月,詔免河北、山東、陜西、河東、西京、遼東等十路去年被旱蝗租稅。十八年正月。免中都、河北、河東、山東、河南、陜西等路前年被災租稅。十九年秋,中都、西京、河北、山東、河東、陜西以水旱傷民田十三萬七千七百餘頃,詔蠲其租。二十年三月,以中都、西京、河北、山東、河東、陜西路前歲被災,詔免其租稅。以戶部尚書曹望之之言,詔減鄜
 延及河東南路稅五十二萬餘石,增河北西路稅八萬八千石。又詔諸稅粟非關邊要之地者,除當儲數外,聽民從便折納。二十一年九月,以中都水災,免租。前時近官路百姓以牛夫充遞運者,復於它處未嘗就役之家徵錢償之。二十三年,宗州民王仲規告乞征還所役牛夫錢,省臣以奏,上曰:「此既就役,復徵錢於彼,前雖如此行之,復恐所給錢未必能到本戶,是兩不便也。不若止計所役,免租稅及鋪馬錢為便。其預計實數以聞。若和雇價直亦須裁定也。」有司上其數,歲約給六萬四千餘貫,計折粟八萬六千餘石。上復命,自今役牛夫之家,以
 去道三十里內居者充役。二十六年,軍民地罹水旱之災者,二十一萬頃免稅凡四十九萬餘石。二十七年六月,免中都、河北等路嘗被河決水災軍民租稅。十一月,詔河水泛溢,農田被災者,與免差稅一年。懷、衛、孟、鄭四州塞河勞役,并免今年差稅。章宗大定二十九年,赦民租十之一。河東南北路則量減之。尚書省奏,兩路田多峻阪,磽瘠者往往再歲一易,若不以地等級蠲除,則有不均。遂敕以赦書特免一分外,中田復減一分,下田減二分。舊制,夏、秋稅納麥、粟、草三色,以各處所須之物不一,戶部復令以諸所用物折納。上封事者言其不可,戶
 部謂如此則諸路所須之物要當和市,轉擾民矣。遂命太府監,應折納之物為祗承宮禁者,治黃河薪芻增直二錢折納,如黃河岸所用木石固非土產,乃令所屬計置,而罷它應折納者。



 明昌元年四月,上封事者乞薄民之租稅,恐廩粟積久腐敗。省臣奏曰:「臣等議,大定十八年戶部尚書曹望之奏,河東及鄜延兩路稅頗重,遂減五十二萬餘石。去年赦十之一,而河東瘠地又減之。今以歲入度支所餘無幾,萬一有水旱之災,既蠲免其所入,復出粟以賑之,非有備不可。若復欲減,將何以待之。如慮腐敗,令諸路以時曝晾,毋令致壞,違者論如律。」制可。



 十一月,
 尚書省奏:「河南荒閑官地,許人計丁請佃,願仍為官者免租八年,願為己業者免稅三年。」詔從之。明昌二年二月,敕自今民有訴水旱災傷者,即委官按視其實,申所屬州府,移報提刑司,同所屬檢畢,始令翻耕。三年六月,有司言河州災傷,闕食之民猶有未輸租者,詔蠲之。九月,以山東、河北三路被災,其權閣之租及借貸之粟,令俟歲豐日續征。上如秋山,免圍場經過人戶今歲夏秋租稅之半。四年冬十月,上行幸,諭旨尚書省曰:「海壖石城等縣,地瘠民困,所種惟黍稗而已。及賦於官,必以易粟輸之。或令止課所產,或依河東路減稅,至還京當定
 議以聞。」五年,敕免河決被菑之民秋租。泰和四年四月,以久旱下詔責躬,免所旱州縣今年夏稅。九月,陳言者謂:「河間、滄州逃戶,物力錢至數千貫,而其差發,有司止取辦於見戶,民不能堪矣!」詔令按察司,除地土物力命隨其業,而權止其浮財物力。五年正月,詔有司:「自泰和三年嘗所行幸至三次者,被科之民特免半年租稅。」八年五月,以宋謀和,詔天下,免河南、山東、陜西六路今年夏稅,河東、河北、大名等五路半之。八月,詔諸路農民請佃荒田者,與免租賦三年,作己業者一年,自首冒佃、及請佃黃河退灘地者,不在免例。



 宣宗貞祐三年十月,御
 史田迥秀言:「方今軍國所需,一切責之河南。有司不惜民力,徵調太急,促其期限,痛其棰楚。民既罄其所有而不足,遂使奔走傍求於它境,力竭財殫,相踵散亡,禁之不能止也,乞自今凡科征必先期告之,不急者皆罷,庶民力寬而逋者可復。」詔行之。十二月,詔免逃戶租稅。四年三月,免陜西逃戶租。五月,山東行省僕散安貞言:「泗州被災,道殣相望,所食者草根木皮而已。而邳州戍兵數萬,急征重役,悉出三縣,官吏酷暴,擅括宿藏,以應一切之命。民皆逋竄,又別遣進納閑官以相迫督。皆怙勢營私,實到官者纔十之一,而徒使國家有厚斂之名。乞
 命信臣革此弊以安百姓。」詔從之。興定元年二月,免中京、嵩、汝等逋租十六萬石。四年,御史中丞完顏伯嘉奏:「亳州大水,計當免租三十萬石,而三司官不以實報,止免十萬而已。」詔命治三司官虛妄之罪。七月,以河南大水,下詔免租勸種,且命參知政事李復亨為宣慰使,中丞完顏伯嘉副之。十月,以久雨,令寬民輸稅之限。十一月,上曰:「聞百姓多逃,而逋賦皆抑配見戶,人何以堪?軍儲既足,宜悉除免。今又添軍須錢太多,亡者詎肯復業乎?」遂命行部官閱實免之,已代納者給以恩例,或除它役,仍減桑皮故紙錢四之一。三年,令逃戶復業者但輸
 本租,餘差役一切皆免。能代耕者,免如復戶。有司失信擅科者,以違制論。



 四年十二月,鎮南軍節度使溫迪罕思敬上書言:「今民輸稅,其法大抵有三,上戶輸遠倉,中戶次之,下戶最近。然近者不下百里,遠者數百里,道路之費倍於所輸,而雨雪有稽違之責。遇賊有死傷之患。不若止輸本郡,令有司檢算倉之所積,稱屯兵之數,使就食之。若有不足,則增斂于民,民計所斂不及道里之費,將忻然從之矣!」五年十月,上諭宰臣曰:「比欲民多種麥,故令所在官貸易麥種。今聞實不貸與,而虛立案簿,反收其數以補不足之租。其遣使究治。」



 元光元年,上聞
 向者有司以征稅租之急,民不待熟而刈之,以應限。今麥將熟矣,其諭州縣,有犯者以慢軍儲治罪。九月,權立職官有田不納租罪。京南司農卿李蹊言:「按《齊民要術》,麥晚種則粒小而不實,故必八月種之。今南路當輸秋稅百四十餘萬石,草四百五十餘萬束,皆以八月為終限。若輸遠倉及泥淖,往返不下二十日,使民不暇趨時,是妨來歲之食也。乞寬征斂之限。使先盡力於二麥。」朝廷不從。元光二年,宰臣奏:「去歲正月京師見糧纔六十餘萬石,今三倍矣,計國用頗足,而民間租稅征之不絕,恐貧民無所輸而逋亡也。」遂以中旨遍諭止之。



 ○
 牛頭稅



 即牛具稅,猛安謀克部女直戶所輸之稅也。其制每耒牛三頭為一具,限民口二十五受田四頃四畝有奇,歲輸粟大約不過一石,官民占田無過四十具。天會三年,太宗以歲稔,官無儲積無以備饑謹,詔令一耒賦粟一石,每謀克別為一廩貯之。四年,詔內地諸路,每牛一具賦粟五斗,為定制。



 世宗大定元年,詔諸猛安不經遷移者,徵牛具稅粟,就命謀克監其倉,虧損則坐之。十二年,尚書省奏:「唐古部民舊同猛安謀克定稅,其後改同州縣,履畝立稅,頗以為重。」遂命從舊制。二十年,定功授世襲謀克,許以親族從行,當給以地者,除牛九具
 以下全給,十具以上四十具以下者,則於官豪之家量撥地六具與之。二十一年,世宗謂宰臣曰:「前時一歲所收可支三年,比聞今歲山西豐稔,所獲可支三年。此間地一歲所獲不能支半歲,而又牛頭稅粟,每牛一頭止令各輸三斗,又多逋懸,此皆遞互隱匿所致,當令盡實輸之。」二十三年,有司奏其事,世宗謂左丞完顏襄曰:「卿家舊止七具,今定為四十具,朕始令卿等議此,而卿皆不欲,蓋各顧其私爾。是後限民口二十五,算牛一具。」七月,尚書省復奏其事,上慮版籍歲久貧富不同,猛安謀克又皆年少,不練時事,一旦軍興,按籍征之必有不均
 之患。乃令驗實推排。閱其戶口、畜產之數,其以上京二十二路來上。八月,尚書省奏,推排定猛安謀克戶口、田畝、牛具之數。猛安二百二,謀克千八百七十八,戶六十一萬五千六百二十四,口六百一十五萬八千六百三十六,內正口四百八十一萬二千六百六十九,奴婢口一百三十四萬五千九百六十七,田一百六十九萬三百八十頃有奇,牛具三十八萬四千七百七十一。在都宗室將軍司,戶一百七十,口二萬八千七百九十,內正口九百八十二,奴婢口二萬七千八百八,田三千六百八十三頃七十五畝有奇,牛具三百四。迭剌、唐古二部
 五颭,戶五千五百八十五,口一十三萬七千五百四十四,內正口十一萬九千四百六十三,奴婢口一萬八千八十一,田四萬六千二十四頃一十七畝,牛具五千六十六。後二十六年,尚書省奏並徵牛頭稅粟,上曰:「積壓五年,一見並徵,民何以堪?其令民隨年輸納。被災者蠲之,貸者俟豐年征還。」



\end{pinyinscope}