\article{列傳第六十五}

\begin{pinyinscope}

 孝友



 ○溫迪罕斡魯補陳顏劉瑜孟興王震劉政



 孝友者,人之至行也,而恒性存焉。有子者欲其孝,有弟者欲其友,豈非人之恒情乎?為子而孝,為弟而友,又豈非人之恒性乎?以人之恒情責人之恒性,而不副所欲者恒有焉。有竭力於是,豈非難乎。天生五穀以養人,五
 穀之有恒性也。服田力穡以望有秋,農夫之有恒情也。五穀熟,人民育,豈異事乎。然以唐、虞之世,「黎民阻飢」不免以命稷,「百姓不親、五品不遜」不免以命契,以是知順成之不可必,猶孝友之不易得也。是故「有年」、「大有年」以異書於聖人之經,孝友以至行傳於歷代之史,勸農興孝之教不廢於歷代之政,孝弟力田自漢以來有其科。章宗嘗曰:「孝義之人,素行已備,雖有希覬,猶不失為行善。」庶幾帝王之善訓矣。夫金世孝友見於旌表、載於史冊者僅六人焉。作《孝友傳》。



 溫迪罕斡魯補,西北路宋葛斜斯渾猛安人。年十五,居
 父喪,不飲酒食肉,廬于墓側。母疾,刲股肉療之,疾愈。詔以為護衛。



 陳顏,衛州汲縣人。世業農。父光,宋季擢武舉第,調壽陽尉,未赴。值金兵取汴,光病,圍城中。顏間關渡河,往省其父,因扶疾北歸。光家奴謀良不可,誣告光與賊殺人。光繫獄,榜掠不勝,因自誣服。顏詣郡請代父死,太守徐某哀之,不敢決,適帥臣至郡,以其狀白,帥曰:「此真孝子也。」遂併釋之。天會七年,詔旌表其門閭。



 劉瑜,棣州人。家貧甚,母喪不能具葬,乃質其子以給喪事。明昌三年,詔賜粟帛,復其終身。



 孟興,蚤喪父,事母孝謹,母沒,喪葬盡禮。事兄如事其父。明昌三年,詔賜帛十匹、粟二十石。



 王震,寧海州文登縣人。為進士學。母患風疾,刲股肉雜飲食中,疾遂愈。母沒,哀泣過禮,目生翳。服除,目不療而愈,皆以為孝感所致。特賜同進士出身,詔尚書省擬注職任。



 劉政,洺州人。性篤孝,母老喪明,政每以舌舐母目,逾旬母能視物。母疾,晝夜侍側,衣不解帶,刲股肉啖之者再三。母死,負土起墳,鄉鄰欲佐其勞,政謝之。葬之日,飛鳥哀鳴,翔集丘木間。廬於墓側者三年。防禦使以聞,除太
 子掌飲丞。



 ◎隱逸



 褚承亮王去非趙質杜時昇郝天挺薛繼元高仲振



 張潛王汝梅宋可辛願王予可1111



 孔子稱逸民伯夷、叔齊、夷逸、朱張、柳下惠、少連,其立心造行之異同,各有所稱謂,而柳下惠則又嘗仕於當世者也。長沮、桀溺之徒,則無所取焉。後世,凡隱遁之士,其名皆列於史傳,何歟?蓋古之仕者,其志將以行道,其為
 貧而仕下列者,猶必先事而後食焉。後世干祿者多,其先人尚人之志與歎老嗟卑之心,能去是者鮮矣。故君子于士之遠引高蹈者特稱述之,庶聞其風猶足以立懦廉頑也。作《隱逸傳》。



 褚承亮,字茂先,真定人。宋蘇軾自定武謫官過真定,承亮以文謁之,大為稱賞。宣和五年秋,應鄉試,同試者八百人,承亮為第一。明年,登第。調易州戶曹,未赴,會金兵南下。天會六年,斡離不既破真定,拘籍境內進士試安國寺,承亮名亦在籍中,匿而不出。軍中知其才,嚴令押赴,與諸生對策。策問「上皇無道、少帝失信。」舉人承風旨,
 極口詆毀。承亮詣主文劉侍中曰:「君父之罪,豈臣子所得言耶?」長揖而出。劉為之動容。餘悉放第,凡七十二人,遂號七十二賢榜。狀元許必仕為郎官,一日出左掖門,墮馬,首中閫石死,餘皆無顯者。劉多承亮之誼,薦知槁城縣。漫應之,即棄去。年七十終,門人私謚曰「玄貞先生。」



 子席珍,正隆二年進士,官州縣有聲。



 王去非,字廣道,平陰人。嘗就舉,不得意即屏去,督妻孥耕織以給伏臘。家居教授,束脩有餘輒分惠人。弟子班冘貧不能朝夕,一女及笄,去非為辦資裝嫁之。北鄰有喪忌東出,西與北皆人居,南則去非家,去非壞蠶室使
 喪南出,遂得葬焉。大定二十四年卒,年八十四。



 趙質,字景道,遼相思溫之裔。大定末,舉進士不第,隱居燕城南,教授為業。明昌間,章宗遊春水過焉,聞絃誦聲,幸其齋舍,見壁間所題詩,諷詠久之,賞其志趣不凡。召至行殿,命之官。因辭曰:「臣僻性野逸,志在長林豐草,金鑣玉絡非所願也。況聖明在上,可不容巢、由為外臣乎。」上益奇之,賜田畝千,復之終身。泰和二年卒,年八十五。



 杜時昇,字進之,霸州信安人。博學知天文,不肯仕進。承安、泰和間,宰相數薦時昇可大用。時昇謂所親曰:「吾觀正北赤氣如血,東西亙天,天下當大亂,亂而南北當合
 為一。消息盈虛,循環無端,察往考來,孰能違之。」是時,風俗侈靡,紀綱大壞,世宗之業遂衰。時昇乃南渡河,隱居嵩、洛山中,從學者甚眾。大抵以「伊洛之學」教人自時昇始。正大間,大元兵攻潼關,拒守甚堅,眾皆相賀,時昇曰:「大兵皆在秦、鞏間,若假道於宋,出襄、漢入宛、葉,鐵騎長驅勢如風雨,無高山大川為之阻,土崩之勢也。」頃之,大元兵果自饒峰關涉襄陽出南陽,金人敗績于三峰山,汴京不守,皆如時昇所料云。正大末,卒。



 郝天挺,字晉卿,澤州陵川人。早衰多疾,厭於科舉,遂不復充賦。太原元好問嘗從學進士業,天挺曰:「今人賦學
 以速售為功,六經百家分磔緝綴,或篇章句讀不之知,幸而得之,不免為庸人。」又曰:「讀書不為藝文,選官不為利養,唯通人能之。」又曰:「今之仕多以貪敗,皆苦饑寒不能自持耳。丈夫不耐饑寒,一事不可為。子以吾言求之,科舉在其中矣。」或曰:「以此學進士無乃戾乎?」天挺曰:「正欲渠不為舉子爾。」貞祐中,居河南,往來淇衛間。為人有崖岸,耿耿自信,寧落魄困窮,終不一至豪富之門。年五十,終于舞陽。



 薛繼先,字曼卿。南渡後,隱居洛西山中,課童子讀書。事母孝,與人交謙遜和雅,所居化之。子純孝,字方叔,有父
 風。有詐為曼卿書就方叔取物者,曼卿年已老狀貌如少者,客不知其為曼卿而以為方叔也,而與之書,曼卿如所取付之。監察御史石玠行部過曼卿,曼卿不之見。或言:「君何無鄉曲情。」曼卿曰:「君未之思耳。凡今時政未必皆善,御史一有所劾,將謂自我發之。同惡相庇,他日並鄉里必有受禍者。」其畏慎皆此類。壬辰之亂,病沒宜陽。



 高仲振,字正之,遼東人。其兄領開封鎮兵,仲振依之以居。既而以家業付其兄,挈妻子入嵩山。博極群書,尤深《易》《皇極經世》學。安貧自樂,不入城市,山野小人亦知敬
 之。嘗與其弟子張潛、王汝梅行山谷間,人望之翩然如仙。或曰仲振嘗遇異人教以養生術,嘗終日燕坐,骨節戛戛有聲,所談皆世外事,有扣之者輒不復語云。



 張潛,字仲升,武清人。幼有志節,慕荊軻、聶政為人,年三十始折節讀書。時人高其行誼,目曰「張古人。」後客崧山,從仲振受《易》。年五十,始娶魯山孫氏,亦有賢行,夫婦相敬如賓,負薪拾穗,行歌自得,不知其貧也。鄰里有為潛種瓜者,及熟讓潛,潛弗許,竟分而食之。嘗行道中拾一斧,夫婦計度移時,乃持歸訪其主還之。里有兄弟分財者,其弟曰:「我家如此,獨不畏張先生知耶?」遂如初。天興
 間,潛挈家避兵少室,乃不食七日死,孫氏亦投絕澗死焉。



 王汝梅,字大用,大名人。始由律學為伊陽簿,秩滿,遂隱居不仕。性嗜書,動有禮法。生徒以法經就學者,兼授以經學。諸生服其教,無敢為非義者。同業嘗憫其貧,時周之,皆謝不受。後不知所終。



 宋可,字予之,武陟人。其姑適大族槁氏,貞祐之兵,夫及子皆死於難。姑以白金五十笏遺可,可受不辭。其後姑得槁氏疏族立為後,挈之省外家。可乃置酒會鄉鄰,謂姑曰:「姑往時遺可以金,可以槁氏無子故受之。今有子
 矣,此金槁氏物,非姑物也,可何名取之。」因呼妻子舁金歸之,鄉里用是重之。未幾,北兵駐山陽,軍中有聞可名者,訪知所在,質其子,使人招之曰:「從我者禍福共之,不然,汝子死矣。」親舊競勸之往,可皆謝不從,曰:「吾有子無子,與吾兒死生,皆有命焉。豈以一子故,併平生所守者亡之。」後竟以無子。



 辛愿,字敬之,福昌人。年二十五始知讀書,取《白氏諷諫集》自試,一日便能背誦。乃聚書環堵中讀之,至《書伊訓》、《詩河廣》頗若有所省,欲罷不能,因更致力焉。由是博極書史,作文有繩尺,詩律精嚴有自得之趣。性野逸,不修
 威儀,貴人延客,麻衣草屨、足脛赤露坦然於其間,劇談豪飲,傍若無人。嘗謂王鬱曰:「王侯將相,世所共嗜者,聖人有以得之亦不避。得之不以道,與夫居之不能行己之志,是欲澡其身而伏於廁也。是難與他人道,子宜保之。」其志趣如此。



 後為河南府治中高廷玉客。廷玉為府尹溫迪罕福興所誣,愿亦被訊掠,幾不得免,自是生事益狼狽。愿雅負高氣,不以從俗俯仰,迫以饑凍流離,往往見之於詩。有詩數千首,常貯竹橐中。正大末,歿洛下。其詩有云:「黃綺暫來為漢友,巢由終不是唐臣。」真處士語也。



 王予可,字南雲,河東吉州人。父本軍校,予可亦嘗隸籍。年三十許,大病後忽發狂,久之能把筆作詩文,及說世外恍惚事。南渡後,居上蔡、遂平、郾城之間,遇文士則稱「大成將軍」,於佛前則稱「諦摩龍什」,於道則稱「騶天玄俊」,於貴游則稱「威錦堂主人」。



 為人軀幹雄偉,貌奇古,戴青葛巾,項後垂雙帶若牛耳,一金鏤環在頂額之間,兩頰以青涅之為翠靨。衣長不能掩脛。落魄嗜酒,每入城,市人爭以酒食遺之。夜宿土室中,夏月或尸穢在傍、蛆蟲狼籍不恤也。人與之紙,落筆數百言,或詩或文,散漫碎雜,無句讀、無首尾,多六經中語及韻學家古文奇字,字
 畫峭勁,遇宋諱亦時避之。或問以故事,其應如響,諸所引書,皆世所未見。談說之際稍若有條貫,則又以誕幻語亂之。麻九疇、張玨與之游最狎,言其詩以百分為率,可曉者才二三耳。



 壬辰兵亂,為順天將領所得,知其名,竊議欲挈之北歸,館於州之瑞雲觀。予可明日見將領自言曰:「我不能住君家瑞雲觀也。」不數日卒。後復有見於淮上者。



 贊曰:金世隱逸不多見,今於簡冊所有,得十有二人焉。其卓爾不群者三人。褚承亮宋人,勒試進士,主司發策問宋徽、欽之罪,承亮長揖而去之。方金人重舉業,杜時
 升居山中,首以「伊洛之學」教後進。宋可不願仕,人執其子為質,寧棄而不就,遂以無子。雖制行過中,豈不賢於殺妻以求大將者乎。大夫士見善明、用心剛,故能為人所難為者如此。



\end{pinyinscope}