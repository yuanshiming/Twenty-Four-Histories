\article{志第一}

\begin{pinyinscope}

 天文



 自伏羲仰觀俯察,黃帝迎日推策,重黎序天地,堯歷日月星辰,舜齊七政,周武王訪箕子,陳《洪範》,協五紀,而觀天之道備矣。《易》曰:「天垂象,見吉兇,聖人象之。」故孔子因魯史作《春秋》,於日星風雨霜雹雷霆皆書變而不書常,所以明天道、驗人事也。秦漢而下,治日患少,陰陽愆違,天象錯迕,無代無之。金百有十九年,而日食四十二,
 星辰風雨霜雹雷霆之變,不知其幾。金九主,莫賢於世宗,二十九年之間,猶日食者十有一,日珥虹貫者四五。然終金之世,慶雲環日者三,皆見於世宗之世。



 羲、和之後,漢有司馬,唐有袁、李,皆世掌天官,故其說詳。且六合為一,推步之術不見異同。金、宋角立,兩國置歷,法有差殊,而日官之選亦有精粗之異。今奉詔作《金史》,於志天文,各因其舊,特以《春秋》為準云。



 ○日薄食煇珥雲氣



 太祖天輔三年夏四月丙子朔,日食。四年冬十月戊辰朔,日食。六年春二月庚寅朔,日食。七年秋八月辛巳朔,日食。



 太宗天會七年三月己卯,日中有黑子。九月丙午朔,日食。十三年正月丙午朔,日食。



 熙宗天會十四年十一月丙寅,日中有黑子,斜角交行。



 天眷三年七月癸卯朔,日食。



 皇統三年十二月癸未朔,日食。四年六月辛巳朔,日食。五年六月乙亥朔,日食。八年四月戊子朔,日食。九年三月癸未朔,日食。



 海陵庶人天德二年正月甲辰,日有暈珥,白虹貫之。十一月丙戌,白虹貫日。十二月乙卯,慶雲見,狀如鸞鳳,五彩。三年正月丁酉,白虹貫日。



 貞元二年五月癸丑朔,日食。三年四月丁丑朔,昏霧四塞,日無光,凡十有七日乃霽。五月丁未朔,日食。



 正隆三年三月辛酉朔,有司奏日食,候之不見。海陵敕,自今日食皆面奏,不須頒告中外。五年八月丙午朔,日食。庚午,日中有黑子,狀如人。六年二月甲辰朔,日有暈珥,戴背。十月丙午,慶雲見。



 世宗大定二年正月戊辰朔,日食,伐鼓用幣,命壽王京代拜行禮。為制,凡遇日月虧食,禁酒、樂、屠宰一日。三年六月庚申朔,日食,上不視朝,命官代拜。有司不治務,過時乃罷。後為常。四年六月甲寅朔,日食。七年四
 月戊辰朔,日食,上避正殿、減膳,伐鼓應天門內,百官各於本司庭立,明復乃止。閏七月己卯午刻,慶雲環日。八月辛亥午刻,慶雲環日。九年八月甲申朔,有司奏日當食,以雨不見。為近奉安太社,乃伐鼓于社,用幣于應天門內。十三年五月壬辰朔,日食。十四年十一月甲申朔,日食。十六年三月丙午朔,日食。十七年九月丁酉朔,日食。二十三年十月己未,慶雲見於日側。十一月壬午朔,日食。二十八年八月甲子朔,日食。二十九年正月乙卯巳初,日有暈,左右有珥,上有背氣兩重,其色青赤而厚。復有白虹貫之亙天,其東有戰氣
 長四尺餘,五刻而散。丁巳巳初,日有兩珥,上有背氣兩重,其色青赤而淡。頃之,背氣於日上為冠,已而俱散。二月辛酉朔,日食。甲子辰刻,日上有重暈兩*,抱而復背,背而復抱,凡二三次。乙丑,日暈兩珥,有負氣承氣,而白虹亙天,左右有戟氣。



 章宗明昌三年十二月丙辰,北方微有赤氣。四年九月癸未,日上有抱氣二,戴氣一,俱相連。左右有珥,其色鮮明。六年三月丙戌朔,日食。



 承安三年正月己亥朔,日食,陰雲不見。五年十一月癸丑,日食。《宋史》作六
 月乙酉朔。



 泰和二年五月甲辰朔,日食。三年十月戊戌,日將沒,色赤如赭。甲辰,申酉間,天色赤,夜將旦復然。四年三月丁卯,日昏無光。五年九月戊子戌時,西北方黑雲間有赤氣如火,次及西南、正南、東南方皆赤,中有白氣貫徹,乍隱乍見。既而為雨,隨作風。至二更初,黑雲間赤氣復起於西北方,及正西、正東、東北,往來遊曳,內有白氣數道,時復出沒。其赤氣又滿中天,約四更皆散。六年正月,北京申,龍山縣西見有雲結成車牛行帳之狀,或如前後摧損之勢,晡時乃散。二月壬子朔,日食。七月癸巳,申刻,日有上背氣一,內赤外青,須臾散。九月乙酉,
 夜將曙,北方有赤白氣數道,歷王良下,徐行至北斗開陽、搖光之東而散。八年四月癸卯,巳刻,日暈二重,風黃外赤,移時而散。



 衛紹王大安元年四月壬申,北方有黑氣如大道,東西竟天,至五更散。十二月辛酉朔,日食。三年三月辛酉辰刻,北方有黑氣如堤,內有白氣三,似龍虎之狀。十月己卯,東北、西北每至初更如月將出狀,明至夜半而滅,經月乃已。



 宣宗貞祐元年十月丙午,夜有白氣三,衝紫微而貫。十一月丙申,白氣東西竟天,移時散。二年九月壬戌
 朔,日食,大星皆見。三年正月壬戌,日有左右珥,上有冠氣,移刻散。二月丁巳,日初出赤如血,將沒復然。六月戊申,夜有黑氣,廣如大路,自東南至於西北,其長竟天。四年二月甲申朔,日食。閏七月壬午朔,日食。



 興定元年七月丙子朔,日食。二年七月庚午朔,日食。三年七月庚申,五色雲見。十月乙丑,平涼府慶雲見,遣官驗實,以告太廟,詔國中。五年正月,山東行省蒙古綱奏慶雲見,命圖以進。四月丙子,日正午,有黃暈四匝,其色鮮明。五月甲申朔,日食。



 六月戊寅,日將出,有氣如大道,經丑未,歷虛危,東西不見首尾,移時沒。十二月
 己巳,北方有白氣,廣三尺餘,東西亙天。



 元光元年十一月丁未,東北有赤雲如火。二年五月辛未,日暈不匝而有背氣。九月庚子朔,日食。



 哀宗正大二年正月甲申,有黃黑昆。三年月三月庚午,省前有氣微黃,自東北亙西南,其狀如虹,中有白物十餘,往來飛翔,又有光倏見如二星,移時方滅。四年十一月乙未,日上有虹,背而向外者二,約長丈餘,兩旁俱有白虹貫之。是年六月丙辰,有白氣經天。或云太白入井。五年十二庚子朔,日食。八年三月庚戌酉正,日忽白而失色,乍明乍暗,左右有氣似日而無光,與日
 相凌,而日光四出搖盪至沒。



 天興元年正月壬午朔,日有兩珥。三年正月己酉,日大赤無光,京、索之間雨血十餘里。是日,蔡城陷,金亡。



 ○月五星凌犯及星變



 太宗天會七年十月甲寅,天旗明,河鼓直。十年閏四月丙申,熒惑入氐。八月辛亥,彗星出於文昌。十一年五月乙丑,月忽失行而南。頃之復故。七月己巳昏,有大星隕于東南,如散火。十二月丙戌,月食昴。



 熙宗天會十三年十一月乙酉,月食,命有司用幣以救,著為命。十四年正月辛巳,太白晝見,凡四十餘日伏。壬
 辰,熒惑入月。三月丁酉夜,中星搖。九月癸未,有星大如缶,起西南,流于正西。十一月己巳,狼星搖。十五年正月戊辰,歲星犯積尸氣。



 天眷二年三月辛巳朔,歲星留逆在太微。五月戊子,太白晝見。八月丁丑,太白晝見;九月辛巳,犯軒轅左星;乙巳,犯左執法;十一月戊寅,入氐。三年七月壬戌,月犯畢。十二月壬午,月掩東井東轅南第一星。



 皇統元年二月甲戌,月犯掩畢大星。二年十一月己酉,月犯軒轅大星。甲寅,月犯氐東北星。三年正月己丑,熒惑逆犯軒轅次北一星。二月乙丑,月犯畢大星。閏四
 月癸巳,月掩軒轅左角星。八月丙申,老人星見。九月丁丑,月犯軒轅大星。四年八月癸未,熒惑入輿鬼。五年四月丙申,彗星見於西北,長丈餘,至五月壬戌始滅。六月甲辰,熒惑犯左執法。六年九月戊寅,熒惑犯西垣上將。己丑,月犯軒轅第二星。七年正月辛未,彗星出東方,長丈餘,凡十五日滅。丁亥,太白經天。七月己巳,太白經天。庚辰,熒惑犯房第二星。十一月壬戌,歲星逆犯井東扇第二星。八年閏八月丙子,熒惑入太微垣。十月甲申,太白晝見;十一月壬辰,經天。十二月丙寅,太白晝見。九年二年癸亥,月掩軒轅第二星。七月甲辰,太白、
 辰星、歲星合于張。丁未,熒惑犯南斗第四星。八月壬子,又歷南斗第三星。



 海陵天德元年十二月甲子,土犯東井東星。二年正月乙酉,月犯昴;壬辰,犯木;乙未,犯角;二月丙寅,犯心大星。九月乙亥,太白晝見,至明年正月辛卯後不見。丁酉,月犯軒轅左角;十月乙丑,犯太微上將;十二月癸丑,犯昴。三年二月丙辰,月食。十月丁亥,月犯軒轅左角。四年正月癸卯,太白經天。二月乙亥,月掩鬼,犯鎮星。五月己亥,太白經天;丁巳,又經天。六月癸巳,太白犯井東第二星。八月辛未,太白犯軒轅大星。十一月甲辰,熒惑
 犯鉤鈐。丙午,月犯井北第一星。十二月乙卯朔,太白經天。丙子,月食。閏月己亥,太白經天。



 貞元年正月辛丑,月犯井東第一星。四月戊寅,有星如杯,自氐入于天市,其光燭地。十二月乙卯,太白經天。庚午,月食。閏月乙酉,太白經天。二年正月庚申,太白經天。是夜,月掩昴;二月辛丑,犯心前星,三月辛巳,月食。七月癸丑,太白晝見,凡三十有三日伏。八月戊戌,熒惑入井,凡十一日而出。十一月甲子,月食。三年八月乙酉,月犯牛;九月辛亥,犯建星;十一月戊午,掩井鉞星。



 正隆二年正月庚辰,太白晝見,凡六十七日伏。三年
 正月丁亥,有流星如杯。長二丈餘,其光燭地,出太微,沒於梗河之北。二月乙卯,熒惑入鬼。辛巳,月食。甲午,月掩歲星;六月丁酉,犯氐。九月己未,太白經天,至明年正月二十一日不見。十二月戊申,月入氐。四年九月壬寅,月掩軒轅右角;十一月壬辰,入畢,犯大星。十二月,太白晝見,凡七日。五年正月,海陵問司天提點馬貴中曰:「朕欲自交伐宋,天道如何?」貴中對曰:「去年十月甲戌,熒惑順入太微,至屏星,留退西出。《占書》熒惑常以



 十月入太微庭,受制出伺無道之國。又去年十二月,太白晝見經天,占為兵喪,為不臣,為更主,又主有兵兵罷,無兵兵
 起。」甲午,月食。二月丁卯,太白晝見。四月甲戌,復見,凡百六十有九日乃伏。六年七月乙酉,月食。九月丙申,太白晝見。先是,海陵問司馬貴中曰:「近日天道何如?」貴中曰:「前年八月二十九日太白入太微右掖門,九月二日至端門,九日至左掖門出,並歷左右執法。太微為天子南宮,太白兵將之象,其占:兵入天子之庭。」海陵曰:「今將征伐,而兵將出入太微,正其事也。」貴中又言:「當端門而出,其占為受制,歷左右執法為受事,此當有出使者,或為兵,或為賊。」海陵曰:「兵興之際,小賊固不能無也。」是歲,海陵南伐,遇弒。



 世宗大定元年十月丙午,熒惑入太微垣,在上將東。丁巳,月犯井西扇北第二星。二年正月癸巳,太白晝見。閏二月戊寅,月掩軒轅大星;三月戊申,掩太微東籓南第一星;八月乙酉,犯井西扇北第二星;九月庚戌,犯畢距星。十月戊辰,有大星如太白,起室壁間,沒於羽林軍,尾跡長丈餘。三年正月庚子,太白晝見,凡百有十日乃伏。五月辛丑,月入氐。七月庚戌,太白晝見,百二十有七日乃伏。八月丁未,月犯井距星。丙寅,太白晝見,經天。十月庚辰,月犯太微垣西上將星。十一月庚寅,太白晝見。經天。歲星入氐。凡二十四日伏。壬子,月入氐。四年
 正月戊子,熒惑、歲星同居氐。己丑,熒惑出氐。二月壬午,歲星退入氐,凡二十九日。九月丙午,月犯軒轅大星北次星。十一月丙申,月食,既。十二月辛卯,太白晝見經天。癸卯,月掩房北第一星。五年正月癸亥,月掩軒轅大星北次星;八月丁酉,犯井東扇第一星。十一月癸丑,熒惑入氐,凡二十一日。六年二月丙申,月犯南半東南第二星;三月己未,入氐。四月辛丑,太白晝見,八十有八日伏。六月辛巳,太白晝見;經天。九月壬子,太白晝見,百有三日乃伏;丙辰,經天;十月壬辰,復晝見,經天。十一月辛亥,金入氐,凡七日。庚申,太白晝見,經天;十二月戊子,
 復見,經天。癸巳,月犯上房北第二星。七年十月乙巳,火入氐,凡四日。十一月壬申,太白晝見,九十有一日伏。丁丑,歲星晝見,二日。八年正月癸未,月掩心大星;三月庚午,掩軒轅大星北一星。己丑,太白晝見,百五十有八日乃伏。五月丁卯,歲星晝見。八月甲午,太白軒轅大星。十月庚子,月掩熒惑;十一月庚午,犯昴。九年正月戊寅,月掩心後星;四月庚子,掩心前星;八月癸卯,掩昴;十二月丙戌,犯土。丁酉,太白晝見,十有六日伏。十年正月丙寅,月掩軒轅大星;七月庚子,犯五車東南星。八月戊申朔,木星掩熒惑,在參畢間。十一年二月壬戌,
 熒惑犯井東扇北第一星。八月癸卯,太白晝見,十二年五月辛巳,月犯心後星;



 八月癸卯,犯心大星。辛亥,熒惑掩井東扇北第二星。九月丁亥,太白晝見,在日前,九十有八日伏。十月己酉,熒惑掩鬼西北星。歲星晝見,在日後,四十有七日伏。十三年閏正月辛酉,太白晝見,四十有九日伏。二月己丑,熒惑犯鬼西北星;三月癸巳朔,入鬼;次日,犯積尸氣。六月辛未,月犯心前星。十月乙丑,歲星晝見於日後,五十有三日伏。十四年三月辛丑,太白歲星晝見,十有八日伏;丙辰,二星經天,凡二日。六月己未,太白晝見,三十有九日;八月己卯,晝見,又百三十
 二日乃伏。庚辰,熒惑犯積尸氣。十月丙寅,歲星晝見,六日。十五年十一月甲子,太白晝見,八十有六日伏。



 十二月乙丑,月掩井西扇北第一星。十六年三月庚申,月食。五月甲寅,太白晝見,五十有四日伏。庚午,月掩太白;七月丁未,犯角宿距星;甲子,掩畢宿距星。八月丙子,太白犯軒轅大星。九月丁巳,月食。十月丁丑,熒惑入太微。十一月甲寅,月掩畢星。戊辰,太白犯軒轅大星。九月丁巳,月食。十月丁丑,熒惑入太微。十一月甲寅,月掩畢距星。戊辰,熒惑犯太微上將。十二月己丑,月掩太微左執法。十七年春正月丙寅,熒惑犯太微西籓上相。九月庚戌,歲星、熒惑、太白聚於尾。十二月己巳,太白晝見,四十有四日伏。十八年七月庚辰,土星
 犯井東扇北第二星。九月己丑,熒惑犯左執法。十二月甲午,鎮星掩井西扇第一星,凡十日。十九年正月甲戌,月食,既。三月甲戌,熒惑犯氐距星。四月丁巳,歲星晝見,凡七日。七月丙子,太白晝見,四十有五日伏;八月癸卯,犯軒轅御女。辛亥,熒惑掩南斗杓第二星。九月壬申,月掩畢大星。十一月辛未,熒惑掩歲星。十二月丁亥,月犯歲星。二十年二月己丑,月掩畢大星;三月丙辰,掩畢西第二星。二十一年二月戊子,月犯鎮星。戊戌,太白晝見。三月甲子,太白晝見。四月壬申,熒惑掩斗魁第二星,十有四日。六月甲戌,客星見于華蓋,凡百五十
 有六日滅。七月乙亥朔,熒惑順入斗魁中,五日。以下史闕。



 二十二年五月甲申,太白書見,六十有四日伏。七月戊子,歲星晝見,二日。八月戊辰,太白晝見,百二十有八日,其經天者六十四日。十一月辛未,熒惑行氐中。乙亥,太白入氐。辛巳夜,月食,既。癸未,熒惑太白皆出氐中。十二月戊戌,熒惑犯鉤鈐。二十三年五月己卯,月食,既。九月甲申,歲星晝見,五十有五日伏。十月辛酉,太白晝見,百四十有九日乃伏。十一月丁卯,歲星晝見,三十有三日伏。閏十一月庚申,歲星晝見,九十日伏。二十四年四月己未朔,太白晝見,百四十有五日乃伏。甲申,
 月掩太白。九月庚子,歲星犯軒轅大星,甲辰晝見,凡五十二日伏。十月壬申,太白、辰星同度。二十五年三月乙酉,太白與月相犯,九月丁亥,月在斗魁中,犯西第五星。十一月庚辰朔,歲星晝見,在日後,凡七十四日。壬午,太白晝見,在日後,百十有一日乃伏。十二月己未,月犯熒惑。甲子,太白晝見經天。二十六年三月丙戌,熒惑入井。鎮星犯太微東籓上相。壬辰,月食。四月丁丑,熒惑犯鬼西南星。七月丙申,月掩心前星。八月乙亥朔,日月五星會于軫。十二月乙未,月掩心前大星,又犯于後星。二十七年五月壬子,月犯心大星。六月庚辰,太白晝
 見,百七十有三日乃伏。癸巳,月掩昴;七月丙午,犯房南第一星。是日,太白晝見經天。十月己丑,太白入氐。十二月丁丑,月掩昴。二十八年正月己未,歲星留於房;甲子,守房北第一星。十一月丙申,鎮星入氐。庚子,太白晝見,在日前,四十有九日伏。十二月壬申,月掩昴。二十九年正月丁酉,土星留氐中,三十有七日逆行,後七十九日出氐。五月庚寅朔,太白晝見,在日後。六月丙辰,月犯太白,月北星南,同在柳宿。十一月己未,熒惑守軒轅,至戊辰退行,其色稍怒。十二月辛丑,月食,既。



 章宗明昌元年二月丁亥,太白晝見。六月丁酉,月食,既。
 十二月乙未,月食。二年六月壬辰,月食。十一月乙丑,金木二星見在日前,十三日方伏而順行,危宿在羽林軍士、壘壁陣下,光芒明天。十二月戊子,木金相犯,有光芒。三年三月戊戌,熒惑順行犯太微西籓上將。四月丁巳,月食。己未,熒惑掩右執法,色怒而稍赤。四年正月丙子,月有暈,白虹貫其中。八月己亥,卯初三刻,歲星見,未正二刻,太白見,俱在午位。其夜歲星留胃十三度,守天廩。十月戊申,月食。五年十月癸卯,月食。十一月癸丑,太白晝見,在日前,三十有三日伏。六年正月庚寅,太白晝見,在日前,百有二日乃伏。六月庚辰,復晝見,
 在日後,百六十七日,唯是日經天。



 承安元年四月,司天奏河津星象事,上諭宰相曰:「天道不測,當預防之。」八月壬戌,月食,九月壬午,太白晝見,在日前,百有七日乃伏。二年二月丁巳,太白晝見,在日後,百九十有五日乃伏;己未,經天。是夜,月食,既。三年正月甲寅,月食。七月庚戌,月食。五年五月庚午,月食,六月庚戌,月掩太白。



 泰和元年十一月辛酉,月食。二年五月己未,月食。三年三月癸未,月食。六月戊戌,太白晝見,在日後,百有十日乃伏。四年九月乙亥,月食。五年三月壬申,月
 食。閏八月己巳,月食。六年五月甲申,太白晝見,在日前,七十有六日;庚戌,經天。六月辛未,歲星晝見,在日後;七月戊申,經天。八月癸卯,月暈圍太白、熒惑二星。辛亥,歲星辰見,至夜五更,與東井距星相去七寸內。癸丑,夜半有流星如太白,其色赤,起於婁宿。己未卯正初刻,太白晝見,在日前。其夜五更,熒惑與輿鬼、積尸氣相犯,在七寸內。庚申卯正初,太白晝見,在日後。其夜五更初,熒惑在輿鬼、積尸氣中。壬申,太白晝見,經天,在日後。十月丙午,歲星犯東井距星。十一月壬午,太白入氐。七年正月丙戌初更,月有暈圍歲、鎮二星,在參畢間。辛卯,月
 食。三月癸丑,月掩軒轅大星。七月戊子,月食。九月己卯初更,月在南斗魁中。旦,歲星在輿鬼中。八年正月丙戌,月食。七月戊戌朔,太白晝見,在日後。八月壬戌,太白、歲星光芒相及,同在張一度。十一月庚子未刻,有流星如太白者二,光芒如炬,幾一丈,起東北,沒東南。



 衛紹王大安元年正月辛丑,有飛星如火,起天市垣,尾跡如赤龍之狀,移刻散。二月乙丑朔,太白晝見,經天。六月丁丑,月食。十月乙丑,月食熒惑。丙寅,歲星犯左執法。二年正月庚戌朔,日中有流星出,大如盆,其色碧,西行,漸如車輪,尾長數丈,沒于濁中,至地復起,光散如火,
 移刻滅。二月,客星入紫微中,其光散如赤龍之狀。三年正月乙酉,熒惑入氐中,凡十有一日乃出。二月,熒惑犯房;閏月,犯鍵閉星;十月癸巳,犯壘壁陣。



 崇慶元年春三月,日正午,日、月、太白皆相去咫尺。



 宣宗貞祐元年十一月丙子,熒惑入壘壁陣。二年二月庚戌,月食。八月丁未,月食。



 九月丁亥,太白晝見於軫。十一月庚辰,鎮星犯太微東垣上相。辛巳,熒惑犯房、鉤鈐。三年七月庚申,有流星如太白,其色青白,有尾出紫微垣北極之旁,入貫索中。己卯,月入畢,至戊夜犯畢大星。八月辛丑,月食,既。十二月庚寅,太白晝見於危,八
 十有五日伏。四年正月乙卯夜,中天有流星大如十,色赤長丈餘,墜於西南,其聲如雷。二月己亥,月食。四月丁酉,太白晝見於奎,百九十有六日乃伏。六月丙申,歲星晝見於奎,百有一日乃伏。閏七月乙未,月食;辛丑,犯畢。十一月丙戌,月暈歲星,歲在奎,月在壁;己丑,犯畢大星;十二月戊午,復犯畢大星。



 興定元年正月乙酉,月犯畢左股第二星。四月戊辰,太白晝見於井,百六十有二日乃伏。八月戊申,歲星晝見於昴,六十有七日伏。九月癸巳,月犯東井西扇第二星。十月癸丑,夜有流星大如杯,尾長丈餘,自軒轅起貫太
 微,沒於角宿之上。十一月癸未,月暈歲星、熒惑二星,木在胃,火在昴。丙戌,太白晝見。十二月戊午,月食。二年六月乙卯,月食。八月壬戌,有流星大如杯,尾長丈餘,其光燭地,起建星,沒尾中。一云自東北至西北而墜,其光如塔狀,先有聲如風,後若雷者三,窗紙皆震。十月庚申,月犯軒轅左角之少民星。十二月壬子,月食,既。三年五月庚戌,月食,既。壬子,太白晝見於參,三十有六日經天,又百八十四日乃伏。七月壬寅初昏,在星自西南來,其光燭地,狀如月而稍不圓,色青白,有小星千百環之,若迸火然,墜於東北,少頃有聲如鼓。八月丁卯,歲星犯
 輿鬼東南星。己巳,歲星晝見於柳,百有九日乃伏。十一月乙巳,月食。癸丑,白虹二,夾月,尋復貫之。四年正月庚子,月犯東井。三月甲寅,歲星犯鬼、積尸氣。五月甲辰,月食;六月戊辰,犯鎮星。己巳,太白晝見於張,百八十有四日乃伏。十一月壬辰,歲星晝見於翼,六十有七日,夜又犯靈臺北第一星。五年正月辛丑,太白晝見於牛,二百三十有二日乃伏。司天夾谷德玉等奏以為臣強之象,請致祭以禳之。宣宗曰:「斗、牛吳分,蓋宋境也。他國有災,吾禳之可乎。」九月庚戌,歲星犯左執法。閏十二月戊子,熒惑犯軒轅。甲午,月犯熒惑。戊戌,鎮星晝見於軫。
 己亥,太白晝見於室。六年正月辛酉,月犯熒惑;壬戌,犯軒轅。三月壬子,月食太白。癸亥,月食。丙寅,歲星犯太微左執法。七月乙亥,太白經天,與日爭光。八月己卯,彗星出於亢宿、右攝提、周鼎之間,指大角,太史奏:「除舊布新之象,宜改元修政以消天變。」於是改是年為元光元年。九月丁未,滅。壬申,月食歲星。



 元光二年八月乙亥,熒惑入輿鬼,掩積尸氣;十月壬午,犯靈臺;十一月,又犯心大星。



 哀宗正大元年正月丙午,月犯昴;三月癸丑,犯熒惑。是月,熒惑逆行犯左執法。四月癸酉,熒惑犯右執法。乙未,
 太白、辰星相犯。三年十一月丙辰,月掩熒惑。丁巳,熒惑犯歲星;庚申,犯壘壁陣。癸酉,五星並見於西南。十二月,熒惑入月。四年正月壬戌,熒惑犯太白。六月丙辰,太白入井。七月丁亥,熒惑犯斗從西第二星。五年五月乙酉,月掩心大星。七年十月己巳,月暈,至五更復有大連環貫之,絡北斗,內有戟氣。十二月庚寅,有星出天津下,大如鎮星而色不明,初犯輦道,二日見於東北,在織女南;乙未,入天市垣,戊申方出;癸丑,歷房北,復東南行,入積薪,凡二十五日而滅。



 天興元年七月乙巳,太白、歲星、熒惑、太陰俱會於軫、翼,
 司天武亢極言天變,上惟嘆息,竟亦不之罪也。八月甲戌,太白、歲星交。閏九月己酉,彗星見東方,色白,長丈餘,彎曲如象牙,出角、軫南行,至十二日長二丈,十六日月燭不見,二十七日五更復出東南,約長四丈餘,至十月一日始滅,凡四十有八日。司天奏其咎在北,哀宗曰:「我亦北人,今日之事,我當滅也,何乃不先不後,適丁此乎!」



\end{pinyinscope}