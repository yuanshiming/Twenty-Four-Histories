\article{志第十二}

\begin{pinyinscope}

 禮四



 ○奏告儀皇帝恭謝儀皇后恭謝儀皇太子恭謝儀薦新功臣配享寶玉雜儀奏告儀



 皇帝即位、加元服、受尊號、納后、冊命、巡狩、征伐、封祀、請謚、營修廟寢,凡國有大事皆告。或一室,或遍告及原廟,並一獻禮,用祝幣。皇統以後,凡皇帝受尊號、冊皇后太子、禘祫、升祔、奉安、奉遷等事皆告,郊祀則告配帝之室。大定十四年三月十七日,詔更御名,命左丞相
 良弼告天地,平章守道告太廟,左丞石琚告昭德皇后廟,禮部尚書張景仁告社稷,及遣官祭告五嶽。前期二日,太廟令掃除廟內外,設告官以下次所。前一日,行事官赴祀所清齋。告日前三刻,禮直官引太廟令帥其屬,入殿開室戶,掃除鋪筵,設几於北墉下,如時享儀。禮直官帥祀祭官陳幣篚於室戶之左,陳祝版於室戶之右案上。及設香案祭器,皆藉以席,每位各左一籩實以鹿脯,右一豆實以鹿臡。犧尊一,置於坫,加勺、冪,在殿上室戶之左,北向,實以酒,每位一瓶。設燭於神位前。又設盥爵洗位橫街之南稍東。設告官褥位,於殿下東階之南,
 西向,餘官在其後稍南。又設望燎位於西神門外之北。告日未明,禮直官引太廟令、太祝、宮闈令入,當階間北面西上立定。奉禮贊:「再拜。」訖,升自西階,太祝、宮闈令各入室,出神主設於座,如常儀。次引告官入,就位。禮直官稍前,贊:「有司謹具,請行事。」又贊:「再拜。」在位者拜,訖,禮直官引告官就盥洗位,盥手,訖,詣神位前,搢笏,跪,三上香。執事者以幣授奉禮郎,西向授告官。告官受幣,奠訖,執笏,俯伏,興,退就戶外位,再拜。詣次位行禮如上儀,訖,降復位。少頃,引告官再詣爵洗位,讀祝、舉祝官後從。至位,北向立,搢笏,洗拭爵,訖,授執事者。執笏升,詣酒尊所,西
 向立,執爵,執尊者舉冪酌酒,告官以授執事者。詣神位前,北向,搢笏,跪,執爵三祭酒,執笏,俯伏,興,退就戶外位,北向立俟,讀祝文,訖,再拜。詣次位行禮如上儀。訖,與讀祝官皆復位。禮直官贊曰:「再拜。」在位者皆再拜。次引告官以下詣望燎位,執事者取幣帛祝版置於燎,禮直官曰:「可燎。」半柴,禮直官贊:「禮畢。」告官以下退。署令闔廟門,瘞祝於坎。貞元四年正月,上尊號。前三日,遣使奏告天地,於常武殿拜天臺設褥位,昊天上帝居中,皇地祇居西少卻,行一獻禮。



 大定七年正月十三日,上尊號。前三日,命皇子判大興尹許王告天地,判宗正英王文告太
 廟。於自來拜天處設昊天上帝位,當中南向,皇地祇位次西少卻,並用坐褥位牌及香酒脯臡等。祝版三,學士院撰告祝文,書寫訖,進請御署,訖,以付禮部,移文宣徽院,并差控鶴官用案舁,覆以黃羅帕,隨所差告官詣祀所。前一日,告官等就局所致齋一日。告日質明,宣徽院、太常寺鋪設供具如儀。閣門舍人一員、太常博士一員引告官各服其服,以次就位。禮直官、舍人稍前,贊:「有司謹具,請行事。」贊者曰:「拜。」在位者皆再拜。禮直官先引執事官各就位。舍人博士次引告官詣盥洗、爵洗位,北向立,搢笏,盥手,帨手,洗爵,拭爵。執笏,詣酒尊所,搢笏,執爵,
 司尊者舉冪酌酒,告官以爵授奉爵酒官,執笏詣昊天上帝,皇地祇神位前再拜,每位三上香,跪奠酒,訖,以爵授奉爵官,執笏,俯伏,興。舉祝官跪舉,讀訖,俯伏,興。告官再拜。告畢。引告官以下降復位,再拜,訖,詣望燎位,燔祝版,再拜。半燎,告官已下皆退。



 皇帝恭謝儀



 大定七年正月,世宗受尊號,禮畢恭謝。前三日,太廟令帥其屬,灑掃廟庭之內外及陳設。尚舍於廟南門之西,設饌幔一十一室。殿中監帥尚舍視大次殿,又設皇帝版位於始祖神位前北向,又設飲福位於版位西南少卻,又設隨室奠拜褥位於神座前。大樂令
 設登歌於殿上,宮縣於殿下。又設皇太之位於阼階東南,又設親王位於其南稍東,宗室王使相位於其後。又設太尉、司徒以下行事官位於殿西階之西,東向,每等異位。又設文武群官位於橫階之南,東、西向。又設御洗位於阼階之東,又設太尉洗位於西階下橫階之南。又設齋郎位於東班群官之後,又設盥洗等官、并奉禮、贊者、大司樂、協律郎、大樂令等位,各如祫享之儀。又設尊彞祭器等於殿之上下,如時享之儀。前一日,禮官御史帥其屬,省牲、視濯滌,如常儀。



 其日質明,禮官御史帥太廟官、太祝官、宮闈令出神主,如時享儀。有司列黃麾仗
 二千人於應天門外。尚輦進金輅於應天門內。午後三刻、宣徽院奏請皇帝赴齋宿殿,文武群官並齋宿於所司。次日質明,俟諸衛各勒所部屯門列仗。導駕官分左右侍立於殿階下,並朝服。通事舍人引侍中詣齋殿,俯伏,跪稱:」臣某言,請中嚴。」俯伏,興。凡侍中奏請,准此。皇帝服通天冠、絳紗袍。少頃,侍中奏:「外辦。」皇帝出齋殿。即御座,群官起居訖,侍中奏:「請升輦。」皇帝升輦以出,侍衛警蹕如常儀。導駕官前導,至應天門,侍中奏:「請降輦升輅。」皇帝升輅,門下侍郎俯伏,跪奏:「請車駕進發。」俯伏,興。凡門下侍郎奏請,准此。車駕動,警蹕如常儀。至應天門外,
 門下侍郎奏:「請車駕少駐,敕侍臣上馬。」侍中前承旨,退稱曰:「制可。」門下侍郎退,傳制稱:「侍臣上馬。」通事舍人承傳:「敕侍臣上馬。」導駕官分左右前導,門下侍郎奏:「請車駕進發」。車駕動,稱:「警蹕。」不鳴鼓吹。典贊儀引皇太子常服乘馬至廟中幕次,更服遠遊冠、朱明衣,執圭。通事舍人文武群官並朝服於廟門外班迎。車駕至廟門,侍中於輅前奏:「請降輅。」導駕官步入廟門稍東,侍中奏:「請升輦。」皇帝升輦,傘扇侍衛如常儀。至大次,侍中奏:「請降輦,入就大次。」皇帝入大次。



 通事舍人分引文武群官由南神東西偏門入廟庭,東西相向立。禮直官引太尉以下行
 事官詣橫街北向,再拜,訖,禮直官引太尉詣盥洗位,搢笏,盥手,帨手,執笏,詣爵洗位,北向立,搢笏,洗瓚,拭瓚,以瓚授執事者,執笏,由西階升殿,詣始祖尊彞所,西向立。執事者以瓚奉太尉,太尉搢笏,執瓚酌鬯,詣神位前,以鬯稞地,訖,以虛瓚授執事者,執笏,俯伏,興,出戶外北向,再拜,訖。次詣隨室並如上儀。禮畢,降自西階,復位。禮直官引司徒出詣饌所,引薦俎齋郎奉俎、并薦籩豆簠簋官奉籩豆簠簋,及太官令,以序入自正門,宮縣樂作,至大階,樂止。諸太祝迎於階上,各設於神座前。先薦牛,次薦羊,次薦豕,訖,禮直官引司徒已下降階復位。典贊儀
 引皇太子,通事舍人引親王,由南神東偏門入,詣褥位。禮直官引中書侍郎、舉冊官等升自西階,詣始祖室前,東西立。通事舍人引侍中詣大次前,奏:「請中嚴。」皇帝服袞冕。少頃,侍中奏:「外辦。」侍中詣廟庭本位立,皇帝將出大次,禮儀使與太常卿贊導。凡禮儀使與太常卿贊導,並博士前引,俯伏,跪稱:「臣某贊導皇帝行禮。」俯伏,興。前導至東神門,撤傘扇,近侍者從入。殿中監跪進鎮圭,禮儀使奏:「請執圭。」皇帝執圭,宮縣樂作。奏:「請詣罍洗位。」至位,樂止。內侍跪取匜,興,沃水。又內侍跪取槃,承水。時寒,預備溫水。禮儀使奏:「請搢鎮圭。」皇帝搢鎮圭,盥手。內侍
 跪取巾於篚,興,進,皇帝帨手,訖,奉爵官以爵跪進,皇帝受爵,內侍捧匜沃水,又內侍跪捧槃承水,皇帝洗爵,訖,內侍跪奉巾以進,皇帝拭爵,訖,內侍奠槃匜,又奠巾於篚。奉爵官受爵。禮儀使奏:「請執鎮圭。」前導皇帝升殿,左右侍從量人數升,宮縣樂作。皇帝至阼階下,樂止。皇帝升自阼階,登歌樂作。禮儀使前導,皇帝至版位,樂止。奏:「請再拜。」奉禮郎贊:「皇太子已下在位群官皆再拜。」贊者承傳,皆再拜。禮儀使前導,皇帝詣始祖尊彞所,樂作,至尊所,樂止。奉爵官以爵蒞尊,執尊者舉冪,侍中跪酌犧尊之汎齊,訖,禮儀使導皇帝至版位,再拜,訖,禮儀使奏:「
 請詣始祖神位前褥位。」登歌樂作。禮儀使奏:「請搢圭。」跪,奉爵官以爵授奉爵酒官以進。禮儀使奏:「請執爵。」皇帝執爵,二奠酒,訖,以虛爵授奉爵酒官。禮儀使奏:「請執圭。」俯伏,興,樂止。奉爵酒官以爵授奉爵官。禮儀使奏:「請詣隨室。」並如上儀。禮直官先引司徒升自西階,立於飲福位之側,酌獻將畢,奉胙,酌福酒。太祝從司徒立於其側,酌獻畢,侍中亦立於其側。禮儀使奏:「請皇帝詣版位。」北向立,登歌樂作,至位樂止。中書侍郎跪讀冊,訖,舉冊官奠,訖,禮儀使奏:「請皇帝再拜。」拜訖,禮儀使奏:「請詣飲福位。」登歌樂作。至位,太祝酌福酒於爵,時寒預備溫酒,以
 奉侍中,侍中受爵奉以立,禮儀使奏:「請搢圭。」跪,侍中以爵北向跪以進,禮儀使奏:「請執爵。」三祭酒。禮儀使奏:「請飲福。」飲福訖,以虛爵授侍中。禮儀使奏:「請受胙。」司徒跪以黍稷飯籩進,皇帝受以授左右。司徒又跪以胙肉進,皇帝受以授左右。禮儀使奏:「請執圭。」興,再拜訖,樂止。禮儀使前導,皇帝還版位,登歌樂作,至位樂止。太祝各進徹籩豆,登歌樂作。卒徹,樂止。奉禮曰:「賜胙。」贊:「皇太子已下在位群官皆再拜。」贊者承傳,皆再拜,宮縣作,一成止。禮儀使奏:「請皇帝再拜。」奉禮郎贊:「皇太子已下在位官皆再拜。」拜訖,禮儀使奏:「禮畢。」前導皇帝降阼階,登歌樂作,
 至階下樂止。宮縣作,前導皇帝出東神門,樂止。傘扇侍衛如常儀。禮儀使奏:「請釋圭。」殿中監跪受鎮圭。至大次,轉仗衛於還途,如來儀。禮宮御史帥其屬,納神主,藏冊如儀。少頃,通事舍人引侍中奏:「請中嚴。」皇帝服通天冠、絳紗袍。少頃,侍中奏:「外辦。」俟尚輦進輦,侍中奏:「請降座升輦。」皇帝升輦,傘扇侍衛如常儀。至南神門稍東,侍中奏:「請降輦步出廟門。」皇帝步出廟門,至輅,侍中奏:「請升輅。」皇帝升輅。門下侍郎奏:「請車駕少駐,敕侍臣上馬。」侍中前承旨,退稱曰:「制可。」門下侍郎退,傳制稱:「侍臣上馬。」通事舍人承傳:「敕侍臣上馬。」車駕還內,鼓吹振作,至應
 天門外,百官班迎起居,宮縣奏《采茨之曲》。入應天門內,侍中奏:「請降輅乘輦。」皇帝降輅乘輦以入,傘扇侍衛警蹕如常儀。皇帝入宮,至致齋殿,侍中奏:「解嚴。」通事舍人承旨:「敕群臣各還次,將士各還本所。」



 皇后恭謝儀



 皇后既受冊,前一日,齋戒於別殿。內命婦應從入廟者俱齋戒一日。其日未明二刻,有司陳設儀仗於后車之左右,以次排列。外命婦先自太廟後門入,內命婦妃嬪已下俱詣殿庭,起居訖,宜徽使版奏:「中嚴。」少頃,又奏:「外辦。」首飾禕衣,御肩輿,取便路至車所。內侍奏:「請降輿升車。」既升車,奏:「請進發。」車出元德東偏門,內
 命婦妃嬪已下自殿門外上車,由左掖門出,從至太廟門外,儀仗止於門外,回車南向。內侍奏:「請降車升輿。」后降車升輿,就東神門外幄次,下簾。內命婦妃嬪已下降車,入就陪列位。內侍引外命婦詣幄次前,起居訖,並赴殿庭陪列位。少頃,宣徽使詣幄次,贊:「行朝謁之禮。」簾卷,宜徽使前導,詣殿庭階下西向褥位立。宣徽使贊:「再拜。」內外命婦皆再拜。宣徽使前導,升東階,詣始祖皇帝神位香案前褥位,宣徽使奏:「請三上香。」又奏:「再拜。」拜訖。宣徽使前導,次詣獻祖已下十室,並如上儀。宣徽使奏:「禮畢。」導歸幄次。宣徽使奏:「請解身體。嚴。」內外命婦還幕次。少頃,
 轉仗還內如來儀,外命婦退。內侍奏:「請御輿。」出至車所,奏:「請升車。」既升車,奏:「請進發。」內命婦上車。至元德東偏門,內侍奏:「請降車升輿。」后御輿,取便路還內,內命婦從入。冊禮畢,百官上表稱賀,並以箋賀中宮。



 皇太子恭謝儀



 其日質明,東宮應從官各服朝服,所司陳鹵簿金輅於左掖門外。皇太子服遠游冠、朱明衣,升輿以出,至金輅所,降輿升輅。左庶子已下夾侍。三帥、三少乘馬導從,餘官亦皆乘馬以從。東行,由太廟西階轉至廟,不鳴鐃吹。至廟西偏門外降輅步進,由東偏門入幄次,改服袞冕。出次,執圭自南神東偏門入,宮官并太
 常寺官皆從。皇太子入詣殿庭東階之東,西向立,典儀贊:「再拜。」訖,升自西階,詣始祖神位前北向,再拜,訖,以次詣逐室行禮,並如上儀。訖,降自西階,復西向位俟,典儀稱:「禮畢。」出東神北偏門,謁別廟如上儀。訖,歸幄次,改服遠游冠、朱明衣。出次,步至廟門外升輅,過廟門鳴鐃而行。至左掖門外降輅,升輿以入。將士各還本所。後一日於東宮受群官賀,如元正受賀之儀。



 薦新



 天德二年,命有司議薦新禮,依典禮合用時物,令太常卿行禮。正月,鮪,明昌間用牛魚,無則鯉代。二月,雁。三月,韭,以卵、以葑。四月,薦冰。五月,荀、蒲,羞以含桃。六月,
 彘肉、小麥仁。七月,嘗雛雞以黍,羞以瓜。八月,羞以芡、以菱、以栗。



 九月,嘗粟與稷,羞以棗、以梨。十月,嘗麻與稻,羞以兔。十一月,羞以麕。十二月,羞以魚。從之。大定三年,有司言:「每歲太廟五享,若復薦新,似涉繁數。擬遇時享之月,以所薦物附於籩豆薦之,以合古者『祭不欲數』之義。」制可。牛魚狀似鮪,鮪之類也。



 功臣配享。



 明昌五年閏十月丙寅,以儀同三司代國公歡都、銀青光祿大夫冶訶、特進劾者、開府儀同三司盆納、儀同三司拔達,配享世祖廟庭。天德二年二月,太廟祫享,有司擬上配享功臣,詔以撒改,辭不失、斜也杲、斡
 魯。阿思魁忠東向,配太祖位。以粘哥宗翰、翰里不宗望、闍母。婁室、銀術可西向,配太宗位。大定三年十月,祫享,又以斜也、斡魯、撒改、習不失、阿思魁配享太祖,宗望、闍〗母、宗翰、婁室、銀術哥配享太宗。其後,次序屢有更易。八年,上命圖畫功臣於太祖廟,有司第祖宗佐命之臣,勛績之大小、官資之崇卑以次上聞。乃定左廡:開府金源郡王撒改、皇伯太師右副元帥宋王宗望、開府金源郡王斡魯、皇伯太師粱王宗弼、開府金源郡王婁室、皇叔祖元帥左都監魯王闍母、開府隋國公阿離合懣、儀同三司袞國公劉彥宗、右丞相齊國簡懿公韓企先、特進宗
 人習失;右廡:太師秦王宗翰、皇叔祖遼王杲、開府金源郡王習不失、開府金源郡王完顏希尹、太傅楚王宗雄、開府前燕京留守金源郡王完顏銀哥、開府金源郡王完顏忠、金源郡王完顏撒離喝、特進宗人斡魯古、右丞相金源郡王紇石烈志寧。十六年,左廡遷粱王宗弼於斡魯上。十八年,黜習失,而次蒲家奴於阿離合懣下。二十二年,增皇伯太師遼王,斜也、撒改、宗乾、宗翰、宗望,其下以次列。



 至明昌四年,次序始定。東廊:皇叔祖遼智烈王斜也杲、皇伯太師遼忠烈王宗乾斡本、皇伯太師右副元帥宋桓肅王訛魯補宗望、開府儀同三司金
 源郡毅武王習不失、開府儀同三司金源郡貞憲王顏谷神希尹、太傅楚威敏王謀良虎宗雄、開府儀同三司燕京留守金源郡襄武王完顏銀術可、開府儀同三司金源郡明毅王完顏忠阿思魁、金源郡莊襄王杲撒離喝、特進宗人斡里古莊翼,特進完顏習失威敬、太師尚書令淄忠烈王徒單克寧、太師尚書令南陽郡文康王張浩。西廊:開府儀同三司金源郡忠毅王撒改、太師秦桓忠王粘罕宗翰、皇伯太師粱忠烈王斡出宗弼、開府儀同三司金源郡剛烈王斡魯、開府儀同三司金源郡莊義王完顏婁室、皇叔祖元師左都監魯莊明王闍
 母、開府儀同三司隋國剛憲公阿離合懣、開府儀同三司豫國襄毅公蒲家奴昱、開府儀同三司袞國英敏公劉彥宗、右丞相齊國簡懿公韓企先、太保尚書令廣平郡襄簡王李石、開府儀同三司右丞相金源郡武定王紇石烈志寧、開府儀同三司左丞相沂國公僕散忠義、儀同三司左丞相崇國公紇石烈良弼、右丞相莘國公石琚、右丞相申國公唐括安禮、開府儀同三司平章政事徒單合喜、參知政事宗敘。每一朝為一列,著為令。



 寶玉



 凡天子大祀,則陳八寶及勝國寶於庭,所以示守也。金克遼、宋所得寶玉,及本朝所製,今并載焉。獲於遼
 者,玉寶四、金寶二。玉寶:「通天萬歲之璽」一,「受天明命惟德乃昌」之寶一皆方三寸,「嗣聖」寶一,御封不辨印文寶一。金寶:「御前之寶」一,「書詔之寶」一,二寶金初用之。獲於宋者,玉寶十五,金寶七、印一,金塗銀寶五。玉寶:受命寶一,咸陽所得,三寸六分,文曰「受命于天,既壽永昌」,相傳為秦璽,白玉蓋,螭紐。傳國寶一,螭紐。鎮國寶一,二面並碧色,文曰「承天休,延萬億,永無極」。又受命寶一,文曰「受命于天,既壽永昌」。「天子之寶」一。「天子信寶」一。「天子行寶」一。「皇帝之寶」一。「皇帝信寶」一。「皇帝行寶」一。「皇帝恭膺天命之寶」二。皆四寸八分,螭紐。「御書之寶」二,一龍紐,一螭紐。「宣
 和御筆之寶」一,螭紐。金寶并印:「天下同文之寶」一,龍紐。「御前之寶」二。「御書之寶」一。「宣和殿寶」一。「皇后之寶」一。「皇太子寶」一,龜紐。「皇太子妃」印一,龜紐。金塗銀寶:「皇帝欽崇國祀之寶」一,「天下合同之寶」一,「御前之寶」一,「御前錫賜之寶」一,「書詔之寶」一。外有宋內府圖書印三十八。「內府圖書之印」一、「御書」三、「御筆」一、「御畫」一,「御書玉寶」一、「天子萬年」一、「天子萬壽」一、「龜龍上珍」一、「河洛元瑞」二、「雲漢之章」一、「奎璧之文」一、「華國之瑞」一、「大觀中秘」一、「大觀寶篆」一、「政和」一、「宜和」三、「宜和御覺」一、「宣和中秘」一、「宣和殿制」一、「宣和大寶」一、「宣和書寶」二、「宣和畫寶」一、「常樂未央」一、古文二、「封」四,共三十五面,並玉。「封」字一、「御畫」一,二面並馬瑙。「政和御筆」一,系水晶。玄圭一,白玉圭一十九。



 本朝所製。國初就用遼寶,皇統五年始鑄金「御前之寶」一、「書詔之寶」一。
 大定十八年,得美玉,詔作「大金受命萬世之寶」,其制徑四寸八分,厚寸四分,盤龍紐高厚各四寸六分。二十三年,又鑄「宣命之寶」,其徑四寸二厘,厚一寸四分,紐高一寸九分,字深二分。敕有司議所當用,奏:「今所收八寶及皇統五年造『御前之寶』,賜宋國書及常例奏目則用之,『書詔之寶』,賜高麗、夏國詔并頒詔則用之。大定十八年造『大金受命萬世之寶』,奉敕再議。今所鑄金寶宜以進呈為始,一品及王公妃用玉寶,二品以下用金『宣命之寶』。」又有「禮信之寶」,用銅,歲賜三國禮物緘封用之,明昌間更以銀。又有太皇太后、皇太后、皇后、皇太妃寶,又有
 皇太子及守國寶,皆用金。大定二十四年,皇太子寶,金鑄龜紐,有司定其文曰「監國」,上命以「守」易「監」,比親王印廣長各加一分。



 雜儀



 大定三年八月,有司議:「祫享犧牲品物,按唐《開元禮》宋《開寶禮》每室犢一、羊一、豬一,《五禮新儀》每室復加魚十有五尾。天德、貞元例,與唐、宋同,有司行事則不用太牢,七祀功臣羊各二,酒共二百一十瓶。正降減定,通用犢一,兩室共用羊一豕一,酒百瓶,此於禮有闕。今七祀功臣牲酒請依天德制,宗廟每室則用宋制,加魚。然每室一犢恐太豐。」世宗乃命每祭共用一犢,羊豕如
 舊。又以九月五日祫享,當用鹿肉五十斤、麞肉三十五斤、兔十四頭為臡醢,以貞元、正降時方禁獵,皆以羊代,此禮殊為未備,詔從古制。十年正月,詔宰臣曰:「古禮殺牛以祭,後世有更者否?其檢討典故以聞。」有司謂:「自周以來,下逮唐、宋,祫享無不用牛者。唐《開元禮》時享每室各用太牢一,至天寶六年始減牛數,太廟每享用一犢。宋《政和五禮新儀》時享太廟,親祀用牛,有司行事則不用。宋開寶二年詔,昊天上帝、皇地祇用犢,餘大祀皆以羊豕代之。合二羊五豕足代一犢。今三年一祫乃為親祠,其禮至重,每室一犢後恐難省減。」遂命時享與祭社稷
 如舊,若親祠宗廟則共用一犢,有司行事則不用。十二年十月,祫享,以攝官行事,詔共用三犢。二十二年十月,詔祫禘共用三犢,有司行事則以鹿代。昭德皇后廟大定十九年禘祭,不用犢。



 大定二十九年,章宗即位,禮官言:「自大定二十七年十月袷享,至今年正月世宗升遐,故四月不行禘禮。按《公羊傳》,閔公二年『吉禘于莊公,言吉者未可以吉,謂未三年也』。注:『謂禘祫從先君數,朝聘從今君數,三年喪畢,遇禘則禘,遇祫則祫。』故事,宜於辛亥歲為大祥,三月禫祭,踰月則吉,則四月一日為初吉,適當孟夏禘祭之時,可
 為親祠。」詔從之。及期,以孝懿皇后崩而止。



 五月,禮官言:「世宗升祔已三年,尚未合食於祖宗,若來冬遂行祫禮,伏為皇帝見居心喪,喪中之吉《春秋》譏其速,恐冬祫未可行。然《周禮》王有哀慘則春官攝事,竊以世宗及孝懿皇后升祔以來,未曾躬謁,豈可令有司先攝事哉!況前代攝事者止施於常祀,今乞依故事,三年喪畢,祫則祫,禘則禘,於明昌四年四月一日釋心喪,行禘禮」。上從之。明昌三年十二月,尚書省奏:「明年親禘,室當用犢一。欽懷皇后祔於明德之廟,按大定三年祫享,明德皇后室未嘗用犢。」敕欽懷皇后亦用之。上因問拜數,右丞瑋
 具對,上曰:「世宗聖壽高,故殺其數,並不立於位,今當從禮而已。」



 大定六年,定晨稞行禮,自大次至板位先見神之禮,兩拜。再至板位,又兩拜。稞鬯畢,還板位,再兩拜。還小次,酌獻時,罍洗位盥訖,至板位,先兩拜。酌獻畢還板位,再兩拜。止將始祖祝冊於板位西南安置,讀冊訖,又兩拜。還小次,又至飲福位,先兩拜,飲畢兩拜。凡十六拜。貞祐四年,命參知政事李革為修奉太廟使,七月吉日親行祔享,有司以故事用皇帝時享儀,初至版位兩拜,晨稞及酌獻則每位三拜,飲福五拜,總七十九拜。今升祔則遍及祧廟五室,則為一百九拜也。明昌間嘗減每位
 酌獻奠爵後一拜,則為九十二拜而已。然大定六年,世宗嘗令禮官通減為十六拜。又皇帝當散齋四日于別殿,致齋三日于大慶殿,今國事方殷,宜權散齋二日,致齋一日。上曰:「拜數從大定例,餘準奏。」禮部尚書張行信言:「近奉詔從世宗十六拜之禮,臣與太常參定儀注,竊有疑焉。謹按唐、宋親祠典禮,皆有通拜及隨位拜禮。世宗大定三年親行奉安之禮,亦通七拜,每室各五拜,合七十二拜。逮六年禘,始敕有司減為十六拜,仍存七十二拜之儀,其意亦可見矣。蓋初年享禮以備,故後從權,更定通拜。今陛下初廟見奉安,而遽從此制,是於隨室
 神位並無拜禮,此臣之所疑一也;大定間十有二室,姑從十六拜,猶可。今十有七室。而拜數反不及之,此臣之所疑二也;況六年所定儀注,惟於皇帝板位前讀始祖一室祝冊。夫祭有祝辭,本告神明,今諸祝冊各書帝后尊謚,及高曾祖考世次不一,皇帝所自稱亦自不同,而乃止讀一冊,餘皆虛設,恐於禮未安,此臣之所疑三也。先王之禮順時施宜,不可多寡,惟稱而已。今近年禮官酌古今,別定四十四拜之禮。初見神二拜,晨稞通四拜,隨室酌獻讀祝畢兩拜,飲福四拜,似為得中。」上從之,乃定祔享如時享十二室之儀。又以祧廟五主始祖室不
 能容,止於室戶外東西一列,以西為上。神主闕者以升祔前三日廟內敬造,以享日醜前題寫畢,以次奉升。十月己未,親王百官自明俊殿奉迎祖宗神主於太廟幄次。辛酉行禮,用四十四拜之儀,無宮縣樂,犧牲從儉,十七室用犢三、羊豕九而已。以皇太子為亞獻,濮王守純為終獻。皇帝權服靴袍,行禮日服袞冕,皇太子以下公服,無鹵簿儀仗,禮畢乘馬還宮。



\end{pinyinscope}