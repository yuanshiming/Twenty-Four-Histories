\article{志第三十六}

\begin{pinyinscope}

 百官一



 ○三師三公尚書省六部都元帥府樞密院大宗正府御史臺宣撫司勸農使司司農司三司國史院翰林學士院審官院太常寺



 金自景祖始建官屬,統諸部以專征伐,嶷然自為一國。其官長皆稱曰勃極烈,故太祖以都勃極烈嗣位,太宗以諳版勃極烈居守。諳版,尊大之稱也。其次曰國論忽魯勃極烈,國論言貴,忽魯猶總帥也。又有國論勃極烈,
 或左右置,所謂國相也。其次諸勃極烈之上,則有國論、乙室、忽魯、移賚、阿買、阿舍、昊、迭之號,以為升拜宗室功臣之序焉。其部長曰孛堇,統數部者曰忽魯。凡此,至熙宗定官制皆廢。其後惟鎮撫邊民之官曰禿里,烏魯骨之下有掃穩脫朵,詳穩之下有麼忽、習尼昆,此則具於官制而不廢,皆踵遼官名也。



 漢官之制,自平州人不樂為猛安謀克之官,始置長吏以下。天輔七年以左企弓行樞密院于廣寧,尚踵遼南院之舊。天會四年,建尚書省,遂有三省之制。至熙宗頒新官制及換官格,除拜內外官,始定勳封食邑入銜,而後其制定。然大率皆循遼、
 宋之舊。海陵庶人正隆元年罷中書門下省,止置尚書省。自省而下官司之別,曰院、曰臺、曰府、曰司、曰寺、曰監、曰局、曰署、曰所,各統其屬以修其職。職有定位,員有常數,紀綱明,庶務舉,是以終金之世守而不敢變焉。大定二十八年,在仕官一萬九千七百員,四季赴選者千餘,歲數監差者三千。明昌四年奏,周歲,官死及事故者六百七十,新入仕者五百一十,見在官萬一千四百九十九,內女直四千七百五員,漢人六千七百九十四員。至泰和七年,在仕官四萬七千餘,四季部擬授者千七百,監官到部者九千二百九十餘,則三倍世宗之時矣。若
 宣宗之招賢所、經略司,義宗之益政院,雖危亡之政亦必列于其次,以著一時之事云。



 三師



 太師、太傅、太保各一員,皆正一品。師範一人,儀刑四海。



 三公



 太尉、司徒、司空各一員,皆正一品。論道經邦,燮理陰陽。



 尚書省



 尚書令一員,正一品。總領紀綱,儀刑端揆。左丞相、右丞相各一員,從一品。平章政事二員,從一品。為宰相,掌丞天子,平章萬機。左丞、右丞各一員,正二品。參知政事二員,從二品。為執政官,為宰相之貳,佐治
 省事。



 左司,郎中一員,正五品。國初置左、右司侍郎,天眷三年始更今名。舊凡視朝,執政官親執奏,自天德二年詔以付左、右司官,為定制。員外郎一員,正六品。掌本司奏事,總察吏、戶、禮三部受事付事,兼帶修起居注官,回避其間記述之事。每月朔朝,則先集是月秩滿者為簿,名曰闕本,及行止簿、貼黃簿、並官制同進呈,御覽畢則受而藏之。每有除拜,凡尚書省所不敢擬注者,則一闕具二三人以聽制授焉。都事二員,正七品。貞元二年,左右司官,宮中出身,並進士、令史三色人內通選。三年,以監察御史相應人取次稟奏,不復擬注。掌本司受事付事,檢勾稽失、省署文牘,兼知省內宿直,檢校架閣等事。右司所掌同。右司,郎中
 一員,正五品。員外郎一員,正六品。掌本司奏事,總察兵、刑、工三部受事付事,兼帶修注官,回避其間記述之事。都事二員,正七品。



 尚書省祗候郎君管勾官,從七品。掌祗候郎君,謹其出入及差遣之事。承安二年以前,走馬郎君擬注。《泰和令》,以左右女直都事兼。正大間,改用親從人。



 架閣庫大定二十一年六月設,仍以都事提控之。管勾舊二員,正大省一員,正八品。同管勾舊二員,正大省一員,從八品。總掌察左右司大程官追付文牘,並提控小都監給受紙筆,餘管勾同。女直省令史三十五人,左二十人,右十五人。大定二十四年為三十人,進士十人,宰執子、宗室子十人,密院臺部統軍司
 令史十人。漢令史三十五人,左二十一人、右十四人。省譯史十四人,左右各七人。女直譯史同。通事八人,左右各四人。高麗、夏國、回紇譯史四人,左右各二人。諸部通事六人。曳剌二十人。走馬郎君五十人。提點歲賜所,左右司郎中,員外郎兼之,掌提點歲賜出入錢幣之事。



 堂食公使酒庫。使一員,從八品。掌受給歲賜錢,總領庫事。副一員,正九品。掌貳使事。



 直省局。局長,從八品。掌都堂之禮及官員參謝之儀。副局長,正九品,掌貳局長。管勾尚書省樂工,從九品。



 行臺之制。熙宗天會十五年,罷劉豫,置行臺尚書省於
 汴。天眷元年,以河南地與宋,遂改燕京樞密為行臺尚書省。天眷三年,復移置於汴京。皇統二年,定行臺官品皆下中臺一等。



 六部,國初與左、右司通署,天眷三年始分治。



 吏部



 尚書一員,正三品。侍郎一員,正四品。郎中二員,從五品。天德二年,增作四員,後省。員外郎,從六品。天德二年,增作四員,後省。掌文武選授、勳封、考課、出給制誥之政。以才行勞效,比仕者之賢否;以行止、文冊、貼黃簿,制名闕之機要。正七品以上,以名上省,聽制授。從七品以下,每至季月則循資格而擬注,自八品以上則奏,以下則否。侍郎以
 下,皆為尚書之貳。郎中掌文武選、流外遷用、官吏差使、行止名簿、封爵制誥。一員掌勳級酬賞、承襲用廕、循遷、致仕、考課、議謚之事。員外郎分判曹務及參議事,所掌與郎中同。



 文官九品,階凡四十有二:從一品上曰開府儀同三司,中曰儀同三司,中次曰特進,下曰崇進。正二品上曰金紫光祿大夫,下曰銀青榮祿大夫。從二品上曰光祿大夫,下曰榮祿大夫。正三品上曰資德大夫,中曰資政大夫,下曰資善大夫。從三品上曰正奉大夫,中曰通奉大夫,下曰中奉大夫。正四品上曰正議大夫,中曰通議大夫,下曰嘉議大夫。
 從四品上曰大中大夫,中曰中大夫,下曰少中大夫。正五品上曰中議大夫,中曰中憲大夫,下曰中順大夫。從五品上曰朝請大夫,中曰朝散大夫,下曰朝列大夫。舊日奉德大夫,天德二年更。正六品上曰奉政大夫,下曰奉議大夫。從六品上曰奉直大夫,下曰奉訓大夫。正七品上曰承德郎,下曰承直郎。從七品上曰承務郎,下曰儒林郎。正八品上曰文林郎,下曰承事郎。從八品上曰徵事郎,下曰從仕郎。正九品上曰登仕郎,下曰將仕郎。從九品上曰登仕佐郎,下曰將仕佐郎。此二階,大定十四年創增。



 武散官,凡仕至從二品以上至從一品者,皆
 用文資。自正三品以下,階與文資同:正三品上曰龍虎衛上將軍,中曰金吾衛上將軍,下曰驃騎衛上將軍。從三品上曰奉國上將軍,中曰輔國上將軍,下曰鎮國上將軍。正四品上曰昭武大將軍,中曰昭毅大將軍,下曰昭勇大將軍。從四品上曰安遠大將軍,中曰定遠大將軍,下曰懷遠大將軍。正五品上曰廣威將軍,中曰宣威將軍,下曰明威將軍。從五品上曰信武將軍,中曰顯武將軍,下曰宣武將軍。正六品上曰武節將軍,下曰武德將軍。從六品上曰武義將軍,下曰武略將軍。正七品上曰承信校尉,下曰昭信校尉。
 從七品上曰忠武校尉,下曰忠顯校尉。正八品上曰忠勇校尉,下曰忠翊校尉。從八品上曰修武校尉,下曰敦武校尉。正九品上曰保義校尉,下曰進義校尉。從九品上曰保義副尉,下曰進義副尉。此二階,大定十四年創增。



 封爵:正從一品曰郡王,曰國公。正從二品曰郡公。正從三品曰郡侯。正從四品曰郡伯。舊曰縣伯,承安二年更。正五品曰縣子,從五品曰縣男。



 凡勳級:正二品曰上柱國,從二品曰柱國。正三品曰上護軍,從三品曰護軍。正四品曰上輕車都尉,從四品曰輕車都尉。正五品曰上騎都尉,從五品曰騎都尉。正六品曰驍騎尉,從六
 品曰飛騎尉。正七品曰雲騎尉,從七品曰武騎尉。



 凡食邑:封王者萬戶,實封一千戶。郡王五千戶,實封五百戶。國公三千戶,實封三百戶。郡公二千戶,實封二百戶。郡侯一千戶,實封一百戶。郡伯七百戶,縣子五百戶,縣男三百戶,皆無實封。自天眷定制,凡食邑,同散官入銜。



 司天翰林官,舊制自從七品而下止五階,至天眷定制,司天自從四品而下,立為十五階:從四品上曰欽象大夫,中曰正儀大夫,下曰欽授大夫。正五品上曰靈憲大夫,中曰明時大夫,下曰頒朔大夫。從五品上曰雲紀大夫,中曰協紀大夫,下曰保章大
 夫。正六品上曰紀和大夫,下曰司玄大夫。從六品上曰探賾郎,下曰授時郎。正七品上曰究微郎,下曰靈臺郎。從七品上曰明緯郎,下曰候儀郎。正八品上曰推策郎,下曰司正郎。從八品上曰校景郎,下曰平秩郎。正九品上曰正紀郎,下曰挈壺郎。從九品上曰司歷郎,下曰司辰郎。



 太醫官,舊自從六品而下止七階,天眷制,自從四品而下,立為十五階:從四品上曰保宜大夫,中曰保康大夫,下曰保平大夫。正五品上曰保頤大夫,中曰保安大夫,下曰保和大夫。從五品上曰保善大夫,中曰保嘉大夫,下曰保順大夫。正六品
 上曰保合大夫,下曰保沖大夫。從六品上曰保愈郎,下曰保全郎。正七品上曰成正郎,下曰成安郎。從七品上曰成順郎,下曰成和郎。正八品上曰成愈郎,下曰成全郎。從八品上曰醫全郎,下曰醫正郎。正九品上曰醫效郎,下曰醫候郎。從九品上曰醫痊郎,下曰醫愈郎。



 內侍,天德創制,自從四品以下,十五階:從四品上曰中散大夫,中曰中尹大夫,下曰中侍大夫。正五品上曰中列大夫,中曰中御大夫,下曰中儀大夫。從五品上曰中常大夫,中曰中益大夫,下曰中衛大夫。正六品上曰中良大夫天德作中亮,下曰中涓大夫。從
 六品上曰通禁郎,下曰通侍郎。正七品上曰通掖郎,下曰通御郎。從七品上曰禁直郎,下曰侍直郎。正八品上曰掖直郎,下曰內直郎。從八品上曰司贊郎,下曰司謁郎。正九品上曰司閽郎,下曰司僕郎。從九品上曰司奉郎,下曰司引郎。



 教坊,舊用武散官,大定二十九年以為不稱,乃創定二十五階。明昌三年,自從四品以下,更立為十五階:從四品上曰雲韶大夫,中曰仙韶大夫,下曰成韶大夫。正五品上曰章德大夫,中曰長寧大夫,下曰德和大夫。從五品上曰景雲大夫,中曰雲和大夫,下曰協律大夫。正六品上曰慶喜
 大夫,下曰嘉成大夫。從六品上曰肅和郎,下曰純和郎。正七品上曰舒和郎,下曰調音郎。從七品上曰比音郎,下曰司樂郎。正八品上曰典樂郎,下曰協樂郎。從八品上曰掌樂郎,下曰和樂郎。正九品上曰司音郎,下曰司律郎。從九品上曰和聲郎,下曰和節郎。



 凡內外官之政績,所歷之資考,更代之期,去就之故,秩滿皆備陳於解由,吏部據以定能否。又撮解由之要,於銓擬時讀之,謂之銓頭。又會歷任銓頭,而書于行止簿。行止簿者,以姓為類,而書各人平日所歷之資考功過者也。又為簿,列百司官名,有所更代,則以小
 黃綾書更代之期,及所以去就之故,而制其銓擬之要領焉。



 凡縣令,則省除、部除者通書而各疏之。泰和四年,定考課法,準唐令,作四善、十七最之制。四善之一曰德義有聞,二曰清慎明著,三曰公平可稱,四曰勤恪匪懈。十七最之一曰禮樂興行,肅清所部,為政教之最。二曰賦役均平,田野加闢,為牧民之最。三曰決斷不滯,與奪當理,為判事之最。四曰鈐束吏卒,姦盜不滋,為嚴明之最。五曰案簿分明,評擬均當,為檢校之最。以上皆謂縣令、丞簿、警巡使副、錄事、司候、判官也。六曰詳斷合宜,咨執當理,為幕職之最。七曰盜
 賊消弭,使人安靜,為巡捕之最。八曰明於出納,物無損失,為倉庫之最。九曰訓導有方,生徒充業,為學官之最。十曰檢察有方,行旅無滯,為關津之最。十一曰隄防堅固,備禦無虞,為河防之最。十二曰出納明敏,數無濫失,為監督之最。十三曰謹察禁囚,輕重無怨,為獄官之最。十四曰物價得實,姦濫不行,為市司之最,謂市令也。十五曰戎器完肅,扞守有方,為邊防之最,謂正副部隊將、鎮防官也。十六曰議獄得情,處斷公平,為法官之最。十七曰差役均平,盜賊止息,為軍職之最,謂都軍、軍轄也。



 凡縣令以下,三最以上有四
 善或三善者為上,陞一等,三最以上有二善者為中,減兩資歷,三最以上有一善為下,減一資歷。節度判官、防禦判官、軍判以下,一最而有四善或三善為上,減一資歷,一最而有二善為中,升為榜首,一最而有一善為下,升本等首。又以明昌四年所定,軍民俱稱為廉能者是為廉能官之制,參于其間而定其甄擢焉。宣宗興定元年,行辟舉縣令法,以六事考之,一曰田野闢,二曰戶口增,三曰賦役平,四曰盜賊息,五曰軍民和,六曰詞訟簡。六事俱備為上等,升職一等。兼四事者為中等,減二資歷。其次為下等,減一資歷。否
 則為不稱職,罷而降之。平常者依本格。



 凡封王:大國號二十,曰:恒舊為遼,明昌二年以漢、遼、唐、宋、梁、秦、殷、楚之類,皆昔有天下者之號,不宜封臣下,遂皆改之、邵舊為梁、汴舊為宋、鎬舊為秦、並舊為晉、益舊為漢、彭舊為齊、趙、越、譙舊為殷、郢舊為楚、魯、冀、豫、絳舊為唐、袞、鄂舊為吳、夔舊為蜀、宛舊為陳、曹。次國三十,曰:涇舊為隋、鄭、衛、韓、潞、豳、沈、岐、代、澤、徐、滕、薛、紀、昇舊為原、邢、翼、豐、畢、鄧、鄆、霍、蔡、瀛按金格,葛當在此、沂、荊、榮、英、壽、溫。小國三十:濮、遂舊曰濟、道、定、景後改為鄒、申、崇、宿、息、莒、鄴、郜、舒、淄、郕、萊舊為宗,以避諱改、鄖、郯、杞、向、管舊曰郇,興定元年改、密、胙、任、戴、鞏、蔣《士民須知》云舊為葛、蕭、莘、芮。封王之郡號十:金源、廣平、平原、南陽、常山、太原、平陽、東
 平、安定、延安。封公主之縣號三十:樂安、清平、蓬萊、榮安、棲霞、壽光、靈仙、壽陽、鐘秀、惠和、永寧、慶雲、靜樂、福山、隆平、德平、文安、福昌、順安、樂壽、靜安、靈壽、大寧、聞喜、秀容、宜芳、真寧、嘉祥、金鄉、華原。



 凡白號之姓,完顏、溫迪罕、夾谷、陀滿、僕散、術虎、移剌荅、斡勒、斡準、把、阿不罕、卓魯、回特、黑罕、會蘭、沈谷、塞蒲里、吾古孫、石敦、卓陀、阿廝準、匹獨思、潘術古、諳石剌、石古苦、綴罕、光吉剌,皆封金源郡。斐滿、徒單、溫敦、兀林荅、阿典、紇石烈、納闌、孛術魯、阿勒根、納合、石盞、蒲鮮、古里甲、阿迭、聶摸欒、抹拈、納坦、兀撒惹、阿鮮、把古、溫古孫、耨碗、撒
 合烈、吾塞、和速嘉、能偃、阿里班、兀里坦、聶散、蒲速烈,皆封廣平郡。吾古論、兀顏、女奚烈、獨吉、黃摑、顏盞、蒲古里,必蘭、斡雷、獨鼎、尼厖窟窟亦作古、拓特、盍散、撒荅牙、阿速、撒刬、準土穀、納謀魯、業速布、安煦烈、愛申、拿可、貴益昆、溫撒、梭罕、霍域,皆封隴西郡。黑號之姓,唐括舊書作同古、蒲察、術甲、蒙古、蒲速、粘割、奧屯、斜卯、準葛、諳蠻、獨虎、術魯、磨輦、益輦、帖暖、蘇孛輦,皆封彭城郡。



 親王母妻,封一字王者舊封王妃,為正從一品。次室封王夫人。承安二年,敕王妃止封王夫人,次室封孺人。郡王母妻封郡王夫人,國公母妻封國公夫人,郡公
 母妻封郡公夫人,郡侯母妻封郡君承安二年更為郡侯夫人,四品文散少中大夫、武散懷遠大將軍以上母妻封縣君承安二年為郡君,五品文散朝列大夫、武散宣武將軍以上母妻封鄉君承安二年為縣君。



 皇統五年,以古官曰「牧」、曰「長」,各有總名,今庶官不分類為名,於文移不便。遂定京府尹牧、留守、知州、縣令、詳穩、群牧為「長官」,同知、簽院、副使、少尹、通判、丞曰「佐貳官」,判官、推官、掌書記、主簿、縣尉為「幕職官」,兵馬司及它司軍者曰「軍職官」,警巡、市令、錄事、司候、諸參軍、知律、勘事、勘判為「釐務官」,應管倉庫院務者曰「監當官」監當官出大定制,知事孔目以
 下行文書者為「吏」。凡除拜,尚書令、左右丞相以下,品不同者,則帶「守」字。左右丞則帶「行守」字。凡臺官、御史、部官、京尹、少尹、守令、丞、簿、尉、錄事、諸卿少至協律、評事、諫官、國子監學官、諸監至丞郎、符寶郎、東宮詹事、率府、僕正副、令丞、王府官,散官高於職事者帶「行」字,職事高於散官一品者帶「守」字、二品者帶「試」字,品同者皆否。猛安、謀克、翰林待制、修撰、判、推、勘事官、都事、典事、知事、內承奉、押班、通事舍人、通進、編修、勾當、頓舍、部役、廂官、受給管勾、巡河官、直省直院長副、諸檢法、知法、司正、教授、司獄、司候、東宮諭德、贊善、掌寶、典
 儀以下,王府文學、記事參軍,並帶「充」字。樞密、宣徽、勸農、諸軍都指揮、統軍、轉運使、招討、提刑、節度、群牧、防禦、客省、引進、四方館、閣門、太醫、教坊、鷹坊、警巡、巡檢、諸司局倉庫務使副,皆帶「充」字及「知某事」。凡帶「知」、「判」、「簽書」字者,則不帶「行」、「守」、「試」字。以上所帶字,品同者則否。自三師、三公、平章政事、元帥以下至監軍、東宮三師、三少、點檢至振肅、承旨、學士、王傅、副統、招討、及前所不載者,皆不帶「行」、「守」、「試」、「知」、「充」字。



 主事四員,從七品。掌知管差除、校勘行止,分掌封勛資考之事,惟選事則通署,及掌受事付事、檢勾稽失省署文牘,兼知本
 部宿直、檢校架閣。餘部主事,自受事付事以下,所掌並同此。皇統四年,六部主事始用漢士人。大定三年,用進士,非特旨不得擬吏人,如宰執保奏人材,不入常例。承安五年,增女直主事一員。令史六十九人,內女直二十九人。譯史五人,通事二人,與令史同。泰和八年,令史增十人。



 架閣庫大定二十一年六月設,仍以主事提控之。管勾,正八品。掌吏、兵兩部架閣,兼檢校吏部行止。以識女直、契丹、漢字人充,如無,擬識女直、漢字人充。同管勾一員。



 官誥院。提舉二員,掌署院事。以吏部郎中、翰林修撰各一人充。



 戶部



 尚書一員,正三品。侍郎二員,正四品。泰和八年減一員,大安二年復增。郎中二員,從五品。天德二年置五員,泰和省作二員,又作四員,貞祐四年置八員,五年作六員。員外郎三員,從六品。郎中而下,皆以一員
 掌戶籍、物力、婚姻、繼嗣、田宅、財業、鹽鐵、酒曲、香茶、礬錫、丹粉、坑冶、榷場、市易等事,一員掌度支、國用、俸祿、恩賜、錢帛、寶貨、貢賦、租稅、府庫、倉廩、積貯、權衡、度量、法式、給授職田、拘收官物、並照磨計帳等事。《泰和令》作二員,後增一員,貞祐四年作六員,又作八員,五年作四員。主事五員,從七品。女直司二員,通掌戶度金倉等事。漢人司三員,同員外郎分掌曹事。泰和八年減一員,貞祐四年作八員,五年六員。兼提控編附條格、管勾架閣等事。令史七十二人,內女直十七人。譯史五人,通事二人。泰和八年增八人。



 架閣庫。管勾一員,正八品。掌戶、禮兩部架閣。大安三年以主事各兼之。同管勾,從八品。檢法,從八品。勾當官五員,正八品。
 貞元二年,設乾辦官十員,從七品。三年,置四員,尋罷之。四年,更設為勾當官,專提控支納、管勾勘覆、經歷交鈔及香、茶、鹽引、照磨文帳等事。承安二年作四員,貞祐四年作十五員,五年作十員,興定元年五員,二年復作十員。



 禮部



 尚書一員,正三品。侍郎一員,正四品。郎中一員,從五品。員外郎一員,從六品。掌凡禮樂、祭祀、燕享、學校、貢舉、儀式、制度、符印、表疏、圖書、冊命、祥瑞、天文、漏刻、國忌、廟諱、醫卜、釋道、四方使客、諸國進貢、犒勞張設之事。凡試僧、尼、道、女冠,三年一次,限度八十人,差京府幕職或節鎮防禦佐貳官二員、僧官二人、道官一人、司吏一名、從人各一人、廚子二人、把門官一名、雜役三人。僧童能讀《法華》、《心地觀》、《金光明》、《報恩》、《華
 嚴》等經共五部,計八帙。《華嚴經》分為四帙,每帙取二卷,卷舉四題,讀百字為限。尼童試經半部,與僧童同。道士、女冠童行念《道德》、《救苦》、《玉京山》、《消災》、《靈寶度人》等經。皆以誦成句、依音釋為通。中選者試官給據,以名報有司。凡僧尼官見管人及八十、道士女冠及三十人者放度一名,死者令監壇以度牒申部毀之。主事二員,從七品。令史十五人,內女直五人。譯史二人,通事一人。



 左三部檢法司。司正二員,正八品。掌披詳法狀。興定二年,右部額外設檢,知法及掌法,四年罷。檢法二十二員,從八品。掌檢斷各司取法文字。右三部檢法職事同。元受札付,大定三年命給敕。



 兵部



 尚書一員,正三品。侍郎一員,正四品。郎中一員,從五品。員外郎二員,從六品。掌兵籍、軍器、城隍、鎮戍、廄牧、鋪驛、車輅、儀杖、郡邑圖志、險阻、障塞、遠方歸化之
 事。凡給馬者,從一品以上,從八人,馬十疋,食錢三貫十四文。從二品以上,從五人,馬七匹、食錢二貫九十八文。從三品以上,從三人,馬五匹,錢一貫五百十一文。從五品以上,從二人,馬四匹,錢九百六十八文。從七品以上,從一人,馬三匹,錢六百十七文。從九品以上,從一人,馬二匹,錢四百六十四文。無從人,減七十八文。御前差無官者,視從五品。省差若有官者,人支錢四百五十一文,有從人加六十八文。走馬人支錢百五十七文。赦書日行五百里。此《天興近鑒》所載之制也。泰和六年置遞鋪,其制,該軍馬路十里一鋪,鋪設四人,內鋪頭一人,鋪兵三人,以所轄軍射糧軍內差充,腰鈴日行三百里。凡元帥府、六部文移,以敕遞、省遞牌子,入鋪轉送。主事二員,從七品。貞祐五年以承發司管勾兼漢人主事。令史二十七人,內女直十二人。譯史三人,通事二人。



 刑部



 尚書一員,正三品。侍郎一員,正四品。郎中一員,從五品。員外郎二員,從六品。一員掌律令格式、審定刑
 名、關津譏察、赦詔勘鞫、追徵給沒等事;一員掌監戶、官戶、配隸、訴良賤、城門啟閉、官吏改正、功賞捕亡等事。主事二員,從七品。令史五十一人,內女直二十二人,譯史五人,通事二人。架閣庫。管勾一員,正八品。掌刑、工兩部架閣。大安二年以主事各兼。同管勾一員,從八品。



 工部



 尚書一員,正三品。侍郎一員,正四品。郎中一員,從五品。掌修造營建法式、諸作工匠、屯田、山林川澤之禁、江河隄岸、道路橋梁之事。員外郎一員,從六品。貞祐五年,兼覆實司官。天德三年,增二員。主事二員,從七品。令史十八人,內女直四人。譯史二人,通事一人。覆實司。管勾一員,從七品。隸戶、工部,掌覆實
 營造材物、工匠價直等事。大安元年,隸三司、工部,罷同管勾。貞祐五年併罷之,以二部主事兼。興定四年復設,從省擬,不令戶、工部舉。



 右三部檢法司。司正二員,正八品。檢法,從八品。二十二員。



 都元帥府。掌征討之事,兵罷則省。天會二年,伐宋始置。泰和八年,復改為樞密院。



 都元帥一員,從一品。左副元帥一員,正二品。右副元帥一員,正二品。元帥左監軍一員,正三品。元帥右監軍一員,正三品。左都監一員,從三品。右都監一員,從三品。經歷一員,都事一員,知事一員見興定三年,正七品。檢法一員,從八品。元帥府女直令史十二人,承安二年十六人,漢人令史六人,譯史三人,女直譯史一人,
 承安二年二人。通事,女直三人,後作六人,承安二年復作三人,漢人二人。



 正隆六年,海陵南伐,立三道都統制府及左右領軍大都督,將三十二總管,有神策、神威、神捷、神銳、神毅、神翼、神勇、神果、神略、神鋒、武勝、武定、武威、武安、武捷、武平、武成、武毅、武銳、武揚、武翼、武震、威定、威信、威勝、威捷、威烈、威毅、威震、威略、威果、威勇之號。泰和六年伐宋,權設平南撫軍上將軍,正三品。至殄寇果毅都尉,從六品。凡九階,曰平南撫軍上將軍、平南冠軍大將軍、平南龍驤將軍、平南虎威將軍、平南盪江將軍、殄寇中郎將、殄寇郎將、殄寇折衝都尉、殄寇果毅都尉,軍還罷。置
 令譯史八十人,正三十三人,餘四十七人從本府選擢。元光間,招義軍,置總領使,從五品。副使,從六品。訓練官,從八官。正大二年,更總領名都尉,升秩為四品。四年,又升為從三品。有建威、折衝、振武、盪寇、果毅、殄寇、虎賁、鷹揚、破虜之名。



 樞密院。天輔七年,始置于廣寧府。天會三年下燕山,初以左企弓為使,後以劉彥宗。初猶如遼南院之制,後則否。泰和六年嘗改為元帥府。



 樞密使一員,從一品。掌凡武備機密之事。樞密副使一員,從二品。泰和四年置二人,後不為例。簽書樞密院事一員,正三品。同簽樞密院事一員,正四品。大定十七年增一員,尋罷。明昌初,復增一員,尋又省。三年九月復增一員。經歷一員,從五品。興定三年置。都事一員,正七品。掌受事付
 事、檢勾稽失省署文牘、兼知宿直之事。架閣庫管勾一員,正八品。知法二員,從八品。掌檢斷各司取法之事。餘檢法同。樞密院令史,女直十二人,漢人六人,三品官子弟四人,吏員轉補二人。譯史三人,通事三人,回紇譯史一人,曳剌十五人。



 大宗正府。泰和六年避睿宗諱,改為大睦親府。



 判大宗正事一員,從一品。以皇族中屬親者充,掌敦睦糾率宗屬欽奉王命,泰和六年改為判大睦親事。同判大宗正事一員,從二品。泰和六年改為同判大睦親事。同簽大宗正事一員,正三品。宗室充。大定元年置。泰和六年改同簽大睦親事。大宗正丞二員,從四品。一員於宗
 室中選能幹者充,一員不限親疏,分司上京長貳、兼管治臨潢以東六司屬,泰和六年改為大睦親丞。知事一員,從七品。檢法,從八品。諸宗室將軍,正七品。上京、東溫忒二處皆有之。世宗時始命遷官,其戶凡百二十。明昌二年更名曰司屬,設令、丞。承安二年以令同隨朝司令,正七品,丞正八品。中都、上京、扎里瓜、合古西南、梅堅寨、蒲與、臨潢、泰州、金山等處置,屬大宗正府。



 ○御史臺。



 登聞檢院隸焉。見《士民須知》。《總格》、《泰和令》皆不載。



 御史大夫,從二品。舊正三品,大定十二年升。掌糾察朝儀、彈劾官邪、勘鞫官府
 公事。凡內外刑獄所屬理斷不當,有陳訴者付臺治之。御史中丞,從三品。貳大夫。侍御史二員,從五品。以上官品皆大定十二年遞升。掌奏事、判臺事。治書侍御史二員,從六品。掌同侍御史。殿中侍御史二員,正七品。每遇朝對立於龍墀之下,專劾朝者儀矩,凡百僚假告事具奏目進呈。監察御史十二員,正七品。掌糾察內外非違、刷磨諸司察帳並監祭禮及出使之事。參注諸色人,大定二年八員,承安四年十員,承安五年兩司各添十二員。典事二員,從七品。架閣庫管勾一員,從八品。檢法四員,從八品。獄丞一員,從九品。御史臺令史,女直十三人,內班內祗六人,終場舉人七人。漢人十五人,內
 班內祗七人,終場舉人八人。譯史四人,內班內祗二人,終場舉人二人。通事三人。



 宣撫司。



 泰和六年置陜西路宣撫使,節制陜西右監軍、右都監兵馬公事,八年,改陜西宣撫司為安撫司。山東東西、大名、河北東西、河東南北、遼東、陜西、咸平、隆安、上京、肇州、北京凡十處置司。



 使,從一品。副使,正三品。



 勸農使司。泰和八年罷,貞祐間復置。興定六年罷勸農司,改立司農司。



 使一員,正三品。副使一員,正五品。掌勸課天下力田之事。



 司農司。興定六年置,兼採訪公事。



 大司農一員,正二品。
 卿三員,正四品。少卿三員,正五品。知事二員,正七品。興定六年,陜西并河南三路置行司農司,設官五員。正大元年,歸德、許州、河南、陜西各置,作三員。卿一員,正四品。少卿一員,正五品。丞一員,正六品。卿以下迭出巡案,察官吏臧否而升黜之。使節所過,姦吏屏息,十年之間民政修舉,實賴其力。



 三司。泰和八年,省戶部官員置三司,謂兼勸農、鹽鐵、度支、戶部三科也。貞祐罷之。



 使一員,從二品。副使一員,正三品。簽三司事一員,正四品。同簽三司事一員,正五品。掌勸農、鹽鐵、度支。判官三員,從六品。本參幹官,
 大安元年更參議。規措審計官三員,正七品。掌同參幹官。知事二員,正七品。以識女直、漢字人充。勾當官二員,正八品。大安元年置三員,照磨吏員七人。管勾架閣庫一員,正八品。三司令史五十人,內女直十人,漢人四十人。大安元年增八人。譯史二人,大安元年增一人。通事二人。知法三員,從八品。女直知法一員,大安元年增二員。



 國史院。先嘗以諫官兼其職,明昌元年詔諫官不得兼,恐於其奏章私溢已美故也。監修國史,掌監修國史事。修國史,掌修國史,判院事。



 同修國史二員。女直人、漢人各一員。承安四年更擬女直一員,罷契丹同修國史。



 編修官,正八品。女直、漢人各四員。明昌二年罷契丹編修三員,添女直一員。大定十八年用書寫出職人。檢閱官,從九品。書寫,女直、漢人各五人。修《遼史》刊修官
 一員,編修官三員。



 翰林學士院。天德二年,命翰林學士院自侍讀學士至應奉文字,通設漢人十員,女直、契丹各七員。



 翰林學士承旨,正三品。掌制撰詞命。凡應奉文字,銜內帶「知制誥」。直學士以上同。貞祐三年升從二品。翰林學士,正三品。翰林侍讀學士,從三品。翰林侍講學士,從三品。翰林直學士,從四品。不限員。翰林待制,正五品。不限員,分掌詞命文字,分判院事,銜內不帶「知制誥」。翰林修撰,從六品。不限員,掌與待制同。應奉翰林文字,從七品。



 審官院。承安四年設,大安二年罷之,若注擬失當,上令御史臺官論列。



 知院一員,從三
 品。掌奏駁除授失當事。隨朝六品、外路五品以上官除授,並送本院審之。補闕、拾遺、監察雖七品,亦送本院。或御批亦送稟,惟部除不送。同知審官院事一員,從四品。掌書四人。女直、漢人各二人,以御史臺終場舉人闢充。



 太常寺。皇統三年正月始置。太廟、廩犧、郊社、諸陵、大樂等署隸焉。



 卿一員,從三品。少卿一員,正五品。丞一員,正六品。掌禮樂、郊廟、社稷、祠祀之事。博士二員,正七品。掌檢討典禮。檢閱官一員,從九品。掌同博士。泰和元年置,四年罷。檢討二員,從九品。明昌元年置,以品官子孫及終場舉人,同國史院漢人書寫例,試補。太祝,從八品。掌奉祀神主。奉禮郎,從八品。掌設版位,執儀行事。協律郎,從八品。掌以麾節樂,調和律呂,監視音
 調。



 太廟署。皇統八年太廟成,設署,置令丞,仍兼提舉慶元、明德、永祚三宮。令一員,從六品。掌太廟、衍慶、坤寧宮殿神御諸物,及提控諸門關鍵,掃除、守衛,兼廩犧令事。丞一員,從七品。兼廩犧署丞。直長,明昌三年罷。



 廩犧署。令、丞,以太廟令、丞兼,掌薦犧牲及養飼等事。



 郊社署。承安三年設祝史、齋郎百六十人,作班祗儤使,周年一替。大安元年,奏兼武成王廟署。令一員,從六品。丞一員,從七品。掌社稷、祠祀、祈禱並舍祭器等物。直長,明昌三年廢。



 武成王廟署。大安元年置。令,從六品。丞,從七品。掌春
 秋祀享,以郊社令、丞兼。



 諸陵署。大安四年同隨朝。提點山陵,正五品。涿州刺史兼。令,從六品。丞一員,從七品。掌守山陵。直長,正八品。



 園陵署。令,宛平縣丞兼。貞祐二年以園陵遷大興縣境,遂以大興縣令、丞兼。



 大樂署。兼鼓吹署。樂工百人。令一員,從六品。丞,從七品。掌調和律呂,教習音聲並施用之法。樂工部籍直長一員,正八品。大樂正,從九品。掌祠祀及行禮陳設樂縣。大樂副正,從九品。



 右屬太常寺。



\end{pinyinscope}