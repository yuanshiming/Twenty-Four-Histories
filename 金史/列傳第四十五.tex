\article{列傳第四十五}

\begin{pinyinscope}

 ○高汝礪張行信



 高汝礪,字巖夫,應州金城人。登大定十九年進士第,蒞官有能聲。明昌五年九月,章宗詔宰執,舉奏中外可為刺史者,上親閱闕點注,蓋取兩員同舉者升用之。於是,汝礪自同知絳陽軍節度事起為石州刺史。承安元年七月,入為左司郎中。一日奏事紫宸殿,時侍臣皆迴避,上所御涼扇忽墮案下,汝礪以非職不敢取以進。奏事
 畢,上謂宰臣曰:「高汝礪不進扇,可謂知體矣。」



 未幾,擢為左諫議大夫。以賦調軍須,郡縣有司或不得人,追胥走卒利其事急,規取貨賂,深為民害,建言:「自今若因兵調發,有犯者乞權依『推排受財法』治之,庶使小人有所畏懼。」二年六月,定制,因軍前差發受財者,一貫以下徒二年,以上徒三年,十貫處死,從汝礪之言也。時遇奏事,臺臣亦令迴避,汝礪乃上言:「國家置諫臣以備侍從,蓋欲周知時政以參得失,非徒使排行就列而已。故唐制,凡中書、門下及三品以上入閣,必遣諫官隨之,俾預聞政事,冀其有所開說。今省臺以下,遇朝奏事則一切回避,
 與諸侍衛之臣旅進旅退。殿廷論事初莫得聞,及其已行,又不詳其始末,遂事而諫,斯亦難矣。顧諫職為何如哉?若曰非材,擇人可也,豈可置之言責而疏遠若此。乞自今以往,有司奏事諫官得以預聞,庶望少補。且修注之職,掌記言動,俱當一體。」上從之。



 又言:「年前十月嘗舉行推排之法,尋以踰時而止,誠知聖上愛民之深也。切聞周制,以歲時定民之眾寡,辨物之多少,入其數于小司徒,以施政教,以行徵令,三年則天下大比,按為定法。伏自大定四年通檢前後,迄今三十餘年,其間雖兩經推排,其浮財物力,惟憑一時小民之語以為增減,有司
 惟務速定,不復推究其實。由是豪強有力者符同而幸免,貧弱寡援者抑屈而無訴。況近年以來,邊方屢有調發,貧戶益多。如止循例推排,緣去歲條理已行,人所通知,恐新強之家預為請囑狡獪之人,冀望至時同辭推唱。或虛作貧之,故以產業低價質典,及將財物徙置他所,權止營運。如此姦弊百端,欲望物力均一,難矣。欲革斯弊,莫若據實通檢,預令有司照勘大定四年條理,嚴立罪賞,截日立限,關防禁約。其間有可以輕重者斟酌行之,去煩碎而就簡易,戒搔擾而事鎮靜,使富者不得以茍避,困者有望於少息,則賦稅易辦,人免不均之患
 矣。」詔尚書省俟邊事息行之。



 是歲十月,上諭尚書省,遣官詣各路通檢民力,命戶部尚書賈執剛與汝礪先推排在都兩警巡院,令諸路所差官視以為法焉。尋為同知大興府事。四年十二月,為陜西東路轉運使。泰和元年七月,改西京路轉運使。二年正月,為北京臨潢府路按察使。四年二月,遷河北西路轉運使。十一月,進中都路都轉運使。



 六年六月,拜戶部尚書。時鈔法不能流轉,汝礪隨事上言,多所更定,民甚便之,語在《食貨志》。上嘉其議,敕尚書省曰:「內外百官所司不同,比應詔言事者不啻千數,俱不達各司利害,汗漫陳說,莫能詳盡。近惟
 戶部尚書高汝礪,論本部數事,並切事情,皆已行之。其諭內外百司各究利害舉明,若可舉而不即申聞,以致上司舉行者,量制其罰。」



 貞祐二年六月,宣宗南遷,次邯鄲,拜汝礪為參知政事。次湯陰,上聞汴京穀價騰踴,慮扈從人至則愈貴,問宰臣何以處之。皆請命留守司約束,汝礪獨曰:「物價低昂,朝夕或異,然糴多糶少則貴。蓋諸路之人輻湊河南,糴者既多,安得不貴?若禁止之,有物之家皆將閉而不出,商旅轉販亦不復入城,則糴者益急而貴益甚矣。事有難易,不可不知,今少而難得者穀也,多而易致者鈔也,自當先其所難,後其所易,多方
 開誘,務使出粟更鈔,則穀價自平矣。」上從之。



 三年五月,朝廷議徙河北軍戶家屬於河南,留其軍守衛郡縣,汝礪言:「此事果行,但便於豪強家耳,貧戶豈能徙?且安土重遷,人之情也。今使盡赴河南,彼一旦去其田園,扶攜老幼,驅馳道路,流離失所,豈不可憐。且所過百姓見軍戶盡遷,必將驚疑,謂國家分別彼此,其心安得不搖。況軍人已去其家,而令護衛他人,以情度之,其不肯盡心必矣。民至愚而神者也,雖告以衛護之意,亦將不信,徒令交亂,俱不得安,此其利害所繫至重。乞先令諸道元帥府、宣撫司、總管府熟論可否,如無可疑,然後施行。」不
 報。



 軍戶既遷,將括地分授之,未有定論,上敕尚書省曰:「北兵將及河南,由是盡起諸路軍戶,共圖保守。今既至矣,糧食所當必與,然未有以處之。可分遣官聚耆老問之,其將益賦,或與之田,二者孰便。」又以諭汝礪。既而所遣官言:「農民並稱,比年以來,租賦已重,若更益之,力實不足,不敢復佃官田,願以給軍。」於是汝礪奏:「遷徙軍戶,一時之事也。民佃官田,久遠之計也。河南民地、官田,計數相半。又多全佃官田之家,墳塋、莊井俱在其中。率皆貧民,一旦奪之,何以自活?夫小民易動難安,一時避賦,遂有此言。及其與人,即前日之主,今還為客,能勿悔乎?
 悔則忿心生矣。如山東撥地時,腴田沃壤盡入勢家,瘠惡者乃付貧戶。無益於軍,而民則有損,至於互相憎疾,今猶未已,前事不遠,足為明戒。惟當倍益官租,以給軍糧之半,復以係官荒田、牧馬草地量數付之,令其自耕,則百姓免失業之艱,而官司不必為厲民之事矣。且河南之田最宜麥,今雨澤霑足,正播種之時,誠恐民疑以誤歲計,宜早決之。」上從其請。



 尋遷尚書右丞。時上以軍戶地當撥付,使得及時耕墾,而汝礪復上奏曰:「在官荒田及牧馬地,民多私耕者。今正藝麥之時,彼知將以與人,必皆棄去。軍戶雖得,亦已逾時,徒成曠廢。若候畢功
 而後撥,量收所得,以補軍儲,則公私俱便。乞盡九月然後遣官。」十月,汝礪言:「今河北軍戶徙河南者幾百萬口,人日給米一升,歲率三百六十萬石,半給其直猶支粟三百萬石。河南租地計二十四萬頃,歲征粟纔一百五十六萬有奇,更乞於經費之外倍徵以給,仍以系官閑田及牧馬地可耕者畀之。」奏可。乃遣右司諫馮開等分詣諸郡就給之,人三十畝,以汝礪總之。既而括地官還,皆曰:「頃畝之數甚少,且瘠惡不可耕。計其可耕者均以與之,人得無幾,又僻遠處不免徙就之,軍人皆以為不便。」汝礪遂言於上,詔有司罷之,但給軍糧之半,而半折
 以實直焉。



 四年正月,拜尚書左丞,連上表乞致仕,皆優詔不許。會朝廷議發兵河北,護民芟麥,而民間流言謂官將盡取之。上聞,以問宰職曰:「為之奈何?」高琪等奏:「若令樞密院遣兵居其衝要,鎮遏土寇,仍許收逃戶之田,則軍民兩便。或有警急,軍士亦必盡心。」汝礪曰:「甚非計也。蓋河朔之民所恃以食者,惟此麥耳。今已有流言,而復以兵往,是益使之疑懼也。不若聽其自便,令宣撫司禁戢無賴,不致侵擾足矣。逃戶田令有司收之,以充軍儲可也。」乃詔遣戶部員外郎裴滿蒲剌都閱視田數,及訪民願發兵以否,還奏:「臣西由懷、孟,東抵曹、單,麥苗苦
 亦無多,訊諸農民,往往自為義軍。臣即宣布朝廷欲發兵之意,皆感戴而不願也。」於是罷之。



 汝礪以數乞致仕不從,乃上言曰:「立非常之功,必待非常之人。今大兵既退,正完葺關隘、簡練兵士之時,須得通敏經綸之才預為籌畫,俾濟中興。伏見尚書左丞兼行樞密副使胥鼎,才擅眾長,身兼數器,乞召還朝省。」不從。時高琪欲從言事者歲閱民田征租,朝廷將從之。汝礪言:「臣聞治大國者若烹小鮮,最為政之善喻也。國朝自大定通檢後,十年一推物力,惟其貴簡靜而重勞民耳。今言者請如河北歲括實種之田,計數征斂,即是常時通檢,無乃駭人
 視聽,使之不安乎。且河南、河北事體不同。河北累經劫掠,戶口亡匿,田疇荒廢,差調難依元額,故為此權宜之法,蓋軍儲不加多,且地少而易見也。河南自車駕巡幸以來,百姓湊集,凡有閑田及逃戶所棄,耕墾殆遍,各承元戶輸租,其所徵斂,皆準通推之額,雖軍馬益多,未嘗闕誤,詎宜一概動擾。若恐豪右蔽匿而逋征賦,則有司檢括亦豈盡實。但嚴立賞罰,許其自首,及聽人告捕,犯者以盜軍儲坐之,地付告者,自足使人知懼,而賦悉入官,何必為是紛紛也。抑又有大不可者三:如每歲檢括,則夏田春量,秋田夏量,中間雜種亦且隨時量之,一歲
 中略無休息,民將厭避,耕種失時,或止耕膏腴而棄其餘,則所收仍舊而所輸益少,一不可也。檢括之時,縣官不能家至戶到,里胥得以暗通貨賂,上下其手,虛為文具,轉失其真,二不可也。民田與軍田犬牙相錯,彼或陰結軍人以相冒亂,而朝廷止憑有司之籍,倘使臨時少於元額,則資儲闕誤必矣,三不可也。夫朝廷舉事,務在必行,既行而復中止焉,是豈善計哉。」議遂寢。



 興定元年十月,上疏曰:「言者請姑與宋人議和以息邊民,切以為非計。宋人多詐無實,雖與文移往來,而邊備未敢遽撤。備既不撤,則議和與否蓋無以異。或復蔓以浮辭,禮例
 之外別有求索,言涉不遜,將若之何?或曰:『大定間亦嘗先遣使,今何不可?』切謂時殊事異,難以例言。昔海陵師出無名,曲在於我,是以世宗即位,首遣高忠建等報諭宋主,罷淮甸所侵以修舊好。彼隨遣使來,書辭慢易,不復奉表稱臣,願還故疆,為兄弟國。雖其樞密院與我帥府時通書問,而侵軼未嘗已也。既而征西元帥合喜敗宋將吳璘、姚良輔於德順、原州,右丞相僕散忠義、右副元帥紇石烈志寧敗李世輔于宿州,斬首五萬,兵威大振。世宗謂宰臣曰:『昔宋人,言遣使請和,乘吾無備遂攻宿州,今為我軍大敗,殺戮過當,故不敢復通問。朕哀南
 北生靈久困于兵,本欲息民,何較細故,其令帥府移書宋人,以議和好。』宋果遣使告和,以當時堂堂之勢,又無邊患,竟免其奉表稱臣之禮。今宋棄信背盟,侵我邊鄙,是曲在彼也。彼若請和,於理為順,豈當先發此議而自示弱耶?恐非徒無益,反招謗侮而已。」



 十一月,汝礪言:「臣聞國以民為基,民以財為本,是以王者必先愛養基本。國家調發,河南為重,所征稅租率常三倍于舊。今省部計歲收通寶不敷所支,乃于民間科斂桑皮故紙錢七千萬貫以補之。近以通寶稍滯,又加兩倍。河南人戶,農民居三之二,今稅租猶多未足,而此令復出,彼不糶所
 當輸租,則必減其食以應之。夫事有難易,勢有緩急。今急用而難得者,芻糧也,出於民力,其來有限,可緩圖。而易為者,鈔法也,行于國家,其變無窮。向者大鈔滯,更為小鈔,小鈔弊,改為寶券,寶券不行,易為通寶,從權制變,皆由於上,尚何以煩民為哉。彼悉力以奉軍儲已患不足,而又添徵通寶,茍不能給,則有逃亡。民逃亡則農事廢,兵食何自而得?有司不究遠圖而貪近效,不固本原而較末節,誠恐軍儲、鈔法兩有所妨。臣非於鈔法不為意也,非於省部故相違也,但以鈔法稍滯物價稍增之害輕,民生不安軍儲不給之害重耳。惟陛下外度事勢,
 俯察臣言,特命有司減免,則群心和悅,而未足之租有所望矣。」



 時朝廷以賈仝、苗道潤等相攻不和,將分畀州縣、別署名號以處之。汝礪上書曰:「甚非計也。蓋河北諸帥多本土義軍,一時權為隊長,亦有先嘗叛亡者,非若素宦於朝,知禮義、識名分之人也。貪暴不法,蓋無足怪。朝廷以時方多故,姑牢籠用之,庶使遣民少得安息。彼互相攻劫則勢浸弱,勢力既弱則朝廷易制。今若分地而與之,州縣官吏得輒署置,民戶稅賦得擅徵收,則地廣者日益強,狹者日益弱。久之,弱者皆併於強,強者之地不可復奪,是朝廷愈難制也。昔唐分河朔地授諸叛
 將,史臣謂其護養孽萌以成其禍,此可為今日大戒也。不若姑令行省羈縻和輯,多方牽制,使之不得逞。異時邊事稍息,氣力漸完,若輩又何足患哉。」議遂寢。



 上嘗謂汝礪曰:「朕每見卿侍朝,恐不任其勞,許坐殿下,而卿終不從何哉?夫君臣相遇,貴在誠實,小謹區區,朕固不較也。」汝礪以君臣之分甚嚴,不敢奉命。



 三年,河南頗豐稔,民間多積粟,汝礪乃奏曰:「國家之務,莫重於食,今所在屯兵益眾,而修築新城其費亦廣,若不及此豐年多方營辦,防秋之際或乏軍興。乞於河南州府驗其物價低昂,權宜立式,凡內外四品以下雜正班散官及承廕人,
 免當暴使監官功酬,或僧道官師德號度牒、寺觀院額等,並聽買之。司縣官有能勸誘輸粟至三千石者,將來注授升本榜首,五千石以上遷官一階,萬石以上升職一等,並注見闕。庶幾人知勸慕,多所收獲。」上從之。



 同提舉榷貨司王三錫建議榷油,高琪以用度方急,勸上行之。汝礪上言曰:「古無榷法,自漢以來始置鹽鐵酒榷均輸官,以佐經費。末流至有算舟車、稅間架,其征利之術固已盡矣,然亦未聞榷油也。蓋油者世所共用,利歸於公則害及於民,故古今皆置不論,亦厭苛細而重煩擾也。國家自軍興,河南一路歲入稅租不啻加倍,又有額
 徵諸錢、橫泛雜役,無非出於民者,而更議榷油,歲收銀數十萬兩。夫國以民為本,當此之際,民可以重困乎!若從三錫議,是以舉世通行之貨為榷貨,私家常用之物為禁物,自古不行之法為良法,切為聖朝不取也。若果行之,其害有五,臣請言之:河南州縣當立務九百餘所,設官千八百餘員,而胥隸工作之徒不與焉。費既不貲,而又創構屋宇,奪買作具,公私俱擾,殆不勝言。至於提點官司有升降決罰之法,其課一虧,必生抑配之弊,小民受病,益不能堪,其害一也。夫油之貴賤所在不齊,惟其商旅轉販有無相易,所以其價常平,人易得之。今既
 設官各有分地,輒相侵犯者有罪,是使貴處常貴而賤處常賤,其害二也。民家日用不能躬自沽之,而轉鬻者增取利息,則價不得不貴,而用不得不難,其害三也。鹽、鐵、酒、醋,公私所造不同,易於分別,惟油不然,莫可辨記。今私造者有刑,捕告者有賞,則無賴輩因之得以誣構良民枉陷於罪,其害四也。油戶所置屋宇、作具,用錢已多,有司按業推定物力,以給差賦。今奪其具、廢其業而差賦如前,何以自活,其害五也。惟罷之便。」上是之,然重違高琪意,乃詔集百官議于尚書省。戶部尚書高夔、工部侍郎粘割荊山、知開封府事溫迪罕二十等二十六
 人議同高琪,禮部尚書楊雲翼、翰林侍讀學士趙秉文、南京路轉運使趙瑄、吏部侍郎趙伯成、刑部郎中姬世英、右司諫郭著、提舉倉場使時戩皆以為不可。上曰:「古所不行者而今行之,是又生一事也,其罷之。」



 十月,賜金鼎一,重幣三。四年三月,拜平章政事,俄而進拜尚書右丞相,監修國史,封壽國公。五年二月,上表乞致政,不許。九月,上諭汝礪曰:「昨日視朝,至午方罷。卿老矣,不任久立,奏事畢,用寶之際,可先退坐,恐以勞致疾,反妨議政也。」是月,復乞致仕,上諭之曰:「丞相之禮盡矣,然今廷臣誰如丞相者,而必欲求去乎,姑留輔朕可也。」十月,躐遷
 榮祿大夫,仍諭曰:「丞相數求去,朕以社稷事重,故堅留之。丞相老矣,而官猶未至二品,故特升兩階。」十二月,上復諭曰:「向朕以卿年老,視朝之日侍立為勞,令用寶時退坐廊下,而卿違之,復侍立終朝,豈有司不為設榻耶?卿其勉從朕意。」元光元年四月,汝礪跪奏事,上命起曰:「卿大臣也,所言皆社稷計。朕之責卿,惟在盡誠,何事小謹,自今勿復爾也。」



 七月,上謂宰臣曰:「昔有言世宗太儉者,或曰不爾則安得廣畜積。章宗時用度甚多,而得不闕乏者,蓋先朝有以遺之也。」汝礪因進言曰:「儉乃帝王大德,陛下言及此,天下福也。」九月,上又謂宰臣曰:「有功
 者雖有微過亦當貸之,無功者豈可貸耶?然有功者人喜謗議。凡有以功過言於朕者,朕必深求其實,雖近侍為言不敢輕信,亦未嘗徇一己之愛憎也。」汝礪因對曰:「公生明,偏生暗。凡人多徇愛憎,不合公議。陛下聖明,故能如是耳。」



 二年正月,復乞致政,上面諭曰:「今若從卿,始終之道俱盡,於卿甚安,在朕亦為美事。但時方多故,而朕復不德,正賴舊人輔佐,故未能遂卿高志耳。」汝礪固辭,竟不許,因謂曰:「朕每聞人有所毀譽,必求其實。」汝礪對曰:「昔齊威王封即墨大夫,烹阿大夫及左右之嘗毀譽者,由是群臣恐懼,莫敢飾非,齊國大治。陛下言及此,
 治安可期也。」二月,上以汝礪年高,免朝拜,侍立久則憩于殿下,仍敕有司設榻焉。三月,又乞致仕,復優詔不許。上謂群臣曰:「人有才堪任事,而處心不正者,終不足貴。」汝礪對曰:「其心不正而濟之以才,所謂虎而翼者也,雖古聖人亦未易知。」上以為然。他日復謂宰臣曰:「凡人處心善良而行事忠實,斯為難得。若言巧心偽,亦復何用。然善良者,人又多目為平常。」汝礪對曰:「人材少全,亦隨其所長取之耳。」上然之。五月,上問宰執以修完京城樓櫓事,汝礪奏:「所用皆大木,顧今難得,方令計置。」上曰:「朕宮中別殿有可用者即用之。」汝礪對以不宜毀,上曰:「所
 居之外,毀亦何害,不愈於勞民遠致乎!」



 哀宗初即位,諫官言汝礪欺君固位,天下所共嫉,宜黜之以厲百官。哀宗曰:「昔惠帝言,我不如高帝,當守先帝法耳。汝礪乃先帝立以為相者,又可黜歟!」又有投匿名書云:「高某不退當殺之。」汝礪因是告老,優詔不許。正大元年三月,薨,年七十一,配享宣宗廟。



 為人慎密廉潔,能結人主知,然規守格法,循嘿避事,故為相十餘年未嘗有譴訶。貪戀不去,當時士論頗以為譏云。



 張行信,字信甫,先名行忠,避莊獻太子諱改焉。行簡弟也。登大定二十八年進士第,累官銅山令。明昌元年,以
 廉擢授監察御史。泰和三年,同知山東西路轉運使,俄簽河東路按察司事。四年四月,召見于泰和殿,行信因言二事,一依舊移轉吏目以除民害,一徐、邳地下宜麥,稅粟許納麥以便民。上是其言,令尚書省議行之。崇慶二年,為左諫議大夫。時胡沙虎已除名為民,賂遺權貴,將復進用。舉朝無敢言者,行信乃上章曰:「胡沙虎殘忍凶悖,跋扈強梁,媚結近習,以圖稱譽。自其廢黜,士庶莫不忻悅。今若復用,惟恐為害更甚前日,況利害之機更有大於此者。」書再上,不報。及胡沙虎弒逆,人甚危之,行信坦然不顧也。



 是歲九月,宣宗即位,改元貞祐。行信以
 皇嗣未立,無以係天下之望,上疏曰:「自古人君即位,必立太子以為儲副,必下詔以告中外。竊見皇長子每遇趨朝,用東宮儀衛,及至丹墀,還列諸王班。況已除侍臣,而今未定其禮,可謂名不正言不順矣。昔漢文帝元年,首立子啟為太子者,所以尊祖廟、重社稷也。願與大臣詳議,酌前代故事,早下明詔,以定其位,慎選宮僚,輔成德器,則天下幸甚。」上嘉納之。



 胡沙虎誅,上封事言正刑賞,辭載《胡沙虎傳》。又言:「自兵興以來,將帥甚難其人,願陛下令重臣各舉所知,才果可用,即賜召見,褒顯獎諭,令其自效,必有奮命報國者。昔李牧為趙將,軍功爵賞
 皆得自專,出攻入守不從中覆,遂能北破大敵,西抑強秦。今命將若不以文法拘繩、中旨牽制,委任責成,使得盡其智能,則克復之功可望矣。」上善其言。時方擢任王守信、賈耐兒者為將,皆鄙俗不材、不曉兵律。行信懼其誤國,上疏曰:「《易》稱『開國承家,小人勿用』。聖人所以垂戒後世者,其嚴如此。今大兵縱橫,人情洶懼,應敵興理,非賢智莫能。狂子庸流,猥蒙拔擢,參預機務,甚無謂也。」於是上皆罷之。權元帥右都監內族訛可率兵五千護糧通州,遇兵輒潰,行信上章曰:「御兵之道,無過賞罰,使其臨敵有所慕而樂於進,有所畏而不敢退,然後將士用
 命而功可成。若訛可敗衄,宜明正其罪,朝廷寬容,一切不問,臣恐御兵之道未盡也。」詔報曰:「卿意具悉,訛可等已下獄矣。」



 時中都受兵,方遣使請和,握兵者畏縮不敢戰,曰:「恐壞和事。」行信上言:「和與戰二事本不相干,奉使者自專議和,將兵者惟當主戰,豈得以和事為辭。自崇慶來,皆以和誤,若我軍時肯進戰,稍挫其鋒,則和事成也久矣。頃北使既來,然猶破東京,略河東。今我使方行,將帥輒按兵不動,於和議卒無益也。事勢益急,芻糧益艱,和之成否蓋未可知,豈當閉門坐守以待弊哉。宜及士馬尚壯,擇猛將銳兵,防衛轉輸,往來拒戰,使之少
 沮,則附近蓄積皆可入京師,和議亦不日可成矣。」上心知其善而不能行。



 二年三月,以朝廷括糧恐失民心,上書言:「近日朝廷令知大興府胥鼎便宜計畫軍食,鼎因奏許人納粟買官。既又遣參知政事奧屯忠孝括官民糧,戶存兩月,餘悉令輸官,酬以爵級銀鈔。時有粟者或先具數於鼎,未及入官。忠孝復欲多得以明己功,凡鼎所籍者不除其數,民甚苦之。今米價踴貴,無所從糴,民糧止兩月又奪之,將不獨歸咎有司,亦怨朝廷不察也。大兵在邇,人方危懼,若復無聊,或生他變,則所得不償所損矣。」上深善其言,即命與近臣往審處焉。仍諭忠孝
 曰:「極知卿盡心于公,然國家本欲得糧,今既得矣,姑從人便可也。」四月,遷山東東路按察使,兼轉運使,仍權本路宣撫副使。將行,求入見,上御便殿見之。奏曰:「臣伏見奧屯忠孝飾詐不忠,臨事慘刻,與胡沙虎為黨。」歷數其罪,且曰:「無事時猶不容一相非才,況今多故,可使斯人與政乎?願即罷之。」上曰:「朕始即位,進退大臣自當以禮,卿語其親知,諷令求去可也。」行信以告右司郎中把胡魯白忠孝,忠孝不恤也。



 三年二月,改安武軍節度使,兼冀州管內觀察使。始至,即上書言四事,其一曰:「楊安兒賊黨旦暮成擒,蓋不足慮。今日之急,惟在收人心而已。
 向者官軍討賦,不分善惡,一概誅夷,劫其資產,掠其婦女,重使居民疑畏,逃聚山林。今宜明敕有司,嚴為約束,毋令劫掠平民。如此則百姓無不安之心,姦人誑脅之計不行,其勢漸消矣。」其二曰:「自兵亂之後,郡縣官豪,多能糾集義徒,摧擊土寇,朝廷雖授以本處職任,未幾遣人代之。夫舊者人所素服,新者未必皆才,緩急之間,啟釁敗事。自今郡縣闕員,乞令尚書省選人擬注。其舊官,民便安者宜就加任使,如資級未及,令攝其職,待有功則正授。庶幾人盡其才,事易以立。」其三曰:「掌軍官敢進戰者十無一二,其或有之,即當責以立功,不宜更授他
 職。」其四曰:「山東軍儲皆鬻爵所獲,及或持敕牒求仕,選曹以等級有不當鬻者往往駮退。夫鬻所不當,有司罪也,彼何責焉。況海岱重地,群寇未平,田野無所收,倉廩無所積,一旦軍餉不給,復欲鬻爵,其誰信之?」朝廷多用其議。八月,召為吏部尚書。九月,改戶部尚書。十二月,轉禮部尚書,兼同修國史。



 四年二月,為太子少保,兼前職。時尚書省奏:「遼東宣撫副使完顏海奴言,參議官王澮嘗言,本朝紹高辛,黃帝之後也。昔漢祖陶唐,唐祖老子,皆為立廟。我朝迄今百年,不為黃帝立廟,無乃愧於漢、唐乎!」又云:「本朝初興,旗幟尚赤,其為火德明矣。主德之
 祀,闕而不講,亦非禮經重祭祀之意。臣聞於澮者如此,乞朝廷議其事。」詔問有司,行信奏曰:「按《始祖實錄》止稱自高麗而來,未聞出於高辛。今所據欲立黃帝廟,黃帝高辛之祖,借曰紹之,當為木德,今乃言火德,亦何謂也?況國初太祖有訓,因完顏部多尚白,又取金之不變,乃以大金為國號,未嘗議及德運。近章宗朝始集百僚議之,而以繼亡宋火行之絕,定為土德,以告宗廟而詔天下焉。顧澮所言特狂妄者耳。」上是之。



 八月,上將祔享太廟,詔依世宗十六拜之禮。行信與禮官參定儀注,上言宜從四十四拜之禮,上嘉納焉,語在《禮志》。祭畢,賜行信
 寶券二萬貫、重幣下端,諭之曰:「太廟拜禮,朕初欲依世宗所行,卿進奏章,備述隨室讀祝,殊為中理。向非卿言,朕幾失之,故特以是旌賞,自今每事更宜盡心。」是年十二月,行信以父暐卒,去官。



 興定元年三月,起復舊職,權參知政事。六月,真拜參知政事。時高琪為相,專權用事,惡不附己者,衣冠之士,動遭窘辱,惟行信屢引舊制力抵其非。會宋兵侵境,朝廷議遣使詳問,高琪等以為失體,行信獨上疏曰:「今以遣使為不當,臣切惑之。議者不過曰:『遣使則為先示弱,其或不報,報而不遜,則愈失國體。』臣獨以為不然。彼幸吾釁隙,數肆侵掠,邊臣以兵卻
 之復來,我大國不責以辭而敵以兵,茲非示弱乎。至於問而不報,報而不遜,曲自在彼,何損於我。昔大定之初,彼嘗犯順,世宗雖遣丞相烏者行省于汴,實令元帥撒合輦先為辭詰之,彼遂伏罪。其後宋主奪取國書,朝廷復欲加兵,丞相婁室獨以為不可,及刑部尚書梁肅銜命以往,尋亦屈焉。在章宗時,猖狂最甚,猶先理問而後用兵。然則遣使詳問正國家故事,何失體之有。且國步多艱,戍兵滋久,不思所以休息之,如民力何。臣書生無甚高論,然事當機會,不敢不罄其愚,惟陛下察之。」上復令尚書省議,高琪等奏:「行信所言固遵舊制,然今日之
 事與昔不同。」詔姑待之。已而高汝礪亦上言先遣使不便,議遂寢,語在汝礪傳。



 時監察御史多被的決,行信乃上言曰:「大定間,監察坐罪大抵收贖,或至奪俸,重則外降而已,間有的決者,皆有為而然。當時執政程輝已嘗面論其非是,又有敕旨,監察職主彈劾,而或看循者,非謂凡失察皆然也。近日無問事之大小、情之輕重,一概的決,以為大定故實、先朝明訓,過矣。」於是詔尚書省更定監察罪名制。



 史館修《章宗實錄》,尚書省奏:「舊制,凡修史,宰相執政皆預焉。然女直、漢人各一員。崇慶中,既以參知政事梁絪兼之,復命翰林承旨張行簡同事,蓋行
 簡家學相傳,多所考據。今修《章宗實錄》,左丞汝礪已充兼修,宜令參知政事行信同修如行簡例。」制可。



 二年二月,出為彰化軍節度使,兼涇州管內觀察使,諭之曰:「初,朕以朝臣多稱卿才,乃令參決機務。而廷議之際,每不據正,妄為異同,甚非為相之道。復聞邇來殊不以乾當為意,豈欲求散地故耶?今授此職,卿宜悉之。」初,內族合周避敵不擊,且詭言密奉朝旨,下獄當誅。諸皇族多抗表乞從末減,高琪以為自古犯法無告免者,行信獨曰:「事無古今,但合周平昔忠孝,或可以免。」又以行信族弟行貞居山東,受紅襖賊偽命,樞密院得宋人書,有干涉
 行信事,故出之。其子莒,時為尚書省令史,亦命別加注授焉。



 初,行信言:「今法,職官論罪,多從的決。伏見大定間世宗敕旨,職官犯故違聖旨,徒年、杖數並的決。然其後三十餘年,有司論罪,未嘗引用,蓋非經久為例之事也。乞詳定之。」行信既出,上以其章付尚書省。至是,宰臣奏:「自今違奏條之所指揮、及諸條格,當坐違制旨者,其徒年、杖數論贖可也。特奉詔旨違者,依大定例。」制可。行信去未久,上嘗諭宰臣曰:「自張行信降黜,卿等遂緘默,此殊非是。行信事,卿等具知,豈以言之故耶!自今宜各盡言,毋復畏忌。」



 行信始至涇,即上書曰:「馬者甲兵之本,方
 軍旅未息,馬政不可緩也。臣自到涇,聞陜右豪民多市於河州,轉入內地,利蓋百倍。及見省差買馬官平涼府判官烏古論桓端市於洮州,以銀百鋌幾得馬千疋,云生羌木波諸部蕃族人戶畜牧甚廣。蓋前所遣官或抑其直,或以勢陵奪,遂失其和,且常患銀少,所以不能多得也。又聞蕃地今秋薄收,鬻馬得銀輒以易粟。冬春之交必艱食,馬價甚低。乞令所司輦銀粟于洮、河等州,選委知蕃情、達時變如桓端者貿易之。若捐銀萬兩,可得良馬千疋,機會不可失,惟朝廷亟圖之。」



 又曰:「此者沿邊戰士有功,朝廷遺使宣諭,賜以官賞,莫不感戴聖恩,願
 出死力,此誠得激勸之方也。然贈遺使者或馬或金,習以為常,臣所未諭也。大定間,嘗立送宣禮,自五品以上各有定數,後竟停罷。況今時務與昔不同,而六品以下及止遷散官者,亦不免饋獻,或莫能辦,則斂所部以應之,至有因而獲罪者。彼軍士效死立功,僅蒙恩賞,而反以饋獻為苦,是豈朝廷之意哉。乞令有司依大定例,參以時務,明立等夷,使取予有限,無傷大體,則上下兩得矣。」



 又曰:「近聞保舉縣令,特增其俸,此朝廷為民之善意也。然自關以西,尚未有到任者,遠方之民不能無望。豈舉者猶寡,而有所不敷耶?乞詔內外職事官,益廣選舉,
 以補其闕,使天下均受其賜。且丞、簿、尉亦皆親民,而獨不增俸,彼既不足以自給,安能禁其侵牟乎。或謂國用方闕,不宜虛費,是大不然。夫重吏祿者,固使之不擾民也,民安則國定,豈為虛費。誠能裁減冗食,不養無用之人,亦何患乎不足。今一軍充役,舉家廩給,軍既物故,給其子弟,感悅士心,為國盡力耳。至於無男丁而其妻女猶給之,此何謂耶?自大駕南巡,存贍者已數年,張頤待哺,以困農民。國家糧儲,常患不及,顧乃久養此老幼數千萬口,冗食虛費,正在是耳。如即罷之,恐其失所,宜限以歲月,使自為計,至期而罷,復將何辭。」上多採納焉。



 元
 光元年正月,遷保大軍節度使,兼鄜州管內觀察使。二月,改靜難軍節度使,兼邠州管內觀察使。未幾,致仕。哀宗即位,徵用舊人,起為尚書左丞。言事稍不及前,人望頗減。尋復致仕家居,惟以抄書教子孫為事,葺園池汴城東,築亭號「靜隱」,時時與侯摯輩游詠其間。正大八年二月乙丑,薨于嵩山崇福宮,年六十有九。初遊嵩山,嘗曰:「吾意欲主此山。」果終于此。



 為人純正真率,不事修飾,雖兩登相位,殆若無官然。遇事輒發,無所畏避,每奏事上前,旁人為動色,行信處之坦如也。及薨之日,雖平昔甚媢忌者,亦曰正人亡矣。初至汴,父暐以御史大夫致
 仕,猶康健,兄行簡為翰林學士承旨,行信為禮部尚書,諸子侄多中第居官,當世未之有也。



 贊曰:高汝礪禔身清慎,練達事宜,久居相位,雖為大夫士所鄙,而人主寵遇不衰。張行信礪志謇諤,言無避忌,然一簉政途,便多坎壈,及其再用,論事稍不及前,豈以汝礪為真可法耶。宣宗伐宋,本非萬全之策,行信諫,汝礪不諫,又沮和議。胡沙虎之惡未著,行信兩疏擊之。汝礪與高琪共事,人疑其黨附。優劣可概見於斯矣。



\end{pinyinscope}