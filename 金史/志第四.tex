\article{志第四}

\begin{pinyinscope}

 五行



 五行之精氣,在天為五緯,在地為五材,在人為五常及五事。五緯志諸《天文》,歷代皆然。其形質在地,性情在人,休咎各以其類,為感應於兩間者,歷代又有《五行志》焉。兩漢以來,儒者若夏侯勝之徒,專以《洪範五行》為學,作史者多采其說,凡言某征之休咎,則以某事之得失繫之,而配之以五行。謂其盡然,其弊不免於傅會;謂其不
 然,「肅,時雨若」、「蒙,恒風若」之類,箕子蓋嘗言之。金世未能一天下,天文災祥猶有星野之說,五行休咎見於國內者不得他諉,乃匯其史氏所書,仍前史法,作《五行志》。至於五常五事之感應,則不必泥漢儒為例云。



 初,金之興,平定諸部,屢有禎異,故世祖每與敵戰,嘗以夢寐卜其勝負。烏春兵至蘇束海甸,世祖曰:「予夙昔有異夢,不可親戰,若左軍有力戰者當克。」既而與肅宗等擊之,敵大敗。



 太祖之生也,常有五色雲氣若二千斛廩囷之狀,屢見東方。遼司天到孔致和曰:「其下當生異人,建非常之事,天以象告,非人力所能為也。」



 溫都部跋忒畔,
 穆宗遣太祖討之,入辭,奏曰:「昨夕見赤祥,往必克。」遂與跋忒戰,殺之。



 穆宗攻阿疏日,辰巳間,忽暴雨昏曀,雷電環阿疏所居,是夕有巨火聲如雷,墜阿疏城中,遂攻下之。



 太祖嘗往寧江,夢斡帶之禾場焚,頃刻而盡。覺而大戚,即馳還,斡帶已寢疾,翌日不起。



 斡塞伐高麗,太祖臥而得夢,及起曰:「今日捷音必至。」乃為具於球場以待。有二麞渡水而至,獲之,太祖曰:「此休徵也。」言未既,捷書至,眾大異之。



 他日軍寧江,駐高阜,撒改仰見太祖體如喬松,所乘馬如岡阜之大,太祖亦視撒改人馬異常,撒改因白所見,太祖喜曰:「此吉兆也。」即舉酒酹之曰:「異日成
 功,當識此地。」師次唐括帶斡甲之地,諸軍介而立,有光起於人足及戈矛上,明日,至札只水,光復如初。



 收國元年,上在寧江州,有光正圓,自空而墜。八月己卯,黃龍見空中。十二月丁未,上候遼軍還至熟結濼,有光復見於矛端。



 天輔六年三月,師攻西京,有火如斗,墜其城中。是月,城降而復叛,四月辛卯,取之。



 太宗天會二年,曷懶移鹿古水霖雨害稼,且為蝗所食。秋,泰州潦,害稼。三年七月,錦州野蠶成繭。九月,廣寧府進嘉禾。四年十月,中京進嘉禾。六年冬,移懶路饑。九年七月丙申,上御西樓聽政,聞咸州所貢白鵲
 音忽異常,上起視之,見東樓上光明中有像巍然高五丈許,下有紅雲承之,若世所謂佛者,乃擎跽修虔,久之而沒。十年冬,移懶、曷懶等路饑。



 熙宗天會十三年五月,甘露降於盧州熊岳縣。十五年七月辛巳,有司進四足雀。丙戌夜,京師地震。



 天眷元年夏,有龍見於熙州野水,凡三日。初,於水面見一蒼龍,良久而沒。次日,見金龍一,爪承一嬰兒,兒為龍所戲,略無懼色,三日如故。又見一人,乘白馬,紅袍玉帶,如少年官狀,馬前有六蟾蜍,凡三時乃沒,郡人競往觀之。七月丁酉,按出滸河溢,壞民廬舍。三年十二月丁
 丑,地震。



 皇統元年秋,蝗。十一月己酉,稽古殿火。二年二月,熙河路饑。三月辛丑,大雪。秋,燕、西東二京、河東、河北、山東、汴、平州大熟。三年,陜西旱。五月丁巳,京兆府貢瑞麥。七月丙庚,太原進獬豸及瑞麥。四年正月乙丑,陜西進嘉禾,十有二莖,一本七穎。十月甲辰,地震。五年閏月戊寅,大名府進牛生麟。壬辰,懷州進嘉禾。七年十一月,完顏秉德進三角牛。九年四月壬申夜,大風雨,雷電震寢殿鴟尾壞。有火入帝寢,燒帷幔,上懼,徙別殿。丁丑,有龍鬥于利州榆林河上。大風壞民居官舍十六
 七,木瓦人畜皆飄揚十餘里,死傷者數百,同知州事石抹里壓死。



 海陵天德二年十二月,野人採石炭,獲異香。



 貞元三年五月癸丑,南京大內災。三年十二月己丑,雨,木冰。



 正隆二年六月壬辰,蝗飛入京師。秋,中都、山東、河東蝗。四年十一月庚寅,霜附木。五年二月辛未,河東、陜西地震。鎮戎、德順等軍大風,壞廬舍,民多壓死。海陵問司天馬貴中等曰:「何為地震?」貴中等曰:「伏陽逼陰所致。」又問:「震而大風,何也?」對曰:「土失其性,則地以震。風為號
 令,人君嚴急則有風及物之災。」六年六月壬戌,大風壞承天門鴟尾。



 是歲,世宗居貞懿皇后憂,在遼陽,一日方寢,有紅光照其室,及其龍見於室上,又夜有大星流入其邸。八月,復有雲氣自西來,黃龍見其中,人皆見之。是時,臨潢府聞空中有車馬聲,仰視見風雲杳靄,神鬼兵甲蔽天,自北而南,仍有語促行者。未歲,海陵下詔南征。



 世宗大定二年閏二月辛卯,神龍殿十六位焚,延及太和、厚德殿。三年三月丙申,中都以南八路蝗。四年三月庚子夜,京師地震。七年辛丑,大風雷雨,拔木。臨潢
 府境禾黍穞生。嵐州進白兔二。八月,永興進嘉禾,異畝同穎。中都南八路蝗飛入京畿。十一月辛丑,尚書省火。是歲,有年。五年六月戊子,河南府進芝草十三本,得於芝田石上,薦之太廟。六月甲辰,大安殿楹產芝,有色如玉。丙午,京師地震,有聲自西北來,殷殷如雷,地生白毛。七月戊申,又震。十一月癸酉,大霧,晝晦。七年九月庚辰,地震。八月五月甲子,北望澱大風,雨雹,廣十里,長六十里。六月,河決李固渡,水入曹州。十年正月,鄧州進芝草。十一年六月戊申,西南路招討司苾里海水之地雨雹三十餘里,小者如雞卵,其一最大,廣三尺,
 長太餘,四五日始消。十二年三月庚寅,雨土。四月,旱。十三年正月,尚書省奏:「宛平張孝善有子曰合得,大定十二年三月旦以疾死,至暮復活,云是本良鄉人王建子喜兒。而喜兒前三年已死,建驗以家事,能具道之。此蓋假屍還魂,擬付王建為子。」上曰:「若是則姦倖小人競生詐偽,瀆亂人倫。」止付孝善。八月丁丑,策試進士於憫忠寺,夜半忽聞音樂聲起東塔上,西達於宮。考官完顏蒲捏、李晏等以為文運始開,得賢之兆。十四年八月丁巳朔,次颭里舌,是午,白龍見於御帳之東小港中,既乘臨雷雲而上,尾猶曳地,良久北去。十六年三月
 戊申,雨豆於臨潢之境,其形上銳而赤,食之味頗苦。五月戊申,南京宮殿火。日歲,中都、河北、山東、陜西、河東、遼東等十路旱、蝗。十七年七月,大雨,滹沱、盧溝水溢,河決白溝。二十年四月己亥,太寧宮門火。五月丙寅,京師地震,生黑白毛。七月,旱。秋,河決衛州。二十二年五月,慶都蝗蝝生,散漫十餘里。一夕大風,蝗皆不見。二十三年正月辛巳,廣樂園燈山焚,延及熙春殿。三月乙酉,氛埃雨土。四月庚子亦如之。五月丁亥,雨雹,地生白毛。二十四年正月辛卯朔,徐州進芝十有八莖。真定嘉禾二本,異畝同穎。二十六年正月庚辰,河南府
 進芝三本。秋,河決,壞衛州城。二十七年四月辛丑,京師地微震。



 章宗大定二十九年五月丁未,地生白毛。五月,曹州河溢。十二月,密州進白鶉、白雉各一。河間府進嘉禾。日冬無雪。



 明昌元年正月,懷州、河間等處進芝草、嘉禾。二月,地生白毛。六月庚子,都水進異卵。夏,旱。七月,淫雨傷稼。二年五月,桓、撫等州旱。秋,山東、河北旱,饑。三年秋,綏德虸蚄蟲生,旱。四年三月,御史中丞董師中奏:「迺者太白晝見,京師地震,北方有赤氣,遲明始散。天之示象,冀有以警悟聖主也。」上問:「所言天象何從得之?」師中
 曰:「前監察御史陳元升得之於一司天長行。」上曰:「司天臺官不奏固有罪,其以語人尤非。朕欲令自今司天有事而不奏者長行得言之,何如?」師中曰:「善。」五月,霖雨,命有司祈晴。六月,河決衛州,魏、清、滄皆被害。是歲,河北、山東、南京、陜西諸路大稔。邢、洺、深、冀及河北西路十六謀克之地,野蠶成繭。



 十一月壬午,木冰。五年七月丙戌,天壽節,先陰雨連日,至是開霽,有龍曳尾於殿前雲間。八月,河決陽武故堤,灌封丘而東。六年二月丁丑,京師地震,大雨雹,晝晦,大風,震應天門左鴟尾壞。六年八月,大雨震電,有龍起於渾儀鰲趺,臺忽中裂而摧。儀
 仆於臺下。



 承安元年五月,自正月不雨,至是月雨。六月,平晉縣民利通家蠶自成綿段,長七尺一寸五分,闊四尺九寸。二年,自正月至四月不雨。六月丙午,雨雹。四年三月戊午,雨雹。五月,旱。五年五月庚辰,地震。十月庚子,天久陰,是日雲色黃而風霾。癸卯晨,陰霜附木,至日入亦如之。



 泰和二年八月丙申,磁州武安縣鼓山石聖臺,有大鳥十集於臺上,其羽五色爛然,文多赤黃,赭冠雞項,尾闊而修,狀若鯉魚尾而長,高可逾人,九子差小侍傍,亦高四五尺。禽鳥萬數形色各異,或飛或蹲,或步或
 立,皆成行列,首皆正向如朝拱然。初自東南來,勢如連雲,聲如殷雷,林木震動,牧者驚惶,即驅牛擊物以驚之,殊不為動。俄有大鳥如雕鶚者怒來搏擊之,民益恐,奔告縣官,皆以為鳳凰也,命工圖上之。留二日西北去。按視其處,糞迹數頃,其色各異。遺禽數千,累日不能去。所食皆巨鯉,大者丈餘,魚骨蔽地。章宗以其事告宗廟,詔中外。三年四月,旱。十月己亥,大風。四年正月壬申,陰霧,木冰。三月丁卯,大風,毀宣陽門鴟尾。四月,旱。壬戌,萬寧宮端門災。十一月丁卯,陰。木冰凡三日。五年夏,旱。八年閏四月甲午,雨雹。河南路蝗。



 六月戊子,飛蝗入京
 畿。八月乙酉,有虎至陽春門外,駕出射獲之。時又有童謠云:「易水流,汴水流,更年易過又休休。兩家都好住,前後總成留。」至貞*中,舉國遷汴。



 衛紹王大安元年,徐、邳界黃河清五百餘里,幾二年,以其事詔中外。臨洮人楊珪上書曰:「河性本濁,而今反清,是水失其性也。正猶天動地靜,使當動者靜,當靜者動,則如之何,其為災異明矣。且《傳》曰:『黃河青,聖人生。』假使聖人生,恐不在今日。又曰『黃河清,諸侯為天子。』正當戒懼,以銷災變,而復誇示四方,臣所未喻。」宰相以為妖言,議誅之,慮絕言路,即詔大興鎖銷還本管。十一月丙申,
 平陽地震,有聲自西北來。戊戌夜,又震,自此時復震動,浮山縣尤劇,城廨民居圮者十七八,死者凡二三千人。二年二月乙酉,地大震,有聲殷殷然。六月、七月至九月晦。其震不一。十一月,京師民周修武宅前渠內火出,高二尺,焚其板橋。又旬日,大悲閣幡竿下石隙中火出,高二三尺,人近之即滅,凡十餘日。自是都城連夜燔爇二三十處。是歲四月,山東、河北大旱,至六月,雨復不止,民間斗米至千餘錢。三年二月乙亥夜,大風從西北來,發屋折木,吹清夷門關折。三月戊午,大悲閣災,延燒萬餘家,火五日不絕。山東、河北、河東諸路大旱。是歲,有
 男子郝贊詣省言:「上即位之後,天變屢見,火焚萬家,風折門關,非小異也,宜退位讓有德。」有司問:「爾狂疾乎?」贊大言曰:「我不狂疾,但為社稷計,宰相皆非其才。」每日省前大呼,凡半月。上怒,誅之隱處。



 崇慶元年七月辛未未時,有風從東來,吹帛一段高數十丈,宛轉如龍,墜於拱辰門內。是歲,河東、陜西、南京諸路旱。二年二月,放進士榜,有狂僧公言:「殺天子。」求之不知所在。是歲,河東、陜西大旱,京兆斗米至八千錢。



 至寧元年,宣宗彰德故園竹開白花,如鷺鷥藤。紫雲覆城上數日,俄而入繼大統。七月,以河東、陜西諸處旱,遣
 工部尚書高朵剌祈雨于巖瀆,至是雨足。時斗米有至錢萬二千者。八月癸巳,衛紹王遇弒。是日,海水不潮,寶坻鹽司懼其虧課,致禱無應。九月丙午,宣宗即位乃潮。初,衛王即位改元大安,四年改曰崇慶,即而又改曰至寧,有人謂曰:「三元大崇至矣。」俄而有胡沙虎之變。



 宣宗貞祐元年八月戊子夜,將曙,大霧蒼黑,跂步無所見,至辰巳間始散。十二月乙卯,雨,木冰。時衛州有童謠曰:「團鸑冬,劈半年。寒食節,沒人煙。」明年正月,元兵破衛,遂丘墟矣。二年六月,潮白河溢,漂古北口鐵裹關門至老王谷。庚申,南京寶鎮閣災。壬戌,上次宜村,有黃龍
 見於西北。冬,黃河自陜州界至衛州八柳樹,清十餘日,纖鱗皆見。十二月己酉,雨,木冰。三年二月戊午,大風,隆德殿鴟尾壞。三月戊辰,大風,霾。四月,自去冬不雨,至於是月。五月,河南大蝗。六月,京城中夜妄相驚逐狼,月餘方息。十月丙申,西北有霧氣如積土,至二更乃散。四年正月己未旦,黑霧四塞,巳時乃散,是春,河朔人相食。五月,河南、陜西大蝗。鳳翔、扶風、歧山、郿縣騑蟲傷麥。七月,旱。癸丑,飛蝗過京師。



 興定元年三月,宮中有蝗。四月,單州雹傷稼。陳州商水縣進瑞麥,一莖四穗。開封府進瑞麥,一莖三穗、二莖四
 穗。五月乙丑,河南大風,吹府門署以去。延州原武縣雹傷稼。七月癸卯,大社壇產嘉禾,一莖十五穗。秋,霖雨。十月,邠州進白兔。丹州進嘉禾。異畝同穎。二年四月,河南諸郡蝗。五月,秦、陜狼害人。六月,旱。是歲,京師屢火,遣禮部尚書楊雲翼鋋之。三年春,吏部火。四月癸未,陜右黑風晝起,有聲如雷,頃之地大震,平涼、鎮戎、德順尤甚,廬舍傾,壓死者以萬計,雜畜倍之。夏,旱。十二月壬申,雨,木冰。四年正月戊辰二更,天鳴有聲。壬子,晝晦,有頃大雷風雨。四月丁丑,大風吹河南府署飛百餘步,戶案門鑰開,文牘飄散,不知所在。六月,旱。七月,河南大水,
 唐、鄧尤甚。十二月癸酉,火。是歲,華州渭南縣民裴德寧家伐樹,破其中有赤色「太」字,表裏吻合。有司言與唐大歷中成都瑞木有文「天下太平」者,其事頗同,蓋太平之兆也。乞付史館。五年三月,以久旱,詔中外,仍命有司祈禱。十一月壬寅,京師相國寺火。十二月丁丑,霜附木。先是,有童謠云:「青山轉,轉山青。耽誤盡,少年人。」蓋言是時人皆為兵,轉鬥山谷,戰伐不休,當至老也。



 元光元年四月,京畿旱。十月,上獵近郊,獲白兔,群臣以為瑞。明日,御便殿,置鈴於項,將縱之,兔驚躍不已,忽斃几上。二年正月辛酉日午,有鶴千餘翔于殿庭,移刻
 乃去。七月乙卯,丹鳳門壞,壓死者數人。十一月,開封有虎害人。是時屢有妖怪,二年之中,白日虎入鄭門,吏部及宮中有狐狼,鬼夜哭於輦路,烏鵲夜驚,飛鳴蔽天。



 十二月,宣宗崩。



 哀宗正大元年正月戊午,上初視朝,尊太后為仁聖宮皇太后,太元妃為慈聖宮皇太后。是日,大風飄端門瓦,昏霾不見日,黃氣塞天,仁聖又夢乞丐萬數踵其後,心惡之,占者曰:「后為天下母,百姓貧窶,將誰訴焉?」遂敕京城設粥與冰藥以應之,人以為壬辰、癸巳之兆。又有人衣麻衣,望承天門大笑者三,大哭者三,有司拘而問之,
 其人曰:「我先笑者,笑許大天下將相無人。後哭者,哀祖宗家國破蕩至此也。」有司以為妖言,處之重典。上曰:「近詔草澤之士並許直言,雖涉譏訕亦不治罪,況此人言亦有理。止不應哭笑闕下耳。」乃杖之。二年正月甲申,有黃黑之昆。四月,旱。京畿大雨雹。三年春,大寒。三月乙丑,有火自吏部中出,大如斛,流行展轉,人皆驚避,踰時而滅。四月,旱、蝗。六月,京東雨雹,蝗死。四年六月丙辰,地震。八月癸亥,大風吹左掖門鴟尾墜,丹鳳門扉壞。是日,風、霜損禾皆盡。五年春,大寒。二月,雷而雪。木之華者皆敗。四月,鄭州大雨雹,桑柘皆枯。京畿旱。八月,御
 座上聞若有言者曰:「不放捨則何?」索之不見。七年十二月,新衛州北三里許,有影在沙上,如舊衛州城狀,寺塔宛然,數日乃滅。



 天興元年正月丁酉,大雪。二月癸丑,又雪。戊午,又雪。是時,鈞州、陽邑、盧氏兵皆大敗。五月,大寒如冬。七月庚辰,兵刃有火。閏八月己未,有箭射入宮中。九月辛丑府,大雷,工部尚書蒲乃速震死。二年六月,上遷蔡,自發歸德,連日暴雨,平地水數尺,軍士漂沒。及蔡始晴,復大旱數月。識者以為不祥。初,南京未破一二年間,市中有一僧不知所從來,持一布囊貯棗,日散與市人無窮,所在
 兒童百十從之。又有一人拾街中破瓦,復以石擊碎之。人皆以為狂,不曉其理,後乃知之,其意蓋欲使人早散,國家將瓦解矣。



\end{pinyinscope}