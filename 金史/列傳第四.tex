\article{列傳第四}

\begin{pinyinscope}

 始祖以下諸子



 勖本名烏野子宗秀隈可



 ◎宗室



 胡十門合住子布輝摑保衷本名醜漢齊本名掃合術魯胡石改宗賢本名阿魯撻懶卞本名吾母膏本名阿里刺弈本名三寶阿喜



 勖,字勉道。本名烏野,穆宗第五子。好學問,國人呼為秀
 才。年十六,從太祖攻寧江州,從宗望襲遼主于石輦鐸。太宗嗣位,自軍中召還,與謀政事。宗翰、宗望定汴州,受宋帝降。太宗使勖就軍中往勞之。宗翰等問其所欲。曰:「惟好書耳。」載數車而還。



 女直初無文字,及破遼,獲契丹、漢人,始通契丹、漢字,於是諸子皆學之。宗雄能以兩月盡通契丹大小字,而完顏希尹乃依仿契丹字製女直字。女直既未有文字,亦未嘗有記錄,故祖宗事皆不載。宗翰好訪問女直老人,多得祖宗遺事。太宗初即位,復進士舉,而韓昉輩皆在朝廷,文學之士稍拔擢用之。天會六年,詔書求訪祖宗遺事,以備國史,命勖與耶律迪
 越掌之。勖等採摭遺言舊事,自始祖以下十帝,綜為三卷。凡部族,既曰某部,復曰某水之某,又曰某鄉某村,以別識之。凡與契丹往來及征伐諸部,其間詐謀詭計,一無所隱。事有詳有略,咸得其實。



 自太祖與高麗議和,凡女直入高麗者皆索之,至十餘年,索之不已。勖上書諫曰:「臣聞德莫大於樂天,仁莫先於惠下。所索戶口,皆前世姦宄叛亡,烏蠢、訛謨罕、阿海、阿合束之緒裔。先世綏懷四境,尚未賓服,自先君在與高麗通,聞我將大,因謂本自同出,稍稍款附。高麗既不聽許,遂生邊釁,因致交兵,久方連和,蓋三十年。當時壯者今皆物故,子孫安於土
 俗,婚姻膠固,徵索不已,彼固不敢稽留,骨肉乖離,誠非眾願。人情怨甚可愍者,而必欲求為己有,特彼我之蔽,非一視同仁之大也。國家民物繁夥,幅員萬里,不知得此果何益耶。今索之不還,我以強兵勁卒取之無難。然兵凶器,戰危事,不得已而後用。高麗稱籓,職貢不闕,國且臣屬,民亦非外。聖人行義,不責小過,理之所在,不俟終日。臣愚以為宜施惠下之仁,弘樂天之德,聽免征索,則彼不謂己有,如自我得之矣。」從之。



 十五年,為尚書左丞加鎮東軍節度使、同中書門下平章事。預平宗磐之難,賜與甚多,加儀同三司,以「皇叔祖」字冠其銜。勖皆力
 辭不受。



 皇統元年,撰定熙宗尊號冊文。上召勖飲於便殿,以玉帶賜之。所撰《祖宗實錄》成,凡三卷,進入,上焚香立受之,賞賚有差。制詔左丞勖、平章政事弈職俸外別給二品親王俸傔。舊制,皇兄弟、皇子為親王給二品俸。宗室封一字王者給三品俸,勖等別給親王俸,皆異數也。宴群臣于五雲樓,勖進酒稱謝。帝起立,宰臣進曰:「至尊為臣下屢起,於禮未安。」上曰:「朕屈己待臣下,亦何害。」是日,上及群臣盡歡。俄同監修國史,進拜平章政事。光懿皇后忌辰,熙宗將出獵,勖諫而止。



 熙宗獵於海島,三日之間,親射五虎獲之。勖獻《東狩射虎賦》,上悅,賜以佩
 刀、玉帶、良馬。能以契丹字為詩文,凡游宴有可言者,輒作詩以見意。時上日與近臣酣飲,或繼以夜,莫能諫之。勖上疏諫,乃為止酒。進拜左丞相,兼侍中、監修如故。八年,奏上《太祖實錄》二十卷,賜黃金八十兩,銀百兩,重彩五十端,絹百匹,通犀、玉鉤帶各一。出領行臺尚書省事,召拜太保,領三省、領行臺如故,封魯國王。



 勖剛正寡言。海陵方用事,朝臣多附之者。一日,大臣會議,海陵後至,勖面責之曰:



 「吾年五十餘,猶不敢後,爾少年強健,肪敢如此。」海陵跪謝。九年,進拜太師,進封漢國王。海陵篡立,加恩大臣以收人望,封秦漢國王,領三省、監修如故。



 及
 宗本無罪誅,勖髭鬢頓白,因上表請老。海陵不許,賜以玉帶,優詔諭之。有大事令宰臣就第商議,入朝不拜。勖遂稱疾篤不言,表請愈切,海陵不懌,從之。以本官致仕,進封周宋國王。正隆元年,與宗室俱遷中都。二年,例降封金源郡王。薨,年五十九。



 撰定《女直郡望姓氏譜》及他文甚眾。大定二十年,詔曰:「太師勖諫表詩文甚有典則,朕自即位所未嘗見。其諫表可入《實錄》,其《射虎賦》詩文等篇什,可鏤版持之。」子宗秀。



 宗秀,字實甫,本名撕里忽。涉獵經史,通契丹大小字。善騎射,與平宗磐、宗雋之亂,授定遠大將軍,以宗磐世襲
 猛安授之。



 宗弼復取河南,宗秀與海陵俱赴軍前任使。宋將岳飛軍于亳、宿之間,宗秀率步騎三千扼其衝要,遂與諸軍逆擊敗之。師還,為太原尹,改婆速路統軍使、不受。高麗遣使以士產獻,卻之。入為刑部尚書,改御史中丞,授翰林學士。天德初,轉承旨,封宿國公,賜玉帶。歷平陽尹、昭義軍節度使,封廣平郡王。正隆二年卒官,年四十二。是歲,例降二品以上封爵,改贈金紫光祿大夫。



 康宗敬僖皇后生楚王謀良虎。次室溫都氏生昭武大將軍同喬茁。次室僕散氏坐事早死,生龍虎衛上將軍隈可。



 隈可亦作偎喝,美髯鬚,勇健有材略。從太祖伐遼,
 取寧江州,戰出河店。天眷二年,授驃騎上將軍,除迭魯苾撒糺詳隱,遷忠順軍節度使、興平軍節度使。天德二年,入為大宗正丞。四年,出為昭德軍節度使。以兄謀良虎孫喚端合扎謀克餘戶,授偎上京路扎里瓜猛安所屬世襲謀克。改德昌軍節度使,封廣平郡王。正隆二年,例奪王爵,改曷速館節度使,再改忠順軍節度使。大定元年,封宗國公,為勸農使,卒官,年六十五。



 始祖兄弟三人,保活里之後為神士懣、迪古乃,別有傳。



 胡十門者,曷蘇館人也。父撻不野,事遼為太尉。胡十門善漢語,通契丹大小字,勇而善戰。高永昌據東京,招曷
 蘇館人,眾畏高永昌兵彊,且欲歸之。胡十門不肯從,召其族人謀曰:「吾遠祖兄弟三人,同出高麗。今大聖皇帝之祖入女直,吾祖留高麗,自高麗歸于遼。吾與皇帝皆三祖之後。皇帝受命即大位,遼之敗亡有征,吾豈能為永昌之臣哉!」始祖兄阿古乃留高麗中,胡十門自言如此,蓋自謂阿古乃之後云。於是率其族屬部眾詣撒改,烏蠢降,營於馳回山之下。永昌攻之,胡十門力戰不能敵,奔于撒改。及攻開州,胡十門以糧餉給軍。後攻保州,遼獎以舟師遁,胡十門邀擊敗之,降其士卒。賞賜甚厚,以為曷蘇館七部勃堇,給銀牌一、木牌三。天輔二年卒。
 贈監門衛上將軍,再贈驃騎衛上將軍。



 子鉤室,嘗從攻顯州,領四謀克軍,破梁魚務,功最,以其父所管七部為曷蘇館都勃堇。



 有合住者,亦稱始祖兄苗裔,但不知與胡十門相去幾從耳。



 合住,曷速館苾里海水人也。仕遼,領辰、復二州漢人、渤海。



 子蒲速越,襲父職,再遷靜江中正軍節度使,佩金牌,為曷速館女直部長。



 子餘里也與胡十門同時歸朝,屢以糧餉助伐高永昌及高麗、新羅。後從宗望伐宋,以功遷真定府路安撫使兼曹州防禦使,佩金牌。授苾里海水世襲猛安。



 長子布輝,識女直、契丹、漢字,善騎射。年十
 八,宗弼選為扎也,從阿里、蒲盧渾追宋康王於明州。睿宗聞其才,召置麾下,從經略山東、河北、陜西,襲其父猛安,授昭勇大將軍。海陵伐宋,以本猛安兵從,半道與南征萬戶完顏福壽等俱亡歸,謁世宗於遼陽。



 世宗即位,除同知曷蘇館節度使事。刑部侍郎斜哥為都統,布輝副之,坐擅署置官吏、私用官中財物,削兩階解職。未浹旬,世宗獻享山陵。兵部尚書可喜、昭毅大將軍斡論、中都同知完顏璋等謀反,欲因上謁山陵舉事。斡論與布輝親舊,與之謀議,事具《可喜傳》。既知事不可成,乃與可喜、璋執斡論等上變。可喜不肯以始謀盡首,遂并誅之,
 而賞布輝、璋。除布輝浚州防禦使,累遷順天軍節度使。致仕,卒,年六十七。



 昭祖族人摑保者,從昭祖耀武於青嶺、白山。還至姑里甸,昭祖得疾,寢于村舍,洞無門扉,乃以車輪當門為蔽,摑保臥輪下為扞禦。已而賊至,刃交於輪輻間。摑保洞腹見膏,恐昭祖知之,乃然薪取膏以為炙,問之,以他肉對。昭祖心知之,遂中夜啟行。



 衷,本名醜漢,中都司屬司人,世祖曾孫。祖霸合布里封鄆王,父悟烈官至特進。大定中,收充閤門祗候,授代州宣銳軍都指揮使。歲旱,州委禱雨于五臺靈潭,步致其
 水,雨隨下,人為刻石紀之。四遷引進使,兼典客署令,改尚輦局使。扈從北幸,賜廄馬二以旌其勤。尋為夏國王李仁孝封冊使,歷寧海、蠡州刺史,入為大睦親府丞。除順義軍節度使,陛辭,賜金幣,特寵異之。移鎮鎮西。泰和六年,致仕,卒。



 衷孝悌貞謹,深悉本朝婚禮,皇族婚嫁每令衷相之。治復有能稱,其在寧海、蠡州,平賦役無擾,民立石頌遺愛。大安初,追贈輔國上將軍。



 齊,本名掃合,穆宗曾孫。父胡八魯,寧州刺史。大定中,以族次充司屬司將軍,授同知復州軍州事,累遷刑部員外郎。上諭曰:「本朝以來,未嘗有內族為六部郎官者,以
 卿歷職廉能,故授之。」先是,復州合廝罕關地方七百八里,因圍獵,禁民樵捕。齊言其地肥衍,令賦民開種則公私有益。上然之,為弛禁。即牧民以居,田收甚利,因名其地曰合廝罕猛安。



 章宗立,改戶部員外郎,出為磁州刺史,治以寬簡,未嘗留獄。屬邑武安,有道士視觀宇不謹,吏民為請鄰郡王師者代主之。道士忿奪其利,告王私置禁銅器,法當徒。縣令惡其為人,反坐之,具獄上。齊審其誣。又以王有德,不忍坐之,問同僚,無以對。齊曰:「道士同請即同居也,當准首,俱釋其罪。」其寬明有體,皆此類也。



 磁,名郡,刺史皆朝廷遴選,郡人以前政有聲如劉徽
 柔、程輝、高德裕皆不及也。河北提刑司以治狀聞。明昌三年,始議置諸王傅,頗難其選,乃以齊傅袞王。王將至任郡,猛安迎接,齊峻卻之。王怪問故,曰:「王國籓輔,猛安皆總戎職,於王何利焉,卻之以遠嫌也。」王悅服。王府家奴為不法,輒發還本猛安,終更無敢犯者。



 明年,授山東東、西路副統軍,兼同知益都府事。有惠愛,郡人為之立碑。轉彰化軍節度使。六年,移利涉軍。召見,勞尉有加。詔留守上京。承安二年,致仕,卒。齊明法識治體,所至有聲,內族中與丞相承暉並稱云。



 術魯,宗室子。從鄭王斡賽敗高麗于曷懶,取亞魯城,克
 寧江州,取黃龍府。出河店之役、達魯古城之役、護步荅岡之役皆力戰有功。東京降,為本路招安副使。敗遼兵,破同刮營。蘇州漢民叛走,術魯追復之,以功為謀克。天輔四年卒,年四十一。皇統中,贈鎮國上將軍。



 胡石改,宗室子也。從太祖攻寧江,敗遼兵於達魯古城,破遼主親兵,皆有功。遼軍來援濟州,胡石改與其兄實古乃以兵迎擊,敗之。還攻濟州,中流矢,戰益力,克其城。軍中稱其勇。從攻春、泰州,降之,並降境內諸部族,其不降者皆攻拔之。遼主西走,胡石改追至中京,獲其宮人、輜重凡八百兩。



 有思泥古者,復以本部叛去,胡石改以
 兵五百追及之,獲其親屬部人以還。德州復叛,胡石改以兵五千克其城。從婁室擊敗敵兵二萬於歸化之南,並降歸化。從取居庸關,並燕之屬縣及其山谷諸屯。移失部既降,復叛去,胡石改引兵追及,戰敗之,俘獲甚眾。澤州諸部有逃者,皆追復之。又敗叛人於臨潢,誅其酋領而安撫其人民。



 天眷二年,遷永定軍節度使,改武定軍,徙汴京留守。天德三年,授世襲猛安。卒,年六十八。



 宗賢,本名阿魯。太祖伐遼,從攻寧江州、臨潢府。太宗監國,選侍左右,甚見親信。臨潢復叛,從宗望復取之。為內庫都提點,再遷歸德軍節度使。政寬簡,境內大治。秩滿,
 士民數百人相率詣朝廷請留。及改武定軍,百姓扶老攜幼送數十里,悲號而去。改永定軍。秉德廉訪官吏,士民持盆水與鏡,前拜言曰:「使君廉明清直類此,民實賴之。」秉德曰:「吾聞郡僚廉能如一,汝等以為如何?」眾對曰:「公勤清儉皆法則於使君耳。」因謂宗賢曰:「人謂君善治,當在甲乙,果然賢使君也。」用是超遷兩階。



 天德初,授世襲謀克,馳驛召之。雄州父老相率張青繩懸明鏡於公署,老幼填門,三日乃得去。封定國公,再除忠順軍節度使,賜以玉帶。捕盜司執數人至府,宗賢問曰:「罪狀明白否?」對曰:「獄具矣。」宗賢閱其案,謂僚佐曰:「吾察此輩必
 冤。」不數日,賊果得,人服其明。改曷懶路兵馬都總管,歷廣寧尹,封廣平郡王。改崇義軍節度使,兼領北京宗室事。正隆例奪王爵,加金紫光祿大夫,改臨海軍。大定初,遣使召之。宗賢率諸宗室見於遼陽,除同簽大宗正事,封景國公,致仕。起為婆速路兵馬都總管,復致仕。卒。



 特進撻懶,宗室子。年十六,事太祖,未嘗去左右。出河店之役,太祖欲親戰,撻懶控其馬而止之曰:「主君何為輕敵。臣請效力。」即挺槍前,手殺七人。已而槍折,騎士曳而下者九人。太祖壯之曰:「誠得此輩數十,雖萬眾不能當也。」及戰於達魯古城,遼兵一千陣于營外,太祖遣撻懶
 往擊之。撻懶衝出敵陣,大敗其眾。攻臨潢府、春、泰州、中、西二京,皆有功。天輔六年,授謀克。



 天會四年,從伐宋,屢以功受賞。明年,再舉至汴。宗望聞宋人會諸路援兵於睢陽,遣撻懶與阿里刮將兵二千往拒之。敗其前鋒軍三萬于杞縣,又破三寨,擒宋京東路都總管胡直孺、南路都統制隋師元及其三將并直孺二子,遂取拱州,降寧陵。復破二萬于睢陽,進取亳州。聞宋兵十萬且至,會宗望益兵四千,合擊,大敗之。其卒二千,陣而立,馳之不動,即麾軍去馬擊之,盡殪,擒其將石瑱而還。帥府嘉其功,賞賚優渥。睿宗駐兵熙州,分遣諸將略地。撻懶以軍
 五百入六盤山十六寨,降其官八十餘,民戶四千,獲馬二千疋。



 皇統中,累加銀青光祿大夫。天德初,加特進,授世襲猛安。卒,年六十五。海陵遷諸陵于大房山,以撻懶嘗給事太祖,命作石像,置睿陵前。



 卞,本名吾母,上京司屬司人,大定二年,收充護衛,積勞授彰化軍節度副使,入為都水監丞,累遷中都、西京路提刑使,徙知歸德府、河平軍節度使。王汝嘉奏卞前在都水監導河有勞,除北京留守。未幾,改知大興府事。時有言,尚書左丞夾谷衡在軍不法,詔刑部問狀。事下大興府,卞輒令追攝,上以為失體,杖四十。久之,乞致仕,不
 許。拜御史大夫。先是,左司諫赤盞高門上言,御史大夫久闕,憲紀不振,宜選剛正疾惡之人肅清庶務。上由是用卞。前時孫鐸、賈鉉俱為尚書,鉉拜參知政事,而鐸再任,對賀客誦唐張在詩,有鬱鬱意。卞劾奏之,鐸坐降黜。既而復申前請,遂以金吾衛上將軍致仕,薨。



 膏,本名阿里刺,隸上京司屬司。大定十年,以皇家近親,收充東宮護衛。轉十人長,授御院通進,從世宗幸上京。會皇太子守國薨,世宗以膏親密可委,特命與滕王府長史臺馳驛往護喪。時章宗為金源郡王,亦留中都,且命膏等保護,諭之曰:「郡王遭此家難,哀哭當以禮節之,
 飲食尤宜謹視。」世宗還都,遷符寶郎,除吏部郎中。



 章宗即位,坐與御史大夫唐括貢為壽,犯夜禁,奪官一階,罷。明昌元年,起為同知棣州防禦使事,上書歷詆宰執。帝以小臣敢譏訕宰輔,杖八十,削一官,罷之,發還本猛安。



 明年,隆授同知宣德州事。召授武衛軍副都指揮使,四遷知大興府事,轉左右宣徽使。承安二年,拜尚書右丞,出為泰定軍節度使,移知濟南府。卒。



 弈,本名三寶,隸梅堅塞吾司屬司。大定七年,以近親充東宮護衛十人長,轉為尚廄局使。章宗即位,遷左衛副將軍,累遷右副都點檢,兼提點尚廄局使。諭旨曰:「汝非
 有過人才,第以久次遷授。當謹乃職,勿復有非違事,使朕聞之。」未幾,坐廄馬瘦,決三十。承安二年,改左副都點檢,兼職如舊。俄授同簽大睦親府事,卒。



 弈為人貪鄙,數以贓敗,帝愛其能治圍場,故進而委信之。



 阿喜,宗室子,好學問。襲父北京路筈柏山猛安,聽訟明決,人信而愛之。察廉能,除彰國軍節度副使,改上京留守判官。提刑司奏彰國軍治狀,遷同知速頻路節度事,改歸德軍,歷海、邳二州刺史,皆兼總押軍馬。



 宋統領劉文謙以兵犯宿遷,阿喜逆擊,破之。復破戚春、夏興國舟兵萬餘人,斬夏興國於陣。遷鎮國上將軍,再賜銀幣,為
 元帥左監軍紇石列執中前鋒。渡淮,破寶應、天長二縣。師還,遷同知歸德府事,改泗州防禦使。丁母憂,起復。大安二年,改華州防禦使,遷鎮南軍節度使。貞祐二年,改知大名府,充馬軍都提控,歷橫海、安化軍節度使,充宣差山東路左翼都提控。尋知濟南府事,徙沁南軍節度使,遷河南統軍使,兼昌武軍節度使,卒。



 贊曰:金諸宗室,自始祖至康宗凡八世。獻祖徙居海姑水納葛里村,再徙安出虎水。世祖稱海姑兄弟,蓋指其所居也。完顏十二部,皆以部為氏,宣宗詔宗室皆書姓氏,然亦有部人以部為氏,非宗室同姓者,遂不可辨矣。



\end{pinyinscope}