\article{列傳第五十五}

\begin{pinyinscope}

 ○徒單益都粘哥荊山劉均附王賓王進等附國用安時青



 徒單益都,不詳其履歷,嘗累官為延安總管。正大九年正月,行省事於徐州。時慶山奴撤東方之備入援,未至睢州,徐、邳義勝軍總領侯進、杜政、張興率本軍降大兵於永州。辛丑,大兵守徐張盆渡。益都到官才三日,懼兵少不能守,即令移剌長壽率甲士千人迎大兵。長壽軍
 無紀律,大兵掩之,一軍皆覆,徐危甚。益都籍州人及運糧埽兵得萬人。乙巳,大兵傅城,燒南關而去。侯進既降北,即以為京東行省,進遂請千人來襲。



 二月庚申,未明,大兵坎南城而上,守者皆散走,城中大呼曰:「大兵入南門矣!」益都聞之不及甲,率州署夜直兵三百,由黃樓而南,力戰禦敵。亂定,遷賞有差。由是軍勢稍振,復奪張盆渡,取蕭縣,破白塔,戰於土山,救被俘老幼五千還徐。既而,侯進亡命駐靈璧,杜政、張興亦慮為北所害,窮窘自歸。益都撫而納之,興留徐,杜政還邳州。



 益都資稟仁厚,持大體,二子兩姪為軍將,頗侵漁軍民。青州人王祐為
 埽兵總領。將兵千七百人,益都常倚之,雖有過亦不責。以故祐亦橫恣,與河間張祚、下邑令李閏、義勝都統封仙、遙授永州刺史成進忠輩,乘軍政廢弛,城中空虛。以六月丁巳夜燒草場作亂。時張興臥病,祐恐事不成,起興與同行。益都疑左右皆叛,挈妻子縋城而出,就從宜眾僧奴及東面總領劉安國軍。張興推祐為都元帥,復懼祐圖己,遂誅祐,並張祚殺之。因大掠城中。壬戌,國用安以行山東路尚書省事率兵至徐,張興率甲士迎之。用安輕騎而入,執興與其黨十餘人,斬之于市,遂以封仙為元帥,兼節度使,主徐州。



 益都窘無所歸,乃奔宿州,
 節度使紇石烈阿虎以益都為人所逐不納,乃與諸將駐于城南。時宿之鎮防軍有逃還者,阿虎以為叛歸亦不納。城中鎮防千戶高臘哥,結小吏郭仲安,謀就徐州將士內外相應以取宿,因歸楊妙真。甲戌夜半,開門納徐州總領王德全及妻弟高元哥軍。劉安國尋亦入城,縛阿虎父子殺之。州中請益都主帥府事,益都不從,曰:「吾國家舊人,為將帥亦久,以資性疏迂,不能周防,遂失重鎮。今大事已去,方逃罪不暇,豈有改易髻髮、奪人城池以降外方乎!」即日,率官吏而行,至穀熟東,遇大兵,不屈而死。



 徐州既歸海州,邳帥兀林答某亦讓印於杜政,
 遂送款於用安。已而宿州王德全、劉安國亦送款海州。惟益都不改髻髮,以至於死云。



 粘哥荊山,不知其所始,正大中,累官亳州節度使。九年正月己丑,游騎自鄧至亳,鈔鹿邑,營於衛真西北五十里。鹿邑令高昂霄知太康已降,即夜趨亳,道出衛真,呼縣令楚珩約同行。珩知勢不支,即明諭縣人以避遷之意,遂同走亳。丁未,二邑皆降。是日,軍至亳州城下。州止有單州兵四百人,號「鎮安軍」,提控楊春、邢某、都統戴興屯已六年。荊山悉籍城中丁壯為軍,修守具,而大兵亦不暇攻。四月,擁降民而北,城門閉,不之知也。



 五月,縱遷
 民收麥,老幼得出,丁壯悉留之。民往往不肯留而遁,數日,城為之空。荊山遣將領各詣所屬招之,並將領亦不返。「鎮安」者皆紅襖餘黨,力盡來歸,變詐反復,朝廷終以盜賊待之。荊山以遷民為軍,蓋防之也。及召外兵不至,乃請於歸德,得甲騎百餘,兩總領統之。既至,「鎮安」疑其謀己,乃乘將士新到不設備,至夜,掩殺殆盡。荊山出走衛真,楚珩與之馬而去,州中豪貴悉被剽略。



 劉堅者,初為大兵守城父,亳州復,擒之,囚之於獄。楊春謀欲北降,乃出之,使為宣差。乙巳,大兵石總管入州,改州為順天府,春為總管,戴興為同知,劉順治中,留黨項軍千人戍
 之。屬縣皆下,惟城父令李用宜不降,其妻子在亳,春以為質,竟不屈而死。春既據州,與劉堅坐樓上,召副提控邢某。邢剛直循理,將士嚴憚之,時臥病,聞春亂,流涕不自禁。春遣人舁致之,邢指春大罵,春慚恧無言。春欲殺荊山家,邢力勸止之,且令給道路費送之出城,邢尋病卒。二年夏四月,北省忒木泬攻歸德,春以戴興提精卒以往,獨與疲弱者守城。州人王賓遂反正,春渡河北遁。既而崔七斤為亂,殺王賓。朝廷不得已,以七斤為節度使,就其兵仗入蔡。八月,劉順攻亳州,破之,七斤為城父令所殺。未幾,單州軍以州人殺其家屬,召大兵來攻,蚧
 能拔,殺屬縣民而去。既渡河,知亳人不疑,復來攻,州竟為春所破。是年六月,宋人來攻,春出降,劉堅北走。



 劉均者,林慮人,時為亳州觀察判官。春既逐荊山,納款大兵,脅均同降。均佯應之,歸其家取朝服服之,顧謂妻子曰:「我起身刀筆,仰荷上知,始列朝著,又佐大籓,死亦足矣。今頭顱已如此,假使有十年壽,何以見先帝於地下乎。」即仰藥而死。



 王賓,字德卿,亳州人。貞祐二年進士。外若曠達,而深有謀畫。初調蘭陵主簿,辟虹縣令,尋入為尚書省令史,坐事罷歸鄉里。天興元年正月,亳州軍變,節度使粘哥荊
 山出走,楊春以州出降。既而,自以羸兵守之。賓與前譙縣尉王進、魏節亨、呂鈞約城中軍民復其州,楊春遂遁,遣節亨詣歸德以聞。哀宗嘉之,授進節度使,賓同知節度使,節亨節度副使,鈞觀察判官。楊春復以兵來攻,月餘不能拔,即渡河而北。



 六月,哀宗遷蔡,賓奉迎於州北之高安。上與語,大悅,恨用之晚,擢為行部尚書、世襲謀克。上初至亳,賓等適徵民丁負鐵甲入蔡,及會計忠孝軍家屬口糧,故留參知政事張天綱董之,就遷有功將士。時亳之糧儲不廣,賓等常吝惜,軍士以此歸怨。及運甲之役,復不欲行。會天綱與賓等於一樓上銓次立功
 等第,鎮防軍崔復哥、王六十之徒擐甲嘩噪登樓,天綱問曰:「即欲見殺,容我望闕拜辭。」賊曰:「無預相公。」即拽賓及呂鈞往市中。鈞且行且跪,涕淚俱下。賓岸然不懼,大叫曰:「不過殺我。但殺,但殺!」乃並害之。節度副使魏節亨、節度判官孫良、觀察副使孫九住皆被害。又數日,殺節度使王進。進嘗應荊山之募,由間道入汴京納奏,賞以物不受,又散家所有濟貧民,以死自勵。至汴,以勞遷本州節度判官。賜以白金,亦不受,一時甚稱之。



 有李喜住者,本宿州眾僧奴下宣差。天興二年四月,進糧入歸德,將還,聞亳州王進反正,制旨以喜住為振武都尉,將兵
 三千應援。是時,太赤圍亳步騎十萬,喜住以眾寡不敵,獨與三人間道入城,王進方議遷左軍林,喜住不可,進即以兵付喜住。大兵攻八日不能下。五月壬子,兵退。己未,官奴與阿里合提忠孝軍百人至亳,與諸將議遷可否。以為不可,當留輜重於蔡,選軍扈從入聖朵就武仙軍,遂入關中。關中地利可恃,又有郭蝦蟆等軍在西可恃。



 五月甲子,召官奴還歸德,不赴,再召,留其軍半於亳乃赴。六月壬辰,車駕舟行至亳,王進奏:「臣本軍伍,不知治體,如李喜住扈從入蔡,則毫不守矣。乞留治此州。」詔以喜住為集慶軍節度使,便宜從事,進領帥職。七月,進
 死。喜住先往城父督糧餫,聞亂遂不敢入亳,後投宋。



 論曰:金季之亂,軍士欲代其偏裨,偏裨欲代其主將,即群起而僨之。無復忌憚。益都、荊山皆忠亮之士,賓、進才略尤足取焉,而並不免於難,惜哉!



 國用安,先名安用,本名咬兒,淄州人。紅襖賊楊安兒、李全餘黨也。嘗歸順大元,為都元帥、行山東路尚書省事。天興元年六月,徐州埽兵總領王祐、義勝軍都統封仙、總領張興等夜燒草場作亂,逐元帥徒單益都。安用率兵入徐,執張興與其黨十餘人斬之,以封仙為元帥兼節度使,主徐州。宿州鎮防軍千戶高臘哥與東面總帥
 劉安國構徐州總帥王德全,殺宿帥紇石烈阿虎,以其州歸海州。邳州從宜兀林答某亦讓州於杜政,送款海州。既而皆歸安用。



 北大將阿術魯聞安用據徐、宿、邳,大怒曰:「此三州我當攻取,安用何人,輒受降。」遣信安、張進等率兵入徐,欲圖安用,奪其軍。安用懼,謀於德全,劫殺張進及海州元帥田福等數百人,與楊妙真絕,乃還邳州。會山東諸將及徐、宿、邳主帥,刑馬結盟,誓歸金朝。既盟,諸將皆散去,安用無所歸,遂同德全、安國託從宜眾僧奴自通於朝廷。眾僧奴遣人上奏:「安用以數州反正,功甚大。且其兵力強盛,材略可稱。國家果欲倚用,非極
 品重權不足以堅其許國之心。」未報。安用率兵萬人攻海州,未至,眾稍散去。安國因勸安用當赤心歸國,安用亦自知反復失計,事已無可奈何,於是復金朝衣冠。妙真怒其叛己,又懼為所圖,悉屠安用家走益都。安用遂選兵分將,期必得妙真,自此淮海之上無寧歲矣。



 未幾,朝廷遣近侍局直長因世英、都事高天祐持手詔至邳,以安用為開府儀同三司、平章政事、兼都元帥、京東山東等路行尚書省事,特封兗王,賜號「英烈戡難保節忠臣」,錫姓完顏,附屬籍,改名用安,賜金鍍銀印、駝紐金印、金虎符、世襲千戶宣命、敕樣、牌樣、御畫體宣、空頭河朔
 山東赦文,便宜從事,且以彭王妃誥委用安招妙真。用安始聞使者至,猶豫未決,以總領楊懋迎使者入,監于州廨,問所以來。世英對以封建事,意頗順。諸帥王、杜輩皆不欲宣言,欲殺使者。明日,用安乃出見使者,跪揖如等夷。坐定,語世英曰:「予向隨大兵攻汴,嘗於開陽門下與侯摯議內外夾擊。此時大兵病死者眾,十七頭項皆在京城,若從吾計出軍,中興久矣。朝廷乃無一人敢決者,今日悔將何及。」言竟而起。既而選人取朝廷賜物遍觀之,喜見顏色。復與使者私議,欲不以朝禮受之,世英等不可,即設宴拜授如儀,以主事常謹等隨使者奉表
 入謝。



 上復遺世英、天祐賜以鐵券一、虎符六、龍文衣一、玉魚帶一、弓矢二、封贈其父母妻誥命,及郡王宣、世襲宣、大信牌、玉兔鶻帶各十,聽同盟可賜者賜之。使者至邳,用安迎受如禮,始有入援意。及聞上將遷蔡州,乃遣人以蠟書言遷蔡有六不可,大率以謂:「歸德環城皆水,卒難攻擊,蔡無此險,一也。歸德雖乏糧儲,而魚芡可以取足,蔡若受圍,廩食有限,二也。大兵所以去歸德者,非畏我也,縱之出而躡其後,舍其難而就其易者攻焉,三也。蔡去宋境不百里,萬一資敵兵糧,禍不可解,四也。歸德不保,水道東行猶可以去,蔡若不守,去將安之,五也。
 時方暑雨,千里泥淖,聖體豐澤,不便鞍馬,倉卒遇敵,非臣子所敢言,六也。雖然,陛下必欲去歸德,莫如權幸山東。山東富庶甲天下,臣略有其地,東連沂、海,西接徐、邳,南扼盱、楚,北控淄、齊。若鑾輿少停,臣仰賴威靈,河朔之地可傳檄而定。惟陛下審察。」上以其言示宰臣。宰臣奏用安反復,本無匡輔志,此必參議張介等議之,業已遷蔡,議遂寢。



 初,世英等過徐,王德全、劉安國說之曰:「朝廷恩命豈宜出自用安,郡王宣吾二人最當得者,乞就留之。」世英乃留郡王宣、世襲宣、玉帶各二。由是與用安有隙,又懼為所圖,皆不聽其節制。十郡王者,李明德、封仙、
 張瑀、張友、卓翼、康琮、杜政、吳歪頭、王德全、劉安國也。用安必欲取山東,累征徐、宿兵,止以勤王為辭,二帥不應。用安怒,令杜政等率兵三千,以取糧為名,襲徐、宿。既入城,德全覺之,就留杜政、封仙不遣。用安愈怒,謂德全、安國必有謀,乃執桃園帥吳某等八九人下獄鞫問。二帥遣溫特罕張哥以杜政、封仙欲襲取徐州白用安,不聽,驅吳帥、張哥輩九人併斬之。張哥將死大呼曰:「國咬兒,汝無尺寸功,受國家大封爵,何負於汝,而從杜政等變亂,又殺無罪之人。今雖死,當與汝辨於地下矣。」會上遣臧國昌以密詔徵兵東方,故用安假朝命聲言入援,檄
 劉安國為前鋒,親率兵三千駐徐州城下招德全。德全終疑見圖,不出,繫封仙於獄,殺之,遣杜政出城。安國既至宿州,用安復召安國還,安國不從,獨與眾僧奴赴援。行及臨渙龍山寺,用安使人劫殺之,遂攻徐州,踰三月不能下,退歸漣水。於是,因世英以用安終不赴援,乃還朝,至宿州西,遇大兵,不屈而死,事聞,贈汝州防禦使。



 既而用安軍食不給,乞糧於宋,宋陽許之,即改從宋衣冠,而私與朝使相親。尋益乏食,軍民多亡去,乃命蕭均以嚴刑禁亡者,血流滿道。大元東平萬戶查剌將兵至漣水,遂降焉。查剌既渡河,趨蔡州,用安以詭計還漣水,復叛
 歸於宋,受浙東總管、忠州團練使,隸淮閫。甲午正月,聞大兵圍沛,用安往救之,敗走徐州。會移兵攻徐,用安投水死,求得其尸,皮刂面繫馬尾,為怨家田福一軍臠食而盡。



 用安形狀短小無須,喜與輕薄子游,日擊鞠衢市間,顧眄自矜,無將帥大體。



 介字介甫,平州人,正大元年經義進士第一,時為用安參議。



 初,天祐等出汴,微服間行,經北軍營幕,至通許崔橋,始有義軍招撫司官府,去京師二百里矣。至陳州,防禦使粘葛奴申始立州事。留二日,至項城,縣令朱珍立縣事,有士卒千二百人。至泰和縣,縣令王義立縣已五月矣。八月,至宿州,眾僧奴得報,
 且知朝廷授以權宿州節度使、兼元帥左都監之命,具彩輿儀衛出城五里奉迎。時東方不知朝廷音問已八月矣,官民見使者至,且拜且哭。有張顯者任俠尚氣知義理,即謂天祐曰:「東方不知朝廷音問已數月,今見使者,百姓皆感動。若不以聖旨撫慰之,恐失東民之必。我欲矯稱制旨宣諭,如何。」天祐書生,守規矩,不敢從,但以宰相旨集州民慰撫之,州民復大哭。明日,往徐州。



 時青,滕陽人。初與叔父全俱為紅襖賊,及楊安兒、劉二祖敗,承赦來降,隸軍中。興定初,青為濟州義軍萬戶。是時,叔父全為行樞密院經歷官。興定二年冬,全馳驛過
 東平,青來見,因告全將叛入宋,全秘之。頃之,青率其眾入于宋。宋人置之淮南,屯龜山,有眾數萬。



 興定四年,泗州行元帥府紇石烈牙吾塔遣人招之,青以書來。書曰:「青本滕陽良民,遭時亂離,扶老攜幼避地草莽。官吏不明此心,目以叛逆,無所逃死,竄匿淮海。離親舊、去鄉邑,豈人情之所樂哉。僕雖偷生寄食他國,首丘之念,未嘗一日忘之。如朝廷赦青之罪,乞假邳州以屯老幼。當襲取盱眙,盡定淮南,以贖往昔之過。」牙吾塔復書曰:「公等初亦無罪,誠能為國建功,全軍來歸,即吾人也。邳州吾城,以吾人居之,亦何不可。《易》曰:『君子見幾而作,不俟終
 日。』公其亟圖之。生還父母之邦,富貴終身,傳芳後世,與其羈縻異域,目以兵虜,孰愈哉?」牙吾塔奏其事。十月,詔加青銀青榮祿大夫,封滕陽公,仍為本處兵馬總領元帥、兼宣撫使。青潛表陳謝,復以邳州為請。樞密院奏:「恐青意止欲得邳州。可諭牙吾塔,若青誠實來歸,即當授之。如審其詐,可使人入宋境宣布往來之言,及所援官爵,亦行間之術也。」青既不得邳州,復為宋守。



 興定五年正月二十五日夜,青襲破泗州西城,提控王祿遇害。是時,時全為同簽樞密院事,朝廷不知青襲破西城,止稱宋人而已。詔全往督泗州兵取西城。全至泗州,獲紅襖
 賊一人,詰問之,乃知青為宋京東鈐轄,襲破西城。全頗喜,乃殺其人以滅口。牙吾塔晝夜力戰,募死士以梯衝逼城,青縋兵出拒不得前。牙吾塔遣提控王應孫穴城,東北隅,青夜出兵來襲,擊卻之。越二日,復出又卻之。攻城益急,青以舟兵二千合城中兵來犯牙吾塔營,提控斡魯朵先知,設伏掩擊,青兵大敗,溺淮水死者千人,自是不復出矣。王應孫穴城將及城中,青隧地然薪,逼出之。青乘城指麾,流矢中其目,餘眾往往被創,樓堞相繼摧壞,城中恟懼,遂無固志。二月二十六日夜,青拔眾走,遂復西城。



 元光元年二月,全與元帥左監軍訛可,節制
 三路軍馬伐宋。詔曰:「卿等重任,毋致不和,以貽喪敗。其資糧可取,規取失宜不能得之,罪在訛可,既已得之,不能運致以為我用,罪在全。」全與訛可由潁、壽進渡淮,敗宋人于高塘市,攻固始縣,破宋廬州將焦思忠兵。無何,獲生口言,時青受宋詔,與全兵相拒,全匿其事。



 五月,兵還,距淮二十里,諸軍將渡,全矯稱密詔「諸軍且留收淮南麥」,遂下令人獲麥三石以給軍。眾惑之,訛可及諸將佐勸之不聽,軍留三日。訛可謂全曰:「今淮水淺狹,可以速濟。時方暑雨,若值暴漲,宋乘其後,將不得完歸矣。」全力拒之。從宜達阿、移失不、斜烈、李辛稍稍不平,全怒曰:「
 訛可一帥耳,汝曹黨之。汝曹致身至此,皆吾之力。吾院官也,於汝無不可者。」眾乃不敢言。是夜,大雨。明日,淮水暴漲,乃為橋渡軍。宋兵襲之,軍遂敗績。橋壞,全以輕舟先濟,士卒皆覆沒。宣宗乃下詔誅之,遣官招集潰軍,詔曰:「大軍渡淮,每立功效。諸將謬誤,部曲散亡,流離憂苦,朕甚閔焉。各歸舊營,勉圖自效。」又詔曰:「陣亡把軍品官子孫,十五以上者依品官子孫例隨局承應,十五以下、十歲以上者依品從隨局給俸,至成人本局差使。無子孫官,依例給俸。應贈官、賻錢、軍人家口當養贍者。並如舊制。」



 贊曰:金自章宗季年,宋韓侂胄構難,招誘鄰境亡命以撓中原,事竟無成。而青、徐、淮海之郊民心一搖,歲遇饑饉,盜賊蜂起,相為長雄,又自屠滅,害及無辜,十餘年糜沸未息。宣宗不思靖難,復為伐宋之舉,迄金之亡,其禍尤甚。簡書所載國用安、時青等遺事,至今仁人君子讀之猶蹙頞終日。當時烝黎,如魚在釜,其何以自存乎。兵,兇器也。金以兵得國,亦以兵失國,可不慎哉,可不慎哉!



\end{pinyinscope}