\article{列傳第五十七}

\begin{pinyinscope}

 ○粘葛奴申劉天起附完顏婁室烏古論鎬張天綱完顏仲德



 粘葛奴申,由任子入宮,或曰策論進士。天興初,卒開封府,以嚴乾稱。其年五月,擢為陳州防禦使。時兵戈搶攘,道路不通,奴申受命,毅然策孤騎由間道以往。陳自兵興,軍民皆避遷他郡,奴申為之擇官吏,明號令,完顏郭,立廬舍,實倉廩,備器械。未幾,聚流亡數十萬口,米一斛
 直白金四兩,市肆喧哄,如汴之闤鋏,京城危困之民望而歸者不絕,遂指以為東南生路。



 明年,哀宗走歸德,改陳州為金興軍,馳使褒諭,以奴申為節度使。俄拜參知政事,行尚書省于陳。於是,奴申立五都尉以將其兵,建威來豬糞、虎威蒲察合達、振武李順兒、振威王義、果毅完顏某,凡招撫司至者皆使隸都尉司。



 是時,交戰無虛日,州所屯軍十萬有餘。奴申與官屬謀曰:「大兵日至,而吾州糧有盡,奈何?」乃減軍所給,月一斛五斗者作一斛,又作八斗,又作六斗。將領則不給。人心稍怨。故李順兒、崔都尉因而有異志,劉提控及完顏不如哥提控者預
 焉。奴申知其謀,常以兵自防。及聞大元兵往朱仙鎮市易,奴申遣五都尉軍各二百人,以李順兒、副都尉崔某將之,襲項城寨。令孫鎮撫者召順兒議兵事,孫至其家,順兒已擐甲,孫欲觀其刀,順兒拔示之,孫色動,即出門奔去。順兒追殺之,乃上馬,引兵二百人入省,說軍士曰:「行省剋減軍糧,汝輩欲飽食則從我,不欲則從行省。」於是,省中軍士皆坐不起。奴申聞變走後堂,追殺之。提控劉某加害,解其虎符以與順兒,并殺其子侄婿及鄉人王都尉。順兒令五都尉軍皆甲,守街曲。自稱行省,署元帥,都尉。以劉提控語不順,斬之坐中。明日,遂遣剋石烈
 正之送款於汴。崔立乃遣其弟倚就加順兒淮陽軍節度使,行省如故。



 未幾,虎威都尉蒲察合達與高元帥者盡殺順兒之徒,舉城走蔡州。大兵覺,追及孫家林,老幼數十萬少有脫者。



 初,奴申聞崔立之變,遣人探其事情,而順兒、崔都尉亦密令人結構崔立,適與奴申所遣者同往同還。順兒懼其謀泄,故發之益速。奴申亦知其謀,故遣襲項城,欲因其行襲殺之,然已為所先。



 劉天起者,起於匹夫,初甚庸鄙。汴京戒嚴,嘗上書以干君相,願暫假一職以自效。每言戰國兵法,平章白撒等信之,令景德寺監造革車三千兩。天興元年,授都招撫使,佩金符。
 召見,乞往陳州運糧,上從之,一時皆竊笑其僥倖。及至陳,行軍殊有方略,每出戰,數有功,陳人甚倚重之。順兒之變,天起偃蹇不從,為所殺。同時一唐括招撫者亦不屈而死。



 完顏婁室三人,皆內族也,時以其名同,故各以長幼別之。



 正大八年,慶山奴棄京兆,適鷹揚都尉大婁室運軍器至白鹿原,遇大兵與戰,兵刃既盡,以絳繫掉金牌,力戰而死。



 九年正月,大兵至襄城,元帥中婁室、小婁室以馬軍三千遇之於汝墳。時大兵以三四十騎入襄城,驅驛馬而出,又入東營,殺一千夫長,金人始覺之。兩婁室
 以正旦飲將校,皆醉不能軍,遂敗,退走許州。會中使召入京師。天興二年正月,河朔軍潰,哀宗走歸德,中婁室為北面總帥,小婁室左翼元帥,收潰卒及將軍夾谷九十奔蔡州。蔡帥烏古論栲栳知其跋扈不納,遂走息州,息帥石抹九住納之。時白華以上命送虎符於九住為息州行帥府事。九住出近侍,好自標致,騶從盈路。三人者妒之,各以招集勤王軍士為名,得五六百人,州以甲仗給之。久之,漸生猜貳,九住亦招負販牙儈數百人為「虎子軍」,夜則擐甲為備。一日,九住使一萬戶巡城,三帥執而驅之,使大呼云:「勿學我欲開西門反!」即斬之。乃召
 九住,九住欲不往,懼州人及禍,乃從三百卒以往。三帥令甲士守街曲,九住從者過,處處執之。九住獨入,三帥問汝何為欲反,九住曰:「我何緣反?」三帥怒,欲殺者久之。小婁室意稍解,頗為救護,得不殺,使人鎖之。以夾谷九十為帥,兼權息州。



 蔡帥栲栳聞九住為三帥所誣,上奏辨之,三帥亦捃摭九住之過上聞。朝廷主栲栳之辨,且不直三帥。六月,赦至蔡,栲栳懼九住為三帥所誅,遣二卒馳送詔書於息,乃得免。及上將幸蔡,密召中婁室引兵來迓,婁室遲疑久之,乃率所招卒奉迎。七月,上遣近侍局使入息州括馬,即召九住。九住至,與中婁室辨於
 上前。時中婁室已授同簽樞密院事,上不欲使之終訟,乃罷九住帥職,授戶部郎中,以烏古論忽魯為息州刺史。



 時有土豪劉禿兒、馬安撫者自蔡朝還,以軍儲不給叛入宋,州之北關為所焚毀。是時城中軍無幾,日有叛去者,且覘知宋人有窺息之意,息帥懼,上奏請益兵為備。朝廷以參知政事抹捻兀典行省事于息州,中婁室以同簽樞密院事為總帥,小婁室以副點檢為元帥,王進為彈壓帥,夾谷九十為都尉,以忠孝馬軍二百、步軍五百屬之,行省、院於息。將行,上諭之曰:「北兵所以常取全勝者,恃北方之馬力,就中國之技巧耳,我實難與之
 敵。至於宋人,何足道哉。朕得甲士三千,縱橫江、淮間有餘力矣。卿等勉之。」



 八月壬辰,行省遣人奏中渡店之捷。初,兀典等赴息,既至之夜,潛遣忠孝軍百餘騎襲宋營於中渡。我軍皆北語,又散漫似之,宋人望之駭愕奔潰,斬獲甚眾。復奏元帥張閏不遵約束,失亡軍士,乞正典刑。婁室表閏無罪,上遣人赦之,比至,已死獄中。蓋閏為婁室腹心,九住之獄皆閏發之。兀典廉得其事,因其失律而誅之也。九月,以忽魯退縮,不能撫御,民多叛去,奪其職,以夾谷九十權息州事。



 十一月,宋人以軍二萬來攻。城中食盡,乃和糴,既而括之,每石止留一斗,並括金
 帛衣物,城中皆無聊矣。前兩月,蔡州以軍護老幼萬口來就食,北兵覺之,追及於二十里之外,至息者才十餘人。至是,蔡問不通。行省及諸帥日以歌酒為事,聲樂不絕。下及軍士強娶寡婦幼女,絕滅人理,無所不至。



 三年甲午正月,蔡凶問至,諸帥殺之以滅口,然民間亦頗有知者。初,諸帥欲北降,而遞相猜忌,無敢先發者。數日,蔡信哄然,諸帥屏人聚議,皆言送款南中為便。時李裕為睦親府同僉桓端國信使下經歷官,乃使送款於宋。遂發喪設祭,謚哀宗曰昭宗。州民奉行省為領省,丞相、總帥、左平章皆娶婦。十三日,舉城南遷,宋人焚州樓櫓。州
 人老幼渡淮南行,入羅山,委曲之信陽。北兵見火起,追及之,無有免者,且誅索行省已下官屬于宋。宋人令官屬入城,托以犒賞,從萬戶以上六七百人皆殺之,軍中亦有奪命死敵者。宋人諭諸軍,行省已下有罪已處置,汝等就迷魂寨安屯,遂以軍防之。既而與北軍接,南軍斂避,一軍悉為所殺。



 烏古論鎬,本名栲栳,東北路招討司人。由護衛起身,累官慶陽總管。天興初,遷蔡、息、陳、潁等州便宜總帥。二年,哀宗在歸德,蒲察官奴、國用安欲上幸海州,未決。會鎬餫米四百餘斛至歸德,且請幸蔡,上意遂決。先遣直學
 士烏古論蒲鮮如蔡,告蔡人以臨幸之意。六月,徵蔡、息軍馬來迓,以蔡重鎮,且慮有不測,詔鎬勿遠迎。



 辛卯,車駕發歸德,時久雨,朝士扈從者徒行泥水中,掇青棗為糧,數日足脛盡腫,參政天綱亦然。壬辰,至亳,上黃衣皁笠,金兔鶻帶,以青黃旗二導前,黃傘擁後,從者二三百人,馬五十餘匹而已。行次城中,僧道父老拜伏道左,上遣近侍諭以「國家涵養汝輩百有餘年,今朕無德,令爾塗炭。朕亦無足言者,汝輩無忘祖宗之德可也。」皆呼萬歲,泣下。留一日,進亳之南六十里,避雨雙溝寺中,蒿艾滿目,無一人迹,上太息曰:「生靈盡矣。」為之一慟。是日,小
 婁室自息來迓,得馬二百。己亥,入蔡。蔡之父老千人羅拜於道,見上儀衛蕭條,莫不感泣,上亦歔欷者久之。



 七月,以鎬為御史大夫,總帥如故。初,鎬守蔡,門禁甚嚴,男女樵采,必以墨識其面,人有以錢出者,十取一分有半以贍軍。上至蔡,或言其非便,即弛其禁。時大兵去遠,商販頗集,小民鼓舞,以為復見太平,公私宿釀,一日俱盡。



 郾城土豪盧進殺其長吏,自稱招撫使,以前關、陜帥府經歷范天保為副。至是,天保來見,進麥三百石及麞鹿脯、茶、蜜等物,遂賜進金牌,加天保官,自是進物者踵至。既而遣內侍殿頭宋珪與鎬妻選室女備後宮,已得數
 人,右丞忽斜虎諫曰:「小民無知,將謂陛下駐蹕以來,不聞恢復遠略,而先求處女以示久居。民愚而神,不可不畏。」上曰:「朕以六宮失散,左右無人,故令採擇。今承規誨,敢不敬從。止留解文義者一人,餘皆放遣。」



 是時,從官近侍率皆窮乏,悉取給於鎬,鎬亦不能人滿其欲,日夕交譖於上,甚以尚食闕供為言。上怒,雖擢拜大夫,而召見特疏。小婁室之在息州也,與石抹九住有隙,怨鎬為九住辨曲直。及上幸蔡,婁室見於雙溝,因厚誣鎬罪,上頗信之。鎬自知被讒,憂憤鬱抑,常稱疾在告。會前參知政事石盞女魯懽姪大安來,以女魯懽無反狀,為官奴所
 殺,白尚書省求改正,尚書省以聞。上曰:「朕嘗謂女魯懽反邪,而無迹可尋。謂不反邪,朕方暴露,遣人征援兵,彼留精銳自防,發其羸弱者以來。既到睢陽,彼厚自奉養,使朕醯醬有闕。朕為人君,不當語此細事,但四海郡縣,孰非國家所有?坐保一城,臣子之分,彼乃自負而有驕君上之心,非反而何?然朕方駕馭人材以濟艱難,錄功忘過此其時也,其釐正之。」群臣知上意之在鎬也,數為右丞仲德言之。仲德每見上,必稱鎬功業,宜令預參機務,又薦以自代,上怒少解。及參政抹捻兀典行省息州,鎬遂以御史大夫權參知政事。



 九月,大兵圍蔡,鎬守南面,
 忠孝軍元帥蔡八兒副之。未幾,城破被執,以招息州不下,殺之。



 烏古論先生者,本貴人家奴,為全真師。佯為狂態,裸顛露足,綴麻為衣,人亦謂之「麻帔先生」。宣宗嘗召入宮,問以秘術。因出入大長主家,殊有穢迹,上微聞之,敕有司掩捕,已逃去。正大末,從鎬來官汝南,人皆知與其妻通,而鎬不知。生不自安,求出,鎬為營道宇,親率僧道送使居之。車駕將至蔡,生欲遁無所往,因自言能使軍士服氣不費糧。右丞仲德知其妄,乃奏:「欲如田單假神師退敵之意,授一真人之號,旋出奇計,北兵信巫必駭異之,或可以有成功。」參政天綱以為不可,遂止。復
 求入見,言有詭計可以退敵。及見,長揖不拜,且多大言,欲出說大帥噴盞為脫身計。時郎中移剌克忠、員外郎王鶚具以向者「麻帔」為言,上怒殺之。



 贊曰:晉劉越石長於撫納,短於駕馭,以故取敗。粘葛奴申陳州之事,殆類之矣。三婁室皆金內族,唯大婁室死得其所,其兩婁室讒賊人也,襄城事急,醉不能軍,乃逭一死,金失政刑,一至於是。烏古論鎬幸蔡之請,雖非至謀,區區效忠以讒見忌,哀宗之明,蓋可知矣。



 張天綱,字正卿,霸州益津人也。至寧元年詞賦進士。性寬厚端直,論議醇正,造次不少變。累官咸寧、臨潼令,入
 補尚書省令史,拜監察御史,以鯁直聞。陞戶部郎中,權左右司員外郎。哀宗東幸,遷左右司郎中,扈從至歸德,改吏部侍郎。知元帥官奴有反狀,屢為上言之,上不從,官奴果變,遂擢天綱權參知政事。及從上遷蔡,留亳州,適軍變,天綱以便宜授作亂者官,州賴之以安。及蔡,轉御史中丞,仍權參政。



 扶溝縣招撫司知事劉昌祖上封事,請大舉伐宋,其略云:「官軍在前,饑民在後,南踐江、惟,西入邑、蜀。」頗合上意。上命天綱面詰其蘊藉,召與語無可取者,然重違上命,且恐閉塞言路,奏以為尚書省委差官。護衛女奚烈完出、近侍局直長粘合斜烈、奉御陳
 謙、權近侍局直長內族泰和四人,以食不給出怨言,乞往陳州就食。天綱奏令監之出門任所往。才出及汝南岸,遇北兵皆見殺,時人快之。妖人烏古論先生者自言能使軍士服氣,可不費糧。右丞仲德援田單故事,欲假其術以駭敵,語在《烏古論鎬傳》。上頗然之,天綱力辨以為不可,遂止,且曰:「向非張天綱,幾為此賊所誑。」軍吏石抹虎兒者求見仲德,自謂有奇計退敵,出馬面具如獅子狀而惡,別制青麻布為足、尾,因言:「北兵所恃者馬而已,欲制其人,先制其馬。如我軍進戰,尋少卻,彼必來追。我以馴騎百餘皆此狀,仍繫大鈴于頸,壯士乘之,以突彼
 騎,騎必驚逸,我軍鼓噪繼其後,此田單所以破燕也。」天綱曰:「不可。彼眾我寡,此不足恃,縱使驚去,安保其不復來乎?恐徒費工物,只取敵人笑耳。」乃罷之。



 蔡城破,為宋將孟珙得之,檻車械至臨安,備禮告廟。既而,命臨安知府薛瓊問曰:「有何面目到此?」天綱對曰:「國之興亡,何代無之。我金之亡,比汝二帝何如?」瓊大叱曰:「曳去。」明日,遂奏其語,宋主召問曰:「天綱真不畏死耶?」對曰:「大丈夫患死之不中節爾,何畏之有。」因祈死不已。宋主不聽。初,有司令供狀必欲書虜主,天綱曰:「殺即殺,焉用狀為!」有司不能屈,聽其所供,天綱但書故主而已。聞者憐之。後不
 知所終。



 完顏仲德,本名忽斜虎,合懶路人。少穎悟不群,讀書習策論,有文武才。初試補親衛軍,雖備宿衛而學業不輟。中泰和三年進士第,歷仕州縣。貞祐用兵,辟充軍職,嘗為大元兵所俘,不踰年盡解其語,尋率諸降人萬餘來歸。宣宗召見,奇之,授邳州刺史、兼從宜。增築城壁,匯水環之,州由是可守。哀宗即位,遙授同知歸德府事,同簽樞密院事,行院於徐州。徐州城東西北三面皆黃河而南獨平陸,仲德疊石為基,增城之半,復浚隍引水為固,民賴以安。



 正大五年,詔關陜以南行元帥府事,以備小
 關及扇車回。時北兵叩關,仲德適與前帥奧屯阿里不酌酒更代,而兵猝至,遂驅而東。阿里不素無守禦之策,為有司所劾,罪當死。仲德上書引咎,以謂「北兵越關之際,符印已交,安得歸罪前帥,臣請受戮。」上義之,止杖阿里不而貰其死。



 六年,移知鞏昌府,兼行總帥府事。時陜西諸郡已殘,仲德招集散亡,得軍數萬,依山為柵,屯田積穀,人多歸焉。一方獨得小康,號令明肅,至路不拾遺。八年四月,詔授仲德鞏昌行省及虎符、銀印。天興元年九月,拜工部尚書、參知政事,行尚書省事於陜州。時兀典新敗,陜州殘破,仲德復立山寨,安撫軍民。會上以蠟
 丸書徵諸道兵入援,行省院帥府往往觀望不進,或中道遇兵而潰,惟仲德提孤軍千人,歷秦、藍、商、鄧,擷果菜為食,間關百死至汴。至之日,適上東遷。妻子在京師五年矣,仲德不入其家,趨見上於宋門,問東幸之意。知欲北渡,力諫云:「北兵在河南,而上遠徇河北,萬一無功,得完歸乎?國之存亡,在此一舉,願加審察。臣嘗屢遣人奏,秦、鞏之間山巖深固,糧餉豐贍。不若西辛,依險固以居,命帥臣分道出戰,然後進取興元,經略巴蜀,此萬全策也。」上已與白撒議定,不從,然素重仲德,且嘉其赴難,進拜尚書省右丞、兼樞密副使,軍次黃陵岡。



 二年正月,車駕
 至歸德,以仲德行尚書省于徐州。既至,遣人與國用安通問。沛縣卓翼、孫璧沖者初投用安,用安封翼為東平郡王,璧沖博平公,升沛縣為源州。已而翼、璧沖來歸,仲德畀之舊職,令統河北諸砦,行源州帥府事。用安累檄王德全入援,不赴。仲德至徐,德全大恐,求赴歸德。仲德留之,遣人納奏帖云:「徐州重地,德全不宜離鎮。」仲德虛州廨不居,亦無兵衛自防,日以觀書為事,而德全自疑益甚。



 二月,魚山總領張獻作亂,殺元帥完顏胡土降北。仲德累議討之,德全不從,即領麾下十許人,親勸民兵得三百人,徑往魚山,而從宜嚴祿已誅獻反正,仲德撫
 慰軍民而還。有曹總領者,盜御馬東行,制旨諭行省討之,仲德既殺賊,德全欲功出己,殺曹黨四十八人。



 三月,阿術魯攻蕭縣,游騎至徐,德全馬悉為所邀。仲德時往宿州,德全以失馬故,始議救蕭縣,遣張元哥、苗秀昌率騎八百以往。未及交戰,元哥退走,北兵掩之,皆為所擒殺之,蕭縣遂破。四月,仲德陽以關糧往邳州,州官出迎,就執德全並其子殺之,餘黨之外,一無所問,闔郡稱快。



 初,完顏胡土以遙授徐州節度,往帥嚴祿軍於永州北保安鎮。時祿已為從宜,在碭山數年,又得士心。忽土到,軍士不悅,二月辛卯夜,遂為總領張獻、崔振所害。吏部
 郎中張敏修,忽土下經歷官,乃以軍變脅嚴祿降北。祿佯應之,陰召永州守陳立、副招撫郭升,會諸義軍赴保安鎮誅作亂者。軍夜至,祿遣敏修召獻、振計事,二人不疑,介胄而至,及其黨與皆為祿所殺。徐州去保安百里,行省聞之來討,會祿已反正,乃以便宜授祿行元帥左都監,就佩忽土虎符。朝廷復授祿遙領歸德知府、兼行帥府事。未幾,大元將阿術魯兵至保安,祿夜遁。後祿聞官奴變,一軍頓徐、宿間幾一月,遂投漣水,敏修入徐。



 五月,詔仲德赴行在。時官奴已變,官屬懼為所紿,勸勿往。仲德曰:「君父之命,豈辨真偽耶?死亦當行。」尋使者至,果
 官奴之詐。六月,官奴誅,詔仲德議遷蔡,仲德雅欲奉上西幸,因贊成之。及蔡,領省院,事無巨細,率親為之,選士括馬,繕治甲兵,未嘗一日無西志。近侍左右久困睢陽,幸即汝陽之安,皆娶妻營業,不願遷徙,日夕為上言西行不便。未幾,大兵梗路,竟不果行。仲德每深居燕坐,瞑目太息,以不得西遷為恨。



 是月,上至蔡,命有司修見山亭及同知衙,為遊息之所。仲德諫曰:「自古人君遭難,播越于外,必痛自刻苦貶損,然後可以克復舊物。況今諸郡殘破,保完者獨一蔡耳。蔡之公廨固不及宮闕萬一,方之野處露宿則有加矣。且上初行幸,已嘗勞民葺治,
 今又興土木之役以求安逸,恐人心解弛,不足以濟大事。」上遽命止之。



 七月,定進馬遷賞格。每甲馬一匹或二匹以上,遷賞有差。自是,西山帥臣范真、姬汝作等各以馬進,凡得千餘匹,以抹捻阿典領之。又遣使分詣諸道徵兵赴蔡,得精銳萬人。又以器甲不完,命工部侍郎術甲咬住監督修繕,不踰月告成。軍威稍振,扈從諸人茍一時之安,遂以蔡為可守矣。



 魯山元帥元志領軍千餘來援。時諸帥往往擁兵自固,志獨冒險數百里,且戰且行,比至蔡,幾喪其半。上表異之,賜以大信牌,升為總帥。息州忠孝軍帥蔡八兒、王山兒亦來援。



 壬午,忠孝軍提
 控李德率十餘人乘馬入省大呼,以月糧不優,幾於罵詈。郎中移剌克忠白之仲德,仲德大怒,縛德堂下,杖之六十。上諭仲德曰:「此軍得力,方欲倚用,卿何不容忍,責罰乃爾。」仲德曰:「時方多故,錄功隱過,自陛下之德。至於將帥之職則不然,小犯則決,大犯則誅,強兵悍卒,不可使一日不在紀律。蓋小人之情縱則驕,驕則難制,睢陽之禍,豈獨官奴之罪,亦有司縱之太過也。今欲更易前轍,不宜愛克厥威,賞必由中,罰則臣任其責。」軍士聞之,至于國亡不敢有犯。



 九月,蔡城戒嚴。行六部尚書蒲察世達以大兵將至,請諭民併收晚田,不及者踐毀之,毋
 資敵,制可。丙辰,詔裁冗員,汰冗軍,及定官吏軍兵月俸,自宰執以下至于皁隸,人月支六斗。初,有司定減糧,人頗怨望。上聞之,欲分軍為三,上軍月給八斗,中七斗,下六斗,人復怨不均。乃立射格,而上中軍輒多受賞,連中者或面賜酒,人益為勸,且陰有所增而人不知,仲德之謀也。甲子,分軍防守四面。



 十月壬申朔,大兵壕壘成,耀兵城下,旗幟蔽天。城中駭懼,及暮,焚四關,夷其墻而退。十一月辛丑,大兵以攻具傅城,有司盡籍民丁防守,不足則括婦女壯健者,假男子衣冠使運木石。蔡既受圍,仲德營畫禦備,未嘗一至其家,拊存軍士,無不得其歡
 心,將校有戰亡者,親為賻祭,哭之盡哀。己丑,西城破,城中前期築柵浚濠為備,雖克之不能入也。但於城上立柵,南北相去百餘步而已。仲德摘三面精銳日夕戰禦,終不能拔。



 三年正月庚子朔,大兵以正旦會飲,鼓吹相接,城中飢窘,愁嘆而已。圍城以來,戰歿者四帥、三都尉,其餘總帥以下,不可勝紀。至是,盡出禁近,至於舍人、牌印、省部掾屬,亦皆供役。戊申,大兵鑿西城為五門,整軍以入,督軍鏖戰,及暮乃退,聲言來日復集。己酉,大兵果復來,仲德率精兵一千巷戰,自卯及巳,俄見子城火起,聞上自縊,謂將士曰:「吾君已崩,吾何以戰為?吾不能死
 於亂兵之手,吾赴汝水,從吾君矣。諸君其善為計。」言訖,赴水死。將士皆曰:「相公能死,吾輩獨不能耶?」於是參政孛術魯婁室、兀林答胡土,總帥元志,元帥王山兒、紇石烈柏壽、烏古論恆端及軍士五百餘人,皆從死焉。



 仲德狀貌不踰常人,平生喜怒未嘗妄發,聞人過,常護諱之。雖在軍旅,手不釋卷,門生故吏每以名分教之。家素貧,敝衣糲食,終其身晏如也。雅好賓客,及薦舉人材,人有寸長,極口稱道。其掌軍務,賞罰明信,號令嚴整,故所至軍民為用,至危急死生之際,無一士有異志者。南渡以後,將相文武,忠亮始終無瑕,仲德一人而已。



 贊曰:金之亡,不可謂無人才也。若完顏仲德、張天綱,豈非將相之器乎。昔者智伯死又無後,其臣豫讓不忘國士之報,君子謂其無所為而為之,真義士也。金亡矣,仲德、天綱諸臣不變所守,豈愧古義士哉!



\end{pinyinscope}