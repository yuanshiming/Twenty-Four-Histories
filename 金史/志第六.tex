\article{志第六}

\begin{pinyinscope}

 地理中



 ○南京路河北東路河北西路山東東路山東西路



 南京路,國初曰汴京,貞元元年更號南京。府三,領節鎮三,防禦八,刺史郡八,縣一百五。都城門十四,曰開陽,曰宣仁,曰安利,曰平化。曰通遠,曰宜照,曰利川,曰崇德,曰迎秋,曰廣澤,曰順義,曰迎朔,曰順常,曰廣智。宮城門,南外門曰南薰,南薰北新城門曰豐宜,橋曰龍津橋,北門曰丹鳳,其門三。丹鳳北曰舟橋,橋少北曰文武樓,遵禦路而北橫街也。東曰太廟,西曰郊社,正北曰承天門,其門五,雙闕前引,東曰登聞檢院,西曰登聞鼓院。檢院東曰左掖門,門南曰待漏院。鼓院西曰右掖門,門南曰都堂。直承天門北曰大慶門,門東曰日精門,又東曰左昇平
 門。大慶門西曰月華門,又西曰右昇平門。正殿曰大慶殿,前有龍墀,又南有丹墀,又南曰沙墀,東廡曰嘉福樓,西廡曰嘉瑞樓。大慶後曰德儀殿。殿東曰左升龍門,西曰右升龍門。正門曰隆德,內有隆德殿,有蕭牆,有丹墀。隆德殿左曰東上閣門,右曰西上閣門,皆南向。鼓樓在東,鐘樓在西。隆德之次曰仁安門、仁安殿,東則內侍局,又東曰近侍局,又東則嚴祗門,宮中則稱曰撒合門,少南曰東樓,則授除樓也。西曰西樓。仁安之次曰純和殿,正寢也。純和西曰雪香亭,亭北則後妃位也,有樓,樓西曰瓊香亭,亭西曰涼位,有樓,樓北少西曰玉清殿。純和之次曰福寧殿,殿後曰苑門,內曰仁智殿,有二太湖石,左曰敷錫神運萬歲峰,右曰玉京獨秀太平巖,殿曰山莊,其西南曰翠微閣。苑門東曰仙韶院,院北曰翠峰,峰之洞曰大滌湧翠,東連長生殿,又東曰湧金殿,又東曰蓬萊殿。長生西曰浮玉殿,又西曰瀛洲殿。長生殿南曰閱武殿,又南曰內藏庫。嚴祗門東曰尚食局,又東曰宣徽院,院北曰御藥院,又北右藏庫,東則左藏庫。宣徽院東曰點檢司,司北曰秘書監,又北曰學士院,又北曰諫院,又北曰武器署。點檢司南曰儀鸞局,又南曰尚輦局。宣徽院南曰拱衛司,又南曰尚衣局。其南為繁禧門,又南安泰
 門,門與左昇龍門相直。東則壽聖宮,兩宮太后位也,本明俊殿,試進士之所。宮北曰徽音院,又北燕壽殿,殿垣後少西曰振肅衛司,東曰中衛尉司。儀鸞局東曰小東華門,更漏在焉。中衛尉司東曰祗肅門,少東南曰將軍司。徽音、壽聖東曰太后苑,苑殿曰慶春,與燕壽殿並。小東華與正東華門對。東華門內正北尚廄局,其西北曰臨武殿。左掖門北,尚食局南曰宮苑司。其西北尚醖局、湯藥局。侍儀司少西曰府寶局、器物局,又西則撒合門也。嘉瑞樓西曰三廟,正殿曰德昌,東曰文昭,西曰光興。德昌後,宣宗廟也。宮西門曰西華,與東華相直,北門曰安貞。



 開封府,上。留守司留守帶本府尹,兼本路兵馬都總管。天德二年罷行臺尚書省,置轉運司、提刑司。天德二年置統軍司。有藥市四,榷場。產蜜蠟、香茶、心紅、硃紅、地龍、黃柏。天德四年,戶二十三萬五千八百九十。泰和末,戶百七十四萬
 六千二百一十。縣十五、鎮十五:



 開封東附郭。有古通津、臨蔡關、汴河。鎮一延嘉。



 祥符西附郭。有岳臺、浚水、沙臺、崇臺、夷門山、蔡河、金水河、廣濟河、寒泉河。鎮三陳橋、八角、郭橋。



 陽武有沙池、黑陽山。黃河、汴河、白溝河。



 通許宋名咸平,大定二十九年以與咸平府重,更。有牛首城、裘亭。



 泰康有魯溝、蔡河、渦河、鎮一崔橋。



 中牟有汴河、鄭河、中牟臺。鎮四圃田、陽武、萬勝、白沙鎮。



 杞宋雍丘縣,杞國也,正隆後更今名。鎮一圉城。



 鄢陵有洧水、潩水、太丘城。鎮一馬欄橋。



 尉氏有惠民河、長明溝。鎮二硃家曲、宋樓。



 扶溝有祁耶山、洧水、白亭。鎮二建雄、義店。舊有赤倉鎮。



 陳留有皇柏山、狼丘、汴河。



 延津貞祐三年七月升為延州。有土山、黃河。



 洧川貞祐二年置惠民倉,興定二年四月以尉氏縣之宋樓鎮升。



 長垣



 封丘



 睢州,下,刺史。宋拱州保慶軍,國初猶稱拱州,天德三
 年更。戶四萬六千三百六十。縣三、鎮一:



 襄邑古襄牛地。有汴河、睢水、渙水、承匡城。鎮一重華。



 考城宋隸東京,正隆前隸曹州,後來屬。有葵丘、黃河、黃陵岡,元光二年改為通安堡。



 柘城古株林,首止地在焉。有渙水、泡水、泓水。



 歸德府,散,中,宣武軍。故宋州,宋南京應天府河南郡歸德軍,國初置宣武軍。戶七萬六千百八十九。縣六、鎮四:



 睢陽宋名宋城,承安五年更名。有鷹鷺池、汴水、睢水、渙水。鎮一葛驛。



 寧陵大定二十二年徙於汴河堤南古城。有汴水、睢水、渙水。



 下邑有汴水、黃水。鎮一會亭。



 虞城有孟諸藪。



 穀熟有汴水、穀水。鎮二營城、洛場。又有舊高辛鎮。



 楚丘國初隸曹州,海陵後來屬,興定元年以限河不便,改隸單州。有景山、京岡。



 單州,中,刺史。宋碭郡,貞祐四年二月升為防禦,興定
 五年二月置招撫司,以安集河北遺黎。戶六萬五千五百四十五。縣四:



 單父有棲霞山、泡溝。



 成武有堂溝。



 魚臺有泗水,涓溝、五丈溝。



 碭山興定元年以限河不便,改隸歸德府。有芒碭山、古汴渠、午溝。



 壽州,下,刺史。宋隸壽春府,貞元元年來屬,泰和六年六月升為防禦。戶八千六百七十七。縣二、鎮一:



 下蔡有硤石山、潁水、淮水。



 蒙城宋隸亳州,國初來屬。有狼山、渦水。鎮一蒙館。



 陜州,下、防禦。宋陜郡保平軍節度,皇統二年降為防禦,貞祐二年七月升為節鎮。戶四萬一千一十。縣四、鎮七:



 陜倚。有虢山、峴頭山、三崤山、底柱山、黃河、橐水。鎮一石壕。



 靈寶有夸父山、黃河、稠桑澤、古函谷關。鎮二乾壕、關東。



 湖城有荊山、鑄鼎原、鳳林泉、鼎湖。鎮二
 三門、集津。



 閿鄉有太華山、黃河、玉澗水、潼關、太谷關。鎮二張店、故鎮。舊又曹張鎮,恐誤。



 鄧州,武勝軍節度使。宋南陽郡,嘗置榷場。戶二萬四千九百八十九。縣三、鎮六:



 穰城倚。有五壟山、覆釜山、湍水、朝水。鎮四順陽、新野、穰東、板橋。



 南陽有豫山、百重山、豐山、梅溪水、白水、清泠水。鎮一張村。



 內鄉有高前山、熊耳山、黃水、菊水、淅水、富水。鎮一峽口。



 唐州,中,刺史。宋準安郡,嘗置榷場。戶一萬一千三十一。縣四、鎮四:



 泌陽倚。有泌水、醴水。鎮一胡陽。



 比陽大明湖、中陽山、比水。鎮一羊棚。



 湖陽貞祐元年廢。鎮一羅渠。



 桐柏大定十年始置正官,興定五年六月廢。有桐柏山、淮水、柘河。鎮一許封。大定二十八年命規措界壕於唐、鄧間。



 裕州,本方城縣,泰和八年正月升置,以方城縣為倚
 郭,割汝州葉縣、許州舞是隸焉。戶八千三百。縣三、鎮四:



 方城倚。有方城山、衡山、堵水。鎮一青臺。



 葉隸汝州,泰和八年來屬。有方城山、石塘河、澧水。鎮一臨墳。



 舞陽本隸許州,泰和八年來屬。有伏牛山、馬鞍山、舞水、汝水、渫水、水雖水。鎮二吳城、北舞。



 河南府,散、中。宋西京河南雒陽郡。初置德昌軍,興定元年八月升為中京,府曰金昌。戶五萬五千六百三十五。縣九、《正隆郡志》有壽安縣,紀錄皆無。鎮四:



 洛陽倚。有北邙山,正隆六年更名太平山,稱舊名者以違制論。有伊、洛、璟、澗、金水,銅駝街、金粟山,金谷。鎮一龍門。



 澠池有天壇山、廣陽山、黃河、澠河。



 登封有太室山、箕山、陽城山、少室山,宣宗置御寨其上。舊有潁陽鎮,後廢。



 孟津貞祐三年七月升為陶州,十二月復為縣。鎮一長泉。舊有河清鎮,後廢。



 芝
 田宋名永安,貞元元年更。有轘轅山、青龍山。



 新安有闕門山、長石山、金水、穀水、陂水。



 偃師有北邙山、緱氏山、半石山、景山、黃河、洛水。鎮一緱氏。



 宜陽有錦屏山、鹿蹄山、憩鶴山、女几山、洛水,昌水,少水。



 鞏有侯山、九山、黃河、洛水。鎮一洛口。



 嵩州,中,刺史。舊名順州,天德三年更。戶二萬六千六百四十九。縣四、鎮四:



 伊陽宋隸河南府。有三塗山、陸渾山、鼓鐘山,伊水、淯陽水。鎮一鳴皋。舊有伊闕鎮,後廢。



 永寧宋隸河南府,正隆六年以前寄治於府,後即鎮為縣。有三肴山、熊耳山、嶕嶢山、天柱山、黃河、杜陽水。鎮一府店。



 福昌宋隸河南府。有女幾山、金門山。鎮二韓城、三鄉。



 長水宋隸河南府。有壇山、松陽山、洛水、松陽水。



 汝州,上,刺史。宋臨汝郡陸海軍度使,國初為刺郡,貞祐三年八月升為防禦。戶三萬五千二百五十四。
 縣四、鎮三:



 梁有霍陽山、崆峒山、紫邏山、汝水、廣潤河。正隆六年,敕環汝州百五十里內州縣商賈,赴溫湯置市。



 郟城宋隸許州。有汝水、扈澗河。鎮一黃道。



 魯山有堯山、水雖水、鴉河。



 寶豐有豢龍城。鎮一汝南。



 許州,下,昌武軍節度使。宋潁昌府許昌郡忠武軍。戶四萬五千五百八十七。縣五、鎮七:



 長社倚。有潩水、潁水。鎮二許田、椹澗。



 郾城有長沙河、五溝水。鎮二駝口、新寨。



 長葛有小陘、洧水。



 臨潁鎮二合流、繁城。



 襄城本隸汝州,泰和七年來屬。鎮一潁橋。



 鈞州,中,刺史。舊陽翟縣,偽齊升為潁順軍。大定二十二升為州,仍名潁順,二十四年更今名。戶一萬八千五百一十。縣二、鎮一:



 陽翟倚。有具茨山、三封山、荊山、潁水。



 新鄭宋隸鄭州。有溱、洧、潩三水。鎮一郭店。



 亳州,上,防禦使。宋譙郡集慶軍,隸揚州。貞祐三年升為節鎮,軍名集慶。戶六萬五百三十五。縣六、鎮五:舊有福寧、馬頭二鎮。



 譙倚。有渦水、泡水。鎮一雙溝。



 鹿邑有渦水、明水。鎮一戰城。



 衛真有洵水、沙水。鎮一穀陽。



 城父有渦水、淝水、殳水。



 酂有睢水、汴河、白龍潭。鎮一酂陽。



 永城興定五年十二月升為永州,以下邑、碭山、酂縣隸焉。有芒山、汴河。鎮一保安。



 陳州,下,防禦使。宋淮寧府淮陽郡鎮安軍。戶二萬六千一百四十五。縣五、鎮二:



 宛丘有蔡河、潁水、洧水。



 項城有潁水、百尺堰。



 南頓鎮一殄寇。



 商水本激水,宋避宣祖諱改。有商水、潁水。



 西華有宜陽山、蔡河、潁水。鎮一長
 平。



 蔡州,中,防禦使。宋汝南郡淮康軍,泰和八年升為節度,軍曰鎮南,嘗置榷場。戶三萬六千九十三。縣六、鎮二:



 汝陽有溱水、澺水。鎮一保城。



 遂平有吳房山、吳城山、龍泉水、瀙水。



 上蔡



 西平有九頭山、滾水、鄧艾陂。



 確山有確山、沒水、溱水。鎮一毛宗。



 平輿



 息州,本新息縣,泰和八年升為息州,以新息為倚郭,割真陽、褎信、新蔡隸焉,為蔡州支郡。戶九千六百八十五。縣四、鎮一:



 新息倚。鎮一王務。



 真陽本隸蔡州,泰和八年來屬。有淮水、汝水、石塘陂。



 褎信本隸蔡州,泰和八年來屬。有汝水、葛陂。



 新蔡本隸蔡州,泰和八年來
 屬。有汝水。



 鄭州,中,防禦,宋滎陽郡奉寧軍節度。戶四萬五千六百五十七。縣七、鎮三:



 管城倚。貞祐四年更名故市。有圃田澤。



 滎陽有鴻溝,京、索二水。



 密有大騩山、溱水、洧水。鎮二大騩、鎖水。



 河陰



 原武鎮一陳橋。



 汜水有虎牢關。



 滎澤有廣武澗。舊有許橋、賈谷二鎮,在鄭境。



 潁州,下,防禦。宋順昌府汝陰郡。嘗置榷場,正隆四年罷榷場。戶一萬六千七百一十四。縣四、鎮十一:舊有萬善鎮,後廢。



 汝陰倚。有潁水、淮水、淝水、汝水。



 潁上元光二年十一月改隸壽州。有潁水、淮水。鎮十永寧、漕口、王家市、櫟頭、永清、椒陂、正陽、江陂、界溝、斤溝。



 泰和有潁水。



 沈丘有武丘。鎮一永安。



 宿州,中,防禦。宋符離郡保靜軍節度,隸揚州。國初隸
 山東西路,大定六年來屬。貞祐三年陞為節鎮,軍曰保靜。戶五萬五千五十八。縣四、鎮八:舊有荊山鎮。



 符離倚。有諸陽山、汴河、睢水、陴湖。鎮三曲溝、符離、黃團。



 臨渙有嵇山、汴河、肥水。鎮三柳子、蘄澤、桐墟。



 靈壁宋元祐元年置。鎮一西固



 蘄有渙水、渦水、蘄水。鎮一靜安。



 泗州,中,防禦使。宋臨淮郡。正隆四年正月罷鳳翔府、唐、鄧、潁、蔡、鞏、洮等州并膠西縣諸榷場,但置榷場於泗州。先隸山東西路,大定六年來屬。戶八千九十二。縣四、鎮六:



 淮平舊盱眙縣,明昌六年以宋有盱眙軍,故更。



 虹有朱山、汴河、淮水、廣濟渠。鎮二千仙、通海。



 臨淮鎮四安河、吳城、青陽、翟家灣。



 睢寧興定二年四月以宿遷縣之古城置。又有淮濱,興定二年四月以桃園置,元光二年四月廢。



 邊戍,皇統元年十月,都元帥宗弼與宋約,以淮水中流為界,西自鄧州南四十里、西南四十里為界。泰和八年設沿淮巡檢使,及朐山縣完瀆村創立巡路,置巡檢。



 河北東路。天會七年析河北為東、西路,名置本路兵馬都總管。府一,領節鎮二,防禦一,刺郡五,縣三十,鎮三十五。



 河間府,中,總管府,瀛海軍。宋河間郡瀛海軍。天會七年置總管府,正隆間升為次府,置瀛州瀛海軍節度使兼總管,置轉運司。後復置總管府,河北東西大名
 等路提刑司產無縫綿、滄鹽、藺席、馬藺花、香附子、錢蝦蟹、乾魚。戶三萬一千六百九十一。縣二、鎮三:



 河間倚。有滹沱河、君子館。鎮三束城、永寧、北林。



 肅寧



 蠡州,下,刺史。宋永寧軍,國初因之,天會七年升為寧州博野郡軍,天德三年更為蠡州。戶二萬九千七百九十七。縣一、鎮一:



 博野倚。有沙河、唐河。鎮一新橋。



 莫州,下,刺史。宋文安郡軍防禦,治任丘。貞祐二年五月降為鄚亭縣。戶二萬二千九百三十三。縣一、鎮一:



 任丘鎮一長豐。



 獻州,下,刺史。本樂壽縣,天會七年升為壽州,天德三
 年更今名。戶五萬六百三十二。縣二、鎮十:



 樂壽倚。有徒駭河、房淵、漢獻王陵。



 交河大定七年以石家圈置。鎮十景城、南大樹、劉解、槐家、參軍、貫河、北望、夾灘、策河、沙渦。



 冀州,上。宋信都郡,天會七年仍舊置安武軍節度。戶三千六百七十。縣五、鎮三:



 信都倚。有胡盧河、降水。鎮一來遠,後廢。



 南宮有降枯瀆。鎮三唐陽、後增寧化、七公二鎮。



 衡水有長蘆河、降水。



 武邑有漳河、長蘆河。鎮一觀津,後廢。



 棗強鎮一廣川,後廢。



 深州,上刺史。宋饒陽郡防禦,因初為刺郡。戶五萬六千三百四十。縣五、鎮一:



 靜安倚。有衡漳水、大陸澤。鎮一下博。



 束鹿有衡漳水、滹沱河。



 武強置河倉。有衡漳水、武強泉。



 饒陽有滹沱河。



 安平有沙
 水、滹沱河。



 清州,中。宋乾寧郡軍,國初因置軍,天會七年以守邊置防禦。戶四萬七千八百七十五。縣三、鎮一:



 會川本名乾寧,貞元元年更名。置河倉。鎮一範橋。



 興濟本隸滄州,大定六年來屬。



 靖海明昌四年以清州窩子口置。



 滄州,上,橫海軍節度。宋景城郡。貞元二年來屬。戶一十萬四千七百七十四。縣五、鎮十一:



 清池置河倉。有浮陽水、徒駭河鎮五長蘆、舊饒安、乾符、郭畽。舊有郭僑,後廢。



 無棣有老烏山、鬲津河。鎮一分水。



 鹽山有鹽山、浮水。鎮四海豐、海潤,後增利豐、撲頭二鎮。



 南皮置河倉。有大、小台山、永濟渠、潔河。鎮一馬明。



 樂陵有鬲津河、篤馬河、鉤盤河、舊有會寧河、永利、東
 中三鎮、後廢。



 景州,上,刺史。宋永靜軍同下州,治東光。國初陞為景州,貞元二年來屬。大安間更為觀州,避章廟諱也。戶六萬五千八百二十八。縣六、鎮四:



 東光倚。置河倉。有永濟渠、漳河。鎮一建橋。



 阜城有衡水、漳水河。劉豫祖塋在縣南十二里。



 將陵置河倉。有永濟渠、鉤盤河。



 吳橋有永濟渠。



 蓚宋隸冀州。有漳河、蓚市。



 寧津鎮三西保安、廣平、會津。



 河北西路。天會七年析為西路。府三,領節鎮二,防禦二,刺郡五,縣六十一。



 真定府,上,總管府,成德軍。宋常山郡鎮州成德軍節
 度,正隆間依舊次府,置本路兵馬都總管府、轉運司。產瓷器、銅、鐵。有丹粉場、烏梨。藥則有茴香、零陵香、御米殼、天南星、皁角、木瓜、芎、井泉石。戶一十三萬七千一百三十七。縣九、鎮三:



 真定倚。有大茂山、滋水、滹沱水。



 槁城有滋水、滹沱水。



 平山



 欒城有泜水、洨水。



 獲鹿興定三年三月升為鎮寧州,權河北西路,以經略使武仙駐焉。有萆山、滹沱水。



 行唐有玉女山、常山。鎮二嘉祐、北鎮。舊有行臺、新年二鎮,後廢。



 阜平明昌四年以北鎮置。



 靈壽鎮一慈谷。



 元氏有封龍山、槐河。



 威州,下,刺史。天會七年以井陘縣升,置陘山郡軍,後為刺郡。戶八千三百一十。縣一:



 井陘



 沃州,上,刺史。宋徽宗升為慶源府趙郡慶源軍,治平
 棘。天會七年改為趙州,天德三年更為沃州,蓋取水沃火之義,軍曰趙郡軍。後廢軍。戶三萬八千一百八十五。縣七、鎮一:



 平棘倚。有洨水、槐水。



 臨城有敦輿山、彭山、泜水。



 高邑有贊皇山、濟水。



 贊皇



 寧晉有洨水、寢水。鎮一奉城。



 柏鄉



 隆平



 邢州,上,安國軍節度。宋信德府鉅鹿郡安國軍節度,天會七年降為邢州,仍置安國軍節度。產玄精石。戶八萬二百九十二。縣八、鎮四:



 邢臺有石門山、百巖山、蓼水、渦水。



 唐山有堯山、泜水。



 內丘有乾言山、內丘山、泜水、渚水。



 平鄉鎮一道武。



 任有溹水、任水。鎮一新店。



 沙河有湯水、湡水。鎮一綦村。



 南和有任水、泜水。



 鉅
 鹿有大陸澤、漳河、落漠水。鎮一團城。



 洺州,上,防禦,廣平郡。治永年。天會七年以守邊置防禦使。戶七萬三千七十。縣九、鎮四:



 永年有榆溪山、洺水、漳水。鎮一西臨洺。



 廣平本魏縣,大定七年更。



 宗城



 新安



 成安



 肥鄉鎮一新安。



 雞澤有洺水、漳水、沙河。



 曲周鎮二平恩、白家灘。



 洺水



 彰德府,散,下。宋相州鄴郡彰德軍節度,治安陽。天會七年仍置彰軍節度,明昌三年升為府,以軍為名。戶七萬七千二百七十六。縣五、鎮五。



 安陽倚。有韓陵山、龍山、洹水、防水。鎮三天祐、永和、豐樂。



 林盧舊林盧鎮,貞祐三年十月升為林州,置元
 帥府。興定三年九月升為節鎮,以安陽縣水治村為輔巖縣隸焉。有隆盧山、洹水、漳水。



 湯陰有牟山、羑水、蕩水、通漕、羑里。鎮一鶴壁。



 臨漳東山、漳水。鎮一鄴鎮。



 輔巖本水治村,興定三年置。



 磁州,中,刺史。宋滏陽郡,國初置滏陽郡軍。戶六萬三千四百一十七。縣三、鎮八:



 滏陽有滏山、磁山、漳水、滏水。鎮四臺城、觀城、昭德,後廢二祖增臨水鎮。



 武安有錫山、武安山鎮一固鎮。



 邯鄲有邯山、靈山、漳水、牛首山。鎮三大趙、北陽、邑城。《士民須知》惟有邯山鎮。



 中山府。宋府,天會七年降為定州博陵郡定武軍節度使,後復為府。戶八萬三千四百九十。縣七、鎮二:



 安喜倚。有滱水、盧奴水、長星川。



 新樂有派水、木刀溝。



 無極有澬河。



 永平
 貞祐二年四月升為完州。



 慶都有堯山、都山、唐水。



 曲陽劇。有常山、曲防水。鎮一龍泉。



 唐有孤山、唐山、滱水。鎮一軍城。



 祁州,中,刺史。宋蒲陰郡,國初置蒲陰郡軍。戶二萬三千三百八十二。縣三:



 蒲陰



 鼓城



 深澤



 濬州,中,防禦。宋大邳郡通利軍,又改平川軍。天會七年以邊境置防禦使。皇統八年,嫌與宗峻音同。更為通州,天德三年復。戶二萬九千三百一十九。縣二、鎮二:



 黎陽有大伾山、枉人山。



 衛有蘇門山、鹿臺、糟丘、酒池、枋頭城。鎮二衛橋、淇門。



 衛州,下,河平軍節度。守汲郡,天會七年因宋置防禦
 使,明昌三年升為河平軍節度,治汲縣、以滑州為支郡。大定二十六年八月以避河患,徙於共城。二十八年復舊治。貞祐二年七月城宜村,三年五月徙治于宜村新城,以胙城為倚郭。正大八年以石甃其城。戶九萬一百一十二。縣四、鎮二



 汲有蒼山、黃河。



 新鄉



 蘇門本共城,大定二十九年改為河平,避顯宗諱也。明昌三年改為今名。貞祐三年九月升為輝州,興定四年置山陽縣隸焉。有白鹿山、天門山、淇水、百門陂。鎮一早生。



 獲嘉鎮一大寧。



 胙城本隸南京,海陵時割隸滑州,泰和七年復隸南京,八年以限河來屬。貞祐五年五月為衛州倚郭。增置主薄。興定四年以修武縣重泉村置縣,來隸。



 滑州,下,刺史。宋靈河郡武成軍。本南京屬郡,大定六
 年割隸大名府。戶二萬二千五百七十。縣二、鎮二:



 白馬鎮二衛南、武城。



 內黃本隸大名府,大定六年來屬。



 山東東路,宋為京東東路,治益都。府二,領節鎮二,防禦二,刺郡七,縣五十三,鎮八十三。



 益都府,上,總管府,宋鎮海軍,國初仍舊置軍,置南青州節度使,後升為總管府,置轉運司。大定八年置山東東西路統軍司。產石器、玉石、沙魚皮、天南星、半夏、澤瀉、紫草。戶一十一萬八千七百一十八。縣七、鎮七:



 益都



 臨朐有朐山、几山、洱水、般水。



 穆陵貞祐四年四月升臨朐之穆陵置。



 壽光有甘水、澠水。鎮一廣陵,有鹽場。



 博興有濟水、時水。鎮二博昌、淳化



 臨淄有南郊山、牛山、天齊淵、康浪水。



 樂安鎮四新鎮、高家港、清河、王家。



 濰州,中,刺史。戶三萬九百八十九。縣三、鎮一:



 北海倚。有浮煙山、溉原山、溉水、汶水。鎮一固底。



 昌邑有霍侯山、濰水。



 昌樂有方山、聚角山、丹水、朐水。



 濱州,中,刺史。宋軍事。戶一十一萬八千五百八十九。縣四、鎮十:



 渤海有黃河。鎮五豐國、寧海、濱海、蒲臺、安平。



 利津明昌三年十二年以永和鎮升置。



 蒲臺鎮二安定、合波。



 霑化本名招安,明昌六年更。鎮三永豐、永阜、永科。



 沂州,上,防禦。宋琅邪郡。戶二萬四千三十五。縣二、鎮三:



 臨沂劇。鎮三長任、向城、利城。



 費



 密州,宋為密州高密郡安化節度。戶一萬一千八十二。縣四、鎮七:



 諸城劇。有瑯邪山、濰水、荊水、盧水。鎮三普慶、信陽、草橋。



 安丘有安丘山、劉山、汶、濰、浯水。鎮一李文。



 高密有礪阜山、密水、膠水。



 膠西鎮三張倉、梁鄉、陳村。



 海州,中,刺史。戶三萬六百九十一。縣五、鎮四:



 岣山。



 贛榆本懷仁,大定七年更。鎮二荻水、臨洪。



 東海



 漣水本漣水軍,皇統二年降為縣來屬。鎮二太平、金城。



 莒州,中,刺史。本城陽軍,大定二十二年升為城陽州,二十四年更今名。戶四萬三千二百四十。縣三、鎮三:



 莒



 日照鎮一濤洛



 沂水鎮一沂安。舊有扶溝、洛鎮
 二鎮,後廢。



 棣州,上,防禦。宋安樂郡。戶八萬二千三百三。縣三、鎮九:



 厭次鎮五清河、歸化、達多、永利、脂角。



 陽信有黃河、鉤盤河。鎮二欽風、西界。



 商河有黃河、馬頰河、商河。鎮二歸仁、官口。



 濟南府,散,上。宋齊州濟南郡。初置興德軍節度使,後置尹,置山東東西路提刑司。戶三十萬八千四百六十九。縣七、鎮二十九:



 歷城鎮六盤水、中宮、老僧口、上洛口、王舍人店、遙墻。



 臨邑鎮三新鎮、安肅、新市。



 齊河鎮三晏城、劉宏、新孫耿。



 章丘有長白山、東陵山、百脈水、楊緒水。鎮四普濟、延安、臨濟、明水。



 禹城有黃河、濟河、淇河、水水。鎮三新安、仁水寨、黎濟寨。



 長清劇。有磨笄山、隔馬山、黃河、清水。鎮六
 赤莊、莒鎮、李家莊、歸德、豐濟、陰河。



 濟陽鎮四回河、曲堤、舊孫耿、仁豐。



 淄州,中,刺史。宋淄川郡軍。戶一十二萬八千六百二十二。縣四、鎮六:



 淄川倚。有幹山、夾谷山、商山、淄水。鎮三金嶺、張店、顏神店。



 長山有長白山、慄水。



 鄒平有系河、濟河。鎮三淄鄉、齊東、孫家嶺。舊有緌店鎮,後廢。



 高苑有濟河。



 萊州,上,定海軍節度。宋軍萊郡。戶八萬六千六百一十五。縣五、鎮一:



 掖倚。有三山、夜居山、掖水。



 萊陽有高麗山、七子山。鎮一衡村。舊有海倉、西由、移風三鎮。



 即墨有牢山、不其山、天室山、沽水、曲裏鹽場。



 膠水



 招遠



 登州,中,刺史。宋東牟郡。戶五萬五千九百一十三。
 縣四、鎮二:



 蓬萊有巨風鹽場。



 福山鎮一孫大川。



 黃有萊山、蹲狗山。鎮一馬停。



 棲霞



 寧海州,刺史。本寧海軍,大定二十二年升為州。戶六萬一千九百三十三。縣二、鎮二:



 牟平有東牟山、之罘山、清陽水。鎮一湯泉。



 文登劇。有文登山、成山、昌陽山。鎮一溫水。



 山東西路,府一,領節鎮二,防禦二,刺郡五。



 東平府,上,天平軍節度。宋東平郡,舊鄆州,後以府尹兼總管,置轉運司。產天麻、全蠍、阿膠、薄荷、防風、絲、綿、綾、錦、絹。戶一十萬八千四十六。縣六、鎮十九:



 須城有梁山、濟水、清河。



 東
 阿有吾山、穀城山、黃河、阿井。鎮五景德、木仁、關山、銅城、陽劉。



 陽穀有黃河、碻磝津。鎮二樂安、定水。



 汶上本名中都、貞元元年更為汶陽,泰和八年更今名。有汶水,大野陂。鎮一柴城。



 壽張大定七年河水壞城,遷於竹口鎮,十九年復舊治鎮一竹口。



 平陰有鬱蔥山、鴟夷山。鎮九但歡、安寧、寧鄉、翔鸞、固留、滑口、廣里、石橫、澄空、傅家岸。



 濟州,中刺史。宋濟陽郡。舊治鉅野,天德二年徙治任城縣,分鉅野之民隸嘉祥、鄆城、金鄉三縣。戶四萬四百八十四。縣四、鎮二:



 任城倚。有承注山、泗水、新河。鎮一魯橋。



 金鄉有桓溝。鎮一昌邑。



 嘉祥舊有合來、山口二鎮,後廢。



 鄆城大定六年五月徙治盤溝村以避河決。有馬頰河、濮水。



 徐州,下,武寧軍節度使。宋彭城郡,貞祐三年九月改
 隸河南路。戶四萬四千六百八十九。縣三、鎮五:



 彭城倚。有九里山、赭土山、泗水、猴水、沛澤。鎮三呂梁、利國、卞唐。又有厥堌鎮,元光二年升為永固縣。



 蕭有綏輿山、丁公山、古汴渠。鎮二白土、安民。舊有晉城、雙溝二鎮。



 豐有泡水、大澤。



 邳州,中,刺史。宋淮陽軍,貞祐三年九月改隸河南路。戶二萬七千二百三十二。縣三:



 下邳有嶧陽山,磬石山、艾山、沂水、泗水、沭水、睢水。



 蘭陵本承縣,明昌六年更名。貞祐四年三月徙治土婁村。



 宿遷元光二年四月廢。有泗水、汜水。



 滕州,上,刺史。本宋滕陽軍,大定二十二年升為滕陽州,二十四年更今名。貞祐三年九月為兗州支郡。戶四萬九千九。縣三、鎮一:



 滕舊名滕陽,大定二十四年更。有桃
 山、抱犒山、漷水。



 沛有微山、泗水、泡水、漷水。鎮一陶陽。



 鄒宋隸泰寧軍。有嶧山、鳧山、泗水、漷水。



 博州,上防禦。宋博平郡。戶八萬八千四十六。縣五、鎮十一:



 聊城倚。有茌山、黃河、金沙水。鎮二王館、武水。



 堂邑鎮二回河、侯固。



 博平有漯河。鎮一博平。



 茌平鎮二廣平、興利。



 高唐有黃河、鳴犢溝。鎮四固河、齊城、靈城、夾灘。



 兗州,中,泰軍節度使。宋襲慶府魯郡。舊名泰寧軍,大定十九年更。戶五萬九十九。縣四:



 嵫陽本瑕丘。



 曲阜宋名仙源。有防山、曲阜山,泗、洙、沂水。



 泗水有陪尾山、尼丘山、泗水、洙水。



 寧陽舊名龔縣,大定二十九年以避顯宗諱改。



 泰安州,上,刺史。本泰安軍,大定二十二年升。戶三萬
 一千四百三十五。縣三、鎮二:



 奉符倚。有泰山,社首山、龜山、徂徠山、亭亭山。有汶水、梁水。鎮二太平、靜封。



 萊蕪有肅然山、安期山、贏汶水、牟汶水。



 新泰



 德州,上,防禦。宋平原郡軍。戶一萬五千五十三。縣三、鎮七:



 安德有鬲津河。鎮四磁博、嚮化、盤河、德安。



 平原有金河。鎮一水務。



 德平鎮二懷仁、孔家鎮。



 曹州,中,刺史。宋興仁府濟陰郡彰信軍。本隸南京,泰和八年來屬。大定八年城為河所沒,遷州治于古乘氏縣。戶一萬二千六百七十七。縣三、鎮一:



 濟陰倚。有曹南山、定濮岡、左山、祝丘、荷水、氾水、饗城、鄸城。鎮一濮水。


定陶本宋廣濟軍,熙寧間
 廢為定陶縣。城中有梁王臺。有
 \gezhu{
  髟方}
 山、獨孤山。



 東明初隸南京,後避河患,徙河北冤句故地。後以故縣為蘭陽、儀封,有舊東明城。



\end{pinyinscope}