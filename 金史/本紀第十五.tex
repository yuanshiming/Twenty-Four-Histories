\article{本紀第十五}

\begin{pinyinscope}

 宣宗中



 興定元年春正月己卯朔,宋遣使來賀。癸未,宋使朝辭。上謂宰臣曰:「聞息州南境有盜,此乃彼界饑民沿淮為亂耳。宋人何故攻我。」高琪請伐之,以廣疆土。上曰:「朕意不然,但能守祖宗所付足矣,安事外討?」乙未,詔中都、西京、北京等路策論進士及武舉人權試于南京、東平、婆速、上京等四路。丙申,東平行省言:「調兵以來,吏卒因勞
 進爵多至五品,例獲封贈,及民年七十並該覃恩。若人往自陳,公私俱費。請令本路為製誥敕,類赴朝廷,以求印署。使受命者量輸諸物而給之。人力不勞,兵食少濟。」從之。皇子平章政事濮王守純授世襲東平府路三屯猛安。尚書左丞胥鼎進平章政事,封莘國公。癸卯,議減庶官冗員。乙巳,大元兵攻觀州。



 二月戊申朔,初用「貞祐通寶」,凡一貫當「貞祐寶券」千貫。己酉,命樞密院汰罷軟軍士。諭尚書者,用官馬給驛傳以紓民力。庚戌,皇后生辰,詔百官免賀,仍諭旨曰:「時方多難,將來長春節亦免賀禮。」辛亥,以崇進、元帥右都監完顏賽不簽樞密院事。
 癸丑,罷招賢所。乙卯,皇孫生,宣徽請稱賀,詔無用樂。己未,大元兵徇忻、代。詔定州、縣官雖積階至三品,坐乏軍儲者,聽行部決遣。壬戌,尚書省以軍儲不繼,請罷州府學生廩給。上曰:「自古文武並用,向在中都,設學養士猶未嘗發,況今日乎?其令仍舊給之。」丙子,議置莊獻太子廟。



 三月戊寅,敕事關刑名,當面議之,勿聽轉奏。以絳陽軍節度使李革知平陽府,兼河東南路兵都總管,權參知政事,行尚書省。壬午,定民間收潰軍亡馬之法,及以馬送官酬直之格。乙酉,上宮中見蝗,遣官分道督捕,仍戒其勿以苛暴擾民。庚寅,長春節,宋遣使來賀。辛卯,
 詔罷平陽、河中元帥。乙未,先征山東兵接應苗道潤共復中都,而石海據真定叛,慮為所梗,乃集粘割貞、郭文振、武仙所部精銳與東平軍為掎角之勢,圖之。己亥,大元兵攻新城。庚子,攻霸州。甲辰,威州刺史武仙率兵斬石海及其黨二百餘人,降葛仲、趙林、張立等軍,盡獲海僭擬之物。尋進仙權知真定府事。



 夏四月丁未朔,以宋歲幣不至,命烏古論慶壽、完顏賽不等經略南邊。戊申,孟州經略司萬戶宋子玉率所部叛,斬關而出,經略使從坦等追敗之。庚戌,花帽軍作亂于滕州,詔山東行省討之。南陽五朵山盜發,眾至千餘人,節度副使移剌羊
 哥出討,遇之方城,招之不從,乃進擊之,殺其眾殆盡。癸丑,以安化軍節度使完顏宇權元帥左都監,行元帥府事,督經略使苗道潤進復都城,且令和輯河間招撫使移剌鐵哥等軍。鐵哥與道潤不協,互言共有異志,故命重臣臨鎮之。戊午,單州雨雹傷稼,詔遣官勸諭農民改蒔秋田,官給其種。平定州賊閻德用之黨閻顯殺德用,以其眾降。己未,以權參知政事遼東路行省完顏阿里不孫為參知政事,行尚書省、元帥府于婆速路。以權遼東路宣撫使蒲察五斤權參知政事,行尚書省、元帥府于上京。庚申,李革請罷義軍總領使副,以畀州縣。尚書
 省以秋防在邇,改法非便,姑如舊制,州縣各司察之。甲子,元帥守顏賽不破宋兵于信陽,使來奏捷。乙丑,濟南、泰安、滕、兗等州賊並起,侯摯遣棣州防禦使完顏霆討平之,降其壯士二萬人、老幼五萬人。完顏賽不復奏敗宋軍于隴山等處,俘馘甚眾。戊辰,太白晝見于井。辛未,權孟州經略使從坦追賊宋子玉至輝州境上,其黨邢福殺子玉,以眾來歸。壬申,以萬奴叛逆未殄,詔諭遼東諸將。完顏賽不軍渡淮,破光州兩關,獲軍實分給將士。



 五月戊寅,陜西行省破夏人于大北岔,是日捷至。丁亥,民苑汝濟上書陳利害,上以示宰臣曰:「卑賤小人,猶能
 盡言如此,有可采者即行之。」己丑,賊宋子玉餘黨家屬悉放歸農。尚書右丞蒲察移剌都棄官擅赴京師,降知河南府事,行樞密院兼行六部事。壬辰,延州原武縣雨雹傷稼,詔官貸民種改蒔。癸巳,宋人攻潁州,焚掠而去。戊戌,行樞密院兵敗宋人於泥河灣,又敗之樊城縣。山東行元帥府事蒙古綱擅械轉運使李秉鈞,法當決,秉鈞返詈綱,應論贖,詔兩釋之。宋人取漣水縣。癸卯,蘭州水軍千戶李平等苦提控蒲察燕京貪暴,殺之。構夏人以叛,脅其徒張扆俱行,扆以計盡獲之。陜西行省便宜遷扆官四階,授同知蘭州事,賞士卒有差,以其事上聞。甲
 辰,大元兵下沔城縣,軍官任福死之。丙午,定河北求仕官渡河之法,曾經總兵者白樞密院,餘驗據聽渡。行樞密院事烏古論慶壽南伐還,奏不以實,詔鞫之。



 六月己酉,苗道潤表歸國人李琛復以眾叛,琛亦表道潤異謀,詔山東行省察之。脩潼關,遣中使持詔及署藥勞夫匠。權參知政事張行信進參知政事。庚戌,詔捕治遼東受偽署官家屬,得按察使高禮妻子,皆戮之。壬子,制鄜、坊、丹州四品以下州縣官視環、慶例,以二十月終更。甲寅,招撫使惟宏言彰德府守臣擅徙民山砦避兵,上曰:「難保之城,守之何益,徒傷吾民耳。勿治。」乙卯,顯宗忌日,謁
 奠于啟慶宮。丙辰,詔樞密院遣經歷官分諭行院,嚴兵利器以守衝要,仍禁飲宴,違以軍律論。宋人合土寇攻東海境。戊午,以宋遣兵數犯境,及歲幣不至,詔諭沿邊罪宋。己未,詔凡上書人其言已采用者,上其姓名。辛酉,以進士朱蓋、草澤人李維巖論議可取,詔給八貫石俸。乙丑,設潼關使、副,及三門、集津提舉官。尚書左丞相兼都元帥僕散端薨,輟朝。置南京流泉務。遼東行省遣使來上正月中敗契丹之捷。



 秋七月丙子朔,日有食之。辛巳,宋人圍泗州。壬午,圍靈璧縣。癸未,庾州振威軍萬戶馬寬逐其刺史李策,據城叛。遣使招之,乃降。已而復謀
 變,州吏擒戮之,夷其族。甲申,詔諭遼東諸路。乙酉,宋人襲破東海縣。丙申,置提舉倉場使、副。癸卯,太社壇產嘉禾,一莖十有五穗。甲辰,夏人犯黃鶴岔,官軍敗之。乙巳,初置集賢院知院事、同知院事等官。宋人及土寇攻海州,經略使擊破其眾。夏人圍羊狼寨,帥府發諸鎮兵擊走之。



 八月戊申,陜西行省報木波賊犯洮州敗績,遁去。木星晝見于昴,六十有七日乃伏。己酉,海州經略司表官軍與宋人戰石湫南,戰漣水縣,戰中土橋,宋兵敗績。壬子,削御史大夫永錫官爵,有司論失律當斬,上以近族,特貰其死。癸丑,宋人攻確山縣,為官軍所敗,詔諭國
 內軍士,使知宋人渝盟之故,仍命大臣議其事。乙卯,集賢院諮議官硃蓋上書陳禦敵三策。壬戌,海州經略使阿不罕奴失剌敗宋人于其境。提控李元與宋人戰,屢捷,多所俘獲。徙欄通渡經略司於黃陵崗。乙丑,制增定擒捕逃軍賞格及居停人罪。丙寅,左司諫僕散毅夫乞更開封府號,賜美名,以尉氏縣為刺郡,睢州為防禦使,與鄭、延二州左右前後輔京師。上曰:「山陵在中都,朕豈樂久居此乎?」遂止。癸酉,太祖忌日,謁奠于啟慶宮。甲戌,元帥左都監承裔遣其部將納蘭記僧等,合葩俄族都管尼旁古,以兵掩襲瓜黎餘族諸蕃帳,屢破之,斬馘士
 卒,禽其首領,俘獲人畜甚多,是日捷至。



 九月丁丑,更定監察御史失察法。以元帥左監軍必蘭阿魯帶權參知政事,行省于益都。戊寅,夏人犯綏德之克戎寨,都統羅世暉逆擊,卻之。己卯,蔡州帥府偵宋人將窺息州,以輕兵誘其進,別以銳師邀擊之,虜其將沈俊。壬午,以改元興定,赦國內。甲申,罷規運所,設行六部。辛卯,大元兵徇隰州及汾西縣,癸巳,攻沁州。遼東行省完顏阿里不孫為叛人伯德胡土所殺。月犯東井西扇北第二星。乙未,大元兵攻太原簸箕掌寨。丁酉,薄太原城,攻交城、清源。癸卯,立沿河冰墻鹿角。



 冬十月丁未,以霖雨,詔寬農民
 輸稅之限。庚戌,以將有事于宋,詔帥臣整厲師徒。辛亥,遣官括市民馬,紅賞格以示勸。甲寅,命高汝礪、張行簡同脩《章宗實錄》。息州帥府獻破宋人于中渡之捷。乙卯,大元兵徇中山府及新樂縣。丙辰,丹州進嘉禾,異畝同穎。辛酉,制定州府司縣官失覺姦細罪。壬戌,右司諫兼侍御史許古上疏,請先遣使與宋議和。乙丑,大元兵下磁州。丙寅,定職官不求仕及規避不赴任法。高汝礪上疏言,和議先發於我,恐自示弱,非便。戊辰,上命許古草通宋議和牒,既進以示宰臣,宰臣以其言有祈哀之意,徒示微弱,無足取者,議遂寢。辛未,罷流泉務。大元兵收
 鄒平、長山及淄州。壬申,改郇國號為管,避上嫌名。高汝礪表致仕,不允。



 十一月壬午,從宜移剌買奴言:「五朵山賊魚張二等,若悉誅之,屢詔免罪,恐乖恩信。且其親屬淪落宋境,近在均州,或相構亂。乞貸其死,徙之歸德、睢、陳、鈞、許間為便。」詔許之。癸未,月暈木、火二星,木在胃,火在昴。丙戌,太白晝見,遣翰林侍講學士楊雲翼鋋之。大元兵收山東濱、棣、博三州,己丑,下淄州。庚寅,下沂州。甲午,河西掬納、篯納等族千餘戶來歸。丁酉,詔唐、鄧、蔡州行元帥府舉兵伐宋。戊戌,大元兵攻太原府。庚子,上謂宰臣曰:「朕聞百姓流亡,逋賦皆配見戶,人何以堪?又添征軍須
 錢太多,亡者詎肯復業,其並讓除之。」宰臣請命行部官閱實蠲貸,已代納者以恩例,或除他役,或減本戶雜征四之一。上曰:「朕於此事未嘗去懷,其亟行之。」



 十二月甲辰朔,大元兵攻潞州,都統馬甫死之。戊申,即墨移風砦於大舶中得日本國太宰府民七十二人,因糴遇風,飄至中國。有司覆驗無他,詔給以糧,俾還本國。庚戌,元帥左監軍蒲察五斤進右副元帥,權參知政事,充遼東行省。是日,大元兵平益都府。辛亥,陜西行省胥鼎諫伐宋,不報。甲寅,海州經略使報提控韓璧敗宋人於鹽倉。己未,大元兵復攻沂州,官民棄城遁。辛酉,下密州,節度
 使完顏宇死之。壬戌,侯摯兼三司使。庚午,免逃戶復業者差賦。



 二年春正月乙亥,詔議賑恤。辛巳,敕南征將帥所至毋縱殺掠。壬午,宋人攻淮北,唐州元帥府擊敗之,獲統領李雄韜、陳皋以歸。癸未,近侍局副使訛可遣使報南師之捷。乙酉,陜西行省獲歸國人,言大元兵圍夏王城,李遵頊命其子居守而出走西涼。詔諭諸帥府明斥候,嚴守備。戊子,唐、鄧元帥完顏賽不報連破宋人之捷。宋人攻泗州,又戰卻之。



 二月癸卯,宋人侵青口,行樞密院遣兵敗之。甲辰,免中京、嵩、汝等州逋租。諭胥鼎,克宋散關,
 可保則保,不可保則焚毀而還。定奴婢救主法。丙午,訛可敗宋人於防山。紇石烈桓端亦遣使來上光州、信陽之捷。庚戌,海州經略敗宋兵于朐山,表請繼其軍儲,督東平帥府發兵護送資糧以應之。許州長社縣何冕等謀反,伏誅。辛亥,張行信出為彰化軍節度使兼涇州管內觀察使。壬子,御史以北兵退,請汰各處行樞密院、元帥府冗官。尚書以為非便,上從尚書言,仍舊制。完顏賽不報棗陽之捷。癸丑,完顏阿鄰報皂郊堡之捷。丁巳,壽州行樞密院破宋人高柳橋水砦,夷其砦而還。壬戌,訛可遣兵拔宋柵棋盤嶺,又破其眾於裴家莊、寒山嶺、龍
 門關等處,得粟二千餘石。乙丑,諭樞密曰:「中京商、虢諸州軍人願耕屯田,比括地授之。聞徐、宿軍獨不願受,意謂予田必絕其廩給也。朕肯爾耶?其以朕意曉之。」丙寅,諭尚書省曰:「聞中都納粟官多為吏部繳駁,殊不思方闕乏時,利害為如何。又立功戰陣人,必責保官,若輩皆義軍白丁,豈識職官,茍文牒可信,即當與之。至若在都時,規運薪炭入城者,朕嘗植恩授以官。此豈容偽,而間亦為所沮格。其悉諭之,勿復若是。」紇石烈牙吾塔破宋人盱眙軍,上俘獲之數。己巳,以侯摯行省河北,兼行三司安撫司事。



 三月庚辰,尚書集文資官雜議進士之
 選,詔依泰和例行之。癸未,訛可敗宋人於光化軍。甲申,長春節。戊子,諭宰臣曰:「舊制,廷試進士日晡後出宮。近欲復舊,恐能文而思遲者,不得盡其才,其令日沒乃出。」以御史中丞把胡魯為參知政事。陜西行六部尚書楊貞削五官,累杖一百七十,解職。訛可表言,官軍自桐柏入宋境,所向多克捷。癸巳,宋人爭皁郊堡,擊官軍,軍潰,主將完顏阿鄰戰沒。丙申,更定京城捕告強盜官賞制。辛丑,上京行省蒲察五斤表,左監軍哥不靄誣坊州宣撫副使紇石烈按敦將叛而殺之。事下尚書省,宰臣以為按敦之死徐議恤典,哥不靄亦姑牢籠使之,上勉從
 其言。



 夏四月壬寅朔,蒲察五斤表,遼東便宜阿里不孫貸糧高麗不應,輒以兵掠其境。上命五斤遣人以詔往諭高麗,使知興兵非上國意。乙巳,詔河南路行總管府節鎮以上官,充宣差捕盜使,以防禦刺史以上長貳官,及世襲猛安之才武者為之副,又命濮王府尉完顏毛良虎為宣差提控以巡督之。是日,曲赦遼東等路。以戶部尚書夾谷必蘭為翰林學士承旨,權參知政事,行省于遼東。丁未,承裔敗宋人於皁郊堡。庚戌,御史劾集賢院諮議官李維巖本中山府無極縣進士趙孝選家奴,乞正其事。上曰:「國家用人,奚擇貴賤?」命以官銀五十兩
 贖放為良,任使仍舊。壬子,遣侍御史完顏素蘭、近侍局副使訛可同赴遼東,察訪叛賊萬奴事體。行省侯摯督兵復密州。提控朱琛復高密縣。癸丑,完顏素蘭請宣諭高麗復開互市,從之。乙卯,特賜武舉溫迪罕繳住以下一百四十人及第。丁巳,陜西行省破宋雞公山,取和州、成州,至河池縣黑谷關,守者皆遁,前後獲糧九萬斛,錢數千萬,軍實不可勝計。戊午,紅襖賊犯徐、邳,行樞密院兵大破之。己未,阿里不孫自潼關之敗,失其所在,變姓名匿居柘城,為御史覺察,繫其家屬,將窮治之,乃遣子上書詣吏待罪。臺臣力請誅之,以懲不忠。上卒赦其
 罪,諭以自效。癸亥,遣重臣審理京師冤獄。丁卯,河南諸郡蝗。臨洮路報敗宋人之捷。東平行省敗黑旗賊,拔膠西縣,渠賊李全來援,併破之。戊辰,河北行省敗紅襖賊,進至密州,降偽將校數十人,士卒七百人,悉復其業。



 五月辛未朔,鳳翔元帥完顏閭山破宋人步落堝、香爐堡諸屯。甲戌,招撫副使黃摑阿魯答襲破李全於莒州及日照縣之南,三道擊之,追奔四十里。丙子,夏人自葭州入鄜延,元帥承立遣兵敗之馬吉峰,是日捷至。詔遣官督捕河南諸路蝗。辛巳,策論詞賦經義進士及武舉人入見,賜告命章服。萊州民曲貴殺節度經略使內族轉
 奴,自稱元帥,構宋人據城叛。山東招撫司遣提控王庭玉、招撫副使黃摑阿魯答等討平之,斬偽統制白珍及牙校數十人,生禽貴及偽節度使呂忠等十餘人,誅之。乃命庭玉保萊,朱琛保密,阿魯答保寧海,以安輯其民。丙戌,陜西行省言:「四月中,鞏州行元帥承裔遣提控烏古論長壽、納蘭記僧分道伐宋。長壽出鹽川鎮,記僧出鐵城堡,皆克捷而還。」辛卯,壽州行樞密院南城軍攻辛城鎮,一軍趣史河,與宋人戰,勝之。壬辰,河北行省復黃縣。乙未,第鳳翔、秦、鞏三道南征將士功,各遷其官。丙申,增隨朝官及諸承應俸。戊戌,陜西行省連報承裔等
 入宋境之捷。己亥,大元兵徇錦州,元帥仲亨死之。庚子,陜州群狼傷百餘人,立賞募人捕殺。



 六月甲辰,樞密院言:「諸道表稱大元集兵應州、飛狐,將分道南下,觀其意不在河北,而在陜西。河東各路義士、土兵、蕃漢弓箭手,宜於農隙教閱,以備緩急。東平、單州衝重,豫徙其農民糧畜,置可守之城,修近城水砦,因以為固。潼谷遠連商、虢,宜令兩帥府選官按視扼塞。」又言:「賈瑀等刺殺苗道潤,乞治瑀等專殺之罪,餘州郡各以正職授頭目,使分治一方。」上諭之曰:「道潤之眾亟收集之,瑀等是非未明,姑置勿問。諸頭目各制一方,利害至重,更審處之。」石
 州賊馮天羽眾數千,據臨泉縣為亂。帥府命將討捕之,為賊所敗,旁郡縣將謀應之。州刺史紇石烈公順赴以兵,天羽等數十人迎降,公順殺之。餘賊走保積翠山,遣將王九思攻之,不下。詔國史院編脩官馬季良持告敕金幣往招之。比至,九思先破柵,殺賊二千人,餘復走險。已而其黨安國用等詣季良降者五千餘人,就署國用同知孟州防禦使事,以次遷擢有差。分其眾于絳、霍間。丁未,以參知政事把胡魯權左副元帥,與平章政事胥鼎協力防秋。己酉,苗道潤所部軍請隸潞州元帥府,詔河北行省審處之。壬子,紅襖賊犯沂州,官軍敗之,追至
 白里港,都提控齊信沒於陣,詔有司議贈恤。丙辰,遣監察御史粘割梭失往河中、絳、解等郡,同守土官商度可保城池。丁巳,上以久旱,諭宰臣治京獄冤。因及京城小民,中納石炭,既給其價,御史劾以過請官錢,並繫之獄,有論至極刑者,欲悉從寬宥,何如?高琪對不然,遂止。壬戌,御史言戶部員外郎臧伯昇供億息州,偶遇官軍戰勝,亦冒遷一官,乞論其罪。上曰:「軍前如此者,何止伯昇,今遽見罪,餘皆不安。且詰所從來,勢連及帥府。多故之秋,豈為一官,遂忘大計?但令釐正之。」癸亥,遣高汝礪、徒單思忠禱雨。



 秋七月庚午朔,日有食之。辛未,詔賞南伐將
 有士有差。夏人犯龕谷,提控夾谷瑞及其副趙防擊走之。甲戌,以旱災,詔中外。己卯,遣官望祀嶽鎮海瀆于北郊,享太廟,祭太社、太稷,祭九宮貴神于東郊,以禱雨。遣太子太保阿不罕德剛、禮部尚書楊雲翼分道審理冤獄。癸未,大雨。太子、親王、百官表請御正殿,復常膳。庚寅,擇明幹官提控銓選無違失者與升擢,令譯史不任事者,驗已歷俸月放滿,別選能者。甲午,夏人復犯龕谷,夾谷瑞大破之。用點檢承玄言,遣官詣諸道選寄居守闕丁憂官及親軍入仕才堪總兵者,得一百六人,付樞密任使。



 八月庚子朔,河北行省以苗道潤軍隸涿州刺史李
 瘸驢,副以張甫、張柔。戊申,敕親軍百戶以下授職待闕者給本俸,仍充役,俟當赴任遣之。己酉,詔河北行省完顏霆進軍援山東招撫使田琢,自今將士立功聽琢先賞以聞。大元遣木華里等帥步騎數萬自太和嶺徇河東。乙卯,大元兵收代州。辛酉,棣州提控紇石烈醜漢討賊張聚,大破其眾,復濱、棣二州。姦人李宜伏誅。復禁北歸民渡河。戊辰,大元兵收隰州。



 九月乙亥,下太原府,元帥左監軍兼知樞府事烏古論德升死之。丙戌,論皇太子曰:「軍務之速,動關機會,悉從中覆,則或稽緩。自今有當亟行者,先行後聞。」以戶部尚書納合蒲剌都為元帥
 右監軍,行元帥府事于潞州。戊子,置秦關等處九守禦使,命完顏蒲察等分戍諸阨。議遷海州,侯摯言不便,止。大元兵徇汾州,節度使兀顏訛出虎死之。庚寅,李全破密州,執招撫副使黃摑阿魯答、同知節度使夾谷寺家奴。辛卯,大元兵下孝義縣。乙未,設隨處行六部官,以京府節鎮長官充尚書,次侍郎、郎中、員外郎;防刺長官侍郎,次郎中、員外郎、主事;勾當官聽所屬任使。州府官並充勸農事,防刺長官及京府節鎮同知以下充副使。丙申,李全破壽光縣。



 冬十月甲辰,李全破鄒平縣,戊申,破臨朐縣。己酉,大元兵徇絳、潞。壬子,攻平陽,提控郭用死
 之。癸丑,下平陽,知府事,權參知政事、行尚書省李革及從坦死之。甲寅,權平定州刺史范鐸以棄城伏誅。詔諸郡錄囚官,凡坐軍期者皆奏讞。山東路轉運副使兼同知沂州防禦使程戩及邳州副提控王汝霖等通宋人為變,伏誅。宋人攻漣水縣,提控劉瑛敗之。丁巳,大元兵攻澤州。戊午,尚書省言獲姦細叛亡,率多僧道。詔沿邊諸州,惟本處受度聽依舊居止,來自河北、山東遣入內郡,譏其出入。己未,李全據安丘,提控王政屯昌樂俟王庭玉兵同進討。宣差太府少監伯德玩擅率政兵攻全,為全所敗,提控王顯死焉。田琢上言乞正玩罪。癸亥,月
 犯軒轅左角之少民星。甲子,詔河東北路忻、代、寧化、東勝諸州並受嵐州帥府節制。



 十一月庚午,大赦。庚辰,御登賢門召致政舊臣賜食,訪以時政得失。辛巳,以行元帥府紇石烈桓端權簽樞密院事,行院于徐州,權右都監訛可行元帥府事于息州。甲申,詔河東南路隰、吉等州聽絳州元帥府節制。大元兵收潞州,元帥右監軍納合蒲剌都、參議官修起居注王良臣死之。戊子,龕谷提控夾谷瑞敗夏人於質孤堡。河北行省報海州之捷。壬辰,定經兵州縣職官子孫非本貫理蔭及過期不蔭等格。丙申,大元兵下太原之韓村砦。定京師失火法。



 十二月
 己亥朔,以御史中丞完顏伯嘉權參知政事、元帥左監軍,行河中府尚書省元帥府,控制河東南、北路便宜從事。升絳州為晉安府,總管河東南路兵,降平陽為散府。辛丑,簽樞密院事蒲察移剌都伏誅。壬寅,前山東西路轉運使致仕移剌福僧上章言時事。癸卯,詔大理卿溫迪罕達權同簽樞密院事,行院於許州。甲辰,以誅移剌都,詔中外。乙巳,命徒單思忠祈雪,已而大雪。甲寅,以開封府治中呂子羽等使宋講和。紅襖賊攻彭城之胡材寨,徐州兵討敗之。乙卯,以禮部侍郎抹撚胡魯剌為汾陽軍節度使,權元帥右監軍,與嵐州元帥古里甲石倫完復河
 東。丁巳,籍瀕河埽兵。癸亥,尚書省言:「樞密掌天下兵,皇太子撫軍,而諸道又設行院。其有功及失律者,須白院,啟東宮,至於奏可,然後誅賞,有司但奉行而已。自今軍中號令關賞罰者,皆明注詔旨、教令,毋容軍司售其姦欺。」上從之。以樞密副使駙馬都尉僕散安貞為左副元帥,權參知政事,行尚書省元帥府事,伐宋。甲子,上諭旨有司:「京師丐食死於祁寒,朕甚憫之。給以後苑竹木,令居獲燠所。」



 三年春正月庚千,呂子羽至淮,宋人不納而還。詔伐宋。丙子,稅民種地畝,議行均輸。戊寅,敕和市邊城軍需,無
 至配民。定鎮戍征行軍官減資歷月日格。壬午,大雪。上聞東掖有撤瓦聲,問左右,知為丁夫葺器物庫廡舍,上惻然,諭主者曰:「雪寒役人不休,可乎?姑止之。」丙戌,紇石烈牙吾塔上濠州香山村之捷。丁亥,諭宣徽,皇后生日免百官賀。壬辰,以大元兵已定太原,河北事勢非復向日,集百官議備禦長久之計。伐宋捷至,上謂侍臣曰:「此事豈得已哉。近日遣使實欲講和,彼既不從,安得不用兵也?」免單丁民戶月輸軍需錢。甲午,有司請立價以買南征軍士所獲馬,上恐失眾心,因至敗事,不聽。乙未,敕尚書省,自今六部稟議常事,但可再送,不得趣召辨正。
 餘應入法寺定斷而再送,猶未當者具以聞,下吏治之。宰相執政以下皆不得召部寺官,部寺官亦不得詣省,犯者論違制。丁酉,鄧州元帥府提控婁室有罪,減死削爵。



 二月庚子,上與太子謀南征帥,不得其人,歎曰:「天下之廣,緩急無可使者,朕安得不憂?」紇石烈牙吾塔敗宋人於滁。甲辰,胥鼎言:「軍中誅賞,近制須聞朝廷。賞由中出,示恩有歸,可。部分失律,主將不得即治其罪,不可。」詔尚書樞密雜議。宰臣請城野戰將校有罪,從七品以下許便宜決罰,餘悉奏裁。上曰:「七品以下財令治之,將權太輕,或至誤事。自今四品以下聽決。」乙巳,攻宋光山
 縣,俘其統制蔡從定等,光州以兵求援,復敗之。丙午,上謂宰臣:「江淮之人,號稱選心耎,然官軍攻蔓菁堝,其眾困甚,脅之使降,無一肯從者。我家河朔州郡,一遇北警,往往出降,此何理也?」丁未,敕凡立功將居喪者特起復遷授。戊申,拔宋小江寨,殺其統制王大蓬。己酉,取宋武休關。庚戌,元帥左都監承立,以綏德、保安之境,各獲夏人統軍司文移來上,其辭雖涉不遜,而皆有保境息民之言,詔尚書省議之。宰臣言:「鎮戎、靈平等鎮近耗,夏人數犯疆埸。此文正緩我耳,宜嚴備禦,以破姦計。」上然其言。又曰:「頃近侍還自陜西,謂白撒已得鳳州,如得武
 休關,將遂取蜀。朕意殊不然,假令得之,亦何可守?此舉蓋為宋人渝盟,初豈貪其土地耶?朕重惜生靈,惟和議早成為佳耳。」高汝礪乞致事,優詔不允。甲寅,詔陜西行省,從七品以下官許注擬,有罪許決罰,丁憂等闕隨宜任使。軍官徒以上罪及軍事怠慢者,巡按御史治之。己未,行省安貞入宋境,破梁縣等軍,擒統制李申之。右副元帥完顏賽不、左都監牙吾塔,白石關、平山砦之捷俱至。



 三月丁卯朔,陜西兵破宋虎頭關,取興元、洋州。捷至,上大悅。庚午,破宋人於七口倉。甲戌,高麗先請朝貢,因遣使撫諭之,使還,表言道路不通,俟平定後議通款。命
 行省姑示羈縻,勿絕其好。戊寅,蔡州行元帥府右都監完顏合達破宋人於梅林關,擒統制張時。己卯,長春節,免朝賀。提控奧屯吾里不敗宋人於上津縣,軍還至濠州,宋人來拒,牙吾塔擊走之。乙酉,河南路節鎮以上立軍器庫,設使、副各一員,防刺郡設都監、同監各一員、完顏合達敗宋人于馬嶺堡。丙戌,行省安貞破宋人於石堌山。己丑,追賜皇后父太尉汴國公彥昌姓溫敦。庚寅,攻宋麻城縣,拔之,獲其令張倜等。辛卯,行省安貞破宋兵于塗山。壬辰,賽不敗宋兵于老口鎮,又敗宋人于石鶻崖。甲午,錄用罪廢官副元帥蒲察阿里不孫、御史大
 夫永錫等七十人。詔太原等路,州縣闕正授官,令民推其所愛為長,從行省量與職任。及運解鹽人陜西,以濟調度,命胥鼎兼領其事。閏三月丙申朔,申明屠宰牛罪律。以雄、霸以東付權中都經略李瘸驢,易州以西付權中都西路經略靖安民治之。遙授金安軍節度使完顏和尚、故行軍副提控夾古吾典皆除名。庚子,皇子平章政事濮王守純進封英王。壬寅,叛賊王公喜構宋人取沂州。甲辰,以沂國公主薨,輟朝。丙午,給空名宣敕及金銀符,付嵐州帥古里甲石倫,許便宜遷注,以招脅從。丁未,諭樞密院議晉安、東平、河中諸郡備兵之策。庚戌,行省左
 副元帥僕散安貞至自軍前,入見于仁安殿。辛亥,少府少監粘割梭失言利害七事。甲寅,以南伐師還,罷南邊州郡籍民為兵者。戊午,夏人破葭州之通秦砦,刺史紇石烈王家奴戰沒。壬戌,治書侍御史蒲魯虎上書,請選太子師傅。甲子,胥鼎等各遷官,賞南伐之功。



 夏四月丙寅朔,裕、宿等州置元帥府,選陜西步騎精銳六千人實京兆。戊辰,選精銳六萬分屯涼、涇、邠、乾、耀等州。庚午,以秦州主禦使女奚烈古里間行元帥府于平涼。罷募民運解鹽。築京師裏城,命侯摯董役,高琪總監之。甲戌,以知臨洮府事石盞合喜為元帥左都監,行元帥府事
 於鞏州。壬午,遣近侍四人巡視築城丁夫,時其飲食,聽其更休,督吏慘酷,悉禁止之。癸未,陜西黑風晝起,有聲如雷,地大震。甲申,詔河北州縣官止令土著推其所愛者充,朝廷已授者別議任使。乙酉,夏人據通秦寨,提控納合買住擊敗之。己丑,林州都統霍成以疑貳誣殺降人,論罪當死,元帥惟良不欲以殺敵人誅邊將,請寬其罰,仍請立護送降民賞格,以杜後患。上為之赦成,而命有司班賞格焉:護送十人以至者遷一官,不及者每名賞錢二百緡,五十人以上兩官,百人以上兩官雜班任使。庚寅,以時暑,詔朝臣四日一奏事。高汝礪請備防
 秋之糧,宜及年豐於河南州郡驗直立式,募民入粟。上與議定其法而行之。同提舉榷貨司王三錫請榷油,歲可入銀數萬兩,高琪主之,眾以為不便,遂止。辛卯,夏人犯通秦砦,元帥完顏合達出兵安塞堡以搗其巢。至隆州,夏人逆戰,官軍擊之,眾潰,進薄城,俄陷其西南隅,會日暮,還。壬辰,以同知平陽府事胡天作充便宜招撫使。



 五月乙未朔,鳳翔元帥府遣兵敗宋人于黃牛等堡。壬子,太白晝見於參。



 六月甲子朔,時暑,給修城夫病者藥餌。遣諭元帥合達曰:「以卿幹局,故有唐、鄧之委。或有侵軼,戰退不宜遠追,第固吾圉。」以驃騎上將軍河南路統
 軍使石盞女魯歡為元帥右都監,行平涼元帥府事。詔付遼東等處行省金銀符及空名宣敕,聽便宜處置。壬申,制沿河戍兵逃亡罪並同征行軍人例。詔御史中丞完顏伯嘉行樞密院于許州。甲戌,定防秋將校擊球飲燕之罰。李全寇日照、博興,紇石烈萬奴敗之;寇即墨,完顏僧壽又敗之,復萊州。戊寅,詔陜西簽軍如河南例,曲赦河東南、北路。丁亥,命防禦使徒單福定等帥所部義軍,與沂州民老幼盡徙于邳。戊子,遼州總領提控唐括狗兒帥師復太原府。平涼等處地震,詔右司諫郭著撫諭其軍民。



 秋七月丁酉,籍邳、海等州義軍及脅從歸國
 而充軍者,人給地三十畝,有力者五十畝,仍蠲差稅,日支糧二升,號「決勝軍」。戊戌,上進樞密臣僚諭之曰:「里城久未畢功,尚書欲增調民,朕慮妨農。況糧儲不繼,將若之何,盍改圖之。」樞臣言:「是役之興,實為大計,今功已過半,偶值霖潦,成功差遲。尚書議增丁夫,勢必驗口,不令妨業。比及防秋,當告成矣。」上曰:「卿等善為計畫,無貽朕憂。」庚子,以地震,曲赦陜西路。甲辰,置京東、西、南三路行三司。乙卯,曲赦山東西路。丁巳,遣單思忠以地震祭地祇于上清宮。



 八月丙寅,補闕許古等削官解職。丁卯,木星犯輿鬼東南星。戊辰,遣禮部尚書楊雲翼祭社稷,翰
 林侍讀學士趙秉文祭后土于河中府。京西行三司李復亨言汝、鄧冶鐵,河南、北食鹽之利。木星晝見于柳,百有九日乃滅。壬申,上敕臺臣:「朕處分尚書事,或至數日不奉行,及再問則巧飾次第以對。大臣容有遺忘,左右司玩弛,臺臣當糾。今後復爾,併罪卿等。」乃定御史上下半月勾檢省中制敕文字。大元兵下武州,軍事判官郭秀死之。丁丑,緩在京差徭。中山治中王善殺權知府事李仲等以叛。大元兵下合河縣,縣令喬天翼等死之。乙酉,命樞密遣官簡嶺外諸軍之武健者,養之彰德、邢、洺、衛、浚、懷、孟等城,弱者罷遣,戊子,敕侯摯諭三司行部官
 勸民種麥,無種粒者貸之。



 九月甲午,詔單州經略使完顏仲元屯宿州,與右都監紇石烈德同行帥府事。丙申,唐州從宜夾谷天成敗宋人於桐柏。丁酉,尚書省請申命侯摯廣營策積貯,上不許,曰:「徵斂已多,今更規畫,不過復取於民耳。防秋稍緩,當量減戍兵,用度幸足,何至是耶。」甲辰,大元兵徇東勝州,節度使伯德窳哥死之。庚戌,命行省胥鼎領兵赴河中。壬子,真定招撫使武仙請給金銀符賞有功,從之。沿河造戰艦,付行院帥府。



 冬十月癸亥朔,定保舉縣令能否升黜舉主制。乙丑,用蒙古綱言,招集義軍各置都統、副統等官,如貞祐三年制。平涼府
 先以地震被命醮祭,方行事,慶雲見,以圖來上。遣官覆驗得實,是日,百官奉表稱賀。丁卯,以完顏開權元帥左都監,郭文振權右都監,並行元帥府事,謀復太原。壬申,定贓吏計罪以銀為則。癸酉,以慶雲遣官告太廟。甲戌,以慶雲詔國內。己卯,大元兵次單州境,詔諸路民應遷避兵而不欲者,亟遣人以利害曉之。癸未,裏城畢工,百官稱賀。宴宰臣便殿。遷右丞摯官一階,賜右丞相琪、左丞汝礪、參知政事思忠金鼎各一,重幣三。是役,上慮擾民,募人能致甓五十萬者遷一官,百萬仍升一等。平陽判官完顏阿剌、左廂譏察霍定和發宋蔡京故居,得二
 百萬有奇,準格遷賞。甲申,宰臣請以里城之功建碑會朝門,從之。丁亥,大元兵屯綿上。壬辰,命有司葺閑舍,給薪米,以濟貧民,期明年二月罷,俟時平則贍之以為常。



 十一月癸巳朔,前嵐州倉使張祐自夏國來歸。以樞密副使僕散安貞、同簽院事訛可行院事于河北。乙未,以官驢借朝士之無馬者乘之,仍給芻豆。己亥,大元兵徇彰德府。辛丑,詔朝官七品、外路六品以上官,二歲舉縣令一人。戶部令史蘇唐催租封丘,期限迫促,民有生刈禾輸租者。上聞之,遣吏按問,杖唐五十,縣令高希隆減二等。尚書以希隆罰輕,上曰:「使臣至外路,自非至剛者,
 孰能不從?其依前詔。」甲寅,徐州總領納合六哥大破紅襖賊於狄山。禮部郎中抹撚魯剌上疏言時事。丁巳,右丞相高琪下獄。



 泰安軍副使張天翼為賊張林所執以歸宋,縶之楚州,至是逃歸,授睢州刺史,超兩官,進職一等。戊午,大元兵平晉安府,行元帥府事、工部尚書粘割貞死之。



 十二月,誅高琪。



\end{pinyinscope}