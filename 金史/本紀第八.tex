\article{本紀第八}

\begin{pinyinscope}

 世宗下



 二十一年正月戊申朔,宋、高麗、夏遣使來賀。壬子,以夏國請,詔復綏德軍榷場,仍許就館市易。上聞山東、大名等路猛安謀克之民,驕縱奢侈,不事耕稼。詔遣閱實,計口授地,必令自耕,地有餘而力不贍者,方許招人租佃,仍禁農時飲酒。丙辰,追貶海陵煬王亮為庶人,詔中外。甲子,如春水。丙子,次永清縣。有移剌餘里也者,契丹人
 也,隸虞王猛安,有一妻一妾。妻之子六,妾之子四。妻死,其六子廬墓下,更宿守之。妾之子皆曰:「是嫡母也,我輩獨不當守墳墓乎?」於是,亦更宿焉,三歲如一。上因獵,過而聞之,賜錢五百貫,仍令縣官積錢於市,以示縣民,然後給之,以為孝子之勸。



 二月戊戌,太白晝見。庚子,還都。壬寅,以河南尹張景仁為御史大夫。乙巳,以元妃李氏之喪,致祭興德宮,過市肆不聞樂聲,謂宰臣曰:「豈以妃故禁之耶?細民日作而食,若禁之是廢其生計也,其勿禁。朕前將詣興德宮,有司請由薊門,朕恐妨市民生業,特從他道。顧見街衢門肆,或有毀撤,障以簾箔,何必爾
 也。自今勿復毀撤。」



 三月丁未朔,萬春節,宋、高麗、夏遣使來賀。上初聞薊、平、灤等州民乏食,命有司發粟糶之,貧不能糴或貸之。有司以貸貧民恐不能償,止貸有戶籍者。上至長春宮,聞之,更遣人閱實,賑貸。以監察御史石抹元禮、鄭達卿不糾舉,各笞四十,前所遣官皆論罪。甲子,太白晝見。乙丑,詔山後冒占官地十頃以上者皆籍入官,均給貧民。遼州民硃忠等亂言,伏誅。上謂宰臣曰:「近聞宗州節度使阿思懣行事多不法,通州刺史完顏守能既與招討職事,猶不守廉。達官貴要多行非理,監察未嘗興劾。斡睹只群牧副使僕散那也取部人二球
 仗,至細事也,乃便劾奏。謂之稱職,可乎?今監察職事修舉者與遷擢,不稱者,大則降罰,小則決責,仍不許去官。」



 閏月己卯,恩州民鄒明等亂言,伏誅。辛卯,漁陽令夾谷移里罕、司候判官劉居漸以被命賑貸,止給富戶,各削三官,通州刺史郭邦傑總其事,奪俸三月。乙未,上謂宰臣曰:「朕觀自古人君多進用讒諂,其間蒙蔽,為害非細,若漢明帝尚為此輩惑之。朕雖不及古之明君,然近習讒言,未嘗入耳。至於宰輔之臣,亦未嘗偏用一人私議也。」癸卯,以尚書左丞相完顏守道為太尉、尚書令,尚書左丞蒲察通為平章政事,右丞襄為左丞,參知政事張
 汝弼為右丞,彰德軍節度使梁肅為參知政事。



 四月戊申,以右丞相徒單克寧為左丞相,平章政事唐括安禮為右丞相。增築泰州、臨潢府等路邊堡及屋宇。庚戌,奉安昭祖以下三福三宗御容於衍慶宮,行親祀禮。上諭宰臣曰:「朕之言行豈能無過?常欲人直諫而無肯言者。使其言果善,朕從而行之,又何難也。」戊辰,以滕王府長史把德固為橫賜夏國使。壬申,幸壽安宮。



 五月戊子,西北路招討使完顏守能以贓罪,杖二百,除名。



 七月丙戌,還都。丁酉,樞密使趙王永中罷。己亥,以左丞相徒單克寧為樞密使。辛丑,以太尉、尚書令完顏守道復為左丞
 相,太尉如故。



 八月乙丑,以右副都點檢胡什賚等為賀宋生日使,吏部郎中奚胡失海為夏國生日使。



 二十二年三月辛未朔,萬春節,宋、高麗、夏遣使來賀,丁丑,命尚書省申敕西北路招討司勒猛安謀克官督部人習武備。甲申,諭戶部:「今歲行幸山後,所須並不得取之民間,雖所用人夫,並以官錢和雇,違者杖八十,罷職。」癸巳,詔頒重修制條。以吏部尚書張汝霖為御史大夫。



 四月乙卯,行監臨院務官食直法。以削明肅尊號,詔中外,從皇太子請也。甲子,上如金蓮川。



 五月甲申,太白晝見。



 六月庚子朔,制立限放良之奴,限內娶良人為妻,所
 生男女即為良。丁巳,右丞相致仕石琚薨。七月辛巳,宰臣奏事,上頗違豫,宰臣請退。上曰:「豈以朕之微爽於和,而倦臨朝之大政耶?」使終其奏。甲午,秋獵。



 八月戊辰,太白經天。



 九月戊寅,至自金蓮川。以左衛將軍禪赤等為賀宋生日使,尚輦局使僕散曷速罕為夏國生日使。己丑,以同知東京留守司事裔在任專恣,失上下之分,謫授復州刺史。乙未,壽刺史訛里也、同知查剌、軍事判官孫紹先、榷場副使韓仲英等以受商賂縱禁物出界,皆處死。



 十月辛丑,從河間宗室于平州。庚戌,袷享于太廟。



 十一月丙子,以吏部尚書孛術魯阿魯罕等為賀宋
 正旦使。東京留守徒單貞以與海陵逆謀,伏誅。妻永平縣主,子慎思並賜死。甲申,以宿直將軍僕散忠佐為高麗生日使。玉田縣令移剌查坐贓,伏誅。戊子,冬獵。



 十二月庚子,還都。癸丑,獵近郊。辛酉,立強取諸部羊馬法。



 二十三年正月丁卯,宋、高麗、夏遣使來賀,庚午,詔有司但獲強盜,迹狀既明,賞隨給之,勿得更待。丁丑,參知政事梁肅致仕。辛巳,廣樂園燈山火。壬午,如春水,詔夾道三十里內被役之民與免今年租稅,仍給傭直。甲午,大邦基伏誅。



 二月乙巳,還都。戊申,以尚書右丞張汝弼攝太尉,致祭于至聖文宣王廟。庚戌,以戶部尚書張仲
 愈為參知政事。御史臺進所察州縣官罪,上覽之曰:「卿等所廉皆細碎事,又止錄其惡而不舉其善,審如是,其為官者不亦難乎?其併察善惡以聞。」



 三月丙寅朔,萬春節,宋、高麗、夏遣使來賀。丙子,初製宣命之寶,金、玉各一。尚書右丞相烏古論元忠罷。潞州涉縣人陳圓亂言,伏誅。乙酉,雨土。丙戌,詔戒諭中外百官。



 四月辛丑,更定奉使三國人從差遣格。祁州刺史大磐坐無罪掠死染工,妄認良人二十五口為奴,削官四階,罷之。癸丑,地生白毛。以大理正紇石主列速為橫賜高麗使,壬戌,幸壽安宮,剌有司為民禱雨。是夕,雨。



 五月庚午,縣令大雛訛只等
 十人以不任職罷歸。六十以上者進官兩階,六十以下者進官一階,並給半俸。甲戌,命應部除官嘗以罪罷而再敘者,遣使按其治迹,如有善狀,方許授以縣令,無治狀者,不以任數多少,並不得授。丁亥,雷,雨雹,地生白毛。



 六月壬子,有司奏右司郎中段珪卒,上曰:「是人甚明正,可用者也。如知登聞檢院巨構,每事但委順而已。燕人自古忠直者鮮,遼兵至則從遼,宋人至則從宋,本朝至則從本朝,其俗詭隨,有自來矣!雖屢經遷變而未嘗殘破者,凡以此也。南人勁挺,敢言直諫者多,前有一人見殺,後復一人諫之,甚可尚也。」又曰:「昨夕苦暑,朕通宵不
 寐,因念小民比屋卑隘,何以安處?」



 七月乙酉,平章政事移剌道,參知政事張仲愈皆罷。御史大夫張汝霖坐失糾舉,降授棣州防禦使。



 八月乙未,觀稼于東郊。以女直字《孝經》千部付點檢司分賜護衛親軍。癸卯,還都。乙巳,大名府猛安人馬和尚謀叛,伏誅。括定猛安謀克戶口田土牛具。以戶部尚書程輝為參知政事。



 九月己巳,以同僉大宗正事方等為賀宋生日使,宿直將軍完顏斜里虎為夏國生日使。譯經所進所譯《易》、《書》、《論語》、《孟子》、《老子》、《楊子》、《文中子》、《劉子》及《新唐書》。上謂宰臣曰:「朕所以令譯《五經》者,正欲女直人知仁義道德所在耳!」命頒行之。
 辛未,秋獵。



 十月癸巳,還都。庚戌,幸東宮,賜皇孫吾都補洗兒禮。己未,慶雲見。辛酉,太白晝見。



 十一月壬戌朔,日有食之。丙寅,平章政事蒲察通罷。丁卯,歲星晝見。壬申,以樞密副使崇尹為平章政事。



 閏月甲午,上謂宰臣曰:「帝王之政,固以寬慈為德,然如梁武帝專務寬慈,以至綱紀大壞。朕嘗思之,賞罰不濫,即是寬政也,餘復何為?」以尚書左丞襄為平章政事,右丞張汝弼為左丞,參知政事粘割斡特剌為右丞,禮部尚書張汝霖為參知政事。以西京留守婆廬火等為賀宋正旦使。制外任官嘗為宰執者,凡吏牘上省部,依親王例,免書名。戊午,歲星
 晝見。上謂宰臣曰:「女直進士可依漢兒進士補省令史。夫儒者操行清潔,非禮不行。以吏出身者,自幼為吏,習其貪墨,至於為官,習性不能遷改。政道興廢,實由於此。」庚申,尚書省左司員外郎徐偉奏事,上謂宰臣曰:「斯人純而幹,有司郎中郭邦傑直而頗躁。」十二月癸酉,上謂宰臣曰:「海陵自以失道,恐上京宗室起而圖之,故不問疏近,並徙之南。豈非以漢光武、宋康王之疏庶得繼大統,故有是心。過慮若此,何其謬也。」乙酉,高麗以母喪來告。丁亥,以真定尹烏古論元忠復為尚書右丞相。



 二十四年正月辛卯朔,宋、夏遣使來賀。徐州進芝草十
 有八莖,真定進嘉禾二本,六莖,異畝同穎。戊戌,如長春宮春水。



 二月壬申,還都。癸酉,上曰:「朕將往上京。念本朝風俗重端午節,比及端午到上京,則燕勞鄉間宗室父老。」甲戌,制一品職事官庶孽子承蔭,更不引見。丙戌,以東上閣門使完顏進兒等為高麗敕祭使,西上閣門使大仲尹為慰問使,虞王府長史永明為起復使,以器物局使皞為橫賜夏國使。



 三月庚寅朔,萬春節,宋、夏遣使來賀。甲午,以上將如上京,尚書省奏定「皇太子守國諸儀」。丙申,尚書省進「皇太子守國寶」,上召皇太子授之,且諭之曰:「上京祖宗興王之寺,欲與諸王一到,或留三二
 年,以汝守國。譬之農家種田,商人營財,但能不墜父業,即為克家子,況社稷任重,尤宜畏慎。常時觀汝甚謹,今日能紓朕優,乃見中心孝也。」皇太子再三辭讓,以不諳政務,乞備扈從。上曰:「政事無甚難,但用心公正,毋納讒邪,久之自熟。」皇太子流涕,左右皆為之感動。皇太子乃受寶。丁酉,如山陵。己亥,還都。壬寅,如上京。皇太子允恭守國。癸卯,宰執以下奉辭於通州。上謂宰執曰:「卿輩皆故老,皇太子守國,宜悉心輔之,以副朕意。」又謂樞密使徒單克寧曰:「朕巡省之後,脫或有事,卿必親之。毋忽細微,大難圖也。」又顧六部官曰:「朕聞省部文字多以小不
 合而駁之,茍求自便,致累歲不能結絕,朕甚惡之。自今可行則行,可罷則罷,毋使在下有滯留之嘆!」時諸王皆從,以趙王永中留輔太子。



 四月己未朔,太白晝見。咸平尹移剌道薨。庚申,次廣寧府。丙寅,次東京。丁卯,朝謁孝寧宮。給復東京百里內夏秋稅租一年。在城隨關年七十者補一官。曲赦百里內犯徒二年以下罪。乙酉,觀漁于混同江。



 五月己丑,至上京,居于光興宮。庚寅,朝謁于慶元宮。戊戌,宴于皇武殿。上謂宗戚曰:「朕思故鄉,積有日矣,今既至此,可極歡飲,君臣同之。」賜諸王妃、主,宰執百官命婦各有差。宗戚皆霑醉起舞,竟日乃罷。



 六月辛
 酉,幸按出虎水臨漪亭。壬戌,閱馬于綠野澱。



 七月乙未,上謂宰臣曰:「天子巡狩當舉善罰惡。凡士民之孝弟淵睦者舉而用之,其不顧廉恥無行之人則教戒之,不悛者則加懲罰。」丙午,獵于勃野澱。乙卯,上謂宰臣曰:「今時之人,有罪不問,既過之後則謂不知。有罪必責,則謂每事尋罪。風俗之薄如此。不以文德感化,不能復于古也。卿等以德輔佐,當使復還古風。」



 八月癸亥,以太府監張大節等為賀宋生日使,侍御史遙里特末哥為夏國生日使。乙亥,詔免上京今年市稅。



 九月甲辰,歲星晝見。



 十月丁卯,獵于近郊。



 十一月辛卯,還宮。甲午,詔以上京天
 寒地遠,宋正旦、生日,高麗、夏國生日,並不須遣使,令有司報諭。丙午尚書省奏徙速頻、胡里改三猛安二十四謀克以實上京。



 十二月丙辰,獵於近郊。己卯,還宮。



 二十五年正月乙酉朔。丁亥,宴妃嬪、親王、公主、文武從官于光德殿,宗室、宗婦及五品以上命婦,與坐者千七百餘人,賞賚有差。



 二月癸酉,以東平尹鳥古論思列怨望,殺之。丁丑,如春水。



 四月己未,至自春水。癸亥,幸皇武殿擊球,許士民縱觀。甲子,詔於速頻、胡里改兩路猛安下選三十謀克為三猛安,移置于率督畔窟之地,以實上京。壬申,曲赦會寧府仍放免今年租稅,百姓年七十
 以上者補一官。甲戌,以會寧府官一人兼大宗正丞,以治宗室之政。上謂群臣曰:「上京風物朕自樂之,每奏還都,輒用感愴。祖宗舊邦,不忍捨去,萬歲之後,當置朕於太祖之側,卿等無忘朕言。」丁丑,宴宗室、宗婦于皇武殿,大功親賜官三階,小功二階,緦麻一階,年高屬近者加宣武將軍。及封宗女,賜銀、絹各有差。曰:「朕尋常不飲酒,今日甚欲成醉,此樂亦不易得也!」宗室婦女及群臣故老以次起舞,進酒。上曰:「吾來數月,未有一人歌本曲者,吾為汝等歌之。」命宗室弟敘坐殿下者皆坐殿上,聽上自歌。其詞道王業之艱難,及繼述之不易,至「慨想祖
 宗,宛然如睹」,慷慨悲激,不能成聲,歌畢泣下。右丞相元忠率群臣、宗戚捧觴上壽,皆稱萬歲。於是,諸夫人更歌本曲,如私家之會。既醉,上復續調,至一鼓乃罷。己卯,發上京。庚辰,宗室戚屬奉辭。上曰:「朕久思故鄉,甚欲留一二歲,京師天下根本,不能久於此也。太平歲久,國無征徭,汝等皆奢縱,往往貧乏,朕甚憐之。當務儉約,無忘祖先艱難。」因泣數行下,宗室戚屬皆感泣而退。



 五月庚寅,平章政事襄、奉御平山等射懷孕兔。上怒杖平山三十,召襄誡飭之,遂下詔禁射兔。壬寅,次天平山好水川。癸卯,遣使臨潢、泰州勸農。丙午,命尚書省奏事衣窄紫。



 六
 月甲寅,獵近山,見田壟不治,命笞田者。庚申,皇太子允恭薨。丙寅,尚書右丞相烏古論元忠罷。庚午,遣左宣徽使唐括鼎詣京師,致祭皇太子。戊寅,命皇太子妃及諸皇孫執喪。並用漢儀。



 七月戊申,發好水川。九月辛巳朔,次轄沙河,賜百歲老嫗帛。甲申,次遼水,召見百二十歲女直老人,能道太祖開創事,上嘉歎,賜食,併賜帛。己酉,至自上京。是日,上臨奠宣孝皇太子于熙春園。十月丙辰,尚書省奏親軍數多,宜稍減損,詔定額為三千。宰臣退,上謂左右曰:「宰相年老艱于久立,可置小榻廊下,使少休息。」甲子,禁上京等路大雪及含胎時採捕。上謂宰
 臣曰:「護衛年老出職而授臨民,手字尚不能畫,何以治民?人胸中明暗外不能知,精神昏耄已見於外,是強其所不能也。天子以兆民為子,不能家家而撫,在用人而已。知其不能而強授之,百姓其謂我何?」丁丑,命學士院、講院、秘書監、司天臺、著作局、閣門、通進、拱衛、直武器署等官,凡直宮中,午前許退。十一月庚辰朔,詔曰:「豺未祭獸,不許採捕。冬月,雪尺以上,不許用網及速撒海,恐盡獸類。」歲星晝見。壬午,太白晝見。甲午,以臨潢尹僕散守中等為賀宋正旦使。丙申,夏國遣使問起居。戊戌,以曹王永功為御史大夫。壬寅,以禮部員外郎移剌履為高
 麗生日使。十二月戊午,以皇孫金源郡王麻達葛判大興尹,進封原王。甲子,太白晝見,經天。丙寅,左相完顏守道、左丞張汝弼、右丞粘割斡特剌、參知政事張汝霖坐擅增東宮諸皇孫食料,各削官一階。甲戌,制增留守、統軍、總管、招討、都轉運、府尹、轉運、節度使月俸。上謂宰臣曰:「太尉守道論事止務從寬,犯罪罷職者多欲復用。若懲其首惡,後來知畏,罪而復用,何以示戒。」是日,命範銅為「禮信之寶」,凡賜外方禮物,給信袋則用之。丙子,上問宰臣曰:「原王大興行事如何?」右丞斡特剌對曰:「聞都人皆稱之。」上曰:「朕令察于民間,咸言見事甚明,予奪皆
 不失當,曹、豳二王弗能及也。又聞有女直人訴事,以女直語問之,漢人訴事,漢語問之。大習不失本朝語為善,不習,則淳風將棄。」汝弼對曰:「不忘本者,聖人之道也。」斡特剌曰:「以西夏小邦,崇尚舊俗,獨能保國數百年。」上曰:「事當任實,一事為偽則喪百真,故凡事莫如真實也。」



 二十六年正月庚辰朔,宋、高麗、夏遣使來賀。甲辰,如長春宮春水。二月癸酉,還都。乙亥,詔曰:「每季求仕人,問以疑難,令剖決之。其才識可取者,仍訪察政迹,如其言行相副,即加升用。」三月乙卯朔,萬春節,宋、高麗、夏遣使來賀。丁亥,以大理卿闕,上問誰可?右丞粘割斡特剌言,前
 使部尚書唐括貢可,乃授以是職。己丑,尚書省擬奏除授,上曰:「卿等在省未嘗薦士,止限資級,安能得人?古有布衣人相者,聞宋亦多用山東、河南流寓疏遠之人,皆不拘於貴近也。以本朝境土之大,豈無其人,朕難遍知,卿又不舉。自古豈有終身為相者,外官三品以上,必有可用之人,但無故得進耳。」左丞張汝弼曰:「下位雖有才能,必試之乃見。」參政程輝曰:「外官雖有聲,一旦入朝,卻不稱任,亦在沙汰而已。」癸巳,香山寺成,幸其寺,賜名大永安,給田二千畝,栗七千株,錢二萬貫。丁酉,以親軍完顏乞奴言,制猛安謀克皆先讀女直字經史然後承襲。
 因曰:「但令稍通古今,則不肯為非。爾一親軍粗人,乃能言此,審其有益,何憚而不從。」



 四月壬子,尚書省奏定院務監官虧陪兌納法及橫班格。因曰:「朕常日御膳亦從減省,嘗有一公主至,至無餘膳可與,當直官皆目睹之。若欲豐腆,雖日用五十羊亦不難矣!然皆民之脂膏,不忍為也。監臨官惟知利己,不知其利自何而來?朕嘗歷外任,稔知民間之事,想前代之君,雖享富貴,不知稼穡艱難者甚多,其失天下,皆由此也!遼主聞民間乏食,謂何不食乾臘,蓋幼失師保之訓,及其即位,故不知民間疾苦也。隨煬帝時,楊素專權行事,乃不慎委任之過也。
 與正人同處,所知必正道,所聞必正言,不可不慎也。今原王府官屬,當選純謹秉性正直者充,勿用有權術之人。」戊午,尚書左丞張汝弼罷。己未,幸壽安宮。壬戌,太尉、左丞相完顏過道致仕。以客省使李磐為橫賜高麗使。尚書省奏北京轉運使以贓除名。尚書省奏事,上曰:「比有上書言,職官犯除名不可復用,朕謂此言極當。如軍期急速,權可使用。今天下無事,復用此輩,何以戒將來。」又奏:「年前以諸路水旱,於軍民地土二十一萬餘頃內,擬免稅四十九萬餘石。」從之。詔曰:「今之稅,考古行之,但遇災傷,常加蠲免。」



 五月甲申,以司徒、樞密使徒單克寧
 為太尉、尚書左丞相,判大宗正事趙王永中復為樞密使,大興尹原王麻達葛為尚書右丞相,賜名璟。參加政事程輝致仕。戊子,盧溝決於上陽村,湍流成河,遂因之。庚寅,御史大夫曹王永功罷,以豳王永成為御史大夫。戊戌,以尚書右丞粘割斡特剌為左丞,參知政事張汝霖為右丞。



 六月癸亥,尚書省奏速頻、胡里改世襲謀克事,上曰:「其人皆勇悍,昔世祖與之鄰,苦戰累年,僅能克復。其後乍服乍叛,至穆、康時,始服聲教。近世亦嘗分徙。朕欲稍遷其民上京,實國家長久之計。」己巳,上謂宰執曰:「齊桓中庸主也,得一管仲,遂成霸業。朕夙夜以思,惟
 恐失人。朕既不知,卿等又不薦,必俟全才而後舉,蓋亦難矣!如舉某人長於某事,朕亦量材用之。朕與卿等俱老矣!天下至大,豈得無人?薦舉人材,當今急務也。」又言:「人之有幹能,固不易得,然不若德行之士最優也。」上謂右丞相原王曰:「爾嘗讀《太祖實錄》乎?太祖征麻產,襲之,至泥淖馬不能進,太祖捨馬而步,歡都射中麻產,遂擒之。創業之難如此,可不思乎。」甲戌,詔曰:「凡陳言文字詣登聞檢院送學士院聞奏,毋經省廷。」



 七月壬午,詔給內外職事官兼職俸錢。丙申,御史中丞馬惠迪為參知政事。庚子,上聞同知中都路都轉運使事趙曦瑞,其在職
 應錢穀利害文字多不題署,但思安身,降授積石州刺史。



 閏月己未,還都。



 八月丁丑,上謂宰臣曰:「親軍雖不識字,亦令依例出職,若涉贓賄,必痛繩之。」太尉左丞相克寧曰:「依法則可。」上曰:「朕於女直人未嘗不知優恤。然涉於贓罪,雖朕子弟亦不能恕。太尉之意,欲姑息女直人耳!」戊寅,尚書省奏,河決,衛州壞。命戶部侍郎王寂、都水少監汝嘉徙限衛州胙城縣。丁亥,尚書省奏,遣吏部侍郎李晏等二十六人分路推排諸路物力,從之。己丑,以宿直將軍李達可為夏國生日使。辛卯,以益都尹宗浩等為賀宋生日使。甲午,秋獵。庚子,次薊州。辛丑,幸
 仙洞寺。壬寅,幸香林、凈名二寺。



 九月甲辰朔,幸盤山上方寺,因篇歷中盤、天香、感化諸寺。庚申,還都。丙寅,上謂宰臣曰:「烏底改叛亡,已遣人討之,可益以甲士,毀其船筏。」參知政事馬惠迪曰:「得其人不可用,有其地不可居,恐不足勞聖慮。」上曰:「朕亦知此類無用,所以毀其船筏,欲不使再邊境耳!」



 十月戊寅,定職官犯贓同職相糾察法。庚寅,上謂宰臣曰:「西南、西北兩路招討司地隘,猛安人戶無處圍獵,不能閑習騎射。委各猛安謀克官依時教練,其弛慢過期及不親監視,並決罰之。」甲午,詔增河防軍數。戊戌,寧昌軍節度使崇肅、行軍都統忠道以
 討烏底改,不待克敵而還,崇肅杖七十,削官一階,忠道杖八十,削官三階。



 十一月甲辰朔,定閔宗陵廟薦享禮。上謂宰臣曰:「女直人中材傑之士,朕少有識者,蓋亦難得也。新進士如徒單鎰、夾古阿里補、尼厖古鑑輩皆可用之材也。起身刀筆者,雖用才力可用,其廉介之節,終不及進士。今五品以上闕員甚多,必資級相當,至老有不能得者,況欲至卿相乎?古來宰相率不過三五年而退,罕有三二十年者,卿等特不舉人,甚非朕意。」上顧修起居注崇璧曰:「斯人孱弱,付之以事,未必能辦,以其謹厚長者,故置諸左右,欲諸官效其為人也。」辛亥,以刑部尚
 書移剌子元等為賀宋正旦使。戊午,以左警巡副使鶻沙通敏善斷,擢殿中侍御史兼右三部司正。庚申,立右丞相原王璟為皇太孫。甲子,上謂宰臣曰:「朕聞宋軍自來教習不輟,今我軍專務游惰,卿等勿謂天下既安而無豫防之心,一旦有警,軍不可用,顧不敗事耶?其令以時訓練。」丙寅,上謂侍臣曰:「唐太子承乾所為多非度,太宗縱而弗檢,遂至於廢,如早為禁止,當不至是。朕於聖經不能深解,至於史傳,開卷輒有所益。每見善人不忘忠孝,檢身廉潔,皆出天性。至於常人多喜為非,有天下者茍無以懲之,何由致治。孔子為政七日而誅少正卯,
 聖人尚爾,況餘人乎?」戊辰,上謂宰臣曰:「朕雖年老,聞善不厭。孔子云:『見善如不及,見不善如探湯。』大哉言乎!」右丞張汝弼對曰:「知之非艱,行之惟艱。」以拱衛直副都指揮使韓景懋為高麗生日使。以近侍局直長尼厖古鑑純直通敏,擢皇太孫侍丞。己巳,獵近郊。庚午,上謂宰臣曰:「朕方前古明君,固不可及。至於不納近臣讒言,不受戚里私謁,亦無愧矣!朕嘗自思,豈能無過,所患過而不改,過而能改,庶幾無咎。省朕之過,頗喜興土木之工,自今不復作矣。」



 十二月甲申,上退朝,御香閤,左諫議大夫黃久約言遞送荔支非是,上諭之曰:「朕不知也,今令罷
 之。」丙戌,上謂宰臣曰:「有司奉上,惟沽辦事之名,不問利害如何。朕嘗欲得新荔支,兵部遂於道路特設鋪遞。比因諫官黃久約言,朕方知之。夫為人無識,一旦臨事,便至顛沛。宮中事無大小,朕常親覽者,以不得人故也,如使得人,寧復他慮。」丁亥,上謂宰臣曰:「朕年來惟以省約為務,常膳止四五味,已厭飫之,比初即位十減七八。」宰臣曰:「天子自有制,不同餘人。」上曰:「天子亦人耳,枉費安用。」丙申,上謂宰臣曰:「比聞河水泛溢,民罹其害者貲產皆空。今復遣官於彼推排,何耶?」右丞張汝霖曰:「今推排皆非被災之處。」上曰:「必鄰道也。既鄰水而居,豈無驚擾
 遷避者乎?計其貲產,豈有餘哉!尚何推排為。」又曰:「平時用人,宜尚平直。至於軍職,當用權謀,使人不易測,可以集事。唐太宗自少年能用兵,其後雖居帝位,猶不能改,吮瘡剪鬚,皆權謀也。」



 二十七年正月癸卯朔,宋、高麗、夏遣使來賀。己酉,以襄城令趙渢為應奉翰林文字。渢入謝,上問宰臣曰:「此黨懷英所薦耶?」對曰:「諫議黃久約亦嘗薦之。」上曰:「學士院比舊殊無人材,何也?」右丞張汝霖曰:「人材須作養,若令久任練習,自可得人。」庚戌,如長春宮春水。



 二月乙亥,還都。乙卯,改閔宗廟號曰熙宗。癸未,命曲陽縣置錢監,賜
 名「利通」。乙酉,上謂宰執曰:「朕自即位以來,言事者雖有狂妄,未嘗罪之。卿等未嘗肯盡言,何也?當言而不言,是相疑也。君臣無疑,則謂之嘉會。事有利害,可竭誠言之。朕見緘默不言之人,不欲觀之矣。」丁亥,命沿河京、府、州、縣長貳官,並帶管勾河防事。己丑,諭宰執曰:「近侍局官須選忠直練達之人用之。朕雖不聽讒言,使佞人在側,將恐漸漬聽從之矣!」上謂宰執曰:「朕聞寶坻尉蒙括末也清廉,其為政何如?」左丞斡特剌對曰:「其部民亦稱譽之,然不知所稱何事?」上曰:「凡為官但得清廉亦可矣,安得全才之人。可進官一階,升為令。」又言:「朕時或體中
 不佳,未嘗不視朝。諸王、查官但有微疾,便不治事,自今宜戒之。」丙申,命罪人在禁有疾,聽親屬人視。



 三月癸卯朔,萬春節,宋、高麗、夏遣使來賀。辛亥,皇太孫受冊,赦。乙卯,尚書省言:「孟家山金口閘下視都城百四十餘尺,恐暴水為害,請閉之。」從之。上謂大臣曰:「十室之邑,必有忠信。今天下之廣,人民之眾,豈得無人?唐之顏真卿、段秀實皆節義之臣也,終不升用,亦當時大臣固蔽而不舉也。卿等當不私親故,而特舉忠正之人,朕將用之。」又言:「國初風俗淳儉,居家惟衣布,非大會賓客,未當輒烹羊豕。朕嘗念當時節儉之風,不欲妄費,凡宮中之官與賜
 之食者,皆有常數。



 四月丙戌,以刑部尚書宗浩為參知政事。丙申,上如金蓮川。辛丑,京師地震。五月壬子,詔罷曷懶路所進海葱及太府監日進時果。曰:「葱、果應用幾何?徒勞人耳!惟上林諸果,三日一進。」庚午,以所進御膳味不調適,有旨問之。尚食局直長言:「臣聞老母病劇,私心憒亂,如喪魂魄,以此有失嘗視,臣罪萬死!」上嘉其孝,即令還家侍疾,俟平愈乃來。



 六月戊寅,免中都、河北等路嘗被河決水災軍民租稅。庚辰太白晝見。



 七月丙午,太白晝見,經天。壬子,秋獵。



 八月丙戌,次雙山子。



 九月己亥朔,還都。己酉,上謂宰臣曰:「朕今歲春水所過州縣,其
 小官多幹事,蓋朕前嘗有賞擢,故皆勉力。以此見專任責罰,不如用賞之有激勸也。」以河中尹田彥皋等為賀宋生日使,武器署令斜卯阿土為夏國生日使。



 十月乙亥,宋前主構殂。庚辰,祫享于太廟。庚寅,上謂宰臣曰:「朕觀唐史,惟魏徵善諫,所言皆國家大事,甚得諫臣之體。近時臺諫惟指摘一二細碎事,姑以塞責,未嘗有及國家大利害者,豈知而不言歟?無乃亦不知也。」宰臣無以對。



 十一月庚戌,以左副都點檢崇安為賀宋正旦使。甲寅,詔:「河水泛溢,農夫被災者,與免差稅一年。衛、懷、孟、鄭四州塞河勞役,併免今年差稅。庚申,平章政事崇尹致
 仕。甲子,上謂宰臣曰:「卿等老矣,殊無可以自代者乎,必待朕知而後進乎?」顧右丞張汝霖曰:「若右丞者亦石丞相所言也。平章政事襄及汝霖對曰:「臣等茍有所知,豈敢不言,但無人耳!」上曰:「春秋諸國分裂,土地褊小,皆稱有賢。卿等不舉而已。今朕自勉,庶幾致治,他日子孫,誰與共治者乎?」宰臣皆有慚色。



 十二月庚午,以翰林待制趙可為高麗生日使。丁丑,獵于近郊,壬午,宋遣使告哀。甲申,上諭宰臣曰:「人皆以奉道崇佛設齋讀經為福,朕使百姓無冤,天下安樂,不勝於彼乎?爾等居輔相之任,誠能匡益國家,使百姓蒙利,不惟身享其報,亦將施及
 子孫矣!」左丞斡特剌曰:「臣等敢不盡以為,第才不逮,不能稱職耳。」上曰:「人亦安能每事盡善,但加勉勵可也。」戊子,禁女直人不得改稱漢姓、學南人衣裝,犯者抵罪。



 二十八年正月丁酉朔,宋、高麗、夏遣使來賀。癸卯,遣宣徽使蒲察克忠為宋弔祭使。甲辰,如春水。



 二月乙亥,還都。乙丑,宋遣使獻先帝遺留物。癸巳,宋使朝辭,以所獻禮物中玉器五,玻璃器二十,及弓劍之屬使還遺宋,曰:「此皆爾國前主珍玩之物,所宜寶藏,以無忘追慕。今受之,義有不忍,歸告爾主,使知朕意也。」



 三月丁酉朔,萬春節,宋、高麗、夏遣使來賀。御慶和殿受群臣朝,復宴于神
 龍殿,諸王、公主以次捧觴上壽。上歡甚,以本國音自度曲。蓋言臨御久,春秋高,渺然思國家基緒之重,萬世無窮之託。以戒皇太孫,當修身養德,善于持守,及命太尉、左丞相克寧盡忠輔導之意。於是,上自歌之,皇太孫及克寧和之。極歡而罷。戊申,命隨朝六品、外路五品以上職事官,舉進士已在仕、才可居翰苑者,試制詔等文字三道,取文理優贍者補充學士院職任。應赴部求仕人,老病昏昧者,勒令致仕,止給半俸,更不遷官。甲寅,幸壽安宮。



 四月癸酉,命增外任小官及繁難局分承應人俸。丁丑,以陜西路統軍使孛術魯阿魯罕為參知政事。癸未,
 命建女直大學。



 五月丙午,制諸教授必以宿儒高才者充,給俸與丞簿等。戊申,宋使來謝弔祭。



 七月辛亥,尚書左丞粘割斡特剌罷。



 八月甲子朔,日有食之。辛未,還都。庚辰,上謂宰臣曰:「近聞烏底改有不順服之意,若遣使責問,彼或抵捍不遜,則邊境之事有不可已者。朕嘗思之,抬徠遠人,於國家殊無所益。彼來則聽之,不來則勿強其來,此前世羈縻之長策也。」參知政事孛術魯阿魯罕罷。壬午,以山東路統軍使完顏婆盧火為參知政事。甲申,上謂宰臣曰:「用人之道,當自其壯年心力精強時用之,若拘以資格,則往往至於耄老,此不思之甚也。阿
 魯罕使其早用,朝廷必得補助之力,惜其已衰老矣!凡有可用之材,汝等宜早思之。」



 九月甲午朔,以鷹坊使崇夔為夏國生日使。丙申,以安武軍節度使王克溫等為賀宋生日使。乙亥,秋獵。乙卯,還都。十月乙丑,京、府及節度州增置流泉務,凡二十八所。禁糠禪、瓢禪,其停止之家抵罪。乙酉,尚書省奏擬除授而拘以資格,上曰:「日月資考所以待庸常之人,若才行過人,豈可拘以常例?國家事務皆須得人,汝等不能隨才委使,所以事多不治。朕固不知用人之術,汝等務循資守格,不思進用才能,豈以才能見用,將奪己之祿位乎?不然,是無知人之明也。群
 臣皆曰:「臣等豈敢蔽賢,才識不逮耳。」上顧謂右丞張汝霖曰:「前世忠言之臣何多,今日何少也?」汝霖對曰:「世亂則忠言進,承平則忠言無所施。」上曰:「何代無可言之事,但古人知無不言,今人不肯言耳!」汝霖不能對。十一月戊戌,以改葬熙陵,詔中外。上謂侍臣曰:「凡修身者,喜怒不可太極,怒極則心勞,喜極則氣散,得中甚難,是故節其喜怒,以思安身。今宮中一歲未嘗責罰人也。」庚子,太白晝見。詔南京、大名府等處避水逃移不能復業者,官與津濟錢,仍量地頃畝給以耕牛。甲辰,以河中尹田彥皋等為賀宋正旦使。戊申,上謂宰臣曰:「制條以拘於舊
 律,間有難解之辭。夫法律歷代損益而為之,彼智慮不及而有乖違本意者,若行刪正,令眾易曉,有何不可。宜修之,務令明白。」有司奏重修上京御容殿,上謂宰臣曰:「宮殿制度,茍務華飾,必不堅固。今仁政殿遼時所建,全無華飾,但見它處歲歲修完,惟此殿如舊,以此見虛華無實者,不能經久也。今土木之工,滅裂尤甚,下則吏與工匠相結為姦,侵剋工物,上則戶工部官支錢度材,惟務茍辦,至有工役纔畢,隨即欹漏者,姦弊茍且,勞民費財,莫甚於此。自今體究,重抵以罪。」庚戌,上謂宰臣曰:「朕近讀《漢書》,見光武所為,人有所難能者。更始既害其兄
 伯升,當亂離之際,不思報怨,事更始如平日,人不見戚容,豈非人所難能乎?此其度量蓋將大有為者也,其他庸主豈可及哉。」右丞張汝霖曰:「湖陽公主奴殺人,匿主車中,洛陽令董宣從車中曳奴下,殺之。主人奏,光武欲殺宣,及聞宣言,意遂解,使宣謝主,宣不奉詔。主以言激怒光武,光武但笑而已,更賜宣錢三十萬。」上曰:「光武聞直言而怒解,可謂賢主矣,令宣謝主,則非也。高祖英雄大度,駕馭豪傑,起自布衣,數年而成帝業,非光武所及,然及即帝位,猶有布衣粗豪之氣,光武所不為也。」癸丑,幸太尉克寧第。



 十二月丙寅,以大理正移剌彥拱為高
 麗生日使。乙亥,上不豫。庚辰,赦天下。乙酉,詔皇太孫景攝政,居慶和殿東廡。丙戌,以太尉、左丞相徒單克寧為太尉兼尚書令,平章政事襄為尚書右相,右丞張汝霖為平章政事。參知政事完顏婆盧火罷,以戶部尚書劉暐為參知政事。戊子,詔尚書令徒單克寧、右丞相相襄、平章政事張汝霖宿於內殿。



 二十九年正月壬辰朔,上大漸,不能視朝。詔遣宋高麗、夏賀正旦使還。癸巳,上崩于福安殿,壽六十七。皇太孫即皇帝位。己亥,殯于大安殿。三月辛卯朔,上尊謚曰光天興運文德武功聖明仁孝皇帝,廟號世宗。四月乙酉,
 葬興陵。



 贊曰:世宗之立,雖由勸進,然天命人心之所歸,雖古聖賢之君,亦不能辭也。蓋自太祖以來,海內用兵,寧歲無幾。重以海陵無道,賦役繁興,盜賊滿野,兵甲並起,萬姓盼盼,國內騷然,老無留養之丁,幼無顧復之愛,顛危愁困,待盡朝夕。世宗久典外郡,明禍亂之故,知吏治之得失。即位五載,而南北講好,與民休息。於是躬節儉,崇孝弟,信賞罰,重農桑,慎守令之選,嚴廉察之責,卻任得敬分國之請,拒趙位寵郡縣之獻,孳孳為治,夜以繼日,可謂得為君之道矣!當此之時,群臣守職,上下相安,家給
 人足,倉廩有餘,刑部歲斷死罪,或十七人,或二十人,號稱「小堯舜」,此其效驗也。然舉賢之急,求言之切,不絕於訓辭,而群臣偷安茍祿,不能將順其美,以底大順,惜哉!



\end{pinyinscope}