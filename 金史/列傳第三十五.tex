\article{列傳第三十五}

\begin{pinyinscope}

 ○裴滿亨斡勒忠張大節子巖叟張亨韓錫鄧儼巨構賀揚庭閻公貞焦旭劉仲洙李完馬百祿楊伯珫劉璣兄珫康元弼移剌益



 裴滿亨,字仲通,本名河西,臨潢府人。其先世居遼海,祖諱虎山者,天輔間移屯東受降城,以禦夏人,後徙居臨
 潢。亨性敦敏習儒,大定間,收充奉職,世宗謂曰:「聞爾業進士舉,其勿忘為學也。」二十八年,擢第,世宗嘉之,升為奉御。一日問以上古為治之道,亨奏:「陛下欲興唐、虞之治,要在進賢,退不肖,信賞罰,薄徵斂而已。」章宗即位,諭之曰:「朕左右侍臣多以門第顯,惟爾由科甲進,且先朝信臣,國家利害,為朕盡言。」俄擢監察御史。內侍梁道兒恃恩驕橫,朝士側目,亨劾奏其姦。遷鎬王府尉,出為定國軍節度副使,三遷同知大名府事。先是,豪猾從衡,前政莫制,亨下車宣明約束,闔境帖然。承安四年,改河南路按察副使,就遷本路副統軍。中都、西京等路按察使。
 時世襲家豪奪民田,亨檢其實,悉還正之。泰和五年,改安武軍節度使。歲大雪,民多凍殍,亨輸己俸為之周贍,及勸率僚屬大姓同出物以濟。轉河東南北路按察使,卒於官。上聞而惜之,贈嘉議大夫,賻物甚厚。



 亨性尤謹密,出入宮禁數年,讜議忠言多所裨益,有稿則焚之,雖家人輩莫知也。所歷州郡,皆有政績可紀云。



 斡勒忠,本名宋浦,蓋州人也。習女直、契丹字,歷兵部、樞密院、尚書省令史,再轉大理寺知法,遷右三部司正。練達邊事,嘗奉命使北,歸致馬四千餘匹,詔褒諭之。大定二十六年,為監察御史,轉尚書省都事。章宗立,遷尚書
 兵部員外郎,出為滄州刺史。河東路提刑副使徒單移刺古舉以自代,改滕州刺史。嘗調發黃河船,數以稽期聽贖。授北京副留守,入為同簽樞密院事,兼沂王傅。承安二年,拜武寧軍節度使,致仕。泰和三年卒,年七十一。忠性敦愨,通法律,以直自守,不交權貴,故時譽歸之。



 張大節,字信之,代州五臺人。擢天德三年進士第,調崞縣丞。改東京市令。世宗判留務,甚愛重之。海陵修汴京,以大節領其役。世宗改元於遼東,或勸赴之,富貴可一朝遂,大節曰:「自有定分,何遽爾。」隨例補尚書省令史,擢秘書郎、大理司直。會左警巡使闕,世宗謂宰臣曰:「朕得
 其人矣。」遂授大節。俄以杖殺豪民為有司所劾,削一階解職。未幾,授同知洺州防禦使事。



 入為太府丞、工部員外郎。盧溝水嚙安次,承詔護視堤城。擢修內司使,推排東京路戶藉,人服其平。進工部郎中。時阜通監鑄錢法弊,與吏部員外郎麻珪蒞其事。積銅皆窳惡,或欲徵民先所給直,大節曰:「此有司受納之過,民何與焉。」以其事聞,卒得免征。就改戶部郎中,定襄退吏誣縣民匿銅者十八村,大節廉得其實,抵吏罪,民斲石頌之。召授工部侍郎,改戶部。世宗東巡,徙太府監,諭之曰:「侍郎與太府監品同,以從行支應籍卿辦耳。」尋為宋生日使,還授橫
 海軍節度使,過闕謁謝東宮,顯宗撫慰良久,曰:「萬事惟中可也。」因榜其公堂曰「惟中」。郡境有巨盜久不獲,大節以方略擒之。後河決於衛,橫流而東,滄境有九河故道,大節即相宜繕堤,水不為害。



 章宗即位,擢中都路都轉運使,因言河東賦重宜減,議者或不同,大節以他路田賦質之,遂命減焉。乞致仕,不許,徙知太原府,以并、代鄉郡,故優寵之。近郭有男子被殺者,聞其妻哭聲不哀,召而審之,果為姦夫所殺,人以為神。西山有晉叔虞祠,舊以施錢輸公使庫,大節還其廟以給營繕。選授河東路提刑使,未赴,留知大興府事,治有能名。閱歲,移知廣寧
 府,復請老,授震武軍節度使。部有銀冶,有司以為爭盜由此生,付河東、西京提刑司與州同議,皆以官榷為便,大節曰:「山澤之利,當與民共,且貧而無業者,雖嚴刑能禁其竊取乎?宜明諭民,授地輸課,則其游手者有所資,於官亦便。」上從其議。復乞致仕,許之,仍擢其子尚書刑部員外郎巖叟為忻州刺史,以便祿養。承安五年卒,年八十。



 大節素廉勤好學,能勵勉後進,自以得學于任倜,待倜子如親而加厚。又善弈棋,當世推為第一,嘗被召與禮部尚書張景仁弈。世宗嘗謂宰臣曰:「人多稱王翛能官,以朕觀之,凡事不肯盡心,一老姦耳。張大節賦性
 剛直,果於從政,遠在王翛之上,惜乎用之太晚。」又屢語近臣曰:「某某非不幹,然不及張大節忠實也。」其見知如此。



 巖叟,字孟弼,大節子也。大定十九年進士,調葭州司候判官,再除雄州觀察判官,補尚書省令史,除大理評事,再遷監察御史、同知河東北路轉運使事、中都路都轉運副使、刑部員外郎、忻州刺史,以父憂去官。起復大理少卿、河北東西大名等路按察轉運副使,累遷刑部侍郎,兼夔王傅,太常卿兼國子祭酒。大安三年,朝廷欲塞諸城門以為兵備,集三品官議於尚書省,巖叟曰:「塞門
 所以受兵,是任城而不任人。莫若遣兵擇將,背城疾戰。」時議多之。除鎮西軍節度使,移定國軍。貞祐二年,改昭義,復移沁南。逾年,按察司言其年老不任邊要,乃致仕,退寓洛陽,卒。



 張亨,字彥通,大興漷陰人。登皇統六年進士第,調樊山丞,以廉幹聞。授弘州軍事判官,歷巨鹿、宜川令。大定二年,補尚書省令史,除大理司直,累遷尚書左司郎中,授戶部侍郎,移吏部。擢中都路都轉運使,坐草場使鄧汝霖盜草失舉劾,解職,削一官。起授戶部尚書。世宗問宰臣曰:「御史中丞馬惠迪與張亨人才孰優?」平章政事張
 汝霖曰:「惠迪為人雖正,於事不敏,亨吏才極高。」上曰:「如汝父浩,於事明敏少有及者,但臨事多徇,若無此過則誠難得之賢相也。」時車駕車巡,費用百出,自遼以東泉貨甚少,計司患其不給,欲輦運以支調度,亨謂:「上京距都四千里,若挽錢而行,是率三而致一也,不獨枉費國用,無乃重勞民力乎。不若行會便法,使行旅便於囊橐,國家無轉輸之勞而用自足矣。」出為絳陽軍節度使。已而復謂宰臣曰:「漢人三品以上官常少得人,如張亨近令補外,頗為眾議所歸,以朕觀之,無甚過人。小官中豈無才能之士,第未知耳。」又曰:「亨嘗為左司,奏事多有脫
 略,是亦謬庸人也。」章宗即位,初置九路提刑司,時方重其選,上以亨為河東南北路提刑使,兼勸農採訪事。訪其利病,條為十三事以聞,上嘉納之。亨在職每事存大體、略苛細,御史以寬緩不事事劾之,降授蔡州防禦使。明年,遷南京路轉運使,轉知歸德府事,致仕。泰和二年卒,年七十八。亨才識強敏,明達吏事,終始有可稱云。



 韓錫,字難老,其先自析津徙薊之漁陽。祖貽愿,遼宣徽北院使。父秉休,歸朝,領忠正軍節度使。錫以蔭補閣門祗候。天會中,南伐,錫從軍掌禮儀,俄以母,老乃就監差。久之,授神銳軍都指揮使,入為宮苑使。天德元年,擢尚
 書工部員外郎,領燕都營繕。特賜胡礪榜進士及第,四遷尚書戶部侍郎,以母喪解。旋起復舊職,付金牌一、銀牌十、籍水手於山東。時蘇保衡為水軍都統制,趨杭州,俾錫部船三百會廣陵。適保衡敗還,喪船過半,令錫補足之。時水淺,船不得進,海陵遣使急責之,眾稍亡,錫召諸豪諭之曰:「今連保法嚴,逃將安往,縱一身偶脫,其如妻子何?」眾悟,亡者稍止。大定改元於遼東,錫奔赴行在,詔復前職。明年,授同知河間府事,引見於香閣,誡之曰:「聞皇族居彼者縱甚,卿當以法繩之。」錫下車宣布詔言,後無有撓政害民者。遷孟州防禦使,累拜絳陽軍節度
 使,改知濟南府事。告老,許之。明昌五年卒,年八十三。



 鄧儼,字子威,懿州宜民人也。天德三年,擢進士第。大定中,為左司員外郎、右司郎中,尋轉左司,掌機務者數年。有司奏使宋者,世宗命選漢官一人,參知政事梁肅以戶部侍郎王翛、工部侍郎張大節、左司郎中鄧儼對,世宗曰:「王翛、張大節苦無資歷,與左右司官辛苦不同,其命儼往。」嘗謂宰臣曰:「人言鄧儼用心不正,朕視儼奏事其心識甚明,在太府監心亦向公。」宰臣因奏儼明事機、有心力,於是擢戶部侍郎。翌日,復謂宰臣曰:「吏部掌銓選,當得通練人,可置儼於吏部。」因改命焉。累遷中都路
 都轉運使。明昌初,為戶部尚書。上命尚書省集百官議,如何使民棄末務本以廣儲蓄。儼言:「今之風俗競為侈靡,莫若定立制度,使貴賤、上下、衣冠、車馬、室宇、器用各有等差,裁抑婚姻喪葬過度之禮,罷去鄉社追逐無名之費,用度有節則蓄積日廣矣。」尋知歸德府事,致仕,卒。



 初,儼致仕復夤緣求進,上問左右:「鄧儼可復用乎?」平章政事完顏守貞曰:「儼有才力,第以謀身為心。」上曰:「朕亦知之。然儼可以誰比?」守貞曰:「臨事則不後於人,但多務自便耳。儼前乞致仕,陛下以其頗黠故許之,甚合眾議。今使復列于朝,恐風化從此壞矣。」上然之,遂不復用云。



 巨構,字子成,薊州平谷人。幼篤學,年二十登進士第。由信都丞察廉為石城令,補尚書省令史,授振武軍節度副使。改同提舉解鹽司事,以課增入為少府監丞。再遷知登聞檢院,兼都水少監。時右司郎中段珪卒,世宗曰:「是人甚明正可用,如巨構每事但委順而已。」二十五年,除南京副留守,上謂宰臣曰:「巨構外淳質而內明悟,第乏剛鯁耳。佐貳之任貴能與長官辨正,恐此人不能爾。若任以長官,必有可稱。」章宗即位,擢橫海軍節度使。承安五年致仕,卒。



 構性寬厚寡言,所治以鎮靜稱,性尤恬退,故人既貴不復往來,先遺以書則裁答寒溫而已。大
 定中,詔與近臣同經營香山行宮及佛舍,其近臣私謂構曰:「公今之德人,我欲舉奏,公行將大任矣。」構辭之。以廉慎守法在考功籍,始終無過云。



 賀揚庭,字公叟,曹州濟陰人也。登天德三年經義進士第,調范縣主簿兼尉,籍有治聲。大定十三年,由安肅令補尚書省令史,授沁南軍節度副使,入為監察御史,歷右司都事、戶部員外郎、侍御史、右司員外郎。世宗喜其剛果,謂揚庭曰:「南人礦直敢為,漢人性姦,臨事多避難。異時南人不習詞賦,故中第者少,近年河南、山東人中第者多,殆勝漢人為官。」俄以廉能遷戶部郎中,進官二
 階。頃之,授左司郎中,改刑部侍郎、山東東路轉運使。章宗即位,初置九路提刑司,驛召赴闕,授山東東西路提刑使。揚庭性疾惡,纖介不少容。明昌改元,詔諸路提刑使入見,親問所察事條,至揚庭則斥之曰:「爾何治之煩也。」明年,下除洺州防禦使,時歲歉民飢,揚庭諭蓄積之家令出所餘以糶之,飢者獲濟,洺人為之立石頌德。改陜西西路轉運使,表乞致仕,上曰:「揚庭能幹者也,當何如?」右丞劉瑋言其疾,遂許之。卒年六十七。



 贊曰:裴滿亨以進士選奉御,能陳唐、虞致治之道於宮庭燕私之地,又能斥中貴梁道兒之姦。斡勒忠以吏道
 致身,始終不交權貴。世宗自立於遼東,歸者如市,張大節獨守正不赴。韓錫出守河間,面諭皇族之居彼者恣睢不道,俾繩以法,佞者必希旨以市權,錫下車宣布告戒而已。是皆有識之士,不為富貴所移者也。巨構骫骳,賀揚庭骨鯁,大定於二人而屢評南北士習之優劣,亶其然乎。張亨始以繆庸見薄,晚以論列稱賞,亦砥礪之功歟。鄧儼專務謀身,上下稱黠,致仕又求進用,弗可改也夫。



 閻公貞,字正之,大興宛平人。大定七年擢進士第,調朝邑主簿。由普潤令補尚書省令史,察廉,升同知亳州防
 禦事,改中都左警巡使。以政績聞,遷同知武定軍節度使。明昌初,召為大理正,累進大理卿。承安元年,遷翰林侍讀學士,仍兼前職,命與登聞檢院賈益同看讀陳言文字。公貞居法寺幾十年,詳慎周密,未嘗有過舉。被命校定律令,多所是正,金人以為法家之祖云。



 焦旭,字明銳,沃州柏鄉人。第進士,調安喜主簿。再轉大興令,攝左警巡事,以杖親軍百人長,有司議其罪當杖決,世宗曰:「旭親民吏也,若因杖有官人復行杖之,何以行事?其令收贖。」改良鄉令。世宗幸春水,見石城、玉田令皆年老不治,謂宰臣曰:「縣令最親民,當得賢才。畿甸尚
 如此,天下可知矣。」平章政事石琚薦旭幹能可甄用,上然之,召為右警巡使。旭為人剛果自任,不避權勢。初,旭部民訴良,旭以無文據付本主,道逢監察御史訴其事,語涉訛亂,即收付旭,旭釋之不問,為御史所劾,削官兩階,杖百八十,出為大名府推官。尋授右三部檢法司正,代韓天和為監察御史。時御史臺言:「監察糾彈之司,天和諸科出身,難居是職。」上命別舉,中丞李晏薦旭剛正可任,遂授之,而改天和獲鹿令。章宗初即位,太傅克寧、右丞相襄請上出獵,旭劾奏其非,上慰諭之,為罷獵。明昌元年,登聞鼓院初設官,宰執奏司諫郭安民、補闕許
 安仁及旭皆堪擢用。改侍御史,四遷都水監,以治河防勞進官一階,授西京路轉運使,卒。旭性警敏,練達時政,與王翛,劉仲洙輩世稱能吏云。



 劉仲洙,字師魯,大興宛平人。大定三年,登進士第。歷龍門主簿、香河酒稅使,再調深澤令。縣近滹沱河,時秋成,水忽暴溢,仲洙極力護塞,竟無害。有盜夜發,居民震驚,仲洙率縣卒生執其一,餘眾遂潰,旦日掩捕皆獲。尋以廉能進官一階,升河北西路轉運司支度判官,入為刑部主事,六遷右司員外郎,俄轉吏部。世宗謂宰臣曰:「人有言語敏辯而庸常不正者,有語言拙訥而才智通達、
 存心向正者,如劉仲洙頗以才行見稱,然而口語甚訥也。」右丞張汝霖曰:「人之若是者多矣,願陛下深察之。」二十九年,出為祁州刺史,以六善為教,民化之。章宗即位,除中都、西京等路提刑副使。先是,田玨等以黨罪廢錮者三十餘家,仲洙知其冤,上書力辨,帝從之,乃復玨官爵而黨禁遂解。明昌二年,授并王傅、兼同知大同府事,尋改平陽,移德州防禦使。轉運郭邦傑、節度李晏皆舉仲洙以自代。升為定海軍節度使。歲饑,仲洙表請開倉,未報,先為賑貸,有司劾之,罪以贖論。時仲洙兄仲淵以罪責石州,仲洙上書請以萊易石,朝廷義而不許。久之,
 以年老乞致仕,累表方聽。泰和八年卒,年七十五。



 仲洙性剛直,果於從政,尤長於治民,所在皆有功迹,蓋一時之能吏云。



 李完,字全道,朔州馬邑人。經童出身,復登詞賦進士第。調澄城主簿,有遺愛,民為立祠。用廉,遷定襄令,召補尚書省令史。時以縣令闕人廉問,世宗選能吏八人按行天下,完其一也。明昌初,為監察御史。故事,臺令史以六部令史久次者補,吏皆同類,莫肯舉劾。完言:「尚書省令史,正隆間用雜流,大定初以太師張浩奏請,始純取進士,天下以為當。令乞以三品官子孫及終場舉人,委臺
 官辟用。」上納其言。擢尚書省都事,出為同知橫海軍節度使事、河間府治中。提刑司言:「完習法律,有治劇才,軍民無間語。」升沁州刺史,仍以璽書褒諭。遷同知廣寧府。初,遼濱民崔元入城飲不歸,求得尸於水中。有司執同飲者訊之,皆誣服,提刑司疑其冤,以獄畀完。完廉得其賊乃舟師也,遂免同飲人。改北京臨潢路提刑副使。承安二年,遷陜西西路轉運使,尋授南京路按察使,卒。完長於吏治,所至姦惡屏迹,民皆便之。



 馬百祿,字天錫,通州三河人。父柔德,天會初第進士,累遷翰林修撰,坐田玨黨免官,迨世宗朝解黨禁,復召用
 焉。百祿幼志學,事繼母以孝聞,登大定三年詞賦進士第,調武清主簿。由龍山令召補尚書省令史,不就,改榷貨副使、平陽府判官,入為國子博士。朝廷以宰縣日清白有治迹,特遷官一階,升同知北京路轉運事。委錄南北路刑獄,所至無冤。召為尚書戶部員外郎,與同知河北東路轉運事李京為中都等路推排使。明昌初,遷耀州刺史,吏民畏愛。提刑司以狀聞,授韓王傅、同知安武軍節度事。俄改兼同知興平軍,以提刑司復舉廉,升孟州防禦使,再遷南京路提刑使。御史臺以剛直能幹聞,轉知河中府。承安四年致仕,卒。謚曰貞忠。



 楊伯元,字長卿,開封尉氏人。登大定三年進士第,調郾城主簿。升榆次令,召為大理評事,累除定海軍節度副使,用廉,超授同知河東北路轉運事,入為尚書刑部員外郎,以憂免,起為遼州刺史。明昌元年,移涿州。久之,擢工部侍郎,四遷安武軍節度使。泰和三年致仕,卒。



 伯元以才幹多被委注,凡兩為推排定課使,累為審錄官,人稱其平。每有疑獄,必專遣決,明辯多中理。賜謚曰達。



 劉璣,字仲璋,益都人也。登天德三年進士第。大定初,為太常博士,改左拾遺,兼許王府文學。璣奏王府事,世宗責之曰:「汝職掌教道,何預奏事!」因命近侍諭旨永中曰:「
 卿有長史,而令文學奏事何也?後勿復爾。」累除同知漕運司事,嘗奏言:「漕戶顧直太高,虛費官物,宜約量裁損。若減三之一,歲可省官錢一十五萬餘貫。」世宗是其言。授戶部員外郎,條上便宜數事,世宗謂宰臣曰:「璣言河堤種柳可省每歲隄防之費,及言官錢利害,甚可取。前後戶部官往往偷延歲月,如璣者不可多得,卿等議其可者行之。璣向言漕運省費事,盡心公家,不厚賞無以勸來者。」乃賜錢三千貫。擢濰州刺史,徙知濟州。未幾,遷同知北京留守事,坐曲法放免奴婢訴良者,左降管州刺史。世宗謂宰臣曰:「璣為人何如?」參知政事程輝曰:「璣
 執強跋扈,嘗追濟南府官錢,以至委曲生意而害及平民。」上曰:「朕聞璣在北京,凡奴隸訴良,不問契券真偽,輒放為良,意欲徼福於冥冥,則在己之奴何為不放?」又曰:「璣放朕之家奴,意欲以此邀福,存心若是,不宜再用。」明昌二年,入為國子司業,乞致仕不許,轉國子祭酒,尋擢太常卿,以昏耄不任職為御史臺所糾罷。承安二年卒。年八十二。兄珫。



 珫字伯玉,幼名太平。以功臣子補閣門祗候,遭父喪求終制,會海陵篡立,不許,改充護衛。海陵忌宗室,珫坐與往來,斥居鄉里。世宗即位,珫晝夜兼馳上謁,世宗大悅,
 以為護衛十人長。往招宗敘、白彥敬、紇石烈志寧,皆相繼來附。還報,上喜其有功,呼其小字而謂之曰:「太平所至,庶幾能贊朕致太平矣。」改御院通進。與烏居仁等往南京發遣六宮百司,珫建議留尚書右丞紇石烈良弼經略淮右,餘皆北來,詔從之。丁母憂,起復,三遷武庫署令。車駕幸西京,留珫為中都總管判官。再轉近侍局使,遷太子少詹事,兼引進使,賜襲衣。未幾,為陜西統軍都監,賜廄馬、金帶,皇太子以馬與幣為贐。召為同知宣徽院事,遷太子詹事、右宣徽使,與張僅言典領昭德皇后園陵,襄事,太子贈以廄馬。轉左宣徽使,以疾求補外,除
 定海軍節度使,以其弟太府監瑋為同知宣徽院事。珫朝辭,上曰:「卿舊臣,今補外,寧不惻然。東萊瀕海,風物亦佳,卿到必得調養。朕用卿弟在近密,如見卿也。」仍賜廄馬、金帶、彩十端、絹百匹。卒官,年五十七。珫柩過京畿,敕有司致祭,賻銀三百兩、重綵三十端。



 康元弼,字輔之,大同雲中人。幼敏學,善屬文,登正隆二年進士第。調汝陽簿,改崇義軍節度判官。由垣曲縣令補尚書省令史,累遷同知河北西路轉運使事,召為大理丞。



 大定二十七年,河決曹、濮間,瀕水者多墊溺,朝延遣元弼往視,相其地如盎,而城在盎中,水易為害,請命
 於朝以徙之,卒改築於北原,曹人賴焉。出為弘州刺史,閱歲授大理少卿。先是,衛州為河所壞,增築蘇門以寓州治。水既退,民不樂遷,欲復歸衛,於是遣元弼按視,還言治故城便,遂復其舊。轉秘書少監,兼著作郎,改通州刺史,兼領漕事。章宗立,尊孝懿皇后為皇太后,以元弼舊臣詔充副衛尉。再轉大理卿,以喪去,起復為尚書刑部侍郎,兼鄆王傅,遷南京路轉運使。承安三年致仕,卒。



 移剌益,字子遷,本名特末阿不,中都路胡魯土猛安人也。以蔭補國史院書寫,積勞調徐州錄事,召為樞密院知法,三遷翰林修撰。時北邊有警,詔百官集尚書省議
 之,太尉克寧銳意用兵,益言天時未利,宜俟後圖。御史臺舉益剛正可任,遂兼監察御史。未幾,改戶部員外郎。明昌三年,畿內饑,擢授霸州刺史,同授刺史者十一人,既入謝,詔諭之曰:「親民之職,惟在守令,比歲民饑,故遣卿等往撫育之。其資序有過者有弗及者,朕不計此,但以材選,爾其知之。」既至,首出俸粟以食飢者,于是倅以下及郡人遞出粟以佐之,且命屬縣視以為法,多所全活。郡東南有堤久頹圮,水屢為害,益增修之,民以為便,為益立祠。升遼東路提刑副使。五年,宋主新立,詔以泗州當使客所經,守臣宜擇人,宰臣進擬數人,皆不合上
 意,上曰:「特末阿不安在?此人可也。」即授防禦使。召為尚書戶部侍郎,尋轉兵部。屬群牧人叛,命益同殿前都點檢兗往招降之。承安二年,邊鄙弗寧,上御便殿,召朝官四品以上入議,益謂「守為便。天子之兵當取萬全,若王師輕出,少有不利,非惟損大國之威,恐啟敵人侵玩之心。」出為山東西路轉運使。有敕使按鷹于山東,益奏:「乞止令調於近甸,何必驚遠方耳目。」書聞,上命有司治使者罪。遷河東南北路按察使。舊制,在位官有不任職,委所屬上司體訪。州府長貳幕職,許互相舉申。益上言以為:「傷禮讓之風,亦恐同官因之不睦,別生姦弊。乞止令
 按察司糾劾,似為得體」。又言:「隨路點軍官與富人飲會,公通獻遺,宜依准監臨官於所部內犯罪究治。」上皆納焉。泰和二年,卒于官。



 贊曰:閻公貞定金律令,楊伯元定金推排,人皆以平稱之,難矣。焦旭畿內小官,聽斷不受御史風指,遂罹深憲。大臣請人主遊獵,劾奏其非,為之罷獵,誠有古人之風焉。李完、康元弼無他足稱,完論臺令史一事,元弼論曹、衛兩城,各當其可。馬百祿初坐黨廢,晚著治跡。劉璣初以理財得幸,晚以曲法得罪,人有前後遭遇不同,而百祿求福不回,非璣所及也。劉珫以大定之立馳赴行在,
 雖終身榮寵,蓋一趨時之士耳。劉仲洙剛而訥於言,移剌益剛而敢言。益以志寧北伐為不可,仲洙釋田玨黨禍三十家。《語》曰:「剛毅木訥近仁。」豈不信哉!



\end{pinyinscope}