\article{列傳第四十七}

\begin{pinyinscope}

 ○完顏素蘭陳規許古



 完顏素蘭,一名翼,字伯揚,至寧元年策論進士也。貞祐初,累遷應奉翰林文字,權監察御史。二年,宣宗遷汴,留皇太子於燕都,既而召之,素蘭以為不可,平章高琪曰:「主上居此,太子宜從。且汝能保都城必完否?」素蘭曰:「完固不敢必,但太子在彼則聲勢俱重,邊隘有守則都城可無虞。昔唐明皇幸蜀,太子實在靈武,蓋將以繫天下
 之心也。」不從,竟召太子從。



 七月,車駕至汴,素蘭上書言事,略曰:「昔東海在位,信用讒諂,疏斥忠直,以致小人日進,君子日退,紀綱紊亂,法度益隳。風折城門之關,火焚市里之舍,蓋上天垂象以儆懼之也。言者勸其親君子、遠小人、恐懼修省,以答天變,東海不從,遂至亡滅。夫善救亂者必迹其亂之所由生,善革弊者必究其弊之所自起,誠能大明黜陟以革東海之政,則治安之效可指日而待也。陛下龍興,不思出此,輒議南遷,詔下之日,士民相率上章請留,啟行之日,風雨不時、橋梁數壞,人心天意亦可見矣。此事既往,豈容復追,但自今尤宜戒慎,
 覆車之轍不可引轅而復蹈也。」



 又曰:「國家不可一日無兵,兵不可一日無食。陛下為社稷之計,宮中用度皆從貶損,而有司復多置軍官,不恤妄費,甚無謂也。或謂軍官之眾所以張大威聲,臣竊以為不然。不加精選而徒務其多,緩急臨敵其可用乎?且中都惟其糧乏,故使車駕至此。稍獲安地,遂忘其危而不之備,萬一再如前日,未知有司復請陛下何之也。」



 三年正月,素蘭自中都計議軍事回,上書求見,乞屏左右。上遣人諭之曰:「屏人奏事,朕固常爾。近以游茂因緣生疑間之語,故凡有所引見,必令一近臣立侍,汝有封章,亦無患不密也。」尋召至
 近侍局,給紙劄令書所欲言,書未及半,上出御便殿見之,悉去左右,惟近侍局直長趙和和在焉。素蘭奏曰:「臣聞興衰治亂有國之常,在所用之人如何耳。用得其人,雖衰亂尚可扶持,一或非才,則治安亦亂矣。向者颭軍之變,中都帥府自足剿滅,朝廷乃令移剌刺塔不也等招誘之,使帥府不敢盡其力,既不能招,愈不可制矣。至於伯德文哥之叛,帥府方議削其權,而朝廷傳旨俾領義軍,文哥由是益肆,改除之令輒拒不受,不臣之狀亦顯矣。帥府方且收捕,而朝廷復赦之,且不令隸帥府。國家付方面於重臣,乃不信任,顧養叛賊之姦,不知誰為陛
 下畫此計者。臣自外風聞,皆平章高琪之意,惟陛下裁察。」上曰:「汝言皆是。文哥之事,朕所未悉,誠如所言,朕肯赦之乎?且汝何以知此事出於高琪?」素蘭曰:「臣見文哥牒永清副提控劉溫云:『所差人張希韓至自南京,道副樞平章處分,已奏令文哥隸大名行省,勿復遵中都帥府約束』。溫即具言於帥府。然則,罪人與高琪計結明矣。」上頷之。素蘭續奏曰:「高琪本無勛勞,亦無公望,向以畏死故擅誅胡沙虎,蓋出無聊耳。一旦得志,妒賢能,樹姦黨,竊弄國權,自作威福。去歲,都下書生樊知一者詣高琪言:『颭軍不可信,恐終作亂。』遂以刀杖決殺之,自是無
 復敢言軍國利害者。宸聰之不通,下情之不達,皆此人罪也。及颭軍為變,以黨人塔不也為武寧軍節度使往招之,已而無成,則復以為武衛軍使。塔不也何人,且有何功,而重用如此。以臣觀之,此賊變亂紀綱,戕害忠良,實有不欲國家平治之意。昔東海時,胡沙虎跋扈無上,天下知之,而不敢言,獨臺官烏古論德升、張行信彈劾其惡,東海不察,卒被其禍。今高琪之姦,過於胡沙虎遠矣。臺諫職當言責,迫於兇威,噤不敢忤。然內外臣庶見其恣橫,莫不扼腕切齒,欲一剚刃,陛下何惜而不去之耶。臣非不知言出而患至,顧臣父子迭仕聖朝,久食厚
 祿,不敢偷安。惟陛下斷然行之,社稷之福也。」上曰:「此乃大事,汝敢及之,甚善。」素蘭復奏:「丞相福興,國之勳舊,乞召還京,以鎮雅俗,付左丞彖多以留後事,足也。」上曰:「如卿所言,二人得無相惡耶?」素蘭曰:「福興、彖多同心同德,無不協者。」上曰:「都下事殷,恐丞相不可輟。」素蘭曰:「臣聞朝廷正則天下正,不若令福興還,以正根本。」上曰:「朕徐思之。」素蘭出,上復戒曰:「今日與朕對者止汝二人,慎無泄也。」厥後,上以素蘭屢進直言,命再任監察御史。



 四年三月,言:「臣近被命體問外路官,廉幹者擬不差遣,若懦弱不公者罷之,具申朝廷,別議擬注。臣伏念彼懦弱不
 公之人雖令罷去,不過止以待闕者代之,其能否又未可知,或反不及前官,蓋徒有選人之虛名,而無得人之實跡。古語曰:『縣令非其人,百姓受其殃。』今若後官更劣,則為患滋甚,豈朝廷恤民之意哉?夫守令,治之本也。乞令隨朝七品、外路六品以上官,各舉堪充司縣長官者,仍明著舉官姓名,他日察其能否,同定賞罰,庶幾其可。議者或以閡選法、紊資品為言,是不知方今之事與平昔不同,豈可拘一定之法,坐視斯民之病而不權宜更定乎。」詔有司議行之。



 時哀宗為皇太子,春宮所設師保贊諭之官多非其人,於是素蘭上章言:「臣聞太子者天
 下之本也,欲治天下先正其本,正本之要無他,在選人輔翼之耳。夫生于齊者能齊言而不能楚語,未習之故也。人之性亦在夫習之而已。昔成王在襁褓中,即命周、召以為師保,戒其逸豫之心,告以持守之道,終之功光文、武,垂休無窮。欽惟陛下順天人之心,預建春宮。皇太子仁孝聰明出于天資,總制樞務固已綽然有餘,倘更選賢如周、召之儔者使之夾輔,則成周之治不足侔矣。」上稱善。未幾,擢為內侍局直長,尋遷諫議大夫,進侍御史。



 興定二年四月,以蒲鮮萬奴叛,遣素蘭與近侍局副使內族訛可同赴遼東,詔諭之曰:「萬奴事竟不知果何
 如,卿等到彼當得其詳,然宜止居鐵山,若復遠去,則朕難得其耗也。」又曰:「朕以訛可性頗率易,故特命卿偕行,每事當詳議之。」素蘭將行,上言曰:「臣近請宣諭高麗復開互市事,聞以詔書付行省必蘭出。若令行省就遣諭之,不過鄰境領受,恐中間有所不通,使聖恩不達於高麗,高麗亦無由知朝廷本意也。況彼世為籓輔,未嘗闕臣子禮,如遣信使明持恩詔諭之,貸糧、開市二者必有一濟。茍俱不從,則其曲在彼,然後別議圖之可也。」上是其言,於是遣典客署書表劉丙從行。及還,授翰林待制。



 正大元年正月,詔集群臣議修復河中府,素蘭與陳規
 等奏其未可,語在《規傳》。是月,轉刑部郎中。時南陽人布陳謀反,坐繫者數百人,司直白華言於素蘭曰:「此獄詿誤者多,新天子方務寬大,他日必再詔推問,比得昭雪,死於榜笞之下者多矣。」素蘭命華及檢法邊澤分別當死、當免者,素蘭以聞,止坐首惡及擬偽將相者數人,餘悉釋之。



 八月,權戶部侍郎。二年三月,授京西司農卿,俄改司農大卿,轉御史中丞。七年七月,權元帥右都監、參知政事,行省於京兆。未幾,遷金安軍節度使,兼同、華安撫使。既而召還朝,行至陜被圍,久之,亡奔行在,道中遇害。



 素蘭蒞官以修謹得名,然苛細不能任大事,較之輩
 流頗可稱。自擢為近侍局直長,每進言多有補益。其居父喪,不飲酒,廬墓三年,時論以為難。



 陳規,字正叔,絳州稷山人。明昌五年詞賦進士,南渡為監察御史。貞祐三年十一月,上章言:「參政侯摯初以都西立功,獲不次之用,遂自請鎮撫河北。陛下遽授以執政,蓋欲責其報效也。既而盤桓西山,不能進退,及召還闕,自當辭避,乃恬然安居,至於按閱倉庫,規畫榷酤,豈大臣所宜親。方今疆土日蹙,將帥乏人,士不選練,冗食猥多,守令貪殘,百姓流亡,盜賊滋起,災變不息,則當日夜講求其故,啟告陛下者也,而摯未嘗及之。伏願陛下
 特賜省察,量其才分別加任使,無令負天下之謗。」不報。又言:「警巡使馮祥進由刀筆,無他才能,第以慘刻督責為事。由是升職,恐長殘虐之風,乞黜退以勵餘者。」詔即罷祥職,且諭規曰:「卿知臣子之分,敢言如此,朕甚嘉之。」



 四年正月,上言:「伏見沿河悉禁物斛北渡,遂使河北艱食,人心不安。昔秦、晉為仇,一遇年饑則互輸之粟。今聖主在上,一視同仁,豈可以一家之民自限南北,坐視困餒而不救哉。況軍民效死禦敵,使復乏食,生亦何聊,人心一搖,為害不細。臣謂宜於大陽、孟津等渡委官閱視,過河之物,每石官收不過其半,則富有之家利其厚息,
 輻湊而往,庶幾公私俱足。」宰執以河南軍儲為重,詔兩渡委官取其八,二以與民,至春澤足,大兵北還,乃依規請。制可。



 三月,上言:「臣因巡按至徐州。去歲河北紅襖盜起,州遣節度副使紇石烈鶴壽將兵討之,而乃大掠良民家屬為驅,甚不可也。乞明敕有司,凡鶴壽所虜俱放免之,餘路軍人有掠本國人為驅者,亦乞一體施行,庶幾河朔有所係望,上恩無有極已。」事下尚書省,命徐州、歸德行院拘括放之,有隱匿者坐掠人為奴婢法,仍許諸人告捕,依令給賞,被虜人自訴者亦賞之。



 四月,上言:「河北瀕河州縣,率距一舍為一寨,籍居民為兵。數寨置
 總領官一人,並以宣差從宜為名。其人大抵皆閑官,義軍之長、偏裨之屬尤多無賴輩,征逐宴飲取給于下,日以為常。及敵至則伏匿不出,敵去騷擾如初。此輩小人假以重柄,朝廷號令威權無乃太輕乎。臣謂宜皆罷之,第委宣撫司從宜措畫足矣。」制可。



 七月,上章言:



 陛下以上聖寬仁之姿,當天地否極之運,廣開言路以求至論,雖狂妄失實者亦不坐罪。臣忝耳目之官,居可言之地,茍為緘默,何以仰酬洪造。謹條陳八事,願不以人微而廢之,即無可採,乞放歸山林以懲尸祿之罪。



 一曰責大臣以身任安危。今北兵起自邊陲,深入吾境,大小之戰
 無不勝捷,以致神都覆沒,翠華南狩,中原之民肝腦塗地,大河以北莽為盜區。臣每念及此,驚怛不已。況宰相大臣皆社稷生靈所繫以安危者,豈得不為陛下憂慮哉。每朝奏議,不過目前數條,特以碎末,互生異同,俱非救時之急者。況近詔軍旅之務,專委樞府,尚書省坐視利害,泛然不問,以為責不在己,其於避嫌周身之計則得矣,社稷生靈將何所賴。古語云:「疑則勿任,任則勿疑。」又曰:「謀之欲眾,斷之欲獨。」陛下既以宰相任之,豈可使親其細而不圖其大者乎。伏願特同睿斷,若軍伍器械、常程文牘即聽樞府專行,至于戰守大計、征討密謀皆
 須省院同議可否,則為大臣者知有所責,而天下可為矣。



 二曰任臺諫以廣耳目。人主有政事之臣,有議論之臣。政事之臣者宰相執政,和陰陽,遂萬物,鎮撫四夷,親附百姓,與天子經綸於廟堂之上者也。議論之臣者諫官御史,與天子辨曲直、正是非者也。二者豈可偏廢哉。昔唐文皇制中書門下入閣議事皆令諫官隨之,有失輒諫。國朝雖設諫官,徒備員耳,每遇奏事皆令迴避。或兼他職,或為省部所差,有終任不覿天顏、不出一言而去者。雖有御史,不過責以糾察官吏、照刷案牘、巡視倉庫而已,其事關利害或政令更革,則皆以為機密而不
 聞。萬一政事之臣專任胸臆、威福自由,或掌兵者以私見敗事機,陛下安得而知之。伏願遴選學術言夾博、通曉世務、骨鯁敢言者以為臺諫,凡事關利害皆令預議,其或不當,悉聽論列,不許兼職及充省部委差,茍畏徇不言則從而黜之。



 三曰崇節儉以答天意。昔衛文公乘狄人滅國之餘,徙居楚丘,纔革車三十兩,乃躬行儉約,冠大帛之冠,衣大布之衣,季年致騋牝三千,遂為富庶。漢文帝承秦、項戰爭之後,四海困窮,天子不能具鈞駟,乃示以敦朴,身衣弋綈,足履革舄,未幾天下富安,四夷咸服。國家自兵興以來,州縣殘毀,存者復為土寇所擾,獨
 河南稍完,然大駕所在,其費不貲,舉天下所奉責之一路,顧不難哉。賴陛下慈仁,上天眷佑,蝗災之餘而去歲秋禾、今年夏麥稍得支持。夫應天者要在以實,行儉者天必降福,切見宮中及東宮奉養與平時無異,隨朝官吏、諸局承應人亦未嘗有所裁省。至於貴臣、豪族、掌兵官,莫不以奢侈相尚,服食車馬惟事紛華。今京師鬻明金衣服及珠玉犀象者日增於舊,俱非克己消厄之道。願陛下以衛文公、漢文帝為法,凡所奉之物痛自樽節,罷冗員,減浮費,戒豪侈,禁戢明金服飾,庶皇天悔禍,太平可致。



 四曰選守令以結民心。方今舉天下官吏軍兵
 之費、轉輸營造之勞,皆仰給河南、陜西。加之連年蝗旱,百姓薦饑,行賑濟則倉廩懸乏,免徵調則用度不足,欲其實惠及民,惟得賢守令而已。當賦役繁殷、期會促迫之際,若措畫有方則百姓力省而易辦,一或乖謬有不勝其害者。況縣令之弊無甚于今,由軍衛監當進納勞效而得者十居八九,其桀黠者乘時貪縱,庸懦者權歸猾吏。近雖遣官廉察,治其姦濫,易其疲軟,然代者亦非選擇,所謂除狼得虎也。伏乞明敕尚書省,公選廉潔無私、才堪牧民者,以補州府官。仍清縣令之選,及責隨朝七品,外任六品以上官各保堪任縣令者一員,如他日
 犯贓並從坐。其資歷已係正七品,及見任縣令者,皆聽寄理,俟秩滿升遷。復令監察以時巡按,有不法及不任職者究治之,則實惠及民而民心固矣。



 五曰博謀群臣以定大計。比者徙河北軍戶百萬餘口于河南,雖革去冗濫而所存猶四十二萬有奇,歲支粟三百八十餘萬斛,致竭一路終歲之斂,不能贍此不耕不戰之人。雖無邊事,亦將坐困,況兵事方興,未見息期耶。近欲分布沿河,使自種殖,然游惰之人不知耕稼,群飲賭博習以成風,是徒煩有司徵索課租而已。舉數百萬眾坐糜廩給,緩之則用闕,急之則民疲,朝遷惟此一事已不知所處,
 又何以待敵哉。是蓋不審於初,不計其後,致此誤也。使初遷時去留從其所願,則欲來者是足以自贍之家,何假官廩,其留者必有避難之所,不必強遣,當不至今日措畫之難。古昔人君將舉大事,則謀及乃心,謀及卿士、庶人、卜筮,乞自今凡有大事必令省院臺諫及隨朝五品以上官同議為便。



 六曰重官賞以勸有功。陛下即位以來,屢沛覃恩以均大慶,不吝官爵以激人心,至有未滿一任而併進十級,承應未出職而已帶驃騎榮祿者,冗濫之極至于如此,復開鬻爵進獻之門,然則被堅執銳效死行陣者何所勸哉。官本虛名,特出於人主之口,
 而天下之人極意趨慕者,以朝廷愛重耳。若不計勛勞,朝授一官,暮升一職,人亦將輕之而不慕矣。已然之事既不可咎,伏願陛下重惜將來,無使公器為尋常之具,功賞為僥倖所乘。又今之散官動至三品,有司艱於遷授,宜於減罷八資內量增階數,易以美名,庶幾歷官者不至于太驟,而國家恩權不失之太輕矣。



 七曰選將帥以明軍法。夫將者國之司命,天下所賴以安危者也。舉萬眾之命付之一人,呼吸之間以決生死,其任顧不重歟?自北兵入境,野戰則全軍俱殃,城守則闔郡被屠,豈皆士卒單弱、守備不嚴哉,特以庸將不知用兵之道而
 已。古語云:「三辰不軌,取士為相。四夷交侵,拔卒為將。」今之將帥,大抵先論出身官品,或門閥膏粱之子,或親故假託之流,平居則意氣自高,遇敵則首尾退縮,將帥既自畏怯,士卒夫誰肯前。又居常裒刻,納其饋獻,士卒因之以擾良民而莫可制。及率之應敵,在途則前後亂行,屯次則排門擇屋,恐逼小民,恣其求索,以此責其畏法死事,豈不難哉。況今軍官數多,自千戶而上,有萬戶、有副統、有都統、有副提控,十羊九牧,號令不一,動相牽制。切聞國初取天下,元帥而下,惟有萬戶,所統軍士不下數萬人,專制一路,豈在多哉?多則難擇,少則易精。今之
 軍法,每二十五人為一謀克,四謀克為一千戶,謀克之下有蒲輦一人、旗鼓司火頭五人,其任戰者纔十有八人而已。又為頭目選其壯健以給使令,則是一千戶所統不及百人,不足成其隊伍矣。古之良將常與士卒同甘苦,今軍官既有俸廩,又有券糧,一日之給兼數十人之用。將帥則豐飽有餘,士卒則飢寒不足,曷若裁省冗食而加之軍士哉。伏乞明敕大臣,精選通曉軍政者,分詣諸路,編列隊伍,要必五十人為一謀克,四謀克為一千戶,五千戶為一萬戶,謂之散將。萬人設一都統,謂之大將,總之帥府。數不足者皆併之,其副統、副提控及無
 軍虛設都統、萬戶者悉罷省。仍敕省院大臣及內外五品以上,各舉方略優長,武勇出眾、材堪將帥者一二人,不限官品,以充萬戶以上都統、元帥之職。千戶以下,選軍中有謀略武藝為眾所服者充。申明軍法,居常教閱,必使將帥明於奇正虛實之數,士卒熟于坐作進退之節。至于弓矢鎧仗須令自負,習於勞苦。若有所犯,必刑無赦。則將帥得人,士氣日振,可以待敵矣。



 八曰練士卒以振兵威。昔周世宗常曰:「兵貴精而不貴多,百農夫不能養一戰士,奈何朘民脂膏養此無用之卒。茍健懦不分,眾何以勸。」因大搜軍卒,遂下淮南,取三關,兵不血刃,
 選練之力也。唐魏徵曰:「兵在以道御之而已。御壯健足以無敵于天下,何取細弱以增虛數。」比者凡戰多敗,非由兵少,正以其多而不分健懦,故為敵所乘,懦者先奔,健者不能獨戰而遂潰,此所以取敗也。今莫若選差習兵公正之官,將已籍軍人隨其所長而類試之。其武藝出眾者別作一軍,量增口糧,時加訓練,視等第而賞之。如此,則人人激厲,爭效所長,而衰懦者亦有可用之漸矣。昔唐文皇出征,常分其軍為上中下,凡臨敵則觀其強弱,使下當其上,而上當其中,中當其下。敵乘下軍不過奔逐數步,而上軍中軍已勝其二軍,用是常勝。蓋古
 之將帥亦有以懦兵委敵者,要在預為分別,不使混淆耳。



 上覽書不悅,詔付尚書省詰之。宰執惡其紛更諸事,謂所言多不當。於是規惶懼待罪,詔諭曰:「朕始以規有放歸山林之語,故令詰之,乃辭以不職忌諱,意謂朕惡其言而怒也。朕初無意加罪,其令御史臺諭之。」尋出為徐州帥府經歷官。



 正大元年,召為右司諫,數上章言事,尋權吏部郎中。時詔群臣議修復河中府,規與楊雲翼等言:「河中今為無人之境,陜西民力疲乏,修之亦不能守,不若以見屯軍士量力補治,待其可守即修之未晚也。」從之。未幾,坐事解職。初,吏部尚書趙伯成坐銓選吏
 員出身王京與進士王著填開封警巡判官見闕,為京所訟免官,規亦坐之。是年十一月,改充補闕。十二月,言將相非材,且薦數人可用者。



 二年正月,規及臺諫同奏五事:一,乞尚書省提控樞密院,如大定、明昌故事。二,簡留親衛軍。三,沙汰冗軍,減行樞密院、帥府。四,選大臣為宣撫使,招集流亡以實邊防。五,選官置所,議一切省減。略施行之。



 四月,以大旱詔規審理冤滯,臨發上奏:「今河南一路便宜、行院、帥府、從宜凡二十處,陜西行尚書省、帥府五,皆得以便宜殺人,冤獄在此,不在州縣。」又曰:「雨水不時則責審理,然則職燮理者當何如?」上善其言
 而不能有為也。



 十一月,上召完顏素蘭及規入見,面諭曰:「宋人輕犯邊界,我以輕騎襲之,冀其懲創告和,以息吾民耳。宋果行成,尚欲用兵乎。卿等當識此意。」規進曰:「帝王之兵貴於萬全,昔光武中興,所征必克,猶言『每一出兵,頭須為白』。兵不妄動如此。」上善之。四年三月,上召群臣喻以陜西事曰:「方春北方馬漸羸瘠,秋高大勢併來,何以支持。朕已喻合達盡力決一戰矣,卿等以為如何?」又言和事無益,撒合輦力破和議,賽不言:「今已遣和使,可中輟乎。」餘皆無言,規獨進曰:「兵難遙度,百聞不如一見。臣嘗任陜西官,近年又屢到陜西,兵將冗懦,恐不
 可用,未如聖料。」言未終,烏古論四和曰:「陳規之言非是,臣近至陜西,軍士勇銳,皆思一戰。」監察御史完顏習顯從而和之,上首肯,又泛言和事。規對曰:「和事固非上策,又不可必成,然方今事勢不得不然。使彼難從,猶可以激厲將士,以待其變。」上不以為然。明日,又令集議省中,欲罷和事,群臣多以和為便,乃詔行省斟酌發遣,而事竟不行。



 十月,規與右拾遺李大節上章,劾同判大睦親事撒合輦諂佞,招權納賄及不公事。由是撒合輦竟出為中京留守,朝廷快之。五年二月,又與大節言三事:「一,將帥出兵每為近臣牽制,不得專輒。二,近侍送宣傳旨,
 公受賂遺,失朝廷體,可一切禁絕。三,罪同罰異,何以使人。」上嘉納焉。



 初,宣宗嘗召文繡署令王壽孫作大紅半身繡衣,且戒以勿令陳規知。及成,進,召壽孫問曰:「曾令陳規輩知否?」壽孫頓首言:「臣侍禁庭,凡宮省大小事不敢為外人言,況親被聖訓乎。」上因歎曰:「陳規若知,必以華飾諫我,我實畏其言。」蓋規言事不假借,朝望甚重,凡宮中舉事,上必曰:「恐陳規有言。」一時近臣切議,惟畏陳正叔耳,挺然一時直士也。後出為中京副留守,未赴,卒,士論惜之。



 規博學能文,詩亦有律度。為人剛毅質實,有古人風,篤於學問,至老不廢。渾源劉從益見其所上八
 事,歎曰:「宰相材也。」每與人論及時事輒憤惋,蓋傷其言之不行也。南渡後,諫官稱許古、陳規,而規不以訐直自名,尤見重云。死之日,家無一金,知友為葬之。子良臣。



 許古,字道真,汾陽軍節度使致仕安仁子也。登明昌五年詞賦進士第。貞祐初,自左拾遺拜監察御史。時宣宗遷汴,信任丞相高琪,無恢復之謀,古上章曰:



 自中都失守,廟社陵寢、宮室府庫,至于圖籍重器,百年積累,一朝棄之。惟聖主痛悼之心至為深切,夙夜思懼所以建中興之功者,未嘗少置也。為臣子者食祿受責,其能無愧乎!且閭閻細民猶顒望朝廷整訓師徒,為恢復計。而今
 才聞拒河自保,又盡徙諸路軍戶河南,彼既棄其恒產無以自生,土居之民復被其擾,臣不知誰為此謀者。然業已如是,但當議所以處之,使軍無妄費,民不至困窮則善矣。



 臣聞安危所繫,在於一相,孔子稱:「危而不持,顛而不扶,則將焉用?」事勢至此,不知執政者每對天顏,何以仰答清問也。今之所急,莫若得人,如前御史大夫裴滿德仁、工部尚書孫德淵,忠諒明敏,可以大用,近皆許告老,願復起而任之,必能有所建立以利國家。太子太師致仕孫鐸,雖頗衰疾,如有大議猶可賜召,或就問之。人才自古所難,凡知治體者皆當重惜,況此耆舊,豈宜
 輕棄哉。若乃臨事不盡其心,雖盡心而不明於理,得無益、失無損者,縱其尚壯,亦安所用。方時多難,固不容碌碌之徒備員尸素,以塞賢路也。惟陛下宸衷剛斷,黜陟一新,以幸天下。臣前為拾遺時,已嘗備論擇相之道,乞取臣前奏并今所言,加審思焉。



 臣又聞將者民之司命,國家安危所系,故古之人君必重其選,為將者亦必以天下為己任。夫將者貴謀而賤戰,必也賞罰使人信之而不疑,權謀使人由之而不知,三軍奔走號令以取勝,然後中心誠服而樂為之用。邇來城守不堅,臨戰輒北,皆以將之不才故也。私於所暱,賞罰不公,至於眾怨,而
 懼其生變,則撫摩慰籍,一切為姑息之事。由是兵輕其將,將畏其兵,尚能使之出死力以禦敵乎?願令腹心之臣及閑於兵事者,各舉所知,果得真才,優加寵任,由戰功可期矣。如河東宣撫使胥鼎、山東宣撫使完顏弼、涿州刺史內族從坦,昭義節度使必蘭阿魯帶,或忠勤勇幹,或重厚有謀,皆可任之,以扞方面。



 又曰:



 河北諸路以都城既失,軍戶盡遷,將謂國家舉而棄之,州縣官往往逃奔河南。乞令所在根括,立期遣還,違者勿復錄用。未嘗離任者議加恩賚,如願自效河北者亦聽陳請,仍先賞之,減其日月。州縣長貳官並令兼領軍職,許擇軍中
 有才略膽勇者為頭目,或加爵命以收其心,能取一府者即授以府長官,州縣亦如之,使人懷復土之心。別遣忠實幹濟者,以文檄官賞招諸脅從人,彼既苦於敵役,來者必多,敵勢當自削。有司不知出此,而但為清野計,事無緩急惟期速辦,今晚禾十損七八,遠近危懼,所謀可謂大戾矣。



 又曰:



 京師諸夏根本,況今常宿重兵,緩急征討必由於此,平時尚宜優於外路,使百姓有所蓄積,雖在私室猶公家也。今有司搜括餘糧,致轉販者無復敢入,宜即止之。臣頃看讀陳言,見其盡心竭誠以吐正論者,率皆草澤疏賤之人,況在百僚,豈無為國深憂進
 章疏者乎?誠宜明敕中外,使得盡言不諱,則太平之長策出矣。



 詔付尚書省,略施行焉。



 尋遷尚書左司員外郎,兼起居注,無何,轉右司諫。時丞相高琪立法,職官有犯皆的決,古及左司諫抹捻胡魯剌上言曰:「禮義廉恥以治君子,刑罰威獄以治小人,此萬世不易論也。近者朝廷急於求治,有司奏請從權立法:職官有犯應贖者亦多的決。夫爵祿所以馭貴也,貴不免辱,則卑賤者又何加焉。車駕所駐非同征行,而凡科征小過皆以軍期罪之,不已甚乎。陛下仁恕,決非本心,殆有司不思寬靜可以措安,而專事督責故耳。且百官皆朝廷遴選,多由文
 行、武功、閥閱而進,乃與凡庶等,則享爵祿者亦不足為榮矣。抑又有大可慮者,為上者將曰官猶不免,民復何辭,則苛暴之政日行。為下者將曰彼既亦然,吾復何恥,則陵犯之心益肆。其弊豈勝言哉。伏願依元年赦恩『刑不上大夫』之文,削此一切之法,幸甚。」上初欲行之,而高琪固執以為不可,遂寢。



 四年,以右司諫兼侍御史。時大兵越潼關而東,詔尚書省集百官議,古上言曰:「兵踰關而朝廷甫知,此蓋諸將欺蔽罪也。雖然,大兵駐閿鄉境,數日不動,意者恐吾河南之軍逆諸前,陜西之眾議其後,或欲先令覘者伺趨向之便,或以深入人境非其地
 利而自危,所以觀望未遽進也。此時正宜選募銳卒併力擊之,且開其歸路,彼既疑惑,遇敵必走,我眾從而襲之,其破必矣。」上以示尚書省,高琪沮其議,遂不行。是月,始置招賢所,令古等領其事。



 興定元年七月,上聞宋兵連陷贛榆、漣水諸縣,且獲偽檄,辭多詆斥,因諭宰臣曰:「宋人構禍久矣,朕姑含容者,眾慮開兵端以勞吾民耳。今數見侵,將何以處,卿等其與百官議。」於是集眾議于都堂,古曰:「宋人孱弱,畏我素深,且知北兵方強,將恃我為屏蔽,雖時跳梁,計必不敢深入,其侮嫚之語,特市井屠沽兒所為,烏足較之。止當命有司移文,諭以本朝累
 有大造,及聖主兼愛生靈意。彼若有知,復尋舊好,則又何求。其或怙惡不悛,舉眾討之,顧亦未晚也。」時預議者十餘人,雖或小異而大略則一,既而丞相高琪等奏:「百官之議,咸請嚴兵設備以逸待勞,此上策也。」上然之。時朝廷以諸路把軍官時有不和不聽,更相訴訟,古上言曰:「臣以為善者有勸,惡者有懲,國之大法也。茍善惡不聞,則上下相蒙,懲勸無所施矣。」上嘉納之。



 古以朝廷欲舉兵伐宋,上疏諫曰:「昔大定初,宋人犯宿州,已而屢敗,世宗料其不敢遽乞和,乃敕元帥府遣人議之,自是太平幾三十年。泰和中,韓侂胄妄開邊釁,章宗遣駙馬僕
 撒揆討之。揆慮兵興費重不能久支,陰遣侂胄族人齎乃祖琦畫像及家牒,偽為歸附,以見丘崇,因之繼好,振旅而還。夫以世宗、章宗之隆,府庫充實,天下富庶,猶先俯屈以即成功,告之祖廟,書之史冊,為萬世美談,今其可不務乎?今大兵少息,若復南邊無事,則太平不遠矣。或謂專用威武可使宋人屈服,此殆虛言,不究實用。借令時獲小捷,亦不足多賀。彼見吾勢大,必堅守不出,我軍倉猝無得,須還以就糧,彼復乘而襲之,使我欲戰不得、欲退不能,則休兵之期殆未見也。況彼有江南蓄積之餘,我止河南一路征斂之弊,可為寒心。願陛下隱忍
 包容,速行此策,果通知,則大兵聞之,亦將斂跡,以吾無掣肘故也。河南既得息肩,然後經略朔方,則陛下享中興之福,天下賴涵養之慶矣。惟陛下略近功、慮後患,不勝幸甚。」上是其言,即命古草議和牒文。既成,以示宰臣,宰臣言其有哀祈之意,自示微弱,遂不用。



 監察御史粘割梭失劾榷貨司同提舉毛端卿貪污不法,古以詞理繁雜,輒為刪定,頗有脫漏,梭失以聞,削官一階,解職,特免殿年。三年正月,尚書省奏諫官闕員,因以古為請,上曰:「朕昨暮方思古,而卿等及之,正合朕意,其趨召之。」復拜左補闕。八月,削官四階,解職。初,朝廷遣近侍局直長溫
 敦百家奴暨刑部侍郎奧屯胡撒合徙吉州之民於丹以避兵鋒,州民重遷,遮道控訴,百家奴諭以天子恐傷百姓之意,且令召晉安兵將護老幼以行。眾意兵至則必見強也,迺噪入州署,索百家奴殺之。胡撒合畏禍,矯徇眾情,與之會飲歌樂盡日,眾肩舁導擁,讙呼拜謝而去。既還,詔古與監察御史紇石烈鐵論鞫之,諭旨曰:「百家奴之死,皆胡撒合所賣也,其閱實以聞。」奧屯胡撒合既下獄,上怒甚,亟欲得其情以正典刑,而古等頗寬縱之。胡撒合自縊死,有司以故出論罪,遂有是罰。



 哀宗初即位,召為補闕,俄遷左司諫,言事稍不及昔時。未幾,致
 仕,居伊陽,郡守為起伊川亭。古性嗜酒,老而未衰,每乘舟出村落間,留飲或十數日不歸,及溯流而上,老稚爭為挽舟,數十里不絕,其為時人愛慕如此。正大七年卒,年七十四。古平生好為詩及書,然不為士大夫所重,時論但稱其直云。



 天興間,有右司諫陳岢者,遇事輒言無少隱,上嘗面獎。及汴京被兵,屢上封事言得失,請戰一書尤為剴切,其略云:「今日之事,皆出陛下不斷,將相怯懦,若因循不決,一旦無如之何,恐君臣相對涕泣而已。」可謂切中時病,而時相赤盞合喜等沮之,策為不行,識者惜焉。岢字和之,滄州人,大安元年進士。



 贊曰:宣宗即位,孜孜焉以繼述世宗為志,而其所為一切反之。大定講和,南北稱治,貞祐用兵,生民塗炭。石琚為相,君臣之間務行寬厚。高琪秉政,惡儒喜吏,上下苛察。完顏素蘭首攻琪惡,謂琪必亂紀綱。陳規力言刀筆吏殘虐,恐壞風俗。許古請與宋和,辭極忠愛。三人所言皆切中時病,有古諍臣之風焉。宣宗知其為直,而不用其言,如是而欲比隆世宗,難矣。



\end{pinyinscope}