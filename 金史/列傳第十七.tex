\article{列傳第十七}

\begin{pinyinscope}

 ○酈瓊李成孔彥舟徐文施宜生張中孚張中彥宇文虛中王倫



 酈瓊,字國寶,相州臨漳人。補州學生。宋宣和間,盜賊起,瓊乃更學擊刺挽強,試弓馬,隸宗澤軍,駐于磁州。未幾告歸,括集義軍七百人,復從澤,澤署瓊為七百人長。澤死,調戍滑州。時宗望伐宋,將渡河。戍軍亂,殺其統制趙
 世彥,而推瓊為主。瓊因誘眾,號為勤王,行且收兵。比渡淮,有眾萬餘。康王以為楚州安撫使、淮南東路兵馬鈐轄,累遷武泰軍承宣使。未幾,率所領步騎十餘萬附於齊,授靜難軍節度使,知拱州。齊國廢,以為博州防禦使。用廉,遷驃騎上將軍。宗弼復河南,以瓊為山東路弩手千戶,知亳州事。丁母憂,去官。



 宗弼再伐江南,以瓊素知南方山川險易,召至軍與計事。從容語同列曰:「瓊嘗從大軍南伐,每見元帥國王親臨陣督戰,矢石交集,而王免胄,指麾三軍,意氣自若,用兵制勝,皆與孫、吳合,可謂命世雄材矣。至於親冒鋒鏑,進不避難,將士視之,孰敢
 愛死乎。宜其所向無前,日闢國千里也。江南諸帥,才能不及中人。每當出兵,必身居數百里外,謂之持重。或督召軍旅,易置將校,僅以一介之士持虛文諭之,謂之調發。制敵決勝委之偏裨,是以智者解體,愚者喪師。幸一小捷,則露布飛馳,增加俘級以為己功,斂怨將士。縱或親臨,亦必先遁。而又國政不綱,纔有微功,已加厚賞,或有大罪,乃置而不誅。不即覆亡,已為天幸,何能振起耶?」眾以為確論。元帥,謂宗弼也。



 及宗弼問瓊以江南成敗,誰敢相拒者。瓊曰:「江南軍勢怯弱,皆敗亡之餘,又無良帥,何以禦我。頗聞秦檜當國用事。檜,老儒,所謂亡國之
 大夫,兢兢自守,惟顛覆是懼。吾以大軍臨之,彼之君臣,方且心破膽裂,將哀鳴不暇,蓋傷弓之鳥可以虛弦下也。」既而江南果稱臣,宗弼喜瓊為知言。



 初,瓊去亳未幾,宋兵陷之而不守,復棄去,乃以州人宋超守之。及大軍至,超復以州事委其鈐轄衛經而遁去。帥府使人招經,經不下。及城潰,百姓惶懼待命,瓊請於元帥曰:「城所不下者,凶豎劫之也。民何罪,願慰安之。」元帥以瓊先嘗守亳,因止戮經而釋其州人,復命瓊守亳。凡六年,亳人德之。遷武寧軍節度使。八年,為泰寧軍節度使。九年,遷歸德尹。貞元元年,加金紫光祿大夫,卒于官,年五十。



 李成,字伯友,雄州歸信人。勇力絕倫,能挽弓三百斤。宋宣和初,試弓手,挽強異等。累官淮南招捉使。成乃聚眾為盜,鈔掠江南,宋遣兵破之,成遂歸齊,累除知開德府,從大軍伐宋。齊廢,再除安武軍節度使。



 成在降附諸將中最勇鷙,號令甚嚴,眾莫敢犯。臨陣身先諸將。士卒未食不先食,有病者親視之。不持雨具,雖沾濕自如也。有告成反者,宗弼察其誣,使成自治,成杖而釋之,其不校如此。以此士樂為用,所至克捷。



 宗弼再取河南,宋李興據河南府。成引軍入孟津。興率眾薄城,鼓噪請戰,成不應。日下昃,興士卒倦且飢,成開門急擊,大破之。興走漢
 南,成遂取洛陽、嵩、汝等。河南平,宗弼奏成為河南尹,都管押本路兵馬。嘗取官羨粟充公費,坐奪兩官,解職。正隆間,起為真定尹,封郡王,例封濟國公。卒,年六十九。



 孔彥舟,字巨濟,相州林慮人。亡賴,不事生產,避罪之汴,占籍軍中。坐事繫獄,說守者解其縛,乘夜踰城遁去。已而殺人,亡命為盜。宋靖康初,應募,累官京東西路兵馬鈐轄。聞大軍將至山東,遂率所部,劫殺居民,燒廬舍,掠財物,渡河南去。宋人復招之,以為沿江招捉使。彥舟暴橫,不奉約束,宋人將以兵執之,彥舟走之齊,從劉麟伐宋,為行軍都統,改行營左總管。



 齊國廢,累知淄州。從宗
 弼取河南,克鄭州,擒其守劉政,破孟邦傑於登封,授鄭州防禦使。討平太行車轅嶺賊。從征江南,渡淮,破孫暉兵萬餘人,下安豐、霍丘。及攻濠州,以彥舟為先鋒,順流薄城,擒其水軍統制邵青,遂克濠州。師還,累官工、兵部尚書,河南尹,封廣平郡王。正隆例降金紫光祿大夫,改西京留守。



 彥舟荒于色,有禽獸行。妾生女姿麗,彥舟苦虐其母,使自陳非己女,遂納為妾。其官屬負官錢,私其妻與折券。惟破濠州時,諸軍凡係獲皆殺之,彥舟號令毋輒殺,免者數千人,人頗以此稱之。然自幼至老常在行伍,習兵事,知利鈍。海陵欲以為征南將佐,正隆五年,
 除南京留守。



 彥舟有疾,朝臣有傳彥舟死者,而彥舟尚無恙,海陵盡杖妄傳彥舟死者,以激勵之。無何竟死於汴,年五十五。遺表言「伐宋當先取淮南」云。



 徐文,字彥武,萊州掖縣人,徙膠水。少時販鹽為業,往來瀕海數州,剛勇尚氣,儕輩皆憚之。宋季盜起,募戰士,為密州板橋左十將。勇力過人,揮巨刀重五十斤,所向無前,人呼為「徐大刀」。後隸王龍圖麾下,與夏人戰,生擒一將,補進武校尉。東還,破群賊楊進等,轉承信郎。



 宋康王渡江,召文為樞密院準備將,擒苗傅及韓世績,以功遷淮東、浙西、沿海水軍都統制。諸將忌其材勇。是時,李成、
 孔彥舟皆歸齊,宋人亦疑文有北歸志,大將閻皋與文有隙,因而譖之。宋使統制朱師敏來襲文,文乃率戰艦數十艘泛海歸于齊。齊以文為海、密二州滄海都招捉使兼水軍統制,遷海道副都統兼海道總管,賜金帶。文以策干劉豫,欲自海道襲臨安,豫不能用。齊國廢,元帥府承制以文為南京步軍都虞候,權馬步軍都指揮使。天眷元年,破太行賊梁小哥,以本職兼水軍統制。朝廷以河南與宋,除文山東路兵馬鈐轄。



 宗弼復取河南,文破宋將李寶於濮陽、孟邦傑於登封。宋蔣知軍據河陽,文遲明至其城下,使別將攻城東北,自將精銳潛師襲
 南門。城中悉眾救東北,文乃自南門斬關入城。宋軍潰去,追擊敗之。破郭清、郭遠於汝州。鄭州叛,復取之,擊走宋將戚方。河南既平,宗弼勞賞將士,賞文銀幣鞍馬。充行軍萬戶,從宗弼取廬、濠等州,超換武義將軍。知濟州,在職七年,移知泰安軍。海陵即位,錄舊功,累遷中都兵馬都指揮使,賜金帶,改浚州防禦使。未幾,海陵謀伐宋,改行都水監,監造戰船於通州。



 東海縣人徐元、張旺作亂,縣人房真等三人走海州,及走總管府,上變。州、府皆遣使效隨真等詣東海觀賊形勢,皆為賊所害。州、府合兵攻之,累月不下。海陵且欲伐宋,惡聞其事,詔文與步
 軍指揮使張弘信、同知大興尹李惟忠、宿直將軍蕭阿窊率舟師九百浮海討之,謂文等曰:「朕意不在一邑,將以試舟師耳。」文等至東海,與賊戰,敗之,斬首五千餘級,獲徐元、張旺,餘眾請降。是役也,張弘信行至萊州,稱疾留止,日與妓樂飲酒。海陵聞之,師還,杖弘信二百。文遷定海軍節度使。房真三人官賞有差。死賊者皆贈官三級,以銀百兩、絹百匹賜其家。



 大定二年,詣闕自陳年老目昏,懇求致仕。許之。以覃恩遷龍虎衛上將軍,卒于家。



 施宜生,字明望,邵武人也。博聞強記,未冠,由鄉貢入太學。宋政和四年,擢上舍第,試學官,授潁州教授。及王師
 入汴,宜生走江南。復以罪北走齊,上書陳取宋之策,齊以為大總管府議事官。失意於劉麟,左遷彰信軍節度判官。齊國廢,擢為太常博士,遷殿中侍御史,轉尚書吏部員外郎,為本部郎中。尋改禮部,出為隰州刺史。天德二年,用參知政事張浩薦宜生可備顧問,海陵召為翰林直學士,撰《太師梁王宗弼墓銘》,進官兩階。正隆元年,出知深州,召為尚書禮部侍郎,遷翰林侍講學士。



 四年冬,為宋國正旦使。宜生自以得罪北走,恥見宋人,力辭,不許。宋命張燾館之都亭,因間以首丘風之。宜生顧其介不在旁,為廋語曰:「今日北風甚勁。」又取幾間筆扣之
 曰:「筆來,筆來。」於是宋始警。其副使耶律闢離剌使還以聞,坐是烹死。



 初,宜生困於場屋,遇僧善風鑒,謂之曰:「子面有權骨,可公可卿。而視子身之毛,皆逆上,且覆腕,必有以合乎此而後可貴也。」宜生聞其言,大喜,竟從范汝為於建、劍。已而汝為敗,變服為傭泰之吳翁家三年。翁異之,一日屏人詰其姓名,宜生曰:「我服傭事惟謹,主人乃亦置疑邪?」翁固詰之,則請其故。翁曰:「日者燕客,執事咸餕,而汝獨孫諸儕,且撤器有歎聲,是以識汝非真傭也。」宜生遂告之故。翁贐之金,夜濟淮以歸。試《一日獲熊三十六賦》擢第一,其後竟如僧言。



 張中孚,字信甫,其先自安定徙居張義堡。父達,仕宋至太師,封慶國公。中孚以父任補承節郎。宗翰圍太原,其父戰歿,中孚泣涕請迹父屍,乃獨率部曲十餘人入大軍中,竟得其屍以還。累官知鎮戎軍兼安撫使,屢從吳玠、張浚以兵拒大軍。浚走巴蜀,中孚權帥事。天會八年,睿宗以左副元帥次涇州,中孚率其將吏來降,睿宗以為鎮洮軍節度使知渭州,兼涇原路經略安撫使。



 齊國建,以什一法括民田,籍丁壯為鄉軍。中孚以為涇原地瘠無良田,且保甲之法行之已習,今遽紛更,人必逃徙,只見其害,未見其利也。竟執不行。時齊政甚急,莫敢違,
 人為中孚懼,而中孚不之顧。未幾齊國廢,一路獨免掊克之患。



 天眷初,為陜西諸路節制使知京兆府,朝廷賜地江南,中孚遂入宋。宗弼再定河南、陜西,移文宋人,使歸中孚。至汴,就除行臺兵部尚書,遷除參知行臺尚書省事。明年,拜參知政事。貞元元年,遷尚書左丞,封南陽郡王。三年,以疾告老,乃為濟南尹,加開府儀同三司,封宿王。移南京留守,又進封崇王。卒,年五十九,加贈鄧王。



 中孚天性孝友剛毅,與弟中彥居,未嘗有間言。喜讀書,頗能書翰。其御士卒嚴而有恩,西人尤畏愛之。葬之日,老稚扶柩流涕蓋數萬人,至為罷市,其得西人之望如
 此。正隆例封崇進、原國公。



 張中彥,字才甫,中孚弟。少以父任仕宋,為涇原副將,知德順軍事。睿宗經略陜西,中彥降,除招撫使。從下熙、河、階、成州,授彰武軍承宣使,為本路兵馬鈐轄,遷都總管。



 宋將關師古圍鞏州,與秦鳳李彥琦會兵攻之。王師下饒風關,得金、洋諸州,以中彥領興元尹,撫輯新附。師還,代彥琦為秦鳳經略使。秦州當要衝而城不可守,中彥徙治北山,因險為壘,今秦州是也。築臘家諸城,以扼蜀道。帥秦凡十年,改涇原路經略使知平涼府。



 朝廷以河南、陜西賜宋,中孚以官守隨例當留關中。熙河經略使
 慕洧謀入夏,將窺關、陜,中彥與環慶趙彬會兩路兵討之,洧敗入于夏。中彥與兄中孚俱至臨安,被留,以為龍神衛四廂都指揮使,清遠軍承宣使,提舉佑神觀,靖海軍節度使。



 皇統初,恢復河南,詔徵中彥兄弟北歸,為靜難軍節度使,歷彰化軍、鳳翔尹,改尹慶陽,兼慶原路兵馬都總管、寧州刺史。宗室宗淵毆死僚佐梁郁。郁遠人,家貧無能赴告者。中彥力為正其罪,竟置于法。改彰德軍節度使,均賦調法,姦豪無所蔽匿,人服其明。



 正隆營汴京新宮,中彥採運關中材木。青峰山巨木最多,而高深阻絕,唐、宋以來不能致。中彥使構崖駕壑,起長橋十
 數里,以車運木,若行平地,開六盤山水洛之路,遂通汴梁。明年,作河上浮梁,復領其役。舟之始製,匠者未得其法,中彥手製小舟纔數寸許,不假膠漆而首尾自相鉤帶,謂之「鼓子卯」,諸匠無不駭服,其智巧如此。浮梁巨艦畢功,將發旁郡民曳之就水。中彥召役夫數十人,治地勢順下傾瀉于河,取新秫秸密布於地,復以大木限其旁,凌晨督眾乘霜滑曳之,殊不勞力而致諸水。



 俄遷平陽。海陵將伐宋,驛召赴闕,授西蜀道行營副都統制,賜細鎧,使先取散關俟後命。世宗即位,赦書至鳳翔,諸將惶惑不能決去就,中彥曉譬之,諸將感悟,受詔。上召中
 彥入朝,以軍付統軍合喜。及見,上賜以所御通犀帶,封宗國公。尋為吏部尚書。上疏曰:「古者關市譏而不征,今使掌關市者征而不譏。苛留行旅,至披剔囊笥甚於剽掠,有傷國體,乞禁止。」從之。



 踰年,除南京留守。時淮楚用兵,土民與戍兵雜居,訟牒紛紜,所司皆依違不決。中彥得戍兵為盜者,悉論如法,帥府怒其專決,劾奏之,朝廷置而不問。秩滿,轉真定尹兼河北西路兵馬都總管。未幾,致仕,西歸京兆。明年,起為臨洮尹兼熙秦路兵馬都總管。鞏州劉海構亂,既敗,籍民之從亂者數千人,中彥惟論為首者戮之。



 西羌吹折、密臧、隴逋、龐拜四族恃險
 不服,使侍御史沙醇之就中彥論方略,中彥曰:「此羌服叛不常,若非中彥自行,勢必不可。」即至積石達南寺,酋長四人來,與之約降,事遂定,賞而遣之。還奏,上大悅,遣張汝玉馳驛勞之,賜以球文金帶,用郊恩加儀同三司。以疾卒官,年七十五。百姓哀號輟市,立像祀之。



 贊曰:自古健將武夫,其不才者,遭世變遷,賣降恐後。此其常態,君子之所不責也,酈瓊、徐文是已。施宜生反覆壬人,李成盜賊之靡,孔彥舟漁色親出,自絕人類,又何責也。張中孚、中彥雖有小惠足稱,然以宋大臣之子,父戰沒於金,若金若齊,義皆不共戴天之仇。金以地與齊
 則甘心臣齊,以地歸宋則忍恥臣宋,金取其地則又比肩臣金,若趨市然,唯利所在,於斯時也,豈復知所謂綱常也哉。吁!



 宇文虛中,字叔通,蜀人。初仕宋,累官資政殿大學士。天會四年,宋少帝已結盟,宗望班師至孟陽,宋姚平仲乘夜來襲,明日復進兵圍汴。少帝使虛中詣宗望軍,告以襲兵皆將帥自為之,復請和議如初,且視康王安否。頃之,臺諫以和議歸罪虛中,罷為青州,復下遷祠職。建炎元年,貶韶州。二年,康王求可為奉使者,虛中自貶中應詔,復資政殿大學士,為祈請使。是時,興兵伐宋,已留王
 倫、朱弁不遣,虛中亦被留,實天會六年也。朝廷方議禮制度,頗愛虛中有才藝,加以官爵,虛中即受之,與韓昉輩俱掌詞命。明年,洪皓至上京,見虛中,甚鄙之。



 天會十三年,熙宗即位。宗翰為太保領三省事,封晉國王,乞致仕。批答不允,其詞虛中作也。天眷間,累官翰林學士知制誥兼太常卿,封河內郡開國公。書《太祖睿德神功碑》,進階金紫光祿大夫。皇統二年,宋人請和,其誓表曰:「自來流移在南之人,經官陳說,願自歸者,更不禁止。上國之於弊邑,亦乞並用此約。」於是,詔尚書省移文宋國,理索張中孚、張中彥、鄭億年、杜充、張孝純、宇文虛中、王進
 家屬,發遣李正民、畢良史還宋,惟孟庾去留聽其所欲。時虛中子師瑗仕宋,至轉運判官,攜家北來。四年,轉承旨,加特進。遷禮部尚書,承旨如故。



 虛中恃才輕肆,好譏訕,凡見女直人輒以礦鹵目之,貴人達官往往積不能平。虛中嘗撰宮殿榜署,本皆嘉美之名,惡虛中者擿其字以為謗訕朝廷,由是媒糵以成其罪矣。六年二月,唐括酬斡家奴杜天佛留告虛中謀反,詔有司鞫治無狀,乃羅織虛中家圖書為反具,虛中曰:「死自吾分。至於圖籍,南來士大夫家家有之,高士談圖書尤多於我家,豈亦反耶?」有司承順風旨并殺士談,至今冤之。



 士談字季
 默,高瓊之後。宣和末,為忻州戶曹參軍。入朝,官至翰林直學士。虛中、士談俱有文集行于世。



 王倫,字正道,故宋宰相王旦弟王勉玄孫。俠邪無賴,年四十餘尚與市井惡少群遊汴中。天會五年,宋人以倫為假刑部侍郎,與閣門舍人朱弁充通問使。是時,方議伐宋,凡宋使者如倫及宇文虛中、魏行可、顧縱、張邵等,皆留之不遣。居數年,倫久困,乃唱為和議求歸。元帥府使人謂之曰:「此非江南情實,特汝自為此言耳。」倫曰:「使事有指,不然何為來哉。惟元帥察之。」



 天會十年,劉豫連歲出師皆無功,撻懶為元帥左監軍經略南邊,密主和
 議,乃遣倫歸。先此,宋已遣使乞和,朝廷未之許也。倫見康王言和議事,康王大喜,遷倫官,并官其子弟。宋方與齊用兵,未可和。



 天會十五年,康王聞天水郡王已薨,以倫假直學士來請其喪,使倫請撻懶曰:「河南之地,上國既不自有,與其封劉豫,曷若歸之趙氏?」是歲,劉豫受封已八年,不能自立其國,尚勤屯戍,朝廷厭其無能為也,乃廢劉豫。撻懶以左副元帥守汴京,於是倫適至。撻懶,太祖從父兄弟,於熙宗為祖行。太宗長子宗磐以太師領三省事,位在宗乾上。宗翰薨已久,宗乾不能與宗磐獨抗。明年,天眷元年,撻懶與東京留守宗雋俱入朝,熙
 宗以宗雋為左丞相。宗雋,太祖子也。撻懶、宗磐、宗雋三人皆跋扈嗜利,陰有異圖,遂合議以齊地與宋,自宗乾以下爭之不能得。以侍郎張通古為詔諭江南使,遣倫先歸。



 明年,宋以倫為端明殿學士,簽書樞密院事,進金器千兩、銀器萬兩,復來請天水郡王喪柩,及請母韋氏兄弟宗族等。保信軍節度使藍公佐副之。是歲,宗磐、宗雋、撻懶皆以謀反屬吏,熙宗誅宗磐、宗雋,以撻懶屬尊,赦其死,以為行臺尚書省事左丞相,奪其兵權。右副元帥宗弼奏曰:「撻懶、宗磐陰與宋人交通,遂以河南、陜西地與宋人。」會撻懶復謀反,捕而殺之於祁州。倫至上京,
 有司詳讀康王表文,不書年,閱進奉狀,稱禮物不言職貢,上使宰相責問倫曰:「汝但知有元帥,豈知有上國耶。」遂留不遣,遣其副藍公佐歸。



 三年五月,宗弼復取河南、陜西地,遂伐江南,已渡淮。皇統元年,宋人請和。二年二月,宋端明殿學士何鑄、容州觀察使曹勛進誓表。三月,遣左副點檢賽里、山東西路都轉運使劉祹送天水郡王喪柩,及宋帝母韋氏還江南。五月,李正明、畢良史南歸。七月,朱弁、張邵、洪皓南歸。



 四年,以倫為平州路轉運使,倫已受命,復辭遜,上曰:「此反復之人也。」遂殺之於上京,年六十
 一。



 贊曰:孔子云:「行己有恥,使於四方不辱君命,可謂士矣」。宇文虛中朝至上京,夕受官爵。王倫紈褲之子,市井為徒。此豈「行己有恥」之士,可以專使者耶?二子之死雖冤,其自取亦多矣。



\end{pinyinscope}