\article{列傳第三十七}

\begin{pinyinscope}

 ○徒單鎰賈鉉孫鐸孫即康李革



 徒單鎰本名按出,上京路速速保子猛安人。父烏輦,北京副留守。鎰穎悟絕倫,甫七歲,習女直字。大定四年,詔以女直字譯書籍。五年,翰林侍講學士徒單子溫進所譯《貞觀政要》、《白氏策林》等書。六年,復進《史記》、《西漢書》,詔頒行之。選諸路學生三十餘人,令編修官溫迪罕締達
 教以古書,習作詩、策。鎰在選中,最精詣,遂通契丹大小字及漢字,該習經史。久之,樞密使完顏思敬請教女直人舉進士,下尚書省議。奏曰:「初立女直進士科,且免鄉、府兩試,其禮部試、廷試,止對策一道,限字五百以上成。在都設國子學,諸路設府學,並以新進士充教授,士民子弟願學者聽。歲久,學者當自眾,即同漢人進士三年一試。」從之。十三年八月,詔策女直進士,問以求賢為治之道。侍御史完顏蒲涅、太常博士李晏、應奉翰林文字阿不罕德甫、移刺傑、中都路都轉運副使奚釭考試鎰等二十七人及第。鎰授兩官,餘授一官,上三人為中都路
 教授,四名以下除各路教授。十五年,詔譯諸經,著作佐郎溫迪罕締達、編修官宗璧、尚書省譯史阿魯、吏部令史楊克忠譯解,翰林修撰移刺傑、應奉翰林文字移刺履講究其義。鎰自中都路教授選為國子助教。左丞相紇石烈良弼嘗到學中與鎰談論,深加禮敬。丁母憂,起復國史院編修官。



 世宗嘗問太尉完顏守道曰:「徒單鎰何如人也?」守道對曰:「有材力,可任政事。」上曰:「然,當以劇任處之。」又曰:「鎰容止溫雅,其心平易。」久之,兼修起居注,累遷翰林待制,兼右司員外郎。獻《漢光武中興賦》,世宗大悅曰:「不設此科,安得此人。」



 章宗即位,遷左諫議大夫,
 兼吏部侍郎。明昌元年,為御史中丞。無何,拜參知政事,兼修國史。鎰言:「人生有欲,不限以制,則侈心無極。今承平日久,當慎行此道,以為經久之治。」章宗銳意於治平,鎰上書,其略曰:「臣竊觀唐、虞之書,其臣之進言於君曰『戒哉』,『懋哉』,曰『吁』,曰『都』。既陳其戒,復導其美。君之為治也,必曰:『稽于眾,舍己從人』。既能聽之,又能行之,又從而興起之。君臣上下之間相與如此。陛下繼興隆之運,撫太平之基,誠宜稽古崇德,留意於此,無因物以好惡喜怒,無以好惡喜怒輕忽小善,不恤人言。夫上下之情有通塞,天地之運有否泰。唐陸贄嘗陳隔塞之九弊,上有其
 六,下有其三。陛下能慎其六,為臣子者敢不慎其三哉!上下之情既通,則大綱舉而群目張矣。」進尚書右丞,修史如故。



 三年,罷為橫海軍節度使,改定武軍節度使,知平陽府事。先是,鄭王永蹈判定武軍,鎬王永中判平陽府,相繼得罪,連引者眾,上疑其有黨,或命節度定武,繼又知平陽焉。改西京留守。承安三年,改上京留守。五年,上問宰臣:「徒單鎰與宗浩孰優?」平章政事張萬公對曰:「皆才能之士,鎰似優者。鎰有執守,宗浩多數耳。」上曰:「何謂多數?」萬公曰:「宗浩微似趨合。」上曰:「卿言是也。」頃之,鎰拜平章政事,封濟國公。



 淑妃李氏擅寵,兄弟恣橫,朝臣
 往往出入其門。是時烈風昏噎連日,詔問變異之由。鎰上疏略曰:「仁、義、禮、智、信謂之五常,父義、母慈、兄友、弟敬、子孝謂之五德。今五常不立,五德不興,縉紳學古之士棄禮義,忘廉恥,細民違道畔義,迷不知返,背毀天常,骨內相殘,動傷和氣,此非一朝一夕之故也。今宜正薄俗,順人心,父父子子夫夫婦婦,各得其道,然後和氣普洽,福祿薦臻矣。」因論:「為政之術,其急有二。一曰正臣下之心。竊見群下不明禮義,趨利者眾,何以責小民之從化哉。其用人也,德器為上,才美為下,兼之者待以不次,才下行美者次之,雖有才能,行義無取者,抑而下之,則臣
 下之趨向正矣。其二曰導學者之志。教化之行,興于學校。今學者失其本真,經史雅奧,委而不習,藻飾虛詞,釣取祿利,乞令取士兼問經史故實,使學者皆守經學,不惑於近習之靡,則善矣。」又曰:「凡天下之事,叢來者非一端,形似者非一體,法制不能盡,隱於近似,乃生異論。孔子曰:『義者天下之制也。』《記》曰:『義為斷之節。』伏望陛下臨制萬機,事有異議,少凝聖慮,尋繹其端,則裁斷有定,而疑可辨矣。」鎰言皆切時弊,上雖納其說,而不能行。上問漢高帝、光武優劣。平章政事張萬公對曰:「高祖優甚。」鎰曰:「光武再造漢業,在位三十年,無沈湎冒色之事。高祖
 惑戚姬,卒至於亂。由是言之,光武優。」上默然。鎰蓋以元妃李氏隆寵過盛,故微諫云。泰和四年,罷知咸平府。五年,改南京留守。六年,徙知河中府,兼陜西安撫使。



 僕散揆行省河南、陜西,元帥府雖受揆節制,實顓方面,上思用謀臣制之,由是升宣撫使一品,鎰改知京兆府事,充宣撫使,陜西元帥府並受節制。詔曰:「將帥雖武悍,久歷行陣,而宋人狡獪,亦資算勝。卿之智略,朕所深悉,且股肱舊臣,故有此寄。宜以長刺御敵,厲兵撫民,稱朕意焉。」鎰言:「初置急遞鋪,本為轉送文牒,今一切乘驛,非便。」上深然之。始置提控急遞鋪官。自中都至真定、平陽置者,
 達于京兆。京兆至鳳翔置者,達於臨洮。自真定至彰德置者,達于南京。自南京分至歸德置者,達于泗州、壽州,分至許州置者,達于鄧州。自中都至滄州置者,達於益都府。自此郵達無復滯焉。



 七年,吳曦死,宋安丙分兵出秦、隴間。十月,詔鎰出兵金、房以分掣宋人梁、益、漢、沔兵勢。鎰遣行軍都統斡勒葉祿瓦、副統把回海、完顏摑刺以步騎五千出商州。十一月,葉祿瓦拔鶻嶺關,摑刺別將攻破燕子關新道口,回海取小湖關敖倉,至營口鎮,破宋兵千餘人,追至上津縣,斬首八百餘級,遂取上津縣。葉祿瓦破宋兵二千于平溪,將趨金州。宋王柟以書
 乞和,詔鎰召葉祿瓦軍退守鶻嶺關。八年正月,宋安丙遣景統領由梅子溪、新道口、朱砂谷襲鶻嶺關,回海,摑刺擊走之,斬景統領於陣。是歲,罷兵。鎰遷特進,賜賚有差。改知真定府事。



 大安初,加儀同三司,封濮國公。改東京留守,過闕入見。衛紹王謂鎰曰:「卿兩朝舊德,欲用卿為相。太尉匡,卿之門人,朕不可屈卿下之。」遷開府儀同三司,佩金符,充遼東安撫副使。三年,改上京留守。平章政事獨吉思忠敗績於會河堡,中都戒嚴,鎰曰:「事急矣。」乃選兵二萬,遣同知烏古孫兀屯將之,入衛中都。朝廷嘉之,徵拜尚書右丞相,監修國史。



 鎰言:「自用兵以來,彼
 聚而行,我散而守,以聚攻散,其敗必然。不若入保大城,併力備禦。昌、桓、撫三州素號富實,人皆勇健,可以內徙,益我兵勢,人畜貨財,不至亡失。」平章政事移刺、參知政事梁絪曰:「如此是自蹙境土也。」衛紹王以責鎰。鎰復奏曰:「遼東國家根本,距中都數千里,萬一受兵,州府顧望,必須報可,誤事多矣。可遣大臣行省以鎮之。」衛紹王不悅曰:「無故置行省,徒搖人心耳。」其後失昌、桓、撫三州,衛紹王乃大悔曰:「從丞相之言,當不至此!」頃之,東京不守,衛紹王自訟曰:「我見丞相恥哉!」



 術虎高琪駐兵縉山,甚得人心,士樂為用。至寧元年,尚書左丞完顏綱將行省
 於縉山,鎰謂綱曰:「行省不必自往,不若益兵為便。」綱不聽,且行,鎰遣人止之曰:「高琪之功,即行省之功也。」亦不聽。綱至縉山,遂敗績焉。



 頃之,鎰墜馬傷足在告,聞胡沙虎難作,命駕將入省。或告之曰:「省府相幕皆以軍士守之,不可入矣。」少頃,兵士索人于閭巷,鎰乃還第。胡沙虎意不可測,方猶豫,不能自定,乃詣鎰問疾,從人望也。鎰從容謂之曰:「翼王,章宗之兄,顯宗長子,眾望所屬,元帥決策立之,萬世之功也。」胡沙虎默然而去,乃迎宣宗于彰德。胡沙虎既殺徒單南平,欲執其弟知真定府事銘,鎰說之曰:「車駕道出真定,鎬王家在威州,河北人心易
 搖,徒單銘有變,朝廷危矣。不如與之金牌,奉迎車駕,銘必感元帥之恩。」胡沙虎從之。至寧、貞祐之際,轉敗為功,惟鎰是賴焉。



 宣宗即位,進拜左丞相,封廣平郡王,授中都路迭魯都世襲猛安蒲魯吉必剌謀克。鎰尚有足疾,詔侍朝無拜。明年,鎰建議和親。言事者請罷按察司。鎰曰:「今郡縣多殘毀,正須按察司撫集,不可罷。」遂止。宣宗將幸南京,鎰曰:「鑾輅一動,北路皆不守矣。今已講和,聚兵積粟,固守京師,策之上也。南京四面受兵。遼東根本之地,依山負海,其險足恃,備禦一面,以為後圖,策之次也。」不從。是歲,薨。詔賻贈從優厚。



 鎰明敏方正,學問該貫,
 一時名士,皆出其門,多至卿相。嘗歎文士委頓,雖巧拙不同,要以仁義道德為本,乃著《學之急》、《道之要》二篇。太學諸生刻之于石。有《弘道集》六卷。



 賈鉉,字鼎臣,博州博平人。性純厚,好學問。中大定十三年進士,調滕州軍事判官、單州司候,補尚書省令史。章宗為右丞相,深器重之,除陜西東路轉運副使。入為刑部主事,遷監察御史。遷侍御史,改右司諫。上疏論邊戍利害,上嘉納之,遷左諫議大夫兼工部侍郎,與黨懷英同刊修《遼史》。



 鉉上書曰:「親民之官,任情立威,所用決杖,分徑長短不如法式,甚者以鐵刃置於杖端,因而致死。
 間者陰陽愆戾,和氣不通,未必不由此也。願下州郡申明舊章,檢量封記,按察官其檢察不如法者,具以名聞。內庭敕斷,亦依已定程式。」制可。復上書論山東採茶事,其大概以為「茶樹隨山皆有,一切護邏,已奪民利,因而以揀茶樹執誣小民,嚇取貨賂,宜嚴禁止。仍令按察司約束。」上從之。承安四年,遷禮部尚書,諫議如故。是時有詔,凡奉敕商量照勘公事皆期日聞奏。鉉言:「若如此,恐官吏迫於限期,姑務茍簡,反害事體。況簿書自有常程,御史臺治其稽緩,如事有應密,三月未絕者,令具次第以聞。下尚書省議。如省部可即定奪者,須三月擬奏,如
 取會案牘卒難補勘者,先具次第奏知,更限一月結絕,違者準稽緩制書罪之。」



 上議置相,欲用鉉,宰臣薦孫即康。張萬公曰:「即康及第在鉉前。」上曰:「用相安問榜次?朕意以為賈鉉才可用也。」然竟用即康焉。



 泰和二年,興陵崇妃薨,上欲成服苑中,行登門送喪之禮,以問鉉,鉉對曰:「故宋嘗行此禮,古無是也。」遂已。改刑部尚書。泰和三年,拜參知政事。亳州醫者孫士明輒用黃紙大書「敕賜神針先生」等十二字,及於紙尾年月間摹作寶樣朱篆青龍二字,以誑惑市人。有司捕治款伏。值赦,大理寺議宜準偽造御寶,雖遇赦不應原。已奏可矣。鉉奏:「天子有
 八寶,其文各異,若偽造,不限用泥及黃蠟。今用筆描成青龍二字,既非八寶文,論以偽造御寶,非本法意。」上悟,遂以赦原。明日,上謂大臣曰:「已行之事,賈鉉猶執奏,甚可嘉也,群臣亦當如此矣。」



 泰和六年,御試,鉉為監試官。上曰:「丞相宗浩嘗言試題頗易,由是進士例不讀書。朕今以《日合天統》為賦題。」鉉曰:「題則佳矣,恐非所以牢籠天下士也。」上曰:「帝王以難題窘舉人,固不可,欲使自今積致學業而已。」遂用之。久之,鉉與審官院掌書大中漏言除授事。上謂鉉曰:「卿罪自知之矣。然卿久參機務,補益弘多,不深罪也。」乃出為安武軍節度使,改知濟南府。
 致仕。貞祐元年薨。



 孫鐸,字振之,其先滕州人,徙恩州歷亭縣。鐸性敏好學,遼陽王遵古一見器之,期以公輔。登大定十三年進士第,調海州軍事判官、衛縣丞,補尚書省令史。章宗為右丞相,語人曰:「治官事如孫鐸,必無錯失。」初即位,問鐸安在,有司奏為右都管,使宋。及還,除同知登聞檢院事。鐸言:「凡上訴者皆因尚書省斷不得直,若上訴者復送省,則必不行矣,乞自宸衷斷之。」上以為然。詔登聞檢院,凡上訴者,每朝日奏十事。詔刊定舊律,鐸先奏《名例》一篇。



 承安元年,遷左諫議大夫,改河東南路轉運使,召為中
 都路都轉運使。初置講議錢穀官十人,鐸為選首。承安四年,遷戶部尚書。鐸因轉對奏曰:「比年號令,或已行而中輟,或既改而復行,更張太煩,百姓不信。乞自今凡將下令,再三講究,如有益于治則必行,無恤小民之言。」國子司業紇石烈善才亦言:「頒行法令,絲綸既出,尤當固守。」上然之。泰和二年閏十二月,上召鐸、戶部侍郎張復亨議交鈔。復亨曰:「三合同鈔可行。」鐸請廢不用,詰難久之,復亨議詘。上顧謂侍臣曰:「孫鐸剛正人也,雖古魏徵何加焉!」



 三年,御史中丞孫即康、刑部尚書賈鉉皆除參知政事,鐸再任戶部尚書。鐸心少之,對賀客誦古人詩曰:「
 唯有庭前老柏樹,春風來似不曾來。」御史大夫卞劾鐸怨望,降同知河南府事。改彰化軍節度使,復為中都轉運使。泰和七年,拜參知政事。



 蒲陰縣令大中與左司郎中劉昂、通州刺史史肅、前臨察御史王宇、吏部主事曹元、戶部員外郎李著、監察御史劉國樞、尚書省都事曹溫、雄州都軍馬師周,吏部員外郎徒單永康、太倉使馬良顯、順州刺史唐括直思白坐私議朝政,下獄,尚書省奏其罪。鐸進曰:「昂等非敢議朝政,但如鄭人游鄉校耳。」上悟,乃薄其罪。鐸上言:「民間鈔多,宜收斂。院務課程及諸窠名錢須要全收交鈔。秋夏稅本色外,盡令折鈔,不
 拘貫例,農民知之,迤漸重鈔。比來州縣抑配行市買鈔,無益,徒擾之耳。乞罷諸處鈔局,惟省庫仍舊,小鈔無限路分,可令通行。」上覽奏,即詔有司曰:「可速行之。」大安初,議誅黃門李新喜。鐸曰:「此先朝用之太過耳。」衛紹王不察,即曰:「卿今日始言之何耶?」既而復曰:「後當盡言,勿以此介意。」頃之,遷尚書左丞,兼修國史。議鈔法忤旨,猶以論李新喜降濬州防禦使。改安國軍節度使,徙絳陽軍。



 宣宗即位,召赴闕,以兵道阻。宣宗遷汴,鐸上謁於宜村,除太子太師。在疾,累遣使候問。貞祐三年,致仕。是歲薨。



 孫即康,字安伯,其先滄州人。石晉之末,遼徙河北實燕、
 薊,八代祖延應在徙中,占籍析津,實大興,仕至涿州刺史。延應玄孫克構,遼檢校太傅、啟聖軍節度使。即康,克構曾孫,中大定十年進士第。章宗為右丞相,是時,即康為尚書省令史,由是識其人。章宗即位,累遷戶部員外郎,講究鹽法利害,語在《食貨志》。除耀州刺史,入為吏部左司郎中。上謂宰臣曰:「孫即康向為省掾,言語拙訥,今才力大進,非向時比也。」宰臣因曰:「即康年已高,幸及早用之。」上問:「年幾何矣?」對曰:「五十六歲。」上復問:「其才何如張萬公?」平章政事守貞對曰:「即康才過之。」上曰:「視萬公為通耳。」由是遷御史中丞。



 初,張汝弼妻高陀斡不道,伏
 誅。汝弼,鎬王永中舅也,上由是頗疑永中。永中府傅尉奏永中第四子阿離合懣語涉不軌,詔同簽大睦親府事褲與即康鞫之。第二子神土門嘗撰詞曲,頗輕肆,遂以語涉不遜就逮。家奴德哥首永中嘗與侍妾瑞雲言:「我得天下,以爾為妃,子為大王。」褲、即康還奏,詔禮部尚書張暐復訊。永中父子皆死,時論冤之。頃之,遷泰寧軍節度使,改知延安府事。



 承安五年,上問宰相:「今漢官誰可用者?」司空襄舉即康。上曰:「不輕薄否?」襄曰:「可再用為中丞觀之。」上乃復召即康為御史中丞。泰和三年,除參知政事。明年,進尚書右丞。六年,宋渝盟有端,大臣猶以
 為小盜竊發不足恤。即康與左丞僕散端、參政獨吉思忠以為必當用兵,上以為然。



 上問即康、參知政事賈鉉曰:「太宗廟諱同音字,有讀作『成』字者,既非同音,便不當缺點畫。睿宗廟諱改作『崇』字,其下卻有本字全體,不若將『示』字依《蘭亭貼》寫作『未』字。顯宗廟諱『允』,『充』字合缺點畫,如『統』傍之『充』,似不合缺。」即康奏曰:「唐太宗諱世民,偏傍犯如『葉』字作『筼』字,『泯』字作『泜』字。」乃擬「熙宗廟諱從『面』從『且』。睿宗廟諱上字從『未』,下字從『簹』。世宗廟諱從『系』。顯宗廟諱如正犯字形,止書斜畫,『沇』字『鈗』字各從『口』,『兌』『悅』之類各從本傳。」從之,自此不勝曲避矣。進左丞。宋人請和,進官一階。



 舊
 制,尚書省令史考滿優調,次任回降。崔建昌已優調興平軍節度副使,未回降即除大理司直。詔知除郭邦傑、李蹊杖七十勒停,左司員外郎高庭玉決四十解職,即康待罪,有詔勿問。章宗崩,衛紹王即位,即康進拜平章政事,封崇國公。大安三年,致仕。是歲,薨。遣使致祭。



 李革,字君美,河津人。父餘慶,三至廷試,不遂,因棄去。革穎悟,讀書一再誦,輒記不忘。大定二十五年進士。調真定主簿。察廉,遷韓城令。同知州事納富商賂,以歲課軍須配屬縣,革獨不聽,提刑司以為能。遷河北東路轉運都勾判官、太原推官。丁母憂,起復,遷大興縣令、中都左
 警巡使、南京提刑判官、監察御史、同知昭義軍節度事。丁父憂,起復,簽南京按察事。



 泰和六年,伐宋,尚書省奏:「軍興,隨路官,差占者別注,闕者選補,老不任職者替罷,及司、縣各存留強幹正官一員。」革與簽陜西高霖、簽山東孟子元俱被詔,體訪三路官員能否,籍存留正官,行省、行部、元帥府差占員數及事故闕員,老不任職,赴闕奏事。改刑部員外郎,調觀州刺史兼提舉漕運,陜西西路按察副使,大興府治中。知府徒單南平貴倖用事,勢傾中外,遣所親以進取誘革,革拒之。貞祐二年,遷戶部侍郎。宣宗遷汴,行河北西路六部事,遷知開封府事,河
 南勸農使,戶部、吏部尚書,陜西行省參議官。



 四年,拜參知政事。革奏:「有司各以情見引用斷例,牽合附會,實啟倖門。乞凡斷例敕條特旨奏斷不為永格者,不許引用,皆以律為正。」詔從之。是歲,大元兵破潼關,革自以執政失備禦之策,上表請罪。不許,罷為絳陽軍節度使。興定元年,胥鼎自平陽移鎮陜西,革以知平陽府事,權參知政事,代鼎為河東行省。是時興兵伐宋,革上書曰:「今之計當休兵息民,養銳待敵。宋雖造釁,止可自備。若不忍小忿以勤遠略,恐或乘之,不能支也。」不納。太原兵後闕食,革移粟七萬石以濟之。二年,宣差粘割梭失至河東,
 於是晚禾未熟,牒行省耕毀清野。革奏:「今歲雨澤及時,秋成可待。如令耕毀,民將不堪。」詔從革奏。十月,平陽被圍,城中兵不滿六千,屢出戰,旬日間傷者過半。徵兵吉、隰、霍三州,不時至。裨將李懷德縋城出降,兵自城東南入。左右請革上馬突圍出,革歎曰:「吾不能保此城,何面目見天子!汝輩可去矣。」乃自殺。贈尚書右丞。



 贊曰:《傳》曰:「君子之言,其利博哉!」徒單鎰拱挹一語而宣宗立,厥功懋矣。賈鉉、孫鐸皆舊臣,鉉久致仕,鐸忤旨衛王,皆不復見用。徒單鎰亦外官,惟孫即康詭隨,乃驟至宰相。古所謂斗筲之人,即康之謂矣。鐸論李新喜,其言
 似漢耿育,有旨哉。貞祐執政李革,可謂君子,其進退之際,有古人為相之風焉。



\end{pinyinscope}