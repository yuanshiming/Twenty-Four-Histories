\article{列傳第十一}

\begin{pinyinscope}

 ○阿離合懣晏本名斡論宗尹本名阿里罕宗寧本名阿土古宗道本名八十宗雄本名謀良虎阿鄰按荅海希尹本名穀神守貞本名左靨守能本名胡刺



 阿離合懣,景祖第八子也。健捷善戰。年十八,臘醅、麻產起兵據暮棱水,烏春、窩謀罕以姑里甸兵助之。世祖擒臘醅,暮棱水人尚反側,不自安,使阿離合懣往撫差之,
 與斜缽合兵攻窩謀罕。烏春已死,窩謀罕棄城遁去。後從撒改討平留可,阿離合懣功居多。



 太祖擒蕭海里,使阿離合懣獻馘於遼。太祖謀伐遼,阿離合懣實贊成之。及舉兵,阿離合懣在行間屢戰有功。及太宗等勸進,太祖未之許也。阿離合懣、昱、宗翰等曰:「今大功已集,若不以時建號,無以繫天下心。〕太祖曰:「吾將思之。」收國元年,太祖即位。阿離合懣與宗翰以耕具九為獻,祝曰:「使陛下毋忘稼穡之艱難。」太祖敬而受之。頃之,為國論乙室勃極烈。



 為人聰敏辨給,凡一聞見,終身不忘。始未有文字,祖宗族屬時事並能默記,與斜葛同脩本朝譜牒。見
 人舊未嘗識,聞其父祖名,即能道其部族世次所出。或積年舊事,偶因他及之,人或遺忘,輒一一辨析言之,有質疑者皆釋其意義。世祖嘗稱其強記,人不可及也。



 天輔三年,寢疾,宗翰日往問之,盡得祖宗舊俗法度。疾病,上幸其家問疾,問以國家事,對曰:「馬者甲兵之用,今四方未平,而國俗多以良馬殉葬,可禁止之。」乃獻平生所乘戰馬。及以馬獻太宗,使其子蒲里迭代為奏,奏有誤語,即哂之,宗翰從傍為改定。進奏訖,薨,年四十九。



 上聞阿離合懣臨薨有奏事,曰:「臨終不亂,念及國家事,真賢臣也。」哭之慟。及葬,上親臨。熙宗時,追封隋國王。天德中,
 改贈開府儀同三司、隋國公。大定間,配饗太祖廟廷,謚曰剛憲。子賽也、斡論。賽也子宗尹。



 晏本名斡論,景祖之孫,阿離合懣次子也。明敏多謀略,通契丹字。天會初,烏底改叛。太宗幸北京,以晏有籌策,召問,稱旨,乃命督扈從諸軍往討之。至混同江,諭將士曰:「今叛眾依山谷,地勢險阻,林木深密,吾騎卒不得成列,未可以歲月破也。」乃具舟楫艤江。,令諸軍據高山,連木為柵,多張旗幟,示以持久計,聲言俟大軍畢集而發。乃潛以舟師浮江而下,直搗其營,遂大破之,據險之眾不戰而潰。月餘,一境皆定。師還,授左監門衛上將軍,為
 廣寧尹,入為吏、禮兩部尚書。



 皇統元年,為北京留守,改咸平尹,徙東京。天德初,封葛王,入拜同判大宗正事,進封宋王,授世襲猛安。海陵遷都,晏留守上京,授金牌一、銀牌二,累封豫王、許王,又改越王。貞元初,進封齊。時近郊禁圍獵,特畀晏三百人從獵。在上京凡五年。正隆二年,例削王爵,改西京留守。未幾,為臨潢尹,遂致仕,還居會寧。



 海陵南伐,世宗為東京留守,將士皆自淮南來歸,晏之子恧里乃亦自軍前率眾來歸世宗。白彥敬等在北京聞恧里乃等逃還,使會寧同知高國勝拘晏家族。上既即位,遣使召晏,既又遣晏兄子鶻魯補馳驛促之。
 晏遂率宗室數人入見,即拜左丞豐,封廣平郡王,宴勞彌日。未幾,兼都元帥。



 大定二年正月,上如山陵。禮畢,上將獵,有司已夙備。晏諫曰:「邊事未寧,畋游非所宜也。」上嘉納之。因謂晏等曰:「古者帝王虛心受諫,朕常慕之。卿等盡言毋隱。」進拜太尉。復致仕,還鄉里。是歲,薨。詔有司致祭,賻贈銀幣甚厚。



 宗尹,本名阿里罕。以宗室子充護衛,改牌印祗候,授世襲謀克,為右衛將軍。歷順天、歸德、彰化、唐古部族、橫海軍節度使。正隆南伐,領神略軍都總管,先鋒渡淮,取揚州及瓜洲渡。大定二年,改河南路副都統,駐軍許州之
 境。



 是時,宋陷汝州,殺刺史烏古孫麻潑及漢軍二千人。宗尹遣萬戶孛術魯定方、完顏阿喝懶、夾谷清臣、烏古論三合、渠雛訛只將騎四千往攻之,遂復取汝州。除大名尹,副統如故。頃之,為河南路統軍使,遷元帥左都監,除南京留守。上曰:「卿年少壯,而心力多滯。前任點檢京尹,勤力不怠,而處事迷錯。勉脩職業,以副朕意。」賜通犀帶、廄馬。八年,置山東路統軍司,宗尹為使。遷樞密副使。錄其父功,授世襲蒲與路屯河猛安,並親管謀克。隊太子太保,樞密副使如故。



 上問宰臣曰:「宗尹雖才無大過人者,而性行淳厚,且國之舊臣,昔為達官,卿等尚未仕
 也。朕欲以為平章政事何如?」宰執皆曰:「宗尹為相,甚協眾望。」即日拜平章政事,封代國公,兼太子太傅。



 是時民間苦錢幣不通,上問宗尹,對曰:「錢者有限之物,積於上者滯於下,所以不通。海陵軍興,為一切之賦,有菜園、房稅、養馬錢。大定初,軍事未息,調度不繼,故因仍不改。今天下無事,府庫充積,悉宜罷去。」上曰:「卿留意百姓,朕復何慮。太尉守道老矣,捨卿而誰。」於是,養馬等錢始罷。



 他日,上謂宰臣曰:「宗尹治家嚴密,他人不及也。」顧謂宗尹曰:「政事亦當如此矣。」有頃,北方歲饑,軍食不足,廷議輸粟賑濟。或謂比雖不登,而舊積有餘,秋成在近,不必更
 勞輸挽,宗尹曰:「國家平時積粟,本以備兇歲也,必待秋成,則憊者眾矣。人有捐瘠,其如防戍何。」上從之。



 宗尹乞令子銀術可襲其猛安,會太尉守道亦乞令其子神果奴襲其謀克。凡承襲人不識女直字者,勒令習學。世宗曰:「此二子,吾識其一習漢字,未習女直字。自今女直、契丹、漢字曾學其一者,即許承襲。」遂著於令。



 宗尹有疾,不能赴朝。上問宰臣曰:「宗尹何為不入朝?」太尉守道以疾對。曰:「丞相志寧嘗言,『若詔遣征伐,所不敢辭。宰相之職,實不敢當』。宗尹亦豈此意耶。」



 二十四年,世宗將幸上京。上曰:「臨潢、烏古里石壘歲皆不登,朕欲自南道往,三
 月過東京,謁太后陵寢,五月可達上京。春月烏獸孳孕,東作方興,不必搜田講事,卿等以為何如?」宗尹曰:「南道歲熟,芻粟賤,宜如聖旨。」遂由南道往焉。世宗至上京,聞同簽大宗正事宗寧不能撫治上京宗室,宗室子往往不事生業。上謂宗尹曰:「汝察其事,宜懲戒之。」宗尹奏曰:「隨仕之子,父沒不還本土,以此多好游蕩。」上命召還。宴宗室於皇武殿,擊球為樂。上曰:「賞賜宗室,亦是小惠,又不可一概遷官,欲令諸局分收補,其間人材孰可者?」宗尹對曰:「奉國斡準之子按出虎、豫國公昱之曾孫阿魯可任使。」上曰:「度可任何職,更訪其餘以聞。」詔以按出虎、
 阿魯為奉御。



 二十七年,乞致仕。」世宗曰:「此老不事事,從其請可也。」宰臣奏曰:「舊臣宜在左右。」上曰:「宰相總天下事,非養老之地。若不堪其職,朕亦有愧焉。如賢者在朝,利及百姓,四方瞻仰,朕亦與其光美。」宰臣無以對。宗尹入謝。上曰:「卿久任外官,不聞有過失,但恨用卿稍晚,今精力似衰矣。省事至煩,若勉留卿,則四方以朕為私,卿亦不自安也。」頃之,上問宗尹子:「汝父致仕,將居何所?」其子曰:「聚屬既多,不能復在京師。」上遣使問宗尹曰:「朕欲留卿,時相從游,卿子之言如此,今定何如?」宗尹曰:「臣豈不欲在此,但餘閑之年,猶在輦下,恐聖主心困耳。既哀
 老臣不忍擯棄,時時得瞻望天顏,臣豈敢他往。鄉里故老無存者,雖到彼,尚將與誰游乎。」於是賜甲第一區,凡宴集畋獵皆從焉。二十八年,薨。



 宗寧本名阿土古,系出景祖,太尉阿離合懣之孫。性勤厚,有大志。起家為海陵征南都統,戰瓜洲渡,功最。歷祁州刺史。



 大定二年,為會寧府路押軍萬戶,擢歸德軍節度使。時方旱蝗,守寧督民捕之,得死蝗一斗,給粟一斗,數日捕絕。移鎮寧昌軍,改知臨潢府事,移天德軍。世宗嘗謂宰臣曰:「宗寧智慮雖淺,然所至人皆愛之。」即命為行軍右翼都統,為賀宋正旦使。累遷兵部尚書,授隆州
 路和團猛安烈里沒世襲謀克。出知大名府事,徙鎮利涉軍,俄同簽大睦親府事。



 宗寧多病,世宗欲以涼地處之,俾知咸平,詔以其子符寶郎畝為韓州刺史,以便養。無幾,入授同判大睦親府事,拜平章政事。明昌二年,薨。宗寧居家約儉如寒素,臨事明敏。其鎮臨潢,鄰國有警,宗寧聞知乏糧,即出倉粟,令以牛易之,敵知得粟,即遁去。邊人以窩斡亂後,苦無牛,宗寧復令民入粟易牛,既而民得牛而倉粟倍於舊,其經畫如此。



 宗道本名八十,上京司屬司人,系出景祖,太尉訛論之少子也。通《周易》、《孟子》,善騎射,大定五年,充閤門祗候,累
 除近侍局使。



 右丞相烏古論元忠、左衛將軍僕散揆等嘗燕集,有所竊議,宗道即密以聞。世宗嘉之,授右衛將軍,出為西南路副招討。章宗即位,改同知平陽府事。陜西路副統軍、左宣徽使移刺仲方舉以自代,除西北路招討使。故事,諸部賀馬八百餘疋,宗道辭不受,諸部悅服,邊鄙順治。提刑司察廉,召為殿前右副都點檢。尋除陜西路統軍使,以鎮靜得軍民心,特遷三階,兼知京兆府事。時夏早,俾長安令取太白湫水,步迎於遠郊,及城而雨。是歲大稔,人以為精意所感,刊石紀之。



 承安二年,為賀宋正旦使,尋授河南路統軍使。泗州民張偉獲宋人
 王萬,言彼界事情,宗道疑其冤,乃廉問得實。萬,楚州賈人,偉負萬貨五千餘貫,三年不償,萬理索,為偉所誣。乃坐偉而歸萬,時人服其明。後乞至仕,朝廷知非本心,改知河中府,有惠政,民立像於層觀,以時祭之。移知臨洮,以病解。泰和四年,卒。贈龍虎衛上將軍。



 宗雄本名謀良虎,康宗長子。其始生也,世祖見而異之,曰:「此兒風骨非常,他日必為國器。」因解佩刀,使常置其側,曰:「俟其成人則使佩之。」九歲能射逸兔。年十一,射中奔鹿。世祖坐之膝上曰:「兒幼已然,異已出倫輩矣。」以銀酒器賜之。既長,風表奇偉,善談辯,多智略,孝敬謙謹,人
 愛敬之。康宗沒,遼使阿息保來,乘馬至靈帷階下,擇取賵贈之馬。太祖怒,欲殺阿息保,宗雄諫,太祖乃止。



 太祖將舉兵,宗雄曰:「遼主驕侈,人不知兵,可取也。不能擒一蕭海里,而我兵擒之。」太祖善其言。攻寧江州,渤海兵銳甚。宗雄以所部敗渤海兵,以功授世襲千戶謀克。太祖敗遼兵于出河店,宗雄推鋒力戰,功多。達魯古城之役,宗雄將右軍,身先士卒戰,遼兵當右軍者已卻,上命宗雄助左軍擊遼兵。宗雄繞遼兵後擊之,遼兵遂大潰,乘勝逐北。日已暮,圍之。黎明,遼兵突圍出,追殺至乙呂白石而還。上撫其背曰:「朕有此子,何事不濟。」以御服賜之。



 及遼帝以七十萬眾至馳門,諸將皆曰:「遼軍勢甚盛,不宜速戰。」宗雄曰:「不然。遼兵雖眾,而皆庸將,士卒惴惴,不足畏也。戰則破之掌握間耳。」上曰:「善。」追及遼帝于護步荅岡。宗雄率眾直前,短兵接。宗雄令前行持挺擊遼兵馬首,後行者射之,大敗遼兵。上嘉宗雄功,執其手勞之,以御介胄及御戰馬、寶貨、奴婢賜之。



 斜也攻春州,宗雄與宗乾、婁室取金山縣。行近白鷹林,獲候者七人,縱其一人使歸。縣人聞大軍至,迺潰,遂下金山縣。與斜也俱取泰州。



 太祖自將取臨潢府,遣宗雄先啟行,遇遼兵五千,宗雄與戰,大軍亦至,大破之。及留守撻不野降,上以
 其女與宗雄,賞其啟行破遼援兵之功也。既而與蒲家奴按視泰州地土,宗雄包其土來奏曰:「其土如此,可種植也。」上從之。由是徙萬餘家屯田泰州,以宗雄等言其地可種藝也。



 西京既降復叛,時糧餉垂盡,議欲罷攻。宗雄曰:「西京,都會也,若委而去之,則降者離心,遼之餘黨與夏人得以窺伺矣。」乃立重賞以激士心。既而,夜中有火,大如斗,墜于城中。宗雄曰:「此城破之象也。」及克西京,賜宗雄黃金百兩,衣十襲及奴婢等。



 與宗翰等擊耿守忠兵七千于西京之東四十里,大破之。迎謁太祖于鴛鴦濼,從至歸化州。疾篤,宗幹問所欲言。宗雄曰:「國家大
 業既成,主上壽考萬年,肅清四方,死且無恨。」天輔六年,薨,年四十。太祖來問疾,不及見,哭之慟。謂群臣曰:「此子謀略過人,臨陣勇決,少見其比。賻贈加等。詔合扎千戶駙馬石家奴護喪歸,葬於歸化州,仍於死所建佛寺。



 宗雄好學嗜書,嘗從上獵,誤中流矢,而神色不變,恐上知之而罪及射者。既拔去其矢,託疾歸家,臥兩月,因學契丹大小字,盡通之。凡金國初建,立法定制,皆與宗乾建白行焉。及與遼議和,書詔契丹、漢字,宗雄與宗翰、希尹主其事。而材武蹻捷,挽強射遠,幾三百步。嘗走馬射三麞,已中其二,復彎弓,馬蹶,躍而下,控弦如故,遂彀滿步
 射獲之。宗雄方逐兔,撻懶亦從後射之,已發矢,撻懶大呼曰:「矢及矣。」宗雄反顧,以手接其矢,就射兔,中之,其輕健如此。



 天眷中,追封太師、齊國王。天德二年,加秦漢國王。正隆二年,改太傅、金源郡王。大定二年,追封楚王,謚威敏,配享太祖廟廷。十五年,詔圖像于衍慶宮。子蒲魯虎、按荅海、阿鄰。孫常春、胡里刺、胡刺、鶻魯、茶扎、怕八、訛出。



 初,宗乾納宗雄妻,海陵銜之。及篡位,使宿直將軍晁霞、牌印閭山往河間,囚宗雄妻於府署,明日,與其子婦及常春兄弟、茶扎之子七人皆殺而焚之,棄其骨於濠水。大定十七年,詔有司收葬。



 初,蒲魯虎襲猛安。蒲魯虎
 卒,贈金紫光祿大夫,子桓端襲之,官至金吾衛上將軍。桓端卒,子裊頻未襲而死。章宗命宗雄孫蒲帶襲之。



 蒲帶,大定末,累官同簽大睦親府事。章宗即位,初置九路提刑司,蒲帶為北京臨潢提刑使。詔曰:「朕初即位,憂勞萬民,每念刑獄未平,農桑未勉,吏或不循法度,以隳吾治。朝廷遣使廉問,事難周悉。惟提刑勸農采訪之官,自古有之。今分九路專設是職,爾其盡心,往懋乃事。」自熙宗時,遣使廉問吏治得失。世宗即位,凡數歲輒一遣黜陟之,故大定之間,郡縣吏皆奉法,百姓滋殖,號為小康。或謂廉問使者,頗以愛憎立殿最,以問宰相。宰相曰:「臣
 等復為陛下察之。」是以世宗嘗欲立提刑司而未果。章宗追述先朝,遂於即位之初行之。



 及九路提刑使朝辭于慶和殿,上曰:「建立官制,當寬猛得中。凡軍民事相涉者,均平決遣,鈐束家人部曲,勿使沮擾郡縣事。今以司獄隸提刑司,惟翼獄犴無冤耳。」既退,復遣近臣諭之曰:「卿等皆妙簡才良,付以專責,盡心舉職,別有旌賞,否則有罰。」明年,蒲帶乃襲猛安云。



 阿鄰,穎悟辯敏,通女直、契丹大小字及漢字。幼時嘗入宮,熙宗見而奇之,曰:「是兒他日必能宣力國家。」年十八,授定遠大將軍,為順天軍節度使。天德二年,用廉,遷益
 都尹兼山東東路兵馬都總管,歷泰寧、定海、鎮西、安國等軍節度。



 海陵南伐,以為神勇、武平等軍都總管,由壽州道渡淮,與勸農使移刺元宜合兵三萬為先鋒。是歲十月,至廬州,與宋將王權軍十餘萬戰于柘皋鎮,渭子橋,敗之。至和州南,復與王權軍八萬餘會戰,又敗之,追殺至江上,斬首數千級。



 上即位于遼陽。海陵死,大軍北還。將渡淮而舟楫甚少,軍士爭舟不得亟渡。阿鄰得生口,知可涉處,識以柳枝,命本部涉濟。既至北岸,而諸軍之爭渡者果為宋人邀擊之。及入見,上聞阿鄰淮止戰功,又以全軍還,遷兵部尚書,監督經畫征窩斡諸軍糧
 餉,授以金牌一、銀牌四。窩斡敗,還至懿州,以疾卒。喪至京師,上命致祭于永安寺,百官赴吊,賻銀五百兩、重綵三十端、絹百匹。



 按荅海,又名阿魯綰,宗雄次子也。性端重,不輕發,有父之風。年十五,太祖賜以一品傘。二十餘,御球場分朋擊球,連勝三算,宗工舊老咸異之。進呈所勝禮物,按荅海為班首,太宗喜曰:「今日之勝,此孫之力也。」賞力獨厚。



 天眷二年,襲父猛安。除大宗正丞,以猛安讓兄子喚端,加武定軍節度使,奉朝請。改侍衛親軍都指揮使,封金源郡王,進封譚王,遷同判大宗正事,別授世襲猛安。



 海陵
 將遷中都,按荅海諫曰:「棄祖宗興王之地而他徙,非義也。」海陵不悅,留之上京。久之,進封鄆王,改封魏王,除濟南尹。按荅海不堪卑濕,多在病告,海陵聞之,改西京留守。正隆例奪王爵,改廣寧尹。



 世宗即位于東京,赦令至廣寧,弟燕京勸按荅海拒弗受。按荅海受之。會海陵遣使至城下,按荅海登城告使者曰:「此府迫近遼陽,勢不能抗,聊且從命,非得已也。」燕京亦登譙樓與使者語,指斥不遜。及諸郡皆詣東京,按荅海兄弟亦上謁。有司議,既拜赦令,復有異言,持兩端,請併誅之。上曰:「正隆剪刈宗室,朕不可效尤。按荅海為弟所惑耳。」於是釋按荅海,
 乃誅燕京。不數日,復判大宗正事,再遷太子太保,封蘭陵郡王。改勸農使。



 海陵時,自上京徙河間,土瘠,詔按荅海一族二十五家,從便遷居近地,乃徙平州。詔給平州官田三百頃,屋三百間,宗州官田一百頃。進金源郡王,致仕。



 大定八年,召見,上曰:「宗室耆老如卿者,能幾人邪。」賜錢萬貫,甲第一區,留京師,使預巡幸球獵宴會。十四年,薨,年六十七。臨終,戒諸子曰:「汝輩勿以生富貴中而為暴戾,宜自謙退。海陵以猜忌剪滅宗室,我以純謹得免死耳。汝輩惟日為善,勿墜吾家。」



 完顏希尹本名谷神,歡都之子也。自太祖舉兵,常在行
 陣,或從太祖、或從撒改,或與諸將征伐,比有功。



 金人初無文字,國勢日強,與鄰國交好,迺用契丹字。太祖命希尹撰本國字,備制度。希尹乃依仿漢人楷字,因契丹字制度,合本國語,製女直字。天輔三年八月,字書成,太祖大悅,命頒行之。賜希尹馬一匹、衣一襲。其後熙宗亦製女直字,與希尹所製字俱行用。希尹所撰謂之女直大字,熙宗所撰謂之小字。



 遼人迪六、和尚、雅里斯充中京走,希尹與迪古乃、婁室、余睹襲之。迪六等聞希尹兵,復走。遂降其旁近人民而還。奚人落虎來降,希尹使落虎招其父西節度使訛里刺。訛里刺以本部降。



 宗翰駐軍
 北安,使希尹經略近地,獲遼護衛耶律習泥烈,知遼主獵于鴛鴦濼。宗翰遂請進兵。宗翰將會都統杲于奚王嶺。遼兵屯古北口。使婆盧火將兵二百擊之,渾黜亦將二百人為後援。渾黜聞遼兵眾,請益兵。宗翰欲親往,希尹、婁室曰:「此小寇,請以千兵為公破之。」渾黜至古北口,遇遼遊兵,逐之入谷中。遼步騎萬餘迫戰,死者數人。渾黜據關口,希尹等至,大破遼兵,斬馘甚眾,盡獲甲胄輜重。復敗其伏兵,殺千餘人,獲馬百餘匹。遂與宗翰至奚王嶺,期會於羊城濼。



 宗翰襲遼帝於五院司,希尹為前驅,所將纔八騎,與遼主戰,一日三敗之。明日,希尹得降
 人麻哲,言遼主在漠,委輜重,獎奔西京。幾及遼主于白水濼南。遼主以輕騎遁去。盡獲其內庫寶物,遂至西京。西京降,使蒲察守之。希尹至乙室部,不及遼主而還。及宗翰入朝,希尹權西南、西北兩路都統。



 是時,夏人已受盟,遼主已獲,耶律大石自立,而夏國與婁室書責諸帥棄盟,軍入其境,多掠取者。希尹上其書,且奏曰:「聞夏使人約大石取山西諸郡,以臣觀之,夏盟不可信也。」上曰:「夏事酌宜行之。軍入其境,不知信與否也。大石合謀,不可不察,其嚴備之。」



 及大舉伐宋,希尹為元帥右監軍。再伐宋,執二主以歸。師還,賜希尹鐵券,除常赦不原之罪,
 餘釋不問。宗翰伐康王,希尹追之於揚州,康王遁去。後與宗翰俱朝京師,請立熙宗為儲嗣,太宗遂以熙宗為諳班勃極烈。



 熙宗即位,希尹為尚書左丞相兼侍中,加開府儀同三司。希尹為相,有大政皆身先執咎。天眷元年,乞致仕,不許,罷為興中尹。二年,復為左丞相兼侍中,俄封陳王。與宗乾共誅宗磐、宗雋。三年,賜希尹詔曰:「帥臣密奏,姦狀已萌,心在無君,言宣不道。逮燕居而竊議,謂神器以何歸,稔於聽聞,遂致章敗。」遂賜死,并殺右丞蕭慶并希尹子同脩國史把荅、符寶郎漫帶。是時,熙宗未有皇子,故嫉希尹者以此言譖之。



 皇統三年,上知希
 尹實無他心,而死非其罪,贈希尹儀同三司、邢國公,改葬之,蕭慶銀青光祿大夫。天德三年,追封豫王。正隆二年,例降金源郡王。大定十五年,謚貞憲。孫守道、守貞、守能。守道自有傳。



 守貞本名左靨,貞元二年,襲祖谷神謀克。大定改元,收充符寶祗候,授通進,除彰德軍節度副使,遷北京留守,移上京。坐安置契丹戶民部內娶妻,杖一百,除名。二十五年,起為西京警巡使。世宗愛其剛直,授中都左警巡使,遷大興府治中,進同知,改同知西京留守事。御史臺奏守貞治有善狀,世宗因謂侍臣曰:「守貞勳臣子,又有
 材能,全勝其兄守道,它日可用也。」



 章宗即位,召為刑部尚書,兼右諫議大夫。守貞與脩起居注張暐奏言:「唐中書門下入閤,諫官隨之,欲其預聞政事,有所開說。又起居郎、起居舍人,每皇帝視朝,左右對立,有命則臨階俯聽,退而書之,以為起居注。緣侍從官每遇視朝,正合侍立。自來左司上殿,諫官、脩起居注不避,或侍從官除授及議便遣,始令避之。比來一例令臣等迴避,及香閤奏陳言文字,亦不令臣等侍立。則凡有聖訓及所議政事,臣等無緣得知,何所記錄,何所開說,似非本設官之義。若漏泄政事,自有不密罪。」上從之。尋為賀宋生日使,還
 拜參知政事。時上新即政,頗銳意於治,嘗問漢宣帝綜核名實之道,其施行之實果何如。守貞誦「樞機周密,品式詳備」以對,上曰:「行之果何始?」守貞曰:「在陛下厲精無倦耳。」久之,進尚書左丞,授上京世襲謀克。



 明昌三年夏,旱,天子下詔罪己。守貞惶恐,表乞解職。詔曰:「天墻時雨,薦歲為災,所以警懼不逮。方與二三輔弼圖回遺闕,宜思有以助朕脩政。上答天戒,消沴召和,以康百姓。卿達機務,朕所親倚,而引咎求去,其如思助何。」守貞懇辭,乃出知東平府事。命參知政事夾谷衡諭之曰:「卿勳臣之裔,早登無仕,才用聲績,朕所素知。故嗣位之初,擢任政
 府,于今數載,毗贊實多。既久任繁劇,宜均適逸安,矧內外之職,亦當更治,今特授卿是命。東平素號雄籓,兼比年饑歉,正賴經畫,卿其為朕往綏撫之。」仍賜金幣、廄馬,以寵其行。它日,上問宰臣:「守貞治東平如何?」對曰:「亦不勞力。」上曰:「以彼之才,治一路誠有餘矣。」右丞劉瑋曰:「方今人材無出守貞者,淹留於外,誠可惜也。」上默然。尋改西京留守。



 監察御史蒲刺都劾奏守貞前宴賜北部有取受事,不報。右拾遺路鐸上章辯之。四年,召拜平章政事,封蕭國公。上御後閤,召守貞曰:「朕以卿乃太師所舉,故特加委用。然比者行事多太過,門下人少慎擇,復與
 丞相不協,以是令卿補外。載念我昭祖、太祖開創以來,乃祖佐命,積有勳勞,茲故召用。卿其勉盡乃心,與丞相議事宜相和諧,率循舊章,無輕改革。」因賜玉帶,併以蒲刺都所彈事與之,曰:「朕度卿必不爾,故以示卿。」



 舊制,監宗御史凡八員,漢人四員皆進士,而女直四員則文資右職參注。守貞曰:「監察乃清要之職,流品自異,俱宜一體純用進士。」一日奏事次,上問司吏移轉事。守貞曰:「今吏權重而積弊深,移轉為便。」上嘗歎文士卒無如黨懷英者,守貞奏進士中若趙渢、王庭筠甚有時譽。上曰:「出倫者難得耳。」守貞曰:「間世之才,自古所難。然國家培養
 久,則人材將自出矣。」守貞因言:「國家選舉之法,惟女直、漢人進士得人居多,此舉更宜增取。其諸司局承應人舊無出身,大定後才許敘使。經童之科,古不常設,唐以諸道表薦,或取五人至十人。近代以為無補,罷之。本朝皇統間,取及五十人,因為常選。天德間,尋以停罷。陛下即位,復立是科,朝廷寬大,放及百數,誠恐積久不勝銓擬。宜稍裁減,以清流品。」又言節用省費之道,並嘉納焉。



 先是,鄭王允蹈等伏誅,上以其家產均給諸王,戶部郎中李敬義言恐因之生事,上又以董壽為宮籍監都管勾,並下尚書省議。守貞奏:「陛下欲以允蹈等家產分賜
 懿親,恩命已出,恐不可改。今已減諸王弓矢,府慰司其出入,臣以為賜之無害。如董壽罪人也,特恩釋之,已為幸矣,不宜更加爵賞。」上是守貞所言。



 自明昌初,北邊屢有警,或請出兵擊之。上曰:「今方南議塞河,而復用兵於北,可乎?守貞曰:「彼屢突軼吾圉,今一懲之,後當不復來,明年可以見矣。」上因論守禦之法。守貞曰:「惟有皇統以前故事,捨此無法耳。」



 守貞讀書,通法律,明習國朝故事。時金有國七十年,禮樂刑政因遼、宋舊制,雜亂無貫,章宗即位,乃更定脩正,為一代法。其儀式條約,多守貞裁訂,故明昌之治,號稱清明。又喜推轂善類,接援後進,朝
 廷正人,多出入門下。



 先是,上以疑忌誅鄭王允蹈,後張汝弼妻高陀斡獄起,意又若在鎬王允中。時右諫議大夫賈守謙上疏陳時事,思有以寬解上意。右拾遺路鐸繼之,言尤切直。帝不悅。守貞持其事,獄久不決。帝疑有黨,乃出守貞知濟南府事,仍命即辭,前舉守貞者董師中、路鐸等皆補外,上語宰臣曰:「守貞固有才力,至其讀書,方之真儒則未也。然太邀權譽,以彼之才而能平心守正,朝廷豈可少離。今茲令出,蓋思之熟矣。」俄以在政府日嘗與近侍竊語宮掖事,而妄稱奏下,上命有司鞫問,守貞款伏,奪官一階,解職。遣中使持詔責諭之曰:「挾
 姦罔上,古有常刑,結援養交,臣之大戒。孰謂予相,乃蹈厥辜。爾本出勛門,浸登膴仕。朕初嗣位,亟欲用卿。未閱歲時,升為宰輔,每期納誨,共致太平。蓋求所長,不考其素,拔擢不為不峻,任用不為不專。曾報效之綽思,輒私權之自樹,交通近侍,密問起居,窺測上心,預圖趨向。繇患失之心重,故欺君之罪彰,指所無之事而妄以肆誣,實未始有言而謂之嘗諫。義豈知於歸美,意專在於要君。其飾詐之若然,豈為臣之當耳。復觀彈奏,益見私情,求親識之援而列布宮中,縱罪廢之餘而出入門下。而又凡有官使,斂為己恩,謂皆涉於回邪,不宜任之中外。
 質之清議,固所不容,揆之乃心,烏得無愧。姑從輕典,庸示蒲懲。」仍以守貞不公事,宣諭百官於尚書省。



 承安元年,降授河中防禦使。五年,改部羅火扎石合節度使。過闕,上賜手詔責諭之,令赴職。久之,遷知都府事。時南鄙用兵,上以山東重地,須大臣安撫,乃移知濟南府,卒。上聞而悼之。敕有司致祭,賻贈禮物依故平章政事蒲察通例。謚曰肅。



 守貞剛直明亮,凡朝廷論議及上有所問,皆傳經以對。上嘗與泛論人材,守貞乃迹其心術行事,臧否無少隱,故為胥持國輩所忌,竟以直罷。後趙秉文由外官入翰林,遽上書言:「願陛下進君子退小人。」上問
 君子小人謂誰。」秉文對:「君子故相完顏守貞,小人今參知政事胥持國。」其為天下推重如此。



 守能本名胡刺,累官商州刺史。正隆末,宋人陷商州,守能被執。大定五年,宋人請和,誓書曰:「俘虜之人,盡數發還。」完顏仲為報問國信使,求守能及新息縣令完顏按辰於宋,遂與俱歸。守能等至京師,入見,詔給舊官之俸。



 大定十九年,為西北路招討使。是時,詔徙窩斡餘黨於臨潢、泰州。押刺民列嘗從窩斡,其弟閘敵也當徙,偽稱身亡,以馬賂守能,固匿不遣。及受賕補賽也蕃部通事,事覺。是時,烏古里石壘部族節度副使奚沙阿補杖殺
 無罪鎮邊猛安,尚書省俱奏其事。上曰:「守能由刺史超擢至此,敢恣貪墨。向者招討司官多進良馬、橐駝、鷹鶻等物,蓋假此以率斂爾,自今並罷之。」因責其兄守道曰:「守能自刺史躐遷招討,外官之尊,無以踰此。前招討哲典以貪墨伏誅,守能豈不知,乃敢如此,其意安在。爾之親弟,何不先訓戒之也。」上謂宰臣曰:「監察專任糾彈。宗州節度使阿思懣初之官,途中侵擾百姓,到官舉動皆違法度。完顏守能為招討使,貪冒狼籍。凡達官貴人,皆未嘗舉劾。斡睹只群牧副使僕散那也取部人球杖兩枝,即便彈奏。自今,監察御史職事脩舉,然後遷除。不舉
 職者,大則降罰,小則決責,仍不得去職。」尚書省奏,守能兩贓俱不至五十貫,抵罪。奚沙阿補解見居官,並解世襲謀克。上曰:「此舊制之誤。居官犯除名者,與世襲並罷之,非犯除名者勿罷。」遂著於令。特詔守能杖二百,除名。



 贊曰:阿離合懣之善頌,宗雄之強識,希尹之敏學,益之以征伐之功,豈不偉哉。



\end{pinyinscope}