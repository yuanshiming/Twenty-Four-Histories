\article{列傳第七十}

\begin{pinyinscope}

 逆臣



 ○秉德本名乙辛唐括辯烏帶大興國徒單阿里出虎僕散師恭本名忽土徒單貞李老僧完顏元宜紇石烈執中本名胡沙虎



 昔者孔子作《春秋》而亂臣賊子懼,其法有五焉:微而顯,
 志而晦,婉而成章,盡而不污,懲惡而勸善。夫懲惡乃所以勸善也,作《逆臣傳》。



 秉德,本名乙辛。初為西南路招討使,改汴京留守。丁母憂,起復為兵部尚書,拜參知政事。皇統八年,與烏林答蒲盧虎等廉察郡縣,使還,拜平章政事。廷議欲徙遼陽渤海人屯燕南,秉德及左司郎中三合議其事。近侍高壽星在徙中,壽星訴於悼后,后以白帝,帝怒,杖秉德而殺三合。是時熙宗在位久,悼后干政,而繼嗣未立,帝無聊不平,屢殺宗室,箠辱大臣。秉德以其故懷忿,乃與唐括辯、烏帶等謀廢立。



 烏帶以其謀告海陵,海陵乃與
 秉德謀弒熙宗。皇統九年十二月九日,遂與唐括辯、烏帶、忽土、阿里出虎、大興國、李老僧、海陵妹夫特廝,弒熙宗于寢殿。秉德初意不在海陵,已弒熙宗,未有所屬,忽土奉海陵坐,秉德等皆拜稱萬歲。殺曹國王宗敏、左丞相宗賢。時秉德位在海陵上,因被杖怨望謀廢立,而海陵因之以為亂。既立,以秉德為左丞相,兼侍中、左副元帥,封蕭王,賜鐵券,與錢二千萬、絹一千匹、馬牛各三百、羊三千。久之,為烏帶所譖,出領行臺尚書省事。



 時秉德方在告,亟召之,限十日內發行。會海陵欲除太宗諸子,並除秉德,以秉德首謀廢立,及弒熙宗下即勸進,銜之。
 烏帶因言秉德與宗本謀反有狀,曰:「昨來秉德曾於宗本家飲酒,海州刺史子忠言,秉德有福,貌類趙太祖,秉德偃仰笑受其言。臣妻言秉德妻嘗指斥主上,語皆不順。及秉德與宗本相別時,指斥尤甚,且謂歷數有歸。秉德招刑部侍郎漫獨曰『已前曾說那公事,頗記憶否』。漫獨曰,『不存性命事何可對眾便說』。似此逆狀甚明。」海陵遣使就行臺殺秉德,并殺前行臺參知政事烏林答贊謀。



 贊謀妻,秉德乳母也。初,贊謀與前行臺左丞溫敦思忠同在行臺,思忠黷貨無厭,贊謀薄之,由是有隙,故思忠乘是並誣贊謀及其子,殺之。贊謀不肯跪受刑,行刑
 者立而縊殺之。海陵以贊謀家財奴婢盡賜思忠。



 秉德與烏帶以口語致怨,既死遂并殺其弟特里、颭里,及宗翰子孫,死者三十餘人,宗翰之後遂絕。世宗即位,追復秉德官爵,贈儀同三司。



 初,撒改薨,宗翰襲其猛安親管謀克。秉德死,海陵以賞烏帶,傳其子兀答補,大定六年,世宗憫宗翰無後,詔以猛安謀克還撒改曾孫盆買,遣使改葬撒改、宗翰於山陵西南二十里,百官致奠,其家產給近親以奉祭祀。



 秉德既死,其中都宅第,左副元帥杲居之。杲死,海陵遷都,迎其嫡母徒單氏居之。徒單遇害,世宗惡其不祥,施為佛寺。



 唐括辯,本名斡骨剌。尚熙宗女代國公主,為駙馬都尉。累官參知政事、尚書左丞。與右丞相秉德謀廢立,而烏帶以告海陵,海陵謂辯曰:「我輩不以匡救,旦暮且及禍。若行大事,誰可立者?」辯曰:「無乃胙王常勝乎?」海陵問其次,辯曰:「鄧王子阿楞。」海陵曰:「阿楞屬疏,安得立。」辯曰:「公豈有意邪?」海陵曰:「若不得已,捨我其誰。」於是,旦夕相與密謀。護衛將軍特思疑之,以告悼后曰:「辯等因間每竊竊偶語,不知議何事。」悼后以告熙宗,熙宗怒,召辯責之曰:「爾與亮謀何事,將如我何。」杖而遣之。自是謀益甚。



 十二月九日,代國公主為其母悼后作佛事,居寺中,故海
 陵、秉德等俱會於辯家。至夜,辯等以刀藏衣下,相隨入宮,門者以辯駙馬不疑,皆內之。至殿門,直宿護衛覺之,辯舉刀呵之使無動。既弒熙宗,立海陵,辯為尚書右丞相兼中書令,封王,賜錢二千萬、絹千匹、馬牛各三百、羊三千、並鐵券。進拜左丞相。父彰德軍節度使重國,遷東平尹。



 初,辯與海陵謀逆,辯嘗言其家奴多可用者,海陵固已懷之。及行弒之夕會於辯家,待興國出宮,辯因設饌,眾皆恇懼不能食,辯獨飽食自若,海陵由此知其忮忍,畏忌之,及即位,嘗與辯觀太祖畫像,海陵指示辯曰:「此眼與爾相似。」辯色動,海陵亦色動,由是疑辯,益忌之。
 及與蕭裕謀致宗本罪,並致辯嘗與宗本謀反,即殺之。



 重國坐奪官,正隆二年,起為沂州防禦使,改清州防禦使。大定初,重國與徒單拔改俱以政跡著聞,歷安國、彰化、橫海軍節度使。



 後辯子孫上書,言辯死天德間,祖重國亦坐追削。正隆初,重國已復官職,乞追復辯官爵。是時,海陵已降為庶人,以辯與弒逆,不許。



 言本名烏帶,行臺左丞相阿魯補子也。熙宗時,累官大理卿。熙宗晚年喜怒不常,大臣往往危懼,右丞相秉德、左丞唐括辯謀廢立,烏帶即詣海陵啟之,遂與俱弒熙宗。海陵即位,烏帶為平章政事,封許國王,賜錢、絹、馬、牛、羊、鐵券,
 並如其黨。



 烏帶妻唐括氏淫泆,舊與海陵通,又私其家奴閻乞兒,秉德嘗對熙宗斥其事,烏帶銜之未發也。時海陵多忌,會有疾,少間,烏帶遂誣奏:「秉德有指斥語,曰:主上數日不視朝,若有不諱,誰當繼者?臣曰:主上有皇子。秉德曰:嬰兒豈能勝天下大任,必也葛王乎」。海陵以為實然,故出秉德,已而殺之,以秉德世襲猛安謀克授烏帶。進右丞相。烏帶與宗本有親,海陵以烏帶告秉德事,故宗本之禍烏帶獨免,遂以秉德千戶謀克及其子婦家產盡賜之。進司空、左丞相、兼侍中。



 居數月,烏帶早朝,以日陰晦將雨,意海陵不視朝,先趨出朝,百官皆隨
 之去。已而海陵御殿,知烏帶率百官出朝,惡之,遂落司空,出為崇義軍節度使。後海陵思慕唐括容色,因其侍婢來候問起居,海陵許立為后,使殺烏帶。海陵詐為烏帶哀傷,使其子兀答補佩金符乘驛赴喪,追封為王,仍詔有司送其靈車,賜絹三百為道途費。納唐括於宮中,封貴妃。



 兀答補襲猛安謀克。大定六年,以猛安謀克還撒改曾孫,以阿魯補謀克授兀答補,終同知大興尹。子瑭,本名烏也阿補,以曾祖阿魯補功,充筆硯祗候。



 大興國,事熙宗為寢殿小底,權近侍局直長,最見親信,未嘗去左右。每逮夜,熙宗就寢,興國時從主者取符鑰
 歸家,主者即以付之,聽其出入以為常。皇統九年,海陵生日,熙宗使興國以宋司馬光畫像及他珍玩賜海陵,悼后亦以物附賜,熙宗不悅,杖興國一百。



 海陵謀弒,意先得興國乃可伺間入宮行大事,且度興國無罪被杖必有怨望心,可乘此說之,乃因李老僧結興國。既而,知無異心可與謀,乃召至臥內,令解衣,欲與之俱臥,意有所屬者。興國固辭不敢,曰:「即有使,惟大王之命。」海陵曰:「主上無故殺常勝,又殺皇后。乃以常勝家產賜阿楞,既又殺阿楞,遂以賜我。我深以為憂,奈何?」興國曰:「是固可慮也。」海陵曰:「朝臣旦夕危懼,皆不自保。向者我生日,因
 皇后附賜物,君遂被杖,我亦見疑。主上嘗言會須殺君,我與君皆將不免,寧坐待死何如舉大事。我與大臣數人謀議已定,爾以為如何?」興國曰:「如大王言,事不可緩也。」乃約十二月九日夜起事。興國取符鑰開門,矯詔召海陵入。夜二更,海陵、秉德等入。熙宗常置佩刀於御榻上,是夜興國先取投榻下,及亂作,熙宗求佩刀不得,遂遇弒。



 海陵既立,以興國為廣寧尹,賜奴婢百口、犀玉帶各一、錢絹馬牛鐵券如其黨,進階金紫光祿大夫。再賜興國錢千萬、黃金四百兩、銀千兩、良馬四匹、駝車一乘、橐駝三頭、真珠巾、玉鉤帶、玉佩刀、及玉校鞍轡。天德四
 年,改崇義軍節度使,賜名邦基。再授絳陽、武寧節度使,改河間尹。



 世宗即位,廢于家,凡海陵所賜皆奪之。大定中,邦基兄邦傑自京兆判官還,世宗曰:「大邦傑因其弟進,濫廁縉紳,豈可復用。」併罷其子弟與所贈父官。及海陵降為庶人,詔曰:「大邦基與海陵同謀弒逆,逋誅至今,為幸多矣。」遂磔于思陵之側。



 徒單阿里出虎,會寧葛馬合窟申人,徙懿州。父拔改,太祖時有戰功,領謀克,曷速館軍帥,皇統四年為兵部侍郎,歷天德軍節度使,改興中尹,與宗幹世為姻家。皇統九年,阿里出虎與僕散忽土俱為護衛十人長。海陵將
 弒熙宗,欲得二人者為內應,遂許以女妻阿里出虎子,而以逆謀告之。阿里出虎素凶暴,聞其言喜甚,曰:「阿家此言何晚邪,廢立之事亦男子所為。主上不能保天下,人望所屬惟在阿家,今日之謀乃我素志也。」遂與忽土俱以十二月九日直禁中,海陵故以是夜二更入宮,至寢殿,阿里出虎先進刃,忽土次之,熙宗頓仆,海陵復刃之,血濺其面及衣。



 海陵既立,以阿里出虎為右副點檢,賜錢絹馬牛羊如其黨,子術斯剌尚榮國公主合女,加昭毅大將軍駙馬都尉。天德二年,留守東京,加儀同三司。八月,改河間尹,世襲臨潢府路斜剌阿猛安領親管
 謀克。以憂去職,起復為太原尹,封王。



 阿里出虎自謂有佐立功,受鐵券,凶狠益甚,奴視僚屬,少忤其意輒箠辱無所恤。嘗問休咎於卜者高鼎,遂以鼎所占問張王乞。王乞以謂當有天命,阿里出虎喜,以王乞語告鼎。鼎上變,阿里出虎伏誅,并殺其妻及王乞。海陵使其子術斯剌焚其屍,投骨水中。



 拔改自西京留守歷西南路招討使、忠順軍節度使,入為勸農使,復為河間尹,改臨洮尹,入為工部尚書,改興平軍節度使,濟南尹,卒。



 僕散師恭,本名忽土,上京老海達葛人。本微賤,宗乾嘗周恤之,擢置宿衛為十人長。海陵謀逆,以忽土出自其
 家,有恩,欲使為內應,謂之曰:「我有一言欲告君久矣,恐泄於人,未敢也。」忽土曰:「肌肉之外,皆先太師所賜,茍有補於國王,死不敢辭。」先太師,謂宗乾也。海陵曰:「主上失道,吾將行廢立事,必得君為助乃可。」忽土許之。



 十二月九日,忽土直宿,海陵因之入宮。至寢殿,熙宗聞步屣聲,咄之,眾皆卻立不敢動,忽土曰:「事至此,不進得乎?」乃相與排闥而入。既弒熙宗,秉德等尚未有所屬,忽土曰:「始者議立平章,今復何疑。」乃奉海陵坐,眾前稱萬歲。遂召曹國王宗敏至,即使忽土殺之。



 既即位,忽土為左副點檢,賜錢絹馬牛羊鐵券。轉都點檢,改名師恭。遷會寧牧,
 拜太子少師、工部尚書,封王。頃之,以憂解職。起復為樞密副使,進拜樞密使。貞元三年,為右丞相。正隆初,拜太尉,復為樞密使。無何,以憂去,起復為太尉、樞密使。



 海陵至汴京,賜忽土第一區,鄰寧德宮。宮,徒單太后所居也,忽土時時入見太后。及契丹撒八反,海陵命忽土與蕭懷忠北伐。比行,忽土入辭寧德,太后與語久之。海陵聞而惡之,疑其與太后有異謀。是時,蕭禿剌、斡盧補與契丹撒八連戰皆無功,糧運不繼,乃退軍臨潢。而撒八聞師恭以大軍且至,乃謀歸大石,沿龍駒河西去。師恭至臨潢,追之不及。海陵使樞密副使白彥敬等討撒八,師
 恭還,遣其子忽殺虎乘傳逆之,至則執而戮于市。師恭臨刑,繩枚窒口不能言,但舉首視天日而已。遂族滅之,並誅滅蕭禿剌、蕭賾、蕭懷忠家。



 大定初,皆復官爵。及海陵降為庶人,師恭以預弒復削之。世宗幸上京,過老海達葛。師恭族人臨潢尹守中、定遠大將軍阿里徒等皆奪官。二十八年,上謂宰相曰;「海陵遣僕散師恭、蕭禿剌、蕭懷忠追撒八不及,皆坐誅,遂夷其族,虐之甚也。」平章政事襄對曰:「是時臣在軍中,忽土、賾有精甲一萬三千有餘,賊軍雖多皆脅從之人,以氈紙為甲,易與也。忽土等恇怯遷延,賊乃遁去。」上曰:「審如是,則誅之可也。」兄渾
 坦。



 徒單貞,本名特思,忒黑闢剌人也。祖抄,從太祖伐遼有功,授世襲猛安。父婆盧火,以戰功累官開府儀同三司。貞娶遼王宗乾女,海陵同母女弟也。皇統九年、貞與海陵俱弒熙宗。海陵既立,以貞為左衛將軍,封貞妻平陽長公主,貞為駙馬都尉、殿前左副點檢。轉都點檢,兼太子少保,封王。改大興尹,都點檢如故。俄授臨潢府路昏斯魯猛安。



 居二年,海陵召貞勖之曰:「汝自幼常在左右,頗著微勞,而近日乃怠忽,縱有罪,樹私恩。凡人富貴而驕,皆死徵也。汝若不制汝心,將無所不至,賜之死復何
 辭。朕念弟襄及公主與朕同胞,故少示懲戒。」貞但號泣。即日解點檢職,仍為大興尹,復戒之曰:「今而後能以勤自勵,朕當思之。不然,黜爾歸田里矣。」逾月,復為都點檢、大興尹如故。正隆二年,例封沈。遷樞密副使,賜佩刀入宮,轉同判大宗正事。



 海陵將伐宋,詔朝官除三國人使宴飲,其餘飲酒者死。六年正月四日立春節,益都尹京、安武節度使爽、金吾上將軍阿速飲於貞第。海陵使周福兒賜土牛至貞第,見之以告,海陵召貞詰之曰:「戎事方殷,禁百官飲酒,卿等知之乎?」貞等伏地請死,海陵數之曰:「汝等若以飲酒殺人太重,固當諫,古人三諫不聽
 亦勉從君命。魏武帝《軍行令》曰『犯麥者死』。已而所乘馬入麥中,乃割髮以自刑。犯麥,微事也,然必欲以示信。朕為天下主,法不能行于貴近乎?朕念慈憲太后子四人,惟朕與公主在,而京等皆近屬,曲貸死罪。」於是杖貞七十,京等三人各杖一百,降貞為安武軍節度使,京為灤州刺史,爽歸化州刺史。



 無何,拜貞御史大夫,以本官為左監軍,從伐宋。至揚州,海陵死,北還。見世宗於中都,詔以貞女為皇太子妃,除貞為太原尹,改咸平。貞在咸平貪汙不法,累贓巨萬,徙真定尹,事覺。世宗使大理卿李昌圖鞫之,貞即引伏,昌圖還奏,上問之曰:「貞停職否?」對
 曰:「未也。」上怒,抵昌圖罪,復遣刑部尚書移剌道往真定問之,徵其贓還主。有司徵給不以時,詔先以官錢還其主,而令貞納官。凡還主臟,皆準此例。降貞為博州防禦使,降貞妻為清平縣主。



 頃之,遷震武節度使,遣使者往戒敕之,詔曰:「朕念卿懿戚,不待終考,更遷大鎮。非常之恩不可數得,卿勿蹈前過。」轉河中尹。進封其妻為任國公主,賜黃金百兩、重彩二十端,賜貞擊球馬二匹。改東京留守,賜玉吐鶻、弓矢,賜貞妻錢萬貫。



 有司奏:「海陵已貶為庶人,宗乾不當猶稱帝。」於是,以宗乾有社稷功,詔追封為遼主,其子孫及諸女皆降,貞妻降永平縣主,貞
 自儀同三司降特進,奪猛安,不稱駙馬都尉。再徙臨潢尹。



 初,與弒熙宗凡九人,海陵以暴虐自斃,秉德、辯、忽土、阿里出虎以疑見殺,言以妻殞,裕、老僧以反誅,至是貞與大興國尚在。而興國擯棄不用,獨貞以世姻籍恩寵,雖夫婦降削爵號,而世宗慮久遠,終不以私恩曲庇,久之,詔誅貞及其妻與二子慎思、十六,而宥其諸孫。俄而,興國亦誅,皇統逆黨盡矣。



 章宗即位,尊母皇太子妃為皇太后,追封貞為太尉梁國公,貞祖抄司空魯國公,父婆盧火司徒齊國公,貞妻梁國夫人,子陀補火、慎思、十六俱為鎮國上將軍。無何,再贈貞太師、廣平郡王,謚莊
 簡。貞妻進封梁國公主。



 李老僧,舊為將軍司書吏,與大興國有親,素相厚。海陵秉政,興國屬諸海陵,海陵以為省令史。及將舉事,使老僧結興國,興國終為海陵取符鑰,納海陵宮中成弒逆者,老僧為之也。海陵既立,以老僧為同知廣寧尹事,賜錢千萬、絹五百匹、馬牛各二百、羊二千。



 久之,海陵惡韓王亨,將殺之,求其罪不可得,遂以亨為廣寧尹,再任老僧同知,使伺察亨,構致其罪。亨喜博,及至廣寧,常與老僧博,待之甚厚。老僧由是不忍致亨死罪,遲疑者久之。海陵再使小底訛論促老僧,老僧乃與亨家奴六斤謀,
 殺亨獄中,語在亨傳。及耶律安禮自廣寧還朝,海陵謂之曰:「孛迭三罪,伏其一已見觖望。爾乃梁王故吏,若亨伏辜,必罪及親族,故榜殺之。」



 海陵以老僧於亨有遲回意,遂降老僧為易州刺史。久之,遷同知大興尹,賜名惟忠,改延安府同知,大定二年,與兵部尚書可喜謀反,誅。



 論曰:《書》曰:「王左右常伯、常任、準人、綴衣、虎賁。周公曰:嗚呼,休茲知恤,鮮哉!」穆王告伯冏曰:「慎簡乃僚,其無以巧言令色、便辟側媚,其惟吉士。」金人所謂寢殿小底猶周之綴衣,所謂護衛猶周之虎賁也,則皆群僕侍御之臣矣。海陵弒逆,而大興國、忽土、阿里出虎為之扼擘,皆出
 於小底護衛之中,熙宗固不知恤之也。一日,熙宗與近侍飲酒,會夜,稽古殿火,上欲往視,都點檢辭不失引帝裾止之,奏曰:「臣在此,陛下何患,願無親往。」熙宗謂辭不失被酒,甚怒之,明日,杖而出之,已而思其忠,復見召用。海陵與唐括辯時時屏人私語,護衛特思察其非常,海陵擠而殺之。皇統末年,群臣解體,無尊君謹上之心,而群姦竊發,僕御之臣不復有如辭不失、特思者矣。《綿》之詩曰:「予曰有疏附,予曰有先後,予曰有奔走,予曰有禦侮。」嗚呼,先後禦侮之臣,豈可少哉!



 完顏元宜,本名阿列,一名移特輦,本姓耶律氏。父慎思,
 天輔七年,宗望追遼主至天德,慎思來降,且言夏人以兵迎遼主,將渡河去。宗望移書夏人諭以禍福,夏人乃止。賜慎思姓完顏氏,官至儀同三司。



 元宜便騎射,善擊球。皇統元年,充護衛,累遷甌里本群牧使,入為武庫署令,轉符寶郎,海陵篡立,為兵部尚書。天德三年,詔凡賜姓者皆復本姓,元宜復姓耶律氏。歷順義、昭義節度使,復為兵部尚書、勸農使。



 海陵伐宋,以本官領神武軍都總管,以大名路騎兵萬餘益之。前鋒渡淮,拔昭關,遇宋兵萬餘於柘皋,力戰卻之。至和州,宋兵十萬來拒,元宜麾軍力戰,抵暮而罷。宋人乘夜襲營,元宜擊走之,黎明
 追及宋兵,斬首數萬,以功遷銀青光祿大夫。海陵增置浙西道都統制,使元宜領之,督諸軍渡江,佩金牌,賜衣一襲。



 是時,世宗已即位于遼陽,軍中多懷去就。海陵軍令慘急,亟欲渡江,眾欲亡歸,決計於元宜。猛安唐括烏野曰:「前阻淮渡,皆成擒矣。比聞遼陽新天子即位,不若共行大事,然後舉軍北還。」元宜曰:「待王祥至謀之。」王祥者元宜子,為驍騎副都指揮使,在別軍。元宜使人密召王祥,既至,遂約詰旦衛軍番代即行事。元宜先欺其眾曰:「有令,爾輩皆去馬,詰旦渡江。」眾皆懼,乃以舉事告之,皆許諾。



 十月乙未黎明,元宜、王祥與武勝軍都總管徒
 單守素、猛安唐括烏野、謀克斡盧保、婁薛、溫都長壽等率眾犯御營。海陵聞亂,以為宋兵奄至,攬衣遽起,箭入帳中,取視之,愕然曰:「乃我兵也。」大慶山曰:「事急矣,當出避之。」海陵曰:「走將安往。」方取弓,已中箭仆地。延安少尹納合斡魯補先刃之,手足猶動,遂縊殺之。驍騎指揮使大磐整兵來救,王祥出語之曰:「無及矣。」大磐乃止。軍士攘取行營服用皆盡,乃取大磐衣巾裹海陵尸,焚之。遂收尚書右丞李通、浙西道副統制郭安國、監軍徒單永年、近侍局使梁珫、副使大慶山,皆殺之。元宜行左領軍副大都督事,使使者殺皇太子光英于南京。大軍北還。



 大定二年春,入見,拜御史大夫,詔曰:「高楨為御史大夫,號為正直,頗涉煩碎,臣下衣冠不正亦被糾舉。職事有大於此者,爾宜勉之。」未幾,拜平章政事,封冀國公。賜玉帶、甲第一區,復賜姓完顏氏。



 往泰州路規措討契丹事,元宜使忠勇校尉李榮招窩斡,窩斡殺榮,詔追贈榮進官四階。五月,上聞元宜將還,遣使止之。契丹已平、元宜還朝,奏請益諸群牧鎧甲。詔從之,每群牧益二十副。元宜復請益臨潢戍軍士馬,詔給馬六百匹。久之,罷為東京留守。乞還所賜甲第,上從之,賜以襲衣、吐鶻、廄馬、海東青鶻。未幾,致仕,薨于家。上聞之,遣使致祭,賻贈甚厚。



 大定十一年,尚書省奏擬納合斡魯補除授,上曰:「昔廢海陵,此人首入弒之,人臣之罪莫大於是,豈可復加官使?其世襲謀克姑聽仍舊。」大定十八年,扎里海上言:「凡為人臣能捍災禦侮有功者,宜錄用之。今弒海陵者以為有功,賞以高爵,非所以勸事君也。宜削奪,以為人臣之戒。臣在當時亦與其黨,如正名定罪,請自臣始。」上曰:「扎里海自請其罪以勸事君,此亦人之所難。」遂以扎里海充趙王府祗候郎君。



 元宜子習涅阿補,大定二十五年為符寶祗候,乞依女直人例遷官,上曰:「賜姓一時之權宜。」令習涅阿補還本姓。



 論曰:《春秋》書「齊公子商人弒其君舍」,又曰:「齊人弒其君商人。」嗟乎,弒舍者商人也,弒商人者邴埸、閻職也。海陵弒熙宗,完顏元宜弒海陵。商人之弒也,邴埸、閻職去之。海陵之弒也,元宜歸于世宗。邴、閻賤役,元宜都將也,握君之親兵,窺利以弒之,其罪豈容誅乎,世宗僅能不大用之而已。扎里海猶殺人而自首者也,在律,殺人未聞準首免罪而又予賞者也,況弒逆乎。海陵弒五十三年,得有胡沙虎之事。



 紇石烈執中,本名胡沙虎,阿疏裔孫也。徙東平路猛安。大定八年,充皇太子護衛,出職太子僕丞,改鷹坊直長,
 再遷鷹坊使、拱衛直指揮使。明昌四年,使過阻居,監酒官移剌保迎謁後時,飲以酒,酒味薄,執中怒,毆傷移剌保,詔的決五十。未幾,遷右副點檢,肆傲不奉職,降肇州防禦使。踰年,遷興平軍節度使。丁母憂,起復歸德軍節度使,改開遠軍兼西南路招討副使。俄知大名府事。承安二年,召為簽樞密院事。詔佐丞相襄征伐,執中不欲行,奏曰:「臣與襄有隙,且殺臣矣。」上怒其言不遜,事下有司,既而赦之,出為永定軍節度使。改西北路招討使,復為永定軍,坐奪部軍馬解職。



 泰和元年,起知大興府事。詔契丹人立功官賞恩同女直人,許存養馬匹,得充司
 吏譯人,著為令。執中格詔不下,上責之曰;「汝雖意在防閑,而不知朝廷自有定格,自今勿復如此煩碎生事也。」乃下詔行之。



 淶水人魏廷實祖任兒,舊為靳文昭家放良,天德三年,編籍正戶,已三世矣。文昭孫勍詆廷實為奴,及妄訴毆詈,警巡院鞫對無狀,法當訴本貫。勍訴于府,執中使廷實納錢五百貫與勍。廷實不從,還淶水,執中徑遣鎖致廷實。御史臺請移問,執中轉奏御史臺不依制,府未結斷,令移推。詔吏部侍郎李柄、戶部侍郎粘割合答推問。炳、合答奏御史臺理直,詔乃切責執中。



 御史中丞孟鑄奏彈執中「貪殘專恣,不奉法令。釋罪之後,
 累過不悛。既蒙恩貸,轉生跋扈。如雄州詐認馬,平州冒支俸,破魏廷實家。發其塚墓,拜表不赴,祈雨聚妓,毆詈同僚擅令停職,失師帥之體,不稱京尹之任」。上曰:「執中粗人,似有跋扈爾。」鑄對曰:「明天子在上,豈容有跋扈之臣。」上意寤,取閱奏章,詔尚書省問之。由是改武衛軍都指揮使。



 平章政事僕散揆宣撫河南,執中除山東東西路統軍使。揆行省汴京伐宋,升諸道統軍司為兵馬都統府,執中為山東兩路兵馬都統,定海軍節度使完顏撒剌副之。執中分兵駐金城、朐山,請益發東平路兵屯密、沂、寧海、登、萊以遏兵衝,詔從之,時泰和六年四月也。



 五月,宋兵犯金城,執中遣巡檢使周奴以騎兵三百禦之。會宋益兵轉趨沭陽,謀克三合伏卒五十人篁竹中,伺宋兵過突出擊之,殺十數人,追至縣城,宋兵不敢出。會周奴以兵入城,宋兵踰城走,三合已焚其舟,合擊大破之,斬首五百餘級,殺宋統領李藻,擒忠義軍將呂璋。



 十月,執中率兵二萬出清口,宋以步騎萬餘列南岸,戰艦百艘拒上流,相持累日。執中以舟兵二千搏戰,遏宋舟兵,遣副統移剌古與涅率精騎四千自下流徑渡。宋兵望騎兵登南岸,水陸俱潰。追斬及溺死者甚眾,盡獲其戰艦及戰馬三百,遂克淮陰,進兵圍楚州。遷元帥左
 監軍。執中縱兵虜掠,上聞之,杖其經歷官阿里不孫,放還所掠。未幾,宋人請和,詔罷兵。除西南路招討使,改西京留守。



 大安元年,授世襲謀克,復知大興府事,出知太原府,復為西京留守,行樞密院,兼安撫使。以勁兵七千遇大兵,戰于定安之北,薄暮,先以麾下遁去。眾遂潰。行次蔚州,擅取官庫銀五千兩及衣幣諸物,奪官民馬,與從行私人入紫荊關,杖殺淶水令。至中都,朝廷皆不問。乃遷右副元帥,權尚書左丞。執中益無所忌憚,自請步騎二萬屯宣德州,與之三千,令駐媯川。



 崇慶元年正月,執中乞移屯南口或屯新莊,移文尚書省曰:「大兵來必
 不能支,一身不足惜,三千兵為可憂,十二關、建春、萬寧宮且不保。」朝廷惡其言,下有司按問,詔數其十五罪,罷歸田里。



 明年,復召至中都,預議軍事。左諫議大夫張行信上書曰:「胡沙虎專逞私意,不循公道,蔑省部以示強梁,媚近臣以求稱譽,骫法行事,枉害平民。行院山西,出師無律,不戰先退,擅取官物,杖殺縣令。屯駐媯川,乞移內地,其謀略概可見矣。欲使改易前非。以收後效,不亦難乎。才誠可取,雖在微賤皆當擢用,何必老舊始能立功。一將之用,安危所係,惟朝廷加察,天下幸甚。」丞相徒單鎰以為不可用,參知政事絪跪奏其姦惡,乃止。執中
 善結近倖,交口稱譽。五月,詔給留守半俸,預議軍事。張行信復諫曰:「伏聞以胡沙虎老臣,欲起而用。人之能否,不在新舊。彼向之敗,朝廷既知之矣。乃復用之,無乃不可乎。」遂止。



 上終以執中為可用,賜金牌,權右副元帥,將武衛軍五千人屯中都城北。執中乃與其黨經歷官文繡局直長完顏醜奴、提控宿直將軍蒲察六斤、武衛軍鈐轄烏古論奪剌謀作亂。是時,大元大兵在近,上使奉職即軍中責執中止務馳獵。不恤軍事。執中方飼鷂,怒擲殺之,遂妄稱知大興府徒單南平及其子刑部侍郎駙馬都尉沒烈謀反,奉詔討之。南平姻家福海,別將兵
 屯於城北,遣人以好語招之,福海不知,既至乃執之。



 八月二十五日未五更,分其軍為三軍,由章義門入,自將一軍由通玄門入。執中恐城中出兵來拒,乃遣一騎先馳抵東華門大呼曰:「大軍至北關,已接戰矣。」既而再遣一騎亦如之。使徒單金壽召知大興府徒單南平,南平不知,行至廣陽門西富義坊,馬上與執中相見,執中手槍刺之墮馬下,金壽斫殺之。使烏古論奪剌召沒烈,殺之。符寶祗候鄯陽、護衛十人長完顏石古乃聞亂,遽召大漢軍五百人赴難,與執中戰不勝,皆死之。執中至東華門,使呼門者親軍百戶冬兒、五十戶蒲察六斤,皆不應,
 許以世襲猛安、三品職事官,亦不應。呼都點檢徒單渭河,謂河即徒單鎬也。渭河縋城出見執中,執中命聚薪焚東華門,立梯登城。護衛斜烈、乞兒、親軍春山共掊鎖開門納執中。執中入宮,盡以其黨易宿衛。自稱監國都元帥,居大興府,陳兵自衛。急召都轉運使孫椿年取銀幣賞金壽、奪剌及軍官軍士、大興府輿隸。是夜,召聲妓與親黨會飲。明日,以兵逼上出居衛邸,誘左丞完顏綱至軍中,即殺之。執中意不可測,丞相徒單鎰勸執中立宣宗,執中然之。



 是時,莊獻太子在中都,執中以皇太子儀仗迎莊獻入居東宮。召符寶郎徒單福壽取符寶,陳
 於大興府露階上。盜用御寶出制,除完顏醜奴德州防禦使,烏古論奪剌順天軍節度使,蒲察六斤橫海軍節度使,徒單金壽永定軍節度使,雖除外官,皆留之左右。其餘除拜猶數十人。同時有兩蒲察六斤,其一守東華門不肯從亂者。召禮部令史張好禮欲鑄監國元帥印,好禮曰:「自古無異姓監國者。」乃止。遣奉御完顏忽失來等三人,護衛蒲鮮班底、完顏醜奴等十人,迎宣宗於彰德。使宦者李思忠弒上於衛邸。盡撤沿邊諸軍赴中都平州、騎兵屯薊州以自重,邊戍皆不守矣。



 九月甲辰,宣宗即位,拜執中太師、尚書令、都元帥、監修國史,封澤王,
 授中都路和魯忽土世襲猛安。以其弟同知河南府特末也為都點檢,兼侍衛親軍都指揮使,子豬糞除濮王傅、兵部侍郎,都點檢徒單渭河為御史中丞,烏古論奪剌遙授知真定府事,徒單金壽遙授知東平府事,蒲察六斤遙授知平陽府事,完顏醜奴同知河中府事,權宿直將軍。詔以烏古論誼居第賜執中,儀鸞局給供張,妻王賜紫結銀鐸車。



 戊申,執中侍朝,宣宗賜之坐,執中就坐不辭。無何,執中奏請降衛紹王為庶人,奏再上,詔百官議于朝堂。太子少傅奧屯忠孝、侍讀學士蒲察思忠附執中議,眾相視莫敢言,獨文學田廷芳奮然曰:「先朝
 素無失德,尊號在禮不當削。」於是從之者禮部張敬甫、諫議張信甫、戶部武文伯、龐才卿、石抹晉卿等二十四人。宣宗曰:「闢諸問途,百人曰東行是,十人曰西行是,行道之人果適東乎、適西乎。豈以百人、十人為是非哉?」既而曰:「朕徐思之。」數日,詔降為東海郡侯。



 大元遊騎至高橋,宰臣以聞。宣宗使人問執中,執中曰:「計畫已定矣。」既而讓宰執曰:「吾為尚書令,豈得不先與議而遽奏耶?」宰執遜謝而已。



 提點近侍局慶山奴、副使惟弼、奉御惟康請除執中,宣宗念援立功,隱忍不許。元帥右監軍術虎高琪屢戰不利,執中戒之曰:「今日出兵果無功,當以軍
 法從事矣。」高琪出戰復敗,自度不免,頗聞慶山奴諸人有謀,十月辛亥,高琪遂率所將颭軍入中都,圍執中第。執中聞變,彎弓注矢外射,不勝,登後垣欲走,衣絓墮而傷股,軍士就斬之。高琪持執中首詣闕待罪,宣宗赦之。以為左副元帥。



 執中之黨呼於衢路曰:「颭軍反矣,殺之者有賞。」市人從之。颭軍死者甚眾,一軍皆恟恟,宣宗遣近侍撫諭之,詔有司量加賻贈,眾乃稍安。明日,除特末也泰寧軍節度使,烏古論奮剌真授知濟南府事,徒單金壽真授知歸德府事,蒲察六斤真授知平陽府事。



 甲寅,左諫議大夫張行信上封事曰:「《春秋》之法,國君立不
 以道,若嘗與諸侯盟會,即列為諸侯。東海在位已六年矣,為其臣者誰敢干之。胡沙虎握兵入城,躬行弒逆,當是時惟鄯陽、石古乃率眾赴援,至于戰死,論其忠烈,在朝食祿者皆當愧之。陛下始親萬機,海內望化,褒顯二人,延及子孫,庶幾少慰貞魂,激天下之義氣。宋徐羨之、傅亮、謝晦弒營陽王立文帝,文帝誅之,以江陵奉迎之誠,免其妻子。胡沙虎國之大賊,世所共惡,雖已死而罪名未正,合暴其過惡,宣布中外,除名削爵,緣坐其家,然後為快。陛下若不忍援立之勞,則依仿元嘉故事,亦足以示懲戒。」宣宗乃下詔暴執中過惡,削其官爵。贈鄯陽、
 石古乃,加恩其子。慶山奴、惟弼、惟康皆遷賞,近侍局自此用事矣。



 論曰:金九主,遇弒者三,其逆謀者十人。熙宗之弒,惟大興國一人世宗聲其罪而磔之思陵之側。徒單貞雖誅。未聞暴其罪狀,後以戚畹又復贈官追封。餘秉德、唐括辯等六人,皆以他罪誅,海陵之弒,其首惡為完顏元宜,則令終焉。衛紹王之弒曰胡沙虎,不死於司敗之誅,而死於高琪之手。古所謂弒君之賊人得而討之者,謂請于公上而致討焉。如孔子之請討陳恒是也。豈有如琪之擅殺而以為功者乎。金之政刑,其亂若此,國欲不亡,
 其可得乎!



\end{pinyinscope}