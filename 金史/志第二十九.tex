\article{志第二十九}

\begin{pinyinscope}

 食貨三



 ○錢幣



 錢幣。金初用遼、宋舊錢,天會末,雖劉豫「阜昌元寶」、「阜昌重寶」亦用之。海陵庶人貞元二年遷都之後,戶部尚書蔡松年復鈔引法,遂製交鈔,與錢並用。正隆二年,歷四十餘歲,始議鼓鑄。冬十月,初禁銅越外界,懸罪賞格。括民間銅鍮器,陜西、南京者輸京兆,他路悉輸中都。三年二月,中都置錢監二,東曰寶源,西曰寶豐。京兆置監
 一。曰利用。三監鑄錢,文曰「正隆通寶」,輕重如宋小平錢,而肉好字文峻整過之,與舊錢通用。



 世宗大定元年,用吏部尚書張中彥言,命陜西路參用宋舊鐵錢。四年,浸不行,詔陜西行戶部、并兩路通檢官,詳究其事。皆言:「民間用錢,名與鐵錢兼用,其實不為準數,公私不便。」遂罷之。八年,民有犯銅禁者,上曰:「銷錢作銅,舊有禁令。然民間猶有鑄鏡者,非銷錢而何?」遂併禁之。十年,上諭戶部臣曰:「官錢積而不散,則民間錢重,貿易必艱,宜令市金銀及諸物。其諸路酤榷之貨,亦令以物平折輸之。」十月,上責戶部官曰:「先以官錢率多,恐民間不得流通,令諸處
 貿易金銀絲帛,以圖流轉。今知乃有以抑配反害百姓者。前許院務得折納輕齎之物以便民,是皆朕思而後行者也,此尚出朕,安用若為。又隨處時有賑濟,往往近地無糧,取於它處,往返既遠,人愈難之。何為不隨處起倉。年豐則多糴以備賑贍,設有緩急,亦豈不易辦乎?而徒使錢充府庫,將安用之。天下之大,朕豈能一一遍知,凡此數事,汝等何為而使至此。且戶部與它部不同,當從宜為計,若但務因循,以守其職,則戶部官誰不能為。」十一年二月,禁私鑄銅鏡。舊有銅器悉送官,給其直之半。惟神佛像、鐘、磬、鈸、鈷、腰束帶、魚袋之屬,則存之。十二
 年正月,以銅少,命尚書省遣使諸路規措銅貨。能指坑冶得實者,賞。上與宰臣議鼓鑄之術,宰臣曰:「有言所在有金銀坑冶,皆可採以鑄錢,臣竊謂工費過於所得數倍,恐不可行。」上曰:「金銀,山澤之利,當以與民,惟錢不當私鑄。今國家財用豐盈,若流布四方與在官何異?所費雖多,但在民間,而新錢日增爾。其遣能吏經營之。」左丞石琚進曰:「臣聞天子之富藏在天下,錢貨如泉,正欲流通。」上復問琚曰:「古亦有民自鑄錢者乎?」琚對曰:「民若自鑄,則小人圖利,錢益薄惡,此古所以禁也。」十三年,命非屯兵之州府,以錢市易金帛,運致京師,使錢幣流通,以
 濟民用。十五年十一月,上謂宰臣曰:「或言鑄錢無益,所得不償所費。朕謂不然。天下如一家,何公私之間,公家之費私家得之,但新幣日增,公私俱便也。」十六年三月,遣使分路訪察銅礦苗脈。十八年,代州立監鑄錢,命震武軍節度使李天吉、知保德軍事高季孫往監之,而所鑄斑駁黑澀不可用,詔削天吉、季孫等官兩階,解職,仍杖季孫八十。更命工部郎中張大節、吏部員外郎麻珪監鑄。其錢文曰「大定通寶」,字文肉好又勝正隆之制,世傳其錢料微用銀云。十九年,始鑄至萬六千餘貫。二十年,詔先以五千進呈,而後命與舊錢並用。



 初,新錢之未行
 也,以宋大觀錢作當五用之。二月,上聞上京修內所,市民物不即與直,又用短錢,責宰臣曰:「如此小事,朕豈能悉知?卿等何為不察也。」時民間以八十為陌,謂之短錢,官用足陌,謂之長錢。大名男子斡魯補者上言,謂官私所用錢皆當以八十為陌,遂為定制。二十年十一月,名代州監曰阜通,設監一員,正五品,以州節度兼領。副監一員,正六品,以州同知兼領。丞一員,正七品,以觀察判官兼領。設勾當官二員,從八品。給銀牌,命副監及丞更馳驛經理。二十二年十月,以參加政事粘割斡特剌提控代州阜通監。二十三年,上以阜通監鼓鑄歲久,而錢
 不加多,蓋以代州長貳幕兼領,而奪於州務,不得專意綜理故也。遂設副監、監丞為正員,而以節度領監事。二十六年,上曰:「中外皆言錢難,朕嘗計之,京師積錢五百萬貫亦不為多,外路雖有終亦無用,諸路官錢非屯兵處可盡運至京師。」太慰丞相克寧曰:「民間錢固已艱得,若盡歸京師,民益艱得矣!不若起其半至都,餘半變折輕齎,則中外皆便。」十一月,上諭宰臣曰:「國家銅禁久矣,尚聞民私造腰帶及鏡,託為舊物,公然市之,宜加禁約。」二十七年二月,曲陽縣鑄錢別為一監,以利通為名,設副監、監丞,給驛更出經營銅事。二十八年,上謂宰臣
 曰:「今者外路見錢其數甚多,聞有六千餘萬貫,皆在僻處積貯。既不流散,公私無益,與無等爾。今中都歲費三百萬貫,支用不繼,若致之京師,不過少有挽運之費,縱所費多,亦惟散在民爾。」章宗大定二十九年十二月,鴈門、五臺民劉完等訴:「自立監鑄錢以來,有銅礦之地雖曰官運,其顧直不足則令民共償。乞與本州司縣均為差配。」遂命甄官署丞丁用楫往審其利病,還言:「所運銅礦,民以物力科差濟之,非所願也。其顧直即低,又有刻剝之弊。而相視苗脈工匠,妄指人之垣屋及寺觀謂當開採,因以取賄。又隨冶夫匠,日辦凈銅四兩,多不及數,
 復銷銅器及舊錢,送官以足之。今阜通,利通兩監,歲鑄錢十四萬餘貫,而歲所費乃至八十餘萬貫,病民而多費,未見其利便也。」宰臣以聞,遂罷代州、曲陽二監。



 初,貞元間既行鈔引法,遂設印造鈔引庫及交鈔庫,皆設使、副、判各一員,都監二員,而交鈔庫副則專主書押、搭印合同之事。印一貫、二貫、三貫、五貫、十貫五等,謂之大鈔;一百、二百、三百、五百、七百五等,謂之小鈔。與錢並行,以七年為限,納舊易新。猶循宋張詠四川交子之法而紓其期爾,蓋亦以銅少,權制之法也。時有欲罷之者,至是二監既罷,有司言:「交鈔舊同見錢,商旅利於致遠,往往
 以錢買鈔,蓋公私俱便之事,豈可罷去!止因有釐革年限,不能無疑,乞削七年釐革之法,令民得常用。若歲久字文磨滅,許於所在官庫納舊換新,或聽便支錢。」遂罷七年釐革之限,交鈔字昏方換。法自此始,而收斂無術,出多入少,民浸輕之。厥後其法屢更,而不能革,弊亦始於此焉。



 交鈔之制,外為闌,作花紋,其上衡書貫例,左曰:「某字料。」右曰:「某字號。」料號外,篆書曰:「偽造交鈔者斬,告捕者賞錢三百貫。」料號衡闌下曰:「中都交鈔庫,准尚書戶部符,承都堂札付,戶部覆點勘,令史姓名押字。」又曰:「聖旨印造逐路交鈔,於某處庫納錢換鈔,更許於某處
 庫納鈔換錢,官私同見錢流轉。」其鈔不限年月行用,如字文故暗,鈔紙擦磨,許於所屬庫司納舊換新。若到庫支錢,或倒換新鈔,每貫剋工墨錢若干文。庫掐、攢司、庫副、副使、使各押字,年月日。印造鈔引庫庫子、庫司、副使各押字,上至尚書戶部官亦押字。其搭印支錢處合同,餘用印依常例。


初,大定間定制,民間應許存留銅鍮器物,若申賣入官,每斤給錢二百文。其
 \gezhu{
  去廾}
 藏應禁器物,首納者每斤給錢百文,非器物銅貨一百五十文,不及斤者計給之。在都官局及外路造賣銅器價,令運司佐貳檢校,鏡每斤三百十四文,鍍金御仙花腰帶十七貫六
 百七十一文,五子荔支腰帶十七貫九百七十一文,抬鈒羅文束帶八貫五百六十文,魚袋二貫三百九文,鈸鈷鐃磬每斤一貫九百二文,鈴杵坐銅者二貫七百六十九文,鍮石者三貫六百四十六文。明昌二年十月,敕減賣鏡價,防私鑄銷錢也。舊嘗以夫匠逾天山北界外採銅,明昌三年,監察御史李炳言:「頃聞有司奏,在官銅數可支十年,若復每歲令夫匠過界遠採,不惟多費,復恐或生邊釁。若支用將盡之日,止可於界內採煉。」上是其言,遂不許出界。



 五月,敕尚書省曰:「民間流轉交鈔,當限其數,毋令多於見錢也。」四年,上諭宰臣曰:「隨處有無
 用官物,可為計置,如鐵錢之類是也。」或有言鐵錢有破損,當令所司以銅錢償之者,參知政事胥持國不可,上曰:「令償之尚壞,不償將盡壞矣!若果無用,曷別為計?」持國曰:「如江南用銅錢,江北、淮南用鐵錢,蓋以隔閡銅錢不令過界爾。如陜西市易亦有用銀布姜麻,若舊有鐵錢,宜姑收貯,以備緩急。」遂令有司籍鐵錢及諸無用之物,貯於庫。



 八月,提刑司言:「所降陜西交鈔多於見錢,使民艱於流轉。」宰臣以聞,遂令本路榷稅及諸名色錢,折交鈔。官兵俸,許錢絹銀鈔各半之,若錢銀數少,即全給交鈔。五年三月,宰臣奏:「民間錢所以艱得,以官豪家多
 積故也。在唐元和間,嘗限富家錢過五千貫者死,王公重貶沒入,以五之一賞告者。」上令參酌定制,令官民之家以品從物力限見錢,多不過二萬貫,猛安謀克則以牛具為差,不得過萬貫,凡有所餘,盡令易諸物收貯之。有能告數外留錢者,奴婢免為良,傭者出離,以十之一為賞,餘皆沒入。又諭旨有司,凡使高麗還者,所得銅器令盡買之。



 承安二年十月,宰臣奏:「舊立交鈔法,凡以舊易新者,每貫取工墨錢十五文。至大定二十三年,不拘貫例,每張收八文,既無益於官,亦妨鈔法,宜從舊制便。若以鈔買鹽引,每貫權作一貫五十文,庶得多售。」上曰:「工
 墨錢,貫可令收十二文。買鹽引者,每貫可權作一貫一百文。」時交鈔所出數多,民間成貫例者艱於流轉,詔以西北二京、遼東路從宜給小鈔,且許於官庫換錢,與它路通行。



 十二月,尚書省議,謂時所給官兵俸及邊戍軍須,皆以銀鈔相兼,舊例銀每鋌五十兩,其直百貫,民間或有截鑿之者,其價亦隨低昂,遂改鑄銀名「承安寶貨」,一兩至十兩分五等,每兩折錢二貫,公私同見錢用,仍定銷鑄及接受稽留罪賞格。



 承安三年正月,省奏:「隨處榷場若許見錢越境,雖非銷毀,即與銷毀無異。」遂立制,以錢與外方人使及與交易者,徒五年,三斤以上死,駔
 儈同罪。捕告人之賞,官先為代給錢五百貫。其逮及與接引、館伴,先排、通引、書表等以次坐罪,仍令均償。時交鈔稍滯,命西京、北京、臨潢,遼東等路一貫以上俱用銀鈔、寶貨,不許用錢,一貫以下聽民便。時既行限錢法,人多不遵,上曰:「已定條約,不為不重,其令御史臺及提刑司察之。」九月,以民間鈔滯,盡以一貫以下交鈔易錢用之,遂復減元限之數,更定官民存留錢法,三分為率,親王、公主、品官許留一分,餘皆半之,其贏餘之數期五十日內盡易諸物,違者以違制論,以錢賞告者。於兩行部各置回易務,以綿絹物段易銀鈔,亦許本務納銀鈔。赴
 榷貨出鹽引,納鈔於山東、河北、河東等路,從便易錢。各降補官及德號空敕三百、度牒一千,從兩行部指定處,限四月進納補換。又更造一百例小鈔,並許官庫易錢。一貫、二貫例並支小鈔,三貫例則支銀一兩、小鈔一貫,若五貫、十貫例則四分支小鈔、六分支銀,欲得寶貨者聽,有阻滯及輒減價者罪之。四年三月,又以銀鈔阻滯,乃權止山東諸路以銀鈔與綿絹鹽引從便易錢之制。令院務諸科名錢,除京師、河南、陜西銀鈔從便,餘路並許收銀鈔各半,仍於鈔四分之一許納其本路。隨路所收交鈔,除本路者不復支發,餘通行者並循環用之。榷
 貨所鬻鹽引,收納寶貨與鈔相半,銀每兩止折鈔兩貫。省許人依舊詣庫納鈔,隨路漕司所收,除額外羨餘者,亦如之。所支官錢,亦以銀鈔相兼,銀已零截者令交鈔庫不復支,若寶貨數少,可浸增鑄。銀鈔既通則物價自平,雖有禁法亦安所施、遂除阻滯銀鈔罪制。四年,以戶部言,命在都官錢、榷貨務鹽引,並聽收寶貨,附近鹽司貼錢數亦許帶納。民間寶貨有所歸,自然通行,不至銷毀。先是,設四庫印小鈔以代鈔本,令人便齎小鈔赴庫換錢,即與支見錢無異。今更不須印造,俟其換盡,可罷四庫。但以大鈔驗錢數支易見錢。時私鑄「承安寶貨」者
 多雜以銅錫,浸不能行,京師閉肆。五年十二月,宰奏臣:「比以軍儲調發,支出交鈔數多。遂鑄寶貨,與錢兼用,以代鈔本,蓋權時之制,非經久之法。」遂罷「承安寶貨」。



 泰和元年六月,通州刺史盧構言:「民間鈔固已流行,獨銀價未平,官之所定每鋌以十萬為準,而市肆纔直八萬,蓋出多入少故也。若令諸稅以錢銀鈔三分均納。庶革其弊。」下省議,宰臣謂『軍興以來,全賴交鈔佐用,以出多遂滯,頃令院務收鈔七分,亦漸流通。若與銀均納,則彼增此減,理必偏勝,至礙鈔法。必欲銀價之平,宜令諸名若『鋪馬』『軍須』等錢,許納銀半,無者聽便。」先是,嘗行三合同
 交鈔,至泰和二年,止行於民間,而官不收斂,朝廷慮其病民,遂令諸稅各帶納一分,雖止係本路者,亦許不限路分通納。戶部見征累年鋪馬錢,亦聽收其半。閏十二月,上以交鈔事,召戶部尚書孫鐸、侍郎張復亨,議於內殿。復亨以三合同鈔可行,鐸請廢不用,既而復亨言竟詘。自是而後,國虛民貧,經用不足,專以交鈔愚百姓,而法又不常,世宗之業衰焉。以至泰和三年,其弊彌甚,乃謂宰臣曰:「大定間,錢至足,今民間錢少,而又不在官,何耶?其集問百官,必有能知之者。」四年七月,罷限錢法,從戶部尚書上官瑜所請也。四年,欲增鑄錢,命百官議所
 以足銅之術。中丞孟鑄謂:「銷錢作銅,及盜用出境者不止,宜罪其官及鄰。」太府監梁絪等言:「鑄錢甚費,率費十錢可得一錢。識者謂費雖多猶增一錢也,乞採銅、拘器以鑄。」宰臣謂:「鼓鑄未可速行,其銅治聽民煎煉,官為買之。凡寺觀不及十人,不許畜法器。民間鍮銅器期以兩月送官給價。匿者以私法坐,限外人告者,以知而不糾坐其官。寺觀許童行告者賞。俟銅多,別具以聞。」八月,定從便易錢法,聽人輸納於京師,而於山東、河北、大名、河東等路依數支取。後鑄大錢一直十,篆文曰「泰和重寶」,與鈔參行。五年,上欲罷交鈔工墨錢,復以印時常費遂
 命貫止收六文。



 六年四月,陜西交鈔不行,以見錢十萬貫為鈔本,與鈔相易,復以小鈔十萬貫相參用之。六年十一月,復許諸路各行小鈔。中都路則於中都及保州,南京路則於南京、歸德、河南府,山東東路則於益都、濟南府,山東西路則於東平、大名府,河北東路則於河間府、冀州,河北西路則於真定、彰德府,河東南路則於平陽,河東北路則於太原、汾州,遼東則於上京、咸平,西京則於西京、撫州,北京則於臨潢府官庫易錢。令戶部印小鈔五等,附各路同見錢用。七年正月,敕在官毋得支出大鈔,在民者令赴庫,以多寡制數易小鈔及見錢,院
 務商稅及諸名錢,三分須納大鈔一分,惟遼東以便。時民以貨幣屢變,往往怨嗟,聚語於市。上知之,諭旨於御史臺曰:「自今都市敢有相聚論鈔法難行者,許人捕告,賞錢三百貫。」五月,以戶部尚書高汝礪議,立「鈔法條約」,添印大小鈔,以鈔庫至急切,增副使一員。汝礪又與中都路轉運使孫鐸言錢幣,上命中丞孟鑄、禮部侍郎喬宇、國子司業劉昂等十人議,月餘不決。七月,上召議于泰和殿,且諭汝礪曰:「今後毋謂鈔多,不加重而輒易之。重之加於錢,可也。」明日,敕:「民間之交易、典質、一貫以上並用交鈔,毋得用錢。須立契者,三分之一用諸物。六盤
 山西、遼河東以五分之一用鈔,東鄙屯田戶以六分之一用鈔。不須立契者,惟遼東錢鈔從便。犯者徒二年,告者賞有差,監臨犯者杖且解職,縣官能奉行流通者升除,否者降罰,集眾沮法者以違制論。工墨錢每張止收二錢。商旅齎見錢不得過十貫。所司籍辨鈔人以防偽冒。品官及民家存留見錢,比舊減其數,若舊有見錢多者,許送官易鈔,十貫以上不得出京。」又定制,按察司以鈔法流通為稱職,而河北按察使斜不出巡按所給券應得鈔一貫,以難支用,命取見錢。御史以沮壞鈔法劾之,上曰:「糾察之官乃先壞法,情不可恕。」杖之七十,削官
 一階,解職。



 戶部尚書高汝礪言:「鈔法務在必行,府州縣鎮宜各籍辨鈔人,給以條印,聽與人辨驗,隨貫量給二錢,貫例雖多,六錢即止。每朝官出使,則令體究通滯以聞。民間舊有宋會子,亦令同見錢用,十貫以上不許持行。榷鹽許用銀絹,餘市易及俸,並用交鈔,其奇數以小鈔足之,應支銀絹而不足者亦以鈔給之。」上遣近侍諭旨尚書省:「今既以按察司鈔法通快為稱職,否則為不稱職,仍於州府司縣官給由內,明書所犯之數,但犯鈔法者雖監察御史舉其能幹,亦不準用。」十月,楊序言:「交鈔料號不明,年月故暗,雖令赴庫易新,然外路無設定
 庫司,欲易無所,遠者直須赴都。」上以問汝礪,對曰:「隨處州府庫內,各有辨鈔庫子,鈔雖弊不偽,亦可收納。去都遠之城邑,既有設置合同換錢,客旅經之皆可相易。更慮無合同之地,難以易者,令官庫凡納昏鈔者受而不支,於鈔背印記官吏姓名,積半歲赴都易新鈔。如此,則昏鈔有所歸而無滯矣!」



 十一月,上諭戶部官曰:「今鈔法雖行,卿等亦宜審察,少有壅滯,即當以聞,勿謂已行而憚改。」汝礪對曰:「今諸處置庫多在公廨內,小民出入頗難,雖有商賈易之,然患鈔本不豐。比者河北西路轉運司言,一富民首其當存留錢外,見錢十四萬貫。它路臆
 或有如此者,臣等謂宜令州縣委官及庫典,於市肆要處置庫支換。以出首之錢為鈔本,十萬戶以上州府,給三萬貫,以次為差,易鈔者人不得過二貫。以所得工墨錢充庫典食直,仍令州府佐貳及轉運司官一員提控。」上是之,遂命移庫於市肆之會,令民以鈔易錢。



 是月,敕捕獲偽造交鈔者,皆以交鈔為賞。



 時復議更鈔法,上從高汝礪言,命在官大鈔更不許出。聽民以五貫十貫例者赴庫易小鈔,欲得錢者五貫內與一緡,十貫內與兩緡,惟遼東從便。河南、陜西、山東及它行鈔諸路,院務諸稅及諸科名錢,並以三分為率,一分納十貫例者,二分
 五貫例者,餘並收見錢。



 八年正月,以京師鈔滯,定所司賞罰格。時新制,按察司及州縣官,例以鈔通滯為升降。遂命監察御史賞罰同外道按察司,大興府警巡院官同外路州縣官。



 是月,收毀大鈔,行小鈔。



 八月,從遼東按察司楊雲翼言,以咸平、東京兩路商旅所集,遂從都南例,一貫以上皆用交鈔,不得用錢。十月,孫鐸又言:「民間鈔多,正宜收斂,院務稅諸名錢,可盡收鈔,秋夏稅納本色外,亦令收鈔,不拘貫例。農民知之則漸重鈔,可以流通。比來州縣抑配市肆買鈔,徒增騷擾,可罷諸處創設鈔局,止令赴省庫換易。今小鈔各限路分,亦甚未便,可令
 通用。」上命亟行之。



 十二月,宰臣奏:「舊制,內外官兵俸皆給鈔,其必用錢以足數者,可以十分為率,軍兵給三分,官員承應人給二分,多不過十貫。凡前所收大鈔,俟至通行當復計造,其終須當精致以圖經久。民間舊鈔故暗者,乞許於所在庫易新。若官吏勢要之家有賤買交鈔,而於院務換錢興販者,以違制論。復遣官分路巡察,其限錢過數雖許奴婢以告,乃有所屬默令其主藏匿不以實首者,可令按察司察之。若舊限已滿,當更展五十日,許再令變易鈔引諸物。」是制既行之後,章宗尋崩,衛紹王繼立,大安三年會河之役,至以八十四車為軍
 賞,兵衄國殘,不遑救弊,交鈔之輕幾於不能市易矣。至宣宗貞祐二年二月,思有以重之,乃更作二十貫至百貫例交鈔,又造二百貫至千貫例者。然自泰和以來,凡更交鈔,初雖重,不數年則輕而不行,至是則愈更而愈滯矣。南遷之後,國蹙民困,軍旅不息,供億無度,輕又甚焉。



 三年四月,河東宣撫使胥鼎上言曰:「今之物重,其弊在於鈔窒,有出而無入也。雖院務稅增收數倍,而所納皆十貫例大鈔,此何益哉?今十貫例者民間甚多,以無所歸,故市易多用見錢,而鈔每貫僅直一錢,曾不及工墨之費。臣愚謂,宜權禁見錢,且令計司以軍須為名,量
 民力征斂,則泉貨流通,而物價平矣。」自是,錢貨不用,富家內困藏鏹之限,外弊交鈔屢變,皆至窘敗,謂之「坐化」。商人往往舟運貿易于江淮,錢多入于宋矣。宋人以為喜,而金人不禁也,識者惜其既不能重無用之楮,而又棄自古流行之寶焉。



 五月,權西安軍節度使烏林達與言:「關陜軍多,供億不足,所仰交鈔則取於京師,徒成煩費,乞降板就造便。」又言:「懷州舊鐵錢鉅萬,今既無用,願貫為甲,以給戰士。」時有司輕罪議罰,率以鐵贖,而當罪不平,遂命贖銅計贓皆以銀價為準。



 六月,敕議交鈔利便。七月,改交鈔名為「貞祐寶券」,仍立沮阻罪。九月,御史
 臺言:「自多故以來,全藉交鈔以助軍需,然所入不及所出,則其價浸減,卒無法以禁,此必然之理也。近用『貞祐寶券』以革其弊,又慮既多而民輕,與舊鈔無異也,乃令民間市易悉從時估,嚴立罪賞,期於必行,遂使商旅不行,四方之物不敢入。夫京師百萬之眾,日費不貲,物價寧不日貴耶?且時估月再定之,而民間價旦暮不一,今有司強之,而市肆盡閉。復議搜括隱匿,必令如估鬻之,則京師之物指日盡,而百姓重困矣。臣等謂,惟官和買計贓之類可用時估,餘宜從便。」制可。



 十二月,上聞近京郡縣多糴於京師,穀價翔踴,令尚書省集戶部、講議所、
 開封府、轉運司,議所以制之者。戶部及講議所言,以五斗出城者可闌糴其半,轉運司謂宜悉禁其出,上從開封府議,謂:「寶券初行時,民甚重之。但以河北、陜西諸路所支既多,人遂輕之。商賈爭收入京,以市金銀,銀價昂,穀亦隨之。若令寶券路各殊制,則不可復入河南,則河南金銀賤而穀自輕。若直閉京城粟不出,則外亦自守,不復入京,穀當益貴。宜諭郡縣小民,毋妄增價,官為定制,務從其便。」



 四年正月,監察御史田迥秀言:「國家調度皆資寶券,行才數月,又復壅滯,非約束不嚴、奉行不謹也。夫錢幣欲流通,必輕重相權、散斂有術而後可。今之
 患在出太多、入太少爾。若隨時裁損所支,而增其所收,庶乎或可也。」因條五事,一曰省冗官吏,二曰損酒使司,三曰節兵俸,四曰罷寄治官,五曰酒稅及納粟補官皆當用寶券。詔酒稅從大定之舊,餘皆不從。尋又更定捕獲偽造寶券官賞。



 三月,翰林侍講學士趙秉文言:「比者寶券滯塞,蓋朝廷將議更張,已而妄傳不用,因之抑遏,漸至廢絕,此乃權歸小民也。自遷汴以來,廢回易務,臣愚謂當復置,令職官通市道者掌之,給銀鈔粟麥縑帛之類,權其低昂而出納之。仍自選良監當官營為之,若半年無過,及券法通流,則聽所指任便差遣。」詔議行之。



 四月,河東行省胥鼎言:「交鈔貴乎流通,今諸路所造不充所出,不以術收之,不無缺誤。宜量民力征斂,以裨軍用。河中宣撫司亦以寶券多出,民不之貴,乞驗民貧富征之。雖為陜西,若一體徵收,則彼中所有日湊于河東,與不斂何異?又河北寶券以不許行于河南,由是愈滯。」宰臣謂:「昨以河北寶券,商旅齎販繼踵南渡,遂致物價翔踴,乃權宜限以路分。今鼎既以本路用度繁殷,欲征軍須錢,宜從所請。若陜西可徵與否,詔令行省議定而後行。」五月,上以河北州府官錢散失,多在民間,命尚書省經畫之。



 八月,平章高琪奏:「軍興以來,用度不貲,惟賴
 寶券,然所入不敷所出,是以浸輕,今千錢之券僅直數錢,隨造隨盡,工物日增,不有以救之,弊將滋甚。宜更造新券,與舊券權為子母而兼行之,庶工物俱省,而用不乏。」濮王守純以下皆憚改,奏曰:「自古軍旅之費皆取於民,向朝廷以小鈔殊輕,權更寶券,而復禁用錢。小民淺慮,謂楮幣易壞,不若錢可久,於是得錢則珍藏,而券則亟用之,惟恐破裂而至於廢也。今朝廷知支而不知收,所以錢日貴而券日輕。然則券之輕非民輕之,國家致之然也。不若量其所支復斂于民,出入循環,則彼知為必用之物,而知愛重矣。今徒患輕而即欲更造,不惟信
 令不行,且恐新券之輕復同舊券也。」既而,隴州防禦使完顏宇及陜西行省令史惠吉繼言券法之弊。宇請姑罷印造,以見在者流通之,若滯塞則驗丁口之多寡、物力之高下而征之。吉言:「券者所以救弊一時,非可通流與見錢比,必欲通之,不過多斂少支爾。然斂多則傷民,支少則用不足,二者皆不可。為今日計,莫若更造,以『貞祐通寶』為名,自百至三千等之為十,聽各路轉運司印造,仍不得過五千貫,與舊券參用,庶乎可也。」詔集百官議。戶部侍郎奧屯阿虎、禮部侍郎楊雲翼、郎中蘭芝、刑部侍郎馮鶚皆主更造。戶部侍郎高夔、員外郎張師魯、
 兵部侍郎徒單歐里白皆請徵斂。惟戶部尚書蕭貢謂止當如舊,而工部尚書李元輔謂二者可並行。太子少保張行信亦言不宜更造,但嚴立不行之罪,足矣。侍御史趙伯成曰:「更造之法,陰奪民利,其弊甚於征。征之為法,特徵於農民則不可,若徵於市肆商賈之家,是亦敦本抑末之一端。」刑部主事王壽寧曰:「不然,今之重錢輕券者皆農爾,其斂必先於民而後可。」轉運使王擴曰:「凡論事當究其本,今歲支軍士家口糧四萬餘石,如使斯人地著,少寬民力,然後征之,則行之不難。」榷貨司楊貞亦欲節無名之費,罷閑冗之官。或有請鑄大錢以當百,
 別造小鈔以省費。或謂縣官當擇人者。獨吏部尚書溫迪罕思敬上書言:「國家立法,莫不備具,但有司不克奉之而已。誠使臣得便宜從事,凡外路四品以下官皆許杖決,三品以上奏聞,仍付監察二人弛驛往來,法不必變,民不必征,一號令之,可使上下無不奉法。如其不然,請就重刑。」上以示宰臣曰:「彼自許如此,試委之可乎?」宰臣未有以處,而監察御史陳規,完顏素蘭交諍,以為:「事有難行,聖哲猶病之,思敬何為者,徒害人爾。」上以眾議紛紛,月餘不決,厭之,乃詔如舊,紓其徵斂之期焉。未幾,竟用惠吉言,造「貞祐通寶」。興定元年二月,始詔行之,凡
 一貫當千貫,增重偽造沮阻罪及捕獲之賞。



 五月,以鈔法屢變,隨出而隨壞,製紙之桑皮故紙皆取于民,至是又甚艱得,遂令計價,但徵寶券、通寶、名曰「桑皮故紙錢」。謂可以免民輸挽之勞,而省工物之費也。高汝礪言:「河南調發繁重,所征租稅三倍於舊,僅可供億,如此其重也。而今年五月省部以歲收通寶不充所用,乃於民間斂桑皮故紙鈔七千萬貫以補之,又太甚矣!而近又以通寶稍滯,又增兩倍。河南人戶農居三之二,今年租稅征尚未足,而復令出此,民若不糶當納之租,則賣所食之粟,舍此將何得焉?今所急而難得者芻糧也,出於民
 而有限。可緩而易為者交鈔也,出於國而可變。以國家之所自行者而強求之民,將若之何?向者大鈔滯則更為小鈔,小鈔弊則改為寶券,寶券不行則易為通寶,變制在我,尚何煩民哉!民既悉力以奉軍而不足,又計口、計稅、計物、計生殖之業而加征,若是其剝,彼不能給,則有亡而已矣!民逃田穢,兵食不給,是軍儲鈔法兩廢矣。臣非於鈔法不加意,非故與省部相違也。但以鈔滯物貴之害輕,民去軍飢之害重爾。」時不能用。



 三年十月,省臣奏:「向以物重錢輕,犯贓者計錢論罪則太重,於是以銀為則,每兩為錢二貫,有犯通寶之贓者直以通寶論,
 如因軍興調發,受通寶及三十貫者,已得死刑,準以金銀價,纔為錢四百有奇,則當杖。輕重之間懸絕如此。」遂命准犯時銀價論罪。四年三月,參知政事李復亨言:「近制,犯通寶之贓者並以物價折銀定罪,每兩為錢二貫,而法當贖銅者,止納通寶見錢,亦乞令依上輸銀,既足以懲惡,又有補於官。」詔省臣議,遂命犯公錯過誤者止征通寶見錢,贓污故犯者輸銀。



 十二月,鎮南軍節度使溫迪罕思敬上書言:「錢之為泉也,貴流通而不可塞,積於官而不散則病民,散於民而不斂則闕用,必多寡輕重與物相權而後可。大定之世,民間錢多而鈔少,故貴
 而易行。軍興以來,在官殊少,民亦無幾,軍旅調度悉仰於鈔,日之所出動以萬計,至於填委市肆,能無輕乎?不若馳限錢之禁,許民自採銅鑄錢,而官製模範,薄惡不如法者令民不得用,則錢必日多,鈔可少出,少出則貴而易行矣。今日出益眾,民日益輕,有司欲重之而不得其法,至乃計官吏之俸、驗百姓之物力以斂之,而卒不能增重,曾不知錢少之弊也。臣謂宜令民鑄錢,而當斂鈔者亦聽輸銀,民因以銀鑄錢為數等,文曰「興定元寶」,定直以備軍賞,亦救弊之一法也。」朝廷不從。



 五年閏十二月,宰臣奏:「向者寶券既弊,乃造『貞祐通寶』以救之,迄
 今五年,其弊又復如寶券之末。初,通寶四貫為銀一兩,今八百餘貫矣。宜復更造『興定寶泉』,子母相權,與通寶兼行,每貫當通寶四百貫,以二貫為銀一兩,隨處置庫,許人以通寶易之。縣官能使民流通者,進官一階、升職一等,其或姑息以致壅滯,則亦追降的決為差。州府官以所屬司縣定罪賞,命監察御史及諸路行部官察之,定撓法失糾舉法,失舉則御史降決,行部官降罰,集眾妄議難行者徒二年,告捕者賞錢三百貫。」元光元年二月,始詔行之。二年五月,更造每貫當通寶五十,又以綾印製「元光珍貨」,同銀鈔及餘鈔行之。行之未久,銀價日
 貴,寶泉日賤,民但以銀論價。至元光二年,寶泉幾於不用,乃定法,銀一兩不得過寶泉三百貫,凡物可直銀三兩以下者不許用銀,以上者三分為率,一分用銀,二分用寶泉及珍貨、重寶。京師及州郡置平準務,以寶泉銀相易,其私易及違法而能告者罪賞有差。是令既下,市肆晝閉,商旅不行,朝廷患之,乃除市易用銀及銀寶泉私相易之法。然上有限用之名,而下無從令之實,有司雖知,莫能制矣。義宗正大間,民間但以銀市易。天興二年十月印「天興寶會」於蔡州,自一錢至四錢四等,同見銀流轉,不數月國亡。



\end{pinyinscope}