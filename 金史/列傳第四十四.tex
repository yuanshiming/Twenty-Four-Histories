\article{列傳第四十四}

\begin{pinyinscope}

 ○張暐張行簡賈益謙劉炳術虎高琪塔不也



 張暐,字明仲,莒州日照縣人。博學該通。登正隆五年進士。調陳留主簿、淄州酒稅副使,課增羨,遷昌樂令。改永清令,補尚書省令史,除太常博士,兼國子助教。丁父憂,服除,調山東東路轉運副使,入為太常丞,兼左贊善大夫。章宗封原王,兼原王府文學。章宗冊為皇太孫,復為
 左贊善,轉左諭德,兼太常丞,充宋國報諭使。至盱眙,宋人請赴宴,暐曰:「大行在殯,未可。」及受賜,不舞蹈,宋人服其知禮。使還,遷太常少卿,兼修起居注。改禮部郎中,修起居注如故。遷右諫議大夫,兼禮部侍郎。



 明昌二年,太傅徒單克寧薨,章宗欲親為燒飯,是時,孝懿皇后梓宮在殯,暐奏:」仰惟聖慈,追念勛臣,恩禮隆厚,孰不感勸。太祖時享,尚且權停,若為大臣燒飯,禮有未安。今已降恩旨,聖意至厚,人皆知之,乞俯從典禮,則兩全矣。」章宗從之。上封事者言提刑司可罷,暐上疏曰:「陛下即位,因民所利,更法立制,無慮數十百條。提刑之設,政之大者,若
 為浮議所搖,則內外無所取信。唐開元中,或請選擇守令,停採訪使,姚崇奏『十道採訪猶未盡得人,天下三百餘州,縣多數倍,安得守令皆稱其職?』然則提刑之任,誠不可罷,擇其人而用之,生民之大利,國家之長策也。」因舉漢刺史六條以奏。上曰:「卿言與朕意合。」



 拜禮部尚書。孫即康鞫治鎬王永中事,還奏,有詔復訊,群臣舉暐及兵部侍郎烏古論慶裔。上使參知政事馬琪諭暐曰:「百官舉閱實鎬王事,要勿屈抑其人,亦不可虧損國法。」上因謂宰臣曰:「鎬王視永蹈為輕。」馬琪曰:「人臣無將。」由是永中之獄決矣。霍王從彞母早死,溫妃石抹氏養之,明昌
 六年溫妃薨,上問從彞喪服。暐奏:「慈母服齊衰三年,桐杖布冠,禮也。從彞近親,至尊壓降與臣下不同,乞於未葬以前服白布衣絹巾,既葬止用素服終制,朝會從吉。」上從其奏。



 承安元年八月壬子,上召暐至內殿,問曰:「南郊大祀,今用度不給,俟他年可乎?」暐曰:「陛下即位于今八年,大禮未舉,宜亟行之。」上曰:「北方未寧,致齋之際,有不測奏報何如?」對曰:「豈可逆度而妨大禮。今河平歲豐,正其時也。」上復問曰:「僧道三年一試,八十而取一,不亦少乎?」對曰:「此輩浮食,無益有損,不宜滋益也。」上曰:「周武帝、唐武宗、後周世宗皆賢君,其壽不永,雖曰偶然,似亦
 有因也。」對曰:「三君矯枉太過。今不毀除、不崇奉,是為得中矣。」是歲,郊見上帝焉。



 頃之,翰林修撰路鐸論胥持國不可再用,因及董師中趨走持國及丞相襄之門,上曰:「張暐父子必不如是也。」三年,為御史大夫,懇辭,不許。明年,坐奏事不實,奪一官,解職。起為安武軍節度使。致仕,例給半俸,久之,暐不復請,遂止。



 暐自妻卒後不復娶,亦無姬侍,齋居與子行簡講論古今,諸孫課誦其側,至夜分乃罷,以為常。歷太常,禮部二十餘年,最明古今禮學,家法為士族儀表。子行簡、行信,行信自有傳。



 行簡字敬甫。穎悟力學,淹貫經史。大定十九年進士第
 一,除應奉翰林文字。丁母憂,歸葬益都,杜門讀書,人莫見其面。服除,復任。章宗即位,轉修撰,進讀陳言文字,攝太常博士。夏國遣使陳慰,欲致祭大行靈殿。行簡曰:「彼陳慰非專祭,不可。」廷議遣使橫賜高麗,「比遣使報哀,彼以細故邀阻,且出嫚言,俟移問還報,橫賜未晚」。徒單克寧韙其言,深器重之。轉翰林修撰,與路伯達俱進讀陳言文字,累遷禮部郎中。



 司天臺劉道用改進新歷,詔學士院更定歷名,行簡奏乞復校測驗,俟將來月食無差,然後賜名。詔翰林侍講學士黨懷英等復校。懷英等校定道用新歷:明昌三年不置閏,即以閏月為三月;二年
 十二月十四日,金木星俱在危十三度,道用歷在十三日,差一日;三年四月十六日夜月食,時刻不同。道用不會考驗古今所記,比登事迹,輒以上進,不可用。道用當徒一年收贖,長行彭徽等四人各杖八十罷去。



 群臣屢請上尊號,章宗不從,將下詔以示四方,行簡奏曰:「往年饑民棄子,或丐以與人,其後詔書官為收贖,或其父母衣食稍充,即識認,官亦斷與之。自此以後,饑歲流離道路,人不肯收養,肆為捐瘠,餓死溝中。伏見近代禦災詔書,皆曰『以後不得復取』今乞依此施行。」上是其言,詔書中行之。久之,兼同修國史。改禮部侍郎、提點司天臺,直
 學士,同修史如故。



 行簡言:「唐制,僕射、宰相上日,百官通班致賀,降階答拜。國朝皇太子元正、生日,三師、三公、宰執以下須群官同班拜賀,皇太子立受再答拜。今尚書省宰執上日,分六品以下別為一班揖賀,宰執坐答揖,左右司郎中五品官廷揖,亦坐答之。臣謂身坐舉手答揖,近於坐受也。宰執受賀,其禮乃重於皇太子,鞏於義未安。別嫌明微,禮之大節,伏請宰執上日令三品以下官同班賀,宰執起立,依見三品官儀式通答揖。」上曰:「此事何不早辨正之,如都省擅行,卿論之是矣。」行簡對曰:「禮部蓋嘗參酌古今典禮,擬定儀式,省廷不從,輒改以
 奏。」下尚書省議,遂用之。宰執上日,三品以下群官通班賀,起立答拜,自此始。



 行簡轉對,因論典故之學,乞於太常博士之下置檢閱官二員,通禮學資淺者使為之,積資乃遷博士。又曰:「今雖有《國朝集禮》,至於食貨、官職、兵刑沿革,未有成書,乞定會要,以示無窮。」承安五年,遷侍講學士,同修史、提點司天如故。



 泰和二年,為宋主生日副使。上召生日使完顏瑭戒之曰:「卿過界勿飲酒,每事聽於行簡。」謂行簡曰:「宋人行禮,好事末節,茍有非是,皆須正之,舊例所有,不可不至。」上復曰:「頗聞前奉使者過淮,每至中流,即以分界爭渡船,此殊非禮。卿自戒舟人,
 且語宋使曰:『兩國和好久矣,不宜爭細故傷大體。』丁寧諭之,使悉此意也。」四年,詔曰:「每奏事之際,須令張行簡常在左右。」



 五年,群臣復請上尊號,上不許,詔行簡作批答,因問行簡宋范祖禹作《唐鑑》論尊號事。行簡對曰:「司馬光亦嘗諫尊號事,不若祖禹之詞深至,以謂臣子生謚君父,頗似慘切。」上曰:「卿用祖禹意答之,仍曰太祖雖有尊號,太宗未嘗受也。」行簡乞不拘對偶,引祖禹以微見其意。從之。其文深雅,甚得代言之體。



 改順天軍節度使。上謂行簡曰:「卿未更治民,今至保州,民之情偽,卒難臆度,如何治之則可?」對曰:「臣奉行法令,不敢違失,獄訟
 之事,以情察之,鈐制公吏,禁抑豪猾,以鎮靜為務,庶幾萬分之一。」上曰:「在任半歲或一年,所得利害上之。」行簡到保州,上書曰:「比者括官田給軍,既一定矣,有告欲別給者,輒從其告,至今未已。名曰官田,實取之民以與之,奪彼與此,徒啟爭端。臣所管已撥深澤縣地三百餘頃,復告水占沙堿者三分之二,若悉從之,何時可定。臣謂當限以月日,不許再告為便。」下尚書省議,奏請:「如實有水占河塌,不可耕種,本路及運司佐官按視,尚書省下按察司復同,然後改撥。若沙堿瘠薄,當準已撥為定。」制曰:「可。」



 六年,召為禮部尚書,兼侍講、同修國史。秘書監進《太
 一新歷》,詔行簡校之。七年,上遣中使馮賢童以實封御扎賜行簡曰:「朕念鎬、鄭二王誤干天常,自貽伊戚。槁葬郊野,多歷年所,朕甚悼焉。欲追復前爵,備禮改葬,卿可詳閱唐貞觀追贈隱、巢,并前代故事,密封以聞。」又曰:「欲使石古乃於威州擇地營葬,歲時祭奠,兼命衛王諸子中立一人為鄭王後,謹其祭祀。此事既行,理須降詔,卿草詔文大意,一就封進。」行簡乃具漢淮南厲王長、楚王英、唐隱太子建成、巢剌王元吉、譙王重福故事為奏,并進詔草,遂施行焉。累遷太子太保、翰林學士承旨,尚書、修史如故。



 貞祐初,轉太子太傅,上書論議和事,其略曰:「
 東海郡候嘗遣約和,較計細故,遷延不決。今都城危急,豈可拒絕。臣願更留聖慮,包荒含垢,以救生靈。或如遼、宋相為敵國,歲奉幣帛,或二三年以繼。選忠實辨捷之人,往與議之,庶幾有成,可以紓患。」是時,百官議者,雖有異同,大概以和親為主焉。莊獻太子葬後,不置宮師官,升承旨為二品,以寵行簡,兼職如故。



 三年七月,朝廷備防秋兵械,令內外職官不以丁憂致仕,皆納弓箭。行簡上書曰:「弓箭非通有之物,其清貧之家及中下監當,丁憂致仕,安有所謂如法軍器。今繩以軍期,補弊修壞,以求應命而已,與倉猝製造何以異哉。若於隨州郡及猛
 安謀克人戶拘括,擇其佳者買之,不足則令職輸所買之價,庶不擾而事可辦。」左丞相僕散端、平章政事高琪、盡忠、右丞賈益謙皆曰:「丁憂致仕者可以免此。」權參政烏古論德升曰:「職官久享爵祿,軍興以來,曾無寸補,況事已行而復改,天下何所取信。」是議也,丁憂致仕官竟得免。是歲,卒,贈銀青榮祿大夫,謚文正。



 行簡端愨慎密,為人主所知。自初入翰林,至太常、禮部,典貢舉終身,縉紳以為榮。與弟行信同居數十年,人無間言。所著文章十五卷,《禮例纂》一百二十卷,會同、朝獻、禘佩、喪葬,皆有記錄,及《清臺》,《皇華》、《戒嚴》、《為善》、《自公》等記,藏于家。



 贊曰:張暐、行簡世為禮官,世習禮學。其為禮也,行於家庭,講於朝廷,施用於鄰國,無不中度。古者官有世掌,學有專門,金諸儒臣,唯張氏父子庶幾無愧於古乎。



 賈益謙,字彥亨,沃州人也,本名守謙,避哀宗諱改焉。大定十年詞賦進士,歷仕州郡,以能稱。明昌間,入為尚書省令史,累遷左司郎中。章宗諭之曰:「汝自知除至居是職,左司事不為不練,凡百官行止、資歷固宜照勘,勿使差繆。若武庫署直長移刺郝自平定州軍事判官召為典輿副轄,在職才五月,降授門山縣簿尉。朕比閱貼黃,行止乃俱書作一十三月,行止尚如此失實,其如選法
 何?蓋是汝不用心致然爾。今姑杖知除掾,汝勿復犯之。」



 五年,為右諫議大夫,上言:「提刑司官不須遣監察體訪,宜據其任內行事,考其能否而升黜之。」上曰:「卿之言其有所見乎?」守謙對曰:「提刑官若不稱職,眾所共知,且其職與監察等,臣是故言之。」上嘉納焉。是年夏,上將幸景明宮清暑,守謙連上疏,極諫之。上御後閣,召守謙入對,稱旨。進兼尚書吏部侍郎。時鎬王以疑忌下獄,上怒甚,朝臣無敢言者。守謙上章論其不可,言極懇切。上諭之曰:「汝言諸王皆有覬心,而游其門者不無橫議。此何等語,固當罪汝。以汝前言事亦有當處,故免。」既而以議鎬
 王事有違上意,解職,削官二階。承安元年七月,降為寧化州刺史。五年八月,改為山東路按察使,轉河北西路轉運使。泰和三年四月,召為御史中丞。四年三月,出為定武軍節度使。



 八年六月,復為御史中丞。八月,改吏部尚書。九月,詔守謙等一十三員分詣諸路,與本路按察司官一員同推排民戶物力。上召見於香閣,諭之曰:「朕選卿等隨路推排,除推收外,其新強、銷乏戶,雖集眾推唱,然銷乏者勿銷不盡,如一戶元物力三百貫,今蠲減二百五十貫,猶有不能當。新強者勿添盡,量存氣力,如一戶添三百貫而止添二百貫之類。卿等宜各用心。百
 姓應當賦役,十年之間,利害非細。茍不稱所委,治罪當不輕也。」尋出知濟南府,移鎮河中。大安末,拜參知政事。貞祐二年二月,改河東南路安撫使,俄知彰德府。



 三年,召為尚書省右丞。會宣宗始遷汴梁,益謙乃建言:「汴之形勢,惟恃大河。今河朔受兵,群盜並起,宜嚴河禁以備不虞,凡自北來而無公憑者,勿聽渡。」是時,河北民遷避河南者甚眾。侍御史劉無規上言:「僑戶宜與土民均應差役。」上留中,而自以其意問宰臣。丞相端、平章盡忠以為便。益謙曰:「僑戶應役,甚非計也。蓋河北人戶本避兵而來,兵稍息即歸矣。今旅寓倉皇之際,無以為生,若又
 與地著者並應供憶,必騷動不能安居矣。豈主上矜恤流亡之意乎。」上甚嘉賞,曰:「此非朕意也。」因出元規章示之。三年八月,進拜尚書左丞。四年正月,致仕,居鄭州。



 興定五年正月,尚書省奏:「《章宗實錄》已進呈,衛王事迹亦宜依《海陵庶人實錄》,纂集成書,以示後世。」制可。初,胡沙虎弒衛王,立宣宗,一時朝臣皆謂衛王失道,天命絕之,虎實無罪,且有推戴之功,獨張行信抗章言之,不報,舉朝遂以為諱。及是,史官謂益謙嘗事衛王,宜知其事,乃遣編修一人就鄭訪之。益謙知其旨,謂之曰:「知衛王莫如我。然我聞海陵被弒而世宗立,大定三十年,禁近
 能暴海陵蟄惡者,輒得美仕,故當時史官修實錄多所附會。衛王為人勤儉,慎惜名器,較其行事,中材不及者多矣。吾知此而已,設欲飾吾言以實其罪,吾亦何惜餘年。」朝議偉之。正大三年,年八十,薨。三子:賢卿、頤卿、翔卿,皆以門資入仕。



 贊曰:賈益謙於衛紹王,可謂盡事君之義矣。海陵之事,君子不無憾焉。夫正隆之為惡,暴其大者斯亦足矣。中綍之醜,史不絕書,誠如益謙所言,則史亦可為取富貴之道乎?嘻,其甚矣。《傳》曰:「不有廢者,其何以興!」



 劉炳,葛城人。每讀書,見前古忠臣烈士為國家畫策慮
 萬世安,輒歎息景慕。貞祐三年,中進士第,即日上書條便宜十事:



 其一曰,任諸王以鎮社稷。臣觀往歲,王師屢戰屢衄,率皆自敗。承平日久,人不知兵,將帥非才,既無靖難之謀,又無效死之節,外託持重之名,而內為自安之計,擇驍果以自隨,委疲懦以臨陣,陣勢稍動,望塵先奔,士卒從而大潰。朝廷不加詰問,輒為益兵。是以法度日紊,倉庾日虛,閭井日凋,土地日蹙。自大駕南巡,遠近相望,益無固志。吏任河北者以為不幸,逡巡退避,莫之敢前。昔唐天寶之末,洛陽、潼關相次失守,皇輿夜出,向非太子回趨靈武,率先諸將,則西行之士當終老於劍
 南矣。臣願陛下擇諸王之英明者,總監天下之兵,北駐重鎮,移檄遠近,戒以軍政。則四方聞風者皆將自奮,前死不避。折衝厭難,無大於此。夫人情可以氣激不可以力使,一卒先登,則萬夫齊奮,此古人所以先身教而後威令也。



 二曰,結人心以固基本。天子惠人,不在施予,在于除其同患,因所利而利之。今艱危之後,易於為惠,因其欲安而慰撫之,則忠誠親上之心,當益加於前日。臣願寬其賦役,信其號令,凡事不便者一切停罷。時遣重臣按行郡縣,延見耆老,問其疾苦,選廉正,黜貪殘,拯貧窮,恤孤獨,勞來還定,則效忠徇義,無有二志矣。故曰安
 民可與行義,危民易與為亂,惟陛下留神。



 三曰,廣收人材以備國用。備歲寒者必求貂狐,適長途者必畜騏驥。河南、陜西,車駕臨幸,當有以大慰士民之心。其有操行為民望者,稍擢用之,平居可以勵風俗,緩急可以備驅策。昭示新恩,易民觀聽,陰係天下之心也。



 四曰,選守令以安百姓。郡守、縣令,天子所恃以為治,百姓所依以為命者也。今眾庶已弊,官吏庸暗,無安利之才,貪暴昏亂,與姦為市,公有斗粟之賦,私有萬錢之求,遠近囂囂,無所控告。自今非才器過人,政跡卓異者,不可使在此職。親勳故舊,雖望隆資高,不可使為長吏。則賢者喜於殊
 用,益盡其能,不肖者愧慕而思自勵矣。



 五曰,褒忠義以勵臣節。忠義之士,奮身效命,力盡城破而不少屈。事定之後,有司略不加省,棄職者顧以恩貸,死事者反不見錄,天下何所慕憚,而不為自安之計邪?使為臣者皆知殺身之無益,臨難可以茍免,甚非國家之利也。



 六曰,務農力本以廣蓄積。此最強兵富民之要術,當今之急務也。



 七曰,崇節儉以省財用。今海內虛耗,田疇荒蕪,廢奢從儉以紓生民之急,無先於此者。



 八曰,去冗食以助軍費。兵革之後,人物凋喪者十四五,郡縣官吏署置如故,甚非審權救弊之道。



 九曰,修軍政以習守戰。自古名將
 料敵制勝,訓練士卒,故可使赴湯蹈火,百戰不殆。孔子曰:「以不教民戰,是謂棄之。」兵法曰:「器械不利,以其卒與敵也。卒不服習,以其將與敵也。將不知兵,以其主與敵也。主不擇將,以其國與敵也。」可不慎哉。



 十曰,修城池以備守禦。保障國家,惟都城與附近數郡耳。北地不守,是無河朔矣,黃河豈足恃哉。



 書奏,宣宗異焉。復試之曰:「河北城邑,何術可保?兵民雜居,何道可和?鈔法如何而通?物價如何而平?」炳對大略以審擇守將則城邑固,兵不侵民則兵民和,斂散相權則鈔法通,勸農薄賦則物價平。宣宗雖異其言,而不能用,但補御史臺令史而已。



 論曰:劉炳可謂能言之士矣。宣宗召試既不失對,而以一臺令史賞之,足以倡士氣乎?



 術虎高琪,或作高乞,西北路猛安人。大定二十七年充護衛,轉十人長,出職河間都總管判官,召為武衛軍鈐轄,遷宿直將軍,除建州刺史,改同知監洮府事。泰和六年,伐宋,與彰化軍節度副使把回海備鞏州諸鎮,宋兵萬餘自鞏州轆轤嶺入,高琪奮擊破之,賜銀百兩、重彩十端。青宜可內附,詔知府事石抹仲溫與高琪俱出界,與青宜可合兵進取。詔高琪曰:「汝年尚少,近聞與宋人力戰奮勇,朕甚嘉之。今與仲溫同行出界,如其成功,高
 爵厚祿,朕不吝也。」



 詔封吳曦為蜀國王,高琪為封冊使。詔戒諭曰:「卿讀書解事,蜀人亦識威名,勿以財賄動心,失大國體。如或隨去奉職有違禮生事,卿與喬宇體察以聞。」使還,加都統,號平南虎威將軍。



 宋安丙遣李孝義率步騎三萬攻秦州,先以萬人圍皁角堡,高琪赴之。宋兵列陣山谷,以武車為左右翼,伏弩其下來逆戰。既合,宋兵陽卻。高琪軍見宋兵伏不得前,退整陣,宋兵復來。凡五戰,宋兵益堅,不可以得志。高琪分騎為二,出者戰則止者俟,止者出則戰者還,還者復出以更。久之,遣蒲察桃思剌潛兵上山,自山馳下合擊,大破宋兵,斬首四
 千級,生擒數百人,李孝義乃解圍去。宋兵三千致馬連寨以窺湫池,遣夾谷福壽擊走之,斬七百餘級。



 大安三年,累官泰州刺史,以颭軍三千屯通玄門外。未幾,升縉山縣為鎮州,以高琪為防禦使,權元帥右都監,所部颭軍賞賚有差。至寧元年八月,尚書左丞完顏綱將兵十萬行省於縉山,敗績。貞祐初,遷元帥右監軍。閏月,詔高琪曰:「聞軍事皆中覆,得無失機會乎?自今當即行之,朕但責成功耳。」



 是月,被詔自鎮州移軍守禦中都迤南,次良鄉不得前,乃還中都。每出戰輒敗,紇石烈執中戒之曰:「汝連敗矣,若再不勝,當以軍法從事。」及出,果敗,高琪
 懼誅。十月辛亥,高琪自軍中入,遂以兵圍執中第,殺執中,持其首詣闕待罪。宣宗赦之,以為左副元帥,一行將士遷賞有差。丙寅,詔曰:「胡沙虎畜無君之心,形迹露見,不可盡言。武衛副使提點近侍局慶山奴、近侍局使斜烈、直長撒合輦累曾陳奏,方慎圖之。斜烈漏此意於按察判官胡魯,胡魯以告翰林待制訛出,訛出達於高琪,今月十五日將胡沙虎戮訖。惟茲臣庶將恐有疑,肆降札書,不匿厥旨。」論者謂高琪專殺,故降此詔。頃之,拜平章政事。



 宣宗論馬政,顧高琪曰:「往歲市馬西夏,今肯市否?」對曰:「木波畜馬甚多,市之可得,括緣邊部落馬,亦不
 少矣。」宣宗曰:「盡括邊馬,緩急如之何?」閱三日,復奏曰:「河南鎮防二十餘軍,計可得精騎二萬,緩急亦足用。」宣宗曰:「馬雖多,養之有法,習之有時,詳諭所司令加意也。」貞祐二年十一月,宣宗問高琪曰:「所造軍器往往不可用,此誰之罪也?」對曰:「軍器美惡在兵部,材物則戶部,工匠則工部。」宣宗曰:「治之!且將敗事。」宣宗問楊安兒事,高琪對曰:「賊方據險,臣令主將以石墻圍之,勢不得出,擒在旦夕矣。」宣宗曰:「可以急攻,或力戰突圍,我師必有傷者。」



 應奉翰林文字完顏素蘭自中都議軍事還,上書求見,乞屏左右。故事,有奏密事輒屏左右。先是,太府監丞游
 茂以高琪威權太重,中外畏之,常以為憂,因入見,屏人密奏,請裁抑之。宣宗曰:「既委任之,權安得不重?」茂退不自安,復欲結高琪,詣其第上書曰:「宰相自有體,豈可以此生人主之疑,招天下之議。」恐高琪不相信,復曰:「茂嘗間見主上,實惡相公權重。相公若能用茂,當使上不疑,而下無所議。」高琪聞茂嘗請間屏人奏事,疑之,乃具以聞。游茂論死,詔免死,杖一百,除名。自是凡屏人奏事,必令近臣一人侍立。及素蘭請密,召至近侍局,給筆札,使書所欲言。少頃,宣宗御便殿見之,惟留近侍局直長趙和和侍立。素蘭奏曰:「日者元帥府議削伯德文哥兵權,
 朝廷乃詔領義軍。改除之命拒而不受,元帥府方欲討捕,朝廷復赦之,且不令隸元帥府。不知誰為陛下畫此計者,臣自外風聞皆出平章高琪。」宣宗曰:「汝何以知此事出於高琪?」素蘭曰:「臣見文哥與永清副提控劉溫牒云,差人張希韓至自南京,道副樞平章處分,已奏令文哥隸大名行省,毋遵中都帥府約束。溫即具言於帥府。然則文哥與高琪計結,明矣。」上頷之。素蘭復奏曰:「高琪本無勳望,嚮以畏死擅殺胡沙虎,計出於無聊耳。妒賢能,樹黨與,竊弄威權,自作威福。去歲,都下書生樊知一詣高琪,言颭軍不可信,恐生亂。高琪以刀杖決殺之,自
 是無復敢言軍國利害者。使其黨移剌塔不也為武寧軍節度使,招颭軍,已而無功,復以為武衛軍使。以臣觀之,此賊滅亂紀綱,戕害忠良,實有不欲國家平治之意。惟陛下斷然行之,社稷之福也。」宣宗曰:「朕徐思之。」素蘭出,復戒曰:「慎無泄也。」



 四年十月,大元大兵取潼關,次嵩、汝間,待闕臺院令史高嶷上書曰:「向者河朔敗績,朝廷不時出應,此失機會一也。及深入吾境,都城精兵無慮數十萬,若效命一戰,必無今日之憂,此失機會二也。既退之後,不議追襲,此失機會三也。今已度關,不亟進禦,患益深矣。乞命平章政事高琪為帥,以厭眾心。」不報。御
 史臺言:「兵逾潼關、崤、澠,深入重地,近抵西郊。彼知京師屯宿重兵,不復叩城索戰,但以遊騎遮絕道路,而別兵攻擊州縣,是亦困京師之漸也。若專以城守為事,中都之危又將見於今日,況公私蓄積視中都百不及一,此臣等所為寒心也。不攻京城而縱其別攻州縣,是猶火在腹心,撥置於手足之上,均一身也,願陛下察之。請以陜西兵扼拒潼關,與右副元帥蒲察阿里不孫為掎角之勢,選在京勇敢之將十數人,各付精兵數千,隨宜伺察,且戰且守,復諭河北,亦以此待之。」詔付尚書省,高琪奏曰:「臺官素不習兵,備禦方略,非所知也。」遂寢。高琪止
 欲以重兵屯駐南京以自固,州郡殘破不復恤也。宣宗惑之,計行言聽,終以自斃。



 未幾,進拜尚書右丞相,奏曰:「凡監察有失糾彈者從本法。若人使入國,私通言語,說知本國事情,宿衛、近侍官、承應人出入親王、公主、宰執之家,災傷闕食,體究不實,致傷人命,轉運軍儲,而有私載,及考試舉人關防不嚴者,並的杖。在京犯至兩次者,臺官減監察一等論贖,餘止坐專差者。任滿日議定升降。若任內有漏察之事應的決者,依格雖為稱職,止從平常,平常者從降罰。」制可。高琪請修南京裏城,宣宗曰:「此役一興,民滋病矣。城雖完固,能獨安乎?」



 初,陳言人王
 世安獻攻取盱眙、楚州策,樞密院奏乞以世安為招撫使,選謀勇二三人同往淮南,招紅襖賊及淮南宋官。宣宗可其奏,詔泗州元帥府遣人同往。興定元年正月癸未,宋賀正旦使朝辭,宣宗曰:「聞息州透漏宋人,此乃彼界饑民沿淮為亂,宋人何敢犯我?」高琪請伐之以廣疆土。上曰:「朕但能守祖宗所付足矣,安事外討。」高琪謝曰:「今雨雪應期,皆聖德所致。而能包容小國,天下幸甚,臣言過矣。」四月,遣元帥左都監烏古論慶壽、簽樞密院事完顏賽不經略南邊,尋復下詔罷兵,然自是與宋絕矣。



 興定元年十月,右司諫許古勸宣宗與宋議和,宣宗命
 古草牒,以示宰臣,高琪曰:「辭有哀祈之意,自示微弱不足取。遂寢。集賢院諮議官呂鑑言:「南邊屯兵數十萬,自唐、鄧至壽、泗沿邊居民逃亡殆盡,兵士亦多亡者,亦以人煙絕少故也。臣嘗比監息州榷場,每場所獲布帛數千匹、銀數百兩,大計布帛數萬匹,銀數千兩,兵興以來俱失之矣。夫軍民有逃亡之病,而國家失日獲之利,非計也。今隆冬冱寒,吾騎得騁,當重兵屯境上,馳書諭之,誠為大便。若俟春和,則利在於彼,難與議矣。昔燕人獲趙王,趙遣辯士說之,不許,一牧豎請行,趙王乃還。孔子失馬,馭卒得之。人無貴賤,茍中事機,皆可以成功。臣雖
 不肖,願效牧豎馭卒之智,伏望宸斷。」詔問尚書省。高琪曰:「鑒狂妄無稽,但其氣岸可尚,宜付陜西行省備任使。」制可。十二月,胥鼎諫伐宋,語在鼎傳。高琪曰:「大軍已進,無復可議。」遂寢。



 二年,胥鼎上書諫曰:「錢穀之冗,非九重所能兼,天子總大綱,責成功而已。」高琪曰:「陛下法上天行健之義,憂勤庶務,夙夜不遑,乃太平之階也。鼎言非是。」宣宗以南北用兵,深以為憂,右司諫呂造上章:「乞詔內外百官各上封事,直言無諱。或時召見,親為訪問。陛下博採兼聽,以盡群下之情,天下幸甚。」宣宗嘉納,詔集百官議河北、陜西守禦之策。高琪心忌之,不用一言。是
 時,築汴京城裏城,宣宗問高琪曰:「人言此役恐不能就,如何?」高琪曰:「終當告成,但其濠未及浚耳。」宣宗曰:「無濠可乎?」高琪曰:「茍防城有法,正使兵來,臣等愈得效力。」宣宗曰:「與其臨城,曷若不令至此為善。」高琪無以對。



 高琪自為宰相,專固權寵,擅作威福,與高汝礪相唱和。高琪主機務,高汝礪掌利權,附己者用,不附己者斥。凡言事忤意,及負材力或與己頡頑者,對宣宗陽稱其才,使乾當於河北,陰置之死地。自不兼樞密元帥之後,常欲得兵權,遂力勸宣宗伐宋。置河北不復為意,凡精兵皆置河南,茍且歲月,不肯輒出一卒,以應方面之急。平章政
 事英王守純欲發其罪,密召右司員外郎王阿里、知案蒲鮮石魯剌、令史蒲察胡魯謀之。石魯剌、胡魯以告尚書省都事僕散奴失不,僕散奴失不以告高琪。英王懼高琪黨與,遂不敢發。頃之,高琪使奴賽不殺其妻,乃歸罪於賽不,送開封府殺之以滅口。開封府畏高琪,不敢發其實,賽不論死。事覺,宣宗久聞高琪姦惡,遂因此事誅之,時興定三年十二月也。尚書省都事僕散奴失不以英王謀告高琪,論死。蒲鮮石魯剌、蒲察胡魯各杖七十,勒停。



 初,宣宗將遷南,欲置颭軍於平州,高琪難之。及遷汴,戒彖多厚撫此軍,彖多輒殺颭軍數人,以至于敗。
 宣宗末年嘗曰:「壞天下者,高琪、彖多也。」終身以為恨云。



 移剌塔不也,東北路猛安人。明昌元年,累官西上閣門使。二年,襲父謀克。泰和伐宋,有功,遙授同知慶州事,權迪列颭詳穩。丁父憂,起復西北路招討判官,改尚輦局使、曹王傅。貞祐二年,遷武寧軍節度使,招徠中都颭軍,無功,平章高琪芘之,召為武衛軍都指揮使。應奉翰林文字完顏素蘭嘗面奏高琪黨比,語在《高琪傳》。尋知河南府事,兼副統軍,徙彰化軍節度使。上言:「盡籍山東、河間、大名猛安人為兵,老弱城守,壯者捍禦。」又言:「河東地險人勇,步兵為天下冠,可盡調以戍諸隘。」從之。自是河
 東郡縣屯兵少,不可守矣。改知臨洮府事,兼陜西副統軍。貞祐三年十一月,破夏兵于熟羊寨。平章高琪率宰臣入賀曰:「塔不也以少敗眾,蓋陛下威德所致。」宣宗曰:「自古興國皆賴忠賢,今茲立功,皆將率諸賢之力也。」乃以塔不也為勸農使,兼知平涼府事,進階銀青榮祿大夫。四年,伐西夏,攻威、靈、安、會等州。興定元年,知慶陽府事。三年,遷元帥左都監,卒。



 論曰:高琪擅殺執中,宣宗不能正其罪,又曲為之說,以詔臣下。就其事論之,人君欲誅大臣,而與近侍密謀于宮中,已非其道。謀之不密,又為外臣所知,以告敗軍之
 將,因殺之以為說,此可欺後世邪?金至南渡,譬之尪羸病人,元氣無幾。琪喜吏而惡儒,好兵而厭靜,沮遷颭之議,破和宋之謀,正猶繆醫,投以烏喙、附子,只速其亡耳。使宣宗於擅殺之日,即能伸大義而誅之,何至誤國如是邪。



\end{pinyinscope}