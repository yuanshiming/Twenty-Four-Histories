\article{志第二十一}

\begin{pinyinscope}

 樂下



 ○宗廟樂歌殿庭樂歌鼓吹導引曲採茨曲



 禘祫親饗,皇帝入門。宮縣無射宮《昌寧之曲》:出、入步武同。



 惟時升平,禮儀肇興。鳴鑾至止,穆穆造庭。百辟卿士,恪謹迎承。恭款祖考,神宇攸寧。



 皇帝升殿,登歌夾鐘宮《昌寧之曲》:升階及將還板位,皆同登歌。



 笙鏞既陳,罍樽在戶。升降有容,惟規惟矩。恭敬明神,上儀交舉。永言保之,承天之祜。



 皇帝盥洗,宮縣無射宮《昌寧之曲》:



 惟水之功,潔凈精微。洗爵奠斝,于德有輝。皇皇穆穆,宗廟之威。宜其感格,福祉交歸。



 皇帝降階,宮縣無射宮《昌寧之曲》:



 於皇神宮,象天清明。有來肅肅,相維公卿。禮儀卒度,君子攸寧。孔時孔惠,綏我思成。



 迎神,宮縣《來寧之曲》。黃鐘宮三奏,大呂角二奏,大蔟徵二奏,應鐘羽二奏,詞同:



 八音克諧,百禮具舉。明德維清,至誠永慕。神之格思,雲軿風馭。來止來臨,千祀燕處。



 司徒引俎,宮縣無射宮《豐寧之曲》:



 維牲維犧,齊明致祠。我將我享,吉蠲奉之。博碩肥腯,神嗜為宜。千秋歆此,永綏黔黎。



 始祖酌獻,宮縣無射宮《大元之曲》:



 惟酒既清,惟肴既馨。苾芬孝祀,在廟之庭。羞於皇祖,來燕來寧。象功昭德,先祖是聽。



 德皇帝,《大熙之曲》:



 萬方欣戴,鴻業創基。瑤源垂裕,綿瓞重熙。式崇毖祀,爰考成規。籩豆有楚,益臻皇儀。



 安皇帝,《大安之曲》:



 爰圖造邦,載德其昌。皇儀允穆,誕集嘉祥。明誠昭格,積厚流光。祇嚴清廟,鐘石瑯瑯。



 獻祖,《大昭之曲》:



 惟聖興邦,經始之初。鳩民化俗,還定攸居。迪德純儉,志規遠圖。時哉顯祀,精誠有孚。



 昭祖,《大成之曲》:



 天啟璇源,貽慶定基。率義為勇,施德為威。耀武拓境,功烈巍巍。永昌皇祚,均福黔黎。



 景祖,《大昌之曲》:



 丕顯鴻烈,基緒降昌。聖期誕集,邦宇斯張。尊嚴廟祏,
 昭格休祥。煌煌縟典,億載彌光。



 世祖,《大武之曲》:



 桓桓伐功,天監其明。惟威震疊,惟德綏寧。神策無遺,鴻圖以興。會孫孝祀,遹昭厥成。



 肅宗,《大明之曲》:



 於皇神人,武烈文謨。左右世祖,懷柔掃除。威震遐邇,化漸蟲魚。垂光綿永,成帝之孚。



 穆宗,《大章之曲》:



 烝哉文祖,欽聖弘淵。慈愛忠信,典策昭然。歆此明祀,繁祉綿綿。時純熙矣,流慶萬年。



 康宗,《大康之曲》:



 惟明惟聽,曄曄神功。儀刑世業,昭格上穹。持盈孝孫,薦芳斯豐。錫我祉福,皇化益隆。



 太祖,《大定之曲》:



 功超殷周,德配唐虞。天人協應,平統寰區。開祥垂裕,肇基永圖。明明天子,敬承典謨。



 太宗,《大惠之曲》:



 巍巍德鴻,無為端扆。祚承神功,究馴俗美。清宮緝熙,孝毖時祀。欽奠羞誠,犧樽嘉旨。



 熙宗《大同之曲》:



 昭顯令德,神基丕承。對越在天,享用躋升。於穆清廟,來燕來寧。神其醉止,惟欽克誠。



 睿宗,《大和之曲》:



 皇祖開基,周武殷湯。猗歟聖考,嗣德彌光。啟祐洪緒,長發其祥。嚴恭廟享,萬世烝嘗。



 世宗,《大鈞之曲》:



 神之來思,甫登于堂。稞圭有瓚,秬鬯芬芳。巍巍先功,啟祐無疆。萬年肆祀,孝心不忘。



 顯宗,《大寧之曲》:



 於皇神宮,有嚴惟清。吉蠲孝祀,惟神之寧。對越在天,
 綏我思誠。敷祐億年,邦家之慶。



 章宗,《大隆之典》:



 兩紀踐阼,萬方寧康。文經天地,武服遐荒。禮備制定,德隆業昌。居歆典祀,億載無疆。



 宣宗,《大慶之曲》:



 猗歟聖皇,三代之英。功光先后,德被群生。牲粢惟馨,鼓鐘其鏗。神兮來思,歆于克誠。



 文舞退,武舞進,宮縣無射宮《肅寧之曲》:



 明明先皇,神武維揚。開基垂統,萬世無疆。干戚象功,威儀有光。神保是饗,昭哉降康。



 亞終獻,無射宮《肅寧之曲》:



 涓辰之休,昭祀惟恭。威儀陟降,惟禮是從。籩豆靜嘉,於論鼓鐘。惟皇受祉,監斯德容。



 皇帝飲福,登歌夾鐘宮《福寧之曲》:



 犧牲充潔,粢盛馨香。來格來享,精神用彰。飲此純禧,簡簡穰穰。文明天子,萬壽無疆。



 徹豆,登歌夾鐘宮《豐寧之曲》:



 孝祀肅睦,明德以薦。樂奏九成,禮終三獻。百辟卿士,進徹以時。小大稽首,神保聿歸。



 送神,宮縣黃鐘宮《來寧之曲》:



 潔茲牛羊,清茲酒醴。三獻攸終,神既燕喜。神之去兮,載錫繁祉。萬壽無疆,永保禋祀。



 郊祀前,朝享太廟樂歌。



 皇帝入門,宮縣無射宮《昌寧之曲》:



 郊將升禋,廟當告虔。錫鑾戾止,孝實奉先。祀事斯舉,有序無愆。祗見祖考,神意歡然。



 皇帝升殿,登歌夾鐘宮《昌寧之曲》:



 皇皇天子,升自阼階。奠見祖禰,肅然有懷。百禮已洽,八音克諧。既昌且寧,萬福沓來。



 迎神,宮縣《來寧之曲》。黃鐘宮三奏,大呂角二奏,太蔟徵二
 奏,應鐘羽二奏,詞同:



 以實應天,報本反始。潔粢豐盛,禮先肆祀。風馬雲車,神之弔矣。來止來宜,而燕翼子。



 皇帝盥洗,宮縣無射宮《昌寧之曲》:



 有水于罍,有巾於篚。帨手拭爵,圭袞有煒。玄酒大羹,德磬維菲。萬年昌寧,皇皇負扆。



 皇帝升階,宮縣無射宮《昌寧之曲》:降階,同。



 巍巍京師,有嚴神宮。聖主戾止,多士雲從。來享來獻,肅肅其容。將昭大報,庸示推崇。



 司徒奉俎,宮縣無射宮《豐寧之曲》:



 陳其犧牲,惟純與精。苾芬孝祀,於昭克誠。不疾瘯蠡,或剝或亨。洋洋在上,以交神明。



 始祖酌獻,宮縣《大元之曲》:



 猗歟初基,兆我王迹。其命維新,貽謀丕赫。綿綿瓜瓞,國步日闢。堂構之成,焜煌今昔。



 獻祖,《大昭之曲》:



 以聖繼興,成王之孚。民從其化,咸奠攸居。清廟觀德,猗歟偉歟。金石備樂,以奉神娛。



 昭祖,《大成之曲》:



 東夷不庭,皇祖震怒。神武削平,貽厥聖緒。猶室有基,
 垣墉乃樹。億萬斯年,天保孔固。



 景祖,《大昌之曲》:



 於皇藝祖,其智如神。修法施令,百度惟新。疆宇日廣,海隅咸賓。功高德厚,耀耀震震。



 世祖,《大武之曲》:



 於皇先王,昭假于天。長駕遠馭,麾斥無前。王業猶生,孫謀有傳。圓壇展禮,敢先告虔。



 肅宗,《大明之曲》:



 猗歟前人,簡惠昭融。相我世祖,成茲伐功。敷佑來葉,帝圖其隆。將修熙事,先款神宮。



 穆宗,《大章之曲》:



 仁慈忠信,惟祖之休。功光岐下,迹掩商丘。言瞻清廟,懷想前修。神其來格,歆茲庶羞。



 康宗,《大康之曲》:



 猗歟前王,惠我無疆。儀刑典法,日靖四方。永言孝思,於乎不忘。昭告大祀,祗率舊章。



 太祖,《大定之曲》:



 天生聰明,俾乂蒸人。惟此二國,為我驅民。撻彼威武,萬邦咸賓。明昭大報,推而配神。



 太宗,《大惠之曲》:



 維清緝熙,於昭明德。我其收之,駿奔萬國。南郊肇修,大典增飾。清廟吉蠲,純禧申錫。



 睿宗,《大和之曲》:



 維時祖功,肇開神基。昭哉聖考,其德增輝。上動天監,明命攸歸。謀貽翼子,無疆之辭。



 文舞退,武舞進,宮縣《肅寧之曲》:



 先皇開基,比迹殷湯,功加天下,武德彌光。容舞象成,干戈戚揚。於昭報本,懷哉不忘。



 亞終獻,宮縣《肅寧之曲》:



 於皇宗祊,朝獻維時。芬芬酒醴,棣棣威儀。誠則有餘,
 神之格思。神孫千億,神其相之。



 皇帝飲福,登歌夾鐘宮《福寧之曲》:



 皇皇穆穆,丕承丕基。躬親于禋,載肅載祗。對越在天,神歆其誠。于以飲酒,如川之增。



 徹豆,登歌夾鐘宮《豐寧之曲》:



 物維其時,既豐且旨。苾苾德馨,或將或肆。神之居歆,洽于百禮。於萬斯年,穰穰介祉。



 送神,宮縣黃鐘宮《來寧之曲》:



 濟濟多儀,皇皇雅奏。獻終反爵,薦餘徹豆。神監昭回,有秩斯祐。無疆之福,申錫厥後。



 昭德皇后別廟,郊祀前薦享,登歌樂曲。



 初獻盥洗,夷則宮《肅寧之曲》:



 神無常享,時歆精誠。惟誠惟潔,感通神明。先事盥滌,注茲清冷。巾篚既奠,尊彞薦馨。



 初獻升、降殿,中呂宮《嘉寧之曲》:



 有來肅肅,登降以敬。粲粲袨服,鏘鏘佩聲。金石節奏,既協且平。其儀不忒,乃終有慶。



 司徒奉俎,奏夷則宮《豐寧之曲》:



 馨我黍稷,潔我牲牷。降升有節,薦是吉蠲。工祝致告,威儀肅然。神之弔矣,元吉其旋。



 酌獻,奏夷則宮《儀坤之曲》:



 伣天之妹,坤德利貞。圓丘有事,先薦以誠。我酒既旨,我肴既盈。神其居饗,福祿來成。



 徹豆,奏中呂宮《豐寧之曲》:



 明昭祀事,舊典無違。樂既云闋,神其聿歸。禮之克成,神保斯饗。於萬斯年,迓續丕貺。



 祫禘有司攝事。



 初獻盥洗,宮縣無射宮《肅寧之曲》:



 祀事之大,齊栗為先。潔精以獻,沃盥于前。既灌以升,乃薦豆籩。神其感格,歆於吉蠲。



 升自西階,登歌奏夾鐘宮《嘉寧之曲》:餘並同親祀。



 國有太宮,合食以禮。躋階肅肅,降陛濟濟。鏘然純音,節乃容止。神之格思,永綏福履。



 時享,攝事登歌樂章。



 初獻盥洗,無射宮《肅寧之曲》:



 酌彼行潦,維挹其請。潔齊以祀,祀事昭明。顯允辟公,沃盥乃升。神之至止,歆于克誠。



 初獻升殿,夾鐘宮《嘉寧之曲》:餘同親祀,惟不用宮縣。



 濟濟在庭,祗薦有序。雍容令儀,旋規折矩。爰徂于基,鳴佩接武。敬恭神明,來寧來處。



 昭德皇后時享,登歌樂章。



 初獻盥洗,無射宮《肅寧之曲》:



 時祀有章,禮備樂舉。爰潔其盥,亦豐其俎。俯仰升降,中規中矩。神其來格,百神是與。



 初獻升殿,夾鐘宮《嘉寧之曲》:三獻及司徒降,同。



 假哉神宮,神宮有侐。惟時吉蠲,登降翼翼。歌鐘鏘煌,笙磬翕繹。於昭肅恭,靈釐來格。



 司徒奉俎,無射宮《豐寧之曲》:



 宮庭枚枚,鐘磬喤喤。既儀圭鬯,既奠惣薌。齊莊奉饋,籩豆大房。靈之右饗,流慶無疆。



 酌獻,無射宮《儀坤之曲》:



 於皇坤德,作合乾儀。塗山懿範,京室芳徽,容聲如在,典祀惟時。神其克享,薦祉來宜。



 亞終獻,無射宮《儀坤之曲》:



 嘉羞實俎,高張在庭。申獻合禮終獻改申為三,坤德儀刑。神其是聽,用鬯清明。清明既鬯,來享來寧。



 徹豆,夾鐘宮《豐寧之曲》:



 禮成於終,神心禗禗。惣蕭發馨,樂闕獻已。徒馭孔多,靈輿載轙。青玄悠悠,歸且億矣。



 宣孝太子別廟,登歌樂章。



 初獻升殿,夾鐘宮《承安之曲》:



 有腯斯牲,有馨斯齊。美哉洋洋,升降以禮。禮容既莊,樂亦諧止。神之格思,式歆明祀。



 酌獻,無射宮《和寧之曲》:



 於惟光靈,孝德昭宣。高麗有奕,來寧來燕。於薦惟祫,既時既蠲。從我烈祖,載享億年。



 亞終獻,《和寧之曲》:



 金石和奏,豆籩惟豐。祠宮奉事,齊敬精衷。笙吟伊浦,鶴駐緱峰。是保是饗,靈德無窮。



 徹豆,夾鐘宮《和安之曲》:



 寢成奕奕,今茲其時。明稱肇祀,將禮之儀。侯安以懌,羞嘉且時。樂闋獻已,神其饗思。



 大定三年十月,追上睿宗冊寶,應鐘宮《顯寧之曲》:



 天開休運。積仁而昌。命茲昭考,敢忘顯揚。上儀肇舉,涓日之良。來格來享,惠我無疆。



 大定十九年,升祔熙宗冊寶樂曲:



 恢大帝業,敉寧多方。懿德茂烈,金書發揚。肇舉上儀,涓擇吉日。鴻名赫赫,與天無極。



 上冊寶,宮縣《靜寧之曲》:



 日卜其吉,承祀孔肅。廣號追崇,孝心克篤。於乎悠哉,
 來思晬穆。寶冊既陳,委於宗祝。



 皇帝降殿,宮縣《鴻寧之曲》:



 繼世隆昌,臨朝靜默。追謚鴻名,發輝潛德。玉質金章,煌煌簡冊。涓辰展儀,永傳無極。



 殿庭樂歌。



 大定七年正月,上冊寶,皇帝將升御座,宮縣奏太蔟宮《泰寧之曲》:降座,同。



 德隆帝位,承天而興。侯邦來庭,民居安寧。歸美以報,傳之無極。鴻名徽稱,壽時萬億。



 冊寶入門,奏《天保報上之曲》:



 四方既平,功歸聖明。定功巍巍,丕享鴻名。股肱良哉,揄揚元首。儲精優游,南山等壽。



 奉冊寶官將復班位,奏《歸美揚功之曲》:



 聖德高明,萬邦咸休。錙銖唐虞,糠骰商周。維時群臣,對揚稽首。天子明明,令聞不朽。



 冊寶初行,奏《和寧之曲》:冊寶將升殿,皇太子自侍立位至降階,曲並同。



 四方攸同,昭哉成功。時和年豐,諸福來崇。英聲昭騰,和氣充塞。於乎皇王,維壽時億。



 皇太子升殿賀,奏《同心戴聖之曲》:



 穆清皇風,遐方來同。於昭于天,物和歲豐。丕受鴻名,
 對揚偉跡。純釐穰穰,敷錫罔極。



 上壽,皇帝將升御座,宮縣《和寧之曲》。同前。



 舉酒,《萬壽無疆之曲》:



 四海太平,吾皇之功。群臣對揚,誕受鴻名。霞觴瓊腴,君王樂豈。皇天垂休,萬壽無極。



 皇太子升階、降階,及與宴官升殿,並奏《和寧之曲》。同前。



 進第一爵,登歌奏《王道昌明之曲》:



 對天鴻休,於以鋪張。巍巍煌煌,超冠百王。皇圖皇綱,時維明昌。祉福無疆,于民敷揚。



 行群官酒,宮縣《和寧之曲》。文舞入,設群官食,奏《功成治
 定之舞》,三成止:



 聖德高明,如天強名。多方治平,功大有成。流於聲音,形於蹈舞。頒觴群臣。以昭禮遇。



 進第二爵,登歌奏《天子萬年之曲》:



 惟明后,馭寰瀛。躋升平,飛英聲。功三王,德五帝。游巖廊,億萬歲。



 行群官酒,宮縣《和寧之曲》。武舞入,設群官食,奏《四海會同之舞》,三成止:



 地平天成,時和歲豐。迓衡弗迷,率惟敉功。受天之祜,四方來荷。於萬斯年,不遐有佐。



 進第三爵,登歌《嘉禾之曲》:



 景命赫斯歸吾皇,仁風洋洋被遠荒。琛贄旅庭趨明光,氣和薰蒸為嘉祥。殊本合穗真異常,庾如坻京歲且穰。猗歟鴻休超前王,播為聲詩傳無疆。行群官酒、設群官食、群官降階,宮縣並奏《和寧之曲》,皇帝將降御座,奏《泰寧之曲》,並用太簇宮。



 大定十一年十一月,行冊禮,皇帝升御座,宮縣《泰寧之曲》:



 皇皇穆穆,袞服玉趾,如日之升,如山仰止。九賓在列,媚茲天子。願言無疆,介以繁祉。



 冊寶入門,奏《天保報上之曲》:



 穆穆元聖,天迪子保。相維臣工,以奏丕號。揚于路朝,玉牒神寶。於萬斯年,吾君壽考。



 奉冊寶官將復班位,奏《歸美揚功之曲》:



 玉冊玉寶,尊聖天子,丕揚鴻名,昭受帝祉。閎休對天,其隆孰比。臣下同心,翼戴歸美。



 皇太子升殿賀,奏《同心戴聖之曲》:



 大矣我后,徽冊膺受。歡趨彤庭,拜手稽首。休明御辰,無疆萬壽。靈貺沓來,天地長久。



 舉酒,奏《萬壽無疆之曲》:



 聖德懋昭,民歸天祐。煌煌金書,典冊光受。備樂在庭,
 八音諧奏。群公奉觴,天子萬壽。



 進第一爵,登歌《王道昌明之曲》:



 明明我皇,道光化溥。百度惟新,禮修樂舉。藻飾太平,爛然可睹。超躋三王,暉映千古。



 設群官食,奏《和寧之曲》、《功成治定之舞》:



 穆穆我君,威折群醜。輝光日新,仁洽九有。容典葳蕤,超前絕後。端拱深嚴,寶冊膺受。



 第二爵,登歌奏《天子萬年之曲》:



 典禮修,惟明后。揚鴻名,燦瓊玖。羅華紳,為萬壽。歌南山,堅且久。



 行群官酒,奏《和寧之曲》、《四海會同之舞》:



 道隆政平,天開有德。萬國和寧,來王來極。昭受鴻名,俯徇列辟。錫飲行觴,歡心各得。



 第三爵,登歌奏《嘉禾之曲》:



 眾瑞畢至昭升平,爰生嘉禾迺合穗。膴膴大田無南東,稼茂如雲成豐歲。既刈既獲百室盈,擊壤歌沸野老聲。陶唐之民茲其比,帝力何有若自遂。



 大定十八年十二月,上受命寶,皇帝將升御座,宮縣奏《泰寧之曲》。並大呂宮:



 上帝有赫,懷此明德。畀之神寶,庸鎮萬國。臨軒是膺,
 登降維則。群臣拜首,年卜萬億。



 寶入門,奏《天保報上之曲》:



 受命大寶,昭答眷祐。珍符明貺,人為天授。文物具舉,《韶》、《濩》迭奏。群臣上之,天子萬壽。



 群臣合班,奏《歸美揚功之曲》:



 德冒生民,明明元后。端冕臨軒,神寶是受。群工來賀,咸拜稽首。無疆無期,享祚長久。



 皇太子升殿、并自侍立位降階,宮縣《稱觴介壽之曲》:



 上儀昭舉,膺時瑞玉。群辟在列,蹌蹌肅肅。袞衣桓圭,歸美稽首。升降惟時,天子萬壽。



 舉酒,登歌奏《萬壽無疆之曲》:



 上帝眷命,純休茲至。誕膺洪寶,光臨大器。稱觴對揚,嵩嶽萬歲。其寧惟永,無疆卜世。



 天德二年十月,冊立中宮,皇帝將升御座,宮縣奏《乾寧之曲》:降座,同。



 人道大倫,王化所基。明聖稽古,陰教欲施。臨軒發冊,備舉彝儀。《麟趾》《關睢》,宜播聲詩。



 冊寶入門,奏《昌寧之曲》:出門,同。



 羽衛充庭,淑旂徽章。禮儀具舉,涓辰以良。相我內訓,來儀椒房。億萬斯年,邦家之光。



 將受冊寶、以冊寶入門,宮縣奏《肅寧之曲》:命婦升、降,同。



 塗山興夏,《關睢》美周。坤儀之尊,母臨九州。瑤冊禕衣,光配凝旒。地久天長。福祿是遒。



 后出閣,奏《順寧之曲》:升、降座,同。



 天立厥配,任姒比隆。母儀四海,化行六宮。日月並明,乾坤合德。於萬斯年,作儷宸極。



 受冊,奏《坤寧之曲》:



 風化之始,由于壺闈。禮文斯備,爰正坤儀。維順以慈,儷聖同德。則百斯男,垂統無極。



 天德四年二月,冊皇太子,皇帝將升御座,宮縣奏《乾寧
 之曲》:皆用夾鐘宮。



 大君有為,先圖本固。涓辰之吉,禮成儲副。文物備陳,聲樂皆具。人心載寧,克昌福祚。



 冊使入門,《昌寧之曲》:



 在天成象,煥乎前星。惟聖時憲,典禮以行。一人有慶,萬邦以貞。社稷之福,浸昌浸明。



 皇太子入門,奏《元寧之曲》:出門,同。



 皇矣上帝,純佐明聖。篤生元良,日躋德性,冊命主器,萬邦以正。龍樓問寢,億年之慶。



 大定八年正月,冊皇太子,皇帝將升御座,宮縣《洪寧之
 曲》:並用太簇宮。



 會朝清明,臨軒備禮。天威皇皇,臣工濟濟。於昭元良,膺茲典冊。對揚閎休,卜年萬億。



 皇太子入門,奏《肅寧之曲》:



 光昭前星,惟天垂象。稽古而行,主器以長。曲禮告成,邇遐屬望。國本既隆,繁釐永享。



 群臣合班,奏《嘉寧之曲》:



 於皇臨軒,禮崇上嗣,維眷之祺,傃方正位。言觀其儀,翔翔濟濟。美歸吾君,太平萬歲。



 皇太子復受冊位,奏《和寧之曲》:



 祖功艱難,經營締構。基牢根深,枝繁葉茂。於昭貽謀,駢休集祐。元良斯貞,吾皇萬壽。



 大定二十七年三月,冊皇太孫,皇帝將升御座,宮縣《泰寧之曲》:並姑洗宮。



 上天叢休,申錫祚胤。孫謀有詒,臨軒體正。煌煌上儀,欣欣眾聽。隆我邦本,無疆惟慶。



 皇太孫入門,奏《慶寧之曲》:出門,同。



 寶源流光,流光惟遠。孫謀有貽,慶序昭衍。於樂眾望,於皇備典。動容周旋,承茲嘉羨。



 群臣合班,奏《順寧之曲》:



 冕旒當寧,徽章備舉。綵仗充庭,金石列虡,濟濟多士,翼翼就序。海潤山暉,傾聽樂府。



 皇太孫復受冊位,奏《保寧之曲》:



 禮之攸聞,丕建世嫡。眾論協從,天心不易。名崇震宮,辭著瑞冊。社稷宗廟,無疆夷懌。



 鼓吹導引曲



 天眷三年九月,駕幸燕京,導引曲:無射宮。



 五年一狩,仙仗到人間,問稼穡艱難。蒼生洗眼秋光裏,今日見天顏。金戈玉斧臨香火,馳道六龍閑。歌謠到處皆相似,天子壽南山。



 天德二年三月,祫享迴鑾,導引曲:



 禮成廟享,御衛拱飛龍,諸道起祥風。太平天子多受福,孝德與天通。鳳簫龍管《韶》音奏,聲在五雲中。粲然文物昭治世,萬億祀無窮。



 貞元元年三月,駕幸中都,導引曲:並姑洗宮。



 鑾輿順動,嘉氣滿神京,輦路宿塵清。鉤陳萬旅隨天仗,縹緲轉霓旌。都人望幸傾堯日,鰲撲溢歡聲。臨觀八極辰居正,寰宇慶昇平。



 《采茨曲》:



 新都春色滿,華蓋定全燕。時運千齡協,星辰五緯連。
 六龍承曉日,丹鳳倚中天。王氣盤山海,皇居億萬年。



 貞元三年十一月,祫享迴鑾,《采茨曲》:並用。



 慶成迴大駕,仙仗紫雲深。龍袞輝千騎,嵩呼間八音。太平興縟禮,萬國得歡心。孝格迎遐福,穰穰永降臨。



 正隆六年六月,駕幸南京,導引曲:並林鐘宮。



 神宮壯麗,宮殿壓蓬萊,向曉九門開。聖明天子初巡幸,遙駕六龍來。五雲影裏排仙仗,清蹕絕纖埃。都人齊唱昇平曲,更進萬年盃。



 《採茨曲》:



 雙闕層雲表,澄景開清曉。六龍天上來,馳道平如掃。
 虞巡五載合,夏諺一遊同。都人欣豫意,寫入頌聲中。



 大定三年十月,祫享迴鑾,《采茨》、導引曲:皆應鐘宮。自後親祀,二曲並用。



 太宮崇烈考,大禮慶初成。彩仗迴雲步,天階嚴蹕聲。舜宮合至孝,周《頌》詠維清。介福應穰簡,歡交萬國情。



 導引曲:



 禮行清廟,華黍薦年豐,聖孝與天通。六龍迴馭千官衛,玉振珮環風。黃麾金輅嚴天仗,非霧鬱葱葱。工歌疊奏升平曲,福祿自來崇。



 大定二十七年三月,皇太孫受冊,謝廟,導引曲:



 璇
 源濬發,衍慶自靈長,聖運日隆昌。震闈顯冊遵彞典,基緒煥重光。練時廟見嚴昭報,禮樂粲成章。精誠潛格神明助,福祿永無疆。



\end{pinyinscope}