\article{列傳第六十九}

\begin{pinyinscope}

 宦者



 ○梁珫宋珪潘守恆附



 古之宦者皆出於刑人,刑餘不可列於士庶,故掌宮寺之事,謂之「婦寺」焉。東漢以來,宦者養子以繼世。唐世,繼者皆為閹人,其初進也,性多巧慧便僻、善固恩寵,及其得志,黨比糾結不可制。東漢以宦者亡,唐又甚焉。世儒
 論宦者之害,如毒藥猛虎之不可拯也。金法置近侍局,嘗與政事,而宦者少與焉。惟海陵時有梁珫,章宗時有梁道、李新喜干政,二君為所誤多矣。世傳梁道勸章宗納李妃後宮,金史不載梁道始末,弗得而論次之。惟宋珪、潘守恒頗能諷諫宣、哀,時有裨益,蓋傭之佼佼、鐵之錚錚者也。作《宦者傳》。



 梁珫,本大抃家奴,隨元妃入宮,以閹豎事海陵。珫性便佞,善迎合,特見寵信,舊制,宦者惟掌掖廷宮闈之事。天德三年,始以王光道為內藏庫使,衛愈、梁安仁皆以宦官領內藏。海陵謂光道等曰:「人言宦者不可用,朕以為
 不然。後唐莊宗委張承業以軍,竟立大功,此中豈無人乎。卿等宜悉此意。帑藏之物皆出民力,費十致一,當糾察姦弊,犯者必罰無赦。」宦者始與政事,而珫委任尤甚,累官近侍局使。及營建南京宮室,海陵數數使珫往視工役。是時,一殿之費已不可勝計,珫或言其未善,即盡撒去。雖丞相張浩亦曲意事之,與之均禮。



 海陵欲伐宋,珫因極言宋劉貴妃絕色傾國。海陵大喜,及南征將行,命縣君高師姑兒貯衾褥之新潔者俟得劉貴妃用之。議者言珫與宋通謀,勸帝伐宋,徵天下兵以疲弊中國。



 海陵至和州,聞珫與宋人交通有狀,謂珫曰:「聞汝與宋
 國交通,傳泄事情。汝本奴隸,朕拔擢至此,乃敢爾耶。若至江南詢得實跡,殺汝亦未晚也。」又謂校書郎田與信曰:「爾面目亦可疑,必與珫同謀者。」皆命執於軍中。海陵遇弒,珫、與信皆為亂軍所殺。



 宋珪,本名乞奴,燕人也。為內侍殿頭。宣宗嘗以元夕欲觀燈戲,命乞奴監作,乞奴誶語云:「社稷棄之中都,南京作燈戲有何看耶。」宣宗微聞之,杖之二十,既而悔之,有旨宣諭。



 哀宗放鷂後苑,鷂逸去,敕近侍追訪之,市中一農民臂此鷂,近侍不敢言宮中所逸者,百方索之,農民不與,與之物直,僅乃得。事聞,哀宗欲送其人於有司,乞
 奴從旁諫曰:「貴畜賤人,豈可宣示四方。」哀宗惡其大訐,又仗之,尋亦悔,賜物慰遣之。



 及哀宗至歸德,馬軍元帥蒲察官奴為變,殺左丞李蹊、參政石盞女魯歡以下從官三百餘人。倉皇之際,哀宗不得已,以官奴權參知政事,既為所制,含恨欲誅之未能也。及官奴往亳州,珪陰與奉御吾古孫愛實、納蘭忔答,護衛女奚烈完出、范陳僧、王山兒等謀誅之。官奴自亳還,哀宗御臨漪亭,召參政張天綱及官奴議事。官奴入見,珪等即從旁殺之,及其黨阿里合、白進、習顯。及蔡城破,哀宗自縊於幽蘭軒,珪與完顏斜烈、焦春和等皆從死。



 有潘守恒者亦內侍
 也,素稱知書,南遷後規益甚多。及哀宗自蒲城走歸德,道次民家,守恒進櫛,曰:「願陛下還宮之日無忘此草廬中,更加儉素,以濟大業。」上聞其言,悽惋咨嗟久之。



 ◎方伎



 劉完素從正慶嗣天錫張元素馬貴中武禎子亢李懋胡德新



 太史公敘九流,述《日者》、《龜策》、《扁鵲倉公列傳》。劉歆校中秘書,以術數、方伎載之《七略》。後世史官作《方伎傳》,蓋祖其意焉。或曰《素問》、《內經》言天道消長、氣運贏縮,假醫術,
 託岐黃,以傳其秘奧耳。秦人至以《周易》列之卜筮,斯豈易言哉!第古之為術,以吉凶導人而為善,後世術者,或以休咎導人為不善,古之為醫,以活人為功,後世醫者,或因以為利而誤殺人。故為政於天下,雖方伎之事,亦必慎其所職掌,而務旌別其賢否焉。金世,如武禎、武亢之信而不誣,劉完素、張元素之治療通變,學其術者皆師尊之,不可不記云。



 劉完素,字守真,河間人。嘗遇異人陳先生,以酒飲守真,大醉,及寤洞達醫術,若有授之者。乃撰《運氣要旨論》、《精要宣明論》,慮庸醫或出妄說,又著《素問玄機原病式》,特
 舉二百八十八字,注二萬餘言。然好用涼劑,以降心火、益腎水為主。自號「通元處士」云。



 張從正,字子和,睢州考城人。精於醫,貫穿《難》、《素》之學,其法宗劉守真,用藥多寒涼,然起疾救死多取效。古醫書有《汗下吐法》,亦有不當汗者汗之則死,不當下者下之則死,不當吐者吐之則死,各有經絡脈理,世傳黃帝、岐伯所為書也。從正用之最精,號「張子和汗下吐法」。妄庸淺術習其方劑,不知察脈原病,往往殺人,此庸醫所以失其傳之過也。其所著有「六門、二法」之目,存於世云。



 李慶嗣,洺人。少舉進士不第,棄而學醫,讀《素問》諸書,洞
 曉其義。天德間,歲大疫,廣平尤甚,貧者往往闔門臥病。廣嗣攜藥與米分遺之,全活者眾。慶嗣年八十餘,無疾而終。所著《傷寒纂類》四卷、《改證活人書》三卷、《傷寒論》三卷、《針經》一卷,傳於世。



 紀天錫,字齊卿,泰安人。早棄進士業,學醫,精於其技,遂以醫名世。集註《難經》五卷,大定十五年上其書,授醫學博士。



 張元素,字潔古,易州人。八歲試童子舉。二十七試經義進士,犯廟諱下第。乃去學醫,無所知名,夜夢有人用大斧長鑿鑿心開竅,納書數卷於其中,自是洞徹其術。河
 間劉完素病傷寒八日,頭痛脈緊,嘔逆不食,不和所為。元素往候,完素面壁不顧,元素曰:「何見待之卑如此哉。」既為診脈,謂之曰脈病云云,曰:「然。」「初服某藥,用某味乎?」曰:「然。」元素曰:「子誤矣。某味性寒,下降走太陰,陽亡汗不能出。今脈如此,當服某藥則效矣。」完素大服,如其言遂愈,元素自此顯名。平素治病不用古方,其說曰:「運氣不齊,古今異軌,古方新病不相能也。」自為家法云。



 馬貴中,天德中,為司天提點。與校書郎高守元奏天象災異忤旨,海陵皆杖之,黜貴中為大同府判官。久之,遷司天監。正隆三年三月辛酉朔,日當食。是日,候之不食,
 海陵謂貴中曰:「自今凡遇日食皆面奏,不須頒示內外。」



 海陵伐宋,問曰:「朕欲自將伐宋,天道何如?」貴中對曰:「去年十月甲戌,熒惑順入太微,至屏星,留、退、西出。《占書》,熒惑常以十月入太微庭,受制出伺無道之國。十二月,太白晝見經天,占為兵喪、為不臣、為更主,又主有兵兵罷、無兵兵起。」鎮戎軍地震大風,海陵以問,貴中對曰:「伏陰逼陽,所以震也。」又問曰:「當震,大風何也?」對曰:「土失其性則地震,風為號令,人君命令嚴急,則有烈風及物之災。」六年二月甲辰朔,日有暈珥戴背,海陵問:「近日天道何如?」貴中對曰:「前年八月二十九日,太白入太微右掖門,
 九月二日,至端門,九日,至左掖門出,並歷左右執法。太微為天子南宮,太白兵將之象,其占,兵入天子之廷。」海陵曰:「今將征伐而兵將出入太微,正其事也。」貴中又曰;「當端門而出,其占為受制,歷左右執法為受事,此當有出使者,或為兵,或為賊。」海陵曰:「兵興之際,小盜固不能無也。」及被害于揚州,貴中之言皆驗。



 大定八年,世宗擊球於常武殿,貴中上疏諫曰:「陛下為天下主,守宗廟社稷之重,圍獵擊球皆危事也。前日皇太子墜馬,可以為戒,臣願一切罷之。」上曰:「祖宗以武定天下,豈以承平遽忘之邪。皇統嘗罷此事,當時之人皆以為非,朕所親見,
 故示天下以習武耳。」



 十年十一月,皇太子生日,世宗宴百官于東宮。上飲歡甚,貴中被酒,前跪欲言事,錯亂失次,上不之罪,但令扶出。



 武禎,宿州臨渙人。祖官太史,靖康後業農,後畫界屬金。禎深數學。貞祐間,行樞密院僕散安貞聞其名,召至徐州,以上客禮之,每出師必資焉。其占如響。正大初,徵至汴京,待詔東華門。其友王鉉問禎曰:「朝廷若問國祚脩短,子何以對?」禎曰:「當以實告之,但更言周過其歷,秦不及期,亦在修德耳。」時久旱祈禱不應,朝廷為憂,禎忽謂鉉曰:「足下今日早歸,恐為雨阻。」鉉曰:「萬里無雲,赤日如
 此,安得有雨?」禎笑曰:「若是,則天不誠也。天何嘗不誠。」既而東南有雲氣,須臾蔽天,平地雨注二尺,眾皆驚嘆。尋除司天臺管勾。



 子亢,寡言笑,不妄交。嘗與一學生終日相對,握籌布畫,目炯炯若有所營,見者莫測也。哀宗至蔡州,右丞完顏仲德薦其術。召至,屏人與語,大悅,除司天長行,賞賚甚厚。上書曰:「比者有星變於周、楚之分,彗星起于大角西,掃軫之左軸,蓋除舊布新之象。」又言:「鄭、楚、周三分野當赤地千里,兵凶大起,王者不可居也。」又曰:「蔡城有兵喪之兆,楚有亡國之徵,三軍苦戰於西垣前後有日矣。城壁傾頹,內無見糧,外無應兵,君臣數盡
 之年也。」聞者悚然奪氣,哀宗惟嗟嘆良久,不以此罪。性頗倨傲,朝士以此非之。



 天興二年九月,蔡州被圍,亢奏曰:「十二月三日必攻城。」及期果然。末帝問曰:「解圍當在何日?」對曰:「明年正月十三日,城下無一人一騎矣。」帝不知其由,乃喜圍解有期,日但密計糧草,使可給至其日不闕者。明年甲午正月十日,蔡州破,十三日,大元兵退。是日,亢赴水死云。



 李懋,不知何許人。有異術。正大間,游京兆,行省完顏合達愛其術,與俱至汴京,薦於哀宗。遣近侍密問國運否泰,言無忌避。居之繁臺寺,朝士日走問之,或能道隱事
 及吉凶之變,人以為神。帝惡其言太洩,遣使者殺之。使者乃持酒肴入寺,懋出迎,笑曰:「是矣。」使者曰:「何謂也?」懋曰:「我數當盡今日,尚復何言。」遂索酒,痛飲就死。



 胡德新,河北士族也。寓居南陽,往來宛、葉間,嗜酒落魄不羈,言禍福有奇驗。正大七年夏,與燕人王鉉邂逅於葉縣村落中。與鉉初不相識,坐中謬以兵官對,胡曰:「此公在吾法中當登科甲,何以謂之兵官。」眾愕然,遂以實告。二人相得甚歡,即命家人具雞酒以待,酒酣、舉大白相屬曰:「君此去事業甚遠,不必置問。某有所見,久不敢對人言,今欲告子。」遂邀至野田,密謂曰:「某自去年來,行
 宛、葉道中,見往來者十且八九有死氣。今春至陳、許間,見其人亦有大半當死者。若吾目可用,則時事可知矣。」鉉驚問應驗遲速,曰:「不過歲月間耳,某亦不逃此厄,請密志之。」明年,大元兵由金、房入,取峭石灘渡漢,所過廬舍蕭然,胡亦舉家及難,其精驗如此。



\end{pinyinscope}