\article{志第十四}

\begin{pinyinscope}

 禮六



 ○原廟



 太宗天會二年,立大聖皇帝廟于西京。熙宗天眷二年九月,又以上京慶元宮為太祖皇帝原廟。皇統七年,有司奏:「慶元宮門舊日景暉,殿日辰居,似非廟中之名,今宜改殿名曰世德。」是歲,東京御容殿成。世宗大定二年十二月,詔以「會寧府國家興王之地,宜就慶元宮址建正殿九間,仍其舊號,以時薦享」。海陵天德四年,有司言:「
 燕京興建太廟,復立原廟。三代以前無原廟制,至漢惠帝始置廟於長安渭北,薦以時果,其後又置於豐、沛,不聞享薦之禮。今兩都告享宜止於燕京所建原廟行事。」於是,名其宮曰衍慶,殿曰聖武,門曰崇聖。



 大定二年,以睿宗御容奉遷衍慶宮。五年,會寧府太祖廟成,有司言宜以御容安置。先是,衍慶宮藏太祖御容十有二:法服一、立容一、戎衣一、佩弓矢一、坐容二、巾服一,舊在會寧府安置;半身容二,春衣容一、巾而衣紅者二,舊在中都御容殿安置,今皆在此。詔以便服容一,遣官奉安,擇日啟行。前一日,夙興,告廟,用酒饌,差奏告官
 一員,以所差使充,進請御署祝板。其日質明,有司設龍車於衍慶宮門外少西,東向。宰執率百官公服詣本宮殿下,班立,再拜。班首升殿,跪上香、奠酒,教坊樂作,少退,再拜。班首降階復位,陪位官皆再拜。奉送使副率太祝捧御容匣出,宰執以下分左右前導,出衍慶宮門外,俟御容匣升車,百官上馬後從,旗幟甲馬錦衣人等分左右導,香輿扇等前行。至都門郊外,俟御容車少駐,導從官下馬,車前立班,再拜。奉送使副側侍不拜。班首詣香輿,跪上香,俯伏,興,還班,再拜辭訖,退。使副遂行。每程到館或廨舍內安駐。其道路儀衛,紅羅傘一,龍車一,其制
 以青布為亭子狀,安車上,駕以牛。又用駝五,旗鼓共五十,舁香輿一十人,導從六十人,執扇八人,兵士百人,護衛二十人以宗室猛安謀克子孫充。所過州縣,官屬公服出郭香果奉迎,再拜,班首上香奠酒,又再拜。送至郊外,再拜乃退。至會寧府,官屬備香輿奉迎如上儀,乘馬從至廟門外下馬,分左右導引。使副率太祝四員,捧御容入廟,於中門外東壁幄次內奉置定,再拜,訖,退,擇日奉安。至日質明,差去官與本府官及建廟官等並公服,詣幄次前排立,先再拜,跪上香,樂作,奠酒,訖,又再拜。太祝捧御容,眾官前導引,至殿下排立。御容升殿奉安,訖,
 再拜,班首升殿,跪上香,讀祝,奠酒,樂作,少退再拜,訖,班首降階復位,同
 執事官再拜,訖,退。



 十五年二月,有司言東京開覺寺藏睿宗皇帝皁衣展裹真容,敕遷本京祖廟奉祀,仍易袍色。明年四月,詔依奉安睿宗禮,奉安世祖御容於衍慶宮。前期,有司備香案、酒果、教坊樂。至日質明,親王宰執率百官公服迎引至衍慶宮,凡用甲騎百人,傘二人,扇十二人、香輿八人、彩輿十六人、從者二十四人、執事官二人,弩手控鶴各五十人、贊者二人、禮直官二人,六品以下官三十員公服乘馬前導。奉安訖,百官再拜,禮畢,退立宮
 門之外,迎駕朝謁。十六年正月,有司奏:「奉敕議世祖皇帝御容當於何處安置。臣等參詳衍慶宮即漢之原廟,每遇太祖皇帝忌辰,百官朝拜。今世祖皇帝擇地修建殿位,庶可副嚴奉之意。」從之。乃敕於聖武殿東西興建世祖、太宗、睿宗殿位。既而復欲擇地建太宗殿於歸仁館,有司言:「山陵太祖、太宗、睿宗共一兆域,太廟
 世祖、太祖、太宗、睿宗亦同堂異室。今於歸仁館興建太宗殿位,似與山陵、太廟之制不同。」詔從前議,止於衍慶宮各建殿七間、閣五間、三門五間。乃定世祖殿曰廣德、閣曰燕昌,太宗殿曰丕承、閣曰光昭,睿宗殿曰天興、閣曰景福。



 十九年五月六日,奏告。七日,奉安。執事
 禮官二人,每位香案一、祭器席一、拜褥二、盥洗一、大勺篚巾全。前一日,太廟令率其屬掃除宮內外,又各設神座於殿上,又設親王宰執以下百官拜位於殿庭。又設盥洗位於東階下,執罍篚者位於其後。又於神位前各設北向拜褥位,並各設香案香爐匙合香酒花果器皿物等,依前來例。又於聖武殿上設香案爐匙合香等,又於殿下各設腰輿一、舁士一十六人、傘子各二人、執扇各十二人、導從弩手各三十人。前一日,清齋,親王於本府,百官於其第。行禮官執事人等習儀,就祠所清齋。其日質明,禮官率太廟署官等詣崇聖閣奉
 世祖御容,每匣用內侍二人、太祝一員,禮官、署官前導,置於聖武殿神座。禮直官引親王宰執百官公服於殿庭班立,七品以下班於殿門之外,贊者曰:「拜。」在位官皆再拜。禮直官引班首詣罍洗,盥手訖,升殿,詣神座前跪上香,訖,少退,再拜。禮直官引班首降殿復位,贊者曰:「拜。」在位官皆再拜,訖,禮直官導
 世祖御容升腰輿,儀衛依次序導從,至廣德殿,百官後從,至庭下班位立。禮官率太廟署官就腰輿內捧禦容,於殿上正面奉安訖,百官於階下,六品已下官於殿門外,立班。贊者曰:「再拜。」在位官皆再拜。禮直官引班首盥洗,盥手訖,升殿,執事官等從升,詣御容前,跪上香,奠酒,教坊樂作,少退再拜,訖,樂止。禮直官引班首降殿復位,贊者曰:「拜。」在位官皆再拜。訖,禮官率太廟署
 官詣崇聖閣。太祝內侍捧太宗御容,禮官導太宗御容置於聖武殿,行禮畢,以次奉安於丕承殿,行禮並如上儀。次睿宗御容奉安於天興殿,禮亦如之。俟奉安禮畢,百官退。



 二十一年閏三月,奉旨昭祖、景祖奉安燕昌閣上,肅宗、穆宗、康宗奉安閣下,明肅皇帝奉安崇聖閣下。每位設黃羅幕一、黃羅明金柱衣二、紫羅地褥一、龍床一、踏床二、衣全,前期奏告。四月一日奉安,五日親祀。是年五月,遷聖安寺睿宗皇帝御容於衍慶宮,皇太子親王宰執奉迎安置。



 ○朝謁儀



 大定十六年四月十九日,奉安世祖御容,行朝謁之禮。皇帝前一日齋於內殿,皇太子齋於本宮,親王齋於本府,百官齋於其第。太廟令率其屬,於衍慶宮內外掃除,設親王百官拜位於
 殿庭,又設皇太子拜褥於親王百官位前。宣徽院率其屬,於聖武門外之東設西向御幄,靈星門東設皇太子幄次。其日,有司列仗衛於應天門,俟奉安御容訖,有司於殿上並神御前設北向拜褥位,安置香爐香案並香酒器物等。皇太子比至車駕進發已前,公服乘馬,本宮官屬導從,至衍慶宮門西下馬,步入幄次。親王百官於衍慶宮門外西向立班。俟車駕將至,典贊儀引皇太子出幄次,於親王百官班前奉迎。導駕官,五品六品七品職官內差四十員於應天門外道南立班以俟。皇帝服靴袍乘輦,從官傘扇侍衛如常儀。敕旨用大安輦、儀仗一千人。出應天門,閣門通喝:「導駕官再拜。」訖,閣門傳敕:「道駕官上馬。」分左右前導,至廟門外西偏下馬。車駕至衍慶宮門外稍西降輦。左右宣徽使前導,皇帝步入御幄,簾降。閣門先引親王、宰執、四品已上執事官,由東西偏門入,至殿庭分東西班相向立。典贊儀引皇太子入,立於褥位之西,東向。進香進酒等執事官並升階,於殿上分東西向以次立。宣徽使跪奏:「請皇帝行朝謁之禮。」簾卷,皇帝出幄。宣徽使前導,至殿上褥位,北向立。典贊儀引皇太子就褥位,閣門引親王宰執四品已上職事官回班,並北向立。令中間歇空,不礙奏樂。五品以下聖武門外、八品以下宮門外陪拜。二宣徽使奏請,皇
 帝再拜,教坊樂作。皇太子已下群官皆再拜。請皇帝詣神御前褥位,北向立,又請皇帝再拜,皇太子已下群官皆再拜。請皇帝跪,三上香,三奠酒,俯伏,興。又請皇帝再拜,皇太子已下群官皆再拜,訖,皇帝復位。又請皇帝再拜,皇太子已下群官皆再拜。宣徽使奏:「禮畢。」已上擬八拜,宣徽院奏過,依舊例十二拜。典贊儀引皇太子復立於褥位之西,東向。閣門引親王宰執以下群官,東西相向立。先引五品已下官出。宣徽使前導,皇帝還御幄,簾降。典贊儀引皇太子,閣門分引殿庭百官,以次出。宣徽使跪奏:「請皇帝還宮。」簾卷,步出廟門外,升輦還宮,如來儀。十九年舉安禮同。



 ○朝拜儀



 初,太祖忌辰,皇帝至褥位立,再拜。稍東,西向,詣香案前,又再拜。上香訖,復位,又再拜。進食、奠茶、辭神皆再拜而退。大定二十一年五月十二日,睿宗忌辰,有司更定儀
 禮。前一日,宣徽院設御幄於天興殿門外稍西。至日質明,皇太子親王百官具公服於衍慶宮門外立班,奉迎。皇帝乘馬至衍慶宮門外下馬,二宣徽使前導,步入宮門稍東。皇帝乘輦,傘扇侍衛如常儀,至天興殿門外稍西。皇帝降輦,入幄次,簾降,典贊儀引皇太子、閣門引親王宰執四品已上官由偏門入,至於殿庭,左右分班立定,二宣徽使導皇帝由天興門正門入,自東階升殿,詣褥位立定。皇太子已下官合班,五品以下班於殿門外。宣徽使奏:「請皇帝先再拜。請詣侍神位立。」俟有司置香案酒卓訖,請詣褥位,又再拜,三上香、奠酒,復位,再拜。已上,皇太子以下皆陪拜。再奏:「請詣稍東侍神位立。」典贊儀引皇太子升殿赴褥位,先兩拜,奠酒再兩拜,降復褥位。次閣門引終獻官趙王上殿行禮。宣徽使奏:「請皇帝詣褥位。」再兩拜。皇太子以下官皆再拜。禮畢,百官依前分班立。
 皇帝出殿門外,入幄次,簾降,更衣。次引皇太子以下官出宮門外立班。皇帝乘輦,至宮門稍東降輦,步出宮門外,上馬還宮,導從侍衛如來儀。皇太子以下官,俟車駕行然後退。大定五年,奏旨:「太祖忌辰,衍慶宮薦享止用素食,諸京凡御容所在皆同。又朔望皆行朝拜禮。」六年,有司奏:「太祖皇帝忌辰,車駕親奠,百官陪拜。今車駕巡幸,合以宰臣為班首,率百官詣衍慶宮行禮。」從之。十六年,奉旨:「世祖、太宗忌辰,一體奉奠。」十八年八月,太祖忌辰,世祖、太宗同在一處致祭,有司言「歷代無一聖忌辰列聖預祭之典。」擬議間,敕遣太子,一位行禮,並諸祭功臣。二十六年,以內外祖廟不同,定擬:「太廟每歲五享,山陵朔、望、忌辰及節辰祭奠並依前代典故外,衍慶宮自來車駕行幸,遇祖宗忌辰百官行禮,並詣京祖廟節辰、忌辰、朔、望拜奠,雖無典故參酌,恐合依舊,以盡崇奉之意。」從之。



 ○別廟



 大定二年,有司擬奏閔宗無嗣,合別立廟,有司以時祭享,不稱宗,以武靈為廟號。又奏:「唐立別廟,不必專在太廟垣內。今武靈皇帝既不稱宗,又不與祫享,其廟擬於太廟東墉外隙地建立。」從之。十四年,廟成,以武靈后謚孝成,又謂之孝成廟。十五年三月戊申,奉安武靈皇帝及悼皇后。前期一日,奏告太廟十一室及昭德皇后廟,餘如昭德過廟之儀。四月十七日,夏享太廟,同時行禮,命判宗正英王爽攝太尉,充初獻官。兵部尚書讓攝司徒,差大理卿天錫攝太常卿,充亞獻。大興少尹高居中攝光祿卿,充終獻。自是,歲常五
 享。十七年十月,祫享太廟,「檢討唐禮,孝敬皇帝廟時享用廟舞、宮縣、登歌,讓皇帝廟至禘祫月一祭,只用登歌,其禮制損益不同。今武靈皇帝廟庭與太廟地步不同,難以容設宮縣樂舞,並樂器亦是闕少,看詳恐合依唐讓皇帝祫享典故,樂用登歌,所有牲牢樽俎同太廟一室行禮。及契勘得自來祫享,遇親祠每室一犢,攝官行禮共用三犢。今添武靈皇帝別廟行禮,合無依已奏定共用三犢,或增添牛數。」奏奉敕旨:「太廟、別廟共用三犢,武靈皇帝廟樂用登歌,差官奏告,並準奏」。大定十九年四月,升祔太廟,其舊廟遂毀。



 昭德皇后廟。大定二年,有司援唐典,昭德皇后合立別廟,擬於太廟內垣東北起建,從之。三年十月七日,太廟祫享,升祔睿宗皇帝并昭德皇后,神主同時製造題寫,奉詣殿庭,謁畢祔於祖姑欽仁皇后之左,享祀畢,奉主還本廟。十二月二十一日,臘享,禮官言:「唐
 禮,別廟薦享皆準太廟一室之儀,伏恐今廟享畢已過質明,請別差官攝祭。」制可。後以殿制小,又於太廟之東別建一位。十二年八月,廟成,正殿三間,東西各空半間,以兩間為室,從西一間西壁上安置祏室。廟置一便門,與太廟相通。仍以舊殿為冊寶殿,祏室奏毀。十三年六月二十一日,奏告太廟,祭告別廟。二十三日,奉安,用前祫享過廟儀。有司言當用鹵簿,以廟相去不遠,參酌擬用清道二人,次圍扇二人,次職掌八人,次衙官二十六人為十三重,供奉官充。次腰輿,輿士一十六人,傘子二人,次圍扇十四為七重,方扇四,次排列職掌六人,燭籠
 十對,輦官並錦襖盤裹。仍令皇太子率百官行禮。前一日,行事執事官就祠所清齋一宿,仍習儀。執事者視醴饌,太廟令帥其屬掃除廟之內外。禮直官設皇太子西向位,執事官位皇太子後,近南,西向,各依品從立。監祭,殿西階下東向立。及親王百官位於廟庭,北向,西上,又設祝案於神位之右,設尊彞之位於左,各加勺、冪、坫。又設祭器,皆籍以席,左一籩實以鹿脯,右一豆實以鹿臡。又設盥洗、爵洗位于橫街之南稍東。罍在洗東,加勺。篚在洗西,南肆,實以巾。執罍篚者位于其後。太廟令又設神位於室內北墉下,當戶南向。設直几一、黼扆
 一、莞席一、繅席一、次席二、紫綾厚褥一、紫綾蒙褥一并幄帳等,諸物並如舊廟之儀。又設望燎位于西神門外之北,設燎柴於位之北,預掘瘞坎于燎所,所司陳儀衛於舊廟門之外。奉安日未明二刻,所司進方扇燭籠於舊廟殿門外,設腰輿一、傘一於殿階之下,南向。質明,皇太子公服乘馬,本宮官屬導從,至廟門外下馬,步入廟門,至幕。次引親王百官常服由廟門入,於殿庭北向西上、重行立定。次引皇太子於百官前絕席位立,贊者曰:「再拜。」皆再拜。宮闈令升殿,捧昭德皇后神主置于座,贊者曰:「再拜。」皆再拜。次引內常侍北向俯伏,跪奏:「請昭德皇后神
 主奉安于新廟,降殿升輿。」奏訖,俯伏,興。捧幾內侍先捧几匱跪置於輿,又宮闈令接神主,內侍前引,跪置于輿上幾後,覆以紅羅帕。內常侍以下分左右前引,皇太子步自舊廟先從行,親王次之,百官分左右後從,儀衛導從,至別廟殿下北向。內常侍於腰輿前俯伏,興,跪奏:「請降輿升殿。」內侍捧几匱前,宮闈令捧接神主升殿,置於座。禮直官引皇太子以下親王百官入殿庭,北向西上、重行立,皇太子在絕席立,禮直官贊曰:「再拜。」皆再拜。又贊曰:「行事官各就位。」禮直官引皇太子西向位立定。禮直官少前贊曰:「有司謹具,請行事。」即引皇太子就盥洗
 位,北向,搢笏,盥手,帨手,執笏。詣爵洗位,北向立,搢笏,洗爵,拭爵以授執事者。執笏,升,詣酒尊所,西向立,執事者以爵授皇太子,搢笏,執爵。執事者舉冪酌酒,皇太子以爵授執事者,詣神位前北向,搢笏,跪。執事者以爵授皇太子,執爵三祭酒,反爵于坫,執笏,俯伏,興,少立。次引太祝、舉祝官詣讀祝位東北向,舉祝官跪舉祝版,太祝跪讀祝,訖,置祝于案,俯伏,興。舉祝官皆卻立北向。贊者曰:「再拜。」皇太子就兩拜,降階復位。舉祝、讀祝官後從,復本位。禮直官曰:「再拜。」在位者皆再拜。宮闈令納神主於室,贊者曰:「再拜。」皆再拜,禮畢,退。署令闔廟門,瘞祝于坎,儀
 物各還所司。十一年,郊祀前一日朝享,與太廟同日,用登歌樂,行三獻禮,有司攝事。二十六年,敕別建昭德皇后影廟于太廟內。有司言:「宜建殿三間,南面一屋三門,垣周以甓,外垣置靈星門一,神廚及西房各三間。然禮無廟中別建影廟之例,今皇后廟西有隙地,廣三十四步,袤五十四步,可以興建。」制可。仍於正南別創正門,門以坤儀為名。仍留舊有便門,遇禘祫祔享由之。每歲五享并影廟行禮於正南門出入。又於廟外起齋廊房二十三間。



 宣孝太子廟。大定二十五年七月,有司奏:「依唐典,故太子置廟,設官屬奉祀。擬於法物庫東建殿三
 間,南垣及外垣皆一屋三間,東西垣各一屋一門,門設九戟。齋房、神廚,度地之宜。」又奉旨:「太子廟既安神主,宜別建影殿。」有司定擬制度,於見建廟稍西中間,限以磚墉,內建影殿三間。南面一屋三門,垣周以甓,無闕角及東西門。外垣正南建三門一,左右翼廊二十間,神廚、齋室各二屋三間,是歲十月,廟成。十一日奉安神主,十四日奉遷畫像。神主用栗,依唐制諸侯用一尺,刻謚于背。省部遣官於本廟西南隅面北設幄次,監視製造,於行禮前一日製造訖。其日晚,奉神主官奉承以箱,覆以帕,捧詣題神主幄中。次日丑前五刻,題神主官與典儀并
 禮官詣幄次前,題神主官詣罍洗位,盥手、帨手訖,奉神主官先以香湯奉沐,拭以羅巾。題神主官就褥位,題謚號於背云「宣孝太子神主」,墨書,用光漆模,訖,授奉神主官,承以箱,覆以梅紅羅帕,藉以素羅帕,詣座置於匱,乃下簾帷,侍衛如式。俟典儀俯伏,跪請,備腰輿傘扇詣神位。導引侍衛皆減昭德廟儀。祭儀,有司言:「當隨祖廟四時祭享。初獻於皇孫皇族、亞獻於皇族或五品以下差。樂用登歌,今量減用二十五人,其接神用無射宮,升降徹豆則歌夾鐘。牲羊、豕各一、籩豆各八,簠簋各二,登鉶各一,其餘祭食亦量減之。」二十六年十一月一日,奏:「神
 主廟,牲牢樂縣官給。影廟,皇孫奉祀。」



\end{pinyinscope}