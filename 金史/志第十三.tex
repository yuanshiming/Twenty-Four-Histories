\article{志第十三}

\begin{pinyinscope}

 禮
 五



 ○上尊謚



 天會三年六月,諳班勃極烈杲等表請追冊先大聖皇帝。十二月二十五日,奉玉冊、玉寶,恭上尊謚曰大聖武元皇帝,廟號太祖。天會十三年三月七日,遣攝太尉皇叔祖大司空昱奉玉冊、玉寶,上尊謚曰文烈皇帝,廟號太宗。九月,追謚皇考曰景宣皇帝,廟號徽宗。



 十四年八月庚戌,文武百僚、太師宗磐等上議曰:「國家肇造區夏,
 四征弗庭,太祖武元皇帝受命撥亂,光啟大業。太宗文烈皇帝繼志卒伐,奮張皇威。原其積德累功,所由來者遠矣!且禮多為貴,固前籍之美談;德厚流光,實本朝之先務。伏惟皇九代祖,廓君人之量,挺御世之姿。虞舜生馮,遷於負夏,太王避狄,邑此岐山,聖姥來歸,天原肇發。皇八代祖、皇七代祖,承家襲慶,裕後垂芳,不求赫赫之名,終大振振之族。皇六代祖,徒居得吉,播種是勤,去暴露獲棟宇之安,釋負載興車輿之利。皇五代祖孛堇,雄姿邁世,美略齊時。成百里日辟之功,戎車既飾;著五教在寬之訓,人紀肇修。皇高祖太師,質自天成,德為民望,
 兼精騎射,往無不摧,始置官師,歸者益眾。皇曾祖太師,威靈震遠,機警絕人,雅善運籌,未嘗衿甲,臨敵愈奮,應變若神。皇曾叔祖太師,機獨運心,公無私物,四方聳動,諸部歸懷,德威兩隆,風俗大定。皇伯祖太師,友于盡愛,國爾惟忠,謀必罔愆,舉無不濟。累代祖妣,婦道警戒,王業艱難,俱殫內助之勞,實著始基之漸。是宜采群臣之僉議,酌故事以遵行,款帝于郊,稱天以誄。謹按謚法,布義行剛曰『景』,主義行德曰『元』,保民耆艾曰『明』,溫柔聖善曰『懿』,請上皇九代祖尊謚曰景元皇帝,廟號始祖,妣曰明懿皇后。中和純備曰『德』,道德純一曰『思』,請上皇八代
 祖尊謚曰德皇帝,妣曰思皇后。好和不爭曰『安』,好廉自克曰『節』,請上皇七代祖尊謚曰安皇帝,妣曰節皇后。安民治古曰『定』,明德有勞曰『昭』,尊賢讓善曰『恭』,柔德好眾曰『靖』,請上皇六代祖尊謚曰定昭皇帝,廟號獻祖,妣曰恭靖皇后。愛民立政曰『成』,辟土有德曰『襄』,強毅執正曰『威』,慈仁和民曰『順』,請上皇五代祖孛堇尊謚曰成襄皇帝,廟號昭祖,妣曰威順皇后。愛民好與曰『惠』,辟土兼國曰『桓』,明德有勞曰『昭』,執心決斷曰『肅』,請上皇高祖太師尊謚曰惠桓皇帝,廟號景祖,妣曰昭肅皇后。大而化之曰『聖』,剛德克就曰『肅』,思慮深遠曰『翼』,一德不懈曰『簡』,請
 上皇曾祖太師尊謚曰聖肅皇帝,廟號世祖,妣曰翼簡皇后。申情見貌曰『穆』,博聞多能曰『憲』,柔德好眾曰『靜』,聖善周聞曰『宣』,請上皇曾叔祖太師尊謚曰穆憲皇帝,廟號肅宗,妣曰靜宣皇后。慈愛忘勞曰『孝』,執事有制曰『平』,清白守節曰『貞』,愛民好與曰『惠』,請上皇曾叔祖太師尊謚曰孝平皇帝,廟號穆宗,妣曰貞惠皇后。愛民長悌曰『恭』,一德不懈曰『簡』,夙夜共事曰『敬』,小心畏忌曰『僖』,請上皇伯祖太師尊謚曰恭簡皇帝,廟號康宗、妣曰敬僖皇后。仍請以始祖景元皇帝、景祖惠桓皇帝、世祖聖肅皇帝、太祖武元皇帝、太宗文烈皇帝為永永不祧之廟。須
 廟室告成,涓日備物,奉上寶冊,藏于天府,施之罔極。」丙辰,奉上九代祖妣尊謚廟號,是日百僚上表稱賀。



 皇統五年,增上太祖尊謚,禮官議:「自古辨祀,以南北郊、太社、太稷、太廟為序。若太廟神主造畢,即合題尊謚,擇日奉安,恐在郊社之前於禮未倫。候築郊兆畢,擇日奏告昊天上帝、皇地祇,次奉安社稷神主及奏告,其次恭造太廟神主,題號奉安入室,以此為序。元奉敕旨,候到上京行禮,不見元奏目內,有無指定候修建太廟奉安神主以後行禮,或只於慶元宮奉上謚號。若候奉安太廟神主禮畢,方奉上謚號冊寶,即百官並合法服,兼於皇帝
 所御殿合立黃麾仗及殿中省細仗,太廟殿前亦合立黃麾仗,其冊寶在路亦合量設儀仗。若太廟未奉安,只於慶元宮上冊寶,即行事及立班官並用常服,及依例量用大小旗、甲騎、門仗官,供奉官引從冊寶彩服。若奉安後發冊,即御服通天冠、絳紗袍。若只就慶元宮,即襆頭紅袍。並慶元官上冊寶,即將來題太廟本室神主,便可用新謚。若於太廟先奉安神主,即先題舊謚。及至就本室上冊寶,又須改題新謚。有兩節不同。五月九日擬奏告於太廟,上冊寶,竊慮法物樂舞難辦,只於慶元宮上冊寶。」從之。



 十月三日,奉上尊謚冊寶儀:前期,有司供張
 辰居殿神御床案。少府監、鉤盾署設燎薪于殿庭西南,掘坎於其側。儀鸞司設小次於辰居殿下東廂,又設冊寶幄殿于景輝門外東仗舍。殿前司、宜徽院量差甲騎、大小旗鼓、門仗官、香輿,自製造冊寶所迎奉冊寶,奉安於幄殿,行事官、製造官皆騎馬引從,門下中書侍郎在前,侍中中書令在後,大禮使又在其後,舉舁奉冊寶官、製造官分左右夾侍,以北為上,皆給人從錦帽衫帶。是日未明,翰林使、大官令丞鋪設香案酒果、供具牲體膳羞於神御前,儀鸞司設皇帝拜褥四:一在阼階上,面西;一在香案南,面北;一在殿上東欄子內,面西;一在燎薪
 之東,面西。設黃道,自小次至阼階褥位。質明,有司備常行儀仗,駕頭扇筤,常朝官常服騎馬執鞭前導,以北為上,造冊寶官,排辦管勾官常服,於慶元宮門外立班,迎駕再拜。皇帝自宮中服靴袍、御馬,至景暉門外下馬,步入小次。少頃,御史臺催班,大禮使、行事官自幄殿奉冊寶入正門,置于辰居殿西階下。大禮使歸押班位,閣門使奏:「班齊。」太常卿奏:「請皇帝行奉上冊寶之禮。」宣徽使、太常卿分引前導,皇帝由黃道升阼階上面西褥位立,贊:「請再拜」,閤門使臚傳,在位官皆再拜。乃引皇帝由殿上正門入殿,於香案前褥位再拜,上香,又再拜,退稍東
 於欄子內面西褥位立定。儀鸞司徹香案前拜褥,設冊寶褥位於香案南,舉冊、舁冊官取冊匣于床,對捧由西階升,中書侍郎分左右前導。奉冊中書令、讀冊中書令並後從,候於褥位。置定,奉冊中書令於褥位南再拜,退就殿階上西南柱外,面東立。讀冊官、中書令稍前,再拜。舁冊官取匣蓋下,置於西階下冊床。舉冊官對舉冊,讀冊官中書令一拜起,跪,搢笏,讀冊文曰:「孝孫嗣皇帝臣某,謹拜手稽首奉玉冊玉寶,恭上尊謚曰應乾興運昭德定功睿神莊孝仁明大聖武元皇帝。」讀冊畢,就拜,興,又再拜,退立于奉冊中書令之次。奉冊官進,與中書侍
 郎率舉冊、舁冊官奉冊匣由西階下,引從如上儀,復置於冊床。置定,舉寶官以寶盝進,至侍中讀畢,由西階下,復置于床,皆如冊匣之儀。有司徹冊寶褥位,復設香案南拜褥。宣徽使、太常卿導皇帝進就褥位,再拜,上香、茶、酒,樂作,三酹酒,樂止。太祝讀祝文,訖,皇帝再拜,復歸阼階褥位,立定。大禮使升殿,於香案南宣徽使處授福酒臺盞,行至皇帝阼階褥位前,宣徽使贊:「皇帝再拜飲福。」閣門臚傳:「賜胙,再拜。」應在位官皆再拜。大禮使跪,以酒盞進授皇帝,樂作,飲訖,又再拜。大禮使受酒盞,復以授宣徽使,訖,由西階下,歸押班位。太祝奉祝版,翰林使酌
 酒,太官令丞量取牲羞,自西階下,置于燎薪之上。文武班皆回班向燎所立,禮官贊:「請皇帝就望潦位。」宣徽使取酒盞臺于翰林使,以進授皇帝。皇帝酹酒於燎薪之上,執事者舉燎,半燎,瘞于坎。宣徽使贊:「皇帝再拜。」閣門喝:「百官皆再拜。」太常卿、宣徽使前導,皇帝歸小次,即御座,簾降。太常卿俯伏,興,跪奏:「太常卿臣某言,禮畢。」百官皆卷班西出。大禮使以下奉冊寶床,納于慶元宮收掌去處。皇帝進膳于別殿,侍食官取旨,有司轉仗由來路,皇帝便服還內,教坊作樂前導。次日,大禮使率百官稱賀。



 是歲閏十一月,增上祖宗尊謚,始祖景元皇帝曰懿
 憲景元皇帝,德皇帝曰淵穆玄德皇帝,安皇帝曰和靖慶安皇帝,獻祖定昭皇帝曰純烈定昭皇帝,昭祖成襄皇帝曰武惠成襄皇帝,景祖惠桓皇帝曰英烈惠桓皇帝,世祖聖肅皇帝曰神武聖肅皇帝,肅宗穆憲皇帝曰明睿穆憲皇帝,穆宗孝平皇帝曰章順孝平皇帝,康宗恭簡皇帝曰獻敏恭簡皇帝,太宗文烈皇帝曰體元應運世德昭功哲惠仁聖文烈皇帝,徽宗景宣皇帝曰允恭克讓孝德玄功佑聖景宣皇帝,已上廟號如故。十二月一日,奏告如儀。



 大定三年,增上睿宗尊謚。先是,元年十一月十六日,追冊皇考曰簡肅皇帝,廟號睿宗,皇妣
 蒲察氏欽慈皇后,皇妣李氏貞懿皇后。二年八月一日,有司奏:「祖宗謚號或十六字,或十四字,或十二字,即今睿宗皇帝更合增上尊謚,於升祔前奉冊寶。」制可。十七日,左平章元宜等奏請增上尊謚曰睿宗立德顯仁啟聖廣運文武簡肅皇帝。有司奏:「睿宗皇帝未經升祔,合無於衍慶宮聖武殿設神御床案?」奉旨崇聖閣借設正位。又奏:「皇帝親授冊寶,太尉行事。」制可。



 九月二十二日,奏告太廟。二十八日,大安殿置大樂,閱習。前一日,自衍慶宮奉迎冊寶,於大安殿安置。



 授冊日未明三刻,有司各勒所部,整肅儀衛,群臣集于殿門,行事官各法服,陪
 位官公服。皇帝自宮中常服乘輿,侍衛如儀,赴大安殿後更衣幄次。御史臺催班,通事舍人引太尉及群臣就位。侍中跪奏:「中嚴。」少頃,又跪奏:「外辦。」皇帝服通天冠、絳紗袍出。太常卿跪奏稱:「太常卿臣某言,請皇帝行奉上冊寶之禮。」奏訖,俯伏,興。宣徽使分左右前導,皇帝步詣冊寶幄次。將至幄次,登歌樂作,至幄次前北向,宣微使贊:「請皇帝再拜。」典儀贊:「在位官再拜。」拜訖,奏:「請皇帝搢圭。」三上香,訖,執圭。奏:「請皇帝再拜。」典儀贊:「在位官再拜。」訖,各分班東西序立。奏:「請皇帝詣稍東褥位。」樂止。中書令、中書侍郎奉引冊,侍中、門下侍郎奉引寶,行,登歌樂
 作。宣徽使贊導皇帝隨冊寶降自西階,登歌樂止,宮縣樂作,至大安殿下當中褥位。中書令、侍中奉冊寶於皇帝褥位之西,樂止。宣徽使奏:「請皇帝再拜。」典儀贊:「在位官皆再拜。」拜訖,中書令搢笏,奉冊匣,宮縣樂作,至皇帝褥位前,俯伏,跪,奉置訖,執笏,俯伏,興,退稍西立,東向。太常博士引太尉至褥位,北向立。宣徽使奏:「請皇帝搢圭。」跪捧冊匣授太尉,太尉搢笏,跪受訖,執笏,少東立,宣徽使奏:「請執圭。」俯伏,興。舁冊官捧冊匣,中書侍郎奉冊匣置於冊床,樂止。侍中搢笏,奉寶盝,宮縣樂作,至皇帝褥位前,俯伏,跪,奉置訖,執笏,俯伏,興,退稍西立,東向。太常博士引太
 尉至褥位,北向立。宣徽使奏:「皇帝搢圭。」跪捧寶盝授太尉,太尉搢笏,跪,受訖,執笏,少東立。宣徽使奏:「請執圭。」俯伏,興。舁寶官捧寶盝,門下侍郎奉置於寶床,樂止。宣徽使奏:「皇帝再拜。」典儀贊:「在位官再拜。」皇帝南向立,宮縣樂作。太常博士引太尉奉冊寶出,主節者持節前導,冊床在前,寶床次之,樂止。中書門下侍郎各導於冊寶之前,太尉居其後,至大安門外,太尉以次跪奉冊寶於玉輅中,中書侍郎於輅旁夾侍,所司迎衛如式。太尉奉冊寶訖,步出通天門外,革車用本品鹵簿,導從如儀,鼓吹不振作。俟冊寶出大安門,太常卿跪奏稱:「太常卿臣某
 言,禮畢。」奏訖,俯伏,興,前導皇帝升自東階,登歌樂作,還大安殿後幄次,樂止。侍中跪奏:「解嚴。」乘輿還內,侍衛如來儀。



 十月一日,攝太尉特進平章政事兼太子太師定國公臣完顏宗憲率百官赴衍慶宮行禮。前一日,設冊寶幄次於聖武殿門外,西向。其日質明,太常寺官率所屬,於聖武殿設神御床案,宣徽院排備茶酒果、時饌、茶食、香花等,並如太祖皇帝忌辰供備之數。大樂署設登歌之樂於殿上前楹間稍南,北向。迎衛冊寶至衍慶宮門外,中書門下侍郎各奉冊寶降幣,各置於床。太尉至門外降車,率中書令以下導從,赴聖武殿門外幄次,奉
 安如式。其儀仗兵士並退。次引文武百官各服其服,以次就位。大樂令率工人就位,禮直官亦先就位。應執事者並先入殿庭北向立,禮直官贊:「再拜。」訖,升殿。次引太尉就東階下褥位西向立,禮直官贊:「拜。」在位官俱再拜。禮直官曰:「有司謹具,請行事。」禮直官贊:「拜。」在位官俱再拜,訖,引太尉詣罍洗盥手,升殿,詣神座前,搢笏,跪,三上香,樂作,奠茶、奠酒,訖,執笏,俯伏,興,樂止。太尉再拜,訖,還位少立。次引太尉出,率中書門下侍郎等,奉冊寶床入自殿門,中書令侍中等並導從,登歌樂作,冊寶床至殿庭,列於西階之下,承以席褥,樂止。太尉以下各就面北
 褥位立定,禮直官贊:「拜。」在位官俱再拜,訖,太尉率中書令侍郎奉冊匣升殿,登歌樂作,至殿上,冊匣置於食案之前,仍設褥位,樂止。次引太尉詣神位前,俯伏,跪,稱:「攝太尉臣某言,謹上加尊謚冊,寶。」奏訖,俯伏,興,稍西立。次引中書令立於冊匣南,舉冊官舉冊,中書令俯伏,跪讀冊,訖,俯伏,興。中書令奉冊匣降自西階,置于床,登歌樂作,置訖,樂止。次引侍中門下侍郎奉寶盝升殿,樂作,置于食案之前。仍設褥位,樂止。舉寶官舉寶盝,侍中俯伏,跪讀寶,訖,俯伏,興。侍中奉寶盝降自西階,置于床,登歌樂作,置訖,樂止。太尉詣殿門外褥位,再拜,訖,太尉而下
 俱降階,以次就位。禮直官贊:「拜。」在位官皆再拜,訖,以次出。寺官、署官率拱衛直,舁冊寶床置于冊寶殿,各退。次日,百官稱賀如常儀。



 大定十九年,奉上孝成皇帝謚號。元年十一月十六日,詔曰:「前君乃太祖之長孫,受太宗之遺命,嗣膺神器,十有五年。垂拱仰成,委任勳戚,廢齊國以省徭賦,柔宋人而息兵戈,世格泰和,俗躋仁壽,混車書於南北,一尉候於東西。晚雖淫刑,幾於恣意,冤施弟后,戮及良工,虐不及民,事猶可諫,過之至此,古或有焉。右丞相岐國王亮不務弼諧,反行篡弒,妄加黜廢,抑損徽稱。遠近傷嗟,神
 人憤怒,天方悔禍,朕乃繼興,受天下之樂推,居域中之有大。將撥亂而反正,務在革非。期事亡以如存,聿思盡禮。宜上謚號曰閔宗武靈皇帝。」十八年,有司言:「本朝祖宗尊謚或十八字,或十四字,或十二字,或四字。今擬增上閔宗尊謚曰弘基纘武莊靖孝成皇帝,仍加謚悼皇后曰悼平皇后。」又言:「大定三年追尊睿宗皇帝禮儀,大安殿前立黃麾仗一千人,應天門外行仗二千人,皇帝服通天冠、絳紗袍,隨冊寶降自西階,搢圭,跪,捧冊寶授太尉。今擬大安殿行禮,及依唐、周典故,降階捧冊寶授太尉。所有冠冕儀仗擬依已行禮例。」上命儀仗人數約
 量減之,餘略同前儀。明年四月十日,奉上冊寶,升祔太廟。二十六年,敕再議閔宗廟號,禮官擬上「襄、威、敬、定、桓、烈、熙」七字,奉旨用「熙」字,乃以明年四月一日,遣官奏告太廟及閔宗本室,易新廟號。



 大定二十九年四月乙丑,謚大行皇帝曰光天興運文德武功聖明仁孝皇帝,廟號世宗。五月丙午,以祔廟禮成,大赦。大定二十九年五月甲午,上皇考尊謚曰體道弘仁英文睿德光孝皇帝,廟號顯宗。大定元年二月丁卯,謚大行皇帝曰憲天光運仁文義
 武神聖英孝皇帝,廟號章宗。正大元年正月戊戌,謚大行皇帝曰繼天興統述道勤仁英武聖孝皇帝,廟號宣宗。



\end{pinyinscope}