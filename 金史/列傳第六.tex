\article{列傳第六}

\begin{pinyinscope}

 ○歡都子謀演冶訶子阿魯補骨赧訛古乃蒲查



 歡都,完顏部人。祖石魯,與昭祖同時同部同名,交相得,誓曰:「生則同川居,死則同谷葬。」土人呼昭祖為勇石魯,呼石魯為賢石魯。



 初,烏薩扎部有美女名罷敵悔,青嶺東混同江蜀束水人掠而去,生二女,長曰達回,幼曰滓
 賽。昭祖與石魯謀取之,遂偕至嶺右,炷火於箭端而射。蜀束水人怪之,皆走險阻,久之,無所復見,卻還所居。昭祖及石魯以眾至,攻取其貲產,虜二女子以歸。昭祖納其一,賢石魯納其一,皆以為妾。是時,諸部不肯用條教,昭祖耀武于青嶺、白山,入于蘇濱、耶懶之地,賢石魯佐之也。其後別去。



 至景祖時,石魯之子劾孫舉部來歸,居於安出虎水源胡凱山南。胡凱山者,所謂和陵之地是也。



 歡都,劾孫子。世祖初,襲節度使。而跋黑以屬尊,蓄異謀,不可制。諸部不肯受約束,相繼為變。歡都入與謀議,出臨戰陣,未嘗去左右。



 斡勒部人盃乃,自景祖時與其
 兄弟俱居安出虎水之北,及烏春作難,盃乃將與烏春合,間誘斡魯紺出水居人與之相結,欲先除去歡都。會其家被火,陰約隸人不歌束,詭稱放火乃歡都、胡土二人,使注都來謂世祖曰:「不歌束來告曰『前日之火,歡都等縱之』。若不棄舊好,其執縱火之人以來。」世祖疑之。石盧斡勒勃堇曰:「盃乃兄弟也,豈以一二人之故,而與兄弟構怨乎。彼自取之,又將尤誰,不如與之便。」歡都被甲執戟而起曰:「彼為亂之人也,若取太師兄弟,則亦與之乎。今取我輩,我輩決不可往,若必用戰,當盡力致死。」穆宗曰:「壯哉歡都,以我所見,正如此爾。」贈歡都以馬,曰:「戰
 則乘此。」眾皆稱善。世祖乃往見盃乃,隔鱉刺水而與之言曰:「不歌束既告縱火由歡都等,謹當如約。當先遣不歌束來。」不歌束至,世祖於馬前殺之,使杯乃見之。既而聞之,放火者盃乃家人阿出胡山也,盃乃欲開此釁,故以誣歡都云。



 臘醅、麻產與世祖遇于野鵲水。日已曛,惟從五六十騎,歡都入敵陣鏖擊之,左右出入者數四,世祖中創乃止。烏春、窩謀罕據活刺渾水,世祖既許之降,遂還軍。於是騷臘勃堇、富者撻懶歡勝負不助軍,而騷臘、撻懶先曾與臘醅、麻產合,世祖欲因軍還而遂滅之,馳馬前進。撻懶者,貞惠皇后之弟也。歡都下馬執轡而
 諫曰:「獨不念愛弟蒲陽溫與弟婦乎。」世祖感其言,遂止。蒲陽溫者,漢語云幼弟也。世祖母弟中穆宗最少,故云然。穆宗德歡都言,後以撻懶女曷羅哂妻其子谷神。太祖追麻產,歡都射中其首,遂獲之。遼人命穆宗、太祖、辭不失、歡都俱為詳穩。



 斡善、斡脫以姑里甸兵來歸,使斜缽勃堇撫定之。蒲察部故石、拔石等,誘其眾入城,陷三百餘人。歡都為都統,往治斜缽失軍之狀,盡解斜缽所將軍、大破烏春、窩謀罕於斜堆,擒故石、拔石。



 初,耶悔水納喝部撒八之弟曰阿注阿,與人爭部族官,不得直,來歸穆宗。阿注阿之甥曰三濱、曰撒達。辭不失破烏春窩
 謀罕城,獲三濱、撒達,并獲其母,以為次室,撫其二子。撒達告阿注阿必為變,不信而殺之。撒達臨刑歎曰:「後必知之。」至是,阿注阿果為變。因穆宗晨出獵,糾率七八人操兵入宅,奪據寢門,劫貞惠皇后及家人等。歡都入見阿注阿曰:「汝輩所謀之事奈何。閨門眷屬豈足劫質,徒使之驚恐耳。汝固識我,盍以我為質也。」再三言之,阿注阿從之,貞惠皇后乃得解,而質歡都。而撒改、辭不失使人告急於獵所。穆宗亦心動,罷獵。中途逢告者,日午至,阿注阿謂穆宗曰:「可使係案女直知名官僚相結,送我兄弟親屬由咸州路入遼國,庫金廄馬與我勿惜,歡都
 亦當送我至遼境,然後還。」而耍穆宗盟,穆宗皆從之。遂執歡都及阿魯太彎、阿魯不太彎等七人,以衣裾相結,與阿注阿俱行,至遼境,乃釋歡都。歡都至濟州,實黃龍府,使人馳驛要遮阿注阿黨屬,惟縱其親人使去。遂殺三濱并其母,具報於遼,乞還阿注阿,遼人流之曷堇城。其後,阿注阿懷思鄉土,亡歸,附於係案女直,因亂其官僚之室,捕之,不伏,乃見殺。



 穆宗襲位之初,諸父之子習烈、斜缽及諸兄有異言,曰:「君相之位,皆渠輩為之,奈何?」歡都曰:「汝輩若紛爭,則吾必不默默但已。」眾聞之遂帖然,自是不復有異言者。



 歡都事四君,出入四十年,征伐
 之際遇敵則先戰,廣廷大議多用其謀。世祖嘗曰:「吾有歡都,則何事不成。」肅宗時,委任冠於近僚。穆宗嗣位,凡圖遼事皆專委之。康宗以為父叔舊人,尤加敬禮,多所補益。



 康宗十一年癸巳二月,得疾,避疾於米里每水,薨,年六十三。喪歸,康宗親迓於路,送至其家,親視葬事。天會十五年,追贈儀同三司、代國公。明昌五年,贈開府儀同三司,謚曰忠敏。子谷神、謀演。谷神別有傳。



 謀演,當阿注阿之難,從歡都代為質。後與宗峻俱侍太祖,宗峻坐謀演上,上怒,命坐其下。孛堇老孛論、拔合汝、轄拔速三人爭千戶,上曰:「汝輩能如歡都父子有勞於國者乎。」乃
 命謀演為千戶,三人者皆隸焉,其眷顧如此。天輔五年十二月卒,天會十五年贈太子少傅。



 冶訶系出景祖,居神隱水完顏部,為其部勃堇。與同部人把里勃堇,斡泯水蒲察部胡都化勃堇、廝都勃堇,泰神忒保水完顏部安團勃堇,統門水溫迪痕部活里蓋勃堇,俱來歸,金之為國,自此益大。



 肅宗拒桓赧已再失利,世祖命歡都、冶訶,以本部謀克之兵助之。冶訶與歡都常在世祖左右,居則與謀議,出則泣行陣,未嘗不在其間。



 天會十五年,贈銀青光祿大夫。明昌五年,贈特進,謚忠濟,與代國公歡都、特進劾者、開府儀同三司盆
 納、儀同三司拔達,俱配享世祖廟廷。



 冶訶子阿魯補、骨赧、訛古乃、散荅。散荅子蒲查。



 阿魯補,冶訶之子。為人魁偉多智略,勇於戰。未冠從軍,下咸州、東京。遼人來取海州,從勃堇麻吉往援,道遇重敵,力戰,斬首千級。從斡魯古攻豪、懿州,以十餘騎破敵七百,進襲遼主。阿魯補徇北地,招降營帳二十四,民戶數千。時已下西京,闍母攻應州未下,退營於州北十餘里,夜遣阿魯補率兵四百伺敵,城中果出兵三千平襲,阿魯補道與之遇,斬首百餘,獲馬六十。後遼兵三萬出馬邑之境,以千兵擊之,斬其將於陣。



 天會初,宋王宗望
 討張覺於平州,聞應州有兵萬餘來援,遣阿魯補與阿里帶迎擊之,斬馘數千而還。復從其兄虞劃,率兵三千攻乾州,虞劃道病卒,代領其眾,至乾州,降其軍及營帳三十,獲印四十,與僕虺攻下義州。



 宗望伐宋,與郭藥師戰於白河。宗望命阿魯補以二謀克先登,奮戰,賞賚特異。至汴,破淮南援兵,斬其二將。大軍退次孟陽。姚平仲夜以重兵來襲,阿魯補適當其中,力戰敗之。既還,聞大名、開德合兵十餘萬來爭河。至河上,知去敵尚遠,乃以輕兵夜發,詰旦至衛縣,遇敵,斬首數千級,餘皆潰去。師次邢州,滹沱橋已焚,阿魯補先以偏師營於水上,比軍
 至而橋成。宗望嘉其功,出真定庫物賞之,為長勝軍千戶。



 及再伐宋,從宗望破敵於井陘,遂下欒城。師自大名濟河,阿魯補屯於洺州之境。時康王留相州,大名府以兵來攻我營,阿魯補乘夜以騎二百潛出其後,反擊敗之。居數日,敵復來,蘇統制以兵二萬先至,阿魯補乘其未集,以三百騎出戰,大敗其眾,生擒蘇統制,殺之。大軍既克汴京,攻洺州,敗大名救兵,遂下洺州。從撻懶文攻恩州還,洺人復叛,阿魯補先至城下,城中出兵來戰,敗之,執其守佐,,遂與蒲魯懽取信德軍。



 梁王宗弼取開德,阿魯補以步兵五千赴之。大名境內多盜,命阿魯補留屯
 其地。賊犯莘縣,聞阿魯補至,即潰去,追襲一晝夜,至館陶及之,皆俘以歸。



 從宗弼襲康王,即渡淮,阿魯補以兵四千留和州,總督江、淮間戍將,以討未附郡縣。遂攻下太平州,隳其城。廬州叛,以偏師討之,敗其騎六千,擒三校。明日復破敵二萬於慎縣,斬首五百。張永合步騎數萬來戰,阿魯補兵止二千,敵圍之,阿魯補潰圍力戰,竟敗之,追殺四十里,獲馬三百而還。再攻廬州,與迪古不敗敵萬眾於拓皋,至廬州,騎兵五百出戰,敗之,斬其二校。師還。宗弼趨陜西,道聞大名復叛,遣阿魯補經略之,獨與譯者至城下,招之,大名果降。翌日,下令民間兵器,
 悉上送官,於是吏民按堵如故。為大名開德路都統。



 齊國建,阿魯補屯兵於汴城外。天會十五年,詔廢齊國,已執劉麟,阿魯補先入汴京備變。明年,除歸德尹。割河南地與宋,入為燕京內省使。宗弼復河南,阿魯補先濟河,撫定諸郡,再為歸德尹、河南路都統。宋兵來取河南地,宗弼召阿魯補,與許州韓常、潁州大臭、陳州赤盞暉、皆會於汴,阿魯補以敵在近,獨不赴。而宋將岳飛、劉光世等,果乘間襲取許潁三州,旁郡皆響應。其兵犯歸德者,阿魯補連擊敗之,復取亳、宿等州,河南平,阿魯補功最。



 皇統五年,為行臺參知政事,授世襲猛安,兼合扎謀
 克。改元帥右監軍,婆速路統軍,歸德軍節度使,累階儀同三司。



 其在汴時,嘗取官舍材木,構私第於恩州,至是事覺,法當「議勳」「議親」。海陵嘗在軍中,惡阿魯補,詔曰:「若論勳勞,更有過於此者。況官至一品,足以酬之。國家立法,貴賤一也,豈以親貴而有異也。」遂論死。年五十五。



 阿魯補以將家子從征伐,屢立功,歷官有惠愛,得民心。及死,人皆惜之。大定三年,贈儀同三司,詔以其子為右衛將軍,襲猛安及親管謀克,賜銀五百兩、重彩二十端、絹三百匹。



 骨赧,冶訶子,善騎射,有材幹。從討桓赧散達、烏春、窩謀
 罕、留可之叛,皆有功。從太祖伐遼,骨赧從軍戰寧江州出河店,破遼主親軍,皆以力戰受賞,襲其父謀克。領秦王宗翰千戶,攻下中、西兩京。



 宗翰伐宋,圍太原未下,宗翰還西京,骨赧以右翼軍佐銀術可守太原。是時汾州、團柏、榆次、嵐、憲、潞皆有兵來援,骨赧凡四戰,皆破之。大軍團汴,骨赧引萬戶軍,屢敗其援兵。憲、潞等州復叛,骨赧引兵復取之,并收撫保德、火山而還。



 後領軍鎮夏邊,在職十二年。天會八年,授世襲猛安。天眷初,為天德軍節度使,致仕。累遷開府儀同三司,卒,年八十五。子喜哥襲猛安,加宣武將軍。



 訛古乃,冶訶子,姿質魁偉。年十四,隸秦王宗翰軍中,常領兵行前為偵候。及大軍襲遼主,訛古乃以甲騎六十,追遼招討徒山,獲之,又以七騎追獲遼公主牙不里以獻。有軍來為遼援,方臨陣,中有躍馬而出者,軍帥謂之曰:「爾能為我取此乎?」訛古乃曰:「諾。」果生擒而還,問其名,曰同瓜,蓋北部中之勇者也。



 訛古乃善馳驛,日能千里。及伐宋,屢遣將命以行。天會八年,從秦王在燕,聞余睹反於西北,秦王令訛古乃馳驛以往,訛古乃黎明走天德,及至,日未曛也。



 皇統元年,以功授寧遠大將軍,迭刺唐古部節度使。五年,授千戶。六年,遷西北路招討使。九
 年,再遷天德尹、西南路招討使。天德二年,召見。四年,遷臨洮尹,加金紫光祿大夫。卒官,年五十三。



 蒲查,自上京梅堅河徙屯天德。初為元帥府扎也,使於四方稱職,按事能得其實,領猛安。皇統間,除同知開遠軍節度使,斥候嚴整,邊境無事。正隆初,為中都路兵馬判官。是時,京畿多盜,蒲查捕得大盜四十餘人,百姓稍安。改安化軍節度副使。大定二年,領行軍萬戶,充邳州刺史、知軍事,領本州萬戶,管所屯九猛安軍,昌武軍節度使,山東副都統。撒改南征,元帥府以蒲查行副統事。入為太子少詹事,再遷開遠軍節度使,襲伯父骨赧猛
 安,歷婆速路兵馬都總管,西北路招討使,卒。



 蒲查性廉潔忠直,臨事能斷,凡被任使,無不稱云。



 贊曰:賢石魯與昭祖為友,歡都事景祖、世祖為之臣。蓋金自景祖始大,諸部君臣之分始定,故傳異姓之臣,以歡都為首。冶訶雖宗室,與歡都同功,故列敘焉。



\end{pinyinscope}