\article{列傳第三十九}

\begin{pinyinscope}

 ○承暉本名福興抹捻盡忠僕散端本名七斤耿端義李英孛術魯德裕烏古論慶壽



 承暉,字維明,本名福興。好學,淹貫經史。襲父益都尹鄭家塔割剌訛沒謀克。大定十五年,選充符寶祗候,遷筆硯直長,轉近侍局直長,調中都右警巡使。章宗為皇太孫,選充侍正。章宗即位,遷近侍局使。孝懿皇后妹夫吾
 也藍,世宗時以罪斥去,乙夜,詔開宮城門召之。承暉不奉詔,明日奏曰:「吾也藍得罪先帝,不可召。」章宗曰:「善。」未幾,遷兵部侍郎兼右補闕。



 初置九路提刑司,承暉東京咸平等路提刑副使,改同知上京留守事。御史臺奏:「承暉前為提刑,豪猾屏息。」遷臨海軍節度使。歷利涉、遼海軍,遷北京路提刑使。歷知咸平、臨潢府,為北京留守。副留守李東陽素貴,承暉自非公事,不與交一言。改知大名府,召為刑部尚書,兼知審官院。惠民司都監餘里痕都遷織染署直長,承暉駮奏曰:「痕都以蔭得官,別無才能,前為大陽渡譏察,纔八月擢惠民司都監,已為太優,
 依格兩除之後,當再入監差,今乃超授隨朝八品職任。況痕都乃平章鎰之甥,不能不涉物議。」上從承暉議,召徒單鎰深責之。改知大興府事。宦者李新喜有寵用事,借大興府妓樂。承暉拒不與,新喜慚。章宗聞而嘉之。豪民與人爭種稻水利不直,厚賂元妃兄左宣徽使李仁惠。仁惠使人屬承暉右之。承暉即杖豪民而遣之,謂其人曰:「可以此報宣徽也。」復改知大名府事。雨潦害稼,承暉決引潦水納之濠隍。



 及伐宋,遷山東路統軍使。山東盜賊起,承暉言:「捕盜不即獲,比奏報或遷官去官,請權行的決。」尚書省議:「猛安依舊收贖,謀克奏報,其餘鈐轄
 都軍巡尉先決奏聞,俟事定復舊。」從之。及罷兵,盜賊渠魁稍就招降,猶往往潛匿泰山巖穴間。按察司請發數萬人刊除林木,則盜賊無所隱矣。承暉奏曰:「泰山五岳之宗,故曰岱宗。王者受命,封禪告代,國家雖不行此事,而山亦不可赭也。齊人易動,驅之入山,必有凍餓失所之患,此誨盜,非止盜也。天下之山亦多矣,豈可盡赭哉。」議遂寢。



 是時,行限錢法。承暉上疏,略曰:「貨聚於上,怨結於下。」不報。改知興中府事。衛紹王即位,召為御史大夫,拜參知政事。駙馬都慰徒單沒烈與其父南平干政事,大為姦利,承暉面質其非。進拜尚書左丞,行省于宣德。
 參知政事承裕敗績于會河堡,承暉亦坐除名。至寧元年,起為橫海軍節度使。貞祐初,召拜尚書右丞。承暉即日入朝,妻子留滄州。滄州破,妻子皆死。紇石烈執中伏誅。進拜平章政事,兼都元帥,封鄒國公。



 中都被圍,承暉出議和事。宣宗遷汴,進拜右丞相,兼都元帥,徙封定國公,與皇太子留守中都。承暉以尚書左丞抹捻盡忠久在軍旅,知兵事,遂以赤心委盡忠,悉以兵事付之,己乃總持大綱,期於保完都城。頃之,莊獻太子去之,右副元帥蒲察七斤以其軍出降,中都危急。詔以抹捻盡忠為平章政事,兼左副元帥。三年二月,詔元帥左監軍永錫
 將中山、真定兵,元帥左都監烏古論慶壽將大名軍萬八千人、西南路步騎萬一千、河北兵一萬,御史中丞李英運糧,參知政事、大名行省孛術魯德裕調遣繼發,救中都。承暉間遣人以礬寫奏曰:「七斤既降,城中無有固志,臣雖以死守之,豈能持久。伏念一失中都,遼東、河朔皆非我有,諸軍倍道來援,猶冀有濟。」詔曰:「中都重地,廟社在焉,朕豈一日忘也。已趣諸路兵與糧俱往,卿會知之。」及詔中都官吏軍民曰:「朕欲紓民力,遂幸陪都,天未悔禍,時尚多虞,道路久梗,音問難通。汝等朝暮矢石,暴露風霜,思惟報國,靡有貳心,俟兵事之稍息,當不愆地旌
 賞。今已會合諸路兵馬救援,故茲獎諭,想宜知悉。」永錫、慶壽等軍至霸州北。三月乙亥,李英被酒,軍無紀律,大元兵攻之,英軍大敗。是時,高琪居中用事,忌承暉成功,諸將皆顧望。既而以刑部侍郎阿典宋阿為左監軍,行元帥府于清州,同知真定府事女奚烈胡論出為右都監,行元帥府于保州,戶部侍郎侯摯行尚書六部,往來應給,終無一兵至中都者。慶壽軍聞之亦潰。



 承暉與抹捻盡忠會議于尚書省。承暉約盡忠同死社稷。盡忠謀南奔,承暉怒,即起還第,亦無如盡忠何。召盡忠腹心元帥府經歷官完顏師姑至,謂曰:「始我謂平章知兵,故推
 心以權畀平章,嘗許與我俱死。今忽異議,行期且在何日,汝必知之。」師姑曰:「今日向暮且行。」曰:「汝行李辦未?」曰:「辦矣」。承暉變色曰:「社稷若何?」師姑不能對。叱下斬之。承暉起,辭謁家廟,召左右司郎中趙思文與之飲酒,謂之曰:「事勢至此,惟有一死以報國家。」作遺表付尚書省令史師安石。其表皆論國家大計,辨君子小人治亂之本,歷指當時邪正者數人,曰:「平章政事高琪,賦性陰險,報復私憾,竊弄威柄,包藏禍心,終害國家。」因引咎以不能終保都城為謝。復謂妻子死于滄州,為書以從兄子永懷為後。從容若平日,盡出財物,召家人,隨年勞多寡而
 分之,皆與從良書。舉家號泣,承暉神色泰然,方與安石舉白引滿,謂之曰:「承暉於《五經》皆經師授,謹守而力行之,不為虛文。」既被酒,取筆與安石訣,最後倒寫二字,投筆嘆曰:「遽爾謬誤,得非神志亂邪?」謂安石曰:「子行矣。」安石出門,聞哭聲,復還問之,則已仰藥薨矣。家人匆匆瘞庭中。是日暮,盡忠出奔,中都不守。貞祐三年五月二日也。師安石奉遺表奔赴行在奏之。宣宗設奠於相國寺,哭之盡哀。贈開府儀同三司、太尉、尚書令、廣平郡王,謚忠肅。詔以永懷為器物局直長。永懷子撒速為奉御。



 承暉生而貴富,居家類寒素,常置司馬光、蘇軾像於書室,
 曰:「吾師司馬而友蘇公。」平章政事完顏守貞素敬之,與為忘年交。



 扶捻盡忠,本名彖多,上京路猛安人。中大定二十八年進士第,調高陽、朝城主簿,北京、臨潢提刑司知事。御史臺舉廉能,遷順義軍節度副使。以憂去官,起復翰林修撰,同知德昌軍節度事,簽北京按察司、滑州刺史,改恩州。上言:「凡買賣軍器,乞令告給憑驗,以防盜賊私市。」尚書省議,「止聽係籍人匠貨賣,有知情售不應存留者同私造法。」從之。遷山東按察副使,坐虛奏田稼豐收請糴常平粟,詐稱宣差和糴,降虢州刺史,改乾州。



 泰和六年,
 伐宋,為元帥右監軍完顏充經歷官,坐奏報稽滯,杖五十。八年,入為吏部郎中,累遷中都、西京按察使。是時,紇石烈執中為西京留守,與盡忠爭,私意不協。盡忠陰伺執中過失,申奏。執中雖跋扈,善撫御其部曲,密於居庸、北口置腹心刺取按察司文字。及執中自紫荊關走還中都,詔盡忠為左副元帥兼西京留守。以保全西京功進官三階,賜金百兩、銀千兩、重彩百段、絹二百疋。未幾,拜尚書右丞,行省西京。貞祐初,進拜左丞。詔曰:「卿總領行省,鎮撫陪京,守御有功,人民攸賴。朕新嗣祚,念爾重臣,益勉乃力,以副朕懷。」二年五月,自西京入朝,加崇進,封
 申國公,賜玉帶、金鼎、重幣。二年,進拜都元帥,左丞如故。



 宣宗遷汴,與右丞相承暉守中都。承暉為都元帥,盡忠復為左副元帥。十月,進拜平章政事,監修國史,左副元帥如故。宣宗詔盡忠善撫颭軍,盡忠不察,殺颭軍數人。已而中都受圍,承暉以盡忠久在軍旅,付以兵事,嘗約同死社稷。及烏古論慶壽等兵潰,外援不至,中都危急,密與腹心元帥府經歷官完顏師姑謀棄中都南奔,已戒行李,期以五月二日向暮出城。是日,承暉、盡忠會議于尚書省,承暉無奈盡忠何,徑歸家,召師姑問之,知將以其夜出奔,乃先殺師姑,然後仰藥而死。是日,凡在中
 都妃嬪,聞盡忠出奔,皆束裝至通玄門。盡忠謂之曰:「我當先出,與諸妃啟途。」諸妃以為信然。盡忠乃與愛妾及所親者先出城,不復顧矣。中都遂不守。盡忠行至中山,謂所親曰:」若與諸妃偕來,我輩豈能至此!」



 盡忠至南京,宣宗釋不問棄中都事,仍以為平章政事。盡忠言:「記注之官,奏事不當回避,可令左右司官兼之。」宣宗以為然。盡忠奏應奉翰林文字完顏素蘭可為近侍局。宣宗曰:「近侍局例注本局人及宮中出身,雜以他色,恐或不和。」盡忠曰:「若給使左右,可止注本局人。既令預政,固宜慎選。」宣宗曰:「何謂預政?」盡忠曰:「中外之事得議論訪察,即
 為預政矣。」宣宗曰:「自世宗、章宗朝許察外事,非自朕始也。如請謁營私,擬除不當,臺諫不職,非近侍體察,何由知之?」盡忠乃謝罪。參政德升繼之曰:「固當慎選其人。」宣宗曰:「朕於庶官曷嘗不慎,有外似可用而實無才力者,視之若忠孝而包藏悖逆者。蒲察七斤以刺史立功,驟升顯貴,輒懷異志。蒲鮮萬奴委以遼東,乃復肆亂。知人之難如此,朕敢輕乎!眾以蒲察五斤為公幹,乃除副使。眾以斜烈為淳直,乃用為提點。若烏古論石虎,乃汝等共舉之,朕豈不盡心哉!」德升曰:「比來訪察,開決河隄,水損田禾等,復之皆不實。」上曰:「朕自今不敢問若輩,外間
 事皆不知,朕幹何事,但終日默坐聽汝等所為矣。方朕有過,汝等不諫,今乃面訐,此豈為臣之義哉!」德升亦謝罪。紇石烈執中之誅,近侍局嘗先事啟之,遂以為功,陰秉朝政。高琪託此輩以自固。及盡忠、德升面責,愈無所忌。未幾,德升罷相,盡忠下獄,自是以後,中外蔽隔,以至於亡。



 盡忠與高琪素不相能,疑宣宗頗疏己,高琪間之。其兄吾里也為許州監酒,秩滿,求調南京。盡忠與吾里也語及中都事,曰:「邇來上頗疏我,此高琪所為也。若再主兵,必不置此,胡沙虎之事,孰為為之!」吾里也曰:「然。」九月,尚書省奏:「遙授武寧軍節度副使徒單吾典告盡忠
 謀逆。」上憮然曰:「朕何負彖多,彼棄中都,凡祖宗御容及道陵諸妃皆不顧,獨與其妾偕來,此固有罪。」乃命有司鞫治,問得與兄吾里也相語事,遂並吾里也誅之。



 僕散端,本名七斤,中都路火魯虎必剌猛安人。事親孝,選充護衛,除太子僕正、滕王府長史、宿直將軍、邳州刺史、尚廄局副使、右衛將軍。章宗即位,轉左衛。章宗朝隆慶宮,護衛花狗邀駕陳言:「端叔父胡睹預弒海陵,端不宜在侍衛。」詔杖花狗六十,代撰章奏人杖五十。丁憂,起復東北路招討副使,改左副點檢,轉都點檢,歷河南、陜西統軍使,復召為都點檢。承安四年,上如薊州秋山獵,
 端射鹿誤入圍,杖之,解職。泰和三年,起為御史大夫。明年,拜尚書左丞。



 泰和六年,詔大臣議伐宋,皆曰無足慮者。左丞相宗浩、參知政事賈鉉亦曰:「狗盜鼠竊,非舉兵也。」端曰:「小寇當晝伏夜出,豈敢白日列陳,犯靈璧、入渦口、攻壽春邪?此宋人欲多方誤我,不早為之所,一旦大舉入寇,將墮其計中。」上深然之。未幾,丁母憂,起復尚書左丞。平章政事僕散揆伐宋,發兵南京,詔端行省,主留務。僕散揆已渡淮,次盧州。宋使皇甫拱奉書乞和,端奏其書。朝議諸道兵既進,疑宋以計緩師,詔端遣拱還宋。七年,僕散揆以暑雨班師,端還朝。



 初,婦人阿魯不嫁為
 武衛軍士妻,生二女而寡,常託夢中言以惑眾,頗有驗,或以為神。乃自言夢中屢見白頭老父指其二女曰:「皆有福人也。若侍掖廷,必得皇嗣。」是時,章宗在位久,皇子未立,端請納之。章宗從之。既而京師久不雨,阿魯不復言:「夢見白頭老父使己祈雨,三日必大澍足。」過三日雨不降,章宗疑其誕妄,下有司鞫問,阿魯不引伏。詔讓端曰:「昔者所奏,今其若何?後人謂朕信其妖妄,實由卿啟其端,倪鬱於予懷,念之難置。其循省于往咎,思善補于將來。恪整乃心,式副朕意!」端上表待罪,詔釋不問。頃之,進拜平章政事,封申國公。八年,宋人請盟,端遷一官。



 章
 宗遺詔:「內人有娠者兩位,生子立為儲嗣。」衛紹王即位,命端與尚書左丞孫即康護視章宗內人有娠者。泰和八年十一月二十日,章宗崩。二十二日,太醫副使儀師顏狀:「診得范氏胎氣有損。」明年四月,有人告元妃李氏教承御賈氏詐稱有身。元妃、承御皆誅死。端進拜右丞相,授世襲謀克。



 貞祐二年五月,判南京留守,與河南統軍使長壽、按察轉運使王質表請南遷,凡三奏,宣宗意乃決。百官士庶皆言其不可,太學生趙昉等四百人上書極論利害,宣宗慰遣之,乃下詔遷都。明年,中都失守。宣宗至南京,以端知開封府事。頃之,為御使大夫,無何,
 拜尚書左丞相。三年,兼樞密副使,未幾,進兼樞密使。數月,以左丞相兼都元帥行省陜西,給親軍三十人、騎兵三百為衛,次子宿直將軍納坦出侍行。賜契紙勘同曰:「緩急有事,以此召卿。」端招遙領通遠軍節度使完顏狗兒即日來歸,奏遷知平涼府事,諸將聞之,莫不感激。遣納蘭伴僧招諭臨洮祇黎五族都管青覺兒、積石州章羅謁蘭冬及鐸精族都管阿令結、蘭州葩俄族都管汪三郎等,皆相繼內附。汪三郎賜姓完顏,後為西方名將。



 四年,以疾請致仕,不許,遣近侍與太醫診視。端雖癃老,凡朝廷使至,必遠迓,宴勞不懈,故讒構不果行。宣宗聞
 之,詔自今專使酒三行別于儀門,他事經過者一見而止。初,同、華舊屯陜西軍及河南步騎九千餘人,皆隸陜州宣撫副使永錫,端奏:「潼關之西,皆陜西地,請此軍隸行省,緩急可使。」朝廷從之。及大元兵入潼關,永錫坐誅,而罪不及端。



 興定元年,朝廷以知臨洮府事承裔為元帥左都監,行元帥府於鳳翔。端奏:「隴外十州,介宋、夏之間,與諸番雜處,先於鞏州置元帥府以鎮之。今承裔以隴外萬兵移居鳳翔,臣恐一旦有警,援應不及。乞令承裔行元帥府於鞏州。若以鳳翔密邇宋界,則本路屯兵已多,但令總管攝行帥事,與京兆、鞏相為首尾,足以備
 緩急矣。」從之。是歲,薨。訃聞,宣宗震悼,輟朝。贈延安郡王,謚忠正。正大三年,配享宣宗廟廷。



 子納坦出,為定國軍節度使。天興元年十一月,納坦出之子忙押門與兄石里門及護衛顏盞宗阿同飲,忙押門詐以事出投北兵,省以刑部郎中趙楠推其家屬及同飲人。時上下迎合,必欲以知情處之,至於忙押門妻皆被訊掠。其母完顏氏曰:「忙押門通其父妾,父殺此妾,忙押門不自安,遂叛,求脫命而已。」委曲推問,無知情之狀。省中微聞之,召小吏郭從革喻以風旨,從革言之。楠方食,擲匕箸於案,大言曰:「寧使趙楠除名,亦不能屈斷無辜人。」遂以不知情
 奏,且以妾事上聞。上曰:「丞相功臣,納坦出父子俱受國恩,吾已保其不知情也。」立命赦出之。楠字才美,進士,高平人。



 耿端義,字忠嗣,博州博平人。大定二十八年進士。調滑州軍事判官,歷上洛縣令,安化、順義軍節度判官,補尚書省令史,除汾陽軍節度副使,改都轉運司戶籍判官,轉太常博士,遷太常丞兼秘書郎,再除左司員外郎,歷太常少卿兼吏部員外郎,同修國史,戶部郎中,河北東路按察副使,同知東平府事,充山東安撫使。宣宗判汾陽軍,是時端義為副使。宣宗即位,召見,訪問時事,遷翰
 林侍講學士兼戶部侍郎,未幾,拜參知政事。貞祐二年,中都被圍,將帥皆不肯戰。端義奏曰:「今日之患,衛王啟之。士卒縱不可使,城中軍官自都統至謀克不啻萬餘,遣此輩一出,或可以得志。」議竟不行。中都解圍,端義請遷南京。既而僕散端三表皆言遷都事,宣宗意遂決。是歲,薨。宣宗輟朝,賻贈甚厚,遣使祭葬。



 李英,字子賢,其先遼陽人,徙益都。中明昌五年進士第,調淳化主簿、登州軍事判官、封丘令。丁父憂,服除,調通遠令。蕃部取民物不與直,攝之不時至,即掩捕之,論如法。補尚書省令史。大安三年,集三品以上官議兵事,英
 上疏曰:「軍旅必練習者,術虎高琪、烏古孫兀屯、納蘭頭、抹捻盡忠先朝嘗任使,可與商略。餘者紛紛,恐誤大計。」又曰:「比來增築城郭,修完樓櫓,事勢可知,山東、河北不大其聲援,則京師為孤城矣。」不報。除吏部主事。



 貞祐初,攝左司都事,遷監察御史。右副元帥術虎高琪辟為經歷官,乃上書高琪曰:「中都之有居庸,猶秦之崤、函,蜀之劍門也。邇者撤居庸兵,我勢遂去。今土豪守之,朝廷當遣官節制,失此不圖,忠義之士,將轉為他矣。」又曰:「可鎮撫宣德、德興餘民,使之從戎。所在自有宿藏,足以取給,是國家不費斗糧尺帛,坐收所失之關隘也。居庸咫
 尺,都之北門,而不能衛護,英實恥之。」高琪奏其書,即除尚書工部員外郎,充宣差都提控,居庸等關隘悉隸焉。二年正月,乘夜與壯士李雄、郭仲元、郭興祖等四百九十人出城,緣西山進至佛巖寺。令李雄等下山招募軍民,旬日得萬餘人。擇眾所推服者領之,詭稱土豪,時時出戰。被創,召還。遷翰林待制,因獻十策,其大概謂:「居中土以鎮四方,委親賢以守中都,立潘屏以固關隘,集人力以防不虞,養馬力以助軍威,愛禾稼以結民心,明賞罰以勸百官,選守令以復郡縣,并州縣以省民力。」頗施行之。



 宣宗南遷,與左諫議大夫把胡魯俱為御前經歷
 官。詔曰:「扈從軍馬,朕自總之,事有利害,可因近侍局以聞。」宣宗次真定,以英為國子祭酒,充宣差提控隴右邊事。無何,召為御史中丞。英言:「兵興以來,百務皆弛,其要在于激濁揚清,獎進人材耳。近年改定四善、二十七最之法,徒為虛文。大定間,數遣使者分道考察廉能,當時號為得人。願改前日徒設之文,遵大定已試之效,庶幾人人自勵,為國家用矣。」宣宗嘉納之。



 自兵興以來,亟用官爵為賞,程陳僧敗官軍于龕谷,遣偽統制董九招西關堡都統王狗兒,狗兒立殺之。詔除通遠軍節度使,加榮祿大夫,賜姓完顏氏。英言:「名器不可以假人,上恩以
 難得為貴。比來醲於用賞,實駭聞聽。帑藏不足,惟恃爵命,今又輕之,何以使人?伏見蘭州西關堡守將王狗兒向以微勞,既蒙甄錄,頃者堅守關城,誘殺賊使,論其忠節,誠有可嘉。若官之五品,命以一州,亦無負矣。急於勸獎,遂擢節鉞,加階二品,賜以國姓,若取蘭州,又將何以待之?陜西名將項背相望,曹記僧、包長壽、東永昌、徒單醜兒、郭祿大皆其著者。狗兒藐然賤卒,一朝處眾人之右,為統領之官,恐眾望不厭,難得其死力。」宣宗以英奏示宰臣。宰臣奏:「狗兒奮發如此,賞以異恩,殆不為過。」上然其言。



 中都久圍,丞相承暉遣人以礬寫奏告急。詔元
 帥左監軍永錫、左都監烏古論慶壽將兵,英收河間清、滄義軍自清州督糧運救中都。英至大名,得兵數萬,馭眾素無紀律。貞祐三年三月十六日,英被酒,與大元兵遇于霸州北,大敗,盡失所運糧。英死,士卒殲焉。慶壽、永錫軍聞之,皆潰歸。五月,中都不守,宣宗猶加恩,贈通奉大夫,謚剛貞,官護葬事,錄用其子云。



 孛術魯德裕,本名蒲剌都,隆安路猛安人。補樞密院尚書省令史,右三部檢法、監察御史,遷少府監丞。明昌末,修北邊壕塹,立堡塞,以勞進官三階,授大理正。丁母憂,起復廣寧治中,歷順州、濱州刺史。坐前在順州市物虧
 直,遇赦,改刺沈州,累官北京路按察使、太子詹事、元帥左都監,遷左監軍兼監潢府路兵馬都總管。坐士馬物故多,及都統按帶私率官兵救護家屬,德裕蔽之,御史劾奏逮獄。遇赦,謫寧海州刺史,稍遷泗州防禦使、武勝軍節度使。貞祐二年,改知臨洮府事,兼陜西路副統軍。召為御史中丞,拜參知政事兼簽樞密院事,行省大名。詔發河北兵救中都。凡真定、中山、保、涿等兵,元帥左監軍永錫將之,大名、河間、清、滄、觀、霸、河南等兵,德裕將之,並護清、滄糧運。德裕不時發。及李英至霸州兵敗,糧盡亡失,坐弛慢兵期,責授沂州防禦使,尋知益都府事。興
 定元年二月,卒。



 烏古論慶壽,河北西路猛安人,由知把書畫充奉御,除近侍局直長,再轉本局使。禦邊有勞,進一階,賜金帶。泰和四年,遷本局提點。是時,議開通州漕河,詔慶壽按視。漕河成,賜銀一百五十兩、重幣十端。



 泰和六年,伐宋,從右副元帥完顏匡出唐鄧,為先鋒都統,賜御弓二。以騎兵八千攻下棗陽。頃之,完顏匡軍次白虎粒,遣都統完顏按帶取隨州,遣慶壽以兵五千扼赤岸,斷襄漢路。行與宋兵遇,斬首五百級,宋隨州將雷太尉遁去,遂克隨州。於是宋鄧城、樊城戍兵皆潰,遂與大軍渡漢江,圍襄
 陽。元帥匡表薦慶壽謀略出眾。上嘉之,進一官,遷拱衛直都指揮使,提點如故。



 初,慶壽上書云:「汝州襄城縣去汝州遠於許州兩舍,請割隸許州便。」尚書省議:「汝州南有鴉路舊屯四千,其三千在襄城,今割襄隸許州,道里近便,仍食用解鹽,其屯軍三千,依舊汝州總押。」從之。八年,罷兵,遷兩階,賜銀二百五十兩、重幣十端。有疾,賜御藥。衛紹王即位,改左副點檢、近侍局如故。未幾,坐與黃門李新喜題品諸王,免死除名。久之,起為保安州刺史,歷同知延安府,西北、西南招討副使,棣州防禦使,興平軍節度使。



 貞祐二年,遷元帥右都監,以保全平州功進
 官五階,賜金吐鶻、重幣十端。頃之,宣宗遷汴,改右副點檢兼侍衛親軍副都指揮使。閱月,知大興府事。未行,改左副點檢兼親軍副都指揮。數月,知彰德府事。三年,中都危急,改元帥左都監,將大名兵萬八千、西南路步騎萬一千、河北兵一萬救中都。次霸州北,兵潰。頃之,中都不守,改大名府權宣撫使。未幾,知河中府,權河東南路宣撫副使。四年,遷元帥左監軍兼陜西統軍使。駐兵延安,敗夏人于安塞堡。戰於鄜州之倉曲谷,有功。



 興定元年,與簽樞密院事完顏賽不經略伐宋,敗宋兵于泥河灣石壕村,斬首三千級,獲馬四百匹、牛三百頭,器械稱
 是。復破宋兵七千於樊城縣。既而,以軍士多被傷,奏不以實,詔有司鞫問,已而釋之。歷鎮南集慶軍節度使,卒。



 贊曰:承暉守中都期年,相為存亡,臨終就義,古人所難也。大抵宣宗既遷,則中都必不能守,中都不守,則土崩之勢決矣。僕散端、耿端義似忠而實愚,抹捻盡忠委中都,庸何議焉。高琪忌承暉成功,孛術魯德裕緩師期,姦人之黨,於是何誅。李英被酒敗軍,雖死不能贖也。烏古論慶壽無罰,貞祐之刑政,從可知矣。



\end{pinyinscope}