\article{志第三十一}

\begin{pinyinscope}

 食貨五



 ○榷場



 與敵國互市之所也。皆設場官,嚴厲禁,廣屋宇以通二國之貨,歲之所獲亦大有助於經用焉。熙宗皇統二年五月,許宋人之請,遂各置於兩界。九月,命壽州、鄧州、鳳翔府等處皆置。海陵正隆四年正月,罷鳳翔府、唐、鄧、潁、蔡、鞏、洮等州并膠西縣所置者,而專置於泗州。尋伐宋,亦罷之。五年八月,命榷場起赴南京。國初於西北
 招討司之燕子城、北羊城之間嘗置之,以易北方牧畜。世宗大定三年,市馬於夏國之榷場。四年,以尚書省奏,復置泗、壽、蔡、唐、鄧、潁、密、鳳翔、秦、鞏、洮諸場。七年,禁秦州場不得賣米面、及羊豕之臘、並可作軍器之物入外界。十七年二月,上謂宰臣曰:「宋人喜生事背盟,或與大石交通,恐枉害生靈,不可不備。其陜西沿邊榷場可止留一處,餘悉罷之。令所司嚴察姦細。」前此,以防姦細,罷西界蘭州、保安、綏德三榷場。二十一年正月,夏國王李仁孝上表乞復置,以保安、蘭州無所產,而且稅少,惟於綏德為要地,可復設互市,命省臣議之。宰臣以陜西鄰西
 夏,邊民私越境盜竊,緣有榷場,故姦人得往來,擬東勝可依舊設,陜西者並罷之。上曰:「東勝與陜西道路隔絕,貿易不通,其令環州置一場。」尋於綏德州復置一場。



 十二月,禁壽州榷場受分例。分例者,商人贄見場官之錢幣也。



 章宗明昌二年七月,尚書省以泗州榷場自前關防不嚴,遂奏定從大定五年制,官為增修舍屋,倍設闌禁,委場官及提控所拘榷,以提刑司舉察。惟東勝、凈、慶州,來遠軍者仍舊,餘皆修完之。泗州場,大定間,歲獲五萬三千四百六十七貫,承安元年,增為十萬七千八百九十三貫六百五十三文。所須雜物,泗州場歲供進新
 茶千胯、荔支五百斤、圓眼五百斤、金橘六千斤、橄欖五百斤、芭蕉乾三百個、蘇木千斤、溫柑七千箇、橘子八千箇、沙糖三百斤、生姜六百斤、梔子九十稱、犀象丹砂之類不與焉。宋亦歲得課四萬三千貫。秦州西子城場,大定間,歲獲三萬三千六百五十六貫,承安元年,歲獲十二萬二千九十九貫。承安二年,復置於保安、蘭州。三年九月,行樞密院奏:「斜出等告開榷場,擬於轄里尼要安置。」許自今年十一月貿易。尋定制,隨路榷場若以見錢入外界、與外人交易者,徒五年,三斤以上死。宋界諸場,以伐宋皆罷。泰和八年八月,以與宋和,宋人請如舊置之,
 遂復置於唐、鄧、壽、泗、息州及秦、鳳之地。



 宣宗貞祐元年,秦州榷場為宋人所焚。二年,陜西安撫副使烏古論袞州復開設之,歲所獲以十數萬計。三年七月,議欲聽榷場互市用銀,而計數稅之。上曰:「如此,是公使銀入外界也。」平章盡忠、權參知政事德升曰:「賞賜之用莫如銀絹,而府庫不足以給之。互市雖有禁,而私易者自如。若稅之,則斂不及民而用可足。」平章高琪曰:「小人敢犯,法不行爾,況許之乎?今軍未息,而產銀之地皆在外界,不禁則公私指日罄矣!」上曰:「當熟計之。」興定元年,集賢咨議官呂鑒言:「嘗監息州榷場,每場獲布數千匹,銀數百兩,
 兵興之後皆失之。」



 金銀之稅。世宗大定五年,聽人射買寶山縣銀冶。九年,御史臺奏河南府以和買金銀,抑配百姓,且下其直。上曰:「初,朕欲泉貨流通,故令行,豈可反害民乎?」遂罷之。十二年,詔金銀坑冶,恣民採,毋收稅。二十七年,尚書省奏:「聽民於農隙採銀,承納官課。」明昌二年,天下見在金千二百餘鋌,銀五十五萬二千餘鋌。三年,以提刑司言,封諸處銀冶,禁民採煉。五年,以御史臺奏,請令民採煉隨處金銀銅冶,上命尚書省議之。宰臣議謂:「國家承平日久,戶口增息,雖嘗禁之,而貧人茍求生計,聚眾私煉。上有禁之之名。而無杜絕之實,故官無
 利而民多犯法。如令民射買,則貧民壯者為夫匠,老稚供雜役,各得均齊,而射買之家亦有餘利。如此,則可以久行。比之官役顧工,糜費百端者,有間矣。」遂定制,有冶之地,委謀克縣令籍數,召募射買。禁權要、官吏、弓兵、里胥皆不得與。如舊場之例,令州府長官一員提控,提刑司訪察而禁治之。上曰:「此終非長策。」參知政事胥持國曰:「今姑聽如此,後有利然後設官可也。譬之酒酤,蓋先為坊場,從後官榷也。」上亦以為然,遂從之。墳山、西銀山之銀窟凡百一十有三。



 ○和糴



 熙宗皇統二年十月,燕、西、東京、河東、河北、山東,汴
 京等路秋熟,命有司增價和糴。世宗大定二年,以正隆之後倉廩久匱,遣太子少師完顏守道等,山東東、西路收糴軍糧,除戶口歲食外,盡令納官,給其直。三年,謂宰臣曰:「國家經費甚大,向令山東和糴,止得四十五萬餘石,未足為備。自古有水旱,所以無患者,由蓄積多也。山東軍屯處須急為二年之儲,若遇水旱則用賑濟。自餘宿兵之郡,亦須糴以足之。京師之用甚大,所須之儲,其敕戶部宜急為計。」五年,責宰臣曰:「朕謂積貯為國本,當修倉廩以廣和糴。今聞外路官文具而已。卿等不留心,甚不稱委任之意。」六年八月,敕有司,秋成之後,可於諸
 路廣糴,以備水旱。九年正月,諭宰臣曰:「朕觀宋人虛誕,恐不能久遵誓約。其令將臣謹飭邊備,以戒不虞。去歲河南豐,宜令所在廣糴,以實倉稟。詔州縣和糴,毋得抑配百姓。」十二年



 十二月,詔在都和糴以實倉廩,且使錢幣通流。又詔凡秋熟之郡,廣糴以備水旱。十六年五月,諭左丞相紇石烈良弼曰:「西邊自來不備儲蓄,其令所在和糴,以備緩急。」十七年春,尚書省奏:「先奉詔賑濟東京等路飢民,三路粟數不能給。」上曰:「朕嘗諭卿等,豐年廣糴以備凶歉。卿等皆言天下倉廩盈溢,今欲賑濟,乃云不給。自古帝王皆以蓄積為國長計,朕之積粟豈欲
 獨用。即今不給,可於鄰道取之。自今多備,當以為常。」四月,尚書省奏:「東京三路十二猛安尤闕食者,已賑之矣。尚有未賑者。」詔遣官詣復州、曷蘇館路,檢視富家,蓄積有餘增直以糴。令近地居民就往受糧。十八年四月,命泰州所管諸猛安、西北路招討司所管奚猛安,咸平府慶雲縣寔松河等處遇豐年,多和糴。



 章宗明昌四年七月,諭旨戶部官:「聞通州米粟甚賤,若以平價官糴之,何如?」於是,有司奏:「中都路去歲不熟,今其價稍減者,以商旅運販繼至故也。若即差官爭糴,切恐市價騰踴,貧民愈病,請俟秋收日,依常平倉條理收糴。」詔從之。明昌五
 年五月,上曰:「聞米價騰踴,今官運至者有餘,可減直以糶之。其明告民,不須貴價私糴也。」六年七月,敕宰臣曰:「詔制內饑謹之地令減價糶之,而貧民無錢者何以得食,其議賑濟。」省臣以為:「闕食州縣,一年則當賑貸,二年然後賑濟,如其民實無恒產者,雖應賑貸,亦請賑濟。」上遂命間隔饑荒之地,可以辨錢收糴者減價糶之,貧乏無依者賑濟。



 宣宗貞祐三年十月,命高汝礪糴於河南諸郡,令民輸挽入京,復命在京諸倉糴民輸之餘粟。侍御史黃摑奴申言:「汝礪所糴足給歲支,民既於租賦之外轉挽而來,亦已勞矣!止將其餘以為歸資,而又強取
 之,可乎?且糴此有日矣,而止得二百餘石,此何濟也。」詔罷之。十二月,附近郡縣多糴於京師,穀價騰踴,遂禁其出境。四年,河北行省侯摯言:「河北人相食,觀、滄等州斗米銀十餘兩。伏見沿河諸津許販粟北渡,然每石官糴其八,商人無利,誰肯為之。且河朔之民皆陛下赤子,既罹兵革,又坐視其死,臣恐弄兵之徒得以籍口而起也。願止其糴,縱民輸販為便。」詔從之。又制凡軍民客旅粟不於官糴處糶,而私販渡河者,杖百。沿河軍及譏察權豪家犯者,徒年、杖數並的決從重,以物沒官。上以河北州府錢多,其散失民間頗廣,命尚書省措畫之。省臣奏:「
 已命山東、河北榷酤及濱、滄鹽司,以分數帶納矣。今河北艱食,販粟北渡者眾,宜權立法以遮糴之。擬於諸渡口南岸,選通練財貨官,先以金銀絲絹等博易商販之糧,轉之北岸,以迴易糴本,兼收見錢。不惟杜姦弊,亦使錢入京師。」從之。又上封事者曰:「比年以來屢艱食,雖由調度徵斂之繁,亦兼並之家有以奪之也。收則乘賤多糴,困急則以貸人,私立券質,名為無利而實數倍。饑民惟恐不得,莫敢較者,故場功甫畢,官租未了,而囤已空矣!此富者益富,而貧者益貧者也。國朝立法,舉財物者月利不過三分,積久至倍則止,今或不期月而息三倍。
 願明敕有司,舉行舊法,豐熟之日增價和糴,則在公有益,而私無損矣。」詔宰臣行之。是年,權河東南路宣撫副使烏古論慶壽言邀糴事。見《鹽志》下。



 興定元年,上頗聞百姓以和糴太重,棄業者多,命宰臣加意焉。八月,以戶部郎中楊貞權陜西行六部尚書,收給潼、陜軍馬之用,奏糴販糧濟河者之半,以寬民。從之。



 六月,立和糴賞格。



 ○常平倉



 世宗大定十四年,嘗定制,詔中外行之,其法尋廢。章宗明昌元年八月,御史請復設,敕省臣詳議以聞。省臣言:「大定舊制,豐年則增市價十之二以糴,儉歲則減市價十之一以出,平歲則已。夫所以豐則增價以收
 者,恐物賤傷農。儉則減價以出者,恐物貴傷民。增之損之以平粟價,故謂常平,非謂使天下之民專仰給於此也。今天下生齒至眾,如欲計口使餘一年之儲,則不惟數多難辦,又慮出不以時而致腐敗也。況復有司抑配之弊,殊非經久之計。如計諸郡縣驗戶口例以月支三斗為率,每口但儲三月,已及千萬數,亦足以平物價救荒凶矣。若令諸處,自官兵三年食外,可充三月之食者免糴,其不及者俟豐年糴之,庶可久行也。然立法之始貴在必行,其令提刑司各路計司兼領之,郡縣吏沮格者糾,能推行者加擢用。若中都路年穀不熟之所,則依
 常平法,減其價三之一以糴。」詔從之。



 三年八月,敕:「常平倉豐糴儉糶,有司奉行勤惰褒罰之制,其遍諭諸路,其奉行滅裂者,提刑司糾察以聞。」又謂宰臣曰:「隨處常平倉,往往有名無實。況遠縣人戶豈肯跋涉,直就州府糶糴。可各縣置倉,命州府縣官兼提控管勾。」遂定制,縣距州六十里內就州倉,六十里外則特置。舊擬備戶口三月之糧,恐數多致損,改令戶二萬以上備三萬石,一萬以上備二萬石,一萬以下、五千以上備萬五千石,五千戶以下備五千石。河南、陜西屯軍貯糧之縣,不在是數。州縣有倉仍舊,否則創置。郡縣吏受代,所糴粟無壞,
 一月內交割給由。如無同管勾,亦準上交割。違限,委州府並提刑司差官催督監交。本處歲豐,而收糴不及一分者,本等內降,提刑司體察,直申尚書省,至日斟酌黜陟。



 九月,敕置常平倉之地,令州府官提舉之,縣官兼董其事,以所糴多寡約量升降,為永制。又諭尚書省曰:「上京路諸縣未有常平倉,如亦可置,定其當備粟數以聞。」四年十月,尚書省奏:「今上京、蒲與、速頻、曷懶、胡里改等路,猛安謀克民戶計一十七萬六千有餘,每歲收稅粟二十萬五千餘石,所支者六萬六千餘石,總其見數二百四十七萬六千餘石。臣等以為此地收多支少,遇災足
 以賑濟,似不必置。」遂止。



 五年九月,尚書省奏:「明昌三年始設常平倉,定其永制。天下常平倉總五百一十九處,見積粟三千七百八十六萬三千餘石,可備官兵五年之食,米八百一十餘萬石,可備四年之用,而見在錢總三千三百四十萬貫有奇,僅支二年以上,見錢既少,且比年稍豐而米價猶貴,若復預糴,恐價騰踴,於民未便。」遂詔權罷中外常平倉和糴,俟官錢羨餘日舉行。



 ○水田



 明昌五年閏十月,言事者謂郡縣有河者可開渠,引以溉田,詔下州郡。既而八路提刑司雖有河者皆言不可溉,惟中都言安肅、定興二縣可引河溉田四千
 餘畝,詔命行之。六年十月,定制,縣官任內有能興水利田及百頃以上者,升本等首注除。謀克所管屯田。能創增三十頃以上,賞銀絹二十兩匹,其租稅止從陸田。承安二年,敕放白蓮潭東閘水與百姓溉田。三年,又命勿毀高梁河閘,從民灌溉。泰和八年七月,詔諸路按察司規畫水田,部官謂:「水田之利甚大,沿河通作渠,如平陽掘井種田俱可灌溉。比年邳、沂近河布種豆麥,無水則鑿井灌之,計六百餘頃,比之陸田所收數倍。以此較之,它境無不可行者。」遂令轉運司因出計點,就令審察,若諸路按察司因勸農,可按問開河或掘井如何為便,規
 畫具申,以俟興作。



 貞祐四年八月,言事者程淵言:「碭山諸縣陂湖,水至則畦為稻田,水退種麥,所收倍於陸地。宜募人佃之,官取三之一,歲可得十萬石。」詔從之。興定五年五月,南陽令李國瑞創開水田四百餘頃,詔陞職二等,仍錄其最狀遍諭諸道。



 十一月,議興水田。省奏:「漢召信臣於南陽灌溉三萬頃。魏賈逵堰汝水為新陂,通運二百餘里,人謂之賈侯渠。鄧艾修淮陽、百尺二渠,通淮、潁、大治諸陂於潁之南,穿渠三百餘里,溉田二萬頃。今河南郡縣多古所開水田之地,收獲多於陸地數倍。」敕令分治戶部按行州郡,有可開者誘民赴功,其租止
 依陸田,不復添征,仍以官賞激之。陜西除三白渠設官外,亦宜視例施行。元光元年正月,遣戶部郎中楊大有等詣京東、西、南三路開水田。



 ○區田之法



 見嵇康《養生論》,自是歷代未有天下通用如趙過一畝三甽之法者。章宗明昌三年三月,宰執嘗論其法於上前,上曰:「卿等所言甚嘉,但恐農民不達此法。如其可行,當遍諭之。」四年夏四月,上與宰執復言其法,久之,參知政事胥持國曰:「今日方之大定間,戶口既多,費用亦厚。若區種之法行,良多利益。」上曰:「此法自古有之,若其可行,則何為不行也?」持國曰:「所以不行者,蓋民
 未見其利。今已令試種於城南之地,乃委官往監督之。若使民見收成之利,當不率行者自效矣。」參知政事夾谷衡以為:「若有其利,古已行矣。且用功多而所種少,復恐廢壟畝之田功也。」上曰:「姑試行之。」六月,上問參知政事胥持國曰:「區種事如何?」對曰:「六七月之交,方可見矣。」「河東及代州田種今歲佳否?」曰:「比常年頗登。」是日,命近侍二人馳驛巡視京畿禾稼。五年正月,敕諭農民使區種,先是,陳言人武陟高翌上區種法,且請驗人丁地土多少,定數令種。上令尚書省議既定,遂敕令農田百畝以上,如瀕河易得水之地,須區種三十餘畝,多種者
 聽。無水之地則從民便。仍委各千戶謀克縣官依法勸率。



 承安元年四月,初行區種法,男年十五以上、六十以下有土田者丁種一畝,丁多者五畝止。二年二月,九路提刑馬百祿奏:「聖訓農民有地一頃者區種一畝,五畝即止。臣以為地肥瘠不同,乞不限畝數。」制可。



 泰和四年九月,尚書省奏:「近奉旨講議區田,臣等謂此法本欲利民,或以天旱乃始用之,倉卒施功未必有益也。且五方地肥瘠不同,使皆可以區種,農民見有利自當勉效之。不然,督責雖嚴,亦徒勞耳。」敕遂令所在長官及按察司隨宜勸諭,亦竟不能行。



 ○
 入粟鬻度牒



 熙宗皇統三年三月,陜西旱饑,詔許富民入粟補官。世宗大定元年,以兵興歲歉,下令聽民進納補官。又募能濟饑民者,視其人數為補官格。五年,上謂宰臣曰:「頃以邊事未定,財用闕乏,自東、南兩京外,命民進納補官,及賣僧、道、尼、女冠度牒,紫、褐衣師號,寺觀名額。今邊鄙已寧,其悉罷之。慶壽寺、天長觀歲給度牒,每道折錢二十萬以賜之。」明昌二年,敕山東、河北闕食之地,納粟補官有差。承安二年,賣度牒、師號、寺觀額,復令人入粟補官。三年,西京饑,詔賣度牒以濟之。



 宣宗貞祐二年,從知大興府事胥鼎所請,定權宜鬻恩例格,進宮
 升職、丁憂人許應舉求仕、監戶從良之類,入粟草各有數。三年,制無問官民,有能勸率諸人納物入官者,米百五十石遷官一階,正班任使。七百石兩階,除諸司。千石三階,除丞簿。過此數則請於朝廷議賞。推司縣官有能勸二千石遷一階,三千石兩階,以濟軍儲。又定制,司縣官能勸率進糧至五千石以上者減一資考,萬石以上遷一官,減二等考,二萬石以上遷一官、陞一等,皆注見闕。四年,河東行省胥鼎言:「河東兵多民少,倉空歲饑。竊見潞州元帥府雖設鬻爵恩例,然條目至少,未盡勸率之術。今擬凡補買正班,依格止廕一名。若願輸許增廕
 一名。僧道已具師號者,許補買本司官。職官願納粟或不願給俸及券糧者,宜量數遷加。三舉終場人年五十以上,四舉年四十五以上,並許入粟,該恩大小官及承應人。令譯史吏員,雖未係班,亦許進納遷官。其有品官應注諸司者,聽獻物借注丞簿。丞簿注縣令,差使免一差。掌軍官能自備芻糧者,依職官例遷官如舊。」四年,耀州僧廣惠言:「軍儲不足,凡京府節鎮以上僧道官,乞令納粟百石。防刺郡副綱、威儀等,七十石者乃充,三十月滿替。諸監寺十石,周年一代,願復買者聽。」詔從之。



 興定元年,潞州行元帥府事粘割貞言:「近承奏格,凡去歲覃
 恩之官,以品從差等聽其入粟,委帥府書空宣敕授之,則人無陳訴之勞,而官有儲蓄矣。比年屢降覃恩,凡羈縻軍職者多未暇授,若止許遷新覃,則將隔越矣。乞令計前後所該輸粟積遷。」詔從之。



\end{pinyinscope}