\article{列傳第二十八}

\begin{pinyinscope}

 ○趙元移剌道本名按高德基馬諷完顏兀不喝劉徽柔賈少沖子益移剌斡里朵阿勒根彥忠張九思高衎楊邦基丁暐仁



 趙元,字善長,涿州范陽人。遼天慶八年登進士第,仕至尚書金部員外郎。遼亡,郭藥師為宋守燕,以元掌機宜
 文字。王師取燕,藥師降,樞密使劉彥宗辟元為本院令史。天會間,同知薊州事。有賊殺人橫道,官吏圜視莫知所為,路人耕夫聚觀甚眾。元指田中釋耒而來者曰:「此賊也。」叱左右縛之,遂伏。僚吏問其故,元曰:「偶得於眉睫間耳。」其後朝廷立磨勘格,凡嘗仕宣和者皆除名籍,元在磨勘中。



 齊國廢,置行臺省于汴,選名士十餘人備官屬,元在選中,授行兵部郎中。行臺徙大名,再徙祁州,及宗弼再取河南,元皆攝戶部事,賦調兵食取辦。天眷三年,為行臺右司員外郎,囚有殺人當死者,行臺欲宥之,元不從,反復數四,勢不可奪,乃仰天嘆曰:「如殺人者可
 宥,死者復何辜,何欲徼己福而亂天下法乎?」行臺竟不能奪。改左司員外郎,攝吏部事。在行臺凡十年,吏事明敏,宗弼深知之,行臺或有事上相府,宗弼必問「曾經趙元未也?」其見重如此。為同簽汴京留守事,改同知大名尹,用廉遷河北西路轉運使,歷彰德、武勝等軍節度使,以老致仕,卒于家。



 移剌道,本名按。宗室移剌古為山東東路兵馬都總管,辟掌軍府簿書,往來元帥府計議邊事,右副元帥宗弼愛其才,召為元帥府令史。補尚書省令史,特除監察御史,再遷大理丞,兼工部員外郎。海陵南伐,使督運芻糧,
 所在盜起,道路梗澀,間關僅至淮南。上謁,承問,具言四方盜賊狀,海陵惡聞其言,杖之七十,使督戰艦渡江,會海陵死,軍還。



 大定二年,除工部郎中。奉詔招撫諸奚。是時,抹白猛安下謀克徐列等皆欲降,制於猛安合住,不敢即降。道發兵掩襲合住子婦孫男女甥,及謀克留住,及蒲輦白撒妻孥。是日,適窩斡遣白撒發抹白猛安軍,白撒聞其家人被獲,遂來降。改禮部郎中。從討窩斡,佩金牌,與應奉翰林文字訛里也招降叛奚。



 奉使河南,勸課農桑,密訪吏治得失。累遷御史中丞、同修國史,廉問職官殿最,還奏。上曰:「職官貪汙罪廢,其餘因循以茍歲
 月。今廉能即與升除,無以慰百姓愛留之意,可就遷秩,秩滿升除。」於是,廉能官景州刺史耶律補進一階,單州刺史石抹靳家奴、泰寧軍節度副使尹昇卿、寧陵縣令監邦彥、濬州司候張匡福各進兩階。貪汙官同知濬州防禦使事蒲速越、真定縣令特謀葛並免死,杖一百五十,除名。同知睢州事烏古孫阿里補杖一百,削四階,非奉旨不得錄用。於是道改同知大興尹事。詔曰:「京師士民輻湊,犯法者眾,罪狀自實,毋為文所持,斷之以公可也。朕嘗諭執政矣,必不以小苛譴卿,勉副朕意。」



 遷刑部尚書。尚廄局使宗夔、副使石抹青狗私用官芻,事覺。尚
 廄局隸點檢司,刑部當自問。點檢烏林答天錫屬刑部使輕其罪,刑部以付大興府鞫治,於是道及天錫、郎中丁暐仁皆坐解職。尋起為大理卿,兼簽書樞密院事,再遷西京留守,卒。



 高德基,字元履,遼陽渤海人。皇統二年登進士第。六年,為尚書省令史。海陵為相,專愎自用,人莫敢拂其意,德基每與之詳辨。及篡位,命左司郎中賈昌祚諭旨曰:「卿公直果敢,今委卿南京行省勾當。」未行,會海陵欲都燕京,命德基攝燕京行臺省都事。改攝右司員外郎,除戶部員外郎,改中都路都轉運副使,遷戶部郎中。



 正隆三
 年,詔左丞相張浩、參知政事敬嗣暉營建南京宮室。明年,德基與御史中丞李籌、刑部侍郎蕭中一俱為營造提點。海陵使中使謂德基等曰:「汝等欲乘傳往邪?欲乘己馬往邪?銀牌可於南京尚書省取之。」籌乞先降銀牌,復遣中使謂籌曰:「牌之與否,當出朕意,爾敢輒言,豈以三人中官獨高邪?」遂杖之三十,遣乘己馬往,德基、中一乘傳往。轉同知開封尹。



 大定三年,以察廉治狀不善,下遷同知北京路都轉運使事。是年秋,土河泛濫,水入京城,德基遽命開長樂門,疏分使入御溝,以殺其勢,水不能為害。遷刑部侍郎。七年,改中都路都轉運使。九年,轉
 刑部尚書。有犯罪當死者,宰相欲從末減,德基曰:「法無二門,失出猶失入也。」不從。及奏,上曰:「刑部議,是也。」因召諸尚書諭之曰:「自朕即位以來,以政事與宰相爭是非者,德基一人而已。自今部上省三議不合,即具以聞。」為宋主生日使。及還,宋人禮物外附進臘茶三千胯,不親封署。德基曰:「侄獻叔,而不署,是無名之物也。」卻之。



 十一年,改戶部尚書。德基上疏,乞免軍須房稅等錢,減農稅及鹽酒等課,未報。隨朝官俸粟折錢,增高市價與之,多出官錢幾四十萬貫。上使人諭之曰:「卿為尚書,取悅宰執近臣,濫出官錢。卿之官爵,一出於朕,奈何如此。」於是
 決杖八十,戶部郎中王佐、員外郎盧彥沖、同知中都轉運使劉兟、副使石抹長壽、支度判官韓鎮、左警巡使李克勤、右警巡使李寶、判官強銳昌、姚宗奭、尼龐古達吉不,皆決杖有差。詔自大定十一年十一月郊祀赦後,尚書省、御史臺、戶部、轉運司、警巡院多支俸粟折錢,皆追還之。德基降蘭州刺史,王佐降大興府推官,盧彥沖河北西路戶籍判官,劉兟東京警巡使,石抹長壽東京留守推官,韓鎮河東南路戶籍判官,李克勤通遠縣令,李寶清水縣令,強銳昌、姚宗奭、尼龐古達吉不皆除司候。大定十二年,德基卒,年五十四。子錫。



 馬諷,字良弼,大興漷陰人。國初以燕與宋,諷游學汴梁,登宣和六年進士第。宗翰克汴京,諷歸朝,復登進士第,調蔚州廣靈丞,遷雄州歸信令。境有河曰八尺口,每秋潦漲溢害民田,諷視地高下,疏決之,其患遂息。召為尚書省令史,除獻州刺史。天德初,改寧州,民有告謀不軌者,株連數十百人,諷察其無狀,乃究問告者,告者具伏其誣,眾懽呼感泣。再遷南京副留守,入為大理少卿。是時,高楨為御史大夫,素貴重,繩治無所避,權貴憚其威嚴,乃以諷及張忠輔為中丞,欲有以中傷之者。諷、忠輔皆文吏巧法,不能與楨絲髮相假借,楨畏其害己,因訴
 於海陵,海陵以楨太祖舊臣,每慰安之。諷改大理卿,歲餘出為順天軍節度使。大定二年,復為大理卿,遷刑部尚書,改忠順軍節度使,致仕。卒。



 完顏兀不喝,會寧府海姑寨人。年十三,選充女直字學生。補上京女直吏,再習小字兼通契丹文字。充尚書省令史。天德初,除吏部主事,鞫問押懶路詐襲謀克事,人稱其能,擢右拾遺。海陵謂之曰:「始聞汝名,試以吏部主事。今計其實,優於所聞遠矣。」累遷右司郎中。從海陵伐宋,至淮南,聞世宗即位于遼陽,兀不喝入白其事,海陵沉思良久,曰:「卿等始聞之邪?我已知之,遣人往矣。此
 大事勿泄于外。」大定二年,秩滿當代,世宗嘉其善敷奏,特詔再任,謂宰臣曰:「兀不喝為人公忠,後來有如斯人者,卿等宜薦舉之。」其見知如此。



 窩斡已平,詔罷契丹猛安謀克,其元管戶口,及從窩斡作亂來降者,皆隸女直猛安謀克,遣兀不喝於猛安謀克人戶少處分置。未經罷去猛安謀克合承襲者,仍許承襲,賑贍其貧乏者,仍括買契丹馬匹,官員年老之馬不在括限。頃之,世宗以諸契丹未嘗為亂者與來降者一概隸女直猛安中,非是,未嘗從亂可且仍舊。平章政事完顏元宜奏,已遷契丹所棄地,可遷女直人與不從亂契丹雜處。上以問右
 丞蘇保衡、參政石琚,皆不能對。上責之曰:「卿等每事先熟議然後奏,有問即對,豈容不知此。」保衡、琚頓首謝,上曰:「分隸契丹,以本猛安租稅給贍之,所棄地與附近女直人及餘戶,願居者聽,其猛安謀克官,選契丹官員不預亂者充之。」改同知大興尹,遷橫海軍節度使。初到官,讞囚能得其情,人以為不冤。五年,卒官。



 劉徽柔,字君美,大興安次人。天眷二年,擢進士第。初為真定欒城主簿,轉開遠軍節度掌書記,遷洪洞令。徽柔明敏善聽斷。縣人楊遠者,投牒於縣,以為夜雨屋壞,壓其姪死,號訴哀切。徽柔熟視而笑曰:「汝利侄財而殺之,
 乃誣雨耶?」叱付獄,其人立伏曰:「公神明也,不敢延死。」遂置於法。秩滿,縣人遮戀不得去者彌日,為立生祠,刻石頌德。正隆二年,入為大理評事,遷司直。大定二年,同知河東南路轉運使事,以廉第一,改知平定軍,入為大理少卿。七年,知磁州,改同知南京留守事。十年,遷中都路轉運使,卒官。



 賈少沖,字若虛,通州人。勤學,日誦數百千言。家貧甚,嘗道中獲遺金,訪其主歸之。天會中,再伐宋,調及民兵。少沖甫冠,代其叔行,雖行伍間,未嘗釋卷。中天眷二年進士。劉筈欲以妹妻之,少沖辭不就曰:「富貴當自致之。」調
 營州軍事判官,遷定安令。蔚州刺史恃貴不法,屬吏畏之,每事輒曲從其意,少沖守正不阿。用廉進官一階,再遷吏部主事、定武軍節度副使、河中府判官。海陵浸以失道,少沖謂所親曰:「天下且亂,不可仕也。」秩滿,乃不求仕。大定二年,調御史臺典事,累遷刑部郎中。往北京決獄,奏誅首惡,誤牽連其中者皆釋不問,全活凡千人。以本職攝右司員外郎。嘗執奏刑名甚堅,既退,上謂侍臣曰:「少沖居下位,有守如此。」除同知河間尹。數月,入為秘書少監,兼起居注、左補闕。



 少沖外柔內剛,每從容進諫,世宗稱美之。十四年,為宋主生日副使,宋國方有祈請,
 上以意諭少沖,少沖對曰:「臣有死無辱。」宋人別致珍異,少沖笑謂其人曰:「行人受賜自有常數,寧敢以賂辱君命乎。」遂不受。使還,世宗嘉之,遷右諫議大夫,秘書、起居注如故。十七年請老,除衛州防禦使,遷河東南路轉運使,召為太常卿,兼祕書少監。復請致仕,不許,改順天軍節度使,卒。



 少沖性夷簡,不喜言利,嘗教諸子曰:「蔭所以庇身,筦庫不可為也。」聞者尚之。子益。



 益字損之,少穎悟如成人。大定十四年,父少沖為祕書少監,充宋主生日副使,益侍行。是時,宋人常爭起立接受國書之禮,少沖問益曰:「即宋人欲變禮,持議不決,奈
 何?」益曰:「守死無辱,可謂使矣。」少沖大奇之。中大定十九年進士,調河津主簿。丁父憂去官,察廉起復礬山令,補尚書省令史。丁母憂,服闋,除定海軍節度副使,監察御史,治書侍御史,轉侍御史,知登聞鼓院,兼少府少監。未幾,改禮部郎中,兼知登聞鼓院,看讀陳言文字,遷左司郎中,改吏部侍郎,兼蔡王傅。以病免。除鄭州防禦使,陜西東路轉運使,順天軍節度使。大安初,召為吏部尚書,有疾,改安國軍節度使。益調民夫修完城郭,為戰守備,按察司止之,不聽,曰:「治城,守臣事也,按察何預。」既而兵至,以有備解去。改橫海、定國軍節度使,道阻不赴。宣宗
 初為吏部尚書,益為侍郎,相得歡甚,貞祐二年至汴京,訪益所在,召為太常卿。上防秋十三事,與戶部尚書李革論遷河北軍民不便,不報。貞祐三年,致仕。元光元年,卒。



 移剌斡里朵,一名八斤,系出遼五院司,通契丹字。天會三年伐宋,隸軍中,遇戰輒先登,屢獲偵人,有司上其功,補尚書省令史。十五年,籍發諸部兵於山後,將與右丞蕭慶會,時官軍竄而南者凡數千,斡里朵以兵邀擊之,盡獲其輜重財物,悉送有司而去,一毫弗取。以勞遷修武校尉。宗弼復河南,斡里朵督諸路帥臣進討,事定以
 勞遷宣武將軍。時六部未分,乃以為兵刑二部主事。未幾,遷右司都事。皇統二年,授大理正,歷同知昭德軍節度使事,以廉升孟州防禦使。正隆間,轉同知北京留守事。會遊古河闌子山等猛安契丹謀亂,時方發兵討之,別遣斡里朵押軍南下。至松山縣,為賊黨江哥所執,且欲推為主盟,要以契約,斡里朵怒曰:「我受國厚恩,豈能從汝反耶?寧殺我,契約不可得也。」賊知不可屈,乃困辱之,使布衣草履逐馬而行,且欲害之。斡里朵說其監奴,因得脫還。六年九月,改北京路轉運使。大定初,為博州防禦使,再遷利涉軍節度使。先是,有農民避賊入保郡
 城,以錢三十千寄之鄰家,賊平索之,鄰人諱不與,訴於縣,縣官以無契驗卻之,乃訴于州。斡里朵陽怒械繫之,捕其鄰人,關以三木,詰之曰:「汝鄰乙坐劫殺人,指汝同盜。」鄰人大懼,始自陳有欺錢之隙,乃責歸所隱錢而釋之,郡人駭服。改通遠軍節度使,卒。



 阿勒根彥忠,本名窊合山,曷速館人也。好學,通吏事。天會十四年,選充尚書兵部孔目官,升尚書省令史,除右司都事。七年,改大理丞,為會寧少尹,進同知會寧府事,入為尚書吏禮部郎中。貞元二年,進本部侍郎。海陵庶人凡有所疑,常使彥忠裁決,彥忠據法以對。間有不合,
 則召讓之,彥忠執奏如前,終無阿屈,同列咸為懼,彥忠固執不變,海陵壯之。明年,除御史中丞,歷尚書戶部侍郎、侍衛親軍副都指揮使。海陵南伐,除南京路都轉運使。大定二年,改大名尹,兼本路兵馬都總管。四年,入為刑部尚書。詔規措北邊艱食戶口。及泰州、臨潢接境,度宜安置堡戍七十,駐兵萬三千,芻糧之用就經畫之。還朝未及入對,以疾卒,年五十三。



 彥忠性孝友,嘗使宋,所得金帛,盡分兄弟親友。贈榮祿大夫,命有司致祭,并以銀絹賜其家。



 張九思,字全行,錦州人。皇統初,補行臺省女直譯史,除
 同知易州事,三遷亳州防禦使、歸德尹。劉仲延受宋國歲貢於泗州,九思副之。往歲受歲貢者,每以幣物不精責宋使者,宋使者私饋銀幣各直數百千以為常,九思獨不肯受,仲延從之,自是私饋遂絕。自大理評事,再遷大理少卿。清池令雙申自陳:「父虔,天眷初,知永安軍,遇叛寇孟邦傑,執而脅之,不從,遂被害。乞正班用蔭。」大理寺議,虔子止合雜班敘,九思曰:「虔奮不顧身,守節以死,其子正班用蔭,以勸忠孝。」世宗從九思議。改工部郎中,大興少尹,同知中都都轉運使事,轉刑部侍郎,改工部。



 九思所守清約,然急於進取,一切以功利為務,率意任
 情不恤百姓。詔檢括官田,凡地名疑似者,如皇后店、太子莊、燕樂城之類,不問民田契驗,一切籍之,復有鄰接官地冒占幸免者。世宗聞其如是,召還戒之曰:「如遼時支撥地土,及國初元帥府拘刷民間指射租田,近歲冒為己業,此類當拘籍之。其餘民田,一旦奪之則百姓失業,朕意豈如此也。」轉御史中丞。九思言屯田猛安人為盜徵償,家貧輒賣所種屯地。凡家貧不能征償者,止令事主以其地招佃,收其租入,估賈與征償相當,即以其地還之。臨洮尹完顏讓亦論屯田貧人征償賣田,乞用九思議,詔從之。



 遷工部尚書。年高,愈自用,上謂左丞張
 汝弼曰:「九思耄矣,頗執強自用,欲令外補,何如?」於是,九思男若拙為尚書省令史,冒填詔敕,事覺,亡命。汝弼因奏其事,上曰:「九思豈不知若拙處邪?可免其官,捕若拙,獲日授職。」九思聞命惶懼,因感疾,卒。



 高衎,字穆仲,遼陽渤海人。敏而好學,自少有能賦聲,同舍生欲試其才,使一日賦十題戲之,衎執筆怡然,未暮,十賦皆就,彬彬然有可觀。年二十六登進士第,乞歸養,逾二年方調漷陰丞。召為尚書省令史,除右司都事。母喪去官,起復吏部員外郎,攝左司員外郎。



 王彥潛、常大榮、李慶之皆在吏部選中,吏部擬彥潛、大榮皆進士第
 一,次當在慶之上,彥潛洺州防禦判官,大榮臨海軍節度判官,慶之沈州觀察判官。左司郎中賈昌祚挾私,欲與慶之洺州,詭曰:「洺雖佳郡,防禦幕官在節鎮下。」乃改擬彥潛臨海軍,大榮沈州,慶之洺州。慶之初赴選,昌祚以慶之為會試詮讀官,而慶之弟慶雲為尚書省令史,多與權貴游,海陵心惡之,嘗謂左右司「昌祚必與慶之善闕」。大奉國臣者,遼陽人,永寧太后族人,先為東京警巡院使,以贓免去,欲因太后求見,海陵不許。衎與奉國臣有鄉里舊,擬為貴德縣令。海陵大怒,於是昌祚、衎、吏部侍郎馮仲等,各杖之有差,慶雲決杖一百五十,罷去。
 未幾,仲、昌祚、慶雲皆死,衎降為清水縣主簿,兵部員外郎攝吏部主事楊邦基降宜君縣主簿,吏部主事宋仝降漷陰縣主簿,尚書省知除楊伯傑,降閭陽縣主簿。



 居二年,為大理司直,遷戶部員外郎,同知中都都轉運使,太常少卿,吏部郎中。大定初,轉左司郎中。世宗孜孜求諫,群臣承順旨意,無所匡正,上曰:「朕初即位,庶政多未諳悉,實賴將相大臣同心輔佐。百姓且上書言事,或有所補。夫聽斷獄訟,簿書期會,何人不能,如唐、虞之聖,猶曰『稽于眾,舍己從人』。正隆專任獨見,不謀臣下,以取敗亂。卿等其體朕意。」使潔傳詔臺省百司曰:「凡上書言事,
 或為有司沮遏,許進表以聞。」



 遷吏部尚書。每季選人至,吏部託以檢閱舊籍,謂之檢卷,有滯留至後季猶不得去者。衎三為吏部,知其弊,歲餘銓事修理,選人便之。五年,為賀宋國生日使,中道得疾,去職。大定七年,卒。



 楊邦基,字德懋,華陰人。父綯,宋末為易州州佐。宗望伐宋,蔡靖以燕山降,易州即日來附,綯被殺,邦基年十餘歲,匿僧舍中,得免。既長,好學。天眷二年,登進士第,調灤州軍事判官,遷太原交城令。太原尹徒單恭貪汙不法,託名鑄金佛,命屬縣輸金,邦基獨不與。徒單恭怒,召至府,將以手持鐵拄杖撞邦基面,邦基不動。秉德廉察官
 吏,尹與九縣令皆免去,邦基以廉為河東第一,召為禮部主事。以兵部員外郎攝吏部差除,坐銓注李慶之、大奉國臣,與高衎等皆貶官,邦基降坊州宜君簿。轉高密令。大定初,尚書省擬邦基刑部郎中,世宗曰:「縣官即除郎中,如何?」太師張浩對曰:「邦基前為兵部員外郎矣,且其人材可用。」上許之。改太府少監,知登聞檢院,為祕書少監,遷翰林直學士,再遷秘書監兼左諫議大夫,修起居注。中都警巡使張子衍與邦基姻家,子衍道中遇皇太子衛仗,立馬市門不去傘,衛士訶之,子衍以鞭鞭衛士訶己者。御史臺劾奏子衍,邦基見臺官為子衍求解,
 及入見顯宗,求脫子衍罪。詔削子衍官兩階。邦基坐削官一階,出為同知西京留守事,徙山東東路轉運使,永定軍節度使,致仕。大定二十一年,卒。邦基能屬文,善畫山水人物,尤以畫名當世云。



 丁暐仁,字藏用,大興府宛平人。曾祖奭。祖惟壽。父筠,以吏補州縣,所至有治聲,其後致仕,杜門不出,鄉里有鬥訟者,不之官而就筠質焉。暐仁沖澹寡欲,讀書之外,無他好,遼季避難,雖間關道途,未嘗釋卷。皇統二年,登進士第,調武清縣丞。縣經兵革後,無學校,暐仁召邑中俊秀子弟教之學,百姓欣然從之。調磁州軍事判官。是時,
 詔使廉察官吏,暐仁以廉攝守事。遷和川令。前令罷耎不事事,群小越法干禁無所憚,暐仁申明法禁,皆屏息,或走入他縣以避之。有董祐者最強悍,畏服暐仁,以刀斷指,誓終身不復犯法。凡租賦與百姓前為期率,比他邑先辦。歷北京推官,再遷大理司直,以憂去官,尋起復。大定三年,除定武軍節度副使,而節度使、同知皆闕,暐仁為政無留訟。改大理丞,吏部員外郎,轉戶部郎中。於是,賈少沖為刑部郎中,上謂左丞相紇石烈良弼曰:「少沖為人柔緩,不稱刑部之職,其議易之。」乃以暐仁為刑部郎中。坐尚廄局官私用官芻,違格付大興府鞫問,解
 職。改祁州刺史。祁州為定武支郡,士民聞暐仁之官,相率歡迎界上,相屬不絕。改同知西京留守事,首興學校,以明養士之法。遷陜西西路轉運使。大定二十一年,卒官。



 贊曰:吏之興,其秦之季邪?吏有選試,其遼、金之際邪?其文從一從史,守法不貳之謂邪?守法不貳,斯真吏矣。巧者舞文以亂法,窒者執一而弗通,此皆吏道之自失者也。高衎、高德基、張九思之徒,皆詭法以自失者矣。



\end{pinyinscope}