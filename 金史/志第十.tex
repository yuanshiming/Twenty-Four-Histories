\article{志第十}

\begin{pinyinscope}

 禮二



 ○方丘儀



 齋戒:祭前三日質明,有司設三獻以下行事官位於尚書省。初獻南面,監祭御史位於西,東向,監禮博士位於東,西向,俱北上。司徒亞、終獻位於南,北向。次光祿卿、太常卿,次第一等分獻官、司天監,次第二等分獻官、光祿丞、郊社令、大樂令、良醖令、廩犧令、司尊彝,次內濆內外分獻官、太祝官、奉禮郎、協律郎、諸執事官,就位,
 立定。次禮直官引初獻就位,初獻讀誓曰:「今年五月幾日夏至,祭皇地祇於方丘,所有攝官,各揚其職。其或不敬,國有常刑。」讀畢,禮直官贊:「七品以下官先退。」餘官對拜,訖,退。散齋二日,宿於正寢,治理如故。齋禁並如郊祀。守濆門兵衛與大樂工人,俱清齋一宿。行禮官前期習儀於祠所。



 陳設:祭前三日,所司設三獻官以下行事執事官次於外壝東門之外,道南,北向,西上,隨地之宜。又設饌幕於內濆東門之外,道北南向。祭前二日,所司設兵衛,各服其服,守衛濆門,每門二人。大樂令帥其屬,設登歌之樂於壇上,如郊祀。郊社令帥其屬,掃除壇之
 上下,為瘞坎在內濆外之壬地。祭前一日,司天監、郊社令各服其服,帥其屬,升設皇地祇神座於壇上北方,南向,席以槁秸。又設配位神座於東方,西向,席以蒲越。又設神州地祇神座於壇之第一等東南方,席以槁秸。又設五神、五官、嶽鎮海瀆二十九座於第二等階之間,各依方位。又設崑崙、山大川澤二十一座於內濆之內,又設丘陵墳衍原隰三十座於內濆外,席皆以莞。



 又設神位版,各於座首。子陛之西,水神玄冥、北嶽、北鎮、北海、北瀆於壇之第二等,北山、北林、北川、北澤、於內濆內,北丘、北陵、北墳、北衍、北原、北隰於內濆外,皆各為一列,以東
 為上。卯陛之北,木神勾芒,東嶽、長白山、東鎮、東海、東瀆於壇之第二等,東山、東林、東川、東澤於內濆內,東丘、東陵、東墳、東衍、東原、東隰於內濆外,皆各為一列,以南為上。午陛之東,神州地祇於壇之第一等,火神祝融,南嶽、南鎮、南海、南瀆於壇之第二等,南山、南林、南川、南澤於內濆內,南丘、南陵、南墳、南衍、南原、南隰於內濆外,皆各為一列,以西為上。午陛之西,土神后土、中嶽、中鎮於壇之第二等,中山、中林、中川、中澤於內濆內,中丘、中陵、中墳、中衍、中原、中隰於內濆外,皆各為一列,以南為上。酉陛之南,金神蓐收、西嶽、西鎮、西海、西瀆於壇之第二等,
 昆崙、西山、西林、西川、西澤於內濆內,西丘、西陵、西墳、西衍、西原、西隰於內壝外,皆各為一列,以北為上。其皇地祇、及配位、神州地祇之座,并禮神之玉,設訖,俟告潔畢權徹,祭日早重設。其第二等下神座,設定不收。



 奉禮郎、禮直官又設三獻官位於卯陛之東稍北,西向。司徒位於卯陛之東,道南,西向。太常卿、光祿卿位次之。第一等分獻官、司天監位於其東,光祿丞、郊社令、太官令、廩犧令位又在其東,每等異位重行,俱西向北上。又設太祝、奉禮郎及諸執事位於內壝東門外道南,每等異位重行,俱西向北上。設監祭御史二位,一於壇下午陛之西
 南,一於子陛之西北,俱東向。設監禮博士二位,一於壇下午陛之東南,一於子陛之東北,俱西向。奉禮郎位於壇之東南,西向。協律郎位於樂虡西北,東向。大樂令位於樂虡之間,西向。司尊彝位於酌尊所,俱北向。設望瘞位坎之南,北向。又設牲榜位於內濆東門之外,西向。太祝、祝史各位於牲後,俱西向。設省饌位於牲西,太常卿、光祿卿、太官令位於牲北,南向,西上。監祭、監禮位在太常卿之西稍卻,西上。廩犧令位於牲西南,北向。又陳禮饌於內壝東門之外,道北,南向。設省饌位於禮饌之南。太常卿、光祿卿、太官令位在東,西向,監祭、監禮位在西,
 東向,俱北上。設祝版於神位之右。



 司尊及奉禮郎帥其屬,設玉幣篚於酌尊所,次及籩豆之位。正、配位各左有十一籩、右有十一豆,俱為三行。登三,在籩豆間。鉶三,在登前。簠一、簋一,各在鉶前。又設尊罍之位,皇地祇太尊二、著尊二、犧尊二、山罍二,在壇上東南隅。配位著尊二,犧尊二、象尊二、山罍二,在正位酒尊之東,俱北向西上,皆有坫,加勺、冪,為酌尊所。又設皇地祇位象尊二、壺尊二、山罍四,在壇下午陛之西,北向西上。配位犧尊二、壺尊二、山罍四,在酉陛之北,東向北上,皆有坫,加冪,設而不酌。神州地祇位左八籩、右八豆,登一在籩豆間,簠一、簋一在登
 前,爵坫一,在神座前。又設第二等諸神位每位籩二、豆二、簠一、簋一、俎一、爵坫一。內濆之內外諸神每位籩一、豆一、簠一、簋一、俎一、爵坫一。陳列皆與上同。又設神州地祇太尊二、著尊二,皆有坫。第二等諸神每方山尊二,內濆內每方蜃尊二,內濆外每方概尊二,皆加勺、冪。又設正、配位籩一、豆一、簠一、簋一、俎三、及毛血豆一、并神州地祇位俎一,各於饌冪內。又設二洗於壇下卯陛之東,北向,盥洗在東,爵洗在西,並有罍加勺。篚在洗西,南肆,實以巾。爵洗之篚實以匏爵,加坫。又設第一等分獻官盥洗爵洗位,第二等以下分獻官盥洗位,各於其方道
 之左,罍在洗左,篚在洗右,俱內向。執罍篚者各於其後。



 祭日丑前五刻,司天監、郊社令帥其屬,升設皇地祇及配位神座於壇上,設神州地祇座於第一等。又設玉幣,皇地祇玉以黃琮,神州地祇玉以兩圭有邸,皆置於匣。正、配位幣並以黃色,神州地祇幣以玄色,五神、五官、嶽鎮海瀆之幣各從其方色,皆陳於篚。太祝取瘞玉加於幣,於禮神之玉各置於神座前。光祿卿帥其屬,入實正、配位籩豆。籩三行以右為上,豆三行以左為上,其實並如郊祀。登實以大羹,鉶實以和羹。又設從祭第一等神州地祇之饌。籩三行以右為上,豆三行以左為上,其實
 並如郊祀。登實以大羹,簠實以稷,簋實以黍。第二等每位,左二籩,栗在前,鹿脯次之。右二豆,菁菹在前,鹿臡次之。簠實以稷,簋實以黍。俎,一羊、一豕。內濆內外每位,左籩一,鹿脯。右豆一,鹿臡。簠稷,簋黍,俎以羊。良醖令帥其屬,入實酒尊。皇地祇太尊為上,實以泛齊。著尊次之,實以醴齊。犧尊次之,實以盎齊。象尊次之,實以醍齊。壺尊次之,實以沈齊。山罍為下,實以三酒。配位,著尊為上,實以汎齊。犧尊次之,實以醴齊。象奠次之,實以盎齊。壺尊次之,實以醍齊,山罍為下,實以三酒。皆左實明水,右實玄酒,皆尚醖代。次實從祭第一等神州地祇酒尊,太尊
 為上,實以汎齊。著尊次之,實以醴齊。第二等,山尊實以是齊。內濆內,蜃尊實以汎齊。內濆外,概尊實以三酒。以上尊皆左以明水,右以玄酒,皆尚醖代之。太常卿設燭於神座前。



 省牲器:祭前一日午後八刻,去壇二百步禁止行者。未後二刻,郊社令帥其屬,掃除壇之上下。司尊與奉禮郎,帥執事者以祭器入,設於位。郊社令陳玉幣於篚。未後三刻,廩犧令與諸太祝、祝史,以牲就省位。禮直官、贊者分引太常卿,光祿卿、丞、監禮、祭,太官令等詣內壝東門外省牲位。其視滌濯、告潔、省牲饌,並同郊祀。俱畢,廩犧令、諸太祝、祝史以次牽牲詣廚,授太官令。
 次引光祿卿以下詣廚,省鼎鑊,視滌溉,乃還齋所。晡後一刻,太官令帥宰人以鸞刀割牲,祝史各取毛血,實以豆,置於饌幔。遂烹牲,又祝史取瘞血貯於盤。



 奠玉幣:祭日丑前五刻,獻官以下行事官,各服其服。有司設神位版,陳玉幣,實籩豆簠簋奠罍,俟監祭、監禮按視壇之上下,乃徹去蓋冪。大樂令帥工人,及奉禮郎、贊者先入。禮直官、贊者分引分獻官以下,監祭、監禮、諸大祝、祝史、齋郎與執事者,入自南濆東門,當壇南,重行,北向,西上,立定。奉禮郎贊:「拜。」獻官以下皆再拜,訖,以次分引各就壇陛上下位。次引監祭、監禮按視壇之上下,訖,退復位。
 禮直官分引三獻官以下行事官俱入就位。行禮官皆自南濆東門入。禮直官進立初獻之左,白曰:「有司謹具,請行事。」退復位。協律郎高舉笏,執麾者舉麾,俯伏,興。工鼓柷,樂作《坤寧之曲》,八成,偃麾,戛敔,樂止。俟太常卿瘞血,訖,奉禮郎贊:「拜。」在位者皆再拜。又贊:「諸執事者各就位。」禮直官引諸執事各就其位俟。太祝跪取玉幣於篚,立於尊所。諸位太祝亦各取玉幣立於奠所。



 禮直官引初獻詣盥洗位,樂作《肅寧之曲》。至位,北向立,樂止。搢笏,盥手,帨手,執笏,詣壇,樂作《肅寧之曲》。凡初獻升降,皆作《肅寧之曲》。升自卯階,至壇,樂止。詣皇地祇神座前,北向
 立,樂作《靜寧之曲》。搢笏,跪。太祝加玉於幣,西向跪以授初獻。初獻受玉幣奠訖,執笏,俯伏,興,再拜,訖,樂止。次詣配位神座前,東向立,樂作《億寧之曲》,奠幣如上儀,樂止。降自卯陛,樂作,復位,樂止。初獻將奠配位之幣,贊者引第一等分獻官詣盥洗位,搢笏,盥手,帨手,執笏,由卯陛詣神州地祇神座前,搢笏,跪。太祝以玉幣授分獻官,分獻官受玉幣,奠訖,執笏,俯伏,興,再拜,訖,退。初,第一分獻官將升,贊者引第二分獻官詣盥洗位,盥手、帨手,執笏,各由其陛升,唯不由午陛,詣於首位神座前,奠幣如上儀。餘以次祝史、齋郎助奠訖,各引還位。初獻奠幣將畢,祝
 史奉毛血豆,各由午陛升,諸太祝迎於壇上,進奠於正、配位神座前,太祝與祝史俱退,立於尊所。



 進熟:初獻既升奠玉幣。有司先陳牛鼎二、羊鼎二、豕鼎二於神廚,各在鑊右。太官帥進饌者詣廚,以匕升牛、羊、豕,自鑊實於各鼎。牛、羊、豕各肩、臂、臑、肫、胳、正脊一、橫脊一、長脅一、短脅一、代脅一,皆二骨以並,冪之。祝史以扃各對舉鼎,有司執匕以從,陳於饌幔內。從祀之俎實以羊,更陳於饌幔內。光祿卿實以籩豆簠簋。籩實以粉餈,豆實以糝食,簠實以稷,簋實以黍。實訖,去鼎之扃冪,匕加於鼎。太官令以匕升牛羊豕,載於俎,肩臂臑在上端,肫胳在下
 端,脊脅在中。俟初獻還位,樂止。禮直官引司徒出詣饌所,同薦籩豆簠簋俎。齋郎各奉皇地祇配位之饌,升自卯陛,諸太祝各迎於壇上。司徒詣皇地祇神座前,搢笏,奉籩豆簠簋,次奉俎,北向跪奠,訖,執笏,俯伏,興,設籩於糗餌之前,豆於醓醢之前,簠簋在登前,俎在籩前。次於卯陛奉配位之饌,東向跪奠於神座前,並如上儀。各降自卯陛,還位。太官令又同齋郎奉神州地祇之饌,升自卯陛,太祝迎於壇陛之道間,奠於神座前,在籩前,訖,樂止。太官令進饌者降自卯陛,還位。



 禮直官引初獻官詣盥洗位,樂作。至位,樂止。北向立,搢笏,盥手、帨手,執笏,詣
 爵洗位。至位,北向立,搢笏,洗爵,拭爵以授執事者。執笏,詣壇,樂作。升自卯陛,至壇上,樂止。詣皇地祇酌尊所,西向立。執事者以爵授初獻。初獻搢笏,執爵。司尊舉冪,良醖令跪酌太尊之汎齊,酌訖,初獻以爵授執事者,執笏,詣皇地祇神座前,北向立,搢笏,跪。執事者以爵授初獻,初獻執爵,三祭酒於茅苴,奠爵,三獻奠爵,皆執事者受以興。執笏,俯伏,興,少退,跪,樂止。舉祝官跪,對舉祝版。讀祝,太祝東向跪,讀祝訖,俯伏,興。舉祝奠版於案,再拜,興。次詣配位酌尊所,執事者以爵授初獻,初獻搢笏,執爵。司尊舉冪,良醖令跪酌著尊之汎齊,樂作太簇宮《保寧之曲》。初獻以爵授
 執事者,執笏,詣配位神座前,東向立,搢笏,跪。執事者以爵授初獻,初獻執爵,三奠酒於茅苴。奠爵,執笏,俯伏,興。少退,跪,樂止。讀祝,訖,樂作,就拜,興,拜,興。降自卯陛,讀祝、舉祝官俱從,樂作,復位,樂止。次引亞獻詣盥洗位,北向立,搢笏,盥手,帨手。執笏,詣爵洗位,北向立,搢笏,洗爵,拭爵授執事者。執笏,升自卯陛,詣皇地祇酌尊所,西向立。執事者以爵授亞獻。亞獻搢笏執爵,司尊舉冪,良醖令酌著尊之醴齊,酌訖,以爵授執事者,執笏,詣皇地祇神座前,北向立,搢笏、跪。執事以爵授亞獻,亞獻執爵,三祭酒於茅苴,奠爵,執笏,俯伏,興,少退,再拜。次詣配位酌獻如
 上儀,唯酌犧尊為異。樂止,降復位。次引終獻詣盥洗位,盥手,帨手,洗爵,拭爵,以爵授執事者,升壇。正位,酌犧尊之盎齊,配位,酌象尊之醴齊,奠獻並如亞獻之儀。禮畢,降復位。



 初,終獻將升,贊者引第一等分獻官詣盥洗位,搢笏,盥手,帨手,洗爵,拭爵,以爵授執事者。執笏,詣神州地祇酌尊所,搢笏,執事者以爵授獻官。獻官執爵,執事者酌太尊之汎齊,酌訖,以爵授執事者。進詣神座前,搢笏,跪,執事者以爵授獻官,獻官執爵,三祭酒於茅苴,奠爵,俯伏,興,少退,跪,再拜,訖,還位。初,第一等分獻官將升,贊者分引第二等分獻官詣盥洗位,搢笏,盥手,帨手,執
 笏詣酌尊所,執事以爵授分獻官,分獻酌以授執事者,進詣首位神座前,奠獻並如上儀。祝史、齋郎以次助奠,訖,各引還位。諸獻俱畢,諸太祝進徹籩豆,籩豆各一,少移故處。樂作《豐寧之曲》,卒徹、樂止。奉禮官贊曰:「賜胙。」眾官再拜,樂作,一成,止。初,送神樂止,引初獻官詣望瘞位,樂作太簇宮《肅寧之曲》,至位,南向立,樂止。初,在位官將拜,諸太祝、祝史各奉篚進詣神座前,玉幣,從祭神州地祇以下,並以俎載牲體,并取黍稷飯爵酒,各由其陛降壇,北詣瘞坎,實於坎中,又以從祭之位禮幣皆從瘞,禮直官曰:「可瘞。」東西六行,置土半坎,禮直官贊:「禮畢。」引初
 獻出,禮官贊者各引祭官及監祭、監禮、太祝以下,俱復壇南,北向立定,奉禮郎贊曰:「再拜。」監祭以下皆再拜,訖,奉禮以下及工人以次出。光祿卿以胙奉進,監祭、監禮展視。其祝版燔於齋坊。



 ○朝日夕月儀



 齋戒、陳設、省牲器、奠玉幣、進熟,其節並如大祀之儀。朝日玉用青壁,夕月用白壁,幣皆如玉之色。牲各用羊一、豕一。有司攝三獻司徒行事。其親行朝日,金初用本國禮,天會四年正月,始朝日于乾元殿,而後受賀。天眷二年,定朔望朝日儀。皇帝服靴袍,百官常服。有司設爐案、御褥位於所御殿前陛上,設百官褥位于
 殿門外,皆向日。宣徽使奏導皇帝至位,南向,再拜,上香,又再拜。閣門皆相應贊,殿門外臣僚陪拜如常儀。大定二年,以無典故罷。十五年,言事者謂今正旦并萬春節,宜令有司定拜日之禮。有司援據漢、唐春分朝日,升煙奠玉如圜丘之儀。又按唐《開元禮》,南向設大明神位,天子北向,皆無南向拜日之制。今已奉敕以月朔拜日,宜遵古制,殿前東向拜。詔姑從南向。其日,先引臣僚於殿門外立,陪位立殿前班露臺左右,皇帝於露臺香案拜如上儀。十八年,上拜日於仁政殿,始行東向之禮。皇帝出殿,東向設位,宣徽贊:「拜。」皇帝再拜,上香,訖,又再拜。臣
 僚並陪拜,依班次起居,如常儀。



 ○高禖



 明昌六年,章宗未有子,尚書省臣奏行高禖之祀,乃築壇于景風門外東南端,當闕之卯辰地,與圜丘東西相望,壇如北郊之制。歲以春分日祀青帝、伏羲氏、女媧氏,凡三位,壇上南向,西上。姜嫄、簡狄位於壇之第二層,東向,北上。前一日未三刻,布神位,省牲器,陳御弓矢弓蜀於上下神位之右。其齋戒、奠玉幣、進熟,皆如大祀儀。青帝幣玉皆用青,餘皆無玉。每位牲用羊一、豕一。有司攝三獻司徒行事。禮畢,進胙,倍於他祀之肉。進胙官佩弓矢蜀以進,上命后妃嬪御皆執弓矢東向而射,
 乃命以次飲福享胙。



\end{pinyinscope}