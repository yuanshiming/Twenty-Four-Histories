\article{列傳第六十七}

\begin{pinyinscope}

 酷吏



 ○高閭山蒲察合住



 太史公有言:「法家嚴而少恩。」信哉斯言也。金法嚴密,律文雖因前代而增損之,大抵多準重典。熙宗迭興大獄,海陵翦滅宗室,鉤棘傅會,告姦上變者賞以不次。於是中外風俗一變,咸尚威虐以為事功,而讒賊作焉。流毒遠邇,慘矣。金史多闕逸,據其舊錄得二人焉。作《酷吏傳》。



 高閭山,澄州析木人。選充護衛,調順義軍節度副使,轉唐括、移剌都颭詳穩,改震武軍節度副使、曹王府尉,大名治中。遷汝州刺史,改單州。制禁不依法用杖決人者,閭山見之笑曰:「此亦難行。」是日,特用大杖杖死部民楊仙,坐削一官,解職。久之,降鳳翔治中,歷原州、濟州、泗州刺史,改鄭州防禦使,遷蒲與路節度使,移臨海軍、盤安軍、寧昌軍。貞祐二年,城破死之。



 蒲察合住,以吏起身,久為宣宗所信,聲勢烜赫,性復殘刻,人知其蠹國而莫敢言。其子充護衛,先逐出之。繼而合住為恒州刺史,需次近縣。後大兵入陜西,關中震動。
 或言合住赴恒州為北走計,朝廷命開封羈其親屬,合住出怨言曰:「殺卻我即太平矣。」尋為御史所劾,初議笞贖,宰相以為悖理,斬於開封府門之下。故當時有宣朝三賊之目,謂王阿里、蒲察咬住,合住其一也。



 興定中,駙馬僕散阿海之獄,京師宣勘七十餘所,阿里輩乘時起事以肆其毒,朝士惴惴莫克自保,惟獨吉文之在開封府幕,明其不反,竟不署字,阿海誅,文之亦無所問。



 咬住,正大初致仕,居睢陽,潰軍變,與其家皆被殺。



 初,宣宗喜刑罰,朝士往往被笞楚,至用刀杖決殺言者。高琪用事,威刑自恣。南渡之後,習以成風,雖士大夫亦為所移,如
 徒單右丞思忠好用麻椎擊人,號「麻椎相公」。李運使特立號「半截劍」,言其短小鋒利也。馮內翰璧號「馮劊」。雷淵為御史,至蔡州得姦豪,杖殺五百人,號曰「雷半千」。又有完顏麻斤出,皆以酷聞,而合住、王阿里、李渙之徒,胥吏中尤狡刻者也。



 ◎佞幸



 蕭肄張仲軻李通馬欽高懷貞蕭裕胥持國



 世之有嗜慾者,何嘗不被其害哉。龍,天下之至神也,一
 有嗜慾,見制於人,故人君亦然。嗜慾不獨柔曼之傾意也,征伐、畋獵、土木、神仙,彼為佞者皆有以投其所好焉。金主內蠱聲色,外好大喜功,莫甚於熙宗、海陵,而章宗次之。《金史》自蕭肄至胥持國,得佞臣之尤者七人,皆被寵遇於三君之朝,以亡其身,以蠹其國,其禍皆始於此,可不戒哉。作《佞幸傳》。



 蕭肄,本奚人,有寵於熙宗,復諂事悼后,累官參知政事。皇統九年四月壬申夜,大風雨,雷電震壞寢殿鴟尾,有火自外入,燒內寢幃幔。帝徙別殿避之,欲下詔罪己。翰林學士張鈞視草。鈞意欲奉答天戒,當深自貶損,其文
 有曰:「惟德弗類,上干天威」及「顧茲寡昧眇予小子」等語。肄譯奏曰:「弗類是大無道,寡者孤獨無親,昧則於人事弗曉,眇則目無所見,小子嬰孩之稱,此漢人託文字以詈主上也。」帝大怒,命衛士拽鈞下殿,榜之數百,不死。以手劍剺其口而醢之。賜肄通天犀帶。憑恃恩倖,倨視同列,遂與海陵有惡。及篡立,加大臣官爵,例加銀青光祿大夫。數日,召肄詰之曰:「學士張鈞何罪被誅,爾何功受賞?」肄不能對。海陵曰:「朕殺汝無難事,人或以我報私怨也。」於是,詔除名,放歸田里,禁錮不得出百里外。



 張仲軻,幼名牛兒,市井無賴,說傳奇小說,雜以俳優詼
 諧語為業。海陵引之左右,以資戲笑。海陵封岐國王,以為書表,及即位,為秘書郎。海陵嘗對仲軻與妃嬪褻瀆,仲軻但稱死罪,不敢仰視。又嘗令仲軻惈形以觀之,侍臣往往令惈褫,雖徒單貞亦不免此。兵部侍郎完顏普連、大興少尹李惇皆以贓敗,海陵置之要近。伶人於慶兒官五品、大氏家奴王之彰為秘書郎。之彰睪珠偏僻,海陵親視之,不以為褻。唐括辯家奴和尚、烏帶家奴葛溫、葛魯,皆置宿衛,有僥倖至一品者。左右或無官職人,或以名呼之,即授以顯階,海陵語其人曰:「爾復能名之乎?」常置黃金茵褥間,喜之者令自取之,其濫賜如此。宋
 餘唐弼賀登寶位,且還,海陵以玉帶附賜宋帝,使謂宋帝曰:「此帶卿父所常服,今以為賜,使卿如見而父,當不忘朕意也。」使退,仲軻曰:「此希世之寶,可惜輕賜。」上曰:「江南之地,他日當為我有,此置之外府耳。」由是知海陵有南伐之意。



 俄遷秘書丞,轉少監。是時,營建燕京宮室,有司取真定府潭園材木,仲軻乘間言其中材木不可用,海陵意仲軻受請託,免仲軻官。未幾,復用為少監。海陵獵于途你山,次于鐸瓦,酹天而拜,謂群臣曰:「朕幼時習射,至一門下,默祝曰:『若我異日大貴,當使一矢橫加門脊上。』及射,果橫加門脊上。後為中京留守,嘗大獵于此
 地,圍未合,禱曰:『我若有大位,百步之內當獲三鹿。若止為公相,獲一而已。』於是不及百步連獲三鹿。又祝曰:『若統一海內,當復獲一大鹿。』於是果獲一大鹿。此事嘗與蕭裕言之,朕今復至此地,故拜奠焉。」海陵意欲取江南,故先設禨祥以諷群臣,是以仲軻每先逢其意,導之南伐。



 貞元二年正月,宋賀正旦使施巨朝辭,海陵使左宣徽使敬嗣暉問施巨曰:「宋國幾科取士?」對曰:「詩賦、經義、策論兼行。」又問:「秦檜作何官,年今幾何?」對曰:「檜為尚書左僕射中書門下平章事,年六十五矣。」復謂之曰:「我聞秦檜賢,故問之。」



 正隆二年,仲軻為左諫議大夫,修起居
 注,但食諫議俸,不得言事。三年正月,宋賀正使孫道夫陛辭,海陵使左宣徽使敬嗣暉諭之曰:「歸白爾帝,事我上國多有不誠,今略舉二事:爾民有逃入我境者,邊吏皆即發還,我民有逃叛入爾境者,有司索之往往託辭不發,一也。爾於沿邊盜買鞍馬,備戰陣,二也。且馬待人而後可用,如無其人,得馬百萬亦奚以為?我亦豈能無備。且我不取爾國則已,如欲取之,固非難事。我聞接納叛亡、盜買鞍馬,皆爾國楊太尉所為,常因俘獲問知其人,無能為者也。」又曰:「聞秦檜已死,果否?」道夫對曰:「檜實死矣,陪臣亦檜所薦用者。」又曰:「爾國比來行事,殊不似
 秦檜時何也?」道夫曰:「容陪臣還國,一一具聞宋帝。」海陵蓋欲南伐,故先設納叛亡、盜買馬二事,而雜以他辭言之。



 海陵召仲軻、右補闕馬欽、校書郎田與信、直長習失入便殿侍坐。海陵與仲軻論《漢書》,謂仲軻曰:「漢之封疆不過七八千里,今吾國幅員萬里,可謂大矣。」仲軻曰:「本朝疆土雖大,而天下有四主,南有宋,東有高麗,西有夏,若能一之,乃為大耳。」海陵曰:「彼且何罪而伐之?」仲軻曰:「臣聞宋人買馬修器械,招納山東叛亡,豈得為無罪?」海陵喜曰:「向者梁珫嘗為朕言,宋有劉貴妃者姿質艷美,蜀之華蕊、吳之西施所不及也。今一舉而兩得之,俗所
 謂『因行掉手』也。江南聞我舉兵,必遠竄耳。」欽與與信俱對曰:「海島、蠻越,臣等皆知道路,彼將安往?」欽又曰:「臣在宋時,嘗帥軍征蠻,所以知也。」海陵謂習失曰:「汝敢戰乎?」對曰:「受恩日久,死亦何避。」海陵曰:「汝料彼敢出兵否,彼若出兵,汝果能死敵乎?」習失良久曰:「臣雖懦弱,亦將與之為敵矣。」海陵曰:「彼將出兵何地?」曰:「不過淮上耳。」海陵曰:「然則天與我也。」既而曰:「朕舉兵滅宋,遠不過二三年,然後討平高麗、夏國。一統之後,論功遷秩,分賞將士,彼必忘勞矣。」



 四年三月,仲軻死。冬至前一夕,海陵夢仲軻求酒,既覺,嗟悼良久,遣使者奠其墓。



 李通,以便辟側媚得幸於海陵。累官右司郎中,遷吏部尚書。請謁賄賂輻輳其門。正隆二年正月乙酉,詔左右司御史中丞以下奏事便殿,海陵曰:「知子莫若父,知臣莫若君,朕嘗試之矣。朕詢及人材,汝等若不舉同類,必舉其相善者。朕聞女直、契丹之仕進者,必賴刑部尚書烏帶、簽書樞密遙設為之先容,左司員外郎阿里骨列任其事。渤海、漢人仕進者,必賴吏部尚書李通、戶部尚書許霖為之先容,左司郎中王蔚任其事。凡在仕版,朕識者寡,不識者眾,莫非人臣,豈有遠近親疏之異哉。茍奉職無愆,尚書侍郎節度使便可得,萬一獲罪,必罰無
 赦。」頃之,拜參知政事。



 海陵恃累世彊盛,欲大肆征伐,以一天下,嘗曰:「天下一家,然後可以為正統。」通揣知其意,遂與張仲軻、馬欽、宦者梁珫近習群小輩,盛談江南富庶,子女玉帛之多,逢其意而先道之。海陵信其言,以通為謀主,遂議興兵伐江南。四年二月,海陵諭宰相曰「宋國雖臣服,有誓約而無誠實,比聞沿邊買馬及招納叛亡,不可不備。」遣使籍諸路猛安部族、及州縣渤海丁莊充軍,仍括諸道民馬。於是,遣使分往上京、速頻路、胡里改路、曷懶路、蒲與路、泰州、咸平府、東京、婆速路、曷蘇館、臨潢府、西南招討司、西北招討司、北京、河間府、真定府、
 益都府、東平府、大名府、西京路,凡年二十以上、五十以下者皆籍之,雖親老丁多,求一子留侍,亦不聽,五年十一月,使益都尹京等三十一人押諸路軍器於軍行要會處安置,俟軍至分給之。其分給之餘與繕完不及者,皆聚而焚之。



 六年正月,海陵使通諭旨宋使徐度等曰:「朕昔從梁王嘗居南京,樂其風土。帝王巡狩,自古有之。淮右多隙地,欲校獵其間,從兵不踰萬人。汝等歸告汝主,令有司宣諭朕意,使淮南之民無懷疑懼。」二月,通進拜右丞,詔曰:「卿典領繕完兵械,今已畢功,朕嘉卿忠謹,故有是命,俟江南事畢,別當旌賞。」



 四月,簽書樞密院事
 高景山為賜宋帝生日使,右司員外郎王全副之,海陵謂全曰:「汝見宋主,即面數其焚南京宮室、沿邊買馬、招致叛亡之罪,當令大臣某人某人來此,朕將親詰問之,且索漢、淮之地,如不從,即厲聲詆責之,彼必不敢害汝。」海陵蓋使王全激怒宋主,將以為南伐之名也。謂景山曰:「回日,以全所言奏聞。」全至宋,一如海陵之言詆責宋主,宋主謂全曰:「聞公北方名家,何乃如是?」全復曰:「趙桓今已死矣。」宋主遽起發哀而罷。海陵至南京,宋遣使賀遷都,海陵使韓汝嘉就境上止之曰:「朕始至此,比聞北方小警,欲復歸中都,無庸來賀。」宋使乃還。



 於是,大括天
 下騾馬,官至七品聽留一馬,等而上之。并舊籍民馬,其在東者給西軍,在西者給東軍,東西交相往來,晝夜絡繹不絕,死者狼籍于道。其亡失多者,官吏懼罪或自殺。所過蹂踐民田,調發牽馬夫役。詔河南州縣所貯糧米以備大軍,不得他用,而騾馬所至當給芻粟,無可給,有司以為請,海陵曰:「此方比歲民間儲畜尚多,今禾稼滿野,騾馬可就牧田中,借令再歲不獲,亦何傷乎。」及徵發諸道工匠至京師,疫死者不可勝數,天下始騷然矣。調諸路馬以戶口為率,富室有至六十匹者。凡調馬五十六萬餘匹,仍令本家養飼,以俟師期。



 海陵因出獵,遂至
 通州觀造戰船,籍諸路水手得三萬餘人。及東海縣人張旺、徐元反,遣都水監徐文等率師浮海討之,海陵曰:「朕意不在一邑,將試舟師耳。」於是民不堪命,盜賊蜂起,大者連城邑,小者保山澤,遣護衛普連二十四人,各授甲士五十人,分往山東、河北、河東、中都等路節鎮州郡屯駐,捕捉盜賊。以護衛頑犀為定武軍節度副使,尚賢為安武軍節度副使,蒲甲為昭義軍節度副使,皆給銀牌,使督責之。是時,山東賊犯沂州,臨沂令胡撒力戰而死。大名府賊王九等據城叛,眾至數萬。契丹邊六斤、王三輩皆以十數騎張旗幟,白晝公行,官軍不敢誰何,所
 過州縣,開劫府庫物置于市,令人攘取之,小人皆喜賊至,而良民不勝其害。太府監高彥福、大理正耶律道、翰林待制大穎出使還朝,皆言盜賊事。海陵惡聞,怒而杖之,穎仍除名,自是人人不復敢言。



 海陵自將,分諸道兵為神策、神威、神捷、神銳、神毅、神翼、神勇、神果、神略、神鋒、武勝、武定、武威、武安、武捷、武平、武成、武毅、武銳、武揚、武翼、武震、威定、威信、威勝、威捷、威烈、威毅、威震、威略、威果、威勇三十二軍,置都總管、副總管各一員,分隸左右領軍大都督及三道都統制府。置諸軍巡察使、副各一員。以太保奔睹為左領軍大提督,通為副大都督。海陵以
 奔睹舊將,使帥諸軍以從人望,實使通專其事。



 海陵召諸將授方略,賜宴於尚書省。海陵曰:「太師梁王連年南伐,淹延歲月。今舉兵必不如彼,遠則百日,近止旬月。惟爾將士無以征行為勞,戮力一心,以成大功,當厚加旌賞,其或弛慢,刑茲無赦。」海陵恐糧運不繼,命諸軍渡江無以僮僕從行,聞者莫不怨咨。徒單后與太子光英居守,尚書令張浩、左丞相蕭玉、參知政事敬嗣暉留治省事。



 九月甲午,海陵戎服乘馬,具裝啟行。明日,妃嬪皆行,宮中慟哭久之。十月乙巳,陰晦失路,是夜二更始至蒙城。丁未,大軍渡淮,至中流,海陵拜而酹之。至宿次,見築
 繚垣者,殺四方館使張永鈐。將至廬州,見白兔,馳射不中。既而後軍獲之以進,海陵大喜,以金帛賜之,顧謂李通曰:「昔武王伐紂,白魚躍於舟中。今朕獲此,亦吉兆也。」癸亥,海陵至和州,百官表奉起居,海陵謂其使:「汝等欲伺我動靜邪?自今勿復來,俟平江南始進賀表。」



 是時,梁山濼水涸,先造戰船不得進,乃命通更造戰船,督責苛急,將士七八日夜不得休息,壞城中民居以為材木,煮死人膏為油用之。遂築臺於江上,海陵被金甲登臺,殺黑馬以祭天,以一羊一豕投於江中。召都督昂、副都督蒲盧渾謂之曰:「舟楫已具,可以濟江矣。」蒲盧渾曰:「臣觀
 宋舟甚大,我舟小而行遲,恐不可濟。」海陵怒曰:「爾昔從梁王追趙構入海島,豈皆大舟邪?明日汝與昂先濟。」昂聞令己渡江,悲懼欲亡去。至暮,海陵使謂昂曰:「前言一時之怒耳,不須先渡江也。」明日,遣武平軍都總管阿鄰、武捷軍副總管阿撒率舟師先濟。宿直將軍溫都奧剌、國子司業馬欽、武庫直長習失皆從戰。海陵置黃旗紅旗於岸上,以號令進止,紅旗立則進,黃旗仆則退。既渡江,兩舟先逼南岸,水淺不得進,與宋兵相對射者良久,兩舟中矢盡,遂為所獲,亡一猛安、軍士百餘人。海陵遂還和州。



 於是尚書省使右司郎中吾補可、員外郎王全
 奏報:世宗即位於東京,改元大定。海陵前此已遣護衛謀良虎、特離補往東京,欲害世宗。行至遼水,遇世宗詔使撒八,執而殺之,遂還軍中。海陵拊髀嘆曰:「朕本欲平江南改元大定,此豈非天乎!」乃出素所書取一戎衣天下大定改元事,以示群臣。遂召諸將帥謀北歸,且分兵渡江。



 議定,通復入奏曰:「陛下親師深入異境,無功而還,若眾散於前,敵乘於後,非萬全計。若留兵渡江,車駕北還,諸將亦將解體。今燕北諸軍近遼陽者恐有異志,宜先發兵渡江,斂舟焚之,絕其歸望。然後陛下北還,南北皆指日而定矣。」海陵然之,明日遂趨揚州。過烏江縣,觀
 項羽祠,嘆曰:「如此英雄不得天下,誠可惜也。」



 海陵至揚州,使符寶耶律沒答護神果軍扼淮渡,凡自軍中還至淮上,無都督府文字皆殺之。乃出內箭飾以金龍,題曰御箭,繫帛書其上,使人乘舟射之南岸,其書言「宋國遣人焚毀南京宮室、及沿邊買馬、招誘軍民,今興師問罪,義在弔伐,大軍所至,必無秋毫之犯。」以此招諭宋人。於是,宋將王權亦縱所獲金軍士三人,齎書數海陵罪,通奏其書,即命焚之。



 海陵怒,亟欲渡江。驍騎高僧欲誘其黨以亡,事覺,命眾刃剉之。乃下令,軍士亡者殺其蒲里衍,蒲里衍亡者殺其謀克,謀克亡者殺其猛安,猛安亡
 者殺其總管,由是軍士益危懼。甲午,令軍中運鴉鶻船及糧船於瓜州渡,期以明日渡江,敢後者死。



 乙未,完顏元宜等以兵犯御營,海陵遇弒。都督府以南伐之計皆通等贊成之,徒單永年乃其姻戚,郭安國眾所共惡,皆殺之。大定二年,詔削通官爵,人心始快。



 馬欽,幼名韓哥,嘗仕江南,故能知江南道路。正隆三年,海陵將南伐,遂召用欽,自貴德縣令為右補闕。欽為人輕脫不識大體,海陵每召見與語,欽出宮輒以語人曰:「上與我論某事,將行之矣。」其視海陵如僚友然。累遷國子司業。海陵至和州,欲遣蒲盧渾渡江,蒲盧渾言舟小
 不可濟,海陵使人召欽,先戒左右曰:「欽若言舟小不可渡江,即殺之。」欽至,問曰:「此舟可渡江否?」欽曰:「臣得筏亦可渡也。」大定二年,除名。是日,起前翰林待制大穎為秘書丞。穎在正隆間嘗言山東盜賊,海陵惡其言,杖之除名。世宗嘉穎忠直,惡欽巧佞,故復用穎而放欽焉。



 高懷貞,為尚書省令史,素與海陵狎暱。海陵久蓄不臣之心,嘗與懷貞各言所志,海陵曰:「吾志有三:國家大事皆自我出,一也。帥師伐國,執其君長問罪於前,二也。得天下絕色而妻之,三也。」由是小人佞夫皆知其志,爭進諛說。大定縣丞張忠輔謂海陵曰:「夢公與帝擊球,公乘
 馬衝過之,帝墜馬下。」海陵聞之大喜。會熙宗在位久,委政大臣,海陵以近屬為宰相,專威福柄,遂成弒逆之計,皆懷貞輩小人從臾導之。海陵篡立,以懷貞為修起居注,懷貞故父濱州刺史贈中奉大夫。懷貞累遷禮部侍郎。大定二年,降奉政大夫,放歸田里。五年,與許霖俱賜起復,懷貞為定國軍節度使。上戒之曰:「汝等在正隆時,姦佞貪私,物論鄙之。朕念沒身不齒則無以自新。若怙舊不悛,必不貸汝矣。」



 蕭裕,本名遙折,奚人。初以猛安居中京,海陵為中京留守,與裕相結,每與論天下事。裕揣海陵有覬覦心,密謂
 海陵曰:「留守先太師,太祖長子。德望如此,人心天意宜有所屬,誠有志舉大事,顧竭力以從。」海陵喜受之,遂與謀議。海陵竟成弒逆之謀者,裕啟之也。



 海陵為左丞,除裕兵部侍郎,改同知南京留守事,改北京。海陵領行臺尚書省事,道過北京,謂裕曰:「我欲就河南兵建立位號,先定兩河,舉兵而北。君為我結諸猛安以應我。」定約而去。海陵雖自良鄉召還,不能如約,遂弒熙宗篡立,以裕為秘書監。



 海陵心忌太宗諸子,欲除之,與裕密謀。裕傾險巧詐,因構致太傅宗本、秉德等反狀,海陵殺宗本,唐括辯遣使殺秉德、宗懿及太宗子孫七十餘人、秦王宗
 翰子孫三十餘人。宗本已死,裕乃求宗本門客蕭玉,教以具款反狀,令作主名上變。海陵既詔天下,天下冤之。海陵賞誅宗本功,以裕為尚書左丞,加儀同三司,授猛安,賜錢二千萬,馬四百匹、牛四百頭、羊四千口。再閱月,為平章政事、監修國史。舊制,首相監修國史,海陵以命裕,謂裕曰:「太祖以神武受命,豐功茂烈光於四海,恐史官有遺逸,故以命卿。」久之,裕為右丞相、兼中書令。裕在相位,任職用事頗專恣,威福在己,勢傾朝廷。海陵倚信之,他相仰成而已。



 裕與高藥師善,嘗以海陵密語告藥師,藥師以其言奏海陵,且曰:「裕有怨望心。」海陵召裕戒
 諭之,而不以為罪也。或有言裕擅權者,海陵以為忌裕者眾,不之信。又以為人見裕弟蕭祚為左副點檢,妹夫耶律闢離剌為左衛將軍,勢位相憑藉,遂生忌嫉,乃出祚為益都尹,闢離剌為寧昌軍節度使,以絕眾疑。裕不知海陵意,遽見出其親表補外,不令己知之,自是深念恐海陵疑己。海陵弟太師袞領三省事,共在相位,以裕多自用,頗防閑之,裕乃謂海陵使袞備之也。而海陵猜忍嗜殺,裕恐及禍,遂與前真定尹蕭馮家奴、前御史中丞蕭招折、博州同知遙設、裕女夫遏剌補謀立亡遼豫王延禧之孫。裕使親信蕭屯納往結西北路招討使蕭
 好胡,好胡即懷忠。懷忠依違未決,謂屯納曰:「此大事,汝歸遣一重人來。」裕乃使招折往。招折前為中丞,以罪免,以此得詣懷忠。懷忠問招折與謀者復有何人,招折曰:「五院節度使耶律朗亦是也。」懷忠舊與朗有隙,而招折嘗上撻懶變事,懷忠疑招折反復,因執招折,收朗繫獄,遣使上變。遙設亦與筆硯令史白答書,使白答助裕以取富貴,白答奏其書。海陵信裕不疑,謂白答構誣之,命殺白答於市。執白答出宣華門,點檢徒單貞得蕭懷忠上變事入奏,遇見白答,問其故,因止之。徒單貞已奏變事,以白答為請,海陵遽使釋之。



 海陵使宰相問裕,裕即
 款伏。海陵甚驚愕,猶未能盡信,引見裕,親問之。裕曰:「大丈夫所為,事至此又豈可諱。」誨陵復問曰:「汝何怨於朕而作此事?」裕曰:「陛下凡事皆與臣議,及除祚等乃不令臣知之。領省國王每事謂臣專權,頗有提防,恐是得陛下旨意。陛下與唐括辯及臣約同生死,辯以強忍果敢致之死地,臣皆知之,恐不得死所,以此謀反,幸茍免耳。太宗子孫無罪,皆死臣手,臣之死亦晚矣。」海陵復謂裕曰:「朕為天子,若於汝有疑,雖汝弟輩在朝,豈不能施行,以此疑我,汝實錯誤。太宗諸子豈獨在汝,朕為國家計也。」又謂之曰:「自來與汝相好,雖有此罪,貸汝性命,惟不
 得作宰相,令汝終身守汝祖先墳壟。」裕曰:「臣子既犯如此罪逆,何面目見天下人,但願絞死,以戒其餘不忠者。」海陵遂以刀刺左臂,取血塗裕面,謂之曰:「汝死之後,當知朕本無疑汝心。」裕曰:「久蒙陛下非常眷遇,仰戀徒切,自知錯繆,雖悔何及。」海陵哭送裕出門,殺之,并誅遙設及馮家奴。馮家奴妻,豫王女也,與其子穀皆與反謀,并殺之。遣護衛龐葛往西北路招討司誅朗及招折,而屯納、遏剌補皆出走,捕得屯納棄市,遏剌補自縊死。



 屯納出走,過河間少尹蕭之詳,之詳初不知裕事,留之三日。屯納往之詳茶扎家,茶扎遣人詣之詳告公引,得之,付
 屯納遣之他所。茶扎家奴發其事,吏部侍郎窊產鞫之,之祥曰:「屯納宿二日而去。」法家以之詳隱其間,欺尚書省,罪當贖。海陵怒,命殺之,杖窊產及議法者,茶扎杖四百死。



 龐葛殺招折等,并殺無罪四人,海陵不問,杖之五十而已。以裕等罪詔天下。賞上變功,懷忠遷樞密副使,以白答為牌印云。高藥師遷起居注,進階顯武將軍。藥師嘗奏裕有怨望,至是賞之云。



 胥持國,字秉鈞,代州繁畤人。經童出身,累調博野縣丞。上書者言民間冒占官地,如「太子務」、「大王莊」,非私家所宜有。部委持國按核之。持國還言「此地自異代已為民
 有,不可取也。」事遂寢。尋授太子司倉,轉掌飲令,兼司倉。皇太子識之,擢祗應司令,章宗即位,除宮籍副監,賜宮籍庫錢五十萬、宅一區。俄改同簽宣徽院事、工部侍郎,並領宮籍監。閱三月,遷工部尚書,使宋。明昌四年,拜參知政事,賜孫用康榜下進士第。會河決陽武,持國請督役,遂行尚書省事。明年,進尚書右丞。



 持國為人柔佞有智術。初,李妃起微賤,得幸於上。持國久在太子宮,素知上好色,陰以秘術干之,又多賂遺妃左右用事人。妃亦自嫌門地薄,欲藉外廷為重,乃數稱譽持國能,由是大為上所信任,與妃表裏,筦擅朝政。誅鄭王永蹈、鎬王永
 中,罷黜完顏守貞等事,皆起於李妃、持國。士之好利躁進者皆趨走其門下。四方為之語曰:「經童作相,監婢為妃。」惡其卑賤庸鄙也。



 承安三年,御史臺劾奏:「右司諫張復亨、右拾遺張嘉貞、同知安豐軍節度使事趙樞、同知定海軍節度使事張光庭、戶部主事高元甫、刑部員外郎張巖叟、尚書省令史傅汝梅、張翰、裴元、郭郛,皆趨走權門,人戲謂『胥門十哲』。復亨、嘉貞尤卑佞茍進,不稱諫職。俱宜黜罷。」奉可。於是持國以通奉大夫致仕,嘉貞等皆補外。



 頃之,起知大名府事,未行,改樞密副使,佐樞密使襄治軍於北京。一日,上召翰林修撰路鐸問以他事,
 因語及董師中、張萬公優劣,鐸曰:「師中附胥持國進。持國奸邪小人,不宜典軍馬,以臣度之,不惟不允人望,亦必不能服軍心,若回日再相,必亂天下。」上曰:「人臣進退人難,人君進退人易,朕豈以此人復為相耶。第遷官二階,使之致仕耳。」尋卒於軍,謚曰「通敏」。後上問平章政事張萬公曰:「持國今已死,其為人竟如何?」萬公對曰:「持國素行不純謹,如貨酒平樂樓一事,可知矣。」上曰:「此亦非好利。如馬琪位參政,私鬻省醖,乃為好利也。」子鼎,別有
 傳。



\end{pinyinscope}