\article{列傳第四十}

\begin{pinyinscope}

 ○僕散安貞田琢完顏弼蒙古綱必蘭阿魯帶



 僕散安貞,本名阿海,以大臣子充奉御。父揆,尚韓國公主,鄭王永蹈同母妹也。永蹈誅,安貞罷歸,召為符寶祗候。復為奉御,尚邢國長公主,加駙馬都尉,襲胡王愛割蠻猛安。歷尚衣直長、御院通進、尚藥副使。丁母憂,起復,轉符寶郎,除同知定海軍節度使事。歷邳、淄、涿州刺史,
 拱衛直都指揮使。貞祐初,改右副點檢兼侍衛親軍副都指揮使,遷元帥左都監。二年,中都解嚴,河北州郡未破者惟真定、大名、東平、清、沃、徐、邳、海州而已。朝廷遣安貞與兵部尚書裴滿子仁、刑部尚書武都分道宣撫。於是除安貞山東路統軍安撫等使。



 初,益都縣人楊安國自少無賴,以鬻鞍材為業,市人呼為「楊鞍兒」,遂自名楊安兒。泰和伐宋,山東無賴往往相聚剽掠,詔州郡招捕之。安兒降,隸諸軍,累官刺史、防禦使。大安三年,招鐵瓦敢戰軍,得千餘人,以唐括合打為都統,安兒為副統,戍邊。至雞鳴山不進。衛紹王驛召問狀,安兒乃曰:「平章參
 政軍數十萬在前,無可慮者。屯駐雞鳴山,所以備間道透漏者耳。」朝廷信其言。安兒乃亡歸山東,與張汝楫聚黨攻劫州縣,殺略官吏,山東大擾。



 安貞至益都,敗安兒於城東。安兒奔萊陽。萊州徐汝賢以城降安兒,賊勢復振。登州刺史耿格開門納偽鄒都統,以州印付之,郊迎安兒,發帑藏以勞賊。安兒遂僭號,置官屬,改元天順,凡符印詔表儀式皆格草定,遂陷寧海,攻濰州。偽元帥方郭三據密州,略沂、海。李全略臨朐,扼穆陵關,欲取益都。安貞以沂州防禦使僕散留家為左翼,安化軍節度使完顏訛論為右翼。



 七月庚辰,安貞軍昌邑東,徐汝賢等
 以三州之眾十萬來拒戰。自午抵暮,轉戰三十里,殺賊數萬,獲器械不可勝計。壬午,賊棘七率眾四萬陣于辛河。安貞令留家由上流膠西濟,繼以大兵,殺獲甚眾。甲申,安貞軍至萊州,偽寧海州刺史史潑立以二十萬陣于城東。留家先以輕兵薄賊,諸將繼之,賊大敗,殺獲且半,以重賞招之,不應。安貞遣萊州黥卒曹全、張德、田貴、宋福詐降於徐汝賢以為內應。全與賊西南隅戍卒姚雲相結,約納官軍。丁亥夜,全縋城出,潛告留家。留家募勇敢士三十人從全入城,姚雲納之,大軍畢登,遂復萊州,斬徐汝賢及諸賊將以徇。安兒脫身走,訛論以兵追
 之。耿格、史潑立皆降。留家略定膠西諸縣,宣差伯德玩襲殺方郭三,復密州。餘賊在諸州者皆潰去。安兒嘗遣梁居實、黃縣甘泉鎮監酒石抹充浮海赴遼東構留哥,已具舟,皆捕斬之。



 十一月戊辰,曲赦山東,除楊安兒、耿格及諸故官家作過驅奴不赦外,劉二祖、張汝楫、李思溫及應脅誘從賊,並在本路自為寇盜,罪無輕重,並與赦免。獲楊安兒者,官職俱授三品,賞錢十萬貫。十二月辛亥,耿格伏誅,妻子皆遠徙。諸軍方攻大沫堌,赦至,宣撫副使、知東平府事烏林答與即引軍還。賊眾乘之,復出為患。詔以陜西統軍使完顏弼知東平府事,權宣撫
 副使。其後楊安兒與汲政等乘舟入海,欲走岠嵎山。舟人曲成等擊之,墜水死。



 三年二月,安貞遣提控紇石烈牙吾塔破巨蒙等四堌,及破馬耳山,殺劉二祖賊四千餘人,降餘黨八千,擒偽宣差程寬、招軍大使程福,招降脅從百姓三萬餘人。安貞遣兵會宿州提控夾谷石里哥同攻大沫堌,賊千餘逆戰。石里哥,以騎兵擊之,盡殪。提控沒烈奪其北門以入,別軍取賊水寨,諸軍繼進,殺賊五千餘人。劉二祖被創,獲之,及偽參謀官崔天佑,楊安兒偽太師李思溫。餘眾保大小峻角子山,前後追擊,殺獲以萬計,斬劉二祖。詔遷賞沒烈等有差。詔尚書省
 曰:「山東東、西路賊黨猶嘯聚作過者,詔書到日,並與免罪,各令復業。在處官司盡心招撫,優加存恤,無令失所。」十月,安貞遷樞密副使,行院于徐州。



 四年二月,楊安兒餘黨復擾山東。詔安貞與蒙古綱、完顏弼以近詔招之。五月,安貞遣兵討郝定,連戰皆克,殺九萬人,降者三萬餘,郝定僅以身免。獲偽金銀牌、器械甚眾,來歸且萬人,皆安慰復業。自楊安兒、劉二祖敗後,河北殘破,干戈相尋。其黨往往復相團結,所在寇掠,皆衣紅納襖以相識別,號「紅襖賊」。官軍雖討之,不能除也。大概皆李全、國用安、時青之徒焉。



 興定元年十月,詔安貞曰:「防河卒多老
 幼疲軟不勝執役之人,其令速易之。」二年十二月,開封治中呂子羽等以國書議和于宋,宋人不受。以安貞為左副元帥權參知政事行尚書省元帥府,及唐、息、壽、泗行元帥府分道各將兵三萬,安貞總之,畫定期日,下詔伐宋。安貞至安豐,宋兵七千拒戰,權都事完顏胡魯剌衝擊敗之,追至淝水,死者二千餘人。安貞至大江,乃班師。三年閏月,安貞至自軍中,入見於仁安殿。胡魯剌進一階。久之,安貞燕見,奏曰:「淝水之捷,胡魯剌功第一,臣之兵事皆咨此人,功厚賞薄,乞加賞以勸來者。」尚書省奏:「凡行省行院帥府參議左右司經歷官都事以下皆遷
 一官,所以絕求請之路,塞姦倖之門也。安貞之請不可從。」遂止。



 五年,復伐宋。二月,安貞出息州,軍于七里鎮,宋兵據凈居山,遣兵擊敗之。宋兵保山寺。縱火焚寺,乘勝追至洪門山。宋兵方浚濠立柵,安貞軍亟戰,奪其柵。宋黃統制團兵五千保黃土關,關絕險,素有備,堅壁不出。安貞遣輕兵分為左右軍潛登,別以兵三千直逼關門。翼日,左右軍會於山顛,俯瞰關內。宋人守關者望之,駭咢不能立。中軍急攻,宋兵潰,遂奪黃土關。遂入梅林關,拔麻城縣,抵大江,至黃州,克之。進克蘄州,前後殺略不可勝計。獲宋宗室男女七十餘口,獻之,師還。安貞每獲宋
 壯士,輒釋不殺,無慮數萬,因用其策,輒有功。宣宗謂宰臣曰:「阿海將略善固矣,此輩得無思歸乎?南京密邇宋境,此輩既不可盡殺,安所置之?朕欲驅之境上,遣之歸如何?」宰臣不對。



 六月甲寅朔,尚書省奏安貞謀叛。宣宗謂平章政事英王守純曰:「朕觀此奏,皆飾詞不實,其令覆案之。」戊寅,並其二子殺之,以祖忠義、父揆有大功,免兄弟緣坐。詔曰:「銀青榮祿大夫、左副元帥兼樞密副使、駙馬都尉僕散阿海,早藉世姻,浸馳仕軌,屬當軍旅之事,益厚朝廷之恩,爰自帥籓,擢居樞府。頃者南伐,時乃奏言,是俾行鱗介之誅,而盡露梟獍之狀。二城雖得,
 多罪稔彰,念勝負之靡常,肯刑章之輕用。始自畫因糧之計,乃更嚴橫斂之期,督促計司,凋弊民力,信其私意,或失防秋。顧利害之實深,尚優容而弗問。頃因近侍,悉露姦謀,蓋虞前後罪之上聞,迺以金玉帶而夜獻。審事情之詭秘,命信臣而鞫推,迨致款詞,乃詳實狀。自以積愆之著,必非公憲所容,欲結近臣之歡心,俾伺內庭之指意,如釁端之少露,得先事而易圖。因其方握兵權,得以謀危廟祝願祏,事或不濟,計即外奔。前日之俘,隨時誅戮,獨於宋族,曲活全門,示其悖德于敵仇,豫冀全身而納用。」



 初,安貞破蘄州,獲宋宗室不殺而獻之,遂以為罪。安
 貞憂讒,以賄近侍局,乃以質成其誣。安貞典兵征伐,嘗曰:「三世為將,道家所忌。」自忠議、揆至安貞,凡三世大將焉。



 初,安貞破蘄州,所得金帛,分給將士。南京都轉運使行六部事李特立,金安軍節度副使紇石烈蒲剌都、大名路總管判官銀術可因而欺隱。事覺,特立當死、蒲剌都、銀術可當杖一百除名。詔薄其罪,特立奪三官、降三等,蒲剌都、銀術可奪兩官、降二等云。



 田琢,字器之,蔚州定安人。中明昌五年進士,調寧邊、茌平主簿,潞州觀察判官,中都商稅副使。丁父憂,起復懷安令,補尚書省令史。貞祐二年,中都被圍,琢請由間道
 往山西招集義勇,以為宣差兵馬提控、同知忠順軍節度使事,經略山西。琢與弘州刺史魏用有隙,琢自飛狐還蔚州,用伏甲於路,將邀而殺之。琢知其謀,自別道入定安。用入蔚州,殺觀察判官李宜,錄事判官馬士成、永興縣令張福,劫府庫倉稟,以兵攻琢於定安。琢與戰,敗之。用脫身走,易州刺史蒲察縛送中都元帥府殺之。是時,勸農副使侯摯提控紫荊等關隘,朝廷聞蔚州亂,欲以摯就代琢守蔚州,令軍中推可為管押者,即以魏用金牌佩之,以安其眾。丞相承暉奏:「田琢實得軍民心,諳練山西利害,魏用將士本無勞效,以用弄兵死禍,遽爾
 任用,恐開倖門。」詔從之。



 琢至蔚州,誅與用同惡數人。募兵旬日,得二萬人。十月,琢兵敗,僅以身免。招集散亡,得三萬餘,入中山界屯駐,而遣沈思忠招集西京蕩析百姓,得萬餘人,皆願徙河南。琢上書:「此輩與河南鎮防,往往鄉舊,若令南渡,擇壯健為兵,自然和協,且可以招集其餘也。」從之。加沈思忠同知深州軍州事。琢復遣沈思忠、宮楫招弘州、蔚州百姓,得五萬餘人,可充軍者萬五千人,分屯蔚州諸隘,皆願得沈思忠為將。詔加思忠順天軍節度副使,提控弘、蔚州軍馬,宮楫副之。頃之,西山諸隘皆不能守。琢移軍沃州。沃州刺史完顏僧家奴奏:「
 田琢軍二千五百人,官廩不足,發民窖粟猶不能贍。其中多女直人,均為一軍,不可復有厚薄,可令於衛、輝、大名就食。」制可。加琢河北西路宣撫副使,遙授濬州防禦使,屯濬州。琢欲陂西山諸水以衛濬州。



 貞祐三年十一月,河北行省侯摯入見,奏:「河北兵食少,請令琢汰遣老弱,就食歸德。」琢奏:「此輩嶺外失業,父子兄弟合為一軍,若離而分之,定生他變,乞以全軍南渡,或徙衛州防河。」詔盡徙屯陜。琢復奏:「臣幸徙安地,然浚乃河北要郡,今見糧可支數月,乞俟來春乃行。」數日,琢復奏:「濬不可守,惟當遷之。」宰臣劾琢前後奏陳不一,請逮鞫問。宣宗不
 許。



 琢至陜,上書曰:「河北失業之民僑居河南、陜西,蓋不可以數計。百司用度,三軍調發,一人耕之,百人食之,其能贍乎?春種不廣,收成失望,軍民俱困,實繫安危。臣聞古之名將,雖在征行,必須屯田,趙充國、諸葛亮是也。古之良吏,必課農桑以足民,黃霸、虞詡是也。方今曠土多,游民眾,乞明敕有司,無蹈虛文,嚴升降之法,選能吏勸課,公私皆得耕墾。富者備牛出種,貧者傭力服勤。若又不足,則教之區種,期於盡闢而後已。官司圉牧,勢家兼並,亦籍其數而授之農民,寬其負算,省其徭役,使盡力南畝,則蓄積歲增,家給人足,富國強兵之道也。」宣宗深
 然之。



 陜西元帥府請益兵,詔以琢眾與之。興定元年,朝廷易置諸將,遷山東西路轉運使。二年,改山東東路轉運使,權知益都府事,行六部尚書宣差便宜招撫使。李旺據膠西,琢遣益都治中張林討之,生擒李旺。八月,萊州經略使術虎山壽襲破李旺黨偽鄒元帥于小堌,獲其前鋒于水等三十人,追擊偽陳萬戶,斬首八百級。明日,復破之于朱寒寨。膠西、高密官軍亦屢破之于諸村及海島間。



 是月,棣州裨將張聚殺防禦使斜卯重興,遂據棣州,襲濱州,其眾數千人。琢遣提控紇石烈醜漢會兵討之。聚棄濱專保棣州。諸軍趣棣,聚出戰,敗之,斬首
 百級,生擒偽都統王仙等十三人。餘眾奔潰,追及于別寨,攻拔之,聚僅以身免。遂復二州。李全據安丘,琢遣總領提控王政、王庭玉討之。宣差提控、太府少監伯德玩率政兵攻安丘,敗焉,提控王顯死之。琢奏:「伯德玩本相視山東山堌水寨,未嘗遍行,獨留密州,輒為此舉,乞治其罪。」詔遣官鞫玩,會赦而止。既而昌樂縣令術虎桓都、臨朐縣令兀顏吾丁、福山縣令烏林答石家奴、壽光縣巡檢紇石烈醜漢破李全于日照縣,琢承制各遷官一階,進職一等,詔許之。



 三年,沂州注子堌王公喜構宋兵據沂州,防禦使徒單福定徒跣脫走,百姓潰散。琢奏:「去
 歲顧王二嘗據沂州,邳州總領提控納合六哥前為同知沂州防禦事,招集餘眾攻取之,百姓歸心。可用六哥取沂州,今方在行省侯摯麾下,乞發還,取便道進討。」制可。既而莒州提控燕寧復沂州,王公喜復保注子堌。琢奏:「沂州須知兵者守之。徒單福定已衰老,納合六哥善治兵,識沂形勢。」詔福定專治州事,以六哥為沂州總領。琢奏:「濰州刺史致仕獨吉世顯能招集猛安餘眾及義軍,卻李全,保濰州。六哥破灰山堌,沂境以安。守兗州觀察判官梁昱嘗攝淄州刺史,率軍民力田,徵科有度,饋餉不乏,保全淄州,土賊不敢發。前猗氏主簿張亞夫嘗
 權行部官,主餉密州,委曲購得糧二萬斛,兵儲乃足,行至高密,征他州兵拒李全。」詔世顯升職從四品,遙授同知海州事。六哥遷一官,升一等,充沂州宣差都提控。梁昱遷一官,同知淄州事。張亞夫遷兩官,密州觀察判官。



 初,張林本益都府卒,有復立府事之功,遂為治中,而兇險不逞,恥出琢下。琢在山東徵求過當,頗失眾心,林欲因眾以去琢,未有間也。會于海、牟佐據萊州,琢遣林分兵討之。林既得兵,伺琢出,即率眾噪入府中。琢倉猝入營,領兵與林戰,不勝,欲就外縣兵,且戰且行。至章丘,兵變,求救於鄰道,不時至。東平行省蒙古綱以狀聞。宣宗
 度不能制林,而欲馴致之,乃遣人召琢還。行至壽張,疽發背卒。



 完顏弼,本名達吉不,蓋州猛安人。充護衛,轉十人長。從丞相襄戍邊,功最,除同知德州防禦使事,武衛軍鈐轄,轉宿直將軍、深州刺史。泰和六年,從左副元帥完顏匡攻襄陽,破雷太尉兵,積功加平南盪江將軍。丁母憂,起復。八年,除南京副留守、壽州防禦使。大安二年,入為武衛軍副都指揮使。三年,以本官領兵駐宣德。會河之敗,弼被創,馬中流矢,押軍千戶夾谷王家奴以馬授弼,遂得免。遷右副都點檢。



 至寧元年,東京不守,弼為元帥左
 監軍,扞禦遼東。請「自募二萬人為一軍,萬一京師有急,亦可以回戈自救。今驅市人以應大敵,往則敗矣。」衛紹王怒曰:「我以東北路為憂,卿言京師有急何邪?就如卿言,我自有策。以卿皇后連姻,故相委寄,乃不體朕意也。」弼曰:「陛下勿謂皇后親姻俱可恃也。」時提點近侍局駙馬都尉徒單沒烈侍側,弼意竊譏之。衛紹王怒甚,顧謂沒烈曰:「何不叱去?」沒烈乃引起,付有司。論以奏對無人臣禮,詔免死,杖一百,責為雲內州防禦使。



 貞祐初,宣宗驛召弼赴中都,是時雲內已受兵,弼善馬槊,與數騎突出,由太原出澤、潞,將從清、滄赴闕。會有詔除定武軍節
 度使,尋為元帥左都監,駐真定。弼奏:「賞罰所以勸善懲惡,有功必賞,有罪必罰,而後人可使、兵可強。今外兵日增,軍無鬥志。亦有逃歸而以戰潰自陳者,有司從而存恤之,見聞習熟,相效成風。」又曰:「村寨城邑,兵退之後,有心力勇敢可使者,乞招用之。」又曰:「河朔郡縣,皆以拘文不相應救,由此殘破。乞敕州府,凡有告急徵兵,即須赴救,違者坐之。」又曰:「河北軍器,乞權宜弛禁,仍令團結堡寨以備外兵。」又曰:「今雖議和,萬一輕騎復來,則吾民重困矣。願速講防禦之策。」及勸遷都南京,阻長淮,拒大河,扼潼關以自固。



 宣宗將遷汴,弼兼河北西路兵馬都總
 管。宣宗次真定,弼言:「皇太子不可留中都,蓋軍少則難守,軍多則難養。」又奏:「將帥以閫外為威,今生殺之權皆從中覆。」又奏:「瑞州軍頗狡,左丞盡忠多疑,乞付他將。」宣宗頗采用其言。



 大名軍變,殺蒲察阿里,詔弼鎮撫之。未幾,改陜西路統軍使、京兆兵馬都總管。宣撫副使烏古論兗州置秦州榷場,弼以擅置,移文問之。兗州曰:「近日入見,許山外從宜行事,秦州自宋兵焚蕩榷場,幾一年矣,今既安帖,復宜開設,彼此獲利,歲收以十萬計。對境天水軍移文來請,如俟報可,實慮後時。」弼奏其事,宰臣以兗州雖擅舉而無違失,茍利於民,專之亦可。宣宗曰:「
 朕固嘗許其從宜也。」



 三年,改知東平府事、山東西路宣撫副使。是時,劉二祖餘黨孫邦佐、張汝楫保濟南勤子堌,弼遣人招之,得邦佐書云:「我輩自軍興屢立戰功,主將見忌,陰圖陷害,竄伏山林,以至今日,實畏死耳。如蒙湔洗,便當釋險面縛,餘賊未降者保盡招之。」弼奏:「方今多故,此賊果定,亦一事畢也。乞明以官賞示之。」詔曰:「孫邦佐果受招,各遷五官職。」於是邦佐、汝楫皆降。邦佐遙授濰州刺史,汝楫遙授淄州刺史,皆加明威將軍。頃之,弼薦邦佐、汝楫改過用命,招降甚眾,稍收其兵仗,放歸田里。詔邦佐遙授同知益都府事,汝楫遙授同知東平
 府事,皆加懷遠大將軍。梁聚寬遙授泰定軍節度副使,加宣武將軍。四年,弼遷宣撫使。已而汝楫復謀作亂,邦佐密告弼,弼饗汝楫,伏甲廡下,酒數行,鐘鳴伏發,殺汝楫并其黨與。手詔褒諭,封密國公。其後邦佐屢立功。元光末,累官知東平府事、山東西路兵馬都總管,充宣差招撫使。



 弼上書曰:「山東、河北、河東數鎮僅能自守,恐長河之險有不足恃者。河南嘗招戰士,率皆游惰市人,不閑訓練。若遷簽驅丁監戶數千,別為一軍,立功者全戶為良,必將爭先效命以取勝矣。武衛軍家屬嘗苦于兵,人人懷憤,若擇驍悍千餘,加以爵賞,亦可得其死力。」又
 曰:「老病之官,例許致仕,居河北者嫌于避難,居河南者茍于尸祿,職事曠廢。乞遍諭核實,其精力可用者仍舊,年高昏聵不事事者罷之。」又曰:「賦役頻煩,河南百姓新強舊乏,諸路豪民行販市易,侵土人之利,未有定籍,一無庸調,乞權宜均定。如知而輒避、事過復來者,許諸人捕告,以軍興法治之。」詔下尚書省議,惟老病官從所言,餘皆不允。



 大元兵圍東平,弼百計應戰,久之,乃解圍去。宣宗賜詔,獎諭將士,賞賚有差。是歲五月,疽發于腦。詔太醫診視,賜御藥。俄卒。



 弼平生無所好,惟喜讀書,閑暇延引儒士,歌詠投壺以為常。所辟如承裔、陀滿胡士門、
 紇石烈牙吾塔,皆立方面功。治東平,愛民省費,井邑之間,軍民無相訟,有古良將之風焉。



 蒙古綱,本名胡里綱,咸平府猛安人。承安五年進士,累調補尚書省今史,除國子助教。貞祐初,自請招集西山兵民,進官一階,賜錢二百萬,遷都水監丞,尋加遙授永定軍節度副使。招捕有功,遷太子左諭德,除順州刺史,遷同知大興府事。三年,知河間府事,權河北東路宣撫使,屯冀州。軍食不足,徙濟南。綱欲徙河南,行至徐州,未渡河,尚書省奏:「東平宣撫使完顏弼行事多不盡。」乃以綱權山東宣撫副使。改山東路統軍使,兼知益都府事,
 權元帥右都監,宣撫如故。四年十月,行元帥府事。綱奏:「山東兵後,楊安兒黨內有故淄王習顯、故留守術羅等家奴,不在赦原,據險作亂,至今未息,民多歸之,乞普賜恩宥。」宣宗即命赦之,仍贖為良。



 興定元年,徙知東平府事,遷元帥右監軍。久之,拜右副元帥權參知政事,行尚書省。先是,東平治中沒烈坐事削降殿年,詔仍從軍,有功復用。綱遣沒烈討花帽賊于曹、濟間,捷報,沒烈乃復前職。興定二年,詔曰:「卿以忠貞,為國捍難,保完城邑,朕甚嘉之。可進官二階,賜金帶一重,幣十端。」



 興定三年,奏曰:「濟南介山東兩路之間,最為衝要,被兵日久,雖與東
 平鄰接,不相統屬,緩急不相應,乞權隸本路,且差近於益都。」詔從之。綱奏:「恩州武城縣艾家凹水濼,清河縣澗口河濼,其深一丈,廣數十里,險固可恃。因其地形,少加浚治,足以保禦。請遷州民其中,多募義軍以實之。」綱以山東恃東平為重鎮,兵卒少,守城且不足,況欲分部出戰,是安坐以待困也。乃上奏曰:「伏見貞祐三年古里甲石倫招義軍,設置長校,各立等差,都統授正七品職,副統正八品,萬戶正九品,千戶正班任使,謀克雜班,仍三十人為一謀克,五謀克為一千戶,四千戶為一萬戶,四萬戶為一副統,兩副統為一都統,設一總領提控。今乞
 依此格募選,以益兵威。」制可。



 是歲,益都桃林寨總領張林號「張大刀」,據險為亂,自稱安化軍節度使。綱奏:「林勢甚張,乞遣河南馬軍千人,單州經略司以眾接應。」左司郎中李蹊請令綱約燕寧同力殄滅,單州經略使完顏仲元分兵三千人同往。宰相以糧運不給,益都以東,嘯聚不止一張林,宜令綱設備禦,俟來春議之。四年,張林侵掠東平,綱遣元帥右監軍行樞密院事王庭玉討之。至舊縣,遇張林眾萬餘人據嶺為陣,庭玉督兵踰嶺搏戰。林眾少卻,且欲東走。庭玉踵擊,大破之,殺數千人,生擒張林,獲雜畜兵仗萬計。招降虎窟諸寨,悉令歸業。詔
 賜空名宣敕,聽綱第功遷賞。遣樞密院令史劉顒蒞殺張林于東平。張林乞貰死自效,請曰:「臣兄演在宋為統制,有眾三千,駐即墨、萊陽之境,請以書招之,使轉致諸賊之款密者,相為表裏,然後以檄招益都張林,不從則合擊之,山東不足平也。」所謂益都張林,即據府事逐田琢者也,事見琢傳。綱以林策請于朝,樞密院請羈縻使之。制可,以為萊州兵馬鈐轄。久之,山東不能守,林乃降于宋云。



 初,東平提控鄭倜生擒宋將李資,綱奏賞倜。宰臣謂:「李資自稱宋將,無所憑據,請詳究其實。」綱奏:「臣自按問俱獲宋將統制十餘人,皆以資為將無異辭。此輩
 力屈就擒,豈肯虛稱偽將,以重獲者之功?今多故之際,賞功後時,將士且解體。凡行賞必求形迹,過為逗遛,甚未可也。」詔即賞之。綱奏:「遼東渡海,必由恩、博二州之間,乞置經略司鎮撫。」從之。興定五年二月,東平解圍,宣宗曲赦境內。凡東平府試諸科中選人,嘗被任使,已逾省試期日,特免省試。惟經童律科即為及第,似涉太優,別日試之。皆從綱所請也。詔以綱、王庭玉、東莒公燕寧保全東平,各遷一階。



 是歲,燕寧戰死。綱奏:「寧所居天勝寨,乃益都險要之地。寧嘗招降群盜胡七、胡八,用為牙校,委以腹心,群盜皆有歸志。及寧死,復懷顧望,胡七、胡八
 亦反側不安。臣以提控孫邦佐世居泰安,眾心所屬,遂署招撫使。以提控黃摑兀也充總領,副之。此當先奏可,顧事勢危迫,故輒授之。」燕寧死而綱勢孤矣。綱奏請移軍於河南,詔百官議,御史大夫紇石烈胡失門以下皆曰:「金城湯池,非粟不守。東平孤城,四無應援,萬一失之,則官吏兵民俱盡。宜徙之河南,以助防秋。」翰林待制抹捻阿虎德奏曰:「車駕南遷,恃大河以為險。大河以東平為籓籬,今乃棄之,則大河不足恃矣。兵以將為主,將以心為主,蒙古綱既欲棄之,決不可使之守矣。宜就選將士之願守者擢用之,別遣官為行省,付以兵馬鎧仗,從
 宜規畫軍食。」樞密院請用胡失門議,焚其樓櫓廨舍而徙之。宣宗曰:「此事朕不能決擇,眾議可者行之。」樞密院頗采阿虎德議,許綱內徙,率所部女直、契丹、漢軍五千人,行省邳州。元帥左監軍王庭玉將餘軍屯黃陵岡,行元帥府事。於是,綱改兼靜難軍節度使,行省邳州。自此山東事勢去矣。



 是歲六月,以歸德、邳、宿、徐、泗乏軍食,詔綱率所部就食睢州。綱奏:「宿州連年饑饉,加之重斂,百姓離散。鎮防軍遽徵逋課,窘迫陵辱有甚于官,眾不勝其酷,皆懷報復之心。近日,高羊哥等苦其佃戶,佃戶憤怒,執羊哥等投之井中。武夫不識緩急,乃至于此。乞
 一切所負並令停止,俟夏秋收成徵還,軍人量增廩給,可也。」詔議行之。元光二年三月,以邳州經略司隸綱,令募勇敢,收復山東。



 初,碭山首領數人,以減罷懷忿怨,誘脅餘眾作亂,引水環城以自固,構浮橋於河上,結紅襖賊為援。同簽樞密院事徒單牙剌哥會諸道兵討之。綱云:「碭山北近大河,南近汴堤,東西二百里,大河分派其間,乾灘泥淖,步騎俱不可行,惟宜輕舟往來。可選銳卒數千與水軍埽兵,以舟二百艘,由便道斷浮梁,絕紅襖之援。募膽勇有口辯者,持牒密諭之以離間其黨,與臣已遣三人入賊中。復分兵屯要害,別以三百人巡邏。乞賜
 空名告身,從便遷賞。」樞密院奏:「已委監軍王庭玉駐歸德、寧陵備之矣。仍令牙剌哥水陸並進,先行招誘,不從,乃合擊之。其空名告身,宜從所請,以責成功。」



 無何,碭山賊夜襲永城縣,行軍副總領高琬、萬戶麻吉擊走之,殺傷及溺死者甚眾,奪其所俘掠而還。詔綱併力討之。綱遣降人陳松持牒招李全,全縛松將斬之,已而但黥其面遣還。綱奏:「全有歸國意,嚴實、張林亦可招之。」此謂益都張林也。詔擬實一品官職,封國公,仍世襲。全階正三品、職正二品。林山東西路宣撫使兼知益都府事,與全皆賜田百頃。受命往招者先授正七品官職,賜銀二十
 五兩,事成遷五品。會綱遇害而止。



 綱御下嚴,信賞必罰,邳州軍不樂屬綱。八月辛未朔,邳州從宜經略使納合六哥、都統金山顏俊率沂州軍士百餘人晨入行省,殺綱及僚屬于省署,遂據州反。樞密院奏請出空名宣敕,設重賞招誘。丞相高汝礪曰:「懸重賞募死士,必有能取之者。」宣宗不得已,下詔罪綱,以撫諭六哥。六哥遣人送綱尸及虎符牌印,終不肯出。乃升經略司為元帥府,加六哥泗州防禦使,權元帥左監軍,副使烏古論老漢加邳州刺史,權右監軍。頃之,邳州卒逃歸,詣總帥牙吾塔言,六哥已結李全為助。遣總領孛術魯留住等毀其橋
 梁,攻破承安、青陽寨,留兵戍守。六哥惶懼,乃言待李全兵入邳州,誘而殺之,以圖報效。宣宗曰:「李全豈無心者,六哥能誘而殺之,殆詐耳。」十月壬辰,牙吾塔圍邳州,急攻之。紅襖賊高顯等殺六哥,函首以獻。詔加顯三品官職,授世襲謀克,侯進四品,陳榮、邢進、邊全、魏興、孫仲皆五品,賞銀有差。



 必蘭阿魯帶,貞祐初,累官寧化州刺史。二年,同知真定府事,權河北、大名宣撫副使。三年,保全贊皇,加遙授安武軍節度使,改昭義軍節度使、充宣撫副使。閱月,權元帥左都監行元帥府事,節度、宣撫如故。遣都統奧屯喜哥
 復取威州及獲鹿縣。既而詔擇義軍為三等,阿魯帶奏:「自去歲初置帥府,已按閱本軍,去其冗食。部分既定,上下既親,故能所向成功,此皆血戰屢試而可者。父子兄弟自相救援,各顧其家,心一力齊,勢不可離。今必析之,將互易其處,不相諳委矣。國家糧儲常患不繼,豈容僥冒其間?但本府之兵不至是耳。事勢方殷,分別如此,彼居中下,將氣挫心懈而不可用。且義軍率皆農民,已散歸田畝,趨時力作,徵集旬日,農事廢而歲計失矣。乞本府所定,無輕變易。」詔許之。阿魯帶繕完州縣之可守者,其不可守者遷徙其民,依險為柵以備緩急。



 澤州舊
 隸昭義軍,近年改隸孟州,阿魯帶奏:「澤州城郭堅完,器械具備,若屯兵數千,臣能保守之。今聞議遷於青蓮寺山寨,距州既遠,地形狹隘,所容無幾。一旦有急,所保者少,所遣者多,徒棄名城以失太行之險,則沁南、昭義不通問矣。」詔澤州復隸昭義軍。



 是歲,潼關失守,阿魯帶趨備藍田、商州,乃陳河北利害,略曰:「今忻、代撤戍,太原帥府眾纔數千,平陽行省兵亦不多,河東、河北之勢,全恃潞州,潞州兵強,則國家基本漸可復立。臣已將兵離境,乞復置潞州帥府。」阿魯帶行次澠池,右副元帥蒲察阿里不孫敗績,逃匿不知所在。阿魯帶亦被創,收集潰卒,
 臥澠池。詔還潞州。



 興定元年,改簽樞密院事。數月,以元帥左監軍兼山東路統軍使,知益都府事。未幾,權參知政事,行尚書省于益都。阿魯帶復立潞州,最有功,識遼州刺史郭文振,舉以為將。既而去潞州,張開代領其眾,與郭文振不相得,文振漸不能守矣。



 贊曰:貞祐之時,僕散安貞定山東,僕散端鎮陜西,胥鼎控制河東,侯摯經營趙、魏,其措注施設有可觀者。故田琢撫青、齊,完顏弼保東平,必蘭阿魯帶守上黨,皆嚮用有功焉。高琪忌功,汝礪固位,西啟夏釁,南挑宋兵。宣宗道謀是用,煦煦以為慈,皦皦以為明,孑孑以為強。既而
 潼關破毀,崤、澠喪敗,汴州城門不啟連月,高琪方且增陴浚隍為自守計,繕御寨以祈逃死。然後田琢走益都而青、齊裂,蒙古綱去東平而兗、魯蹙,僕散安貞死而南伐無功。雖曰天道,亦由人事。自是以往,無足言者矣。



\end{pinyinscope}