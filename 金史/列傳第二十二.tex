\article{列傳第二十二}

\begin{pinyinscope}

 ○杲本名撒離喝耨碗溫敦思忠子乙迭溫敦兀帶奔睹高楨白彥敬張景仁



 杲,本名撒離喝,安帝六代孫,泰州婆盧火之族,胡魯補山之子。雄偉有才略,太祖愛之,常在軍中。及婆盧火為泰州都統,宗族皆隨遷泰州。撒離喝嘗為世祖養子,獨
 得不遷,仍居安出虎水。



 宗翰、宗望已再克汴,執宋二主北還。宗望分遣諸將定河北。左都監闍母攻下河間。雄州李成棄城走,撒離喝邀擊,大破之,雄州遂降。睿宗經略山東,留撒離喝於河上,而真定境內有賊眾,自稱元帥秦王。撒離喝擊破其眾,執而戮之。從平陜西,撒離喝徇地自渭以西,降德順軍,又降涇原路鎮戎軍,進平熙河,降甘泉等三堡,遂取保川城,明年,同奔睹討平河外,降寧洮、安隴二寨,並降下河及樂州。至西寧,盡降其都護官屬,於是木波族長等皆迎降。攻慶陽,敗其拒者,遂降其城。慕洧以環州來降,得城寨十三,步騎一萬。於是,
 宗弼軍敗于和尚原,上褒美撒離喝而戒勵宗弼。



 睿宗已定陜西,留兵屯衝要,使撒離喝總之。居無何,請收劍外十三州。與宋王彥之軍七千人遇于沙會濼,敗之,遂克金州。連破吳玠諸軍于饒峰關,遂取真符縣,取洋州入興元府。敗吳玠兵於固鎮,擒其兩將。撒葛柷等破宋兵,盡下諸砦及仙人關。天會十四年,為元帥右監軍。



 天眷三年,宗弼復取河南。撒離喝自河中出陜西。既至鳳翔,擊走宋軍。是時,宋軍在京兆西者甚眾。諸將以暑雨,欲駐軍。且聞宋兵九萬會于涇州,都元帥遣河南步卒來會軍。撒離喝留諸將屯環慶,獨以輕騎取涇州。六月,
 敗宋兵於涇州。宋兵走渭州,拔離速追擊,大敗之。未幾,為右副元帥。皇統三年,封應國公,錫賚甚厚。熙宗出獵,賜具裝馬二,命射于圍中。加開府儀同三司。將還軍,命宰臣餞之。



 海陵升蒲州為河中府,撒離喝為河中尹,左副元帥如故。自陜西入朝,因從容言曰:「唐建成不道,太宗以義除之,即位之後,力行善政,後世稱賢。陛下以前主失德,大義廢絕,力行善政,則如唐太宗矣。」海陵聞其言,色變,撒離喝亦悔其言。既而進封國王,從行官吏皆官賞之。海陵念撒離喝久握兵在外,頗得士心,忌之,以為行臺左丞相兼左副元帥。又恐不奉命,陽尊以殊禮,
 使係屬籍,以玉帶璽書賜之。撒離喝至汴,詔諭行臺右丞相、右副元帥撻不野無使撒離喝預軍事。撒離喝不知,每事輒爭之。撻不野詭曰:「太師梁王以陜西事屬公,以河南事屬撻不野,今未嘗別奉詔命。陜西之事,撻不野固不敢干涉。」撻不野久在河南,將帥畏而附之。撒離喝始至勢孤,爭之不得,白於朝。大臣知上旨,報曰:「如梁王教。」及詔使至汴,諭旨於撻不野。使還,撻不野獨有附奏,撒離喝不得與聞,人皆知海陵使撻不野圖之矣。



 會海陵欲除遼王斜也子孫及平章政事宗義等,元帥府令史遙設希海陵旨,誣撒離喝父子謀反,并平章宗義、
 尚書謀里野等。遙設學撒離喝手署及印文,詐為契丹小字家書與其子宗安,從左都監奔睹上變。封題作已經開拆者,書紙隱約有白字,作曾經水浸,致字畫分明者,稱御史大夫宗安於宮門外遺下此書,遙設拾得之。其書略曰:「撻不野自來於我不好,凡事常有隄防,應是知得上意。移剌補丞相於我不好,若遲緩分毫,猜疑必落他手也。」又曰:「阿渾每見此書,約定月日,教掃胡令史卻寫白字書來。」有司鞫問,宗安不服曰:「使真有此書,我剖肌肉藏之,猶恐漏泄,安得於朝門下遺之?」有司掠笞楚毒,宗安神色不變。乃置掃胡爐炭上,掃胡不能堪,自
 誣服。宗安謂掃胡曰:「爾苦矣。」宗義被掠笞,不能當,亦自誣服,曰:「我輩知不免矣,不早決,徒自苦。」宗安曰:「今雖無以自明,九泉之下當有冤對,吾終不能引屈。」竟不服而死。使廝魯渾殺撒離喝于汴,族其家,而無寫書及傳書者主名。



 有折哥者,能契丹小字,舊嘗從撒離喝。特末者,陜西舊將,嘗以左副元帥事馳驛赴闕。兩人者皆族誅。撒離喝親屬坐是死者二十餘人。魯王斡者孫耶魯候撒離喝于汴,廝魯渾執之,耶魯曰:「願付有司,若法當同坐,雖死不恨。」廝魯渾亦殺之。其家訟于朝,海陵不問,但賜錢二百萬。



 奔睹遷元帥左監軍,加開府儀同三司。遙
 設為同知博州事,賜錢三百萬,謂之曰:「爾無自比老人。老人親告朕,爾以告有司,設有撒離喝黨人在其間,敗吾事矣。」老人指蕭玉也。蕭玉名老人,故云然。遙設在博州數歲,後與蕭裕謀反,伏誅。



 大定初,詔復撒離喝官爵。三年,追封金源郡王,謚莊襄,以郡王品秩官為營葬。十七年,配享太宗廟廷。



 耨碗溫敦思忠,本名乙剌補,阿補斯水人。太祖伐遼,是時未有文字,凡軍事當中復而應密者,諸將皆口授思忠,思忠面奏受詔,還軍傳致詔辭,雖往復數千言,無少誤。及遼人議和,思忠與烏林答贊謀往來專對其間,號
 閘剌。閘剌者,漢語云行人也。自收國元年正月,遼人遣僧家奴來,使者三往反,議不決。使者賽剌至遼,遼人殺之。遼主自將,至駝門,大敗,歸,復遣使議和。太祖使胡突袞往,書曰:「若不從此,胡突袞但使人送至界上,或如賽剌殺之,惟所欲者。」



 天輔三年六月,遼大冊使太傅習泥烈以冊璽至上京一舍,先取冊文副錄閱視,文不稱兄,不稱大金,稱東懷國。太祖不受,使宗翰、宗雄、宗乾、希尹商定冊文義指,揚樸潤色,胡十答、阿撒、高慶裔譯契丹字,使贊謀與習泥烈偕行。贊謀至遼,見遼人再撰冊文,復不盡如本國旨意,欲見遼主自陳,閽者止之。贊謀不
 顧,直入。閽者相與搏撠,折其信牌。遼人懼,遽遣贊謀歸。太祖再遣贊謀如遼。遼人前後十三遣使,和議終不可成。太祖自將,遂克臨潢。



 其後伐宋,思忠從宗翰軍,封劉豫為齊帝,思忠為傳宣使,俄授謀克。從宗弼克和尚原。還為同知西京留守事。天眷初,改蒲州防禦使。元帥府在陜西者,其官屬往往豪壓貧民為奴,起遣工匠千人東來,至河上,思忠留止其人以聞,詔皆還之。為行臺尚書左丞。是時,贊謨為行臺參知政事,思忠黷貨無厭,贊謨鄙之,兩人由是交惡。海陵殺左丞相秉德于行臺。贊謨妻,秉德乳母也。思忠因構贊謨。殺之。是歲,思忠入為尚
 書右丞。俄進平章政事,封郜國公。進拜左丞相兼侍中,封沂國公。



 天德三年,致仕。貞元二年十月,海陵率三品以上官幸思忠第,使以家禮見,謂思忠曰:「卿神氣康實,習先朝舊事,舍卿無能知者,當為朕起,共治國政。」對曰:「君之命,臣敢不敬從,但恨老病疏謬,無以塞責耳。」遂命思忠乘馬從入宮,拜太傅,領三省事,封齊國王。尋拜太師兼勸農使。已而罷中書門下省,不置領三省事。置尚書令,位丞相上。思忠為尚書令,特置散從八人,聽隨至宮,省奏賜坐。海陵欲定封爵制度,風思忠建白之。封王者皆降封,異姓或封公或一品、二品階。惟封思忠廣平
 郡王,賜以玉帶。思忠言百官不當封妻,海陵從之。惟封思忠次室為郡夫人。而思忠亦自謂太祖舊臣,頗自任,雖海陵遂非拒諫,而思忠盡言無所避。



 海陵將伐宋,問諸大臣,皆不敢對。思忠曰:「不可。」海陵不悅,謂思忠曰:「汝勿論可否,但云何時克之。」思忠曰:「以十年為期。」海陵曰:「何久也?期月耳。」思忠曰:「太祖伐遼,猶且數年。今百姓愁怨,師出無名。江、淮間暑熱湫濕,不堪久居,未能以歲月期也。」海陵怒,顧視左右,若欲取兵刃者。思忠無所畏恐,復曰:「老臣歷事四朝,位至公相,茍有補於國家,死亦何憾。」有頃,海陵曰:「自古帝王混一天下,然後可為正統。爾
 耄夫固不知此,汝子乙迭讀書,可往問之。」思忠曰:「臣昔見太祖取天下,此時豈有文字耶?臣年垂七十,更事多矣,彼乳臭子,安足問哉!」



 海陵既不用思忠言,運四方甲仗於中都。思忠曰:「州郡無兵,何以備盜賊?」海陵盡籍丁壯為兵,思忠曰:「山後契丹諸部,恐未可盡起。」皆不聽。其後,州郡盜起,守令不能制。契丹撒八、窩斡果反,期年乃克之。



 當是時,海陵伐宋,祁宰諫而死,張浩進言被杖,思忠見疏,孔彥舟畫策先取兩淮,他無及者。正隆六年,思忠薨,年七十三。海陵深悼惜之,親臨奠,賻贈加等,賜金螭頭車,使者監護,給道路費。



 大定十二年,詔復烏林答
 贊謨官爵,贈特進。上謂宰臣曰:「贊謨忠實剛毅,雖古人無以過。與思忠有隙,遂勸海陵殺之。今思忠子孫皆不肖,亦陰報也。」初,思忠已構殺贊謨,遂納其妻曹氏,盡取其家財產。章宗即位,贊謨女五十九乞改葬。詔賜葬地於懷州,并以思忠元取家貲付之。



 謙,本名乙迭,累官御史中丞。世宗謂之曰:「省部官受請託,有以室家傳達者。官刑不肅,士風頹弊如此,其糾正之。」初,世宗至中都,多放宮人還家,有稱心等數人在放遣之例,所司失於檢照,不得出宮,心常怏怏。大定二年閏二月癸巳夜,遂於十六位放火,延燒太和、神龍殿。上
 命近臣跡火之所發。十六位宮人袁六娘等六人告,實稱心等為之。稱心等伏誅,賞賜袁六娘六人,放出宮為良。謙意宮殿被火,將復興工役,勞民傷財,乃上表乞權紓修建。上使張汝弼詔謙曰:「朕思正隆比年徭役,百姓瘡痍未復,邊事未息,豈遽有營繕也。卿可悉之。」



 久之,襲父思忠濟州猛安、利涉軍節度副使。烏林答鈔兀追捕逃軍,至猛安中,謙畏其擾,乃醵民財買銀賂鈔兀。事覺,鈔兀抵罪,謙坐奪猛安。遇赦,求敘。上曰:「乙迭無自與贓,使復其所。」



 耨碗溫敦兀帶,太師思忠侄也。天會間,充女直字學生,
 學問通達,觀書史,工為詩。選為尚書省令史,除右司都事,轉行臺右司郎中,入為左司員外郎。累官同知大興尹,京師盜賊止息,事無留滯。再遷刑部尚書,改定海軍節度使。除兵部尚書,改吏部。正隆伐宋。為武定軍都總管。世宗即位,遣使召之,授咸平尹,為北邊行軍都統。改會寧尹,都統如故。是時初定窩斡,人心未安,兀帶為治寬簡,多備禦,謹斥候,邊郡以寧。改北京留守。以廉察舉「兀帶所在有能名,無私過」,由是入拜參知政事。世宗諭之曰:「凡在卿上者,行事或不當理,咨稟不從,卿以所見奏聞。下位有可用之才,當推薦之。」久之,屬疾,上命左宣
 徽使敬嗣暉往視,遣醫治療。薨,年四十七。上聞悼惜之,賻銀千兩、重綵四十端、絹四百匹,敕有司致祭。久之,上謂侍臣曰:「故參知政事兀帶、刑部尚書彥忠、滄州節度使兀不喝、侍郎敵斡、郎中骨赧皆為人忠直,後進中少有能及之者。朕樂得忠直之人,有如兀帶輩者乎,卿等為朕舉之。」其見思如此。



 昂,本名奔睹,景祖弟孛黑之孫,斜斡之子。幼時侍太祖。太祖令數人兩兩角力。時昂年十五,太祖顧曰:「汝能此乎?」對曰:「有命,敢不勉。」遂連仆六人。太祖喜曰:「汝,吾宗弟也,自今勿遠左右。」居數日,賜金牌,令佩以侍。年十七,太
 祖伐遼,謂之曰:「汝可擐甲從軍矣。」昂遂佩所賜金牌從軍。太祖平燕,策功,賜甲第一區。天輔六年,宗翰駐北安州,聞遼主延禧在鴛鴦濼,遣耨碗溫敦思忠請於國論勃極烈杲,願以所部軍追之。杲不能決,乃遣昂與思忠詣宗翰議,其事遂定。天會二年,南京叛,軍帥闍母遣昂、劉彥宗分兵討之。



 宗望伐宋,承制以為河南諸路兵馬都統,稱「金牌郎君」。及攻汴州,宗弼與昂以兵三千為前鋒。比暮,昂先以兵千人馳至其北門。時軍中遣使入城,宋人不納。昂諭之以事,遂得入。宗望至汴,令闍母、撻懶等屯于城之東北隅。慮宋主遁去,遣昂等率輕騎環城
 巡邏。昂所領止八謀克,遇敵萬人,與戰,敗之,其步軍溺死於汴者過半。七年,大軍渡江,敗宋兵於江上。帥府遣昂等以兵追宋主。宋主入會稽,若為堅守計,有兵數千列陣於郭東竹葦間。諸將欲擊之,昂曰:「此詐也。不若急攻城,不然將由他門逸去。」諸將猶豫未決,而宋主果於他門以單舟入海,不獲而還。



 宗輔定陜西,宗弼經略熙秦,遣昂與撒離喝領兵八千攻取河西郡縣。昂等遂取寧洮、安隴二寨。進至河州,其通判率士民迎降。攻樂州,其都護及河州安撫使郭寧偕降。復進取三寨,至西寧州,都護許居簡以城降,吐蕃酋長之孫趙鈐轄率其所
 部木波首領五人來降。昂別領軍四千往積石軍,降其軍及所部五寨官吏。追吐蕃鈐轄等十二人至廓州,招之不下,攻取之。



 天眷元年,授鎮國上將軍,除東平尹。明年夏,宋將岳飛以兵十萬,號稱百萬,來攻東平。東平有兵五千,倉卒出禦之。時桑柘方茂,昂使多張旗幟於林間,以為疑兵,自以精兵陣于前。飛不敢動,相持數日而退。昂勒兵襲之,至清口,飛眾泛舟逆水而去。時霖雨晝夜不止,昂乃附水屯營。夜將半,忽促眾北行。諸將諫曰:「軍士遠涉泥淖,饑憊未食,恐難遽行。」昂怒不應,鳴鼓督之,下令曰:「鼓聲絕而敢後者斬。」遂棄營去,幾二十里而
 止。是夜,宋人來劫營,無所得而去。諸將入賀,且問其故。昂曰:「沿流而下者,走也;溯流而上者,誘我必追也。今大雨泥淖,彼舟行安,我陸行勞。士卒飢乏,弓矢敗弱,我軍居其下流,勢不便利,其襲我必矣。」眾皆稱善。岳飛以兵十萬圍邳州甚急,城中兵纔千餘,守將懼,遣人求救。昂曰:「為我語守將,我嘗至下邳,城中西南隅有塹深丈餘,可速實之。」守將如其教,填之。岳飛果自此穴地以入,知有備,遂止。昂舉兵以為聲援,飛乃退。



 在東平七年,改益都尹,遷東北路招討使,改崇義軍節度使,遷會寧牧。天德初,改安武軍節度使,遷元帥右都監,轉左監軍,授上
 京路移里閔斡魯渾河世襲猛安。海陵曰:「汝有大功,一猛安不足酬也。」益以四謀克。昂受親管謀克,餘三謀克讓其族兄弟。拜樞密副使,轉太子少保,進樞密使、尚書左丞相。昂怒族弟妻,去衣杖其脊,海陵聞之,杖昂五十。久之,拜太尉,封沈國公。進太保,判大宗正事,封楚國公,累進封莒、衛、齊,兼樞密使,太保如故。



 海陵南伐,分諸路軍為三十二總管,分隸左右領軍大都督府,遂以昂為左領軍大都督。海陵築臺于江上,召昂及右領軍副大都督蒲盧渾謂之曰:「舟楫已具,可以濟矣。」蒲盧渾曰:「舟小不可濟。」海陵怒,詔昂與蒲盧渾明日先濟。昂懼,欲亡
 去。抵暮,海陵遣人止之曰:「前言一時之怒耳。」既而至揚州,軍變,海陵死。



 世宗即位遼陽,昂使人殺皇太子光英於南京,遣其子寢殿小底宗浩與其婿牌印祗候回海等奉表賀登寶位。大軍北還,昂恐宋人躡其後,即以罷兵移書于宋。二年,入見世宗,深慰勞之。進封漢國公,拜都元帥,太保如故,置元帥府於山東,經略邊事。未幾,奉遷睿宗皇帝梓宮於山陵,以昂為敕葬使。事畢,還山東。三年,召至京師,以疾薨,年六十四。上為輟朝,親臨奠,賻銀千兩、重彩五十端、絹五百匹。



 昂在海陵時,縱飲沉酣,輒數日不醒。海陵聞之,常面戒不令飲。得閑輒飲如故。
 大定初,還自揚州,妻子為置酒私第,未數行,輒臥不飲。其妻大氏,海陵庶人從母姊也,怪而問之。昂曰:「吾本非嗜酒者,但向時不以酒自晦,則汝弟殺我久矣。今遇遭明時,正當自愛,是以不飲。」聞者稱之。睦於兄弟,尤善施予,其親族有貧困者,必厚給之。至於茵帳、衣衾、器皿、僕馬之屬,常預設於家。即命駕相就,為具,歡樂終日,盡以遺之,即日使富足。人或以子孫計為言,答曰:「人各有命,但使其能自立爾,何至為子孫奴耶?」君子以為達。



 贊曰:撒離喝、溫敦思忠、奔睹皆有功舊臣,當天會、皇統之際,戰勝攻取,可謂壯哉。及海陵之世,崎嶇嫌忌,撒離
 喝既自以言致疑,猶與大抃辨爭軍事,何見幾之不早也。烏林答贊謨廉直自奮,思忠擠之於死,自謂固結海陵,堅若金石,豈意執議不合而遽棄耶。始之不以道,未有能終者也。且思忠之最可罪者,構害贊謨,又納其室而敓其貲,此何異於殺越人于貨者乎!陰報不在其身,在其子孫,亦已晚矣。正隆之末,奔睹位三公,居上將,內不肯與謀,外不肯與戰,逼側趑趄,茍免自全,大臣之道,固若是乎?



 高楨,遼陽渤海人。五世祖牟翰仕遼,官至太師。楨少好學,嘗業進士。斡魯討高永昌,已下沈州,永昌懼,偽送款
 以緩師。是時,楨母在沈州,遂來降,告以永昌降款非誠,斡魯乃進攻。既破永昌,遂以楨同知東京留守事,授猛安。天會六年,遷尚書左僕射,判廣寧尹,加太子太傅。在鎮八年,政令清肅,吏畏而人安之。十五年,加太子太師,提點河北西路錢帛事。天眷初,同簽會寧牧。及熙宗幸燕,兼同知留守,封戴國公,改同知燕京留守。魏王道濟出守中京,以楨為同判,俄改行臺平章政事,為西京留守,封任國公。



 是時,奚、霫軍民皆南徙,謀克別術者因之嘯聚為盜。海陵患之,即以楨為中京留守,命乘驛之官,責以平賊之期。賊平,封河內郡王。海陵至中京,楨警夜
 嚴肅。有近侍馮僧家奴李街喜等皆得幸海陵,嘗夜飲干禁,楨杖之瀕死,由是權貴皆震懾。遷太子太保,行御史大夫,封莒王。策拜司空,進封代王,太子太保、行御史大夫如故。



 楨久在臺,彈劾無所避,每進對,必以區別流品,進善退惡為言,當路者忌之。薦張忠輔、馬諷為中丞,二人皆險詖深刻,欲令以事中楨。正隆例封冀國公,楨因固辭曰:「臣為眾小所嫉,恐不能免,尚可受封爵耶?」海陵知其忠直,慰而遣之。及疾革,書空獨語曰:「某事未決,某事未奏,死有餘恨。」薨,年六十九。海陵悼惜之,遣使致奠,賻贈加等。



 楨性方嚴,家居無聲伎之奉。雖甚暑,未嘗
 解衣緩帶。對妻孥危坐終日,不一談笑,其簡默如此。



 白彥敬,本名遙設,部羅火部族人。初名彥恭,避顯宗諱,改焉。祖屋僕根。父阿斯,仕遼為率府率。彥敬善騎射,起家為吏,補元帥府令史。伐宋,為錢帛司都管勾。立三省,選為尚書省令史,除都元帥府知事。招諭諸部,授以金牌,行數千里,有功,超遷兵部郎中。熙宗罷統軍司改招討司,遣彥敬分僚屬改牌印,諭諸部隸招討司。還為本部侍郎,遷大理卿,出為通州防禦使,改刑部侍郎。怨家告誣開府慎思與西北路部族謀叛,彥敬鞫得其實,海陵嘉之。遷簽書樞密院事,以便宜措置邊防。



 正隆六年,
 調諸路兵伐宋,及調民馬,使彥敬主會寧、蒲與、胡里改三路事。改吏部尚書,充南征萬戶,遷樞密副使。契丹撒八反,樞密使僕散忽土等以無功坐誅,以彥敬為北面行營都統,與副統紇石烈志寧以便宜往,賜御服皮襖。行至北京,聞南征諸軍逃歸者皆奔東京,欲推戴世宗。彥敬與志寧謀,陰結會寧尹完顏蒲速賚、利涉軍節度使獨吉義以圖之。



 世宗已即位,使石抹移迭、移剌曷補等九人招彥敬、志寧。彥敬拒之,使移迭跪。移迭不屈,皆殺之。及完顏謀衍將兵攻北京,彥敬使偏將率兵拒於建州之境,而獨吉義先歸世宗,蒲速賚稱疾不至。世宗
 密遣人乘夜揭榜於北京市,購以官賞。彥敬、志寧恐為人圖己,遂降。以為曷速館節度使。不數月,召為御史大夫。



 窩斡心慄帝號。諸軍馬瘦弱,遣彥敬往西北路招討司市馬,得六千餘匹。窩斡敗,西走山後。完顏思敬以新馬三千備追襲。彥敬屯于夏國兩界間。窩斡平,召還為兵部尚書,出為鳳翔尹,改太原尹,兼河北東路兵馬總管,尋改河中尹。大定九年,卒于官。



 張景仁,字壽甫,遼西人。累官翰林待制。貞元二年,與翟永固俱試禮部進士,以「尊祖配天」為賦題,忤海陵旨,語在永固傳。大定二年,僕散忠義伐宋,景仁掌其文辭。宋
 人議和,朝廷已改奉表為國書,稱臣為姪,但不肯世稱侄國。往復凡七書,然後定,其書皆景仁為之。世宗稱其能,嘗曰:「今之文章,如張景仁與宋人往復書,指事達意,辨而裁,真能文之士也。」五年,罷兵,入為翰林直學士。七年,遷侍講。八年,為詳讀官。宋國書中有「寶鄰」字,景仁奏「鄰」字太涉平易。上問累年國書有「鄰」字否,命一一校勘。六年書中亦有之,上責問六年詳讀官劉仲淵,右丞石琚亦請罪曰:「臣嘗預六年詳讀。」上曰:「此有司之過,安得一一責宰臣邪?」詔有司就諭宋臣王瀹,使歸告其主,後日國書不得復爾。仲淵時為禮部侍郎,降石州刺史,景
 仁遷翰林學士兼同修國史。



 久之,上召景仁讀陳言文字。上問「事款幾何?」景仁率易,少周密,對曰:「二十餘事。」復曰:「其中如某事某事十事可行,餘皆無謂也。」明日,上召景仁責之曰:「卿昨言可行者,朕觀之,中復有不可行者。卿謂無謂者,中亦有可行者。朕未嘗使卿分別可否,卿輒專可否,何也?自今戒之。」十年,兼太常卿,學士、同修國史如故。轉承旨,兼修國史。改河南尹。二十一年,召為御史大夫,仍兼承旨、修國史。



 世宗謂景仁曰:「卿博學老儒,求如古之御史大夫,然後行之,期為稱矣。不能如古之人,眾人不獨誚卿,亦謂朕不能知人。卿醉中頗輕脫失
 言,當以酒為戒。」初,朝臣言景仁有文藝而頗率易,不可任臺察。景仁被詔,就臺中治監察罪,輒以便服視決罰。上聞之,責景仁曰:「朕初用卿為大夫,或言卿不可居此官,今果不用故事,率易如此。卿自慎,不然黜罰及矣!」景仁頓首謝。



 未幾,詔葬元妃李氏于海王莊。平章政事烏古論元忠提控葬事,都水監丞高杲壽治道路不如式,元忠不奏,決之四十。景仁劾奏元忠輒斷六品官,無人臣禮。上曰:「卿劾奏甚當。」使左宣徽使蒲察鼎壽傳詔戒敕元忠曰:「監丞六品,有罪聞奏,今乃一切趨辦,擅決六品官,法當如是耶?御史在尊朝廷,汝當自咎,勿復再!」元
 忠尚豫國公主,怙寵自任,倨慢朝士。景仁劾之,朝廷肅然。是歲,薨。



 贊曰:高楨以舊勞為御史大夫,剛明自任,繩治無所避,幾不免於怨憎之荼毒。直己而行,自古難之。白彥敬不受大定之詔而世宗賢之。嚮使久在此位,其深謀讜論,必有竦動人者。張景仁儒者之勇,廷論元忠,正矣。



\end{pinyinscope}