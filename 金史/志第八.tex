\article{志第八}

\begin{pinyinscope}

 河渠



 ○黃河



 金始克宋,兩河悉畀劉豫。豫亡,河遂盡入金境。數十年間,或決或塞,遷徙無定。金人設官置屬,以主其事。沿河上下凡二十五埽,六在河南,十九在河北,埽設散巡河官一員。雄武、滎澤、原武、陽武、延津五埽則兼汴河事,設黃汴都巡河官一員於河陰以蒞之。懷州、孟津、孟州及城北之四埽則兼沁水事,設黃沁都巡河官一員
 於懷州以臨之。崇福上下、衛南、淇上四埽屬衛南都巡河官,則居新鄉。武城、白馬、書城、教城四埽屬浚滑都巡河官,則處教城。曹甸都巡河官則總東明、西佳、孟華、凌城四埽。曹濟都巡河官則司定陶、濟北、寒山、金山四埽者也。故都巡河官凡六員。後又特設崇福上下埽都巡河官兼石橋使。凡巡河官,皆從都水監廉舉,總統埽兵萬二千人,歲用薪百一十一萬三千餘束,草百八十三萬七百餘束,樁杙之木不與,此備河之恆制也。



 大定八年六月,河決李固渡,水潰曹州城,分流於單州之境。九年正月,朝廷遣都水監梁肅往視之。河南統軍使宗室
 宗敘言:「大河所以決溢者,以河道積淤,不能受水故也。今曹、單雖被其患,而兩州本以水利為生,所害農田無幾。今欲河復故道,不惟大費工役,又卒難成功。縱能塞之,他日霖潦,亦將潰決,則山東河患又非曹、單比也。又沿河數州之地,驟興大役,人心動搖,恐宋人乘間構為邊患。」而肅亦言:「新河水六分,舊河水四分,今若塞新河,則二水復合為一。如遇漲溢,南決則害於南京,北決『則山東、河北皆被其害。不若李固南築隄以防決溢為便。」尚書省以聞,上從之。十年三月,拜宗敘為參知政事,上諭之曰:「卿昨為河南統軍時,嘗言黃河隄埽利害,甚合朕意。
 朕每念百姓凡有差調,吏互為姦,若不早計而迫期徵斂,則民增十倍之費。然其所征之物,或委積經年,至腐朽不可復用,使吾民數十萬之財,皆為棄物,此害非細。卿既參朝政,凡類此者皆當革其弊,擇所利而行之。」十一年,河決王村,南京孟、衛州界多被其害。十二年正月,尚書省奏:「檢視官言,水東南行,其勢甚大。可自河陰廣武山循河而東,至原武、陽武、東明等縣孟、衛等州增築隄岸,日役夫萬一千,期以六十日畢。」詔遣太府少監張九思、同知南京留守事紇石烈邈小字阿補孫監護工作。十三年三月,以尚書省請修孟津、滎澤、崇福埽堤以備水
 患,上乃命雄武以下八埽並以類從事。十七年秋七月,大雨,河決白溝。十二月,尚書省奏:「修築河隄,日役夫一萬一千五百,以六十日畢工。」詔以十八年二月一日發六百里內軍夫,并取職官人力之半,余聽發民夫,以尚書工部郎中張大節、同知南京留守事高蘇董役。先是,祥符縣陳橋鎮之東至陳留潘崗,黃河堤道四十餘里以縣官攝其事,南京有司言,乞專設埽官,十九年九月,乃設京埽巡河官一員。二十年,河決衛州及廷津京東埽,彌漫至于歸德府。檢視官南京副留守石抹輝者言:「河水因今秋霖潦暴漲,遂失故道,勢益南行。」宰臣以聞。乃
 自衛州埽下接歸德府南北兩岸增築堤以捍湍怒,計工一百七十九萬六千餘,日役夫二萬四千餘,期以七十日畢工。遂於歸德府創設巡河官一員,埽兵二百人,且詔頻役夫之地與免今年稅賦。二十一年十月,以河移故道,命築堤以備。



 二十六年八月,河決衛州堤,壞其城。上命戶部侍郎王寂、都水少監王汝嘉馳傳措畫備禦。而寂視被災之民不為拯救,乃專集眾以網魚取官物為事,民甚怨嫉。上聞而惡之。既而,河勢泛濫及大名。上於是遣戶部尚書劉瑋往行工部事,從宜規畫,黜寂為蔡州防禦使。冬十月,上謂宰臣曰:「朕聞亡宋河防
 一步置一人,可添設河防軍數。」它日,又曰:「比聞河水泛溢,民罹其害者,貲產皆空。今復遣官於被災路分推排,何耶?」右丞張汝霖曰:「今推排者皆非被災之處。」上曰:「雖然,必其鄰道也。既鄰水而居,豈無驚擾遷避者乎?計其貲產,豈有餘哉!尚何推排為。」十一月,又謂宰臣曰:「河未決衛州時嘗有言者,既決之後,有司何故不令朕知。」命詢其故。



 二十七年春正月,尚書省言:「鄭州河陰縣聖后廟,前代河水為患,屢禱有應,嘗加封號廟額。今因禱祈,河遂安流,乞加褒贈。」上從其請,特加號曰昭應順濟聖后,廟曰靈德善利之廟。



 二月,以衛州新鄉縣令張虡、丞
 唐括唐古出、主簿溫敦偎喝,以河水入城閉塞救護有功,皆遷賞有差。御史臺言:「自來沿河京、府、州、縣官坐視管內河防缺壤,特不介意。若令沿河京、府、州縣長貳官皆於名銜管勾河防事,如任內規措有方能禦大患,或守護不謹以致疏虞,臨時聞奏,以議賞罰。」上從之,仍命每歲將泛之時,令工部官一員沿河檢視。於是以南京府及所屬延津、封丘、祥符、開封、陳留、胙城、杞縣、長垣、歸德府及所屬宋城、寧陵、虞城,河南府及孟津,河中府及河東,懷州河內、武陟,同州朝邑,衛州汲、新鄉、獲嘉、徐州彭城、蕭、豐,孟州河陽、溫,鄭州河陰、滎澤、原武、汜水,浚州
 衛,陜州閿鄉、湖城、靈寶,曹州濟陰,滑州白馬,睢州襄邑,滕州沛,單州單父,解州平陸,開州濮陽,濟州嘉祥、金鄉、鄆城,四府、十六州之長貳皆提舉河防事,四十四縣之令佐皆管勾河防事。初,衛州為河水所壞,乃命增築蘇門,遷其州治。至二十八年,水息,居民稍還,皆不樂遷。於是遣大理少卿康元弼按視之。元弼還奏:「舊州民復業者甚眾,且南使驛道館舍所在,向以不為水備,以故被害。若但修其堤之薄缺者,可以無虞,比之遷治,所省數倍,不若從其民情,修治舊城為便。」乃不遷州,仍敕自今河防官司怠慢失備者,皆重抵以罪。



 二十九年五月,河
 溢於曹州小堤之北。六月,上諭旨有司曰:「比聞五月二十八日河溢,而所報文字如此稽滯。水事最急,功不可緩,稍緩時頃,則難固護矣。」十二月,工部言:「營築河堤,用工六百八萬餘,就用埽兵軍夫外,有四百三十餘萬工當用民夫。」遂詔命去役所五百里州、府差顧,於不差夫之地均徵顧錢,驗物力科之。每工錢百五十交外,日支官錢五十文,米升半。仍命彰化軍節度使內族裔、都水少監大齡壽提控五百人往來彈壓。先是,河南路提刑司言:「沿河居民多困乏逃移,蓋以河防差役煩重故也。竊惟禦水患者,不過堤埽,若土功從實計料,薪槁樁杙
 以時徵斂,亦復何難。今春築堤,都水監初料取土甚近,及其興工乃遠數倍,人夫懼不及程,貴價買土,一隊之間多至千貫。又許州初科薪槁十八萬餘束,既而又配四萬四千,是皆常歲必用之物,農隙均科則易輸納。自今堤埽興工,乞令本監以實計度,量一歲所用物料,驗數折稅,或令和買,於冬月分為三限輸納為便。」詔尚書省詳議以聞。



 明昌元年春正月,尚書省奏:「臣等以為,自今凡興工役,先量負土遠近,增築高卑,定功立限,榜諭使人先知,無令增加力役。并河防所用物色,委都水監每歲於八月以前,先拘籍舊貯物外實闕之數,及次年
 春工多寡,移報轉運司計置,於冬三月分限輸納。如水勢不常,夏秋暴漲危急,則用相鄰埽分防備之物,不足,則復於所近州縣和買。然復慮人戶道塗泥淖,艱于運納,止依稅內科折他物,更為增價,當官支付,違者並論如律,仍令所屬提刑司正官一員馳驛監視體究,如此則役作有程,而河不失備。」制可之。四年十一月,尚書省奏:「河平軍節度使王汝嘉等言:『大河南岸舊有分流河口,如可疏導,足泄其勢,及長堤以北恐亦有可以歸納排瀹之處,乞委官視之。濟北埽以北宜創起月堤。』臣等以為宜從所言。其本監官皆以諳練河防故注以是職,
 當使從汝嘉等同往相視,庶免異議。如大河南北必不能開挑歸納,其月堤宜依所料興修。」上從之。



 十二月,敕都水監官提控修築黃河堤,及令大名府差正千戶一員,部甲軍二百人彈壓勾當。



 五年春正月,尚書省奏:「都水監丞田櫟同本監官講議黃河利害,嘗以狀上言,前代每遇古堤南決,多經南、北清河分流,南清河北下有枯河數道,河水流其中者長至七八分,北清河乃濟水故道,可容三二分而已。令河水趨北,齧長堤而流者十餘處,而堤外率多積水,恐難依元料增修長堤與創築月堤也。可於北岸墻村決河入梁山濼故道,依舊作南、
 北兩清河分流。然北清河舊堤歲久不完,當立年限增築大堤,而梁山故道多有屯田軍戶,亦宜遷徙。今擬先於南岸王村、宜村兩處決堤導水,使長堤可以固護,姑宜仍舊,如不能疏導,即依上開決,分為四道,俟見水勢隨宜料理。」尚書省以櫟等所言與明昌二年劉瑋等所案視利害不同,及令陳言人馮德輿與櫟面對,亦有不合者,送工部議。復言:「若遽於墻村疏決,緣瀕北清河州縣二十餘處,兩岸連亙千有餘里,其隄防素不修備,恐所屯軍戶亦卒難徙。今歲先於南岸延津縣堤決堤洩水,其北岸長堤自白馬以下,定陶以上,並宜加功築護,
 庶可以遏將來之患。若定陶以東三埽棄堤則不必修,止決舊壓河口,引導積水東南行,流堤北張彪、白塔兩河間,礙水軍戶可使遷徙,及梁山濼故道分屯者,亦當預為安置。」宰臣奏曰:「若遽從櫟等所擬,恐既更張,利害非細。比召河平軍節度使王汝嘉同計議,先差幹濟官兩員行戶工部事覆視之,同則就令計實用工物、量州縣遠近以調丁夫,其督趣春工官即充今歲守漲,及與本監官同議經久之利。」詔以知大名府事內族裔、尚書戶部郎中李敬義充行戶工部事,以參知政事胥持國都提控。又奏差德州防禦使李獻可、尚書戶部郎中焦
 旭於山東當水所經州縣築護城堤,及北清河兩岸舊有堤處別率丁夫修築,亦就令講究河防之計。



 他日,上以宋閻士良所述《黃河利害》一帙,付參知政事馬琪曰:「此書所言亦有可用者,今以賜卿。」二月,上諭平章政事守貞曰:「王汝嘉、田櫟專管河防,此國家之重事也。朕比問其曾於南岸行視否?」乃稱:『未也。』又問水決能行南岸乎?又云:『不可知。』且水趨北久矣,自去歲便當經畫,今不稱職如是耶!可諭旨令往盡心固護,無致失備,及講究所以經久之計。稍涉違慢,當併治罪。」三月,行省并行戶工部及都水監官各言河防利害事。都水監元擬於南
 岸王村、宜村兩處開導河勢,緣北來水勢去宜村堤稍緩,唯王村岸向上數里臥捲,可以開決作一河,且無所犯之城市村落。又擬於北岸墻村疏決,依舊分作兩清河入梁山故道,北清河兩岸素有小堤不完,復當築大堤。尚書省謂:「以黃河之水勢,若於墻村決注,則山東州縣膏腴之地及諸鹽場必被淪溺。設使修築壞堤,而又吞納不盡,功役至重,虛困山東之民,非徙無益,而又害之也。況長堤已加護,復於南岸疏決水勢,已寢決河入梁山濼之議,水所經城邑已勸率作護城堤矣,先所修清河舊堤宜遣罷之。監丞田櫟言定陶以東三埽棄
 堤不當修,止言:『決舊壓河口以導漸水入堤北張彪、白塔兩河之間,凡當水衝屯田戶須令遷徙。』臣等所見,止當堤前作木岸以備之,其間居人未當遷徙,至夏秋水勢泛溢,權令避之,水落則當各復業,此亦戶工部之所言也。」上曰:「地之相去如此其遠,彼中利害,安得悉知?惟委行省盡心措畫可也。」四月,以田櫟言河防事,上諭旨參知政事持國曰:「此事不惟責卿,要卿等同心規畫,不勞朕心爾。如櫟所言,築堤用二十萬工,歲役五十日,五年可畢,此役之大,古所未有。況其成否未可知,就使可成,恐難行也。遷徒軍戶四千則不為難,然其水特決,尚
 不知所歸,儻有潰走,若何枝梧。如令南岸兩處疏決,使其水趨南,或可分殺其勢。然水之形勢,朕不親見,難為條畫,雖卿亦然。丞相、左丞皆不熟此,可集百官詳議以行。」百官咸謂:「櫟所言棄長堤,無起新堤,放河入梁山故道,使南北兩清河分流,為省費息民長久之計。臣等以為黃河水勢非常,變易無定,非人力可以斟酌,可以指使也。況梁山濼淤填已高,而北清河窄狹不能吞伏,兼所經州縣農民廬井非一,使大河北入清河,山東必被其害。櫟又言乞許都水監符下州府運司,專其用度,委其任責,一切同於軍期,仍委執政提控。緣今監官已經
 添設,又於外監署司多以沿河州府長官兼領之,及令佐管勾河防,其或怠慢已有同軍期斷罪的決之法,凡櫟所言無可用。」遂寢其議。



 八月,以河決陽武故堤,灌封丘而東,尚書省奏:「都水監、行部官有失固護。」詔命同知都轉運使高旭、武衛軍副都指揮使女奚列奕小字韓家奴同往規措。尚書省奏:「都水監官前來有犯,已經戒諭,使之常切固護。今王汝嘉等殊不加意,既見水勢趨南,不預經畫,承留守司累報,輒為遷延,以至害民。即是故違制旨,私罪當的決。」詔汝嘉等各削官兩階,杖七十罷職。



 上謂宰臣曰:「李愈論河決事,謂宜遣大臣往,以慰人心,
 其言良是。嚮慮河北決,措畫堤防,猶嘗置行省,況今方橫潰為害,而止差小官,恐失眾望,自國家觀之,雖山東之地重於河南,然民皆赤子,何彼此之間。」乃命參知政事馬琪往,仍許便宜從事。上曰:「李愈不得為無罪,雖都水監官非提刑司統攝,若與留守司以便宜率民固護,或申聞省部,亦何不可使朕聞之。徒能張皇水勢而無經畫,及其已決,乃與王汝嘉一往視之而還,亦未嘗有所施行。問王村河口開導之月,則對以四月終,其實六月也,月日尚不知,提刑司官當如是乎?」尋命戶部員外郎何格賑濟被浸之民。時行省參知政事胥持國、馬琪
 言:「已至光祿村周視堤口。以其河水浸漫,堤岸陷潰,至十餘里外乃能取土。而堤面窄狹,僅可數步,人力不可施,雖窮力可以暫成,終當復毀。而中道淤澱,地有高低,流不得泄,且水退,新灘亦難開鑿。其孟華等四埽與孟陽堤道,沿汴河東岸,但可施功者,即悉力修護,將於農隙興役,及凍畢工,則京城不至為害。」參知政事馬琪言:「都水外監員數冗多,每事相倚,或復邀功,議論紛紜不一,隳廢官事。擬罷都水監掾,設勾當官二員。又自昔選用都、散巡河官,止由監官辟舉,皆諸司人,或有老疾,避倉庫之繁,行賄請託,以致多不稱職。擬升都巡河作從
 七品,於應入縣令兼舉人內選注外,散巡河依舊,亦於諸司及丞簿廉舉人內選注,並取年六十以下有精力能幹者。到任一年,委提刑司體察,若不稱職,即日罷之。如守禦有方,致河水安流,任滿,從本監及提刑司保申,量與升除。凡河橋司使副亦擬同此選注。」繼而胥持國亦以為言,乃從其請。閏十月,平章政事守貞曰:「馬琪措畫河防事,未見功役之數,加之積歲興功,民力將困,今持國復病,請別遣有材幹者往議之。」上曰:「堤防救護若能成功,則財力固不敢惜。第恐財殫力屈,成而復毀,如重困何?」宰臣對曰:「如盡力固護,縱為害亦輕,若恬然不
 顧,則為害滋甚。」上曰:「無乃因是致盜賊乎?」守貞曰:」宋以河決興役,亦嘗致盜賊,然多生於凶歉。今時平歲豐,少有差役,未必至此。且河防之役,理所當然,今之當役者猶為可耳。至於科征薪芻,不問有無,督輸迫切則破產業以易之,恐民益困耳。」上曰:「役夫須近地差取,若遠調之,民益艱苦,但使津濟可也。然當俟馬琪至而後議之。」庚辰,琪自行省還,入見,言:「孟陽河堤及汴堤已填築補修,水不能犯汴城。自今河勢趨北,來歲春首擬於中道疏決,以解南北兩岸之危。凡計工八百七十餘萬,可於正月終興工。臣乞前期再往河上監視。」上以所言付尚
 書省,而治檢覆河堤並守漲官等罪有差。他日,尚書省奏事,上語及河防事,馬琪奏言:「臣非敢不盡心,然恐智力有所不及。若別差官相度,儻有奇畫,亦未可知。如適與臣策同,方來興功,亦庶幾稍寬朝延憂顧。」上然之,命翰林待制奧屯忠孝權尚書戶部侍郎、太府少監溫昉權尚書工部侍郎,行戶、工部事,修治河防,且諭之曰:「汝二人皆朕所素識,以故委任,冀副朕意。如有錯失,亦不汝容。」



 承安元年七月,敕自今沿河傍側州、府、縣官雖部除者皆勿令員闕。泰和二年九月,敕御史臺官:「河防利害初不與卿等事,然臺官無所不問,應體究者亦體究
 之。」五年二月,以崔守真言:「黃河危急,芻槁物料雖云折稅,每年不下五六次,或名為和買,而未嘗還其直。」敕委右三部司正郭澥、御史中丞孟鑄講究以聞。澥等言:「大名府、鄭州等處自承安二年以來,所科芻槁未給價者,計錢二十一萬九千餘貫。」遂命以各處見錢差能幹官同各州縣清強官一一酬之,續令按察司體究。



 宣宗貞祐三年十一月壬申,上遣參知政事侯摯祭河神於宜村。三年四月,單州刺史顏盞天澤言:「守禦之道,當決大河使北流德、博、觀、滄之境。今其故堤宛然猶在,工役不勞,水就下必無漂沒之患。而難者若不以犯滄鹽場損國利為
 說,必以浸沒河北良田為解。臣嘗聞河側故老言,水勢散漫,則淺不可以馬涉,深不可以舟濟,此守禦之大計也。若日浸民田,則河徙之後,淤為沃壤,正宜耕墾,收倍于常,利孰大焉。若失此計,則河南一路兵食不足,而河北、山東之民皆瓦解矣!」詔命議之。四年三月,延州刺史溫撒可喜言:「近世河離故道,自衛東南而流,由徐、邳入海,以此,河南之地為狹。臣竊見新鄉縣西河水可決使東北,其南有舊隄,水不能溢,行五十餘里與清河合,則由浚州、大名、觀州、清州、柳口入海,此河之故道也,皆有舊隄,補其缺罅足矣!如此則山東、大名等路,皆在河南,
 而河北諸郡亦得其半,退足以為禦備之計,進足以壯恢復之基。」又言:「南岸居民,既已籍其河夫修築河堰,營作戍屋,又使轉輸芻糧,賦役繁殷,倍於他所,夏秋租稅,猶所未論,乞減其稍緩者,以寬民力。」事下尚書省,宰臣謂:「河流東南舊矣。一旦決之,恐故道不容,衍溢而出,分為數河,不復可收。水分則淺狹易渡,天寒輒凍,禦備愈難,此甚不可!」詔但令量宜減南岸郡縣居民之賦役。五年夏四月,敕樞密院,沿河要害之地,可壘石岸,仍置撒星樁、陷馬塹以備敵。



 ○漕渠



 金都於燕,東去潞水五十里,故為閘以節高良河、
 白蓮潭諸水,以通山東、河北之粟。凡諸路瀕河之城,則置倉以貯傍郡之稅,若恩州之臨清、歷亭,景州之將陵、東光,清州之興濟、會川,獻州及深州之武強,是六州諸縣皆置倉之地也。其通漕之水,舊黃河行滑州、大名、恩州、景州、滄州、會州之境,漳水東北為御河,則通蘇門、獲嘉、新鄉、衛州、浚州、黎陽、衛縣、彰德、磁州、洺州之饋,衡水則經深州會于滹沱,以來獻州、清州之餉,皆合于信安海壖。溯流而至通州,由通州入閘,十餘日而後至于京師。其它若霸州之巨馬河,雄州之沙河,山東之北清河,皆其灌輸之路也。然自通州而上,地峻而水不留,其勢
 易淺,舟膠不行,故常徙事陸挽,人頗艱之。世宗之世,言者請開盧溝金口以通漕運,役眾數年,竟無成功,事見《盧溝河》。其後亦以閘河或通或塞,而但以車挽矣。其制,春運以冰消行,暑雨畢。秋運以八月行,冰凝畢。其綱將發也。乃合眾,以所載之粟苴而封之,先以付所卸之地,視與所封樣同則受。凡綱船以前期三日修治,日裝一綱,裝畢以三日啟行。計道里分溯流、沿流為限,至所受之倉,以三日卸,又三日給收付。凡挽漕腳直,水運鹽每石百里四十八文,米五十文一分二釐七毫,粟四十文一分三毫,錢則每貫一文七分二釐八毫。陸運傭直,米每
 石百里百一十二文一分五毫,粟五十七文六分八釐四毫,錢每貫三文九釐六毫。餘物每百斤行百里,平路則春冬百三十一文五分,夏秋百五十七文八分,山路則春冬百四十九文,夏秋二百一文。凡使司院務納課傭直,春冬九十文三分,夏秋百一十四文。諸民戶射賃官船漕運者,其腳直以十分為率,初年克二分,二年克一分八釐,三年克一分七釐,四年剋一分五釐,五年以上剋一分。



 初,世宗大定四年八月,以山東大熟,詔移其粟以實京師。十月,上出近郊,見運河湮塞,召問其故。主者云戶部不為經畫所致。上召戶部侍郎曹望之,責曰:「
 有河不加浚,使百姓陸運勞甚,罪在汝等。朕不欲即加罪,宜悉力使漕渠通也。」五年正月,尚書省奏,可調夫數萬,上曰:「方春不可勞民,令宮籍監戶、東宮親王人從、及五百內里軍夫浚治。」二十一年,以八月京城儲積不廣,詔沿河恩獻等六州粟百萬餘石運至通州,輦入京師。明昌三年四月,尚書省奏:「遼東、北京路米粟素饒,宜航海以達山東。昨以按視東京近海之地,自大務清口並咸平銅善館皆可置倉貯粟以通漕運,若山東、河北荒歉,即可運以相濟。」制可。承安五年,邊河倉州縣,可令折納菽二十萬石,漕以入京,驗品級養馬於俸內帶支,仍
 漕麥十萬石,各支本色。乃命都水監丞田櫟相視運糧河道。



 泰和元年,尚書省以景州漕運司所管六河倉,歲稅不下六萬餘石,其科州縣近者不下二百里,官吏取賄延阻,人不勝苦,雖近官監之亦然。遂命監察御史一員往來糾察之。五年,上至霸州,以故漕河淺澀,敕尚書省發山東、河北、河東、中都、北京軍夫六千,改鑿之。犯屯田戶地者,官對給之。民田則多酬其價。六年,尚書省以凡漕河所經之地,州縣官以為無與於己,多致淺滯,使綱戶以盤淺剝載為名,姦弊百出。於是遂定制,凡漕河所經之地,州府官銜內皆帶「提控漕河事」,縣官則帶「管
 勾漕河事」,俾催檢綱運,營護堤岸。為府三:大興、大名、彰德。州十二:恩、景、滄、清、獻、深、衛、濬、滑、磁、洺、通。縣三十三:大名、元城、館陶、夏津、武城、歷亭、監清、吳橋、將陵、東光、南皮、清池、靖海、興濟、會川、交河、樂壽、武強、安陽、湯陰、監漳、成安、滏陽、內黃、黎陽、衛、蘇門、獲嘉、新鄉、汲、潞、武清、香河、漷陰。



 十二月,通濟河創設巡河官一員,與天津河同為一司,通管漕河閘岸,止名天津河巡河官,隸都水監。八年六月,通州刺史張行信言:「船自通州入閘,凡十餘日方至京師,而官支五日轉腳之費。」遂增給之。



 貞祐三年,既遷於汴,以陳、潁二州瀕水,欲借民船以漕,不便。遂依觀
 州漕運司設提舉官,募船戶而籍之,命戶部勾當官往來巡督。四年,從右丞侯摯言,開沁水以便餽運。上又念京師轉輸之勞,命出尚廄牛及官車,以助其力。興定四年十月,諭皇太子曰:「中京運糧護送官,當擇其人,萬有一失,樞密官亦有罪矣!其船當用毛花輦所造兩首尾者,仍張幟如渡軍之狀,勿令敵知為糧也。」陜西行省把胡魯言:」陜西歲運糧以助關東,民力浸困,若以舟自渭入河,順流而下,可以紓民力。」遂命嚴其偵候,如有警,則皆維於南岸。時朝延以邳、徐、宿、泗軍儲,京東縣挽運者歲十餘萬石,民甚苦之。元光元年,遂於歸德府置通濟
 倉,設都監一員,以受東郡之粟。定國軍節度使李復亨言:「河南駐蹕,兵不可闕,糧不厭多,比年,少有匱乏即仰給陜西,陜西地腴歲豐,十萬石之助不難。但以車運之費先去其半,民何以堪?宜造大船二十,由大慶關渡入河,東抵湖城,往還不過數日,篙工不過百人,使舟皆容三百五十斛,則是百人以數日運七千斛矣!自夏抵秋可漕三千餘萬斛,且無稽滯之患。」上從之。時又於靈璧縣潼郡鎮設倉都監及監支納,以方開長直溝,將由萬安湖舟運入汴至泗,以貯粟也。



 ○盧溝河



 大定十年,議決盧溝以通京師漕運,上忻然曰:「
 如此,則諸路之物可徑達京師,利孰大焉!」命計之,當役千里內民夫,上命免被災之地,以百官從人助役。已而,敕宰臣曰:「山東歲饑。工役興則妨農作,能無怨乎?開河本欲利民,而反取怨,不可!其姑罷之。」十一年十二月,省臣奏復開之,自金口疏導至京城北入壕,而東至通州之北,入潞水,計工可八十日。十二年三月,上令人覆按,還奏:「止可五十日。」上召宰臣責曰:「所餘三十日徒妨農費工,卿等何為慮不及此。」及渠成,以地勢高峻,水性渾濁。峻則奔流漩洄,齧岸善崩,濁則泥淖淤塞,積滓成淺,不能勝舟。其後,上謂宰臣曰:「分盧溝為漕渠,竟未見功,
 若果能行,南路諸貨皆至京師,而價賤矣。」平章政事駙馬元忠曰:「請求識河道者,按視其地。」竟不能行而罷。二十五年五月,盧溝決於上陽村。先是,決顯通寨,詔發中都三百里內民夫塞之,至是復決,朝延恐枉費工物,遂令且勿治。二十七年三月,宰臣以「孟家山金口閘下視都城,高一百四十餘尺,止以射糧軍守之,恐不足恃。儻遇暴漲,人或為姦,其害非細。若固塞之,則所灌稻田俱為陸地,種植禾麥亦非曠土。不然則更立重閘,仍於岸上置埽官廨署,及埽兵之室,庶幾可以無虞也」。上是其言,遣使塞之。夏四月丙子,詔封盧溝水神為安平侯。二
 十八年五月,詔盧溝河使旅往來之津要,令建石橋。未行而世宗崩。章宗大定二十九年六月,復以涉者病河流湍急,詔命造舟,既而更命建石橋。明昌三年三月成,敕命名曰廣利。有司謂車駕之所經行,使客商旅之要路,請官建東西廊,令人居之。上曰:「何必然,民間自應為耳。」左丞守貞言:「但恐為豪右所占,況罔利之人多止東岸,若官築則東西兩岸俱稱,亦便於觀望也。」遂從之。



 六月,盧溝隄決,詔速遏塞之,無令泛溢為害。右拾遺路鐸上疏言:「當從水勢分流以行,不必補修玄同口以下、丁村以上舊堤。」上命宰臣議之,遂命工部尚書胥持國及
 路鐸同檢視其隄道。



 ○滹沱河



 大定八年六月,滹沱犯真定,命發河北西路及河間、太原、冀州民夫二萬八千,繕完其隄岸。十年二月,滹沱河創設巡河官二員。十七年,滹沱決白馬崗,有司以聞,詔遣使固塞,發真定五百里內民夫,以十八年二月一日興役,命同知真定尹鶻沙虎、同知河北西路轉運使徐偉監護。



 漳河



 大定二十年春正月,詔有司修護漳河閘,所須工物一切並從官給,毋令擾民。明昌二年六月,漳河及盧溝隄皆決,詔命速塞之。四年春正月癸未,有司言修漳
 河堤埽計三十八萬餘工,詔依盧溝河例,招被水闕食人充夫,官支錢米,不足則調礙水人戶,依上支給。



\end{pinyinscope}