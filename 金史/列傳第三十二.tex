\article{列傳第三十二}

\begin{pinyinscope}

 ○夾谷清臣內族襄夾谷沖完顏安國瑤里孛迭



 夾谷清臣,本名阿不沙,胡里改路桓篤人也。姿狀雄偉,善騎謝。皇統八年,襲祖駮達猛安。大定元年,聞世宗即位,率本部軍六千赴中都會之,以功遷昭武大將軍。從右副元帥紇石烈志寧為管押萬戶,接應左都監完顏思敬,逐窩斡餘黨,敗之柔遠,至抹拔里達悉獲之。賊平,
 遷鎮國上將軍,知潁順軍事。會宋兵二萬襲陷汝州,殺刺史烏古孫麻發及漢軍二千。河南統軍宗尹遣萬戶孛術魯定方與清臣等領騎兵四千往擊之。宋人棄城遁,遂復汝州。三年五月,從志寧復取宿州,宋將李世輔大敗遁去,志寧復遣清臣等以兵追襲,又敗之。捷聞,授宿州防禦使。移博州,改西北路招討都監,遷烏古十壘部族節度使。十二年,授右副都點檢,遷左副都點檢,出為陜西路統軍使,兼知京兆府事。朝辭,賜以金帶廄馬,仍諭之曰:「卿典禁兵,日侍左右,勤勞久矣,故以是授卿,宜益思勉。」二十六年,改西京留守。閱三歲,遷樞密副使。
 明昌元年,初議出師,以本職充東北路兵馬都統制使,既而詔止之。俄以其女為昭儀,眷倚益重。二年,拜尚書左丞。頃之,進平章政事,封芮國公,賜同本朝人。四年,遷右丞相,監修國史。



 時議簽軍戍邊,上問:「漢人與夏人孰勇?」清臣曰:「漢人勇。」上曰:「昔元昊擾邊,宋終不能制,何也?」清臣曰:「宋馭軍法不可得知,今西南路人殊勝彼也。」未幾,遷崇進,改封戴。一日,上謂宰臣曰:「人有以《八陣圖》來上者,其圖果何如?朕嘗觀宋白所集《武經》,然其載攻守之法亦多難行。」清臣曰:「兵書皆定法,難以應變。本朝行兵之術,惟用正奇二軍,臨敵制變,以正為奇,以奇為正,
 故無往不克。」上曰:「自古用兵亦不出奇正二法耳。且學古兵法如學弈棋,未能自得於心,而欲用舊陣勢以接敵,亦以疏矣。」



 尋上表丐閑,不許。固請,乃賜告省親,諭之曰:「聞卿母老,欲令歸省,故特給假五十日,馳驛以往,至彼可為一月留也。」五年二月,上御凝和殿,清臣省覲還,謁上。上問:「卿母健否?其壽幾何?相別幾年矣?」清臣對曰:「臣母年八十三矣,別十年,幸頗強健。」上曰:「何不來此?」曰:「急於家務,故不欲離耳。」上曰:「老人多如是,所謂『血氣既衰,戒之在得』也。」復謂清臣:「胡里改路風俗何如?」對曰:「視舊則稍知禮貌,而勇勁不及矣。」因言西南、西北等路軍
 人,其閑習弓矢,亦非復曩時。



 六年,遷儀同三司,進拜左丞相,改封密。受命出師,行尚書省事於臨潢府。清臣遣人偵知虛實,以輕騎八千,令宣徽使移剌敏為都統,左衛將軍充、招討使完顏安國為左右翼,分領前隊,自選精兵一萬以當後隊。進至合勒河,前隊敏等於栲栳濼攻營十四,下之,回迎大軍,屬部斜出掩其所獲羊馬資物以歸。清臣遣人責其賧罰,北阻珝由此叛去,大侵掠。上遣責清臣,命右丞相襄代之。承安五年,降授橫海軍節度使兼滄州管內觀察使。



 初,上諭宰臣曰:「清臣舊有勞效,罪狀未甚明,若降授,應須告致仕耳。」初擬知廣寧
 府,上曰:「姑與滄州。」既而又曰:「與則與之,第恐有人言也。」尋復致仕。泰和二年薨,年七十。子麼查剌襲猛安。初議征討,清臣主其事,既而領軍出征,雖屢獲捷,而貪小利,遂致北邊不寧者數歲,天下尤之。


丞相襄,本名唵,昭祖五世孫也。祖什古迺從太祖平遼,以功授上京世襲猛安,歷東京留守。父阿魯帶,皇統初北伐有功,拜參知政事。襄幼有志節,善騎射,多勇略,年十八襲世爵。大定初,契丹叛,從左副元帥謀衍以本部兵討賊,戰于肇州之長濼。襄先登鏖擊,足中流矢,裹創以戰,氣愈厲,七戰皆勝。謀衍握其手曰:「今日之捷,皆公
 力也。」賊走渡霿
 \gezhu{
  松}
 河,追及之。所駐地多草,賊乘風縱火,襄亦縱火,立空地以俟。戰十餘合,賊益困。襄謂謀衍曰:「今不乘此平殄,後將有悔。」謀衍然之。襄率眾搏戰。大敗之,俘獲萬計。會朝廷遣平章政事僕散忠義代謀衍將,襄復從忠義追賊至裊嶺西之陷泉,及之,率右翼身先奮擊,賊大潰,人馬相蹂而死,陷泉幾平。賊酉窩斡僅與數十騎遁去,卒就擒,論功為第一。有司擬淄州刺史,詔特授亳州防禦使,時年二十三。



 宋人犯南鄙,襄為潁、壽都統,率甲士二千人渡潁水,敗敵兵五千,復潁州,生擒宋帥楊思。次濠州,宋將郭太尉退保橫澗山,襄攻之,伏
 弩射中其膝,督攻愈急,拔之,獲郭太尉。既而趨滁州,襄為先鋒,將至清流關,得宋偵者,知敵欲三道夜出,掩我不備。左副元帥紇石烈志寧問計。襄曰:「今兵少地隘,儻不得關,敵至,我無所據,必先取之。」曰:「我與若孰往?」襄曰:「元帥國家大臣,詎宜輕動?襄當為公往取。」志寧韙之。襄率騎二千,分二道,一由衝路,自以千兵間道潛登。既近,敵始覺。襄攻克之,據其關,志寧履行戰地,顧謂曰:「克敵於不可勝之地,真天下英傑也。」及宋乞盟,班師,召為拱衛直都指揮使,改殿前右衛將軍,轉左衛,出為東北路招討都監,遷速頻路節度使,移曷懶路兵馬都總管。



 左
 丞相志寧疾甚,世宗臨問之,志寧薦襄「智勇兼濟,有經世才,他人莫及,異時任用,殆勝于臣」。即召授殿前左副都點檢。為宋生日使,宋方祈免親接國書,襄至,宋人屢來議,皆折之,迄成禮而還。授陜西路統軍使,賜之尚服、廄馬、鞍勒、佩刀。改河南統軍使。



 入為吏部尚書,轉都點檢,賜錢千萬。世宗謂宰執曰:「襄為人甚蘊藉,非直日,亦入宮規畫諸事,事有所付乃退,其公勤如此。若襄之才豈多得哉!」擢御史大夫。踰月,拜尚書右丞,諭之曰:「卿在河南經制邊事,甚有統紀,及在吏部,至為點檢,尤奉公守法,朕甚嘉之。近長憲臺,亦以剛直聞,是用委以機政,
 其益勉之!」未幾,進拜左丞。襄在外任,治有異效,至是朝廷以褒賞廉吏詔天下,列其名以示獎勵。二十三年,進拜平章政事,封蕭國公。



 世宗以金源郡王世嫡皇孫,將加王爵,詔擇國號。襄曰:「為天下大計,必先正其本,原者本也,請封原。」從之。故事,諸部族節度使及其僚屬多用颭人,而頗有私縱不法者,議改用諸色人。襄曰:「北邊雖無事,恒須經略之,若杜此門,其後有勞績,何以處之?請如舊。」他日,議及古有監軍之事。襄曰:「漢、唐初無監軍,將得專任,故戰必勝,攻必克。及叔世始以內臣監其軍,動為所制,故多敗而少功。若將得其人,監軍誠不必置。」並
 嘉納之。詔受北部進貢。使還,世宗問邊事,具圖以進,因上羈縻屬部、鎮服大石之策,詔悉行之。進拜右丞相,徙封戴。



 世宗不豫,與太尉徒單克寧、平章政事張汝霖宿內殿,同受顧命。章宗初即政,議罷僧道奴婢。太尉克寧奏曰:「此蓋成俗日久,若遽更之,於人情不安。陛下如惡其數多,宜嚴立格法,以防濫度,則自少矣。」襄曰:「出家之人安用僕隸?乞不問從初如何所得,悉放為良。若寺觀物力元係奴婢之數推定者,並合除免。」詔從襄言。由是二稅戶多為良者。



 明昌元年,同知棣州防禦使鷿上封事,歷詆宰執。太傅克寧奏,膏所言襄預知之。於是詔鷿
 還本猛安,而襄出知平陽府事。移知鳳翔,歷西京留守,召授同判大睦親府事,進樞密使,復拜右丞相,改封任。時左丞相夾谷清臣北禦邊,措畫乖方,屬邊事急,命襄代將其眾,佩金牌,便宜從事。臨宴慰遣,賜以貂裘、鞍山、細鎧及戰馬二。時胡里颭亦叛,嘯聚北京、臨潢之間。襄至,遣人招之,即降,遂屯臨潢。頃之,出師大鹽濼,復遣右衛將軍完顏充進軍斡魯速城,欲屯守,俟隙進兵。繪圖以聞,議者異同,即召面論,厚賜遣還。



 未幾,遣西北路招討使完顏安國等趨多泉子。密詔進討,乃命支軍出東道,襄由西道。而東軍至龍駒河為阻珝所圍,三日不得
 出,求援甚急,或請俟諸軍集乃發。襄曰:「我軍被圍數日,馳救之猶恐不及,豈可後時?」即鳴鼓夜發。或請先遣人報圍中,使知援至。襄曰:「所遣者儻為敵得,使知我兵寡而糧在後,則吾事敗矣。」乃益疾馳。遲明,距敵近,眾請少憩。襄曰:「吾所以乘夜疾馳者,欲掩其不備爾。緩則不及。」嚮晨壓敵,突擊之,圍中將士亦鼓噪出,大戰,獲輿帳牛羊。眾皆奔斡里札河。遣安國追躡之。眾散走,會大雨,凍死者十八九,降其部長,遂勒勳九峰石壁。捷聞,上遣使厚賜以勞之,別詔許便宜賞賚士卒。九月,赴闕,拜左丞相,監修國史,封常山郡王。宴慶和殿,上親舉酒飲,解所
 服玉具佩刀以賜,俾即服之。



 十月,阻珝復叛,襄出屯北京,會群牧契丹德壽、陀鎖等據信州叛,偽建元曰身聖,眾號數十萬,遠近震駭。襄閑暇如平日,人心乃安。初,襄之出鎮也,至石門鎮,密謂僚屬曰:「北部犯塞奚足慮。第恐姦人乘隙而動。北京近地軍少,當預為之備。」即遣官發上京等軍六千,至是果得其用。臨潢總管烏古論道遠、咸平總管蒲察守純分道進討,擒德壽等送京師。



 契丹之亂,廷臣議罷郊祀,又欲改用正月上辛,上遣使問之,對曰:」郊為重禮,且先期詔天下,又籓國已報表賀,今若中罷,何以副四方傾望之意?若改用正月上辛,乃祈
 穀之禮,非郊見上帝之本意也。大禮不可輕廢,請決行之,臣乞於祀前滅賊。」既而賊破,果如所料。郊禮成,進封南陽郡王。始討契丹,自龍虎衛上將軍、節度使以下許承制授之。襄以為賞罰之柄非人臣所預,不敢奉詔。賊平,請委近臣諭旨將士,使知上恩。乃遣李仁惠持宣三十、敕百五十,視功給之。



 方德壽之叛,諸颭亦剽略為民患,襄慮其與之合,乃移諸颭居之近京地,撫慰之。或曰:「颭人與北俗無異,今置內地,或生變奈何?」襄笑曰:「颭雖雜類,亦我之邊民,若撫以恩,焉能無感?我在此,必不敢動。」後果無患。尋詔參知政事裔代領其軍。入見,賜錢五
 千萬。明年,以內艱免。翌日,起復視事。時議以契丹戶之驅奴尚眾,乞盡鬻以散其黨,襄以為非便,奏請量存口數,餘悉官贖為良,上納之。



 北部復叛,裔戰失律,復命襄為左副元帥蒞師,尋拜樞密使兼平章政事,屯北京。民方艱食,乃減價出糶倉粟以濟之。或以兵食方闕為言,襄曰:「烏有民足而兵不足者?」卒行之,民皆悅服。時議北討,襄奏遣同判大睦親府事宗浩出軍泰州,又請左丞衡於撫州行樞密院,出軍西北路以邀阻珝,而自帥兵出臨潢。上從其策,賜內庫物即軍中用之。其後斜出部族詣撫州降,上專使問襄,襄以為受之便。賜寶劍,詔度
 宜窮討。乃令士自齎糧以省挽運,進屯於沔移剌烈、烏滿掃等山以逼之。因請就用步卒穿壕築障,起臨潢左界北京路以為阻塞。言者多異同,詔問方略。襄曰:「今茲之費雖百萬貫,然功一成則邊防固而戍兵可減半,歲省三百萬貫,且寬民轉輸之力,實為永利。」詔可。襄親督視之,軍民並役,又募饑民以傭即事,五旬而畢。於是西北、西南路亦治塞如所請。無何,泰州軍與敵接戰,宗浩督其後,殺獲過半,諸部相率送款,襄納之。自是北陲遂定。



 襄還臨潢,減屯兵四萬、馬二萬疋。上以信符召還,遣近臣迎勞于途。既至,復撫問于第,入獻邊機十事,皆為
 施行,仍厚賜之,復拜左丞相。初,襄至自軍,上諭宰臣曰:「樞密使襄築立邊堡完固。古來立一城一邑,尚有賞賚,即欲拜三公,三公非賞功官,如左丞相亦非賞功者,雖然可特授之。」遣左司郎中阿勒根阿海降詔褒諭。四年正月,進拜司空,領左丞相如故。



 襄重厚寡言,務以鎮靜守法。每掾有所稟,必問曰:「諸相云何?」掾對某相如是,某相如是。襄曰:「從某議。」其事無有異者。識者謂襄誠得相體。時上頗更定制度,初置提刑司,又議設清閑職位,如宋朝宮觀使,以待年高致仕之官。襄言:「年老致仕,朝廷養以俸廩,恩禮至渥。老不為退,復有省會之法,所以抑
 貪冒,長廉節。若擬別設,恐涉於濫。」又言:「省事不如省官,今提刑官吏,多無益於治,徒亂有司事。議者以謂斯乃外臺,不宜罷。臣恐混淆之辭,徒煩聖聽。且憲臺所掌者察官吏非違,正下民冤枉,亦無提點刑獄、舉薦之權。若已設難以遽更,其採訪廉能不宜隸本司,宜令監察御史歲終體究,仍不時選官廉訪。」上皆聽納。俄乞致仕,不許。



 時方旱,命有司祈雨,襄及平章政事張萬公、參政僕散揆等上表待罪。上召翰林學士黨懷英草罪己詔,仍慰諭襄等視事。泰和元年春,承命馳禱于亳州太清宮及后土方嶽。以其世封遠,特改授河間府路算術海猛
 安。明年,皇子生,襄復自請報謝。既祀嵩嶽,還次芝田之府店,遂以疾薨,年六十三。訃聞,輟朝,遣使祭于路,葬禮依太師淄王克寧。謚曰武昭。命張行簡銘其碑。



 襄明敏,才武過人,上親待之厚,故所至有功。其駐軍臨潢也,有以偽書遺西京留守徒單鎰,欲構以罪。書聞,上以書還畀襄,其明信如此。既而果獲為偽書者。在政府二十年,明練故事,簡重能斷,器局尤寬大,待掾吏盡禮,用人各得所長,為當世名將相。大安間,配享章宗廟廷。



 夾谷衡,本名阿里不,山東西路三土猛安益打把謀克人也。大定十三年,創設女直進士舉,衡中第四人,補東
 平府教授。調范陽簿,選充國史院編修官,改應奉翰林文字。世宗嘗謂宰臣曰:「女直進士中才傑之士蓋亦難得,如徒單鎰、夾谷衡、尼龐古鑒,皆有用材也。」遷修起居注。章宗立,為侍御史,轉右司員外郎,敷奏稱旨,升左司郎中。明昌二年,擢御史中丞,未幾,拜參知政事。三年八月,以病,表乞致仕,詔撫慰不許。



 衡久在告,承詔始出,上見其羸瘠,復賜告一月。四年,詔賜今名,諭之曰:「朕選大臣,俾參機務,必資謀畫,協贊治平。其或得失晦而未形,利害膠而未決,正須識見純直,方能去取合公。比來議事之臣,鮮有一定之論,蓋以內無所守,故臨事而惑,致
 有中失,朕將何賴?卿忠實公方,審其是則執而不回,見其非則去而能果,度其事勢,有若權衡。汝之所長,衡實似之,可賜名衡。古者命名,將以責實,汝先有實,可謂稱名,行之克終,乃副朕意。」



 參知政事胥持國言區種法。衡曰:「若茍有利,古已行之,且用功多而所種少,復恐荒廢土田,徒勞民,無益也。」進尚書右丞。舊制,久歷隨朝職任者,得奉使江表。衡未使而拜執政,特賜錢六千貫。六年,遷尚書左丞,尋出行省于撫州。洎還入朝,聞父憂去,上亟召回,起復本職。承安二年,出為上京留守,尋改樞密副使,行院規畫邊事。三年,以修完封界,賜詔褒諭。四年
 正月,就拜平章政事,封英國公。薨,年五十一。上聞之惻然,為輟朝,命官致祭,賻贈有加。遣使敕葬,謚曰貞獻。



 完顏安國,字正臣,本名闍母。其先占籍上京,世有戰功。祖斜婆,授西南路世襲合札謀克。安國沉雄有謀畫,尤善騎射。正隆元年,從軍為謀克,常以少擊眾。大定中,為常山簿,轉虹縣令。會王府新建,選充虞王府掾。再遷儀鸞局副使。明昌元年,改本局使。會大石部長有乞修歲貢者,朝廷許其請,詔安國往使之。至則率眾遠迓至帳,望闕羅拜,執禮無惰容。



 時北阻珝迫近塞垣,鄰部欲立功以誇雄上國,議邀安國俱行討之。安國以未奉詔為
 辭。強之,不可。或以危言怵之,安國曰:「大丈夫豈以生死易節。暴骨邊庭,不猶愈於病死牖下。」眾壯其言,饋贐如禮。既還,以奉使稱旨,升武衛軍都指揮使。出為東北路副招討,未赴,改西北路副招討。



 六年,左丞相夾谷清臣用兵,以安國為先鋒都統。適臨潢、泰州屬部叛,安國先討定之,以功遷本路招討使,兼威遠軍節度使。承安元年,大鹽濼之戰,殺獲甚眾,詔賜金幣。既而右丞相襄總大軍進,安國為兩路都統,大捷於多泉子。襄遣安國追敵,僉言糧道繼,不可行也。安國曰:「人得一羊可食十餘日,不如驅羊以襲之便。」遂從其計。安國統所部萬人
 疾驅以薄之,降其部長。捷聞,進官四級,遷左翼都統。



 承安二年,以營邊堡功,召簽樞密院事。賜虎符還邊,得以便宜從事。時並塞諸部降,諭使輸貢如初。進拜樞密副使。泰和元年,特授世襲西南路延晏河猛安,兼合札謀克。帝幸慶寧宮,命安國嚴飭邊備。奏西南路邊戍私竄者乞招誘以安人心,上是其言。三年,以疾致仕,封道國公。四年,起復前職,卒。上聞之,輟朝。敕有司葬以執政禮。贈特進。



 安國在軍旅幾十五年,號令嚴明,指麾卒伍如左右手。又善伺知敵人虛實及山川險易,戰必身先士卒,故所向輒克。諸部入貢,安國能一一呼其祖先弟姪
 名字以戒諭之,諸部皆震悚,甚為鄰國所畏服。



 瑤里孛迭,北京路窟白猛安陀羅山謀克人也。以軍功歷海濱令,遷徐王府掾。以稱職,再任御史臺。察廉,升同知震武軍節度使事。明昌初,為唐州刺史,尋授西北路招討副使。未幾,改東北路。六年正月,北邊有警,聚兵圍慶州急,孛迭率本路軍往救,敵解去,州竟無患。承安元年,丞相襄北伐,孛迭為先鋒副統,進軍至龍駒河,受圍,會襄引大軍至,得解。後授鎮寧軍節度使,以六群牧人叛,改寧昌軍。孛迭為都統,領步騎萬次懿州,敵數萬來逆戰,兵勢甚張,孛迭親陷陣,奮力鏖擊卻之,身中二創,
 捷聞,遷一官。承安二年,颭軍千餘出沒剽掠錦、懿間,孛迭追敗之,復獲所掠,悉還本戶。三年,從同判大睦親府事宗浩為左翼都統,戰移密河,勝;戰骨堡子西,殺獲甚眾。五年,授知廣寧府事,俄改東北路招討使。以捍邊有功,賜詔褒諭,三遷為崇義軍節度使。泰和六年,卒。訃聞,遣官致祭,賜銀五百兩,贈金紫光祿大夫。



 孛迭勇決善戰,自幼以軍功顯,任兵鎮十餘年,所向克捷,凡再遷官,賜金幣,甚為上倚注云。



 贊曰:《易·師》之初六:「師出以律,否臧兇。」蓋初為師之始,出師之道,當慎其始。清臣首議出師,遽以貪小利敗。襄雖
 賢,竭力而後勝其任。衡、安國、孛迭之功又亞於襄者也。然而兵連禍結,以終金世。故兵無常勝,制勝在勢。勢制兵者強,兵制勢者亡。跡襄之開築壕塹以自固,其猶元魏、北齊之長城歟?金之勢可知矣。勢屈而兵勝,亡國之道也。金以兵始,亦以兵終。嗚呼!用兵之始,可不慎歟,可不慎歟!



\end{pinyinscope}