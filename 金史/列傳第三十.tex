\article{列傳第三十}

\begin{pinyinscope}

 ○毛碩李上達曹望之大懷貞盧孝儉盧庸李偲徒單克寧本名習顯



 毛碩,字仲權,甘陵人。宋末,試弓馬子弟,碩中選,調高陽關路安撫司準備差使。尋辟河間尉,再辟兵馬都監。宗望軍至,碩以本部迎降。齊國建,由淮東路第一副將擢知滑州。劉麟伐宋,充行營中軍統制軍馬。天眷間,歷汴京路、山東西路兵馬都監。皇統元年,權知拱州。宋將張
 俊據亳州,而柘城酒監房人傑叛以應俊,碩發兵討之。至柘城,躬扣城門,呼耆老以諭意。縣人縛人傑以降。碩徑入縣署,召百姓慰安之,眾皆感悅,刻石紀其事。四年,真授拱州刺史。元帥梁王宗弼承制超武義將軍,改知曹州。有書生投書於碩,辭涉謗訕,僚屬皆不能堪。碩延之上座,謝曰:「使碩常聞斯言,庶乎寡過。」士論以故嘉之。遷鄭州防禦使,尋改通州。天德二年,充陜西路轉運使。碩以陜右邊荒,種藝不過麻、粟、蕎麥,賦入甚薄,市井交易惟川絹、乾姜,商賈不通,酒稅之入耗減,請視汴京、燕京例給交鈔通行。而鞏、會、德順道路多險,鹽引斤數太
 重,請一引分作三四,以從輕便。朝廷皆從之。秦州倉粟陳積,而百姓有支移者,止就本州折納其直,公私便之。改河東南路轉運使。上言:「頃者,定立商酒課,不量土產厚薄、戶口多寡及今昔物價之增耗,一概理責之,故監官被系,失身破家,折傭逃竄。或為姦吏盜有實錢,而以賒券輸官,故河東有積負至四百餘萬貫,公私苦之。請自今禁約酒官,不得折準賒貸,惟許收用實錢,則官民俱便。」至今行之。秩滿,除南京路都轉運使。大定六年致仕,卒于家。碩文雅好事,性謹飭,每見古人行事有益於時者,常書置座右,以為蒞官之戒云。



 李上達,字達道,曹州濟陰人。在宋時以蔭補官,累東平府司戶參軍。撻懶取東平,上達給軍須,號辦治。齊國建,為吏部員外郎,攝戶部事。劉豫行什一之法,樂歲輸多,歉歲寡取之,蓋古人助法也。收斂之時,蓄積蓋藏,民或不以實輸官,官亦不肯盡信,於是告訐起而獄訟繁,公私苦之。上達論其弊,豫改定為五等之制。齊國廢,以河南與宋人,上達隨地入宋。宗弼復取河南,上達為同知大名尹,按察陜西、河南。是時,關、陜、蒲、解、汝、蔡民飢,上達輒以便宜發倉粟賑百姓。累遷知山東西路轉運使。上達到官再期,比舊增三十餘萬貫。戶部以其法頒之鄰
 路。上達長於吏事,能治繁劇,猾吏不能欺,所至稱之。卒官,年六十一。



 曹望之,字景蕭,其先臨潢人,遼季移家宣德。天會間,以秀民子選充女直字學生。年十四,業成,除西京教授。為元帥府書令史,補正令史,轉行臺省令史。錄教授資,補修武校尉,除右司都事。吏部侍郎田玨素薄望之,望之願交不肯納,遂與蔡松年、許霖構致黨獄。改行臺吏部員外郎。



 海陵為相,嘗以書致其私,望之不從。天德元年,調同知石州軍州事,坐事免。丁母憂,久之,除絳陽軍節度副使,入為戶部員外郎。詔買牛萬頭給按出虎八猛
 安徙居南京者,望之主給之。撒八反,轉致甲仗八萬自洺州輸燕子城。運米八十萬斛由蔡水入淮,饋伐宋諸軍,期以一日。望之如期集事。進本部郎中,特賜進士及第。



 大定初,討窩斡,望之主軍食,給與有節,凡省糧三十萬石,省剉草五十萬石。帥府以捷入告,議者欲遂罷轉輸,望之以為元惡未誅,不可弛備。既而大軍追討,果賴以濟。以勞進一階,兼同修國史。請於大鹽濼設官榷鹽,聽民以米貿易,民成聚落,可以固邊圉,其利無窮。從之。其後凡貯米二十餘萬石。及東北路歲饑,賴以濟者不可勝數。



 三年,上曰:「自正隆兵興,農桑失業,猛安謀克屯
 田多不如法。」詔遣戶部侍郎魏子平、大興少尹同知中都轉運事李滌、禮部侍郎李願、禮部郎中移剌道、戶部員外郎完顏兀古出、監察御史夾谷阿里補及望之分道勸農,廉問職官臧否。望之還言,乞汰諸路胥吏,可減其半。詔胥吏如故。於是始禁用貼書云。遷本部侍郎,領復實繕修大內財用,費用大省。復以勞進階,上召見諭勉之。



 望之家奴袁一言涉妖妄,大興府鞫治。望之恐,使戶部令史劉公輔問其事于大興少尹王全,全具其事語公輔,公輔以語望之。御史臺劾奏劉公輔言泄獄情。上曰:「妖妄之言,交相傳說何也?」於是,望之決杖一百,王
 全杖八十,劉公輔杖一百五十,除名。



 頃之,運河堙塞,世宗出郊見之,問其故。主者奏曰:「戶部不肯經畫,歲久以致如此。」上責望之曰:「有水運不濬治,乃用陸運,煩費民力,罪在汝等,其往治之。」尚書省奏當用夫役數萬人。上曰:「方春耕作,不可勞民。以宮籍監戶及摘東宮、諸王人從充役,若不足即以五百里內軍夫補之。」



 《太宗實錄》成,監修國史紇石烈良弼賜金帶一、重彩二十端。同修國史張景仁、劉仲淵、望之皆賜銀幣有差。望之嘆賞薄,謂人曰:「栽花接本乃加爵命,勤勞者不遷官。」無何,張景仁遷翰林學士,望之又曰:「止與他人便遣,獨不及我哉。」世
 宗聞之,出望之德州防禦使,謂之曰:「汝為人能幹而心不忠實。朕前往安州春水,人言汝無事君之義。朕敕臣下,有過即當諫爭。汝但面從,退則謗議,此不忠不孝也。汝自五品起遷四品,《太宗皇帝實錄》成,優賜銀幣,不思盡心竭力,惟官賞是覬。今出汝於外,宜改心滌慮。不然,則身亦莫保。」望之到德州,有惠政,百姓為立生祠。改同知西京留守事。



 上書論便宜事:其一,論山東、河北猛安謀克與百姓雜處,民多失業。陳、蔡、汝、潁之間土廣人稀,宜徙百姓以實其處,復數年之賦以安輯之。百姓亡命及避役軍中者,閱實其人,使還本貫。或編近縣以為客
 戶,或留為佃戶者,亦籍其姓名。州縣與猛安事干涉者無相黨匿,庶幾軍民協和,盜賊弭息。其二,論薦舉之法虛文無實。宰相拔擢及其所識,不及其所不識。內外官所舉亦輒不用,或指以為朋黨,遂不敢復舉。宜令宰執歲舉三品二人,御史大夫以下內外官終秩舉二人,自此以下以品殺為差等。終秩不舉者遇轉官勒不遷,三品者削後任俸三月。其舉者已改除,吏部以類品第,季而上之。三品闕則於類第四品中補授,四品五品以下視此為差。其待以不次者,宰執具才行功實以聞。舉當否罪當如律。廉介之士老於令幕無舉主者、七考無贓
 私罪者,準朝官三考勞敘。吏部每季圖上外路職官姓名,路為一圖,大書贓污者於其名下,使知畏慎。外任五品以上官改除,令代之者具功過以聞。年六十以上者,終更赴調,有司察其視聽精力,老疾不堪釐務,給以半祿罷遣。其三,論守邊將帥及沿邊州縣官漁剝軍民,擅興力役,宜歲遣監察御史周行察之。邊部有訟,招討司無得輒遣白身人征斷,宜於省部有出身女直、契丹人及縣令丞簿中擇廉能者,因其風俗,略定科條,務為簡易。征斷羊馬入官籍數,如邊部遇饑饉,即以此賑給之。招討及都監視事,宜限邊部饋送駝馬。招討司女直人
 戶,或擷野菜以濟艱食,而軍中舊籍馬死,則一村均錢補買,往往鬻妻子、賣耕牛以備之。臣恐數年之後,邊防困弊,臨時賑濟,費財十倍而無益,早為之所,則財用省而邊備實矣。官給軍箭用盡,則市以補之,皆朽鈍不堪用,可每歲給官箭一分,以補其闕。邊民闕食給米,地遠負重,往往就倉賤賣而去,可計口支錢,則公私兩便。陜西正副,宜如猛安謀克用土人一員,隊將亦宜參用土人,久居其任。增弓箭田,復其賦役。以廉吏為提舉,舉察總管府以下官。農隙校閱,以嚴武備。則太平之時有經略之制矣。



 又論六鹽場用人,宜令戶部公議辟舉。論漕
 運,先計河倉見在幾何,通州容受幾何,京師歲費幾何。今近河州縣歲稅或六七萬石,小民有入資之費,富室收轉輸之利,宜計實數以科稅入。論民間私錢苦惡,宜以官錢五百易私錢千,期以一月易之,過期以銷錢法坐之。論州府力役錢物,戶部頒印署白簿,使盡書之,以俟審閱,有畏避不書者坐之。論工部營造調發,妨民生業。諸路射糧軍約量人數,習武藝,期以三年成,以息調民。



 書奏,多見采納。以本官行六部事於北邊,召拜戶部尚書。上數之曰:「汝前為侍郎,以不忠外補,頗能練習錢穀,故任以尚書之重,宜改前非,以圖新效也。」



 是時,戶部
 尚書高德基坐高估俸粟責降,世宗念望之吝出納或懲德基也,既出,使人諭之曰:「勿以高德基下粟直,要在平估而已。」十五年新宮成,世宗幸新宮,敕望之曰:「新宮中所須,毋取于民間也。」有良民夫婦質身於東京留守完顏彀英家,期終而不遣,尚書省下東京鞫治。望之言彀英為留守,其同官必且阿徇,不肯窮竟,當移他州。



 望之久習事,有治錢穀名,性剛愎,頗沾沾自露,希覬執政。而刑部尚書梁肅自詳問宋國使還,世宗嘗欲以為執政,久而未用,亦頗炫耀求進。世宗謂左丞相紇石烈良弼日:「曹望之、梁肅急於見知,涉於躁進。」遂出梁肅為濟
 南尹。數年,乃召拜參知政事。而望之終於戶部尚書,年五十六。世宗惜其未及用,賜錢三千貫,敕使致祭,賻銀五百兩、重彩二十端、絹二百匹,以其子淵為奉御,澤為筆硯承奉。



 其後,尚輦局舉出身人年六十餘可以臨事,世宗曰:「豈為此輩惜官邪,但此輩專以盜取官錢為謀生計,不可用也。」由是欲更改監臨格式,以問戶部尚書劉瑋。瑋恐監官謗己,不肯實對。世宗因思望之,嘆曰:「不如望之之敢行也。」



 望之初不學,及貴,稍知讀書,遂刻苦自致,有詩集三十卷。



 大懷貞,字子正,遼陽人。皇統五年,除閤門祗候,三遷東
 上閣門使。丁母憂,起復符寶郎,累官右宣徽使。正隆伐宋,為武勝軍都總管。大定二年,除洺州防御使兼押軍萬戶,改沂州,再遷彰國、安武軍節度使。縣尉獲盜,得一旗,上圖亢宿。詰之,有謀叛狀,株連幾萬人。懷貞當以亂民之刑,請誅其首亂者十八人,餘皆釋之。嘗以私忌飯僧數人,就中一僧異常,懷貞問曰:「汝何許人也?」對曰:「山西人。」復問:「曾為盜殺人否?」對曰:「無之。」後三日詰盜,果引此僧,皆服其明察。改興中尹。錦州富民蕭鶴壽途中殺人,匿府少尹家,有司捕不得,懷貞以計取之,置於法。改彰德軍節度使,卒。



 盧孝儉,宣德州人。登天眷二年第,調憲州軍事判官,補尚書省令史,累官太原少尹。大定二年,陜西用兵,尚書省發本路稅粟赴平涼充軍實,期甚嚴迫。孝儉輒易以金帛,馳至平涼,用省而不失期,并人稱之。用廉,進官二階,遷同知廣寧尹。廣寧大饑,民多流亡失業,乃借僧粟,留其一歲之用,使平其價市與貧民,既以救民,僧亦獲利。累遷山東東路轉運使。孝儉素褊躁,與同僚王公謹失歡。其子嘗私用官帑,孝儉不知也。既而改河北西路轉運使,公謹乃發其事。孝儉聞被逮,莫測所以,行至章丘,自縊死。



 盧庸,字子憲,薊州豐潤人。大定二十八年進士,調唐州軍事判官,再調定平縣令。庸治舊堰,引涇水溉田,民賴其利。補尚書省令史,除南京轉運副使,改中都戶籍判官。察廉,遷禮部主事,累官鳳翔治中。大安三年,徵陜西屯田軍衛中都,以庸簽三司事,主兵食。至潞州,放還屯田軍,庸改乾州刺史,入為吏部郎中。至寧元年,改陜西按察副使。夏人犯邊,庸繕治平涼城池,積芻粟,團結土兵為備。



 十一月,夏人掠鎮戎,陷涇、邠,遂圍平涼。庸矢盡,募人取夏兵射城上箭以濟急用,出府庫賞有功者,人樂為死,平涼賴以完。貞祐二年,庸移書陜西行省僕散
 端,大概謂慶陽、平涼、德順陜西重地,長安以西邠為阨塞,當重兵屯守。詔賞平涼功,庸進官四階,遷按察轉運使。三年,詔諸道按察司講究防秋,庸陳便宜曰:「自延至積石,雖多溝阪,無長河大山為之屏蔽,恃弓箭手以禦侮,其人皆剛猛善鬥,熟於地利,夏人畏之。向者徙屯他所,夏人即時犯邊,此近年深患也。人情樂土,且耕且戰,緩急將自奮。」又曰:「防秋之際,宜先清野。」又曰:「掌軍之官不宜臨時易代,兵家所忌,將非其人,屢代何益?」無何,有言庸老不勝任者,即罷之。未幾,改定海軍節度使,山東亂,不能赴,按察司劾之,當奪兩官,審理官直之。庸以
 病請求醫藥,遂致仕。興定三年,卒。



 李偲,字子友,定州安喜人。中天眷二年進士,調遼山簿,累官戶部主事。丁母憂,起復舊職,除同知河東南路轉運使事。大定初,改同知中都路都轉運使事。僕散忠義行省事於汴京,奏偲幕府,世宗曰:「李偲方治京畿漕事,行省可他選也。」三年,權知登聞檢院,再遷戶部侍郎。上曰:「戶部財用出入,朕難其人。卿非舊勞,資敘尚淺,勿以秩滿例升三品,因循歲月,若不自勉,必不汝貸。」偲每朝會與高德基屏人私語。上聞而怪之,問右丞石琚曰:「李偲果何如人?」琚曰:「亦幹事吏耳。」改同知北京留守、沂州
 防御使。沂南邊郡,戶部符借民閑田,種禾取槁秸,備警急用度。偲曰:「如此則農民失業。」具奏止之。轉運司牒郡輸粟朐山,調急夫數萬人,是時久雨泥濘,挽運不能前進。偲遣吏往朐山刺取其官廩,見儲糧數可支半歲,即具其事牒運司,請緩期,毋自困百姓。先是,郡縣街陌間聽民作廛舍,取其僦直。至是,罷收僦直,廛舍一切撤毀。他郡奉承號令,督百姓必盡撤去,使街陌繩齊矢棘如初時然後止。偲獨教民撤治前卻不齊一者三五所,使巷道端正即已,民便之。改陜西西路轉運使,卒。



 贊曰:毛碩、李上達、曹望之、李偲之流,皆金之能吏也。望
 之悻悻然以求大用,君子無取焉。



 徒單克寧,本名習顯,其先金源縣人,徙居比古土之地,後徙置猛安于山東,遂占籍萊州。父況者,官至汾陽軍節度使。克寧資質渾厚,寡言笑,善騎射,有勇略,通女直、契丹字。左丞相希尹,克寧母舅。熙宗問希尹表戚中誰可侍衛者,希尹奏曰:「習顯可用。」以為符寶祗候。是時,悼后干政,后弟裴滿忽土侮克寧,克寧毆之。明日,忽土以告悼后,后曰:「習顯剛直,必汝之過也。」已而充護衛,轉符寶郎,遷侍衛親軍馬步軍都指揮使,改忠順軍節度使。


克寧娶宗乾女嘉祥縣主,同母兄蒲甲判大宗正事,海
 陵心忌之,出為西京留守,構致其罪誅之,因降克寧知滕陽軍。歷宿州防禦使、胡里改路節度使、曷懶路兵馬都總管。大定初,詔克寧以本路兵會東京。遷左翼都統。詔與廣寧尹僕散渾坦、同知廣寧尹完顏巖雅、肇州防禦使唐括烏也,從右副元帥完顏謀衍討契丹窩斡。趨濟州。謀衍用契丹降吏颭者計策襲賊輜重,克寧與紇石烈志寧為殿,與賊遇于長濼。謀衍使伏兵于左翼之側。賊二萬餘躡吾後,又以騎四百餘突出左翼伏兵之間,欲繞出陣後攻我。克寧與善射二十餘人拒之。眾曰:「賊眾我寡,不若與伏兵合擊,或與大軍相依,可以萬全。」
 克寧曰:「不可。若賊出陣後,則前後夾擊,我敗矣,大軍不可俟也。」於是奮擊,賊乃卻。左翼萬戶襄與大軍合擊之,賊遂敗,追奔十餘里,二年四月一日也。越九日,復追及賊於霿
 \gezhu{
  松}
 河。左翼軍先與賊戰,克寧以騎二千追掩十五里,賊迫澗不得亟渡,殺傷甚眾。賊收軍返旆,大軍尚未至,克寧令軍士下馬射賊,賊遂引而南。



 是時,窩斡已再北,元帥謀衍利鹵掠,駐師白濼。世宗訝其持久,遣問之。謀衍曰:「賊騎壯,我騎弱,此少駐所以完養馬力也。不然,非益萬騎不可勝。」克寧奮然而言曰:「吾馬固不少,但帥不得人耳。其意常利虜掠,賊至則引避,賊去則緩隨
 之,故賊常得善牧,而我常拾其蹂踐之餘,此吾馬所以弱也。今誠能更置良帥,雖不益兵,可以有功。不然,騎雖十倍,未見其利也。」朝廷知其議,召還謀衍,以平章政事僕散忠義兼右副元帥。師將發,賊聲言乞降。克寧曰:「賊初困蹙,且無降意,所以揚言者,是欲緩吾師期也。不若攻其未備,賊若挫衄,則其降必速。如其不降,乘其怠而急擊之,可一戰而定也。」忠義以為然,乃與克寧出中路,遂敗賊兵於羅不魯之地。賊奔七渡河,負險為柵,克寧覘知賊柵之背其勢可上,乃潛師夜登,俯射之,大軍自下攻,賊潰,皆遁去。



 契丹平,克寧除太原尹。未閱月,宋吳璘
 侵陜右,元帥左都監徒單合喜乞益兵,遣克寧佩金牌,駐軍平涼。詔合喜曰:「朕遣克寧參議軍事,此其智勇足敵萬人,不必益軍也。」克寧至,下令安輯,未幾,民皆完聚。治兵伐宋,右丞相僕散忠義駐南京節制諸軍,左副元帥紇石烈志寧經略邊事,克寧改益都尹,兼山東路兵馬都總管、行軍都統。四年,元帥府欲遣左都監璋以兵四千由水路進,詔曰:「可付都統徒單習顯,仍益兵二千,擇良將副之。璋可經略山東。」於是,克寧出軍楚、泗之間,與宋將魏勝相拒于楚州之十八里口。魏勝取弊舟鑿其底,貫以大木,列植水中,別以船載巨石貫以鐵鎖,沉
 之水底,以塞十八里口及淮渡舟路。以步兵四萬人屯於淮渡南岸、運河之間。克寧使斜卯和尚選善游者沒水,繫大繩植木上,數百人於岸上引繩曳一植木,皆拔出之,徹去沉船。進至淮口,宋兵來拒,隔水矢石俱發。斜卯和尚以竹編籬捍矢石,復拔去植木沉船,師遂入淮。與宋兵奪渡口,合戰數四,猛安長壽先行薄岸,水淺,先率勁卒數人涉水登岸,敗其津口兵五百人,餘眾皆濟。宋兵四百餘自清河口來,鎮國上將軍蒲察阿離合懣以步兵百人禦之。克寧自與扎也銀術可五騎先行六七里與戰,銀術可先登,奮擊敗之。宋大兵整陣來拒,克
 寧麾兵前戰,自旦至午,宋兵敗,踰運河為陣,餘眾數千皆走入營中。克寧使以火箭射其營舍,盡焚,踰河撤橋,與其大軍相會。隔水射之,宋兵不能為陣。猛安鈔兀以六十騎擊宋騎兵千餘,不利,少卻。克寧以猛安賽剌九十騎橫擊之,宋兵大敗。追至楚州,射殺魏勝,遂取楚州及淮陰縣。是役也,賽剌功居多。是時,宋屢遣使請和,僕散忠義、紇石烈志寧約以世為叔姪國,割還海、泗、唐、鄧四州。宋人尚遷延有請,及克寧取楚州,宋人乃大懼,一一如約。



 兵罷,改大名尹,歷河間、東平尹,召為都點檢。十一年,從丞相志寧北伐,還師。十一月皇太子生日,世宗
 置酒東宮,賜克寧金帶。明年,遷樞密副使,兼知大興府事,改太子太保,樞密副使如故。拜平章政事,封密國公。



 克寧女嫁為沈王永成妃,得罪,克寧不悅,求致仕,不許,罷為東京留守。明年,上將復相克寧,改南京留守,兼河南統軍使。遣使者諭之曰:「統軍使未嘗以留守兼之,此朕意也。可過京師入見。」克寧至京師,復拜平章政事,授世襲不扎土河猛安兼親管謀克。



 世宗欲以制書親授克寧,主者不知上意,及克寧已受制,上謂克寧曰:「此制朕欲親授與卿,誤授之於外也。」又曰:「朕欲盡徙卿宗族在山東者居之近地,卿族多,官田少,無以盡給之。」乃選
 其最親者徙之。十九年,拜右丞相,徙封譚國公。克寧辭曰:「臣無功,不明國家大事,更內外重任,當自愧。乞歸田里,以盡餘年。」上曰:「朕念眾人之功無出卿右者,卿慎重得大臣體,毋復多讓。」克寧出朝,上使徒單懷忠諭之曰:「凡人醉時醒時處事不同,卿今日親賓慶會,可一飲,過今日可勿飲也。」克寧頓首謝曰:「陛下念臣及此,臣之福也。」



 克寧為相,持正守大體,至於簿書期會,不屑屑然也。世宗嘗曰:「習顯在樞密,未嘗有過舉。」謂克寧曰:「宰相之職,進賢為上。」克寧謝曰:「臣愚幸得備位宰輔,但不能明於知人,以此為恨耳。」二十一年,左丞相守道為尚書令,
 克寧為左丞相,徙封定國公,懇求致仕。上曰:「汝立功立事,迺登相位,朝廷是賴,年雖及,未可去也。」後三日,與守道奏事,俱跪而請曰:「臣等齒髮皆衰,幸陛下賜以餘年。」上曰:「上相坐而論道,不惟其官惟其人,豈可屢改易之邪?」頃之,克寧改樞密使,而難其代。復以守道為左丞相,虛尚書令位者數年,其重如此。未幾,以司徒兼樞密使。二十二年,詔賜今名。二十三年,克寧復以年老為請。上曰:「卿昔在政府,勤勞夙夜,除卿樞密使亦可以優逸矣。朕念舊臣無幾人,萬一邊隅有警,選將帥,授方略,山川險要,兵道軍謀,舍卿誰可與共者?勉為朕留!」克寧乃不
 敢復言。



 二十四年,世宗幸上京,皇太子守國,詔左丞相守道與克寧俱留中都輔太子。上謂克寧曰:「朕巡省之後,萬一有事,卿必躬親之,毋忽細微,圖難於其易可也。」二十五年,左丞相守道賜宴北部,詔克寧行左丞相事。



 是時,世宗自上京還,次天平山清暑,皇太子薨於京師,諸王妃主入宮弔哭,奴婢從入者多,頗喧雜不嚴。克寧遣出之,身護宮門,嚴飭殿廷宮門禁衛如法,然後聽宗室外戚入臨,從者有數。謂東宮官屬曰:「主上巡幸,未還宮闕,太子不幸至于大故,汝等此時能以死報國乎?吾亦不敢愛吾生也。」辭色俱厲,聞者肅然敬憚。章宗時為
 金源郡王,哀毀過甚,克寧諫曰:「哭泣,常禮也。郡王身居冢嗣,豈以常禮而忘宗社之重乎?」召太子侍讀完顏匡曰:「爾侍太子日久,親臣也。郡王哀毀過甚,爾當固諫。謹視郡王,勿去左右。」世宗在天平山,皇太子訃至,哀慟者屢矣。聞克寧嚴飭宮衛,謹護皇孫,嘉其忠誠而愈重之。



 九月,世宗還京師。十一月,克寧表請立金源郡王為皇太孫,以係天下之望。其略曰:「今宣孝皇太子陵寢已畢,東宮虛位,此社稷安危之事,陛下明聖超越前古,寧不察此。事貴果斷,不可緩也。緩之則起覬覦之心,來讒佞之言。讒佞之言起,雖欲無疑得乎?茲事深可畏、大可慎,
 而不畏不慎,豈惟儲位久虛,而骨肉之禍,自此始矣。臣愚不避危身之罪,伏願亟立嫡孫金源郡王為皇太孫,以釋天下之惑,塞覬覦之端,絕構禍之萌,則宗廟獲安,臣民蒙福。臣備位宰相,不敢不盡言,惟陛下裁察。」踰月,有詔起復皇孫金源郡王判大興尹,封原王。世宗諸子中趙王永中最長,其母張玄征女,玄徵子汝弼為尚書左丞。二十六年,世宗出汝弼為廣寧尹。於是,左丞相守道致仕,遂以克寧為太尉,兼左丞相,原王為右丞相,因使克寧輔導之。原王為丞相方四日,世宗問之曰:「汝治事幾日矣?」對曰:「四日。」「京尹與省事同乎?」對曰:「不同。」上笑
 曰:「京尹浩穰,尚書省總大體,所以不同也。」數日,復謂原王曰:「宮中有四方地圖,汝可觀之,知遠近阨塞也。」世宗與宰相論錢幣,上曰:「中外皆患錢少,今京師積錢止五百萬貫,除屯兵路分其他郡縣錢可運至京師。」克寧曰:「郡縣錢盡入京師,民間錢益少矣。若起運其半,其半變折輕齎,庶幾錢貨流布也。」上嘉納之。章宗雖封原王,為丞相,克寧猶以未正太孫之位,屢請於世宗,世宗嘆曰:「克寧,社稷之臣也。」十一月戊午,宰相入見於香閣,既退,原王已出,克寧率宰臣屏左右奏立太孫,世宗許之。庚申,詔立原王右丞相為皇太孫。



 明日,徒單公弼尚息國
 公主納幣,賜六品以上宴于慶和殿。上謂諸王大臣曰:「太尉忠實明達,漢之周勃也。」稱嘆再三。克寧進酒,上舉觴為之酹。有詔給太尉假三日。明年正月,復求解機務。上曰:「卿遽求去邪?豈朕用卿有未盡乎?或因喜怒用刑賞乎?其他宰相未有能如卿者,宜勉留以輔朕。卿若思念鄉土,可以一往,不必謝政事。三月一日朕之生辰,卿不必到,從容至暑月還京師相見。」四月,克寧還朝,入見上。上問曰:「卿往鄉中,百姓皆安業否?」克寧曰:「生業頗安,然初起移至彼,未能滋殖耳。」未幾,以丞相監修國史。上問史事,奏曰:「臣聞古者人君不觀史,願陛下勿觀。」上曰:「
 朕豈欲觀此?深知史事不詳,故問之耳。」初,瀘溝河決久不能塞,加封安平侯,久之,水復故道。上曰:「鬼神雖不可窺測,即獲感應如此。」克寧奏曰:「神之所佑者正也,人事乖,則弗享矣。報應之來皆由人事。」上曰:「卿言是也。」世宗頗信神仙浮圖之事,故克寧及之。宋前主殂,宋主遣使進遺留物,上怪其禮物薄。克寧曰:「此非常貢,責之近於好利。」上曰:「卿言是也。」乃以其玉器五事、玻璃器大小二十事及茶器刀劍等還之。



 二十八年十一月癸丑,上幸克寧第。初,上欲以甲第賜克寧,克寧固辭,乃賜錢,因其舊居宏大之。畢工,上臨幸,賜金器錦繡重彩,克寧亦有
 獻。上飲懽甚,解御衣以衣之。詔畫克寧像藏內府。



 十二月乙亥,世宗不豫。甲申,克寧率宰執入問起居。上曰:「朕疾殆矣。」謂克寧曰:「皇太孫年雖弱冠,生而明達,卿等竭力輔之。」又曰:「尚書省政務權聽於皇太孫。」克寧奏曰:「陛下幸上京時,宣孝太子守國,許除六品以下官,今可權行也。」上曰:「五品以下亦何不可。」乙酉,詔皇太孫攝行政事,注授五品以下官。詔太孫與諸王大臣俱宿禁中。克寧奏曰:「皇太孫與諸王宜別嫌疑,正名分,宿止同處,禮有未安。」詔太孫居慶和殿東廡。丙戌,詔克寧以太尉兼尚書令,封延安郡王。平章政事襄為右丞相,右丞張汝
 霖為平章政事。戊子,詔克寧、襄、汝霖宿於內殿。



 二十九年正月癸巳,世宗崩于福安殿。是日,克寧等宣遺詔,立皇太孫為皇帝,是為章宗。徙封為東平郡王。詔克寧朝朔望,朝日設坐殿上。克寧固辭,詔近臣勉諭。克寧涕泣謝曰:「憐憫老臣,幸免常朝,豈敢當坐禮。」其後,每朝必為克寧設坐,克寧侍立益敬。即位詔文「凡除名開落官吏並量材錄用」,張汝霖奏真盜枉法不可恕,克寧曰:「陛下初即位,行非常之典,贓吏誤沾恩宥其害小,國之大信不可失也。」章宗深然之。無何,進拜太傅,兼尚書令,賜尚衣玉帶。乞致仕,不許。詔譯《諸葛孔明傳》賜之。詔尚書省
 曰:「太傅年高,旬休外四日一居休,大事錄之,細事不須親也。」賜金五百兩、銀五千兩、錢千萬、重彩二百端、絹二千匹。



 尚書省奏猛安謀克願試進士者聽之,上曰:「其應襲猛安謀克者學於太學可乎?」克寧曰:「承平日久,今之猛安謀克其材武已不及前輩,萬一有警,使誰禦之?習辭藝,忘武備,於國弗便。」上曰:「太傅言是也。」章宗初即位,頗好辭章,而疆埸方有事,故克寧言及之。



 明昌二年,克寧屬疾,章宗往視之。克寧頓首謝曰:「臣無似,嘗蒙先帝任使,陛下即位,屬以上相,今臣老病,將先犬馬填溝壑,無以輔明主綏四方。陛下念臣駑怯,親枉車駕臨幸,死
 有餘罪矣。」是日,即榻前拜太師,封淄王,加賜甚厚。是歲二月,薨,遺表,其大概言:「人君往往重君子而反疏之,輕小人而終暱之。願陛下慎終如始,安不忘危,而言不及私。」詔有司護喪事,歸葬于萊州,謚曰忠烈。明昌五年,配享世宗廟廷,圖像衍慶宮。大安元年,改配享章宗廟廷。



 贊曰:徒單克寧可謂大臣矣,功高而身愈下,位盛而心愈勞。《經》曰:「在上不驕,高而不危,制節謹度,滿而不溢」,所以長守富貴。故曰忠信匪懈,不施其功,履盛滿而不忘,德之上也。孜孜勉勉,恪守職業,不居不可成,不事不可行,人主知之,次也。諫期必行,言期必聽,為其事必有其
 功者,又其次也。



\end{pinyinscope}