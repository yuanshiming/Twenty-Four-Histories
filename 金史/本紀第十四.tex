\article{本紀第十四}

\begin{pinyinscope}

 宣宗上



 宣宗繼天興統述道勤仁英武聖孝皇帝諱珣,本名吾睹補,顯宗長子,母曰昭華劉氏。



 大定三年癸未歲生,世宗養于宮中。十八年,封溫國公,加特進。二十六年,賜今名。二十九年,進封豐王,加開府儀同三司,累判兵、吏部,又判永定、彰德等軍。承安元年,進封翼王。泰和五年,改賜名從嘉。八年,進封邢王,又封昇王。所至著祥異。



 至寧
 元年八月,衛紹王被弒,徒單銘等迎于彰德府。既至京,親王、百官上表勸進。



 九月甲辰,既皇帝位於大安殿。以紇石烈胡沙虎為太師、尚書令兼都元帥,封澤王。乙巳,諭尚書省,事有規畫者皆即規畫,悉依世宗所行行之。丙午,以駙馬雄名第賜胡沙虎。丁未,諭宰臣曰:「朕即大位,群臣凡有所見,各直言勿隱。」臨奠於衛紹王第。有司奏,舊禮當設坐哭。上命撤坐,伏哭盡哀。敕有司以禮改葬。戊申,御仁政殿視朝。賜胡沙虎坐,胡沙虎不辭。辛亥,封皇子守禮為遂王,守純為濮王,皇女溫國公主。夔王永升薨,上親臨奠。大元遣乙里只來。壬子,改元貞祐,
 大赦。恩賚中外臣民有差。丙辰,左諫議大夫張行信上章言崇節儉、廣聽納、明賞罰三事。尚書右丞相徒單鎰進左丞相,封廣平郡王。庚申,澤王胡沙虎等議廢故衛王為庶人,上曰:「朕徐思之,以諭卿等。」壬戌,授胡虎中都路和魯忽土世襲猛安。丙寅,詔諭六品以下官,有事可言者言之無隱。



 閏月戊辰朔,拜日于仁政殿,自是每月吉為常。授尚書左丞相徒單鎰中都路迭魯猛安。庚午,上復舊名珣,詔所司,告天地廟社。前所更名二字,自今不須迴避。辛未,詔追尊皇妣為皇太后。是日,皇妃皇子至自彰德府。遣使使宋。己卯,左諫議大夫張行信上
 疏請立皇太子。甲申,立子守忠為皇太子。丙戌,詔降故衛王為東海郡侯。甲午,減定監察御史為十二員。



 冬十月丁酉朔,京師戒嚴。辛丑,大元乙里只來。乙巳,詔應遷加官賞,諸色人與本朝人一體。庚戌,敕有司,皇太子冊禮俟邊事息然後舉行。辛亥,元帥右監軍術虎高琪戰于城北,凡兩敗績而歸,就以兵殺胡沙虎於其第,持其首詣闕待罪。赦之,仍授左副元帥。壬子,殿前都點檢紇石烈特末也等補外。甲寅,張行信上封事,言正刑賞、擇將帥,及鄯陽、石古乃之冤。大元兵下涿州。設京城鎮撫彈壓官。置招賢所。癸亥,放宮女百三十人。



 十一月戊辰,夏人
 攻會州,徒單醜兒兵擊走之。庚午,將乞和于大元,詔百官議於尚書省。以橫海軍節度使承暉為尚書右丞,耿端義為參知政事。癸未,詔贈死事裴滿福興及鄯陽、石古乃官。大元兵徇觀州,刺史高守約死之。又徇河間府、滄州。乙未,定亡失告身文憑格。



 十二月丁酉朔,上御應天門,詔諭軍士,仍出銀以賜之。平章政事徒單公弼進尚書右丞相,尚書右丞承暉進都元帥兼平章政事,左副元帥術虎高琪進平章政事兼前職。



 二年春正月丁卯朔,以邊事未息,詔免朝賀。辛未,大元兵徇彰德府,知府事黃摑九住死之。宋人攻秦州,統軍
 使石抹仲溫擊卻之。癸未,有司奏,請權止今年禘享朝獻原廟及皇太后冊禮,從之。乙酉,徵處士王澮,不至。大元兵徇益都府。命有司復議本朝德運。乙未,大元兵徇懷州,沁南軍節度使宋扆死之。二月丙申朔。壬子,大元乙里只扎八來。丙辰,罷按察司。壬戌,大元乙只復來。



 三月辛未,遣承暉詣大元請和。丁丑,赦國內。癸未,京師大括粟。甲申,大元乙里只扎八來。詔百官議于尚書省。戊子,以濮王守純為殿前都點檢兼侍衛親軍都指揮使,權都元帥府事。庚寅,奉衛紹王公主歸于大元太祖皇帝,是為公主皇后。辛卯,詔許諸人納粟買官。京師戒
 嚴。壬辰,大元兵下嵐州,鎮西軍節使烏古論仲溫死之。夏四月乙未朔,以知大興府事胥鼎為尚書右丞。戊戌,奉遷昭聖皇后柩於新寺。時山東、河北諸郡失守,惟真定、清、沃、大名、東平、徐、邳、海數城僅存而已,河東州縣亦多殘毀。兵退,命僕散安貞等為諸路宣撫使,安集遺黎。至是以大元允和議,大赦國內。癸卯,權厝昭聖皇后於新寺。甲辰,詔有司具陣亡人子孫以備錄用。丁未,以都元帥承輝為右丞相。庚戌,左丞相、監修國史廣平郡主徒單鎰薨。乙卯,尚書省奏巡幸南京,詔從之。己未,葬衛紹王。



 五月癸酉,承暉加金紫光祿大夫,封定國公。
 尚書左丞抹撚盡忠加崇進,封申國公。



 甲戌,霍王從彞薨。乙亥,輟朝。上決意南遷,詔告國內。太學生趙昉等上章極論利害,以大計已定,不能中止,皆慰諭而遣之。詣原廟奉辭。戊寅,將發,雨,不果行。以南京留守僕散端等嘗請臨幸,及行,先詔諭之。辛巳,詔遷衛紹、鎬厲王家屬于鄭州。壬午,車駕發中都。是日雨,至甲申止。丙戌,次定興。禁有司扈從踐蹂民田。丁亥,次安肅州,元帥右監軍完顏弼以兵迎見。癸巳,次中山府,敕扈從軍所踐禾稼,計直酬之。



 六月甲午朔,以按察轉運使高汝礪為參知政事。癸丑,次內丘縣。大元乙里只來。戊午,次彰德府,曲
 赦其境內。庚申,次鉅橋鎮。是日,南京行宮寶鎮閣災。壬戌,次宜村。黃龍見西北。



 秋七月,車駕至南京。詔立元妃溫敦氏為皇后。



 八月甲午,以立后,百官上表稱賀。庚子,皇太子至自中都。丁未,夏人入邊,命移文責之。甲寅,罷經略司。應奉翰林文字完顏素蘭上書言事。



 九月壬戌朔,日有食之。皇孫生。癸亥,山東路報萊州之捷。辛未,立監察御史陞黜格。庚辰,詔訓練軍士。丁亥,諭宣徽院,正旦生辰不須進物。太白晝見于軫。戊子,禁軍官圍獵。



 冬十月甲午,詔遣官市木波、西羌馬。陜西軍戶戰死者給糧贍其家。丁酉,大元兵徇順州,勸農使王晦死之。壬寅,
 左副元帥兼尚書左丞抹撚盡忠進平章政事。以御史中丞孛術魯德裕為參知政事兼簽樞密院事。曲赦中都路。乙卯,遣參知政事孛術魯德裕行尚書省于大名府。丙辰,大元兵收成州。諭大名行省貶損用度。德州防禦使完顏醜奴伏誅。



 十一月丁卯,以御史大夫僕散端為尚書左丞相。曲赦山東路。辛未,詔賜衛紹王家屬既稟。詔有司答夏國牒。丙子,許諸色人試武舉。蘭州譯人程陳僧叛,西結夏人為援。辛巳,熒惑犯房宿鉤鈐星。癸未,曲赦遼東路。敕罷宣撫司輒擬官。



 十二月戊戌,遣真定行元帥府事永錫等援中都。頒勸農詔。丁未,以和議
 既定,聽民南渡。乙卯,登州刺史耿格伏誅,流其妻孥。大元兵徇懿州,節度使高閭山死之。



 三年春正月辛酉朔,宋遣使來賀。壬戌,遣內侍諭永錫防邊,毋以和議為辭。癸亥,曲宴群臣、宋使。定文武五品以上侍坐員,遂為常制。乙丑,詔宣撫阿海、總管合住討賊劉二祖、張汝楫。戊辰,尚書省言:「內外軍人入粟補官者多,行伍浸虛。請俟平定,應監差者與三酬,門戶有職事者升一等,其子弟應廕者罷之。」上可其奏。乙亥,夏人犯環州。北京軍亂,殺宣撫使奧屯襄。丁丑,右副元帥蒲察七斤以其軍降於大元。辛巳,皇太子疾。輟朝。乙酉,皇
 太子薨。



 二月辛卯,環州刺烏古論延壽及斜卯毛良虎等敗夏人於州境,詔進官有差。大元乙里只來。壬辰,上臨奠皇太子殯所。有司奏辰日不哭,上曰:「父子至親,何可拘忌!」命御史中丞李英、元帥左都監烏古論慶壽領兵護餉中都,付以空名宣敕,許視功遷敘,逗撓者治以軍律。乙未,改寧邊州隸嵐州。丁酉,詔諸色人遷官並視女直人,有司妄生分別,以違制論,從戶部郎中奧屯阿虎請也。辛丑,敕宰臣饋乙里只酒饌。壬寅,頒獎諭官吏軍民詔,曲赦,招撫北京作亂者。丙午,尚書省以南遷後,吏部秋冬置選南京,春夏置選中都,赴調者不便,請
 併選於南京。從之。武清縣巡檢梁佐、柳口巡檢李咬住以誅颭賊張暉、劉永昌等功進官有差,皆賜姓完顏。丁未,山東宣撫使僕散安貞遣提控僕散留家等破賊楊安兒步騎三萬,殲其眾,降偽頭目三百餘人、脅從民三萬餘戶。戊申,減沿邊州府官資考有差。壬子,立保城無虞及捕獲姦叛遷賞格。乙卯,敕奏急事不拘假日。丁巳,日初出赤如血,欲沒復然。戊午,隆德殿鴟尾壞。



 三月壬戌,詔河北州縣官,令文武五品以上辟舉,不聽以它事差占,仍勒終任。有勞績者但升遙領之職,應降罰者亦止本處居住。時河北殘毀,吏治多茍且以求代易,
 故著是令。癸亥,詔百官各陳防邊利害,封章以聞。丙寅,敕河東、河北、大名長貳官訓練隨處義兵,鄰境有警,責其救援。降人自拔歸國者遷職,仍列其姓名,以招諭來者。沿河州縣官罷軟不勝職任者汰去,令五品已上官公舉,仍許今季到部人內先擇能者量緩急易之。丁卯,安武軍節度使張行信上書言急務四事。庚午,諭遼東宣撫使蒲鮮萬奴選精銳屯沈州、廣寧,以俟進止。壬申,長春節,宋遣使來賀。戊寅,諭尚書省,歲旱,議弛諸處碾磑,以其水溉民田。己卯,雨。自去冬不雨雪,至是始雨。勸農使李革言:「河北州縣官吏多求河南差占以避難,宜
 發元任領戍兵者。不可離則別注以往。」庚辰,御史臺言:「在京軍官及委差官芻糧券例悉同征行,乞減其給。樞密院委差有俸有吏,非征行不必給。」皆從之。敕尚書省,入粟補官者毋括其戶為軍。有司議賞軍功,毋有所沮格。壬午,山東宣撫司報大沫堌之捷,夾谷石里哥及沒烈擒賊渠劉二祖等斬之,前後殆賊萬計。西京軍民變,遣官撫諭之。己丑,禁州縣置刃於杖以決罪人。前年,京兆治中李友直私逃華州,結同知防禦使馮朝、河州防禦判官郝遵甫、平涼府同知致仕楊庭秀、水洛縣主簿宿徽等團集州民,號「忠義扈駕都統府」,相手延為亂,殺其
 防禦判官完顏八斤及城中女直人,以書約都統楊珪,為府兵所得。珪諱之,請自效,誘友直等執之,麾所招千餘人納仗,坑諸城下。時京師道路隔絕,安撫司以便宜族友直等,至是以狀聞。乃贈八斤及被害官軍十餘人各一官,賻錢三百貫。



 夏四月癸巳,河東宣撫使胥鼎言利害十三事。長勝軍都統楊珪伏誅。丙申,河南路蝗,遣官分捕。上諭宰臣曰:「朕在潛邸,聞捕蝗者止及道傍,使者不見處即不加意,當以此意戒之。」權參知政事德升言:「舊制夏至後免朝,四日一奏事。」上曰:「此在平時可耳。方今多故,勿謂朕勞,遂云當免,但使國事無廢則善矣。」
 己亥,曲赦山東路。癸卯,籍赴選臨當官為軍。乙巳,罷都南行尚書六部。侯摯言九事。曲赦蒲察七斤脅從之黨,募能殺獲七斤者,以其官官之。丙午,以調度不給,凡隨朝六品以下官及承應人,罷其從己人力輸傭錢。經兵州、府其吏減半,司、縣吏減三之一。其餘除開封府、南京轉運司外,例減三之一。有祿官吏被差不出本境並罷給券,出境者以其半給之。脩內司軍夫亦減其半。丁未,故皇太子啟菆,賜謚曰莊獻,戊申,權葬迎朔門外。



 詔自今策論詞賦進士,第一甲第一人特遷奉直大夫,第二人以下,經義第一人並儒林郎,第二甲以下徵事郎,
 同進士從仕郎,經童將仕郎。壬子,芮國公從厚薨。詔遣使同山西宣撫司選其民勇健者為軍。論有司,勿拒河北避兵之民,所至加存恤。用山東西路宣撫副使完顏弼言,招大沫堌渠賊孫邦佐、張汝楫以五品職,下詔湔洗其罪。乙卯,詔檢核朝廷差遣官券歷,無故稽留中道者罪之。丙辰,諭田琢留山西流民少壯者充軍,老幼者令就食於邢、洺等州,欲趣河南者聽。上議遣親軍六千餘及所募二千七百人援中都,宰臣以為行宮單弱,親軍不可遣,遂止。



 五月庚申,招撫山西軍民,仍降詔諭之。是日,中都破,尚書右丞相兼都元帥定國公承暉死之。
 戶部尚書任天寵、知大興府事高霖皆及於難。壬戌,降空名宣敕、紫衣師德號度牒,以補軍儲。辛未,立皇孫鏗為皇太孫。癸酉,劉炳上書言十事。辛巳,上諭宰臣:「多事之秋,陳言者悉送省。恐卿等不暇,朕於宮中置局,命方正官數員擇可取者付出施行,何如?」宰臣請如聖諭。詔削納馬補官恩例。戊子,謀代西夏,遣大臣鎮撫京兆。



 秋七月戊午朔,大元兵收濟源縣。己未,徵弓箭於內外品官,三品以上三副,四品、五品二副,餘以等級征之。庚申,置陳、潁漕運提舉官,以戶部勾當官往來督察。有星如太白,色青白,有尾,出紫微垣北極傍,入貫索中。上聞河
 北譏察官有要求民財始聽民渡河者,避兵民至或餓死、自溺,特命御史臺體訪之。又禁隨朝職官奪民碾磑以自營利。詔河間有孤城,移其軍民就粟清州,括民間騾付諸軍,與馬參用。辛酉,議括官田及牧馬地以贍河北軍戶之徙河南者,已為民佃者俟獲畢日付之。群臣迭言其不便,遂寢。癸亥,詔河北郡縣軍須並減河南之半。定尚書所造諸符:樞密院鹿,宣撫司魚,統軍司虎。丙寅,遣參知政事高汝礪往河南,便宜措置糧儲。制品官納弓箭之令,丁憂致仕者免。甲戌,借平陽民租一年。詔職官更兵亡失告身,見任者保識即重給之,妄冒者從詐
 偽法。丙子,尚書省奏給皇太孫歲賜錢,上不從,曰:「襁褓兒安所用之。」詔致仕官俸給比南征時減其半。丁丑,肅宗神主至自中都,奉安於明俊殿。戊寅,月入畢宿中,戊夜犯畢大星。己卯,明德皇后神主至中都。裁損宮中歲給有差。甲申,詔尚書省,行六部太多,其令各路運司兼之。改交鈔名「貞祐寶券」。



 八月戊子朔,以陜西統軍使完顏合打簽樞密院事。己丑,制軍府庶事樞密院官須與經歷官裁決,經歷議是而院官不從,許直以聞。癸巳,詔遣官體究京西路親遷軍戶。丙申,諭尚書省,職官犯罪。大者即施行之,小者籍之,事定始論其罪,諭樞密院,
 撒合輦所簽軍有具戒僧人,可罷遣之。己亥,詔武舉官非見任及已從軍者,隨處調赴京師,別為一軍,以備用。被薦未授官者,量才任之。庚子,上慮平陽城大,兵食不足,議棄之,宰臣持不可。賞前冀州教授粘割忒鄰,集義兵,出方略,遏土寇,兵後攝州,復立州治,積芻糧,招徠民戶至五萬,特遷三官,升正五品職。置山東西路總管府于歸德府及徐、亳二州。以太常卿侯摯為參知政事,行尚書省于河北東、西兩路。太祖御容至自西京,奉安于啟慶宮。甲辰,置行樞密院於徐州、歸德府。詔諸職官不拘何從出身,其才可大用者尚書省具以聞。丙午,山東
 西路宣撫使完顏弼表:「遙授同知東平府事張汝楫將謀復叛,密遣人招同知益都府事孫邦佐。邦佐斬其人,馳報弼,弼殺汝楫及其黨萬餘。承制升邦佐德州防禦使,餘立功者賞有差。」上嘉弼功,加崇進,封密國公,詔獎諭之。丁未,詔近臣舉良將,加孫邦佐昭毅大將軍、泰定軍節度使,仍官其子。戊申,東平、益都、太原、潞州置元帥府。大赦。己酉,監察御史許古獻恢復中都之策。紅襖賊掠成武,宣撫副使顏盞天澤討走之,斬首數百級。進天澤一官,將校有功者命就遷賞。命侯摯招邢州賊程邦傑以官,不從則誘其黨圖之。減戶部乾辦官四員及
 委差官有差。壬子,置行省於陜西。乙卯,增沿河闌糴之法,十取其八,以抑販粟之幣,仍嚴禁私渡。增步軍萬人戍京以西,四萬人戍京以東。選陜西騎兵二千,增京畿之衛。諭陜西,堅守延安、臨洮、環、慶、蘭、會、保安、綏德、平涼、德順、鎮戎、涇原、酈、坊、邠、寧、乾、耀等處要害。分渭南州郡步兵屯平涼,令宣撫使治邠州,副使治同州之澄城以統之。更以步騎守沿渭諸津。丙辰,元帥左監軍兼知真定府事永錫坐援中都失律,削官爵,杖之八十。



 九月丁巳朔,戶部侍郎奧屯阿虎言:「國家多故,職官往往不仕。乞限以兩季,違者勿復任用。」上嫌其太重,命違限者止奪
 三官,降職三等,仍永不升注。辛酉,除名永錫特遷信武將軍、息州刺史。甲子,諭宰臣,沿淮塘路以南地向授民業,今為豪勢據奪者,其令有司察之。丙寅,樞密院言:「陜西、河東世襲蕃部巡檢,昨與世襲猛安謀克例罷其俸。今邊事方急,宜仍給之,庶獲其用。又西邊弓箭手有才武出眾,獲功未推賞者,令宣撫司核實以聞。」從之。丁卯,以秋稼未獲,禁軍官圍獵。詔授隱士王澮太中大夫、右諫議大夫,充遼東宣撫司參謀官。戊辰,遙授武寧軍節度副使徒單吾典告平章政事抹撚盡忠逆謀,詔有司鞫之。設潼關提控總領軍馬等官。辛未,置河北東路行
 總管府於原武、陽武、封丘、陳留、延津、通許、杞諸縣,以治所徙軍戶。命司屬令和尚等護治鞏國公按辰第。上謂宰臣曰:「按辰所為不慎,或至犯法。舍之則理所不容,治之則失親親之道,但當設官以防之耳。」按辰尋以不法,謫博州防禦使。黜衛紹王母李氏光獻皇后尊謚,神主在太廟,畫像在啟慶宮,並遷出之。陳州鎮防軍段仲連進羊三百,詔遷三官。命右丞汝礪詣陳州規畫糧儲。壬申,以蘇門縣為輝州。癸酉,朝謁世祖、太祖御容于啟慶宮,行獻享禮,始用樂。賜東永昌姓為溫敦氏,包世顯、包疙疸為烏古論氏,睹令孤為和速嘉氏,何定為必蘭氏,
 馬福德、馬柏壽為夾谷氏,各遷一官。甲戌,朝謁太宗、熙宗、睿宗御容,行獻享禮。詔開、滑、浚、濟、曹、滕諸州置連珠寨,如衛州。乙亥,詔河北、山東等路及平涼、慶陽、臨洮府,涇、邠、秦、鞏、德順諸州經兵,四品以下職事官並以二十月為滿。募隨處主帥及官軍、義軍將校,有能率眾復取中都者封王,遷一品階,授二品職。能戰卻敵、善誘降人、取附都州縣者,予本處長官、散官,隨職遷授,餘州縣遞減二等。



 紅襖賊周元兒陷深、祁州,束鹿、安平、無極等縣,真定帥府以計破之,斬元兒及殺其黨五百餘人。丁丑,詔司、縣官能募民進糧五千石以上,減一資考,萬石以
 上,遷一官,減二資考,二萬石以上遷一官,升一等,注見闕。諸色人以功賜國姓者,能以千人敗敵三千人,賜及緦麻以上親,二千人以上,賜及大功以上親,千人以上,賜止其家。庚辰,陜西宣撫司來上第五將城萬戶楊再興擊走夏人之捷。壬午,以空名宣敕付陜西宣撫司,凡夏人入寇,有能臨陣立功者,五品以下並聽遷授。乙酉,置大名府行總管府於柘城縣,以治所徙軍戶。



 冬十月丙戌朔,翰林侍讀學士、權參知政事烏古論德升出為集慶軍節度使兼亳州管內觀察使。丁亥,尚書右丞汝礪言:「河北軍戶之徙河南省,宜以系官閒田及牧馬草
 地之可耕者賜之,使自耕以食,而罷其月糧。」上從其請。命右司諫馮開隨處按視,人給三十畝。夏人入保安,都統完顏國家奴破之;攻延安,戍將又敗之。是日,捷至。戊子,以御史中丞徒單思忠為參知政事。己丑,平章抹撚盡忠下獄既久,監察御史許古言:「盡忠逮繫有司,此必重罪,而莫知其由,甚駭眾聽。乞遣公正重臣鞫之。如得其實,明示罪目,以厭中外之心。」書上,不報。庚寅,遂誅盡忠。癸巳,罪狀盡忠告中外。詔樞密副使僕散安貞行樞密院于徐州。戊戌,遼東宣撫司報敗留哥之捷。甲辰,詔求廣平郡王承暉之後,得其猶子歷亭縣丞永懷,以為
 器物直長。丙午,夏人陷臨洮,陜西宣撫副使完顏胡失剌被執。庚戌,詔尚書左丞相僕散端兼都元帥,行尚書省于陜西。辛亥,蒙古綱奏:「昨被旨權山東路宣副使,屯東平。行至徐北岸,北兵已偪徐,不可往。」詔樞密副使僕散安貞權於沿河任使之。壬子,以同、華舊屯陜西兵及河南所移步騎舊隸陜州宣撫司者,改隸陜西行省。召中奉大夫、襲封衍聖公孔元措為太常博士。上初用元措於朝,或言宣聖墳廟在曲阜,宜遣之奉祀。既而上念元措聖人之後,山東寇盜縱橫,恐罹其害,是使之奉祀而反絕之也,故有是命。遼東賊蒲鮮萬奴僭號,改元
 天泰。



 十一月丙辰朔,河北行尚書省侯摯入見。詔河北西路宣撫副使田琢自浚徙其兵屯陜。戊午,樞密院進王世安取盱眙、楚州之策,遂以世安為招撫使,與泗州元帥府所遣人同往淮南計度其事。戊辰,夏人犯綏德之克戎寨,官軍敗之,犯綏平,又敗之。賞有功將士告捷者。參知政事徒單思忠言:「今陳言者多掇拾細故,乞不送省,止令近侍局度其可否發遣。」上曰:「若爾,是塞言路。凡係國家者,豈得不由尚書省乎?」庚午,上與尚書右丞汝礪商略遣官括田賜軍之利害,汝礪言不便者數端。乃詔有司罷其令,仍給軍糧之半,其半給詣實之
 價。壬申,遣參知政事侯摯祭河神於宜村。甲戌,移剌塔不也以軍萬人破夏人數萬於熟羊寨。丙子,詔市民間輓車羸疾牝馬置群牧中,以圖滋息。知臨洮府陀滿胡土門破夏人八萬於城下。丁丑,監察御史陳規劾參知政事侯摯,上不允所言,而慰答之。庚辰,上謂宰臣曰:「朕恐括地擾民,罷其令矣。官荒牧馬地軍戶願耕者聽,已為民承種者勿奪。舊列點檢左右將軍、近侍局官、護衛、承應人秩滿皆賜匹帛,雖所司為之製造,然不免賦取於民,近亦罷之,止給寶券。至於朕所服御,亦以官絲付府監織之,自今勿復及民也。」大元兵徇彰德府,知府
 陀滿斜烈死之。



 十二月乙酉朔,徙朔州民分屯嵐、石、隰、吉、絳、解等州。戊子,以軍事免樞密院官朝拜。己丑,侯摯復行尚書省于河北。庚寅,太白晝見。壬辰,詔免元日朝賀。乙未,敕贈昭聖皇后三代官爵。太康縣人劉全、時溫、東平府民李寧謀反,伏誅。戊戌,陜西行元帥府乞益兵,以田琢之眾隸之,仍獎諭以詔。壬寅,詔林州刺史惟宏與都提控從坦同經理邊事,諸將功賞次第便宜行之。乙巳,大元兵徇大名府。癸丑,皇太孫薨,以殤,無祭享之制,戒勿勞民。諭宣徽院免元日親王、公主進酒。甲寅,禮官奏,正旦宋遣使來賀。不宜輟朝。命舉樂、服色如常儀。
 詔臨洮路兵馬都總管陀滿胡土門進官三階,再任。



 四年春正月癸亥,監察御史田迥秀條陳五事。丙寅,紅襖賊犯泰安、德、博等州,山東西路行元帥府敗之。丁卯,諭御史臺曰:「今旦視朝,百官既拜之後,始聞開封府報衙聲。四方多故之秋,弛慢如此,可乎?中丞福興號素謹于官事者,當一詰之。」己巳,尚書右丞高汝礪進左丞。庚午,大元兵收曹州。辛未,參知政事侯摯進尚書右丞。壬申,太原元帥左監軍烏古論德升招其民降北者,得四千三百餘人。癸酉,詔賜故皇太孫謚曰沖懷。更定捕獲偽造寶券者官賞。乙亥,以殿前都點檢皇子遂王守
 禮為樞密使,樞密使濮王守純為平章政事。己卯,立遂王守禮為皇太子。庚辰,詔免逃戶租。壬午,言者請遣官勸農,至秋成,考其績以甄賞。宰臣言:「民恃農以生,初不待勸,但寬其力,勿奪其時而已。遣官不過督州縣計頃畝、嚴期會而已。吏卒因為姦利,是乃妨農,何名為勸。」上是其言,不遣。



 二月甲申朔,日有食之。上不視朝,詔皇太子控制樞密院事。大元兵圍太原。乙酉,以信武將軍、宣撫副使永錫簽樞密院事,權尚書右丞。皇太子既總樞務,詔有司議典禮,以金鑄「撫軍之寶」授太子,啟稟之際用之。平章政事高琪表乞致仕,不允。如樞密院官問所
 以備禦之策。丁亥,以河東南路宣撫使胥鼎為樞密副使,權尚書左丞,行省于平陽。鼎方抗表求退,詔勉諭就職,因有是命。行省左丞相僕散端先亦告老,遣太醫往鎮護視其疾。戊子,宰臣以皇太子既立,服御儀物悉與已受冊同,今邊事未寧,請少緩冊寶之禮,從之。戊戌,免親王、公主長春節入賀致禮。己亥,大元兵攻下霍山諸隘。甲辰,命參知政事李革為修奉太廟使,禮部尚書張行信提控修奉社稷。權祔肅宗神主于世祖室,奉始祖以下神主于隨室,祭器以瓦代銅,獻官以公服行事,供張等物並從簡約。庚戌,詔凡死節之臣籍其數,立廟致
 祭。壬子,任國公瑋薨,輟朝。是月,同知觀州軍州事張開復河間府滄、獻等州并屬縣十有三,表請赦旁郡脅從之臣。又請以宣撫司空名宣敕二百道付之,從權署補,仍以糧繼其軍食。詔樞密措畫。



 三月乙卯,以將修太廟,遣李革奏告祖宗神主于明俊殿。丁巳,曲赦中都、河北等路。議軍戶給地事。乙丑,延州刺史溫撒可喜上疏言:「皇太子宜選正人為師保。」丙寅,長春節,宋遣使來賀。己巳,以將修社稷,遣太子少保張行信預告。滄州經略副使張文破趙福,復恩州。丙子,曲赦遼東路。己卯,處士王澮以右諫議大夫復遷中奉大夫、翰林學士,仍賜詔褒
 諭。庚辰,復邢州捷至。



 夏四月己丑,陜西行省來報秦州官軍破妖賊趙用、劉高二之捷。遣官鞫單州防禦使僕散倬之罪,罷其城單州之役。癸巳,張開奏復清州等十有一城,詔遷官兩階,賞將士有差。甲午,改賜皇太子名守緒。詔諭陜西路軍民。丙申,河北行省侯摯言:「北商敗粟渡河,官遮糴其什八,商遂不行,民飢益甚。請罷其令。」從之。河南、陜西蝗。丁酉,太白晝見于奎。己亥,夏人葩俄族都管汪三郎率其蕃戶來歸,以千羊進,詔納之,優給其直。辛丑,侯摯言:「紅襖賊掠臨沂、費縣之境,官軍敗之。獲其黨訊之,知其渠賊郝定僭號署官,已陷滕、兗、單諸
 州,萊蕪、新泰等十餘縣。」時道路不通,宰臣請諭摯為備。仍詔樞密招捕。蔡、息行元帥府兵拔木陡關,斬首千級。甲辰,有司言,扶風、郿縣有騑傷麥。



 五月癸丑朔,禮官言:「太廟既成,行都禮雖簡約,惟以親行祔享為敬,請權不用鹵簿儀仗及宮縣樂舞。」從之。山東行省上沂州之捷。甲寅,鳳翔之華、汝等州蝗。辛酉,以尚書右丞侯摯行省事于東平。己巳,來遠鎮獲夏諜者陳絺等,知夏人將圖臨洮、鞏州,窺長安。命陜西行省嚴為之備。丙子,上將以七月行祔享禮,慮時雨有妨,詔改用十月。夏人修來羌城界河橋。元帥右都監完顏賽不遣兵焚之,俘馘甚
 多。戊寅,京兆、同、華、鄧、裕、汝、亳、宿、泗等州蝗。



 六月戊子,詔凡進奏帖及申尚書省、樞密院關應大事,私發視者絞,誤者減二等,制書應密者如之。壬辰,遼西偽瀛王張致遣完顏南合、張頑僧上表來歸。詔授致特進,行北京路元帥府事,兼本路宣撫使,南合同知北京兵馬總管府,頑僧同知廣寧府。丙申,木星晝見于奎,百有一日乃伏。癸卯,詔有司祈雨。丁未,河南大蝗傷稼,遣官分道捕之。罷河北諸路宣撫司,更置經略司。壬子,以旱,詔參知政事李革審決京師冤獄。秋七月癸丑朔,昭義軍節度使必蘭阿魯帶復威州及獲鹿縣。飛蝗過京師。甲寅,山
 東行省檻賊郝定等至京師,伏誅。乙卯,以旱蝗,詔中外。己未,敕減尚食數品及後宮歲給縑帛有差。辛酉,監察御史陳規上章條陳八事。



 閏月壬午朔,日有食之。辛卯,復深州。癸巳,翰林學士完顏孛迭進《中興事跡》。甲午,命掌軍官舉奇才絕力之人,提控、都副統等官互舉其屬。頒舉官賞罰格,許功過相除。品官及草澤人有才武者,舉薦升降亦如之。庚子,詔河南、陜西鎮防軍應蔭及納粟補官者,當役如舊,俟事定乃聽赴銓。



 八月甲寅,太子少保兼禮部尚書張行信定祔享親祀之儀以進。上嘉納之。三原縣僧廣惠進僧道納粟多寡與都副威儀及
 監寺等格,從其言鬻之。夏人入安塞堡,元帥左監軍烏古論慶壽遣軍敗之。壬戌,賜張行信寶券二萬貫、重幣十端,旌其議禮之當。乙亥,詔論中都民,命大名招撫使募人持詔以往。丙子,大元兵攻延安。己卯,夏人入結耶觜川,官軍擊走之。



 九月辛巳朔,大元兵攻坊州。以簽樞密院事永錫為御史大夫,領兵赴陜西,便宜從事。壬辰,大元兵攻代州。經略使奧屯醜和尚戰沒。以中衛尉完顏奴婢等充賀宋生日使。冬十月己未,親王、百官奉迎祖宗神主于太廟。招射生獵戶練習武藝知山徑者分屯陜、虢要地。命元帥左監軍必蘭魯帶守潼關,遙授
 知歸德府事完顏仲元軍盧氏。大元兵攻潼關,西安軍節度使泥旁古蒲魯虎戰沒。辛酉,上親行祔享禮。甲子,祔享禮成。赦。乙丑,詔諭河南官吏軍民,以賞格募立功之士。命參知政事徒單思忠提控鎮撫京師,移剌周剌阿不屯關、陜。丙寅,詔京師具防城器械,多鑿坎熥,築垣墻於隙地。徙衛紹及鎬厲王家屬于京師。丁卯,以奉安社稷,遣官預告。戊辰,命張行信攝太尉,奉安社稷,禮樂咸殺其數。詔吏、禮、兵、工四部尚書董防城之役。大元兵徇汝州。己巳,沿河唯存通報小舟,餘皆焚之。庚午,詔宿糧州縣屯兵,其簽民為兵者就署隊長,以自防遏。河東
 行省胥鼎,遣潞州元帥左監軍必蘭阿魯帶以軍一萬,孟州經略使徒單百家以軍五千,由便道濟河趣關、陜,自將平陽精兵援京師。命樞府督軍應之。辛未,置官領招賢所事。命內外官探訪有才識勇略能區畫防城者具以聞,得實超任,仍賞舉主。內負長才不為人所知者,聽赴招賢所自陳。壬申,以龍虎衛上將軍裴滿羊哥知歸德府事,行樞密院事。癸酉,詔罷遣有司所拘民間輸稅車牛以運軍士衣糧者。甲戌,諭附京民盡徙其芻糧入城,官儲併運之。丙子,行樞密院知河南府事完顏合打以徵兵失應,坐誅。戶部郎中魏琦以沒王事,官其子。
 己卯,議禁京師靡穀,近侍以寶券方行,恐滯其用,不果。吏部令史韓希祖陳言,曾以戰功致身者盡拘京師備用,從之。



 十一月庚辰朔,增定守禦官及軍人遷賞格。辛巳,詔止附京農民自撤其廬舍。壬午,河東行省胥鼎入援京師,用其言以知平陽府王質權元帥左監軍,同知完顏僧家奴權右監軍,代鎮河東。拜鼎為尚書左丞兼樞密副使,知歸德府完顏伯嘉簽樞密院事。以完顏合打伏誅,詔中外。乙酉,元帥右都監完顏賽不來獻其提控石盞合喜、楊斡烈等大敗夏人於定西之捷,命行省視其功賞之。大元兵至澠池,右副元帥蒲察阿里不孫
 軍潰而逃,失其所佩虎符。丙戌,前臨潢府推官權元帥右監軍完顏合達率官軍老幼自北歸國,升鎮南軍節度使,進官三階。詔出公帑綿絹付有司償所括民服以衣軍者。是夕,月暈木星,木在奎,月在壁。己丑,定毀防城器具法。辛卯,詔立功五品以上官賜饌御前,六品以下官賜饌近侍局。癸巳,上諭皇太子:「京城提控官有以文資充者,彼豈知兵?其速易之。」甲午,放免諸職官傔從及諸司局射糧兵卒嘗選充軍者。戊戌,敕諸州縣簽籍軍民,以備土寇。華州元帥府復潼關。庚子,罷在京防城民軍。遣御史陳規等充河南宣差安撫捕盜官。河南路統
 軍使紇石烈掃合以發兵後期,坐誅。甲辰,以尚書工部侍郎和尚等充賀宋正旦使。丙午,河南行樞密院從坦言,其族人道哥願隸行伍以自效。上嘉其忠,許之。內族承立進所獲馬駝。上曰:「此軍士所得,即以予之可也,朕安用哉。」因遍諭諸道將帥,後勿復如是。



 十二月辛亥,平章政事術虎高琪加崇進、尚書右丞相。參知政事李革罷。癸亥,大元兵攻平陽。丙寅,皇太子議伐西夏。大元兵徇大名府。壬申,大元兵進自代州神仙橫城及平定承天鎮諸隘,攻太原府。宣撫使烏古論禮遣人間道齎礬書至京師告急。詔發潞州元帥府,平陽、河中、絳、孟宣撫
 司兵援之。乙亥,高琪請修南京裏城。上曰:「民力已困,此役一興,病滋甚矣。城雖完固,朕亦何能獨安此乎?」



\end{pinyinscope}