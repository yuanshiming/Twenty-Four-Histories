\article{本紀第六}

\begin{pinyinscope}

 世宗上



 世宗光天興運文德武功聖明仁孝皇帝,諱雍,本諱烏祿,太祖孫,睿宗子也。母曰貞懿皇后李氏。天輔七年癸卯歲,生于上京。體貌奇偉,美鬚髯,長過其腹,胸間有七子如北斗形。性仁孝,沉靜明達。善騎射,國人推為第一,每出獵,耆老皆隨而觀之。皇統間,以宗室子例授光祿大夫,封葛王,為兵部尚書。天德初,判會寧牧。明年,判大
 宗正事,改中京留守,俄改燕京,未幾,為濟南尹。貞元初,為西京留守,三年,改東京,進封趙王。正隆二年,例降封鄭國公,進封衛國。三年,再任留守,徙封曹國。六年五月,居貞懿皇后喪。一日方寢,有紅光照室,及黃龍見寢室上。又嘗夜有大星流入留守第中。是歲,東梁水漲溢,暴至城下,水與城等,決女墻石罅中流入城,湍激如涌,城中人惶駭,上親登城,舉酒酹之,水退。



 海陵南伐,天下騷動。是時,籍契丹部人丁壯為兵,部人不願行,以告使者,使者燥合畏海陵不以告,部人遂反。至是,咸平府謀克括里陷韓州,據咸平,將犯東京。



 八月,起復東京留守。
 婆速路兵四百來會討括里,復得城中子弟願為兵者數百人。帝舅興中少尹李石以病免,家居遼陽。戊午,發東京,以石主留務。賊覘者聞鼙鼓聲震天,見旌旗蔽野,傅言國公兵十萬且至,賊眾至瀋州,遁去。會烏延查剌等敗賊兵,還至常安縣,海陵使婆速路總管完顏謀衍來討賊,以兵屬之。



 九月,至東京。副留守高存福,其女在海陵後宮,海陵使存福伺起居。適以造兵器餘材造甲數十,存福宣言,留守何為造甲,密使人以白海陵,遂與推官李彥隆託為擊球,謀不利。存福家人以其謀來告,平定知軍李蒲速越亦言其事。海陵嘗聞上有疾,即使
 近習來觀動靜,至是,又使謀良虎圖淮北諸王,上知之,心常隱憂。及討括里還至清河,遇故吏六斤乘傳自南來,具言海陵殺其母,殺兄子檀奴、阿里白及樞密使僕散忽土等,又曰:「且遣人來害宗室兄弟矣!」上聞之,益懼。及聞存福圖己,事且有迹,李石勸上早圖之。於是,以議備賊事,召官屬會清安寺,彥隆先到,存福累召始來,並於座上執之。是月,復有雲來自西,黃龍見雲中。



 十月辛丑,南征萬戶完顏福壽、高忠建、盧萬家奴等自山東率所領兵二萬,完顏謀衍自常安率兵五千皆來附。謀衍即以臣禮上謁。乙巳,諸軍入城,共擊殺存福等。是夜,諸
 軍被甲環衛皇城。丙午,慶雲見,官屬諸軍勸進,固讓良久,於是親告天太祖廟,還御宣政殿,即皇帝位。以完顏謀衍為右副元帥,高忠建元帥左監軍,完顏福壽右監軍,盧萬家奴顯德軍節度使。丁未,大赦,改元大定。下詔暴揚海陵罪惡數十事。己酉,饗將士,賜官賞各有差,仍給復三年。會寧、胡里改、速頻等路南伐諸軍,會尚書省,奏請以從軍來者補諸局司承應人及官吏闕員。上曰:「舊人南征者即還,何以處之。必不可闕者,量用新人可也。」辛亥,以利涉軍節度使獨吉義為參知政事。中都留守、西北面行營都統完顏彀英將兵三萬駐歸化,以為
 左副元帥。丁巳,出內府金銀器物贍軍,吏民出財物佐官用者甚眾。壬戌,以前臨潢尹晏為左丞相。癸亥,詔諭南京太傅、尚書令張浩。甲子,興平軍節度使張玄素上謁。尚書省奏:「正隆軍興之餘,進錢粟者宜量授以官。」從之。詔遣移剌札八招契丹諸部為亂者。以前肇州防禦使神土懣為元帥右都監。



 十一月己巳朔,以左丞相晏兼都元帥。辛未,以戶部尚書李石為參知政事。己卯,詔調民間馬充軍用,事畢還主,死者給價。阿瑣、璋殺同知中都留守事沙離只,阿瑣自稱中都留守,璋自稱同知留守事,使石家奴等來上表賀。辛巳,以如中都期日詔
 群臣。壬午,詔中都轉運使左淵曰:「凡宮殿張設毋得增置,無役一夫以擾百姓,但謹圍禁,嚴出入而己。」以尚書右司員外郎完顏兀古出為詔諭高麗使。癸未,遣權元帥左都監吾札忽、右都監神土懣、廣寧尹僕散渾坦討契丹諸部。甲申,追尊皇考幽王為皇帝,謚簡肅,廟號睿宗,皇妣蒲察氏曰欽慈皇后,李氏曰貞懿皇后。群臣上尊號曰仁明聖孝皇帝。乙酉,追復東昏王帝號,謚武靈,廟號閔宗,詔中外。封子實魯剌為許王,胡土瓦為楚王。戊子,辭謁太祖廟及貞懿皇后園陵。己丑,如中都。次小遼口。使中都留守宗憲先往。壬辰,次梁魚務。樞密副使,
 北面行營都統白彥敬、南京留守北面行營副統紇石烈志寧以所統軍數來上。安武軍節度使爽來歸。乙未,完顏元宜等弒海陵於揚州。丙申,次義州。丁酉,宋人破陜州,防禦使折可直降,同知防禦使事李柔立死之。



 十二月乙卯,次三河縣,左副元帥完顏彀英來朝。丙辰,次通州,延安尹唐括德溫來朝。丁巳,至中都。戊午,謁太祖廟。巳未,御貞元殿,受群臣朝。庚申,以元帥左監軍高忠建等為報諭宋國使。壬戍,詔軍士自東京扈從至京師者復三年。同知河間尹高昌福上書陳便宜,上覽之再三。詔內外大小職官陳便宜。丙寅,詔左副元帥完顏彀
 英規措南邊及陜西等路事。



 二年正月戊辰朔,日有食之。伐鼓用幣,上徹樂減膳,不視朝。庚午,上謂宰相曰:「進賢退不肖,宰相之職也。有才能高於己者,或懼其分權,往往不肯引置同列,朕甚不取。卿等毋以此為心。」以前翰林學士承旨致仕翟永固為尚書左丞,濟南尹僕散忠義為右丞。都統斜哥、副統完顏布輝坐擅易置中都官吏,斜哥除名,布輝削兩階,罷之。辛未,御太和殿,宴百官,宗戚命婦賜賚有差。壬申,敕御史臺檢察六部文移,稽而不行,行而失當,皆舉劾之。甲戌,除迎賽神佛禁令。乙亥,如大房山。丙子,獻享山
 陵,禮畢,欲獵而還,左丞相晏等諫曰:「邊事未寧,不宜游幸。」戊寅,還宮。因諭晏等曰:「朕常慕古之帝王,虛心受諫。卿等有言即言,毋緘默以自便。」辛巳,以兵部尚書可喜等謀反,伏誅,詔中外。是日,賜扈從猛安謀克甲士下至阿里喜有差。遣左副點檢蒲察阿孛罕等賞賚河南將士。以前勸農使移剌元宜為御史大夫。詔前工部尚書蘇保衡、太子少保高思廉振賜山東百姓粟帛,無妻者具姓名以聞。庚寅,行納粟補官法。遣右副元帥完顏謀衍率師討蕭窩斡。壬辰,上謂宰執曰:「朕即位未半年,可行之事甚多,近日全無敷奏。朕深居九重,正賴卿等贊
 襄,各思所長以聞,朕豈有倦怠。」癸巳,太白晝見。甲午,上謂宰執曰:「卿等當參民間利害,及時事之可否,以時敷奏。不可公餘輒從自便,優游而已。」命河北、山東、陜西等路征南步軍並放還家。咸平、濟州軍二萬入屯京師。丙申,以西南路招討使完顏思敬、兵部尚書阿鄰督北邊將士。



 二月己亥,前翰林待制大穎以言盜賊忤海陵,杖而除名,起為秘書丞。補闕馬欽以諂事海陵得幸,除名。庚子,詔前戶部尚書梁金求、戶部郎中耶律道安撫山東百姓。招諭盜賊或避賊及避徭役在他所者,並令歸業,及時農種,無問罪名輕重,並與原免。壬寅,太傳、尚書令
 張浩來見。癸卯,以上初即位,遣遼陽主簿石抹移迭、東京曲院都監移剌葛補招契丹叛人,為白彥敬、紇石烈志寧所害,並贈鎮國上將軍,令其家各食五品俸,仍收錄其子。甲辰,以張浩為太師,尚書令如故,御史大夫移剌元宜為平章政事。辛亥,定世襲猛安謀克遷授格。壬子,以太保、左領軍大都督奔睹為都元帥,太保如故。癸丑,詔降蕭玉、敬嗣暉、許霖等官,放歸田里。甲寅,復用進士為尚書省令史。丙辰,嵩州刺史石抹術突刺等敗宋兵於壽安縣。丁巳,鄭州防禦使蒲察世傑取陜州。甲子,詔都元帥奔睹開府山東,經略邊事。澤州刺史特末哥
 及其妻高福娘伏誅。



 閏月甲戌,上謂宰臣曰:「比聞外議言,奏事甚難。朕於可行者未嘗不從。自今敷奏勿有所隱,朕固樂聞之。」戊子,上謂宰臣曰:「臣民上書者,多敕尚書省詳閱,而不即具奏,天下將謂朕徒受其言而不行也。其亟條具以聞。」庚寅,詔平章政事移剌元宜泰州路規措邊事。辛卯,太和、厚德殿火。乙未,尚書兵部侍郎溫敦術突剌等與窩斡戰,敗于勝州。



 三月癸亥,參知政事獨吉義罷。元帥左都監徒單合嘉敗宋將吳璘于德順州。甲辰,追削李通官職。乙巳,免南京正隆丁夫貸役錢。辛亥,以廉平誡諭中外官吏。癸亥,詔河南、陜西、山東,昨
 因捕賊,良民被虜為賊者,釐正之。


四月己巳,右副元帥完顏謀衍等敗窩斡于長濼。辛未,降廢帝亮為海陵郡王。乙亥,詔減御膳及宮中食物之半。夏國遣使來賀即位,及進方物,及賀萬春節。右副元帥完顏謀衍復敗窩斡於霿
 \gezhu{
  松}
 河。辛巳,宴夏使貞元殿。故事,外國使三節人從皆坐廡下賜食。上察其食不精腆,曰:「何以服遠人之心。」掌食官皆杖六十。癸未,夏使朝辭,乞互市,從之。己丑,以左丞相晏為太尉。壬辰,詔徵契丹部將士曰:「應契丹與大軍未戰而降者,不得殺傷,仍安撫之。後招誘來降者,除奴婢以已虜為定,其親屬使各還其家,仍官為贖
 之。」



 五月丁酉朔,以曷速館節度使白彥敬為御史大夫。戊戌,遣元帥左監軍高忠建會北征將帥討契丹。己亥,以臨海軍節度使紇石烈志寧為元帥右監軍。右副元帥完顏謀衍、元帥右監軍完顏福壽坐逗遛,召還京師,皆罷之。壬寅,立楚王允迪為皇太子,詔中外。丁巳,押軍萬戶裴滿按剌、猛安移剌沙里剌敗宋兵于華州。



 六月戊辰,命御史大夫白彥敬西北路市馬。庚午,以尚書右丞僕散忠義為平章政事兼右副元帥,經略契丹。詔出內府金銀給徵契丹軍用。戊寅,詔居庸關、古北口譏察契丹姦細,捕獲者加官賞。己卯,詔守禦古北口及石門
 關。庚辰,宋遣使賀即位。壬午,右副元帥僕散忠義與窩斡戰于花道。戊子,以南京留守紇石烈良弼為尚書右丞。庚寅,右副元帥僕散忠義大敗窩斡于裊嶺西陷泉。獲其弟裊。壬辰,以西南路招討使完顏思敬為元帥右都監。



 七月丁酉,復取原州。丙午,宋主傅位于子甗。甲寅,詔諭契丹。丁巳,速頻軍士術里古等誣完顏謀衍子斜哥寄書其父謀反,并以其書上之。上覽書曰:「此誣也,止訊告者。」訊之,果誣也。術里古伏誅。庚申,太尉、尚書左丞相晏致仕。壬戌,詔發濟州會寧府軍在京師者,以五千人赴北京都統府。陜西都統璋敗宋將吳璘于張義堡。



 八月乙丑朔,奚抹白謀克徐列等降。左監軍高忠建破奚于栲栳山,及招降旁近奚六營,有不降者,攻破之。盡殺其男子,以其婦女童孺分給諸軍。丁卯,永興縣進嘉禾。壬申,萬戶溫迪罕阿魯帶與奚戰于古北口,敗焉,詔同判大宗正事完顏謀衍等禦之。癸酉,上謂宰臣曰:「百姓上書陳時政,其言猶有所補。卿等位居機要,略無獻替,可乎?夫聽斷獄訟,簿書期會,何人不能?唐、虞之聖,猶務兼覽博照,乃能成治。正隆專任獨見,故取敗亡。朕早夜孜孜,冀聞讜論,卿等宜體朕意。」詔:「百司官吏,凡上書言事或為有司所抑,許進表以聞,朕將親覽,以觀人材
 優劣。」夏國遣使賀尊號。丁丑,免齊國妃、韓王亨、樞密忽土、留守賾等家親屬在宮籍者。詔元帥右都監完顏思敬以所部軍與大軍會討窩斡。乙酉,詔左諫議大夫石琚、監察御史馮仲廉察河北東路。丁亥,詔御史臺曰:「卿等所劾,惟諸局行移稽緩,及緩於赴局者耳,此細事也。自三公以下,官僚善惡邪正,當審察之。若止理細務而略其大者,將治卿等罪矣!」契丹老和尚降。辛卯,罷諸關征稅。



 九月甲午朔,完顏謀衍擒奚猛安合住。元帥左都監徒單合喜大敗宋將吳璘于德順州。乙未,詔尚書右丞紇石烈良弼以便宜招撫奚、契丹之叛者。庚子,元
 帥右都監完顏思敬獲契丹窩斡,餘眾悉平。以尚書左司員外郎完顏正臣為夏國生日使。壬寅,獵于近郊。乙巳,以移剌窩斡平,詔中外。庚戌,改葬睿宗皇帝。壬子,以元帥右都監完顏思敬為右副元帥。戊午,詔思敬經略南邊。辛酉,奉遷睿宗皇帝梓宮于磐寧宮。癸亥,元帥左監軍徒單合喜等敗宋兵于德順州。河南統軍使宗尹復取汝州。



 十月丁卯,以左副元帥完顏彀英為平章政事。戊辰,如山陵,謁睿宗皇帝梓宮,哭盡哀。平章政事、右副元帥僕散忠義等還自軍,上謁。丙戌,以僕散忠義為尚書右丞相、元帥左監軍紇石烈志寧為左副元帥。戊
 子,葬睿宗皇帝于景陵,大赦。己丑,詔左副元帥紇石烈志寧經略南邊。壬辰,華州防禦使蒲察世傑、丹州刺史赤盞胡速魯改敗宋兵于德順州。



 十一月癸巳朔,詔右丞相僕散忠義伐宋。丁酉,第職官,廉能、汙濫、不職各為三等而黜陟之。



 十二月乙酉,遣尚書刑部侍郎劉仲淵等廉察宣諭東京、北京等路。



 三年正月壬辰朔,高麗、夏遣使來賀。庚子,太白晝見。壬子,遣客省使烏居仁賞勞河南軍士。癸丑,復取德順州。



 二月甲子,詔太子少詹事楊伯雄等廉問山西路。庚午,上謂宰相曰:「灣州飢民,流散逐食,甚可矜恤。移於山西,
 富民贍濟,仍于道路計口給食。」壬申,詔撫諭陜西。庚辰,太保、都元帥奔睹薨。丙戌,趙景元等以亂言伏誅。庚寅,高麗、夏遣使來賀萬春節。高麗遣使賀即位。東京僧法通以妖術亂眾,都統府討平之。



 三月丙申,中都以南八路蝗,詔尚書省遣官捕之。壬寅,詔戶部侍郎魏子平等九人,分詣諸路猛安謀克,勸農及廉問。詔臨潢漢民遂食於會寧府濟、信等州。庚戌,詔免去年租稅。



 四月辛酉朔,右副元帥完顏思敬罷。丁卯,平章政事完顏彀英、御史大夫白彥敬罷。以參知政事李石為御史大夫。丁丑,詔吏犯贓罪,雖會赦不敘。己卯,以引進使韓綱為橫賜
 高麗使。乙酉,賑山西路猛安謀克貧民,給六十日糧。是月,取商、虢、環州,宋所侵一十六州至是皆復。



 五月辛卯朔,右丞相僕散忠義朝京師。乙未,以重五,幸廣樂園射柳,命皇太子、親王、百官皆射,勝者賜物有差。上復御常武殿,賜宴擊球。自是歲以為常。丙申,宋人攻破靈璧、虹縣。己亥,罷河南、山東、陜西統軍司,置都統、副統。以太子詹事完顏守道從皇太子,上召諭守道曰:「卿任執政,所責非輕,自今毋從行。」辛丑,以右丞相僕散忠義兼都元帥。癸卯,僕散忠義還軍。河南路都統奚撻不也叛入于宋。丙午,宋人攻破宿州。辛亥,更定出征軍逃亡法。尚書
 省請籍天德間被誅大臣諸奴隸及從窩斡亂者為軍,上以四方甫定,民意稍蘇,而復簽軍,非長策,不聽。癸丑,詔諭契丹餘黨蒲速越等,如能自新,並釋其罪。若執蒲速越父子以來者,仍官賞之。左副元帥紇石烈志寧復取宿州,河南副統孛術魯定方死于陣。乙卯,以北京留守完顏思敬復為右副元帥。中都蝗。詔參知政事完顏守道按問大興府捕蝗官。



 六月庚申朔,日有食之。以刑部尚書蘇保衡為參知政事。丙子,詔曰:「正隆之末,濟州路逃回軍士為中都軍所邀殺者,官為收葬。」己卯,觀稼于近郊。甲申,太師、尚書令張浩罷。以宿直將軍阿勒
 根和衍為橫賜夏國使。



 七月庚戌,太白晝見。以太子太師宗憲為平章政事。以孔總為襲封衍聖公。



 八月丙寅,太白經天。庚午,詔曰:「祖宗時有勞效未曾遷賞者,五品以上奏聞,六品以下及無職事者尚書省約量升除。」甲戌,詔參知政事完顏守道招撫契丹餘黨。戊寅,詔罷契丹猛安謀克,其戶分隸女直猛安謀克。命諸官員年老者,許存馬一二匹,餘並括買入官。敕殿前都點檢唐括德溫:「重九出獵,國朝舊俗。今扈從軍二千,能無擾民?可嚴為約束,仍以錢萬貫分賜之。」乙酉,如大房山。丁亥,薦享于睿陵。戊子,還宮。



 九月癸巳,以宿直將軍僕散習尼列為夏
 國生日使。丁酉,秋獵。以重九,拜天于北郊。丙午,詔翰林待制劉仲誨等廉問車駕所經州縣。乙卯,還宮。



 十月甲子,大享于太廟。丙寅,以許王府長史移剌天佛留為高麗生日使。癸酉,冬獵。



 十一月庚寅,太白晝見,經天。壬辰,還都。戊申,詔:「求仕官輒入權要之門,追一官,仍降除。以請求有所饋獻及受之者,具狀奏裁。」庚戌,百官請上尊號,不允。詔:「中都、平州及饑荒地並經契丹剽掠,有質賣妻子者,官為收贖。」壬子,尚書左丞翟永固罷。癸丑,罷貢金線段匹。甲寅,以尚書右丞紇石烈良弼為左丞,吏部尚書石琚為參知政事。



 十二月丁丑,臘,獵于近郊。
 以所獲薦山陵,自是歲以為常。詔流民未復業,增限招誘。己卯,參知政事蘇保衡至自軍,辛巳,以為尚書右丞。



 四年正月丁亥朔,高麗、夏遣使來賀。戊子,罷路府州元日及萬春節貢獻。上謂侍臣曰:「秦王宗翰有功於國,何乃無嗣?」皆未知所對。上曰:「朕嘗聞宗翰在西京坑殺丐者千人,得非其報耶?」癸巳,百官復請上尊號,不允。丁酉,如安州春水。壬寅,至安州。大雪。詔扈從人舍民家者,人日支錢一百與其主。甲辰,元帥府言:「宋遣審議官胡昉致尚書右僕射書,來議和好。以其言失信,拘昉軍中,以書答之。」及以書進,上覽之曰:「宋之失信,行人何罪?當即
 遣還。邊事令元帥府從宜措畫。」乙巳,尚書省奏:「徐州民曹珪討賊江志,而子弼亦在賊中,并殺之。法當補二官,敘雜班。」上以所奏未當,進一官,正班用之。辛亥,獲頭鵝。遣使薦山陵,自是歲以為常。



 二月丁巳,免安州今年賦役,及保塞縣御城邊吳二村凡扈從人嘗止其家者,亦復一年。辛酉,獵于高陽之北。庚午,還都。庚辰,以北京粟價踴貴,詔免今年課甲。



 三月丙戌朔,萬春節,高麗、夏遣使來賀。詔免北京歲課段匹一年。庚子,京師地震。壬寅,百官復請上尊號,不允。



 四月丁巳,平章政事完顏元宜罷。甲戌,出宮女二十一人。



 五月,旱。癸卯,敕有司審冤獄,
 禁宮中音樂,放球場役夫。乙巳,詔禮部尚書王競禱雨於北岳。己酉,命參知政事石琚等於北郊望祭禱雨。壬子,雨。窩乾餘黨蒲速越伏誅。



 六月甲寅朔,日有食之。壬戌,尚書左丞紇石烈良弼至自征南元帥府。甲子,以雨足,命有司祭謝嶽鎮海瀆于北郊。己巳,幸東宮,視皇太子疾。庚午,初定祭五嶽四瀆禮。辛未,觀稼于近郊。庚辰,詔諭元帥府曰:「所請伐宋軍萬五千,今以騎三千,步四千赴之。」詔陜西元帥府議入蜀利害以聞。



 七月壬辰,故衛王襄妃及其子和尚以妖妄伏誅。庚子,以尚書左丞紇石烈良弼為平章政事。辛丑,大風雷雨,拔木。



 八月甲
 寅朔,詔征南元帥府曰:「前所請收復舊疆,乞候秋涼進發,今已秋涼,復俟何時?」戊午,以參知政事完顏守道為尚書左丞,大興尹唐括安禮為參知政事。壬申,上謂宰臣曰:「卿每奏皆常事,凡治國安民及朝政不便於民者,未嘗及也。如此,則宰相之任誰不能之?」己卯,如大房山。辛巳,致祭于山陵。



 九月癸未朔,還都。乙酉,上謂宰臣曰:「形勢之家,親識訴訟,請屬道達,官吏往往屈法徇情,宜一切禁止。」己丑,上謂宰臣曰:「北京、懿州、臨潢等路嘗經契丹寇掠,平、薊二州近復蝗旱,百姓艱食,父母兄弟不能相保,多冒鬻為奴,朕甚閔之。可速遣使閱實其數,出
 內庫物贖之。」乙未,幸鷹房,主者以鷹隼置內省堂上,上怒曰:「此宰相聽事,豈置鷹隼處耶?」痛責其人,俾置他所。己亥,以宿直將軍烏里雅為夏國生日使。辛亥,以太子少詹事烏古論三合為高麗生日使。



 十月癸丑朔,獵于密雲縣。丙寅,還都。己卯,命泰寧軍節度使張弘信等二十四人分路通檢諸路物力。



 十一月乙酉,征南都統徒單克寧敗宋兵,取楚州。己丑,封子永功為鄭王。辛卯,冬獵。乙未,詔進師伐宋。戊戌,次河間府。辛丑,尚書省火。甲辰,次清州。閏月壬子朔,還都。



 十二月丁亥,尚書省奏都統高景山取商州。己丑,臘,獵于近郊。辛卯,太白晝見,經
 天。是歲,大有年。斷死罪十有七人。



 五年正月辛亥朔,高麗、夏遣使來賀。乙卯,詔泰州、臨潢接境設邊堡七十,駐兵萬三千。己未,宋通問使魏杞等以國書來。書不稱「大」,稱「姪宋皇帝」,稱名「再拜奉書于叔大金皇帝」。歲幣二十萬。辛未,詔中外。復命有司,旱、蝗、水溢之處,與免租賦。癸酉,命元帥府諸新舊軍以六萬人留戍,餘並放還。以宋國歲幣悉賞諸軍。



 二月壬午,以左副都點檢完顏仲等為宋報問使。壬寅,罷納粟補官令。戊申,萬春節,宋、高麗、夏遣使來賀。



 三月壬申,群臣奉上尊號曰應天興祚仁德聖孝皇帝,詔中外。



 四月癸卯,西
 京留守壽王京謀反,獄成,特免死,杖之,除名,嵐州安置。乙巳,右副元帥完顏思敬罷。丁未,右丞相、都元帥僕散忠義還自軍。



 五月壬子,左副元帥紇石烈志寧以召入見。丁巳,以僕散忠義為尚書左丞相,紇石烈志寧為平章政事,還軍。乙丑,以平章政事宗憲為尚書右丞相。癸酉,罷山東路都統府,以其軍各隸總管府。



 六月甲辰,芝產大安殿柱。丙午,京師地震,雨毛。



 七月戊申朔,京師地復震。罷陜西都統府,復置統軍司京兆,徙陜西元帥府河中。



 八月己卯,前宿州防禦使烏林答剌撒以與宋李世輔交通,伏誅。癸巳,宋、夏遣使賀尊號。



 九月丁未朔,以
 吏部尚書高衎等為賀宋生日使。戊申,秋獵。庚戌,以宿直將軍術虎蒲查為夏國生日使。甲戌,還都。



 十月丁丑朔,地震。辛巳,以大宗正丞璋為高麗生日使。乙未,冬獵。辛丑,還都。



 十一月丙午朔,上謂宰臣曰:「朕在位日淺,未能遍識臣下賢否?全賴卿等盡公舉薦。今六品以下殊乏人材,何以副朕求賢之意。」癸丑,幸東宮。戊午,以右副都點檢烏古論粘沒曷為賀宋正旦使。癸亥,立諸路通檢地土等稅法。癸酉,大霧。晝晦。



 十二月己丑,獵于近郊。高麗遣使賀尊號。



 六年正月丙午朔,宋、高麗、夏遣使來賀。庚午,敕有司:「宮
 中張設毋以塗金為飾。」



 二月丁亥,尚書左丞相兼都元帥沂國公僕散忠義薨。壬寅,萬春節,宋、高麗、夏遣使來賀。



 三月甲寅,上如西京。庚申,次歸化州,西京留守唐括德溫上謁。戊辰,至西京。庚午,朝謁太祖廟。壬申,擊球,百姓縱觀。



 四月甲戌朔,詔月朔禁屠宰。戊戌,以尚書右司郎中移剌道為橫賜高麗使,宿直將軍斜卯摑剌為橫賜夏國使。辛丑,太白晝見。



 五月戊申,幸華嚴寺,觀故遼諸帝銅像,詔主僧謹視之。壬子,詔雲中大同縣及警巡院給復一年。壬戌,詔將幸銀山,諸扈從軍士賜錢五萬貫,有敢損苗稼者,並償之。



 六月辛巳,太白晝見,經天。丙
 戌,發自西京。庚子,獵于銀山。



 七月辛酉,次三叉口。



 八月辛未朔,次涼陘。庚辰,獵于望雲之南山。



 九月辛丑朔,至自西京。丁未,以戶部尚書魏子平為賀宋生日使。辛亥,以翰林待制移剌熙載為夏國生日使。澤州刺史劉德裕等以盜用官錢伏誅。壬子,太白晝見。癸丑,尚書右丞相宗憲薨。丙辰,太白晝見,經天。



 十月己卯,以尚書兵部侍郎移剌按答為高麗生日使。甲申,朝享于太廟。詔免雄、莫等州今年租。壬辰,太白晝見,經天。丁酉,如安肅州。冬獵。



 十一月丙午,還都。癸丑,以右副都點檢烏古論元忠為賀宋正旦使。上謂宰臣曰:「朝官當慎選其人,庶可
 激勵其餘,若不當,則啟覬覦之心。卿等必知人才優劣,舉實才用之。」庚申,太白晝見,經天。丁卯,參知政事石琚以母憂罷。



 十二月甲戌,詔有司,每月朔望及上七日毋奏刑名。戊子,太白晝見,經天。甲午,泰州民合住謀反,伏誅。丙申,以平章政事紇石烈良弼為尚書右丞相,紇石烈志寧為樞密使。



 七年正月庚子朔,宋、高麗、夏遣使來賀。辛亥,石琚起復參知政事。壬子,上服袞冕,御大安殿,受尊號冊寶禮。癸丑,大赦,庚申,以元帥左監軍徒單合喜為樞密副使。



 二月庚寅,尚書右丞蘇保衡薨。丙申,以參知政事石琚為
 尚書右丞。



 三月己亥朔,萬春節,宋、高麗、夏遣使來賀。



 四月戊辰朔,日有食之。壬辰,以御史大夫李石為司徒,大夫如故。



 五月丙午,大興府獄空,詔賜錢三百貫為宴樂之用,以勞之。甲寅,以北京留守耨碗溫敦兀帶為參知政事。



 六月癸酉,命地衣用龍文者罷之。



 七月戊申,禁服用金線,其織賣者,皆抵罪。丙辰,幸東宮。己未,幸東宮視皇太子疾。



 閏月丁卯,觀稼于近郊。戊辰,許王永中進封越王,鄭王永功封隨王,永成封審王。甲戌,詔遣秘書監移剌子敬經略北邊。戊寅,幸東宮。己卯,慶雲環日。壬午,觀稼于近郊。戊子,觀稼于北郊。



 八月辛亥,慶雲環日。癸
 丑,尚書右丞相監修國史紇石烈良弼進《太宗實錄》,上立受之。己未,如大房山。壬戌,致祭睿陵。



 九月乙丑朔,還宮。己巳,右三部檢法官韓贊以捕蝗受賂,除名。詔吏人但犯贓罪,雖會赦,非待旨不敘。以勸農使蒲察莎魯窩等為賀宋生日使。辛未,參知政事唐括安禮罷。乙亥,以宿直將軍唐括鶻魯為夏國生日使。庚辰,地震。辛巳,以都水監李衛國為高麗生日使。乙酉,秋獵。庚寅,次保州。詔修起居注王天祺察訪所經過州縣官。



 十月乙未朔,上謂侍臣曰:「近聞朕所幸郡邑,會宴寢堂宇,後皆避之,此甚無謂,可宣諭,令仍舊居止。」戊申,還都。丁巳,上謂宰
 臣曰:「海陵不辨人才優劣,惟徇己欲,多所升擢。朕即位以來,以此為戒,止取實才用之。近聞蠡州同知移剌延壽在官污濫,詢其出身,乃正隆時鷹房子。如鷹房、廚人之類,可典城牧民耶?自今如此局分,不得授以臨民職任。」以御史中丞孟浩為參知政事。是日,參知政事耨碗溫敦兀帶薨。辛酉,敕有司於東宮涼樓前增建殿位,孟浩諫曰:「皇太子雖為儲貳,宜示以儉德,不當與至尊宮室相俟。」乃罷之。



 十一月乙丑朔,上謂宰臣曰:「聞縣令多非其人,其令吏部察其善惡,明加黜陟。」辛未,以河間尹徒單克寧等為賀宋正旦使。壬申,太白晝見。丁丑,歲星
 晝見。丁亥,樞密副使徒單合喜罷。



 十二月戊戌,東京留守徒單合喜、北京留守完顏謀衍、肇州防禦使蒲察通朝辭,賜通金帶,諭之曰:「卿雖有才,然用心多詐,朕左右須忠實人,故命卿補外。賜卿金帶者,答卿服勞之久也!」又顧謂左宣徽使敬嗣暉曰:「如卿不可謂無才,所欠者純實耳!」甲辰,以北京留守完顏思敬為平章政事。是歲,斷死囚二十人。



 八年正月甲子朔,宋、高麗、夏遣使來賀。乙丑,上謂宰臣曰:「朕治天下,方與卿等共之,事有不可,各當面陳,以輔朕之不逮,慎毋阿順取容。卿等致位公相,正行道揚名
 之時,茍或偷安自便,雖為今日之幸,後世以為何如?」群臣皆稱萬歲。辛未,謂秘書監移剌子敬等曰:「昔唐、虞之時,未有華飾,漢惟孝文務為純儉。朕於宮室惟恐過度,其或興修,即損宮人歲費以充之,今亦不復營建矣!如宴飲之事,近惟太子生日及歲元嘗飲酒,往者亦止上元、中秋飲之,亦未嘗至醉。至於佛法,尤所未信。梁武帝為同泰寺奴,遼道宗以民戶賜寺僧,復加以三公之官,其惑深矣!」庚辰,行皇太子冊禮。



 二月甲午朔,制子為改嫁母服喪三年。上諭左宣徽使敬嗣暉曰:「凡為人臣,上欲要君之恩,下欲干民之譽,必虧忠節,卿宜戒之!」



 三月
 癸亥朔,萬春節,宋、高麗、夏遣使來賀。己巳,命以職官子補令史。丁丑,命護衛親軍百戶、五十戶,非直日不得帶刀入宮,己丑,太白晝見。



 四月丙午,詔曰:「馬者軍旅所用,牛者農耕之資,殺牛有禁,馬亦何殊,其令禁之。」戊申,擊球常武殿,司天馬貴中諫曰:「陛下為天下主,繫社稷之重,又春秋高,圍獵擊球危事也,宜悉罷之。」上曰:「朕以示習武耳!」五月甲子,北望澱大震、風、雨雹,廣十里,長六十里。詔戶、工兩部,自今宮中之飾,並勿用黃金。乙丑,上如涼陘。丁卯,歲星晝見。庚寅,改旺國崖曰靜寧山,曷里滸東川曰金蓮川。



 六月,河決李固渡,水入曹州。



 七月甲子,
 制盜群牧馬者死,告者給錢三百貫。戊辰,上謂平章政事完顏思敬等曰:「朕思得賢士,寤寐不忘。自今朝臣出外,即令體訪外任職官廉能者,及草萊之士可以助治者,具姓名以聞。」甲戌,秋獵。己卯,次三叉口。上諭點檢司曰:「沿路禾稼甚佳,其扈從人少有蹂踐,則當汝罪。」



 八月乙卯,至自涼陘。



 九月辛酉,上諭尚書右丞石琚、參政孟浩曰:「聞蔚州採地蕈,役夫數百千人,朕所用幾何?而擾動如此。自今差役凡稱御前者,皆須稟奏,仍令附冊。」癸亥,以右宣徽使移剌神獨斡等為賀宋生日使。己巳,以引進使高希甫為夏國生日使。庚午,上幸東宮。癸酉,上
 諭宰臣曰:「卿等舉用人材,凡己所知識,必使他人舉奏,朕甚不喜。如其果賢,何必以親疏為避忌也。」以戶部尚書魏子平為參知政事。辛巳,上謂御史大夫李石曰:「臺憲固在分別邪正,然內外百司豈謂無人?惟見卿等劾人之罪,不聞舉善。自今宜令監察御史分路刺舉善惡以聞。」上嘗命左衛將軍大磐訪求良弓,而磐多自取,護衛婁室以告,上命點檢司鞫磐。磐妹為寶林,磐屬內侍僧兒言之寶林,寶林以聞,命杖僧兒百,出磐為隴州防禦使。



 十月己丑朔,以戒諭官吏貪墨,詔中外。乙未,命涿州刺史兼提點山陵,每以朔望致祭,朔則用素,望則用
 肉,仍以明年正月為首。及命圖畫功臣於太祖廟,其未立碑者立之。以翰林待制靖為高麗生日使。上謂宰臣曰:「海陵時,修起居注不任直臣,故所書多不實。可訪求得實,詳而錄之。」參政孟浩進曰:「良史直筆,君舉必書,自古帝王不自觀史,意正在此。」辛亥,詔罷復州歲貢鹿筋。



 十一月乙丑,幸東宮。以同簽大宗正事闢合土等為賀宋正旦使。



 十二月戊子朔,遣武定軍節度使移剌按等招諭阻珝。



 九年正月戊午朔,宋、高麗、夏遣使來賀。辛酉,上與宣徽使敬嗣暉、秘書監移剌子敬論古今事,因曰:「亡遼日屠
 食羊三百,亦豈能盡用,徒傷生耳!朕雖處至尊,每當食,常思貧民飢餒,猶在己也。彼身為惡而口祈福,何益之有?如海陵以張仲軻為諫議大夫,何以得聞忠言。朕與大臣論議一事,非正不言,卿等不以正對,豈人臣之道也!」庚午,詔諸州縣和糴,毋得抑配百姓。戊寅,契丹外失剌等謀叛,伏誅。丙戌,制漢人、渤海兄弟之妻,服闋歸宗,以禮續婚者,聽。



 二月庚寅,制妄言邊關兵馬者,徒二年。丙申,詔改葬漢二燕王於城東。庚子,以中都等路水,免稅,詔中外。又以曹、單二州被水尤甚,給復一年。甲寅,詔女直人與諸色人公事相關,只就女直理問。



 三月丁巳
 朔,萬春節,宋、高麗、夏遣使來賀。丁卯,以尚書省定網捕走獸法,或至徒,上曰:「以禽獸之故而抵民以徒,是重禽獸而輕民命也,豈朕意哉!自今有犯,可杖而釋之。」詔御史中丞移剌道廉問山東、河南。辛未,禁民間稱言「銷金」,條理內舊有者,改作「明金」字。辛巳,以大名路諸猛安民戶艱食,遣使發倉廩減價出之。



 四月己丑,謂宰臣曰:「朕觀在位之臣,初入仕時,競求聲譽以取爵位,亦既顯達,即徇默茍容為自安計,朕甚不取。宜宣諭百官,使知朕意。」癸巳,遣翰林脩撰蒲察兀虎、監察御史完顏鶻沙分詣河北西路、大名、河南、山東等路勸猛安謀克農。



 五月
 丙辰朔,以符寶郎徒單懷貞為橫賜高麗使,宿直將軍完顏賽也為橫賜夏國使。戊辰,尚書省奏越王永中、隋王永功二府有所興造,發役夫。上曰:「朕見宮中竹有枯瘁者,欲令更植,恐勞人而止。二王府各有引從人力,又奴婢甚多,何得更役百姓。爾等但以例為請,海陵橫役無度,可盡為例耶!自今在都浮役,久為例者仍舊,於並官給傭直,重者奏聞。」



 六月庚寅,冀州張和等反,伏誅。戊戌,以久旱,命宮中毋用扇。庚子,雨。



 七月乙卯朔,罷東北路採珠。壬申,觀稼于近郊。



 八月甲申朔,有司奏日食,以雨不見,伐鼓用幣如常禮。



 九月甲寅朔,以刑部尚書高
 德基等為賀宋生日使,宿直將軍僕散守中為夏國生日使,提點司天臺馬貴中為高麗生日使。罷皇太子月料,歲給錢五萬貫。上謂臺臣曰:「比聞朝官內有攬中官物以規貸利者,汝何不言?」皆對曰:「不知。」上曰:「朕尚知之,汝有不知者乎?朕若舉行,汝將安用!」壬戌,秋獵。



 十月丁亥,還都。辛丑,以尚書右丞相紇石烈良弼為左丞相,樞密使紇石烈志寧為右丞相。詔宗廟之祭,以鹿代牛,著為令。丙午,大享于太廟。辛亥,以平章政事完顏思敬為樞密使。



 十一月己未,以尚書左丞完顏守道為平章政事,右丞石琚為左丞,參知政事孟浩為右丞。庚申,上幸
 東宮。辛酉,以京兆尹毅等為賀宋正旦使。壬戌,冬獵。丙子,還都。



 十二月丙戌,詔賑臨潢、泰州、山東東路,河北東路諸猛安民。以東京留守徒單合喜為平章政事。丁酉,太白晝見。辛丑,獵于近郊。丙午,制職官犯公罪,在官已承伏者,雖去官猶論。



 十年正月壬子朔,宋、高麗、夏遣使來賀。甲子,命宮中元宵無得張燈。甲戌,以司徒、御史大夫李石為太尉、尚書令。



 二月甲午,安化軍節度使徒單子溫、副使老君奴以贓罪,伏誅。戊申,上謂近臣曰:「護衛以後皆是治民之官,其令教以讀書。」



 三月壬子朔,萬春節,宋、高麗、夏遣使來
 賀。丙辰,上因命護衛中善射者押賜宋使射弓宴,宋使中五十,押宴者纔中其七,謂左右將軍曰:「護衛十年出為五品職官,每三日上直,役亦輕矣,豈徒令飽食安臥而已!弓矢不習,將焉用之?」戊午,以河南統軍使宗敘為參知政事。庚午,上謂參政宗敘曰:「卿昨為河南統軍時,言黃河堤埽利害,甚合朕意。朕每念百姓差調,官吏互為姦弊,不早計料,臨期星火率斂,所費倍蓰,為害非細。卿既參朝政,皆當革弊,擇利行之。」又諭左丞石琚曰:「女直人徑居達要,不知閭閻疾苦。汝等自丞簿至是,民間何事不知,凡有利害,宜悉敷陳。」



 四月丁酉,制命婦犯姦,
 不用夫廕以子封者,不拘此法。



 五月乙卯,如柳河川。



 閏月庚辰,夏國任得敬脅其主李仁孝,使上表,請中分其國。上問宰臣李石,石等以為事繫彼國,不如許之。上曰:「彼劫於權臣耳!」詔不許,并卻其貢物。



 七月壬午,秋獵。戊戌,放圍場役夫。詔扈從糧食並從官給。乙巳,敕扈從人縱畜牧蹂踐禾稼者,杖之,仍償其直。



 八月己未,至自柳河川。壬申,遣參知政事宗敘北巡。



 九月庚辰,尚書左丞相紇石烈良弼丁憂,起復如故。壬午,以簽書樞密院事移剌子敬為賀宋生日使。庚寅,以戶部郎中夾谷阿里補為夏國生日使。



 十月己酉,以大宗正丞糺為高麗生
 日使。甲寅,如霸州,冬獵。乙丑,上謂大臣曰:「比因巡獵,聞固安縣令高昌裔不職,已令罷之。霸州司候成奉先奉職謹恪,可進一階,除固安令。」辛未,上謂宰臣曰:「朕凡論事有未能深究其利害者,卿等宜悉心論列,無為面從而退有後言。」



 十一月辛巳,制盜太廟物者與盜宮中物論同。甲申,上幸東宮。丁亥,以太子詹事蒲速察蒲越等為賀宋正旦使。癸巳,夏國以誅任得敬遣使來謝,詔慰諭之。



 十二月丙寅,上謂宰臣曰:「比體中不佳,有妨朝事。今觀所奏事,皆依條格,殊無一利國之事。若一朝行一事,歲計有餘,則其利博矣!朕居深宮,豈能悉知外事?卿
 等尤當注意。」



 十一年正月丙子朔,宋、夏遣使來賀。丁丑,封子永升為徐王,永蹈為滕王,永濟為薛王。壬午,詔職官年七十以上致仕者,不拘官品,並給俸祿之半。丙申,命賑南京屯田猛安被水災者。戊戌,尚書省奏汾陽軍節度副使牛信昌生日受饋獻,法當奪官。上曰:「朝廷行事茍不自正,何以正天下。尚書省、樞密院生日節辰饋獻不少,此而不問,小官饋獻即加按劾,豈正天下之道?自今宰執樞密饋獻亦宜罷去!」上謂宰臣曰:「往歲清暑山西,近路禾稼甚廣,殆無畜牧之地,因命五里外乃得耕墾。今聞民
 皆去之他所,甚可矜憫,其令依舊耕種。事有類此,卿等宜即告朕。」



 三月乙亥朔,萬春節,宋、夏遣使來賀。辛巳,命有司以天水郡公旅櫬依一品禮葬於鞏洛之原。



 四月丁未,歸德府民臧安兒謀反,伏誅。大理卿李昌圖以廉問真定尹徒單貞、咸平尹石抹阿沒剌受贓不法,既得罪狀,不即黜罷,杖之四十。癸亥,參知政事魏子平罷。高麗國王晛弟皓,廢其主自立,詐稱讓國,遣使以表來上。



 五月辛卯,詔遣吏部侍郎靖使高麗問故。癸巳,以南京留守移剌成為樞密副使。



 六月己酉,詔曰:「諸路常貢數內,同州沙苑羊非急用,徒勞民爾,自今罷之。朕居深宮,
 勞民之事豈能盡知?似此當具以聞。」戊午,觀稼于近郊。甲子,平章政事徒單合喜薨。



 七月甲申,參知政事宗敘薨。



 八月癸卯朔,太白晝見。詔朝臣曰:「朕嘗諭汝等,國家利便,治體遣闕,皆可直言。外路官民亦嘗言事,汝等終無一語。凡政事所行,豈能皆當?自今直言得失,毋有所隱。」乙巳,上謂宰臣曰:「隨朝之官,自謂歷一考則當得某職,兩考則當得某職。第務因循,碌碌而已。自今以外路官與內除者,察其公勤則升用之,但茍簡於事,不須任滿,便以本品出之。賞罰不明,豈能勸勉。」庚戌,詔曰:「應因窩斡被掠女直及諸色人未經刷放者,官為贖放。隱匿
 者,以違制論。其年幼不能稱說住貫者,從便住坐。」上謂宰臣曰:「五品以下闕員甚多,而難於得人。三品以上朕則知之,五品以下不能知也。卿等會無一言見舉者。欲畫久安之計,興百姓之利,而無良輔佐,所行皆尋常事耳,雖日日視朝,何益之有。卿等宜勉思之。」己巳,以尚書刑部侍郎烏林答天錫等為賀宋生日使,近侍局使劉珫為夏國生日使。



 九月癸未,獵于橫山。庚寅,還都。



 十月壬寅朔,以左宣徽使敬嗣暉為參知政事。甲寅,上謂宰臣曰:「朕已行之事,卿等以為成命不可復更,但承順而已,一無執奏。且卿等凡有奏,何嘗不從。自今朕旨雖出,
 宜審而行,有未便者,即奏改之。或在下位有言尚書省所行未便,亦當從而改之,毋拒而不從。」丙寅,尚書左丞相紇石烈良弼進《睿宗實錄》。戊辰,上謂宰臣曰:「衍慶宮圖畫功臣,已命增為二十人。如丞相韓企先,自本朝興國以來,憲章法度,多出其手。至於關決大政,但與大臣謀議,終不使外人知覺。漢人宰相,前後無比,若褒顯之,亦足示勸,慎無遺之。」



 十一月丁丑,以西南路招討使宗寧等為賀宋正旦使。戊寅,幸東宮。上謂皇太子曰:「吾兒在儲貳之位,朕為汝措天下,當無復有經營之事。汝惟無忘祖宗純厚之風,以勤修道德為孝,明信賞罰為治
 而已。昔唐太宗謂其子高宗曰:『吾伐高麗不克終,汝可繼之。』如此之事,朕不以遺汝。如遼之海濱王,以國人愛其子,嫉而殺之,此何理也。子為眾愛,愈為美事,所為若此,安有不亡。唐太宗有道之君,而謂其子高宗曰:『爾於李績無恩。今以事出之,我死,宜即授以僕射,彼必致死力矣。』君人者,焉用偽為。受恩於父,安有忘報於子者乎?朕御臣下,惟以誠實耳。」群臣皆稱萬歲。丙戌,朝享于太廟。丁亥,有事于圓丘,大赦。癸巳,群臣奉上尊號曰應天興祚欽文廣武仁德聖孝皇帝。乙未,詔中外。



 十二月癸卯,冬獵。乙卯,還宮。丙辰,參知政事敬嗣暉薨。辛酉,進封
 越王永中趙王,隨王永功曹王,沈王永成豳王,徐王永升虞王,滕王永蹈徐王,薛王永濟滕王。乙丑,趙王永中、曹王永功俱授猛安,仍命永功親治事,以習為政。



\end{pinyinscope}