\article{列傳第二十九}

\begin{pinyinscope}

 ○完顏撒改龐迪溫迪罕移室懣神土懣移剌成石抹卞楊仲武蒲察世傑本名阿撒蕭懷忠移剌按答孛術魯阿魯罕趙興祥石抹榮敬嗣暉



 完顏撒改,上京納魯渾河人也,其先居於兀冷窟河。身長多力,善用槍。王師南征,睿宗為右副元帥,置之麾下,佩以金牌,使督軍事。天眷元年,授本班祗候郎君詳穩。
 其後從軍泰州路,軍帥以撒改為萬戶,領銀術可等猛安,戍北邊,數有戰功。天德二年正月,海陵庶人遣使夏國,諭以即位事,因令伺彼之意。既還,稱旨,為尚書兵部郎中。改同知會寧尹,遷迭剌部族節度使,改甌里本群牧使,為曷懶路都總管。海陵伐宋,授衛州防禦使,為武震軍都總管。世宗即位,遣使召撒改,既至,除昌武軍節度使。已而為山東路元帥副都統,改安化軍節度使,兼副都統如故。四年,徙鎮安武,仍兼副統。領山東、大名、東平三路軍八萬餘渡淮,會大軍伐宋。進至楚州,宋遣使奉歲幣。還邳州,卒。



 龐迪,字仲由,延安人。少倜儻,喜讀兵書,習騎射,學推步孤虛之術,無所效用。應募,隸涇原路第三副將,破賊有功,授保義郎。嘗從百餘騎經行山谷,遇夏人數千,眾皆駭懼請避,迪遂躍馬犯陣,敵皆披靡,身被重創,神色自若,完軍以還。自是知名,擢為正將,權發遣涇原路兵馬都監。



 齊國建,涇原路經略使張中孚舉迪權知懷德軍,兼沿邊安撫使。夏人合軍五萬薄懷德城,迪開門待之,夏人不敢入。因以數千騎分門突出,遂破之,斬首五百級,獲軍資羊馬甚眾。復破關師古兵,擢知涇州。未到官,改知鎮戎軍、沿邊安撫使。已而權淮南東路馬步軍副
 總管,總制沂、密、淮陽,兼權知沂州。丁父憂,去官。尋起復為環慶路兵馬都鈐轄,權知邠州。齊國廢,改華州防禦使。頃之,軍變,被執入山。已而賊眾悔曰:「公為政素善,豈宜劫辱。」遂縱之還,復領州事。



 天眷元年,除永興軍路兵馬都總管兼知京兆府,徙臨洮尹,兼熙秦路兵馬都總管。陜右大饑,流亡四集,迪開渠溉田,流民利其食,居民藉其力,各得其所,郡人立碑紀其政績。官制行,吏部以武功大夫、博州團練使特授定遠大將軍。七年,除慶陽尹。歷三考不易,以治最聞,詔書褒美,西人榮之。正隆元年,遷鳳翔尹,屢上章求退,不許。



 海陵南伐,徵斂煩急,官
 吏因緣為姦,富者用賄以免,貧者破產益困。迪悉召民使共議增減,不加威督而役力均,人情大悅。五年,徙汾陽軍節度使。大定初,復為臨洮尹,遷南京路都轉運使,以省事惜費,安靜為政,河南稱之。徙絳陽軍節度使。卒官,年七十。



 迪性純孝,父病,醫藥弗效,迪仰天泣禱,刲股作羹,由是獲安。昆弟析家財,迪盡以與之,一無所取。官爵之蔭,率先諸姪。疾革,沐浴朝服而逝。



 溫迪罕移室懣,速頻屯懣歡春人,徙上京忽論失懶。兄術輦,國初有功,授世襲謀克。移室懣性忠正強毅,善騎射,膂力過人。皇統初,襲其兄謀克,積戰功,為洮州刺史。
 謂人曰:「謀克,兄職也。兄子斡魯古今已長矣。」遂以謀克讓還兄子。宗弼聞而嘉之曰:「能讓世襲,可謂難矣。」除貴德州刺史,改移典颭詳穩,遷烏古里部族節度使,改德昌軍。正隆四年,大徵兵南伐,泰州猛安定遠阿補以所部叛還,移室懣以七謀克執定遠阿補,勒其眾付大軍。契丹反,敗會寧六猛安於締母嶺,屯於信、韓二州之境。移室懣率數千人殺賊萬餘于伊改河,以功遷臨潢尹。世宗即位,賜手詔曰:「南征諸路將士及卿子姪安遠、斡魯古、斜普兄弟,具甲仗悉來推戴,朕勉即大位。卿累世有功耆舊之臣,緣邊事未寧,臨潢劇任,姑仍舊職。聞樞
 密副使白彥敬、南京留守紇石烈志寧來討契丹,今已遣人往招之。其家皆在南京,恐或遁去,兼起異謀,若至則已,若不至,卿當以計執而獻之。兩次遣人招誘招討都監老和尚,去人不知彼之所在,久而不還。兼老和尚不知朕已即位,卿可使人諭以朕意。如來降,悉令復舊,邊關之事,可設耳目。」是時窩斡已反,領兵數萬來攻臨潢,諸路軍未至,窩斡勢益大。移室懣領城中軍士六百人邀擊窩斡,凡數接戰,剿殺甚眾,所乘馬中流矢而仆,為賊所執。賊使移室懣招城中人曰:「爾生死在頃刻,能使城中出降,官爵如故;不然殺汝矣。」移室懣怒罵賊曰:「
 我受國家爵祿,肯從汝叛賊乎?」賊執之至城中,迫脅之使招城中。其妻子官屬將士皆登城臨望。移室懣厲聲曰:「我恨軍少不能滅賊。人生會有一死耳,汝輩慎勿降賊!一旦開門納賊,城中百姓皆被殺掠,毋以我故敗國家事,賊無能為也。」賊怒殺之。城中人皆為之感激,推官麻珪益繕完城郭,右監軍神土懣、輔國上將軍阿思懣乘城固守。賊不克攻,遂引眾東行。


神土懣,本諸宗室,贈銀青光祿大夫胡速魯改子也。年十五,事太宗為左奉宸。皇統二年,充護衛,除武器署丞,累官肇州防禦使。大定初,除元帥右都監,與咸平尹吾
 扎忽率泰州兵及曷懶路兵千五百人,會臨潢尹移室懣討契丹。契丹犯臨潢,移室懣死,攻之不能克,迺引眾東行。神土懣表乞濟師。十二月甲辰,世宗次海濱縣,得奏,上曰:「神土懣、吾扎忽軍不少,可以從長攻襲矣。」會右副元帥謀衍以大軍至,神土懣改曷速館節度使,隸右翼,與紇石烈志寧敗賊於長濼,戰霿
 \gezhu{
  松}
 河,皆有功,改婆速路兵馬都總管,卒。



 移剌成,本名落兀,其先遼橫帳人也。沉勇有謀,通契丹、漢字。天會間,隸撻懶下為行軍猛安,與宋人戰於楚、泗之間,成以所部先登,大破宋軍,功最諸將。劉麟約會天
 長軍議進止。成與夾古查合你俱為撻懶前鋒,得宋生口為鄉導,遂達天長,睿宗嘉之。後從宗弼將兵廢齊國。及再伐宋,攻濠州,每戰輒先登,多所摧破。宗弼再取河南,成及蕭懷忠等八猛安先渡。河南平,第功授宣武將軍,除威州刺史。用廉,擢同知延安尹,再遷昭義軍節度使。正隆南伐,為武毅軍都總管。撒八反,海陵以事誅契丹名將,成以本軍守磁,即遣妻子還汴。海陵用是不疑。時人高其有識。改神武軍都總管,與孛術魯定方為浙東道先鋒,使由淮陰進兵。以所部護糧赴揚州,敵兵乘夜來攻,成整兵奮擊,斬刈甚眾。會海陵庶人死,軍還,復
 鎮昭義。大定二年,以廉在優等,改河中尹。再除臨洮尹,招降喬家等族首領結什角。遷南京留守,召拜樞密副使,封任國公。改北京留守。卒。訃聞,上悼惜之,授其子順思阿不武功將軍,世襲咸平路鈔赤鄰猛安下查不魯謀克。



 結什角者,西番既衰,其苗裔曰堇氈,其子曰巴氈。角始附宋,賜姓趙,改名順忠。順忠子永吉,永吉子世昌,皆受宋官,為左武大夫,遙領萊州防禦使,襲把羊族長。朝廷定陜西,世昌換忠翊校尉。既而鬼蘆族長京臧殺世昌,朝廷遣兵執京臧,斬之臨洮市,以世昌子鐵哥為把羊族都管。大定四年,宋人破洮州,鐵哥弟結什角與
 其母走入喬家族避之。喬家族首領播逋與鄰族木波隴逋、龐拜、丙離四族耆老大僧等立結什角為木波四族長,號曰「王子」。其地北接洮州、積石軍。其南隴逋族,南限大山,八百餘里不通人行。東南與疊州羌接。其西丙離族,西與盧甘羌接。其北龐拜族,與西夏容魯族接。地高寒,無絲枲五穀,惟產青稞,與野菜合酥酪食之。其疆境共八千里,合四萬餘戶。其居隨水草畜牧,遷徙不常。結什角念朝廷為其父報仇,欲棄四族歸朝,四族不許。成至臨洮,使人招結什角,乃率四族來附,進馬百匹,仍請每年貢馬。詔曰:「遠人慕義,朕甚嘉之。其遣能吏往撫
 其眾,厚其賞賜。」



 初,天會中,詔以舊積石地與夏人,夏人謂之祈安城。有莊浪四族,一曰吹折門,二曰密臧門,三曰隴逋門,四曰龐拜門,雖屬夏國,叛服不常。大定六年,夏人破滅吹折、密臧二門,其隴逋、龐拜二門與喬家族相鄰,遂歸結什角。夏國遣使來告莊浪族違命作亂,欲興兵剪除。朝廷不知隴逋、龐拜二門舊屬夏國,報以將檢會其地舊所隸屬,毋擅出兵。



 結什角之母居于莊浪族中。大定九年,結什角往省其母,夏人伺知之,遂出兵圍結什角,招之使降。結什角不從,率所部力戰,潰圍出,夏人斫斷其臂,虜其母去,部兵亦多亡者。結什角尋亦
 死,遺言請命朝廷,復立喬家族首領。陜西奏:「聞知夏國王李仁孝與其臣任得敬中分其國,發兵四萬,役夫三萬,築祈安城,殺喬家等族首領結什角。屢獲宋諜人,言宋欲結夏國謀犯邊境。」詔遣大理卿李昌圖、左司員外郎粘割斡特剌往按之,且止夏人毋築祈安城及處置喬家等族別立首領。夏國報云:「祈安本積石舊城,久廢,邊臣請設戍兵鎮撫莊浪族,所以備盜,非有他也。結什角以兵入境,以是殺之,不知為喬家族首領也。」李昌圖等按視,殺結什角之地本在夏境,築祈安城已畢工,皆罷歸,不得宋、夏交通之狀,乃於熙秦迫近宋、夏衝要量
 添戍兵。及問喬家等族民戶,願以結什角姪趙師古為首領,於是詔以趙師古為木波喬家、丙離、隴逋、龐拜四族都鈐轄,加宣武將軍。



 石抹卞,本名阿魯古列。五代祖王五,遼駙馬都尉。父五斤,為群牧使,從睿宗秋山,卞年十三,已能射,連獲二鹿,睿宗奇之,賜以良馬及金吐鶻。天會末,宗弼為右監軍,召卞隸帳下。丁父憂,是時宗磐為太師,撻懶為左副元帥,人爭附之,使人召卞,卞不往。宗磐、撻懶皆以罪誅,人多其有識。宗弼復取河南,與宋人戰於潁州,漢軍少卻,卞身被七創,率勇士十餘騎奮擊,敗之。及宋稱臣,宗弼
 選嘗有勞者與俱入朝,授卞忠勇校尉。遷宣武將軍,除河間少尹。察廉,升遂州刺史,改壽州,再改唐州。丁母憂去官,起復唐州刺史。海陵伐宋,卞為武毅軍都總管,由別道進兵。遇宋伏兵數百人,以三十騎擊敗之,遂下信陽軍及羅山縣。至蔣州,宋守將棄城遁,因取其城。頃之,軍士皆欲逃歸,闌子山猛安結漢軍三猛安謀克劫卞還,舍於獎水之曲。卞乃陰約漢軍將吏乘夜掩殺闌子山猛安,復將其軍。大定二年,除鄭州防禦使,以本官領行軍萬戶伐宋。遷武勝軍節度使。宋人請和,明年,有水牛數百頭自淮南走入州境,僚佐欲收之充官用,卞
 不聽,復驅過淮還之。遷河南尹,轉西南路招討使,改大名尹。大名多盜,而城郭不完,卞請修大名城。奏可。城完葺,盜賊不得發。徙臨洮尹。卒官,年六十三。



 楊仲武,字德威,保安人。父遇,以勇聞關西,為宥州團練使。宋末,仲武謁經略使王庶求自效,遂用為先鋒。婁室入關,仲武與鄜延路兵馬都監鄭建充俱降,為安塞堡。環慶路兵馬都監。皇統初,復陜西,將兵戍鳳翔,屢卻宋軍。除知寧州。關中薦饑,境內盜賊縱橫,仲武悉平之。改坊州刺史,復知寧州,遷同知臨洮尹,改同知河中府。海陵營繕南京,典浮橋工役。臨洮地接西羌,與木波雜居,
 邊將貪暴,木波苦之,遂相率為寇掠。仲武前治臨洮,乃從數騎入其營諭之曰:「此皆將校侵漁汝等,以至此爾。今懲治此輩,不復擾害汝也。」並以禍福曉之。羌人喜悅,寇掠遂息。至是,木波復掠熙河,熙河主帥使人諭之,不肯去,曰:「楊總管來,我乃解去。」熙河具奏,詔復遣仲武。當是時,木波謂仲武不能復來。及仲武至,與其酋帥相見,責以負約,對曰:「邊將苦我,今之來,求訴於上官耳。今幸見公,願終身不復犯塞。」乃舉酒酹天,折箭為誓。仲武因以卮酒飲之曰:「當更為汝請,若復背約,必用兵矣。」羌人羅拜而去。及伐宋,以仲武為威定軍都總管,駐兵歸德。
 大定三年,除武勝軍節度使,改陜西西路轉運使,卒。



 蒲察世傑,本名阿撒,曷速館斡篤河人,徙遼陽。初在梁王宗弼軍中。為人多力,每與武士角力賭羊,輒勝之。能以拳擊四歲牛,折脅死之。有糧車陷淖中,七牛挽不能出,世傑手挽出之。宗敏為東京留守,召置左右。海陵篡立,即以為護衛。海陵謂世傑曰:「汝勇力絕倫,今我兄弟有異志者,期以十日除之,則有非常之賞,仍盡以各人家產賜汝。」世傑受詔而不肯為。已過十日,海陵怒,面責之。世傑曰:「臣自誓不以非道害物,雖死不敢奉詔。」海陵愛其勇,不之罪也。



 正隆四年,調諸路兵伐宋,年二十
 以上、五十以下皆籍之。他使者唯恐不如詔書,得數多,世傑往喝懶路,得數少。海陵怪問之,對曰:「曷懶地接高麗,今若多籍其丁,即有緩急,何以為備?」海陵喜曰:「他人用心不能及也。」除同知安國軍節度使事,賜銀二百五十兩、絹彩六百匹、馬二疋。是時徵發不已,民不堪命,犯法者眾,邢久無長吏,獄囚積四百餘人。世傑到官月餘,決遣略盡。入為宿直將軍,以事往胡里改路,還奏:「契丹部族大抵皆叛,百姓驚擾不安。今舉國南伐,賊若乘虛入據東土根本之地,雖得江、淮,無益也。宜先討平契丹,南伐未晚。」海陵不悅曰:「詔令已出矣。今以三萬兵選將
 屯中都以北,足以鎮壓。」世傑又曰:「若東土大族附於賊,恐三萬眾未易當也。」海陵不聽。



 及發汴京,授鄭州防禦使,領武捷軍副總管。大軍渡淮,世傑以軍三千護糧餫東下,敗宋兵數千人,奪其戰船甚眾。至和州境,擊宋兵五萬人走之。明日,使其子兀迭領二百八十騎為應兵,自領八百騎前戰。連射六十餘人,皆應弦而斃,宋兵遂奔潰。海陵欲觀水戰,使世傑領水軍百人試之。宋人舟大而多,世傑舟小,乃急進,至中流取勝而還。大定初,世傑復取陜州,敗宋兵石壕鎮,復敗宋援兵三千人,遂圍陜州。宋兵二千自潼關來,世傑以兵二百四十迎擊之,
 射殺十餘人,宋兵敗走。復敗之於土壕山,生擒一將。復以兵三百至斗門城,遇宋兵萬餘,宋將三人挺槍來刺世傑,世傑以刀斷其鎗,宋兵乃退。復以四謀克軍敗宋兵於土華,復圍陜州。世傑嘗擐甲佩刀,腰箭百隻,持槍躍馬,往來軍中。敵人見而異之,曰:「真神將也。」親率選卒二百餘人穴地以入,城遂拔。再破宋軍三萬人,復虢州。



 未幾,為衛州防禦使,改河南路統軍都監。召赴闕,上慰勞良久,除西北路副統,賜廄馬、弓矢、佩刀。從僕散忠義討契丹。賊平,改華州防禦使,與徒單合喜經略隴右。合喜復德順,至東山堡,宋兵捍絕樵路,世傑擊走之,追至
 城下。城中出兵約二萬餘,敗之,殺傷甚眾。宋經略使荊皋棄德順走,世傑與左都監璋追破其軍。改亳州防禦使,四遷通遠軍節度使。宋人輒入鞏州境糶米面,有司執之,世傑署案作歸附人,縱遣之。譯吏蔡松壽誣府主謀叛,坐斬。十八年,起為弘州刺史。母憂去職。累遷亳州防禦使,卒。



 世傑少貧,然疏財尚氣,每臨陣,敵眾既敗,必戒士卒毋縱殺掠。平居非忠孝不言,親賢樂善,甚獲當世之譽云。



 蕭懷忠,本名好胡,奚人也。為西北路招討使。蕭裕等謀立遼後,使蕭招折往西北路結懷忠,并結節度使耶律
 朗為助。懷忠與朗有隙,遂執招折并執朗,遣使上變。裕等既誅,懷忠為樞密副使,賜今名。復為西北路招討使,西京留守,封王。改南京留守。契丹撒八反,復以懷忠為西京留宋、西南面兵馬都統,與樞密使僕散思恭、北京留守蕭賾、右衛將軍蕭禿剌、護衛十人長斡盧保往討之。蕭禿剌戰無功,大軍追撒八不及。而海陵意謂懷忠與蕭裕皆契丹人,本同謀,逾年乃執招折上變,而撒八亦契丹部族,恐其合,以師恭與太后密語,而禿剌無功,懷忠、賾、師恭逸賊,既殺師恭,族滅其家,使使即軍中殺賾、懷忠,皆族之。斡盧保、禿剌初為罪首,但誅之而已。大
 定三年,追復賾、懷忠、禿剌、斡盧保官爵。賾弟安州刺史頤求襲賾之謀克,上不許謀克,而以賾家產付之。



 移剌按答,遼橫帳人也。父留斡,與耶律餘睹俱來降。西京下,復叛,留斡遇害,按答以死事之子,授左奉宸。熙宗初,充護衛,除安州刺史,累官東京副留守。參知政事完顏守道經略北方,攝咸平路屯軍都統。入為兵部侍郎,徙西北、西南兩路舊設堡戍迫近內地者於極邊安置,仍與泰州、臨潢邊堡相接。除武定軍節度使,以招徠邊部功遷東北路招討使,改臨潢尹,卒。



 按答騎射絕倫,善相馬,嘗論及善射者,世宗曰:「能如卿乎?」閱馬于市,見良
 馬,雖羸瘦,輒與善價取之,他日果良馬也。



 孛術魯阿魯罕,隆州琶離葛山人。年八歲,選習契丹字,再選習女直字。既壯,為黃龍府路萬戶令史。貞元二年,試外路胥吏三百人補隨朝,阿魯罕在第一,補宗正府令史。累擢尚書省令史。僕散忠義討窩斡,辟置幕府,掌邊關文字,甚見信任。窩斡既平,阿魯罕招集散亡,復業者數萬人。復從忠義伐宋,屢入奏事,論列可否。上謂宰相曰:「阿魯罕所言,可行者即行之。」宋人請和,忠義使阿魯罕往。和議定,阿魯罕入奏,賜銀百兩、重彩十端。忠義薦阿魯罕有才幹,可任尚書省都事,詔以為大理司直。
 未幾,授尚書省都事,除同知順天軍節度事。紇石烈志寧北巡,阿魯罕攝左右司郎中。還朝,除刑部員外郎,再遷侍御史。上問紇石烈良弼曰:「阿魯罕何如人也?」對曰:「有幹材,持心忠正,出言不阿順。」數日,遷勸農副使,兼同修國史,侍御史如故。改右司郎中。奏請徙河南戍軍屯營城中者於十里外,從之。遷吏部侍郎,附山東統軍都監,徙置河南八猛安。遷武勝軍節度使。入為吏部尚書,改西南路招討使。有司督本路猛安人戶所貸官粟,阿魯罕乞俟豐年,從之。軍人有以甲葉貿易諸物,天德榷場及界外歲采銅礦,或因私挾兵鐵與之市易,皆一切禁
 絕之。上番軍不許用親戚、奴婢及傭雇者,營塹損圮以時葺治,不與所部猛安謀克會宴,故兵民皆畏愛之。



 上謂太尉守道曰:「阿魯罕及上京留守完顏烏里也皆起身胥吏,阿魯罕為人沉厚,其賢過之。」改陜西路統軍使兼京兆尹。陜西軍籍有闕,舊例用子弟補充,而材多不堪用,阿魯罕於阿里喜旗鼓手內選補。軍人以春牧馬,經夏不收飼,瘠弱多死,阿魯罕命以時收秣之,故死損者少。仍春秋督閱軍士騎射,以嚴武備。終南采漆者,節其期限,檢其出入,以防姦細。上謂宰相曰:「阿魯罕所至稱治,陜西政績尤著,用之雖遲,亦可得數年力也。」召為
 參知政事,命條上天德、陜西行事,上稱善。以疾乞致仕,除北京留守,卒。



 贊曰:《記》曰:「君子聽磬聲,則思死封疆之臣。」《傳》曰:「疆埸之事,慎守其一而備其不虞。」故守戍邊圉之臣,不可以不論焉。



 趙興祥,平州盧龍人。六世祖思溫,遼燕京留守,封天水郡王。父瑾,遼靜江軍節度使。興祥以父任閣門祗候,謁告省親于白霫。會遼季土賊據郡作亂,興祥攜母及弟妹奔燕京,不能進,乃自柳城涉砂磧,夜視星斗而行。僅達遼軍,而不知遼主所向,遂還柳城。及婁室獲遼主,興
 祥乃歸國,從宗望伐宋,為六宅使。天眷初,累官同知宣徽院事。母憂去官。熙宗素聞興祥孝行,及英悼太子受冊,以本官起復,護視太子。轉右宣徽使。天德初,改左宣徽使。海陵嘗問興祥,欲使子弟為官,當自言。興祥辭謝。海陵善之,賜以玉帶,詔曰:「汝官雖未至一品,可佩此侍立。」為濟南尹,賜車馬、金幣、金銀器皿,改絳陽軍節度使,召為太子少保,封廣平郡王,改封鉅鹿。正隆初,例奪王爵,遷太子少傅,封申國公,起為定武軍節度使。海陵伐宋,興祥二子從軍。世宗即位,海陵尚在淮南,二子未得還。興祥來見於平州,世宗嘉其誠款,以為秘書監,復為
 左宣徽使。上曰:「尚食庖人猥多,徒費廩祿。朕在籓邸時,家務皆委執事者,自即位以來,事皆留心。俸祿出於百姓,不可妄費,庖人可約量損減。」近臣獻琵琶,世宗卻之,謂興祥曰:「朕憂勞天下,未嘗以聲伎為心,自今勿復有獻,宜悉諭朕意。」有司奏南北邊事未息,恐財用未給,乞罷修神龍殿涼位工役。上即日使興祥傳詔罷之。久之,以其孫珣為閣門祗候。十五年,上幸安州春水,召興祥赴萬春節。上謁於良鄉,賜銀五百兩,感風眩,賜醫藥。未幾,卒官。



 石抹榮,字昌祖。七世祖仕遼,封順國王。遼主奔天德,榮
 父惕益挺身赴之。是時,榮方六歲,母忽土特滿攜之流離道路,宗室谷神得之,納為次室,榮就養於谷神家。惕益既見遼主,委以軍事。軍敗被執,將殺之,金源郡王銀術可曰:「彼忠於所事,殺之何以勸後。」遂釋之。後從伐宋,卒於軍中。榮年長,事秦王宗翰,居幕府。天眷二年,充護衛。熙宗宴飲,命胙王元與榮角力,榮勝之,連仆力士六七人。熙宗親飲之酒,賜以金幣,遷宿直將軍。天德初,除開遠軍節度使。入謝,不覺泣下。海陵問其故。對曰:「老母在谷神家,違去膝下,是以感泣。」乃詔其母與之俱行,仍賜錢萬貫。改天德尹,徙泰寧軍,再除延安、東平尹。海陵
 南征,為神果軍都總管,留駐泗州,以遏逋卒。大定初,還鎮東平,與戶部尚書梁金求按治山東盜賊。二年,以本官充山東東西、大名等路都統。有疾,改太原尹,徙益都尹。丁母憂,起復召為簽書樞密院事,北京、東京留守,陜西路統軍使,南京、西京留守。榮與河南尹婁室、陜州防禦使石抹靳家奴皆坐高賈賣私物、抑賈買民物得罪。靳家奴前為單州刺史,廉察官行郡,乃劫制民使作虛譽,用是得遷同知太原尹,復多取民利。及為陜州,尚書省奏其事,法當解職削階,上以靳家奴鼓虛聲以誑朝廷,不可恕,特詔除名。榮與婁室削兩階解職。久之,榮除臨
 潢尹,改臨洮尹。卒,年六十三。



 敬嗣暉,字唐臣,易州人。登天眷二年進士第,調懷安丞,遷弘政令,補尚書省令史。有才辯,海陵為宰相,愛之,及篡立,擢起居注,歷諫議大夫、吏部侍郎、左宣徽使。貞元三年八月,尚食烹飪失宜,庖官各杖二百,嗣暉與同知宣徽院事烏居仁各杖有差。久之,拜參知政事。正隆六年伐宋,留張浩及嗣暉于南京,治尚書省事。世宗即位,惡嗣暉巧佞,御史大夫完顏元宜劾奏蕭玉、嗣暉、許霖等六人不可用。嗣暉降通議大夫,放歸田里。嗣暉練習朝儀,進止應對閑雅,由是起為丹州刺史,戒諭之曰:「卿
 為正隆執政,阿順取容,朕甚鄙之。今當竭力奉職,以洗前日之咎。茍或不悛,必罰無赦。」未幾,丁母憂,起復為左宣徽使。世宗頗好道術,謂嗣暉曰:「尚食官毋於禁中殺羊豕,朔望上七日有司毋奏刑名。」大定七年,蒲察通除肇州防禦使,上責其飾詐,因顧嗣暉曰:「如卿不可謂無才,但純實不足耳。」久之,有榜匿名書于通衢者,稱海陵舊臣不得用者有怨望心,將圖不軌。上曰:「豈有是哉。」謂嗣暉曰:「正隆時,卿為執政,今指卿以為怨望,朕極知其不然。卿性明達能辨,但頗自炫,釣眾人之譽,所以致此媒蘗,後當改之。」十年,將有事南郊,廷議嗣暉在海陵時
 凡宗廟禘祫輒行太常事,復拜參知政事,詔以執政冠服攝太常。禮成,薨。



 贊曰:趙興祥、石抹榮自拔流離艱阨中,而克有所樹立,固其識之過人,亦其所遭際致然也。迹世宗之卻聲伎、減庖人,仁愛若是,而其下孰不興起哉!



\end{pinyinscope}