\article{本紀第十八}

\begin{pinyinscope}

 哀宗下



 九月戊寅朔,詔減親衛軍。己丑,軍士殺鄭門守者出奔。壬辰,起上黨公張開及臨淄郡王王義深、廣平郡王范成進為元帥。以前御史大夫完顏合周權參知政事。乙未,以榜召民賣放下年軍需錢,上戶田租如之。辛丑,夜大雷,工部尚書蒲乃速震死。



 閏月戊申朔,遣使以鐵券一、虎符六、大信牌十、織金龍文御衣一、越王玉魚帶一、弓
 矢二賜兗王用安,其父母妻皆贈封之。又以世襲宣命十、郡王宣命十、玉免鶻帶十付用安,其同盟可賜者即賜之。辛亥,遣張開、溫撒辛、劉益、高顯率步軍護陳留、通許糧道。罷貧民進獻糧。戊午,招鄉導。己未,有箭射入宮中,書姦臣姓名,兩日而再得之。辛酉,再括京城粟,以御史大夫合周、點檢徒單百家等主之。丙寅,括粟使者兵馬都總領完顏九住以粟有蓬稗,杖殺孝婦于省門。



 十月,以前司農卿李渙飛語,詔左丞李蹊、戶部侍郎楊綎繫獄,將以軍儲失計坐罪。俄蹊、綎並除名,而止籍綎家貲。渙遂權戶部尚書。尋赦殘欠糧,其應以糧事繫者皆
 釋之。詔徵諸道軍,期以十二月一日入援。



 十一月丁未朔,賜貧民粥。平章政事侯摯致仕。左司郎中斜卯愛實以言事忤近侍,送有司,尋釋之。己酉,衛州軍校白晝取豐備倉米。壬子,京城人相食。癸丑,詔曹門、宋門放士民出就食。壬戌,召諸將相入議事。兗王用安率兵至徐州,元帥王德全閉城不納。會劉安國與宿帥眾僧奴引兵入援,至臨渙,用安使人劫殺之,攻徐州久不能下,退保漣水。制使因世英以用安不赴援,還至宿州西,遇大元兵,死之。丙寅,河、解元帥權興寶軍節度使趙偉襲據陜州以叛,殺行省阿不罕奴十剌以下凡二十一人,誣阿
 不罕奴十剌等反狀以聞。上知其冤,不能直其事,就授偉元帥左監軍,兼西安軍節度使,行總帥府事。偉尋亦歸北。



 十二月丙子朔,以事勢危急,遣近侍即白華問計,華對以紀季以酅入齊之義,遂以為右司郎中。甲申,詔議親出。乙酉,再議於大慶殿,上欲以官奴、高顯、劉益為元帥,不果。是日,除拜扈從及守京城官。以右丞相、樞密使兼左副元帥賽不,平章政事、權樞密使兼右副元帥白撒,右副元帥兼樞密副使權參知政事訛出,兵部書權尚書左丞李蹊,元帥左監軍行總帥府事徒單百家等率諸軍扈從。參知政事兼樞密院副使完顏奴
 申,樞密副使兼知開封府權參知政事習捏阿不,裏城四面都總領、戶部尚書完顏珠顆,外城東面元帥把撒合,南面元帥術甲咬住,西面元帥崔立,北面元帥孛術魯買奴等留守。除拜既定,以京城付之。擢魏璠翰林脩撰,如鄧州招武仙入援。丁亥,上御端門,發府庫及兩府器皿宮人衣物賜將士。戊戌,官奴、阿里合謀立荊王不果,朝廷知其謀,置不問。庚子,上發南京,與太后、皇后、諸妃別,太慟。行次公主苑,太后遣中官持米肉遍犒軍士。辛丑,至開陽門外,麾百官退。詔諭戍兵曰:「社稷宗廟在此,汝等壯士也,毋以不預進發之數,便謂無功,若保
 守無虞,將來功賞顧豈在戰士下?」聞者皆灑泣。是日,鞏昌元帥完顏忽斜虎至自金昌,為上言京西三百里之間無井灶,不可往,東行之議遂決,以為尚書右丞從行,遂次陳留。壬寅,次杞縣。癸卯,次黃城。丞相完顏賽不之子按春有罪,伏誅。甲辰,次黃陵堈。乙巳,諸將請幸河朔,從之。



 二年正月丙午朔,濟河,北風大作,後軍不克濟。丁未,大元兵追擊于南岸,元帥完顏豬兒、賀都喜死之,建威都尉完顏兀論出降。己酉,上哭祭戰死士于河北岸,皆贈官,斬兀論出二弟以殉。赦河朔,招集兵糧,議取衛州。元
 帥蒲察官奴將忠孝軍千人,東面元帥高顯、果毅都尉粘哥咬住領軍萬人為前鋒,至蒲城。庚戌,上次漚麻岡,平章政事白撒、元帥和速嘉兀底不繼至。辛亥,白撒引兵攻衛州,不克。乙卯,聞大元兵自河南渡河,至衛之西南,遂退師,丁巳,戰於白公廟,白撒敗績,棄軍東遁。元帥劉益、上黨公張開亦遁,並為民家所殺。益部曲王全降。戊午,上進次蒲城,復還魏樓村。李辛自汴京出奔,伏誅。己未,上以白撒謀,夜棄六軍渡河,與副元帥、合里合六七人走歸德。庚申,諸軍始知上已往,遂潰。辛酉,司農大卿蒲察世達、元帥完顏忽土出歸德西門,奉迎上入歸
 德。赦在府囚。軍民普覃一官,賜進士終場王輔以下十六人出身。遣奉御術甲塔失不、后弟徒單四喜往汴京奉迎兩宮。白撒還自蒲城,聚兵於大橋,不敢入。壬戌,遣使召白撒至,數其罪,下之獄,仍籍其家財賜將士,曰:「汝輩宜竭忠力,毋如斯人誤國。」人予金一兩。七日,白撒及其子忽土鄰皆死獄中。右丞相賽不致仕。右丞完顏忽斜虎行省事於徐州。官奴再請率兵北渡,女魯懽不可。遣歸德知府行戶部尚書蒲察世達、都轉運使張俊民如陳、蔡取糧,以元帥李琦、王璧護之。戊辰,安平都尉、京城西面元帥崔立與其黨韓鐸、藥安國等舉兵為亂,
 殺參知政事完顏奴申、樞密副使完顏斜捻阿不,勒兵入見太后,傳令召衛王子從恪為梁王,監國。即自為太師、軍馬都元帥、尚書令,尋自稱左丞相、都元帥、尚書令、鄭王。弟倚平章政事,侃殿前都點檢,其黨孛術魯哥御史中丞,韓鐸副元帥兼知開封府,折希顏、藥安國、張軍奴、完顏合答並元帥,師肅左右司郎中,賈良兵部郎中兼右司都事,又署工部尚書溫迪罕二十、吏部侍郎劉仲周並為參知政事,宣徽使奧屯舜卿為尚書左丞,戶部侍郎張正倫為尚書右丞,左右司都事張節為左右司郎中,尚書省掾元好問為左右司員外郎,都轉運
 知事王天祺、懷州同知康瑭並為左右司都事。開封判官李禹翼棄官去。戶部主事鄭著召不起。是日,右副點檢溫敦阿里,左右懷員外郎聶天驥,御史大夫裴滿阿虎帶,諫議大夫、左右司郎中烏古孫奴申,左副點檢完顏阿散,奉御忙哥,講議蒲察琦並死之。遂送款大元軍前。癸酉,大元將碎不泬進兵汴京。甲戌,立閱隨駕官屬軍民子女於省署,及禁民間嫁娶,括京城財。兩宮值變不果行,答失不以其父咬住、四喜以其妻奪門而出,庚午至歸德。上怒二人,皆斬於市。乙亥,遣右宣徽提點近侍局事移剌粘古如徐州,相地形,察倉庫虛實。白華如
 鄧州召兵。



 二月丙子朔。魚山張獻殺元帥完顏忽土,行省忽斜虎自率兵討之,會從宜嚴祿誅獻,乃還。括城中糧。知歸德府事石盞女魯懽為樞密副使、權參知政事。留元帥官奴忠孝軍四百五十人,都尉馬用軍二百八十餘人,發餘軍赴宿、徐、陳三州就糧。



 三月乙丑,石盞女魯懽乞盡散衛兵出城就食。官奴私與國用安謀,邀上幸海州,不從。蔡帥烏石論鎬以糧四百餘斛至歸德,表請臨幸,上遣學士烏石論蒲鮮以幸蔡之意諭其州人。戊辰,官奴以忠孝軍為亂,攻殺馬用,遂殺尚書左丞李蹊、參知政事石盞女魯懽、點檢徒單長樂,從官右丞已
 下三百餘人。上赦官奴,暴女魯懽罪狀,以官奴為樞密副使、權參知政事,左右司郎中張天綱為戶部侍郎、權參知政事。辛卯,官奴真授參知政事,兼左副元帥。官奴以上居照碧堂,禁近諸臣無一人敢奏對者。上日悲泣言曰:「自古無不亡之國、不死之主,但恨朕不知用人,致為此奴所囚耳。」遂與內局令宋珪等謀誅官奴。夏四月壬午,徐州行省完顏忽斜貢執王德全并其子誅之,及其黨王琳、楊璝、斜卯延壽。召經歷商瑀用之。魚山從宜嚴祿叛歸漣水。庚寅,陳州都尉李順兒殺行省粘葛奴申及招撫使劉天起,送款於崔立。張俊民、李琦奔汴京。王
 壁還歸德。癸巳,崔立以梁王從恪、荊王守純及及諸宗室男女五百餘人至青城,皆及於難。甲午,兩宮北遷。甲辰,鄧州節度使移剌瑗以其城叛,與白華俱亡入宋。



 六月己卯,官奴及其黨阿里合、白進皆伏誅。上御雙門,赦忠孝軍,以安反側。遂決策遷蔡,詔蔡、息、陳、潁各以兵來迓。中京留守、權參政烏林答胡土棄城奔蔡。壬午,中京破,留守兼便宜總帥強伸死之。戊子,召徐州行省完顏忽斜虎赴行在所,以抹撚兀典代行省事,郭恩為總帥兼節度使。辛卯,上發歸德,留元帥王璧守之。壬辰,次亳州。癸巳,以亳州節度使王進、同知節度使王賓徵民丁運
 鐵甲糗糧,留權參政張天綱董之,就遷有功將士。臨淄郡王王義深據靈璧望口寨以叛,遣近侍直長女奚烈完出將徐、宿兵討之,義深敗走漣水,入宋。丙申,亳州鎮防軍崔復哥殺守臣王賓等,張天綱以便宜授復哥節度使,罷運鐵甲糗糧,州人乃安。己亥,上入蔡州,詔尚書省為書召武仙會兵入援。徐州行省抹撚兀典赴蔡州。起復右丞相致仕賽不代行省事。



 七月癸卯朔,曲赦蔡州管內雜犯死罪以下。官吏軍民普覃兩官,經應辦者更遷一官。弛門禁,通眾貨,蔡人便之。乙巳,以烏古論鎬為御史大夫,總帥如故,張天綱為御史中丞,仍權參政,
 完顏藥師為鎮南軍節度使,兼蔡州管內觀察使。戊申,左右司郎中烏古論蒲鮮兼息州刺史,權元帥右都監,行帥府事。征行元帥權總帥婁室簽樞密院事。己酉,選室女備宮中使令,已得數人,以右丞忽斜虎諫,留識文義者一人,余聽自便。乙卯,遣魏璠征武仙兵。丁巳,護衛蒲鮮石魯負祖宗御容至自汴,敕有司奉安于乾元寺。前御史中丞蒲察世達、西面元帥把撒合自汴來歸。辛酉,武仙劫將士,謀取宋金州,至淅水眾潰。行六部尚書盧芝、侍郎石玠謀歸蔡州,仙追芝不及,遂殺玠。丁卯,定進馬遷賞格,又定括馬罪格,以簽樞密院事權參政事
 抹撚兀典領其事。遣使分詣諸道,選兵會於蔡。己巳,以蒲察世達為吏部侍郎,權行六部尚書。



 八月癸酉朔,以秦州元帥粘哥完展權參知政事,行省事於陜西。諭以蠟書,期九月中徵兵上會於饒豐關,欲出宋不意,以取興元。甲戌,大元使王楫諭宋還,宋以軍護其行,青山招撫盧進得邏吏言以聞,上為之懼。丁丑,上閱兵于見山亭。癸未,元帥楚弁復立壽州於蒙城,詔遷賞有差,州縣官皆令真授。乙酉,大元召宋兵攻唐州,元帥右監軍烏古論黑漢死於戰,主帥蒲察某為部曲兵所食。城破,宋人求食人者盡戮之,餘無所犯。宋人駐兵息州南。丙戌,詔
 權參政抹撚兀典、簽樞密院事婁室行省、院於息州。丁亥,烏古論鎬權參知政事,兀林答胡土為殿前都點檢。庚寅,初設四隅機察官。壬辰,息州行省抹撚兀典以兵襲宋人中渡店,斬獲甚眾。乙未,萬年節,州郡以表來賀二十餘所。辛丑,設四隅和糴官及惠民司,以太醫數人更直,病人官給以藥,仍擇年老進士二人為醫藥官。是月,假蔡州都軍致仕內族阿虎帶同僉大睦親府事,使宋借糧,入辭,上諭之曰:「宋人負朕深矣。朕自即位以來,戒飭邊將無犯南界。邊臣有自請征討者,未嘗不切責之。向得宋一州,隨即付與。近淮陰來歸,彼多
 以金幣為贖,朕若受財,是貨之也,付之全城,秋毫無犯。清口臨陣生獲數千人,悉以資糧遣之。今乘我疲敝,據我壽州,誘我鄧州,又攻我唐州,彼為謀亦淺矣。大元滅國四十,以及西夏,夏亡必及於我。我亡必乃於宋。脣亡齒寒,自然之理。若與我連和,所以為我者亦為彼也。卿其以此曉之。」至宋,宋不許。



 九月戊申,魯山元帥元志率兵入援,賜以大信牌,升為總帥。庚戌,以重九拜天於節度使,群臣陪從成禮,上面諭之曰:「國家自開創涵養汝等百有餘年。汝等或以先世立功,或以勞效起身,被堅執銳,積有年矣。今當厄運,與朕同患,可謂忠矣。比聞北兵將
 至,正汝等立功報國之秋,縱死王事,不失為忠孝之鬼。往者汝等立功,常慮不為朝廷所知,今日臨敵,朕親見之矣,汝等勉之。」因賜卮酒。酒未竟,邏騎馳奏,敵兵數百突至城下。將士踴躍咸請一戰,上許之。是日,分軍防守四面及子城,以總帥孛術魯婁室守東面,內族承麟副之;參知政事烏古論鎬守南面,總帥元志副之;殿前都點檢兀林答胡土守西面,忠孝軍元帥蔡八兒副之;忠孝軍元帥、權殿前右副點檢王山兒守北面,元帥紇石烈柏壽副之;遙授西安軍節度使兼殿前右衛將軍、行元帥府事女奚烈出守東南,元帥左都監夾谷當哥
 副之;殿前右衛將軍、權左副都點檢內族斜烈守子城,都尉王愛實副之。辛亥,大元兵築長壘圍蔡城。己未,括蔡城粟。辛酉,禁公私釀酒。



 十月戊寅,更造「天興寶會」。辛巳,縱飢民老稚羸疾者出城。癸未,徐州守臣郭恩殺逐官吏以叛,行省賽不死之。甲申,給飢民船,聽採城壕菱芡水草以食。戊子,徵諸道兵。辛卯,上閱射于子城,中者賞麥有差。丙申,殿前左副都點檢溫敦昌孫戰歿。戊戌,賜義軍戰歿被創者麥。



 十一月辛丑朔,以右副都點檢阿勒根移失剌為宣差鎮撫都彈壓,別設彈壓四員副之,四隅機察亦隸焉。宋遣其將江海、孟珙帥兵萬人,獻糧
 三十萬石助大元兵攻蔡。



 十二月甲戌,盡籍民丁防守,括婦人壯捷者假男子衣冠,運大石。上親出撫軍。丁丑,大元兵決練江,宋兵決柴潭入汝水。己卯,大元兵破外城,宿州副總帥高剌哥戰歿。辛巳,以總帥孛術魯婁室、殿前都點檢兀林答胡土皆權參政,都尉完顏承麟為東面元帥,權總帥。己丑,大元兵墮西城,上謂侍臣曰:「我為金紫十年,太子十年,人主十年,自知無大過惡,死無恨矣。所恨者祖宗傳祚百年,至我而絕,與自古荒淫暴亂之君等為亡國,獨此為介介耳。」又曰:「古無不亡之國,亡國之君往往為人囚縶,或為俘獻,或辱於階庭,閉之
 空谷。朕必不至於此。卿等觀之,朕志決矣。」都尉王愛實戰歿。砲軍總帥王銳殺元帥谷當哥,率三十人降大元。庚寅,以御用器皿賞戰士。甲午,上微服率兵夜出東城謀遁,及柵不果,戰而還。乙未,殺尚廄馬五十疋、官馬一百五十疋犒將士。



 三年正月壬寅,冊柴潭神為護國靈應王。甲辰,以近侍分守四城。戊申,夜,上集百官,傳位于東面元帥承麟,承麟固讓。詔曰:「朕所以付卿者,豈得已哉?以肌體肥重,不便鞍馬馳突。卿平日捷有將略,萬一得免,祚胤不絕,此朕志也。」己酉,承麟即皇帝位。百官稱賀。禮畢,亟出捍
 敵,而南面已立幟。俄頃,四面呼聲震天地。南面守者棄門,大軍入,與城中軍巷戰,城中軍不能禦。帝自縊於幽蘭軒。末帝退保子城,聞帝崩,率群臣入哭,謚曰哀宗。哭奠未畢,城潰,諸禁近舉火焚之。奉御絳山收哀宗骨瘞之汝水上。末帝為亂兵所害,金亡。



 贊曰:金之初興,天下莫彊焉。太祖、太宗威制中國,大概欲效遼初故事,立楚立齊,委而去之,宋人不競,遂失故物。熙宗、海陵濟以虐政,中原觖望,金事幾去。天厭南北之兵,挺生世宗,以仁易暴,休息斯民。是故金祚百有餘年,由大定之政有以固結人心,乃克爾也。章宗志存潤
 色,而秕政日多,誅求無藝,民力浸竭,明昌、承安盛極衰始。至於衛紹,紀綱大壞,亡徵已見。宣宗南度,棄厥本根,外狃餘威,連兵宋、夏,內致困憊,自速土崩。哀宗之世無足為者。皇元功德日盛,天人屬心,日出爝息,理勢必然。區區生聚,圖存於亡,力盡乃斃,可哀也矣。雖然,在《禮》「國君死社稷」,哀宗無愧焉。



\end{pinyinscope}