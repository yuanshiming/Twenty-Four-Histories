\article{本紀第十二}

\begin{pinyinscope}

 章宗四



 四年春正月乙丑朔,宋、高麗、夏遣使來賀。丁卯,諭外方使人不得佩刀入宮。庚午,幸豫王永成第視疾。辛未,如光春宮春水。壬申,陰霧,木冰。丁丑,行尚書省奏,宋賀正使還至慶都卒。詔遣防禦使女奚烈元往祭,致賻絹布各二百二十匹,仍命送伴使張雲護喪以歸。豫王永成薨。辛卯,高麗國王王晫沒,嗣子韺遣使來告哀。



 二月乙未
 朔,還宮。丁酉,以山東、河北旱,詔祈雨東、北二嶽。己亥,命購豫王永成遺文。庚戌,始祭三皇、五帝、四王。癸丑,詔刺史,州郡無宣聖廟學者並增修之。



 三月丁卯,日昏無光,大風毀宣陽門鴟尾。癸酉,命大興府祈雨。戊寅,幸太極宮。詔定前代帝王合致祭者。尚書省奏:「三皇、五帝、四王,已行三年一祭之禮。若夏太康,殷太甲、太戊、武丁,周成王、康王、宣王,漢高祖、文、景、武、宣、光武、明帝、章帝,唐高祖、文皇一十七君致祭為宜。」從之。乙酉,祈雨于北郊。丁亥,如萬寧宮。壬辰,祈雨于社稷。遼陽府判官民斜卯劉家以上書論列朝臣,削官一階,罷之。夏四月丙申,詔定縣令
 以下考課法。已亥,祈雨于太廟。庚子,增定關防姦細格。丙午,定衣服制。以祈雨,望祀嶽鎮海瀆于北郊。癸丑,祈雨于社稷。甲寅,以久旱,下詔責躬,求直言,避正殿,減膳撤樂,省御廄馬,免旱菑州縣徭役及今年夏稅。遣使審繫囚,理冤獄。乙卯,宰臣上表待罪。詔答曰:「朕德有愆,上天示異。卿等各趨乃職,思副朕懷。」戊午,以西上閣門使張偁等為故高麗國王王晫敕祭使,東上閣門使石愨等為高麗國王王韺慰問起復橫賜使。庚申,祈雨于太廟。壬戌,萬寧宮端門災。



 五月乙丑,祈雨于北郊。有司請雩,詔三禱嶽瀆社稷宗廟,不雨,乃行之。癸酉,平章政事
 徒單鎰、尚書左丞完顏匡罷。甲戌,雨。乙亥,百官上表請御正殿,復常儀。乙酉,謝雨于宗廟。丁亥,報祀社稷。汰隨朝冗官。定省令史關決公務,詭稱已稟,擅退六部、大理寺法狀及妄有所更易者罪。辛卯,報謝嶽鎮海瀆。



 六月壬辰朔,罷兼官俸給。壬寅,復行吏目移轉法。乙巳,始祭中溜。戊申,罷惠、川、高三州,秀巖、灤陽、徽川、咸寧、金安、利民六縣,及北京宮苑使。諸群牧提舉,居庸、紫荊、通會三關使,西北路鎮防十三千戶,諸路醫學博士。壬子,司天臺長行張冀進《天象傳》。秋七月丁卯,定申報盜賊制。戊辰,朝獻于衍慶宮。庚午,幸望京甸。壬申,如萬寧宮。甲戌,
 罷限錢法。甲申,改葬鎬王永中於威州。



 八月,大理丞姬端修、司直溫敦按帶論奏知大興府事紇石烈執中,坐所言不當,各削一官,罷職。丁酉,以尚書右丞相宗浩為左丞相,右丞僕散揆為平章政事,參知政事孫即康為尚書右丞,御史大夫僕散端為左丞,吏部尚書獨吉思忠為參知政事。庚子,詔完顏綱、喬宇、宋元吉等編類陳言文字,其言涉宮庭,若大臣、省臺、六部,各以類從,凡二千卷。辛丑,以西京留守崇肅為御史大夫。癸卯,更定閣門祗候出職格。先是以天旱詔求直言。至是尚書省奏:「河南府盧顯達、汝州王大材所陳,言涉不遜,請以情理
 切害論其罪。」從之,仍遍諭中外。命諸路學校生徒少者罷教授,止以本州、府文資官提控之。丁未,以安州軍事判官劉常言,諸按察司體訪不實,輒加糾劾者,從故出入人罪論,仍勒停。若事涉私曲,各從本法。辛亥,還宮。乙卯,以知真定府事完顏昌等為賀宋生日使。丁巳,幸太極宮。弛圍場遠地禁,縱民耕捕樵採。減教坊長行五十人,渤海教坊長行三十人,文繡署女工五十人。出宮女百六十人。



 九月庚申朔,天壽節,宋、高麗、夏遣使來賀。丙寅,如薊州秋山。壬申,定屯田戶自種及租佃法。冬十月甲午,定私鹼法。丙申,詔親軍三十五以下令習《孝經》、《論
 語》。癸卯,至自秋山。甲寅,以提點尚衣局完顏燮為夏國生日使。



 十一月丁卯,以殿前右副都點檢烏林答毅等為賀宋正旦使。癸酉,木冰,凡三日。丁丑,定收補承應人格。



 十二月己丑朔,新平等縣虸蚄蟲生。己亥,左丞相宗浩等請上尊號。不許。辛丑,敕陜西、河南饑民所鬻男女,官為贖之。乙卯,百官再表乞受尊號。不許。



 五年春正月己未朔,大雪。宋、高麗、夏遣使來賀。庚申,謁衍慶宮。乙丑,幸太極宮。丁卯,如光春宮春水。壬申,朝獻于衍慶宮。乙亥,詔有司:「自泰和三年郡縣三經行幸、民嘗供億者,賜今年租稅之半。」丁丑,次霸州。調山東、河北
 軍夫改治漕渠。



 二月己丑朔,諭按察司:「近制以鎮靜而知大體為稱職,苛細而暗於大體為不稱。由是各路按察以因循為事,莫思舉刺,郡縣以貪黜相尚,莫能畏戢。自今若糾察得實,民無冤滯,能使一路鎮靜者為稱職。其或煩紊使民不得伸愬者,是為曠廢。」癸巳,定鞫勘官受飲宴者罪。己亥,如建春宮。甲寅,制盜用及偽造都門契者罪,視宮城門減一等。



 三月庚申,還宮。癸亥,更定兩稅輸限。乙丑,宋兵入秦川界。庚午,親王、百官請上尊號,不許。甲戌,諭有司,進士名有犯孔子諱者避之,仍著為令。命給米諸寺,自



 十月十五日至次年正月十五日作
 糜以食貧民。戊寅,罷獄空錢。辛巳,宋失入鞏州來遠鎮。唐州得宋諜者,言韓侂胄屯兵鄂、岳,將謀北侵。



 四月戊子朔,如萬寧宮。癸巳,命樞密院移文宋人,依誓約撤新兵,毋縱入境。壬子,定隨路轉運司及府官每季檢視庫物法。



 五月甲子,以平章政事僕散揆為河南宣撫使,籍諸道兵以備宋。癸酉,詔定遼東邑社人數。戊寅,更定檢、知法勒留格。己卯,如慶寧宮。制司屬丞凡遭父母喪止給卒哭假,為永制。甲申,宋人入漣水縣。



 六月戊子,復漣水縣。丁酉,制定本朝婚禮。更定鬻米面入外界法。己酉,制鎮防軍逃亡致邊事失錯、陷敗戶口者罪。甲寅,詔拜
 禮不依本朝者罰。召諸大臣問備宋之策,皆以設備養惡為言。上以南北和好四十餘載,民不知兵,不忍先發。



 七月戊辰,如錦屏山。壬申,朝獻于衍慶宮。乙亥,宣撫使揆奏定姦細罪賞法。丙子,定圍場誤射中人罪。壬午,招諸縣盜賊多所選注巡尉。



 八月辛卯,詔罷宣撫司。時宋殿帥敦倪、濠州守將田俊邁誘虹縣民蘇貴等為間,河南將臣亦屢縱諜,往往利俊邁之賂,反為遊說。皆言宋之增戍,本虞他盜,及聞行臺之建,益畏懾不敢去備。且兵皆白丁,自裹糧Я,窮蹙饑疫,死者十二三,由是中外信之。宣撫司以宋三省、樞密院及盱眙軍牒來上,又皆
 鐫點邊臣為辭。宣撫使揆因請罷司,從之。揆又奏罷臨洮、德順、秦、鞏新置弓箭手。



 閏月乙卯朔,罷典衛司。丙子,還宮。



 九月甲申朔,天壽節,宋、高麗、夏遣使來賀。戊子,西北方黑雲間有赤氣如火色,次及西南、正南、東南路方皆赤,有白氣貫其中,至中夜,赤氣滿天,四更乃散。以河南路統軍使紇石烈子仁等為賀宋生日使。戊戌,宋兵三百攻比陽寺莊,副巡檢阿里根寺家奴死之。甲辰,宋人焚黃澗,虜巡檢高顥。冬十月庚申,以刑部員外郎李元忠為高麗生日使。丁丑,宋人襲比陽。唐州軍事判官撒睹死之。



 十一月乙酉,宋人入內鄉,攻洛南之固縣,商州司
 獄壽祖追至丹河,擊敗之。己丑,以太常卿趙之傑等為賀宋正旦使。癸巳,山東闕食,賜錢三萬貫以賑之。乙未,初定武舉格。丁酉,詔山東、陜西帥臣訓練士卒,以備非常。仍以銀十五萬兩分給邊帥,募民偵伺。復遣武衛軍副都指揮使完顏太平、殿前右衛副將軍蒲察阿里赴邊,伺其入,伏兵掩之。戊戌,大雪,免朝參。己亥,更定宮中局、署承應收補格。宋吳曦擁眾興元,欲窺關、隴,皇甫斌益募兵擾淮北,所掠即以與之,使自為戰。



 六年春正月癸未朔,宋、高麗、夏遣使來賀。丁亥,宋使陳克俊等朝辭。遣御史大夫孟鑄就館諭克俊等曰:「大定
 初,世宗皇帝許宋世為姪國,朕遵守遺法,和好至今。豈意爾國屢有盜賊犯我邊境,以此遣大臣宣撫河南軍民。及得爾國有司公移,稱已罷黜邊臣,抽去兵卒,朕方以天下為度,不介小嫌,遂罷宣撫司。未幾,盜賊甚于前日,比來群臣屢以爾國渝盟為言,朕惟和好歲久,委曲涵容。恐姪宋皇帝或未詳知。若依前不息,臣下或復有云,朕雖兼愛生靈,事亦豈能終已。卿等歸國,當以朕意具言之汝主。」辛卯,朝享于衍慶宮。丙申,宋興元守將吳曦遣兵圍抹熟龍堡,部將蒲鮮長安擊走之,斬其將。辛丑,更定保伍法。癸卯,始以沿河縣官兼管勾漕河事,州、
 府官兼提控。丁未,如春水。庚戌,宋人入撒牟谷。陜西統軍判官完顏摑剌、鞏州兵馬鈐轄完顏七斤約宋西和州守將會境上。俄伏發,為所襲,木波部長趙彥雄等七人死焉。摑剌馬陷淖中,中流矢,七斤僅以身免。



 二月甲戌,御史中丞孟鑄言:「提刑改為按察司,又差官覆察,權削而望輕,非便。」參知政事賈鉉曰:「按察司既差監察體訪,復遣官覆察之,誠為繁冗。請自今差監察時即遣官與俱,更不覆察。」從之。



 三月甲午,尚書省奏,商州刺史烏古論袞州請賻押軍官與南兵戰沒者,又奏遷右振肅蒲察五斤官,皆從之。明昌初,五斤嘗為奉御,出使山
 東,至河間,以百姓饑,輒移提刑司開倉賑之,還具以聞。上初甚悅。太傅徒單克寧言:「陛下始親大政,不宜假近侍人權,乞正專擅之罪。」詔杖之二十。克寧又以為言,乃罷之。後上思之,由泰州都軍召為振肅。己亥,如萬寧宮。甲辰,敕尚書省:「祖父母、父母無人侍養,而子孫遠遊至經歲者,甚傷風化,雖舊有徒二年之罪,似涉太輕。其考前律,再議以聞。」己酉,宋人攻靈璧,南京按察使行部至縣,匿民舍得免。



 四月丙辰,宋人圍壽春。壽春告急于亳,同知防禦使賢聖奴將步騎六百赴之,乃退。癸亥,尚書省奏:「河南統軍司言,統軍使紇石烈子仁等遣嚴整、閻
 忠、周秀輩入襄陽。覘敵陰事。還言皇甫斌遣兵四萬規取鄧,以我叛人田元為鄉導,三萬人規取唐,以張真、張勝為鄉導,俱授統領官,故不敢無備。乃聚鄭、汝、陽翟之兵於昌武,以南京副留守兼兵馬副都總管紇石烈毅統之,聚亳、陳、襄邑之兵于歸德,以河南路副統軍徒單鐸統之,而自以所部兵駐汴。及擬山東東、西路軍七千付統軍紇石烈執中駐大名,河北東、西路軍萬七千屯河南,皆給以馬,有老弱者易其人。」皆從之。甲子,宋人攻天水界,乙丑,入東柯谷,部將劉鐸戰敗之。丙寅,詔平章政事僕散揆領行省于汴,許以便宜從事。升諸道統軍
 司為兵馬都統府,以山東東、西路統軍使紇石烈執中為山東西路兵馬都統使,定海軍節度使、副都統軍使完顏撒剌副之,陜西統軍使充為陜西五路兵馬都統使,通遠軍節度使胡沙、知臨洮府事石抹仲溫副之。河南皆聽揆節制如故。盡征諸道籍兵。辛未,宋吳曦攻來遠鎮之蘭家嶺。丙子,招內外職官納馬各有數。丁丑,宋人入新息、內鄉,又入泗州。戊寅,入褒信。己卯,入虹縣。庚辰,入潁上。



 五月壬午,宋李爽圍壽州,田俊邁入蘄縣,秦詵攻蔡州。防禦使完顏佛住敗之。又入金城海口,殺長山尉,執二巡檢以去。甲申,太白晝見。丙戌,以宋叛盟出
 師,告于天地太廟社稷。丁亥,親告于衍慶宮。戊子,平章政事僕散揆兼左副元帥,陜西兵馬都統使充為元帥右監軍,知真定府事烏古論誼為元帥左都監。辛卯,以征南詔中外。賜唐州刺史吾古孫兀屯、總押鄧州軍馬事完顏江山爵各二級,蔡州防禦使完顏佛住爵一級,餘賞賚有差。又以非嚴整上變,必為所誤,授整嵩州巡檢使,賜爵八級,錢二百萬。上以宋兵方熾,東北新調之兵未集,河南之眾不足支,命河北、大名、北京、天山之兵萬五千屯真定、河間、清、獻等以為應。壬辰,諭尚書省:「今國家多故,凡言軍國利害,五品以上官以次奏陳,朕將
 親問之。六品以下則具帖子以進。」癸巳,山東路災,赦死罪已下。以樞密副使完顏匡為右副元帥。宋田俊邁攻宿州,安國軍節度副使納蘭邦烈等出兵擊之。邦烈中流矢,宋郭悼、李汝翼參眾繼至,遂圍宿州。壬寅,納蘭邦烈等擊敗之,俊邁退保於蘄。癸卯,執俊邁於蘄。甲辰,皇甫斌攻唐州,刺史吾古孫兀屯拒之。行省遣泌陽副巡檢納合軍勝來援,遂擊敗之。庚戌,太白經天。



 六月辛亥朔,左丞僕散端以母憂罷。平章政事揆報蘄之捷,並送所獲宋將田俊邁至闕。上降詔褒諭,賜紇石烈貞、納蘭邦烈、史扢搭等爵賞有差。宋將李爽以兵圍壽州,刺史
 徒單羲拒守,踰月不能下。壬子,河南統軍判官乞住及買哥等以兵來援,羲出兵應之,爽大敗,同知軍州事蒲烈古中流矢死。乙卯,初置急遞鋪,腰鈴轉遞,日行三百里,非軍期、河防不許起馬。定軍前差發受贓罪。除飛蝗入境雖不損苗稼亦坐罪法。丁巳,詔彰德府,宋韓侂胄祖琦墳毋得損壞,仍禁樵採。庚申,右翼都統完顏賽不敗宋曹統制于溱水。辛酉,詔有司,有宋宗族所居,各具以聞。長官常加提控。壬戌,平章政事揆報壽州之捷。戊辰,詔升壽州為防禦,免今年租稅諸科名錢,釋死罪以下。以徒單羲為防禦使。贈蒲烈古昭勇大將軍,賜錢三
 百貫,官其子圖剌。擢乞住同知昌武軍節度使事,買哥河南路統軍判官。都統賽不、副統蒲鮮萬奴各進爵一級,賜金幣有差。辛未,木星晝見,至七月戊申,經天。乙亥,宋吳曦攻鹽川,戍將完顏王喜敗之。秋七月癸未,宋商榮復攻東海,縣令完顏卞僧復敗之。還,中伏矢死,贈海州刺史,以銀五百兩,絹百匹給其家,仍官其一子。甲申,朝獻于衍慶宮。丁亥,敕翰林直學士陳大任妨本職專修《遼史》。甲午,宋統制戚春以舟師攻邳州,刺史完顏從正敗之,春赴水死,斬其副夏統制。吳曦兵五萬入秦州,陜西路都統副使承裕等敗之。丙申,夏國王李純佑廢,
 侄安全立,遣使奉表來告。詔禁賣馬入外境,但至界欲賣而為所捕即論死。



 八月庚戌,山東帥來報邳州之捷。辛亥,木星晨見。乙卯,以羌酋青宜可為壘州副都總管。己未,太白晝見。丙寅,左丞僕散端起復前職。詔設平南諸將軍。辛未,宋程松襲取方山原,蒲察貞破走之。壬申,太白晝見,經天。甲戌,至自萬寧宮。乙亥,赦唐、鄧、潁、蔡、宿、泗六州,免來年租稅三分之一。



 九月己卯朔,天壽節,高麗遣使來賀。辛巳,元帥右都監蒲察貞取和尚原,臨洮蕃部遵寧獻芻粟、戰馬以助軍。乙酉,將五鼓,北方有赤白氣數道,起于王良之下,行至北斗開陽、搖光之東。丙
 戌,幸香山。庚寅,敕行尚書省,有方略出眾、武藝絕倫、才幹辦事、工巧過人者,其招選之。甲午,參知政事賈鉉乞致政,不許。戊戌,尚書左丞僕散端行省于汴。己亥,尚書戶部侍郎梁鏜行六部尚書事於山東。辛丑,遣尚書左司郎中溫迪罕思敬冊李安全為夏國王。甲辰,宋吳曦將馮興、楊雄、李珪等入秦州,陜西都統副使承裕等擊破之,斬楊雄、李珪。冬十月戊申朔,平章政事僕散揆督諸道兵伐宋。庚戌,揆以行省兵三萬出潁、壽,河南路統軍使紇石烈子仁以兵三萬出渦口,元帥匡以兵二萬五千出唐、鄧,左監軍紇石烈執中以山東兵二萬出清
 口,右監軍充以關中兵一萬出陳倉,右都監蒲察貞以岐、隴兵一萬出成紀,蜀漢路安撫使完顏綱以漢、蕃步騎一萬出臨潭,臨洮路兵馬都總管石抹仲溫以隴右步騎五千出鹽川,隴州防禦使完顏璘以本部兵五千出來遠。甲子,獵于近郊。



 十一月戊寅朔,詔定諸州府物力差役式。壬午,完顏匡攻下棘陽。乙酉,詔屯田軍戶與所居民為婚姻者聽。丁亥,僕散揆克安豐軍,取霍丘縣。紇石烈執中克淮陰,遂圍楚州。己丑,尚書省奏,減朝官及承應人月俸折支錢。庚寅,完顏匡克光化軍及神馬坡。壬辰,僕散揆次盧江。宋督視江淮兵馬事丘灊遣劉
 祐來乞和。紇石烈子仁克定遠縣。乙未,完顏匡取隨州。丙申,紇石烈子仁克滁州。戊戌,詔諸路行用小鈔。完顏匡圍德安,別以兵徇下安陸、應城、雲夢、孝感、漢川、荊山等縣。庚子,日斜,有流星二,光芒如炬,幾及一丈,起東北沒東南。初定茶禁。完顏綱圍祐州,降之。宋丘灊遣林拱持書乞和。辛丑,完顏匡攻襄陽,破其外城。僕散揆克含山,蒲察貞克天水,紇石烈子仁徇下來安、全椒二縣。壬寅,完顏綱徇下荔川、閭川等城。癸卯,丘灊復遣宋顯等以書幣乞和。乙巳,完顏綱克宕昌。丙午,蒲察貞克西和州。



 十二月丁未朔,完顏匡克宜城,僕散揆攻和州,史乂
 搭中流矢死。壬子,完顏綱次大潭縣,降之。蒲察貞克成州。癸丑,宋太尉、昭信軍節度使、四川宣撫副使吳曦納款于完顏綱。戊午,右監軍充攻下大散關。己未,紇石烈子仁克真州,丘灊復遣陳璧等奉書乞和。辛酉,右監軍充遣兀顏抄合以兵趣鳳州,城潰入焉。完顏綱遣京兆錄事張仔會吳曦于興州之置口。曦具言所以歸朝之意,仔請以告身為報,盡出以付之,仍獻階州。乙丑,初設都提控急遞鋪官。平章政事僕散揆班師。完顏綱以朝命,假太倉使馬良顯齎詔書、金印立吳曦為蜀王。戊辰,蒲察貞以西和、天水等捷來報。完顏匡進所掠女子百
 人。己巳,曦遣其果州團練使郭澄、提舉仙人關使任辛奉表及蜀地圖志、吳氏譜牒來上。壬申,詔完顏匡權尚書右丞,行省事、右副元帥如故。以紇石烈執中縱下虜掠,遣近臣杖其經歷阿里不孫等,仍詔放還所掠。



 七年春正月丁丑朔,高麗、夏遣使來賀。完顏匡進攻襄陽。戊寅,敕宰臣舉材幹官同議南征事。辛巳,詔御史大夫崇肅、同判大睦親府事徒單懷忠、吏部尚書範楫、戶部尚書高汝礪、禮部尚書張行簡、知大興府事溫迪罕思齊等十有四人同對于慶和殿。壬午,詔百官及前十四人同對于廣仁殿。甲申,朝獻于衍慶宮。乙酉,贈故壽
 州死節軍士魏全宣武將軍、蒙城令,封其妻鄉君,子俟年至十五收充八貫石正班局分承應,仍賜錢百萬。初,李爽圍壽州,刺史羲募人往斫敵營,全在選中,而為敵所執。敵令罵羲則免,全陽許,及至城下,反罵敵,遂殺之。至死罵不絕聲,故是有恩。戊子,召完顏綱赴闕。庚寅,僕散揆還駐下蔡而病。丙申,以左丞相宗浩兼都元帥,行省于南京以代揆。己亥,有司奏更定茶禁。辛丑,完顏匡取穀城。



 二月丙辰,赦鳳、成、西和、階、山五州,丁巳,詔追復永中、永蹈王爵。宋知樞密院張嚴遣方信孺以書詣平章政事揆、左丞端乞和。己未,獵于近郊。完顏匡克荊門
 軍。癸亥,如建春宮。吳曦遣使奉三表來:謝封爵,陳誓言,賀全蜀內附。丙寅,還宮。戊辰,平章政事兼左副元帥僕散揆薨于軍。癸酉,遣同知府事術虎高琪等冊吳曦為蜀國王。判平陽府事衛王永濟改武定軍節度使,兼奉聖州管內觀察使。是月,蜀國王吳曦為宋臣安丙所殺。



 三月戊子,幸太極宮。庚寅,詔撫陜西軍士。壬辰,初定蟲蝻生發地主及鄰主首不申之罪。宋復攻破階州。癸巳,復攻破西州。乙未,宣撫副使完顏綱至鳳翔,詔撤五州之兵,分保要害,綱召諸軍還。庚子,以完顏匡為左副元帥。壬寅,如萬寧宮。甲辰,幸西園。夏四月壬子,遣宮
 籍副監楊序為橫賜高麗王使。癸丑,宋人攻破散關,鞏州鈐轄兀顏阿失死之。丙辰,以紇石烈子仁為右副元帥。戊辰,詔元帥府分遣諸將遊奕淮南諸州。癸西,復下散關。



 五月己卯,幸東園射柳。己丑,幸玉泉山。丙申,宋知樞密院事張嚴復遣方信孺以書至都元帥府,增歲弊乞和。四川安撫使安丙遣西和州安撫使李孝羲率步騎三萬攻秦州,圍皁角堡。術虎高琪以兵赴之,七戰而解其圍。是月,放宮女二十人。



 六月乙巳朔,詔朝官六品、外官五品以上,及親王舉通錢穀官一人。不舉者罰,舉不當者論如律。己酉,以山東盜,制同黨能自殺捕出首
 官賞法。戊午,烏古論誼為元帥左監軍,完顏撒剌為元帥左都監。乙丑,遣使捕蝗。秋七月庚辰,朝獻于衍慶宮。壬午,詔民間交易、典質,一貫以上並用交鈔,毋用錢。乙酉,敕尚書省:「自今初受監察者令進利害帖子,以待召見。」甲午,左副元帥匡至自許州。乙未,詔核西夏人口,盡贖放還,敢有藏匿者以違制論。



 八月戊申,宋張嚴復遣方信孺齊其主誓書槁來乞和。庚戌,割汝州襄城縣於許州。戊辰,至自萬寧宮。



 九月甲戌朔,天壽節,高麗、夏遣使來賀。左丞相兼都元帥宗浩薨於軍。甲申,定西、北京,遼東鹽司判官諸場管勾增虧升降格。以尚書左丞僕
 散端為平章政事,封申國公,左副元帥完顏匡為平章政事兼左副元帥,封定國公。丙戌,獵于近郊。壬辰,還宮。戊戌,更定受制忘誤及誤寫制書事重加等罪。壬寅,敕女直人不得改為漢姓及學南人裝束。冬十月甲辰,詔應廕之家,旁正蔭足,其正廕者未出官而亡,許補廕一人。辛亥,以武庫令術甲法心為高麗生日使。丙辰,獵于近郊。己巳,詔定隨軍遷賞格。辛未,陜西宣撫使徒單鎰分遣副統把回海攻下蘇嶺關。是月,定南征將士功賞格。



 十一月癸酉,詔新定學令內削去薛居正《五代史》,止用歐陽修所撰。是日,都統押剌拔鶻嶺關、新道口,副統
 回海取小湖關、敖倉,進至營口鎮,遂取其城。丙子,宋韓侂胄遣左司郎中王柟以書來乞和,請稱伯,復增歲幣、犒軍錢,誅蘇師旦函首以獻。丙戌,上聞陜州防禦使紇石烈孛孫禁民糶,命尚書省罪之。壬辰,宋參知政事錢象祖以誅韓侂胄移書行省。甲午,獵于近郊。戊戌,參知政事賈鉉罷。詔完顏匡檄宋,函侂胄首以贖淮南故地。



 十二月壬寅朔,《遼史》成。丙午,以符寶郎烏古論福齡為夏國生日使。戊午,詔策論進士免試弓箭、擊球。庚申,以尚書右丞孫即康為左丞,參知政事獨吉思忠為右丞,中都路都轉運使孫鐸為參知政事。



 八年春正月辛未朔,高麗、夏遣使來賀。壬申,朝謁於衍慶宮。癸酉,收毀大鈔,行小鈔。以元帥左都監完顏撒剌為參知政事。乙亥,宋安丙遣兵襲鶻嶺關,副統把回海、完顏摑剌擊走之,斬其將景統領。丙子,左司郎中劉昂、通州刺史史肅、監察御史王宇、吏部主事曹元、吏部員外郎徒單永康、太倉使馬良顯、順州刺史唐括直思白坐與蒲陰令大中私議朝政,皆杖之。癸未,如春水。丙戌,如光春宮。



 二月乙巳,宋參知政事錢象祖遣王柟來,以書上行省,復請川、陜關隘。甲寅,如建春宮。庚申,諭有司曰:「方農作時,雖在禁地亦令耕種。」己巳,還宮。



 三月丁亥,
 幸瀛王第視疾。庚寅,以與宋和,諭尚書省。壬辰,宰臣上表謝罪。甲午,瀛王從憲薨。乙未,上親臨祭。夏四月癸卯,日暈三重,皆內黃外赤。戊申,禘于太廟。庚戌,如萬寧宮。甲寅,以北邊無事,敕尚書省:「命東北路招討司還治泰州,就兼節度使,其副招討仍置于邊。」詔諭有司:「以苗稼方興,宜速遣官分道巡行農事,以備蟲蝻。」詔更定猛安謀克承襲程試格。宋錢象祖復遣王柟以書上行省。庚申,詔諸路按察司歲公用錢。



 閏月辛未,諭尚書省曰:「翰林侍請學士蒲察畏也言,使宋官當選人,其言甚當。彼通謝使雖未到闕,其報聘人當行議擇。此乃更始,凡
 有禮數,皆在奉使。今既行之,遂為永例,不可不慎也。」甲戌,制諸州府司縣造作,不得役諸色人匠。違者準私役之律,計備以受所監臨財物論。甲申,定承應人收補年甲格。甲午,雨雹。定保甲軍殺獲南軍官賞。乙未,宋獻韓侂胄等首于元帥府。



 五月丁未,御應天門,備橫麾立杖,親王文武合班起居。中路兵馬提控、平南撫軍上將軍紇石烈貞以宋賊臣韓侂胄、蘇師旦首獻,并奉元帥府露布以聞。懸其首并畫像于市,以露布頒中外。丙辰,平章政事匡至自軍。己未,更元帥府為樞密院。癸亥,詔移天壽節於十月十五日。丁卯,遣使分路捕蝗。



 六月癸酉,
 宋通謝使朝議大夫、試禮部尚書許奕,福州觀察使、右武衛上將軍吳衡等奉其主書入見。甲戌,謁謝于衍慶宮。癸未,以許宋平,詔中外。免河南、山東、陜西等六路今年夏稅,河東、河北、大名等五路半之。丁亥,以元帥左都監烏古論誼為御史大夫。戊子,飛蝗入京畿。乙未,定服飾明金象金制。丁酉,以左副都點檢完顏侃為宋諭成使,禮部侍郎喬宇副之。秋七月戊戌朔,太白晝見。庚子,詔更定蝗蟲生發坐罪法。乙巳,朝獻于衍慶宮。詔頒《捕蝗圖》于中外。戊申,宋使朝辭,致答通謝書及誓書于宋主。



 八月壬申,更定遼東行使鈔法。癸酉,如建春宮。己丑,以
 戶部尚書高汝礪等為宋生日使。庚寅,如秋山。



 九月甲子,遣吏部尚書賈守謙等一十三人與各路按察司官推排民戶物力。乙丑,至自秋山。冬十月辛未,以吏部郎中郭郛為高麗生日使。辛巳,宋、高麗、夏遣使來賀。夏國有兵,遣使來告。癸未,更定安泊強竊盜罪格。辛卯,以軍民共譽為廉能官條附善最法。



 十一月丁酉朔,詔諸路按察使並兼轉運使。初設三司使,掌判鹽鐵、度支、勸農事。以樞密使紇石烈子仁兼三司使。癸卯,詔戒諭尚書省曰:「國家之治,在於紀綱。紀綱所先,賞罰必信。今迺上自省部之重,下逮司縣之間,律度弗循,私懷自便。遷延
 曠歲,茍且成風,習此為恒,從何致理?朝廷者百官之本,京師者諸夏之儀。其勖自今,各懲已往,遵繩奉法,竭力赴功。無枉撓以循情,無依違而避勢,壹歸於正,用範乃民。」是日,御臨武殿試護衛。丁未,敕諭臨潢泰州路兵馬都總管承裔等修邊備。乙卯,上不豫。丙辰,崩于福安殿,年四十一。大安元年春正月,謚曰憲天光運仁文義武神聖英孝皇帝,廟號章宗。二月甲申,葬道陵。



 贊曰:章宗在位二十年,承世宗治平日久,宇內小康,乃正禮樂,修刑法,定官制,典章文物粲然成一代治規。又數問群臣漢宣綜核名實、唐代考課之法,蓋欲跨遼、宋
 而比跡於漢、唐,亦可謂有志於治者矣!然婢寵擅朝,塚嗣未立,疏忌宗室而傳授非人。向之所謂維持鞏固於久遠者,徒為文具,而不得為後世子孫一日之用,金源氏從此衰矣!昔揚雄氏有云:「秦之有司負秦之法度,秦之法度負聖人之法度。」蓋有以夫。



\end{pinyinscope}