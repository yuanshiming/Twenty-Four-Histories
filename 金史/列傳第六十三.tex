\article{列傳第六十三}

\begin{pinyinscope}

 文藝上



 ○韓昉蔡松年子珪吳激馬定國任詢趙可郭長倩蕭永祺胡礪王競楊伯仁鄭子聃黨懷英



 金初未有文字。世祖以來,漸立條教。太祖既興,得遼舊
 人用之,使介往復,其言已文。太宗繼統,乃行選舉之法,及伐宋,取汴經籍圖,宋士多歸之。熙宗款謁先聖,北面如弟子禮。世宗、章宗之世,儒風丕變,庠序日盛,士由科第位至宰輔者接踵。當時儒者雖無專門名家之學,然而朝廷典策、鄰國書命,粲然有可觀者矣。金用武得國,無以異於遼,而一代制作,能自樹立唐、宋之間,有非遼世所及,以文而不以武也。《傳》曰:「言之不文,行之不遠。」文治有補於人之家國,豈一日之效哉。作《文藝傳》。



 韓昉,字公美,燕京人。仕遼,累世通顯。昉五歲喪父,哭泣能盡哀。天慶二年,中進士第一。補右拾遺,轉史館修撰。
 累遷少府少監、乾文閣待制。加衛尉卿,知制誥,充高麗國信使。高麗雖舊通好,天會四年,奉表稱籓而不肯進誓表,累使要約,皆不得要領。而昉復至高麗,移督再三。高麗徵國中讀書知古今者,商榷辭旨,使酬答專對。凡涉旬乃始置對,謂昉曰:「小國事遼、宋二百年無誓表,未嘗失籓臣禮。今事上國,當與事遼、宋同禮。而屢盟長亂,聖人所不與,必不敢用誓表。」昉曰:「貴國必欲用古禮,舜五載一巡狩,群后四朝。周六年五服一朝,又六年王乃時巡,諸侯各朝于方岳。今天子方事西狩,則貴國當從朝會矣。」高麗人無以對,乃曰:「徐議之。」昉曰:「誓表朝會,一
 言決耳。」於是高麗乃進誓表如約,昉乃還。宗乾大說曰:「非卿誰能辦此。」因謂執事者曰:「自今出疆之使,皆宜擇人。」



 明年,加昭文館直學士,兼堂後官。再加諫議大夫,遷翰林侍講學士。改禮部尚書,遷翰林學士,兼太常卿、修國史,尚書如故。昉自天會十二年入禮部,在職凡七年。當是時,朝廷方議禮,制度或因或革,故昉在禮部兼太常甚久云。除濟南尹,拜參知政事。皇統四年,表乞致仕,不許。六年,再表乞致仕,乃除汴京留守,封鄆國公。復請如初,以儀同三司致仕。天德初,加開府儀同三司。薨。年六十八。



 昉性仁厚,待物甚寬。有家奴誣告昉以馬資送
 叛人出境,考之無狀,有司以奴還昉,昉待之如初,曰:「奴誣主人以罪,求為良耳,何足怪哉。」人稱其長者。昉雖貴,讀書未嘗去手,善屬文,最長於詔冊,作《太祖睿德神功碑》,當世稱之。自使高麗歸,後高麗使者至,必問昉安否云。



 蔡松年,字伯堅。父靖,宋宣和末,守燕山。松年從父來,管勾機宜文字。宗望軍至白河,郭藥師敗,靖以燕山府降,元帥府辟松年為令史。天會中,遼、宋舊有官者皆換授,松年為太子中允,除真定府判官,自此為真定人。



 嘗從元帥府與齊俱伐宋。是時,初平真定西山群盜,山中居
 民為賊汙者千餘家,松年力為辨論,竟得不坐。齊國廢,置行臺尚書省於汴,松年為行臺刑部郎中,都元帥宗弼領行臺事,伐宋,松年兼總軍中六部事。宋稱臣,師還,宗弼入為左丞相,薦松年為刑部員外郎。皇統七年,尚書省令史許霖告田玨黨事,松年素與玨不相能。是時宗弼當國,玨性剛正,好評論人物,其黨皆君子,韓企先為相愛重之。而松年、許霖、曹望之欲與玨相結,玨拒之,由是構怨。故松年、許霖構成玨等罪狀,勸宗弼誅之,君子之黨熄焉。是歲,松年遷左司員外郎。



 松年前在宗弼府,而海陵以宗室子在宗弼軍中任使,用是相厚善。天
 德初,擢吏部侍郎,俄遷戶部尚書。海陵遷中都,徙榷貨物以實都城,復鈔引法,皆自松年啟之。海陵謀伐宋,以松年家世仕宋,故亟擢顯位以聳南人觀聽,遂以松年為賀宋正旦使,使還改吏部尚書,尋拜參知政一。是年,自崇德大夫進銀青光祿大夫,遷尚書右丞。未幾,為左丞,封郜國公。



 初,海陵愛宋使人山呼聲,使神衛軍習之。及孫道夫賀正隆三年正旦,入見,山呼聲不類往年來者。道夫退,海陵謂宰臣曰:「宋人知我使神衛軍習其聲,此必蔡松年、胡礪泄之。」松年惶恐對曰:「臣若懷此心,便當族滅。」



 久之,進拜右丞相,加儀同三司,封衛國公。正隆四
 年薨,年五十三。海陵悼惜之,奠于其第,命作祭文以見意。加封吳國公,謚文簡。起復其子三河主簿珪為翰林修撰,璋賜進士第。遣翰林待制蕭籲護送其喪,歸葬真定,四品以下官離都城十里送之,道路之費,皆從官給。



 松年事繼母以孝聞,喜周恤親黨,性復豪侈,不計家之有無。文詞清麗,尤工樂府,與吳激齊名,時號「吳蔡體。」有集行于世。子珪。



 珪字正甫。中進士第,不求調,久乃除澄州軍事判官,遷三河主簿。丁父憂,起復翰林修撰,同知制誥。在職八年,改戶部員外郎,兼太常丞。珪號為辨博,凡朝廷制度損
 益,珪為編類詳定檢討刪定官。



 初,兩燕王墓舊在中都東城外,海陵廣京城圍,墓在東城內。前嘗有盜發其墓,大定九年詔改葬於城外。俗傳六國時燕王及太子丹之葬,及啟壙,其東墓之柩題其和曰「燕靈王舊。」「舊」,古「柩」字,通用。乃西漢高祖子劉建葬也。其西墓,蓋燕康王劉嘉之葬也。珪作《兩燕王墓辯》,據葬制名物款刻甚詳。



 安國軍節度判官高元鼎坐監臨姦事,求援於太常博士田居實、大理司直吳長行、吏部主事高震亨、大理評事王元忠。震亨以屬鞫問官御史臺典事李仲柔,仲柔發之。珪與刑部員外郎王翛、宛平主簿任詢、前衛州防禦
 判官閻恕、承事郎高復亨、文林郎翟詢、敦武校尉王景晞、進義校尉任師望,坐與居實等轉相傳教,或令元鼎逃避,居實、長行、震亨、元忠各杖八十,翛、珪、詢、恕、復亨、霍詢各笞四十,景晞、師望各徒二年,官贖外並的決。



 久之,除河東北路轉運副使,復入為脩撰,遷禮部郎中,封真定縣男。珪已得風疾,失音不能言,乃除濰州刺史,同輩已奏謝,珪獨不能入見。世宗以讓右丞唐括安禮、參政王蔚曰:「卿等閱書史,亦有不能言之人可以從政者乎。」又謂中丞劉仲誨曰:「蔡珪風疾不能奏謝,卿等何不糾之。人言卿等相為黨蔽,今果然邪?」珪乃致仕。尋卒。



 珪之
 文有《補正水經》五篇,合沈約、蕭子顯、魏收宋、齊、北魏志作《南北史志》三十卷,《續金石遺文跋尾》十卷,《晉陽志》十二卷,《文集》五十五卷。《補正水經》、《晉陽志》、《文集》今存,餘皆亡。



 吳激,字彥高,建州人。父拭,宋進士,官終朝奉郎、知蘇州。激,米芾之婿也。工詩能文,字畫俊逸,得芾筆意。尤精樂府,造語清婉,哀而不傷。將宋命至金,以知名留不遣,命為翰林待制。皇統二年,出知深州,到官三日卒。詔賜其子錢百萬、粟三百斛、田三頃以周其家。有《東山集》十卷行于世。「東山」,其自號也。



 馬定國字子卿,茌平人。自少志趣不群。宣、政未末題詩酒家壁,坐譏訕得罪,亦因以知名。阜昌初,遊歷下,以詩撼齊王豫,豫大悅,授監察御史,仕至翰林學士。《石鼓》自唐以來無定論,定國以字畫考之,云是宇文周時所造,作辯萬餘言,出入傳記,引據甚明,學者以比蔡正甫《燕王墓辯》。初,學詩未有入處,夢其父與方寸白筆,從是文章大進。有集傳于世。



 任詢,字君謨,易州軍市人。父貴,有才幹,善畫,喜談兵,宣、政間游江、浙。詢生於虔州,為人慷慨多大節。書為當時第一,畫亦入妙品。評者謂畫高於書,書高於詩,詩高於
 文,然王庭筠獨以其才具許之。登正隆二年進士第。歷益都都勾判官,北京鹽使。年六十四致仕,優游鄉里,家藏法書名畫數百軸。年七十卒。



 趙可,字獻之,高平人。貞元二年進士。仕至翰林直學士。博學高才,卓犖不羈。天德、貞元間,有聲場屋。後入翰林,一時詔誥多出其手,流輩服其典雅。其歌詩樂府尤工,號《玉峰散人集》。



 郭長倩,字曼卿,文登人。登皇統丙寅經義乙科。仕至秘書少監,兼禮部郎中,修起居注。與施朋望、王無競、劉巖老、劉無黨相友善。所撰《石決明傳》為時輩所稱。有《崑崙
 集》行于世。



 蕭永祺,字景純,本名蒲烈。少好學,通契丹大小字。廣寧尹耶律固奉詔譯書,時置門下,因盡傳其業。固卒,永祺率門弟子服齊衰喪。固作《遼史》未成,永祺繼之,作紀三十卷、志五卷、傳四十卷,上之。加宣武將軍,除太常丞。



 海陵為中京留守,永祺特見親禮。天德初,擢左諫議大夫,遷翰林侍講學士,同修國史,再遷翰林學士。明年,遷承旨。尚書左丞耶律安禮出守南京,海陵欲以永祺代之,召見于內閣,諭以旨意,永祺辭曰:「臣才識卑下,不足以辱執政。」海陵曰:「今天下無事,朕方以文治,卿為是優矣。」
 永祺固辭。既出,或問曰:「公遇知人主,進取爵位,以道佐時,何多讓也?」永祺曰:「執政繫天下休戚,縱欲貪冒榮寵,如蒼生何!」海陵嘗選廷臣十人備諮訪,獨永祺議論寬厚,時稱長者。卒年五十七。



 胡礪,字元化,磁州武安人。少嗜學。天會間,大軍下河北,礪為軍士所掠,行至燕,亡匿香山寺,與傭保雜處。韓昉見而異之,使賦詩以見志,礪操筆立成,思致清婉,昉喜甚,因館置門下,使與其子處,同教育之,自是學業日進。昉嘗謂人曰:「胡生才器一日千里,他日必將名世。」十年,舉進士第一,授右拾遺,權翰林修撰。久之,改定州觀察
 判官。定之學校為河朔冠,士子聚居者常以百數,礪督教不倦,經指授者悉為場屋上游,稱其程文為「元化格」。



 皇統初,為河北西路轉運都勾判官。礪性剛直無所屈。行臺平章政事高楨之汴,道真定,燕於漕司。礪欲就坐,楨責之,礪曰:「公在政府則禮絕百僚,今日之會自有賓主禮。」楨曰:「汝他日為省吏當何如?」礪曰:「當官而行,亦何所避。」楨壯其言,改謝之。



 改同知深州軍州事,加朝奉大夫。郡守暴戾,蔑視僚屬,礪常以禮折之,守愧服,郡事一委于礪。州管五縣,例置弓手百餘,少者猶六七十人,歲徵民錢五千餘萬為顧直。其人皆市井無賴,以迹盜為名,
 所至擾民。礪知其弊,悉罷去。繼而有飛語曰:「某日賊發,將殺通守。」或請為備,礪曰:「盜所利者財耳,吾貧如此,何備為。」是夕,令公署撤關,竟亦無事。



 再補翰林修撰,遷禮部郎中,一時典禮多所裁定。海陵拜平章政事,百官賀於廟堂,礪獨不跪。海陵問其故,礪以令對,且曰:「朝服而跪,見君父禮也。」海陵深器重之。天德初,再遷侍講學士,同修國史。以母憂去官。起復為宋國歲元副使,刑部侍郎白彥恭為使,海陵謂礪曰:「彥恭官在卿下,以其舊勞,故使卿副之。」遷翰林學士,改刑部尚書。扈從至汴得疾,海陵數遣使臨問,卒,深悼惜之。年五十五。



 王競,字無競,彰德人。警敏好學。年十七以廕補官。宋宣和中,太學兩試合格,調屯留主簿。入國朝,除大寧令,歷寶勝鹽官,轉河內令。時歲饑盜起,競設方略以購賊,不數月盡得之。夏秋之交,沁水泛溢,歲發民築堤,豪民猾吏因緣為姦,競核實之,減費幾半,縣民為之諺曰:「西山至河岸,縣官兩人半。」蓋以前政韓希甫與競相繼治縣,皆有幹能,絳州正平令張元亦有治績而差不及,故云然。



 天眷元年,轉固安令。皇統初,參政韓昉薦之,召權應奉翰林文字,兼太常博士。詔作《金源郡王完顏婁室墓碑》,競以行狀盡其實,乃請國史刑正之,時人以為法。二
 年,試館閣,競文居最,遂為真。



 遷尚書禮部員外郎。時海陵當國,政由己出,欲令百官避堂諱,競言人臣無公諱,遂止。蕭仲恭以太傅領三省事封王,欲援遼故事,親王用紫羅傘。事下禮部,競與郎中翟永固明言其非是,事竟不行,海陵由是重之。天德初,轉翰林待制,遷翰林直學士,改禮部侍郎,遷翰林侍講學士,改太常卿,同修國史,擢禮部尚書,同修國史如故。大定二年春,從太傅張浩朝京師,詔復為禮部尚書。是歲,奉遷睿宗山陵,儀注不應典禮,競削官兩階。詔改創五龍車,兼翰林學士承旨,修國史。四年,卒官。



 競博學而能文,善草隸書,工大字,
 兩都宮殿榜題,皆競所書,士林推為第一云。



 楊伯仁,字安道,伯雄之弟也。天性孝友,讀書一過成誦。登皇統九年進士第,事親不求調。天德二年,除應奉翰林文字。初名伯英,避太子光英諱,改今名。海陵嘗夜召賦詩,傳趣甚亟,未二鼓奏十詠,海陵喜,解衣賜之。海陵射烏,伯仁獻《獲烏詩》以諷。丁父憂,起復,賜金帶襲衣,及賜白金以奉母。改左拾遺。進士呂忠翰廷試已在第一,未唱名,海陵以忠翰程文示伯仁,問其優劣,伯仁對曰:「當在優等。」海陵曰:「此今試狀元也。」伯仁自以知忠翰姓名在第一,遂宿諫省,俟唱名乃出,海陵嘉其慎密。轉翰
 林修撰。孟宗獻發解第一,伯仁讀其程文,稱之「此人當成大名」。是歲,宗獻府試、省試、廷試皆第一,號「孟四元」,時論以為知文。故事,狀元官從七品,階承務郎,世宗以宗獻獨異等,與從六品,階授奉直大夫。



 改著作郎。居母喪,服除,調鎮西節度副使。入為起居注兼左拾遺,上書論時務六事。改大名少尹。郡中豪民橫恣甚,莫可制,民受其害,伯仁窮竟渠黨,四境帖然。讞館陶大辟,得其冤狀,館陶人為立祠。府尹荊王文坐贓削封,降德州防禦使,同知裴滿子寧及伯仁、判官謝奴皆以不能匡正解職。伯仁降南京留守判官,改同知安化軍節度使,到官三
 日,召為太子右諭德、兼侍御史,改翰林待制,復兼右諭德。



 除濱州刺史。郡俗有遣奴出亡,捕之以規賞者,伯仁至,責其主而杖殺其奴,如是者數輩,其弊遂止。入為左諫議大夫,兼禮部侍郎、翰林直學士。故事,諫官詞臣入直禁中,上閔其勞,特免入直。改吏部侍郎,直學士如故。鄭子聃卒,宰相舉伯仁代之,乃遷待講兼禮部侍郎。



 伯仁久在翰林,文詞典麗,上曰:「自韓昉、張鈞後,則有翟永固,近日則張景仁、鄭子聃,今則伯仁而已,其次未見能文者。呂忠翰草《降海陵庶人詔》,點竄再四,終不能盡朕意,狀元雖以詞賦甲天下,至於辭命,未必皆能。凡進士
 可令補外,考其能文者召用之。」不數月,兼左諫議大夫,俄兼太常卿。大臣舉可修起居注者數人,上以伯仁領之。從幸上京,伯仁多病,至臨潢,地寒因感疾,還中都。明年,上還幸中都,遣使勞問,賜以丹劑。是歲,卒。



 鄭子聃,字景純,大定府人。父宏,遼金源令,二子子京、子聃。楊丘行嘗謂人曰:「金源二子,鳳毛也。小者尤特達,後必名世。」子聃及冠,有能賦聲。天德三年,丘行為太子左衛率府率,廷試明日,海陵以子聃程文示丘行,對曰:「可入甲乙。」及拆卷,果中第一甲第三人。調翼城丞,遷贊皇令,召為書畫直長。



 子聃頗以才望自負,常慊不得為第
 一甲第一人。正隆二年會試畢,海陵以第一人程文問子聃,子聃少之。海陵問作賦何如,對曰:「甚易。」因自矜,且謂他人莫己若也。海陵不悅,乃使子聃與翰林修撰綦戩、楊伯仁、宣徽判官張汝霖、應奉翰林文字李希顏同進士雜試。七月癸未,海陵御寶昌門臨軒觀試,以「不貴異物民乃足」為賦題,「忠臣猶孝子」為詩題,「憂國如飢渴」為論題。上謂讀卷官翟永固曰:「朕出賦題,能言之或能行之,未可知也。詩、論題,庶戒臣下。」丁亥,御便殿親覽試卷,中第者七十三人,子聃果第一,海陵奇之。有頃,進官三階,除翰林修撰。改侍御史。



 京畿旱,詔子聃決囚,遂澍
 雨,人以比顏真卿。遷待制,兼吏部郎中,改秘書少監。遷翰林直學士,兼太子左諭德,顯宗深器重之。以疾求補外,遂為沂州防禦使,皇太子幣贐甚厚,命以安輿之官。召還,為左諫議大夫、兼直學士。改吏部侍郎、同修國史,直學士如故。遷侍講、兼修國史,上曰:「修《海陵實錄》,知其詳無如子聃者。」蓋以史事專責之也。二十年,卒,年五十五。子聃英俊有直氣,其為文亦然。平生所著詩文二千餘篇。



 黨懷英,字世傑,故宋太尉進十一代孫,馮翊人。父純睦,泰安軍錄事參軍,卒官,妻子不能歸,因家焉。應舉不得
 意,遂脫略世務,放浪山水間。簞瓢屢空,晏如也。大定十年,中進士第,調莒州軍事判官,累除汝陰縣令、國史院編修官、應奉翰林文字、翰林待制、兼同修國史。



 懷英能屬文,工篆籀,當時稱為第一,學者宗之。大定二十九年,與鳳翔府治中郝俁充《遼史》刊修官,應奉翰林文字移剌益、趙渢等七人為編修官。凡民間遼時碑銘墓志及諸家文集,或記憶遼舊事,悉上送官。是時,章宗初即位,好尚文辭,旁求文學之士以備侍從,謂宰臣曰:「翰林闕人如之何?」張汝霖奏曰:「郝俁能屬文,宦業亦佳。」上曰:「近日制詔惟黨懷英最善。」移剌履進曰:「進士擢第後止習
 吏事,更不復讀書,近日始知為學矣。」上曰:「今時進士甚滅裂,《唐書》中事亦多不知,朕殊不喜。」上謂宰臣曰:「郝俁賦詩頗佳,舊時劉迎能之,李晏不及也。」



 明昌元年,懷英再遷國子祭酒。二年,遷侍講學士。明年,議開邊防濠塹,懷英等十六人請罷其役,詔從之。遷翰林學士。七年,有事于南郊,攝中書侍郎讀祝冊,上曰:「讀冊至朕名,聲微下,雖曰尊君,然在郊廟,禮非所宜,當平讀之。」承安二年乞致仕,改泰寧軍節度使。明年,召為翰林學士承旨。泰和元年,增修《遼史》編修官三員,詔分紀、志、列傳刊修官,有改除者以書自隨。久之,致仕。大安三年卒,年七十八,
 謚文獻。懷英致仕後,章宗詔直學士陳大任繼成《遼史》云。



\end{pinyinscope}