\article{志第十九}

\begin{pinyinscope}

 禮十一



 ○外國使入見儀



 皇帝即御座,鳴鞭、報時畢,殿前班小起居,引至侍立位。引臣僚左右入,至丹墀,小起居畢,宰執上殿,其餘臣僚分班出。閣門使奏使者入見榜子。先引宋使、副,出笏,捧書左入,至丹墀北向立。閣使左下接書,捧書者單跪授書,拜,起立。閣使左上露階,右入欄內奏:「封全。」轉讀畢,引使、副左上露階,齊揖入欄內,揖使副鞠躬,
 使少前拜跪,附奏畢,拜起,復位立。待宣問宋皇帝時並鞠躬,受敕旨,再揖鞠躬,使少前拜跪,奏畢,起復位,齊退卻,引使、副左下,至丹墀北嚮立。禮物右入左出,盡,揖使、副傍折通班,再引至丹墀,舞蹈,五拜,不出班奏:「聖躬萬福。」再拜。揖使副鞠躬,使出班謝面天顏,復位,舞蹈,五拜。再揖副使鞠躬,使出班謝遠差接伴、兼賜湯藥諸物等,復位,舞蹈,五拜。各祗候,引右出,賜衣。次引宋人從入,通名已下再拜不出班,又再拜,各祗候,亦引右出。次引高麗使左入,至丹墀北嚮略立,引使左上露階,立定。揖橫使鞠躬,正使少前拜跪,附奏畢,拜起,復位立,閣使宣問
 高麗王時並鞠躬,受敕旨畢,再揖橫使鞠躬,正使少前拜跪,奏畢,拜起,復位,齊退卻,引左下,至丹墀,面殿立定。禮物右入左出,盡,揖使傍折通班,畢,引至丹墀,通一十七拜,各祗候,平立,引左階立。次引夏使見如上儀,引右階立。次再引宋使副左入,至丹墀,謝恩,舞蹈,五拜,各祗候,平立。次引高麗、夏使並至丹墀。三使並鞠躬,有敕賜酒食,舞蹈,五拜,各祗候,引右出。次引宰執下殿,禮畢。



 ○曲宴儀



 皇帝即御座,鳴鞭,報時畢,殿前班小起居,到侍立位。引臣僚並使客左入,傍折通班,至丹墀舞蹈,五拜,不出班奏:「聖躬萬福。」又再拜。出班謝宴,舞蹈,五拜,各上
 殿祗候。分引預宴官上殿,其餘臣僚右出。次引宋使從人入,至丹墀再拜,不出班奏:「聖躬萬福。」又再拜。有敕賜酒食,又再拜,引左廊立。次引高麗、夏從人入,分引左右廊立。果床入,進酒。皇帝舉酒時,上下侍立官並再拜,接盞,畢,候進酒官到位,當坐者再拜,坐,即行臣使酒。傳宣,立飲畢,再拜,坐。次從人再拜,坐。至四盞,餅茶入,致語。聞鼓笛時,揖臣使并人從立,口號絕,坐宴并侍立官並再拜,坐,次從人再拜,坐。食入,五盞,歇宴。教坊謝恩畢,揖臣使起,果床出。皇帝起入閣,臣使下殿歸幕次。賜花,人從隨出戴花畢,先引人從入,左右廊立,次引臣使入,左右
 上殿位立。皇帝出閣坐,果床入,坐立並再拜,坐,次從人再拜,坐。九盞,將曲終,揖從人至位再拜,引出。聞曲時,揖臣使起,再拜,下殿。果床出。至丹墀謝宴,舞蹈,五拜。分引出。



 ○朝辭儀



 皇帝即御座,鳴鞭、報時畢,殿前班小起居,至侍立位。引臣僚合班入,至丹墀小起居,引宰執上殿,其餘臣僚分班出。閣門使奏辭榜子。先引夏使左入,傍折通班畢,至丹墀再拜,不出班奏:「聖躬萬福。」又再拜。揖使副鞠躬,使出班,戀闕致詞,復位,又再拜,唱:「各好去。」引右出。次引高麗使,如上儀,亦引右出。次引宋使副左入,傍折通
 班畢,至丹墀,依上通六拜,各祗候,平立。閣使賜衣馬,鞠躬,聞敕,再拜。賜衣馬畢,平身,搢笏,單跪,受別錄物過盡,出笏,拜起,謝恩,舞蹈,五拜。有敕賜酒食,舞蹈,五拜。引使副左上露階,齊揖入欄內,揖鞠躬,大使少前拜跪受書,起復位。揖使副齊鞠躬,受傳達畢,齊退,引左下至丹墀,鞠躬,喝:「各好去。」引右出。次引宰執下殿,禮畢。熙宗時,夏使入見,改為大起居。定制以宋使列於三品班,高麗、夏列於五品班。皇統二年六月,定臣使辭見,臣僚服色拜數止從常朝起居,三國使班品如舊。俟殿前班及臣僚小起居畢,宰執升殿,餘臣分班畢,乃令行入見及朝辭
 之禮。凡入見則宋使先,禮畢夏使入,禮畢而高麗使入。其朝辭則夏使先,禮畢而高麗使入,禮畢而宋使入。夏、高麗朝辭之賜,則遣使就賜於會同館。惟宋使之賜則庭授。舊高麗使至闕皆有私進禮,大定五年,上以宋、夏使皆無此禮,而小國獨有之,不可。遂命罷之。六年,詔外國使初見、朝辭則於左掖門出入,朝賀,賜宴則由應天門東偏門出入。



 大定二十九年三月,章宗以在諒暗,免宋使朝辭,太常寺言:「若不面授書及傳達語言,恐後別有違失。」遂令宋使先辭靈幄,然後詣仁政殿朝辭,授書。時右丞相襄言:「伏見熙宗聖誕七月七日,以景宣忌辰
 避之,更為翌日,復用正月十七日受外國賀。今聖誕節若依期,令外方人使過界,恐為雨潦所滯,設能到闕,或值陰雨亦難行禮,乞以正月十一日或三月十五日為聖節,定宋人過界之期。」平章政事張汝霖、參知政事劉瑋等言:「帝王當示信,以雨潦路阻輒改之,或恐失信。且宋帝生日亦五月也,是時都在會寧,上國遣使賜生日,萬里渡越江、河,尚不避霖潦,如期而至。今久與宋好,不可以小阻示以不實。彼若過界,多作程頓亦不至留滯,縱使雨水愆期而入見,猶勝更用他日也。」御史大夫唐括貢、中丞李晏、刑部尚書兼右諫議大夫完顏守貞等
 亦皆言不可,上初從之,既而竟用襄議,令有司移報,使明知聖誕之實,特改其日以示優待行人之意。承安三年正月,上諭旨有司曰:「此聞宋國花宴,殿上不設肴饌,至其歇時乃備於廊下。今花宴上賜食甚為拘束,若依彼例可乎?且向者人使見辭,殿上亦嘗有酒禮,今已移在館宴矣」。有司奏曰:「曲宴之禮舊矣。彼方,酒一行、食一上必相須成禮。而國朝之例,酒既罷而食始進。至於花宴日,宋使至客省幕次有酒禮,而我使至其幕則有食而無酒,各因其舊,不必相同。古者宴禮設食以示慈惠,今遽更之,恐遠人有疑,失朝廷寵待臣子之意。」乃命止
 如舊。正大元年十月,夏國遣使修好。二年九月,夏國和議定,以兄事金,各用本國年號,定擬使者見辭儀注云。蓋夏人自天會議和,臣屬於金八十餘年,無兵革事。及貞祐之初,小有侵掠,以至構難十年,兩國俱敝,至是,始以兄弟之國成和。十月,遣禮部尚書奧敦良弼、大理卿裴滿欽甫、侍御史烏古孫弘毅為報成使。三年十月,夏人告哀,遣中大夫完顏履信為弔祭使。夏人以兵事方殷,各停使聘。四年,遣王立之來聘,未復命而夏亡。



 ○新定夏使儀注



 夏國使、副及參議各一,謂之使。都管三。上節、中節各五、下節二十四,謂之三節人從。報至行省,
 差接伴使與書表人迓於境。入界,則先具驛程腰宿之次。始至京兆行省,翌日賜宴,至河南行省亦然,謂之來宴。將至京,遣內侍一人以油絹復韜三銀盒,貯湯樂二十六品,逆於近境尉氏縣賜之。至恩華館舊名燕賓館,承安三年更名更衣,由宜照門入,預差館伴使、副使二員,書表四人,牽攏官三十人以俟。來使三節人從至會同館,謂之聚,先以館伴使名銜付之,而使者亦以其銜呈,然後使、副、都管、上中節人從以次見館伴使。接伴使初相見之儀亦然。次以館伴所書表見人使,館伴所牽攏官與下節人互相參見,畢,乃請館伴、接伴人,使、副,各公服齊出
 幕次,對行上欄子外,館伴在北,對立。先接伴揖,次來使副與館伴互展狀,揖,各傳示,再揖。各就位,請收笏坐,先湯,次酒三盞,置果殽。茶罷,執笏,近前齊起,欄子外館伴在南,對立。先館伴揖,次展接伴辭狀,相別揖,各傳示,再揖,通揖分位。是日,皇帝遣使撫問。天使至館,轉銜如館伴初見之儀。館伴與天使、來使副各公服,齊行至位,對立。請來使副升拜褥望闕立,次請天使升拜褥稍前立。來使副鞠躬,天使言:「有敕。」乃再拜鞠躬。天使口宣辭畢,復位。來使再拜,舞蹈,三拜,復位立。來使與天使各展狀,相見揖,次館伴揖。來使令人傳示,請館伴、天使與來
 使對行上,各赴椅子立,通揖。謹收笏坐,湯酒殽茶並如前,畢,執笏,近前,齊請起,至拜褥,依前對立。請來使副升褥位,進表謝撫問,再拜,副使平立,使跪奉表,天使近前搢笏受之,出笏復位,來使就拜,退,復對立。來使令人傳示館伴,依例書送天使土物,畢,展天使辭狀,相別揖,次館伴揖,各請分位。是後,每旦暮傳示,并牽隴官聲喏如儀。到館之明日,遣使賜酒果,天使初至轉銜後,望拜傳宣皆如撫問之儀。使副單跪,以酒果過其側,拜、舞蹈如儀。上湯酒茶畢,詣拜褥位,跪進謝賜酒果表,贈天使土物皆如撫問使禮,押酒果軍亦有土物之贈。乃命
 閣門副使至館習儀,初轉銜前後皆如館伴相見之儀。湯茶罷,館伴閣副傳示使副,來日入見,例當習儀。來使副回傳示,習儀畢。第二盞後,當面勸習儀承受人酒一盞,先揖,飲酒,再拜退。三盞果茶罷,執笏近前齊起,欄子外南為上,對立。以來日入見,故但揖而不展辭狀,分位。乃以入見榜子付閣門持去,以付禮進司。來使副以書送土物於引進使,及交進物軍員人等,閣門副及習儀承受人各贈土物。



 第三日,入見。其日質明,都管、三節人從皆裹帶,館伴與來使副各公服,齊請赴馬臺,館伴牽攏官喝:「排馬。」來使牽攏官喝:「牽馬。」各上馬張蓋。都管馬
 上奉書在使前,至中門外,以外為上,對立,先來使牽攏官兩聲喏,次館伴牽攏官亦然,齊揖,各傳示,再揖,請行。至左掖門外五百步,館伴與使副乃左右易位而行。揖畢,去門百步去傘下馬,出笏,對行。凡後入稱賀、曲宴皆同是儀。來使人從持物者不得入門,牽攏官權收之。客省令二人傳示,館伴與來使各令人回傳示。至客省幕前,館伴所書表在上立,齊揖,乃入幕。先館伴所書表傳示,次來使書表傳示,依前欄子外立,先揖,當面勸酒一盞,再揖,退。引館伴來使入客省幕,內為上,對立揖畢,請分位立。先館伴揖,次展客省起居狀,揖,各傳示,再揖,通
 揖。請赴位立,再揖,請收笏坐。先湯,次酒三盞,各有果殽。第二盞酒畢,客省乃傳示來使,請都管、上中節勸酒。回傳示畢,引都管、上中節於幕次前階下排立,先揖,飲酒,再揖,引退。第三盞酒畢,茶罷,執笏,近前齊起,幕次前立,通揖畢,各歸本幕次。俟殿上小起居畢,宰執升殿,餘臣分班退,閣使奏來使見榜子。乃先請館伴入班。俟閣門招引,乃請客省與來使副對立於幕前,外為上。使者奉書,揖畢對行,至三門外,與引揖閣副揖。使奉書,副出笏後隨,左上露臺殿簷柱外,奉書單跪舊儀於丹墀內奉書,閣使接書,使副就拜,立。閣使右入欄子內,奏:「封全。」轉讀畢,故事
 皆不讀。引使副入殿欄子內,揖使副鞠躬再拜,引少前跪奏:「弟大夏皇帝致問兄大金皇帝,聖躬萬福。」再拜,興,復位。皇帝乃宣問夏皇帝,使副鞠躬受旨,畢,引使少前跪奏:「弟大夏皇帝聖躬萬福。」拜,復位,立。齊退,左下階,至丹墀北向立。以禮物右入左出,盡,揖使副傍折通班。再引至丹墀,舞蹈,五拜,不出班代奏:「聖躬萬福。」畢,再拜。引使副前,雙跪,皇帝遣人勞問,復位,謝恩,舞蹈,五拜。再揖使副出班,謝面天顏,復位,舞蹈,五拜。再揖閣副鞠躬,引使出班,謝遠差接伴兼賜湯藥諸物,復位,舞蹈,五拜。喝:「各祗候。」引右出,至三門階下,與閣副揖別,與客省同行至幕
 次前對揖,各歸幕次。引都管、上中節左入,丹墀立,下節於門外階下立,齊鞠躬通名,先再拜,不出班奏:「聖躬萬福。」再拜。下節鞠躬聲喏,初一拜呼「萬歲」,次一拜呼「萬歲」,臨起呼「萬萬歲」。喝:「各祗候。」平立,引右出。乃賜使者衣,拜舞皆如賜酒果之儀,畢,使者與天使對立。次請都管、三節人從望闕立,天使稍前立,都管人從鞠躬,天使傳敕,拜謝如使儀,就拜畢,謝恩再拜。下節鞠躬聲喏,如入見儀,乃再引入,賜以酒食,閣門招、客省皆如入見儀。至丹墀,謝賜衣物,再拜,舞蹈,三拜,鞠躬。贊:「有敕賜酒食。」舞蹈,五拜。喝:「各祗候。」引右出,如前儀,歸幕。乃請出,館伴與使
 副幕前對立揖,各傳示,再揖,請行。至元下馬所,復左右易位而行,揖畢,各收笏,上馬至館。又左右易位入門,內為上,對立。先來使牽攏官,次館伴牽攏官,各聲喏,再拜揖,畢,請分位。乃以押伴使賜宴於館。押伴至館,轉名銜回畢,與館伴、來使公服,齊詣褥位對立,押伴稍前立。先請押伴、館伴上褥位,望闕拜,謝坐,再拜,舞蹈,三拜,起。先請押伴上副階上立,乃引使副上褥位,望闕亦謝坐,儀同上。乃與館伴對行上。押伴在副階上,與使副展參狀。來使副先令人報上聞,押伴回傳示,再揖。請押伴先入,於卓前椅位立。館伴與使副對揖,各就位立,通揖,請
 端笏坐,湯入,乃於拜席上排立都管人從。湯盞出,揖起,押伴等離位立。都管人從鞠躬拜,下節人聲喏,如入見儀。呼「萬歲」,畢,喝:「押伴及使副皆就坐。」引三都管、上中節分左右上,南入,北為上,下節在西廊下立。候押伴等初盞畢,樂聲盡,坐。至三盞下,食畢,四盞下,酒畢。押伴傳示來使,面勸都管、上中節酒一盞,來使答上聞,以都管、上中節於副階下排立,先揖,飲,傳台旨勸,再揖,退。至五盞下,酒畢,茶入。都管人從於拜席上排立,待茶罷,揖押伴等起,離位立,都管人從鞠躬,喝:「謝恩。」拜,下節聲喏如上儀,就位立。請押伴等齊下,赴拜褥對立。先請使副
 就褥位,謝恩,再拜,舞蹈,三拜,復位。乃請押伴、館伴就褥位,謝如上儀,復位。



 第四日,命押宴官、賜宴官就館宴。先賜宴天使轉銜如前儀,各公服,請館伴、天使與來使就褥位對立。先請使副就褥位,望闕立。次請賜宴天使就褥位稍前,使副鞠躬,天使傳宣,使副拜謝,皆如前儀。使副與天使互展狀,起居,揖。次館伴揖,使副令人傳示館伴,依例請賜宴天使茶酒,館伴暫歸幕。來使副與天使主賓對行上,於西間內各詣椅位揖,收笏坐。先湯,次酒三盞,果殽。茶罷,執笏,近前請起,賜宴天使暗退。請押宴使至褥位立,次請館伴齊就褥位,望闕再拜。平身,搢
 笏,鞠躬,三舞蹈,跪左膝三叩頭,出笏就拜,興,再拜復位,對立。請押宴上。次請來使副詣褥位,謝坐,再拜,舞蹈,三拜,請分階升,欄子外,內為上,對立。先館伴揖,次互展押宴起居狀,相見,揖。各傳示,再揖,通揖,請就位,詣椅位立。通揖,請端笏坐,以御宴不敢用踏床。湯入,都管、三節人從於拜席上排立。湯盞出,押宴離位立揖,都管人從鞠躬,下節人從聲喏,呼「萬歲」,如入見儀,喝:「各就坐。」請押宴等坐。引都管、上中節分左右上,北入,南為上,立。下節於西廊下南入,北為上,立。候押宴等初盞畢,樂聲盡,坐。至五盞後食,六盞、七盞雜劇。八盞下,酒畢。押宴傳
 示使副,依例請都管、上中節當面勸酒。使者答上聞,復引都管、上中節於欄子外階下排立,先揖,飲酒,再揖,退。至九盞下,酒畢,教坊退。乃請賜宴天使於幕次前。候茶入,乃於拜席排立都管、三節人從。茶盞出,揖起,押宴官等離位立,揖,都管人從鞠躬,喝:「謝恩。」拜,下節聲喏,呼「萬歲」,如入見儀,齊鞠躬,喝:「各祗候。」請押宴等官齊出,分階下,與天使對行至拜褥前立。請使副就位望闕謝恩,再拜,舞蹈,三拜,畢,依位立。請押宴、館伴齊詣褥位謝恩。來使乃進謝御宴表,先再拜,平身立。使跪捧表,天使近前搢笏受表,出笏復位。使就拜,退復位,立。使副上聞,依
 例書送天使土物,領畢,天使即以物報之,然後展天使辭狀,再揖,次館伴揖,通揖,請分位。是日,來使於宴下監酒等官及教坊人等皆有所贈。



 第五日,稱賀。比至客省幕次對立,皆如入見儀。至收笏坐,先湯,次酒三盞,畢,客省傳示來使,辭曰:「請都管、上中節當面勸酒。」回傳示畢,引都管、上中節於幕次前階下排立,先揖,飲酒,再揖,引退。至三盞酒畢,茶罷,出笏近前,齊請出幕次,前外為上,對立,通揖,分位,各歸幕次。候閣門招引時,請客省與使副幕次前,外為上,對立揖。對行至門外階下,與引揖閣副揖。引使副左入,與臣僚合班,至丹墀北嚮立定。同臣
 僚先再拜,平身,搢笏,鞠躬,三舞蹈,跪左膝三叩頭,出笏就拜,興,再拜,平立。俟進酒致辭畢,再拜,宣徽使稱:「有制。」又再拜,宣答畢,先再拜,舞蹈,平立,分班。俟皇帝舉酒時,再拜,合班又再拜,上殿,夏使副在御座右第二行北端立。次引都管、上中節左入,至丹墀立,下節門外階下排立,齊鞠躬,通名畢,先再拜,鞠躬,不出班奏:「聖躬萬福。」喝:「拜。」又再拜,下節聲喏呼「萬歲」,如前儀。喝:「各祗候。」畢,平立,再鞠躬,喝:「賜酒食。」聲喏再拜呼「萬歲」,如前儀。引左廊立。待床入,進酒。皇帝飲酒時,上下侍立皆再拜。俟進酒官至位,合坐官再拜,皆坐。即行臣使酒,普傳宣,立飲,再拜,
 復坐。次人從鞠躬聲喏再拜呼「萬歲」之儀如前。皆坐。至第三盞,傳宣立飲,畢,再拜,復坐。次人從如前,畢,坐。俟致語,聞鼓笛時,揖臣使皆立,俟口號絕,臣使再拜,坐,次人從如前儀,復坐。次至五盞,將曲終,人從立,再如前儀,畢,先引出。臣使起再拜,退至丹墀,合班,謝宴,再拜,舞蹈,三拜,喝:「各祗候。」引出,至三門階下,與閣門副使相揖別,與客省同行,至幕次前對立,先揖,各傳示,再揖,請分位,就幕次。少頃,請館伴與使副出幕次,外為上,對立,先揖,各傳示,再揖,引行,至元下馬處,請左右易位,對立揖,收笏上馬,至館,聲喏相揖分位,與初入見還禮同。



 第六日,賜
 分食,並賜酒果禮。天使至館,與第二日賜酒果禮同。是日,支押分食酒果軍土物,並在館隨局分官員承應人例物。凡裏外門將軍、監廚直長、館都監、監酒食官、承應班祗候、眾廚子、館子、巡護軍、館伴所牽攏官,皆溥及之。第七日,曲宴禮,如前儀。第八日,奉辭之儀。至小起居畢,閣使先奏來使辭榜子。引使者左入,傍折通班,至丹墀再拜,不出班奏8「聖躬萬福。」又再拜。揖副鞠躬,使出班戀闕致詞,復位,再拜,喝:「各好去。」引右出,次引宰執下殿,禮畢。第九日,聚,送至恩華館,更衣而行。凡使將至界,報至則差接伴使,至則差館伴使,去則差送伴使,皆有副,
 皆差書表以從。凡行省來宴、回宴之押宴官,皆從行省定差,就借以文武高爵長官之職,以為轉銜之光。來回之賜宴天使,皆以閣門祗候往,詔書、口宣皆稟命於都省,以翰林院定撰焉。



 夏使至,或許貿易於市二日。使至,所差者館伴使、副各一,監察、奉職、省令史各一,書表四,總領提控官、酒食官、監廚、稱肉官各一,牽攏官三十,尚食局直長、知書、都管、接手、湯藥直長、長行各一,廚子五,奉飲直長一、長行二,奉珍二,儀鸞直長一、長行十,把內外門官二,館外巡防軍三十,把館甲軍六十二,雜役軍六十,過位不通漢語軍十,凡雜役皆衣皂,過食司吏八
 十,街市廚子四十,方脈雜科醫各一,醫獸一,鞍馬二十四匹,後止備八匹,押馬官一員。又差說儀承受禮直官一員。凡在館鋪陳繳絡器皿什物,戶部差官與東上直閣同點檢。所經橋道皆先期命工部修治之。凡賜衣,使副各三對,人從衣各二對,使副幣帛百四十段,舊又賜貂裘二,無則使者代以銀三錠,副代以帛六十匹,後削之。惟生餼則代以綾羅三十九匹、帛六十二匹、布四匹。金帶三,金鍍銀束帶三,金塗銀鬧裝鞍轡三,金塗銀渾裹書匣、間金塗銀裝釘黑油詔匣及包書、詔匣復各一。朝辭,賜人從銀二百三十五兩,絹二百三十五匹。賜宋、
 高麗使之物,其數則無所考。



\end{pinyinscope}