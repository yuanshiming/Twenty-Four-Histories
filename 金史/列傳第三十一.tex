\article{列傳第三十一}

\begin{pinyinscope}

 ○顯宗諸子章宗諸子衛紹王子
 宣宗三子獨吉思忠承裕僕散揆抹捻史乂搭宗浩



 顯宗孝懿皇后生章宗,昭聖皇后生宣宗,諸姬田氏生鄆王琮、瀛王瑰、霍王從彞,劉氏生瀛王從憲,王氏生溫王玠。



 鄆王琮,本名承慶,母田氏,其後封裕陵充華。琮儀觀豐偉,機警清辯,性寬厚,好學。世宗選進士之有名行者納
 坦謀嘉教之,女直小字及漢字皆通習。及長,輕財好施,無慍色,善吟詠,不喜聞人過,至于騎射繪塑之藝,皆造精妙。大定十八年,封道國公。二十六年,加崇進。章宗即位,遷開府儀同三司,封鄆王。明昌元年,授婆速路獲火羅合打世襲猛安,留京師。五年,薨。上輟朝,親臨奠於殯所。謚曰莊靖,改莊惠。



 瀛王瑰,本名桓篤,鄆王琮之同母弟也。重厚寡言,內行修飭,工詩,精於騎射、書藝、女直大小字。大定二十二年,封崇國公。二十六年,加崇進。章宗即位,遷開府儀同三司,封瀛王。明昌三年,薨。敕葬事所須皆從官給,命工部
 侍郎胥持國等典喪事。比葬,帝三臨奠,哭之慟。謚曰文敬。其後帝謂輔臣曰:「王性忠孝,兄弟中最為善人,故朕嘗令在左右。溫王雖幼,亦佳。不二旬俱逝,良可哀悼。」



 霍王從彞,本名阿憐,母田氏早卒,溫妃石抹氏養為己子。大定二十五年,封宿國公,加崇進。二十六年,賜名瓚。章宗即位,封沂王。明昌元年,諭旨有司曰:「豐、鄆、瀛、沂四王府各賜奴婢七百人。」四年,詔追封故魯王孰輦為趙王,以從彞為趙王後。承安元年,為兵部尚書,改封蔡。四年,除秘書監。泰和五年,賜今名。八年,封霍。貞祐二年,薨。



 瀛王從憲,本名吾里不,母劉氏,後封裕陵茂儀。大定二
 十六年,賜名琦。章宗即位,加開府儀同三司,封壽王。承安元年,以郊祀恩進封英。四年,改封瀛。泰和五年,更賜今名。六年,授祕書監。八年,薨。



 從憲風儀秀峙,性寬厚,善騎射,待府僚以禮,秩滿去者皆有贐。帝尤愛重,初以病聞,即臨問之,賜錢五百萬。還宮,詔府僚上其疾增損狀,仍敕門司夜一鼓即奏,比五更重言之。及薨,上哭之慟,為輟朝臨奠者再。諭旨判大睦親府事宛王永升曰:「瀛王家事,叔宜規畫。聞其二姬方孕,若生子,即以付之。」以右宣徽使移剌都護其喪葬,斂以內庫之服,其餘所須,亦從官給。謚曰敦懿。



 溫王玠,本名謀良虎,母王氏,後封裕陵婉儀。玠幼穎秀,性溫厚,好學。大定二十九年,章宗即位,加開府儀同三司,封溫王。明昌三年,薨,年十一。訃聞,上為輟朝,親臨奠哭之。謚曰悼敏。



 章宗欽懷皇后生絳王洪裕,資明夫人林氏生荊王洪靖,諸姬生榮王洪熙、英王洪衍、壽王洪輝。元妃李氏生葛王忒鄰。



 洪裕,大定二十六年生。是時顯宗薨逾年,世宗深感,及聞皇曾孫生,喜甚。滿三月,宴于慶和殿,賜曾孫金鼎,金香合,重彩二十端,骨睹犀、吐鶻玉山子、兔兒垂頭一副,
 名馬二匹。章宗進玉雙駝鎮紙、玉琵琶撥、玉鳳鉤、骨睹犀具佩刀、衣服一襲。世宗御酒歌歡,乙夜方罷。二十八年十月丙寅,薨。明昌三年,追封絳王,賜名。



 洪靖,本名阿虎懶,明昌三年生。生而警秀,上所鍾愛。四年,薨。承安四年,追封荊王,賜名,加開府儀同三司。



 洪熙,本名訛魯不,明昌三年生,未彌月薨。承安四年,追封榮王,賜名,加開府儀同三司。



 洪衍,本名撒改,明昌四年生,未幾薨。承安四年,追封英王,賜名,加開府儀同三司。



 洪輝,本名訛論,承安二年五月生,彌月,封壽王。閏六月
 壬午,病急風,募能醫者加宣武將軍,賜錢五百萬。甲申,疾愈,印《無量壽經》一萬卷報謝,衍慶宮作普天大醮七日,無奏刑名,仍禁屠宰。十月丁亥,薨,備禮葬。



 忒鄰,泰和二年八月生。上久無皇嗣,祈禱于郊、廟、衍慶宮、亳州太清宮,至是喜甚。彌月,將加封,三等國號無愜上意者,念世宗在位最久,年最高,初封葛王,遂封為葛王。十二月癸酉,生滿百日,放僧道度牒三千道,設醮玄真觀,宴于慶和殿。百官用天壽節禮儀,進酒稱賀,三品以上進禮物。泰和三年,薨。



 衛紹王六子,大定二十六年,賜名猛安曰琚,按出曰瑄,
 按辰曰璪。泰和七年,詔按辰出繼鄭王永蹈後,詔曰:「朕追惟鄭邸,誤蹈非彞,槁窆原野,多歷歲年,怛然軫懷,有不能已,乃詔追復王爵,備禮改葬。今稽式古典,命汝為鄭王後,守其祭祀。」大安元年,封子六人為王,從恪胙王,有任王、鞏王,餘弗傳。是歲,從恪為左丞相。二年八月,立從恪為皇太子。至寧末,胡沙虎殺衛王,從恪兄弟皆廢居中都。貞祐二年,徙鄭州。四年,徙居南京。天興元年,崔立以從恪為梁王,汴京破,死焉。



 贊曰:章宗晚年,繼嗣不立,遂屬意衛紹王。衛紹歷年不永,諸子凡禁錮二十餘年,鎬厲王諸子禁錮四十餘年,
 長女鰥男皆不得婚嫁。天興初,方弛其禁,金亡祚後可知矣。



 莊獻太子,名守忠,宣宗長子也。其母未詳,說在《王后傳》。胡沙虎既廢衛王,時上未至,迎守忠入居東宮。貞祐元年閏九月甲申,立為皇太子,詔曰:「朕以眇躬,嗣服景命,念祖宗之遺統,方夙夜以靡遑,將上以承九廟之靈,而下以係多方之望。皇太子守忠性秉溫良,地居長嫡,以次第言之,則宜升儲嗣,以典禮質之,則足愜群情,其立為皇太子。」十月己未,以鎮國上將軍。太子少保阿魯罕為太子少師。庚申,上遣諭曰:「朕宮中每事裁減,汝亦
 宜知時難,斟酌撙節也。」又謂曰:「時方多艱,每事當從貶損,吾已放宮人百餘矣,東宮無用者亦宜出之。汝讀書人,必能知此也。」二年四月,宣宗遷汴,留守中京。七月,召至汴。三年正月,薨。上臨奠殯所凡四次。四月,葬迎朔門外五里。謚莊獻。五月,立其子鏗為皇太孫,始二歲。十二月薨,四年正月,賜謚沖懷太孫。



 玄齡,或曰莊獻太子母弟,早卒,未封爵。或曰麗妃史氏所生。



 荊王守純,本名盤都,宣宗第二子也。母曰真妃龐氏。貞祐元年,封濮王。二年,為殿前都點檢兼侍衛親軍都指
 揮使,權都元帥。上諭帥府曰:「濮王年幼,公事殊未諳,卿等毋以朕子故不相規戒。凡見將校,令謙和接遇可也。」三年,為樞密使。四年,拜平章政事。興定元年,授世襲東平府路三屯猛安。三年,以知管差除令史梁



 瓛,誤書轉運副使張正倫宣命,奏乞治罪。上曰:「令史有犯,宰臣自當治之,何必關朕耶?」是年三月,進封英王。時監察御史程震言其不法,宣宗切責,杖司馬及大奴尤不法者數人。四年九月,守純欲發丞相高琪罪,密召知案蒲鮮石魯剌、令史蒲察胡魯、員外郎王阿里謀之,且屬令勿泄,而石魯剌、胡魯輒以告都事僕散奴失不,奴失不白高
 琪。及高琪伏誅,守純劾三人者泄密事,奴失不免死,除名,石魯剌、胡魯各杖七十,勒停。



 元光二年三月壬子,上戒諭守純曰:「始吾以汝為相者,庶幾相輔,不至為人譏病耳。汝乃惟飲酒耽樂,公事漫不加省,何耶?吾常聞人言己過,雖自省無之,亦未敢容易去懷也。」又曰:「吾所以責汝者,但以崇飲不事事之故,汝勿過慮,遂至奪權。今諸相皆老臣,每事與之商略,使無貽物議足矣。」



 是年十二月庚寅,宣宗病喉痺,危篤,將夕,守純趣入侍。哀宗後至,東華門已閉,聞守純在宮,分遣樞密院官及東宮親衛軍總領移剌蒲阿集軍三萬餘屯東華門外。部署定,
 扣門求見。都點檢駙馬都尉徒單合住奏中宮,得旨,領符鑰開門。哀宗入,宰相把胡魯已遣人止丞相高汝礪,不聽入宮,以護衛四人監守純於近侍局。是夕,宣宗崩。明日,哀宗即位。



 正大元年正月,進封荊王,罷平章政事、判睦親府,封真妃龐氏為荊國太妃,三月,或告守純謀不軌,下獄推問。慈聖宮皇太后有言於帝,由是獲免,語在《皇后傳》。守純三子,長曰訛可,封肅國公,天興元年三月進封曹王,出質於軍前。次曰某,封戴王。次曰孛德,封鞏王。



 天興初,守純府第產肉芝一株,高五寸許,色紅鮮可愛,既而枝葉津流,濡地成血,臭不可聞,鏟去復生者
 再。夜則房榻間群狐號鳴,秉燭逐捕則失所在。未幾,訛可出質,哀宗遷歸德。明年正月,崔立亂。四月癸巳,守純及諸宗室皆死青城。



 贊曰:《詩》云:「天難忱斯,不易維王,天位殷適,使不挾四方。」信哉!守忠立為太子,未幾而薨,其子鏗立,又薨,哀宗復乏嗣,豈非天乎。正大間,國勢日蹙,本支殆盡,哀宗尚且疏忌骨肉,非明惠之賢,荊王幾不能免,豈「宗子維城」之道哉!



 獨吉思忠,本名千家奴。明昌六年,為行省都事,累遷同簽樞密院事。承安三年,除興平軍節度使,改西北路招
 討使。初,大定間修築西北屯戍,西自坦舌,東至胡烈麼,幾六百里。中間堡障,工役促迫,雖有墻隍,無女墻副堤。思忠增繕,用工七十五萬,止用屯戍軍卒,役不及民。上嘉其勞,賜詔獎諭曰:「直乾之維,扼邊之要,正資守備,以靖翰籓,垣壘弗完,營屯未固。卿督茲事役,唯用戍兵,民不知勞,時非淹久,已臻休畢,仍底工堅。賴爾忠勤,辦茲心畫,有嘉乃力,式副予懷。」賜銀五百兩、重幣十端。入為簽樞密院事,轉吏部尚書,拜參知政事。



 泰和五年,宋渝盟有端,平章政事僕散揆宣撫河南。揆奏宋人懦弱,韓侂胄用事,請遣使詰問。上召大臣議。左丞相宗浩曰:「宋
 久敗之國,必不敢動。」思忠曰:「宋雖羈棲江表,未嘗一日忘中國,但力不足耳。」其後果如思忠策。六年四月,上召大臣議伐宋事,大臣猶言無足慮者。或曰:「鼠竊狗盜,非用兵也。」思忠執前議曰:「不早為之所,彼將誤也。」上深然之。



 七年正月,元帥左監軍紇石烈執中圍楚州,久不能下,宰臣奏請命大臣節制其軍,及益兵攻之。思忠請行。上曰:「以執政將兵攻一小州,克之亦不武。」乃用唐宰相宣慰諸軍故事,以思忠充淮南宣慰使,持空名宣敕賞立功者。詔大臣宿于秘書監,各具奏帖以聞。明日,詔百官集議于廣仁殿,問對者久之。既而宋人來請和,議遂
 寢。



 頃之,進拜尚書右丞。大安初,拜平章政事。三年,與參知政事承裕將兵屯邊,方繕完烏沙堡,思忠等不設備,大元前兵奄至,取烏月營,思忠不能守,乃退兵,思忠坐解職。衛紹王命參知政事承裕行省,既而敗績于會河堡云。



 承裕,本名胡沙,頗讀孫、吳書,以宗室子充符寶祗候。除中都左警巡副使,通括戶籍,百姓稱其平。遷殿中侍御史,改右警巡使、彰德軍節度副使、刑部員外郎,轉本部郎中。歷會州、惠州刺史、遷同知臨潢府事,改東北路招討副使。以病免,起為西南招討副使。



 泰和六年,伐宋,遷
 陜西路統軍副使,俄改通遠軍節度使、陜西兵馬都統副使,與秦州防禦使完顏璘屯成紀界。宋吳曦兵五萬由保岔、姑蘇等谷襲秦州,承裕、璘以騎兵千餘人擊走之,追奔四十里,凡六戰,宋兵大敗,斬首四千餘級。詔承裕曰:「昔乃祖乃父,戮力戎旅,汝年尚少,善於其職,故命汝與完顏璘同行出界。昔汝自言得兵三萬足以辦事,今以石抹仲溫、術虎高琪及青宜可與汝軍相合,計可六萬,斯亦足以辦矣。仲溫、高琪兵道險阻,汝兵道甚易也。自秦州至仙人關纔四百里耳,從長計畫,以副朕意。」詔完顏璘曰:「汝向在北邊,以幹勇見稱,頃以過失,逮問
 有司。近知與宋人奮戰,故特赦免,仍充副統,如能佐承裕立功業,朕於官賞,豈復吝惜。聞汝臨事頗黠,若復自速罪,且不赦汝矣。」宋吳曦使其將馮興、楊雄、李珪以步騎八千入赤谷,承裕、璘及河州防禦使蒲察秉鉉逆擊破之。宋步兵保西山,騎兵走赤谷。承裕遣部將唐括按答海率騎二百馳擊宋步兵,甲士蒙括挺身先入乘之,宋步兵大潰。追奔至皂郊城,斬二千餘級。猛安把添奴追宋騎兵,殺千餘人,斬楊雄、李珪于陣,馮興僅以身免。承裕進兵,克成州。



 八年,罷兵,遷河南東路統軍使,兼知歸德府事,俄改知臨潢府事。賜金帶、重幣十端、銀百五
 十兩。大安初,召為御史中丞。三年,拜參知政事,與平章政事獨吉思忠行省戍邊。烏沙堡之役,不為備,失利,朝廷獨坐思忠,詔承裕主兵事。



 八月,大元大兵至野狐嶺,承裕喪氣,不敢拒戰,退至宣平。縣中土豪請以土兵為前鋒,以行省兵為聲援,承裕畏怯不敢用,但問此去宣德間道而已。土豪嗤之曰:「溪澗曲折,我輩諳知之。行省不知用地利力戰,但謀走耳,今敗矣。」其夜,承裕率兵南行,大元兵踵擊之。明日,至會河川,承裕兵大潰。承裕僅脫身,走入宣德。大元游兵入居庸關,中都戒嚴。識者謂金之亡,決於是役。衛紹王猶薄其罪,除名而已。崇慶元
 年,起為陜西安撫使。至寧元年,遷元帥右監軍,兼咸平府路兵馬都總管,與契丹留可戰,敗績。改同判大睦親府事、遼東宣撫使。貞祐初,改臨海軍節度使,卒。



 贊曰:曹劌有言:「一鼓作氣,再而衰,三而竭。」夫兵以氣為主,會河堡之役,獨吉思忠、承裕沮喪不可復振,金之亡國,兆於此焉。



 僕散揆,本名臨喜,其先上京人,左丞相兼都元帥沂國武莊公忠義之子也。少以世胄,選為近侍奉御。大定十五年,尚韓國大長公主,擢器物局副使,特授臨潢府路赫沙阿世襲猛安。歷近侍局副使、尚衣局使、拱衛直副
 都指揮使,為殿前左衛將軍。罷職,世宗諭之曰:「以汝宣獻皇后之親,故令尚主,置之宿衛,謂當以忠孝自勵。日者乃與外人竊議,汝腹中事,朕不能測,其罷歸田里。」尋起為濼州刺史,改蠡州,入為兵部侍郎、大理卿、刑部尚書。



 章宗即位,出為泰定軍節度使,改知臨洮府事。以政績聞。升河南路統軍使。陜西提刑司舉揆「剛直明斷,獄無冤滯。禁戢家人,百姓莫識其面。積石、洮二州舊寇皆遁,商旅得通」。於是進官一階,仍詔褒諭。



 明昌四年,鄭王永蹈謀逆,事覺,揆坐嘗私品藻諸王,獨稱永蹈性善,靜不好事,乃免死,除名。未幾,復五品階,起為同知崇義軍節
 度使事。以戰功遷西北路副招討,進官七階,賜金馬盂一、銀二百兩、重彩一十端。復以戰功升西南路招討使兼天德軍節度使,賜金五十兩、重彩一十端。復出禦邊,當轉戰出塞七百里,至赤胡睹地而還。優詔褒諭,遷一官,仍許其子安貞尚邢國長公主,且許揆入謝,禮成,歸鎮。



 會韓國大長公主薨,揆來赴,上諭之曰:「北邊之事,非卿不能辦。」乃賜戰馬二,即日遣還。揆沿徼築壘穿塹,連亙九百里,營柵相望,烽候相應,人得恣田牧,北邊遂寧。復以手詔褒諭,且欲大用,以知興中府事紇石烈子仁代之,敕盡以方略授子仁。既入,拜參知政事,改授中都
 路胡土愛割蠻世襲猛安。進拜尚書右丞。尋出經略邊事,還拜平章政事,封濟國公。



 泰和五年,宋人渝盟,以揆為宣撫河南軍民使。上諭之曰:「朕即位以來,任宰相未有如卿之久者,若非君臣道合,一體同心,何以及此。先丞相亦嘗總師南邊,效力先朝,今復委卿,諒無過舉。朕非好大喜功,務要寧靜內外。宋人屈服,無復可議,若恬不改,可整兵渡淮,掃蕩江左,以繼爾先公之功。」即以尚廄名馬、玉束帶、內府重彩及御藥賜之。揆至汴,搜練將士,軍聲大振。會天壽節,特遣其子安貞賜宴。且命持白玉杯以飲揆,及上秋獵所親獲鹿尾舌為賜。宋人服罪,
 即罷宣撫使,召揆還。



 六年春,宋人復數路來侵,取泗州,取靈璧,圍壽春。命揆為左副元帥以討之。揆至軍前,集諸將校告以朝廷弔伐之意,分遣將士禦敵。復取臨淮、蘄縣,而符離、壽春之圍亦解去,敵屢敗衄,悉遁出境。上即遣提點近侍局烏古論慶壽持手詔勞問征討事宜,仍賜玉具劍一、玉荷蓮盞一、金器一百兩、重彩一十端。尋復以詔褒諭,賜玉鞍勒馬二及玉具佩刀、內府重彩、御藥,以旌其功。



 宋人既敗退,上欲進討,乃召揆赴闕,戒以師期,宴於慶和殿,親諭之曰:「朕以趙擴背盟,侵我疆埸,命卿措畫。曾未期月,諸處累報大捷。振我國威,挫彼
 賊鋒,皆卿之力,朕不能忘。」是日寵錫甚厚,特收其次子寧壽為奉御,乃密授以成算,俾還軍。



 十月,揆總大軍南伐,分兵為九路進。揆以行省兵三萬出潁、壽,至淮,宋人旅拒于水南。揆密遣人測淮水,惟八疊灘可涉,即遣奧屯驤揚兵下蔡,聲言欲渡。宋帥何汝礪、姚公佐悉銳師屯花靨以備。揆乃遣右翼都統完顏賽不、先鋒都統納蘭邦烈潛渡八疊,駐南岸。揆麾大軍直壓其陣。敵不虞我卒至,皆潰走,自相蹂踐,死於水者不可勝計。進奪潁口,下安豐軍,遂攻合肥,取滁州,盡獲其軍實。上遣使諭之曰:「前得卿奏,先鋒已奪潁口,偏師又下安豐,斬馘
 之數,各以萬計。近又西帥奏捷,棗陽、光化既為我有,樊城、鄧城亦自潰散。又聞隨州闔城歸順,山東之眾久圍楚州,隴右之師剋期出界。卿提大兵攻合肥,趙擴聞之,料已破膽,失其神守。度彼之計,乞和為上,昔嘗畫三事付卿,以今事勢計之,徑渡長江,亦其時矣。淮南既為我有,際江為界,理所宜然。如使趙擴奉表稱臣,歲增貢幣,縛送賊魁,還所俘掠,一如所諭,亦可罷兵。卿宜廣為渡江之勢,使彼有必死之憂,從其所請而縱之,僅得餘息偷生,豈敢復萌他慮。卿於此時,經營江北,勞徠安集,除其虐政橫賦,以良吏撫字疲民,以精兵分守要害,雖未
 係趙擴之頸,而朕前所畫三事,上功已成矣。前入見時,已嘗議定,今復諄諄者,欲決卿成功爾。機會難遇,卿其勉之。」



 既而宋帥丘灊果奉書乞和,揆以前五事諭而遣之。復進軍圍和州,敵以騎萬五千駐六合,揆偵知之,即以右翼掩擊,斬首八千級,進屯於瓦梁河,以控真、揚諸路之衝。乃整列軍騎,畢張旗幟,沿江上下,皆金兵焉。於是江表震恐。宋真州兵數萬保河橋,復遣統軍紇石烈子仁往攻之,分軍涉淺,潛出敵後。敵見之大驚,不戰而潰,斬首二萬餘級,生擒其帥劉侹、常思敬、蕭從德、莫子容,皆宋驍將也。遂下真州。宋復遣陳璧來告和,揆以乞
 辭未誠,徒欲緩師,欲之。宋人既喪敗,不獲請成,乃決巨勝、成公、雷塘渚積水以為阻,盡焚其廬舍儲積,過江遁去。



 揆以方春地濕,不可久留,且欲休養士馬,遂振旅而還。次下蔡,遇疾。詔遣宣徽使李仁惠及其子寧壽引太醫診視,仍遣中使撫問。泰和七年二月,薨。訃聞,上哀悼之,輟朝,遣使迎喪殯于都城之北。百官會弔,車駕臨奠哭之,賻銀一千五百兩、重幣五十端、絹五百疋,其葬祭物皆從官給。謚曰武肅。



 揆體剛內和,與物無忤,臨民有惠政。其為將也,軍門鎮靜,嘗罰必行。初渡淮,即命徹去浮梁。所至皆因糧于敵,無餽運之勞。未嘗輕用士卒,而
 與之同甘苦,人亦樂為之用。故南征北伐,為一代名將云。



 抹捻史乂搭,臨潢路人也。其先以功授世襲謀克。史乂搭幼襲爵,守邊有勞。泰和六年,南鄙用兵,授同知蔡州防禦使事。



 五月,宋將李爽圍壽州,田俊邁陷蘄縣,平章政事僕散揆謂諸將曰:「符離、彭城,齊魯之蔽,符離不守,是無彭城,彭城陷則齊魯危矣。」乃遣安國軍節度副使納蘭邦烈與史乂搭以精騎三千戍宿州。俊邁果率步騎二萬來襲,邦烈、史乂搭逆擊,大破之。邦烈中流矢。宋郭倬、李汝翼以眾五萬繼至,遂圍城,攻之甚力,城中叢射,
 敵不能逼。會淫雨潦溢,敵露處勞倦,邦烈遣騎二百潛出敵後突擊之。敵亂,史乂搭率騎蹂之,殺傷數千人。敵復聞援軍將至,遂夜遁。邦烈、史乂搭躡其後,黎明合擊,大破之,獲田俊邁。十月,揆以行省兵三萬出潁、壽,史乂搭為驍騎將中軍副統,克安豐軍,戰霍丘、花靨,功居多。十二月,從攻和州,中流矢卒。



 史乂搭形不過中人,而拳勇善鬥,所用槍長二丈,軍中號為「長槍副統」。又工用手箭,箭長不盈握,每用百數,散置鎧中,遇敵抽箭,以鞭揮之,或以指鉗取飛擲,數矢齊發,無不中,敵以為神。其箭皆以智創,雖子弟亦不能傳其法。在北部守厭山營,敵
 尤畏之,不敢近。及死,將士皆惋惜之。



 內族宗浩,字師孟,本名老,照祖四世孫,太保兼都元帥漢國公昂之子也。貞元中,為海陵庶人入殿小底。世宗即位遼陽,昂遣宗浩馳賀。世宗見之喜,命充符寶祗候。大定二年冬,昂以都元帥置幕山東,宗浩領萬戶從行,仍授山東東路兵馬都總管判官。丁父憂,起復,承襲因閔斡魯渾猛安,授河南府判官。以母喪解,服闋,授同知陜州防禦使事。察廉能第一等,進官一階,陞同知彰化軍節度使事,累遷同簽樞密院事,改曷蘇館節度使。



 世宗謂宰臣曰:「宗浩有才幹,可及者無幾。」二十三年,徵為
 大理卿,踰年授山東路統軍使,兼知益都府事。陛辭,世宗諭之曰:「卿年尚少,以卿近屬,有治迹,故以此授卿,宜體朕意。」因賜金帶遣之。二十六年,為賜宋主趙甗生日使。還,授刑部尚書,俄拜參知政事。



 章宗即位,出為北京留守,三轉同判大睦親府事。北方有警,命宗浩佩金虎符駐泰州便宜從事。朝廷發上京等路軍萬人以戍。宗浩以糧儲未備,且度敵未敢動,遂分其軍就食隆、肇間。是冬,果無警。北部廣吉剌者尤桀驁,屢脅諸部入塞。宗浩請乘其春暮馬弱擊之。時阻珝亦叛,內族襄行省事于北京,詔議其事。襄以謂若攻破廣吉剌,則阻珝無東
 顧憂,不若留之,以牽其勢。宗浩奏:「國家以堂堂之勢,不能掃滅小部,顧欲藉彼為捍乎?臣請先破廣吉剌,然後提兵北滅阻珝。」章再上,從之。詔諭宗浩曰:「將征北部,固卿之誠,更宜加意,毋致後悔。」宗浩覘知合底忻與婆速火等相結,廣吉剌之勢必分,彼既畏我見討,而復掣肘仇敵,則理必求降,可呼致也。因遣主簿撒領軍二百為先鋒,戒之曰:「若廣吉剌降,可就徵其兵以圖合底忻,仍偵餘部所在,速使來報,大軍當進,與汝擊破之必矣。」合底忻者,與山只昆皆北方別部,恃強中立,無所羈屬,往來阻珝、廣吉剌間,連歲擾邊,皆二部為之也。撒入敵境,
 廣吉剌果降,遂徵其兵萬四千騎,馳報以待。



 宗浩北進,命人齎三十日糧,報撒會于移米河共擊敵,而所遣人誤入婆速火部,由是東軍失期。宗浩前軍至忒里葛山,遇山只昆所統石魯、渾灘兩部,擊走之,斬首千二百級,俘生口車畜甚眾。進至呼歇水,敵勢大蹙,於是合底忻部長白古帶、山只昆部長胡必剌及婆速火所遣和火者皆乞降。宗浩承詔,諭而釋之。胡必剌因言,所部迪列土近在移米河不肯偕降,乞討之。乃移軍趨移米,與迪列土遇,擊之,斬首三百級,赴水死者十四五,獲牛羊萬二千,車帳稱是。合底忻等恐大軍至,西渡移米,棄輜重
 遁去。撒與廣吉剌部長忒里虎追躡及之,於窊里不水縱擊大破之。婆速火九部斬首、溺水死者四千五百餘人,獲駝馬牛羊不可勝計。軍還,婆速火乞內屬,并請置吏。上優詔褒諭,遷光祿大夫,以所獲馬六千置牧以處之。明年,宴賜東北部,尋拜樞密使,封榮國公。初,朝廷置東北路招討司泰州,去境三百里,每敵入,比出兵追襲,敵已遁去。至是,宗浩奏徙之金山,以據要害,設副招討二員,分置左右,由是敵不敢犯。



 會中都、山東、河北屯駐軍人地土不贍,官田多為民所冒占,命宗浩行省事,詣諸道括籍,凡得地三十餘萬頃。還,坐以倡女自隨,為憲
 司所糾,出知真定府事。徙西京留守,復為樞密使,進拜尚書右丞相,超授崇進。時懲北邊不寧,議築壕壘以備守戍,廷臣多異同。平章政事張萬公力言其不可,宗浩獨謂便,乃命宗浩行省事,以督其役。功畢,上賜詔褒賚甚厚。撒里部長陀括里入塞,宗浩以兵追躡,與僕散揆軍合擊之,殺獲甚眾,敵遁去。詔徵還,入見,優詔獎諭,躐遷儀同三司,賜玉束帶一、金器百兩、重幣二十端,進拜左丞相。



 宋人畔盟,王師南伐,會平章政事揆病,乃命宗浩兼都元帥往督進討。宗浩馳至汴,大張兵勢,親赴襄陽巡師而還。宋人大懼,乃命知樞密院事張巖以書乞
 和。宗浩以辭旨未順卻之,仍諭以稱臣、割地、縛送元謀姦臣等事。巖復遣方信孺齎其主趙擴誓稿來,且言擴併發三使,將賀天壽節及通謝,仍報其祖母謝氏殂,致書于都元帥宗浩曰:



 方信孺還,遠貽報翰及所承鈞旨,仰見以生靈休息為重,曲示包容矜軫之意。聞命踴躍,私竊自喜,即具奏聞,備述大金皇帝天覆地載之仁,與都元帥海涵春育之德。旋奉上旨,亟遣信使通謝宸庭,仍先令信孺再詣行省,以請定議。區區之愚,實恃高明,必蒙洞照,重布本末,幸垂聽焉。



 兵端之開,雖本朝失於輕信,然痛罪姦臣之蔽欺,亦不為不早。自去歲五月,編
 竄鄧友龍,六月又誅蘇師旦等。是時大國尚未嘗一出兵也,本朝即捐已得之泗州,諸軍屯于境外者盡令徹戍而南,悔艾之誠,於茲可見。惟是名分之諭,今昔事殊,本朝皇帝本無佳兵之意,況關繫至重,又豈臣子之所敢言?



 江外之地,恃為屏蔽,儻如來諭,何以為國?大朝所當念察。至于首事人鄧友龍等誤國之罪,固無所逃,若使執縛以送,是本朝不得自致其罰于臣下。所有歲幣,前書已增大定所減之數,此在上國,初何足以為重輕,特欲藉手以見謝過之實。儻上國諒此至情,物之多寡,必不深計。矧惟兵興以來,連歲創殘,賦入屢蠲,若又重
 取于民,豈基元元無窮之困,竊計大朝亦必有所不忍也。於通謝禮幣之外,別致微誠,庶幾以此易彼。



 其歸投之人,皆雀鼠偷生,一時竄匿,往往不知存亡,本朝既無所用,豈以去來為意。當隆興時,固有大朝名族貴將南來者,洎和議之定,亦嘗約各不取索,況茲瑣瑣,誠何足云。儻大朝必欲追求,尚容拘刷。至如泗州等處驅掠人,悉當護送歸業。



 夫締新好者不念舊惡,成大功者不較小利。欲望力賜開陳,捐棄前過,闊略他事,玉帛交馳,歡好如初,海內寧謐,長無軍兵之事。功烈昭宣,德澤洋溢,鼎彞所紀,方冊所載,垂之萬世,豈有既乎!重惟大金皇
 帝誕節將臨,禮當修賀,兼之本國多故,又言合遣人使,接續津發,已具公移,企望取接。伏冀鑒其至再至三有加無已之誠,亟踐請盟之諾,即底于成,感戴恩德永永無極。誓書副本慮往復遷延,就以錄呈。



 初,信孺之來,自以和議遂成,輒自稱通謝使所參議官。大定中,宋人乞和,以王抃為通問使所參議官,信孺援以為例。宗浩怒其輕妄,囚之以聞。朝廷亦以其為行人而不能孚兩國之情,將留之,遣使問宗浩。宗浩曰:「今信孺事既未集,自知還必得罪,拘之適使他日有以藉口。不若數其恌易,而釋遣之使歸,自窮無辭以白其國人,則擴、侂胄必擇
 謹厚者來矣。」於是遣之,而復張巖書曰:



 方信孺重以書來,詳味其辭,於請和之意雖若婉遜,而所畫之事猶未悉從,惟言當還泗州等驅掠而已。至於責貢幣,則欲以舊數為增,追叛亡,則欲以橫恩為例,而稱臣、割地、縛送姦臣三事,則並飾虛說,弗肯如約。豈以為朝廷過求有不可從,將度德量力,足以背城借一,與我軍角一日勝負者哉?既不能強,又不能弱,不深思熟慮以計將來之利害,徒以不情之語形于尺牘而勤郵傳,何也?



 兵者凶器,佳之不祥,然聖人不得已而用之,故三皇、五帝所不能免。夫豈不以生靈為念,蓋犯順負義有不可恕者。乃
 者彼國犯盟,侵我疆埸,帥府奉命征討,雖未及出師,姑以逐處戍兵,隨宜捍禦,所向摧破,莫之敢當,執俘折馘,不可勝計,餘眾震懾靡然奔潰。是以所侵疆土,旋即底平,爰及泗州,亦不勞而復。今乃自謂捐其已得,斂軍徹戍,以為悔過之效,是豈誠實之言!據陜西宣撫司申報,今夏宋人犯邊者十餘次,並為我軍擊退,梟斬捕獲,蓋以億計。夫以悔艾罪咎,移書往來丐和之間,乃暗遣賊徒突我守圉,冀乘其不虞,以徼倖毫末,然則所為來請和者,理安在哉!



 其言名分之諭,今昔事殊者,蓋與大定之事固殊矣。本朝之於宋國,恩深德厚,莫可殫述,皇統
 謝章,可概見也。至於世宗皇帝俯就和好,三十年間恩澤之渥,夫豈可忘?江表舊臣於我,大定之初,以失在正隆,致南服不定,故特施大惠,易為姪國,以鎮撫之。今以小犯大,曲在於彼,既以絕大定之好,則復舊稱臣,於理為宜。若為非臣子所敢言,在皇統時何故敢言而今獨不敢,是又誠然乎哉!又謂江外之地將為屏蔽,割之則無以為國。夫籓籬之固,當守信義,如不務此,雖長江之險,亦不可恃,區區兩淮之地,何足屏蔽而為國哉!昔江左六朝之時,淮南屢嘗屬中國矣。至後周顯德間,南唐李景獻廬、舒、蘄、黃,畫江為界,是亦皆能為國。既有如此
 故實,則割地之事,亦奚不可!



 自我師出疆,所下州軍縣鎮已為我有,未下者即當割而獻之。今方信孺齎到誓書,乃云疆界並依大國皇統、彼之隆興年已畫為定,若是則既不言割彼之地,又翻欲得我之已有者,豈理也哉!又來書云通謝禮幣之外,別備錢一百萬貫,折金銀各三萬兩,專以塞再增幣之責,又云歲幣添五萬兩疋,其言無可準。況和議未定,輒前具載約,擬為誓書,又直報通謝等三番人使,其自專如是,豈協禮體。此方信孺以求成自任,臆度上國,謂如此徑往,則事必可集,輕瀆誑紿,理不可容。



 尋具奏聞,欽奉聖訓:「昔宣、靖之際,棄信
 背盟,我師問罪,嘗割三鎮以乞和。今既無故興兵,蔑棄信誓,雖盡獻江、淮之地,猶不足以自贖。況彼國嘗自言,叔父姪子與君臣父子略不相遠,如能依應稱臣,即許以江、淮之間取中為界。如欲世為子國,即當盡割淮南,直以大江為界。陜西邊面並以大軍已占為定據。元謀姦臣必使縛送,緣彼懇欲自致其罰,可令函首以獻。外歲幣雖添五萬兩疋,止是復皇統舊額而已,安得為增?可令更添五萬兩疋,以表悔謝之實。向汴陽乞和時嘗進賞軍之物,金五百萬兩、銀五千萬、表段裏絹各一百萬、牛馬騾各一萬、駝一千、書五監。今即江表一隅之地。
 與昔不同,特加矜憫,止令量輸銀一千萬兩以充犒軍之用。方信孺言語反復不足取信,如李大性、朱致知、李璧、吳琯輩似乎忠實,可遣詣軍前稟議。據方信孺詭詐之罪,過於胡昉,然自古兵交,使人容在其間,姑放令回報。」伏遇主上聖德寬裕光大,天覆地容,包荒宥罪,其可不欽承以仰副仁恩之厚!儻猶有所稽違,則和好之事,勿復冀也。夫宋國之安危存亡,將繫于此,更期審慮,無貽後悔!



 泰和七年九月,薨於汴。其後宋人竟請以叔為伯,增歲幣,備犒軍銀,函姦臣韓侂胄、蘇師旦首以獻而乞盟焉。訃聞,上震悼,輟朝,命其子宿直將軍天下奴奔
 赴喪所,仍命葬畢持繪像至都,將親臨奠。以南京副留守張巖叟為敕祭兼發引使,莒州刺史女奚列孛葛速為敕葬使,仍摘軍前武士及旗鼓笛角各五十人,外隨行親屬官員親軍送至葬所,賻贈甚厚。謚曰通敏。



 贊曰:金自宗弼渡江而還,既而畫淮為界。厥後海陵咈眾舉兵,國用虛耗,上下離心,內難先作。故世宗之初,章宗之末,有事於南,皆非得已,而詳問之使每先發焉。侂胄狂謀誤國,動非其時,取敗宜也。揆、宗浩雖師出輒捷,而行成之使,不拒其來。儀幣書辭,抑揚增損之際,有可藉口,即許其平矣。函首之事,宋人亦欲因是以自除其
 禍耳。雖然,揆、宗浩常勝之家,史乂搭驍勇之將,三人相繼而死,和議亦成,天意蓋已休息南北之人歟?



\end{pinyinscope}