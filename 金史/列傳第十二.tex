\article{列傳第十二}

\begin{pinyinscope}

 ○宗翰本名粘罕子斜哥宗望本名斡離不子齊京文



 宗翰本名粘沒喝,漢語訛為粘罕,國相撒改之長子也。年十七,軍中服其勇。及議伐遼,宗翰與太祖意合。太祖敗遼師于境上,獲耶律謝十。撒改使宗翰及完顏希尹來賀捷,即稱帝為賀。及太宗以下宗室群臣皆勸進,太祖猶謙讓。宗翰與阿離合懣、蒲家奴等進曰:「若不以時
 建號,無以系天下心。」太祖意乃決。遼都統耶律訛里朵以二十餘萬戍邊,太祖逆擊之,宗翰為右軍,大敗遼人於達魯古城。



 天輔五年四月,宗翰奏曰:「遼主失德,中外離心。我朝興師,大業既定,而根本弗除,後必為患。今乘其釁,可襲取之。天時人事,不可失也。」太祖然之,即命諸路戒備軍事。五月戊戌,射柳,宴群臣。上顧謂宗翰曰:「今議西征,汝前後計議多合朕意。宗室中雖有長於汝者,若謀元帥,無以易汝。汝當治兵,以俟師期。」上親酌酒飲之,且命之釂,解御衣以衣之。群臣言時方暑月,乃止。無何,為移賚勃極烈,副蒲家奴西襲遼帝,不果行。



 十一月,
 宗翰復請曰:「諸軍久駐,人思自奮,馬亦壯健,宜乘此時進取中京。」群臣言時方寒,太祖不聽,竟用宗翰策。於是,忽魯勃極烈杲都統內外諸軍,蒲家奴、宗翰、宗乾、宗磐副之,宗峻領合扎猛安,皆受金牌,余睹為鄉導,取中京實北京。既克中京,宗翰率偏師趨北安州,與婁室、徒單綽里合兵,大敗奚王霞末,北安遂降。



 宗翰駐軍北安,遣希尹經略近地,獲遼護衛耶律習泥烈,乃知遼主獵於鴛鴦濼,殺其子晉王敖魯斡,眾益離心,西北、西南兩路兵馬皆羸弱,不可用。宗翰使耨碗溫都、移刺保報都統杲曰:「遼主窮迫於山西,猶事畋獵,不恤危亡,自殺其
 子,臣民失望。攻取之策,幸速見諭。若有異議,此當以偏師討之。」杲使奔睹與移刺保同來報曰:「頃奉詔旨,不令便趨山西,當審詳徐議。」當時,宗翰使人報杲,即整眾俟兵期。及奔睹至,知杲無意進取,宗翰恐待杲約或失機會,即決策進兵。使移刺保復往報都統曰:「初受命雖未令便取山西,亦許便宜從事。遼人可取,其勢已見,一失機會,後難圖矣。今已進兵,當與大軍會于何地,幸以見報。」宗乾勸杲當如宗翰策,杲意乃決,約以奚王嶺會議。



 宗翰至奚王嶺,與都統杲會。杲軍出青嶺,宗翰軍出瓢嶺,期於羊城濼會軍。宗翰以精兵六千襲遼主,聞遼主
 自五院司來拒戰,宗翰倍道兼行,一宿而至,遼主遁去。乃使希尹等追之。西京復叛,耿守忠以兵五千來救,至城東四十里,蒲察烏烈、谷赧先擊之,斬首千餘。宗翰、宗雄、宗乾、宗峻繼至,宗翰率麾下自其中衝擊之,使餘兵去馬從旁射之。守忠敗走,其眾殲焉。宗翰弟扎保迪沒於陣。天眷中,贈扎保迪特進云。



 宗翰已撫定西路州縣部族,謁上于行在所,遂從上取燕京。燕京平,賜宗翰、希尹、撻懶、耶律餘睹金器有差。太祖既以燕京與宋人,還軍次鴛鴦濼,不豫,將歸京師。以宗翰為都統,昃勃極烈昱、迭勃極烈斡魯副之,駐軍雲中。



 太宗即位,詔宗翰曰:「
 寄爾以方面,當遷官資者,以便宜除授。」因以空名宣頭百道給之。宋人來請割諸城,宗翰報以武、朔二州。宗翰請曰:「宋人不歸我叛亡,阻絕燕山往來道路,後必敗盟,請勿割山西郡縣。」太宗曰:「先皇帝嘗許之矣,當與之。」



 諸將獲耶律馬哥,宗翰歸之京師。詔以馬七百匹給宗翰軍,以田種千石、米七千石賑新附之民。詔曰:「新附之民,比及農時,度地以居之。」宗翰請分宗望、撻懶、石古乃精兵討諸部。詔曰:「宗望軍不可分,別以精銳五千給之。」宗翰朝太祖陵,入見上,奏曰:「先皇帝時,山西、南京諸部漢官,軍帥皆得承制除授。今南京皆循舊制,惟山西優以
 朝命。」詔曰:「一用先皇帝燕京所降詔敕從事,卿等度其勤力而遷授之。」



 宗翰復奏曰:「先皇帝征遼之初,圖宋協力夾攻,故許以燕地。宋人既盟之後,請加幣以求山西諸鎮,先皇帝辭其加幣。盟書曰:『無容匿逋逃,誘擾邊民。』今宋數路招納叛亡,厚以恩賞。累疏叛人姓名,索之童貫,嘗期以月日,約以哲書,一無所至。盟未期年,今已如此,萬世守約,其可望乎。且西鄙未寧,割付山西諸郡,則諸軍失屯據之所,將有經略,或難持久,請姑置勿割。」上悉如所請。



 上以宗翰破遼,經略夏國奉表稱籓,深嘉其功,以馬十匹,使宗翰自擇二匹,余賜群帥。



 及斡魯奏宋
 不遣歲幣戶口事,且將渝盟,不可不備。太宗命宗翰取諸路戶籍按籍索之。而闍母再奏宋敗盟有狀,宗翰、宗望俱請伐宋。於是,諳班勃極烈杲領都元帥,居京師,宗翰為左副元帥,自太原路代宋。



 宗翰發自河陰,遂降朔州,克代州,圍太原府。宋河東、陜西軍四萬救太原,敗于汾河之北,殺萬餘人。宗望自河北趨汴,久不聞問,遂留銀術可等圍太原,宗翰率師而南。天會四年降定諸縣及威勝軍,下隆德府實潞州。軍至澤州,宋使至軍中,始知割三鎮講和事。路允迪以宋割太原詔書來,太原人不受詔。宗翰取文水及盂縣,復留銀術可圍太原。宗翰乃還山西。



 宋少帝誘蕭仲恭貽書余睹,以興復遼社稷以動之。蕭仲恭獻其書,詔復伐宋。八月,宗翰發自西京。九月丙寅,宗翰克太原,執宋經略使張孝純等。鶻沙虎取平遙,降靈石、介休、孝義諸縣。十一月甲子,宗翰自太原趨汴,降威勝軍,克隆德府,遂取澤州。撒刺荅等先已破天井關,進逼河陽,破宋兵萬人,降其城。宗翰攻懷州,克之。丁亥,渡河。閏月,宗翰至汴,與宗望會兵。宋約畫河為界,復請修好。不克和。丙辰,銀術可等克汴州。辛酉,宋少帝詣軍前,舍青城。十二月癸亥,少帝奏表降。詔元帥府曰:「將帥士卒立功者,第其功之高下遷賞之。其殞身行陣,沒於
 王事者,厚恤其家,賜贈官爵務從優厚。」使勖就軍中勞賜宗翰、宗望,使皆執其手以勞之。五年四月,以宋二主及其宗族四百七十餘人及珪璋、寶印、袞冕、車輅、祭器、大樂、靈臺、圖書,與大軍北還。七月,賜宗翰鐵券,除反逆外,餘皆不問,賜與甚厚。



 宗翰奏河北、河東府鎮州縣請擇前資官良能者任之,以安新民。上遣耶律暉等從宗翰行。詔黃龍府路、南路、東京路於所部各選如耶律暉者遣之。宗翰遂趨洛陽。宋董植以兵至鄭州,鄭州人復叛。宗翰使諸將擊董植軍,復取鄭州。遂遷洛陽、襄陽、潁昌、汝、鄭、均、房、唐、鄧、陳、蔡之民於河北,而遣婁室平陜西州郡。
 是時河東寇盜尚多,宗翰乃分留將士,夾河屯守,而還師山西。昏德公致書「請立趙氏,奉職脩貢,民心必喜,萬世利也。」宗翰受其書而不答。



 康王遣王師正奉表,密以書招誘契彤、漢人。獲其書奏之。太宗下詔伐康王。河北諸將欲罷陜西兵,併力南伐。河東諸將不可,曰:「陜西與西夏為鄰,事重體大,兵不可罷。」宗翰曰:「初與夏約夾攻宋人,而夏人弗應。而耶律大石在西北,交通西夏。吾舍陜西而會師河北,彼必謂我有急難。河北不足虞,宜先事陜西,略定五路,既弱西夏,然後取宋。」宗翰蓋有意于夏人也。議久不決,奏請于上,上曰:「康王構當窮其所往
 而追之。俟平宋,當立籓輔如張邦昌者。陜右之地,亦未可置而不取。」於是婁室、蒲察帥師,繩果、婆盧火監戰,平陜西。銀術可守太原,耶律餘睹留西京。



 宗翰會東軍于黎陽津,遂會睿宗于濮。進兵至東平,宋知府權邦彥棄家宵遁,降其城,駐軍東平東南五十里。復取徐州。先是,宋人運江、淮金幣皆在徐州官庫,盡得之,分給諸軍。襲慶府來降。宋知濟南府劉豫以城降于撻懶。乃遣拔離速、烏林荅泰欲、馬五襲康王于揚州,未至百五十里,馬五以五百騎先馳至揚州城下。康王聞兵來,已於前一夕渡江矣。於是,康王以書請存趙氏社稷。先是,康王嘗
 致書元帥府,稱「大宋皇帝構致書大金元帥帳前」,至是乃貶去大號,自稱「宋康王趙構謹致書元帥閤下」。其四月、七月兩書皆然。元帥府答其書,招之使降。於是,撻懶、宗弼、拔離速、馬五等分道南伐。宗弼之軍渡江取建康,入于杭州。康王入海,阿里、蒲盧渾等自明州行海三百里,追之弗及。宗弼乃還。其後宗翰欲用徐文策伐江南,睿宗、宗弼議不合,乃止。語在《劉豫傳》。歸德叛,都統大糺里平之。



 初,太宗以斜也為諳班勃極烈,天會八年,斜也薨,久虛此位。而熙宗宗峻子,太祖嫡孫,宗乾等不以言太宗,而太宗亦無立熙宗意。宗翰朝京師,謂宗乾曰:「儲
 嗣虛位頗久,合刺先帝嫡孫,當立,不早定之,恐授非其人。宗翰日夜未嘗忘此。」遂與宋乾、希尹定議,入言於太宗,請之再三。太宗以宗翰等皆大臣,義不可奪,乃從之,遂立熙宗為諳班勃極烈。於是,宗翰為國論右勃極烈,兼都元帥。



 熙宗即位,拜太保、尚書令,領三省事,封晉國王。乞致仕,詔不許。天會十四年薨,年五十八。追封周宋國正。正隆二年,例封金源郡王。大定間,改贈秦王,謚桓忠,配享太祖廟廷。



 孫秉德、斜哥。秉德別有傳。



 斜哥,累官同知曷蘇館節度使事。大定初,除刑部侍郎,充都統,與副統完顏布輝自東京先赴中都,輒署置官
 吏,私用官中財物。世宗至中都,事覺,斜哥當死,布輝當除名。詔寬減,斜哥除名,布輝削兩階,解職。



 二年,起為大宗正丞,除祁州刺史。坐贓枉法,當死,詔杖一百五十,除名。遣左衛將軍夾谷查刺諭斜哥曰:「卿何面目至鄉中與宗族相見。今徙鄜州,以家人自隨,俟汝身死,聽家人從便。」久之,起同知興中尹,遷唐括部族節度使,歷開遠、順義軍。



 斜哥前在雲內受贓,御史臺劾奏,上謂宰臣曰:「斜哥今三犯矣,蓋其資質鄙惡如此。」令強幹吏鞫之。獄成,法當死。上曰:「斜哥祖父秦王宗翰有大功,特免死,杖一百五十,除名。」久之,復起為勸農副使。



 贊曰:宗翰內能謀國,外能謀敵,決策制勝,有古名將之風。臨潢既捷,諸將皆有怠忽之心,而請伐不已。越千里以襲遼主,諸將皆有畏顧之心,而請期不已。觀其欲置江、淮,專事陜服,當時無有能識其意者。甫釋干戈,斂衽歸朝,以定熙宗之位,精誠之發,孰可掩哉。



 宗望本名斡魯補,又作斡離不,太祖第二子也。每從太祖征伐,常在左右。



 都統杲已克中京,宗翰在北安州,獲遼護衛習泥烈,知遼主在鴛鴦濼,宗翰請襲之。杲出青嶺,遼兵三百餘掠降人家貲。宗望曰:「若生致此輩,可審得遼主所在虛實。」遂與宗弼率百騎進。騎多罷乏,獨與
 馬和尚逐越盧、孛古、野里斯等,留一騎趣後軍,即馳擊敗之,生擒五人。因審遼主尚在鴛鴦濼未去無疑也。於是進兵。宗翰倍道兼行,追遼主於五院司,不及。婁室等追之至白水濼,遼主走陰山。遼秦晉國王捏里自立于燕京。新降州部,人心不固,杲使宗望請太祖臨軍。



 宗望至京師,百官入賀。上曰:「宗望與十餘騎經涉兵寇數千里,可嘉也。」上宴群臣,歡甚,宗望奏曰:「今雲中新定,諸路遼兵尚數萬,遼主尚在陰山、天德之間,而捏里自立于燕京,新降之民,其心未固,是以諸將望陛下幸軍中也。」上曰:「懸軍遠伐,授以成算,豈能盡合機事。朕以六月朔
 啟行。」既次大濼西南,果使希尹奏請徙西南招討司諸部于內地。上顧謂群臣曰:「徙諸部人當出何路?」宗望對曰:「中京殘弊,芻糧不給,由上京為宜。然新降之人,遽爾騷動,未降者必皆疑懼。勞師害人,所失多矣。」上京謂臨潢府也。上乃下其議,命軍帥度宜行之。



 上聞遼主在大魚濼,自將精兵萬人襲之。蒲家奴、宗望率兵四千為前鋒,晝夜兼行,馬多乏,追及遼主于石輦驛,軍士至者才千人,遼軍餘二萬五千。方治營壘,蒲家奴與諸將議。余睹曰:「我軍未集,人馬疲劇,未可戰。」宗望曰:「今追及遼主而不亟戰,日入而遁,則無及。」遂戰,短兵接,遼兵圍之數
 重,士皆殊死戰。遼主謂宗望兵少必敗,遂與嬪御皆自高阜下平地觀戰。余睹示諸將曰:「此遼主麾蓋也。若萃而薄之,可以得志。」騎兵馳赴之,遼主望見大驚,即遁去,遼兵遂潰。宗望等還。上曰:「遼主去不遠,亟追之。」宗望以騎兵千餘追之,蒲家奴為後繼。



 太祖已定燕京,斡魯為都統,宗望副之,襲遼主於陰山、青塚之間。宗望、婁室、銀術可以三千軍分路襲之。將至青塚,遇泥濘,眾不能進。宗望與當海四騎以繩繫遼都統林牙大石,使為鄉導,直至遼主營。時遼主往應州,其嬪御諸女見敵兵奄至驚駭欲奔,命騎下執之。有頃,後軍至。遼太叔胡盧瓦妃,
 國王捏里次妃,遼漢夫人,并其子秦王、許王,女骨欲、餘里衍、斡里衍、大奧野、次奧野、趙王妃斡里衍,招討迪六,詳穩六斤,節度使孛迭、赤狗兒皆降。得車萬餘乘,惟梁王雅里及其長女乘軍亂亡去。婁室、銀術可獲其左右輿帳。進至掃里門,為書以招遼主。



 遼主自金城來,知其族屬皆見俘,率兵五千餘決戰。宗望以千兵擊敗之。遼主相去百步,遁去。獲其子趙王習泥烈及傳國璽。追二十餘里,盡得其從馬,而照里、特末、胡巴魯、背荅別獲牧馬萬四千匹、車八千乘。及獻傳國璽於行在,太祖曰:「此群臣之功也。」遂置璽于懷中,東面恭謝天地,乃大錄諸
 帥功,加賞焉。



 遼主乃使謀盧瓦持兔鈕金印請降。宗望受之,視其文,乃「元帥燕國王之印」也。宗望復以書招之,諭以石晉北遷事。遂使使諭夏國,示以和好,所以沮疑其救遼之心也。宗望趨天德,遼耶律慎思降。及候人吳十回,皆言夏國迎護遼主度大河矣。宗望乃傳檄夏國曰:「果欲附我,當如前諭,執送遼主。若猶疑貳,恐有後悔。」及遼秦王等以俘見太祖,太祖嘉宗望功,以遼蜀國公主餘里衍賜之。



 闍母與張覺戰,大敗於兔耳山。上使宗望問狀,就以闍母軍討張覺,降瀕海郡縣。遂與覺戰于南京城東。覺敗,宵遁奔宋,語在《覺傳》。城中人執覺父及
 其二子來獻,宗望殺之。使以詔書宣諭城中張敦固等出降。使使與敦固俱入城收兵仗。城中人殺使者,立敦固為都統,劫府庫,掠居民,乘城拒守。太宗賞破張覺功及有功將士各有差。



 初,張覺奔宋,入於燕京,宗望責宋人納叛人,且征軍糧。久不聞問,宗望欲移書督之,請空名宣頭千道,增信牌,安撫新降之民。詔以「新附長吏職員仍舊。已命諸路轉輸軍糧,勿督於宋。給銀牌十、空名宣頭五十道。及遷、潤、來、隰四州人徙于瀋州者,俟畢農各復其業。」乃詔咸州輸粟宗望軍。



 張敦固以兵八千分四隊出戰,大敗。宗望再三開諭,敦固等曰:「屢嘗拒戰,不
 敢遽降。」宗望許其望闕遙拜。敦固乃開其一門。宗望使闍母奏其事,乃下詔赦南京官民,大小罪皆釋之,官職如舊。別敕有司輕徭賦、勸稼穡,疆場之事,一決於宗望。又曰:「議索張覺及逋亡戶口於宋。聞此歲不登,若如舊征斂,恐民匱乏,度其糧數賦之。射糧軍願為民者,使復田里。小大之事關白軍帥,無得專達朝廷。」詔宗望曰:「選勛賢及有民望者為南京留守,及諸闕員,仍具姓名官階以聞。」是時,遷、潤、來、隰四州之民保山砦者甚眾,宗望乙選良吏招撫。上從之。



 上召宗望赴闕,而闍母克南京,兵執偽都統張敦固殺之,南京平。赴京師。於是,宗翰請
 無割山西地與宋,斡魯亦言之。闍母論奏宋渝盟有驗,不可不備。及宗望還軍,上曰:「征歲幣於宋,以銀二十萬兩、絹三十萬匹分賜爾軍及六部東京諸軍。」宗望至軍,宋兵三千自海道來,破九寨,殺馬城縣戍將節度使度盧斡,取其銀牌兵仗及馬而去。宗望索戶口,宋人弗言,且聞童貫、郭藥師治軍燕山。宗望奏請伐宋曰:「茍不先之,恐為後患。」宗翰亦以為方。故伐宋之策,宗望實啟之。



 宗望為南京路都統,闍母副之,自燕山路伐宋。宗望奏曰:「闍母於臣為叔父,請以闍母為都統,臣監戰事。」上從之。以宗望監闍母、劉彥宗兩軍戰事。宗望至三河,破郭
 藥師兵四萬五千於白河,蒲莧敗宋兵三千於古北口,郭藥師降。遂取燕山府,盡收其軍實,馬萬匹、甲胄五萬、兵七萬,州縣悉平。宋中山戍將王彥、劉璧率兵二千來降。蒲察、繩果以三百騎遇中山三萬人於阨隘之地,力戰,死之。術烈速、活里改軍繼至,殺二萬餘人。宗望破宋真定兵五千人,遂克信德府,次邯鄲。宋李鄴請修舊好。宗望留軍中不遣。



 自郭藥師降,益知宋之虛實。宗望請以為燕京留守。及董才降,益知宋之地里。宗望請任以軍事。太宗俱賜姓完顏氏,皆給以金牌。



 四年正月己巳,諸軍渡河,取滑州。使吳孝民入汴,以詔書問納平州張
 覺事,令執送童貫、譚積、詹度,以黃河為界,納質奉貢。癸酉,諸軍圍汴。宋少帝請為伯姪國,效質納地,增歲幣請和。遂割太原、中山、河間三鎮,書用伯姪禮,以康王構、太宰張邦昌為質。沈晦以哲書、三鎮地圖至軍中,發幣割地一依定約,語在宋事中。



 二月丁酉朔,與宋平,退軍孟陽。是夜,姚平仲兵四十萬來襲。候騎覺之,分遣諸將迎擊,大破平仲軍,復進攻汴城,問舉兵之狀。少帝大恐,使宇文虛中來辨曰:「初不知其事,且將加罪其人。」宗望輟弗攻,改肅王樞為質,康王構遣歸。師還,河北兩鎮不下,遂分兵討之。



 宗望罷常勝軍,給還燕人田業,命將士分
 屯安肅、雄、霸、廣信之境。宗望還山西。未幾。為右副元帥,有功將士遷賞有差。



 頃之,宋少帝以書誘餘睹,肅仲恭獻其書,詔復伐宋。八月,宗望會諸將,發自保州。耶律鐸破敵兵三萬于雄州,殺萬餘人。那野敗宋軍七千於中山。高六、董才破宋兵三千於廣信。宋種師閔軍四萬人駐井陘,宗望大破之,遂取天威軍。東還,遂克真定,殺知府李邈,得戶三萬,降五縣。遂自真定趨汴。



 十一月戊辰,宗望至河上,降魏縣。諸軍渡河,留諸將分出大名之境。降臨河縣,至大名縣,德清軍、開德府,皆克之。阿里刮以騎兵三千先趨汴,破宋軍六千於路。取胙城,抵汴城下,
 覆宋兵千人,擒數將。宗望至汴,分遣諸將遏宋援兵,奔睹、那野、賽刺、臺實連破宋援兵。閏月壬辰朔,宋兵一萬出自汴城來戰。宗望選勁勇五千,使當海、忽魯、雛鶻失擊敗之。癸巳,宗翰自太原會軍于汴。丙辰,克汴州。辛酉,宋少帝詣軍前。十二月癸亥,宋帝奉表降。上使勖就軍中勞賜宗翰、宗望,使皆執其手以勞之。五年四月,以宋二主及其宗族四百七十餘人,及珪璋、寶印、袞冕、車輅、祭器、大樂、靈臺、圖書,與大軍北還。



 宗望乃分諸將鎮守河北。董才降廣信軍及旁近縣鎮。宗望乃西上涼陘。詔宗望曰:「自河之北,今既分畫,重念其民見城邑有被殘者,遂
 阻命堅守,其申諭招輯安全之。儻堅執不移,自當致討。若諸軍敢利於俘掠,輒肆毀蕩者,當底於罰。」



 是月,宗望薨。天會十三年,封魏王。皇統三年,進許國王,又徙封晉國王。天德二年,贈太師,加遼燕國王,配享太宗廟廷。正隆二年,例降封。大定三年,改封宋王,謚桓肅。子齊、京、文。



 初。遼帝之奔陰山也,遼節度使和尚與林牙馬哥、男慎思俱被擒,都統杲使阿鄰護送得里底、和尚、雅里斯等入京師。得里底道亡,太祖誅阿鄰。和尚弟道溫為興中尹,太祖使謾都本以兵千人與和尚往招之。和尚欲亡去,不克,至興中城下,以矢繫書射城中,教道溫毋降。事
 泄,謾都本責之曰:「汝何反覆如此?」對曰:「以忠報國,何反覆之有,雖死不恨。」乃殺之。既而宗望軍遇遼都統孛迭等,道溫在其中,相與隔水而語。宗望承制招之,孛迭唯諾,無降意。宗望謂道溫曰:「汝兄和尚因戰而獲,未嘗加罪,後以叛誅,能無痛悼。」道溫曰:「吾兄辱於見獲,榮於死國。」宗望顧馬和尚曰:「能為我取此乎?」對曰:「能。」遂以所部渡水擊敗其眾,直趨道溫,射中其臂,獲而殺之。



 齊本名受速,長身美髯。天眷三年,以宗室子授鎮國上將軍。皇統元年,遷光祿大夫。正隆六年,遷銀青榮祿大夫。大定初,遷特進,加安武軍節度使,留京師奉朝請。齊
 以近屬,上所寵遇,而性庸滯無材能。大定三年,罷節度官,給隨朝三品俸,累官特進。卒。



 弟京、弟文皆以謀反誅。世宗盡以其家財產與齊之子咬住。詔齊妻曰:「汝等皆當緣坐,有至大辟及流竄者。朕念宋王,故置而不問,且以其家產賜汝子。宜悉朕意。」十五年,上召英王爽謂曰:「卿於讀者公主女子中為咬住擇婚,其禮幣命有司給之。」俄襲叔父京山東西路徒毋堅猛安。



 京本名忽魯,以宗室子累遷特進。天德二年,除翰林學士承旨,兼修國史,加開府儀同三司,遷工部尚書,改禮部、兵禮部,判大宗正事,封曹王,除河間尹。正隆二年,例封
 瀋國公,北京留守,以喪去官。起復益都尹。六年,坐違制,立春日與徒單貞飲酒,降灤州刺史。未幾,改絳陽軍節度使。海陵遣護衛忽魯往絳州殺之。京由間道走入汾州境得免。



 世宗即位,來見於桃花塢。復判大宗正事,封壽王。二年正月戊辰朔,日食,伐鼓用幣,上不視朝,減膳徹樂。詔京代拜行禮。世宗懲創海陵疏忌宗室,加禮京兄弟,情若同生。謂京等曰:「朕每見天象變異,輒思政事之闕,寤寐自責不遑。凡事必審思而後行,猶懼獨見未能盡善,每令群臣集議,庶幾無過舉也。」是時,伐宋未罷兵,用度不足,百官未給全俸。京家人數百口,財用少,上
 聞之,賜金一百五十兩、重綵百端、絹五百匹。改西京留守,賜佩刀廄馬。



 京到西京,京妻嘗召日者孫邦榮推京祿命。邦榮言留守官至太師,爵封王。京問:「此上更無否?」邦榮曰:「止於此。」京曰:「若止於此,所官何為。」邦榮察其意,乃詐為圖讖,作詩,中有「鶻魯為」之語,以獻於京。京曰:「後誠如此乎。」遂受其詩,再使卜之。邦榮稱所得卦有獨權之兆。京復使邦榮推世宗當生年月。家人孫小哥妄作謠言誑惑京,如邦榮指,京信之。京妻公壽具知其事。大定五年三月,孫邦榮上變。詔刑部侍郎高德基、戶部員外郎完顏兀古出往鞫之。京等皆款伏。獄成,還奏。上曰:「
 海陵無道,使光英在,朕亦保全之,況京等哉。」於是,京夫婦特免死,杖一百,除名,嵐州樓煩縣安置,以奴婢百口自隨,官給上田。遣兀古出、劉珫宣諭京,詔曰:「朕與汝皆太祖之孫。海陵失道,翦滅宗支,朕念兄弟無幾,於汝尤為親愛,汝亦自知之,何為而懷此心。朕念骨肉,不忍盡法。汝若尚不思過,朕雖不加誅,天地豈能容汝也。」十年四月,詔于樓煩縣,為京作第一區,月給節度廩俸。



 十二年,兄德州防禦使文謀反。上問皇太子、趙王允中及宰臣曰:「京謀不軌,朕特免死,今復當緣坐,何如。」宰臣或言京圖逆,今不除之,恐為後患。上曰:「天下大器歸於有德,
 海陵失道,朕乃得之。但務修德,餘何足慮。」太子曰:「誠如聖訓。」乃遣使宣諭京,詔曰:「卿兄文,舊封國公,不任職事,朕進封王爵,委以大籓。頃在大名,以贓得罪,止削左遷,不知恩幸,乃蓄怨心,謀不軌,罪及兄弟。朕念宋王,皆免緣坐。文之家產應沒入者,盡與卿兄子咬住。卿宜悉此意。」



 二十年十一月,上問宰臣曰:「京之罪始於其妻,妄卜休咎。太祖諸孫存者無幾,朕欲召置左右,不使任職,但廩給之,卿等以為何如?」皆曰:「置之近密,臣等以為非宜。」上曰:「朕若修德,何以豫懷疑忌。」久之,上復欲召京,宰臣曰:「京,不赦之罪也,赦之以為至幸矣,豈可復。」上默良久,
 乃止。



 文本名胡刺。皇統間,授世襲謀克,加奉國上將軍,居中京。



 海陵篡立,賜錢二萬貫。是時,左淵為中京轉運使,市中有穢術敲仙者,文與淵皆與之游。海陵還中京,聞,召敲仙詰問,窮竟本末。既而殺之于市,責讓文、淵。貞元元年,除秘書,坐與靈壽縣主阿里虎有姦,杖二百,除名。俄復為秘書監,封王。正隆例封鄖國公,以喪去官。起復翰林學士承旨、同判大宗正事、昌武軍節度使。



 大定初,改武定軍,留京師,奉朝請。三年,賜上常御條服佩刀而遣之。謂文曰:「朕無兄弟,見卿往外郡,惻然傷懷。卿頗自放,
 宜加檢束。」除廣寧尹,召為判大宗正事,封英王。是時,弟京得罪,上謂文曰:「朕待京不薄,乃包藏禍心,圖不軌,不忍刑及骨肉,遂從輕典。卿亦驕縱無度。宋王有社稷功。武靈封太祖諸孫為王,卿獨不封。朕即位,封卿兄弟為王。自今懲咎悔過,赤心事朕,無患朕不知也。」除真定尹,賜以衣帶。改大名尹,徙封荊王。



 文到大名,多取猛安謀克良馬,或以駑馬易之,買民物與價不盡其直。尋常占役弓手四十餘人,詭納稅草十六萬束。公用闕,取民錢一萬九千餘貫。坐是奪爵,降德州防禦使,僚佐皆坐不矯正解職。監察御史董師中按文事失糾察,已除尚書
 省都事,降沁南軍節度副使。詔曰:「自今長官不法,僚佐不矯正,又不言上,並嚴行懲斷。」



 文既失職,居常怏怏,日與家奴石抹合住、忽里者為怨言。合住揣知其意,因言南京路猛安阿古、合住、謀克頗里,銀術可與大王厚善,果欲舉大事,彼皆願從。文信其言。乃召日者康洪占休咎,密以謀告洪。洪言來歲甚吉。文厚謝洪,使家僮剛哥等往南京以書幣遺阿古等。剛哥問合住何以知阿古等必從。合住曰:「阿古等與大王善,以此意其必從耳。」剛哥到南京,見阿古等,不言其本來之事。及還,紿文曰:「阿古從大王矣。」文乃造兵仗,使家奴斡敵畫陣圖。家奴重
 喜詣河北東路上變,府遣總管判官孛特馳往德州捕文。孛特至德州,日已晚。會文出獵,召防禦判官酬越謀就獵所執之。酬越言:「文兵衛甚眾,且暮夜,明日文生日,可就會上執之。」孛特乃止。是夜,文知本府使至,意其事覺,乃與合住、忽里者等俱亡去。河間府使奏文事,詔遣右司郎中紇石烈哲典、斡林修撰阿不罕訛里也往德州鞫問。



 上聞文亡命,謂宰臣曰:「海陵翦滅宗室殆盡,朕念太祖孫存者無幾人,曲為寬假,而文曾不知幸,尚懷異圖,何狂悖如此。」上恐文久不獲,詿誤者多,督所在捕之。詔募獲文者遷官五階,賜錢三千貫。文以大定十二
 年九月事覺,亡命凡四月,至十二月被獲,伏誅。康洪論死,餘皆坐如律。詔釋其妻術實懶。孛特、酬越不即捕,致文亡去,孛特杖二百,除名,酬越杖一百,削兩階。詔曰:「德州防禦使文、北京曹貴、鄜州李方皆因術士妄談祿命,陷于大戮。凡術士多務茍得,肆為異說。自今宗室、宗女有屬籍者及官職三品者,除占問嫁娶、修造、葬事,不得推算相命,違者徒二年,重者從重。」上以文家財產賜其故兄特進齊之子咬住,并以西京留守京沒入家產賜之。



 贊曰:宗望啟行平州,戰勝白河,席卷而南,風行電舉,兵
 無留難,再閱月而汴京圍矣。所謂敵不能與校者耶。既取信德,留兵守之,以為後距,此豈輕者耶。《管子》曰:「徑於絕地,攻於恃固,獨出獨入,而莫之能止。」其宗望之謂乎。



\end{pinyinscope}