\article{志第十六}

\begin{pinyinscope}

 禮八



 ○
 宣聖廟



 皇統元年二月戊子,熙宗詣文宣王廟奠祭,北面再拜,顧儒臣曰:「為善不可不勉。孔子雖無位,以其道可尊,使萬世高仰如此。」大定十四年,國子監言:「歲春秋仲月上丁日,釋奠於文宣王,用本監官房錢六十貫,止造茶食等物,以大小楪排設,用留守司樂,以樂工為禮
 生,率倉場等官陪位,於古禮未合也。伏睹國家承平日久,典章文物當粲然備具,以光萬世。況京師為首善之地,四方之所觀仰,擬釋奠器物、行禮次序,合行下詳定。兼兗國公親承聖教者也,鄒國公功扶聖教者也,當於宣聖像左右列之。今孟子以燕服在後堂,宣聖像側還虛一位,禮宜遷孟子像於宣聖右,與顏子相對,改塑冠冕,妝飾法服,一遵舊制。」



 禮官參酌唐《開元禮》,定擬釋奠儀數:文宣王、兗國公、鄒國公每位籩豆各十、犧尊一、象尊一、簠簋各二、俎二、祝板各一、皆設案。七十二賢、二十一先儒,每位各籩一、豆一、爵一,兩廡各設象尊二。總用
 籩、豆各一百二十三,簠簋各六,俎六,犧尊三,象尊七,爵九十四。其尊皆有坫。罍二,洗二,篚勺各二,冪六。正位并從祀藉尊、罍、俎、豆席,約用三十幅,尊席用葦,俎、豆席用莞。牲用羊、豕各三,酒二十瓶。禮行三獻,以祭酒、司業、博士充。分獻官二,讀祝官一,太官令一,捧祝官二,罍洗官一,爵洗官一,巾篚官二,禮直官十一,學生以儒服陪位。樂用登歌,大樂令一員,本署官充,樂工三十九人。迎神,三奏姑洗宮《來寧之曲》,辭曰:「上都隆化,廟堂作新。神之來格,威儀具陳。穆穆凝旒,巍然聖真。斯文伊始,群方所視。」初獻盥洗,姑洗宮《靜寧之曲》,辭曰:「偉矣素王,風猷至
 粹。垂二千年,斯文不墜。涓辰維良,爰修祀事。沃盥于庭,嚴禋禮備。」升階,南呂宮《肅寧之曲》,辭曰:「巍乎聖師,道全德隆。修明五常,垂教無窮。增崇儒宮,遹追遺風。嚴祀申虔,登降有容。」奠幣,姑洗宮《和寧之曲》,辭曰:「天生聖人,賢於堯舜。仰之彌高,磨而不磷。新廟告成,宮墻數仞。遣使陳祠,斯文復振。」降階,姑洗宮《安寧之曲》辭曰:「稟靈尼丘,垂芳闕里。生民以來,孰如夫子。新祠巋然,四方所視。酹觴告成,祗循典禮。」兗國公酌獻,姑洗宮《輯寧之曲》,辭曰:「聖師之門,顏惟居上。其殆庶幾,是宜配饗。桓圭袞衣,有嚴儀象。載之神祠,增光吾黨。」鄒國公酌獻,姑洗宮《泰寧
 之曲》,辭曰:「有周之衰,王綱既墜。是生真儒,宏才命世。言而為經,醇乎仁義。力扶聖功,同垂萬祀。」亞、終獻,姑洗宮《咸寧之曲》,辭曰:「於昭聖能,與天立極。有承其流,皇仁帝德。豈伊立言,訓經王國。煥我文明,典祀千億。」送神,姑洗宮《來寧之曲》,辭曰:「吉蠲為饎,孔惠孔時。正辭嘉言,神之格思。是饗是宜,神保聿歸。惟時肇祀,太平極致。」



 承安二年,春丁,章宗親祀,以親王攝亞、終獻,皇族陪祀,文武群臣助奠。上親為贊文,舊封公者升為國公,侯者為國侯,郕伯以下皆封侯。宣宗遷汴,建廟會朝門內,歲祀如儀,宣聖、顏、孟各羊一、豕一,餘同小祀,共用羊八,無豕。其諸
 州釋奠並遵唐儀。



 ○武成王廟



 泰和六年,詔建昭烈武成王廟于闕庭之右,麗澤門內。其制一遵唐舊,禮三獻,官以四品官已下,儀同中祀,用二月上戊。七年,完顏匡等言:「我朝創業功臣,禮宜配祀。」於是,以秦王宗翰同子房配武成王,而降管仲以下。又躋楚王宗雄、宗望、宗弼等侍武成王坐,韓信而下降立於廡。又黜王猛、慕容恪等二十餘人,而增金臣遼王斜也等。其祭,武成王、宗翰、子房各羊一、豕一,餘共用羊八,無豕。宣宗遷汴,餘會朝門內闕庭之右營廟如制,春秋上戊之祭仍舊。



 ○
 諸前代帝王



 三年一祭,於仲春之月祭伏犧於陳州,神農於亳州,軒轅於坊州,少昊於兗州,顓頊於開州,高辛於歸德府,陶唐於平陽府,虞舜、夏禹、成湯於河中府,周文王、武王於京兆府。泰和三年、尚書省奏:「太常寺言:『《開元禮》祭帝嚳、堯、舜、禹、湯、文、武、漢祖祝版請御署。《開寶禮》犧、軒、顓頊、帝嚳、陶唐、女媧、成湯、文、武請御署,自漢高祖以下二十七帝不署。』平章政事鎰、左丞匡、太常博士溫迪罕天興言:『方嶽之神各有所主,有國所賴,請御署固宜。至於前古帝王,寥落杳茫,列于中祀亦已厚矣,不須御署。』參知政事即康及鉉以為三皇、五帝、禹、湯、文、武皆
 垂世立教之君,唐、宋致祭皆御署,而今降祝版不署,恐於禮未盡。不若止從外路祭社稷及釋奠文宣王例,不降祝版,而令學士院定撰祝文,頒各處為常制。」敕命依期降祝板,而不請署。



 ○諸神雜祠·長白山



 大定十二年,有司言:「長白山在興王之地,禮合尊崇,議封爵,建廟宇。」十二月,禮部、太常、學士院奏奉敕旨封興國靈應王,即其山北地建廟宇。十五年三月,奏定封冊儀物,冠九旒,服九章,玉圭、玉冊、函、香、幣、冊、祝。遣使副各一員,詣會寧府。行禮官散齋二日,致齋一日。所司於廟中陳設如儀。廟門外設玉冊、袞冕幄次,牙杖旗
 鼓從物等視一品儀。禮用三獻,如祭嶽鎮。其冊文云:「皇帝若曰:自兩儀剖判,山嶽神秀各鐘於其分野。國將興者,天實作之。對越神休,必以祀事。故肇基王迹,有若岐陽。望秩山川,於稽虞《典》。厥惟長白,載我金德,仰止其高,實惟我舊邦之鎮。混同流光,源所從出。秩秩幽幽,有相之道。列聖蕃衍熾昌,迄于太祖,神武徵應,無敵于天下,爰作神主。肆予沖人,紹休聖緒,四海之內,名山大川,靡不咸秩。矧王業所因,瞻彼旱麓,可儉其禮?服章爵號非位於公侯之上,不足以稱焉。今遣某官某,持節備物,冊命茲山之神為興國靈應王,仍敕有司歲時奉祀,於戲!
 廟食之享,亙萬億年。維金之禎,與山無極,豈不偉歟?」自是,每歲降香,命有司春秋二仲擇日致祭。明昌四年十月,備袞冕、玉冊、儀物,上御大安殿,用黃麾立仗八百人,行仗五百人,復冊為開天弘聖帝。



 ○諸神雜祠·大房山



 大定二十一年,敕封山陵地大房山神為保陵公,冕八旒、服七章、圭、冊、香、幣,使副持節行禮,並如冊長白山之儀。其冊文云:「皇帝若曰:古之建邦設都,必有名山大川以為形勝。我國既定鼎於燕,西顧郊圻,巍然大房,秀撥混厚,雲雨之所出,萬民之所瞻,祖宗陵寢於是焉依。仰惟嶽鎮古有秩序,皆載祀典,矧茲大房,禮可闕歟?其爵號服章俾
 列於侯伯之上,庶足以稱。今遣某官某,備物冊命神為保陵公。申敕有司,歲時奉祀。其封域之內,禁無得樵採弋獵。著為令。」是後,遣使山陵行禮畢,山陵官以一獻禮致奠。



 ○諸神雜祠·混同江



 大定二十五年,有司言:「昔太祖征遼,策馬徑渡,江神助順,靈應昭著,宜修祠宇,加賜封爵。」乃封神為興國應聖公,致祭如長白山儀,冊禮如保陵公故事。其冊文云:「昔我太祖武元皇帝,受天明命,掃遼季荒茀,成師以出,至于大江,浩浩洪流,不舟而濟,雖穆滿渡江面黿粱,光武濟河而水冰,自今觀之無足言矣!執徐之歲,四月孟夏,朕時邁舊邦,臨江永歎,仰藝祖之開基,
 佳江神之效靈,至止上都,議所以尊崇之典。蓋古者五嶽視三公,四瀆視諸侯,至有唐以來,遂享帝王之尊稱,非直後世彌文,而崇德報功理亦有當然者。矧茲江源出於長白,經營帝鄉,實相興運,非錫以上公之號,則無以昭答神休。今遣某官某。持節備物冊命神為興國應聖公。申命有司,歲時奉祀。於戲!嚴廟貌,正封爵,禮亦至矣!惟神其衍靈長之德,用輔我國家彌億年,神亦享廟食於無窮,豈不休哉!」



 ○諸神雜祠·嘉蔭侯



 大定二十五年,敕封上京護國林神為護國嘉蔭侯,毳冕七旒,服五章,圭同信圭,遣使詣廟,以三獻禮祭告。其祝文曰:「蔚彼長林,實壯
 於邑,廣袤百里,惟神主之。廟貌有嚴,侯封是享,歆時蠲潔,相厥滋榮。」是後,遇月七日,上京幕官一員行香,著為令。



 ○諸神雜祠·瀘溝河神



 大定十九年,有司言:「瀘溝河水勢泛決嚙民田,乞官為封冊神號。」禮官以祀典所不載,難之。已而,特封安平侯,建廟。二十七年,奉旨,每歲委本縣長官春秋致祭,如令。



 ○諸神雜祠·昭應順濟聖后



 大定十七年,都水監言:「陽武上埽黃河神聖后廟,宜依唐仲春祭五龍祠故事。」二十七年春正月,尚書省言:「鄭州河陰縣聖后廟,前代河水為患屢禱有應,嘗加封號廟額。今因禱祈,河遂安流,乞加褒贈。」上從其請,特加號曰昭應順濟聖后。廟
 曰靈德善利之廟。每歲委本縣長官春秋致祭,如令。



 ○諸神雜祠·鎮安公



 舊名旺國崖,太祖伐遼嘗駐蹕於此。大定八年五月,更名靜寧山,後建廟。明昌六年八月,以冕服玉冊,冊山神為鎮安公。冊文曰:「皇帝若曰:古之名山,咸在祀典。軒皇之世,神靈所奉者七千。虞氏之時,望秩每及於五載。蓋惟有益于國,是以必報其功。逮乎後王,申以徽冊,至于嶽鎮之外,亦或封爵之加。故太白有神應之稱,而終南有廣惠之號。禮由義起,事與時偕,載籍所傳,於今猶監,朕修和有夏,咸秩無文,眷茲靜寧,秀峙朔野,厓澤布氣,幽贊乎坤元,導風出雲,協符乎乾造。一方之表,
 萬物所瞻,南直都畿,北維障徼,連延廣厚,寶藏攸興,盤固高明,謻宮斯奠。昔有遼嘗恃以富國,迄大定更為之錫名。洪惟世宗,功昭列聖,亦越顯考,德利生民。爰即歲時,駕言臨幸,兵革不試,遠人輯寧。雨暘常調,品匯蕃廡,此上帝無疆之貺,亦英靈有相之符。比即輿情,載修故事。顧先皇帝駐蹕之地,揖累世承平之風。迓續遺休,式甄神祐,肆象德以畀號,仍班台而闡儀。宇像一新,采章具舉。今遣使某、副某,持節備物,冊命神為鎮安公,仍敕歲時奉祀。於戲!容典焜耀,精明感通,惟永億年,翊我昌運。神其受職,豈不偉歟?」



 ○諸神雜祠·瑞聖公



 即麻達葛山也,章宗
 生於此。世宗愛此山勢衍氣清,故命章宗名之。後更名胡土白山,建廟。明昌四年八月,以冕服玉冊,封山神為瑞聖公。建廟,命撫州有司,春秋二仲,擇日致祭為常。其冊文曰:「皇帝若曰:國家之興,命歷攸屬。天地元化,惟時合符。山川百神,無不受職。粹精薦瑞,明聖繼生。著丕應於殊禎,啟昌期於幽贊。裒對信猶之典,咸修望秩之文。嘉乃名山,奠茲勝地,下綿乾分,上直樞輝。盤析木之津,達中原之氣。廓除氛昆,函毓泰和。仰惟光烈昭垂,徽音如在,即高明而清暑,克靜壽以安仁。周廬安寧,厚澤浹洽。朕祗循祖武,順講時巡,感美號以興懷,佩聖謨而介
 福。言念誕彌之初度,言由翊衛之效靈。然猶祀秩無章,神居不屋,非所以盡報功崇德之義,副追始樂原之心。爰飾名稱,載新祠宇。勒忱辭於貞琰,涓良日於元龜,彰服采以辨威,潔庪縣而致祭。闡揚茂實,敷繹多儀。今遣使某、副某,持節備物,冊命神為瑞聖公,仍敕有司歲時奉祀。於戲!尚其聰明,歆此誠意,孚休惟永,亦莫不寧。」



 ○諸神雜祠·貞獻郡王廟



 明昌五年正月,陳言者謂:「葉魯、谷神二賢創制女直文字,乞各封贈名爵,建立祠廟。令女直、漢人諸生隨拜孔子之後拜之。」有司謂:「葉魯難以致祭,若金源郡貞獻王谷神則既已配享太廟矣,亦難特立廟也。」
 有旨,令再議之。禮官言:「前代無創製文字入孔子廟故事,如於廟後或左右置祠,令諸儒就拜,亦無害也。」尚書省謂:「若如此,恐不副國家厚功臣之意。」遂詔令依蒼頡立廟于盩厔例,官為立廟于上京納里渾莊,委本路官一員與本千戶春秋致祭,所用諸物從宜給之。



 ○祈禜



 大定四年五月,不雨。命禮部尚書王競祈雨北嶽,以定州長貳官充亞、終獻。又卜日於都門北郊,望祀嶽鎮海瀆,有司行事,禮用酒脯醢。後七日不雨,祈太社、太稷。又七日祈宗廟,不雨,仍從嶽鎮海瀆如初祈。其設神座,實樽罍,如常儀。其樽罍用瓢齊,擇甘瓠去柢以為尊。
 祝版惟五岳、宗廟、社稷御署,餘則否。後十日不雨,乃徒市,禁屠殺,斷傘扇,造土龍以祈。雨足則報祀,送龍水中。十七年夏六月,京畿久雨,遵祈雨儀,命諸寺觀啟道場祈禱。



 ○拜天



 金因遼舊俗,以重五、中元、重九日行拜天之禮。重五於鞠場,中元於內殿,重九於都城外。其制,刳木為盤,如舟狀,赤為質,畫雲鶴文。為架高五六尺,置盤其上,薦食物其中,聚宗族拜之。若至尊則於常武殿築臺為拜天所。重五日質明,陳設畢,百官班俟於球場樂亭南。皇帝靴袍乘輦,宣徽使前導,自球場南門入,至拜天臺,降
 輦至褥位。皇太子以下百官皆詣褥位,宣徽贊:「拜。」皇帝再拜。上香,又再拜。排食拋盞畢,又再拜。飲福酒,跪飲畢,又再拜。百官陪拜,引皇太子以下先出,皆如前導引。皇帝回輦至幄次,更衣,行射柳、擊球之戲,亦遼俗也,金因尚之。凡重五日拜天禮畢,插柳、球場為兩行,當射者以尊卑序,各以帕識其枝,去地約數寸,削其皮而白之。先以一人馳馬前導,後馳馬以無羽橫鏃箭射之,既斷柳,又以手接而馳去者,為上。斷而不能接去者,次之。或斷其青處,及中而不能斷,與不能中者,為負。每射,必伐鼓以助其氣。已而擊球,各乘所常習馬,持鞠杖。杖長數尺,
 其端如偃月。分其眾為兩隊,共爭擊一球。先於球場南立雙桓,置板,下開一孔為門,而加網為囊,能奪得鞠擊人網囊者為勝,或曰:「兩端對立二門,互相排擊,各以出門為勝。」球狀小如拳,以輕韌木枵其中而朱之。皆所以習蹺捷也。既畢賜宴,歲以為常。



 ○本國拜儀



 金之拜制,先袖手微俯身,稍復卻,跪左膝,左右搖肘,若舞蹈狀。凡跪,搖袖,下拂膝,上則至左右肩者,凡四。如此者四跪,復以手按右膝,單跪左膝而成禮。國言搖手而拜謂之「撒速」。承安五年五月,上諭旨有司曰:「女直、漢人拜數可以相從者,酌中議之。」禮官奏曰:「《周官》
 九拜,一曰稽首,拜中至重,臣拜君之禮也。乞自今,凡公服則用漢拜,若便服則各用本俗之拜。」主事陳松曰:「本朝拜禮,其來久矣,乃便服之拜也。可令公服則朝拜,便服則從本朝拜。」平章政事張萬公謂拜禮各便所習,不須改也,司空完顏襄曰:「今諸人衽發皆從本朝之制,宜從本朝拜禮,松言是也。」上乃命公裳則朝拜,諸色人便服則皆用本朝拜。



\end{pinyinscope}