\article{列傳第七}

\begin{pinyinscope}

 太祖諸子



 宗雋本名訛魯觀宗傑本名沒里野宗強本名阿魯爽本名阿鄰可喜阿瑣宗敏本名阿魯補元



 太祖聖穆皇后生景宣帝、豐王烏烈、趙王宗傑。光懿皇后生遼王宗乾。欽憲皇后生宋王宗望、陳王宗雋、瀋王
 訛魯。宣獻皇后生睿宗、豳王訛魯朵。元妃烏古論氏生梁王宗弼、衛王宗強、蜀王宗敏。崇妃蕭氏生紀王習泥烈、息王寧吉、莒王燕孫。娘子獨奴可生鄴王斡忽。宗乾、宗望、宗弼自有傳。



 宗雋,本名訛魯觀。天會十四年,為東京留守。天眷元年,入朝,與左副元帥撻懶建議,以河南、陜西地與宋。俄為尚書左丞相,加開府儀同三司,兼侍中,封陳王。二年,拜太堡,領三省事,進封兗國王,既而以謀反,誅。



 宗傑,本名沒里野。天會五年,薨。天會十三年,謚孝悼。天眷元年,追封越王。以其長子奭為會寧牧,封鄧王。後為
 上京留守,再改燕京、西京。皇統三年,薨。子阿楞、撻楞。海陵為相,將謀弒立,構而殺之。海陵篡立,并殺宗傑妻。大定間,贈宗傑太師,進封趙王。



 宗強,本名阿魯。天眷元年,封紀王。三年,代宗固為燕京留守,封衛王,太師。皇統二年十月,薨,輟朝七日。喪至上京,上親臨哭之慟,仍親視喪事。子阿鄰、可喜、阿瑣,爽,本名阿鄰。天德三年,授世襲猛安。正隆二年,除橫海軍節度使,改安武軍,留京師奉朝請。海陵將伐宋,嚴酒禁,爽坐與其弟阿瑣,及從父兄京、徒單貞會飲,被杖,下遷歸化州刺史,奪猛安。未幾,復除安武軍節度使。



 海陵
 渡淮,分遣使者翦滅宗室,爽憂懼不知所出。會世宗即位東京,宗室璋推爽弟阿瑣行中都留守,遣人報爽。爽棄妻子來奔,與弟忻州刺史可喜,俱至中都。東迎車駕,至梁魚務入見,世宗大悅,即除殿前馬步軍都指揮使。封溫王,改秘書監。母憂,尋起復,遷太子太保,進封壽王。



 頃之,世宗第五女蜀國公主下嫁唐括鼎,賜宴神龍殿,謂爽曰:「朕與卿兄弟,在正隆時,朝夕常懼不保,豈意今日賴爾兄弟之福,可以享安樂矣。」爽泣下,頓首謝。未幾,判大宗正事,太子太保如故。



 爽有疾,詔除其子符寶祗候思列為忠順軍節度副使。爽入謝,上曰:「朕以卿疾,使
 卿子遷官,冀卿因喜而愈也。思列年少,未閑政事,卿訓以義方,使有善可稱,別加升擢。」爽疾少間,將從上如涼陘,賜錢千萬,進封英王,轉太子太傅。復世襲猛安,進封榮王,改太子太師。



 顯宗長女鄴國公主下嫁烏古論誼,賜宴慶和殿,爽坐西向,迎夕照,面發赤似醉。上問曰:「卿醉邪?」對曰:「未也,臣面迎日色,非酒紅也。」上悅,顧群臣曰:「此弟出言,未嘗不實,自小如此。」因謂顯宗兄弟曰:「汝等可以為法。」以爽貲用有闕,特賜錢一萬貫。二十三年,爽疾久不愈,敕有司曰:「榮王告滿百日,當給以王俸。」



 既薨,上悼痛,輟朝,遣官致祭,賻銀千兩、重彩四十端、絹四百
 匹。陪葬山陵,親王、百官送葬。他日,謂大臣曰:「榮王之葬,朕以不果親送為恨。」其見友愛如此。



 可喜,以宗室子,累官唐括部族節度使,降忻州刺史。海陵遣使殺之,可喜聞世宗即位,即棄州來歸,與其兄歸化州刺史阿鄰會于中都。是時,弟阿瑣權中都留守事,可喜謂阿鄰曰:「阿瑣愚戇,恐不能撫治,欲少留以助之。」阿鄰乃行。可喜留中都,聞世宗發東京,乃迎見于麻吉鋪。除兵部尚書,佩金牌,將兵往南京。行至中都,聞南京已定,遂止。



 可喜材武過人,狠戾好亂,自以太祖孫,頗有異志。世宗初至中都,倥傯多事,扈從諸軍未暇行賞,或
 有怨言。昭武大將軍斡論,正隆末,被詔佩金牌,取河南兵四百人,監完顏彀英軍于歸化,次彰德。會獨吉和尚持大定赦文至。和尚使人招之,斡論不聽,率兵來迎,和尚亦以所將蒲輦兵,列陣待之。斡論兵皆不肯戰,遂請降。和尚邀之入相州,收其甲兵,置酒相勞,斡論托腹疾,不肯飲。至夜,已張燈,時時出門,與其心腹密謀,欲就執和尚。稍具弓矢,和尚覺之,佯為不知,使其從者迫而伺之,斡論不得發。上至中都近郊,斡論上謁,上亦撫慰之。斡論自慊,初無降志。及河南統軍司令史斡里朵,為人狡險,憙圖事,斡論取兵于河南統軍使陁滿訛里也,斡
 里朵與俱來,俱不自安。同知延安尹李惟忠,與熙宗弒逆,構殺韓王亨,世宗疏斥之。同知中都留守璋,初自領其職,因而授之。完顏布輝為副統,以罪解職,居京師。於是可喜、斡論、李惟忠、斡里朵、璋、布輝謀,欲因扈從軍士怨望作亂。斡論曰:「押軍猛安沃窟刺,必不違我。」惟忠曰:「惟忠嘗為神翼軍總管,有兩銀牌尚在,可以矯發內藏賞士。萬戶高松與我舊,必見聽。」眾曰:「若得此軍,舉事無難矣。」斡論往約沃窟刺,沃窟刺從之。惟忠旆說高松,高松不聽,語在《松傳》。



 大定二年正月甲戌,上謁山陵,可喜中道稱疾而歸。乙亥夜,召斡論、惟忠、斡里朵、璋、布輝會
 其家,沃窟刺以兵赴之,璋曰:「今不高松軍,事不可成矣。」可喜、璋、布輝乃擒斡論、惟忠、斡里朵、沃窟刺,詣有司自首。既下詔獄,可喜不肯自言其始謀,及與斡論面質,然後款伏。上念兄弟少,太祖孫惟數人在,惻然傷之。詔罪止可喜一身,其兄弟子孫皆不緣坐。遂誅斡論、惟忠、斡里朵、沃窟刺等,其沃窟刺下謀克士卒皆釋之。除璋彰化軍節度使,布輝浚州防禦使。辛巳,詔天下。是日,賜扈從萬戶銀百兩,猛安五十兩,,謀克絹十匹,甲士絹五匹、錢六貫,阿里喜以下賜各有差。



 阿瑣,宗強之幼子也。長身多力。天德二年,以宗室子,授
 奉國上將軍,累加金吾衛上將軍,居於中都。



 海陵伐宋,以左衛將軍蒲察沙離只同知中都留守事,佩金牌,守管簽。世宗即位東京,阿瑣與璋等守城軍官烏林荅石家奴等,入留守府,殺沙離只、府判抹捻撒離喝。眾以阿瑣行留守事,璋自署同知留守事,即遣謀克石家奴、烏林荅愿、蒲察蒲查、大興少尹李天吉子磐等,奉表東京。



 大定二年,授橫海軍節度使,剛以名鷹,詔曰:「卿方年少,宜自戒慎,留心政事。」改武定軍,以母憂去官。起復興平軍節度使,賜以襲衣廄馬。遷廣寧尹,坐贓一萬四千餘貫,詔杖八十,削兩階,解職。入見于常武殿,上曰:「朕謂汝
 有才力,使之臨民。今汝在法當死,朕以親親之故,曲為全貸。當思自今戒懼,勿復使惡聲達於朕聽。」改平涼、濟南尹,卒官,年三十七。上命有司致祭,賻銀千兩、重彩四十端、絹四百匹。



 宗敏,本名阿魯補。天眷元年,封邢王。皇統三年,為東京留守,拜左副元帥,兼會寧牧。進拜都元帥,兼判大宗正事。再遷太保,領三省事,兼左副元帥,領行臺尚書省事,封曹國王。



 海陵謀弒立,畏宗敏尊且材勇,欲構誣以除之。時熙宗屢殺大臣,宗敏憂之,謂海陵曰:「主上喜殘殺,而國家事重,奈何。」宗敏言時,適左右無人,海陵將以
 此為指斥構害之,自念無證不可發,乃止。



 及弒熙宗,使葛王宗敏。葛王者,世宗初封也。宗敏聞海陵召,疑懼不敢往。葛王曰:「叔父今不即往,至明日,如何與之相見。」宗敏入宮,海陵欲殺之,尚猶豫,以問左右。烏帶曰:「彼太祖子也,不殺之,眾人必有異議,不如除之。」乃使僕散忽土殺之,忽土刃擊宗敏,宗敏左右走避,膚髮血肉,狼藉遍地。葛王見殺宗敏,問於眾曰:「國王何罪而死?」烏帶曰:「天許大事,尚已行之,此蟣虱耳,何足道者。」天德三年,海陵追封宗敏為太師,進封爵。妃蒲察氏,進國號。封子撒合輦舒國公,賜名褒,進封王;阿里罕封密國公。正隆六
 年,契丹撒八反,海陵遣使殺諸宗室,阿里罕遂見殺。大定間,詔復官爵。



 胙王元,景宣皇帝峻子也,本名常勝,為北京留守,弟查刺為安武軍節度使。



 皇統七年四月戊午,左副點檢蒲察阿虎特子尚主,進禮物,賜宴便殿。熙宗被酒,酌酒賜元,元不能飲,上怒,仗劍逼之,元逃去。命左丞宗憲召元,宗憲與元俱去,上益怒,是時戶部尚書宗禮在側,使之跪,手殺之。



 海陵與唐括辯謀廢立,海陵曰:「若舉大事,誰當立者。」海陵意謂己乃太祖長房之孫,當立。而辯與秉德初意不在海陵,常勝乃熙宗之弟,辯答曰:「無胙王
 常勝乎。」海陵復問其次,辯曰:「鄧王子阿楞。海陵曰:「阿楞屬疏。」由是海陵謂胙王有人望,不除之將不得立,故心忌常勝並阿楞。是時,阿楞方為奉國上將軍。



 河南軍士孫進自稱「皇弟按察大王」,熙宗疑「皇弟」二字或在常勝也,使特思鞫之,無狀。特思乃嘗疑海陵與唐括辯時時竊議,告之悼后者。海陵知熙宗有疑常勝心,因此可以除之,謂熙宗曰:「孫進反有端,不稱他人,乃稱皇弟大王。陛下弟止有常勝、查刺。特思鞫不以實,故出之矣。」熙宗以為然,使唐括辯、蕭肄按問特思,特思自誣服,故出常勝罪。於是,乃殺常勝及其弟查刺,并殺特思。海陵乘此并
 擠阿楞殺之。阿楞弟撻楞,熙宗本無意殺之,海陵曰:「其兄既已伏誅,其弟安得獨存。」又殺之。熙宗以海陵為忠,愈益任之,而不知其詐也。



 海陵篡立,追封常勝、查刺、阿楞官爵,親臨葬所致祭。大定十三年六月丁巳,世宗召皇太子諸王,侍食于清輝殿,曰:「或稱海陵多能,何也。海陵譎詐,睢盱殺人,空虛天下三分之二。太祖諸孫中,惟胙王元天性賢者也。」



 元子育,本名合住,大定二十七年,自南京副留守遷大宗正丞,兼勸農副使。上問宰臣曰:「合住為人如何?」平章政事襄、參政宗浩對曰:「為人清廉幹治。」上曰:「乃父亦然。」又曰:「蒲陽溫胙王元,外若愚訥,臨
 事明敏過人。朕於兄弟間,於元尤款密。」



 贊曰:「太祖躬擐甲胄,以定國家,舉無遺策,而諸子勇略材識,足以遂父之志。傳及太宗,而諸孫享其成矣。



\end{pinyinscope}