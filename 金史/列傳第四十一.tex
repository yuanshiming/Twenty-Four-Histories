\article{列傳第四十一}

\begin{pinyinscope}

 ○完顏仲元完顏阿鄰完顏霆烏古論長壽完顏佐石抹仲溫烏古論禮蒲察阿里奧屯襄完顏蒲剌都夾谷石里哥術甲臣嘉紇石烈桓端完顏阿里不孫完顏鐵哥納蘭胡魯剌



 完顏仲元,本姓郭氏,中都人。大安中,李雄募兵,仲元與
 完顏阿鄰俱應募,數有功。貞祐三年,與阿鄰俱累功至節度。仲元為永定軍節度使,賜姓完顏氏。仲元在當時兵最強,號「花帽軍」,人呼為「郭大相公」,以與阿鄰相別。頃之,兼本路宣撫使。八月,遙授知河間府事。數月,改知濟南府事,權山東東路宣撫副使。



 貞祐四年,山東乏糧,仲元軍三萬欲於黃河之側或陜右分屯,上書乞補京官,且言恢復河朔之策,當詣闕面陳。詔曰:「卿兄弟鳩集義旅,所在立功,忠義之誠,皎然可見。朕以參政侯摯與卿素厚,命於彼中行省,應悉朕心。卿求入見,其意固嘉,東平方危,正賴卿等相為聲援,俟兵勢稍緩,即徙軍附河
 屯駐,此時卿來,蓋未晚也。尚思戮力,朕不汝忘。」未幾,改河北宣撫副使。



 仲元部將李霆等積功至刺史、提控,仲元奏賜金牌,霆等皆為名將,功名與仲元相埒。仲元屢有功,以本職為從宜招撫使,計約從坦等軍圖恢復。詔以仲元軍猥多,差為三等,上等備征伐,中下給戍守,懦弱者皆罷去。紅襖賊千餘人據漣水縣,仲元遣提控婁室率兵擊破之,斬首數百,敗祝春,擒郭偉,餘眾奔潰,遂復漣水縣。仲元兼單州經略使,婁室遷兩階,升職一等。未幾,仲元遙授知歸德府事。



 是歲十月,徙軍盧氏,改商州經略使,權元帥右都監。詔曰:「商、虢、潼關,實相連屬,卿
 思為萬全之計。」未幾,潼關失守,仲元軍趨商、虢,復至嵩、汝,皆弗及。仲元上書曰:「去年六月,臣嘗請於朝廷,乞選名將督諸軍,臣得推鋒,身先士卒,糧儲不繼,竟不果行。今將坐甲待敵,則師老財殫,日就困弊。」其大概欲伐西夏以張兵勢。又曰:「陜西一路,最為重地,潼關、禁坑及商州諸隘,俱當預備。向者中都,居庸最為要害,乃由小嶺、紫荊繞出,我軍腹背受兵,卒不能守。近日由禁坑出,遂失潼關。可選精兵分地戍之。」其後乃置秦、藍守禦,及用兵西夏矣。



 興定元年,復為單州經略使,敗宋人二千于龜山,復敗步騎千餘于盱眙,敗紅襖於白里港,獲老幼
 萬餘人,皆縱遣之。宋人圍海州,仲元軍高橋,令提控完顏阿鄰領騎繞出其後夾擊之。宋兵解去。賜金帶,優詔獎諭。紅襖賊陷曹馬城,剽掠徐、單之間。提控高琬等分兵擊之,俘生口二千。三年,仲元奏:「州城既固,積糧二十萬石,集鄉義軍萬餘人,並閑訓練,足以守禦,乞以所部渡河。」詔屯宿州,與右都監紇石烈德同行帥府事。仲元有足疾,滿百日,詔曰:「卿處置機務,撫存將士,出兵使李辛可也。」四年,兼保靜軍節度使,尋為勸農使。五年,為鎮南節度使。



 元光元年,知鳳翔府事。鳳翔被圍,左監軍石盞合喜來濟軍。仲元讓合喜總兵事。合喜曰:「公素得眾
 心,不必以官位見讓。」仲元請身先士卒,諭諸將士曰:「凡有奇功者,即承制超擢。」及危急乃輒注四品以下。顏盞蝦蟆力戰功最,輒授通遠軍節度使。圍解,奏請擅除拜之罪。宣宗嘉其功,皆許之。遷元帥右監軍,授河北東路洮委必剌猛安,賜金五十兩、重幣十五端、通犀帶,優詔褒諭。正大間,為兵部尚書,皇太后衛尉,卒。仲元為將,沈毅有謀,南渡後最稱名將云。



 完顏阿鄰,本姓郭氏,以功俱賜姓完顏。大安中,李雄募兵,阿鄰與完顏仲元等俱應募,數有功。宣宗即位,遷通州防禦使。宣宗遷汴,阿鄰改同知河間府事兼清州防
 禦使,將所部兵駐清、滄,控扼山東。遷橫海軍節度使,賜以國姓。阿鄰與山東路宣撫副使顏盞天澤不相能,詔阿鄰當與天澤共濟國事,無執偏見,妄分彼此。尋改泰定軍節度使、山東西路宣撫使。是時,仲元亦積功勞,知濟南府,賜姓完顏,與阿鄰俱加從宜招撫使,詔書獎諭,且令計約涿州刺史從坦等軍恢復中都。於是,仲元、阿鄰部兵猥多,詔以三等差第之,上等備征伐,中下戍守,懦弱者罷去,量給地以贍其家。阿鄰所部「黃鶴袖軍」駐魚臺者,桀驁不法,掠平民,劫商旅,道路不通,有司乞徙於滕州。詔阿鄰就處置之。頃之,破紅襖賊郝定于泗水
 縣柘溝村,生擒郝定,送京師斬之。



 近制,賜本朝姓者,凡以千人敗敵三千者賜及緦麻以上,敗二千人以上者賜及大功以上,敗千人以上賜止其家。阿鄰既賜姓,以兄守楫及從父兄弟為請。宰臣奏阿鄰功止賜一家,宣宗特詔許之。至是仲元上奏曰:「臣頃在軍旅,纔立微功,遽蒙天恩,賜之國姓,非臣殺身所能仰報。族兄徐州譏察副使僧喜、前汾州酒同監三喜、前解州鹽管勾添章、守興平縣監酒添福猶姓郭氏。念臣與僧喜等昔同一家,今為兩族,完顏阿鄰與臣同功,皇恩所加併及本族,僧喜等四人乞依此例。」不許。改輝州經略使。



 阿鄰有
 眾萬五千,詔分五千隸東平行省,其眾泣訴云:「我曹以國家多難,奮義相從,捐田宅,離親戚,轉戰至此,誓同立功,偕還鄉里。今將分配他軍,心實艱苦。乞以全軍分駐懷、衛、輝州之間,捍蔽大河,惟受阿鄰節制。」阿鄰亦不欲分之,因以為請。宰臣奏:「若遂聽之,非唯東平失備,他將仿傚,皆不可使矣。」宣宗以為然。加遙授知河南府事,應援陜西。阿鄰將兵八千,西赴至潼關,聞京兆已被圍,游騎至華州,陜西行院欲令阿鄰駐軍商、虢,拒東向之路。阿鄰上奏:「臣本援陜西,遇難而止,豈人臣之節?夫自古用兵,步騎相參,乃可以得志。今乃各有所屬,臨難不救,
 互分彼此。今臣所統皆步卒,願賜馬軍千人,則京兆之圍不足解矣。」宣宗謂皇太子曰:「阿鄰赴難不回,固善矣。而軍勢單弱,且駐內地以觀事變,併以虢州兵五千付之,使乘隙而進,卿以此意諭之也。」



 興定元年,遷元帥右都監,出秦州伐宋。宋統制吳筠守皁角角又作郊堡,城三重,據山之巔。阿鄰分兵絕其汲路,克其外城,再克其次城。宋兵縱火而出,阿鄰以騎兵邀之,遣步卒襲其後,宋兵敗,生獲吳筠及將校二百人,馬數百匹,糧萬石及兵甲衣襖。復敗宋兵於裴家莊六谷中,斬五百級,墜澗死者甚眾。又敗之于寒山嶺、龍門關、大石渡,得粟二千餘石。
 復敗之於稍子嶺,斬首二千餘級,生擒百人。是時三月,宿麥方滋,阿鄰留兵守之。已而宋兵大至,金兵敗,阿鄰戰沒。贈金紫光祿大夫、西京留守。



 完顏霆,本姓李氏,中都寶坻人。粗知書,善騎射,輕財好施,得鄉曲之譽。貞祐初,縣人共推霆為四鄉部頭。霆招集離散,糾合義兵,眾賴以安。招撫司奏其事,遷兩官。霆與弟雲率眾數千巡邏固安、永清間,遙授寶坻縣丞,充義軍都統。劉璋說霆使出降,霆縛送經略司。遷三階,攝寶坻令,升都提控,遙授同知通州軍州事。



 中都食盡,霆遣軍分護清、滄河路,召募賈船通餉道。遙授同知清州
 防禦事,從河北路宣撫使完顏仲元保清、滄。遙授通州刺史、河北東路行軍提控,佩金牌。舊制,宣撫副使乃佩金牌,仲元奏:「臣軍三萬,管軍官三人,皆至五品,乞各賜金牌。」廷議霆輩忠勇絕人,遂與之。改大名路提控,復取玉田、三河、香河三縣。徙屯濱、棣、淄,留副將孫江守滄州。江以滄州降于王楫,而江將兵圍觀州。霆乃詐作書與孫江,約同取滄州者。王楫得其書,果疑孫江與霆有謀,召江還,殺之。霆乃定觀州而還。進官三階,充濱、棣行軍都提控。未幾,遙授同知益都府事,加宣差都提控,遷棣州防禦使,賜姓完顏氏,屯海州。俄權單州經略司事,充
 宣差總領都提控。



 興定元年,泰安、滕、兗土寇蜂起,東平行省侯摯遣霆率兵討之,降石花五、夏全餘黨二萬人,老幼五萬口,充權海州經略副使。紅襖賊於忙兒寇海州,霆擊走之。二年,宋高太尉兵三萬駐朐山。霆軍乏糧,採野菜麥苗雜食之。宋兵柵朐山,下隔湖港,霆作港中暗橋,遣萬戶胡仲珪、副統劉斌率死士由暗橋登山,霆率兵四千人趨山下,約以昏時舉火為期,上下夾擊,宋兵大敗,墜澗溺水死者不可勝計,斬高太尉、彭元帥于陣,餘眾潰去。遷安化軍節度使,經略副使如故。以其子為符寶典書。逾月,宋兵復至,霆逆戰,駐兵城外。夜半,宋
 人乘虛踰城而入。經略使阿不罕奴失剌率兵扼戰,都統溫迪罕五兒、副統蒲察永成、蒲察只魯身先士卒,殺二百餘人,城賴以完。詔五兒等各遷兩階。



 四年,改集慶軍節度使,兼同知歸德府事。五年,改定國軍節度使,兼同知京兆府事,擢其子為護衛。元光元年,陜西行省白撒奏:「京兆南山密邇宋境,官民遷避其間者,無慮百萬人。可遣官鎮撫,庶幾不生他變。」宣宗以為然。十月,霆以本官為安撫使,守同知歸德府惟宏、大司農丞郭皓為副使,分護百姓之遷南山者。元光二年,卒。



 烏古論長壽,臨洮府第五將突門族人也。本姓包氏,襲
 父永本族都管。泰和伐宋,充緋翮翅軍千戶,取褷川寨及祐州、宕昌、辛城子,以功進官二階。貞祐初,夏人攻會州,統軍使署征行萬戶,陞副統,與夏人戰於窄土峽,先登陷陣,賞銀五十兩。戰東關堡,以功署都統,兼充安定、定西、保川、西寧軍馬都彈壓。詔錄前後功,遙授同知隴州防禦事,世襲本族都巡檢。三年,賜今姓。攻蘭州程陳僧,為先鋒都統。夏人圍監洮,扼渭源堡,內外不通。統軍司募人偵候臨洮消息,長壽應募,馘二人,擒一人,問得臨洮及夏兵事勢。以勞遷宣武將軍,遙授通遠軍節度副使。招降諸蕃族及熟羊寨秦州逋亡者。復遷懷遠大將
 軍,升提控。興定元年,夏人大入隴西,長壽拒戰,遷平涼府治中,兼節度副使,充宣差鞏州規措官。頃之,遙授同知鳳翔府事,兼同知通遠軍節度事,提控如故。



 興定二年,遷同知臨洮府事。與提控洮州刺史納蘭記僧分兵伐宋。長壽由鹽川鎮進兵,宋人守戍者走保馬頭山,合諸部族兵來拒。長壽擊敗之,復破其援兵四千於荔川寨。即趨宕昌縣。破宋兵二千于八斜谷,拔宕昌縣,進攻西和州,先敗其州兵。明日,木波兵三千與宋兵合,依川為陣,長壽奮擊,宋兵入保城,堅壁不復出,長壽乃還。凡斬馘八千,獲馬二百餘、牛羊三萬,器械軍實甚多。納蘭
 記僧出洮州鐵城堡,屢敗宋人,完軍而還。詔賞鳳翔、秦、鞏伐宋將士,長壽遙授隴安軍節度使,同知通遠軍、提控如故。頃之,長壽升總領都提控,改通遠軍節度使。



 夏人攻定西,是時弟世顯已降夏人,夏人執世顯至定西城下,謂長壽曰:「若不速降,即殺汝弟。」長壽不顧,奮戰。夏兵退,加榮祿大夫,賜金二十五兩、重幣三端。世顯既降,二子公政、重壽當緣坐。宣宗嘉長壽守定西功,釋公政兄弟,有司廩給之。詔長壽曰:「汝久在戎行,盡忠國事。世顯之降,必不得已,汝永念國恩,益思自效。」未幾,夏人復攻會州,行元帥府事石盞合喜發兵救未至,夏人移兵
 臨洮,長壽伏精兵五千於定西險要間,敗夏兵三萬騎,殺千餘人,獲馬數百。夏人已破西寧,乃犯定西,長壽擊卻之,斬首三百級。既而三萬騎復至,攻城甚急。長壽乘城拒戰,矢石如雨,夏兵死者數千,被創者眾,乃解去。是歲卒。



 完顏佐,本姓梁氏,初為武清縣巡檢。完顏咬住,本姓李氏,為柳口鎮巡檢。久之,以佐為都統,咬住副之,戍直沽寨。貞祐二年,颭軍遣張暉等三人來招佐,佐執之。翌日,劉永昌率眾二十人持文書來,署其年曰天賜,佐擲之,麾眾執永昌,及暉等併斬之。宣宗嘉其功,遷佐奉國上
 將軍,遙授德州防禦使,咬住鎮國上將軍,遙授同知河間府事,皆賜姓完顏氏。詔曰:「自今有忠義如是者,並一體遷授。」



 贊曰:古者天子胙土命氏,漢以來乃有賜姓。宣宗假以賞一時之功,郭仲元、郭阿鄰以功皆賜國姓。女奚烈資祿、烏古論長壽皆封疆之臣而賜以他姓。貞祐以後,賜姓有格。夫以名使人,用之貴則貴,用之賤則賤,使人計功而得國姓,則以其貴者反賤矣。完顏霆、完顏佐皆賜國姓者,併附于此。



 石抹仲溫,本名老斡,懿州胡土虎猛安人。充護衛十人
 長、太子僕正,除同知武寧軍節度使事、宿直將軍、器物局使。坐前在武寧造馬鞍虧直,章宗原之,改左衛將軍,遷左副點檢。坐徵契丹逗遛,降蔡州防禦使。復召為左副點檢,遷知臨洮府事。泰和伐宋,青宜可內附,進爵二級,賜銀二百五十兩、重幣十端。詔曰:「青宜可之來,乃汝管內,與有勞焉。比與青宜可相合,其間諸事量宜而行。」頃之,諸道進兵,仲溫以隴右步騎五千出鹽川。八年,罷兵,改知河中府。崇慶初,遷陜西統軍使。貞祐二年,宋人攻秦州,促溫率兵敗之。尋充本路安撫使,改鎮南軍節度使。致仕。興定三年,卒。



 烏古論禮,本名六斤,益都猛安人。充習騎,累擢近侍局直長,轉本局副使、左衛副將軍。坐受沁南軍節度使兗王永成名馬玉帶,杖一百,削官解職。起為蒲速碗群牧副使,改武庫署令、宿直將軍,復為左衛副將軍、順州刺史,累遷武寧軍節度。泰和伐宋,為山東路兵馬都統副使兼副統軍、安化軍節度。八年,宋人請盟,罷兵馬都統官,仍以節度兼副統軍。大安三年,改知歸德府兼河南副統軍,歷知河南府。至寧初,改知太原府事。貞祐二年,兼河東北路安撫使。三年,充本路宣撫使,頃之,兼左副元帥。四年,太原被圍,未幾圍解,進官二階。興定三年,卒。



 蒲察阿里,興州路人。以蔭補官,充護衛十人長、武器署令,轉宿直將軍,遷右衛副將軍。宋兵犯分道鋪,馳驛赴邊,伺其入,以伏兵掩之。改提點器物局。泰和伐宋,從右副元帥匡為副統,攻宜城縣,取之。八年,以功遷武衛軍副都指揮使。大安元年,同知南京留守事,徙壽州防禦使,遷興平軍節度使。崇慶初,遷元帥右都監,明年,轉左都監。時都城被圍,道路梗塞,阿裏由太原至真定,率師赴援,抵中山,不克進。貞祐二年,移駐大名。征河南鎮防軍圖再舉,眾既憚于行,而阿里遇之有厚薄,軍變,遇害,眾因逃散。宣宗詔元帥左都監完顏弼安集其軍,赦首
 惡以下,河南統軍司更加撫諭。



 奧屯襄,本名添壽,上京路人。大定十年,襲猛安。丞相襄舉通練邊事,授崇義軍節度副使,改烏古里颭詳穩,召為都水少監、石州刺史。未幾,為平南盪江將軍,以功升壽州防禦使,遷河南路副統軍兼同知歸德府事、昌武軍節度使,仍兼副統軍。崇慶改元,為元帥左都監,救西京,至墨谷口,一軍盡殪,襄僅以身免,坐是除名。明年,授上京兵馬使。宣宗即位,擢遼東路宣撫副使。未幾,改速頻路節度使,兼同知上京留守事。二年二月,為元帥右都監,行元帥府事于北京。五月,改留守,兼前職,俄遷宣
 撫使兼留守。



 十一月,詔諭襄及遼東路宣撫使蒲鮮萬奴、宣差蒲察五斤曰:「上京、遼東,國家重地,以卿等累效忠勤,故委腹心,意其協力盡公,以徇國家之急。及詳來奏,乃大不然,朕將何賴。自今每事同心,併力備禦,機會一失,悔之何及!且師克在和,善鈞從眾,尚懲前過,以圖後功。」三年正月,襄為北京宣差提控完顏習烈所害。未幾,習烈復為其下所殺,詔曲赦北京。



 完顏蒲剌都,西南路按出灰必剌罕猛安人。充護衛,除泰定軍節度副使。以憂去官,起復唐古部族節度副使,徙安國軍,移颭詳穩,累官原州刺史。坐買部內馬虧直,
 奪官一階,降北京兵馬都指揮使、寧遠軍刺史,歷同知臨洮府、西京留守事。崇慶元年,遷震武軍節度,備禦有功,遷一官。貞祐初,置東西面經略司,就充西面經略使,上言:「管內大和嶺諸隘屯兵,控制邊要。行元帥府輒分臣兵萬二千戍真定,餘眾不足守禦,近日復簡精銳二千七百人以往。今見兵不滿萬,老羸者十七八。臣死固不足惜,顧國家之事不可不慮,新設經略移文西京、太原、河東取軍馬,大數並稱非臣所統。」詔真定元帥府還其精銳二千七百人。西京、太原、嵐州有警急,約為應援。州郡皆不欲屬經略司,遂罷經略官,入為簽樞密院事,改左
 副點檢。四年,遷兵部尚書,興定元年,致仕。四年,卒。



 夾谷石里哥,上京路猛安人。明昌五年進士,泰州防禦判官,補尚書省令史,歷臨潢、婆速路都總管判官,累除刑部主事,改薊州副提控,駐軍大名。俄遷翰林待制,為宿州提控。與山東宣撫完顏弼攻大沫堌,賊眾千餘逆戰,石里哥以騎兵擊之,盡殪。提控沒烈入自北門,遂擒劉二祖。以功遷武衛軍副都指揮使。坐前在宿州掠良人為生口,當死,特詔決杖八十。徙洺州防禦使、山東路副統軍。坐不時進兵,往宿遷取妻子,解職。起為東平行軍提控。興定元年,破宋兵于宿州,以功遙授安化軍節
 度使,移定海軍,卒。



 術甲臣嘉,北京路猛安人,襲父謀克。泰和伐宋,隸陜西完顏綱麾下。歷通州、海州同知軍州事。貞祐二年,除武器署丞。救集寧有功,遷河南統軍判官、拱衛直副都指揮使、河南治中,遙領綏州刺史兼延安治中,就遷同知府事,改同知河間府事。興定元年,行樞密院于壽州,由壽、泗渡淮伐宋。二年二月,破宋兵三千於漸湖灘,斬三百級。有詔蹂踐宋境上,毋深入。臣嘉駐霍丘楂岡村,縱輕騎鈔掠,焚毀積聚。獲宋諜者張聰,知宋兵二千屯高柳橋,老幼甚眾,其寨兩城,環之以水。臣嘉遣張聰持牒招之,
 不從。先令水軍徑渡攻之。軍士牛青操戈刺門卒,皆披靡散去,遂登陴,大軍繼之,夷其寨而還。遇宋兵數千於梅景村。臣嘉伏兵林間,以步卒誘致之,伏發,宋兵潰,追奔十餘里,生擒其將阮世安等五人,獲器仗甚眾。七月,賞征南功,升職一等,遷元帥右都監,充陜西行省參議官。四年,兼金安軍節度使。五年,改知延安府事,轉左都監,駐兵京兆。元光元年,卒。



 紇石烈桓端,西南路忽論宋割猛安人,襲兄銀術可謀可。泰和伐宋,充行軍萬戶,破宋兵二千於蔡州,加宣武將軍。自壽州渡淮,敗宋步騎一萬五千于鷂子嶺,遂克
 安豐軍。軍還,除同知懷遠軍節度事,權木典颭詳穩。大安三年,西京行省選充合扎萬戶,遙授同知清州防禦事,改興平軍節度副使,遙授顯德軍節度副使,徙遼東路宣撫司都統。敗移剌留哥萬五千眾於御河寨,奪車數千兩,降萬餘人。加驃騎衛上將軍,遙授同知順天軍節度事。



 貞祐二年,為宣差副提控,同知婆速路兵馬都總管,行府事。貞祐三年,蒲鮮萬奴取咸平、東京沈、澄諸州,及猛安謀克人亦多從之者。三月,萬奴步騎九千侵婆速近境,桓端遣都統溫迪罕怕哥輦擊卻之。四月,復掠上京城,遣都統兀顏缽轄拒戰。萬奴別遣五千人攻望
 雲驛,都統奧屯馬和尚擊之。都統夾谷合打破其眾數千于三叉里。五月,都統溫迪罕福壽攻萬奴之眾於大寧鎮,拔其壘,其眾殲焉。九月,萬奴眾九千人出宜風及湯池,桓端率兵與戰,其眾潰去,因招奄吉斡、都麻渾、賓哥、出臺、答愛、顏哥、不灰、活拙、按出、孛德、烈鄰十一猛安復來附,擇其丁男補軍,攻城邑之未下者。貞祐四年,桓端遣王汝弼由海道奏事,宣宗嘉其功,桓端遷遼海軍節度使、同知行府事,宣差提控如故。婆速路溫甲海世襲猛安、權同知府事溫迪罕哥不靄遷顯德軍節度使,兼婆速府治中。權判官、前脩起居注裴滿按帶遷兩階,
 升二等。王汝弼遷四階,陞四等。餘將士有功者,詔遼東宣撫承制遷賞。是歲,改邳州刺史,充徐州界都提控。



 紅襖賊數萬攻邳州,桓端破之於黃山。賊復來,桓端薄其營,走保北山,追擊敗之,溺沂水死者甚眾。賊數萬圍沂州,同知防禦事僕散撒合突圍出求救,桓端率兵赴之。撒合還入沂州,與桓端內外夾擊之,殺萬餘人,賊乃去。樞密副使僕散安貞上其功,因奏曰:「桓端天資忠實,深有計畫,曉習軍事,撒合勇而有謀,皆得軍民心,乞加擢用。」桓端進金紫光祿大夫,兼同知武寧軍節度事,提控如故。召為勸農副使,充都提控,屯陳州。



 興定元年,自新
 息渡淮伐宋,破中渡店,至定城,以少擊眾,戰不留行。未幾,充宣差參議官,復渡淮,連破宋兵,獲其將沈俊,遷武衛軍副都指揮使。宋人城守不出,分兵攻其山寨水堡,殺獲甚眾。興定二年,遷鎮南軍節度使,權元帥右都監。數月,改武衛軍都指揮使,仍權右都監,行元帥府于息州。



 徐州行樞密院石盞女魯歡剛愎自用,詔桓端以本官權簽樞密院事,往代之。四年冬,上言:「竊聞宋人與李全將併力來攻,當預為之防。」樞密院奏可,召桓端與朝臣面議。尋有疾,賜太醫御藥。五年正月,召至京師,疾病不能入見,力疾草奏,大略以南北皆用兵,當豫防其患,
 及防河數策。無何,卒,年四十五。敕有司給喪事。



 完顏阿里不孫,字彥成,曷懶路泰申必剌猛安人。明昌五年進士,調易州、忻州軍事判官、安豐縣令。補尚書省令史,除興平軍節度副使,應奉翰林文字,轉修撰,充元帥左監軍紇石烈執中經歷官。執中圍楚州,縱兵大掠,坐不諫正,決杖五十。大安初,改戶部員外郎、鈞州刺史。執中行樞密院於西京,復以為經歷官。改威州刺史。貞祐初,累遷國子祭酒,歷越王、濮王傅,改同知平陽府事,兼本路宣撫副使。召為兵部侍郎,遷翰林侍講學士。改陜西路宣撫副使,遷元帥左都監。改河平軍節度使、河
 北西路宣撫副使。改御史中丞、遼東宣撫副使。再閱月,權右副元帥、參知政事、遼東路行尚書省事,賜御衣、廄馬、安山甲。上京行省蒲察五斤奏其功,賜金百兩、絹百匹。



 興定元年,真拜參知政事,權右副元帥,行尚書省、元帥府于婆速路,承制除拜剌史以下。不協。是時,蒲鮮萬奴據遼東,侵掠婆速之境,高麗畏其強,助糧八萬石。上京行省蒲察五斤入朝,遼東兵勢愈弱,五斤留江山守肇州,江山亦頗懷去就。及上京宣撫使蒲察移刺都改陜西行省參議官,而伯德胡土遂有異志。宣撫使海奴不迎制使,坐而受詔,阿里不孫械繫之。頃之,阿里不孫
 輒矯制大赦諸道,眾乃稍安,而請罪于朝。



 初,留哥據廣寧,知廣寧府事溫迪罕青狗居蓋州,妻子留廣寧,與伯德胡土約為兄弟。青狗兵隸阿里不孫,內猜忌不協,蒲察移剌都嘗奏青狗無隸阿里不孫。宣宗乃召青狗,青狗不受詔,阿里不孫殺之。胡土乃怨阿里不孫。既而胡土率眾伐高麗,乃以兵戕殺阿里不孫。權左都監納坦裕與監軍溫迪罕哥不靄、遙授東平判官參議軍事郭澍謀誅胡土,未敢發,會上京留守蒲察五斤遣副留守夾谷愛答、左右司員外郎抹捻獨魯詣裕計事。裕以謀告二人,二人許諾,遂召胡土至帳中殺之。阿里不孫已
 死,朝廷始得矯赦奏疏,詔有司獎諭。未幾,聞阿里不孫死于亂,詔贈平章政事、芮國公。納合裕真授左都監,哥不靄進一階,愛答、獨魯、郭澍遷官升職有差。



 阿里不孫寬厚愛人,敏於吏事,能治劇要,識者以為用之未盡云。



 完顏鐵哥,性淳直,體貌雄偉,粗通書。年二十四,襲父速頻路曷懶合打猛安。授廣威將軍。御下惠愛。察廉,除臨海軍節度副使,改底剌颭詳穩。丞相襄行省于北京,鐵哥為先鋒萬戶,有功。丁母憂,服除,遷同知武勝軍節度使事,充右副元帥完顏匡副統,號平南盪江將軍。攻光化軍,王統制以步騎出東門逆戰,鐵哥擊卻之,拔鹿角,
 奪門以入,遂克之。進攻襄陽,為前驅,獲生口,知江渡可涉處,陰植標以識之。大軍至,鐵哥導之濟,屢戰皆捷,以勞進官兩階。匡圍德安,鐵哥總領攻城,築壘于德安南鳳凰臺,並城作甬道,立鵝車,對樓攻之,擊走張統制兵。時暑,還屯鄧州。兵罷,進官兩階,遷同知臨潢府事,改西南路副招討、宿州防禦使。貞祐二年,樞密使徒單度移剌以鐵哥充都統,入衛中都。遷東北路招討使,兼德昌軍節度使。



 蒲鮮萬奴在咸平,忌鐵哥兵強,牒取所部騎兵二千,又召泰州軍三千及戶口遷咸平。鐵哥察其有異志,不遣。宣撫使承充召鐵哥赴上京,命伐蒲與路。既
 還,適萬奴代承充為宣撫使,摭前不發軍罪,下獄被害。謚勇毅。



 納蘭胡魯剌,大名路怕魯歡猛安人。性淳直,寡言笑,好讀書,博通今古。承安二年,進士第一,除應奉翰林文字。被詔括牛于臨潢、上京等路。丞相襄有田在肇州,家奴匿牛不以實聞,即械繫正其罪而盡括之。於是豪民皆懼,無敢匿者。使還,襄稱其能。居父喪盡禮,御史舉其清節。服除,轉修撰。平章政事僕散端舉廉能有文采,遷同知順天軍節度使事,從伐宋。以勞加朝請大夫,改禮部員外郎、曹州刺史。豪民僕散掃合立私渡於定陶間,逃兵
 盜劫,皆籍為囊橐,累政莫敢問。胡魯剌捕治之,窮竟其黨,闔郡肅然。改沃州。改南京路按察副使。貞佑二年,改泗州防禦使。召為吏部侍郎,遷絳陽軍節度使,權河東南路宣撫副使。是時兵興,胡魯剌完城郭,繕器械,料丁壯為鄉兵。延問耆老,招致儒士,咨以備御之策。鹽米儲偫,勸富民出粟,郡賴以完。賜詔褒諭,加資善大夫,官其次子吾申。改權經略使,被召,以疾不能行,卒於絳州。



 贊曰:泰和、貞祐,其間相去五年耳,故將遺老往往在焉。高琪得君,宿將皆斥外矣。高汝礪任職,舊臣皆守籓矣。假以重任,其實疏之。故石抹仲溫以下,以見當時之將
 校焉。



\end{pinyinscope}