\article{志第九}

\begin{pinyinscope}

 禮一



 ○郊



 金人之入汴也,時宋承平日久,典章禮樂粲然備具。金人既悉收其圖籍,載其車輅、法物、儀仗而北,時方事軍旅,未遑講也。既而,即會寧建宗社,庶事草創。皇統間,熙宗巡幸析津,始乘金輅,導儀衛,陳鼓吹,其觀聽赫然一新,而宗社朝會之禮亦次第舉行矣。繼以海陵狼顧,志欲併吞江南,乃命官修汴故宮,繕宗廟社稷,悉載宋故
 禮器以還。外行黷武,內而縱欲,其猷既失,奚敢議禮樂哉!世宗既興,復收嚮所遷宋故禮器以旋,乃命官參校唐、宋故典沿革,開「詳定所」以議禮,設「詳校所」以審樂,統以宰相通學術者,於一事之宜適、一物之節文,既上聞而始匯次,至明昌初書成,凡四百餘卷,名曰《金纂修雜錄》。凡事物名數,支分派引,珠貫棋布,井然有序,炳然如丹。又圖吉、凶二儀:鹵簿十三節以備大葬,小鹵簿九節以備郊廟。而命尚書左右司、春官、兵曹、太常寺各掌一本,其意至深遠也。是時,宇內阜安,民物小康,而維持幾百年者實此乎基。鳴呼,禮之為國也信矣夫!而況《關雎》、《
 麟趾》之化,其流風遺思被於後世者,為何如也。宣宗南播,疆宇日蹙,旭日方升而爝火之燃,蔡流弗東而餘燼滅矣!圖籍散逸既莫可尋,而其宰相韓企先等之所論列,禮官張暐與其子行簡所私著《自公紀》,亦亡其傳。故書之存,僅《集禮》若干卷,其藏史館者又殘缺弗完,姑掇其郊社宗廟諸神祀、朝覲會同等儀而為書,若夫凶禮則略焉。蓋自熙宗、海陵、衛紹王之繼弒,雖曰「鹵簿十三節以備大葬」,其行乎否耶?蓋莫得而考也,故宣孝之喪禮存,亦不復紀。噫!告朔餼羊雖孔子所不去,而史之缺文則亦慎之。作《禮志》。



 南北郊



 金之郊祀,本於其俗有拜天之禮。其後,太宗即位,乃告祀天地,蓋設位而祭也。天德以後,始有南北郊之制,大定,明昌其禮浸備。



 南郊壇,在豐宜門外,當闕之巳地。圓壇三成,成十二陛,各按辰位。濆墻三匝,四面各三門。齋宮東北,廚庫在南。壇、濆皆以赤土圬之。北郊方丘,在通玄門外,當闕之亥地。方壇三成,成為子午卯酉四正陛。方濆三周,四面亦三門。朝日壇曰大明,在施仁門外之東南,當闕之卯地,門濆之制皆同方丘。夕月壇曰夜明,在彰義門外之西北,當闕之酉地,掘地汙之,為壇其中。常以冬至日合祀昊天上帝、皇地祗於圜丘,夏至
 日祭皇地祇於方丘,春分朝日於東郊,秋分夕月於西郊。



 大定十一年始郊,命宰臣議配享之禮。左丞石琚奏曰:「按《禮記》:『萬物本乎天,人本乎祖,此所以祖配上帝也。』蓋配之者,侑神作主也。自外至者無主不止,故推祖考配天,尊之也。兩漢、魏、晉以來,皆配以一祖。至唐高宗,始以高祖、太宗崇配。垂拱初,又加以高宗,遂有三祖同配之禮。至宋,亦嘗以三帝配,後禮院上議,以為對越天地,神無二主,由是止以太祖配。臣謂冬至親郊宜從古禮。」上曰:「唐、宋以私親,不合古,不足為法。今止當以太祖配。」又謂宰臣曰:「本國拜天之禮甚重。今汝等言依古制築
 壇,亦宜。我國家絀遼、宋主、據天下之正,郊祀之禮豈可不行?」乃以八月詔曰:「國莫大于祀,祀莫大於天,振古所行,舊章咸在。仰惟太祖之基命,詔我本朝之燕謀,奄有萬邦,于今五紀。因時制作,雖增飾於國容,推本奉承,猶未遑于郊見。況天休滋至而年穀屢豐,敢不敷繹曠文、明昭大報。取陽升之至日,將親享于圓壇,嘉與臣工,共圖熙事。以今年十一月十七日有事于南郊,咨爾有司,各揚乃職,相予肆祀,罔或不欽。」乃於前一日,遍見祖宗,告以郊祀之事。其日,備法駕鹵簿,躬詣郊壇行禮。



 儀注



 齋戒:用唐制。大祀,散齋四日,臻齋三日。中祀,散齋二
 日,致齋一日。



 天子親祀,皆前期七日,攝太慰誓亞終獻官、親王、陪祀皇族於宮省。皇族十五以上,官雖不至七品者亦助祭受誓。又誓百官於尚書省。攝太尉南向。司徒北向,監祭御史在西,監禮博士居東,皆相向。太常卿、光祿卿在司徒後,重行北向。司天監、光祿丞、太廟令丞、大樂令丞、太官令丞、良醖令、廩犧令、郊社丞、司尊、太祝、奉禮郎、協律郎、諸執事官皆重行西上北向。禮直官以誓文授攝太尉,乃誓曰:「維某年歲次某甲,某月,某日,某甲,皇帝有事於南郊,各揚其職。其或不恭,國有常刑。」禮直官贊曰:「七品以下官皆退。」餘皆再拜,退。誓於宮省之儀皆同。
 於是,皇帝散齋於別殿。前致齋一日,尚舍設御坐於大安殿,當中南向。設東西房於御坐之側,設御幄於室內,施簾於楹下。享前三日,陳設小次。享前一日,設拜褥,及皇帝版位、皇帝飲福位,及黃道氈褥,自玉輅下至升輿所。及致齋之日,通事舍人引文武五品以上官,陪位如式。諸侍衛之官,各服其器服,並結佩,俱詣閣奉迎。上水二刻,侍中版奏「外辦」。皇帝服袞冕,結佩,乘輿出,警蹕、侍衛如常儀。皇帝即御座,東向坐。通事舍人承傳,殿上下俱拜,訖,西面,贊「各祗候」。一刻頃,侍中跪奏:「臣某言,請降就齋。」俛伏,興,還侍位。皇帝降座,入室,群官皆退。諸執事
 官皆宿於正寢,治事如故,不弔喪問疾,不判署刑殺文字,不決罰罪人,不與穢惡事。致齋日,惟祀事則行,餘悉禁。已齋而闕者,通攝行事。



 陳設:前祀五日,儀鸞、尚舍陳設齋宮。有司設扈從侍衛次於宮東西。又設陪祀親王次宮東稍南,西向北上,宗室子孫位於其後。又設司徒亞終獻行事執事官次於壇南外濆門之西,東向北上,重行異位。又設天名房,在壇南外濆門之東,西向。大禮使次於其後,皆西向。又設席大屋於壇外西北,駐車輅以備風雪。



 祀前三日,尚舍設大次於東濆外門內道北,南向。又設小次於壇下卯陛之北,南向。有司設饌幔
 於東濆中門之北,南向。設兵衛,各服其器服,守衛濆門,每門二人。郊社令帥其屬,埽除壇之上下及濆之內外。乃為燎位,在南中濆東門之東,壇之巳位。又為瘞坎,在中濆內戌位。祀前二日,太樂令帥其屬,設登歌之樂於壇上稍南,北向。玉磬在午陛之西,金鐘在午陛之東,柷一在鐘前稍北,敔一在磬前稍北,東西相向,歌工次之,餘工各位於縣後。琴瑟在前,匏竹在後,於壇下第一等上,皆重行異位,北向。又設宮縣樂南濆外門之外,八佾二舞表於樂前。又設《采茨》樂於應天門前。祀前一日,奉禮郎升設皇帝版位於壇上辰巳之間,北向。又設皇帝
 飲福位於其左稍卻,北向。又帥禮直官設亞終獻位於卯陛之東北,西向北上。司徒位於卯陛之東,道南西向。禮部尚書、太常卿、光祿卿、禮部侍郎位各次之,太常丞、光祿丞又次之。又設大禮使位於小次之左少卻,西向。又設分獻官、司天監、讀冊中書侍郎位於中濆門道北,西向。郊社令、廩犧令、太官令、良醖令位於其後。又設郊社丞、太祝、奉禮郎以下諸執事官位於其後,皆西向,重行異位。又設從祀文武群官一品至五品位於中濆門內道南,西向,皆重行立。又設助奠祝史齋郎位於東濆門外道北,西向。又設陪祀皇族於道南,西向。六品至九品
 從祀群官,又於其南,皆西向,重行異位,各依其品。又設監祭御史二員,一員在午陛之西南,一員在子陛之西北,皆東向。又設監禮博士二員,一員在午陛之東南,一員在子陛之東北,皆西向。又設太樂令位於樂虡之間稍東,西向。協律郎位於樂虡之西,東向。又設奉禮郎位於壇南稍東,西向。贊者次之。司尊位於酌尊所,俱北向。又設牲榜於外濆東門之外,西向。饌榜於其北稍西,南向。牲榜之東,牲位。太史、太祝各位於牲後,俱西向。又設禮部尚書、太常卿、光祿卿位於牲榜南稍北,西向。太常丞、光祿丞、太官令位於其後。監祭御史、監禮博士於禮部
 尚書位之西稍卻,北向。廩犧令位在牲位西南,北向。又陳禮饌於榜之前案上。



 未後三刻,陳饌之時,又設禮部尚書、太常卿、光祿卿位於案前稍東,北上,西向。太常丞、光祿丞、太官令位於其後,西向。又設監祭御史、監禮博士位於案前稍西,北上,東向。又設異寶嘉瑞位於宮縣西北,太府少監位於寶後。諸州歲貢位於宮縣東北,戶部郎中位於其後。天子八寶位於宮縣西南,符寶郎八員各於寶後。伐國毀寶位於宮縣東南,少府少監位於其後。又設大樂令位於宮縣之北稍東,協律郎二在大樂令南,東西相向。司天監,未後二刻,同郊社令升設
 昊天上帝、皇地祗神座於壇上北方南向,地祗位在東稍卻,席皆以槁秸。太祖配位座於東方西向,席以蒲越。五方帝、日、月、神州地祗、天皇大帝、北極神座於壇上第一等,席皆槁秸,內官五十四座、五神、五官、嶽鎮海瀆二十九座於壇上第二等,中官一百五十有八座、崑崙、山林川澤二十一座於壇上第三等,外官一百六座、丘陵墳衍原隰三十座於內濆之內,眾星三百六十座在內濆之外,席皆以莞。神座版各設於座首。又設禮神玉。俟告潔畢,權徹去壇上及第一等神位,祀日醜前五刻重設。



 奉禮郎同司尊及執事者設天、地、配位各左十有一籩,右
 十有一豆,俱為三行。登三在籩豆間。簠一簋一於登前,簠在左,簋在右。各於神座前藉以席。又設天、地位太尊各二、著尊各二、犧尊各二,山罍各二,壇上東南隅配位著尊二、犧尊二、象奠二,在天、地位酒尊之東,俱北向西上,皆有坫,加勺、冪,為酌尊所。又天、地位象尊各二、壺尊各二、山罍各四,在壇下午陛之南,北向西上。配位壺尊二、山罍四在酉陛之北,東向北上,皆有坫,設而不酌,亦左以明水,右以玄酒。



 又設五方帝、日、月、神州地祗、天皇大帝、北極,第一等皆左八籩、右八豆,登在籩豆間,簠一簋一在登前,爵坫一在神座前。第二等內官五十四座,
 五神、五官、嶽鎮海瀆二十九座,每座邊二、豆二、簠一、簋二、俎一、爵坫一。第三等中官一百五十八座,崑崙、山林川澤二十一座,及內濆內外官一百六座,丘陵墳衍原隰三十座,內濆外眾星三百六十座,每位籩二、豆二、簠一、簋一、俎一、爵一。又設第一等每位太尊二、著尊二、皆有坫加勺。第二等每陛山尊二,第三等每位蜃尊二,內濆內外每辰概尊二,皆加勺。自第二等已下皆用匏爵,先洗拭訖,置於尊所,其尊所皆在神位之左。凡祭器皆藉以席,籩豆各加巾蓋。又設天、地及配位籩一、豆一、簠一、簋一、俎四、及毛血豆各一,并第一等神位每位俎
 二,於饌幔內。又設皇帝洗位於卯陛下,道北,南向。盥洗在東,爵洗在西。匜在東,巾在西。篚南肆,實玉爵坫。又設亞終獻洗位在小次之東,南向。盥洗在東,爵洗在西,加勺。篚在西,南肆,加巾。又設第一等分獻官盥洗爵洗位,及第二等分獻官盥洗位,各於其辰陛道之左,罍在洗左,篚在洗右,俱內向,執罍篚者位於其後。



 太府監、少府監祀前一日未後二刻,帥其屬升壇陳玉幣。昊天上帝以蒼壁、蒼幣,皇地祗以黃琮、黃幣,配位以蒼幣,黃帝以黃琮,青帝以青珪,赤帝以赤璋,大明以青珪璧,白帝以白琥,黑帝以玄璜,北極以青珪璧,天皇大帝以玄珪璧,
 神州地祗以玄色兩珪有邸,皆置於匣。五帝之幣各從其方色。凡幣皆陳於篚。設訖,俟告潔訖權徹去,祀日重設。祀日丑前五刻,禮部設祝冊神座之右,皆藉以案。太常卿明燈燎。戶部郎中設諸州歲貢於宮縣東北,金為前列,玉帛次之,餘為從列,皆藉以席,立於歲貢之後,北向。太府監、少府監設異寶嘉瑞於宮縣西,北上,瑞居前,中下次之,皆藉以席,立於實後,北向。少府少監設伐國毀寶於宮縣東南,皆藉以席,立於寶後,北向。符寶郎設八寶於宮縣西南,各分立於寶南,皆北向。司天監、太府監、少府監、郊社令、奉禮郎升設昊天上帝、皇地祗、配位、
 及壇上第一等神座,又設玉幣,各於其位。太祝取瘞玉加於幣,以禮神之玉各置於神座前,乃退。光祿卿帥其屬入實祭器。昊天上帝、皇地祇、配位每位籩三行,以右為上,形鹽在前,魚鱐糗餌次之,第二行榛實在前,乾桃乾裛乾棗次之,第三行乾菱在前,乾芡乾栗鹿脯次之。豆三行,以左為上,芹菹在前,筍菹葵菹次之,第二行韭菹在前,菁菹魚醢兔醢次之,第三行豚胉在前,醓醢酏食鹿臡次之。簠黍,簋稷,登皆大羹。第一等壇上一十位,每位皆實籩三行,以右為上,形鹽在前,魚鱐次之,第二行乾裛在前,桃棗次之,第三行乾芡在前,榛實鹿脯次
 之。豆三行以左為上,芹菹在前,筍菹次之,第二行菁菹在前,韭菹魚醢次之,第三行豚胉在前,醓醢鹿臡次之。簠黍,簋稷,登大羹,第二、第三等每位籩二,鹿脯、乾棗。豆二,鹿臡、菁菹。俎,羊一段。內濆內、內濆外每位籩鹿脯,豆鹿臡,俎羊一段。良醖令帥其屬入實尊罍,昊天上帝、皇地祇大尊為上,實以汎齊;著尊次之,實以醴齊;犧尊次之,實以盎齊;壺尊次之,實以醍齊,象尊次之;實以沈齊,山罍為下,實以三酒。配位著尊為上,實以汎齊;犧尊次之,實以醴齊;象尊次之,實以盎齊;壺尊次之,實以醍齊;山罍為下,實以三酒。第一等每位大尊實以汎齊,
 著尊實以醴齊。第二等山尊實以醍齊。第三等及內濆內,蜃尊實以汎齊。內濆外及眾星,概尊實以三酒。



 省牲器:祀前一日午後八刻,去壇二百步禁止行人。未後二刻,郊社令丞帥其屬掃除壇之上下,司尊、奉禮郎帥執事者以祭器入,設於位。司天監設神位,太府監、少府監陳玉幣於篚。未後三刻,禮直官引廩犧令與諸太祝、祝史以牲就位。又禮直官贊者分引禮部尚書、太常卿、光祿卿、禮部侍郎、太常丞、監祭御史、監禮博士、廩犧令,太官令、太官丞詣內濆東門外省牲位。立定,乃引禮部尚書、侍郎、太常丞、及監祭御史、監禮博士升自卯階,視
 濯滌,執事者皆舉冪告潔,俱畢,降復位。禮直官稍前曰:「告潔畢,請省牲。」禮部尚書侍郎及太常卿丞稍前,省牲訖,退,復位。次引光祿卿丞巡牲一匝,光祿卿退,光祿丞西向折身曰:「備訖。」乃復位。次引廩犧令巡牲一匝,西向躬身曰:「充。」又引諸祝史巡牲一匝,首一員西向躬身曰:「腯。」畢,俱復位。禮直官稍前曰:「請省饌。」乃引禮部尚書以下各就位,立定,省饌,訖,禮直官引禮部尚書侍郎、太常卿丞各還齋所,餘官廩犧令與諸太祝祝史以次牽牲詣廚,授太官令丞。次引光祿卿丞、監祭、監禮詣廚,省鼎鑊,視滌濯畢,乃還齋所。晡後一刻,太官令帥宰人以鸞
 刀割牲,祝史各取毛血實以豆,置於饌幔,遂烹牲,祝史乃取瘞血貯於盤。



 奠玉幣:祀日丑前五刻,亞終獻司徒已下,應行事陪從群官,各服其服就次。司天監復設壇上及第一等神位。太府監、少府監陳玉幣。太常卿、郊社令丞明燭燎。光祿卿丞實籩豆簠簋尊罍,俟監祭、監禮案視訖,徹去巾蓋。大樂令帥工人布於宮縣之內、文舞八佾立於縣前表後,武舞八佾各為四佾立於宮縣左右,引舞執纛等在前,又引登歌樂工由卯陛而升,各就其位。歌、擊、彈者坐,吹者立。奉禮郎贊者先入就位,餘禮直官、贊者分引分獻宮,監祭御史、監禮博士、諸執事
 及太祝、祝史、齋郎、助奠、執尊罍、舉冪等官,入自中壝東門,當壇南重行西上、北向立定。奉禮郎贊:「拜。」分獻官以下皆再拜,訖,奉禮贊曰:「各就位。」贊者、禮直官分引監祭御史、監禮博士,按視壇之上下,糾察不如儀者,退復位。禮直官引司徒入就位,西向立。禮直官引博士,博士引亞獻,自東濆偏門人就位,西向立。又禮直官引終獻,次於其位。



 祀日未明一刻,通事舍人引侍中詣齋殿,跪奏:「請中嚴。」俯伏,興。又少頃,乃跪奏:「外辦。」俟尚輦進輿,乃跪奏稱:「具官臣某,請皇帝降座升輿」。皇帝至大次,乃跪奏稱:「具官臣某,請皇帝降輿。」皇帝入次,即位於大次外。質
 明,詣次前跪奏:「請中嚴。」少頃,又奏:「外辦。」訖,太常卿乃當次前跪稱:「具官臣某,請皇帝行事。」俯伏,興。凡跪奏,準此。皇帝出次,乃前導至中濆門,殿中監進大圭,太常卿奏:「請執大圭。」入自正門,皇帝入小次位,西向立,太常卿乃與博士分左右立定,乃奏:「有司謹具,請行事。」降神,六成,樂止。太常卿別一員,乃升煙瘞血,訖,乃奏:「拜。」訖,俟侍中升壇,請詣盥洗位。至位,奏:「請搢大圭、盥手。」訖,奏:「請帨手。」皇帝帨手,訖,奏:「請執大圭。」乃引至壇上,殿中監進鎮圭,乃奏:「請搢大圭、執鎮圭。」皇帝執鎮圭,詣昊天上帝神座前,奏:「請跪,奠鎮圭。」皇帝奠,訖,執大圭,俯伏,興。侍中進玉
 幣,乃奏:「請搢大圭、跪奠玉幣。」訖,乃奏:「請執大圭。」俯伏,興。少退,又奏:「請再拜。」詣皇地祇及配位,奠鎮圭玉幣,並如儀。配位唯奏請奠鎮圭及幣。奠玉幣畢,皇帝還版位,乃奏:「請還小次、釋大圭。」皇帝入小次,乃立於小次之南稍東,以俟。皇帝將奠配位之幣也,贊者分引第一等分獻官詣盥洗位,搢笏、盥手、帨手、執笏,各由其陛升,唯不由午陛。詣神前,搢笏、跪,太祝以玉幣授之,奠訖,俯伏,興。再拜,訖,各由本陛降,復位。初,分獻將降也,禮直官引諸祝史、齋郎、應助奠者再拜,祝史各奉毛血之豆入,各由其陛升,諸太祝迎取於壇上,奠訖,退立於尊所。



 進熟:
 奠玉幣訖,降還小次。有司先陳牛鼎三、羊鼎三、豕鼎三、魚鼎三,各在鑊右。太官令丞帥進饌者詣廚,以匕升牛羊豕魚,自鑊各實於鼎。牛羊豕皆肩、臂、臑、肫、胳、正脊各一,長脅二,短脅二、代脅二、凡十一體。牛豕皆三十斤,羊十五斤,魚十五頭一十五斤,實訖,冪之。祝史二人以扃對舉一鼎,牛鼎在前,羊豕次之,魚又次之,有司執匕以從,各陳於每位饌幔位。從祀壇上第一等五方帝、大明、夜明、天皇大帝、神州地祇、北極,皆羊豕之體並同。光祿卿帥祝史、齋郎、太官令丞各以匕升牛羊豕魚於俎,肩臂臑在上端,肫胳在下端,脊脅在中,魚即橫置,頭在尊
 位,設去鼎冪。光祿卿丞同太官令丞實籩豆簠簋,籩實以粉餈,豆實以糝食,簠實稻,簋實粱。



 俟皇帝還小次,樂止。禮直官引司徒出詣饌幔所,與薦籩豆簠簋俎齋,各奉天、地、配位之饌。司徒帥太官令以序入內濆正門,樂作,至壇下,俟。祝史進徹毛血豆,降自卯陛,以次出,訖,司徒與薦籩豆簠簋俎齋郎,奉昊天上帝、皇地祇之饌,升自午陛。太官令丞與薦籩豆簠簋俎齊郎,奉配位及第一等神位之饌,升自卯陛。各位太祝迎於壇陛之道間。於昊天上帝位,司徒搢笏北向跪奉,粉餈籩在糗餌之前,糝食豆在醓醢之前,簠左簋右,皆在登前,牛俎在
 豆前,羊豕魚俎次之,以右為上。司徒俯伏,興,奉饌者奉訖,皆出笏就位,一拜。司徒次詣皇地祇奉奠,並如上儀。配位亦同。司徒及奉於、地、配位饌者以次降。太官令帥奉第一等神位之饌,各於其位,並如前儀。俱畢,樂止。司徒、太官令以下皆就位,訖,侍中升自卯陛,立於昊天上帝酌尊所,以俟。太常卿乃當次前俯伏,跪奏:「請皇帝詣盥洗位。」俯伏,興。皇帝出次,殿中監進大圭,乃奏:「請執大圭。」至盥洗位,奏:「請搢大圭、盥手。」皇帝盥手,訖,奏:「請帨手。」皇帝帨手,訖,奏:「請執大圭。」乃詣爵洗位。至位,奏:「請搢大圭、受爵」,又奏:「請洗爵。」皇帝洗爵,訖,奏:「請拭爵。」皇帝拭爵,
 訖,奏:「請執大圭。」以爵奉爵官。皇帝詣昊天上帝酌尊所,執爵,良醖令舉冪,侍中跪酌太尊之汎齊,酌訖,皇帝以爵授侍中。皇帝乃詣昊天上帝神座前,侍中進爵,乃奏「請搢大圭,跪執爵三祭酒。」訖,奏:「請奠爵。」奠爵訖,奏:「請執大圭。」俯伏,興。又奏:「請少退。」立俟。中書侍郎讀冊文,訖,乃奏:「請再拜。」詣皇地祇位及配位,並如上儀。獻畢,皇帝還版位,乃奏「請還小釋大圭。」皇帝入小次,太常卿立於小次東南。禮直官引博士,博士引亞獻,詣次,盥洗位,搢笏、盥手、帨手、訖,詣爵洗位,搢笏、洗爵,拭爵,訖,以爵授執事者,執笏升自卯陛,詣昊天上帝酌尊所,西向立。執事
 者以爵授之,乃搢笏執爵。執尊者舉冪,良醖令跪酌著尊之醴齊,酌訖,復以爵授執事者,執笏詣昊天上帝神座前。初,亞獻至盥洗位,文舞退,武舞進,樂作。亞獻詣昊天上帝神座前,搢笏跪,執事者以爵授之,乃執爵三祭酒,奠爵,執笏,俯伏,興,少退,再拜。次詣皇地祗及配位,並如上儀。獻畢,降復位。禮直官引博士,博士引終獻,詣盥洗位,盥手,洗爵,升壇奠獻,並如上儀。



 初,終獻將升壇,禮直官分引第一等分獻官詣盥洗位,搢笏,盥手,帨手。執笏,各由其陛,唯不由午陛,詣神位酌尊所,執事者以爵授之,乃酌汎齊,訖,以爵授執事者,共詣神座前,搢笏跪,
 執事者以爵授之,乃執爵三祭酒,奠爵,執笏,俯伏,興,少退,再拜,訖,各引還本位。初,第一等分獻官將升,贊引引第二等、第三等、內濆內外眾星位分獻各詣盥洗位,搢笏、盥手、帨手、酌酒、奠拜,並同上儀。祝史、齋郎以次助奠,訖,各還本位。諸太祝各進徹籩、豆各一,少移故處,樂作。卒徹,樂止。初,終獻禮畢,降復位,太常卿乃當次前俯伏,跪奏:「請皇帝詣飲福位。」皇帝出次。殿中監進大圭。乃奏:「請執爵,三祭酒。」又奏:「請啐酒。」皇帝啐酒,訖,以爵授侍中,乃奏:「請受胙。」侍中再以爵酒進,乃奏:「請受爵飲福。」皇帝飲福,訖,奏:「請執大圭。」俯伏,興。又奏:「請再拜。」訖,乃導還版
 位,西向立,俟送神樂止。乃奏:「請詣望燎位。」至位,南向立,俟火半柴,乃跪奏:「具官臣某言禮畢。」皇帝還大次,出中濆門外,奏:「請釋大圭。」皇帝入大次。初,終獻禮畢,司徙、侍中、太祝各升自卯陛,太祝持胙俎進,減天、地、配位前胙肉加於俎,皆取前腳第二節,又以黍稷飯共置一籩,奉詣司徒侍中後,北向立。俟皇帝至飲福位,太常卿奏:「請皇帝搢大圭啐酒。」訖,司徒乃進胙俎,皇帝受胙,訖,奉禮郎贊曰:「賜胙。」贊者唱曰:「再拜。」在位者皆再拜,送神,樂一成止。皇帝既入大次,更通天冠、絳紗袍,升輿,至齋宮,乘金輅。通事舍人引門下侍郎當輅前跪奏,稱:「具官臣某
 請車駕進發。」至侍臣上馬所,乃跪奏:「具官臣某請車駕少駐,敕侍臣上馬。」侍中稱:「制可。」乃退,傳制稱:「侍臣上馬。」侍臣上馬畢,乃跪奏,稱:「具官臣某請敕車右升。」千牛將軍升訖,跪奏稱:「具官臣某請車駕進發。」車駕動,前中後三部鼓吹凡十二隊齊作。應行禮陪從祀官先詣應天門奉迎,再拜。大樂令先詣應天門外,準備奏樂如儀。訖,擇日稱賀。



 承安元年,將郊,禮官言:「禮神之玉當用真玉,燔玉當用次玉。昔大定十一年,天、地之玉皆以次玉代之,臣等疑其未盡。禮貴有恆,不能繼者不敢以獻。若燔真玉,常祀
 用之恐有時或闕,反失禮制。若從近代之典及本朝儀禮,真玉禮神,次玉燔瘞,於禮為當。近代郊,自第二等升天皇大帝、北極於第一等,前八位舊各有禮玉燔玉,而此二位尚無之。按《周禮典瑞》云:『以圭璧祀日月星辰。』近代禮九宮貴神、大火星位,猶用《周禮》之說。其天皇大帝、北極二位,固宜用禮神之玉及燔玉也。」上命俱用真玉。省臣又奏:「前時郊,天、地、配位各用一犢,五方帝、日、月、神州、天皇大帝、北極十位皆大祀,亦當用犢,當時止以羊代。第二等以下從祀神位則分刲羊豕以獻。竊意天、地之祀,籩豆尚多者以備陰陽之物,鼎俎尚少者以人之
 烹薦無可以稱其德,則貴質而已。故天地日月星辰之位皆用一俎,前時第一等神位偏用二俎,似為不倫。今第一等神位亦當各用犢一,餘位以羊豕分獻,及朝享太廟則用犢十二。」上從之。



\end{pinyinscope}