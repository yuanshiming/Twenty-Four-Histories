\article{志第三十五}

\begin{pinyinscope}

 選舉四



 ○部選省選廉察薦舉功酬虧永



 凡吏部選授之制,自太宗天會十二年,始法古立官,至天眷元年,頒新官制。及天德四年,始以河南、北選人並赴中京,吏部各置局銓注。又命吏部尚書蕭賾定河南、北官通注格,以諸司橫班大解、并大將軍合注差人,依年例一就銓注,餘求仕人分四季擬授,遂為定制。貞元二年,命擬注時,依舊令,求仕官明數謂面授也,不許就本鄉,
 若衰病年老者毋授繁劇處。



 世宗大定元年,敕從八品以下除授,不須奏聞。又制,求仕官毋入權門,違者追一官降除,有所餽獻而受之者,奏之。二年,詔隨季選人,如無過或有功酬者,依格銓注。有廉能及污濫者,約量升降,呈省。七年,命有司,自今每季求仕人到部,令本部體問,政跡出眾者,及贓污者,申省核實以聞,約量升擢懲斷,年老者勿授縣令。又謂宰臣曰:「隨朝官能否,大率可知。若外路轉運司幕官以至縣令,但驗資考,其中縱有忠勤廉潔者,無路而進,是此人終身不敢望三品矣,豈進賢退不肖之道哉!自今通三考視其能否,以定升降
 為格。」又曰:「今用人之法甚弊,其有不求聞達者,人仕雖久,不離小官,至三四十年不離七品者。而新進者結朝貴,致顯達,此豈示激勸之道。卿等當審於用人,以革此弊。」時清州防禦使常德輝上言:「吏部格法,止敘年勞,是以雖有才能,拘於法而不得升,以致人材多滯下位。又刺史縣令親民之職,多不得人,乞加體察,然後公行廉問,庶使有懼心。且今酒稅使尚選能者,況承流宣化之官,可不擇乎?自今宜以能吏當任酒使者授親民之職。」從之。十年,上謂宰臣曰:「守令以下小官,能否不能遍知。比聞百姓或請留者,類皆不聽。凡小官得民悅,上官多
 惡之,能承事上官者,必不得民悅。自今民願留者,許直赴部,告呈省。遣使覆實,其績果善可超升之,如丞簿升縣令之類,以示激勸。」二十六年,以闕官,敕:「見行格法合降資歷內,三降兩降各免一降,一降者勿降。省令譯史合得縣令資歷內,免錄事及下縣令各一任。密院令史三考以上者,同前免之。臺、部、宗正府、統軍司令譯史,合歷縣令任數,免下令一任。外路右職文資諸科,合歷縣令亦免一任。當過檢法知法,三考得錄事者,已後兩除一差。」



 明昌三年,上曰:「舊制,每季到部求仕人,識字者試以書判,不識字者問以疑難三事,體察言行相副者。其
 令自今隨季部人並令依條試驗。」宰執奏曰:「既體察知與所舉相同,又試中書判,若不量與升除,無以示勸。」遂定制,若隨朝及外路六品以上官則隨長任用,外路正七品官擬升六品縣令一等除授,任滿合降者免降,從七品以下於各等資歷內減兩任擬注,以後體察相同即依已升任使,若體察不同者本等注授,若見任縣令升中上令者、並掌錢穀及丁憂去者,候解由到部。諸局分人亦候將來出職日準上擬注。猛安謀克擬依前提刑司保舉到升任例,施行時嘗令隨門戶減一資歷。明昌七年,敕復令如舊。泰和元年,上以縣令見守闕,近者
 十四月,遠者十六月,又以縣令丞簿員闕不相副,敕省臣:「右選官見格,散官至明威者注縣令,宣武者注丞簿,雖曾犯選格及虧永者亦注,是無別也。」遂定制,曾犯選格及虧永者,廣威注令,明威注丞簿。衛紹王大安元年,以縣令闕少,令初入上中下令者,與其守闕可令再注丞簿一任,俟員闕相副則當復舊。



 宣宗貞祐二年,以播越流離,官職多闕,權命河朔諸道宣撫司得擬七品以下,尋以所注吏部不知,季放之闕多至重復,乃奏罷之。時李英言:「兵興以來,百務煩冗,政在用人,舊雖有四善、十七最之法,而拔擢蔑聞,幾為徒設。大定間,以監察御
 史及審錄官分詣諸路,考核以擬,號為得人,可依已試之效,庶幾使人自勵。」詔從之。三年,戶部郎中奧屯阿虎言:「諸色遷官並與女直一體,而有司不奉,妄生分別,以至上下相疑。」詔以違制禁之。初,宣宗之南遷也,詔吏部以秋冬於南京、春夏於中都置選,而赴調者憚於北行,率皆南來,遂併於南京設之。三月,命汰不勝官者,令五品以上官公舉,今季赴部人內,先擇材幹者量緩急易之。興定元年,詔有司議減冗員。又詔,自今吏部每季銓選,差女直、漢人監察各一員監視,又盡罷前犯罪降除截罷、及承應未滿解去而復為隨處官司委使者。又定
 制,權依劇縣例俱作正七品,令隨朝七品、外路六品以上職事官,舉正七品以下職事官年未六十無公私罪堪任使者,歲一人,仍令兼領樞密院彈壓之職,以鎮軍人。凡上司不得差占及凌辱決罰。到任半年,委巡按官體訪具申籍記。又半年覆察,考滿日分等升用。如六事備為上等,升職一等,四事為中等,減二資歷,其次下等減一資歷,不稱者截罷。



 凡省選之制,自熙宗皇統八年以上京僻遠,始命詣燕京擬注,歲以為常。貞元遷都,始罷是制。其常調制,正七品兩任陞六品,六品三任升從五品,從五品兩任升正
 五品,正五品三任陞刺史。凡內外官皆以三十月為考,隨朝官以三十月為任,升職一等。自非制授,尚書選在外官,命左司移文勾取。承安三年,始命置簿勾取。



 大定十五年,制凡二品官及宰執樞密使不理任,每及三十月則書于貼黃,不及則附于闕滿簿。內外三品官以五十月為任。泰和三年,制凡文資右職官應遷三品職事者,五品以上歷五十月,六品以下及門蔭雜流職事至四品以上而散官應至三品者,皆歷六十月,方許告遷。七年,自按察使副依舊三十月理考外,內外四品以四十月理考,通八十月遷三品。泰和八年,詔以門廕官職
 事至四品者甚少,自今至刺史而散官應至三品者,即許告遷三品。此省選資考之制也。



 世宗大定元年,上謂宰臣曰:「朕昔歷外任,不能悉知人之優劣,每除一官必以不稱職為憂。夫薦賢乃相職,卿等其各盡乃心,勿貽笑天下。」又曰:「凡擬注之際當為官擇人,勿徒任親舊,庶無曠官矣。」又曰:「守令之職當擇材能,比聞近邊殘破多用年老及罪降者,是益害邊民也。若資歷高者不當任邊遠,可取以下之才能者升授,回不復降,庶可以完復邊陲也。」邊升之制,蓋始于此。三年,詔監當官遷散官至三品尚任縣令者,與省除。四年,敕隨朝六品以繁劇局
 分官有闕者,省不得擬注,令具闕及人以聞。六年,制官至三品除,朝廷約量勞績歲月,特恩遷官。七年,制內外三品官遇擬注,其歷過成考以上月日,不曾遷加,或經革撥,可於除目內備書以聞。又敕,外路四品以上職事官、並五品合陞除官,皆具闕及人以聞。六品以下官,命尚書省擬定而復奏。上又謂宰臣曰:「擬注外官,往往未當。州縣之官良則政舉,否則政隳。卿宜辨論人材,優劣參用,則遞相勉勵,庶幾成治矣。」又曰:「從來頓舍人例為節副,今宣徽院同簽銀術可以特收頓舍,然後授以滄州同知,此亦何功,但其人有足任使,故授以同簽也。且
 如自護衛、符寶、頓舍考滿者與六品五品之職,而與元苦辛特收頓舍者例除,則是不倫也。」十年,謂宰臣曰:「凡在官者,若不為隨朝職任,便不能離常調。若以卿等所知任使恐有滯,如驗入仕名項或廉等第用之亦可。若不稱職,即與外除。」十一年,上謂宰臣曰:「隨朝官多自計所歷,一考謂當得某職,兩考又當得某職,故但務因循而已。及被差遣,又多稽違。近除大理司直李寶為警巡使,而奏謝言『臣內歷兩考』,意謂合得五品則除六品也。朕以此人幹事,嘗除監察御史,及為大理司直,未嘗言情見一事,由是除長官,欲視其為政,故授是職。自今外
 路與內除者,察其為政公勤則升用,若但務茍簡者,不必待任滿即當依本等出之。不明賞罰,何以示勸勉也。」十二年,上謂宰臣曰:「朕嘗取尚書省百官行止觀之,應任刺史知軍者甚少,近獨深州同知辭不習為可,故用之。即今居五品者皆再任當例降之人,故不可也。護衛中有考滿者,若令出職,慮其年幼不閑政事,兼宿衛中如今日人材亦難得也。若勒留承應,累其資考,令至正五品可乎?」皆曰:「善。」十六年,敕宰臣:「選調擬注之際,須引外路求仕人,引至尚書省堂量材受職。」二十一年,謂宰臣曰:「海陵時,與人本官太濫,今復太隘,令散官小者奏
 之。」二十四年,以舊資考太滯,命各減一任,臨時量人材、辛苦、資歷、年甲,以次奏稟。



 章宗大定二十九年,定制,自正七品而上皆以兩任而後升。明昌四年,以前制有職官已帶三品者不許告遷,有司因之不舉,以致無由遷敘。上慮其滯,遂定制,已帶三品散官寶歷五十月,從有司照勘,格前進官一階,格後為始再算。五年,命宰臣擬注之際,召赴選人與之語,以觀其人。六年,命隨朝五品之要職,及外路三品官,皆具人闕進呈,以聽制授。七年,敕隨朝除授必欲至三十月,如有急闕,則具闕及人奏稟。尋復令,不須待考滿後,當通算其所歷而已。承安四
 年,敕宰臣曰:「凡除授,恐未盡當。今無門下省,雖有給事中而無封駁司,若設之,使於擬奏未受時詳審得當,然後授之可也。」乃立審官院,凡所送令詳審者,以五日內奏或申省。承安五年,以六品、從五品闕少,敕命歷三任正七品而後升六品。泰和元年,諭旨宰臣曰:「凡遇急闕,與其用資歷未及之人,何如止起復丁憂舊人也。」命內外官通算,合得升等而少十五月者,依舊在職補足,而後升除,或有餘月日以後積算。遇闕而無相應人,則以資歷近者奏稟。二年,命少五月以下者本任補,六月至十四月者本任或別除補之。是制既行之後,至六年,以
 一例遞升復恐太濫,命量材續稟。衛紹王大安元年,定文資本職出身內,有至一品職事官應遷一品散官者,實歷五十月方許告遷。二品三品職事官應告本品循遷者,亦歷五十月,不得過本品外。四品以下職事官如遷三品者,亦歷五十月,止許告遷三品一資。六品以下職事官歷六十月告遷,帶至三品更不許告。犯選格者皆不許。如已至三品以上職事者,六十月亦聽。凡遷三品官資及致仕並橫遷三品者,則具行止以聞。四品則六十月告遷,雜班則否。宣宗興定元年,徒單頑僧言:「兵興以來,恩命數出,以勞進階者比年尤多。賤職下僚散
 官或至極品,名器之輕莫此為甚。自今非親王子及職一品,餘人雖散官至一品乞皆不許封公。若已封者,雖不追奪其儀衛,亦當降從二品之制。」從之。



 凡選監察御史,尚書省具才能者疏名進呈,以聽制授。任滿,御史臺奏其能否,仍視其所察公事具書於解由,以送尚書省。如所察事皆無謬戾為稱職,則有升擢。庸常者臨期取旨,不稱者降除,任未滿者不許改除。大定二十七年前,嘗令六十以上者為之。後,臺官以年老者多廢事為言,乃敕尚書省於六品七品內取六十以下廉幹者備選。二十九年,令臺官得自辟舉。明昌三年,復命尚書省擬
 注,每一闕則具三人或五人之名,取旨授之。承安三年,敕監察給由必經部而後呈省。泰和四年,制以給由具所察事之大小多寡定其優劣。八年,定制,事有失糾察者以怠慢治罪。貞祐二年,定制以所察大事至五、小事至十為稱職,數不及且無切務者為庸常,數內有二事不實者為不稱職。四年,命臺官辟舉,以名申省,定其可否。



 廉察之制,始見於海陵時,故正隆二年六月有廉能官復與差除之令。大定三年,命廉到廉能官第一等進官一階升一等,其次約量注授。污濫官第一等殿三年降
 二等,次二年,又次一年,皆降一等。詔廉問猛安謀克,廉能者第一等遷兩官,其次遷一官。污濫者第一等決杖百,罷去,擇其兄弟代之。第二等杖八十,第三等杖七十,皆令復職。蒲輦決則罷去,永不補差。八年,省臣奏御史中丞移剌道所廉之官,上曰:「職官多貪污,以致罪廢,其餘亦有因循以茍歲月者。今所察能實可甄獎,若即與升除,恐無以慰民愛留之意,且可遷加,候秩滿日升除。」十年正月,上謂宰臣曰:「今天下州縣之職多闕員,朕欲不限資歷用人,何以遍知其能。擬欲遣使廉問,又慮擾民而未得其真。若令行辟舉之法,復恐久則生弊。不若
 選人暗察明廉,如其相同,然後升黜之,何如?」宰臣曰:「當如聖訓。」十一年,奏所廉善惡官,上曰:「罪重者遣官就治,所犯細微者蓋不能禁制妻拏耳,其誡勵而釋之。凡廉能官,四品以下委官覆實,同則升擢。三品以上以聞,朕自處之。」時陳言者有云:「每三年委宰執一員廉問者。」上以大臣出則郡縣動搖,誰復敢行事者。今默察明問之制,蓋得其中矣。又謂宰臣曰:「朕以欲遍知天下官吏善惡,故每使採訪,其被升黜者多矣,宜知勸也。若常設訪察,恐任非其人以之生弊,是以姑罷之。」皆曰:「是官不設,何以知官吏之善惡也?」左丞相良弼曰:「自今臣等盡心
 親察之。」上曰:「宜加詳,勿使名實淆混。」十二年,以同知城陽軍山和尚等清強,上曰:「此輩,暗察明訪皆著政聲。夫賞罰必信,則善者勸、惡者懼,此道久行庶可得人也。其第其政績旌賞之。」三月,詔贓官既已被廉,若仍舊在職必復害民,其遣驛使遍詣諸道,即日罷之。大定二十八年,制以閣門祗候、筆硯承奏、奉職、妃護衛,東宮入殿小底,宗室郎君,王府郎君、省郎君,始以選試才能用之,不須體察。內藏本把、不入殿小底、與入殿小底、及知把書畫,則亦不體察。



 明昌三年,以所廉察則有清廉之聲,而政績則平常者,敕命不降注。以石仲淵等四人,雖清廉為百
 姓所喜,而復有行事邀順人情之語,則與公正廉能人不同,敕命降注。凡治績平常者,奪元舉官俸一月。四年,上曰:「凡被舉者,或先選者不同,其後為人再舉而察者同,或先察者同,而後察者不同,當何以處之?其議可久通行無窒之術以聞。」省臣奏曰:「保舉與體察不一者,可除不相攝提刑司境內職事,再令體察,如果同則依格用,不同則還本資歷。」時有議「凡當舉人之官,歲限以數,減資注受者。」是日,省臣並奏,以謂如此恐滋久長求請僥倖之弊。遂擬:「被舉官如體察相同,隨常陞用,不如所舉者元舉官約量降除。如自囑求舉,或因勢要及
 為人請囑而舉之者,各追一官,受賄者以枉法論,體察官亦同此。歲舉不限數,不舉不坐罪,但不如所舉則有降罰,如此則必不敢濫舉,而實材可得。」上曰:「是可止作條理,施行一二年,當別思其法。」承安四年,以按察司不兼採訪,遂罷平倒別路除授之制。泰和元年,定制,自第一等闕外,第二等闕滿,合注縣令者升上令,少一任與中令,少二任與下令,少三任以上者與錄事軍防判,仍減一資,注令。少五任以上者注丞簿。第三等任滿,合注縣令者升中令,少一任與下令,少二任以上者與錄事防判,亦減一資,注令。少四任以上者並注丞簿。已入縣
 令者,秩滿日與上令,仍依各等資考內通減兩任呈省。已任七品、六品者減一資注授,經保充縣令,明問相同,依資考不待滿升除,見隨朝者考滿升注,既升除後將來覆察公正廉能者不降。宣宗南遷,嘗以御史巡察。興定元年,以縣官或非材,監察御史一過不能備知,遂令每歲兩遣監察御史巡察,仍別選官巡訪,以行黜陟之政。哀宗正大元年,設司農司,自卿而下迭出巡察吏治臧否,以升黜之。



 舉薦。大定二年,詔隨朝六品、外路五品以上官,各舉廉能官一員。三年,定制,若察得所舉相同者,即議旌除。若
 聲跡穢濫,所舉官約量降罰。九年,上曰:「朕思得忠廉之臣,與之共治,故嘗命五品以上各舉所知,于今數年矣!以天下之大,豈無其人?由在上者知而不舉也。」參知政事魏子平奏曰:「可令當舉官者,每任須舉一人,視其當否以為旌賞。」上曰:「一任舉一人,則人材或難,恐涉於濫。又少有所犯則罪舉者,故人益畏而不敢舉。宋國被舉之官有犯罪者,所舉官雖宰執亦不免降黜,若有能名,則被遷賞。且人情始慕進,故多廉慎,既得任用,或失所守。宰執自掌黜陟之權,豈可因所舉而置罪耶?」左丞相紇石列良弼曰:「已申前令,命舉之矣。」十年,上曰:「舉人之
 法,若定三品官當舉幾人,是使小官皆諂媚於上也。惟任滿詢察前政,則得人矣。」十一年,上謂宰臣曰:「昨觀貼黃,五品以下官多闕,而難於得人。凡三品以上,朕則自知,五品以下,不能盡識,卿等曾無一言見舉者。國家之務,朕豈能獨盡哉!蓋嘗思之,欲畫久安之計,興百姓之利,而無良輔佐,雖有所行皆尋常事耳。」十九年,時朝廷既取民所譽望之官而升遷之,後,上以隨路之民赴都舉請者,往往無廉能之實,多為所使而來沽名者,不須舉行。



 章宗大定二十九年,上以選舉十事,命奉御合魯諭尚書省定擬。



 其一曰:「舊格,進士、軍功最高,尚且初除
 丞簿,第五任縣令升正七品,兩任正七品升六品,三任六品升從五品,兩任從五升正五品,正五三任而後升刺史,計四十餘年始得至刺史也,其他資格出職者可知矣。拘於資格之滯,至於如此,其令提刑司採訪可用之才,減資考而用之,庶使可用者不至衰老。」省臣遂擬,凡三任升者減為兩任,於此資歷內,遇各品闕多,則於第二任未滿人內,選人材、苦辛可以超用者,及外路提刑司所採訪者,升擢之。



 其二曰:「舊格,隨朝苦辛驗資考升除者,任滿回日一而復降之。如正七滿回降除從七品,從五品回降為六品之類。今若其人果才能,可為免降。」
 尚書吏部遂擬,今隨朝考滿,遷除外路五品以下職事,并應驗考次職滿有才能者,以本官任滿已前十五月以上、二十月以內,察訪保結呈省。



 其三曰:「隨路提刑所訪廉能之官,就令定其堪任職事,從宜遷注。」



 其四曰:「從來宰相不得與求仕官相見,如此何由知天下人材優劣。其許相見,以訪才能。」尚書刑部謂:「在制,求仕官不得於私第謁見達官,違者追一官降等奏除。若有求請饋遺,則以奏聞,仍委御史糾察。」上遂命削此制。



 其五曰:「舊時,臣下雖知親友有可用者,皆欲遠嫌而不引薦。古者舉賢不避親仇,如祁奚舉仇,仁傑舉子,崔祐甫除吏八
 百皆親故也。其令五品以上官,各舉所知幾人,違者加以蔽賢之罪。」吏部議,內外五品以上職事官,每歲保廉能官一人。外路五品,隨朝六品願舉者聽。若不如所舉者,各約量降罰。今擬賢而不舉者,亦當約量降罰。



 其六曰:「前代官到任之後,即舉可自代者,其令自今五品以上官,舉自代以備交承。」吏部按《唐會要》,建中元年赦文,文武常參官外,節度、觀察、防禦、軍使、刺史、赤令、畿令、并七品以上清官,大理司直評事,受命之三日,於四方館上表,讓一人以自代,外官則馳驛奏聞。表付中書門下,每官闕即以所舉多者量授。今擬內外官五品以上到
 任,須舉所知才行官一員以自代。太傅、丞相、平章謂:「自古人材難得,若令舉以自代,恐濫而不得實材。」參政謂:「自代非謂即令代其人也,止類姓名,取所舉多者約量授之爾,此蓋舜官相讓,《周官》推賢之遺意。」上以參政所言與吏部同,從之。



 其七曰:「隨朝、外路長官,一任之內足知僚屬之能否,每任可令舉幾人。」吏部擬,今內外五品以上職事官長,於僚屬內須舉才能官一人,數外舉者聽。



 其八曰:「人才隨色有之,監臨諸物料及草澤隱逸之士,不無人材,宜薦舉用之。」吏部擬,監臨諸物料內,以外路五品、隨朝六品以上,舉廉能者,直言所長,移文轉申
 省,差官察訪得實,隨材任使。草澤隱逸,當遍下司縣,以提刑司察訪呈省。隨色人材,令內外五品以上職官薦之。



 其九曰:「親軍出職,內有尤長武藝,勇敢過人者,其令內外官舉、提刑司察,如資考高者,可參注沿邊刺史、同知、縣令。」吏部擬,若依本格資歷,恐妨才能,若舉察得實者,依本格減一資歷擬注。尚書省擬,依旨升品擬注。



 其十曰:「內外官所薦人材,即依所舉試之,委提刑司採訪虛實,若果能稱職,更加遷擢,如或碌碌,即送常調。古者進賢受上賞,進不肖有罰,其立定賞罰條格,庶使人不敢徇私也。」省臣議,隨款各欲舉人,則一人內所舉不下
 五七人。自古知人為難,人材亦自難得,限數多則猥避責罰、務茍簡,不副聖主求賢之意。擬以前項各款,隨色能舉一人,即充歲舉之數。如此則不濫,而實材得矣。每歲貢人數,尚書省覆察相同,則置簿籍之,如有闕則當隨材奏擬。



 明昌元年,敕齊民之中有德行才能者,司縣舉之,特賜同四舉五舉人下。明昌元年,制如所舉碌碌無過人跡者,元舉官依例治罪。



 宣宗興定元年,令隨朝七品、外路六品以上職事官,舉正七品以下職事官年未六十、不犯贓,堪任使者一人。三年,定辟舉縣令制。稱職,則元舉官減一資歷。中平,約量升除。不稱,罰俸一月。
 犯免官,免所居官。及官當私罪解任、杖罪、贓污者,約量降除。污贓至徒以上及除名者,一任不理資考。三品以上舉縣令,稱職者約量升除,不稱奪俸一月。若被舉者犯免官等罪,奪俸兩月。贓污至徒以上及除名者,奪俸三月,獄成,而會赦原者,亦原之。五年,制辟舉縣令考平者,元舉者不得復舉,他人舉之者聽。又舊制,保舉縣令秩滿之後,以六事論升降,三事以下減一資歷,四事減兩資歷,六事皆備則升職一等。既而御史張升卿言:「進士中下甲及第人、及監官至明威當入縣丞主簿,而三事以下減一資歷注下令,四事減注中令,令皆七品也,
 若復八品矣。輕重相戾,宜更定之。」遂定制,自今四事以下如前條,六事完者,進士中下甲及第、監官當入縣丞主簿人,減三資歷,注上令。餘出身者亦同此。任二十月以上,雖未秩滿,若以理去官,六事之跡已經覆察,論升如秩滿例。五年,以舉官或私其親,或徇於請求,或謬於鑒裁而妄舉,數歲之間以濫去者九十餘人,乃罷辟舉縣令之制。至哀宗正大元年,乃立法,命監察御史、司農司官,先訪察隨朝七品、外路六品以上官,清慎明潔可為舉主者,然後移文使舉所知,仍以六事課殿最,而升黜舉主。故舉主既為之盡心,而被舉者亦為之盡力。是
 時雖迫危亡,而縣令號為得人,由作法有足取云。



 功酬虧永之制。凡諸提點院務官,三十月遷一官,周歲為滿,止取無虧月日用之。大定四年,定制,一任內虧一分以上降五人,二分以上降十人,三分以上降十五人,若有增羨則依此升遷,其升降不盡之數,於後任充折。二十一年,以舊制監當官並責決,而不顧廉恥之人,以謂已決即得赴調,不以刑罰為畏。擬自今,若虧永及一酬以上,依格追官殿一年外,虧永不及酬者,亦殿一年。



 章宗大定二十九年,罷年遷之法,更定制,比永課增及一酬遷一官,兩酬遷兩官,如虧課則削亦如之,各兩官
 止。又罷使司小都監與使副一體論增虧者,及罷餘前升降不盡之數後任充折之制。泰和元年,制犯選及虧永者,右職漢人至宣武將軍從五品、女直至廣威將軍正五品,方注縣令。又吏格,曾犯選及虧永者,女直至武義從六,漢人及諸色人至武略從六,皆注諸司,亦兩除一差,至明威方注丞簿。貞祐三年,制曾虧永、犯選者,遷至宣武,注諸司,至懷遠從四下,方注丞簿,至安遠從四上,注下令。



 正大元年,制曾犯選、曾虧永者,至廣威與諸司、兩除一差,至安遠注丞簿,三任,其至鎮國從三品下,方注下令。群牧官三周歲為滿,所牧之畜以十為率,駝
 增二頭,馬增二匹,牛亦如之,羊增四口,而大馬百死十五匹者,及能征前官所虧,三分為率,能盡征及征二分半以上,為上等,升一品級。駝增一,馬牛增二,羊增三,大馬百死二十五,徵前官所虧二分以上,為中等,約量升除。駝不增,馬牛增一,羊增二,大馬百死三十,徵虧一分以上,為下等,依本等除。餘畜皆依元數,而大馬百死四十,徵虧不及一分者,降一等。此明昌四年制也。五年,制馬牛羊虧元數十之一,騬馬百死四十,徵虧不及一分者,降一等,決四十。若駝馬牛羊虧元數一分、馬百死四十,徵虧不得者,杖八十,降同前。



\end{pinyinscope}