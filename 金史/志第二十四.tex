\article{志第二十四}

\begin{pinyinscope}

 輿服上



 ○天子車輅皇后妃嬪車輦皇太子車制王公以下車制及鞍勒飾



 古者軍輿之制,各有名物表識,以祀以封,以田以戎,所以別上下、明等威也。歷代相承,互有損益,或因時創始,或襲舊致文,奇巧日滋,浮靡益蕩。加以後世便習騎乘,車用蓋寡,惟於郊廟祀享法駕導引,為一代令儀而不敢廢也。其於先王經世立法之意,寥乎闊哉!金初得遼之儀物,既而克宋,於是乎有車輅之制。熙宗幸燕,始用
 法駕。迨至世宗,制作乃定,班班乎古矣!考禮文,證國史,以見一代之制度云。



 大定十一年,將有事於南郊,命太常寺檢宋南郊禮,鹵簿當用玉輅、金輅、象輅、革輅、木輅、耕根車、明遠車、指南車、記里鼓車、崇德車、皮軒車、進賢車、黃鉞車、白鷺車、鸞旗車、豹尾車、軺車、羊車各一,革車五,屬車十二。除見有車輅外,闕象、木、革輅、耕根、明遠、皮軒、進賢、白鷺、羊車,大輦各一,革車三,屬車四。



 按《五禮新儀》,玉輅以青,金輅以緋,象輅以銀褐,革輅以黃,木輅以皁,蓋其物有合隨輅之色者,有當用別色者,如玉輅用青絲繡雲龍絡帶,青
 羅繡寶相花帶,青畫輪轅,青氂牛尾,此隨輅之色者也。若象、木、革輅則當用緋、用銀褐、用黃及皁。若至尊乘御步武所及,非若餘物但為美觀,其踏床、倚背、踏道之褥皆用紅錦,座褥、及行馬褥、透壁軟簾三,用銀褐、黃、青羅錦三色。又大輦,宋陶穀創意為之,至祥符中以其太重,減七百餘斤,可見當時亦無定制,各以意從長斟酌造之。其制,金玉輅闕,可見者象輅、革輅、木輅,耕根、皮軒、進賢、明遠、白鷺、羊車、革車、大輦,凡十有一:



 象輅,黃質,金塗銅裝,以象飾諸末。輪衣以銀褐。建大赤。餘同玉輅。



 革輅,黃質,鞔之以革,金塗銅裝,輪衣以黃,建大白。餘同玉
 輅。



 木輅,黑質,漆之,輪衣以皂,建大麾。餘同玉輅。



 耕根車,青質,蓋三重,制如玉輅而無玉飾。



 皮軒車,赤質,上有漆柱,貫五輪相重,畫虎紋,一轅。



 進賢車,赤質,如革車,緋輪衣、絡帶、門簾並繡鳳。上設朱漆床、香案,紫綾衣。一轅。



 明遠車,制如屋,銳頂,重簷,勾欄。頂上有金龍,四角垂鐸。上層四面垂簾,下層周以花板。三轅。



 白鷺車,赤質,周施花板,上有漆柱,柱杪刻為鷺鷥,銜鵝毛筒,紅綬帶。柱貫五輪相重。輪衣、皂頂、緋裙、緋絡帶,並繡飛鷺。一轅。



 羊車,赤質,兩壁油畫龜紋,金鳳翅,幰衣、結帶並繡瑞羊。二轅。



 大輦,赤質,正方,油畫,金塗銀葉龍鳳裝。其
 上四面施行龍、雲朵、火珠,方鑑、銀絲囊網,珠翠結雲龍,鈿窠霞子。四角龍頭銜香囊。頂輪施耀葉,中有銀蓮花,坐龍。紅綾裏,碧牙壓帖。內設圓鑑、香囊,銀飾勾欄臺坐,紫絲條網帉錔。中施黃褥,上置御座、曲几,香爐、錦結綬。几衣、輪衣、絡帶並緋繡雲龍寶相花,金線壓。長竿四,飾以金塗銀龍頭。畫梯、托叉、行馬。



 七寶輦,制如大輦,飾以玉裙網,七寶,滴子用真珠。宋欽宗為上皇製,海陵自汴取而用之。



 皇后之車六。



 一曰重翟車,青質,金飾金塗銅鈒花葉段裝釘,燿葉二十四,明金立鳳一,紫羅銷金生色寶相帷
 一,青羅、青油幰衣各一,朱絲絡網、紫羅明金生色雲龍絡帶各二,兩廂明金五彩間裝翟羽二,金塗鍮石長轅鳳頭三,橫轅立鸞八,香爐香寶子一副,宜男錦帶結,朱紅漆杌子、踏床各一,扶板扶魚一副,紅羅明金衣褥,紅羅襯褥一,青羅行道褥四,青羅明金生色雲鳳夾幔一,紅羅明金緣紅竹簾二,金塗銅葉段行馬二,朱紅漆金塗銀葉裝釘胡梯一,青羅胡梯尋儀褥二,踏道褥十,青絹裹大麻索二,油蒙帕一。二曰厭翟車,赤質,倒仙錦帷一,紫羅、紫油幰衣各一,朱絲絡網,宜男錦絡帶各二,餘同重翟,惟行道褥、夾幔、尋儀褥羅及裹索等用紅。三曰翟
 車,黃質,金飾鍮石葉段裝釘,宜男錦帷,黃羅油幰衣,鍮石長轅鳳頭三,而無橫轅立鸞,餘同厭翟,而羅色用黃。四曰安車,赤質,倒仙錦帷,紫、油幰衣,朱絲絡網,天下樂錦絡帶,鍮石長轅鳳頭三,無橫轅立鸞及香爐香寶子,餘同翟車,而色皆用紅。五曰四望車,朱質,宜男錦帷,青、油幰衣,轅端螭頭二,餘並同安車。六曰金根車,朱質,紫羅、紫油幰衣,朱絲絡網、倒仙錦絡帶各二,踏床衣褥用紅綾,尋儀褥、踏道褥並用綾,餘並同安車。



 造六車成後,復改造圓輅、重簷,方輅、五華、亭頭、平頭六等之制,又增製九龍車一,高二丈、廣一丈一尺、長二丈六尺。五鳳車四,各
 高一丈八尺,長廣如之。圓輅車一、方輅車一、重簷車一,各高一丈七尺,長一丈八尺,廣八尺。皆駕馬四,駕士各五十人,並平巾幘、生色青緋黃三色寶相花衫、銀褐抹帶、大口褲。平頭輦一、五華輦一、亭頭輦一,各高一丈九尺,廣丈五寸,長三丈。舁士各九十六人作兩番代,並生色緋寶相花衫,餘如前製。管押人員三十五人,長腳襆頭、紫羅窄衫、金銅帶束。駕馬繁纓、涼屜、鈴拂、包尾皆從車色,金銅面,插翟尾,朱轡,朱總。龍車合用紅羅傘一,傘子二人用本服錦帽襆帶。又檢定扇、障等制。偏扇如仙人羽扇。行障六扇,各長八尺、高六尺,用紅羅表、朱裏,畫
 雲鳳,龍首竿銜鞶結,每障用宮人四。坐障三扇,各長七尺、高五尺,畫雲鳳,紅羅表、朱裏,餘同行障。錦六柱八扇,各闊二尺、高三尺,冒以錦,內給使八人執。宮人車制如屬車,駕士八人,平巾幘、緋衫、大口褲、鞋襪、供奉宮人三十人,雲腳紗帽、紫衫束帶,綠靴。明昌元年三月,定妃嬪車輦同鍍金鳳頭、黃結。御妻、世婦用間金鳳頭、梅紅結子。



 皇太子車制。



 大定六年十二月,奏皇太子金輅典故制度,及上用金輅儀式,奉敕詳定。輈、旗、旂首及應用龍者更以麟為飾,省去障塵等物。上用金輅名件色數,依上
 公以九為節,減四分之一。上用輅,軾前有金龍改為伏鹿,軾上坐龍改為鳳,旂十二旒減為九,駕赤騮六減為四,及簾褥用黃羅處改用梅紅,餘並具體成造。其制,赤質,金飾諸末,重較。箱畫虞文鳥獸,黃屋。軾作赤伏鹿,龍輈。金鳳一,在軾前。設障塵。朱蓋黃裏。輪畫朱牙。左建九旒,右載闟戟。旂首銜金龍頭,結綏及鈴緌。八鸞在衡,二鈴在軾。駕赤騮四,金勣釳方,插翟尾,鏤錫鞶,纓九就。皇帝輅自頂至地高一丈七尺,今閷四分之一為一丈三尺二寸,修廣之閷亦如之。



 王公以下車制。



 一品,轅用銀螭頭,涼棚桿子、月板並許
 以銀裝飾。三品以上,螭頭不得施銀,涼棚桿子、月板亦聽用銀為飾。五品以上,轅獅頭。六品以下,轅雲頭。庶人坐車平頭,止用一色黑油。親王鞍,塗金銀裹,仍鈒以開花。障泥用紫羅,飾以錦。轡以塗金銀裝,束用絲結。皇家小功以上、太皇太后皇太后大功以上、皇后期親以上、并一品官、及官職俱至三品以上者,障泥許用金花。若經賜或御球場內,不在禁限。舊制,親王、宰執任外者,與大興尹,皆服小帽、束帶、銀鞍、絲鞭。大定中,世宗以京尹亦外官三品,而與親王無別,遂命不得御銀鞍、絲鞭,惟同外三品例,襆頭、帶、展皁
 視事。承安二年,制護衛銅裝鞍轡不得借人。庶人馬鞍許用黑漆,以骨、角、鐵為飾,不得用玉較具及金、銀、犀、象飾鞍轡。



 ◎輿服中



 ○天子袞冕



 昔者聖人制為玄黃黼黻之服,以象天地之德,以章貴賤之儀,夏、商損益,至周大備,不可以有加矣。自秦滅棄禮法,先王之制靡敝不存,漢初猶服袀玄以從大祀,歷代雖漸復古,終亦不純而已。金制皇帝服通天、絳紗、兗
 冕、偪,即前代之遺制也。其臣有貂蟬法服,即所謂朝服者。章宗時,禮官請參酌漢、唐,更製祭服,青衣朱裳,去貂蟬豎筆,以別於朝服。惟公朝則又有紫、緋、綠三等之服,與夫窄紫、展皁等事,悉著于篇云。



 天眷三年,有司以車駕將幸燕京,合用通天冠、絳紗袍,據見闕名件,依式成造。禮服,袍、裳、方心曲領、中單、蔽膝、革帶、大帶、玉具劍、綬、佩、褵、襪。乘輿服,大綬六采,黑、黃、赤、白、縹、綠、小綬三色,同大綬,間施三玉環,大綬五百首,小綬半之。白玉雙佩、革帶、玉鉤。



 冕制。天板長一尺六寸,廣八寸,前高八寸五分,後高九
 寸五分,身圍一尺八寸三分,並納言,並用青羅為表,紅羅為裏,周迴用金棱。天板下有四柱,四面珍珠網結子,花素墜子,前後珠旒共二十四,旒各長一尺二寸。青碧線織造天河帶一,長一丈二尺,闊二寸,兩頭各有真珠金碧旒三節,玉滴子節花。紅線組帶二,上有真珠金翠旒,玉滴子節花,下有金鐸子二。梅紅線款幔帶一。黈纊二,真珠垂繫,上用金萼子二。簪窠,款幔、組帶鈿窠,各二,內組帶鈿窠四並玉鏤塵碾造。玉簪一,頂方二寸,導長一尺二寸,簪頂刻鏤塵雲龍。



 袞,用青羅夾製,五彩間金繪畫,正面日一、月一、昇龍四、
 山十二,上下襟華蟲、火各六對,虎、蜼各六對。背面星一,昇龍四、山十二,華蟲、火各二十對,虎、蜼各六對。中單一,白羅單製,羅領、褾、襈。裳一,帶、褾襈,紅羅八幅夾製,繡藻三十二,粉十六、米十六、黼三十二、黻三十二。蔽膝一,帶、褾、襈,並紅羅夾製,繡昇龍二。綬一副:大綬以赤黃黑白綠縹六彩織,紅羅托裏,小綬三色,同大綬,銷金黃羅綬頭,上間施三玉環,皆刻雲龍,大綬五百首,小綬半之。緋白大帶一,銷金黃羅帶頭,鈿窠二十四。紅羅勒帛一,青羅抹帶一。玉佩二,白玉上中下璜各一,半月各二,皆刻雲龍,玉滴子各二,皆以紅真珠穿製。金篦鉤、獸面、水葉、環、
 釘。涼帶一,紅羅裹,縷金,上有玉鵝七,金宅尾束各一,金攀龍口,以玳瑁板襯釘腳。褵,重底、紅羅面,白綾托裏,如意頭,銷金黃羅緣口,玉鼻仁飾以珠。襪用緋羅加綿。凡大祭祀、加尊號、受冊寶,則服袞冕。行幸、齋戒出宮或御正殿,則通天冠、絳紗袍。



 鎮圭,大圭。皇統九年十月二十四日,禮部下太常,畫鎮圭式樣,大禮使據《三禮圖》以進,用之。大定十一年,太常寺按《禮》「大圭長三尺,抒上終葵首,天子服之」。自西魏、隋、唐以來,大圭長尺二寸,與鎮圭同。蓋鎮圭以鎮天下,以四鎮山為飾,今其圭已依古制,惟無大圭。今御府有故
 宋白玉圭,圓,無上閷及終葵首。自西魏以來,所制玉笏皆長尺有二寸,方而不折,雖非先王之法,蓋後世玉難得,隨宜故也。擬合以御府所藏,行禮就用。



 ○視朝之服



 初,太宗即位,始服赭黃,自後視百官朝御袍帶。章宗即位,以世宗之喪,有司請御純吉,不從,乃服淡黃袍、烏犀帶。常朝則服小帽、紅襴、偏帶或束帶。



 ○皇后冠服



 花株冠,用盛子一,青羅表、青絹襯金紅羅托裏,用九龍、四鳳,前面大龍銜穗球一朵,前後有花株各十有二,及鸂鶒、孔雀、雲鶴、王母仙人隊、浮動插瓣等,後有納言,上有金蟬鑻金兩博鬢,以上並用鋪翠滴粉縷
 金裝珍珠結製,下有金圈口,上用七鈿窠,後有金鈿窠二,穿紅羅鋪金款幔帶一。禕衣,深青羅織成翬翟之形,素質,十二等,領、褾、襈並紅羅織成雲龍,中單以素青紗製,領織成黼形霰十二,褾、袖襈、織成雲龍,並織紅縠造。裳,八副,深青羅織成翟文六等,褾、襈織成紅羅雲龍,明金帶腰。蔽膝,深青羅織成翟文三等,領緣,緅色羅織成雲龍,明金帶大綬一,長五尺,闊一尺,黃赤白黑縹綠六彩織成,小綬三色同大綬,間七寶鈿窠,施三玉環。上碾雲龍,拈金線織成大小綬頭,紅羅花襯,大帶,青羅朱裏,紕其外,上以朱錦,下以綠錦,
 紐約用青組,拈金線織成帶頭。玉佩二朵,每朵上中下璜各一,半月墜子各二,並玉碾,縷金打鈒獸面、篦鉤佩子各一,水葉子真珠穿綴。青衣革帶,用縷金青羅裹造,上用金打鈒水地龍,鵝眼金宅尾,龍口攀束子共八事,以玳瑁襯金釘腳。抹帶二,紅羅、青羅各一,並明金造,各長一丈五寸。褵以青羅製,白綾裏,如意頭,明金、黃羅準上用,玉鼻仁真珠裝,綴繫帶。襪,青羅表裏,綴繫帶。犀冠,減撥花樣,縷金裝造,上有玉簪一,下有玳瑁盤一。



 ○皇太子冠服



 冕用白珠九旒,紅絲組為纓,青纊充耳,犀簪導。袞,青衣朱裳,五章在衣,山、龍、華蟲、火、宗彞,四章在
 裳,藻、粉米、黼、黻。白紗中單,青褾示巽裾。革帶,塗金銀鉤。蔽膝,隨裳色,為火、山二章。瑜玉雙佩,四采織成大綬,間施玉環三。白襪,朱褵,褵加金塗銀釦。謁廟則服之。遠遊冠,十八梁,金塗銀花,飾博山附蟬,紅絲組為纓,犀簪導。朱明服,紅裳,白紗中單,方心曲領。絳紗蔽膝,白襪黑褵。餘同袞冕。冊寶則服之。桓圭,長九寸、廣三寸、厚半寸、用白玉,若屋之桓楹,為二棱。太子入朝起居及與宴,則朝服,紫袍、玉帶、雙魚袋。其視事及見師少賓客,則服小帽、皂衫、玉束帶。



 ○宗室及外戚並一品命婦



 衣服聽用明金,期親雖別籍、
 女子出嫁並同。又五品以上官母、妻,許披霞帔。唯首飾、霞帔、領袖、腰帶,許用明金、籠金、間金之類。其衣服止用明銀、象金及金條壓繡。正班局分承應帶官人,雖未出職係班,其祖母及母、妻、子孫之婦、同籍兄弟之妻、及在室女、孫、姊妹並同。又禁私家用純黃帳幕陳設,若曾經宣賜鸞輿服御,日月雲肩、龍文黃服、五箇鞘眼之鞍皆須更改。



 ○臣下朝服



 凡導駕及行大禮,文武百官皆服之。正一品:貂蟬籠巾,七梁額花冠,貂鼠立筆,銀立筆,犀簪導,佩劍,緋羅大袖、緋羅裙、緋羅蔽膝各一,緋白羅大帶,天下樂
 暈錦玉環綬一,白羅方心曲領、白紗中單、銀褐勒帛各一,玉珠佩二,金塗銀革帶,烏皮履,白綾襪。正二品:七梁冠,銀立筆,犀簪導,不佩劍,緋羅大袖,雜花暈錦玉環綬,餘並同。正四品:五梁冠,銀立筆,犀簪,白獅錦銀環綬,珠佩,銀革帶,御史中丞則獬豸冠、青荷蓮綬,餘並同。正五品:四梁冠,簇四金雕錦銅環綬,銀珠佩,餘並同。正六品至七品:三梁冠,黃獅錦銅環綬,銅珠佩,銅束帶,餘並同。大定二十二年祫享,攝官、導駕二品冠七梁,三品四品冠六梁,服有金花,五品冠五梁,六品冠四梁,七品冠三梁,監察御史獬豸冠、青綬,八品九品冠二梁,餘製並同。三
 品舊無。



 ○祭服



 皇統七年,太常寺言:「太廟成後,奉安神主,祫享行禮,凡行事、執事、助祭、陪位官,準古典當服袞冕、九章畫降龍,隨品各有等差。《通典》云虞、夏、殷並十二章,日、月、星辰、山、龍、華蟲作繪於衣,宗彞、藻、火、粉米、黼、黻絺繡於裳。周升三辰於旂,登龍於山,登火於宗彞,作九章之服,龍、山、華蟲、火、宗彞繪於衣,藻、粉米、黼、黻繡於裳。『公之服自袞冕而下如王之服,候伯之服自鷩冕而下如公之服』。又後魏帝服袞冕,與祭者皆朝服。又《開元禮》一品服九章。又《五禮新儀》正一品服九旒冕、犀簪,青衣畫降龍。今
 汴京舊禮直官言,自宣和二年已後,一品祭服七旒冕、大袖無龍。唐雖服九章服,當時司禮少常伯孫茂道言:『諸臣之章雖殊,然飾龍名袞,尊卑相亂,請三公服鷩冕八章為宜。』臣等竊謂歷代衣服之制不同,若從後魏則止服朝服,或用宋服則為七章,若遵唐九章,則有飾龍名袞尊卑相亂之議。」尚書省乃奏用後魏故事,止用燕京大冊禮時所服朝服以祭。大定三年八月,詔遵皇統制,攝官則朝服,散官則公服,以皇太子為亞獻,服袞冕。十四年,用唐制,若祭遇雨雪則服常服,謂今之公服也。泰和元年八月,禮官言:「祭服所以接神,朝服所以事君,
 雖歷代損益不同,然未嘗不有分別。是以袞冕十二旒,玄衣纁裳備十二章,天子之祭服也。通天冠、絳紗袍、紅羅裳,天子之視朝服也。臣下之服則用青衣朱裳以祭,朱衣朱裳以朝。國朝惟天子備袞冕、通天冠二等之服,今群臣但有朝服,而祭服尚闕,每有祀事但以朝服從事,實於典禮未當。請依漢、唐故事,祭服冕旒畫章,然君臣冕服雖章數各殊而俱飾龍名袞,而唐孫茂道已有尊卑相亂之論。然三公法服有龍,恐涉於僭,國初禮官亦嘗駮議。乞參酌古今,改置祭服,其冠則如朝冠,而但去其貂蟬、豎筆,其服用青衣、朱裳、白襪、朱履,非攝事者則
 用朝服,庶幾少有差別。」上曰:「朝、祭之服,固宜分也。」



 ○公服



 大定官制,文資五品以上官服紫。三師、三公、親王、宰相一品官服大獨科花羅,徑不過五寸,執政官服小獨科花羅,徑不過三寸。二品、三品服散搭花羅,謂無枝葉者,徑不過寸半。四品、五品服小雜花羅,謂花頭碎小者,徑不過一寸。六品、七品服緋芝麻羅。八品、九品服綠無紋羅。應武官皆服紫。凡散官、職事皆從一高,上得兼下,下不得僭上,窄紫亦同服色,各依官制品格。其諸局分承應人並服無紋素羅。十五年制曰:「袍不加襴,非古也。」遂命文資官公服皆加襴。帶制,皇太子玉帶,佩玉雙魚
 袋。親王玉帶,佩玉魚。一品玉帶,佩金魚。二品笏頭球文金帶,佩金魚。三品、四品荔枝或御仙花金帶,並佩金魚。五品,服紫者紅鞓烏犀帶,佩金魚,服緋者紅鞓烏犀帶,佩銀魚,服綠者並皂鞓烏犀帶。武官,一品、二品佩帶同,三品、四品金帶,五品、六品、七品紅鞓烏犀帶,皆不佩魚,八品以下並皂鞓烏犀帶。司天、太醫、內侍、教坊,服皆同文武官,惟不佩魚。應殿庭承應五品以下官,非入內不許金帶,又展紫入殿庭者,並許服紅鞓,不佩魚。又二品以上官,許兼服通犀帶,三品官若治事及見賓客,許兼服花犀帶。大定二年制,百官趨朝、赴省,並須裹帶。五品
 以上官,趨朝則朝服,赴省則展皁,雨雪沾衣則從便。凡朝參,主寶、主符展紫,御仙花或太平花金束帶。近侍給使、供御筆硯、直長、符寶吏紫襖子,塗金束帶。輪直,則近侍給使並常服,常服則展紫。閣門六尚,遇朝參侍立則服本品服,若宮中當直則服窄紫、金帶。學士院官、修起居注、補闕、拾遺、秘書丞、秘書郎,朝參侍立則服本品服、色帶。當直則窄紫、金帶。東宮左右衛率、僕正、副僕正、典儀、贊儀、內直郎丞,當直亦許服之。太子太師出入宮中則展紫,至東宮則展皂,三少則展紫。



 ◎輿服下



 ○衣服通制



 君子之服,以稱德也,故德之備者其文備。古者王公及士庶人莫不各有一定之制,而不敢相逾者,蓋風俗之奢儉,法令之齊一,必於是而觀焉。《詩》曰:「彼都人士,狐裘黃黃。其容不改,出言有章。」其三章曰:「彼都人士,充耳琇實。彼君子女,謂之尹吉。」此言都邑之盛,人物之懿也。明昌間,章宗謂宰臣曰:「今風俗侈靡,莫若律以制度,使貴賤有等。其令禮部具典故以聞。」他日又謂參知政事張萬公曰:「山東風俗如何?」萬公對以奢,左丞守貞因言衣服之制,上曰:「如卿所言,正恐失人心耳。」守貞曰:「止是商賈有不悅者。」萬公曰:「乞寬與之期,三年之內當如制矣。」
 於是,上以禮部所擬太繁,以尚書省所擬而行之。嗟乎!人君以風俗為言,其亦知所務矣。



 金人之常服四:帶,巾,盤領衣,烏皮靴。其束帶曰吐鶻。巾之制,以皂羅若紗為之,上結方頂,折垂于後。頂之下際兩角各綴方羅徑二寸許,方羅之下各附帶長六七寸。當橫額之上,或為一縮襞積。貴顯者於方頂,循十字縫飾以珠,其中必貫以大者,謂之頂珠。帶旁各絡珠結綬,長半帶,垂之,海陵賜大興國者是也。其衣色多白,三品以皂,窄袖,盤領,縫腋,下為襞積,而不缺褲。其胸臆肩袖,或飾以金繡,其從春水之服則多鶻捕鵝,雜花卉之飾,
 其從秋山之服則以熊鹿山林為文,其長中骭,取便於騎也。吐鶻,玉為上,金次之,犀象骨角又次之。銙鞓,小者間置於前,大者施於後,左右有變雙金宅尾,納方束中,其刻琢多如春水秋山之飾。左佩牌,右佩刀。刀貴鑌,柄尚雞舌木,黃黑相半,有黑雙距者為上,或三事五事。室飾以醬瓣樺,金剽口飾以鮫,或屑金鍮和漆,塗鮫隙而礲平之。醬瓣樺者,謂樺皮班文色殷紫如醬中豆瓣也,產其國,故尚之。



 初,女直人不得改為漢姓及學南人裝束,違者杖八十,編為永制。



 婦人服襜裙,多以黑紫,上編繡全枝花,周身六襞積。上衣謂之團衫,用黑紫或皂及紺,直
 領,左衽,掖縫,兩傍復為雙襞積,前拂地,後曳地尺餘。帶色用紅黃,前變垂至下齊。年老者以皂紗籠髻如巾狀,散綴玉鈿於上,謂之玉逍遙。此皆遼服也,金亦襲之。許嫁之女則服綽子,製如婦人服,以紅或銀褐明金為之,對襟彩領,前齊拂地,後曳五寸餘。



 明昌六年制,文武官六貫石以上承應人並及廕者,許用牙領,紫圓板皁條羅帶,皂靴,上得兼下。係籍儒生止服白衫領,繫背帶並以紫圓絳羅帶,乾皂靴。餘人用純紫領,不得用緣,雜色圓板絳羅帶不得用紫,靴用黃及黑油皂蠟等,婦人各從便。泰和四年,以親王品官既分
 領緣,而復有皂靴之禁,似涉太煩,遂聽親王用銀褐領紫緣,品官皆紫領白緣,餘從明昌制。



 書袋之制。大定十六年,世宗以吏員與士民之服無別,潛入民間受賕轀獄,有司不能檢察,遂定懸書袋之制。省、樞密院令、譯史用紫襜絲為之,臺、六部、宗正、統軍司、檢察司以黑斜皮為之,寺、監、隨朝諸局、并州縣,並黃皮為之,各長七寸,闊二寸、厚半寸,並於束帶上懸帶,公退則懸於便服,違者所司糾之。



 大定十三年,太常寺擬士人及僧尼道女冠有師號、並良閑官八品以上,許服花紗綾羅絲綢。在官承應有出
 身人、帶八品以下官,未帶官亦同,許服花紗綾羅襜絲絲綢,家屬同,婦人許用珠為首飾。其都孔目與八品良閑官同,京府州縣司吏皆與庶人同。庶人止許服霡綢、絹布、毛褐、花紗、無紋素羅、絲綿,其頭巾、系腰、領帕許用芝麻羅、絳用絨織成者,不得以金玉犀象諸寶瑪瑙玻璃之類為器皿、及裝飾刀把鞘、並銀裝釘床榻之類。婦人首飾,不許用珠翠鈿子等物,翠毛除許裝飾花環冠子,餘外並禁。兵卒許服無紋壓羅、霡綢、絹布、毛褐。奴婢止許服霡綢、絹布、毛褐。倡優遇迎接、公筵承應,許暫服繪畫之服,其私服與庶人同。



\end{pinyinscope}