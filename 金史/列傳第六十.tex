\article{列傳第六十}

\begin{pinyinscope}

 忠義二



 ○吳僧哥烏古論德升張順馬驤伯德窊哥奧屯醜和尚從坦孛術魯福壽吳邦傑納合蒲剌都女奚烈斡出時茂先溫迪罕老兒梁持勝賈邦獻移剌阿里合完顏六斤紇石烈鶴壽
 蒲察婁室女奚烈資祿趙益侯小叔王佐黃摑九住烏林答乞住陀滿斜烈尼龐古蒲魯虎兀顏畏可兀顏訛出虎粘割貞



 吳僧哥,西南路唐古乙剌颭上沙燕部落人。拳勇善騎射。大安間,選籍山西人為兵,僧哥充馬軍千戶,有功。貞祐初,遷萬戶,權順義軍節度使。朔州失守,僧哥復取之,真授同知節度使事。弟權同知節度使事迪剌真授節度副使。權節度副使燕曹兒真授節度判官。提控馬壽兒以下,遷授有差。眾苦乏食,僧哥乞賜糧十五萬斛。朝
 廷以為應州已破,朔為孤城,其勢不可守,乃遷朔之軍民九萬餘口分屯於嵐、石、隰、吉、絳、解之間。未行,大元兵至朔州,戰七晝夜,有功,加遙授同知太原府事、兼同知節度使事、迪剌石州刺史,曹兒同知岢嵐州防禦使事。四年,始遷其民南行,且戰且行者數十里,僧哥力憊馬躓死焉,時年三十。詔贈鎮國上將軍、順義軍節度使。



 烏古論德升,本名六斤,益都路猛安人。明昌二年進士。累官補尚書省令史,知管差除。除吏部主事、絳陽軍節度副使。丁父憂,起復太常博士、東平治中。大安初,知弘文院。改侍御史,論西京留守紇石烈執中姦惡,衛紹王
 不聽,遷肇州防禦使。宣宗遷汴,召赴闕,上言:「泰州殘破,東北路招討司猛安謀克人皆寓于肇州,凡徵調往復甚難。乞升肇州為節度使,以招討使兼之。置招討副使二員,分治泰州及宜春。」詔從之。進翰林侍讀學士、兼戶部侍郎。俄以翰林侍讀權參知政事,與平章政事抹捻盡忠論近侍局預政,宣宗怒,語在《盡忠傳》。無何,出為集慶軍節度使,改汾陽軍節度使、河東北路宣撫副使,復改知太原府事、權元帥左監軍。興定元年,大元兵急攻太原,糧道絕。德升屢出兵戰,糧道復通,詔遷官一階。德升上言:「皇太子聰明仁孝、保訓之官已備,更宜選德望
 素著之士朝夕左右之。日聞正言、見正行,此社稷之洪休、生民之大慶也。」宣宗嘉納之。二年,真授左監軍,行元帥府事。大元兵復圍太原,環之數匝,已破濠垣,德升植柵為拒,出其家銀幣及馬賞戰士。北軍壞城西北隅以入,德升聯車塞之。三卻三登,矢石如雨,守陴者不能立。城破,德升至府署,謂其姑及其妻曰:「吾守此數年,不幸力窮。」乃自縊而死。其姑及其妻皆自殺。詔贈翰林學士承旨。子兀里偉尚幼,詔以奉御俸養之。



 張順,淄州士伍。淄州被圍,行省侯摯遣總領提控王庭玉將兵救之。庭玉募順等三十人往覘兵勢,且欲令城
 中知援兵之至。乘夜潛至城下,順為所得。執之使宣言行省軍敗績,庭玉亦死,宜速降。順陽許諾,既乃呼謂城中曰:「外兵無多,王節度軍且至,堅守毋降!」兵刃交下,順曰:「得為忠孝鬼,足矣。」遂死。淄人知救兵至,以死守,城賴以完。後贈宣武將軍、同知棣州防禦使事。詔有司給養其親,且訪其子孫,優加任用。



 馬驤,禹城人也。登進士,歷官有聲。貞祐三年,為曹州濟陰令。四月,大元克曹州,驤被執。軍卒搒掠求金,驤曰:「吾書生,何從得是。」又使跪,驤曰:「吾膝不能屈,欲殺即殺,得死為大金鬼,足矣。」遂死。贈朝列大夫、泰定軍節度副使,
 仍樹碑于州,歲時致祭。貞祐四年七月,詔以其男惟賢於八貫石局分收補。



 伯德窊哥,西南路咩颭奚人。壯健沉勇。大元兵克西南路,鄰郡皆降,窊哥獨不屈。貞祐五年,東勝州已破,窊哥與姚里鴉胡、姚里鴉兒招集義軍,披荊棘復立州事。河東北路行元帥府承制除窊哥武義將軍、寧遠軍節度副使,姚里鴉胡武義將軍、節度判官,姚里鴉兒武義將軍、觀察判官。窊哥等以恩不出朝廷,頗懷觖望,縱兵剽掠。興定元年,詔窊哥遙授武州刺史、權節度使,姚里鴉胡權同知節度使事,姚里鴉兒權節度副使,各遷官兩
 階。興定三年,窊哥特遷三官,遙授同知晉安府事,尋真授東勝軍節度使。東勝被圍,城中糧盡,援兵絕,窊哥率眾潰圍,走保長寧寨,詔各進一官,戰沒者贈三官。九月,復被圍,窊哥死之。



 奧屯醜和尚,為代州經略使。貞祐四年八月,大元兵攻代州,和尚禦戰敗績,身被數創,被執。欲降之,不屈,遂死。



 從坦,宗室子。大安中,充尚書省祗候郎君。貞祐二年,自募義兵數千,充宣差都提控,詔從提舉奉先、范陽三都統兵。除同知涿州事,遷刺史,佩金牌,經略海州。頃之,充宣差都提控,安撫山西軍民,應援中都。上書曰:「絳、解二
 州僅能城守,而村落之民皆嘗被兵,重以連歲不登,人多艱食,皆恃鹽布易米。今大陽等渡乃不許粟麥過河,願罷其禁,官稅十三,則公私皆濟矣。」又曰:「絳、解、河中必爭之地,惟令寶昌軍節度使從宜規畫鹽池之利,以實二州,則民受其利,兵可以強矣。」又曰:「中條之南,垣曲、平陸、芮城、虞鄉,河東之形勢,陜、洛之襟喉也。可分陜州步騎萬二千人為一提控、四都統,分戍四縣,此萬全之策也。」又曰:「平陸產銀鐵,若以鹽易米,募工煉冶,可以廣財用、備戎器,小民傭力為食,可以息盜。」又曰:「河北貧民渡河逐食,已而復還濟其饑者,艱苦殊甚。苛暴之吏抑止
 誅求,弊莫大焉。」又曰:「河南、陜西調度未急,擇騎軍牝馬群牧,不二三年可增數萬騎,軍勢自振矣。」又曰:「諸路印造寶券,久而益多,必將積滯。止於南京印造給降,庶可久行。」又曰:「河北職任雖除授不次,而人皆不願者,蓋以物價十倍河南,祿廩不給,飢寒且至。若實給俸粟之半,少足養廉,則可責其效力。」又曰:「河北之官,朝廷減資遷秩躐等以答其勞。聞河南官吏以貶逐目之,彼若以為信然,誰不解體?」書奏,下尚書省議,惟許放大陽等渡、宣撫司量民力給河北官俸、目河北為貶所者有禁而已。四年,行樞密院于河南府,上書曰:「用兵累年,出輒無功
 者,兵不素勵也。士庶且充行伍,況於皇族與國同休戚哉。皆當從軍,親冒矢石,為士卒先,少寬聖主之憂。族人道哥實同此心,願隸臣麾下。」宣宗嘉其忠,許之。



 興定元年,改輝州刺史,權河平軍節度使、孟州經略使。初,御史大夫權尚書右丞永錫被詔經略陜西,宣宗曰:「敵兵強則謹守潼關,毋使得東。」永錫既行,留澠池數日,至京兆駐兵不動。頃之,潼關破,大元兵次近郊。由是永錫下獄,久不決。從坦乃上疏救之,略曰:「竊聞周祚八百,漢享國四百餘載,皆以封建親戚,犬牙相制故也。孤秦、曹魏亡國不永,晉八王相魚肉,猶歷過秦、魏,自古同姓之親,未
 有不與國存亡者。本朝胡沙虎之難,百僚將士無敢誰何,鄯陽、石古乃奮身拒戰,盡節而死。御史大夫永錫才不勝任,而必用之,是朝廷之過也。國之枝葉已無幾矣,伏惟陛下審圖之。」於是,宗室四百餘人上書論永錫,皆不報。久之,永錫杖一百,除名。



 當是時,諸路兵皆入城自守,百姓耕稼失所,從坦上書曰:「養兵所以衛民。方今河朔惟真定、河間之眾可留扞城,其餘府州皆當散屯于外,以為民防,俟稼穡畢功然後移於屯守之地,是為長策。」從之。加遙授同知東平府事,權元帥左監軍、行元帥府事,與參知政事李革俱守平陽。興定二年十月,從坦
 上奏:「太原已破,行及平陽。河東郡縣皆不守,大抵屯兵少、援兵不至故耳。行省兵不滿六千。平陽,河東之根本,河南之籓籬也。乞併懷、孟、衛州之兵以實潞州,調澤州、沁水、端氏、高平諸兵並山為營,為平陽聲援。惟祈聖斷,以救倒懸之急。」是月壬子,大元兵至平陽,提控郭用戰于城北濠垣,被執不屈而死。癸丑,城破,從坦自殺。贈昌武軍節度使。



 孛術魯福壽,為唐邑主簿。大元兵攻唐邑,福壽與戰,死之。贈官三階,賻錢五百貫。



 吳邦傑,登州軍事判官。邦傑寓居日照之村墅,為大元
 兵所得,驅令攻城,邦傑曰:「吾荷吾國恩,詎忍攻吾君之城。」與之酒食不顧,乃殺之。詔贈朝列大夫、定海軍節度副使。



 納合蒲剌都,大名路猛安人。承安二年進士,調大名教授。累除比陽令,補尚書省令史,除彰德軍節度副使,以憂去官。貞祐二年,調同知西安軍節度使事,歷同知臨洮、平涼府事,河州防禦使。三年,夏人圍定羌,蒲剌都擊走之,以功加遙授彰化軍節度使。四年,升河州為平西軍,就以蒲剌都為節度使。上言:「古者一人從軍,七家奉之,興十萬之師,不得操事者七十萬家。今籍諸道民為
 兵者十之七八,奉之者纔二三,民安得不困。夫兵貴精,不在眾寡。擇勇敢謀略者為兵,脆懦之徒使歸農畝,是亦紓民之一端也。」又請補官贖罪以足用,及請許人射佃陜西荒田、開採礦冶,不報。改知平涼府事,入為戶部尚書。是時,伐宋大捷,蒲剌都奏:「宋人屢敗,其氣必沮,可乘此遣人諭說,以尋舊盟。若宋人不從,然後伐之,疾仇怒頑,易以成功。」朝廷不能用。蒲剌都又言:「諸軍當汰去老弱,妙選精銳,庶可取勝。陜西弓箭手不習騎射,可選善騎者代之。延安屯兵甚眾,分徙萬人駐平涼。關中元帥猥多,除京兆重鎮,其餘皆可罷。鞏縣以北,黃河南岸,
 及金鉤、弔橋、虎牢關、虢州崿嶺,凡斜徑僻路俱當置兵防守。」詔下尚書省、樞密院議,竟不施行。未幾,改元帥右監軍、兼昭義軍節度使、行元帥府事。興定二年,潞州破,力戰而死。贈御史大夫。



 女奚烈斡出,仕至楨州刺史,被行省牒徙州人於金勝堡。已而大兵至,斡出拒戰,中流矢,病創臥。花帽軍張提控言:「兵勢不可當,宜速降。」斡出曰:「吾曹坐食官祿,可忘國家恩乎。汝不聞趙坊州乎,以金帛子女與敵人,終亦不免。我輩但當力戰而死耳。」至夜,張提控引數人持兵仗以入,脅斡出使出降,斡出曰:「聽汝所為,吾終不屈也。」
 遂殺之,執其妻子出降。



 初,楨州人遷金勝堡多不能至,軍事判官王謹收遺散之眾,別屯周安堡。周安堡不繕完樓堞、置戰守之具,兵至,謹拒戰十餘日,內潰,被執不屈而死。詔斡出、謹各贈官六階、升職三等。



 時茂先,日照縣沙溝酒監,寓居諸城。紅襖賊方郭三據密州,過其村,居民相率迎之。賊以元帥自稱,茂先怒謂眾曰:「此賊首耳,何元帥之有。」方郭三聞而執之,斷其腕,茂先大罵,賊不勝忿,復剔其目,亂刃剉之,至死罵不絕。詔贈武節將軍、同知沂州防禦使事。



 溫迪罕老兒,為同知上京留守事。蒲鮮萬奴攻上京,其
 子鐵哥生獲老兒,脅之使招餘人,不從。鐵哥怒,亂斫而死。贈龍虎衛上將軍、婆速兵馬都總管,以其姪黑廝為後,特授四官。



 梁持勝,字經甫,本名詢誼,避宣宗嫌名改焉。保大軍節度使襄之子。多力善射。泰和六年進士,復中宏詞。累官太常博士,遷咸平路宣撫司經歷官。興定初,宣撫使蒲鮮萬奴有異志,欲棄咸平徙曷懶路,持勝力止之,萬奴怒,杖之八十。持勝走上京,告行省太平。是時,太平已與萬奴通謀,口稱持勝忠,而心實不然,署持勝左右司員外郎。既而太平受萬奴命,焚毀上京宗廟,執元帥承充,
 奪其軍。持勝與提控咸平治中裴滿賽不、萬戶韓公恕約,殺太平,復推承充行省事,共伐萬奴。事泄,俱被害。詔贈持勝中順大夫、韓州刺史,賽不鎮國上將軍、顯德軍節度使,公恕明威將軍、信州刺史。



 賈邦獻,霍州霍邑縣陳村人也。舉進士第。質直有勇略。大元攻河東,邦獻集居民為守禦計。既而,兵大至,居民悉降。邦獻棄其家,獨與子懿保於松平寨。是時,權知州事劉珍在寨,與之共守,竟能成功。珍每欲辟之,邦獻輒以衰老為辭。興定四年十月,兵復大至,病不能避,與懿俱被執。欲以為鎮西元帥,且持刃脅之,邦獻不屈,密遣
 懿歸松平,遂自剄。贈奉直大夫、本縣令。



 移剌阿里合,遼人。興定間,累遷霍州刺史。興定四年正月,移霍州治好義堡。大元兵至,阿里合力戰不能敵,兵敗被執。誘使降,阿里合曰:「吾有死無貳。」叱使跪,但向闕而立,於是叢矢射殺之。



 寶昌軍節度副使孔祖湯同時被獲。既又令祖湯跪,祖湯不從,亦死。詔贈阿里合龍虎衛上將軍、泰定軍節度使,祖湯資善大夫、同知平陽府事。祖湯,泰和三年進士。



 完顏六斤,中都路胡土愛割蠻猛安人。大安中,以蔭補官,選充親軍。調阜平尉,遷方城令,改通州軍事判官,以
 功遷本州刺史。頃之,元帥右都監蒲察七斤執之以去。未幾,挈家脫歸,除同知臨洮府事,徙慶陽,遷保大軍節度使。興定五年,鄜州破,六斤自投崖下死焉。贈特進、知延安府事。詔陜西行省訪其子孫以聞。



 紇石烈鶴壽,河北西路山春猛安人。性淳質,軀幹雄偉。初充親軍。中泰和三年武舉,調褒信縣副巡檢。六年,宋人圍蔡州,鶴壽請于防禦使,與勇士五十人夜斫宋營,使諸軍噪于城上,斬三百餘級。宋兵自相蹂踐,死者千餘人。遲明,宋人解圍去。鶴壽追之,使殿曳柴。宋人顧塵起,以為大兵且至,遂奔,追至陳寨而還。已而,宋兵復據
 新蔡、新息、褒信三縣,鶴壽皆復取之,得馬三百匹,充行軍萬戶,從大軍出壽春,敗宋人于渦口,奪馬千餘匹,攻下真、滁二州及盱眙軍。軍還,進九官,遷同知息州軍州事。改萬寧宮同提舉。



 大安三年,充西南路馬軍萬戶。夏人五萬圍東勝,鶴壽救之,突圍入城,夏兵解去。遷兩階,賜銀百兩、重彩十端。遷尚方署令,充行軍副統,升充行省左翼都統。轉武衛軍都統,充馬軍副提控。轉鈐轄,充都城東面宣差副提控。



 貞祐二年,丁父憂,起復武寧軍節度副使。破紅襖賊於蘭陵石城堌,一切掠良人為生口。監察御史陳規奏:「乞敕有司,凡鶴壽所獲,俱從放免。」
 詔徐州、歸德行院拘括放之。尋遙授同知武寧軍節度使事,兼節度副使。坐出獵縱火延燒官草,杖一百,改同知河平軍節度使事。



 興定元年,充馬軍都提控,入宋襄陽界,遙授同知武勝軍節度使事,改遙授睢州刺史。二年,攻棗陽,三敗宋兵,改遙授同知歸德府事。三年,奪宋石渠寨,決去棗陽濠水,加宣差鄧州路軍馬從宜,遙授汝州防禦使。四年,宋扈太尉步騎十萬圍鄧州,鶴壽分兵拒守,出府庫金帛賞士,許以遷官加爵。自將餘眾日出搏戰,宋兵焚營去,鶴壽被創,不能騎馬,遣招撫副使術虎移剌答追及之,殺數十人,奪其俘而還。詔所散金
 帛勿問,將士優遷官爵,鶴壽遷金紫光祿大夫,遙授武勝軍節度使。



 俄丁母憂,以本官起復,權元帥左都監,行元帥府于鄜州。興定五年閏十二月,鄜州破,鶴壽與數騎突出城,追及之,鶴壽據土山力戰而死。謚果勇。



 蒲察婁室,東北路按出虎割里罕猛安人。泰和三年進士。調慶都、牟平主簿,以廉能遷中都右警巡副使。補尚書省令史,知管差除。貞祐初,除吏部主事、監察御史。丁母憂,服闕,充行省經歷官,改京兆治中,遙授定西州刺史,充元帥參議官。興定二年,與元帥承裔攻下西和州。白撒由秦州進兵抵棧道,宋人悉銳來拒。婁室乘高立
 幟,策馬旋走,揚塵為疑兵,別遣精騎掩出其後,宋兵大潰,乘勝遂拔興元。進一階,除丹州刺史。再遷同知河中府事,權元帥右都監、河東路安撫使。復取平陽、晉安,優詔褒寵,進一階,賜銀二百兩、重幣二十端,遙授孟州防禦使,權都監如故。將兵救鄜州,轉戰而至,城破死之。贈資德大夫、定國軍節度使,謚襄勇。敕行省求其尸以葬。



 女奚烈資祿,本姓張氏,咸平府人。泰和伐宋,從軍有功,調易縣尉,遷潞縣主簿。貞祐初,遙授同知德州防禦事,改秦州。三年,遙授同知通遠軍節度事。興定元年,改西寧州刺史,賜今姓。久之,遙授同知臨洮府事,兼定西州
 刺史。從元帥右都監完顏阿鄰破宋兵於梢子嶺。三年,攻破武休關,資祿功最。詔比將士遷五官、職二等外,資祿更加官、職一等,遙授通遠軍節度使,刺史如故。五年,遙授隴安軍節度使,俄改金安軍,詔曰:「陜西行省奏軍官闕員。卿久在行陣,御下有法,舊隸士卒多在京兆。今正防秋,關、河要衝,悉心備禦。」將兵救鄜州。閏十二月,鄜州破,被執不肯降,遂死。贈銀青榮祿大夫、中京留守。元光元年,言事者謂資祿褒贈尚薄,詔錄其二子烈山、林泉,升職一等,陜西行省軍中用之。



 趙益,太原人。讀書肄業。大元兵入境,益鳩合土豪,保聚
 山硤,屢戰有功。晉陽公郭文振署為壽陽令,駐兵榆次重原寨。遂率眾收復太原,夜登其城,斬馘甚眾,所獲馬仗不可計,護老幼二萬餘口以出。升太原治中,復擢同知府事、兼招撫使。元光元年八月,大元兵大至,攻城益急,知不可支,乃自焚其府庫,殺妻子,沉其符印于井,遂自殺。宣宗聞之嘉歎,贈銀青榮祿大夫、河東北路宣撫使,仍諭有司求其子孫錄用。



 侯小叔,河東縣人。為河津水手。貞祐初,籍充鎮威軍,以勞補官。元光元年,遷河中府判官,權河東南路安撫副使。小叔盡護農民入城,以家財賞戰士。河中圍解,遷治
 中,安撫如故。樞密院奏:「小叔才能可用,權位輕不足以威眾,乞假符節。」十二月,詔權元帥右都監,便宜從事。提控吳德說小叔出降,叱出斬之。表兄張先從容言大兵勢重,可出降以保妻子,小叔怒謂先曰:「我舟人子,致身至此,何謂出降。」縛先於柱而殺之,飯僧祭葬,以盡戚黨之禮。頃之,樞密院遣都監訛論與小叔議兵事,小叔出城與訛論會,石天應乘之取河中府,作浮橋通陜西。小叔駐樂李山寨,眾兵畢會,夜半坎城以登,焚樓櫓,火照城中,天應大驚不知所為,盡棄輜重、牌印、馬牛雜畜,死于雙市門。小叔燒絕浮橋,撫定其眾。遷昭毅大將軍,遙
 授孟州防禦使、同知府事,監軍、安撫如故。



 二年正月,大元軍騎十萬圍河中,總師訛可遣提控孫昌率兵五千,樞密副使完顏賽不遣李仁智率兵三千,俱救河中。小叔期以夜中鳴鉦,內外相應。及期,小叔出兵戰,昌、仁智不敢動。小叔斂眾入城,圍益急,眾議出保山寨,小叔曰:「去何之?」密遣經歷官張思祖潰圍出,奔告于汴京。明日,城破,小叔死,不得其尸。總帥訛可以河中府推官籍阿外代小叔權右都監。樞密院奏:「小叔功卓異,或疑尚在,遽令阿外代之,絕歸向之路。」至是,小叔已亡四十餘日,中條諸寨無所統領,乃詔阿外權領。宣宗思小叔功,下
 詔褒贈,切責訛可不救河中之罪。



 王佐,字輔之,霍州農家子。豁略不事產業,輕財好施,善騎射。興定中,聚兵數千人,權領霍州事。平陽胡天作承制加忠勇校尉、趙城丞,遷霍邑令、同知蒲州軍事,權招撫副使、蒲州經略使。詔遷宣武將軍,遙授寶昌軍節度副使。大元兵取青龍堡,佐被獲,署霍州守將,隸元帥崔環,質其妻子。招撫使成天祐與環有隙,佐與天祐謀殺環,天祐曰:「君妻子為質奈何?」佐曰:「佐豈顧家者邪?」元光二年七月,因環出獵殺之,率軍民數萬請命,加龍虎衛上將軍、元帥右監軍、兼知平陽府事。佐與平陽公史詠
 素不協,請徙沁州玉女寨,詔從之,仍令聽上黨公完顏開節制。是歲七月,救襄垣,中流矢卒。贈金吾衛上將軍,以其子為符寶典書。



 黃摑九住,臨潢人。大定間,以廕補部令史,轉樞密院令史,調安肅州軍事判官。明昌四年,為大理執法,同知薊州軍事,再遷潞王府司馬,累官河東北路按察使、轉運使,改知彰德府事。戰歿。贈榮祿大夫、南京留守,仍錄用其子孫。



 烏林答乞住,大名路猛安人。大定二十八年進士。累官補尚書省令史,除山東提刑判官、英王府司馬。御史臺
 舉前在山東稱職,改太原府治中。簽陜西按察司事,歷汝州、沁州刺史,北京、臨潢按察副使,遷蒲與路節度使。未幾,以罪奪三官,解職,降德昌軍節度副使。崇慶初,戍邊有功,遷一官,賞銀百兩、重幣十端,轉利州刺史。貞祐初,改同知咸平府事,遷歸德軍節度使,改興平軍,就充東面經略使。尋罷經略司,改元帥右都監。赴援中都戰歿。贈榮祿大夫、參知政事,以參政半俸給其家。



 陀滿斜烈,咸平路猛安人。襲父猛安。明昌中,以所部兵充押軍萬戶,戍邊。承安中,討契丹有功,除陳州防禦使。遷知平涼府事,改保大軍節度使,徙知彰德府事。貞祐
 四年,大元兵復取彰德,斜烈死焉。



 尼龐古蒲魯虎,中都路猛安人。明昌五年進士。累官補尚書省令史,從平章政事僕散揆伐宋。兵罷,除同知崇義軍節度使事。察廉,改東平府治中。歷環州、裕州刺史,翰林待制,開封府治中,大理卿。尋擢知河南府事,兼河南路副統軍。貞祐四年,急備京西,為陜州宣撫副使、兼西安軍節度使。是歲,大元兵取潼關,戍卒皆潰,蒲魯虎禦戰,兵敗死焉。



 兀顏畏可,隆安路猛安人。補親軍,充護衛,除益都總管府判官、中都兵馬副都指揮使,累官會州刺史。貞祐初,
 為左衛將軍、拱衛直都指揮使、山東副統軍、安化軍節度使。土賊據九仙山為巢穴,畏可擁眾不擊,賊愈熾。東平行省蒙古綱劾奏畏可不任將帥,朝廷不問。改鎮西軍,權經略副使,歷金安、武勝軍。興定四年,改泰定軍。是歲五月,袞州破,死焉。



 兀顏訛出虎,隆安府猛安人。大定二十八年進士。累官補尚書省令史,除順天軍節度副使,召為治書侍御史、刑部員外郎、單州刺史、戶部郎中、河東北路按察副使、同知大興府事、秦州防禦使。丁母憂,起復泗州防禦使,遷武寧軍節度使,徙河平軍、兼都水監。坐前在武寧奏
 軍功不實,降沂州防禦使,遷汾陽軍節度使、兼經略使。興定二年九月,城破死焉。



 粘割貞,本名抄合,西南路招討司人。大定二十八年進士。歷教授、主簿,用薦舉除河北大名提刑知事。察廉遷都轉運戶籍判官,累官泰定軍節度副使。丁父憂,服闋,除德興治中、宣德州刺史。貞祐元年十二月,貞以禮部郎中攝國子祭酒,與恩州刺史攝武衛軍副都指揮使粘割合達、河間府判官攝同知順天軍節度使事梅只乞奴、保州錄事攝永定軍節度副使伯德張奴出議和事。二年,和議成,賞銀二百兩、重幣十端、玉吐鶻。改戶部侍
 郎,歷沁南、河平、鎮南、集慶、汾陽軍節度使。貞祐四年,改昭義軍,充潞州經略使。興定二年,入為工部尚書。由壽州伐宋,攻正陽有功。權元帥左都監,守晉安府。興定三年十一月,城破,貞與府官十餘人皆死之。



\end{pinyinscope}