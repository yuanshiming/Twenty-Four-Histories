\article{列傳第十九}

\begin{pinyinscope}

 ○鶻謀琶迪姑迭阿徒罕夾谷謝奴阿勒根沒都魯黃摑敵古本蒲察胡盞夾谷吾里補王伯龍高彪溫迪罕蒲里特伯德特離補耶律懷義蕭王家奴
 田顥趙隇



 鶻謀琶,術吉水斜卯部人也。性忠直寬厚,重節義,勇於戰。父阿鶻土,贈金吾衛上將軍。穆宗時,鶻謀琶內附,先遣子寧吉從間道送款。遂使活里畽與鶻謀琶合軍攻降諸部,因領其眾,與弟胡麻谷、渾坦、姪阿里等攻下諸城,從撒改破塢塔城,穆宗屢賞之。破高麗戍兵。與石適歡討平諸部。蒲察部雅里孛堇與其兄弟胡八、雙括等欲叛歸遼,鶻謀琶執之,送于康宗,賜賚甚厚。破高麗曷懶甸及下陀魯城有功。天輔六年卒,年七十二。天眷中,贈銀青光祿大夫。



 迪姑迭,溫迪罕部人。祖扎古乃,父阿胡迭,世為胡論水部長。迪姑迭年二十餘代領父謀克,攻寧江州,敗遼援兵,獲甲馬財物。攻破奚營,回至韓州,遇敵二千人,擊走之。斡魯古與遼人戰於咸州,兵已卻,迪姑迭以本部兵力戰,諸軍復振,遂大破之。護步答岡之役,乙里補孛堇陷敵中,迪姑迭援出之。攻黃龍府,身被數創,授猛安。天輔七年,從上至山西,病卒,年四十七。天眷中,贈光祿大夫。



 阿徒罕,溫迪罕部人。年十七從撒改、斡帶等討平諸部,皆身先力戰。高麗築九城于曷懶甸,斡塞禦之,阿徒罕
 為前鋒。高麗有屯于海島者,阿徒罕率眾三十人夜渡,焚其營柵戰艦,大破之,遂下陀吉城。既而八城皆下,功最。遼兵自寧江州東門出,阿徒罕逆擊,盡殪之,以功授謀克。從攻黃龍府,力戰,身被數十創,竟登其城。後與烏論石準援照散城,阿徒罕請乘不備急擊之,遂夜過益褪水,詰朝,大敗之。斡魯上其功,賜幣與馬。天輔四年五月疾病,賜良馬一匹,詔曰:「汝安則乘之。」年六十五卒。上悼惜之,遣使弔祭,以馬為贈。阿徒罕為人孝弟,好施惠,健捷善弋獵,至角牴、擊鞠,咸精其能。



 夾谷謝奴,隆州納魯悔河人也。國初,祖阿海率所部來
 歸,獻器用甲仗。父不剌速,襲本部勃堇,從太祖伐遼,授世襲猛安,親管謀克,為曷懶路都統。謝奴其長子也,長身多髯,善騎射,通女直、契丹大小字及漢字。既冠,隨其父見太祖,命佩金牌,總領左翼護衛。西京未下,謝奴獲城中生口,乃知城中潛遣人求救於外,都統府得為之備,卻其救兵,西京乃下。自燕京還,過判泥恩納阿,遇敵於隘。謝奴身先士卒,射殺敵中先鋒二人,敵潰走,總管蒲魯虎以甲及馬贈之。後領其父猛安,從攻和尚原,出仙人關。宋兵據險,猛安雛訛只突戰不克,謝奴選麾下五十人戰,克之。與吳玠相拒,烏里雅行陣不整,吳玠乘
 之,謝奴領兵逆戰,遂大破敵。計前後功,襲其父猛安謀克。宗弼復取河南、陜西,宋人欲潛兵襲取石閏諸營,謝奴自渭南大禹鎮掩其伏兵,射中其軍帥,宋兵敗走,多獲旗幟兵仗,帥府厚賞之。除華州防禦使。入為工部侍郎,遷本部尚書。改平涼尹、昭義軍節度使。大定初,卒。阿勒根沒都魯,上京納鄰河人也,後徙咸平路梅黑河。雄偉美鬚髯,勇毅善射。國初伐遼,沒都魯在軍中,領謀克猛安,每遇敵,往來馳突,人莫敢當,故所戰皆克。皇統元年,計功擢宣威將軍。明年,授同知通遠軍節度使,改移剌都抃詳穩。授世襲本路寧打渾河謀克。為滑州刺
 史,改肇州防禦使、蒲與路節度使,遷驃騎上將車。累官金吾衛上將軍。是歲,以年老致仕,卒。年七十三。



 黃摑敵古本,世居星顯水。從破寧江,取咸州,平東京路及諸山寨柵,皆有功。從麻吉破遼將和尚節使兵七千於上京,復破那野軍二萬。再從麻吉遇敵於阿鄰甸,麻吉被創,不能戰,敵古本率兵擊敗之,剿殺殆盡。從攻回鶻城,破其兵九萬,敗木匠直撒兵於山後,俘獲甚眾。敗昭古牙之兵三千,獲其家屬而還。攻平州張覺,吾春被圍於西山,敵古本引兵救之,解其圍,並獲糧五千斛,招降戶口甚眾。從平興中,撫安其民人。天會間,大軍伐宋,
 敵古本從取濬、開德、大名,及取濟南、高唐、棣、密等州。皇統間,以功襲謀克,移屯於壽光縣界為千戶。六年,授世襲千戶,棣州防禦使。卒。



 蒲察胡盞,案出滸水人。年十八從軍,其父特廝死,襲為謀克。天輔間,夏以兵三萬出天德路,胡盞從婁室迎戰,以兵三百敗敵二千。天會三年,大軍攻太原,城中出兵萬餘來戰,胡盞以所領千戶軍擊之,復敗敵兵三萬餘於榆次境。六年,從婁室攻京兆,以所部兵屢與宋人接戰,皆先登有功。七年,取邠州,遇宋人二十餘萬,我軍右翼少卻。時胡盞為左翼千戶,摧鋒陷陣,敵遂敗去。敗張
 浚富平復有功。十三年,擊關師古於臨洮眾三萬餘。從攻涇州,從破德順、秦、鞏、臨洮、河、蘭等州,破吳璘兵,胡盞皆有力焉。授德順州刺史,改隴州防禦使。鳳翔尹。卒,年五十五。



 夾谷吾里補,暗土渾河人,徙天德。父兀屯,討烏春、窩謀罕有功。吾里補隸婁室帳下,攻係遼女直,招降太彎照三等。從婁室救斡魯古于咸州,敗遼兵于押魯虎城。遼軍營遼水,吾里補五謀克軍乘夜擊之,遼軍驚潰,殺獲幾盡。斡魯伐高永昌,吾里補以數騎奮擊於遼水之上,復以四十騎伏於津要,遇其候騎,擊之,獲生口,因盡知
 永昌虛實。太祖嘉之,賞奴婢八人。永昌駐軍於兔兒陀,先據津要,軍不得渡。吾里補與撒八射殺其先鋒二人,永昌眾稍卻,大軍遂渡遼水。及攻廣寧,軍帥選勇士先登,吾里補與赤盞忽沒渾各領所部,突入其陣,大軍繼之,遂拔廣寧。太祖攻臨潢,吾里補面被重創,奮擊自若,賞以遼宮女二人。遼王杲已取中京,吾里補以四十騎覘敵,獲遼喉舌人,因知遼主所在。後從都統斡魯定雲中,從宗翰屯應州,遼軍在近境,吾里補以所部擊敗之。宗望伐宋,宋安撫使蔡靖詣吾里補降。婁室攻陜西,諸郡往往復叛,吾里補攻敗之。敗張浚軍于富平,吾里補
 先登,睿宗賞以金器名馬。遂以先鋒攻蘭州,下其城。加昭武大將軍,授世襲猛安。累官孛特本部族節度使,以老致仕,封芮國公。



 吾里補多智略,膂力過人,雖甚老,勇健不少衰。大定初,劇賊嘯聚,出特鄙關,吾里補率鄉里年少逆擊之,賊黨遂潰。事聞,賞賚甚厚。大定二十六年卒,一百有五歲。



 王伯龍,沈州雙城人也。遼末,聚黨為盜。天輔二年,率眾二萬及其輜重來降,授世襲猛安,知銀州,兼知雙州。四年,太祖攻臨潢,伯龍與韓慶和以兵護糧餉。挽夫千五百人皆授甲,慶和已將兵行前,伯龍從糧居後,遇遼兵
 五千餘邀於路,伯龍率挽夫擊敗之,獲馬五百匹。六年,從攻下中京,並克境內諸山寨,為靜江軍節度留後。天會元年,真授節度使,從宗望討張覺於平州,伯龍先登馳擊,手殺數十百人,遷右金吾衛將軍。白河之戰,伯龍當其左軍,麾兵疾馳蹂之,宋軍亂,我師乘勝奮擊敗之。



 宗望伐宋,伯龍為先鋒,次保州,遇敵五萬,破之,招降新樂軍民十餘萬。大軍圍汴,宋太尉何巉以軍數萬出酸棗門,伯龍以本部遮擊,多所斬獲。及破汴,伯龍以治攻具有功。進破孔彥舟、酈瓊眾三萬於洺州。是年,同知保州兵馬安撫司事,將兵數千攻北平,拔之。復取保州、河間。睿
 宗經略山東,伯龍從攻青州,未下,城中夜出兵襲伯龍營,伯龍不及甲,獨被衣挺刃拒營門,敵不得入,因奮擊殺數十人。已而軍士皆甲出,殺傷宋兵不可勝計,並獲其一將,斬之。及下青州,第功,伯龍第一。



 六年,還攻莫州,降之,加太子少保、莫州安撫使。破李固寨眾十餘萬於濮州。濮城守,城中鎔鐵揮我軍,攻之不能剋。伯龍被重甲,首冠大釜,挺槍先登,殺守陴者二十餘人。大軍相繼而上,遂剋之。進攻徐州,伯龍復先登,充徐、宿、邳三路軍馬都統。敗高托山之眾十五萬餘於清河。進擊韓世忠於邳州,走之,與大軍會於宿遷,追世忠至揚州。還攻泗
 州。泗州守將以城降。屯軍嵫陽,破陳宏賊眾四十餘萬。破黃戩於單州。進攻歸德,軍帥遣伯龍立攻具,伯龍從二十餘騎行視地形,城中忽出兵千餘,欲生得伯龍,伯龍縱騎馳之,敵兵亂,墮隍而死者幾二百人。破王善之眾於巢縣,取廬州、和州,伯龍之功多。軍渡采石,擊敗岳飛、劉立、路尚等兵,獲芻糧數百萬計。還過真、揚,道遇酈瓊、韓世忠軍,復戰敗之。復為莫州安撫,改知澤州。太行群賊往往嘯聚,伯龍皆平之。



 天眷元年,為燕京馬軍都指揮使。從元帥府復收河南,權武定軍節度使,兼本路都統。宋兵據許州,伯龍擊走之,招復其人民。是年秋,泰
 安卒徒張貴驅脅良民,據險作亂,伯龍討平之。



 皇統元年,以本部從宗弼南伐,攻破濠州而還。三年,為武定軍節度使,改延安尹,寧昌軍節度使。天德三年,改河中尹,徙益都尹,封廣平郡王。卒,年六十五。正隆間,例贈特進、定國公。



 高彪,本名召和失,辰州渤海人。祖安國,遼興、辰、開三鎮節度使。父六哥,左承制,官至刺史。彪始生,其父用術者言,為其時日不利於己,欲不舉,其母為營護。居數歲,竟逐之,彪匿於外家。遼人調兵東京時,六哥已老,當從軍,悵然謂所親曰:「吾兒若在,可勝兵矣。」所親具以實告,因
 代其父行。戰於出河店,遼兵敗走,彪獨力戰,軍帥見之曰:「此勇土也。」令生致之。斡魯攻東京,六哥率其鄉人迎降,以為榆河州千戶。久之告老,彪代領其眾。



 都統杲攻中京,彪領謀克,從斡魯破遼將合魯燥及韓慶民於高、惠之境。已而駐軍武安,合魯燥以勁兵二萬來襲,從斡魯出戰,與所部皆去馬先登,奮擊敗之。奚人負險拒命,所在屯結,彪屢戰有功。宗望攻平州,彪徇地西北道,破敵,招降石家山寨。再從宗望伐宋,為猛安。師次真定,彪率兵士七十人,臨城築甬道,城中夜出兵焚攻具,彪擊走之。大軍圍汴,以五十騎屯於東南水門。宋人再以重
 兵出戰,彪皆敗之。師還,屯鎮河朔,復破敵於霸州,擒其裨將祝昂。河間夜出兵二萬襲我營壘,彪率三謀克兵擊敗之。天會五年,授靜江軍節度使、壽州刺史。



 明年,伐宋,從帥府徇地山東,攻城克敵,數被重賞。七年,師至睢,彪以所部招誘京西人民。次柘城縣,其官吏出降,彪獨與五十餘騎入城。繼而城中三千餘人復叛,彪率其眾力戰敗之,撫安其民而還。從梁王宗弼襲康王,至杭州。師還,宋將韓世忠以戰艦數百扼於江北。宗弼引而西,將至黃天蕩,敵舟三十餘來逼南岸,其一先至者載兵士二百餘。彪度垂及,以鉤拽之,率勇士數十,躍入敵舟,所
 殺甚眾,餘皆逼死於水中。明年,從攻陜西,師至寧州,彪與宗人昂率兵三千取廓州。始至,有來降者言:「城東北隅守兵將謀為內應。」彪即夜從家奴二人以登,左右守者覺之,彪與從者皆殊死戰,諸軍繼進,遂克其城。從攻和尚原及仙人關。與阿里監護漕糧並戰艦至亳州,宋人以舟五十艘阻河路,擊敗之,擒其將蕭通。擊漣水賊水寨,進取漣水軍,其官民已遁去,悉招降之。



 彪勇健絕人,能日行三百里,身被重鎧,歷險如飛。及臨敵,身先士卒,未嘗反顧,大小數十戰,率以少擊眾,無不勝捷。



 齊國既廢,攝滕陽軍以東諸路兵馬都統,撫諭徐、宿、曹、單,滕
 陽及其屬邑皆按堵如故。為武寧軍節度使,頗黜貨,嘗坐贓,海陵以其勳舊,杖而釋之。改沂州防禦使,歷安化、安國、武勝軍節度使,遷行臺兵部尚書,改京兆尹,封郜國公。以憂去官,起復為武定軍節度使,歸德尹。正隆例授金紫光祿大夫。久之致仕,復起為樞密副使、舒國公,賜名彪。卒年六十七,謚桓壯。彪性機巧,通音律,人無貴賤,皆溫顏接之。



 溫迪罕蒲里特,隆州移離閔河胡勒出寨人也。魁梧美髯,有謀略,以智勇聞。都統杲取中京,蒲里特權猛安,領軍五千,遇契丹賊萬餘,與戰敗之。出袞古里道,敗敵八
 千餘。至臘門華道,復以伏兵敗敵萬人。太祖定燕,自儒州至居庸關,執其喉舌人。有頃,賊三千餘人復寇臘門華道,蒲里特整隊先登,賊識其旗幟,望風而遁,遂奮擊之,親執賊帥。皇統元年,從梁王宗弼伐宋,留軍唐州。敵眾奄至,蒲里特擊之,大名軍萬四千號二十萬,蒲里特率親管猛安,身先士卒,衝擊。敵少卻,乃張左右翼併擊之,敵眾散走。而別遇兵二萬來援,復以兵三千擊走之。時邳州土賊嘯聚,幾二十萬,蒲里特軍三千,分為數隊急攻之,賊潰去。南京路遇敵軍二萬,蒲里特以軍三千擊敗之。是日,有兵自城中出者,復擊敗之。皇統二年,遷
 定遠大將軍,同知鳳翔尹。六年,改京兆尹,轉寧州刺史,改西北路招討都監,遷永定軍節度使。海陵南征,改武衛軍都總管。大定三年,授開遠軍節度使,改泰寧軍。卒。十九年,以功授其子兀帶武功將軍、本猛安奚出痕世襲謀克。



 伯德特離補,奚五王族人也,遼御院通進。天會初,與父撻不也歸朝,授世襲謀克,後以京兆尹致仕。



 特離補招降松山等州未附軍民,及招降平州、薊州境內,督之耕作。宗望伐宋,特離補為軍馬猛安,與諸將留,規取保、遂、安三州。攻安肅軍,河間、雄、保等兵十餘萬來救,特離補
 率所部先戰,大軍繼之,大破其兵,遂拔安肅。特離補攝通判事,降將胡愈陰結眾謀亂,特離補勒兵擒愈及其眾五十餘人。安肅軍改為州,就除同知州事。改磁州,捕獲太行群盜。元帥府以磁、相二州屯兵屬之,擒王會、孫小十、苗清等,群盜遂平。遷濱州刺史,廉入優等。以母憂去官,起復本職,改涿州刺史。入為工部郎中,從張浩營繕東京宮室。及田玨黨事起,朝省為之一空,特離補攝行六部事,遷大理卿,出為同知東京留守。天德三年,復為大理卿,同知南京留守。丁父憂,起復洺州防禦使。正隆盜起,州縣無兵,不能禦。洺舊有河附于城下,特離補
 乃引水注濠中以為固,盜弗能近,州賴以安。遷崇義軍節度使。未幾,告老歸田里,卒。



 特離補為人孝謹,為政簡靜不積財,常曰:「俸祿已足養廉,衣食之外,何用蓄積。」凡調官,行李止車一乘、婢僕數人而已。



 耶律懷義,本名孛迭,遼宗室子。年二十四,以戰功累遷同知點檢司事。宗翰已取西京,遼主謀奔于夏,懷義諫止之,不見聽,乃竊取遼主廄馬來降。太祖自燕還師,留宗翰、斡魯經略西方,懷義領謀克從軍。天會初,帥府以新降諸部大小遠近不一,令懷義易置之,承制以為西南路招討使。乃擇諸部衝要之地,建城市,通商賈。諸部
 兵革之餘,人多匱乏,自是衣食歲滋,畜牧蕃息矣。



 從宗翰伐宋,降馬邑,破鴈門,屯兵,進攻太原。以所部別降清源縣徐溝鎮,遂與諸將列屯汾州之境。時河東、陜西路兵來救太原,劉光世、折可求柵於文水西山,懷義捕得生口,盡知宋兵屯守要害,乃分兵襲敗之。明年,再伐宋,從婁室取汾州及其屬邑,遂過平陽,出澤、潞以趨河陽,所至皆降。及大軍圍汴,懷義屯京西,汴城既下,宋兵之出奔者,邀擊盡之。從攻鄭、鄧州及討平鄭州叛者,攻下濮州及雷澤縣,從破大名、東平府、徐、兗等州,皆有功。七年,還鎮。十年,加尚書左僕射,改西北路招討使。



 懷義在
 西陲幾十年,撫御有恩,及去,老幼遮道攀戀,數日不得發。天眷初,為太原尹,治有能聲。改中京留守。從宗弼過烏納水,還中京,以老乞致仕,不許。改大名尹,命不赴治所,止以俸傔給之。每歲春水扈從,余聽自便。明年,再請老得謝,給俸傔之半。海陵即位,封漆水郡王,進封莘王。久之,進封蕭王。正隆例封景國公。其子神都斡為西北路招討都監,迎侍之官。神都斡從海陵南征,懷義卒于雲中,年八十二。



 蕭王家奴,奚人也,居庫黨河。為人魁偉多力,未冠仕遼,為太子率府率。天輔七年,都統杲定奚地,王家奴率其
 鄉人來降,命為千戶領之。奚王回離保既死,其親黨金臣阿古者猶保撒葛山,王家奴與突捻往討之,生擒金臣阿古者,降其餘眾。時平、灤多盜,王家奴以所部屢破賊兵,斬馘執俘,數被賞賚。宗望伐宋,敗郭藥師於白河,亦與有功。至河上,宋兵扼津要,與諸將擊敗之。進圍汴,破其東門兵。明年,再伐宋,宗望軍至中山,諸門分兵出戰,焚我攻具,祁州、河間各以兵來援,皆敗之。師還,屯鎮河朔。濱州賊葛進聚眾數萬臨淄,孛堇照里以騎兵二千討之,王家奴領謀克先登,力戰大破其眾。明年,攻滄州,宋兵拒戰,復從照里擊走之。宋將徐文以舟百艘泊
 海島,即以商船十八進襲,斬首七百級,獲舟二十。天會八年,除靜江軍節度使,授世襲千戶。從梁王宗弼征伐,為萬戶,還為五院部節度使。天德二年,改烏古迪烈招討都監,卒。



 田顥,字默之,興中人。遼天慶八年進士,歷官金部員外郎,權歸德節度使。太祖定燕,顥舉四州版圖歸朝,加都官郎中,權節度使事,四遷知真定府事。招降齊博、游貴等賊眾五千餘人。已而貴復叛去,顥遣齊博偽叛從貴,因令伺間殺之,降其眾,賊壘悉平。三遷行臺左丞、彰德軍節度使。是時,新定力役,顥蠲籍之半而上之,故相之
 徭賦比他州獨輕。徙同知河北東路都總管,改同簽燕京留守司事,民遮留不得出,易服夜去。改河東南路轉運使,尋改絳陽軍節度使。居三年,以疾請謝事,徑解印歸。數奏不允,移鎮振武軍。入為刑部尚書,居三月請老,卒于家。



 趙隇,字德固,遼陽人。其婦翁以優伶得幸於遼主,隇補閣門祗候,累遷太子左衛率。後居灤州。宗望討張覺,隇踰城出降,授洛苑副使,為灤州千戶。遷洛苑使,檢校工部尚書。從伐宋,至汴,遷棣州刺史、侍衛步軍都虞候。及再伐宋,攻真定,與有功,改商州刺史,檢校尚書右僕射。
 五年,同知信德府路統押軍兵,兼沿邊安撫司事。明年,權知濟州事。八年,從定河南,授隴州團練使。十年,改知石州。隇久在兵間,不善治民,坐謗議,謫監平州甜水鹽。齊國廢,河南皆以宿將守之,授隇宿州防禦使,統本路軍兵。隇重義,接儒士。嘗以事至汴,有故人子負官錢百萬,隇以橐金贈之,其子悉為私費,復代輸之。頃之,有訟徐帥不法者,朝廷使隇鞫治,隇委曲營護,坐是廢罷,寓居於燕。海陵出領行臺省,至燕,隇往見之,因訴其事。及海陵即位,起為保大軍節度使。貞元初,改內省使。未幾,為中都路都轉運使。明年,再徙順義、興平,入為太子詹
 事,鎮沁南,以疾卒,年六十六。



 後十餘年,隇子孫、司徒張通古子孫皆不肖淫蕩,破貲產,賣田宅。世宗聞之,詔曰:「自今官民祖先亡沒,子孫不得分割居第,止以嫡幼主之,毋致鬻賣。」仍著于令。



\end{pinyinscope}