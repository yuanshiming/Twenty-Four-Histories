\article{志第二十五}

\begin{pinyinscope}

 兵



 ○兵
 制



 金興,用兵如神,戰勝功取,無敵當世,曾未十年遂定大業。原其成功之速,俗本鷙勁,人多沉雄,兄弟子姓才皆良將,部落保伍技皆銳兵。加之地狹產薄,無事苦耕可給衣食,有事苦戰可致俘獲,勞其筋骨以能寒暑,徵發調遣事同一家。是故將勇而志一,兵精而力齊,一旦奮起,變弱為彊,以寡制眾,用是道也。及其得志中國,自顧
 其宗族國人尚少,乃割土地、崇位號以假漢人,使為之效力而守之。猛安謀克雜廁漢地,聽與契丹、漢人昏因以相固結。迨夫國勢浸盛,則歸土地、削位號,罷遼東渤海、漢人之襲猛安謀克者,漸以兵柄歸其內族。然樞府簽軍募軍兼采漢制,伐宋之役參用漢軍及諸部族而統以國人,非不知制勝長策在於以志一之將、用力齊之兵也,第以土宇既廣,豈得盡任其所親哉!馴致極盛,乃自患其宗族國人之多,積其猜疑,卒自戕賊,遂致強本刊落,醇風鍥薄,將帥攜離,兵士驕惰。迄其亡也,「忠孝」等軍構難于內,颭軍雜人召禍於外,向之所謂志一而
 力齊者,不見可恃之勢焉。豈非自壞其家法而致是歟?抑是道也可用於新造之邦,不可以保長久之天下歟?金以兵得國,奉詔作《金史》,故於金之《兵志》考其興亡得失之跡,特著於斯。兵制、馬政、養兵等法載諸舊史者,戶列于篇。



 金之初年,諸部之民無它徭役,壯者皆兵,平居則聽以佃漁射獵習為勞事,有警則下令部內,及遣使詣諸孛堇徵兵,凡步騎之仗糗皆取備焉。其部長曰孛堇,行兵則稱曰猛安、謀克,從其多寡以為號,猛安者千夫長也,謀克者百夫長也。謀克之副曰蒲里衍,士卒之副從曰
 阿里喜。部卒之數,初無定制。至太祖即位之二年,既以二千五百破耶律謝十,始命以三百戶為謀克,謀克十為猛安。繼而諸部來降,率用猛安、謀克之名以授其首領而部伍其人。出河之戰兵始滿萬,而遼莫敵矣!及來流、鴨水、鐵驪、鱉古之民皆附,東京既平,山西繼定,內收遼、漢之降卒,外籍部族之健士。嘗用遼人訛里野以北部百三十戶為一謀克,漢人王六兒以諸州漢人六十五戶為一謀克,王伯龍及高從祐等並領所部為一猛安。至天會二年,平州既平,宗望恐風俗揉雜,民情弗便,乃罷是制。諸部降人但置長吏,以下從漢官之號。四年,
 伐宋之役,調燕山、雲中、中京、上京、東京、遼東、平州、遼西、長春八路民兵,隸諸萬戶,其間萬戶亦有專統漢軍者。熙宗皇統五年,又罷遼東漢人、渤海猛安謀克承襲之制,浸移兵柄於其國人,乃分猛安謀克為上中下三等,宗室為上,餘次之。至海陵庶人天德二年,省併中京、東京、臨潢、咸平、泰州等路節鎮及猛安謀克,削上中下之名,但稱為「諸猛安謀克,」循舊制間年一徵發,以補老疾死亡之數。貞元遷都,遂徙上京路太祖、遼王宗乾、秦王宗翰之猛安,併為合扎猛安,及右諫議烏里補猛安,太師勖、宗正宗敏之族,處之中都。斡論、和尚、胡剌三國公,
 太保昂,詹事烏里野,輔國勃魯骨,定遠許烈,故杲國公勃迭八猛安處之山東。阿魯之族處之北京。按達族屬處之河間。正隆二年,命兵部尚書蕭恭等,與舊軍皆分隸諸總管府、節度使,授田牛使之耕食,以蕃衛京國。六年,南伐,立三道都統制府及左右領軍大都督,將三十二軍,以神策、神威、神捷、神銳、神毅、神翼、神勇、神果、神略、神鋒、武勝、武定、武威、武安、武捷、武平、武成、武毅、武銳、武揚、武翼、武震、威定、威信、威勝、威捷、威烈、威毅、威震、威略、威果、威勇為名,軍置都總管、副總管及巡察使、副各一員。而沿邊契丹恐妻孥被鄰寇鈔掠,不可盡行,遂皆背判。
 而大名續授甲之士還迎立世宗于東京。



 及大定之初,窩斡既平,乃散契丹隸諸猛安謀克。至三年,詔河北、山東等路所簽軍,有父兄俱已充甲軍,子弟又為阿里喜,恐其家更無丁男,有誤農種,與免一丁,以驅丁充阿里喜,無驅丁者於本猛安謀克內驗富強有驅丁者簽充。十三年,徙東北等戌邊漢軍於內地。十五年十月,遣吏部郎中蒲察兀虎等十人分行天下,再定猛安謀克戶,每謀克戶不過三百,七謀克至十謀克置一猛安。十七年,又以西南、西北招討司契丹餘黨心素狠戾,復恐生事,它時或有邊隙,不為我用,令遷之於烏十里石壘部
 及上京之地。上謂宰臣曰:「北邊番戍之人,歲冒寒暑往來千里,甚為勞苦。縱有一二馬牛,一往則無還理,且奪其農時不得耕種。故嘗命卿等議,以何術得罷其役,使安于田里,不知卿議何如也?」左丞相良弼對曰:「北邊之地,不堪耕種,不能長戍,故須番戍耳。」上曰:「朕一日萬幾,安能遍及,卿等既為宰相,以此急務反以為末事,竟無一言,甚勞朕慮。往者參政宗敘屢為朕言,若以貧戶永屯邊境,使之耕種,官給糧廩,則貧者得濟,富戶免於更代之勞,使之得勤農務。若宗敘者可謂盡心為國矣!朕嘗思之,宜以兩路招討司及烏古里石壘部族、臨潢府、
 泰州等路分定保戍,具數以聞,朕親覽焉。」十八年,命部族、颭分番守邊。二十年,以祖宗平定天下以來,所建立猛安謀克,因循既久,其間有戶口繁簡、地里遠近不同,又自正隆之後所授無度,及大定間亦有功多未酬者,遂更定以詔天下。復命新授者並令就封,其謀克人內有六品以下職及諸局承應人,皆為遷之。三從以上族人願從行者,猛安不得過十戶,謀克不得過六戶。詔戍邊軍士年五十五以上,許以其子及同居弟姪承替,以奴代者罪之。二十一年三月,詔遣大興尹完顏迪古速遷河北東路兩猛安,上曰:「朕始令移此,欲令與女直戶
 相錯,安置久則自相姻親,不生異意,此長久之利也。今者移馬河猛安相錯以居,甚符朕意,而遙落河猛安不如此,可再遣兵部尚書張那也按視其地以雜居之。」二十二年,以山東屯田戶鄰之於邊鄙,命聚之一處,俾協力蠶種。右丞相烏古論元忠曰:「彼方之人以所得之地為家,雖兄弟不同處,故貧者眾。」參政粘割斡特剌曰:「舊時兄弟雖析猶相聚種,今則不然,宜令約束之。」又以猛安謀克舊籍不明,遇簽軍與諸差役及賑濟,增減不以實,命括其口,以實籍之。二十三年,遣刑部尚書移剌綎遷山東東路八謀克處之河間,其棄地以山東東路忒
 黑河猛安下蘸荅謀克,移里閔斡魯渾猛安下翕浦謀克、什母溫山謀克九村人戶徙於劉僧、安和二謀克之舊地。其未徙者之地皆薄惡且鄰寇,遣使詢願徙者,相可居之地,圖以進。



 上嘗以速頻、胡里改人驍勇可用,海陵嘗欲徙之而未能,二十四年以上京率、胡剌溫之地廣而腴,遂遣刑部尚書烏里也出府庫錢以濟行資牛畜,遷速頻一猛安、胡里改二猛安二十四謀克以實之。蓋欲上京兵多,它日可為緩急之備也。當是時,多易置河北、山東所屯之舊,括民地而為之業,戶頒牛而使之耕,畜甲兵而為之備,乃大重其權,授諸王以猛安之號,
 或新置者特賜之名。制其奢靡,禁其飲酒,習其騎射,儲其糧Я,其備至嚴也。是時宗室戶百七十,猛安二百二,謀克千八百七十八,戶六十一萬五千六百二十四。東北路部族颭軍曰迭剌部承安三年改為土魯渾尼石合節度使,曰唐古部承安三年改為部魯火札石合節度使,二部五颭,戶五千五百八十五。其它若助魯部族、烏魯古部族、石壘部族、萌骨部族、計魯部族、孛特本部族數皆稱是。西北、西南二路之颭軍十,曰蘇謨典颭、曰耶刺都颭、曰骨典颭、唐古颭、霞馬颭、木典颭、萌骨颭、咩颭、胡都颭凡九,其諸路曰曷懶、曰蒲與、曰婆速、曰恤頻、曰胡里改、曰移懶,移懶後廢,皆在上
 京之鄙,或置總管府,或置節度使。至章宗明昌間,欲國人兼知文武,令猛安謀克舉進士,試以策論及射,以定其科甲高下。承安四年,上謂宰臣曰:「人有以《八陣圖》來上者,其圖果何如?朕嘗觀宋白所集《武經》,具載攻守之法,亦多難行。」右丞相清臣曰:「兵書一定之法,難以應變。本朝行兵惟用正奇二軍,臨敵制變,以正為奇,以奇為正,故無往不克。」上曰:「自古用兵亦不出奇正二法耳。且學古兵法如學弈棋,未能自得於心,欲用舊陣勢以接敵,疏矣。敵所應與舊勢異,則必不可支。然《武經》所述雖難遵行,然知之猶愈不知。」泰和間,又制武舉,其制具在《
 選舉志》。



 所謂渤海軍,則渤海八猛安之兵也。所謂奚軍者,奚人遙輦昭古牙九猛安之兵也。奚軍初徙于山西,後分遷河東。其漢軍中都永固軍,大定所置者也。所謂鎮防軍,則諸軍中取以更代戍邊者也。在西北邊則有分番屯戍軍及永屯軍驅軍之別。驅軍則國初所免遼人之奴婢,使屯守于泰州者也。邊鋪軍則河南、陜西居守邊界者。河東三虞候順德軍及章宗所置諸路效節軍,京府節鎮設三十人,防刺設二十人。掌同弓手者也。諸路所募射糧軍,五年一籍三十以下、十七以上強壯者,皆刺其缺,所以兼充雜役者也。京師防城軍,世宗大定十七年三月改
 為武衛軍,則掌京師巡捕者也。其曰牢城軍,則嘗為盜竊者,以充防築之役。曰土兵,則以司警捕之事。凡漢軍,有事則簽取於民,事已則或亦放免。初,天會間,郭藥師降,有曰長勝軍者,皆遼水側人也,以鄉土歸金,皆愁怨思歸,宗望及令罷還。正隆間,又嘗罷諸路漢軍,而所存者猶有威勇、威烈、威捷、順德及「韓常之軍」之號。



 凡邊境置兵之州三十八:鳳翔、延安、鄧、鞏、熙、泗、潁、蔡、隴、秦、河、海、壽、唐、商、洮、蘭、會、積石、鎮戎、保安、綏德、保德、環、葭、庾、寧邊、東勝、凈、慶、來遠、桓、昌、曷懶、婆速、蒲與、恤品、胡里改。置於要州者十一:南京、東京、益都、京兆、太原、臨洮、臨潢、豐、泰、
 撫、蓋。及宣宗南遷,颭軍潰去,兵勢益弱,遂盡擁猛安戶之老稚渡河,僑置諸總管府以統之,器械既缺,糧Я不給,朘民膏血而不足,乃行括糧之法,一人從征,舉家待哺。又謂無以堅戰士之心,乃令其家盡入京師,不數年至無以為食,乃聽其出,而國亦屈矣。然初南渡時,盡以河朔戰兵三十萬分隸河南行樞密及帥府,往往蔽匿強壯,驅羸弱使戰,不能取勝。後乃至以二十五人為謀克,四謀克為猛安。每謀克除旗鼓司火頭五人,任戰者止十八人,不足成隊伍,但務存其名而已。故混源劉祁謂:「金之兵制最弊,每有征伐及邊釁,輒下令簽軍,使遠
 近騷動。民家丁男若皆強壯,或盡取無遺,號泣動乎鄰里,嗟怨盈於道路,驅此使戰,欲其勝敵,難矣!」初,貞祐時,下令簽軍,會一時任子為監當者春赴吏部選,宰執命取為監官軍,皆憤慍哀號交醖臺省,至衝宰相鹵簿以告,丞相僕散七斤大怒,趣左右取弓矢射去。已而,上知其不可用,命免之。元光末,備潼關黃河,又簽軍,諸使者歷縣邑,自見居官外,無文武小大職事官皆充軍。至許州,前侍御史劉元規年幾六十,亦選為千戶。至陳州,以祁父從益以前監察御史亦為千戶,餘不可悉紀。既立部伍,必以軍律相臨,物議紛然,後亦罷之。



 哀宗正大二
 年,議選諸路精兵,直隸密院。先設總領六員,分路揀閱,因相合併。每總領司率數萬人,軍勢既張,乃易總領之名為都尉,班在隨朝四品之列,曰建威、曰虎威、曰破虜、振威、鷹揚、虎賁、振武、折衝、盪寇、殄寇。必以先嘗秉帥權者居是職,雖帥府行院亦不敢以貴重臨之。天興初元,有十五都尉。先六人升授,在京建威奧屯斡里卜,許州折衝夾谷澤本姓樊,陳州振武溫撒辛本姓李,蔡州盪寇蒲察打吉卜,申裕安平完顏斜列,嵩汝振武唐括韓僧。續封金昌府虎威紇石烈乞兒,宣權歸德果毅完顏豬兒,南京殄寇完顏阿拍。宣權潼關都尉三:虎賁完顏陳兒、
 鷹揚內族大婁室、全節。復取河朔諸路歸正人,不問鞍馬有無、譯語能否,悉送密院,增月給三倍它軍,授以官馬,得千餘人,歲時犒燕,名曰忠孝軍。以石抹燕山奴、蒲察定住統之。加以正大已後諸路所虜、臨陳所獲,皆放歸鄉土,同忠孝軍給其犒賞,使河朔俘係知之。故此軍迄于天興至七千,千戶以上將帥尚不預焉。又以歸正人過多,乃係於忠孝籍中別為一軍,減忠孝所給之半,不能射者令閱習一再月,然後試補忠孝軍,是所謂合里合軍也。又以親衛馬軍,舊時所選未精,必加閱試,直取武藝如忠孝軍者得五千人,餘罷歸為步軍。凡進徵,
 忠孝居前,馬軍次之。自正大改立馬軍,隊伍鞍勒兵甲一切更新,將相舊人自謂國家全盛之際馬數則有之,至於軍士精銳、器仗堅整,較之今日有不侔者,中興之期為有望矣。一日布列曹門內教埸,忠孝軍七千,馬軍五千,京師所屯建威都尉軍萬人,內族九住所統親衛軍三千,及阿排所統四千,皆哀宗控制樞密院時所選,教場地約三十頃尚不能容,餘都尉十三四軍猶不在是數。此外,招集義軍名曰忠義,要皆燕、趙亡命,雖獲近用,終不可制,異時擅殺北使唐慶以速金亡者即此曹也。



 ○
 禁軍之制



 本於合扎謀克。合扎者,言親軍也,以近親所領,故以名焉。貞元遷都,更以太祖、遼王宗幹、秦王宗翰之軍為合扎猛安,謂之侍衛親軍,故立侍衛親軍司以統之。舊常選諸軍之材武者為護駕軍,海陵又名上京龍翔軍為神勇軍,正隆二年將南伐,乃罷歸,使就僉調,復於侍衛親軍四猛安舊止曰太祖、遼王、秦王猛安凡三,今曰四猛安,未詳,豈太祖兩猛安耶?內,選三十以下千六百人,騎兵曰龍翔,步兵曰虎步,以備宿衛。五年,罷親軍司,以所掌付大興府,置左右驍騎,所謂從駕軍也,置都副指揮使隸點檢司,步軍都副指揮使隸宣徽院。大定初,親軍置四千人。二十二年,省
 為三千五百。上京亦設守衛軍。是年,尚書省奏:「上京既設皇城提舉官,亦當設軍守衛。」上曰:「可設四百五十,馬一百二十,分三番更代。異時朕至上京,即作兩番巡警,限以半年交替。人日給錢五十、米一升半,馬給芻粟,猛安謀克官可差年四十上下者、軍士並取三十以上者充。」章宗承安四年,增為五千,又增至六千。又有威捷軍。承安增簽弩手千人。凡選弩手之制,先以營造尺度杖,其長六尺,立之謂之等杖。取身與杖等,能踏弩至三石,鋪弦解索登踏閑習,射六箭皆上垛,內二箭中貼者。又選親軍,取身長五尺五寸善騎射者,猛安謀克以名上
 兵部,移點檢司、宣徽院試補之。又設護衛二百人,近侍之執兵仗者也,取五品至七品官子孫及宗室并親軍、諸局分承應人,身長五尺六寸者,選試補之。又設控鶴二百人,皆以備出入者也。



 大將府治之稱號。收國元年十二月,始置咸州軍帥司,以經略遼地,討高永昌,置南路都統司,且以討張覺。天輔五年襲遼主,始有內外諸軍都統之名。時以奚未平,又置奚路都統司,後改為六部路都統司,以遙輦九營為九猛安隸焉,與上京及泰州凡六處置,每司統五六萬人,又以渤海軍為八猛安。凡猛安之上置軍帥,軍帥之上置萬戶,萬戶之上置都
 統。然時亦稱軍帥為猛安,而猛安則稱親管猛安者。燕山既下,循遼制立樞密院于廣寧府,以總漢軍。太宗天會元年,以襲遼主所立西南都統府為西南、西北兩路都統府。三年,以伐宋更為元帥府,置元帥及左、右副,及左、右監軍,左、右都監。金制,都元帥必以諳版孛極烈為之,恒居守而不出。六年,詔還二帥以鎮方面。諸路各設兵馬都總管府,州鎮置節度使,沿邊州則置防禦使。凡州府所募射糧軍、牢城軍,每五百人為一指揮使司,設使,分為四都,都設左右什將及承局押官。其軍數若有餘或不足,則與近者合置,不可合者以三百人或二百
 人亦設指揮使,若百人則止設軍使,百人以上立為都,不及百人止設什將及承局管押官各一員。十年,改南京路都統司為東南路都統司,治東京以鎮高麗。後又置統軍司于大名府。及海陵天德二年八月,改諸京兵馬都部署司為本路都總管府。九月,罷大名統軍司,而置統軍司于山西、河南、陜西三路。以元帥府都監、監軍為使,分統天下之兵。又改烏古迪烈路統軍司為招討司,以婆速路統軍司為總管府。三年,以元帥府為樞密院,罷萬戶之官,詔曰:「太祖開創,因時制宜,材堪統眾授之萬戶,其次千戶及謀克。當時官賞未定,城郭未下,設
 此職許以世襲,乃權宜之制,非經久之利。今子孫相繼專攬威權,其戶不下數萬,與留守總管無異,而世權過之。可罷是官。若舊無千戶之職者,續思增置。國初時賜以國姓,若為子孫者皆令復舊。」正隆末,復升陜西統軍司為都統府。大定五年,復罷府,降為統軍司。尋又設兩招討司,與前凡三,以鎮邊陲。東北路者,初置烏古迪烈部,後置于泰州。泰和間,以去邊尚三百里,宗浩乃命分司于金山。西北路者置於應州,西南路者置於桓州,以重臣知兵者為使,列城堡濠墻,戍守為永制。樞密院每行兵則更為元帥府,罷則復為院。宣宗貞祐三年,徵代
 州戍兵五千,從胥鼎言,留代以屏平陽。興定二年,選募河南、陜西弩手軍二千人為一軍,賜號威勇。及南遷,河北封九公,因其兵假以便宜從事,沿河諸城置行樞密院元帥府,大者有「便宜」之號,小者有「從宜」之名。元光間,時招義軍以三十人為謀克,五謀克為一千戶,四千戶為一萬戶,四萬戶為一副統,兩副統為一都統,此復國初之名也。然又外設一總領提控,故時皆稱元帥為總領云。



 金初因遼諸抹而置群牧,抹之為言無蚊蚋、美水草之地也。天德間,置迪河斡朵、斡里保保亦作本、蒲速里、燕恩、兀者五群牧所,皆仍遼舊名,各設官以治之。又於諸
 色人內,選家富丁多,及品官家子、猛安謀克蒲輦軍與司吏家餘丁及奴,使之司牧,謂之「群子」,分牧馬駝牛羊,為之立蕃息衰耗之刑賞。後稍增其數為九。契丹之亂遂亡其五,四所之所存者馬千餘、牛二百八十餘、羊八百六十、駝九十而已。世宗置所七:曰特滿、忒滿在撫州、斡睹只、蒲速碗、蒲速碗本斡睹只之地,大定七年分其地置之。承安三年改為板底因烏魯古。甌里本、承安三年改為烏鮮烏魯古。烏魯古者言滋息也。合魯碗、耶盧碗。在武平縣、臨潢、泰州之境。大定二十年三月,更定群牧官、詳穩脫朵、知把、群牧人滋息損耗賞罰格。二十一年,敕諸所,馬三歲者付女直人牧之,牛或以借民耕,或又令民畜羊,或以賑貧
 戶。時遣使閱實其數,缺則杖其官,而令牧人償之,匿其實者監察舉覺之。二十八年,蕃息之久,馬至四十七萬,牛十三萬,羊八十七萬,駝四千。明昌五年,散騬馬,令中都、西京、河北東、西路驗民物力分畜之。又令它路民養馬者,死則於前四路所養者給換,若欲用則悉以送官。此金之馬政也。然每有大役,必括於民,及取群官之餘騎,以供戰士焉。宣宗興定元年,定民間收潰軍亡馬之法,及以馬送官酬直之格:「上等馬一匹銀五十兩,中下遞減十兩。不願酬直者,上等二匹補一官,雜班任使,中等三匹,下等四匹,如之。令下十日陳首,限外匿及殺,並
 絞。」又遣官括市民馬,立賞格以示勸,五百匹以上鈔千貫,千匹以上一官,二千匹以上兩官。



 ○養兵之法



 熙宗天眷三年正月,詔歲給遼東戍卒綢絹有差。正隆四年,命河南、陜西統軍司並虞候司順德軍,官兵並增廩給。六年,將南征,以絹萬匹于京城易衣襖穿膝一萬,以給軍。世宗大定三年,南征,軍士每歲可支一千萬貫,官府止有二百萬貫,外可取於官民戶,此軍須錢之所由起也。時言事者,以山東、河南、陜西等路循宋、齊舊例,州縣司吏、弓手於民間驗物力均敷顧錢,名曰「免役」,請以是錢贍軍。至是,省具數以聞,詔罷弓手錢,
 其司吏錢仍舊。四年六月,奏,元帥府乞降軍須錢,上曰:「帥府支費無度,例皆科取於民,甚非朕意。仰會計軍須支用不盡之數,及諸路轉運司見在如實缺用,則別具以聞。」十年四月,命德順州建營屋以處屯軍。十七年七月,歲以羊皮三萬賜西北路戍兵。承安三年,以軍須所費甚大,乞驗天下物力均徵。擬依黃河夫錢例,征軍須錢,驗各路新籍物力,每貫徵錢四貫,西京、北京、遼東路每貫徵錢二貫,臨潢、全州則免征,周年三限送納。恐期遠,遂定制作半年三限輸納。



 凡河南、陜西、山東放老千戶、謀克、蒲輦、正軍、阿里喜等給賞之例,舊軍千戶十年
 以上賞銀五十兩、絹三十匹,不及十年,比附十年以上謀克支。謀克十年以上銀四十兩,絹二十五匹,不及十年銀三十兩、絹二十匹。蒲輦十年以上銀三十兩,絹二十匹,不及十年銀二十兩,絹一十五匹。馬步正軍、阿里喜等勾當不拘年分,放老正軍銀一十五兩、絹一十匹,阿里喜、旗鼓、吹笛、本司火頭人等同銀八兩、絹五匹。三虞候千戶,十年以上銀四十兩,絹二十五匹。不及十年銀三十兩、絹二十匹。謀克二十年以上銀五十兩、絹三十匹,十年以上銀三十兩、絹二十匹,不及十年銀一十兩、絹一十五匹。蒲輦十年以上銀二十兩、絹一十五匹,
 不及十年銀一十五兩、絹一十匹。正軍、阿里喜、勾當不拘年分,放老正軍銀一十兩、絹七匹,阿里喜、旗鼓、吹笛、本司火頭人等同銀五兩,絹四匹。北邊萬戶、千戶、謀克等,歷過軍功及年老放罷給賞之例遷官同從吏部格,正千戶管押萬戶,勾當過一十五年,遷兩官與從五品。不及一十五年年老放罷,遷一官與正六品。若十年以下,遷一官賞銀絹六十兩匹。正謀克管押萬戶,勾當一十五年遷兩官與正六品,不及一十五年年老放罷,遷一官與正七品,若十年以下遷一官賞銀絹五十兩匹。正千戶官押千戶,勾當過二十年,遷一官與正六品,不及二十
 年年老放罷,遷一官與正七品,若十年以下遷一官賞銀絹四十兩匹。正謀克管押千戶以下,依河南、陜西體例。凡鎮防軍,每年試射,射若有出眾,上等賞銀四兩,特異眾者賞十兩銀馬盂。簽充武衛軍,挈家赴京者,人日給六口糧,馬四匹芻槁。



 諸招軍月給例物。邊鋪軍錢五十貫、絹十匹。軍匠上中等錢五十貫、絹五匹,下等錢四十貫、絹四匹。黃河埽兵錢三十貫、絹五匹,射糧軍及溝渠等處埽兵水手,錢二十貫、絹二匹,士兵錢十貫、絹一匹。凡射糧軍指揮使及黃、沁埽兵指揮使,錢粟七貫石、絹六匹,軍使錢粟六貫石、絹同上,什將錢二貫、粟三石,
 春衣錢五貫、秋衣錢十貫。承局押官錢一貫五百文、粟二石,春衣錢五貫、秋衣錢七貫。牢城并士兵錢八百文、粟二石,春衣錢四貫、秋衣錢六貫。邊鋪軍請給與射糧軍同。河南、陜西、山東路統軍司鎮防甲軍、馬軍,猛安錢八貫、米五石二斗、絹八匹、六馬芻粟,謀克錢六貫、米二石八斗、絹六匹、五馬芻粟,蒲輦錢四貫、米石七斗、絹五匹、四馬芻粟,正軍錢二貫、米石五斗、絹四匹、綿十五兩、兩馬芻粟,阿里喜錢一貫五百文、米七斗、絹三匹、綿十兩。步軍,猛安馬二匹、謀克馬一匹芻粟。每馬給芻一束、粟五升,歲仲青野有青草馬可收養則止,惟每猛安當
 差馬七十二匹,四時皆給。又定制河南、山東、河東歲給五月,陜西六月。鎮防軍補買馬錢,河南路正軍五百文,阿里喜隨色人三百文,陜西、山東路正軍三百文,阿里喜隨色人二百文。諸屯田被差及緣邊駐扎捉殺軍,猛安月給錢六貫、米一石八斗、五馬芻粟,謀克錢四貫、米一石二斗、三馬芻粟,蒲輦錢二貫、米六斗、二馬芻粟,正軍錢一貫五百文、米四斗、一馬芻粟,阿里喜隨色人錢一貫、米四斗、一馬芻粟。德順軍指揮使錢六貫、米二石八斗、絹六匹、三馬芻粟,軍使什將錢四貫、米一石七斗、絹五匹,給兩馬料,長行錢二貫、米一石五斗、絹四匹、綿
 十五兩,給一馬料,奚軍謀克錢一貫五百文、米一石五斗、綢絹春秋各一匹,給三馬料,蒲輦錢一貫、米二石七斗、綢絹同上,給二馬料,長行錢一貫、米一石八斗、綢絹同上,飼一馬。北邊臨潢等處永屯駐軍,千戶錢八貫、米五石二斗、絹八匹、飼馬六匹,步軍飼兩馬、地五頃,謀克錢六貫、米二石八斗、絹六匹、飼五馬、地四頃,蒲輦錢四貫、米一石七斗、絹五匹、飼四馬、地三頃,正軍錢二貫、米一石四斗五升、絹四匹、綿十五兩、飼兩馬、地二頃,阿里喜錢一貫五百文、米七斗、絹三匹、綿十兩、地一頃,旗鼓司人與阿里喜同,交替軍錢二貫、米四斗,阿里喜錢
 一貫五百文、米四斗。上番漢軍,千戶月給錢三貫、糧四石、絹八匹、飼四馬,謀克錢二貫五百文、糧一石、絹六匹、飼二馬,正軍錢二貫、米九斗五升、絹四匹。上京路永屯駐軍所除授,千戶月給錢粟十五貫石、絹十匹、綿二十兩、飼三馬,謀克錢六貫、米二石八斗、絹六匹、飼二馬,正軍月支錢二貫五百文、米一石二斗、絹四匹、綿十五兩、飼一馬,阿里喜隨色人錢二貫、米一石二斗,絹四匹、綿十五兩。諸北邊永駐軍,月給補買馬錢四百文,隨色人三百文。貞祐三年,軍前委差及掌軍官,規圖糧料,冒占職役,皆無實員,又見職及遙授者,已有俸給,又與無職事
 者同支券糧,故時議欲省員減所給之數,俟征行則全給之。及興定二年,彰化軍節度使張行信言:「一軍充役,舉家廩給,蓋欲感悅士心,使為國盡力耳!至於無軍之家,復無丁男,而其妻女猶受給何謂耶?」五年,京南行三司官石抹斡魯言:「京南、東、西三路見屯軍戶,老幼四十萬口,歲費糧百四十餘萬石,皆坐食民租,甚非善計。」語在《田制》。諸屯田軍人,如差防送,日給錢一百五十文。看管孝寧宮人,月各給米五斗、柴一車、春秋衣粗布一段、秋絹二匹、綿一十五兩。諸黃院子年滿者,以元請錢糧三分內,給一貫石養老。



\end{pinyinscope}