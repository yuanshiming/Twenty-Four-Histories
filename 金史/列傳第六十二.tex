\article{列傳第六十二}

\begin{pinyinscope}

 忠義四



 ○馬慶祥商衡術甲脫魯灰楊達夫馮延登烏古孫仲端烏古孫奴申蒲察琦蔡八兒溫敦昌孫完顏絳山畢資倫郭蝦蟆



 馬慶祥,字瑞寧,本名習禮吉思。先世自西城入居臨洮
 狄道,以馬為氏,後徙家凈州天山。泰和中,試補尚書省譯史。大安初,衛王始通問大元,選使副,上曰:「習禮吉思智辯通六國語,往必無辱也。」使還,授開封府判官。內城之役充應辦使,不擾而事集。未幾,大元兵出陜右,朝廷命完顏仲元為鳳翔元帥,舉慶祥為副,上曰:「此朕志也,且築城有勞。」即拜鳳翔府路兵馬都總管判官。



 元光元年冬十一月,聞大將萌古不花將攻鳳翔,行省檄慶祥與治中胥謙分道清野。將行,命畫工肖其貌,付其家人。或曰:「君方壯,何乃為此不祥?」慶祥曰:「非汝所知也。」明日遂行。遇先鋒于澮水,戰不利。且行且戰,將及城,會大兵
 邀其歸路,度不能脫,令其騎曰:「吾屬荷國厚恩,竭力效死,乃其職也。」諸騎皆曰:「諾。」人殊死戰,良久矢盡。大兵圍數匝,欲降之,軍擁以行,語言往復,竟不屈而死,年四十有六。元帥郭仲元輿其尸以歸,葬鳳翔普門寺之東。事聞,詔贈輔國上將軍、恒州刺史,謚忠愍。



 胥謙及其子嗣亨亦不屈死,謙贈輔國上將軍、彰化軍節度使,嗣亨贈威遠將軍、鳳翔府判官。



 楨州金勝堡提控僕散胡沙亦死,贈銀青榮祿大夫。



 正大二年,哀宗詔褒死節士,若馬習禮吉思、王清、田榮、李貴、王斌、馮萬奴、張德威、高行中、程濟、姬芃、張山等十有三人,為立褒忠廟,仍錄其孤。二
 人者逸其名,餘亦無所考。



 商衡,字平叔,曹州人。至寧元年,特恩第一人,授鄜州洛郊主簿。以廉能換郿縣,尋辟威戎令。興定三年,歲饑,民無所於糴,衡白行省,得開倉賑貸,全活者甚眾。後因地震城圮,夏人乘釁入侵,衡率蕃部土豪守禦應敵,保以無虞。秩滿,縣人為立生祠。再辟原武令。未幾,入為尚書省令史,轉戶部主事,兩月拜監察御史。



 哀宗姨郕國夫人不時出入宮闈,干預政事,聲跡甚惡。衡上章極言,自是郕國被召乃敢進見。內族慶山奴將兵守盱眙,與李全戰敗,朝廷置而不問。衡上言:「自古敗軍之將必正典
 刑,不爾則無以謝天下。」詔降慶山奴為定國軍節度使。戶部侍郎權尚書曹溫之女在掖庭,親舊干預權利,其家人填委諸司,貪墨彰露。臺臣無敢言者,衡歷數其罪。詔罷溫戶部,改太后府衛尉。再上章言:「溫果可罪,當貶逐,無罪則臣為妄言,豈有是非不別而兩可之理。」哀宗為之動容,乃出溫為汝州防禦使。



 未幾,為右司都事,改同知河平軍節度使。未赴,改樞密院經歷官,遙領昌武軍同知節度使事。丞相完顏賽不領陜西行省,奏衡為左右司員外郎,密院表留,有旨:「行省地重,急於得人,可從丞相奏。」明年,召遷,行省再奏留之。正大八年,以母喪
 還京師。十月,起復為秦藍總帥府經歷官。天興元年二月,關陜行省徒單兀典等敗於鐵嶺,衡未知諸帥存歿,招集潰軍以須其至。遂為兵士所得,欲降之,不為屈。監至長水縣東岳祠前,誘之使招洛陽,衡曰:「我洛陽識何人,為汝招之耶?」兵知不可誘,欲捽其巾。衡瞋目大呼曰:「汝欲脅從我耶?」終不肯降,望闕瞻拜曰:「主將無狀,亡兵失利。臣子罪責,亦無所逃,但以一死報國耳。」遂引佩刀自剄,年四十有六。



 正大初,河間許古詣闕拜章,言:「八座率非其材,省寺小臣有可任宰相者,不大升黜之則無以致中興。」章奏,詔古赴都堂,問孰為可相者,古以衡對,
 則衡之材可知矣。



 術甲脫魯灰,上京人,世為北京路部長。其先有開國功,授北京路宋阿答阿猛安,脫魯灰自幼襲爵。貞祐二年,宣宗遷汴,率本部兵赴中都扈從,上喜,特授御前馬步軍都總領。宋人略南鄙,命同簽樞密院事時全將大軍南伐,脫魯灰率本部屢摧宋兵破城寨,以功遙授昌武軍節度使、元帥右都監、行蔡、息等路元帥府事。既而,宋人有因畜牧越境者,邏卒擒之,法當械送朝廷,脫魯灰曰:「國家自遷都以來,境土日蹙,民力凋耗,幸邊無事,人稍得息。若戮此曹,則邊釁復生,兵連禍結矣。不如釋之,
 以絕兵端。」



 哀宗即位,授鎮南軍節度使、蔡州管內觀察使、行戶、工部尚書。時大元兵入陜西。乃上章曰:「宋人與我為仇敵,頃以力屈自保,非其本心。今陜西被兵,河南出師,轉戰連年不絕,兵死于陣,民疲于役,國力竭矣。壽、泗一帶南接盱、楚,紅襖賊李全巢穴也。萬一宋人諜知,與全乘虛而入,腹背受敵,非計之得者也。臣已令所部沿邊警斥,以備非常。宜敕壽、泗帥臣謹斥候,嚴烽燧,常若敵至,此兵法所謂『無恃其不來,特吾有以待之』之道也。」上是而行之。



 正大二年秋,傳言宋人將入侵,農司令民先期刈禾,脫魯灰曰:「夫民所恃以仰事俯育及供億國家
 者,秋成而已。今使秋無所獲,國何以仰,民何以給?」遂遣軍巡邏,聽民待熟而刈,宋人卒不入寇。諜者又報光州汪太尉將以八月發兵來取真陽,議者請籍丁男以備,脫魯灰曰:「汪太尉恇怯人耳,寧敢為此?必姦人聲言來寇,欲使吾民廢務也,不可信。」已而果然。



 叛人焦風子者,沿河南北屢為反復,朝廷授以提控之職,令將三千人戍遂平。四年春,風子謀率其眾入宋,脫魯灰策之,以兵數千伏鄱陽道,賊果夜出此途,伏發殪之。



 七年,大元兵攻藍關,至八渡倉退。舉朝皆賀,以為無事。脫魯灰獨言曰:「潼關險隘,兵精足用。然商、洛以南瀕於宋境,大山重
 復,宋人不知守,國家亦不能逾宋境屯戍。大兵若由散關入興元,下金、房,繞出襄、漢,北入鄧鄙,則大事去矣。宜與宋人釋怨,諭以輔車之勢,脣亡齒寒,彼必見從。據其險要以備,不然必敗。」是秋,必授小關子元帥,屯商州大吉口。



 九年春,從行省參政徒單吾典將潼關兵入援,至商山遇雪,大兵邀擊之,士卒飢凍,不能戰而潰。脫魯灰被執不屈,拔佩刀自殺。



 楊達夫,字晉卿,耀州三原人。泰和三年進士。有才幹,所至可紀。召補省掾,草奏章,坐字誤,降平涼府判官。嘗主鄠縣簿,事一從簡,吏民樂之。達夫亦愛其山水之勝,因
 家焉。日以詩酒自娛,了無宦情。會有詔徙民東入關,達夫與眾行,及韶,避兵于州北之橫嶺,為游騎所執,將褫衣害之。達夫挺然直立馬首,略無所懼。稍侵辱之,即大言曰:「我金國臣子,即為汝所執,不過一死,忍裸袒以黷天日耶!」遂見殺。兩山潛伏之民竊觀之者,皆相告曰:「若此好官,異日祠之,當作我橫嶺之神。」



 馮延登,字子俊,吉州吉鄉人。世業醫。延登幼穎悟,既長,事舉業,承安二年登詞賦進士第。調臨真簿、德順州軍事判官。泰和元年,轉寧邊令。大安元年秋七月,霜害稼,民艱于食,延登發粟賑貸,全活甚眾。貞祐二年,補尚書
 省令史,尋授河中府判官、兼行尚書省左右司員外郎。興定五年,入為國史院編修官,改太常博士。元光二年,知登聞鼓院,兼翰林修撰,奉使夏國,就充接送伴使。正大七年十二月,遷國子祭酒。假翰林學士承旨,充國信使。以八年春奉國書朝見於虢縣御營。有旨問:「汝識鳳翔帥否?」對曰:「識之。」又問:「何如人?」曰:「敏於事者也。」又問:「汝能招之使降即貰汝死,不則殺汝矣。」曰:「臣奉書請和,招降豈使職乎。招降亦死,還朝亦死,不若今日即死為愈也。」明日,復問:「汝曾思之不?」對如前,問至再三,執義不回。又明日,乃喻旨云:「汝罪應死,但古無殺使者理,汝愛汝
 須髯猶汝命也。」叱左右以刀截去之,延登岸然不動,乃監之豐州。二年後放還,哀宗撫慰久之,復以為祭酒,歷禮、吏二部侍郎,權刑部尚書。明年,大元兵圍汴京,倉猝逃難,為騎兵所得,欲擁而北行。延登辭情慷慨,義不受辱,遂躍城旁井中,年五十八。



 烏古孫仲端,本名卜吉,字子正。承安二年策論進士。宣宗時,累官禮部侍郎。與翰林待制安延珍奉使乞和于大元,謁見太師國王木華黎,於是安延珍留止,仲端獨往。並大夏,涉流沙,踰葱嶺,至西域,進見太祖皇帝,致其使事乃還。自興定四年七月啟行,明年十二月還至。朝
 廷嘉其有奉使勞,進官兩階,延珍進一階。歷裕州刺史。正大元年,召為御史中丞,奉詔安撫陜西。及歸,權參知政事。



 正大五年十二月,知開封府事完顏麻斤出、吏部郎中楊居仁以奉使不職,尚書省具獄,有旨釋之備再使。仲端言曰;「麻斤出等辱君命,失臣節,大不敬,宜償禮幣誅之。」奏上,麻斤出等免死除名。會議降大軍事,及諍太后奉佛,涉亡家敗國之語,上怒,貶同州節度使。



 哀宗將遷歸德,召為翰林學士承旨,兼同簽大睦親府事,留守汴京。及大元兵圍汴,日久食盡,諸將不相統一,仲端自度汴中事變不測。一日與同年汝州防禦裴滿思忠
 小飲,談太學同舍事以為笑樂,因數言「人死亦易事耳。」思忠曰:「吾兄何故頻出此語?」仲端因寫一詩示之,其詩大概謂人生大似巢燕,或在華屋杏梁,或在村居茅茨,及秋社甫臨,皆當逝去。人生雖有富貴貧賤不同,要之終有一死耳。書畢,連飲數杯,送思忠出門,曰:「此別終天矣。」思忠去,仲端即自縊,其妻亦從死。明日,崔立變。



 仲端為人樂易寬厚知大體,奉公好善,獨得士譽。一子名愛實,嘗為護衛、奉御,以誅官奴功授節度、世襲千戶。



 思忠名正之,本名蒲剌篤,亦承安二年進士。



 烏古孫奴申,字道遠。由譯史入官。性伉特敢為,有直氣。
 嘗為監察御史,時中丞完顏百家以酷烈聞,奴申以事糾罷,朝士聳然。後為左司郎中、近侍局使,皆有名。哀宗東遷,為諫議大夫、近侍局使、行省左右司郎中、兼知宮省事,留汴京居守。崔立變之明日,同御史大夫裴滿阿虎帶自縊死於臺中。是日,戶部尚書完顏珠顆亦自縊。



 阿虎帶字仲寧,珠顆字仲平,皆女直進士。



 時不辱而死者,奉御完顏忙哥、大睦親府事烏古孫仲端。大理裴滿德輝、右副點檢完顏阿撒、參政完顏奴申之子麻因,可知者數人,餘各有傳。



 蒲察琦,本名阿憐,字仁卿,棣州陽信人。試補刑部掾。兄
 世襲謀克,兄死,琦承襲。正大六年,秦、藍總帥府辟琦為安平都尉粘葛合典下都統兼知事。其冬,小關破,事勢已迫。琦常在合典左右,合典令避矢石,琦不去,曰:「業已從公,死生當共之,尚安所避耶。」哀宗遷歸德,汴京立講議所,受陳言文字,其官則御史大夫納合寧以下十七人,皆朝臣之選,而琦以有論議預焉。時左司都事元好問領講議,兼看讀陳言文字,與琦甚相得。崔立變後,令改易巾髻,琦謂好問曰:「今日易巾髻,在京人皆可,獨琦不可。琦一刑部譯史,襲先兄世爵,安忍作此?今以一死付公。然死則即死,付公一言亦剩矣。」因泣涕而別。琦既
 至其家,母氏方晝寢,驚而寤。琦問阿母何為,母曰:「適夢三人潛伏梁間,故驚寤。」仁卿跪曰:「梁上人,鬼也。兒意在懸梁,阿母夢先見耳。」家人輩泣勸曰:「君不念老母歟?」母止之曰:「勿勸,兒所處是矣。」即自縊,時年四十餘。



 琦性沉靜好讀書,知古今事。其母完顏氏,以孝謹稱。



 蔡八兒,不知其所始。矯捷有勇,性純質可任。時為忠孝軍元帥。天興二年,自息州入援,會大將奔盞遣數百騎駐城東,令人大呼曰:「城中速降,當免殺戮,不然無噍類矣。」於是,上登城,遣八兒率挽強兵百餘潛出暗門,渡汝水,左右交射之。自是兵不復薄城,築長壘為久困計。上
 令分軍防守四城,以殿前都點檢兀林答胡土守西面,八兒副之。已而哀宗度蔡城不守,傳位承麟。群臣入賀,班定,八兒不拜,謂所親曰:「事至於此,有死而已,安能更事一君乎!」遂戰死。



 毛牷者,恩州人。貞祐中為盜,宣宗南渡,率眾歸國,署為義軍招撫。哀宗遷蔡,以牷為都尉。圍城之戰。牷力居多,城破自縊。其子先牷戰歿。



 時死事者則有閻忠、郝乙、王阿驢、樊喬焉。



 忠,滑州人。衛王時,開州剌史賽哥叛,忠單騎入城,縛賽哥以出,由是漸被擢用。



 乙,磁州人,同日戰死,哀宗贈官。



 阿驢、樊喬,皆河中人,初為砲軍萬戶。鳳翔
 破,北降,從軍攻汴,司砲如故,即紿主者曰:「砲利於短,不利於長。」信之,使截其木數尺、綆十餘握,由是機雖起伏,所擊無力。即日二人皆捐家走城。



 是時,女直人無死事者,長公主言於哀宗曰:「近來立功效命多諸色人,無事時則自家人爭強,有事則他人盡力,焉得不怨。」上默然。餘各有傳。



 溫敦昌孫,皇太后之姪,衛尉七十五之子。為人短小精悍,性復愷弟。累遷諸局分官。上幸蔡,授殿前左副點檢。圍城中,數引軍潛出巡邏。時尚食須魚,汝河魚甚美,上以水多浮尸,惡之。城西有積水曰練江,魚大且多,往捕
 必軍衛乃可。昌孫常自領兵以往,所得動千餘斤,分賜將士。後知其出,左右設伏,伺而邀之,力戰而死。蔡城破,前監察御史納坦胡失打聞之,慟哭,投水而死。



 完顏絳山,哀宗之奉御也,系出始祖。天興二年十月,蔡城被圍,城中飢民萬餘訴於有司求出,有司難之,民大呼於道。上聞之,遣近侍官分監四門,門日出千人,必老稚羸疾者聽其出。絳山時在北門,憫人之飢,出過其數,命杖之四十。然出者多泄城中虛實,尋止之。



 三年正月己酉,蔡城破,哀宗傳立承麟,即自縊于幽蘭軒。權點檢內族斜烈矯制召承御石盞氏、近侍局大使焦春和、內
 侍局殿頭宋珪赴上前,曉以名分大義,及侍從官巴良弼、阿勒根文卿皆從死。斜烈將死,遺言絳山,使焚幽蘭軒。火方熾,子城破,大兵突入,近侍左右皆走避,獨絳山留不去,為兵所執,問曰:「汝為誰?」絳山曰:「吾奉御絳山也。」兵曰:「眾皆散走,而獨後何也?」曰:「吾君終于是,吾候火滅灰寒,收瘞其骨耳。」兵笑曰:「若狂者耶?汝命且不能保,能瘞而君耶?」絳山曰:「人各事其君。吾君有天下十餘年,功業弗終,身死社稷,忍使暴露遺骸與士卒等耶?吾逆知君輩必不遺吾,吾是以留。果瘞吾君之後,雖寸斬吾不恨矣。」兵以告其帥,奔盞曰:「此奇男子也。」許之。絳山乃掇
 其餘燼,裹以弊衾,瘞于汝水之旁。再拜號哭,將赴汝水死。軍士救之得免,後不知所終。



 畢資倫,縉山人也。泰和南征,以傭雇從軍,軍還,例授進義副尉。崇慶元年,改縉山為鎮州,術虎高琪為防禦使、行元帥府事于是州,選資倫為防城軍千戶。至寧元年秋,大元兵至鎮州,高琪棄城遁。資倫行及昌平,收避遷民兵,轉戰有功,擢授都統軍。軍數千,與軍中將領沉思忠、寧子都輩同隸一府,屯鄭州及衛州,時號「沈、畢軍」。積功至都總領,思忠為副都尉。



 僕散阿海南征,軍次梅林關不得過,阿海問諸將誰能取此關者,資倫首出應命。
 問須軍士幾何,曰:「止用資倫所統足矣,不煩餘軍。」明日遲明,出宋軍不意,引兵簿之,萬眾崩,遂取梅林關。阿海軍得南行,留提控王祿軍萬人守關。不數日,宋兵奪關守之,阿海以梅林歸途為敵據,計無所出,復問:「誰能取梅林者,以帥職賞之。」資倫復出應命,以本軍再奪梅林。阿海破蘄、黃,按軍而還,論功資論第一,授遙領同知昌武軍節度使、宣差總領都提控。



 既而樞密院以資倫、思忠不相能,恐敗事,以資倫統本軍屯泗州。興定五年正月戊戌,提控王祿湯餅會軍中宴飲,宋龜山統制時青乘隙襲破泗州西城。資倫知失計,墮南城求死,為宋軍
 所執,以見時青。青說之曰:「畢宣差,我知爾好男子,亦宜相時達變。金國勢已衰弱,爾肯降我,宋亦不負爾。若不從,見劉天帥即死矣。」資倫極口罵曰:「時青逆賊聽我言。我出身至貧賤,結柳器為生,自征南始得一官,今職居三品。不幸失國家城池,甘分一死尚不能報,肯從汝反賊求生耶!」青知無降意,下盱眙獄。時臨淮令李某者亦被執,後得歸,為泗州從宜移剌羊哥言其事。羊哥以資倫惡語罵時青必被殺,即以死不屈節聞于朝。時資倫子牛兒年十二,居宿州,收充皇后位奉閣舍人。



 宋人亦賞資倫忠憤不撓,欲全活之,鈐以鐵繩,囚於鎮江府土
 獄,略給衣食使不至寒餓,脅誘百方,時一引出問云:「汝降否?」資倫或罵或不語,如是十四年。及盱眙將士降宋,宋使總帥納合買住已下北望哭拜,謂之辭故主,驅資倫在旁觀之。資倫見買住罵曰:「納合買住,國家未嘗負汝,何所求死不可,乃作如此觜鼻耶!」買住俯首不敢仰視。



 及蔡州破,哀宗自縊,宋人以告資倫。資倫歎曰:「吾無所望矣。容我一祭吾君乃降耳。」宋人信之,為屠牛羊設祭鎮江南岸。資倫祭畢,伏地大哭,乘其不防投江水而死。宋人義之,宣示四方,仍議為立祠。鎮江之囚有方士者親嘗見之,以告元好問,及言泗州城陷資倫被執事,
 且曰:「資倫長身,面赤色,顴頰微高,髯疏而黃。資稟質直,重然諾,故其堅忍守節卓卓如此。」《宣宗實錄》載資倫為亂兵所殺,當時傳聞不得其實云。



 郭蝦蟆,會州人。世為保甲射生手,與兄祿大俱以善射應募。興定初,祿大以功遷遙授同知平涼府事、兼會州刺史,進官一階,賜姓顏盞。夏人攻會州,祿大遙見其主兵者人馬皆衣金,出入陣中,約二百餘步,一發中其吭,殪之。又射一人,矢貫兩手於樹,敵大駭。城破,祿大、蝦蟆俱被禽。夏人憐其技,囚之,兄弟皆誓死不屈。朝廷聞之,議加優獎,而未知存沒,乃特遷祿大子伴牛官一階,授
 巡尉職,以旌其忠。其後兄弟謀奔會,自拔其鬚,事覺,祿大竟為所殺,蝦蟆獨拔歸。上思祿大之忠,命復遷伴牛官一階,遙授會州軍事判官,蝦蟆遙授鞏州鈐轄。會言者乞獎用祿大弟,遂遷蝦蟆官兩階,授同知蘭州軍州事。



 興定五年冬,夏人萬餘侵定西,蝦蟆敗之,斬首七百,獲馬五十匹,以功遷同知臨洮府事。元光二年,夏人步騎數十萬攻鳳翔甚急,元帥赤盞合喜以蝦蟆總領軍事。從巡城,濠外一人坐胡床,以箭力不及,氣貌若蔑視城守者。合喜指似蝦蟆云:「汝能射此人否?」蝦蟆測量遠近,曰:「可。」蝦蟆平時發矢,伺腋下甲不掩處射之無不中,
 即持弓矢伺坐者舉肘,一發而斃。兵退,升遙授靜難軍節度使,尋改通遠軍節度使,授山東西路斡可必剌謀克,仍遣使賞賚,遍諭諸郡焉。



 是年冬,蝦蟆與鞏州元帥田瑞攻取會州。蝦蟆率騎兵五百皆被赭衲,蔽州之南山而下,夏人猝望之以為神。城上有舉手於懸風版者,蝦蟆射之,手與版俱貫。凡射死數百人。夏人震恐,乃出降。蓋會州為夏人所據近四年,至是復焉。



 正大初,田瑞據鞏州叛,詔陜西兩行省併力擊之。蝦蟆率眾先登,瑞開門突出,為其弟濟所殺,斬首五千餘級,以功遷遙授知鳳翔府事、本路兵馬都總管、元帥左都監、兼行蘭、會、
 洮、河元帥府事。六年九月,蝦蟆進西馬二匹,詔曰:「卿武藝超絕。此馬可充戰用,朕乘此豈能盡其力。既入進,即尚廄物也,就以賜卿。」仍賜金鼎一、玉兔鶻一,并所遣郭倫哥等物有差。



 天興二年,哀宗遷蔡州,慮孤城不能保,擬遷鞏昌,以粘葛完展為鞏昌行省。三年春正月,完展聞蔡已破,欲安眾心,城守以待嗣立者,乃遣人稱使者至自蔡,有旨宣諭。綏德州帥汪世顯者亦知蔡凶問,且嫉完展制己,欲發矯詔事,因以兵圖之,然懼蝦蟆威望,乃遣使約蝦蟆併力破鞏昌。使者至,蝦蟆謂之曰:「粘葛公奉詔為行省,號令孰敢不從。今主上受圍於蔡,擬遷
 鞏昌。國家危急之際,我輩既不能致死赴援,又不能葉眾奉迎,乃欲攻粘葛公,先廢遷幸之地,上至何所歸乎。汝帥若欲背國家,任自為之,何及於我。」世顯即攻鞏昌破之,劫殺完展,送款於大元,復遣使者二十餘輩諭蝦蟆以禍福,不從。



 甲午春,金國已亡,西州無不歸順者,獨蝦蟆堅守孤城。丙申歲冬十月,大兵併力攻之。蝦蟆度不能支,集州中所有金銀銅鐵,雜鑄為砲以擊攻者,殺牛馬以食戰士,又自焚盧舍積聚,曰:「無至資兵。」日與血戰,而大兵亦不能卒拔。及軍士死傷者眾,乃命積薪於州廨,呼集家人及城中將校妻女,閉諸一室,將自焚之。
 蝦蟆之妾欲有所訴,立斬以徇。火既熾,率將士於火前持滿以待。城破,兵填委以入,鏖戰既久,士卒有弓盡矢絕者,挺身入火中。蝦蟆獨上大草積,以門扉自蔽,發二三百矢無不中者,矢盡,投弓劍于火自焚。城中無一人肯降者。蝦蟆死時年四十五。土人為立祠。



 完展字世昌。泰和三年策論進士。初為行省,以蠟丸為詔,期以天興二年九月集大軍與上會於饒峰關,出宋不意取興元。既而不果云。



\end{pinyinscope}