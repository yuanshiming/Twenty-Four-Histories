\article{本紀第一}

\begin{pinyinscope}

 世紀



 金之先,出靺鞨氏。靺鞨本號勿吉。勿吉,古肅慎地也。元魏時,勿吉有七部:曰粟末部、曰伯咄部、曰安車骨部、曰拂涅部、曰號室部、曰黑水部、曰白山部。隋稱靺鞨,而七部並同。唐初,有黑水靺鞨、慄末靺鞨,其五部無聞。粟末靺鞨始附高麗,姓大氏。李績破高麗,粟末靺鞨保東牟山。後為渤海,稱王,傳十餘世。有文字、禮樂、官府、制度。有五京、十五府、六十二州。黑水靺鞨居肅慎地,東瀕海,南接高麗,亦附於高麗。嘗以兵十五萬眾助高麗拒唐太宗,敗于安市。開元中,來朝,置黑水府,以部長為都督、刺史,置長史監之。賜都督姓李氏,名獻誠,領黑水經略使。其後渤海盛強,黑水役屬之,朝貢遂絕。五代時,契丹盡取渤海地,而黑水靺鞨附屬于契丹。其在南者籍契丹,號熟女直;其在北者不在契丹籍,號生女直。生女直地有混同江、長白山,混同江亦號黑龍江,所謂「白山黑水」是也。



 金之始祖諱函普,初從高麗來,年已六十餘矣。兄阿古迺好佛,留高麗不肯從,曰:「後世子孫必有能相聚
 者,吾不能去也。」獨與弟保活里俱。始祖居完顏部僕乾水之涯,保活里居耶懶。其後胡十門以曷蘇館歸太祖,自言其祖兄弟三人相別而去,蓋自謂阿古乃之後。石土門、迪古乃,保活里之裔也。及太祖敗遼兵于境上,獲耶律謝十,乃使梁福、斡荅刺招諭渤海人曰:「女直、渤海本同一家。」蓋其初皆勿吉之七部也。始祖至完顏部,居久之,其部人嘗殺它族之人,由是兩族交惡,哄鬥不能解。完顏部人謂始祖曰:「若能為部人解此怨,使兩族不相殺,部有賢女,年六十而未嫁,當以相配,仍為同部。」始祖曰:「諾。」乃自往諭之曰:「殺一人而鬥不解,損傷益多。曷
 若止誅首亂者一人,部內以物納償汝,可以無鬥,而且獲利焉。」怨家從之。乃為約曰:「凡有殺傷人者,徵其家人口一、馬十偶、牸牛十、黃金六兩,與所殺傷之家,即兩解,不得私斗。」曰:「謹如約。」女直之俗,殺人償馬牛三十,自此始。既備償如約,部眾信服之,謝以青牛一,并許歸六十之婦。始祖乃以青牛為聘禮而納之,并得其貲產。後生二男,長曰烏魯,次曰斡魯,一女曰注思板,遂為完顏部人。天會十四年,追謚景元皇帝,廟號始祖。皇統四年,號其藏曰光陵。五年,增謚始祖懿憲景元皇帝。



 子德帝,諱烏魯。天會十四年,追謚德皇帝。皇統四年,號其藏曰熙
 陵。五年,增謚淵穆玄德皇帝。



 子安帝,諱跋海。天會十四年,追謚安皇帝。皇統四年,號其藏建陵。五年,增謚和靖慶安皇帝。



 子獻祖,諱綏可。黑水舊俗無室廬,負山水坎地,梁木其上,覆以土,夏則出隨水草以居,冬則入處其中,遷徙不常。獻祖乃徙居海古水,耕墾樹藝,始築室,有棟宇之制,人呼其地為納葛里。納葛里者,漢語居室也。自此遂定居於安出虎水之側矣。天會十四年,追謚定昭皇帝,廟號獻祖。皇統四年,號其藏曰輝陵。五年,增謚獻祖純烈定昭皇帝。



 子昭祖,諱石魯,剛毅質直。生女直無書契,無約束,不可檢制。昭祖欲稍立條教,諸父、部人
 皆不悅,欲坑殺之。已被執,叔父謝里忽知部眾將殺昭祖,曰:「吾兄子,賢人也,必能承家,安輯部眾,此輩奈何輒欲坑殺之!」亟往,彎弓注矢射于眾中,劫執者皆散走,昭祖乃得免。昭祖稍以條教為治,部落浸強。遼以惕隱官之。諸部猶以舊俗,不肯用條教。昭祖耀武至于青嶺、白山,順者撫之,不從者討伐之,入于蘇濱、耶懶之地,所至克捷,還經僕燕水。僕燕,漢語惡瘡也。昭祖惡其地名,雖已困憊,不肯止。行至姑里甸,得疾。迨夜,寢于村舍。有盜至,遂中夜啟行,至逼剌紀村止焉。是夕,卒。載柩而行,遇賊於路,奪柩去。部眾追賊與戰,復得柩。加古部人蒲虎
 復來襲之,垂及,蒲虎問諸路人曰:「石魯柩去此幾何?」其人曰:「遠矣,追之不及也。」蒲虎遂止。於是乃得歸葬焉。生女直之俗,至昭祖時稍用條教,民頗聽從,尚未有文字,無官府,不知歲月晦朔,是以年壽脩短莫得而考焉。天會十五年,追謚成襄皇帝,廟號昭祖。皇統四年,藏號安陵。五年,增謚昭祖武惠成襄皇帝。



 子景祖,諱烏古迺。遼太平元年辛酉歲生。自始祖至此,已六世矣。景祖稍役屬諸部,自白山、耶悔、統門、耶懶、土骨論之屬,以至五國之長,皆聽命。是時,遼之邊民有逃而歸者。及遼以兵徙鐵勒、烏惹之民,鐵勒、烏惹多不肯徙,亦逃而來歸。遼使
 曷魯林牙將兵來索逋逃之民。景祖恐遼兵深入,盡得山川道路險易,或將圖之,乃以計止之曰:「兵若深入,諸部必驚擾,變生不測,逋戶亦不可得,非計也。」曷魯以為然,遂止其軍,與曷魯自行索之。是時,鄰部雖稍從,孩懶水烏林答部石顯尚拒阻不服。攻之,不克。景祖以計告於遼主,遼主遣使責讓石顯。石顯乃遣其子婆諸刊入朝,遼主厚賜遣還。其後石顯與婆諸刊入見遼主於春蒐。遼主乃留石顯於邊地,而遣婆諸刊還所部。景祖之謀也。既而五國蒲聶部節度使拔乙門叛遼,鷹路不通。遼人將討之,先遣同乾來諭旨。景祖曰:「可以計取。若用
 兵,彼將走保險阻,非歲月可平也。」遼人從之。蓋景祖終畏遼兵之入其境也,故自以為功。於是景祖陽與拔乙門為好,而以妻子為質,襲而擒之,獻於遼主。遼主召見於寢殿,燕賜加等,以為生女直部族節度使。遼人呼節度使為太師,金人稱都太師者自此始。遼主將刻印與之,景祖不肯繫遼籍,辭曰:「請俟他日。」遼主終欲與之,遣使來。景祖詭使部人揚言曰:「主公若受印繫籍,部人必殺之!」用是以拒之,遼使乃還。既為節度使,有官屬,紀綱漸立矣。生女直舊無鐵,鄰國有以甲胄來鬻者,傾貲厚賈以與貿易,亦令昆弟族人皆售之。得鐵既多,因之以
 修弓矢,備器械,兵勢稍振,前後願附者眾。斡泯水蒲察部、泰神忒保水完顏部、統門水溫迪痕部、神隱水完顏部,皆相繼來附。



 景祖為人寬恕,能容物,平生不見喜慍。推財與人,分食解衣,無所吝惜。人或忤之,亦不念。先時,有叛去者,遣人諭誘之。叛者曰:「汝主,活羅也。活羅,吾能獲之,吾豈能為活羅屈哉!」活羅,漢語慈烏也。北方有之,狀如大雞,善啄物,見馬牛橐駝脊間有瘡,啄其脊間食之,馬牛輒死,若飢不得食,雖砂石亦食之。景祖嗜酒好色,飲啖過人,時人呼曰活羅,故彼以此訕之,亦不以介意。其後訕者力屈來降,厚賜遣還。曷懶水有率眾降者,
 錄其歲月姓名,即遣去,俾復其故。人以此益信服之。遼咸雍八年,五國沒拈部謝野勃堇叛遼,鷹路不通。景祖伐之,謝野來禦。景祖被重鎧,率眾力戰。謝野兵敗,走拔里邁濼。時方十月,冰忽解,謝野不能軍,眾皆潰去,乃旋師。道中遇逋亡,要遮險阻,晝夜拒戰,比至部已憊。即往見遼邊將達魯骨,自陳敗謝野功。行次來流水,未見達魯骨,疾作而復,卒於家,年五十四。天會十四年,追謚惠桓皇帝,廟號景祖。皇統四年,藏號定陵。五年,增謚景祖英烈惠桓皇帝。



 第二子襲節度使,是為世祖,諱劾里缽。生女直之俗,生子年長即異居。景祖九子,元配唐括氏
 生劾者,次世祖,次劾孫,次肅宗,次穆宗。及當異居,景祖曰:「劾者柔和,可治家務。劾里缽有器量智識,何事不成。劾孫亦柔善人耳。」乃命劾者與世祖同居,劾孫與肅宗同居。景祖卒,世祖繼之。世祖卒,肅宗繼之。肅宗卒,穆宗繼之。穆宗復傳世祖之子,至於太祖,竟登大位焉。世祖,遼重熙八年己卯歲生。遼咸雍十年,襲節度使。景祖異母弟跋黑有異志,世祖慮其為變,加意事之,不使將兵,但為部長。跋黑遂誘桓赧、散達、烏春、窩謀罕為亂,及間諸部使貳于世祖。世祖猶欲撫慰之,語在跋黑、桓赧等傳中。世祖嘗買加古部鍛工烏不屯被甲九十,烏春欲
 托此以為兵端,世祖還其甲,語在《烏春傳》。部中有流言曰:「欲生則附於跋黑,欲死則附于劾里缽、頗剌淑。」世祖聞之,疑焉。無以察之,乃佯為具裝,欲有所往者,陰遣人揚言曰:「寇至!」部眾聞者莫知虛實,有保於跋黑之室者,有保於世祖之室者,世祖乃盡得兄弟部屬向背彼此之情矣。



 間數年,烏春來攻,世祖拒之。時十月已半,大雨累晝夜,冰澌覆地,烏春不能進。既而悔曰:「此天也!」乃引兵去。烏春舍於阿里矮村滓不乃家,而以兵圍其弟勝昆於胡不村。兵退,勝昆執其兄滓不乃,而請蒞殺于世祖,且請免其孥戮。從之。桓赧、散達亦舉兵,遣肅宗拒之。
 當是時,烏春兵在北,桓赧兵在南,其勢甚盛。戒之曰:「可和則與之和,否則決戰!」肅宗兵敗。會烏春以久雨解去,世祖乃以偏師涉舍很水,經貼割水,覆桓赧、散達之家。明日,大霧晦冥,失道,至婆多吐水乃覺。即還至舍很、貼割之間,升高阜望之,見六騎來,大呼,馳擊之。世祖射一人斃,生獲五人,問之,乃知卜灰、撒骨出使助恆赧、散達者也。世祖至桓赧、散達所居,焚蕩其室家。殺百許人,舊將主保亦死之。比世祖還,與肅宗會,肅宗兵又敗矣。世祖讓肅宗失利之狀。遣人議和,桓赧、散達曰:「以爾盈歌之大赤馬、辭不失之紫騮馬與我,我則和。」二馬皆女直
 名馬,不許。



 桓赧、散達大會諸部來攻,過裴滿部,以其附於世祖也,縱火焚之。蒲察部沙祇勃堇、胡補答勃堇使阿喜來告難,世祖使之詭從以自全,曰:「戰則以旗鼓自別。」世祖往禦桓赧之眾,將行,有報者曰:「跋黑食於愛妾之父家,肉脹咽死矣!」乃遣肅宗求援於遼,遂率眾出。使辭不失取海姑兄弟兵,已而乃知海姑兄弟貳於桓赧矣。欲併取其眾,徑至海姑。偵者報曰:「敵已至。」將戰,世祖戒辭不失曰:「汝先陣於脫豁改原,待吾三揚旗,三鳴鼓,即棄旗決戰。死生惟在今日,命不足惜!」使裴滿胡喜牽大紫騮馬以為貳馬,馳至陣。時桓赧、散達盛強,世祖軍吏
 未戰而懼,皆植立無人色。世祖陽陽如平常,亦無責讓之言,但令士卒解甲少憩,以水沃面,調蒨水飲之。有頃,訓勵之,軍勢復振。乃避眾獨引穆宗,執其手密與之言曰:「今日之事,若勝則已,萬一有不勝,吾必無生。汝今介馬遙觀,勿預戰事。若我死,汝勿收吾骨,勿顧戀親戚,亟馳馬奔告汝兄頗剌淑,于遼繫籍受印,乞師以報此仇!」語畢,袒袖,不被甲,以縕袍垂襴護前後心,韔弓提劍,三揚旗,三鳴鼓,棄旗搏戰,身為軍鋒,突入敵陣,眾從之。辭不失從後奮擊,大敗之。乘勝逐之,自阿不彎至于北隘甸,死者如仆麻,破多吐水水為之赤,棄車甲馬牛軍
 實盡獲之。世祖曰:「今日之捷,非天不能及此,亦可以知足矣。雖縱之去,敗軍之氣,沒世不振。」乃引軍還。世祖視其戰地,馳突成大路,闊且三十隴。手殺九人,自相重積,人皆異之。桓赧、散達自此不能復聚,未幾,各以其屬來降,遼大安七年也。



 初,桓赧兄弟之變,魯部卜灰、蒲察部撒骨出助之。至是,招之,不肯和。卜灰之黨石魯遂殺卜灰來降。撒骨出追躡亡者,道傍人潛射之,中口而死。自是舊部悉歸。景祖時,斡勒部人盃乃來屬,及是,有他志。會其家失火,因以縱火誣歡部,世祖征償如約。盃乃不自安,遂結烏春、窩謀罕舉兵。使肅宗與戰,敗之,獲
 杯乃,世祖獻之於遼。臘醅、麻產侵掠野居女直,略來流水牧馬。世祖擊之,中四創,久之疾愈。臘醅等復略穆宗牧馬,交結諸部。世祖復伐之,臘醅等紿降,乃旋。臘醅得姑里甸兵百十有七人,據暮稜水守險,石顯子婆諸刊亦在其中。世祖圍而克之,盡獲姑里甸兵。麻產遁去。遂擒臘醅及婆諸刊,皆獻之遼。既已,復請之,遼人與之,并以前後所獻罪人歸之。歡都在破烏春等於斜堆,故石、拔石皆就擒。世祖自將與歡都合兵嶺東,諸軍皆至。是時,烏春已前死,窩謀罕請于遼,願和解。既與和,復來襲,乃進軍圍之。窩謀罕棄城遁去。破其城,盡俘獲之,以功
 差次分賜諸軍。城始破,議渠長生殺,眾皆長跪,遼使者在坐。忽一人佩長刀突前咫尺,謂世祖曰:「勿殺我!」遼使及左右皆走匿。世祖色不少動,執其人之手,語之曰:「吾不殺汝也。」於是罰左右匿者,曰:「汝等何敢失次耶?」罰既已,乃徐使執突前者殺之。其膽勇鎮物如此。



 師還,寢疾,遂篤。元娶拏懶氏哭不止,世祖曰:「汝勿哭,汝惟後我一歲耳。」肅宗請後事,曰:「汝惟後我三年。」肅宗出,謂人曰:「吾兄至此,亦不與我好言。」乃叩地而哭。俄呼穆宗謂曰:「烏雅束柔善,若辦集契丹事,阿骨打能之。」遼大安八年五月十五日卒。襲位十九年,年五十四。明年,拏懶氏卒。又
 明年,肅宗卒。肅宗病篤,歎曰:「我兄真多智哉!」世祖天性嚴重,有智識,一見必識,暫聞不忘。凝寒不縮慄,動止不回顧。每戰未嘗被甲,先以夢兆候其勝負。嘗乘醉騎驢入室中,明日見驢足跡,問而知之,自是不復飲酒。襲位之初,內外潰叛,締交為寇。世祖乃因敗為功,變弱為彊。既破桓赧、散達、烏春、窩謀罕,基業自此大矣。天會十五年,追謚聖肅皇帝,廟號世祖。皇統四年,號其藏曰永陵。五年,增謚世祖神武聖肅皇帝。



 母弟頗剌淑襲節度使,景祖第四子也,是為肅宗。遼重熙十一年壬午歲生。在父兄時號國相。國相之稱不知始何時。初,雅達為國相。
 雅達者,桓赧、散達之父也。景祖以幣馬求之於雅達,而命肅宗為之。肅宗自幼機敏善辯。當其兄時,身居國相,盡心匡輔。是時,叔父跋黑有異志,及桓赧、散達、烏春、窩謀罕、石顯父子、臘醅、麻產作難,用兵之際,肅宗屢當一面。尤能知遼人國政人情。凡有遼事,一切委之肅宗專心焉。凡白事於遼官,皆令遠跪陳辭,譯者傳致之,往往為譯者錯亂。肅宗欲得自前委曲言之,故先不以實告譯者。譯者惑之,不得已,引之前,使自言。乃以草木瓦石為籌,枚數其事而陳之。官吏聽者皆愕然,問其故,則為卑辭以對曰:「鄙陋無文,故如此。」官吏以為實然,不復疑
 之,是以所訴無不如意。



 桓赧、散達之戰,部人賽罕死之,其弟活羅陰懷忿怨。一日,忽以劍脊置肅宗項上曰:「吾兄為汝輩死矣!剄汝以償,則如之何?」久之,因其兄柩至,遂怒而攻習不出,習不出走避之。攻肅宗於家,矢注次室之裙,著于門扉。復攻歡都,歡都衷甲拒于室中,既不能入,持其門旃而去,往附盃乃。盃乃誘烏春兵度嶺,世祖與遇於蘇素海甸。世祖曰:「予昔有異夢,今不可親戰。若左軍中有力戰者,則大功成矣!」命肅宗及斜列、辭不失與之戰。肅宗下馬,名呼世祖,復自呼其名而言曰:「若天助我當為眾部長,則今日之事神祇監之。」語畢再拜。
 遂炷火束縕。頃之,大風自後起,火益熾。是時八月,并青草皆焚之,煙焰漲天。我軍隨煙衝擊,大敗之。遂獲盃乃,囚而獻諸遼。並獲活羅,肅宗釋其罪,左右任使之,後竟得其力焉。



 大安八年,自國相襲位。是時,麻產尚據直屋鎧水,繕完營堡,誘納亡命。招之,不聽,遣康宗伐之。太祖別軍取麻產家屬,錡釜無遺。既獲麻產,殺之,獻馘于遼。陶溫水民來附。二年癸酉,遣太祖以偏師伐泥厖古部帥水抹離海村跋黑、播立開,平之,自是寇賊皆息。三年八月,肅宗卒。天會十五年,追謚穆憲皇帝。皇統四年,藏號泰陵。五年,增謚肅宗明睿穆憲皇帝。



 母弟穆宗,諱盈
 歌,字烏魯完,景祖第五子也。南人稱揚割太師,又曰揚割追謚孝平皇帝,號穆宗,又曰揚割號仁祖。金代無號仁祖者,穆宗諱盈歌,謚孝平,「盈」近「揚」,「歌」近「割」,南北音訛。遼人呼節度使為太師,自景祖至太祖皆有是稱。凡《叢言》、《松漠記》、張棣《金志》等書皆無足取。穆宗,遼重熙二十二年癸巳歲生。肅宗時擒麻產,遼命穆宗為詳穩。大安十年甲戌,襲節度使,年四十二。以兄劾者子撒改為國相。



 三年丙子,唐括部跋葛勃堇與溫都部人跋忒有舊,跋葛以事往,跋忒殺跋葛。使太祖率師伐跋忒,跋忒亡去,追及,殺之星顯水,紇石烈部阿疏、毛睹祿阻兵為難,
 穆宗自將伐阿疏,撒改以偏師攻鈍恩城,拔之。阿疏初聞來伐,乃自訴於遼。遂留劾者守阿疏城,穆宗乃還。會陶溫水、徒籠古水紇石烈部阿閣版及石魯阻五國鷹路,執殺遼捕鷹使者。遼詔穆宗討之,阿閣版等據險立柵。方大寒,乃募善射者操勁弓利矢攻之。數日,入其城,出遼使存者數人,俾之歸。統門、渾蠢水之交烏古論部留可、詐都與蘇濱水烏古論敵庫德,起兵于米里迷石罕城,納根涅之子鈍恩亦亡去,於是兩黨作難。



 八月,撒改為都統,辭不失、阿里合懣、斡帶副之,以伐留可、詐都、塢塔等。謾都訶、石土門伐敵庫德。撒改欲先平邊地城
 堡,或欲先取留可,莫能決,乃命太祖往。鈍恩將援留可,乘謾都訶兵未集而攻之。石土門軍既與謾都訶會,迎擊鈍恩,大敗之,降米里迷石罕城,獲鈍恩、敵庫德,釋弗殺。太祖度盆搦嶺,與撒改會,攻破留可城,留可已先往遼矣,盡殺其城中渠長。還圍塢塔城,塢塔先已亡在外,城降於軍。詐都亦降於蒲家奴,於是撫寧諸路如舊時。太師因致穆宗,教統門、渾蠢、耶悔、星顯四路及嶺東諸部自今勿復稱都部長。命勝官、醜阿等撫定乙離骨嶺注阿門水之西諸部居民,又命斡帶及偏裨悉平二涅囊虎、二蠢出等路寇盜而還。



 七年庚辰,劾者尚守阿疏
 城,毛睹祿來降。阿疏猶在遼,遼使使來罷兵。未到,穆宗使烏林答石魯往佐劾者,戒之曰:「遼使來罷兵,但換我軍衣服旗幟與阿疏城中無辨,勿令遼使知之。」因戒劾者曰:「遼使可以計卻,勿聽其言遽罷兵也。」遼使果來罷兵,穆宗使蒲察部胡魯勃堇、邈遜孛堇與俱至阿疏城。劾者見遼使,詭謂胡魯、邈遜曰:「我部族自相攻擊,干汝等何事?誰識汝之太師?」乃援創刺殺胡魯、邈遜所乘馬。遼使驚駭遽走,不敢回顧,徑歸。居數日,破其城。狄故保還自遼。在城中,執而殺之。阿疏復訴於遼。遼遣奚節度使乙烈來。穆宗至來流水興和村,見乙烈。問阿疏城事,
 命穆宗曰:「凡攻城所獲,存者復與之,不存者備償。」且征馬數百匹。穆宗與僚佐謀曰:「若償阿疏,則諸部不復可號令任使也。」乃令主隈、禿答兩水之民陽為阻絕鷹路,復使鱉故德部節度使言于遼曰:「欲開鷹路,非生女直節度使不可。」遼不知其為穆宗謀也,信之,命穆宗討阻絕鷹路者,而阿疏城事遂止。穆宗聲言平鷹路,畋於土溫水而歸。是歲,留可來降。八年辛巳,遼使使持賜物來賞平鷹路之有功者。



 九年壬午,使蒲家奴以遼賜,給主隈、禿答之民,且修鷹路而歸。冬,蕭海里叛,入于係案女直阿典部,遣其族人斡達剌來給結和,曰:「願與太師為友,
 同往伐遼。」穆宗執斡達剌。會遼命穆宗捕討海里,穆宗送斡達剌于遼,募軍得甲千餘。女直甲兵之數,始見于此,蓋未嘗滿千也。軍次混同水,蕭海里再使人來,復執之。既而與海里遇。海里遙問曰:「我使者安在?」對曰:「與後人偕來。」海里不信。是時,遼追海里兵數千人,攻之不能克。穆宗謂遼將曰:「退爾軍,我當獨取海里。」遼將許之。太祖策馬突戰,流矢中海里首,海里墮馬下,執而殺之,大破其軍。使阿離合懣獻馘于遼。金人自此知遼兵之易與也。是役也,康宗最先登,於是以先登并有功者為前行,次以諸軍護俘獲歸所部。穆宗朝遼主于漁所,大被
 嘉賞,授以使相,錫予加等。



 十年癸未二月,穆宗還。遼使使授從破海里者官賞。高麗始來通好。十月二十九日,穆宗卒,年五十有一。初,諸部各有信牌,穆宗用太祖議,擅置牌號者置于法,自是號令乃一,民聽不疑矣。自景祖以來,兩世四主,志業相因,卒定離析,一切治以本部法令,東南至于乙離骨、曷懶、耶懶、土骨論,東北至于五國、主隈、禿答,金蓋盛于此。天會十五年,追謚孝平皇帝,廟號穆宗。皇統四年,號其藏曰獻陵。五年,增謚章順孝平皇帝。



 兄子康宗,諱烏雅束,字毛路完,世祖長子也。遼清寧七年辛丑歲生。乾統三年癸未,襲節度使,年四十
 三。穆宗末年,阿疏使達紀誘扇邊民,曷懶甸人執送之。穆宗使石適歡撫納曷懶甸,未行,穆宗卒,至是遣焉。先是,高麗通好,既而頗有隙,高麗使來請議事,使者至高麗,拒而不納。五水之民附于高麗,執團練使十四人。語在《高麗傳》中。二年甲申,高麗再來伐,石適歡再破之。高麗復請和,前所執團練十四人皆遣歸,石適歡撫定邊民而還。蘇濱水民不聽命,使斡帶至活羅海川,召諸官僚告諭之。含國部蘇濱水居斡豁勃堇不至。斡准部、職德部既至,復亡去。塢塔遇二部於馬紀嶺,執之而來,遂伐斡豁,克之。斡帶進至北琴海,攻拔泓忒城,乃還。四
 年丙戌,高麗遣黑歡方石來賀襲位,遣杯魯報之。高麗約還諸亡在彼者,乃使阿聒、勝昆往受之。高麗背約,殺二使,築九城於曷懶甸,以兵數萬來攻。斡賽敗之。斡魯亦築九城,與高麗九城相對。高麗復來攻,斡賽復敗之。高麗約以還逋逃之人,退九城之軍。復所侵故地。九月,乃罷兵。七年己丑,歲不登,減盜賊徵償,振貧乏者。十一年癸巳,康宗卒,年五十三。天會十五年,追謚恭簡皇帝。皇統四年,號其藏曰喬陵。五年,增謚康宗獻敏恭簡皇帝。



 贊曰:金之厥初,兄弟三人,亦微矣。熙宗追帝祖宗,定著
 始祖、景祖、世祖廟,世世不祧。始祖娶六十之婦而生二男一女,豈非天耶?景祖不受遼籍遼印,取雅達國相以與其子。世祖既破桓赧、散達,遼政日衰,而以太祖屬之穆宗,其思慮豈不深遠矣夫!



\end{pinyinscope}