\article{志第五}

\begin{pinyinscope}

 地
 理上



 ○上京路咸平路東京路北京路西京路中都路



 金之壤地封疆,東極吉里迷兀的改諸野人之境,北自蒲與路之北三千餘里,火魯火畽謀克地為邊,右旋入泰州婆盧火所浚界壕而西,經臨潢、金山,跨慶、桓、撫、昌、凈州之北,出天山外,包東勝,接西夏,逾黃河,復西歷葭州及米脂寨,出臨洮府,會州、積石之外,與生羌地相錯。復自積石諸山之南左折而東,逾洮州,越鹽川堡,循渭
 至大散關北,並山入京兆,絡商州,南以唐鄧西南皆四十里,取淮之中流為界,而與宋為表裏。



 襲遼制,建五京,置十四總管府,是為十九路。其間散府九,節鎮三十六,防禦郡二十二,刺使郡七十三,軍十有六,縣六百三十二。後復盡升軍為州,或升城堡寨鎮為縣,是以金之京府州凡百七十九,縣加於舊五十一,城寨堡關百二十二,鎮四百八十八。雖貞祐、興定危亡之所廢置,既歸大元,或有因之者,故凡可考必盡著之,其所不載則闕之。



 上京路,即海古之地,金之舊土也,國言「金」曰「按出虎」,按出虎水源於此,故名金源,建國之號蓋取諸此。國初
 稱為內地,天眷元年號上京。海陵貞元元年遷都于燕,削上京之號,止稱會寧府,稱為國中者以違制論。大定十三年七月,後為上京。其山有長白、青嶺、馬紀嶺、完都魯,水有按出虎水、混同江、來流河、宋瓦江、鴨子河。府一,領節鎮四,防禦一,縣六,鎮一。舊有會平州,天會二年築,契丹之周特城也,後廢。其宮室有乾元殿,天會三年建,天眷元年更名皇極殿。慶元宮,天會十三年建,殿曰辰居,門曰景暉,天眷二年安太祖以下御容,為原廟。朝殿,天眷元年建,殿曰敷德,門曰延光,寢殿曰宵衣,書殿曰稽古。又有時德宮、明德殿,熙宗嘗享太宗御容於此,太后所居也。涼殿,皇統二年構,門曰延福,樓曰五雲,殿曰重明。東廡南殿曰東華,次曰廣仁。西廡南殿曰西清,次曰明義。重明後,東殿曰龍壽,西殿曰奎文。時令殿及其曰奉元。有泰和殿,有武德殿,有薰風殿。其行宮有天開殿,爻剌春水之地
 也。有混同江行宮。太廟、社稷,皇統三年建,正隆二年毀。原廟,天眷元年以春亭名天元殿,安太祖、太宗、徽宗及諸後御容。春亭者,太祖所嘗御之所也。天眷二年作原廟,皇統七年改原廟乾文殿曰世德,正隆二年毀。大定五年復建太祖廟。興聖宮,德宗所居也,天德元年名之。興德宮,後更名永祚宮,睿宗所居也,光興宮,世宗所居也。正隆二年命吏部郎中蕭彥良盡毀宮殿、宗廟、諸大族邸第及儲慶寺,夷其趾,耕墾之。大定二十一年後修宮殿,建城隍廟。二十三年以甓束其城。有皇武殿,擊球校射之所也。有雲錦亭,有臨漪亭,為籠鷹之所,在按出虎水側。



 會寧府,下。初為會寧州,太宗以建都,升為府。天眷元年,置上京留守司,以留守帶本府尹,兼本路兵馬都總管。後置上京曷懶等路提刑司。戶三萬一千二百七十。舊歲貢秦王魚,大定十二年罷之,又貢豬二萬,二十五年罷之。東至胡里改六百三十里,西
 到肇州五百五十里,北至蒲與路七百里,東南至恤品路一千六百里,至曷賴路一千八百里縣三:



 會寧倚,與府同時置。有長白山、青嶺、馬紀嶺、勃野澱、綠野澱。有按出虎河,又書作阿術滸。有混同江、淶流河。有得勝陀,國言忽土皚葛蠻,太祖誓師之地也。



 曲江初名鎮東,大定七年置,十三年更今名。



 宜春大定七年置。有鴨子河。



 肇州,下,防禦使。舊出河店也。天會八年,以太祖兵勝遼,肇基王績於此,遂建為州。天眷元年十月,置防禦使,隸會寧府。海陵時,嘗為濟州支郡。承安三年,復以為太祖神武隆興之地,陞為節鎮,軍名武興。五年,置漕運司,以提舉兼州事。後廢軍。貞祐二年復升為武興軍節鎮,置招討司,以使兼州事。戶五千三百七十
 五。縣一:



 始興倚,興州同時置。有鴨子河、黑龍江。



 隆州,下,利涉軍節度使。古扶餘之地,遼太祖時,有黃龍見,遂名黃龍府。天眷三年,改為濟州,以太祖來攻城時大軍徑涉,不假舟楫之祥也,置利涉軍。天德三年置上京路都轉運司,四年,改為濟州轉運司。大定二十九年嫌與山東路濟州同,更今名。貞祐初,升為隆安府,戶一萬一百八十。縣一:



 利涉倚,興州同時置。有混同江、淶流河。鎮一興縣同時置,有混同館。



 信州,下,彰信軍刺史。本渤海懷遠軍,遼開泰七年建,取諸路漢民置。戶七千三百五十九。縣一:



 武昌
 本渤海懷福縣地。鎮一八十戶。



 蒲與路,國初置萬戶,海陵例罷萬戶,乃改置節度使。承安三年,設節度副使。南至上京六百七十里,東南至胡里改一千四百里,北至北邊界火魯火畽謀克三千里。



 合懶路,置總管府。貞元元年,改總管為尹,仍兼兵馬都總管。承安三年,設兵馬副總管。舊貢海葱,大定二十七年罷之。有移鹿古水。西至上京一千八百里,東南至高麗界五百里。



 恤品路,節度使。遼時,為率賓府,置刺史。本率賓故地,太宗天會二年,以耶懶路都孛堇所居地瘠,遂遷于此。以海陵例罷萬戶,置節度使,因名速頻路節度使。
 世宗大定十一年,以耶懶、速頻相去千里,既居速頻,然不可忘本,遂命名古土門親管猛安曰押懶猛安。承安三年,設節度副使。西北至上京一千五百七十里,東北至胡里改一千一百,西南至合懶一千二百,北至邊界斡可阿憐千戶二千里。「耶懶」又書作「押懶」。



 曷蘇館路,置節度使。天會七年,徙治寧州,嘗置都統司,明昌四年廢。有化成關,國言曰曷撒罕關。



 胡里改路,國初置萬戶,海陵例罷萬戶,乃改置節度使。承安三年,置節度副使。西至上京六百三十里,北至邊界合里賓忒千戶一千五百里。



 烏古迪烈統軍司,後升為招討司,與蒲與路近。



 咸平路,府一,領刺郡一,縣十。



 咸平府,下,總管府,安東軍節度使,本高麗銅山縣地,遼為咸州,國初為咸州路,置都統司。天德二年八月,升為咸平府,後為總管府。置遼東路轉運司、東京咸平路提刑司。戶五萬六千四百四。縣八:



 平郭倚,舊名咸平,大定七年更。



 銅山遼同州鎮安軍,本漢襄平縣,遼太祖時以東平寨置,因名東平,軍曰鎮東。章宗大定二十九年,以與東平重,故更。南有柴河,北有清河,西有遼河。



 新興遼銀州富國軍,本渤海富州,熙宗皇統三年廢州,更名來屬。有范河,北有柴河,西有遼河。



 慶雲遼祺州祐聖軍,本以所俘檀州密雲民建州密雲,後更名,有遼河。



 清安遼肅州信陵軍,熙宗皇統三年降為縣



 榮安東有遼河。



 歸仁遼舊隸通州安遠軍,本渤海強師縣,遼更名,金因之。北有細河。



 玉山章
 宗承安三年,以烏速集、平郭、林河之間相去六百餘里之地置,貞祐二年四月升為節,軍曰鎮安。



 韓州,下,刺史。遼置東平軍,本渤海鄚頡府。戶一萬五千四百一十二。舊有營。縣二:



 臨津倚,未詳何年置。



 柳河本渤海粵喜縣地,遼以河為名。有狗河、柳河。



 東京路,府一,領節鎮一,刺郡四,縣十七,鎮五。皇統四年二月,立東京新宮,寢殿曰保寧,宴殿曰嘉惠,前後正門曰天華、曰乾貞。七月,建宗廟,有孝寧宮。七年,建御容殿。



 遼陽府,中。東京留守司。本渤海遼陽故城,遼完葺之,郡名東平。天顯三年,升為南京,府曰遼陽。十三年,更為東京。太宗天會十年,改南京路平州軍帥司為東南路都統司之時,嘗治於此,以鎮高麗。後置兵馬都部署司,天德
 二年,改為本路都總管府,後更置留守司。產白兔、師姑布、鼠毫、白鼠皮、人參、白附子。戶四萬六百四。縣四、鎮一:



 遼陽倚。東梁河、國名兀魯忽必剌,俗名太子河。



 鶴野鎮一長宜,曷蘇館在其地。



 宜豐遼舊衍州安廣軍,皇統三年廢為縣,有東梁河。



 石城興定三年九月,以縣之靈巖寺為巖州,名其倚郭縣曰東安,置行省。



 澄州,南海軍刺史,下。本遼海州,天德三年改州名。戶一萬一千九百三十五。縣二。鎮一:



 臨溟鎮一新昌。



 析木遼銅州廣利軍附郭析木縣地,皇統三年廢州來屬。有沙河。



 沈州,昭德軍刺史,中。本遼定理府地,遼太宗時軍曰興遼,後為昭德軍,置節度。明昌四年改為刺史,與通、貴德、澄三州皆隸東京。戶三萬六千八百九十二。
 縣五:



 樂郊遼太祖俘三河之民建三河縣於此,後改更今名。有渾河。



 章義遼舊廣州,皇統三年降為縣來屬。有遼河、東梁河、遼河大口。



 遼濱遼舊遼州東平軍,遼太宗改為始平軍,皇統三年廢為縣。有遼河。



 邑樓遼舊興州人中軍常安縣,遼嘗置定理府刺史於此,本邑樓故地,大定二十九年章宗更名。有范河、清河,國名叩隈必剌。



 雙城遼雙州保安軍也,皇統三年降為縣,章宗時廢。



 貴德州,刺史,下。遼貴德州寧遠軍,國初廢軍,降為刺郡。戶二萬八百九十六。縣二:



 貴德倚。有範河。



 奉集遼集州懷遠軍奉集縣,本渤海舊縣。有渾河。



 蓋州,奉國軍節度使,下。本高麗蓋葛牟城,遼辰州。明昌四年,罷曷蘇館,建辰州遼海軍節度使。六年,以與「
 陳」同音,更取蓋葛牟為名。戶一萬八千四百五十六。縣四、鎮二:



 湯池遼鐵州建武軍湯池縣。鎮一神鄉。



 建安遼縣。鎮一大寧。



 秀巖本大寧鎮,明昌四年升。泰和四年廢為鎮,貞祐四年復升置。



 熊岳遼盧州玄德軍熊岳縣。遼屬南女直湯河司。



 復州,下,刺史。遼懷還軍節度,明昌四年降為刺史。舊貢鹿筋,大定八年罷之。戶一萬三千九百五十。縣二、鎮一:



 永康倚。舊名永寧,大定七年更。



 化成遼蘇州安復軍,本高麗地,興宗置。皇統三年降為縣來屬。貞祐四年五月升為金州,興定二年陞為防禦。鎮一歸勝。



 來遠州,下。舊來遠城,本遼熟女直地,大定二十二年升為軍,後升為州。



 婆速府路,國初置統軍司,天德二處置總管府,貞元元年與曷懶路總管並為尹,兼本路兵馬都總管。此路皆猛安戶。



 北京路,府四,領節鎮七,刺郡三,縣四十二,鎮七,寨一。



 大定府,中,北京留守司,遼中京。統和二十五年建為中京,國初因稱之。海陵貞元元年更為北京,置留守司、都轉運司、警巡院。產紘鼠、螺杯、茱萸梳、玳瑁鞍、酥乳餅、五味子。戶六萬四千四十七。縣十一、鎮二:



 大定倚,遼縣舊名。在土河、七金山、陰涼河。鎮一恩化。



 長興有塗河。



 富庶有心河。鎮一文安。



 松山遼松
 山州勝安軍松山縣,開泰中置,舊置刺史。太祖天輔七年置觀察使。皇統三年廢州來屬。承安三年隸高州,泰和四年後復。有陰涼河、落馬河。



 神山遼澤州神山縣,遼太祖俘蔚州之民置。章宗承安二年嘗置惠州,升孩兒館為灤陽縣,以隸之。泰和四年罷州及灤陽縣。



 惠和皇統三年以遼惠州惠和縣置。



 金源唐青山縣,遼開泰二年置,以地有金甸為名。有駱駝山。



 和眾遼榆州和眾縣,皇統三年罷州來屬。



 武平遼築城杏堝,初名新州,統和間更為武安州。皇統三年降為武安縣來屬,大定七年更名。承安三年隸高州,泰和四年復來屬。



 靜封承安二年以胡設務置,隸全州,三年隸高州,泰和四年來屬。



 三韓遼伐高麗,遷馬韓、辰韓、弁韓三國民為縣,置高州。太祖天輔七年以高州置節度使,皇統三年廢為縣,承安三年復陞為高州,置刺史,為全州支郡,分武平、松山、靜封三縣隸焉。泰和四年廢。有落馬河,塗河。



 利州,下,刺史。遼統和十六年置。戶二萬一千二百九
 十六。縣二、鎮一、寨一:



 阜俗遼統和四年置,金因之。



 龍山遼故潭州廣潤軍縣故名,熙宗皇統三年廢州來屬。有榆河。寨一蘭州。鎮一漆河。



 義州,下,崇義軍節使。遼宜州,天德三年更州名。戶三萬二百三十三。縣三、鎮一:



 弘政有凌河。



 開義遼海北州廣化軍縣故名,熙宗皇統三年廢州來屬。鎮一饒慶。



 同昌遼成州興府軍縣故名,國初隸川州,大定六年罷川州,隸懿州,承安二年復隸川州,泰和四年來屬。



 錦州,下,臨海軍節度使。舊隸興中府,後來屬。戶三萬九千一百二十三。縣三:



 永樂本慕容皝之西樂縣地。



 安昌



 神水遼開泰二年置,皇統三年廢為鎮,大定二十九年復升為縣。有土河。



 瑞州,下,歸德軍節使。本來州,天德三年更為宗州,
 泰和六年以避睿宗諱,謂本唐瑞州地,故更今名。戶一萬九千九百五十三。縣三、鎮一:



 瑞安舊名來賓,唐來遠縣民。明昌六年更為宗安,泰和六年復更今名。



 海陽遼潤州海陽軍故縣,皇統三年廢州來屬。鎮一遷民。



 海濱本慕容皝集寧縣地,遼隰州海平軍故縣,皇統三年廢州來屬。



 廣寧府,散,下,鎮寧軍節度使。本遼顯州奉先軍,漢望平縣地,天輔七年升為府,因軍名置節度。天會八年改軍名鎮寧。天德二年隸咸平,後廢軍隸東京。泰和元年七月來屬。戶四萬三千一百六十一。縣三、舊有奉玄縣,天會八年改為鐘秀縣。鎮六、寨四。鎮二歡城、遼西。



 廣寧舊名山東縣,大定二十九年更名。有遼世宗顯陵。寨二閭城、兔兒窩。



 望平大定二十九年升梁漁務置。鎮
 二梁漁務、山西店。



 閭陽遼乾州廣德軍,以乾陵故名奉陵縣。天會八年廢州更名來屬。有凌河。有遼景宗乾陵。鎮二閭陽、衡家。寨二大斧山、北川。



 懿州,下,寧昌軍節度使。遼嘗更軍名慶懿,又為廣順,復更今名。金因之,先隸咸平府,泰和末來屬。戶四萬二千三百五十一。縣二:大定六年罷川州,以宜民、同昌二縣來屬。承安二年復以二縣隸川州。泰和四年罷川州,以宜民隸興中,同昌隸義州。



 順安



 靈山本渤海靈峰縣地。



 興中府,散,下。本唐營州城,遼太祖遷漢民以實之,曰霸州彰武軍,重熙十一年升為府,更今名,金因之。戶四萬九百二十七。縣四、鎮三:



 興中本漢柳城地。
 南有凌河。鎮一黔城。



 永德遼安德州化平軍安德縣,世宗大定七年更今名。北有凌河。鎮一阜安。



 興城遼嚴州保肅軍縣故名,皇統三年廢州隸錦州。有桃花島。



 宜民遼川州長寧軍,會同中嘗名白川州,天祿五年去「白」字,國初因之,與同昌縣皆隸焉。大定六年降為宜民縣,隸懿州。承安二年復置川州,改徽川寨為徽川縣,為懿州支郡。泰和四年罷州及徽川縣求屬。鎮一咸康,遼縣也,國初廢為鎮。



 建州,下,保靖軍刺史。遼初名軍曰武寧,後更,金因之。戶一萬一千四百三十九。縣一:



 永霸本唐昌黎縣地。



 全州,下,盤安軍節度使。承安二年置,改胡設務為靜封縣,黑河鋪為盧川縣,撥北京三韓縣烈虎等五猛安以隸焉。貞祐二年四月嘗僑置於平州。戶九千三
 百一十九。縣一:



 安豐承安元年十月改豐州鋪為安豐縣,隸臨潢府,二年置全州盤安軍節度使治。有黃河、黑河。



 臨潢府,下,總管府。地名西樓,遼為上京。國初因稱之,天眷元年改為北京。天德二年改北京為臨潢府路,以北京路都轉運司為臨潢府路轉運司,天德三年罷。貞元元年以大定府為北京後,但置北京臨潢路提刑司。大定後罷路,併入大定府路。貞祐二年四月嘗僑置于平州。有天平山、好水川,行宮地也,大定二十五年命名。有撒里乃地。熙宗皇帝統九年嘗避暑于此。有陷泉,國言曰落孛魯。有合裊追古思阿不漠合沙地。戶六萬七千九百七。縣五、堡三十七:大定間二十四,後增。



 臨潢倚。有金粟河。



 長
 泰有立列只山,其北千餘里有龍駒河,國言曰喝必剌。有撒里葛睹地。



 盧川承安二年以黑河鋪升,隸全州,後復來屬。有潢河。



 寧塞泰和元年五月置。有滑河。



 長寧遼永州永昌軍縣故名,太祖天輔七年嘗置節度使,皇統三年廢州來屬。



 慶州,下,玄寧軍刺史。境內有遼祖州,天會八年改為奉州,皇統三年廢,遼太祖祖陵在焉。境內有遼懷州,舊置奉陵軍,天會八年更為奉德軍,皇統三年廢,遼太宗、穆宗懷陵在焉。北山有這聖宗、興宗、道宗慶陵。城中有遼行宮,比他州為富庶,遼時刺此郡者非耶律、蕭氏不與,遼國寶貨多聚藏于此。北至界二十里,南至盧川二百二十,西至桓州九百,東至臨潢一百六十。戶
 二千七。縣一:舊有孝安縣,天會八年改為慶民縣,皇統三年廢。



 朔平有榷場務。



 興州,寧朔軍節度使。本遼北安州興化軍,皇統三年降軍置興化縣,承安五年陞為興州,置節度,軍名寧朔,改利民寨為利民縣,撥梅堅河徒門必罕、寧江、速馬剌三猛安隸焉。貞祐二年四月僑置于密雲縣。戶一萬五千九百七十。縣二:又有利民縣,承安五年以利民寨升,泰和四年廢。



 興化倚。遼舊縣,皇統三年降興化軍置,隸大定府,承安五千建興州於縣,為倚郭。舊有白檀鎮。



 宜興本興化縣白檀鎮,泰和三年陞為縣來屬。



 泰州,德昌軍節度使。遼時本契丹二十部族牧地,海陵正隆間,置德昌軍,隸上京,大定二十五年罷之。承
 安三年復置于長春縣,以舊泰州為金安縣,隸焉。北至邊四百里,南至懿州八百里,不至肇州三百五十里。戶三千五百四。縣一、舊有金安縣,承安三年置,尋廢。堡十九:



 長春遼長春州韶陽軍,天德二年降為縣,隸肇州,承安三年來屬。有撻魯古河、鴨子河。有別里不泉。



 邊堡,大定二十一年三月,世宗以東北如討司十九堡在泰州之境,及臨潢路舊設二十四堡障參差不齊,遣大理司直蒲察張家奴等往視其處置。於是東北自達里帶石堡子至鶴五河地分,臨潢路自鶴五河堡子至撒里乃,皆取直列置堡戍。評事移剌每言:「東北及臨潢所置,土脊樵絕,當令
 所徙之民姑逐水草以居,分遣丁壯營畢,開壕塹以備邊。」上令無水草地官為建屋,及臨潢路諸堡皆以放良人戍守。省議:「臨潢路二十四堡,堡置戶三十,共為七百二十,若營建畢,官給一歲之食。」上以年饑權寢,姑令開壕為備。四月,遣吏部郎中奚胡失海經畫壕塹,旋為沙雪堙塞,不足為禦。乃言:「可築二百五十堡,堡日用工三百,計一月可畢,糧亦足備,可為邊防久計,泰州九堡、臨潢五堡之地斥鹵,官可為屋外,自撒里乃以西十九堡,舊戍軍舍少,可令大鹽濼官木三萬餘,與直東堡近嶺求
 木,每家官為構室一椽以處之。」



 西京路,府二,領節鎮七,刺郡八,縣三十九,鎮九。大定五年建宮室,名其殿曰保安,其門南曰奉天,東曰宣仁,西曰阜成。天會三年建太祖原廟。



 大同府,中,西京留守司。晉雲州大同軍節度,遼重熙十三年,升為西京,府名大同,金因之。皇統元年,以燕京路隸尚書省,西京及山後諸部族隸元帥府。舊置兵馬都部署司,天德二年,改置本路都總管府,後更置留守司。置轉運司及中都西京路提刑司。貢瑪瑙環子、瑪瑙數珠。產白駝、安息香、松明、松脂、黃連、百藥煎、芥子煎、鹽、撈鹽、石綠、綠礬、鐵、甘草、枸杞、碾玉砂、地蕈。戶九萬八千四百四十四。縣七、鎮三:



 大同倚。遼
 析雲中置,金因之。有牛皮關、武周山、方山、奚望山、盛樂城、御河、鬥雞臺、平城外郭鹽場、如渾水、桑乾河、紇真山。有遼帝后像。在華嚴寺。鎮一奉義。



 雲中晉舊縣名。



 宣寧遼德州昭聖軍宣德縣,大定八年更名。有官山、彌陀山、石綠山、產碾玉砂。鎮一窟龍城。



 懷安晉故縣名。



 天成遼析雲中置。



 白登本名長清,大定七年更。有白登臺、採掠山。



 懷仁遼析雲中置,貞祐二年五月升為雲州。有黃花嶺、錦屏山、清涼山、金龍山、早起城、日中城。鎮一安七畽。



 豐州,下,天德軍節度使。遼嘗更軍名應天,尋復,金因之。皇統九年升為天德總管府,置西南路招討司,以天德尹兼領之。大定元年降為天德軍節度使,兼豐州管內觀察使,以元管部族直撒、軍馬公事,並隸西南路詔討司產不灰木、地蕈。戶二萬二千六百八十三。縣
 一、鎮一:



 富民晉舊名。有黑山、神山。鎮一振武。



 弘州,下,刺史。遼名軍曰博寧,本襄陰村,統和中建。國初置保寧軍,後廢軍。產瑪瑙。戶二萬二千二。縣二、鎮二:



 襄陰倚。本名永寧,大定七年改。



 順聖本安塞軍故地,遼應歷中置,金因之。鎮二陽門,貞祐二年七月升為縣。大羅。



 凈州,下,刺史。大定十八年以天山縣升,為豐州支郡,刺史兼權譏察。北至界八十里戶五千九百三十八。縣一:



 天山舊為榷場,大定十八年置,為倚郭。



 桓州,下,威遠軍節度使。軍兵隸西北路招討司。明昌七年改置刺史。北至舊界一里半。戶五百七十八。
 縣一:曷里滸東川,更名金蓮川,世宗曰:「蓮者連也,取其金枝玉葉相連之義。」景明宮,避暑宮也,在涼陘,有殿、揚武殿,皆大定二十年命名。有查沙,有白濼,國言曰勺赤勒。



 清塞倚。明昌四年以罷錄事司置。



 撫州,下,鎮寧軍節度使。遼秦國大長公主建為州,章宗昌三年復置刺史,為桓州支郡,治柔遠。明昌四年置司候司。承安二年升為節鎮,軍名鎮寧,撥西北路招討司所管梅堅必剌、王敦必剌、拿憐術花速、宋葛斜忒渾四猛安以隸之。戶一萬一千三百八十。縣四:有旺國崖,大定八年五月更名靜寧山。有麻達葛山,大定二十九年更名胡土白山。有冰井。



 柔遠倚。大定十年置于燕子城,隸宣德州,明昌三年來屬,有燕子城,國言曰吉甫魯灣城,北羊城,國
 言曰火唵榷場,查剌嶺,沔山,大漁濼,行宮有樞光殿。有雙山,七里河,石井,蝦蟆山,昂吉濼又名鴛鴦濼,得勝口舊名北望澱,大定二十年更。



 集寧明昌三年以春市場置,北至界二百七十里。



 豐利明昌四年以泥濼置。有蓋里泊。



 威寧承安二年以撫州新城鎮置。



 德興府,晉新州,遼奉聖州武定軍節度,國初因之。大安元年陞為府,名德興。戶八萬八百六十八。縣六、有漫天堝,泰和二年更名拂雲,平惡崖,更名壘翠巖。鎮一:



 德興倚。舊名永興縣,大安元年更名。有涿鹿定、東水鎮。有雞鳴山。



 媯川遼可汗州清平軍,本晉媯州,會同元年遼太祖嘗名可汗州,縣舊曰懷戎,更名懷來,明昌六年更今名。西北有合河龜館石橋,明昌四年建。



 縉山遼儒州縉陽軍縣故名,皇統元年廢州來屬,崇慶元年升為鎮州。鎮一永安



 望雲本望雲川地,遼帝嘗居,號曰御莊,後更為縣,金因之。



 礬山晉故縣,國初隸弘州,明昌三年來屬。



 龍門晉縣,
 國初隸弘州,後來屬。明昌三年割隸宣德州。有慶寧宮,行宮也,泰和五年以提舉兼龍門令。



 昌州,天輔七年降為建昌縣,隸桓州。明昌七年以狗濼復置,隸撫州,後來屬。戶一千二百四十一。縣一:



 寶山有狗濼,國言曰押恩尼要。其北五百餘里有日月山,大定二十年更曰抹白山。國言涅里塞一山。



 宣德州,下,刺史。遼改晉武州為歸化州雄川武軍,大定七年更為宣化州,八年復更為宣德。戶三萬二千一百四十七。縣二:



 宣德舊文德縣,大定二十九年更名。



 宣平承安二年以大新鎮置,以北邊用兵嘗駐此地也。



 朔州,中,順義軍節度使。貞祐三年七月,嘗割朔州廣
 武縣隸代州。產鐵、荊三棱、枸杞。戶四萬四千八百九十二。縣二:



 鄯陽晉故縣。有桑乾河、大和嶺、天池、鴈門關、霸德山。



 馬邑晉故縣,貞祐二年五月升為固州。有洪濤山、鑊水,又曰桑乾河。



 武州,邊,下,刺史。大定前仍置宣威軍。戶一萬三千八百五十一。縣一:



 寧遠晉故縣。黃河。



 應州,下,彰國軍節度使。戶三萬二千九百七十七。縣三:



 金城晉故縣。有黃瓜堆、復宿山、桑乾河、渾河、崞川水、黃花城。



 山陰本名河陰,大定七年以興鄭州屬縣同,故更焉。貞祐二年五月升忠州。有黃花嶺、桑乾河。



 渾源晉縣,貞祐二年五月升為渾源州。產鹽。



 蔚州,下,忠順軍節度使。遼嘗更為武安軍,尋復。貢地
 蕈。戶五萬六千六百七十四。縣五:



 靈仙北有桑乾河、代王城、薄家村。



 廣靈亦作「陵」,遼統和三年析靈仙置。



 靈丘晉縣,貞祐二年四月升為成州,四年割為代州支郡。



 定安晉縣。有桑乾河。貞祐二年四月升為定安州。



 飛狐。晉縣。



 雲內州,下,開遠軍節度使。天會七年徙奚第一、第三部來戍。產青鑌鐵。戶二萬四千八百六十八。縣二、鎮一:



 柔服夾山在城北六十里。鎮一:寧仁,舊縣也,大定後廢為鎮。



 雲川本曷董館,後升裕民縣,皇統元年復廢為曷董館,大定二十九年復升,更為今名。



 寧邊州,下,刺史。國初置鎮西軍,貞祐三年隸嵐州,四年二月升為防禦。戶六千七十二。縣一:



 寧邊正隆三年置。



 東勝州,下,邊,刺史。國初置武興軍,有古東勝城。戶三千五百三十一。縣一、鎮一:



 東勝鎮一寧化。



 部族節度使:



 烏昆神魯部族節度使,軍兵事屬西北路招討司,明昌三年罷節度使,以招討司兼領。



 烏古里部族節度使。



 石壘部族節度使。



 助魯部族節度使。



 孛特本部族節度使。



 計魯部族節度使。



 唐古部族,承安三年改為部羅火扎石合節度使。



 迪烈又作迭剌。女古部族,承安三年改為土魯渾扎石合節度使。



 詳穩九處:



 咩颭詳穩,貞祐四年六月改為葛也阿鄰猛安。



 木典颭詳穩,貞祐四年改為抗葛阿鄰謀克。



 骨典颭穩,貞祐四年改為撒合輦必剌謀克。



 唐古颭詳穩。



 耶剌都颭詳穩。



 移典颭穩。



 蘇木典颭詳穩,近北京。



 胡都颭詳穩。



 霞馬颭詳穩。



 群牧十二處:



 斡獨椀群牧,大定四年改為斡睹只群牧。



 蒲速斡群牧。本斡睹只地,大定七年分置。



 耶魯椀群牧。



 訛里都群牧。



 颭斡群牧。



 歐里本群牧。



 烏展群牧。



 特滿群牧。



 駝駝都群牧。



 訛魯都群牧。



 忒恩群牧。承安四年創置。



 蒲鮮群牧。承安四年創置。



 中都路,遼會同元年為南京,開泰元年號燕京。海陵貞元元年定都,以燕乃列國之名,不當為京師號,遂改為中都。府一,領節鎮三,刺郡九,縣四十九。天德三年,始圖上燕城宮室制度,三月,命張浩等增廣燕城。城門十三,東曰施仁、曰宣曜、曰陽春,南曰影風、曰豐宜、曰端禮,西曰麗
 澤、曰顥華、曰彰義,北曰會城、曰通玄、曰崇智、曰光泰。浩等取真定府潭園材木,營建宮室及涼位十六。應天門十一楹,左右有樓,門內有左、右翔龍門,及日華、月華門,前殿曰大安,左、右掖門,內殿東廊曰敷得門。大安殿之東北為東宮,正北列三門,中曰粹英,為壽康宮,母后所居也,西曰會通門,門北曰承明門,又北曰昭慶門。東曰集禧門,尚書省在其外,其東西門左、右嘉會門也,門有二樓,大安殿後門之後也。其北曰宣明門,則常朝後殿門也。北曰仁政門,傍為朵殿,朵殿上為兩高樓,曰東、西上閣門,內有仁政殿,常朝之所也。宮城之前廊,東西各二百餘間,分為三節,節為一門。將至宮城,東西轉各有廊百許間,馳道兩傍植柳,廊脊覆碧瓦,宮闕殿門則純用碧瓦。應天門舊名通天門,大定五年更。七年改福壽殿曰壽安宮。明昌五年復以隆慶宮為東宮,慈訓殿為承華殿,承華殿者,皇太子所居之東宮也。泰和殿,泰和二年更名慶寧殿。又有崇慶殿。魚藻池、瑤池殿位,貞元元年建。有神龍殿,又有觀會亭。又有安仁殿、隆德殿、臨芳殿。皇統元年有元和殿。有常武殿,有廣武殿。為擊球、習射之所。京城北離宮有太寧宮,大定十九年建,後更為壽寧,又更為壽安,明昌二年更為萬寧
 宮。瓊林苑有橫翠殿。寧德宮西園有瑤光臺,又有瓊華島,又有瑤光樓。皇統元年有宣和門。正隆三年有宣華門,又有撒合門。



 大興府,上。晉幽州,遼會同元年陞為南京,府曰幽都,仍號盧龍軍,開泰元年更為永安析津府。天會七年析河北為東、西路時屬河北東路,貞元元年更今名。戶二十二萬五千五百九十二。大定四年十月,命都門外夾道重行植柳各百里。產金銀銅鐵。藥產滑石、半夏、蒼術、代赭石、白龍骨、薄荷、五味子、白牽牛。縣十、鎮一:



 大興倚。遼名析津,貞元二年更今名。有建春宮。鎮一廣陽。



 宛平倚。本晉幽都縣,遼開泰元提更今名。有玉泉山行宮。



 安次晉舊名。



 漷陰遼太平中,以漷陰村置。



 永清晉舊名。



 寶坻本新倉鎮,大定十二年置,以香河縣近民附之。承安三年置盈州,為大興府支郡,以香河、武
 清隸焉。尋廢州。



 香河遼以武清縣之孫村置。



 昌平有居庸關,國名查剌合攀。



 武清晉縣。



 良鄉有料石岡、閻溝。



 通州,下,刺史。天德三年升潞縣置,以三河隸焉。興定二年五月陞為防禦。戶三萬五千九十九。縣二:



 潞晉縣名。有潞水。



 三河晉縣名。



 薊州,中,刺史。遼置上武軍。戶六萬九千一十五。產粟。縣五、舊又有永濟縣,大定二十七年以永濟務置,未詳何年廢。又有黎豁縣,廢置皆未詳。鎮二:



 漁陽倚。



 遵化遼景州清安軍。鎮一石門。



 豐潤泰和間置。



 玉田有行宮、偏林,大定二十年改為御林。鎮一韓城。



 平峪大定二十七年,以漁陽縣大王鎮升。



 易州,下,刺史。遼置高陽軍。戶四萬一千五右七十七。
 縣二:



 易有易水。



 淶水有淶水。



 涿州,中,刺史。遼為永泰軍。貢羅。戶一十一萬四千九百一十二。縣五、鎮一:



 范陽倚。晉縣。有湖梁河。有劉李河。鎮一政滿。



 固安晉縣。



 新城



 定興大定六年以范陽縣黃村置,割淶水、易縣近民屬之。有巨馬河。



 奉先大定二十九年置萬寧縣以奉山陵,明昌二年更今名。有房山、龍泉河、盤寧宮。



 順州,下,刺史。遼置歸化軍。戶三萬二千四百三十三。縣二:



 溫陽舊名懷柔,明昌六年更。有螺山、漵水、兔耳山。



 密雲遼檀州武威軍。有古北山,國言曰留斡嶺。



 平州,中,興平節度使。遼為遼興軍。天輔七年以燕西地與宋,遂以平州為南京,以錢帛司為三司,天會
 四年復平州,嘗置軍帥司。天會十年徙軍帥司治遼陽府,後置轉運司。貞元元年以轉運司併隸中都路。貞祐二年四月置東面經略司,八月罷。貢櫻桃、綾。戶四萬一千七百四十八。縣五、鎮一:



 盧龍倚。



 撫寧本新安鎮,大定二十九年置。



 海山本漢海陽故城,遼以所俘望都縣民置,故名望都,大定七年更名。



 遷安本漢令支縣故城,遼以所俘安喜縣民置,因名安喜,大定七年更今名。鎮一建昌。



 昌黎遼營州鄰海軍,以所俘定州民置廣寧縣。皇統二年降州來屬,大定二十九年以與廣寧府重,故更今名。



 灤州,中,刺史。本黃落故城,遼為永安軍,天輔七年因置節度使。戶六萬九千八百六。縣四、有松亭關,國名斜烈只。鎮
 二:



 義豐倚。



 石城有長春行宮。長春澱舊名大定澱,大定二十年更。鎮一榛子。



 馬城



 樂亭鎮一新橋。



 雄州,中。宋名易陽郡。天會七年置永定軍節度使。隸河北東路,貞元二年來屬。戶二萬四百一十一。縣三:



 歸信倚。有易水、巨馬河。



 容城泰和八年割隸安州,貞祐二年隸安肅州。有南易水、大泥澱、渾泥城。



 保定宋保定軍,後廢為縣。



 霸州,下,刺史。遼益津郡。隸河北東路,貞元二年來屬。戶四萬一千二百七十六。縣四:



 益津倚。大定二十九年創置,倚郭。



 文安



 大城



 保安國初因宋為信安軍,大定七年降為信安縣,隸霸州。元光元年四月升為鎮安府。所以重高陽公張甫也。



 保州,中,順天軍節度使。宋舊軍事,天會七年置順天軍節度使,隸河北東路,貞元二年來屬。海陵賜名清苑郡。戶九萬三千二十一。縣二:



 清苑倚。宋名保塞,大定十六年更。有抱陽山、沉水、饋軍河。



 滿城大定三十八年以清苑縣塔院村置。



 安州,下,刺史。宋順安軍治高陽,天會七年升為安州,隸河北東路,後置高陽軍。大定二十八年徙治葛城,因升葛城為縣,用倚郭。泰和四年改混泥城為渥城縣,來屬,八年移州治於渥城,以葛城為屬縣。戶三萬五百三十二。縣三:



 渥城倚。泰和四年置。



 葛城大定二十八年置。



 高陽泰和八年正月改隸莫州,四月復。有徐河、百濟河。



 遂州,下,刺史。宋廣信軍,天會七年改為遂州,隸河北東路,貞元二年來隸,號龍山郡。泰和四年廢為遂城縣,隸保州,貞祐二年復置州。戶一萬一千一百七十四。縣一:



 遂城倚。有光春宮行宮。有遂城山、易水、漕水、鮑河。



 安肅州,下,刺史。宋安肅軍,天會七年升為徐州,軍如舊,隸河北東路,貞元二年來屬。天德三年改為安肅州,軍名徐郡軍。大定後降為刺郡,廢軍。戶一萬二千九百八十。縣一:



 安肅按《金初州郡志》,雄、霸、保、安、遂、安肅六州皆隸廣寧府。《太宗紀》載天會七年分河北為東、西路,則隸河北東路,豈以平州為南京之後,以六州隸廣寧也?不然則郡
 志誤。



\end{pinyinscope}