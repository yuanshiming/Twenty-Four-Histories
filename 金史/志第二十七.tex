\article{志第二十七}

\begin{pinyinscope}

 食貨一



 國之有食貨,猶人之有飲食也。人非飲食不生,國非食貨不立。然燧人、庖犧能為飲食之道以教人,而不能使人無飲食之疾。三王能為食貨之政以遺後世,而不能使後世無食貨之弊。唯善養生者如不欲食啖,而飲食自不闕焉,故能適飢飽之宜,可以疾少而長壽。善裕國者初不事貨殖,而食貨自不乏焉,故能制豐約之節,可
 以弊少而長治。



 金於食貨,其立法也周,其取民也審。太祖肇造,減遼租稅,規模遠矣。熙宗、海陵之世,風氣日開,兼務遠略,君臣講求財用之制,切切然以是為先務。雖以世宗之賢,儲積之志曷嘗一日而忘之。章宗彌文煟興,邊費亦廣,食貨之議不容不急。宣宗南遷,國土日蹙,污池數罟,往往而然。考其立國以來,所謂食貨之法,犖犖大者曰租稅、銅錢、交鈔三者而已。三者之法數變而數窮。官田曰租,私田曰稅。租稅之外算其田園屋舍車馬牛羊樹藝之數,及其藏鏹多寡,徵錢曰物力。物力之徵,上自公卿大夫,下逮民庶,無茍免者。近臣出使外國,
 歸必增物力錢,以其受饋遺也。猛安謀克戶又有所謂牛頭稅者,宰臣有納此稅,庭陛間諮及其增減,則州縣徵求於小民蓋可知矣。故物力之外又有鋪馬、軍須、輸庸、司吏、河夫、桑皮故紙等錢,名目瑣細,不可殫述。其為戶有數等,有課役戶、不課役戶、本戶、雜戶、正戶、監戶、官戶、奴婢戶、二稅戶。有司始以三年一籍,後變為通檢,又為推排。凡戶隸州縣者,與隸猛安謀克,其輸納高下又各不同。法之初行,唯恐不密,言事者謂其厲民,即命罷之。罷之未久,會計者告用乏,又即舉行。其罷也志以便民,而民未見德。其行也志以足用,而用不加饒。一時君
 臣節用之言不絕告誡。嘗自計其國用,數亦浩瀚,若足支歷年者,郡縣稍遇歲侵,又遽不足,竟莫詰其故焉。



 至於銅錢、交鈔之弊,蓋有甚者。初用遼、宋舊錢,雖劉豫所鑄,豫廢,亦兼用之。正隆而降,始議鼓鑄,民間銅禁甚至,銅不給用,漸興窯冶。凡產銅地脈,遺吏境內訪察無遣,且及外界,而民用銅器不可闕者,皆造於官而鬻之。既而官不勝煩,民不勝病,乃聽民冶銅造器,而官為立價以售,此銅法之變也。若錢法之變,則鼓鑄未廣,斂散無方,已見壅滯。初恐官庫多積,錢不及民,立法廣布。繼恐民多匿錢,乃設存留之限,開告訐之路,犯者繩以重罰,
 卒莫能禁。州縣錢艱,民間自鑄,私錢苦惡特甚。乃以官錢五百易其一千,其策愈下。及改鑄大錢,所準加重,百計流通,卒莫獲效。濟以鐵錢,鐵不可用,權以交鈔,錢重鈔輕,相去懸絕,物價騰踴,鈔至不行。權以銀貨,銀弊又滋,救亦無策,遂罷銅錢,專用交鈔、銀貨。然而二者之弊乃甚於錢,在官利於用大鈔,而大鈔出多,民益見輕。在私利於得小鈔,而小鈔入多,國亦無補。於是,禁官不得用大鈔,已而恐民用銀而不用鈔,則又責民以鈔納官,以示必用。先造二十貫至百貫例,後造二百貫至千貫例,先後輕重不倫,民益眩惑。及不得已,則限以年數,限
 以地方,公私受納限以分數,由是民疑日深。其間,易交鈔為寶券,寶券未久更作通寶,準銀并用。通寶未久復作寶泉,寶泉未久織綾印鈔,名曰珍貨。珍貨未久復作寶會,汔無定制,而金祚訖矣。



 歷觀自古財聚民散,以至亡國,若鹿臺、鉅橋之類,不足論也。其國亡財匱,比比有之,而國用之屈,未有若金季之甚者。金之為政,常有恤民之志,而不能已苛征之令,徒有聚斂之名,而不能致富國之實。及其亡也,括粟、闌糴,一切掊克之政靡不為之。加賦數倍,豫借數年,或欲得鈔則豫賣下年差科。高琪為相,議至榷油。進納濫官,輒售空名宣敕,或欲與以
 五品正班。僧道入粟,始自度牒,終至德號、綱副威儀、寺觀主席亦量其貲而鬻之。甚而丁憂鬻以求仕,監戶鬻以從良,進士出身鬻至及第。又甚而叛臣劇盜之效順,無金帛以備賞激,動以王爵固結其心,重爵不蔇,則以國姓賜之。名實混淆,倫法颻壞,皆不暇顧,國欲不亂,其可得乎?迨夫宋絕歲幣而不許和,貪其淮南之蓄,謀以力取,至使樞府武騎盡於南伐。訛可、時全之出,初志得糧,後乃尺寸無補,三軍僨亡,我師壓境,兵財俱困,無以禦之。故志金之食貨者,不能不為之掩卷而興慨也。《傳》曰:「作法於涼,其弊猶貪。作法於貪,弊將若何。」金起東海,
 其俗純實,可與返古。初入中夏,兵威所加,民多流亡,土多曠閒,遺黎惴惴,何求不獲。使於斯時,縱不能復井地溝洫之制,若用唐之永業、口分以制民產,仿其租庸調之法以足國計,何至百年之內所為經畫紛紛然,與其國相終始耶!其弊在於急一時之利,踵久壞之法,及其中葉,鄙遼儉朴,襲宋繁縟之文;懲宋寬柔,加遼操切之政。是棄二國之所長,而併用其所短也。繁褥勝必至於傷財,操切勝必至於害民,訖金之世,國用易匱,民心易離,豈不由是歟?作法不慎厥初,變法以救其弊,只益甚焉耳。其他鹽策、酒曲、常平、和糴、茶稅、征商、榷場等法,大
 概多宋舊人之所建明,息耗無定,變易靡恆,視錢鈔何異?田制、水利、區田之目,或驟行隨輟,或屢試無效,或熟議未行,咸著無篇,以備一代之制云。



 ○戶口



 金制,男女二歲以下為黃,十五以下為小,十六為中,十七為丁,六十為老,無夫為寡妻妾,諸篤廢疾不為丁。戶主推其長充,內有物力者為課役戶,無者為不課設戶。令民以五家為保。泰和六年,上以舊定保伍法,有司滅裂不行,其令結保,有匿姦細、盜賊者連坐。宰臣謂舊以五家為保,恐人易為計構而難覺察,遂令從唐制,五家為鄰、五鄰為保,以相檢察。京府州縣郭下則置坊
 正,村社則隨戶眾寡為鄉置里正,以按比戶口,催督賦役,勸課農桑。村社三百戶以上則設主首四人,二百戶以上三人,五十戶以上二人,以下一人,以佐里正禁察非違。置壯丁,以佐主首巡警盜賊。猛安謀克部村寨,五十戶以上設寨使一人,掌同主首。寺觀則設綱首。凡坊正、里正,以其戶十分內取三分,富民均出顧錢,募強幹有抵保者充,人不得過百貫,役不得過一年。大定二十九年,章宗嘗欲罷坊、里正,復以主首遠,入城應代,妨農不便,乃以有物力謹愿者二年一更代。凡戶口計帳,三年一籍。自正月初,州縣以里正、主首,猛安謀克則以寨使,詣編戶家責手實,具男女老幼年與姓名,生者增之,
 死者除之。正月二十日以實數報縣,二月二十日申州,以十日內達上司,無遠近皆以四月二十日到部呈省。凡漢人、渤海人不得充猛安謀克戶。猛安謀克之奴婢免為良者,止隸本部為正戶。凡沒入官良人,隸宮籍監為監戶,沒入官奴婢,隸太府監為官戶。



 當收國二年時,法制未定,兵革未息,貧民多依權右為茍安,多隱蔽為奴婢者。太祖下詔曰:「比以歲凶民飢,多附豪族,因陷為奴隸。及有犯法,徵償莫辦,折身為奴。或私約立限,以人對贖,過期則以為奴者。並聽以兩人贖一為良,元約以一人贖者從便。」天輔五年,以境土既拓,而舊部多瘠鹵,
 將移其民于泰州,乃遣皇弟昱及族子宗雄按視其地。昱等苴其土以進,言可種植,遂摘諸猛安謀克中民戶萬餘,使宗人婆盧火統之,屯種於泰州。婆廬火舊居阿注滸水又作按出虎,至是遷焉。其居寧江州者,遣拾得、查端、阿里徒歡、奚撻罕等四謀克,挈家屬耕具,徙于泰州,仍賜婆盧火耕牛五十。天輔六年,既定山西諸州,以上京為內地,則移其民實之。又命耶律佛頂以兵護送諸降人於渾河路,以皇弟昂監之,命從便以居。七年,以山西諸部族近西北二邊,且遼主未獲,恐陰相結誘,復命皇弟昂與孛堇稍喝等以兵四千護送,處之嶺東,惟西京
 民安堵如故,且命昂鎮守上京路。既而,上聞昂已過上京,而降人復苦其侵擾多叛亡者,遂命孛堇出里底往戒諭之,比至,而諸部已叛去。又以猛安詳穩留住所領歸附之民還東京,命有司常撫慰,且貸一歲之糧,其親屬被虜者皆令聚居。及七年取燕京路,二月,盡徙六州氏族富強工技之民於內地。太宗天會元年,以舊徙潤、隰等四州之民於沈州之境,以新遷之戶艱苦不能自存,詔曰:「比聞民乏食至鬻子者,聽以丁力等者贖之。」又詔孛堇阿實賚曰:「先皇帝以同姓之人昔有自鬻及典質其身者,命官為贖。今聞尚有未復者,其悉閱贖之。」又
 命以官粟贖上京路新遷置寧江州戶口貧而賣身者,六百餘人。二年,民有自鬻為奴者,詔以丁力等者易之。三年,禁內外官及宗室毋得私役百姓,權勢家不得買貧民為奴,其脅買者一人償十五人,詐買者一人償二人,罪皆杖百。七年,詔兵興以來,良人被略為驅者,聽其父母妻子贖之。熙宗皇統四年詔陜西、蒲、解、汝、蔡等州歲饑,百姓流落典雇為驅者,官以絹贖為良,丁男三匹,婦人幼小二匹。



 世宗大定二年,詔免二稅戶為民。初,遼人佞佛尤甚,多以良民賜諸寺,分其稅一半輸官,一半輸寺,故謂之二稅戶。遼亡,僧多匿其實,抑為賤,有援左
 證以告者,有司各執以聞,上素知其事,故特免之。十七年五月,省奏:「咸平府路一千六百餘戶,自陳皆長白山星顯、禪春河女直人,遼時簽為獵戶,移居於此,號移典部,遂附契丹籍。本朝義兵之興,首詣軍降,仍居本部,今乞釐正。」詔從之。二十年,以上京路女直人戶,規避物力,自賣其奴婢,致耕田者少,遂以貧乏,詔定制禁之。又謂宰臣曰:「猛安謀克人戶,兄弟親屬若各隨所分土,與漢人錯居,每四五十戶結為保聚,農作時令相助濟,此亦勸相之道也。」二十一年六月,徙銀山側民於臨潢。又命避役之戶舉家逃於他所者,元貫及所寓司縣官同罪,
 為定制。二十三年,定制,女直奴婢如有得力,本主許令婚娉者,須取問房親及村老給據,方許娉於良人。是年八月,奏猛安謀克戶口、墾地、牛具之數。猛安二百二,謀克千八百七十八,戶六十一萬五千六百二十四,口六百一十五萬八千六百三十六,內正口四百八十一萬二千六百六十九,奴婢口一百三十四萬五千九百六十七。墾田一百六十九萬三百八十頃有奇,牛具三十八萬四千七百七十一。在都宗室將軍司,戶一百七十,口二萬八千七百九十,內正口九百八十二,奴婢口二萬七千八百八。墾田三千六百八十三頃七十五畝,牛具三百四。迭剌、唐西二部五颭,戶五千五百八十五,口十三萬七
 千五百四十四,內正口十一萬九千四百六十三,奴婢口一萬八千八十一。墾田萬六千二十四頃一十七畝,牛具五千六十六。二十五年,命宰臣禁有祿人一子、及農民避課役,為僧道者。大定初,天下戶纔三百餘萬,至二十七年天下戶六百七十八萬九千四百四十九,口四千四百七十萬五千八十六。



 章宗大定二十九年十一月,上封事者言,乞放二稅戶為良。省臣欲取公牒可憑者為準,參知政事移剌履謂:「憑驗真偽難明,凡契丹奴婢今後所生者悉為良,見有者則不得典賣,如此則三十年後奴皆為良,而民且不病焉。」上以履言未當,令再議。省奏謂不拘括則訟終不
 絕,遂遣大興府治中烏古孫仲和、侍御史范楫分括北京路及中都路二稅戶,凡無憑驗,其主自言之者及因通檢而知之者,其稅半輸官,半輸主,而有憑驗者悉放為良。明昌元年正月,上封事者言:「自古以農桑為本,今商賈之外又有佛、老與他游食,浮費百倍。農歲不登,流殍相望,此末作傷農者多故也。」上乃下令,禁自披剃為僧、道者。是歲,奏天下戶六百九十三萬九千,口四千五百四十四萬七千九百,而粟止五千二百二十六萬一千餘石,除官兵二年之費,餘驗口計之,口月食五斗,可為四十四日之食。上曰:「蓄積不多,是力農者少故也。其集
 百官,議所以使民務本廣儲之道,以聞。」六月,奏北京等路所免二稅戶,凡一千七百餘戶,萬三千九百餘口,此後為良為驅,皆從已斷為定。明昌六年二月,上謂宰臣曰:「凡言女直進士,不須稱女直字。卿等誤作迴避女直、契丹語,非也。今如分別戶民,則女直言本戶,漢戶及契丹,餘謂之雜戶。」明昌六年十二月,奏天下女直、契丹、漢戶七百二十二萬三千四百,口四千八百四十九萬四百,物力錢二百六十萬四千七百四十二貫。泰和七年六月,敕,中物力戶,有役則多逃避,有司令以次戶代之,事畢則復業,以致大損不逃之戶。令省臣詳議。宰臣奏:「
 舊制太輕。」遂命課役全戶逃者徒二年,賞告者錢五萬。先逃者以百日內自首,免罪。如實銷乏者,內從御史臺,外從按察司,體究免之。十二月,奏天下戶七百六十八萬四千四百三十八,口四千五百八十一萬六千七十九。戶增於大定二十七年一百六十二萬三千七百一十五,口增八百八十二萬七千六十五。此金版籍之極盛也。



 及衛紹王之時,軍旅不息,宣宗立而南遷,死徙之餘,所在為虛矣。戶口日耗,軍費日急,賦斂繁重,皆仰給於河南,民不堪命,率棄廬田,相繼亡去。及屢降詔招復業者,免其歲之租,然以國用乏竭,逃者之租皆令居者代出,以故多不敢還。興定元年十二月,宣宗
 欲懸賞募人捕亡戶,而復慮騷動,遂命依已降詔書,已免債逋,更招一月,違而不來者然後捕獲治罪,而以所遺地賜人。四年,省臣奏:「河南以歲饑而賦役不息,所亡戶令有司招之,至明年三月不復業者,論如律。」時河壖為疆,烽鞞屢警,故集慶軍節度使溫迪罕達言:「亳州戶舊六萬,自南遷以來,不勝調發,相繼逃去,所存者曾無十一,碭山下邑,野無居民矣!」



 ○通檢推排



 通檢,即《周禮》大司徒三年一大比,各登其鄉之眾寡、六畜、車輦,辨物行征之制也。金自國初占籍之後,至大定四年,承正隆師旅之餘,民之貧富變更,賦役
 不均,世宗下詔曰:「粵自國初,有司常行大比,于今四十年矣。正隆時,兵役並興,調發無度,富者今貧不能自存,版籍所無者今為富室而猶幸免。是用遣信臣泰寧軍節度使張弘信等十三人,分路通檢天下物力而差定之,以革前弊,俾元元無不均之歎,以稱朕意。凡規措條理,命尚書省畫一以行。」又命:「凡監戶事產,除官所撥賜之外,餘凡置到百姓有稅田宅。皆在通檢之數。」時諸使往往以苛酷多得物力為功,弘信檢山東州縣尤為酷暴,棣州防禦使完顏永元面責之曰:「朝廷以正隆後差調不均,故命使者均之。今乃殘暴,妄加民產業數倍,一
 有來申訴者,則血肉淋離,甚者即殞杖下,此何理也?」弘信不能對,故惟棣州稍平。五年,有司奏諸路通檢不均,詔再以戶口多寡、貧富輕重,適中定之。既而,又定通檢地土等第稅法。十五年九月,上以天下物力,自通檢以來十餘年,貧富變易,賦調輕重不均,遣濟南尹梁肅等二十六人,分路推排。



 二十年四月,上謂宰臣曰:「猛安謀克戶,富貧差發不均,皆自謀克內科之,暗者惟胥吏之言是從,輕重不一。自窩斡叛後,貧富反復,今當籍其夾戶,推其家貲,儻有軍役庶可均也。」詔集百官議,右丞相克寧、平章政事安禮,樞密副使宗尹言:「女直人除猛安
 謀克僕從差使,餘無差役。今不推奴婢孳畜、地土數目,止驗產業科差為便。」左丞相守道等言:「止驗財產,多寡分為四等,置籍以科差,庶得均也。」左丞通、右丞道、都點檢襄言:「括其奴婢之數,則貧富自見,緩急有事科差,與一例科差者不同。請俟農隙,拘括地土牛具之數,各以所見上聞。」上曰:「一謀克戶之貧富,謀克豈不知。一猛安所領八謀克,一例科差。設如一謀無內,有奴婢二三百口者,有奴婢一二人者,科差與同,豈得平均。正隆興兵時,朕之奴婢萬數,孳畜數千,而不差一人一馬,豈可謂平。朕於庶事未嘗專行,與卿謀之。往年散置契丹戶,安
 禮極言恐擾動,朕決行之,果得安業。安禮雖盡忠,未審長策。其從左丞通等所見,拘括推排之。」十二月,上謂宰臣曰:「猛安謀克多新強舊弱,差役不均,其令推排,當自中都路始。」至二十二年八月,始詔令集耆老,推貧富,驗土地牛具奴婢之數,分為上中下三等。以同知大興府事完顏烏里也先推中都路,續遣戶部主事按帶等十四人與外官同分路推排。



 九月,詔:「毋令富者匿隱畜產,貧戶或有不敢養馬者。昔海陵時,拘括馬畜,絕無等級,富者倖免,貧者盡拘入官,大為不均。今並核實貧富造籍,有急即按籍取之,庶幾無不均之弊。」張汝弼、梁肅奏:「
 天下民戶通檢既定,設有產物移易,自應隨業輸納。至於浮財,須有增耗,貧者自貧,富者自富,似不必屢推排也。」上曰:「宰執家多有新富者,故皆不願也。」肅對曰:「如臣者,能推排中都物力。臣以嘗為南使,先自添物力錢至六十餘貫,視其他奉使無如臣多者。但小民無知,法出姦生,數動搖則易駭。如唐、宋及遼時,或三二十年不測通比則有之。頻歲推排,似為難爾。」二十六年,復以李晏等分路推排。二十七年,奏晏等所定物力之數。上曰:「朕以元推天下物力錢三百五萬餘貫,除三百萬貫外,令減五萬餘貫。今減不及數,復續收二萬餘貫,即是實二萬
 貫爾,而曰續收,何也?」對曰:「此謂舊脫漏而今首出者,及民地舊無力耕種,而今耕種者也。」上曰:「通檢舊數,止於視其營運息耗,與房地多寡,而加減之。彼人賣地,此人買之,皆舊數也。至如營運。此強則彼弱,強者增之,弱者減之而已。且物力之數蓋是定差役之法,其大數不在多寡也。朕恐實有營運富家所當出者,反分與貧者爾。」



 章宗大定二十九年六月,命為國信使之副者,免增物力。又命農民如有積粟,毋充物力,錢慳之郡,所納錢貨則許折粟帛。九月,以曹州河溢,遣馬百祿等推排遭墊溺州縣之貧乏者。明昌元年四月,刑部郎中路伯達等
 言:「民地已納稅,又通定物力,比之浮財所出差役,是為重併也。」遂詳酌民地定物力,減十之二。尚書戶部言,中都等路被水,詔委官推排,比舊減錢五千六百餘貫。明昌三年八月,敕尚書省:「百姓當豐稔之時不務積貯,一遇凶儉輒有阻飢,何法可使民重穀而多積也。」宰臣對曰:「二十九年,已詔農民能積粟免充物力。明昌初,命民之物力與地土通推者,亦減十分之二,此固其術也。」承安元年,尚書省奏:「是年九月當推排,以有故不克。」詔以冬已深,比事畢恐妨農作,乃權止之。二年冬十月,敕令議通檢,宰臣奏曰:「大定二十七年通檢後,距今已十年,
 舊戶貧弱者眾,儻遲更定,恐致流亡。」遂定制,已典賣物業,止隨物推收,析戶異居者許令別籍,戶絕及困弱者減免,新強者詳審增之,止當從實,不必敷足元數。邊城被寇之地,皆不必推排。於是,令吏部尚書賈執剛、吏部侍郎高汝礪先推排在都兩警巡院,示為諸路法。每路差官一員,命提刑司官一員副之。三年九月,奏十三路籍定推排物力錢二百五十八萬六千七百二貫四百九十文,舊額三百二萬二千七百十八貫九百二十二文,以貧乏除免六十三萬八千一百一十一貫。除上京,北京、西京路無新強增者,餘路計收二十萬二千九十
 五貫。泰和二年閏十二月,上以推排時,既問人戶浮財物力,而又勘當比次,期迫事繁,難得其實,敕尚書省,定人戶物力隨時推收法,令自今典賣事產者隨業推收,別置標簿,臨時止拘浮財物力以增減之。泰和四年十二月,上以職官仕於遠方,其家物力有應除而不除者,遂定典賣實業逐時推收,若無浮財營運,應除免者,令本家陳告,集坊村人戶推唱,驗實免之。造籍後如無人告,一月內以本官文牒推唱,定標附于籍。五年,以西京、北京邊地常罹兵荒,遣使推排之。舊大定二十六年所定三十五萬三千餘貫,遂減為二十八萬七千餘貫。五
 年六月,簽南京按察司事李革言:「近制,令人戶推收物力,置簿標題,至通推時,止增新強,銷舊弱,庶得其實。今有司奉行滅裂,恐臨時冗併,卒難詳審,可定期限,立罪以督之。」遂令自今年十一月一日,令人戶告詣推收標附,至次年二月一日畢,違期不言者坐罪。且令諸處稅務,具稅訖房地,每半月具數申報所屬,違者坐以怠慢輕事之罪。仍敕物力既隨業,通推時止令定浮財。八年九月,以吏部尚書賈守謙、知濟南府事蒲察張家奴、莒州刺史完顏百嘉、南京路轉運使宋元吉等十三員,分路同本路按察司官一員,推排諸路。上召至香閣,親諭
 之曰:「朕選卿等隨路推排,除推收外,其新強消乏戶,雖集眾推唱,然消乏者勿銷不盡,如一戶物力元三百貫,今蠲免二百五十貫猶有未當者。新強勿添盡,量存其力,如一戶可添三百貫,而止添二百貫之類。卿等各宜盡心,一推之後十年利害所關,茍不副所任,罪當不輕也。」



\end{pinyinscope}