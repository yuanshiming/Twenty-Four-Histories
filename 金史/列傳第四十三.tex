\article{列傳第四十三}

\begin{pinyinscope}

 ○程寀任熊祥孔璠子拯范拱張用直劉樞王翛楊伯雄兄伯淵蕭貢溫迪罕締達張翰任天寵



 程寀,字公弼,燕之析津人。祖冀,仕遼廣德軍節度使。冀凡六男,父子皆擢科第,士族號其家為「程一舉」。冀次子四穆,遼崇義軍節度使。寀,四穆之季子也。自幼如成人。
 及冠,篤學,中進士甲科,累遷殿中丞。天輔七年,太祖入燕,授尚書都官員外郎、錦州安昌令,累加起居郎,為史館修撰,以從軍有勞,加少府少監。熙宗時,歷翰林待制,兼右諫議大夫。寀上疏言事,其略曰:「殿前點檢司,古殿巖環衛之任,所以肅禁禦,尊天子、備不虞也。臣幸得近清光,從天子觀時畋之禮。比見陛下校獵,凡羽衛從臣,無貴賤皆得執弓矢馳逐,而聖駕崎嶇沙礫之地,加之林木叢鬱,易以迷失。是日自卯及申,百官始出沙漠,獨不知車駕何在。瞻望久之,始有騎來報,皇帝從數騎已至行在。竊惟古天子出入警蹕,清道而行。至於楚畋雲
 夢,漢獵長楊,皆大陳兵衛,以備非常。陛下膺祖宗付託之重,奈何獨與數騎出入林麓沙漠之中,前無斥候,後無羽衛,甚非肅禁禦之意也。臣願陛下熟計之。後若復獵,當預戒有司,圖上獵地,具其可否,然後下令清道而行。擇衝要稍平之地,為駐蹕之所,簡忠義爪牙之士,統以親信腹心之臣,警衛左右。俟其麋鹿既來,然後馳射。仍先遣搜閱林藪,明立標幟,為出入之馳道。不然,後恐貽宗朝社稷之憂。」



 又曰:「臣伏讀唐史,追尊高祖以下,謚號或加至十八字。前宋大中祥符間亦加至十六字,亡遼因之,近陛下亦受『崇天體道欽明文武聖德』十字。臣
 竊謂人臣以歸美報上為忠,天子以追崇祖考為孝。太祖武元皇帝受命開基,八年之間,奄有天下,功德茂盛,振古無前,止謚『武元』二字,理或未安,何以示將來?臣願詔有司定議謚號,庶幾上慰祖宗在天之靈,使耿光丕烈,傳于無窮。」



 又曰:「古者天子皆有巡狩,無非事者。或省察風俗,或審理冤獄,或問民疾苦,以布宣德澤,皆巡狩之名也。國家肇興,誠恐郡國新民,逐末棄本,習舊染之汙,奢侈詐偽,或有不明之獄,僭濫之刑,或力役無時,四民失業。今鑾輅省方,將憲古行事,臣願天心洞照,委之長貳,釐正風俗,或置匭匣,以申冤枉,或遣使郡國,問民
 無告,皆古巡狩之事。昔漢昭帝問疾苦,光武求民瘼,如此則和氣通,天下丕平可坐而待也。」



 又曰:「臣聞,善醫者不視他人之肥瘠,察其脈之病否而已;善計天下者不視天下之安危,察其紀綱理否而已。天下者人也,安危者肥瘠也,紀綱者脈也,脈不病雖瘠不害,脈病而肥者危矣。是故,四肢雖無故,不足恃也,脈而已矣。天下雖無事,不足矜也,綱紀而已矣。尚書省,天子喉舌之官,綱紀在焉。臣願詔尚書省,戒勵百官,各揚其職,以立綱紀。如吏部天官以進賢退不肖為任,誠使升黜有科,任得其人,則綱紀理而民受其賜,前代興替,未始不由此者。」



 又曰:「虞
 舜不告而娶二妃。帝嚳娶四妃,法天之四星。周文王一後、三夫人,嬪御有數。選求淑媛以充後宮,帝王之制也。然女無美惡,入宮見妒,陛下欲廣嗣續,不可不知而告戒之。」



 又曰:「臣伏見本朝富有四海,禮樂制度,莫不一新。宮禁之制,尚未嚴密,胥吏健卒之輩,皆得出入,莫有呵止,至淆混而無別。雖有闌入之法,久尚未行,甚非嚴禁衛、明法令之意,陛下不可不知而必行。」



 疏奏,上嘉納之,於是始命有司議增上太祖尊謚。皇統八年十二月,由翰林侍講學士為橫海軍節度使,移彰德軍節度使。卒官,年六十二。寀剛直耿介,不諂奉權貴以希茍進,有古
 君子之風云。



 任熊祥,字子仁。八代祖圜,為後唐宰相。圜孫睿,隨石晉北遷,遂為燕人。熊祥登遼天慶八年進士第,為樞密院令史。太祖平燕,以其地畀宋,熊祥至汴,授武當丞。宋法,新附官不釐務,熊祥言於郡守楊皙曰:「既不與事,請止給半俸以養親。」皙雖不許,而喜其廉。金人取均、房州,熊祥歸朝,復為樞密院令史。時西京留守高慶裔攝院事,無敢忤其意者,熊祥未嘗阿意事之。其後杜充、劉筈同知燕京行省,法制未一,日有異論,熊祥為折衷之。歷深、磁州刺史,開封少尹,行臺工部郎中,同知汴京留守事。
 天德初,為山東東路轉運使,改鎮西軍節度使。是時,詔徐文、張弘信討東海縣,弘信逗遛,稱疾不進,決杖二百。熊祥被詔為會試主文,以「事不避難臣之職」為賦題。及御試,熊祥復以「賞罰之令信如四時」為賦題,海陵大喜,以為翰林侍讀學士。大定初,起為太子少師。時契丹賊窩斡竊號,北鄙用兵未息,上以為憂,詔公卿百官議所以招伐之宜。眾皆異議,熊祥徐進曰:「陛下以勞民為憂,用兵為重,莫若以恩信招懷之。」上問:「孰可使者?」對曰:「臣雖老,憑國威靈,尚堪一行。」上曰:「卿老矣,無煩為此。」七年,復致仕。熊祥事母以孝聞,母沒時,熊祥年已七十,不食
 三日,人皆稱之。卒于家。



 孔璠,字文老,至聖文宣王四十九代孫,故宋朝奉郎襲封端友弟端操之子。齊阜昌三年補迪功郎,襲封衍聖公,主管祀事。天會十五年,齊國廢。熙宗即位,興制度禮樂,立孔子廟於上京。天眷三年,詔求孔子後,加璠承奉郎,襲封衍聖公,奉祀事。是時,熙宗頗讀《論語》、《尚書》、《春秋左氏傳》及諸史、《通歷》、《唐律》,乙夜乃罷。皇統元年三月戊午,上謁奠孔子廟,北面再拜,顧謂侍臣曰:「朕幼年游佚,不知志學,歲月逾邁,深以為悔。大凡為善,不可不勉。孔子雖無位,其道可尊,萬世高仰如此。」皇統三年,璠卒。子
 拯襲封,加文林郎。



 拯字元濟。天德二年,定襲封衍聖公俸格,有加于常品。是歲立國子監,久之,加拯承直郎。大定元年卒。弟總襲封,加文林郎。



 總字元會。大定二十年,召總至京師,欲與之官。尚書省奏:「總主先聖祀事,若加任使,守奉有闕。」上曰:「然。」乃授曲阜縣令。明昌元年卒。子元措襲封,加文林郎。



 元措字夢得。三年四月詔曰:「衍聖公視四品,階止八品,不稱。可超遷中議大夫,永著于令。」四年八月丁未,章宗行釋奠禮,北面再拜,親王、百官、六學生員陪位。承安二年正月,詔元措兼曲阜縣令,仍世襲。元措歷事宣宗、
 哀宗,後歸大元終焉。



 四十八代端甫者,明昌初,學士黨懷英薦其年德俱高,讀書樂道,該通古學。召至京師,特賜王澤榜及第,除將仕郎、小學教授,以主簿半俸致仕。



 范拱,字清叔,濟南人。九歲能屬文,深於《易》學。宋末登進士第,調廣濟軍曹,權邦彥辟為書記,攝學事。劉豫鎮東平,拱撰謁廟文,豫奇之,深加賞識。拱獻《六箴》。



 齊國建,累擢中書舍人。上《初政錄》十五篇:一曰《得民》,二曰《命將》,三曰《簡禮》,四曰《納諫》,五曰《遠圖》,六曰《治亂》,七曰《舉賢》,八曰《守令》,九曰《延問》,十曰《畏慎》,十一曰《節祥瑞》十二曰《戒雷同》,十三曰《用人》,十四曰《御將》,十五曰《御軍》。豫納其說而不
 能盡用也。久之,權尚書右丞,進左丞,兼門下侍郎。



 豫以什一稅民,名為古法,其實裒斂,而刑法嚴急,吏夤緣為暴。民久罹兵革,益窮困,陷罪者眾,境內苦之。右丞相張孝純及拱兄侍郎巽,極言其弊,請仍因履畝之法,豫不從。巽坐貶官,自是無復敢言者。拱曰:「吾言之則為黨兄,不言則百姓困弊。吾執政也,寧為百姓言之。」乃上疏,其大略以為「國家懲亡宋重斂弊,什一稅民,本務優恤,官吏奉行太急,驅民犯禁,非長久計也」。豫雖未即從,而亦不加譴。拱令刑部條上諸路以稅抵罪者凡千餘人,豫見其多,乃更為五等稅法,民猶以為重也。



 齊廢,梁王宗
 弼領行臺省事,拱為官屬。宗弼訪求百姓利病,拱以減稅為請,宗弼從之,減舊三分之一,民始蘇息。拱慎許可,而推轂士,李南、張輔、劉長言皆拱薦也。長言自汝州郟城酒監擢省郎,人不知其所以進,拱亦不自言也。以久病乞近郡,除淄州刺史。皇統四年,以疾求退,以通議大夫致仕。齋居讀書,罕對妻子。



 世宗在濟南聞其名。大定初,拱上封事。七年,召赴闕,除太常卿。議郊祀。或有言前代都長安及汴、洛,以太、華等山列為五岳,今既都燕,當別議五岳名。寺僚取《崧高》疏「周都酆鎬,以吳嶽為西岳」。拱以為非是,議略曰:「軒轅居上谷,在恒山之西,舜居蒲
 阪,在華山之北。以此言之,未嘗據所都而改岳祀也。」後遂不改。拱嘗言:「禮官當守禮,法官當守法,若漢張釋之可謂能守法矣。」故其議論確然不可移奪。九年,復致仕,卒於家,年七十四。



 張用直,臨潢人。少以學行稱。遼王宗乾聞之,延置門下,海陵與其兄充皆從之學。天眷二年,以教宗子賜進士及第,除禮部郎中。皇統四年,為宣徽判官,歷橫海軍節度副使,改寧州刺史。海陵即位,召為簽書徽政院事、太常卿、太子詹事。海陵嘗謂用直曰:「朕雖不能博通經史,亦粗有所聞,皆卿平昔輔導之力。太子方就學,宜善導
 之。朕父子並受卿學,亦儒者之榮也。」為賀宋國正旦使,卒於汴。海陵深悼惜之,遣使迎護其喪,官給道途費。喪至,親臨奠,賜錢千萬。其養子始七歲,特授武義將軍。



 劉樞,字居中,通州三河人。少以良家子從軍,屯河間。同輩皆騎射,獨樞刻意經史。登天眷二年進士,調唐山主簿。改飛狐令,蔚州刺史恃功貪汙無所顧忌,屬邑皆厭苦之,樞一無所應,乃摭以他事繫獄,將致之死。郡人有憐樞者,導樞脫走,訴於朝。會廉察使至,守倅而下皆抵罪廢,獨樞治狀入優等,躐遷奉直大夫。張浩營建燕京宮室,選樞分治工役。遷尚書刑部員外郎,鞫治太原尹
 徒單阿里出虎反狀,旬日獄具。轉工部郎中,進本部侍郎。正隆末,從軍還自江上。大定初,與左司郎中王蔚、右司員外郎王全俱出補外,樞為南京路轉運使事。初,世宗欲復用樞等,御史臺奏:「樞等在正隆時皆以巧進,敗法蠹政,人多怨嫉之。」上以樞等頗幹濟,猶用之,戒之曰:「能悛心改過,必加升擢。不然,則斥汝等矣。」是時,阿勒根彥忠為南京都轉運使,不閑吏事,故用樞以佐之。遷山東路轉運使,改中都路轉運使。大定四年,卒于官。



 王翛,字翛然,涿州人也。登皇統二年進士第,由尚書省令史除同知霸州事。累遷刑部員外郎。坐請囑故人姦
 罪,杖四十,降授泰定軍節度副使。四遷大興府治中,授戶部侍郎。世宗謂宰臣曰:「王翛前為外官,聞有剛直名。今聞專務出罪為陰德,事多非理從輕。又巧倖偷安,若果剛直,則當忘身以為國,履正以無偏,何必賣法以徼福耶?」尋命賑濟密雲等三十六縣猛安人戶,冒請粟三萬餘石,為尚書省奏奪官一階,出為同知北京留守事。上曰:「人多言王翛能官,以朕觀之,凡事不肯盡力,直一老姦耳。」二十四年,遷遼東路轉運使。歲餘,改顯德軍節度使。以前任轉運使拽辱倉使王祺致死,追兩官解職,敕杖七十,降授鄭州防禦使。



 章宗即位,擢同知大興府事。
 審錄官奏,翛前任顯德潔廉剛直,軍吏斂迹,無訟獄。遷禮部尚書,兼大理卿。使宋還,會改葬太師廣平郡王徒單貞。貞,章宗母孝懿皇后父也。帝欲用前代故事,班劍、鼓吹、羽葆等儀衛。宰臣以貞與弒熙宗誅死,意難之。於是詔下禮官議。翛言:「晉葬丞相王導,給前後羽葆、鼓吹、武賁、班劍百人。唐以來,大駕鹵簿有班劍,其王公以下鹵簿並無班劍,兼羽葆非臣下所宜用,國朝葬大臣亦無之。」上先知唐葬大臣李靖等皆用班劍、羽葆,怒曰:「典故所無,固可從,然用之亦不過禮。」一日,詔翛及諫議大夫兼禮部侍郎張暐詣殿門,諭之曰:「朝廷之事,汝諫
 官、禮官即當辯析。且小民言可採,朕尚從之,況卿等乎?自今議事,毋但附合尚書省。」



 明昌二年,改知大興府事。時僧徒多游貴戚門,翛惡之,乃禁僧午後不得出寺。嘗一僧犯禁,皇姑大長公主為請,翛曰:「奉主命,即令出之。」立召僧,杖一百死,京師肅然。後坐故出人罪,復削官解職。明年,特授定海軍節度使。諭旨曰:「卿賦性太剛,率意行事,乃自陷於刑。若殿年降敘,念卿入仕久,頗有執持,故特起於罪謫之中,授以見職。且彼歲歉民飢,盜賊多,須用舊人鎮撫,庶得安治。勉晝乃心,以圖後效。」未幾,表乞致仕。上曰:「翛能幹者,得力為多。」不許。復申請,從之。泰
 和七年,卒,年七十五。



 翛性剛嚴,臨事果決,吏民憚其威,雖豪右不敢犯。承安間,知大興府事闕,詔諭宰臣曰:「可選極有風力如王翛輩者用之。」其為上所知如此。



 楊伯雄,字希雲,真定槁城人。八世祖彥稠,後唐清泰中為定州兵馬使。後隨晉主北遷,遂居臨潢。父丘行,太子左衛率府率。



 伯雄登皇統二年進士,海陵留守中京,丘行在幕府,伯雄來省視,海陵見之,深加器重。久之,調韓州軍事判官。有二盜詐稱賈販,逆旅主人見欺,至州署陳訴,實欲劫取伯雄。伯雄心覺其詐,執而詰之,并獲其黨十餘人,一郡駭服。遷應奉翰林文字。是時,海陵執政,
 自以舊知伯雄,屬之使時時至其第,伯雄諾之而不往也。他日海陵怪問之,對曰:「君子受知於人當以禮進,附麗奔走,非素志也。」由是愈厚待之。



 海陵篡立,數月,遷右補闕,改修起居注。海陵銳於求治,講論每至夜分。嘗問曰:「人君治天下,其道何貴?」對曰:「貴靜。」海陵默然。明日,復謂曰:「我遷諸部猛安分屯邊戍,前夕之對,豈指是為非靜邪?」對曰:「徙兵分屯,使南北相維,長策也。所謂靜者,乃不擾之耳。」乙夜,復問鬼神事。伯雄進曰:「漢文帝召見賈生,夜半前席,不問百姓而問鬼神,後世頗譏之。陛下不以臣愚陋,幸及天下大計,鬼神之事,未之學也。」海陵曰:「但
 言之,以釋永夜倦思。」伯雄不得已,乃曰:「臣家有一卷書,記人死復生,或問冥官何以免罪,答曰,汝置一歷,白日所為,暮夜書之,不可書者是不可為也。」海陵為之改容。夏日,海陵登瑞雲樓納涼,命伯雄賦詩,其卒章云:「六月不知蒸鬱到,清涼會與萬方同。」海陵忻然,以示左右曰:「伯雄出語不忘規戒,為人臣當如是矣。」再遷兵部員外郎。丁父憂,起復翰林待制,兼修起居注。遷直學士,再遷右諫議大夫,兼著作郎,修起居注如故。



 皇子慎思阿不薨,伯雄坐與同直者竊議被責,語在《海陵諸子傳》。海陵議征江南,伯雄奏:「晉武平吳,皆命將帥,何勞親總戎律?」
 不聽。乃落起居注,不復召見。大定初,除大興少尹,丁母憂。顯宗為皇太子,遷東宮官屬,張浩薦伯雄,起復少詹事,兄子蟠為左贊善,言聽諫從,時論榮之。集古太子賢不肖為書,號《瑤山往鑒》,進之。及進《羽獵》、《保成》等箴,皆見嘉納。復為左諫議大夫、翰林直學士。會太子詹事闕,宰相復舉伯雄。上曰:「伯雄不可去朕左右,而東宮亦須輔導。」遂以太子詹事兼諫議。



 六年,上幸西京,欲因往涼陘避暑,伯雄率眾諫官入諫。上曰:「朕徐思之。」伯雄言之不已,同列皆引退,久之乃起。是年,至涼陘,徼巡果有疏虞。上思伯雄之言,及還,遷禮部尚書,謂近臣曰:「群臣有幹
 局者眾矣,如伯雄忠實,皆莫及也。」上謂伯雄曰:「龍逄、比干皆以忠諫而死,使遇明君,豈有是哉!」伯雄對曰:「魏徵願為良臣,正謂遇明君耳。」因顧謂宰相曰:「《書》曰:『汝無面從,退有後言。』朕與卿等共治天下,事有可否,即當面陳。卿等致位卿相,正行道揚名之時,偷安自便,徼倖一時,如後世何?」群臣皆稱萬歲。



 十二年,改沁南軍節度使,召為翰林學士承旨。丞相石琚致仕,上問:「誰可代卿者?」琚對曰:「伯雄可。」時論以琚舉得其人。復權詹事,伯雄知無不言,匡救弘多。後宮僚有詭隨者,人必稱楊詹事以愧之。除定武軍節度使,改平陽尹。先是,張浩治平陽,有惠
 政,及伯雄為尹,百姓稱之,曰:「前有張,後有楊。」徙河中尹。卒,年六十五。謚莊獻。弟伯傑、伯仁,族兄伯淵。



 伯淵字宗之。父丘文,遼中書舍人。伯淵早孤,事母以孝聞,疏財好施,喜收古書。天會初,以名家子補尚書省令史。十四年,賜進士第,歷吏、禮二部主事、御前承應文字,秩滿,除同知永定軍節度使事。召為司計郎中。知平定軍,用廉,遷平州路轉運使。知泰安軍,有惠政,百姓刻石紀其事。四遷山東東路轉運使。正隆末,群盜蜂起,州郡往往罹害,獨濟南賴伯淵保全。大定三年,致仕,卒于家。



 蕭貢,字真卿,京兆咸陽人。大定二十二年進士,調鎮戎
 州判官,涇陽令,涇州觀察判官。補尚書省令史。舊例,試補兩月,乃補用。貢至數日,執政以為能,即用之。擢監察御史。提刑司奏涇州有美政,遷北京轉運副使。親老,歸養。左丞董師中、右丞楊伯通薦其文學,除翰林修撰。上書論:「比年之弊,人才不以器識、操履,巧於案牘,不涉吏議者為工。用人不務因才授官,惟泥資敘。名器不務慎與,人多僥倖。守令不務才實,民罹其害。伏望擢真才以振澆俗,核功能以理職業,慎名器以抑僥倖,重守令以厚邦本。然後政化可行,百事可舉矣。」詔詞臣作《唐用董重質誅郭誼得失論》,貢為第一,賜重幣四端。貢論時政
 五弊,言路四難,詞意切至,改治書侍御史。丁父憂,起復,改右司員外郎,尋轉郎中,遷國子祭酒,兼太常少卿,與陳大任刊修《遼史》。改刑部侍郎,歷同知大興府事、德州防禦使,三遷河東北路按察轉運使。大安末,改彰德軍節度事。坐兵興不能守城,亡失百姓,降同知通遠軍節度使。未幾,改靜難軍節度使,歷河東北路、南京路轉運使、御史中丞,戶部尚書。南京戒嚴,坐乏軍儲,詔釋不問。興定元年,致仕。元光二年卒,謚文簡。貢好學,讀書至老不倦,有注《史記》一百卷。



 溫迪罕締達,該習經史,以女直字出身,累官國史院編
 修官。初,丞相希尹制女直字,設學校,使訛離剌等教之。其後學者漸盛,轉習經史,故納合椿年、紇石烈良弼皆由此致位宰相。締達最號精深。大定十二年,詔締達所教生員習作詩、策,若有文采,量才任使,其自願從學者聽。十三年,設女直進士科。是歲,徒單鎰等二十七人登第。十五年,締達遷著作佐郎,與編修官宗璧、尚書省譯史阿魯、吏部令史張克忠譯解經書。累遷祕書丞。十九年,改左贊善,以母老求養。顯宗使內直丞六斤謂締達曰:「贊善,初未除此官,天子謂孤曰:『朕得一出倫之才,學問該貫,當令輔汝德義。』既數日,贊善除此官。自謂親炙
 德義,不勝其喜。未可去也,勿難于懷。」久之,轉翰林待制,卒。明昌五年,贈翰林學士承旨,謚文成。



 子二十,章宗即位,以為符寶典書,累官左諫議大夫。貞祐四年,上疏,略曰:「今邊備未撤,徵調不休,州縣長吏不知愛養其民,督責徵科,鞭笞逼迫,急於星火,文移重復,不勝其弊,宜敕有司務從簡易。兵興以來,忠臣烈士,孝子順孫,義夫節婦,湮沒無聞者甚眾,乞遣史官一員,廣為采訪,以議褒嘉。」興定元年,遷武勝軍節度使,改吏部尚書,知開封府。坐縱軍人家屬出城,當杖,詔解職。四年,復知開封府,復坐以事囑警巡使完顏金僧奴,降為鄭州防禦使。未幾,
 復為知開封府事。



 張翰,字林卿,忻州秀容人。大定二十八年進士,調隰州軍事判官。有誣昆弟三人為劫者,翰微行廉得其狀,白於州釋之。歷東勝、義豐、會川令,補尚書省令史,除戶部主事,遷監察御史。丁母憂,服闋,調山東路鹽使。丁父憂,起復尚書省都事、戶部員外郎。大安間,平章政事獨吉思忠、參知政事承裕行省戍邊,翰充左右司郎中,論議不相協。處置乖方,翰屢爭之不見省。承裕就逮,衛紹王知翰嘗有言,召見撫慰之。改知登聞鼓院,兼前職,遷侍御史。貞祐初,為翰林直學士,充元帥府經歷官。中都戒嚴,調
 度方殷,改戶部侍郎。宣宗遷汴,翰規措扈從糧草至真定,上書言五事:「一曰強本。謂當裒兵徒、徙豪民,以實南京。二曰足用。謂當按蔡、汴舊渠以通漕運。三曰防亂。謂當就集義軍假之官印,使相統攝,以安反側。四曰省事。謂縣邑不能自立者宜稍併之,既以省官,且易於備盜。五曰推恩。謂當推恩以示天子所在稱幸之意。」上略施行之。翰雅有治劇才,所至輒辦。遷河平軍節度使、都水監、提控軍馬使,俄改戶部尚書。是時,初至南京,庶事草略,翰經度區處,皆有條理。是歲卒,謚達義。



 任天寵,字清叔,曹州定陶人也,明昌二年進士,調考城
 主簿,再遷威戎縣令。縣故堡寨,無文廟學舍,天寵以廢署建。有兄弟訟田者,天寵諭以理義,委曲周至,皆感泣而去。調泰定軍節度判官。丁父憂,服闋,調崇義軍節度判官。補尚書省令史、右三部檢法司正,遷監察御史。改右司都事,遷員外郎。改左司諫,轉左司郎中,遷國子祭酒。貞祐初,轉祕書監兼吏部侍郎,改中都路都轉運使。時京師戒嚴,糧運艱阻,天寵悉力營辦,曲盡勞瘁,出家貲以濟飢者,全活甚眾。監察御史高夔、劉元規舉天寵二十人公勤明敏,有材幹,可安集百姓。遷戶部尚書。三年,中都不守,天寵繼走南京,中道遇兵,死之。謚純肅。



 贊曰:程寀、任熊祥,遼之進士,孔璠、範拱事宋、事齊,太祖皆見禮遇,而金之文治日以盛矣。張用直,海陵父子並列舊學。劉樞之練達,王翛之強敏於事,楊伯雄之善諷諫、工辭藻,蕭貢、溫迪罕締達之文藝適時,之數人者迭用於正隆、大定、明昌之間。張翰、任天寵之經理調度,宣宗南遷,猶賴其用焉。金源氏百餘年所以培植人才而獲其效者,於斯可概見矣。



\end{pinyinscope}