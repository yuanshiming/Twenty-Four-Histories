\article{列傳第五}

\begin{pinyinscope}

 ○石顯桓赧弟散達烏春溫敦蒲刺附臘醅弟麻產鈍恩留可阿疏奚王回離保



 石顯,孩懶水烏林荅部人。昭祖以條教約束諸部,石顯陸梁不可制。及昭祖沒于逼刺紀村,部人以柩歸,至孩懶水,石顯與完顏部窩忽窩出邀於路,攻而奪之柩,揚
 言曰:「汝輩以石魯為能而推尊之,吾今得之矣。」昭祖之徒告于蒲馬太彎,與馬紀嶺劾保村完顏部蒙葛巴土等募軍追及之,與戰,復得柩。眾推景祖為諸部長,白山、耶悔、統門、耶懶、土骨論、五國皆從服。



 及遼使曷魯林牙來索逋人,石顯皆拒阻不聽命,景祖攻之,不能克。景祖自度不可以力取,遂以詭計取之。乃以石顯阻絕海東路請於遼,遼帝使人讓之曰:「汝何敢阻絕鷹路?審無他意,遣其酋長來。」石顯使其長子婆諸刊入朝,曰:「不敢違大國之命。」遼人厚賜遣還,謂婆諸刊曰:「汝父信無他,宜身自入朝。」石顯信之,明年入見於春搜,婆諸刊從。遼主
 謂石顯曰:「罪惟在汝,不在汝子。」乃命婆諸刊還,而流石顯於邊地。蓋景祖以計除石顯而欲撫有其子與部人也。



 婆諸刊蓄怨未發,會活刺渾水紇石烈部臘醅、麻產起兵,婆諸刊往從之。及敗於暮棱水,麻產先遁去,婆諸刊與臘醅就擒,及其黨與,皆獻之遼主。久之,世祖復使人言曰:「婆諸刊不還,則其部人自知罪重,因此恐懼,不肯歸服。」遼主以為然,遂遣婆諸刊及前後所獻罪人皆還之。



 桓赧、散達兄弟者,國相雅達之子也。居完顏部邑屯村。雅達稱國相,不知其所從來。景祖嘗以幣與馬求國相
 於雅達,雅達許之。景祖得之,以命肅宗,其後撒改亦居是官焉。



 桓赧兄弟嘗事景祖。世祖初,季父跋黑有異志,陰誘桓赧欲與為亂。昭肅皇后往邑屯村,世祖、肅宗皆從行,遇桓赧、散達各被酒,言語紛爭,遂相毆擊,舉刃相向。昭肅皇后親解之,乃止,自是謀益甚。



 是時烏春、窩謀罕亦與跋黑相結,詭以烏不屯賣甲為兵端,世祖不得已而與之和。間數年,烏春以其眾涉活論、來流二水,世祖親往拒之。桓赧、散達遂起兵。



 肅宗以偏師拒桓赧、散達。世祖畏其合勢也,戒之曰:「可和則和,否則戰。」至斡魯紺出水,既陣成列,肅宗使盆德勃堇議和。桓赧亦恃烏
 春之在北也,無和意。盆德報肅宗曰:「敵欲戰。」或曰:「戰地迫近村墟,雖勝不能盡敵,宜退軍誘之寬地。」肅宗惑之,乃令軍少卻,未能成列。桓赧、散達乘之,肅宗敗焉。桓赧乘勝,大肆鈔略。是役也,烏春以久雨不能前,乃罷兵。



 世祖聞肅宗敗,乃自將,經舍很、貼割兩水取桓赧、散達之家、桓赧、散達不知也。世祖焚其所居,殺略百許人而還。未至軍,肅宗之軍又敗。世祖至,責讓肅宗失利之狀,使歡都、冶訶以本部七謀克助之,復遣人議和。桓赧、散達欲得盈歌之大赤馬、辭不失之紫騮馬,世祖不許,遂與不術魯部卜灰、蒲察部撒骨出及混同江左右匹古敦
 水北諸部兵皆會,厚集為陣,嗚鼓作氣馳騁。桓赧恃其眾,有必勝之心,下令曰:「今天門開矣,悉以爾車自隨。凡烏古迺夫婦寶貨財產恣爾取之,有不從者俘略之而去。」於是婆多吐水裴滿部斡不勃堇附於世祖,桓赧等縱火焚之。斡不死,世祖厚撫其家,既定桓赧,以舊地還之。



 桓赧軍復來,蒲察部沙祗勃堇、胡補荅勃堇使阿喜間道來告,且問曰:「寇將至,吾屬何以待之?」世祖復命曰:「事至此,不及謀矣。以眾從之,自救可也,惟以旗幟自別耳。」每有兵至,則輒遣阿喜穿林潛來,令與畢察往還大道,即故潛往來林中路也。桓赧至北隘甸,世祖將出兵,
 聞跋黑食於馳滿村死矣。乃沿安術虎水行,且欲並取海故術烈速勃堇之眾而後戰。覘者來報曰:「敵至矣。」世祖戒辭不失整軍速進,使待於脫豁改原。當是時,桓赧兵眾,世祖兵少,眾寡不敵。比世祖至軍,士氣恤甚。世祖心知之而不敢言,但令解甲少憩,以水洗面,飲鮮水。頃之,士氣稍蘇息。是時,肅宗求救於遼,不在軍中。將戰,世祖屏人獨與穆宗私語,兵敗,則就與肅宗乞師以報仇。仍令穆宗勿預戰事,介馬以觀勝負,先圖去就。乃袒袖韔弓服矢,以縕袍下幅護前後心,三揚旗,三撾鼓,棄旗提劍,身為軍鋒,盡銳搏戰。桓赧步軍以干盾進,世祖之
 眾以長槍擊之,步軍大敗。辭不失從後奮擊之,桓赧之騎兵亦敗。世祖乘勝逐北,破多退水水為之赤。世祖止軍勿追,盡獲所棄車甲馬牛軍實,以戰勝告于天地,頒所獲於將士,各以功為差。



 未幾,桓赧、散達俱以其屬來降。卜灰猶保撒阿辣村,招之不出。撒骨出據阿魯紺出村,世祖遣人與之議和,撒骨出謾言為戲,答之曰:「我本欲和,壯士巴的懣不肯和,泣而謂我曰:『若果與和,則美衣肥羊不可復得。』是以不敢從命。」遂縱兵俘略鄰近村墅。有人從道傍射之,中口死。



 卜灰之屬曰石魯,石魯之母嫁丁馳滿部達魯罕勃堇而為之妾。達魯罕與族兄
 弟抹腮引勃堇俱事世祖,世祖欲間石魯於卜灰,謂達魯罕曰:「汝之事我,不如抹腮引之堅固也。」蓋謂石魯母子一彼焉,一此焉,以此撼石魯。石魯聞之,遂殺卜灰而降。



 石魯通於卜灰之妾,常懼得罪,及聞世祖言,惑之,使告於達魯罕曰:「將殺卜灰而來,汝待我于江。」伺卜灰睡熟,剚刃於胸而殺之。追者急,白日露鼻匿水中,逮夜,至江,方游以濟。達魯罕使人待之,乃得免。久之,醉酒,而與達魯罕狠爭,達魯罕殺之。



 烏春,阿跋斯水溫都部人,以鍛鐵為業。因歲歉,策杖負簷與其族屬來歸。景祖與之處,以本業自給。既而知其
 果敢善斷,命為本部長,仍遣族人盆德送歸舊部。盆德,烏春之甥也。



 世祖初嗣節度使,叔父跋黑陰懷覬覦,間誘桓赧、散達兄弟及烏春、窩謀罕等。烏春以跋黑居肘腋為變,信之,由是頗貳於世祖,而虐用其部人。部人訴於世祖,世祖使人讓之曰:「吾父信任汝,以汝為部長。今人告汝有實狀,殺無罪人,聽訟不平,自今不得復爾為也。」烏春曰:「吾與汝父等輩舊人,汝為長能幾日,於汝何事。世祖內畏跋黑,恐郡朋為變,故曲意懷撫,而欲以婚姻結其歡心。使與約婚,烏春不欲,笑曰:「狗彘之子同處,豈能生育。胡里改與女直豈可為親也。」烏春欲發兵,而
 世祖待之如初,無以為端。



 加古部烏不屯,亦鐵工也,以被甲九十來售。烏春聞之,使人來讓曰:「甲,吾甲也。來流水以南、匹古敦水以北,皆吾土也。何故輒取吾甲,其亟以歸我。」世祖曰:「彼以甲來市,吾與直而售之。」烏春曰:「汝不肯與我甲而為和解,則使汝叔之子斜葛及廝勒來。」斜葛蓋跋黑之子也。世祖度其意非真肯議和者,將以有為也,不欲遣。眾固請曰:「不遣則必用兵。」不得已,遣之。謂廝勒曰:「斜葛無害。彼且執汝矣,半途辭疾勿往。」既行,廝勒曰:「我疾作,將止不往。」斜葛曰:「吾亦不能獨往矣。」同行者強之使行。既見烏春,烏春與斜葛厚為禮,而果執
 廝勒,曰:得甲則生,否則殺汝。」世祖與其甲,廝勒乃得歸。烏春自此益無所憚。



 後數年,烏春舉兵來戰,道斜寸嶺,涉活論、來流水,舍於術虎部阿里矮村滓布乃勃堇家。是時十月中,大雨累晝夜不止,冰澌覆地,烏春不能進,乃引去。於是桓赧、散達亦舉兵。世祖自拒烏春,而使肅宗拒桓赧。巳而烏春遇雨歸,叔父跋黑亦死,故世祖得併力於桓赧、散達,一戰而遂敗之。



 斡勒部人盃乃,舊事景祖,至是亦有他志,徙於南畢懇忒村,遂以縱火誣歡都,欲因此除去之,語在《歡都傳》中。世祖獲盃乃,釋其罪,盃乃終不自安,徙居吐窟村,與烏春、窩謀罕結約。烏春
 舉兵度嶺,世祖駐軍屋闢村以待之。進至蘇素海甸,兩軍皆陣,將戰,世祖不親戰,命肅宗以左軍戰,斜列、辭不失助之,徵異夢也。肅宗束縕縱火,大風從後起,火熾烈,時八月,野草尚青,火盡燎,煙焰張天。烏春軍在下風,肅宗自上風擊之,烏春大敗,復獲盃乃,獻於遼,而城蘇素海甸以據之。



 紇石烈臘醅、麻產與世祖戰於野鵲水。世祖中四創,軍敗。臘醅使舊賊禿罕等過青嶺,見烏春,賂諸部與之交結。臘醅、麻產求助於烏春,烏春以姑里甸兵百十七人助之。世祖擒臘醅獻于遼主,并言烏春助兵之狀,仍以不修鷹道罪之。遼主使人至烏春問狀,烏春
 懼,乃為讕言以告曰:「未嘗與臘醅為助也。德鄰石之北,姑里甸之民,所管不及此。」



 臘醅既敗,世祖盡得烏春姑里甸助兵一百十七人,而使其卒長斡善、斡脫往招其眾,繼遣斜缽勃堇撫定之。斜缽不能訓齊其人,蒲察部故石、跋石等誘三百餘人入城,盡陷之。世祖治鷹道還,斜列來告,世祖使歡都為都統,破烏春、窩謀罕於斜堆,故石、跋石皆就擒。世祖自將過烏紀嶺,至窩謀海村,胡論加古部勝昆勃堇居,烏延部富者郭赧請分一軍由所部伐烏春,蓋以所部與烏春近,欲以自蔽故也。乃使斜列、躍盤以支軍道其所居,世祖自將大軍與歡都合。
 至阿不塞水,嶺東諸部皆會,石土門亦以所部兵來。



 是時,烏春前死,窩謀罕聞知世祖來伐,訴於遼人,乞與和解。使者已至其家,世祖軍至,窩謀罕請緩師,盡以前所納亡人歸之。世祖使烏林荅故德黑勃堇往受所遣亡者。窩謀罕以三百騎乘懈來攻,世祖敗之。遼使惡其無信,不復為主和,乃進軍圍之。太祖衣短甲行圍,號令諸軍,窩謀罕使太峪潛出城攻之。太峪馳馬援槍,將及太祖,活臘胡擊斷其槍,太祖乃得免。斜列至斜寸水,用郭赧計,取先在烏春軍者二十二人。烏春軍覺之,殺二人,餘二十人皆得之,益以土軍來助。窩謀罕自知不敵,乃
 遁去。遂克其城,盡以貲產分賚軍中,以功為次,諸部皆安輯焉。穆宗常嘉郭赧功,後以斜列之女守寧妻其子胡里罕。



 烏春之後為溫敦氏,裔孫曰蒲刺。



 溫敦蒲刺始居長白山阿不辛河,徙隆州移里閔河。蒲刺初從希尹征伐,攝猛安謀克事,遇賊突出,力擊敗之,手殺二十餘人,用是擢修武校尉。天德初,充護衛,遷宿直將軍,與眾護衛射遠,皆莫能及,海陵以玉鞍、銜賞之。往曷懶路選可充護衛者,使還稱旨,遷耶盧椀群牧使,改遼州刺史。正隆伐宋,召為武翼軍副都總管,將兵二千,至汝州南,遇宋兵二萬餘,邀擊敗之,手殺將士十餘
 人。是時,嵩、汝兩州百姓多逃去,蒲刺招集,使之復其業。改莫州刺史,徵為太子左衛率府率,再遷隴州防禦使,歷鎮西、胡里改、顯德軍節度使。致仕,卒。



 臘醅、麻產兄弟者,活刺渾水訶鄰鄉紇石烈部人。兄弟七人,素有名聲,人推服之。及烏春、窩謀罕等為難,故臘醅兄弟乘此際結陶溫水之民,浸不可制。其同里中有避之者,徙於苾罕村野居女直中,臘醅怒,將攻之,乃約烏古論部騷臘勃堇、富者撻懶、胡什滿勃堇、海羅勃堇、斡茁火勃堇。海羅、斡茁火間使人告野居女直,野居女直有備,臘醅等敗歸。臘醅乃由南路復襲野居女直,勝
 之,俘略甚眾。海羅、斡茁火、胡什滿畏臘醅,求援于世祖。斜列以輕兵邀擊臘醅等於屯睦吐村,敗之,盡得所俘。



 臘醅、麻產驅掠來流水牧馬。世祖至混同江,與穆宗分軍。世祖自妒骨魯津倍道兼行,馬多乏,皆留之路傍,從五六十騎,遇臘醅于野鵲水。日已曛,臘醅兵眾,世祖兵少,歡都鏖戰,出入數四,馬中創,死者十數。世祖突陣力戰,中四創,不能軍。穆宗自庵吐渾津度江,遇敵于蒲盧買水。敵問為誰,應之曰:「歡都。」問者射穆宗,矢著於弓箙。是歲,臘醅、麻產使其徒舊賊禿罕及馳朵剽取戶魯不濼牧馬四百,及富者粘罕之馬合七百餘匹,過青嶺東,
 與烏春、窩謀罕交結。世祖自將伐之,臘醅等偽降,還軍。臘醅復求助於烏春、窩謀罕。窩謀罕以姑里甸兵百有十七人助之。臘醅據暮棱水,保固險阻,石顯子婆諸刊亦往從之。世祖率兵圍之,克其軍,麻產遁去,遂擒臘醅及婆諸刊,皆獻之遼。盡獲其兵,使其卒長斡善、斡脫招撫其眾,使斜缽撫定之。復使阿離合懣察暮棱水人情,並募兵與斜缽合,語在《烏春傳》。



 世祖既沒,肅宗襲節度使。麻產據直屋鎧水,繕完營堡,招納亡命,杜絕往來者。恃陶溫水民為之助,招之不聽,使康宗伐之。是歲,白山混同江大溢,水與岸齊,康宗自阿鄰岡乘舟至於帥水,
 舍舟沿帥水而進。使太祖從東路取麻產家屬,盡獲之。康宗圍麻產急,太祖來會軍,於是麻產先亡在外,其人乘夜突圍遁去。太祖曰:「麻產之家蕩盡矣,走將安歸。」追之。麻產不知太祖急求己也,與三騎來伺軍,其一人墜馬下,太祖識之,問狀。其人曰:「我隨麻產來伺軍,彼走者二人,麻產在焉。」麻產與其人分道走,太祖命劾魯古追東走者,而自追西走者。至直屋鎧水,失麻產不見,急追之,得遺甲於路,迹而往,前至大澤,濘淖。麻產棄馬入萑葦,太祖亦棄馬追及之,與之挑戰。烏古論壯士活臘胡乘馬來,問曰:「此何人也。」太祖初不識麻產,佯應曰:「麻產
 也。」活臘胡曰:「今亦追及此人邪。」遂下馬援槍進戰。麻產連射活臘胡,活臘胡中二矢,不能戰。有頃,軍至,圍之。歡都射中麻產首,遂擒之。無有識之者,活臘胡乃前扶其首而視之,見其鹵豁,曰:「真麻產也。」麻產張目曰:「公等事定矣。」遂殺之。太祖獻馘於遼。



 鈍恩,阿里民忒石水紇石烈部人。祖曰劾魯古,父納根涅,世為其部勃堇。斡准部人冶刺勃堇、海葛安勃堇暴其族人斡達罕勃堇及諸弟屋里黑、屋徒門,抄略其家,及抄略阿活里勃堇家,侵及納根涅所部。穆宗使納根涅以本部兵往治冶刺等。行至蘇濱水,輒募人為兵,主
 者拒之,輒抄略其人。遂攻烏古論部敵庫德,入米里迷石罕城。及斡賽、冶訶來問狀,止蘇濱水西納木汗村,納根涅止蘇濱水東屋邁村。納根涅雖款伏而不肯征償,時甲戌歲十月也。明年八月,納根涅遁去,斡賽追而殺之,執其母及其妻子以歸,而使鈍恩復其所。



 留可,統門、渾蠢水合流之地烏古論部人,忽沙渾勃堇之子。詐都,渾蠢水安春之子也。間誘奧純、塢塔兩部之民作亂。敵庫德、鈍恩皆叛而與留可、詐都合。兩黨揚言曰:「徒單部之黨十四部為一,烏古論部之黨十四部為一,蒲察部之黨七部為一,凡三十五部。完顏
 部十二而已,以三十五部戰十二部,三人戰一人也,勝之必矣。」世祖降附諸部亦皆有離心。當是時,惟烏延部斜勒勃堇及統門水溫迪痕部阿里保勃堇、撒葛周勃堇等皆使人來告難。斜勒,達紀保之子也,先使其兄保骨臘來,既而以其甲來歸。阿里保等曰:「吾等必不從亂,但乞兵為援耳。」



 穆宗使撒改伐留可,使謾都訶伐敵庫德。既而太祖以七十甲詣撒改軍,中道以四十甲與謾都訶。石土門之軍與謾都訶會于米里迷石罕城下。而鈍恩將援留可,聞謾都訶之兵寡,以為無備,而未知石土門之來會也,欲先攻謾都訶。謾都訶、石土門迎擊,大
 破鈍恩。米里迷石罕城遂降,獲鈍恩、敵庫德,皆釋弗誅。太祖至撒改軍,明日遂攻破留可城,城中渠帥皆誅之,取其孥累貲產而還。塢塔城亦撒守備而降。留可先在遼,塢塔已脫身在外,由是皆未獲。詐都亦詣蒲家奴降,太祖釋之。於是,諸部皆安業如故。久之,留可、塢塔皆來降。



 阿疏,星顯水紇石烈部人。父阿海勃堇事景祖、世祖。世祖破烏春還,阿海率官屬士民迎謁于雙宜大濼,獻黃金五斗。世祖喻之曰:「烏春本微賤,吾父撫育之,使為部長,而忘大恩,乃結怨於我,遂成大亂,自取滅亡。吾與汝
 等三十部人之人,自今可以保安休息。吾大數亦將終。我死,汝等當念我,竭力以輔我子弟,若亂心一生,則滅亡如烏春矣。」阿海與眾跪而泣曰:「太師若有不諱,眾人賴誰以生,勿為此言。」未幾,世祖沒,阿海亦死,阿疏繼之。



 阿疏自其父時常以事來,昭肅皇后甚憐愛之,每至,必留月餘乃遣歸。阿疏既為勃堇,嘗與徒單部詐都勃堇爭長,肅宗治之,乃長阿疏。



 穆宗嗣節度,聞阿疏有異志,乃召阿疏賜以鞍馬,深加撫諭,陰察其意趣。阿疏歸,謀益甚,乃斥其事。復召之,阿疏不來,遂與同部毛睹祿勃董等起兵。



 穆宗自馬紀嶺出兵攻之。撒改自胡論嶺往略,
 定潺春、星顯兩路,攻下鈍恩城。穆宗略阿茶檜水,益募軍,至阿疏城。是日辰巳間,忽暴雨,晦曀,雷電下阿疏所居,既又有大光,聲如雷,墜阿疏城中。識者以謂破亡之徵。



 阿疏聞穆宗來,與其弟狄故保往訴於遼。遼人來止勿攻。穆宗不得已,留劾者勃堇守阿疏城而歸。金初亦有兩劾者,其一撒改父,贈韓國公。其一守阿疏城者,後贈特進云。



 劾者以兵守阿疏城者二年矣。阿疏在遼不敢歸,毛睹祿乃降。遼使復為阿疏來。穆宗聞之,使烏林荅石魯濟師,且戒劾者令易衣服旗幟與阿疏城中同色,使遼使不可辨。遼使至,乃使蒲察部胡魯勃堇、邈遜
 勃堇與俱至劾者軍,而軍中已易衣服旗幟,與阿疏城中如一,遼使果不能辨。劾者詭曰:「吾等白相攻,干汝何事,誰識汝之太師。」乃刺殺胡魯、邈遜所乘馬,遼使驚怖走去,遂破其城。狄故保先歸,殺之。



 阿疏聞穆宗以計卻遼使,破其城,殺狄故保,復訴於遼。遼使奚節度使乙烈來問狀,且使備償阿疏。穆宗復使主隈、禿荅水人偽阻絕鷹路者,而使鱉故德部節度使言於遼,平鷹路非己不可。遼人不察也,信之。穆宗畋於土溫水,謂遼人曰:「吾平鷹路也。」遼人以為功,使使來賞之。穆宗盡以其物與主隈、禿荅之人而不復備償阿疏。遼人亦不復問。



 阿疏
 在遼無所歸,後二年,使其徒達紀至生女直界上,曷懶甸人畏穆宗,執而送之,阿疏遂終于遼。



 及太祖伐遼,底遼之罪告于天地,而以阿疏亡命遼人不與為言,凡與遼往復書命必及之。天輔六年,闍母、婁室略定天德、雲內、寧邊、東勝等州,獲阿疏。軍士問之曰:「爾為誰?」曰:「我破遼鬼也。」



 贊曰:金之興也,有自來矣。世祖擒臘醅、婆諸刊,既獻之遼以為功,則又曰:「若不遣還,其部人疑懼,且為亂階。」遼人不察,盡以前後所獻罪人歸之。景祖止曷魯林牙、止同乾,穆宗止遼使阿疏城,始終以鷹路誤之,而遼人不
 悟。景祖有黃馬,服乘如意,景祖沒,遼貴人爭欲得之。世祖弗與,曰:「難未息也,馬不可以與人。」遂割其兩耳,謂之禿耳馬,遼貴人乃弗取。其前平諸部則借遼以為己重,既獻而求之則市以為己重。戰陣一良馬終弗與遼人,而遼人終不悟,豈興亡有數,蓋天奪其魄歟。



 奚,與契丹俱起,在元魏時號庫莫奚,歷宇文周、隋、唐,皆號兵強。其後契丹破走奚,奚西保冷陘,其留者臣服于契丹,號東、西奚。厥後遼太祖稱帝,諸部皆內屬矣。鐵勒者,古部族之號,奚有其地,號稱鐵勒州,又書作鐵驪州。奚有五王族,世與遼人為昏,因附姓述律氏中,事具《遼
 史》,今不載。



 奚有十三部、二十八落、一百一帳、三百六十二族。甲午歲,太祖破耶律謝十,諸將連戰皆捷,奚鐵驪王回離保以所部降,未幾,遁歸于遼。及遼主使使請和,太祖曰:「歸我叛人阿疏、降人回離保、迪里等,餘事徐議之。」久之,遼主至鴛鴦濼,都統杲襲之,亡走天德。



 回離保與遼大臣立秦晉國王耶律捏里于燕京。捏里死,蕭妃權國事。太祖入居庸關,蕭妃自古北口出奔。回離保至盧龍嶺,遂留不行,會諸奚吏民於越里部,僭稱帝,改元天復,改置官屬,籍渤海、奚、漢丁壯為軍。太祖詔回離保曰:「聞汝脅誘吏民,僭竊位號。遼主越在草莽,大福不再。
 汝之先世臣服于遼,今來臣屬,與昔何異。汝與余睹有隙,故難其來。余睹設有睚眥,朕豈從之。儻能速降,盡釋汝罪,仍俾主六部族,總山前奚眾,還其官屬財產。若尚執迷,遣兵致討,必不汝赦。」回離保不聽。天輔七年五月,回離保南寇燕地,敗於景、薊間,其眾奔潰。耶律奧古哲及甥八斤、家奴白底哥等殺之。其妻阿古聞之,自剄而死。



 先是,速古部人據劾山,奚路都統撻懶招之不服,往討之。鐵泥部眾扼險拒戰,殺之殆盡。至是,速古、啜里、鐵泥三部所據十三巖皆討平之。達魯古部節度使乙列已降復叛,奚馬和尚討達魯古並五院司等諸部,諸部
 皆降,遂執乙列,杖之一百,其父及其家人先被獲者皆還之。



 初,太祖破遼兵于達魯古城,九百奚營來降。至是,回離保死,奚人以次附屬,亦各置猛安謀克領之。



 贊曰:庫莫奚、契丹起於漢末,盛於隋、唐之間,俱彊為鄰國,合並為群臣,歷八百餘年,相為終始。奚有五,大定間,類族著姓有遙里氏、伯德氏、奧里氏、梅知氏、揣氏。



\end{pinyinscope}