\article{列傳第五十}

\begin{pinyinscope}

 ○完顏合達移剌蒲阿



 完顏合達,名瞻,字景山。少長兵間,習弓馬,能得人死力。貞祐初,以親衛軍送岐國公主,充護衛。三年,授臨潢府推官,權元帥右監軍。時臨潢避遷,與全、慶兩州之民共壁平州。合達隸其經略使烏林答乞住,乞住以便宜授軍中都統,累遷提控,佩金符。未幾,會燕南諸帥將兵復中都城,行至平州遷安縣,臨潢、全慶兩軍變,殺乞住,擁
 合達還平州,推為帥,統乞住軍。合達以計誅首亂者數人。其年六月,北兵大將喊得不遣監戰提軍至平州城下,以州人黃裳入城招降,父老不從,合達引兵逆戰,知事勢不敵,以本軍降於陣。監戰以合達北上,留半發,令還守平州。已而,謀自拔歸,乃遣奉先縣令紇石烈布里哥、北京教授蒲察胡里安、右三部檢法蒲察蒲女涉海來報。



 四年十一月,合達果率所部及州民並海西南歸國。詔進官三階,升鎮南軍節度使,駐益都,與元帥蒙古綱相應接,充宣差都提控。十二月,大元兵徇地博興、樂安、壽光,東涉濰州之境,蒙古綱遣合達率兵屢戰於壽
 光、臨淄。興定元年正月,轉通遠軍節度使、兼鞏州管內觀察使。七月,改平西軍節度使、兼河州管內觀察使。二年正月,知延安府事、兼鄜延路兵馬都總管。



 三年正月,詔伐宋,以合達為元帥右都監。三月,破宋兵於梅林關,擒統領張時。又敗宋兵於馬嶺堡,獲馬百匹。又拔麻城縣,獲其令張倜、乾辦官郭守紀。



 四月,夏人犯通秦寨,合達出兵安塞堡,抵隆州,夏人自城中出步騎二千逆戰,進兵擊之,斬首數十級,俘十人,遂攻隆州,陷其西南隅,會日暮乃還。六月,行元帥府事於唐、鄧,上遣諭曰:「以卿才幹,故委卿,無使敵人侵軼,第固吾圉可也。」四年正月,復
 為元帥右都監,屯延安。十月,夏人攻綏德州,駐兵于拄天山。合達將兵擊之,別遣先鋒提控樊澤等各率所部分三道以進,畢會於山顛。見夏人數萬餘傅山而陣,即縱兵分擊。澤先登,摧其左軍,諸將繼攻其右,敗之。五年五月,知延安府事,兼前職。上言:「諸軍官以屢徙,故往往不知所居地形迂直險易,緩急之際恐至敗事,自今乞勿徙。」又言:「河南、陜西鎮防軍皆分屯諸路,在營惟老稚而已。乞選老成人為各路統軍以鎮撫之,且督其子弟習騎射,將來可用。」皆從之。



 十一月,夏人攻安塞堡,其軍先至,合達與征行元帥納合買住禦之。合達策之曰:「比
 北方兵至,先破夏人則後易為力。」於是潛軍裹糧倍道兼進,夜襲其營,夏人果大潰,追殺四十里,墜崖谷死者不可勝計。上聞之,賜金各五十兩、重幣十端,且詔諭曰:「卿等克成大功,朕聞之良喜。經畫如此,彼當知畏,期之數年,卿等可以休息矣。」仍詔以合達之功遍諭河南帥臣。是月,與元帥買住又戰延安,皆被重創。十二月,以保延安功賜金帶一、玉吐鶻一,重幣十端。



 元光元年正月,遷元帥左監軍,授山東西路吾改必剌世襲謀克。權參知政事,行省事於京兆。未幾,真拜。是年五月,上言:「頃河中安撫司報,北將按察兒率兵入隰、吉、翼州,浸及榮、解
 之境,今時已暑,猶無回意,蓋將蹂吾禾麥。倘如此,則河東之土非吾有也。又河南、陜西調度仰給解鹽,今正漉鹽之時,而敵擾之,將失其利。乞速濟師,臣已擬分兵二萬,與平陽、上黨、晉陽三公府兵同力禦之。竊見河中、榮、解司縣官與軍民多不相諳,守禦之間或失事機。乞從舊法,凡司縣官使兼軍民,庶幾上下相得,易以集事。」又言鹽利,「今方敵兵迫境,不厚以分人,孰肯冒險而取之?若自輸運者十與其八,則人爭赴以濟國用。」從之。



 葭州提控王公佐言於合達曰:「去歲十月,北兵既破葭州,構浮梁河上。公佐寓州治北石山子,招集餘眾得二千餘
 人,欲復州城。以士卒皆自北逃歸者,且無鎧仗,故嘗請兵於帥府,將焚其浮橋,以取葭州,帥府不聽。又請兵援護老幼稍徙內地,而帥府亦不應。今葭州之民迫於敵境,皆有動搖之心。若是秋敵騎復來,則公佐力屈死於敵手,而遺民亦俱屠矣。」合達乃上言:「臣願馳至延安,與元帥買住議,以兵護公佐軍民來屯吳堡,伺隙而動。」詔省院議之,於是命合達率兵取葭州。行至鄜州,千戶張子政等殺萬戶陳紋,將掠城中。合達已勒兵為備,子政等乃出城走,合達追及之,眾復來歸,斬首惡數十人,軍乃定。



 六月,合達上言:「累獲諜者,皆云北方已約夏人,將
 由河中、葭州以入陜西。防秋在近,宜預為計。今陜西重兵兩行省分制之,然京兆抵平涼六百餘里,萬一敵梗其間,,使不得通,是自孤也。宜令平涼行省內族白撒領軍東下,與臣協力禦敵,以屏潼、陜,敵退後復議分司為便。」詔許之。二年二月,以保鳳翔之功進官,賜金幣及通犀帶一。是時,河中已破,合達提兵復取之。



 正大二年七月,陜西旱甚,合達齋戒請雨,雨澍,是歲大稔,民立石頌德。延安既殘毀,合達令於西路買牛付主者,招集散亡,助其耕墾,自是延安之民稍復耕稼之利。八月,鞏州田瑞反,合達討之,諸軍進攻,合達移文諭之曰:「罪止田瑞
 一身,餘無所問。」不數日,瑞弟濟殺瑞以降,合達如約撫定一州,民賴以寧。三年,詔遷平涼行省。四年二月,徵還,拜平章政事,芮國公。七年七月庚寅朔,以平章政事妨職樞密副使。初,蒲阿面奏:「合達在軍中久,今日多事之際乃在於省,用違其長。臣等欲與樞密協力軍務,擢之相位似亦未晚。」故有此授。



 十月己未朔,詔合達及樞密副使蒲阿救衛州。初,朝廷以恒山公仙屯衛州,公府節制不一,欲合而一之。至是,河朔諸軍圍衛,內外不通已連月,但見塔上時舉火而已。合達等既至,先以親衛兵三千嘗之,北兵小退,翼日圍解。上登承天門犒軍,皆授
 世襲謀克,賜良馬玉帶,全給月俸本色,蓋異恩也。



 未幾,以蒲阿權參和政事,同合達行省事於閿鄉,以備潼關。先是,陜省言備禦策,朝官集議,上策親征,中策幸陜,下策棄秦保潼關。議者謂止可助陜西軍以決一戰,使陜西不守,河南亦不可保。至是,自陜以西亦不守矣。



 八年正月,北帥速不泬攻破小關,殘盧氏、朱陽,散漫百餘里間。潼關總帥納合買住率夾谷移迪烈、都尉高英拒之,求救地二省。省以陳和尚忠孝軍一千,都尉夾谷澤軍一萬往應,北軍退,追至谷口而還。兩省輒稱大捷,以聞。既而北軍攻風翔,二省提兵出關二十里,與渭北軍交,
 至晚復收兵入關,鳳翔遂破。二省遂棄京兆,與牙古塔起遷居民於河南,留慶山奴守之。九月,北兵入河中,時二相防秋還陜,量以軍馬出冷水谷以為聲援。



 十一月,鄧州報,北兵道饒峰關,由金州而東。於是,兩省軍入鄧,遣提控劉天山以劄付下襄陽制置司,約同禦北兵,且索軍食。兩省以前月癸卯行,留楊沃衍軍守閿鄉。沃衍尋被旨取洛南路入商州,屯豐陽川備上津,與恒山公仙相掎角。合達復留禦侮中郎將完顏陳和尚於閿鄉南十五里,乃行。陳和尚亦隨而往。沃衍軍八千及商州之木瓜平,一日夜馳三百里入桃花堡,知北兵由豐陽
 而東,亦東還,會大軍於鎮平。恒山公仙萬人元駐胡陵關,至是亦由荊子口、順陽來會。十二月朔,俱至鄧下,屯順陽。乃遣天山入宋。



 初,宋人於國朝君之、伯之、叔之,納歲幣將百年。南渡以後,宋以我為不足慮,絕不往來。故宣宗南伐,士馬折耗十不一存,雖攻陷淮上數州,徒使驕將悍卒恣其殺虜、飽其私欲而已。又宣徽使奧敦阿虎使北方,北中大臣有以輿地圖指示之曰:「商州至此中軍馬幾何?」又指興元云:「我不從商州,則取興元路入汝界矣。」阿虎還奏,宣宗甚憂之。哀宗即位,群臣建言,可因國喪遣使報哀,副以遺留物,因與之講解,盡撤邊備,
 共守武休之險。遂下省院議之,而當國者有仰而不能俯之疾,皆以朝廷先遣人則於國體有虧為辭。元年,上諭南鄙諸帥,遣人往滁州與宋通好。宋人每以奏稟為辭,和事遂不講。然十年之間,朝廷屢敕邊將不妄侵掠,彼我稍得休息,宋人始信之,遂有繼好之意。及天山以劄付至宋,劄付者指揮之別名,宋制使陳該怒辱天山,且以惡語復之。報至,識者皆為竊嘆。



 戊辰,北兵渡漢江而北,諸將以為可乘其半渡擊之,蒲阿不從。丙子,兵畢渡,戰於禹山之前,北兵小卻,營於三十里之外。二相以大捷驛報,百官表賀,諸相置酒省中,左丞李蹊且喜且
 泣曰:「非今日之捷,生靈之禍,可勝言哉!」蓋以為實然也。先是,河南聞北兵出饒峰,百姓往往入城壁、保險固,及聞敵已退,至有晏然不動者,不二三日游騎至,人無所逃,悉為捷書所誤。



 九年正月丁酉,兩省軍潰於陽翟之三峰山。初,禹山之戰,兩軍相拒,北軍散漫而北,金軍懼其乘虛襲京城,乃謀入援。時北兵遣三千騎趨河上,已二十餘日,泌陽、南陽、方城、襄、郟至京諸縣皆破,所有積聚焚毀無餘。金軍由鄧而東,無所仰給,乃並山入陽翟。既行,北兵即襲之,且行且戰,北兵傷折亦多。恒山一軍為突騎三千所衝,軍殊死鬥,北騎退走。追奔之際,忽大
 霧四塞,兩省命收軍。少之,霧散乃前,前一大澗,長闊數里,非此霧則北兵人馬滿中矣。明日,至三峰山,遂潰,事載蒲阿傳。合達知大事已去,欲下馬戰,而蒲阿已失所在。合達以數百騎走鈞州,北兵塹其城外攻之,走門不得出,匿窟室中,城破,北兵發而殺之。時朝廷不知其死,或云已走京兆,賜以手詔,募人訪之。及攻汴,乃揚言曰:「汝家所恃,惟黃河與合達耳。今合達為我殺,黃河為我有,不降何待?」



 合達熟知敵情,習於行陣,且重義輕財,與下同甘苦,有俘獲即分給,遇敵則身先之而不避,眾亦樂為之用,其為人亦可知矣。左丞張行信嘗薦之曰:「完
 顏合達,今之良將也。」



 移剌蒲阿,本契丹人,少從軍,以勞自千戶遷都統。初,哀宗為皇太子,控制樞密院,選充親衛軍總領,佩金符。元光二年冬十二月庚寅,宣宗疾大漸,皇太子異母兄英王守純先入侍疾,太子自東宮扣門求見,令蒲阿衷甲聚兵屯於艮嶽,以備非常。哀宗即位,嘗謂近臣言:「向非蒲阿,何至於此。」遂自遙授同知睢州軍州事,權樞密院判官,自是軍國大計多從決之。



 正大四年十二月,河朔軍突入商州,殘朱陽、盧氏。蒲阿逆戰至靈寶東,遇游騎十餘,獲一人,餘即退,蒲阿輒以捷聞。賞世襲謀克,仍厚
 賜之。人共知其罔上,而無敢言,吏部郎中楊居仁以微言取怒。



 六年二月丙辰,以蒲阿權樞密副使。自去年夏,北軍之在陜西者駸駸至涇州,且阻慶陽糧道。蒲阿奏:「陜西設兩行省,本以籓衛河南,今北軍之來三年於茲,行省統軍馬二三十萬,未嘗對壘,亦未嘗得一折箭,何用行省。」院官亦俱奏將來須用密院軍馬勾當,上不語者久之。是後,以丞相賽不行尚書省事於關中,召平章政事合達還朝,白撒亦召至闕,蒲阿率完顏陳和尚忠孝軍一千駐邠州,且令觀北勢。八月丙申,蒲阿再復潞州。十月乙未朔,蒲阿東還。



 十二月乙未,詔蒲阿與總帥
 牙吾塔、權簽樞密院事訛可救慶陽。七年正月,戰北兵於大昌原,北軍還,慶陽圍解。詔以訛可屯邠州,蒲阿、牙吾塔還京兆。未幾,以權參知政事與合達行省于閿鄉。八年正月,北軍入陜西,鳳翔破,兩行省棄京兆而東,至洛陽驛,被召議河中事,語在白華傳。



 十二月,北兵濟自漢江,兩省軍入鄧州,議敵所從出,謂由光化截江戰為便,放之渡而戰為便、張惠以「截江為便,縱之渡,我腹空虛,能不為所潰乎?」蒲阿麾之曰:「汝但知南事,於北事何知。我向於裕州得制旨云,『使彼在沙磧,且當往求之』,況今自來乎。汝等更勿似大昌原、舊衛州、扇車回縱出之。」
 定住、高、樊皆謂蒲阿此言為然。合達乃問按得木,木以為不然。軍中以木北人,知其軍情,此言為有理,然不能奪蒲阿之議。



 順陽留二十日,光化探騎至,云「千騎已北渡」,兩省是夜進軍,比曉至禹山,探者續云「北騎已盡濟」。癸酉,北軍將近,兩省立軍高山,各分據地勢,步迎於山前,騎屯於山後。甲戌,日未出,北兵至,大帥以兩小旗前導來觀,觀竟不前,散如雁翅,轉山麓出騎兵之後,分三隊而進,輜重外餘二萬人。合達令諸軍,「觀今日事勢,不當戰,且待之。」俄而北騎突前,金兵不得不戰,至以短兵相接,戰三交,北騎少退。北兵之在西者望蒲阿親繞甲
 騎後而突之,至於三,為蒲察定住力拒而退。大帥以旗聚諸將,議良久。合達知北兵意向。時高英軍方北顧,而北兵出其背擁之,英軍動,合達幾斬英,英復督軍力戰。北兵稍卻觀變,英軍定,復擁樊澤軍,合達斬一千夫長,軍殊死鬥,乃卻之。



 北兵回陣,南向來路。兩省復議:「彼雖號三萬,而輜重三之一焉。又相持二三日不得食,乘其卻退當擁之。」張惠主此議,蒲阿言:「江路已絕,黃河不冰,彼入重地,將安歸乎?何以速為。」不從。乙亥,北兵忽不知所在,營火寂無一耗。兩省及諸將議,四日不見軍,又不見營,鄧州津送及路人不絕,而亦無見者,豈南渡而歸
 乎?己卯,邏騎乃知北軍在光化對岸棗林中,晝作食,夜不下馬,望林中往來,不五六十步而不聞音響,其有謀可知矣。



 初,禹山戰罷,有二騎迷入營,問之,知北兵凡七頭項,大將統之。復有詐降者十人,弊衣羸馬泣訴艱苦,兩省信之,易以肥馬,飲之酒,及煖衣食而置之陣後,十人者皆鞭馬而去,始悟其為覘騎也。



 庚辰,兩省議入鄧就糧,辰巳間到林後,北兵忽來突,兩省軍迎擊,交綏之際,北兵以百騎邀輜重而去,金兵幾不成列,逮夜乃入城,懼軍士迷路,鳴鐘招之。樊澤屯城西,高英屯城東。九年正月壬午朔,耀兵於鄧城下。北兵不與戰,大將使來
 索酒,兩省與之二十瓶。癸未,大軍發鄧州,趨京師,騎二萬,步十三萬,騎帥蒲察定住,蒲察答吉卜,郎將按忒木,忠孝軍總領夾谷愛答、內族達魯歡,總領夾谷移特剌,提控步軍臨淄郡王張惠,殄寇都尉完顏阿排、高英、樊澤,中軍陳和尚,與恒山公武仙、楊沃衍軍合。是日,次五朵山下,取鴉路,北兵以三千騎尾之,遂駐營待楊武。



 楊武至,知申、裕兩州已降。七日至夜,議北騎明日當復襲我,彼止騎三千,而我示以弱,將為所輕,當與之戰。乃伏騎五十於鄧州道。明日軍行,北騎襲之如故,金以萬人擁之而東,伏發,北兵南避。是日雨,宿竹林中。庚寅,頓安
 皋。辛卯,宿鴉路、魯山。河西軍已獻申、裕,擁老幼牛羊取鴉路,金軍適值之,奪其牛羊餉軍。



 癸巳,望鈞州,至沙河,北騎五千待於河北,金軍奪橋以過,北軍即西首斂避。金軍縱擊,北軍不戰,復南渡沙河。金軍欲盤營,北軍復渡河來襲。金軍不能得食,又不得休息。合昏,雨作,明旦變雪。北兵增及萬人,且行且戰,致黃榆店,望鈞州二十五里,雨雪不能進,盤營三日。丙申,一近侍入軍中傳旨,集諸帥聽處分,制旨云:「兩省軍悉赴京師,我御門犒軍,換易御馬,然後出戰未晚。」復有密旨云:「近知張家灣透漏二三百騎,已遷衛、孟兩州,兩省當常切防備。」領旨訖,蒲
 阿拂袖而起,合達欲再議,蒲阿言:「止此而已,復何所議。」蓋已奪魄矣。軍即行。



 北軍自北渡者畢集,前後以大樹塞其軍路,沃衍軍奪路,得之。合達又議陳和尚先擁山上大勢,比再整頓,金軍已接竹林,去鈞州止十餘里矣。金軍遂進,北軍果卻三峰之東北、西南。武、高前鋒擁其西南,楊、樊擁其東北,北兵俱卻,止有三峰之東。張惠、按得林立山上望北兵二三十萬,約厚二十里。按得木與張惠謀曰:「此地不戰,欲何為耶?」乃率騎兵萬餘乘上而下擁之,北兵卻。須臾雪大作,白霧蔽空,人不相覿。時雪已三日,戰地多麻田,往往耕四五過,人馬所踐泥淖
 沒脛。軍士被甲骨僵立雪中,槍槊結凍如椽,軍士有不食至三日者。北兵與河北軍合,四外圍之,熾薪燔牛羊肉,更遞休息。乘金困憊,乃開鈞州路縱之走,而以生軍夾擊之。金軍遂潰,聲如崩山,忽天氣開霽,日光皎然,金軍無一人得逃者。



 武仙率三十騎入竹林中,楊、樊、張三軍爭路,北兵圍之數重,及高英殘兵共戰於柿林村南,沃衍、澤、英皆死,惟張惠步持大槍奮戰而歿。蒲阿走京師,未至,追及,擒之。七月,械至官山,召問降否,往復數百言,但曰:「我金國大臣,惟當金國境內死耳。」遂見殺。



 贊曰:金自南渡,用兵克捷之功,史不絕書,然而地不加
 闢,殺傷相當,君子疑之。異時伐宋,唐州之役,喪師七百,主將訛論匿之,而以捷聞。御史納蘭糾之,宣宗獎御史,而不罪訛論,是君臣相率而為虛聲也。禹山之捷,兩省為欺,遂致誤國,豈非宣宗前事有以啟之耶?至於三峰山之敗,不可收拾,上下咢眙,而金事已去十九。天朝取道襄、漢,懸軍深入,機權若神,又獲天助,用能犯兵家之所忌,以建萬世之俊功,合達雖良將,何足以當之。蒲阿無謀,獨以一死無愧,猶足取焉爾。



\end{pinyinscope}