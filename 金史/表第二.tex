\article{表第二}

\begin{pinyinscope}

 交聘表上



 天下之勢,曷有常哉。金人日尋干戈,撫制諸郡,保其疆圉,以求逞志於遼也,豈一日哉。及太祖再乘勝,已即帝位,遼乃招之使降,是猶能蒸虎變,欲誰何而止之。厥後使者八九往反,終不能定約束,何者,取天下者不徇小節,成算既定矣,終不為卑辭厚禮而輟攻。



 遼人過計,宋人亦過計,海上之書曰:「克遼之後,五代時陷入契丹漢
 地願畀下邑。」此何計之過也。血刃相向百戰而得之,卑辭厚幣以求之,難得而易與人,豈人之情哉。宋之失計有三,撤三關故塞不能固燕山塞,汴京城下之盟竭公私之帑以約質,立染楚而不力戰而江左稱臣。金人豈愛宋人而為和哉!策既失矣,名既屈矣,假使高宗立歸德,不得河北,可保河南、山東,不然,亦不失為晉元帝,其孰能亡之。金不能奄有四海,而宋人以尊稱與之,是誰強之邪。



 金人出于高麗,始通好為敵國,後稱臣。夏國始稱臣,末年為兄弟,於其國自為帝。宋於金初或以臣禮稱「表」,終以姪禮往復稱「書」。故識其通好與間有兵爭之
 歲,其盛衰大指可觀也已。使者或書本階,或用借授,兩國各因舊史,不必強同雲。



 表略



\end{pinyinscope}