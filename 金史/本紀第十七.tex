\article{本紀第十七}

\begin{pinyinscope}

 哀宗上



 哀宗諱守緒,初諱守禮,又諱寧甲速,宣宗第三子。母曰明惠皇后王氏,賜姓溫敦氏,仁聖皇后之女兄也。承安三年八月二十三日生於翼邸,仁聖無子,養為己子。泰和中,授金紫光祿大夫。宣宗登極,進封遂王,授秘書監,改樞密使。貞祐初,莊獻太子守忠薨,立皇孫鏗為皇太孫,尋又薨。四年正月己卯,立守禮為皇太子,仍控制樞
 密院事,詔略曰:「子以母貴,遂王守禮地鄰冢嫡,慶集元妃,立為皇太子,其典禮有司條具以聞。」四月甲午,用太子少保張行信言,更賜名守緒。元光二年十二月庚寅,宣宗崩。辛卯,奉遺詔即皇帝位于柩前。壬辰,詔大赦,略曰:「朕述先帝之遺意,有便於時欲行而未及者,悉奉而行之。國家已有定制,有司往往以情破法,使人罔遭刑憲,今後有本條而不遵者,以故入人罪罪之。草澤士庶,許令直言軍國利害,雖涉譏諷無可採取者,並不坐罪。」



 正大元年春正月戊戌朔,詔改元正大。庚子,上居廬,百官始奏事。祕書監、權吏部侍郎蒲察合住改恒州刺史,
 左司員外郎泥旁古華山同知楨州軍州事,逐二姦臣,大夫士相賀。邠州節度使移剌術納阿卜貢白兔,詔曰:「得賢臣輔佐,年穀豐登,此上瑞也,焉事此為。令有司給道里費,從之本土。禮部其遍諭四方,使知朕意。」丁巳,詔朝臣議修復河中府。禮部尚書趙秉文、太常卿楊雲翼等言,陜西民方疲敝,未堪力役。遂止。戊午,上始視朝。大司農、守汝州防禦使李蹊為太常卿,權參知政事。平章政事荊王守純罷,判睦親府。參知政事僕散五斤罷,充大行山陵使。尊皇后溫敦氏、元妃溫敦氏皆為皇太后,號其宮一曰仁聖,一曰慈聖。百官入賀于隆德殿。是日,
 大風飄端門瓦。赤盞合喜權樞密副使。有男子服麻衣,望承天門且笑且哭。詰之,則曰:「吾笑,笑將相無人;吾哭,哭金國將亡。」群臣請置重典,上持不可,曰:「近詔草澤諸人直言,雖涉譏訕不坐。」法司唯以君門非笑哭之所,重杖而遣之。南陽民布陳謀反,伏誅。



 三月,熒惑犯左執法。戊申,奉安宣宗御容于孝嚴寺。辛亥,丞相高汝礪薨。癸丑,葬宣宗于德陵。甲寅。起復邠州節度使致仕張行信為尚書左丞。以延安帥臣完顏合達戰禦有功,授金虎符,權參知政事,行尚書省事于京兆,兼統河東兩路。夏四月癸酉,宣宗祔廟,大赦中外。熒惑犯右執法。



 五月戊
 戌,平章政事把胡魯薨。癸卯,樞密副使完顏賽不為平章政事,權參知政事石盞尉忻為尚書右丞,太常卿李蹊為翰林承旨,仍權參政。甲辰,賜策論進士孛術論長河以下十餘人及第,經義進士張介以下五人及第。戊申,賜詞賦進士王鶚以下五十人及第。詔刑部,登聞檢、鼓院,毋銷閉防護,聽有冤者陳訴。



 六月甲戌,宰執請擊鞠,上以心喪不許,辛卯,立妃徒單氏為皇后。遣樞密判官移剌蒲阿率兵至光州,榜諭宋界軍民更不南伐。秋七月己亥,詔諭百官各勤乃職。癸卯,補修大樂。



 九月,樞密判官移剌蒲阿復澤、潞、獲馬千疋。冬十月戊午,夏國遣
 使來修好。



 十二月乙巳,恆州刺史蒲察合住有罪,伏誅。甲寅,宣宗小祥,燒飯于德陵。改定辟舉縣令法,以六事課縣令。京東、西、南,陜西設大司農司,兼採訪公事,京師大司農總之。左丞張行信言:「先帝詔國內,刑不上大夫,治以廉恥。丞相高琪所定職官犯罪的決百餘條,乞改依舊制。」上不欲彰先帝之過,略施行之。



 二年春正月甲申,有黃黑之昆。夏四月辛卯朔,恒山公武仙自真定府來奔。起復平章政事致仕莘國公胥鼎為平章政事,行省事于衛州,進封英國公。甲午,以京畿旱,遣使慮囚。鈞、許州大雨雹。丁酉,宿、鄭州雨傷麥。



 五月丁丑,以旱甚責
 己,避正殿,減常膳,赦罪。蘇椿自大名來奔,詔置椿許州。秋七月,都水蒲察毛花輦殺人,免死除名。



 八月,鞏州元帥田瑞反,行省軍圍之,其母弟十哥殺瑞出降,赦其罪,以為涇州節度使,世襲猛安。



 九月,夏國和議定,以兄事金,各用本國年號,遣使來聘,奉國書稱弟。冬十月,以夏國修好,詔中外。新軍政改總領為都尉。己酉,以誅田瑞詔中外。癸亥,遣禮部尚書奧敦良弼、大理卿裴滿欽甫、侍御史烏古孫弘毅為夏國報成使,國書稱兄。乙亥,面諭臺諫完顏素蘭、陳規曰:「宋人輕犯邊界,我以輕騎襲之,冀其懲創通好,以息吾民耳。夏人從來臣屬我朝,今
 稱弟以和,我尚不以為辱。果得和好,以安吾民,尚欲用兵乎。卿等宜悉朕意。」移剌蒲阿及宋人戰于光州,獲馬數千,數人千餘而還。內族王家奴故殺鮮于主簿,權貴多救之者,上曰:「英王朕兄,敢妄撻一人乎?朕為人主,敢以無罪害一人乎?國家衰弱之際,生靈有幾何,而族子恃殺一主簿,吾民無主矣。」特命斬之。詔有司為死節士十有三人立褒忠廟。禁宿、泗、青口巡邊官兵,毋復擅殺過淮紅衲軍。詔趙秉文、楊雲翼作《龜鏡萬年錄》。



 三年春正月丁巳朔,夏國遣使來賀。



 三月,陜西旱。平章政事胥鼎復請致仕,不許。詔尚書省議省減用度。夏四
 月辛卯,親享于太廟。郕國夫人車經御路,過廟前,馭者乘馬,二婢坐車中,俱不下,詔繫獄杖之。辛丑,以旱,遣官禱于濟瀆。癸卯,祈于太廟。禁傘扇。河南大雨雹。己酉,遣使慮囚,遣使捕蝗。



 五月己未,大雨。宋兵掠壽州境。癸亥,永州桃園軍失利,死者四百人。乙丑,大雨。壬申,詔諭庾州趙甫等,能以土地來歸,當任使之。



 六月辛卯,京東大雨雹,蝗盡死。壬子,詔諭高麗及遼東行省葛不靄,討反賊萬奴,赦脅從者。秋七月庚午,平章政事英國公胥鼎薨。



 八月,移剌蒲阿復曲沃及晉安。辛卯,詔設益政院于內廷,以禮部尚書楊雲翼等為益政院說書官,日二
 人直,備顧問。冬十月丁酉,夏使來報哀。



 十一月庚申,議與宋修好。戊辰,又議之。己巳,宋忠義軍夏全自楚州來歸,楚州王義深、張惠、范成進以城降,封四人為郡王。辛未,改楚州為平淮府,以夏全等來降,赦諸路從宋及淮、楚官吏軍民,並其家屬。甲戌,遣使夏國賀正旦。丙子,夏以兵事方殷來報,各停使聘。大元兵征西夏,平中興府。召陜西行省及陜州總帥完顏訛可、靈寶總帥紇石烈牙吾塔赴汴議兵事。詔諭陜西兩省,凡戎事三品以下官聽以功過賞罰之,銀二十五萬兩從其給賞。遣中奉大夫完顏履信等為弔祭夏國使。



 四年春正月辛亥朔。壬戌,增築中京城,浚汴城外濠。



 二月,蒲阿、牙吾塔復平陽,執知府李七斤,獲馬八千。



 三月,簽勞效官充軍,有怨言,不果用。以銀贖平陽虜獲男女,分賜官軍者聽自便。大元兵平德順府,節度使愛申、攝府判馬肩龍死之。大元兵復下平陽。己巳,征夏稅二倍。夏五月丁丑,議乞和於大元。大元兵平臨洮府,總管陀滿胡土門死之。陜西行省進三策:上策自將出戰,中策幸陜州,下策棄秦保潼關。不從。



 六月戊申朔,遣前御史大夫完顏合周為議和使。丙辰,地震。太白入井。賜詞賦經義盧亞以下進士第。秋七月,大元兵自鳳翔徇京兆,
 關中大震。工部尚書師安石為尚書右丞。壬辰,以中丞烏古孫卜吉、祭酒裴滿阿虎帶兼司農卿,簽民軍,勸率富民入保城聚,兼督秋稅,令百姓知避遷之計。丁酉,赦陜西東、西兩路,賜民今年租。



 八月庚戌,詔有司罷遣防備丁壯、修城民夫,軍須差發應不急者權停。己巳,萬年節,同知集賢院史公奕進《大定遺訓》,待制呂造進《尚書要略》。是日,大風落左掖門鴟尾,壞丹鳳扉。隕霜,禾盡損。李全自都復入楚州據之,遣總帥完顏訛可、元帥慶山奴守盱眙,與全戰于龜山,敗績。冬十月辛酉,右拾遺李大節、右司諫陳規劾同判睦親府事撒合輦姦贓,
 不報。壬戌,外臺監察御史諫獵,上怒,以邀名賣直責之。詔贈德順府死事愛申、馬肩龍等官。以淮南王爵招李全。



 十一月乙未,未時,日上有二白虹貫之。丁酉,獵于近郊。



 十二月,真授李蹊參知政事。大元兵下商州。壬子,遣使安撫陜西,以牛千頭賜貧民。



 五年春正月丁丑,親祭三廟。庚辰,遣知開封府事完顏麻斤出如大元弔慰。丙戌,議擊盱眙。辛卯,以龜山之敗,降元帥慶山奴為定國軍節度使。



 二月乙巳朔,大寒,雷,雨雪,木之華者盡死。癸丑,詔有司以臨洮總管陀滿胡土門塑像入褒忠廟。書死節子孫於御屏,量材官使之。



 三月甲戌朔,群臣請依祖宗故事,樞密院聽尚書省節制,不從。乙酉,監察御史烏古論不魯剌劾近侍張文壽、張仁壽、李麟之受饋遺,曲赦其罪而出之。夏四月甲辰朔,以御史言三姦不已,凡四日不視朝。八日,議放還西夏人口。丙寅,右丞師安石薨。親衛軍王咬兒酗酒殺其孫,大理寺當以徒刑,特命斬之。



 五月癸巳,定國軍節度使慶山奴以受賂,奪一官。



 六月壬戌,以旱,赦雜犯罪死已下。秋七月戊子,同判睦親府事撒合輦出為中京留寧,行樞密院事。



 八月乙卯,以旱,遣使禱于上清宮。甲子,參知政事白撒為尚書右丞,太常卿顏盞世魯權參知
 政事。增築歸德行樞密院,擬工役數百萬,詔遣權樞密院判官白華喻以農夫勞苦,減其工三之二。以節制不一,併衛州帥府於恒山公府,命白華往經畫之。



 九月庚寅,雨足,始種麥。冬十一月辛巳,進宣《宣宗實錄》。



 十二月庚子朔,日有食之。完顏麻斤出以奉使不職,免死除名。壬子,完顏奴申改侍講學士,充國信使。以陜西大寒,賜軍士柴炭銀有差。京兆、鳳翔府司竹監進竹,令分給之。



 六年春二月丙辰,樞密院判官移剌蒲阿權樞密副使。耀州刺史李興有戰功,詔賜玉兔鶻帶、金器。以丞相完顏賽不行尚書省事于關中,召平章政事完顏合達還
 朝。移剌蒲阿率忠孝軍總領完顏陳和尚忠孝軍一千騎駐邠州。遣白華馳諭蒲阿以用兵之意。詔樞密更給忠孝軍馬疋,以漸調發都尉司步卒及忠孝馬軍屯京西。以白華專備軍須。



 三月乙亥,忠孝軍總領陳和尚有戰功,授定遠大將軍、平涼府判官,世襲謀克。夏五月,隴州防禦使石抹冬兒進黃鸚鵡,詔曰:「外方獻珍禽異獸,違物性,損人力,令勿復進。」秋七月,罷陜西行省軍中浮費。



 八月,移剌蒲阿再復澤、潞。



 九月,洮、河、蘭、會元帥顏盞蝦麻進西馬二疋,詔曰:「卿武藝超絕,此馬可充戰用,朕乘豈能盡其力。既入進,即尚廄物也,今以賜卿,其悉
 朕意。」冬十月,移剌蒲阿東還,令陳和尚率陜西歸順馬軍屯鈞、許。大元兵駐慶陽界。詔陜西行省遣使奉羊酒幣帛乞緩師請和。



 十一月,遣使鈞、許選試陜西歸順人,得軍二千,以藝優者充忠孝軍,次充合里合軍。



 十二月,詔副樞蒲阿、總帥紇石烈牙吾塔、權簽樞密院事完顏訛可救慶陽。罷附京獵地百里,聽民耕稼。



 七年春正月,副樞蒲阿、總帥牙吾塔、權簽院事訛可解慶陽之圍。以訛可屯邠州,蒲阿、牙吾塔還京兆。夏五月,詔釋清口宋敗軍三千人,願留者五百人,以屯許州,餘悉縱遣之。賜經義詞賦李瑭以下進士第。秋七月,以平
 章政事合達權樞密副使。



 八月,賜陜西死事孤鹽引及絹,仍量材任使。大元兵圍武仙于舊衛州。冬十月,平章合達、副樞蒲阿引兵救衛州。衛州圍解,上登承天門犒軍,合達、蒲阿並世襲謀克。移剌蒲阿權參政事,同合達行省事于閿鄉,以備潼關。



 八年春正月,大元兵圍鳳翔府。遣樞密院判官白華、右司郎中夾谷八里門諭閿鄉行省進兵,合達、蒲阿以未見機會不行。復遣白華諭合達、蒲達將兵出關以解鳳翔之圍,又不行。夏四月丁巳朔,赦。全免京西路軍錢一年。旱災州縣,差稅從實減貸。大元兵平鳳翔府。兩行
 省棄京兆,遷居民於河南,留慶山奴守之。



 五月,李全妻楊妙真以全陷沒于宋,構浮梁楚州北,欲復宋讎。遣合達、蒲阿屯桃源界滶河口,以備侵軼。宋八里莊人拒其主將納合達、蒲阿。詔改八里莊為鎮淮府。秋七月,宋將焚浮梁。



 九月丙申,慈聖宮皇太后溫敦氏崩,遣誥園陵制度務從儉約。大元兵駐河中府。慶山奴棄京兆東還。召合達、蒲阿赴汴,議引兵趨河中府,懼不敢行,還陜州,出師至冷水穀而歸。大元兵攻河中府,合達、蒲阿遣元帥王敢率兵萬人救之。冬十月,右丞相賽不致仕。



 十一月丁未,大元進兵嶢峰關,由金州而東。省院議以逸侍
 勞,未可與戰。上諭之曰:「南渡二十年,所在之民,破田宅,鬻妻子,竭肝腦以養軍。今兵至不能逆戰,止以自護,京城縱存,何以為國,天下其謂我何?朕思之熟矣,存與亡有天命,惟不負吾民可也。」乃詔諸將屯軍襄、鄧。



 十二月己未,葬明惠皇后。河中府破,權簽樞密院事草火訛可死之,元帥板子訛可提敗卒三千走閿鄉。詔赦將佐以下,杖訛可二百以死。合達、蒲阿率諸軍入鄧州,楊沃衍、陳和尚、武仙皆引兵來會。出屯順陽。戊辰,大元兵渡漢江而北,丙子,畢渡。合達、蒲阿將兵禦于禹山之前。大元兵分道趨汴京,京城戒嚴。是夜二鼓,合達、蒲阿引軍還
 鄧州。大元兵躡其後,盡獲其輜重。



 天興元年,是年本正大九年,正月改元開興,四月又改元天興。春正月壬午朔,日有兩珥。大元兵道唐州,元帥完顏兩婁室與戰襄城之汝墳,敗績。兩婁室走汴京。遣完顏麻斤出等部民丁萬人,決河水衛京城。癸未,置尚書省、樞密院于宮中,以便召問。起前元帥古里甲石倫權昌武軍節度使,行元帥府事。合達、蒲阿引軍自鄧州赴汴京。乙酉,以點檢夾谷撒合為總帥,將步騎三萬巡河渡,權近侍局使徒單長樂監其軍。起近京諸色軍家屬五十萬口入京。丙戌,大元兵既定河中,由河清縣白坡渡河。丁亥,長樂、撒合引
 兵至封丘而還。戊子,左司郎中斜卯愛實上書請斬長樂、撒合以肅軍政,不從。都尉烏林答胡土一軍自潼關入援,至偃師,聞大元兵渡河,遂走登封少室山。壬辰,衛州節度使完顏斜捻阿不棄城走汴。甲午,修京城樓櫓及守禦備。大元兵薄鄭州,與白坡兵合,屯軍元帥馬伯堅以城降,防御使烏林答咬住死之。乙未,大元游騎至汴城。丁酉,大雪。大元兵及兩省軍戰鈞州之三峰山,兩省軍大潰,合達、陳和尚、楊沃衍走鈞州,城破皆死之。樞密副使蒲阿就執,尋亦死。武仙走密縣。自是,兵不復振,己亥,徐州行省完顏慶山奴引兵入援,義勝軍校侯進、
 杜正、張興率所部北降,慶山奴入睢州。庚子,御端門肆赦,改元開興。辛丑,潼關守將李平以關降大元。壬寅,扶溝民錢大亨、李鈞叛,殺縣令王浩及其簿尉。庚戌,許州軍變,殺元帥古里甲石倫、粘合仝周、蘇椿等,以城降大元。



 二月壬子朔,慶山奴謀走歸德,至陽驛店遇大元兵,徐帥完顏兀里力戰而死,慶山奴被擒,使招京城,不從。睢州刺史張文壽棄城從慶山奴,皆死之。甲寅,大元兵徇臨渙,攝縣令張若愚死之。戊午,次盧氏。關、陜行省總帥兩軍及秦、藍帥府軍棄潼關而東,與之遇,天又大雪,未戰而潰。行省徒單兀典、總帥納合合閏敗死,完顏重
 喜降,斬于馬前。都尉鄭倜殺都尉苗英亦降。秦、藍總帥府經歷商衡死之。大元兵下睢州。庚申,翰林待制馮延登北來歸。乙丑,大元兵攻歸德。庚午,起復右丞相致仕賽不為左丞相。括京師民軍二十萬分隸諸帥,人月給粟一石有五斗。



 三月丁亥,大元軍平中京,留守撒合輦投水死。甲午,命平章政事白撒宿上清宮,樞密副使合喜宿大佛寺,以備緩急。大元遣使自鄭州來諭降,使者立出國書以授譯史,譯史以授宰相,宰相跪進,上起立受之,以付有司。書索翰林學士趙秉文、衍聖公孔元措等二十七家,及歸順人家屬,蒲阿妻子,繡女、弓匠、鷹
 人又數十人。庚子,封荊王子訛可為曹王,議以為質。密國公璹以曹王幼,請代行,上慰遣之,不聽其代。壬寅,尚書左丞李蹊送曹王出質,諫議大夫裴滿阿虎帶、太府監國世榮為請和使。戶部侍郎楊慥權參知政事。分軍防守四城。大元兵攻汴城,上出承天門撫西面將士。千戶劉壽語不遜,詔釋勿問。癸卯,上復出撫東面將士,親傅戰傷者藥于南薰門下,仍賜卮酒。出內府金帛器皿以賞戰士。乙巳,鳳翔府砲軍萬戶王阿驢、樊喬來歸。己酉,造革車三千兩,已而不用。置局養無家俘民。夏四月癸丑,兵士李新有功,擢四方館使。元帥劉益叱其子戰
 死。丁巳,遣戶部侍郎楊居仁奉金帛詣大元兵乞和。戊午,又以珍異往謝許和。癸亥,明惠皇后陵被發,失柩所在,遣中官往視之,至是始得。以兵護宮女十人出迎朔門奉柩至城下,設御幄安置,是夜復葬之。戮鄭倜妻子。甲子,御端門肆赦,改元天興。詔內外官民能完復州郡者功賞有差。出金帛酒炙犒飫軍士。減御膳,罷冗員,放宮女。上書不得稱聖,改聖旨為制旨。釋鎬厲王、衛紹王二族禁錮,聽自便。乙丑,百官初起居于隆德殿前。丙寅,以尚書省兼樞密院事。丁卯,放宮女,聽以衣裝自隨,金珠留犒士卒。汴京解嚴,步軍始出封丘門采薪蔬。己巳,建
 威都尉完顏兀論同大元使沒忒入城。庚午,見使臣於隆德殿。放宮女如前。辛未,開鄭門聽百姓男子出入。甲戌,御承天門大饗將士,聞有聲屈者乃還宮。乙亥,有詔止奏事。許州進櫻桃。



 五月辛巳,遷民告出城者以萬數,賽不、白撒不聽。乙酉,以南陽郡王子思烈行尚書省于鄧州,召援兵。丙戌,拜天於大慶殿,詔白撒致仕。放京城四面軍,李辛不奉詔。丁亥,鑿洧川漕渠,尋罷之。馮延登以奉使有勞,授禮部侍郎。戊子,裕州鎮防軍將領賀都喜率西軍二千人入援,放遷民出京。辛卯,大寒如冬。密國公璹薨。汴京大疫,凡五十日,諸門出死者九十餘萬
 人,貧不能葬者不在是數。癸巳,楊春入據亳州,觀察判官劉均死之。辛丑,上御香合,面責宰相。乙巳,將相受保城爵賞。



 六月庚戌朔,詔百官舉大將,眾舉劉益,不能用。癸丑,飛虎軍二百人奪封丘門出奔。甲寅,以出師錮門禁。乙卯,白撒開渠於私第東。丙辰,閱官馬,擇瘠者殺以食。丁巳,封仙據徐州,徒單益都走宿州,推張興行省事。庚申,塞京城四門,以便守禦。壬戌,國安用入徐州,殺張興,推封仙為元帥,以主州事。己巳,詔贈禦侮中郎將完顏陳和尚鎮南軍節度使。立褒忠廟碑。權參知政事楊糸廷罷。辛未,復修汴城。以疫後,園戶、僧道、醫師、鬻棺者擅
 厚利,命有司倍征之,以助其用。甲戌,宿州鎮防千戶高臘哥、李宣殺節度使紇石烈阿虎父子,請行省徒單益都主帥事,益都不從,率其將吏西走,至穀熟遇大元軍,死之。乙亥,左丞李蹊送曹王與其子仝俱還。丁丑,恒山公武仙殺士人李汾。



 七月庚辰朔,兵刃有火。辛巳,軍士撾登聞鼓乞將劉益。癸未,尚書右丞顏盞世魯罷。吏部尚書完顏奴申為參知政事。甲申,飛虎軍士申福、蔡元擅殺北使唐慶等三十餘人於館,詔貰其罪,和議遂絕。乙酉,都人揚言欲殺白撒,密詔遣衛士護其家。丙戌,軍士毀白撒別墅。斜捻阿不妄殺市人之過其門者以靖
 亂。丁亥,拜天於承天門下,出內府及兩宮物賜軍士。戊子,下令招軍。辛卯,簽民為兵。鞏昌民百二十人赴援。乙未,宿州帥眾僧奴稱國安用降,遣近侍直長因世英等持詔封安用為兗王,行京東等路尚書省事,賜姓完顏,改名用安。新軍有撾登聞鼓者,杖殺之。乙巳,金、木、火、太陰會于軫、翼。丙午,參知政事完顏思烈、恒山公武仙、鞏昌總帥完顏忽斜虎率諸將兵自汝州入援,以合喜為樞密使,將兵一萬應之,命左丞李蹊勸諭出師,乃行。



 八月己酉朔,合喜屯杏花營,又益兵五千人,始進屯中牟故城。庚戌,發丁壯五千人運糧,餉合喜軍。辛亥,完顏思
 烈遇大元兵于京水,遂潰,武仙退保留山,思烈走御寨,中京元帥左監軍任守貞死之。合喜棄輜重奔至鄭門,聚兵乃入。甲寅,免合喜為庶人,籍其家以賜軍士。降監軍長東為符寶郎。丁巳,釋奠孔子。戊午,括民間粟,己未,籍徒單兀典、完顏重喜、納合合閏家貲。前儀封令魏璠上言,「鞏昌帥完顏仲德沉毅有遠謀,臣請奉命往召。」不報。戊辰,免府試。起復前大司農侯摯為平章政事,進封蕭國公,行京東路尚書省事。己巳,摯帥兵行至封丘,將士將潰,摯止之,乃與眾還汴。壬申,聽無軍家口戍京。甲戌,金木星交。乙亥,賣官,及許買進士第。丙子,詔罷
 括粟,復以進獻取之。丁丑,京城民楊興入貲,授延州刺史。戊寅,劉仲溫入貲,授許州刺史。



\end{pinyinscope}