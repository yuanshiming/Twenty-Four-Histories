\article{志第三十四}

\begin{pinyinscope}

 選舉三



 ○右職吏員雜選



 右職。省令史、譯史。皇統八年格,初考遷一重,女直人依本法外,諸人越進義,每三十月各遷兩重,百二十月出職,除正六品以下,正七品以上職官。正隆二年,更為五十月遷一重。初考,女直人遷敦武校尉,餘人遷保義校尉,百五十月出職,係正班與從七品。若自樞密院臺六部轉省者,以前已成考月數通算出職。大定二年,復以
 三十月遷一官,亦以百二十月出職,與正、從七品。院臺六部及它府司轉省而不及考者,以三月折兩月,一考與從七,兩考正七品,三考與六品。三年,定格,及七十五月出職者,初上令,二中令,三下令,四、五錄事,六下令,七中令,八上令。百五十月出職者,初刺同、運判、推官等,二、三中令,四上令,回呈省。大定二十七年,制一考及不成考者,除從七品,須歷縣令三任,第五任則升正七品。兩考以上除正七品,再任降除縣令,三、四皆與正七品,第五任則升六品。三考以上者除六品,再任降正七品,三任、四任與六品,第五任則升從五品。



 省女直譯史。大定二十八年,制以見任從七、從八人內,勾六十歲以上者相視用之。明昌三年,取見役契丹譯史內女直、契丹字熟閑者,無則以前省契丹譯史出職官及國史院女直書寫,見任七品、八品、九品官充。



 省通事。大定二十年格,三十月遷一重,百二十月出職。一考兩考與八品,三考者從七品,餘與部令譯史一體免差。



 御史臺令史、譯史。皇統八年遷考之制,百二十月出職,正隆二年格,百五十月出職,皆九品,係正班。大定二年,百二十月出職,皆以三十月遷一官。其出職,一考、兩考皆與九品,三考與八品。明昌三年,截罷見役吏人,
 用三品職事官子弟試中者、及終場舉人本臺試補者,若不足,於密院六部見役品官、及契丹品官子孫兄弟選充。承安三年,敕凡補一人必詢於眾,雖為公選,亦恐久漸生弊。況又在書史之上,不試而即用,本臺出身門戶似涉太優,遂令除本臺班內祗、令譯史名闕外,於試中樞密院令譯史人內以名次取用,不足,即於隨部班祗令譯史上名轉充。若須用終場舉人之闕,則令三次終場舉人,每科舉後與它試書史人同程試驗,榜次用之。女直十三人,內班內祗六人,終場舉人七人。漢人十五人,內班內祗七人,終場舉人八人。譯史四人,內班內
 祗二人,終場舉人二人。



 樞密院令史、譯史。令史。正隆二年,制遷考與省同,出職除係正班正、從八品。大定二十一年,定元帥府令譯史三十月遷一官,百二十月出職,一考、兩考與八品除授,三考與從七品。十四年,遂命內祗、并三品職事官承蔭人、與四品五品班祗、及吏員人通試,中選者用之。



 十六年,定一考、兩考者,初錄事、軍判、防判,再除上簿,三中簿,四同初,五、六下令,七、八中令,九、十上令。二十六年,兩考者免下令一任。三考以上,初上令,二中令,三下令,四錄事、軍防判二十六年免此除,五下令二十六年亦免此除,六、七中令,八上令。十七年,制試
 補緦麻袒免以上宗室郎君。又定制,三品職事子弟設四人,吏員二人。



 睦親府、宗正府、統軍司令譯史,遷考出職,與臺部同。部令史、譯史,皇統八年格,初考三十月遷一重,女直人依本格,餘人越進義,第二、第三考各遷一重,第四考並遷兩重,百二十月出職八品已下。正隆二年,遷考與省右職令史同,出職九品。大定二十一年,宗正府、六部、臺、統軍司令史,番部譯史,元帥府通事,皆三十月遷一重,百二十月出職係班,一考、兩考與九品,三考已上與八品除授。十四年,以三品至七品官承廕子孫一混試充,尋以為不倫,命以四品五品子孫及吏員
 試中者,依舊例補,六品以下不與。十五年,命免差使。十六年格,一考兩考者,初除上簿,再除中簿,三下簿,四上簿,五錄事、軍防判,六、七下令,八、九中令,十上令。三考以上者,初除錄事、軍防判,再除上簿,三中簿,四如初,五下令後免此除,六、七下令,八中令,九上令。按察司書吏,以終場舉人內選補,遷加出職同臺部。



 凡內外諸吏員之制,自正隆二年,定知事孔目出身俸給,凡都目皆自朝差。海陵初,除尚書省、樞密院、御史臺吏員外,皆為雜班,乃召諸吏員於昌明殿,諭之曰:「爾等勿以班次稍降為歉,果有人才,當不次擢用也。」又定少
 府監吏員,以內省司舊吏員、及外路試中司吏補。



 大定二年,戶部郎中曹望之言:「隨處胥吏猥多,乞減其半。」詔胥吏仍舊,但禁用貼書。又命縣吏闕,則令推舉行止修舉為鄉里所重者充。三年,以外路司吏久不升轉,往往交通豪右為姦,命與孔目官每三十月則一轉,移於它處。七年,敕隨朝司屬吏員通事譯史勾當過雜班月日,如到部者並不理算。又詔吏人但犯贓罪罷者,雖遇赦,而無特旨,不許復敘。又命京府州縣及轉運司胥吏之數,視其戶口與課之多寡,增減之。十二年,上謂宰臣曰:「外路司吏,止論名次上下,恐未得人。若其下有廉慎、熟
 閑吏事,委所屬保舉。試不中程式者,付隨朝近下局分承應,以待再試。彼既知不得免試,必當盡心以求進也。」



 章宗大定二十九年,上封事者言:「諸州府吏人不宜試補隨朝吏員,乞以五品以上子孫試補。蓋職官之後清勤者多,故為可任也。」尚書省謂:「吏人試補之法,行之已久,若止收承廕人,復恐不閑案牘,或致敗事。舊格惟許五品職官子孫投試,今省部試者尚少,以所定格法未寬故也。」遂定制,散官五品而任七品,散官未至五品而職事五品,其兄弟子孫已承蔭者並許投試,而六部令史內吏人試補者仍舊。泰和四年,簽河東按察司事張
 行信言:「自罷移轉法後,吏勢浸重,恣為豪奪,民不敢言。今又無朝差都目,止令上名吏人兼管經歷六案文字,與同類分受賄賂。吏目通歷三十年始得出職,常在本處侵漁,不便。」遂定制,依舊三十月移轉,年滿出職,以杜把握州府之弊。八年,以僉東京按察司事楊雲翼言,書吏書史皆不用本路人,以別路書吏許特薦申部者類試,取中選者補用。



 凡右職官,天德制,忠武以下與差使,昭信以上兩除一差。大定十二年,敕鎮國以上即與省除。十三年,制明威注下令,宣威注中令,廣威注上令,信武權注下令,宣武、顯武免差,權注丞簿。又制宣武、顯武,功酬
 與上簿,無虧與中簿。二十六年,制遷至宣武、顯武始令出職。又以舊制通歷五任令呈省,詔減為四任。明昌三年,以諸司除授,守闕近三十月,於選調窒礙,今後依舊兩除一差,候員闕相副,則復舊制。



 泰和元年,以縣令見闕,近者十四月,遠者至十六月,蓋以見格,官至明威者並注縣令,或犯選並虧永人,若帶明威人亦注,是無別也。遂令曾虧永及犯選格,女直人展至廣威,漢人至宣武,方注縣令。又以守闕簿丞,近者十九月、遠者二十一月,依見格官至宣武、顯武、信武者合注丞簿,遂命但曾虧永,直至明威方注丞簿。又吏格,凡諸右職正雜班
 謂無資歷者,班內祗同。,皆驗官資注授。帶忠武以下者與監當差使,昭信以上擬諸司除授,仍兩除一差。宣武以上與中簿功酬人與上簿,明威注下令,宣威注中令,廣威注上令,通歷縣令四任,如帶定遠已歷縣令三任者,皆呈省,若但曾虧永及犯選格諸曾犯公罪追官、私罪解任、及犯贓、廉訪不好、併體察不堪臨民,謂之犯選格。女直人選至武義,漢人諸色人武略,並注諸司除授,皆兩除一差。若至明威方注丞簿,女直人遷至廣威,漢人、諸色人遷至宣威者,皆兩任下令,一任中令,回呈省。貞祐三年,制遷至宣武者,皆與諸司除授,亦兩除一差。凡不犯選格者,若懷遠方注丞簿,至安遠則注下令、上
 令各一任,呈省。四年,復以官至懷遠注下令,定遠注中令,安遠注上令,四任呈省。



 檢法、知法。正隆二年,嘗定六部所用人數及差取格法,初考、兩考皆除司候,三考者除上簿。五年,定制,十年內者初考除下簿,兩考除中簿,三考除警判。十年外者初考除第二任司候,兩考除上簿,三考則除市丞。大定二年,制曾三考者,不拘十年內外,皆與八品錄事、市令,擬當合得本門戶。除授,舊授扎付,大定三年始命給敕,以律科人為之。七年,定制,驗榜次勾取,如勾省令史之制。二十六年,命三考除錄事,以後則兩除一差。



 女直知法、檢法。大定三年格,以臺部統軍司出職令譯史,曾任縣佐市令差使人內奏差,考滿比元出身升一等,依隨路知事例給敕,以三十月為任。明昌五年,以省院臺部統軍司令譯史書史內擬,年五十以下、無過犯、慎行止,試一月,以能者充,再勒留者升一等,一考者初上令,二、三中令,四上令、兩考升二等,呈省。



 太常寺檢討二人。正隆二年,五十月遷一重,女直遷敦武,餘人進義,百五十月出職,係雜班。大定二年,制以三十月遷一重,百二十月出職,係正班九品。



 省祗候郎君。大定三年,制以袒免以上親願承應已試
 合格而無闕收補者及一品官子,已引見,止在班祗候,三十月循遷。初任與正,從七品,次任呈省。內祗在班,初、次任注正、從八品,三、四注從七品,而後呈省。班祗在班,初九品,次、三正、從八品,四、五從七品。而後呈省。已上三等,並以六十月為滿,各遷一重。八年,定制,先役六十月以試驗其才,不能幹者進一官黜之。才幹者再理六十月。每三十月遷加,百二十月為滿,須用識女直字者。十六年,定制,以制文試之,能解說得制意者為中選。十八年,制一品官子,初都軍,二錄事,軍防判,三都軍,四下令,五、六上令,回呈省。內祗,初錄事,軍防判,二上簿,三同初,四
 錄事,五都軍,六下令,七中令,八上令,回呈省。班祗,初上簿,二中簿,三同初,四錄事,軍防判,五錄事,六都軍,七下令,八中令,九上令,回呈省。



 國史院書寫。正隆元年,定制,女直書寫,試以契丹字書譯成女直字,限三百字以上。契丹書寫,以熟於契丹大小字,以漢字書史譯成契丹字三百字以上,詩一首,或五言七言四韻,以契丹字出題。漢人則試論一道。遷考出職同太常檢討。



 宗室將軍。六十月為任,初刺同,二都軍,三刺同,四從六。副將軍以七品出職人充。明昌元年,以九十月為滿,中
 都、上京初從七,二錄事、軍防判,三入本門戶。餘路,初錄事、軍防判,二上簿,三入本門戶。承安二年改司屬令作隨朝。



 內侍御直。內直六十四人,正隆二年格,長行人五十月遷一重,女直人遷敦武,餘人遷進義,無出身。大定二年格,同上。大定六年,更定收補內侍格,能誦一大經、以《論語》、《孟子》內能誦一書,并善書札者,月給奉八貫石,稍識字能書者七貫石,不識字六貫石。泰和二年,以參用外官失防微之道,乃創寄錄官名,以專任之,既足以酬其勞,而無侵官之弊。



 凡宮中諸局分,大定元年,世宗謂諸局分承應人,班敘俸給涉於太濫,正隆時乃無出身,涉于太刻,又其官品不以勞逸為制,遂命更定之。大定六年,諭有司曰:「宮中諸局分承應人,有年滿數差使者,往往苦於稽留,而卒不得。其差者,復多不解文字而不幹,故公私不便。今從願出局者聽,願留者各增其秩,依舊承應。其十人長,雖老願留者亦增秩,作長行承應,餘依例放還。」七年,詔宰臣曰:「女直人自來諸局分不經收充祗候。可自今除太醫、司天、內侍外、餘局分並令收充勾當。」



 護衛,正隆二年格,每三十月遷一重,初考,女直遷敦武,
 餘遷保義,百五十月出職,與從五品以下、從六品以上除。大定二年格,更為初遷忠勇,百二十月出職。大定十四年官制,從下添兩重,遂命女直初遷修武,餘人敦武。十八年,制初除五品者次降除六品,第三復除從五品。初任六品者不降,第四任升授從五品,再勒留者各遷一官。明昌元年資格,初任不算資歷,不勒留者,初從六品,二、三皆同上,第四任升從五。勒留者,初從五,二、三同上,第四正五品。再勒留者,初正五品,二同上,三少尹,四刺史。明昌四年,降作六品、七品除。貞祐制,一考八品,兩考除縣令,三考正七品,四考六品。五年,定一考者注上
 令。兩考者一任正七品回降從七,兩任正七回陞六品。三考者正七一任回,再任正七升六品。四考者,三任六品陞從五品。



 符寶郎,十二人,正隆二年格,皆同護衛,出職與從七品除授。大定二年格,並同護衛。十四年,初收。餘人遷進義,二十一年,英俊者與六品除,常人止與七品除。



 奉御,十六人,以內駙馬充,舊名入寢殿小底。大定十二年,更今名。正隆二年格,同符寶郎。大定二年,出職從七品。



 奉職,三十人,舊名不入寢殿小底,又名外帳小底。大定
 十二年更今名。正隆二年格,女直遷敦武,餘人歷進義,無出身。大定二年格。出職正班九品。大定十四年定新官制,從下添兩重,女直初考進義,餘人進義副尉。十七年格,有廕者初中簿,二下簿,無蔭者注縣尉,已後則依格。明昌元年格,有廕者每勒留一考則減一資。二年,以八品出職。六年定格,初錄事、軍防判,正從八品丞,二上簿、三中簿,四正從八品,若不犯選格者則免此除,五下令,六、七中令,八上令。勒留一考者升下令,四、五中令,六上令,回呈省。勒留兩考者陞上令,二中令,三、四上令,回呈省。凡奉御奉職之出職,大定十二年增為百五十月,
 二十九年復舊,承安四年復增。



 東宮護衛,正隆二年,出職正班從八品。大定二年,正從七品。初收女直遷敦武,餘人保義。



 閣門祗候,正隆二年格,女直初遷敦武,餘人保義,出職正班從八品。大定二年格,出職從七品。八年定格,初都軍,二錄事,三軍防判,四都軍,五下令,六中令,七上令。已帶明威者即與下令,二錄事、軍防判,三都軍,四下令,五中令,六上令。泰和四年格,初都軍,二錄事、軍防判、三下令,四中令,五上令。



 筆硯承奉,舊名筆硯令史,大定三年,更為筆硯供奉,後
 以避顯宗諱,復更今名。正隆二年,女直人遷敦武,餘歷進義,無出身。大定二年格,初考女直遷敦武,餘保義,出職正班從七品。吏格,初都軍,二、三下令,四、五中令,六上令。



 妃護衛,正隆二年格,與奉職同。大定二年,出職與八品。



 符寶典書,四人,舊名牌印令史,以皇家袒免以上親、有服外戚、功臣子孫為之。正隆二年格,出職九品。大定二十八年,出職八品,二上簿,回驗官資注授。



 尚衣承奉,天德二年格,以班內祗人選充。大定三年,女直人遷敦武,餘人遷進義,出職九品。



 知把書畫,十人,正隆二年格,與奉職同。大定二年,出職九品。十四年格,同奉職。二十一年定格,有廕者,初中簿,二軍器庫副,後依本門戶差注。無廕者,與差使。



 凡已上諸局分承應人,正隆二年格,有出身者皆以五十月為一考,五考出職,無出身者五十月止遷一官。大定二年,三年格,皆三十月為考,遷一重,四考出職。十二年,復加為五考。大定二十九年,又為四考。承安四年,復為五考。自大定十二年,凡增考者,惟護衛則否。



 隨局內藏四庫本把,二十八人,正隆二年格,同奉職。大定二年格,十人長,每三十月遷一重,四考出職九品。長行,每五十月遷
 一重,初考女直敦武,餘人進義。轉十人長者其後依親軍例,轉五十人長者以三十月遷加,雖未至十人長而遷加至敦武者,依本門戶出職。十二年,加為五考。二十一年格,與知把書畫同。二十八年,以合數監同人內,從下選差。明昌元年,如八貫石本把闕,六貫石局內選。六年,半於隨局承應人內選。左右藏庫本把,八人,格同內藏。大定二十九年設,三十月遷一重,百二十月出職。儀鸞局本把,大定二十七年,三人。明昌元年,設十五人,格比內藏本把。
 尚食局本把,四人,大定二十八年設,格同儀鸞。尚輦局本把,六人,二十八年設,格同儀鸞。



 典客署書表,十八人,大定十二年,以班內祗、并終場舉人慎行止者,試三國奉使接送禮儀、并往復書表,格同國史院書寫。十四年,以女直人識漢字班內祗一同試補。大定二十四年,終場舉人出職八品注上簿,次下簿,三任依本門戶。明昌五年,復許終場舉人材質端偉、言語辯捷者,與內班祗同試,與正九除。



 捧案,八人,大定十九年,以已承三品官廕人,命宣徽院揀試儀觀修整者,格同尚衣承奉。二十一年,格同知把
 書畫。



 擎執儤使,大定四年,以內職及承奉班內選。明昌六年,以皇家袒免以上親、不足則於外戚,并三品已上散官、五品以上職事官應廕子孫弟兄姪,以宣徽院選有德而美形貌者。



 奉輦,舊名拽輦兒,大定二十九年更名,格同擎執。



 妃奉事,舊名不入寢殿小底,大定十一年又名妃奉職,大定十八年更今名。格同知把書畫。



 東宮妃護衛,十人,大定十三年,格同親王府祗候郎君。二十八年,有蔭人與副巡檢、譏察,無廕人與司軍,軍轄等除。
 東宮入殿小底,三十月遷一重。初考,女直人遷敦武,餘人遷保義。吏格,有廕無廕其出職,初八品,二上簿,三中簿,四八品。五下令,六中令,八上令,回呈省。東宮筆硯,五十月遷一重,百五十月出職正班九品。無蔭人差使。有廕人,大定二十一年格,與二十一年知把書畫格同。



 正班局分,尚藥、果子本把、奉膳、奉飲、司裀、儀鸞、武庫本把、掌器、掌輦、習騎、群子都管、生料庫本把。大定二十一年格,有廕人,知把書畫格同。章宗大定二十九年,諸局分長行並歷三百月。十
 人長九十月出職。



 雜班局分,鷹坊子、尚食局廚子、果子廚子、食庫車本把、儀鸞典幄、武庫槍寨、司獸、錢帛庫官、旗鼓笛角唱曲子人、弩手、傘子。貞元元年,制弩手、傘子、尚廄局小底、尚食局廚子,並授府州作院都監。大定二十九年,長行三百月、十人長九十月出職。弩手、傘子四百月出職。其他局分,若秘書監楷書及琴、棋、書、阮、象、說話待詔,尚廄局醫獸、駝馬牛羊群子、酪人,皆無出身。



 侍衛親軍長行,初收,遷一重,女直敦武,餘人進義。每五
 十月遷一重。以次轉五十人長者,則每三十月遷一重。如五十人長內遷至武義者,以五十人長本門戶出職。五十人長每三十月遷一重,六十月出職,係正班,與九品除授,有廕者八品除授。如轉百人長者,則三十月遷一重,六十月出職,係正班八品,有蔭者七品。大定六年,百戶任滿,有廕者注七品都軍、正將,無蔭及五十戶有廕者,注八品刺郡、都巡檢、副將。五十戶無廕者及長行有廕者,注縣尉,無蔭注散巡檢。十六年,有廕百戶,初中令,二都軍、正將,三、四祿事,五下令,六中令、七上令,回呈省。無廕者,初都軍、正將,二錄事,三、四副將、巡檢,五都軍、
 正將,六下令,七中令,八上令,回呈省。此言識字者也。不識字者,初止縣尉,次主簿。二十一年,有廕者初中簿,二縣尉。無蔭者初縣尉,二散巡檢。已後,依本門戶,識字、不識字並用差注。二十九年,定女直二百五十月出職,餘三百月出職。吏格,先察可親民、及不可者,驗其資歷,若已任回帶明威、懷遠者,驗資擬注。



 拱衛直,正隆名龍翔軍,無出身。大定二年,改龍翔軍為拱衛司。定格,軍使,什將,長行,每五十月遷一重,女直人敦武,餘人進義。遷至指揮使,則三十月出職,遷一重,係正班,與諸司都監。雖未至指揮使,遷至武義出職,係雜
 班,與差使。



 司天長行,正隆二年,定五十月遷一重,女直敦武,餘人進義,無出身。



 太醫,格同。貞元元年,嘗罷去六十餘人。正隆二年格,五十月遷一重,女直人敦武,餘人進義,無出身。



 教坊,正隆間有典城牧民者,大定間罷,遂定格同上。



\end{pinyinscope}