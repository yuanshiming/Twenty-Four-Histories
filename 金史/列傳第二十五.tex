\article{列傳第二十五}

\begin{pinyinscope}

 ○紇石烈志寧僕散忠義徒單合喜



 紇石烈志寧,本名撒曷輦,上京胡塔安人。自五代祖太尉韓赤以來,與國家世為甥舅。父撒八,海陵時賜名懷忠,為泰州路顏河世襲謀克,轉猛安,嘗為東平尹、開遠軍節度使。志寧沉毅有大略,娶梁王宗弼女永安縣主,宗弼於諸婿中最愛之。皇統間,為護衛。海陵以為右宣徽使,出為汾陽軍節度使,入為兵部尚書,改左宣徽使、
 都點檢,遷樞密副使,開封尹。



 契丹撒八反,樞密使僕散忽土、北京留守蕭賾、西京留守蕭懷忠皆以征討無功,坐誅。於是,志寧為北面副統,與都統白彥敬,以北京、臨潢、泰州三路軍討之。志寧至北京,而海陵伐宋已渡淮。彥敬、志寧聞世宗有異志,乃陰結會寧尹完顏蒲速賚、利涉軍節度使獨吉義,將攻之。而世宗已即位,使石抹移迭、移剌曷補來招,彥敬、志寧殺其使者九人。世宗使完顏謀衍來伐,眾不肯戰,乃與彥敬俱降。世宗問曰:「正隆暴虐,人望既絕,朕以太祖之孫即大位。汝殺我使者,又不能為正隆死節,恐為人所圖,然後來降。朕今殺
 汝等,將何辭?」彥敬未有以對,志寧前奏曰:「臣等受正隆厚恩,所以不降,罪當萬死。」上曰:「汝輩初心亦可謂忠於所事,自今事朕,宜勉忠節。」


世宗使扎八招窩斡,扎八乃勸之,遂稱帝。世宗使右副元帥完顏謀衍征之,志寧以臨海節度使,都統右翼軍。窩斡敗于長濼,西走,志寧追及於霿
 \gezhu{
  松}
 河。賊已先渡,依岸為陣,毀橋岸以為阻。志寧與賊夾河,為疑兵,與萬戶夾谷清臣、徒單海羅於下流涉渡。已渡,前有支港岸斗絕,其中泥濘,乃束柳填藉,士卒畢濟。行數里,得平地,將士方食,賊奄至。賊據南岡,三馳下志寧陣。陣堅,力戰,流矢中左臂,戰自若。賊據上風
 縱火,乘煙勢馳擊。志寧步軍繼至,轉戰十餘合,火益熾,風煙突人,不可當。會雨作,風煙乃熄,遂奮擊,大破之。於是,元帥謀衍、右監軍福壽不急擊賊,久無功,右丞僕散忠義請自討賊,而志寧擊賊有功,上以忠義代謀衍,志寧代福壽,封定國公,使蒲察通至軍中宣諭之。賊略懿州界,陷靈山、同昌、惠和三縣,睥睨北京。會土河水漲,賊不得渡,乃西趨三韓縣。志寧方追躡之,元帥忠義與賊遇於花道,軍頗失利。賊見志寧踵其後,不敢乘勝,遂西走。是時,大軍馬瘦弱,不堪追襲,諸將欲止軍勿追。志寧獲賊候人,知賊自選精銳,與老小輜重分道,期山後會
 集,可擊其輜重。忠義以為然,遂過移馬嶺,進及裊嶺西陷泉。賊見左翼據南岡為陣,不敢犯。右翼萬戶烏延查剌擊賊少卻,志寧與夾谷清臣等擊之,賊眾大敗,涉水走。窩斡母徐輦舉營由落括岡西去,志寧追及之,盡獲其輜重,俘五萬餘人,雜畜不可勝計。偽節度使六,及其部族皆降。窩斡走奚中,至七渡河,志寧復敗之。賊過渾嶺,入于奚中。志寧獲賊將稍合住,釋弗殺,許以官賞,縱之歸,約以捕窩斡自效。稍合住既去,見窩斡,秘不言見獲事,乃反間奚人於窩斡曰:「陷泉失利,奚人有貳志,不可不察。」當是時,窩斡屢敗,其下亦各有心,稍合住乃與
 賊帥神獨斡執窩斡,詣右都監完顏思敬降。志寧與萬戶清臣,宗寧、速哥等,追捕餘黨至燕子城,盡得所畜善馬,因至抹拔里達之地,悉獲之。逆黨既平,入朝,為左副元帥,賜以玉帶。



 經略宋事,駐軍睢陽,都元帥忠義居南京,節制諸軍。宋將黃觀察據蔡州,楊思據潁昌。志寧使完顏王祥復取蔡州,黃觀察遁去。完顏襄攻潁州,拔之,獲楊思。乃移牒宋樞密使張浚,使依皇統以來舊式,浚復書曰:「謹遣使者至麾下議之。」是時,宋得窩斡黨人括里、扎八,用其謀攻靈璧、虹縣,都統奚撻不也叛入于宋,遂陷宿州。括里等謀曰:「北人恃騎射,戰勝攻取。今夏月
 久雨,膠解,弓不可用。」故李世輔與之來攻宿州。歸德尹術甲撒速、宿州防禦使烏林答剌撒、萬戶溫迪罕速可、裴滿婁室,不守約束,不肯堅壁俟大軍,輒出與戰,由是軍敗,城陷。剌撒嘗遣人入宋界貿易,交通李世輔,受其賂遺,久之,事覺,伏誅。謀克賽一坐故知不舉,除名。撻不也母斡里懶,緣坐當死,上曰:「撻不也背國棄母,殺之何益?朕閔其老。」遂原其死。詔撒速、剌撒、速可、婁室各杖有差,撒速、剌撒仍解職。世輔自以為得志,日與括里、扎八置酒高會。志寧以精兵萬人,發自睢陽,趨宿州。中使來督軍,志寧附奏曰:「此役不煩聖慮,臣但恐世輔遁去耳。」
 世輔聞志寧軍止萬人,甚易之,曰:「當令十人執一人也。」括里等問候人所見上將旗幟,知是志寧,謂世輔曰:「此撒合輦監軍也,軍至萬人,慎毋輕之。」大定三年五月二十日,志寧將至宿州,乃令從軍盡執旗幟,駐州西為疑兵,三猛安兵駐州南。志寧自以大軍駐州東南,阨其歸路。世輔望見州西兵旌旗蔽野,果謂大軍在州西,而謂東南兵少不足慮,先擊之。以步騎數萬,皆執盾,背城為陣,外以行馬捍之。使別將將兵三千,出自東門,欲自陣後攻志寧軍,萬戶蒲查擊敗之。右翼萬戶夾谷清臣為前行,撤毀行馬,短兵接戰,世輔軍亂,諸將乘之,追殺至城下。是夕,
 世輔盡按敗將,將斬之,其統制常吉懼而來奔,盡得城中虛實。明日,世輔悉兵出戰,騎兵居前,志寧使夾谷清臣當之。世輔別將以五六千騎為一隊,與清臣遇,清臣踵擊之,宋將不能反旆。志寧麾諸軍力戰,世輔復大敗,走者自相蹈藉,僵尸相枕,爭城門而入。門填塞,人人自阻,遂緣城而上。我軍自濠外射之,往往墮死於隍間。殺騎士萬五千,步卒三萬餘人。世輔乘夜脫走。明日,夾谷清臣、張師忠追及世輔,斬首四千餘,赴水死者不可勝計,獲甲三萬,他兵仗甚眾。上以御服金線袍、玉吐鶻、賓鐵佩刀,使移剌道就軍中賜之。凡有功將士,猛安、謀克
 並如陜西遷賞,蒲輦進官三階、重彩三端、絹六匹,旗鼓笛手、吏人各賜錢十貫。詔志寧曰:「卿雖年少,前征契丹戰功居最,今復破大敵,朕甚嘉之。」



 宋人議和不能決,都元帥僕散忠義移軍泰和,志寧移軍臨渙,遂渡淮,徒單克寧取盱眙、濠、廬、和、滁等州。宋人懼,乃決意請和。使者六七往反,議遂定,宋世為姪國,約歲幣二十萬兩、匹。魏杞奉誓書入見,復通好。志寧還軍睢陽,上以御服、玉佩刀、通犀御帶賜之。詔曰:「靈璧、虹縣、宿州兵士死者,朕實閔焉。宜歸葬鄉里,官為齎送,人賻錢三十貫。」鳳翔尹孛術魯定方以下猛安謀克,官為致祭。定方賻銀五百兩、
 重彩二十端,猛安三百貫,謀克二百貫,蒲里衍一百貫,權猛安二百貫,權謀克一百五十貫,權蒲里衍七十貫。



 五年三月,忠義朝京師,志寧駐軍南京。五月,志寧召至京師,拜平章政事,左副元帥如故。志寧復還軍,賜玉束帶,上曰:「卿壯年能立功如此,朕甚嘉之。南服雖定,日月尚淺,須卿一往規畫。」六年二月,志寧還京師,拜樞密使。七年十一月八日,皇太子生日,宴群臣於東宮,志寧奉觴上壽,上悅,顧謂太子曰:「天下無事,吾父子今日相樂,皆此人力也。」使太子取御前玉大杓酌酒,上手飲志寧,即以玉杓及黃金五百兩賜之。以第十四女下嫁志寧
 子諸神奴,八年十月,進幣,宴百官于慶和殿。皇女以婦禮謁見,志寧夫婦坐而受之,歡飲終日,夜久乃罷。九年,拜右丞相。十一年,代宗敘北征。既還,遣使者迎勞,賜以弓矢、玉吐鶻。入見,上慰勞良久。是日,封廣平郡王,復遣使就第慰勞之。皇太子生日,宴群臣於東宮,以玉帶賜志寧,上曰:「此梁王宗弼所服者,故以賜卿。」郊祀覃恩,從征護衛,皆有賜,進封金源郡王。



 十二年,志寧有疾,中使看問,日三四輩。疾亟,賜金丹三十粒,詔曰:「此丹未嘗以賜人也。」使者至,志寧已不能言,但稽首而已。是歲,薨。上輟朝,臨其喪,行哭而入,哀動左右。將葬,上致祭,見陳甲
 柩前,復慟哭之。賻銀千五百兩、重彩五十端、絹五百匹,葬事祠堂,皆從官給,謚武定。十五年,圖像衍慶宮。



 志寧妻永安縣主妒甚,嘗殺孕妾,及志寧薨後,諸神奴兄弟皆病亡,世宗甚惜之,遣使諭永安縣主曰:「丞相有大功三,先朝舊臣,惟秦、宋二王功大,餘不及也。今養其孽子,當如親子視之。」二十二年,上問宰臣:「僕散忠義、紇石烈志寧孰愈?」尚書左丞襄奏曰:「忠義兵權精致,此其所長也。」上曰:「不然。志寧臨敵,身先士卒,勇敢之氣,自太師梁王未有如此人者也。」明昌五年,配享世宗廟廷。



 僕散忠義,本名烏者,上京拔盧古河人,宣獻皇后侄,元
 妃之兄也。高祖斡魯補。曾祖班睹。祖胡闌。父背魯,國初世襲謀克,婆速路統軍使,致仕。忠義魁偉,長髯,喜談兵,有大略。年十六,領本謀克兵,從宗輔定陜西,行間射中宋大將,宋兵遂潰,由是知名。帥府錄其功,承制署為謀克。宗弼再取河南,表薦忠義為猛安。攻冀州,先登,攻大名府,以本部兵力戰,破其軍十餘萬,賞以奴婢、馬牛、金銀、重彩。從宗弼渡淮攻壽、廬等州,宗弼稱之曰:「此子勇略過人,將帥之器也。」賞馬五匹、牛一百五十頭、羊五百口,領親軍萬戶,超寧遠大將軍,承其父世襲謀克。



 皇統四年,除博州防禦使,公餘學女直字,及古算法,閱月,盡
 能通之。在郡不事田獵、燕游,以職業為務,郡中翕然稱治。忽一夕陰晦,囚徒謀為反獄。倉猝間,將校皆惶駭失措,忠義從容,但使守更吏撾鼓鳴角。囚徒以為天且曉,不敢出,自就桎梏。及考,郡民詣闕願留,詔從之。八年,改同知真定尹,兼河北西路兵馬都總管,遷西北路招討使,入為兵部尚書。



 僕散忽土嘗與海陵篡立,恃勢陵傲同列,忠義因會飲眾辱之,海陵不悅,出為震武軍節度使。火山賊李鐵槍乘暑來攻,忠義單衣從一騎迎擊之,射殺數人,賊乃退。改臨洮尹,兼熙秦路兵馬都總管。海陵召至京師謂之曰:「洮河地接吐蕃、木波,異時剽害良
 民,州縣不能制。汝宿將,故以命汝。」賜條服、玉具、佩刀。閱再考,徙平陽尹,再徙濟南尹。以本官為漢南路行營副統制,伐宋,克通化軍。


世宗立,海陵死揚州,罷兵入朝京師,拜尚書右丞。移剌窩斡心慄號,兵久不決。右副元帥完顏謀衍既敗之於霿
 \gezhu{
  松}
 河,乃擁眾,貪鹵掠,不追討,而縱其子斜哥暴橫軍中,士卒不用命。賊得水草善地,官軍踵其遺餘,水草乏,馬益弱,賊軼出山西,久無功。忠義請曰:「契丹小寇,不時殄滅,致煩聖慮。臣聞主憂臣辱,願效死力除之。」世宗大悅。即召還謀衍,勒歸斜哥本貫。拜忠義平章政事,兼右副元帥,封榮國公,賜以御府貂裘、
 賓鐵吐鶻弓矢大刀、具裝對馬及安山鐵甲、金牌,詔曰:「軍中將士有犯,連職之外並以軍法從事,有功者依格遷賞。」詔諸將士曰:「兵久駐邊陲,蠹費財用,百姓不得休息。今以右丞忠義為平章政事、右副元帥,宜同心戮力,無或弛慢。」忠義至軍,賊陷靈山、同昌、惠和等縣,陣而西行。忠義追之,及于花道,宗亨為左翼,宗敘為右翼,與賊夾河而陣。賊渡河,先攻左翼,偏敗,右翼救之,賊引去。窩斡乃以精銳自隨,以羸兵護其母妻輜重由別道西走,期於山後會集。追復及于裊嶺西陷泉。與賊遇,時昏霧四塞,跬步莫睹物色,忠義禱曰:「狂寇肆暴,殺戮無辜,天
 不助惡,當為開霽。」奠已,昏霧廓然。及戰,忠義左據南岡,為偃月陣,右迤而北,大敗之,獲其弟裊,俘生口三十萬,獲雜畜十餘萬,車帳金珍以鉅萬計,悉分諸軍。賊走趨奚地,遣將追躡,至七渡河,又敗之。既踰渾嶺,復進軍襲之,望風奔潰,遁入奚中,降者相屬於路。詔忠義曰:「卿材能素著,果能大破賊眾,朕甚嘉之。今遣勞卿,如朕親往。賜卿御衣、及骨睹犀具佩刀、通犀帶等。就以俘獲,均散軍士。」窩斡既敗,遂入于奚中。高忠建敗奚于栲栳山,移剌道取抹白諸奚之家,抹白奚乃降,窩斡勢益弱。紇石烈志寧獲賊將稍合住,縱之使歸,約以捕窩斡自贖,仍
 許以官賞。稍合住與其黨,執窩斡詣完顏思敬降。契丹平。忠義朝京師,拜尚書右丞相,改封沂國公,以玉帶賜之。



 自海陵遇弒,大軍北還,而窩斡鴟張,命將徂征。及窩斡敗,其黨括里、扎八奔入于宋,宋人用其謀,侵掠邊鄙,攻取泗、壽、唐、海州。於是,宋主傳位于宗室子甗,是為宋孝宗,雖嘗遣使來,而欲用敵國禮。世宗以紇石烈志寧經略宋事,制詔忠義以丞相總戎事,居南京節制諸將,時大定二年也。忠義將行,陛辭,上諭之曰:「彼若歸侵疆,貢禮如故,則可罷兵。」既至南京,簡閱士卒,分屯要害,戒諸將嚴守備。使左副元帥志寧移牒宋樞密使張浚,其
 略曰:「可還所侵本朝內地,各守自來畫定疆界,凡事一依皇統以來舊約,帥府亦當解嚴。如必欲抗衡,請會兵相見。」宋宣撫使張浚復書志寧曰:「疆埸之一彼一此,兵家之或勝或負,何常之有,當置勿道。謹遣官僚,敬造麾下議之。」是時,已復泗、壽、鄧州,請隳其城,遷其民于宿、亳、蔡州,上曰:「三州本吾土也,得之則已。」忠義使將士擇善水草休息,且牧馬,俟來歲取淮南。初,世宗詔諸將由泗、壽、唐鄧三道進發,宋人聞之,即自方城、葉縣以來田野皆燒夷之,使無所芻牧。忠義命唐、鄧道軍芻牧許、汝間。



 三年,忠義入奏事,遂以丞相兼都元帥。無何,還軍中。忠
 義與宋相持日久,慮夏久雨,弓力易減,宋或乘時見攻,豫選勁弓萬張於別庫。及自汴赴闕議事,次濬州,宋將李世輔果掩取靈璧、虹縣,遂陷宿州。忠義使人還汴,發所貯勁弓給志寧軍,與宋人戰,遂大捷,竟復宿州。忠義還,以書責宋。宋同知樞密院事洪遵、計議官盧仲賢,遣使二輩持與志寧書及手狀,歸海、泗、唐、鄧州所侵地,約為叔侄國。報書期十一月使入境,宋又使人來言,禮物未備,請俟十二月行成。忠義以其事馳奏,請定書式,且言宋書如式,則許其入界,如其不然,勢須遣還本國,復稟其主,若是往復,動經七八十日,恐誤軍馬進取。世宗
 以詔諭之曰:「若宋人歸疆,歲幣如昔,可免奉表稱臣,許世為侄國。」忠義乃貽書宋人,前後凡七,宋人他託未從。忠義移大軍壓淮境,遣志寧率偏師渡淮,取盱眙、濠、廬、和、滁等州,宋人懼。而世宗意天下厭苦兵革,思與百姓休息,詔忠義度宜以行。



 四年正月,忠義使右監軍宗敘入奏,將近暑月,乞俟秋涼進發。詔從之。宋使胡昉敢以右僕射湯思退書來,宋稱侄國,不肯加世字。忠義執昉留軍中,答其書,使使以聞。詔曰:「行人何罪,遣胡昉還國。邊事從宜措畫。」八月,詔忠義曰:「前請俟秋涼進發,今已八月,復俟何時?」先是,忠義乞增金、銀牌,上曰:「太師梁王兼
 數職,未嘗增也。」至是增都元帥金牌一、銀牌二十,左右副元帥金牌各一、銀牌各十,左右監軍金牌各一、銀牌各六,左右都監金牌各一、銀牌各四,三路都統府銀牌各二。乃定南界官員、百姓歸附遷賞格。



 元帥府獲宋諜人符忠。忠前嘗至中都,大興府官詰問,忠執文據,及與泗州防禦判官張德亨知識,由是獲免,厚謝德亨,德亨受之。忠具款服,乃奏其事于朝,於是,大興少尹王全解職,德亨除名。和議始于張浚,中更洪遵、湯思退,及徒單克寧敗宋魏勝于十八里莊,取楚州,世宗下詔進師,於是宋知樞密院周葵、同知樞密院事王之望書一一如
 約,和議始定。宋遣試禮部尚書魏杞,崇信軍、承宣使康湑,充通問國信使,取到宋主國書式,并國書副本,宋世為侄國,約歲幣為二十萬兩、匹,國書仍書名再拜,不稱「大」字。大定五年正月,魏杞、康湑入見,其書曰:「侄宋皇帝甗謹再拜致書於叔大金聖明仁孝皇帝闕下。」魏杞還,復書「叔大金皇帝」不名,不書「謹再拜」,但曰「致書于侄宋皇帝」,不用尊號,不稱闕下。和好已定,罷兵,詔天下。以左副都點檢完顏仲為報問國信使,太子詹事楊伯雄副之。



 忠義奏官軍一十七萬三千三百餘人,留馬步軍一十一萬六千二百屯戍。上曰:「今已許宋講好,而屯戍尚
 多,可除舊軍外,選馬一萬二千,阿里喜稱是,步軍虞候司軍共選一萬五千,及簽軍一萬,與舊軍通留六萬。富強丁多者摘留,貧難者阿里喜官給,富者就用其奴。其存留馬步軍於河北東西、大名府、速頻、胡里改、會寧、咸平府、濟州、東京、曷速館等路軍內,約量揀取。其西南、西北招討司,臨潢府、泰州、北京、婆速、曷懶、山東東西路,並行放還。」詔近侍局使裴滿子寧佩金牌,護衛醜底、符寶祗候駝滿回海佩銀牌,諭諸路將帥,以宋國進到歲幣銀絹二十萬兩、匹,盡數給與見存留及放散軍充賞。曾過界者,人給絹二匹、銀二兩,不曾過界者銀二兩、絹一
 匹。阿里喜絹一匹。謀克倍軍人,猛安倍謀克。押軍猛安謀克年老有勞績者,量與除授。又詔曰:「其令一路全罷者,先發遣之。」賜忠義玉束帶。三月,詔曰:「如大軍已放還,丞相忠義宜先還,左副元帥志寧、右監軍宗敘留駐南京,餘官非急用者並勒還任。」



 忠義朝京師,上勞之曰:「宋國請和,偃兵息民,皆卿力也。」拜左丞相,兼都元帥。大定初,事多權制,詔有司刪定,上謂宰臣曰:「凡已奏之事,朕嘗再閱,卿等毋懷懼。朕於大臣,豈有不相信者?但軍國事,不敢輕易,恐或有誤也。」忠義對曰:「臣等豈敢竊意陛下,但智力不及耳。陛下留神萬幾,天下之福也。」



 大定六
 年正月,忠義有疾,上遣太醫診視,賜以御用藥物,中使撫問,相繼於道。二月,薨。上親臨哭之慟,輟朝奠祭,賻銀千五百兩、重彩五十端、絹五百匹。世宗將幸西京,復臨奠焉。命參知政事唐括安禮護喪事,凡葬祭從優厚,官為給之。大宗正丞竟充敕祭使,中都轉運副使王震充敕葬使,百官送葬,具一品儀物,建大將旗鼓,送至墳域。謚武莊。



 忠義動由禮義,謙以接下,敬儒士,與人極和易,侃侃如也。善御將士,能得其死力。及為宰輔,知無不言。自漢、唐以來,外家多緣恩戚以致富貴,又多不克其終,未有兼任將相功名始終如忠義者。十一年,詔曰:「故左
 丞相忠義族人,及昭德皇后親族,人材可用者,左副點檢烏古論元忠體察以聞。」二十一年,上思忠義功,勒銘墓碑。泰和元年,圖像衍慶宮,配享世宗廟廷。子揆,別有傳。



 徒單合喜,上京速蘇海水人也。父蒲涅,世襲猛安。合喜魁偉,膂力過人,一經聞見,終身不忘。天輔間,從金源郡王婁室為扎也,甚愛之。天會六年,以功為謀克,尋領婁室親管猛安。元帥府聞其才,命權左翼軍事。皇統二年,為隴州防禦使。以兵十五人敗宋兵二百於高陵,以兵五百人敗宋兵二千於秦州,以兵八百人敗宋兵三千
 五百於鳳翔。以二謀克拒饒風關,宋兵二千來奪其關口,奮擊敗之,諸軍乃得過險。遷平涼尹,再徙臨洮、延安尹。是時,關、陜以西,初去兵革,百姓多失業,合喜守之以靜,民多還歸者。天德二年,為元帥左都監,陜西統軍使。貞元二年,以本官兼河中尹。正隆六年,為西蜀道兵馬都統。



 世宗即位,以手詔賜合喜曰:「岐國失道,殺其母后,橫虐兄弟,流毒兆庶。朕惟太祖創業之艱難,勉膺大位。卿之子弟皆自軍中來歸,卿國家舊臣,豈不知天道人事?卿軍不多,未宜深入,當領軍屯境上。陜右重地,非卿無能措畫者。俟兵革既定,即當召卿,宜自勉之。」大定二
 年,復為陜西路統軍使。未幾,改元帥右都監。表陳伐宋方略,詔許以便宜從事。轉左都監。破宋兵于華州。是時宋吳璘侵古鎮,分據散關、和尚原、神叉口、玉女潭、大蟲嶺、石壁寨、寶雞縣,兵十餘萬,陷河州、鎮戎軍。合喜乞濟師,詔以河南兵萬人益之。合喜遣丹州刺史赤盞胡速魯改以兵四千守德順,吳璘以二十萬人圍之。統軍都監石抹迭勒將兵萬人,破宋兵于河州,還過德順,駐兵平涼,求益兵於合喜,以解德順之圍。合喜遣萬戶完顏習尼列、大良順,寧州刺史顏盞門都各將本部兵,合二萬人,以順義軍節度使烏延蒲離黑統押之,與迭勒會。
 吳璘聞之,使偏將將兵五千人來迎,前鋒特里失烏也、奚王和尚擊敗之,追至德順城南小溪邊,璘自將大軍蔽岡阜而出,烏也等馳擊之,迭勒、蒲離黑繼至,併力戰,日已暮,兩軍不相辨,乃解。已而璘報云:「宋主遣使至,兩國講和,請各罷兵。」璘遂遁去。蒲離黑亦引軍還。自宋兵圍城,至是凡四十餘日乃解。



 初,德順在圍中,押軍猛安溫敦蒲里海身先士卒,力戰未嘗少挫,及救兵至,圍解,蒲里海之功為多。頃之,吳璘復來犯陜西州郡,兵十餘萬。詔以兵七千益合喜兵,號二萬人,慶陽尹烏延蒲轄奴、延安尹高景山分領之。彰化軍節度使璋、通遠軍節
 度使烏延吾里補、寧州刺史移剌高山奴、京兆少尹宗室泥河、恩州刺史完顏謀良虎,皆備軍前任使。宋人驅率商、虢及華山、南山之民五萬人,來圍華州。押軍萬戶裴滿挼剌欲堅壁守之,猛安移剌沙里剌曰:「宋兵雖多,半是居民,不習戰,不如擊之。」於是挼剌以騎兵千人敗宋前鋒,追至其大軍,亦敗之,斬首五千餘級。已而,璋敗宋姚良輔軍于原州,宋戍軍自寶雞以西,至于大蟲嶺,皆自散關遁去。頃之,吳璘聞赤盞胡速魯改、烏延蒲里黑軍已去德順,率兵號二十萬,復據德順,陷鞏州、臨洮府。臨洮少尹紇石烈騷洽死之,詔贈官一階,賜錢五百貫。合
 喜以璋權都統,習尼列權副統,將兵二萬攻之。連戰,宋兵雖敗,璘恃其眾,不肯去,分其兵之半,守秦州。合喜乃自行,駐水洛城,東自六盤山,西抵石山頭,分兵守之,當德順、秦州之兩間,斷其餉道,璘乃引去。



 都統璋、副統習尼列邀擊宋經略使荊皋,自上八節至甘谷城,殺數千人。習尼列擒宋將硃永以下將校十二人。宋張安撫守德順,亦棄城遁,胡速魯改邀擊之,所殺過半,擒將校十餘人,遂復德順州。宋之守秦州者,亦自退。高景山定商、虢,宗室泥河取環州。於是,臨洮、鞏、秦、河、隴、蘭、會、原、洮、積石、鎮戎、德順、商、虢、環、華等州府一十六盡復之,陜西平。詔
 書獎諭,賜以玉帶。詔陜西將士,猛安,階昭毅以下遷兩資,昭武以上遷一資。謀克,階六品以下遷兩資,五品以上遷一資。押軍猛安,階昭武以上者遷一資,昭毅以下、武義以上遷兩資,昭信以下,女直人遷宣武,餘人遷奉信,無官者,女直人授敦信,餘人授忠武。押軍謀克,武功以下、忠顯以上遷兩資,忠勇以下,女直人遷昭信,餘人遷忠顯,無官者,女直人授忠顯,餘人授忠翊。正軍,有官者遷一資,無官者授兩資。猛安賞銀五十兩、重彩五端、絹十匹,權、正同之。正軍人給錢三十貫,阿里喜十貫。戰沒軍官、軍士、長行,贈官賜錢有差。



 五年,置陜西路統軍
 使,兼京兆尹。元帥府移治河中府。統軍使璋朝辭,上曰:「合喜年老,以陜西軍事委卿,凡鎮防利害,可訪問合喜也。」七年,入為樞密副使,改東京留守,賜以衣帶、佩刀,詔曰:「卿年老,以此職優佚,宜勉之。」九年,入為平章政事,奏睿宗收復陜西功數事,上嘉納之,藏之秘府。封定國公。



 十一年,薨。上方擊球,聞訃遂罷。有司致祭,備禮以葬。賻銀一千二百五十兩及重彩幣帛。二十一年,上念其功,遷其孫三合武功將軍,授世襲本猛安曷懶若窟申謀克。泰和元年,配享世宗廟廷。



 贊曰:大定之初,兵連於江、淮,難作於契丹,謀衍挾功,窩
 斡橫噬,有弗戢之畏焉。世宗獨斷,召還謀衍,僕散忠義受任責成矣。故曰:「兵主於將,將賢則士勇。」其此之謂邪!紇石烈志寧有言:「受詔征伐,則不敢辭,為宰相則誠不能。」如知為相之難,固所謂賢也。秦、隴之兵,殆哉岌岌乎,徒單合喜料敵應變若此之審,亦難矣哉。



\end{pinyinscope}