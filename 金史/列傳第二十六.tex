\article{列傳第二十六}

\begin{pinyinscope}

 紇
 石烈良弼完顏守道(本名習尼列)石琚唐括安禮移剌道(本名趙三子光祖)



 紇石烈良弼,本名婁室,回怕川人也。曾祖忽懶。祖忒不魯。父太宇,世襲蒲輦,徙宣寧。天會中,選諸路女直字學生送京師,良弼與納合椿年皆童丱,俱在選中。是時,希尹為丞相,以事如外郡,良弼遇之途中,望見之,嘆曰:「吾
 輩學丞相文字,千里來京師,固當一見。」乃入傳舍求見,拜於堂下。希尹問曰:「此何兒也?」良弼自贊曰:「有司所薦學丞相文字者也。」希尹大喜,問所學,良弼應對,無懼色。希尹曰:「此子他日必為國之令器。」留之數日。年十四,為北京教授,學徒常二百人。時人為之語曰:「前有谷神,後有婁室。」其從學者,後皆成名。年十七,補尚書省令史。簿書過目,輒得其隱奧。雖大文牒,口占立成,詞理皆到。時學希尹之業者稱為第一。除吏部主事。



 天德初,累官吏部郎中,改右司郎中,借秘書少監為宋主歲元使。是時,納合椿年為參知政事,薦良弼才出己右,用是為刑部
 尚書,賜今名。丁父憂,以本官起復。海陵嘗曰:「左丞相張浩練達事務,而頗不實。刑部尚書婁室言行端正,無所阿諂。」因謂椿年曰:「卿可謂舉能矣。常人多嫉勝己者,卿舉勝於己者,賢於人遠矣。」改侍衛親軍馬步軍都指揮使。良弼音吐清亮,海陵詔諭臣下,必令良弼傳旨,聞者莫不聳動,以故常被召問。不踰年,拜參知政事,進尚書右丞,賜佩刀入宮,轉左丞。海陵伐宋,良弼諫不聽,以為右領軍大都督。海陵在淮南,詔良弼與監軍徒單貞撫定上京、遼右。既而,諸軍往往道亡北歸,而世宗即位于遼陽,良弼乃還汴京。



 海陵死,世宗就以良弼為南京留
 守兼開封尹,再兼河南都統,召拜尚書右丞。世宗謂良弼曰:「卿嘗諫正隆伐宋,不用卿言,以至廢殞。當時懷祿偷安之人,朕皆黜之矣。今復用卿,凡於國家之事,當盡言,無復顧忌也。」良弼頓首謝。窩斡敗於陷泉,入奚中,詔良弼佩金牌及銀牌四,往北京招撫奚、契丹。還,拜尚書左丞。上言:「祖宗以來未錄功賞者,臣考按得凡三十二人,宜差第封賞。」詔曰:「已有五品以上官者,聞奏。六品以下及無官者,尚書省約量遷除。」自是功勞畢賞矣。進拜平章政事,封宗國公。



 初,山東兩路猛安謀克與百姓雜居,詔良弼度宜易置,使與百姓異聚,與民田互相犬牙
 者,皆以官田對易之,自是無復爭訴。六年十一月,皇太子生日,上置酒于東宮,良弼、志寧同賜酒。上曰:「邊境無事,中外晏然,將相之力也。」良弼奏曰:「臣等不才,備位宰相,敢不竭犬馬之力。」上悅。進拜右丞相,監修國史。世宗謂良弼曰:「海陵時,記注皆不完。人君善惡,為萬世勸戒,記注遺逸,後世何觀?其令史官旁求書之。」又曰:「五從以上宗室在省祗候者,才有可用,具名聞奏。其猥冗不足蒞官者,亦聞奏罷去。」左丞完顏守道奏:「近都兩猛安,父子兄弟往往析居,其所得之地不能贍,日益困乏。」上以問宰臣,良弼對曰:「必欲父兄聚居,宜以所分之地與土
 民相換易。雖暫擾,然經久甚便。」右丞石琚曰:「百姓各安其業,不若依舊便。」上竟從良弼議。《太宗實錄》成,賜良弼金帶、重彩二十端,同修國史張景仁、曹望之、劉仲淵以下賜有差。



 世宗與侍臣論古今為臣孰賢不肖,因謂宰相曰:「皇統、正隆多殺臣僚,往往死非其罪。朕委卿等以大政,毋違道以自陷,毋曲從以誤朕。惟忠惟孝,匡救輔益,期致太平。」良弼對曰:「臣等過蒙嘉惠,雖譾薄,敢不盡心。聖諭諄諄,臣等不勝萬幸。」良弼請於榷場市馬,毋拘牝牡,「今官馬甚少,一旦邊境有警,乃調於民,不亦晚乎。」上從之。八年,選侍衛親軍,世宗聞其中多不能弓矢,詔
 使習射。頃之,問良弼及平章政事思敬曰:「女直人習射尚未行耶?」良弼對曰:「已行之矣。」同知清州防禦事常德暉上書言:「吏部格法,止敘年勞,雖有材能,拘滯下位。刺史、縣令,多不得人。乞密加訪察,然後廉問。今酒稅使尚選能吏,縣令可不擇人才,乞以能吏當任酒稅使者,任親民之職。」上是其言,謂宰相曰:「朕思庶職多不得人,中夜而寤,或達旦不能寐。卿等注意選擇,朕亦密加體察。」良弼對曰:「女直、契丹人,須是曾習漢人文字,然後可。方今大率多為黨與,或稱譽於此,或見毀於彼,所以難也。」上曰:「朕所以密令體察也。」上謂良弼曰:「猛安謀克牛頭
 稅粟,本以備兇年,凡水旱乏糧處就賑給之。」進拜左丞相,監修國史如故。



 良弼為相既久,練達朝政,上所詢訪盡誠開奏,垂紳正笏不動聲氣,議政多稱上意。以母憂去,起復舊職。是時,夏國王李仁孝乞分國之半,以封其臣任得敬。上以問群臣,群臣多言此外國事,從之可也。上曰:「此非是仁孝本心,不可從。」良弼議與上意合。既而,夏國果誅任得敬,上表來謝。參知政事宗敘請置沿邊壕塹,良弼曰:「敵國果來伐,此豈可禦哉?」上曰:「卿言是也。」高麗國王王晛表讓國於其弟皓,上疑之,以問宰相良弼。良弼策以為讓國非王見本心。其後趙位寵求以四
 十州來附,其表果言王皓弒其兄晛,如良弼策,語在《高麗傳》中。



 世宗罷採訪官,謂宰臣曰:「官吏之善惡,何由知之?」良弼對曰:「臣等當為陛下訪察之。」以進《睿宗實錄》,賜通犀帶、重彩二十端。是年,有事南郊,良弼為大禮使。自收國以來,未嘗講行是禮,歷代典故又多不同,良弼討論損益,各合其宜,人服其能。上與良弼、守道論猛安謀克官多年幼,不習教訓,無長幼之禮。曩時鄉里老者輒教導之。今鄉里中耆老有能教導者,或謂事不在己而不問,或非其職而人不從。可依漢制置鄉老,選廉潔正直可為師範者,使教導之。良弼奏曰:「聖慮及此,億兆之
 福也。」他日,上問曰:「朕觀前史,有在下位而存心國家,直言為民者。今無其人,何也?」良弼曰:「今豈無其人哉。蓋以直道而行,反被謗毀,禍及其身,是以不為也。」



 大定十四年,歲在甲午,大興尹璋為賀宋正旦使,宋人就館奪其國書,詔梁肅詳問。眾議紛紛,謂凡午年必用兵,上以問良弼,對曰:「太祖皇帝以甲午年伐遼,太宗皇帝以丙午年克宋,今茲宋人奪我國書,而適在午年,故有此語,未必然也。」既而,梁肅至宋,宋主起立授受國書,如舊儀。梁肅既還,宋主遣工部尚書張子顏、知閣門事劉灊來祈請,其書曰:「言念眇躬,夙承大統。荷上國照臨之惠,尋盟
 遂閱於十年。修兩朝聘問之勤,繼好靡忘于一日。惟是函書之受,當新賓接之儀。嘗空臆以屢陳,飭行人而再請。仰祈眷顧,俯賜矜從。」上與大臣議,良弼奏曰:「宋國免稱臣為姪,免奉表為書,恩賜亦已多矣。今又乞免親接國書,是無厭也,必不可從。」平章政事完顏守道、參知政事移剌道與良弼議合。左丞石琚、右丞唐括安禮以為不從所請,必至于用兵。上謂琚等曰:「卿等所言,非也。所請有大於此者,更欲從之乎。」遂從良弼議,答其書,略曰:「弗循定分之常,復有授書之請。謂承大統,愈見自尊。奈何以若所為,尚求其欲。矧曰已行之禮,靡得而更。」其授
 受禮儀,終不復改。



 上問宰臣:「嘗求內外官舉賢能,未聞有舉者,何也?」參政魏子平請當舉者每任須舉一人,視其當不,以為賞罰。上曰:「宋制薦舉,其人犯私罪者,舉主雖至宰執,亦坐降罰。人心有恒者鮮,財利怵于前,或喪其所守。宰臣任大責重,豈坐是以為升黜邪?」良弼曰:「前詔朝官六品以上,外官五品以上,各舉所知,盍申明前詔?」從之。上曰:「朕欲周知官吏善惡,若尋常遣官采訪,恐用非其人。然則官吏善惡,何以知之?」良弼曰:「臣等當為陛下訪察。」上曰:「然,但勿使名實混淆耳。」上欲徙窩斡逆黨,分散置之遼東。良弼奏:「此輩已經赦宥,徙之生怨望。」
 上曰:「此目前利害,朕為子孫後世慮耳。」良弼曰:「非臣等所及也。」於是以嘗預亂者徙居烏古里石壘部。上問宰臣曰:「堯有九年之水,湯有七年之旱,而民不病飢。今一二歲不登,而人民乏食,何也?」良弼對曰:「古者地廣人淳,崇尚節儉,而又惟農是務,故蓄積多,而無饑饉之患也。今地狹民眾,又多棄本逐末,耕之者少,食之者眾,故一遇凶歲而民已病矣。」上深然之,於是命有司懲戒荒縱不務生業者。



 十七年,以疾辭相位,不許。告滿百日,詔賜告,遣太醫診視,屢使中使問疾。良弼在告既久,省多滯事,上以問宰相、參政,張汝弼對曰:「無之。」上曰:「豈曰無之。
 自今疑事久不能決者,當具以聞。」十八年,表乞致仕歸田里,上遣使慰諭之曰:「卿比以疾在告,朕甚憂之。今聞卿將往西京養疾,彼中風土,非老疾所宜。京師中倦於人事,若就近都佳郡居處,待疾少間,速令朕知之。」良弼奏曰:「臣遭遇聖明,濫膺大任,夙夜憂懼,以至成疾。比蒙聖恩,數遣使存問,賜以醫藥,臣之茍活至今,皆陛下之賜也。臣豈敢望到鄉里,便可愈疾。臣去鄉歲久,親識多已亡沒,惟老臣獨在,鄉土之戀,誠不能忘。臣竊惟自來人臣受知人主,無逾臣者,臣雖粉骨碎身無以圖報。若使一還鄉社,得見親舊,則死無恨矣。」上問宰相曰:「丞相
 良弼必欲歸鄉里,朕以世襲猛安封其子符寶曷答,俾之侍行,何如?」右丞相完顏守道曰:「不若以猛安授良弼,使其子攝事。」上從之。於是授胡論宋葛猛安,給丞相俸傔,良弼乃致仕歸。上謂宰相曰:「卿等非不盡心,但才力不及良弼,所以惜其去也。」其後,尚書省奏差除,上曰:「丞相良弼擬注差除,未嘗茍與不當得者,而薦舉往往得人。粘割斡特剌、移剌綎、裴滿餘慶,皆其所舉。至于私門請託,絕然無之。」嘗問良弼:「每旦暮日色皆赤,何也?」良弼曰:「旦而色赤應在東,高麗當之。暮而色赤應在西,夏國當之。願陛下修德以應天,則災變自弭矣。」既而夏國有
 任德敬之亂,高麗有趙位寵之難,其言皆驗云。是歲,薨。年六十。上悼惜之,遣太府監移剌綎、同知西京留守王佐為敕葬祭奠使,賻白金、彩幣加等,喪葬皆從官給。追封金源郡王,命翰林待制移剌履勒銘墓碑,謚誠敏。



 良弼性聰敏忠正,善斷決,言論器識出人意表。雖起寒素,致位宰相,朝夕惕惕盡心於國,謀慮深遠,薦舉人材,常若不及。居家清儉,親舊貧乏者周給之,與人交久而愈敬。居位幾二十年,以成太平之功,號賢相焉。明昌五年,配饗世宗廟廷。



 守道,本名習尼列,以祖谷神功,擢應奉翰林文字。皇統
 九年,同知盧龍軍節度使事,歷獻、祁、濱、薊四州刺史。世宗幸中都,過薊,父老遮道請留再任。平章政事移剌元宜舉以自代,於是遷昭毅大將軍,授左諫議大夫。內族晏以恩舊拜左丞相,守道諫曰:「陛下初即位,天下略定,邊警未息,方大有為之時,恐晏非其材。必欲親愛,莫若厚與之祿,俾勿事事。」乃授以太尉,致仕。世宗錄扈從將士之勞,欲行賞賚,而帑藏空竭,議貸民財以與之。守道曰:「人罹虐政,方喜更生,今仁恩未及,而徵斂遽出,如群望何,寧出宮中所有,無取於民。」遂從其言。契丹叛,遼東猛安謀克在其境者,或附從之,朝議欲徙之內地,守道
 極陳其不可。右副元帥謀衍將兵討賊,不即擊,守道力言於朝,詔遣僕散忠義、紇石烈志寧往代之,東方以平。



 大定二年,宮中十六位火,方事完葺,時已入夏,頗妨民力,守道諫而罷。未幾,改太子詹事,兼右諫議大夫,馳驛規畫山東兩路軍糧,及賑民饑。守道籍大姓戶口,限以歲儲,使盡輸其贏入官,復給其直,以是軍民皆足。拜參知政事、兼太子少保,守道懇辭,世宗諭之曰:「乃祖勳在王室,朕亦悉卿忠謹,以是擢用,無為多讓。」時契丹餘黨未附者尚眾,北京、臨潢、泰州民不安,詔守道佩金符往安撫之,給群牧馬千疋,以備軍用。守道招致契丹骨迭
 聶合等內附,民以寧息。還進尚書左丞,兼太子少師。嘗從獵近郊,有虎傷獵夫,帝欲親射之,守道叩馬極諫而止。俄拜平章政事。十四年,宋人遣使因陳請手接書事,左丞石琚等議從其請,帝意未決,守道等以為不可許,帝卒從之,詳在《紇石烈良弼傳》中,既而,遷右丞相,監修國史,復遷左丞相,授世襲謀克。



 二十年,修《熙宗實錄》成,帝因謂曰:「卿祖谷神行事有未當者,尚不為隱,見卿直筆也。」尋請避賢路,帝不許。進拜太尉、尚書令,改授尚書左丞相,諭之曰:「丞相之位不可虛曠,須用老成人,故復以卿處之,卿宜悉此。」未幾,復乞致仕,帝曰:「以卿先朝勳
 臣之後,特委以三公重任,自秉政以來,效竭忠勤,朕甚嘉之。今引年求退,甚得宰相體,然未得代卿者,以是難從,汝勉之哉。」二十五年,坐擅支東宮諸皇孫食廩,奪官一階。尋改兼太子太師,特錄其子珪襲謀克,充符寶祗候。章宗為原王,詔習騎鞠,守道諫曰:「哀制中未可。」帝曰:「此習武備耳,自為之則不可,從朕之命,庸何傷乎?然亦不可數也。」二十六年,懇求致仕,優詔許之,特賜宴於慶春殿,帝手飲以卮酒,錫與甚厚,以其子珪侍行,又賜次子璋進士第。明昌四年卒,年七十四。上聞之震悼,遣其弟點檢司判官蒲帶致祭,賻銀千兩、重彩五十端、絹五
 百疋。太常議謚曰簡憲,上改曰簡靖,蓋重其能全終始云。



 石琚,字子美,定州人。沉厚好學。父皋,補郡吏,廉潔自將,稱為長者。從魯王闍母攻青州,州人堅守不降。闍母怒之,及城破,命皋計州民之數,將使諸軍分掠有之,皋緩其事。闍母讓之,皋曰:「大王將為朝廷撫定郡縣,當使百姓按堵,無或侵苦之。若取城邑而殘其民,則未下者必死守以拒我。皋之稽緩,安敢逃罪。」闍母感悟,乃下令曰:「敢有犯州人者,以軍法論。」指其坐謂皋曰:「汝之子孫必有居此坐者。」皋隨守定州,唐縣人王八謀為亂,書其縣
 人姓名于籍,無慮數千人,其黨持其籍詣州發之,皋主鞫治。是時冬月,皋抱籍上事,佯為頓仆,覆其籍爐火中,盡焚之,不可復得其姓名,止坐為首者,餘皆得釋。



 琚生七歲,讀書過目即成誦。既長,博通經史,工詞章。天眷二年,中進士第一,再調弘政、邢臺縣令。邢守貪暴屬縣,掊取民財,以奉所欲,琚獨一物無所與。既而守以贓敗,他令佐皆坐累,琚以廉辦,改秀容令。復擢行臺禮部主事,召為左司都事。累遷吏部郎中。貞元三年,以父喪去官,尋起復為本部侍郎。世宗舊聞其名,大定二年,擢左諫議大夫,侍郎如故。奉命詳定制度,琚上疏六事,大概
 言正紀綱,明賞罰,近忠直,遠邪佞,省不急之務,罷無名之役。上嘉納之。遷吏部尚書。琚自員外郎至尚書,未嘗去吏部,且十年。典選久,凡宋、齊換授官格,南北通注銓法,能僂指而次第之,當時號為詳明。頃之,拜參知政事,琚辭讓再三,上曰:「卿之材望,無不可者,何以辭為。」右丞蘇保衡監護十六位工役,詔共典其事,給銀牌二十四,許從宜規畫。上謂琚曰:「此役不欲煩民,丁匠皆給雇直,毋使貪吏夤緣為姦利,以興民怨。卿等勉力,稱朕意焉。」徒單合喜定陜西,琚請曲赦秦、隴,以安百姓,上從之。丁母憂,尋起復,進拜尚書右丞。天長觀災,詔有司營繕,有
 司闢民居以廣大之,費錢三十萬貫。蔚州采地蕈,役數百千人。琚奏之,上曰:「自今凡稱御前者,皆稟奏。」琚與孟浩對曰:「聖訓及此,百姓之福也。」是時,議禁網捕狐、兔等野物,累計其獲,或至徒罪,琚奏曰:「捕禽獸而罪至徒,恐非陛下意,杖而釋之可也。」上曰:「然。」久之,進拜左丞,兼太子少師。上問宰相:「古有居下位能憂國為民直言無忌者,今何以無之?」琚對曰:「是豈無之,但未得上達耳。」上曰:「宜盡心采擢之。」



 世宗將行郊祀,議配享,琚曰:「配者,侑神作主也。自外至者無主不止,故推祖考以配天,同尊之也。《孝經》曰:『郊祀后稷以配天。』漢、魏、晉皆以一帝配之。唐
 高宗始以高祖、太宗崇配。垂拱初,以高祖、太宗、高宗並配。玄宗開元十一年,罷同配之禮,以高祖配。宋太宗時,以宣祖、太祖配。真宗時以太祖、太宗配。仁宗時,有司請以三帝並侑,遂以太祖、太宗、真宗並配。其後禮院議對越天地、神無二主,當以太祖配。此唐、宋變古以三帝配天,終竟依古以一祖配也。將來親郊合依古禮,以一祖配之。」上曰:「唐、宋不足為法,止當奉太祖皇帝配之。」琚嘗請命太子習政事,或譖之曰:「琚希恩東宮。」世宗察其無他,以此言告之,琚對曰:「臣本孤生,蒙陛下拔擢,備位執政,兼師保之任。臣愚以為太子天下之本,當使知民事,
 遂言及之。」因乞解少師。十年二月,祭社,有司奏請御署祝版,上問琚曰:「當署乎?」琚曰:「故事有之。」上曰:「祭祀典禮,卿等慎之,無使後世譏誚。熙宗尊謚太祖,宇文虛中定禮儀,以常朝服行事。當時朕雖童稚,猶覺其非。」琚曰:「祭祀,大事也,非故事不敢行。」



 上謂琚曰:「女直人往往徑居要達,不知閭閻疾苦。卿嘗為丞簿,民間何事不知,凡利害極陳之。」上與宰臣議鑄錢,或以鑄錢工費數倍,欲採金銀坑冶,上曰:「山澤之利可以與民,惟錢幣不當私鑄。若財貨流布四方,與在官何異。」琚進曰:「臣聞天子之富藏於天下,正如泉源欲其流通耳。」上問琚曰:「古亦有百
 姓鑄錢者乎?」對曰:「使百姓自鑄,則小人圖厚利,錢愈薄惡,古所以禁也。」



 時民間往往造作妖言,相為黨與謀不軌,事覺伏誅。上問宰臣曰:「南方尚多反側,何也?」琚對曰:「南方無賴之徒,假託釋道,以妖幻惑人。愚民無知,遂至犯法。」上曰:「如僧智究是也。此輩不足恤,但軍士討捕,利取民財,害及良民,不若杜之以漸也。」智究,大名府僧,同寺僧苑智義與智究言,《蓮華經》中載五濁惡世佛出魏地,《心經》有夢想究竟涅槃之語,汝法名智究,正應經文,先師藏瓶和尚知汝有是福分,亦作頌子付汝。智究信其言,遂謀作亂,歷大名、東平州郡,假託抄化,誘惑愚民,
 潛結姦黨,議以十一年十二月十七日先取兗州,會徒嶧山,以「應天時」三字為號,分取東平諸州府。及期向夜,使逆黨胡智愛等,劫旁近軍寨,掠取甲仗,軍士擊敗之。會傅戩、劉宣亦於陽穀、東平上變。皆伏誅,連坐者四百五十餘人。



 宗室子或不勝任官事,世宗欲授散官,量與廩祿,以贍足之,以問宰臣曰:「於前代何如?」琚對曰:「堯親九族,周家內睦九族,皆帝王盛事也。」琚之將順,多此類。



 十三年,上表乞致仕。十六年,再表乞致仕。皆不許。參知政事唐括安禮忤上意,出為橫海軍節度使,數年不復召。琚對便殿,從容進曰:「唐括安禮忠直,久在外官。」世宗
 深然之,遂自南京留守召為尚書右丞。琚嘗舉室紹先以為右司員外郎,紹先中風暴卒,上甚惜之,謂琚曰:「卿之所舉也。」感歎者再三。



 十七年,拜平章政事,封莘國公。明年,拜右丞相。修起居注移剌傑上書言:「朝奏屏人議事,史官亦不與聞,無由紀錄。」上以問宰相,琚與右丞唐括安禮對曰:「古者史官,天子言動必書,以儆戒人君,庶幾有畏也。周成王剪桐葉為圭,戲封叔虞,史佚曰:『天子不可戲言,言則史書之。』以此知人君言動,史官皆得記錄,不可避也。」上曰:「朕觀《貞觀政要》,唐太宗與臣下議論,始議如何,後竟如何,此政史臣在側記而書之耳。若恐
 漏泄幾事,則擇慎密者任之。」朝奏屏人議事,記注官不避自此始。



 以年老衰病固辭,上曰:「朕知卿年老,勉為朕留,俟一二年,朕將思之。」上謂宰臣曰:「朕為天子,未嘗敢專行獨斷,每事遍問卿等,可行則行之,不可則止也。」琚與平章政事唐括安禮奏曰:「好問則裕,自用則小,陛下行之,天下幸甚。」居一年,復表致仕,乃許。詔以一孫為閣門祗候。即命駕歸鄉里。久之,世宗謂宰臣:「知人最為難事,近來左選多不得人。惟石琚為相時,往往舉能其官,左丞移剌道、參政粘割斡特剌舉右選,頗得之。朕常以不能遍識人材為不足。此宰相事也,左右近侍雖常有
 言,朕未敢輕信。」又曰:「近日刺史縣令多闕員,當擇幹濟者除之,資級不到庸何傷。」又曰:「惟石琚最為知人。」



 唐括鼎為定武軍節度使,上謂鼎曰:「久不見石琚,精力比舊何如?汝到官往視之。」顯宗亦思之,因琚生日,寄詩以見意。二十二年,以疾薨于家,年七十二。謚文憲。泰和元年,圖像衍慶宮,配享世宗廟廷。



 唐括安禮,本名斡魯古,字子敬。好學,通經史,工詞章,知為政大體。貞元中,累官臨海軍節度使,入為翰林侍讀學士,改浚州防禦使、彰化軍節度使。大定初,遷益都尹,召為大興尹,上曰:「京師好訛言。府中姦吏為民患。卿雖
 年少,有治才,去其宿弊,毋為因仍。」察廉入第一等,進階榮祿大夫。



 七年五月,大興府獄空,詔錫宴勞之。凡州郡有獄空者,皆賜錢為錫宴費,大興府錫宴錢三百貫,其餘有差。久之,拜參知政事,罷為橫海軍節度使,歷河間尹、南京留守。以喪去官,起復尚書右丞。詔曰:「南路女直戶頗有貧者,漢戶租佃田土,所得無幾,費用不給,不習騎射,不任軍旅。凡成丁者簽入軍籍,月給錢米,山東路沿邊安置。其議以聞。」浹旬,上問曰:「宰臣議山東猛安貧戶如之何?」奏曰:「未也。」乃問安禮曰:「於卿意如何?」對曰:「猛安人與漢戶,今皆一家,彼耕此種,皆是國人,即日簽軍,
 恐妨農作。」上責安禮曰:「朕謂卿有知識,每事專傚漢人。若無事之際可務農作,度宋人之意且起爭端,國家有事,農作奚暇?卿習漢字,讀《詩》、《書》,姑置此以講本朝之法。前日宰臣皆女直拜,卿獨漢人拜,是邪非邪?所謂一家者,皆一類也,女直、漢人,其實則二。朕即位東京,契丹、漢人皆不往,惟女直人偕來,此可謂一類乎?」又曰:「朕夙夜思念,使太祖皇帝功業不墜,傳及萬世,女直人物力不困。卿等悉之。」因以有益貧窮猛安人數事,詔左司郎中粘割斡特剌使書之,百官集議於尚書省。



 十七年,詔遣監察御史完顏覿古速行邊,從行契丹押剌四人,挼剌、
 招得、雅魯、斡列阿,自邊亡歸大石。上聞之,詔曰:「大石在夏國西北。昔窩斡為亂,契丹等響應,朕釋其罪,俾復舊業,遣使安輯之,反側之心猶未已。若大石使人間誘,必生邊患。遣使徙之,俾與女直人雜居,男婚女聘,漸化成俗,長久之策也。」於是遣同簽樞密院事紇石烈奧也、吏部郎中裴滿餘慶、翰林修撰移剌傑,徙西北路契丹人嘗預窩斡亂者上京、濟、利等路安置。以兵部郎中移剌子元為西北路招討都監,詔子元曰:「卿可省諭徙上京、濟州契丹人,彼地土肥饒,可以生殖,與女直人相為婚姻,亦汝等久安之計也。卿與奧也同催發徙之。仍遣猛
 安一員以兵護送而東,所經道路勿令與群牧相近,脫或有變,即便討滅。俟其過嶺,卿即還鎮。」上已遣奧也、子元等,謂宰臣曰:「海陵時,契丹人尤被信任,終為叛亂,群牧使鶴壽、駙馬都尉賽一、昭武大將軍術魯古、金吾衛上將軍蒲都皆被害。賽一等皆功臣之後,在官時未嘗與契丹有怨,彼之野心,亦足見也。」安禮對曰:「聖主溥愛天下,子育萬國,不宜有分別。」上曰:「朕非有分別,但善善惡惡,所以為治。異時或有邊釁,契丹豈肯與我一心也哉。」



 他日,上又曰:「薦舉,大臣之職。外官五品猶得舉人,宰相無所舉,何也?」安禮對曰:「孔子稱才難。賢人君子,世不
 多有。陛下必欲得人,當廣取士之路,區別器使之,斯得人矣。」上曰:「除授格法不倫。奉職皆閥閱子孫,朕所知識,有資考出身月日。親軍不以門第收補,無廕者不至武義不得出職。但以女直人有超遷官資,故出職反在奉職上。天下一家,獨女直有超遷格,何也?」安禮對曰:「祖宗以來立此格,恐難輒改。」



 轉左丞,與右丞蒲察通同日拜,上謂之曰:「朕今年五十有五,若過六十,必倦於政事。宜及朕之康強,凡女直猛安謀克當修舉政事,改定法令。宗族中鮮有及朕之壽者,朕頗習女直舊風,子孫豈能知之,況政事乎。卿等宜悉此意。」上又曰:「大理寺事多留
 滯,宰執不督責之,何也?」安禮對曰:「案牘疑難者舊例給限。」上曰:「舊例是邪非邪,今不究其事,輒給以限邪?」參政移剌道曰:「臣在大理時,未嘗有滯事。」上曰:「卿在大理無滯事,為宰執而不能檢治,何也?」道無以對而退。上問宰臣曰:「御史臺官,亦與親知往來否?」皆曰:「往來殊少。」上曰:「臺官當盡絕人事。諫官、記注官與聞議論,亦不可與人游從。」安禮對曰:「親知之間,恐不可盡絕也。」上曰:「職任如是,何恤人之言。」



 進拜平章政事,封芮國公,授世襲謀克。上諭安禮,前代史書詳備,今祖宗實錄太簡略。對曰:「前代史皆成書,有帝紀、列傳。他日修史時,亦有帝紀、列傳,
 其詳自見于列傳也。」安禮嘗議科目,言于上曰:「臣觀近日士人不以策論為意。今若詩賦策論各場考試,文理俱優者為中選,以時務策觀其器識,庶得人也。」上曰:「卿等議之。」上謂宰臣曰:「賞有功不可緩,緩賞無以勸善。」安禮對曰:「古所謂賞不踰時者,正謂此也。」



 二十一年,拜右丞相,進封申國公,固辭曰:「臣備位宰相,無補於國家,夙夜憂懼,惟恐得罪,上負陛下,下負百姓。臣實不敢受丞相位,惟陛下擇賢於臣者用之。」上曰:「朕知卿正直,與左丞相習顯無異。且練習政事,無出卿之右者。其毋多讓。」安禮頓首謝。是歲,薨。泰和元年,配享世宗廟廷。



 移剌道,本名趙三。其先乙室部人也,初徙咸平。為人寬厚,有大志,以篤孝著名。通女直、契丹、漢字。皇統初,補刑部令史,轉尚書省令史,再遷大理司直。丁母憂,起復,遷戶部員外郎。正隆三年,徙臨潢、咸平路、畢沙河等三猛安,屯戍斡盧速。還奏,海陵謂侍臣曰:「道骨相異常,他日必登公輔。」明年,遷本部郎中。



 海陵伐宋,為都督府長史。海陵死,師還,無復紀律,士卒掠淮南,百姓苦之。有男女二百餘人,自願與道為奴,道受之,至淮,俟諸軍畢濟,乃悉遣還。大定二年,復為戶部郎中,與梁金求安撫山東,招諭盜賊。民或避盜避役者,並令歸業,不問罪名輕重皆
 原之,軍人不得並緣虜掠。僕散忠義討窩斡,道參謀幕府事。賊平,元帥府以俘獲生口分給官僚,道悉縱遣之。



 還京師,入見,既退,世宗目送之,曰:「此人有幹才,可大用也。」遷翰林直學士,兼修起居注。頃之,世宗曰:「道清廉有幹局,翰林文雅之職,不足以盡其才。」中都轉運繁劇,乃改同知中都路都轉運事。詔道送河北、山東等路廉察善惡升降官員制敕,上曰:「卿從討契丹,不貪俘獲,其志可嘉。故命卿為使。卿其勉之。」是歲,以廉升者,磁州刺史完顏蒲速列為北京副留守,濰州刺史蒲察蒲查為博州防禦使,威州刺史完顏兀答補為磁州刺史。治狀不
 善下遷者,登州刺史大磐為嵩州刺史,同知南京留守高德基為同知北京轉運事,衛州防禦使完顏阿鄰為陳州防禦使,真定尹徒單拔改為興平軍節度使,安國軍節度使唐括重國為彰化軍節度使。仍具功過善惡宣諭,毋受饋獻。遷大理卿。五年,宋人請和,罷兵。道往山東,閱實軍器,振贍戍兵妻子。再除同知大興尹。



 親軍百人長完顏阿思缽非禁直日帶刀入宮,其夜入左藏庫,殺都監郭良臣,盜取金珠。點檢司執其疑似者八人,掠笞三人死,五人者自誣,其贓不可得。上疑之,命道參問。道持久其獄,既而阿思缽鬻金事覺,伏誅。上曰:「箠楚之
 下,何求不得。奈何點檢司不以情求之乎!」賜掠死者錢人二百貫,周其家,不死者人五十貫。詔自今護衛親軍百人長、五十人長,非直日不得帶刀入宮。



 遷戶部尚書。上曰:「朕初即位,卿為戶部員外郎,聞卿孳孳為善,進卿郎中,果有可稱。及貳京尹,亦能善治。戶部經治國用,卿其勉之。」道頓首謝。改西北路招討使,賜金帶。故事,招討使到官,諸部皆獻駝馬,多至數百,道皆卻之,數月皆復貢職。父喪去官,起復參知政事。初,諸部有獄訟,招討司例遣胥吏按問,往往為姦利。道請專設一官,上嘉納之,招討司設勘事官自此始。上謂宰臣曰:「比聞大理寺斷
 獄,輒經旬月,何邪?」道奏曰:「在法,決死囚不過七日,徒刑五日,杖刑三日。」上曰:「法有程限,而輒違之,此官吏之責也,嚴戒約以去其弊。」進尚書右丞。乞致仕,上曰:「卿孝於家,忠於朕,通習法令政事,雖踰六十,心力未衰,未可退也。」乃除南京留守,賜通犀帶。上曰:「河南統軍烏古論思列為人少戇,凡邊事須與卿共議。卿以朕意諭思列也。」入拜平章政事。



 道弟臨潼令幼阿補犯罪至死,道待罪於家。皇太子生日,宴于慶和殿,上問道何故不在,參知政事粘割斡特剌奏曰:「其弟犯死刑,據制不合入內。」上曰:「此何傷也。」即詔道起視事。是時縣令多闕,上以問宰
 相,道奏曰:「散官宣武以上借除以充之。」上曰:「廉察八品以下已去官者,錄事丞簿有清幹之譽者,縣尉入優等者,皆與縣令。散官至五品,無貪汙曠職之名者,亦可與之。俟縣令不闕,即如舊制。」



 二十三年,罷為咸平尹,封莘國公。上曰:「卿數年前嘗乞致仕,朕不許卿。卿今老矣。咸平卿故鄉,地涼事少,老者所宜。」賜通犀帶。明日,復遣近侍曹淵諭旨曰:「咸平自窩斡亂後,民業尚未復舊,朕聽卿歸鄉里,所以安輯一境也。」二十四年,薨。上聞之,悼惜良久。是歲幸上京,道過咸平,遣使致祭,賻贈有加。詔圖像藏秘府,擢其子八狗為閣門祗候。



 光祖字仲禮,幼名八狗。以廕補閣門祗候,調平晉令、衛州都巡河、內承奉押班,累轉東上閤門使,兼典客署令。大安中,改少府少監。丁母憂,起復儀鸞局使,同知宣徽院使事,秘書監右宣徽使。興定二年十一月,詔集百官議所以為長久之利者,光祖等三人議曰:「募土人假以方面權任,俾人自勸,各保一方。」由是公府封建之論興焉,語在九公傳。三年,轉左宣徽使。五年,卒。



 贊曰:良弼、守道、琚、安禮、道,皆無聞正隆時,及其簉治朝,佐明主,諫行言聽,膏澤下於民,豈非遇其時邪。官序無闕,上下相安,君享其名,臣終其祿,可謂盛哉。海陵能知
 移剌道有公輔之器,而不能用,故其治績亦待大定而後著焉。人才之顯晦,有係於世道之污隆也,尚矣。金世內燕,惟親王公主駙馬得與,世宗一日特召琚入,諸王以下竊語,心蓋易之。世宗覺之,即語之曰:「使我父子家人輩得安然無事,而有今日之樂者,此人力也。」乃歷舉近事數十顯著為時所知者以曉之,皆俯伏謝罪。君臣相知如此,有不竭忠者乎!大定末,世宗將立元妃為后,以問琚,琚屏左右曰:「元妃之立,本無異辭,如東宮何?」世宗愕然曰:「何謂也?」琚曰:「元妃自有子,元妃立,東宮搖矣。」世宗悟而止。且人主家事,人臣之所難言者,許敬宗以
 一言幾亡唐祚,琚之對,其為金謀者至矣。



\end{pinyinscope}