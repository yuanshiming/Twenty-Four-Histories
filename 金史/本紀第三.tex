\article{本紀第三}

\begin{pinyinscope}

 太宗



 太宗體元應運世德昭功哲惠仁聖文烈皇帝,諱晟,本諱吳乞買,世祖第四子,母曰翼簡皇后拏懶氏,太祖母弟也。遼太康元年乙卯歲生。初為穆宗養子。收國元年十月,命為諳班勃極烈。太祖征伐,常居守。天輔五年,賜詔曰:「汝惟朕之母弟,義均一體,是用汝貳我國政。凡軍事違者,閱實其罪,從宜處之。其餘事無大小,一依本朝
 舊制。」



 天輔七年六月,太祖次鴛鴦濼,有疾。至斡獨山驛,召赴行在。詔曰:「今遼主盡喪其師,奔于夏國。遼官特列、遙設等劫其子雅里而立之,已留宗翰等措畫。朕親巡已久,功亦大就,所獲州部,政須綏撫,是用還都。



 八月中旬,可至春州,汝率內戚迎我,若至豹子崖尤善。」八月乙未,會于渾河北。戊申,太祖崩。



 九月乙卯,葬太祖于宮城西。國論勃極烈杲、鄆王昂、宗峻、宗乾率宗親百宮請正帝位,不許,固請,亦不許。宗乾率諸弟以赭袍被體,置璽懷中。丙辰,即皇帝位。己未,告祀天地。丙寅,大赦中外。改天輔七年為天會元年。癸酉,發春州粟,賑降人之徙于
 上京者。戊寅,詔諸猛安賦米,給戶口在內地匱乏者。南路軍帥闍母,敗張覺于樓峰口。



 十月壬辰,詔以空名宣頭百道給西南、西北兩路都統宗翰,曰:「今寄爾以方面,如當遷授必待奏請,恐致稽滯,其以便宜從事。」己亥,上京慶元寺僧獻佛骨,卻之。闍母及張覺戰于兔耳山,闍母敗績。



 十一月壬子,命宗望問闍母罪,以其兵討張覺。壬戌,復以空名宣頭及銀牌給上京路軍帥實古迺、婆盧火等。癸亥,宗望以闍母軍發廣寧,下瀕海諸郡縣。詔諭南京,割武、朔二州入于宋。婁室破朔州西山,擒其帥趙公直。勃堇斡魯別及勃剌速破走乙室白答於歸化。
 己巳,徙遷、潤、來、顯四州之民于瀋州。庚午,宗望及張覺戰于南京東,大敗之。張覺奔宋,城中人執其父及二子以獻,戮之軍中。壬申,張忠嗣、張敦固以南京降,遣使與張敦固入諭城中,復殺其使者以叛。己卯,詔女直人,先有附于遼,今復虜獲者,悉從其所欲居而復之。其奴婢部曲,昔雖逃背,今能復歸者,並聽為民。



 十二月辛巳,蠲民間貸息。詔以咸州以南,蘇、復州以北,年穀不登,其應輸南京軍糧免之。甲午,詔曰:「此聞民間乏食,至有鬻其子者,其聽以丁力等者贖之。」是日,以國論勃極烈杲為諳班勃極烈,宗乾為國論勃極烈。遣勃堇李靖如宋告
 哀。



 二年春正月庚戌朔,以謾都訶為阿捨勃極烈,參議國政。壬子,命賞宗望及將士克南京之功,赦闍母罪。甲寅,以空名宣頭五十、銀牌十給宗望。戊午,詔孛堇完顏阿實賚曰:「先帝以同姓之人有自育及典質其身者,命官為贖。今聞尚有未復者,其悉閱贖之。」癸亥,以東京比歲不登,詔減田租、市租之半。甲戌,西南、西北兩路都統宗翰、宗望請勿割山西郡縣與宋,上曰:「是違先帝之命也,其速與之。」夏國奉表稱籓,以下寨以北,陰山以南、乙室耶剌部吐祿濼西之地與之。丙子,貽宋書,索俘虜叛亡。
 丁丑,始自京師至南京每五十里置驛。



 二月,詔有盜發遼諸陵者,罪死。庚寅,詔命給宗翰馬七百匹、田種千石、米七千石,以賑新附之民。丁酉,命徙移懶路都勃堇完顏忠于蘇瀕水。乙巳,詔諭南京官僚,小大之事,必關白軍帥,無得專達朝廷。丙午,宗翰乞濟師,詔有司選精兵五千給之。丁未,命宗望,凡南京留守及諸闕員,可選勳賢有人望者就注擬之,具姓名官階以聞。



 三月己酉朔,命宗望以宋歲幣銀絹分賜將士之有功者。庚戌,叛人活孛帶降,詔釋之。宗望請選良吏招撫遷、潤、來、顯之民保山寨者,從之。己未,宗望以南京反覆,凡攻取之行,乞
 與知樞密院事劉彥宗裁決之。劉公胄、王永福棄家踰城來降,以公胄為廣寧尹,永福為奉先軍節度使。辛未,夏國王李乾順遣使上誓表。



 閏月戊寅朔,賜夏國誓詔。辛巳,命置驛上京、春、泰之間。己丑,烏虎里、迪烈底兩部來降。丙午,既許割山西諸鎮與宋,以宗翰言罷之。



 是月,斜野襲遙輦昭古牙,走之,獲其妻孥群從及豪族。勃堇渾啜等破奚七崖而撫其民人。



 四月己酉,以宗翰經略西夏及破遼功,賜以十馬,使自擇其二,餘以分諸帥。賑上京路、西北路降者及新徙嶺東之人。戊午,以實古迺所築上京新城名會平州。乙亥,詔贖上京路新遷寧江
 州戶口賣身者六百餘人。宋遣使來弔喪。以高術僕古等充遺留國信使,高興輔、劉興嗣等充告即位國信使,如宋。



 五月丁丑朔,上京軍帥實古乃以所獲印綬二十二及銀牌來上。癸未,詔曰:「新降之民,訴訟者眾,今方農時,或失田業,可俟農隙聽決。」丁亥,婆速路猛安僕盧古以贓罷,以謀克習泥烈代之。乙巳,曷懶路軍帥完顏忽剌古等言:「往者歲捕海狗、海東青、鴉、鶻於高麗之境,近以二舟往,彼乃以戰艦十四要而擊之,盡殺二舟之人,奮其兵杖。」上曰:「以小故起戰爭,甚非所宜。今後非奉命,毋輒往。闍母克南京,殺都統張敦固。



 七月壬午,皇子宗
 峻薨。丙戌,禁外方使介冗從多者。壬辰,鶻實答言:「高麗約吾叛亡,增其邊備,必有異圖。」詔曰:「納我叛亡而弗歸,其曲在彼。凡有通問,毋違常式。或來侵略,整爾行列,與之從事。敢先犯彼,雖捷必罰。」乙未,以烏虎部及諸營叛,以昊勃極烈昱等討平之。



 八月乙巳朔,以孛堇烏爪乃等為賀宋生辰使。丁巳,撒離改部猛安雛思以贓罷,以奚金家奴代之。六部都統撻懶擊走昭古牙,殺其隊將曷魯燥、白撒曷等。又破降駱駝山、金源、興中諸軍,詔增給銀牌十。



 十月甲辰朔,夏國遣使謝誓詔。戊午,天清節,宋、夏遣使來賀。甲子,詔發寧江州粟,賑泰州民被秋潦
 者。遙輦昭古牙率眾來降。興中府降。丙寅,詔有司運米五萬石于廣寧,以給南京、潤州戍卒。命南路軍帥闍母,以甲士千人益合蘇館路孛堇完顏阿實賚,以備高麗。戊辰,西南、西北兩路權都統斡魯言:「遼祥穩撻不野來奔,言耶律大石自稱為王,置南北官屬,有戰馬萬匹。遼主從者不過四千戶,有步騎萬餘,欲趨天德,駐餘都谷。」詔曰:「追襲遼主,必酌事宜。其討大石,則俟報下。」



 十一月癸未,闍母下宜州,拔杈丫山,殺節度使韓慶民。癸卯,詔以米五萬石給撻懶、實古迺。



 十二月戊申,以孛堇高居慶等為賀宋正旦使。



 三年正月癸酉朔,宋、夏遣使來賀。戊子,同知宣徽院事韓資正加尚書左僕射,為諸宮都部署。乙未,夏國遣使奠幣及賀即位。宋遣使賀即位。



 二月壬戌,婁室獲遼主於余睹谷,丁卯,以厖葛城地分授所徙烏虎里、迪烈底二部及契丹民。



 三月乙亥,阿捨勃極烈謾都訶薨。丙子,賑奚、契丹新附之民。辛巳,建乾元殿。斡魯獻傳國寶,以謀葛失來附,請授印綬。是日,賜完顏婁室鐵券。



 四月壬寅朔,詔以遼主赴京師。丁巳,南路軍帥察剌以罪罷。



 五月己丑,蕭八斤獲遼玉寶來獻。



 六月庚申,以獲遼主,遣李用和等充告慶使如宋。



 七月壬申,禁內外官、宗室毋私
 役百姓。己卯,南京帥以錦州野蠶成繭,奉其絲綿來獻,命賞其長吏。詔權勢之家毋買貧民為奴。其脅買者一人償十五人,詐買者一人償二人,皆杖一百。甲申,詔南京括官豪牧馬,以等第取之,分給諸軍。以耶律固等為宋報謝使。



 八月癸卯,斡魯以遼主至京師。甲辰,告于太祖廟。丙午,遼主延禧入見,降封海濱王。壬子,詔有司揀閱善射勇健之士以備宋。



 九月壬午,廣寧府獻嘉禾。癸巳,保州路都孛堇加古撒曷有罪伏誅,以孛堇徒單烏烈代之。



 十月甲辰,詔諸將伐宋。以諳班勃極烈杲兼領都元帥,移賚勃極烈宗翰兼左副元帥先鋒,經略使完
 顏希尹為元帥右監軍,左金吾上將軍耶律餘睹為元帥右都臨,自西京入太原。六部路軍帥撻懶為六部路都統,斜也副之,宗望為南京路都統,闍母副之,知樞密院事劉彥宗兼領漢軍都統,自南京入燕山。詔建太祖廟于西京。召耶魯赴京師教授女直字。戊申,有司言權南路軍帥鶻實荅官吏貪縱,詔鞫之。壬子,天清節,宋、夏遣使來賀。丁巳,以闍母為南京路都統,埽喝副之,宗望為闍母、劉彥宗兩軍監戰。壬戌,詔曰:「今大有年,無儲蓄則何以備饑饉,其令牛一具賦粟一石,每謀克為一廩貯之。」宋易州戍將韓民毅以軍降,處之蔚州。



 十一月庚
 辰,以降封遼主為海濱王詔中外。辛卯,南路軍帥司請禁契丹、奚、漢人挾兵器,詔勿禁。以張忠嗣權簽南京中書樞密院事。



 十二月庚子,宗翰下朔州。甲辰,宗望諸軍及宋郭藥師、張企徽、劉舜仁戰於白河。大破之。蒲莧敗宋兵於古北口。丙午,郭藥師降,燕山州縣悉平。戊申,宗翰克代州。乙卯,中山降。丙辰,宗望破宋兵五千于真定。戊午,宗翰圍太原。耶律餘睹破宋河東、陜西援兵于汾河北。甲子,宗望克信德府。



 四年春正月丁卯朔,始朝日。降臣郭藥師、董才皆賜姓完顏氏。戊辰,宗弼取湯陰,大抃攻下濬州,迪古補取黎
 陽。己巳,諸軍渡河。庚午,取滑州。宗望使吳孝民等入汴,問宋取首謀平山者童貫、譚稹、詹度及張覺等。宋太上皇帝出奔。癸酉,諸軍圍汴。甲戌,宋使李棁來謝罪,且請修好。宗望許宋修好,約質,割三鎮地,增歲幣,載書稱伯姪。戊寅,宋以康王構、少宰張邦昌為質。辛巳,宋上誓書、地圖,稱侄大宋皇帝、伯大金皇帝。癸未,諸軍解圍。



 二月丁酉朔,夜,宋將姚平仲兵四十萬來襲宗望營,敗之。己亥,復進師圍汴。宋使宇文虛中以書來,改以肅王樞為質,遣康王構歸。師還。壬子,以滑、浚二州與宋。宗翰定威勝軍,攻下隆德府。丁巳,次澤州。海濱王家奴誣其主欲亡
 去,詔誅其首惡,餘並杖之。



 三月癸未,銀術可圍太原,宗翰還西京。



 四月癸卯,宗望使宗弼來奏捷。乙丑,耿守忠等大敗宋兵于西都谷。



 五月辛未,宋種師中以兵出井陘。癸酉,完顏活女敗之於殺熊嶺,斬師中於陣。是日,拔離速敗宋姚古軍於隆州谷。



 六月丙申朔,高麗國王王楷奉表稱籓。庚戌,宗望獻所獲三象。庚申,以宗望為右副元帥。



 七月丙寅,遣高伯淑等宣諭高麗。壬申,出金牌,命孛堇大抃以所領渤海軍八猛安為萬戶。戊子,以鐵勒部長奪離剌不從其兄夔里本叛,賜馬十一、豕百、錢五百萬。蕭仲恭使宋還,以所持宋帝與耶律餘睹蠟書
 自陳。



 八月庚子,詔左副元帥宗翰、右副元帥宗望伐宋。宋張灝率兵出汾州,拔離速擊走之。劉臻以兵出壽陽,婁室破之。庚戌,宗翰發西京。辛亥,婁室等破宋張灝軍于文水。癸丑,宗望發保州。是日,耶律鐸破宋兵于雄州,那野等敗宋兵于中山。甲寅,新城縣進白烏。庚申,突拈取新樂。



 九月丙寅,宗翰克太原,執經略使張孝純。鶻沙虎取平遙、靈石、孝義、介休諸縣。己巳,復以南京為平州。辛未,宗望破宋種師閔於井陘,取天威軍,克真定,殺其守李邈。



 十月,婁室克汾州,石州降。蒲察克平定軍,遼州降。丁未,天清節,高麗、夏遣使來賀,中京進嘉禾。



 十一
 月甲子,宗翰自太原趨汴。丙寅,宗望自真定趨汴。戊辰,宗翰下威勝軍。癸酉,撒剌荅破天井關。乙亥,宗翰克隆德府。活女渡盟津。西京、永安軍、鄭州皆降。庚辰,宗翰克澤州。宗望諸軍渡河,臨河、大名二縣、德清軍、開德府皆下。丙戌,克懷州。是日,宗望至汴。



 閏月壬辰朔,宋出兵拒戰,宗望等擊敗之。癸巳,宗翰至汴。丙辰,克汴城。庚申,以高隨充高麗生日使。辛酉,宋主桓出居青城。



 十二月癸亥,宋主桓降,是日,歸于汴城。庚辰,詔曰:「朕惟國家,四境雖遠而兵革未息,田野雖廣而畎畝未闢,百工略備而祿秩未均,方貢僅修而賓館未贍。是皆出乎民力,茍不
 務本業而抑游手,欲上下皆足,其可得乎?其令所在長吏,敦勸農功。」



 五年正月辛卯朔,高麗、夏遣使來賀。癸巳,宗翰、宗望使使以宋降表來上。乙未,知樞密院事劉彥宗上表,請復立趙氏,不聽。丁巳,回鶻喝里可汗遣使入貢。



 二月丙寅,詔降宋二帝為庶人。三月丁酉,立宋太宰張邦昌為大楚皇帝。割地賜夏國。四月乙酉,克陜府,取虢州。丙戌,以六部路都統撻懶為元帥左監軍,南京路都統闍母為元帥左都監。宗翰、宗望以宋二帝歸。己丑,詔曰:「合蘇館諸部與新附人民,其在降附之後同姓為婚者,離之。」五
 月庚寅朔,宋康王構即位於歸德。宋殺張邦昌。婁室降解、絳、慈、顯、石、河中、岢嵐,寧化、保德、火山諸城。撻懶徇地山東,下密州。迪虎下單州,廣信軍降。



 六月庚申,詔曰:「自河之北,今既分畫,重念其民或見城邑有被殘者,不無疑懼,遂命堅守。若即討伐,生靈可愍。其申諭以理,招輯安全之。儻執不移,自當致討。若諸軍敢利於俘掠輒肆蕩毀者,底於罰。」庚辰,右副元帥宗望薨。漢國王宗傑繼薨。



 七月甲午,賜宗翰券書,除反逆外,咸貰勿論。以石州戍將烏虎棄城喪師,杖之,削其宮。八月戊寅,以宋捷,遣耶律居謹等充宣慶使使高麗。丙戌,以宗輔為右副元
 帥。詔曰:「河北、河東郡縣職員多闕,宜開貢舉取士,以安新民。其南北進士,各以所業試之。」



 九月丁未,詔曰:「內地諸路,每耕牛一具賦粟五斗,以備歉歲。」辛亥,賜元帥右監軍完顏希尹、萬戶銀術可券書,除赦所不原,餘並勿論。闍母取河間,大敗宋兵于莫州,雄州降。撻懶克祁州,永寧軍、保州、順安軍皆降。



 冬十月丁卯,沙州回鶻活剌散可汗遣使入貢。辛未,天清節,高麗、夏遣使來賀。宋二帝自燕徙居于中京。



 十二月丙寅,右副元帥宗輔伐宋,徇地淄、青。烏林荅泰欲敗宋將李成於淄州。趙州降。阿里刮徇地濬州,敗敵兵,遂取滑州。乙亥,西南路都統斡
 魯薨。己卯,賽里下汝州。



 六年正月丙戌朔,高麗、夏遣使來賀。宗弼破宋鄭宗孟軍于青州。銀術可取鄧州。薩謀魯入襄陽。拔離速入均州。馬五取房州。癸巳,克青州。癸卯,闍母克濰州。丁未,迪古補敗宋將趙子昉兵。撒離喝敗宋于河上。甲寅,宋將馬括兵次樂安,宗輔擊敗之,聞宋主在維揚,以農時還師。宗弼敗宋兵于河上。



 二月乙卯朔,拔離速取唐州,癸亥,取蔡州。己巳,移剌古敗宋將臺宗雋等兵于大名。庚午,再破其軍,獲臺宗雋及宋忠。甲戌,拔離速取陳州。癸未,克穎昌府。鄭州叛入于宋,復取鄭州。遷洛陽、襄陽、
 穎昌、汝、鄭、均、房、唐、鄧、陳、蔡之民于河北。宗翰復遣婁室攻下同、華、京兆、鳳翔,擒宋經制使傅亮。阿鄰破河中。斡魯入馮翊。



 三月壬辰,命南路軍帥實古迺,籍節度使完顏慎思所領諸部及未置猛安謀克戶來上。己酉,撻懶下恩州。五月戊戌,移沙土古思以本部來附。六月己未,詔求祖宗遺事。撻懶遣兵徇下磁州、信德府。真定賊自稱元帥,秦王撒離喝討平之。七月乙巳,宋主遣使奉表請和,詔進兵伐之。以宋二庶人赴上京。八月乙卯,婁室敗宋兵于華州,訛特剌破敵于渭水,遂取下邽。丁丑,以宋二庶人素服見太祖廟,遂入見于乾元殿。封其父昏
 德公、子重昏侯。是日,告于太祖廟。以州郡職員名稱及俸給因革詔中外。



 九月辛丑,繩果等敗宋兵于蒲城,甲申,又破敵于同州。乙丑,取丹州。十月丙寅,天清節,高麗、夏遣使來賀。癸酉,知樞密院事劉彥宗薨。丁丑,蒲察、婁室敗宋兵于臨真。戊寅,徙昏德公、重昏侯于韓州。庚辰,宗翰、宗輔會于濮,伐宋。



 十一月庚寅,蒲察、婁室取延安府。壬辰,賑移懶路。乙未,取濮州,綏德軍降。婁室再攻晉寧軍,其守徐徽言固守,不能克。十二月丙辰,宗弼取開德府。丁卯,宗輔克大名府。鶻沙虎敗宋兵于鞏。



 七年正月庚辰朔,高麗、夏遣使來賀。辛巳,吳國王闍母
 薨。甲午,以西京留守韓企先同中書門下平章事、知樞密院事。



 二月戊辰,宋麟府路安撫使折可求以麟、府、豐三州降。己巳,婁室、塞里、鶻沙虎等破晉寧軍,其守徐徽言據子城拒戰。庚午,率眾潰圍走,擒之。使之拜,不拜。臨之以兵,不動。命降將折可求諭之降,指可求大罵,出不遜語,遂殺之。其統制孫昂及士卒皆不屈,盡殺之。甲戌,詔禁醫巫閭山遼代陵樵採。



 三月己卯朔,日中有黑子。壬寅,詔軍興以來,良人被略為驅者,聽其父母夫妻子贖之。尚書左僕射高楨罷。四月,蒲察、婁室取鄜、坊二州。五月乙卯,拔離速等襲宋主于揚州。九月丙午朔,日
 有食之。庚午,宗弼敗宋兵于睢陽。辛未,降其城。是月,曹州降。十月丙子朔,京兆府降。丁丑,鞏州降。庚寅,天清節,高麗、夏遣使來賀。丁酉,阿里、當海、大抃破敵于壽春。己亥,安撫使馬世元以城降。甲辰,廬州降。十一月庚戌,徙曷蘇館都統司治寧州。乙卯,高麗遣使來貢。丙辰,宗弼取和州。壬戌,宗弼渡江,敗宋副元帥杜充軍于江寧。丁卯,守臣陳邦光以城降。



 十二月丙戌,宗弼取湖州。丁亥,克杭州。阿里、蒲盧渾追宋主于明州。越州降。大抃敗宋樞密使周望於秀州,又敗宋兵于杭州東北。戊戌,阿里、蒲盧渾敗宋兵于東關,遂濟曹娥江。壬寅,敗宋兵于高
 橋。宋主入于海。



 八年正月甲辰朔,高麗、夏遣使來賀。丁巳,以同中書門下平章事韓企先為尚書左僕射兼侍中。己未,阿里、蒲盧渾克明州,執其守臣趙伯諤。庚申,詔曰「避役之民,以微直鬻身權貴之家者,悉出還本貫。」阿魯補、斜里也下太平、順昌及濠州。



 是月,宋副元帥杜充以其眾降。二月乙亥,宗弼還自杭州。庚寅,取秀州。戊戌,取平江。汴京亂,三月丁卯,大迪里復取之。宗弼及宋韓世忠戰于鎮江,不利。四月丙申,復戰于江寧,敗之。諸軍渡江。是日,阿魯補戰于拓皋,己亥,周企戰于壽春,辛丑,婁室戰于淳化,
 皆勝之。醴州降,遂克邠州。



 五月癸卯,禁私度僧尼及繼父繼母之男女無相嫁娶。戊申,詔曰:「河北、河東簽軍。其家屬流寓河南被俘掠為奴婢者,官為贖之,俾復其業。」六月壬申,詔遣遼統軍使耶律曷禮質、節度使蕭別離剌等十人。分治新附州鎮。癸酉,詔以昏德公六女為宗婦。七月辛亥,詔給泰州都統婆盧火所部諸謀克甲胄各五十。先遣婁室經略陜西,所下城邑叛服不常,其監戰阿盧補請益兵。帥府會諸將議曰:「兵威非不足,綏懷之道有所未盡。誠得位望隆重、恩威兼濟者以往,可指日而定。若以皇子右副元帥宗輔往,為宜。」以聞。詔曰:「婁
 室往者所向輒克,今使專征陜西,淹延未定,豈倦于兵而自愛耶?關、陜重地,卿等其戮力焉。」丁卯,上如東京溫湯。徙昏德公、重昏侯于鶻里改路。



 九月戊申,立劉豫為大齊皇帝,世修子禮,都大名府。辛酉,諳班勃極烈、都元帥杲薨。癸亥,宗輔等敗宋張浚軍于富平。耀州降。乙丑,鳳翔府降。十月乙亥,上至自東京。齊帝劉豫遣使謝封冊。甲申,天清節,齊、高麗、夏遣使來賀。以鐵驪突離剌同中書門下平章事。詔遼、宋官上本國誥命,等第換授。



 十一月甲辰,宗輔下涇州。丁未,渭州降。敗宋劉倪軍于瓦亭。戊申,原州降。宋涇原路統制張中孚、知鎮戎軍李彥
 琦以眾降。馬五等擊宋吳玠軍于隴州。庚戌,以遙鎮節度使烏克壽等為齊劉豫生日使。癸亥,宗輔以陜西事狀聞,詔獎諭之。十二月丁丑,完顏婁室薨。乙酉,宗輔敗宋劉維輔軍。壬辰,熙州降。



 九年正月己亥朔,齊、高麗、夏遣使來賀。戊申,命以徒門水以西,渾畽、星顯、僝蠢三水以北閑田,給曷懶路諸謀克。辛亥,蒲察鶻拔魯、完顏忒里討張萬敵於白馬湖,陷于敵。癸丑,以同中書門下平章事時立愛為侍中,知樞密院張忠嗣為宣政殿大學士、知三司使事。宗弼、阿盧補撫定鞏、洮、河、樂、西寧、蘭、廓、積石等州。涇原、熙河兩路
 皆平。



 四月己卯,詔「新徙戍邊戶,匱於衣食,有典質其親屬奴婢者,官為贖之。戶計其口而有二三者,以官奴婢益之,使戶為四口。又乏耕牛者,給以官牛,別委官勸督田作。戍戶及邊軍資糧不繼,傘粟于民而與賑恤。其續遷戍戶在中路者,姑止之,即其地種藝,俟畢獲而行,及來春農時,以至戍所。」五月丙午,分遣使者諸路勸農。



 六月壬辰,賜昏德公、重昏侯時服各兩襲。八月辛巳,回鶻隈欲遣使來貢。九月己酉,和州回鶻執耶律大石之黨撒八、迪里、突迭來獻。十月戊寅,天清節,齊、高麗、夏遣使來賀。撒離喝攻下慶陽。慕洧以環州降。宗弼與宋吳玠
 戰於和尚原,敗績。十一月己未,遷趙氏疏屬于上京。以陜西地賜齊。



 十年正月癸巳朔,齊、高麗、夏遣使來賀。己酉,齊表謝賜地。壬子,詔曰:「昔遼人分士庶之族,賦役皆有等差,其悉均之。」二月庚午,賑上京路戍邊猛安民。



 四月丁卯,詔:「諸良人知情嫁奴者,聽如故為妻,其不知而嫁者,去住悉從所欲。」移賚勃極烈,左副元帥宗翰朝京師。庚午,以太祖孫亶為諳班勃極烈,皇子宗磐為國論忽魯勃極烈,國論勃極烈宗乾為國論左勃極烈,移賚勃極烈、左副元帥宗翰為國論右勃極烈兼都元帥,右副元帥宗輔
 為左副元帥。庚寅,聞鴨綠、混同江暴漲,命賑徙戍邊戶在混同江者。



 閏月辛卯,詔分遣鶻沙虎等十三人閱諸路丁壯,調赴軍。七月甲午,賑泰州路戍邊戶。上如中京。九月,元帥右都監耶律餘睹謀反,出奔。其黨燕京統軍使蕭高六伏誅,蔚州節度使蕭特謀葛自殺。



 十月壬寅,天清節,大赦。齊、高麗、夏遣使來賀。上如興中府。齊使使來告母喪。十一月癸亥,以武良謨為齊弔祭使。癸未,撒離喝請取劍外十三州,從之。部族節度使土古廝捕斬余睹及其諸子,函其首來獻。十二月庚子,撒離喝克金州。上至自興中府。



 十一年正月丁巳朔,齊、高麗、夏遣使來賀。丁卯,撒離喝敗吳玠於饒峰關。戊辰,取洋州。甲戌,入興元府。二月己亥,元帥府言:「承詔賑軍士,臣恐有司錢幣將不繼。請自元帥以下有祿者出錢助給之。」詔曰:「官有府庫而取於臣下,此何理耶?其悉從官給。」八月甲申,黃龍府置錢帛司。戊子,趙咢誣告其父昏德公謀反,咢及其婿劉文彥伏誅。戊戌,詔曰:「比以軍旅未定,嘗命帥府自擇人授官,今並從朝廷選注。」十月丙申,天清節,齊、高麗、夏遣使來賀。十一月丙寅,賑移懶路。宗弼克和尚原。十二月癸未,賑曷懶路。



 十二年正月辛亥朔,齊、高麗、夏遣使來賀。甲子,初改定制度,詔中外。丙寅,如東京。二月丁酉,撒離喝敗宋吳玠軍于固鎮。四月,至自東京。六月甲午,以阿盧補為元帥右都監。十月庚寅,天清節,齊、高麗、夏遣使來賀。



 十三年正月丙午朔,日有食之。己巳,上崩於明德宮,年六十一。庚午,諳班勃極烈即皇帝位于柩前。三月庚辰,上尊謚曰文烈皇帝,廟號太宗。乙酉,葬和陵。皇統四年,改號恭陵。五年,增上尊謚曰體元應運世德昭功哲惠仁聖文烈皇帝。貞元三年十一月戊申,改葬于大房山,
 仍號恭陵。



 贊曰:天輔草創,未遑禮樂之事。太宗以斜也、宗乾知國政,以宗翰、宗望總戎事。既滅遼舉宋,即議禮制度,治歷明時,纘以武功,述以文事,經國規摹,至是始定。在位十三年,宮室苑無所增益。末聽大臣計,傳位熙宗,使太祖世嗣不失正緒,可謂行其所甚難矣!



\end{pinyinscope}