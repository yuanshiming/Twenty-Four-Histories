\article{列傳第十五}

\begin{pinyinscope}

 ○宗弼本名兀術亨本名孛迭張邦昌劉豫撻懶



 宗弼,本名斡啜,又作兀術,亦作斡出,或作晃斡出,太祖第四子也。希尹獲遼護衛習泥烈,問知遼帝獵鴛鴦濼。都統杲出青嶺,宗望、宗弼率百騎與馬和尚逐越盧、孛古、野里斯等,馳擊敗之。宗弼矢盡,遂奪遼兵士槍,獨殺
 八人,生獲五人,遂審得遼主在鴛鴦濼畋獵,尚未去,可襲取者。



 及宗望伐宋,宗弼從軍。取湯陰縣,降其卒三千人。至御河,宋人已焚橋,不得渡,合魯索以七十騎涉之,殺宋焚橋軍五百人。宗望遣吳孝民先入汴諭宋人,宗弼以三千騎薄汴城。宋上皇出奔,選百騎追之,弗及,獲馬三千而還。



 宗望薨,宗輔為右副元帥,徇地淄、青。宗弼敗宋鄭宗孟數萬眾,遂克青州。復破賊將趙成于臨朐,大破黃瓊軍,遂取臨朐。宗輔軍還,遇敵三萬眾于河上,宗弼擊敗之,殺萬餘人。詔伐宋康王,宗輔發河北,宗弼攻開德府,糧乏,轉攻濮州。前鋒烏林答泰欲破王善二
 十萬眾,遂克濮州,降旁近五縣。攻開德府,宗弼以其軍先登,奮擊破之。攻大名府,宗弼軍復先登,破其城。河北平。



 宋主自揚州奔于江南,宗弼等分道伐之。進兵歸德,城中有自西門北門出者,當海復敗之。乃絕隍築道,列炮隍上,將攻之。城中人懼,遂降。先遣阿里、蒲盧渾至壽春,宗弼軍繼之。宋安撫使馬世元率官屬出降。進降盧州,再降巢縣王善軍。當海等破酈瓊萬餘眾于和州,遂自和州渡江。將至江寧西二十里,宋杜充率步騎六萬來拒戰,鶻盧補、當海、迪虎、大抃合擊破之。宋陳邦光以江寧府降。留長安奴、斡里也守江寧。使阿魯補、斡里也
 別將兵徇地,下太平州、濠州及句容、溧陽等縣,溯江而西,屢敗張永等兵,杜充遂降。



 宗弼自江寧取廣德軍路,追襲宋主于越州。至湖州,取之。先使阿里、蒲盧渾趨杭州,具舟于錢塘江。宗弼至杭州,官守巨室皆逃去,遂攻杭州,取之。宋主聞杭州不守,遂自越奔明州。宗弼留杭州,使阿里、蒲盧渾以精兵四千襲之。訛魯補、術列速降越州。大抃破宋周汪軍,阿里、蒲魯渾破宋兵三千,遂渡曹娥江。去明州二十五里,大破宋兵,追至其城下。城中出兵,戰失利,宋主走入于海。宗弼中分麾下兵,會攻明州,克之。阿里、蒲盧渾泛海至昌國縣,執宋明州守趙伯
 諤,伯諤言:「宋主奔溫州,將自溫州趨福州矣。」遂行海追三百餘里,不及,阿里、蒲盧渾乃還。



 宗弼還自杭州,遂取秀州。赤盞暉敗宋軍於平江,遂取平江。阿里率兵先趨鎮江,宋韓世忠以舟師扼江口。宗弼舟小,契丹、漢軍沒者二百餘人,遂自鎮江溯流西上。世忠襲之,奪世忠大舟十艘,於是宗弼循南岸,世忠循北岸,且戰且行。世忠艨艟大艦數倍宗弼軍,出宗弼軍前後數里,擊柝之聲,自夜達旦。世忠以輕舟來挑戰,一日數接。將至黃天蕩,宗弼乃因老鸛河故道開三十里通秦淮,一日一夜而成,宗弼乃得至江寧。撻懶使移剌古自天長趨江寧援
 宗弼,烏林答泰欲亦以兵來會,連敗宋兵。



 宗弼發江寧,將渡江而北。宗弼軍渡自東,移剌古渡自西,與世忠戰於江渡。世忠分舟師絕江流上下,將左右掩擊之。世忠舟皆張五糸兩,宗弼選善射者,乘輕舟,以火箭射世忠舟上五糸兩,五糸兩著火箭,皆自焚,煙焰滿江,世忠不能軍,追北七十里,舟軍殲焉,世忠僅能自免。



 宗弼渡江北還,遂從宗輔定陜西。與張浚戰于富平。宗弼陷重圍中,韓常流矢中目,怒拔去其矢,血淋漓,以土塞創,躍馬奮呼搏戰,遂解圍,與宗弼俱出。既敗張浚軍于富平,遂與阿盧補招降熙河、涇原兩路。及攻吳玠于和尚原,抵險不可
 進,乃退軍。伏兵起,且戰且走。行三十里,將至平地,宋軍陣于山口,宗弼大敗,將士多戰沒。明年,復攻和尚原,克之。天會十五年,為右副元帥,封沈王。



 天眷元年,撻懶、宗磐執議以河南之地割賜宋,詔遣張通古等奉使江南。明年,宋主遣端明殿學士韓肖胄奉表謝,遣王倫等乞歸父喪及母韋氏兄弟。宗弼自軍中入朝,進拜都元帥。宗弼察撻懶與宋人交通賂遺,遂以河南、陜西與宋,奏請誅撻懶,復舊疆。是時,宗磐已誅,撻懶在行臺,復與鶻懶謀反。會置行臺於燕京,詔宗弼為太保,領行臺尚書省,都元帥如故,往燕京誅撻懶。撻懶自燕京南走,將亡
 入于宋,追至祁州,殺之。



 詔「諸州郡軍旅之事,決於帥府。民訟錢穀,行臺尚書省治之」。宗弼兼總其事,遂議南伐。太師宗乾以下皆曰:「構蒙再造之恩,不思報德,妄自鴟張,祈求無厭,今若不取,後恐難圖。」上曰:「彼將謂我不能奄有河南之地。且都元帥久在方面,深究利害,宜即舉兵討之。」遂命元帥府復河南疆土,詔中外。



 宗弼由黎陽趨汴,右監軍撒離喝出河中趨陜西。宋岳飛、韓世忠分據河南州郡要害,復出兵涉河東,駐嵐、石、保德之境,以相牽制。宗弼遣孔彥舟下汴、鄭兩州,王伯龍取陳州,李成取洛陽,自率眾取亳州及順昌府,嵩、汝等州相次皆
 下。時暑,宗弼還軍于汴,岳飛等軍皆退去,河南平,時天眷三年也。上使使勞問宗弼以下將士,凡有功軍士三千,並加忠勇校尉。攻嵐、石、保德皆克之。



 宗弼入朝,是時,上幸燕京,宗弼見於行在所。居再旬,宗弼還軍,上起立,酌酒飲之,賜以甲胄弓矢及馬二匹。宗弼已啟行四日,召還。至日,希尹誅。越五日,宗弼還軍,進伐淮南,克廬州。



 上幸燕京,宗弼朝燕京,乞取江南,上從之。制詔都元帥宗弼比還軍,與宰臣同入奏事。俄為尚書左丞相兼侍中,太保、都元帥、領行臺如故。詔以燕京路隸尚書省,西京及山後諸部族隸元帥府。乃還軍,遂伐江南。既渡淮,
 以書責讓宋人,宋人答書乞加寬宥。宗弼令宋主遣信臣來稟議,宋主乞「先斂兵,許弊邑拜表闕下」,宗弼以便宜約以畫淮水為界。上遣護衛將軍撒改往軍中勞之。



 皇統二年二月,宗弼朝京師,兼監修國史。宋主遣端明殿學士何鑄等進誓表,其表曰:「臣構言,今來畫疆,合以淮水中流為界,西有唐、鄧州割屬上國。自鄧州西四十里并南四十里為界,屬鄧州。其四十里外並西南盡屬光化軍,為弊邑沿邊州城。既蒙恩造,許備籓方,世世子孫,謹守臣節。每年皇帝生辰并正旦,遣使稱賀不絕。歲貢銀、絹二十五萬兩、匹,自壬戌年為首,每春季差人般送至
 泗州交納。有渝此盟,明神是殛,墜命亡氏,踣其國家。臣今既進誓表,伏望上國蚤降誓詔,庶使弊邑永有憑焉。」



 宗弼進拜太傅。乃遣左宣徽使劉筈使宋,以袞冕圭寶佩璲玉冊冊康王為宋帝。其冊文曰「皇帝若曰:咨爾宋康王趙構。不弔,天降喪于爾邦,亟瀆齊盟,自貽顛覆,俾爾越在江表。用勤我師旅,蓋十有八年于茲。朕用震悼,斯民其何罪。今天其悔禍,誕誘爾衷,封奏狎至,願身列于籓輔。今遣光祿大夫、左宣徽使劉筈等持節冊命爾為帝,國號宋,世服臣職,永為屏翰。嗚呼欽哉,其恭聽朕命。」仍詔天下。賜宗弼人口牛馬各千、駝百、羊萬,仍每歲
 宋國進貢內給銀、絹二千兩、匹。



 宗弼表乞致仕,不許,優詔答之,賜以金券。皇統七年,為太師,領三省事,都元帥、領行臺尚書省事如故。皇統八年,薨。大定十五年,謚忠烈,十八年,配享太宗廟廷。子孛迭。



 亨,本名孛迭。熙宗時,封芮王,為猛安,加銀青光祿大夫。天德初,加特進。海陵忌太宗諸子,將謁太廟,以亨為右衛將軍,語在《太宗諸王傳》。海陵賜良弓,亨性直,材勇絕人,喜自負,辭曰:「所賜弓,弱不可用。」海陵遂忌之,出為真定尹,謂亨曰:「太宗諸子方強,多在河朔、山東,真定據其衝要,如其有變,欲倚卿為重耳。」其實忌亨也。歷中京、東
 京留守。家奴梁遵告亨與衛士符公弼謀反,考驗無狀,遵坐誅。海陵益疑之。改廣寧尹,再任李老僧使伺察亨動靜,且令構其罪狀。



 亨初除廣寧,諸公主宗婦往賀其母徒單氏,太祖長女兀魯曰:「孛迭雖稍下遷,勿以為嫌,國家視京府一也,況孛迭年富,何患不貴顯乎!」是時,兀魯與徒單斜也為室,斜也妾忽撻得幸於徒單后,忽撻詣后,告「兀魯語涉怨望,且指斥,又言孛迭當大貴」。海陵使蕭裕鞫之,左驗皆不敢言,遂殺兀魯而杖斜也,免其官,以兀魯怨望,斜也不先奏聞故也。乃封忽撻為莘國夫人。



 久之,亨家奴六斤頗黠,給使總諸奴,老僧謂六斤
 曰:「爾渤海大族,不幸坐累為奴,寧不念為良乎!」六斤識其意。六斤嘗與亨侍妾私通,亨知之,怒曰:「必殺此奴!」六斤聞之懼,密與老僧謀告亨謀逆。亨有良馬,將因海陵生辰進之,以謂生辰進馬者眾,不能以良馬自異,欲他日入見進之。六斤言亨笑海陵不識馬,不足進。亨之奴有自京師來者,具言徒單阿里出虎誅死。亨曰:「彼有貸死誓券,安得誅之。」奴曰:「必欲殺之,誓券安足用哉。」亨曰:「然則將及我矣。」六斤即以為怨望,遂誣亨欲因間刺海陵。老僧即捕繫亨以聞。工部尚書耶律安禮、大理正忒里等鞫之,亨言嘗論鐵券事,實無反心,而六斤亦自引
 伏與妾私通,亨嘗言欲殺之狀。安禮等還奏,海陵怒,復遣與老僧同鞫之。與其家奴並加榜掠,皆不伏。老僧夜至亨囚所,使人蹴其陰間殺之。亨比至死,不勝楚痛,聲達於外。海陵聞亨死,佯為泣下,遣人諭其母曰:「爾子所犯法,當考掠,不意飲水致死。」



 亨擊鞠為天下第一,常獨當數人。馬無良惡,皆如意。馬方馳,輒投杖馬前,側身附地,取杖而去。每畋獵,持鐵連錘擊狐兔。一日與海陵同行道中,遇群豕,亨曰:「吾能以錘殺之。」即奮錘遙擊,中其腹,穿入之。終以勇力見忌焉。



 正隆六年,海陵遣使殺諸宗室,於是殺亨妃徒單氏、次妃大氏及子羊蹄等三人。
 大定初,追復亨官爵,封韓王。十七年,詔有司改葬亨及妻子。



 贊曰:宗弼蹙宋主于海島,卒定畫淮之約。熙宗舉河南、陜西以與宋人,矯而正之者,宗弼也。宗翰死,宗磐、宗雋、撻懶湛溺富貴,人人有自為之心,宗幹獨立,不能如之何,時無宗弼,金之國勢亦曰殆哉。世宗嘗有言曰:「宗翰之後,惟宗弼一人。」非虛言也。



 張邦昌,《宋史》有傳。天會四年,宗望軍圍汴,宋少帝請割三鎮地及輸歲幣、納質修好。於是,邦昌為宋太宰,與肅王樞俱為質以來。而少帝以書誘耶律餘睹,宗翰、宗望
 復伐宋,執二帝以歸。劉彥宗乞復立趙氏,太宗不許。宋吏部尚書王時雍等請邦昌治國事,天會五年三月,立邦昌為大楚皇帝。



 初,少帝以康王構與邦昌為質,既而肅王樞易之,康王乃歸。及宗望再舉兵,少帝復使康王奉玉冊玉寶袞冕,增上太宗尊號請和。康王至磁州,而宗望已自魏縣渡河圍汴矣。及二帝出汴州,從大軍北來,而邦昌至汴,康王入于歸德。邦昌勸進于歸德,康王已即位,罪以隱事殺之。



 邦昌死,太宗聞之,大怒,詔元帥府伐宋,宋主走揚州,事具宗翰等傳。其後,太宗復立劉豫繼邦昌,號大齊。



 劉豫,字彥游,景州阜城人。宋宣和末,仕為河北西路提刑。徙浙西。抵儀真,喪妻翟氏,繼值父憂。康王至揚州,樞密使張愨薦知濟南府。是時,山東盜賊滿野,豫欲得江南一郡,宰相不與,忿忿而去。撻懶攻濟南,有關勝者,濟南驍將也,屢出城拒戰,豫遂殺關勝出降。遂為京東東、西、淮南安撫使,知東平府兼諸路馬步軍都總管,節制河外諸軍。以豫子麟知濟南府。撻懶屯兵衝要,以鎮撫之。



 初,康王既殺張邦昌,自歸德奔揚州,詔左右副元帥合兵討之,詔曰:「俟宋平,當援立籓輔,以鎮南服,如張邦昌者。」及宋主自明州入海亡去,宗弼北還,乃議更立其
 人。眾議折可求、劉豫皆可立,而豫亦有心。撻懶為豫求封,太宗用封張邦昌故事,以九月朔旦授策。受策之後,以籓王禮見使者。臣宗翰、臣宗輔議:「既策為籓輔,稱臣奉表,朝廷報諭詔命,避正位與使人抗禮,餘禮並從帝者。」詔曰:「今立豫為子皇帝,既為鄰國之君,又為大朝之子,其見大朝使介,惟使者始見躬問起居與面辭有奏則立,其餘並行皇帝禮。」



 天會八年九月戊申,備禮冊命,立豫為大齊皇帝,都大名,仍號北京,置丞相以下官,赦境內。復自大名還居東平,以東平為東京,汴州為汴京,降宋南京為歸德府,降淮寧、永昌、順昌、興仁府俱為州。
 張孝純等為宰相,弟益為北京留守,母翟氏為皇太后,妾錢氏為皇后。錢氏,宣和內人也。以辛亥年為阜昌元年。以其子麟為尚書左丞相、諸路兵馬大總管。宋人畏之,待以敵國禮,國書稱大齊皇帝。豫宰相張孝純、鄭億年、李鄴家人皆在宋,宋人加意撫之。阜昌二年,豫遷都於汴。睿宗定陜西,太宗以其地賜豫,從張邦昌所受封略故也。



 元帥府使蕭慶如汴,與豫議以伐宋事,豫報曰:「宋主軍帥韓世忠屯潤州,劉光世屯江寧。今舉大兵,欲往采石渡江,而劉光世拒守江寧。若出宿州抵揚州,則世忠必聚海船截瓜洲渡。若輕兵直趨采石,彼未有備,
 我必徑渡江矣。光世海船亦在潤州,韓世忠必先取之,二將由此必不和。以此逼宋主,其可以也。」



 未幾,宋主閣門宣贊舍人徐文將大小船六十隻、軍兵七百餘人來奔,至密州界中,率將佐至汴。豫與元帥府書曰:「徐文一行,久在海中,盡知江南利害。文言:宋主在杭州,其候潮門外錢塘江內有船二百隻。宋主初走入海時,於此上船,過錢塘江別有河入越州,向明州定海口迤邐前去昌國縣。其縣在海中,宋人聚船積糧之處。今大軍可先往昌國縣,攻取船糧,還趨明州城下,奪取宋主御船,直抵錢塘江口。今自密州上船,如風勢順,可五日夜到昌
 國縣,或風勢稍慢,十日或半月可至。」



 初,宗弼自江南北還,宗翰將入朝,再議以伐宋事。宗翰堅執以為可伐。宗弼曰:「江南卑濕,今士馬困憊,糧儲未豐足,恐無成功。」宗翰曰:「都監務偷安爾。」及豫以書報,而睿宗亦不肯用豫策,使撻懶帥師至瓜洲而還。



 天會十四年,制詔「齊國與本朝軍民相訴,關涉文移,署年止用天會」。天會十五年,詔廢齊國,降封豫為蜀王。豫稱大號凡八年。於是,置行臺尚書省於汴,除去豫弊政,人情大悅。以故齊宰相張孝純權行臺左丞相,遂遷豫家屬於臨潢府。



 皇統元年,賜豫錢一萬貫、田五十頃、牛五十頭。二年,進封曹王。六
 年,薨。子麟。



 麟字元瑞,豫之子也。宋宣和間,父廕補將仕郎,累加承務郎。天會七年,豫以濟南降,麟因從軍,討水賊王江,破降之。豫節制東平,以麟知濟南府事。齊國建,以濟南為興平軍,麟為節度使、開府儀同三司、梁國公,充諸路兵馬大總管,判濟南府事。明年,為齊尚書左丞相。明年,從豫遷汴,罷判濟南,依前開府,聽置參謀。豫請立麟為太子,朝廷不許,曰:「若與我伐宋有功則立之。」於是,麟連歲帥兵南伐,皆無功而還。



 及朝廷議廢齊,報以南伐之期,俾豫先遣兵駐淮上。撻懶以軍廢豫,止刁馬河。麟從數百騎出迎,撻懶諭麟,止從騎南岸,獨召
 麟渡河,因執麟。豫廢,麟遷臨潢。頃之,授北京路都轉運使,歷中京、燕京路都轉運使、參知政事、尚書左丞,復為興平軍節度使、上京路轉運使、開府儀同三司,封韓國公。薨,年六十四。正隆間,降二品以上官封,改贈特進、息國公。



 昌,本名撻懶,穆宗子。宗翰襲遼主于鴛鴦濼,遼都統馬哥奔搗里,撻懶收其群牧。宗翰使撻懶追擊之,不及,獲遼樞密使得里底及其子磨哥、那野以還。太祖自將襲遼主于大魚濼,留輜重於草濼,使撻懶、牙卯守之。奚路兵官渾黜不能安輯其眾,遂以撻懶為奚六路軍帥鎮
 之。習古迺、婆盧火護送常勝軍及燕京豪族工匠自松亭關入內地,上戒之曰:「若遇險阨,則分兵以往。」習古乃、婆盧火乃合於撻懶。



 久之,討劾山速古部奚人。奚人據險戰,殺且盡,速古、啜里、鐵尼十三砦皆平之。詔曰:「朕以奚路險阻,經略為難,命汝往任其事,而克副所託,良用嘉歎。今回離保部族來附,餘眾奔潰,無能為已。比命習古迺、波盧火獲送降人,若遇險阻,即分兵以行,餘眾悉與汝合。降詔二十,招諭未降,汝當審度其事,從宜處之。」其後撫定奚部及分南路邊界,表請設官鎮守。上曰:「依東京渤海列置千戶、謀克。」



 遼外戚遙輦昭古牙部族在
 建州,斜野襲走之,獲其妻孥及官豪之族。撻懶復擊之,擒其隊將曷魯燥、白撒葛,殺之,降民戶千餘,進降金源縣。詔增賜銀牌十。又降遙輦二部,再破興中兵,降建州官屬,得山砦二十,村堡五百八十。阿忽復敗昭古牙,降其官民尤多。昭古牙勢蹙亦降,興中、建州皆平。詔第將士功賞,撫安新民。



 撻懶請以遙輦九營為九猛安。上以奪鄰有功,使領四猛安,昭古牙仍為親管猛安。五猛安之都帥,命撻懶擇人授之。撻懶與劉彥宗舉蕭公翊為興中尹,郡府各以契丹、漢官攝治,上皆從之。及宗翰、宗望伐宋,撻懶為六部路都統。宗望已受宋盟,軍還,撻懶
 乃歸中京。



 天會四年八月,復伐宋。閏月,宗翰、宗望軍皆至汴州。撻懶、阿里刮破宋兵二萬於杞,覆其三營,獲京東路都總管胡直孺及其二子與南路都統制隋師元及其三將,遂克拱州,降寧陵,破睢陽,下亳州。宋兵來復睢陽,又擊走之,擒其將石瑱。



 宋二帝已降,大軍北還,撻懶為元帥左監軍,徇地山東,取密州。迪虎取單州,撻懶取巨鹿,阿里刮取宗城,迪古不取清平、臨清,蒙刮取趙州,阿里刮徇下浚、滑、恩及高唐,分遣諸將趣磁、信德,皆降之。劉豫以濟南府降,詔以豫為安撫使,治東平,撻懶以左監軍鎮撫之,大事專決焉。後為右副元帥。天會十五年為左
 副元帥,封魯國王。



 初,宋人既誅張邦昌,太宗詔諸將復求如邦昌者立之,或舉折可求,撻懶力舉劉豫。豫立為帝,號大齊。豫為帝數年,無尺寸功,遂廢豫為蜀王。撻懶與右副元帥宗弼俱在河南,宋使王倫求河南、陜西地于撻懶。明年,撻懶朝京師,倡議以廢齊舊地與宋,熙宗命群臣議,會東京留守宗雋來朝,與撻懶合力,宗乾等爭之不能得。宗雋曰:「我以地與宋,宋必德我。」宗憲折之日:「我俘宋人父兄,怨非一日。若復資以土地,是助仇也,何德之有。勿與便。」撻懶弟勖亦以為不可。既退,撻懶責勖曰:「他人尚有從我者,汝乃異議乎。」勖曰:「茍利國家,豈
 敢私邪。」是時,太宗長子宗磐為宰相,位在宗乾上,撻懶、宗雋附之,竟執議以河南、陜西地與宋。張通古為詔諭江南使。



 久子,宗磐跋扈尤甚,宗雋亦為丞相,撻懶持兵柄,謀反有狀。宗磐、宗雋皆伏誅,詔以撻懶屬尊,有大功,因釋不問,出為行臺尚書左丞相,手詔慰遣。撻懶至燕京,愈驕肆不法,復與翼王鶻懶謀反,而朝議漸知其初與宋交通而倡議割河南、陜西之地。宗弼請復取河南、陜西。會有上變告撻懶者,熙宗乃下詔誅之。撻懶自燕京南走,追而殺之于祁州,并殺翼王及宗人活離胡土、撻懶二子斡帶、烏達補,而赦其黨與。



 宗弼為都元帥,再
 定河南、陜西。伐宋渡淮,宋康王乞和,遂稱臣,畫淮為界,乃罷兵。



 贊曰:君臣之位,如冠屨定分,不可頃刻易也。五季亂極,綱常斁壞。遼之太宗,慢褻神器,倒置冠屨,援立石晉,以臣易君,宇宙以來之一大變也。金人效尤,而張邦昌、劉豫之事出焉。邦昌雖非本心,以死辭之,孰曰不可。豫乘時徼利,金人欲倚以為功,豈有是理哉。撻懶初薦劉豫,後以陜西、河南歸宋,視猶儻來,初無固志以處此也。積其輕躁,終陷逆圖,事敗南奔,適足以實通宋之事爾,哀哉!



\end{pinyinscope}