\article{列傳第九}

\begin{pinyinscope}

 ○斡魯斡魯古勃堇婆盧火吾扎忽闍母宗敘本名德壽



 斡魯,韓國公劾者第三子。康宗初,蘇濱水含國部斡豁勃堇及斡準、聽備二部有異志,斡帶治之,斡賽、斡魯為之佐,遂伐斡豁,拔其城以歸。高麗築九城於曷懶甸。斡賽母疾病,斡魯代將其兵者數月。斡魯亦對築九城與高麗抗,出則戰,入則守,斡賽用之,卒城高麗。



 收國二年
 四月,詔斡魯統諸軍,與闍母、蒲察、迪古乃合咸州路都統斡魯古等,伐高永昌。詔曰:「永昌誘脅戍卒,竊據一方,直投其隙而取之耳。此非有遠大計,其亡可立而待也。東京渤海人德我舊矣,易為招懷。如其不從,即議進討,無事多殺。」



 高永昌渤海人,在遼為裨將,以兵三千,屯東京八甔口。永昌見遼政日敗,太祖起兵,遼人不能支,遂覬覦非常。是時,東京漢人與渤海人有怨,而多殺渤海人。永昌乃誘諸渤海,并其戍卒人據東京,旬月之間,遠近響應,有兵八千人,遂僭稱帝,改元隆基。遼人討之,久不能克。



 永昌使撻不野、杓合,以幣求救於太祖,且曰:「願
 併力以取遼。」太祖使胡沙補往諭之曰:「同力取遼固可。東京近地,汝輒據之,以僭大號可乎。若能歸款,當處以王爵。仍遣係遼籍女直胡突古來。」高永昌使撻不野與胡沙補、胡突古偕來,而永昌表辭不遜,且請還所俘渤海人。太祖留胡突古不遣,遣大藥師奴與撻不野往招諭之。



 斡魯方趨東京,遼兵六萬來攻照散城,阿徒罕勃堇、烏論石準與戰於益褪之地,大破之。五月,斡魯與遼軍遇於瀋州,敗之,進攻瀋州,取之。永昌聞取瀋州,大懼,使家奴鐸刺以金印一、銀牌五十來,願去名號,稱籓。斡魯使胡沙補、撒八往報之。會渤海高楨降,言永昌非真
 降者,特以緩師耳。斡魯進兵,永昌遂殺胡沙補等,率眾來拒。遇于沃里活水,我軍既濟,永昌之軍不戰而卻,逐北至東京城下。明日,永昌盡率其眾來戰,復大敗之,遂以五千騎奔長松島。



 初,太祖下寧江州,獲東京渤海人皆釋之,往往中道亡去,諸將請殺之,太祖曰:「既以克敵下城,何為多殺。昔先太師嘗破敵,獲百餘人,釋之,皆亡去。既而,往往招其部人來降。今此輩亡,後日當有效用者。」至是,東京人恩勝奴、仙哥等,執永昌妻子以城降,即寧江州所釋東京渤海人也。先太師,蓋謂世祖云。未幾,撻不野執永昌及鐸刺以獻,皆殺之。於是,遼之南路係
 籍女直及東京州縣盡降。



 以斡魯為南路都統、迭勃極烈,留烏蠢知東京事。詔除遼法,省賦稅,置猛安謀克一如本朝之制。九月,斡魯上謁于婆魯買水,上慰勞之。辛亥,幸斡魯第,張宴,官屬皆預,賜賚有差。



 燭偎水部實里古達,殺酬斡、僕忽得,斡魯分胡刺古、烏蠢之兵討之。酬斡宗室子,魁偉善戰,年十五,隸軍中,多見任用。以兵五百,敗室韋,獲其民眾。及招降燭偎水部,以功為謀克。僕忽得初事撒改,從討蕭海里,降燭偎水部,領行軍千戶。從破黃龍府,戰達魯古城,皆有功。其破寧江州,渤海乙塞補叛去,僕忽得追復之。至是,與酬斡同被害。



 斡魯至
 石里罕河,實里古達遁去,追及于合撻刺山,誅其首惡四人,撫定餘眾。詔曰:「汝討平叛亂,不勞師眾,朕甚嘉之。酬斡等死於國事,聞其尸棄于河,俟冰釋,必求以葬。其民可三百戶為一謀克,以眾所推服者領之,仍以其子弟等為質。」斡魯乃還。天眷中,酬斡贈奉國上將軍,僕忽得贈昭義大將軍。



 斡魯從都統襲遼主,遼主西走,西京已降復叛,敵據城西浮圖,下射攻城者。斡魯與鶻巴魯攻浮圖,奪之,復以精銳乘浮圖下射城中,遂破西京。夏國王使李良輔將兵三萬來救遼,次於天德之境。婁室與斡魯合軍擊敗之,追至野谷,殺數千人。夏人渡澗水,
 水暴至,漂溺者不可勝計。遼主在陰山、青塚之間,斡魯為西南路都統,往襲之。使勃刺淑、撒曷懣以兵二百,襲遼權六院司喝離質於白水濼,獲之。遼主留輜重於青塚,領兵一萬,往應州。遣照里、背答各率兵邀之,宗望奄至遼主營,盡俘其妻、子、宗族,得其傳國璽。斡魯使使奏捷曰:「賴陛下威靈,屢敗敵兵,遼主無歸,勢必來降。已嚴戒鄰境,毋納宋人,合饋軍糧,令銀術可往代州受之。」詔:「遍諭有功將士,俟朕至彼,當次第推賞。遼主戚屬勿去其輿帳,善撫存之。遼主伶俜去國,懷悲負恥,恐隕其命。孽雖自作,而嘗居大位,深所不忍。如招之肯來,以其宗
 族付之。已遣楊璞徵糧於宋,銀術可不須往矣。遼趙王習泥烈及諸官吏,並釋其罪,且撫慰之。」



 太祖還京師,宗翰為西北、西南兩路都統,斡魯及蒲家奴副之。宗翰朝京師,詔:「以夏人言,宋侵略新割地,以便宜決之。」斡魯奏曰:「夏人不盡歸戶口資帑,又以宋人侵賜地求援兵。宋之邊臣將取所賜夏人疆土,蓋有異圖。」詔曰:「夏人屢求援兵者,或不欲歸我戶口,沮吾追襲遼主事也。宋人敢言自取疆土於夏,誠有異圖。宜謹守備,盡索在夏戶口,通聞兩國,事審處之。」斡魯復請弗割山西與宋,則遼主不能與宋郭藥師交通。復詔曰:「宗翰請毋與宋山西地,
 卿復及此,疆場之事當慎毋忽。」及宗翰等伐宋,斡魯行西南、西北兩路都統事。天會五年,薨。皇統五年,追封鄭國王。天德二年,酏享太祖廟廷。



 子撒八,銀青光祿大夫。子賽里。



 斡魯古勃堇,宗室子也。太祖伐遼,使斡魯古、阿魯撫諭斡忽、急賽兩路係遼女直,與遼節度使撻不也戰,敗之,斬撻不也,酷輦嶺阿魯臺罕等十四太彎皆降,斡忽、急賽兩路亦降。與遼都統實婁戰于咸州西,敗之,斬實婁于陣,與婁室克咸州。陁滿忽吐以所部降于斡魯古,鄰部戶七千亦來歸,遂與遼將喝補戰,破其軍數萬人。太
 祖嘉之,以為咸州軍帥。



 斡魯伐高永昌于東京,斡魯古以咸州軍佐之。遼秦晉國王耶律捏里來伐,迪古乃、婁室、婆盧火等將二萬眾,合斡魯古咸州兵往擊之。



 胡突古嘗叛入于遼,居于東京,高永昌據東京,太祖索之以歸。斡魯古伐永昌,以便宜署胡突古為千戶。散都魯、訛魯補皆無功,亦以便宜除官。及以便宜解權謀克斛拔魯、黃哥、達及保等職,皆非其罪。太祖聞之,盡復斛拔魯等謀克,胡突古等皆罷去。



 太祖聞斡魯古軍中往往闕馬,而官馬多匿於私家,遂檢括之。耶律捏里、佛頂遺斡魯古書,請和。斡魯古以捏里書并所答書來上,且請曰:「
 復有書問,宜如何報之?」詔曰:「若彼再來請和,汝當以阿疏等叛亡,索而不獲至於交兵,我行人賽刺亦不遣還。若歸賽刺,及送阿疏等,則和好之議方敢奏聞。仍恐議和非實,無失備禦。」



 耶律捏里軍蒺藜山,斡魯古以兵一萬,戍東京。太祖使迪古乃、婁室復以兵一萬益之,詔曰:「遼主失道,肆命徂征,惟爾將士,當體朕意,拒命者討之,服者撫安之。毋貪俘掠,毋肆殺戮。所賜捏里詔書,可傳致也。」詔捏里曰:「汝等誠欲請和,當廢黜昏主,擇立賢者,副朕弔伐之意,然後可議和約。不然,當盡并爾國。其審圖之。」捏里復書斡魯古,云:「降去人痕孛見還,則當送阿
 疏等。」上曰:「痕孛等乃交兵之後來降,阿疏則平日以罪亡去,其事特異。」復詔捏里,令此月十三日送阿疏至顯州,各遣重臣議疆場事。



 斡魯古等攻顯州,知東京事完顏斡論以兵來會,即以兵三千先渡遼水,得降戶千餘,遂薄顯州。郭藥師乘夜來襲,斡論擊走之。斡魯古等遂與捏里等戰于蒺藜山,大敗遼兵,追北至阿里真陂,獲佛頂家屬。遂圍顯州,攻其城西南,軍士神篤踰城先入,燒其佛寺,煙焰撲人,守陴者不能立,諸軍乘之,遂拔顯州。於是、乾、懿、豪、徽、成、川、惠等州皆降。乾州後為閭陽縣,遼諸陵多在此,禁無所犯。徙成、川州人於同、銀二州居
 之。



 捏里再以書來請和,斡魯古承前詔,以阿疏為言,答之。駐軍顯州以聽命。賜斡魯古等馬十匹,詔曰:「汝等力摧大敵,攻下諸城,朕甚嘉之。遼主未獲,人心易搖,不可恃戰勝而失備禦。」遼雙州節度使張崇降,斡魯古以便宜命復其職,仍令世襲。



 斡魯古久在咸州,多立功,亦多自恣,劾里保、雙古等告斡魯古不法事:遼帝在中京可追襲而不追襲,咸州糧草豐足而奏數不以實,攻顯州獲生口財畜多自取。捏里、孛刺束等亦告孛堇瞢葛、麻吉、窩論、赤閏、阿刺本、乙刺等多取生口財畜。遂以闍哥代為咸州路都銑。



 闍哥亦宗室子也,既代斡魯古治咸
 州。初,迪古乃、婁室奏,攻顯州新降附之民,可遷其富者於咸州路,其貧者徙內地。於是,詔使闍哥擇其才可幹事者授之謀克,其豪右誠心歸附者擬為猛安,錄其姓名以聞,饑貧之民,官賑給之,而使闍母為其副統云。久之,遼通、祺、雙、遼四州之民八百餘家,詣咸州都統降。上曰:「遼人賦斂無度,民不堪命,相率求生,不可使失望,分置諸部,擇善地以處之。」



 太祖召斡魯古自問之,斡魯古引伏。闍哥鞫窩論等。詔降斡魯古為謀克,而禁錮窩論等。天輔六年,討賊于牛心山,道病卒。天眷中,贈特進。天德二年,配享太祖廟廷。大定十五年,謚莊翼。



 婆盧火,安帝五代孫。太祖伐遼,使婆盧火徵迪古乃兵,失期,杖之。後與渾黜以四千人,往助婁室、銀術哥攻黃龍府。辭勒罕、撒孛得兄弟,直攧里部人,嘗寇耶懶路,穆宗遣婆盧火討之。至阿里門河,辭勒罕偽降,遂略馬畜三百而去,復掠兀勒部二十五寨。太祖復使婆盧火討之。婆盧火渡蘇袞河,招降旁近諸部,因籍丁壯為軍,至特滕吳水,轍孛得偽降,復叛去,執而殺之。婆盧火至特鄰城,圍之,辭勒罕遁去。婆盧火破其城,執其妻子,辭勒罕遂降,曰:「我之馬牛財貨盡矣,何以為生。」婆盧火與之馬十匹。直攧里部產良馬,太祖使紇石烈阿習罕掌其
 畜牧,婆盧火及子婆速,俱為謀克。



 天輔五年,摘取諸路猛安中萬餘家,屯田于泰州,婆盧火為都統,賜耕牛五十。婆盧火舊居按出虎水,自是徙居泰州,而遣拾得、查端、阿里徒歡、奚撻罕等俱徙焉。唯族子撒刺喝嘗為世祖養子,獨得不徙。



 太祖取燕京,婆盧火為右翼,兵出居庸關,大敗遼兵,遂取居庸。蕭妃遁去,都監高六等來送款乞降。習古乃追蕭妃至古北口,蕭妃已過三日,不及而還。上令婆盧火、胡實賚率輕騎追之,蕭妃已遠去,獲其從官統軍察刺、宣徽查刺,并其家族,及銀牌二、印十有一。



 及迭刺叛,婆盧火、石古乃討平之,其群官率眾降
 者,就使領其所部。太宗以空名宣頭及銀牌給之。



 同時有婆盧火者,婁室平陜西,婆盧火、繩果監戰。後為平陽尹,西南路招討使,終於慶陽尹。



 泰州婆盧火守邊屢有功,太宗賜衣一襲,并賜其子剖叔。八年,以甲胄賜所部諸謀克。天會十三年,加同中書門下平章事。天眷元年,駐烏骨迪烈地,薨。贈開府儀同三司,謚剛毅。



 子剖叔,襲猛安,天眷二年,為泰州副都統,子斡帶,廣威將軍。



 婆速,官特進,子吾扎忽。



 吾扎忽,善騎射,年二十,以本班祗候郎君都管,從征伐有功,授脩武校尉。皇統二年,權領泰州軍。平陜西,至涇
 州,大破宋兵於馬西鎮,超遷寧遠大將軍,襲猛安。復以本部軍從宗弼,權都統。正隆末,從海陵伐宋。契丹反,與德昌軍節度使移至懣同討契丹,許以便宜從事。


大定初,除咸平尹,駐軍泰州。俄改臨潢尹,攝元帥左都監。與廣寧尹僕散渾坦俱從元帥右都監神土懣解臨潢之圍。契丹引眾東行,吾扎忽追及于窊歷山。押軍猛安契丹忽刺叔以所部助敵,攻官軍,官軍失利。泰州節度使烏里雅來救,未至臨潢與敵遇,烏里雅敗,僅以數騎脫歸。敵攻泰州,其勢大振,城中震駭,將士不敢出戰,敵四面登城。押軍猛安烏古孫阿里補率軍士數人持金算刀
 循城,應敵力戰,斫刈甚眾,敵乃退,泰州得完。吾扎忽乃使謀克蒲盧渾徙百姓旁邑及險阨之地,以俟大軍。明年,聚甲士萬三千於濟州,會元帥謀衍,敗窩斡於長濼。戰霿
 \gezhu{
  松}
 河,戰陷泉,皆有功,改胡里改節度使,卒。



 吾扎忽性聰敏,有才智,善用軍,常出敵之不意,故能以寡敵眾,而所往無不克,號為「鶻軍」云。



 闍母,世祖第十一子,太祖異母弟也。高永昌據東京,斡魯往伐之,闍母等為之佐。已克瀋州,城中出奔者闍母邀擊殆盡。與永昌隔沃里活水,眾遇淖不敢進,闍母以所部先濟,諸軍畢濟。軍東京城下,城中人出城來戰,闍
 母破之于首山,殲其眾,獲馬五百匹。



 及斡魯古以罪去咸州,闍母氏代之,於是闍母為咸州路副統。遼議和久不成,太祖進兵,詔咸州路都統司,令斜葛留兵一千鎮守,闍母以餘兵會于渾河。太祖攻上京,實臨潢府,諭之不下。遼人恃儲蓄自固。上親臨陣,闍母以眾先登,克其外城,留守撻不野率眾出降。都統杲兵至中京,闍母自城西沿土河以進,城中兵尚餘三千,皆不能守,遂克之。



 宗翰等攻西京,闍母、婁室等於城東為木洞以捍蔽矢石,於北隅以芻茭塞其隍,城中出兵萬餘,將燒之。溫迪罕蒲匣率眾力戰,執旗者被創,蒲匣自執旗,奮擊卻之。又
 為四輪革車,高出於堞,闍母與麾下乘車先登,諸軍繼之,遂克西京。



 與遼步騎五千戰於朔州之境,,斬首三百級。復敗遼騎三百於河陰。遼兵五千屯于馬邑縣南,復擊破之,隳其營壘,盡得其車馬、器械。遼兵三萬,列營於西京之西,闍母以三千擊之。闍母使士卒皆去馬,陣於溝塹之間,曰:「以一擊十,不致之死地,不可使戰也。」謂眾曰:「若不勝敵,不可以求生。」於是人皆殊死戰,遼兵遂敗,追至其營而止。明日,復敗其兵七百餘人。



 興中府宜州復叛,闍母討之,并下詔招諭,詔闍母曰:「遼之土地皆為我有,彼雖復叛,終皆吾民,可縱其耕稼,毋得侵掠。」勃堇
 蒙刮、斜缽、吾撻等獲契丹九斤,興中平。



 闍母為南路都統,討回離保,詔曰:「回離保以烏合之眾,保據險阻,其勢必將自斃。若彼不出掠,毋庸攻討。」耶律奧古哲等殺回離保于景、薊之間,其眾遂潰。



 張覺據平州叛,入于宋,闍母自錦州往討之。覺將以兵脅遷、來、潤、隰四州之民闍母至潤州,擊走張覺軍,逐北至榆關,遣俘侍書招之。復敗覺兵於營州東北,欲乘勝進取南京。時方暑雨,退屯海需,逐水草休息,使僕虺、蒙刮兩猛安屯潤州,制未降州縣,不得與覺交通。九月,闍母破覺將王孝古於新安,敗覺軍於樓峰口。復與覺戰於兔耳山,闍母大敗。太宗
 使宗望問闍母敗軍之狀,宗望遂以闍母軍討覺。及宗望破張覺,太宗乃赦闍母,召宗望赴闕。



 闍母連破偽都統張敦固,遂克南京,執敦固殺之。上遣使迎勞之,詔曰:「聞下南京,撫定兵民,甚善。諸軍之賞,卿差等以給之。」又詔曰:「南京疆場如舊,屯兵以鎮之。命有司運米五萬石於廣寧,給南京、潤州戍卒。」遂下宜州,拔叉牙山,殺其節度使韓慶民,得糧五千石。詔以南路歲饑,許田獵。



 其後宋童貫、郭藥師治兵,闍母輒因降人知之,即具奏,語在宋事中。而宗翰、宗望皆請伐宋,於是闍母副宗望伐宋,宗望以闍母屬尊,先皇帝任使有功,請以為都統,己監
 戰事。於是闍母為都統,掃喝副之,敗郭藥師兵于白河,遂降燕山,以先鋒渡河圍汴,宋人請盟。將士分屯于安肅、雄、霸、廣、信之境,宗望還山西,闍母與劉彥宗留燕京,節制諸軍。



 八月,復伐宋,大軍克汴州,諸軍屯于城上。城中諸軍潰而西出者十三萬人,闍母、撻懶分擊,大敗之。師還,闍母為元帥左都監,攻河間,下之,大破敵兵萬餘於莫州。宗輔為右副元帥,徇地淄、青。闍母與宗弼分兵破山谷諸屯。宋李成兵圍淄州,烏林荅泰欲破之。闍母克濰州。迪古補、術烈速連破趙子昉等兵,至于河上。烏林荅泰欲破敵于靈城鎮。及儀伐康王,闍母欲先定河
 北,然後進討,太宗乃酌取群議之中,使婁室取陜西,宗翰、宗輔南伐。



 天會七年,薨,年四十。熙宗時,追封吳國王。天德二年,配享太祖廟廷。正隆,改封譚王。大定二年,徙封魯王,謚莊襄。



 子宗敘。



 宗敘,本名德壽,闍母第四子也。奇偉有大志,喜談兵。天德二年,充護衛,授武義將軍。明年,授世襲謀克,擢御院通進,遷翰林待制,兼脩起居注,轉國子司業,兼左補闕。正隆初,轉符寶郎,在宮職凡五年,皆帶劍押領宿衛。遷大宗正丞,以母憂去官。以本官起復,未幾,遷侍衛親軍馬軍都指揮使,改左驍騎都指揮使。明年,海陵幸南京,
 宗敘至汴。契丹撒八反,宗敘為咸平尹,兼本路兵馬都總管,以甲仗四千付之,許以便宜。



 宗敘出松亭關,取牛遞于廣寧。聞世宗即位,將歸之。廣寧尹按荅海弟燕京勸宗敘,乃還興中。白彥敬、紇石烈志寧使宗敘奉表降。宗敘見世宗於梁魚務,授寧昌軍節度使。



 明年二月,契丹攻寧昌,宗敘止有女直、渤海騎兵三十、漢兵百二十人,自將擊之。遇賊千餘騎,漢兵皆散走,宗敘與女直、渤海三十騎盡銳力戰,身被二創,所乘馬中箭而仆,遂為所執。居百餘日,會賊中有臨潢民移刺阿塔等,盜馬授之,得脫歸。



 宗敘陷賊久,盡得其虛實,見元帥完顏謀衍、
 平章政事完顏元宜,謂之曰:「賊眾烏合,無紀律,破之易耳。」於是帥府欲授軍職,宗敘見謀衍貪鹵掠,失事機,欲歸白上,不肯受職,曰:「我有機密,須面奏。」是夕,乃遁去,至廣寧,矯取驛馬,馳至京師。而帥府先事以聞,上遣中使詰之曰:「汝為節度,不度眾寡,戰敗被獲,幸得脫歸,乃拒帥府命,輒自乘傳赴都,朕姑置汝罪,可速還軍,併力破賊。」宗敘附奏曰:「臣非辭難者,事須面奏,不得不來。」遂召入,乃條奏賊中虛實,及諸軍進退不合事機狀。詔大臣議,皆以其言為然。是時,已詔僕散忠義代謀衍為元帥進討,於是拜宗敘為兵部尚書,以本職領右翼都統,率
 宗寧、烏延查刺、烏林荅刺撒兵各千人,號三萬,佐忠義軍。至花道,遇賊,與戰,左翼都統宗亨先敗走,忠義亦引卻,宗敘勒本部遮擊之,麾帳下士三百,捨馬步戰,賊不得逞。大軍整列復至,合勢擊之,賊遂敗去。而元帥右監軍紇石烈志寧率軍至,追及窩斡於陷泉,大破之。復與志寧及徒單克寧,追至七渡河,復大敗之。元帥忠義遂留宗敘自從。賊平,入為右宣徽使。



 宋兵據海州,將謀深入。詔以宗敘為元帥右監軍,往禦之。宗敘駐山東,分兵據守要害,敵不得西。尋奉詔,與左副元帥紇石烈志寧參議軍事。四年,宗敘入朝,奏曰:「暑月在近,頓兵邊陲,飛
 挽頗艱,乞俟秋涼進發。」上從其請。及還軍,授以成算,賜襲衣、弓矢。九月,渡淮,宗敘出唐、鄧,比至襄陽,屢戰皆捷。明年,宋人請和,軍還,除河南路統軍使。



 河決李固渡,分流曹、單之間。詔遣都水監梁肅視河決,宗敘言:「河道填淤不受水,故有決溢之患。今欲河復故道,卒難成功,幸而可塞,它日不免決溢山東,非曹、單比也。沿河數州,驟興大役,人心動搖,恐宋人乘間扇誘,構為邊患。」梁肅亦請聽兩河分流,以殺水勢,遂止不塞。



 十年,召至京師,拜參知政事,上曰:「卿奏黃河利害,甚合朕意。朕念百姓差調,官吏為姦,率斂星火,所費倍蓰,委積經年,腐朽不可
 復用,若此等類,百孔千瘡,百姓何以堪之。卿參朝政,擇利而行,以副朕心。」及與上論南邊事,宗敘曰:「南人遣諜來,多得我事情。我遣諜人,多不得其實。蓋彼以厚賞故也。」上曰:「彼以厚利資諜人,徒費其財,何能為也。」



 十一年,奉詔巡邊。六月,至軍中,將戰,有疾,詔以右丞相紇石烈志寧代,宗敘還。七月,病甚,遺表朝政得失,及邊防利害,力疾,使其子上之。薨,年四十六。上見其遺表,傷悼不已,輟朝,遣宣徽使敬嗣暉致祭,賻銀千兩、彩四十端、絹四百匹。上謂宰臣曰:「宗敘勤勞國家,他人不能及也。」



 初,宗敘嘗請募貧民戍邊屯田,給以廩粟,既貧者無艱食之
 患,而富家免更代之勞,得專農業。上善其言,而未行也。十七年,上謂宰臣曰:「戍邊之卒,歲冒寒暑,往來番休,以馬牛往戍,往往皆死。且奪其農時,敗其生業,朕甚閔之。朕欲使百姓安於田里,而邊圉強固,卿等何術可以致死。」左丞相良弼曰:「邊地不堪耕種,不能久戍,所以番代耳。」上曰:「卿等以此急務為未事耶。往歲,參政宗敘嘗為朕言此事。若宗敘,可謂盡心於國者矣。今以兩路招討司、烏古里石壘部族、臨潢、泰州等路,分置堡戍,詳定以聞,朕將親覽。」



 上追念宗敘,聞其子孫家用不給,詔賜錢三千貫。明昌五年,配享世宗廟廷。



\end{pinyinscope}