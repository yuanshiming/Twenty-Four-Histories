\article{列傳第四十八}

\begin{pinyinscope}

 ○楊雲翼趙秉文韓玉馮璧李獻甫雷淵程震



 楊雲翼,字之美,其先贊皇檀山人,六代祖忠,客平定之樂平縣,遂家焉。曾祖青、祖郁、考恆,皆贈官于朝。雲翼天資穎悟,初學語輒畫地作字,日誦數千言。登明昌五年進士第一,詞賦亦中乙科,特授承務郎、應奉翰林文字。承安四年,出為陜西東路兵馬都總管判官。泰和元年,
 召為太學博士,遷太常寺丞,兼翰林修撰。七年,簽上京、東京等路按察司事,因召見,章宗咨以當世之務,稱旨。大安元年,翰林承旨張行簡薦其材,且精術數,召授提點司天臺,兼翰林修撰,俄兼禮部郎中。崇慶元年,以病歸。貞祐二年,有司上官簿,宣宗閱之,記其姓名,起授前職,兼吏部郎中。三年,轉禮部侍郎,兼提點司天臺。



 四年,大元及西夏兵入鄜延,潼關失守,朝議以兵部尚書蒲察阿里不孫為副元帥以禦之。雲翼言其人言浮於實,必誤大事。不聽,後果敗。興定元年六月,遷翰林侍講學士,兼修國史,知集賢院事,兼前職,詔曰:「官制入三品者
 例外除,以卿遇事敢言,議論忠讜,故特留之。」時右丞相高琪當國,人有請榷油者,高琪主之甚力,詔集百官議,戶部尚書高夔等二十六人同聲曰:「可。」雲翼獨與趙秉文、時戩等數人以為不可,議遂格。高琪後以事譴之,雲翼不恤也。二年,拜禮部尚書,兼職如故。三年,築京師子城,役兵民數萬,夏秋之交病者相籍,雲翼提舉醫藥,躬自調護,多所全濟。四年,改吏部尚書。凡軍興以來,入粟補官及以戰功遷授者,事定之後,有司苛為程式,或小有不合輒罷去,雲翼奏曰:「賞罰國之大信,此輩宜從寬錄,以勸將來。」



 是年九月,上召雲翼及戶部尚書夔、翰林
 學士秉文於內殿,皆賜坐,問以講和之策,或以力戰為言,上俯首不樂,雲翼徐以《孟子》事大、事小之說解之,且曰:「今日奚計哉,使生靈息肩,則社稷之福也。」上色乃和。



 十一月,改御史中丞。宗室承立權參知政事,行尚書省事於京兆,大臣言其不法,詔雲翼就鞫之,獄成,廷奏曰:「承立所坐皆細事,不足問。向大兵掠平涼以西,數州皆破,承立坐擁彊兵,瞻望不進。鄜延帥臣完顏合達以孤城當兵衝,屢立戰績。其功如此,而承立之罪如彼,願陛下明其功罪以誅賞之,則天下知所勸懲矣。自餘小失,何足追咎。」承立由是免官,合達遂掌機務。



 哀宗即位,首
 命雲翼攝太常卿,尋拜翰林學士。正大二年二月,復為禮部尚書,兼侍讀。詔集百官議省費,雲翼曰:「省費事小,戶部司農足以辦之。樞密專制軍政,蔑視尚書。尚書出政之地,政無大小,皆當總領。今軍旅大事,社稷繫焉,宰相乃不得預聞,欲使利病兩不相蔽得乎。」上嘉納之。



 明年,設益政院,雲翼為選首,每召見賜坐而不名。時講《尚書》,雲翼為言帝王之學不必如經生分章析句,但知為國大綱足矣。因舉「任賢」「去邪」、「與治同道」「與亂同事」、「有言逆於汝心」「有言遜於汝志」等數條,一皆本於正心誠意,敷繹詳明。上聽忘倦。尋進《龜鑒萬年錄》、《聖學》、《聖孝》之類
 凡二十篇。



 當時朝士,廷議之際多不盡言,顧望依違,浸以成俗。一日,經筵畢,因言:「人臣有事君之禮,有事君之義。禮,不敢齒君之路馬,蹴其芻者有罰,入君門則趨,見君之几杖則起,君命召不俟駕而行,受命不宿於家,是皆事君之禮,人臣所當盡者也。然國家之利害,生民之休戚,一一陳之,則向所謂禮者特虛器耳。君曰可,而有否者獻其否;君曰否,而有可者獻其可。言有不從,雖引裾、折檻、斷鞅、軔輪有不恤焉者。當是時也,姑徇事君之虛禮,而不知事君之大義,國家何賴焉。」上變色曰:「非卿,朕不聞此言。」雲翼嘗患風痺,至是稍愈,上親問愈之之
 方,對曰:「但治心耳。心和則邪氣不干,治國亦然,人君先正其心,則朝廷百官莫不一於正矣。」上矍然,知其為醫諫也。



 夏人既通好,遣其徽猷閣學士李弁來議互市,往返不能決,朝廷以雲翼往議乃定。五年卒,年五十有九,謚文獻。



 雲翼天性雅重,自律甚嚴,其待人則寬,與人交分一定,死生禍福不少變。其於國家之事,知無不言。貞祐中,主兵者不能外禦而欲取償於宋,故頻歲南伐。有言之者,不謂之與宋為地,則疑與之有謀。至於宰執,他事無不言者,獨南伐則一語不敢及。雲翼乃建言曰:「國家之慮,不在於未得淮南之前,而在城既得淮南之後。
 蓋淮南平則江之北盡為戰地,進而爭利於舟楫之間,恐勁弓良馬有不得騁者矣。彼若扼江為屯,潛師於淮以斷餉道,或決水以瀦淮南之地,則我軍何以善其後乎。」及時全倡議南伐,宣宗以問朝臣,雲翼曰:「朝臣率皆諛辭,天下有治有亂,國勢有弱有彊,今但言治而不言亂,言彊而不言弱,言勝而不言負,此議論所以偏也。臣請兩言之。夫將有事於宋者,非貪其土地也,第恐西北有警而南又綴之,則我三面受敵矣,故欲我師乘勢先動,以阻其進。借使宋人失淮,且不敢來,此戰勝之利也。就如所料,其利猶未可必然。彼江之南其地尚廣,雖無
 淮南豈不能集數萬之眾,伺我有警而出師耶。戰而勝且如此,如不勝害將若何。且我以騎當彼之步,理宜萬全,臣猶恐其有不敢恃者。蓋今之事勢與泰和不同。泰和以冬征,今我以夏往,此天時之不同也。冬則水涸而陸多,夏則水潦而塗淖,此地利之不同也。泰和舉天下全力,驅颭軍以為前鋒,今能之乎?此人事之不同也。議者徒見泰和之易,而不知今日之難。請以夏人觀之,向日弓箭手之在西邊者,一遇敵則搏而戰、袒而射,彼已奔北之不暇,今乃陷吾城而虜守臣,敗吾軍而禽主將。曩則畏我如彼,今則侮我如此。夫以夏人既非前日,奈
 何以宋人獨如前日哉。願陛下思其勝之之利,又思敗之之害,無悅甘言,無貽後悔。」章奏不報。時全果大敗於淮上,一軍全沒。宣宗責諸將曰:「當使我何面目見楊雲翼耶?」



 河朔民十有一人為游騎所迫,泅河而南,有司論罪當死,雲翼曰:「法所重私渡者,防姦偽也。今平民為兵所迫,奔入於河,為逭死之計耳。今使不死於敵而死於法,後惟從敵而已。」宣宗悟,盡釋之。哀宗以河南旱,詔遣官理冤獄,而不及陜西,雲翼言:「天地人通為一體,今人一支受病則四體為之不寧,豈可專治受病之處而置其餘哉。」朝廷是之。



 司天有以《太乙新歷》上進者,尚書省
 檄雲翼參訂,摘其不合者二十餘條,歷家稱焉。所著文集若干卷,校《大金禮儀》若干卷,《續通鑑》若干卷,《周禮辨》一篇,《左氏》、《莊》、《列賦》各一篇,《五星聚井辨》一篇,《縣象賦》一篇,《勾股機要》、《象數雜說》等著藏于家。



 趙秉文,字周臣,磁州滏陽人也。幼穎悟,讀書若夙習。登大定二十五年進士第,調安塞簿,以課最遷邯鄲令,再遷唐山。丁父憂,用薦者起復南京路轉運司都勾判官。明昌六年,入為應奉翰林文字,同知制誥。上書論宰相胥持國當罷,宗室守貞可大用。章宗召問,言頗差異,於是命知大興府事內族膏等鞫之。秉文初不肯言,詰其
 僕,歷數交游者,秉文乃曰:「初欲上言,嘗與修撰王庭筠、御史周昂、省令史潘豹、鄭贊道、高坦等私議。」庭筠等皆下獄,決罰有差。有司論秉文上書狂妄,法當追解,上不欲以言罪人,遂特免焉。當時為之語曰:「古有朱雲,今有秉文,朱雲攀檻,秉文攀人。」士大夫莫不恥之。坐是久廢,後起為同知岢嵐軍州事,轉北京路轉運司支度判官。承安五年冬十月,陰晦連日,宰相張萬公入對,上顧謂萬公曰:「卿言天日晦冥,亦猶人君用人邪正不分,極有理。若趙秉文曩以言事降授,聞其人有才藻,工書翰,又且敢言,朕非棄不用,以北邊軍事方興,姑試之耳。」泰和
 二年,召為戶部主事,遷翰林修撰。十月,出為寧邊州刺史。三年,改平定州。前政苛於用刑,每聞赦將至,先掊賊死乃拜赦,而盜愈繁。秉文為政,一從寬簡,旬月盜悉屏跡。歲饑,出祿粟倡豪民以賑,全活者甚眾。



 大安初,北兵南嚮,召秉文與待制趙資道論備邊策,秉文言:「今我軍聚於宣德,城小,列營其外,涉暑雨,器械弛敗,人且病,俟秋敵至將不利矣。可遣臨潢一軍搗其虛,則山西之圍可解,兵法所謂『出其不意、攻其必救』者也。」衛王不能用,其秋宣德果以敗聞。尋為兵部郎中,兼翰林修撰,俄轉翰林直學士。



 貞祐初,建言時事可行者三:一遷都,二導
 河,三封建。朝廷略施行之。明年,上書願為國家守殘破一州,以宣布朝廷恤民之意,且曰:「陛下勿謂書生不知兵,顏真卿、張巡、許遠輩以身許國,亦書生也。」又曰:「使臣死而有益於國,猶勝坐糜廩祿為無用之人。」上曰:「秉文志固可尚,然方今翰苑尤難其人,卿宿儒,當在左右。」不許。四年,拜翰林侍講學士,言:「寶券滯塞,蓋朝廷初議更張,市肆已妄傳其不用,因之抑遏,漸至廢絕。臣愚以為宜立回易務,令近上職官通市道者掌之,給以銀鈔粟麥縑帛之類,權其低昂而出納。」詔有司議行之。



 興定元年,轉侍讀學士。拜禮部尚書,兼侍讀學士,同修國史,知
 集賢院事。又明年,知貢舉,坐取進士盧亞重用韻,削兩階,因請致仕。金自泰和、大安以來,科舉之文其弊益甚。蓋有司惟守格法,所取之文卑陋陳腐,茍合程度而已,稍涉奇峭,即遭絀落,於是文風大衰。貞祐初,秉文為省試,得李獻能賦,雖格律稍疏而詞藻頗麗,擢為第一。舉人遂大喧噪,愬於臺省,以為趙公大壞文格,且作詩謗之,久之方息。俄而獻能復中宏詞,入翰林,而秉文竟以是得罪。



 五年,復為禮部尚書,入謝,上曰:「卿春秋高,以文章故須復用卿。」秉文以身受厚恩,無以自效,願開忠言、廣聖慮,每進見從容為上言,人主當儉勤、慎兵刑,所以
 祈天永命者,上嘉納焉。哀宗即位,再乞致仕,不許。改翰林學士,同修國史,兼益政院說書官。以上嗣德在初,當日親經史以自裨益,進《無逸直解》、《貞觀政要》、《申鑒》各一通。



 正大九年正月,汴京戒嚴,上命秉文為赦文,以布宣悔悟哀痛之意。秉文指事陳義,辭情俱盡。及兵退,大臣欲稱賀,且命為表,秉文曰:「《春秋》『新宮火,三日哭』。今園陵如此,酌之以禮,當慰不當賀。」遂已。時年已老,日以時事為憂,雖食息頃不能忘。每聞一事可便民,一士可擢用,大則拜章,小則為當路者言,殷勤鄭重,不能自已。三月,草《開興改元詔》,閭巷間皆能傳誦,洛陽人拜詔畢,舉城
 痛哭,其感人如此。是年五月壬辰,卒,年七十四,積官至資善大夫、上護軍、天水郡侯。



 正大間,同楊雲翼作《龜鑒萬年錄》上之。又因進講,與雲翼共集自古治術,號《君臣政要》為一編以進焉。秉文自幼至老未嘗一日廢書,著《易叢說》十卷,《中庸說》一卷,《揚子發微》一卷,《太玄箋贊》六卷,《文中子類說》一卷,《南華略釋》一卷,《列子補注》一卷,刪集《論語》、《孟子解》各一十卷,《資暇錄》一十五卷,所著文章號《滏水集》者三十卷。



 秉文之文長於辨析,極所欲言而止,不以繩墨自拘。七言長詩筆勢縱放,不拘一律,律詩壯麗,小詩精絕,多以近體為之,至五言古詩則沉鬱頓
 挫。字畫則草書尤遒勁。朝使至自河、湟者,多言夏人問秉文及王庭筠起居狀,其為四方所重如此。



 為人至誠樂易,與人交不立崖岸,未嘗以大名自居。仕五朝,官六卿,自奉養如寒士。楊雲翼嘗與秉文代掌文柄,時人號「楊趙」。然晚年頗以禪語自污,人亦以為秉文之恨云。



 贊曰:楊雲翼、趙秉文,金士巨擘,其文墨論議以及政事皆有足傳。雲翼諫伐宋一疏,宣宗雖不見聽,此心何愧景略。庭筠之累,秉文所為,茲事大愧高允。



 韓玉,字溫甫,其先相人,曾祖錫仕金,以濟南尹致仕。玉明昌五年經義、辭賦兩科進士,入翰林為應奉。應制一
 日百篇,文不加點。又作《元勳傳》,稱旨,章宗歎曰:「勛臣何幸,得此家作傳耶!」泰和中,建言開通州潞水漕渠,船運至都。升兩階,授同知陜西東路轉運使事。



 大安三年,都城受圍。夏人連陷邠、涇,陜西安撫司檄玉以鳳翔總管判官為都統府募軍,旬日得萬人,與夏人戰,敗之,獲牛馬千餘。時夏兵五萬方圍平涼,又戰于北原,夏人疑大軍至,是夜解去。當路者忌其功,驛奏玉與夏寇有謀,朝廷疑之,使使者授玉河平軍節度副使,且覘其軍。先是,華州李公直以都城隔絕,謀舉兵入援,而玉恃其軍為可用,亦欲為勤王之舉,乃傳檄州郡云:「事推其本,禍有
 所基,始自賊臣貪容姦賂,繼緣二帥貪固威權。」又云:「裹糧坐費,盡膏血於生民。棄甲復來,竭資儲於國計。要權力而望形勢,連歲月而守妻孥。」又云:「人誰無死,有臣子之當然。事至于今,忍君親之弗顧。而謂百年身後,虛名一聽史臣。只如今日目前,何顏以居人世。」公直一軍行有日矣,將有違約、國朝人有不從者,輒以軍法從事。京兆統軍便謂公直據華州反,遣都統楊珪襲取之,遂置極刑。公直曾為書約玉,玉不預知,其書乃為安撫所得。及使者覘玉軍,且疑預公直之謀,即實其罪。玉道出華州,被囚,死於郡學。臨終書二詩壁間,士論冤之。



 子不疑,
 字居之。以父死非罪,誓不祿仕。藏其父臨終時手書云:「此去冥路,吾心皓然,剛直之氣,必不下沉。兒可無慮。世亂時艱,努力自護,幽明雖異,寧不見爾。」讀者惻然。



 馮璧,字叔獻,真定縣人。幼穎悟不凡,弱冠補太學生。承安二年經義進士,制策復優等,調莒州軍事判官,宰相奏留校秘書。未幾,調遼濱主簿。縣有和糴粟未給價者餘十萬斛,散貯民居,以富人掌之,有腐敗則責償於民,民殊苦之。璧白漕司,即日罷之,民大悅。



 泰和四年,調鄜州錄事。明年,伐蜀,行部檄充軍前檢察,帥府以書檄委之。章宗欲招降吳曦,詔先以文告曉之,然後用兵。蜀人守散
 關不下,金兵殺獲甚眾,璧言:「彼軍拒守而并禍其民,無乃與詔旨相戾乎?」主帥憾之,以璧招兩當潰卒,璧即日率風州已降官屬淡剛、李果偕行。道逢軍士所得子女金帛牛馬皆奪付剛,使歸其家,軍士則以違制決遣之。比到兩當,軍民三萬餘眾鼓舞迎勞,璧以朝旨慰遣之。及還,主帥嘉其能,奏遷一官。五年,自東阿丞召補尚書省令史,用宗室承暉薦授應奉翰林文字,兼韓王府記室參軍。俄轉太學博士。至寧初,忽沙虎弒逆,遂去官。



 宣宗南遷,璧時避兵東方,由單父渡河詣汴梁,時相奏復前職。貞祐三年,遷翰林修撰。時山東、河朔軍六十餘萬
 口,仰給縣官,率不逞輩竄名其間。詔璧攝監察御史,汰逐之。總領撒合問冒券四百餘口,劾案以聞,詔杖殺之,故所至爭自首,減幾及於半。復進一官。初,監察御史本溫被命汰宗室從坦軍於孟州,軍士欲謀變,本溫懼不知所為。尋有旨,北軍沈思忠以下四將屯衛州,餘眾果叛入太行。於是,密院奏以璧代本溫竟其事。璧馳至衛,召四將喻以上意。思忠等挾叛者請還奏之,璧責以大義,將士慚服,不日就汰者三千人。



 六月,改大理丞,與臺官行關中,劾奏姦臟之尤者商州防禦使宗室重福等十數人,自是權貴側目。



 興定四年,以宋人拒使者於淮上,遣
 兵南伐,詔京東總帥紇石烈牙吾塔攻盱眙,牙吾塔不從命,乃率精騎由滁州略宣化,縱兵大掠。故兵所至原野蕭條,絕無所資,宋人堅壁不戰,乃無功而歸。行省奏牙吾塔故違節制,詔璧佩金符鞫之。璧馳入牙吾塔軍,奪其金符,易以他帥攝。牙吾塔入獄,兵士嘩噪,以吾帥無罪為言,璧怒責牙吾塔曰:「元帥欲以兵抗制使耶?待罪之禮恐不如此,使者還奏,獄能竟乎。」牙吾塔伏地請死,璧曰:「兵法,進退自專,有失機會以致覆敗者斬。」即擬以聞,時議壯之。



 十月,改禮部員外郎,權右司諫、治書侍御史。詔問時務所當先者,璧上六事,大略言減冗食,備
 選鋒,緩疑似以慎刑,擇公廉以檢吏,屯戍革朘削之弊,權貴嚴請託之科。又條自治之策四,謂別賢佞,信賞罰,聽覽以通下情,貶損以謹天戒。詔以東方饑饉,盜賊並起,以御史中丞完顏伯嘉為宣慰使,監察御史道遠從行。道遠發永城令簿姦贓,伯嘉與令有違,付令有司,釋簿不問,燕語之際,又許參佐克忠等臺職。璧皆劾之,伯嘉竟得罪去。



 初,諜者告歸德行樞密院言,河朔叛軍有竊謀南渡者,行院事胡土門、都水監使毛花輦易其人,不為備。一日,紅衲數百聯筏南渡,殘下邑而去。命璧鞫之。璧以二將託疾營私,聞寇弛備,且來不戰、去不追,在
 法皆當斬。或以為言:「二將皆寵臣,而都水者貲累巨萬,若求援禁近,必從輕典。君徒結怨權貴,果何益耶?」璧歎曰:「睢陽行闕,東籓重兵所宿,門廷之寇且不能禦,有大於此者,復何望乎!」即具所擬聞。



 四年,遷刑部郎中。關中旱,詔璧與吏部侍郎畏忻審理冤獄。時河中帥阿虎帶及僚屬十數人皆以棄城罪當死,繫同州獄待報。同州官僚承望風旨,問璧何以處之,璧曰:「河中今日重地,朝議擬為駐蹕之所,若失此則河南、陜西有脣亡之憂。以彼宗室勛貴故使鎮之,平居無事竭民膏血為浚築計,一旦有警乃遽焚蕩而去,此而不誅,三尺法無用矣。」竟
 以無冤上之。



 冬十月,出為歸德治中。未幾,改同知保靜軍節度使。又改同知集慶軍節度使,到官即上章乞骸骨,進一官致仕。正大九年,河南破,北歸,又數年卒,年七十有九。



 李獻甫,字欽用,獻能從弟也。博通書傳,尤精《左氏》及地理學。為人有幹局,心所到則絕人遠甚,故時人稱其精神滿腹。興定五年登進士第,歷咸陽簿,辟行臺令史。正大初,夏使來請和,朝廷以翰林待制馮延登往議,時獻甫為書表官,從行。夏使有口辯,延登不能折,往復數日不定,至以歲幣為言,獻甫不能平,從旁進曰:「夏國與我
 和好百年,今雖易君臣之名為兄弟之國,使兄輸幣,寧有據耶?」使者曰:「兄弟且不論。宋歲輸吾國幣二十五萬疋,典故具在,君獨不知耶?金朝必欲修舊好,非此例不可。」獻甫作色曰:「使者尚忍言耶?宋以歲幣餌君家而賜之姓,岸然以君父自居,夏國君臣無一悟者,誠謂使者當以為諱,乃今公言之。使者果能主此議,以從賜姓之例,弊邑雖歲捐五十萬,獻甫請以身任之。」夏使語塞,和議乃定。後朝廷錄其功,授慶陽總帥府經歷官。尋辟長安令。京兆行臺所在,供億甚繁,獻甫處之常若有餘,縣民賴之以安。入為尚書省令史。天興元年,充行六部員
 外郎,守備之策時相倚任之。以功遷鎮南軍節度副使,兼右警巡使,死於蔡州之難,年四十。



 所著文章號《天倪集》,留汴京。獻甫死,其家亦破,同年華陰王元禮購得之,傳于世。



 雷淵,字希顏,一字季默,應州渾源人。父思,名進士,仕至同知北京轉運使,註《易》行于世。淵庶出,年最幼,諸兄不齒。父歿,不能安於家,乃發憤入太學。衣弊履穿,坐榻無席,自以跣露,恒兀坐讀書,不迎送賓客,人皆以為倨。其友商衡每為辯之,且周恤焉。後從李之純游,遂知名。登至寧元年詞賦進士甲科,調涇州錄事,坐高庭玉獄,幾
 死。後改東平,河朔重兵所在,驕將悍卒倚外敵為重,自行臺以下皆摩撫之,淵出入軍中,偃然不為屈。不數月,閭巷間多畫淵像,雖大將不敢以新進書生遇之。尋遷東阿令,轉徐州觀察判官。興定末,召為英王府文學兼記室參軍,轉應奉翰林文字。拜監察御史,言五事稱旨,又彈劾不避權貴,出巡郡邑所至有威譽,奸豪不法者立箠殺之。至蔡州,杖殺五百人,時號曰「雷半千」。坐此為人所訟,罷去。久之,用宰相侯摯薦,起為太學博士、南京轉運司戶籍判官,遷翰林修撰。一夕暴卒,年四十八。



 正大庚寅倒迴谷之役,淵嘗上書破朝臣孤注之論,引援
 深切,灼然易見,主兵者沮之,策竟不行。為人軀幹雄偉,髯張口哆,顏渥丹,眼如望洋,遇不平則疾惡之氣見於顏間,或嚼齒大罵不休,雖痛自懲創,然亦不能變也。為文章詩喜新奇。善結交,凡當塗貴要與布衣名士無不往來。居京師,賓客踵門未嘗去舍,家無餘貲,及待賓客甚豐腆。蒞官喜立名,初登第攝遂平縣事,年少氣稅,擊豪右,發姦伏,一邑大震,稱為神明。嘗擅笞州魁吏,州檄召之不應,罷去。後凡居一職輒震耀,亦坐此不達。



 程震,字威卿,東勝人。與其兄鼎俱擢第。震入仕有能聲。興定初,詔百官舉縣令,震得陳留,治為河南第一,召拜
 監察御史,彈劾無所撓。時皇子荊王為宰相,家僮輩席勢侵民,震以法劾之,奏曰:「荊王以陛下之子,任天下之重。不能上贊君父,同濟艱難。顧乃專恃權勢,蔑棄典禮,開納貨賂,進退官吏。縱令奴隸侵漁細民,名為和市,其實脅取。諸所不法不可枚舉。陛下不能正家,而欲正天下,難矣。」於是,上責荊王,出內府銀以償物直,杖大奴尤不法者數人。未幾,坐為故吏所訟,罷官。歲餘。嘔血卒。



 震為人剛直有材幹,忘身徇國,不少私與。及為御史,臺綱大振,以故小人側目者眾,不能久留於朝,士論惜之。



 贊曰:韓玉、馮璧、李獻甫、雷淵,皆金季豪傑之士也。邠、涇
 之變,玉募兵旬日而得萬人。牙吾塔之兇暴,璧以王度繩之,卒不敢動。夏人援宋例以邀歲幣,獻甫以宋賜夏姓一事折之,夏使語塞而和議定。淵為御史,權貴斂避,古之國士何加焉。玉以疑見冤,璧、淵疾惡太甚,議者以酷譏之,瑕豈可以掩瑜哉。程震劾荊抵罪,比蹤馮、雷,然亦以群小齟齬而死,直士之不容於世也久矣。籲!



\end{pinyinscope}