\article{列傳第十三}

\begin{pinyinscope}

 ○盧彥
 倫子璣孫亨嗣毛子廉李三錫孔敬宗李師夔沈璋左企弓虞仲文曹勇義康公弼附左水必弟淵侄光慶



 盧彥倫,臨潢人。遼天慶初,蕭貞一留守上京,置為吏,以材乾稱。是時,臨潢之境多盜,而城中兵無統屬者,府以
 彥倫為材,薦之於朝,即授殿直、勾當兵馬公事。



 遼兵敗於出河店,還至臨潢,散居民家,令給養之,而軍士縱恣侵擾,無所不至,百姓殊厭苦之。留守耶律赤狗兒不能禁戢,乃召軍民諭之曰:「契丹、漢人久為一家,今邊方有警,國用不足,致使兵士久溷父老間,有侵擾亦當相容。」眾皆無敢言者。彥倫獨曰:「兵興以來,民間財力困竭,今復使之養土,以國家多故,義固不敢辭。而此輩恣為強暴,人不能堪。且番、漢之民皆赤子也,奪此與彼,謂何。」



 初取臨潢,軍中有辛訛特刺者,舊為臨潢驛吏,與彥倫善,使往招諭,彥倫殺之。遼授彥倫團練使、勾當留守司公
 事。



 天輔四年,彥倫從留守撻不野出降。授夏州觀察使,權發遣上京留守事。師還,撻不野以城叛,彥倫乃率所部逐撻不野,盡殺城中契丹,遣使來報。未幾,遼將耶律馬哥以兵取臨潢,彥倫拒守者七月。會援兵至,敵解圍去,因赴闕。



 天會二年,知新城事。城邑初建,彥倫為經畫,民居、公宇皆有法。改靜江軍節度留後,知咸州煙火事。未幾,遷靜江軍節度使。天眷初,行少府監兼都水使者,充提點京城大內所,改利涉軍節度使。未閱月,還,復為提點大內所。彥倫性機巧,能迎合悼合意,由是頗見寵用。歲餘,遷侍衛親軍馬步軍都指揮使,為宋國歲元使。
 改禮部尚書,加特進,封郇國公。天德二年,出為大名尹。明年,詔彥倫營造燕京宮室,以疾卒,年六十九。子璣。



 璣字正甫,以蔭補閤門祗候,累遷客省使,兼東上閤門使,改提點太醫、教坊、司天,充大定十五年宋主生日副使,遷同知宣徽院事。丁母憂,起復太府監,改開遠軍節度使,入為右宣徽使。章宗即位,轉左宣徽使,致仕。明昌四年,起復左宣徽使,改定武軍節度使,復為左宣徽使。



 是時,璣年已七十,詔許朝參得坐於廊下。復致仕。泰和初,詔璣天壽節預宴。二年,元妃李氏生皇子,滿三月,章宗以璣老而康強,命以所策杖為洗兒禮物。章宗幸玉
 泉山,詔璣與致仕宰相俱會食,許策杖給扶。後預天壽節,上命璣與大臣握槊戲,璣獲勝焉。從上秋山,賜名馬。上曰:「酬卿博直。」其眷遇如此。泰和六年卒,年八十。子亨嗣。



 亨嗣字繼祖,以廕補閤門祗候,內供奉。調同監平涼府醋務,改同監天山鹽場。丁母憂,服闋,監萊州酒課,累調監豐州、任丘、汲縣、東平酒務。課最,遷白登縣令。明昌四年,行六部差規措軍前糧料,入為典給直長,改西京戶籍判官,歷官西京、中都太倉使,中都戶籍判官,尚醖署丞。丁父憂。大安初,復為典給署丞兼太子家令。崇慶元
 年,遷同知順天軍節度使事。是時,兵興,徵調煩急,亨嗣以辦最,遷定遠大將軍,入為戶部員外郎。貞祐二年,遷莒州刺史。三年,山東宣撫司討楊安兒,亨嗣行六部,兵罷,還州。興定二年,卒,年六十一。



 亨嗣與弟亨益,盡友愛之道。亨嗣初以祖蔭得官,大定十六年,父璣為同知宣徽院事,當廕子,亨嗣以讓弟亨益。亨益早卒,子兟。兟幼稚,亨嗣盡以舊業田宅奴畜財物與之。



 毛子廉本名八十,臨潢長泰人,材勇善射。遼季群盜起,募勇士,子廉應募。遼主召見,賜甲仗,率百人,會所在官兵捕盜。以功授東頭供奉官,賜良馬。



 天輔四年,遣謀克
 辛斡特刺、移刺窟斜招諭臨潢,子廉率戶二千六百來歸。今就領其眾,佩銀牌,招未降軍民。盧彥倫怒於廉先降,殺子廉妻及二子,使騎兵二千伺取子廉。子廉與窟斜經險阻中,騎兵圍之,兩騎突出直犯子廉。子廉引弓斃其一人,其一人挺槍幾中子廉腋。子廉避其槍,與搏戰,生擒之,乃彥倫健將孫延壽也。餘眾潰去。



 天會三年,除上京副留守。久之,兼鹽鐵事。天眷中,除燕京院都監。遼王宗幹問宰相曰:「子廉有功,何為下遷。」;宰相以例對。宗乾曰:「盧彥倫何不除此職?子廉之功十倍彥倫,在臨潢十餘年,吏民畏愛如一日,誰能及此。」是時盧彥倫
 已以少府監除節度使,故宗乾引以為比。除寧昌軍節度使。海陵弒熙宗,子廉聞之,嘆曰:「曾不念國王定策之功耶。」乃致仕。大定二年,卒。



 李三錫字懷邦,錦州安昌人,以貲得官。遼季,盜攻錦州,州人推三錫主兵事,設機應變,城賴以完。錄功授左承制。遼主走天德,劉彥宗辟三錫將兵保白雲山。



 金兵次來州,三錫以其眾降。攝臨海軍節度副使,參預元帥府軍事,改知嚴州。宗望伐宋,三錫領行軍猛安,敗郭藥師軍於白河。進官安州防禦使。再克汴京,三錫從闍母護宋二主北歸。復知嚴州,改歸德軍節度副使。詔廢齊國,
 擇吏三十人與俱行,三錫在選中。還為慶州刺史,三遷武勝軍節度使。察廉第一,遷三階,改安國軍節度使,除河北西路轉運使,致仕。



 三錫政事強明,所至稱治。世宗舊聞其名,大定初,起為北京路都轉運使。制下,而三錫已卒。



 孔敬宗字仲先,其先東垣人,石晉末,徙遼陽。遼季,敬宗為寧昌劉宏幕官。斡魯古兵至境上,敬宗勸劉宏迎降,遂以敬宗為鄉導,拔顯州,以功補順安令。天輔二年,詔敬宗與劉宏率懿州民徙內地,授世襲猛安,知安州事。將兵千人從宗望伐宋。汴京平,宗望命敬宗守汴。嘗自
 汴馳驛至河北,還至河上,會日暮無舟,敬宗策馬亂流,遂達南岸。遷靜江軍節度使,歷石、辰、信、磁四州刺史,階光祿大夫。



 海陵問張浩曰:「卿識孔敬宗否,何階高職下也。」浩對曰:「國初,敬宗勸劉宏以懿州效順,其後從軍積勞,有司不知,故一概常調耳。」明日,除寧昌軍節度使。徙歸德軍,致仕。大定二年,卒。



 李師夔字賢佐,奉聖永興人。少倜儻,有大志。以廕入仕,為本州監。天輔六年,太祖襲遼主于鴛鴦濼,郡守委城遁去,眾無所屬,相與叩門請師夔主郡事。師夔許之,乃搜卒治兵。



 迪古乃兵至奉聖州,師夔與其故人沈璋
 密謀出降,曰:「一城之命懸於此舉。」璋曰:「君言是矣。如軍民不從,奈何。」師夔即率親信十數輩詰旦出城,見余睹,與之約曰:「今已服從,願無以兵入城及俘掠境內。」余睹許諾。詔以師夔領節度,以璋佐之。賜師夔駿馬二,俾招未附者,許以便宜從事。明年,加左監門衛大將軍。



 劇賊張勝以萬人逼城,師夔度眾寡不敵,乃偽與之和,日致饋給,勝信之。師夔乘其不備,使人刺勝,殺之。以其首徇曰:「汝輩皆良民,脅從至此,今元惡已誅,可棄兵歸復其所。」賊眾大驚,皆散去。別賊焦望天、尹智穆率兵數千來寇。師夔以兵臨之,設伏歸路,使人反間之。智穆果疑,望
 天先引去。智穆勢孤,亦還,遇伏而敗,遂執斬之。是後賊眾不敢入境。以勞遷靜江軍節度留後,累遷武平軍節度使,改東京路轉運使,徙陜西東路轉運使。致仕,封任國公。卒,年八十五。



 沈璋字之達,奉聖州永興人也。學進士業。迪古乃軍至上谷,璋與李師夔謀,開門迎降。明日,擇可為守者,眾皆推璋,璋固稱李師夔,於是授師夔武定軍節度使,以璋副之。授太常少卿,遷鴻臚卿。丁母憂,起復山西路都轉運副使,加衛尉卿。從伐宋。汴京平,眾爭趨貲貨,璋獨無所取,惟載書數千卷而還。



 太行賊陷潞州,殺其守姚璠,
 官軍討平之,命璋權知州事。璋至,招復逋逃,賑養困餓,收其橫屍葬之。未幾,民頗軍輯。初,賊黨據城,潞之軍卒當緣坐者七百人,帥府牒璋盡誅之,璋不從。帥府聞之,大怒,召璋呵責,且欲殺璋,左右震恐,璋顏色不動,從容對曰:「招亡撫存,璋之職也。此輩初無叛心,蓋為賊所脅,有不得已者,故招之復來。今欲殺之,是殺降也。茍利於眾,璋死何憾。」少頃,怒解。因召潞軍曰:「吾始命戮汝,今汝使君活爾矣。」皆感泣而去。朝廷聞而嘉之,拜左諫議大夫,知潞州事。百姓為之立祠。移知忻州,改同知太原尹,加尚書禮部侍郎。



 時介休人張覺聚黨亡命山谷,鈔掠
 邑縣,招之不肯降,曰:「前嘗有降者,皆殺之。今以好言誘我,是欲殺我耳。獨得侍郎沈公一言,我乃無疑。」於是,命璋往招之,覺即日降。



 轉尚書吏部侍郎、西京副留守、同知平陽尹,遷利涉軍節度使,為東京路都轉運使,改鎮西軍節度使。天德元年,以病致仕。卒,年六十。



 子宜中,天德三年,賜楊建中榜及第。



 贊曰:危難之際,兩軍方爭,專城之將,國家之輕重繫焉。李師夔非有君命,為眾所推,又能全活其人,猶有說也。盧彥倫之降,雖云城潰,初志不確,何尤乎毛子廉。至如子廉不仕海陵,沈璋以片言降張覺,一善足稱,何可掩
 也。



 左企弓字君材。八世祖皓,後唐棣川刺史,以行軍司馬戍燕,遼取燕,使守薊,因家焉。企弓讀書,通《左氏春秋》。中進士,再遷來州觀察判官。蕭英弼賊昭懷太子,窮治黨與,多連引。企弓辨析其冤,免其甚眾。自御史知難事,出為中京副留守,按刑遼陽。有獄本輕而入之重者,已奏待報,企弓釋之以聞。累遷知三司使事。天慶末,拜廣陵軍節度使,同中書門下平章事、知樞密院事。



 金兵已拔上京,北樞密院恐忤旨,不以時奏。遼故事,軍政皆關決北樞密院,然後奏御。企弓以聞。遼主曰:「兵事無乃非卿
 職邪?」對曰:「國勢如此,豈敢循例為自容計。」因陳守備之策。拜中書侍郎平章事,監修國史。時遼主聞金已克中京,將西幸以避之。企弓諫不聽。



 遼主自鴛鴦濼亡保陰山。秦晉國王耶律捏里自立於燕,廢遼主為湘陰王,改元德興。企弓守司徒,封燕國公。虞仲文參知政事,領西京留守、同中書門下平章事、內外諸軍都統。曹勇義中書侍郎平章事、樞密使、燕國公。康公弼參知政事、簽樞密院事,賜號「忠烈翊聖功臣」。德妃攝政,企弓加侍中。宋兵襲燕,奄至城中,已而敗走。或疑有內應者,欲根株之,企弓爭之,乃止。



 太祖至居庸關,蕭妃自古北口遁去。都
 監高六等送款于太祖,太祖徑至城下。高六等開門待之。太祖入城受降,企弓等猶不知。太祖駐蹕燕京城南,企弓等奉表降,太祖俾復舊職,皆受金牌。企弓守太傅、中書令,仲文樞密使、侍中、秦國公,勇義以舊官守司空,公弼同中書門下平章事、樞密副使權知院事、簽中書省、封陳國公。遼致仕宰相張琳進上降表,詔曰:「燕京應琳田宅財物並給還之。」琳年高,不能入見,止令其子弟來。



 太祖既定燕,從初約,以與宋人。企弓獻詩,略曰:「君王莫聽捐燕議,一寸山河一寸金。」太祖不聽。



 是時,置樞密院于廣寧府。企弓等將赴廣寧,張覺在平州有異志,太
 祖欲以兵送之。企弓等辭兵曰:「如此,是促之亂也。」及過平州,舍于慄林下,張覺使人殺之。企弓年七十三,謚恭烈。天會七年,贈守太師,遣使致奠。正隆二年,改贈特進、濟國公。



 虞仲文字質夫,武州寧遠人也。七歲知作詩,十歲能屬文,日記千言,刻苦學問。第進士,累仕州縣,以廉能稱。舉賢良方正,對策優等。擢起居郎、史館修撰,三遷至太常少卿。宰相有左降,仲文獨出餞之。或指以為黨,仲文乃求養親。久之,召復前職。宰相薦文行第一,權知制誥,除中書舍人。討平白霫,拜樞密直學士,權翰林學士,為翰
 林侍講學士。年五十五,卒,謚文正。天會七年,贈兼中書令。正隆二年,改贈特進、濮國公。



 曹勇義,廣寧人。第進士,除長春令。樞府辟令史。上書陳時政,累擢館閣,遷樞密副都承旨,權燕京三司使,加給事中。召為樞密副使,加太子少保。與大公鼎、虞仲文、龔誼友善。與虞仲文同在樞密,群小擠之。復出為三司使,加宣政殿大學士。卒,謚文莊。天會七年,贈守太保。正隆二年,改贈特進、定國公。



 康公弼字伯迪,其先應州人。曾祖胤,遼保寧間以戰功授質券,家于燕之宛平。公弼好學,年二十三中進士,除
 著作郎、武州軍事判官。辟樞府令史,求外補,出為寧遠令。縣中隕霜殺禾稼,漕司督賦急,系之獄。公弼上書,朝廷乃釋之,因免縣中租賦,縣人為立生祠。監平州錢帛庫,調役糧于川州。大盜侯概陷川州,使護送公弼出境,曰:「良吏也。」權乾州節度使。卒,謚忠肅。天會七年,贈侍中。正隆二年,改贈特進、道國公。



 企弓子泌、瀛、淵。



 泌字長源,企弓長子也。仕遼,官至棣州刺史。太祖平燕,泌從企弓歸朝。既而東遷至平州,企弓為張覺所害,泌復還燕。是時,以燕與宋,宣撫司遣至汴,泌以平州仇人在是,乃間道奔還。朝廷嘉之,擢西上閤
 門使。從宋王宗望南伐,破真定有功,知祁州,歷刺澤、隰等州。貞元初,為濬州防禦使,遷陜西路轉運使,封戴國公。



 泌性夷澹,好讀《莊》、《老》,年六十一,即請致仕。親友或以為早,泌嘆曰:「予年三十秉旄鉞,侵尋仕路又三十年,名遂身退,可矣。」時人高之。卒年七十四。



 淵累官燕京副留守、中京路都轉運使,歷河北東路、中都路都轉運使。淵貪鄙,三任漕事,務以錢穀自營。在中都凡八年,不求遷。與李通、許霖交關賄賂,詭納漕司諸物,規取財利。世宗即位,淵使其子貽慶詣東京上表,特賜貽慶任忠傑榜第三甲進士,授從仕郎。貽慶還中都,
 世宗詔淵曰:「凡殿位張設悉依舊,毋增益。不得役使一夫,以擾百姓。謹宮禁出入而已。」大定二年,改沁南軍節度使。世宗素知其為人,戒之曰:「卿宰相子,練習朝政,前為漕司,朕甚鄙之。毋或刻削百姓,若復敢爾,勿思再用。」淵到懷州未幾,坐前為中都轉運嘗盜用官材木,除名。子光慶。



 光慶字君錫,幼潁悟,沉厚少言。淵嘗謂所親曰:「世吾家者,此子也。」以蔭,補閤門祗候,遷西上閤門副使。丁父憂,起復東上閤門副使,再轉西上、東上閤門使,兼太廟署令。



 光慶好古,讀書識大義,喜為詩,善篆隸,尤工大字。世
 宗行郊禮,受尊號,及受命寶,皆光慶篆。凡宮廟榜署經光慶書者,人稱其有法。典領原廟、坤厚陵、壽安宮工役,不為苛峻,使勞逸相均。身兼數職,勤慎周密,未嘗自伐,世宗獨察之。



 初,御史大夫璋請製大金受命寶,有司以秦璽文進,上命以「大金受命萬世之寶」為文。徑四寸八分,厚一寸四分,蟠龍紐,高厚各四寸六分有半。禮部尚書張景仁、少府監張僅言典領工事,詔光慶篆之。遷同知宣徽院事,改少府監。丁母憂,起復右宣徽使。世宗幸上京,光慶往上京治儀仗制度,時人以為得宜。



 二十五年,卒,年五十一。上遣使至祭,賻銀三百兩、重彩十端、絹
 百匹。平時喜為善言,蓄善藥,號「善善道人」。晚信浮屠法,自作真贊,語皆任達云。



 贊曰:左企弓、虞仲文、曹勇義、康公弼四子者,皆有才識之士,其事遼主數有論建。及其受爵僭位,委質二君,隕身逆黨,三者胥失之,哀哉。



\end{pinyinscope}