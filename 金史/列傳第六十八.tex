\article{列傳第六十八}

\begin{pinyinscope}

 列女



 ○阿鄰妻李寶信妻韓慶民妻雷婦師氏康住住李文妻李英妻相琪妻阿魯真撒合輦妻許古妻馮妙真蒲察氏烏古論氏素蘭妻忙哥妻尹氏白氏
 聶孝女仲德妻寶符李氏張鳳奴附



 漢成帝時,劉向始述三代賢妃淑女,及淫泆奢僭、興亡盛衰之所由,匯分類別,號《列女傳》,因以諷諫。范曄始載之漢史。古者女子生十年有女師,漸長,有麻枲絲繭之事,有祭祀助奠之事;既嫁,職在中饋而已,故以無非無儀為賢。若乃嫠居寡處,患難顛沛,是皆婦人之不幸也。一遇不幸,卓然能自樹立,有烈丈夫之風,是以君子異之。



 阿鄰妻沙里質者,金源郡王銀術可之妹。天輔六年,黃
 龍府叛卒攻鈔旁近部族。是時,阿鄰從軍,沙里質糾集附近居民得男女五百人,樹營柵為保守計。賊千餘來攻,沙里質以氈為甲,以裳為旗,男夫授甲,婦女鼓噪,沙里質仗劍督戰,凡三日賊去。皇統二年,論功封金源郡夫人。大定間,以其孫藥師為謀克。



 李寶信妻王氏。寶信為義豐縣令,張覺以平州叛,王氏陷賊中。賊欲逼室之,王氏罵賊,賊怒遂支解之。大定十二年,贈「貞烈縣君」。



 韓慶民妻者,不知何許人,亦不知其姓氏。慶民事遼為宜州節度使。天會中,攻破宜州,慶民不屈而死,以其妻
 配將士,其妻誓死不從,遂自殺。世宗讀《太宗實錄》,見慶民夫婦事,嘆曰:「如此節操,可謂難矣。」



 雷婦師氏,夫亡,孝養舅姑。姑病,刲臂肉飼之,姑即愈。舅姑既歿。兄師逵與夫姪規其財產,乃偽立謀證致之官,欲必嫁之。縣官不能辨曲直,師氏畏逼,乃投縣署井中死。詔有司祭其墓,賜謚曰「節」。



 康住住,鄜州人。夫早亡,服闋,父取之歸家,許嚴沂為妻。康氏誓死弗聽,欲還夫家不可得,乃投崖而死。詔有司致祭其墓。



 李文妻史氏,同州白水人。夫亡,服闋,誓死弗嫁。父強取
 之歸,許邑人姚乙為妻。史氏不聽,姚訴之官,被逮,遂自縊死。詔有司致祭其墓。



 李英妻張氏。英初為監察御史,在中都,張居濰州。貞祐元年冬,大元兵取濰州,入其家,張氏盡以所有財物與之。既而,令張氏上馬,張曰:「我盡以物與汝,猶不見贖邪?」答曰「汝品官妻,當復為夫人。」張曰:「我死則為李氏鬼。頓坐不起,遂見殺。追封隴西郡夫人,謚「莊潔」。英仕至御史中丞,有傳。



 相琪妻欒氏,有姿色。琪為萊州掖縣司吏。貞祐三年八月,紅襖賊陷掖縣,琪與欒氏及子俱為所得。賊見欒悅
 之,殺琪及其子而誘欒。欒奮起以頭觸賊而仆,罵曰:「我豈為犬彘所汙者哉。」賊怒,殺之。追封西河縣君,謚「莊潔」。



 阿魯真,宗室承充之女,胡里改猛安夾谷胡山之妻。夫亡寡居,有眾千餘。興定元年,承充為上京元帥,上京行省太平執承充應蒲鮮萬奴。阿魯真治廢壘,修器械,積芻糧以自守。萬奴遣人招之,不從,乃射承充書入城,阿魯真得而碎之,曰:「此詐也。」萬奴兵急攻之,阿魯真衣男子服,與其子蒲帶督眾力戰,殺數百人,生擒十餘人,萬奴兵乃解去。後復遣將擊萬奴兵,獲其將一人。詔封郡公夫人,子蒲帶視功遷賞。承充已被執,乘間謂其二子
 女胡、蒲速乃曰:「吾起身宿衛,致位一品,死無恨矣。若輩亦皆通顯,未嘗一日報國家,當思自處,以為後圖。」二子乃冒險自拔南走,是年四月至南京。



 獨吉氏,平章政事千家奴之女,護衛銀術可妹也。自幼動有禮法,及適內族撒合輦。閨門肅如。撒合輦為中京留守,大兵圍之,撒合輦疽發背不能軍,獨吉氏度城必破,謂撒合輦曰:「公本無功能,徒以宗室故嘗在禁近,以至提點近侍局,同判睦親府,今又為留守外路第一等官,受國家恩最厚。今大兵臨城,公不幸病,不能戰禦,設若城破,公當率精銳奪門而出,攜一子走京師。不能則
 獨赴京師,又不能,戰而死猶可報國,幸無以我為慮。」撒合輦出巡城,獨吉氏乃取平日衣服妝具玩好布之臥榻,資貨悉散之家人,艷妝盛服過於平日,且戒女使曰:「我死則扶置榻上,以衾覆面,四圍舉火焚之。無使兵見吾面。」言訖,閉門自經而死。家人如言,臥尸榻上,以衾覆之。撒合輦從外室,家人告以夫人之死,撒合輦拊榻曰:「夫人不辱我,我肯辱朝廷乎!」因命焚之。年三十有六。少頃,城破,撒合輦率死士欲奪門出,不果,投壕水死,有傳。



 許古妻劉氏,定海軍節度使仲洙之女也。貞祐初,古挈家僑居蒲城,從留劉氏母子于蒲,仕于朝。既而,兵圍蒲,
 劉謂二女曰:「汝父在朝,而兵勢如此,事不可保。若城破被驅,一為所汙奈何?不若俱死以自全。」已而,攻城益急,於是劉氏與二女相繼自盡。有司以聞于朝,四年五月,追封劉氏為郡君,謚曰「貞潔」,其長女謚曰「定姜」,次「肅姜」,以其事付史館。



 馮妙真,刑部尚書延登之女也。生十有八年,適進士張綎。興定五年,綎為洛川主簿。大元兵破葭州、綏德,遂入鄜延。鄜人震恐具守備,守臣以西路輸芻粟不時至,檄慥詣平涼督之。時延登為平涼行省員外郎,綎欲偕妙真以往,妙真辭曰:「舅姑老矣。雖有叔姒,妾能安乎。子行,
 妾留奉養。」十一月,洛川破,妙真從舅姑匿窟室,兵索得之。妙真泣與舅姑訣曰:「婦生不辰,不得終執箕帚,義不從辱。」即攜三子赴井死。縣人從而死者數十人。明年春,綎發井得屍,殯于縣之東郭外。死時年二十四。



 蒲察氏,字明秀,鄜州帥訥申之女,完顏長樂之妻也。哀宗遷歸德,以長樂為總領,將兵扈從。將行,屬蒲察氏曰:「無他言,夫人慎毋辱此身。」明秀曰:「君第致身事上,無以妾為念。妾必不辱。」長樂一子在幼,出妻柴氏所生也,明秀撫育如己出。崔立之變,驅從官妻子于省中,人自閱之。蒲察氏聞,以幼子付婢僕,且與之金幣,親具衣棺祭
 物,與家人訣曰:「崔立不道,強人妻女,兵在城下,吾何所逃,惟一死不負吾夫耳。汝等惟善養幼子。」遂自縊而死,欣然若不以死為難者。時年二十七。



 烏古論氏,伯祥之妹,臨洮總管陀滿胡土門之妻也。伯祥朝貴中聲譽藉甚,胡土門死王事。崔立之變,衣冠家婦女多為所汙,烏古論氏謂家人曰:「吾夫不辱朝廷,我敢辱吾兄及吾夫乎。」即自縊。一婢從死。



 參政完顏素蘭妻,亡其姓氏。當崔立之變,謂所親曰:「吾夫有天下重名,吾豈肯隨眾陷身以辱吾夫乎。今日一死固當,但不可無名而死,亦不可離吾家而死。」即自縊
 於室。



 溫特罕氏,夫完顏忙哥,五朵山宣差提控回里不之子也,系出蕭王。忙哥叔父益都,節度秦州,為大元兵所攻,適病不能軍,忙哥為提控,獨當一面。兵退而益都死,忙哥以城守功世襲謀克,收充奉御。及崔立之變,忙哥義不受辱,與其妻訣。妻曰:「君能為國家死,我不能為君死乎。」一婢曰:「主死,婢將安歸。」是日,夫婦以一繩同縊,婢從之。



 尹氏,完顏豬兒之妻也。豬兒系出蕭王,天興二年正月從哀宗為南面元帥,戰死黃陵岡。其妻金源郡夫人聞
 豬兒死,聚家資焚之,遂自縊,年三十一。豬兒贈官,弟長住即日詔補護衛。



 白氏,蘇嗣之之母,許州人,宋尚書右丞子由五世孫婦也。初,東坡、潁濱、叔黨俱葬郟城之小峨嵋山,故五世皆居許昌。白氏年二十餘即寡居,服除,外家迎歸,兄嫂竊議改醮。白氏微聞之,牽車徑歸,曰:「我為蘇學士家婦,又有子,乃欲使我失身乎。」自是,外家非有大故不往也。嘗於宅東北為祭室,畫兩先生像,圖黃州、龍川故事壁間,香火嚴潔,躬自灑掃,士大夫求瞻拜者往往過其家奠之。天興元年正月庚戌,許州被兵,嗣之為汴京廂官,白
 拜辭兩先生前曰:「兒子往京師,老婦死無恨矣,敢以告。」即自縊於室側。家人并屋焚之。年七十餘。嗣之本名宗之,避諱改焉。



 聶孝女,字舜英,尚書左右司員外郎天驥之長女也。年二十三,適進士張伯豪。伯豪卒,歸父母家。及哀宗遷歸德,天驥留汴。崔立劫殺宰相,天驥被創甚,日夜悲泣,恨不即死。舜英謁醫救療百方,至刲其股雜他肉以進,而天驥竟死。時京城圍久食盡,閭巷間有嫁妻易一飽者,重以崔立之變,剽奪暴凌,無復人理。舜英頗讀書知義理,自以年尚少艾,夫既亡,父又死非命,比為兵所汙,何
 若從吾父于地下乎。葬其父之明日,絕脰而死。一時士女賢之,有為泣下者。其家以舜英合葬張伯豪之墓。



 完顏仲德妻,不知其族氏。崔立之變,妻自毀其容服,攜妾及二子紿以采蔬,自汴走蔡。蔡被圍,丁男皆乘城拒守,謂仲德曰:「事勢若此,丈夫能為國出力,婦人獨不能耶!」率諸命婦自作一軍,親運矢石於城下,城中婦女爭出繼之。城破自盡。



 哀宗寶符李氏,國亡從后妃北遷,至宣德州,居摩訶院,日夕寢處佛殿中,作幡旆。會當赴龍庭,將發,即於佛像前自縊死,且自書門紙曰:「寶符御侍此處身故。」後人至
 其處,見其遺跡,憐而哀之。



 天興元年,北兵攻城,矢石之際忽見一女子呼於城下曰:「我倡女張鳳奴也,許州破被俘至此。彼軍不日去矣,諸君努力為國堅守,無為所欺也。」言竟,投濠而死。朝廷遣使馳祭於西門。



 正大、天興之際,婦人節義可知者特數人耳。鳳奴之事別史錄之。蓋亦有所激云。



\end{pinyinscope}