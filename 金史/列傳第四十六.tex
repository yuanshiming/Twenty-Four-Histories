\article{列傳第四十六}

\begin{pinyinscope}

 ○胥鼎侯摯把胡魯師安名



 胥鼎,字和之,尚書右丞持國之子也。大定二十八年擢進士第,入官以能稱,累遷大理丞。承安二年,持國卒,去官。四年,尚書省起復為著作郎。上曰:「鼎故家子,其才如何?」宰臣奏曰:「為人甚幹濟。」上曰:「著作職閑,緣今無他闕,姑授之。」未幾,遷右司郎中,轉工部侍郎。泰和六年,鼎言急遞鋪轉送文檄之制,上從之,時以為便。至寧初,中都
 受兵,由戶部尚書拜參知政事。



 貞祐元年十一月,出為泰定軍節度使,兼兗州管內觀察使,未赴,改知大興府事,兼中都路兵馬都總管。二年正月,鼎以在京貧民闕食者眾,宜立法振救,乃奏曰:「京師官民有能贍給貧人者,宜計所贍遷官陞職,以勸獎之。」遂定權宜鬻恩例格,如進官升職、丁憂人許應舉求仕、官監戶從良之類,入粟草各有數,全活甚眾。四月,拜尚書右丞,仍兼知府事。五月,宣宗將南渡,留為汾陽軍節度使,兼汾州管內觀察使。十一月,改知平陽府事,兼河東南路兵馬都總管,權宣撫使。



 三年四月,建言利害十三事,若積軍儲、備黃
 河、選官讞獄、簡將練卒、鈔法、版籍之類,上頗採用焉。又言:「平陽歲再被兵,人戶散亡,樓櫓修繕未完,衣甲器械極少,庾廩無兩月食。夏田已為兵蹂,復不雨,秋種未下。雖有復業殘民,皆老幼,莫能耕種,豈足徵求。比聞北方劉伯林聚兵野狐嶺,將深入平陽、絳、解、河中,遂抵河南。戰禦有期,儲積未備,不速錯置,實關社稷生靈大計。乞降空名宣敕一千、紫衣師德號度牒三千,以補軍儲。」上曰:「鼎言是也,有司其如數亟給之。」



 七月,就拜本路宣撫使,兼前職。朝廷欲起代州戍兵五千,鼎上言:「嶺外軍已皆南徙,代為邊要,正宜益兵保守,今更損其力,一朝兵
 至,何以待之?平陽以代為籓籬,豈可撤去。」尚書省奏宜如所請,詔從之。又言:「近聞朝廷令臣清野,切謂臣所部乃河東南路,太原則北路也,大兵若來,必始於北,故清野當先北而後南。況北路禾稼早熟,其野既清,兵無所掠,則勢當自止。不然,南路雖清,而穀草委積於北,是資兵而召之南也。臣已移文北路宣撫司矣,乞更詔諭之。」既而大兵果出境,賜詔獎諭曰:「卿以文武之才,膺兵民之寄,往鎮方面,式固邊防,坐釋朕憂,孰如卿力。益懋忠勤之節,以收綏靜之功,仰副予心,嗣有後寵。」尋以能設方略退兵,進官一階。



 十月,鼎上言:「臣所將義軍,皆從來
 背本趨末,勇猛兇悍、盜竊亡命之徒,茍無訓練統攝官以制之,則朋聚黨植,無所不至。乞許臣便宜置總領義軍使、副及彈壓,仍每五千人設訓練一員,不惟預為防閑,使有畏忌,且令武藝精熟,人各為用。」上從之。



 四年正月,大兵略霍、吉、隰三州,已而步騎六萬圍平陽,急攻者十餘日,鼎遣兵屢卻之,且上言:「臣以便宜立官賞,預張文榜,招還脅從人七千有奇,績至者又六千餘,俱令復業。竊謂凡被俘未歸者,更宜多方招誘,已歸者所居從便,優加存恤,無致失所。」制可。二月,拜樞密副使,權尚書左丞,行省于平陽。時鼎方抗表求退,上不許,因進拜焉,
 且遣近侍諭曰:「卿父子皆朕所知,向卿執政時,因有人言,遂以河東事相委,果能勉力以保無虞。方國家多難,非卿孰可倚者?卿退易耳,能勿慮社稷之計乎!今特授卿是任,咫尺防秋,更宜悉意。」



 時河南粟麥不令興販渡河,鼎上言曰:「河東多山險,平時地利不遺,夏秋薦熟,猶常藉陜西、河南通販物斛。況今累值兵戎,農民浸少,且無雨雪,闕食為甚。又解州屯兵數多,糧儲僅及一月。伏見陜州大陽渡、河中大慶渡皆邀阻粟麥,不令過河,臣恐軍民不安,或生內患。伏望朝廷聽其輸販,以紓解州之急。」從之。



 又言:「河東兵革之餘,疲民稍復,然丁牛既少,
 莫能耕稼,重以亢旱蝗螟,而餽餉所須,徵科頗急,貧無依者俱已乏食,富戶宿藏亦為盜發,蓋絕無而僅有焉,其憔悴亦已甚矣。有司宜奉朝廷德意,以謀安集,而潞州帥府遣官於遼、沁諸郡搜括餘粟,懸重賞誘人告訐,州縣憚帥府,鞭箠械繫,所在騷然,甚可憐憫。今大兵既去,惟宜汰冗兵,省浮費,招集流亡,勸督農事。彼不是務,而使瘡痍之民重罹茲苦,是兵未來而先自弊也。願朝廷亟止之,如經費果闕,以恩例勸民入粟,不猶愈於強括乎!」又言:「霍州回牛、夙樓嶺諸阨,戍卒幾四千。今兵既去而農事方興,臣乞量留偵候,餘悉遣歸,有警復徵。既
 休民力,且省縣官,萬一兵來,亦足禦遏。舉一事而獲二利,臣敢以為請。」詔趨行之。



 又言:「河東兩路農民浸少,而兵戍益多,是以每歲糧儲常苦不繼。臣切見潞州元帥府雖設鬻爵恩例,然條目至少,未盡勸誘之術,故進獻者無幾。宜增益其條,如中都時,仍許各路宣撫司俱得發賣,庶幾多獲貯儲,以濟不給。」於是尚書省更定制奏行焉。



 又言:「交鈔貴於通流,今諸路所造不敷所出,茍不以術收之,不無闕誤。宜從行省行部量民力征斂,以裨軍用。河中宣撫司亦以寶券所支已多,民不貴,乞驗民貧富征之。雖然,陜西若一體徵收,則彼中所有,日湊于
 河東,其與不斂何異。又河北寶券以不許行于河南,由是愈滯,將誤軍儲而啟釁端。」時以河北寶券商旅齎販南渡,致物價翔貴,權限路分行用,因鼎有言,罷之。



 又言:「比者朝廷命擇義軍為三等,臣即檄所司,而潞帥必蘭阿魯帶言:『自去歲初置帥府時已按閱本軍,去其冗者。部分既定,上下既親,故能所向成功。此皆血戰之餘,屢試可者。且又父子兄弟自相赴援,各顧其家,心一而力齊,勢不可離。今必析之,將互易而不相諳矣。國家糧儲,常恐不繼,豈容僥冒,但本府兵不至是耳。況潞州北即為異境,日常備戰,事務方殷,而分別如此,彼居中下者,
 皆將氣挫心懈而不可用,慮恐因得測吾虛實。且義軍率皆農民,已各散歸田畝,趨時力作。若徵集之,動經旬日,農事廢而歲計失矣。乞從本府所定,無輕變易。』臣切是其言。」時阿魯帶奏亦至,詔遂許之。



 又言:「近偵知北兵駐同、耀,竊慮梗吾東西往來之路,遂委河中經略使陀滿胡土門領軍赴援。今兵勢將叩關矣,前此臣嘗奏聞,北兵非止欲攻河東、陜西,必將進取河南。雖已移文陜州行院及陜西鄰境,俱令設備,恐未即遵行。乞詔河南行院統軍司,議所以禦備之策。」上以示尚書省,宰臣奏:「兵已踰關,惟宜嚴責所遣帥臣趨迎擊之,及命鼎益兵
 渡河以掣其肘。」制可。既而鼎聞大兵已越關,乃急上章曰:「臣叨蒙國恩擢列樞府,凡有戎事,皆當任之。今入河南,將及畿甸,豈可安據一方,坐視朝廷之急,而不思自奮以少寬陛下之憂乎。去歲頒降聖訓,以向者都城被圍四方無援為恨,明敕將帥,若京師有警,即各提兵奔赴,其或不至自有常刑。臣已奉詔,先遣潞州元帥左監軍必蘭阿魯帶領軍一萬,孟州經略使徒單百家領兵五千,由便道濟河以趨關、陜,臣將親率平陽精兵直抵京師,與王師相合。」又奏曰:「京師去平陽千五百餘里,倘俟朝廷之命方圖入援,須三旬而後能至,得無失其機
 耶?臣以身先士卒倍道兼行矣。」上嘉其意,詔樞府督軍應之。



 初,鼎以將率兵赴援京師,奏乞委知平陽府事王質權元帥左監軍,同知府事完顏僧家奴權右監軍,以鎮守河東,從之。至是,鼎拜尚書左丞,兼樞密副使。是時,大兵已過陜州,自關以西皆列營柵,連亙數十里。鼎慮近薄京畿,遂以河東南路懷、孟諸兵合萬五千,由河中入援,又遣遙授河中府判官僕散掃吾出領軍趨陜西,併力禦之。且慮北兵扼河,移檄絳、解、吉、隰、孟州經略司,相與會兵以為夾攻之勢。已而北兵果由三門、集津北渡而去。



 鼎復上言:「自兵興以來,河北潰散軍兵、流亡人
 戶,及山西、河東老幼,俱徙河南。在處僑居,各無本業,易至動搖。竊慮有司妄分彼此,或加迫遣,以致不安。今兵日益盛,將及畿甸,倘復誘此失職之眾使為鄉導,或驅之攻城,豈不益資其力。乞朝廷遣官撫慰,及令所司嚴為防閑,庶幾不至生釁。」上從其計,遣監察御史陳規等充安撫捕盜官,巡行郡邑。大兵還至平陽,鼎遣兵拒戰,不利乃去。



 興定元年正月,上命鼎選兵三萬五千,付陀滿胡土門統之西征。至是,鼎馳奏以為非便,略曰:「自北兵經過之後,民食不給,兵力未完。若又出師,非獨饋運為勞,而民將流亡,愈至失所。或宋人乘隙而動,復何以
 制之?此繫國家社稷大計。方今事勢,止當禦備南邊,西征未可議也。」遂止。是月,進拜平章政事,封莘國公。又上奏曰:「臣近遣太原、汾、嵐官軍以備西征,而太原路元帥左監軍烏古論德升以狀白臣,甚言其失計。臣愚以為德升所言可取,敢具以聞。」詔付尚書省議之,語在德升傳。三月,鼎以祖父名章,乞避職,詔不從。



 朝廷詔鼎舉兵伐宋,且令勿復有言,以沮成算。鼎已分兵由秦、鞏、鳳翔三路並進,乃上書曰:「竊懷愚懇,不敢自默,謹條利害以聞。昔泰和間,蓋嘗南伐,時太平日久,百姓富庶,馬蕃軍銳,所謂萬全之舉也,然猶亟和,以偃兵為務。大安之後,
 北兵大舉,天下騷然者累年,然軍馬氣勢,視舊纔十一耳。至於器械之屬,亦多損弊,民間差役重繁,浸以疲乏,而日勤師旅,遠近動搖,是未獲一敵而自害者眾,其不可一也。今歲西北二兵無入境之報,此非有所憚而不敢也,意者以去年北還,姑自息養,不然則別部相攻,未暇及我。如聞王師南征,乘隙併至,雖有潼關、大河之險,殆不足恃,則三面受敵者首尾莫救,得無貽後悔乎?其不可二也。凡兵雄于天下者,必其士馬精強,器械犀利,且出其不備而後能取勝也。宋自泰和再修舊好,練兵峙糧,繕修營壘,十年于茲矣。又車駕至汴益近宋境,彼
 必朝夕憂懼,委曲為防。況聞王師已出唐、鄧,必徙民渡江,所在清野,止留空城,使我軍無所得,徒自勞費,果何益哉?其不可三也。宋我世仇,比年非無恢復舊疆、洗雪前恥之志,特畏吾威力,不能窺其虛實,故未敢輕舉。今我軍皆山西、河北無依之人,或招還逃軍,脅從歸國,大抵烏合之眾,素非練習,而遽使從戎,豈能保其決勝哉?雖得其城,內無儲蓄,亦何以守?以不練烏合之軍,深入敵境,進不得食,退無所掠,將復遁逃嘯聚為腹心患,其不可四也。發兵進討,欲因敵糧,此事不可必者。隨軍轉輸,則又非民力所及。沿邊人戶雖有恒產,而賦役繁重,
 不勝困憊。又凡失業寓河南者,類皆衣食不給。貧窮之迫,盜所由生,如宋人陰為招募,誘以厚利,使為鄉導,伺我不虞突而入寇,則內有叛民,外有勍敵,未易圖之,其不可五也。今春事將興,若進兵不還,必違農時,以誤防秋之用,此社稷大計,豈特疆埸利害而已哉!其不可六也。臣愚以為止當遴選材武將士,分布近邊州郡,敵至則追擊,去則力田,以廣儲蓄。至于士氣益強,民心益固,國用豐饒,自可恢廓先業,成中興之功,一區區之宋何足平乎。」詔付尚書省,宰臣以為諸軍既進,無復可議,遂寢。



 既而元帥承裔等取宋大散關,上諭鼎曰:「所得大散
 關,可保則保,不可則焚毀而還。」於是鼎奏:「臣近遣官問諸帥臣,皆曰散關至驀關諸隘,其地遠甚,中間堡壘相望,如欲分屯,非萬人不可。則又有恒州、虢縣所直數關,宋兵皆固守如舊,緩急有事,當復分散關之兵。餘眾數少,必不能支,而鳳翔、恒、隴亦無應援,恐兩失之。且比年以來,民力困於調度,今方春,農事已急,恐妨耕墾,不若焚毀此關,但屯邊隘以張其勢,彼或來侵,互相應援易為力也。」制可。



 二年四月,鼎乞致仕,上遣近侍諭曰:「卿年既耄,朕非不知,然天下事方有次第,卿舊人也,姑宜勉力以終之。」鼎以宣宗多親細務,非帝王體,乃上奏曰:「天
 下之大,萬機之眾,錢穀之冗,非九重所能兼,則必付之有司,天子操大綱、責成功而已。況今多故,豈可躬親細務哉?惟陛下委任大臣,坐收成算,則恢復之期不遠矣。」上覽其奏不悅,謂宰臣曰:「朕惟恐有怠,而鼎言如此何耶?」高琪奏曰:「聖主以宗廟社稷為心,法上天行健之義,憂勤庶政,夙夜不遑,乃太平之階也。鼎言非是。」上喜之。



 三年正月,上言:「沿邊州府官既有減定資歷月日之格,至于掌兵及守禦邊隘者,征行暴露,備歷艱險,宜一體減免,以示激勸。」從之。二月,上言:「近制,軍前立功犯罪之人,行省、行院、帥府不得輒行誅賞。夫賞由中出則恩有
 所歸,茲固至當。至于部分犯罪,主將不得施行,則下無所畏而令莫得行矣。」宰臣難之,上以問樞密院官,對如鼎言,乃下詔,自今四品以下皆得裁決。



 時元帥內族承裔、移剌粘何伐宋,所下城邑多所焚掠,於是鼎上言:「承裔等奉詔宣揚國威,所謂『弔民伐罪』者也。今大軍已克武休,將至興元。興元乃漢中、西蜀喉衿之地,乞諭帥臣,所得城邑姑無焚掠,務慰撫之。誠使一郡貼然,秋毫不犯,則其餘三十軍將不攻自下矣。若拒王師,乃宜有戮。」上甚是其言,遂詔諭承裔。鼎以年老屢上表求致仕,上謂宰臣曰:「胥鼎以老求退,朕觀其精力未衰,已遣人往
 慰諭之。鼎嘗薦把胡魯,以為過己遠甚,欲以自代。胡魯固佳,至於駕馭人材,處決機務,不及鼎多矣。」俄以伐宋有功,遷官一階。



 八月,上言:「臣奉詔兼節制河東,近晉安帥府令百里內止留桑棗果木,餘皆伐之。方今秋收,乃為此舉以奪其事,既不能禦敵而又害民,非計也。且一朝警急,其所伐木豈能盡去,使不資敵乎?他木雖伐,桑棗舍屋獨非木乎,此殆徒勞。臣已下帥府止之,而左都監完顏閭山乃言嘗奉旨清野,臣不知其可。」詔從鼎便宜規畫。是時,大元兵大舉入陜西,鼎多料敵之策,朝臣或中沮之,上諭樞密院官曰:「胥鼎規畫必無謬誤,
 自今卿等不須指授也。」尋又遣諭曰:「卿專制方面,凡事得以從宜規畫,又何必一一中復,徒為逗留也。」



 四年,進封溫國公,致仕,詔諭曰:「卿屢求退,朕初不許者,俟其安好,復為朕用爾。今從卿請,仍可來居京師,或有大事,得就諮決也。」五年三月,上遣近侍諭鼎及左丞賈益謙曰:「自去冬至今,雨雪殊少,民心不安,軍用或闕,為害甚重。卿等皆名臣故老,今當何以處之。欲召赴尚書省會議,恐與時相不合,難於面折,故令就第延問,其悉意以陳,毋有所隱。」元光元年五月,上敕宰相曰:「前平章胥鼎、左丞賈益謙、工部尚書札里吉、翰林學士孛迭,皆致政老
 臣,經練國事,當邀赴省與議利害。」仍遣侍官分詣四人者諭意焉。



 六月,晉陽公郭文振奏:「河朔受兵有年矣,向皆秋來春去,今已盛暑不回,且不嗜戕殺,恣民耕稼,此殆不可測也。樞府每檄臣會合府兵進戰,蓋公府雖號分封,力實單弱,且不相統攝,方自保不暇,朝廷不即遣兵為援,臣恐人心以謂舉棄河北,甚非計也。伏見前平章政事胥鼎,才兼將相,威望甚隆,向行省河東,人樂為用。今雖致政,精力未衰,乞付重兵,使總制公府,同力戰禦,庶幾人皆響應,易為恢復,惟陛下圖之。」



 明年,宣宗崩,哀宗即位。正大二年,起復,拜平章政事,進封英國公,行
 尚書省于衛州。鼎以衰病辭,上諭曰:「卿向在河東,朝廷倚重。今河朔州郡多歸附,須卿圖畫。卿先朝大臣,必濟吾事,大河以北,卿皆節制。」鼎乃力疾赴鎮,來歸者益眾。鼎病不能自持,復申前請,優詔不許。三年,復上章請老,且舉朝賢練軍政者自代。詔答曰:「卿往在河東,殘破孤危,殆不易保,卿一至而定。迄卿移鎮,敵不復侵。何乃過為嫌避?且君臣均為一體,朕待下亦豈自殊,自外之語,殆為過計。況餘人才力孰可副卿者?卿年高久勞於外,朕豈不知,但國家百年積累之基,河朔億萬生靈之命,卿當勉出壯圖,同濟大事。」鼎奉詔惶懼不敢退。是年七
 月,薨。



 鼎通達吏事,有度量,為政鎮靜,所在無賢不肖皆得其懽心。南渡以來,書生鎮方面者,惟鼎一人而已。



 侯摯,初名師尹,避諱改今名,字莘卿,東阿人。明昌二年進士,入官慷慨有為。承安間,積遷山東路鹽使司判官。泰和元年,以課增四分,特命遷官二階。八年七月,追官一階,降授長武縣令。初,摯為戶部主事,與王謙規措西北路軍儲以代張煒,摯上章論本路財用不實,至是降除焉。貞祐初,大兵圍燕都,時摯為中都曲使,請出募軍,已而嬰城有功,擢為右補闕。二年正月,詔摯與少府監丞李向秀分詣西山招撫。宣宗南渡,轉勸農副使,提控
 紫荊等關。俄遷行六部侍郎。三年四月,同簽樞密院阿勒根訛論等以謂「今車駕駐南京,河南兵不可易動,且兵不在多,以將為本。侯摯有過人之才,倘假以便宜之權,使募兵轉糧,事無不克,可陞為尚書,以總制永錫、慶壽兩軍。」於是以摯為太常卿,行尚書六部事,往來應給之。



 摯遂上章言九事,其一曰:「省部所以總天下之紀綱,今隨路宣差便宜、從宜,往往不遵條格,輒劄付六部及三品以下官,其於紀綱豈不紊亂,宜革其弊。」其二曰:「近置四帥府,所統兵校不為不眾,然而弗克取勝者,蓋一處受敵,餘徒傍觀,未嘗發一卒以為援,稍見小卻,則棄
 戈遁去,此師老將怯故也。將將之道,惟陛下察之。」其三曰:「率兵禦寇,督民運糧,各有所職,本不可以兼行,而帥府每令雜進,累遇寇至,軍未戰而丁夫已遁,行伍錯亂,敗之由也。夫前陣雖勝,而後必更者,恐為敵所料耳,況不勝哉。用兵尚變,本無定形,今乃因循不改覆轍,臣雖素不知兵,妄謂率由此失。」其四曰:「雄、保、安肅諸郡據白溝、易水、西山之固,今多闕員,又所任者皆柔懦不武,宜亟選勇猛才幹者分典之。」其五曰:「漳水自衛至海,宜沿流設備,以固山東,使力穡之民安服田畝。」其六曰:「近都州縣官吏往往逋逃,蓋以往來敵中失身者多,兼轉輸
 頻併,民力困弊,應給不前復遭責罰,秩滿乃與他處一體計資考,實負其人。乞詔有司優定等級,以別異之。」其七曰:「兵威不振,罪在將帥輕敵妄舉,如近日李英為帥,臨陣之際酒猶未醒,是以取敗。臣謂英既無功,其濫注官爵並宜削奪。」其八曰:「大河之北,民失稼穡,官無俸給,上下不安,皆欲逃竄。加以潰散軍卒還相剽掠,以致平民愈不聊生。宜優加矜恤,亟招撫之。」其九曰:「從來掌兵者多用世襲之官,此屬自幼驕惰不任勞苦,且心膽懦怯何足倚辦。宜選驍勇過人、眾所推服者,不考其素用之。」上略施行焉。



 時元帥蒲察七斤以通州叛,累遣諜者
 間摯,摯恐為所陷,上章自辯。詔諭之曰:「卿朕素知,豈容間耶。其一意於職,無以猜嫌自沮也。」八月,權參知政事。俄拜參知政事,行尚書省于河北。先是,摯言:「河北東、西兩路最為要地,而真定守帥胡論出輒棄城南奔,州縣危懼。今防秋在邇,甚為可憂,臣願募兵與舊部西山忠義軍往安撫之。」制可,故是有命。十一月,入見。壬申,遣祭河神於宜村。十二月,復行省于河北。



 四年正月,進拜尚書右丞。嘗上言,宜開沁水以便饋運,至是,詔有司開之。是時,河北大饑,摯上言曰:「今河朔饑甚,人至相食,觀、滄等州斗米銀十餘兩,殍殣相屬。伏見沿河上下許販粟
 北渡,然每石官糴其八,彼商人非有濟物之心也,所以涉河往來者特利其厚息而已,利既無有,誰復為之?是雖有濟物之名,而實無所渡之物,其與不渡何異。昔春秋列國各列疆界,然晉饑則秦輸之粟,及秦饑,晉閉之糴,千古譏之。況今天下一家,河朔之民皆陛下赤子,而遭罹兵革,尤為可哀,其忍坐視其死而不救歟!人心惟危,臣恐弄兵之徒,得以藉口而起也。願止其糴,縱民輸販為便。」詔尚書省行之。



 時紅襖賊數萬人入臨沂、費縣之境,官軍敗之,生擒偽宣徽使李壽甫。訊之,則云其眾皆楊安兒、劉二祖散亡之餘,今復聚及六萬,賊首郝定
 者兗州泗水人,署置百官,僭稱大漢皇帝,已攻泰安、滕、兗、單諸州,及萊蕪、新泰等十餘縣,又破邳州硇子堌,得船數百艘,近遣人北構南連皆成約,行將跨河為亂。摯以其言聞于上,且曰:「今邳、滕之路不通,恐實有此謀。」遂詔摯行省事于東平,權本路兵馬都總管,以招誘之,若不從即率兵捕討。興定元年四月,濟南、泰安、滕、兗等州土賊並起,肆行剽掠,摯遣提控遙授棣州防禦使完顏霆率兵討之,前後斬首千餘,招降偽元帥石花五、夏全餘黨壯士二萬人,老幼五萬口。



 是年冬,升資德大夫,兼三司使。二年二月,摯上言:「山東、河北數罹兵亂,遺民嗷
 嗷,實可哀恤,近朝廷遣官分往撫輯,其惠大矣。然臣忝預執政,敢請繼行,以宣布國家德信,使疲瘵者得以少蘇,是亦圖報之一也。」宰臣難之,無何,詔遣摯行省于河北,兼行三司安撫事。既行,又上言曰:「臣近歷黃陵崗南岸,多有貧乏老幼自陳本河北農民,因敵驚擾故南遷以避,今欲復歸本土及春耕種,而河禁邀阻。臣謂河禁本以防閑自北來者耳,此乃由南而往,安所容姦,乞令有司驗實放渡。」詔付尚書省,宰臣奏「宜令樞府講究」,上曰:「民飢且死,而尚為次第何耶?其令速放之。」



 四月,招撫副使黃摑阿魯答破李全於密州。初,賊首李全據密州
 及膠西、高密諸縣,摯督兵討之。會高密賊陳全等四人默白招撫副使黃摑阿魯答,願為內應,阿魯答乃遣提控朱琛率兵五百赴之。時李全暨其黨於忙兒者皆在城中,聞官軍且西來,全潛逸去,忙兒不知所為。阿魯答馳抵城下,鼓噪逼之,賊守陴者八百人皆下乞降,餘賊四千出走,進軍邀擊之,斬首千級,俘百餘人,所獲軍實甚眾,遂復其城。是夜,琛又用陳全計,拔高密焉。六月,上遣諭摯曰:「卿勤勞王家,不避患難,身居相職而往來山堌水寨之間,保庇農民收獲二麥,忠恪之意朕所具知。雖然,大臣也,防秋之際亦須擇安地而處,不可墮其計
 中。」摯對曰:「臣蒙大恩,死莫能報,然承聖訓,敢不奉行。擬駐兵于長清縣之靈巖寺,有屋三百餘間,且連接泰安之天勝寨,介於東平、益都之間,萬一兵來,足相應援。」上恐分其兵糧,乃詔權移邳州行省。



 九月,摯上言:「東平以東累經殘毀,至于邳、海尤甚,海之民戶曾不滿百而屯軍五千,邳戶僅及八百,軍以萬計。夫古之取兵以八家為率,一家充軍七家給之,猶有傷生廢業、疲於道路之嘆。今兵多而民不足,使蕭何、劉晏復生,亦無所施其術,況於臣者何能為哉。伏見邳,海之間,貧民失業者甚眾,日食野菜,無所依倚,恐因而嘯聚以益敵勢。乞募選為
 兵,自十月給糧,使充戍役,至二月罷之,人授地三十畝,貸之種粒而驗所收獲,量數取之,逮秋復隸兵伍。且戰且耕,公私俱利,亦望被俘之民易於招集也。」詔施行之。



 是時,樞密院以海州軍食不足,艱于轉輸,奏乞遷于內地。詔問摯,摯奏曰:「海州連山阻海,與沂、莒、邳、密皆邊隅衝要之地,比年以來為賊淵藪者,宋人資給之故。若棄而他徙,則直抵東平無非敵境,地大氣增,後難圖矣,臣未見其可。且朝廷所以欲遷者,止慮糧儲不給耳。臣請盡力規畫,勸喻農民趨時耕種,且令煮鹽易糧,或置場宿遷,以通商旅,可不勞民力而辦。仍擇沭陽之地可以
 為營屯者,分兵護邏,雖不遷無患也。」上是其言,乃止。



 十月,先是,邳州副提控王汝霖以州廩將乏,扇其軍為亂。山東東路轉運副使兼同知沂州防禦使程戩懼禍及己,遂與同謀,因結宋兵以為外應。摯聞,即遣兵捕之,訊竟具伏,汝霖及戩并其黨彈壓崔榮、副統韓松、萬戶戚誼等皆就誅,至是以聞。三年七月,設汴京東、西、南三路行三司,詔摯居中總其事焉。十月,以裏城畢工,遷官一階。四年七月,遷榮祿大夫,致仕。



 天興元年正月,起復為大司農。四月,歸大司農印,復致仕。八月,復起為平章政事,封蕭國公,行京東路尚書省事。以軍三千護送就舟張
 家渡,行至封丘,敵兵覺,不能進。諸將卒謀倒戈南奔,留數騎衛摯。摯知其謀,遂下馬,坐語諸將曰:「敵兵環視,進退在我。汝曹不思持重,吾寧死於汝曹之手,不忍為亂兵所蹂,以辱君父之命。」諸將諾而止,得全師以還,聞者壯之。十一月,復致仕。居汴中,有園亭蔡水濱,日與耆舊宴飲。及崔立以汴城降,為大兵所殺。



 摯為人威嚴,御兵人莫敢犯。在朝遇事敢言,又喜薦士,如張文舉、雷淵、麻九疇輩皆由摯進用。南渡後宰執中,人望最重。



 把胡魯,不詳其初起。貞祐二年五月,宣宗南遷,由左諫議大夫擢為御前經歷官,上面諭之曰:「此行,軍馬朕自
 總之,事有利害可因近侍局以聞。」三年十一月,出為彰化軍節度使,兼涇州管內觀察使。四年五月,改知京兆府事,兼本路兵馬都總管,充行省參議官。



 興定元年三月,授陜西路統軍使,兼前職。二年正月,召為御史中丞。三月,上言:「國家取人,惟進士之選為重,不求備數,務在得賢。竊見今場會試,考官取人泛濫,非求賢之道也。宜革其弊,依大定舊制。」詔付尚書省集文資官雜議,卒依泰和例行之。



 是月,拜參知政事。六月,詔權左副元帥,與平章胥鼎同事防秋。三年六月,平涼等處地震,胡魯因上言:「皇天不言,以象告人,災害之生,必有其故,乞明諭
 有司,敬畏天戒。」上嘉納之,遣右司諫郭著往閱其迹,撫諭軍民焉。四年四月,權尚書右丞、左副元帥,行尚書省、元帥府于京兆。時陜西歲運糧以助關東,民力浸困,胡魯上言:「若以舟楫自渭入河,順流而下,庶可少紓民力。」從之。時以為便。



 五年正月,朝議欲復取會州,胡魯上言:「臣竊計之,月當費米三萬石、草九萬稱,轉運丁夫不下十餘萬人。使此城一月可拔,其費已如此,況未必耶。臨洮路新遭劫掠,瘡痍未復,所須芻糧決不可辦,雖復取之慶陽、平涼、鳳翔及邠、涇、寧、原、恒、隴等州,亦恐未能無闕。今農事將興,沿邊常費已不暇給,豈可更調十餘萬
 人以餉此軍。果欲行之,則數郡春種盡廢矣。政使此城必得,不免留兵戍守,是飛挽之役,無時而已也。止宜令承裔軍于定西、鞏州之地,護民耕稼,俟敵意怠,然後取之。」詔付省院曰:「其言甚當,從之可也。」



 三月,上言:「禦敵在乎強兵,強兵在乎足食,此當今急務也。竊見自陜以西,州郡置帥府者九,其部眾率不過三四千,而長校猥多,虛糜廩給,甚無謂也。臣謂延安、風翔、恐州邊隅重地固當仍舊,德順、平涼等處宜皆罷去。河南行院、帥府存沿邊並河者,餘亦宜罷之。」制可。



 是年十月,西北兵三萬攻延安,胡魯遣元帥完顏合達、元帥納合買住禦之,遂保
 延安。先是,胡魯以西北兵勢甚大,屢請兵於朝,上由是惡之。元光元年正月,遂罷參知政事,以知河中府事權安撫使。於是陜西西路轉運使夾谷德新上言曰:「臣伏見知河中府把胡魯廉直忠孝,公家之利知無不為,實朝廷之良臣也。去歲,兵入延安,胡魯遣將調兵,城賴以無,不為無功。今合達、買住各授世封,而胡魯改知河中府。切謂方今用人之時,使謀略之臣不獲展力,緩急或失事機。誠宜復行省之任,使與承裔共守京兆,令合達、買住捍禦延安,以籓衛河南,則內外安矣。」不報。



 六月,召為大司農,既至汴,遂上言曰:「邇來群盜擾攘,侵及內地,陳、
 潁去京不及四百里,民居稀闊,農事半廢、蔡、息之間十去八九。甫經大赦,賊起益多,動計數百,驅牛焚舍,恣行剽掠,田穀雖熟,莫敢獲者。所在屯兵率無騎士,比報至而賊已遁,叢薄深惡,復難追襲,則徒形跡而已。今向秋成,奈何不為處置也。」八月,復拜參知政事,上謂之曰:「卿頃為大司農,巡行郡縣,盜賊如何可息?」對曰:「盜賊之多,以賦役多也。賦役省則盜賊息。」上曰:「朕固省之矣。」胡魯曰:「如行院、帥府擾之何。」上曰:「司農官既兼採訪,自今其令禁止之。」



 初,胡魯拜命日,巡護衛紹王宅都將把九斤來賀,御史粘割阿里言:「九斤不當遊執政門,胡魯亦不
 當受其賀,請併案之。」於是詔諭曰:「卿昔行省陜西,擅出繫囚,此自人主當行,非臣下可專,人茍有言,其罪豈特除名。朕為卿地,因而肆赦,以弭眾口,卿知之乎?今九斤有職守,且握兵柄,而縱至門下,法當責降,朕重卿素有直氣,故復曲留。公家事但當履正而行,要取人情何必爾也,卿其戒之。」是年十二月,進拜尚書右丞。



 元光二年正月,上諭宰臣曰:「陜右之兵將退,當審後圖,不然今秋又至矣。右丞胡魯深悉彼中利害,其與共議之。」尋遣胡魯往陜西,與行省賽不、合達從宜規畫焉。哀宗即位,以有冊立功,進拜平章政事。正大元年四月,薨。詔加贈右
 丞相、東平郡王。胡魯為人忠實,憂國奉公。及亡,朝廷公宰,下迨吏民,皆嗟惜之。



 師安石,字子安,清州人,本姓尹氏,避國諱更焉。承安五年詞賦進士。為人輕財尚義。初補尚書省令史,適宣宗南遷,留平章完顏承暉守燕都。承暉將就死,以遺表託安石使赴行在,安石間道走汴以聞。上嘉之,擢為樞密院經歷官。時哀宗在春宮,領密院事,遂見知遇。元光二年,累遷御史中丞。其七月,上章言備禦二事,其一曰:「自古所以安國家、息禍亂,不過戰、守、避、和四者而已。為今之計,守、和為上。所謂守者,必求智謀之士,使內足以得
 戍卒之心,外足以挫敵人之銳,不惟彼不能攻,又可以伺其隙而敗之。其所謂和,則漢、唐之君固嘗用此策矣,豈獨今日不可用乎。乞令有司詳議而行。」其二曰:「今敵中來歸者頗多,宜豐其糧餉,厚其接遇,度彼果肯為我用,則擇有心力者數十人,潛往以誘致其餘。來者既眾,彼必轉相猜貳,然後徐起而圖之,則中興之功不遠矣。」上嘉納之。



 九月,坐劾英王守純附奏不實,決杖追官。及哀宗即位,正大元年,擢為同簽樞密院事。二年,復御史中丞。三年,工部尚書、權左參政。四年,進尚書右丞。五年,臺諫劾近侍張文壽、張仁壽、李麟之,安石亦論列三人
 不已,上怒甚,有旨謂安石曰:「汝便承取賢相,朕為昏主,止矣。」如是數百言。安石驟蒙任用,遽遭摧折,疽發腦而死,上甚悼惜之。



 贊曰:宣宗南遷,天命去矣,當是時雖有忠良之佐、謀勇之將,亦難為也。然而汝礪、行信拯救于內,胥鼎、侯摯守禦于外,訖使宣宗得免亡國,而哀宗復有十年之久,人才有益于人國也若是哉。胡魯養兵惜穀之論,善矣。安石不負承暉之託,遂見知遇,以論列近侍觸怒而死,悲夫!



\end{pinyinscope}