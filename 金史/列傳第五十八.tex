\article{列傳第五十八}

\begin{pinyinscope}

 世戚



 ○石家奴裴滿達忽睹徒單恭烏古論蒲魯虎唐括德溫烏古論粘沒曷蒲察阿虎迭烏林答暉蒲察鼎壽徒單思忠徒單繹烏林答復烏古論元忠子誼唐括貢烏林答琳徒單公弼徒單銘
 徒單四喜



 金昭祖娶徒單氏,后妃之族,自此始見。世祖時,烏春為難,世祖欲求昏以結其歡心,烏春曰:「女直與胡里改豈可為昏。」世宗時,賜夾谷清臣族同國人。清臣,胡里改人也。然則四十七部之中亦有不通昏因者矣,其故則莫能詰也。有國家者,昏因有恒族,能使風氣淳固,親義不渝,而貴賤等威有別焉,蓋良法也歟。作《世戚傳》。



 石家奴,蒲察部人,世居案出虎水。祖斛魯短,世祖外孫。桓赧、散達之亂,昭肅皇后父母兄弟皆在敵境,斛魯短以計迎還之。石家奴自幼時撫養於太祖家,及長,太祖
 以女妻之。年十五,從攻寧江州,敗遼主親軍,攻臨潢府皆有功,襲謀克。。其後,自山西護齊國王謀良虎之喪歸上京,道由興中。是時,方攻興中未下,石家奴置柩於驛,率其所領猛安兵助王師,遂破其城。



 從宗望討張覺。再從宗翰伐宋。宗翰聞宗望軍已圍汴,遣石家奴計事,抵平定軍遇敵兵數萬,敗之,遂見宗望。已還報,宗翰聞其平定之戰,甚嘉之。明年,復伐宋,石家奴隸婁室軍。婁室討陜西未下,石家奴領所部兵援之。既而,以本部屯戍西京,會契丹大石出奔,以余睹為元帥,石家奴為副,襲諸部族以還。未幾,有疾,退居鄉里。



 天眷間,授侍中、駙馬
 都尉。再以都統定邊部,熙宗賜御書嘉獎之。封蘭陵郡王。除東京留守,以病致仕。卒,年六十三,加贈鄖王。正隆奪王爵,封魯國公。



 裴滿達,本名忽撻,婆盧木部人。為人淳直孝友。天輔六年,從蒲家奴追叛寇於鐵呂川,力戰有功。熙宗娶忽達女,是為悼平皇后。天眷元年,授世襲猛安。明年,以皇后父拜太尉,封徐國公。皇統元年,除會寧牧。居數歲,以太尉奉朝請。九年,悼后死。無何,海陵弒熙宗,欲邀眾譽,揚熙宗過惡,以悼后死非罪,於是封忽撻為王。天德三年,薨。子忽睹,為燕京留守,以罪免,居中都,海陵命馳驛赴
 之。及葬,使秘書監納合椿年致祭,賻銀五百兩。



 忽睹,天眷三年權猛安,皇統元年為行軍猛安。歷橫海、崇義軍節度使,以后戚怙勢贓汙不法。其在橫海,拜富人為父,及死,為之行服而分其資。在崇義,諷寺僧設齋而受其施。及留守中京,益驕恣,茍可以得財無不為者。選諸猛安富人子弟為扎野,規取財物,時號「閑郎君」。朝廷以忽睹與徒單恭等汙濫至甚,命秉德黜陟天下官吏,忽睹以贓罷。海陵以忽睹所至縱家奴擾民,乃定禁外官任所閑雜人條約。天德三年,復起為鄭州防禦使,改安國軍節度使。卒,年三十九。



 徒單恭,本名斜也。天眷二年,為奉國上將軍。以告吳十反事,超授龍虎衛上將軍。為戶部侍郎,出為濟南尹,遷會寧牧,封譚國公。復出為太原尹。斜也貪鄙,使工繪一佛像,自稱嘗見佛,其像如此,當以金鑄之。遂賦屬縣金,而未嘗鑄佛,盡入其家,百姓號為「金總管」。秉德廉訪官吏,斜也以贓免。



 海陵篡立,海陵后徒單氏,斜也女,由是復用為會寧牧,封王。未幾,拜平章政事,海陵獵於胡剌渾水,斜也編列圍場,凡平日不相能者輒杖之。海陵謂宰相曰:「斜也為相,朕非私之。今聞軍國大事凡斜也所言,卿等一無取,豈千慮無一得乎?」他宰相無以對,溫都
 思忠舉數事對曰:「某事本當如此,斜也輒以為如彼,皆妄生異議,不達事宜。臣逮事康宗,累朝宰相未嘗有如斜也專恣者。」海陵默然。斜也於都堂脊杖令史馮仲尹,御史臺劾之,海陵杖之二十。斜也猛安部人撒合出者,言斜也強率取部人財物。海陵命侍御史保魯鞫之。保魯鞫不以實,海陵杖保魯,而以撒合出為符寶祗候,改隸合扎猛安。



 斜也兄定哥尚太祖長女兀魯,定哥死無子,以季弟之子查剌為後。斜也謀取其兄家財,強納兀魯為室而不相能,兀魯嘗怨詈斜也。斜也妾忽撻與兀魯不葉,乃譖兀魯於海陵后徒單氏曰:「兀魯怨上殺其
 兄宗敏,有怨望語。」會韓王亨改廣寧尹,諸公主宗婦往賀其母,兀魯以言慰亨母,忽撻亦以怨望指斥誣兀魯。海陵使蕭裕鞫之,忽撻得幸于徒單后,左驗皆不敢言,遂殺兀魯,斜也因而盡奪查剌家財。大定間皆追正之。海陵以兀魯有怨望語,斜也不奏,遂杖斜也,免所居官。俄,復為司徒,進拜太保,領三省事,兼勸農使。再進太師,封梁晉國王。



 貞元二年九月,斜也從海陵獵于順州。方獵,聞斜也薨,即日罷獵,臨其喪,親為擇葬地,遣使營治。及葬,賜轀輬車,上及后率百官祭之,賜謚曰忠。正隆間,改封趙國王,再進齊國公。



 其妻先斜也卒,海陵嘗至其
 葬所致祭,起復其子率府率吾里補為諫議大夫。大定間,海陵降為庶人,徒單氏為庶人妻,斜也降特進鞏國公。



 烏古論蒲魯虎,父當海,國初有功。蒲魯虎通契丹大小字,娶宋王宗望女昭寧公主什古。熙宗初,為護衛,改牌印,常侍左右。轉通進。襲父謀克,再遷臨海軍節度使,改衛州防禦使。海陵賜食內殿,謂之曰:「衛州風土甚佳,勿以防禦為降也。」對曰:「頗聞衛州官署不利守者。」即日改汾陽軍節度使,賜衣服、佩玉、帶劍。入為太子詹事,卒,年四十一。海陵親臨哭之,后妃皆弔祭,賻贈甚厚。有司給
 喪事,贈特進駙馬都尉。正隆例贈光祿大夫。



 唐括德溫,本名阿里,上京率河人也。曾祖石古,從太祖平臘醅麻產,領謀克。祖脫孛魯,領其父謀克,從太祖伐遼,攻寧江、泰州戰有功。父撻懶,尚康宗女,從宋王宗望以軍二萬收平州,至城東十里許遇敵兵甚眾,戰敗之,太祖賞賚甚厚,授行軍猛安。皇統初,遷龍虎衛上將軍,歷興平、臨海等軍節度使。



 德溫善射,尚睿宗皇帝女楚國長公主。天眷三年,授宣武將軍。皇統元年,從都元帥宗弼南征,以善突戰遷廣威將軍。六年,遷定遠大將軍。七年,授殿前右副都點檢。天德初,改殿前左副都點檢,
 遷兵部尚書。出為大名尹兼本路兵馬都總管,改橫海軍節度使,延安尹兼鄜延路兵馬都總管。世宗即位,封道國公,為殿前都點檢、駙馬都尉。大定二年,以父祖功授按出虎猛安所管世襲謀克。三年九月九日,世宗以故事出獵,謂德溫曰:「扈從軍士二千,飲食芻秣能無擾百姓乎。」嚴為約束,仍以錢一萬貫分給之。四年,為勸農使,出為西京留守,賜犀弓玉帶,召入為皇太子太傅,卒。上輟朝,親臨喪奠祭,賻贈甚厚。



 十八年,追錄其父撻懶并德溫前後功,授其長子駙馬都尉鼎世襲西北路沒里山猛安,徙隸泰州。



 烏古論粘沒曷,上京胡剌溫屯人也,移屯河間。祖喚端,太祖伐遼,常侍左右,追遼主延禧、卻夏人援兵皆有功,授世襲謀克。父歡睹,官至廣威將軍。粘沒曷尚睿宗女冀國長公主,初為護衛,天德二年襲謀克。海陵伐宋,為押軍猛安。世宗即位,軍還,授侍衛親軍步軍都指揮使,加駙馬都尉。歷左副點檢,禁直被酒不親視扃鐍,杖四十。遷右宣徽使、勸農使,出為興平軍節度使。改廣寧尹,賜錢三千貫。粘沒曷至廣寧,嗜酒不視事,上以兵部員外郎宗安為少尹,詔宗安戒諭之,上謂宗安曰:「汝能繼修前政,朕不忘汝,勉之。」大定中,粘沒曷卒。上聞之,遣其
 子駙馬都尉公說馳驛奔喪,賜錢三千貫,沿路祭物並從官給。



 蒲察阿虎迭,初授信武將軍,尚海陵姊遼國長公主迪缽,為駙馬都尉。遼國薨,繼尚鄧國長公主崔哥。皇統三年,為右副點檢。五年,使宋為賀正旦使,改左副點檢,禮部、工部尚書,廣寧、咸平、臨潢尹,武定軍節度使,封葛王。薨年二十八。海陵親臨葬,贈譚王。正隆例贈特進楚國公。



 烏林答暉,本名謀良虎,明德皇后兄也。天眷初,充護衛,以捕宗磐、宗雋功授忠勇校尉,遷明威將軍。從宗弼北
 征,遷廣威將軍,賞以金幣、尚廄擊球馬。久之,除殿中侍御史,再除蒲速碗群牧使,謹畜牧,不事遊宴,孳產蕃息,進秩,改特滿群牧使。世宗即位,召見行在,除中都兵馬都指揮使。世宗至中都,將遣使於宋,以暉為使。世宗曰:「暉嘗私用官錢五百貫。」迺數其罪而罷之,遣高忠建往。因謂宰臣曰:「朕於賞罰,豪髮無所假借。果公廉辦治,雖素所不喜,必加升擢,若抵冒公法,雖至親不少恕。」遷都點檢、兼侍衛親軍副都指揮使,卒。遣官致祭,皇太子諸王百官會喪,賻銀千兩、重彩四十端、絹四十匹。詔以暉第三子天錫世襲納鄰河猛安親管謀克。



 蒲察鼎壽,本名和尚,上京曷速河人,欽懷皇后父也。賦性沉厚有明鑒,通契丹、漢字,長於吏事。尚熙宗女鄭國公主。貞元三年,以海陵女弟慶宜公主子加定遠大將軍,為尚衣局使,累官器物局使。大定二年,加駙馬都尉,職如故。歷符寶郎、蠡州刺史、浚州防禦使,有惠政,兩州百姓刻石紀之。遷泰寧軍節度使,歷東平府、橫海軍,入為右宣徽使,改左宣徽,授中都路昏得渾山猛安曷速木單世襲謀克。改河間尹。號令必行,豪右屏跡。有宗室居河間,侵削居民,鼎壽奏徙其族于平州,郡內大治。卒官。上聞之深加悼惜。喪至香山,皇太子往奠,百官致祭,
 賻銀綵絹。明昌三年,以皇后父贈太尉、越國公。



 鼎壽既世連姻戚,女為皇后,長子辭不失凡三尚定國、景國、道國公主。其寵遇如此,未嘗以富貴驕人,當時以為外戚之冠云。



 徒單思忠,字良弼,本名寧慶。曾祖賽補,尚景祖女。從太祖伐遼,戰歿于臨潢之渾河。父賽一,尚熙宗妹。正隆末,為颭碗群牧使,契丹賊窩斡擾北邊,賽一與戰死之。大定初,贈金吾衛上將軍。



 思忠通敏有才,頗通經史。世宗在潛邸,撫養之。賦性寬厚。十有二歲從上在濟南,一日,與姻戚公子出遊近郊,有醉人腰弓矢策馬突過,諸公
 子怒欲鞭之,思忠曰:「醉人昏昧,又何足責。」遂釋之。其人行數十步,忽執弓矢,思忠恐欲傷人,速馳至其傍,奪其弓,弛而還之。上聞之,嘉有識量,由是常使侍側。尚皇弟二女唐國公主。大定初,世宗使思忠迎南征萬戶高忠建、完顏福壽於遼口,察其去就,思忠知其誠意,乃與俱至東京。世宗即位,如中都,思忠從行,軍國庶事補益弘多。大定元年十月,拜殿前左衛將軍,二年,加駙馬都尉,卒。上為輟朝,即喪所臨奠,命有司備禮葬之,營費從官給。



 十九年,上追念思忠輔立功,贈驃騎衛上將軍,乃授其子鐸武功將軍、世襲中都路烏獨渾謀克。



 徒單繹,本名術輩,其先上京按出虎達阿人。祖撒合懣,國初有功,授隆安府路合扎謀克、奪古阿鄰猛安。繹美姿儀,通諸國語。尚熙宗第七女沈國公主。充符寶祗候,遷御院通進,授符寶郎。歷宣德、泰安、淄州刺史,有廉名。改同知廣寧府事,以母鄂國公主憂,不赴。世宗特許以憂制中襲父封。服闕,授同知濟南府事。二十六年,遷棣州防禦使,以政迹聞,升臨海軍節度使,卒。



 繹家世貴寵,自會祖照至繹尚公主者凡四世云。



 烏林答復,本名阿里剌,東平人也。奉御出身,大定七年尚世宗第七女宛國公主,授駙馬都尉。改引進使、兼符
 寶郎,出為蠡州刺史,三遷歸德軍節度使。明昌三年,轉知興中府事,久之,為曷懶路兵馬都總管。承安四年,拜絳陽軍節度使。卒。



 烏古論元忠,本名訛里也,其先上京獨拔古人。父訛論,尚太祖女畢國公主。元忠幼秀異,世宗在潛邸以長女妻之,後封魯國大長公主。正隆末,從海陵南伐。世宗即位遼陽,時太保昂為海陵左領軍大都督,遣元忠朝于行在,遂授定遠大將軍,擢符寶郎。諭之曰:「朕初即位,親密無如汝者,侍從宿衛,宜戒不虞。」大定二年,加駙馬都尉,除近侍局使,遷殿前左衛將軍。從世宗獵,上欲射虎,
 元忠諫止之。進殿前右副都點檢,為賀宋正旦使,還,轉左副都點檢。坐家奴結攬民稅,免官。十一年,復舊職。明年,升都點檢。十五年,北邊淮獻,命元忠往受之,及還,詔諭曰:「朕每遇卿直宿,其寢必安。今夏幸景明宮,卿去久,朕甚思之。」



 會大興府守臣闕,遂以元忠知府事。有僧犯法,吏甫得置獄,皇姑梁國大長公主屬使釋之,元忠不聽,主奏其事,世宗召謂曰:「卿不徇情,甚可嘉也,治京如此,朕復何憂。」秩滿,授吏部尚書。以其子誼尚顯宗長女薛國公主。十八年,擢御史大夫,授撒巴山世襲謀克。世宗問左丞相紇石烈良弼孰可相者,良弼以元忠對,乃
 拜平章政事,封任國公,進尚書右丞相。策論進士之科設,元忠贊成之。世宗將幸會寧,元忠進諫不聽,出知真定府,尋復詔為右丞相。



 世宗欲甓上京城,元忠曰:「此邦遭正隆軍興,百姓凋弊,陛下休養二十餘年,尚未完復。況土性疏惡,甓之恐難經久,風雨摧壞,歲歲繕完,民將益因矣。」駕東幸久之未還,元忠奏曰:「鸞輿駐此已閱歲,倉儲日少,市買漸貴,禁衛暨諸局署多逃者,有司捕置諸法恐傷陛下仁愛。」世宗嘉納之。



 尋出為北京留守,責諭之曰:「汝強悍自用,顓權而結近密。汝心叵測,其速之官。」後左丞張汝弼奏事,世宗惡其阿順,謂左右日:「卿等
 每事依違茍避,不肯盡言,高爵厚祿何以勝任。如烏古論元忠為相,剛直敢言,義不顧身,誠可尚也。」於是,改知真定府事,移知河間。明昌二年,知廣寧府。以河間修築球場擾民,會赦下,除順義軍節度使。乞致仕不許,特加開府儀同三司、北京留守。徙知濟南府,過闕,令預宴,班平章政事之上。承安二年,移守南京,尋改知彰德府,卒。訃聞,上遣宣徽使白琬燒飯,賻物甚厚。元忠素貴,性粗豪而內深忌,世宗嘗責之。又所至不能戢奴僕,世以此為訾云。子誼。



 誼本名雄名。大定八年,尚海陵女。宴宗室及六品以上
 官,命婦預焉,上曰:「此女亦太祖之曾孫,猶朕之女,乃父廢亡,非其女之罪也。」海陵女卒,大定二十一年,尚顯宗女廣平郡主。誼歷仕宮衛,為人粗豪類其父。二十六年,上謂原王曰:「元忠勿望其可復相也。雄名又不及乃父,朕嘗宥待,殊不知恩,汝宜知其為人。」謂平章政事襄曰:「雄名可令補外。自今宮掖官已有旨補外者,比及廷授,即毋令入宮。」於是,誼除同知澄州軍州事。章宗即位,廣平郡主進封鄴國長公主,誼改順天軍節度副使,加駙馬都尉。承安元年,累遷秘書監兼吏部侍郎,改刑部,遷工部尚書。泰和元年,遇父元忠憂。二年,以本官起復。三
 年,知東平府事,改知真定府事。六年,伐宋,遷元帥左都監。七年,轉左監軍。八年,拜御史大夫。大安中,知大名府。至寧初,以謀逆伏誅。



 唐括貢,本名達哥,太傅阿里之子也。尚世宗第四女吳國公主,授駙馬都尉,充奉御。特授拱衛直副都指揮使,五遷刑部侍郎,坐擅離職削官一階,出為德州防禦使。升順天軍節度使,移鎮橫海。召為左宣徽使,遷兵部尚書,改吏部,轉禮部尚書、兼大理卿。先是,大理卿闕,世宗命宰臣選可授者,左丞張汝弼舉西京副留守楊子益法律詳明。上曰:「子益雖明法,而用心不正,豈可任之以
 分別天下是非也?大理須用公正人。」左丞粘割斡特剌舉貢可任以閑簡部分而兼領是職,遂以貢為之。二十八年,拜樞密副使。章宗立,為御史大夫。會貢生日,右丞相襄、參知政事劉瑋、吏部郎中鷿、中都兵馬都指揮使和喜為貢壽,遂犯夜禁,和喜遣軍人送襄至第。監察御史徒單德勝劾其事,下刑部逮鷿等問狀。上以襄、瑋大臣釋之,而貢等各解職。尋知大興府事,復為樞密副使。乞致仕不許,進樞密使,封莘國公,改封蕭。復上表乞退,上曰:「向已嘗告,續知意欲外除,今之告將復若何。」遂優詔許之。尋起知真定府事。泰和二年,薨。



 烏林答琳,本名留住。尚郜國公主,加駙馬都尉。貞祐元年為靜難軍節度使。夏人犯邠州,琳降。會延安府遣通事張福孫至夏國,夏人使福孫見琳,時已中風,公主令人以狀付福孫,屬以懇禱朝廷,冀早太平得還鄉之意。福孫具以聞,詔賜以藥物。



 徒單公弼,本名習烈,河北東路算主海猛安人。父府君奴,尚熙宗女,加駙馬都尉,終武定軍節度使。公弼初充奉御,大定二十七年,尚世宗女息國公主,加定遠大將軍、駙馬都尉,改器物局直長。轉副使、兼近侍局直長。丁父憂,起復本局副使。章宗秋山射中虎,虎怒突而前,侍
 衛皆避去,公弼不動,虎亦隨斃。詔責侍衛而慰諭公弼。除濱州刺史,再遷兵部侍郎,累除知大名府事。是時,伐宋軍興,有司督逋租及牛頭稅甚急,公弼奏:「軍士從戎,民亦疲弊,可緩徵以紓民。」朝廷從之。大安初,知大興府事,讞武清盜,疑其有冤,已而果獲真盜。歲餘拜參知政事,進右丞,轉左丞。至寧初,拜平章政事,封定國公。貞祐初,進拜右丞相,罷知中山府事。是時,中都圍急不可行,圍解,宣宗曰:「中山新被兵,不如河中善。」乃改知河中府。歷定國軍節度使事、太孫太師、同判大睦親府事。興定五年薨,宣宗輟朝,賻贈,謚恪愿。



 徒單銘,字國本,顯宗賜名重泰。祖貞,別有傳。父特進、涇國公。性重默寡言,粗通經史,事母盡孝。大定末,充奉御。章宗即位,特敕襲中都路渾特山猛安。明昌五年,授尚醖署直長,累遷侍儀司令、宿直將軍、尚衣局使、兵部郎中,與大理評事孫人鑑為採訪使,覆按提刑司事。改右衛將軍,轉左衛,出為永定軍節度使,移河東北路按察使、轉運使。大安三年,改知大名府,就升河北東西、大名路安撫使。大名薦饑重困,銘乞大出交鈔以賑之。崇慶初,移知真定府,復充河北東西、大名路宣撫使。至寧元年九月,奉迎宣宗于彰德府,俄拜尚書右丞,出為北京
 留守,以路阻不能赴。貞祐二年,卒。



 贊曰:天子娶后,王姬下嫁,豈不重哉。秦、漢以來,無世世甥舅之家。《關雎》之道缺,外戚驕盈,《何彼穠矣》不作,王姬肅雝之義幾希矣。蓋古者異姓世爵公侯與天子為昏因,他姓不得參焉。女為王后,己尚王姬,而自貴其貴,富厚不加焉,寵榮不與焉。使漢、唐行此道,則無呂氏、王氏、武氏之難,公主下嫁各安其分、各得其所矣。金之徒單、拿懶、唐括、蒲察、裴滿、紇石烈、僕散皆貴族也,天子娶后必于是,公主下嫁必于是,與周之齊、紀無異,此昏禮之最得宜者,盛於漢、唐矣。



 徒單四喜,哀宗皇后之弟也。天興二年正月辛酉夜,四喜、內侍馬福惠至自歸德,時河朔已失利,京城猶未知,二人被旨迎兩宮,遂託以報捷,執小黃旗以入,至則奏兩宮以奉迎之意。是日,召二相入議,二相及烏古孫奴申諫不可行。四喜作色曰:「我奉制旨迎兩宮,有敢言不行者,當以別敕從事矣。」二相不復敢言,行議遂決。制旨所取兩宮、柔妃裴滿氏及令人張秀蕊、都轄、承御、湯藥、皇乳母鞏國夫人等十餘人外,皆放遣之。又取宮中寶物,馬蹄金四百杖、大珠如慄黃者七千杖、生金山一、龍腦板二及信瑞御璽,仍許賜忠孝軍以兩宮隨行物之
 半。



 壬寅,太后御仁安殿,出錠金及七寶金洗分賜忠孝軍。是夜,兩宮騎而出,至陳留,見城外二三處火起,疑有兵,遲回間,奴申初不欲行,即承太后旨馳還。癸卯,入京頓四喜家,少頃,還宮。復議以是夜再往,太后憊於鞍馬不能動,遂止。



 明日,崔立變。四喜、術甲塔失不及塔失不之父咬住、四喜妻完顏氏,以忠孝卒九十七騎奪曹門而出,將往歸德,不得出,轉陳州門,亦為門卒所止。門帥裕州防禦使阿不罕斜合已遁去,經歷官完顏合住權帥職,麾門卒放塔失不等去,且曰:「罪在我,非汝等之過。」明日,立以數十騎召合住,合住自分必死,易衣冠而往。
 立左右扼腕欲加刃。立遙見,問:「汝是放忠孝軍出門者耶?」合住曰:「然。天子使命,某實放之,罪在某。」立忽若有所省,顧群卒言:「此官人我識之,前築裏城時與我同事。我所部十餘卒盜官木罪當死,此官人不之問,但笞數十而已。此家能殺人,能救人。」因好謂合住曰:「業已放出,吾不汝罪也。」



 四喜等至歸德,上驚問兩宮何如,二人奏京城軍變不及入宮。上曰:「汝父汝妻獨得出耶。」下之獄,皆斬於市。



 贊曰:四喜奉迎兩宮,而值崔立之變,智者居此,與兩宮周旋兵間,以俟事變之定而徐圖之。萬一不然,以一死
 徇之耳,他無策也。四喜奉其私親以歸,而望人主貸其死,豈非愚乎!



\end{pinyinscope}