\article{志第三十八}

\begin{pinyinscope}

 百官三



 ○內命婦宮人女職東宮官屬親王府屬太后兩宮官屬大興府諸京留守司諸京城宮苑提舉等職按察司諸路總管府諸府諸節鎮防禦刺史縣鎮等職諸轉運泉穀等職諸府鎮兵馬等職諸猛安部族及群牧等職



 內命婦品



 元妃、貴妃、淑妃、德妃、賢妃,正一品。昭儀、昭容、昭媛、修儀、修容、修媛、充儀、充容、充媛曰九嬪,正二品。婕妤,正三品。美人,正四品。才人,正五品。各九員,曰二
 十七世婦。寶林,正六品。御女,正七品。采女,正八品。各二十七員,曰八十一御妻。按金格,貞祐後之制,貴妃下有真妃,淑妃下有麗妃、柔妃,而無德妃、賢妃。九嬪同。婕妤下有麗人、才人為正三品。順儀、淑華、淑儀為正四品。尚宮夫人,尚宮左夫人、尚宮右夫人、宮正夫人、寶華夫人、尚儀夫人、尚服夫人、尚寢夫人、欽聖夫人、資明夫人為正五品。尚儀御侍、尚服御侍、尚寢御侍、尚正御侍、寶符宸侍、奉恩令人、奉光令人、奉徽令人、奉美令人為正六品,司正御侍、寶符御侍、司儀御侍、司符御侍,司寢御侍、司飾御侍、司設御侍、司衣御侍、司膳御侍、司藥御侍、仙韶使、光訓良侍、明訓良侍、遵訓良侍、從訓良侍為正七品。典儀御侍、典膳御侍、典寢御侍、典飾御侍、典設御侍、典衣御侍、典藥御侍、仙韶副使、承和良侍、承惠良侍、承宜良侍為正八品。掌儀御侍、掌服御侍、掌寢御侍、掌飾禦侍、掌設御侍、掌衣御侍、掌膳御侍、掌藥御侍、仙韶掌音、祗肅良侍、祗敬良侍、祗願良侍為正九品。



 宮人女官職員品秩,皆同唐制。



 尚宮二人,掌導引皇后,管司記、司言、司簿、司闈,仍總知五尚須物出納等事。司記二人、典記二人、掌記二人,掌在內諸文書出入目錄,為記審訖付行縣印等事。女史六人,掌職文簿。司言二人、典言二人、掌言二人、女史四人,掌宣傳啟奏之事。司簿二人、典簿二人、掌簿二人、女史六人,掌宮人名簿廩賜之事。司闈六人、典闈六人、掌闈六人、女史四人,掌宮闈管鑰之事。尚儀二人,掌禮儀起居、管司籍、司樂、司賓、司贊事。司籍二人、典籍二人、掌籍二人、女史十人,掌經籍教學紙筆几案之事。司樂四
 人、典樂四人、掌樂四人、女史二人,掌音樂之事。司賓二人、典賓二人、掌賓二人、女史二人,掌賓客參見、朝會引導之事。司贊二人、典贊二人、掌贊二人、女史二人、彤史二人,掌禮儀班序、設板贊拜之事。尚服二人,掌管司寶、司衣、司飾、司仗之事。司寶二人、典寶二人、掌寶二人、女史四人,掌珍寶符契圖籍之事。司衣二人、典衣二人、掌衣二人、女史四人,掌御衣服首飾之事。司飾二人、典飾二人、掌飾二人、女史二人,掌膏沐巾櫛服玩之事。司仗二人、典仗二人、掌仗二人、女史二人,掌仗衛兵器之事。尚食二人,掌知御膳、進食先嘗,
 管司膳、司醖、司藥、司饎事。司膳四人、典膳四人、掌膳四人、女史四人,掌膳羞器皿。司醖二人、典醖二人、掌醖二人、女史二人,掌酒醴。司藥二人、典藥二人、掌藥二人、女史二人,掌醫藥。司饎二人、典饎二人、掌饎二人,女史二人,掌宮人食并柴炭之事。尚寢二人,管司設、司輿、司苑、司燈事。司設二人、典設二人、掌設二人、女史二人,掌帷帳、床褥、枕席、灑掃、鋪設。司輿二人、典輿二人、掌輿二人、女史二人,掌輿傘扇羽儀。司苑二人、典苑二人、掌苑二人、女史二人,掌苑囿種植蔬果。司燈二人、典燈二人、掌燈二人、女史二人,掌燈油火燭。
 尚功二人,掌女功,管司製、司珍、司彩、司計事。司製二人、典製二人、掌製二人、女史二人,掌裁縫衣服篡組之事。司珍二人、典珍二人、掌珍二人、女史二人,掌金珠玉寶財貨之事。司彩二人、典彩二人、掌彩二人、女史二人,掌錦文緋彩絲帛之事。司計二人、典計二人、掌計二人、女史二人,掌支度衣服飲食柴炭雜物之事。宮正二人,掌總知宮內格式、糾正推罰之事。司正二人,同掌。典正二人,糾察違失。女史四人。



 皇后位下女職依隆慶宮所設人數,大安元年定。



 司閨一員,八品。掌宮內諸事並給散宮人俸給食料。秉儀一員,八品。丞儀
 一員,九品。掌左右給事、宣傳啟奏、經籍紙筆之事。直閣一員、司陳一員,九品。掌帳幕床褥輿傘、灑掃鋪陳、薪炭燈燭之事。秉衣一員、奉衣一員,九品。掌首飾衣服器玩諸寶財貨,裁製縑彩之事。掌饌一員,八品。奉饌一員,九品。掌飲食湯藥酒醴蔬果之事。



 東宮官宮師府



 太子太師、太子太傅、太子太保,正二品。太子少師、太子少傅、太子少保,正三品。掌保護東宮,導以德義。海陵天德四年,始定制宮師府三師、三少、詹事院詹事、三寺、十率府皆隸焉。左右諭德,為東宮僚屬。



 詹事院太子詹事,從三品。少詹事,從四品。掌總統東宮內外庶務。左右衛率府率,從五品。掌周衛導從儀仗。左右監門,正六品。掌門衛禁鑰。僕正,正六品。副僕,正七品。僕丞,正九品。掌車馬廄牧弓箭鞍轡器物等事。掌寶二人,從六品。掌奉寶,謹其出入。典儀,從六品。贊儀,從七品。司贊禮儀。侍正,正七品。侍丞,正八品。掌冠帶衣服、左右給使之事。典食令,正八品。丞,正九品。承奉膳羞。侍藥,正八品。奉藥,正九品。承奉醫藥。掌飲令,正八品。丞,正九品。承奉賜茶及酒果之事。家令,正八品。家丞,正九品。掌營繕栽植鋪設及燈燭之事。司經,正
 八品。副,正九品。掌經史圖籍筆硯等事。司藏,從八品。副,從九品。掌庫藏財貨出入之事。司倉,從八品。副,從九品。掌倉廩出納薪炭等事。中侍局都監,正九品。同監,從九品。掌東閣內之禁令、省察宮人廩賜給納諸物、轄侍人等。左諭德、右諭德,正五品。左贊善、右贊善,正六品。掌贊諭道德、侍從文章。內直郎,正七品。



 右屬宮師府。



 親王府屬官



 傅,正四品。掌師範輔導、參議可否,若親王在外,亦兼本京節鎮同知。府尉,從四品。本府長史,從五品。明昌三年改,掌警嚴侍從、兼總統本府之事。司
 馬,從六品。同檢校門禁、總統府事。文學二人,從七品。掌贊導禮義、資廣學問。記室參軍,正八品。掌表箋書啟之事。大定七年八月始置。二十年,不專除,令文學兼之。



 諸駙馬都尉,正四品。



 提舉衛紹王家屬。提舉,從六品。同提舉,從七品。舊為東海郡侯邑令,丞。



 提舉鎬厲王家屬。提舉,同提舉。以上二宅,天興元年始聽自便。



 提控鞏國公家屬。提控。同提控。



 太后兩宮官屬。正大元年置。



 衛尉,從三品。副衛尉,從四
 品。左典禁,右典禁,從五品。奉令,正七品。奉丞,正八品。太僕,正六品。副僕,正七品。門衛二員,正六品。典寶二員,正六品。謁者二員,從六品。閣正,從七品。閣丞,從八品。食官令,正八品。食官丞,正九品。宮令,正八品。宮丞,正九品。醫令,正八品。醫丞,正九品。飲官令,正八品。飲官丞,正九品。主藏,正八品。副主藏、主廩,從八品。副主廩,正九品。



 大興府



 尹一員,正三品。掌宣風導俗、肅清所部,總判府事。餘府尹同。兼領本路兵馬都總管府事。車駕巡幸,則置留守同知、少尹、判官。惟留判不別置,以總判兼
 之。同知一員,從四品。掌通判府事。餘府同知同此。少尹一員,正五品。掌同同知。總管判官一員,從五品。掌紀綱總府眾務,分判兵案之事。府判一員,從五品。掌諮議參佐、糾正非違、紀綱眾務,分判吏部、工案事。推官二員,從六品。掌同府判,分判戶、刑案事,內戶推掌通檢推排簿籍。舊一員,大定五年增一員。知事,正八品。掌付事勾稽省署文牘、總錄諸案之事。都孔目官,女直司一員,漢人司一員,職同知事,掌監印、監受案牘。餘都孔目官同此。不常置,省則吏目攝。六案司吏七十五人,內女直十五人,漢人六十人。司吏分掌六案,各置孔目官一員,掌呈覆糾正本案文書。餘分前後行,其他處應設十人以下、六人以上者,置孔目
 官三人,及置提點所處仍舊。女直司吏若十二人以上,分設六案,不及者設三案,五人以下設一案,通掌六案事。以上名充孔目官。知法三員,從八品。女直一員、漢人二員,掌律令格式、審斷刑名。抄事一人,掌抄事目、寫法狀,以前後行吏人選。公使百人。女直教授一員。東京、北京、上京、河東東西路、山東東西路、大名、咸平、臨潢、陜西統軍司、西南招討司、西北路招討司、婆速路、曷懶路、速頻、蒲與、胡里改、隆州、泰州、蓋州並同此。皆置醫院,醫正一人,醫工八人。



 諸京留守司



 留守一員,正三品。帶本府尹兼本路兵馬都總管。同知留守事一員,正四品。帶同知本府尹兼
 本路兵馬都總管。副留守一員,從四品。帶本府少尹兼本路兵馬副都總管。留守判官一員,從五品。都總管判官一員,從五品。掌紀綱總府眾務、分判兵案之事。推官一員,從六品。掌同府判,分判刑案之事,上京兼管林木事。司獄一員,正八品。司吏。女直司吏,上京二十人,北京十三人,東京十人,南京、西京各五人。漢人司吏,三十萬戶以上六十人,二十五萬戶五十五人,十萬戶以上四十人,七萬戶以上三十五人,五萬戶以上三十人,三萬戶以上二十四人,不及萬戶十人。譯人,上京、北京各三人,東京、西京、南京各二人。通事二人。知法,女直、漢人各一員,南京漢人二員。抄事一人,掌抄錄事目、書寫法狀。公事百人。



 京城門收支器物使。貞祐元年置,每城一面設一員。五年,南京隨門添設。舊有小都監,後
 省。正八品,十四員,戶部闢舉。開陽門、宣仁門、安利門、平化門、通遠門、宜照門、利川門、崇德門、迎秋門、廣澤門、順義門、迎朔門、順常門、廣智門,已上各門副尉兼職。貞祐五年制,乃罷小都監。十四門尉,從七品。副尉,正九品。



 上京提舉皇城司提舉一員,從六品。同提舉一員,從七品。司吏一人。



 南京提舉京城所提舉一員,正七品。同提舉一員,從七品。掌本京城壁及繕修等事,不常置。上京同此。管勾一員,正八品。掌佐繕治。受給官一員,掌收支之事。壕
 寨官一員,掌監督修造。



 皇城使一員,正八品。副使一員,正九品。掌宮闕繕修之事,不常置。



 管勾北太一宮,同樂園二員,正八品。掌守宮園繕修之事。



 慶元宮小都監三員,掌鋪陳祭器諸物。餘宮同。



 花園小都監二員。



 東京宮苑使一員。西京、北京同。



 東京、西京御容殿,閣門各二員,掌享祀禮數、鋪陳祭器。



 東京萬寧宮小都監一員。



 按察司



 本提刑司,承安三年以上京、東京等提刑司併為一提刑使,兼宣撫使勸農採訪事,為官稱。副使、判官以兼宣撫副使、判官為名。復改宣撫為安撫,各設安撫判官一員、提刑一員,通四員。安撫司,掌鎮撫人民、譏察邊防軍旅、審錄重刑事。安撫判官則銜內不帶勸農採訪事,令專管千戶謀克。安撫使副內,差一員於咸平、一員於上京分司。承安四年罷咸平分司,使在上京、副在東京,各設簽事一員。承安四年改按察司,貞祐三年罷,止委監察採訪。



 使一員,正三品。掌審察刑獄、照刷案牘、糾察濫官汙吏豪猾之人、私鹽
 酒曲並應禁之事,兼勸農桑,與副使、簽事更出巡案。副使,正四品。兼勸農事。簽按察司事,正五品。承安四年設。判官二員,從六品。大定二十九年設。明昌元年以陜西地闊,添一員。知事,正八品。承安三年,上京者兼經歷安撫司使。泰和八年十一月,省議以轉運司權輕,州縣不畏,不能規措錢穀,遂詔中都都轉運,依舊專管錢穀事,自餘諸路按察使並兼轉運使,副使兼同知,簽按察並兼轉運副,添按察判官一員,為從六品。中都、西京路按察司官止兼西京路轉運司事。遼東路惟上京按察安撫使及簽事依舊署本司事。
 遼東轉運使兼按察副使,同知轉運使兼簽按察司事,轉運副使兼按察判官,添知事一員。知法二員,從八品。書史四人,書吏十人,抄事一人,公使四十人。右中都、西京並依此置。陜西、上京兩路設簽按察司事二員,上京簽安撫司事。



 上京、東京等路按察司並安撫司。使,正三品。鎮撫人民、譏察邊防軍旅之事,仍專管猛安謀克,教習武藝及令本土純愿風俗不致改易。副使二員,正四品。簽安撫司事,正五品。簽按察司事,正五品。知事兼安撫司事,正八品。知法四員,從八品。書史四人,上京、東京書吏十八人,女直十二人、漢人六人。中都、西京,女直五人,漢人五人。北京、臨潢,女直三人、漢人五人。南京,女直二人、漢人
 七人。山東,女直三人、漢人七人。大名,女直三人、漢人六人。抄事一人,公使十人。右按察使於上京、副使於東京各路設簽事一員,分司勾當。惟安撫司不帶「勸農」字,內知事於上京、自餘並於兩處分減存設。



 諸總管府謂府尹兼領者。



 都總管一員,正三品。掌統諸城隍兵馬甲仗,總判府事。同知都總管一員,從四品。掌通判府事,惟婆速路同知都總管兼來遠軍事兵馬。副都總管一員,正五品。所掌與同知同。總管判官一員,從六品。掌紀綱總府眾務,分判兵案之事。府判一員,從六品。掌紀綱眾務,分判戶、禮案,仍掌通檢推
 排簿籍。推官一員,正七品。掌同府判,分判工、刑案事。知法一員。司吏,女直,山東西路十五人,大名十四人,山東東路、咸平府、臨潢府各十二人,曷懶路、河北西路各十人,婆速路十一人,河北東路八人,河東南北路、京兆、慶陽、臨洮、鳳翔、延安各四人。漢人,戶十八萬以上四十二人,十五萬以上四十人,十三萬以上三十八人,十萬以上三十五人,七萬以上三十二人,五萬以上二十八人,三萬以上二十二人,不及三萬戶二十人,婆速路、曷懶路各二人。譯人,咸平三人,河北東西、山東東西、曷懶、大名、臨潢各二人,餘各一人。通事,婆速、曷懶路高麗通事一人,臨潢北部通事一人,部落通事一人、小部落通事二人,慶陽府通事一人。抄事一人。公使八十人。臨潢別置移剌十三人。凡諸府置員並同,惟曷懶路無府事。



 諸府謂非兼總管府事者。



 尹一員,正三品。同知一員,正四品。少尹一員,正五品。府判一員,從六品。掌紀綱眾務,分判吏、
 戶、禮案事,專掌通檢推排簿籍。推官一員,正七品。掌同府判兵、刑、工案事。府教授一員。知法一員。司吏,女直皆三人,漢人,若管十六萬戶四十人,十四萬以上三十八人,十二萬以上三十五人,十萬以上三十二人,七萬以上三十人,五萬以上二十五人,三萬戶以上二十人,不及三萬戶十七人。譯人一人,通事一人,抄事一人,公使七十人。



 諸節鎮



 節度使一員,從三品。掌鎮撫諸軍防刺,總判本鎮兵馬之事,兼本州管內觀察使事。其觀察使所掌,並同府尹兼軍州事管內觀察使。同知節度使一員,正五品。通判節度使事,兼州事者仍帶同知管內觀察使。副使一員,從五品。節度判官一員,正七品。掌紀
 綱節鎮眾務、僉判兵馬之事,兼判兵、刑、工案事。觀察判官一員,正七品。掌紀綱觀察眾務,僉判吏、戶、禮案事,通檢推排簿籍。知法一員,州教授一員,司獄一員,正八品。司吏,女直,隆州十四人,蓋州十二人,泰州十一人,速頻、胡里改各十人,蒲與八人,平、宗、懿、定、衛、萊、密、滄、冀、邢、同、雄、保、兗、邠、涇、朔、奉聖、豐、雲內、許、徐、鄧、鞏、鄜、全、肇各三人,餘各二人。漢人,依府尹數例。譯人一人,通事二人,抄事一人。公使人,上鎮七十、中六十五、下六十人,惟蒲與、胡里改、速頻各二十人。曷速館路、蒲與路、胡里改路、速頻路四節鎮,省觀察判官而無州事。



 諸防禦州。防禦使一員,從四品。掌防捍不虞、禦制盜賊,餘同府尹。同知防禦使事一員,正六品。掌通判防禦使事。判官一員,正八品。掌簽判州事,專掌通檢推排
 簿籍。知法,從九品。州教授一員。司軍,從九品。軍轄兼巡捕使,從九品。司吏,女直一人,漢人管戶五萬以上二十人,以率而減。譯人一人,通事一人,抄事一人。公使,上州六十人、中五十五人、下五十人。



 諸刺史州。刺史一員,正五品。掌同府尹兼治州事。同知一員,正七品。通判州事。判官一員,從八品。簽判州事,專掌通檢推排簿籍。司軍,從九品。知法一員。軍轄兼巡捕使,從九品。司吏,女直,韓、慶、信、灤、薊、通、澄、復、沈、貴德、涿、利、建州、來遠軍各三人,餘各二人。抄事一人。公使,上州五十、中四十五、下四十。惟來遠軍同下州,省同知。凡諸州以上知印,並於孔目官內輪差,運司押司官並同。無孔目官,以上各司吏充,司、縣同此。



 諸京警巡院。使一員,正六品。掌平理獄訟、警察別部,總
 判院事。副一員,從七品。掌警巡之事。判官二員,正九品。掌檢稽失,簽判院事。司吏,女直,中都三人,上、東、西三京各二人,餘各一人。漢人,中都十五人,南京九人,西京八人,東京六人,北京五人,上京四人。惟東、西、北、上京無副使。



 諸府節鎮錄事司。錄事一員,正八品。判官一員,正九品。掌同警巡使。司吏,戶萬以上設六人,以下為率減之。凡府鎮二千戶以上則依此置,以下則止設錄事一員,不及百戶者並省。



 諸防刺州司候司。司候一員,正九品。司判一員,從九品。司吏、公使七人。然亦驗戶口置。



 赤縣。謂大興、宛平縣。令一員,從六品,掌養百姓、按察所部、宣導風化、勸課農桑、平理獄訟、捕除盜賊、禁止游惰,兼管
 常平倉及通檢推排簿籍,總判縣事。丞一員,正八品。掌貳縣事。主簿一員,正九品。掌同縣丞。尉四員,正八品。專巡捕盜賊。餘縣置四尉者同此。司吏十人,內一名取識女直、漢字者充。公使十人。



 次赤縣又曰劇縣。令一員,正七品。丞一員,正九品。主簿一員,正九品。尉一員,正九品。



 諸縣。令一員,從七品。丞一員,正九品。主簿一員,正九品。尉一員,正九品。凡縣二萬五千戶以上為次赤、為劇,二萬以上為次劇,在諸京倚郭者曰京縣。自京縣而下,以萬戶以上為上,三千戶以上為中,不滿三千為
 下。中縣而下不置丞,以主簿與尉通領巡捕事。下縣則不置尉,以主簿兼之。中縣司吏八人,下縣司吏六人,公使皆十人。



 諸知鎮、知城、知堡、知寨,皆從七品。其設公使皆與縣同,惟驗戶口置司吏。



 諸司獄。司獄一員,正九品。提控獄囚。司吏一人。公使二人。典獄二人,防守獄囚門禁啟閉之事。獄子,防守罪囚者。



 市令司。唯中都置。令一員,正八品。南遷以左、右警巡使兼。丞一員,正九品。掌平物價,察度量權衡之違式、百貨之估直。司吏四人、公使八人。



 軍器庫。使一員,正八品。副使一員,從九品。掌甲胄兵仗。
 司吏二人。庫子,掌出納之數、看守巡護。中都、南京依此置,西京省副使,北京惟副使,仍兼八作使。隨府節鎮設使、副,若軍器兼作院,軍資兼軍器庫,及防刺郡,則置都監一員,以軍資監兼者如舊。



 作院。使一員,副使一員,掌監造軍器,兼管徒囚,判院事。都監一員,掌收支之事。牢長,監管囚徒及差設牢子。中都、南京依此置,仍加「都」字。南京省都監一員,東京、西京置使或副一員,上京並省。隨府節鎮作院使副,並以軍器使副兼之。其或置一員,或以軍資庫兼之,若元設甲院都監處,並薊州專設使副者,並仍舊。



 都轉運司



 使,正三品。掌稅賦錢穀、倉庫出納、權衡度量之制。同知,從四品。副使,正五品。都勾判官,從六品。紀綱眾務、分判勾案,惟南京勾判兼上林署丞。戶籍判官二員,從六品。舊止一員,承安四年增置一員,不許
 別差,專管拘收征剋等事。支度判官二員,從六品。掌勾判、分判支度案事。鹽鐵判官一員,從六品。都孔目官二員,勾稽文牘。知法二員,從八品。都勾案、戶籍案、鹽鐵案、支度案、開拆案司吏,女直八人,漢人九十人。抄事一人,譯史三人,通事一人,押遞五十人,監運諸物公使八十人。惟中都路置都轉運司,餘置轉運司,省戶、度判官各一員。南京、西京、北京、遼東、山東西路,河北東路則置女直知法、漢知法各一員。山東東路、河東南路北路、河北西路、陜西東西路則置漢知法一員。餘官皆同中都置。女直司,司吏,遼東路十人,西京、北京、山東西路各五人,餘路皆四人。譯史,遼東路三人,餘各二人。通事各一人。漢人司,司吏,課額一百八十萬貫以上者五十人,百五十萬貫以上四十五人,百二十萬貫以上四十人,九十萬貫以上三十五人,六十萬貫以上三十人,三十萬貫以上二十五人,不及三十萬貫二十人。公使人,各七十人。押遞,南京、山東東西路、河東南路、河北西路各五十人,西京、河東北路、河北東
 路各四十人,餘路各三十人。



 山東鹽使司。與寶坻、滄、解、遼東、西京、北京凡七司。使一員,正五品。他司皆同。副使二員,正六品。它司皆一員。判官三員,正七品。泰和作四員,寶坻、解州設二員,餘司皆一員。掌乾鹽利以佐國用。管勾二十二員,正九品。寶坻、解、西京則設六員,北京、遼東、滄州則設四員。同管勾、都同監皆省。掌分管諸場發買收納恢辦之事。同管勾五員。都監八員。監、同各七員。知法一員。司吏二十二人,女直三人,漢人十九人。譯人一人,抄事、公使四十人,它司皆同。



 中都都曲使司。酒使司、院務、稅醋使司,榷場兼酒使司附。使,從六品。副使,正七品。掌監知人戶醖造曲蘗,辦課
 以佐國用。餘酒使監醖辦課同此。都監二員,正八品。掌簽署文簿、檢視醖造。司吏四人,公使十人。凡京都及真定皆為都曲酒使司,設官吏同此。它處置酒使司,課及十萬貫以上者設使、副、小都監各一員,五萬貫以上者設使、副各一員,以上皆設司吏三人。三萬貫以上者設使及都監各一員,司吏二人。不及二萬貫者為院務,設都監、同監各一員。不及千貫之院務止設都監一員。其它稅醋使司、及榷場與酒稅相兼者,視課多寡設官吏,皆同此。諸酒稅使三萬貫以上者正八品,諸酒榷場使從七品,五萬貫以上副使正八品。



 提舉南京路榷貨事,從六品。



 中都都商稅務司。使一員,正八品。副使一員,正九品。正大元年升為從七品。掌從實辦課以佐國用。都監一員,從九品。掌簽署文簿、巡察匿稅。司吏四人,公使十
 人,餘置官吏同酒使司。



 中都廣備庫。使一員,從七品。副使一員,從八品。判官一員,正九品。掌匹帛顏色,油漆諸物出納之事。攢典四人。庫子十四人,內十二人收支,二人應辦。掌排數出納、看守巡護之事,與庫官通管。



 永豐庫。鍍鐵院都監隸焉。使一員,從七品。副使一員,從八品。判官一員,正九品。掌泉貨金銀珠玉出納之事。攢典三人。庫子十二人,內十人收支,二人應辦。凡歲收二十五萬貫者置庫子十人,不及二萬貫者置二人。鍍鐵院都監二員,管勾生熟鐵釘線。攢典一人。京、府、鎮、通州並依此置,判官、都監皆省。或兼軍器並作院,或設使若副一員。防刺郡設都監一員,仍兼軍器庫。



 南京交鈔庫。使一員,正八品。副使一員,正九品。掌出入
 錢鈔兌便之事。攢典二人,攢寫計帳、類會合同。庫子八人,掌受納錢數、辨驗交鈔、毀舊主簿曆。



 中都流泉務。大定十三年,上謂宰臣曰:「聞民間質典,利息重者至五七分,或以利為本,小民苦之。若官為設庫務,十中取一為息,以助官吏廩給之費,似可便民。卿等其議以聞。」有司奏於中都、南京、東平、真定等處並置質典庫,以流泉為名,各設使、副一員。凡典質物,使、副親評價直,許典七分,月利一分,不及一月者以日計之。經二周年外,又逾月不贖,即聽下架出賣。出帖子作寫質物人姓名,物之名色,金銀等第分兩,及
 所典年月日錢貫,下架年月之類。若亡失者,收贖日勒合干人,驗元典官本,並合該利息,陪償入官外,更勒庫子,驗典物日上等時估償之,物雖故舊,依新價償。仍委運司佐貳幕官識漢字者一員提控,若有違犯則究治。每月具數申報上司。大定二十八年十月,京府節度州添設流泉務,凡二十八所。明昌元年,皆罷之。二年,在都依舊存設。使一員,正八品。副使一員,正九品。掌解典諸物、流通泉貨。勾當官一員。攢典二人。



 中都店宅務。管勾四員,正九品。各以二員分左右廂,掌官房地基,徵收官錢、檢料修造摧毀房舍。攢典,左右廂各
 五人,掌徵收及檢料修造房屋之事。庫子,左右廂各三人。催錢人,左右廂各十五人,又別設左廂平樂樓花園子一名,右廂館子四人。



 南京店宅務同。



 中都左右廂別貯院。使一員,從八品。副使一員,正九品。判官,從九品。掌拘收退朴等物及出給之事。攢典、庫子,同前。



 中都木場。使一員,從八品。副使一員,判官一員,皆正九品。掌拘收材木諸物及出給之事。司吏一人,庫子四人,花料一人,木匠一人。



 中都買物司。使一員,從八品。副使一員,正九品。掌收買官中所用諸物。都監四員,從九品。掌支應等事。司吏二人。



 京兆府司竹監。管勾一員,從七品。掌耨養竹園採斫之
 事。司吏一人,監兵百人,給耨養採斫之役。



 諸綾綿院。置於真定、平陽、太原、河間、懷州。使一員,正八品。副使一員,正九品。掌織造常課匹段之事。



 規措京兆府、耀州、三白渠公事。規措官,正七品。掌灌溉民田。點檢渠堰官一員,掌點檢啟閉涇陽等縣渠堰。司吏三人。



 漕運司。提舉一員,正五品。景州刺史兼領。掌河倉漕運之事。同提舉一員,正六品。勾當官,從八品。掌催督起運綱船。司吏六人,分掌課使、起運兩科,各設孔目官,前後行各一人。儤使科,掌吏、戶、禮案。起運科,掌兵、刑、工案。公使八十一人,押綱官七十六人。景州依此置。肇州以提舉兼
 本州同知,同提舉兼州判。



 諸倉。使,正八品。副使,正九品。掌倉廩畜積、受納租稅、支給祿廩之事。攢典,掌收支文曆、行署案牘。歲收一萬石以上設二人。倉子,掌斛斗盤量、出納看守之事。



 草場。使,副使,掌儲積受給之事。攢典二人。場子,掌積垛、出納、看守、巡護之事,歲收五萬以上設四人。中都、南京、歸德、河南、京兆、鳳翔依此置。西京省副使,餘京節鎮科設使副一員,防刺仍舊,置都監一員。



 南京諸倉監支納官、草場監支納官,正八品。



 南京提控規運柴炭場。使,從五品。副使,正六品。



 京西規運柴炭場。使,從八品。副使,正九品。



 諸總管府節鎮兵馬司



 都指揮使一員,正五品。巡捕盜賊,提控禁夜,糾察諸博徒、屠宰牛馬,總判司事。副都指揮使二員,正六品。貳使職,通判司事,分管內外,巡捕盜賊。軍典十二人,掌本庫名籍、差遣文簿、行署文書、巡捕等事,餘軍典同此。司吏一人,譯人一人,公使十人。指揮使一員,從六品。鈐轄四都之兵以屬都指揮使,專署本指揮使事。軍使一員,正七品。指揮之職,左右什將各一人,共管一都。軍典二人,營典一人,左、右承局各一人,左、右押官各一人。以上軍員每百人為一指揮使,各一員分四都,每都設左右什將、承局、押官各一。若人數不及,附近相合者,並依上置。如無可相合者,三百人以上為一指揮,二百人以上止設指揮使,一百人止設軍使,仍每百人以上立為一都,不及百人設什將、承局、押官各一。其指揮下軍使,什將下軍典、營典、各
 同此置。惟北京、西京止設使、副各一員。



 諸府鎮都軍司。都指揮使一員,正七品。節鎮軍都指揮使則從七品。掌軍率差役、巡捕盜賊,總判軍事,仍與錄事同管城隍。軍典二人,公使六人,凡諸府及節鎮並依此置。



 諸防刺州。軍轄一員,掌同都軍,兼巡捕,仍與司候同管城壁。軍典二人。



 諸府州。兵馬鈐轄一員,從六品。掌巡捕盜賊。若有盜,則總押隨處巡尉,併力擒捕。司吏二人。京兆、咸平、濟南、鳳翔、萊、密、懿、鞏州並依此置。惟京兆、咸平府置兵馬都鈐轄,餘並省。



 諸巡檢。中都東北都巡檢使一員,正七品。通州置司,分
 管大興、漷陰、昌平、通、順、薊、盈州界盜賊事。司吏一人,掌行署文書。馬軍十五人,於武衛馬軍內選少壯熟閑弓馬人充。



 西南都巡檢一員,正七品。良鄉縣置司,分管良鄉、宛平、安次、永清縣並涿、易州界盜賊事。



 諸州都巡檢使各一員,正七品。副都巡檢使各一員,正八品。司吏各一人。右宿、泗、唐、鄧、察、亳、陳、穎、德、華、河、隴、泰等州并西北路依此置,餘不加「使」字。



 散巡檢,正九品。內泗州以管勾排岸兼之。皆設副巡檢一員,為之佐。右地險要處置司。唐、鄧、宿、泗、穎、壽、蔡等州及緣邊二十五處置。大定二十二年,廣寧府大斧山置巡檢司。明昌五年七月,升蔡州劉輝村置巡檢。



 潼關。關使兼譏察官,正七品。掌關禁、譏察姦偽及管鑰
 啟閉。副譏察,正九品。掌任使之事。司吏二人,女直、漢人各一。



 居庸關、紫荊關、通會關、會安關及他關。皆設使,從七品。



 大慶關。



 管勾河橋官兼譏察事一員,正八品。掌解繫浮橋、濟渡舟楫、巡視河道、修完埽岸、兼率埽兵四時功役、栽植榆柳、預備物料、譏察姦偽等事。同管勾一員。司吏二人,女直、漢人各一人。九鼎、大陽津渡,惟置譏察官一員。



 孟津渡。譏察一員,正八品。掌譏察姦偽。副譏察一員,正九品。司吏二人。



 提舉譏察使,正五品。副使,從五品。陜西一員,河南二員。南遷置譏察使,從七品。副使,正八品。南遷後,陜西置於秦州,河
 南置於唐、鄧、息、壽、泗五州。



 提舉秦、藍兩關,提舉,從五品。同提舉,正六品。南遷後置。



 提舉三門,集津南北岸,正六品。南遷後置。



 沿淮譏察使,從五品。



 管勾泗州兼排岸巡檢,正九品。



 諸邊將。正將一員,正七品。掌提控部保將、輪番巡守邊境。副將一員,正八品。部將一員,正九品。輪番巡守邊境。隊將,正九品。鄜延九將,慶陽十將,臨洮十四將,鳳翔十六將,河東三將,並依此置。



 統軍司河南、山西、陜西,益都。使一員,正三品。督領軍馬、鎮攝封陲、分營衛、視察姦。副統軍一員,正四品。判官一員,從五
 品。紀綱庶務,簽判司事。大定九年置。知事一員,從七品。知法二員,從八品。女直、漢人各一。書史十三人,女直八人。漢人五人,掌行署文牘、上名監印。守當官四人,譯書四人,通事一人,抄事一人,公使五十人。河南依此置,山東不設判官,知法以益都府知法兼之。



 招討司。三處置,西北路、西南路、東北路。使一員,正三品。副招討使二員,從四品。招懷降附、征討攜離。判官一員,從六品,紀綱職務、簽判司事。勘事官一員,從七品。知事一員,正八品。知法二員,從八品。女直、漢人各一。司吏十九人。譯人三人。通事六人,內諸部三人、河西一人。移剌三十人,以上各充都管。抄事一人。公使五十人。西北路增勘事官一員。東北路不置漢人知法。



 諸猛安。謀克隸焉。



 猛安,從四品。掌修理軍務、訓練武藝、勸課農桑,餘同防禦。司吏四人,譯一人,撻馬、差役人數並同舊例。諸謀克,從五品。掌撫輯軍戶、訓練武藝。惟不管常平倉,餘同縣令。女直司吏一人,譯人一人,撻馬。



 諸部族節度使。節度使一員,從三品。統制各部,鎮撫諸軍,餘同州節度。副使一員,從五品。判官一員。知法一員。司吏四人,女直、漢人各半。通事一人,譯人一人,撻馬。右部羅火部族、土魯渾部族並依此置。



 諸颭詳穩一員,從五品。掌守戍邊堡,餘同謀克。皇統八年六月,設本班左右詳穩,定為從五品。麼忽一員,從八品。掌貳詳穩。司史三人,習尼昆,掌本颭差役等事。撻馬,隨從也。咩颭、唐古颭、移剌颭、木
 典颭、骨典颭、失魯颭並依此置。惟失魯颭添設譯人一名。《士民須知》某年有慈謨典颭、胡都颭、霞馬颭,無失魯颭、移典颭。



 諸移里堇司。移裏堇一員,從八品。分掌部族村寨事。司吏,女直一人、漢人一人。習尼昆,掌本颭差役等事,撻馬。右土魯渾部族南北移裏堇司依此置。部羅火部族左右移裏堇司置女直司吏一人。



 諸禿里。禿里一員,從七品。掌部落詞訟、防察違背等事。女直司吏一人,通事一人。



 諸群牧所,又國言謂烏魯古。提控諸烏魯古一員,正四品。明昌四年置。是年以安遠大將軍尚廄局使石抹貞兼慶州刺史為之,設女直司吏三人,譯一人,通事一人。使一員,從四品。國言作烏魯古使。副使一員,從六
 品。掌檢校群牧畜養蕃息之事。判官一員,正八品。掌簽判本所事。知法一員,從八品。女直司吏四人,譯人一人,撻馬十六人,使八人,副五人,判三人。又設埽穩脫朵,分掌諸畜,所謂牛馬群子也。惟板底因、烏解、忒恩、蒲鮮群牧依此置。



\end{pinyinscope}