\article{志第三十七}

\begin{pinyinscope}

 百官二



 ○殿前都點檢司宣徽院祕書監國子監太府監少府監軍器監都水監諫院大理寺弘文院登聞鼓院登聞檢院記注院集賢院益政院武衛軍都指揮使司衛尉司戶禮兵刑工部所轄諸司三路檢察及外路倉庫牧圉等職



 殿前都點檢司。天眷元年置。掌親軍,總領左右衛將軍、符寶郎、宿直將軍、左右振肅,宮籍監、近侍等諸局署、鷹坊、頓舍官隸焉。



 殿前都點檢,正三品。兼侍衛將軍都指
 揮使。掌行從宿衛,關防門禁,督攝隊仗,總判司事。殿前左副都點檢,從三品。兼侍衛將軍副都指揮使。殿前右副都點檢,從三品。兼侍衛將軍副都指揮使。掌宮掖及行從。殿前都點檢判官,從六品。大定十二年設。知事一員,從七品。



 殿前左衛將軍,殿前右衛將軍,殿前左衛副將軍,殿前右衛副將軍。掌宮禁及行從宿衛警嚴,仍總領護衛。右衛同此。



 符寶郎四員,掌御寶及金銀等牌。舊名牌印祗候,大定二年改為符寶祗候,改牌印令史為符寶典書,四人。



 左右宿直將軍,從五品。掌總領親軍。凡宮城諸門衛禁。並行從宿衛之事,八員。大定二十九年作十員,復作十一員。



 左右振肅,正七品。掌妃嬪出入總領護衛導從。本妃嬪護衛之長,大定二年改今名。



 宮籍監。提點,正五品。監,從五品。副監,從六品。丞,從七品。掌內外監戶及地土錢帛小大差發。直長二員,正八品。掌同丞。



 近侍局。提點,正五品。泰和八年創設。使,從五品。副使,從六品。掌侍從、承敕令、轉進奏帖。直長,正八品。大定十八年增二員。奉御十六人,舊名入寢殿小底。奉職三十人,舊名不入寢殿小底,又名外帳小底,皆大定十二
 年更。



 器物局。提點,正五品。使,從五品。副使,從六品。掌進御器械鞍轡諸物。直長,正八品。都監,正九品。明昌三年省罷。同監,從九品。泰和四年設。



 尚廄局。提點,正五品。使,從五品。副使,從六品。掌御馬調習牧養,以奉其事。大定二十九年添副使一員,管小馬群。直長一員,司馬牛群。掌廄都轄,正九品。不限員。副轄,從九品。不限員數資考。



 尚輦局。使,從五品。副使,從六品。掌承奉輿輦等事。直長,正八品。不限資考,大定十九年,除年六十以下人充。典輿都轄,從九品。
 不限資考。收支都監,正九品。大定二十年設,掌給受之事。同監,泰和四年設。大安二年省。本把,四人。



 鷹坊。提點,正五品。使,從五品。副使,從六品。掌調養鷹鶻海東青之類。直長,正八品。不限員。管勾,從九品。不限員數資考。



 武庫署。令,從六品。掌收貯諸路常課甲仗。以曉軍器女直人充。丞,從七品。直長二員,正八品。大定二年省一員。



 武器署。提點,從五品。令,從六品。丞,從七品。掌祭祀、朝會、巡幸及公卿婚葬鹵簿儀仗旗鼓笛角之事。直長,正八品。或二員。頓舍官二員《泰和令總格》作四員,正八品。直長。
 見《士民須知》。《泰和令》無。



 右屬殿前都點檢司。



 宣徽院



 左宣徽使,正三品。右宣徽使,正三品。同知宣徽院事,正四品。同簽宣徽院事,正五品。宣徽判官,從六品。掌朝會、燕享,凡殿庭禮儀及監知御膳。所隸弩手、傘子二百三十九人,控鶴二百人。



 拱衛直使司,威捷軍隸焉。舊名龍翔軍,正隆二年更為神衛軍,大定二年更名為拱衛司。都指揮使,從四品。舊曰使。副使指揮使,從五品。舊曰副使。掌總統本直,謹嚴儀衛。大定五年,詔以使為都指揮使,副使為副都指
 揮使。什將。長行。威捷軍。承安二年,簽弩手千人。泰和四年,以之備邊事。鈐轄,正六品。都轄,從九品。不奏。



 客省。使,正五品。副使,從六品。掌接伴人使見辭之事。



 引進司。使,正五品。副使,從六品。掌進外方人使貢獻禮物事。



 閣門。明昌五年,閣門官以次排轉除授。東上閣門使二員,正五品。明昌六年省一員,作從五品。西同。副使二員,正六品。明昌六年,省一員。西同。簽事一員,從六品。掌簽判閣門事。西同。明昌六年,以減副使置。西上閣門使二員,正五品。副使二員,正六品。簽事一員,從六品。掌贊導殿庭禮儀。西閣門餘副貳同。閣門祗候二十五人。
 正大間三十二。閣門通事舍人二員,從七品。掌通班贊唱、承奏勞問之事。承奉班都知,正七品。掌總率本班承奉之事。舊置判官,後罷。內承奉班押班,正七品。掌總率本班承奉之事。御院通進四員,從七品。掌諸進獻禮物及薦享編次位序。



 尚衣局。提點,正五品。使,從五品。副使,從六品。掌御用衣服、冠帶等事。都監,正九品。舊設,後罷。直長,正八品。同監,從九品。



 儀鸞局。泰和四年,或以少府監官兼,或兼少府監官。提點,正五品。使,從五品。副使,從六品。掌殿庭鋪設、帳幕、香燭等事。直長四
 員,正八品。《泰和令》三員。收支都監,正九品,二員,一員掌給受鋪陳諸物,一員掌萬寧宮收支庫。大定七年置,明昌二年增一員。同監二員,從九品。司使二人,如內藏庫知書例。



 尚食局。元光二年,參用近侍、奉御、奉職。提點,正五品。使,從五品。副使,從六品。掌總知御膳、進食先嘗、兼管從官食。直長一員,正八品。不限資考。都監三員,正九品。不限資考。生料庫都監、同監各一員,掌給受生料物色。收支庫都監、同監各一員,掌給受金銀裹諸色器皿。以外路差除人內選充。



 尚藥局。提點,正五品。使,從五品。出職官內選除。副使,從六品。掌進湯藥茶果。直長,正八品。都監,正九品。果子都監、
 同監各一員,掌給受進御果子。本局本把四人。



 太醫院。提點,正五品。使,從五品。副使,從六品。判官,從八品,掌諸醫藥,總判院事。管勾,從九品。隨科至十人設一員,以術精者充。如不至十人併至十人置。不限資考。正奉上太醫一百二十月升除,副奉上太醫不算月日,長行太醫不算月日,十科額五十人。



 御藥院。提點,從五品。直長,正八品。掌進御湯藥。明昌五年設,以親信內侍人充。都監,正九品。不限員,《泰和令》四員。同監,從九品。不常除,《泰和令》無。



 教坊。提點,正五品。使,從五品。副使,從六品。判官,從八
 品。掌殿庭音樂,總判院事。諧音郎,從九品。不限資考、員數。



 內藏庫。大定二年,分為四庫。使,從五品。副使,從六品。掌內府珍寶財物,率隨庫都監等供奉其事。直長一員。承安三年增。



 頭面庫。都監,正九品。同監,從九品。本把七人,大定二年定出身,依不入寢殿小底例。



 段匹庫。都監,正九品。同監,從九品。本把十二人。金銀庫都監,正九品。本把八人。



 雜物庫。都監,正九品。同監,從九品。本把八人。每庫知書各二人。



 宮闈局。舊名宮闈司,大定二年改為局,舊設令,丞,改為使、副。提點,正五品。使,從五品。副使,從六品。掌宮中合門之禁,率隨位都監、同監及內直各給其事。直長,正八品。內直一百七十人。
 後作百七十九人。



 內侍局。令二員,從八品。興定五年,陞作從六品。丞二員,從九品。興定五年,升從七品。掌正位合門之禁,率殿位都監、同監及御直各給其事。局長二員,從九品,興定五年升正八品。御直、內直共六十四人。明昌元年,分宮闈局正位內直置,初隸宮闈局。東門都監、同監。諸隨殿位承應都監,同監,掌各位承應及門禁管鑰。昭明殿都監、同監。大定二十九年設,各一員。承徽殿都監、同監。麗妃位。隆徽殿都監、同監。本隆和殿,系皇后位。鸞翔殿都監、同監。崇儀殿都監、同監。迎暉殿都監、同監。七妃
 充容,泰和三年罷。蕊珠殿都監、同監。瑞寧殿都監、同監。回春殿都監、同監。芸香殿都監、同監。瑞像殿都監、同監。系佛殿。以上「殿」字下無「位」字。凝福改韶景、溫芳二位都監、同監。瑤華、柔則二位都監、同監。以上無「殿」字及「承應」字。嘉福等殿位都監、同監。四位。廣仁殿都監、同監。睿思殿都監、同監。以上有「承應」字。滋福殿都監、同監。本以隆慶改,無「位」字。咨正殿都監、同監。邇英殿都監、同監。長慶院都監、同監。仙韶院都監、同監。貞和門都監、同監。應系錢帛經此門出入。明昌四年添一員。右升平門都監、同監。長樂門都監、同監。瓊林苑都監、同監。各二員。廣樂園都
 監、同監。順儀位提控、都監、同監。舊寶林位。瑞華門俗名金骨朵門都監一員,同監三員。太師位提控、都監、同監。寶昌門都監、同監。會昌門都監、同監。東京孝寧宮都監、同監。崇妃位提控。世宗夫人,興陵。惠妃位提控、都監、同監。裕陵。溫妃位提控、都監、同監。裕陵二位,明昌四年添。報德寺提控、都監、同監。世宗御容。光泰門街。報恩寺提控、都監、同監。世宗御容。清夷門街。明昌三年設,三。孝嚴寺都監、同監。在南京,安宣宗御容,改興國感誠寺。正大元年設,三。以下皆在南京。福寧殿都監、同監。三。純和殿都監、同監。三。仁安殿都監、同監。三。真妃位都監、同監。三。麗妃位都監、同監。宣儀
 位都監、同監。莊獻妃位都監、同監。三廟都監、同監。貞祐二年設。西華門都監、同監。京後園都監、同監。



 內侍寄祿官。泰和二年設,初隸宮闈局,尋直隸宣徽院。所以升用內侍局御直、內直有年勞者。中常侍。正五品。給事中。從五品。內殿通直。正六品。先名內殿給使。黃門郎。從六品。內謁者。正七品。內侍殿頭。從七品。內侍高品。正八品。不限員。內侍高班。從八品。



 典衛司。大定二十九年,世宗才人、寶林位各設。泰和五年閏八月,以崇妃薨罷。興定元年復設。世宗妃、才人、寶林位各設防衛軍導從人。令,正七品。丞,從七品。直長。見《士民湏知》。



 孝靖宮。章宗五妃位。大安元年以有監同、無總領者,故設。令,從八品。丞,正九
 品。端妃位同監。真妃徒單氏。慧妃位同監。麗妃徒單氏。貞妃位同監。柔妃唐括氏。靚儀位同監。昭儀夾谷氏。才媛位同監。修儀吾古論氏。



 懿安家。貞祐三年,為莊獻太子設。令,從八品。丞,正九品。



 宮苑司。令,從六品。丞,從七品。掌宮庭修飭灑掃、啟閉門戶、鋪設氈席之事。直長,正八品一員。《泰和令》二員。都監、同監二員。泰和元年設。泰和四年罷同監。



 尚醖署。令,從六品。丞,從七品。掌進御酒醴。直長,正八品。二員。



 典客署。令,從六品。丞,從七品。直長,後罷。書表十八人。



 侍儀司。舊名擎執局,大定元年改為侍儀局,大定五年陞局為司。令,從六品。舊日局使。掌侍奉朝儀,率捧案、擎執、奉輦各給其事。直長,正七品。舊設局副,品從七。



 右屬宣徽院。



 秘書監。著作局、筆硯局、書畫司、司天臺隸焉。



 監一員,從三品。少監一員,正五品。丞一員,正六品。祕書郎二員,正七品。泰和元年定為二員。通掌經籍圖書。校書郎一員,從七品。承安五年二員。泰和五年以翰林院官兼,大安二年省一員。專掌校勘在監文籍。



 著作局。著作郎一員,從六品。著作佐郎一員,正七品。掌修日歷。皇統六年,著作局設著作郎、佐郎各二員,編修日歷,以學士院兼領之。



 筆硯局。直長二員,正八品,掌御用筆墨硯等事。泰和七年以女直應奉兼。舊名筆硯令史,大定三年改為筆硯供奉,以避諱改為承奉。



 書畫局。直長一員,正八品。掌御用書畫紙札。都監,正九品。二員或一員。



 司天臺。提點,正五品。監,從五品。掌天文歷數、風雲氣色,密以奏聞。少監,從六品。判官,從八品。教授,舊設二員,正大初省一員。係籍學生七十六人,漢人五十人,女直二十六人,試補長行。司天管勾,從九品。不限資考、員數,隨科十人設一員,以藝業尤精者充。長行人五十人。未授職事者,試補管勾。天文科,女直、漢人各六人。算歷科,八人。三式科,四人。測驗科,八人。漏刻科,二十五人。銅
 儀法物舊在法物庫,貞元二年始付本臺。



 右屬祕書監。



 國子監。國子學、太學隸焉。



 祭酒,正四品。司業,正五品。掌學校。丞二員,從六品。明昌二年增一員,兼提控女直學。



 國子學。博士二員,正七品。分掌教授生員、考藝業。太學同。明昌二年添女直一員,泰和四年減,大安二年並罷。助教二員,正八品。女直、漢人各一員。教授四員,正八品。分掌教誨諸生。明昌二年,小學各添二員,承安五年一員不除。國子校勘,從八品。掌校勘文字。國子書寫官,從八品。掌書寫實錄。



 太學。博士四員,正七品。大安二年減二員。助教四員,正八品。明昌二年不除一員,大安二年減一員。



 右屬國子監。



 太府監。左右藏、支應所、太倉、酒坊、典給署、市買司隸焉。



 監,正四品。少監,從五品。丞二員,從六品。掌出納邦國財用錢穀之事。



 左藏庫。使,從六品。副使,從七品。興定三年增一員。掌金銀珠玉、寶貨錢幣。本把四人。



 右藏庫。使,從六品。副使,從七品。興定三年添一員。掌錦帛絲綿毛褐、諸道常課諸色雜物。本把四人。



 支應所。又作支承所。都監二員,正九品。掌宮中出入、御前
 支賜金銀幣帛。大安三年省。



 太倉。使,從六品。掌九穀廩藏、出納之事。預除人。副使,從七品。



 酒坊。部除。使,從八品。副使,正九品。掌醖造御酒及支用諸色酒醴。



 典給署。本鉤盾署,明昌三年更。令,從六品。舊曰鉤盾使。丞,從七品。舊曰鉤盾副使。掌宮中所用薪炭冰燭、並管官戶。直長一員,正八品。



 市買司。天德二年更為市買局。使,從八品。副使,正九品。掌收買宮中所用果實生料諸物。



 右屬太府監。



 少府監。尚方、織染、文思、裁造、文繡等署隸焉。泰和四年,選能幹官兼儀鸞局近上官。



 監,正四品。少監,從五品。丞二員,從六品。大定十一年省,二十一年復置。掌邦國百工營造之事。



 尚方署。令,從六品。丞,從七品。掌造金銀器物、亭帳、車輿、床榻、簾席、鞍轡、傘扇及裝釘之事。大定二十年,令不專除人,令人兼。直長,正八品。



 圖畫署。明昌七年,省入祗應司。令,從六品。丞,從七品。掌圖畫縷金匠。直長,正八品。明昌三年罷。



 裁造署。令,從六品。丞,從七品。掌造龍鳳車具、亭帳、鋪
 陳諸物,宮中隨位床榻、屏風、簾額、絳結等,及陵廟諸物并省臺部內所用物。《泰和令》有畫繪之事。直長,從八品。明昌三年省,裁造匠六人,針工婦人三十七人。



 文繡署。令,從六品。丞,從七品。掌繡造御用並妃嬪等服飾、及燭籠照道花卉。貞祐二年,止設官一員。直長,正八品。繡工一人,都繡頭一人,副繡頭四人,女四百九十六人,內上等七十人,次等凡四百二十六人。



 織染署。令,從六品。丞,從七品。直長,正八品。掌織紝、色染諸供御及宮中錦綺幣帛紗縠。



 文思署。明昌七年,省入祗應司。令,從六品。丞,從七品。掌造內外局分印合,傘浮圖金銀等尚輦儀鸞局車具亭帳之
 物並三國生日等禮物,織染文繡兩署金線。直長,正八品。明昌三年省去。



 右屬少府監。



 軍器監。承安二年設,泰和四年罷,復並甲坊、利器兩署為軍器署,置令、丞、直長,直隸兵部。至寧元年復為軍器監,軍器庫、利器署隸焉。舊轄甲坊、利器兩署。



 監,從五品。少監,從六品。丞,從七品。掌修治邦國戎器之事。直長,正八品。《泰和令》無,《總格》有。



 軍器庫。至寧元年隸大興府,貞祐三年來屬。使,正八品。副使,正九品。省擬,不奏。掌收支河南一路並在京所造
 常課橫添和買軍器。大定五年設。



 甲坊署。泰和四年廢,舊置令、丞、直長。



 利器署。本都作院,興定二年更今名。同隨朝來屬。令,從六品。丞,從七品。掌修弓弩刀槊之屬。直長,正八品。



 右屬軍器監。



 都水監:街道司隸焉。分治監,專規措黃、沁河、衛州置司。



 監,正四品。掌川澤、津梁、舟楫、河渠之事。興定五年兼管勾沿河漕運事,作從五品,少監正六品以下皆同兼漕事。少監,從五品。明昌二年增一員,衛州分治。丞二員,正七品。內一員外監分治。貞元元年置。掾,正八品。掌與丞同,
 外監分治。大定二十七年添一員,明昌三年並罷之,六年復置二員。勾當官四員,準備分治監差委。明昌五年以罷掾設二員,興定五年設四員。



 街道司。管勾,正九品。掌灑掃街道、修治溝渠。舊南京街道司,隸都水外監,貞元二年罷歸京城所。



 都巡河官,從七品。掌巡視河道、修完堤堰、栽植榆柳、凡河防之事。分治監巡河官同此。其瀘溝、崇福上下埽都巡河兼石橋使,通濟河節巡官兼建春官地分河道。諸都巡河官,掌提控諸埽巡河官、明昌五年設,以合得縣令人年六十者選充。大定二年設滹沱河巡河官二員。散巡河官。於諸局及丞簿廉舉人,並見勾當人六十以下者充。



 黃汴都巡河官,下六處河陰、雄武、滎澤、原武、陽武、延津,各設散巡河官一員。



 黃沁都巡河官,下四處懷州、孟津、孟州、城北,各設黃沁散巡河官各一員。



 衛南都巡河官,下四處崇福上、崇福下、衛南、淇上,散巡河官各一員。



 滑濬都巡河官,下四處武城、白馬、書城、教城,散巡河官各一員。



 曹甸都巡河官,下四處東明、西佳、孟華、凌城,散巡河官各一員。



 曹濟都巡河官,下四處定陶、濟北、寒山、金山,散巡河官各一員。凡二十五埽,埽兵萬二千人。



 諸埽物料場官,掌受給本場物料。分治監物料場官同此。惟崇福上、下埽物料場官與當界官通管收支。



 南京延津渡河橋官,兼譏察事。管勾一員,同管勾一員,掌橋船渡口譏察濟渡、給受本橋諸物等事,內譏察事隸留守司。餘浮橋官同此。



 右屬都水監。皇統三年四月,懷州置黃沁河堤大管勾司,未詳何年罷。正大二年,外監東置於歸德,西置于河陰。



 諫院



 左諫議大夫、右諫議大夫,皆正四品。左司諫、右司諫,皆從四品。左補闕、右補闕,正七品。左拾遺、右拾遺,正七品。



 大理寺。天德二年置。自少卿至評事,漢人通設六員,女直、契丹各四員。



 卿,正四品。少卿,從五品。正,正六品。丞,從六品。掌審斷天下奏案、詳讞疑獄。司直四員,正七品。掌參議疑獄、披詳法狀。舊有契丹司直一員,明昌二年罷。評事三員,正八品。掌同司直。明昌二年省契丹評事二員,大安二年省漢人一員。知法十一員,從八品。女直司五員,漢人司六員。掌檢斷刑名事。明法二員,從八品。興定二年置,同流外,四年罷之。



 弘文院



 知院,從五品。同知弘文院事,從六品。校理,正八品。掌校譯經史。



 登聞鼓院



 知登聞鼓院,從五品。同知登聞鼓院事,正六品。掌奏進告御史臺、登聞檢院理斷不當事,承安二年以諫官兼。知法二員,從八品。女直、漢人各一員。



 登聞檢院



 知登聞檢院,從五品。同知登聞檢院,正六品。掌奏御進告尚書省、御史臺理斷不當事。知法,從八品。女直、漢人各一員。



 記注院。修起居注,掌記言、動。明昌元年,詔毋令諫官兼或以左右衛將軍兼。貞祐三年,以左右司首領官兼,
 為定制。



 集賢院。貞祐五年設。



 知集賢院,從四品。正大元年,授馬璘額外兼吏部郎中。同知集賢院,從五品。司議官,正八品。不限員。咨議官,正九品。不限員。



 益政院。正大三年置於內庭,以學問該博、議論宏遠者數人兼之。日以二人上直,備顧問,講《尚書》、《通鑑》、《貞觀政要》。名則經筵,實內相也。末帝出,遂罷。



 武衛軍都指揮使司隸尚書兵部。



 都指揮使,從三品。大定二十九年,以武衛軍六十人,兵馬一員、副都二員其職低,故設使,品正四,承安三年升。副都指揮使二員,從四品。副都一員,從四品。初正五品,承安三年升。判官一
 員。承安三年設。掌防衛都城、警捕盜賊。



 鈐轄司。鈐轄十員,正六品。初設二員。都鈐轄四員,從七品。興定三年權設,巡把兩宅。都將二十員,從九品。大定十六年立名。掌管轄軍人、防衛警捕之事。承安元年設萬人,內軍八千九百四十九人,忠衛二百人,隊正四百人。



 右屬武衛軍都指揮使司。



 衛尉司大安元年,擬隆慶宮人數定之。



 中衛尉,從三品,掌總中宮事務。副尉,從四品。左常侍,從五品。掌周護導從儀仗之事。右常侍,從五品。常侍官。護衛三十人同東宮,奉引八十人同控鶴,傘子四人同控鶴,執旗二人同儀
 鸞。



 給事局。使,正七品。副使,正八品。內謁者兼司寶二員,從六品。內直充。奉閣一十人。同東宮入殿小底。閣直二十人。同宮闈局內直。



 掖庭局。令,正九品。內直充。,掌皇后宮事務。丞,從九品。內直充。宮令。宮苑司、儀鸞局兼。食官。尚食局兼。飲官。尚醖署兼。醫官。尚藥局、太醫院兼。主藏。內藏、典給署兼。主廩。太倉兼。



 右屬衛尉司。



 榷貨務。在京諸稅係中運司,見錢皆權於本務收。使,從六品。副使,從七品。掌發賣給隨路香茶鹽鈔引。



 交鈔庫。使,舊正八品,後升從七品,貞祐復。掌諸路交鈔及檢勘錢鈔、換
 易收支之事。副使,從八品。掌書押印合同。判官,正九品。貞祐二年作從九品。都監,二員。見《泰和令》。



 印造鈔引庫。大安二年兼抄紙坊。使,從八品。副,正九品。判,正九品。掌監視印造勘覆諸路交鈔、鹽引,兼提控抄造鈔引紙。承安四年,罷四小庫,並罷庫判四員。至寧元年設二員,貞祐二年作從九品。



 抄紙坊。大安二年以印造鈔引庫兼。貞祐二年復置,仍設小都監二員。使,從八品。貞祐二年同隨朝。副使,正九品。判,從九品。



 交鈔庫物料場。至寧元年置。場官。舊正八品,後作正九品。掌收支交鈔物料。



 隨處交鈔庫抄紙坊。使,從八品。貞祐二年,設於上京、西京、北京、東平、大名、益都、咸平、真定、河間、平陽、太原、京兆、平涼、廣寧等府,瑞、蔚、平、清、通、順、薊等州,貞祐三年罷之。



 平準務。元光二年五月設,十月罷。使,從六品。副使,從七品。勾當官六員。



 右自榷貨務以下,皆屬尚書戶部。



 惠民司。令,從六品。掌修合發賣湯藥。舊又設丞一員。大定三年,有司言:「惠民歲入息錢不償官吏俸。」上曰:「設此本欲濟民,官非人,怠於監視藥物,財費何足計哉!可減員而已。」直長,正八品。都監,正九品。



 右屬尚書禮部。



 四方館。使,正五品。副使,從六品。掌提控諸路驛舍驛馬并陳設器皿等事。



 法物庫。元兼管大樂,貞元二年改付太常寺。使,從六品。副使,從七品。掌鹵
 簿儀仗車輅法服等事。直長,正八品。泰和三年省。



 承發司。管勾,從七品。同管勾,從八品。掌受發省部及外路文字。



 右屬尚書兵部。



 萬寧宮提舉司。舊大寧宮,更名壽安宮,又更今名。提舉,從六品。同提舉,從七品。掌守護宮城殿位。本把五人。



 慶寧宮提舉司。提舉,正七品。兼龍門縣令。同提舉,正八品。兼儀鸞監。



 右屬尚書刑部。



 修內司。大定七年設。使,從五品。副使,從六品。掌宮中營造事。
 兵匠一千六十五人,兵夫二千人,仍命少府監長官提控。直長二員,正八品。部役官四員,正八品。掌監督工役。受給官二員,正八品。掌支納諸物。



 都城所。提舉,從六品。同提舉,從七品。掌修完廟社及城隍門鑰、百司公廨、係官舍屋並栽植樹木工役等事。左右廂官各二員,正八品。掌監督工役。受給官二員,正八品。掌支納諸物及埏埴等事。



 祗應司。提點,從五品。令,從六品。丞,從七品。掌給宮中諸色工作。直長,正八品。收支庫都監、同監。泰和元年置。



 甄官署。令,從六品。丞,從七品。直長,正八品。掌劖石及埏
 埴之事。



 上林署。提點,從五品。泰和八年創,大安二年省。令,從六品。掌諸苑園池沼、種植花木果蔬及承奉行幸舟船事。丞,從七品。大定七年,增一員,分司南京,以勾判兼之。大安三年復省一員。直長二員,正八品。花木局都監、同監。舊設接手官四人,泰和元年罷,復以諸司人內置都監、同監二員。貞祐三年罷都、同監、以同樂園管勾兼。熙春園都監、同監三員。泰和四年置,貞祐三年省。同樂園管勾二員,每年額辦課程,隸南運司。宣宗南
 遷,罷課,改為隨朝職,正八品。



 右皆屬尚書工部。



 京東西南三路檢察司。興定四年置。使,從六品。副使,正七品。掌檢察支散軍糧,驗軍戶實給,均軍戶差役,勸農種,毋犯私殺馬牛、私鹽酒曲。



 南京豐衍東西庫。隸運司,貞祐二年同隨朝。使,正八品。副使,從八品。判二員,正九品。監支、納各一員,正八品。



 提舉南京榷貨司。貞祐四年置。提舉,從五品。同提舉,從六品。勾當官三員,正九品。



 提舉倉場司。貞祐五年置,先吏部闢舉,從省擬。使,從五品。副使,從六品。
 掌出納公平及毋致虧敗。監支納官,八品,十六員。以年六十以下廉幹人充,女直、漢人各一。廣盈倉、豐盈倉、永豐倉、廣儲倉、富國倉、廣衍倉、三登倉、常盈倉、西一場、西二場、西三場、東一場、東二場、南一場、北一場、北二場。通濟倉與在京倉,置監支納使副各一員。豐備倉、豐贍倉、廣濟倉、潼關倉,興定五年創置潼關倉監支納一員,兼樞密院彈壓。陳州倉四員。洧川倉二員。



 八作左右院。設官同上,掌收軍須、軍器。



\end{pinyinscope}