\article{列傳第二}

\begin{pinyinscope}

 后妃下



 ○睿宗欽慈皇后睿宗貞懿皇后世宗昭德皇后世宗元妃張氏世宗元妃李氏顯宗孝懿皇后暈宗昭聖皇后章宗欽懷皇后章宗元妃李氏衛紹王后徒單氏宣宗皇后王氏宣宗明惠皇后
 哀宗徒單皇后



 睿宗欽慈皇后,蒲察氏。睿宗元配。后之母,太祖之妹也。睿宗為左副元帥,天會十三年薨,追封潞王,后封路遷妃。皇統六年,進號冀國王妃。天德間,進國號。正隆例,親王止封一字王,睿宗封許王,后封許王妃。世宗即位,睿宗升祔,追謚欽慈皇后。贈后曾祖賽補司空、韓國公,祖蒲刺司徒、鄭國公,父按補太尉、曹國公。大定二年,祔葬景陵。



 世宗嘗曰:「今之女直,不比前輩,雖親戚世敘,亦不能知其詳。太后之母,太祖之妹,人亦不能知也。」謂宗敘曰:「亦是卿父譚王之妹,知之乎?」宗敘曰:「臣不能知也。」上
 曰:「父之妹且不知,其如疏遠何」。十九年,后族人勸農使莎魯窩請致仕,宰相以莎魯窩未嘗歷外,請除一外官,以均勞佚。上曰:「莎魯窩不閑政事,不可使治民。雖太后戚屬,富貴之可也。」不聽。



 貞懿皇后,李氏,世宗母,遼陽人。父雛訛只,仕遼,官至桂州觀察使。天輔間,選東京士族女子有姿德者赴上京,後入睿宗邸。七年,世宗生。天會十三年,睿宗薨,世宗時年十三。后教之有義方,嘗密謂所親曰:「吾兒有奇相,貴不可言。」居上京,內治謹嚴,臧獲皆守規矩,衣服飲食器皿無不精潔,敦睦親族,周給貧乏,宗室中甚敬之。后性
 明敏,剛正有決,容貌端整,言不妄發。



 舊俗,婦女寡居,宗族接續之。后乃祝髮為比丘尼,號通慧圓明大師,賜紫衣,歸遼陽,營建清安禪寺,別為尼院居之。貞元三年,世宗為東京留守。正隆六年五月,后卒。世宗哀毀過禮,以喪去官。未幾,起復為留守。是歲十月,后弟李石定策,世宗即位于東京,尊謚為貞懿皇后,其寢園曰孝寧宮。



 大定二年,改葬睿宗於景陵。初,后自建浮圖于遼陽,是為垂慶寺,臨終謂世宗曰:「鄉土之念,人情所同,吾已用浮屠法置塔于此,不必合葬也。我死,毋忘此言。」世宗深念遺命,乃即東京清安寺建神御殿,詔有司增大舊塔,起
 奉茲殿於塔前。敕禮部尚書王競為塔銘以敘其意。贈后曾祖參君司空、潞國公,祖波司徒、衛國公,父雛訛只太尉、隋國公。四年,封后妹為邢國夫人,賜銀千兩、錦綺二十端、絹五百匹。九年,神御殿名曰報德殿。詔翰林學士張景仁作《清安寺碑》,其文不稱旨,詔左丞石琚共修之。十三年,東京垂慶寺起神御殿,寺地褊狹,詔賈傍近民地,優與其直,不願鬻者以官地易之。二十四年,世宗至東京,幸清安、垂慶寺。



 世宗昭德皇后。烏林荅氏,其先居羅伊河,世為烏林荅部長,率部族來歸,居上京,與本朝為婚姻家。曾祖勝
 管,康宗時累使高麗。父石土黑,騎射絕倫,從太祖伐遼,領行軍猛安。雖在行伍間,不嗜殺人。以功授世襲謀克,為東京留守。



 后聰敏孝慈,容儀整肅,在父母家,宗族皆敬重之。既歸世宗,事舅姑孝謹,治家有敘,甚得婦道。睿宗伐宋,得白玉帶,蓋帝王之服御也。睿宗沒後,世宗寶畜之。后謂世宗曰:「此非王邸所宜有也,當獻之天子。」世宗以為然,獻之熙宗,於是悼后大喜。熙宗晚年頗酗酒,獨於世宗無間然。



 海陵篡立,深忌宗室。烏帶譖秉德以為意在葛王。秉德誅死,后勸世宗多獻珍異以說其心,如故遼骨睹犀佩刀、吐鶻良玉茶器之類,皆奇寶也。海
 陵以世宗恭順畏己,由是忌刻之心頗解。



 后不妒忌,為世宗擇後房,廣繼嗣,雖顯宗生後而此心不移。后嘗有疾,世宗為視醫藥,數日不離去。后曰:「大王視妾過厚,其知者以為視疾,不知者必有專妒之嫌。」又曰:「婦道以正家為大,第恐德薄,無補內治,安能效嬪妾所為,惟欲己厚也。」



 世宗在濟南,海陵召后來中都。后念若身死濟南,海陵必殺世宗,惟奉詔,去濟南而死,世宗可以免。謂世宗曰:「我當自勉,不可累大王也。」召王府臣僕張僅言諭之曰:「汝,王之腹心人也。為我禱諸東嶽,我不負王,使皇天后土明鑒我心。」召家人謂之曰:「我自初年為婦以至
 今日,未嘗見王有違道之事。今宗室往往被疑者,皆奴僕不良,傲恨其主,以誣陷之耳。汝等皆先國王時舊人,當念舊恩,無或妄圖也。違此言者,我死後於冥中觀汝所為。」眾皆泣下。后既離濟南,從行者知后必不肯見海陵,將自為之所,防護甚謹。行至良鄉,去中都七十里,從行者防之稍緩,后得間即自殺。海陵猶疑世宗教之使然。



 世宗自濟南改西京留守,過良鄉,使魯國公主葬后于宛平縣土魯原。大定二年,追冊為昭德皇后,立別廟。贈三代,曾祖勝管司空、徐國公,曾祖母完顏氏徐國夫人,祖術思黑司徒、代國公,祖母完顏氏,代國夫人,父石
 土黑太尉、瀋國公,母完顏氏瀋國夫人。敕有司改葬,命皇太子致奠。以后兄暉子天錫為太尉,石土黑後授世襲猛安。上謂天錫曰:「朕四五歲時與皇后定婚,乃祖太尉置朕于膝上曰:『吾婿七人,此婿最幼,後來必大吾門。』今卜葬有期,疇昔之言驗矣。」



 六年,利涉軍節度副使烏林荅鈔兀捕逃軍受贓,當殆。有司奏,鈔兀,后大功親,當議。詔論如法。



 八年七月,章宗上,世宗喜甚。謂顯宗曰:「得社稷冢嗣,朕樂何極。此皇后貽爾以陰德也。」



 十年十月,將改葬太尉石土黑,有司奏禮儀,援唐葬太尉李良器、司徒馬燧故事,百官便服送至都門外五里。上曰:「前改
 葬太后父母,未嘗用此故事。但以本朝禮改葬之,惟親戚皆送。」詔皇太子臨奠。



 十一年,皇太子生日,世宗宴於東宮。酒酣,命豫國公主起舞。上流涕曰:「此女之母皇后,婦道至矣。朕所以不立中宮者,念皇后之德今無其比故也。」



 十二年四月,立皇后別廟于太廟東北隅。是歲五月,車駕幸士魯原致奠。十九年,改卜于大房山。十一月甲寅,皇后梓宮至近郊,百官奉迎。乙卯,車駕如楊村致祭。丙辰,上登車送,哭之慟。戊午,奉安于磐寧宮。庚申,葬於坤厚陵,諸妃祔焉。二十九年,祔葬興陵。章宗時,有司奏太祖謚有「昭德」字,改謚明德皇后。



 元妃張氏,父玄征。母高氏,與世宗母貞懿皇后葭莩親。世宗納為次室,生趙王永中,而張氏卒。大定二年,追封宸妃。是歲十月,追進惠妃。十九年,追進元妃。



 大定二十五年,皇太子薨。永中於諸子最長,而世宗與徒單克寧議立章宗為太孫。世宗嘗曰:「克寧與永中有親,而建議立太孫,真社會稷臣也。」尚書左丞汝弼者,玄徵子,永中每舅。汝弼妻高陀斡屢以邪言怵永中,畫元妃像,朝夕事之,覬望徼福,及挾左道。明昌五年,高陀斡誅死,事連汝弼及永中,汝弼以死後事覺,得不追削官爵,而章宗心疑永中,累年不釋。諫官賈守謙、路鐸上疏欲寬解上意,章
 宗愈不悅。平章政事完顏守貞持其事不肯決,章宗怒守貞,罷知濟南府,諸諫官皆斥外,賜永中死。金代外戚之禍,惟張氏云。



 元妃李氏,南陽郡王李石女。生鄭王允蹈、衛紹王允濟、潞王允德。豫王允成母昭儀梁氏早卒,上命允成為妃養子。大定元年,封賢妃。二年,進封貴妃。七年,進封元妃。世宗即位,感念昭德皇后,不復立后。嘗曰:「朕所以不復立后者,今後宮無皇后之賢故也。」元妃下皇后一等,在諸妃上。石有定策功,世宗厚而深制之,寵以尚書令之位,而責成左右丞相以下,妃雖貴,不得預政,宮壺無
 事。



 大定二十一年二月,上如春水,次長春宮。戊子,妃以疾薨。詔允成、允蹈、允濟、允德皆服衰絰居喪。己丑,皇太子及扈從臣僚,奉慰于芳明殿。辛卯,留守官平章政事唐括安禮、曹王允功等上表奉慰。御史中丞張九思提控殯事,少府監左光慶、大興少尹王翛典領鹵簿儀仗。宮籍監別治殯所,還殯京師。乙未,入自崇智門,百官郊迎,親戚迎奠道路,殯于興德宮西位別室。庚子,上至京師,幸興德宮致奠。此葬,三致尊焉。詔平章政事烏古論元忠監護葬事。癸未,啟取,上輟朝。皇太子、親王、宗戚、百官送葬。甲申,葬於海王莊。丙戌,上如海王莊燒飯。二十
 八年九月,與賢妃石抹氏、德妃徒單氏、柔妃大氏俱陪葬于坤厚陵。衛紹王即位,追謚光獻皇后,贈妃弟獻可特進。貞祐三年九月,削皇后號。



 顯宗孝懿皇后,徒單氏。其先忒里闢刺人也。曾祖抄,從太祖取遼有功,命以所部為猛安,世襲之。祖婆盧火,以戰功多,累官開府儀同三司,贈司徒、齊國公。父貞尚遼王宗乾女梁國公主,加駙馬都尉,贈太師、廣平郡王。



 后以皇統七年生於遼陽。母夢神人授以寶珠,光焰滿室,既寤而生,紅光燭于庭。后性莊重寡言,父母嘗令總家事,細大畢辦,諸男不及也。



 世宗初即位,貞為御史大夫,
 自南京馳見。世宗喜謂之曰:「卿雖廢主腹心臣,然未嘗助彼為虐,況卿家法可尚,其以卿女為朕子妃。」及顯宗為皇太子,大定四年九月,備禮親迎於貞弟。世宗臨宴,盡歡而罷。是年十一月,顯宗生辰,初封為皇太子妃。



 八年七月,上遣宣徽使移刺神獨斡以名馬、寶刀、御饍賜太子及妃,仍諭之曰:「妃今臨蓐,願平安得雄。有慶之後,宜以此刀置左右。」既而皇孫生,是為章宗。時上幸金蓮川,次冰井,翌日,上臨幸撫視,宴甚歡。又賜御服佩刀等物,謂顯宗曰:「祖宗積慶,且皇后陰德至厚,而有今日,社稷之洪福也。」又謂李石、紇石烈志寧曰:「朕諸子雖多,皇
 后止有太子一人而已。今幸得嫡孫,觀其骨相不凡,又生麻達葛山,山勢衍氣清,朕甚嘉之。」因以山名為章宗小字。



 后素謙謹,每畏其家世崇寵,見父母流涕而言曰:「高明之家,古人所忌,願善自保持。」其後,家果以海陵事敗,盡其遠慮如此。世宗嘗謂諸王妃、公主曰:「皇太子妃容止合度,服飾得中,爾等當法效之。」章宗即位,尊為皇太后,更所居仁壽宮名曰隆慶宮。詔有司歲奉金千兩、銀五千兩、重幣五百端、絹二千疋、綿二萬兩、布五百疋、錢五萬貫。他所應用,內庫奉之,毋拘其數。



 上月或五朝六朝,而后愈加敬儉,見諸大長公主,禮如平時,敦睦九
 族,恩紀皆合。尤惡聞人過,諛佞之言無所得入。恕以容物,未嘗見喜慍。然御下公平,雖至親無所阿徇。嘗誡諸侄曰:「皇帝以我故,乃推恩外家,當盡忠圖報。勿謂小善為無益而弗為,小惡為無傷而弗去。毋藉吾之貴,輒肆非違,以干國家常憲。」一日,妹並國夫人、嫂涇國夫人等侍側,因諭之曰:「爾家累素重,且非豐厚,宜節約財用,勿以吾為可恃。吾受天下之養,豈有所私積哉。況財用者,天下之財用也。吾終不能多取以富爾之私室。」家人有以玉盂進者,卻之,且曰:「貴異物而殫財用,非我所欲也。況我之賜予有度,今爾以此為獻,何以自給。徒費汝財,
 我實無用,後勿復爾。」明昌元年,禮官議以五月奉上冊寶,后弗許。上屢為之請,后曰:「今世宗服未終,遽衣錦繡、佩珠玉,於禮何安。當俟服闋行之。」上諭有司曰:「太后執意甚堅,其待來年。」明昌二年正月,崩於隆慶宮,年四十五。謚曰孝懿,祔葬裕陵。



 后好《詩》、《書》尤喜《老》、《莊》,學純淡清懿,造次心於禮。逮嬪御以和平,其有生子而母亡者,視之如己所生,慈訓無間。上時問安,見事有未當者,必加之嚴誡云。



 昭聖皇后,劉氏,遼陽人。天眷二年九月己亥夜,后家若見有黃衣女子入其母室中者,俄頃,后生。性聰慧,凡字
 過目不忘。初讀《孝經》,旬日終卷。最喜佛書。世宗為東京留守,因擊球,見而奇之,使見貞懿皇后于府中,進退閑雅,無恣睢之色。大定元年,選入東宮,時年二十三。



 三年三月十三日,宣宗生。是日,大雨震電,后驚悸得疾,尋卒。承安五年,贈裕陵昭華。宣宗即位,追尊為皇太后,升祔顯宗廟,追謚昭聖皇后。



 章宗欽懷皇后,蒲察氏,上京路曷速河人也。曾署祖太神,國初有功,累階光祿大夫,贈司空、應國公。祖阿胡迭,官至特進,贈司徒、譙國公。父鼎壽尚熙宗鄭國公主,授駙馬都尉、中都路昏得渾山猛安、曷速木單世襲謀克,累
 官至金吾衛上將軍,贈太尉、越國公。



 后之始生,有紅光被體,移時不退。就養於姨冀國公主,既長,孝謹如事所生。大定二十三年,章宗為金源郡王,行納采禮。世宗遣近侍局使徒單懷忠就賜金百兩、銀千兩、廄馬六匹、重彩三十端。拜命間,慶雲見於日側,觀者異之。是年十一月,備禮親迎。詔親王宰執三品已上官及命婦會禮,封金源郡王夫人,後進封妃,崩。



 后性淑明,風儀粹穆,知讀書為文」帝即位,遂加追冊,仍詔告中外,奉安神主于坤寧宮,歲時致祭。大安初,祔于道陵。



 元妃李氏師兒,其家有罪,沒入宮籍監。父湘,母王盻兒,
 皆微賤。大定末,以監戶女子入宮。是時宮教張建教宮中,師兒與諸宮女皆從之學。故事,宮教以青紗隔障蔽內外,宮教居障外,諸宮女居障內,不得面見。有不識字及問義,皆自障內映紗指字請問,宮教自障外口說教之。諸女子中惟師兒易為領解,建不知其誰,但識其音聲清亮。章宗嘗問建,宮教中女子誰可教者。建對曰:「就中聲音清亮者最可教。」章宗以建言求得之。宦者梁道譽師兒才美,勸章宗納之。章宗好文辭,妃性慧黠,能作字,知文義,尤善伺候顏色,迎合旨意,遂大愛幸。明昌四年,封為昭容。明年,進封淑妃,父湘追贈金紫光祿大夫、
 上柱國、隴西郡公。祖父、曾祖父皆追贈。



 兄喜兒舊嘗為盜,與弟鐵哥皆擢顯近,勢傾朝廷,風采動四方,射利競進之徒爭趨走其門,南京李炳、中山李著與通譜系,超取顯美。胥持國附依以致宰相。怙財固位,上下紛然,知其姦蠹,不敢擊之,雖擊之,莫能去也。紇石烈執中貪愎不法,章宗知其跋扈,而屢斥屢起,終亂天下。



 自欽懷皇后沒世,中宮虛位久,章宗意屬李氏。而國朝故事,皆徒單、唐括、蒲察、拏懶僕散、紇石烈、烏林荅、烏古論諸部部長之家,世為姻婚,娶后尚主,而李氏微甚。至是,章宗果欲立之,大臣固執不從,臺諫以為言,帝不得已,進封為
 元妃,而勢位熏赫,與皇后侔矣。一日,章宗宴宮中,優人瑇瑁頭者戲于前。或問:「上國有何符瑞?」優曰:「汝不聞鳳皇見乎。」其人曰:「知之,而未聞其詳。」優曰:「其飛有四,所應亦異。若響上飛則風雨順時,響下飛則五穀豐登,響外飛則四國來朝,向裏飛則加官進祿。」上笑而罷。



 欽懷后及妃姬嘗有子,或二三歲或數月輒夭。承安五年,帝以繼嗣未立,禱禮太廟、山陵。少府監張汝猷因轉對,奏「皇嗣未立,乞聖主親行禮事之後,遣近臣詣諸岳觀廟祈禱。」詔司空襄往亳州禱太清宮,既而止之,遣刑部員外郎完顏匡往焉。



 泰和二年八月丁酉,元妃生皇子忒鄰,
 群臣上表稱賀。宴五品以上於神龍殿,六品以下宴于東廡下。詔平章政事徒單鎰報謝太廟,右丞完顏匡報謝山陵,使使亳州報謝太清宮。既彌月,詔賜名,封為葛王。葛王,世宗初封,大定後不以封臣下,由是三等國號無葛。尚書省奏,請於瀛王下附葛國號,上從之。十二月癸酉,忒鄰生滿三月,敕放僧道度牒三千道,設醮于玄真觀,為忒鄰祈福。丁丑,御慶和殿,浴皇子。詔百官用元旦禮儀進酒稱賀,五品以上進禮物。生凡二歲而薨。



 兄喜兒,累官宣徽使、安國軍節度使。弟鐵哥,累官近侍局使、少府監。



 至八年,承御賈氏及范氏皆有娠,未及
 乳月,章宗已得嗽疾,頗困。是時衛王永濟自武定軍來朝。章宗於父兄中最愛衛王,欲使繼體立之,語在《衛紹王紀》。衛王朝辭,是日,章宗力疾與之擊球,謂衛王曰:「叔王不欲作主人,遽欲去邪?」元妃在傍,謂帝曰:「此非輕言者。」十一月乙卯,章宗大漸,衛王未發,元妃與黃門李新喜議立衛王,使內侍潘守恒召之。守恒頗知書,識大體,謂元妃曰:「此大事,當與大臣議。」乃使守恒召平章政事完顏匡。匡,顯宗侍讀,最為舊臣,有征伐功,故獨召之。匡至,遂與定策立衛王。丙辰,章宗崩,遺詔皇叔衛王即皇帝位。詔曰:「朕之內人,見有娠者兩位。如其中有男,當立
 為儲貳。如皆是男子,擇可立者立之。」



 衛紹王即位,大安元年二月,詔曰:「章宗皇帝以天下重器畀于眇躬,遺旨謂掖庭內人有娠者兩位,如得男則立為儲貳。申諭多方,皎如天日。朕雖涼菲,實受付託,思克副於遺意,每曲為之盡心,擇靜舍以俾居,遣懿親而守視。欽懷皇后母鄭國公主及乳母蕭國夫人晝夜不離。昨聞有爽於安養,已用軫憂而弗寧,爰命大臣專為調護。今者平章政事僕散端、左丞孫即康奏言,承御賈氏當以十一月免乳,今則已出三月,來事未可度知。范氏產期,合在正月,而太醫副使儀師顏言,自年前十一月診得范氏胎氣
 有損,調治迄今,脈息雖和,胎形已失。及范氏自願於神御前削髮為尼。重念先皇帝重屬大事,豈期聞此,深用怛然。今范氏既已有損,而賈氏猶或可冀,告於先帝,願降靈禧,默賜保全,早生聖嗣。尚恐眾庶未究端由,要不匿於播敷,使咸明於吾意。」



 四月,詔曰:「近者有訴元妃李氏,潛計負恩,自泰和七年正月,章宗暫嘗違豫,李氏與新喜竊議,為儲嗣未立,欲令宮人詐作有身,計取他兒詐充皇嗣。遂於年前閏月十日,因賈承御病嘔吐,腹中若有積塊,李氏與其母王盻兒及李新喜謀,令賈氏詐稱有身,俟將臨月,於李家取兒以入,月日不偶則規別
 取,以為皇嗣。章宗崩,謀不及行。當先帝彌留之際,命平章政事完顏匡都提點中外事務,明有敕旨,『我有兩宮人有娠』,更令召平章,左右並聞斯語。李氏並新喜乃敢不依敕旨,欲喚喜兒、鐵哥,事既不克,竊呼提點近侍局烏古論慶壽與計,因品藻諸王,議復不定。知近侍局副使徒單張僧遣人召平章,已到宣華門外,始發勘同。平章入內,一遵遺旨,以定大事。方先帝疾危,數召李氏,李氏不到。及索衣服,李氏承召亦不即來,猶與其母私議。先皇平昔或有幸御,李氏嫉妒,令女巫李定奴作紙木人、鴛鴦符以事魘魅,致絕聖嗣。所為不軌,莫可殫陳。事
 既發露,遣大臣按問,俱已款服。命宰臣往審,亦如之。有司議,法當極刑。以其久侍先帝,欲免其死。王公百僚,執奏堅確。今賜李氏自盡。王盻兒、李新喜各正典刑。李氏兄安國軍節度使喜兒、弟少府監鐵哥如律,仍追除復係監籍,於遠地安置。諸連坐並依律令施行。承御賈氏亦賜自盡。」



 蓋章宗崩三日而稱范氏胎氣有損。章宗疾彌留,亦無完顏匡都提點中外事務敕旨。或謂完顏匡欲奪定策功,構致如此。自後天下不復稱元妃,但呼曰李師兒。



 及胡沙虎弒衛王,立宣宗,請貶降衛王,降為東海郡侯。其詔曰:「大安之初,頒諭天下,謂李氏與其母王
 盻兒及李新喜同謀,令賈氏虛稱有身,各正罪法。朕惟章宗皇帝聖德聰明,豈容有此欺紿。近因集議,武衛軍副使兼提點近侍局完顏達、霍王傅大政德皆言賈氏事內有冤。此時,達職在近侍,政德護賈氏,所以知之。朕親臨問左證,其事暖昧無據,當時被罪貶責者可俱令放免還家。」由是李氏家族皆得還。



 衛紹王后徒單氏,大安元年,立為皇后。至寧元年,胡沙虎亂,與衛王俱遷於衛邸。帝遇弒,宣宗即位,衛王降為東海郡侯,徒單氏削皇后號。貞祐二年,遷都汴,詔凡衛紹王及鄗厲王家人皆徙鄭州,仍禁錮,不得出入。男女
 不得婚嫁者十九年。天興元年,詔釋禁錮。是時,河南已不能守,子孫不知所終。



 宣宗皇后王氏,中都人,明惠皇后妹也。其父微時嘗夢二玉梳化為月,已而生二后,及沒,有芝生于柩。初,宣宗封翼王,章宗詔諸王求民家子,以廣繼嗣。是時,后與龐氏偕入王邸,及見后姊有姿色,又納之。貞祐元年九月,封后為元妃,姊為淑妃,龐氏為真妃。淑妃生哀宗,真妃生守純,后無子,養哀宗為己子。貞祐二年七月,賜姓溫敦氏,立為皇后。追封后曾祖得壽司空、冀國公,曾祖母劉氏冀國夫人,祖璞司徒、益國公,祖母楊氏益國夫人,
 父彥昌太尉、汴國公,母馬氏汴國夫人。



 三年,莊獻太子薨,哀宗為皇太子。宣宗崩,哀宗即位。正大元年,尊后為皇太后,號其宮曰仁聖,進封后父曰南陽郡王。



 或曰:宣宗為諸王時,莊獻太子母為正妃,及即位,尊為皇后。貞祐元年九月,詔曰:「元妃某氏久奉侍於潛籓,已賜封於國號,可立為皇后。」其名氏蓋不可考也。或又曰:自王氏姊妹入宮而后寵衰,尋為尼,王氏遂立為后,皆后姊明惠之謀也。



 初,王氏姊妹受封之日,大風昏霾,黃氣充塞天地。已而,后夢丐者數萬踵其後,心甚惡之。占者曰:「后者,天下之母也。百姓貧窶,將誰訴焉?」后遂敕有司,京城
 設粥與冰藥。及壬辰、癸巳歲,河南饑饉。大元兵圍汴,加以大疫,汴城之民,死者百餘萬,后皆目睹焉。



 哀宗釋服,將禘饗太廟,先期,有司奏冕服成,上請仁聖、慈聖兩宮太后御內殿,因試衣之以見,兩宮大悅。上更便服,奉觴為兩宮壽。仁聖太后諭上曰:「祖宗初取天下甚不易。何時使四方承平,百姓安樂,天子服此法服,於中都祖廟行禘饗乎?」上曰:「阿婆有此意,臣亦何嘗忘。」慈聖太后亦曰:「恒有此心,則見此當有期矣。」遂酌酒為上壽,歡然而罷。



 天興元年冬,哀宗遷歸德。二年正月,遣近侍徒單四喜、術甲荅失不奉迎兩宮。后御仁安殿,出鋌金及七寶
 金洗,分賜從行忠孝軍。是夜,兩宮及柔妃裴滿氏等乘馬出宮,行至陳留,城左右火起,疑有兵,不敢進。后亟命還宮。明日,入京憩四喜家。少頃,輦迎入宮。方謀再行,京城破,后及諸妃嬪北遷,不知所終。惟寶符李氏從至宣德州,居摩訶院。李氏自入院,止寢佛殿中,作為幡旆。會當同后妃北行,將發,佛像前自縊死,且自書門紙曰「寶符御侍此處身故。」



 宣宗明惠皇后,王皇后之姊也。生哀宗。宣宗即位,封為淑妃。及妹立為后,進封元妃。哀宗即位,詔尊為皇太后,號其宮曰慈聖。



 后性端嚴,頗達古今。哀宗已立為皇太
 子,有過尚切責之,及即位,始免賈楚。一日,宮中就食,尚器有玉碗楪三,一奉太后,二奉帝及中宮。荊王母真妃龐氏以瑪瑙器進食,后見之怒,召主者責曰:「誰令汝妄生分別,荊王母豈卑我兒婦耶。非飲食細故,已令有司杖殺汝矣。」是後,宮中奉真妃有加。或告荊王謀不軌者,下獄,議已決。帝言于后,后曰:「汝止一兄,奈何以讒言欲害之。章宗殺伯與叔,享年不永,皇嗣又絕,何為欲效之耶。趣赦出,使來見我。移時不至,吾不見汝矣。」帝起,后立待,王至,涕泣慰撫之。



 哀宗甚寵一宮人,欲立為后。后惡其微賤,固命出之。上不得已,命放之出宮,語使者曰:「爾
 出東華門,不計何人,首遇者即賜之。」於是遇一販繒者,遂賜為妻。點檢撒合輦教上騎鞠,后傳旨戒之云:「汝為人臣,當輔主以正,顧乃教之戲耶。再有聞,必大杖汝矣。」



 比年小捷,國勢頗振,文士有奏賦頌以聖德中興為言者。后聞不悅曰:「帝年少氣銳,無懼心則驕怠生。今幸一勝,何等中興,而若輩諂之如是。」



 正大八年九月丙申,后崩,遺命園陵制度,務從儉約。十二月己未,葬汴城迎朔門外五里莊獻太子墓之西。謚明惠皇后。



 哀宗皇后,徒單氏。宣宗及后有疾,后嘗刲膚以進,宣宗聞而嘉之。興定四年,后父鎮南軍節度使頑僧有罪,宣
 宗以后純孝,因曲赦之,聽其致仕。正大元年,詔立為皇后。哀宗遷歸德,遣后弟四喜等詣汴奉迎,夜至陳留,不敢進,復歸于汴。未幾,城破北遷,不知所終。



 贊曰:《周禮》「九嬪,掌婦學之法,婦德、婦言、婦容、婦功。」班昭氏論之曰:「婦德,不必才明絕異也。婦言,不必便口利辭也。婦容,不必顏色美麗也。婦功,不必功巧過人也。清閑貞靜,守節整齊,行己有恥,動靜有法,是謂婦德。擇辭而說,不道惡語,時然後言,不厭於人,是謂婦言。盥浣塵穢,服飾鮮潔,沐浴以時,身不垢辱,是謂婦容。專心紡績,不好戲笑,潔齊酒食,以奉賓客,是謂婦功。」後世婦學不脩,
 麗色以相高,巧言以相傾,銜能以市恩,逢迎以固寵。是故悼平掣頓皇統,以隕其身,海陵蠱惑群嬖,幾亡其國。道陵李氏擅寵蠹政,卒憤其宗。嗚呼,可不戒哉。



\end{pinyinscope}