\article{列傳第三十八}

\begin{pinyinscope}

 ○孟鑄宗端修完顏閭山路鐸完顏伯嘉術虎筠壽張煒高竑李復亨



 孟鑄,大定末,補尚書省令史。明昌元年,御史臺奏薦戶部員外郎李獻可、完顏掃合、太府丞徒單繹、宮籍監丞張庸、右警巡使袞、禮部主事蒲察振壽、戶部主事郭蛻、應奉翰林文字移刺益、中都鹽鐵判官趙皓、尚書省令
 史劉昂及鑄十一人皆剛正可用。詔除獻可右司諫,掃合磁州刺史,繹祕書丞,庸中都右警巡使,袞彰國軍節度副使,振壽治書侍御史,蛻同知定武軍節度使事,益翰林修撰,皓都水丞,昂戶部主事,鑄刑部主事。累遷中都路按察副使、南京副留守、河平軍節度使。



 泰和四年,入為御史中丞,召見於香閣。上謂鑄曰:「朕自知卿,非因人薦舉也。御史責任甚重,往者臺官乃推求細故,彈劾小官,至於巨室重事,則畏徇不言。其勤乃職,無廢朕命。」是歲,自春至夏,諸郡少雨。鑄奏:「今歲愆陽,已近五月,比至得雨,恐失播種之期,可依種麻菜法,擇地形稍下處
 撥畦種穀,穿土作井,隨宜灌溉。」上從其言,區種法自此始。



 無何,奏彈知大興府事紇石烈執中過惡,其文略曰:「京師百郡之首,四方取則。知府執中貪殘專恣,不奉法令,自奉聖州罪解以後,怙罪不悛,蒙朝廷恩貸,轉生跋扈。雄州詐奪人馬,平州冒支己俸,無故破魏廷碩家,發其冢墓。拜表以調鷹不赴,祈雨聚妓戲嬉,毆詈同僚,擅令住職,失師帥之體。乞行黜退,以厭人望。」上以執中東宮舊人,頗右之,謂鑄曰:「執中麤人,似有跋扈者。」鑄曰:「明天子在上,豈容有跋扈之臣?」上悟,詔尚書省問之。



 泰和五年,唐、鄧、河南屢有警,議者謂宋且敗盟。六年正月,宋
 賀正旦使陳克俊等朝辭,上使鑄就館諭克俊以國家涵容之意,果不詳此旨,恐兵未可息也。使以上言達宋主。章宗本無意用兵,故再三諭之。



 鑄論提刑司改按察司,差官復察,權削望輕。下尚書省議。參知政事賈鉉奏:「乞差監察時,即別遣官偕往,更不復察,諸疑獄並令按察司從正與決,庶幾可慰人望。」從之。



 永豐庫官不守宿,因而被盜,上召登聞鼓院官欲有所問,皆不在。上諭鑄曰:「此輩慢法如此,御史臺所職何事也!」復諭御史大夫宗肅及鑄曰:「朕聞唐宰相宿省中,卿等所知也。臺官、六部官、其餘司局亦嘗宿直。今尚書省左右司官宿直,餘
 亦當准此。」八年,除絳陽軍節度使。至寧元年,復為御史中丞。



 紇石烈執中作亂,召鑄及右諫議大夫張行信俱至大興府,問曰:「汝輩向來彈我者耶?」鑄等各以正言答之。執中乃遣還家,曰:「且須後命。」既而執中死,鑄亦尋卒。



 宗端脩,字平叔,汝州人。章宗避睿宗諱上一字,凡太祖諸子皆加「山」為「崇」,改「宗」氏為「姬」氏。端脩好學,喜名節,中大定二十二年進士第。明昌間,補尚書省令史。承安元年,監察御史孫椿年、武簡職事不脩舉,詔以端脩及范鐸代之。是時元妃李氏兄弟干預朝政,端脩上書乞遠小人。上遣李喜兒傳詔部端脩:「小人為誰,其以姓名對。」
 端脩對曰:「小人者,李仁惠兄弟。」仁惠,喜兒賜名也。喜兒不敢隱,具奏之。上雖責喜兒兄弟,而不能去也。四年,復上書言事,宰相惡之,坐以不經臺官直進奏帖,準上書不以實,削一官,期年後敘。章宗知端脩不為眾所容,釋之,改大理司直。泰和四年,遷大理丞,召見於香閣。上謂端脩曰:「汝前為御史,以幹能見用。汝言多細碎,不究其實,嘗令問汝,亦不汝罪。及為大理司直,乃能稱職,用是擢汝為丞,盡乃心力,惟法是守,勿問上位宰執所見何如,汝其志之!」知大同府紇石烈執中陳言,下大理寺議。端脩謂執中言事涉私治罪。詔以端脩別出情見不當,
 與司直溫敦按帶各削一官解職。久之,為節度副使,卒官。



 端脩終以直道不振於時,自守愈篤。妻死不復更娶,獨居二十年,士論高之。汝州司候游彥哲將之官,問為政。端脩曰:「為政不難,治氣養心而已。」彥哲不達,端脩曰:「心正則不私,氣平則不暴。為政之術,盡於此矣。」



 完顏閭山,蓋州猛安人。明昌二年進士,累調觀察判官,補尚書省令史,知管差除。授都轉運都勾判官,改河東南路轉運都勾判官、南京警巡使。丁母憂,起復南京按察判官,累遷沁南軍節度使,入為工部尚書。貞祐三年,知京兆府事,充行省參議官。四年,知鳳翔府事。興定元
 年冬,詔陜西行省伐宋,閭山權元帥右都監,參議諸軍事。宋兵千餘人伏吳寨谷,閭山率騎兵掩擊敗之,追襲十五里,殺三百餘,獲牛羊以千計。改知平涼府,敗宋人于步落堝。遷官一階。三年,召為吏部尚書。廷議選戶部官,往往舉聚斂苛刻以應詔。閭山曰:「民勞至矣,復用此輩,將何以堪。」識者稱之。三年,朝廷以晉安行元帥府陀滿胡土門暴刻,以閭山代之。是歲十月,卒。



 路鐸,字宣叔,伯達子也。明昌三年,為左三部司正。上書言事,召見便殿,遷右拾遣。明年,盧溝河決,鐸請自玄同口以下、丁村以上無修舊堤,縱使分流,以殺減水勢。詔
 工部尚書胥持國與鐸同檢視。章宗將幸景明宮,是歲民饑,不可行。御史中丞董師中上書諫,鐸與左補闕許安仁繼之,賜對御閣。詔尚書省曰:「朕不禁暑熱,欲往山後。今臺諫言民間多闕食,朕初不盡知,既已知之,其忍自奉以重困民哉。」乃罷行。



 尚書左丞完顏守貞每論政事,守正不移,與同列不合,罷知東平府事,臺諫因而擠之。鐸上書論守貞賢,可復用,其言太切,召對于崇政殿。既而章宗以鐸書語大臣,於是尚書左丞烏林答愿、參知政事夾谷衡、胥持國奏路鐸以梁冀比右丞相,所言狂妄,不稱諫職。右丞相,夾谷清臣也。上曰:「周昌以傑、紂
 比漢高祖,高祖不以為忤。路鐸以梁冀比丞相耳。」頃之,守貞入為平章政事。五年,復與禮部尚書張暐、御史中丞董師中、右諫議大夫賈守謙、翰林脩撰完顏撒刺諫幸景明宮,語多激切,章宗不能堪,遣近侍局直長李仁願召凡諫北幸者詣尚書省,詔曰:「卿等諫北幸甚善,但其間頗失君臣之體耳。」



 是歲,郝忠愈獄起,事密,諫官不能察其詳,議者頗謂事涉鎬王永中,思有以寬解上意。右諫議大夫賈守謙上封事,鐸繼之,尤切直。上優容之,謂鐸曰:「汝言諸王皆有覬心,游其門者不無橫議,是何言也。但朕不罪諫官耳。」頃之,尚書省奏擬鐸同知河北
 西路轉運使事,詔再任右拾遺,謂宰相曰:「鐸敢言,但識短耳。朕嘗詰責而氣不沮。」鐸因召對,論宰相權太重。上曰:「凡事由朕,宰相安得權重。」既而復奏曰:「乞陛下勿泄此言,泄則臣齏粉矣。」上曰:「宰相安能齏粉人!」至是,章宗並以此言告宰相,雖留再任,宰相愈銜之。改右補闕。



 自完顏守貞再入相,以政事為己任,胥持國方幸,尤忌守貞,并忌鐸輩。鐸輩雖嘗為守貞論辨而不相附。鐸論邊防,守貞以為掇拾唐人餘論,皆不行。及守貞持鎬王永中事久不決,鐸等亦上言切諫,並指以為黨。上乃出守貞知濟南府,凡曾薦守貞者皆黜降,謂宰臣曰:「董師中
 謂臺省無守貞不可治,路鐸、李敬義皆稱舉之者。然三人者後俱可用,今姑出之。」上復曰:「路鐸敢言,甚有時名,一旦外補,人將謂朕不能容直臣。可選敢言及才識處鐸右者。」參知政事馬琪奏曰:「鐸雖知無不言,然亦多不當理。」上曰:「諫官非但取敢言,亦須間有出朕意表者,乃有裨益耳。」於是,吏部尚書董師中出為陜西路轉運使,鐸為南京留守判官。戶部郎中李敬義方使高麗還,即出為安化軍節度副使。詔曰:「卿等昨來交薦守貞公正可用,今坐所舉失實耳。」



 承安二年,召為翰林修撰,同看讀陳言文字。上召禮部尚書張暐、大理卿麻安上及
 鐸,問趙晏所言十事,因問董師中、張萬公優劣。鐸奏:「師中附胥持國以進,趙樞、張復亨、張嘉貞皆出持國門下,嘉貞復趨走襄之門。持國不可復用,若再相,必亂綱紀。」上曰:「朕豈復相此人,但遷官二階使致仕,何為不可?」持國黨聞之,怒愈甚。改監察御史。



 參知政事楊伯通引用鄉人李浩,鐸劾奏:「伯通以公器結私恩,左司郎中賈益、知除武郁承望風旨,不詳檢起復條例。」涉妄冒,大夫張暐抑之不行。上命同知大興府事賈鉉詰問。張暐、伯通待罪于家。賈鉉奏:「近詔書詰問御史大夫張暐。暐言路鐸嘗稟會楊伯通私用鄉人李浩。暐以為彈絀大臣,須
 有阿曲實迹,恐所劾不當,臺綱愈壞,令再體察。賈益言除授皆宰執公議,奏稟,不見伯通私任形迹。」於是,詔責鐸言事輕率,慰諭伯通治事如故。



 頃之,遷侍御史,主奏事。監察御史姬端脩以言事下吏,使御史臺令史郭公仲達意于大夫張暐及鐸。暐與鐸奏事殿上,上問:「姬端脩彈事嘗申臺官否?」對曰:「嘗來面議。」端脩款伏乃云:「只曾與侍御私議,大夫不知也。」既而端脩杖七十收贖,公仲杖七十替罷。暐、鐸坐奏事不實,暐追一官,鐸兩官,皆解職。頃之,起為泰定軍節度副使。上謂宰臣曰:「凡言事者,議及朕躬亦無妨,語涉宰相,間有憎嫌,何以得進?」詔
 左司計鐸資考至正五品,即除東平府治中。未幾,景州闕刺史,尚書省已奏郭歧為之,詔特改鐸為景州刺史,仍勿送審官院。鐸述十二訓以教民。詔曰:「路鐸十二訓皆勸人為善,遍諭州郡使知之。」遷陜西路按察副使。坐以糾彈之官與京兆府治中蒲察張鐵、總管判官辛孝儉、推官愛剌宴飲,奪路一官解職。泰和六年,召為翰林待制兼知登聞鼓院,累除孟州防禦使。貞祐初,城破,投沁水死。



 鐸剛正,歷官臺諫,有直臣之風。為文尚奇,詩篇溫潤精致,號《虛舟居士集》云。



 完顏伯嘉字,輔之,北京路訛魯古必剌猛安人。明昌二
 年進士,調中都左警巡判官。孝懿皇后妹晉國夫人家奴買漆不酬直,伯嘉鉤致晉國用事奴數人繫獄。晉國白章宗,章宗曰:「姨酬其價,則奴釋矣。」由是豪右屏迹。改寶坻丞。補尚書省令史,除太學助教、監察御史。劾奏平章政事僕散揆。或曰:「與宰相有隙,奈何?」伯嘉曰:「職分如此。」遷平涼治中。累官莒州刺史。讞屬縣盜,伯嘉曰:「飢寒為盜,得錢二千,經月不使一錢云何?此必官兵捕他盜不獲,誣以準罪耳。」詰之,果然。詔與按察官俱推排物力,召見于香閣。



 大安中,三遷同知西京留守,權本路安撫使。貞祐初,遷順義軍節度使。居父母喪,卒哭,起復震武
 軍節度使兼宣撫副使,提控太和嶺諸隘。副統李鵬飛誣殺彰國軍節度使牙改,詔伯嘉治之。貞祐四年三月,伯嘉奏:「西京副統程琢智勇過人,持心忠孝,以私財募集壯士二萬,復取渾源、白登,有恢復山西之志,已命駐于弘州矣。近者靖大中、完顏毛吉打以三千人歸國,各遷節度副使。今山西已不守,琢收合餘眾,盡忠於國,百戰不挫。臣恐失機會,輒擬琢昭勇大將軍,同知西京留守事,兼領一路義軍,給以空名敕二十道,許擇有謀略者充州縣。」制可,仍賜琢姓夾谷氏。琢請曰:「前代皆賜國姓,不繫他族,如蒙更賜,榮莫大焉。」詔更賜完顏氏。



 是月,
 伯嘉遷元帥左監軍,知太原府事,河東北路宣撫使。以同知太原府斡勒合打為彰國軍節度使、宣撫副使。六月,斡勒合打奏:「同知西京留守完顏琢恃與宣撫使伯嘉雅善,徙居代州,肆為侵掠。遙授太原治中,權堅州刺史完顏斜烈私離邊面,臣白伯嘉,伯嘉不悅,遣臣護送糧運于代州。臣請益兵,乃以羸卒數百見付,半無鎧仗。臣復為言,伯嘉怒臣,榜掠幾死。臣立功累年,頗有寸效,伯嘉挾私陵轢,無復宣撫同僚之禮。臣欲不言,恐他日反為所誣,無以自明。」上問宰臣,奏曰:「太原重鎮,防秋在邇,請敕諭和解。」詔曰:「太原兵衝,若以私忿廢國事,國家
 何賴焉!卿等同心戮力,以分北顧之憂,無執前非,誤大計也。」七月,伯嘉改知歸德府事,合打改武寧軍節度使。御史臺奏:「宣撫副使合打訴元帥伯嘉以私忿加箠楚,令本臺廉問,既得其事,遂不復窮治。若合打奏實,伯嘉安得無罪,伯嘉無罪,合打合坐欺罔,乞審正是非,明示黜陟。」宣宗曰:「今正防秋,且已。」



 初,河東行省胥鼎奏:「完顏伯嘉屢言同知西京留守兼臺州刺史完顏琢,可倚之以復山西,朝廷遷官賜姓,令屯代北,扼太和嶺。今聞諸隘悉無琢兵,蓋琢挈太原之眾,保五臺剽掠耳。如尚以伯嘉之言為可信,乞遣琢出太原,或徙之內地,分處其
 眾,以備不測之變。」宰臣奏:「已遣官體究琢軍,且令太原元帥府烏古論德升召琢使之矣。當以此意報鼎。」無何,德升奏:「琢兵數萬分屯代州諸險,拒戰甚力,其眾烏合,非琢不可制。」胥鼎復奏:「宣差提控古里甲石倫言,琢方招降人,謀復山西,盤桓于忻、代、定、襄間,恣為侵擾,無復行意。發掘民粟,並且。戕殺無辜,雖曰不煩官廩,博易為名,實則攘劫,欺國害民無如琢者。石倫之言如此,臣已令帥府禁止之矣。」宰臣奏:「所遣官自忻、代來,云不見劫掠之迹,惟如德升言便。」從之。



 伯嘉至歸德,上言,乞雜犯死罪以下納粟贖免。宰臣奏:「伯嘉前在代州嘗行之,蓋一時
 之權,不可為常法。」遂寢。俄改簽樞密院事。未閱月,改知河南府事。是時,甫經兵後,乏兵食,伯嘉令輸棗栗菜根足之,皆以為便。興定元年,知河中府,充宣差都提控,未幾召為吏部尚書。二年,改御史中丞。



 初,貞祐四年十月,詔以兵部尚書、簽樞密院事蒲察阿里不孫為右副元帥,備禦潼關、陜州。次澠池土濠村,兵不戰而潰。阿里不孫逸去,亡所佩虎符,變易姓名,匿柘城縣,與其妻妹前韓州刺史合喜男婦紇石烈氏及僕婢三人僦民舍居止。合喜母徒單氏聞之,捕執紇石烈,斷其髮,拘之佛寺中。阿里不孫復亡去。監察御史完顏藥師劾奏:「乞就詰
 紇石烈及僕婢,當得所在。其妻子見在京師,亦無容不知,請窮治。」有司方繫其家人,特命釋之,詔曰:「阿里不孫若能自出,當免極罪。」阿里不孫乃使其子上書,請圖後效。尚書省奏:「阿里不孫幸特赦死,當詣闕自陳,乃令其子上書,猶懷顧望。」伯嘉劾之曰:「古之為將者,受命之日忘其家,臨陣之日忘其身,服喪衣、鑿凶門而出,以示必死。進不求名,退不避罪,惟民是保。阿里不孫膺國重寄,握兵數萬,未陣而潰,委棄虎符,既不得援枹鼓以死敵,又不能負斧鑕而請罪,逃命竄伏,猥居里巷,挾匿婦人,為此醜行。聖恩寬大,曲赦其死,自當奔走闕庭,皇恐待
 命。安坐要君,略無忌憚,迹其情罪,實不容誅。此而不懲,朝綱廢矣。乞尸諸市以戒為臣之不忠者!」宣宗曰:「中丞言是,業已赦之矣。」阿里不孫乃除名。



 五月,充宣差河南提控捕蝗,許決四品以下。宣宗憂旱。伯嘉奏曰:「日者君之象,陽之精,旱乃人君自用亢極之象,宰執以為冤獄所致。夫燮和陰陽,宰相之職,而猥歸咎於有司。高琪武弁出身,固不足論,汝礪輩不知所職,其罪大矣。漢制,災異策免三公,顧歸之有司邪。臣謂今日之旱,聖主自用,宰相諂諛,百司失職,實此之由。」高琪、汝礪深怨之。禮部郎中抹捻胡魯剌以言事忤旨,集五品以上官顯責
 之。明日,伯嘉諫曰:「自古帝王莫不欲法堯、舜而恥為桀、紂,蓋堯、舜納諫,桀、紂拒諫也。故曰:『納諫者昌,拒諫者亡』。胡魯剌所言是,無益於身,所言不是,無損於國。陛下廷辱如此,獨不欲為堯、舜乎?近日言事者語涉謗訕,有司當以重典,陛下釋之。與其釋之以為恩,曷若置之而不問。」宰相請脩山寨以避兵,伯嘉諫曰:「建議者必曰據險可以安君父,獨不見陳後主之入井乎?假令入山寨可以得生,能復為國乎?人臣有忠國者,有媚君者,忠國者或拂君意,媚君者不為國謀。臣竊論之,有國可以有君,有君未必有國也。」高琪、汝礪聞之,怒愈甚。



 十二月,以御
 史中丞、權參知政事,元帥左監軍,行尚書省、元帥府于河中,控制河東南北路便宜從事。興定三年,伯嘉至河中,奏曰:「本路衝要,不可闕官,凡召辟者每以艱險為辭。乞凡檄召無故不至者宜令降罰,悉心幹當者視所歷升遷。」詔召不至者決杖一百,餘如所請。廷議欲棄河東,其民以實陜西。伯嘉上書諫曰:「中原之有河東,如人之有肩背。古人云『不得河東不雄』,萬一失之,恐未易取也。」大忤宰執意。



 頃之,召還,罷為中丞。伯嘉入見,奏曰:「如臣駑鈍,固宜召還,更須速遣大臣鎮撫。」宣宗深然之。伯嘉上疏曰:「國家兵不強,力不足以有為,財不富,賞不足
 以周眾,獨恃官爵以激勸人心。近日以功遷官赴都求調者,有司往往駮之,冒濫者固十之?,既與而復奪之,非所以勸功也。乞應軍功遷官,宣敕無偽者即準用之。」又曰:「自兵興以來,河北桀黠往往聚眾自保,未有定屬。乞賜招撫,署以職名,無為他人所先。」又曰:「河東、河北有能招集餘民完守城寨者,乞無問其門地,皆超踰等級,授以本處見任之職。」又曰:「河中、晉安被山帶河,保障關、陜,此必爭之地。今雖殘破,形勢猶存,若使他人據之,因鹽池之饒,聚兵積糧,則河津以南,太行以西,皆不足恃矣。」



 四年秋,河南大水,充宣慰副使,按行京東。奏曰:「亳州
 災最甚,合免三十餘萬石。三司止奏除十萬石,民將重困,惟陛下憐之。」詔治三司奏災不以實罪。伯嘉行至蘄縣,聞前有紅襖賊,不敢至泗州。監察御史烏古孫奴申劾伯嘉違詔,不遍按視。又曰:「伯嘉知永城縣主簿蒙古訛里刺不法,沈丘令夾谷陶也受賄,匿而不發。前穀城縣令獨吉鼎術可嘗受業伯嘉,伯嘉諷御史辟之。」詔有司鞫問,會赦免。



 五年,起為彰化軍節度使,改翰林侍講學士。伯嘉純直,不能與時低昂,嘗曰:「生為男子,當益國澤民,其他不可學也。」高汝礪方希寵固位,伯嘉論事輒與之忤,由是毀之者眾。元光元年,坐言事過切,降遙授
 同知歸德府事。二年三月,遙授集慶軍節度使,權參知政事,行尚書省于河中,率陜西精銳與平陽公史詠共復河東。頃之,伯嘉有疾。六月,薨。



 伯嘉去太原後,完顏琢寓軍平定石仁寨,權平定州刺史范鐸以閻德用充本州提控。德用桀驁,蓄姦謀,鐸不能制,委曲容庇之。興定元年,德用率所部掩襲,殺琢及官屬程珪等百餘人,遂據石仁寨。鐸懼,挈家奔太原。德用遂據平定州。二年十月,詔誅范鐸。



 術虎筠壽,貞祐間為器物局直長,遷副使。貞祐三年七月,工部下開封市白牯取皮治御用鞠仗。筠壽以其家
 所有鞠仗以進,因奏曰:「中都食盡,遠棄廟社,陛下當坐薪懸膽之日,奈何以球鞠細物動搖民間,使屠宰耕牛以供不急之用,非所以示百姓也。」宣宗不懌,擲仗籠中。明日,出筠壽為橋西提控。



 贊曰:孟鑄、宗端脩、路鐸盡言於章宗,皆擯斥不遂。鑄劾胡沙虎,可謂先知,雖行其言,弗究厥罰。厥後胡沙虎逆謀,胥持國終至于誤國,而不悟也。宣宗時,完顏素蘭、許古皆敢言者,亦挫于高琪、汝礪之手。匱土不能塞河決,有以也夫!完顏伯嘉以著功參大政,亦不能一朝而安,言之難也如是哉!術虎筠壽,所謂執藝事以諫者邪。



 張煒,字子明,洺州永年人,本名燝,避章宗嫌名改焉。大定二十五年進士,調葭州軍事判官,再遷中都左警巡使。煒喜言功利,寡廉節,交通部民閻元翬,縉紳薄之。累官部員外郎。



 承安五年,天色久陰晦,平章政事張萬公奏:「此由君子小人邪正不分所致,君子宜在內,小人宜在外。」章宗問:「孰為小人?」萬公對曰:「戶部員外郎張煒、文繡署丞田櫟、都水監丞張嘉貞雖有幹才,無德而稱,好奔走以取勢利。大抵論人當先德後才。」詔三人皆與外除,煒出為同知鎮西軍節度使事,轉同知西京轉運使事。是時,大築界牆,被行戶工部牒主役事。丁母憂,起
 復桓州刺史,奏請以鹽易米事,且所言利害甚多,恐涉細碎,不敢盡上。詔尚書省曰:「張煒通曉人也,朕不敢縷詰,卿等詳問之,毋為虛文。」充宣差西北路軍儲,自言斂不及民,可以足用。大抵募商賈縱其販易,不問所從來。姦人往往投牒,妄指產業,疏鄰保姓名,煒信之,多與之錢。已而亡去,即逮繫鄰保,使之代償,一路為之疲弊。以故舊氈罽繒絮皮革折給軍士,皆棄於道而去。歲餘,改戶部郎中,遷翰林直學士,俱兼規措職事。左丞相宗浩奏:「張煒長於恢辦,比戶部給錢三十萬,已增息十四萬矣。請給錢通百萬,今從長恢辦,乞不隸省部,委臣專
 一提控,有應奏者,許煒專達,歲差幹事官計本息具奏。」上從其請。



 泰和六年,伐宋,煒進銀五千兩。詔曰:「汝幹集資儲,固其職也,毋令軍士有議國家。人之短汝,朕皆知之,惟能興利,斯惟汝功。」自西北路召還,勾計諸道倉庫,除簽三司事。上問:「誰可代卿規措者?」煒舉中都轉運戶籍判官王謙。謙至西北路,盡發煒前後散失錢物以巨萬計,對獄者積年。大安三年,起為同簽三司事。會河堡兵敗,軍士猶去張宣差刻我,欲倒戈殺之。累遷戶部侍郎。貞祐初。遷河北西路按察轉運使。



 貞祐二年春,中都乏糧,詔同知都轉運使事。邊源以兵萬人護運通州積粟,
 軍敗死焉,平章政事高琪舉煒代源行六部事。以勞進官一階,改河北東路轉運使。宣宗遷汴,佐尚書右丞胥鼎前路排頓,及脩南京宮闕。無何,坐事降孟州防禦使。三年,遷安國軍節度使。致仕。宣宗初以煒有才,既察其無實,遂不復用。貞祐四年卒。



 高竑,渤海人。以廕補官,累調貴德縣尉。提刑司舉任繁劇,遷奉聖州錄事。察廉,遷內黃令,累官左藏庫副使。元妃李氏以皁弊易紅幣,竑獨拒不肯易。元妃奏之。章宗大喜,遣人諭之曰:「所執甚善。今姑與之,後不得為例。」轉儀鸞局、少府少監,改戶部員外郎、安州刺史。大安中,越
 王永功判中山,竑以王傅同知府事。改同知河南府,充安撫使。徙同知大名府,兼本路安撫使。貞祐二年,遷河北西路按察轉運使,錄大名功,遷三官,致仕。興定四年,卒。



 李復亨,字仲脩,榮州河津人。年十八,登進士第。復中書判優等,調臨晉主簿。護送官馬入府,宿逆旅,有盜殺馬,復亨曰:「不利而殺之,必有仇者。」盡索逆旅商人過客。同邑人橐中盛佩刀,謂之曰:「刀蔑馬血,火煆之則刃青。」其人款服,果有仇。以提刑薦遷南和令。盜割民家牛耳。復亨盡召里中人至,使牛家牽牛遍過之,至一人前,牛忽
 驚躍,詰之,乃引伏。察廉,遷臨洮府判官,改陜西東路戶籍判官,轉河東北路支度判官。



 泰和中,伐宋,充宣撫司經歷官,遷解鹽副使,歷保大、震武同知節度事。丁母憂,起復同知震武節度,加遙授忻州刺史。貞祐間,歷左司員外郎、郎中,遷翰林直學士行三司事。興定三年,上言:「近日興師伐宋,恐宋人乘虛掩襲南鄙,故籍邊郡民為軍。今大軍已還,乞罷遣歸本業。」從之。復亨舉陳留縣令程震等二十九人農桑有效,徵科均一,朝廷皆遷擢之。



 是歲七月,置京東、京西、京南三路行三司,掌勸農催租、軍須科差及鹽鐵酒榷等事,戶部侍郎張師魯攝東路,治
 歸德,戶部侍郎完顏麻斤出攝南路,治許州,復亨攝西路,治中京實河南府,三司使侯摯總之。復亨奏:「民間銷毀農具以供軍器,臣竊以為未便。汝州魯山、寶豐,鄧州南陽皆產鐵,募工置冶,可以獲利,且不厲民。」又奏:「陽武設賣鹽官以佐軍用,乞禁止滄,濱鹽勿令過河,河南食陽武、解鹽,河北食滄、濱鹽,南北俱濟。」詔尚書省行之。九月,以勸農有勞,遷兵部尚書。再閱月,轉吏部尚書,權參知政事。四年三月,真拜參知政事,兼修國史。



 七月,河南雨水害稼,復亨為宣慰使,御史中丞完顏伯嘉副之,循行郡縣,凡官吏貪汙不治者,得廢罷推治。復亨奏乞禁宣
 慰司官吏不得與州府司縣行總管府及管軍官會飲。又奏曰:「詔書令臣,民間差發可免者免之。民養驛馬,此役最甚,使者求索百端,皆出養馬之家,人多逃竄,職此之由。可依舊設回馬官,使者食料皆官給之,歲終會計,均賦於民。」又奏:「河南閑田多,可招河東、河北移民耕種。被災及沿邊郡縣租稅全免,內地半之,以救塗炭之民,資蓄積之用。」詔有司議行焉。還奏:「南陽禾麥雖傷,土性宜稻,今因久雨,乃更滋茂。田凡五百餘頃,畝可收五石,都得二十五萬餘石。可增直糴稻給唐、鄧軍食。緣詔書不急科役即令免罷,臣不敢輒行,如以臣言為然,乞付
 有司計之。」制可。無何,被詔提控軍興糧草。復亨奏:「河渡不通,陜西鹽價踴貴,乞以粟互易足兵食。」詔戶部從長規措。



 復亨有會計才,號能吏,當時推服,故驟至通顯。既執政,頗矜持,以私自營,譽望頓減。五年三月,廷試進士,復亨監試。進士盧元謬誤,濫放及第。讀卷官禮部尚書趙秉文、翰林待制崔禧、歸德治中時戩、應奉翰林文字程嘉善當奪三官降職,復亨當奪兩官。趙秉文嘗請致仕,宣宗憐其老,降兩階,以禮部尚書致仕。復亨罷為定國軍節度使。元光元年十一月,城破自殺,年四十六。贈資德大夫、知河中府事。



 贊曰:大凡兵興則財用不足,是故張煒、李復亨乘時射利,聚斂為功。大安,軍士欲倒戈殺煒。復亨宣慰南陽,還奏稻熟可糴。所謂聚斂之臣者,二子之謂矣。高竑之守藏,君子頗有取焉。



\end{pinyinscope}