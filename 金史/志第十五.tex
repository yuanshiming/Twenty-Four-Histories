\article{志第十五}

\begin{pinyinscope}

 禮七



 ○社稷



 貞
 元元年閏十二月,有司奏建社稷壇于上京。大定七年七月,又奏建壇于中都。社為制,其外四周為垣,南向開一神門,門三間。內又四周為垣,東西南北各開一神門,門三間,各列二十四戟。四隅連飾罘罳,無屋,於中稍南為壇位,令三方廣闊,一級四陛。以五色土各飾其方,中央覆以黃土,其廣五丈,高五尺。其主用白石,下廣二
 尺,剡其上,形如鐘,埋其半,壇南,栽栗以表之。近西為稷壇,如社壇之制而無石主。四濆門各五間,兩塾三門,門列十二戟。濆有角樓,樓之面皆隨方色飾之。饌幔四楹,在北濆門西,北向。神廚在西濆門外,南向。廨在南圍墻內,東西向。有望祭堂三楹,在其北,雨則於是堂望拜。堂之南北各為屋二楹,三獻及司徒致齋幕次也。堂下南北相向有齋舍二十楹。外門止一間,不施鴟尾。



 祭用春秋二仲月上戊日,樂用登歌,遣官行事。太尉一,司徒一,已上奏差。亞獻太常卿一,終獻光祿卿一,省差。太常卿一,光祿卿一,郊社令一,學士院官一,請御署祝版。大樂
 令一,太官令二,監祭御史二,太常博士二,廩犧令一,奉禮郎一,協律郎二,司尊罍二,奉爵酒官一,太祝七,祝史四,盥洗官二,爵洗官二,執巾篚官四,齋郎四十八,贊者一,禮直官十,已上部差。守衛十二人,各衣其方色,其服官給。舉瘞四,衣皂,軍人內差,其衣自備。前三日質明,行事官受誓戒於尚書省、御史臺,太常寺引眾官就位,禮直官贊:「揖。」對揖,訖,太尉誓曰:「某月某日上戊,祭于太社,各揚爾職。不恭其事,國有常刑。」讀訖,對拜,訖,退。凡與祭官散齋二日,致齋一日,已齋而闕者通攝行事,仍習禮於社宮。諸衛令率其屬,各以其方器服守衛社宮門。大
 樂工人俱清齋一宿。前三日,陳設局設祭官公卿已下次於齋房之內。及設饌幔四於社宮西神門之外,門南,西向。前二日,郊社令率其屬,掃除壇之上下。大樂令設樂於壇上。郊社令為瘞坎二於壬地,方深取足容物,南出陛。又設望瘞位於坎之北,南向。前一日,奉禮郎帥禮直官,設祭官分卿以下褥位於西神門之內道南,執事官於道北,每等異位,俱重行,東向,南上。設御史位二於壇下,一在太社東北,西向,一在太稷西北,東向,博士各在其北。設奉禮郎位於稷壇上西北,贊者一在北,東向。設協律郎位二於壇上東北隅,俱西向。設大樂令位於
 兩壇之間,南向。設獻官褥位於逐壇上神座前。設省牲位於西神門外。設牲榜於當門,黝牲二居前,又黝牲二少退二牲皆用黝,北上。設廩犧令位於牲東北,南向。設諸太祝位於牲西,各當牲後,祝史陪其後,具東向。設太常卿省牲位於前近南,北向。又設御史位於太常卿之東,北向。太常卿帥其屬,設酒樽之位。太樽二、著樽二、犧樽二、山罍二在壇上北隅,南向。象樽二、壺樽二、山罍二在壇下北陛之西,南向。后土氏象樽二、著樽二、山罍二在太社酒樽之西,俱東南上。設太稷、后稷酒樽於壇之上下,如太社、后土之儀。設洗位二於社壇西北,南向。罍在洗
 東,篚在洗西,北肆。司樽罍篚冪者,各位於其後。設玉帛之篚於壇上樽坫之所。設四座,各籩十、豆十、簠二、簋二、鉶三、槃一、俎三、坫四,內籩一、豆一、簠一、簋一、俎三各設於饌幔內。光祿卿率其屬,入實。籩之實,魚鱐、乾棗、形鹽、鹿脯、榛實、乾裛、桃、菱、芡、慄,以序為次。豆之實,芹菹、筍菹、葵菹、菁菹、韭菹、魚醢、兔醢、豚拍、鹿臡、醓醢以序為次。鉶實以羹,加芼滑。簠實以稻、粱、簋實以黍、稷、粱在稻前,稷在黍前。太官令入實樽罍以酒,各一樽實以玄酒。



 祭日未明五刻,郊社令升設太社太稷神座,各於壇上近南,北向。設后土氏神座於太社神座之左,后稷氏神座於
 太稷神座之左,俱東向。席皆以莞,加裀褥如幣之色。神位版各於座首。前一日,諸衛之屬禁斷行人。郊社令與其屬,以樽坫罍洗篚冪入設於位,司樽罍奉禮郎及執事者升自太社壇西陛以俟。其省牲器、視滌溉,並如郊廟儀。祭日未明十刻,太官令率宰人以鸞刀割牲,祝史以豆取毛血,各置於饌所,以盤取血置神座前,遂烹牲。未明三刻,諸祭官各服其服。郊社令、太官令入實玉幣樽罍。太官令帥進饌者實諸籩豆簠簋。未明一刻,奉禮郎、贊者先入就位。禮直官引光祿卿、御史、博士、諸太祝、祝史、司樽罍篚冪者入自西門,當太社壇北,重行南向
 東上立定,奉禮曰:「再拜。」贊者承傳,御史以下皆再拜,訖,司樽罍篚冪者皆就位。奉盤血祝史與太祝由西陛升壇,各於樽所立,祝史以俟瘞血,太祝以次取玉幣。大樂令帥工人入。禮直官各引祭官入,就位立定,奉禮曰:「眾官再拜。」贊者曰:「在位者皆再拜。」其先拜者不拜。禮直官進太尉之左曰:「有司謹具,請行事。」退復位。禮直官引光祿卿就瘞血所,又引祝史奉盤血降自西陛,至瘞位,光祿卿瘞血,訖,復位。祝史以盤還饌幔,以俟奉毛血豆。奉禮曰:「眾官再拜。」在位者皆再拜。諸太祝取玉幣於篚,各立於尊所。禮直官引太尉詣盥洗位。協律郎跪,俯伏,舉
 麾,樂作太簇宮《正寧之曲》。後盥洗同。至洗位南向立,樂止。搢笏,盥手、帨手訖,詣太社壇,樂作應鐘宮《嘉寧之曲》。後升壇同。升自北陛,樂止,南向立。太祝以玉帛西向授太尉,太尉受玉帛。禮神之玉奠於神前,瘞玉加於幣,配位不用玉。玉用兩圭有邸。盛以匣。瘞玉以玉石為之。帛用黑繒,長一丈八尺。樂作太簇宮《嘉寧之曲》,太稷同。禮直官引太尉進,南向跪奠於太社座前,俯伏,興。引太尉少退,詣褥位南向再拜。太祝以幣授太尉,太尉受幣,西向跪奠於后土神座前,俯伏,興。禮直官引太尉少退,西向再拜,訖,樂止。禮直官引太尉降自北陛,詣太稷壇,盥
 洗、升奠玉幣如太社后土之儀。祝史奉毛血入,各由其陛升,毛血豆係別置一豆。諸太祝迎取於壇上,俱進奠於神座前,祝史退立於樽所。太尉既升奠玉幣,太官令出帥進饌者,奉饌陳於西門外。禮直官引司徒出詣饌所,司徒奉太社之俎。諸太祝既奠毛血,禮直官太官令引太社太稷之饌入自正門,配座之饌入自左闥。饌初入門,樂作太簇宮《正寧之曲》,饌至陛,樂止。祝史俱進徹毛血豆,降自西陛以出。太社太稷之饌升自北陛,配座之饌升自西陛,諸太祝迎引於壇上,各於神座前設訖,禮直官引司徒已下降自西陛,樂作,復位,樂止。諸太祝
 還樽所。禮直官引太尉詣罍洗位,樂作,至位,樂止。盥手、洗爵訖,禮直官引太尉詣太社壇,升自北陛,樂作,至太社酒樽所,樂止。執樽者舉冪,執事者以爵授太尉,太尉執爵,太官令酌酒,訖,樂作太簇宮《阜寧之曲》。太稷同。太尉以爵授執事者。禮直官引太尉詣太社神座前,執事者以爵授太尉,南向跪奠爵,訖,以爵授執事者,俯伏,興。太尉少退,樂止。讀祝官與捧祝官進於神座前右,西向跪讀祝,讀訖,讀祝官就一拜,各還樽所。太尉拜訖,詣配位酒樽所。執事者舉冪,執事者以爵授太尉,太尉執爵,太官令酌酒,訖,樂作太簇宮《昭寧之曲》。太尉以爵授執事者。
 禮直官引太尉進后土神座前,執事者以爵授太尉,西向跪奠爵,訖,以爵授執事者,俯伏,興。太尉少退,樂止。讀祝如上儀。太尉再拜,訖,禮直官引太尉降自北陛,樂作,至罍洗位,樂止。盥手、洗爵訖,禮直官引太尉詣太稷壇,升自北陛,並如太社后土之儀,樂曲同。訖,禮直官引太尉還本位。



 亞、終獻,盥洗升獻並如太尉之儀。禮直官引終獻降復位,樂止。太祝各進徹豆,樂作應鐘宮《娛寧之曲》,還樽所,樂止。徹者籩豆各一,少移於故處。奉禮曰:「賜胙。」贊者曰:「眾官再拜。」在位者皆再拜。禮直官進太尉之右,請就望瘞位,御史博士從,南向立。於眾官將拜之前,
 太祝執篚進於神座前取玉幣,齋郎以俎載牲體、稷黍飯、爵酒體謂牲之左髀,各由其陛降壇,以玉幣饌物置於坎,訖,奉禮曰:「可瘞。」坎東西二人置土半坎,訖,禮直官進太尉之左曰:「禮畢。」遂引太尉出,祭官以下以次出。禮直官引御史博士以下俱復執事位,立定。奉禮曰:「再拜。」御史以下皆再拜,訖,出。工人以次出。祝版燔於齋坊。光祿卿以胙奉進,御史就位展視,光祿卿望闕再拜,乃退。其州郡祭享,一遵唐、宋舊儀。



 ○風雨雷師



 明昌五年,禮官言:「國之大事,莫重於祭。王者奉神靈,祈福祐,皆為民也。我國家自祖廟禘祫五享外,
 惟社稷、嶽鎮海瀆定為常祀,而天地日月風雨雷師其禮尚闕,宜詔有司講定儀注以聞。」尚書省奏:「天地日月,或親祀或令有司攝事。若風雨雷師乃中祀,合令有司攝之。且又州縣之所通祀者也,合先舉行。」制可。乃為壇於景豐門外東南,闕之巽地,歲以立春後丑日,以祀風師。牲、幣、進熟,如中祀儀。又為壇於端禮門外西南,闕之坤地,以立夏後申日以祀雨師,其儀如中祀,羊豕各一。是日,祭雷師於位下,禮同小祀,一獻,羊一,無豕。其祝稱:「天子謹遣臣某」云。



 ○嶽鎮海瀆



 大定四年,禮官言:「嶽鎮海瀆,當以五郊迎氣
 日祭之。」詔依典禮以四立、土王日就本廟致祭,其在他界者遙祀。立春,祭東嶽於泰安州、東鎮于益都府、東海於萊州、東瀆大淮于唐州。立夏,望祭南嶽衡山、南鎮會稽山于河南府,南海、南瀆大江于萊州。季夏土王日,祭中嶽于河南府、中鎮霍山于平陽府。立秋,祭西嶽華山於華州、西鎮吳山于隴州,望祭西海、西瀆于河中府。立冬,祭北嶽恒山于定州、北鎮醫巫閭山于廣寧府,望祭北海、北瀆大濟于孟州。其封爵並仍唐、宋之舊。明昌間,從沂山道士楊道全請,封沂山為東安王,吳山為成德王,霍山為應靈王,會稽山為永興王,醫巫閭山為廣寧
 王,淮為長源王,江為會源王,河為顯聖靈源王,濟為清源王。



 每歲遣使奉御署祝版奩薌,乘驛詣所在,率郡邑長貳官行事。禮用三獻。讀祝官一、捧祝官二。盥洗官二、爵洗官二、奉爵官一、司尊彝一、禮直官四,以州府司吏充。前三日,應行事執事官散齋二日,治事如故,宿於正寢,如常儀,前二日,有司設行事執事官次於廟門外。掌廟者掃除廟之內外。前一日,有司牽牲詣祠所,享官以下常服閱饌物,視牲充腯。享日丑前五刻,執事者設祝版於神位之右,置於玷,及以血豆設於饌所。次設祭器,皆藉以席,掌饌者實之。左十籩為三行,以右為上,實以
 乾裛、乾棗、形鹽、魚鱐、鹿脯、榛實、乾桃、菱、芡、慄。右十豆為三行,以左為上,實以芹菹、筍菹、韭菹、葵菹、菁菹、魚醢、兔醢、豚拍。鹿臡、醓醢。左簠二,實以粱、稻。右簋二,實以稷、黍。俎二,實以牲體。次設犧樽二、象樽二,在堂上東南隅,北向西上。犧樽在前,實以法酒。犧樽,初獻官酌。象樽,亞、終獻酌。又設太樽一、山樽一,在神位前,設而不酌。有司設燭於神位前。洗二,在東階之下,直東霤北向,罍在洗東,加勺。篚在洗西,南肆,實以巾。執罍篚者位於其後。又設揖位於廟門外,初獻在西,東向,亞、終及祝在東,南向,北上。開瘞坎於廟內廷之壬地。享日丑前五刻,執事官各
 就次。掌饌者帥其屬,實饌具畢。凡祭官各服其服,與執事官行止皆贊者引,點視陳設訖,退就次。引初獻以下詣廟南門外揖位,立定,贊禮者贊:「揖。」次引祝升堂就位立。次引初獻詣盥洗位北向立,搢笏、盥手、帨手、執笏,詣爵洗位北向立,搢笏,洗爵,以爵授執事者,執笏,升堂,詣酌樽所西向立。執事者以爵授初獻。初獻搢笏執爵,執樽者舉冪,執事者酌酒。初獻以爵授執事者,執笏,詣神位前北向立,搢笏,跪,執事者以爵授初獻。初獻執爵三祭酒,奠爵訖,執笏,俯伏,興,少立。次引祝詣神位前東向立。搢笏,跪,讀祝,訖,執笏、興、退復位。初獻再拜,贊禮者引
 初獻復位。次引亞獻酌獻,並如初獻之儀,次引終獻,並如亞獻之儀。



 贊者引初獻官詣神位前北向立,執事者以爵酌清酒,進初獻之右,初獻跪,祭酒,啐酒,奠爵。執事者以俎進,減神座前胙肉前腳第二節,共置一俎上,以授初獻,初獻以授執事者。初獻取爵,遂飲,卒爵,執事者進受爵,復於坫。初獻興,再拜,贊者引初獻復位。贊者曰:「再拜。」已飲福、受胙者不拜。亞獻官以下皆再拜,拜訖,次引初獻已下就望瘞位,以饌物置於坎,東西廂各二人,贊者曰:「可瘞。」置土關坎,又曰:「禮畢。」遂引初獻官已下出。祝與執樽罍篚冪者俱復位立定,贊者曰:「再拜。」再拜訖,遂出。
 祝版燔於齋所。



\end{pinyinscope}