\article{本紀第五}

\begin{pinyinscope}

 海陵



 廢帝海陵庶人亮,字元功,本諱迪古乃,遼王宗乾第二子也。母大氏。天輔六年壬寅歲生。天眷三年,年十八,以宗室子為奉國上將軍,赴梁王宗弼軍前任使,以為行軍萬戶,遷驃騎上將軍。皇統四年,加龍虎衛上將軍,為中京留守,遷光祿大夫。為人僄急,多猜忌,殘忍任數。初,熙宗以太祖嫡孫嗣位,亮意以為宗乾太祖長子,而己
 亦太祖孫,遂懷覬覦。在中京,專務立威,以厭伏小人。猛安蕭裕傾險敢決,亮結納之,每與論天下事。裕揣知其意,因勸海陵舉大事,語在《裕傳》。



 七年五月,召為同判大宗正事,加特進。十一月,拜尚書左丞,務攬持權柄,用其腹心為省臺要職,引蕭裕為兵部侍郎。一日因召對,語及太祖創業艱難,亮因嗚咽流涕,熙宗以為忠。八年六月,拜平章政事。十一月,拜右丞相。



 九年正月,兼都元帥。熙宗使小底大興國賜亮生日,悼后亦附賜禮物,熙宗不悅,杖興國百,追其賜物,海陵由此不自安。三月,拜太保、領三省事,益邀求人譽,引用勢望子孫,結其歡心。四
 月,學士張鈞草詔忤旨死,熙宗問:「誰使為之?」左丞相宗賢對曰:「太保實然。」熙宗不悅,遂出為領行臺尚書省事。過中京,與蕭裕定約而去。至良鄉,召還。海陵莫測所以召還之意,大恐。既至,復為平章政事,由是益危迫。



 熙宗嘗以事杖左丞唐括辯及右丞相秉德,辯及與大理卿烏帶謀廢立,而烏帶先此謀告海陵。他日,海陵與辯語及廢立事,曰:「若舉大事,誰可立者?」辯曰:「胙王常勝乎?」問其次,曰:「鄧王子阿懶。」亮曰:「阿懶屬疏,安得立?」辯曰:「公豈有意邪?」海陵曰:「果不得已,捨我其誰!」於是旦夕相與密謀。護衛將軍特思疑之,以告悼后曰:「辯等公餘每竊
 竊聚語,竊疑之。」后以告熙宗。熙宗怒,召辯謂曰:「爾與亮謀何事,將如我何?」杖之。亮因此忌常勝、阿懶,且惡特思。因河南兵士孫進自稱皇弟按察大王,而熙宗之弟止有常勝、查剌,海陵乘此構常勝、查剌、阿懶、達懶。熙宗使特思鞫之,無狀。海陵曰:「特思鞫不以實。」遂俱殺之。護衛十人長僕散忽土舊受宗乾恩。徒單阿里出虎與海陵姻家。大興國給事寢殿,時時乘夜從主者取符鑰歸家,以為常。興國嘗以李老僧屬海陵,得為尚書省令史,故使老僧結興國為內應,而興國亦以被杖怨熙宗,遂與亮約。



 十二月丁巳,忽土、阿里出虎內直。是夜,興國取符
 鑰啟門納海陵、秉德、辯、烏帶、徒單貞、李老僧等人至寢殿,遂弒熙宗。秉德等未有所屬。忽土曰:「始者議立平章,今復何疑。」乃奉海陵坐,皆拜,稱萬歲。詐以熙宗欲議立後,召大臣,遂殺曹國王宗敏,左丞相宗賢。是日,以秉德為左丞相兼侍中、左副元帥,辯為右丞相兼中書令,烏帶為平章政事,忽土為左副點檢,阿里出虎為右副點檢,貞為左衛將軍,興國為廣寧尹。於是自太師、領三省事勖以下二十人進爵增職各有差。己未,大赦。改皇統九年為天德元年。參知政事蕭肄除名。鎮南統軍孛極為尚書左丞。賜左丞相秉德、右丞相辯、平章政事烏帶、
 廣寧尹興國、點檢忽土、阿里出虎、左衛將軍貞、尚書省令史老僧、辯父刑部尚書阿里等錢絹馬牛羊有差。甲子,誓太祖廟,召秉德、辯、烏帶、忽土、阿里出虎、興國六人賜誓券。丙寅,以燕京路都轉運使劉麟為參知政事。癸酉,太傳、領三省事蕭仲恭,尚書右丞稟罷。以行臺尚書左丞溫都思忠為右丞。乙亥,追謚皇考太師憲古弘道文昭武烈章孝睿明皇帝,廟號德宗,名其故居曰興聖宮。宋、高麗、夏賀正旦使中道遣還。



 二年正月辛巳,以同知中京留守事蕭裕為秘書監。癸巳,尊嫡母徒單氏及母大氏皆為皇太后。名徒單氏宮
 曰永壽,大氏宮曰永寧。乙巳,以勵官守、務農時、慎刑罰、揚側陋、恤窮民、節財用、審才實七事詔中外。遣侍衛親步軍都指揮使完顏思恭等以廢立事報諭宋、高麗、夏國。以左丞相兼左副元帥秉德領行臺尚書省事。



 二月戊申朔,封子元壽為崇王。庚戌,降前帝為東昏王。給天水郡公孫女二人月俸。甲子,以兵部尚書完顏元宜等充賀宋生日使。戊辰,群臣上尊號曰法天膺運睿武宣文大明聖孝皇帝,詔中外。永壽、永寧兩太后父祖贈官有差。以右丞相唐括辯為左丞相,平章政事烏帶為右丞相。三月丙戌,宋、高麗遣使賀即位。以弟袞為司徒
 兼都元帥。詔以天水郡王玉帶歸宋。四月戊午,殺太傅、領三省事宗本,尚書左丞相唐括辯,判大宗正府事宗美。遣使殺領行臺尚書省事秉德,東京留守宗懿,北京留守卞及太宗子孫七十餘人,周宋國王宗翰子孫三十餘人,諸宗室五十餘人。辛酉,以尚書省譯史蕭玉為禮部尚書,秘書監蕭裕為尚書左丞,司徒袞領三省事、封王,都元帥如故,右丞相烏帶司空、左丞相兼侍中,平章政事劉筈為尚書右丞相兼中書令,左丞宗義,右丞溫都思忠為平章政事,參知政事劉麟為尚書右丞,殿前左副點檢僕散忽土為殿前都點檢。



 五月戊子,以
 平章行臺尚書省事、右副元帥大抃為行臺尚書右丞相,元帥如故。壬辰,以左副元帥撒離喝為行臺尚書左丞相,元帥如故。同判大宗正事宗安為御史大夫。六月丙午朔,高麗遣使賀即位。甲子,太廟初設四神門及四隅罘罳。



 七月己丑,司空、左丞相兼侍中烏帶罷。以平章政事溫都思忠為左丞相,尚書左丞蕭裕為平章政事,右丞劉麟為左丞,侍衛親軍步軍都指揮使完顏思恭為右丞。參知政事張浩丁憂,起復如故。戊戌,夏國遣使賀即位及受尊號。八月戊申,以司徒袞為太尉、領三省事,都元帥如故。以禮部尚書蕭玉為參知政事。



 九月甲
 午,立惠妃徒單氏為皇后。十月癸卯,太師、領三省事勖致仕。辛未,殺太皇太妃蕭氏及其子任王偎喝。使使殺行臺左丞相、左副元帥撒離喝於汴,並殺平章政事宗義、前工部尚書謀里野、御史大夫宗安,皆夷其族。以魏王斡帶之孫活里甲好脩飾,亦族之。十一月癸未,尚書右丞相劉筈罷。以會寧牧徒單恭為平章政事。尚書左丞劉麟、右丞完顏思恭罷。以參知政事張浩為尚書右丞。乙酉,以行臺尚書左丞張通古為尚書左丞。丙戌,白虹貫日。丁亥,以太后旨稱令旨。戊子,以十二事戒約官吏。己丑,命庶官許求次室二人,百姓亦許置妾。



 十二月
 癸卯朔,詔去群臣所上尊號。丙午,初定襲封衍聖公俸格。命外官去所屬百里外者不許參謁,百里內者往還不得過三日。癸丑,立太祖射碑于紇石烈部中,上及皇后致奠于碑下。甲寅,野人來獻異香,卻之。乙卯,有司奏慶雲見,上曰:「朕何德以當此。自今瑞應毋得上聞,若有妖異,當以諭朕,使自警焉。」己未,罷行臺尚書省。改都元帥府為樞密院。詔改定繼絕法。以右副元帥大抃為尚書右丞相兼中書令,參知行臺尚書省事張中孚為參知政事,都元帥袞為樞密使,太尉、領三省事如故,元帥左監軍昂為樞密副使,刑部尚書趙資福為御史大夫。



 三年正月癸酉朔,宋、夏、高麗遣使來賀。乙亥,參知政事蕭玉丁憂,起復如故。癸未,立春,觀擊土牛。丁亥,初造燈山於宮中。戊子,生辰,宋、高麗、夏遣使來賀。甲午,初置國子監。謂御史大夫趙資福曰:「汝等多徇私情,未聞有所彈劾,朕甚不取。自今百官有不法者,必當舉劾,無憚權貴。」乙未,上出獵,宰相以下辭於近郊。上駐馬戒之曰:「朕不惜高爵厚祿以任汝等,比聞事多留滯,豈汝等茍圖自安不以民事為念耶?自今朕將察其勤惰,以為賞罰,其各勉之。」丁酉,白虹貫日。二月丁巳,還宮。



 三月庚寅,以翰林學士劉長言等為宋生日使。壬辰,詔廣燕城,建宮
 室。己亥,謂侍臣曰:「昨太子生日,皇后獻朕一物,大是珍異,卿試觀之。」即出諸絳囊中,乃田家稼穡圓。「后意太子生深宮之中,不知民間稼穡之艱難,故以為獻,朕甚賢之。」四月丙午,詔遷都燕京。辛酉,有司圖上燕城宮室制度,營建陰陽五姓所宜。海陵曰:「國家吉凶,在德不在地。使桀、紂居之,雖卜善地何益。使堯、舜居之,何用卜為。」丙寅,罷歲貢鷹隼。沂州男子吳真犯法,當死,有司以其母老疾無侍為請,命官與養濟,著為令。閏月辛未朔,命尚書右丞張浩調選燕京,仍諭浩無私徇。丙子,命太官常膳惟進魚肉,舊貢鵝鴨等悉罷之。丁丑,罷皇統間苑中
 所養禽獸。歸德軍節度使阿魯補以撤官舍材木構私第,賜死。戊戌,詔朝官稱疾不治事者,尚書省令監察御史與太醫同診視,無實者坐之。



 五月壬子,以戒敕宰相以下官,詔中外。戊辰,宰臣請益嬪御以廣嗣續。上命徒單貞語宰臣,前所誅黨人諸婦人中多朕中表親,欲納之官中。平章政事蕭裕不可,上不從。遂納宗本子莎魯啜,宗固子胡里剌、胡失打,秉德弟糺里等妻宮中。六月丙子,殺太府監完顏馮六。宋遣使祈請山陵,不許。九月庚戌,賜燕京役夫帛,人一匹。以東京路兵馬都總管府判官蕭子敏為高麗生日使,修起居注蕭彭哥為夏國
 生日使。



 十月己巳,殺蘭子山猛安蕭拱。以右副點檢不術魯阿海等為宋正旦使。十一月癸亥,詔罷世襲萬戶官,前後賜姓人各復本姓。十二月戊辰,杖壽寧縣主徐輦。癸酉,獵于近郊。乙酉,還宮。是歲,子崇王元壽薨。



 四年正月丁酉朔,宋、高麗、夏遣使來賀。群臣請立皇太子,從之。戊戌,初定東宮官屬。立捕盜賞格。癸卯,太白經天。壬子,生辰,宋、高麗、夏遣使來賀。癸亥,朝謁世祖、太祖、太宗、德宗陵。甲子,還宮。二月丁卯,立子光英為皇太子,庚午,詔中外。甲戌,如燕京。昭義軍節度使蕭仲宣家奴告其主怨謗。上曰:「仲宣之姪迪輦阿不近以誹謗誅,故
 敢妄愬。」命殺告者。迪輦阿不者,蕭拱也。戊子,次泰州。



 三月丙申朔,以刑部尚書田秀穎等為宋生日使。四月丙寅朔,有司請今歲河南、北選人並赴中京銓注,從之。壬辰,上自泰州如涼陘。五月丁酉,獵于立列只山。甲寅,賜獵士,人一羊。乙卯,次臨潢府。丁巳,太白經天。六月甲子朔,駐綿山。戊寅,權楚底部猛安那野伏誅。七月癸卯,命崇義軍節度使烏帶之妻唐括定哥殺其夫而納之。八月癸亥朔,獵于途你山。甲戌,以侍御史保魯鞫事不實,杖之。丙子,次于鐸瓦。



 九月甲午,次中京。丙午,尚書右丞相大抃罷。殺太府少監劉景。以都水使者完顏麻潑為
 高麗生日使,吏部郎中蕭中立為夏國生日使。十月壬戌朔,使使奉遷太廟神主。御史大夫趙資福罷。甲申,以太子詹事張用直等為賀宋正旦使。殺太祖長公主兀魯,杖罷其夫平章政事徒單恭,封其侍婢忽撻為國夫人。恭之兄定哥初尚兀魯,定哥死,恭強納焉,而不相能,又與侍婢忽撻不協。忽撻得幸于后,遂譖於上,故見殺,而並罷恭。



 十一月戊戌,以咸平尹李德固為平章政事。辛丑,買珠于烏古迪烈部及蒲與路,禁百姓私相貿易,仍調兩路民夫,採珠一年。戊申,以前平章政事徒單恭為司徒。



 十二月甲子,斬妄人敲仙于中京市。辛未,以汴
 京路都轉運使左瀛等為賀宋正旦使。庚寅,太尉、領三省事、樞密使袞薨。



 貞元元年正月辛卯朔,上不視朝。詔有司受宋、高麗、夏、回紇貢獻。丙午,生辰,宋、高麗、夏遣使來賀。以中京留守高楨為御史大夫。



 二月庚申,上自中京如燕京。三月辛亥,上至燕京,初備法駕,甲寅,親選良家子百三十餘人充後宮。乙卯,以遷都詔中外。改元貞元。改燕京為中都,府曰大興,汴京為南京,中京為北京。丙辰,以司徒徒單恭為太保、領三省事,平章政事蕭裕為右丞相兼中書令,右丞張浩、左丞張通古為平章政事,參知政事張中
 孚為左丞,蕭玉為右丞,平章政事李德固為司空,左宣徽使劉萼為參知政事,樞密副使昂為樞密使,工部尚書僕散師恭為樞密副使。四月辛酉,以右宣徽使紇石烈撒合輦等為賀宋生日使。辛未,特封唐括定哥為貴妃。戊寅,皇太后大氏崩。



 五月辛卯,殺弟西京留守蒲家。西京兵馬完顏謨盧瓦、編修官圓福奴、通進孛迭坐與蒲家善,并殺之。乙卯,以京城隙地賜朝官及衛士。六月乙丑,以安國軍節度使耶律恕為參知政事。



 七月戊子朔,元賜朝官京城隙地,徵錢有差。八月壬戌,司空李德固薨。禁中都路捕射麞兔。戊寅,賜營建宮室工匠及役
 夫帛。九月丁亥朔,以翰林待制謀良虎為夏國生日使,吏部郎中窊合山為高麗生日使。



 十月丁巳,獵于良鄉。封料石岡神為靈應王。初,海陵嘗過此祠,持杯珓禱曰:「使吾有天命,當得吉卜。」投之,吉。又禱曰:「果如所卜,他日當有報,否則毀爾祠宇。」投之,又吉,故封之。戊午,還宮。壬戌,有司言:「太后園陵未畢,合停冬享及祫祭。」從之。丙子,命內外官聞大功以上喪,止給當日假,若父母喪,聽給假三日,著為令。



 十一月丙戌朔,定州獻嘉禾,詔自今不得復進。己丑,瑤池殿成。丙申,以戶部尚書蔡松年等為賀宋正旦使。戊戌,左丞相耨碗溫都思忠致仕。庚戌,以
 樞密使昂為左丞相,樞密副使僕散思恭為樞密使。十二月,太白經天。戊午,特賜貴妃唐括定哥家奴孫梅進士及第。壬戌,以簽書樞密院事南撒為樞密副使。辛未,對所納皇叔曹國王宗敏妃阿懶為昭妃。丙子,貴妃唐括定哥坐與舊奴姦,賜死。



 閏月乙酉朔,殺護衛特謨葛。癸巳,定社稷制度。太白經天。癸卯,以太保、領三省事徒單恭為太保、領三省事如故。命西京路統軍撻懶、西北路招討蕭懷忠、臨潢府總管馬和尚、烏古迪烈司招討斜野等北巡。



 二年正月甲寅朔,上不豫,不視朝。賜宋、高麗、夏使就館
 燕。庚申,太白經天。尚書右丞相蕭裕與前真定尹蕭馮家奴、前御史中丞蕭招折、博州同知遙設等謀反,伏誅,詔中外。己巳,生辰,宋、高麗、夏遣使來賀。



 二月甲申朔,以平章政事張浩為尚書右丞相兼中書令。甲午,以尚書右丞蕭玉為平章政事,前河南路統軍使張輝為尚書右丞,西北路招討使蕭好胡為樞密副使。三月戊辰,夏遣使賀遷都。四月丙戌,幸大興府及都轉運使司。遣薦含桃于衍慶宮。



 五月癸丑朔,日有食之,避正殿,敕百官勿治事。己未,詔自今每月上七日不奏刑名,尚食進饌不進肉。丁卯,始置交鈔庫,設使副員。丁丑,太原尹徒單阿里
 出虎伏誅,復命其子術斯剌乘傳焚其骨,擲水中。七月庚申,初設鹽鈔香茶文引印造庫使副。丙子,參知政事耶律恕罷。



 八月丙午,以左丞相昂去衣杖其弟婦,命杖之。戊申,以御史大夫高楨為司空,御史大夫如故。九月己未,常武殿擊鞠,令百姓縱觀。辛酉,以吏部尚書蕭頤為參知政事。癸亥,獵于近郊。丁卯,次順州。太師、領三省事徒單恭薨。是夜,還宮。乙亥,復獵于近郊。十月庚辰朔,殺廣寧尹韓王亨。庚寅,還宮。庚子,以左丞相致仕溫都思忠起為太傅、領三省事。以刑部侍郎白彥恭等為賀宋正旦使。



 十一月戊辰,上命諸從姊妹皆分屬諸妃,出
 入禁中,與為淫亂,臥內遍設地衣,裸逐為戲。是月,初置惠民局。高麗遣使謝賜生日。



 十二月乙酉,以太傅溫都思忠為太師,領三省事如故,平章政事張通古為司徒,平章政事如故。



 三年正月己酉朔,宋、高麗、夏遣使來賀。辛酉,以判東京留守大抃為太傅、領三省事。甲子,生辰,宋、高麗、夏遣使來賀。二月壬午,以左丞相昂為太尉、樞密使,右丞相張浩為左丞相兼侍中,樞密使僕散思恭為右丞相兼中書令。尚書左丞張中孚罷,右丞張暉為平章政事。參知政事劉萼為左丞,參知政事蕭頤為右丞,吏部尚書蔡
 松年為參知政事。



 三月壬子,以左丞相張浩、平章政事張暉每見僧法寶必坐其下,失大臣體,各杖二十。僧法寶妄自尊大,杖二百。乙卯,命以大房山雲峰寺為山陵,建行宮其麓。庚午,以左司郎中李通為賀宋生日使。夏四月丁丑朔,昏霧四塞,日無光,凡十有七日。五月丁未朔,日有食之。癸丑,南京大內火。乙卯,命判大宗正事京等如上京,奉遷太祖、太宗梓宮。丙寅,如大房山,營山陵。



 六月丙戌,登寶昌門觀角抵,百姓縱觀。乙未,命右丞相僕散思恭、大宗正丞胡拔魯如上京,奉遷山陵及迎永壽宮皇太后。



 七月癸丑,太白晝見。辛酉,如大房山,杖提
 舉營造官吏部尚書耶律安禮等。乙亥,還宮。八月壬午,如大房山。甲申,啟土,賜役夫,人絹一匹。是日,還宮。甲午,遣平章政事蕭玉迎祭祖宗梓宮於廣寧。乙未,增置教坊人數。庚子,杖左宣徽使敬嗣暉、同知宣徽事烏居仁及尚食官。



 九月戊申,平章政事張輝迎祭梓宮于宗州。乙卯,上謂宰臣及左司官曰:「朝廷之事,尤在慎密。昨授張中孚、趙慶襲官,除書未到,先已知之,皆汝等泄之也。敢復爾者,殺無赦。」己未,如大房山。庚申,還宮。丙寅,以殿前都點檢納合椿年為參知政事。丁卯,上親迎梓宮及皇太后于沙流河,命左右持杖二束,跽太后前,曰:「某不
 孝,久失溫凊,願痛笞之。」太后掖起之,曰:「凡民有子克家,猶愛之,況我有子如此。」叱持杖者退。庚午,獵,親射麞以薦梓宮。壬申,至自沙流河。



 十月丙子,皇太后至中都,居壽康宮。戊寅,權奉安太廟神主于延聖寺,致奠梓宮於東郊,舉哀。己卯,梓宮至中都,以大安殿為丕承殿,安置。壬午,命省部諸司便服治事,不奏死刑一月。辛卯,告於丕承殿。乙未,如菆宮,冊謚永寧皇太后曰慈憲皇后。丁酉,大房山行宮成,名曰磐寧。戊戌,還宮。己亥,以翰林學士承旨耶律歸一等為賀宋正旦使。



 十一月乙巳朔,梓宮發丕承殿。戊申,山陵禮成。甲寅,詔內外大小職官覃
 遷一重,貞元四年租稅並與放免,軍士久於屯戍不經替換者,人賜絹三匹、銀三兩。群臣稱賀。丙辰,燕百官於泰和殿。丁卯,奉安神主于太廟。戊辰,群臣稱賀。辛未,獵於近郊。十二月己丑,還宮。木冰。乙未,上朝太后于壽康宮。己亥,太傅、領三省事大抃薨,親臨哭之,命有司廢務及禁樂三日。



 正隆元年正月癸卯朔,宋、高麗、夏遣使來賀。己酉,群臣奉上尊號曰聖文神武皇帝。上自九月廢朝,常數月不出,有急奏,召左右司郎中省於臥內。庚戌,始視朝。戊午,生辰,宋、高麗、夏遣使來賀。乙丑,觀角抵戲。罷中書門下
 省。以太師、領三省事溫都思忠為尚書令,太尉、樞密使昂為太保,右丞相僕散思恭為太尉、樞密使。左丞劉萼、右丞蕭頤罷,參知政事蔡松年為尚書右丞。樞密副使蕭懷忠罷,吏部尚書耶律安禮為樞密副使。平章政事蕭玉為右丞相,平章政事張暉罷,不置平章政事官。



 二月癸酉朔,改元正隆,大赦。庚辰,御宣華門觀迎佛,賜諸寺僧絹五百匹、彩五十段、銀五百兩。辛巳,改定內外諸司印記。乙未,司徒張通古致仕。庚子,謁山陵。辛丑,還都。三月壬寅朔,始定職事官朝參等格。仍罷兵衛。庚申,以左宣徽使敬嗣暉等為賀宋生日使。



 四月,太尉、樞密使
 僕散思恭以父憂,起復如故。五月辛亥,修容安氏閣女御為妖所憑,舞噪宮中,命殺之。是月,頒行正隆官制。六月庚辰,天水郡公趙桓薨。丙戌,以尚書右丞蔡松年為左丞,樞密副使耶律安禮為右丞,附馬都尉烏古論當海為樞密副使。七月己酉,命太保昂如上京,奉遷始祖以下梓宮。



 八月丁丑,如大房山行視山陵。



 十月乙酉,葬始祖以下十帝于大房山。丁酉,還宮。閏月己亥朔,山陵禮成,群臣稱賀。甲辰,回鶻使使寅術烏籠骨來貢。庚寅,杖右丞相蕭玉、左丞蔡松年、右丞耶律安禮、御史中丞馬諷等。十一月己巳朔,以右司郎中梁金求等為賀宋正
 旦使。癸巳,禁二月八日迎佛。



 二年正月戊辰朔,宋、高麗、夏遣使來賀。庚辰,太白晝見。癸未,生辰,宋、高麗夏遣使來賀。庚寅,以工部侍郎韓錫同知宣徽院事,錫不謝,杖百二十,奪所授官。二月辛丑,初定太廟時享牲牢禮儀。癸卯,改定親王以下封爵等第,命置局追取存亡告身,存者二品以上,死者一品,參酌削降。公私文書,但有王爵字者,皆立限毀抹,雖墳墓碑志並發而毀之。



 三月丙寅朔,高麗遣使賀受尊號。



 四月戊戌,追降景宣皇帝為豐王。以僉書宣徽院事張哲為橫賜高麗使,宿直將軍溫敦斡喝為橫賜夏國使。六
 月乙未,參知政事納合椿年薨。以禮部尚書耶律守素等為賀宋生日使。八月癸卯,始置登聞院。甲寅,罷上京留守司。



 九月乙丑,以宿直將軍僕散烏里黑為夏國生日使。戊子,罷護駕軍,置龍翔虎步軍。罷尚書省文資令史出為外官。是秋,中都、山東、河東蝗。十月壬寅,命會寧府毀舊宮殿、諸大族第宅及儲慶寺,仍夷其址而耕種之。丁未,禁賣古器入他境。乙卯,初鑄銅錢。十一月辛未,以侍衛親軍副指揮使高助不古等為賀宋正旦使。十二月己亥,以侍衛親軍都指揮使紇石烈良弼為參知政事。



 三年正月壬戌朔,宋、高麗、夏遣使來賀。丙寅,子矧思阿不死,殺太醫副使謝友正及其乳母等。丁丑,生辰,宋、高麗、夏遣使來賀。己卯,杖右諫議大夫楊伯雄。



 二月壬辰朔,都城及京兆初置錢監。甲午,遣使檢視隨路金銀銅鐵冶。



 三月辛酉朔,司天奏日食,侯之不見。命自今遇日食,面奏,不須頒告。辛巳,以兵部尚書蕭恭等為賀宋生日使。四月丙辰,樞密副使烏古論當海罷,以北京留守張暉為樞密副使。六月壬辰,蝗入京師。



 七月庚申,封子廣陽為滕王。甲申,以右丞相蕭玉為司徒,尚書左丞蔡松年為右丞相,右丞耶律安禮為左丞,參知政事紇石
 烈良弼為右丞,左宣徽使敬嗣暉、吏部尚書李通為參知政事。九月己未,太白經天。甲子,滕王廣陽薨。庚午,以宿直將軍阿魯保為夏國生日使。丁丑,以教坊提點高存福為高麗生日使。辛巳,遷中都屯軍二猛安於南京,遣吏部尚書李惇等分地安置。十月戊戌,詔尚書省:「凡事理不當者,許詣登聞檢院投狀,院類奏覽訖,付御史臺理問。」



 十一月辛酉,以工部尚書蘇保衡等為賀宋正旦使。癸亥,詔有司勤政安民。癸未,尚書左丞耶律安禮罷。參知政事李通以憂制,起復如故。詔左丞相張浩、參知政事敬嗣暉營建南京宮室。十二月乙卯,以樞密副
 使張暉為尚書左丞。歸德尹致仕高召和式起為樞密副使。



 四年正月丙辰朔,宋、高麗、夏遣使來賀。上朝太后于壽康宮。丁巳,御史大夫高楨薨。庚申,更定私相越境法,並論死。辛酉,罷鳳翔、唐、鄧、潁、蔡、鞏、洮、膠西諸榷場,置場泗州。辛未,生辰,宋、高麗、夏遣使來賀。二月己丑,以左宣徽使許霖為御史大夫。丁未,修中都城。造戰船于通州。詔諭宰臣以伐宋事。調諸路猛安謀克軍年二十以上、五十以下者,皆籍之,雖親老丁多亦不許留侍。



 三月丙辰朔,遣兵部尚書蕭恭經畫夏國邊界。遣使分詣諸道總
 管府督造兵器。四月辛丑,命增山東泉水、畢括兩營兵士廩給。庚戌,詔諸路舊貯軍器並致於中都。時方建宮室于南京,又中都與四方所造軍器材用皆賦於民,箭翎一尺至千錢,村落間往往椎牛以供筋革,至於烏鵲狗彘無不被害者。辛亥,尚書左丞張暉、御史大夫許霖罷。以大興尹徒單貞為樞密副使。以秘書監王可道等為賀宋生日使。八月,詔諸路調馬,以戶口為差,計五十六萬餘匹,富室有至六十匹者,仍令戶自養飼以俟。己卯,尚書右丞相蔡松年薨。



 九月,以翰林待制完顏達紀為高麗生日使,宿直將軍加古撻懶為夏國生日使。
 十月乙亥,獵于近郊,觀造船于通州。賜尚書右丞紇石烈良弼、樞密副使徒單貞佩刀入宮。十一月甲辰,以翰林侍講學士施宜生等為賀宋正旦使。十二月乙卯,宋遣使告母韋氏哀。甲子,太白晝見。乙丑,以左副點檢大懷忠等為宋弔祭使。乙亥,太醫使祁宰上疏諫伐宋,殺之。



 五年正月庚辰朔,宋、高麗、夏遣使來賀。乙未,生辰,宋、高麗、夏遣使來賀。二月壬子,宋遣使獻母后遣留物。丁卯,太白晝見。辛未,河東、陜西地震,鎮戎、德順軍大風,壞廬舍,人多壓死。甲戌,遣引進使高植、刑部郎中海狗分道
 監視所獲盜賊,並凌遲處死,或鋸灼去皮截手足。仍戒屯戍千戶謀克等,後有獲者,並處死,總管府官亦決罰。



 三月辛巳,東海縣民張旺、徐元等反,遣都水監徐文、步軍指揮使張弘信、同知大興尹事李惟忠、宿直將軍蕭阿窊率舟師九百,浮海討之,命之曰:「朕意不在一邑,將試舟師耳。」庚子,以司徒判大宗正事蕭玉為御史大夫,司徒如故,尚書右丞紇石烈良弼為左丞,橫海軍節度使致仕劉長言起為右丞。



 四月庚戌,昭妃蒲察阿里忽有罪賜死。甲寅,宿州防禦使耶律翼使宋失體,杖二百,除名。甲戌,太白晝見。六月,徐文等破賊張旺、徐元,東海
 平。



 七月辛巳,詔東海縣徐元、張旺詿誤者,並釋之。壬午,以張弘信被命討賊,稱疾逗遛萊州,與妓樂飲燕,杖之二百。癸卯,遣使簽諸路漢軍。八月丙午朔,日有食之。辛亥,命榷貨務并印造鈔引庫起赴南京。己巳,樞密副使徒單貞罷,以太子少保徒單永年為樞密副使。辛未,謁山陵,見田間獲者,問其豐耗,以衣賜之。



 九月己卯,還宮。十月庚午,遣護衛完顏普連等二十四人督捕山東、河東、河北、中都盜賊。籍諸路水手得三萬人。十一月乙酉,以濟南尹僕散烏者等為賀宋正旦使。尚書右丞劉長言罷。命親軍司以所掌付大興府。置左右驍騎都副指
 揮使,隸點檢司。步軍都副指揮使,隸宣徽院。



 十二月癸丑,禁中都、河北、山東、河南、河東、京兆軍民綱捕禽獸及畜養雕隼者。戊辰,禁朝官飲酒,犯者死,三國人使燕飲者罪。



 六年正月甲戌朔,宋、高麗、夏遣使來賀。丁丑,判大宗正徒單貞、益都尹京、安武軍節度使爽、金吾衛上將軍阿速飲酒,以近屬故,杖貞七十,餘皆杖百。壬午,上將如南京,以司徒、御史大夫蕭玉為大興尹,司徒如故。樞密副使徒單永年罷,以都點檢紇石烈志寧為樞密副使。己丑,生辰,宋、高麗、夏遣使來賀。癸巳,命參知政事李通諭
 宋使徐度等曰:「朕昔從梁王軍,樂南京風土,常欲巡幸。今營繕將畢功,期以二月末先往河南。帝王巡守,自古有之。以淮右多隙地,欲校獵其間,從兵不踰萬人。況朕祖宗陵廟在此,安能久于彼乎。汝等歸告汝主,令有司宣諭朕意,使淮南之民無懷疑懼。」庚子,詔自中都至河南府所過州縣調從獵騎士二千。辛丑,殺蒲察阿虎迭女義察。義察,慶宜公主出,幼鞠宮中,上屢欲納之,太后不可。至是,以罪殺之。



 二月乙巳,杖衛王襄之妃及左宣徽使許霖。甲寅,以參知政事李通為尚書右丞。己未,禁扈從縱獵擾民。庚申,徵諸道水手運戰船。癸亥,發中都。
 丙寅,次安肅州。三月己卯,改河南北邙山為太平山,稱舊名者以違制論。丁亥,將至獲嘉,有男子上書言事,斬之,所言莫得聞。癸巳,次河南府,因出獵,幸汝州溫湯,視行宮地。自中都至河南,所過麥皆為空。復禁扈從毋輒離次及游賞飲酒,犯者罪皆死,而莫有從者。詔內地諸猛安赴山後牧馬,俟秋並發。弟袞之妻烏延氏有罪,賜死。烏延氏之弟南京兵馬副都指揮使習泥烈亦以罪誅。



 四月丁未,詔百官先赴南京治事,尚書省、樞密院、大宗正府、勸農司、太府、少府皆從行,吏、戶、兵、刑部,四方館,都水監,大理司官各留一員。以簽書樞密院事高景山等
 為賀宋生日使。戊申,詔汝州百五十里內州縣,量遣商賈赴溫湯置市。詔有司移問宋人,蔡、穎、壽諸州對境創置堡戍者。庚戌,發河南府。契丹不補自山馳下,伏道左,自陳破東海賊有功,為李惟忠所抑,立命斬之。丁卯,次溫湯。誡扈從毋輒過汝水。上獵,奔鹿突之墮馬,嘔血數日。遣使徵諸道兵。



 五月庚辰,太師、尚書令耨碗溫都思忠薨。契丹諸部反,遣右衛將軍蕭禿剌等討之。六月癸卯,命樞密使僕散思恭、西京留守蕭懷忠將兵一萬討契丹諸部。上自汝州如南京。壬戌,次南京近郊,左丞相張浩率百官迎謁。是夜,大風,壞承天門鴟尾。癸亥,上備法
 駕入於南京。



 七月丁亥,以左丞相張浩為太傅、尚書令,司徒、大興尹蕭玉為尚書左丞相,吏部尚書白彥恭為樞密副使,樞密副使紇石烈志寧為開封尹,安武軍節度使徒單貞為御史大夫。己丑,賜從駕、從行、從軍及千戶謀克錢帛。大括天下羸馬。殺亡遼耶律氏、宋趙氏子男凡百三十餘人。



 八月壬寅,單州賊杜奎據城叛,遣都點檢耶律湛、右驍騎副都指揮大磐討之。以樞密副使白彥恭為北面兵馬都統,開封尹紇石烈志寧副之,中都留守完顏彀英為西北面兵馬都統,西北路招討使唐括孛古的副之,討契丹。癸丑,以諫伐宋弒皇太后
 徒單氏于寧德宮,仍命即宮中焚之,棄其骨水中,並殺其侍婢等十餘人。癸亥,殺右衛將軍蕭禿剌、護衛十人長斡盧保,族樞密使僕散思恭、北京留守蕭賾、西京留守蕭懷忠,杖尚書令張浩、左丞相蕭玉。以太常博士張崇為高麗生日使,蕭誼忠為夏國生日使。甲子,封所幸太后侍婢高福娘為鄖國夫人。



 九月庚午朔,以太保、判大宗正事昂為樞密使,太保如故。戊子,殺前壽州刺史毛良虎。庚寅,大名府賊王九據城叛,眾至數萬,所至盜賊峰起,大者連城邑,小者保山澤,或以十數騎張旗幟而行,官軍莫敢近。上又惡聞盜賊事,言者輒罪之。



 上自將
 三十二總管兵伐宋,進自壽春。以太保、樞密使昂為左領軍大都督,尚書右丞李通副之,尚書左丞紇石烈良弼為右領軍大都督,判大宗正烏延蒲盧渾副之,御史大夫徒單貞為左監軍,同判大宗正事徒單永年為右監軍,左宣徽使許霖為左都監,河南尹蒲察斡論為右都監,皆從。工部尚書蘇保衡為浙東道水軍都統制,益都尹鄭家副之,由海道徑趨臨安。太原尹劉萼為漢南道行營兵馬都統制,濟南尹僕散烏者副之,進自蔡州。河中尹徒單合喜為西蜀道行營兵馬都統制,平陽尹張中彥副之,由鳳翔取散關,駐軍以俟後命。武勝、武平、
 武捷三軍為前鋒。徒單貞別將兵二萬入淮陰。甲午,上發南京,詔皇后及太子光英居守,尚書令張浩、左丞相蕭玉、參知政事敬嗣暉留治省事。丙申,太白晝見。將士自軍中亡歸者相屬于道。曷蘇館猛安福壽、東京謀克金住等始授甲於大名,即舉部亡歸,從者眾至萬餘,皆公言於路曰:「我輩今往東京,立新天子矣!」



 十月乙巳,陰迷失道,二鼓始達營所。丙午,慶雲見。東京留守曹國公烏祿即位于遼陽,改元大定,大赦。數海陵過惡:弒皇太后徒單氏,弒太宗及宗翰、宗弼子孫及宗本諸王,毀上京宮室,殺遼豫王、宋天水郡王、郡公子孫等數十事。丁
 未,大軍渡淮,將至廬州,獲白鹿,以為武王白魚之兆。漢南道劉萼取通化軍、蔣州、信陽軍。徒單貞敗宋將王權於盱眙,進取揚州。前鋒軍至段寨,宋戍兵皆遁去,敗宋兵於蔚子橋,敗宋兵于巢縣,斬二百級,至和州。王權夜以兵千餘來襲,射卻之。翼日,雨。宋人夜焚其積遁去。詰旦追之,宋人逆戰,猛安韓棠軍卻,遂失利。溫都奧剌奔北,武捷軍副總管阿散率猛安謀克力戰,卻之。王權退保南岸。癸亥,上次和州,阿散等進階賞賚有差。西蜀道徒單合喜駐散關,宋人攻秦州臘家城、德順州,克之。浙東道蘇保衡與宋人戰于海道,敗績,副統制鄭家死
 之。



 十一月庚午,左司郎中兀不喝等聞赦,入白東京即位改元事,上拊髀歎曰:「我本欲滅宋後改元大定,豈非天命乎?」出其書示之,即預志改元事也。以勸農使完顏元宜為浙西道兵馬都統制,刑部尚書郭安國副之。上駐軍江北。遣武平總管阿鄰先渡江至南岸,失利。上還和州,遂進兵揚州。甲午,會舟師于瓜洲渡,期以明日渡江。乙未,浙西兵馬都統制完顏元宜等軍反,帝遇弒,崩,年四十。



 海陵在位十餘年,每飾情貌以御臣下。卻尚食進鵝以示儉,及游獵頓次,不時需索,一鵝一鶉,民間或用數萬售之,有以一牛易一鶉者。或以弊衾覆衣,以示
 近臣。或服補綴,令記注官見之。或取軍士陳米飯與尚食同進,先食軍士飯幾盡。或見民車陷泥澤,令衛士下挽,俟車出然後行。與近臣燕語,輒引古昔賢君以自況。顯責大臣,使進直言。使張仲軻輩為諫官,而祁宰竟以直諫死。比暱群小,官賞無度,左右有曠僚者,人或以名呼之,即授以顯階。常置黃金裀褥間,有喜之者,令自取之。而淫嬖不擇骨肉,刑殺不問有罪。至營南京宮殿,運一木之費至二千萬,牽一車之力至五百人。宮殿之飾,遍傅黃金而後間以五采,金屑飛空如落雪。一殿之費以億萬計,成而復毀,務極華麗。其南征造戰艦江上,毀
 民廬舍以為材,煮死人膏以為油,殫民力如馬牛,費財用如土苴,空國以圖人國,遂至於敗。都督府以其柩置之南京班荊館。大定二年,降封為海陵郡王,謚曰煬。二月,世宗使小底婁室與南京官遷其柩於寧德宮。四月,葬於大房山鹿門谷諸王兆域中。二十年,熙宗既祔廟,有司奏曰:「煬王之罪未正。準晉趙王倫廢惠帝自立,惠帝反正,誅倫,廢為庶人。煬帝罪惡過於倫,不當有王對,亦不當在諸王塋域。」乃詔降為海陵庶人,改葬于山陵西南四十里。



 贊曰:海陵智足以拒諫,言足以飾非。欲為君則弒其君,
 欲伐國則弒其母,欲奪人之妻則使之殺其夫。三綱絕矣,何暇他論。至於屠滅宗族,剪刈忠良,婦姑姊妹盡入嬪御。方以三十二總管之兵圖一天下,卒之戾氣感召,身由惡終,使天下後世稱無道主以海陵為首。可不戒哉!可不戒哉!



\end{pinyinscope}