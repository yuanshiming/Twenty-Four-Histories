\article{志第七}

\begin{pinyinscope}

 地理下



 ○大名府路河東北路河東南路京兆府路鳳翔路鄜延路慶原路臨洮路



 大名府路,宋北京魏郡。府一,領刺郡三,縣二十,鎮二十二。貞祐二年十月置行尚書省。



 大名府,上,天雄軍。舊為散府,先置統軍司,天德二年罷,以其所轄民戶分隸旁近總管府。正隆二年升為總管府,附近十二猛安皆隸焉,兼漕河事。產皺、穀、絹、梨肉、櫻桃、
 煎木耳、硝。戶三十萬八千五百一十一。縣十、鎮十三:舊有柳林、侯固二鎮。



 元城有愜山、漕運御河、屯氏河。鎮二安定、安賢。



 大名倚。鎮一



 魏縣



 冠氏有弇山、水沙河。鎮四普通、清水、博寧、桑橋。



 南樂鎮一南樂。



 館陶有漕運御河。鎮一館陶。



 夏津有屯氏河、潤溝河。鎮一孫生。



 朝城鎮一韓張。



 清平有新渠金堤。鎮一清平。



 莘鎮一馬橋。



 恩州,中,刺史。宋清河郡軍事,治清河,今治歷亭。戶九萬九千一百一十九。縣四、鎮六:



 歷亭倚。有永濟渠,置河倉。鎮四漳南、新安樂、舊安樂、王杲。



 武城有永濟渠、沙河。鎮一武城。



 清河有永濟渠、漳渠。



 臨清有河倉。鎮一曹仁。



 濮州,下,刺史。宋濮陽郡。戶五萬二千九百四十八。
 縣二、鎮三:



 鄄城倚。有旄丘、陶丘、金堤。鎮二臨濮、雷澤、皆舊縣、貞元二年為鎮。



 范鎮一定安。



 開州,中,刺史。宋開德府澶淵郡鎮寧軍節度,降為澶州,皇統四年復更今名。戶三萬三千八百三十六。縣四、鎮一:



 濮陽倚。有衛陽山、鮒鰨山、黃河、淇河、瓠子口。



 清豐有廣陽山、黃河。



 觀城有泉源河。鎮一武鄉。



 長垣本隸南京,泰和八年以限河不便。來屬。



 河東北路。宋河東路,天會六年析河東為南、北路,各置兵馬都總管。府一,領節鎮三,刺郡九,縣三十九,鎮四十,堡十,寨八。



 太原府,上,武勇軍。宋太原郡河東軍節度,國初依舊
 為次府,復名并州太原郡河東軍總管府,置轉運司。有造墨場,煉銀洞、瑪瑙石。藥產松脂、白膠香、五靈脂。大黃、白玉石。戶一十六萬五千八百六十二。縣十一、鎮八:



 陽曲倚。有罕山、蒙山、汾水。鎮五陽曲、百井、赤塘關、天門關、陵井驛。



 太谷有太谷山、蔣水。



 平晉貞祐四年七月廢,興定元年復置。有龍山、晉水。鎮二晉寧、晉祠。



 清源有清源水、汾水。



 徐溝本清源縣之徐溝鎮,大定二十九年升。



 榆次有麓臺山、塗水。



 祁有幘山、太谷水鎮一團柏。



 文水有隱泉山、汾水、文水。



 交城有少楊山、狐突山、汾水。



 盂興定中升為州,聽絳州元帥府節制,置刺史,尋復。有白馬山、原仇山、滹沱水。



 壽陽興定二年九月嘗割隸平定州,有方山、洞過水。



 晉州。興定四年正月以壽陽縣西張寨置。



 忻州,下,刺史。舊定襄郡軍。戶三萬二千三百四十一。
 縣二、鎮四:



 秀容有程候山、雲母山、忻水、滹沱水。鎮四忻口、雲內、徒合、石嶺。



 定襄



 平定州,中,刺史。本宋平定軍,大定二年升為州。興定二年為防禦,十一月復降為刺郡。戶一萬八千二百九十六。縣二、鎮三:



 平定倚。有浮山、浮濼水。鎮二承天、東百井。



 樂平興定四年正月升為皋州。有樂平山、清漳水。鎮一凈陽。



 汾州,上。宋西河郡軍事,天會六年置汾陽軍節度使,後又置河東、南、北路提刑司。戶八萬七千一百二十七。縣五、鎮二:



 西河有謁泉山、比干山、文水、汾水。鎮一郭柵。



 孝義有勝水。



 介休有介山、汾水。鎮一洪山。



 平遙有鹿臺山、汾水。



 靈石貞祐
 三年割霍州,四年五月後來屬。有靜巖山、汾水。



 石州,上,刺史。舊昌化軍。興定五年復隸晉陽,從郭振之請也。戶三萬六千五百二十八。縣六、鎮四:



 離石倚。有胡公山、離石水。鎮一石窟。



 方山貞祐四年徙治於積翠山。有方山、赤洪水。



 孟門舊名定胡,明昌六年更。宋隸晉寧軍。有黃河、寧鄉水。鎮二吳保、天澤。



 溫泉貞祐四年五月改隸汾州。有遠望山、溫泉。



 臨泉宋隸晉寧軍。有黃河、臨泉水。鎮一克胡。



 寧鄉舊名平夷,明昌六年更。



 葭州,下,刺史。本晉寧軍,貞元元年隸汾州,大定二十二年升為晉寧州,二十四年更今名。在黃河西,興定二年五月以河東殘破,改隸延安府。戶八千八百六
 十四。寨八、堡九:神泉寨、永祚堡、烏龍寨、康定堡、寧河寨、寧河堡、太和寨、神木寨、通津堡、彌川寨、護川堡、強川堡、清川堡、通秦寨、通秦堡、晉安堡、吳堡寨。已上皆在黃河西,臨西夏界。



 代州,中。宋雁門郡防禦,天會六年置震武軍節度使。貞祐二年四月僑置西經略司,八月罷。戶五萬七千六百九十。縣五、鎮十三:



 鴈門倚。有夏屋山、鴈門山、滹沱水。鎮三雁門、西陘、胡谷。



 崞倚。有崞山、石鼓山、滹沱河、沙河。鎮一樓板。



 五臺貞祐四年三月升為臺州。有五臺山、蠙慮水。鎮二興善、石觜。



 廣武貞祐三年七月來屬。



 繁畤貞祐三年九月升為堅州。鎮七茹越、大石、義興、麻穀、瓶形、梅乃、寶興。



 庾州,下。中宋舊火山軍,大定二十二年升為火山州,後更今名。興定二年九月改隸嵐州,四年以殘破徙
 治于黃河灘許父寨。戶七千五百九十二。縣一、鎮一:



 河曲貞元元年置。有火山、黃河。鎮一鄴鎮。



 寧化州,下,刺史。本寧化軍,大定二十二年陞為州。戶六千一百。縣一、鎮一



 寧化鎮一窟谷。



 嵐州,下,宋舊樓煩郡軍事,天會六年置鎮西節度使。戶一萬七千五百五十七。縣三、鎮四:



 宜芳鎮一飛鳶。



 合河鎮三合河津、乳浪、鹽院渡。



 樓煩



 岢嵐州,下,刺史。本宋岢嵐軍,大定二十二年為州,貞祐三年九月升為防禦,四年正月升為節鎮,五月復防禦。戶五千八百五十一。縣一、堡一:



 嵐谷
 有岢嵐山、雪山、岢嵐水。堡一寒光。



 保德州,下,刺史。本宋保德軍,大定二十二年升為州,元光元年六月升為防禦。戶三千一百九十一。縣一:



 保德大定十一年置。有大堡津、沙谷津。



 管州,下,刺史。本宋憲州靜樂郡,天德三年更。興定三年升為防禦。戶五千八百八十一。縣一:



 靜樂



 河東南路,府二,領節鎮三,防禦一,刺郡六,縣六十八,鎮二十九,關六。



 平陽府,上。宋平陽郡建雄軍節度。本晉州,初為次府,置建雄軍節度使。天會六年升總管府。置轉運司。興
 定二年十二月以殘破降為散府。有書籍。產解鹽、隰州綠、卷子布、龍門椒、紫團參、甘草、蒼術。戶一十三萬六千九百三十六。縣十、鎮一:



 臨汾天會六年定臨汾為次赤,餘並次畿,置丞、簿、尉各一。有姑射山、平水、壺口山、汾水。



 襄陵倚。有浮山、汾水、橘水。鎮一故關。



 洪洞有霍山、汾水。



 趙城有姑射山、汾水、霍水。



 霍邑貞祐三年七月升為霍州,以趙城、汾西、靈石隸焉。興定元年七月升為節鎮,軍曰鎮定。有霍山、汾水、彘水。



 汾西有汾西山、汾水。



 岳陽有烏嶺山、通軍水。



 浮山舊名神山,大定七年更為浮山,興定四年更名曰忠孝。



 和川



 冀氏



 隰州,上,刺史。宋大寧郡。團練。舊大寧郡軍刺史,天會六年改為南隰州,以與北京隰州重也,天德三年去「南」字。戶二萬五千四百四十五。縣六、關四:



 隰
 川倚。有石馬山、石樓山。



 仵城興定五年正月升為隰川之午城鎮置。



 蒲興定五年正月升為蒲州,以大寧隸焉。有孤石山、橫木嶺。



 大寧有孔山、黃河、日斤水。關一馬門關。



 永和有樓山、黃河、仙芝水。關一永和關。



 石樓有石樓山、黃河、龍泉。關二永寧、上平關。



 吉州,下,刺史。宋置團練。舊名慈州,天德三年改為耿州,置文成郡軍,明昌元年更名吉。戶一萬三千三百二十四。縣二:



 吉鄉有壺口山、孟門山、黃河、蒲水。



 鄉寧



 河中府,散,上。宋河東郡。舊置護國軍節度使,天會六年降為蒲州,置防禦使。天德元年升為河中府,仍舊護國軍節度使。大定五年置陜西元帥府。戶十萬六千五百三十九。縣七、鎮四:



 河東倚。有中條山、五老山、黃河、
 媯水、汭水。鎮二永樂、合河。



 滎河貞祐三年升為滎州,以河津、萬泉隸焉。有黃河、汾水。睢丘。鎮一北郎。



 虞鄉有雷首山、中條山、壇道山。



 萬泉鎮一胡壁。



 臨晉有三疑山、黃河。



 河津



 猗氏有涑水。



 絳州,上。宋置絳郡防禦。天會六年絳陽軍節度使。興定二年十二月升為晉安府,總管河東南路兵馬,三年三月置河東南路轉運司。戶一十三萬一千五百一十。縣七、鎮五、關一:



 正平倚。劇。有定境山、汾水、澮水、鼓水。鎮一澤掌。



 曲沃劇。有絳山、絳水、汾水、澮水。鎮二柴村、九王。



 稷山有稷山、汾水。



 翼城興定四年七月升為翼州,以垣曲、絳縣隸焉。元光二年升為節鎮,軍曰翼安。有澮高山、清野山、烏嶺山。



 太平有汾水。



 垣曲有王屋山、清廉山、黃河、清山。鎮一皋落。關一行
 臺。



 絳有太陰山、教山、絳水。鎮一繪交。



 平水興定四年七月徙置汾河之西,從平陽公胡天作之請也。



 解州,上,刺史。宋慶成軍防禦,國初置解梁郡軍,後廢為刺郡。貞祐三年復升為節鎮,軍名寶昌。興定四年徙治平陸縣。戶七萬一千二百三十二。縣六、鎮四:



 解倚。有壇道山、鹽池。



 平陸有吳山、黃河。鎮一張店。



 芮城宋隸陜州。有中條山、黃河、龍泉。



 夏有巫咸山、中條山、淡水。鎮一曹張。



 安邑有中條山、稷山、鹽池、涑水。



 聞喜有九龍山、湯山、涑水。鎮二東鎮、劉莊。



 澤州,上,刺史。宋高平郡。天會六年以北京澤州同,加「南」字。天德三年復去「南」字。貞祐四年隸潞州昭義
 軍,後又改隸孟州。元光二年升為節鎮,軍曰忠昌。戶五萬九升四百一十六。縣六、鎮二:



 晉城倚。有太行山、丹水、白水、天井關。鎮二周村、巴公。舊又置星軺鎮。



 端氏有石門山、巨峻山。



 陵川有太行山、九仙山。



 陽城元光二年十一月升為績州。有王屋山、濩澤。



 高平有頭顱山、米山、丹水。



 沁水有鹿臺山、沁水、馬邑山。



 潞州,上。宋隆德府上黨郡昭德軍節度使。天會六年,節度使兼潞南沁觀察處置使。戶七九千二百三十二。縣八、鎮四:



 上黨倚。鎮一八義。



 壺關有抱犢山、紫團山、赤壤山。



 屯留有盤秀山、絳水。鎮一寺底。



 長子有羊頭山、發鳩山、堯水鎮一橫水。



 潞城有三垂山、伏牛山、潞水、漳水。



 襄垣有鹿臺山、涅水、漳水。鎮一示虎亭。



 黎城
 有白巖山、故壺口關。



 涉貞祐三年七月升為崇州,以黎城縣隸焉。四年八月以殘破復為縣。興定五年九月復升為州。有崇山、涉水。



 遼州,中,刺史。宋本樂平郡刺史,天會六年以與東京遼州同,加「南」字,天德三年復去「南」字。戶一萬五千八百五十。縣四、鎮一、關一:



 遼山倚。有箕山、青谷水。鎮一平城、舊縣也、貞元間廢為鎮,屬遼山縣,及廢舊芹泉鎮。關一黃澤。



 榆社有武鄉水、石勒漚麻池。



 和順有九原山。



 儀城舊為平城縣,貞元二年廢入遼山為鎮,貞祐四年復升為縣,更今名。



 沁州,中,刺史錦山郡。宋威勝軍,天會六年升為州。元光二年升為節鎮,軍曰義勝。戶一萬八千五十九。縣四、
 鎮一:



 銅鞮倚。有銅鞮山、石梯山、洹水、交水。



 武鄉有胡甲山、武鄉水。鎮一南關。



 沁原元光二年十一月升為穀州。有霍山、沁水。



 綿上有羊頭山、沁水。



 懷州,上,宋河內郡防禦,天會六年以與臨潢府懷州同、加「南」字,仍舊置沁南軍節度使,天德三年去「南」字。皇統三年閏四月置黃沁河堤都大管勾司。大定五年置行元帥府。興定五年置招撫司。戶八萬六千七百五十六。縣四、鎮六:



 河內倚。有太行陘、太行山、黃河、沁水、浿水。鎮四武德、柏鄉、萬善、清化。



 修武有濁鹿城。鎮一承恩。



 山陽興定四年以修武縣重泉村為山陽縣,隸輝州。



 武陟有太行山、天門山、黃河、沁水。鎮一宋郭。



 孟州,上。宋濟源郡節度,天會六年降河陽府為孟州,
 置防禦,守盟津。宣宗朝置經略司。戶四萬一千六百四十九。縣四、鎮二:



 河陽倚。有嶺山、黃河、湛水、同水。鎮二穀羅、沇河。



 王屋有王屋山、天壇山、析城山、黃河。



 濟源有太行山、孔山、濟水、澳水、沁水。



 溫有黃河、湋水。



 京兆府路,宋為永興軍路。皇統二年省併陜西六路為四,曰京兆,曰慶原,曰熙秦,曰鄜延。府一,領節鎮一,防禦一,刺郡四,縣三十六、鎮三十七。



 京兆府,上。宋京兆郡永興軍節度使。皇統二年置總管府,天德二置陜西路統軍司、陜西東路轉運司。產白芷、麻黃、白蒺藜、茴香、細辛。戶九萬八千一百七十七。縣十二、
 鎮十:舊又有中橋、臨涇二鎮,後廢。



 長安倚。有終南山、龍首山、灃水、渭水、鎬水。鎮一子午。



 咸寧倚。本萬年,後更名。泰和四年廢,尋復。鎮二鳴犢、乾祐。



 興平有渭水、醴泉。



 涇陽



 臨潼有驪山、渭水、戲水。鎮一零口。



 藍田有藍田山、匱山、灞水。



 雲陽鎮一孟店。



 高陵有涇水、渭水、白渠鎮二毗沙、渭城。



 終南宋清平軍。鎮一甘河。



 櫟陽有渭水、沮河、清泉陂。鎮一粟邑。



 鄠有終南山、牛首山、水美陂、渭水。鎮一秦渡。



 咸陽



 商州,下,刺史。宋上洛郡軍事。貞祐四年升為防禦,尋隸陜州,興定二年正月復來屬,元光二年五月改隸河南路。戶三千九百九十九。縣二、鎮一:舊又有西市、黃川、青雲三鎮,後廢。



 上洛有楚山、熊耳山、丹水、嶢關。鎮二商洛、豐陽,皆舊為縣,貞元二年廢為鎮。



 洛南有塚嶺山、洛水。



 虢州,下,刺史。宋虢郡軍事。貞祐二年割為陜州支郡,以備潼關。戶一萬二十二。縣三、鎮五:



 虢略有鹿蹄山、黃河。燭水。鎮三靖遠、玉城、硃陽。



 盧氏有硃陽山、熊耳山、洛水、鄢水。鎮二社管,欒川舊為縣,海陵貞元二年廢為鎮。



 朱陽海陵時當廢,後復置。有地肺山。



 乾州,中,刺史。宋嘗改為醴州,天德三年復。戶二萬六千八百五十六。縣四、鎮三:



 奉天有梁山、莫谷水、甘谷水。鎮一薛祿。



 醴泉有九峻山、浪水。鎮一甘北。



 武亭本武功,大定二十九年以嫌顯宗諱更。有敦物山、武功山、渭水。鎮一長寧。



 好畤有梁山、武亭河。



 同州,中,宋馮翊郡定國節度,治馮翊。後改安國軍節度使。舊貢圓筋繭耳羊,大定十一年罷之。戶三萬五千五百六十一。
 縣六、鎮九:



 馮翊倚。有洛水、渭水。鎮二沙苑並臨。



 朝邑有黃河、渭水。鎮四朝邑、新市、延祥、洿谷。



 白水有五龍山、洛水、白水。



 郃陽有非山、瀵水、黃河。鎮一夏陽。



 澄城有梁山、洛水。



 韓城貞祐三年為楨州,以郃陽縣隸焉。鎮二寺前、良輔。



 耀州,上,刺史。宋華原郡感德軍節度,皇統二年降為軍事,後為刺史州。戶五萬二百一十一。縣四、鎮二:



 華原有土門山、漆水、沮水。



 同官有白馬山、同官川。鎮一黃堡。



 美原有頻陽山。



 三原有堯門山、中白渠。鎮一龍橋。



 華州,中。宋華陰郡鎮潼軍節度,治鄭,國初因之,後置節度使,皇統二年降為防禦使。貞祐三年八月升為節鎮,軍曰金安,以商州為支郡。戶五萬三千八百。
 縣五、鎮六:



 鄭倚。有少華山、聖山、渭水、符禺水。鎮一赤水。



 華陰有太華山、松果山、黃河、渭水、潼關。鎮二關西、敷水。



 下邽有渭水、太白渠。鎮二素化、新市。



 蒲城有金粟山、洛水。鎮一荊姚。



 渭南有靈臺山、渭水。



 鳳翔路,宋秦鳳路,治秦州。府二、領防禦二,刺郡二,縣三十三,城一,堡四,寨十四,鎮十五。



 鳳翔府,中。宋扶風郡鳳翔軍節度。皇統二年升為府,軍名天興,大定十九年更軍名為鳳翔。大定二十七年升總管府。產芎藭、獨活、燈草、無心草、升麻、秦艽、骨碎補、羌活。戶六萬三千三百三。縣九、鎮四:舊有橫水、驛店、崔模、麻務,長清五鎮,後廢。



 鳳翔倚。有杜陽山、吳嶽、雍水。舊名天興縣,大定十九年更。



 寶雞有陳倉山、渭水、水幵水、大散關。鎮一
 武城。



 虢有楚山、渭水。鎮一陽平。



 郿有太白山、渭水。



 盭厔南至巡馬道二十里。貞祐四年升為恒州,以郿縣隸焉。有終南山、渭水、浴谷。



 扶風國初作扶興。有渭水、湋水。鎮一岐陽。



 岐山有岐山、終南山、渭水、姜水、水幵水。鎮一馬跡。



 普潤有杜水、漆水、岐山。



 麟遊有五將山、黝土山。



 德順州,上,刺史。宋德順軍,國初隸熙秦路,皇統二年升為州,大定二十七年來屬。貞祐四年四月升為防禦,十月升為節鎮,軍曰隴安。戶三萬五千四百四十九。縣六、寨四、堡一:舊有上接鎮、通安寨、王家城、牧龍城、同家堡、後廢。



 隴干倚。



 水洛本中安堡城。堡一中安。



 威戎本威戎堡城。



 隆德本隆德寨。



 通邊本通邊寨。寨三靜邊舊為縣,得勝,寧安。



 治平本治平寨。寨一
 懷遠。



 平涼府,散,中。宋渭州隴西郡平涼軍節度。舊為軍,後置陜西西路轉運司、陜西東、西路提刑司。大定二十六年來屬。戶三萬一千三十三。縣五、鎮五、寨一:



 平涼倚。有羊頭山、馬屯山。



 潘原有鳥鼠山、銅城山。



 崇信有閣川水。鎮一西赤城。



 華亭有小隴山。



 化平本名安化,大定七年更。鎮四安化、安國、白巖河、耀武。寨一瓦亭。



 鎮戎州,下,刺史。本鎮戎軍,大定二十二年為州,二十七年來屬。戶一萬四百四十七。縣二、堡三、寨八:



 東山本東山寨。



 三川本三川寨。堡三彭陽、乾興、開遠。寨八天聖、飛泉、熙寧、靈平、通峽、蕩羌、九羊、張義。



 秦州,下。宋天水郡雄武軍節度,後置秦鳳路。國初置節度,皇統二年置防禦使,隸熙秦路,大定二十七年來屬,元光二年四月升為節鎮,軍曰鎮遠,後罷,貞祐三年復置。戶四萬四百四十八。縣八、城一、寨三、鎮二:舊有甘谷城、甘泉城、結藏城、定西寨、西顧堡,後廢。



 成紀倚。有龍馬泉。



 冶坊



 甘谷



 清水宋舊縣。有中隴山、嶓冢山、清水。



 雞川



 隴城有大隴山、瓦亭山。寨一隴城。



 西寧貞祐四年十月升為西寧州,以甘谷、雞川、治平三縣隸焉。



 秦安城一伏羌。寨二三陽務、弓門。鎮二靜戎、褷穰。



 隴州,下,宋水幵陽郡,防禦。海陵時隸熙秦路,大定二十七年來屬。戶一萬六千四百四十二。縣三、鎮五:



 水幵陽倚。有水幵水、隃麋澤。鎮二安化、新興。



 水幵源有吳巖山、白環水。鎮三吳山、定戎、隴西。



 隴安泰和八年以隴安寨升。有秦嶺山、渭水。



 鄜延路,府一,領節鎮一,刺郡四,縣十六,鎮五,城二,堡四,寨十八,關二。



 延安府,下。宋延安郡彰武軍節度使,皇統二年置彰武軍總管府。戶八萬中九百九十四。縣七、寨四、堡一、鎮一。



 膚施倚。有五龍山、伏龍山、洛水、清水、濯巾水。鎮一樂盤。



 延川有濯巾水、黃河、吐延水。寨一永平。



 延長有獨戰山、濯巾水。



 臨真有庫利川。



 甘泉有洛水。



 敷政有三捶山、洛水。



 門山有重覆山、黃河、渭牙川水。堡二安定,置第六正將。安寨。寨四萬安,興定二年廢。德安,置第五副將。招安。永平,有丹陽驛。



 丹州,中,刺史。宋咸寧郡軍事,國初因之。戶一萬三千七十八。縣一、鎮一、關一:



 宜川有雲巖山、孟門山、黃河、庫利川。鎮一雲巖。關一烏仁。



 保安州,下,刺史。宋保安軍,大定二十二年升為州。戶七千三百四十。縣一、寨三、鎮二、堡一、城一:



 保安大定十二年以保安軍置。寨三德靖、順寧、平戎。鎮二靜邊、永和。堡一園林。城一金湯。



 綏德州,下,刺史。唐綏州,宋綏德軍,大定二十二年升為州。戶一萬二千七百二十。縣一、寨十、城一、堡一、關一



 清澗本宋清澗城,大定二十二年升。寨十暖泉,義合,清邊,
 臨夏,白草,米脂置第二將,懷寧,鎮邊,綏平,克戎置第四將。城一嗣武。堡一開光。關一永寧。



 鄜州,下。宋洛交郡康定軍節度,國初因之,置保大軍節度使。戶六萬二千九百三十一。縣四、鎮一:



 洛交倚。有疏屬山、洛水、華池水。鎮一三川。



 直羅有大盤山、羅川水。



 酈城有楊班湫。



 洛川有洛川水、圜水。



 坊州,中,刺史。宋中部郡軍事。戶二萬七百四十六。縣二、鎮一:



 中部有沮河、橋山、石堂山、洛水、蒲谷水。



 宜君有沮水。鎮一玉華。



 天會五年,元帥府宗翰、宗望奉詔伐宋,若克宋則割地以賜夏。及宋既克,乃分割楚、夏疆封,自麟府
 路洛陽溝距黃河西岸,西歷暖泉堡,鄜延路米脂谷至累勝寨,環慶路威邊寨踰九星原至委布穀,涇原路威川寨略古蕭關至北谷,秦鳳路通懷堡至古會州,自此距黃河,依見流分熙河路盡西邊,以限楚、夏之封,或指定地名有懸邈者,相地勢從便分畫。



 慶原路,舊作陜西西路。府一,領節鎮二,刺郡三,縣十八,鎮二十三,城二,堡四,寨二十二,邊將營八。



 慶陽府,中。宋安化郡慶陽軍節度。本慶州軍事,國初改安國軍,後置定安軍節度使兼總管,皇統二年置
 總管府。戶四萬六千一百七十一。縣三、城二、堡一、寨三、鎮七:



 安化倚。有馬嶺山、延慶水。



 彭原有彭池原、睦陽川。鎮二董志、赤城。



 合水有子午山。鎮五金櫃、懷安、業樂、五交、景山。城二白豹、大順。寨三安疆、華池、柔遠。堡一荔原。



 環州,上,刺史。宋軍事,國初因之,大定間升為刺郡。戶九千五百四。縣一、堡三、寨六、鎮三:



 通遠倚。有鹵河、馬嶺阪、塔子平榷場。堡三木瓜、歸德、興平。舊有惠丁、射香、流井三堡,後廢。寨六定邊、平遠、永和、洪德、烏倫、安邊。鎮三合道、馬嶺、木波。



 寧州,中,刺史。宋彭原郡興寧軍節度,國初因之,皇統二年降為軍,仍加「西」字,天德二年去「西」字,為刺郡。戶
 三萬四千七百五十七。縣四、鎮五:



 安定本名定安,大定七年更。倚。有洛水、九陵水。鎮一交城。



 定平鎮二棗社、大昌。



 真寧有子午山、羅川水。鎮二要關、山河。



 襄樂有延川水。



 邠州,中。宋新平郡靜難軍節度使,國初因之。戶四萬七千二百九十一。縣五、鎮三、寨一:



 新平倚。有涇水、潘水。



 淳化有仲山、車箱阪。



 宜祿有涇水、汭水。鎮一亭口。



 永壽宋隸醴州。有高泉山。鎮一永壽。舊有邵寨鎮,後割隸涇州。寨一常寧。



 三水有石門山、涇水、羅川水。鎮一清泉。



 原州,上,刺史。宋平涼郡軍事,大定二十七年為涇州支郡,後復軍事。戶一萬七千八百。縣二、鎮三、寨
 五:



 臨涇倚。有陽晉水、朝那水。



 彭陽有大湖河、蒲川河。鎮三蕭鎮、柳泉、新城。寨五綏寧、平寧、靖安、開邊、西壕。



 涇州,中,彰化軍節度使。本治涇川,元光二年徙治長武。戶二萬六千二百九十。縣四、寨一、鎮二:



 涇川本保定縣,大定七年更。寨一官地。



 長武



 良原



 靈臺鎮二百里、邵寨。



 邊將:



 第二將營,在荔原堡西,白豹城南七十五里,戶三千七百一十六。



 次西第四將營,戶一千二百三十二。



 次西第三將營,戶二千一百五。



 次西第
 八將營,戶一千二百二十二。



 次西第七將營,戶八百五十。



 次西第九將營,戶七百二十七。



 次西第六將營,戶九百八十九。



 次西第五將營,戶三百六十四。



 皇統六年,以德威城、西安州、定邊軍等沿邊地賜夏國,從所請也。正隆元年,命與夏國邊界對立烽候,以防侵軼。



 臨洮路,皇統二年改熙州為臨洮府,置熙秦路總管府,大定二十七年更今名。府一,領節鎮一,防禦一,刺郡四,縣一十三,鎮六,城六,堡十二,寨九,關二。



 臨洮府,中。宋舊熙州臨洮郡鎮洮軍節度,後更為德順軍,皇統二年置總管府。產甘草、庵珣子、大黃。戶一萬九升七百二十一。縣三、鎮一、城一、堡四:



 狄道有白石山、洮水、浩亹河。鎮一慶平。城一景骨。



 當川堡一通谷。



 康樂堡三渭源,臨洮,南川臨宋界。



 積石州,下,刺史。本宋積石這溪哥城,大定二十二年為州,戶五千一百八十五。縣一、城三、堡三:



 懷羌西至生羌界八十里。城三循化,西至生羌界一百里。大通,臨河、夏界。來羌,臨夏邊。堡三通津、臨灘、來同。



 洮州,下。宋嘗置團練。刺史。舊軍事。臨宋界,至西生羌
 界八十里。戶一萬一千三百三十七。堡二:通祐,臨宋界,無民戶,置軍守。鐵城,臨宋界,無民戶,置軍守。



 蘭州,上,刺史。宋金城郡軍事。戶一萬一千三百六十、鎮三、城二、堡三、關一:



 定遠兼第十將,去質孤堡一十五里。



 龕谷宋舊寨。



 阿干宋舊寨。城二寧遠、安羌。堡三東關、質孤,臨夏邊,兼第八將。西關,臨黃河、夏邊鎮三原川、豬觜、納米。關一京玉。



 鞏州,下,節度。宋通遠軍,皇統二年升軍事為通遠軍節度使。戶三萬六千三百一。縣五、寨四、鎮一:



 隴西宋舊縣。



 通渭



 定西貞祐四年六月升為州,以通西、安西隸焉。鎮一鹽川。舊有赤觜鎮,後廢。



 通西



 安西寨四熟羊,臨宋界。來遠,去宋界二十五里,
 舊為鎮。永寧,去宋界三十里。南川。舊有平西、寧遠二寨,及南三岔堡。



 會州,上,刺史。宋前舊名汝遮。戶八千九百一十八。縣一、舊有會川城。寨二關一:



 保川寨二平西、通安。關一會安,舊作會寧。



 河州,下,防禦,宋安鄉郡軍事。至都西千七百一十里。皇統二年升軍事為防禦,貞祐四年十月升為節鎮,軍曰平西。戶一萬四千九百四十二。縣二、城一、寨三、鎮一:



 枹罕國初廢,貞元二年復置。



 寧河城一安鄉關。寨三南川、通會關、定羌城。鎮
 一積慶。



\end{pinyinscope}