\article{本紀第七}

\begin{pinyinscope}

 世宗中



 十二年正月庚午朔,宋、高麗、夏遣使來賀。戊寅,詔有司:「凡陳言文字,皆國政利害,自今言有可行,以其本封送秘書監,當行者錄副付所司。」丙申,以水旱,免中都、西京、南京、河北、河東、山東、陜西去年租稅。



 二月壬寅,上召諸王府長史諭之曰:「朕選汝等,正欲勸導諸王,使之為善。如諸王所為有所未善,當力陳之,尚或不從,則具某日
 行某事以奏。若阿意不言,朕惟汝罪。」丙午,尚書省奏,廉察到同知城陽軍事山和尚等清強官,上曰:「此輩暗察明訪皆著政聲,可第其政績,各進官旌賞。其速議升除。」庚戌,上如順州春水。癸丑,還都。丙辰,詔:「自今官長不法,其僚佐不能糾正又不言上者,並坐之。」戶部尚書高德基濫支朝官俸錢四十萬貫,杖八十。



 三月己巳朔,萬春節,宋、高麗、夏遣使來賀。乙亥,詔尚書省:「贓汙之官,已被廉問,若仍舊職,必復害民。其遣使諸道,即日罷之。」丁丑,詔遣宿直將軍烏古論思列,冊封王皓為高麗國王。庚寅,雨土。癸巳,以前西北路招討使移剌道為參知政事。
 回紇遣使來貢。丁酉,北京曹貴等謀反,伏誅。



 四月,旱。癸卯,尚書右丞孟浩罷。丁巳,西北路納合七斤等謀反,伏誅。癸亥,以久旱,命鑄祠山川。詔宰臣曰:「諸府少尹多闕員,當選進士雖資敘未至而有政聲者,擢用之。」以宿直將軍唐括阿忽里為橫賜夏國使。乙丑,大名尹荊王文以贓罪奪王爵,降授德州防禦使。回紇使使來貢。丙寅,尚書右丞相紇石烈志寧薨。丁卯,宋、高麗遣使賀尊號。阻珝來貢。



 五月癸酉,上如百花川。甲戌,命賑山東東路胡剌溫猛安民飢。丁丑,次阻居。久旱而雨。戊寅,觀稼。禁扈從蹂踐民田。禁百官及承應人不得服純黃油衣。癸
 未,諭宰臣曰:「朕每次舍,凡就秣馬之具皆假於民間,多亡失不還其主。此彈壓官不職,可擇人代之。所過即令詢問,但亡失民間什物,並償其直。」乙酉,詔給西北路人戶牛。



 六月甲寅,如金蓮川。



 九月丙子,至自金蓮川。辛巳,以右副都點檢來谷清臣等為賀宋生日使,右衛將軍粘割斡特剌為夏國生日使。丁亥,太白晝見,在日前。鄜州李方等謀反,伏誅。



 十月,高麗國王王皓遣使謝封冊。乙未,臨奠故右丞相紇石烈志寧喪,志寧妻永安縣主進鎧甲、弓矢、鷹鶻、重彩。壬子,召皇太子及趙王永中上殿,上顧謂宰臣曰:「京嘗圖逆,今不除之,恐為後患。」又曰:「天
 下大器歸於有德。海陵失道,朕乃得之。但務修德,餘何足慮。」皇太子及永中皆曰:「誠如聖訓。」遂釋之。丙辰,以德州防禦使文貲產賜其兄之子咬住,且諭其母:「文之罪,汝等皆當連坐。念宋王有大功於國,故置不問,仍以家產賜汝子。」



 十一月甲戌,上謂宰臣曰:「宗室中有不任官事者,若不加恩澤,於親親之道,有所未弘。朕欲授以散官,量予廩祿,未知前代何如?」左丞石琚曰:「陶唐之親九族,周家之內睦九族,見於《詩》、《書》,皆帝王美事也。」丙子,上以曹國公主家奴犯事,宛平令劉彥弼杖之,主乃折辱令,既深責公主,又以臺臣徇勢偷安,畏忌不敢言,奪俸一
 月。以陜西統軍使璋為御史大夫。以戶部尚書曹望之為賀宋正旦使。壬午,同州民屈立等謀反,伏誅。戊子,上屏侍臣,與宰臣議事,記注官亦退,上曰:「史官記人君善惡,朕之言動及與卿等所議,皆當與知。其於記錄無或有隱,可以朕意諭之。」



 十二月乙未朔,以濟南尹劉萼在定武軍貪墨不道,命大理少卿張九思鞫之。丁酉,詔遣官及護衛二十人,分路選年二十以上四十以下有門地才行及善射者,充護衛,不得過百人。冀州王瓊等謀反,伏誅。德州防禦使文以謀反,伏誅。辛丑,出宮女二十餘人。己酉,樞密副使移剌成罷。辛亥,禁審錄官以宴
 飲廢公務。詔金、銀坑冶聽民開採,毋得收稅。癸丑,獵于近郊。以殿前都點檢徒單克寧為樞密副使。己未,詔自今除名人子孫有在仕者並取奏裁。



 十三年正月乙丑朔,宋、高麗、夏遣使來賀。癸酉,尚書省奏,南客車俊等因榷場貿易,誤犯邊界,罪當死。上曰:「本非故意,可免罪發還,毋令彼國知之,恐復治其罪。」詔有司嚴禁州縣坊里為民害者。



 閏月壬子,詔太子詹事曰:「東宮官屬尤當選用正人,如行檢不修及不稱職者,具以名聞。」辛酉,太白晝見。洛陽縣賊聚眾攻盧氏縣,殺縣令李庭才,亡入于宋。



 三月癸巳朔,萬春節,宋、高麗、夏遣
 使來賀。乙卯,上謂宰臣曰:「會寧乃國家興王之地,自海陵遷都永安,女直人浸忘舊風。朕時嘗見女直風俗,迄今不忘。今之燕飲音樂,皆習漢風,蓋以備禮也,非朕心所好。東宮不知女直風俗,第以朕故,猶尚存之。恐異時一變此風,非長久之計。甚欲一至會寧,使子孫得見舊俗,庶幾習效之。」太子詹事劉仲誨請增東宮牧人及張設,上曰:「東宮諸司局人自有常數,張設已具,尚何增益。太子生於富貴,易入於侈,惟當導以淳儉。朕自即位以來,服御器物,往往仍舊,卿以此意諭之。」



 四月己巳,定出繼子所繼財產不及本家者,以所繼與本家財產通數
 均分制。以有司言,特授洺州孝子劉政太子掌飲丞。乙亥,上御睿思殿,命歌者歌女直詞。顧謂皇太子及諸王曰:「朕思先朝所行之事,未嘗塹忘,故時聽此詞,亦欲令汝輩知之。汝輩自幼惟習漢人風俗,不知女直純實之風,至於文字語言,或不通曉,是忘本也。汝輩當體朕意,至於子孫,亦當遵朕教誡也。」辛巳,更定盜宗廟祭物法。



 五月壬辰朔,日有食之。戊戌,禁女直人毋得譯為漢姓。壬寅,真定尹孟浩薨。甲辰,尚書省癸,鄧州民范三毆殺人,當死,而親老無侍。上曰:「在醜不爭謂之孝,孝然後能養。斯人以一朝之忿忘其身,而有事親之心乎?可論如
 法,其親,官與養濟。」六月,樞密使完顏思敬薨。



 七月庚子,復以會寧府為上京。庚戌,罷歲課雉尾。



 八月丁卯,以判大興尹趙王永中為樞密使。詔賜諸猛安謀克廉能三等官賞。己卯,御史大夫璋罷。丙戌,以左副都點檢襄等為賀宋生日使。丁亥,秋獵。



 九月辛卯朔,以宿直將軍胡什賚為夏國生日使。辛亥,還都。大名府僧李智究等謀反,伏誅。



 十月乙丑,歲星晝見。丙子,以前南京留守唐括安禮為尚書右丞。



 十一月,以大興尹璋為賀宋正旦使,引進使大洞為高麗生日使。上謂宰臣曰:「外路正五品職事多闕員,何也?」太尉李石對曰:「資考少有及者。」上曰:「
 茍有賢能,當不次用之。」壬子,吏部尚書梁肅請禁奴婢服羅綺。上曰:「近已禁其服明金。行之以漸可也。且教化之行,當自貴近始。朕宮中服御,常自節約,舊服明金者,已減太半矣!近民間風俗,比正隆時聞稍淳儉,卿等當更務從儉素,使民知所效也。」



 十四年正月已丑朔,宋、高麗、夏遣使來賀。



 二月壬戌,以大興尹璋使宋有罪,杖百五十,除名,仍以所受禮物入官。丙寅,以刑部尚書梁肅等為宋詳問使。庚午,以太尉、尚書令李石為太保,致仕。戊寅,詔免去年被水旱百姓租稅。



 三月戊子朔,萬春節,宋、高麗、夏遣使來賀。甲午,上
 謂大臣曰:「海陵純尚吏事,當時宰執止以案牘為功。卿等當思經濟之術,不可狃於故常也。」又詔:「猛安謀克之民,今後不許殺生祈祭。若遇節辰及祭天日,許得飲會。自二月一日至八月終,並禁絕飲燕,亦不許赴會他所,恐妨農功。雖閒月亦不許痛飲,犯者抵罪。可遍諭之。」又命:「應衛士有不閑女直語者,並勒習學,仍自後不得漢語。」辛丑,太白、歲星晝見。甲辰,上更名雍,詔中外。丙辰,太白、歲星晝見,經天。



 四月乙丑,上諭宰臣曰:「聞愚民祈福,多建佛寺,雖已條禁,尚多犯者,宜申約束,無令徒費財用。」戊辰,有事于太廟,以皇太子攝行事。乙亥,以勸農副
 使完顏蒲涅為橫賜高麗使。上御垂拱殿,顧謂皇太子及親王曰:「人之行,莫大於孝弟,孝弟無不蒙天日之祐。汝等宜盡孝于父母,友于兄弟。自古兄弟之際,多因妻妾離間,以至相違。且妻者乃外屬耳,可比兄弟之親乎?若妻言是聽,而兄弟相違,甚非理也。汝等當以朕言常銘于心。」戊子,以樞密副使徒單克寧兼大興尹。



 五月丙戌朔,詳問使梁肅等還自宋。甲午,如金蓮川。



 六月乙未,太白晝見。



 八月丁已,次糺里舌。日中,白龍見御帳東小港中,須臾,乘雲雷而去。癸亥,獵於彌離補。己卯,太白晝見。



 九月丁亥,還都。乙未,以兵部尚書完顏讓等為賀宋
 生日使,宿直將軍崇肅為夏國生日使。癸卯,上退朝,謂侍臣曰:「朕自在潛邸及踐阼以至于今,於親屬舊知未嘗欺心有徇。近御史臺奏,樞密使永中嘗致書河南統軍使完顏仲,託以賣為。朕知而不問。朕之欺心,此一事耳,夙夜思之,其如有疾。」己酉,宋遣使報聘。



 十月乙卯朔,詔圖畫功臣二十人衍慶宮聖武殿之左右廡。



 十一月甲申朔,日有食之。丙申,御史中丞劉仲誨等為賀宋正旦使。戊戌,召尚食局使,諭之曰;「太官之食,皆民脂膏。日者品味太多,不可遍舉,徒為虛費。自今止進可口者數品而已。」戊申,以儀鸞局使曹士元為高麗生日使。



 十
 二月戊寅,以平章政事完顏守道為右丞相,樞密副使徒單克寧為平章政事。



 十五年正月。此下闕。



 七月丙午,粘拔恩與所部康里孛古等內附。



 九月戊子,至自金蓮川。辛卯,高麗西京留守趙位寵叛其君,請以慈悲嶺以西,鴨淥江以東四十於城內附,不納。丙申,幸新宮。



 閏月己酉朔,定應禁弓箭鎗刀路分品官家奴客旅等許帶弓箭制。上謂左丞相良弼曰:「今之在官者,須職位稱愜所望,然後始加勉力。其或稍不如意,則止以度日為務,是豈忠臣之道耶?」丁巳,又謂良弼曰:「海陵時,領省秉德、左丞相言皆有能名,然
 為政不務遠圖,止以苛刻為事。言及可喜等在會寧時,一月之間,杖而殺之者二十人,罪皆不至於死,於理可乎?海陵為人如虎,此輩尚欲以術數要之,以至賣直取死,得為能乎?」己未,以歸德尹完顏王祥等為賀宋生日使,符寶郎斜卯和尚為夏國生日使。辛酉,高麗國王奏告趙位寵伏誅,詔慰答之。詔親王、百官傔人所服紅紫改為黑紫。甲戌,詔年老之人毋注縣令。年老而任從政,其佐亦擇壯者參用。



 十月乙卯,冬獵。丁未,還都。十一月乙卯,上幸東宮。初,唐古部族節度使移剌毛得之子殺其妻而逃,上命捕之。至是,皇姑梁國公主請赦之。上謂
 宰臣曰:「公主婦人,不識典法,罪尚可恕。毛得請託至此,豈可貸宥?」不許。戊午,以右宣征使靖等為賀宋正旦使。甲子,太白晝見。戊辰,以宿直將軍阿典蒲魯虎為高麗生日使。



 十六年正月戊申朔,宋、高麗、夏遣使來賀。甲寅,詔免去年被水、旱路分租稅。甲子,詔宗屬未附玉牒者並與編次。丙寅,上與親王、宰執、從官從容論古今興廢事,曰:「經籍之興,其來久矣,垂教後世,無不盡善。今之學者,既能誦之,必須行之。然知而不能行者多矣,茍不能行,誦之何益?女直舊風最為純直,雖不知書,然其祭天地,敬親
 戚,尊耆老,接賓客,信朋友,禮意款曲,皆出自然,其善與古書所載無異。汝輩當習學之,舊風不可忘也!」戊辰,宮中火。庚午,上按鷹高橋,見道側醉人墮驢而臥,命左右扶而乘之,送至其家。辛未,皇姑邀上至私第,諸妃皆從,宴飲甚歡。公主每進酒,上立飲之。



 二月庚寅,皇子豳王妃徒單氏以姦,伏誅。己亥,平章政事單克寧罷,以女故。



 三月丙午朔,日有食之。是日,萬春節,改用明日,宋、高麗、夏遣使來賀。戊申,雨豆於臨潢之境。戊午,上御廣仁殿,皇太子、親王皆侍膳,上從容訓之曰:「大凡資用當務節省,如其有餘,可周親戚,勿妄費也。」因舉所御服曰:「此
 服已三年未嘗更換,尚爾完好,汝等宜識之。」壬申,復置吾都碗部禿里。



 四月丙戌,詔京府設學養土,及定宗室、宰相子程試等第。戊子,制商賈舟車不得用馬。以東京留守崇尹為樞密副使。壬寅,如金蓮川。



 五月戊申,南京宮殿火。甲寅,太白晝見。庚申,遣使禱雨靜寧山神,有頃而雨。



 六月,山東兩路蝗。



 七月壬子,夏津縣令移剌山往坐贓,伏誅。



 八月辛巳,次霹靂濼。



 九月乙巳,至自金蓮川。己酉,諭左丞相紇石烈良弼曰:「西邊自來不備儲蓄,其令所在和糴,以為緩急之備。」癸丑,以殿前都點檢蒲察通等為賀宋生日使,宿直將軍完顏覿古速為夏國生
 日使。諭左丞相良弼曰:「海陵非理殺戮臣下,甚可哀憫。其孛論出等遺骸,仰逐處訪求,官為收葬。」辛酉,以南京宮殿火,留守、轉運兩司皆抵罪。



 十月丙申,詔諭宰執曰:「諸王小字未嘗以女直語命之,今皆當更易,卿等擇名以上。」



 十一月壬寅朔,參知政事王蔚罷。尚書省奏,河北東路胡剌溫猛安所轄謀克孛術魯舍廝,參謀克讓其兄子蒲速列。上賢而從之,仍令議加舍廝恩賞。戊午,以同知宣徽院事劉珫等為賀宋正旦使。庚申,以吏部尚書張汝弼為參知政事。甲子,以粘割韓奴之子詳古為尚輦局直長,婁室為武器直長。初,韓奴被旨招契丹
 大石,後不知所終,至是因粘拔恩部長撒里雅寅特斯等來,詢知其死節之詳,故錄其後。遣兵部郎中移剌子元為高麗生日使。



 十二月壬申朔,詔諸科人出身四十年方注縣令,年歲太遠,今後仕及三十二年,別無負犯贓染追奪,便與縣令。丙子,詔諸流移人老病者,官與養濟。上諭宰臣曰:「凡已經奏斷事有未當,卿等勿謂已行,不為奏聞改正。朕以萬幾之繁,豈無一失?卿等但言之,朕當更改,必無吝也!」庚寅,定榷場香茶罪賞法。



 十七年正月壬寅朔,宋、高麗、夏遣使來賀。高麗并表謝不納趙位寵。丙午,有司奏,高麗所進玉帶乃石似玉者,
 上曰:「小國無能辨識者,誤以為玉耳!且人不易物,惟德其物,若復卻之,豈禮體耶?」戊申,詔於衍慶宮聖武殿西建世祖神御殿,東建太宗、睿宗神御殿。詔西北路招討司契丹民戶,其嘗叛亂者已行措置,其不與叛亂及放良奴隸可徙烏古里石壘部,令及春耕作。尚書省奏,吾都碗部體土胡魯雅里密斯請入獻,許之。庚戌,詔諸大臣家應請功臣號者,既不許其子孫自陳,吏部考功郎其詳考其勞績,當賜號者,即以聞。壬子,上謂宰臣曰:「宗室中年高者,往往未有官稱。其先皆有功於國,朕欲稍加以官,使有名位可稱,如何?」對曰:「親親報功,先王之令
 則。」丁巳,詔朝官嫁娶給假三日,不須申告。壬戌,詔宰臣:「海陵時,大臣無辜被戮家屬籍沒者,並釋為良。遼豫王、宋天水郡王被害子孫,各葬於廣寧、河南舊塋。」其後復昭:「天水郡王親屬於都北安葬外,咸平所寄骨殖,官為葬於本處。遼豫王親屬未入本塋者,亦遷祔之」。



 三月辛丑朔,宋、高麗、夏遣使來賀。辛亥,詔免河北、山東、陜西、河東、西京、遼東等十路去年被旱、蝗租稅。賑東京、婆速、曷速館三路。乙丑,尚書省奏,三路之粟,不能周給。上曰:「朕嘗語卿等,遇豐年即廣糴以備凶歉。卿等皆言天下倉廩盈溢。今欲賑濟,乃云不給。自古帝王皆以蓄積為國
 家長計,朕之積粟,豈欲獨用之耶?今既不給,可於鄰道取之以濟。自今預備,當以為常。」



 四月甲戌,制世襲猛安謀克若出仕者,雖年未及六十,欲令子孫襲者,聽。戊寅,諭宰臣曰:「郡縣之官雖以罪解,一二歲後,亦須再用。猛安謀克皆太祖創業之際於國勤勞有功之人,其世襲之官,不宜以小罪奪免。」戊子,以滕王府長史徒單烏者為橫賜高麗使。



 五月,尚書省奏,定皇家袒免以上親燕饗班次,並從唐制。癸卯,幸姚村澱,閱七品以下官及宗室子、諸局承應人射柳,賞有差。



 六月乙卯,謂宰臣曰:「朕年老矣!恐因一時喜怒,處置有所不當,卿等即當執奏,
 毋為面從,成朕之失。」乙未,以英王爽之子思列為忠順軍節度副使。爽入謝,上曰:「朕以卿疾故,特任卿子,所冀卿因喜而愈也。欲即加峻授,恐思列年幼,未閑政事。汝當訓之,使有善可觀,更當升擢。」



 七月壬子,尚書省奏,歲以羊三萬賜西北路戍兵,上問如何運致,宰臣不能對。上曰:「朕雖退朝,留凡政務,不遑安寧。卿等勿謂細事非帝王所宜問,以卿等於國家之事未嘗用心,故問之耳。」是月,大雨,河決。



 八月己巳,觀稼于近郊。壬申,以監察御察御史體察東北路官吏,輒受訟牒,為不稱職,笞之五十。庚辰,上謂宰臣曰:「今之在官者,同僚所見,事雖當理,必以
 為非,意謂從之則恐人謂政非己出。如此者多,朕甚不取。今觀大理寺所斷,雖制有正條,理不能行者別具情見,朕惟取其所長。夫為人之理,他人之善者從之,則可謂善矣。」壬午,上謂宰臣曰:「今在下僚豈無人材,但在上者不為汲引,惡其材勝己故耳。」丙戌,上謂御史中丞紇石烈邈曰:「臺臣糾察吏治之能否,務去其擾民,且冀其得賢也。今所至輒受訟牒,聽其妄告,使為政者如何則可也。」



 九月丁酉朔,日有食之。辛丑,封子永德為薛王。以右副都點檢完顏習尼烈等為賀宋生日使。癸卯,以兵部郎中石抹忽土為夏國生日使。戊申,秋獵。庚戌,歲星、
 熒惑、太白聚於尾。甲子,還都。



 十月己巳,夏國進百頭帳,詔卻之境上。癸酉,有司奏:「衍慶宮所畫功臣二十人,惟五人有謚,今考檢餘十五人功狀,擬定謚號以進。」詔可,詔以羊十萬付烏古里石壘部畜牧,其滋息以予貧民。丁丑,制諸猛安,父任別職,子須年二十五以上方許承襲。辛巳,上謂宰臣曰:「今在位不聞薦賢,何也?昔狄仁傑起自下僚,力扶唐祚,使既危而安,延數百年之永。仁傑雖賢,非婁師德何以自薦乎?」癸未,更護送罪人逃亡制。上謂宰臣曰:「近觀上封章者,殊無大利害。且古之諫者既忠於國,亦以求名,今之諫者為利而已。如戶部尚書
 曹望之、濟南尹梁肅皆上書言事,蓋覬覦執政耳,其於國政竟何所補。達官如此,況餘人乎!昔海陵南伐,太醫使祁宰極諫,至戮於市,此本朝以來一人而已。」丁亥,上命宰臣曰:「監察御史田忠孺嘗上書言事,今當升擢,以勵其餘。」



 十一月戊戌,以南京留守徒單克寧為平章政事。庚戌,上謂宰臣曰:「朕常恐重斂以困吾民,自今諸路差科之煩細者,亦具以聞。」有司奏,夏國進御帳使因邊臣懇求進入,乃許之。以尚書左丞石琚為平章政事。丙辰,以延安尹完顏蒲剌睹等為賀宋正旦使。



 十二月戊辰,以渤海舊俗男女婚娶多不以禮,必先攘竊以奔,詔
 禁絕之,犯者以姦論。以宿直將軍僕散懷忠為高麗生日使。己巳,太白晝見。壬申,以尚書右丞唐括安禮為左丞,殿前都點檢蒲察通為右丞。上謂宰執曰:「朕今年已五十有五,若年踰六十,雖欲有為,而莫之能矣!宜及朕之康強,其女直人猛安謀克及國家政事之未完,與夫法令之未一者,宜皆脩舉之。凡所施行,朕不為怠。」



 十八年正月丙申朔,宋、高麗、夏遣使來賀。壬寅,定殺異居周親奴婢、同居卑幼,輒殺奴婢及妻無罪而輒毆殺者罪。庚戌,修起居注移剌傑上書言:「每屏人議事,雖史官亦不與聞,無由紀錄。」上以問平章政事石琚、左丞唐
 括安禮,對曰:「古者,天子置史官於左右,言動必書,所以儆戒人君,庶幾有所畏也。」庚申,免中都、河北、河東、山東、河南、陜西等路前年被災租稅。壬戌,如春水。



 二月丙寅朔,次管莊。丙子,次華港。己丑,還宮。



 三月乙未朔,萬春節,宋、高麗、夏遣使來賀。乙巳,命戍邊女直人遇祭祀、婚嫁、節辰許自造酒。丁未,上謂宰執曰:「縣令之職最為親民,當得賢材用之。邇來犯法者眾,殊不聞有能者。比在春水,見石城、玉田兩縣令,皆年老,茍祿而已。畿甸尚爾,遠縣可知。」平章政事石琚對曰:」良鄉令焦旭、慶都令李伯達皆能吏,可任。」上曰:「審如卿言,可擢用之。」己酉,禁民間
 無得創興寺觀。獻州人殷小二等謀反,伏誅。



 四月己巳,上謂宰臣曰:「朕巡幸所至,必令體訪官吏臧否。向玉田知主簿石抹杳乃能吏也,可授本縣令。」己丑,以太子左贊善阿不罕德甫為橫賜夏國使。



 五月丙午,上如金蓮川。



 六月庚午,尚書左丞相紇石烈良弼薨。



 閏月辛丑,命賑西南、西北兩招討司民,及烏古里石壘部轉戶饑。



 七月丙子,上謂宰臣曰:「職官始犯贓罪,容有過誤,至於再犯,是無改過之心。自今再犯不以贓數多寡,並除名。」



 八月乙巳,至自金蓮川。丙辰,以尚書右丞相完顏守道為左丞相,平章政事石琚為右丞相。



 九月辛未,以大理卿
 張九思等為賀宋生日使,侍御史完顏蒲魯虎為夏國生日使。癸酉,以尚書左丞唐括安禮為平章政事。乙亥,以右丞蒲察通為左丞,參知政事移剌道為右丞,刑部尚書粘割斡特剌為參知政事。



 十月庚寅朔,陜州防禦使石抹靳家奴以罪除名。甲午,御史中丞劉仲誨、侍御史李瑜坐失糾察大長公主事,各削官一階。



 十一月庚申朔,尚書省奏,擬同知永寧軍節度使事阿可為刺史,上曰:「阿可年幼,於事未練,授佐貳官可也。」平章政事唐括安禮奏曰:「臣等以阿可宗室,故擬是職。」上曰:「郡守係千里休戚,安可不擇人而私其親耶?若以親親之恩,賜
 與雖厚,無害於政。使之治郡而非其才,一境何賴焉。」壬申,以靜難軍節度使烏延查剌等為賀宋正旦使。丙子,尚書省奏,崇信縣令石安節買車材於部民,三日不償其直,當削官一階,解職。上因言:「凡在官者,但當取其貪污與清白之尤者數人黜陟之,則人自知懲勸矣。夫朝廷之政,太寬則人不知懼,太猛則小玷亦將不免於罪,惟當用中典耳。」戊寅,上責宰臣曰:「近問趙承元何故再任,卿等言,曹王嘗遣人言其才能幹敏,故再任之。官爵擬注,雖由卿輩,予奪之權,當出于朕。曹王之言尚從之,假皇太子有所諭,則其從可知矣。此事因卿言始知,其
 不知者知復幾何?且卿等公受請屬,可乎?」蓋承元前為曹王府文學,與王邸婢姦,杖百五十除名,而復用也。丙戌,以吏部尚書烏古論元忠為御史大夫,以東上閣門使左光慶為高麗生日使。



 十二月庚戌,封孫吾都補溫國公,麻達葛金源郡王,承慶道國公。壬子,群臣奉上「大金受命萬世之寶。」



 十九年正月庚申朔,宋、高麗、夏遣使來賀。丁卯,如春水。



 二月己酉,還宮。乙卯,免去年被水旱民田租稅。



 三月乙未朔,萬春節,宋、高麗、夏遣使來賀。乙丑,尚書省奏,虧課院務官顏葵等六十八人,各合削官一階。上曰:「以承廢
 人主榷沽,此遼法也。法弊則當更張,唐、宋法有可行者則行之。」己巳,上與宰臣論史事,且曰:「朕觀前史多溢美。大抵史書載事貴實,不必浮辭諂諛也。」辛未,上謂宰臣曰:「姦邪之臣,欲有規求,往往私其黨與,不肯明言,託以他事,陽不與而陰為之力。朕觀古之姦人,當國家建儲之時,恐其聰明不利於己,往往風以陰事,破壞其議,惟擇昏懦者立之,冀他日可弄權為功利也。如晉武欲立其弟,而姦臣沮之,竟立惠帝,以致喪亂,此明驗也。」丁丑,上謂宰臣曰:「朕觀前代人臣將諫於朝,與父母妻子決,示以必死。同列目睹其死,亦不顧身,又為之諫。此盡忠
 於國者,人所難能也。」己卯,制糾彈之官知有犯法而不舉者,減犯人罪一等科之,關親者許回避。上謂宰臣曰:「人多奉釋老,意欲徼福。朕蚤年亦頗惑之,旋悟其非。且上天立君,使之治民,若盤樂怠忽,欲以僥倖祈福,難矣!果能愛養下民,上當天心,福必報之。」四月己丑朔,詔賑西南路招討司所部民。己酉,以升祔閔宗,詔中外。丁巳,歲星晝見。五月戊寅,幸太寧宮。六月戊子朔,詔更定制條。七月辛未,有司奏擬趙王子石古乃人從,上不從,謂宰相曰:「兒罪尚幼,若奉承太過,使侈心滋大,卒難節抑,此不可長。諸兒每入侍,當其語笑娛樂之際,朕必淵默,
 蒞之以嚴,庶其知朕教戒之意,使常畏慎而寡過也。」癸酉,密州民許通等謀反,伏誅。丙子,太白晝見。庚辰,至自太寧宮。



 八月壬辰,尚書右丞相石琚致仕。戊戌,以宋大觀錢當五用。丙午,濟南民劉溪忠謀反,伏誅。



 九月戊午,以左宣徽使蒲察鼎壽等為賀宋生日使,太子左衛率府率裴滿胡剌為夏國生日使。癸亥,秋獵。癸未,還都。



 十月辛卯,西南路招討使哲典以贓罪,伏誅。辛亥,制知情服內成親者,雖自首仍依律坐之。



 十一月壬戌,改葬昭德皇后,大赦。以御史中丞移剌綎等為賀宋正旦使。戊辰,以西上閣門使盧拱為高麗生日使。壬申,上如河間
 冬獵。癸未,至自河間。



 二十年正月甲寅朔,宋、高麗、夏遣使來賀。戊午,定試令史格。壬戌,命歲以錢五千貫造隨朝百官節酒及冰、燭、藥、炭,視品秩給之。己巳,如春水。丙子,幸石城縣行宮。丁丑,以玉田縣行宮之地偏林為御林,大澱濼為長春澱。



 二月丁未,還都。



 三月癸丑朔,萬春節,宋、高麗、夏遣使來賀。己未,詔凡犯罪被問之官,雖遇赦,不得復職。乙丑,以新定猛安謀克,詔免中都、西京、河北、山東、河東、陜西路去年租稅。辛巳,以平章政事徒單克寧為尚書右丞相,御史大夫烏古論元忠為平章政事。



 四月丁亥,定冒蔭
 罪賞。己亥,制宗室及外并一品命婦,衣服聽用明金。以西上閣門使郭喜國為橫賜高麗使。太寧宮火。乙巳,上謂宰臣曰:「女直官多謂朕食用太儉,朕謂不然。夫一食多費,豈為美事。況朕年高,不欲屠宰物命。貴為天子,能自節約,亦不惡也。朕服御或舊,常使浣濯,至于破碎,方始更易。向時帳幕常用塗金為飾,今則不爾,但令足用,何必事紛華也。」庚戌,如金蓮川。



 五月丙寅,京師地震,生黑白毛。



 七月,旱。



 八月壬午,秋獵。



 九月壬戌,至自金蓮川。以太府監李佾等為賀宋生日使,少府少監賽補為夏國生日使。丙子,蒲速宛群牧老忽謀叛,伏誅。



 十月庚
 辰朔,更定銓注縣令丞簿格。詔西北路招討司每進馬駝鷹鶻等,輒率斂部內,自今並罷之。壬午,上謂宰臣曰:「察問細微,非人君之體,朕亦知之。然以卿等殊不用心,故時或察問。如山後之地,皆為親王、公主、權勢之家所占,轉和於民,皆由卿等之不察。卿等當盡心勤事,毋令朕之煩勞也。」詔徙遙落河、移馬河兩猛安於大名、東平等路安置。戊戌,上謂宰臣曰:「凡人在下位,欲冀升進,勉為公廉賢肖何以知之。及其通顯,觀其施為,方見本心。如招討哲典,初任定州同知,繼為都司,未嘗少有私徇,所至皆有清名,及為招討,不固守。人心險于山川,誠
 難知也。」壬寅,上謂宰臣曰:「近覽《資治通鑑》,編次累代廢興,甚有鑒戒,司馬光用心如此,古之良史無以加也。校書郎毛麾,朕屢問以事,善於應對,真該博老儒,可除太常職事,以備討論。」甲辰,以殿前都點檢襄為御史大夫。



 十一月丁巳,尚書右丞移剌道罷。乙丑,以真定尹單守素等為賀宋正旦使。癸酉,以御史大夫襄為尚書右丞。乙亥,上諭宰臣曰:「郡守選人,資考雖未及,廉能者則升用之,以勵其餘。」以太常少卿任倜為高麗生日使。



 十二月辛巳,上謂宰臣曰:「岐國用人,但一言合意便升用之,一言之失便責罰之。凡人言辭,一得一失,賢者不免。
 自古用人咸試以事,若止以奏對之間,安能知人賢否?朕之取人,眾所與者用之,不以獨見為是也。」己亥,河決衛州。辛丑,獵於近郊。癸卯,特授襲封衍聖公孔總兗州曲阜令,封爵如故。



\end{pinyinscope}