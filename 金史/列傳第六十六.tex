\article{列傳第六十六}

\begin{pinyinscope}

 循吏



 ○盧克忠牛德昌范承吉王政張奕李瞻劉敏行傅慎微劉煥高昌福孫德淵趙鑑蒲察鄭留女奚烈守愚石抹元張彀趙重福武都
 紇石烈德張特立王浩



 金自穆宗號令諸部不得稱都孛堇,於是諸部始列於統屬。太祖命三百戶為謀克,十謀克為猛安,一如郡縣置吏之法。太宗既有中原,申畫封疆,分建守令。熙宗遣廉察之使循行四方。世宗承海陵凋之餘,休養生息,迄於明昌、承安之間,民物滋殖,循吏迭出焉。泰和用兵,郡縣多故,吏治衰矣。宣宗尚刀筆之習,嚴考核之法,能吏不乏,而豈弟之政罕見稱述焉。金百餘年吏治,始終可攷,於是作《循吏傳》。



 盧克忠,貴德州奉集人。高永昌據遼陽,克忠走詣金源
 郡王斡魯營降,遂以撒屋出為鄉導。斡魯克東京,永昌走長松島,克忠與渤海人撻不也追獲之。收國二年,授世襲謀克。其後,定燕伐宋皆與有功,除登州刺史,改刺澶州。天德間,同知保大軍節度使。綏德州軍卒數人道過鄜城,求宿民家,是夜有賊剽主人財而去。有司執假宿之卒,繫獄榜掠誣服。克忠察其冤,獨不肯署,未幾果得賊,假宿之卒遂釋。大定二年,除北京副留守。會民艱食,克忠下令凡民有蓄積者計留一歲,悉平其價糴之,由是無捐瘠之患。轉陳州防禦使,後以靜難軍節度使致仕,卒。



 牛德昌,字彥欽,蔚州定安人。父鐸,遼將作大監。德昌少孤,其母教之學,有勸以就蔭者,其母曰:「大監遺命不使作承奉也。」中皇統二年進士第,調礬山簿。遷萬泉令。屬蒲、陜薦饑,群盜充斥,州縣城門晝閉。德昌到官,即日開城門縱百姓出入,榜曰:「民苦饑寒,剽掠鄉聚以偷旦夕之命,甚可憐也。能自新者一不問。」賊皆感激解散,縣境以安。府尹王伯龍嘉之,禮待甚厚。累官刑部、吏部侍郎,中都路都轉運使,廣寧、太原尹。卒,贈中奉大夫。



 范承吉,字寵之。好學問,屬遼季盜賊起,雖避地未嘗廢書。天慶八年中進士丙科,授秘書省校書郎,至大定府
 金源令。歸朝為御前承應文字。天會初,遷殿中少監。四年,從攻太原,遷少府監。五年,宗翰克宋,所得金珠承吉司其出入,無毫髮欺,及還,犢車載書史而已,尋遷昭文館直學士,知絳州。



 先是,軍興,民有為將士所掠而逃歸者,承吉使吏遍諭,俾其自實,凡數千人,具白元帥府,許自贖為良,或貧無貲者以公廚代輸。六年,改河東北路轉運使。時承宋季之弊,民賦繁重失當,承吉乃為經畫,立法簡便,所入增十數萬斛,官既足而民有餘。歷同知平陽尹、西京副留守,遷河東南路轉運使,改同簽燕京留守事、順天軍節度使,屬地震壞民廬舍,有欲爭先營葺
 者,工匠過取其直,承吉命官屬董其役,先後以次,不間貧富,民賴以省費。



 歷鎮西軍節度使、行臺禮部尚書、泰寧軍節度使,復鎮順天。奚卒散居境內,率數千人為盜,承吉繩以法不少貸,懼而不敢犯。貞元二年,以光祿大夫致仕,卒年六十六。



 王政,辰州熊岳人也。其先仕渤海及遼,皆有顯者。政當遼季亂,浮沈州里。高永昌據遼東,知政材略,欲用之。政度其無成,辭謝不就。永昌敗,渤海人爭縛永昌以為功,政獨逡巡引退。吳王闍母聞而異之,言於太祖,授盧州渤海軍謀克。從破白霫,下燕雲。及金兵伐宋,滑州降,留
 政為安撫使。前此數州既降,復殺守將反為宋守,及是人以為政憂。政曰:「茍利國家,雖死何避。」宋王宗望壯之,曰:「身沒王事,利及子孫,汝言是也。」政從數騎入州。是時,民多以饑為盜,坐繫。政皆釋之,發倉廩以賑貧乏,於是州民皆悅,不復叛。傍郡聞之,亦多降者。宋王召政至轅門,撫其背曰:「吾以汝為死矣,乃復成功耶。」慰諭者久之。



 天會四年,為燕京都曲院同監。未幾,除同知金勝軍節度使事。改權侍衛親軍都指揮使、兼掌軍資。是時,軍旅始定,管庫紀綱未立,掌吏皆因緣為姦。政獨明會計,嚴局鐍,金帛山積而出納無錙銖之失。吳王闍母戲之曰:「
 汝為官久矣,而貧不加富何也?」對曰:「政以楊震四知自守,安得不貧。」吳王笑曰:「前言戲之耳。」以黃金百兩、銀五百兩及所乘馬遺之。六年,授左監門將軍,歷安州刺史、檀州軍州事、戶吏房主事。天眷元年,遷保靜軍節度使,致仕卒,年六十六。



 政本名南撒里,嘗使高麗,因改名政。子遵仕、遵義、遵古。遵古子庭筠有傳。



 張奕,字彥微,其先澤州高平人。以廕補官,仕齊為歸德府通判。齊國廢,齊兵之在郡者二萬人謀為亂,約夜半舉燎相應。奕知之,選市人丁壯授以兵,結陣扼其要巷,開小南門以示生路,亂不得作,比明亡匿略盡,擒其首
 惡誅之。後五日,都統完顏阿魯補以軍至歸德,欲根株餘黨,奕以闔門保郡人無他,遂止。行臺承制除同知歸德尹。



 天眷元年,以河南與宋,改同知沂州防禦使事。三年,宗弼復取河南,徵奕赴行省,既定汴京,授汴京副留守。歷陳、秦州防禦使,同知太原尹。晉寧軍報夏人侵界,詔奕往征之。奕至境上,按籍各歸所侵土,還奏曰:「折氏世守麟府,以抗夏人。本朝有其地遂以與夏。夏人夷折氏墳壟而戮其屍,折氏怨入骨髓而不得報也。今復使守晉寧,故激怒夏人使為鼠侵,而條上其罪,茍欲開邊釁以雪私仇耳。獨可徙折氏他郡,則夏人自安。」朝廷從之,
 遂移折氏守青州。正隆間,同知西京留守事,遷河東北路轉運使。大定二年,徵為戶部尚書,甫視事,得疾卒。



 李瞻,薊州玉田人。遼天慶二年進士,為平州望雲令。張覺據平州叛,以瞻從事。宗望復平州,覺亡去,城中復叛,瞻踰城出降,其子不能出,為賊所害。宋王宗望嘉之。承制以為興平府判官。天會三年,遷大理少卿,從宗望南伐,為漢軍糧料使。四年,金兵圍汴,宋人請割河北三鎮,瞻與禮部侍郎李天翼安撫河北東、西兩路,略定懷、濬、衛等州,衛、湯陰等縣。七年,知寧州,累遷德州防禦使。為政寬平,民懷其惠,相率詣京師請留者數百千人。貞元
 三年,遷濟州路轉運使,改忠順軍節度使。正隆末,盜賊蜂起,瞻增築城壘為備,蔚人賴之以安。大定初,卒于官。



 劉敏行,平州人。登天會三年進士。除太子校書郎,累遷肥鄉令。歲大饑,盜賊掠人為食。諸縣老弱入保郡城,不敢耕種,農事廢,畎畝荒蕪。敏行白州,借軍士三十護縣民出耕,多張旗幟為疑兵,敏行率軍巡邏,日暮則閱民入城,由是盜不敢犯而耕稼滋殖。轉高平令。縣城圮壞久不修,大盜橫恣,掠縣鎮不能御。敏行出己俸,率僚吏出錢顧役繕治,百姓欣然從之,凡用二千人,版築遂完。鄉村百姓入保,賊至不能犯。凡九遷,為河北東路轉運使。
 致仕。卒。



 傅慎微,字幾先。其先秦州沙溪人,後徙建昌。慎微遷居長安。宋末登進士,累官河東路經制使。宗翰已克汴京,使婁室定陜西,慎微率眾迎戰,兵敗被獲,送至元帥府。元帥宗翰愛其才學,弗殺,羈置歸化州,希尹收置門下。宗弼復取河南地,起為陜西經略使,尋權同州節度使事。明年,陜西大旱,饑死者十七八,以慎微為京兆、鄜延、環慶三路經濟使,許以便宜。慎微募民入粟,得二十餘萬石,立養濟院飼餓者,全活甚眾。改同知京兆尹,權陜西諸路轉運使。復修三白、龍首等渠以溉田,募民屯種,
 貸牛及種子以濟之,民賴其利。轉中京副留守,用廉,改忻州刺史,累遷太常卿,除定武軍節度使,移靜難軍,忤用事者,蘇保衡救之得免。大定初,復為太常卿,遷禮部尚書,與翰林侍講學士徒單子溫、翰林待制移剌熙載俱兼同修國史。卒官,年七十六。



 慎微博學喜著書,嘗奏《興亡金鏡錄》一百卷。性純質,篤古喜談兵,時人以為迂闊云。



 劉煥,字德文,中山人。宋末起兵,城中久乏食,煥尚幼,煮糠核而食之,自飲其清者,以醲厚者供其母,鄉里異之。稍長就學,天寒擁糞火讀書不怠。登天德元年進士。調
 任丘尉。縣令貪污,煥每規正之,秩滿,令持盃酒謝曰:「尉廉慎,使我獲考。」調中都市令。樞密使僕散忽土家有絳結工,牟利於市,不肯從市籍役,煥繫之。忽土召煥,煥不往,暴工罪而笞之。煥初除市令,過謝鄉人吏部侍郎石琚,琚不悅曰:「京師浩穰,不與外郡同,棄簡就煩,吾所不曉也。」至是,始重之。



 以廉升京兆推官,再遷北京警巡使。捕二惡少杖于庭中,戒之曰:「孝弟敬慎,則為君子。暴戾隱賊,則為小人。自今以往,毋狃于故習,國有明罰,吾不得私也。」自是,眾皆畏憚,毋敢犯者。召為監察御使,父老數百人或臥車下,或挽其靴鐙,曰:「我欲復留使君期年,
 不可得也。」



 以本官攝戶部員外郎,代州錢監雜青銅鑄錢,錢色惡,類鐵錢。民間盜鑄,抵罪者眾,朝廷患之,下尚書省議。煥奏曰:「錢寶純用黃銅精治之,中濡以錫,若青銅可鑄,歷代無緣不用。自代州取二分與四六分,青黃雜糅,務省銅而功易就。由是,民間盜鑄,陷罪者眾,非朝廷意也。必欲為天下利,宜純用黃銅,得數少而利遠。其新錢已流行者,宜驗數輸納準換。」從之。



 再遷管州刺史,耆老數百人疏其著跡十一事,詣節鎮請留煥,曰:「刺史守職奉法,乞留之。」以廉升鄭州防禦使,遷官一階,轉同知北京留守事。世宗幸上京,所過州郡大發民夫治橋
 梁馳道,以希恩賞,煥所部惟平治端好而已。上嘉其意,遷遼東路轉運使,卒。



 高昌福,中都宛平人。父履,遼御史中丞致仕,太宗聞其名召之,未及入見而卒,特詔昌福釋服應舉。登天會十年進士第,補樞密院令史。明年,辟元帥府令史。皇統初,宗弼復河南,元帥府治汴,人有疑似被獲,皆目為宋諜者,即殺之。昌福讞得其實,釋去者甚眾。許州都統韓常用法嚴,好殺人,遣介送囚於汴,或道亡,監吏自度失囚恐得罪,欲盡殺諸囚以滅口。昌福識監吏意,窮竟其狀,免死者十七八,而諸吏遂怨昌福,欲構害之。是時方用
 兵,梁、楚間夜多陰雨,元帥府選人偵宋兵動靜,諸吏遣昌福。昌福不辭即行,盡得敵軍虛實報元帥府。師還,除震武軍節度副使,轉行臺禮部員外郎。天德間,行臺罷,改絳陽軍節度副使,入為兵部員外郎,改河間少尹。



 世宗即位,上書陳便宜事,上披閱再三,因謂侍臣曰:「內外官皆上書言事,可以知人材優劣,不然,朕何由知之。」三除同知東京留守事,治最,遷山東西路轉運使、工部尚書,改彰德軍節度使。上書言賦稅太重,上問翰林學士張景仁曰:「稅法比近代為輕,而以為重何也?」景仁曰:「今之稅殊輕,若復輕之,國用且不足。」事遂寢。累遷河中尹,
 致仕,卒。



 孫德淵,字資深,興中府人也。大定十六年進士,調石州軍事判官、淶水丞,察廉遷沙河令。有盜秋桑者,主逐捕之,盜以叉自刺其足面,曰:「秋桑例不禁採,汝何得刺我?」主懼,賂而求免,盜不從,訴之縣。德淵曰:「若逐捕而傷,瘡必在後,今在前,乃自刺也。」盜遂引服。選尚書省令史,不就。丁父憂去官,民為刻石祠之。察廉,起復北京轉運司都勾判官,以累薦遷中都左警巡使、監察御史、山東東路轉運副使,累官大理丞、兼左拾遺。審官院奏德淵剛正幹能,可任繁劇,遂再任。丁母憂,服除特遷恩州刺史,
 入為右司郎中,滕州刺史,遷同知河間府事,歷大興治中、同知府事。大安初,遷盤安軍節度使,改河北西路按察轉運使,改昭義軍節度使。潞州破被執,俄有拜于前者,皆沙河舊民也,密護德淵,由是得脫。貞祐二年,拜工部尚書,攝御史中丞。是時,山東乏兵食,有司請鬻恩例舉人,居喪者亦許納錢就試。德淵奏,此大傷名教,事遂寢。尋致仕。監察御史許古論德淵「忠亮明敏,可以大用,近許告老,士大夫竊歎,望朝廷起復,必能建明以利國家。」宣宗嘉納。未及用而卒。



 趙鑒,字擇善,濟南章丘人。宋建炎二年進士,調廬州司
 理參軍。是時江、淮方用兵,鑑棄官還鄉里。齊國建,除歷城丞,轉長清令,皆劇邑難治,鑑政甚著。劉豫召見,遷直秘閣、提舉涇原路弓箭手、兼提點本路刑獄公事,誡之曰:「邊將多不法,可痛繩之。」原州守將武悍自用,以鑑年少易之,鑑發其奸,守將坐免,郡縣聞風無敢犯者。齊廢,除知城陽軍,改山東東路轉運副使,攝行臺左司郎中。行臺宰相欲以故宋宦者權都水監,鑑曰:「誤國閹豎,汴人視為寇仇,付以美官,將失人望。」遂不用。以母憂解職,天德初,起為濟州刺史,移涿州。海陵召鑒入朝,應對失旨,遣還郡,俄除知火山軍,以病免。大定初,起知寧海軍。
 秋禾方熟,子方蟲生,鑑出城行視,蟲乃自死。再遷鎮西軍節度使,改河北西路轉運使,致仕,卒。



 蒲察鄭留,字文叔,東京路斡底必剌猛安人。大定二十二年進士,調高苑主薄、濬州司候,補尚書省令史,除鑒察御史,累遷北京、臨潢按察副使、戶部侍郎。御史臺奏鄭留前任北京稱職,遷陜西路按察使,改順義軍節度使。西京人李安兄弟爭財,府縣不能決,按察司移鄭留平理。月餘不問,會釋奠孔子廟,鄭留乃引安兄弟與諸生敘齒,列坐會酒,陳說古之友悌數事。安兄弟感悟,謝曰:「節使父母也,誓不復爭。」乃相讓而歸。朔州多盜,鄭留
 禁絕游食,多蓄兵器,因行春撫諭之,盜乃衰息,獄空。賜錫宴錢以褒之。改利涉軍節度使。詔括馬,鄭留使百姓飼養以須,御史劾之。既而伐宋,諸語括馬皆瘦,惟隆州馬肥,乃釋鄭留。大安初,徙安國軍。二年,知慶陽府事。三年,夏人犯邊,鄭留擊走之。至寧元年,改知平涼府。是時,平涼新被兵,夏人復來攻,鄭留招潰卒為禦守計,夏兵退,遷官四階。貞祐二年,改東京留守,致仕。貞祐四年,卒。



 鄭留重厚寡言笑,人不見其喜慍,臨終取奏稿盡焚之。



 女奚烈守愚,字仲晦,本名胡里改門,真定府路吾直克猛安人也。六歲知讀書。既齔,或謂食肉昏神識,乃戒而
 不食。性至孝,父沒時年十五,營葬如禮,治家有法,鄉人稱之。中明昌二年進士。調深澤主簿,治有聲。遷懷仁令,改弘文校理,秩滿為臨沂令。有不逞輩五百人,結為黨社,大擾境內,守愚下車,其黨散去。蝗起莒、密間,獨不入臨沂境。先是,朝廷括河朔、山東地,隱匿者沒入宮。告者給賞。莒州刺史教其奴告臨沂人冒地,積賞錢三百萬,先給官鏹乃徵于民,民甚苦之。守愚列其冤狀白州,州不為理,即聞於戶部而徵還之,流民歸業,縣人勒其事於石。



 改秘書郎。母喪,勺飲不入口三日,終喪未嘗至內寢。太常寺、勸農司交辟守愚,皆不聽,服除,除同知登聞
 檢院,改著作郎、永定軍節度副使。泰和伐宋,守愚為山東行六部員外郎,改大興都總管判官。大安元年,除修起居注,轉刑部員外郎、戶部郎中、太子左諭德。貞祐初,除戶部侍郎,數月拜諫議大夫、提點近侍局。二年,除保大軍節度使,改翰林學士、參議陜西路安撫司事。安撫完顏弼重其為人,每事咨而後行。未幾,有疾,詔賜御藥。三年,卒。



 守愚為人忠實無華,孜孜于公,蓋天性然也。



 石抹元,字希明,懿州路胡土虎猛安人。七歲喪父,號泣不食者數日。十三居母喪如成人。嘗為擊鞠戲,馬踣,歎曰:「生無兄弟,而數乘此險,設有不測,奈何?」由是終身不
 復為之。補樞密院尚書省譯史,調同知恩州軍州事,遷監察御史,為同知淄州軍州事。劇盜劉奇久為民患,一日捕獲,方訊鞫,聞赦將至,亟命杖殺之,闔郡稱快。改大興府判官,沂王府司馬、沁南軍節度副使。河內民家有多美橙者,歲獲厚利。仇家夜入殘毀之,主人捕得,乃以劫財誣其人,仇家引服,贓不可得。元攝州事,究得其情。尋改河北西路轉運副使,累遷山東西路按察轉運使。貞祐初,黃摑吾典徵兵東平,擁眾不進,大括民財,眾皆忿怨。副統僕散掃合殺吾典於坐,取其符佩之,縱恣尤甚。元密疏劾掃合擅殺近臣,無上不道,掃合坐誅。移知
 濟南府,到官六月卒。



 元生平寡言笑,尚節儉,居官自守,不交權要,人以是稱之。



 張彀,字伯英,許州臨潁人。大定二十八年進士,調寧陵縣主簿。改泰定軍節度判官。率儒士行鄉飲酒禮。改同州觀察判官。是時,出兵備邊,州征箭十萬,限以雕鴈羽為之,其價翔躍不可得。彀曰:「矢去物也,何羽不可。」節度使曰:「當須省報。」彀曰:「州距京師二千里,如民急何。萬一有責,下官身任其咎。」一日之間,價減數倍。尚書省竟如所請。補尚書省令史,除同知鄭州防禦使事,改北京鹽使。丁父憂,服除,再遷監察御史。從伐宋,遷武寧軍節度
 副使。居母憂。貞祐二年,改惠民司令,歷河南治中、顯州刺史、刑部郎中、同知河南府事,遷河東南路轉運使、權行六部尚書,安撫使。興定元年,以疾卒。



 彀天性孝友,任子悉先諸弟,俸入所得亦委其弟掌之,未嘗問有無云。



 趙重福,字履祥,豐州人。通女直大小字,試補女直誥院令史。轉兵部譯史、陜西提刑知法,遷陜西東路都勾判官、右藏庫副使、同知陳州防御事。宋諜人蘇泉入河南,重福迹之,至魚臺將渡河,見前一舟且渡,令從者大呼泉姓名,前舟中忽有蒼惶失措者,執之果泉也。改滄州鹽副使。歲饑,民煮鹵為鹽賣以給食,鹽官往往杖殺之。
 重福曰:「寧使課殿,不忍殺人。」歲滿,課殿當降,尚書右丞完顏匡、三司使按出虎知其事,乃以歲荒薄其罰,除織染署令。大安三年,佐戶部尚書張煒調兵食于古北口,遷都水少監,行西北路六部郎中,治密雲縣,俄兼戶部員外郎。貞祐二年,以守密雲功遷同知河間府事,行六部侍郎,權清州防禦使,攝河北東路兵馬都總管。三年,河間被圍,有劉中者嘗與重福密雲聯事,勸重福出降,重福不聽。是時,河間兵少,多羸疾不任戰,欲亡去。重福勸其父老率其子弟,強者戰、弱者守,會久雨圍乃解去。遷河東北路轉運使,致仕。元光二年,卒。



 武都,字文伯、東勝州人。大定二十二年進士,調陽穀主簿,遷商水令。縣素多盜,凡姦民嘗縱火行劫、椎埋發冢者,都皆廉得姓名,榜之通衢,約毋再犯,悉奔他境。察廉,遷南京路轉運支度判官,累遷中都路都轉運副使。以親老,與弟監察御史郁俱乞侍。尋丁憂。服除,調太原治中,復為都轉運副使,遷灤州刺史。充宣差北京路規措官,都拘括散逸官錢百萬。入為戶部郎中,權右司郎中,奏事稱旨。被詔由海道漕遼東粟賑山東,都高其價直募人入粟,招海賈船致之。三遷中都、西京按察副使。大安三年,充宣差行六部侍郎,以勞遷本路按察使,行西
 南路六部尚書,佐元帥抹捻盡忠備禦西京,有勞,召為戶部尚書,賞銀二百兩、絹一百匹。宣宗即位,議衛紹王降封,語在《衛紹王紀》。頃之,中都戒嚴,都知大興府,佩虎符便宜行事,彈壓中外軍民。都醉酒以褻衣見詔使,坐是解職。起為刑部尚書。中都解圍,為河東路宣撫使,俄以參知政事胥鼎代之。興定元年,以疾卒。



 紇石烈德,字廣之,真定路山春猛安人。明昌二年進士,調南京教授。察廉能,遷厭次令,補尚書省令史,除同知泗州防禦事、監察御史、大名治中、安、曹、裕三州刺史,歷同知臨潢,大興府事。貞祐二年,遷肇州防禦使。是歲,肇
 州升為武興軍節度,德為節度使宣撫司署都提控。肇州圍急,食且盡,有糧三百船在鴨子河,去州五里不能至。德乃浚濠增陴,築甬道導濠水屬之河。鑿陷馬阱,伏甲其傍以拒守,一日兵數接,士殊死戰。渠成,船至城下,兵食足,圍乃解。改遼東路轉運使,軍民遮道挽留,乘夜乃得去。蒲鮮萬奴逼上京,德與部將劉子元戰卻之。遷東京留守,歷保靜、武勝軍節度使。興定二年,以本官行六部事。三年,以節度權元帥右都監,與左都監單州經略使完顏仲元俱行元帥府于宿州。四年,遷工部尚書。明年,召還中都。是歲,卒。



 張特立,字文舉,曹州東明人。泰和三年中進士第,調宣德州司候。郡多皇族巨室,特立律之以法,闔境肅然。調萊州節度判官,不赴,躬耕杞之圉城,以經學自樂。正大初,左丞侯摯、參政師安石薦其才,授洛陽令。四年,拜監察御史。拜章言:「鎬、厲二宅,久加禁錮,棘圍柝警,如防寇盜。近降赦恩。謀反大逆,皆蒙湔雪,彼獨何罪,幽囚若是。世宗神靈在天,得無傷其心乎!聖嗣未立,未必不由是也。」又言:「方今三面受敵,百姓凋敝,宰執非才,臣恐中興之功未可以歲月期也。」又言:「尚書右丞顏盞世魯遣其奴與小民爭田,失大臣體。參知政事徒單兀典諂事近
 習,得居其位。皆宜罷之。」當路者忌其直,陰有以擠之。因劾省掾高楨輩受請託,飲娼家。時平章政事白撒犒軍陜西歸,楨等泣訴于道,以當時同席并有省掾王賓,張為其進士,故不劾。白撒以其私且不實,并治特立及賓。特立左遷邳州軍事判官,杖五十,賓亦勒停。士論皆惜特立之去。後卒癸丑歲,年七十五。



 王浩,由吏起身,初辟涇陽令,廉白為關輔第一。時西臺檄州縣增植棗果,督責嚴急,民甚被擾,浩獨無所問,主司將坐之,浩曰:「是縣所植已滿其數,若欲增植,必盜他人所有,取彼置此,未見其利。」其愛民多此類。所在有善
 政,民絲毫無所犯,秦人為立生祠,歲時思之。南遷後,為扶溝令。開興元年正月,民錢大亨等執縣官送款于北,大亨以浩有恩於民,不忍加刃,日遣所知勸之降,浩終不聽,於是殺之,無血。主簿劉坦、尉宋乙並見害。棄屍道路,自春徂夏,獨浩屍儼然如生,目且不瞑,烏犬莫敢近,殆若有神護者。



 初,辟舉法行,縣官甚多得人。如咸寧令張天綱、長安令李獻甫、洛陽令張特立三人有傳。餘如興平師夔、臨潼武天禎、氾水黨君玉、偃師王登庸、高陵宋九嘉、登封薛居中、長社李天翼、河津孫鼎臣、郟城李無黨、滎陽李過庭、尉氏張瑜、長葛張子玉、猗氏安德璋、
 三原蕭邦傑、藍田張德直、葉縣劉從益皆清慎才敏,極一時之選,而能扶持百年將傾之祚者,亦曰吏得其人故也。



\end{pinyinscope}