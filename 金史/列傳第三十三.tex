\article{列傳第三十三}

\begin{pinyinscope}

 ○移剌履張萬公蒲察通粘割斡特剌程輝劉瑋董師中王蔚馬惠迪馬琪楊伯通尼龐古鑑



 移剌履字履道,遼東丹王突欲七世孫也。父聿魯,早亡。聿魯之族兄興平軍節度使德元無子,以履為後。方五歲,晚臥廡下,見微雲往來天際,忽謂乳母曰:「此所謂『臥看青天行白雲』者耶?」德元聞之,驚曰:「是子當以文學名
 世。」及長,博學多藝,善屬文。初舉進士,惡搜檢煩瑣,去之。廕補為承奉班祗候、國史院書寫。世宗方興儒術,詔譯經史,擢國史院編修官,兼筆硯直長。一日,世宗召問曰:「朕比讀《貞觀政要》,見魏徵嘉謀忠節,良可稱歎。近世何故無如征者?」履曰:「忠嘉之士,何代無之,但上之人用與不用耳。」世宗曰:「卿不見劉仲誨、張汝霖耶,朕超用二人者,以嘗居諫職,屢有忠言故也。安得謂之不用,第人材難得耳。」履曰:「臣未聞其諫也。且海陵杜塞言路,天下緘口,習以成風。願陛下懲艾前事,開諫諍之門,天下幸甚。」



 初議以時務策設女直進士科,禮部以所學不同,未可
 概稱進士,詔履定其事,乃上議曰:「進士之科,起於隋大業中,始試以策。唐初因之,高宗時雜以箴銘賦詩,至文宗始專用賦。且進士之初,本專試策,今女直諸生以試策稱進士,又何疑焉。」世宗大悅,事遂施行。十五年,授應奉翰林文字,兼前職,俄遷修撰。二十年,詔提控衍慶宮畫功臣像,過期,降應奉。踰年,復為修撰,轉尚書禮部員外郎。



 章宗為金源郡王,喜讀《春秋左氏傳》,聞履博洽,召質所疑。履曰:「左氏多權詐,駁而不純。《尚書》、《孟子》皆聖賢純全之道,願留意焉。」王嘉納之。二十六年,進本部郎中,兼同修國史、翰林修撰,表進宋司馬光《古文孝經指解》
 曰:「臣竊觀近世,皆以兵刑財賦為急,而光獨以此進其君。有天下者,取其辭施諸宇內,則元元受賜。」俄以疾,乞補外,世宗曰:「履多病,可與便州。」遂授薊州刺史。無幾,召為翰林待制,同修國史。明年,擢尚書禮部侍郎,兼翰林直學士。



 世宗崩,遺詔移梓宮壽安宮。章宗詔百官議,皆謂當如遺詔,履獨曰:「非禮也。天子七月而葬,同軌畢至。其可使萬國之臣朝大行於離宮乎?」上曰:「朕日夜思之,捨正殿而奠於別宮,情有所不忍,且於禮未安。」遂殯於大安殿。二十九年三月,進禮部尚書,兼翰林直學士,賜大定三年孟崇獻榜下進士及第。七月,拜參知政事,提
 控刊修《遼史》。明昌元年,進尚書右丞。



 初,河溢曹州,帝問曰:「《春秋》二百四十二年,不言河決,何也?」履曰:「《春秋》止是魯史,所以鮮及他國事。」二年六月,薨,年六十一。是日,履所生也。謚曰文獻。



 履秀峙通悟,精歷算書繪事。先是,舊《大明歷》舛誤,履上《乙未歷》,以金受命于乙未也,世服其善。初,德元未有子,以履為後,既而生子震,德元歿,盡推家貲與之。其自禮部兼直學士為執政,乃舉前代光院故事,以錢五十萬送學士院,學者榮之。



 張萬公,字良輔,東平東阿人也。幼聰悟,喜讀書。父彌學,夢至一室,榜曰「張萬相公讀書堂」,已而萬公生,因以名
 焉。登正隆二年進士第,調新鄭簿。以憂去。服闋,除費縣簿。大定四年,為東京辰淥鹽副使,課增,遷長山令。時土寇未平,一旦至城下者幾萬人,萬公登陴諭以鄉里親舊意,眾感悟相率而去,邑人賴之,為立生祠。久之,補尚書省令史,擢河北西路轉運司都勾判官,改大理評事,就升司直,四遷侍御史、尚書右司員外郎。丞相徒單克寧嘗謂曰:「後代我者必汝也。」俄授郎中,敷奏明敏,世宗嘉之,謂侍臣曰:「張萬公純直人也。」尋遷刑部侍郎。



 章宗即位,初置九路提刑司,選為南京路提刑使。以治最,遷御史中丞。會北邊屢有警,上命樞密使夾谷清臣發兵
 擊之。萬公言:「勞民非便。」詔百官議於尚書省,遂罷兵。尋為彰國軍節度使。明昌二年,知大興府事,拜參知政事。踰年,以母老乞就養,詔不許,賜告省親。還,上問山東、河北粟貴賤,今春苗稼,萬公具以實對。上謂宰臣曰:「隨處雖得雨,尚未霑足,奈何?」萬公進曰:「自陛下即位以來,興利除害,凡益國便民之事,聖心孜孜,無不舉行。至於旱災,皆由臣等,若依漢典故,皆當免官。」上曰:「卿等何罪,殆朕所行有不逮者。」對曰:「天道雖遠,實與人事相通,唯聖人言行可以動天地。昔成湯引六事自責,周宣遇災而懼,側身修行,莫不修飭人事。方今宜崇節儉,不急之務、
 無名之費,可俱罷去。」上曰:「災異不可專言天道,蓋必先盡人事耳,故孟子謂王無罪歲。」左丞完顏守貞曰:「陛下引咎自責,社稷之福也。」上由是以萬公所言下詔罪己。進士李邦乂者上封事,因論世俗侈靡,譏涉先朝,有司議言者罪,上謂宰臣曰:「昔唐張玄素以桀、紂比文皇。今若方我為桀、紂,亦不之罪。至於世宗功德,豈容譏毀。」顧問萬公曰:「卿謂何如?」萬公曰:「譏斥先朝,固當治罪,然舊無此法。今宜定立,使人知之。」乃命免邦乂罪,惟殿三舉。其奏對詳敏,多類此。



 四年,復申前請,授知東平府事,諭之曰:「卿在政府,非不稱職,以卿母老,乞侍養,特畀鄉郡,
 以遂孝養。朕心所屬,不汝忘也。」萬公謝,且捧書言曰:「臣狂妄,有一言欲今日以聞,會受除未及耳。夫內外之職,憂責如一,畎畝之臣猶不忘君,芻蕘之言,明主所擇,伏望聖聰省察。」上嘉納之。六年,改知河中府,時軍興,調發叢劇,悉為寬假,使民力易辦。人為繪像於薰風樓,又建「去思堂」。



 移鎮濟南,以母憂去職。卒哭,詔起復,拜平章政事,躐遷資善大夫,封壽國公。時李淑妃有寵,用事,帝意惑之,欲立為后,大臣多不可。御史姬端修上書論之,帝怒,御史大夫張暐削一官,侍御史路鐸削兩官,端修杖七十,以贖論。淑妃竟進封元妃。又大兵雖罷,而邊事方
 殷,連歲旱,災異數見。又多變更制度,民以為弗便而又改之。紛紛無定。萬公素沉厚深謹,務安靜少事以為治,與同列議多不合。然頗嫌畏,不敢犯顏強諫,須帝有問,然後審畫利害而質言之,帝雖從而弗行也。萬公於是兩上表以衰病丐閒,詔諭曰:「近卿言數事,朕未嘗行,乃朕之過。卿年未老,而遽告病,今特賜告兩月,復起視事。」



 初,明昌間,有司建議,自西南、西北路,沿臨潢達泰州,開築壕塹以備大兵,役者三萬人,連年未就。御史臺言:「所開旋為風沙所平,無益於禦侮,而徒勞民。」上因旱災,問萬公所由致。萬公對以「勞民之久,恐傷和氣,宜從御
 史臺所言,罷之為便」。後丞相襄師還,卒為開築,民甚苦之。主兵者又言:「比歲征伐,軍多敗衄,蓋屯田地寡,無以養贍,至有不免饑寒者,故無鬥志。願括民田之冒稅者分給之,則戰士氣自倍矣。」朝臣議已定,萬公獨上書,言其不可者五,大略以為:「軍旅之後,瘡痍未復,百姓拊摩之不暇,何可重擾,一也。通檢未久,田有定籍,括之必不能盡,適足以增猾吏之敝,長告訐之風,二也。浮費侈用,不可勝計,推之以養軍,可斂不及民而足,無待於奪民之田,三也。兵士失於選擇,強弱不別,而使同田共食,振厲者無以盡其力,疲劣者得以容其姦,四也。奪民而與軍,
 得軍心而失天下心,其禍有不可勝言者,五也。必不得已,乞以冒地之已括者,召民蒔之,以所入贍軍,則軍有坐獲之利,而民無被奪怨矣。」皆不報。一日奏事,上謂萬公曰:「卿昨言天久陰晦,亦由人君用人邪正不分。君子當在內,小人當在外,甚有理也,然孰謂小人?」萬公奏「張煒、田櫟、張嘉貞等,雖有才幹,無德可稱」。上即命三人補外。



 泰和元年,連章請老,不許,遷榮祿大夫,賜其子進士及第。明年,章再上,有旨:「得非卿有所言,朕有不從者乎?或同列情見不一,而多違卿意邪?不然,何求去如是之數也。」萬公謝無他,第以病言。三年正月,章再上,不允,
 加銀青光祿大夫。三月,歷舉朝臣有名者以自代,求去甚力。上知其不能留,諭曰:「朕初即位,擢卿執政,繼遷相位,以卿先朝舊人,練習典故,朕甚重之。且年雖高而精力未衰,故以機務相勞。為卿屢求退去,故勉從之,甚非朕意也。」加金紫光祿大夫,致仕。



 六年,南鄙用兵,上以山東重地,須大臣鎮撫之,先任完顏守貞卒,於是特起萬公知濟南府、山東路安撫使。山東連歲旱蝗,沂、密、萊、莒、濰五州尤甚。萬公慮民饑盜起,當預備賑濟。時兵興,國用不給,萬公乃上言乞將僧道度牒、師德號、觀院名額並鹽引,付山東行部,於五州給賣,納粟易換。又言督責
 有司禁戢盜賊之方。上皆從之。宋人請和,復乞致仕,許之,加崇進,仍給平章政事俸之半。泰和七年,薨。命依宰臣故事,燒飯,賻葬。贈儀同三司,謚曰文貞。



 萬公淳厚剛正,門無雜賓,典章文物,多所裁正。上嘗與司空襄言秋山之樂,意將有事於春蒐也。顧視萬公,萬公曰:「動何如靜。」上改容而止。輔政八年,其所薦引,多廉讓之士焉。大安元年,配享章宗廟廷。



 蒲察通,本名蒲魯渾,中都路胡土愛割蠻猛安人也。熙宗選護衛,見通名,以筆識之。通以父老,懇乞就養。眾訝之曰:「得充侍衛,終身榮貴,今乃辭,過人遠矣。」朝廷義而
 從之。後因會葬宋王宗望於房山,以門閥,加昭信校尉,授頓舍。改御院通進。



 海陵伐宋,隆州諸軍尤精銳,付通總之。兵壓淮,令通率騎二百先濟覘敵。及弇中,敵兵躍出,通按兵直前,傍有舞槊來刺者,回身射之,應弦而斃。諸軍併擊,敗之。海陵召見,喜形於色,曰:「兵事定,汝勿憂爵賞。」至揚州,通營別屯。是夜,海陵遇弒,有來告者,通欲執而殺之,續聞其實,哀悶仆地,眾掖而起,徑入營門哭之。



 軍還,入見,世宗顧謂近臣曰:「朕素知是人,幼嘗從游,性溫厚,有識慮,又精騎射。」授尚廄局副使。又諭近臣曰:「常令見朕,欲問以事而考其言,朕將用之。」窩斡反,命通
 佩金符,詣軍前督戰。賊破,以功授世襲謀克。奚人亂,承詔繼往蒞軍。遷本局使,以母喪免。起為殿前右衛將軍,兼領閑廄。尋命其子蒲速烈尚衛國公主。出為肇州防禦使,賜以金帶,仍諭以補外之意,因戒敕之,語在《世宗紀》中。尋擢蒲與路節度使,移鎮歸德軍,遷西南路招討,入知大興府事,除殿前都點檢。初,大理卿闕,世宗欲令通為之,問宰臣,對曰:「通,點檢器也。」上曰:「點檢繁冗,無由顯其能。通明敏才幹,正掌法之官。」又曰:「通之機識,崇尹不及也。」



 大定十七年,拜尚書右丞,轉左丞。詔議推排猛安謀克事,大臣皆以為止驗見在產業,定貧富,依舊科差
 為便。通言:「必須通括各謀克人戶物力多寡,則貧富自分。貧富分,則版籍定,如有緩急,驗籍科差,富者不得隱,貧者不重困。與一例科差者,大不侔矣。」上是通言,謂宰臣曰:「議事當如通之盡心也。」閱三歲,進平章政事,封任國公。



 世宗將幸上京,以通朝廷舊人,命為上京留守,先往鎮撫之。二十五年,除知真定府事,世宗曰:「朕復欲相卿,惜卿老矣,故以此授卿。」仍賜錢千貫。未幾,改知平陽府事,移鳳翔,致仕。明昌四年,上諭宰臣曰:「通先朝重臣,年雖高而未衰。」因命知廣寧府事。累表請老,復以開府儀同三司致仕。承安三年薨。諭旨於其弟曰:「舊制,致仕
 宰相無祭葬禮,通舊臣懿戚,故特命敕祭及葬。」初,通在政府,舉太子率府完顏守貞、監察御史裔俱可大用,其後皆為名臣,世多其知人云。



 粘割斡特剌,蓋州別里賣猛安奚屈謀克人也。貞元初,以習女直字試補戶部令史,轉尚書省令史。大定七年,選授吏部主事,歷右補闕、修起居注。九年,河南路統軍使宗敘以宋人欲啟兵釁,上言求入見,世宗遣斡特剌就問之,仍究其實。至汴,問宗敘,及召凡嘗言邊事者詰之,皆無狀。還報,世宗喜曰:「朕固知妄也。」授左司員外郎。



 十年,以夏國發兵築祁安城及襲殺喬家族首領結什
 角,又諜者言夏與宋人通謀犯邊,詔大理卿李昌圖與斡特剌往按其事。夏人報言,結什角以兵犯夏境故殺之,祁安城本上國所賜舊積石地,發兵修築以備他盜耳。又察知宋、夏無交通狀,及喬家族民戶願令結什角侄趙師古為首領,具以聞。世宗甚悅,轉右衛將軍,賜衣馬車牛弓矢器仗。十二年,為夏國生日使,還授右司郎中,遷右副都點檢。久之,出為河南路統軍都監,賜金帶及具裝馬。



 十七年,授昌武軍節度使,兼領前職。明年,入為刑部尚書,拜參知政事。世宗嘗諭平章政事唐括安禮曰:「朕思為治之道,考擇人材最為難事,其餘常務各有
 程式,非此比也。如斡特剌所舉者,頗稱朕意。」時右三部檢法蒙括蠻都告斡特剌與招討哲典朋黨,乞付刑部詰問,世宗曰:「若哲典免死,則可謂朋黨。今已伏誅,乃誣謗耳。」又謂宰臣曰:「朕素知此人極有識慮,貌雖柔而心甚剛直,所行不率易也。」二十二年,委提控代州阜通監,召見諭之曰:「朕自任卿以來,悉卿材幹,故擢為執政。卿亦體朕待遇之意,能勉盡所職,凡謀議奏對多副朕心,莫倚上有宰相而自嫌外。蓋舊人年老,新人未苦經練,是以委責於卿,但有所見悉心以言,勿持嫌以為不知也。」二十三年,進尚書右丞,兼樞密副使,表乞解一職,詔
 許解樞密。世宗以猛安謀克拋留土田,責宰臣曰:「此事皆卿輩所當陳舉,乃俟朕言而後行,蓋卿輩以為細務非天子所親。朕嘗思之,獄訟簿書有斡特剌在,餘事卿輩略不介意,朕亦安能置而不問邪?」俄坐事削一階,令視事如故。



 二十六年,轉尚書左丞,世宗謂曰:「朕昨與宰臣議可授執政者,卿不在焉。今阿魯罕年老,斡魯也多病,吾欲用宗浩,何如?」斡特剌奏曰:「彼二人者恐不得力,獨宗浩幹能可任。」遂用宗浩。又謂曰:「朕於天下事無不用心,一如草創時。」斡特剌曰:「自古人君,始勤終怠者多矣,有始有終,惟聖人能之。」上曰:「唐太宗,至明之主也,然
 魏徵諫以十事,謂其不能有終,是則有終始者,實為難矣。」二十八年,為上京留守,賜通犀帶及射生馬一。



 明昌二年致仕。承安初,有事北方,朝廷欲得舊臣任之,乃起為東京留守,遣監察御史完顏綱諭旨曰:「知汝精神尚健,故復用也。」明年,改上京留守,又諭之曰:「上京祖先基業之地,卿馳驛之任,到彼便宜行事。邊事稍息,即召卿還。」二年九月,還朝,拜平章政事,封芮國公。在位數月,薨,年六十九。訃聞,上傷悼久之,遣官致祭,賻贈銀千二百五十兩、重幣四十五端、絹四百五十疋、錢二千貫,謚曰成肅。



 斡特剌性溫厚醖藉,嘗為丞相紇石烈良弼所薦,
 後世宗謂宰臣曰:「良弼善知人,如斡特剌輩其才真可用也。」在相位十餘年,甚見寵遇,唯奏定五品官子與外路司吏同試部令史、及令隨朝吏員得試國史院書寫,世宗以為非云。



 程輝,字日新,蔚州靈仙人也。皇統二年,擢進士第,由尚書省令史升左司都事。久之,為南京路轉運使,以宮殿火,降授磁州刺史。有吳僧者殺州人張善友而取其妻,輝督捕之,命張母以長錐刺僧與其妻無完膚以死。改陜西東路轉運使,再遷戶部尚書。



 大定二十三年,拜參知政事。世宗諭之曰:「卿年雖老,猶可宣力。事有當言,毋
 或隱默。卿其勉之。」一日,輝侍朝,世宗曰:「人嘗謂卿言語荒唐,今遇事輒言,過於王蔚。」顧謂宰臣曰:「卿等以為何如?」皆曰:「輝議政可否,略無隱情。」輝對曰:「臣年老耳聵,第患聽聞不審,或失奏對。茍有所聞,敢不盡心。」舊廟祭用牛,世宗晚年欲以他牲易之,輝奏曰:「凡祭用牛者,以牲之最重,故號太牢。《語》曰『犁牛之子騂且角,雖欲勿用,山川其舍諸?』古禮不可廢也。」



 二十四年,世宗幸上京,尚書省奏來歲正旦外國朝賀事,世宗曰:「上京地遠天寒,朕甚憫人使勞苦,欲即南京受宋書,何如?」輝對曰:「外國使來,必面見天子,今半途受書,異時宋人託事效之,何以
 辭為?」世宗曰:「朕以誠實,彼若相詐,朕自有處置耳。」輝以為不可,於是議權免一年。會有司市面不時酬直,世宗怒監察不舉劾,杖責之。以問輝,輝對曰:「監察,君之耳目。所犯罪輕,不贖而杖,亦一時之怒也。」世宗曰:「職事不舉,是故犯也,杖之何不可!」輝對曰:「往者不可諫,來者猶可追。」



 二十六年,以老致仕。次年,復起知河南府事,輝辭以衰老不任,召入香閣,諭之曰:「卿年老而精力尚強,雖久歷外,未嘗得嘉郡。河南地勝事簡,故以處卿,卿可優游頤養。」輝曰:「臣猶老馬也,芻豆待養,豈可責以筋力。向者南京宮殿火,非聖恩寬貸,臣死久矣。今河之徑河南境
 上下千餘里,河防之責視彼尤重,此臣所以憂不任也。」於是特詔不預河事。章宗立,時輝年七十六,復乞致仕,詔許之,仍給參知政事半俸。承安元年卒,謚曰忠簡。



 輝性倜儻敢言,喜雜學,尤好論醫。從河間劉守真說,率用涼藥。神童嘗添壽者方數歲,輝召之,因書「醫非細事」四字,添壽塗「細」字,改書作「相」,輝頗慚,人亦以此為中其病云。



 劉瑋,字德玉,咸平人也。唐盧龍節度使仁敬之裔。祖弘,遼季鎮懿州,王師至,弘以州降,太祖俾知咸州,後以同平章政事致仕。父君詔,同知宣徽院事。瑋幼警悟,業進
 士舉,熙宗錄其舊,特賜及第。調安次丞。由遵化縣令補尚書省令史,歷戶部主事、監察御史,累轉尚書省都事。宰臣奏擬瑋經畫軍民田土,世宗見其名曰:「劉瑋尚淹此乎。」遷戶部員外郎。時將東巡,命瑋同工部郎中宋中往營行宮,就陞郎中。改同知宣徽院事,為使宋國信副使。瑋父兄皆以是官使江左,當時榮之。還授戶部侍郎。



 初,世宗器瑋材幹,以為無施不可,及將幸上京,以行在所須皆隸太府,欲瑋領其事,嫌其稍下,故移戶部侍郎張大節於工部,而以戶部授瑋。上還,謂宰臣曰:「劉瑋極有心力,臨事閑暇,第用心不正耳。若心正當,其人才不
 可得也。」



 明年,擢戶部尚書。時河決于衛,自衛抵清、滄皆被其害,詔兼工部尚書往塞之。或以謂天災流行,非人力所能禦,惟當徙民以避其衝,瑋曰:「不然。天生五材,遞相休王,今河決者土不勝水也。俟秋冬之交,水勢稍殺,以漸興築,庶幾可塞。」明年春,瑋齋戒禱于河,功役齊舉,河乃復故。召還增秩,以為宋弔祭副使。世宗不豫,拜參知政事,仍領戶部,既而為山陵使。尋上表請外,出知濟南府事,移鎮河中。明昌二年,徙知大名府,仍領河防事。



 三年,入拜尚書右丞。上嘗問考課法今可行否,右丞相夾谷清臣曰:「行之亦可,但格法繁則有司難於承用耳。」
 瑋曰:「考課之法,本於總核名實,今提刑司體察廉能贓濫,以行賞罰,亦其意也。若別議設法,恐涉太繁。」上問唐代何如,瑋對以「四善、二十七最」。明年六月,卒。是日,上將擊球於臨武殿,聞瑋卒而止,謚曰安敏。



 後上謂宰臣曰:「人為小官或稱才幹,及其大用則不然。如劉瑋固甚幹,然自世宗朝逮輔朕,於事多有知而不言者。若實愚人,則不足論,知及之而不肯盡心,可乎?」平章政事完顏守貞曰:「《春秋》之法,責備賢者。」上曰:「夫為宰相而欲收恩避怨,使人人皆稱己是,賢者固若是乎?」



 董師中,字紹祖,洺州人也。少敏贍,好學強記。擢皇統九
 年進士第,調澤州軍事判官。改平遙丞。縣有劇賊王乙,素凶悍不可制,師中捕得杖殺之,一境遂安。時大軍後,野多枯胔,縣有遺櫬寓于驛舍者,悉為葬之。遷綿上令,補尚書省令史。右相唐括訛魯古尤器重之,撫其座曰:「子議論英發,襟度開朗,他日必居此座。」再考,擢監察御史,遷尚書省都事。初,師中為監察時,漏察大名總管忽剌不公事,及忽剌以罪誅,世宗怒曰:「監察出使郡縣,職在彈糾,忽剌親貴,尤當用意,乃徇不以聞。」削官一階,降授沁南軍節度副使。累遷坊州刺史。



 明昌元年,初置九路提刑司,師中選為陜西路副使,坐修公廨濫支官錢
 罪,以贖論。及御史臺言其寬和有體,召為大理卿。御史中丞吳鼎樞舉以自代,尚書省亦奏其才行,遂擢中丞。時西北路招討使宗肅以平章夾谷清臣薦,知大興府事。師中上言:「宗肅近以贓罪鞫于有司,獄未竟,不宜改除。」上納其言,曰:「朕知之矣。有功不賞,有罪不罰,雖唐、虞不能化天下。」命復送有司。



 四年,上將幸景明宮,師中及侍御史賈鉉、治書侍御史粘割遵古諫,以謂「勞人費財,蓋其小者,變生不虞,所繫非輕。聖人法天地以順動,故萬舉萬全。今邊鄙不馴,反側無定,必里哥孛瓦貪暴強悍,深可為慮。陛下若問諸左右,必有容悅而言者,謂堂
 堂大國,何彼之恤。夫蜂蠆有毒,患起所忽。今都邑壯麗,內外苑囿足以優佚皇情,近畿山川飛走充牣,足以閱習武事,何必千車萬騎,草居露宿,逼介邊陲,遠煩偵候,以冒不惻之悔哉。」上不納。師中等又上疏曰:「近年水旱為沴,明詔罪己求言,罷不急之役,省無名之費,天下欣幸。今方春東作,而亟遣有司修建行宮,揆之於事,似為不急。況西、北二京,臨潢諸路,比歲不登。加以民有養馬簽軍挑壕之役,財力大困,流移未復,米價甚貴,若扈從至彼,又必增價。日糴升合者口以萬數,舊藉北京等路商販給之,倘以物貴或不時至,則饑餓之徒將復有如
 曩歲,殺太尉馬、毀太府瓜果、出忿怨言、起而為亂者矣。《書》曰:『民情大可見,小人難保。』況南北兩屬部數十年捍邊者,今為必里哥孛瓦誘脅,傾族隨去,邊境蕩搖如此可虞,若忽之而往,豈聖人萬舉萬全之道哉。乃者太白晝見,京師地震,又北方有赤色,遲明始散。天之示象,冀有以警悟聖意,修德銷變。矧夫逸遊,古人所戒,遠自周、秦,近逮隋、唐與遼,皆以是生釁,可不慎哉,可不畏哉。」左補闕許安仁、右拾遺路鐸亦皆上書論諫。是日,上御後閣,召師中等賜對,即從其奏,仍遣諭輔臣曰:「朕欲巡幸山後,無他,不禁暑熱故也。今臺諫官咸言民間缺食處
 甚多,朕初不盡知,既已知之,暑雖可畏,其忍私奉而重民之困哉!」乃罷北幸。尋為宋生日國信使,還以所得金帛分遺親舊。五年,上復如景明宮,師中及臺諫官各上疏極諫,上怒,遣近侍局直長李仁愿詣尚書省,召師中等諭之曰:「卿等所言,非無可取,然亦有失君臣之體者。今命平章諭旨,其往聽焉。」



 戶部尚書馬琪表舉自代,擢吏部尚書。初,完顏守貞改為西京留守,朝京師,上欲復用,監察御史蒲剌都等糾彈數事,師中辨其誣,而舉守貞正人可用,守貞由是復拜平章政事。及守貞以罪斥,上曰:「向薦守貞者應降黜。如董師中言臺省無此人不
 治,路鐸、李敬義亦嘗推舉,可左遷於外。然三人者後俱可用,今姑出之,以正失舉罪。」除陜西西路轉運使。歲餘,徵為御史大夫,命與禮部尚書張暐看讀陳言文字。踰三月,拜參知政事,進尚書左丞。他日奏事,上語輔臣曰:「御史姬端修言小人在側,果誰歟?」師中曰:「應謂李喜兒輩。」上默然。



 師中通古今,善敷奏,練達典憲,處事精敏,嘗言曰:「宰相不當事細務,要在知人才,振綱紀,但一心正、兩目明,足矣。」承安四年,表乞致仕,詔賜宅一區,留居京師。以寒食,乞過家上冢,許之,且命賦《寒食還家上冢詩》。每節辰朝會,召入侍宴,其眷禮如此。泰和二年,薨,年七
 十四。上聞之,甚悼惜,顧謂大臣曰:「凡正人多執方而不通,獨師中正而通。」詔依見任宰執例葬祭,仍賻贈之,謚曰文定。



 師中工文,性通達,疏財尚義,平居則樂易真率,其臨事則剛決,挺然不可奪。弟師儉,初業進士,欲籍其資蔭。師中保任之,密令人代給堂帖,使之肄業。師儉感其義方,力學後遂登第。方在政府,近侍傳詔,將錄用其子,師中奏曰:「臣有侄孤幼,若蒙恩錄,勝于臣子。」上義之,以其姪為筆硯承奉。與胥持國同輔政,頗相親附,世以此少之。



 王蔚,字叔文,香河人也。登皇統二年進士第,調良鄉丞。
 治績優等,補尚書省令史,知管差除。蔚性通敏,曉析吏事。尋授都事,以喪去。起復,行左司員外郎,遷郎中。大定二年,超授河東北路轉運使,諭旨曰:「汝在海陵時,行事多不法。然朕素知爾才幹,欲授以內除,而憲臺有言,以是補外。如能澡心易行,必當升擢,否則勿望再用。」既而察廉為第一,授中都路都轉運使。改吏部尚書,以斷護衛出職事不當,奪官一階。頃之,出知河中府事,遷南京留守。十五年,拜參知政事,蔚懇辭不任負荷,敕諭之曰:「卿但履正奉公,無或阿順,何以辭為?」十六年,出知真定府事,累轉知河中府。明昌元年,召拜尚書右丞,致仕,卒。



 馬惠迪,字吉甫,漷陰人也。擢天德三年進士第,再調昌邑令,察廉第一,補尚書省令史。大定中,出為西京留守判官,以治最,擢同知崇義軍節度事。累遷左司郎中。先是,鄧儼居是職,世宗愛其明敏,惠迪一日奏事退,上謂宰臣曰:「人之聰明,多失於浮炫,若惠迪聰明而朴實,甚可喜也。朕嘗與論事,五品以下朝官少有如者。」未幾,超授御確中丞,拜參知政事。時烏底改叛亡,世宗已遣人討之,又欲益以甲士,毀其船筏。惠迪奏曰:「得其人不可用,有其地不可居,恐不足勞聖慮。」上曰:「朕固知之。所以毀其船筏,正欲不使再窺邊境耳。」尋以憂去。起為昭義
 軍節度使。明昌元年,為南京留守,致仕,卒。



 馬琪,字德玉,大興寶坻人。正隆五年擢進士第,調清源主簿‖三遷永清令。永清畿縣,號難治,前令要介有能聲,琪繼以治聞。補尚書省令史,以永清治最,授同知定武軍節度使事、興中府治中,召為戶部員外郎,改侍御史。



 世宗謂宰臣曰:「比者馬琪主奏高德溫獄,其於富戶寄錢事皆略不奏。朕以琪明法律而正直,所為乃爾,稱職之才何其難也?古人雖云『罪疑惟輕』,非為全尚寬縱也。」尋轉左司員外郎,扈從東巡,遷右司郎中,移左司。時擇使宋國者,世宗欲命琪,宰臣言其資淺,詔特遣之,還授
 吏部侍郎,改戶部。



 章宗即位,除中都路都轉運使。時戶部闕官,上命宰臣選可任者,或舉同知大興府事烏古孫仲和,上曰:「仲和雖有智力,恐不能主錢穀。理財安得如劉晏者,官用足而民不困,唐以來一人而已。」或舉琪,上然之,曰:「琪不肯欺官,亦不肯害民,是可用也。」遂擢為戶部尚書。久之,削官一階。初,琪病告,近侍傳旨,不具服曳履而出,有司議當徒二年,減外猶追官解任。大理少卿閻公貞以為琪本荒遽失措,與非病告有違不同,宜減徒二年三等論之。上從公貞議,任職如故。



 明昌四年,拜參知政事,詔諭之曰:「戶部遽難得人,顧無以代卿者,
 故用卿晚耳。」一日,上謂琪曰:「卿在省久矣,比來事少於往時何也?」琪曰:「昔宰職多有異同,今情見不同者甚少。」上曰:「往多情見為是耶,今無者為是耶?」琪曰:「事狀明者不假情見,便用情見,亦要歸之是而已。」五年,河決陽武,灌封丘而東,琪行尚書省事往治之,訖役而還。遷中大夫。承安元年,北邊用兵,而連歲旱,表乞致仕,不許。明年,出鎮安武軍,致仕,卒。子師周,閣門祗候,當給假,以聞。上悼之,以不奏聞責諭有司,後二品官卒皆具以聞,自琪始。



 琪性明敏,習吏事,其治錢穀尤長,然性吝好利,頗為上所少云。



 楊伯通,字吉甫,弘州人。擢大定三年進士第,由尚書省令史為吏部主事、順義軍節度副使,以憂去。吏部侍郎馬琪表薦伯通廉幹,尚書省復察如所舉,召為尚書省都事,授同知定武軍節度使事。明昌元年,擢左司員外郎,轉郎中,累遷吏部尚書,尋移戶部。



 承安二年,拜參知政事。監察御史路鐸劾奏伯通引用鄉人李浩,以公器結私恩。左司郎中賈益承望風旨,不復檢詳,言之臺端,欲加糾劾,大夫張暐輒尼不行。上命同知大興府事賈鉉詰之,伯通居家待罪。鉉奏:「暐言彈絀大臣,須有實跡,所劾不當,徒壞臺綱。益言除授皆宰執公議,不言伯通
 私枉。」詔責鐸言事輕率,而慰諭伯通治事。伯通再上表辭,不許。四年,進尚書左丞,致仕,卒。



 尼龐古鑑,本名外留,隆州人也。識女直小字及漢字,登大定十三年進士第,調隆安教授。改即墨主簿,召授國子助教,擢近侍局直長。世宗器其材,謂宰臣曰:「新進士中如徒單鎰、夾谷衡、尼龐古鑒,皆可用也。」改太子侍丞。踰年,遷應奉翰林文字,兼右三部司正。世宗復謂宰臣曰:「鑑嘗近侍,朕知其正直幹治。及為東宮侍丞,保護太孫,禮節言動猶有國俗純厚舊風,朕甚嘉之。」章宗立,累遷尚書戶部侍郎,兼翰林直學士。俄轉同知大興府,用
 大臣薦,改知大興府事。明昌五年拜參知政事,薨,謚曰文肅。



 贊曰:移剌履從容進說,信孚於君,至論經純傳駁,以孝行為治本,其得古人遺學歟!昔臧孫達忠諫於魯,君子知其有後,信矣。張萬公引正守己,質言無華。開壕括地之議,明灼利害,如指諸掌,閉於群說而不式,致仕而歸,理勢然也。蒲察通之哭海陵,君臣大義死生一之,其志烈矣。程輝、斡特剌之鯁直,劉瑋、董師中之通敏,才皆足以發聞,然師中有附胥之譏,劉瑋見避事之責,其視前人,多有愧矣。王蔚、馬惠迪之徒,何足算也。



\end{pinyinscope}