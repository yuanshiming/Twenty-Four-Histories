\article{列傳第七十二}

\begin{pinyinscope}

 外國上



 ○西夏



 夏國王李乾順。其先曰托跋思恭,唐僖宗時,為夏、綏、銀、宥節度使,與李茂貞、李克用等破黃巢,復京師,賜姓李氏。唐末,天下大亂,籓鎮連兵,惟夏州未嘗為唐患。歷五代至宋,傳數世至元昊,始稱帝。遼人以公主下嫁李氏,世修朝貢不絕,事具《遼史》。



 天輔六年,金破遼兵,遼主走
 陰山,夏將李良輔將兵三萬來救遼,次天德境野谷。斡魯、婁室敗之於宜水,追至野谷,澗水暴至,漂沒者不可勝計。宗望至陰山,以便宜與夏國議和,其書曰:「奉詔有之:夏王,遼之自出,不渝終始,危難相救。今茲已舉遼國,若能如事遼之日以效職貢,當聽其來,毋致疑貳。若遼主至彼,可令執送。」天會二年,始奉誓表,以事遼之禮稱籓,請受割賜之地。宗翰承制,割下寨以北、陰山以南、乙室耶刮部吐祿濼之西,以賜之。



 天會二年三月,乾順遣把里公亮等來上誓表,曰:「臣乾順言:今月十五日,西南、西北兩路都統遣左諫議大夫王介儒等齎牒奉宣,若夏國
 追悔前非,捕送遼主,立盟上表,仍依遼國舊制及賜誓詔,將來或有不虞,交相救援者。臣與遼國世通姻契,名係籓臣,輒為援以啟端,曾犯威而結釁。既速違天之咎,果罹敗績之憂。蒙降德音以寬前罪,仍賜土地用廣籓籬,載惟含垢之恩,常切戴天之望。自今已後,凡於歲時朝賀、貢進表章、使人往復等事,一切永依臣事遼國舊例。其契丹昏主今不在臣境,至如奔竄到此,不復存泊,即當執獻。若大朝知其所在,以兵追捕,無敢為地及依前援助。其或徵兵,即當依應。至如殊方異域朝覲天闕,合經當國道路,亦不阻節。以上所敘數事,臣誓固此誠,
 傳嗣不變,茍或有渝,天地鑒察,神明殛之,禍及子孫,不克享國。」所謂西北,西南兩路都統者宗翰也。蓋宗望以太祖命與之通書,而宗翰以便宜割地議和云。



 太宗使王阿海、楊天吉往賜誓詔曰:「維天會二年歲次甲辰,閏三月戊寅朔,皇帝賜誓詔於夏國王乾順:先皇帝誕膺駿命,肇啟鴻圖,而卿國據夏臺,境連遼右,以效力於昏主,致結釁於王師。先皇帝以謂忠於所事,務施恩而釋過,迨眇躬之纂紹,仰遺訓以遵行,卿乃深念前非,樂從內附,飭使軺而奉貢,效臣節以稱籓。載錫寵光,用彰復好,所有割賜地土、使聘禮節、相為援助等事,一切恭依
 先朝制詔。其依應徵兵,所請宜允。三辰在上,朕豈食言,茍或變渝,亦如卿誓。遠垂戒諭,毋替厥誠。」



 於是,宋人與夏人俱受山西地,宋人侵取之,乾順遣使表謝賜誓詔、并論宋所侵地。詔曰:「省所上表,具悉,已命西南、西北兩路都統府從宜定奪。」是時,宗翰朝京師未還,錄夏國奏付權都統斡魯,宋人侵略新受疆土、及使人王阿海爭儀物事,與夏通問以便宜決之。



 初,以山西九州與宋人,而天德遠在一隅,緩急不可及,割以與夏。後破宋都獲二帝,乃畫陜西分界,自麟府路洛陽溝東距黃河西岸、西歷暖泉堡,鄜延路米脂谷至累勝寨,環慶路威邊寨
 過九星原至委布谷口,涇原路威川寨略古蕭關至北谷川,秦鳳路通懷堡至古會州,自此直距黃河,依見今流行分熙河路盡西邊以限封域。復分陜西北鄙以易天德、雲內,以河為界。



 及婁室定陜西,婆盧火率兵先取威戎城。軍至威戎東與敵遇,擊走之,生致二人,問之,乃知為夏將李遇取威戎也,乃還其人而與李遇通問。李遇軍威戎西,蒲察軍威戎東,而使使議事于婁室。婁室報曰:「元帥府約束,若兵近夏境,則與夏人相為掎角,毋相侵犯。」李遇使人來曰:「夏國既以天德、雲內歸大國,大國許我陜西北鄙之地,是以至此。」蒲察等遂旋軍。睿宗既
 定陜西,元帥府不欲以陜西北鄙與夏國,詔曰:「卿等審處所宜從事。」



 天眷二年,國王乾順薨,子仁孝立,遣使冊命,加開府儀同三司上柱國。皇統元年,請置榷場、許之。



 初,王阿海等以太宗誓詔賜夏國,乾順以契丹舊儀見使者,阿海不肯曰:「契丹與夏國甥舅也,故國王坐受,使者以禮進。今大金與夏國君臣也,見大國使者當如儀。」爭數日不能決,於是始起立受焉。厥後不遣賜生日使,至是始遣使賜之。



 初,慕洧以環州降,及割陜西、河南與宋人,洧奔夏國,夏人以為山訛首領。及撒離喝再定陜西,洧思歸,夏人知之,遂族洧,以表聞,詔書責讓之。及海
 陵弒熙宗,遣使報諭至境上,夏人問曰:「聖德皇帝何為見廢。」不肯納。朝廷乃使有司以廢立之故移文報之。天德二年七月,夏使御史中丞雜辣公濟等來賀,如舊禮。



 正隆末伐宋,宋人入秦、隴,夏亦乘隙攻取盪羌、通峽、九羊、會川等城寨,宋亦侵入夏境。世宗即位,夏人復以城寨來歸,且乞兵復宋侵地,詔書嘉獎,仍遣吏部郎中完顏達吉體究陜西利害。邊吏奏,夏人已歸城寨,而所侵掠人口財畜尚未還,請索之。大定四年二月甲申,夏遣其武功大夫紐臥文忠等賀萬春節,入見,附狀奏告,略曰:「眾軍破蕩之時,幸而免者十無一二,繼以凍餒死
 亡,其存幾何。兼夏國與宋兵交,人畜之被俘戮亦多,連歲勤動,士卒暴露,勢皆朘削。又坐為宋人牽制,使忠誠之節無由自達,中外咸知,願止約理索,聽納臣言,不勝下國之幸。」其後屢以為請,詔許之。



 久之,其臣任得敬專國政,欲分割夏國。因賀大定八年正旦,遣奏告使殿前太尉芭里昌祖等以仁孝章乞良醫為得敬治疾,詔保全郎王師道佩銀牌往焉。詔師道曰:「如病勢不可療,則勿治。如可治,期一月歸。」得敬疾有瘳,遣謝恩使任得聰來,得敬亦附表進禮物,上曰:「得敬自有定分,附表禮物皆不可受。」並卻之。



 初,仁孝嗣位,其臣屢作亂,任得敬抗
 禦有功,遂相夏二十餘年,陰蓄異志,欲圖夏國,誣殺宗親大臣,其勢漸逼,仁孝不能制。大定十年,乃分西南路及靈州羅龐嶺地與得敬,自為國,且上表為得敬求封。世宗以問宰相,尚書令李石等曰:「事繫彼國,我何預焉,不如因而許之。」上曰:「有國之主豈肯無故分國與人,此必權臣逼奪,非夏王本意。況夏國稱籓歲久,一旦迫於賊臣,朕為四海主,寧容此邪?若彼不能自正,則當以兵誅之,不可許也。」乃卻其貢物,賜仁孝詔曰:「自我國家戡定中原,懷柔西土,始則畫疆於乃父,繼而錫命於爾躬,恩厚一方,年垂三紀,籓臣之禮既務踐修,先業所傳
 亦當固守。今茲請命,事頗靡常,未知措意之由來,續當遣使以詢爾。所有貢物,已令發回。」



 得敬密通宋人求助,宋以蠟丸書答得敬,夏人得之。得敬始因求醫附表進禮物,欲以嘗試世宗,既不可行,而求封又不可得,仁孝乃謀誅之。八月晦,仁孝誅得敬及其黨與,上表謝,并以所執宋人及蠟丸書來上。其謝表曰:「得敬初受分土之後,曾遣使赴大朝代求封建,蒙詔書不為俞納,此朝廷憐愛之恩,夏國不勝感戴。夏國妄煩朝廷,冒求賊臣封建,深虧禮節。今既賊臣誅訖,大朝不用遣使詢問。得敬所分之地與大朝熙秦路接境,恐自分地以來別有生
 事,已根勘禁約,乞朝廷亦行禁約。」



 十二年,上謂宰臣曰:「夏國以珠玉易我絲帛,是以無用易我有用也。」乃減罷保安、蘭州榷場。



 仁孝深念世宗恩厚,十七年,獻本國所造百頭帳,上曰:「夏國貢獻自有方物,可卻之。」仁孝再以表上曰:「所進帳本非珍異,使人亦已到邊,若不蒙包納,則下國深誠無所展效,四方鄰國以為夏國不預大朝眷愛之數,將何所安。」乃許與正旦使同來。



 先是,尚書奏:「夏國與陜西邊民私相越境,盜竊財畜,姦人託名榷場貿易,得以往來,恐為邊患。使人入境與富商相易,亦可禁止。」於是,復罷綏德榷場,止存東勝、環州而已。仁孝表
 請復置蘭州、保安、綏德榷場如舊,并乞使人入界相易用物。詔曰:「保安、蘭州地無絲枲,惟綏德建關市以通貨財。使副往來,聽留都亭貿易。」章宗即位,詔曰:「夏使館內貿易且已。」明昌二年,復舊。頃之,夏人肆牧於鎮戎之境,邏卒逐之,夏人執邏卒而去。邊將阿魯帶率兵詰之,夏廂官吳明契、信陵都、卜祥、徐餘立等伏兵三千於潤中,阿魯帶口中流矢而死,取其弓甲而去。詔索殺阿魯帶者,夏人處以徒刑。詔索之不已,夏人乃殺明契等。



 明昌四年,仁孝薨,子純佑嗣立。承安二年,復置蘭州、保安榷場。承安五年,純佑母病風求醫,詔太醫判官時德元及
 王利貞往,仍賜御藥。八月,再賜醫藥。泰和六年三月,仁孝弟仁友子安全,廢純佑自立,再閱月死于廢所。七月,使純佑母羅氏為表,言純佑不能嗣守,與大臣定議立安全為王,遣使奏告。夏使私問館伴官:「奏告事詔許否?」館伴官曰:「此不當問也。」夏使曰:「明日當問諸客省,若又不答,則升殿奏請。」上聞之,使客省諭以許所祈之意,乃賜羅氏詔詢其意,夏人復以羅氏表來,乃封安全為夏國王。



 大安三年,安全薨,族子遵頊立。遵頊先以狀元及第,充大都督府主,立在安全薨前一月,衛紹王無實錄,不知其故。然是時金兵敗績于會河堡,夏人乘其兵敗
 侵略邊境,而通使如故。



 崇慶元年三月,攻葭州。至寧元年六月,攻保安州。貞祐元年十一月,攻會州,都統徒單醜兒擊走之。十二月,陷涇州。二年八月,歸國人喬成齎夏國書,大概言金邊吏侵略,乞禁戢。詔移文答之,宰臣言:「既非公牒,今將責問,彼必飾詞,徒為虛文,無益於事。」乃止。未幾,夏人攻慶原、延安、積石州,乃詔有司移文責問。



 十一月,蘭州譯人程陳僧結夏人以州叛,邊將敗其兵三千。三年正月,夏兵攻武延川,宣宗曰:「此不足慮,恐由他道入也。」既而聞邊吏侵夏境,夏人乃攻環州,詔治邊吏罪。夏兵攻積石州,都統姜伯通敗之。夏兵入安鄉
 關,都統曹記僧、萬戶忽三十卻之。二月,攻環州,刺史烏古論延壽敗之于境上。



 三月,詔議伐夏,陜西宣撫司奏:「往者,夏人侵我環、慶,河、蘭、積石以兵應之,悉皆遁去,遽還巢穴,蓋為我備也。今蘭州潰兵猶未集,軍實多不完,沿邊地寒,春草始生,未可芻牧,兩界無煙火者三百餘里,不宜輕舉。」從之。



 四月,詔河州提控曹記僧、通遠軍節度使完顏狗兒討程陳僧,夏人援之。九月,遂破西關堡。夏人復攻第五將城,萬戶楊再興擊走之。詔陜西宣撫司及沿邊諸將,降空名宣敕,臨陣立功,五品以下並聽遷授。十月,攻保安及延安,都統完顏國家奴破之。既而
 深入臨洮,總管陀滿胡土門不能禦,陜西宣撫副使完顏胡失來救臨洮,大敗于渭源堡,城破,胡失來被執。十一月,夏兵敗于克戎寨,復敗于熟羊寨,宰相入賀,宣宗曰:「此忠賢之力也。」夏兵進圍臨洮,陀滿胡土門破之。四年四月,夏葩俄族總管汪三郎率眾來降,進羊千口,詔納之,優給其直。來遠鎮獲諜人,言宋、夏相結來攻,詔陜西行省備之。



 夏於來羌城界河起折橋,元帥右都監完顏賽不焚之,斬馘甚眾。六月,鄜延路奏,夏人牒報用彼國光定年號,詔封還其牒。閏月,慶陽總管慶山奴伐夏,出環州,陜西行省請中分其軍,令慶山奴出第三將懷
 安寨,環州刺史完顏胡魯出環州,宣宗曰:「聞夏人移軍備其王城,尚恐詐我,勿墮其計中也。」提控完顏狗兒抵蘭州西關堡,招得舊部曲九人。掩擊夏兵于阿彌灣,殺其將士百餘人。八月,左監軍烏古論慶壽敗夏兵于安塞堡。右都監賽不擊走夏兵於結耶觜川,復破之于車兒堡。十一月,提控石盞合喜、楊斡烈解定西之圍。



 十二月丙寅,宣宗與皇太子議伐夏,左監軍陀滿胡土門、延安總管古里甲石倫攻鹽、宥、夏州,慶陽總管慶山奴、知平涼府移剌答不也攻威、靈、安、會等州。



 興定元年正月,夏兵三萬自寧州還,慶山奴以兵邀擊,敗之。詔河東行
 省胥鼎選兵三萬五千,付陀滿胡土門伐夏,鼎馳奏不可,遂止,語在鼎傳。右都監完顏仲元請試兵西夏,出其不意必獲全勝,兵威既振,國力益完。詔下尚書省、樞密院議。



 夏人福山以俘戶來降,除同知澤州軍州事。



 五月,夏兵入大北岔,都統紇石烈豬狗掩擊,敗之。宣宗欲與夏議和,右都監慶山奴屯延安,奏曰:「夏國決不肯和,徒見欺耳。」既而,獲諜者言,遵頊聞大金將約和,戒諭將士無犯西鄙。宰臣奏曰:「就令如此,邊備亦不宜弛。」宣宗以為然。



 右都監完顏閭山敗夏兵於黃鶴岔。夏人圍羊狠寨,都統黨世昌與戰,完顏狗兒遣都統夾谷瑞夜斫夏
 營,遂解其圍,猶駐近地,左都監白撒發定西銳兵、龕谷副統包孝成緋翮翅軍,合擊走之。八月,安定堡馬家平總押李公直敗夏兵三千。九月,都統羅世暉卻夏兵於克戎寨。



 興定二年三月,右都監慶山奴奏:「夏人有乞和意,保安、綏德、葭州得文報,乞復互市,以尋舊盟。以臣觀之,此出於遵頊,非邊吏所敢專者。」朝廷不以為然。



 五月,夏人入葭州,慶山奴破之於馬吉峰。七月,犯龕谷,夾谷瑞、趙防敗之,追至質孤堡。三年閏月,夏人破通秦寨,提控納合買住擊敗之,自葭盧川遁去。華州元帥完顏合達出安寨堡至隆州,敗其兵二千。進攻隆州,克其西南,
 會暮乃還。十二月,詔有司移文夏國。



 四年二月,夏人犯鎮戎,金師敗績,夏人公移語不遜,詔詞臣草牒折之。四月,夏兵犯邊,元帥石盞合喜遇於鹿兒原,提控烏古論世顯以偏師敗之,都統王定復破其眾于新泉城。元帥慶山奴攻宥州,圍神堆府,穴其城,士卒有登者,援兵至,擊走之,斬首二千,俘百餘人,獲雜畜三千餘。八月,夏人陷會州,刺史烏古論世顯降,復犯龕谷,夾谷瑞連戰敗之,夏人乃去。是月,詔有司移文議和,事竟不克。



 夏人三萬自高峰鎮圍定西,刺史愛申阿失剌、提控烏古論長壽、溫敦永昌擊走之。九月,夏人圍綏平寨、安定堡,未幾,
 陷西寧州,遂攻定西,烏古論長壽擊卻之。乃襲鞏州,石盞合喜逆戰,一日十餘戰,乃解去。



 五年正月,詔樞密院議夏事,奏曰:「夏人聚兵境上,欲由會州入,已遣行省白撒伏兵險要以待之。鄜延元帥府伺便發兵以綴其後,足以無慮。」二月,寧遠軍節度使夾谷海壽破夏兵于搜嵬堡。三月,復取來羌城。十月,攻龕谷,白撒連敗之。元光元年正月,夏人陷大通城,復取之。三月,提控李師林敗夏兵於永木嶺。八月,攻寧安寨,十月,攻神林堡,十二月,入質孤堡,提控唐括昉敗之。



 二年,遵頊使其太子德任來伐,德任諫曰:「彼兵勢尚強,不若與之約和。」遵頊笑曰:「是非爾所知
 也。彼失蘭州竟不能復,何強之有。」德任固諫不從,乞避太子位,願為僧。遵頊怒,幽之靈州,遣人代將,會天旱不果。



 是歲,大元兵問罪夏國,延安、慶原元帥府欲乘夏人之困弊伐之,陜西行省白撒、合達以為不可,乃止。



 隴安軍節度使完顏阿鄰日與將士宴飲,不治軍事,夏人乘之,掠民五千餘口、牛羊雜畜數萬而去。



 自天會議和,八十餘年與夏人未嘗有兵革之事。及貞祐之初,小有侵掠,以至構難十年不解,一勝一負精銳皆盡,而兩國俱弊。



 是歲,遵頊傳位於子德旺。正大元年,和議成,自稱兄弟之國。



 三年二月,遵頊死,七月,德旺死,嗣立者史失其
 名。明年,夏國亡。



 先是,夏使精方匭匣使王立之來聘,未復命國已亡,詔於京兆安置,充宣差彈壓,主管夏國降戶。八年五月,立之妻子三十餘口至環州,詔以歸立之,賜以幣帛。立之上言,先世本申州人,乞不仕,居申州。詔如所請,以本官居申州,主管唐、鄧、申、裕等處夏國降戶,聽唐、鄧總帥府節制,給上田千畝、牛具農作云。



 贊曰:夏之立國舊矣,其臣羅世昌譜敘世次稱,元魏衰微,居松州者因以舊姓為托跋氏。按《唐書》黨項八部有托跋部,自黨項入居銀、夏之間者號平夏部。托跋思恭以破黃巢功賜姓李氏,兄弟相繼為節度使,居夏州,在
 河南。繼遷再立國,元昊始大,乃北渡河,城興州而都之。



 其地初有夏、綏、銀、宥、靈、鹽等州,其後遂取武威、張掖、酒泉、燉煌郡地,南界橫山,東距西河,土宜三種,善水草,宜畜牧,所謂涼州畜牧甲天下者是也。土堅腴,水清冽,風氣廣莫,民俗彊梗尚氣,重然諾,敢戰鬥。自漢、唐以水利積穀食邊兵,興州有漢、唐二渠,甘、涼亦各有灌溉,土境雖小,能以富彊,地勢然也。



 五代之際,朝興夕替,制度禮樂,盪為灰燼,唐節度使有鼓吹,故夏國聲樂清厲頓挫,猶有鼓吹之遺音焉。然能崇尚儒術,尊孔子以帝號,其文章辭命有可觀者。立國二百餘年,抗衡遼、金、宋三國,
 偭鄉無常,視三國之勢彊弱以為異同焉。故近代學者記西北地理,往往皆臆度言之。聖神有作,天下會於一,驛道往來視為東西州矣。



\end{pinyinscope}