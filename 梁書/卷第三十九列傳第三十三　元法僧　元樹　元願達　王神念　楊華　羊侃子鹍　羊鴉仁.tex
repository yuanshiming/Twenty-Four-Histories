\article{卷第三十九列傳第三十三 元法僧 元樹 元願達 王神念 楊華 羊侃子鹍 羊鴉仁}

\begin{pinyinscope}

 元法僧,魏氏之支屬也。其始祖道武帝。父鐘葵,江陽王。法僧仕魏,歷光祿大夫,後為使持節、都督徐州諸軍事、徐州刺史,鎮彭城。普通五年,魏室大亂,法僧遂據鎮稱帝,誅鋤異己,立諸子為王,部署將帥,欲議匡復。既而魏亂稍定,將討法僧。法僧懼,乃遣使歸款,請為附庸,高祖
 許焉,授侍中、司空,封始安郡公,邑五千戶。及魏軍既逼,法僧請還朝,高祖遣中書舍人朱異迎之。既至,甚加優寵。時方事招攜,撫悅降附,賜法僧甲第女樂及金帛,前後不可勝數。法僧以在魏之日,久處疆埸之任,每因寇掠,殺戮甚多,求兵自衛,詔給甲仗百人,出入禁闥。大通二年,加冠軍將軍。中大通元年,轉車騎將軍。四年,進太尉,領金紫光祿。其年,立為東魏主,不行,仍授使持節、散騎常侍、驃騎大將軍、開府同三司之儀、郢州刺史。大同二年,徵為侍中、太尉,領軍師將軍,薨,時年八十三。二子景隆、景仲,普通中隨法僧入朝。



 景隆封沌陽縣公,邑千
 戶,出為持節、都督廣、越、交、桂等十三州諸軍事、平南將軍、平越中郎將、廣州刺史。中大通三年,徵侍中、安右將軍。四年,為征北將軍、徐州刺史,封彭城王,不行,俄除侍中、度支尚書。太清初,又為使持節、都督廣、越、交、桂等十三州諸軍事、征南將軍、平越中郎將、廣州刺史,行至雷首,遇疾卒,時年五十八。



 景仲封枝江縣公,邑千戶,拜侍中、右衛將軍。大通三年,增封,并前為二千戶,仍賜女樂一部。出為持節、都督廣、越等十三州諸軍事、宣惠將軍、平越中郎將、廣州刺史。大同中,徵侍中、左衛將軍。兄景隆後為廣州刺史。侯景作亂,以景仲元氏之族,遣信誘
 之,許奉為主。景仲乃舉兵,將下應景。會西江督護陳霸先與成州刺史王懷明等起兵攻之,霸先徇其眾曰:「朝廷以元景仲與賊連從,謀危社稷,今使曲江公勃為刺史,鎮撫此州。」眾聞之,皆棄甲而散,景仲乃自縊而死。



 元樹,字君立,亦魏之近屬也。祖獻文帝。父僖,咸陽王。樹仕魏為宗正卿,屬爾朱榮亂,以天監八年歸國,封為鄴王,邑二千戶,拜散騎常侍。普通六年,應接元法僧還朝,遷使持節、督郢、司、霍三州諸軍事、雲麾將軍、郢州刺史,增封并前為三千戶。討南蠻賊,平之,加散騎常侍、安西將軍,又增邑五百戶。中大通二年,徵侍中、鎮右將軍。四
 年,為使持節,鎮北將軍,都督北討諸軍事,加鼓吹一部以伐魏,攻魏譙城,拔之。會魏將獨孤如願來援,遂圍樹,城陷被執,發憤卒於魏,時年四十八。



 子貞,大同中,求隨魏使崔長謙至鄴葬父,還拜太子舍人。太清初,侯景降,請元氏戚屬,願奉為主,詔封貞為咸陽王,以天子之禮遣還北,會景敗而返。



 元願達,亦魏之支庶也。祖明元帝。父樂平王。願達仕魏為中書令、郢州刺史。普通中,大軍北伐,攻義陽,願達舉州獻款,詔封樂平公,邑千戶,賜甲第女樂。仍出為使持節、散騎常侍、都督湘州諸軍事、平南將軍、湘州刺史。中
 大通二年,徵侍中、太中大夫、翊左將軍。大同三年,卒,時年五十七。



 王神念,太原祁人也。少好儒術,尤明內典。仕魏起家州主簿,稍遷潁川太守,遂據郡歸款。魏軍至,與家屬渡江,封南城縣侯,邑五百戶。頃之,除安成內史,又歷武陽、宣城內史,皆著治績。還除太僕卿。出為持節、都督青、冀二州諸軍事、信武將軍、青、冀二州刺史。神念性剛正,所更州郡必禁止淫祠。時青、冀州東北有石鹿山臨海,先有神廟,妖巫欺惑百姓,遠近祈禱,糜費極多。及神念至,便令毀撤,風俗遂改。普通中,大舉北伐,徵為右衛將軍。六
 年,遷使持節、散騎常侍、爪牙將軍,右衛如故。遘疾卒,時年七十五。詔贈本官、衡州刺史,兼給鼓吹一部。謚曰壯。



 神念少善騎射,既老不衰,嘗於高祖前手執二刀楯,左右交度,馳馬往來,冠絕群伍。時復有楊華者,能作驚軍騎,並一時妙捷,高祖深歎賞之。



 子尊業,仕至太僕卿。卒,贈信威將軍、青、冀二州刺史,鼓吹一部。次子僧辯,別有傳。



 楊華,武都仇池人也。父大眼,為魏名將。華少有勇力,容貌雄偉,魏胡太后逼通之,華懼及禍,乃率其部曲來降。胡太后追思之不能已,為作《楊白華歌辭》,使宮人晝夜
 連臂蹋足歌之,辭甚心妻惋焉。華後累征伐,有戰功,歷官太僕卿,太子左衛率,封益陽縣侯。太清中,侯景亂,華欲立志節,妻子為賊所擒,遂降之,卒於賊。



 羊侃,字祖忻,泰山梁甫人,漢南陽太守續之裔也。祖規,宋武帝之臨徐州,辟祭酒從事、大中正。會薛安都舉彭城降北,規由是陷魏,魏授衛將軍、營州刺史。父祉,魏侍中,金紫光祿大夫。侃少而瑰偉,身長七尺八寸,雅愛文史,博涉書記,尤好《左氏春秋》及《孫吳兵法》。弱冠隨父在梁州立功。魏正光中,稍為別將。時秦州羌有莫遮念生者,據州反,稱帝,仍遣其弟天生率眾攻陷岐州,遂寇雍
 州。侃為偏將,隸蕭寶夤往討之,潛身巡緌,伺射天生,應弦即倒,其眾遂潰。以功遷使持節、征東大將軍、東道行臺,領泰山太守,進爵鉅平侯。



 初,其父每有南歸之志,常謂諸子曰:「人生安可久淹異域,汝等可歸奉東朝。」侃至是將舉河濟以成先志。兗州刺史羊敦,侃從兄也,密知之,據州拒侃。侃乃率精兵三萬襲之,弗剋,仍築十餘城以守之。朝廷賞授,一與元法僧同。遣羊鴉仁、王弁率軍應接,李元履運給糧仗。魏帝聞之,使授侃驃騎大將軍、司徒、泰山郡公,長為兗州刺史,侃斬其使者以徇。魏人大駭,令僕射于暉率眾數十萬,及高歡、爾朱陽都等相
 繼而至,圍侃十餘重,傷殺甚眾。柵中矢盡,南軍不進,乃夜潰圍而出,且戰且行,一日一夜乃出魏境。至渣口,眾尚萬餘人,馬二千匹,將入南,士卒並竟夜悲歌。侃乃謝曰:「卿等懷土,理不能見隨,幸適去留,於此別異。」因各拜辭而去。



 侃以大通三年至京師,詔授使持節、散騎常侍、都督瑕丘征討諸軍事、安北將軍、徐州刺史,并其兄默及三弟忱、給、元,皆拜為刺史。尋以侃為都督北討諸軍事,出頓日城,會陳慶之失律,停進。其年,詔以為持節、雲麾將軍、青、冀二州刺史。中大通四年,詔為使持節、都督瑕丘諸軍事、安北將軍、兗州刺史,隨太尉元法僧北討。
 法僧先啟云:「與侃有舊,願得同行。」高祖乃召侃問方略,侃具陳進取之計。高祖因曰:「知卿願與太尉同行。」侃曰:「臣拔迹還朝,常思效命,然實未曾願與法僧同行。北人雖謂臣為吳,南人已呼臣為虜,今與法僧同行,還是群類相逐,非止有乖素心,亦使匈奴輕漢。」高祖曰:「朝廷今者要須卿行。」乃詔以為大軍司馬。高祖謂侃曰;「軍司馬廢來已久,此段為卿置之。」行次官竹,元樹又於譙城喪師。軍罷,入為侍中。五年,封高昌縣侯,邑千戶。六年,出為雲麾將軍、晉安太守。閩越俗好反亂,前後太守莫能止息,侃至討擊,斬其渠帥陳稱、吳滿等,於是郡內肅清,莫
 敢犯者。頃之,徵太子左衛率。



 大同三年,車駕幸樂遊苑,侃預宴。時少府奏新造兩刃槊成,長二丈四尺,圍一尺三寸,高祖因賜侃馬,令試之。侃執槊上馬,左右擊刺,特盡其妙,高祖善之,又製《武宴詩》三十韻以示侃,侃即席應詔,高祖覽曰:「吾聞仁者有勇,今見勇者有仁,可謂鄒、魯遺風,英賢不絕。」六年,遷司徒左長史。八年,遷都官尚書。時尚書令何敬容用事,與之並省,未嘗遊造。有宦者張僧胤候侃,侃曰:「我床非閹人所坐。」竟不前之,時論美其貞正。九年,出為使持節、壯武將軍、衡州刺史。



 太清元年,徵為侍中。會大舉北伐,仍以侃為持節、冠軍,監作韓山
 堰事,兩旬堰立。侃勸元帥貞陽侯乘水攻彭城,不納;既而魏援大至,侃頻勸乘其遠來可擊,旦日又勸出戰,並不從,侃乃率所領出頓堰上。及眾軍敗,侃結陣徐還。



 二年,復為都官尚書。侯景反,攻陷歷陽,高祖問侃討景之策。侃曰:「景反迹久見,或容豕突,宜急據采石,令邵陵王襲取壽春。景進不得前,退失巢窟,烏合之眾,自然瓦解。」議者謂景未敢便逼京師,遂寢其策,令侃率千餘騎頓望國門。景至新林,追侃入副宣城王都督城內諸軍事。時景既卒至,百姓競入,公私混亂,無復次第。侃乃區分防擬,皆以宗室間之。軍人爭入武庫,自取器甲,所司不
 能禁,侃命斬數人,方得止。及賊逼城,眾皆恟懼,侃偽稱得射書,云「邵陵王、西昌侯已至近路」。眾乃少安。賊攻東掖門,縱火甚盛,侃親自距抗,以水沃火,火滅,引弓射殺數人,賊乃退。加侍中、軍師將軍。有詔送金五千兩,銀萬兩,絹萬匹,以賜戰士,侃辭不受。部曲千餘人,並私加賞賚。



 賊為尖頂木驢攻城,矢石所不能制,侃作雉尾炬,施鐵鏃,以油灌之,擲驢上焚之,俄盡。賊又東西兩面起土山,以臨城,城中震駭,侃命為地道,潛引其土,山不能立。賊又作登城樓車,高十餘丈,欲臨射城內,侃曰:「車高緌虛,彼來必倒,可臥而觀之,不勞設備。」及車動果倒,眾皆
 服焉。賊既頻攻不捷,乃築長圍。朱異、張綰議欲出擊之,高祖以問侃,侃曰:「不可。賊多日攻城,既不能下,故立長圍,欲引城中降者耳。今擊之,出人若少,不足破賊,若多,則一旦失利,自相騰踐,門隘橋小,必大致挫衄,此乃示弱,非騁王威也。」不從,遂使千餘人出戰,未及交鋒,望風退走,果以爭橋赴水,死者太半。



 初,侃長子躭為景所獲,執來城下示侃,侃謂曰:「我傾宗報主,猶恨不足,豈復計此一子,幸汝早能殺之。」數日復持來,侃謂躭曰:「久以汝為死,猶復在邪?吾以身許國,誓死行陣,終不以爾而生進退。」因引弓射之。賊感其忠義,亦不之害也。景遣儀同
 傅士哲呼侃與語曰:「侯王遠來問訊天子,何為閉距,不時進納?尚書國家大臣,宜啟朝廷。」侃曰:「侯將軍奔亡之後,歸命國家,重鎮方城,懸相任寄,何所患苦?忽致稱兵?今驅烏合之卒,至王城之下,虜馬飲淮,矢集帝室,豈有人臣而至於此?吾荷國重恩,當稟承廟算,以掃大逆耳,不能妄受浮說,開門揖盜。幸謝侯王,早自為所。」士哲又曰:「侯王事君盡節,不為朝廷所知,正欲面啟至尊,以除姦佞,既居戎旅,故帶甲來朝,何謂作逆?」侃曰:「聖上臨四海將五十年,聰明睿哲,無幽不照,有何姦佞而得在朝?欲飾其非,寧無詭說。且侯王親舉白刃,以向城闕,事君
 盡節,正若是邪!」士哲無以應,乃曰:「在北之日,久挹風猷,每恨平生,未獲披敘,願去戎服,得一相見。」侃為之免胄,士哲瞻望久之而去。其為北人所欽慕如此。



 後大雨,城內土山崩,賊乘之垂入,苦戰不能禁,侃乃令多擲火,為火城以斷其路,徐於裏築城,賊不能進。十二月,遘疾卒于臺內,時年五十四。詔給東園秘器,布絹各五百匹,錢三百萬,贈侍中、護軍將軍,鼓吹一部。



 侃少而雄勇,膂力絕人,所用弓至十餘石。嘗於兗州堯廟蹋壁,直上至五尋,橫行得七跡。泗橋有數石人,長八尺,大十圍,侃執以相擊,悉皆破碎。



 侃性豪侈,善音律,自造《採蓮》、《棹歌》兩曲,
 甚有新致。姬妾侍列,窮極奢靡。有彈箏人陸太喜,著鹿角爪長七寸。儛人張凈琬,腰圍一尺六寸,時人咸推能掌中儛。又有孫荊玉,能反腰帖地,銜得席上玉簪。敕賚歌人王娥兒,東宮亦賚歌者屈偶之,並妙盡奇曲,一時無對。初赴衡州,於兩艖符,起三間通梁水齋,飾以珠玉,加之錦繢,盛設帷屏,陳列女樂,乘潮解纜,臨波置酒,緣塘傍水,觀者填咽。大同中,魏使陽斐,與侃在北嘗同學,有詔令侃延斐同宴。賓客三百餘人,器皆金玉雜寶,奏三部女樂,至夕,侍婢百餘人,俱執金花燭。侃不能飲酒,而好賓客交遊,終日獻酬,同其醉醒。性寬厚,有器局,嘗
 南還至漣口,置酒,有客張孺才者,醉於船中失火,延燒七十餘艘,所燔金帛不可勝數。侃聞之,都不挂意,命酒不輟。孺才慚懼,自逃匿,侃慰喻使還,待之如舊。



 第三子鵾。鵾字子鵬。隨侃臺內,城陷,竄於陽平。侯景呼還,待之甚厚。及景敗,鵾密圖之,乃隨其東走。景於松江戰敗,惟餘三舸,下海欲向蒙山。會景倦晝寢,鵾語海師:「此中何處有蒙山!汝但聽我處分。」遂直向京口。至胡豆洲,景覺,大驚,問岸上人,云「郭元建猶在廣陵」,景大喜,將依之。鵾拔刀叱海師,使向京口。景欲透水,鵾抽刀斫之,景乃走入船中,以小刀抉船,鵾以槊入刺殺之。世祖以鵾為持
 節、通直散騎常侍、都督青、冀二州諸軍事、明威將軍、青州刺史,封昌國縣公,邑二千戶,賜錢五百萬,米五千石,布絹各一千匹,又領東陽太守。征陸納,加散騎常侍。平峽中,除西晉州刺史。破郭元建於東關,遷使持節、信武將軍、東晉州刺史。承聖三年,西魏圍江陵,鵾赴援不及,從王僧愔征蕭勃於嶺表。聞大尉僧辯敗,乃還,為侯瑱所破,於豫章遇害,時年二十八。



 羊鴉仁,字孝穆,太山鉅平人也。少驍果有膽力,仕郡為主簿。普通中,率兄弟自魏歸國,封廣晉縣侯。征伐青、齊間,累有功績,稍遷員外散騎常侍、歷陽太守。中大通四
 年,為持節、都督譙州諸軍事、信威將軍、譙州刺史。大同七年,除太子左衛率,出為持節、都督南、北司、豫、楚四州諸軍事、輕車將軍、北司州刺史。侯景降,詔鴉仁督士州刺史桓和之、仁州刺史湛海珍等精兵三萬,趨懸瓠應接景,仍為都督豫、司、淮、冀、殷、應、西豫等七州諸軍事、司、豫二州刺史,鎮懸瓠。會侯景敗於渦陽,魏軍漸逼,鴉仁恐糧運不繼,遂還北司,上表陳謝。高祖大怒,責之,鴉仁懼,又頓軍於淮上。及侯景反,鴉仁率所部入援。太清二年,景既背盟,鴉仁乃與趙伯超及南康王會理共攻賊於東府城,反為賊所敗。臺城陷,鴉仁見景,為景所留,以
 為五兵尚書。鴉仁常思奮發,謂所親曰:「吾以凡流,受寵朝廷,竟無報效,以答重恩。社稷傾危,身不能死,偷生茍免,以至於今。若以此終,沒有餘憤。」因遂泣下,見者傷焉。三年,出奔江西,其故部曲數百人迎之,將赴江陵,至東莞,為故北徐州刺史荀伯道諸子所害。



 史臣曰:高祖革命受終,光期寶運,威德所漸,莫不懷來,其皆殉難投身,前後相屬。元法僧之徒入國,並降恩遇,位重任隆,擊鐘鼎食,美矣。而羊侃、鴉仁值太清之難,並竭忠奉國。侃則臨危不撓,鴉仁守義殞命,可謂志等松筠,心均鐵石,古之殉節,斯其謂乎!



\end{pinyinscope}