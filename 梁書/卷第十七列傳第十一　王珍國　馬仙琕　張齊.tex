\article{卷第十七列傳第十一 王珍國 馬仙琕 張齊}

\begin{pinyinscope}

 王珍國,字德重,沛國相人也。父廣之,齊世良將,官至散騎常侍、車騎將軍。珍國起家冠軍行參軍,累遷虎賁中郎將、南譙太守,治有能名。時郡境苦飢,乃發米散財,以拯窮乏。齊高帝手敕云:「卿愛人治國,甚副吾意也。」永明初,遷桂陽內史,討捕盜賊,境內肅清。罷任還都,路經江州,刺史柳世隆臨渚餞別,見珍國還裝輕素,乃歎曰:「此
 真可謂良二千石也!」還為大司馬中兵參軍。武帝雅相知賞,每歎曰:「晚代將家子弟,有如珍國者少矣。」復出為安成內史。入為越騎校尉,冠軍長史、鐘離太守。仍遷巴東、建平二郡太守。還為游擊將軍,以父憂去職。



 建武末,魏軍圍司州,明帝使徐州刺史裴叔業攻拔渦陽,以為聲援,起珍國為輔國將軍,率兵助焉。魏將楊大眼大眾奄至,叔業懼,棄軍走,珍國率其眾殿,故不至大敗。永泰元年,會稽太守王敬則反,珍國又率眾距之。敬則平,遷寧朔將軍、青、冀二州刺史,將軍如故。



 義師起,東昏召珍國以眾還京師,入頓建康城。義師至,使珍國出屯朱雀
 門,為王茂軍所敗,乃入城。仍密遣卻纂奉明鏡獻誠於高祖,高祖斷金以報之。時城中咸思從義,莫敢先發,侍中、衛尉張稷都督眾軍,珍國潛結稷腹心張齊要稷,稷許之。十二月丙寅旦,珍國引稷於衛尉府,勒兵入自雲龍門,即東昏於內殿斬之,與稷會尚書僕射王亮等於西鐘下,使中書舍人裴長穆等奉東昏首歸高祖。以功授右衛將軍,辭不拜;又授徐州刺史,固乞留京師。復賜金帛,珍國又固讓。敕答曰:「昔田子泰固辭絹穀。卿體國情深,良在可嘉。」後因侍宴,帝問曰:「卿明鏡尚存,昔金何在?」珍國答曰:「黃金謹在臣肘,不敢失墜。」復為右衛將軍,
 加給事中,遷左衛將軍,加散騎常侍。天監初,封灄陽縣侯,邑千戶。除都官尚書,常侍如故。



 五年,魏任城王元澄寇鐘離,高祖遣珍國,因問討賊方略。珍國對曰:「臣常患魏眾少,不苦其多。」高祖壯其言,乃假節,與眾軍同討焉。魏軍退,班師。出為使持節、都督梁、秦二州諸軍事、征虜將軍、南秦、梁二州刺史。會梁州長史夏侯道遷以州降魏,珍國步道出魏興,將襲之,不果,遂留鎮焉。以無功,累表請解,高祖弗許。改封宜陽縣侯,戶邑如前。徵還為員外散騎常侍、太子右衛率,加後軍。頃之,復為左衛將軍。九年,出為使持節、都督湘州諸軍事、信武將軍、湘州刺
 史。視事四年,徵還為護軍將軍,遷通直散騎常侍、丹陽尹。十四年,卒。詔贈車騎將軍,給鼓吹一部,賻錢十萬,布百匹。謚曰威。子僧度嗣。



 馬仙琕,字靈馥,扶風郿人也。父伯鸞,宋冠軍司馬。仙琕少以果敢聞,遭父憂,毀瘠過禮,負土成墳,手植松柏。起家郢州主簿,遷武騎常侍,為小將,隨齊安陸王蕭緬。緬卒,事明帝。永元中,蕭遙光、崔慧景亂,累有戰功,以勳至前將軍。出為龍驤將軍、南汝陰、譙二郡太守。會壽陽新陷,魏將王肅侵邊,仙琕力戰,以寡克眾,魏人甚憚之。復以功遷寧朔將軍、豫州刺史。



 義師起,四方多響應,高祖
 使仙琕故人姚仲賓說之,仙琕於軍斬仲賓以徇。義師至新林,仙琕猶持兵於江西,日鈔運漕,建康城陷,仙琕號哭經宿,乃解兵歸罪。高祖勞之曰:「射鉤斬袪,昔人弗忌。卿勿以戮使斷運,茍自嫌絕也。」仙琕謝曰:「小人如失主犬,後主飼之,便復為用。」高祖笑而美之。俄而仙琕母卒,高祖知其貧,賻給甚厚。仙琕號泣,謂弟仲艾曰:「蒙大造之恩,未獲上報。今復荷殊澤,當與爾以心力自效耳。」



 天監四年,王師北討,仙琕每戰,勇冠三軍,當其衝者,莫不摧破。與諸將論議,口未嘗言功。人問其故,仙琕曰:「丈夫為時所知,當進不求名,退不逃罪,乃平生願也。何功
 可論!」授輔國將軍、宋安、安蠻二郡太守,遷南義陽太守。累破山蠻,郡境清謐。以功封浛洭縣伯,邑四百戶,仍遷都督司州諸軍事、司州刺史,輔國將軍如故。俄進號貞威將軍。



 魏豫州人白皁生殺其刺史瑯邪王司馬慶曾,自號平北將軍,推鄉人胡遜為刺史,以懸瓠來降。高祖使仙琕赴之,又遣直閣將軍武會超、馬廣率眾為援。仙琕進頓楚王城,遣副將齊茍兒以兵二千助守懸瓠。魏中山王元英率眾十萬攻懸瓠,仙琕遣廣、會超等守三關。十二月,英破懸瓠,執齊茍兒,遂進攻馬廣,又破廣,生擒之,送雒陽。仙琕不能救。會超等亦相次退散,魏軍遂
 進據三關。仙琕坐徵還,為雲騎將軍。出為仁威司馬,府主豫章王轉號雲麾,復為司馬,加振遠將軍。



 十年,朐山民殺瑯邪太守劉晣,以城降魏,詔假仙琕節,討之。魏徐州刺史盧昶以眾十餘萬赴焉。仙琕與戰,累破之,昶遁走。仙琕縱兵乘之,魏眾免者十一二,收其兵糧牛馬器械,不可勝數。振旅還京師,遷太子左衛率,進爵為侯,增邑六百戶。十一年,遷持節、督豫、北豫、霍三州諸軍事、信武將軍、豫州刺史,領南汝陰太守。



 初,仙琕幼名仙婢,及長,以「婢」名不典,乃以「玉」代「女」,因成「琕」云。自為將及居州郡,能與士卒同勞逸。身衣不過布帛,所居無帷幕衾屏,
 行則飲食與廝養最下者同。其在邊境,常單身潛入敵庭,伺知壁壘村落險要處所,故戰多克捷,士卒亦甘心為之用,高祖雅愛仗之。在州四年,卒。贈左衛將軍。謚曰剛。子巖夫嗣。



 張齊,字子響,馮翊郡人。世居橫桑,或云橫桑人也。少有膽氣。初事荊府司馬垣歷生。歷生酗酒,遇下嚴酷,不甚禮之。歷生罷官歸,吳郡張稷為荊府司馬,齊復從之,稷甚相知重,以為心腹,雖家居細事,皆以任焉。齊盡心事稷,無所辭憚。隨稷歸京師。稷為南兗州,又擢為府中兵參軍,始委以軍旅。



 齊永元中,義師起,東昏征稷歸,都督
 宮城諸軍事,居尚書省。義兵至,外圍漸急,齊日造王珍國,陰與定計。計定,夜引珍國就稷造膝,齊自執燭以成謀。明旦,與稷、珍國即東昏於內殿,齊手刃焉。明年,高祖受禪,封齊安昌縣侯,邑五百戶,仍為寧朔將軍、歷陽太守。齊手不知書,目不識字,而在郡有清政,吏事甚修。



 天監二年,還為虎賁中郎將。未拜,遷天門太守,寧朔將軍如故。四年,魏將王足寇巴、蜀,高祖以齊為輔國將軍救蜀。未至,足退走,齊進戍南安。七年秋,使齊置大劍、寒冢二戍,軍還益州。其年,遷武旅將軍、巴西太守,尋加征遠將軍。十年,郡人姚景和聚合蠻蜒,抄斷江路,攻破金井。
 齊討景和於平昌,破之。



 初,南鄭沒於魏,乃於益州西置南梁州。州鎮草創,皆仰益州取足。齊上夷獠義租,得米二十萬斛。又立臺傳,興冶鑄,以應贍南梁。



 十一年,進假節、督益州外水諸軍。十二年,魏將傅豎眼寇南安,齊率眾距之,豎眼退走。十四年,遷信武將軍、巴西、梓潼二郡太守。是歲,葭萌人任令宗因眾之患魏也,殺魏晉壽太守,以城歸款。益州刺史鄱陽王遣齊帥眾三萬,督南梁州長史席宗範諸軍迎令宗。十五年,魏東益州刺史元法僧遣子景隆來拒齊師,南安太守皇甫諶及宗範逆擊之,大破魏軍於葭萌,屠十餘城,魏將丘突、王穆等皆
 降。而魏更增傅豎眼兵,復來拒戰,齊兵少不利,軍引還,於是葭萌復沒於魏。



 齊在益部累年,討擊蠻獠,身無寧歲。其居軍中,能身親勞辱,與士卒同其勤苦。自畫頓舍城壘,皆委曲得其便,調給衣糧資用,人人無所困乏。既為物情所附,蠻獠亦不敢犯,是以威名行於庸、蜀。巴西郡居益州之半,又當東道衝要,刺史經過,軍府遠涉,多所窮匱。齊沿路聚糧食,種蔬菜,行者皆取給焉。其能濟辦,多此類也。



 十七年,遷持節、都督南梁州諸軍事、智武將軍、南梁州刺史。普通四年,遷信武將軍、征西鄱陽王司馬、新興、永寧二郡太守。未發而卒,時年六十七。追贈
 散騎常侍、右衛將軍。賻錢十萬,布百匹。謚曰壯。



 陳吏部尚書姚察曰:王珍國、申胄、徐元瑜、李居士,齊末咸為列將,擁強兵,或面縛請罪,或斬關獻捷;其能後服,馬仙琕而已。仁義何常,蹈之則為君子,信哉!及其臨邊撫眾,雖李牧無以加矣。張齊之政績,亦有異焉。胄、元瑜、居士入梁事跡鮮,故不為之傳。



\end{pinyinscope}