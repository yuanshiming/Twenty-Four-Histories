\article{卷第十八列傳第十二 張惠紹 馮道根 康絢 昌義之}

\begin{pinyinscope}

 張惠紹,字德繼,義陽人也。少有武幹。齊明帝時為直閣,後出補竟陵橫桑戍主。永元初,母喪歸葬於鄉里。聞義師起,馳歸高祖,板為中兵參軍,加寧朔將軍、軍主。師次漢口,高祖使惠紹與軍主朱思遠遊遏江中,斷郢、魯二城糧運。郢城水軍主沈難當帥輕舸數十挑戰,惠紹擊破,斬難當,盡獲其軍器。義師次新林、朱雀,惠紹累有戰
 功。建康城平,遷輔國將軍、前軍,直閣、左細仗主。高祖踐阼,封石陽縣侯,邑五百戶。遷驍騎將軍,直閣、細仗主如故。時東昏餘黨數百人,竊入南北掖門,燒神虎門,害衛尉張弘策。惠紹馳率所領赴戰,斬首數十級,賊乃散走。以功增邑二百戶。遷太子右衛率。



 天監四年,大舉北伐,惠紹與冠軍長史胡辛生、寧朔將軍張豹子攻宿預,執城主馬成龍,送于京師。使部將藍懷恭於水南立城為掎角。俄而魏援大至,敗陷懷恭,惠紹不能守,是夜奔還淮陰,魏復得宿預。六年,魏軍攻鐘離,詔左衛將軍曹景宗督眾軍為援,進據邵陽。惠紹與馮道根、裴邃等攻斷
 魏連橋,短兵接戰,魏軍大潰。以功增邑三百戶,還為左驍騎將軍。尋出為持節、都督北兗州諸軍事、冠軍將軍、北兗州刺史。魏宿預、淮陽二城內附,惠紹撫納有功,進號智武將軍,益封二百戶。入為衛尉卿,遷左衛將軍。出為持節、都督司州諸軍事、信威將軍、司州刺史、領安陸太守。在州和理,吏民親愛之。



 征還為左衛將軍,加通直散騎常侍,甲仗百人,直衛殿內。十八年,卒,時年六十三。詔曰:「張惠紹志略開濟,乾用貞果。誠勤義始,績聞累任。爰居禁旅,盡心朝夕。奄至殞喪,惻愴於懷。宜追寵命,以彰勛烈。可贈護軍將軍,給鼓吹一部,布百匹,蠟二百斤。
 謚曰忠。」子澄嗣。



 澄初為直閣將軍,丁父憂,起為晉熙太守,隨豫州刺史裴邃北伐,累有戰功,與湛僧智、胡紹世、魚弘並當時之驍將。歷官衛尉卿、太子左衛率。卒官,謚曰愍。



 馮道根,字巨基,廣平酂人也。少失父,家貧,傭賃以養母。行得甘肥,不敢先食,必遽還以進母。年十三,以孝聞於鄉里。郡召為主簿,辭不就。年十六,鄉人蔡道斑為湖陽戍主,道斑攻蠻錫城,反為蠻所困,道根救之。匹馬轉戰,殺傷甚多,道斑以免,由是知名。



 齊建武末,魏主托跋宏寇沒南陽等五郡,明帝遣太尉陳顯達率眾復爭之。師
 入汮均口,道根與鄉里人士以牛酒候軍,因說顯達曰:「汋均水迅急,難進易退。魏若守隘,則首尾俱急。不如悉棄船艦於酂城,方道步進,建營相次,鼓行而前。如是,則立破之矣。」顯達不聽,道根猶以私屬從軍。及顯達敗,軍人夜走,多不知山路;道根每及險要,輒停馬指示之,眾賴以全。尋為汮均口戍副。



 永元中,以母喪還家。聞高祖起義師,乃謂所親曰:「金革奪禮,古人不避,揚名後世,豈非孝乎?時不可失,吾其行矣。」率鄉人子弟勝兵者,悉歸高祖。時有蔡道福為將從軍,高祖使道根副之,皆隸於王茂。茂伐沔,攻郢城,克加湖,道根常為前鋒陷陳。會道
 福卒於軍,高祖令道根并領其眾。大軍次新林,隨王茂於朱雀航大戰,斬獲尤多。高祖即位,以為驍騎將軍,封增城縣男,邑二百戶。領文德帥,遷游擊將軍。是歲,江州刺史陳伯之反,道根隨王茂討平之。



 天監二年,為寧朔將軍、南梁太守,領阜陵城戍。初到阜陵,修城隍,遠斥候,有如敵將至者,眾頗笑之。道根曰:「怯防勇戰,此之謂也。」修城未畢,會魏將黨法宗、傅豎眼率眾二萬,奄至城下。道根塹壘未固,城中眾少,皆失色。道根命廣開門,緩服登城,選精銳二百人,出與魏軍戰,敗之。魏人見意閑,且戰又不利,因退走。是時魏分兵於大小峴、東桑等,連城
 相持。魏將高祖珍以三千騎軍其間,道根率百騎橫擊破之,獲其鼓角軍儀。於是糧運既絕,諸軍乃退。遷道根輔國將軍。



 豫州刺史韋睿圍合肥,克之。道根與諸軍同進,所在有功。六年,魏攻鐘離,高祖復詔睿救之,道根率眾三千為睿前驅。至徐州,建計據邵陽洲,築壘掘塹,以逼魏城。道根能走馬步地,計馬足以賦功,城隍立辦。及淮水長,道根乘戰艦,攻斷魏連橋數百丈,魏軍敗績。益封三百戶,進爵為伯。還,遷雲騎將軍、領直閣將軍,改封豫寧縣,戶邑如前。累遷中權中司馬、右游擊將軍、武旅將軍、歷陽太守。八年,遷貞毅將軍、假節、督豫州諸軍事、豫
 州刺史、領汝陰太守。為政清簡,境內安定。十一年,徵為太子右衛率。十三年,出為信武將軍、宣惠司馬、新興、永寧二郡太守。十四年,徵為員外散騎常侍、右游擊將軍,領朱衣直閣。十五年,為右衛將軍。



 道根性謹厚,木訥少言,為將能檢御部曲,所過村陌,將士不敢虜掠。每所征伐,終不言功,諸將言雚嘩爭競,道根默然而已。其部曲或怨非之,道根喻曰:「明主自鑒功之多少,吾將何事。」高祖嘗指道根示尚書令沈約曰:「此人口不論勳。」約曰:「此陛下之大樹將軍也。」處州郡,和理清靜,為部下所懷。在朝廷,雖貴顯而性儉約,所居宅不營墻屋,無器服侍衛,入
 室則蕭然如素士之貧賤者。當時服其清退,高祖亦雅重之。微時不學,既貴,粗讀書,自謂少文,常慕周勃之器重。



 十六年,復假節、都督豫州諸軍事、信武將軍、豫州刺史。將行,高祖引朝臣宴別道根於武德殿,召工視道根,使圖其形像。道根踧謝曰:「臣所可報國家,惟餘一死;但天下太平,臣恨無可死之地。」豫部重得道根,人皆喜悅。高祖每稱曰:「馮道根所在,能使朝廷不復憶有一州。」



 居州少時,遇疾,自表乞還朝,徵為散騎常侍、左軍將軍。既至疾甚,中使累加存問。普通元年正月,卒,時年五十八。是日輿駕春祠二廟,既出宮,有司以聞。高祖問中書
 舍人朱異曰:「吉凶同日,今行乎?」異對曰:「昔柳莊寢疾,衛獻公當祭,請於尸曰:『有臣柳莊,非寡人之臣,是社稷之臣也,聞其死,請往。』不釋祭服而往,遂以襚之。道根雖未為社稷之臣,亦有勞王室,臨之,禮也。」高祖即幸其宅,哭之甚慟。詔曰:「豫寧縣開國伯、新除散騎常侍、領左軍將軍馮道根,奉上能忠,有功不伐,撫人留愛,守邊難犯,祭遵、馮異、郭人及、李牧,不能過也。奄致殞喪,惻愴于懷。可贈信威將軍、左衛將軍,給鼓吹一部。賻錢十萬,布百匹。謚曰威。」子懷嗣。



 康絢,字長明,華山藍田人也。其先出自康居。初,漢置都
 護,盡臣西域。康居亦遣侍子待詔於河西,因留為黔首,其後即以康為姓。晉時隴右亂,康氏遷于藍田。絢曾祖因為苻堅太子詹事,生穆,穆為姚萇河南尹。宋永初中,穆舉鄉族三千餘家,入襄陽之峴南。宋為置華山郡藍田縣,寄居于襄陽,以穆為秦、梁二州刺史。未拜,卒。絢世父元隆,父元撫,並為流人所推,相繼為華山太守。



 絢少俶儻有志氣。齊文帝為雍州刺史,所辟皆取名家,絢特以才力召為西曹書佐。永明三年,除奉朝請。文帝在東宮,以舊恩引為直後,以母憂去職。服闋,除振威將軍、華山太守。推誠撫循,荒餘悅服。遷前軍將軍,復為華山太
 守。



 永元元年,義兵起,絢舉郡以應高祖,身率敢勇三千人,私馬二百五十匹以從。除西中郎南康王中兵參軍,加輔國將軍。義師方圍張沖於郢城,曠日持久,東昏將吳子陽壁于加湖,軍鋒甚盛,絢隨王茂力攻屠之。自是常領遊兵,有急應赴,斬獲居多。天監元年,封南安縣男,邑三百戶。除輔國將軍、竟陵太守。魏圍梁州,刺史王珍國使請救,絢以郡兵赴之,魏軍退。七年,司州三關為魏所逼,詔假絢節、武旅將軍,率眾赴援。九年,遷假節、督北兗州緣淮諸軍事、振遠將軍、北兗州刺史。及朐山亡徒以城降魏,絢馳遣司馬霍奉伯分軍據險。魏軍至,不得
 越朐城。明年,青州刺史張稷為土人徐道角所殺,絢又遣司馬茅榮伯討平之。征驃騎臨川王司馬,加左驍騎將軍,尋轉朱衣直閣。十三年,遷太子右衛率,甲仗百人,與領軍蕭景直殿內。



 絢身長八尺,容貌絕倫,雖居顯官,猶習武藝。高祖幸德陽殿戲馬,敕絢馬射,撫弦貫的,觀者悅之。其日,上使畫工圖絢形,遣中使持以問絢曰:「卿識此圖不?」其見親如此。



 時魏降人王足陳計,求堰淮水以灌壽陽。足引北方童謠曰:「荊山為上格,浮山為下格,潼沱為激溝,併灌鉅野澤。」高祖以為然,使水工陳承伯、材官將軍祖芃視地形,咸謂淮內沙土漂輕,不堅實,其
 功不可就。高祖弗納,發徐、揚人,率二十戶取五丁以築之。假絢節、都督淮上諸軍事,並護堰作,役人及戰士,有眾二十萬。於鐘離南起浮山,北抵巉石,依岸以築土,合脊於中流。十四年,堰將合,淮水漂疾,輒復決潰,眾患之。或謂江、淮多有蛟,能乘風雨決壞崖岸,其性惡鐵,因是引東西二冶鐵器,大則釜鬵,小則鋘鋤,數千萬斤,沉于堰所。猶不能合,乃伐樹為井幹,填以巨石,加土其上。緣淮百里內,岡陵木石,無巨細必盡,負擔者肩上皆穿。夏日疾疫,死者相枕,蠅蟲晝夜聲相合。高祖愍役人淹久,遣尚書右僕射袁昂、侍中謝舉假節慰勞之,并加蠲復。
 是冬又寒甚,淮、泗盡凍,士卒死者十七八,高祖復遣賜以衣褲。十一月,魏遣將楊大眼揚聲決堰,絢命諸軍撤營露次以待之。遣其子悅挑戰,斬魏咸陽王府司馬徐方興,魏軍小卻。十二月,魏遣其尚書僕射李曇定督眾軍來戰,絢與徐州刺史劉思祖等距之。高祖又遣右衛將軍昌義之、太僕卿魚弘文、直閣曹世宗、徐元和相次距守。十五年四月,堰乃成。其長九里,下闊一百四十丈,上廣四十五丈,高二十丈,深十九丈五尺。夾之以堤,并樹杞柳,軍人安堵,列居其上。其水清潔,俯視居人墳墓,了然皆在其下。或人謂絢曰:「四瀆,天所以節宣其氣,不
 可久塞。若鑿湫東注,則游波寬緩,堰得不壞。」絢然之,開湫東注。又縱反間於魏曰:「梁人所懼開湫,不畏野戰。」魏人信之,果鑿山深五丈,開湫北注,水日夜分流,湫猶不減。其月,魏軍竟潰而歸。水之所及,夾淮方數百里地。魏壽陽城戍稍徙頓於八公山,此南居人散就岡壟。



 初,堰起於徐州界,刺史張豹子宣言於境,謂己必尸其事。既而絢以他官來監作,豹子甚慚。俄而敕豹子受絢節度,每事輒先諮焉,由是遂譖絢與魏交通,高祖雖不納,猶以事畢征絢。尋以絢為持節、都督司州諸軍事、信武將軍、司州刺史,領安陸太守,增封二百戶。絢還後,豹子不
 修堰,至其秋八月,淮水暴長,堰悉壞決,奔流于海,祖芃坐下獄。絢在州三年,大修城隍,號為嚴政。



 十八年,徵為員外散騎常侍,領長水校尉,與護軍韋睿、太子右衛率周捨直殿省。普通元年,除衛尉卿,未拜,卒,時年五十七。輿駕即日臨哭。贈右衛將軍,給鼓吹一部。賻錢十萬,布百匹。謚曰壯。



 絢寬和少喜懼,在朝廷,見人如不能言,號為長厚。在省,每寒月見省官繿縷,輒遺以襦衣,其好施如此。子悅嗣。



 昌義之,歷陽烏江人也。少有武幹。齊代隨曹虎征伐,累有戰功。虎為雍州,以義之補防閣,出為馮翊戍主。及虎
 代還,義之留事高祖。時天下方亂,高祖亦厚遇之。義師起,板為輔國將軍、軍主,除建安王中兵參軍。時竟陵芊口有邸閣,高祖遣驅,每戰必捷。大軍次新林,隨王茂於新亭,并朱雀航力戰,斬獲尤多。建康城平,以為直閣將軍、馬右夾轂主。天監元年,封永豊縣侯,邑五百戶。除驍騎將軍。出為盱眙太守。二年,遷假節、督北徐州諸軍事、輔國將軍、北徐州刺史,鎮鐘離。魏寇州境,義之擊破之。三年,進號冠軍將軍,增封二百戶。



 四年,大舉北伐,揚州刺史臨川王督眾軍軍洛口,義之以州兵受節度,為前軍,攻魏梁城戍,克之。五年,高祖以征役久,有詔班師,眾
 軍各退散,魏中山王元英乘勢追躡,攻沒馬頭,城內糧儲,魏悉移之歸北。議者咸曰:「魏運米北歸,當無復南向。」高祖曰:「不然,此必進兵,非其實也。」乃遣土匠修塹營鐘離城,敕義之為戰守之備。是冬,英果率其安樂王元道明、平東將軍楊大眼等眾數十萬,來寇鐘離。鐘離城北阻淮水,魏人於邵陽洲西岸作浮橋,跨淮通道。英據東岸,大眼據西岸,以攻城。時城中眾纔三千人,義之督帥,隨方抗禦。魏軍乃以車載土填塹,使其眾負土隨之,嚴騎自後蹙焉。人有未及回者,因以土迮之,俄而塹滿。英與大眼躬自督戰,晝夜苦攻,分番相代,墜而復升,莫有退
 者。又設飛樓及衝車撞之,所值城土輒頹落。義之乃以泥補缺,衝車雖入而不能壞。義之善射,其被攻危急之處,輒馳往救之,每彎弓所向,莫不應弦而倒。一日戰數十合,前後殺傷者萬計,魏軍死者與城平。



 六年四月,高祖遣曹景宗、韋睿帥眾二十萬救焉,既至,與魏戰,大破之,英、大眼等各脫身奔走。義之因率輕兵追至洛口而還。斬首俘生,不可勝計。以功進號軍師將軍,增封二百戶,遷持節、督青、冀二州諸軍事、征虜將軍、青、冀二州刺史。未拜,改督南兗、兗、徐、青、冀五州諸軍事、輔國將軍、南兗州刺史。坐禁物出籓,為有司所奏免。其年,補朱衣直
 閣,除左驍騎將軍,直閣如故。遷太子右衛率,領越騎校尉,假節。八年,出為持節、督湘州諸軍事、征遠將軍、湘州刺史。九年,以本號還朝,俄為司空臨川王司馬,將軍如故。十年,遷右衛將軍。十三年,徙為左衛將軍。



 是冬,高祖遣太子右衛率康絢督眾軍作荊山堰。明年,魏遣將李曇定大眾逼荊山,揚聲欲決堰,詔假義之節,帥太僕卿魚弘文、直閣將軍曹世宗、徐元和等救絢,軍未至,絢等已破魏軍。魏又遣大將李平攻峽石,圍直閣將軍趙祖悅,義之又率朱衣直閣王神念等救之。時魏兵盛,神念攻峽石浮橋不能克,故援兵不得時進,遂陷峽石。義之
 班師,為有司所奏,高祖以其功臣,不問也。



 十五年,復以為使持節、都督湘州諸軍事、信威將軍、湘州刺史。其年,改授都督北徐州緣淮諸軍事、平北將軍、北徐州刺史。義之性寬厚,為將能撫御,得人死力,及居籓任,吏民安之。俄給鼓吹一部,改封營道縣侯,邑戶如先。普通三年,徵為護軍將軍,鼓吹如故。四年十月,卒。高祖深痛惜之,詔曰:「護軍將軍、營道縣開國侯昌義之,幹略沉濟,志懷寬隱,誠著運始,效彰邊服。方申爪牙,寄以禁旅;奄至殞喪,惻愴于懷。可贈散騎常侍、車騎將軍,并鼓吹一部。給東園秘器,朝服一具。賻錢二萬,布二百匹,蠟二百斤。謚
 曰烈。」子寶業嗣,官至直閣將軍、譙州刺史。



 陳吏部尚書姚察曰:張惠紹、馮道根、康絢、昌義之,初起從上,其功則輕。及群盜焚門,而惠紹以力戰顯;合肥、邵陽之逼,而道根、義之功多;浮山之役起,而康絢典其事:互有厥勞,寵進宜矣。先是鎮星守天江而堰興,及退舍而堰決,非徒人事,有天道矣。



\end{pinyinscope}