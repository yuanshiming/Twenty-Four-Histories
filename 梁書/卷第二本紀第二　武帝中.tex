\article{卷第二本紀第二 武帝中}

\begin{pinyinscope}

 天
 監元年夏四月丙寅,高祖即皇帝位於南郊。設壇柴燎,告類於天曰:「皇帝臣衍,敢用玄牡,昭告於皇天后帝:齊氏以歷運斯既,否終則亨,欽若天應,以命於衍。夫任是司牧,惟能是授;天命不于常,帝王非一族。唐謝虞受,漢替魏升,爰及晉、宋,憲章在昔。咸以君德馭四海,元功子萬姓,故能大庇氓黎,光宅區宇。齊代云季,世主昏凶,
 狡焉群慝,是崇是長,肆厥姦回暴亂,以播虐於我有邦,俾溥天惴惴,將墜于深壑。九服八荒之內,連率岳牧之君,蹶角頓顙,匡救無術,臥薪待然,援天靡訴。衍投袂星言,摧鋒萬里,厲其掛冠之情,用拯兆民之切。銜膽誓眾,覆銳屠堅,建立人主,克剪昏亂。遂因時來,宰司邦國,濟民康世,實有厥勞。而晷緯呈祥,川岳效祉,朝夕坰牧,日月郊畿。代終之符既顯,革運之期已萃,殊俗百蠻,重譯獻款,人神遠邇,罔不和會。於是群公卿士,咸致厥誠,並以皇乾降命,難以謙拒。齊帝脫屣萬邦,授以神器。衍自惟匪德,辭不獲許。仰迫上玄之眷,俯惟億兆之心,宸極
 不可久曠,民神不可乏主,遂藉樂推,膺此嘉祚。以茲寡薄,臨御萬方,顧求夙志,永言祗惕。敬簡元辰,恭茲大禮,升壇受禪,告類上帝,克播休祉,以弘盛烈,式傳厥後,用永保于我有梁。惟明靈是饗。」



 禮畢,備法駕即建康宮,臨太極前殿。詔曰:「五精遞襲,皇王所以受命;四海樂推,殷、周所以改物。雖禪代相舛,遭會異時,而微明迭用,其流遠矣。莫不振民育德,光被黎元。朕以寡闇,命不先後,寧濟之功,屬當期運,乘此時來,因心萬物,遂振厥弛維,大造區夏,永言前蹤,義均慚德。齊氏以代終有征,歷數云改,欽若前載,集大命于朕躬。顧惟菲德,辭不獲命,寅畏
 上靈,用膺景業。執禋柴之禮,當與能之祚,繼迹百王,君臨四海,若涉大川,罔知攸濟。洪基初兆,萬品權輿,思俾慶澤,覃被率土。可大赦天下。改齊中興二年為天監元年。賜民爵二級;文武加位二等;鰥寡孤獨不能自存者,人穀五斛。逋布、口錢、宿債勿復收。其犯鄉論清議,贓汙淫盜,一皆蕩滌,洗除前注,與之更始。」



 封齊帝為巴陵王,全食一郡。載天子旌旗,乘五時副車。行齊正朔。郊祀天地,禮樂制度,皆用齊典。齊宣德皇后為齊文帝妃,齊后王氏為巴陵王妃。



 詔曰:「興運升降,前代舊章。齊世王侯封爵,悉皆降省。其有效著艱難者,別有後命。惟宋汝陰
 王不在除例。」又詔曰:「大運肇升,嘉慶惟始,劫賊餘口沒在臺府者,悉可蠲放。諸流徙之家,並聽還本。」



 追尊皇考為文皇帝,廟曰太祖;皇妣為獻皇后。追謚妃郗氏為德皇后。追封兄太傅懿為長沙郡王,謚曰宣武;齊後軍諮議敷為永陽郡王,謚曰昭;弟齊太常暢為衡陽郡王,謚曰宣;齊給事黃門侍郎融為桂陽郡王,謚曰簡。



 是日,詔封文武功臣新除車騎將軍夏侯詳等十五人為公侯,食邑各有差。以弟中護軍宏為揚州刺史,封為臨川郡王;南徐州刺史秀安成郡王;雍州刺史偉建安郡王;左衛將軍恢鄱陽郡王;荊州刺史憺始興郡王。



 丁卯,加領
 軍將軍王茂鎮軍將軍。以中書監王亮為尚書令、中軍將軍,相國左長史王瑩為中書監、撫軍將軍,吏部尚書沈約為尚書僕射,長兼侍中范雲為散騎常侍、吏部尚書。



 詔曰:「宋氏以來,並恣淫侈,傾宮之富,遂盈數千。推算五都,愁窮四海,並嬰罹冤橫,拘逼不一。撫絃命管,良家不被蠲;織室繡房,幽厄猶見役。弊國傷和,莫斯為甚。凡後宮樂府,西解暴室,諸如此例,一皆放遣。若衰老不能自存,官給廩食。」



 戊辰,車騎將軍高句驪王高雲進號車騎大將軍。鎮東大將軍百濟王餘大進號征東大將軍。安西將軍宕昌王梁彌進號鎮西將軍。鎮東大將軍倭
 王武進號征東大將軍。鎮西將軍河南王吐谷渾休留代進號征西將軍。巴陵王薨於姑孰,追謚為齊和帝,終禮一依故事。



 己巳,以光祿大夫張瑰為右光祿大夫。庚午,鎮南將軍、江州刺史陳伯之進號征南將軍。



 詔曰:「觀風省俗,哲后弘規;狩岳巡方,明王盛軌。所以重華在上,五品聿修;文命肇基,四載斯履。故能物色幽微,耳目屠釣,致王業於緝熙,被淳風於遐邇。朕以寡薄,昧於治方,藉代終之運,當符命之重,取監前古,懍若馭朽。思所以振民育德,去殺勝殘,解網更張,置之仁壽;而明慚照遠,智不周物,兼以歲之不易,未遑卜征,興言夕惕,無忘鑒寐。
 可分遣內侍,周省四方,觀政聽謠,訪賢舉滯。其有田野不闢,獄訟無章,忘公殉私,侵漁是務者,悉隨事以聞。若懷寶迷邦,蘊奇待價,蓄響藏真,不求聞達,並依名騰奏,罔或遺隱。使輶軒所屆,如朕親覽焉。」



 又詔曰:「金作贖刑,有聞自昔,入縑以免,施於中世,民悅法行,莫尚乎此。永言叔世,偷薄成風,嬰愆入罪,厥塗匪一。斷弊之書,日纏於聽覽;鉗之刑,歲積於牢犴。死者不可復生,刑者無因自返,由此而望滋實,庸可致乎?朕夕惕思治,念崇政術,斟酌前王,擇其令典,有可以憲章邦國,罔不由之。釋愧心於四海,昭情素於萬物。俗偽日久,禁網彌繁。漢文
 四百,邈焉已遠。雖省事清心,無忘日用,而委銜廢策,事未獲從。可依周、漢舊典,有罪入贖,外詳為條格,以時奏聞。」



 辛未,以中領軍蔡道恭為司州刺史。以新除謝沐縣公蕭寶義為巴陵王,以奉齊祀。復南蘭陵武進縣,依前代之科。征謝朏為左光祿大夫、開府儀同三司,何胤為右光祿大夫。改南東海為蘭陵郡。土斷南徐州諸僑郡縣。



 癸酉,詔曰:「商俗甫移,遺風尚熾,下不上達,由來遠矣。升中馭索,增其懍然。可於公車府謗木肺石傍各置一函。若肉食莫言,山阿欲有橫議,投謗木函。若從我江、漢,功在可策,犀兕徒弊,龍蛇方縣;次身才高妙,擯壓莫通,
 懷傅、呂之術,抱屈、賈之歎,其理有皦然,受困包匭;夫大政侵小,豪門陵賤,四民已窮,九重莫達。若欲自申,並可投肺石函。」甲戍,詔斷遠近上慶禮。



 又詔曰:「禮闈文閣,宜率舊章,貴賤既位,各有差等,俯仰拜伏,以明王度,濟濟洋洋,具瞻斯在。頃因多難,治綱弛落,官非積及,榮由幸至。六軍尸四品之職,青紫治白簿之勞。振衣朝伍,長揖卿相,趨步廣闥,並驅丞郎。遂冠履倒錯,珪甑莫辨。靜言疚懷,思返流弊。且玩法惰官,動成逋弛,罰以常科,終未懲革。夫檟楚申威,蓋代斷趾,笞捶有令,如或可從。外詳共平議,務盡厥理。」



 癸未,詔「相國府職吏,可依資勞度臺;
 若職限已盈,所度之餘,及驃騎府並可賜滿。」



 閏月丁酉,以行宕昌王梁彌邕為安西將軍、河涼二州刺史,正封宕昌王。壬寅,以車騎將軍夏侯詳為右光祿大夫。



 詔曰:「成務弘風,肅厲內外,實由設官分職,互相懲糾。而頃壹拘常式,見失方奏,多容違惰,莫肯執咎,憲綱日弛,漸以為俗,今端右可以風聞奏事,依元熙舊制。」



 五月乙亥夜,盜人南、北掖,燒神虎門、總章觀,害衛尉卿張弘策。戊子,江州刺史陳伯之舉兵反,以領軍將軍王茂為征南將軍、江州刺史,率眾討之。六月庚戌,以行北秦州刺史楊紹先為北秦州刺史、武都王。是月,陳伯之奔魏,江州平。
 前益州刺史劉季連據成都反。八月戊戌,置建康三官。乙巳,平北將軍、西涼州刺史象舒彭進號安西將軍,封鄧至王。丁未,詔中書監王瑩等八人參定律令。是月,詔尚書曹郎依昔奏事。林邑、乾陁利國各遣使獻方物。冬十一月己未,立小廟。甲子,立皇子統為皇太子。十二月丙申,以國子祭酒張稷為護軍將軍。辛亥,護軍將軍張稷免。是歲大旱,米斗五千,人多餓死。



 二年春正月甲寅朔,詔曰:「三訊五聽,著自聖典,哀矜折獄,義重前誥,蓋所以明慎用刑,深戒疑枉,成功致治,罔不由茲。朕自籓部,常躬訊錄,求理得情,洪細必盡。末運
 弛網,斯政又闕,牢犴沉壅,申訴靡從。朕屬當期運,君臨兆億,雖復齋居宣室,留心聽斷;而九牧遐荒,無因臨覽。深懼懷冤就鞫,匪惟一方。可申敕諸州,月一臨訊,博詢擇善,務在確實。」乙卯,以尚書僕射沈約為尚書左僕射;吏部尚書范雲為尚書右僕射;前將軍鄱陽王恢為南徐州刺史;尚書令王亮為左光祿大夫;右衛將軍柳慶遠為中領軍。丙辰,尚書令、新除左光祿大夫王亮免。夏四月癸卯,尚書刪定郎蔡法度上《梁律》二十卷、《令》三十卷、《科》四十卷。五月丁巳,尚書右僕射范雲卒。乙丑,益州刺史鄧元起克成都,曲赦益州。壬申,斷諸郡縣獻奉二
 宮。惟諸州及會稽,職惟嶽牧,許薦任土,若非地產,亦不得貢。六月丁亥,詔以東陽、信安、豊安三縣水潦,漂損居民資業,遣使周履,量蠲課調。是夏多癆疫。以新除左光祿大夫謝朏為司徒、尚書令。甲午,以中書監王瑩為尚書右僕射。秋七月,扶南、龜茲、中天竺國各遣使獻方物。冬十月,魏寇司州。十一月乙卯,雷電大雨,晦。是夜又雷。乙亥,尚書左僕射沈約以母憂去職。



 三年春正月戊申,後將軍、揚州刺史臨川王宏進號中軍將軍。癸丑,以尚書右僕射王瑩為尚書左僕射,太子詹事柳惔為尚書右僕射,前尚書左僕射沈約為鎮軍
 將軍。二月,魏陷梁州。三月,隕霜殺草。五月丁巳,以扶南國王憍陳如闍耶跋摩為安南將軍。六月丙子,詔曰:「昔哲王之宰世也,每歲卜征,躬事巡省,民俗政刑,罔不必逮。末代風凋,久曠茲典。雖欲肆遠忘勞,究臨幽仄,而居今行古,事未易從,所以日晏踟躕,情同再撫。總總九州,遠近民庶,或川路幽遐,或貧羸老疾,懷冤抱理,莫由自申,所以東海匹婦,致災邦國,西土孤魂,登樓請訴。念此于懷,中夜太息。可分將命巡行州部。其有深冤鉅害,抑鬱無歸,聽詣使者,依源自列。庶以矜隱之念,昭被四方,棨聽遠聞,事均親覽。」癸未,大赦天下。秋七月丁未,以光
 祿大夫夏侯詳為車騎將軍、湘州刺史,湘州刺史楊公則為中護軍。甲子,立皇子綜為豫章郡王。八月,魏陷司州,詔以南義陽置司州。九月壬子,以河南王世子伏連籌為鎮西將軍、西秦河二州刺史、河南王。北天竺國遣使獻方物。冬十一月甲子,詔曰:「設教因時,淳薄異政,刑以世革,輕重殊風。昔商俗未移,民散久矣,嬰網陷辟,日夜相尋。若悉加正法,則赭衣塞路;並申弘宥,則難用為國,故使有罪入贖,以全元元之命。今遐邇知禁,圄犴稍虛,率斯以往,庶幾刑措。金作權典,宜在蠲息。可除贖罪之科。」是歲多疾疫。



 四年春正月癸卯朔,詔曰:「今九流常選,年未三十,不通一經,不得解褐。若有才同甘、顏,勿限年次。」置《五經》博士各一人。以鎮北將軍、雍州刺史、建安王偉為南徐州刺史,南徐州刺史鄱陽王恢為郢州刺史,中領軍柳慶遠為雍州刺史。丙午,省《鳳皇銜書伎》。戊申,詔曰:「夫禋郊饗帝,至敬攸在,致誠盡愨,猶懼有違;而往代多令宮人縱觀茲禮,帷宮廣設,輜軿耀路,非所以仰虔蒼昊,昭感上靈。屬車之間,見譏前世,便可自今停止。」辛亥,輿駕親祠南郊,赦天下。二月壬午,遣衛尉卿楊公則率宿衛兵塞洛口。壬辰,交州刺史李凱據州反,長史李畟討平之。曲
 赦交州。戊戌,以前郢州刺史曹景宗為中護軍。是月,立建興苑於秣陵建興里。夏四月丁巳,以行宕昌王梁彌博為安西將軍、河涼二州刺史、宕昌王。是月,自甲寅至壬戌,甘露連降華林園。五月辛卯,建康縣朔陰里生嘉禾,一莖十二穗。六月庚戌,立孔子廟。壬戌,歲星晝見。秋七月辛卯,右光祿大夫張瑰卒。八月庚子,老人星見。冬十月丙午,北伐,以中軍將軍、揚州刺史臨川王宏都督北討諸軍事,尚書右僕射柳惔為副。是歲,以興師費用,王公以下各上國租及田穀,以助軍資。十一月辛未,以都官尚書張稷為領軍將軍。甲午,天晴朗,西南有電光,
 聞如雷聲三。十二月,司徒、尚書令謝朏以所生母憂,去職。是歲大穰,米斛三十。



 五年春正月丁卯朔,詔曰:「在昔周、漢,取士方國。頃代凋訛,幽仄罕被,人孤地絕,用隔聽覽,士操淪胥,因茲靡勸。豈其岳瀆縱靈,偏有厚薄,實由知與不知,用與不用耳。朕以菲德,君此兆民,而兼明廣照,屈於堂戶,飛耳長目,不及四方,永言愧懷,無忘旦夕。凡諸郡國舊族,邦內無在朝位者,選官搜括,使郡有一人。」乙亥,以前司徒謝朏為中書監、司徒、衛將軍,鎮軍將軍沈約為右光祿大夫,豫章王綜為南徐州刺史。丁丑,以尚書左僕射王瑩為
 護軍將軍,僕射如故。甲申,立皇子綱為晉安郡王。丁亥,太白晝見。二月庚戌,以太常張充為吏部尚書。三月丙寅朔,日有蝕之。癸未,魏宣武帝從弟翼率其諸弟來降。輔國將軍劉思效破魏青州刺史元繫於膠水。丁亥,陳伯之自壽陽率眾歸降。夏四月丙申,廬陵高昌之仁山獲銅劍二,始豊縣獲八目龜一。甲寅,詔曰:「朕昧旦齋居,惟刑是恤,三辟五聽,寢興載懷。故陳肺石於都街,增官司於詔獄,殷勤親覽,小大以情。而明慎未洽,囹圄尚壅,永言納隍,在予興愧。凡犴獄之所,可遣法官近侍,遞錄囚徒,如有枉滯,以時奏聞。」五月辛未,太子左衛率張惠
 紹克魏宿預城。乙亥,臨川王宏前軍克梁城。辛巳,豫州刺史韋睿克合肥城。丁亥,廬江太守裴邃克羊石城;庚寅,又克霍丘城。辛卯,太白晝見。六月庚子,青、冀二州刺史桓和前軍克朐山城。秋七月乙丑,鄧至國遣使獻方物。八月戊戌,老人星見。辛酉,作太子宮。冬十一月甲子,京師地震。乙丑,以師出淹時,大赦天下。魏寇鐘離,遣右衛將軍曹景宗率眾赴援。十二月癸卯,司徒謝朏薨。



 六年春正月辛酉朔,詔曰:「徑寸之寶,或隱沙泥;以人廢言,君子斯戒。朕聽朝晏罷,思闡政術,雖百辟卿士,有懷必聞,而蓄響邊遐,未臻魏闕。或屈以貧陋,或間以山川,
 頓足延首,無因奏達。豈所以沉浮靡漏,遠邇兼得者乎?四方士民,若有欲陳言刑政,益國利民,淪礙幽遠,不能自通者,可各詮條布懷於刺史二千石。有可申採,大小以聞。」己卯,詔曰:「夫有天下者,義非為己。凶荒疾癆,兵革水火,有一於此,責歸元首。今祝史請禱,繼諸不善,以朕身當之。永使災害不及萬姓,俾茲下民稍蒙寧息。不得為朕祈福,以增其過。特班遠邇,咸令遵奉。」二月甲辰,老人星見。三月庚申朔,隕霜殺草。是月,有三象入京師。夏四月壬辰,置左右驍騎、左右游擊將軍官。癸巳,曹景宗、韋睿等破魏軍於邵陽洲,斬獲萬計。癸卯,以右衛將軍
 曹景宗為領軍將軍、徐州刺史。己酉,以江州刺史王茂為尚書右僕射,中書令安成王秀為平南將軍、江州刺史。分湘廣二州置衡州。丁巳,以中軍將軍、揚州刺史臨川王宏為驃騎將軍、開府儀同三司,撫軍將軍建安王偉為揚州刺史,右光祿大夫沈約為尚書左僕射,尚書左僕射王瑩為中軍將軍。五月己未,以新除左驍騎將軍長沙王深業為中護軍。癸亥,以侍中袁昂為吏部尚書。己巳,置中衛、中權將軍,改驍騎為雲騎,游擊為游騎。辛未,右將軍、揚州刺史建安王偉進號中權將軍。六月庚戌,以車騎將軍、湘州刺史夏侯詳為右光祿大夫,新
 除金紫光祿大夫柳惔為安南將軍、湘州刺史。新吳縣獲四目龜一。秋七月甲子,太白晝見。丙寅,分廣州置桂州。丁亥,以新除尚書右僕射王茂為中衛將軍。八月戊子,赦天下。戊戌,大風折木。京師大水,因濤入,加御道七尺。九月,嘉禾一莖九穗,生江陵縣。乙亥,改閱武堂為德陽堂,聽訟堂為儀賢堂。丙戌,以左衛將軍呂僧珍為平北將軍、南兗州刺史,豫章內史蕭昌為廣州刺史。冬十月壬寅,以五兵尚書徐勉為吏部尚書。閏月乙丑,以驃騎將軍、開府儀同三司臨川王宏為司徒、行太子太傅,尚書左僕射沈約為尚書令、行太子少傅,吏部尚書袁
 昂為右僕射。戊寅,平西將軍、荊州刺史始興王心詹進號安西將軍。甲申,以右光祿大夫夏侯詳為尚書左僕射。十二月丙辰,尚書左僕射夏侯詳卒。乙丑,魏淮陽鎮都軍主常邕和以城內屬。分豫州置霍州。



 七年春正月乙酉朔,詔曰:「建國君民,立教為首。不學將落,嘉植靡由。朕肇基明命,光宅區宇,雖耕耘雅業,傍闡藝文,而成器未廣,志本猶闕,非所以熔範貴遊,納諸軌度。思欲式敦讓齒,自家刑國。今聲訓所漸,戎夏同風,宜大啟庠斅,博延胄子,務彼十倫,弘此三德,使陶鈞遠被,微言載表。」中衛將軍、領太子詹事王茂進號車騎將軍。
 戊戌,作神龍、仁虎闕於端門、大司馬門外。壬子,以領軍將軍曹景宗為中衛將軍,衛尉蕭景兼領軍將軍。二月乙卯,廬江灊縣獲銅鐘二。新作國門于越城南。乙丑,增置鎮衛將軍以下各有差。庚午,詔於州郡縣置州望、郡宗、鄉豪各一人,專掌搜薦。乙亥,以車騎大將軍高麗王高雲為撫東大將軍、開府儀同三司,平北將軍、南兗州刺史呂僧珍為領軍將軍。丙子,以中護軍長沙王深業為南兗州刺史,兼領軍將軍蕭景為雍州刺史,雍州刺史柳慶遠為護軍將軍。夏四月乙卯,皇太子納妃,赦大辟以下,頒賜朝臣及近侍各有差。辛未,秣陵縣獲靈龜
 一。戊寅,餘姚縣獲古銅劍二。五月己亥,詔復置宗正、太僕、大匠、鴻臚,又增太府、太舟,仍先為十二卿。癸卯,以平南將軍、江州刺史安成王秀為平西將軍、荊州刺史,安西將軍、荊州刺史始興王憺為護軍將軍,中衛將軍曹景宗為安南將軍、江州刺史。六月辛酉,復建、修二陵周回五里內居民,改陵監為令。秋七月丁亥,月犯氐。八月癸丑,安南將軍、江州刺史曹景宗卒。丁巳,赦大辟以下未結正者。甲戌,平西將軍、荊州刺史安成王秀進號安西將軍,雲麾將軍、郢州刺史鄱陽王恢進號平西將軍。老人星見。九月丁亥,詔曰:「芻牧必往,姬文垂則,雉兔有
 刑,姜宣致貶。藪澤山林,毓材是出,斧斤之用,比屋所資。而頃世相承,並加封固,豈所謂與民同利,惠茲黔首?凡公家諸屯戍見封熂者,可悉開常禁。」壬辰,置童子奉車郎。癸巳,立皇子績為南康郡王。己亥,月犯東井。冬十月丙寅,以吳興太守張稷為尚書左僕射。丙子,魏陽關主許敬珍以城內附。詔大舉北伐。以護軍將軍始興王憺為平北將軍,率眾入清;車騎將軍王茂率眾向宿預。丁丑,魏懸瓠鎮軍主白皁生、豫州刺史胡遜以城內屬。以皁生為鎮北將軍、司州刺史,遜為平北將軍、豫州刺史。十一月辛巳,鄞縣言甘露降。



 八年春正月辛巳,輿駕親祠南郊,赦天下,內外文武各賜勞一年。壬辰,魏鎮東參軍成景俊斬宿預城主嚴仲寶,以城內屬。二月壬戌,老人星見。夏四月,以北巴西郡置南梁州。戊申,以護軍將軍始興王憺為中衛將軍,司徒、行太子太傅臨川王宏為司空、揚州刺史,車騎將軍、領太子詹事王茂即本號開府儀同三司。丁卯,魏楚王城主李國興以城內附。丙子,以中軍將軍、丹陽尹王瑩為右光祿大夫。五月壬午,詔曰:「學以從政,殷勤往哲,祿在其中,抑亦前事。朕思闡治綱,每敦儒術,軾閭闢館,造次以之。故負帙成風,甲科間出,方當置諸周行,飾以青
 紫。其有能通一經,始末無倦者,策實之後,選可量加敘錄。雖復牛監羊肆,寒品後門,並隨才試吏,勿有遺隔。」秋七月癸巳,巴陵王蕭寶義薨。八月戊午,老人星見。冬十月乙巳,以中軍將軍始興王慎為鎮北將軍、南兗州刺史,南兗州刺史長沙王深業為護軍將軍。



 九年春正月乙亥,以尚書令、行太子少傅沈約為左光祿大夫,行少傅如故,右光祿大夫王瑩為尚書令,行中撫將軍建安王偉領護軍將軍,鎮北將軍、南兗州刺史始興王憺為鎮西將軍、益州刺史,太常卿王亮為中書監。丙子,以輕車將軍晉安王綱為南兗州刺史。庚寅,新
 作緣淮塘,北岸起石頭迄東冶,南岸起後渚籬門迄三橋。三月己丑,車駕幸國子學,親臨講肆,賜國子祭酒以下帛各有差。乙未,詔曰:「王子從學,著自禮經,貴遊咸在,實惟前誥,所以式廣義方,克隆教道。今成均大啟,元良齒讓,自斯以降,並宜肄業。皇太子及王侯之子,年在從師者,可令入學。」于闐國遣使獻方物。夏四月丁巳,革選尚書五都令史用寒流。林邑國遣使獻白猴一。五月己亥,詔曰:「朕達聽思治,無忘日昃。而百司群務,其途不一,隨時適用,各有攸宜,若非總會眾言,無以備茲親覽。自今臺閣省府州郡鎮戍應有職僚之所,時共集議,各陳
 損益,具以奏聞。」中書監王亮卒。六月癸丑,盜殺宣城太守朱僧勇。癸酉,以中撫將軍、領護軍建安王偉為鎮南將軍、江州刺史。閏月己丑,宣城盜轉寇吳興縣,太守蔡撙討平之。秋七月己巳,老人星見。冬十二月癸未,輿駕幸國子學,策試胄子,賜訓授之司各有差。



 十年春正月辛丑,輿駕親祠南郊,大赦天下,居局治事賜勞二年。癸卯,以尚書左僕射張稷為安北將軍,青冀二州刺史,郢州刺史鄱陽王恢為護軍將軍。甲辰,以南徐州刺史豫章王綜為郢州刺史,輕車將軍南康王績為南徐州刺史。戊申,騶虞一,見荊州華容縣。以左民尚
 書王暕為吏部尚書。辛酉,輿駕親祠明堂。三月辛丑,盜殺東莞、瑯邪二郡太守鄧晣,以朐山引魏軍,遣振遠將軍馬仙琕討之。是月,魏徐州刺史盧昶帥眾赴朐山。夏五月癸酉,安豊縣獲一角玄龜。丁丑,領軍呂僧珍卒。己卯,以國子祭酒張充為尚書左僕射,太子詹事柳慶遠為領軍將軍。六月乙酉,嘉蓮一莖三花生樂遊苑。秋七月丙辰,詔曰:「昔公卿面陳,載在前史,令僕陛奏,列代明文,所以釐彼庶績,成茲群務。晉氏陵替,虛誕為風,自此相因,其失彌遠。遂使武帳空勞,無汲公之奏,丹墀徒闢,闕鄭生之履。三槐八座,應有務之百官,宜有所論,可入陳啟,庶
 藉周爰,少匡寡薄。」九月丙申,天西北隆隆有聲,赤氣下至地。冬十二月癸酉,山車見于臨城縣。庚辰,馬仙琕大破魏軍,斬馘十餘萬,剋復朐山城。是歲,初作宮城門三重樓及開二道。宕昌國遣使獻方物。



 十一年春正月壬辰,詔曰:「夫刑法悼耄,罪不收孥,禮著明文,史彰前事,蓋所以申其哀矜,故罰有弗及。近代相因,厥網彌峻,髫年華髮,同坐入愆。雖懲惡勸善,宜窮其制,而老幼流離,良亦可愍。自今逋謫之家及罪應質作,若年有老小,可停將送。」加左光祿大夫、行太子少傅沈約特進,鎮南將軍、江州刺史建安王偉儀同三司,司空、
 揚州刺史臨川王宏進位為太尉,驃騎將軍王茂為司空,尚書令、雲麾將軍王瑩進號安左將軍,安北將軍、青冀二州刺史張稷進號鎮北將軍。二月戊辰,新昌、濟陽二郡野蠶成繭。三月丁巳,曲赦揚、徐二州。築西靜壇於鐘山。庚申,高麗國遣使獻方物。四月戊子,詔曰:「去歲朐山大殲醜類,宜為京觀,用旌武功;但伐罪弔民,皇王盛軌,掩骼埋胔,仁者用心。其下青州悉使收藏。」百濟、扶南、林邑國並遣使獻方物。六月辛巳,以司空王茂領中權將軍。九月辛亥,宕昌國遣使獻方物。冬十一月乙未,以吳郡太守袁昂兼尚書右僕射。己酉,降太尉、揚州刺史
 臨川王宏為驃騎將軍、開府同三司之儀。癸丑,齊宣德太妃王氏薨。十二月己未,以安西將軍、荊州刺史安成王秀為中衛將軍,護軍將軍鄱陽王恢為平西將軍、荊州刺史。



 十二年春正月辛卯,輿駕親祠南郊,赦大辟以下。二月辛酉,以兼尚書右僕射袁昂為尚書右僕射。丙寅,詔曰:「掩骼埋胔,義重周經,槥櫝有加,事美漢策。朕向隅載懷,每勤造次,收藏之命,亟下哀矜;而珝縣遐深,遵奉未洽,髐然路隅,往往而有,言愍沉枯,彌勞傷惻。可明下遠近,各巡境界,若委骸不葬,或蒢衣莫改,即就收斂,量給棺
 具。庶夜哭之魂斯慰,霑霜之骨有歸。」辛巳,新作太極殿,改為十三間。三月癸卯,以湘州刺史王珍國為護軍將軍。閏月乙丑,特進、中軍將軍沈約卒。夏四月,京邑大水。六月癸巳,新作太廟,增基九尺。庚子,太極殿成。秋九月戊午,以鎮南將軍、開府儀同三司、江州刺史建安王偉為撫軍將軍,儀同如故;驃騎將軍、開府同三司之儀、揚州刺史臨川王宏為司空;領中權將軍王茂為驃騎將軍、開府同三司之儀、江州刺史。冬十月丁亥,詔曰:「明堂地勢卑濕,未稱乃心。外可量就埤起,以盡誠敬。」



 十三年春正月壬戌,以丹陽尹晉安王綱為荊州刺史。
 癸亥,以平西將軍、荊州刺史鄱陽王恢為鎮西將軍、益州刺史。丙寅,以翊右將軍安成王秀為安西將軍、郢州刺史。二月丁亥,輿駕親耕籍田,赦天下,孝悌力田賜爵一級。老人星見。三月辛亥,以新除中撫將軍、開府儀同三司建安王偉為左光祿大夫。夏四月辛卯,林邑國遣使獻方物。壬辰,以郢州刺史豫章王綜為安右將軍。五月辛亥,以通直散騎常侍韋睿為中護軍。六月己亥,以南兗州刺史蕭景為領軍將軍,領軍將軍柳慶遠為安北將軍、雍州刺史。秋七月乙亥,立皇子綸為邵陵郡王,繹為湘東郡王,紀為武陵郡王。八月癸卯,扶南、于闐國
 各遣使獻方物。是歲作浮山堰。



 十四年春正月乙巳朔,皇太子冠,赦天下,賜為父後者爵一級,王公以下班賚各有差,停遠近上慶禮。丙午,安左將軍、尚書令王瑩進號中權將軍。以鎮西將軍始興王憺為中撫將軍。辛亥,輿駕親祠南郊。詔曰:「朕恭祗明祀,昭事上靈,臨竹宮而登泰壇,服裘冕而奉蒼璧,柴望既升,誠敬克展,思所以對越乾元,弘宣德教;而缺于治道,政法多昧,實佇群才,用康庶績。可班下遠近,博採英異。若有確然鄉黨,獨行州閭,肥遁丘園,不求聞達,藏器待時,未加收採;或賢良、方正,孝悌、力田,並即騰奏,具以
 名上。當擢彼周行,試以邦邑,庶百司咸事,兆民無隱。又世輕世重,隨時約法,前以劓墨,用代重辟,猶念改悔,其路已壅,並可省除。」丙寅,汝陰王劉胤薨。二月庚寅,芮芮國遣使獻方物。戊戌,老人星見。辛丑,以中護軍韋睿為平北將軍、雍州刺史,新除中撫將軍始興王憺為荊州刺史。夏四月丁丑,驃騎將軍、開府同三司之儀、江州刺史王茂薨。五月丁巳,以荊州刺史晉安王綱為江州刺史。秋八月乙未,老人星見。九月癸亥,以長沙王深業為護軍將軍。狼牙脩國遣使獻方物。



 十五年春正月己巳,詔曰:「觀時設教,王政所先,兼而利
 之,實惟務本,移風致治,咸由此作。頃因革之令,隨事必下,而張弛之要,未臻厥宜,民瘼猶繁,廉平尚寡,所以佇旒纊而載懷,朝玉帛而興歎。可申下四方,政有不便於民者,所在具條以聞。守宰若清潔可稱,或侵漁為蠹,分別奏上,將行黜陟。長吏勸課,躬履堤防,勿有不脩,致妨農事。關市之賦,或有未允,外時參量,優減舊格。」三月戊辰朔,日有蝕之。夏四月丁未,以安右將軍豫章王綜兼護軍。高麗國遣使獻方物。五月癸未,以司空、揚州刺史臨川王宏為中書監,驃騎大將軍、刺史如故。六月丙申,改作小廟畢。庚子,以尚書令王瑩為左光祿大夫、開府
 儀同三司,尚書右僕射袁昂為尚書左僕射,吏部尚書王暕為尚書右僕射。秋八月,老人星見。芮芮、河南遣使獻方物。九月辛巳,左光祿大夫、開府儀同三司王瑩薨。壬辰,赦天下。冬十月戊午,以丹陽尹長沙王深業為湘州刺史。十一月丁卯,以兼護軍豫章王綜為安前將軍。交州刺史李畟斬交州反者阮宗孝,傳首京師。曲赦交州。壬午,以雍州刺史韋睿為護軍將軍。



 十六年春正月辛未,輿駕親祠南郊,詔曰:「朕當扆思治,政道未明,昧旦劬勞,亟移星紀。今太皞御氣,句芒首節,升中就陽,禋敬克展,務承天休,布茲和澤。尤貧之家,勿
 收今年三調。其無田業者,所在量宜賦給。若民有產子,即依格優蠲。孤老鰥寡不能自存,咸加賑恤。班下四方。諸州郡縣,時理獄訟,勿使冤滯,並若親覽。」二月庚戌,老人星見,甲寅,以安前將軍豫章王綜為南徐州刺史。三月丙子,河南王遣使獻方物。夏四月甲子,初去宗廟牲。潮溝獲白雀一。六月戊申,以廬陵王績為江州刺史。七月丁丑,以郢州刺史安成王秀為鎮北將軍、雍州刺史。八月辛丑,老人星見。扶南、婆利國各遣使獻方物。冬十月,去宗廟薦脩,始用蔬果。



 十七年春正月丁巳朔,詔曰:「夫樂所自生,含識之常性;
 厚下安宅,馭世之通規。朕矜此庶氓,無忘待旦,亟弘生聚之略,每布寬恤之恩;而編戶未滋,遷徙尚有,輕去故鄉,豈其本志?資業殆闕,自返莫由,巢南之心,亦何能弭。今開元發歲,品物惟新,思俾黔黎,各安舊所。將使郡無曠土,邑靡游民,雞犬相聞,桑柘交畛。凡天下之民,有流移他境,在天監十七年正月一日以前,可開恩半歲,悉聽還本,蠲課三年。其流寓過遠者,量加程日。若有不樂還者,即使著土籍為民,准舊課輸。若流移之後,本鄉無復居宅者,村司三老及餘親屬,即為詣縣,占請村內官地官宅,令相容受,使戀本者還有所託。凡坐為市埭諸
 職,割盜衰滅,應被封籍者,其田宅車牛,是民生之具,不得悉以沒入,皆優量分留,使得自止。其商賈富室,亦不得頓相兼併。遁叛之身,罪無輕重,並許首出,還復民伍。若有拘限,自還本役。並為條格,咸使知聞。」二月癸巳,鎮北將軍、雍州刺史安成王秀薨。甲辰,大赦天下。乙卯,以領石頭戍事南康王績為南兗州刺史。三月甲申,老人星見。丙申,改封建安王偉為南平王。夏五月戊寅,驃騎大將軍、揚州刺史臨川王宏免。己卯、乾利國遣使獻方物。以領軍將軍蕭景為安右將軍,監揚州。辛巳,以臨川王宏為中軍將軍、中書監。六月乙酉,以益州刺史鄱
 陽王恢為領軍將軍。中軍將軍,中書監臨川王宏以本號行司徒。癸卯,以國子祭酒蔡撙為吏部尚書。秋八月壬寅,老人星見。詔以兵騶奴婢,男年登六十,女年登五十,免為平民。冬十月乙亥,以中軍將軍、行司徒臨川王宏為中書監、司徒。十一月辛亥,以南平王偉為左光祿大夫、開府儀同三司。



 十八年春正月甲申,以領軍將軍鄱陽王恢為征西將軍、開府儀同三司、荊州刺史,荊州刺史始興王憺為中撫將軍、開府儀同三司、領軍。以尚書左僕射袁昂為尚書令,尚書右僕射王暕為尚書左僕射,太子詹事徐勉
 為尚書右僕射。辛卯,輿駕親祠南郊,孝悌力田賜爵一級。二月戊午,老人星見。四月丁巳,大赦天下。秋七月甲申,老人星見。於闐、扶南國各遣使獻方物。



\end{pinyinscope}