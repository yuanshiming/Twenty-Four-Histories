\article{卷第一十列傳第四 蕭穎達 夏侯詳 蔡道恭 楊公則 鄧元起}

\begin{pinyinscope}

 蕭穎達,蘭陵蘭陵人,齊光祿大夫赤斧第五子也。少好勇使氣,起家冠軍。兄穎胄,齊建武末行荊州事,穎達亦為西中郎外兵參軍,俱在西府。齊季多難,頗不自安。會東昏遣輔國將軍劉山陽為巴西太守,道過荊州,密敕穎胄襲雍州。時高祖已為備矣。仍遣穎胄親人王天虎以書疑之。山陽
 至,果不敢入城。穎胄計無所出,夜遣錢塘人朱景思呼西中郎城局參軍席闡文、諮議參軍柳忱閉齋定議。闡文曰:「蕭雍州蓄養士馬,非復一日,江陵素畏襄陽人,人眾又不敵,取之必不可制,制之,歲寒復不為朝廷所容。今若殺山陽,與雍州舉事,立天子以令諸侯,則霸業成矣。山陽持疑不進,是不信我。今斬送天虎,則彼疑可釋。至而圖之,罔不濟矣。」忱亦勸焉。穎達曰:「善。」及天明,穎胄謂天虎曰:「卿與劉輔國相識,今不得不借卿頭。」乃斬天虎以示山陽。山陽大喜,輕將步騎數百到州。闡文勒兵待於門,山陽車踰限而門闔,因執斬之,傳首高祖。且以
 奉南康王之議來告,高祖許焉。



 和帝即位,以穎胄為假節、侍中、尚書令、領吏部尚書、都督行留諸軍事、鎮軍將軍、荊州刺史,留衛西朝。以穎達為冠軍將軍。及楊公則等率師隨高祖,高祖圍郢城,穎達會軍於漢口,與王茂、曹景宗等攻郢城,陷之。隨高祖平江州。高祖進江州,使與曹景宗先率馬步進趨江寧,破東昏將李居士,又下東城。



 初,義師之起也,巴東太守蕭惠訓子璝、巴西太守魯休烈弗從,舉兵侵荊州,敗輔國將軍任漾之於硤口,破大將軍劉孝慶於上明,穎胄遣軍拒之;而高祖已平江、郢,圖建康。穎胄自以職居上將,不能拒制璝等,憂愧
 不樂,發疾數日而卒。州中秘之,使似其書者假為教命。及璝等聞建康將平,眾懼而潰,乃始發喪,和帝贈穎胄丞相。



 義師初,穎達弟穎孚自京師出亡,廬陵人循景智潛引與南歸,至廬陵,景智及宗人靈祐為起兵,得數百人,屯西昌藥山湖。穎達聞之,假穎孚節、督廬陵豫章臨川南康安成五郡軍事、冠軍將軍、廬陵內史。穎孚率靈祐等進據西昌,東昏遣安西太守劉希祖自南江入湖拒之。穎孚不能自立,以其兵由建安復奔長沙,希祖追之,穎孚緣山踰嶂,僅而獲免。在道絕糧,後因食過飽而卒。



 建康城平,高祖以穎達為前將軍、丹陽尹。上受禪,詔
 曰:「念功惟德,列代所同,追遠懷人,彌與事篤。齊故侍中、丞相、尚書令穎胄,風格峻遠,器珝深邵,清猷盛業,問望斯歸。締構義始,肇基王迹,契闊屯夷,載形心事。朕膺天改物,光宅區宇,望岱觀河,永言號慟。可封巴東郡開國公,食邑三千戶,本官如故。」贈穎孚右衛將軍。加穎達散騎常侍,以公事免。及大論功賞,封穎達吳昌縣侯,邑千五百戶。尋為侍中,改封作唐侯,縣邑如故。遷征虜將軍、太子左衛率。御史中丞任昉奏曰:臣聞貧觀所取,窮視不為。在於布衣窮居,介然之行,尚可以激貪歷俗,惇此薄夫;況乎伐冰之家,爭雞豚之利;衣繡之士,受賈人之
 服。風聞征虜將軍臣蕭穎達啟乞魚軍稅,輒攝穎達宅督彭難當到臺辨問。列稱『尋生魚典稅,先本是鄧僧琰啟乞,限訖今年五月十四日。主人穎達,于時謂非新立,仍啟乞接代僧琰,即蒙降許登稅,與史法論一年收直五十萬。』如其列狀,則與風聞符同,穎達即主。



 臣謹案:征虜將軍、太子左衛率、作唐縣開國侯臣穎達,備位大臣,預聞執憲,私謁亟陳,至公寂寞。屠中之志,異乎鮑肆之求;魚飧之資,不俟潛有之數。遂復申茲文二,追彼十一,風體若茲,準繩斯在!陛下弘惜勳良,每為曲法;臣當官執憲,敢不直繩。臣等參議,請以見事免穎達所居官,以
 侯還第。



 有詔原之。轉散騎常侍、左衛將軍。俄復為侍中,衛尉卿。出為信威將軍、豫章內史,加秩中二千石。治任威猛,郡人畏之。遷使持節、都督江州諸軍事、江州刺史,將軍如故。頃之,徵為通直散騎常侍、右驍騎將軍。既處優閑,尤恣聲色,飲酒過度,頗以此傷生。



 九年,遷信威將軍、右衛將軍。是歲卒,年三十四。車駕臨哭,給東園秘器,朝服一具,衣一襲,錢二十萬,布二百匹。追贈侍中、中衛將軍,鼓吹一部。謚曰康。子敏嗣。



 穎胄子靡,襲巴東公,位至中書郎,早卒。



 夏侯詳,字叔業,譙郡人也。年十六,遭父艱,居喪哀毀。三
 年廬于墓,嘗有雀三足,飛來集其廬戶,眾咸異焉。服闋,刺史殷琰召補主簿。宋泰始初,琰舉豫州叛,宋明帝遣輔國將軍劉勔討之,攻守連月,人情危懼,將請救於魏。詳說琰曰:「今日之舉,本效忠節;若社稷有奉,便歸身朝廷,何可屈身北面異域。且今魏氏之卒,近在淮次,一軍未測去就,懼有異圖。今若遣使歸款,必厚相慰納,豈止免罪而已。若謂不然,請充一介。」琰許之。詳見勔曰:「將軍嚴圍峭壘,矢刃如霜,城內愚徒,實同困獸,士庶懼誅,咸欲投魏。僕所以踰城歸德,敢布腹心。願將軍弘曠蕩之恩,垂霈然之惠,解圍退舍,則皆相率而至矣。」勔許之。詳
 曰:「審爾,當如君言,而詳請反命。」勔遣到城下,詳呼城中人,語以勔辭,即日琰及眾俱出,一州以全。勔為刺史,又補主簿。頃之,為新汲令,治有異績,刺史段佛榮班下境內,為屬城表。轉治中從事史,仍遷別駕。歷事八將,州部稱之。



 齊明帝為刺史,雅相器遇。及輔政,招令出都,將大用之。每引詳及鄉人裴叔業日夜與語,詳輒末略不酬。帝以問叔業,叔業告詳。詳曰:「不為福始,不為禍先。」由此微有忤。出為征虜長史、義陽太守。頃之、建安戍為魏所圍,仍以詳為建安戍主,帶邊城、新蔡二郡太守,并督光城、弋陽、汝陰三郡眾赴之。詳至建安,魏軍引退。先是,魏
 又於淮上置荊亭戍,常為寇掠,累攻不能禦,詳率銳卒攻之,賊眾大潰,皆棄城奔走。



 建武末,徵為游擊將軍,出為南中郎司馬、南新蔡太守。齊南康王為荊州,遷西中郎司馬、新興太守,便道先到江陽。時始安王遙光稱兵京邑,南康王長史蕭穎胄並未至,中兵參軍劉山陽先在州,山陽副潘紹欲謀作亂,詳偽呼紹議事,即於城門斬之,州府乃安。遷司州刺史,辭不之職。



 高祖義兵起,詳與穎胄同創大舉。西臺建,以詳為中領軍,加散騎常侍、南郡太守。凡軍國大事,穎胄多決於詳。及高祖圍郢城未下,穎胄遣衛尉席闡文如高祖軍。詳獻議曰:「窮壁易
 守,攻取勢難;頓甲堅城,兵家所忌。誠宜大弘經略,詢納群言。軍主以下至於匹夫,皆令獻其所見,盡其所懷,擇善而從,選能而用,不以人廢言,不以多罔寡。又須量我眾力,度賊樵糧,窺彼人情,權其形勢。若使賊人眾而食少,故宜計日而守之;食多而力寡,故宜悉眾而攻之。若使糧力俱足,非攻守所屈,便宜散金寶,縱反間,使彼智者不用,愚者懷猜,此魏武之所以定大業也。若三事未可,宜思變通,觀於人情,計我糧穀。若德之所感,萬里同符,仁之所懷,遠邇歸義,金帛素積,糧運又充,乃可以列圍寬守,引以歲月,此王剪之所以剋楚也。若圍之不卒
 降,攻之未可下,間道不能行,金粟無人積,天下非一家,人情難可豫,此則宜更思變計矣。變計之道,實資英斷,此之深要,難以紙宣,輒布言於席衛尉,特願垂採。」高祖嘉納焉。頃之,穎胄卒。時高祖弟始興王憺留守襄陽,詳乃遣使迎憺,共參軍國。和帝加詳禁兵,出入殿省,固辭不受。遷侍中、尚書右僕射。尋授使持節、撫軍將軍、荊州刺史。詳又固讓于憺。



 天監元年,徵為侍中、車騎將軍,論功封寧都縣侯,邑二千戶。詳累辭讓,至於懇切,乃更授右光祿大夫,侍中如故。給親信二十人,改封豊城縣公,邑如故。二年,抗表致仕,詔解侍中,進特進。三年,遷使持
 節、散騎常侍、車騎將軍、湘州刺史。詳善吏事,在州四載,為百姓所稱。州城南臨水有峻峰,舊老相傳,云「刺史登此山輒被代。」因是歷政莫敢至。詳於其地起臺榭,延僚屬,以表損挹之志。



 六年,徵為侍中、右光祿大夫,給親信二十人,未至,授尚書左僕射、金紫光祿大夫,侍中如故。道病卒,時年七十四,上為素服舉哀,贈右光祿。



 先是,荊府城局參軍吉士瞻役萬人浚仗庫防火池,得金革帶鉤,隱起雕鏤甚精巧,篆文曰「錫爾金鉤,既公且侯」。士瞻,詳兄女婿也。女竊以與詳,詳喜佩之,期歲而貴矣。



 蔡道恭,字懷儉,南陽冠軍人也。父郡,宋益州刺史。道恭
 少寬厚有大量。齊文帝為雍州,召補主簿,仍除員外散騎常侍。後累有戰功,遷越騎校尉、後軍將軍。建武末,出為輔國司馬、汝南令。齊南康王為荊州,薦為西中郎中兵參軍,加輔國將軍。義兵起,蕭穎胄以道恭舊將,素著威略,專相委任,遷冠軍將軍、西中郎諮議參軍,仍轉司馬。中興元年,和帝即位,遷右衛將軍。巴西太守魯休烈等自巴、蜀連兵寇上明,以道恭持節、督西討諸軍事。次土臺,與賊合戰,道恭潛以奇兵出其後,一戰大破之,休烈等降於軍門。以功遷中領軍,固辭不受,出為使持節、右將軍、司州刺史。



 天監初,論功封漢壽縣伯,邑七百戶,
 進號平北將軍。三年,魏圍司州,時城中眾不滿五千人,食裁支半歲,魏軍攻之,晝夜不息,道恭隨方抗禦,皆應手摧卻。魏乃作大車載土,四面俱前,欲以填緌,道恭輒於緌內列艨衝鬥艦以待之,魏人不得進。又潛作伏道以決緌水,道恭載土犬屯塞之。相持百餘日,前後斬獲不可勝計。魏大造梯衝,攻圍日急,道恭於城內作土山,厚二十餘丈;多作大槊,長二丈五尺,施長刃,使壯士刺魏人登城者。魏軍甚憚之,將退。會道恭疾篤,乃呼兄子僧勰、從弟錄恩及諸將帥謂曰:「吾受國厚恩,不能破滅寇賊,今所苦轉篤,勢不支久,汝等當以死固節,無令吾沒
 有遺恨。」又令取所持節謂僧勰曰:「稟命出疆,憑此而已;即不得奉以還朝,方欲攜之同逝,可與棺柩相隨。」眾皆流涕。其年五月卒。魏知道恭死,攻之轉急。



 先是,朝廷遣郢州刺史曹景宗率眾赴援,景宗到鑿峴,頓兵不前。至八月,城內糧盡,乃陷。詔曰:「持節、都督司州諸軍事、平北將軍、司州刺史、漢壽縣開國伯道恭器幹詳審,才志通烈。王業肇構,致力陜西。受任邊垂,效彰所蒞。寇賊憑陵,竭誠守禦,奇謀間出,捷書日至。不幸抱疾,奄至殞喪,遺略所固,得移氣朔。自非徇國忘已,忠果並至,何能身沒守存,窮而後屈。言念傷悼,特兼常懷,追榮加等。抑有恆
 數。可贈鎮西將軍,使持節、都督、刺史、伯如故,并尋購喪櫬,隨宜資給。」八年,魏許還道恭喪,其家以女樂易之,葬襄陽。



 子澹嗣,卒於河東太守。孫固早卒,國除。



 楊公則,字君翼,天水西縣人也。父仲懷,宋泰始初為豫州刺史殷琰將。琰叛,輔國將軍劉勔討琰,仲懷力戰,死於橫塘。公則隨父在軍,年未弱冠,冒陣抱屍號哭,氣絕良久,勔命還仲懷首。公則殮畢,徒步負喪歸鄉里,由此著名。歷官員外散騎侍郎。梁州刺史范柏年板為宋熙太守、領白馬戍主。



 氐賊李烏奴作亂,攻白馬,公則固守經時,矢盡糧竭,陷于寇,抗聲罵賊。烏奴壯之,更厚待焉,
 要與同事。公則偽許而圖之,謀泄,單馬逃歸。梁州刺史王玄邈以事表聞,齊高帝下詔褒美。除晉壽太守,在任清潔自守。



 永明中,為鎮北長流參軍。遷扶風太守,母憂去官。雍州刺史陳顯達起為寧朔將軍。復領太守。頃之,荊州刺史巴東王子響構亂,公則率師進討。事平,遷武寧太守。在郡七年,資無擔石,百姓便之。入為前軍將軍。南康王為荊州,復為西中郎中兵參軍。領軍將軍蕭穎胃協同義舉,以公則為輔國將軍、領西中郎諮議參軍,中兵如故,率眾東下。時湘州行事張寶積發兵自守,未知所附,公則軍及巴陵,仍回師南討。軍次白沙,寶積懼,
 釋甲以俟焉。公則到,撫納之,湘境遂定。



 和帝即位,授持節、都督湘州諸軍事、湘州刺史。高祖勒眾軍次於沔口,魯山城主孫樂祖、郢州刺史張沖各據城未下,公則率湘府之眾會于夏口。時荊州諸軍受公則節度,雖蕭穎達宗室之貴亦隸焉。累進征虜將軍、左衛將軍,持節、刺史如故。



 郢城平,高祖命眾軍即日俱下,公則受命先驅,徑掩柴桑。江州既定,連旌東下,直造京邑。公則號令嚴明,秋毫不犯,所在莫不賴焉。大軍至新林,公則自越城移屯領軍府壘北樓,與南掖門相對,嘗登樓望戰。城中遙見麾蓋,縱神鋒弩射之,矢貫胡床,左右皆失色。公則
 曰:「幾中吾腳。」談笑如初。東昏夜選勇士攻公則柵,軍中驚擾,公則堅臥不起,徐命擊之,東昏軍乃退。公則所領多湘溪人,性怯懦,城內輕之,以為易與,每出盪,輒先犯公則壘。公則獎厲軍士,剋獲更多。及平,城內出者或被剝奪,公則親率麾下,列陣東掖門,衛送公卿士庶,故出者多由公則營焉。進號左將軍,持節、刺史如故,還鎮南蕃。



 初,公則東下,湘部諸郡多未賓從,及公則還州,然後諸屯聚並散。天監元年,進號平南將軍,封寧都縣侯,邑一千五百戶。湘州寇亂累年,民多流散,公則輕刑薄斂,頃之,戶口充復。為政雖無威嚴,然保己廉慎,為吏民所
 悅。湘俗單家以賂求州職,公則至,悉斷之,所辟引皆州郡著姓,高祖班下諸州以為法。



 四年,徵中護軍。代至,乘二舸便發,贐送一無所取。仍遷衛尉卿,加散騎常侍。時朝廷始議北伐,以公則威名素著,至京師,詔假節先屯洛口。公則受命遘疾,謂親人曰:「昔廉頗、馬援以年老見遺,猶自力請用。今國家不以吾朽懦,任以前驅,方於古人,見知重矣。雖臨途疾苦,豈可僶俛辭事。馬革還葬,此吾志也。」遂彊起登舟。至洛口,壽春士女歸降者數千戶。魏、豫州刺史薛恭度遣長史石榮前鋒接戰,即斬石榮,逐北至壽春,去城數十里乃反。疾卒于師,時年六十
 一。高祖深痛惜之,即日舉哀,贈車騎將軍,給鼓吹一部。謚曰烈。



 公則為人敦厚慈愛,居家篤睦,視兄子過於其子,家財悉委焉。性好學,雖居軍旅,手不輟卷,士大夫以此稱之。



 子膘嗣,有罪國除。高祖以公則勛臣,特詔聽庶長子朓嗣。朓固讓,歷年乃受。



 鄧元起,字仲居,南郡當陽人也。少有膽幹,膂力過人。性任俠,好賑施,鄉里年少多附之。起家州辟議曹從事史,轉奉朝請。雍州刺史蕭緬板為槐里令。遷弘農太守、平西軍事。時西陽馬榮率眾緣江寇抄,商旅斷絕,刺史蕭遙欣使元起率眾討平之。遷武寧太守。



 永元末,魏軍逼
 義陽,元起自郡援焉。蠻帥田孔明附于魏,自號郢州刺史,寇掠三關,規襲夏口,元起率銳卒攻之,旬月之間,頻陷六城,斬獲萬計,餘黨悉皆散走。仍戍三關。郢州刺史張沖督河北軍事,元起累與沖書,求旋軍。沖報書曰:「足下在彼,吾在此,表裏之勢,所謂金城湯池;一旦捨去,則荊棘生焉。」乃表元起為平南中兵參軍事。自是每戰必捷,勇冠當時,敢死之士樂為用命者萬有餘人。



 義師起,蕭穎胄與書招之。張沖待元起素厚,眾皆懼沖;及書至,元起部曲多勸其還郢。元起大言於眾曰:「朝廷暴虐,誅戮宰臣,群小用命,衣冠道盡。荊、雍二州同舉大事,何患
 不剋。且我老母在西,豈容背本。若事不成,政受戮昏朝,倖免不孝之罪。」即日治嚴上道。至江陵,為西中郎中兵參軍,加冠軍將軍,率眾與高祖會于夏口。高祖命王茂、曹景宗及元起等圍城,結壘九里,張沖屢戰,輒大敗,乃嬰城固守。



 和帝即位,授假節、冠軍將軍、平越中郎將、廣州刺史,遷給事黃門侍郎,移鎮南堂西渚。中興元年七月,郢城降,以本號為益州刺史,仍為前軍,先定尋陽。及大軍進至京邑,元起築壘於建陽門,與王茂、曹景宗等合長圍,身當鋒鏑。建康城平,進號征虜將軍。天監初,封當陽縣侯,邑一千二百戶。又進號左將軍,刺史如故,始
 述職焉。



 初,義師之起,益州刺史劉季連持兩端;及聞元起將至,遂發兵拒守。語在《季連傳》。元起至巴西,巴西太守朱士略開門以待。先時蜀人多逃亡,至是出投元起,皆稱起義應朝廷,師人新故三萬餘。元起在道久,軍糧乏絕。或說之曰:「蜀土政慢,民多詐疾,若儉巴西一郡籍注,困而罰之,所獲必厚。」元起然之。涪令李膺諫曰:「使君前有嚴敵,後無繼援,山民始附,於我觀德,若糾以刻薄,民必不堪,眾心一離,雖悔無及,何必起疾,可以濟師。膺請出圖之,不患資糧不足也。」元起曰:「善,一以委卿。」膺退,率富民上軍資米,俄得三萬斛。



 元起先遣將王元宗等,
 破季連將李奉伯於新巴,齊晚盛於赤水,眾進屯西平。季連始嬰城自守。晚盛又破元起將魯方達於斛石,士卒死者千餘人,師眾咸懼,元起乃自率兵稍進至蔣橋,去成都二十里,留輜重於郫。季連復遣奉伯、晚盛二千人,間道襲郫,陷之,軍備盡沒。元起遣魯方達之眾救之,敗而反,遂不能剋。元起捨郫,逕圍州城,柵其三面而塹焉。元起出巡視圍柵,季連使精勇掩之,將至麾下,元起下輿持楯叱之,眾辟易不敢進。



 時益部兵亂日久,民廢耕農,內外苦饑,人多相食,道路斷絕,季連計窮。會明年,高祖使赦季連罪,許之降。季連即日開城納元起,元起
 送季連于京師。城開,郫乃降。斬奉伯、晚盛。高祖論平蜀勛,復元起號平西將軍,增封八百戶,並前二千戶。



 元起以鄉人庾黔婁為錄事參軍,又得荊州刺史蕭遙欣故客蔣光濟,並厚待之,任以州事。黔婁甚清潔,光濟多計謀,並勸為善政。元起之剋季連也,城內財寶無所私,勤恤民事,口不論財色。性本能飲酒,至一斛不亂,及是絕之。蜀土翕然稱之。元起舅子梁矜孫性輕脫,與黔婁志行不同,乃言於元起曰:「城中稱有三刺史,節下何以堪之!」元起由此疏黔屢、光濟,而治迹稍損。



 在州二年,以母老乞歸供養,詔許焉。徵為右衛將軍,以西昌侯蕭淵藻
 代之。是時,梁州長史夏侯道遷以南鄭叛,引魏人,白馬戍主尹天寶馳使報蜀,魏將王景胤、孔陵寇東西晉壽,並遣告急,眾勸元起急救之。元起曰:「朝廷萬里,軍不卒至,若寇賊侵淫,方須撲討,董督之任,非我而誰?何事匆匆便救。」黔婁等苦諫之,皆不從。高祖亦假元起節,都督征討諸軍事,救漢中。比至,魏已攻陷兩晉壽。淵藻將至。元起頗營還裝,糧儲器械,略無遺者。淵藻入城,甚怨望之,因表其逗留不憂軍事。收付州獄,於獄自縊,時年四十八。有司追劾削爵土,詔減邑之半,乃更封松滋縣侯,邑千戶。



 初,元起在荊州,刺史隨王板元起為從事,別駕庾蓽
 堅執不可,元起恨之。大軍既至京師,蓽在城內,甚懼。及城平,元起先遣迎蓽,語人曰:「庾別駕若為亂兵所殺,我無以自明。」因厚遣之。少時又賞至其西沮田舍,有沙門造之乞,元起問田人曰:「有稻幾何?」對曰:「二十斛。」元起悉以施之。時人稱其有大度。



 元起初為益州,過江陵迎其母,母事道,方居館,不肯出。元起拜請同行。母曰:「貧賤家兒忽得富貴,詎可久保,我寧死不能與汝共入禍敗。」元起之至巴東,聞蜀亂,使蔣光濟筮之,遇《蹇》,喟然歎曰:「吾豈鄧艾而及此乎。」後果如筮。子鏗嗣。



 陳吏部尚書姚察曰:永元之末,荊州方未有釁,蕭穎胄
 悉全楚之兵,首應義舉。豈天之所啟,人惎之謀?不然,何其響附之決也?穎達叔侄慶流後嗣,夏侯、楊、鄧咸享隆名,盛矣!詳之謹厚,楊、蔡廉節,君子有取焉。



\end{pinyinscope}