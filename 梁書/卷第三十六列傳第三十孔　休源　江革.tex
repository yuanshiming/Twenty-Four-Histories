\article{卷第三十六列傳第三十孔 休源 江革}

\begin{pinyinscope}

 孔休源,字慶緒,會稽山陰人也。晉丹陽太守沖之八世孫。曾祖遙之,宋尚書水部郎。父珮,齊廬陵王記室參軍,早卒。



 休源年十一而孤,居喪盡禮,每見父手所寫書,必哀慟流涕,不能自勝,見者莫不為之垂泣。後就吳興沈驎士受經,略通大義。建武四年,州舉秀才,太尉徐孝嗣省其策,深善之,謂同坐曰:「董仲舒、華令思何以尚此,可
 謂後生之准也。觀其此對,足稱王佐之才。」瑯邪王融雅相友善,乃薦之於司徒竟陵王,為西邸學士。梁臺建,與南陽劉之遴同為太學博士,當時以為美選。休源初到京,寓於宗人少府卿孔登宅,曾以祠事入廟,侍中范雲一與相遇,深加褒賞,曰:「不期忽覯清顏,頓袪鄙吝,觀天披霧,驗之今日。」後雲命駕到少府門,登便拂筵整帶,謂當詣己,既而獨造休源,高談盡日,同載還家,登深以為愧。尚書令沈約當朝貴顯,軒蓋盈門,休源或時後來,必虛襟引接,處之坐右,商略文義。其為通人所推如此。



 俄除臨川王府行參軍。高祖嘗問吏部尚書徐勉曰:「今帝
 業初基,須一人有學藝解朝儀者,為尚書儀曹郎。為朕思之,誰堪其選?」勉對曰:「孔休源識具清通,諳練故實,自晉、宋《起居注》誦略上口。」高祖亦素聞之,即日除兼尚書儀曹郎中。是時多所改作,每逮訪前事,休源即以所誦記隨機斷決,曾無疑滯。吏部郎任昉常謂之為「孔獨誦」。



 遷建康獄正,及辨訟折獄,時罕冤人。後有選人為獄司者,高祖尚引休源以勵之。除中書舍人,司徒臨川王府記室參軍,遷尚書左丞,彈肅禮闈,雅允朝望。時太子詹事周捨撰《禮疑義》,自漢魏至于齊梁,並皆搜採,休源所有奏議,咸預編錄。除給事黃門侍郎,遷長兼御史中丞,
 正色直繩,無所回避,百僚莫不憚之。除少府卿,又兼行丹陽尹事。出為宣惠晉安王府長史、南郡太守、行荊州府州事。高祖謂之曰:「荊州總上流衝要,義高分陜,今以十歲兒委卿,善匡翼之,勿憚周昌之舉也。」對曰:「臣以庸鄙,曲荷恩遇,方揣丹誠,效其一割。」上善其對,乃敕晉安王曰:「孔休源人倫儀表,汝年尚幼,當每事師之。」尋而始興王嶦代鎮荊州,復為憺府長史,南郡太守、行府州事如故。在州累政,甚有治績,平心決斷,請託不行。高祖深嘉之。除通直散騎常侍,領羽林監,轉秘書監,遷明威將軍,復為晉安王府長史、南蘭陵太守,別敕專行南徐州
 事。休源累佐名籓,甚得民譽,王深相倚仗,軍民機務,動止詢謀。常於中齋別施一榻,云「此是孔長史坐」,人莫得預焉。其見敬如此。



 徵為太府卿,俄授都官尚書,頃之,領太子中庶子。普通七年,揚州刺史臨川王宏薨,高祖與群臣議代王居州任者久之,于時貴戚王公,咸望遷授,高祖曰:「朕已得人。孔休源才識通敏,實應此選。」乃授宣惠將軍、監揚州。休源初為臨川王行佐,及王薨而管州任,時論榮之。而神州都會,簿領殷繁,休源割斷如流,傍無私謁。中大通二年,加授金紫光祿大夫,監揚州如故。累表陳讓,優詔不許。在州晝決辭訟,夜覽墳籍。每車駕
 巡幸,常以軍國事委之。



 昭明太子薨,有敕夜召休源入宴居殿,與群公參定謀議,立晉安王綱為皇太子。四年,遘疾,高祖遣中使候問,并給醫藥,日有十數。其年五月,卒,時年六十四。遺令薄葬,節朔薦蔬菲而已。高祖為之流涕,顧謂謝舉曰:「孔休源奉職清忠,當官正直,方欲共康治道,以隆王化。奄至殞歿,朕甚痛之。」舉曰:「此人清介彊直,當今罕有,微臣竊為陛下惜之。」詔曰:「慎終追遠,歷代通規;褒德疇庸,先王令典。宣惠將軍、金紫光祿大夫、監揚州孔休源,風業貞正,雅量沖邈,升榮建禮,譽重搢紳。理務神州,化覃歌詠,方興仁壽,穆是倫。奄然永逝,
 倍用悲惻。可贈散騎常侍、金紫光祿大夫,賻第一材一具,布五十匹,錢五萬,蠟二百斤。剋日舉哀。喪事所須,隨由資給。謚曰貞子。」皇太子手令曰:「金紫光祿大夫孔休源,立身中正,行己清恪。昔歲西浮渚宮,東泊枌壤,毘佐蕃政,實盡厥誠。安國之詳審,公儀之廉白,無以過之。奄至殞喪,情用惻怛。今須舉哀,外可備禮。」



 休源少孤,立志操,風範彊正,明練治體。持身儉約,學窮文藝,當官理務,不憚彊禦,常以天下為己任。高祖深委仗之。累居顯職,纖毫無犯。性慎密,寡嗜好。出入帷幄,未嘗言禁中事,世以此重之。聚書盈七千卷,手自校治,凡奏議彈文,勒成十
 五卷。



 長子雲童,頗有父風,而篤信佛理,遍持經戒。官至岳陽王府諮議、東揚州別駕。



 少子宗軌,聰敏有識度,歷尚書都官郎,司徒左西掾,中書郎。



 江革,字休映,濟陽考城人也。祖齊之,宋尚書金部郎。父柔之,齊尚書倉部郎,有孝行,以母憂毀卒。革幼而聰敏,早有才思,六歲便解屬文。柔之深加賞器,曰:「此兒必興吾門。」九歲丁父艱,與弟觀同生孤貧,傍無師友,兄弟自相訓勖,讀書精力不倦。十六喪母,以孝聞。服闋,與觀俱詣太學,補國子生,舉高第。齊中書郎王融、吏部謝朓雅相欽重。朓嘗宿衛,還過候革,時大雪,見革弊絮單席,而
 耽學不倦,嗟嘆久之,乃脫所著襦,并手割半氈與革充臥具而去。司徒竟陵王聞其名,引為西邸學士。弱冠舉南徐州秀才。時豫章胡諧之行州事,王融與諧之書,令薦革。諧之方貢瑯邪王泛,便以革代之。解褐奉朝請。僕射江祏深相引接,祏為太子詹事,啟革為府丞。祏時權傾朝右,以革才堪經國,令參掌機務,詔誥文檄,皆委以具。革防杜形迹,外人不知。祏誅,賓客皆罹其罪,革獨以智免。



 除尚書駕部郎。中興元年,高祖入石頭,時吳興太守袁昂據郡距義師,乃使革製書與昂,於坐立成,辭義典雅,高祖深賞歎之,因令與徐勉同掌書記。建安王為
 雍州刺史,表求管記,以革為征北記室參軍,帶中廬令。與弟觀少長共居,不忍離別,苦求同行,乃以觀為征北行參軍,兼記室。時吳興沈約、樂安任昉,並相賞重,昉與革書云:「此段雍府妙選英才,文房之職,總卿昆季,可謂馭二龍於長途,騁騏驥於千里。」途次江夏,觀遇疾卒。革時在雍,為府王所禮,款若布衣。王被徵為丹陽尹,以革為記室,領五官掾,除通直散騎常侍,建康正。頻遷秣陵、建康令。為治明肅,豪彊憚之。入為中書舍人,尚書左丞,司農卿,復出為雲麾晉安王長史、尋陽太守、行江州府事。徙仁威廬陵王長史,太守、行事如故,以清嚴為百城
 所憚。時少王行事多傾意於簽帥,革以正直自居,不與簽帥等同坐。俄遷左光祿大夫、南平王長史、御史中丞,彈奏豪權,一無所避。



 除少府卿,出為貞威將軍、北中郎南康王長史、廣陵太守,改授鎮北豫章王長史,將軍、太守如故。時魏徐州刺史元法僧降附,革被敕隨府王鎮彭城。城既失守,革素不便馬,乃泛舟而還,途經下邳,遂為魏人所執。魏徐州刺史元延明聞革才名,厚加接待。革稱患腳不拜,延明將加害焉,見革辭色嚴正,更相敬重。時祖芃同被拘執,延明使芃作《欹器》、《漏刻銘》,革罵芃曰:「卿荷國厚恩,已無報答,今乃為虜立銘,孤負朝廷。」延
 明聞之,乃令革作丈八寺碑并祭彭祖文,革辭以囚執既久,無復心思。延明逼之逾苦,將加箠撲。革厲色而言曰:「江革行年六十,不能殺身報主,今日得死為幸,誓不為人執筆。」延明知不可屈,乃止。日給脫粟三升,僅餘性命。值魏主討中山王元略反北,乃放革及祖芃還朝。詔曰:「前貞威將軍、鎮北長史、廣陵太守江革,才思通贍,出內有聞,在朝正色,臨危不撓,首佐台鉉,實允僉諧。可太尉臨川王長史。」



 時高祖盛於佛教,朝賢多啟求受戒,革精信因果,而高祖未知,謂革不奉佛教,乃賜革《覺意詩》五百字,云「惟當勤精進,自彊行勝脩;豈可作底突,如彼
 必死囚。以此告江革,并及諸貴遊。」又手敕云:「世間果報,不可不信,豈得底突如對元延明邪?」革因啟乞受菩薩戒。



 重除少府卿、長史、校尉。時武陵王在東州,頗自驕縱,上召革面敕曰:「武陵王年少,臧盾性弱,不能匡正,欲以卿代為行事。非卿不可,不得有辭。」乃除折衝將軍、東中郎武陵王長史、會稽郡丞、行府州事。革門生故吏,家多在東州,聞革應至,並齎持緣道迎候。革曰:「我通不受餉,不容獨當故人筐篚。」至鎮,惟資公俸,食不兼味。郡境殷廣,辭訟日數百,革分判辨析,曾無疑滯。功必賞,過必罰,民安吏畏,百城震恐。瑯邪王騫為山陰令,贓貨狼藉,望
 風自解。府王憚之,遂雅相欽重。每至侍宴,言論必以《詩》《書》,王因此耽學好文。典簽沈熾文以王所製詩呈高祖,高祖謂僕射徐勉曰:「江革果能稱職。」乃除都官尚書。將還,民皆戀惜之,贈遺無所受。送故依舊訂舫,革並不納,惟乘臺所給一舸。舸艚偏欹,不得安臥。或謂革曰:「船既不平,濟江甚險,當移徙重物,以迮輕艚。」革既無物,乃於西陵岸取石十餘片以實之。其清貧如此。尋監吳郡。於時境內荒儉,劫盜公行。革至郡,惟有公給仗身二十人,百姓皆懼不能靜寇;反省遊軍尉,民下逾恐。革乃廣施恩撫,明行制令,盜賊靜息,民吏安之。



 武陵王出鎮江州,
 乃曰:「我得江革,文華清麗,豈能一日忘之,當與其同飽。」乃表革同行。又除明威將軍、南中郎長史、尋陽太守。徵入為度支尚書。好獎進閭閻,為後生延譽,由是衣冠士子,翕然歸之。時尚書令何敬容掌選,序用多非其人。革性彊直,每至朝宴,恒有褒貶,以此為權勢所疾,乃謝病還家。除光祿大夫、領步兵校尉、南、北兗二州大中正,優遊閑放,以文酒自娛。大同元年二月,卒,謚曰彊子。有集二十卷,行於世。革歷官八府長史,四王行事,三為二千石,傍無姬侍,家徒壁立,世以此高之。



 長子行敏,好學有才俊,官至通直郎,早卒,有集五卷。



 次子從簡,少有文情,
 年十七,作《採荷詞》以刺敬容,為當時所賞。歷官司徒從事中郎。侯景亂,為任約所害。子兼叩頭流血,乞代父命,以身蔽刃,遂俱見殺。天下莫不痛之。



 史臣曰:高祖留心政道,孔休源以識治見知,既遇其時,斯為幸矣。江革聰敏亮直,亦一代之盛名歟。



\end{pinyinscope}