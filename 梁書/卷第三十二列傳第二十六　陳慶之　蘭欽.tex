\article{卷第三十二列傳第二十六 陳慶之 蘭欽}

\begin{pinyinscope}

 陳慶之,字子雲,義興國山人也。幼而隨從高祖。高祖性好棋,每從夜達旦不輟,等輩皆倦寐,惟慶之不寢,聞呼即至,甚見親賞。從高祖東下平建鄴,稍為主書,散財聚士,常思效用。除奉朝請。普通中,魏徐州刺史元法僧於彭城求入內附,以慶之為武威將軍,與胡龍牙、成景俊率諸軍應接。還,除宣猛將軍、文德主帥,仍率軍二千,送
 豫章王綜入鎮徐州。魏遣安豊王元延明、臨淮王元彧率眾二萬來拒,屯據陟□。延明先遣其別將丘大千築壘潯梁,觀兵近境。慶之進薄其壘,一鼓便潰。後豫章王棄軍奔魏,眾皆潰散,諸將莫能制止。慶之乃斬關夜退,軍士得全。普通七年,安西將軍元樹出征壽春,除慶之假節、總知軍事。魏豫州刺史李憲遣其子長鈞別築兩城相拒。慶之攻之,憲力屈遂降,慶之入據其城。轉東宮直閣,賜爵關中侯。



 大通元年,隸領軍曹仲宗伐渦陽。魏遣征南將軍常山王元昭等率馬步十五萬來援,前軍至駝澗,去渦陽四十里。慶之欲逆戰,韋放以賊之前鋒
 必是輕銳,與戰若捷,不足為功,如其不利,沮我軍勢,兵法所謂以逸待勞,不如勿擊。慶之曰:「魏人遠來,皆已疲倦,去我既遠,必不見疑,及其未集,須挫其氣,出其不意,必無不敗之理。且聞虜所據營,林木甚盛,必不夜出。諸君若疑惑,慶之請獨取之。」於是與麾下二百騎奔擊,破其前軍,魏人震恐。慶之乃還與諸將連營而進,據渦陽城,與魏軍相持。自春至冬,數十百戰,師老氣衰,魏之援兵復欲築壘於軍後,仲宗等恐腹背受敵,謀欲退師。慶之杖節軍門曰:「共來至此,涉歷一歲,糜費糧仗,其數極多。諸軍並無鬥心,皆謀退縮,豈是欲立功名,直聚為抄
 暴耳。吾聞置兵死地,乃可求生,須虜大合,然後與戰。審欲班師,慶之別有密敕,今日犯者,便依明詔。」仲宗壯其計,乃從之。魏人掎角作十三城,慶之銜枚夜出,陷其四壘,渦陽城主王緯乞降。所餘九城,兵甲猶盛,乃陳其俘馘,鼓噪而攻之,遂大奔潰,斬獲略盡,渦水咽流,降城中男女三萬餘口。詔以渦陽之地置西徐州。眾軍乘勝前頓城父。高祖嘉焉,賜慶之手詔曰:「本非將種,又非豪家,觖望風雲,以至於此。可深思奇略,善克令終。開朱門而待賓,揚聲名於竹帛,豈非大丈夫哉!」



 大通初,魏北海王元顥以本朝大亂,自拔來降,求立為魏主。高祖納之,以
 慶之為假節、飆勇將軍,送元顥還北。顥於渙水即魏帝號,授慶之使持節、鎮北將軍、護軍、前軍大都督,發自銍縣,進拔滎城,遂至睢陽。魏將丘大千有眾七萬,分築九城以相拒。慶之攻之,自旦至申,陷其三壘,大千乃降。時魏征東將軍濟陰王元暉業率羽林庶子二萬人來救梁、宋,進屯考城,城四面縈水,守備嚴固。慶之命浮水築壘,攻陷其城,生擒暉業,獲租車七千八百輛。仍趨大梁,望旗歸款。顥進慶之衛將軍、徐州刺史、武都公。仍率眾而西。



 魏左僕射楊昱、西阿王元慶、撫軍將軍元顯恭率御仗羽林宗子庶子眾凡七萬,據滎陽拒顥。兵既精強,
 城又險固,慶之攻未能拔。魏將元天穆大軍復將至,先遣其驃騎將軍爾朱吐沒兒領胡騎五千,騎將魯安領夏州步騎九千,援楊昱;又遣右僕射爾朱世隆、西荊州刺史王羆騎一萬,據虎牢。天穆、吐沒兒前後繼至,旗鼓相望。時滎陽未拔,士眾皆恐,慶之乃解鞍秣馬,宣喻眾曰:「吾至此以來,屠城略地,實為不少;君等殺人父兄,略人子女,又為無算。天穆之眾,並是仇讎。我等纔有七千,虜眾三十餘萬,今日之事,義不圖存。吾以虜騎不可爭力平原,及未盡至前,須平其城壘,諸君無假狐疑,自貽屠膾。」一鼓悉使登城,壯士東陽宋景休、義興魚天愍踰堞而
 入,遂克之。俄而魏陣外合,慶之率騎三千背城逆戰,大破之,魯安於陣乞降,元天穆、爾朱吐沒兒單騎獲免。收滎陽儲實,牛馬穀帛不可勝計。進赴虎牢,爾朱世隆棄城走。魏主元子攸懼,奔并州。其臨淮王元彧、安豊王元延明率百僚,封府庫,備法駕,奉迎顥入洛陽宮,御前殿,改元大赦。顥以慶之為侍中、車騎大將軍、左光祿大夫,增邑萬戶。魏大將軍上黨王元天穆、王老生、李叔仁又率眾四萬,攻陷大梁,分遣老生、費穆兵二萬,據虎牢,刁宣、刁雙入梁、宋,慶之隨方掩襲,並皆降款。天穆與十餘騎北渡河。高祖復賜手詔稱美焉。慶之麾下悉著白袍,所
 向披靡。先是洛陽童謠曰:「名師大將莫自牢,千兵萬馬避白袍。」自發銍縣至于洛陽,十四旬平三十二城,四十七戰,所向無前。



 初,元子攸止單騎奔走,宮衛嬪侍無改於常。顥既得志,荒于酒色,乃日夜宴樂,不復視事。與安豊、臨淮共立姦計,將背朝恩,絕賓貢之禮;直以時事未安,且資慶之之力用,外同內異,言多忌刻。慶之心知之,亦密為其計。乃說顥曰:「今遠來至此,未伏尚多,若人知虛實,方更連兵,而安不忘危,須預為其策。宜啟天子,更請精兵;并勒諸州,有南人沒此者,悉須部送。」顥欲從之,元延明說顥曰:「陳慶之兵不出數千,已自難制;今增其
 眾,寧肯復為用乎?權柄一去,動轉聽人,魏之宗社,於斯而滅。」顥由是致疑,稍成疏貳。慮慶之密啟,乃表高祖曰:「河北、河南一時已定,唯爾朱榮尚敢跋扈,臣與慶之自能擒討。今州郡新服,正須綏撫,不宜更復加兵,搖動百姓。」高祖遂詔眾軍皆停界首。洛下南人不出一萬,羌夷十倍,軍副馬佛念言於慶之曰:「功高不賞,震主身危,二事既有,將軍豈得無慮?自古以來,廢昏立明,扶危定難,鮮有得終。今將軍威震中原,聲動河塞,屠顥據洛,則千載一時也。」慶之不從。顥前以慶之為徐州刺史,因固求之鎮。顥心憚之,遂不遣。乃曰:「主上以洛陽之地全相任委,
 忽聞捨此朝寄,欲往彭城,謂君遽取富貴,不為國計,手敕頻仍,恐成僕責。」慶之不敢復言。



 魏天柱將軍爾朱榮、右僕射爾朱世隆、大都督元天穆、驃騎將軍爾朱吐沒兒、榮長史高歡、鮮卑、芮芮,勒眾號百萬,挾魏主元子攸來攻顥。顥據洛陽六十五日,凡所得城,一時反叛。慶之渡河守北中郎城,三日中十有一戰,傷殺甚眾。榮將退,時有劉助者,善天文,乃謂榮曰:「不出十日,河南大定。」榮乃縛木為筏,濟自硤石,與顥戰於河橋,顥大敗,走至臨潁,遇賊被擒,洛陽陷。慶之馬步數千,結陣東反,榮親自來追,值蒿高山水洪溢,軍人死散。慶之乃落鬚髮為沙
 門,間行至豫州,豫州人程道雍等潛送出汝陰。至都,仍以功除右衛將軍,封永興縣侯,邑一千五百戶。



 出為持節、都督緣淮諸軍事、奮武將軍、北兗州刺史。會有妖賊沙門僧強自稱為帝,土豪蔡伯龍起兵應之。僧強頗知幻術,更相扇惑,眾至三萬,攻陷北徐州,濟陰太守楊起文棄城走,鐘離太守單希寶見害,使慶之討焉。車駕幸白下,臨餞謂慶之曰:「江、淮兵勁,其鋒難當,卿可以策制之,不宜決戰。」慶之受命而行。曾未浹辰,斬伯龍、僧強,傳其首。



 中大通二年,除都督南、北司、西豫、豫四州諸軍事、南、北司二州刺史,餘並如故。慶之至鎮,遂圍懸瓠。破魏
 潁州刺史婁起、揚州刺史是云寶於溱水,又破行臺孫騰、大都督侯進、豫州刺史堯雄、梁州刺史司馬恭於楚城。罷義陽鎮兵,停水陸轉運,江湖諸州並得休息。開田六千頃,二年之後,倉廩充實。高祖每嘉勞之。又表省南司州,復安陸郡,置上明郡。



 大同二年,魏遣將侯景率眾七萬寇楚州,刺史桓和陷沒,景仍進軍淮上,貽慶之書使降。敕遣湘潭侯退、右衛夏侯夔等赴援,軍至黎漿,慶之已擊破景。時大寒雪,景棄輜重走,慶之收之以歸。進號仁威將軍。是歲,豫州饑,慶之開倉賑給,多所全濟。州民李昇等八百人表請樹碑頌德,詔許焉。五年十月,卒,
 時年五十六。贈散騎常侍、左衛將軍,鼓吹一部。謚曰武。敕義興郡發五百丁會喪。



 慶之性祗慎,衣不紈綺,不好絲竹,射不穿札,馬非所便,而善撫軍士,能得其死力。長子昭嗣。



 第五子昕,字君章。七歲能騎射。十二隨父入洛,於路遇疾,還京師。詣鴻臚卿朱異,異訪北間形勢,昕聚土畫地,指麾分別,異甚奇之。大同四年,為邵陵王常侍、文德主帥、右衛仗主,敕遣助防義陽。魏豫州刺史堯雄,北間驍將,兄子寶樂,特為敢勇。慶之圍懸瓠,雄來赴其難,寶樂求單騎校戰,昕躍馬直趣寶樂,雄即散潰,仍陷溱城。六年,除威遠將軍、小峴城主,以公事免。十年,妖賊王
 勤宗起於巴山郡,以昕為宣猛將軍,假節討焉。勤宗平,除陰陵戍主、北譙太守,以疾不之官。又除驃騎外兵,俄為臨川太守。太清二年,侯景圍歷陽,敕召昕還,昕啟云:「采石急須重鎮,王質水軍輕弱,恐慮不濟。」乃板昕為雲騎將軍,代質,未及下渚,景已渡江,仍遣率所領遊防城外,不得入守。欲奔京口,乃為景所擒。景見昕殷勤,因留極飲,曰:「我至此得卿,餘人無能為也。」令昕收集部曲,將用之,昕誓而不許。景使其儀同范桃棒嚴禁之,昕因說桃棒令率所領歸降,襲殺王偉、宋子仙為信。桃棒許之,遂盟約,射啟城中,遣昕夜縋而入。高祖大喜,敕即受降,
 太宗遲疑累日不決,外事發洩,昕弗之知,猶依期而下。景邀得之,乃逼昕令更射書城中,云「桃棒且輕將數十人先入。」景欲裹甲隨之。昕既不肯為書,期以必死,遂為景所害,時年三十三。



 蘭欽,字休明,中昌魏人也。父子雲,天監中,軍功官至雲麾將軍,冀州刺史。欽幼而果決,篸捷過人。隨父北征,授東宮直閣。大通元年,攻魏蕭城,拔之。仍破彭城別將郊仲,進攻擬山城,破其大都督劉屬眾二十萬。進攻籠城,獲馬千餘匹。又破其大將柴集及襄城太守高宣、別將范思念、鄭承宗等。仍攻厥固、張龍、子城,未拔,魏彭城守
 將楊目遣子孝邕率輕兵來援,欽逆擊走之。又破譙州刺史劉海游,還拔厥固,收其家口。楊目又遣都督范思念、別將曹龍牙數萬眾來援,欽與戰,於陣斬龍牙,傳首京師。



 又假欽節,都督衡州三郡兵,討桂陽、陽山、始興叛蠻,至即平破之。封安懷縣男,邑五百戶。又破天漆蠻帥晚時得。會衡州刺史元慶和為桂陽人嚴容所圍,遣使告急,欽往應援,破容羅溪,於是長樂諸洞一時平蕩。又密敕欽向魏興,經南鄭,屬魏將托跋勝寇襄陽,仍敕赴援。除持節、督南梁、南、北秦、沙四州諸軍事、光烈將軍、平西校尉、梁、南秦二州刺史,增封五百戶,進爵為侯。破通
 生,擒行臺元子禮、大將薛俊、張菩薩,魏梁州刺史元羅遂降,梁、漢底定。進號智武將軍,增封二千戶。俄改授持節、都督衡、桂二州諸軍事、衡州刺史。未及述職,魏遣都督董紹、張獻攻圍南鄭,梁州刺史杜懷瑤請救。欽率所領援之,大破紹、獻於高橋城,斬首三千餘,紹、獻奔退,追入斜谷,斬獲略盡。西魏相宇文黑泰致馬二千匹,請結鄰好。詔加散騎常侍,進號仁威將軍,增封五百戶,仍令述職。



 經廣州,因破俚帥陳文徹兄弟,並擒之。至衡州,進號平南將軍,改封曲江縣公,增邑五百戶。在州有惠政,吏民詣闕請立碑頌德,詔許焉。徵為散騎常侍、左衛將
 軍,尋改授散騎常侍、安南將軍、廣州刺史。既至任所,前刺史南安侯密遣廚人置藥於食,欽中毒而卒,時年四十二。詔贈侍中、中衛將軍,鼓吹一部。



 子夏禮,侯景至歷陽,率其部曲邀擊景,兵敗死之。



 史臣曰:陳慶之、蘭欽俱有將略,戰勝攻取,蓋頗、牧、衛、霍之亞歟。慶之警悟,早侍高祖,既預舊恩,加之謹肅,蟬冕組珮,亦一世之榮矣。



\end{pinyinscope}