\article{卷第三十一列傳第二十五 袁 昂子君正}

\begin{pinyinscope}

 袁昂,字千里,陳郡陽夏人。祖詢,宋征虜將軍、吳郡太守,父抃,冠軍將軍、雍州刺史,泰始初,舉兵奉晉安王子勛,事敗誅死。昂時年五歲,乳媼攜抱匿於廬山,會赦得出,猶徙晉安。至元徽中聽還,時年十五。初,抃敗,傳首京師,藏於武庫,至是始還之。昂號慟嘔血,絕而復蘇,從兄彖嘗撫視抑譬,昂更制服,廬于墓次。後與彖同見從叔司
 徒粲,粲謂彖曰:「其幼孤而能至此,故知名器自有所在。」



 齊初,起家冠軍安成王行參軍,遷征虜主簿,太子舍人,王儉鎮軍府功曹史。儉時為京尹,經於後堂獨引見昂,指北堂謂昂曰:「卿必居此。」累遷秘書丞,黃門侍郎。昂本名千里,齊永明中,武帝謂之曰:「昂昂千里之駒,在卿有之,今改卿名為昂。即千里為字。」出為安南鄱陽王長史、尋陽公相。還為太孫中庶子、衛軍武陵王長史。



 丁內憂,哀毀過禮。服未除而從兄彖卒。昂幼孤,為彖所養,乃制期服。人有怪而問之者,昂致書以喻之曰:「竊聞禮由恩斷,服以情申。故小功他邦,加制一等,同爨有緦,明之典
 籍。孤子夙以不天,幼傾乾廕,資敬未奉,過庭莫承。藐藐沖人,未達朱紫。從兄提養訓教,示以義方,每假其談價,虛其聲譽,得及人次,實亦有由。兼開拓房宇,處以華曠,同財共有,恣其取足。爾來三十餘年,憐愛之至,無異於己。姊妹孤姪,成就一時,篤念之深,在終彌固,此恩此愛,畢壤不追。既情若同生,而服為諸從,言心即事,實未忍安。昔馬棱與弟毅同居,毅亡,棱為心服三年。由也之不除喪,亦緣情而致制,雖識不及古,誠懷感慕。常願千秋之後,從服期齊;不圖門衰,禍集一旦,草土殘息,復罹今酷,尋惟慟絕,彌劇彌深。今以餘喘,欲遂素志,庶寄其罔
 慕之痛,少申無已之情。雖禮無明據,乃事有先例,率迷而至,必欲行之。君問禮所歸,謹以諮白。臨紙號哽,言不識次。」



 服闋,除右軍邵陵王長史,俄遷御史中丞。時尚書令王晏弟詡為廣州,多納賕貨,昂依事劾奏,不憚權豪,當時號為正直。出為豫章內史,丁所生母憂去職。以喪還,江路風浪暴駭,昂乃縛衣著柩,誓同沉溺。及風止,餘船皆沒,唯昂所乘船獲全,咸謂精誠所致。葬訖,起為建武將軍、吳興太守。



 永元末,義師至京師,州牧郡守皆望風降款,昂獨拒境不受命。高祖手書喻曰:「夫禍福無門,興亡有數,天之所棄,人孰能匡?機來不再,圖之宜早。頃
 藉聽道路,承欲狼顧一隅,既未悉雅懷,聊申往意。獨夫狂悖,振古未聞,窮凶極虐,歲月滋甚。天未絕齊,聖明啟運,兆民有賴,百姓來蘇。吾荷任前驅,掃除京邑,方撥亂反正,伐罪弔民,至止以來,前無橫陣。今皇威四臨,長圍已合,遐邇畢集,人神同奮。銳卒萬計,鐵馬千群,以此攻戰,何往不克。況建業孤城,人懷離阻,面縛軍門,日夕相繼,屠潰之期,勢不云遠。兼熒惑出端門,太白入氐室,天文表於上,人事符於下,不謀同契,實在茲辰。且范岫、申胄,久薦誠款,各率所由,仍為掎角,沈法瑀、孫肸、朱端,已先肅清吳會,而足下欲以區區之郡,御堂堂之師,根本
 既傾,枝葉安附?童兒牧豎,咸謂其非,求之明鑒,實所未達。今竭力昏主,未足為忠,家門屠滅,非所謂孝,忠孝俱盡,將欲何依?豈若翻然改圖,自招多福,進則遠害全身,退則長守祿位。去就之宜,幸加詳擇。若執迷遂往,同惡不悛,大軍一臨,誅及三族。雖貽後悔,寧復云補?欲布所懷,故致今白。」昂答曰:「都史至,辱誨。承藉以眾論,謂僕有勤王之舉,兼蒙誚責,獨無送款,循復嚴旨,若臨萬仞。三吳內地,非用兵之所,況以偏隅一郡,何能為役?近奉敕,以此境多虞,見使安慰。自承麾旆屆止,莫不膝袒軍門,惟僕一人敢後至者,政以內揆庸素,文武無施,直是東
 國賤男子耳。雖欲獻心,不增大師之勇;置其愚默,寧沮眾軍之威。幸藉將軍含弘之大,可得從容以禮。竊以一飡微施,尚復投殞,況食人之祿,而頓忘一旦。非惟物議不可,亦恐明公鄙之,所以躊躇,未遑薦璧。遂以輕微,爰降重命,震灼于心,忘其所厝,誠推理鑒,猶懼威臨。」建康城平,昂束身詣闕,高祖宥之不問也。



 天監二年,以為後軍臨川王參軍事。昂奉啟謝曰:「恩降絕望之辰,慶集寒心之日,焰灰非喻,荑枯未擬,摳衣聚足,顛狽不勝。臣遍歷三墳,備詳六典,巡校賞罰之科,調檢生死之律,莫不嚴五辟於明君之朝,峻三章於聖人之世。是以塗山始
 會,致防風之誅;酆邑方構,有崇侯之伐。未有緩憲於斫戮之人,賒刑於耐罪之族,出萬死入一生如臣者也。推恩及罪,在臣實大,披心瀝血,敢乞言之。臣東國賤人,學行何取,既殊鳴鴈直木,故無結綬彈冠,徒藉羽儀,易農就仕。往年濫職,守秩東隅,仰屬龔行,風驅電掩。當其時也,負鼎圖者日至,執玉帛者相望。獨在愚臣,頓昏大義,殉鴻毛之輕,忘同德之重。但三吳險薄,五湖交通,屢起田儋之變,每懼殷通之禍,空慕君魚保境,遂失師涓抱器。後至者斬,臣甘斯戮。明刑徇眾,誰曰不然。幸約法之弘,承解網之宥,猶當降等薪粲,遂乃頓釋鉗赭。斂骨吹
 魂,還編黔庶,濯疵蕩穢,入楚遊陳,天波既洗,雲油遽沐。古人有言:『非死之難,處死之難。』臣之所荷,曠古不書;臣之死所,未知何地。」



 高祖答曰:「朕遺射鉤,卿無自外。」俄除給事黃門侍郎。其年遷侍中。明年,出為尋陽太守,行江州事。六年,徵為吏部尚書,累表陳讓,徙為左民尚書,兼右僕射。七年,除國子祭酒,兼僕射如故,領豫州大中正。八年,出為仁威將軍、吳郡太守。十一年,入為五兵尚書,復兼右僕射,未拜,有詔即真封。尋以本官領起部尚書,加侍中。十四年,馬仙琕破魏軍於朐山,詔權假昂節,往勞軍。十五年,遷左僕射,尋為尚書令、宣惠將軍。普通三
 年,為中書監、丹陽尹。其年進號中衛將軍,復為尚書令,即本號開府儀同三司,給鼓吹,未拜,又領國子祭酒。大通元年,加中書監,給親信三十人。尋表解祭酒,進號中撫軍大將軍,遷司空、侍中、尚書令,親信、鼓吹並如故。五年,加特進、左光祿大夫,增親信為八十人。大同六年,薨,時年八十。詔曰:「侍中、特進、左光祿大夫、司空昂,奄至薨逝,惻怛於懷。公器珝凝素,志誠貞方,端朝燮理,嘉猷載緝。追榮表德,實惟令典。可贈本官,鼓吹一部,給東園祕器,朝服一具,衣一襲,錢二十萬,絹布一百匹,蠟二百斤,即日舉哀。」



 初,昂臨終遺疏,不受贈謚。敕諸子不得言上
 行狀及立志銘,凡有所須,悉皆停省。復曰:「吾釋褐從仕,不期富貴,但官序不失等倫,衣食粗知榮辱,以此闔棺,無慚鄉里。往忝吳興,屬在昏明之際,既闇於前覺,無識於聖朝,不知天命,甘貽顯戮,幸遇殊恩,遂得全門戶。自念負罪私門,階榮望絕,保存性命,以為幸甚;不謂叨竊寵靈,一至於此。常欲竭誠酬報,申吾乃心,所以朝廷每興師北伐,吾輒啟求行,誓之丹款,實非矯言。既庸懦無施,皆不蒙許,雖欲罄命,其議莫從。今日瞑目,畢恨泉壤,若魂而有知,方期結草。聖朝遵古,知吾名品,或有追遠之恩,雖是經國恆典,在吾無應致此,脫有贈官,慎勿祗奉。」
 諸子累表陳奏,詔不許。冊謚曰穆正公。



 子君正,美風儀,善自居處,以貴公子得當世名譽。頃之,兼吏部郎,以母憂去職。服闋,為邵陵王友、北中郎長史、東陽太守。尋徵還都,郡民徵士徐天祐等三百人詣闕乞留一年,詔不許,仍除豫章內史,尋轉吳郡太守。侯景亂,率數百人隨邵陵王赴援,及京城陷,還郡。



 君正當官蒞事有名稱,而蓄聚財產,服玩靡麗。賊遣于子悅攻之,新城戍主戴僧易勸令拒守;吳陸映公等懼賊脫勝,略其資產,乃曰:「賊軍甚銳,其鋒不可當;今若拒之,恐民心不從也。」君正性怯懦,乃送米及牛酒,郊迎子悅。子悅既
 至,掠奪其財物子女,因是感疾卒。



 史臣曰:夫天尊地卑,以定君臣之位;松筠等質,無革歲寒之心。袁千里命屬崩離,身逢厄季,雖獨夫喪德,臣志不移;及抗疏高祖,無虧忠節,斯亦存夷、叔之風矣。終為梁室臺鼎,何其美焉。



\end{pinyinscope}