\article{卷第二十二列傳第十六 太祖五王}

\begin{pinyinscope}

 太祖十男。張皇后生長沙宣武王懿、永陽昭王敷、高祖、衡陽宣王暢。李太妃生桂陽簡王融。懿及融,齊永元中為東昏所害;敷、暢,建武中卒:高祖踐阼,並追封郡王。陳太妃生臨川靖惠王宏,南平元襄王偉。吳太妃生安成康王秀,始興忠武王憺。費太妃生鄱陽忠烈王恢。



 臨川靖惠王宏,字宣達,太祖第六子也。長八尺,美鬚眉,
 容止可觀。齊永明十年,為衛軍廬陵王法曹行參軍,遷太子舍人。時長沙王懿鎮梁州,為魏所圍,明年,給宏精兵千人赴援,未至,魏軍退。遷驃騎晉安王主簿,尋為北中郎桂陽王功曹史。衡陽王暢,有美名,為始安王蕭遙光所禮。及遙光作亂,逼暢入東府,暢懼禍,先赴臺。高祖在雍州,常懼諸弟及禍,謂南平王偉曰:「六弟明於事理,必先還臺。」及信至,果如高祖策。



 高祖義師下,宏至新林奉迎,拜輔國將軍。建康平,遷西中郎將、中護軍,領石頭戍軍事。天監元年,封臨川郡王,邑二千戶。尋為使持節、散騎常侍、都督揚、南徐州諸軍事、後將軍、揚州刺史,又
 給鼓吹一部。三年,加侍中,進號中軍將軍。



 四年,高祖詔北伐,以宏為都督南北兗、北、徐、青、冀、豫、司、霍八州北討諸軍事。宏以帝之介弟,所領皆器械精新,軍容甚盛,北人以為百數十年所未之有。軍次洛口,宏前軍剋梁城,斬魏將濆清。會征役久,有詔班師。六年夏,遷驃騎將軍、開府儀同三司,侍中如故。其年,遷司徒,領太子太傅。八年夏,為使持節、都督揚、南徐二州諸軍事、司空、揚州刺史,侍中如故。其年冬,以公事左遷驃騎大將軍,開府同三司之儀,侍中如故。未拜,遷使持節、都督揚、徐二州諸軍事、揚州刺史,侍中、將軍如故。十二年,遷司空,使持節、
 侍中、都督、刺史、將軍並如故。



 十五年春,所生母陳太妃寢疾,宏與母弟南平王偉侍疾,並衣不解帶,每二宮參問,輒對使涕泣。及太妃薨,水漿不入口者五日,高祖每臨幸慰勉之。宏少而孝謹,齊之末年,避難潛伏,與太妃異處,每遣使參問起居。或謂宏曰:「逃難須密,不宜往來。」宏銜淚答曰:「乃可無我,此事不容暫廢。」尋起為中書監,驃騎大將軍、使持節、都督如故,固辭弗許。



 十七年夏,以公事左遷侍中、中軍將軍、行司徒。其年冬,遷侍中、中書監、司徒。普通元年,遷使持節、都督揚、南徐州諸軍事、太尉、揚州刺史,侍中如故。二年,改創南、北郊,以本官領起
 部尚書,事竟罷。



 七年三月,以疾累表自陳,詔許解揚州,餘如故。四月,薨,時年五十四。自疾至于薨,輿駕七出臨視。及葬,詔曰:「侍中、太尉臨川王宏,器宇沖貴,雅量弘通。爰初弱齡,行彰素履;逮於應務,嘉猷載緝。自皇業啟基,地惟介弟,久司神甸,歷位台階,論道登朝,物無異議。朕友于之至,家國兼情,方弘燮贊,儀刑列辟。天不裛遺,奄焉不永,哀痛抽切,震慟于厥心。宜增峻禮秩,式昭懋典。可贈侍中、大將軍、揚州牧、假黃鉞,王如故。並給羽葆鼓吹一部,增班劍為六十人。給溫明秘器,斂以袞服。謚曰靖惠。」宏性寬和篤厚,在州二十餘年,未嘗以吏事按郡
 縣,時稱其長者。



 宏有七子:正仁,正義,正德、正則,正立,正表,正信。世子正仁,為吳興太守,有治能。天監十年,卒,謚曰哀世子。無子,高祖詔以羅平侯正立為世子,由宏意也。宏薨,正立表讓正義為嗣,高祖嘉而許之,改封正立為建安侯,邑千戶。卒,子賁嗣。正義先封平樂侯,正德西豊侯,正則樂山侯,正立羅平侯,正表封山侯,正信武化侯,正德別有傳。



 安成康王秀,字彥達,太祖第七子也。年十二,所生母吳太妃亡,秀母弟始興王憺時年九歲,並以孝聞,居喪,累日不進漿飲,太祖親取粥授之。哀其早孤,命側室陳氏
 并母二子。陳亦無子,有母德,視二子如親生焉。秀既長,美風儀,性方靜,雖左右近侍,非正衣冠不見也,由是親友及家人咸敬焉。齊世,弱冠為著作佐郎,累遷後軍法曹行參軍,太子舍人。



 永元中,長沙宣武王懿入平崔慧景,為尚書令,居端右;弟衡陽王暢為衛尉,掌管籥。東昏日夕逸遊,出入無度。眾頗勸懿因其出,閉門舉兵廢之,懿不聽。帝左右既惡懿勛高,又慮廢立,並間懿,懿亦危之,自是諸王侯咸為之備。及難作,臨川王宏以下諸弟侄各得奔避。方其逃也,皆不出京師,而罕有發覺,惟桂陽王融及禍。



 高祖義師至新林,秀與諸王侯並自拔赴
 軍,高祖以秀為輔國將軍。是時東昏弟晉熙王寶嵩為冠軍將軍、南徐州刺史,鎮京口,長史范岫行府州事,遣使降,且請兵於高祖。以秀為冠軍長史、南東海太守,鎮京口。建康平,仍為使持節、都督南徐、兗二州諸軍事、南徐州刺史,輔國將軍如故。天監元年,進號征虜將軍,封安成郡王,邑二千戶。京口自崔慧景作亂,累被兵革,民戶流散,秀招懷撫納,惠愛大行。仍值年饑,以私財贍百姓,所濟活甚多。二年,以本號征領石頭戍事,加散騎常侍。三年,進號右將軍。五年,加領軍、中書令,給鼓吹一部。



 六年,出為使持節、都督江州諸軍事、平南將軍、江州刺
 史。將發,主者求堅船以為齋舫。秀曰:「吾豈愛財而不愛士。」乃教所由,以牢者給參佐,下者載齋物。既而遭風,齋舫遂破。及至州,聞前刺史取征士陶潛曾孫為里司。秀歎曰:「陶潛之德,豈可不及後世!」即日辟為西曹。時盛夏水泛長,津梁斷絕,外司請依舊僦度,收其價直。秀教曰:「刺史不德,水潦為患,可利之乎!給船而已。」七年,遭慈母陳太妃憂,詔起視事。尋遷都督荊、湘、雍、益、寧、南、北梁、南、北秦州九州諸軍事、平西將軍、荊州刺史。其年,遷號安西將軍。立學校,招隱逸。下教曰:「夫鶉火之禽,不匿影於丹山;昭華之寶,乍耀采於藍田。是以江漢有濯纓之歌,空谷著
 來思之詠,弘風闡道,靡不由茲。處士河東韓懷明、南平韓望、南郡庾承先、河東郭麻,並脫落風塵,高蹈其事。兩韓之孝友純深,庾、郭之形骸枯槁,或橡飯菁羹,惟日不足,或葭牆艾席,樂在其中。昔伯武貞堅,就仕河內,史雲孤劭,屈志陳留。豈曰場苗,實惟攻玉。可加引辟,并遣喻意。既同魏侯致禮之請,庶無辟畺三緘之歎。」



 是歲,魏懸瓠城民反,殺豫州刺史司馬悅,引司州刺史馬仙琕,仙琕簽荊州求應赴。眾咸謂宜待臺報,秀曰:「彼待我而為援,援之宜速,待敕雖舊,非應急也。」即遣兵赴之。先是,巴陵馬營蠻為緣江寇害,後軍司馬高江產以郢州軍
 伐之,不剋,江產死之,蠻遂盛。秀遣防閣文熾率眾討之,燔其林木,絕其蹊逕,蠻失其險,期歲而江路清,於是州境盜賊遂絕。及沮水暴長,頗敗民田,秀以穀二萬斛贍之。使長史蕭琛簡府州貧老單丁吏,一日散遣五百餘人,百姓甚悅。



 十一年,徵為侍中、中衛將軍,領宗正卿、石頭戍事。十三年,復出為使持節、散騎常侍、都督郢、司、霍三州諸軍事、安西將軍、郢州刺史。郢州當塗為劇地,百姓貧,至以婦人供役,其弊如此。秀至鎮,務安之。主者或求召吏。秀曰:「不識救弊之術;此州凋殘,不可擾也。」於是務存約己,省去遊費,百姓安堵,境內晏然。先是夏口常
 為兵衝,露骸積骨於黃鶴樓下,秀祭而埋之。一夜,夢數百人拜謝而去。每冬月,常作襦褲以賜凍者。時司州叛蠻田魯生,弟魯賢、超秀,據蒙籠來降。高祖以魯生為北司州刺史,魯賢北豫州刺史,超秀定州刺史,為北境捍蔽。而魯生、超秀互相讒毀,有去就心,秀撫喻懷納,各得其用,當時賴之。



 十六年,遷使持節、都督雍、梁、南、北秦四州郢州之竟陵司州之隨郡諸軍事、鎮北將軍、寧蠻校尉、雍州刺史,便道之鎮。十七年春,行至竟陵之石梵,薨,時年四十四。高祖聞之,甚痛悼焉。遣皇子南康王績緣道迎候。



 初,秀之西也,郢州民相送出境,聞其疾,百姓商
 賈咸為請命。既薨,四州民裂裳為白帽,哀哭以迎送之。雍州蠻迎秀,聞薨,祭哭而去。喪至京師,高祖使使冊贈侍中、司空,謚曰康。



 秀有容觀,每朝,百僚屬目。性仁恕,喜慍不形於色。左右嘗以石擲殺所養鵠,齋帥請治其罪。秀曰:「吾豈以鳥傷人。」在京師,旦臨公事,廚人進食,誤而覆之,去而登車,竟朝不飯,亦不之誚也。精意術學,搜集經記,招學士平原劉孝標,使撰《類苑》,書未及畢,而已行於世。秀於高祖布衣昆弟,及為君臣,小心畏敬,過於疏賤者,高祖益以此賢之。少偏孤,於始興王嶦尤篤。梁興,嶦久為荊州刺史,自天監初,常以所得俸中分與秀,秀
 稱心受之,亦弗辭多也。昆弟之睦,時議歸之。故吏夏侯稟等表立墓碑,詔許焉。當世高才遊王門者,東海王僧孺、吳郡陸倕、彭城劉孝綽、河東裴子野,各製其文,古未之有也。世子機嗣。



 機字智通,天監二年,除安成國世子。六年,為寧遠將軍、會稽太守。還為給事中。普通元年,襲封安成郡王,其年為太子洗馬,遷中書侍郎。二年,遷明威將軍、丹陽尹。三年,遷持節、督湘、衡、桂三州諸軍事、寧遠將軍、湘州刺史。大通二年,薨于州,時年三十。機美姿容,善吐納。家既多書,博學彊記;然而好弄,尚力,遠士子,近小人。為州專意聚斂,無治績,頻被案劾。及將葬,有司
 請謚,高祖詔曰:「王好內怠政,可謚曰煬。」所著詩賦數千言,世祖集而序之。子操嗣。



 南浦侯推,字智進,機次弟也。少清敏,好屬文,深為太宗所賞。普通六年,以王子例封。歷寧遠將軍、淮南太守。遷輕車將軍、晉陵太守,給事中,太子洗馬,秘書丞。出為戎昭將軍、吳郡太守。所臨必赤地大旱,吳人號「旱母」焉。侯景之亂,守東府城,賊設樓車,盡銳攻之,推隨方抗拒,頻擊挫之。至夕,東北樓主許鬱華啟關延賊,城遂陷,推握節死之。



 南平元襄王偉,字文達,太祖第八子也。幼清警好學。齊世,起家晉安鎮北法曹行參軍府,遷驃騎,轉外兵。高祖
 為雍州,慮天下將亂,求迎偉及始興王心詹來襄陽。俄聞已入沔,高祖欣然謂佐吏曰:「吾無憂矣。」義師起,南康王承制,板為冠軍將軍,留行雍州開府事。義師發後,州內儲備及人皆虛竭。魏興太守裴師仁、齊興太守顏僧都並據郡不受命,舉兵將襲雍州,偉與始興王嶦遣兵於始平郡待師仁等,要擊大破之,州境以安。



 高祖既剋郢、魯,下尋陽,圍建業,而巴東太守蕭慧訓子璝及巴西太守魯休烈起兵逼荊州,屯軍上明,連破荊州。鎮軍蕭穎胄遣將劉孝慶等距之,反為璝所敗,穎胄憂憤暴疾卒,西朝兇懼。尚書僕射夏侯詳議徵兵雍州,偉乃割州府
 將吏,配始興王嶦往赴之。嶦既至,璝等皆降。和帝詔以偉為使持節、都督雍、梁、南、北秦四州郢州之竟陵司州之隨郡諸軍事、寧蠻校尉、雍州刺史,將軍如故。尋加侍中,進號鎮北將軍。天監元年,加散騎常侍,進督荊、寧二州,餘如故。封建安郡王,食邑二千戶,給鼓吹一部。四年,徙都督南徐州諸軍事、南徐州刺史,使持節、常侍、將軍如故。五年,至都,改為撫軍將軍、丹陽尹,常侍如故。六年,遷使持節、都督揚、南徐二州諸軍事、右軍將軍、揚州刺史。未拜,進號中權將軍。七年,以疾表解州,改侍中、中撫軍,知司徒事。九年,遷護軍、石頭戍軍事,侍中、將軍、鼓吹
 如故。其年,出為使持節、散騎常侍、都督江州諸軍事、鎮南將軍、江州刺史,鼓吹如故。十一年,以本號加開府儀同三司。其年,復以疾陳解。十二年,徵為撫軍將軍,儀同、常侍如故,以疾不拜。十三年,改為左光祿大夫。加親信四十人,歲給米萬斛,布絹五千匹,藥直二百四十萬,廚供月二十萬,并二衛兩營雜役二百人,倍先。置防閣白直左右職局一百人。偉末年疾浸劇,不復出籓,故俸秩加焉。



 十五年,所生母陳太妃寢疾,偉及臨川王宏侍疾,並衣不解帶。及太妃薨,毀頓過禮,水漿不入口累日,高祖每臨幸譬抑之。偉雖奉詔,而毀瘠殆不勝喪。



 十七年,
 高祖以建安土瘠,改封南平郡王,邑戶如故。遷侍中、左光祿大夫、開府儀同三司。普通四年,增邑一千戶。五年,進號鎮衛大將軍。中大通元年,以本官領太子太傅。四年,遷中書令、大司馬。五年,薨,時年五十八。詔斂以袞冕,給東園秘器。又詔曰:「旌德紀功,前王令典;慎終追遠,列代通規。故侍中、中書令、大司馬南平王偉,器宇宏曠,鑒識弘簡。爰在弱齡,清風載穆,翼佐草昧,勳高樊、沔,契闊艱難,劬勞任寄。及贊務論道,弘茲袞職。奄焉薨逝,朕用震慟於厥心。宜隆寵命,式昭茂典。可贈侍中、太宰,王如故。給羽葆鼓吹一部,并班劍四十人。謚曰元襄。」



 偉少好
 學,篤誠通恕,趨賢重士,常如不及。由是四方遊士,當世知名者,莫不畢至。齊世,青溪宮改為芳林苑,天監初,賜偉為第,偉又加穿築,增植嘉樹珍果,窮極雕麗,每與賓客遊其中,命從事中郎蕭子範為之記。梁世籓邸之盛,無以過焉。而性多恩惠,尤愍窮乏。常遣腹心左右,歷訪閭里人士,其有貧困吉凶不舉者,即遣贍恤之。太原王曼穎卒,家貧無以殯斂,友人江革往哭之,其妻兒對革號訴。革曰:「建安王當知,必為營理。」言未訖而偉使至,給其喪事,得周濟焉。每祁寒積雪,則遣人載樵米,隨乏絕者即賦給之。晚年崇信佛理,尤精玄學,著《二旨義》,別為
 新通。又製《性情》、《幾神》等論其義,僧寵及周捨、殷鈞、陸倕並名精解,而不能屈。



 偉四子:恪,恭,虔,祗。世子恪嗣。



 恭字敬範。天監八年,封衡山縣侯,以元襄功,加邑至千戶。初,樂山侯正則有罪,敕讓諸王,獨謂元襄曰:「汝兒非直無過,並有義方。」



 恭起家給事中,遷太子洗馬。出為督齊安等十一郡事、寧遠將軍、西陽、武昌二郡太守。徵為秘書丞,遷中書郎,監丹陽尹,行徐、南徐州事,轉衡州刺史,母憂去職。尋起為雲麾將軍、湘州刺史。



 恭善解吏事,所在見稱。而性尚華侈,廣營第宅,重齋步櫩,模寫宮殿。尤好賓友,酣宴終辰,座客滿筵,言談不倦。時世祖居籓,頗事
 聲譽,勤心著述,卮酒未嘗妄進。恭每從容謂人曰:「下官歷觀世人,多有不好歡樂,乃仰眠床上,看屋梁而著書,千秋萬歲,誰傳此者。勞神苦思,竟不成名,豈如臨清風,對朗月,登山泛水,肆意酣歌也。」尋以雍州蠻文道拘引魏寇,詔恭赴援,仍除持節、仁威將軍、寧蠻校尉、雍州刺史,便道之鎮。太宗少與恭遊,特被賞狎,至是手令曰:「彼士流骯臟,有關輔餘風,黔首扞格,但知重劍輕死。降胡惟尚貪婪,邊蠻不知敬讓,懷抱不可皁白,法律無所用施。願充實邊戍,無數遷徙,諜候惟遠,箱庾惟積,長以控短,靜以制躁。早蒙愛念,敢布腹心。」恭至州,治果有聲績,
 百姓陳奏,乞於城南立碑頌德,詔許焉。



 先高祖以雍為邊鎮,運數州之粟,以實儲倉,恭後多取官米,贍給私宅,為荊州刺史廬陵王所啟,由是免官削爵,數年竟不敘用。侯景亂,卒於城中,時年五十二。詔特復本封。世祖追贈侍中、左衛將軍。謚曰僖。



 世子靜,字安仁,有美名,號為宗室後進。有文才,而篤志好學,既內足於財,多聚經史,散書滿席,手自讎校。何敬容欲以女妻之,靜忌其太盛,距而不納,時論服焉。歷官太子舍人、東宮領直。遷丹陽尹丞,給事黃門侍郎,深為太宗所愛賞。太清三年,卒,贈侍中。



 鄱陽忠烈王恢,字弘達,太祖第九子也。幼聰穎,年七歲,能通《孝經》、《論語》義,發擿無所遺。既長,美風表,涉獵史籍。齊隆昌中,明帝作相,內外多虞,明帝就長沙宣武王懿求諸弟有可委以腹心者,宣武言恢焉。明帝以恢為寧遠將軍,甲仗百人衛東府,且引為驃騎法曹行參軍。明帝即位,東宮建,為太子舍人,累遷北中郎外兵參軍,前軍主簿。宣武之難,逃在京師。



 高祖義兵至,恢於新林奉迎,以為輔國將軍。時三吳多亂,高祖命出頓破崗。建康平,還為冠軍將軍、右衛將軍。天監元年,為侍中、前將軍,領石頭戍軍事,封鄱陽郡王,食邑二千戶。二年,出為使
 持節、都督南徐州諸軍事、征虜將軍、南徐州刺史。四年,改授都督郢、司二州諸軍事、後將軍、郢州刺史,持節如故。義兵初,郢城內疾疫死者甚多,不及藏殯,及恢下車,遽命埋掩。又遣四使巡行州部,境內大治。七年,進號雲麾將軍,進督霍州。八年,復進號平西將軍。十年,徵為侍中、護軍將軍、石頭戍軍事,領宗正卿。十一年,出為使持節、都督荊、湘、雍、益、寧、南、北梁、南、北秦九州諸軍事、平西將軍、荊州刺史,給鼓吹一部。十三年,遷散騎常侍、都督益、寧、南、北秦、沙七州諸軍事、鎮西將軍、益州刺史,使持節如故,便道之鎮。成都去新城五百里,陸路往來,悉訂
 私馬,百姓患焉,累政不能改。恢乃市馬千匹,以付所訂之家,資其騎乘,有用則以次發之,百姓賴焉。十七年,徵為侍中、安前將軍、領軍將軍。十八年,出為使持節、散騎常侍、都督荊、湘、雍、梁、益、寧、南、北秦八州諸軍事、征西將軍、開府儀同三司、荊州刺史。普通五年,進號驃騎大將軍。七年九月,薨于州,時年五十一。詔曰:「故使持節、散騎常侍、都督荊、湘、雍、梁、益、寧、南、北秦八州諸軍事、驃騎大將軍、開府儀同三司、荊州刺史鄱陽王恢,風度開朗,器情凝質。爰在弱歲,美譽克宣,洎於從政,嘉猷載緝。方入正論道,弘燮台階,奄焉薨逝,朕用傷慟于厥心。宜隆寵
 命,以申朝典。可贈侍中、司徒,王如故。並給班劍二十人。謚曰忠烈。」遣中書舍人劉顯護喪事。



 恢有孝性,初鎮蜀,所生費太妃猶停都,後於都下不豫,恢未之知,一夜忽夢還侍疾,既覺憂遑,便廢寢食。俄而都信至,太妃已瘳。後又目有疾,久廢視瞻,有北渡道人慧龍得治眼術,恢請之。既至,空中忽見聖僧,及慧龍下鍼,豁然開朗,咸謂精誠所致。



 恢性通恕,輕財好施,凡歷四州,所得俸祿隨而散之。在荊州,常從容問賓僚曰:「中山好酒,趙王好吏,二者孰愈?」眾未有對者。顧謂長史蕭琛曰:「漢時王侯,籓屏而已,視事親民,自有其職。中山聽樂,可得任性;彭祖
 代吏,近於侵官。今之王侯,不守籓國,當佐天子臨民,清白其優乎!」坐賓咸服。世子範嗣。



 範字世儀,溫和有器識。起家太子洗馬、秘書郎,歷黃門郎,遷衛尉卿。每夜自巡警,高祖嘉其勞苦。出為益州刺史,開通劍道,剋復華陽,增邑一千戶,加鼓吹。徵為領軍將軍、侍中。



 範雖無學術,而以籌略自命。愛奇玩古,招集文才,率意題章,亦時有奇致。復出為使持節、都督雍、梁、東益、南、北秦五州諸軍事、鎮北將軍、雍州刺史。範作牧蒞民,甚得時譽;撫循將士,盡獲歡心。太清元年,大舉北伐,以範為使持節、征北大將軍、總督漢北征討諸軍事,進伐穰城。尋遷安北將
 軍、南豫州刺史。侯景敗於渦陽,退保壽陽,乃改範為合州刺史,鎮合肥。時景已蓄姦謀,不臣將露,範屢啟言之,朱異每抑而不奏。及景圍京邑,範遣世子嗣與裴之高等入援,遷開府儀同三司,進號征北將軍。京城不守,範乃棄合肥,出東關,請兵于魏,遣二子為質。魏人據合肥,竟不出師助範,範進退無計,乃溯流西上,軍于樅陽,遣信告尋陽王。尋陽要還九江,欲共治兵西上,範得書大喜,乃引軍至湓城,以晉熙為晉州,遣子嗣為刺史。江州郡縣,輒更改易,尋陽政令所行,惟存一郡,時論以此少之。既商旅不通,信使距絕,範數萬之眾,皆無復食,人多
 餓死。範恚,發背薨,時年五十二。



 世子嗣,字長胤。容貌豊偉,腰帶十圍。性驍果有膽略,倜儻不護細行,而能傾身養士,皆得其死力。範之薨也,嗣猶據晉熙,城中食盡,士乏絕,景遣任約來攻,嗣躬擐甲胄,出壘距之。時賊勢方盛,咸勸且止。嗣按劍叱之曰:「今之戰,何有退乎?此蕭嗣效命死節之秋也。」遂中流矢,卒於陣。



 始興忠武王嶦,字僧達,太祖第十一子也。數歲,所生母吳太妃卒,嶦哀感傍人。齊世,弱冠為西中郎法曹行參軍,遷外兵參軍。義師起,南康王承制,以嶦為冠軍將軍、西中郎諮議參軍,遷相國從事中郎,與南平王偉留守。



 和帝立,以嶦為給事黃門侍郎。時巴東太守蕭慧訓子璝等及巴西太守魯休烈舉兵逼荊州,屯軍上明,鎮軍將軍蕭穎胄暴疾卒,西朝甚懼,尚書僕射夏侯祥議徵兵雍州,南平王偉遣嶦赴之。嶦以書喻璝等,旬日皆請降。是冬,高祖平建業。明年春,和帝將發江陵,詔以嶦為使持節、都督荊、湘、益、寧、南、北秦六州諸軍事、平西將軍、荊州刺史,未拜。天監元年,加安西將軍,都督、刺史如故。封始興郡王,食邑二千戶。時軍旅之後,公私空乏,嶦厲精為治,廣闢屯田,減省力役,存問兵死之家,供其窮困,民甚安之。嶦自以少年始居重任,思欲開導物情。乃謂
 佐吏曰:「政之不臧,士君子所宜共惜。言可用,用之可也;如不用,於我何傷?吾開懷矣,爾其無吝。」於是小人知恩,而君子盡意。民辭訟者,皆立前待符教,決於俄頃。曹無留事,下無滯獄,民益悅焉。三年,詔加鼓吹一部。



 六年,州大水,江溢堤壞,嶦親率府將吏,冒雨賦丈尺築治之。雨甚水壯,眾皆恐,或請嶦避焉。嶦曰:「王尊尚欲身塞河堤,我獨何心以免。」乃刑白馬祭江神。俄而水退堤立。邴州在南岸,數百家見水長驚走,登屋緣樹,憺募人救之,一口賞一萬,估客數十人應募救焉,州民乃以免。又分遣行諸郡,遭水死者給棺槥,失田者與糧種。是歲,嘉禾生
 於州界,吏民歸美,嶦謙讓不受。



 七年,慈母陳太妃薨,水漿不入口六日,居喪過禮,高祖優詔勉之,使攝州任。是冬,詔徵以本號還朝。民為之歌曰:「始興王,民之爹。赴人急,如水火。何時復來哺乳我?」八年,為平北將軍、護軍將軍、領石頭戍事。尋遷中軍將軍、中書令,俄領衛尉卿。嶦性勞謙,降意接士,常與賓客連榻而坐,時論稱之。是秋,出為使持節、散騎常侍、都督南、北兗、徐、青、冀五州諸軍事、鎮北將軍、南兗州刺史。



 九年春,遷都督益、寧、南梁、南、北秦、沙六州諸軍事、鎮西將軍、益州刺史。開立學校,勸課就業,遣子映親受經焉,由是多向方者。時魏襲
 巴南,西圍南安,南安太守垣季珪堅壁固守,嶦遣軍救之,魏人退走,所收器械甚眾。十四年,遷都督荊、湘、雍、寧、南梁、南、北秦七州諸軍事、鎮右將軍、荊州刺史。同母兄安成王秀將之雍州,薨於道。嶦聞喪,自投於地,席稿哭泣,不飲不食者數日,傾財產賻送,部伍小大皆取足焉。天下稱其悌。



 十八年,徵為侍中、中撫將軍、開府儀同三司、領軍將軍。普通三年十一月,薨,時年四十五。追贈侍中、司徒、驃騎將軍。給班劍三十人,羽葆鼓吹一部。冊曰:「咨故侍中、司徒、驃騎將軍始興王:夫忠為令德,武謂止戈,于以用之,載在前志。王有佐命之元勳,利民之厚德,
 契闊二紀,始終不渝,是用方軌往賢,稽擇故訓,鴻名美義,允臻其極。今遣兼大鴻臚程爽,謚曰忠武。魂而有靈,歆茲顯號。嗚呼哀哉!」



 嶦未薨前,夢改封中山王,策授如他日,意頗惡之,數旬而卒。世子亮嗣。



 史臣曰:自昔王者創業,廣植親親,割裂州國,封建子弟。是以大旗少帛,崇於魯、衛,盤石凝脂,樹斯梁、楚。高祖遠遵前軌,籓屏懿親。至於安成、南平,鄱陽、始興,俱以名跡著,蓋亦漢之間、平矣。



\end{pinyinscope}