\article{卷第九列傳第三 王茂 曹景宗 柳慶遠}

\begin{pinyinscope}

 王茂,字休遠,太原祁人也。祖深,北中郎司馬。父天生,宋末為列將,於石頭克司徒袁粲,以勳至巴西、梓潼二郡太守,上黃縣男。茂年數歲,為大父深所異,常謂親識曰:「此吾家之千里駒,成門戶者必此兒也。」及長,好讀兵書,駁略究其大旨。性沈隱,不妄交遊,身長八尺,潔白美容觀。齊武帝布衣時,見之歎曰:「王茂年少,堂堂如此,必為
 公輔之器。」宋昇明末,起家奉朝請,歷後軍行參軍,司空騎兵,太尉中兵參軍。魏將李烏奴寇漢中,茂受詔西討。魏軍退,還為鎮南司馬,帶臨湘令。入為越騎校尉。魏寇兗州,茂時以寧朔將軍長史鎮援北境,入為前軍將軍江夏王司馬。又遷寧朔將軍、江夏內史。建武初,魏圍司州,茂以郢州之師救焉。高祖率眾先登賢首山,魏將王肅、劉昶來戰,茂從高祖拒之,大破肅等。魏軍退,茂還郢,仍遷輔國長史、襄陽太守。



 高祖義師起,茂私與張弘策勸高祖迎和帝,高祖以為不然,語在《高祖紀》。高祖發雍部,每遣茂為前驅。師次郢城,茂進平加湖,破光子衿、吳
 子陽等,斬馘萬計,還獻捷於漢川。郢、魯既平,從高祖東下,復為軍鋒。師次秣陵,東昏遣大將王珍國,盛兵朱雀門,眾號二十萬,度航請戰。茂與曹景宗等會擊,大破之。縱兵追奔,積屍與航欄等,其赴淮死者,不可勝算。長驅至宣陽門。建康城平,以茂為護軍將軍,俄遷侍中、領軍將軍。群盜之燒神虎門也,茂率所領到東掖門應赴,為盜所射,茂躍馬而進,群盜反走。茂以不能式遏姦盜,自表解職,優詔不許。加鎮軍將軍,封望蔡縣公,邑二千三百戶。



 是歲,江州刺史陳伯之舉兵叛,茂出為使持節、散騎常侍、都督江州諸軍事、征南將軍、江州刺史,給鼓吹
 一部,南討伯之。伯之奔於魏。時九江新罹軍寇,民思反業,茂務農省役,百姓安之。四年,魏侵漢中,茂受詔西討,魏乃班師。六年,遷尚書右僕射,常侍如故。固辭不拜,改授侍中、中衛將軍,領太子詹事。七年,拜車騎將軍,太子詹事如故。八年,以本號開府儀同三司、丹陽尹,侍中如故。時天下無事,高祖方信仗文雅,茂心頗怏怏,侍宴醉後,每見言色,高祖常宥而不之責也。十一年,進位司空,侍中、尹如故。茂辭京尹,改領中權將軍。



 茂性寬厚,居官雖無譽,亦為吏民所安。居處方正,在一室衣冠儼然,雖僕妾莫見其惰容。姿表瑰麗,須眉如畫。出入朝會,每為
 眾所瞻望。明年,出為使持節、散騎常侍、驃騎將軍、開府同三司之儀、都督江州諸軍事、江州刺史。視事三年,薨于州,時年六十。高祖甚悼惜之,賻錢三十萬,布三百匹。詔曰:「旌德紀勳,哲王令軌;念終追遠,前典明誥。故使持節、散騎常侍、驃騎將軍、開府儀同三司、江州刺史茂,識度淹廣,器宇凝正。爰初草昧,盡誠宣力,綢繆休戚,契闊屯夷。方賴謀猷,永隆朝寄;奄至薨殞,朕用慟于厥心。宜增禮數,式昭盛烈。可贈侍中、太尉,加班劍二十人,鼓吹一部。謚曰忠烈。」



 初,茂以元勳,高祖賜以鐘磬之樂。茂在江州,夢鐘磬在格,無故自墮,心惡之。及覺,命奏樂。既成
 列,鐘磬在格,果無故編皆絕,墮地。茂謂長史江詮曰:「此樂,天子所以惠勞臣也。樂既極矣,能無憂乎!」俄而病,少日卒。



 子貞秀嗣,以居喪無禮,為有司奏,徙越州。後有詔留廣州,乃潛結仁威府中兵參軍杜景,欲襲州城,刺史蕭昂討之。景,魏降人,與貞秀同戮。



 曹景宗,字子震,新野人也。父欣之,為宋將,位至征虜將軍、徐州刺史。景宗幼善騎射,好畋獵。常與少年數十人澤中逐麞鹿,每眾騎赴鹿,鹿馬相亂,景宗於眾中射之,人皆懼中馬足,鹿應弦輒斃,以此為樂。未弱冠,欣之於新野遣出州,以匹馬將數人,於中路卒逢蠻賊數百圍
 之。景宗帶百餘箭,乃馳騎四射,每箭殺一蠻,蠻遂散走,因是以膽勇知名。頗愛史書,每讀《穰苴》、《樂毅傳》,輒放卷歎息曰:「丈夫當如是!」辟西曹不就。宋元徽中,隨父出京師,為奉朝請、員外,遷尚書左民郎。尋以父憂去職,還鄉里。服闋,刺史蕭赤斧板為冠軍中兵參軍,領天水太守。



 時建元初,蠻寇群動,景宗東西討擊,多所擒破。齊鄱陽王鏘為雍州,復以為征虜中兵參軍,帶馮翊太守督峴南諸軍事,除屯騎校尉。少與州里張道門厚善。道門,齊車騎將軍敬兒少子也,為武陵太守。敬兒誅,道門於郡伏法,親屬故吏莫敢收,景宗自襄陽遣人船
 到武陵,收其屍骸,迎還殯葬,鄉里以此義之。



 建武二年,魏主託跋宏寇赭陽,景宗為偏將,每衝堅陷陣,輒有斬獲,以勳除遊擊將軍。四年,太尉陳顯達督眾軍北圍馬圈,景宗從之,以甲士二千設伏,破魏援托跋英四萬人。及剋馬圈,顯達論功,以景宗為後,景宗退無怨言。魏主率眾大至,顯達宵奔,景宗導入山道,故顯達父子獲全。五年,高祖為雍州刺史,景宗深自結附,數請高祖臨其宅。時天下方亂,高祖亦厚加意焉。永元初,表為冠軍將軍、竟陵太守。及義師起,景宗聚眾,遣親人杜思沖勸先迎南康王於襄陽即帝位,然後出師,為萬全計。高祖不
 從,語在《高祖紀》。高祖至竟陵,以景宗與冠軍將軍王茂濟江,圍郢城,自二月至於七月,城乃降。復帥眾前驅至南州,領馬步軍取建康。道次江寧,東昏將李居士以重兵屯新亭,是日選精騎一千至江寧行頓,景宗始至,安營未立;且師行日久,器甲穿弊,居士望而輕之,因鼓噪前薄景宗。景宗被甲馳戰,短兵裁接,居士棄甲奔走,景宗皆獲之,因鼓而前,徑至皁莢橋築壘。景宗又與王茂、呂僧珍掎角,破王珍國於大航。茂衝其中堅,應時而陷,景宗縱兵乘之。景宗軍士皆桀黠無賴,御道左右,莫非富室,抄掠財物,略奪子女,景宗不能禁。及高祖入頓新
 城,嚴申號令,然後稍息。復與眾軍長圍六門。城平,拜散騎常侍、右衛將軍,封湘西縣侯,食邑一千六百戶。仍遷持節、都督郢、司二州諸軍事、左將軍、郢州刺史。天監元年,進號平西將軍,改封竟陵縣侯。



 景宗在州,鬻貨聚斂。於城南起宅,長堤以東,夏口以北,開街列門,東西數里,而部曲殘橫,民頗厭之。二年十月,魏寇司州,圍刺史蔡道恭。時魏攻日苦,城中負板而汲,景宗望門不出,但耀軍遊獵而已。及司州城陷,為御史中丞任昉所奏。高祖以功臣寢而不治,徵為護軍。既至,復拜散騎常侍、右衛將軍。



 五年,魏托跋英寇鐘離,圍徐州刺史昌義之。高祖
 詔景宗督眾軍援義之,豫州刺史韋睿亦預焉,而受景宗節度。詔景宗頓道人洲,待眾軍齊集俱進。景宗固啟,求先據邵陽洲尾,高祖不聽。景宗欲專其功,乃違詔而進,值暴風卒起,頗有淹溺,復還守先頓。高祖聞之,曰:「此所以破賊也。景宗不進,蓋天意乎!若孤軍獨往,城不時立,必見狼狽。今得待眾軍同進,始大捷矣。」及韋睿至,與景宗進頓邵陽洲,立壘去魏城百餘步。魏連戰不能卻,殺傷者十二三,自是魏軍不敢逼。景宗等器甲精新,軍儀甚盛,魏人望之奪氣。魏大將楊大眼對橋北岸立城,以通糧運,每牧人過岸伐芻槁,皆為大眼所略。景宗乃
 募勇敢士千餘人,徑渡大眼城南數里築壘,親自舉築。大眼率眾來攻,景宗與戰破之,因得壘成。使別將趙草守之,因謂為趙草城,是後恣芻牧焉。大眼時遣抄掠,輒反為趙草所獲。先是,高祖詔景宗等逆裝高艦,使與魏橋等,為火攻計。令景宗與睿各攻一橋,睿攻其南,景宗攻其北。六年三月,春水生,淮水暴長六七尺。睿遣所督將馮道根、李文釗、裴邃、韋寂等乘艦登岸,擊魏洲上軍盡殪。景宗因使眾軍皆鼓噪亂登諸城,呼聲震天地,大眼於西岸燒營,英自東岸棄城走。諸壘相次土崩,悉棄其器甲,爭投水死,淮水為之不流。景宗令軍主馬廣,躡
 大眼至濊水上,四十餘里,伏屍相枕。義之出逐英至洛口,英以匹馬入梁城。緣淮百餘里,屍骸枕藉,生擒五萬餘人,收其軍糧器械,積如山岳,牛馬驢騾,不可勝計。景宗乃搜軍所得生口萬餘人,馬千匹,遣獻捷,高祖詔還本軍,景宗振旅凱入,增封四百,并前為二千戶,進爵為公。詔拜侍中、領軍將軍,給鼓吹一部。



 景宗為人自恃尚勝,每作書,字有不解,不以問人,皆以意造焉。雖公卿無所推揖;惟韋睿年長,且州里勝流,特相敬重,同宴御筵,亦曲躬謙遜,高祖以此嘉之。景宗好內,妓妾至數百,窮極錦繡。性躁動,不能沈默,出行常欲褰車帷幔,左右輒
 諫以位望隆重,人所具瞻,不宜然。景宗謂所親曰:「我昔在鄉里,騎快馬如龍,與年少輩數十騎,拓弓弦作霹靂聲,箭如餓鴟叫。平澤中逐麞,數肋射之,渴飲其血,飢食其肉,甜如甘露漿。覺耳後風生,鼻頭出火,此樂使人忘死,不知老之將至。今來揚州作貴人,動轉不得,路行開車幔,小人輒言不可。閉置車中,如三日新婦。遭此邑邑,使人無氣。」為人嗜酒好樂,臘月於宅中,使作野虖逐除,遍往人家乞酒食。本以為戲,而部下多剽輕,因弄人婦女,奪人財貨。高祖頗知之,景宗乃止。高祖數宴見功臣,共道故舊,景宗醉後謬忘,或誤稱下官,高祖故縱之,以為
 笑樂。



 七年,遷侍中、中衛將軍、江州刺史。赴任卒於道,時年五十二。詔賻錢二十萬,布三百匹,追贈征北將軍、雍州刺史、開府儀同三司。謚曰壯。子皎嗣。



 柳慶遠,字文和,河東解人也。伯父元景,宋太尉。慶遠起家郢州主簿,齊初為尚書都官郎、大司馬中兵參軍、建武將軍、魏興太守。郡遭暴水,流漂居民,吏請徙民祀城。慶遠曰:「天降雨水,豈城之所知。吾聞江河長不過三日,斯亦何慮。」命築土而已。俄而水過,百姓服之。入為長水校尉,出為平北錄事參軍、襄陽令。



 高祖之臨雍州,問京兆人杜惲求州綱,惲舉慶遠。高祖曰:「文和吾已知之,所
 問未知者耳。」因辟別駕從事史。齊方多難,慶遠謂所親曰:「方今天下將亂,英雄必起,庇民定霸,其吾君乎?」因盡誠協贊。及義兵起,慶遠常居帷幄為謀主。



 中興元年,西臺選為黃門郎,遷冠軍將軍、征東長史。從軍東下,身先士卒。高祖行營壘,見慶遠頓舍嚴整,每歎曰:「人人若是,吾又何憂。」建康城平,入為侍中,領前軍將軍,帶淮陵、齊昌二郡太守。城內嘗夜失火,禁中驚懼,高祖時居宮中,悉斂諸鑰,問「柳侍中何在」。慶遠至,悉付之。其見任如此。



 霸府建,以為太尉從事中郎。高祖受禪,遷散騎常侍、右衛將軍,加征虜將軍,封重安侯,食邑千戶。母憂去職,以
 本官起之,固辭不拜。天監二年,遷中領軍,改封雲杜侯。四年,出為使持節、都督雍、梁、南、北秦四州諸軍事、征虜將軍、寧蠻校尉、雍州刺史。高祖餞於新亭,謂曰:「卿衣錦還鄉,朕無西顧之憂矣。」



 七年,徵為護軍將軍,領太子庶子。未赴職,仍遷通直散騎常侍、右衛將軍,領右驍騎將軍。至京都,值魏宿預城請降,受詔為援,於是假節守淮陰。魏軍退。八年,還京師,遷散騎常侍、太子詹事、雍州大中正。十年,遷侍中、領軍將軍,給扶,并鼓吹一部。十二年,遷安北將軍、寧蠻校尉、雍州刺史。慶遠重為本州,頗歷清節,士庶懷之。明年春,卒,時年五十七。詔曰:「念往篤終,
 前王令則;式隆寵數,列代恒規。使持節、都督雍、梁、南、北秦四州郢州之竟陵司州之隨郡諸軍事、安北將軍、寧蠻校尉、雍州刺史、雲杜縣開國侯柳慶遠,器識淹曠,思懷通雅。爰初草昧,預屬經綸;遠自升平,契闊禁旅。重牧西籓,方弘治道,奄至殞喪,傷慟于懷。宜追榮命,以彰茂勳。可贈侍中、中軍將軍、開府儀同三司,鼓吹、侯如故。謚曰忠惠。賻錢二十萬,布二百匹。」及喪還京師,高祖出臨哭。子津嗣。



 初,慶遠從父兄衛將軍世隆嘗謂慶遠曰:「吾昔夢太尉以褥席見賜,吾遂亞台司,適又夢以吾褥席與汝,汝必光我公族。」至是,慶遠亦繼世隆焉。



 陳吏部尚書姚察曰:王茂、曹景宗、柳慶遠雖世為將家,然未顯奇節。梁興,因日月末光,以成所志,配跡方、邵,勒勛鐘鼎,偉哉!昔漢光武全愛功臣,不過朝請、特進,寇、鄧、耿、賈咸不盡其器力。茂等迭據方岳,位終上將,君臣之際,邁於前代矣。



\end{pinyinscope}