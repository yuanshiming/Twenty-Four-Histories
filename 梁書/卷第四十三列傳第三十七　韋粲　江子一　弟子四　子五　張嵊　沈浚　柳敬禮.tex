\article{卷第四十三列傳第三十七 韋粲 江子一 弟子四 子五 張嵊 沈浚 柳敬禮}

\begin{pinyinscope}

 韋粲,字長蒨,車騎將軍睿之孫,北徐州刺史放之子也。有父風,好學仗氣,身長八尺,容貌甚偉。初為雲麾晉安王行參軍,俄署法曹,遷外兵參軍,兼中兵。時潁川庾仲容、吳郡張率,前輩知名,與粲同府,並忘年交好。及王遷鎮雍州,隨轉記室,兼中兵如故。王立為皇太子,粲遷步
 兵校尉,入為東宮領直,丁父憂去職。尋起為招遠將軍,復為領直。服闋,襲爵永昌縣侯,除安西湘東王諮議,累遷太子僕、左衛率,領直並如故。粲以舊恩,任寄綢密,雖居職屢徙,常留宿衛,頗擅威名,誕倨,不為時輩所平。右衛硃異嘗於酒席厲色謂粲曰:「卿何得已作領軍面向人!」



 中大同十一年,遷通直散騎常侍,未拜,出為持節、督衡州諸軍事、安遠將軍、衡州刺史。皇太子出餞新亭,執粲手曰:「與卿不為久別。」太清元年,粲至州。無幾,便表解職。二年,徵為散騎常侍。粲還至廬陵,聞侯景作逆,便簡閱部下,得精卒五千,馬百匹,倍道赴援。至豫章,奉命報云「
 賊已出橫江」,粲即就內史劉孝儀共謀之。孝儀曰:「必期如此,當有別敕。豈可輕信單使,妄相驚動,或恐不然。」時孝儀置酒,粲怒,以杯抵地曰:「賊已渡江,便逼宮闕,水陸俱斷,何暇有報;假令無敕,豈得自安?韋粲今日何情飲酒!」即馳馬出,部分將發,會江州刺史當陽公大心遣使要粲,粲乃馳往見大心曰:「上游蕃鎮,江州去京最近,殿下情計,實宜在前;但中流任重,當須應接,不可闕鎮。今直且張聲勢,移鎮湓城,遣偏將賜隨,於事便足。」大心然之,遣中兵柳昕帥兵二千人隨粲。粲悉留家累於江州,以輕舸就路。至南州,粲外弟司州刺史柳仲禮亦帥步
 騎萬餘人至橫江,粲即送糧仗贍給之,並散私金帛以賞其戰士。



 先是,安北將軍鄱陽王範亦自合肥遣西豫州刺史裴之高與其長子嗣,帥江西之眾赴京師,屯於張公洲,待上流諸軍至。是時,之高遣船渡仲禮,與合軍進屯王遊苑。粲建議推仲禮為大都督,報下流眾軍。裴之高自以年位恥居其下,乃云:「柳節下是州將,何須我復鞭板?」累日不決。粲乃抗言於眾曰:「今者同赴國難,義在除賊,所以推柳司州者,政以久捍邊疆,先為侯景所憚;且士馬精銳,無出其前。若論位次,柳在粲下;語其年齒,亦少於粲,直以社稷之計,不得復論。今日形勢,貴在
 將和;若人心不同,大事去矣。裴公朝之舊齒,年德已隆,豈應復挾私情,以沮大計。粲請為諸君解釋之。」乃單舸至之高營,切讓之曰:「前諸將之議,豫州意所未同,即二宮危逼,猾寇滔天,臣子當戮力同心,豈可自相矛盾!豫州必欲立異,鋒鏑便有所歸。」之高垂泣曰:「吾荷國恩榮,自應帥先士卒,顧恨衰老,不能效命,企望柳使君共平凶逆,謂眾議已從,無俟老夫耳。若必有疑,當剖心相示。」於是諸將定議,仲禮方得進軍。



 次新亭,賊列陣於中興寺,相持至晚,各解歸。是夜,仲禮入粲營,部分眾軍,旦日將戰,諸將各有據守,令粲頓青塘。青塘當石頭中路,粲
 慮柵壘未立,賊必爭之,頗以為憚,謂仲禮曰:「下官才非禦侮,直欲以身殉國。節下善量其宜,不可致有虧喪。」仲禮曰:「青塘立柵,迫近淮渚,欲以糧儲船乘盡就泊之,此是大事,非兄不可。若疑兵少,當更差軍相助。」乃使直閣將軍劉叔胤師助粲,帥所部水陸俱進。時值昏霧,軍人迷失道,比及青塘,夜已過半,壘柵至曉未合。景登禪靈寺門閣,望粲營未立,便率銳卒來攻。軍副王長茂勸據柵待之,粲不從,令軍主鄭逸逆擊之,命劉叔胤以水軍截其後。叔胤畏懦不敢進,逸遂敗。賊乘勝入營,左右牽粲避賊,粲不動,猶叱子弟力戰,兵死略盡,遂見害,時年
 五十四。粲子尼及三弟助、警、構、從弟昂皆戰死,親戚死者數百人。賊傳粲首闕下,以示城內,太宗聞之流涕曰:「社稷所寄,惟在韋公,如何不幸,先死行陣。」詔贈護軍將軍。世祖平侯景,追謚曰忠貞,并追贈助、警、構及尼皆中書郎,昂員外散騎常侍。



 粲長子臧,字君理。歷官尚書三公郎、太子洗馬、東宮領直。侯景至,帥兵屯西華門。城陷,奔江州,收舊部曲,據豫章,為其部下所害。



 江子一,字元貞,濟陽考城人,晉散騎常侍統之七世孫也。父法成,天監中奉朝請。子一少好學,有志操,以家貧闕養,因蔬食終身。起家王國侍郎、朝請。啟求觀書秘閣,
 高祖許之,有敕直華林省。其姑夫右衛將軍朱異,權要當朝,休下之日,賓客輻湊,子一未嘗造門,其高潔如此。稍遷尚書儀曹郎,出為遂昌、曲阿令,皆著美績。除通直散騎侍郎,出為戎昭將軍、南津校尉。



 弟子四,歷尚書金部郎。大同初,遷右丞。兄弟性並剛烈。子四自右丞上封事,極言得失,高祖甚善之,詔尚書詳擇施行焉。左民郎沈炯、少府丞顧璵嘗奏事不允,高祖厲色呵責之;子四乃趨前代炯等對,言甚激切,高祖怒呼縛之,子四據地不受,高祖怒亦止,乃釋之。猶坐免職。



 及侯景反,攻陷歷陽,自橫江將渡,子一帥舟師千餘人,於下流欲邀之,其
 副董桃生家在江北,因與其黨散走。子一乃退還南洲,復收餘眾,步道赴京師。賊亦尋至,子一啟太宗曰:「賊圍未合,猶可出盪,若營柵一固,無所用武。」請與其弟子四、子五帥所領百餘人,開承明門挑賊。許之。子一乃身先士卒,抽戈獨進,群賊夾攻之,從者莫敢繼。子四、子五見事急,相引赴賊,並見害。詔曰:「故戎昭將軍、通直散騎侍郎、南津校尉江子一,前尚書右丞江子四,東宮直殿主帥子五,禍故有聞,良以矜惻,死事加等,抑惟舊章。可贈子一給事黃門侍郎,子四中書侍郎,子五散騎侍郎。」侯景平,世祖又追贈子一侍中,謚義子;子四黃門侍郎,謚
 毅子;子五中書侍郎,謚烈子。



 子一續《黃圖》及班固「九品」,並辭賦文筆數十篇,行於世。



 張嵊,字四山,鎮北將軍稷之子也。少方雅,有志操,能清言。父臨青州,為土民所害,嵊感家禍,終身蔬食布衣,手不執刀刃。州舉秀才。起家秘書郎,累遷太子舍人、洗馬、司徒左西掾、中書郎。出為永陽內史,還除中軍宣城王司馬、散騎常侍。又出為鎮南湘東王長史、尋陽太守。中大同元年,徵為太府卿,俄遷吳興太守。



 太清二年,侯景圍京城,嵊遣弟伊率郡兵數千人赴援。三年,宮城陷,御史中丞沈浚違難東歸。嵊往見而謂曰:「賊臣憑陵,社稷
 危恥,正是人臣效命之秋。今欲收集兵力,保據貴鄉。若天道無靈,忠節不展,雖復及死,誠亦無恨。」浚曰:「鄙郡雖小,仗義拒逆,誰敢不從!」固勸嵊舉義。於是收集士卒,繕築城壘。時邵陵王東奔至錢唐,聞之,遣板授嵊征東將軍,加秩中二千石。嵊曰:「朝廷危迫,天子蒙塵,今日何情,復受榮號。」留板而已。賊行臺劉神茂攻破義興,遣使說嵊曰:「若早降附,當還以郡相處,復加爵賞。」嵊命斬其使,仍遣軍主王雄等帥兵於鱧瀆逆擊之,破神茂,神茂退走。侯景聞神茂敗,乃遣其中軍侯子鑒帥精兵二萬人,助神茂以擊嵊。嵊遣軍主范智朗出郡西拒戰,為神茂
 所敗,退歸。賊騎乘勝焚柵,柵內眾軍皆土崩。嵊乃釋戎服,坐於聽事,賊臨之以刃,終不為屈。乃執嵊以送景,景刑之於都市,子弟同遇害者十餘人,時年六十二。賊平,世祖追贈侍中、中衛將軍、開府儀同三司。謚曰忠貞子。



 沈浚,字叔源,吳興武康人。祖憲,齊散騎常侍,齊史有傳。浚少博學,有才幹,歷山陰、吳、建康令,並有能名。入為中書郎、尚書左丞。侯景逼京城,遷御史中丞。是時外援並至,侯景表請求和,詔許之。既盟,景知城內疾疫,復懷姦計,遲疑不去。數日,皇太子令浚詣景所,景曰:「即已向熱,非復行時。十萬之眾,何由可去,還欲立效朝廷,君可見
 為申聞。」浚曰:「將軍此論,意在得城。城內兵糧,尚支百日。將軍儲積內盡,國家援軍外集,十萬之眾,將何所資?而反設此言,欲脅朝廷邪?」景橫刃於膝,目叱之。浚正色責景曰:「明公親是人臣,舉兵向闕,聖主申恩赦過,已共結盟,口血未幹,而有翻背。沈浚六十之年,且天子之使,死生有命,豈畏逆臣之刀乎!」不顧而出。景曰:「是真司直也。」然密銜之。及破張嵊,乃求浚以害之。



 柳敬禮,開府儀同三司慶遠之孫。父津,太子詹事。敬禮與兄仲禮,皆少以勇烈知名。起家著作佐郎,稍遷扶風太守。侯景渡江,敬禮率馬步三千赴援。至都,據青溪埭,與
 景頻戰,恒先登陷陳,甚著威名。臺城沒,敬禮與仲禮俱見於景,景遣仲禮經略上流,留敬禮為質,以為護軍。景餞仲禮於後渚,敬禮密謂仲禮曰:「景今來會,敬禮抱之,兄拔佩刀,便可斫殺,敬禮死亦無所恨。」仲禮壯其言,許之。及酒數行,敬禮目仲禮,仲禮見備衛嚴,不敢動,計遂不果。會景征晉熙,敬禮與南康王會理共謀襲其城,剋期將發,建安侯蕭賁知而告之,遂遇害。



 史臣曰:若夫義重於生,前典垂誥,斯蓋先哲之所貴也。故孟子稱:生者我所欲,義亦我所欲,二事必不可兼得,寧捨生而取義。至如張嵊二三子之徒,捐軀殉節,赴死
 如歸,英風勁氣,籠罩今古,君子知梁代之有忠臣焉。



\end{pinyinscope}