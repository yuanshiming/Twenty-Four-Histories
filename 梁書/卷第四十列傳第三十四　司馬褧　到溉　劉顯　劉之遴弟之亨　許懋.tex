\article{卷第四十列傳第三十四 司馬褧 到溉 劉顯 劉之遴弟之亨 許懋}

\begin{pinyinscope}

 司馬褧,字元素,河內溫人也。曾祖純之,晉大司農高密敬王。祖讓之,員外常侍。父燮,善《三禮》,仕齊官至國子博士。褧少傳家業,強力專精,手不釋卷,其禮文所涉書,略皆遍睹。沛國劉獻為儒者宗,嘉其學,深相賞好。少與樂安任昉善,昉亦推重焉。初為國子生,起家奉朝請,稍遷
 王府行參軍。天監初,詔通儒治五禮,有司舉褧治嘉禮,除尚書祠部郎中。是時創定禮樂,褧所議多見施行。除步兵校尉,兼中書通事舍人。褧學尤精於事數,國家吉凶禮,當世名儒明山賓、賀蒨等疑不能斷,皆取決焉。累遷正員郎、鎮南諮議參軍,兼舍人如故。遷尚書右丞。出為仁威長史、長沙內史。還除雲騎將軍,兼御史中丞,頃之即真。十六年,出為宣毅南康王長史、行府國并石頭戍軍事。褧雖居外官,有敕預文德、武德二殿長名問訊,不限日。十七年,遷明威將軍、晉安王長史,未幾卒。王命記室庾肩吾集其文為十卷,所撰《嘉禮儀注》一百一十
 二卷。



 到溉,字茂灌,彭城武原人。曾祖彥之,宋驃騎將軍。祖仲度,驃騎江夏王從事中郎。父坦,齊中書郎。溉少孤貧,與弟洽俱聰敏有才學,早為任昉所知,由是聲名益廣。起家王國左常侍,轉後軍法曹行參軍,歷殿中郎。出為建安內史,遷中書郎,兼吏部,太子中庶子。湘東王繹為會稽太守,以溉為輕車長史、行府郡事。高祖敕王曰:「到溉非直為汝行事,足為汝師,間有進止,每須詢訪。」遭母憂,居喪盡禮,朝廷嘉之。服闋,猶蔬食布衣者累載。除通直散騎常侍,御史中丞,太府卿,都官尚書,郢州長史、江夏
 太守,加招遠將軍,入為左民尚書。



 溉身長八尺,美風儀,善容止,所蒞以清白自修。性又率儉,不好聲色,虛室單床,傍無姬侍。自外車服,不事鮮華,冠履十年一易,朝服或至穿補,傳呼清路,示有朝章而已。頃之,坐事左遷金紫光祿大夫,俄授散騎常侍、侍中、國子祭酒。



 溉素謹厚,特被高祖賞接,每與對棋,從夕達旦。溉第山池有奇石,高祖戲與賭之,并《禮記》一部,溉並輸焉,未進,高祖謂朱異曰;「卿謂到溉所輸可以送未?」溉斂板對曰:「臣既事君,安敢失禮。」高祖大笑,其見親愛如此。後因疾失明,詔以金紫光祿大夫、散騎常侍,就第養疾。



 溉家門雍睦,兄弟
 特相友愛。初與弟洽常共居一齋,洽卒後,便捨為寺,因斷腥膻,終身蔬食,別營小室,朝夕從僧徒禮誦。高祖每月三致凈饌,恩禮甚篤。蔣山有延賢寺者,溉家世創立,故生平公俸,咸以供焉,略無所取。性又不好交游,惟與朱異、劉之遴、張綰同志友密。及臥疾家園,門可羅雀,三君每歲時常鳴騶枉道,以相存問,置酒敘生平,極歡而去。臨終,託張、劉勒子孫以薄葬之禮,卒時年七十二。詔贈本官。有集二十卷行於世。時以溉、洽兄弟比之二陸,故世祖贈詩曰:「魏世重雙丁,晉朝稱二陸,何如今兩到,復似凌寒竹。」



 子鏡,字圓照,安西湘東王法曹行參軍,太
 子舍人,早卒。



 鏡子藎,早聰慧,起家著作佐郎,歷太子舍人,宣城王主簿,太子洗馬,尚書殿中郎。嘗從高祖幸京口,登北顧樓賦詩,藎受詔便就,上覽以示溉曰:「藎定是才子,翻恐卿從來文章假手於藎。」因賜溉《連珠》曰:「研磨墨以騰文,筆飛毫以書信。如飛蛾之赴火,豈焚身之可吝。必耄年其已及,可假之於少藎。」其見知賞如此。除丹陽尹丞。太清亂,赴江陵卒。



 劉顯,字嗣芳,沛國相人也。父鬷,晉安內史。顯幼而聰敏,當世號曰神童。天監初,舉秀才,解褐中軍臨川王行參軍,俄署法曹。顯好學,博涉多參通,任昉嘗得一篇缺簡書,
 文字零落,歷示諸人,莫能識者,顯云是《古文尚書》所刪逸篇,昉檢《周書》,果如其說,昉因大相賞異。丁母憂,服闋,尚書令沈約命駕造焉,於坐策顯經史十事,顯對其九。約曰:「老夫昏忘,不可受策;雖然,聊試數事,不可至十也。」顯問其五,約對其二。陸倕聞之歎曰:「劉郎可謂差人,雖吾家平原詣張壯武,王粲謁伯喈,必無此對。」其為名流推賞如此。及約為太子少傅,乃引為五官掾,俄兼廷尉正。五兵尚書傅昭掌著作,撰國史,引顯為佐。九年,始革尚書五都選,顯以本官兼吏部郎,又除司空臨川王外兵參軍,遷尚書儀曹郎。嘗為《上朝詩》,沈約見而美之,時
 約郊居宅新成,因命工書人題之於壁。出為臨川王記室參軍。建康平,復入為尚書儀曹侍郎,兼中書通事舍人。出為秣陵令,又除驃騎鄱陽王記室,兼中書舍人,累遷步兵校尉、中書侍郎,舍人如故。



 顯與河東裴子野、南陽劉之遴、吳郡顧協,連職禁中,遞相師友,時人莫不慕之。顯博聞強記,過於裴、顧,時魏人獻古器,有隱起字,無能識者,顯案文讀之,無有滯礙,考校年月,一字不差,高祖甚嘉焉。遷尚書左丞,除國子博士。出為宣遠岳陽王長史,行府國事,未拜,遷雲麾邵陵王長史、尋陽太守。大同九年,王遷鎮郢州,除平西諮議參軍,加戎昭將軍。其
 年卒,時年六十三。友人劉之遴啟皇太子曰:「之遴嘗聞,夷、叔、柳惠,不逢仲尼一言,則西山餓夫,東國黜士,名豈施於後世。信哉!生有七尺之形,終為一棺之土。不朽之事,寄之題目,懷珠抱玉,有歿世而名不稱者,可為長太息,孰過於斯。竊痛友人沛國劉顯,韞櫝藝文,研精覃奧,聰明特達,出類拔群。闔棺郢都,歸魂上國,卜宅有日,須鐫墓板。之遴已略撰其事行,今輒上呈。伏願鴻慈,降茲睿藻,榮其枯骴,以慰幽魂。冒昧塵聞,戰慄無地。」乃蒙令為誌銘曰:「繁弱挺質,空桑吐聲,分器見重,播樂傳名。誰其均之?美有髦士。禮著幼年,業明壯齒。厭飫典墳,研精名理。一見
 弗忘,過目則記。若訪賈逵,如問伯始。穎脫斯出,學優而仕。議獄既佐,芸蘭乃握。摶鳳池水,推羊太學。內參禁中,外相籓岳。斜光已道,殞彼西浮;百川到海,還逐東流。營營返魄,汎汎虛舟。白馬向郊,丹旒背鞏。野埃興伏,山雲輕重。呂掩書墳,揚歸玄冢。爾其戒行,途窮土壟。弱葛方施,叢柯日拱。遂柳荑春,禽寒斂氄。長空常暗,陰泉獨湧。祔彼故塋,流芬相踵。」



 顯有三子:莠,荏,臻。臻早著名。



 劉之遴,字思貞,南陽涅陽人也。父虯,齊國子博士,謚文範先生。之遴八歲能屬文,十五舉茂才對策,沈約、任昉見而異之。起家寧朔主簿。吏部尚書王瞻嘗候任昉,值
 之遴在坐,昉謂瞻曰:「此南陽劉之遴,學優未仕,水鏡所宜甄擢。」瞻即辟為太學博士。時張稷新除尚書僕射,託昉為讓表,昉令之遴代作,操筆立成。昉曰:「荊南秀氣,果有異才,後仕必當過僕。」御史中丞樂藹,即之遴舅,憲臺奏彈,皆之遴草焉。遷平南行參軍,尚書起部郎,延陵令,荊州治中。太宗臨荊州,仍遷宣惠記室。之遴篤學明審,博覽群籍。時劉顯、韋稜並強記,之遴每與討論,咸不能過也。



 還除通直散騎侍郎,兼中書通事舍人。遷正員郎,尚書右丞,荊州大中正。累遷中書侍郎,鴻臚卿,復兼中書舍人。出為征西鄱陽王長史、南郡太守,高祖謂曰:「卿
 母年德並高,故令卿衣錦還鄉,盡榮養之理。」後轉為西中郎湘東王長史,太守如故。初,之遴在荊府,嘗寄居南郡廨,忽夢前太守袁彖謂曰:「卿後當為折臂太守,即居此中。」之遴後果損臂,遂臨此郡。丁母憂,服闋,徵秘書監,領步兵校尉。出為郢州行事,之遴意不願出,固辭,高祖手敕曰:「朕聞妻子具,孝衰於親;爵祿具,忠衰於君。卿既內足,理忘奉公之節。」遂為有司所奏免。久之,為太府卿,都官尚書,太常卿。



 之遴好古愛奇,在荊州聚古器數十百種。有一器似甌,可容一斛,上有金錯字,時人無能知者。又獻古器四種於東宮。其第一種,鏤銅鴟夷榼二枚,
 兩耳有銀鏤,銘云「建平二年造」。其第二種,金銀錯鏤古樽二枚,有篆銘云「秦容成侯適楚之歲造」。其第三種,外國澡灌一口,銘云「元封二年,龜茲國獻」。其第四種,古製澡盤一枚,銘云「初平二年造」。



 時鄱陽嗣王範得班固所上《漢書》真本,獻之東宮,皇太子令之遴與張纘、到溉、陸襄等參校異同。之遴具異狀十事,其大略曰:「案古本《漢書》稱『永平十六年五月二十一日己酉,郎班固上』;而今本無上書年月日字。又案古本《敘傳》號為中篇;今本稱為《敘傳》。又今本《敘傳》載班彪事行;而古本云『稚生彪,自有傳』。又今本紀及表、志、列傳不相合為次,而古本相合
 為次,總成三十八卷。又今本《外戚》在《西域》後;古本《外戚》次《帝紀》下。又今本《高五子》、《文三王》、《景十三王》、《武五子》、《宣元六王》雜在諸傳秩中;古本諸王悉次《外戚》下,在《陳項傳》前。又今本《韓彭英盧吳》述云『信惟餓隸,布實黥徒,越亦狗盜,芮尹江湖,雲起龍驤,化為侯王』;古本述云『淮陰毅毅,杖劍周章,邦之傑子,實惟彭、英,化為侯王,雲起龍驤』。又古本第三十七卷,解音釋義,以助雅詁,而今本無此卷。」



 之遴好屬文,多學古體,與河東裴子野、沛國劉顯常共討論書籍,因為交好。是時《周易》、《尚書》、《禮記》、《毛詩》並有高祖義疏,惟《左氏傳》尚闕。之遴乃著《春秋大意》十科,《
 左氏》十科,《三傳同異》十科,合三十事以上之。高祖大悅,詔答之曰:「省所撰《春秋》義,比事論書,辭微旨遠。編年之教,言闡義繁,丘明傳洙泗之風,公羊稟西河之學,鐸椒之解不追,瑕丘之說無取。繼踵胡母,仲舒云盛,因脩《穀梁》,千秋最篤。張蒼之傳《左氏》,賈誼之襲荀卿,源本分鑣,指歸殊致,詳略紛然,其來舊矣。昔在弱年,乃經研味,一從遺置,迄將五紀。兼晚冬晷促,機事罕暇,夜分求衣,未遑搜括。須待夏景,試取推尋,若溫故可求,別酬所問也。」



 太清二年,侯景亂,之遴避難還鄉,未至,卒於夏口,時年七十二。前後文集五十卷,行於世。



 之亨字嘉會,之遴弟也。少有令名。舉秀才,拜太學博士,稍遷兼中書通事舍人,步兵校尉,司農卿。又代兄之遴為安西湘東王長史、南郡太守。在郡有異績。數年卒於官,時年五十。荊士至今懷之,不忍斥其名,號為「大南郡」、「小南郡」云。



 許懋,字昭哲,高陽新城人,魏鎮北將軍允九世孫。祖珪,宋給事中,著作郎,桂陽太守。父勇惠,齊太子家令,冗從僕射。懋少孤,性至孝,居父憂,執喪過禮。篤志好學,為州黨所稱。十四入太學,受《毛詩》,旦領師說,晚而覆講,座下聽者常數十百人,因撰《風雅比興義》十五卷,盛行於世。
 尤曉故事,稱為儀注之學。



 起家後軍豫章王行參軍,轉法曹,舉茂才,遷驃騎大將軍儀同中記室。文惠太子聞而召之,侍講于崇明殿,除太子步兵校尉。永元中,轉散騎侍郎,兼國子博士。與司馬褧同志友善,僕射江祏甚推重之,號為「經史笥」。天監初,吏部尚書范雲舉懋參詳五禮,除征西鄱陽王諮議,兼著作郎,待詔文德省。時有請封會稽禪國山者,高祖雅好禮,因集儒學之士,草封禪儀,將欲行焉。懋以為不可,因建議曰:臣案舜幸岱宗,是為巡狩,而鄭引《孝經鉤命決》云「封于泰山,考績柴燎,禪乎梁甫,刻石紀號」。此緯書之曲說,非正經之通義也。依《
 白虎通》云,「封者,言附廣也;禪者,言成功相傳也」。若以禪授為義,則禹不應傳啟至桀十七世也,湯又不應傳外丙至紂三十七世也。又《禮記》云:「三皇禪奕奕,謂盛德也。五帝禪亭亭,特立獨起於身也。三王禪梁甫,連延不絕,父沒子繼也。」若謂「禪奕奕為盛德者,古義以伏羲、神農、黃帝,是為三皇。伏羲封泰山,禪云云,黃帝封泰山,禪亭亭,皆不禪奕奕,而云盛德,則無所寄矣。若謂五帝禪亭亭,特立獨起於身者,顓頊封泰山,禪云云,帝嚳封泰山,禪云云,堯封泰山,禪云云,舜封泰山,禪云云,亦不禪亭亭,若合黃帝以為五帝者,少昊即黃帝子,又非獨立之
 義矣。若謂三王禪梁甫,連延不絕,父沒子繼者,禹封泰山,禪云云,周成王封泰山,禪社首,舊書如此,異乎《禮說》,皆道聽所得,失其本文。假使三王皆封泰山禪梁甫者,是為封泰山則有傳世之義,禪梁甫則有揖讓之懷,或欲禪位,或欲傳子,義既矛盾,理必不然。



 又七十二君,夷吾所記,此中世數,裁可得二十餘主:伏羲、神農、女媧、大庭、柏皇、中央、慄陸、驪連、赫胥、尊盧、混沌、昊英、有巢、朱襄、葛天、陰康、無懷、黃帝、少昊、顓頊、高辛、堯、舜、禹、湯、文、武,中間乃有共工,霸有九州,非帝之數,云何得有七十二君封禪之事?且燧人以前至周之世,未有君臣,人心淳朴,
 不應金泥玉檢,升中刻石。燧人、伏羲、神農三皇結繩而治,書契未作,未應有鐫文告成。且無懷氏,伏羲後第十六主,云何得在伏羲前封泰山禪云云?



 夷吾又曰:「惟受命之君然後得封禪。」周成王非受命君,云何而得封泰山禪社首?神農與炎帝是一主,而云神農封泰山禪云云,炎帝封泰山禪云云,分為二人,妄亦甚矣!若是聖主,不須封禪;若是凡主,不應封禪。當是齊桓欲行此事,管仲知其不可,故舉怪物以屈之也。



 秦始皇登泰山中阪,風雨暴至,休松樹下,封為五大夫,而事不遂。漢武帝宗信方士,廣召儒生,皮弁搢紳,射牛行事,獨與霍嬗俱上,
 既而子侯暴卒,厥足用傷。至魏明,使高堂隆撰其禮儀,聞隆沒,歎息曰:「天不欲成吾事,高生捨我亡也。」晉武泰始中欲封禪,乃至太康議猶不定,意不果行。孫皓遣兼司空董朝、兼太常周處至陽羨封禪國山。此朝君子,有何功德?不思古道而欲封禪,皆是主好名於上,臣阿旨於下也。



 夫封禪者,不出正經,惟《左傳》說「禹會諸侯於塗山,執玉帛者萬國」,亦不謂為封禪。鄭玄有參、柴之風,不能推尋正經,專信緯候之書,斯為謬矣。蓋《禮》云「因天事天,因地事地,因名山升中於天,因吉土享帝于郊」。燔柴岱宗,即因山之謂矣。故《曲禮》云「天子祭天地」是也。又祈
 穀一,報穀一,禮乃不顯祈報地,推文則有。《樂記》云:「大樂與天地同和,大禮與天地同節;和故百物不失,節故祀天祭地。」百物不失者,天生之,地養之。故知地亦有祈報,是則一年三郊天,三祭地。《周官》有員丘方澤者,總為三事,郊祭天地。故《小宗伯》云「兆五帝於四郊」,此即《月令》迎氣之郊也。《舜典》有「歲二月東巡狩,至於岱宗」,夏南,秋西,冬北,五年一周,若為封禪,何其數也!此為九郊,亦皆正義。至如大旅於南郊者,非常祭也。《大宗伯》「國有大故則旅上帝」,《月令》云「仲春玄鳥至,祀於高禖」,亦非常祭。故《詩》云「克禋克祀,以弗無子」。并有雩禱,亦非常祭。《禮》云「雩,頠
 水旱也」。是為合郊天地有三,特郊天有九,非常祀又有三。《孝經》云:「宗祀文王於明堂,以配上帝。」雩祭與明堂雖是祭天,而不在郊,是為天祀有十六,地祭有三,惟大禘祀不在此數。《大傳》云:「王者禘其祖之所自出,以其祖配之。」異於常祭,以故云大於時祭。案《系辭》云:「《易》之為書也,廣大悉備。有天道焉,有地道焉,有人道焉,兼三才而兩之,故六。六者非佗,三才之道也。」《乾·彖》云:「大哉乾元,萬物資始,乃統天。雲行雨施,品物流形,大明終始,六位時成。」此則應六年一祭,坤元亦爾。誠敬之道,盡此而備。至於封禪,非所敢聞。



 高祖嘉納之,因推演懋議,稱制旨以答,
 請者由是遂停。



 十年,轉太子家令。宋、齊舊儀,郊天祀帝,皆用袞冕,至天監七年,懋始請造大裘。至是,有事于明堂,儀注猶云「服袞冕」。懋駮云:「《禮》云『大裘而冕,祀昊天上帝亦如之。』良由天神尊遠,須貴誠質。今泛祭五帝,理不容文。」改服大裘,自此始也。又降敕問:「凡求陰陽,應各從其類,今雩祭燔柴,以火祈水,意以為疑。」懋答曰:「雩祭燔柴,經無其文,良由先儒不思故也。按周宣《雲漢》之詩曰:『上下奠瘞,靡神不宗。』毛注云:『上祭天,下祭地,奠其幣,瘞其物。』以此而言,為旱而祭天地,並有瘞埋之文,不見有燔柴之說。若以祭五帝必應燔柴者,今明常之禮,又無
 其事。且《禮》又云『埋少牢以祭時』,時之功是五帝,此又是不用柴之證矣。昔雩壇在南方正陽位,有乖求神;而已移於東,實柴之禮猶未革。請停用柴,其牲牢等物,悉從坎瘞,以符周宣《雲漢》之說。」詔並從之。凡諸禮儀,多所刊正。



 以足疾出為始平太守,政有能名。加散騎常侍,轉天門太守。中大通三年,皇太子召諸儒參錄《長春義記》。四年,拜中庶子。是歲卒,時年六十九。撰《述行記》四卷,有集十五卷。



 陳吏部尚書姚察曰:司馬褧儒術博通,到溉文義優敏,顯、懋、之遴強學浹洽,並職經便繁,應對左右,斯蓋嚴、朱
 之任焉。而溉、之遴遂至顯貴,亟拾青紫;然非遇時,焉能致此仕也。



\end{pinyinscope}