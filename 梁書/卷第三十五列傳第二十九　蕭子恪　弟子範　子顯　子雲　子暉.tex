\article{卷第三十五列傳第二十九 蕭子恪 弟子範 子顯 子雲 子暉}

\begin{pinyinscope}

 蕭子恪,字景沖,蘭陵人,齊豫章文獻王嶷第二子也。永明中,以王子封南康縣侯。年十二,和從兄司徒竟陵王《高松賦》,衛軍王儉見而奇之。初為寧朔將軍、淮陵太守,建武中,遷輔國將軍、吳郡太守。大司馬王敬則於會稽舉兵反,以奉子恪為名,明帝悉召子恪兄弟親從七十餘人入西省,至夜當害之。會子恪棄郡奔歸,是日亦至,
 明帝乃止,以子恪為太子中庶子。東昏即位,遷祕書監,領右軍將軍,俄為侍中。中興二年,遷輔國諮議參軍。天監元年,降爵為子,除散騎常侍,領步兵校尉,以疾不拜,徙為光祿大夫,俄為司徒左長史。



 子恪與弟子範等,嘗因事入謝,高祖在文德殿引見之,從容謂曰:「我欲與卿兄弟有言。夫天下之寶,本是公器,非可力得。茍無期運,雖有項籍之力,終亦敗亡。所以班彪《王命論》云:『所求不過一金,然終轉死溝壑』。卿不應不讀此書。宋孝武為性猜忌,兄弟粗有令名者,無不因事鴆毒,所遺唯有景和。至於朝臣之中,或疑有天命而致害者,枉濫相繼,然而
 或疑有天命而不能害者,或不知有天命而不疑者,于時雖疑卿祖,而無如之何。此是疑而不得。又有不疑者,如宋明帝本為庸常被免,豈疑而得全?又復我于時已年二歲,彼豈知我應有今日?當知有天命者,非人所害,害亦不能得。我初平建康城,朝廷內外皆勸我云:『時代革異,物心須一,宜行處分。』我於時依此而行,誰謂不可!我政言江左以來,代謝必相誅戮,此是傷於和氣,所以國祚例不靈長。所謂『殷鑒不遠,在夏后之世。』此是一義。二者,齊梁雖曰革代,義異往時。我與卿兄弟雖復絕服二世,宗屬未遠。卿勿言兄弟是親,人家兄弟自有周旋
 者,有不周旋者,況五服之屬邪?齊業之初,亦是甘苦共嘗,腹心在我。卿兄弟年少,理當不悉。我與卿兄弟,便是情同一家,豈當都不念此,作行路事。此是二義。我有今日,非是本意所求。且建武屠滅卿門,致卿兄弟塗炭。我起義兵,非惟自雪門恥,亦是為卿兄弟報仇。卿若能在建武、永元之世,撥亂反正,我雖起樊、鄧,豈得不釋戈推奉;其雖欲不已,亦是師出無名。我今為卿報仇,且時代革異,望卿兄弟盡節報我耳。且我自藉喪亂,代明帝家天下耳,不取卿家天下。昔劉子輿自稱成帝子,光武言『假使成帝更生,天下亦不復可得,況子輿乎』。梁初,人勸
 我相誅滅者,我答之猶如向孝武時事:彼若茍有天命,非我所能殺;若其無期運,何忽行此,政足示無度量。曹志親是魏武帝孫,陳思之子,事晉武能為晉室忠臣,此即卿事例。卿是宗室,情義異佗,方坦然相期,卿無復懷自外之意。小待,自當知我寸心。」又文獻王時,內齋直帳閹人趙叔祖,天監初,入為臺齊齋帥,在壽光省,高祖呼叔祖曰:「我本識汝在北第,以汝舊人,故每驅使。汝比見北第諸郎不?」叔祖奉答云:「比多在直,出外甚疏,假使暫出,亦不能得往。」高祖曰:「若見北第諸郎,道我此意:我今日雖是革代,情同一家;但今磐石未立,所以未得用諸郎
 者,非惟在我未宜,亦是欲使諸郎得安耳。但閉門高枕,後自當見我心。」叔祖即出外具宣敕語。



 子恪尋出為永嘉太守。還除光祿卿,秘書監。出為明威將軍、零陵太守。十七年,入為散騎常侍、輔國將軍。普通元年,遷宗正卿。三年,遷都官尚書。四年,轉吏部。六年,遷太子詹事。大通二年,出為寧遠將軍、吳郡太守。三年,卒于郡舍,時年五十二。詔贈侍中、中書令。謚曰恭。



 子恪兄弟十六人,並仕梁。有文學者,子恪、子質、子顯、子雲、子暉五人。子恪嘗謂所親曰:「文史之事,諸弟備之矣,不煩吾復牽率,但退食自公,無過足矣。」子恪少亦涉學,頗屬文,隨棄其本,故不
 傳文集。



 子瑳,亦知名太清中,官至吏部郎,避亂東陽,後為盜所害。



 子範字景則,子恪第六弟也。齊永明十年,封祁陽縣侯,拜太子洗馬。天監初,降爵為子,除後軍記室參軍,復為太子洗馬,俄遷司徒主簿,丁所生母憂去職。子範有孝性,居喪以毀聞。服闋,又為司徒主簿,累遷丹陽尹丞,太子中舍人。出為建安太守,還除大司馬南平王戶曹屬,從事中郎。王愛文學士,子範偏被恩遇,嘗曰:「此宗室奇才也。」使製《千字文》,其辭甚美,王命記室蔡薳注釋之。自是府中文筆,皆使草之。王薨,子範遷宣惠諮議參軍,護
 軍臨賀王正德長史。正德為丹陽尹,復為正德信威長史,領尹丞。歷官十餘年,不出籓府,常以自慨,而諸弟並登顯列,意不能平,及是為到府箋曰:「上籓首佐,於茲再忝,河南雌伏,自此重昇。以老少異時,盛衰殊日,雖佩恩寵,還羞年鬢。」子範少與弟子顯、子雲才名略相比,而風采容止不逮,故宦途有優劣。每讀《漢書》,杜緩兄弟「五人至大官,唯中弟欽官不至而最知名」,常吟諷之,以況己也。



 尋復為宣惠武陵王司馬,不就,仍除中散大夫,遷光祿、廷尉卿。出為戎昭將軍、始興內史。還除太中大夫,遷秘書監。太宗即位,召為光祿大夫,加金章紫綬,以逼賊
 不拜。其年葬簡皇后,使與張纘俱製哀策文,太宗覽讀之,曰:「今葬禮雖闕,此文猶不減於舊。」尋遇疾卒,時年六十四。賊平後,世祖追贈金紫光祿大夫。謚曰文。前後文集三十卷。



 二子滂、確,並少有文章。太宗東宮時,嘗與邵陵王數諸蕭文士,滂、確亦預焉。滂官至尚書殿中郎,中軍宣城王記室,先子範卒。確,太清中歷官宣城王友,司徒右長史。賊平後,赴江陵,因沒關西。



 子顯字景陽,子恪第八弟也。幼聰慧,文獻王異之,愛過諸子。七歲,封寧都縣侯。永元末,以王子例拜給事中。天監初,降爵為子。累遷安西外兵,仁威記室參軍,司徒主
 簿,太尉錄事。



 子顯偉容貌,身長八尺。好學,工屬文。嘗著《鴻序賦》,尚書令沈約見而稱曰:「可謂得明道之高致,蓋《幽通》之流也。」又採眾家《後漢》,考正同異,為一家之書。又啟撰《齊史》,書成,表奏之,詔付秘閣。累遷太子中舍人,建康令,邵陵王友,丹陽尹丞,中書郎,守宗正卿。出為臨川內史,還除黃門郎。中大通二年,遷長兼侍中。高祖雅愛子顯才,又嘉其容止吐納,每御筵侍坐,偏顧訪焉。嘗從容謂子顯曰:「我造《通史》,此書若成,眾史可廢。」子顯對曰:「仲尼贊《易》道,黜《八索》,述職方,除《九丘》,聖製符同,復在茲日。」時以為名對。三年,以本官領國子博士。高祖所製經
 義,未列學官,子顯在職,表置助教一人,生十人。又啟撰高祖集,并《普通北伐記》。其年遷國子祭酒,又加侍中,於學遞述高祖《五經義》。五年,遷吏部尚書,侍中如故。



 子顯性凝簡,頗負其才氣。及掌選,見九流賓客,不與交言,但舉扇一捴而已,衣冠竊恨之。然太宗素重其為人,在東宮時,每引與促宴。子顯嘗起更衣,太宗謂坐客曰:「嘗聞異人間出,今日始知是蕭尚書。」其見重如此。大同三年,出為仁威將軍、吳興太守,至郡未幾,卒,時年四十九。詔曰:「仁威將軍、吳興太守子顯,神韻峻舉,宗中佳器。分竹未久,奄到喪殞,惻愴于懷。可贈侍中、中書令。今便舉哀。」
 及葬請謚,手詔「恃才傲物,宜謚曰驕」。



 子顯嘗為《自序》,其略云:「餘為邵陵王友,忝還京師,遠思前比,即楚之唐、宋,梁之嚴、鄒。追尋平生,頗好辭藻,雖在名無成,求心已足。若乃登高自極,臨水送歸,風動春朝,月明秋夜,早雁初鶯,開花落葉,有來斯應,每不能已也。前世賈、傅、崔、馬、邯鄲、繆、路之徒,並以文章顯,所以屢上歌頌,自比古人。天監十六年,始預九日朝宴,稠人廣坐,獨受旨云:『今雲物甚美,卿得不斐然賦詩。』詩既成,又降帝旨曰:『可謂才子。』余退謂人曰:『一顧之恩,非望而至。遂方賈誼何如哉?未易當也。』每有製作,特寡思功,須其自來,不以力構。少來
 所為詩賦,則《鴻序》一作,體兼眾製,文備多方,頗為好事所傳,故虛聲易遠。」



 子顯所著《後漢書》一百卷,《齊書》六十卷,《普通北伐記》五卷,《貴儉傳》三十卷,文集二十卷。



 二子序、愷,並少知名。序,太清中歷官太子家令,中庶子,並掌管記。及亂,於城內卒。愷,初為國子生,對策高第,州又舉秀才。起家秘書郎,遷太子中舍人,王府主簿,太子洗馬,父憂去職。服闋,復除太子洗馬,遷中舍人,並掌管記。累遷宣城王文學,中書郎,太子家令,又掌管記。愷才學譽望,時論以方其父,太宗在東宮,早引接之。時中庶子謝嘏出守建安,於宣猷堂宴餞,並召時才賦詩,同用十五
 劇韻,愷詩先就,其辭又美。太宗與湘東王令曰:「王筠本自舊手,後進有蕭愷可稱,信為才子。」先是時太學博士顧野王奉令撰《玉篇》,太宗嫌其書詳略未當,以愷博學,於文字尤善,使更與學士刪改。遷中庶子,未拜,徙為吏部郎。太清二年,遷御史中丞。頃之,侯景寇亂,愷於城內遷侍中,尋卒官,時年四十四。文集並亡逸。



 子雲字景喬,子恪第九弟也。年十二,齊建武四年,封新浦縣侯,自製拜章,便有文採。天監初,降爵為子。既長勤學,以晉代竟無全書,弱冠便留心撰著,至年二十六,書成,表奏之,詔付秘閣。子雲性沈靜,不樂仕進。年三十,方
 起家為秘書郎。遷太子舍人,撰《東宮新記》,奏之,敕賜束帛。累遷北中郎外兵參軍,晉安王文學,司徒主簿,丹陽尹丞。時湘東王為京尹,深相賞好,如布衣之交。遷北中郎廬陵王諮議參軍,兼尚書左丞。大通元年,除黃門郎,俄遷輕車將軍,兼司徒左長史。二年,入為吏部。三年,遷長兼侍中。中大通元年,轉太府卿。三年,出為貞威將軍、臨川內史。在郡以和理稱,民吏悅之。還除散騎常侍,俄復為侍中。大同二年,遷員外散騎常侍、國子祭酒,領南徐州大中正。頃之,復為侍中,祭酒、中正如故。



 梁初,郊廟未革牲牷,樂辭皆沈約撰,至是承用,子雲始建言宜改。
 啟曰:「伏惟聖敬率由,尊嚴郊廟,得西鄰之心,知周、孔之迹,載革牢俎,德通神明,黍稷蘋藻,竭誠嚴配,經國制度,方懸日月,垂訓百王,於是乎在。臣比兼職齋官,見伶人所歌,猶用未革牲前曲。圜丘視燎,尚言『式備牲牷』;北郊《諴雅》,亦奏『牲云孔備』;清廟登歌,而稱『我牲以潔』;三朝食舉,猶詠『朱尾碧鱗』。聲被鼓鐘,未符盛制。臣職司儒訓,意以為疑,未審應改定樂辭以不?」敕答曰:「此是主者守株,宜急改也。」仍使子雲撰定。敕曰:「郊廟歌辭,應須典誥大語,不得雜用子史文章淺言;而沈約所撰,亦多舛謬。」子雲答敕曰:「殷薦朝饗,樂以雅名,理應正採《五經》,聖人成
 教。而漢來此製,不全用經典;約之所撰,彌復淺雜。臣前所易約十曲,惟知牲牷既革,宜改歌辭,而猶承例,不嫌流俗乖體。既奉令旨,始得發蒙。臣夙本庸滯,昭然忽朗,謹依成旨,悉改約制。惟用《五經》為本,其次《爾雅》、《周易》、《尚書》、《大戴禮》,即是經誥之流,愚意亦取兼用。臣又尋唐、虞諸書,殷《頌》周《雅》,稱美是一,而復各述時事。大梁革服,偃武脩文,制禮作樂,義高三正;而約撰歌辭,惟浸稱聖德之美,了不序皇朝制作事。《雅》、《頌》前例,於體為違。伏以聖旨所定《樂論》,鐘律緯緒,文思深微,命世一出,方懸日月,不刊之典,禮樂之教,致治所成。謹一二採綴,各隨事顯
 義,以明制作之美。覃思累日,今始克就,謹以上呈。」敕並施用。



 子雲善草隸書,為世楷法。自云善效鐘元常、王逸少而微變字體。答敕云:「臣昔不能拔賞,隨世所貴,規摹子敬,多歷年所。年二十六,著《晉史》,至《二王列傳》,欲作論語草隸法,言不盡意,遂不能成,略指論飛白一勢而已。十許年來,始見敕旨《論書》一卷,商略筆勢,洞澈字體;又以逸少之不及元常,猶子敬之不及逸少。自此研思,方悟隸式,始變子敬,全範元常。逮爾以來,自覺功進。」其書跡雅為高祖所重,嘗論子雲書曰:「筆力勁駿,心手相應,巧踰杜度,美過崔實,當與元常並驅爭先。」其見賞如此。



 七
 年,出為仁威將軍、東陽太守。中大同元年,還拜宗正卿。太清元年,復為侍中、國子祭酒,領南徐州大中正。二年,侯景寇逼,子雲逃民間。三年三月,宮城失守,東奔晉陵,餒卒於顯靈寺僧房,年六十三。所著《晉書》一百一十卷,《東宮新記》二十卷。



 第二子特,字世達。早知名,亦善草隸。高祖嘗謂子雲曰:「子敬之書,不及逸少。近見特迹,遂逼於卿。」歷官著作佐郎,太子舍人,宣惠主簿,中軍記室。出為海鹽令,坐事免。年二十五,先子雲卒。



 子暉字景光,子雲弟也。少涉書史,亦有文才。起家員外散騎侍郎,遷南中郎記室。出為臨安令。性恬靜,寡嗜好,
 嘗預重雲殿聽制講《三慧經》,退為《講賦》奏之,甚見稱賞。遷安西武陵王諮議,帶新繁令,隨府轉儀同從事、驃騎長史,卒。



 陳吏部尚書姚察曰:昔魏藉兵威而革漢運,晉因宰輔乃移魏歷,異乎古之禪授,以德相傳,故抑前代宗枝,用絕民望。然劉曄、曹志,猶顯於朝;及宋遂為廢姓。而齊代,宋之戚屬,一皆殲焉。其祚不長,抑亦由此。有梁革命,弗取前規,故子恪兄弟及群從,並隨才任職,通貴滿朝,不失於舊,豈惟魏幽晉顯而已哉。君子以是知高祖之弘量,度越前代矣。



\end{pinyinscope}