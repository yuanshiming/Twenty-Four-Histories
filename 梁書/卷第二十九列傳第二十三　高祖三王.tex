\article{卷第二十九列傳第二十三 高祖三王}

\begin{pinyinscope}

 高祖八
 男:丁貴嬪生昭明太子統,太宗簡文皇帝,廬陵威王續;阮脩容生世祖孝元皇帝;吳淑媛生豫章王綜;董淑儀生南康簡王績;丁充華生邵陵攜王綸;葛脩容生武陵王紀。綜及紀別有傳。



 南康簡王績,字世謹,高祖第四子。天監八年,封南康郡王,邑二千戶。出為輕車將軍,領石頭戍軍事。十年,遷使
 持節、都督南徐州諸軍事、南徐州刺史,進號仁威將軍。績時年七歲,主者有受貨,洗改解書,長史王僧孺弗之覺,績見而輒詰之,便即時首服,眾咸歎其聰警。十六年,徵為宣毅將軍、領石頭戍軍事。十七年,出為使持節、都督南、北兗、徐、青、冀五州諸軍事、南兗州刺史,在州著稱。尋有詔徵還,民曹嘉樂等三百七十人詣闕上表,稱績尤異一十五條,乞留州任,優詔許之,進號北中郎將。普通四年,徵為侍中、雲麾將軍,領石頭戍軍事。五年,出為使持節、都督江州諸軍事、江州刺史。丁董淑儀憂,居喪過禮,高祖手詔勉之,使攝州任,固求解職,乃徵授安右
 將軍、領石頭戍軍事,尋加護軍。羸瘠弗堪視事。大通三年,因感病薨于任,時年二十五。贈侍中、中軍將軍、開府儀同三司,給鼓吹一部。謚曰簡。



 績寡玩好,少嗜慾,居無僕妾,躬事約儉,所有租秩,悉寄天府。及薨後,府有南康國無名錢數千萬。



 子會理嗣,字長才。少聰慧,好文史。年十一而孤,特為高祖所愛,衣服禮秩與正王不殊。年十五,拜輕車將軍、湘州刺史,又領石頭戍軍事。遷侍中,兼領軍將軍。尋除宣惠將軍、丹陽尹,置佐史。出為使持節、都督南、北兗、北徐、青、冀、東徐、譙七州諸軍事、平北將軍、南兗州刺史。太清元年,督眾軍北討,至彭城,為魏師所
 敗,退歸本鎮。



 二年,侯景圍京邑,會理治嚴將入援,會北徐州刺史封山侯正表將應其兄正德,外託赴援,實謀襲廣陵,會理擊破之。方得進路,臺城陷,侯景遣前臨江太守董紹先以高祖手敕召會理,其僚佐咸勸距之。會理曰:「諸君心事,與我不同,天子年尊,受制賊虜,今有手敕召我入朝,臣子之心,豈得違背。且遠處江北,功業難成,不若身赴京都,圖之肘腋。吾計決矣。」遂席卷而行,以城輸紹先。至京,景以為侍中、司空、兼中書令。雖在寇手,每思匡復,與西鄉侯勸等潛布腹心,要結壯士。時范陽祖皓斬紹先,據廣陵城起義,期以會理為內應。皓敗,辭
 相連及,景矯詔免會理官,猶以白衣領尚書令。



 是冬,景往晉熙,景師虛弱,會理復與柳敬禮謀之。敬禮曰:「舉大事必有所資,今無寸兵,安可以動?」會理曰:「湖熟有吾舊兵三千餘人,昨來相知,克期響集,聽吾日定,便至京師。計賊守兵不過千人耳,若大兵外攻,吾等內應,直取王偉,事必有成。縱景後歸,無能為也。」敬禮曰「善」,因贊成之。於時百姓厭賊,咸思用命,自丹陽至于京口,靡不同之。後事不果,與弟祁陽侯通理並遇害。



 通理字仲宣,位太子洗馬,封祁陽侯。



 通理弟乂理,字季英,會理第六弟也。生十旬而簡王薨,至三歲而能言,見內人分散,涕泣相送,乂理問其故,或曰:「此簡王宮人,喪畢去爾。」
 乂理便號泣,悲不自勝,諸宮人見之,莫不傷感,為之停者三人焉。服闋後,見高祖,又悲泣不自勝。高祖為之流涕,謂左右曰:「此兒大必為奇士。」大同八年,封安樂縣侯,邑五百戶。



 乂理性慷慨,慕立功名,每讀書見忠臣烈士,未嘗不廢卷歎曰:「一生之內,當無愧古人。」博覽多識,有文才,嘗祭孔文舉墓,并為立碑,製文甚美。



 太清中,侯景內寇,乂理聚賓客數百,輕裝赴南兗州,隨兄會理入援,恆親當矢石,為士卒先。及城陷,又隨會理還廣陵,因入齊為質,乞師。行二日,會侯景遣董紹先據廣陵,遂追會理,因為所獲。紹先防之甚嚴,不得與兄弟相見,乃偽請
 先還京,得入辭母,謂其姊安固公主曰:「事既如此,豈可合家受斃。兄若至,願為言之,善為計自勉,勿賜以為念也。家國阽危,雖死非恨,前途亦思立效,但未知天命何如耳!」至京師,以魏降人元貞立節忠正,可以託孤,乃以玉柄扇贈之。貞怪其故,不受。乂理曰:「後當見憶,幸勿推辭。」會祖皓起兵,乂理奔長蘆,收軍得千餘人。其左右有應賊者,因間劫會理,其眾遂駭散,為景所害,時年二十一。元貞始悟其前言,往收葬焉。



 廬陵威王續,字世,高祖第五子,天監八年,封廬陵郡王,邑二千戶。十年,拜輕車將軍、南彭城瑯邪太守。十三
 年,轉會稽太守。十六年,為都督江州諸軍事、雲麾將軍、江州刺史。普通元年,徵為宣毅將軍,領石頭戍軍事。



 續少英果,膂力絕人,馳射游獵,應發命中。高祖常歎曰:「此我之任城也。」嘗與臨賀王正德及胡貴通、趙伯超等馳射於高祖前,續冠於諸人,高祖大悅。三年,為使持節、都督雍、梁、秦、沙四州諸軍事、西中郎將、雍州刺史。七年,加宣毅將軍。中大通二年,又為使持節、都督雍、梁、秦、沙四州諸軍事、平北將軍、寧蠻校尉、雍州刺史,給鼓吹一部。續多聚馬仗,畜養驍雄,金帛內盈,倉廩外實。四年,遷安北將軍。大同元年,為使持節、都督江州諸軍事、安
 南將軍、江州刺史。三年,徵為護軍將軍、領石頭戍軍事。五年,為驃騎將軍、開府儀同三司。又出為使持節、都督荊、郢、司、雍、南、北秦、梁、巴、華九州諸軍事、荊州刺史。中大同二年,薨於州,時年四十四。贈司空、散騎常侍、驃騎大將軍,鼓吹一部,謚曰威。長子安嗣。



 邵陵攜王綸,字世調,高祖第六子也。少聰穎,博學善屬文,尤工尺牘。天監十三年,封邵陵郡王,邑二千戶。出為寧遠將軍、瑯邪、彭城二郡太守,遷輕車將軍、會稽太守。十八年,徵為信威將軍。普通元年,領石頭戍軍事,尋為江州刺史。五年,以西中郎將權攝南兗州,坐事免官奪
 爵。七年,拜侍中。大通元年,復封爵,尋加信威將軍,置佐史。中大通元年,為丹陽尹。四年,為侍中、宣惠將軍、揚州刺史。以侵漁細民,少府丞何智通以事啟聞,綸知之,令客戴子高於都巷刺殺之。智通子訴于闕下,高祖令圍綸第,捕子高,綸匿之,竟不出。坐免為庶人。頃之,復封爵。大同元年,為侍中、雲麾將軍。七年,出為使持節、都督郢、定、霍、司四州諸軍事、平西將軍、郢州刺史,遷為安前將軍、丹陽尹。中大同元年,出為鎮東將軍、南徐州刺史。



 太清二年,進位中衛將軍、開府儀同三司。侯景構逆,加征討大都督,率眾討景。將發,高祖誡曰:「侯景小豎,頗習行
 陣,未可以一戰即殄,當以歲月圖之。」綸次鐘離,景已度采石。綸乃晝夜兼道,遊軍入赴。濟江中流,風起,人馬溺者十一二。遂率寧遠將軍西豊公大春、新淦公大成等,步騎三萬,發自京口。將軍趙伯超曰:「若從黃城大道,必與賊遇,不如徑路直指鐘山,出其不意。」綸從之。眾軍奄至,賊徒大駭,分為三道攻綸,綸與戰,大破之,斬首千餘級。翌日,賊又來攻,相持日晚,賊稍引卻,南安侯駿以數十騎馳之。賊回拒駿,駿部亂。賊因逼大軍,軍遂潰。綸至鐘山,眾裁千人,賊圍之,戰又敗,乃奔還京口。



 三年春,綸復與東揚州刺史大連等入援,至于驃騎洲。進位司空。
 臺城陷,奔禹穴。大寶元年,綸至郢州,刺史南平王恪讓州於綸,綸不受,乃上綸為假黃鉞、都督中外諸軍事。綸於是置百官,改廳事為正陽殿。數有災怪,綸甚惡之。時元帝圍河東王譽於長沙既久,內外斷絕,綸聞其急,欲往救之,為軍糧不繼,遂止。乃與世祖書曰:伏以先朝聖德,孝治天下,九親雍睦,四表無怨,誠為國政,實亦家風。唯餘與爾,同奉神訓,宜敦旨喻,共承無改。且道之斯美,以和為貴,況天時地利,不及人和,豈可手足肱支,自相屠害。日者聞譽專情失訓,以幼陵長,湘、峽之內,遂至交鋒。方等身遇亂兵,斃於行陣,殞于吳局。方此非冤,聞問
 號怛,惟增摧憤,念以兼悼,當何可稱。吾在州所居遙隔,雖知其狀,未喻所然。及屆此籓,備加覿訪,咸云譽應接多替,兵糧閉壅;弟教亦不悛,故興師以伐。譽未識大體,意斷所行,雖存急難,豈知竊思。不能禮爭,復以兵來。蕭墻興變,體親成敵,一朝至此,能不鳴呼。既有書問,雲雨傳流,噂沓其間,委悉無因詳究。



 方今社稷危恥,創巨痛深,人非禽蟲,在知君父。即日大敵猶強,天仇未雪,餘爾昆季,在外三人,如不匡難,安用臣子。唯應剖心嘗膽,泣血枕戈,感誓蒼穹,憑靈宗祀,晝謀夕計,共思匡復。至於其餘小忿,或宜寬貸。誠復子憾須臾,將奈國冤未逞。正
 當輕重相推,小大易奪,遣無益之情,割下流之悼,弘豁以理,通識勉之。今已喪鐘山,復誅猶子,將非揚湯止沸,吞冰療寒。若以譽之無道,近遠同疾,弟復效尤,攸非獨罪。幸寬於眾議,忍以事寧。如使外寇未除,家禍仍構,料今訪古,未或弗亡。



 夫征戰之理,義在克勝;至於骨肉之戰,愈勝愈酷,捷則非功,敗則有喪,勞兵損義,虧失多矣。侯景之軍所以未窺江外者,正為籓屏盤固,宗鎮強密。若自相魚肉,是代景行師。景便不勞兵力,坐致成效,醜徒聞此,何快如之!又莊鐵小豎作亂,久挾觀寧、懷安二侯,以為名號,當陽有事克掣,殊廢備境。第聞征伐,復致
 分兵,便是自於瓜州至於湘、雍,莫非戰地,悉以勞師。侯景卒承虛藉釁,浮江豕突,豈不表裹成虞,首尾難救?可為寒心,其事已切。弟若苦陷洞庭,兵戈不戢,雍州疑迫,何以自安?必引進魏軍,以求形援。侯景事等內癰,西秦外同瘤腫。直置關中,已為咽氣,況復貪狼難測,勢必侵吞。弟若不安,家國去矣。吾非有深鑒,獨能弘理,正是採藉風謠,博參物論,咸以為疑,皆欲解體故耳。



 自我國五十許年,恩格玄穹,德彌赤縣,雖有逆難,未亂邕熙。溥天率土,忠臣憤慨,比屋罹禍,忠義奮發,無不抱甲負戈,衝冠裂眥,咸欲刃於侯景腹中,所須兵主唱耳。今人皆
 樂死,赴者如流。弟英略振遠,雄伯當代,唯德唯藝,資文資武,拯溺濟難,朝野咸屬,一匡九合,非弟而誰?豈得自違物望,致招群讀!其間患難,具如所陳。斯理皎然,無勞請箸;驗之以實,寧須確引。吾所以間關險道,出自東川,政謂上游諸籓,必連師狎至,庶以殘命,預在行間;及到九江,安北兄遂溯流更上,全由餼饋懸斷,卒食半菽,阻以菜色,無因進取。侯景方延假息,復緩誅刑,信增號憤,啟處無地。計瀟湘穀粟,猶當紅委,若阻弟嚴兵,唯事交切,至於運轉,恐無暇發遣。即日萬心慊望,唯在民天,若遂等西河,時事殆矣!必希令弟豁照茲途,解汨川之圍,
 存社稷之計,使其運輸糧儲,應贍軍旅,庶協力一舉,指日寧泰。宗廟重安,天下清復,推弟之功,豈非幸甚。吾才懦兵寡,安能為役,所寄令弟,庶得申情,朝聞夕死,萬殞何恨。聊陳聞見,幸無怪焉。臨紙號迷,諸失次緒。



 世祖復書,陳河東有罪,不可解圍之狀。綸省書流涕曰:「天下之事,一至於斯!」左右聞之,莫不掩泣。於是大修器甲,將討侯景。元帝聞其彊盛,乃遣王僧辯帥舟師一萬以逼綸,綸將劉龍武等降僧辯,綸軍潰,遂與子躓等十餘人輕舟走武昌。



 時綸長史韋質、司馬姜律先在于外,聞綸敗,馳往迎之。於是復收散卒,屯于齊昌郡,將引魏軍共攻
 南陽。侯景將任約聞之,使鐵騎二百襲綸,綸無備,又敗走定州。定州刺史田龍祖迎綸,綸以龍祖荊鎮所任,懼為所執,復歸齊昌。行至汝南,西魏所署汝南城主李素者,綸之故吏,聞綸敗,開城納之。綸乃修浚城池,收集士卒,將攻竟陵。西魏安州刺史馬岫聞之,報于西魏,西魏遣大將軍楊忠、儀同侯幾通率眾赴焉。二年二月,忠等至于汝南,綸嬰城自守。會天寒大雪,忠等攻不能克,死者甚眾。後李素中流矢卒,城乃陷。忠等執綸,綸不為屈,遂害之。投于江岸,經日顏色不變,鳥獸莫敢近焉。時年三十三。百姓憐之,為立祠廟,後世祖追謚曰攜。



 長子
 堅,字長白。大同元年,以例封汝南侯,邑五百戶。亦善草隸,性頗庸短。侯景圍城,堅屯太陽門,終日蒲飲,不撫軍政。吏士有功,未嘗申理,疫癘所加,亦不存恤,士咸憤怨。太清三年三月,堅書佐董勛華、白曇朗等以繩引賊登樓,城遂陷,堅遇害。



 弟確,字仲正。少驍勇,有文才。大同二年,封為正階侯,邑五百戶,後徙封永安。常在第中習騎射,學兵法,時人皆以為狂。左右或以進諫,確曰:「聽吾為國家破賊,使汝知之。」除秘書丞,太子中舍人。



 鐘山之役,確苦戰,所向披靡,群虜憚之。確每臨陣對敵,意氣詳贍。帶甲據鞍,自朝及夕,馳驟往反,不以為勞,諸將服其壯勇。
 及侯景乞盟,確在外,慮為後患,啟求召確入城。詔乃召確為南中郎將、廣州刺史,增封二千戶。確知此盟多貳,城必淪沒,因欲南奔。攜王聞之,逼確使入,確猶不肯。攜王流涕謂曰:「汝欲反邪!」時臺使周石珍在坐,確謂石珍曰:「侯景雖云欲去,而不解長圍,以意而推,其事可見。今召我入,未見其益也。」石珍曰:「敕旨如此,侯豈得辭?」確執意猶堅,攜王大怒,謂趙伯超曰:「譙州,卿為我斬之,當賚首赴闕。」伯超揮刃眄確曰:「我識君耳,刀豈識君?」確於是流涕而出,遂入城。及景背盟復圍城,城陷,確排闥入,啟高祖曰:「城已陷矣。」高祖曰:「猶可一戰不?」對曰:「不可。臣向
 者親格戰,勢不能禁,自縋下城,僅得至此。」高祖歎曰:「自我得之,自我失之,亦復何恨。」乃使確為慰勞文。



 確既出見景,景愛其膂力,恆令在左右。後從景行,見天上飛鳶,群虜爭射不中,確射之,應弦而落。賊徒忿嫉,咸勸除之。先是攜王遣人密導確,確謂使者曰:「侯景輕佻,可一夫力致,確不惜死,正欲手刃之;但未得其便耳。卿還啟家王,願勿以為念也。」事未遂而為賊所害。



 史臣曰:自周、漢廣樹籓屏,固本深根;高祖之封建,將遵古制也。南康、廬陵並以宗室之貴,據磐石之重,績以孝著,續以勇聞。綸聰警有才學,性險躁,屢以罪黜,及太清
 之亂,忠孝獨存,斯可嘉矣。



\end{pinyinscope}