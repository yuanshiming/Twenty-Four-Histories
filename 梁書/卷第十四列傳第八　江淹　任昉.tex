\article{卷第十四列傳第八 江淹 任昉}

\begin{pinyinscope}

 江淹,字文通,濟陽考城人也。少孤貧好學,沉靜少交遊。起家南徐州從事,轉奉朝請。宋建平王景素好士,淹隨景素在南兗州。廣陵令郭彥文得罪,辭連淹,繫州獄。淹獄中上書曰:昔者賤臣叩心,飛霜擊於燕地;庶女告天,振風襲於齊臺。下官每讀其書,未嘗不廢卷流涕。何者?士有一定之論,女有不易之行。信而見疑,貞而為戮,是
 以壯夫義士伏死而不顧者此也。下官聞仁不可恃,善不可依,始謂徒語,乃今知之。伏願大王暫停左右,少加憐鑒。



 下官本蓬戶桑樞之民,布衣韋帶之士,退不飾《詩書》以驚愚,進不買名聲於天下。日者謬得升降承明之闕,出入金華之殿,何嘗不局影凝嚴,側身扃禁者乎?竊慕大王之義,為門下之賓,備鳴盜淺術之餘,豫三五賤伎之末。大王惠以恩光,眄以顏色。實佩荊卿黃金之賜,竊感豫讓國士之分矣。常欲結纓伏劍,少謝萬一,剖心摩踵,以報所天。不圖小人固陋,坐貽謗缺,迹墜昭憲,身限幽圄。履影弔心,酸鼻痛骨。下官聞虧名為辱,虧形次
 之,是以每一念來,忽若有遺。加以涉旬月,迫季秋,天光沉陰,左右無色。身非木石,與獄吏為伍。此少卿所以仰天搥心,泣盡而繼之以血者也。下官雖乏鄉曲之譽,然嘗聞君子之行矣。其上則隱於簾肆之間,臥於巖石之下;次則結綬金馬之庭,高議雲臺之上;次則虜南越之君,係單于之頸:俱啟丹冊,並圖青史。寧當爭分寸之末,競刀錐之利哉!然下官聞積毀銷金,積讒糜骨。古則直生取疑於盜金,近則伯魚被名於不義。彼之二才,猶或如此;況在下官,焉能自免。昔上將之恥,絳侯幽獄;名臣之羞,史遷下室,如下官尚何言哉!夫魯連之智,辭祿而
 不反;接輿之賢,行歌而忘歸。子陵閉關於東越,仲蔚杜門於西秦,亦良可知也。若使下官事非其虛,罪得其實,亦當鉗口吞舌,伏匕首以殞身,何以見齊魯奇節之人,燕趙悲歌之士乎?



 方今聖歷欽明,天下樂業,青雲浮雒,榮光塞河。西洎臨洮、狄道,北距飛狐、陽原,莫不浸仁沐義,照景飲醴。而下官抱痛圜門,含憤獄戶,一物之微,有足悲者。仰惟大王少垂明白,則梧丘之魂,不愧於沉首;鵠亭之鬼,無恨於灰骨。不任肝膽之切,敬因執事以聞。此心既照,死且不朽。



 景素覽書,即日出之。尋舉南徐州秀才,對策上第,轉巴陵王國左常侍。景素為荊州,淹從
 之鎮。少帝即位,多失德。景素專據上流,咸勸因此舉事。淹每從容諫曰:「流言納禍,二叔所以同亡;抵局銜怨,七國於焉俱斃。殿下不求宗廟之安,而信左右之計,則復見麋鹿霜露棲於姑蘇之臺矣。」景素不納。及鎮京口,淹又為鎮軍參軍事,領南東海郡丞。景素與腹心日夜謀議,淹知禍機將發,乃贈詩十五首以諷焉。



 會南東海太守陸澄丁艱,淹自謂郡丞應行郡事,景素用司馬柳世隆。淹固求之,景素大怒,言於選部,黜為建安吳興令。淹在縣三年。昇明初,齊帝輔政,聞其才,召為尚書駕部郎、驃騎參軍事。俄而荊州刺史沈攸之作亂,高帝謂淹曰:「
 天下紛紛若是,君謂何如?」淹對曰:「昔項彊而劉弱,袁眾而曹寡,羽號令諸侯,卒受一劍之辱,紹跨躡四州,終為奔北之虜。此謂『在德不在鼎』。公何疑哉?」帝曰:「聞此言者多矣,試為慮之。」淹曰:「公雄武有奇略,一勝也;寬容而仁恕,二勝也;賢能畢力,三勝也;民望所歸,四勝也;奉天子而伐叛逆,五勝也。彼志銳而器小,一敗也;有威而無恩,二敗也;士卒解體,三敗也;搢紳不懷,四敗也;懸兵數千里,而無同惡相濟,五敗也。故雖豺狼十萬,而終為我獲焉。」帝笑曰:「君談過矣。」是時軍書表記,皆使淹具草。相國建,補記室參軍事。建元初,又為驃騎豫章王記室,帶東武
 令,參掌詔冊,並典國史。尋遷中書侍郎。永明初,遷驍騎將軍,掌國史。出為建武將軍、廬陵內史。視事三年,還為驍騎將軍,兼尚書左丞,尋復以本官領國子博士。少帝初,以本官兼御史中丞。



 時明帝作相,因謂淹曰:「君昔在尚書中,非公事不妄行,在官寬猛能折衷;今為南司,足以震肅百僚。」淹答曰:「今日之事,可謂當官而行,更恐才劣志薄,不足以仰稱明旨耳。」於是彈中書令謝朏,司徒左長史王繢、護軍長史庾弘遠,並以久疾不預山陵公事;又奏前益州刺史劉悛、梁州刺史陰智伯,並贓貨巨萬,輒收付廷尉治罪。臨海太守沈昭略、永嘉太守庾曇
 隆,及諸郡二千石並大縣官長,多被劾治,內外肅然。明帝謂淹曰:「宋世以來,不復有嚴明中丞,君今日可謂近世獨步。」



 明帝即位,為車騎臨海王長史。俄除廷尉卿,加給事中,遷冠軍長史,加輔國將軍。出為宣城太守,將軍如故。在郡四年,還為黃門侍郎、領步兵校尉,尋為秘書監。永元中,崔慧景舉兵圍京城,衣冠悉投名刺,淹稱疾不往。及事平,世服其先見。



 東昏末,淹以祕書監兼衛尉,固辭不獲免,遂親職。謂人曰:「此非吾任,路人所知,正取吾空名耳。且天時人事,尋當翻覆。孔子曰:『有文事者必有武備。』臨事圖之,何憂之有?」頃之,又副領軍王瑩。及義
 師至新林,淹微服來奔,高祖板為冠軍將軍,秘書監如故,尋兼司徒左長史。中興元年,遷吏部尚書。二年,轉相國右長史,冠軍將軍如故。



 天監元年,為散騎常侍、左衛將軍,封臨沮縣開國伯,食邑四百戶。淹乃謂子弟曰:「吾本素宦,不求富貴,今之忝竊,遂至於此。平生言止足之事,亦以備矣。人生行樂耳,須富貴何時。吾功名既立,正欲歸身草萊耳。」其年,以疾遷金紫光祿大夫,改封醴陵侯。四年卒,時年六十二。高祖為素服舉哀。賻錢三萬,布五十匹。謚曰憲伯。



 淹少以文章顯,晚節才思微退,時人皆謂之才盡。凡所著述百餘篇,自撰為前後集,并《齊史》
 十志,並行於世。



 子筼襲封嗣,自丹陽尹丞為長城令,有罪削爵。普通四年,高祖追念淹功,復封筼吳昌伯,邑如先。



 任昉,字彥昇,樂安博昌人,漢御史大夫敖之後也。父遙,齊中散大夫。遙妻裴氏,嘗晝寢,夢有彩旗蓋四角懸鈴,自天而墜,其一鈴落入裴懷中,心悸動,既而有娠,生昉。身長七尺五寸。幼而好學,早知名。宋丹陽尹劉秉辟為主簿。時昉年十六,以氣忤秉子。久之,為奉朝請,舉兗州秀才,拜太常博士,遷征北行參軍。



 永明初,衛將軍王儉領丹陽尹,復引為主簿。儉雅欽重昉,以為當時無輩。遷司
 徒刑獄參軍事,入為尚書殿中郎,轉司徒竟陵王記室參軍,以父憂去職。性至孝,居喪盡禮。服闋,續遭母憂,常廬于墓側,哭泣之地,草為不生。服除,拜太子步兵校尉、管東宮書記。



 初,齊明帝既廢鬱林王,始為侍中、中書監、驃騎大將軍、開府儀同三司、揚州刺史、錄尚書事,封宣城郡公,加兵五千,使昉具表草。其辭曰:「臣本庸才,智力淺短。太祖高皇帝篤猶子之愛,降家人之慈;世祖武皇帝情等布衣,寄深同氣。武皇大漸,實奉詔言。雖自見之明,庸近所蔽,愚夫一至,偶識量己,實不忍自固於綴衣之辰,拒違於玉几之側,遂荷顧託,導揚末命。雖嗣君棄
 常,獲罪宣德,王室不造,職臣之由。何者?親則東牟,任惟博陸,徒懷子孟社稷之對,何救昌邑爭臣之譏。四海之議,於何逃責?陵土未乾,訓誓在耳,家國之事,一至於斯,非臣之尤,誰任其咎!將何以肅拜高寢,虔奉武園?悼心失圖,泣血待旦。寧容復徼榮於家恥,宴安於國危。驃騎上將之元勳,神州儀刑之列岳,尚書是稱司會,中書實管王言。且虛飾寵章,委成禦侮,臣知不愜,物誰謂宜。但命輕鴻毛,責重山岳,存沒同歸,毀譽一貫。辭一官不減身累,增一職已黷朝經。便當自同體國,不為飾讓。至於功均一匡,賞同千室,光宅近甸,奄有全邦,殞越為期,不
 敢聞命,亦願曲留降鑒,即垂聽許。鉅平之懇誠必固,永昌之丹慊獲申,乃知君臣之道,綽有餘裕,茍曰易昭,敢守難奪。」帝惡其辭斥,甚慍昉,由是終建武中,位不過列校。



 昉雅善屬文,尤長載筆,才思無窮,當世王公表奏,莫不請焉。昉起草即成,不加點竄。沈約一代詞宗,深所推挹。明帝崩,遷中書侍郎。永元末,為司徒右長史。



 高祖克京邑,霸府初開,以昉為驃騎記室參軍。始高祖與昉遇竟陵王西邸,從容謂昉曰:「我登三府,當以卿為記室。」昉亦戲高祖曰:「我若登三事,當以卿為騎兵。」謂高祖善騎也。至是故引昉,符昔言焉。昉奉箋曰:「伏承以今月令辰,肅
 膺典策,德顯功高,光副四海,含生之倫,庇身有地;況昉受教君子,將二十年,咳唾為恩,眄睞成飾,小人懷惠,顧知死所。昔承清宴,屬有緒言,提挈之旨,形乎善謔,豈謂多幸,斯言不渝。雖情謬先覺,而迹淪驕餌,湯沐具而非吊,大廈構而相歡。明公道冠二儀,勳超邃古,將使伊周奉轡,桓文扶轂,神功無紀,化物何稱。府朝初建,俊賢驤首,惟此魚目,唐突璵璠。顧己循涯,實知塵忝,千載一逢,再造難答。雖則殞越,且知非報。」



 梁臺建,禪讓文誥,多昉所具。高祖踐阼,拜黃門侍郎,遷吏部郎中,尋以本官掌著作。天監二年,出為義興太守。在任清潔,兒妾食麥而
 已。友人彭城到溉,溉弟洽,從昉共為山澤游。及被代登舟,止有米五斛。既至無衣,鎮軍將軍沈約遣裙衫迎之。重除吏部郎中,參掌大選,居職不稱。尋轉御史中丞,秘書監,領前軍將軍。自齊永元以來,秘閣四部,篇卷紛雜,昉手自讎校,由是篇目定焉。



 六年春,出為寧朔將軍、新安太守。在郡不事邊幅,率然曳杖,徒行邑郭,民通辭訟者,就路決焉。為政清省,吏民便之。視事期歲,卒於官舍,時年四十九。闔境痛惜,百姓共立祠堂於城南。高祖聞問,即日舉哀,哭之甚慟。追贈太常卿,謚曰敬子。



 昉好交結,獎進士友,得其延譽者,率多升擢,故衣冠貴遊,莫不
 爭與交好,坐上賓客,恒有數十。時人慕之,號曰任君,言如漢之三君也。陳郡殷芸與建安太守到溉書曰:「哲人云亡,儀表長謝。元龜何寄?指南誰託?」其為士友所推如此。昉不治生產,至乃居無室宅。世或譏其多乞貸,亦隨復散之親故。昉常歎曰:「知我亦以叔則,不知我亦以叔則。」昉墳籍無所不見,家雖貧,聚書至萬餘卷,率多異本。昉卒後,高祖使學士賀縱共沈約勘其書目,官所無者,就昉家取之。昉所著文章數十萬言,盛行於世。



 初,昉立於士大夫間,多所汲引,有善己者則厚其聲名。及卒,諸子皆幼,人罕贍恤之。平原劉孝標為著論曰:客問主人
 曰:「朱公叔《絕交論》,為是乎?為非乎?」主人曰:「客奚此之問?」客曰:「夫草蟲鳴則阜螽躍,雕虎嘯而清風起。故絪縕相感,霧涌雲蒸;嚶鳴相召,星流電激。是以王陽登則貢公喜,罕生逝而國子悲。且心同琴瑟,言鬱郁於蘭簹,道葉膠漆,志婉孌於塤篪。聖賢以此鏤金版而鐫盤盂,書玉牒而刻鐘鼎。若匠人輟成風之妙巧,伯牙息流波之雅引。范、張款款於下泉,尹、班陶陶於永夕。駱驛縱橫,煙霏雨散,皆巧歷所不知,心計莫能測。而朱益州汨敘,越謨訓,捶直切,絕交遊,視黔首以鷹鸇,媲人倫於豺虎。蒙有猜焉,請辨其惑。」



 主人欣然曰:「客所謂撫絃徽音,未達
 燥濕變響;張羅沮澤,不睹鵠雁高飛。蓋聖人握金鏡,闡風烈,龍驤蠖屈,從道汙隆。日月聯璧,歎亹亹之弘致;雲飛電薄,顯棣華之微旨。若五音之變化,濟九成之妙曲。此朱生得玄珠於赤水,謨神睿而為言。至夫組織仁義,琢磨道德,歡其愉樂,恤其陵夷。寄通靈臺之下,遺迹江湖之上,風雨急而不輟其音,霜雪零而不渝其色,斯賢達之素交,歷萬古而一遇。逮叔世民訛,狙詐飆起,谿谷不能踰其險,鬼神無以究其變,競毛羽之輕,趨錐刀之末。於是素交盡,利交興,天下蚩蚩,鳥驚雷駭。然利交同源,派流則異,較言其略,有五術焉:「若其寵鈞董、石,權壓
 梁、竇。雕刻百工,爐錘萬物,吐漱興雲雨,呼吸下霜露,九域聳其風塵,四海疊其熏灼。靡不望影星奔,藉響川鶩,雞人始唱,鶴蓋成陰,高門旦開,流水接軫。皆願摩頂至踵,隳膽抽腸,約同要離焚妻子,誓徇荊卿湛七族。是曰勢交,其流一也。



 「富埒陶、白,貲巨程、羅,山擅銅陵,家藏金穴,出平原而聯騎,居里閈而鳴鐘。則有窮巷之賓,繩樞之士,冀宵燭之末光,邀潤屋之微澤,魚貫鳧踴,颯沓鱗萃,分雁鶩之稻粱,沾玉斝之餘瀝。銜恩遇,進款誠,援青松以示心,指白水而旌信。是曰賄交,其流二也。



 「陸大夫燕喜西都,郭有道人倫東國,公卿貴其籍甚,搢紳羨其
 登仙。加以頤蹙頞,涕唾流沫,騁黃馬之劇談,縱碧雞之雄辯,敘溫燠則寒谷成暄,論嚴枯則春叢零葉,飛沉出其顧指,榮辱定其一言。於是弱冠王孫,綺紈公子,道不絓於通人,聲未遒於雲閣,攀其鱗翼,丐其餘論,附騏驥之髦端,軼歸鴻於碣石。是曰談交,其流三也。



 「陽舒陰慘,生民大情,憂合歡離,品物恒性。故魚以泉涸而呴沫,鳥因將死而悲鳴。同病相憐,綴河上之悲曲;恐懼置懷,昭《谷風》之盛典。斯則斷金由於湫隘,刎頸起於苫蓋。是以伍員濯溉於宰嚭,張王撫翼於陳相。是曰窮交,其流四也。



 「馳鶩之俗,澆薄之倫,無不操權衡,秉纖纊。衡所以
 揣其輕重,纊所以屬其鼻息。若衡不能舉,纊不能飛,雖顏、冉龍翰,鳳雛曾、史,蘭熏雪白,舒、向金玉,淵海卿、雲,黼黻河漢,視若遊塵。遇同土梗,莫肯費其半菽,罕有落其一毛。若衡重錙銖,纊微彯撇,雖共工之蒐慝,驩兜之掩義,南荊之跋扈,東陵之巨猾,皆為匍匐委蛇,折枝舐痔,金膏翠羽將其意,脂韋便辟導其誠。故輪蓋所遊,必非夷、惠之室;苞苴所入,實行張、霍之家。謀而後動,芒毫寡忒。是曰量交,其流五也。



 「凡斯五交,義同賈鬻,故桓譚譬之於闤闠,林回喻之於甘醴。夫寒暑遞進,盛衰相襲,或前榮而後瘁,或始富而終貧,或初存而末亡,或古約而
 今泰,循環翻覆,迅若波瀾。此則徇利之情未嘗異,變化之道不得一。由是觀之,張、陳所以凶終,蕭、朱所以隙末,斷焉可知矣。而翟公方規規然勒門以箴客,何所見之晚乎?



 「然因此五交,是生三釁:敗德殄義,禽獸相若,一釁也;難固易攜,仇訟所聚,二釁也;名陷饕餮,貞介所羞,三釁也。古人知三釁之為梗,懼五交之速尤。故王丹威子以檟楚,朱穆昌言而示絕,有旨哉!



 「近世有樂安任昉,海內髦傑,早綰銀黃,夙招民譽。遒文麗藻,方駕曹、王;英特俊邁,聯衡許、郭。類田文之愛客,同鄭莊之好賢。見一善則盱衡扼腕,遇一才則揚眉抵掌。雌黃出其脣吻,朱紫
 由其月旦。於是冠蓋輻湊,衣裳雲合,輜軿擊轊,坐客恒滿。蹈其閫閾,若升闕里之堂;入其奧隅,謂登龍門之阪。至於顧盼增其倍價,剪拂使其長鳴,彯組雲臺者摩肩,趨走丹墀者疊迹。莫不締恩狎,結綢繆,想惠、莊之清塵,庶羊、左之徽烈。及瞑目東越,歸骸雒浦,繐帳猶懸,門罕漬酒之彥;墳未宿草,野絕動輪之賓。藐爾諸孤,朝不謀夕,流離大海之南,寄命瘴癘之地。自昔把臂之英,金蘭之友,曾無羊舌下泣之仁,寧慕郈成分宅之德。嗚呼!世路險巇,一至於此!太行孟門,寧云嶄絕。是以耿介之士,疾其若斯,裂裳裹足,棄之長祇。獨立高山之頂,懽與麋
 鹿同群,皦皦然絕其雰濁,誠恥之也,誠畏之也。」



 昉撰《雜傳》二百四十七卷,《地記》二百五十二卷,文章三十三卷。



 昉第四子東里,頗有父風,官至尚書外兵郎。



 陳吏部尚書姚察曰:觀夫二漢求賢,率先經術;近世取人,多由文史。二子之作,辭藻壯麗,允值其時。淹能沉靜,昉持內行,並以名位終始,宜哉。江非先覺,任無舊恩,則上秩顯贈,亦末由也已。



\end{pinyinscope}