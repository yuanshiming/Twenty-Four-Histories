\article{卷第四十八列傳第四十二 儒林伏曼容 何佟之 範縝 嚴植之 賀蒨 子革 司馬筠 卞華崔靈恩 孔僉 盧廣 沈峻 太史叔明 孔子袪 皇侃}

\begin{pinyinscope}

 漢氏承秦燔書,大弘儒訓,太學生徒,動以萬數,郡國黌舍,悉皆充滿。學於山澤者,至或就為列肆,其盛也如是。
 漢末喪亂,其道遂衰。魏正始以後,仍尚玄虛之學,為儒者蓋寡。時荀抃、摯虞之徒,雖刪定新禮,改官職,未能易俗移風。自是中原橫潰,衣冠殄盡;江左草創,日不暇給;以迄于宋、齊。國學時或開置,而勸課未博,建之不及十年,蓋取文具,廢之多歷世祀,其棄也忽諸。鄉里莫或開館,公卿罕通經術。朝廷大儒,獨學而弗肯養眾;後生孤陋,擁經而無所講習。三德六藝,其廢久矣。



 高祖有天下,深愍之,詔求碩學,治五禮,定六律,改斗歷,正權衡。天監四年,詔曰:「二漢登賢,莫非經術,服膺雅道,名立行成。魏、晉浮蕩,儒教淪歇,風節罔樹,抑此之由。朕日昃罷朝,思
 聞俊異,收士得人,實惟酬獎。可置《五經》博士各一人,廣開館宇,招內後進。」於是以平原明山賓、吳興沈峻、建平嚴植之、會稽賀蒨補博士,各主一館。館有數百生,給其餼廩。其射策通明者,即除為吏。十數月間,懷經負笈者雲會京師。又選遣學生如會稽雲門山,受業於廬江何胤。分遣博士祭酒,到州郡立學。七年,又詔曰:「建國君民,立教為首,砥身礪行,由乎經術。朕肇基明命,光宅區宇,雖耕耘雅業,傍闡藝文,而成器未廣,志本猶闕。非以熔範貴遊,納諸軌度;思欲式敦讓齒,自家刑國。今聲訓所漸,戎夏同風。宜大啟癢斅,博延胄子,務彼十倫,弘此三
 德,使陶鈞遠被,微言載表。」於是皇太子、皇子、宗室、王侯始就業焉。高祖親屈輿駕,釋奠於先師先聖,申之以宴語,勞之以束帛,濟濟焉,洋洋焉,大道之行也如是。其伏曼容、何佟之、范縝,有舊名於世;為時儒者,嚴植之、賀蒨等首膺茲選。今並綴為《儒林傳》云。



 伏曼容,字公儀,平昌安丘人。曾祖滔,晉著作郎。父胤之,宋司空主簿。曼容早孤,與母兄客居南海。少篤學,善《老》、《易》,倜儻好大言,常云:「何晏疑《易》中九事。以吾觀之,晏了不學也,故知平叔有所短。」聚徒教授以自業。為驃騎行參軍。宋明帝好《周易》,集朝臣於清暑殿講,詔曼容執經。
 曼容素美風采,帝恆以方嵇叔夜,使吳人陸探微畫叔夜像以賜之。遷司徒參軍。袁粲為丹陽尹,請為江寧令,入拜尚書外兵郎。昇明末,為輔國長史、南海太守。齊初,為通直散騎侍郎。永明初,為太子率更令,侍皇太子講。衛將軍王儉深相交好,令與河內司馬憲、吳郡陸澄共撰《喪服義》,既成,又欲與之定禮樂。會儉薨,遷中書侍郎、大司馬諮議參軍,出為武昌太守。建武中,入拜中散大夫。時明帝不重儒術,曼容宅在瓦官寺東,施高坐於聽事,有賓客輒升高坐為講說,生徒常數十百人。梁臺建,以曼容舊儒,召拜司馬,出為臨海太守。天監元年,卒官,
 時年八十二。為《周易》、《毛詩》、《喪服集解》、《老》、《莊》、《論語義》。子芃,在《良吏傳》。



 何佟之,字士威,廬江灊人,豫州刺史惲六世孫也。祖劭之,宋員外散騎常侍。父歆,齊奉朝請。佟之少好《三禮》,師心獨學,彊力專精,手不輟卷,讀《禮》論二百篇,略皆上口。時太尉王儉為時儒宗,雅相推重。起家揚州從事,仍為總明館學士,頻遷司徒車騎參軍事、尚書祠部郎。齊建武中,為鎮北記室參軍,侍皇太子講,領丹陽邑中正。時步兵校尉劉獻、徵士吳苞皆已卒,京邑碩儒,唯佟之而已。佟之明習事數,當時國家吉凶禮則,皆取決焉,名重
 於世。歷步兵校尉、國子博士,尋遷驃騎諮議參軍,轉司馬。永元末,京師兵亂,佟之常集諸生講論,孜孜不怠。中興初,拜驍騎將軍。高祖踐阼,尊重儒術,以佟之為尚書左丞。是時百度草創,佟之依《禮》定議,多所裨益。天監二年,卒官,年五十五。高祖甚悼惜,將贈之官;故事左丞無贈官者,特詔贈黃門侍郎,儒者榮之。所著文章、《禮義》百許篇。子:朝隱、朝晦。



 范縝,字子真,南鄉舞陰人也。晉安北將軍汪六世孫。祖璩之,中書郎。父蒙,早卒。縝少孤貧,事母孝謹。年未弱冠,聞沛國劉聚眾講說。始往從之,卓越不群而勤學,
 獻甚奇之,親為之冠。在門下積年,去來歸家,恒芒矰布衣,徒行於路。門多車馬貴游,縝在其門,聊無恥愧。既長,博通經術,尤精《三禮》。性質直,好危言高論,不為士友所安。唯與外弟蕭琛相善,琛名曰口辯,每服縝簡詣。



 起家齊寧蠻主簿,累遷尚書殿中郎。永明年中,與魏氏和親,歲通聘好,特簡才學之士,以為行人。縝及從弟雲、蕭琛、瑯邪顏幼明、河東裴昭明相繼將命,皆著名鄰國。於時竟陵王子良盛招賓客,縝亦預焉。建武中,遷領軍長史。出為宜都太守,母憂去職,歸居于南州。義軍至,縝墨絰來迎。高祖與縝有西邸之舊,見之甚悅。及建康城平,以
 縝為晉安太守,在郡清約,資公祿而已。視事四年,徵為尚書左丞。縝去還,雖親戚無所遺,唯餉前尚書令王亮。縝仕齊時,與亮同臺為郎,舊相友,至是亮被擯棄在家。縝自迎王師,志在權軸,既而所懷未滿,亦常怏怏,故私相親結,以矯時云。後竟坐亮徙廣州,語在亮傳。



 初,縝在齊世,嘗侍竟陵王子良。子良精信釋教,而縝盛稱無佛。子良問曰:「君不信因果,世間何得有富貴,何得有貧賤?」縝答曰:「人之生譬如一樹花,同發一枝,俱開一蒂,隨風而墮,自有拂簾幌墜於茵席之上,自有關籬牆落於溷糞之側。墜茵席者,殿下是也;落糞溷者,下官是也。貴賤
 雖復殊途,因果竟在何處?」子良不能屈,深怪之。縝退論其理,著《神滅論》曰:或問予云:「神滅,何以知其滅也?」答曰:「神即形也,形即神也;是以形存則神存,形謝則神滅也。」



 問曰:「形者無知之稱,神者有知之名。知與無知,即事有異,神之與形,理不容一,形神相即,非所聞也。」答曰:「形者神之質,神者形之用;是則形稱其質,神言其用;形之與神,不得相異也。」



 問曰:「神故非質,形故非用,不得為異,其義安在?」答曰:「名殊而體一也。」



 問曰:「名既已殊,體何得一?」答曰:「神之於質,猶利之於刀;形之於用,猶刀之於利;利之名非刀也,刀之名非利也。然而捨利無刀,捨刀無利。未聞刀沒
 而利存,豈容形亡而神在?」



 問曰:「刀之與利,或如來說;形之與神,其義不然。何以言之?木之質無知也,人之質有知也;人既有如木之質,而有異木之知,豈非木有一、人有二邪?」答曰:「異哉言乎!人若有如木之質以為形,又有異木之知以為神,則可如來論也。今人之質,質有知也;木之質,質無知也。人之質非木質也,木之質非人質也,安有如木之質而復有異木之知哉!」



 問曰:「人之質所以異木質者,以其有知耳。人而無知,與木何異?」答曰:「人無無知之質,猶木無有知之形。」



 問曰:「死者之形骸,豈非無知之質邪?」答曰:「是無人質。」



 問曰:「若然者,人果有如木之
 質,而有異木之知矣。」答曰:「死者如木,而無異木之知;生者有異木之知,而無如木之質也。」



 問曰:「死者之骨骼,非生之形骸邪?」答曰:「生形之非死形,死形之非生形,區已革矣。安有生人之形骸,而有死人之骨骼哉?」



 問曰:「若生者之形骸,非死者之骨骼;非死者之骨骼,則應不由生者之形骸;不由生者之形骸,則此骨骼從何而至此邪?」答曰:「是生者之形骸,變為死者之骨骼也。」



 問曰:「生者之形骸雖變為死者之骨骼,豈不因生而有死?則知死體猶生體也。」答曰:「如因榮木變為枯木,枯木之質,寧是榮木之體!」



 問曰:「榮體變為枯體,枯體即是榮體;絲體變為
 縷體,縷體即是絲體,有何別焉?」答曰:「若枯即是榮,榮即是枯,應榮時凋零,枯時結實也。又榮木不應變為枯木,以榮即枯,無所復變也。榮枯是一,何不先枯後榮?要先榮後枯,何也?絲縷之義,亦同此破。」



 問曰:「生形之謝,便應豁然都盡。何故方受死形,綿歷未已邪?」答曰:「生滅之體,要有其次故也。夫惸而生者必惸而滅,漸而生者必漸而滅。惸而生者,飄驟是也;漸而生者,動植是也。有惸有漸,物之理也。」



 問曰:「形即是神者,手等亦是邪?」答曰:「皆是神之分也。」



 問曰:「若皆是神之分,神既能慮,手等亦應能慮也?」答曰:「手等亦應能有痛癢之知,而無是非之慮。」



 問
 曰:「知之與慮,為一為異?」答曰:「知即是慮。淺則為知,深則為慮。」



 問曰:「若爾,應有二慮;慮既有二,神有二乎?」答曰:「人體惟一,神何得二。」



 問曰:「若不得二,安有痛癢之知,復有是非之慮?」答曰:「如手足雖異,總為一人。是非痛癢雖復有異,亦總為一神矣。」



 問曰:「是非之慮,不關手足,當關何處?」答曰:「是非之慮,心器所主。」



 問曰:「心器是五藏之心,非邪?」答曰:「是也。」



 問曰:「五藏有何殊別,而心獨有是非之慮乎?」答曰:「七竅亦復何殊,而司用不均。」



 問曰:「慮思無方,何以知是心器所主?」答曰:「五藏各有所司,無有能慮者,是以知心為慮本。」



 問曰:「何不寄在眼等分中?」答曰:「若慮可寄於眼分,眼何故不寄於耳分邪?」



 問曰:「慮體無本,故可寄之於眼分;眼自有本,不假寄於佗分也。」答曰:「眼何故有本而慮無本;茍無本於我形,而可遍寄於異地。亦可張甲之情,寄王乙之軀;李丙之性,託趙丁之體。然乎哉?不然也。」



 問曰:「聖人形猶凡人之形,而有凡聖之殊,故知形神異矣。」答曰:「不然。金之精者能昭,穢者不能昭,有能昭之精金,寧有不昭之穢質。又豈有聖人之神而寄凡人之器,亦無凡人之神而託聖人之體。是以八采、重瞳,勛、華之容;龍顏、馬口,軒、皞之狀;形表之異也。比干之心,七竅列角;伯約之膽,其大若拳;此心器之殊也。是知聖人定分,每絕常區,非惟道革群生,
 乃亦形超萬有。凡聖均體,所未敢安。」



 問曰:「子云聖人之形必異於凡者。敢問陽貨類仲尼,項籍似大舜;舜、項、孔、陽,智革形同,其故何邪?」答曰:「氏似玉而非玉,雞類鳳而非鳳;物誠有之,人故宜爾。項、陽貌似而非實似,心器不均,雖貌無益。」



 問曰:「凡聖之殊,形器不一,可也。聖人員極,理無有二;而丘、旦殊姿,湯、文異狀,神不侔色,於此益明矣。」答曰:「聖同於心器,形不必同也,猶馬殊毛而齊逸,玉異色而均美。是以晉棘、荊和,等價連城;驊騮、騄驪,俱致千里。」



 問曰:「形神不二,既聞之矣,形謝神滅,理固宜然。敢問經云『為之宗廟,以鬼饗之』,何謂也?」答曰:「聖人之教然也。所
 以弭孝子之心,而厲偷薄之意,神而明之,此之謂矣。」



 問曰:「伯有被甲,彭生豕見,墳素著其事,寧是設教而已邪?」答曰:「妖怪茫茫,或存或亡,彊死者眾,不皆為鬼。彭生、伯有,何獨能然;乍為人豕,未必齊、鄭之公子也。」



 問曰:「《易》稱『故知鬼神之情狀,與天地相似而不違』。又曰:『載鬼一車。』其義云何?」答曰:「有禽焉,有獸焉,飛走之別也;有人焉,有鬼焉,幽明之別也。人滅而為鬼,鬼滅而為人,則未之知也。」



 問曰:「知此神滅,有何利用邪?」答曰:「浮屠害政,桑門蠹俗。風驚霧起,馳蕩不休。吾哀其弊,思拯其溺。夫竭財以赴僧,破產以趨佛,而不恤親戚,不憐窮匱者何?良由厚
 我之情深,濟物之意淺。是以圭撮涉於貧友,吝情動於顏色;千鐘委於富僧,歡意暢於容髮。豈不以僧有多稌之期,友無遺秉之報,務施闕於周急,歸德必於在己。又惑以茫昧之言,懼以阿鼻之苦,誘以虛誕之辭,欣以兜率之樂。故捨逢掖,襲橫衣,廢俎豆,列瓶缽;家家棄其親愛,人人絕其嗣續。致使兵挫於行間,吏空於官府,粟罄於惰遊,貨殫於泥木。所以姦宄弗勝,頌聲尚擁,惟此之故,其流莫已,其病無限。若陶甄稟於自然,森羅均於獨化;忽焉自有,恍爾而無,來也不禦,去也不追,乘夫天理,各安其性。小人甘其壟畝,君子保其恬素;耕而食,食不
 可窮也;蠶而衣,衣不可盡也;下有餘以奉其上,上無為以待其下,可以全生,可以匡國,可以霸君,用此道也。」



 此論出,朝野喧嘩,子良集僧難之而不能屈。



 縝在南累年,追還京。既至,以為中書郎、國子博士,卒官。文集十卷。



 子胥,字長才。傳父學,起家太學博士。胥有口辯,大同中,常兼主客郎,對接北使。遷平西湘東王諮議參軍,侍宣城王讀。出為鄱陽內史,卒於郡。



 嚴植之,字孝源,建平秭歸人也。祖欽,宋通直散騎常侍。植之少善《莊》、《老》,能玄言,精解《喪服》、《孝經》、《論語》。及長,遍治鄭氏《禮》、《周易》、《毛詩》、《左氏春秋》。性淳孝謹厚,不以所長高
 人。少遭父憂,因菜食二十三載,後得風冷疾,乃止。



 齊永明中,始起家為廬陵王國侍郎,遷廣漢王國右常侍。王誅,國人莫敢視,植之獨奔哭,手營殯殮,徒跣送喪墓所,為起冢,葬畢乃還,當時義之。建武中,遷員外郎、散騎常侍。尋為康樂侯相,在縣清白,民吏稱之。天監二年,板後軍騎兵參軍事。高祖詔求通儒治五禮,有司奏植之治凶禮。四年初,置《五經》博士,各開館教授,以植之兼《五經》博士。植之館在潮溝,生徒常百數。植之講,五館生必至,聽者千餘人。六年,遷中撫軍記室參軍,猶兼博士。七年,卒於館,時年五十二。植之自疾後,便不受廩俸,妻子困
 乏。既卒,喪無所寄,生徒為市宅,乃得成喪焉。



 植之性仁慈,好行陰德,雖在闇室,未嘗怠也。少嘗山行,見一患者,植之問其姓名,不能答,載與俱歸,為營醫藥,六日而死。植之為棺殮殯之,卒不知何許人也。嘗緣柵塘行,見患人臥塘側,植之下車問其故,云姓黃氏,家本荊州,為人傭賃,疾既危篤,船主將發,棄之于岸。植之心惻然,載還治之,經年而黃氏差,請終身充奴僕以報厚恩。植之不受,遺以資糧,遣之。其義行多如此。撰《凶禮儀注》四百七十九卷。



 賀瑒,字德璉,會稽山陰人也。祖道力,善《三禮》,仕宋為尚
 書三公郎、建康令。



 瑒少傳家業。齊時,沛國劉獻為會稽府丞,見蒨深器異之。嘗與俱造吳郡張融,指蒨謂融曰:「此生神明聰敏,將來當為儒者宗。」還,薦之為國子生。舉明經,揚州祭酒,俄兼國子助教。歷奉朝請、太學博士、太常丞,遭母憂去職。天監初,復為太常丞,有司舉治賓禮,召見說《禮》義,高祖異之,詔朝朔望,預華林講。四年初,開五館,以瑒兼《五經》博士,別詔為皇太子定禮,撰《五經義》。瑒悉禮舊事。時高祖方創定禮樂,蒨所建議,多見施行。七年,拜步兵校尉,領《五經》博士。九年,遇疾,遣醫藥省問,卒于館,時年五十九。所著《禮》、《易》、《老》、《莊講疏》、《朝廷博議》
 數百篇,《賓禮儀注》一百四十五卷。瑒於《禮》尤精,館中生徒常百數,弟子明經封策至數十人。



 二子。革,字文明。少通《三禮》,及長,遍治《孝經》、《論語》、《毛詩》、《左傳》。起家晉安王國侍郎、兼太學博士,侍湘東王讀。敕於永福省為邵陵、湘東、武陵三王講禮。稍遷湘東王府行參軍,轉尚書儀曹郎。尋除秣陵令,遷國子博士,於學講授,生徒常數百人。出為西中郎湘東王諮議參軍,帶江陵令。王初於府置學,以革領儒林祭酒,講《三禮》,荊楚衣冠聽者甚眾。前後再監南平郡,為民吏所德。尋加貞威將軍、兼平西長史、南郡太守。革性至孝,常恨貪祿代耕,不及養。在荊州歷
 為郡縣,所得俸秩,不及妻孥,專擬還鄉造寺,以申感思。大同六年,卒官,時年六十二。弟季,亦明《三禮》,歷官尚書祠部郎,兼中書通事舍人。累遷步兵校尉、中書黃門郎,兼著作。



 司馬筠,字貞素,河內溫人,晉驃騎將軍譙烈王承七世孫。祖亮,宋司空從事中郎。父端,齊奉朝請。筠孤貧好學,師事沛國劉獻,彊力專精,深為所器異。既長,博通經術,尤明《三禮》。齊建武中,起家奉朝請,遷王府行參軍。天監初,為本州治中,除暨陽令,有清績。入拜尚書祠部郎。



 七年,安成太妃陳氏薨,江州刺史安成王秀、荊州刺史
 始興王憺,並以《慈母表》解職,詔不許,還攝本任;而太妃薨京邑,喪祭無主。舍人周捨議曰:「賀彥先稱『慈母之子不服慈母之黨,婦又不從夫而服慈姑,小功服無從故也。』庾蔚之云:『非徒子不從母而服其黨,孫又不從父而服其慈母。』由斯而言,慈祖母無服明矣。尋門內之哀,不容自同於常;按父之祥禫,子並受弔。今二王諸子,宜以成服日,單衣一日,為位受弔。」制曰:「二王在遠,諸子宜攝祭事。」捨又曰:「《禮》云『縞冠玄武,子姓之冠』。則世子衣服宜異於常。可著細布衣,絹為領帶,三年不聽樂。又《禮》及《春秋》:庶母不世祭,蓋謂無王命者耳。吳太妃既朝命所加,得用
 安成禮秩,則當祔廟,五世親盡乃毀。陳太妃命數之重,雖則不異,慈孫既不從服,廟食理無傳祀,子祭孫止,是會經文。」高祖因是敕禮官議皇子慈母之服。筠議:「宋朝五服制,皇子服訓養母,依《禮》庶母慈己,宜從小功之制。按《曾子問》曰:子游曰:『喪慈母如母,禮歟?』孔子曰:『非禮也。古者男子外有傅,內有慈母,君命所使教子也,何服之有?』鄭玄注云:『此指謂國君之子也。』若國君之子不服,則王者之子不服可知。又《喪服經》云『君子子為庶母慈己者』。《傳》曰:『君子子者,貴人子也。』鄭玄引《內則》:三母止施於卿大夫。以此而推,則慈母之服,上不在五等之嗣,下不逮三
 士之息。儻其服者止卿大夫,尋諸侯之子尚無此服,況乃施之皇子。謂宜依《禮》刊除,以反前代之惑。」高祖以為不然,曰:「《禮》言慈母,凡有三條:一則妾子之無母,使妾之無子者養之,命為母子,服以三年,《喪服齊衰章》所言『慈母』是也;二則嫡妻之子無母,使妾養之,慈撫隆至,雖均乎慈愛,但嫡妻之子,妾無為母之義,而恩深事重,故服以小功,《喪服小功章》所以不直言慈母,而云『庶母慈己』者,明異於三年之慈母也;其三則子非無母,正是擇賤者視之,義同師保,而不無慈愛,故亦有慈母之名。師保既無其服,則此慈亦無服矣。《內則》云『擇於諸母與可者,
 使為子師;其次為慈母;其次為保母』,此其明文。此言擇諸母,是擇人而為此三母,非謂擇取兄弟之母也。何以知之?若是兄弟之母其先有子者,則是長妾,長妾之禮,實有殊加,何容次妾生子,乃退成保母,斯不可也。又有多兄弟之人,於義或可;若始生之子,便應三母俱闕邪?由是推之,《內則》所言『諸母』,是謂三母,非兄弟之母明矣。子游所問,自是師保之慈,非三年小功之慈也,故夫子得有此對。豈非師保之慈母無服之證乎?鄭玄不辨三慈,混為訓釋,引彼無服,以注『慈己』,後人致謬,實此之由。經言『君子子』者,此雖起於大夫,明大夫猶爾,自斯以上,
 彌應不異,故傳云『君子子者,貴人之子也』。總言曰貴,則無所不包。經傳互文,交相顯發,則知慈加之義,通乎大夫以上矣。宋代此科,不乖《禮》意,便加除削,良是所疑。」於是筠等請依制改定:嫡妻之子,母沒為父妾所養,服之五月,貴賤並同,以為永制。累遷王府諮議、權知左丞事,尋除尚書左丞。出為始興內史,卒官。



 子壽,傳父業,明《三禮》。大同中,歷官尚書祠部郎,出為曲阿令。



 卞華,字昭丘,濟陰冤句人也。晉驃騎將軍忠貞公壼六世孫。父倫之,給事中。華幼孤貧好學。年十四,召補國子生,通《周易》。既長,遍治《五經》,與平原明山賓、會稽賀蒨同
 業友善。起家齊豫章王國侍郎,累遷奉朝請、征西行參軍。天監初,遷臨川王參軍事,兼國子助教,轉安成王功曹參軍,兼《五經》博士,聚徒教授。華博涉有機辯,說經析理,為當時之冠。江左以來,鐘律絕學,至華乃通焉。遷尚書儀曹郎,出為吳令,卒。



 崔靈恩,清河武城人也。少篤學,從師遍通《五經》,尤精《三禮》、《三傳》。先在北仕為太常博士,天監十三年歸國。高祖以其儒術,擢拜員外散騎侍郎,累遷步兵校尉,兼國子博士。靈恩聚徒講授,聽者常數百人。性拙朴無風采,及解經析理,甚有精致,京師舊儒咸稱重之,助教孔僉尤
 好其學。靈恩先習《左傳》服解,不為江東所行;及改說杜義,每文句常申服以難杜,遂著《左氏條義》以明之。時有助教虞僧誕又精杜學,因作《申杜難服》,以報靈恩,世並行焉。(僧誕,會稽餘姚人,以《左氏》教授,聽者亦數百人。其該通義例,當時莫及。)先是儒者論天,互執渾、蓋二義,論蓋不合於渾,論渾不合於蓋。靈恩立義,以渾、蓋為一焉。出為長沙內史,還除國子博士,講眾尤盛。出為明威將軍、桂州刺史,卒官。靈恩集注《毛詩》二十二卷,集注《周禮》四十卷,制《三禮義宗》四十七卷,《左氏經傳義》二十二卷,《左氏條例》十卷,《公羊穀梁文句義》十卷。



 孔僉,會稽山陰人。少師事何胤,通《五經》,尤明《三禮》、《孝經》、《論語》,講說並數十遍,生徒亦數百人。歷官國子助教,三為《五經》博士,遷尚書祠部郎。出為海鹽、山陰二縣令。僉儒者,不長政術,在縣無績。太清亂,卒于家。子俶玄,頗涉文學,官至太學博士。僉兄子元素,又善《三禮》,有盛名,早卒。



 盧廣,范陽涿人,自云晉司空從事中郎諶之後也。諶沒死冉閔之亂,晉中原舊族,諶有後焉。廣少明經,有儒術。天監中歸國。初拜員外散騎侍郎,出為始安太守,坐事免。頃之,起為折衝將軍,配千兵北伐,還拜步兵校尉,兼
 國子博士,遍講《五經》。時北來人,儒學者有崔靈恩、孫詳、蔣顯,並聚徒講說,而音辭鄙拙;惟廣言論清雅,不類北人。僕射徐勉,兼通經術,深相賞好。尋遷員外散騎常侍,博士如故。出為信武桂陽嗣王長史、尋陽太守。又為武陵王長史,太守如故,卒官。



 沈峻,字士嵩,吳興武康人。家世農夫,至峻好學,與舅太史叔明師事宗人沈麟士門下積年。晝夜自課,時或睡寐,輒以杖自擊,其篤志如此。麟士卒後,乃出都,遍遊講肆,遂博通《五經》,尤長《三禮》。初為王國中尉,稍遷侍郎,並兼國子助教。時吏部郎陸倕與僕射徐勉書薦峻曰:「《五
 經》博士庾季達須換,計公家必欲詳擇其人。凡聖賢可講之書,必以《周官》立義,則《周官》一書,實為群經源本。此學不傳,多歷年世,北人孫詳、蔣顯亦經聽習,而音革楚、夏,故學徒不至;惟助教沈峻,特精此書。比日時開講肆,群儒劉巖、沈宏、沈熊之徒,並執經下坐,北面受業,莫不歎服,人無間言。第謂宜即用此人,命其專此一學,周而復始。使聖人正典,廢而更興;累世絕業,傳於學者。」勉從之,奏峻兼《五經》博士。於館講授,聽者常數百人。出為華容令,還除員外散騎侍郎,復兼《五經》博士。時中書舍人賀琛奉敕撰《梁官》,乃啟峻及孔子袪補西省學士,助撰
 錄。書成,入兼中書通事舍人。出為武康令,卒官。



 子文阿,傳父業,尤明《左氏傳》。太清中,自國子助教為《五經》博士。傳峻業者,又有吳郡張及、會稽孔子雲,官皆至《五經》博士、尚書祠部郎。



 太史叔明,吳興烏程人,吳太史慈後也。少善《莊》、《老》,兼治《孝經》、《禮記》,其三玄尤精解,當世冠絕,每講說,聽者常五百餘人。歷官國子助教。邵陵王綸好其學,及出為江州,攜叔明之鎮。王遷郢州,又隨府,所至輒講授,江外人士皆傳其學焉。大同十三年,卒,時年七十三。



 孔子袪,會稽山陰人。少孤貧好學,耕耘樵採,常懷書自
 隨,投閑則誦讀。勤苦自勵,遂通經術,尤明《古文尚書》。初為長沙嗣王侍郎,兼國子助教,講《尚書》四十遍,聽者常數百人。中書舍人賀琛受敕撰《梁官》,啟子袪為西省學士,助撰錄。書成,兼司文侍郎,不就。久之兼主客郎、舍人,學士如故。累遷湘東王國侍郎、常侍、員外散騎侍郎,又雲麾廬江公記室參軍,轉兼中書通事舍人。尋遷步兵校尉,舍人如故。高祖撰《五經講疏》及《孔子正言》,專使子袪檢閱群書,以為義證。事竟,敕子袪與右衛朱異、左丞賀琛於士林館遞日執經。累遷通直正員郎,舍人如故。中大同元年,卒官,時年五十一。子袪凡著《尚書義》二十
 卷,《集注尚書》三十卷,續朱異《集注周易》一百卷,續何承天《集禮論》一百五十卷。



 皇侃,吳郡人,青州刺史皇象九世孫也。侃少好學,師事賀蒨,精力專門,盡通其業,尤明《三禮》、《孝經》、《論語》。起家兼國子助教,於學講說,聽者數百人。撰《禮記講疏》五十卷,書成奏上,詔付秘閣。頃之,召入壽光殿講《禮記義》,高祖善之,拜員外散騎侍郎,兼助教如故。性至孝,常日限誦《孝經》二十遍,以擬《觀世音經》。丁母憂,解職還鄉里。平西邵陵王欽其學,厚禮迎之。侃既至,因感心疾,大同十一年,卒於夏首,時年五十八。所撰《論語義》十卷,與《禮記義》
 並見重於世,學者傳焉。



 陳吏部尚書姚察曰:昔叔孫通講論馬上,桓榮精力兇荒;既逢平定,自致光寵;若夫崔、伏、何、嚴互有焉。曼容、佟之講道於齊季,不為時改;賀蒨、嚴植之之徒,遭梁之崇儒重道,咸至高官,稽古之力,諸子各盡之矣。範縝墨絰僥幸,不遂其志,宜哉。



\end{pinyinscope}