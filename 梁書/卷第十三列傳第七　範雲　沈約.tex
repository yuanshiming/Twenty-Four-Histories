\article{卷第十三列傳第七 範雲 沈約}

\begin{pinyinscope}

 范雲,字彥龍,南鄉舞陰人,晉平北將軍汪六世孫也。年八歲,遇宋豫州刺史殷琰於塗,琰異之,要就席,雲風姿應對,傍若無人。琰令賦詩,操筆便就,坐者歎焉。嘗就親人袁照學,晝夜不怠。照撫其背曰:「卿精神秀朗而勤於學,卿相才也。」少機警有識,且善屬文,便尺牘,下筆輒成,未嘗定槁,時人每疑其宿構。父抗,為郢府參軍,雲隨
 父在府,時吳興沈約、新野庾杲之與抗同府,見而友之。



 起家郢州西曹書佐,轉法曹行參軍。俄而沈攸之舉兵圍郢城,抗時為府長流,入城固守,留家屬居外。雲為軍人所得,攸之召與語,聲色甚厲,雲容貌不變,徐自陳說。攸之乃笑曰:「卿定可兒,且出就舍。」明旦,又召令送書入城。城內或欲誅之,雲曰:「老母弱弟,懸命沈氏,若違其命,禍必及親,今日就戮,甘心如薺。」長史柳世隆素與雲善,乃免之。



 齊建元初,竟陵王子良為會稽太守,雲始隨王,王未之知也。會遊秦望,使人視刻石文,時莫能識,雲獨誦之,王悅,自是寵冠府朝。王為丹陽尹,召為主簿,深相
 親任。時進見齊高帝,值有獻白烏者,帝問此為何瑞?雲位卑,最後答曰:「臣聞王者敬宗廟,則白烏至。」時謁廟始畢。帝曰:「卿言是也。感應之理,一至此乎!」轉補征北南郡王刑獄參軍事,領主簿如故,遷尚書殿中郎。子良為司徒,又補記室參軍事,尋授通直散騎侍郎、領本州大中正。出為零陵內史,在任潔己,省煩苛,去游費,百姓安之。明帝召還都,及至,拜散騎侍郎。復出為始興內史。郡多豪猾大姓,二千石有不善者,謀共殺害,不則逐去之。邊帶蠻俚,尤多盜賊,前內史皆以兵刃自衛。雲入境,撫以恩德,罷亭候,商賈露宿,郡中稱為神明。仍遷假節、建武
 將軍、平越中郎將、廣州刺史。初,雲與尚書僕射江祏善,祏姨弟徐藝為曲江令,深以託雲。有譚儼者,縣之豪族,藝鞭之,儼以為恥,詣京訴雲,雲坐徵還下獄,會赦免。永元二年,起為國子博士。



 初,雲與高祖遇於齊竟陵王子良邸,又嘗接里閈,高祖深器之。及義兵至京邑,雲時在城內。東昏既誅,侍中張稷使雲銜命出城,高祖因留之,便參帷幄,仍拜黃門侍郎,與沈約同心翊贊。俄遷大司馬諮議參軍、領錄事。梁臺建,遷侍中。時高祖納齊東昏餘妃,頗妨政事,雲嘗以為言,未之納也。後與王茂同入臥內,雲又諫曰:「昔漢祖居山東,貪財好色,及入關定秦,
 財帛無所取,婦女無所幸,范增以為其志大故也。今明公始定天下,海內想望風聲,奈何襲昏亂之蹤,以女德為累。」王茂因起拜曰:「范雲言是,公必以天下為念,無宜留惜。」高祖默然。雲便疏令以餘氏賚茂,高祖賢其意而許之。明日,賜雲、茂錢各百萬。



 天監元年,高祖受禪,柴燎於南郊,雲以侍中參乘。禮畢,高祖升輦,謂雲曰:「朕之今日,所謂懍乎若朽索之馭六馬。」雲對曰:「亦願陛下日慎一日。」高祖善之。是日,遷散騎常侍、吏部尚書;以佐命功封霄城縣侯,邑千戶。雲以舊恩見拔,超居佐命,盡誠翊亮,知無不為。高祖亦推心任之,所奏多允。嘗侍宴,高祖
 謂臨川王宏、鄱陽王恢曰:「我與范尚書少親善,申四海之敬;今為天下主,此禮既革,汝宜代我呼范為兄。」二王下席拜,與雲同車還尚書下省,時人榮之。其年,東宮建,雲以本官領太子中庶子,尋遷尚書右僕射,猶領吏部。頃之,坐違詔用人,免吏部,猶為僕射。



 雲性篤睦,事寡嫂盡禮,家事必先諮而後行。好節尚奇,專趣人之急。少時與領軍長史王畡善,畡亡於官舍,貧無居宅,雲乃迎喪還家。躬營含殯。事竟陵王子良恩禮甚隆,雲每獻損益,未嘗阿意。子良嘗啟齊武帝論雲為郡。帝曰:「庸人,聞其恒相賣弄,不復窮法,當宥之以遠。」子良曰:「不然。雲動相
 規誨,諫書具存,請取以奏。」既至,有百餘紙,辭皆切直。帝歎息,因謂子良曰:「不謂雲能爾。方使弼汝,何宜出守。」齊文惠太子嘗出東田觀獲,顧謂眾賓曰:「刈此亦殊可觀。」眾皆唯唯。雲獨曰:「夫三時之務,實為長勤。伏願殿下知稼穡之艱難,無徇一朝之宴逸。」既出,侍中蕭緬先不相識,因就車握雲手曰:「不圖今日復聞讜言。」及居選官,任守隆重,書牘盈案,賓客滿門,雲應對如流,無所壅滯,官曹文墨,發擿若神,時人咸服其明贍。性頗激厲,少威重,有所是非,形於造次,士或以此少之。初,雲為郡號稱廉潔,及居貴重,頗通饋餉;然家無蓄積,隨散之親友。



 二年,
 卒,時年五十三。高祖為之流涕,即日輿駕臨殯。詔曰:「追遠興悼,常情所篤;況問望斯在,事深朝寄者乎!故散騎常侍、尚書右僕射、霄城侯雲,器範貞正,思懷經遠,爰初立志,素履有聞。脫巾來仕,清績仍著。燮務登朝,具瞻惟允。綢繆翊贊,義簡朕心,雖勤非負靮,而舊同論講。方騁遠塗,永毘庶政;奄致喪殞,傷悼于懷。宜加命秩,式備徽典。可追贈侍中、衛將軍,僕射、侯如故。并給鼓吹一部。」禮官請謚曰宣,敕賜謚文。有集三十卷。子孝才嗣,官至太子中舍人。



 沈約,字休文,吳興武康人也。祖林子,宋征虜將軍。父璞,
 淮南太守。璞元嘉末被誅,約幼潛竄,會赦免。既而流寓孤貧,篤志好學,晝夜不倦。母恐其以勞生疾,常遣減油滅火。而晝之所讀,夜輒誦之,遂博通群籍,能屬文。起家奉朝請。濟陽蔡興宗聞其才而善之;興宗為郢州刺史,引為安西外兵參軍,兼記室。興宗嘗謂其諸子曰:「沈記室人倫師表,宜善事之。」及為荊州,又為征西記室參軍,帶關西令。興宗卒,始為安西晉安王法曹參軍,轉外兵,並兼記室。入為尚書度支郎。



 齊初為征虜記室,帶襄陽令,所奉之王,齊文惠太子也。太子入居東宮,為步兵校尉,管書記,直永壽省,校四部圖書。時東宮多士,約特被
 親遇,每直入見,影斜方出。當時王侯到宮,或不得進,約每以為言。太子曰:「吾生平懶起,是卿所悉,得卿談論,然後忘寢。卿欲我夙興,可恒早入。」遷太子家令,後以本官兼著作郎,遷中書郎,本邑中正,司徒右長史,黃門侍郎。時竟陵王亦招士,約與蘭陵蕭琛、瑯邪王融、陳郡謝朓、南鄉范雲、樂安任昉等皆遊焉,當世號為得人。俄兼尚書左丞,尋為御史中丞,轉車騎長史。隆昌元年,除吏部郎,出為寧朔將軍、東陽太守。明帝即位,進號輔國將軍,徵為五兵尚書,遷國子祭酒。明帝崩,政歸塚宰,尚書令徐孝嗣使約撰定遺詔。遷左衛將軍,尋加通直散騎常
 侍。永元二年,以母老表求解職,改授冠軍將軍、司徒左長史,征虜將軍、南清河太守。



 高祖在西邸,與約遊舊,建康城平,引為驃騎司馬,將軍如故。時高祖勳業既就,天人允屬,約嘗扣其端,高祖默而不應。佗日又進曰:「今與古異,不可以淳風期萬物。士大夫攀龍附鳳者,皆望有尺寸之功,以保其福祿。今童兒牧豎,悉知齊祚已終,莫不云明公其人也。天文人事,表革運之徵,永元以來,尤為彰著。讖云『行中水,作天子」,此又歷然在記。天心不可違,人情不可失,茍是歷數所至,雖欲謙光,亦不可得已。」高祖曰:「吾方思之。」對曰:「公初杖兵樊、沔,此時應思,今王業
 已就,何所復思。昔武王伐紂,始入,民便曰吾君,武王不違民意,亦無所思。公自至京邑,已移氣序,比於周武,遲速不同。若不早定大業,稽天人之望,脫有一人立異,便損威德。且人非金玉,時事難保。豈可以建安之封,遺之子孫?若天子還都,公卿在位,則君臣分定,無復異心。君明於上,臣忠於下,豈復有人方更同公作賊。」高祖然之。約出,高祖召范雲告之,雲對略同約旨。高祖曰:「智者乃爾暗同,卿明早將休文更來。」雲出語約,約曰:「卿必待我。」雲許諾,而約先期入,高祖命草其事。約乃出懷中詔書并諸選置,高祖初無所改。俄而雲自外來,至殿門不得
 入,徘徊壽光閣外,但云「咄咄」。約出,問曰:「何以見處?」約舉手向左,雲笑曰:「不乖所望。」有頃,高祖召范雲謂曰:「生平與沈休文群居,不覺有異人處;今日才智縱橫,可謂明識。」雲曰:「公今知約,不異約今知公。」高祖曰:「我起兵於今三年矣,功臣諸將,實有其勞,然成帝業者,乃卿二人也。」



 梁臺建,為散騎常侍、吏部尚書,兼右僕射。高祖受禪,為尚書僕射,封建昌縣侯,邑千戶,常侍如故。又拜約母謝為建昌國太夫人。奉策之日,右僕射范雲等二十餘人咸來致拜,朝野以為榮。俄遷尚書左僕射,常侍如故。尋兼領軍,加侍中。天監二年,遭母憂,輿駕親出臨弔,以約
 年衰,不宜致毀,遣中書舍人斷客節哭。起為鎮軍將軍、丹陽尹,置佐史。服闋,遷侍中、右光祿大夫,領太子詹事,揚州大中正,關尚書八條事,遷尚書令,侍中、詹事、中正如故。累表陳讓,改授尚書左僕射、領中書令、前將軍,置佐史,侍中如故。尋遷尚書令,領太子少傅。九年,轉左光祿大夫,侍中、少傅如故,給鼓吹一部。



 初,約久處端揆,有志台司,論者咸謂為宜,而帝終不用,乃求外出,又不見許。與徐勉素善,遂以書陳情於勉曰:「吾弱年孤苦,傍無期屬,往者將墜於地,契闊屯邅,困於朝夕,崎嶇薄宦,事非為己,望得小祿,傍此東歸。歲逾十稔,方忝襄陽縣,公
 私情計,非所了具,以身資物,不得不任人事。永明末,出守東陽,意在止足;而建武肇運,人世膠加,一去不返,行之未易。及昏猜之始,王政多門,因此謀退,庶幾可果,託卿布懷於徐令,想記未忘。聖道聿興,謬逢嘉運,往志宿心,復成乖爽。今歲開元,禮年云至,懸車之請,事由恩奪。誠不能弘宣風政,光闡朝猷,尚欲討尋文簿,時議同異。而開年以來,病增慮切,當由生靈有限,勞役過差,總此凋竭,歸之暮年,牽策行止,努力祗事。外觀傍覽,尚似全人,而形骸力用,不相綜攝,常須過自束持,方可黽勉。解衣一臥,支體不復相關。上熱下冷,月增日篤,取煖則煩,
 加寒必利,後差不及前差,後劇必甚前劇。百日數旬,革帶常應移孔;以手握臂,率計月小半分。以此推算,豈能支久?若此不休,日復一日,將貽聖主不追之恨。冒欲表聞,乞歸老之秩。若天假其年,還是平健,才力所堪,惟思是策。」勉為言於高祖,請三司之儀,弗許,但加鼓吹而已。



 約性不飲酒,少嗜欲,雖時遇隆重,而居處儉素。立宅東田,矚望郊阜。嘗為《郊居賦》,其辭曰:惟至人之非己,固物我而兼忘。自中智以下洎,咸得性以為場。獸因窟而獲騁,鳥先巢而後翔。陳巷窮而業泰,嬰居湫而德昌。僑棲仁於東里,鳳晦跡於西堂。伊吾人之褊志,無經世之大
 方。思依林而羽戢,願託水而鱗藏。固無情於輪奐,非有欲於康莊。披東郊之寥廓,入蓬藋之荒茫。既從豎而橫構,亦風除而雨攘。



 昔西漢之標季,餘播遷之云始。違利建於海昏,創惟桑於江汜。同河濟之重世,踰班生之十紀。或辭祿而反耕,或彈冠而來仕。逮有晉之隆安,集艱虞於天步。世交爭而波流,民失時而狼顧。延亂麻於井邑,曝如莽於衢路。大地曠而靡容,旻天遠而誰訴。伊皇祖之弱辰,逢時艱之孔棘。違危邦而窘驚,訪安土而移即。肇胥宇於朱方,掩閑庭而晏息。值龍顏之鬱起,乃憑風而矯翼。指皇邑而南轅,駕修衢以騁力。遷華扉而來
 啟,張高衡而徙植。傍逸陌之修平,面淮流之清直。芳塵浸而悠遠,世道忽其窊隆。綿四代於茲日,盈百祀於微躬。嗟弊廬之難保,若霣籜之從風。或誅茅而剪棘,或既西而復東。乍容身於白社,亦寄孥於伯通。



 迹平生之耿介,實有心於獨往。思幽人而軫念,望東皋而長想。本忘情於徇物,徒羈紲於天壤。應屢歎於牽絲,陸興言於世網。事滔滔而未合,志悁悁而無爽。路將殫而彌峭,情薄暮而逾廣。抱寸心其如蘭,何斯願之浩蕩。詠歸歟而躑跼,眷巖阿而抵掌。



 逢時君之喪德,何凶昏之孔熾。乃戰牧所未陳,實升陑所不記。彼黎元之喋喋,將垂獸而為
 餌。瞻穹昊而無歸,雖非牢而被胾。始歎絲而未睹,終逌組而後值。尋貽愛乎上天,固非民其莫甚。授冥符於井翼,實靈命之所稟。當降監之初辰,值積惡之云稔。寧方割於下墊,廓重氛於上墋。躬靡暇於朝食,常求衣於夜枕。既牢籠於媯、夏,又驅馳乎軒、頊。德無遠而不被,明無微而不燭。鼓玄澤於大荒,播仁風於遐俗。闢終古而遐念,信王猷其如玉。



 值銜《圖》之盛世,遇興聖之嘉期。謝中涓於初日,叨光佐於此時。闕投石之猛志,無飛矢之麗辭。排陽鳥而命邑,方河山而啟基。翼儲光於三善,長王職於百司。兢鄙夫之易失,懼寵祿之難持。伊前世之貴
 仕,罕紆情於丘窟。譬叢華於楚、趙,每驕奢以相越。築甲館於銅駝,並高門於北闕。闢重扃於華閫,豈蓬蒿所能沒。敖傳嗣於墝壤,何安身於窮地。味先哲而為言,固余心之所嗜。不慕權於城市,豈邀名於屠肆。詠希微以考室,幸風霜之可庇。



 爾乃傍窮野,抵荒郊;編霜菼,葺寒茅。構棲噪之所集,築町畽之所交。因犯簷而刊樹,由妨基而剪巢。決渟洿之汀濙,塞井甃之淪坳。藝芳枳於北渠,樹修楊於南浦。遷甕牖於蘭室,同肩墻於華堵。織宿楚以成門,籍外扉而為戶。既取陰於庭樾,又因籬於芳杜。開閣室以遠臨,闢高軒而旁睹。漸沼沚於溜垂,周塍陌
 於堂下。其水草則蘋萍芡芰,菁藻蒹菰;石衣海髮,黃荇綠蒲。動紅荷於輕浪,覆碧葉於澄湖。飡嘉實而卻老,振羽服於清都。其陸卉則紫鱉綠葹,天著山韭;雁齒麋舌,牛脣彘首。布濩南池之陽,爛漫北樓之後。或幕渚而芘地,或縈窗而窺牖。若乃園宅殊製,田圃異區。李衡則橘林千樹,石崇則雜果萬株。並豪情之所侈,非儉志之所娛。欲令紛披蓊鬱,吐綠攢朱;羅窗映戶,接溜承隅。開丹房以四照,舒翠葉而九衢。抽紅英於紫帶,銜素蕊於青跗。其林鳥則翻泊頡頏,遺音下上;楚雀多名,流嚶雜響。或班尾而綺翼,或綠衿而絳顙。好葉隱而枝藏,乍間關
 而來往。其水禽則大鴻小鴈,天狗澤虞;秋蠙寒褵,修鷁短鳧。曳參差之弱藻,戲瀺灂之輕軀;翅抨流而起沫,翼鼓浪而成珠。其魚則赤鯉青魴,纖倏鉅褷。碧鱗朱尾,修顱偃額。小則戲渚成文,大則噴流揚白。不興羨於江海,聊相忘於余宅。其竹則東南獨秀,九府擅奇。不遷植於淇水,豈分根於樂池。秋蜩吟葉,寒雀噪枝。來風南軒之下,負雪北堂之垂。訪往塗之軫跡,觀先識之情偽。每誅空而索有,皆指難以為易。不自已而求足,並尤物以興累。亦昔士之所迷,而今餘之所避也。



 原農皇之攸始,討厥播之云初。肇變腥以粒食,乃人命之所儲。尋井田之
 往記,考阡陌於前書。顏簞食而樂在,鄭高廩而空虛。頃四百而不足,畝五十而有餘。撫幽衷而跼念,幸取給於庭廬。緯東菑之故耜,浸北畝之新渠。無褰爨於曉蓐,不抱惄於朝蔬。排外物以齊遣,獨為累之在餘。安事千斯之積,不羨汶陽之墟。



 臨巽維而騁目,即堆塚而流眄。雖茲山之培塿,乃文靖之所宴。驅四牡之低昂,響繁笳之清囀。羅方員而綺錯,窮海陸而兼薦。奚一權之足偉,委千金其如線。試撫臆而為言,豈斯風之可扇。將通人之遠旨,非庸情之所見。聊遷情而徙睇,識方阜於歸津。帶修汀於桂渚,肇舉鍤於彊秦。路縈吳而款越,塗被海而
 通閩。懷三鳥以長念,伊故鄉之可珍。實褰期於晚歲,非失步於方春。何東川之濔々,獨流涕於吾人。謬參賢於昔代,亟徒遊於茲所。侍採旄而齊轡,陪龍舟而遵渚。或列席而賦詩,或班觴而宴語。繐帷一朝冥漠,西陵忽其蔥楚。望商飆而永歎,每樂愷於斯觀。始則鐘石鏘珣,終以魚龍瀾漫。或升降有序,或浮白無算。貴則景、魏、蕭、曹,親則梁武、周旦。莫不共霜霧而歇滅,與風雲而消散。眺孫后之墓田,尋雄霸之遺武。實接漢之後王,信開吳之英主。指衡岳而作鎮,苞江漢而為宇。徒徵言於石槨,遂延災於金縷。忽蕪穢而不修,同原陵之膴々。寧知螻蟻
 之與狐兔,無論樵芻之與牧豎。睇東巘以流目,心悽愴而不怡。蓋昔儲之舊苑,實博望之餘基。修林則表以桂樹,列草則冠以芳芝。風臺累翼,月榭重栭。千櫨捷釭,百栱相持。皁轅林駕,蘭枻水嬉。踰三齡而事往,忽二紀以歷茲。咸夷漫以蕩滌,非古今之異時。


回餘眸於艮域,覿高館於茲嶺。雖混成以無跡,實遺訓之可秉。始飡霞而吐霧,終陵虛而倒影。駕雌霓之連卷,泛天江之悠永。指咸池而一息,望瑤臺而高騁,匪爽言以自姱,冀神方之可請。惟鐘巖之隱鬱,表皇都而作峻,蓋望秩之所宗,含風雲而吐潤。其為狀也,則巍峨崇褲,喬枝拂日;嶢嶷岧
 \gezhu{
  山亭}
 ,墜石堆星。岑崟峍屼,或坳或平;盤堅枕臥,詭狀殊形。孤嶝橫插,洞穴斜經;千丈萬仞,三襲九成。亙繞州邑,款跨郊坰;素煙晚帶,白霧晨縈。近循則一巖異色,遠望則百嶺俱青。



 觀二代之塋兆,睹摧殘之餘遂。成顛沛於虐豎,康斂衿於虛器;穆恭已於巖廊,簡遊情於玄肆;烈窮飲以致災,安忘懷而受祟。何宗祖之奇傑,威橫天而陵地。惟聖文之纘武,殆隆平之可至。餘世德之所君,仰遺封而掩淚。神寢匪一,靈館相距。席布騂駒,堂流桂醑。降紫皇於天闕,延二妃於湘渚。浮蘭煙於桂棟,召巫陽於南楚。揚玉桴,握椒糈。怳臨風以浩唱,折瓊茅而延佇。敬
 惟空路邈遠,神蹤遐闊。念甚驚飆,生猶聚沫。歸妙軫於一乘,啟玄扉於三達。欲息心以遣累,必違人而後豁。或結於巖根,或開欞於木末。室闇蘿蔦,簷梢松栝。既得理於兼謝,固忘懷於饑渴。或攀枝獨遠,或陵雲高蹈。因葺茨以結名,猶觀空以表號。得忘己於茲日,豈期心於來報。天假餘以大德,荷茲賜之無疆。受老夫之嘉稱,班燕禮於上庠。無希驥之秀質,乏如珪之令望。邀昔恩於舊主,重匪服於今皇。仰休老之盛則,請微軀於夕陽。勞蒙司而獲謝,猶奉職於春坊。時言歸於陋宇,聊暇日以翱翔。棲餘志於凈國,歸餘心於道場。獸依墀而莫駭,魚
 牣沼而不綱。旋迷塗於去轍,篤後念於徂光。晚樹開花,初英落蕊。或異林而分丹青,乍因風而雜紅紫。紫蓮夜發,紅荷曉舒。輕風微動,其芳襲餘。風騷屑於園樹,月籠連於池竹。蔓長柯於簷桂,發黃華於庭菊。冰懸埳而帶坻,雪縈松而被野。鴨屯飛而不散,雁高翔而欲下。並時物之可懷,雖外來而非假。實情性之所留滯,亦志之而不能捨也。



 傷餘情之頹暮,罹憂患其相溢。悲異軫而同歸,嘆殊方而並失。時復託情魚鳥,歸閑蓬蓽。旁闕吳娃,前無趙瑟。以斯終老,於焉消日。惟以天地之恩不報,書事之官靡述;徒重於高門之地,不載於良史之筆。長太
 息其何言,羌愧心之非一。



 尋加特進,光祿、侍中、少傅如故。十二年,卒官,時年七十三。詔贈本官,賻錢五萬,布百匹,謚曰隱。



 約左目重瞳子,腰有紫志,聰明過人。好墳籍,聚書至二萬卷,京師莫比。少時孤貧,丐于宗黨,得米數百斛,為宗人所侮,覆米而去。及貴,不以為憾,用為郡部傳。嘗侍宴,有妓師是齊文惠宮人。帝問識座中客不?曰:「惟識沈家令。」約伏座流涕,帝亦悲焉,為之罷酒。約歷仕三代,該悉舊章,博物洽聞,當世取則。謝玄暉善為詩,任彥昇工於文章,約兼而有之,然不能過也。自負高才,昧於榮利,乘時藉勢,頗累清談。及居端揆,稍弘止足。每進
 一官,輒殷勤請退,而終不能去,論者方之山濤。用事十餘年,未嘗有所薦達,政之得失,唯唯而已。



 初,高祖有憾於張稷,及稷卒,因與約言之。約曰:「尚書左僕射出作邊州刺史,已往之事,何足復論。」帝以為婚家相為,大怒曰:「卿言如此,是忠臣邪!」乃輦歸內殿。約懼,不覺高祖起,猶坐如初。及還,未至床,而憑空頓於戶下。因病,夢齊和帝以劍斷其舌。召巫視之,巫言如夢。乃呼道士奏赤章於天,稱禪代之事,不由己出。高祖遣上省醫徐奘視約疾,還具以狀聞。先此,約嘗侍宴,值豫州獻栗,徑寸半,帝奇之,問曰:「慄事多少?」與約各疏所憶,少帝三事。出謂人曰:「
 此公護前,不讓即羞死。」帝以其言不遜,欲抵其罪,徐勉固諫乃止。及聞赤章事,大怒,中使譴責者數焉,約懼遂卒。有司謚曰文,帝曰:「懷情不盡曰隱。」故改為隱云。所著《晉書》百一十卷,《宋書》百卷,《齊紀》二十卷,《高祖紀》十四卷,《邇言》十卷,《謚例》十卷,《宋文章志》三十卷,文集一百卷:皆行於世。又撰《四聲譜》,以為在昔詞人,累千載而不寤,而獨得胸衿,窮其妙旨,自謂入神之作,高祖雅不好焉。帝問周捨曰:「何謂四聲?」捨曰:「天子聖哲」是也,然帝竟不遵用。



 子旋,及約時已歷中書侍郎,永嘉太守,司徒從事中郎,司徒右長史。免約喪,為太子僕,復以母憂去官,而蔬
 食辟穀。服除,猶絕粳粱。為給事黃門侍郎、中撫軍長史。出為招遠將軍、南康內史,在部以清治稱。卒官,謚曰恭侯。子實嗣。



 陳吏部尚書姚察曰:昔木德將謝,昏嗣流虐,惵惵黔黎,命懸晷漏。高祖義拯橫潰,志寧區夏,謀謨帷幄,實寄良、平。至於范雲、沈約,參預締構,贊成帝業;加雲以機警明贍,濟務益時,約高才博洽,名亞遷、董,俱屬興運,蓋一代之英偉焉。



\end{pinyinscope}