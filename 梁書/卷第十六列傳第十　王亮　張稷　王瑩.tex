\article{卷第十六列傳第十 王亮 張稷 王瑩}

\begin{pinyinscope}

 王亮,字奉叔,瑯邪臨沂人,晉丞相導之六世孫也。祖偃,宋右光祿大夫、開府儀同三司。父攸,給事黃門侍郎。亮以名家子,宋末選尚公主,拜駙馬都尉、祕書郎,累遷桂陽王文學,南郡王友,祕書丞。齊竟陵王子良開西邸,延才俊以為士林館,使工圖畫其像,亮亦預焉。遷中書侍郎、大司馬從事中郎,出為衡陽太守。以南土卑濕,辭不
 之官,遷給事黃門侍郎。尋拜晉陵太守,在職清公有美政。時齊明帝作相,聞而嘉之,引為領軍長史,甚見賞納。及即位,累遷太子中庶子,尚書吏部郎,詮序著稱,遷侍中。建武末,為吏部尚書,是時尚書右僕射江祏管朝政,多所進拔,為士子所歸。亮自以身居選部,每持異議。始亮未為吏部郎時,以祏帝之內弟,故深友祏,祏為之延譽,益為帝所器重;至是與祏暱之如初。及祏遇誅,群小放命,凡所除拜,悉由內寵,亮更弗能止。外若詳審,內無明鑒,其所選用,拘資次而已,當世不謂為能。頻加通直散騎常侍、太子右衛率,為尚書右僕射、中護軍。既而東
 昏肆虐,淫刑已逞,亮傾側取容,竟以免戮。



 義師至新林,內外百僚皆道迎,其未能拔者,亦間路送誠款,亮獨不遣。及城內既定,獨推亮為首。亮出見高祖,高祖曰:「顛而不扶,安用彼相。」而弗之罪也。霸府開,以為大司馬長史、撫軍將軍、瑯邪、清河二郡太守。梁臺建,授侍中、尚書令,固讓不拜,乃為侍中、中書監,兼尚書令。高祖受禪,遷侍中、尚書令、中軍將軍,引參佐命,封豫寧縣公,邑二千戶。天監二年,轉左光祿大夫,侍中、中軍如故。元日朝會萬國,亮辭疾不登殿,設饌別省,而語笑自若。數日,詔公卿問訊,亮無疾色,御史中丞樂藹奏大不敬,論棄市刑。詔
 削爵廢為庶人。四年夏,高祖宴於華光殿,謂群臣曰:「朕日昃聽政,思聞得失。卿等可謂多士,宜各盡獻替。」尚書左丞范縝起曰:「司徒謝朏本有虛名,陛下擢之如此,前尚書令王亮頗有治實,陛下棄之如彼,是愚臣所不知。」高祖變色曰:「卿可更餘言。」縝固執不已,高祖不悅。御史中丞任昉因奏曰:臣聞息夫歷詆,漢有正刑;白褒一奏,晉以明罰。況乎附下訕上,毀譽自口者哉。風聞尚書左丞臣范縝,自晉安還,語人云:「我不詣餘人,惟詣王亮;不餉餘人,惟餉王亮。」輒收縝白從左右萬休到臺辨問,與風聞符同。又今月十日,御餞梁州刺史臣珍國,宴私既
 洽,群臣並已謁退,時詔留侍中臣昂等十人,訪以政道。縝不答所問,而橫議沸騰,遂貶裁司徒臣朏,褒舉庶人王亮。臣于時預奉恩留,肩隨並立,耳目所接,差非風聞。竊尋王有遊豫,親御軒陛,義深推轂,情均《湛露》。酒闌宴罷,當扆正立,記事在前,記言在後,軫早朝之念,深求瘼之情,而縝言不遜,妄陳褒貶,傷濟濟之風,缺側席之望。不有嚴裁,憲準將頹,縝即主。



 臣謹案:尚書左丞臣范縝,衣冠緒餘,言行舛駁,誇諧里落,喧詬周行。曲學諛聞,未知去代;弄口鳴舌,祇足飾非。乃者,義師近次,縝丁罹艱棘,曾不呼門,墨縗景附,頗同先覺,實奉龍顏。而今黨協
 釁餘,翻為矛楯,人而無恒,成茲姦詖。日者,飲至策勳,功微賞厚,出守名邦,入司管轄,苞篚罔遺,而假稱折轅,衣裙所弊,讒激失所,許與疵廢,廷辱民宗。自居樞憲,糾奏寂寞。顧望縱容,無至公之議;惡直醜正,有私訐之談。宜置之徽纆,肅正國典。臣等參議,請以見事免縝所居官,輒勒外收付廷尉法獄治罪。應諸連逮,委之獄官,以法制從事。縝位應黃紙,臣輒奉白簡。



 詔聞可。璽書詰縝曰:「亮少乏才能,無聞時輩,昔經冒入群英,相與豈薄,晚節諂事江祏,為吏部,末協附梅蟲兒、茹法珍,遂執昏政。比屋罹禍,盡家塗炭,四海沸騰,天下橫潰,此誰之咎!食亂
 君之祿,不死於治世。亮協固凶黨,作威作福,靡衣玉食,女樂盈房,勢危事逼,自相吞噬。建石首題,啟靡請罪。朕錄其白旗之來,貰其既往之咎。亮反覆不忠,姦賄彰暴,有何可論!妄相談述,具以狀對。」所詰十條,縝答支離而已。亮因屏居閉掃,不通賓客。遭母憂,居喪盡禮。



 八年,詔起為秘書監,俄加通直散騎常侍,數日遷太常卿。九年,轉中書監,加散騎常侍。其年卒。詔賻錢三萬,布五十匹。謚曰煬子。



 張稷,字公喬,吳郡人也。父永,宋右光祿大夫。稷所生母遘疾歷時,稷始年十一,夜不解衣而養,永異之。及母亡,
 毀瘠過人,杖而後起。性疏率,朗悟有才略,與族兄充、融、卷等俱知名,時稱之曰:「充融卷稷,是為四張。」起家著作佐郎,不拜,頻居父母憂,六載廬于墓側。服除,為驃騎法曹行參軍,遷外兵參軍。



 齊永明中,為剡縣令,略不視事,多為山水遊。會賊唐瑤作亂,稷率厲縣人,保全縣境。入為太子洗馬,大司馬東曹掾,建安王友,大司馬從事中郎。武陵王渼為護軍,轉護軍司馬,尋為本州治中。明帝領牧,仍為別駕。時魏寇壽春,以稷為寧朔將軍、軍主,副尚書僕射沈文季鎮豫州。魏眾稱百萬,圍城累日,時經略處分,文季悉委稷焉。軍退,遷平西司馬、寧朔將軍、南
 平內史。魏又寇雍州,詔以本號都督荊、雍諸軍事。時雍州刺史曹虎度樊城岸,以稷知州事。魏師退,稷還荊州,就拜黃門侍郎,復為司馬、新興、永寧二郡太守。郡犯私諱,改永寧為長寧。尋遷司徒司馬,加輔國將軍。及江州刺史陳顯達舉兵反,以本號鎮歷陽、南譙二郡太守,遷鎮南長史、尋陽太守、輔國將軍、行江州事。尋徵還,為持節、輔國將軍、都督北徐州諸軍事、北徐州刺史。出次白下,仍遷都督南兗州諸軍事、南兗州刺史。俄進督北徐、徐、兗、青、冀五州諸軍事,將軍並如故。永元末,徵為侍中,宿衛宮城。義師至,兼衛尉江淹出奔。稷兼衛尉,副王瑩
 都督城內諸軍事。



 時東昏淫虐,義師圍城已久,城內思亡而莫有先發。北徐州刺史王珍國就稷謀之,乃使直閣張齊害東昏于含德殿。稷召尚書右僕射王亮等列坐殿前西鐘下,謂曰:「昔桀有昏德,鼎遷于殷;商紂暴虐,鼎遷于周。今獨夫自絕于天,四海已歸聖主,斯實微子去殷之時,項伯歸漢之日,可不勉哉!」乃遣國子博士范雲、舍人裴長穆等使石頭城詣高祖,高祖以稷為侍中、左衛將軍。高祖總百揆,遷大司馬左司馬。梁臺建,為散騎常侍、中書令。高祖受禪,以功封江安縣侯,邑一千戶。又為侍中、國子祭酒,領驍騎將軍,遷護軍將軍、揚州大
 中正,以事免。尋為度支尚書、前將軍、太子右衛率,又以公事免。俄為祠部尚書,轉散騎常侍、都官尚書、揚州大中正,以本職知領軍事。尋遷領軍將軍,中正、侯如故。



 時魏寇青州,詔假節、行州事。會魏軍退,仍出為散騎常侍、將軍,吳興太守,秩中二千石。下車存問遺老,引其子孫,置之右職,政稱寬恕。進號雲麾將軍,徵尚書左僕射。輿駕將欲如稷宅,以盛暑,留幸僕射省,舊臨幸供具皆酬太官饌直,帝以稷清貧,手詔不受。出為使持節、散騎常侍、都督青、冀二州諸軍事、安北將軍、青、冀二州刺史。會魏寇朐山,詔稷權頓六里,都督眾軍。還,進號鎮北將軍。



 初鬱洲接邊陲,民俗多與魏人交市。及朐山叛,或與魏通,既不自安矣;且稷寬弛無防,僚吏頗侵漁之。州人徐道角等夜襲州城,害稷,時年六十三。有司奏削爵土。



 稷性烈亮,善與人交。歷官無蓄聚,俸祿皆頒之親故,家無餘財。初去吳興郡,以僕射徵,道由吳鄉,候稷者滿水陸。稷單裝徑還京師,人莫之識,其率素如此。



 稷長女楚瑗,適會稽孔氏,無子歸宗。至稷見害,女以身蔽刃,先父卒。稷子嵊,別有傳。



 卷字令遠,稷從兄也。少以知理著稱,能清言,仕至都官尚書,天監初卒。



 王瑩,字奉光,瑯邪臨沂人也。父懋,光祿大夫、南鄉僖侯。
 瑩選尚宋臨淮公主,拜駙馬都尉,除著作佐郎,累遷太子舍人,撫軍功曹,散騎侍郎,司徒左西屬。齊高帝為驃騎將軍,引為從事中郎。頃之,出為義興太守,代謝超宗。超宗去郡,與瑩交惡,既還,間瑩於懋。懋言之於朝廷,以瑩供養不足,坐失郡廢棄。久之,為前軍諮議參軍,中書侍郎,大司馬從事中郎,未拜,丁母憂。服闋,為給事黃門郎,出為宣城太守,遷為驃騎長史。復為黃門侍郎、司馬、太子中庶子,仍遷侍中,父憂去職。服闋,復為侍中,領射聲校尉,又為冠軍將軍、東陽太守。居郡有惠政,遷吳興太守。明帝勤憂庶政,瑩頻處二郡,皆有能名。甚見褒美。
 還為太子詹事、中領軍。



 永元初,政由群小,瑩守職而不能有所是非。瑩從弟亮既當朝,於瑩素雖不善,時欲引與同事。遷尚書左僕射,未拜。會護軍崔慧景自京口奉江夏王入伐,瑩假節,率眾拒慧景於湖頭,夜為慧景所襲,眾散,瑩赴水,乘榜入樂遊,因得還臺城。慧景敗,還居領軍府。義師至,復假節,都督宮城諸軍事。建康平,高祖為相國,引瑩為左長史,加冠軍將軍,奉法駕迎和帝於江陵。帝至南州,遜位于別宮。高祖踐阼,遷侍中、撫軍將軍,封建城縣公,邑千戶。尋遷尚書左僕射,侍中、撫軍如故。頃之,為護軍將軍,復遷散騎常侍、中軍將軍、丹陽尹。
 視事三年,遷侍中、光祿大夫,領左衛將軍。俄遷尚書令、雲麾將軍,侍中如故。累進號左中權將軍,給鼓吹一部。瑩性清慎,居官恭恪,高祖深重之。



 天監十五年,遷左光祿大夫、開府儀同三司,丹陽尹、侍中如故。瑩將拜,印工鑄其印,六鑄而龜六毀,既成,頸空不實,補而用之。居職六日,暴疾卒。贈侍中、左光祿大夫、開府儀同三司。



 陳吏部尚書姚察曰:孔子稱「殷有三仁:微子去之,箕子為之奴,比干諫而死。」王亮之居亂世,勢位見矣。其於取捨,何與三仁之異歟?及奉興王,蒙寬政,為佐命,固將愧於心。迺自取廢敗,非不幸也。《易》曰:「非所據而據之,身必
 危。」亮之進退,失所據矣。惜哉!張稷因機制變,亦其時也。王瑩印章六毀,豈神之害盈乎?



\end{pinyinscope}