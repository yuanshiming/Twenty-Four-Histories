\article{卷第二十五列傳第十九 周舍 徐勉}

\begin{pinyinscope}

 周捨,字昇逸,汝南安城人,晉左光祿大夫抃之八世孫也。父顒,齊中書侍郎,有名於時。捨幼聰潁,顒異之,臨卒謂曰:「汝不患不富貴,但當持之以道德。」既長,博學多通,尤精義理,善誦書,背文諷說,音韻清辯。起家齊太學博士,遷後軍行參軍。建武中,魏人吳包南歸,有儒學,尚書僕射江祏招包講。舍造坐,累折包,辭理遒逸,由是名為
 口辯。王亮為丹陽尹,聞而悅之,辟為主簿,政事多委焉。遷太常丞。



 梁臺建,為奉常丞。高祖即位,博求異能之士。吏部尚書范雲與顒素善,重捨才器,言之於高祖,召拜尚書祠部郎。時天下草創,禮儀損益,多自捨出。尋為後軍記室參軍、秣陵令。入為中書通事舍人,累遷太子洗馬,散騎常侍,中書侍郎,鴻臚卿。時王亮得罪歸家,故人莫有至者,舍獨敦恩舊,及卒,身營殯葬,時人稱之。遷尚書吏部郎,太子右衛率,右衛將軍,雖居職屢徙,而常留省內,罕得休下。國史詔誥,儀體法律,軍旅謀謨,皆兼掌之。日夜侍上,預機密,二十餘年未嘗離左右。捨素辯給,
 與人泛論談謔,終日不絕口,而竟無一言漏泄機事,眾尤歎服之。性儉素,衣服器用,居處床席,如布衣之貧者。每入官府,雖廣廈華堂,閨閣重邃,捨居之則塵埃滿積。以荻為鄣,壞亦不營。為右衛,母憂去職,起為明威將軍、右驍騎將軍。服闋,除侍中,領步兵校尉,未拜,仍遷員外散騎常侍、太子左衛率。頃之,加散騎常侍、本州大中正,遷太子詹事。



 普通五年,南津獲武陵太守白渦書,許遺舍面錢百萬,津司以聞。雖書自外入,猶為有司所奏,舍坐免。遷右驍騎將軍,知太子詹事。以其年卒,時年五十六。上臨哭,哀慟左右。詔曰:「太子詹事、豫州大中正舍,奄
 至殞喪,惻愴于懷。其學思堅明,志行開敏,劬勞機要,多歷歲年,才用未窮,彌可嗟慟。宜隆追遠,以旌善人。可贈侍中、護軍將軍,鼓吹一部,給東園秘器,朝服一具,衣一襲,喪事隨由資給。謚曰簡子。」明年,又詔曰:「故侍中、護軍將軍簡子捨,義該玄儒,博窮文史,奉親能孝,事君盡忠,歷掌機密,清貞自居。食不重味,身靡兼衣。終亡之日,內無妻妾,外無田宅,兩兒單貧,有過古烈。往者,南司白渦之劾,恐外議謂朕有私,致此黜免,追愧若人一介之善。外可量加褒異,以旌善人。」二子:弘義,弘信。



 徐勉,字脩仁,東海郯人也。祖長宗,宋高祖霸府行參軍。
 父融,南昌相。勉幼孤貧,早勵清節。年六歲,時屬霖雨,家人祈霽,率爾為文,見稱耆宿。及長,篤志好學。起家國子生。太尉文憲公王儉時為祭酒,每稱勉有宰輔之量。射策舉高第,補西陽王國侍郎。尋遷太學博士,鎮軍參軍,尚書殿中郎,以公事免。又除中兵郎、領軍長史。瑯邪王元長才名甚盛,嘗欲與勉相識,每託人召之。勉謂人曰:「王郎名高望促,難可輕醿衣裾。」俄而元長及禍,時人莫不服其機鑒。



 初與長沙宣武王遊,高祖深器賞之。及義兵至京邑,勉於新林謁見,高祖甚加恩禮,使管書記。高祖踐阼,拜中書侍郎,遷建威將軍、後軍諮議參軍、本邑
 中正、尚書左丞。自掌樞憲,多所糾舉,時論以為稱職。天監二年,除給事黃門侍郎、尚書吏部郎,參掌大選。遷侍中。時王師北伐,候驛填委。勉參掌軍書,劬勞夙夜,動經數旬,乃一還宅。每還,群犬驚吠。勉歎曰:「吾憂國忘家,乃至於此。若吾亡後,亦是傳中一事。」六年,除給事中、五兵尚書,遷吏部尚書。勉居選官,彞倫有序,既閑尺牘,兼善辭令,雖文案填積,坐客充滿,應對如流,手不停筆。又該綜百氏,皆為避諱。常與門入夜集,客有虞皓求詹事五官,勉正色答云:「今夕止可談風月,不宜及公事。」故時人咸服其無私。



 除散騎常侍,領遊擊將軍,未拜,改領太子
 右衛率。遷左衛將軍,領太子中庶子,侍東宮。昭明太子尚幼,敕知宮事。太子禮之甚重,每事詢謀。嘗於殿內講《孝經》,臨川靖惠王、尚書令沈約備二傅,勉與國子祭酒張充為執經,王瑩、張稷、柳憕、王暕為侍講。時選極親賢,妙盡時譽,勉陳讓數四。又與沈約書,求換侍講,詔不許,然後就焉。轉太子詹事,領雲騎將軍,尋加散騎常侍,遷尚書右僕射,詹事如故。又改授侍中,頻表解宮職,優詔不許。



 時人間喪事,多不遵禮,朝終夕殯,相尚以速。勉上疏曰:「《禮記問喪》云:『三日而後斂者,以俟其生也。三日而不生,亦不生矣。』自頃以來,不遵斯制。送終之禮,殯以期
 日,潤屋豪家,乃或半晷,衣衾棺槨,以速為榮,親戚徒隸,各念休反。故屬纊纔畢,灰釘已具,忘狐鼠之顧步,愧燕雀之徊翔。傷情滅理,莫此為大。且人子承衾之時,志懣心絕,喪事所資,悉關他手,愛憎深淺,事實難原。如覘視或爽,存沒違濫,使萬有其一,怨酷已多。豈若緩其告斂之晨,申其望生之冀。請自今士庶,宜悉依古,三日大斂。如有不奉,加以糾繩。」詔可其奏。



 尋授宣惠將軍,置佐史,侍中、僕射如故。又除尚書僕射、中衛將軍。勉以舊恩,越升重位,盡心奉上,知無不為。爰自小選,迄于此職,常參掌衡石,甚得士心。禁省中事,未嘗漏洩。每有表奏,輒焚
 槁草。博通經史,多識前載。朝儀國典,婚冠吉凶,勉皆預圖議。普通六年,上修五禮表曰:臣聞「立天之道,曰陰與陽;立人之道,曰仁與義。」故稱「導之以德,齊之以禮」。夫禮所以安上治民,弘風訓俗,經國家,利後嗣者也。唐虞三代,咸必由之。在乎有周,憲章尤備,因殷革夏,損益可知。雖復經禮三百,曲禮三千,經文三百,威儀三千,其大歸有五,即宗伯所掌典禮:吉為上,凶次之,賓次之,軍次之,嘉為下也。故祠祭不以禮,則不齊不莊;喪紀不以禮,則背死忘生者眾;賓客不以禮,則朝覲失其儀;軍旅不以禮,則致亂於師律;冠婚不以禮,則男女失其時。為國修
 身,於斯攸急。



 洎周室大壞,王道既衰,官守斯文,日失其序。禮樂征伐,出自諸侯,《小雅》盡廢,舊章缺矣。是以韓宣適魯,知周公之德;叔侯在晉,辨郊勞之儀。戰國從橫,政教愈泯;暴秦滅學,掃地無餘。漢氏鬱興,日不暇給,猶命叔孫於外野,方知帝王之為貴。末葉紛綸,遞有興毀,或以武功銳志,或好黃老之言,禮義之式,於焉中止。及東京曹褒,南宮制述,集其散略,百有餘篇,雖寫以尺簡,而終闕平奏。其後兵革相尋,異端互起,章句既淪,俎豆斯輟。方領矩步之容,事滅於旌鼓;蘭臺石室之文,用盡於帷蓋。至乎晉初,爰定新禮,荀抃制之於前,摯虞刪之於
 末。既而中原喪亂,罕有所遺;江左草創,因循而已。釐革之風,是則未暇。



 伏惟陛下睿明啟運,先天改物,撥亂惟武,經世以文。作樂在乎功成,制禮弘於業定。光啟二學,皇枝等於貴遊;闢茲五館,草萊升以好爵。爰自受命,迄于告成,盛德形容備矣,天下能事畢矣。明明穆穆,無德而稱焉。至若玄符靈貺之祥,浮溟機山之贐,固亦日書左史,副在司存,今可得而略也。是以命彼群才,搜甘泉之法;延茲碩學,闡曲臺之儀。淄上淹中之儒,連蹤繼軌;負笈懷鉛之彥,匪旦伊夕。諒以化穆三雍,人從五典,秩宗之教,勃焉以興。



 伏尋所定五禮,起齊永明三年,太子
 步兵校尉伏曼容表求制一代禮樂,于時參議置新舊學士十人,止修五禮,諮稟衛將軍丹陽尹王儉,學士亦分住郡中,製作歷年,猶未克就。及文憲薨殂,遺文散逸,後又以事付國子祭酒何胤,經涉九載,猶復未畢。建武四年,胤還東山,齊明帝敕委尚書令徐孝嗣。舊事本末,隨在南第。永元中,孝嗣於此遇禍,又多零落。當時鳩斂所餘,權付尚書左丞蔡仲熊、驍騎將軍何佟之,共掌其事。時修禮局住在國子學中門外,東昏之代,頻有軍火,其所散失,又踰太半。天監元年,佟之啟審省置之宜,敕使外詳。時尚書參詳,以天地初革,庶務權輿,宜俟隆平,
 徐議刪撰。欲且省禮局,併還尚書儀曹。詔旨云:「禮壞樂缺,故國異家殊,實宜以時修定,以為永準。但頃之修撰,以情取人,不以學進;其掌知者,以貴總一,不以稽古,所以歷年不就,有名無實。此既經國所先,外可議其人,人定,便即撰次。」於是尚書僕射沈約等參議,請五禮各置舊學士一人,人各自舉學士二人,相助抄撰。其中有疑者,依前漢石渠、後漢白虎,隨源以聞,請旨斷決。乃以舊學士右軍記室參軍明山賓掌吉禮,中軍騎兵參軍嚴植之掌凶禮,中軍田曹行參軍兼太常丞賀蒨掌賓禮,征虜記室參軍陸璉掌軍禮,右軍參軍司馬裴掌嘉禮,
 尚書左丞何佟之總參其事。佟之亡後,以鎮北諮議參軍伏芃代之。後又以芃代嚴植之掌凶禮。芃尋遷官,以《五經》博士繆昭掌凶禮。復以禮儀深廣,記載殘缺,宜須博論,共盡其致,更使鎮軍將軍丹陽尹沈約、太常卿張充及臣三人同參厥務。臣又奉別敕,總知其事。末又使中書侍郎周捨、庾於陵二人復豫參知。若有疑義,所掌學士當職先立議,通諮五禮舊學士及參知,各言同異,條牒啟聞,決之制旨。疑事既多,歲時又積,制旨裁斷,其數不少。莫不網羅經誥,玉振金聲,義貫幽微,理入神契。前儒所不釋,後學所未聞。凡諸奏決,皆載篇首,具列聖
 旨,為不刊之則。洪規盛範,冠絕百王;茂實英聲,方垂千載。寧孝宣之能擬,豈孝章之足云。



 五禮之職,事有繁簡,及其列畢,不得同時。《嘉禮儀注》以天監六年五月七日上尚書,合十有二秩,一百一十六卷,五百三十六條;《賓禮儀注》以天監六年五月二十日上尚書,合十有七秩,一百三十三卷,五百四十五條;《軍禮儀注》以天監九年十月二十九日上尚書,合十有八秩,一百八十九卷,二百四十條;《吉禮儀注》以天監十一年十一月十日上尚書,合二十有六秩,二百二十四卷,一千五條;《凶禮儀注》以天監十一年十一月十七日上尚書,合四十有七秩,五
 百一十四卷,五千六百九十三條:大凡一百二十秩,一千一百七十六卷,八千一十九條。又列副秘閣及《五經》典書各一通,繕寫校定,以普通五年二月始獲洗畢。



 竊以撰正履禮,歷代罕就,皇明在運,厥功克成。周代三千,舉其盈數;今之八千,隨事附益。質文相變,故其數兼倍,猶如八卦之爻,因而重之,錯綜成六十四也。昔文武二王,所以綱紀周室,君臨天下,公旦修之,以致太平龍鳳之瑞。自斯厥後,甫備茲日。孔子曰:「其有繼周,雖百世可知。」豈所謂齊功比美者歟!臣以庸識,謬司其任,淹留歷稔,允當斯責;兼勒成之初,未遑表上,實由才輕務廣,思
 力不周,永言慚惕,無忘寤寐。自今春輿駕將親六師,搜尋軍禮,閱其條章,靡不該備。所謂郁郁文哉,煥乎洋溢,信可以懸諸日月,頒之天下者矣。愚心喜抃,彌思陳述;兼前後聯官,一時皆逝,臣雖幸存,耄已將及,慮皇世大典,遂闕騰奏,不任下情,輒具載撰脩始末,并職掌人、所成卷秩、條目之數,謹拜表以聞。



 詔曰:「經禮大備,政典載弘,今詔有司,案以行事也。」又詔曰:「勉表如此。因革允釐,憲章孔備,功成業定,於是乎在。可以光被八表,施諸百代,俾萬世之下,知斯文在斯。主者其按以遵行,勿有失墜。」尋加中書令,給親信二十人。勉以疾自陳,求解內任。
 詔不許,乃令停下省,三日一朝,有事遣主書論決。腳疾轉劇,久闕朝覲,固陳求解,詔乃賚假,須疾差還省。



 勉雖居顯位,不營產業,家無蓄積,俸祿分贍親族之窮乏者。門人故舊或從容致言。勉乃答曰:「人遺子孫以財,我遺之以清白。子孫才也,則自致輜軿;如其不才,終為他有。」嘗為書誡其子崧曰:吾家世清廉,故常居貧素,至於產業之事,所未嘗言,非直不經營而已。薄躬遭逢,遂至今日,尊官厚祿,可謂備之。每念叨竊若斯,豈由才致,仰藉先代風範及以福慶,故臻此耳。古人所謂「以清白遺子孫,不亦厚乎!」又云:「遺子黃金滿惣,不如一經。」詳求此言,
 信非徒語。吾雖不敏,實有本志,庶得遵奉斯義,不敢墜失。所以顯貴以來,將三十載,門人故舊,亟薦便宜,或使創闢田園,或勸興立邸店,又欲舳艫運致,亦令貨殖聚斂。若此眾事,皆距而不納。非謂拔葵去織,且欲省息紛紜。



 中年聊於東田間營小園者,非在播藝,以要利入,正欲穿池種樹,少寄情賞。又以郊際閑曠,終可為宅,儻獲懸車致事,實欲歌哭於斯。慧日、十住等,既應營婚,又須住止,吾清明門宅,無相容處。所以爾者,亦復有以;前割西邊施宣武寺,既失西廂,不復方幅,意亦謂此逆旅舍耳,何事須華?常恨時人謂是我宅。古往今來,豪富繼踵,
 高門甲第,連闥洞房,宛其死矣,定是誰室?但不能不為培塿之山,聚石移果,雜以花卉,以娛休沐,用託性靈。隨便架立,不在廣大,惟功德處,小以為好。所以內中逼促,無復房宇。近營東邊兒孫二宅,乃藉十住南還之資,其中所須,猶為不少,既牽挽不至,又不可中塗而輟,郊間之園,遂不辦保,貨與韋黯,乃獲百金,成就兩宅,已消其半。尋園價所得,何以至此?由吾經始歷年,粗已成立,桃李茂密,桐竹成陰,塍陌交通,渠畎相屬,華樓迥榭,頗有臨眺之美;孤峰叢薄,不無糾紛之興。瀆中並饒菰蔣,湖里殊富芰蓮。雖云人外,城闕密邇,韋生欲之,亦雅有情
 趣。追述此事,非有吝心,蓋是筆勢所至耳。憶謝靈運《山家詩》云:「中為天地物,今成鄙夫有。」吾此園有之二十載矣,今為天地物,物之與我,相校幾何哉!此吾所餘,今以分汝,營小田舍,親累既多,理亦須此。且釋氏之教,以財物謂之外命;儒典亦稱「何以聚人曰財」。況汝曹常情,安得忘此。聞汝所買姑孰田地,甚為舄鹵,彌復何安。所以如此,非物競故也。雖事異寢丘,聊可仿佛。孔子曰:「居家理治,可移於官。」既已營之,宜使成立。進退兩亡,更貽恥笑。若有所收獲,汝可自分贍內外大小,宜令得所,非吾所知,又復應沾之諸女耳。汝既居長,故有此及。



 凡為人
 長,殊復不易,當使中外諧緝,人無間言,先物後己,然後可貴。老生云:「後其身而身先。」若能爾者,更招巨利。汝當自勖,見賢思齊,不宜忽略以棄日也。非徒棄日,乃是棄身,身名美惡,豈不大哉!可不慎歟?今之所敕,略言此意。正謂為家已來,不事資產,既立墅舍,以乖舊業,陳其始末,無愧懷抱。兼吾年時朽暮,心力稍殫,牽課奉公,略不克舉,其中餘暇,裁可自休。或復冬日之陽,夏日之陰,良辰美景,文案間隙,負杖躡履,逍遙陋館,臨池觀魚,披林聽鳥,濁酒一杯,彈琴一曲,求數刻之暫樂,庶居常以待終,不宜復勞家間細務。汝交關既定,此書又行,凡所資須,付
 給如別。自茲以後,吾不復言及田事,汝亦勿復與吾言之。假使堯水湯旱,吾豈知如何;若其滿庾盈箱,爾之幸遇。如斯之事,並無俟令吾知也。《記》云:「夫孝者,善繼人之志,善述人之事。」今且望汝全吾此志,則無所恨矣。



 勉第二子悱卒,痛悼甚至,不欲久廢王務,乃為《答客喻》。其辭曰:普通五年春二月丁丑,餘第二息晉安內史悱喪之問至焉,舉家傷悼,心情若隕。二宮並降中使,以相慰勖,親遊賓客,畢來弔問,輒慟哭失聲,悲不自已,所謂父子天性,不知涕之所從來也。



 於是門人慮其肆情所鐘,容致委頓,乃斂衽而進曰:「僕聞古往今來,理運之常數;春
 榮秋落,氣象之定期。人居其間,譬諸逆旅,生寄死歸,著於通論,是以深識之士,悠爾忘懷。東門歸無之旨,見稱往哲;西河喪明之過,取誚友朋。足下受遇於朝,任居端右,憂深責重,休戚是均,宜其遺情下流,止哀加飯,上存奉國,俯示隆家。豈可縱此無益,同之兒女,傷神損識,或虧生務。門下竊議,咸為君侯不取也。」



 餘雪泣而答曰:「彭殤之達義,延吳之雅言,亦常聞之矣;顧所以未能弭意者,請陳其說。夫植樹階庭,欽柯葉之茂;為山累仞,惜覆簣之功。故秀而不實,尼父為之歎息;析彼歧路,楊子所以留連。事有可深,聖賢靡抑。今吾所悲,亦以悱始踰立
 歲,孝悌之至,自幼而長,文章之美,得之天然,好學不倦,居無塵雜,多所著述,盈帙滿笥,淡然得失之際,不見喜慍之容。及翰飛東朝,參伍盛列,其所遊往,皆一時才俊,賦詩頌詠,終日忘疲。每從容謂吾以遭逢時來,位隆任要,當應推賢下士,先物後身,然後可以報恩明主,克保元吉。俾餘二紀之中,忝竊若是,幸無大過者,繄此子之助焉。自出閩區,政存清靜,冀其旋反,少慰衰暮,言念今日,眇然長往。加以闔棺千里之外,未知歸骨之期,雖復無情之倫,庸詎不痛於昔!夷甫孩抱中物,尚盡慟以待賓;安仁未及七旬,猶殷勤於詞賦。況夫名立宦成,半途
 而廢者,亦焉可已已哉。求其此懷,可謂苗實之義。諸賢既貽格言,喻以大理,即日輟哀,命駕修職事焉。」



 中大通三年,又以疾自陳,移授特進、右光祿大夫、侍中、中衛將軍,置佐史,餘如故。增親信四十人。兩宮參問,冠蓋結轍;服膳醫藥,皆資天府。有敕每欲臨幸,勉以拜伏有虧,頻啟停出,詔許之,遂停輿駕。大同元年,卒,時年七十。高祖聞而流涕,即日車駕臨殯,乃詔贈特進、右光祿大夫、開府儀同三司,餘並如故。給東園秘器,朝服一具,衣一襲。贈錢二十萬,布百匹。皇太子亦舉哀朝堂。謚曰簡肅公。



 勉善屬文,勤著述,雖當機務,下筆不休。嘗以起居注煩
 雜,乃加刪撰為《別起居注》六百卷;《左丞彈事》五卷;在選曹,撰《選品》五卷;齊時,撰《太廟祝文》二卷;以孔釋二教殊途同歸,撰《會林》五十卷。凡所著前後二集四十五卷,又為《婦人集》十卷,皆行於世。大同三年,故佐史尚書左丞劉覽等詣闕陳勉行狀,請刊石紀德,即降詔許立碑於墓云。



 悱字敬業,幼聰敏,能屬文。起家著作佐郎,轉太子舍人,掌書記之任。累遷洗馬、中舍人,猶管書記。出入宮坊者歷稔,以足疾出為湘東王友,遷晉安內史。



 陳吏部尚書姚察曰:徐勉少而厲志忘食,發憤脩身,慎
 言行,擇交游;加運屬興王,依光日月,故能明經術以綰青紫,出閭閻而取卿相。及居重任,竭誠事主,動師古始,依則先王,提衡端軌,物無異議,為梁宗臣,盛矣。



\end{pinyinscope}