\article{卷第五十一列傳第四十五 處士何點 弟胤 阮孝緒 陶弘景 諸葛璩 沈顗 劉慧斐 範元琰 劉訏 劉 庾詵 張孝秀 庾承先}

\begin{pinyinscope}

 《易》曰:「君子遁世無悶,獨立不懼。」孔子稱長沮、桀溺隱者也。古之隱者,或恥聞禪代,高讓帝王,以萬乘為垢辱,之死亡而無悔。此則輕生重道,希世間出,隱之上者也。或
 託仕監門,寄臣柱下,居易而以求其志,處污而不愧其色。此所謂大隱隱於市朝,又其次也。或裸體佯狂,盲喑絕世,棄禮樂以反道,忍孝慈而不恤。此全身遠害,得大雅之道,又其次也。然同不失語默之致,有幽人貞吉矣。與夫沒身亂世,爭利幹時者,豈同年而語哉!《孟子》曰:「今人之於爵祿,得之若其生,失之若其死。」《淮南子》曰:「人皆鑒於止水,不鑒於流潦。」夫可以揚清激濁,抑貪止競,其惟隱者乎!自古帝王,莫不崇尚其道。雖唐堯不屈巢、許,周武不降夷、齊;以漢高肆慢而長揖黃、綺,光武按法而折意嚴、周;自茲以來,世有人矣!有梁之盛,繼紹風猷。斯
 乃道德可宗,學藝可範,故以備《處士篇》云。



 何點,字子晳,廬江灊人也。祖尚之,宋司空。父鑠,宜都太守。鑠素有風疾,無故害妻,坐法死。點年十一,幾至滅性。及長,感家禍,欲絕婚宦,尚之彊為之娶瑯邪王氏。禮畢,將親迎,點累涕泣,求執本志,遂得罷。



 容貌方雅,博通群書,善談論。家本甲族,親姻多貴仕。點雖不入城府,而遨遊人世,不簪不帶,或駕柴車,躡草矰,恣心所適,致醉而歸,士大夫多慕從之,時人號為「通隱」。兄求,亦隱居吳郡虎丘山。求卒,點菜食不飲酒,訖于三年,要帶減半。



 宋泰始末,徵太子洗馬。齊初,累徵中書郎、太子中庶子,並不
 就。與陳郡謝、吳國張融、會稽孔稚珪為莫逆友。從弟遁,以東籬門園居之,稚珪為築室焉。園內有卞忠貞冢,點植花卉於冢側,每飲必舉酒酹之。初,褚淵、王儉為宰相,點謂人曰:「我作《齊書贊》,云『淵既世族,儉亦國華;不賴舅氏,遑恤國家』。」王儉聞之,欲候點,知不可見,乃止。豫章王嶷命駕造點,點從後門遁去。司徒、竟陵王子良欲就見之,點時在法輪寺,子良乃往請,點角巾登席,子良欣悅無已,遺點嵇叔夜酒杯、徐景山酒鐺。



 點少時嘗患渴痢,積歲不愈。後在吳中石佛寺建講,於講所晝寢,夢一道人形貌非常,授丸一掬,夢中服之,自此而差,時人以
 為淳德所感。性通脫,好施與,遠近致遺,一無所逆,隨復散焉。嘗行經朱雀門街,有自車後盜點衣者,見而不言,傍有人擒盜與之,點乃以衣施盜,盜不敢受,點命告有司,盜懼,乃受之,催令急去。點雅有人倫識鑒,多所甄拔,知吳興丘遲於幼童,稱濟陽江淹於寒素,悉如其言。



 點既老,又娶魯國孔嗣女,嗣亦隱者也。點雖婚,亦不與妻相見,築別室以處之,人莫喻其意也。吳國張融少時免官,而為詩有高尚之言,點答詩曰:「昔聞東都日,不在簡書前。」雖戲也,而融久病之。及點後婚,融始為詩贈點曰:「惜哉何居士,薄暮遘荒淫。」點亦病之,而無以釋也。



 高祖
 與點有舊,及踐阼,手詔曰:「昔因多暇,得訪逸軌,坐修竹,臨清池,忘今語古,何其樂也。暫別丘園,十有四載,人事艱阻,亦何可言。自應運在天,每思相見,密邇物色,勞甚山阿。嚴光排九重,踐九等,談天人,敘故舊,有所不臣,何傷於高?文先以皮弁謁子桓,伯況以縠綃見文叔,求之往策,不無前例。今賜卿鹿皮巾等。後數日,望能入也。」點以巾褐引入華林園,高祖甚悅,賦詩置酒,恩禮如舊。仍下詔曰:「前征士何點,高尚其道,志安容膝,脫落形骸,棲志窅冥。朕日昃思治,尚想前哲;況親得同時,而不與為政。喉脣任切,必俟邦良,誠望惠然,屈居獻替。可徵為侍中。」
 辭疾不赴。乃復詔曰:「徵士何點,居貞物表,縱心塵外,夷坦之風,率由自遠。往因素志,頗申宴言,眷彼子陵,情兼惟舊。昔仲虞邁俗,受俸漢朝;安道逸志,不辭晉祿。此蓋前代盛軌,往賢所同。可議加資給,並出在所,日費所須,太官別給。既人高曜卿,故事同垣下。」



 天監三年,卒,時年六十八。詔曰:「新除侍中何點,棲遲衡泌,白首不渝。奄至殞喪,倍懷傷惻。可給第一品材一具,賻錢二萬、布五十匹。喪事所須,內監經理。」又敕點弟胤曰:「賢兄征君,弱冠拂衣,華首一操。心遊物表,不滯近跡;脫落形骸,寄之遠理。性情勝致,遇興彌高;文會酒德,撫際逾遠。朕膺籙
 受圖,思長聲教。朝多君子,既貴成雅俗;野有外臣,宜弘此難進。方賴清徽,式隆大業。昔在布衣,情期早著,資以仲虞之秩,待以子陵之禮,聽覽暇日,角巾引見,窅然汾射,茲焉有託。一旦萬古,良懷震悼。卿友于純至,親從凋亡;偕老之願,致使反奪;纏綿永恨,伊何可任。永矣柰何!」點無子,宗人以其從弟耿子遲任為嗣。



 胤,字子季,點之弟也。年八歲,居憂哀毀若成人。既長好學。師事沛國劉獻,受《易》及《禮記》、《毛詩》,又入鐘山定林寺聽內典,其業皆通。而縱情誕節,時人未之知也,唯獻與汝南周顒深器異之。



 起家齊秘書郎,遷太子舍人。出為
 建安太守,為政有恩信,民不忍欺。每伏臘放囚還家,依期而返。入為尚書三公郎,不拜,遷司徒主簿。注《易》,又解《禮記》,於卷背書之,謂為《隱義》。累遷中書郎、員外散騎常侍、太尉從事中郎、司徒右長史、給事黃門侍郎、太子中庶子、領國子博士、丹陽邑中正。尚書令王儉受詔撰新禮,未就而卒。又使特進張緒續成之,緒又卒;屬在司徒竟陵王子良,子良以讓胤,乃置學士二十人,佐胤撰錄。永明十年,遷侍中,領步兵校尉,轉為國子祭酒。鬱林嗣位,胤為后族,甚見親待。累遷左民尚書、領驍騎、中書令、領臨海、巴陵王師。



 胤雖貴顯,常懷止足。建武初,已築室
 郊外,號曰小山,恒與學徒遊處其內。至是,遂賣園宅,欲入東山,未及發,聞謝朏罷吳興郡不還,胤恐後之,乃拜表辭職,不待報輒去。明帝大怒,使御史中丞袁昂奏收胤,尋有詔許之。胤以會稽山多靈異,往遊焉,居若邪山雲門寺。初,胤二兄求、點並棲遁,求先卒,至是胤又隱,世號點為大山;胤為小山,亦曰東山。



 永元中,徵太常、太子詹事,並不就。高祖霸府建,引胤為軍謀祭酒,與書曰:「想恒清豫,縱情林壑,致足懽也。既內絕心戰,外勞物役,以道養和,履候無爽。若邪擅美東區,山川相屬,前世嘉賞,是為樂土。僕推遷簿官,自東徂西,悟言素對,用成睽闋,傾
 首東顧,曷日無懷。疇昔懽遇,曳裾儒肆,實欲臥遊千載,畋漁百氏,一行為吏,此事遂乖。屬以世道威夷,仍離屯故,投袂數千,剋黜釁禍。思得矚卷諮款,寓情古昔,夫豈不懷,事與願謝。君清襟素託,棲寄不近,中居人世,殆同隱淪。既俯拾青組,又脫屣朱黻。但理存用捨,義貴隨時,往識禍萌,實為先覺,超然獨善,有識欽嗟。今者為邦,貧賤咸恥,好仁由己,幸無凝滯。比別具白,此未盡言。今遣候承音息,矯首還翰,慰其引領。」胤不至。



 高祖踐阼,詔為特進、右光祿大夫。手敕曰:「吾猥當期運,膺此樂推,而顧己蒙蔽,昧於治道。雖復劬勞日昃,思致隆平,而先王遺
 範,尚蘊方策,自舉之用,存乎其人。兼以世道澆暮,爭詐繁起,改俗遷風,良有未易。自非以儒雅弘朝,高尚軌物,則汩流所至,莫知其限。治人之與治身,獨善之與兼濟,得失去取,為用孰多。吾雖不學,頗好博古,尚想高塵,每懷擊節。今世務紛亂,憂責是當,不得不屈道巖阿,共成世美。必望深達往懷,不吝濡足。今遣領軍司馬王果宣旨諭意,遲面在近。」果至,胤單衣鹿巾,執經卷,下床跪受詔書,就席伏讀。胤因謂果曰:「吾昔於齊朝欲陳兩三條事,一者欲正郊丘,二者欲更鑄九鼎,三者欲樹雙闕。世傳晉室欲立闕,王丞相指牛頭山云:『此天闕也』,是則未
 明立闕之意。闕者,謂之象魏。縣象法於其上,浹日而收之。象者,法也;魏者,當塗而高大貌也。鼎者神器,有國所先,故王孫滿斥言,楚子頓盡。圓丘國郊,舊典不同。南郊祠五帝靈威仰之類,圓丘祠天皇大帝、北極大星是也。往代合之郊丘,先儒之巨失。今梁德告始,不宜遂因前謬。卿宜詣闕陳之。」果曰:「僕之鄙劣,豈敢輕議國典?此當敬俟叔孫生耳。」胤曰:「卿詎不遣傳詔還朝拜表,留與我同遊邪?」果愕然曰:「古今不聞此例。」胤曰:「《檀弓》兩卷,皆言物始。自卿而始,何必有例。」果曰:「今君遂當邈然絕世,猶有致身理不?」胤曰:「卿但以事見推,吾年已五十七,月食
 四斗米不盡,何容得有宦情?昔荷聖王跂識,今又蒙旌賁,甚願詣闕謝恩,但比腰腳大惡,此心不遂耳。」



 果還,以胤意奏聞,有敕給白衣尚書祿,胤固辭。又敕山陰庫錢月給五萬,胤又不受。乃敕胤曰:「頃者學業淪廢,儒術將盡,閭閻搢紳,鮮聞好事。吾每思弘獎,其風未移,當扆興言為歎。本欲屈卿暫出,開導後生,既屬廢業,此懷未遂,延佇之勞,載盈夢想。理舟虛席,須俟來秋,所望惠然,申其宿抱耳。卿門徒中經明行修,厥數有幾?且欲瞻彼堂堂,置此周行。便可具以名聞,副其勞望。」又曰:「比歲學者殊為寡少,良由無復聚徒,故明經斯廢。每一念此,為之
 慨然。卿居儒宗,加以德素,當敕後進有意向者,就卿受業。想深思誨誘,使斯文載興。」於是遣何子朗、孔壽等六人於東山受學。



 太守衡陽王元簡深加禮敬,月中常命駕式閭,談論終日。胤以若邪處勢迫隘,不容生徒,乃遷秦望山。山有飛泉,西起學舍,即林成援,因巖為堵。別為小閣室,寢處其中,躬自啟閉,僮僕無得至者。山側營田二頃,講隙從生徒遊之。胤初遷,將築室,忽見二人著玄冠,容貌甚偉,問胤曰:「君欲居此邪?」乃指一處云:「此中殊吉。」忽不復見,胤依其言而止焉。尋而山發洪水,樹石皆倒拔,唯胤所居室巋然獨存。元簡乃命記室參軍鐘嶸
 作《瑞室頌》,刻石以旌之。及元簡去郡,入山與胤別,送至都賜埭,去郡三里,因曰:「僕自棄人事,交遊路斷,自非降貴山藪,豈容復望城邑?此埭之遊,於今絕矣。」執手涕零。



 何氏過江,自晉司空充並葬吳西山。胤家世年皆不永,唯祖尚之至七十二。胤年登祖壽,乃移還吳,作《別山詩》一首,言甚悽愴。至吳,居虎丘西寺講經論,學徒復隨之,東境守宰經途者,莫不畢至。胤常禁殺,有虞人逐鹿,鹿徑來趨胤,伏而不動。又有異鳥如鶴,紅色,集講堂,馴狎如家禽焉。



 初,開善寺藏法師與胤遇於秦望,後還都,卒於鐘山。其死日,胤在般若寺,見一僧授胤香奩并函書,
 云「呈何居士」,言訖失所在。胤開函,乃是《大莊嚴論》,世中未有。又於寺內立明珠柱,乃七日七夜放光,太守何遠以狀啟。昭明太子欽其德,遣舍人何思澄致手令以褒美之。



 中大通三年,卒,年八十六。先是胤疾,妻江氏夢神人告之曰:「汝夫壽盡。既有至德,應獲延期,爾當代之。」妻覺說焉,俄得患而卒,胤疾乃瘳。至是胤夢一神女并八十許人,並衣帢,行列至前,俱拜床下,覺又見之,便命營凶具。既而疾動,因不自治。



 胤注《百法論》、《十二門論》各一卷,注《周易》十卷、《毛詩總集》六卷、《毛詩隱義》十卷、《禮記隱義》二十卷、《禮答問》五十五卷。



 子撰,亦不仕,廬陵王辟為
 主簿,不就。



 阮孝緒,字士宗,陳留尉氏人也。父彥之,宋太尉從事中郎。孝緒七歲,出後從伯胤之。胤之母周氏卒,有遺財百餘萬,應歸孝緒,孝緒一無所納,盡以歸胤之姊瑯邪王晏之母,聞者咸嘆異之。



 幼至孝,性沉靜,雖與兒童遊戲,恆以穿池築山為樂。年十三,遍通《五經》。十五,冠而見其父,彥之誡曰:「三加彌尊,人倫之始。宜思自勖,以庇爾躬。」答曰:「願迹松子於瀛海,追許由於穹谷,庶保促生,以免塵累。」自是屏居一室,非定省未嘗出戶,家人莫見其面,親友因呼為「居士」。外兄王晏貴顯,屢至其門,孝緒度之
 必至顛覆,常逃匿不與相見。曾食醬美,問之,云是王家所得,便吐飧覆醢。及晏誅,其親戚咸為之懼,孝緒曰:「親而不黨,何坐之及?」竟獲免。



 義師圍京城,家貧無以爨,僮妾竊鄰人樵以繼火。孝緒知之,乃不食,更令撤屋而炊。所居室唯有一鹿床,竹樹環繞。天監初,御史中丞任昉尋其兄履之,欲造而不敢,望而歎曰:「其室雖邇,其人甚遠。」為名流所欽尚如此。



 十二年,與吳郡范元琰俱徵,並不到。陳郡袁峻謂之曰:「往者,天地閉,賢人隱;今世路已清,而子猶遁,可乎?」答曰:「昔周德雖興,夷、齊不厭薇蕨;漢道方盛,黃、綺無悶山林。為仁由己,何關人世!況僕非往
 賢之類邪?」



 後於鐘山聽講,母王氏忽有疾,兄弟欲召之。母曰:「孝緒至性冥通,必當自到。」果心驚而返,鄰里嗟異之。合藥須得生人參,舊傳鐘山所出,孝緒躬歷幽險,累日不值。忽見一鹿前行,孝緒感而隨後,至一所遂滅,就視,果獲此草。母得服之,遂愈。時皆歎其孝感所致。



 時有善筮者張有道謂孝緒曰:「見子隱跡而心難明,自非考之龜蓍,無以驗也。」及布卦,既揲五爻,曰:「此將為《咸》,應感之法,非嘉遁之兆。」孝緒曰:「安知後爻不為上九?」果成《遁卦》。有道歎曰:「此謂『肥遁無不利。』象實應德,心迹并也。」孝緒曰:「雖獲《遁卦》,而上九爻不發,升遐之道,便當高謝許
 生。」乃著《高隱傳》,上自炎、黃,終於天監之末,斟酌分為三品,凡若干卷。又著論云:「夫至道之本,貴在無為;聖人之跡,存乎拯弊。弊拯由跡,跡用有乖於本,本既無為,為非道之至。然不垂其跡,則世無以平;不究其本,則道實交喪。丘、旦將存其跡,故宜權晦其本;老、莊但明其本,亦宜深抑其跡。跡既可抑,數子所以有餘;本方見晦,尼丘是故不足。非得一之士,闕彼明智;體二之徒,獨懷鑒識。然聖已極照,反創其跡;賢未居宗,更言其本。良由跡須拯世,非聖不能;本實明理,在賢可照。若能體茲本跡,悟彼抑揚,則孔、莊之意,其過半矣。」



 南平元襄王聞其名,致書
 要之,不赴。孝緒曰:「非志驕富貴,但性畏廟堂。若使籞軿可驂,何以異夫驥騄。」



 初,建武末,青溪宮東門無故自崩,大風拔東宮門外楊樹。或以問孝緒,孝緒曰:「青溪皇家舊宅。齊為木行,東者木位,今東門自壞,木其衰矣。」



 鄱陽忠烈王妃,孝緒之姊。王嘗命駕,欲就之遊,孝緒鑿垣而逃,卒不肯見。諸甥歲時饋遺,一無所納。人或怪之,答云:「非我始願,故不受也。」



 其恒所供養石像,先有損壞,心欲治補,經一夜忽然完復,眾並異之。大同二年,卒,時年五十八。門徒誄其德行,謚曰文貞處士。所著《七錄》等書二百五十卷,行於世。



 陶弘景,字通明,丹陽秣陵人也。初,母夢青龍自懷而出,并見兩天人手執香爐來至其所,已而有娠,遂產弘景。幼有異操。年十歲,得葛洪《神仙傳》,晝夜研尋,便有養生之志。謂人曰:「仰青雲,睹白日,不覺為遠矣。」及長,身長七尺四寸,神儀明秀,朗目疏眉,細形長耳。讀書萬餘卷。善琴棋,工草隸。未弱冠,齊高帝作相,引為諸王侍讀,除奉朝請。雖在朱門,閉影不交外物,唯以披閱為務。朝儀故事,多取決焉。



 永明十年,上表辭祿,詔許之,賜以束帛。及發,公卿祖之於征虜亭,供帳甚盛,車馬填咽,咸云宋、齊以來,未有斯事。朝野榮之。於是止于句容之句曲山。恒
 曰:「此山下是第八洞宮,名金壇華陽之天,周回一百五十里。昔漢有咸陽三茅君得道,來掌此山,故謂之茅山。」乃中山立館,自號華陽隱居。始從東陽孫遊岳受符圖經法。遍歷名山,尋訪仙藥。每經澗谷,必坐臥其間,吟詠盤桓,不能已已。時沈約為東陽郡守,高其志節,累書要之,不至。



 弘景為人,圓通謙謹,出處冥會,心如明鏡,遇物便了,言無煩舛,有亦輒覺。建武中,齊宜都王鏗為明帝所害,其夜,弘景夢鏗告別,因訪其幽冥中事,多說秘異,因著《夢記》焉。



 永元初,更築三層樓,弘景處其上,弟子居其中,賓客至其下,與物遂絕,唯一家僮得侍其旁。特愛
 松風,每聞其響,欣然為樂。有時獨遊泉石,望見者以為仙人。性好著述,尚奇異,顧惜光景,老而彌篤。尤明陰陽五行,風角星算,山川地理,方圖產物,醫術本草。著《帝代年歷》,又嘗造渾天象,云「修道所須,非止史官是用」。義師平建康,聞議禪代,弘景援引圖讖,數處皆成「梁」字,令弟子進之。高祖既早與之遊,及即位後,恩禮逾篤,書問不絕,冠蓋相望。



 天監四年,移居積金東澗。善辟穀導引之法,年逾八十而有壯容。深慕張良之為人,云「古賢莫比」。曾夢佛授其菩提記,名為勝力菩薩。乃詣鄮縣阿育王塔自誓,受五大戒。後太宗臨南徐州,欽其風素,召至後
 堂,與談論數日而去,太宗甚敬異之。大通初,令獻二刀於高祖,其一名養勝,一名成勝,並為佳寶。大同二年,卒,時年八十五。顏色不變,屈申如恒。詔贈中散大夫,謚曰貞白先生,仍遣舍人監護喪事。弘景遺令薄葬,弟子遵而行之。



 諸葛璩,字幼玟,瑯邪陽都人,世居京口。璩幼事徵士關康之,博涉經史。復師征士臧榮緒。榮緒著《晉書》,稱璩有發擿之功,方之壺遂。



 齊建武初,南徐州行事江祀薦璩於明帝曰:「璩安貧守道,悅《禮》敦《詩》,未嘗投刺邦宰,曳裾府寺,如其簡退,可以揚清厲俗。請辟為議曹從事。」帝許
 之,璩辭不去。陳郡謝朓為東海太守,教曰:「昔長孫東組,降龍丘之節;文舉北輜,高通德之稱。所以激貪立懦,式揚風範。處士諸葛璩,高風所漸,結轍前修。豈懷珠披褐,韜玉待價?將幽貞獨往,不事王侯者邪?聞事親有啜菽之窶,就養寡藜蒸之給,豈得獨享萬鐘,而忘茲五秉?可餉穀百斛。」天監中,太守蕭琛、刺史安成王秀、鄱陽王恢並禮異焉。璩丁母憂毀瘠,恢累加存問。服闋,舉秀才,不就。



 璩性勤於誨誘,後生就學者日至,居宅狹陋,無以容之,太守張友為起講舍。璩處身清正,妻子不見喜慍之色。旦夕孜孜,講誦不輟,時人益以此宗之。七年,高祖敕
 問太守王份,份即具以實對,未及徵用,是年卒於家。璩所著文章二十卷,門人劉曒集而錄之。



 沈顗,字處默,吳興武康人也。父坦之,齊都官郎。



 顗幼清靜有至行,慕黃叔度、徐孺子之為人。讀書不為章句,著述不尚浮華。常獨處一室,人罕見其面。顗從叔勃,貴顯齊世,每還吳興,賓客填咽,顗不至其門。勃就之,顗送迎不越於閫。勃歎息曰:「吾乃今知貴不如賤。」



 俄徵為南郡王左常侍,不就。顗內行甚修,事母兄弟孝友,為鄉里所稱慕。永明三年,徵著作郎;建武二年,徵太子舍人,俱不赴。永元二年,又徵通直郎,亦不赴。顗素不治家產,值齊
 末兵荒,與家人並日而食。或有饋其梁肉者,閉門不受。唯以樵採自資,怡怡然恒不改其樂。天監四年,大舉北伐,訂民丁。吳興太守柳惲以顗從役,揚州別駕陸任以書責之,惲大慚,厚禮而遣之。其年卒於家。所著文章數十篇。



 劉慧斐,字文宣,彭城人也。少博學,能屬文,起家安成王法曹行參軍。嘗還都,途經尋陽,遊於匡山,過處士張孝秀,相得甚歡,遂有終焉之志。因不仕,居於東林寺。又於山北構園一所,號曰離垢園,時人乃謂為離垢先生。



 慧斐尤明釋典,工篆隸,在山手寫佛經二千餘卷,常所誦
 者百餘卷。晝夜行道,孜孜不怠,遠近欽慕之。太宗臨江州,遺以几杖。論者云:自遠法師沒後,將二百年,始有張、劉之盛矣。世祖及武陵王等書問不絕。大同二年,卒,時年五十九。



 范元琰,字伯珪,吳郡錢唐人也。祖悅之,太學博士徵,不至。父靈瑜,居父憂,以毀卒。元琰時童孺,哀慕盡禮,親黨異之。及長好學,博通經史,兼精佛義。然性謙敬,不以所長驕人。家貧,唯以園蔬為業。嘗出行,見人盜其菜,元琰遽退走,母問其故,具以實答。母問盜者為誰,答曰:「向所以退,畏其愧恥。今啟其名,願不泄也。」於是母子秘之。或
 有涉溝盜其筍者,元琰因伐木為橋以渡之。自是盜者大慚,一鄉無復草竊。居常不出城市,獨坐如對嚴賓,見之者莫不改容正色。沛國劉獻深加器異,嘗表稱之。齊建武二年,始徵為安北參軍事,不赴。天監九年,縣令管慧辨上言義行,揚州刺史、臨川王宏辟命,不至。十年,王拜表薦焉,竟未征。其年卒于家,時年七十。



 劉訏,字彥度,平原人也。父靈真,齊武昌太守。訏幼稱純孝,數歲,父母繼卒,訏居喪,哭泣孺慕,幾至滅性,赴弔者莫不傷焉。後為伯父所養,事伯母及昆姊,孝友篤至,為宗族所稱。自傷早孤,人有誤觸其諱者,未嘗不感結流
 涕。長兄潔為之娉妻,剋日成婚,訏聞而逃匿,事息乃還。本州刺史張稷辟為主簿,不就。主者檄召,訏乃挂檄於樹而逃。



 訏善玄言,尤精釋典。曾與族兄劉聽講於鐘山諸寺,因共卜築宋熙寺東澗,有終焉之志。天監十七年,卒於舍,時年三十一。臨終,執手曰:「氣絕便斂,斂畢即埋,靈筵一不須立,勿設饗祀,無求繼嗣。」從而行之。宗人至友相與刊石立銘,謚曰玄貞處士。



 劉,字士光,訏族兄也。祖乘民,宋冀州刺史;父聞慰,齊正員郎。世為二千石,皆有清名。幼有識慧,四歲喪父,與群兒同處,獨不戲弄。六歲誦《論語》、《毛詩》,意所不解,便
 能問難。十一,讀《莊子·逍遙篇》,曰:「此可解耳。」客因問之,隨問而答,皆有情理,家人每異之。及長,博學有文才,不娶不仕,與族弟訏並隱居求志,遨遊林澤,以山水書籍相娛而已。常欲避人世,以母老不忍違離,每隨兄霽、杳從宦。少時好施,務周人之急,人或遺之,亦不距也。久而歎曰:「受人者必報,不則有愧於人。吾固無以報人,豈可常有愧乎?」



 天監十七年,無何而著《革終論》。其辭曰:死生之事,聖人罕言之矣。孔子曰:「精氣為物,遊魂為變,知鬼神之情狀,與天地相似而不違。」其言約,其旨妙,其事隱,其意深,未可以臆斷,難得而精核,聊肆狂瞽,請試言之。



 夫
 形慮合而為生,魂質離而稱死;合則起動,離則休寂。當其動也,人皆知其神;及其寂也,物莫測其所趣。皆知則不言而義顯,莫測則逾辯而理微。是以勛、華曠而莫陳,姬、孔抑而不說,前達往賢,互生異見。季札云:「骨肉歸於土,魂氣無不之。」莊周云:「生為徭役,死為休息。」尋此二說,如或相反。何者?氣無不之,神有也;死為休息,神無也。原憲云:「夏后氏用明器示民無知也;殷人用祭器,示人有知也;周人兼用之,示民疑也。」考之記籍,驗之前志,有無之辯,不可歷言。若稽諸內教,判乎釋部,則諸子之言可尋,三代之禮無越。何者?神為生本,形為生具。死者神離
 此具,而即非彼具也。雖死者不可復反,而精靈遞變,未嘗滅絕。當其離此之日,識用廓然,故夏后明器,示其弗反。即彼之時,魂靈知滅,故殷人祭器,顯其猶存。不存則合乎莊周,猶存則同乎季札,各得一隅,無傷厥義。設其實也,則亦無,故周人有兼用之禮,尼父發遊魂之唱,不其然乎?若廢偏攜之論,探中途之旨,則不仁不智之譏,於是乎可息。



 夫形也者,無知之質也;神也者,有知之性也。有知不獨存,依無知以自立,故形之於神,逆旅之館耳。及其死也,神去此而適彼也。神已去此,館何用存?速朽得理也。神已適彼,祭何所祭?祭則失理。而姬、孔之教
 不然者,其有以乎!蓋禮樂之興,出於澆薄,俎豆綴兆,生於俗弊。施靈筵,陳棺槨,設饋奠,建丘隴,蓋欲令孝子有追思之地耳,夫何補於已遷之神乎?故上古衣之以薪,棄之中野,可謂尊盧、赫胥、皇雄、炎帝蹈於失理哉?是以子羽沉川,漢伯方壙,文楚黃壤,士安麻索。此四子者,得理也,忘教也。若從四子而遊,則平生之志得矣。



 然積習生常,難卒改革,一朝肆志,儻不見從。今欲剪截煩厚,務存儉易;進不裸尸,退異常俗;不傷存者之念,有合至人之道。孔子云:「斂首足形,還葬而無槨。」斯亦貧者之禮也,餘何陋焉?且張奐止用幅巾,王肅唯盥手足,范冉殮畢
 便葬,奚珍無設筵几,文度故舟為槨,子廉牛車載柩,叔起誡絕墳隴,康成使無卜吉。此數公者,尚或如之;況於吾人,而當華泰!今欲仿佛景行,以為軌則,儻合中庸之道,庶免徒費之譏。氣絕不須復魂,盥洗而斂。以一千錢市治棺、單故裙衫、衣巾枕履。此外送往之具,棺中常物,及餘閣之祭,一不得有所施。世多信李、彭之言,可謂惑矣。餘以孔、釋為師,差無此惑。斂訖,載以露車,歸於舊山,隨得一地,地足為坎,坎足容棺,不須磚甓,不勞封樹,勿設祭饗,勿置几筵,無用茅君之虛座,伯夷之杅水。其蒸嘗繼嗣,言象所絕,事止餘身,無傷世教。家人長幼,內外
 姻戚,凡厥友朋,爰及寓所,咸願成餘之志,幸勿奪之。



 明年疾卒,時年三十二。



 幼時嘗獨坐空室,有一老公至門,謂曰:「心力勇猛,能精死生;但不得久滯一方耳。」因彈指而去。既長,精心學佛。有道人釋寶誌者,時人莫測也,遇於興皇寺,驚起曰:「隱居學道,清凈登佛。」如此三說。未死之春,有人為其庭中栽柿,謂兄子弇曰:「吾不見此實,爾其勿言。」至秋而亡,人以為知命。親故誄其行跡,謚曰貞節處士。



 庾詵,字彥寶,新野人也。幼聰警篤學,經史百家無不該綜,緯候書射,棋釐機巧,並一時之絕。而性託夷簡,特愛
 林泉。十畝之宅,山池居半。蔬食弊衣,不治產業。嘗乘舟從田舍還,載米一百五十石,有人寄載三十石。既至宅,寄載者曰:「君三十斛,我百五十石。」詵默然不言,恣其取足。鄰人有被誣為盜者,被治劾,妄款,詵矜之,乃以書質錢二萬,令門生詐為其親,代之酬備。鄰人獲免,謝詵,詵曰:「吾矜天下無辜,豈期謝也。」其行多如此類。



 高祖少與詵善,雅推重之。及起義,署為平西府記室參軍,詵不屈。平生少所遊狎,河東柳惲欲與之交,詵距而不納。後湘東王臨荊州,板為鎮西府記室參軍,不就。普通中,詔曰:「明揚振滯,為政所先;旌賢求士,夢佇斯急。新野庾詵,止
 足棲退,自事卻掃,經史文藝,多所貫習;潁川庾承先,學通黃、老,該涉釋教;並不競不營,安茲枯槁,可以鎮躁敦俗。詵可黃門侍郎,承先可中書侍郎。勒州縣時加敦遣,庶能屈志,方冀鹽梅。」詵稱疾不赴。



 晚年以後,尤遵釋教。宅內立道場,環繞禮懺,六時不輟。誦《法華經》,每日一遍。後夜中忽見一道人,自稱願公,容止甚異,呼詵為上行先生,授香而去。中大通四年,因晝寢,忽驚覺曰:「願公復來,不可久住。」顏色不變,言終而卒,時年七十八。舉室咸聞空中唱「上行先生已生彌凈域矣」。高祖聞而下詔曰:「旌善表行,前王所敦。新野庾詵,荊山珠玉,江陵杞梓,
 靜侯南度,固有名德,獨貞苦節,孤芳素履。奄隨運往,惻愴于懷。宜謚貞節處士,以顯高烈。」詵所撰《帝歷》二十卷、《易林》二十卷、續伍端休《江陵記》一卷、《晉朝雜事》五卷、《總抄》八十卷,行於世。



 子曼倩,字世華,亦早有令譽。世祖在荊州,辟為主簿,遷中錄事。每出,世祖常目送之,謂劉之遴曰:「荊南信多君子,雖美歸田鳳,清屬桓階,賞德標奇,未過此子。」後轉諮議參軍。所著《喪服儀》、《文字體例》、《莊老義疏》,注《算經》及《七曜歷術》,並所製文章,凡九十五卷。



 子季才,有學行。承聖中,仕至中書侍郎。江陵陷,隨例入關。



 張孝秀,字文逸,南陽宛人也。少仕州為治中從事史。遭
 母憂,服闋,為建安王別駕。頃之,遂去職歸山,居于東林寺。有田數十頃,部曲數百人,率以力田,盡供山眾,遠近歸慕,赴之如市。孝秀性通率,不好浮華,常冠穀皮巾,躡蒲履,手執並櫚皮麈尾。服寒食散,盛冬能臥於石。博涉群書,專精釋典。善談論,工隸書,凡諸藝能,莫不明習。普通三年,卒,時年四十二,室中皆聞有非常香氣。太宗聞,甚傷悼焉,與劉慧斐書,述其貞白云。



 庾承先,字子通,潁川焉陵人也。少沉靜有志操,是非不涉於言,喜慍不形於色,人莫能窺也。弱歲受學於南陽劉虯,彊記敏識,出於群輩。玄經釋典,靡不該悉;九流《七
 略》,咸所精練。郡辟功曹不就,乃與道士王僧鎮同遊衡嶽。晚以弟疾還鄉里,遂居于土臺山。鄱陽忠烈王在州,欽其風味,要與遊處。又令講《老子》,遠近名僧,咸來赴集,論難鋒起,異端競至,承先徐相酬答,皆得所未聞。忠烈王尤加欽重,徵州主簿;湘東王聞之,亦板為法曹參軍;並不赴。



 中大通三年,廬山劉慧斐至荊州,承先與之有舊,往從之。荊陜學徒,因請承先講《老子》。湘東王親命駕臨聽,論議終日,深相賞接。留連月餘日,乃還山。王親祖道,並贈篇什,隱者美之。其年卒,時年六十。



 陳吏部尚書姚察曰:世之誣處士者,多云純盜虛名而
 無適用,蓋有負其實者。若諸葛璩之學術,阮孝緒之簿閥,其取進也豈難哉?終於隱居,固亦性而已矣。



\end{pinyinscope}