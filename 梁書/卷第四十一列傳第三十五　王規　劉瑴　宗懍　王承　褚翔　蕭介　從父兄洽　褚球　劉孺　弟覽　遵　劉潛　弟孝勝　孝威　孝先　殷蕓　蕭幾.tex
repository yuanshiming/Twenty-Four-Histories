\article{卷第四十一列傳第三十五 王規 劉瑴 宗懍 王承 褚翔 蕭介 從父兄洽 褚球 劉孺 弟覽 遵 劉潛 弟孝勝 孝威 孝先 殷蕓 蕭幾}

\begin{pinyinscope}

 王
 規,字威明,瑯邪臨沂人。祖儉,齊太尉南昌文憲公。父騫,金紫光祿大夫南昌安侯。規八歲,以丁所生母憂,居喪有至性。太尉徐孝嗣每見必為之流涕,稱曰孝童。叔
 父暕亦深器重之,常曰:「此兒吾家千里駒也。」年十二,《五經》大義,並略能通。既長,好學有口辯。州舉秀才,郡迎主簿。



 起家秘書郎,累遷太子舍人、安右南康王主簿、太子洗馬。天監十二年,改構太極殿,功畢,規獻《新殿賦》,其辭甚工。拜秘書丞。歷太子中舍人、司徒左西屬、從事中郎。晉安王綱出為南徐州,高選僚屬,引為雲麾諮議參軍。久之,出為新安太守,父憂去職。服闋,襲封南昌縣侯,除中書黃門侍郎。敕與陳郡殷鈞、瑯邪王錫、范陽張緬同侍東宮,俱為昭明太子所禮。湘東王時為京尹,與朝士宴集,屬規為酒令。規從容對曰:「自江左以來,未有茲舉。」
 特進蕭琛、金紫傅昭在坐,並謂為知言。普通初,陳慶之北伐,剋復洛陽,百僚稱賀,規退曰:「道家有云:非為功難,成功難也。羯寇遊魂,為日已久,桓溫得而復失,宋武竟無成功。我孤軍無援,深入寇境,威勢不接,餽運難繼,將是役也,為禍階矣。」俄而王師覆沒,其識達事機多如此類。



 六年,高祖於文德殿餞廣州刺史元景隆,詔群臣賦詩,同用五十韻,規援筆立奏,其文又美。高祖嘉焉,即日詔為侍中。大通三年,遷五兵尚書,俄領步兵校尉。中大通二年,出為貞威將軍驃騎晉安王長史。其年,王立為皇太子,仍為吳郡太守。主書芮珍宗家在吳,前守宰皆
 傾意附之。是時珍宗假還,規遇之甚薄,珍宗還都,密奏規云「不理郡事」。俄徵為左民尚書,郡吏民千餘人詣闕請留,表三奏,上不許。尋以本官領右軍將軍,未拜,復為散騎常侍、太子中庶子,領步兵校尉。規辭疾不拜,於鐘山宗熙寺築室居焉。大同二年,卒,時年四十五。詔贈散騎常侍、光祿大夫,賻錢二十萬,布百匹。謚曰章。皇太子出臨哭,與湘東王繹令曰:「威明昨宵奄復殂化,甚可痛傷。其風韻遒正,神峰標映,千里絕迹,百尺無枝。文辯縱橫,才學優贍,跌宕之情彌遠,濠梁之氣特多,斯實俊民也。一爾過隙,永歸長夜,金刀掩芒,長淮絕涸。去歲冬中,
 已傷劉子;今茲寒孟,復悼王生。俱往之傷,信非虛說。」規集《後漢》眾家異同,注《續漢書》二百卷,文集二十卷。



 子褒,字子漢,七歲能屬文。外祖司空袁昂愛之,謂賓客曰:「此兒當成吾宅相。」弱冠舉秀才,除秘書郎、太子舍人,以父憂去職。服闋,襲封南昌侯,除武昌王文學、太子洗馬,兼東宮管記,遷司徒屬,秘書丞,出為安成內史。太清中,侯景陷京城,江州刺史當陽公大心舉州附賊,賊轉寇南中,褒猶據郡拒守。大寶二年,世祖命徵褒赴江陵,既至,以為忠武將軍、南平內史,俄遷吏部尚書、侍中。承聖二年,遷尚書右僕射,仍參掌選事,又加侍中。其年,遷左僕
 射,參掌如故。三年,江陵陷,入于周。



 褒著《幼訓》,以誡諸子。其一章云:陶士衡曰:「昔大禹不吝尺璧而重寸陰。」文士何不誦書,武士何不馬射?若乃玄冬脩夜,朱明永日,肅其居處,崇其牆仞,門無糅雜,坐闕號呶。以之求學,則仲尼之門人也;以之為文,則賈生之升堂也。古者盤盂有銘,几杖有誡,進退循焉,俯仰觀焉。文王之詩曰:「靡不有初,鮮克有終。」立身行道,終始若一。「造次必於是」,君子之言歟?



 儒家則尊卑等差,吉凶降殺。君南面而臣北面,天地之義也;鼎俎奇而籩豆偶,陰陽之義也。道家則墮支體,黜聰明,棄義絕仁,離形去智。釋氏之義,見苦斷習,證
 滅循道,明因辨果,偶凡成聖,斯雖為教等差,而義歸汲引。吾始乎幼學,及於知命,既崇周、孔之教,兼循老、釋之談,江左以來,斯業不墜,汝能脩之,吾之志也。



 初,有沛國劉瑴、南陽宗懍與褒俱為中興佐命,同參帷幄。



 劉瑴,字仲寶,晉丹陽尹真長七世孫也。少方正有器局。自國子禮生射策高第,為寧海令,稍遷湘東王記室參軍,又轉中記室。太清中,侯景亂,世祖承制上流,書檄多委瑴焉,瑴亦竭力盡忠,甚蒙賞遇。歷尚書左丞、御史中丞。承聖二年,遷吏部尚書、國子祭酒,餘如故。



 宗懍,字元懍。八世祖承,晉宜都郡守,屬永嘉東徙,子孫
 因居江陵焉。懍少聰敏好學,晝夜不倦,鄉里號為「童子學士」。普通中,為湘東王府兼記室,轉刑獄,仍掌書記。歷臨汝、建成、廣晉等令,後又為世祖荊州別駕。及世祖即位,以為尚書郎,封信安縣侯,邑一千戶。累遷吏部郎中、五兵尚書、吏部尚書。承聖三年,江陵沒,與瑴俱入于周。



 王承,字安期,僕射暕子。七歲通《周易》,選補國子生。年十五,射策高第,除祕書郎。歷太子舍人、南康王文學、邵陵王友、太子中舍人。以父憂去職。服闋,復為中舍人,累遷中書黃門侍郎,兼國子博士。時膏腴貴遊,咸以文學相尚,罕以經術為業,惟承獨好之,發言吐論,造次儒者。在
 學訓諸生,述《禮》、《易》義。中大通五年,遷長兼侍中,俄轉國子祭酒。承祖儉及父暕嘗為此職,三世為國師,前代未之有也,當世以為榮。久之,出為戎昭將軍、東陽太守。為政寬惠,吏民悅之。視事未期,卒於郡,時年四十一。謚曰章子。



 承性簡貴有風格。時右衛朱異當朝用事,每休下,車馬常填門。時有魏郡申英好危言高論,以忤權右,常指異門曰:「此中輻輳,皆以利往。能不至者,惟有大小王東陽。」小東陽,即承弟稚也。當時惟承兄弟及褚翔不至異門,時以此稱之。



 褚翔,字世舉,河南陽翟人。曾祖淵,齊太宰文簡公,佐命
 齊室。祖蓁,太常穆子。父向,字景政。年數歲,父母相繼亡沒,向哀毀若成人者,親表咸異之。既長,淹雅有器量。高祖踐阼,選補國子生。起家秘書郎,遷太子舍人、尚書殿中郎。出為安成內史。還除太子洗馬、中舍人,累遷太尉從事中郎、黃門侍郎、鎮右豫章王長史。頃之,入為長兼侍中。向風儀端麗,眉目如點,每公庭就列,為眾所瞻望焉。大通四年,出為寧遠將軍北中郎廬陵王長史。三年,卒官。外兄謝舉為製墓銘,其略曰:「弘治推華,子嵩慚量;酒歸月下,風清琴上。」論者以為擬得其人。



 翔初為國子生,舉高第。丁父憂。服闋,除秘書郎,累遷太子舍人、宣城
 王主簿。中大通五年,高祖宴群臣樂遊苑,別詔翔與王訓為二十韻詩,限三刻成。翔於坐立奏,高祖異焉,即日轉宣城王文學,俄遷為友。時宣城友、文學加它王二等,故以翔超為之,時論美焉。出為義興太守。翔在政潔已,省繁苛,去浮費,百姓安之。郡之西亭有古樹,積年枯死;翔至郡,忽更生枝葉,百姓咸以為善政所感。及秩滿,吏民詣闕請之,敕許焉。尋徵為吏部郎,去郡,百姓無老少追送出境,涕泣拜辭。



 翔居小選公清,不為請屬易意,號為平允。俄遷侍中,頃之轉散騎常侍,領羽林監,侍東宮。出為晉陵太守,在郡未期,以公事免。俄復為散騎常侍,
 侍東宮。太清二年,遷守吏部尚書。其年冬,侯景圍宮城,翔於圍內丁母憂,以毀卒,時年四十四。詔贈本官。翔少有孝性。為侍中時,母疾篤,請沙門祈福。中夜忽見戶外有異光,又聞空中彈指,及曉,疾遂愈。咸以翔精誠所致焉。



 蕭介,字茂鏡,蘭陵人也。祖思話,宋開府儀同三司、尚書僕射。父惠茜,齊左民尚書。介少穎悟,有器識,博涉經史,兼善屬文。齊永元末,釋褐著作佐郎。天監六年,除太子舍人。八年,遷尚書金部郎。十二年,轉主客郎。出為吳令,甚著聲績。湘東王聞介名,思共遊處,表請之。普通三年,
 乃以介為湘東王諮議參軍。大通二年,除給事黃門侍郎。大同二年,武陵王為揚州刺史,以介為府長史,在職清白,為朝廷所稱。高祖謂何敬容曰:「蕭介甚貧,可處以一郡。」敬容未對,高祖曰:「始興郡頃無良守,嶺上民頗不安,可以介為之。」由是出為始興太守。介至任,宣布威德,境內肅清。七年,徵為少府卿,尋加散騎常侍。會侍中闕,選司舉王筠等四人,並不稱旨,高祖曰:「我門中久無此職,宜用蕭介為之。」介博物強識,應對左右,多所匡正,高祖甚重之。遷都官尚書,每軍國大事,必先詢訪於介焉。高祖謂朱異曰:「端右之材也。」中大同二年,辭疾致事,高
 祖優詔不許。終不肯起,乃遣謁者僕射魏祥就拜光祿大夫。



 太清中,侯景於渦陽敗走,入壽陽。高祖敕防主韋默納之,介聞而上表諫曰:臣抱患私門,竊聞侯景以渦陽敗績,隻馬歸命,陛下不悔前禍,復敕容納。臣聞凶人之性不移,天下之惡一也。昔呂布殺丁原以事董卓,終誅董而為賊;劉牢反王恭以歸晉,還背晉以構妖。何者?狼子野心,終無馴狎之性;養虎之喻,必見飢噬之禍。侯景獸心之種,鳴鏑之類。以凶狡之才,荷高歡翼長之遇,位忝台司,任居方伯;然而高歡墳土未幹,即還反噬。逆力不逮,乃復逃死關西;宇文不容,故復投身於我。陛下
 前者所以不逆細流,正欲以屬國降胡以討匈奴,冀獲一戰之效耳。今既亡師失地,直是境上之匹夫。陛下愛匹夫而棄與國之好,臣竊不取也。若國家猶待其更鳴之晨,歲暮之效,臣竊惟侯景必非歲暮之臣。棄鄉國如脫屣,背君親如遺芥,豈知遠慕聖德,為江淮之純臣!事跡顯然,無可致惑。一隅尚其如此,觸類何可具陳?



 臣朽老疾侵,不應輒干朝政。但楚囊將死,有城郢之忠;衛魚臨亡,亦有屍諫之節。臣忝為宗室遺老,敢忘劉向之心?伏願天慈,少思危苦之語。



 高祖省表歎息,卒不能用。



 介性高簡,少交遊,惟與族兄琛、從兄眎素及洽、從弟淑等
 文酒賞會,時人以比謝氏烏衣之遊。初,高祖招延後進二十餘人,置酒賦詩。臧盾以詩不成,罰酒一斗,盾飲盡,顏色不變,言笑自若;介染翰便成,文無加點。高祖兩美之曰:「臧盾之飲,蕭介之文,即席之美也。」年七十三,卒於家。



 第三子允,初以兼散騎常侍聘魏,還為太子中庶子,後至光祿大夫。



 洽,字宏稱,介從父兄也。父惠基,齊吏部尚書,有重名前世。洽幼敏寤,年七歲,誦《楚辭》略上口。及長,好學博涉,亦善屬文。齊永明中,為國子生,舉明經。起家著作佐郎,遷西中郎外兵參軍。天監初,為前軍鄱陽王主簿、尚書囗
 部郎,遷太子中舍人。出為南徐州治中,既近畿重鎮,史數千人,前後居之者皆致巨富。洽為之,清身率職,饋遺一無所受,妻子不免饑寒。還除司空從事中郎,為建安內史,坐事免。久之,起為護軍長史、北中郎諮議參軍,遷太府卿、司徒臨川王司馬。普通初,拜員外散騎常侍,兼御史中丞,以公事免。頃之,為通直散騎常侍。洽少有才思,高祖令製同泰、大愛敬二寺剎下銘,其文甚美。二年,遷散騎常侍。出為招遠將軍、臨海太守。為政清平,不尚威猛,民俗便之。還拜司徒左長史,又敕撰《當塗堰碑》,辭亦贍麗。六年,卒官,時年五十五。有詔出舉哀,賻錢二萬,
 布五十匹。集二十卷,行於世。



 褚球,字仲寶,河南陽翟人。高祖叔度,宋征虜將軍、雍州刺史;祖曖,太宰外兵參軍;父繢,太子舍人;並尚宋公主。球少孤貧,篤志好學,有才思。宋建平王景素,元徽中誅滅,惟有一女得存。其故吏何昌珝、王思遠聞球清立,以此女妻之,因為之延譽。仕齊,起家征虜行參軍,俄署法曹,遷右軍曲江公主簿。出為溧陽令,在縣清白,資公俸而已。除平西主簿。



 天監初,遷太子洗馬、散騎侍郎,兼中書通事舍人。出為建康令,母憂去職,以本官起之,固辭不拜。服闋,除北中郎諮議參軍,俄遷中書郎,復兼中書
 通事舍人。除雲騎將軍,累兼廷尉、光祿卿,舍人如故。遷御史中丞。球性公強,無所屈撓,在憲司甚稱職。普通四年,出為北中郎長史、南蘭陵太守;入為通直散騎常侍,領羽林監。七年,遷太府卿,頃之,遷都官尚書。中大同中,出為仁威臨川王長史、江夏太守,以疾不赴職。改授光祿大夫,未拜,復為太府卿,領步兵校尉。俄遷通直散騎常侍、秘書監,領著作。遷司徒左長史,常侍、著作如故。自魏孫禮、晉荀組以後,台佐加貂,始有球也。尋出為貞威將軍輕車河東王長史、南蘭陵太守;入為散騎常侍,領步兵。尋表致仕,詔不許。俄復拜光祿大夫,加給事中。卒
 官,時年七十。



 劉孺,字孝稚,彭城安上里人也。祖勔,宋司空忠昭公。父悛,齊太常敬子。孺幼聰敏,七歲能屬文。年十四,居父喪,毀瘠骨立,宗黨咸異之。服闋,叔父瑱為義興郡,攜以之官,常置坐側,謂賓客曰:「此兒吾家之明珠也。」既長,美風采,性通和,雖家人不見其喜慍。本州召迎主簿。起家中軍法曹行參軍。時鎮軍沈約聞其名,引為主簿,常與遊宴賦詩,大為約所嗟賞。累遷太子舍人、中軍臨川王主簿、太子洗馬、尚書殿中郎。出為太末令,在縣有清績。還除晉安王友,轉太子中舍人。



 孺少好文章,性又敏速,嘗
 於御坐為《李賦》,受詔便成,文不加點,高祖甚稱賞之。後侍宴壽光殿,詔群臣賦詩,時孺與張率並醉,未及成,高祖取孺手板題戲之曰:「張率東南美,劉孺雒陽才。攬筆便應就,何事久遲回?」其見親愛如此。



 轉中書郎,兼中書通事舍人。頃之遷太子家令,餘如故。出為宣惠晉安王長史,領丹陽尹丞。遷太子中庶子、尚書吏部郎。出為輕車湘東王長史,領會稽郡丞,公事免。頃之,起為王府記室散騎侍郎,兼光祿卿。累遷少府卿、司徒左長史、御史中丞,號為稱職。大通二年,遷散騎常侍。三年,遷左民尚書,領步兵校尉。中大通四年,出為仁威臨川王長史、江
 夏太守,加貞威將軍。五年,為寧遠將軍、司徒左長史,未拜,改為都官尚書,領右軍將軍。大同五年,守吏部尚書。其年,出為明威將軍、晉陵太守。在郡和理,為吏民所稱。七年,入為侍中,領右軍。其年,復為吏部尚書,以母憂去職。居喪未期,以毀卒,時年五十九。謚曰孝子。



 孺少與從兄苞、孝綽齊名。苞早卒,孝綽數坐免黜,位並不高,惟孺貴顯。有文集二十卷。子芻,著作郎,早卒。孺二弟:覽、遵。



 覽,字孝智,十六通《老》、《易》。歷官中書郎,以所生母憂,廬于墓。再期,口不嘗鹽酪,冬止著單布。家人患其不勝喪,中夜竊置炭於床下,覽因暖氣得睡,既覺知之,號慟歐血。
 高祖聞其有至性,數省視之。服闋,除尚書左丞。性聰敏,尚書令史七百人,一見並記名姓。當官清正,無所私。姊夫御史中丞褚湮、從兄吏部郎孝綽,在職頗通贓貨,覽劾奏,並免官。孝綽怨之,嘗謂人曰:「犬齧行路,覽噬家人。」出為始興內史,治郡尤勵清節。還復為左丞,卒官。



 遵,字孝陵。少清雅,有學行,工屬文。起家著作郎、太子舍人,累遷晉安王宣惠、雲麾二府記室,甚見賓禮,轉南徐州治中。王後為雍州,復引為安北諮議參軍,帶邔縣令。中大通二年,王立為皇太子,仍除中庶子。遵自隨籓及在東宮,以舊恩,偏蒙寵遇,同時莫及。大同元年,卒官。皇
 太子深悼惜之,與遵從兄陽羨令孝儀令曰:賢從中庶,奄至殞逝,痛可言乎!其孝友淳深,立身貞固;內含玉潤,外表瀾清。美譽嘉聲,流於士友;言行相符,終始如一。文史該富,琬琰為心;辭章博贍,玄黃成採。既以鳴謙表性,又以難進自居,未嘗造請公卿,締交榮利。是以新沓莫之舉,社武弗之知。自阮放之官,野王之職,棲遲門下,已踰五載;同僚已陟,後進多升,而怡然清靜,不以少多為念。確爾之志,亦何易得?西河觀寶,東江獨步,書籍所載,必不是過。



 吾昔在漢南,連翩書記,及忝朱方,從容坐首。良辰美景,清風月夜,鷁舟乍動,朱鷺徐鳴,未嘗一日而
 不追隨,一時而不會遇。酒蘭耳熱,言志賦詩,校覆忠賢,榷揚文史,益者三友,此實其人。及弘道下邑,未申善政,而能使民結去思,野多馴雉,此亦威鳳一羽,足以驗其五德。比在春坊,載獲申晤,博望無通賓之務,司成多節文之科。所賴故人,時相媲偶;而此子溘然,實可嗟痛。「惟與善人」,此為虛說;天之報施,豈若此乎!想卿痛悼之誠,亦當何已。往矣奈何,投筆惻愴。



 吾昨欲為誌銘,並為撰集。吾之劣薄,其生也不能揄揚吹歔,使得騁其才用,今者為銘為集,何益既往?故為痛惜之情,不能已已耳。



 劉潛,字孝儀,秘書監孝綽弟也。幼孤,與兄弟相勵勤學,
 並工屬文。孝綽常曰「三筆六詩」,三即孝儀,六孝威也。天監五年,舉秀才。起家鎮右始興王法曹行參軍,隨府益州,兼記室。王入為中撫軍,轉主簿,遷尚書殿中郎。敕令制《雍州平等金像碑》,文甚宏麗。晉安王綱出鎮襄陽,引為安北功曹史,以母憂去職。王立為皇太子,孝儀服闋,仍補洗馬,遷中舍人。出為戎昭將軍、陽羨令,甚有稱績,擢為建康令。大同三年,遷中書郎,以公事左遷安西諮議參軍,兼散騎常侍。使魏還,復除中書郎。頃之,權兼司徒右長史,又兼寧遠長史、行彭城瑯邪二郡事。累遷尚書左丞,兼御史中丞。在職彈糾無所顧望,當時稱之。十
 年,出為伏波將軍、臨海太守。是時政網疏闊,百姓多不遵禁。孝儀下車,宣示條制,勵精綏撫,境內翕然,風俗大革。中大同元年,入守都官尚書。太清元年,出為明威將軍、豫章內史。二年,侯景寇京邑,孝儀遣子勵帥郡兵三千人,隨前衡州刺史韋粲入援。三年,宮城不守,孝儀為前歷陽太守莊鐵所逼,失郡。大寶元年,病卒,時年六十七。



 孝儀為人寬厚,內行尤篤。第二兄孝能早卒,孝儀事寡嫂甚謹,家內巨細,必先諮決。與妻子朝夕供事,未嘗失禮。世以此稱之。有文集二十卷,行於世。



 第五弟孝勝,歷官邵陵王法曹、湘東王安西主簿記室、尚書左丞。出
 為信義太守,公事免。久之,復為尚書右丞,兼散騎常侍。聘魏還,為安西武陵王紀長史、蜀郡太守。太清中,侯景陷京師,紀僭號於蜀,以孝勝為尚書僕射。承聖中,隨紀出峽口,兵敗,被執下獄。世祖尋宥之,起為司徒右長史。



 第六弟孝威,初為安北晉安王法曹,轉主簿,以母憂去職。服闋,除太子洗馬,累遷中舍人、庶子、率更令,並掌管記。大同九年,白雀集東宮,孝威上頌,其辭甚美。太清中,遷中庶子,兼通事舍人。及侯景寇亂,孝威於圍城得出,隨司州刺史柳仲禮西上,至安陸,遇疾卒。



 第七弟孝先,武陵王法曹、主簿。王遷益州,隨府轉安西記室。承聖中,
 與兄孝勝俱隨紀軍出峽口,兵敗,至江陵,世祖以為黃門侍郎,遷侍中。兄弟並善五言詩,見重於世。文集值亂,今不具存。



 殷芸,字灌蔬,陳郡長平人。性倜儻,不拘細行。然不妄交遊,門無雜客。勵精勤學,博洽群書。幼而廬江何憲見之,深相歎賞。永明中,為宜都王行參軍。天監初,為西中郎主簿、後軍臨川王記室。七年,遷通直散騎侍郎,兼中書通事舍人。十年,除通直散騎侍郎,兼尚書左丞,又兼中書舍人,遷國子博士、昭明太子侍讀、西中郎豫章王長史,領丹陽尹丞,累遷通直散騎常侍、秘書監、司徒左長
 史。普通六年,直東宮學士省。大通三年卒,時年五十九。



 蕭幾,字德玄,齊曲江公遙欣子也。年十歲,能屬文。早孤,有弟九人,並皆稚小,幾恩愛篤睦,聞於朝野。性溫和,與物無競,清貧自立。好學,善草隸書。湘州刺史楊公則,曲江之故吏也。每見幾,謂人曰:「康公此子,可謂桓靈寶出。」及公則卒,幾為之誄,時年十五,沈約見而奇之,謂其舅蔡撙曰:「昨見賢甥楊平南誄文,不減希逸之作,始驗康公積善之慶。」釋褐著作佐郎、廬陵王文學、尚書殿中郎、太子舍人、掌管記,遷庶子、中書侍郎、尚書左丞。末年,專尚釋教。為新安太守,郡多山水,特其所好,適性遊履,遂
 為之記。卒於官。



 子為,字元專,亦有文才。仕至太子舍人,永康令。



 史臣曰:王規之徒,俱著名譽,既逢休運,才用各展,美矣。蕭洽《當塗》之制,見偉辭人;劉孝儀兄弟,並以文章顯。君子知梁代之有人焉。



\end{pinyinscope}