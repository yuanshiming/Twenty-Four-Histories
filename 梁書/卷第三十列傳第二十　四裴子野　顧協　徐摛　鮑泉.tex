\article{卷第三十列傳第二十 四裴子野 顧協 徐摛 鮑泉}

\begin{pinyinscope}

 裴子野,字幾原,河東聞喜人,晉太子左率康八世孫。兄黎,弟楷、綽,並有盛名,所謂「四裴」也。曾祖松之,宋太中大夫。祖駰,南中郎外兵參軍。父昭明,通直散騎常侍。子野生而偏孤,為祖母所養,年九歲,祖母亡,泣血哀慟,家人異之。少好學,善屬文。起家齊武陵王國左常侍,右軍江夏王參軍,遭父憂去職。居喪盡禮,每之墓所,哭泣處草
 為之枯,有白兔馴擾其側。天監初,尚書僕射范雲嘉其行,將表奏之,會雲卒,不果。樂安任昉有盛名,為後進所慕,遊其門者,昉必相薦達。子野於昉為從中表,獨不至,昉亦恨焉。久之,除右軍安成王參軍,俄遷兼廷尉正。時三官通署獄牒,子野嘗不在,同僚輒署其名,奏有不允,子野從坐免職。或勸言諸有司,可得無咎。子野笑而答曰:「雖慚柳季之道,豈因訟以受服。」自此免黜久之,終無恨意。



 二年,吳平侯蕭景為南兗州刺史,引為冠軍錄事,府遷職解。時中書范縝與子野未遇,聞其行業而善焉。會遷國子博士,乃上表讓之曰:「伏見前冠軍府錄事參
 軍河東裴子野,年四十,字幾原,幼稟至人之行,長厲國士之風。居喪有禮,毀瘠幾滅,免憂之外,蔬水不進。栖遲下位,身賤名微,而性不憛憛,情無汲汲,是以有識嗟推,州閭歎服。且家傳素業,世習儒史,苑囿經籍,遊息文藝。著《宋略》二十卷,彌綸首尾,勒成一代,屬辭比事,有足觀者。且章句洽悉,訓故可傳。脫置之膠庠,以弘獎後進,庶一夔之辯可尋,三豕之疑無謬矣。伏惟皇家淳耀,多士盈庭,官人邁乎有媯,棫樸越於姬氏,茍片善宜錄,無論厚薄,一介可求,不由等級。臣歷觀古今人君,欽賢好善,未有聖朝孜孜若是之至也。敢緣斯義,輕陳愚瞽,乞以
 臣斯忝,回授子野。如此,則賢否之宜,各全其所,訊之物議,誰曰不允。臣與子野雖未嘗銜杯,訪之邑里,差非虛謬,不勝慺慺微見,冒昧陳聞。伏願陛下哀憐悾款,鑒其愚實,干犯之愆,乞垂赦宥。」有司以資歷非次,弗為通。尋除尚書比部郎,仁威記室參軍。出為諸暨令,在縣不行鞭罰,民有爭者,示之以理,百姓稱悅,合境無訟。



 初,子野曾祖松之,宋元嘉中受詔續修何承天《宋史》,未及成而卒,子野常欲繼成先業。及齊永明末,沈約所撰《宋書》既行,子野更刪撰為《宋略》二十卷。其敘事評論多善,約見而歎曰:「吾弗逮也。」蘭陵蕭琛、北地傅昭、汝南周捨咸稱
 重之。至是,吏部尚書徐勉言之於高祖,以為著作郎,掌國史及起居注。頃之,兼中書通事舍人,尋除通直正員郎,著作、舍人如故。又敕掌中書詔誥。是時西北徼外有白題及滑國,遣使由岷山道入貢。此二國歷代弗賓,莫知所出。子野曰:「漢潁陰侯斬胡白題將一人。服虔《注》云:『白題,胡名也。』又漢定遠侯擊虜,八滑從之,此其後乎。」時人服其博識。敕仍使撰《方國使圖》,廣述懷來之盛,自要服至于海表,凡二十國。



 子野與沛國劉顯、南陽劉之遴、陳郡殷芸、陳留阮孝緒、吳郡顧協、京兆韋棱,皆博極群書,深相賞好,顯尤推重之。時吳平侯蕭勱、范陽張纘,每
 討論墳籍,咸折中於子野焉。普通七年,王師北伐,敕子野為喻魏文,受詔立成,高祖以其事體大,召尚書僕射徐勉、太子詹事周捨、鴻臚卿劉之遴、中書侍郎朱異,集壽光殿以觀之,時並歎服。高祖目子野而言曰:「其形雖弱,其文甚壯。」俄又敕為書喻魏相元叉,其夜受旨,子野謂可待旦方奏,未之為也。及五鼓,敕催令開齋速上,子野徐起操筆,昧爽便就。既奏,高祖深嘉焉。自是凡諸符檄,皆令草創。子野為文典而速,不尚麗靡之詞。其制作多法古,與今文體異,當時或有詆訶者,及其末皆翕然重之。或問其為文速者,子野答云:「人皆成於手,我獨成
 於心,雖有見否之異,其於刊改一也。」



 俄遷中書侍郎,餘如故。大通元年,轉鴻臚卿,尋領步兵校尉。子野在禁省十餘年,靜默自守,未嘗有所請謁,外家及中表貧乏,所得俸悉分給之。無宅,借官地二畝,起茅屋數間。妻子恆苦飢寒,唯以教誨為本,子姪祗畏,若奉嚴君。末年深信釋氏,持其教戒,終身飯麥食蔬。中大通二年,卒官,年六十二。



 先是子野自剋死期,不過庚戌歲。是年自省移病,謂同官劉之亨曰:「吾其逝矣。」遺命儉約,務在節制。高祖悼惜,為之流涕。詔曰:「鴻臚卿、領步兵校尉、知著作郎、兼中書通事舍人裴子野,文史足用,廉白自居,劬勞通事,
 多歷年所。奄致喪逝,惻愴空懷。可贈散騎常侍,賻錢五萬,布五十匹,即日舉哀。謚曰貞子。」



 子野少時,《集注喪服》、《續裴氏家傳》各二卷,抄合後漢事四十餘卷,又敕撰《眾僧傳》二十卷,《百官九品》二卷,《附益謚法》一卷,《方國使圖》一卷,文集二十卷,並行於世。又欲撰《齊梁春秋》,始草創,未就而卒。子謇,官至通直郎。



 顧協,字正禮,吳郡吳人也。晉司空和七世孫。協幼孤,隨母養於外氏。外從祖宋右光祿張永嘗攜內外孫姪遊虎丘山,協年數歲,永撫之曰:「兒欲何戲?」協對曰:「兒正欲枕石漱流。」永歎息曰:「顧氏興於此子。」既長,好學,以精力
 稱。外氏諸張多賢達有識鑒,從內弟率尤推重焉。



 起家揚州議曹從事史,兼太學博士。舉秀才,尚書令沈約覽其策而歎曰:「江左以來,未有此作。」遷安成王國左常侍,兼廷尉正。太尉臨川王聞其名,召掌書記,仍侍西豊侯正德讀。正德為巴西、梓潼郡,協除所部安都令。未至縣,遭母憂。服闋,出補西陽郡丞。還除北中郎行參軍,復兼廷尉正。久之,出為廬陵郡丞,未拜。會西豊侯正德為吳郡,除中軍參軍,領郡五官,遷輕車湘東王參軍事,兼記室。普通六年,正德受詔北討,引為府錄事參軍,掌書記。



 軍還,會有詔舉士,湘東王表薦協曰:「臣聞貢玉之士,歸
 之潤山;論珠之人,出於枯岸。是以芻蕘之言,擇於廊廟者也。臣府兼記室參軍吳郡顧協,行稱鄉閭,學兼文武,服膺道素,雅量邃遠,安貧守靜,奉公抗直,傍闕知己,志不自營,年方六十,室無妻子。臣欲言於官人,申其屈滯,協必苦執貞退,立志難奪,可謂東南之遺寶矣。伏惟陛下未明求衣,思賢如渴,爰發明詔,各舉所知。臣識非許、郭,雖無知人之鑒,若守固無言,懼貽蔽賢之咎。昔孔愉表韓績之才,庾亮薦翟湯之德,臣雖未齒二臣,協實無慚兩士。」即召拜通直散騎侍郎,兼中書通事舍人。累遷步兵校尉,守鴻臚卿,員外散騎常侍,卿、舍人並如故。大
 同八年,卒,時年七十三。高祖悼惜之,手詔曰:「員外散騎常侍、鴻臚卿、兼中書通事舍人顧協,廉潔自居,白首不衰,久在省闥,內外稱善。奄然殞喪,惻怛之懷,不能已已。傍無近親,彌足哀者。大殮既畢,即送其喪柩還鄉,并營冢槨,並皆資給,悉使周辦。可贈散騎常侍,令便舉哀。謚曰溫子。」



 協少清介有志操。初為廷尉正,冬服單薄,寺卿蔡法度謂人曰:「我願解身上襦與顧郎,恐顧郎難衣食者。」竟不敢以遺之。及為舍人,同官者皆潤屋,協在省十六載,器服飲食,不改於常。有門生始來事協,知其廉潔,不敢厚餉,止送錢二千,協發怒,杖二十,因此事者絕於
 饋遺。自丁艱憂,遂終身布衣蔬食。少時將娉舅息女,未成婚而協母亡,免喪後不復娶。至六十餘,此女猶未他適,協義而迎之。晚雖判合,卒無胤嗣。



 協博極群書,於文字及禽獸草木尤稱精詳。撰《異姓苑》五卷,《瑣語》十卷,並行於世。



 徐摛,字士秀,東海郯人也。祖憑道,宋海陵太守。父超之,天監初仕至員外散騎常侍。摛幼而好學,及長,遍覽經史。屬文好為新變,不拘舊體。起家太學博士,遷左衛司馬。會晉安王綱出戍石頭,高祖謂周捨曰:「為我求一人,文學俱長兼有行者,欲令與晉安遊處。」捨曰:「臣外弟徐
 摛,形質陋小,若不勝衣,而堪此選。」高祖曰:「必有仲宣之才,亦不簡其容貌。」以摛為侍讀。後王出鎮江州,仍補雲麾府記室參軍,又轉平西府中記室。王移鎮京口,復隨府轉為安北中錄事參軍,帶郯令,以母憂去職。王為丹陽尹,起摛為秣陵令。普通四年,王出鎮襄陽,摛固求隨府西上,遷晉安王諮議參軍。大通初,王總戎北伐,以摛兼寧蠻府長史,參贊戎政,教命軍書,多自摛出。王入為皇太子,轉家令,兼掌管記,尋帶領直。



 摛文體既別,春坊盡學之,「宮體」之號,自斯而起。高祖聞之怒,召摛加讓,及見,應對明敏,辭義可觀,高祖意釋。因問《五經》大義,次問
 歷代史及百家雜說,末論釋教。摛商較縱橫,應答如響,高祖甚加歎異,更被親狎,寵遇日隆。領軍朱異不說,謂所親曰:「徐叟出入兩宮,漸來逼我,須早為之所。」遂承間白高祖曰:「摛年老,又愛泉石,意在一郡,以自怡養。」高祖謂摛欲之,乃召摛曰:「新安大好山水,任昉等並經為之,卿為我臥治此郡。」中大通三年,遂出為新安太守。至郡,為治清靜,教民禮義,勸課農桑,期月之中,風俗便改。秩滿,還為中庶子,加戎昭將軍。



 是時臨城公納夫人王氏,即太宗妃之姪女也。晉宋已來,初婚三日,婦見舅姑,眾賓皆列觀,引《春秋》義云「丁丑,夫人姜氏至。戊寅,公使大夫
 宗婦覿用幣」。戊寅,丁丑之明日,故禮官據此,皆云宜依舊貫。太宗以問摛,摛曰:「《儀禮》云『質明贊見婦於舅姑』。《雜記》又云『婦見舅姑,兄弟姊妹皆立于堂下』。政言婦是外宗,未審嫻令,所以停坐三朝,觀其七德。舅延外客,姑率內賓,堂下之儀,以備盛禮。近代婦於舅姑,本有戚屬,不相瞻看。夫人乃妃姪女,有異他姻,覿見之儀,謂應可略。」太宗從其議。除太子左衛率。



 太清三年,侯景攻陷臺城,時太宗居永福省,賊眾奔入,舉兵上殿,侍衛奔散,莫有存者。摛獨嶷然侍立不動,徐謂景曰:「侯公當以禮見,何得如此。」凶威遂折。侯景乃拜,由是常憚摛。太宗嗣位,進
 授左衛將軍,固辭不拜。太宗後被幽閉,摛不獲朝謁,因感氣疾而卒,年七十八。長子陵,最知名。



 鮑泉,字潤岳,東海人也。父機,湘東王諮議參軍。泉博涉史傳,兼有文筆。少事元帝,早見擢任。及元帝承制,累遷至信州刺史。太清三年,元帝命泉征河東王譽於湘州,泉至長沙,作連城以逼之,譽率眾攻泉,泉據柵堅守,譽不能克。泉因其弊出擊之,譽大敗,盡俘其眾,遂圍其城,久未能拔。世祖乃數泉罪,遣平南將軍王僧辯代泉為都督。僧辯至,泉愕然,顧左右曰:「得王竟陵助我經略,賊不足平矣。」僧辯既入,乃背泉而坐,曰:「鮑郎有罪,令旨使
 我鎖卿,卿勿以故意見期。」因出令示泉,鎖之床下。泉曰:「稽緩王師,甘罪是分,但恐後人更思鮑泉之憒憒耳。」乃為啟謝淹遲之罪。世祖尋復其任,令與僧辯等率舟師東逼邵陵王於郢州。



 郢州平,元帝以長子方諸為刺史,泉為長史,行府州事。侯景密遣將宋子仙、任約率精騎襲之。方諸與泉不恤軍政,唯蒲酒自樂,賊騎至,百姓奔告,方諸與泉方雙陸,不信,曰:「徐文盛大軍在東,賊何由得至?」既而傳告者眾,始令闔門。賊縱火焚之,莫有抗者,賊騎遂入,城乃陷。執方諸及泉送之景所。後景攻王僧辯於巴陵,不克,敗還,乃殺泉於江夏,沉其屍於黃鵠磯。



 初,泉之為南討都督也,其友人夢泉得罪於世祖,覺而告之。後未旬,果見囚執。頃之,又夢泉著硃衣而行水上,又告泉曰:「君勿憂,尋得免矣。」因說其夢,泉密記之,俄而復見任,皆如其夢。



 泉於《儀禮》尤明,撰《新儀》四十卷,行於世。



 陳吏部尚書姚察曰:阮孝緒常言,仲尼論四科,始乎德行,終乎文學。有行者多尚質樸,有文者少蹈規矩,故衛、石靡餘論可傳,屈、賈無立德之譽。若夫憲章游、夏,祖述回、騫,體兼文行,於裴幾原見之矣。



\end{pinyinscope}