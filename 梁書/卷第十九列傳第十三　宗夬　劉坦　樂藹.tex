\article{卷第十九列傳第十三 宗夬 劉坦 樂藹}

\begin{pinyinscope}

 宗夬,字明揚,南陽涅陽人也,世居江陵。祖炳,宋時徵太子庶子不就,有高名。父繁,西中郎諮議參軍。夬少勤學,有局幹。弱冠,舉郢州秀才,歷臨川王常侍、驃騎行參軍。齊司徒竟陵王集學士於西邸,並見圖畫,夬亦預焉。永明中,與魏和親,敕夬與尚書殿中郎任昉同接魏使,皆時選也。



 武帝嫡孫南郡王居西州,以夬管書記,夬既以
 筆札被知,亦以貞正見許,故任焉。俄而文惠太子薨,王為皇太孫,夬仍管書記。及太孫即位,多失德,夬頗自疏,得為秣陵令,遷尚書都官郎。隆昌末,少帝見誅,寵舊多罹其禍,惟夬及傅昭以清正免。



 明帝即位,以夬為郢州治中,有名稱職,以父老去官還鄉里。南康王為荊州刺史,引為別駕。義師起,遷西中郎諮議參軍,別駕如故。時西土位望,惟夬與同郡樂藹、劉坦為州人所推信,故領軍將軍蕭穎胄深相委仗,每事諮焉。高祖師發雍州,穎胄遣夬出自楊口,面稟經略,并護送軍資,高祖甚禮之。中興初,遷御史中丞,以父憂去職。起為冠軍將軍、衛軍
 長史。天監元年,遷征虜長史、東海太守,將軍如故。二年,徵為太子右衛率。是冬,遷五兵尚書,參掌大選。三年,卒,時年四十九。子曜卿嗣。



 夬從弟岳,有名行,州里稱之,出於夬右。仕歷尚書庫部郎,郢州治中,北中郎錄事參軍事。



 劉坦,字德度,南陽安眾人也,晉鎮東將軍喬之七世孫。坦少為從兄虯所知。齊建元初,為南郡王國常侍,尋補孱陵令,遷南中郎錄事參軍,所居以幹濟稱。南康王為荊州刺史,坦為西中郎中兵參軍,領長流。義師起,遷諮議參軍。時輔國將軍楊公則為湘州刺史,帥師赴夏口,
 西朝議行州事者,坦謂眾曰:「湘境人情,易擾難信。若專用武士,則百姓畏侵漁;若遣文人,則威略不振。必欲鎮靜一州城,軍民足食,則無踰老臣。先零之役,竊以自許。」遂從之。乃除輔國長史、長沙太守,行湘州事。坦嘗在湘州,多舊恩,道迎者甚眾。下車簡選堪事吏,分詣十郡,悉發人丁,運租米三十餘萬斛,致之義師,資糧用給。



 時東昏遣安成太守劉希祖破西臺所選太守范僧簡於平都,希祖移檄湘部,於是始興內史王僧粲應之。邵陵人逐其內史褚洊,永陽人周暉起兵攻始安郡,並應僧粲。桂陽人邵曇弄、鄧道介報復私仇,因合黨亦同焉。僧粲
 自號平西將軍、湘州刺史,以永陽人周舒為謀主,師於建寧。自是湘部諸郡,悉皆蜂起;惟臨湘、湘陰、瀏陽、羅四縣猶全。州人咸欲汎舟逃走,坦悉聚船焚之,遣將尹法略距僧粲,相持未決。前湘州鎮軍鐘玄紹潛謀應僧粲,要結士庶數百人,皆連名定計,刻日反州城。坦聞其謀,偽為不知,因理訟至夜,而城門遂不閉,以疑之。玄紹未及發,明旦詣坦問其故。坦久留與語,密遣親兵收其家書。玄紹在坐未起,而收兵已報具得其文書本末,玄紹即首伏,於坐斬之。焚其文書,其餘黨悉無所問,眾愧且服,州部遂安。法略與僧粲相持累月,建康城平,公則還
 州,群賊始散。



 天監初,論功封荔浦縣子,邑三百戶。遷平西司馬、新興太守。天監三年,遷西中郎長史,卒,時年六十二。子泉嗣。



 樂藹,字蔚遠,南陽淯陽人,晉尚書令廣之六世孫,世居江陵。其舅雍州刺史宗愨,嘗陳器物,試諸甥姪。藹時尚幼,而所取惟書,愨由此奇之。又取史傳各一卷授藹等,使讀畢,言所記。藹略讀具舉,愨益善之。宋建平王景素為荊州刺史,辟為主簿。景素為南徐州,復為征北刑獄參軍,遷龍陽相。以父憂去職,吏民詣州請之,葬訖起焉。時齊豫章王嶷為武陵太守,雅善藹為政,及嶷為荊州
 刺史,以藹為驃騎行參軍、領州主簿,參知州事。嶷嘗問藹風土舊俗,城隍基𧾷寺,山川險易,藹隨問立對,若按圖牒,嶷益重焉。州人嫉之,或譖藹廨門如市,嶷遣覘之,方見藹閉閣讀書。嶷還都,以藹為太尉刑獄參軍,典書記,遷枝江令。還為大司馬中兵參軍,轉署記室。



 永明八年,荊州刺史巴東王子響稱兵反,既敗,焚燒府舍,官曹文書,一時蕩盡。武帝引見藹,問以西事,藹上對詳敏,帝悅焉。用為荊州治中,敕付以脩復府州事。藹還州,繕脩廨署數百區,頃之咸畢,而役不及民。荊部以為自晉王悅移鎮以來,府舍未之有也。



 九年,豫章王嶷薨,藹解官赴
 喪,率荊、湘二州故吏,建碑墓所。累遷車騎平西錄事參軍、步兵校尉,求助戍西歸。南康王為西中郎,以藹為諮議參軍。義師起,蕭穎胄引藹及宗夬、劉坦,任以經略。梁臺建,遷鎮軍司馬、中書侍郎、尚書左丞。時營造器甲,舟艦軍糧,及朝廷儀憲,悉資藹焉。尋遷給事黃門侍郎,左丞如故。和帝東下,道兼衛尉卿。



 天監初,遷驍騎將軍、領少府卿;俄遷御史中丞,領本州大中正。初,藹發江陵,無故於船得八車輻,如中丞健步避道者,至是果遷焉。藹性公彊,居憲臺甚稱職。時長沙宣武王將葬,而車府忽於庫火油絡,欲推主者。藹曰:「昔晉武庫火,張華以為積
 油萬石必然。今庫若有灰,非吏罪也。」既而檢之,果有積灰。時稱其博物弘恕焉。



 二年,出為持節、督廣、交、越三州諸軍、冠軍將軍、平越中郎將、廣州刺史。前刺史徐元瑜罷歸,道遇始興人士反,逐內史崔睦舒,因掠元瑜財產。元瑜走歸廣州,借兵於藹,託欲討賊,而實謀襲藹。藹覺之,誅元瑜。尋進號征虜將軍,卒官。



 藹姊適征士同郡劉虯,亦明識有禮訓。藹為州,迎姊居官舍,參分祿秩,西土稱之。



 子法才,字元備,幼與弟法藏俱有美名。少遊京師,造沈約,約見而稱之。齊和帝為相國,召為府參軍,鎮軍蕭穎胄辟主簿。梁臺建,除起部郎。天監二年,藹出鎮嶺表,
 法才留任京邑,遷金部郎,父憂去官。服闋,除中書通事舍人,出為本州別駕。入為通直散騎侍郎,復掌通事,遷尚書右丞。晉安王為荊州,重除別駕從事史。復徵為尚書右丞,出為招遠將軍、建康令。不受俸秩,比去任,將至百金,縣曹啟輸臺庫。高祖嘉其清節,曰:「居職若斯,可以為百城表矣。」即日遷太府卿。尋除南康內史,恥以讓俸受名,辭不拜。俄轉雲騎將軍、少府卿。出為信武長史、江夏太守。因被代,表便道還鄉。至家,割宅為寺,棲心物表。皇太子以法才舊臣,累有優令,召使東下,未及發而卒,時年六十三。



 陳吏部尚書姚察曰:蕭穎胄起大州之眾以會義,當其時,人心未之能悟。此三人者,楚之鎮也。經營締構,蓋有力焉。方面之功,坦為多矣;當官任事,藹則兼之。咸登寵秩,宜乎!



\end{pinyinscope}