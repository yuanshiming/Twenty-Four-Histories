\article{卷第五十二列傳第四十六 止足顧憲之 陶季直 蕭視素}

\begin{pinyinscope}

 《易》曰:「亢之為言也,知進而不知退,知存而不知亡。知進退存亡而不失其正者,其唯聖人乎!」《傳》曰:「知足不辱,知止不殆。」然則不知夫進退,不達乎止足,殆辱之累,期月而至矣。古人之進也,以康世濟務也,以弘道厲俗也。然其進也,光寵夷易,故愚夫之所乾沒;其退也,苦節艱貞,
 故庸曹之所忌憚。雖禍敗危亡,陳乎耳目,而輕舉高蹈,寡乎前史。漢世張良功成身退,病臥卻粒,比於樂毅、范蠡至乎顛狽,斯為優矣。其後薛廣德及二疏等,去就以禮,有可稱焉。魚豢《魏略·知足傳》,方田、徐於管、胡,則其道本異。謝靈運《晉書·止足傳》,先論晉世文士之避亂者,殆非其人;唯阮思曠遺榮好遁,遠殆辱矣。《宋書·止足傳》有羊欣、王微,咸其流亞。齊時沛國劉獻,字子珪,辭祿懷道,棲遲養志,不戚戚於貧賤,不耽耽於富貴,儒行之高者也。梁有天下,小人道消,賢士大夫相招在位,其量力守志,則當世罔聞,時或有致事告老,或有寡志少欲,國史
 書之,亦以為《止足傳》云。



 顧憲之,字士思,吳郡吳人也。祖抃之,宋鎮軍將軍、湘州刺史。憲之未弱冠,州辟議曹從事,舉秀才,累遷太子舍人、尚書比部郎、撫軍主簿。元徽中,為建康令。時有盜牛者,被主所認,盜者亦稱己牛,二家辭證等,前後令莫能決。憲之至,覆其狀,謂二家曰:「無為多言,吾得之矣。」乃令解牛,任其所去,牛徑還本主宅,盜者始伏其辜。發姦擿伏,多如此類,時人號曰神明。至於權要請託,長吏貪殘,據法直繩,無所阿縱。性又清儉,彊力為政,甚得民和。故京師飲酒者得醇旨,輒號為「顧建康」,言醑清且美焉。



 遷
 車騎功曹、晉熙王友。齊高帝執政,以為驃騎錄事參軍,遷太尉西曹掾。齊臺建,為中書侍郎。齊高帝即位,除衡陽內史。先是,郡境連歲疾疫,死者太半,棺木尤貴,悉裹以葦席,棄之路傍。憲之下車,分告屬縣,求其親黨,悉令殯葬。其家人絕滅者,憲之為出公祿,使綱紀營護之。又土俗,山民有病,輒云先人為禍,皆開冢剖棺,水洗枯骨,名為除祟。憲之曉喻,為陳生死之別,事不相由,風俗遂改。時刺史王奐新至,唯衡陽獨無訟者,乃歎曰:「顧衡陽之化至矣。若九郡率然,吾將何事!」



 還為太尉從事中郎。出為東中郎長史、行會稽郡事。山陰人呂文度有寵於
 齊武帝,於餘姚立邸,頗縱橫。憲之至郡,即表除之。文度後還葬母,郡縣爭赴弔,憲之不與相聞。文度深銜之,卒不能傷也。遷南中郎巴陵王長史,加建威將軍、行婺州事。時司徒、竟陵王於宣城、臨成、定陵三縣界立屯,封山澤數百里,禁民樵採,憲之固陳不可,言甚切直。王答之曰:「非君無以聞此德音。」即命無禁。



 遷給事黃門侍郎,兼尚書吏部郎中。宋世,其祖覬之嘗為吏部,於庭植嘉樹,謂人曰:「吾為憲之種耳。」至是,憲之果為此職。出為征虜長史、行南兗州事,遭母憂。服闋,建武中,復除給事黃門侍郎,領步兵校尉。未拜,仍遷太子中庶子,領吳邑中正。
 出為寧朔將軍、臨川內史;未赴,改授輔國將軍、晉陵太守。頃之遇疾,陳解還鄉里。永元初,徵為廷尉,不拜,除豫章太守。有貞婦萬晞者,少孀居無子,事舅姑尤孝,父母欲奪而嫁之,誓死不許,憲之賜以束帛,表其節義。



 中興二年,義師平建康,高祖為揚州牧,徵憲之為別駕從事史。比至,高祖已受禪,憲之風疾漸篤,固求還吳。天監二年,就家授太中大夫。憲之雖累經宰郡,資無擔石。及歸,環堵,不免飢寒。八年,卒於家,年七十四。臨終為制,以敕其子曰:夫出生入死,理均晝夜。生既不知所從來,死亦安識所往。延陵所云「精氣上歸于天,骨肉下歸于地,魂氣
 則無所不之」,良有以也。雖復茫昧難征,要若非妄。百年之期,迅若馳隙。吾今豫為終制,瞑目之後,念並遵行,勿違吾志也。



 莊周、澹臺,達生者也;王孫、士安,矯俗者也。吾進不及達,退無所矯。常謂中都之制,允理愜情。衣周於身,示不違禮;棺周於衣,足以蔽臭。入棺之物,一無所須。載以輴車,覆以粗布,為使人勿惡也。漢明帝天子之尊,猶祭以杅水脯糗;范史雲烈士之高,亦奠以寒水乾飯。況吾卑庸之人,其可不節衷也?喪易寧戚,自是親親之情;禮奢寧儉,差可得由吾意。不須常施靈筵,可止設香燈,使致哀者有憑耳。朔望祥忌,可權安小床,暫設几席,
 唯下素饌,勿用牲牢。蒸嘗之祠,貴賤罔替。備物難辦,多致疏怠。祠先人自有舊典,不可有闕。自吾以下,祠止用蔬食時果,勿同於上世也。示令子孫,四時不忘其親耳。孔子云:「雖菜羹瓜祭,必齊如也。」本貴誠敬,豈求備物哉?



 所著詩、賦、銘、贊并《衡陽郡記》數十篇。



 陶季直,丹陽秣陵人也。祖愍祖,宋廣州刺史。父景仁,中散大夫。季直早慧,愍祖甚愛異之。愍祖嘗以四函銀列置於前,令諸孫各取,季直時甫四歲,獨不取。人問其故,季直曰:「若有賜,當先父伯,不應度及諸孫,是故不取。」愍祖益奇之。五歲喪母,哀若成人。初,母未病,令於外染衣;
 卒後,家人始贖,季直抱之號慟,聞者莫不酸感。



 及長,好學,淡於榮利。起家桂陽王國侍郎、北中郎鎮西行參軍,並不起,時人號曰「聘君」。父憂服闋,尚書令劉秉領丹陽尹,引為後軍主簿、領郡功曹。出為望蔡令,頃之以病免。時劉秉、袁粲以齊高帝權勢日盛,將圖之,秉素重季直,欲與之定策。季直以袁、劉儒者,必致顛殞,固辭不赴。俄而秉等伏誅。



 齊初,為尚書比部郎,時褚淵為尚書令,與季直素善,頻以為司空司徒主簿,委以府事。淵卒,尚書令王儉以淵有至行,欲謚為文孝公,季直請曰:「文孝是司馬道子謚,恐其人非具美,不如文簡。」儉從之。季直又
 請儉為淵立碑,終始營護,甚有吏節,時人美之。



 遷太尉記室參軍。出為冠軍司馬、東莞太守,在郡號為清和。還除散騎侍郎,領左衛司馬,轉鎮西諮議參軍。齊武帝崩,明帝作相,誅鋤異己,季直不能阿意,明帝頗忌之,乃出為輔國長史、北海太守。邊職上佐,素士罕為之者。或勸季直造門致謝,明帝既見,便留之,以為驃騎諮議參軍,兼尚書左丞。仍遷建安太守,政尚清靜,百姓便之。還為中書侍郎,遷游擊將軍、兼廷尉。



 梁臺建,遷給事黃門侍郎。常稱仕至二千石,始願畢矣,無為務人間之事,乃辭疾還鄉里。天監初,就家拜太中大夫。高祖曰:「梁有天下,遂不見
 此人。」十年,卒于家,時年七十五。季直素清苦絕倫,又屏居十餘載,及死,家徒四壁,子孫無以殯斂,聞者莫不傷其志焉。



 蕭視素,蘭陵人也。祖思話,宋征西儀同三司;父惠明,吳興太守;皆有盛名。視素早孤貧,為叔父惠休所收恤。起家為齊司徒法曹行參軍,遷著作佐郎、太子舍人、尚書三公郎。永元末,為太子洗馬。梁臺建,高祖引為中尉驃騎記室參軍。天監初,為臨川王友,復為太子中舍人、丹陽尹丞。初拜,高祖賜錢八萬,視素一朝散之親友。又遷司徒左西屬、南徐州治中。



 性靜退,少嗜欲,好學,能清言,
 榮利不關於口,喜怒不形於色。在人間及居職,並任情通率,不自矜高,天然簡素,士人以此咸敬之。及在京口,便有終焉之志,乃於攝山築室。會徵為中書侍郎,遂辭不就,因還山宅,獨居屏事,非親戚不得至其籬門。妻,太尉王儉女,久與別居,遂無子。八年,卒。親故迹其事行,謚曰貞文先生。



 史臣曰:顧憲之、陶季直,引年者也,蕭視素則宦情鮮焉。比夫懷祿耽寵,婆娑人世,則殊間矣。



\end{pinyinscope}