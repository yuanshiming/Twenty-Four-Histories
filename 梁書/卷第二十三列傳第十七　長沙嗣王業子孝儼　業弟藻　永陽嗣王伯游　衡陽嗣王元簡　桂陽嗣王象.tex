\article{卷第二十三列傳第十七 長沙嗣王業子孝儼 業弟藻 永陽嗣王伯游 衡陽嗣王元簡 桂陽嗣王象}

\begin{pinyinscope}

 長沙嗣王業字靜曠,高祖長兄懿之子也。懿字元達,少有令譽。解褐齊安南邵陵王行參軍,襲爵臨湘縣侯。遷太子舍人、洗馬、建安王友。出為晉陵太守,曾未期月,訟理人和,稱為善政。入為中書侍郎。永明季,授持節、都督梁、南、北秦、沙四州諸軍事、西戎校尉、梁、南秦二州刺史,
 加冠軍將軍。是歲,魏人入漢中,遂圍南鄭。懿隨機拒擊,傷殺甚多,乃解圍遁去。懿又遣氐帥楊元秀攻魏歷城、皋蘭、駱谷、坑池等六戍,剋之。魏人震懼,邊境遂寧。進號征虜將軍,增封三百戶,遷督益、寧二州軍事、益州刺史。入為太子右衛率、尚書吏部郎、衛尉卿。永元二年,裴叔業據豫州反,授持節、征虜將軍、督豫州諸軍事、豫州刺史,領歷陽、南譙二郡太守,討叔業。叔業懼,降于魏。既而平西將軍崔慧景入寇京邑,奉江夏王寶玄圍臺城。齊室大亂,詔徵懿。懿時方食,投箸而起,率銳卒三千人援城。慧景遣其子覺來拒,懿奔擊,大破之,覺單騎走。乘勝
 而進,慧景眾潰,追斬之。授侍中、尚書右僕射,未拜。仍遷尚書令、都督征討水陸諸軍事,持節、將軍如故,增邑二千五百戶。時東昏肆虐,茹法珍、王咺之等執政,宿臣舊將,並見誅夷,懿既立元勳,獨居朝右,深為法珍等所憚,乃說東昏曰:「懿將行隆昌故事,陛下命在晷刻。」東昏信之,將加酷害,而懿所親知之,密具舟江渚,勸令西奔。懿曰:「古皆有死,豈有叛走尚書令耶?」遂遇禍。中興元年,追贈侍中、中書監、司徒。宣德太后臨朝,改贈太傅。天監元年,追崇丞相,封長沙郡王,謚曰宣武。給九旒、鸞輅、厓輬車,黃屋左纛,前後部羽葆鼓吹,挽歌二部,虎賁班劍百
 人,葬禮一依晉安平王故事。



 業幼而明敏,識度過人。仕齊為著作郎、太子舍人。宣武之難,與二弟藻、象俱逃匿。高祖既至,乃赴于軍,以為寧朔將軍。中興二年,除輔國將軍、南瑯邪、清河二郡太守。天監二年,襲封長沙王,徵為冠軍將軍,量置佐史,遷秘書監。四年,改授侍中。六年,轉散騎常侍、太子右衛率,遷左驍騎將軍,尋為中護軍,領石頭戍軍事。七年,出為使持節、都督南兗、兗、徐、青、冀五州諸軍事、仁威將軍、南兗州刺史。八年,徵為護軍。九年,除中書令,改授安後將軍、鎮瑯、邪彭城二郡、領南瑯邪太守。十年,徵為安右將軍、散騎常侍。十四年,復為護
 軍,領南瑯邪、彭城,鎮於瑯邪。復徵中書令,出為輕車將軍、湘州刺史。



 業性敦篤,所在留惠。深信因果,篤誠佛法,高祖每嘉歎之。普通三年,徵為散騎常侍、護軍將軍。四年,改為侍中、金紫光祿大夫。七年,薨,時年四十八。謚曰元。有文集行於世。子孝儼嗣。



 孝儼字希莊,聰慧有文才。射策甲科,除秘書郎、太子舍人。從幸華林園,於座獻《相風烏》、《華光殿》、《景陽山》等頌,其文甚美,高祖深賞異之。普通元年,薨,時年二十三。謚曰章。子慎嗣。



 藻字靖藝,元王弟也。少立名行,志操清潔。齊永元初,釋
 褐著作佐郎。天監元年,封西昌縣侯,食邑五百戶。出為持節、都督益、寧二州諸軍事、冠軍將軍、益州刺史。時天下草創,邊徼未安,州民焦僧護聚眾數萬,據郫、繁作亂。藻年未弱冠,集僚佐議,欲自擊之。或陳不可,藻大怒,斬于階側。乃乘平肩輿,巡行賊壘。賊弓亂射,矢下如雨,從者舉楯禦箭,又命除之,由是人心大安。賊乃夜遁,藻命騎追之,斬首數千級,遂平之。進號信威將軍,九年,徵為太子中庶子。十年,為左驍騎將軍、領南瑯邪太守。入為侍中。



 藻性謙退,不求聞達。善屬文辭,尤好古體,自非公宴,未嘗妄有所為,縱有小文,成輒棄本。十一年,出為使
 持節、都督雍、梁、秦三州竟陵、隨二郡諸軍事、仁威將軍、寧蠻校尉、雍州刺史。十二年,徵為使持節、都督南兗、兗、徐、青、冀五州諸軍事、兗州刺史,軍號如故。頻蒞數鎮,民吏稱之。推善下人,常如弗及。徵為太子詹事。普通三年,遷領軍將軍,加侍中。六年,為軍師將軍,與西豊侯正德北伐渦陽,輒班師,為有司所奏,免官削爵土。七年,起為宗正卿。八年,復封爵,尋除左衛將軍,領步兵校尉。



 大通元年,遷侍中、中護軍。時渦陽始降,乃以藻為使持節、北討都督、征北大將軍,鎮于渦陽。二年,為中權將軍、金紫光祿大夫,置佐史,加侍中。中大通元年,遷護軍將軍,中
 權如故。三年,為中軍將軍、太子詹事,出為丹陽尹。高祖每歎曰:「子弟並如迦葉,吾復何憂。」迦葉,藻小名也。入為安左將軍、尚書左僕射,加侍中,藻固辭不就,詔不許。大同五年,遷中衛將軍、開府儀同三司、中書令,侍中如故。



 藻性恬靜,獨處一室,床有膝痕,宗室衣冠,莫不楷則。常以爵祿太過,每思屏退,門庭閑寂,賓客罕通,太宗尤敬愛之。自遭家禍,恒布衣蒲席,不食鮮禽,非在公庭,不聽音樂。高祖每以此稱之。出為使持節、督南徐州刺史。侯景亂,藻遣長子彧率兵入援,及城開,加散騎常侍、大將軍。景遣其儀同蕭邕代之,據京口,藻因感氣疾,不自療。或
 勸奔江北,藻曰:「吾國之台鉉,位任特隆,既不能誅剪逆賊,正當同死朝廷,安能投身異類,欲保餘生。」因不食累日。太清三年,薨,時年六十七。



 永陽嗣王伯游,字士仁,高祖次兄敷之子。敷字仲達,解褐齊後將軍、征虜行參軍,輔太子舍人,洗馬,遷丹陽尹丞。入為太子中舍人,除建威將軍、隨郡內史。招懷遠近,黎庶安之,以為前後之政莫之及也。進號寧朔將軍,徵為廬陵王諮議參軍。建武四年,薨。高祖即位,追贈侍中、司空,封永陽郡王,謚曰昭。



 伯游美風神,善言玄理。天監元年四月,詔曰:「兄子伯游,雖年識未弘,意尚粗可。浙東
 奧區,宜須撫蒞,可督會稽、東陽、新安、永嘉、臨海五郡諸軍事、輔國將軍、會稽太守。」二年,襲封永陽郡王。五年,薨,時年二十三。謚曰恭。



 衡陽嗣王元簡,字熙遠,高祖第四弟暢之子。暢仕齊至太常,封江陵縣侯,卒。天監元年,追贈侍中、驃騎大將軍、開府儀同三司。封衡陽郡王。謚曰宣。



 元簡三年襲封,除中書郎,遷會稽太守。十三年,入為給事黃門侍郎,出為持節、都督廣、交、越三州諸軍事、平越中郎將、廣州刺史。還為太子中庶子,遷使持節、都督郢、司、霍三州諸軍事、信武將軍、郢州刺史。十八年正月,卒於州。謚曰孝。子俊
 嗣。



 桂陽嗣王象,字世翼,長沙宣武王第九子也。初,叔父融仕齊至太子洗馬。永元中,宣武之難,融遇害。高祖平京邑,贈給事黃門侍郎。天監元年,加散騎常侍、撫軍大將軍,封桂陽郡王。謚曰簡。無子,乃詔象為嗣,襲封爵。



 象容止閑雅,善於交遊,事所生母以孝聞。起家寧遠將軍、丹陽尹。到官未幾,簡王妃薨,去職。服闋,復授明威將軍、丹陽尹。象生長深宮,始親庶政,舉無失德,朝廷稱之。出為持節、督司、霍、郢三州諸軍事、征遠將軍、郢州刺史。尋遷湘、衡二州諸軍事、輕車將軍、湘州刺史。湘州舊多虎暴,
 及象在任,為之靜息,故老咸稱德政所感。除中書侍郎,俄以本官行石頭戍軍事,轉給事黃門侍郎、兼領軍,又以本官兼宗正卿。尋遷侍中、太子詹事,未拜,改授持節、督江州諸軍事、信武將軍、江州刺史。以疾免。尋除太常卿,加侍中,遷秘書監、領步兵校尉。大同二年,薨,謚曰敦。子慥嗣。



 史臣曰:長沙諸嗣王,並承襲土宇,光有籓服。桂陽王象以孝聞,在於牧湘,猛虎息暴,蓋德惠所致也。昔之善政,何以加焉。



\end{pinyinscope}