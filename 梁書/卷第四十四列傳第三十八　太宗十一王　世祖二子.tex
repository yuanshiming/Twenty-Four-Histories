\article{卷第四十四列傳第三十八 太宗十一王 世祖二子}

\begin{pinyinscope}

 太宗王皇后生哀太子大器、南郡王大連,陳淑容生潯陽王大心,左夫人生南海王大臨、安陸王大春,謝夫人生瀏陽公大雅,張夫人生新興王大莊,包昭華生西陽王大鈞,范夫人生武寧王大威,褚脩華生建平王大球,陳夫人生義安王大昕,朱夫人生綏建王大摯。自餘諸子,本書不載。



 潯陽王大心,字仁恕。幼而聰朗,善屬文。中大通四年,以皇孫封當陽公,邑一千五百戶。大同元年,出為使持節、都督郢、南、北司、定、新五州諸軍事、輕車將軍、郢州刺史。時年十三,太宗以其幼,恐未達民情,戒之曰:「事無大小,悉委行事,纖毫不須措懷。」大心雖不親州務,發言每合於理,眾皆驚服。七年,徵為侍中、兼石頭戍軍事。太清元年,出為雲麾將軍、江州刺史。二年,侯景寇京邑。大心招集士卒,遠近歸之,眾至數萬,與上流諸軍赴援宮闕。三年,城陷,上甲侯蕭韶南奔,宣密詔,加散騎常侍,進號平南將軍。大寶元年,封尋陽王,邑二千戶。



 初,歷陽太守莊
 鐵以城降侯景,既而又奉其母來奔,大心以鐵舊將,厚為其禮,軍旅之事,悉以委之,仍以為豫章內史。侯景數遣軍西上寇抄,大心輒令鐵擊破之,賊不能進。時鄱陽王範率眾棄合肥,屯于柵口,待援兵總集,欲俱進。大心聞之,遣要範西上,以湓城處之,廩饋甚厚,與戮力共除禍難。會莊鐵據豫章反,大心令中兵參軍韋約等將軍擊之,鐵敗績,又乞降。鄱陽世子嗣先與鐵遊處,因稱其人才略從橫,且舊將也,欲舉大事,當資其力,若降江州,必不全其首領,嗣請援之。範從之,乃遣將侯瑱率精甲五千往救鐵,夜襲破韋約等營。大心聞之大懼,於是二
 籓釁起,人心離貳。景將任約略地至於湓城,大心遣司馬韋質拒戰,敗績。時帳下猶有勇士千餘人,咸說曰:「既無糧儲,難以守固。若輕騎往建州,以圖後舉,策之上者也。」大心未決,其母陳淑容曰:「即日聖御年尊,儲宮萬福,汝久奉違顏色,不念拜謁闕庭,且吾已老,而欲遠涉險路,糧儲不給,豈謂孝子?吾終不行。」因撫胸慟哭,大心乃止。遂與約和。二年秋,遇害,時年二十九。



 南海王大臨,字仁宣。大同二年,封寧國縣公,邑一千五百戶。少而敏慧。年十一,遭左夫人憂,哭泣毀瘠,以孝聞。後入國學,明經射策甲科,拜中書侍郎,遷給事黃門侍
 郎。十一年,為長兼侍中。出為輕車將軍,瑯邪、彭城二郡太守。侯景亂,為使持節、宣惠將軍,屯新亭。俄又徵還,屯端門,都督城南諸軍事。時議者皆勸收外財物,擬供賞賜,大臨獨曰:「物乃賞士,而牛可犒軍。」命取牛,得千餘頭,城內賴以饗士。大寶元年,封南海郡王,邑二千戶。出為使持節、都督揚、南徐二州諸軍事、安南將軍、揚州刺史。又除安東將軍、吳郡太守。時張彪起義於會稽,吳人陸令公、潁川庾孟卿等勸大臨走投彪。大臨曰:「彪若成功,不資我力;如其撓敗,以我說焉。不可往也。」二年秋,遇害於郡,時年二十五。



 南郡王大連,字仁靖。少俊爽,能屬文,舉止風流,雅有巧思,妙達音樂,兼善丹青。大同二年,封臨城縣公,邑一千五百戶。七年,與南海王俱入國學,射策甲科,拜中書侍郎。十年,高祖幸朱方,大連與兄大臨並從。高祖問曰:「汝等習騎不?」對曰:「臣等未奉詔,不敢輒習。」敕各給馬試之,大連兄弟據鞍往還,各得馳驟之節,高祖大悅,即賜所乘馬。及為啟謝,詞又甚美。高祖佗日謂太宗曰:「昨見大臨、大連,風韻可愛,足以慰吾年老。」遷給事黃門侍郎,轉侍中,尋兼石頭戍軍事。太清元年,出為使持節、輕車將軍、東揚州刺史。侯景入寇京師,大連率眾四萬來赴。及
 臺城沒,援軍散,復還揚州。三年,會稽山賊田領群聚黨數萬來攻,大連命中兵參軍張彪擊斬之。大寶元年,封為南郡王,邑二千戶。景仍遣其將趙伯超、劉神茂來討,大連設備以待之。會將留異以城應賊,大連棄城走,至信安,為賊所獲。侯景以為輕車將軍、行揚州事,遷平南將軍、江州刺史。大連既迫寇手,恒思逃竄,乃與賊約曰:「軍民之事,吾不預焉。候我存亡,但聽鐘響。」欲簡與相見,因得亡逸,賊亦信之。事未果。二年秋,遇害,時年二十五。



 安陸王大春,字仁經。少博涉書記。天性孝謹,體貌環偉,腰帶十圍。大同六年,封西豊縣公,邑一千五百戶。拜中
 書侍郎。後為寧遠將軍,知石頭戍軍事。侯景內寇,大春奔京口,隨邵陵王入援,戰于鐘山,為賊所獲。京城既陷,大寶元年,封安陸郡王,邑二千戶。出為使持節、雲麾將軍、東揚州刺史。二年秋,遇害,時年二十二。



 瀏陽公大雅,字仁風。大同九年,封瀏陽縣公,邑一千五百戶。少聰警,美姿儀,特為高祖所愛。太清三年,京城陷,賊已乘城,大雅猶命左右格戰,賊至漸眾,乃自縋而下。因發憤感疾,薨,時年十七。



 新興王大莊,字仁禮。大同九年,封高唐縣公,邑一千五百戶。大寶元年,封新興郡王,邑二千戶。出為使持節、都
 督南徐州諸軍事、宣毅將軍、南徐州刺史。二年秋,遇害,時年十八。



 西陽王大鈞,字仁輔。性厚重,不妄戲弄。年七歲,高祖嘗問讀何書,對曰「學《詩》」。因命諷誦,音韻清雅,高祖因賜王羲之書一卷。大寶元年,封西陽郡王,邑二千戶。出為宣惠將軍、丹陽尹。二年,監揚州,將軍如故。至秋遇害,時年十三。



 武寧王大威,字仁容。美風儀,眉目如畫。大寶元年,封武寧郡王,邑二千戶。二年,出為信威將軍、丹陽尹。其年秋,遇害,時年十三。



 建平王大球,字仁珽。大寶元年,封建平郡王,邑二千戶。性明慧夙成。初,侯景圍京城,高祖素歸心釋教,每發誓願,恒云:「若有眾生應受諸苦,悉衍身代當。」時大球年甫七歲,聞而驚謂母曰:「官家尚爾,兒安敢辭?」乃六時禮佛,亦云:「凡有眾生應獲苦報,悉大球代受。」其早慧如此。二年,出為輕車將軍、兼石頭戍軍事。其年秋,遇害,時年十一。



 義安王大昕,字仁朗。年四歲,母陳夫人卒,便哀慕毀悴,有若成人。及高祖崩,大昕奉慰太宗,嗚咽不能自勝。左右見之,莫不掩泣。大寶元年,封義安郡王,邑二千戶。二
 年,出為寧遠將軍、瑯邪、彭城二郡太守,未之鎮,遇害,時年十一。



 綏建王大摯,字仁瑛。幼雄壯有膽氣,及京城陷,乃歎曰:「大丈夫會當滅虜屬。」奶媼驚,掩其口曰:「勿妄言,禍將及!」大摯笑曰:「禍至非由此言。」大寶元年,封綏建郡王,邑二千戶。二年,為寧遠將軍,遇害,時年十歲。



 世祖諸男:徐妃生忠壯世子方等,王夫人生貞惠世子方諸,其愍懷太子方矩(本書不載所生,別有傳),夏賢妃生敬皇帝。自餘諸子,並本書無傳。



 忠壯世子方等,字實相,世祖長子也。母曰徐妃。少聰敏,
 有俊才,善騎射,尤長巧思。性愛林泉,特好散逸。嘗著論曰:「人生處世,如白駒過隙耳。一壺之酒,足以養性;一簞之食,足以怡形。生在蓬蒿,死葬溝壑,瓦棺石槨,何以異茲?吾嘗夢為魚,因化為鳥。當其夢也,何樂如之;及其覺也,何憂斯類;良由吾之不及魚鳥者,遠矣。故魚鳥飛浮,任其志性;吾之進退,恒存掌握。舉手懼觸,搖足恐墮。若使吾終得與魚鳥同遊,則去人間如脫屣耳。」初,徐妃以嫉妒失寵,方等意不自安。世祖聞之,又惡方等,方等益懼,故述論以申其志焉。



 會高祖欲見諸王長子,世祖遣方等入侍,方等欣然升舟,冀免憂辱。行至繇水,值侯景
 亂,世祖召之,方等啟曰:「昔申生不愛其死,方等豈顧其生?」世祖省書歎息,知無還意,乃配步騎一萬,使援京都。賊每來攻,方等必身當矢石。宮城陷,方等歸荊州,收集士馬,甚得眾和,世祖始歎其能。方等又勸修築城柵,以備不虞。既成,樓雉相望,周迴七十餘里。世祖觀之甚悅,入謂徐妃曰:「若更有一子如此,吾復何憂!」徐妃不答,垂泣而退。世祖忿之,因疏其穢行,榜于大閣。方等入見,益以自危。時河東王為湘州刺史,不受督府之令,方等乃乞征之,世祖許焉。拜為都督,令帥精卒二萬南討。方等臨行,謂所親曰:「吾此段出征,必死無二;死而獲所,吾豈
 愛生。」及至麻溪,河東王率軍逆戰,方等擊之,軍敗,遂溺死,時年二十二。世祖聞之,不以為戚。後追思其才,贈侍中、中軍將軍、揚州刺史,謚曰忠壯世子,并為招魂以哀之。



 方等注范曄《後漢書》,未就;所撰《三十國春秋》及《靜住子》,行於世。



 貞惠世子方諸,字智相,世祖第二子。母王夫人。幼聰警博學,明《老》、《易》,善談玄,風采清越,辭辯鋒生,特為世祖所愛,母王氏又有寵。及方等敗沒,世祖謂之曰:「不有所廢,其何以興。」因拜為中撫軍以自副,又出為郢州刺史,鎮江夏,以鮑泉為行事,防遏下流。時世祖遣徐文盛督眾
 軍,與侯景將任約相持未決。方諸恃文盛在近,不恤軍政,日與鮑泉蒲酒為樂。侯景知之,乃遣其將宋子仙率輕騎數百,從間道襲之。屬風雨晦冥,子仙至,百姓奔告,方諸與鮑泉猶不信,曰:「徐文盛大軍在下,虜安得來?」始命閉門,賊騎已入,城遂陷,子仙執方諸以歸。王僧辯軍至蔡洲,景遂害之。世祖追贈侍中、大將軍。謚曰貞惠世子。



 史臣曰:太宗、世祖諸子,雖開土宇,運屬亂離;既拘寇賊,多殞非命。籲!可嗟矣。



\end{pinyinscope}