\article{卷第十一列傳第五 張弘策 庾域 鄭紹叔 呂僧珍}

\begin{pinyinscope}

 張弘策,字真簡,范陽方城人,文獻皇后之從父弟也。幼以孝聞。母嘗有疾,五日不食,弘策亦不食。母彊為進粥,乃食母所餘。遭母憂,三年不食鹽菜,幾至滅性。兄弟友愛,不忍暫離,雖各有室,常同臥起,世比之姜肱兄弟。起家齊邵陵王國常侍,遷奉朝請、西中郎江夏王行參軍。



 弘策與高祖年相輩,幼見親狎,恆隨高祖遊處。每入室,
 常覺有雲煙氣,體輒肅然,弘策由此特敬高祖。建武末,弘策從高祖宿,酒酣,徙席星下,語及時事。弘策因問高祖曰:「緯象云何?國家故當無恙?」高祖曰:「其可言乎?」弘策因曰:「請言其兆。」高祖曰:「漢北有失地氣,浙東有急兵祥。今冬初,魏必動;若動則亡漢北。帝今久疾,多異議,萬一伺釁,稽部且乘機而作,是亦無成,徒自驅除耳。明年都邑有亂,死人過於亂麻,齊之歷數,自茲亡矣。梁、楚、漢當有英雄興。」弘策曰:「英雄今何在?為已富貴,為在草茅?」高祖笑曰:「光武有云:『安知非僕?』」弘策起曰:「今夜之言,是天意也。請定君臣之分。」高祖曰:「舅欲效鄧晨乎?」是冬,魏軍
 寇新野,高祖將兵為援,且受密旨,仍代曹虎為雍州。弘策聞之心喜,謂高祖曰:「夜中之言,獨當驗矣。」高祖笑曰:「且勿多言。」弘策從高祖西行,仍參帷幄,身親軍役,不憚辛苦。



 五年秋,明帝崩,遺詔以高祖為雍州刺史,乃表弘策為錄事參軍,帶襄陽令。高祖睹海內方亂,有匡濟之心,密為儲備,謀猷所及,惟弘策而已。時長沙宣武王罷益州還,仍為西中郎長史,行郢州事。高祖使弘策到郢,陳計於宣武王,語在《高祖紀》。弘策因說王曰:「昔周室既衰,諸侯力爭,齊桓蓋中人耳,遂能一匡九合,民到於今稱之。齊德告微,四海方亂,蒼生之命,會應有主。以郢州
 居中流之要,雍部有戎馬之饒,卿兄弟英武,當今無敵,虎據兩州,參分天下,糾合義兵,為百姓請命,廢昏立明,易於反掌。如此,則桓、文之業可成,不世之功可建。無為豎子所欺,取笑身後。雍州揣之已熟,願善圖之。」王頗不懌而無以拒也。



 義師將起,高祖夜召弘策、呂僧珍入宅定議,旦乃發兵,以弘策為輔國將軍、軍主,領萬人督後部軍事。西臺建,為步兵校尉,遷車騎諮議參軍。及郢城平,蕭穎達、楊公則諸將皆欲頓軍夏口,高祖以為宜乘勢長驅,直指京邑,以計語弘策,弘策與高祖意合。又訪寧遠將軍庾域,域又同。乃命眾軍即日上道,沿江至建
 康,凡磯、浦、村落,軍行宿次、立頓處所,弘策逆為圖測,皆在目中。義師至新林,王茂、曹景宗等於大航方戰,高祖遣弘策持節勞勉,眾咸奮厲。是日,仍破硃雀軍。高祖入頓石頭城,弘策屯門禁衛,引接士類,多全免。城平,高祖遣弘策與呂僧珍先入清宮,封檢府庫。于時城內珍寶委積,弘策申勒部曲,秋毫無犯。遷衛尉卿,加給事中。天監初,加散騎常侍,洮陽縣侯,邑二千二百戶。弘策盡忠奉上,知無不為,交友故舊,隨才薦拔,搢紳皆趨焉。



 時東昏餘黨初逢赦令,多未自安,數百人因運荻炬束仗,得入南北掖作亂,燒神虎門、總章觀。前軍司馬呂僧珍直
 殿內,以宿衛兵拒破之,盜分入衛尉府,弘策方救火,盜潛後害之,時年四十七。高祖深慟惜焉。給第一區,衣一襲,錢十萬,布百匹,蠟二百斤。詔曰:「亡從舅衛尉,慮發所忽,殞身祅豎。其情理清貞,器識淹濟,自籓升朝,契闊夷阻。加外氏凋衰,饗嘗屢絕,興感《渭陽》,情寄斯在。方賴忠勳,翼宣寡薄,報效無征,永言增慟。可贈散騎常侍、車騎將軍。給鼓吹一部。謚曰愍。」



 弘策為人寬厚通率,篤舊故。及居隆重,不以貴勢自高。故人賓客,禮接如布衣時。祿賜皆散之親友。及其遇害,莫不痛惜焉。子緬嗣,別有傳。



 庾域,字司大,新野人。長沙宣武王為梁州,以為錄事參
 軍,帶華陽太守。時魏軍攻圍南鄭,州有空倉數十所,域封題指示將士云:「此中粟皆滿,足支二年,但努力堅守。」眾心以安。虜退,以功拜羽林監,遷南中郎記室參軍。永元末,高祖起兵,遣書招域。西臺建,以為寧朔將軍,領行選,從高祖東下。師次楊口,和帝遣御史中丞宗夬銜命勞軍。域乃諷夬曰:「黃鉞未加,非所以總率侯伯。」夬反西臺,即授高祖黃鉞。蕭穎胄既都督中外諸軍事,論者謂高祖應致箋,域爭不聽,乃止。郢城平。域及張弘策議與高祖意合,即命眾軍便下。每獻謀畫,多被納用。霸府初開,以為諮議參軍。天監初,封廣牧縣子,後軍司馬。出為
 寧朔將軍、巴西、梓潼二郡太守。梁州長史夏侯道遷舉州叛降魏,魏騎將襲巴西,域固守百餘日,城中糧盡,將士皆齕草食土,死者太半,無有離心。魏軍退,詔增封二百戶,進爵為伯。六年,卒於郡。



 鄭紹叔,字仲明,滎陽開封人也。世居壽陽。祖琨,宋高平太守。紹叔少孤貧。年二十餘,為安豊令,居縣有能名。本州召補主簿,轉治中從事史。時刺史蕭誕以弟諶誅,臺遣收兵卒至,左右莫不驚散,紹叔聞難,獨馳赴焉。誕死,侍送喪柩,眾咸稱之。到京師,司空徐孝嗣見而異之,曰:「祖逖之流也。」



 高祖臨司州,命為中兵參軍,領長流,因是
 厚自結附。高祖罷州還京師,謝遣賓客,紹叔獨固請願留。高祖謂曰:「卿才幸自有用,我今未能相益,宜更思他塗。」紹叔曰:「委質有在,義無二心。」高祖固不許,於是乃還壽陽。刺史蕭遙昌苦引紹叔,終不受命。遙昌怒,將囚之,救解得免。及高祖為雍州刺史,紹叔間道西歸,補寧蠻長史、扶風太守。



 東昏既害朝宰,頗疑高祖。紹叔兄植為東昏直後,東昏遣至雍州,託以候紹叔,實潛使為刺客。紹叔知之,密以白高祖。植既至,高祖於紹叔處置酒宴之,戲植曰:「朝廷遣卿見圖,今日閑宴,是見取良會也。」賓主大笑。令植登臨城隍,周觀府署,士卒、器械、舟艫、戰馬,
 莫不富實。植退謂紹叔曰:「雍州實力,未易圖也。」紹叔曰:「兄還,具為天子言之。兄若取雍州,紹叔請以此眾一戰。」送兄於南峴,相持慟哭而別。



 義師起,為冠軍將軍,改驍騎將軍,侍從東下江州,留紹叔監州事,督江、湘二州糧運,事無闕乏。天監初,入為衛尉卿。紹叔忠於事上,外所聞知,纖毫無隱。每為高祖言事,善則曰:「臣愚不及,此皆聖主之策。」其不善,則曰:「臣慮出淺短,以為其事當如是,殆以此誤朝廷,臣之罪深矣。」高祖甚親信之。母憂去職。紹叔有至性,高祖常使人節其哭。頃之,起為冠軍將軍、右軍司馬,封營道縣侯,邑千戶。俄復為衛尉卿,加冠軍
 將軍。以營道縣戶凋弊,改封東興縣侯,邑如故。初,紹叔少失父,事母及祖母以孝聞,奉兄恭謹。及居顯要,祿賜所得及四方貢遺,悉歸之兄室。



 三年,魏軍圍合肥,紹叔以本號督眾軍鎮東關,事平,復為衛尉。既而義陽為魏所陷,司州移鎮關南。四年,以紹叔為使持節、征虜將軍、司州刺史。紹叔創立城隍,繕修兵器,廣田積穀,招納流民,百姓安之。性頗矜躁,以權勢自居,然能傾心接物,多所薦舉,士類亦以此歸之。



 六年,徵為左將軍,加通直散騎常侍,領司、豫二州大中正。紹叔至家疾篤。詔於宅拜授,輿載還府,中使醫藥,一日數至。七年,卒於府舍,時年四
 十五。高祖將臨其殯,紹叔宅巷狹陋,不容輿駕,乃止。詔曰:「追往念功,前王所篤;在誠惟舊,異代同規。通直散騎常侍、右衛將軍、東興縣開國侯紹叔,立身清正,奉上忠恪,契闊籓朝,情績顯著。爰及義始,實立茂勛,作牧疆境,效彰所蒞。方申任寄,協贊心膂;奄至殞喪,傷痛于懷。宜加優典,隆茲寵命。可贈散騎常侍、護軍將軍,給鼓吹一部,東園祕器,朝服一具,衣一襲,兇事所須,隨由資給。謚曰忠。」



 紹叔卒後,高祖嘗潸然謂朝臣曰:「鄭紹叔立志忠烈,善則稱君,過則歸己,當今殆無其比。」其見賞惜如此。子貞嗣。



 呂僧珍,字元瑜,東平范人也。世居廣陵。起自寒賤。始童兒時,從師學,有相工歷觀諸生,指僧珍謂博士曰:「此有奇聲,封侯相也。」年二十餘,依宋丹陽尹劉秉,秉誅後,事太祖文皇為門下書佐。身長七尺五寸,容貌甚偉。在同類中少所褻狎,曹輩皆敬之。



 太祖為豫州刺史,以為典簽,帶蒙令,居官稱職。太祖遷領軍,補主簿。妖賊唐瑀寇東陽,太祖率眾東討,使僧珍知行軍眾局事。僧珍宅在建陽門東,自受命當行,每日由建陽門道,不過私室,太祖益以此知之。為丹陽尹,復命為郡督郵。齊隨王子隆出為荊州刺史,齊武以僧珍為子隆防閣,從之鎮。永明
 九年,雍州刺史王奐反,敕遣僧珍隸平北將軍曹虎西為典簽,帶新城令。魏軍寇沔北,司空陳顯達出討,一見異之,因屏人呼上座,謂曰:「卿有貴相,後當不見減,努力為之。」



 建武二年,魏大舉南侵,五道並進。高祖率師援義陽,僧珍從在軍中。長沙宣武王時為梁州刺史。魏圍守連月,間諜所在不通,義陽與雍州路斷。高祖欲遣使至襄陽,求梁州問,眾皆憚,莫敢行,僧珍固請充使,即日單舸上道。既至襄陽,督遣援軍,且獲宣武王書而反,高祖甚嘉之。事寧,補羽林監。



 東昏即位,司空徐孝嗣管朝政,欲與共事,僧珍揣不久安,竟弗往。時高祖已臨雍州,僧
 珍固求西歸,得補邔令。既至,高祖命為中兵參軍,委以心膂。僧珍陰養死士,歸之者甚眾。高祖頗招武猛,士庶響從,會者萬餘人,因命按行城西空地,將起數千間屋,以為止舍,多伐材竹,沈於檀溪,積茅蓋若山阜,皆不之用。僧珍獨悟其旨,亦私具櫓數百張。義兵起,高祖夜召僧珍及張弘策定議,明旦乃會眾發兵,悉取檀溪材竹,裝為艛艦,葺之以茅,並立辦。眾軍將發,諸將果爭櫓,僧珍乃出先所具者,每船付二張,爭者乃息。



 高祖以僧珍為輔國將軍、步兵校尉,出入臥內,宣通意旨。師及郢城,僧珍率所領頓偃月壘,俄又進據騎城。郢州平,高祖進
 僧珍為前鋒大將軍。大軍次江寧,高祖令僧珍與王茂率精兵先登赤鼻邏。其日,東昏將李居士與眾來戰,僧珍等要擊,大破之。乃與茂進軍於白板橋築壘,壘立,茂移頓越城,僧珍獨守白板。李居士密覘知眾少,率銳卒萬人,直來薄城。僧珍謂將士曰:「今力既不敵,不可與戰;亦勿遙射,須至塹裏,當並力破之。俄而皆越塹拔柵,僧珍分人上城,矢石俱發,自率馬步三百人出其後,守隅者復踰城而下,內外齊擊,居士應時奔散,獲其器甲不可勝計。僧珍又進據越城。東昏大將王珍國列車為營,背淮而陣。王茂等眾軍擊之,僧珍縱火車焚其營。即日
 瓦解。



 建康城平,高祖命僧珍率所領先入清宮,與張弘策封檢府庫,即日以本官帶南彭城太守,遷給事黃門侍郎,領虎賁中郎將。高祖受禪,以為冠軍將軍、前軍司馬,封平固縣侯,邑一千二百戶。尋遷給事中、右衛將軍。頃之,轉左衛將軍,加散騎常侍,入直秘書省,總知宿衛。天監四年冬,大舉北伐,自是軍機多事,僧珍晝直中書省,夜還秘書。五年夏,又命僧珍率羽林勁勇出梁城。其年冬旋軍,以本官領太子中庶子。



 僧珍去家久,表求拜墓。高祖欲榮之,使為本州,乃授使持節、平北將軍、南兗州刺史。僧珍在任,平心率下,不私親戚。從父兄子先以
 販葱為業,僧珍既至,乃棄業欲求州官。僧珍曰:「吾荷國重恩,無以報效,汝等自有常分,豈可妄求叨越,但當速反蔥肆耳。」僧珍舊宅在市北,前有督郵廨,鄉人咸勸徒廨以益其宅。僧珍怒曰:「督郵官廨也,置立以來,便在此地,豈可徙之益吾私宅!」姊適于氏,住在市西,小屋臨路,與列肆雜處,僧珍常導從鹵簿到其宅,不以為恥。在州百日,徵為領軍將軍,尋加散騎常侍,給鼓吹一部,直秘書省如先。



 僧珍有大勛,任總心膂,恩遇隆密,莫與為比。性甚恭慎,當直禁中,盛暑不敢解衣。每侍御座,屏氣鞠躬,果食未嘗舉箸。嘗因醉後,取一柑食之。高祖笑謂曰:「
 便是大有所進。」祿俸之外,又月給錢十萬;其餘賜賚不絕於時。



 十年,疾病,車駕臨幸,中使醫藥,日有數四。僧珍語親舊曰:「吾昔在蒙縣,熱病發黃,當時必謂不濟,主上見語,『卿有富貴相,必當不死,尋應自差』,俄而果愈。今已富貴而復發黃,所苦與昔正同,必不復起矣。」竟如其言。卒于領軍府舍,時年五十八。高祖即日臨殯,詔曰:「思舊篤終,前王令典;追榮加等,列代通規。散騎常侍、領軍將軍、平固縣開國侯僧珍,器思淹通,識宇詳濟,竭忠盡禮,知無不為。與朕契闊,情兼屯泰。大業初構,茂勛克舉。及居禁衛,朝夕盡誠。方參任台槐,式隆朝寄;奄致喪逝,傷
 慟於懷。宜加優典,以隆寵命。可贈驃騎將軍、開府儀同三司,常侍、鼓吹、侯如故。給東園秘器,朝服一具,衣一襲,喪事所須,隨由備辦。謚曰忠敬侯。」高祖痛惜之,言為流涕。長子峻早卒,峻子淡嗣。



 陳吏部尚書姚察曰:張弘策敦厚慎密,呂僧珍恪勤匪懈,鄭紹叔忠誠亮藎,締構王業,三子皆有力焉。僧珍之肅恭禁省,紹叔之造膝詭辭,蓋識為臣之節矣。



\end{pinyinscope}