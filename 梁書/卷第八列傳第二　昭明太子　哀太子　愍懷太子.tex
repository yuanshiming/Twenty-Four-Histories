\article{卷第八列傳第二 昭明太子 哀太子 愍懷太子}

\begin{pinyinscope}

 昭明太子統,字德施,高祖長子也。母曰丁貴嬪。初,高祖未有男,義師起,太子以齊中興元年九月生于襄陽。高祖既受禪,有司奏立儲副,高祖以天下始定,百度多闕,未之許也。群臣固請,天監元年十一月,立為皇太子。時太子年幼,依舊居於內,拜東宮官屬文武,皆入直永福省。



 太子生而聰睿,三歲受《孝經》、《論語》,五歲遍讀五經,悉
 能諷誦。五年五月庚戌,始出居東宮。太子性仁孝,自出宮,恒思戀不樂。高祖知之,每五日一朝,多便留永福省,或五日三日乃還宮。八年九月,於壽安殿講《孝經》,盡通大義。講畢,親臨釋奠於國學。十四年正月朔旦,高祖臨軒,冠太子於太極殿。舊制,太子著遠遊冠,金蟬翠緌纓;至是,詔加金博山。



 太子美姿貌,善舉止。讀書數行並下,過目皆憶。每遊宴祖道,賦詩至十數韻。或命作劇韻賦之,皆屬思便成,無所點易。高祖大弘佛教,親自講說;太子亦崇信三寶,遍覽眾經。乃於宮內別立慧義殿,專為法集之所。招引名僧,談論不絕。太子自立三諦、法身義,並
 有新意。普通元年四月,甘露降於慧義殿,咸以為至德所感焉。



 三年十一月,始興王憺薨。舊事,以東宮禮絕傍親,書翰並依常儀。太子意以為疑,命僕射劉孝綽議其事。孝綽議曰:「案張鏡撰《東宮儀記》,稱『三朝發哀者,踰月不舉樂;鼓吹寢奏,服限亦然』。尋傍絕之義,義在去服,服雖可奪,情豈無悲?鐃歌輟奏,良亦為此。既有悲情,宜稱兼慕,卒哭之後,依常舉樂,稱悲竟,此理例相符。謂猶應稱兼慕,至卒哭。」僕射徐勉、左率周舍、家令陸襄並同孝綽議。太子令曰:「張鏡《儀記》云『依《士禮》,終服月稱慕悼』。又云『凡三朝發哀者,踰月不舉樂』。劉僕射議,云『傍絕之義,
 義在去服,服雖可奪,情豈無悲,卒哭之後,依常舉樂,稱悲竟,此理例相符』。尋情悲之說,非止卒哭之後,緣情為論,此自難一也。用張鏡之舉樂,棄張鏡之稱悲,一鏡之言,取捨有異,此自難二也。陸家令止云『多歷年所』,恐非事證;雖復累稔所用,意常未安。近亦常經以此問外,由來立意,謂猶應有慕悼之言。張豈不知舉樂為大,稱悲事小;所以用小而忽大,良亦有以。至如元正六佾,事為國章;雖情或未安,而禮不可廢。鐃吹軍樂,比之亦然。書疏方之,事則成小,差可緣心。聲樂自外,書疏自內,樂自他,書自己。劉僕射之議,即情未安。可令諸賢更共詳衷。」
 司農卿明山賓、步兵校尉朱異議,稱「慕悼之解,宜終服月」。於是令付典書遵用,以為永準。



 七年十一月,貴嬪有疾,太子還永福省,朝夕侍疾,衣不解帶。及薨,步從喪還宮,至殯,水漿不入口,每哭輒慟絕。高祖遣中書舍人顧協宣旨曰:「毀不滅性,聖人之制。《禮》,不勝喪比於不孝。有我在,那得自毀如此!可即彊進飲食。」太子奉敕,乃進數合。自是至葬,日進麥粥一升。高祖又敕曰:「聞汝所進過少,轉就羸瘵。我比更無餘病,正為汝如此,胸中亦圮塞成疾。故應強加饘粥,不使我恆爾懸心。」雖屢奉敕勸逼,日止一溢,不嘗菜果之味。體素壯,腰帶十圍,至是減削
 過半。每入朝,士庶見者莫不下泣。



 太子自加元服,高祖便使省萬機,內外百司,奏事者填塞於前。太子明於庶事,纖毫必曉,每所奏有謬誤及巧妄,皆即就辯析,示其可否,徐令改正,未嘗彈糾一人。平斷法獄,多所全宥,天下皆稱仁。



 性寬和容眾,喜慍不形於色。引納才學之士,賞愛無倦。恆自討論篇籍,或與學士商榷古今;閒則繼以文章著述,率以為常。于時東宮有書幾三萬卷,名才並集,文學之盛,晉、宋以來未之有也。



 性愛山水,於玄圃穿築,更立亭館,與朝士名素者遊其中。嘗泛舟後池,番禺侯軌盛稱「此中宜奏女樂。」太子不答,詠左思《招隱詩》
 曰:「何必絲與竹,山水有清音。」侯慚而止。出宮二十餘年,不畜聲樂。少時,敕賜太樂女妓一部,略非所好。



 普通中,大軍北討,京師穀貴,太子因命菲衣減膳,改常饌為小食。每霖雨積雪,遣腹心左右,周行閭巷,視貧困家,有流離道路,密加振賜。又出主衣綿帛,多作襦褲,冬月以施貧凍。若死亡無可以斂者,為備棺槥。每聞遠近百姓賦役勤苦,輒斂容色。常以戶口未實,重於勞擾。



 吳興郡屢以水災失收,有上言當漕大瀆以瀉浙江。中大通二年春,詔遣前交州刺史王弁假節,發吳郡、吳興、義興三郡民丁就役。太子上疏曰:「伏聞當發王弁等上東三郡民
 丁,開漕溝渠,導泄震澤,使吳興一境,無復水災,誠矜恤之至仁,經略之遠旨。暫勞永逸,必獲後利。未萌難睹,竊有愚懷。所聞吳興累年失收,民頗流移。吳郡十城,亦不全熟。唯義興去秋有稔,復非常役之民。即日東境穀稼猶貴,劫盜屢起,在所有司,不皆聞奏。今征戍未歸,彊丁疏少,此雖小舉,竊恐難合,吏一呼門,動為民蠹。又出丁之處,遠近不一,比得齊集,已妨蠶農。去年稱為豊歲,公私未能足食;如復今茲失業,慮恐為弊更深。且草竊多伺候民間虛實,若善人從役,則抄盜彌增,吳興未受其益,內地已罹其弊。不審可得權停此功,待優實以不?聖
 心垂矜黎庶,神量久已有在。臣意見庸淺,不識事宜,茍有愚心,願得上啟。」高祖優詔以喻焉。



 太子孝謹天至,每入朝,未五鼓便守城門開。東宮雖燕居內殿,一坐一起,恒向西南面臺。宿被召當入,危坐達旦。



 三年三月,寢疾。恐貽高祖憂,敕參問,輒自力手書啟。及稍篤,左右欲啟聞,猶不許,曰「云何令至尊知我如此惡」,因便嗚咽。四月乙巳薨,時年三十一。高祖幸東宮,臨哭盡哀。詔斂以袞冕。謚曰昭明。五月庚寅,葬安寧陵。詔司徒左長史王筠為哀冊文曰:蜃輅俄軒,龍驂跼步;羽翿前驅,雲旂北御。皇帝哀繼明之寢耀,痛嗣德之殂芳;御武帳而悽慟,臨
 甲觀而增傷。式稽令典,載揚鴻烈;詔撰德於旌旒,永傳徽於舞綴。其辭曰:式載明兩,實惟少陽;既稱上嗣,且曰元良。儀天比峻,儷景騰光;奏祀延福,守器傳芳。睿哲膺期,旦暮斯在;外弘莊肅,內含和愷。識洞機深,量苞瀛海;立德不器,至功弗宰。寬綽居心,溫恭成性,循時孝友,率由嚴敬。咸有種德,惠和齊聖;三善遞宣,萬國同慶。



 軒緯掩精,陰犧弛極;纏哀在疚,殷憂銜恤。孺泣無時,蔬饘不溢;禫遵踰月,哀號未畢。實惟監撫,亦嗣郊禋;問安肅肅,視膳恂恂。金華玉璪,玄駟班輪;隆家幹國,主祭安民。光奉成務,萬機是理;矜慎庶獄,勤恤關市。誠存隱惻,容無
 慍喜;殷勤博施,綢繆恩紀。



 爰初敬業,離經斷句;奠爵崇師,卑躬待傅。寧資導習,匪勞審諭;博約是司,時敏斯務。辨究空微,思探幾賾;馳神圖緯,研精爻畫。沈吟典禮,優遊方冊;饜飫膏腴,含咀肴核。括囊流略,包舉藝文;遍該緗素,殫極丘墳。勣帙充積,儒墨區分;瞻河闡訓,望魯揚芬。吟詠性靈,豈惟薄伎;屬詞婉約,緣情綺靡。字無點竄,筆不停紙;壯思泉流,清章雲委。



 總覽時才,網羅英茂;學窮優洽,辭歸繁富。或擅談叢,或稱文囿;四友推德,七子慚秀。望苑招賢,華池愛客;託乘同舟,連輿接席。摛文手炎藻,飛紵泛幹;恩隆置醴,賞逾賜璧。徽風遐被,盛業日新;
 仁器非重,德輶易遵。澤流兆庶,福降百神;四方慕義,天下歸仁。



 雲物告徵,祲沴褰象;星霾恒耀,山頹朽壤。靈儀上賓,德音長往;具僚無蔭,諮承安仰。嗚呼哀哉!



 皇情悼愍,切心纏痛;胤嗣長號,跗萼增慟。慕結親遊,悲動氓眾;憂若殄邦,懼同折棟。嗚呼哀哉!



 首夏司開,麥秋紀節;容衛徒警,菁華委絕。書幌空張,談筵罷設;虛饋食蒙饛,孤燈翳翳。嗚呼哀哉!



 簡辰請日,筮合龜貞。幽埏夙啟,玄宮獻成。武校齊列,文物增明。昔遊漳滏,賓從無聲;今歸郊郭,徒御相驚。嗚呼哀哉!



 背絳闕以遠徂,轥青門而徐轉;指馳道而詎前,望國都而不踐。陵脩阪之威夷,溯平原之
 悠緬;驥蹀足以酸嘶,挽悽鏘而流泫。嗚呼哀哉!



 混哀音於簫籟,變愁容於天日;雖夏木之森陰,返寒林之蕭瑟。既將反而復疑,如有求而遂失;謂天地其無心,遽永潛於容質。嗚呼哀哉!



 即玄宮之冥漠,安神寢之清颭;傳聲華於懋典,觀德業於徽謚。懸忠貞於日月,播鴻名於天地;惟小臣之紀言,實含毫而無媿。嗚呼哀哉!



 太子仁德素著,及薨,朝野惋愕。京師男女,奔走宮門,號泣滿路。四方氓庶,及疆徼之民,聞喪皆慟哭。所著文集二十卷;又撰古今典誥文言,為《正序》十卷;五言詩之善者,為《文章英華》二十卷;《文選》三十卷。



 哀太子大器,字仁宗,太宗嫡長子也。普通四年五月丁酉生。中大通四年,封宣城郡王,食邑二千戶。尋為侍中、中衛將軍,給鼓吹一部。大同四年,授使持節、都督揚、徐二州諸軍事、中軍大將軍、揚州刺史,侍中如故。



 太清二年十月,侯景寇京邑,敕太子為臺內大都督。三年五月,太宗即位。六月丁亥,立為皇太子。大寶二年八月,賊景廢太宗,將害太子,時賊黨稱景命召太子,太子方講《老子》,將欲下床,而刑人掩至。太子顏色不變,徐曰:「久知此事,嗟其晚耳。」刑者欲以衣帶絞之。太子曰:「此不能見殺。」乃指繫帳竿下繩,命取絞之而絕,時年二十八。



 太子性
 寬和,兼神用端嶷,在于賊手,每不屈意。初,侯景西上,攜太子同行,及其敗歸,部伍不復整肅,太子所乘船居後,不及賊眾,左右心腹並勸因此入北。太子曰:「家國喪敗,志不圖生;主上蒙塵,寧忍違離?吾今逃匿,乃是叛父,非謂避賊。」便涕泗鳴咽,令即前進。賊以太子有器度,每常憚之,恐為後患,故先及禍。承聖元年四月,追謚哀太子。



 愍懷太子方矩,字德規,世祖第四子也。初封南安縣侯,隨世祖在荊鎮。太清初,為使持節、督湘、郢、桂、寧、成、合、羅七州諸軍事、鎮南將軍、湘州刺史。尋徵為侍中、中衛將軍,給鼓吹一部。世祖承制,拜王太子,改名元良。承聖元
 年十一月丙子,立為皇太子。及西魏師陷荊城,太子與世祖同為魏人所害。



 太子聰穎,頗有世祖風,而兇暴猜忌。敬帝承制,追謚愍懷太子。



 陳吏部書姚察曰:孟軻有言:「雞鳴而起,孳孳為善者,舜之徒也。」若乃布衣韋帶之士,在於畎畝之中,終日為之,其利亦已博矣。況乎處重明之位,居正體之尊,克念無怠,烝烝以孝。大舜之德,其何遠之有哉!



\end{pinyinscope}