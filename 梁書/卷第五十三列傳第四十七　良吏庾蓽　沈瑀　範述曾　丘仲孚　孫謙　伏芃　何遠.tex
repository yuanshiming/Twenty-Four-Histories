\article{卷第五十三列傳第四十七 良吏庾蓽 沈瑀 範述曾 丘仲孚 孫謙 伏芃 何遠}

\begin{pinyinscope}

 昔漢宣帝以為「政平訟理,其惟良二千石乎!」前史亦云:「今之郡守,古之諸侯也。」故長吏之職,號為親民,是以導德齊禮,移風易俗,咸必由之。齊末昏亂,政移群小,賦調雲起,徭役無度。守宰多倚附權門,互長貪虐,掊克聚斂,
 侵愁細民,天下搖動,無所厝其手足。高祖在田,知民疾苦,及梁臺建,仍下寬大之書,昏時雜調,咸悉除省,於是四海之內,始得息肩。逮踐皇極,躬覽庶事,日昃聽政,求民之瘼。乃命輶軒以省方俗,置肺石以達窮民,務加隱恤,舒其急病。元年,始去人貲,計丁為布;身服浣濯之衣,御府無文飾,宮掖不過綾彩,無珠璣錦繡;太官撤牢饌,每日膳菜蔬,飲酒不過三盞——以儉先海內。每選長吏,務簡廉平,皆召見御前,親勖治道。始擢尚書殿中郎到溉為建安內史,左民侍郎劉鬷為晉安太守,溉等居官,並以廉潔著。又著令:小縣有能,遷為大縣;大縣有能,遷為
 二千石。於是山陰令丘仲孚治有異績,以為長沙內史;武康令何遠清公,以為宣城太守。剖符為吏者,往往承風焉。若新野庾蓽諸任職者,以經術潤飾吏政,或所居流惠,或去後見思,蓋後來之良吏也。綴為《良吏篇》云。



 庾蓽,字休野,新野人也。父深之,宋應州刺史。蓽年十歲,遭父憂,居喪毀瘠,為州黨所稱。弱冠,為州迎主簿,舉秀才,累遷安西主簿、尚書殿中郎、驃騎功曹史。博涉群書,有口辯。齊永明中,與魏和親,以蓽兼散騎常侍報使,還拜散騎侍郎,知東宮管記事。



 鬱林王即位廢,掌中書詔誥,出為荊州別駕。仍遷西中郎諮議參軍,復為州別駕。
 前後綱紀,皆致富饒。蓽再為之,清身率下,杜絕請託,布被蔬食,妻子不免飢寒。明帝聞而嘉焉,手敕褒美,州里榮之。遷司徒諮議參軍、通直散騎常侍。高祖平京邑,霸府建,引為驃騎功曹參軍,遷尚書左丞。出為輔國長史、會稽郡丞、行郡府事。時承凋弊之後,百姓凶荒,所在穀貴,米至數千,民多流散,蓽撫循甚有治理。唯守公祿,清節逾厲,至有經日不舉火。太守、襄陽王聞而饋之,蓽謝不受。天監元年,卒,停屍無以殮,柩不能歸。高祖聞之,詔賜絹百匹、米五十斛。



 初,蓽為西楚望族,早歷顯官,鄉人樂藹有乾用,素與蓽不平,互相陵競。藹事齊豫章王嶷,
 嶷薨,藹仕不得志,自步兵校尉求助戍歸荊州,時蓽為州別駕,益忽藹。及高祖踐阼,藹以西朝勳為御史中丞,蓽始得會稽行事,既恥之矣;會職事微有譴,高祖以藹其鄉人也,使宣旨誨之,蓽大憤,故發病卒。



 沈瑀,字伯瑜,吳興武康人也。叔父昶,事宋建平王景素,景素謀反,昶先去之;及敗,坐繫獄,瑀詣臺陳請,得免罪,由是知名。起家州從事、奉朝請。嘗詣齊尚書右丞殷濔,濔與語及政事,甚器之,謂曰:「觀卿才幹,當居吾此職。」司徒、竟陵王子良聞瑀名,引為府參軍,領揚州部傳從事。時建康令沈徽孚恃勢陵瑀,瑀以法繩之,眾憚其彊。子
 良甚相知賞,雖家事皆以委瑀。子良薨,瑀復事刺史、始安王遙光。嘗被使上民丁,速而無怨。遙光謂同使曰:「爾何不學沈瑀所為?」乃令專知州獄事。湖熟縣方山埭高峻,冬月,公私行侶以為艱難,明帝使瑀行治之。瑀乃開四洪,斷行客就作,三日立辦。揚州書佐私行,詐稱州使,不肯就作,瑀鞭之三十。書佐歸訴遙光,遙光曰:「沈瑀必不枉鞭汝。」覆之,果有詐。明帝復使瑀築赤山塘,所費減材官所量數十萬,帝益善之。永泰元年,為建德令,教民一丁種十五株桑、四株柿及梨栗,女丁半之,人咸歡悅,頃之成林。



 去官還京師,兼行選曹郎。隨陳伯之軍至江
 州,會義師圍郢城,瑀說伯之迎高祖。伯之泣曰:「餘子在都,不得出城,不能不愛之。」瑀曰:「不然,人情匈匈,皆思改計,若不早圖,眾散難合。」伯之遂舉眾降,瑀從在高祖軍中。



 初,瑀在竟陵王家,素與范雲善。齊末,嘗就雲宿,夢坐屋梁柱上,仰見天中有字曰「范氏宅」。至是,瑀為高祖說之。高祖曰:「雲得不死,此夢可驗。」及高祖即位,雲深薦瑀,自暨陽令擢兼尚書右丞。時天下初定,陳伯之表瑀催督運轉,軍國獲濟,高祖以為能。遷尚書駕部郎,兼右丞如故。瑀薦族人沈僧隆、僧照有吏乾,高祖並納之。



 以母憂去職,起為振武將軍、餘姚令。縣大姓虞氏千餘家,請謁
 如市,前後令長莫能絕。自瑀到,非訟所通,其有至者,悉立之階下,以法繩之。縣南又有豪族數百家,子弟縱橫,遞相庇蔭,厚自封植,百姓甚患之。瑀召其老者為石頭倉監,少者補縣僮,皆號泣道路,自是權右屏跡。瑀初至,富吏皆鮮衣美服,以自彰別。瑀怒曰:「汝等下縣吏,何自擬貴人耶?」悉使著芒矰粗布,侍立終日,足有蹉跌,輒加榜棰。瑀微時,嘗自至此鬻瓦器,為富人所辱,故因以報焉,由是士庶駭怨。然瑀廉白自守,故得遂行其志。



 後王師北伐,徵瑀為建威將軍,督運漕,尋兼都水使者。頃之,遷少府卿。出為安南長史、尋陽太守。江州刺史曹景宗
 疾篤,瑀行府州事。景宗卒,仍為信威蕭穎達長史,太守如故。瑀性屈彊,每忤穎達,穎達銜之。天監八年,因入諮事,辭又激厲,穎達作色曰:「朝廷用君作行事耶?」瑀出,謂人曰:「我死而後已,終不能傾側面從。」是日,於路為盜所殺,時年五十九,多以為穎達害焉。子續累訟之,遇穎達亦尋卒,事遂不窮竟。續乃布衣蔬食終其身。



 范述曾,字子玄,吳郡錢唐人也。幼好學,從餘杭呂道惠受《五經》,略通章句。道惠學徒常有百數,獨稱述曾曰:「此子必為王者師。」齊文惠太子、竟陵文宣王幼時,高帝引述曾為之師友。起家為宋晉熙王國侍郎。齊初,至南郡
 王國郎中令,遷尚書主客郎、太子步兵校尉,帶開陽令。述曾為人謇諤,在宮多所諫爭,太子雖不能全用,然亦弗之罪也。竟陵王深相器重,號為「周舍」。時太子左衛率沈約亦以述曾方汲黯。以父母年老,乞還就養,乃拜中散大夫。



 明帝即位,除游擊將軍,出為永嘉太守。為政清平,不尚威猛,民俗便之。所部橫陽縣,山谷險峻,為逋逃所聚,前後二千石討捕莫能息。述曾下車,開示恩信,凡諸凶黨,涘負而出,編戶屬籍者二百餘家。自是商賈流通,居民安業。在郡勵志清白,不受饋遺,明帝聞甚嘉之,下詔褒美焉。徵為游擊將軍。郡送故舊錢二十餘萬,述曾
 一無所受。始之郡,不將家屬;及還,吏無荷擔者。民無老少,皆出拜辭,號哭聞于數十里。



 東昏時,拜中散大夫,還鄉里。高祖踐阼,乃輕舟出詣闕,仍辭還東。高祖詔曰:「中散大夫范述曾,昔在齊世,忠直奉主,往蒞永嘉,治身廉約,宜加禮秩,以厲清操。可太中大夫,賜絹二十匹。」述曾生平得奉祿,皆以分施。及老,遂壁立無所資。以天監八年卒,時年七十九。注《易文言》,著雜詩賦數十篇。



 丘仲孚,字公信,吳興烏程人也。少好學,從祖靈鞠有人倫之鑒,常稱為千里駒也。齊永明初,選為國子生,舉高第,未調,還鄉里。家貧,無以自資,乃結群盜,為之計畫,劫
 掠三吳。仲孚聰明有智略,群盜畏而服之,所行皆果,故亦不發。太守徐嗣召補主簿,歷揚州從事、太學博士、于湖令,有能名。太守呂文顯當時倖臣,陵詆屬縣,仲孚獨不為之屈。以父喪去職。



 明帝即位,起為烈武將軍、曲阿令。值會稽太守王敬則舉兵反,乘朝廷不備,反問始至,而前鋒已屆曲阿。仲孚謂吏民曰:「賊乘勝雖銳,而烏合易離。今若收船艦,鑿長崗埭,洩瀆水以阻其路,得留數日,臺軍必至,則大事濟矣。」敬則軍至,值瀆涸,果頓兵不得進,遂敗散。仲孚以距守有功,遷山陰令,居職甚有聲稱,百姓為之謠曰:「二傅沈劉,不如一丘。」前世傅琰父子、
 沈憲、劉玄明,相繼宰山陰,並有政績,言仲孚皆過之也。



 齊末政亂,頗有贓賄,為有司所舉,將收之,仲孚竊逃,徑還京師詣闕,會赦,得不治。高祖踐阼,復為山陰令。仲孚長於撥煩,善適權變,吏民敬服,號稱神明,治為天下第一。



 超遷車騎長史、長沙內史,視事未期,徵為尚書右丞,遷左丞,仍擢為衛尉卿,恩任甚厚。初起雙闕,以仲孚領大匠。事畢,出為安西長史、南郡太守。遷雲麾長史、江夏太守,行郢州州府事,遭母憂,起攝職。坐事除名,復起為司空參軍。俄遷豫章內史,在郡更勵清節。頃之,卒,時年四十八。詔曰:「豫章內史丘仲孚,重試大邦,責以後效,非
 直悔吝云亡,實亦政績克舉。不幸殞喪,良以傷惻。可贈給事黃門侍郎。」仲孚喪將還,豫章老幼號哭攀送,車輪不得前。



 仲孚為左丞,撰《皇典》二十卷、《南宮故事》百卷,又撰《尚書具事雜儀》,行於世焉。



 孫謙,字長遜,東莞莒人也。少為親人趙伯符所知。謙年十七,伯符為豫州刺史,引為左軍行參軍,以治乾稱。父憂去職,客居歷陽,躬耕以養弟妹,鄉里稱其敦睦。宋江夏王義恭聞之,引為行參軍,歷仕大司馬、太宰二府。出為句容令,清慎彊記,縣人號為神明。



 泰始初,事建安王休仁,休仁以為司徒參軍,言之明帝,擢為明威將軍、巴
 東、建平二郡太守。郡居三峽,恒以威力鎮之。謙將述職,敕募千人自隨。謙曰:「蠻夷不賓,蓋待之失節耳。何煩兵役,以為國費。」固辭不受。至郡,布恩惠之化,蠻獠懷之,競餉金寶,謙慰喻而遣,一無所納。及掠得生口,皆放還家。俸秩出吏民者,悉原除之。郡境翕然,威信大著。視事三年,徵還為撫軍中兵參軍。元徽初,遷梁州刺史,辭不赴職,遷越騎校尉、征北司馬府主簿。建平王將稱兵,患謙彊直,託事遣使京師,然後作亂。及建平誅,遷左軍將軍。



 齊初,為寧朔將軍、錢唐令,治煩以簡,獄無繫囚。及去官,百姓以謙在職不受餉遺,追載縑帛以送之,謙卻不受。
 每去官,輒無私宅,常借官空車廄居焉。永明初,為冠軍長史、江夏太守,坐被代輒去郡,繫尚方。頃之,免為中散大夫。明帝將廢立,欲引謙為心膂,使兼衛尉,給甲仗百人,謙不願處際會,輒散甲士,帝雖不罪,而弗復任焉。出為南中郎司馬。東昏永元元年,遷囗囗大夫。



 天監六年,出為輔國將軍、零陵太守,已衰老,猶彊力為政,吏民安之。先是,郡多虎暴,謙至絕迹。及去官之夜,虎即害居民。謙為郡縣,常勤勸課農桑,務盡地利,收入常多於鄰境。九年,以年老,徵為光祿大夫。既至,高祖嘉其清潔,甚禮異焉。每朝見,猶請劇職自效。高祖笑曰:「朕使卿智,不使
 卿力。」十四年,詔曰:「光祿大夫孫謙,清慎有聞,白首不怠,高年舊齒,宜加優秩。可給親信二十人,並給扶。」



 謙自少及老,歷二縣五郡,所在廉潔。居身儉素,床施蘧除屏風,冬則布被莞席,夏日無幬帳,而夜臥未嘗有蚊蚋,人多異焉。年逾九十,彊壯如五十者,每朝會,輒先眾到公門。力於仁義,行己過人甚遠。從兄靈慶常病寄於謙,謙出行還問起居。靈慶曰:「向飲冷熱不調,即時猶渴。」謙退遣其妻。有彭城劉融者,行乞疾篤無所歸,友人輿送謙舍,謙開廳事以待之。及融死,以禮殯葬之。眾咸服其行義。十五年,卒官,時年九十二。詔賻錢三萬、布五十匹。高祖為
 舉哀,甚悼惜之。



 謙從子廉,便辟巧宦。齊時已歷大縣,尚書右丞。天監初,沈約、范雲當朝用事,廉傾意奉之。及中書舍人黃睦之等,亦尤所結附。凡貴要每食,廉必日進滋旨,皆手自煎調,不辭勤劇,遂得為列卿、御史中丞、晉陵、吳興太守。時廣陵高爽有險薄才,客於廉,廉委以文記,爽嘗有求不稱意,乃為屐謎以喻廉曰:「刺鼻不知嚏,蹋面不知瞋,齧齒作步數,持此得勝人。」譏其不計恥辱,以此取名位也。



 伏暅,字玄耀,曼容之子也。幼傳父業,能言玄理,與樂安任昉、彭城劉曼俱知名。起家齊奉朝請,仍兼太學博士,
 尋除東陽郡丞,秩滿為鄞令。時曼容已致仕,故頻以外職處暅,令其得養焉。齊末,始為尚書都官郎,仍為衛軍記室參軍。



 高祖踐阼,遷國子博士,父憂去職。服闋,為車騎諮議參軍,累遷司空長史、中書侍郎、前軍將軍、兼《五經》博士,與吏部尚書徐勉、中書侍郎周捨,總知五禮事。出為永陽內史,在郡清潔,治務安靜。郡民何貞秀等一百五十四人詣州言狀,湘州刺史以聞。詔勘有十五事為吏民所懷,高祖善之,徵為新安太守。在郡清恪,如永陽時。民賦稅不登者,輒以太守田米助之。郡多麻苧,家人乃至無以為繩,其厲志如此。屬縣始新、遂安、海寧,並
 同時生為立祠。



 徵為國子博士,領長水校尉。時始興內史何遠累著清績,高祖詔擢為黃門侍郎,俄遷信武將軍、監吳郡。暅自以名輩素在遠前,為吏俱稱廉白,遠累見擢,暅遷階而已,意望不滿,多託疾居家。尋求假到東陽迎妹喪,因留會稽築宅,自表解,高祖詔以為豫章內史,暅乃出拜。治書侍御史虞嚼奏曰:臣聞失忠與信,一心之道以虧;貌是情非,兩觀之誅宜及。未有陵犯名教,要冒君親,而可緯俗經邦者也。風聞豫章內史伏暅,去歲啟假,以迎妹喪為解,因停會稽不去。入東之始,貨宅賣車。以此而推,則是本無還意。暅歷典二邦,少免貪濁,此
 自為政之本,豈得稱功。常謂人才品望,居何遠之右,而遠以清公見擢,名位轉隆,暅深誹怨,形於辭色,興居歎吒,寤寐失圖。天高聽卑,無私不照。去年十二月二十一日詔曰:「國子博士、領長水校尉伏暅,為政廉平,宜加將養,勿使恚望,致虧士風。可豫章內史。」豈有人臣奉如此之詔,而不亡魂破膽,歸罪有司;擢髮抽腸,少自論謝?而循奉慠然,了無異色。暅識見所到,足達此旨,而冒寵不辭,吝斯茍得,故以士流解體,行路沸騰,辯跡求心,無一可恕。竊以暅踉蹡落魄,三十餘年,皇運勃興,咸與維始,除舊布新,濯之江、漢,一紀之間,三世隆顯。曾不能少懷
 感激,仰答萬分,反覆拙謀,成茲巧罪,不忠不敬,於斯已及。請以暅大不敬論。以事詳法,應棄市刑,輒收所近獄洗結,以法從事。如法所稱,暅即主。



 臣謹案:豫章內史臣伏暅,含疵表行,藉悖成心,語默一違,資敬兼盡。幸屬昌時,擢以不次。溪壑可盈,志欲無滿。要君東走,豈曰止足之歸;負志解巾,異乎激處之致。甘此脂膏,孰非荼苦;佩茲龜組,豈殊縲紲。宜明風憲,肅正簡書。臣等參議,請以見事免暅所居官,凡諸位任,一皆削除。



 有詔勿治,暅遂得就郡。



 視事三年,徵為給事黃門侍郎,領國子博士,未及起。普通元年,卒於郡,時年五十九。尚書右僕射徐勉
 為之墓志,其一章曰:「東區南服,愛結民胥,相望伏闕,繼軌奏書。或臥其轍,或扳其車,或圖其像,或式其閭。思耿借寇,曷以尚諸。」



 初,暅父曼容與樂安任瑤皆匿於齊太尉王儉,瑤子昉及暅並見知。頃之,昉才遇稍盛,齊末,昉已為司徒右長史,暅猶滯於參軍事;及其終也,名位略相侔。暅性儉素,車服麤惡,外雖退靜,內不免心競,故見譏於時。能推薦後來,常若不及,少年士子,或以此依之。



 何遠,字義方,東海郯人也。父慧炬,齊尚書郎。遠釋褐江夏王國侍郎,轉奉朝請。永元中,江夏王寶玄於京口為護軍將軍崔慧景所奉,入圍宮城,遠豫其事。事敗,乃亡
 抵長沙宣武王,王深保匿焉。遠求得桂陽王融保藏之,既而發覺,收捕者至,遠逾垣以免;融及遠家人皆見執,融遂遇禍,遠家屬繫尚方。遠亡渡江,使其故人高江產共聚眾,欲迎高祖義師,東昏黨聞之,使捕遠等,眾復潰散。遠因降魏,入壽陽,見刺史王肅,欲同義舉,肅不能用,乃求迎高祖,肅許之。遣兵援送,得達高祖。高祖見遠,謂張弘策曰:「何遠美丈夫,而能破家報舊德,未易及也。」板輔國將軍,隨軍東下,既破朱雀軍,以為建康令。高祖踐阼,為步兵校尉,以奉迎勳封廣興男,邑三百戶。遷建武將軍、後軍鄱陽王恢錄事參軍。遠與恢素善,在府盡其
 志力,知無不為,恢亦推心仗之,恩寄甚密。



 頃之,遷武昌太守。遠本倜儻,尚輕俠,至是乃折節為吏,杜絕交遊,饋遺秋毫無所受。武昌俗皆汲江水,盛夏遠患水溫,每以錢買民井寒水;不取錢者,則摙水還之。其佗事率多如此。跡雖似偽,而能委曲用意焉。車服尤弊素,器物無銅漆。江左多水族,甚賤,遠每食不過乾魚數片而已。然性剛嚴,吏民多以細事受鞭罰者,遂為人所訟,徵下廷尉,被劾數十條。當時士大夫坐法,皆不受立,遠度己無贓,就立三七日不款,猶以私藏禁仗除名。



 後起為鎮南將軍、武康令。愈厲廉節,除淫祀,正身率職,民甚稱之。太守
 王彬巡屬縣,諸縣盛供帳以待焉,至武康,遠獨設糗水而已。彬去,遠送至境,進斗酒雙鵝為別。彬戲曰:「卿禮有過陸納,將不為古人所笑乎?」高祖聞其能,擢為宣城太守。自縣為近畿大郡,近代未之有也。郡經寇抄,遠盡心綏理,復著名迹。期年,遷樹功將軍、始興內史。時泉陵侯淵朗為桂州,緣道剽掠,入始興界,草木無所犯。



 遠在官,好開途巷,修葺牆屋,民居市里,城隍廄庫,所過若營家焉。田秩俸錢,並無所取,歲暮,擇民尤窮者,充其租調,以此為常。然其聽訟猶人,不能過絕,而性果斷,民不敢非,畏而惜之。所至皆生為立祠,表言治狀,高祖每優詔答
 焉。天監十六年,詔曰:「何遠前在武康,已著廉平;復蒞二邦,彌盡清白。政先治道,惠留民愛,雖古之良二千石,無以過也。宜升內榮,以顯外績。可給事黃門侍郎。」遠即還,仍為仁威長史。頃之,出為信武將軍,監吳郡。在吳頗有酒失,遷東陽太守。遠處職,疾彊富如仇讎,視貧細如子弟,特為豪右所畏憚。在東陽歲餘,復為受罰者所謗,坐免歸。



 遠耿介無私曲,居人間,絕請謁,不造詣。與貴賤書疏,抗禮如一。其所會遇,未嘗以顏色下人,以此多為俗士所惡。其清公實為天下第一。居數郡,見可欲終不變其心,妻子飢寒,如下貧者。及去東陽歸家,經年歲口不
 言榮辱,士類益以此多之。其輕財好義,周人之急,言不虛妄,蓋天性也。每戲語人云:「卿能得我一妄語,則謝卿以一縑。」眾共伺之,不能記也。後復起為征西諮議參軍、中撫司馬。普通二年,卒,時年五十二。高祖厚贈賜之。



 陳吏部尚書姚察曰:前史有循吏,何哉?世使然也。漢武役繁姦起,循平不能,故有苛酷誅戮以勝之,亦多怨濫矣。梁興,破觚為圓,斲雕為樸,教民以孝悌,勸之以農桑,於是桀黠化為由餘,輕薄變為忠厚。淳風已洽,民自知禁。堯舜之民,比屋可封,信矣。若夫酷吏,於梁無取焉。



\end{pinyinscope}