\article{卷第五十列傳第四十四 文學下劉峻 劉沼 謝幾卿 劉勰 王籍 何思澄 劉杳 謝徵 臧嚴 伏挺 庾仲容 陸雲公 任孝恭 顏協}

\begin{pinyinscope}

 劉峻,字孝標,平原平原人。父珽,宋始興內史。峻生期月,母攜還鄉里。宋泰始初,青州陷魏,峻年八歲,為人所略
 至中山,中山富人劉實愍峻,以束帛贖之,教以書學。魏人聞其江南有戚屬,更徙之桑乾。峻好學,家貧,寄人廡下,自課讀書,常燎麻炬,從夕達旦,時或昏睡,爇其髮,既覺復讀,終夜不寐,其精力如此。齊永明中,從桑乾得還,自謂所見不博,更求異書,聞京師有者,必往祈借,清河崔慰祖謂之「書淫」。時竟陵王子良博招學士,峻因人求為子良國職,吏部尚書徐孝嗣抑而不許,用為南海王侍郎,不就。至明帝時,蕭遙欣為豫州,為府刑獄,禮遇甚厚。遙欣尋卒,久之不調。天監初,召入西省,與學士賀蹤典校秘書。峻兄孝慶,時為青州刺史,峻請假省之,坐私
 載禁物,為有司所奏,免官。安成王秀好峻學,及遷荊州,引為戶曹參軍,給其書籍,使抄錄事類,名曰《類苑》。未及成,復以疾去,因遊東陽紫巖山,築室居焉。為《山棲志》,其文甚美。



 高祖招文學之士,有高才者,多被引進,擢以不次。峻率性而動,不能隨眾沉浮,高祖頗嫌之,故不任用。乃著《辨命論》以寄其懷曰:主上嘗與諸名賢言及管輅,歎其有奇才而位不達。時有在赤墀之下,預聞斯議,歸以告餘。餘謂士之窮通,無非命也。故謹述天旨,因言其略云。



 臣觀管輅天才英偉,珪璋特秀,實海內之髦傑,豈日者卜祝之流。而官止少府丞,年終四十八,天之報施,
 何其寡歟?然則高才而無貴仕,饕餮而居大位,自古所歎,焉獨公明而已哉?故性命之道,窮通之數,夭閼紛綸,莫知其辨。仲任蔽其源,子長闡其惑。至於鶡冠甕牖,必以懸天有期;鼎貴高門,則曰唯人所召。譊々言雚咋,異端俱起。蕭遠論其本而不暢其流,子玄語其流而未詳其本。嘗試言之曰:夫道生萬物,則謂之道;生而無主,謂之自然。自然者,物見其然,不知所以然;同焉皆得,不知所以得。鼓動陶鑄而不為功,庶類混成而非其力;生之無亭毒之心,死之豈虔劉之志;墜之淵泉非其怒,昇之霄漢非其悅。蕩乎大乎,萬寶以之化;確乎純乎,一作而不
 易。化而不易,則謂之命。命也者,自天之命也。定於冥兆,終然不變。鬼神莫能預,聖哲不能謀;觸山之力無以抗,倒日之誠弗能感;短則不可緩之於寸陰,長則不可急之於箭漏;至德未能踰,上智所不免。是以放勛之代,浩浩襄陵;天乙之時,燋金流石。文公𧾷疐其尾,宣尼絕其糧;顏回敗其叢蘭,冉耕歌其芣苡;夷、叔斃淑媛之言,子輿困臧倉之訴。聖賢且猶若此,而況庸庸者乎!至乃伍員浮屍於江流,三閭沉骸於湘渚;賈大夫沮志於長沙,馮都尉皓髮於郎署;君山鴻漸,鎩羽儀於高雲;敬通鳳起,摧迅翮於風穴:此豈才不足而行有遺哉?



 近代有沛國
 劉、弟璡,並一時之秀士也。獻則關西孔子,通涉《六經》,循循善誘,服膺儒行。璡則志烈秋霜,心貞昆玉,亭亭高竦,不雜風塵。皆毓德於衡門,並馳聲於天地。而官有微於侍郎,位不登於執戟,相繼徂落,宗祀無饗。因斯兩賢,以言古則:昔之玉質金相,英髦秀達,皆擯斥於當年,韞奇才而莫用,候草木以共凋,與麋鹿而同死。膏塗平原,骨填川谷,湮滅而無聞者,豈可勝道哉!此則宰衡之與皂隸,容、彭之與殤子,猗頓之與黔婁,陽文之與敦洽,咸得之於自然,不假道於才智。故曰「死生有命,富貴在天」,其斯之謂矣。然命體周流,變化非一,或先號後笑,或
 始吉終凶,或不召自來,或因人以濟。交錯紛糾,循環倚伏。非可以一理征,非可以一途驗。而其道密微,寂寥忽慌,無形可以見,無聲可以聞。必御物以效靈,亦憑人而成象,譬天王之冕旒,任百官以司職。而惑者睹湯、武之龍躍,謂龕亂在神功;聞孔、墨之挺生,謂英睿擅奇響;視彭、韓之豹變,謂鷙猛致人爵;見張、桓之朱紱,謂明經拾青紫。豈知有力者運之而趨乎?故言而非命,有六蔽焉。餘請陳其梗概:夫靡顏膩理,哆噅頞,形之異也;朝秀辰終,龜鶴千歲,年之殊也;聞言如響,智昏菽麥,神之辨也。固知三者定乎造化,榮辱之境,獨曰由人。是知二五
 而未識於十,其蔽一也。龍犀日角,帝王之表;河目龜文,公侯之相。撫鏡知其將刑,壓紐顯其膺錄。星虹樞電,昭聖德之符;夜哭聚雲,鬱興王之瑞。皆兆發於前期,渙汗於後葉。若謂驅貔虎,奮尺劍,入紫微,升帝道;則未達窅冥之情,未測神明之數,其蔽二也。空桑之里,變成洪川;歷陽之都,化為魚鱉。楚師屠漢卒,睢河鯁其流;秦人坑趙士,沸聲若雷震。火炎昆岳,礫石與琬琰俱焚;嚴霜夜零,蕭艾與芝蘭共盡。雖游、夏之英才,伊、顏之殆庶,焉能抗之哉?其蔽三也。或曰,明月之珠,不能無牴;夏后之璜,不能無考。故亭伯死於縣長,長卿卒於園令,才非不傑
 也,主非不明也,而碎結綠之鴻輝,殘懸黎之夜色,抑尺之量有短哉?若然者,主父偃、公孫弘對策不升第,歷說而不入,牧豕淄原,見棄州部。設令忽如過隙,溘死霜露,其為詬恥,豈崔、馬之流乎?及至開東閣,列五鼎,電照風行,聲馳海外,寧前愚而後智,先非而終是?將榮悴有定數,天命有至極,而謬生妍蚩?其蔽四也。夫虎嘯風馳,龍興雲屬,故重華立而元、凱升,辛受生而飛廉進。然則天下善人少,惡人多;闇主眾,明君寡。而薰蕕不同器,梟鸞不接翼。是使渾沌、檮杌,踵武雲臺之上;仲容、庭堅,耕耘巖石之下。橫謂廢興在我,無繫於天,其蔽五也。彼戎狄
 者,人面獸心,宴安鴆毒,以誅殺為道德,以蒸報為仁義。雖大風立於青丘,鑿齒奮於華野,比其狼戾,曾何足踰。自金行不競,天地版蕩,左帶沸脣,乘間電發。遂覆瀍、洛,傾五都;居先王之桑梓,竊名號於中縣;與三皇競其氓黎,五帝角其區宇。種落繁熾,充牣神州。嗚呼!福善禍淫,徒虛言耳。豈非否泰相傾,盈縮遞運,而汩之以人?其蔽六也。



 然所謂命者,死生焉,貴賤焉,貧富焉,理亂焉,禍福焉,此十者天之所賦也。愚智善惡,此四者人之所行也。夫神非舜、禹,心異硃、均,才絓中庸,在於所習。是以素絲無恆,玄黃代起;鮑魚芳蘭,入而自變。故季路學於仲尼,
 厲風霜之節;楚穆謀於潘崇,成悖逆之禍。而商臣之惡,盛業光於後嗣;仲由之善,不能息其結纓。斯則邪正由於人,吉凶存乎命。或以鬼神害盈,皇天輔德。故宋公一言,法星三徙;殷帝自剪,千里來雲。善惡無征,未洽斯義。且于公高門以待封,嚴母掃墓以望喪。此君子所以自彊不息也。如使仁而無報,奚為修善立名乎?斯徑廷之辭也。夫聖人之言,顯而晦,微而婉,幽遠而難聞,河漢而不極。或立教以進庸惰,或言命以窮性靈。積善餘慶,立教也;鳳鳥不至,言命也。今以其片言辯其要趨,何異乎夕死之類而論春秋之變哉?且荊昭德音,丹雲不卷;周
 宣祈雨,珪璧斯罄。于叟種德,不逮勛、華之高;延年殘獷,未甚東陵之酷。為善一,為惡均,而禍福異其流,廢興殊其迹。蕩蕩上帝,豈如是乎?《詩》云:「風雨如晦,雞鳴不已。」故善人為善,焉有息哉?



 夫食稻梁,進芻豢,衣狐貉,襲冰紈,觀窈眇之奇儛,聽雲和之琴瑟,此生人之所急,非有求而為也。修道德,習仁義,敦孝悌,立忠貞,漸禮樂之腴潤,蹈先王之盛則,此君子之所急,非有求而為也。然而君子居正體道,樂天知命。明其無可奈何,識其不由智力。逝而不召,來而不距,生而不喜,死而不戚。瑤臺夏屋,不能悅其神;土室編蓬,未足憂其慮。不充詘於富貴,不遑
 遑於所欲。豈有史公、董相《不遇》之文乎?



 論成,中山劉沼致書以難之,凡再反,峻並為申析以答之。會沼卒,不見峻後報者,峻乃為書以序之曰:「劉侯既有斯難,值餘有天倫之戚,竟未之致也。尋而此君長逝,化為異物,緒言餘論,蘊而莫傳。或有自其家得而示餘者,悲其音徽未沫,而其人已亡,青簡尚新,而宿草將列,泫然不知涕之無從。雖隙駟不留,尺波電謝;而秋菊春蘭,英華靡絕。故存其梗概,更酬其旨。若使墨翟之言無爽,宣室之談有征。冀東平之樹,望咸陽而西靡;蓋山之泉,聞弦歌而赴節。但懸劍空壟,有恨如何!」其論文多不載。



 峻又嘗為《自
 序》,其略曰:「余自比馮敬通,而有同之者三,異之者四。何則?敬通雄才冠世,志剛金石;餘雖不及之,而節亮慷慨,此一同也。敬通值中興明君,而終不試用;餘逢命世英主,亦擯斥當年,此二同也。敬通有忌妻,至於身操井臼;餘有悍室,亦令家道感軻,此三同也。敬通當更始之世,手握兵符,躍馬食肉;餘自少迄長,戚戚無懽,此一異也。敬通有一子仲文,官成名立;餘禍同伯道,永無血胤,此二異也。敬通膂力方剛,老而益壯;餘有犬馬之疾,溘死無時,此三異也。敬通雖芝殘蕙焚,終填溝壑,而為名賢所慕,其風流郁烈芬芳,久而彌盛;餘聲塵寂漠,世不吾
 知,魂魄一去,將同秋草,此四異也。所以自力為敘,遺之好事云。」峻居東陽,吳、會人士多從其學。普通二年,卒,時年六十。門人謚曰玄靖先生。



 劉沼,字明信,中山魏昌人。六代祖輿,晉驃騎將軍。沼幼善屬文,既長博學。仕齊起家奉朝請,冠軍行參軍。天監初,拜後軍臨川王記室參軍,秣陵令,卒。



 謝幾卿,陳郡陽夏人。曾祖靈運,宋臨川內史;父超宗,齊黃門郎;並有重名於前代。幾卿幼清辯,當世號曰神童。後超宗坐事徙越州,路出新亭渚,幾卿不忍辭訣,遂投赴江流,左右馳救,得不沉溺。及居父憂,哀毀過禮。服闋,
 召補國子生。齊文惠太子自臨策試,謂祭酒王儉曰:「幾卿本長玄理,今可以經義訪之。」儉承旨發問,幾卿隨事辨對,辭無滯者,文惠大稱賞焉。儉謂人曰:「謝超宗為不死矣。」



 既長,好學,博涉有文采。起家豫章王國常侍,累遷車騎法曹行參軍、相國祭酒。出為寧國令,入補尚書殿中郎、太尉晉安王主簿。天監初,除征虜鄱陽王記室、尚書三公侍郎,尋為治書侍御史。舊郎官轉為此職者,世謂為南奔。幾卿頗失志,多陳疾,臺事略不復理。徙為散騎侍郎,累遷中書郎、國子博士、尚書左丞。幾卿詳悉故實,僕射徐勉每有疑滯,多詢訪之。然性通脫,會意便行,
 不拘朝憲。嘗預樂遊苑宴,不得醉而還,因詣道邊酒壚,停車褰幔,與車前三騶對飲,時觀者如堵,幾卿處之自若。後以在省署,夜著犢鼻褌,與門生登閣道飲酒酣呼,為有司糾奏,坐免官。尋起為國子博士,俄除河東太守,秩未滿,陳疾解。尋除太子率更令,遷鎮衛南平王長史。普通六年,詔遣領軍將軍西昌侯蕭淵藻督眾軍北伐,幾卿啟求行,擢為軍師長史,加威戎將軍。軍至渦陽退敗,幾卿坐免官。



 居宅在白楊石井,朝中交好者載酒從之,賓客滿坐。時左丞庾仲容亦免歸,二人意志相得,並肆情誕縱,或乘露車歷遊郊野,既醉則執鐸挽歌,不屑
 物議。湘東王在荊鎮,與書慰勉之。幾卿答曰:「下官自奉違南浦,卷迹東郊,望日臨風,瞻言佇立。仰尋惠渥,陪奉遊宴,漾桂棹於清池,席落英於曾岨。蘭香兼御,羽觴競集,側聽餘論,沐浴玄流。濤波之辯,懸河不足譬;春藻之辭,麗文無以匹。莫不相顧動容,服心勝口,不覺春日為遙,更謂脩夜為促。嘉會難常,摶雲易遠,言念如昨,忽焉素秋。恩光不遺,善謔遠降。因事罷歸,豈云栖息。既匪高官,理就一廛。田家作苦,實符清誨。本乏金羈之飾,無假玉璧為資;徒以老使形疏,疾令心阻,沉滯床簟,彌歷七旬。夢幻俄頃,憂傷在念,竟知無益,思自袪遣。尋理滌意,
 即以任命為膏酥;攬鏡照形,翻以支離代萱樹。故得仰慕徽猷,永言前哲;鬼谷深栖,接輿高舉;遁名屠肆,發跡關市;其人緬邈,餘流可想。若令亡者有知,寧不縈悲玄壤,悵隔芳塵;如其逝者可作,必當昭被光景,懽同游豫;使夫一介老圃,得簉虛心末席。去日已疏,來侍未孱;連劍飛鳧,擬非其類;懷私茂德,竊用涕零。」



 幾卿雖不持檢操,然於家門篤睦。兄才卿早卒,其子藻幼孤,幾卿撫養甚至。及藻成立,歷清官公府祭酒、主簿,皆幾卿獎訓之力也。世以此稱之。幾卿未及序用,病卒。文集行於世。



 劉勰,字彥和,東莞莒人。祖靈真,宋司空秀之弟也。父尚,
 越騎校尉。勰早孤,篤志好學。家貧不婚娶,依沙門僧祐,與之居處,積十餘年,遂博通經論,因區別部類,錄而序之。今定林寺經藏,勰所定也。天監初,起家奉朝請、中軍臨川王宏引兼記室,遷車騎倉曹參軍。出為太末令,政有清績。除仁威南康王記室,兼東宮通事舍人。時七廟饗薦已用蔬果,而二郊農社猶有犧牲。勰乃表言二郊宜與七廟同改,詔付尚書議,依勰所陳。遷步兵校尉,兼舍人如故。昭明太子好文學,深愛接之。



 初,勰撰《文心雕龍》五十篇,論古今文體,引而次之。其序曰:夫文心者,言為文之用心也。昔涓子《琴心》,王孫《巧心》,心哉美矣夫,故
 用之焉。古來文章,以雕糸辱成體,豈取騶奭群言雕龍也。夫宇宙綿邈,黎獻紛雜,拔萃出類,智術而已。歲月飄忽,性靈不居,騰聲飛實,制作而已。夫肖貌天地,稟性五才,擬耳目於日月,方聲氣乎風雷,其超出萬物,亦已靈矣。形甚草木之脆,名踰金石之堅,是以君子處世,樹德建言,豈好辯哉?不得已也。



 予齒在踰立,嘗夜夢執丹漆之禮器,隨仲尼而南行,旦而寤,乃怡然而喜。大哉聖人之難見也!乃小子之垂夢歟!自生人以來,未有如夫子者也。敷贊聖旨,莫若注經,而馬、鄭諸儒,弘之已精,就有深解,未足立家。唯文章之用,實經典枝條,五禮資之以成,
 六典因之致用,君臣所以炳煥,軍國所以昭明。詳其本源,莫非經典。而去聖久遠,文體解散,辭人愛奇,言貴浮詭,飾羽尚畫,文繡鞶帨,離本彌甚,將遂訛濫。蓋《周書》論辭,貴乎體要;尼父陳訓,惡乎異端。辭訓之異,宜體於要。於是搦筆和墨,乃始論文。



 詳觀近代之論文者多矣。至如魏文述《典》,陳思序《書》,應蒨《文論》,陸機《文賦》,仲洽《流別》,弘範《翰林》,各照隅隙,鮮觀衢路。或臧否當時之才,或銓品前脩之文,或汎舉雅俗之旨,或撮題篇章之意。魏《典》密而不周,陳《書》辯而無當,應《論》華而疏略,陸《賦》巧而碎亂,《流別》精而少功,《翰林》淺而寡要。又君山、公幹之徒,吉
 甫、士龍之輩,汎議文意,往往間出,並未能振葉以尋根,觀瀾而索源。不述先哲之誥,無益後生之慮。



 蓋《文心》之作也,本乎道,師乎聖,體乎經,酌乎緯,變乎《騷》,文之樞紐,亦云極矣。若乃論文敘筆,則囿別區分,原始以表末,釋名以章義,選文以定篇,敷理以舉統;上篇以上,綱領明矣。至於割情析表,籠圈條貫,摛神性,圖風勢,苞會通,閱聲字,崇贊於《時序》,褒貶於《才略》,怊悵於《知音》,耿介於《程器》,長懷《序志》,以馭群篇;下篇以下,毛目顯矣。位理定名,彰乎《大易》之數,其為文用,四十九篇而已。



 夫銓敘一文為易,彌綸群言為難。雖復輕采毛髮,深極骨髓,或有曲
 意密源,似近而遠,辭所不載,亦不勝數矣。及其品評成文,有同乎舊談者,非雷同也,勢自不可異也;有異乎前論者,非茍異也,理自不可同也。同之與異,不屑古今,擘肌分理,唯務折衷。案轡文雅之場,而環絡藻繪之府,亦幾乎備矣。但言不盡意,聖人所難,識在瓶管,何能矩矱。茫茫往代,既洗予聞;眇眇來世,儻塵彼觀。



 既成,未為時流所稱。勰自重其文,欲取定於沈約。約時貴盛,無由自達,乃負其書,候約出,乾之於車前,狀若貨鬻者。約便命取讀,大重之,謂為深得文理,常陳諸几案。然勰為文長於佛理,京師寺塔及名僧碑志,必請勰製文。有敕與慧
 震沙門於定林寺撰經證,功畢,遂啟求出家,先燔鬢發以自誓,敕許之。乃於寺變服,改名慧地。未期而卒。文集行於世。



 王籍,字文海,瑯邪臨沂人。祖遠,宋光祿勛。父僧祐,齊驍騎將軍。籍七歲能屬文。及長,好學博涉,有才氣,樂安任昉見而稱之。嘗於沈約坐賦得《詠燭》,甚為約賞。齊末,為冠軍行參軍,累遷外兵、記室。天監初,除安成王主簿、尚書三公郎、廷尉正。歷餘姚、錢塘令,並以放免。久之,除輕車湘東王諮議參軍,隨府會稽。郡境有雲門、天柱山,籍嘗遊之,或累月不反。至若邪溪賦詩,其略云:「蟬噪林逾
 靜,鳥鳴山更幽。」當時以為文外獨絕。還為大司馬從事中郎,遷中散大夫,尤不得志,遂徒行市道,不擇交遊。湘東王為荊州,引為安西府諮議參軍,帶作塘令。不理縣事,日飲酒,人有訟者,鞭而遣之。少時,卒。文集行於世。



 子碧,亦有文才,先籍卒。



 何思澄,字元靜,東海郯人。父敬叔,齊征東錄事參軍、餘杭令。思澄少勤學,工文辭。起家為南康王侍郎,累遷安成王左常侍,兼太學博士,平南安成王行參軍,兼記室。隨府江州,為《遊廬山詩》,沈約見之,大相稱賞,自以為弗逮。約郊居宅新構閣齋,因命工書人題此詩於壁。傅昭
 常請思澄製《釋奠詩》,辭文典麗。除廷尉正。天監十五年,敕太子詹事徐勉舉學士入華林撰《遍略》,勉舉思澄等五人以應選。遷治書侍御史。宋、齊以來,此職稍輕,天監初始重其選。車前依尚書二丞給三騶,執盛印青囊,舊事糾彈官印綬在前故也。久之,遷秣陵令,入兼東宮通事舍人。除安西湘東王錄事參軍,兼舍人如故。時徐勉、周捨以才具當朝,並好思澄學,常遞日招致之。昭明太子薨,出為黟縣令。遷除宣惠武陵王中錄事參軍,卒官,時年五十四。文集十五卷。初,思澄與宗人遜及子朗俱擅文名,時人語曰:「東海三何,子朗最多。」思澄聞之,曰:「此
 言誤耳。如其不然,故當歸遜。」思澄意謂宜在己也。



 子朗,字世明,早有才思,工清言,周捨每與共談,服其精理。嘗為《敗塚賦》,擬莊周馬棰,其文甚工。世人語曰:「人中爽爽何子朗。」歷官員外散騎侍郎,出為固山令。卒,時年二十四。文集行於世。



 劉杳,字士深,平原平原人也。祖乘民,宋冀州刺史。父聞慰齊東陽太守,有清績,在《齊書·良政傳》。杳年數歲,徵士明僧紹見之,撫而言曰:「此兒實千里之駒。」十三,丁父憂,每哭,哀感行路。天監初,為太學博士、宣惠豫章王行參軍。



 杳少好學,博綜群書,沈約、任昉以下,每有遺忘,皆訪
 問焉。嘗於約坐語及宗廟犧樽,約云:「鄭玄答張逸,謂為畫鳳皇尾娑娑然。今無復此器,則不依古。」杳曰:「此言未必可按。古者樽彞,皆刻木為鳥獸,鑿頂及背,以出內酒。頃魏世魯郡地中得齊大夫子尾送女器,有犧樽作犧牛形;晉永嘉賊曹嶷於青州發齊景公冢,又得此二樽,形亦為牛象。二處皆古之遺器,知非虛也。」約大以為然。約又云:「何承天《纂文》奇博,其書載張仲師及長頸王事,此何出?」杳曰:「仲師長尺二寸,唯出《論衡》。長頸是毘騫王,朱建安《扶南以南記》云:古來至今不死。」約即取二書尋檢,一如杳言。約郊居宅時新構閣齋,杳為贊二首,並以所
 撰文章呈約,約即命工書人題其贊于壁。仍報杳書曰:「生平愛嗜,不在人中,林壑之懽,多與事奪。日暮塗殫,此心往矣;猶復少存閑遠,徵懷清曠。結宇東郊,匪云止息,政復頗寄夙心,時得休偃。仲長遊居之地,休璉所述之美,望慕空深,何可仿佛。君愛素情多,惠以二贊。辭采妍富,事義畢舉,句韻之間,光影相照,便覺此地,自然十倍。故知麗辭之益,其事弘多,輒當置之閣上,坐臥嗟覽。別卷諸篇,並為名製。又山寺既為警策,諸賢從時復高奇,解頤愈疾,義兼乎此。遲此敘會,更共申析。」其為約所賞如此。又在任昉坐,有人餉昉曌酒而作榐字。昉問杳:「此
 字是不?」杳對曰:「葛洪《字苑》作木旁絜。」昉又曰:「酒有千日醉,當是虛言。」杳云:「桂陽程鄉有千里酒,飲之至家而醉,亦其例也。」昉大驚曰:「吾自當遺忘,實不憶此。」杳云:「出楊元鳳所撰《置郡事》。元鳳是魏代人,此書仍載其賦,云三重五品,商溪摖里。」時即檢楊記,言皆不差。王僧孺被敕撰譜,訪杳血脈所因。杳云:「桓譚《新論》云:『太史《三代世表》,旁行邪上,並效周譜。』以此而推,當起周代。」僧孺歎曰:「可謂得所未聞。」周捨又問杳:「尚書官著紫荷橐,相傳云『挈囊』,竟何所出?」杳答曰:「《張安世傳》曰『持橐簪筆,事孝武皇帝數十年』。韋昭、張晏注並云『橐,囊也。近臣簪筆,以待顧
 問』。」范岫撰《字書音訓》,又訪杳焉。其博識彊記,皆此類也。



 尋佐周捨撰國史。出為臨津令,有善績。秩滿,縣人三百餘人詣闕請留,敕許焉。杳以疾陳解,還除雲麾晉安王府參軍。詹事徐勉舉杳及顧協等五人入華林撰《遍略》,書成,以本官兼廷尉正,又以足疾解。因著《林庭賦》。王僧孺見之歎曰:「《郊居》以後,無復此作。」普通元年,復除建康正,遷尚書駕部郎;數月,徙署儀曹郎,僕射勉以臺閣文議專委杳焉。出為餘姚令,在縣清潔,人有饋遺,一無所受,湘東王發教褒稱之。還除宣惠湘東王記室參軍,母憂去職。服闋,復為王府記室,兼東宮通事舍人。大通元
 年,遷步兵校尉,兼舍人如故。昭明太子謂杳曰:「酒非卿所好,而為酒廚之職,政為不愧古人耳。」俄有敕,代裴子野知著作郎事。昭明太子薨,新宮建,舊人例無停者,敕特留杳焉。仍注太子《徂歸賦》,稱為博悉。僕射何敬容奏轉杳王府諮議,高祖曰:「劉杳須先經中書。」仍除中書侍郎。尋為平西湘東王諮議參軍,兼舍人、知著作如故。遷為尚書左丞。大同二年,卒官,時年五十。



 杳治身清儉,無所嗜好。為性不自伐,不論人短長,及睹釋氏經教,常行慈忍。天監十七年,自居母憂,便長斷腥膻,持齋蔬食。及臨終,遺命斂以法服,載以露車,還葬舊墓,隨得一地,容
 棺而已,不得設靈筵祭醊。其子遵行之。



 杳自少至長,多所著述。撰《要雅》五卷、《楚辭草木疏》一卷、《高士傳》二卷、《東宮新舊記》三十卷、《古今四部書目》五卷,並行於世。



 謝徵,字玄度,陳郡陽夏人。高祖景仁,宋尚書左僕射。祖稚,宋司徒主簿。父璟,少與從叔朓俱知名。齊竟陵王子良開西邸,招文學,璟亦預焉。隆昌中,為明帝驃騎諮議參軍,領記室。遷中書郎,晉安內史。高祖平京邑,為霸府諮議、梁臺黃門郎。天監初,累遷司農卿、秘書監、左民尚書、明威將軍、東陽太守。高祖用為侍中,固辭年老,求金紫,未序,會疾卒。



 征幼聰慧,璟異之,常謂親從曰:「此兒非
 常器,所憂者壽;若天假其年,吾無恨矣。」既長,美風采,好學善屬文。初為安西安成王法曹,遷尚書金部三公二曹郎、豫章王記室,兼中書舍人。遷除平北諮議參軍,兼鴻臚卿,舍人如故。



 徵與河東裴子野、沛國劉顯同官友善,子野嘗為《寒夜直宿賦》以贈征,徵為《感友賦》以酬之。時魏中山王元略還北,高祖餞於武德殿,賦詩三十韻,限三刻成。征二刻便就,其辭甚美,高祖再覽焉。又為臨汝侯淵猷製《放生文》,亦見賞於世。



 中大通元年,以父喪去職,續又丁母憂。詔起為貞威將軍,還攝本任。服闋,除尚書左丞。三年,昭明太子薨,高祖立晉安王綱為皇太
 子,將出詔,唯召尚書左僕射何敬容、宣惠將軍孔休源及徵三人與議。徵時年位尚輕,而任遇已重。四年,累遷中書郎,鴻臚卿、舍人如故。六年,出為北中郎豫章王長史、南蘭陵太守。大同二年,卒官,時年三十七。友人瑯邪王籍集其文為二十卷。



 臧嚴,字彥威,東莞莒人也。曾祖燾,宋左光祿。祖凝,齊尚書右丞。父夌,後軍參軍。嚴幼有孝性,居父憂以毀聞。孤貧勤學,行止書卷不離於手。初為安成王侍郎,轉常侍。從叔未甄為江夏郡,攜嚴之官,於塗作《屯遊賦》,任昉見而稱之。又作《七算》,辭亦富麗。性孤介,於人間未嘗造請。
 僕射徐勉欲識之,嚴終不詣。



 遷冠軍行參軍、侍湘東王讀,累遷王宣惠輕車府參軍,兼記室。嚴於學多所諳記,尤精《漢書》,諷誦略皆上口。王嘗自執四部書目以試之,嚴自甲至丁卷中,各對一事,並作者姓名,遂無遺失,其博洽如此。王遷荊州,隨府轉西中郎安西錄事參軍。歷監義陽、武寧郡,累任皆蠻左,前郡守常選武人,以兵鎮之;嚴獨以數門生單車入境,群蠻悅服,遂絕寇盜。王入為石頭戍軍事,除安右錄事。王遷江州,為鎮南諮議參軍,卒官。文集十卷。



 伏挺,字士標。父芃,為豫章內史,在《良吏傳》。挺幼敏寤,七
 歲通《孝經》、《論語》。及長,有才思,好屬文,為五言詩,善效謝康樂體。父友人樂安任昉深相歎異,常曰:「此子目下無雙。」齊末,州舉秀才,對策為當時第一。高祖義師至,挺迎謁於新林,高祖見之甚悅,謂曰「顏子」,引為征東行參軍,時年十八。天監初,除中軍參軍事。宅居在潮溝,於宅講《論語》,聽者傾朝。遷建康正,俄以劾免。久之,入為尚書儀曹郎,遷西中郎記室參軍,累為晉陵、武康令。罷縣還,仍於東郊築室,不復仕。



 挺少有盛名,又善處當世,朝中勢素,多與交遊,故不能久事隱靜。時僕射徐勉以疾假還宅,挺致書以觀其意曰:昔士德懷顧,戀興數日;輔嗣思
 友,情勞一旬。故知深心所係,貴賤一也。況復恩隆世親,義重知己,道庇生人,德弘覆蓋。而朝野懸隔,山川邈殊,雖咳唾時沾,而顏色不覯。《東山》之歎,豈云旋復;西風可懷,孰能無思。加以靜居廓處,顧影莫酬,秋風四起,園林易色,涼野寂寞,寒蟲吟叫。懷抱不可直置,情慮不能無託,時因吟詠,動輒盈篇。揚生沉鬱,且猶覆盎;惠子五車,彌多踳駮。一日聊呈小文,不期過賞,還逮隆渥,累牘兼翰,紙縟字磨,誦復無已,徒恨許與過當,有傷準的。昔子建不欲妄贊陳琳,恐見嗤哂後代;今之過奢餘論,將不有累清談?



 挺竄迹草萊,事絕聞見,藉以謳謠,得之輿牧。
 仰承有事砭石,仍成簡通,娛腸悅耳,稍從擯落,宴處榮觀,務在滌除。綺羅絲竹,二列頓遣;方丈員案,三桮僅存。故以道變區中,情沖域外;操彼絃誦,賁茲觀損。追留侯之卻粒,念韓卿之辭榮;眷想東都,屬懷南岳;鑽仰來貺,有符下風。雖云幸甚,然則未喻。雖復帝道康寧,走馬行卻,《由庚》得所,寅亮有歸。悠悠之人,展氏猶且攘袂;浩浩白水,甯叟方欲褰裳。是知君子拯物,義非徇己。思與赤松子遊,誰其克遂。願驅之仁壽,綏此多福。雖則不言,四時行矣。然後黔首有庇,薦紳靡奪;白駒不在空谷,屠羊豫蒙其賚。豈不休哉?豈不休哉?昔杜真自閉深室,郎宗
 絕迹幽野。難矣,誠非所希。井丹高潔,相如慢世,尚復游涉權門,雍容鄉邑,常謂此道為泰,每竊慕之。方念擁帚延思,以陳侍者,請至農隙,無待邀求。



 挺誠好屬文,不會今世,不能促節局步,以應流俗。事等昌菹,謬彼偏嗜,是用不羞固陋,無憚龍門。昔敬通之賞景卿,孟公之知仲蔚,止乎通人,猶稱盛美,況在時宗,彌為未易。近以蒲槧勿用,箋素多闕,聊效東方,獻書丞相,須得善寫,更請潤訶,儻逢子侯,比復削牘。



 勉報曰:復覽來書,累牘兼翰;事苞出處,言兼語默;事義周悉,意致深遠;發函伸紙,倍增憤歎。卿雄州擢秀,弱冠升朝,穿綜百家,佃漁六學;觀眸
 表其韶慧,視色見其英朗,若魯國之名駒,邁雲中之白鶴。及占顯邑,試吏腴壤,將有武城弦歌,桐鄉謠詠,豈與卓魯斷斷同年而語邪?方當見賞良能,有加寵授,飾茲簪帶,置彼周行。而欲遠慕卷舒,用懷愚智,既知益之為累,爰悟滿則辭多,高蹈風塵,良所欽挹。況以金商戒節,素秋御序,蕭條林野,無人相樂,偃臥墳籍,遊浪儒玄,物我兼忘,寵辱誰滯?誠乃歡羨,用有殊同。今逖聽傍求,興懷寤宿,白駒空谷,幽人引領,貧賤為恥,鳥獸難群,故當捐此薜蘿,出從鵷鷺,無乖隱顯,不亦休哉!



 吾智乏佐時,才慚濟世,稟承朝則,不敢荒寧,力弱途遙,愧心非一。天
 下有道,堯人何事?得因疲病,念從閑逸。若使車書混合,尉候無警,作樂制禮,紀石封山,然後乃返服衡門,實為多幸。但夙有風咳,遘茲虛眩,瘠類士安,羸同長孺,簿領沉廢,臺閣未理,娛耳爛腸,因事而息,非關欲追松子,遠慕留侯。若乃天假之年,自當靖恭所職。擬非倫匹,良覺辭費;覽復循環,爽焉如失。清塵獨遠,白雲飄蕩,依然何極。



 猥降書札,示之文翰,覽復成誦,流連縟紙。昔仲宣才敏,藉中郎而表譽;正平穎悟,賴北海以騰聲。望古料今,吾有慚德。儻成卷帙,力為稱首。無令獨耀隨掌,空使辭人扼腕。式閭願見,宜事掃門。亦有來思,赴其懸榻。輕苔
 魚網,別當以薦。城闕之嘆,曷日無懷;所遲萱蘇,書不盡意。



 挺後遂出仕,尋除南臺治書,因事納賄,當被推劾。挺懼罪,遂變服為道人,久之藏匿,後遇赦,乃出大心寺。會邵陵王為江州,攜挺之鎮,王好文義,深被恩禮,挺因此還俗。復隨王遷鎮郢州,徵入為京尹,挺留夏首,久之還京師。太清中,客遊吳興、吳郡,侯景亂中卒。著《邇說》十卷,文集二十卷。



 子知命,先隨挺事邵陵王,掌書記。亂中,王於郢州奔敗,知命仍下投侯景。常以其父宦途不至,深怨朝廷,遂盡心事景。景襲郢州,圍巴陵,軍中書檄,皆其文也。及景篡位,為中書舍人,專任權寵,勢傾內外。景敗
 被執,送江陵,於獄中幽死。挺弟捶,亦有才名,先為邵陵王所引,歷為記室、中記室、參軍。



 庾仲容,字仲容,潁川焉陵人也。晉司空冰六代孫。祖徽之,宋御史中丞。父漪,齊邵陵王記室。仲容幼孤,為叔父泳所養。既長,杜絕人事,專精篤學,晝夜手不輟卷。初為安西法曹行參軍。泳時已貴顯,吏部尚書徐勉擬泳子晏嬰為宮僚,泳垂泣曰:「兄子幼孤,人才粗可,願以晏嬰所忝回用之。」勉許焉,因轉仲容為太子舍人。遷安成王主簿。時平原劉孝標亦為府佐,並以彊學為王所禮接。遷晉安功曹史。歷為永康、錢唐、武康令,治縣並無異績,
 多被劾。久之,除安成王中記室,當出隨府,皇太子以舊恩,特降餞宴,賜詩曰:「孫生陟陽道,吳子朝歌縣。未若樊林舉,置酒臨華殿。」時輩榮之。遷安西武陵王諮議參軍。除尚書左丞,坐推糾不直免。



 仲容博學,少有盛名,頗任氣使酒,好危言高論,士友以此少之。唯與王籍、謝幾卿情好相得,二人時亦不調,遂相追隨,誕縱酣飲,不復持檢操。久之,復為諮議參軍,出為黟縣令。及太清亂,客遊會稽,遇疾卒,時年七十四。



 仲容抄諸子書三十卷,眾家地理書二十卷,《列女傳》三卷,文集二十卷,並行於世。



 陸雲公,字子龍,吳郡人也。祖閑,州別駕。父完,寧遠長史。
 雲公五歲誦《論語》、《毛詩》,九歲讀《漢書》,略能記憶。從祖倕、沛國劉顯質問十事,雲公對無所失,顯歎異之。既長,好學有才思。州舉秀才。累遷宣惠武陵王、平西湘東王行參軍。雲公先製《太伯廟碑》,吳興太守張纘罷郡經途,讀其文歎曰:「今之蔡伯喈也。」纘至都掌選,言之於高祖,召兼尚書儀曹郎,頃之即真,入直壽光省,以本官知著作郎事。俄除著作郎,累遷中書黃門郎,並掌著作。雲公善弈棋,嘗夜侍御坐,武冠觸燭火,高祖笑謂曰:「燭燒卿貂。」高祖將用雲公為侍中,故以此言戲之也。是時天淵池新製扁魚舟,形闊而短,高祖暇日,常泛此舟,在朝唯引
 太常劉之遴、國子祭酒到溉、右衛朱異,雲公時年位尚輕,亦預焉。其恩遇如此。太清元年,卒,時年三十七。高祖悼惜之,手詔曰:「給事黃門侍郎、掌著作陸雲公,風尚優敏,後進之秀。奄然殂謝,良以惻然。可剋日舉哀,賻錢五萬、布四十匹。」



 張纘時為湘州,與雲公叔襄、兄晏子書曰:「都信至,承賢兄子賢弟黃門殞折,非唯貴門喪寶,實有識同悲,痛惋傷惜,不能已已。賢兄子賢弟神情早著,標令弱年,經目所睹,殆無再問。懷橘抱柰,稟自天情;倨坐列薪,非因外獎。學以聚之,則一箸能立;問以辯之,則師心獨寤。始踰弱歲,辭藝通洽,升降多士,秀也詩流。見與
 齒過肩隨,禮殊拜絕,懷抱相得,忘其年義。朝遊夕宴,一載于斯;玩古披文,終晨訖暮。平生知舊,零落稍盡,老夫記意,其數幾何。至若此生,寧可多過,賞心樂事,所寄伊人。弟遷職瀟、湘,維舟洛汭,將離之際,彌見情款。夕次帝郊,亟淹信宿,徘徊握手,忍分歧路。行役數年,羈病侵迫,識慮惛怳,久絕人世。憑几口授,素無其功;翰動若飛,彌有多愧。京洛遊故,咸成雲雨,唯有此生,音塵數嗣。形迹之外,不為遠近隔情;襟素之中,豈以風霜改節?客遊半紀,志切首丘,日望東歸,更敦昔款。如何此別,永成異世!揮袂之初,人誰自保,但恐衰謝,無復前期。不謂華齡,方
 春掩質,埋玉之恨,撫事多情。想引進之情,懷抱素篤,友于之至,兼深家寶。奄有此恤,當何可言!臨白增悲,言以無次。」



 雲公從兄才子,亦有才名,歷官中書郎、宣成王友、太子中庶子、廷尉卿,先雲公卒。才子、雲公文集,並行於世。



 任孝恭,字孝恭,臨淮臨淮人也。曾祖農夫,宋南豫州刺史。孝恭幼孤,事母以孝聞。精力勤學,家貧無書,常崎嶇從人假借。每讀一遍,諷誦略無所遺。外祖丘它,與高祖有舊,高祖聞其有才學,召入西省撰史。初為奉朝請,進直壽光省,為司文侍郎,俄兼中書通事舍人。敕遣製《建
 陵寺剎下銘》,又啟撰高祖集序文,並富麗,自是專掌公家筆翰。孝恭為文敏速,受詔立成,若不留意,每奏,高祖輒稱善,累賜金帛。孝恭少從蕭寺雲法師讀經論,明佛理,至是,蔬食持戒,信受甚篤。而性頗自伐,以才能尚人,於時輩中多有忽略,世以此少之。



 太清二年,侯景寇逼,孝恭啟募兵,隸蕭正德,屯南岸。及賊至,正德舉眾入賊,孝恭還赴臺,臺門已閉,因奔入東府,尋為賊所攻,城陷見害。文集行於世。



 顏協,字子和,瑯邪臨沂人也。七代祖含,晉侍中、國子祭酒、西平靖侯。父見遠,博學有志行。初,齊和帝之鎮荊州
 也,以見遠為錄事參軍,及即位於江陵,以為治書侍御史,俄兼中丞。高祖受禪,見遠乃不食,發憤數日而卒。高祖聞之曰:「我自應天從人,何預天下士大夫事?而顏見遠乃至於此也。」協幼孤,養於舅氏。少以器局見稱。博涉群書,工於草隸。釋褐湘東王國常侍,又兼府記室。世祖出鎮荊州,轉正記室。時吳郡顧協亦在蕃邸,與協同名,才學相亞,府中稱為「二協」。舅陳郡謝暕卒,協以有鞠養恩,居喪如伯叔之禮,議者重焉。又感家門事義,不求顯達,恒辭徵辟,遊於蕃府而已。大同五年,卒,時年四十二。世祖甚歎惜之,為《懷舊詩》以傷之。其一章曰:「弘都多雅
 度,信乃含賓實。鴻漸殊未升,上才淹下秩。」



 協所撰《晉仙傳》五篇、《日月災異圖》兩卷,遇火湮滅。



 有二子:之儀、之推,並早知名。之推,承聖中仕至正員郎、中書舍人。



 陳吏部尚書姚察曰:魏文帝稱古之文人,鮮能以名節自全。何哉?夫文者妙發性靈,獨拔懷抱,易邈等夷,必興矜露。大則凌慢侯王,小則慠蔑朋黨;速忌離訧,啟自此作。若夫屈、賈之流斥,桓、馮之擯放,豈獨一世哉?蓋恃才之禍也。群士值文明之運,摛艷藻之辭,無鬱抑之虞,不遭向時之患,美矣。劉氏之論,命之徒也。命也者,聖人罕言歟,就而必之,非經意也。



\end{pinyinscope}