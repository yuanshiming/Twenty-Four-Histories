\article{附錄梁書序}

\begin{pinyinscope}

 《梁書》,六本紀,五十列傳,合五十六
 篇。唐貞觀三年,詔右散騎常侍姚思廉撰。思廉者,梁史官察之子。推其父意,又頗採諸儒謝吳等所記,以成此書。臣等既校正其文字,又集次為目錄一篇而敘之曰:自先王之道不明,百家並起,佛最晚出,為中國之患,而在梁為尤甚,故不得而不論也。蓋佛之徒自以謂吾之所得者內,而世之論佛者皆外也,故不可絀;雖然,彼惡睹聖人之內哉?《書》曰:「思曰睿,睿作聖。」蓋思者,所以致其知也。能致其知者,察三才之道,辯萬物之理,小大精粗無不盡也。此之謂窮理,知之至也。知至矣,則在我者之足貴,在彼者之不足玩,未有不能明之者也。有知之之明而不能好之,未可也,故加之誠心以好之;有好之之心而不能樂之,未可也,故加之至意以樂之。能樂之則能安之矣。如是,則萬物之自外至者安能累我哉?萬物之所不能累,故吾之所以盡其性也。能盡其性則誠矣。誠者,成也,不惑也。既成矣,必充之使可大焉;既大矣,必推之使可化焉;能化矣,則含智之民,肖翹之物,有待於我者,莫不由之以至其性,遂其宜,而吾之用與天地參矣。德如此其至也,而應乎外者未嘗不與人
 同,此吾之道所以為天下之達道也。故與之為衣冠、飲食、冠昏、喪祭之具,而由之以教其為君臣、父子、兄弟、夫婦者,莫不一出乎人情;與之同其吉兇而防其憂患者,莫不一出乎人理。故與之處而安且治之所集也,危且亂之所去也。與之所處者其具如此,使之化者其德如彼,可不謂聖矣乎?既聖矣,則無思也,其至者循理而已;無為也,其動者應物而已。是以覆露乎萬物,鼓舞
 乎群眾,而未有能測之者也,可不謂神矣乎?神也者,至妙而不息者也,此聖人之內也。聖人者,道之極也,佛之說其有以易此乎?求其有以易此者,固其所以為失也。夫得於內者,未有不可行於外也;有不可行於外者,斯不得於內矣。《易》曰:「智周乎萬物而道濟乎天下,故不過。」此聖人所以兩得之也。智足以知一偏,而不足以盡萬事之理,道足以為一方,而不足以適天下之用,此百家之所以兩失之也。佛之失其不以此乎?則佛之徒自以謂得諸內者,亦可謂妄矣。



 夫學史者將以明一代之得失也,臣等故因梁之事,而為著聖人之所以得及佛之所以失以傳之者,使知君子之所以距佛者非外,而有志於內者,庶不以此而易彼也。



 臣鞏等謹敘目錄,昧
 死上。



\end{pinyinscope}