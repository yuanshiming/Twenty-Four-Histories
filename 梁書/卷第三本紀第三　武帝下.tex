\article{卷第三本紀第三 武帝下}

\begin{pinyinscope}

 普通元年春正月乙亥朔,改元,大赦天下。賜文武勞位,孝悌力田爵一級,尤貧之家,勿收常調,鰥寡孤獨,並加贍恤。丙子,日有蝕之。己卯,以司徒臨川王宏為太尉、揚州刺史,安右將軍、監揚州蕭景為安西將軍、郢州刺史。尚書左僕射王暕以母憂去職,金紫光祿大夫王份為尚書左僕射。庚子,扶南、高麗國各遣使獻方物。二月壬
 子,老人星見。癸丑,以高麗王世子安為寧東將軍、高麗王。三月丙戌,滑國遣使獻方物。夏四月甲午,河南王遣使獻方物。六月丁未,以護軍將軍韋睿為車騎將軍。秋七月己卯,江、淮、海並溢。辛卯,以信威將軍邵陵王綸為江州刺史。八月庚戌,老人星見。甲子,新除車騎將軍韋睿卒。九月乙亥,有星晨見東方,光爛如火。冬十月辛亥,以宣惠將軍長沙王深業為護軍將軍。辛酉,以丹陽尹晉安王綱為平西將軍、益州刺史。



 二年春正月甲戌,以南徐州刺史豫章王綜為鎮右將軍。新除益州刺史晉安王綱改為徐州刺史。辛巳,輿駕
 親祠南郊。詔曰:「春司御氣,虔恭報祀,陶匏克誠,蒼璧禮備,思隨乾覆,布茲亭育。凡民有單老孤稚,不能自存,主者郡縣咸加收養,贍給衣食,每令周足,以終其身。又於京師置孤獨園,孤幼有歸,華髮不匱。若終年命,厚加料理。尤窮之家,勿收租賦。」戊子,大赦天下。二月辛丑,輿駕親祠明堂。三月庚寅,大雪,平地三尺。夏四月乙卯,改作南北郊。丙辰,詔曰:「夫欽若昊天,歷象無違。躬執耒耜,盡力致敬,上協星鳥,俯訓民時,平秩東作,義不在南。前代因襲,有乖禮制,可於震方,簡求沃野,具茲千畝,庶允舊章。」五月癸卯,琬琰殿火,延燒後宮屋三千間。丁巳,詔曰:「
 王公卿士,今拜表賀瑞,雖則百辟體國之誠,朕懷良有多愧。若其澤漏川泉,仁被動植,氣調玉燭,治致太平,爰降嘉祥,可無慚德;而政道多缺,淳化未凝,何以仰葉辰和,遠臻冥貺?此乃更彰寡薄,重增其尤。自今可停賀瑞。」六月丁卯,信威將軍、義州刺史文僧明以州叛入于魏。秋七月丁酉,假大匠卿裴邃節,督眾軍北討。甲寅,老人星見。魏荊州刺史桓叔興帥眾降。八月丁亥,始平郡中石鼓村地自開成井,方六尺六寸,深三十二丈。冬十一月,百濟、新羅國各遣使獻方物。十二月戊辰,以鎮東大將軍百濟王餘隆為寧東大將軍。



 三年春正月庚子,以尚書令袁昂為中書監,吳郡太守王暕為尚書左僕射,尚書左僕射王份為右光祿大夫。庚戌,京師地震。己未,以宣毅將軍廬陵王續為雍州刺史。三月乙卯,巴陵王蕭屏薨。夏四月丁卯,汝陰王劉端薨。五月壬辰朔,日有蝕之,既。癸巳,赦天下。并班下四方,民所疾苦,咸即以聞,公卿百僚各上封事,連率郡國舉賢良、方正、直言之士。秋八月辛酉,作二郊及籍田並畢,班賜工匠各有差。甲子,老人星見。婆利、白題國各遣使獻方物。冬十月丙子,加中書監袁昂中衛將軍。十一月甲午,撫軍將軍、開府儀同三司、領軍將軍始興王憺薨。
 辛丑,以太子詹事蕭淵藻為領軍將軍。



 四年春正月辛卯,輿駕親祠南郊,大赦天下。應諸窮疾,咸加賑恤,并班下四方,時理獄訟。丙午,輿駕親祠明堂。二月庚午,老人星見。乙亥,躬耕籍田。詔曰:「夫耕籍之義大矣哉!粢盛由之而興,禮節因之以著,古者哲王咸用此作。眷言八政,致茲千畝,公卿百辟,恪恭其儀,九推畢禮,馨香靡替。兼以風雲葉律,氣象光華,屬覽休辰,思加獎勸。可班下遠近,廣闢良疇,公私畎畝,務盡地利。若欲附農,而糧種有乏,亦加貸恤,每使優遍。孝悌力田賜爵一級。預耕之司,剋日勞酒。」三月壬寅,以鎮右將軍豫章
 王綜為平北將軍、南兗州刺史。六月乙丑,分益州置信州,分交州置愛州,分廣州置成州、南定州、合州、建州,分霍州置義州。秋八月丁卯,老人星見。冬十月庚午,以中書監、中衛將軍袁昂為尚書令,即本號開府儀同三司。己卯,護軍將軍昌義之卒。十一月癸未朔,日有蝕之。太白晝見。甲辰,尚書左僕射王暕卒。十二月戊午,始鑄鐵錢。狼牙脩國遣使獻方物。



 五年春正月,以左光祿大夫、開府儀同三司南平王偉為鎮衛大將軍,改領右光祿大夫,儀同三司如故。征西將軍、開府儀同三司、荊州刺史鄱陽王恢進號驃騎大
 將軍。太府卿夏侯亶為中護軍。右光祿大夫王份為左光祿大夫,加特進。辛卯,平北將軍、南兗州刺史豫章王綜進號鎮北將軍。平西將軍、雍州刺史晉安王綱進號安北將軍。二月庚午,特進、左光祿大夫王份卒。丁丑,老人星見。三月甲戌,分揚州、江州置東揚州。夏四月乙未,以雲麾將軍南康王績為江州刺史。六月乙酉,龍鬥于曲阿王陂,因西行至建陵城。所經處樹木倒折,開地數十丈。戊子,以會稽太守武陵王紀為東揚州刺史。庚子,以員外散騎常侍元樹為平北將軍、北青、兗二州刺史,率眾北伐。秋七月辛未,賜北討義客位一階。八月庚寅,
 徐州刺史成景雋剋魏童棧。九月戊申,又剋睢陵城。戊午,北兗州刺史趙景悅圍荊山。壬戌,宣毅將軍裴邃襲壽陽,入羅城,弗剋。冬十月戊寅,裴邃、元樹攻魏建陵城,破之。辛巳,又破曲木。掃虜將軍彭寶孫剋瑯邪。甲申,又剋檀丘城。辛卯,裴邃破狄城。丙申,又剋甓城,遂進屯黎漿。壬寅,魏東海太守韋敬欣以司吾城降。定遠將軍太守曹世宗破魏曲陽城。甲辰,又剋秦墟。魏郿、潘溪守悉皆棄城走。十一月丙辰,彭寶孫剋東莞城。壬戌,裴邃攻壽陽之安城,剋之。丙寅,魏馬頭、安城並來降。十二月戊寅,魏荊山城降。乙巳,武勇將軍李國興攻平靜關,
 剋之。辛丑,信威長史楊法乾攻武陽關;壬寅,攻峴關:並剋之。



 六年春正月丙午,安北將軍晉安王綱遣長史柳津破魏南鄉郡,司馬董當門破魏晉城。庚戌,又破馬圈、彫陽二城。辛亥,輿駕親祠南郊,大赦天下。庚申,魏鎮東將軍、徐州刺史元法僧以彭城內附。己巳,雍州前軍剋魏新蔡郡。詔曰:「廟謨已定,王略方舉。侍中、領軍將軍西昌侯淵藻,可便親戎,以前啟行;鎮北將軍、南兗州刺史豫章王綜董馭雄桀,風馳次邁;其餘眾軍,計日差遣,初中後師,善得嚴辦。朕當六軍雲動,龍舟濟江。」癸酉,剋魏鄭城。
 甲戌,以魏鎮東將軍、徐州刺史元法僧為司空。二月丁丑,老人星見。庚辰,南徐州刺史廬陵王續還朝,稟承戎略。乙未,趙景悅下魏龍亢城。三月丙午,歲星見南斗。賜新附民長復除,應諸罪失,一無所問。己酉,行幸白下城,履行六軍頓所。乙丑,鎮北將軍、南兗州刺史豫章王綜權頓彭城,總督眾軍,并攝徐州府事。己巳,以魏假平東將軍元景隆為衡州刺史,魏征虜將軍元景仲為廣州刺史。夏五月己酉,築宿預堰,又修曹公堰於濟陰。太白晝見。壬子,遣中護軍夏侯亶督壽陽諸軍事,北伐。六月庚辰,豫章王綜奔于魏,魏復據彭城。秋七月壬戌,大赦
 天下。八月丙子,以散騎常侍曹仲宗兼領軍。壬午,老人星見。十二月戊子,邵陵王綸有罪,免官,削爵土。壬辰,京師地震。



 七年春正月辛丑朔,赦殊死以下。丁卯,滑國遣使獻方物。二月甲戌,北伐眾軍解嚴。河南遣使獻方物。丁亥,老人星見。三月乙卯,高麗國遣使獻方物。夏四月乙酉,太尉臨川王宏薨。南州津改置校尉,增加俸秩。詔在位群臣,各舉所知,凡是清吏,咸使薦聞,州年舉二人,大郡一人。六月己卯,林邑國遣使獻方物。秋九月己酉,驃騎大將軍、開府儀同三司、荊州刺史鄱陽王恢薨。冬十月
 辛未,以丹陽尹、湘東王繹為荊州刺史。十一月庚辰,大赦天下。是日,丁貴嬪薨。辛巳,夏侯亶、胡龍牙、元樹、曹世宗等眾軍剋壽陽城。丁亥,放魏揚州刺史李憲還北。以壽陽置豫州,合肥改為南豫州。以中護軍夏侯亶為豫、南豫二州刺史。平西將軍、郢州刺史元樹進號安西將軍。魏新野太守以郡降。



 大通元年春正月乙丑,以尚書左僕射徐勉為尚書僕射、中衛將軍。詔曰:「朕思利兆民,惟日不足,氣象環回,每弘優簡。百官俸祿,本有定數,前代以來,皆多評准,頃者因循,未遑改革。自今已後,可長給見錢,依時即出,勿令逋
 緩。凡散失官物,不問多少,並從原宥。惟事涉軍儲,取公私見物,不在此例。」辛未,輿駕親祠南郊。詔曰:「奉時昭事,虔薦蒼璧,思承天德,惠此下民。凡因事去土,流移他境者,並聽復宅業,蠲役五年。尤貧之家,勿收三調。孝悌力田賜爵一級。」是月,司州刺史夏侯夔進軍三關,所至皆剋。三月辛未,輿駕幸同泰寺捨身。甲戌,還宮,赦天下,改元。以左衛將軍蕭淵藻為中護軍。林邑、師子國各遣使獻方物。



 夏五月丙寅,成景雋剋魏臨潼竹邑。秋八月壬辰,老人星見。冬十月庚戌,魏東豫州刺史元慶和以渦陽內屬。甲寅,曲赦東豫州。十一月丁卯,以中護軍蕭
 淵藻為北討都督、征北大將軍,鎮渦陽。戊辰,加尚書令、中衛將軍、開府儀同三司袁昂中書監。以渦陽置西徐州。高麗國遣使獻方物。



 二年春正月庚申,司空元法僧以本官領中軍將軍。中書監、尚書令、中衛將軍、開府儀同三司袁昂進號中撫大將軍。衛尉卿蕭昂為中領軍。乙酉,芮芮國遣使獻方物。二月甲午,老人星見。是月,築寒山堰。三月壬戌,以江州刺史南康王績為安右將軍。夏四月辛丑,魏郢州刺史元願達以義陽內附,置北司州。時魏大亂,其北海王元顥、臨淮王元彧、汝南王元悅並來奔;其北青州刺史
 元世雋、南荊州刺史李志亦以地降。六月丁亥,魏臨淮王元彧求還本國,許之。冬十月丁亥,以魏北海王元顥為魏主,遣東宮直閣將軍陳慶之衛送還北。魏豫州刺史鄧獻以地內屬。



 中大通元年正月辛酉,輿駕親祠南郊,大赦天下,孝悌力田賜爵一級。甲子,魏汝南王元悅求還本國,許之。辛巳,輿駕親祠明堂。二月甲申,以丹陽尹武陵王紀為江州刺史。辛丑,芮芮國遣使獻方物。三月丙辰,以河南王阿羅真為寧西將軍、西秦、河沙三州刺史。庚辰,以中護軍蕭淵藻為中權將軍。夏四月癸未,以安右將軍南康
 王績為護軍將軍。癸巳,陳慶之攻魏梁城,拔之,進屠考城,擒魏濟陰王元暉業。五月戊辰,剋大梁。癸酉,剋虎牢城。魏主元子攸棄洛陽,走河北。乙亥,元顥入洛陽。六月壬午,大赦天下。辛亥,魏淮陰太守晉鴻以湖陽城內屬。閏月己未,安右將軍、護軍南康王績薨。己卯,魏爾朱榮攻殺元顥,復據洛陽。秋九月辛巳,朱雀航華表災。以安北將軍羊侃為青、冀二州刺史。癸巳,輿駕幸同泰寺,設四部無遮大會,因捨身,公卿以下,以錢一億萬奉贖。冬十月己酉,輿駕還宮,大赦,改元。十一月丙戌,加中撫大將軍、開府儀同三司袁昂中書監。加鎮衛大將軍、開府
 儀同三司南平王偉太子少傅。加金紫光祿大夫蕭琛、陸杲並特進。司空、中軍將軍元法僧進號車騎將軍。中權將軍蕭淵藻為中護軍將軍。中領軍蕭昂為領軍將軍。戊子,魏巴州刺史嚴始欣以城降。十二月丁巳,盤盤國遣使獻方物。



 二年春正月戊寅,以雍州刺史晉安王綱為驃騎大將軍、揚州刺史,南徐州刺史廬陵王續為平北將軍、雍州刺史。癸未,老人星見。夏四月庚申,大雨雹。壬申,以河南王佛輔為寧西將軍、西秦、河二州刺史。六月丁巳,遣魏太保汝南王元悅還北為魏主。庚申,以魏尚書左僕射
 范遵為安北將軍、司州牧,隨元悅北討。林邑國遣使獻方物。壬申,扶南國遣使獻方物。秋八月庚戌,輿駕幸德陽堂,設絲竹會,祖送魏主元悅。山賊聚結,寇會稽郡所部縣。九月壬午,假超武將軍湛海珍節以討之。



 三年春正月辛巳,輿駕親祠南郊,大赦天下,孝悌力田賜爵一級。丙申,以魏尚書僕射鄭先護為征北大將軍。二月辛丑,輿駕親祠明堂。甲寅,老人星見。乙卯,特進蕭琛卒。乙丑,以廣州刺史元景隆為安右將軍。夏四月乙巳,皇太子統薨。六月丁未,以前太子詹事蕭淵猷為中護軍。尚書僕射徐勉加特進、右光祿大夫。丹丹國遣使
 獻方物。癸丑,立昭明太子子南徐州刺史華容公懽為豫章郡王,枝江公譽為河東郡王,曲阿公察為岳陽郡王。秋七月乙亥,立晉安王綱為皇太子。大赦天下,賜為父後者及出處忠孝文武清勤,並賜爵一級。乙酉,以侍中、五兵尚書謝舉為吏部尚書。庚寅,詔曰:「推恩六親,義彰九族,班以侯爵,亦曰惟允。凡是宗戚有服屬者,並可賜沐食鄉亭侯,各隨遠近以為差次。其有暱親,自依舊章。」壬辰,以吏部尚書何敬容為尚書右僕射。癸巳,老人星見。九月庚午,以太子詹事蕭淵藻為征北將軍、南兗州刺史。戊寅,狼牙脩國奉表獻方物。冬十月己酉,行幸
 同泰寺,高祖升法座,為四部眾說《大般若涅盤經》義,迄于乙卯。前樂山縣侯蕭正則有罪流徙,至是招誘亡命,欲寇廣州,在所討平之。十一月乙未,行幸同泰寺,高祖升法座,為四部從說《摩訶般若波羅蜜經》義,訖于十二月辛丑。是歲,吳興郡生野穀,堪食。



 四年春正月丙寅朔,以鎮衛大將軍、開府儀同三司南平王偉進位大司馬,司空元法僧進太尉,尚書令、中權大將軍、開府儀同三司袁昂進位司空。立臨川靖惠王宏子正德為臨賀郡王。戊辰,以丹陽尹邵陵王綸為揚州刺史。太子右衛率薛法護為平北將軍、司州牧,衛
 送元悅入洛。庚午,立嫡皇孫大器為宣城郡王。癸未,魏南兗州刺史劉世明以城降,改魏南兗州為譙州,以世明為刺史。二月壬寅,老人星見。新除太尉元法僧還北,為東魏主。以安右將軍元景隆為征北將軍、徐州刺史,雲麾將軍羊侃為安北將軍、兗州刺史,散騎常侍元樹為鎮北將軍。庚戌,新除揚州刺史邵陵王綸有罪,免為庶人。壬子,以江州刺史武陵王紀為揚州刺史,領軍將軍蕭昂為江州刺史。丙辰,邵陵縣獲白鹿一。三月庚午,侍中、領國子博士蕭子顯上表置制旨《孝經》助教一人,生十人,專通高祖所釋《孝經義》。夏四月壬申,盤盤國遣
 使獻方物。秋七月甲辰,星隕如雨。八月丙子,特進陸杲卒。九月乙巳,以太子詹事南平王世子恪為領軍將軍,平北將軍、雍州刺史廬陵王續為安北將軍,西中郎將、荊州刺史湘東王繹為平西將軍,司空袁昂領尚書令。十一月己酉,高麗國遣使獻方物。十二月庚辰,以太尉元法僧為驃騎大將軍、開府同三司之儀、郢州刺史。



 五年春正月辛卯,輿駕親祠南郊,大赦天下,孝悌力田賜爵一級。先是一日丙夜,南郊令解滌之等到郊所履行,忽聞空中有異香三隨風至,及將行事,奏樂迎神畢,有神光滿壇上,朱紫黃白雜色,食頃方滅。兼太宰武陵王
 紀等以聞。戊申,京師地震。己酉。長星見。辛亥,輿駕親祠明堂。癸丑,以宣城王大器為中軍將軍。河南國遣使獻方物。二月癸未,行幸同泰寺,設四部大會,高祖升法座,發《金字摩訶波若經》題,訖于己丑。老人星見。三月丙辰,大司馬南平王偉薨。夏四月癸酉,以御史中丞臧盾兼領軍。五月戊子,京邑大水,御道通船。六月己卯,魏建義城主蘭寶殺魏東徐州刺史,以下邳城降。秋七月辛卯,改下邳為武州。八月庚申,以前徐州刺史元景隆為安右將軍。老人星見。甲子,波斯國遣使獻方物。甲申,中護軍蕭淵猷卒。九月己亥,以輕車將軍、臨賀王正德為中
 護軍。甲寅,以尚書令、司空袁昂為特進、左光祿大夫,司空如故。盤盤國遣使獻方物。冬十月庚申,以尚書右僕射何敬容為尚書左僕射,吏部尚書謝舉為尚書右僕射,侍中、國子祭酒蕭子顯為吏部尚書。



 六年春二月癸亥,輿駕親耕籍田,大赦天下,孝悌力田賜爵一級。三月己亥,以行河南王可沓振為西秦、河二州刺史、河南王。甲辰,百濟國遣使獻方物。夏四月丁卯,熒惑在南斗。秋七月甲辰,林邑國遣使獻方物。八月己未,以南梁州刺史武興王楊紹先為秦、南秦二州刺史。冬十月丁卯,以信武將軍元慶和為鎮北將軍,率眾北
 伐。閏十二月丙午,西南有雷聲二。



 大同元年春正月戊申朔,改元,大赦天下。二月己卯,老人星見。辛巳,輿駕親祠明堂。丁亥,輿駕躬耕籍田。辛丑,高麗國、丹丹國各遣使獻方物。三月辛未,滑國王安樂薩丹王遣使獻方物。夏四月庚子,波斯國獻方物。甲辰,以魏鎮東將軍劉濟為徐州刺史。壬戌,以安北將軍廬陵王續為安南將軍、江州刺史。秋七月乙卯,老人星見。辛卯,扶南國遣使獻方物。冬十月辛卯,以前南兗州刺史蕭淵藻為護軍將軍。十一月丁未,中衛將軍、特進、右光祿大夫徐勉卒。壬戌,北梁州刺史蘭欽攻漢中,剋之,
 魏梁州刺史元羅降。癸亥,賜梁州歸附者復除有差。甲子,雄勇將軍、北益州刺史陰平王楊法深進號平北將軍。月行左角星。十二月乙酉,以魏北徐州刺史羊徽逸為平北將軍。戊戌,以平西將軍、秦、南秦二州刺史武興王楊紹先進號車騎將軍、平北將軍、北益州刺史陰平王楊法深進號驃騎將軍。辛丑,平西將軍、荊州刺史湘東王繹進號安西將軍。



 二年春正月甲辰,以兼領軍臧盾為中領軍。二月乙亥,輿駕躬耕籍田。丙戌,老人星見。三月庚申,詔曰:「政在養民,德存被物,上令如風,民應如草。朕以寡德,運屬時來,
 撥亂反正,倏焉三紀。不能使重門不閉,守在海外,疆埸多阻,車書未一。民疲轉輸,士勞邊防。徹田為糧,未得頓止。治道不明,政用多僻,百辟無沃心之言,四聰闕飛耳之聽,州輟刺舉,郡忘共治。致使失理負謗,無由聞達。侮文弄法,因事生姦,肺石空陳,懸鐘徒設。《書》不云乎:『股肱惟人,良臣惟聖。』實賴賢佐,匡其不及。凡厥在朝,各獻讜言,政治不便於民者,可悉陳之。若在四遠,刺史二千石長吏,並以奏聞。細民有言事者,咸為申達。朕將親覽,以紓其過。文武在位,舉爾所知,公侯將相,隨才擢用,拾遺補闕,勿有所隱。」夏四月乙未,以驃騎大將軍、開府同三
 司之儀元法僧為太尉,領軍師將軍。先是,尚書右丞江子四上封事,極言政治得失。五月癸卯,詔曰:「古人有言,屋漏在上,知之在下。朕所鐘過,不能自覺。江子四等封事如上,尚書可時加檢括,於民有蠹患者,便即勒停,宜速詳啟,勿致淹緩。」乙巳,以魏前梁州刺史元羅為征北大將軍、青、冀二州刺史。六月丁亥,詔曰:「南郊、明堂、陵廟等令,與朝請同班,於事為輕,可改視散騎侍郎。」冬十月乙亥,詔大舉北伐。十一月己亥,詔北伐眾班師。辛亥,京師地震。十二月壬申,魏請通和,詔許之。丁酉,以吳興太守、駙馬都尉、利亭侯張纘為吏部尚書。



 三年春正月辛丑,輿駕親祠南郊,大赦天下;孝悌力田賜爵一級。是夜,朱雀門災。壬寅,天無雲,雨灰,黃色。癸卯,以中書令邵陵王綸為江州刺史。二月乙酉,老人星見。丁亥,輿駕親耕籍田。己丑,以尚書左僕射何敬容為中權將軍,護軍將軍蕭淵藻為安右將軍、尚書左僕射。以尚書右僕射謝舉為右光祿大夫。庚寅,以安南將軍廬陵王續為中衛將軍、護軍將軍。三月戊戌,立昭明太子子嵒為武昌郡王,灊為義陽郡王。夏四月丁卯,以南瑯邪、彭城二郡太守河東王譽為南徐州刺史。五月丙申,以前揚州刺史武陵王紀復為揚州刺史。六月,青州朐
 山境隕霜。秋七月癸卯,魏遣使來聘。己酉,義陽王灊薨。是月,青州雪,害苗稼。八月甲申,老人星見。辛卯,輿駕幸阿育王寺,赦天下。九月,南兗州大饑。是月,北徐州境內旅生稻稗二千許頃。閏月甲子,安西將軍、荊州刺史湘東王繹進號鎮西將軍,揚州刺史武陵王紀為安西將軍、益州刺史。冬十月丙辰,京師地震。是歲,饑。



 四年春正月庚辰,以中軍將軍宣城王大器為中軍大將軍、揚州刺史。二月己亥,輿駕親耕籍田。三月戊寅,河南國遣使獻方物。癸未,芮芮國遣使獻方物。五月甲戌,魏遣使來聘。秋七月己未,以南瑯邪、彭城二郡太守岳
 陽王察為東揚州刺史。癸亥,詔以東冶徒李胤之降如來真形舍利,大赦天下。八月甲辰,詔「南兗、北徐、西徐、東徐、青、冀、南北青、武、仁、潼、睢等十二州,既經饑饉,曲赦逋租宿責,勿收今年三調。」冬十二月丁亥,兼國子助教皇侃表上所撰《禮記義疏》五十卷。



 五年春正月乙卯,以護軍將軍廬陵王續為驃騎將軍、開府儀同三司,安右將軍、尚書左僕射蕭淵藻為中衛將軍、開府儀同三司。中權將軍、丹陽尹何敬容以本號為尚書令,吏部尚書張纘為尚書僕射,都官尚書劉孺為吏部尚書。丁巳,御史中丞、參禮儀事賀琛奏:「今南北
 二郊及籍田往還並宜御輦,不復乘輅。二郊請用素輦,籍田往還乘常輦,皆以侍中陪乘,停大將軍及太僕。」詔付尚書博議施行。改素輦名大同輦。昭祀宗廟乘玉輦。辛未,輿駕親祠南郊,詔孝悌力田及州閭鄉黨稱為善人者,各賜爵一級,并勒屬所以時騰上。三月己未,詔曰:「朕四聽既闕,五識多蔽,畫可外牒,或致紕繆。凡是政事不便於民者,州郡縣即時皆言,勿得欺隱。如使怨訟,當境任失。而今而後,以為永准。」秋七月己卯,以驃騎將軍、開府儀同三司廬陵王續為荊州刺史,湘東王繹為護軍將軍、安右將軍。八月乙酉,扶南國遣使獻生犀及方
 物。九月庚申,以都官尚書到溉為吏部尚書。冬十一月乙亥,魏遣使來聘。十二月癸未,以吳郡太守謝舉為中書監,新除中書令鄱陽王範為中領軍。



 六年春正月庚戌朔,曲赦司、豫、徐、兗四州。二月己亥,輿駕親耕籍田。丙午,以江州刺史邵陵王綸為平西將軍、郢州刺史,雲麾將軍豫章王懽為江州刺史。秦郡獻白鹿一。夏四月癸未,詔曰:「命世興王,嗣賢傳業,聲稱不朽,人代徂遷,二賓以位,三恪義在,時事浸遠,宿草榛蕪,望古興懷,言念愴然。晉、宋、齊三代諸陵,有職司者勤加守護,勿令細民妄相侵毀。作兵有少,補使充足。前無守視,
 並可量給。」五月戊寅,以前青、冀二州刺史元羅為右光祿大夫。己卯,河南王遣使獻馬及方物。六月丁未,平陽縣獻白鹿一。秋七月丁亥,魏遣使來聘。八月戊午,赦天下。辛未,詔曰:「經國有體,必詢諸朝,所以尚書置令、僕、丞、郎,旦旦上朝,以議時事,前共籌懷,然後奏聞。頃者不爾,每有疑事,倚立求決。古人有云,主非堯舜,何得發言便是。是故放勛之聖,猶咨四岳,重華之睿,亦待多士。豈朕寡德,所能獨斷。自今尚書中有疑事,前於朝堂參議,然後啟聞,不得習常。其軍機要切,前須諮審,自依舊典。」盤盤國遣使獻方物。九月,移安州置定遠郡,受北徐州都
 督,定遠郡改屬安州。始平太守崔碩表獻嘉禾一莖十二穗。戊戌,特進、左光祿大夫、司空袁昂薨。冬十一月己卯,曲赦京邑。十二月壬子,江州刺史豫章王懽薨。以護軍將軍湘東王繹為鎮南將軍、江州刺史。置桂州於湘州始安郡,受湘州督;省南桂林等二十四郡,悉改屬桂州。



 七年春正月辛巳,輿駕親祠南郊,赦天下,其有流移及失桑梓者,各還田宅,蠲課五年。辛丑,輿駕親祠明堂。二月乙巳,以行宕昌王梁彌泰為平西將軍、河涼二州刺史、宕昌王。辛亥,輿駕躬耕籍田。乙卯,京師地震。丁巳,以
 中領軍、鄱陽王範為鎮北將軍、雍州刺史。三月乙亥,宕昌王遣使獻馬及方物。高麗、百濟、滑國各遣使獻方物。夏四月戊申,魏遣使來聘。五月癸己,以侍中南康王會理兼領軍。秋九月戊寅,芮芮國遣使獻方物。冬十月丙午,以侍中劉孺為吏部尚書。十一月丙子,詔停在所役使女丁。丁丑,詔曰:「民之多幸,國之不幸,恩澤屢加,彌長姦盜,朕亦知此之為病矣。如不優赦,非仁人之心。凡厥愆耗逋負,起今七年十一月九日昧爽以前,在民間無問多少,言上尚書,督所未入者,皆赦除之。」又詔曰:「用天之道,分地之利,蓋先聖之格訓也。凡是田桑廢宅沒入
 者,公創之外,悉以分給貧民,皆使量其所能以受田分。如聞頃者,豪家富室,多占取公田,貴價僦稅,以與貧民,傷時害政,為蠹已甚。自今公田悉不得假與豪家;已假者特聽不追。其若富室給貧民種糧共營作者,不在禁例。」己丑,以金紫光祿大夫臧盾為領軍將軍。十二月壬寅,詔曰:「古人云,一物失所,如納諸隍,未是切言也。朕寒心消志,為日久矣,每當食投箸,方眠徹枕,獨坐懷憂,憤慨申旦,非為一人,萬姓故耳。州牧多非良才,守宰虎而傅翼,楊阜是故憂憤,賈誼所以流涕。至於民間誅求萬端,或供廚帳,或供廄庫,或遣使命,或待賓客,皆無自費,
 取給於民。又復多遣遊軍,稱為遏防,姦盜不止,暴掠繁多,或求供設,或責腳步。又行劫縱,更相枉逼,良人命盡,富室財殫。此為怨酷,非止一事。亦頻禁斷,猶自未已,外司明加聽採,隨事舉奏。又復公私傳、屯、邸、冶,爰至僧尼,當其地界,止應依限守視;乃至廣加封固,越界分斷,水陸採捕,及以樵蘇,遂致細民措手無所。凡自今有越界禁斷者,禁斷之身,皆以軍法從事。若是公家創內,止不得輒自立屯,與公競作,以收私利。至百姓樵採以供煙爨者,悉不得禁。及以採捕,亦勿訶問。若不遵承,皆以死罪結正。」魏遣使來聘。丙辰,於宮城西立士林館,延集學
 者。是歲,交州土民李賁攻刺史蕭諮,諮輸賂,得還越州。



 八年春正月,安成郡民劉敬躬挾左道以反,內史蕭說委郡東奔,敬躬據郡,進攻廬陵,取豫章,妖黨遂至數萬,前逼新淦、柴桑。二月戊戌,江州刺史湘東王繹遣中兵曹子郢討之。三月戊辰,大破之,擒敬躬送京師,斬于建康市。是月,於江州新蔡、高塘立頌平屯,墾作蠻田。遣越州刺史陳侯、羅州刺史寧巨、安州刺史李智、愛州刺史阮漢,同征李賁於交州。



 九年春閏月丙申,地震,生毛。二月甲戌,使江州民三十家出奴婢一戶,配送司州。三月,以太子詹事謝舉為尚
 書僕射。夏四月,林邑王破德州,攻李賁,賁將范脩又破林邑王於九德,林邑王敗走。冬十一月辛丑,安西將軍、益州刺史武陵王紀進號征西將軍、開府儀同三司。十二月壬戌,領軍將軍臧盾卒;以輕車將軍河東王譽為領軍將軍。



 十年春正月,李賁於交址竊位號,署置百官。三月甲午,輿駕幸蘭陵,謁建寧陵。辛丑,至脩陵。壬寅,詔曰:「朕自違桑梓,五十餘載,乃眷東顧,靡日不思。今四方款關,海外有截,獄訟稍簡,國務小閑,始獲展敬園陵,但增感慟。故鄉老少,接踵遠至,情貌孜孜,若歸於父,宜有以慰其此
 心。並可錫位一階,并加頒賚。所經縣邑,無出今年租賦。監所責民,蠲復二年。并普賚內外從官軍主左右錢米各有差。」因作《還舊鄉》詩。癸卯,詔園陵職司,恭事勤勞,並錫位一階,并加頒賚。丁未,仁威將軍、南徐州刺史臨川王正義進號安東將軍。己酉,幸京口城北固樓,改名北顧。庚戌,幸回賓亭,宴帝鄉故老,及所經近縣奉迎候者少長數千人,各賚錢二千。夏四月乙卯,輿駕至自蘭陵。詔鰥寡孤獨尤貧者贍恤各有差。五月丁酉,尚書令何敬容免。秋九月己丑,詔曰:「今茲遠近,雨澤調適,其獲已及,冀必萬箱,宜使百姓因斯安樂。凡天下罪無輕重,已
 發覺未發覺,討捕未擒者,皆赦宥之。侵割耗散官物,無問多少,亦悉原除。田者荒廢、水旱不作、無當時文列,應追稅者,并作田不登公格者,並停。各備臺州以文最逋殿,罪悉從原。其有因饑逐食,離鄉去土,悉聽復業,蠲課五年。」冬十二月,大雪,平地三尺。



 十一年春三月庚辰,詔曰:「皇王在昔,澤風未遠,故端居玄扈,拱默巖廊。自大道既淪,澆波斯逝,動競日滋,情偽彌作。朕負扆君臨,百年將半。宵漏未分,躬勞政事;白日西浮,不遑飧飯。退居猶於布素,含咀匪過藜藿。寧以萬乘為貴,四海為富;唯欲億兆康寧,下民安乂。雖復三思
 行事,而百慮多失。凡遠近分置、內外條流、四方所立屯、傳、邸、冶,市埭、桁渡,津稅、田園,新舊守宰,遊軍戍邏,有不便於民者,尚書州郡各速條上,當隨言除省,以舒民患。夏四月,魏遣使來聘。冬十月己未,詔曰:「堯、舜以來,便開贖刑,中年依古,許罪身入貲,吏下因此,不無姦猾,所以一日復敕禁斷。川流難壅,人心惟危,既乖內典慈悲之義,又傷外教好生之德。《書》云:『與殺不辜,寧失不經。』可復開罪身,皆聽入贖。」



 中大同元年春正月丁未,曲阿縣建陵隧口石騏驎動,有大蛇鬥隧中,其一被傷奔走。癸丑,交州刺史楊票剋
 交趾嘉寧城,李賁竄入獠洞,交州平。三月乙巳,大赦天下:凡主守割盜、放散官物,及以軍糧器甲,凡是赦所不原者,起十一年正月以前,皆悉從恩,十一年正月已後,悉原加責;其或為事逃叛流移,因饑以後亡鄉失土,可聽復業,蠲課五年,停其徭役;其被拘之身,各還本郡,舊業若在,皆悉還之。庚戌,法駕出同泰寺大會,停寺省,講《金字三慧經》。夏四月丙戌,於同泰寺解講,設法會。大赦,改元。孝悌力田為父後者賜爵一級,賚宿衛文武各有差。是夜,同泰寺災。六月辛巳,竟天有聲,如風雨相擊薄。秋七月辛酉,以武昌王嵒為東揚州刺史。甲子,詔曰:「禽
 獸知母而不知父,無賴子弟過於禽獸,至於父母並皆不知。多觸王憲,致及老人。耆年禁執,大可傷愍。自今有犯罪者,父母祖父母勿坐。唯大逆不預今恩。」丙寅,詔曰:「朝四而暮三,眾狙皆喜,名實未虧,而喜怒為用。頃聞外間多用九陌錢,陌減則物貴,陌足則物賤,非物有貴賤,是心有顛倒。至於遠方,日更滋甚。豈直國有異政,乃至家有殊俗,徒亂王制,無益民財。自今可通用足陌錢。令書行後,百日為期,若猶有犯,男子謫運,女子質作,並同三年。」八月丁丑,東揚州刺史武昌王嵒薨。以安東將軍、南徐州刺史臨川王正義即本號東揚州刺史,丹陽尹
 邵陵王綸為鎮東將軍、南徐州刺史。甲午,渴般國遣使獻方物。冬十月癸酉,汝陰王劉哲薨。乙亥,以前東揚州刺史岳陽王察為雍州刺史。



 太清元年正月壬寅,驃騎大將軍、開府儀同三司、荊州刺史廬陵王續薨;以鎮南將軍、江州刺史湘東王繹為鎮西將軍、荊州刺史。辛酉,輿駕親祠南郊,詔曰:「天行彌綸,覆燾之功博;乾道變化,資始之德成。朕沐浴齋宮,虔恭上帝,祗事+燎,高熛太一,大禮克遂,感慶兼懷,思與億兆,同其福惠。可大赦天下,尤窮者無出即年租調;清議禁錮,並皆宥釋;所討逋叛,巧籍隱年,闇丁匿口,開恩百
 日,各令自首,不問往罪;流移他鄉,聽復宅業,蠲課五年;孝悌力田,賜爵一級;居局治事,賞勞二年。可班下遠近,博採英異,或德茂州閭,道行鄉邑,或獨行特立,不求聞達,咸使言上,以時招聘。」甲子,輿駕親祠明堂。二月己卯,白虹貫日。庚辰,魏司徒侯景求以豫、廣、潁、洛、陽、西揚、東荊、北荊、襄、東豫、南兗、西兗、齊等十三州內屬。壬午,以景為大將軍,封河南王,大行臺,制承如鄧禹故事。丁亥,輿駕躬耕籍田。三月庚子,高祖幸同泰寺,設無遮大會,舍身,公卿等以錢一億萬奉贖。甲辰,遣司州刺史羊鴉仁、兗州刺史桓和、仁州刺史湛海珍等應接北豫州。
 夏四月丁亥,輿駕還宮,大赦天下,改元,孝悌力田為父後者賜爵一級,在朝群臣宿衛文武並加頒賚。五月丁酉,輿駕幸德陽堂,宴群臣,設絲竹樂。六月戊辰,以前雍州刺史鄱陽王範為征北將軍,總督漢北征討諸軍事。秋七月庚申,羊鴉仁入懸瓠城。甲子,詔曰:「二豫分置,其來久矣。今汝、潁剋定,可依前代故事,以懸瓠為豫州,壽春為南豫,改合肥為合州,北廣陵為淮州,項城為殷州,合州為南合州。」八月乙丑,王師北伐,以南豫州刺史蕭淵明為大都督。詔曰:「今汝南新復,嵩、潁載清,瞻言遣黎,有勞鑒寐,宜覃寬惠,與之更始。應是緣邊初附諸州部
 內百姓,先有負罪流亡,逃叛入北,一皆曠蕩,不問往愆。並不得挾以私仇而相報復。若有犯者,嚴加裁問。」戊子,以大將軍侯景錄行臺尚書事。九月癸卯,王遊苑成。庚戌,輿駕幸苑。冬十一月,魏遣大將軍慕容紹宗等至寒山。丙午,大戰,淵明敗績,及北兗州刺史胡貴孫等並陷魏。紹宗進圍潼州。十二月戊辰,遣太子舍人元貞還北為魏主。辛巳,以前征北將軍鄱陽王範為安北將軍、南豫州刺史。



 二年春正月戊戌,詔在位各舉所知。己亥,魏陷渦陽。辛丑,以尚書僕射謝舉為尚書令,守吏部尚書王克為尚
 書僕射。甲辰,豫州刺史羊鴉仁,殷州刺史羊思達,並棄城走,魏進據之。乙卯,以大將軍侯景為南豫州牧,安北將軍、南豫州刺史鄱陽王範為合州刺史。三月甲辰,撫東將軍高麗王高延卒,以其息為寧東將軍、高麗王、樂浪公。己未,以鎮東將軍、南徐州刺史邵陵王綸為平南將軍、湘州刺史、同三司之儀,中衛將軍、開府儀同三司蕭淵藻為征東將軍、南徐州刺史。是日,屈獠洞斬李賁,傳首京師。夏四月丙子,詔在朝及州郡各舉清人任治民者,皆以禮送京師。戊寅,以護軍將軍河東王譽為湘州刺史。五月辛丑,以新除中書令邵陵王綸為安前將
 軍、開府儀同三司,前湘州刺史張纘為領軍將軍。辛亥,曲赦交、愛、德三州。癸丑,詔曰:「為國在於多士,寧下寄於得人。朕暗於行事,尤闕治道,孤立在上,如臨深谷。凡爾在朝,咸思匡救,獻替可否,用相啟沃。班下方岳,傍求俊乂,窮其屠釣,盡其巖穴,以時奏聞。」是月,兩月夜見。秋八月乙未,以右衛將軍朱異為中領軍。戊戌,侯景舉兵反,擅攻馬頭、木柵、荊山等戍。甲辰,以安前將軍、開府儀同三司邵陵王綸都督眾軍討景。曲赦南豫州。九月丙寅,加左光祿大夫元羅鎮右將軍。冬十月,侯景襲譙州,執刺史蕭泰。丁未,景進攻歷陽,太守莊鐵降之。戊申,以新
 除光祿大夫臨賀王正德為平北將軍,都督京師諸軍,屯丹陽郡。己酉,景自橫江濟於采石。辛亥,景師至京,臨賀王正德率眾附賊。十一月辛酉,賊攻陷東府城,害南浦侯蕭推、中軍司馬楊暾。庚辰,邵陵王綸帥武州刺史蕭弄璋、前譙州刺史趙伯超等,入援京師,頓鐘山愛敬寺。乙酉,綸進軍湖頭,與賊戰,敗績。丙戌,安北將軍鄱陽王範遣世子嗣、雄信將軍裴之高等帥眾入援,次于張公洲。十二月戊申,天西北中裂,有光如火。尚書令謝舉卒。丙辰,司州刺史柳仲禮、前衡州刺史韋粲、高州刺史李遷仕、前司州刺史羊鴉仁等並帥軍入援,推仲禮為
 大都督。



 三年春正月丁巳朔,柳仲禮帥眾分據南岸。是日,賊濟軍於青塘,襲破韋粲營,粲拒戰死。庚申,卲陵王綸、東揚州刺史臨成公大連等帥兵集南岸。乙丑,中領軍硃異卒。丙寅,以司農卿傅岐為中領軍。戊辰,高州刺史李遷仕、天門太守樊文皎進軍青溪東,為賊所破,文皎死之。壬午,熒惑守心。乙酉,太白晝見。二月丁未南兗州刺史南康王會理、前青、冀二州刺史湘潭侯蕭退帥江州之眾,頓于蘭亭苑。庚戌,安北將軍、合州刺史鄱陽王範以本號開府儀同三司。三月戊午,前司州刺史羊鴉仁等
 進軍東府北,與賊戰,大敗。己未,皇太子妃王氏薨。丁卯,賊攻陷宮城,縱兵大掠。己巳,賊矯詔遣石城公大款解外援軍。庚午,侯景自為都督中外諸軍事、大丞相、錄尚書。辛未,援軍各退散。丙子,熒惑守心。壬午,新除中領軍傅岐卒。夏四月己丑,京師地震。丙申,地又震。己酉,高祖以所求不供,憂憤寢疾。是月,青、冀二州刺史明少遐、東徐州刺史湛海珍、北青州刺史王奉伯各舉州附于魏。五月丙辰,高祖崩于凈居殿,時年八十六。辛巳,遷大行皇帝梓宮于太極前殿。冬十一月,追尊為武皇帝,廟曰高祖。乙卯,葬于脩陵。



 高祖生知淳孝。年六歲,獻皇太后
 崩,水漿不入口三日,哭泣哀苦,有過成人,內外親黨,咸加敬異。及丁文皇帝憂,時為齊隨王諮議,隨府在荊鎮,仿佛奉聞,便投劾星馳,不復寢食,倍道就路,憤風驚浪,不暫停止。高祖形容本壯,及還至京都,銷毀骨立,親表士友,不復識焉。望宅奉諱,氣絕久之,每哭輒歐血數升。服內不復嘗米,惟資大麥,日止二溢。拜掃山陵,涕淚所灑,松草變色。及居帝位,即於鐘山造大愛敬寺,青溪邊造智度寺,又於臺內立至敬等殿。又立七廟堂,月中再過,設凈饌。每至展拜,恒涕泗滂沲,哀動左右。加以文思欽明,能事畢究,少而篤學,洞達儒玄。雖萬機多務,猶卷
 不輟手,燃燭側光,常至戊夜。造《制旨孝經義》,《周易講疏》,及六十四卦、二《繫》、《文言》、《序卦》等義,《樂社義》,《毛詩答問》,《春秋答問》,《尚書大義》,《中庸講疏》,《孔子正言》,《老子講疏》,凡二百餘卷,並正先儒之迷,開古聖之旨。王侯朝臣皆奉表質疑,高祖皆為解釋。脩飾國學,增廣生員,立五館,置《五經》博士。天監初,則何佟之、賀蒨、嚴植之、明山賓等覆述制旨,并撰吉凶軍賓嘉五禮,凡一千餘卷,高祖稱制斷疑。於是穆穆恂恂,家知禮節。大同中,於臺西立士林館,領軍朱異、太府卿賀琛、舍人孔子袂等遞相講述。皇太子、宣城王亦於東宮宣猷堂及揚州廨開講,於是四方
 郡國,趨學向風,雲集於京師矣。兼篤信正法,尤長釋典,製《涅盤》、《大品》、《凈名》、《三慧》諸經義記,復數百卷。聽覽餘閑,即於重雲殿及同泰寺講說,名僧碩學,四部聽眾,常萬餘人。又造《通史》,躬製贊序,凡六百卷。天情睿敏,下筆成章,千賦百詩,直疏便就,皆文質彬彬,超邁今古。詔銘贊誄,箴頌箋奏,爰初在田,洎登寶歷,凡諸文集,又百二十卷。六藝備閑,棋登逸品,陰陽緯候,卜筮占決,並悉稱善。又撰《金策》三十卷。草隸尺牘,騎射弓馬,莫不奇妙。勤於政務,孜孜無怠。每至冬月,四更竟,即敕把燭看事,執筆觸寒,手為皴裂。糾姦擿伏,洞盡物情,常哀矜涕泣,然後
 可奏。日止一食,膳無鮮腴,惟豆羹糲食而已。庶事繁擁,日儻移中,便嗽口以過。身衣布衣,木綿皁帳,一冠三載,一被二年。常克儉於身,凡皆此類。五十外便斷房室。後宮職司,貴妃以下,六宮褘褕三翟之外,皆衣不曳地,傍無錦綺。不飲酒,不聽音聲,非宗廟祭祀、大會饗宴及諸法事,未嘗作樂。性方正,雖居小殿暗室,恒理衣冠,小坐押衣要,盛夏暑月,未嘗褰袒。不正容止,不與人相見,雖覿內豎小臣,亦如遇大賓也。歷觀古昔帝王人君,恭儉莊敬,藝能博學,罕或有焉。



 史臣曰:齊季告終,君臨昏虐,天棄神怒,眾叛親離。高祖
 英武睿哲,義起樊、鄧,仗旗建號,濡足救焚,總蒼兕之師,翼龍豹之陣,雲驤雷駭,剪暴夷凶,萬邦樂推,三靈改卜。於是御鳳歷,握龍圖,闢四門,弘招賢之路,納十亂,引諒直之規。興文學,脩郊祀,治五禮,定六律,四聰既達,萬機斯理,治定功成,遠安邇肅。加以天祥地瑞,無絕歲時。征賦所及之鄉,文軌傍通之地,南超萬里,西拓五千。其中瑰財重寶,千夫百族,莫不充牣王府,蹶角闕庭。三四十年,斯為盛矣。自魏、晉以降,未或有焉。及乎耄年,委事群倖。然朱異之徒,作威作福,挾朋樹黨,政以賄成,服冕乘軒,由其掌握,是以朝經混亂,賞罰無章。「小人道長」,抑此
 之謂也。賈誼有云「可為慟哭者矣」。遂使滔天羯寇,承間掩襲,鷲羽流王屋,金契辱乘輿,塗炭黎元,黍離宮室。嗚呼!天道何其酷焉。雖歷數斯窮,蓋亦人事然
 也。



\end{pinyinscope}