\article{卷第十五列傳第九 謝朏 弟子覽}

\begin{pinyinscope}

 謝朏,字敬沖,陳郡陽夏人也。祖弘微,宋太常卿,父莊,右光祿大夫,並有名前代。朏幼聰慧,莊器之,常置左右。年十歲,能屬文。莊遊土山賦詩,使朏命篇,朏攬筆便就。瑯邪王景文謂莊曰:「賢子足稱神童,復為後來特達。」莊笑,因撫朏背曰:「真吾家千金。」孝武帝遊姑孰,敕莊攜朏從駕,詔使為《洞井贊》,於坐奏之。帝曰:「雖小,奇童也。」起家撫
 軍法曹行參軍,遷太子舍人,以父憂去職。服闋,復為舍人,歷中書郎,衛將軍袁粲長史。粲性簡峻,罕通賓客,時人方之李膺。朏謁既退,粲曰:「謝令不死。」尋遷給事黃門侍郎。出為臨川內史,以賄見劾,案經袁粲,粲寢之。



 齊高帝為驃騎將軍輔政,選朏為長史,敕與河南褚炫、濟陽江斅、彭城劉俁俱入侍宋帝,時號為天子四友。續拜侍中,并掌中書、散騎二省詔冊。高帝進太尉,又以朏為長史,帶南東海太守。高帝方圖禪代,思佐命之臣,以朏有重名,深所欽屬。論魏、晉故事,因曰:「晉革命時事久兆,石苞不早勸晉文,死方慟哭,方之馮異,非知機也。」朏答曰:「
 昔魏臣有勸魏武即帝位者,魏武曰:『如有用我,其為周文王乎!』晉文世事魏氏,將必身終北面;假使魏早依唐虞故事,亦當三讓彌高。」帝不悅。更引王儉為左長史,以朏侍中,領秘書監。及齊受禪,朏當日在直,百僚陪位,侍中當解璽,朏佯不知,曰:「有何公事?」傳詔云:「解璽授齊王。」朏曰:「齊自應有侍中。」乃引枕臥。傳詔懼,乃使稱疾,欲取兼人。朏曰:「我無疾,何所道。」遂朝服,步出東掖門,乃得車,仍還宅。是日遂以王儉為侍中解璽。既而武帝言於高帝,請誅朏。帝曰:「殺之則遂成其名,正應容之度外耳。」遂廢于家。



 永明元年,起家拜通直散騎常侍,累遷侍中,領
 國子博士。五年,出為冠軍將軍、義興太守,加秩中二千石。在郡不省雜事,悉付綱紀,曰:「吾不能作主者吏,但能作太守耳。」視事三年,徵都官尚書、中書令。隆昌元年,復為侍中,領新安王師。未拜,固求外出。仍為征虜將軍、吳興太守,受召便述職。時明帝謀入嗣位,朝之舊臣皆引參謀策。朏內圖止足,且實避事。弟綍,時為吏部尚書。朏至郡,致綍數斛酒,遺書曰:「可力飲此,勿豫人事。」朏居郡每不治,而常務聚斂,眾頗譏之,亦不屑也。



 建武四年,詔徵為侍中、中書令,遂抗表不應召。遣諸子還京師,獨與母留,築室郡之西郭。明帝下詔曰:「夫超然榮觀,風流自
 遠;蹈彼幽人,英華罕值。故長揖楚相,見稱南國;高謝漢臣,取貴良史。新除侍中、中書令朏,早藉羽儀,夙標清尚,登朝樹績,出守馳聲。遂斂跡康衢,拂衣林沚,抱箕潁之餘芳,甘憔悴而無悶。撫事懷人,載留欽想。宜加優禮,用旌素概。可賜床帳褥席,俸以卿祿,常出在所。」時國子祭酒廬江何胤亦抗表還會稽。永元二年,詔徵朏為散騎常侍、中書監,胤為散騎常侍、太常卿,並不屈。三年,又詔徵朏為侍中、太子少傅,胤散騎常侍、太子詹事。時東昏皆下在所,使迫遣之,值義師已近,故並得不到。



 及高祖平京邑,進位相國,表請朏、胤曰:「夫窮則獨善,達以兼濟。
 雖出處之道,其揆不同,用捨惟時,賢哲是蹈。前新除侍中、太子少傅朏,前新除散騎常侍、太子詹事、都亭侯胤,羽儀世胄,徽猷冠冕,道業德聲,康濟雅俗。昔居朝列,素無宦情,賓客簡通,公卿罕預,簪紱未褫,而風塵擺落。且文宗儒肆,互居其長;清規雅裁,兼擅其美。並達照深識,預睹亂萌,見庸質之如初,知貽厥之無寄。拂衣東山,眇絕塵軌。雖解組昌運,實避昏時。家膺鼎食,而甘茲橡艾;世襲青紫,而安此懸鶉。自澆風肇扇,用南成俗,淳流素軌,餘烈頗存。誰其激貪,功歸有道,康俗振民,朝野一致。雖在江海,而勳同魏闕。今泰運甫開,賤貧為恥;況乎久
 蘊瑚璉,暫厭承明,而可得求志海隅,永追松子。臣負荷殊重,參贊萬機,實賴群才,共成棟幹。思挹清源,取鏡止水。愚欲屈居僚首,朝夕諮諏,庶足以翼宣寡薄,式是王度。請並補臣府軍諮祭酒,朏加後將軍。」並不至。



 高祖踐阼,徵朏為侍中、左光祿大夫、開府儀同三司,胤散騎常侍、特進、右光祿大夫,又並不屈。仍遣領軍司馬王果宣旨敦譬。明年六月,朏輕舟出,詣闕自陳。既至,詔以為侍中、司徒、尚書令。朏辭腳疾不堪拜謁,乃角巾肩輿,詣雲龍門謝。詔見於華林園,乘小車就席。明旦,輿駕出幸朏宅,宴語盡懽。朏固陳本志,不許;因請自還東迎母,乃許之。
 臨發,輿駕復臨幸,賦詩餞別。王人送迎,相望於道。到京師,敕材官起府於舊宅,高祖臨軒,遣謁者於府拜授,詔停諸公事及朔望朝謁。



 三年元會,詔朏乘小輿升殿。其年,遭母憂,尋有詔攝職如故。後五年,改授中書監、司徒、衛將軍,並固讓不受。遣謁者敦授,乃拜受焉。是冬薨於府,時年六十六。輿駕出臨哭,詔給東園祕器,朝服一具,衣一襲,錢十萬,布百匹,蠟百斤。贈侍中、司徒。謚曰靖孝。朏所著書及文章,並行於世。



 子諼,官至司徒右長史,坐殺牛免官,卒於家。次子絪,頗有文才,仕至晉安太守,卒官。



 覽字景滌,朏弟綍之子也。選尚齊錢唐公主,拜駙馬都尉、秘書郎、太子舍人。高祖為大司馬,召補東閣祭酒,遷相國戶曹。天監元年,為中書侍郎,掌吏部事,頃之即真。



 覽為人美風神,善辭令,高祖深器之。嘗侍座,受敕與侍中王暕為詩答贈。其文甚工。高祖善之,仍使重作,復合旨。乃賜詩云:「雙文既後進,二少實名家;豈伊止棟隆,信乃俱國華。」以母憂去職。服闋,除中庶子,又掌吏部郎事,尋除吏部郎,遷侍中。覽頗樂酒,因宴席與散騎常侍蕭琛辭相詆毀,為有司所奏。高祖以覽年少不直,出為中權長史。頃之,敕掌東宮管記,遷明威將軍、新安太守。



 九
 年夏,山賊吳承伯破宣城郡,餘黨散入新安,叛吏鮑敘等與合,攻沒黟、歙諸縣,進兵擊覽。覽遣郡丞周興嗣於錦沙立塢拒戰,不敵,遂棄郡奔會稽。臺軍平山寇,覽復還郡,左遷司徒諮議參軍、仁威長史、行南徐州事,五兵尚書。尋遷吏部尚書。覽自祖至孫,三世居選部,當世以為榮。



 十二年春,出為吳興太守。中書舍人黃睦之家居烏程,子弟專橫,前太守皆折節事之。覽未到郡,睦之子弟來迎,覽逐去其船,杖吏為通者。自是睦之家杜門不出,不敢與公私關通。郡境多劫,為東道患,覽下車肅然,一境清謐。初,齊明帝及覽父綍、東海徐孝嗣,並為吳興,
 號稱名守,覽皆欲過之。昔覽在新安頗聚斂,至是遂稱廉潔,時人方之王懷祖。卒於官,時年三十七。詔贈中書令。子罕,早卒。



 陳吏部尚書姚察曰:謝朏之於宋代,蓋忠義者歟?當齊建武之世,拂衣止足,永元多難,確然獨善,其疏、蔣之流乎。洎高祖龍興,旁求物色,角巾來仕,首陟臺司,極出處之致矣!覽終能善政,君子韙之。



\end{pinyinscope}