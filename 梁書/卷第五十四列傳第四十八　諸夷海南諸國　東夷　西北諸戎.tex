\article{卷第五十四列傳第四十八 諸夷海南諸國 東夷 西北諸戎}

\begin{pinyinscope}

 海南諸國,大抵在交州南及西南大海洲上,相去近者三五千里,遠者二三萬里,其西與西域諸國接。漢元鼎中,遣伏波將軍路博德開百越,置日南郡。其徼外諸國,自武帝以來皆朝貢。後漢桓帝世,大秦、天竺皆由此道遣使貢獻。及吳孫權時,遣宣化從事朱應、中郎康泰通
 焉。其所經及傳聞,則有百數十國,因立記傳。晉代通中國者蓋鮮,故不載史官。及宋、齊,至者有十餘國,始為之傳。自梁革運,其奉正朔,脩貢職,航海歲至,踰於前代矣。今採其風俗粗著者,綴為《海南傳》云。



 林邑國者,本漢日南郡象林縣,古越裳之界也。伏波將軍馬援開漢南境,置此縣。其地縱廣可六百里,城去海百二十里,去日南界四百餘里,北接九德郡。其南界,水步道二百餘里,有西國夷亦稱王,馬援植兩銅柱表漢界處也。其國有金山,石皆赤色,其中生金。金夜則出飛,狀如螢火。又出玳瑁、貝齒、吉貝、沉木香。吉貝者,樹名也,
 其華成時如鵝毳,抽其緒紡之以作布,潔白與籥布不殊,亦染成五色,織為斑布也。沉木者,土人斫斷之,積以歲年,朽爛而心節獨在,置水中則沉,故名曰沉香。次不沉不浮者,曰祼香也。



 漢末大亂,功曹區達,殺縣令自立為王。傳數世,其後王無嗣,立外甥范熊。熊死,子逸嗣。晉成帝咸康三年,逸死,奴文篡立。文本日南西捲縣夷帥范稚家奴,常牧牛於山澗,得鱧魚二頭,化而為鐵,因以鑄刀。鑄成,文向石而咒曰:「若斫石破者,文當王此國。」因舉刀斫石,如斷芻槁,文心獨異之。范稚常使之商賈至林邑,因教林邑王作宮室及兵車器械,王寵任之。後乃
 讒王諸子,各奔餘國。及王死無嗣,文偽於鄰國迓王子,置毒於漿中而殺之,遂脅國人自立。舉兵攻旁小國,皆吞滅之,有眾四五萬人。



 時交州刺史姜莊使所親韓戢、謝稚,前後監日南郡,並貪殘,諸國患之。穆帝永和三年,臺遣夏侯覽為太守,侵刻尤甚。林邑先無田土,貪日南地肥沃,常欲略有之,至是,因民之怨,遂舉兵襲日南,殺覽,以其屍祭天。留日南三年,乃還林邑。交州刺史朱籓後遣督護劉雄戍日南,文復屠滅之。進寇九德郡,殘害吏民。遣使告籓,願以日南北境橫山為界,籓不許,又遣督護陶緩、李衢討之。文歸林邑,尋復屯日南。五年,文死,
 子佛立,猶屯日南。征西將軍桓溫遣督護滕畯、九真太守灌邃帥交、廣州兵討之,佛嬰城固守。邃令畯盛兵於前,邃帥勁卒七百人,自後踰壘而入,佛眾驚潰奔走,邃追至林邑,佛乃請降。哀帝昇平初,復為寇暴,刺史溫放之討破之。安帝隆安三年,佛孫須達復寇日南,執太守炅源,又進寇九德,執太守曹炳。交趾太守杜瑗遣都護鄧逸等擊破之,即以瑗為刺史。義熙三年,須達復寇日南,殺長史,瑗遣海邏督護阮斐討破之,斬獲甚眾。九年,須達復寇九真,行郡事杜慧期與戰,斬其息交龍王甄知及其將范健等,生俘須達息能,及虜獲百餘人。自瑗
 卒後,林邑無歲不寇日南、九德諸郡,殺蕩甚多,交州遂致虛弱。



 須達死,子敵真立,其弟敵鎧攜母出奔。敵真追恨不能容其母弟,捨國而之天竺,禪位於其甥,國相藏膋固諫不從。其甥既立而殺藏膋,藏膋子又攻殺之,而立敵鎧同母異父之弟曰文敵。文敵後為扶南王子當根純所殺,大臣范諸農平其亂,而自立為王。諸農死,子陽邁立。宋永初二年,遣使貢獻,以陽邁為林邑王。陽邁死,子咄立,慕其父,復曰陽邁。



 其國俗:居處為閣,名曰于蘭,門戶皆北向;書樹葉為紙;男女皆以橫幅吉貝繞腰以下,謂之干漫,亦曰都縵;穿耳貫小鐶;貴者著革屣,
 賤者跣行。自林邑、扶南以南諸國皆然也。其王著法服,加瓔珞,如佛像之飾。出則乘象,吹螺擊鼓,罩吉貝傘,以吉貝為幡旗。國不設刑法,有罪者使象踏殺之。其大姓號婆羅門。嫁娶必用八月,女先求男,由賤男而貴女也。同姓還相婚姻,使婆羅門引婿見婦,握手相付,咒曰「吉利吉利」,以為成禮。死者焚之中野,謂之火葬。其寡婦孤居,散髮至老。國王事尼乾道,鑄金銀人像,大十圍。



 元嘉初,陽邁侵暴日南、九德諸郡,交州刺史杜弘文建牙欲討之,聞有代乃止。八年,又寇九德郡,入四會浦口,交州刺史阮彌之遣隊主相道生帥兵赴討,攻區栗城不剋,
 乃引還。爾後頻年遣使貢獻,而寇盜不已。二十三年,使交州刺史檀和之、振武將軍宗愨伐之。和之遣司馬蕭景憲為前鋒,陽邁聞之懼,欲輸金一萬斤,銀十萬斤,還所略日南民戶,其大臣幰僧達諫止之,乃遣大帥范扶龍戍其北界區慄城。景憲攻城,剋之,斬扶龍首,獲金銀雜物,不可勝計。乘勝徑進,即剋林邑。陽邁父子並挺身逃奔。獲其珍異,皆是未名之寶。又銷其金人,得黃金數十萬斤。和之後病死,見胡神為祟。



 孝武建元、大明中,林邑王范神成累遣長史奉表貢獻。明帝泰豫元年,又遣使獻方物。齊永明中,范文贊累遣使貢獻。天監九年,文
 贊子天凱奉獻白猴,詔曰:「林邑王范天凱介在海表,乃心款至,遠脩職貢,良有可嘉。宜班爵號,被以榮澤。可持節、督緣海諸軍事、威南將軍、林邑王。」十年、十三年,天凱累遣使獻方物。俄而病死,子弼毳跋摩立,奉表貢獻。普通七年,王高式勝鎧遣使獻方物,詔以為持節、督緣海諸軍事、綏南將軍、林邑王。大通元年,又遣使貢獻。中大通二年,行林邑王高式律羅跋摩遣使貢獻,詔以為持節、督緣海諸軍事、綏南將軍、林邑王。六年,又遣使獻方物。



 扶南國,在日南郡之南海西大灣中,去日南可七千里,
 在林邑西南三千餘里。城去海五百里。有大江廣十里,西北流,東入於海。其國輪廣三千餘里,土地洿下而平博,氣候風俗大較與林邑同。出金、銀、銅、錫、沉木香、象牙、孔翠、五色鸚鵡。



 其南界三千餘里有頓遜國,在海崎上,地方千里,城去海十里。有五王,並羈屬扶南。頓遜之東界通交州,其西界接天竺、安息徼外諸國,往還交市。所以然者,頓遜迴入海中千餘里,漲海無崖岸,船舶未曾得徑過也。其市,東西交會,日有萬餘人。珍物寶貨,無所不有。又有酒樹,似安石榴,采其花汁停甕中,數日成酒。



 頓遜之外,大海洲中,又有毘騫國,去扶南八千里。傳其
 王身長丈二,頭長三尺,自古來不死,莫知其年。王神聖,國中人善惡及將來事,王皆知之,是以無敢欺者。南方號曰長頸王。國俗,有室屋、衣服,啖粳米。其人言語,小異扶南。有山出金,金露生石上,無所限也。國法刑罪人,並於王前啖其肉。國內不受估客,有往者亦殺而啖之,是以商旅不敢至。王常樓居,不血食,不事鬼神。其子孫生死如常人,唯王不死。扶南王數遣使與書相報答,常遺扶南王純金五十人食器,形如圓盤,又如瓦塸,名為多羅,受五升,又如碗者,受一升。王亦能作天竺書,書可三千言,說其宿命所由,與佛經相似,並論善事。



 又傳扶南
 東界即大漲海,海中有大洲,洲上有諸薄國,國東有馬五洲。復東行漲海千餘里,至自然大洲。其上有樹生火中,洲左近人剝取其皮,紡績作布,極得數尺以為手巾,與焦麻無異而色微青黑;若小垢洿,則投火中,復更精潔。或作燈炷,用之不知盡。



 扶南國俗本裸體,文身被髮,不制衣裳。以女人為王,號曰柳葉。年少壯健,有似男子。其南有徼國,有事鬼神者字混填,夢神賜之弓,乘賈人舶入海。混填晨起即詣廟,於神樹下得弓,便依夢乘船入海,遂入扶南外邑。柳葉人眾見舶至,欲取之,混填即張弓射其舶,穿度一面,矢及侍者,柳葉大懼,舉眾降混填。
 混填乃教柳葉穿布貫頭,形不復露,遂治其國,納柳葉為妻,生子分王七邑。其後王混盤況以詐力間諸邑,令相疑阻,因舉兵攻并之,乃遣子孫中分治諸邑,號曰小王。



 盤況年九十餘乃死,立中子盤盤,以國事委其大將范蔓。盤盤立三年死,國人共舉蔓為王。蔓勇健有權略,復以兵威攻伐旁國,咸服屬之,自號扶南大王。乃治作大船,窮漲海,攻屈都昆、九稚、典孫等十餘國,開地五六千里。次當伐金鄰國,蔓遇疾,遣太子金生代行。蔓姊子旃,時為二千人將,因篡蔓自立,遣人詐金生而殺之。蔓死時,有乳下兒名長,在民間,至年二十,乃結國中壯士
 襲殺旃,旃大將范尋又殺長而自立。更繕治國內,起觀閣遊戲之,朝旦中晡三四見客。民人以焦蔗龜鳥為禮。國法無牢獄。有罪者,先齋戒三日,乃燒斧極赤,令訟者捧行七步。又以金鐶、雞卵投沸湯中,令探取之,若無實者,手即焦爛,有理者則不。又於城溝中養鱷魚,門外圈猛獸,有罪者,輒以喂猛獸及鱷魚,魚獸不食為無罪,三日乃放之。鱷大者長二丈餘,狀如鼉,有四足,喙長六七尺,兩邊有齒,利如刀劍,常食魚,遇得麞鹿及人亦啖之,蒼梧以南及外國皆有之。



 吳時,遣中郎康泰、宣化從事硃應使於尋國,國人猶裸,唯婦人著貫頭。泰、應謂曰:「國
 中實佳,但人褻露可怪耳。」尋始令國內男子著橫幅。橫幅,今干漫也。大家乃截錦為之,貧者乃用布。



 晉武帝太康中,尋始遣使貢獻。穆帝升平元年,王竺旃檀奉表獻馴象。詔曰:「此物勞費不少,駐令勿送。」其後王憍陳如,本天竺婆羅門也。有神語曰「應王扶南」,憍陳如心悅,南至盤盤,扶南人聞之,舉國欣戴,迎而立焉。復改制度,用天竺法。



 憍陳如死,後王持梨陀跋摩,宋文帝世奉表獻方物。齊永明中,王闍邪跋摩遣使貢獻。



 天監二年,跋摩復遣使送珊瑚佛像,并獻方物。詔曰:「扶南王憍陳如闍邪跋摩,介居海表,世纂南服,厥誠遠著,重譯獻賝。宜蒙酬
 納,班以榮號。可安南將軍、扶南王。」



 今其國人皆醜黑,拳髮。所居不穿井,數十家共一池引汲之。俗事天神,天神以銅為像,二面者四手,四面者八手,手各有所持,或小兒,或鳥獸,或日月。其王出入乘象,嬪侍亦然。王坐則偏踞翹膝,垂左膝至地,以白疊敷前,設金盆香爐於其上。國俗,居喪則剃除鬚髮。死者有四葬:水葬則投之江流,火葬則焚為灰燼,土葬則瘞埋之,鳥葬則棄之中野。人性貪吝,無禮義,男女恣其奔隨。



 十年、十三年,跋摩累遣使貢獻。其年死,庶子留陀跋摩殺其嫡弟自立。十六年,遣使竺當抱老奉表貢獻。十八年,復遣使送天竺旃檀
 瑞像、婆羅樹葉,并獻火齊珠、鬱金、蘇合等香。普通元年、中大通二年、大同元年,累遣使瑞獻方物。五年,復遣使獻生犀。又言其國有佛髮,長一丈二尺,詔遣沙門釋雲寶隨使往迎之。



 先是,三年八月,高祖改造阿育王寺塔,出舊塔下舍利及佛爪髮。髮青紺色,眾僧以手伸之,隨手長短,放之則旋屈為蠡形。案《僧伽經》云:「佛髮青而細,猶如藕莖絲。」《佛三昧經》云:「我昔在宮沐頭,以尺量髮,長一丈二尺,放已右旋,還成蠡文。」則與高祖所得同也。阿育王即鐵輪王,王閻浮提,一天下,佛滅度後,一日一夜,役鬼神造八萬四千塔,此即其一也。吳時有尼居其地,為
 小精舍,孫綝尋毀除之,塔亦同泯。吳平後,諸道人復於舊處建立焉。晉中宗初渡江,更脩飾之。至簡文咸安中,使沙門安法師程造小塔,未及成而亡,弟子僧顯繼而修立。至孝武太元九年,上金相輪及承露。



 其後西河離石縣有胡人劉薩何遇疾暴亡,而心下猶暖,其家未敢便殯,經十日更蘇。說云:「有兩吏見錄,向西北行,不測遠近,至十八地獄,隨報重輕,受諸楚毒。見觀世音語云:『汝緣未盡,若得活,可作沙門。洛下、齊城、丹陽、會稽並有阿育王塔,可往禮拜。若壽終,則不墮地獄。』語竟,如墮高巖,忽然醒寤。」因此出家,名慧達。遊行禮塔,次至丹陽,未知
 塔處,乃登越城四望,見長千里有異氣色,因就禮拜,果是阿育王塔所,屢放光明。由是定知必有舍利,乃集眾就掘之,入一丈,得三石碑,並長六尺。中一碑有鐵函,函中有銀函,函中又有金函,盛三舍利及爪髮各一枚,髮長數尺。即遷舍利近北,對簡文所造塔西,造一層塔。十六年,又使沙門僧尚伽為三層,即高祖所開者也。初穿土四尺,得龍窟及昔人所捨金銀鐶釧釵鑷等諸雜寶物。可深九尺許,方至石磉,磉下有石函,函內有鐵壺,以盛銀坩,坩內有金鏤罌,盛三舍利,如粟粒大,圓正光潔。函內又有琉璃碗,內得四舍利及髮爪,爪有四枚,並為沉
 香色。至其月二十七日,高祖又到寺禮拜,設無捴大會,大赦天下。是日,以金缽盛水泛舍利,其最小者隱缽不出,高祖禮數十拜,舍利乃於缽內放光,旋回久之,乃當缽中而止。高祖問大僧正慧念:「今日見不可思議事不?」慧念答曰:「法身常住,湛然不動。」高祖曰:「弟子欲請一舍利還臺供養。」至九月五日,又於寺設無捴大會,遣皇太子王侯朝貴等奉迎。是日,風景明和,京師傾屬,觀者百數十萬人。所設金銀供具等物,並留寺供養,并施錢一千萬為寺基業。至四年九月十五日,高祖又至寺設無捴大會,豎二剎,各以金罌,次玉罌,重盛舍利及爪髮,內
 七寶塔中。又以石函盛寶塔,分入兩剎下,及王侯妃主百姓富室所捨金、銀、鐶、釧等珍寶充積。十一年十一月二日,寺僧又請高祖於寺發《般若經》題,爾夕二塔俱放光明,敕鎮東將軍邵陵王綸製寺《大功德碑》文。



 先是,二年,改造會稽鄮縣塔,開舊塔出舍利,遣光宅寺釋敬脫等四僧及舍人孫照暫迎還臺,高祖禮拜竟,即送還縣,入新塔下,此縣塔亦是劉薩何所得也。



 晉咸和中,丹陽尹高悝行至張侯橋,見浦中五色光長數尺,不知何怪,乃令人於光處掊視之,得金像,未有光趺。悝乃下車,載像還,至長干巷首,牛不肯進,悝乃令馭人任牛所之。牛
 徑牽車至寺,悝因留像付寺僧。每至中夜,常放光明,又聞空中有金石之響。經一歲,捕魚人張係世,於海口忽見有銅花趺浮出水上,係世取送縣,縣以送臺,乃施像足,宛然合。會簡文咸安元年,交州合浦人董宗之採珠沒水,於底得佛光艷,交州押送臺,以施像,又合焉。自咸和中得像,至咸安初,歷三十餘年,光趺始具。



 初,高悝得像後,西域胡僧五人來詣悝,曰:「昔於天竺得阿育王造像,來至鄴下,值胡亂,埋像於河邊,今尋覓失所。」五人嘗一夜俱夢見像曰:「已出江東,為高悝所得。」悝乃送此五僧至寺,見像噓欷涕泣,像便放光,照燭殿宇。又瓦官寺
 慧邃欲模寫像形,寺主僧尚慮虧損金色,謂邃曰:「若能令像放光,回身西向,乃可相許。」慧邃便懇到拜請,其夜像即轉坐放光,回身西向,明旦便許模之。像趺先有外國書,莫有識者,後有三藏冉阜求跋摩識之,云是阿育王為第四女所造也。及大同中,出舊塔舍利,敕市寺側數百家宅地,以廣寺域,造諸堂殿并瑞像周回閣等,窮於輪奐焉。其圖諸經變,並吳人張繇運手。繇,丹青之工,一時冠絕。



 盤盤國,宋文帝元嘉,孝武孝建、大明中,並遣使貢獻。大通元年,其王使使奉表曰:「揚州閻浮提震旦天子:萬善
 莊嚴,一切恭敬,猶如天凈無雲,明耀滿目;天子身心清凈,亦復如是。道俗濟濟,並蒙聖王光化,濟度一切,永作舟航,臣聞之慶善。我等至誠敬禮常勝天子足下,稽首問訊。今奉薄獻,願垂哀受。」中大通元年五月,累遣使貢牙像及塔,并獻沉檀等香數十種。六年八月,復使送菩提國真舍利及畫塔,并獻菩提樹葉、詹糖等香。



 丹丹國,中大通二年,其王遣使奉表曰:「伏承聖主至德仁治,信重三寶,佛法興顯,眾僧殷集,法事日盛,威嚴整肅。朝望國執,慈愍蒼生,八方六合,莫不歸服。化鄰諸天,非可言喻。不任慶善,若暫奉見尊足。謹奉送牙像及塔
 各二軀,并獻火齊珠、吉貝、雜香藥等。」大同元年,復遣使獻金、銀、琉璃、雜寶、香、藥等物。



 乾陀利國,在南海洲上。其俗與林邑、扶南略同。出班布、吉貝、檳榔,檳榔特好,為諸國之極。宋孝武世,王釋婆羅冉阜憐陀遣長史竺留陀獻金銀寶器。



 天監元年,其王瞿曇脩跋陀羅以四月八日夢見一僧,謂之曰:「中國今有聖主,十年之後,佛法大興。汝若遣使貢奉敬禮,則土地豊樂,商旅百倍;若不信我,則境土不得自安。」脩跋陀羅初未能信,既而又夢此僧曰:「汝若不信我,當與汝往觀之。」乃於夢中來至中國,拜覲天子。既覺,心異之。陀羅
 本工畫,乃寫夢中所見高祖容質,飾以丹青,仍遣使並畫工奉表獻玉盤等物。使人既至,模寫高祖形以還其國,比本畫則符同焉。因盛以寶函,日加禮敬。後跋陀死,子毘邪跋摩立。十七年,遣長史毘員跋摩奉表曰:「常勝天子陛下:諸佛世尊,常樂安樂,六通三達,為世間尊,是名如來。應供正覺,遺形舍利,造諸塔像,莊嚴國土,如須彌山。邑居聚落,次第羅滿,城郭館宇,如忉利天宮。具足四兵,能伏怨敵。國土安樂,無諸患難,人民和善,受化正法,慶無不通。猶處雪山,流注雪水,八味清凈,百川洋溢,周回屈曲,順趨大海,一切眾生,咸得受用。於諸國土,殊
 勝第一,是名震旦。大梁揚都天子,仁廕四海,德合天心,雖人是天,降生護世,功德寶藏,救世大悲,為我尊生,威儀具足。是故至誠敬禮天子足下,稽首問訊。奉獻金芙蓉、雜香、藥等,願垂納受。」普通元年,復遣使獻方物。



 狼牙脩國,在南海中。其界東西三十日行,南北二十日行,去廣州二萬四千里。土氣物產與扶南略同,偏多祼沉婆律香等。其俗男女皆袒而被髮,以吉貝為乾縵。其王及貴臣乃加雲霞布覆胛,以金繩為絡帶,金鐶貫耳。女子則被布,以瓔珞繞身。其國累磚為城,重門樓閣。王出乘象,有幡毦旗鼓,罩白蓋,兵衛甚設。國人說,立國以
 來四百餘年,後嗣衰弱,王族有賢者,國人歸之。王聞知,乃加囚執,其鏁無故自斷,王以為神,因不敢害,乃斥逐出境,遂奔天竺,天竺妻以長女。俄而狼牙王死,大臣迎還為王。二十餘年死,子婆伽達多立。天監十四年,遣使阿撤多奉表曰:「大吉天子足下:離淫怒癡,哀愍眾生,慈心無量。端嚴相好,身光明朗,如水中月,普照十方。眉間白毫,其白如雪,其色照曜,亦如月光。諸天善神之所供養,以垂正法寶,梵行眾增,莊嚴都邑。城閣高峻,如乾山。樓觀羅列,道途平正。人民熾盛,快樂安穩。著種種衣,猶如天服。於一切國,為極尊勝。天王愍念群生,民人安
 樂,慈心深廣,律儀清凈,正法化治,供養三寶,名稱宣揚,布滿世界,百姓樂見,如月初生。譬如梵王,世界之主,人天一切,莫不歸依。敬禮大吉天子足下,猶如現前,忝承先業,慶嘉無量。今遣使問訊大意。欲自往,復畏大海風波不達。今奉薄獻,願大家曲垂領納。」



 婆利國,在廣州東南海中洲上,去廣州二月日行。國界東西五十日行,南北二十日行。有一百三十六聚。土氣暑熱,如中國之盛夏。穀一歲再熟,草木常榮。海出文螺、紫貝。有石名蚶貝羅,初採之柔軟,及刻削為物乾之,遂大堅彊。其國人披吉貝如帊,及為都縵。王乃用班絲布,
 以瓔珞繞身,頭著金冠高尺餘,形如弁,綴以七寶之飾,帶金裝劍,偏坐金高坐,以銀蹬支足。侍女皆為金花雜寶之飾,或持白毦拂及孔雀扇。王出,以象駕輿,輿以雜香為之,上施羽蓋珠簾,其導從吹螺擊鼓。王姓憍陳如,自古未通中國。問其先及年數,不能記焉,而言白凈王夫人即其國女也。



 天監十六年,遣使奉表曰:「伏承聖王信重三寶,興立塔寺,校飾莊嚴,周遍國土。四衢平坦,清凈無穢;臺殿羅列,狀若天宮;壯麗微妙,世無與等。聖主出時,四兵具足,羽儀導從,布滿左右。都人士女,麗服光飾。市廛豊富,充積珍寶。王法清整,無相侵奪。學徒皆至,
 三乘競集。敷說正法,雲布雨潤。四海流通,交會萬國。長江眇漫,清泠深廣。有生咸資,莫能消穢。陰陽和暢,災厲不作。大梁揚都聖王無等,臨覆上國,有大慈悲,子育萬民。平等忍辱,怨親無二。加以周窮,無所藏積。靡不照燭,如日之明;無不受樂,猶如凈月。宰輔賢良,群臣貞信,盡忠奉上,心無異想。伏惟皇帝是我真佛,臣是婆利國主,今敬稽首禮聖王足下,惟願大王知我此心。此心久矣,非適今也。山海阻遠,無緣自達,今故遣使獻金席等,表此丹誠。」普通三年,其王頻伽復遣使珠貝智貢白鸚鵡、青蟲、兜鍪、琉璃器、吉貝、螺杯、雜香、藥等數十種。



 中天竺國,在大月支東南數千里,地方三萬里,一名身毒。漢世張騫使大夏,見邛竹杖、蜀布,國人云,市之身毒。身毒即天竺,蓋傳譯音字不同,其實一也。從月支、高附以西,南至西海,東至槃越,列國數十,每國置王,其名雖異,皆身毒也。漢時羈屬月支,其俗土著與月支同,而卑濕暑熱,民弱畏戰,弱於月支。國臨大江,名新陶,源出崑崙,分為五江,總名曰恒水。其水甘美,下有真鹽,色正白如水精。土俗出犀、象、貂、鼲、玳瑁、火齊、金、銀、鐵、金縷織成金皮罽、細摩白疊、好裘、毾。火齊狀如雲母,色如紫金,有光耀,別之則薄如蟬翼,積之則如紗縠之重沓也。其
 西與大秦、安息交市海中,多大秦珍物——珊瑚、琥珀、金碧珠璣、瑯玕、鬱金、蘇合。蘇合是合諸香汁煎之,非自然一物也。又云大秦人採蘇合,先笮其汁以為香膏,乃賣其滓與諸國賈人,是以展轉來達中國,不大香也。鬱金獨出罽賓國,華色正黃而細,與芙蓉華裏被蓮者相似。國人先取以上佛寺,積日香槁,乃糞去之;賈人從寺中徵雇,以轉賣與佗國也。



 漢桓帝延熹九年,大秦王安敦遣使自日南徼外來獻,漢世唯一通焉。其國人行賈,往往至扶南、日南、交趾,其南徼諸國人少有到大秦者。孫權黃武五年,有大秦賈人字秦論來到交趾,交趾太守吳
 邈遣送詣權。權問方土謠俗,論具以事對。時諸葛恪討丹陽,獲黝、歙短人,論見之曰:「大秦希見此人。」權以男女各十人,差吏會稽劉咸送論,咸於道物故,論乃徑還本國。漢和帝時,天竺數遣使貢獻,後西域反叛,遂絕。至桓帝延熹二年、四年,頻從日南徼外來獻。魏、晉世,絕不復通。唯吳時扶南王范旃遣親人蘇物使其國,從扶南發投拘利口,循海大灣中正西北入歷灣邊數國,可一年餘到天竺江口,逆水行七千里乃至焉。天竺王驚曰:「海濱極遠,猶有此人。」即呼令觀視國內,仍差陳、宋等二人以月支馬四匹報旃,遣物等還,積四年方至。其時吳遣
 中郎康泰使扶南,及見陳、宋等,具問天竺土俗,云:「佛道所興國也。人民敦厖,土地饒沃。其王號茂論。所都城郭,水泉分流,繞于渠緌,下注大江。其宮殿皆雕文鏤刻,街曲市里,屋舍樓觀,鐘鼓音樂,服飾香華;水陸通流,百賈交會,奇玩珍瑋,恣心所欲。左右嘉維、舍衛、葉波等十六大國,去天竺或二三千里,共尊奉之,以為在天地之中也。」



 天監初,其王屈多遣長史竺羅達奉表曰:「伏聞彼國據江傍海,山川周固,眾妙悉備,莊嚴國土,猶如化城。宮殿莊飾,街巷平坦,人民充滿,歡娛安樂。大王出遊,四兵隨從,聖明仁愛,不害眾生。國中臣民,循行正法,大王仁
 聖,化之以道,慈悲群生,無所遺棄。常修凈戒,式導不及,無上法船,沉溺以濟。百官氓庶,受樂無恐。諸天護持,萬神侍從,天魔降服,莫不歸仰。王身端嚴,如日初出,仁澤普潤,猶如大雲,於彼震旦,最為殊勝。臣之所住國土,首羅天守護,令國安樂。王王相承,未曾斷絕。國中皆七寶形像,眾妙莊嚴,臣自脩檢,如化王法。臣名屈多,奕世王種。惟願大王,聖體和平。今以此國群臣民庶,山川珍重,一切歸屬,五體投地,歸誠大王。使人竺達多由來忠信,是故今遣。大王若有所須珍奇異物,悉當奉送。此之境土,便是大王之國;王之法令善道,悉當承用。願二國信
 使往來不絕。此信返還,願賜一使,具宣聖命,備敕所宜。款至之誠,望不空返,所白如允,願加採納。今奉獻琉璃唾壺、雜香、吉貝等物。」



 師子國,天竺旁國也。其地和適,無冬夏之異。五穀隨人所種,不須時節。其國舊無人民,止有鬼神及龍居之。諸國商估來共市易,鬼神不見其形,但出珍寶,顯其所堪價,商人依價取之。諸國人聞其土樂,因此競至,或有停住者,遂成大國。



 晉義熙初,始遣獻玉像,經十載乃至。像高四尺二寸,玉色潔潤,形製殊特,殆非人工。此像歷晉、宋世在瓦官寺,寺先有徵士戴安道手製佛像五軀,及
 顧長康維摩畫圖,世人謂為三絕。至齊東昏,遂毀玉像,前截臂,次取身,為嬖妾潘貴妃作釵釧。宋元嘉六年、十二年,其王剎利摩訶遣使貢獻。



 大通元年,後王伽葉伽羅訶梨邪使奉表曰:「謹白大梁明主:雖山海殊隔,而音信時通。伏承皇帝道德高遠,覆載同於天地,明照齊乎日月,四海之表,無有不從,方國諸王,莫不奉獻,以表慕義之誠。或泛海三年,陸行千日,畏威懷德,無遠不至。我先王以來,唯以脩德為本,不嚴而治。奉事正法道天下,欣人為善,慶若己身,欲與大梁共弘三寶,以度難化。信還,伏聽告敕。今奉薄獻,願垂納受。」



 東夷之國,朝鮮為大,得箕子之化,其器物猶有禮樂云。魏時,朝鮮以東馬韓、辰韓之屬,世通中國。自晉過江,泛海東使,有高句驪、百濟,而宋、齊間常通職貢。梁興,又有加焉。扶桑國,在昔未聞也。普通中,有道人稱自彼而至,其言元本尤悉,故並錄焉。



 高句驪者,其先出自東明。東明本北夷丱離王之子。離王出行,其侍兒於後任娠,離王還,欲殺之。侍兒曰:「前見天上有氣如大雞子,來降我,因以有娠。」王囚之,後遂生男。王置之豕牢,豕以口氣噓之,不死,王以為神,乃聽收養。長而善射,王忌其猛,復欲殺之,東明乃奔走,南至淹
 滯水,以弓擊水,魚鱉皆浮為橋,東明乘之得渡,至夫餘而王焉。其後支別為句驪種也。其國,漢之玄菟郡也,在遼東之東,去遼東千里。漢、魏世,南與朝鮮、穢貃,東與沃沮,北與夫餘接。漢武帝元封四年,滅朝鮮,置玄菟郡,以高句驪為縣以屬之。



 句驪地方可二千里,中有遼山,遼水所出。其王都於丸都之下,多大山深谷,無原澤,百姓依之以居,食澗水。雖土著,無良田,故其俗節食。好治宮室,於所居之左立大屋,祭鬼神,又祠零星、社稷。人性凶急,喜寇抄。其官,有相加、對盧、沛者、古鄒加、主簿、優台、使者、皂衣、先人,尊卑各有等級。言語諸事,多與夫餘同,其
 性氣、衣服有異。本有五族,有消奴部、絕奴部、慎奴部、雚奴部、桂婁部。本消奴部為王,微弱,桂婁部代之。漢時賜衣幘、朝服、鼓吹,常從玄菟郡受之。後稍驕,不復詣郡,但於東界築小城以受之,至今猶名此城為幘溝婁。「溝婁」者,句驪名「城」也。其置官,有對盧則不置沛者,有沛者則不置對盧。其俗喜歌儛,國中邑落男女,每夜群聚歌戲。其人潔清自喜,善藏釀,跪拜申一腳,行步皆走。以十月祭天大會,名曰「東明」。其公會衣服,皆錦繡金銀以自飾。大加、主簿頭所著似幘而無後;其小加著折風,形如弁。其國無牢獄,有罪者,則會諸加評議殺之,沒入妻子。其
 俗好淫,男女多相奔誘。已嫁娶,便稍作送終之衣。其死葬,有槨無棺。好厚葬,金銀財幣盡於送死。積石為封,列植松柏。兄死妻嫂。其馬皆小,便登山。國人尚氣力,便弓矢刀矛。有鎧甲,習戰鬥,沃沮、東穢皆屬焉。



 王莽初,發高驪兵以伐胡,不欲行,彊迫遣之,皆亡出塞為寇盜。州郡歸咎於句驪侯騶,嚴尤誘而斬之,王莽大悅,更名高句驪為下句驪,當此時為侯矣。光武八年,高句驪王遣使朝貢,始稱王。至殤、安之間,其王名宮,數寇遼東,玄菟太守蔡風討之不能禁。宮死,子伯固立。順、和之間,復數犯遼東寇抄。靈帝建寧二年,玄菟太守耿臨討之,斬首虜
 數百級,伯固乃降屬遼東。公孫度之雄海東也,伯固與之通好。伯固死,子伊夷摸立。伊夷摸自伯固時已數寇遼東,又受亡胡五百餘戶。建安中,公孫康出軍擊之,破其國,焚燒邑落,降胡亦叛伊夷摸,伊夷摸更作新國。其後伊夷摸復擊玄菟,玄菟與遼東合擊,大破之。



 伊夷摸死,子位宮立。位宮有勇力,便鞍馬,善射獵。魏景初二年,遣太傅司馬宣王率眾討公孫淵,位宮遣主簿、大加將兵千人助軍。正始三年,位宮寇西安、嘉平。五年,幽州刺史母丘儉將萬人出玄菟討位宮,位宮將步騎二萬人逆軍,大戰於沸流。位宮敗走,儉軍追至峴,懸車束馬,登
 丸都山,屠其所都,斬首虜萬餘級。位宮單將妻息遠竄。六年,儉復討之,位宮輕將諸加奔沃沮,儉使將軍王頎追之,絕沃沮千餘里,到肅慎南界,刻石紀功;又到丸都山,銘不耐城而還。其後,復通中夏。



 晉永嘉亂,鮮卑慕容廆據昌黎大棘城,元帝授平州刺史。句驪王乙弗利頻寇遼東,廆不能制。弗利死,子劉代立。康帝建元元年,慕容廆子晃率兵伐之,劉與戰,大敗,單馬奔走。晃乘勝追至丸都,焚其宮室,掠男子五萬餘口以歸。孝武太元十年,句驪攻遼東、玄菟郡,後燕慕容垂遣弟農伐句驪,復二郡。垂死,子寶立,以句驪王安為平州牧,封遼東、帶方
 二國王。安始置長史、司馬、參軍官,後略有遼東郡。至孫高璉,晉安帝義熙中,始奉表通貢職,歷宋、齊並授爵位,年百餘歲死。子雲,齊隆昌中,以為使持節、散騎常侍、都督營、平二州、征東大將軍、樂浪公。高祖即位,進雲車騎大將軍。天監七年,詔曰:「高驪王樂浪郡公雲,乃誠款著,貢驛相尋,宜隆秩命,式弘朝典。可撫東大將軍、開府儀同三司,持節、常侍、都督、王並如故。」十一年、十五年,累遣使貢獻。十七年,雲死,子安立。普通元年,詔安纂襲封爵,持節、督營、平二州諸軍事、寧東將軍。七年,安卒,子延立,遣使貢獻,詔以延襲爵。中大通四年、六年,大同元年、七
 年,累奉表獻方物。太清二年,延卒,詔以其子襲延爵位。



 百濟者,其先東夷有三韓國,一曰馬韓,二曰辰韓,三曰弁韓。弁韓、辰韓各十二國,馬韓有五十四國。大國萬餘家,小國數千家,總十餘萬戶,百濟即其一也。後漸彊大,兼諸小國。其國本與句驪在遼東之東,晉世句驪既略有遼東,百濟亦據有遼西、晉平二郡地矣,自置百濟郡。晉太元中,王須;義熙中,王餘映;宋元嘉中,王餘毘;並遣獻生口。餘毘死,立子慶。慶死,子牟都立。都死,立子牟太。齊永明中,除太都督百濟諸軍事、鎮東大將軍、百濟王。天監元年,進太號征東將軍。尋為高句驪所破,衰弱者
 累年,遷居南韓地。普通二年,王餘隆始復遣使奉表,稱「累破句驪,今始與通好」,而百濟更為彊國。其年,高祖詔曰:「行都督百濟諸軍事、鎮東大將軍、百濟王餘隆,守籓海外,遠脩貢職,迺誠款到,朕有嘉焉。宜率舊章,授茲榮命。可使持節、都督百濟諸軍事、寧東大將軍、百濟王。」五年,隆死,詔復以其子明為持節、督百濟諸軍事、綏東將軍、百濟王。



 號所治城曰固麻,謂邑曰簷魯,如中國之言郡縣也。其國有二十二簷魯,皆以子弟宗族分據之。其人形長,衣服凈潔。其國近倭,頗有文身者。今言語服章略與高驪同,行不張拱、拜不申足則異。呼帽曰冠,襦曰
 復衫,褲曰褌。其言參諸夏,亦秦、韓之遺俗云。中大通六年、大同七年,累遣使獻方物;并請《涅盤》等經義、《毛詩》博士,並工匠、畫師等,敕並給之。太清三年,不知京師寇賊,猶遣使貢獻;既至,見城闕荒毀,並號慟涕泣。侯景怒,囚執之,及景平,方得還國。



 新羅者,其先本辰韓種也。辰韓亦曰秦韓,相去萬里,傳言秦世亡人避役來適馬韓,馬韓亦割其東界居之,以秦人,故名之曰秦韓。其言語名物有似中國人,名國為邦,弓為弧,賊為寇,行酒為行觴。相呼皆為徒,不與馬韓同。又辰韓王常用馬韓人作之,世相係,辰韓不得自立
 為王,明其流移之人故也;恒為馬韓所制。辰韓始有六國,稍分為十二,新羅則其一也。其國在百濟東南五千餘里。其地東濱大海,南北與句驪、百濟接。魏時曰新盧,宋時曰新羅,或曰斯羅。其國小,不能自通使聘。普通二年,王募名秦,始使使隨百濟奉獻方物。



 其俗呼城曰健牟羅,其邑在內曰啄評,在外曰邑勒,亦中國之言郡縣也。國有六啄評,五十二邑勒。土地肥美,宜植五穀。多桑麻,作縑布。服牛乘馬,男女有別。其官名,有子賁旱支、齊旱支、謁旱支、壹告支、奇貝旱支。其冠曰遺子禮,襦曰尉解,洿曰柯半,靴曰洗。其拜及行與高驪相類。無文字,刻
 木為信。語言待百濟而後通焉。



 倭者,自云太伯之後,俗皆文身。去帶方萬二千餘里,大抵在會稽之東,相去絕遠。從帶方至倭,循海水行,歷韓國,乍東乍南,七千餘里始度一海;海闊千餘里,名瀚海,至一支國;又度一海千餘里,名未盧國;又東南陸行五百里,至伊都國;又東南行百里,至奴國;又東行百里,至不彌國;又南水行二十日,至投馬國;又南水行十日,陸行一月日,至祁馬臺國,即倭王所居。其官有伊支馬,次曰彌馬獲支,次曰奴往鞮。民種禾稻籥麻,蠶桑織績。有姜、桂、橘、椒、蘇,出黑雉、真珠、青玉。有獸如牛,名山鼠;又有
 大蛇吞此獸。蛇皮堅不可斫,其上有孔,乍開乍閉,時或有光,射之中,蛇則死矣。物產略與儋耳、朱崖同。地溫暖,風俗不淫。男女皆露紒。富貴者以錦繡雜采為帽,似中國胡公頭。食飲用籩豆。其死,有棺無槨,封土作塚。人性皆嗜酒。俗不知正歲,多壽考,多至八九十,或至百歲。其俗女多男少,貴者至四五妻,賤者猶兩三妻。婦人無淫妒。無盜竊,少諍訟。若犯法,輕者沒其妻子,重則滅其宗族。



 漢靈帝光和中,倭國亂,相攻伐歷年,乃共立一女子卑彌呼為王。彌呼無夫婿,挾鬼道,能惑眾,故國人立之。有男弟佐治國。自為王,少有見者,以婢千人自侍,唯使
 一男子出入傳教令。所處宮室,常有兵守衛。至魏景初三年,公孫淵誅後,卑彌呼始遣使朝貢,魏以為親魏王,假金印紫綬。正始中,卑彌呼死,更立男王,國中不服,更相誅殺,復立卑彌呼宗女臺與為王。其後復立男王,並受中國爵命。晉安帝時,有倭王贊。贊死,立弟彌;彌死,立子濟;濟死,立子興;興死,立弟武。齊建元中,除武持節、督倭、新羅、任那、伽羅、秦韓、慕韓六國諸軍事、鎮東大將軍。高祖即位,進武號征東將軍。



 其南有侏儒國,人長三四尺。又南黑齒國、裸國,去倭四千餘里,船行可一年至。又西南萬里有海人,身黑眼白,裸而醜。其肉美,行者或射而
 食之。



 文身國,在倭國東北七千餘里。人體有文如獸,其額上有三文,文直者貴,文小者賤。土俗歡樂,物豊而賤,行客不齎糧。有屋宇,無城郭。其王所居,飾以金銀珍麗。繞屋為緌,廣一丈,實以水銀,雨則流于水銀之上。市用珍寶。犯輕罪者則鞭杖;犯死罪則置猛獸食之,有枉則猛獸避而不食,經宿則赦之。



 大漢國,在文身國東五千餘里。無兵戈,不攻戰。風俗並與文身國同而言語異。



 扶桑國者,齊永元元年,其國有沙門慧深來至荊州,說
 云:「扶桑在大漢國東二萬餘里,地在中國之東,其土多扶桑木,故以為名。」扶桑葉似桐,而初生如筍,國人食之,實如梨而赤,績其皮為布以為衣,亦以為綿。作板屋,無城郭。有文字,以扶桑皮為紙。無兵甲,不攻戰。其國法,有南北獄。若犯輕者入南獄,重罪者入北獄。有赦則赦南獄,不赦北獄。在北獄者,男女相配,生男八歲為奴,生女九歲為婢。犯罪之身,至死不出。貴人有罪,國乃大會,坐罪人於坑,對之宴飲,分訣若死別焉。以灰繞之,其一重則一身屏退,二重則及子孫,三重則及七世。名國王為乙祁;貴人第一者為大對盧,第二者為小對盧,第三者
 為納咄沙。國王行有鼓角導從。其衣色隨年改易,甲乙年青,丙丁年赤,戊己年黃,庚辛年白,壬癸年黑。有牛角甚長,以角載物,至勝二十斛。車有馬車、牛車、鹿車。國人養鹿,如中國畜牛,以乳為酪。有桑梨,經年不壞。多蒲桃。其地無鐵有銅,不貴金銀。市無租估。其婚姻,婿往女家門外作屋,晨夕灑掃,經年而女不悅,即驅之,相悅乃成婚。婚禮大抵與中國同。親喪,七日不食;祖父母喪,五日不食;兄弟伯叔姑姊妹,三日不食。設靈為神像,朝夕拜奠,不制縗絰。嗣王立,三年不視國事。其俗舊無佛法,宋大明二年,罽賓國嘗有比丘五人游行至其國,流通佛
 法、經像,教令出家,風俗遂改。



 慧深又云:「扶桑東千餘里有女國,容貌端正,色甚潔白,身體有毛,髮長委地。至二、三月,競入水則任娠,六七月產子。女人胸前無乳,項後生毛,根白,毛中有汁,以乳子,一百日能行,三四年則成人矣。見人驚避,偏畏丈夫。食鹹草如禽獸。鹹草葉似邪蒿,而氣香味鹹。」天監六年,有晉安人渡海,為風所飄至一島,登岸,有人居止。女則如中國,而言語不可曉;男則人身而狗頭,其聲如吠。其食有小豆,其衣如布。築土為牆,其形圓,其戶如竇云。



 西北諸戎,漢世張騫始發西域之跡,甘英遂臨西海,或
 遣侍子,或奉貢獻,于時雖窮兵極武,僅而克捷,比之前代,其略遠矣。魏時三方鼎跱,日事干戈,晉氏平吳以後,少獲寧息,徒置戊己之官,諸國亦未賓從也。繼以中原喪亂,胡人遞起,西域與江東隔礙,重譯不交。呂光之涉龜茲,亦獲蠻夷之伐蠻夷,非中國之意也。自是諸國分并,勝負強弱,難得詳載。明珠翠羽,雖仞於後宮;蒲梢龍文,希入於外署。有梁受命,其奉正朔而朝闕庭者,則仇池、宕昌、高昌、鄧至、河南、龜茲、于闐、滑諸國焉。今綴其風俗,為《西北戎傳》云。



 河南王者,其先出自鮮卑慕容氏。初,慕容奕洛干有二
 子,庶長曰吐谷渾,嫡曰廆。洛干卒,廆嗣位,吐谷渾避之西徙。廆追留之,而牛馬皆西走,不肯還,因遂徙上隴,度枹罕,出涼州西南,至赤水而居之。其地則張掖之南,隴西之西,在河之南,故以為號。其界東至壘川,西鄰于闐,北接高昌,東北通秦嶺,方千餘里,蓋古之流沙地焉。乏草木,少水潦,四時恒有冰雪,唯六七月雨雹甚盛;若晴則風飄沙礫,常蔽光景。其地有麥無穀。有青海方數百里,放牝馬其側,輒生駒,土人謂之龍種,故其國多善馬。有屋宇,雜以百子帳,即穹廬也。著小袖袍、小口褲、大頭長裙帽。女子披髮為辮。



 其後吐谷渾孫葉延,頗識書記,
 自謂「曾祖奕洛干始封昌黎公,吾蓋公孫之子也」。禮以王父字為國氏,因姓吐谷渾,亦為國號。至其末孫阿豺,始受中國官爵。弟子慕延,宋元嘉末又自號河南王。慕延死,從弟拾寅立,乃用書契,起城池,築宮殿,其小王並立宅。國中有佛法。拾寅死,子度易侯立;易侯死,子休留代立。齊永明中,以代為使持節、都督西秦、河、沙三州、鎮西將軍、護羌校尉、西秦、河二州刺史。梁興,進代為征西將軍。代死,子休運籌襲爵位。天監十三年,遣使獻金裝馬腦鐘二口,又表於益州立九層佛寺,詔許焉。十五年,又遣使獻赤舞龍駒及方物。其使或歲再三至,或再歲
 一至。其地與益州鄰,常通商賈,民慕其利,多往從之,教其書記,為之辭譯,稍桀黠矣。普通元年,又奉獻方物。籌死,子呵羅真立。大通三年,詔以為寧西將軍、護羌校尉、西秦、河二州刺史。真死,子佛輔襲爵位,其世子又遣使獻白龍駒於皇太子。



 高昌國,闞氏為主,其後為河西王沮渠茂虔弟無諱襲破之,其王闞爽奔于芮芮。無諱據之稱王,一世而滅。國人又立麴氏為王,名嘉,元魏授車騎將軍、司空公、都督秦州諸軍事、秦州刺史、金城郡開國公。在位二十四年卒,謚曰昭武王。子子堅,使持節、驃騎大將軍、散騎常侍、
 都督瓜州諸軍事、瓜州刺史、河西郡開國公、儀同三司、高昌王嗣位。



 其國蓋車師之故地也。南接河南,東連燉煌,西次龜茲,北鄰敕勒。置四十六鎮,交河、田地、高寧、臨川、橫截、柳婆、洿林、新興、由寧、始昌、篤進、白力等,皆其鎮名。官有四鎮將軍及雜號將軍、長史、司馬、門下校郎、中兵校郎、通事舍人、通事令史、諮議、校尉、主簿。國人言語與中國略同。有《五經》、歷代史、諸子集。面貌類高驪,辮髮垂之於背,著長身小袖袍、縵襠褲。女子頭髮辮而不垂,著錦纈纓珞環釧。姻有六禮。其地高燥,築土為城,架木為屋,土覆其上。寒暑與益州相似。備植九穀,人多啖罝
 及羊牛肉。出良馬、蒲陶酒、石鹽。多草木,草實如繭,繭中絲如細纑,名為白疊子,國人多取織以為布。布甚軟白,交市用焉。有朝烏者,旦旦集王殿前,為行列,不畏人,日出然後散去。大同中,子堅遣使獻鳴鹽枕、蒲陶、良馬、氍毹等物。



 滑國者,車師之別種也。漢永建元年,八滑從班勇擊北虜有功,勇上八滑為後部親漢侯。自魏、晉以來,不通中國。至天監十五年,其王厭帶夷栗始遣使獻方物。普通元年,又遣使獻黃師子、白貂裘、波斯錦等物。七年,又奉表貢獻。



 元魏之居桑乾也,滑猶為小國,屬芮芮。後稍
 彊大,徵其旁國波斯、盤盤、罽賓、焉耆、龜茲、疏勒、姑墨、于闐、句盤等國,開地千餘里。土地溫暖,多山川樹木,有五穀。國人以罝及羊肉為糧。其獸有師子、兩腳駱駝,野驢有角。人皆善射,著小袖長身袍,用金玉為帶。女人被裘,頭上刻木為角,長六尺,以金銀飾之。少女子,兄弟共妻。無城郭,氈屋為居,東向開戶。其王坐金床,隨太歲轉,與妻並坐接客。無文字,以木為契。與旁國通,則使旁國胡為胡書,羊皮為紙。無職官。事天神、火神,每日則出戶祀神而後食。其跪一拜而止。葬以木為槨。父母死,其子截一耳,葬訖即吉。其言語待河南人譯然後通。



 周古柯國,滑旁小國也。普通元年,使使隨滑來獻方物。



 呵跋檀國,亦滑旁小國也。凡滑旁之國,衣服容貌皆與滑同。普通元年,使使隨滑使來獻方物。



 胡蜜丹國,亦滑旁小國也。普通元年,使使隨滑使來獻方物。



 白題國,王姓支名史稽毅,其先蓋匈奴之別種胡也。漢灌嬰與匈奴戰,斬白題騎一人。今在滑國東,去滑六日行,西極波斯。土地出粟、麥、瓜果,食物略與滑同。普通三年,遣使獻方物。



 龜茲者,西域之舊國也。後漢光武時,其王名弘,為莎車王賢所殺,滅其族。賢使其子則羅為龜茲王,國人又殺
 則羅。匈奴立龜茲貴人身毒為王,由是屬匈奴。然龜茲在漢世常為大國,所都曰延城。魏文帝初即位,遣使貢獻。晉太康中,遣子入侍。太元七年,秦主苻堅遣將呂光伐西域。至龜茲,龜茲王帛純載寶出奔,光入其城。城有三重,外城與長安城等,室屋壯麗,飾以瑯玕金玉。光立帛純弟震為王而歸,自此與中國絕不通。普通二年,王尼瑞摩珠那勝遣使奉表貢獻。



 于闐國,西域之屬也。後漢建武末,王俞為莎車王賢所破,徙為驪歸王,以其弟君得為于闐王,暴虐,百姓患之。永平中,其種人都末殺君得,大人休莫霸又殺都末,自
 立為王。霸死,兄子廣得立,後擊虜莎車王賢以歸,殺之,遂為彊國,西北諸小國皆服從。



 其地多水潦沙石,氣溫,宜稻、麥、蒲桃。有水出玉,名曰玉河。國人善鑄銅器。其治曰西山城,有屋室市井。果蓏菜蔬與中國等。尤敬佛法。王所居室,加以朱畫。王冠金幘,如今胡公帽;與妻並坐接客。國中婦人皆辮髮,衣裘褲。其人恭,相見則跪,其跪則一膝至地。書則以木為筆札,以玉為印。國人得書,戴於首而後開札。魏文帝時,王山習獻名馬。天監九年,遣使獻方物。十三年,又獻波羅婆步鄣。十八年,又獻琉璃罌。大同七年,又獻外國刻玉佛。



 渴盤陁國,于闐西小國也。西鄰滑國,南接罽賓國,北連沙勒國。所治在山谷中,城周迴十餘里,國有十二城。風俗與于闐相類。衣吉貝布,著長身小袖袍、小口褲。地宜小麥,資以為糧。多牛馬駱駝羊等。出好氈、金、玉。王姓葛沙氏。中大同元年,遣使獻方物。



 末國,漢世且末國也。勝兵萬餘戶。北與丁零,東與白題,西與波斯接。土人剪髮,著氈帽、小袖衣,為衫則開頸而縫前。多牛羊騾驢。其王安末深盤,普通五年,遣使來貢獻。



 波斯國,其先有波斯匿王者,子孫以王父字為氏,因為
 國號。國有城,周迴三十二里,城高四丈,皆有樓觀,城內屋宇數百千間,城外佛寺二三百所。西去城十五里有土山,山非過高,其勢連接甚遠,中有鷲鳥啖羊,土人極以為患。國中有優缽曇花,鮮華可愛。出龍駒馬。鹹池生珊瑚樹,長一二尺。亦有琥珀、馬腦、真珠、玫回等,國內不以為珍。市買用金銀。婚姻法:下聘訖,女婿將數十人迎婦,婿著金線錦袍、師子錦褲,戴天冠,婦亦如之。婦兄弟便來捉手付度,夫婦之禮,於茲永畢。國東與滑國,西及南俱與婆羅門國,北與汎心慄國接。中大通二年,遣使獻佛牙。



 宕昌國,在河南之東南,益州之西北,隴西之西,羌種也。宋孝武世,其王梁帟忽始獻方物。天監四年,王梁彌博來獻甘草、當歸,詔以為使持節、都督河、涼二州諸軍事、安西將軍、東羌校尉、河、涼二州刺史、隴西公、宕昌王,佩以金章。彌博死,子彌泰立;大同七年,復授以父爵位。其衣服、風俗與河南略同。



 鄧至國,居西涼州界,羌別種也。世號持節、平北將軍、西涼州刺史。宋文帝時,王象屈耽遣使獻馬。天監元年,詔以鄧至王象舒彭為督西涼州諸軍事,號安北將軍。五年,舒彭遣使獻黃耆四百斤、馬四匹。其俗呼帽曰突何,
 其衣服與宕昌同。



 武興國,本仇池。楊難當自立為秦王,宋文帝遣裴方明討之,難當奔魏。其兄子文德又聚眾茄盧,宋因授以爵位,魏又攻之,文德奔漢中。從弟僧嗣又自立,復戍茄盧。卒,文德弟文度立,以弟文洪為白水太守,屯武興,宋世以為武都王。武興之國,自於此矣。難當族弟廣香又攻殺文度,自立為陰平王、茄盧鎮主。卒,子炅立;炅死,子崇祖立;崇祖死,子孟孫立。齊永明中,魏氏南梁州刺史、仇池公楊靈珍據泥功山歸款,齊世以靈珍為北梁州刺史、仇池公。文洪死,以族人集始為北秦州刺史、武都王。
 天監初,以集始為使持節、都督秦、雍二州諸軍事、輔國將軍、平羌校尉、北秦州刺史、武都王,靈珍為冠軍將軍,孟孫為假節、督沙州刺史、陰平王。集始死,子紹先襲爵位。二年,以靈珍為持節、督隴右諸軍事、左將軍、北梁州刺史、仇池王。十年,孟孫死,詔贈安沙將軍、北雍州刺史。子定襲封爵。紹先死,子智慧立。大同元年,剋復漢中,智慧遣使上表,求率四千戶歸國,詔許焉,即以為東益州。



 其國東連秦嶺,西接宕昌,去宕昌八百里,南去漢中四百里,北去岐州三百里,東去長安九百里。本有十萬戶,世世分減。其大姓有符氏、姜氏。言語與中國同。著烏皁
 突騎帽、長身小袖袍、小口褲、皮靴。地植九穀。婚姻備六禮。知書疏。種桑麻。出紬、絹、精布、漆、蠟、椒等。山出銅鐵。



 芮芮國,蓋匈奴別種。魏、晉世,匈奴分為數百千部,各有名號,芮芮其一部也。自元魏南遷,因擅其故地。無城郭,隨水草畜牧,以穹廬為居。辮髮,衣錦,小袖袍,小口褲,深雍靴。其地苦寒,七月流澌亙河。宋升明中,遣王洪軌使焉,引之共伐魏。齊建元元年,洪軌始至其國,國王率三十萬騎,出燕然山東南三千餘里,魏人閉關不敢戰。後稍侵弱。永明中,為丁零所破,更為小國而南移其居。天監中,始破丁零,復其舊土。始築城郭,名曰木末城。十四
 年,遣使獻烏貂裘。普通元年,又遣使獻方物。是後數歲一至焉。大同七年,又獻馬一匹、金一斤。其國能以術祭天而致風雪,前對皎日,後則泥潦橫流,故其戰敗莫能追及。或於中夏為之,則曀而不雨,問其故,以暖雲。



 史臣曰:海南東夷西北戎諸國,地窮邊裔,各有疆域。若山奇海異,怪類殊種,前古未聞,往牒不記。故知九州之外,八荒之表,辯方物土,莫究其極。高祖以德懷之,故朝貢歲至,美矣。



\end{pinyinscope}