\article{卷第二十六列傳第二十 範岫 傅昭弟映 蕭琛 陸杲}

\begin{pinyinscope}

 範岫,字懋賓,濟陽考城人也。高祖宣,晉徵士。父羲,宋兗州別駕。岫早孤,事母以孝聞,與吳興沈約俱為蔡興宗所禮。泰
 始中,起家奉朝請。興宗為安西將軍,引為主簿。累遷臨海、長城二縣令,驃騎參軍,尚書刪定郎,護軍司馬,齊司徒竟陵王子良記室參軍。累遷太子家令。文惠太子之在東宮,沈約之徒以文才見引,岫亦預焉。岫文雖不逮約,而名行為時輩所與,博涉多通,尤悉魏晉以來吉凶故事。約常稱曰:「范公好事該博,胡廣無以加。」南鄉范雲謂人曰:「諸君進止威儀,當問范長頭。」以岫多識前代舊事也。遷國子博士。



 永明中,魏使至,有詔妙選朝士有詞辯者,接使於界首,以岫兼淮陰長史迎焉。還遷尚書左丞,母憂去官,尋起攝職。出為寧朔將軍、南蠻長史、南義陽太守,未赴職,遷右軍諮議參軍,郡如故。除撫軍司馬。出為建威將軍、安成內史。入為給事黃門侍郎,遷御史中丞、領前軍將軍、南、北兗二州大中正。永元末,出為輔國將軍、冠軍晉安王長史,行南徐州事。義師平
 京邑,承制徵為尚書吏部郎,參大選。梁臺建,為度支尚書。天監五年,遷散騎常侍、光祿大夫,侍皇太子,給扶。六年,領太子左衛率。七年,徙通直散騎常侍、右衛將軍,中正如故。其年表致事,詔不許。八年,出為晉陵太守,秩中二千石。九年,入為祠部尚書,領右驍騎將軍,其年遷金紫光祿大夫,加親信二十人。十三年,卒官,時年七十五。賻錢五萬,布百匹。



 岫身長七尺八寸,恭敬儼恪,進止以禮。自親喪之後,蔬食布衣以終身。每所居官,恆以廉潔著稱。為長城令時,有梓材巾箱,至數十年,經貴遂不改易。在晉陵,惟作牙管筆一雙,猶以為費。所著文集、《禮論》、《
 雜儀》、《字訓》行於世。二子褒,偉。



 傅昭,字茂遠,北地靈州人,晉司隸校尉咸七世孫也。祖和之,父淡,善《三禮》,知名宋世。淡事宋竟陵王劉誕,誕反,淡坐誅。昭六歲而孤,哀毀如成人者,宗黨咸異之。十一,隨外祖於朱雀航賣曆日。為雍州刺史袁抃客,抃嘗來昭所,昭讀書自若,神色不改。抃歎曰:「此兒神情不凡,必成佳器。」司徒建安王休仁聞而悅之,因欲致昭,昭以宋氏多故,遂不往。或有稱昭於廷尉虞愿,愿乃遣車迎昭。時愿宗人通之在坐,並當世名流,通之贈昭詩曰:「英妙擅山東,才子傾洛陽。清塵誰能嗣,及爾遘遺芳。」太原王延秀薦
 昭於丹陽尹袁粲,深為所禮,辟為郡主簿,使諸子從昭受學。會明帝崩,粲造哀策文,乃引昭定其所制。每經昭戶,輒歎曰:「經其戶,寂若無人,披其帷,其人斯在,豈非名賢!」尋為總明學士、奉朝請。齊永明中,累遷員外郎、司徒竟陵王子良參軍、尚書儀曹郎。



 先是御史中丞劉休薦昭於武帝,永明初,以昭為南郡王侍讀。王嗣帝位,故時臣隸爭求權寵,惟昭及南陽宗夬,保身守正,無所參入,竟不罹其禍。明帝踐阼,引昭為中書通事舍人。時居此職者,皆勢傾天下,昭獨廉靜,無所干豫。器服率陋,身安粗糲。常插燭於板床,明帝聞之,賜漆合燭盤等,敕曰:「卿有古人之風,故賜
 卿古人之物。」累遷車騎臨海王記室參軍,長水校尉,太子家令,驃騎晉安王諮議參軍。尋除尚書左丞、本州大中正。



 高祖素悉昭能,建康城平,引為驃騎錄事參軍。梁臺建,遷給事黃門侍郎,領著作郎,頃之,兼御史中丞,黃門、著作、中正並如故。天監三年,兼五兵尚書,參選事,四年,即真。六年,徙為左民尚書,未拜,出為建威將軍、平南安成王長史、尋陽太守。七年,入為振遠將軍、中權長史。八年,遷通直散騎常侍,領步兵校尉,復領本州大中正。十年,復為左民尚書。



 十一年,出為信武將軍、安成內史。安成自宋已來兵亂,郡舍號凶。及昭為郡,郡內人夜夢見
 兵馬鎧甲甚盛,又聞有人云「當避善人」,軍眾相與騰虛而逝。夢者驚起。俄而疾風暴雨,倏忽便至,數間屋俱倒,即夢者所見軍馬踐蹈之所也。自後郡舍遂安,咸以昭正直所致。郡溪無魚,或有暑月薦昭魚者,昭既不納,又不欲拒,遂委于門側。



 十二年,入為秘書監,領後軍將軍。十四年,遷太常卿。十七年,出為智武將軍、臨海太守。郡有蜜巖,前後太守皆自封固,專收其利。昭以周文之囿,與百姓共之,大可喻小,乃教勿封。縣令常餉慄,置絹于薄下,昭笑而還之。普通二年,入為通直散騎常侍、光祿大夫,領本州大中正,尋領秘書監。五年,遷散騎常侍、金
 紫光祿大夫,中正如故。



 昭所蒞官,常以清靜為政,不尚嚴肅。居朝廷,無所請謁,不畜私門生,不交私利。終日端居,以書記為樂,雖老不衰。博極古今,尤善人物,魏晉以來,官宦簿伐,姻通內外,舉而論之,無所遺失。性尤篤慎。子婦嘗得家餉牛肉以進,昭召其子曰:「食之則犯法,告之則不可,取而埋之。」其居身行己,不負闇室,類皆如此。京師後進,宗其學,重其道,人人自以為不逮。大通二年九月,卒,時年七十五。詔賻錢三萬,布五十匹,即日舉哀,謚曰貞子。長子住,尚書郎,臨安令。次子肱。



 映字徽遠,昭弟也。三歲而孤。兄弟友睦,修身厲行,非禮
 不行。始昭之守臨海,陸倕餞之,賓主俱懽,日昏不反,映以昭年高,不可連夜極樂,乃自往迎候,同乘而歸,兄弟並已斑白,時人美而服焉。及昭卒,映喪之如父,年踰七十,哀戚過禮,服制雖除,每言輒感慟。



 映泛涉記傳,有文才,而不以篇什自命。少時與劉繪、蕭琛相友善,繪之為南康相,映時為府丞,文教多令具草。褚彥回聞而悅之,乃屈與子賁等遊處。年未弱冠,彥回欲令仕,映以昭未解褐,固辭,須昭仕乃官。



 永元元年,參鎮軍江夏王軍事,出為武康令。及高祖師次建康,吳興太守袁昂自謂門世忠貞,固守誠節,乃訪於映曰:「卿謂時事云何?」映答曰:「
 元嘉之末,開闢未有,故太尉殺身以明節,司徒當寄託之重,理無茍全,所以不顧夷險,以殉名義。今嗣主昏虐,狎近群小,親賢誅戮,君子道消,外難屢作,曾無悛改。今荊、雍協舉,乘據上流,背昏向明,勢無不濟。百姓思治,天人之意可知;既明且哲,忠孝之途無爽。願明府更當雅慮,無祇悔也。」尋以公事免。天監初,除征虜鄱陽王參軍,建安王中權錄事參軍,領軍長史,烏程令。所受俸祿,悉歸于兄。復為臨川王錄事參軍,南臺治書,安成王錄事,太子翊軍校尉,累遷中散大夫、光祿卿,太中大夫。大同五年,卒,年八十三。子弘。



 蕭琛,字彥瑜,蘭陵人。祖僧珍,宋廷尉卿。父惠訓,太中大夫。琛年數歲,從伯惠開撫其背曰:「必興吾宗。」



 琛少而朗悟,有縱橫才辯。起家齊太學博士。時王儉當朝,琛年少,未為儉所識,負其才氣,欲候儉。時儉宴于樂遊苑,琛乃著虎皮靴,策桃枝杖,直造儉坐,儉與語,大悅。儉為丹陽尹,辟為主簿,舉為南徐州秀才,累遷司徒記室。



 永明九年,魏始通好,琛再銜命到桑乾,還為通直散騎侍郎。時魏遣李道固來使,齊帝宴之。琛於御筵舉酒勸道固,道固不受,曰:「公庭無私禮,不容受勸。」琛徐答曰:「《詩》所謂『雨我公田,遂及我私』。」座者皆服,道固乃受琛酒。遷司徒右
 長史。出為晉熙王長史、行南徐州事。還兼少府卿、尚書左丞。



 東昏初嗣立,時議以無廟見之典,琛議據《周頌·烈文》、《閔予》皆為即位朝廟之典,於是從之。高祖定京邑,引為驃騎諮議,領錄事,遷給事黃門侍郎。梁臺建,為御史中丞。天監元年,遷庶子,出為宣城太守。徵為衛尉卿,俄遷員外散騎常侍。三年,除太子中庶子、散騎常侍。九年,出為寧遠將軍、平西長史、江夏太守。



 始琛在宣城,有北僧南度,惟賚一葫蘆,中有《漢書序傳》。僧曰:「三輔舊老相傳,以為班固真本。」琛固求得之,其書多有異今者,而紙墨亦古,文字多如龍舉之例,非隸非篆,琛甚秘之。及是
 行也,以書餉鄱陽王範,範乃獻于東宮。



 琛尋遷安西長史、南郡太守,母憂去官,又丁父艱。起為信武將軍、護軍長史,俄為貞毅將軍、太尉長史。出為信威將軍、東陽太守,遷吳興太守。郡有項羽廟,土民名為憤王,甚有靈驗,遂於郡廳事安施床幕為神座,公私請禱,前後二千石皆於廳拜祠,而避居他室。琛至,徙神還廟,處之不疑。又禁殺牛解祀,以脯代肉。



 琛頻蒞大郡,不治產業,有闕則取,不以為嫌。普通元年,徵為宗正卿,遷左民尚書,領南徐州大中正,太子右衛率。徙度支尚書,左驍騎將軍,領軍將軍,轉秘書監、後軍將軍,遷侍中。



 高祖在西邸,早與
 琛狎,每朝宴,接以舊恩,呼為宗老。琛亦奉陳昔恩,以「早簉中陽,夙忝同閈,雖迷興運,猶荷洪慈。」上答曰:「雖云早契闊,乃自非同志;勿談興運初,且道狂奴異。」



 琛常言:「少壯三好,音律、書、酒。年長以來,二事都廢,惟書籍不衰。」而琛性通脫,常自解灶事,畢狖餘,必陶然致醉。



 大通二年,為金紫光祿大夫,加特進,給親信三十人。中大通元年,為雲麾將軍、晉陵太守,秩中二千石。以疾自解,改授侍中、特進、金紫光祿大夫。卒,年五十二。遺令諸子,與妻同墳異藏,祭以蔬菜,葬日止車十乘,事存率素。乘輿臨哭甚哀。詔贈本官,加雲麾將軍,給東園秘器,朝服一具,衣
 一襲,賻錢二十萬,布百匹。謚曰平子。



 陸杲,字明霞,吳郡吳人。祖徽,宋輔國將軍、益州刺史。父睿,揚州治中。杲少好學,工書畫,舅張融有高名,杲風韻舉動,頗類於融,時稱之曰:「無對日下,惟舅與甥。」起家齊中軍法曹行參軍,太子舍人,衛軍王儉主簿。遷尚書殿中曹郎,拜日,八座丞郎並到上省交禮,而杲至晚,不及時刻,坐免官。久之,以為司徒竟陵王外兵參軍,遷征虜宜都王功曹史,驃騎晉安王諮議參軍,司徒從事中郎。梁臺建,以為驃騎記室參軍,遷相國西曹掾。天監元年,除撫軍長史,母憂去職。服闋,拜建威將軍、中軍臨川王
 諮議參軍,尋遷黃門侍郎,右軍安成王長史。五年,遷御史中丞。



 杲性婞直,無所顧望。山陰令虞肩在任,贓污數百萬,杲奏收治。中書舍人黃睦之以肩事託杲,杲不答。高祖聞之,以問杲,杲答曰「有之」。高祖曰:「卿識睦之不?」杲答曰:「臣不識其人。」時睦之在御側,上指示杲曰:「此人是也。」杲謂睦之曰:「君小人,何敢以罪人屬南司?」睦之失色。領軍將軍張稷,是杲從舅,杲嘗以公事彈稷,稷因侍宴訴高祖曰:「陸杲是臣通親,小事彈臣不貸。」高祖曰:「杲職司其事,卿何得為嫌!」杲在臺,號稱不畏彊禦。



 六年,遷秘書監,頃之為太子中庶子、光祿卿。八年,出為義興太守,
 在郡寬惠,為民下所稱。還為司空臨川王長史、領揚州大中正。十四年,遷通直散騎侍郎,俄遷散騎常侍,中正如故。十五年,遷司徒左長史。十六年,入為左民尚書,遷太常卿。普通二年,出為仁威將軍、臨川內史。五年,入為金紫光祿大夫,又領揚州大中正。中大通元年,加特進,中正如故。四年,卒,時年七十四。謚曰質子。



 杲素信佛法,持戒甚精,著《沙門傳》三十卷。



 弟煦,學涉有思理。天監初,歷中書侍郎,尚書左丞,太子家令,卒。撰《晉書》未就。又著《陸史》十五卷,《陸氏驪泉志》一卷,並行於世。



 子罩,少篤學,有文才,仕至太子中庶子、光祿卿。



 史臣曰:范岫、傅昭,並篤行清慎,善始令終,斯石建、石慶之徒矣。蕭琛、陸杲俱以才學著名。琛朗悟辯捷,加諳究朝典,高祖在田,與琛遊舊,及踐天歷,任遇甚隆,美矣。杲性婞直,無所忌憚,既而執法憲臺,糾繩不避權幸,可謂允茲正色。《詩》云:「彼己之子,邦之司直。」杲其有焉。



\end{pinyinscope}