\article{卷第二十一列傳第十五 王瞻 王志 王峻 王暕子訓 王泰 王份孫錫 僉 張充 柳惲蔡撙 江蒨}

\begin{pinyinscope}

 王瞻,字思範,瑯邪臨沂人,宋太保弘從孫也。祖柳,光祿大夫、東亭侯。父猷,廷尉卿。瞻年數歲,嘗從師受業,時有伎經其門,同學皆出觀,瞻獨不視,習誦如初。從父尚書僕射僧達聞而異之,謂瞻父曰:「吾宗不衰,寄之此子。」年
 十二,居父憂,以孝聞。服闋,襲封東亭侯。



 瞻幼時輕薄,好逸遊,為閭里所患。及長,頗折節有士操,涉獵書記,於棋射尤善。起家著作佐郎,累遷太子舍人、太尉主簿、太子洗馬。頃之,出為鄱陽內史,秩滿,授太子中舍人。又為齊南海王友,尋轉司徒竟陵王從事中郎,王甚相賓禮。南海王為護軍將軍,瞻為長史。又出補徐州別駕從事史,遷驃騎將軍王晏長史。晏誅,出為晉陵太守。瞻潔己為政,妻子不免飢寒。時大司馬王敬則舉兵作亂,路經晉陵,郡民多附敬則。軍敗,臺軍討賊黨,瞻言於朝曰:「愚人易動,不足窮法。」明帝許之,所全活者萬數。徵拜給事黃
 門侍郎,撫軍建安王長史,御史中丞。



 高祖霸府開,以瞻為大司馬相國諮議參軍,領錄事。梁臺建,為侍中,遷左民尚書,俄轉吏部尚書。瞻性率亮,居選部,所舉多行其意。頗嗜酒,每飲或竟日,而精神益朗贍,不廢簿領。高祖每稱瞻有三術,射、棋、酒也。尋加左軍將軍,以疾不拜,仍為侍中,領驍騎將軍,未拜,卒,時年四十九。謚康侯。子長玄,著作佐郎,早卒。



 王志,字次道,瑯邪臨沂人。祖曇首,宋左光祿大夫、豫寧文侯;父僧虔,齊司空、簡穆公:並有重名。志年九歲,居所生母憂,哀容毀瘠,為中表所異。弱冠,選尚孝武女安
 固公主,拜駙馬都尉、秘書郎。累遷太尉行參軍,太子舍人,武陵王文學。褚淵為司徒,引志為主簿。淵謂僧虔曰:「朝廷之恩,本為殊特,所可光榮,在屈賢子。」累遷鎮北竟陵王功曹史、安陸南郡二王友。入為中書侍郎。尋除宣城內史,清謹有恩惠。郡民張倪、吳慶爭田,經年不決。志到官,父老乃相謂曰:「王府君有德政,吾曹鄉里乃有此爭。」倪、慶因相攜請罪,所訟地遂為閑田。徵拜黃門侍郎,尋遷吏部侍郎。出為寧朔將軍、東陽太守。郡獄有重囚十餘人,冬至日悉遣還家,過節皆返,惟一人失期,獄司以為言。志曰:「此自太守事,主者勿憂。」明旦,果自詣獄,辭以
 婦孕,吏民益歎服之。視事三年,齊永明二年,入為侍中,未拜,轉吏部尚書,在選以和理稱。崔慧景平,以例加右軍將軍,封臨汝侯,固讓不受,改領右衛將軍。



 義師至,城內害東昏,百僚署名送其首。志聞而歎曰:「冠雖弊,可加足乎?」因取庭中樹葉挪服之,偽悶,不署名。高祖覽箋無志署,心嘉之,弗以讓也。霸府開,以志為右軍將軍、驃騎大將軍長史。梁臺建,遷散騎常侍、中書令。



 天監元年,以本官領前軍將軍。其年,遷冠軍將軍、丹陽尹。為政清靜,去煩苛。京師有寡婦無子,姑亡,舉債以斂葬,既葬而無以還之。志愍其義,以俸錢償焉。時年饑,每旦為粥於郡
 門,以賦百姓,民稱之不容口。三年,為散騎常侍、中書令,領游擊將軍。志為中書令,及居京尹,便懷止足。常謂諸子侄曰:「謝莊在宋孝武世,位止中書令,吾自視豈可以過之。」因多謝病,簡通賓客。遷前將軍、太常卿。六年,出為雲麾將軍、安西始興王長史、南郡太守。明年,遷軍師將軍、平西鄱陽郡王長史、江夏太守,並加秩中二千石。九年,遷為散騎常侍、金紫光祿大夫。十二年,卒,時年五十四。



 志善草隸,當時以為楷法。齊游擊將軍徐希秀亦號能書,常謂志為「書聖」。



 志家世居建康禁中里馬蕃巷,父僧虔以來,門風多寬恕,志尤惇厚。所歷職,不以罪咎劾
 人。門下客嘗盜脫志車憲賣之,志知而不問,待之如初。賓客游其門者,專覆其過而稱其善。兄弟子姪皆篤實謙和,時人號馬蕃諸王為長者。普通四年,志改葬,高祖厚賻賜之。追謚曰安。有五子緝、休、+、操、素,並知名。



 王峻,字茂遠,瑯邪臨沂人。曾祖敬弘,有重名於宋世,位至左光祿大夫、開府儀同三司。祖瓚之,金紫光祿大夫。父秀之,吳興太守。峻少美風姿,善舉止。起家著作佐郎,不拜,累遷中軍廬陵王法曹行參軍,太子舍人,邵陵王文學,太傅主簿。府主齊竟陵王子良甚相賞遇。遷司徒主簿,以父憂去職。服闋,除太子洗馬,建安王友。出為寧
 遠將軍、桂陽內史。會義師起,上流諸郡多相驚擾,峻閉門靜坐,一郡帖然,百姓賴之。



 天監初,還,除中書侍郎。高祖甚悅其風采,與陳郡謝覽同見賞擢。俄遷吏部,當官不稱職,轉征虜安成王長史,又為太子中庶子、游擊將軍。出為宣城太守,為政清和,吏民安之。視事三年,徵拜侍中,遷度支尚書。又以本官兼起部尚書,監起太極殿。事畢,出為征遠將軍、平西長史、南郡太守。尋為智武將軍、鎮西長史、蜀郡太守。還為左民尚書,領步兵校尉。遷吏部尚書,處選甚得名譽。



 峻性詳雅,無趨競心。嘗與謝覽約,官至侍中,不復謀進仕。覽自吏部尚書出為吳興
 郡,平心不畏彊禦,亦由處世之情既薄故也。峻為侍中以後,雖不退身,亦淡然自守,無所營務。久之,以疾表解職,遷金紫光祿大夫,未拜。普通二年,卒。時年五十六,謚惠子。



 子琮,玩。琮為國子生,尚始興王女繁昌縣主,不慧,為學生所嗤,遂離婚。峻謝王,王曰:「此自上意,僕極不願如此。」峻曰:「臣太祖是謝仁祖外孫,亦不藉殿下姻媾為門戶。」



 王暕,字思晦,瑯邪臨沂人。父儉,齊太尉,南昌文憲公。暕年數歲,而風神警拔,有成人之度。時文憲作宰,賓客盈門,見暕相謂曰:「公才公望,復在此矣。」弱冠,選尚淮南長
 公主,拜駙馬都尉,除員外散騎侍郎,不拜,改授晉安王文學,遷廬陵王友、秘書丞。明帝詔求異士,始安王遙光表薦暕及東海王僧孺曰:「臣聞求賢暫勞,垂拱永逸,方之疏壤,取類導川。伏惟陛下道隱旒纊,信充符璽,白駒空谷,振鷺在庭;猶懼隱鱗卜祝,藏器屠保,物色關下,委裘河上。非取製於一狐,諒求味於兼採。而五聲倦響,九工是詢;寢議廟堂,借聽輿皁。臣位任隆重,義兼邦家,實欲使名實不違,僥幸路絕。勢門上品,猶當格以清談;英俊下僚,不可限以位貌。竊見秘書丞瑯邪王暕,年二十一,七葉重光,海內冠冕,神清氣茂,允迪中和。叔寶理遣
 之談,彥輔名教之樂,故以暉映先達,領袖後進。居無塵雜,家有賜書;辭賦清新,屬言玄遠;室邇人曠,物疏道親。養素丘園,台階虛位;庠序公朝,萬夫傾首。豈徒荀令可想,李公不亡而已哉!乃東序之秘寶,瑚璉之茂器。」除驃騎從事中郎。



 高祖霸府開,引為戶曹屬,遷司徒左長史。天監元年,除太子中庶子,領驍騎將軍,入為侍中。出為寧朔將軍、中軍長史。又為侍中,領射聲校尉,遷五兵尚書,加給事中,出為晉陵太守。徵為吏部尚書,俄領國子祭酒。暕名公子,少致美稱,及居選曹,職事脩理;然世貴顯,與物多隔,不能留心寒素,眾頗謂為刻薄。遷尚書右
 僕射,尋加侍中。復遷左僕射,以母憂去官。起為雲麾將軍、吳郡太守。還為侍中、尚書左僕射,領國子祭酒。普通四年冬,暴疾卒,時年四十七。詔贈侍中、中書令、中軍將軍,給東園秘器,朝服一具,衣一襲,錢十萬,布百匹。謚曰靖。有四子,訓、承、穉、訏,並通顯。



 訓字懷範,幼聰警有識量,徵士何胤見而奇之。年十三,暕亡憂毀,家人莫之識。十六,召見文德殿,應對爽徹。上目送久之,顧謂朱異曰:「可謂相門有相矣。」補國子生,射策高第,除秘書郎,遷太子舍人、秘書丞。轉宣城王文學、友、太子中庶子,掌管記。俄遷侍中,既拜入見,高祖從容
 問何敬容曰:「褚彥回年幾為宰相?」敬容對曰:「少過三十。」上曰:「今之王訓,無謝彥回。」



 訓美容儀,善進止,文章之美,為後進領袖。在春宮特被恩禮。以疾終于位,時年二十六。贈本官。謚溫子。



 王泰,字仲通,志長兄慈之子也。慈,齊時歷侍中、吳郡,知名在志右。泰幼敏悟,年數歲時,祖母集諸孫姪,散棗栗於床上,群兒皆競之,泰獨不取。問其故,對曰:「不取,自當得賜。」由是中表異之。既長,通和溫雅,人不見其喜慍之色。起家為著作郎,不拜,改除秘書郎,遷前將軍、法曹行參軍、司徒東閣祭酒、車騎主簿。



 高祖霸府建,以泰為驃
 騎功曹史。天監元年,遷秘書丞。齊永元末,後宮火,延燒秘書,圖書散亂殆盡。泰為丞,表校定繕寫,高祖從之。頃之,遷中書侍郎。出為南徐州別駕從事史,居職有能名。復徵中書侍郎,敕掌吏部郎事。累遷給事黃門侍郎、員外散騎常侍,並掌吏部如故,俄即真。自過江,吏部郎不復典大選,令史以下,小人求競者輻湊,前後少能稱職。泰為之不通關求,吏先至者即補,不為貴賤請囑易意,天下稱平。累遷為廷尉,司徒左長史。出為明威將軍、新安太守,在郡和理得民心。徵為寧遠將軍,安右長史,俄遷侍中。尋為太子庶子、領步兵校尉,復為侍中。仍遷仁
 威長史、南蘭陵太守,行南康王府、州、國事。王遷職,復為北中郎長史、行豫章王府、州、國事,太守如故。入為都官尚書。泰能接人士,士多懷泰,每願其居選官。頃之,為吏部尚書,衣冠屬望,未及選舉,仍疾,改除散騎常侍、左驍騎將軍,未拜,卒,時年四十五。謚夷子。



 初,泰無子,養兄子祁,晚有子廓。



 王份,字季文,瑯邪人也。祖僧朗,宋開府儀同三司、元公。父粹,黃門侍郎。份十四而孤,解褐車騎主簿。出為寧遠將軍、始安內史。袁粲之誅,親故無敢視者,份獨往致慟,由是顯名。遷太子中舍人,太尉屬。出為晉安內史。累遷
 中書侍郎,轉大司農。



 份兄奐於雍州被誅,奐子肅奔於魏,份自拘請罪,齊世祖知其誠款,喻而遣之。屬肅屢引魏人來侵疆埸,世祖嘗因侍坐,從容謂份曰:「比有北信不?」份斂容對曰:「肅既近忘墳柏,寧遠憶有臣。」帝亦以此亮焉。尋除寧朔將軍、零陵內史。徵為黃門侍郎,以父終於此職,固辭不拜,遷秘書監。



 天監初,除散騎常侍、領步兵校尉、兼起部尚書。高祖嘗於宴席問群臣曰:「朕為有為無?」份對曰:「陛下應萬物為有,體至理為無。」高祖稱善。出為宣城太守,轉吳郡太守,遷寧朔將軍、北中郎豫章王長史、蘭陵太守,行南徐府州事。遷太常卿、太子右率、
 散騎常侍,侍東宮,除金紫光祿大夫。復為智武將軍、南康王長史,秩中二千石。復入為散騎常侍、金紫光祿、南徐州大中正,給親信二十人。遷尚書左僕射,尋加侍中。



 時脩建二郊,份以本官領大匠卿,遷散騎常侍、右光祿大夫,加親信為四十人。遷侍中、特進、左光祿,復以本官監丹陽尹。普通五年三月,卒,時年七十九。詔贈本官,賻錢四十萬,布四百匹,蠟四百斤,給東園秘器,朝服一具,衣一襲。謚胡子。



 長子琳,字孝璋,舉南徐州秀才,釋褐征虜建安王法曹、司徒東閣祭酒,南平王文學。尚義興公主,拜駙馬都尉。累遷中書侍郎,衛軍謝朏長史,員外散
 騎常侍。出為明威將軍、東陽太守,徵司徒左長史。



 錫字公嘏,琳之第二子也。幼而警悟,與兄弟受業,至應休散,常獨留不起。年七八歲,猶隨公主入宮,高祖嘉其聰敏,常為朝士說之。精力不倦,致損右目。公主每節其業,為飾居宇。雖童稚之中,一無所好。十二,為國子生。十四,舉清茂,除秘書郎,與范陽張伯緒齊名,俱為太子舍人。丁父憂,居喪盡禮。服闋,除太子洗馬。時昭明尚幼,未與臣僚相接。高祖敕:「太子洗馬王錫、秘書郎張纘,親表英華,朝中髦俊,可以師友事之。」以戚屬封永安侯,除晉安王友,稱疾不行,敕許受詔停都。王冠日,以府僚攝事。



 普通初,魏始連和,使劉善明來聘,敕使中書舍人朱異接之,預宴者皆歸化北人。善明負其才氣,酒酣謂異曰:「南國辯學如中書者幾人?」異對曰:「異所以得接賓宴者,乃分職是司。二國通和,所敦親好;若以才辯相尚,則不容見使。」善明乃曰:「王錫、張纘,北間所聞,云何可見?」異具啟,敕即使於南苑設宴,錫與張纘、朱異四人而已。善明造席,遍論經史,兼以嘲謔,錫、纘隨方酬對,無所稽疑,未嘗訪彼一事,善明甚相歎挹。佗日謂異曰:「一日見二賢,實副所期,不有君子,安能為國!」



 轉中書郎,遷給事黃門侍郎、尚書吏部郎中,時年二十四。謂親友曰:「吾以外戚,
 謬被時知,多叨人爵,本非其志;兼比羸病,庶務難擁,安能捨其所好而徇所不能。」乃稱疾不拜。便謝遣胥徒,拒絕賓客,掩扉覃思,室宇蕭然。中大通六年正月,卒,時年三十六。贈侍中,給東園秘器,朝服一具,衣一襲。謚貞子。子泛、湜。



 僉字公會,錫第五弟也。八歲丁父憂,哀毀過禮。服闋,召補國子生,祭酒袁昂稱為通理。策高第,除長史兼秘書郎中,歷尚書殿中郎,太子中舍人,與吳郡陸襄對掌東宮管記。出為建安太守。山酋方善、謝稀聚徒依險,屢為民患,僉潛設方略,率眾平之,有詔褒美,頒示州郡。除武威
 將軍、始興內史,丁所生母憂,固辭不拜。又除寧遠將軍、南康內史,屬盧循作亂,復轉僉為安成內史,以鎮撫之。還除黃門侍郎,尋為安西武陵王長史、蜀郡太守。僉憚岨嶮,固以疾辭,因以黜免。久之,除戎昭將軍、尚書左丞,復補黃門侍郎,遷太子中庶子,掌東宮管記。太清二年十二月,卒,時年四十五。贈侍中,給東園秘器,朝服一具,衣一襲。承聖三年,世祖追詔曰:「賢而不伐曰恭,謚恭子。」



 張充,字延符,吳郡人。父緒,齊特進、金紫光祿大夫,有名前代。充少時,不持操行,好逸游。緒嘗請假還吳,始入西
 郭,值充出獵,左手臂鷹,右手牽狗,遇緒船至,便放紲脫,拜於水次。緒曰:「一身兩役,無乃勞乎?」充跪對曰:「充聞三十而立,今二十九矣,請至來歲而敬易之。」緒曰:「過而能改,顏氏子有焉。」及明年,便脩身改節。學不盈載,多所該覽,尤明《老》、《易》,能清言,與從叔稷俱有令譽。



 起家撫軍行參軍,遷太子舍人、尚書殿中郎、武陵王友。時尚書令王儉當朝用事,武帝皆取決焉。武帝嘗欲以充父緒為尚書僕射,訪於儉,儉對曰:「張緒少有清望,誠美選也;然東士比無所執,緒諸子又多薄行,臣謂此宜詳擇。」帝遂止。先是充兄弟皆輕俠,充少時又不護細行,故儉言之。
 充聞而慍,因與儉書曰:吳國男子張充致書於瑯邪王君侯侍者:頃日路長,愁霖韜晦,涼暑未平,想無虧攝。充幸以魚釣之閑,鐮採之暇,時復以卷軸自娛,逍遙前史。從橫萬古,動默之路多端;紛綸百年,昇降之途不一。故以圓行方止,器之異也;金剛水柔,性之別也。善御性者,不違金水之質;善為器者,不易方圓之用。所以北海掛簪帶之高,河南降璽書之貴。充生平少偶,不以利欲干懷,三十六年,差得以棲貧自澹。介然之志,峭聳霜崖;確乎之情,峰橫海岸。彯纓天閣,既謝廊廟之華;綴組雲臺,終慚衣冠之秀。所以擯跡江皋,陽狂隴畔者,實由氣岸
 疏凝,情塗狷隔。獨師懷抱,不見許於俗人;孤秀神崖,每邅回於在世。故君山直上,蹙壓於當年;叔陽夐舉,甚稟乎千載。充所以長群魚鳥,畢影松阿。半頃之田,足以輸稅;五畝之宅,樹以桑麻。嘯歌於川澤之間,諷味於澠池之上,泛濫於漁父之遊,偃息於卜居之下。如此而已,充何謝焉。



 若夫驚巖罩日,壯海逢天;竦石崩尋,分危落仞。桂蘭綺靡,叢雜於山幽;松柏森陰,相繚於澗曲。元卿於是乎不歸,伯休亦以茲長往。若迺飛竿釣渚,濯足滄洲;獨浪煙霞,高臥風月。悠悠琴酒,岫遠誰來?灼灼文談,空罷方寸。不覺鬱然千里,路阻江川。每至西風,何嘗不眷?
 聊因疾隙,略舉諸襟;持此片言,輕枉高聽。



 丈人歲路未強,學優而仕;道佐蒼生,功橫海望。入朝則協長倩之誠,出議則抗仲子之節。可謂盛德維時,孤松獨秀者也。素履未詳,斯旅尚眇。茂陵之彥,望冠蓋而長懷;霸山之氓,佇衣車而聳歎。得無惜乎?若鴻裝撰御,鶴駕軒空,則岸不辭枯,山被其潤。奇禽異羽,或巖際而逢迎;弱霧輕煙,乍林端而奄藹。東都不足奇,南山豈為貴。



 充昆西之百姓,岱表之一民。蠶而衣,耕且食,不能事王侯,覓知己,造時人,騁遊說,蓬轉於屠博之間,其歡甚矣。丈人早遇承華,中逢崇禮。肆上之眷,望溢於早辰;鄉下之言,謬延於
 造次。然舉世皆謂充為狂,充亦何能與諸君道之哉?是以披聞見,掃心胸,述平生,論語默,所以通夢交魂,推衿送抱者,其惟丈人而已。



 關山夐隔,書罷莫因,儻遇樵者,妄塵執事。



 儉言之武帝,免充官,廢處久之。後為司徒諮議參軍,與瑯邪王思遠、同郡陸慧曉等,並為司徒竟陵王賓客。入為中書侍郎,尋轉給事黃門侍郎。明帝作相,以充為鎮軍長史。出為義興太守,為政清靜,民吏便之。尋以母憂去職,服闋,除太子中庶子,遷侍中。義師近次,東昏召百官入宮省,朝士慮禍,或往來酣宴,充獨居侍中省,不出閣。城內既害東昏,百官集西鐘下,召充不至。



 高祖霸府開,以充為大司馬諮議參軍,遷梁王國郎中令、祠部尚書、領屯騎校尉,轉冠軍將軍、司徒左長史。天監初,除大常卿。尋遷吏部尚書,居選稱為平允。俄為散騎常侍、雲騎將軍。尋除晉陵太守,秩中二千石。徵拜散騎常侍、國子祭酒。充長於義理,登堂講說,皇太子以下皆至。時王侯多在學,執經以拜,充朝服而立,不敢當也。轉左衛將軍,祭酒如故。入為尚書僕射,頃之,除雲麾將軍、吳郡太守。下車恤貧老,故舊莫不欣悅。以疾自陳,徵為散騎常侍,金紫光祿大夫,未及還朝,十三年,卒于吳,時年六十六。詔贈侍中、護軍將軍。謚穆子。子最嗣。



 柳惲字文暢,河東解人也。少有志行,好學,善尺牘。與陳郡謝淪鄰居,淪深所友愛。初,宋世有嵇元榮、羊蓋,並善彈琴,云傳戴安道之法,惲幼從之學,特窮其妙。齊竟陵王聞而引之,以為法曹行參軍,雅被賞狎。王嘗置酒後園,有晉相謝安鳴琴在側,以授惲,惲彈為雅弄。子良曰:「卿巧越嵇心,妙臻羊體,良質美手,信在今辰。豈止當世稱奇,足可追蹤古烈。」累遷太子洗馬,父憂去官。服闋,試守鄱陽相,聽吏屬,得盡三年喪禮,署之文教,百姓稱焉。還除驃騎從事中郎。



 高祖至京邑,惲候謁石頭,以為冠軍將軍、征東府司馬。時東昏未平,士猶苦戰,惲上箋陳
 便宜,請城平之日,先收圖籍,及遵漢祖寬大愛民之義,高祖從之。會蕭穎胄薨於江陵,使惲西上迎和帝,仍除給事黃門侍郎,領步兵校尉,遷相國右司馬。天監元年,除長史、兼侍中,與僕射沈約等共定新律。



 惲立行貞素,以貴公子早有令名,少工篇什。始為詩曰:「亭皋本葉下,隴首秋雲飛。」瑯邪王元長見而嗟賞,因書齋壁。至是預曲宴,必被詔賦詩。嘗奉和高祖《登景陽樓》中篇云:「太液滄波起,長楊高樹秋。翠華承漢遠,雕輦逐風遊。」深為高祖所美。當時咸共稱傳。



 惲善奕棋,帝每敕侍坐,仍令定棋譜,第其優劣。二年,出為吳興太守。六年。徵為散騎常
 侍,遷左民尚書。八年,除持節、都督廣、交、桂、越四州諸軍事、仁武將軍、平越中郎將、廣州刺史。徵為秘書監,領左軍將軍。復為吳興太守六年,為政清靜,民吏懷之。於郡感疾,自陳解任,父老千餘人拜表陳請,事未施行。天監十六年,卒,時年五十三。贈侍中、中護軍。



 惲既善琴,嘗以今聲轉棄古法,乃著《清調論》,具有條流。



 少子偃,字彥游。年十二引見。詔問讀何書,對曰《尚書》。又曰:「有何美句?」對曰:「德惟善政,政在養民。」眾咸異之。詔尚長城公主,拜駙馬都尉,都亭侯,太子舍人,洗馬,廬陵、鄱陽內史。大寶元年,卒。



 蔡撙,字景節,濟陽考城人。父興宗,宋左光祿大夫、開府儀同三司,有重名前代。撙少方雅退默,與兄寅俱知名。選補國子生,舉高第,為司徒法曹行參軍。齊左衛將軍王儉高選府僚,以撙為主簿。累遷建安王文學,司徒主簿、左西屬。明帝為鎮軍將軍,引為從事中郎,遷中書侍郎,中軍長史,給事黃門侍郎。丁母憂,廬於墓側。齊末多難,服闋,因居墓所。除太子中庶子,太尉長史,並不就。梁臺建,為侍中,遷臨海太守,坐公事左遷太子中庶子。復為侍中,吳興太守。



 天監九年,宣城郡吏吳承伯挾祅道聚眾攻宣城,殺太守朱僧勇。因轉屠旁縣,踰山寇吳興,
 所過皆殘破,眾有二萬,奄襲郡城。東道不習兵革,吏民恇擾奔散,並請撙避之。撙堅守不動,募勇敢固郡。承伯盡銳攻撙,撙命眾出拒,戰於門,應手摧破,臨陣斬承伯,餘黨悉平。加信武將軍。徵度支尚書,遷中書令。復為信武將軍、晉陵太守。還,除通直散騎常侍、國子祭酒。遷吏部尚書,居選,弘簡有名稱。又為侍中,領秘書監,轉中書令,侍中如故。普通二年,出為宣毅將軍、吳郡太守。四年,卒,時年五十七。追贈侍中、金紫光祿大夫、宣惠將軍。謚康子。



 子彥熙,歷官中書郎,宣城內史。



 江蒨,字彥標,濟陽考城人。曾祖湛,宋左光祿、儀同三司;
 父斅,齊太常卿:並有重名於前世。



 蒨幼聰警,讀書過目便能諷誦。選為國子生,通《尚書》,舉高第。起家秘書郎,累遷司徒東閣祭酒、廬陵王主簿。居父憂以孝聞,廬於墓側,明帝敕遣齊仗二十人防墓所。服闋,除太子洗馬,累遷司徒左南屬,太子中舍人,秘書丞。出為建安內史,視事期月,義師下次江州,遣寧朔將軍劉諓之為郡,蒨帥吏民據郡拒之。及建康城平,蒨坐禁錮。俄被原,起為後軍臨川王外兵參軍。累遷臨川王友,中書侍郎,太子家令,黃門侍郎,領南兗州大中正。遷太子中庶子,中正如故。轉中權始興王長史。出為伏波將軍、晉安內史。在政
 清約,務在寬惠,吏民便之。詔徵為寧朔將軍、南康王長史,行府、州、國事。頃之,遷太尉臨川王長史,轉尚書吏部郎,右將軍。



 蒨方雅有風格。僕射徐勉以權重自遇,在位者並宿士敬之,惟蒨及王規與抗禮,不為之屈。勉因蒨門客翟景為第七兒繇求蒨女婚,蒨不答,景再言之,乃杖景四十,由此與勉有忤。除散騎常侍,不拜。是時勉又為子求蒨弟葺及王泰女,二人並拒之。葺為吏部郎,坐杖曹中幹免官,泰以疾假出宅,乃遷散騎常侍,皆勉意也。初,天監六年,詔以侍中、常侍並侍帷幄,分門下二局入集書,其官品視侍中,而非華胄所悅,故勉斥泰為之。
 蒨尋遷司徒左長史。



 初,王泰出閣,高祖謂勉云:「江蒨資歷,應居選部。」勉對曰:「蒨有眼患,又不悉人物。」高祖乃止。遷光祿大夫。大通元年,卒,時年五十三。詔贈本官。謚肅子。



 蒨好學,尤悉朝儀故事,撰《江左遺典》三十卷,未就,卒。文集十五卷。



 子紑、經,在《孝行傳》。



 史臣曰:王氏自姬姓已降,及乎秦漢,繼有英哲。洎東晉王茂弘經綸江左,時人方之管仲。其後蟬冕交映,台袞相襲,勒名帝籍,慶流子孫,斯為盛族矣。王瞻等承藉茲基,國華是貴,子有才行,可得而稱。張充少不持操,晚乃折節,在於典選,實號廉平。柳惲以多藝稱,蔡撙以方雅
 著,江蒨以風格顯,俱為梁室名士焉。



\end{pinyinscope}