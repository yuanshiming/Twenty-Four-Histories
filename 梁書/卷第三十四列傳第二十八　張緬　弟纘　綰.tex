\article{卷第三十四列傳第二十八 張緬 弟纘 綰}

\begin{pinyinscope}

 張緬,字元長,車騎將軍弘策子也。年數歲,外祖中山劉仲德異之,嘗曰:「此兒非常器,為張氏寶也。」齊永元末,義師起,弘策從高祖入伐,留緬襄陽,年始十歲,每聞軍有勝負,憂喜形於顏色。天監元年,弘策任衛尉卿,為妖賊所害,緬痛父之酷,喪過於禮,高祖遣戒喻之。服闋,襲洮陽縣侯,召補國子生。起家秘書郎,出為淮南太守,時年
 十八。高祖疑其年少未閑吏事,乃遣主書封取郡曹文案,見其斷決允愜,甚稱賞之。還除太子舍人、雲麾外兵參軍。緬少勤學,自課讀書,手不輟卷,尤明後漢及晉代眾家。客有執卷質緬者,隨問便對,略無遺失。殿中郎缺,高祖謂徐勉曰:「此曹舊用文學,且居鵷行之首,宜詳擇其人。」勉舉緬充選。頃之,出為武陵太守,還拜太子洗馬,中舍人。緬母劉氏,以父沒家貧,葬禮有闕,遂終身不居正室,不隨子入官府。緬在郡所得祿俸不敢用,乃至妻子不易衣裳,及還都,並供其母賑贍親屬,雖累載所畜,一朝隨盡,緬私室常闃然如貧素者。累遷北中郎諮議
 參軍、寧遠長史。出為豫章內史。緬為政任恩惠,不設鉤距,吏人化其德,亦不敢欺,故老咸云「數十年未之有也」。



 大通元年,徵為司徒左長史,以疾不拜,改為太子中庶子,領羽林監。俄遷御史中丞,坐收捕人與外國使鬥,左降黃門郎,兼領先職,俄復為真。緬居憲司,推繩無所顧望,號為勁直。高祖乃遣畫工圖其形於臺省,以勵當官。中太通三年,遷侍中,未拜,卒,時年四十二。詔贈侍中,加貞威將軍,侯如故。賻錢五萬,布五十匹。高祖舉哀。昭明太子亦往臨哭,與緬弟纘書曰:「賢兄學業該通,蒞事明敏,雖倚相之讀墳典,卻縠之敦《詩》《書》,惟今望古,蔑以斯
 過。自列宮朝,二紀將及,義惟僚屬,情實親友。文筵講席,朝游夕宴,何曾不同茲勝賞,共此言寄。如何長謝,奄然不追!且年甫強仕,方申才力,摧苗落穎,彌可傷惋。念天倫素睦,一旦相失,如何可言。言及增哽,巉筆無次。」



 緬性愛墳籍,聚書至萬餘卷。抄《後漢》、《晉書》,眾家異同,為《後漢紀》四十卷,《晉抄》三十卷。又抄《江左集》,未及成。文集五卷。子傅嗣。



 纘字伯緒,緬第三弟也,出後從伯弘籍。弘籍,高祖舅也,梁初贈廷尉卿。纘年十一,尚高祖第四女富陽公主,拜駙馬都尉,封利亭侯,召補國子生。起家秘書郎,時年十七。身長七尺四寸,眉目疏朗,神採爽發。高祖異之,嘗曰:「張壯武云『後八葉有逮吾者』,其此子乎?」纘好學,兄緬有書萬餘卷,晝夜披讀,殆不輟手。秘書郎有四員,宋、齊以來,為甲族起家之選,待次入補,其居職,例數十百
 日便遷任。纘固求不徙,欲遍觀閣內圖籍。嘗執四部書目曰:「若讀此畢,乃可言優仕矣。」如此數載,方遷太子舍人,轉洗馬、中舍人,並掌管記。



 纘與瑯邪王錫齊名。普通初,魏遣彭城人劉善明詣京師請和,求識纘。纘時年二十三,善明見而嗟服。累遷太尉諮議參軍,尚書吏部郎,俄為長史兼侍中,時人以為早達。河東裴子野曰:「張吏部在喉舌之任,已恨其晚矣。」子野性曠達,自云「年出三十,不復詣人。」初未與纘遇,便虛相推重,因為忘年之交。



 大通元年,出為寧遠華容公長史,行瑯邪、彭城二郡國事。二年,仍遷華容公北中郎長史、南蘭陵太守,加貞威將軍,行府州事。三年,入為度支尚書,母憂去職。服闋,出為吳興太守。纘治郡,省煩苛,務清靜,民吏便之。大同二年,徵為吏部尚書。纘居選,其後門寒素,有一介皆見引拔,不為貴要屈意,人士翕然稱之。



 五年,高祖手詔曰:「纘外氏英華,朝中領袖,司空以後,名冠範陽。可尚書僕射。」初,纘與參掌何敬容意趣不協,敬容居權軸,賓客輻湊,有過詣纘者,輒距不前,曰:「吾不能對何敬容殘客。」及是遷,為表曰:「自出守股肱,入尸衡尺,可以仰首伸眉,
 論列是非者矣。而寸衿所滯,近蔽耳目,深淺清濁,豈有能預。加以矯心飾貌,酷非所閑,不喜俗人,與之共事。」此言以指敬容也。纘在職,議南郊御乘素輦,適古今之衷;又議印綬官備朝服,宜並著綬,時並施行。



 九年,遷宣惠將軍、丹陽尹,未拜,改為使持節、都督湘、桂、東寧三州諸軍事、湘州刺史。述職經途,乃作《南征賦》。其詞曰:歲次娵訾,月惟中呂,餘謁帝於承明,將述職於南楚。忽中川而反顧,懷舊鄉而延佇;路漫漫以無端,情容容而莫與。乃弭節嘆曰:人之寓於宇宙也,何異夫棲蝸之爭戰,附蚋之遊禽。而盈虛倚伏,俯仰浮沉,矜榮華於尺影,總萬慮於寸陰。彼忘機於粹日,乃聖達之明箴。妙品物於貞觀,曾何足而系心。撫余躬之末迹,屬興王之盛世;蒙三欒之休寵,荷通家之渥惠。登石渠之三閣,典校文乎六藝。振長纓於承華,眷儲皇之上睿。居銜觴而接席,出方舟以同濟。彼華坊與禁苑,常宵盤而晝憩。思德音其在耳,若清塵之未逝。經二紀以及茲,悲明離之永翳。惟平生之褊能,實有志於棲息。慚滅沒之千里,謝韓哀於八極。如蓑裘之代用,譬輪轅之曲直。愧周任之清規,諒
 無取於陳力。逢濯纓之嘉運,遇井汲之明時。懷君恩而未答,顧靈瑣而依遲。總端揆以居副,長庶僚而稱師。猶深泉之短綆,若高墉而無基。伊吾人之罪薄,豈斯滿之能持。奉皇命以奏舉,方驅傳於衡疑。遵夕宿以言邁,戒晨裝而永辭。行搖搖於南逝,心眷眷而西悲。



 爾乃橫濟牽牛,傍瞻雉庫;前觀隱脈,卻視雲布。追晉氏之啟戎,覆中州之鼎祚。鞠三川於茂草,霑兩京於朝露。故黃旗紫蓋,運在震方;金陵之兆,允符厥祥。及歸命之銜璧,爰獻璽於武王;啟中興之英主,宣十世而重光。觀其內招人望,外攘干紀;草創江南,締構基址。豈徒能布其德,主晉有祀,《雲漢》作詩,《
 斯干》見美而已哉!乃得正朔相承,于茲四代;多歷年所,二百餘載。割疆埸於華戎,拯生靈於宇內;不被髮而左衽,翽明德其是賚。次臨滄之層巘,尋叔寶之舊埏;蘊珠玉之餘潤,昭羅綺之遺妍。懷若人之遠理,豈喜慍其能遷。雖魂埋於百世,猶映澈於九泉。經法王之梵宇,睹因時之或躍;從四海之宅心,故取亂而誅虐。在蒼精之將季,剪洪柯以銷落;既觀蠍而逞刑,又施獸而為謔。候高熢以巧笑,俟長星而懽噱。何惵惵之黔首,思假命其無託。信人欲而天從,爰物睹而聖作。



 我皇帝膺籙受圖,聰明神武,乘釁而運,席卷三楚。師克在和,仁義必取;形猶
 積決,應若飆舉。於是殪桑林之封豨,繳青丘之大風,戢干戈以耀德,肆《時夏》而成功。放流聲於鄭、衛,屏艷質於傾宮;配軒皇以邁迹,豈商、周之比隆。化致長平,于茲四紀;六夷膜拜,八蠻同軌。教穆於上庠,冤申於大理;顯三光之照燭,降五靈之休祉。諒殊功於百王,固無得而稱矣。



 溯金牛之迅渚,睹靈山之雄壯,實江南之丘墟,平雲霄而竦狀。標素嶺乎青壁,葺赬文於翠嶂;跳巨石以驚湍,批衝巖而駭浪。鏟千尋之峭岸,巘萬流之大壑;隱日月以蔽虧,摶風煙而回薄。崖映川而晃朗,水騰光而倏爍;積霜霰之往還,鼓波濤之前卻。下流沫以洊險,上岑
 崟而將落;聞知命之是虞,故違風而靡託。訊會骸之詭狀,云怒特之來奔。及漁人之垂餌,沉潛鎖於洪源。鑒幽塗於忠武,馳四馬之高軒。不語神以征怪,情存之而勿論。曬姑孰之舊朔,訪遺迹兮宣武;挾仲謀之雄氣,朝委裘而作輔。歷祖宗之明君,猶負芒於盛主;勢傾河以覆岱,威回天而震宇。雖明允之篤誠,在伊、稷而未舉;矧有功而無志,豈季葉其能處。懼貽笑於文、景,憂象賢之覆餗;雖苞蘗以代興,終夷宗而殄族。彼儋石之贏儲,尚邀之而俟福;況神明之大寶,乃暗干於天祿。造扃鍵之候司,發傳書於關尉;據蒐轅乎伊洛,守衡津於河渭。無矯
 且以招賓,闕捐繻而待貴。賓祗敬於王典,懷鞠躬而屏氣。惟函谷之襟帶,疑武庫之精兵。採風謠於往昔,聞乳虎於寧成。在當今而簡易,止譏鑒其姦情;陋文仲之廢職,鄙耏門之食征。



 於是近睇赭岑,遙瞻鵲岸,島嶼蒼茫,風雲蕭散。屬時雨之新晴,觀百川之浩涆;水泓澄以闇夕,山參差而辨旦。忽臨睨於故鄉,眇江天其無畔;逆洄流而右阻,遵長薄而左貫。獨向風以舒情,搴芳洲其誰玩。息銅山而繫纜,訪叔文之靈宇;得舊名而猶存,皆攢蕪而積楚。想夫君之令問,實有聲於前古;拯巴漢之廢業,爰配名於鄒魯。辨山精以息訟,對祠星而寤主。每撫
 事以懷人,非末學其能睹。嘉梅根之孝女,尚乘肥於媵姬;嗟吳人之重辟,憂峻網於將貽。彼沈瓜而顯義,指滄波而為期;此浮履以明節,赴丹沄其何疑。信理感而情悼,實+心妻悵於餘悲;空沈吟以遐想,愧邯鄲之妙詞。望南陵以寓目,美牙門之守志;當晉師之席卷,豈籓籬而不庇。攜老弱於窮城,猶區區乎一簣。雖挈瓶之小善,實君子之所識。……是謂事人之禮。



 入雷池之長浦,想恭、岱之芳塵;臨魚官以輟膳,踐寒蒲之抽筠。又有生為令德,沒為明神。或捐家事主,攜手拜親;或正身殉義,哀感市人。所以家稱純孝,國號能臣。揚清徽於上列,並異世
 而為鄰。發曉渚而溯風,苦神吳之難習。岸曜舟而不進,水騰沙以驚急。天曀曀其垂陰,雨霏霏而來集;愍征夫之勞瘁,每搴帷而佇立。由江沲之派別,望彭匯之通津,塗未中乎及絳,日已盈於浹旬。



 於是千流共歸,萬嶺分狀;倒影懸高,浮天瀉壯。清江洗滌,平湖夷暢;翻光轉彩,出沒搖漾。岷山、嶓冢,悠遠寂寥;青湓、赤岸,控汐引潮。望歸雲之蓊蓊,揚清風之飄飄;界飛流於翠薄,耿長虹於青霄。若夫灌莽川涯,層潭水府,游泳之所往還,喧鳴之所攢聚。群飛沙漲,掩薄草渚;奇甲異鱗,雕文綷羽。聽寡鶴之偏鳴,聞孤鴻之慕侶;在客行而多思,獨傷魂而心
 妻楚。美中流之衝要,因習坎以守固。既固之而設險,又居之而務德。南通珠崖、夜郎,西款玉津、華墨。莫不內清姦宄,外弭苛慝,籬屏京師,事有均於齊德也。



 眄匡嶺以躊躇,想霞裳於雲仞;流亙娥之逸響,發王子之清韻。若夜光而可投,豈榮華之難擯。羨還丹其何術,佇一丸於來信。徑遵途乎鄂渚,迹孫氏之霸基;陳利兵而蓄粟,抗十倍之銳師。在賢才之必用,寧推誠而忍欺;圖富強以法立,屬貞臣而日嬉。識徐基於江畔,云釣臺之舊址;方戰國之多虞,猶從容而宴喜。欽輔吳之忠諒,歎仲謀之虛己;處君臣而並得,良致霸其有以。伊文侯之雅望,誠一
 代之偉人;禰觀書以心服,玉比德而譽均。遘時雄之應運,方協義以經綸;名既逼而愈賞,言雖聞而彌親。惜勤王於延獻,俾漢京之惟新;何天命其弗與,悲盛業之未申。泛蘆洲以延佇,聞伍員之所濟;出懷珠而免仇,歸投金以答惠。彼無求於萬鐘,唯長歌而鼓世;慨斯誠之未感,乃沈軀以明誓。空負恨其何追,徒臨飡而先祭;及旋師於鄭國,美邀福於來裔。入郢都而抵掌,壯天險之難窺;允分荊之勝略,成百代之良規。賈生方於指大,應侯譬之木披。所以居宗振末,強本弱枝,聞古今之通制,歷盛衰而不移,可不謂然與,美經國之遠體也。



 酌忠言於
 城郢,播終古之芳猷;忘我躬之匪閱,顧社稷而懷憂。服莊王之高義,乃徵名於夏州;恥蹊田之過罰,納申叔之嘉謀。觀巫臣之獻箴,鑒《周書》以明喻;何自謀其多僻,要桑中而遠赴。若葆申之誅丹,實匡君以成務;在兩臣而優劣,居二主其並裕。臨赤崖而慷愾,榷雄圖於魏武;乘戰勝以長驅,志吞吳而并楚。總八州之毅卒,期姑蘇而振旅;時有便乎建瓴,事無留於蕭斧。霸孫赫其霆奮,杖邁俗之英輔;裂宇宙而三分,誠決機乎一舉。嗟玄德之矯矯,思興復於舊京;招臥龍於當世,配管仲而稱英。收散亡之餘弱,結與國而連橫,延五紀乎岷漢,紹四百於
 炎精。望巴丘以邅回,遵洞庭而敞恍,沉輕舟而不繫,何靈胥之浩蕩。眺君、褊之雙峰,徒臨風以增想;償瑤觴而一酌,駕彩蜺而獨往。



 爾乃南奠衡、霍,北距沮、漳;包括沅、澧,汲引瀟、湘。滮々長邁,漫漫回翔;蕩雲沃日,吐霞含光。青碧潭嶼,萬頃澄澈;綺蘭從風,素沙被雪。雜雲霞以舒卷,間河洲而斷絕;回曉仄於中川,起長飆而半滅。稅遺構之舊浦,瞻汨羅以隕泗;豈懷寶而迷邦,猶殷勤而一致。蘊芳華以襞積,非黨人之所媚;合《小雅》之怨辭,兼《國風》之美志。譬彈冠而振衣,猶自別於泥滓;且殺身以成義,寧露才而揚己?悲先生之不辰,逢椒、蘭之妒美;有驊
 騮而不馭,焉遑遑於千里。既踐境以思人,彌流連其無已。脩行潦之薄薦,敢憑誠於沼沚。謁黃陵而展敬,奠瑤席乎川湄。具蘭香以膏沐,懷椒糈而要之。延帝子于三后,降夔、龍於九疑。騰河靈之水駕,下太一之靈旗。撫安歌以會儛,疏緩節而依遲。日徘徊以將暮,情眇默而無辭。慍秦皇之巡幸,尤土壤以加戮;昧天道之無親,勤望祀以祈福。將人怨而神怒,故飛川而蕩谷;推冥理以歸愆,遂刊山而赭木。



 於是下車入部,班條理務,砥課庸薄,夕惕兢懼。存問長老,隱恤氓庶,奉宣皇恩,寬徭省賦。遠哉盛乎,斯邦之舊也。有虞巡方以託終,夏后開圖而疏
 決,太伯讓嗣以來遊,□臣祈仙而齊潔。固是明王之塵軌,聖賢之蹤轍也。若夫屈平《懷沙》之賦,賈子遊湘之篇,史遷摛文以投弔,揚雄《反騷》而沉川。其風謠雅什,又是詞人之所流連也。亦有仲寧、咸德,仍世相繼,父子三台,緇衣改敝。古初抱於烈火,劉先高而忤世,蔣公琰之弘通,桓柏緒之匡濟,鄧兗時之絕述,穀思恭之藻麗,實川嶽之精靈,常間出而無替也。至於殊庭之客,帝鄉之賢,神奔鬼化,吐吸雲煙。玉笥登之而卻老,金人植杖以尊泉,蘇生騎龍而出入,處靜駕鹿以周旋。配北燭之神女,偶南榮之偓佺。時仿佛其遙見,亦往往而有焉。



 爾乃歷
 省府庭,周行街術,山川遠覽,邑居近悉。割黔中以置守,獻青陽而背質,鄒生所謂還舟,楚王於焉乘馹。巡高山之累仞,褒吳文之為宰;彼非劉而八王,皆國亡而身醢。在長沙而著令,經五葉其未改;知天道之福謙,勝一時之經始。尋太傅之故宅,今築室以安禪;邑無改於舊井,尚開流而冽泉。懷伊、管之政術,遇庸臣而見遷;終被知於時主,嗟漢宗之得賢。受齊君之遠託,豈理謝而生全;哀懷王之不秀,遂抱恨而傷年。脩定祀于北郭,對林野而幽藹;庶無吐於馨香,祀瓊茅而沃酹。景十三以啟國,惟君王其能大;迨炎正之中微,實斯籓而是賴。顧四阜
 之紆餘,乍升高以遊目;審山川之面帶,將取名於衡麓。下彌漫以爽塏,上欽虧而重復;風瑟瑟以鳴松,水琤琤而響谷。低四照於若華,竦千尋於建木。冀囂塵之可屏,登巖阿而寤宿。捨域中之常戀,慕遊仙之靈族。是時涼風暮節,萬實西成,華池迥遠,飛閣淒明。嘉南州之炎德,愛蘭蕙之秋榮。下名柑於曲榭,採芳菊於高城。樹羅軒而並列,竹被嶺而叢生。玩棲禽之夕返,送旅鴈之晨征。悲去鄉而遠客,寄覽物而娛情。惟傳車之所騖,實鷹揚其是掌,或解組以立威,乍露服而加賞。遵聖主之恩刑,荷天地之厚德。沾河潤於九里,澤自家而刑國。闕小道
 之可觀,寧畏塗其易克;眄高衢而願騁,憂取累於長纆。聞困石之非據,承炯戒乎明則;愧壽陵之餘子,學邯鄲而匍匐也。



 纘至州,停遣十郡慰勞,解放老疾吏役,及關市戍邏先所防人,一皆省併。州界零陵、衡陽等郡,有莫徭蠻者,依山險為居,歷政不賓服,因此向化。益陽縣人作田二頃,皆異畝同穎。纘在政四年,流人自歸,戶口增益十餘萬,州境大安。



 太清二年,徵為領軍,俄改授使持節、都督雍、梁、北秦、東益、郢州之竟陵司州之隨郡諸軍事、平北將軍、寧蠻校尉。纘初聞邵陵王綸當代己為湘州,其後定用河東王譽,纘素輕少王,州府候迎及資待
 甚薄,譽深銜之。及至州,遂託疾不見纘,仍檢括州府庶事,留纘不遣。會聞侯景寇京師,譽飾裝當下援,時荊州刺史湘東王赴援,軍次郢州武城,纘馳信報曰:「河東已豎檣上水,將襲荊州。」王信之,便回軍鎮,荊、湘因構嫌隙。尋棄其部伍,單舸赴江陵,王即遣使責讓譽,索纘部下。既至,仍遣纘向襄陽,前刺史岳陽王察推遷未去鎮,但以城西白馬寺處之。會聞賊陷京師,察因不受代。州助防杜岸紿纘曰:「觀岳陽殿下必不容使君,使君素得物情,若走入西山,招聚義眾,遠近必當投集,又帥部下繼至,以此義舉,無往不克。」纘信之,與結盟約,因夜遁入山。
 岸反以告察,仍遣岸帥軍追纘。纘眾望岸軍大喜,謂是赴期,既至,即執纘並其眾,並俘送之。始被囚縶,尋又逼纘剃髮為道人。其年,察舉兵襲江陵,常載纘隨後。及軍退敗,行至湕水南,防守纘者慮追兵至,遂害之,棄屍而去,時年五十一。元帝承制,贈纘侍中、中衛將軍、開府儀同三司。謚簡憲公。



 纘有識鑒,自見元帝,便推誠委結。及元帝即位,追思之,嘗為詩,其《序》曰:「簡憲之為人也,不事王侯,負才任氣,見餘則申旦達夕,不能已已。懷夫人之德,何日忘之。」纘著《鴻寶》一百卷,文集二十卷。



 次子希,字子顏,早知名,選尚太宗第九女海鹽公主。承聖初,官至
 黃門侍郎。



 綰字孝卿,纘第四弟也。初為國子生,射策高第。起家長兼秘書郎,遷太子舍人,洗馬,中舍人,並掌管記。累遷中書郎,國子博士。出為北中郎長史、蘭陵太守,還除員外散騎常侍。時丹陽尹西昌侯蕭淵藻以久疾未拜,敕綰權知尹事,遷中軍宣城王長史,俄徙御史中丞。高祖遣其弟中書舍人絢宣旨曰:「為國之急,惟在執憲直繩,用人本不限升降。晉宋之世,周閔、蔡廓並以侍中為之,卿勿疑是左遷也。」時宣城王府望重,故有此旨焉。大同四年元日,舊制僕射中丞坐位東西相當,時綰兄纘為僕
 射,及百司就列,兄弟導騶,分趨兩陛,前代未有也,時人榮之。歲餘,出為豫章內史。綰在郡,述《制旨禮記正言》義,四姓衣冠士子聽者常數百人。



 八年,安成人劉敬宮挾祅道,遂聚黨攻郡,內史蕭侻棄城走。賊轉寇南康、廬陵,屠破縣邑,有眾數萬人,進寇豫章新淦縣。南中久不習兵革,吏民恇擾奔散。或勸綰宜避其鋒,綰不從,仍修城隍,設戰備,募召敢勇,得萬餘人。刺史湘東王遣司馬王僧辯帥兵討賊,受綰節度,旬月間,賊黨悉平。



 十年,復為御史中丞,加通直散騎常侍。綰再為憲司,彈糾無所回避,豪右憚之。是時城西開士林館聚學者,綰與右衛朱
 異、太府卿賀琛遞述《制旨禮記中庸》義。



 太清二年,遷左衛將軍。會侯景寇至,入守東掖門。三年,遷吏部尚書。宮城陷,綰出奔,外轉至江陵。湘東王承制,授侍中、左衛將軍、相國長史,侍中如故。出為持節、雲麾將軍、湘東內史。承聖二年,徵為尚書右僕射,尋加侍中。明年,江陵陷,朝士皆俘入關,綰以疾免,後卒於江陵,時年六十三。



 次子交,字少游,頗涉文學,選尚太宗第十一女安陽公主。承聖二年,官至太子洗馬,秘書丞,掌東宮管記。



 陳吏部尚書姚察曰:太清版蕩,親屬離貳,纘不能葉和籓岳,成溫陶之舉,茍懷私怨,構隙瀟湘,遂及禍於身,非
 由忠節;繼以江陵淪覆,實萌於此。以纘之風格,卒為梁之亂階,惜矣哉。



\end{pinyinscope}