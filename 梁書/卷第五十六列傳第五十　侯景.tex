\article{卷第五十六列傳第五十 侯景}

\begin{pinyinscope}

 侯景,字萬景,朔方人,或云鴈門人。少而不羈,見憚鄉里。及長,驍勇有膂力,善騎射。以選為北鎮戍兵,稍立功效。魏孝昌元年,有懷朔鎮兵鮮于脩禮,於定州作亂,攻沒郡縣;又有柔玄鎮兵吐斤洛周,率其黨與,復寇幽、冀,與脩禮相合,眾十餘萬。後脩禮見殺,部下潰散,懷朔鎮將葛榮因收集之,攻殺吐斤洛周,盡有其眾,謂之「葛賊」。四
 年,魏明帝殂,其后胡氏臨朝,天柱將軍爾朱榮自晉陽入殺胡氏,并誅其親屬。景始以私眾見榮,榮甚奇景,即委以軍事。會葛賊南逼,榮自討,命景先驅,至河內,擊,大破之,生擒葛榮,以功擢為定州刺史、大行臺,封濮陽郡公。景自是威名遂著。



 頃之,齊神武帝為魏相,又入洛誅爾朱氏,景復以眾降之,仍為神武所用。景性殘忍酷虐,馭軍嚴整;然破掠所得財寶,皆班賜將士,故咸為之用,所向多捷。總攬兵權,與神武相亞。魏以為司徒、南道行臺,擁眾十萬,專制河南。及神武疾篤,謂子澄曰:「侯景狡猾多計,反覆難知,我死後,必不為汝用。」乃為書召景。景
 知之,慮及於禍,太清元年,乃遣其行臺郎中丁和來上表請降曰:臣聞股肱體合,則四海和平;上下猜貳,則封疆幅裂。故周、邵同德,越常之貢來臻;飛、惡離心,諸侯所以背叛。此蓋成敗之所由,古今如畫一者也。



 臣昔與魏丞相高王並肩戮力,共平災釁,扶危戴主,匡弼社稷。中興以後,無役不從;天平及此,有事先出。攻城每陷,野戰必殄;筋力消於鞍甲,忠貞竭於寸心。乘藉機運,位階鼎輔;宜應誓死罄節,仰報時恩,隕首流腸,溘焉罔貳。何言翰墨,一旦論此?臣所恨義非死所,壯士弗為。臣不愛命,但恐死之無益耳。而丞相既遭疾患,政出子澄。澄天性
 險忌,觸類猜嫉,諂諛迭進,共相構毀。而部分未周,累信賜召;不顧社稷之安危,惟恐私門之不植。甘言厚幣,規滅忠梗。其父若殞,將何賜容。懼讒畏戮,拒而不返,遂觀兵汝、潁,擁璟周、韓。乃與豫州刺史高成、廣州刺史郎椿、襄州刺史李密、兗州刺史邢子才、南兗州刺史石長宣、齊州刺史許季良、東豫州刺史丘元征、洛州刺史朱渾願、揚州刺史樂恂、北荊州刺史梅季昌、北揚州刺史元神和等,皆河南牧伯,大州帥長,各陰結私圖,剋相影會,秣馬潛戈,待時即發。函谷以東,瑕丘以西,咸願歸誠聖朝,息肩有道,戮力同心,死無二志。惟有青、徐數州,僅須
 折簡,一驛走來,不勞經略。



 且臣與高氏釁隙已成,臨患賜征,前已不赴,縱其平復,終無合理。黃河以南,臣之所職,易同反掌,附化不難。群臣顒仰,聽臣而唱。若齊、宋一平,徐事燕、趙。伏惟陛下天網宏開,方同書軌,聞茲寸款,惟應霈然。



 丁和既至,高祖召群臣廷議。尚書僕射謝舉及百辟等議,皆云納侯景非宜,高祖不從是議而納景。及齊神武卒,其子澄嗣,是為文襄帝。高祖乃下詔封景河南王、大將軍、使持節、董督河南南北諸軍事、大行臺,承制輒行,如鄧禹故事,給鼓吹一部。齊文襄遣大將軍慕容紹宗圍景於長社,景請西魏為援,西魏遣其五城王
 元慶等率兵救之,紹宗乃退。景復請兵於司州刺史羊鴉仁,鴉仁遣長史鄧鴻率兵至汝水,元慶軍又夜遁。於是據懸瓠、項城,求遣刺史以鎮之。詔以羊鴉仁為豫、司二州刺史,移鎮懸瓠;西陽太守羊思建為殷州刺史,鎮項城。



 魏既新喪元帥,景又舉河南內附,齊文襄慮景與西、南合從,方為己患,乃以書喻景曰:蓋聞位為大寶,守之未易;仁誠重任,終之實難。或殺身成名,或去食存信;比性命於鴻毛,等節義於熊掌。夫然者,舉不失德,動無過事;進不見惡,退無謗言。



 先王與司徒契闊夷險,孤子相於,偏所眷屬,繾綣衿期,綢繆寤語,義貫終始,情存歲
 寒。司徒自少及長,從微至著,共相成生,非無恩德。既爵冠通侯,位標上等,門容駟馬,室饗萬鐘,財利潤於鄉黨,榮華被於親戚。意氣相傾,人倫所重,感於知己,義在忘軀。眷為國士者,乃立漆身之節;饋以壺飧者,便致扶輪之效。若然尚不能已,況其重於此乎!



 幸以故舊之義,欲持子孫相託,方為秦晉之匹,共成劉范之親。假使日往月來,時移世易,門無強蔭,家有幼孤,猶加璧不遺,分宅相濟,無忘先德,以恤後人。況聞負杖行歌,便已狼顧犬噬,於名無所成,於義無所取,不蹈忠臣之跡,自陷叛人之地。力不足以自強,勢不足以自保;率烏合之眾,為累
 卵之危。西求救於黑泰,南請援於蕭氏,以狐疑之心,為首鼠之事。入秦則秦人不容,歸吳則吳人不信。當今相視,未見其可,不知終久,持此安歸。相推本心,必不應爾。當是不逞之人,曲為口端之說,遂懷市虎之疑,乃致投杼之惑耳。



 比來舉止,事已可見,人相疑誤,想自覺知,合門大小,並付司寇。近者,聊命偏師,前驅致討,南兗、揚州,應時剋復。即欲乘機,長驅懸瓠;屬以炎暑,欲為後圖。方憑國靈,龔行天罰,器械精新,士馬彊盛。內外感德,上下齊心,三令五申,可蹈湯火。若使旗鼓相望,埃塵相接,勢如沃雪,事等注螢。夫明者去危就安,智者轉禍為福。寧使我
 負人,不使人負我。當開從善之門,決改先迷之路。今刷心盪意,除嫌去惡,想猶致疑,未便見信。若能卷甲來朝,垂丱還闕者,當授豫州刺史。即使終君之世,所部文武更不追攝。進得保其祿位,退則不喪功名。君門眷屬,可以無恙;寵妻愛子,亦送相還。仍為通家,卒成親好。所不食言,有如皎日。君既不能東封函谷,南向稱孤,受制於人,威名頓盡。空使兄弟子侄,足首異門,垂髮戴白,同之塗炭,聞者酸鼻,見者寒心,矧伊骨肉,能無愧也?



 孤子今日不應方遣此書,但見蔡遵道云:司徒本無歸西之心,深有悔禍之意,聞西兵將至,遣遵道向崤中參其多少;
 少則與其同力,多則更為其備。又云:房長史在彼之日,司徒嘗欲遣書啟,將改過自新。已差李龍仁,垂欲發遣,聞房已遠,遂復停發。未知遵道此言為虛為實,但既有所聞,不容不相盡告。吉兇之理,想自圖之。



 景報書曰:蓋聞立身揚名者,義也;在躬所寶者,生也。茍事當其義,則節士不愛其軀;刑罰斯舛,則君子實重其命。昔微子發狂而去殷,陳平懷智而背楚者,良有以也。僕鄉曲布衣,本乖藝用。初逢天柱,賜忝帷幄之謀;晚遇永熙,委以干戈之任。出身為國,綿歷二紀,犯危履難,豈避風霜。遂得躬被袞衣,口飧玉食,富貴當年,光榮身世。何為一旦舉
 旌璟,援桴鼓,而北面相抗者,何哉?實以畏懼危亡,恐招禍害,捐軀非義,身名兩滅故耳。何者?往年之暮,尊王遘疾,神不祐善,祈禱莫瘳。遂使嬖幸擅威權,閽寺肆詭惑,上下相猜,心腹離貳。僕妻子在宅,無事見圍;段康之謀,莫知所以;盧潛入軍,未審何故。翼翼小心,常懷戰心慄,有靦面目,寧不自疑。及回師長社,希自陳狀,簡書未達,斧鉞已臨。既旌旗相對,咫尺不遠,飛書每奏,兼申鄙情;而群卒恃雄,眇然不顧,運戟推鋒,專欲屠滅。築圍堰水,三板僅存,舉目相看,命懸晷刻,不忍死亡,出戰城下。禽獸惡死,人倫好生,送地拘秦,非樂為也。但尊王平昔見與,
 比肩共獎帝室,雖形勢參差,寒暑小異,丞相司徒,雁行而已。福祿官榮,自是天爵,勞而後受,理不相干,欲求吞炭,何其謬也!然竊人之財,猶謂為盜,祿去公室,相為不取。今魏德雖衰,天命未改,祈恩私第,何足關言。



 賜示「不能東封函谷,受制於人」,當似教僕賢祭仲而褒季氏。無主之國,在禮未聞,動而不法,何以取訓?竊以分財養幼,事歸令終,捨宅存孤,誰云隙末?復言僕「眾不足以自強,危如累卵」。然紂有億兆夷人,卒降十亂;桀之百剋,終自無後。潁川之戰,即是殷監。輕重由人,非鼎在德。茍能忠信,雖弱必彊。殷憂啟聖,處危何苦。況今梁道邕熙,招攜
 以禮,被我獸文,縻之好爵。方欲苑五岳而池四海,掃夷穢以拯黎元,東羈甌越,西通汧、隴。吳、楚剽勁,帶甲千群;吳兵冀馬,控弦十萬。兼僕所部義勇如林,奮義取威,不期而發,大風一振,枯幹必摧,凝霜暫落,秋蒂自殞。此而為弱,誰足稱彊!



 又見誣兩端,受疑二國。斟酌物情,一何至此!昔陳平背楚,歸漢則王;百里出虞,入秦斯霸。蓋昏明由主,用捨在時,奉禮而行,神其庇也。



 書稱士馬精新,剋日齊舉,誇張形勝,指期盪滅。竊以寒飂白露,節候乃同;秋風揚塵,馬首何異。徒知北方之力爭,未識西、南之合從,茍欲徇意於前途,不覺坑阱在其側。若云去危令
 歸正朔,轉禍以脫網羅,彼既嗤僕之愚迷,此亦笑君之晦昧。今已引二邦,揚旌北討,熊豹齊奮,剋復中原,荊、襄、廣、潁已屬關右,項城、懸瓠亦奉南朝,幸自取之,何勞恩賜。然權變不一,理有萬途。為君計者,莫若割地兩和,二分鼎峙,燕、衛、晉、趙足相奉祿,齊、曹、宋、魯悉歸大梁,使僕得輸力南朝,北敦姻好,束帛交行,戎車不動。僕立當世之功,君卒祖禰之業,各保疆界,躬享歲時,百姓乂寧,四民安堵。孰若驅農夫於隴畝,抗勍敵於三方,避干戈於首尾,當鋒鏑於心腹。縱太公為將,不能獲存,歸之高明,何以剋濟。



 復尋來書云,僕妻子悉拘司寇。以之見要,庶
 其可及。當是見疑褊心,未識大趣。何者?昔王陵附漢,母在不歸;太上囚楚,乞羹自若。矧伊妻子,而可介意。脫謂誅之有益,欲止不能;殺之無損,徒復坑戮。家累在君,何關僕也?而遵道所傳,頗亦非謬,但在縲紲,恐不備盡,故重陳辭,更論款曲。所望良圖,時惠報旨。然昔與盟主,事等琴瑟,讒人間之,翻為仇敵。撫弦搦矢,不覺傷懷,裂帛還書,知何能述。



 十二月,景率軍圍譙城不下,退攻城父,拔之。又遣其行臺左丞王偉、左民郎中王則詣闕獻策,求諸元子弟立為魏主,輔以北伐,許之。詔遣太子舍人元貞為咸陽王,須渡江,許即偽位,乘輿副御以資給之。



 齊文襄又遣慕容紹宗追景,景退入渦陽,馬尚有數千匹,甲卒數萬人,車萬餘輛,相持於渦北。景軍食盡,士卒並北人,不樂南渡,其將暴顯等各率所部降於紹宗。景軍潰散,乃與腹心數騎自峽石濟淮,稍收散卒,得馬步八百人,奔壽春,監州韋黯納之。景啟求貶削,優詔不許,仍以為豫州牧,本官如故。



 景既據壽春,遂懷反叛,屬城居民,悉召募為軍士,輒停責市估及田租,百姓子女悉以配將卒。又啟求錦萬匹,為軍人袍,領軍朱異議以御府錦署止充頒賞遠近,不容以供邊城戎服,請送青布以給之。景得布,悉用為袍衫,因尚青色。又以臺所給仗,
 多不能精,啟請東冶鍛工,欲更營造,敕並給之。景自渦陽敗後,多所徵求,朝廷含弘,未嘗拒絕。



 先是,豫州刺史貞陽侯淵明督眾軍圍彭城,兵敗沒于魏。至是,遣使還述魏人請追前好。二年二月,高祖又與魏連和。景聞之懼,馳啟固諫,高祖不從。爾後表疏跋扈,言辭不遜。鄱陽王範鎮合肥,及司州刺史羊鴉仁俱累啟稱景有異志,領軍朱異曰:「侯景數百叛虜,何能為役?」並抑不奏聞,而逾加賞賜,所以姦謀益果。又知臨賀王正德怨望朝廷,密令要結,正德許為內啟。八月,景遂發兵反,攻馬頭、木柵,執太守劉神茂、戍主曹璆等。於是詔郢州刺史鄱陽
 王範為南道都督,北徐州刺史封山侯正表為北道都督,司州刺史柳仲禮為西道都督,通直散騎常侍裴之高為東道都督,同討景,濟自歷陽;又令開府儀同三司、丹陽尹、邵陵王綸持節,董督眾軍。



 十月,景留其中軍王顯貴守壽春城,出軍偽向合肥,遂襲譙州,助防董紹先開城降之,執刺史豊城侯泰。高祖聞之,遣太子家令王質率兵三千巡江遏防。景進攻歷陽,歷陽太守莊鐵遣弟均率數百人夜斫景營,不克,均戰沒,鐵又降之。蕭正德先遣大船數十艘,偽稱載荻,實裝濟景。景至京口,將渡,慮王質為梗。俄而質無故退,景聞之尚未信也,乃密
 遣覘之。謂使者曰:「質若審退,可折江東樹枝為驗。」覘人如言而返,景大喜曰:「吾事辦矣。」乃自採石濟,馬數百匹,兵千人,京師不之覺。景即分襲姑孰,執淮南太守文成侯寧,遂至慈湖。於是詔以揚州刺史宣城王大器為都督城內諸軍事,都官尚書羊侃為軍師將軍以副焉;南浦侯推守東府城,西豊公大春守石頭城,輕車長史謝禧守白下。



 既而景至朱雀航,蕭正德先屯丹陽郡,至是,率所部與景合。建康令庾信率兵千餘人屯航北,見景至航,命徹航,始除一舶,遂棄軍走南塘,遊軍復閉航渡景。皇太子以所乘馬授王質,配精兵三千,使援庾信。質
 至領軍府,與賊遇,未陣便奔走,景乘勝至闕下。西豊公大春棄石頭城走,景遣其儀同于子悅據之。謝禧亦棄白下城走。景於是百道攻城,持火炬燒大司馬、東西華諸門。城中倉卒,未有其備,乃鑿門樓,下水沃火,久之方滅。賊又斫東掖門將開,羊侃鑿門扇,刺殺數人,賊乃退。又登東宮牆,射城內,至夜,太宗募人出燒東宮,東宮臺殿遂盡。景又燒城西馬廄、士林館、太府寺。明日,景又作木驢數百攻城,城上飛石擲之,所值皆碎破。景苦攻不剋,傷損甚多,乃止攻,築長圍以絕內外,啟求誅中領軍朱異、太子右衛率陸驗、兼少府卿徐膋、制局監周石珍
 等。城內亦射賞格出外:「有能斬景首,授以景位,并錢一億萬,布絹各萬匹,女樂二部。」



 十一月,景立蕭正德為帝,即偽位於儀賢堂,改年曰正平。初,童謠有「正平」之言,故立號以應之。景自為相國、天柱將軍,正德以女妻之。



 景又攻東府城,設百尺樓車,鉤城堞盡落,城遂陷。景使其儀同盧暉略率數千人,持長刀夾城門,悉驅城內文武裸身而出,賊交兵殺之,死者二千餘人。南浦侯推是日遇害。景使正德子見理、儀同盧暉略守東府城。



 景又於城東西各起一土山以臨城內,城內亦作兩山以應之,王公以下皆負土。初,景至,便望克定京師,號令甚明,不
 犯百姓。既攻城不下,人心離阻,又恐援軍總集,眾必潰散,乃縱兵殺掠,交屍塞路,富室豪家,恣意裒剝,子女妻妾,悉入軍營。及築土山,不限貴賤,晝夜不息,亂加毆棰,疲羸者因殺之以填山,號哭之聲,響動天地。百姓不敢藏隱,並出從之,旬日之間,眾至數萬。



 景儀同范桃棒密遣使送款乞降,會事泄見殺。至是,邵陵王綸率西豊公大春、新塗將軍永安侯確、超武將軍南安鄉侯駿、前譙州刺史趙伯超、武州刺史蕭弄璋、步兵校尉尹思合等,馬步三萬發自京口,直據鐘山。景黨大駭,具船舟咸欲逃散,分遣萬餘人距綸,綸擊大破之,斬首千餘級。旦日,景
 復陳兵覆舟山北,綸亦列陣以待之。景不進,相持。會日暮,景引軍還,南安侯駿率數十騎挑之,景迴軍與戰,駿退。時趙伯超陳於玄武湖北,見駿急,不赴,乃率軍前走,眾軍因亂,遂敗績。綸奔京口。賊盡獲輜重器甲,斬首數百級,生俘千餘人,獲西豊公大春、綸司馬莊丘惠達、直閣將軍胡子約、廣陵令霍俊等,來送城下徇之,逼云「已擒邵陵王」,俊獨云「王小小失利,已全軍還京口,城中但堅守,援軍尋至」。賊以刀毆之,俊言辭顏色如舊,景義而釋之。



 是日,鄱陽世子嗣、裴之高至後渚,結營于蔡洲。景分軍屯南岸。



 十二月,景造諸攻具及飛樓、橦車、登城車、
 登堞車、階道車、火車,並高數丈,一車至二十輪,陳於闕前,百道攻城並用焉。以火車焚城東南隅大樓,賊因火勢以攻城,城上縱火,悉焚其攻具,賊乃退。又築土山以逼城,城內作地道以引其土山,賊又不能立,焚其攻具,還入于柵。材官將軍宋嶷降賊,因為立計,引玄武湖水灌臺城,城外水起數尺,闕前御街並為洪波矣。又燒南岸民居營寺,莫不咸盡。



 司州刺史柳仲禮、衡州刺史韋粲、南陵太守陳文徹、宣猛將軍李孝欽等,皆來赴援。鄱陽世子嗣、裴之高又濟江。仲禮營朱雀航南,裴之高營南苑,韋粲營青塘,陳文徹、李孝欽屯丹陽郡,鄱陽世子
 嗣營小航南,並緣淮造柵。及曉,景方覺,乃登禪靈寺門樓望之,見韋粲營壘未合,先渡兵擊之。粲拒戰敗績,景斬粲首徇于城下。柳仲禮聞粲敗,不遑貫甲,與數十騎馳赴之,遇賊交戰,斬首數百,投水死者千餘人。仲禮深入,馬陷泥,亦被重創。自是賊不敢濟岸。



 邵陵王綸與臨成公大連等自東道集于南岸,荊州刺史湘東王繹遣世子方等、兼司馬吳曄、天門太守樊文皎下赴京師,營于湘子岸前,高州刺史李遷仕、前司州刺史羊鴉仁又率兵繼至。既而鄱陽世子嗣、永安侯確、羊鴉仁、李遷仕、樊文皎率眾渡淮,攻賊東府城前柵,破之,遂結營于青
 溪水東。景遣其儀同宋子仙頓南平王第,緣水西立柵相拒。景食稍盡,至是米斛數十萬人相食者十五六。



 初,援兵至北岸,百姓扶老攜幼以候王師,纔得過淮,便競剝掠,賊黨有欲自拔者,聞之咸止。賊之始至,城中纔得固守,平蕩之事,期望援軍。既而四方雲合,眾號百萬,連營相持,已月餘日,城中疾疫,死者太半。



 景自歲首以來乞和,朝廷未之許,至是事急乃聽焉。請割江右四州之地,并求宣城王大器出送,然後解圍濟江;仍許遣其儀同於子悅、左丞王偉入城為質。中領軍傅岐議以宣城王嫡嗣之重,不容許之。乃請石城公大款出送,詔許焉。
 遂於西華門外設壇,遣尚書僕射王克、兼侍中上甲鄉侯韶、兼散騎常侍蕭瑳與于子悅、王偉等,登壇共盟。左衛將軍柳津出西華門下,景出其柵門,與津遙相對,刑牲歃血。



 南兗州刺史南康嗣王會理、前青、冀二州刺史湘潭侯退、西昌侯世子彧率眾三萬,至于馬邛州。景慮北軍自白下而上,斷其江路,請悉勒聚南岸,敕乃遣北軍進江潭苑。景啟稱:「永安侯、趙威方頻隔柵見詬臣,云『天子自與汝盟,我終當逐汝』。乞召入城,即當進發。」敕並召之。景又啟云:「西岸信至,高澄已得壽春、鐘離,便無處安足。權借廣陵、譙州,須征得壽春、鐘離,即以奉還朝廷。」



 初,彭城劉邈說景曰:「大將軍頓兵已久,攻城不拔,今援眾雲集,未易而破;如聞軍糧不支一月,運漕路絕,野無所掠,嬰兒掌上,信在於今。未若乞和,全師而返,此計之上者。」景然其言,故請和。後知援軍號令不一,終無勤王之效;又聞城中死疾轉多,必當有應之者。景謀臣王偉又說曰:「王以人臣舉兵背叛,圍守宮闕,已盈十旬,逼辱妃主,凌穢宗廟,今日持此,何處容身?願王且觀其變。」景然之,乃抗表曰:臣聞「書不盡言,言不盡意」。然則意非言不宣,言非筆不盡,臣所以含憤蓄積,不能默已者也。竊惟陛下睿智在躬,多才多藝。昔因世季,龍翔漢、沔,夷凶
 剪亂,克雪家怨,然後踵武前王,光宅江表,憲章文、武,祖述堯、舜。兼屬魏國凌遲,外無勍敵,故能西取華陵,北封淮、泗,結好高氏,輶軒相屬,疆埸無虞,十有餘載。躬覽萬機,劬勞治道。刊正周、孔之遺文,訓釋真如之秘奧。享年長久,本枝盤石。人君藝業,莫之與京。臣所以踴躍一隅,望南風而歎息也,豈圖名與實爽,聞見不同?臣自委質策名,前後事跡,從來表奏,已具之矣。不勝憤懣,復為陛下陳之:陛下與高氏通和,歲踰一紀,舟車往復,相望道路,必將分災恤患,同休等戚;寧可納臣一介之服,貪臣汝、潁之地,便絕好河北,檄詈高澄,聘使未歸,陷之虎口,
 揚兵擊鼓,侵逼彭、宋。夫敵國相伐,聞喪則止,匹夫之交,託孤寄命。豈有萬乘之主,見利忘義若此者哉!其失一也。



 臣與高澄,既有仇憾,義不同國,歸身有道。陛下授以上將,任以專征,歌鐘女樂,車服弓矢。臣受命不辭,實思報效。方欲掛旆嵩、華,懸旌冀、趙,劉夷蕩滌,一匡宇內;陛下朝服濟江,告成東岳,使大梁與軒黃等盛,臣與伊、呂比功,垂裕後昆,流名竹帛,此實生平之志也。而陛下欲分其功,不能賜任,使臣擊河北,欲自舉徐方,遣庸懦之貞陽,任驕貪之胡、趙,裁見旗鼓,鳥散魚潰,慕容紹宗乘勝席卷,渦陽諸鎮靡不棄甲。疾雷不及掩耳,散地不可
 固全,使臣狼狽失據,妻子為戮,斯實陛下負臣之深。其失二也。



 韋黯之守壽陽,眾無一旅,慕容凶銳,欲飲馬長江,非臣退保淮南,其勢未之可測。既而逃遁,邊境獲寧,令臣作牧此州,以為蕃捍。方欲收合餘燼,勞來安集,勵兵秣馬,剋申後戰,封韓山之屍,雪渦陽之恥。陛下喪其精魄,無復守氣,便信貞陽謬啟,復請通和。臣頻陳執,疑閉不聽。翻覆若此,童子猶且羞之;況在人君,二三其德。其失三也。



 夫畏懦逗留,軍有常法。子玉小敗,見誅於楚;王恢失律,受戮于漢。貞陽精甲數萬,器械山積,慕容輕兵,眾無百乘,不能拒抗,身受囚執。以帝之猶子,而面縛
 敵庭,實宜絕其屬籍,以釁征鼓。陛下曾無追責,憐其茍存,欲以微臣,規相貿易。人君之法,當如是哉?其失四也。



 懸瓠大籓,古稱汝、潁。臣舉州內附,羊鴉仁固不肯入;既入之後,無故棄之,陛下曾無嫌責,使還居北司。鴉仁棄之,既不為罪,臣得之不以為功。其失五也。



 臣渦陽退衄,非戰之罪,實由陛下君臣相與見誤。乃還壽春,曾無悔色,祗奉朝廷,掩惡揚善。鴉仁自知棄州,切齒歎恨,內懷慚懼,遂啟臣欲反。欲反當有形迹,何所徵驗?誣陷頓爾,陛下曾無辯究,默而信納。豈有誣人莫大之罪,而可並肩事主者乎?其失六也。



 趙伯超拔自無能,任居方伯,惟
 漁獵百姓,多蓄士馬,非欲為國立功,直是自為富貴,行貨權幸,徼買聲名。朱異之徒,積受金貝,遂使咸稱胡、趙,比昔關、張,誣掩天聽,謂為真實。韓山之役,女妓自隨,裁聞敵鼓,與妾俱逝,不待貞陽,故隻輪莫返。論其此罪,應誅九族;而納賄中人,還處州任。伯超無罪,臣功何論?賞罰無章,何以為國?其失七也。



 臣御下素嚴,無所侵物,關市征稅,咸悉停原,壽陽之民,頗懷優復。裴之悌等助戍在彼,憚臣檢制,遂無故遁歸;又啟臣欲反。陛下不責違命離局,方受其浸潤之譖。處臣如此,使何地自安?其失八也。



 臣雖才謝古人,實頗更事,撫民率眾,自幼至長,少
 來運動,多無遺策。及歸身有道,罄竭忠規,每有陳奏,恆被抑遏。朱異專斷軍旅,周石珍總尸兵杖,陸驗、徐膋典司穀帛,皆明言求貨,非令不行。境外虛實,定計於舍人之省;舉將出師,責奏於主者之命。臣無賄於中,故恒被抑折。其失九也。



 鄱陽之鎮合肥,與臣鄰接。臣推以皇枝,每相祗敬;而嗣王庸怯,虛見備御,臣有使命,必加彈射,或聲言臣反,或啟臣纖介。招攜當須以禮,忠烈何以堪於此哉!其失十也。



 其餘條目,不可具陳。進退惟谷,頻有表疏。言直辭強,有忤龍鱗,遂發嚴詔,便見討襲。重華純孝,猶逃凶父之杖;趙盾忠賢,不討殺君之賊。臣何親何
 罪,而能坐受殲夷?韓信雄桀,亡項霸漢,末為女子所烹,方悔蒯通之說。臣每覽書傳,心常笑之。豈容遵彼覆車,而快陛下佞臣之手?是以興晉陽之甲,亂長江而直濟,願得升赤墀,踐文石,口陳枉直,指畫臧否,誅君側之惡臣,清國朝之粃政,然後還守籓翰,以保忠節,實臣之至願也。



 三月朔旦,城內以景違盟,舉烽鼓噪,於是羊鴉仁、柳敬禮、鄱陽世子嗣進軍於東府城北。柵壘未立,為景將宋子仙所襲,敗績,赴淮死者數千人。賊送首級於闕下。



 景又遣于子悅至,更請和。遣御史中丞沈浚至景所,景無去意,浚固責之。景大怒,即決石闕前水,百道攻城,
 晝夜不息,城遂陷。於是悉鹵掠乘輿服玩、後宮嬪妾,收王侯朝士送永福省,撤二宮侍衛。使王偉守武德殿,于子悅屯太極東堂,矯詔大赦天下,自為大都督、督中外諸軍事、錄尚書,其侍中、使持節、大丞相、王如故。初,城中積屍不暇埋瘞,又有已死而未斂,或將死而未絕,景悉聚而燒之,臭氣聞十餘里。尚書外兵郎鮑正疾篤,賊曳出焚之,宛轉火中,久而方絕。於是援兵並散。



 景矯詔曰:「日者,姦臣擅命,幾危社稷,賴丞相英發,入輔朕躬,征鎮牧守可各復本任。」降蕭正德為侍中、大司馬,百官皆復其職。景遣董紹先率兵襲廣陵,南兗州刺史南康嗣王
 會理以城降之。景以紹先為南兗州刺史。



 初,北兗州刺史定襄侯祗與湘潭侯退,及前潼州刺史郭鳳同起兵,將赴援。至是,鳳謀以淮陰應景,祗等力不能制,並奔于魏。景以蕭弄璋為北兗州刺史,州民發兵拒之,景遣廂公丘子英、直閣將軍羊海率眾赴援,海斬子英,率其軍降于魏,魏遂據其淮陰。景又遣儀同于子悅、張大黑率兵入吳,吳郡太守袁君正迎降。子悅等既至,破掠吳中,多自調發,逼掠子女,毒虐百姓,吳人莫不怨憤,於是各立城柵拒守。是月,景移屯西州,遣儀同任約為南道行臺,鎮姑孰。



 五月,高祖崩于文德殿。初,臺城既陷,景先遣
 王偉、陳慶入謁高祖,高祖曰:「景今安在?卿可召來。」時高祖坐文德殿,景乃入朝,以甲士五百人自衛,帶劍升殿。拜訖,高祖問曰:「卿在戎日久,無乃為勞?」景默然。又問:「卿何州人,而敢至此乎?」景又不能對,從者代對。及出,謂廂公王僧貴曰:「吾常據鞍對敵,矢刃交下,而意氣安緩,了無怖心。今日見蕭公,使人自懾,豈非天威難犯?吾不可再見之。」高祖雖外跡已屈,而意猶忿憤,時有事奏聞,多所譴卻。景深敬憚,亦不敢逼。景遣軍人直殿省內,高祖問制局監周石珍曰:「是何物人?」對曰:「丞相。」高祖乃謬曰:「何物丞相?」對曰:「是侯丞相。」高祖怒曰:「是名景,何謂丞相!」
 是後,每所徵求,多不稱旨,至於御膳亦被裁抑,遂憂憤感疾而崩。景乃密不發喪,權殯於昭陽殿,自外文武咸莫知之。二十餘日,升梓宮於太極前殿,迎皇太子即皇帝位。於是矯詔赦北人為奴婢者,冀收其力用焉。



 又遣儀同來亮率兵攻宣城,宣城內史楊華誘亮斬之;景復遣其將李賢明討華,華以郡降。景遣儀同宋子仙等率眾東次錢塘,新城戍主戴僧易據縣拒之。



 是月,景遣中軍侯子鑒入吳軍,收于子悅、張大黑,還京誅之。



 時東揚州刺史臨成公大連據州,吳興太守張嵊據郡,自南陵以上,皆各據守。景制命所行,惟吳郡以西、南陵以北而已。



 六月,景以儀同郭元建為尚書僕射、北道行臺、總江北諸軍事,鎮新秦。郡人陸緝、戴文舉等起兵萬餘人,殺景太守蘇單于,推前淮南太守文成侯寧為主,以拒景。宋子仙聞而擊之,緝等棄城走。景乃分吳郡海鹽、胥浦二縣為武原郡。至是,景殺蕭正德於永福省。封元羅為西秦王,元景龍為陳留王,諸元子弟封王者十餘人。以柳敬禮為使持節、大都督,隸大丞相,參戎事。



 景遣其中軍侯子鑒監行臺劉神茂等軍東討,破吳興,執太守張嵊父子送京師,景並殺之。景以宋子仙為司徒,任約為領軍將軍,爾朱季伯、叱羅子通、彭俊、董紹先、張化仁、于慶、
 魯伯和、紇奚斤、史安和、時靈護、劉歸義,並為開府儀同三司。



 是月,鄱陽嗣王範率兵次柵口,江州刺史尋陽王大心要之西上。景出頓姑孰,範將裴之悌、夏侯威生以眾降景。



 十一月,宋子仙攻錢塘,戴僧易降。景以錢塘為臨江郡,富陽為富春郡。又王偉、元羅並為儀同三司。



 十二月,宋子仙、趙伯超、劉神茂進攻會稽,東揚州刺史臨成公大連棄城走,遣劉神茂追擒之。景以裴之悌為使持節、平西將軍、合州刺史,以夏侯威生為使持節、平北將軍、南豫州刺史。



 是月,百濟使至,見城邑丘墟,於端門外號泣,行路見者莫不灑淚。景聞之大怒,送小莊嚴寺
 禁止,不聽出入。



 大寶元年正月,景矯詔自加班劍四十人,給前後部羽葆鼓吹,置左右長史、從事中郎四人。前江都令祖皓起兵於廣陵,斬景刺史董紹先,推前太子舍人蕭勔為刺史;又結魏人為援,馳檄遠近,將以討景。景聞之大懼,即日率侯子鑒等出自京口,水陸並集。皓嬰城拒守,景攻城,陷之。景車裂皓以徇,城中無少長皆斬之。以侯子鑒監南兗州事。



 是月,景召宋子仙還京口。



 四月,景以元思虔為東道行臺,鎮錢塘。以侯子鑒為南兗州刺史。



 文成侯寧於吳西鄉起兵,旬日之間,眾至一萬,率以西上。景廂公孟振、侯子榮擊破之,斬寧,傳首於
 景。



 七月,景以秦郡為西兗州,陽平郡為北兗州。任約、盧暉略攻晉熙郡,殺鄱陽世子嗣。



 景以王偉為中書監。



 任約進軍襲江州,江州刺史尋陽王大心降之。世祖時聞江州失守,遣衛軍將軍徐文盛率眾軍下武昌,拒約。



 景又矯詔自進位為相國,封泰山等二十郡為漢王,入朝不趨,贊拜不名,劍履上殿,如蕭何故事。景以柳敬禮為護軍將軍,姜詢義為相國左長史,徐洪為左司馬,陸約為右長史,沈眾為右司馬。



 是月,景率舟師上皖口。



 十月,盜殺武林侯諮於廣莫門。諮常出入太宗臥內,景黨不能平,故害之。



 景又矯詔曰:「蓋懸象在天,四時取則於辰
 頭;群生育地,萬物仰照於大明。是以垂拱當扆,則八枿共輳;負圖正位,則九域同歸。故乃雲名水號之君,龍官人爵之后,莫不啟符河、洛,封禪岱宗,奔走四夷,來朝萬國。逖聽虞、夏,厥道彌新。爰及商、周,未之或改。逮幽、厲不競,戎馬生郊;惠、懷失御,胡塵犯蹕。遂使豺狼肆毒,侵穴伊、瀍;獫狁孔熾,巢栖咸、洛。自晉鼎東遷,多歷年代,周原不復,歲實永久。雖宋祖經略,中息遠圖,齊號和親,空勞冠蓋。我大梁膺符作帝,出震登皇。浹珝歸仁,綿區飲化。開疆闢土,跨瀚海以揚鑣;來庭入覲,等塗山而比轍。玄龜出洛,白雉歸豊。鳥塞同文,胡天共軌。不謂高澄跋扈,
 虔劉魏邦,扇動華夷,不供王職,遂乃狼顧北侵,馬首南向。值天厭昏偽,醜徒數盡,龍豹應期,風雲會節。相國漢王,上德英姿,蓋惟天授;雄謨勇略,出自懷抱。珠魚表應,辰昴葉暉;剖析六韜,錙銖四履。騰文豹變,鳳集虯翔;奮翼來儀,負圖而降。爰初秉律,實先啟行;奉茲廟算,克除獯醜。直以鼎湖上征,六龍晏駕;干戈暫止,九伐未申。而惡稔貫盈,元凶殞斃;弟洋繼逆,續長亂階。異彼洋音,同茲薦食;偷竊偽號,心希舉斧。豊水君臣,奉圖乞援;關河百姓,泣血請師。咸願承奉國靈,思睹王化。朕以寡昧,纂戎下武,庶拯堯黎,冀康禹跡。且夫車服以庸,名因事著。
 周師克殷,鷹揚創自尚父;漢征戎狄,明友實始度遼。況乃神規睿算,眇乎難測,大功懋績,事絕言象,安可以習彼常名,保茲守固。相國可加宇宙大將軍、都督六合諸軍事,餘悉如故。」以詔文呈太宗,太宗驚曰:「將軍乃有宇宙之號乎!」



 齊遣其將辛術圍陽平,景行臺郭元建率兵赴援,術退。徐文盛入資磯,任約率水軍逆戰,文盛大破之,仍進軍大舉口。時景屯於皖口,京師虛弱,南康王會理及北兗州司馬成欽等將襲之。建安侯賁知其謀,以告景,景遣收會理與其弟祈陽侯通理、柳敬禮、成欽等,並害之。



 十二月,景矯詔封賁為竟陵王,賞發南康之謀
 也。



 是月,張彪起義於會稽,攻破上虞,景太守蔡臺樂討之,不能禁。至是,彪又破諸暨、永興等諸縣,景遣儀同田遷、趙伯超、謝答仁等東伐彪。



 二年正月,彪遣別將寇錢塘、富春,田遷進軍與戰,破之。



 景以王克為太師,宋子仙為太保,元羅為太傅,郭元建為太尉,張化仁為司徒,任約為司空,於慶為太子太師,時靈護為太子太保,紇奚斤為太子太傅,王偉為尚書左僕射,索超世為尚書右僕射。



 北兗州刺史蕭邕謀降魏,事泄,景誅之。



 是月,世祖遣巴州刺史王珣等率眾下武昌助徐文盛。任約以西臺益兵,告急於景。三月,景自率眾二萬,西上援約。四月,
 景次西陽,徐文盛率水軍邀戰,大破之。景訪知郢州無備,兵少,又遣宋子仙率輕騎三百襲陷之,執刺史方諸、行事鮑泉,盡獲武昌軍人家口。徐文盛等聞之,大潰,奔歸江陵,景乘勝西上。



 初,世祖遣領軍王僧辯率眾東下代徐文盛,軍次巴陵,會景至,僧辯因堅壁拒之。景設長圍,築土山,晝夜攻擊,不克。軍中疾疫,死傷太半。世祖遣平北將軍胡僧祐率兵二千人救巴陵,景聞,遣任約以精卒數千逆擊僧祐,僧祐與居士陸法和退據赤亭以待之,約至與戰,大破之,生擒約。景聞之,夜遁。以丁和為郢州刺史,留宋子仙、時靈護等助和守,以張化仁、閻洪
 慶守魯山城,景還京師。王僧辯乃率眾東下,次漢口,攻魯山及郢城,皆陷之。自是眾軍所至皆捷。



 景乃廢太宗,幽於永福省。作詔草成,逼太宗寫之,至「先皇念神器之重,思社稷之固」,歔欷嗚咽,不能自止。是日,景迎豫章王棟即皇帝位,升太極前殿,大赦天下,改元為天正元年。有回風自永福省吹其文物,皆倒折,見者莫不驚駭。



 初,景既平京邑,便有篡奪之志,以四方須定,且未自立;既巴陵失律,江、郢喪師,猛將外殲,雄心內沮,便欲偽僭大號,遂其姦心。其謀臣王偉云「自古移鼎,必須廢立」,故景從之。其太尉郭元建聞之,自秦郡馳還,諫景曰:「四方之
 師所以不至者,政為二宮萬福;若遂行弒逆,結怨海內,事幾一去,雖悔無及。」王偉固執不從。景乃矯棟詔,追尊昭明太子為昭明皇帝,豫章安王為安皇帝,金華敬妃為敬皇后,豫章國太妃王氏為皇太后,妃張氏為皇后;以劉神茂為司空,徐洪為平南將軍,秦晃之、王曄、李賢明、徐永、徐珍國、宋長寶、尹思合並為儀同三司。景以哀太子妃賜郭元建,元建曰:「豈有皇太子妃而降為人妾?」竟不與相見。



 十月壬寅夜,景遣其衛尉彭俊、王修纂奉酒於太宗曰:「丞相以陛下處憂既久,故令臣等奉進一觴。」太宗知其將弒,乃大酣飲酒,既醉還寢,修纂以帊盛
 土加於腹,因崩焉。斂用法服,以薄棺密瘞於城北酒庫。初,太宗久見幽縶,朝士莫得接覲,慮禍將及,常不自安;惟舍人殷不害後稍得入,太宗指所居殿謂之曰:「龐涓當死此下。」又曰:「吾昨夜夢吞土,卿試為思之。」不害曰:「昔重耳饋塊,卒反晉國。陛下所夢,將符是乎?」太宗曰:「儻幽冥有征,冀斯言不妄耳。」至是見弒,實以土焉。



 是月,景司空東道行臺劉神茂、儀同尹思合、劉歸義、王曄、雲麾將軍桑乾王元頵等據東陽歸順,仍遣元頵及別將李占、趙惠朗下據建德江口。尹思合收景新安太守元義,奪其兵。張彪攻永嘉,永嘉太守秦遠降彪。



 十一月,景以趙
 伯超為東道行臺,鎮錢塘,遣儀同田遷、謝答仁等將兵東征神茂。



 景矯蕭棟詔,自加九錫之禮,置丞相以下百官。陳備物於庭,忽有野鳥翔於景上,赤足丹觜,形似山鵲,賊徒悉駭,競射之不能中。景以劉勸、戚霸、朱安王為開府儀同三司,索九昇為護軍將軍。南兗州刺史侯子鑒獻白麞,建康獲白鼠以獻,蕭棟歸之于景。景以郭元建為南兗州刺史,太尉、北行臺如故。



 景又矯蕭棟詔,追崇其祖為大將軍,考為丞相。自加冕,十有二旒,建天子旌旗,出警入蹕,乘金根車,駕六馬,備五時副車,置旄頭雲罕,樂儛八佾,鐘虡宮懸之樂,一如舊儀。



 景又矯蕭棟
 詔,禪位於己。於是南郊,柴燎于天,升壇受禪文物,並依舊儀。以轜車床載鼓吹,橐駝負犧牲,輦上置筌蹄、垂腳坐。景所帶劍水精標無故墮落,手自拾之。將登壇,有兔自前而走,俄失所在;又白虹貫日。景還升太極前殿,大赦,改元為太始元年。封蕭棟為淮陰王,幽于監省。偽有司奏改「警蹕」為「永蹕」,避景名也。改梁律為漢律,改左民尚書為殿中尚書,五兵尚書為七兵尚書,直殿主帥為直寢。景三公之官動置十數,儀同尤多,或匹馬孤行,自執羈絆。其左僕射王偉請立七廟,景曰:「何謂為七廟?」偉曰:「天子祭七世祖考,故置七廟。」并請七世之諱,敕太常
 具祭祀之禮。景曰:「前世吾不復憶,惟阿爺名標。」眾聞咸竊笑之。景黨有知景祖名周者,自外悉是王偉制其名位,以漢司徒侯霸為始祖,晉徵士侯瑾為七世祖。於是追尊其祖周為大丞相,父標為元皇帝。



 十二月,謝答仁、李慶等至建德,攻元頵、李占柵,大破之,執頵、占送景。景截其手足徇之,經日乃死。



 景二年正月朔,臨軒朝會。景自巴丘挫衄,軍兵略盡,恐齊人乘釁與西師掎角,乃遣郭元建率步軍趣小峴,侯子鑒率舟師向濡須,曜兵肥水,以示武威。子鑒至合肥,攻羅城,剋之。郭元建、侯子鑒俄聞王師既近,燒合肥百姓邑居,引軍退,子鑒保姑孰,
 元建還廣陵。時謝答仁攻劉神茂,神茂別將王華、麗通並據外營降答仁。劉歸義、尹思合等懼,各棄城走。神茂孤危,復降答仁。



 王僧辯軍至蕪湖,蕪湖城主宵遁。景遣史安和、宋長貴等率兵二千,助子鑒守姑孰,追田遷等還京師。是月,景黨郭長獻馬駒生角。三月,景往姑孰,巡視壘柵,又誡子鑒曰:「西人善水戰,不可與爭鋒,往年任約敗績,良為此也。若得馬步一交,必當可破,汝但堅壁以觀其變。」子鑒乃捨舟登岸,閉營不出。僧辯等遂停軍十餘日,賊黨大喜,告景曰:「西師懼吾之強,必欲遁逸,不擊,將失之。」景復命子鑒為水戰之備。子鑒乃率步騎萬
 餘人渡洲,並引水軍俱進,僧辯逆擊,大破之,子鑒僅以身免。景聞子鑒敗,大懼涕下,覆面引衾以臥,良久方起,歎曰:「誤殺乃公!」



 僧辯進軍,次張公洲。景以盧暉略守石頭,紇奚斤守捍國城,悉逼百姓及軍士家累入臺城內。僧辯焚景水柵,入淮,至祥靈寺渚。景大驚,乃緣淮立柵,自石頭至朱雀航。僧辯及諸將遂於石頭城西步上連營立柵,至於落星墩。景大恐,自率侯子鑒、于慶、史安和、王僧貴等,於石頭東北立柵拒守。使王偉、索超世、呂季略守臺城,宋長貴守延祚寺。遣掘王僧辯父墓,剖棺焚屍。王僧辯等進營於石頭城北,景列陣挑戰。僧辯率眾
 軍奮擊,大破之,侯子鑒、史安和、王僧貴各棄柵走,盧暉略、紇奚斤並以城降。



 景既退敗,不入宮,斂其散兵,屯于闕下,遂將逃竄。王偉攬轡諫曰:「自古豈有叛天子!今宮中衛士,尚足一戰,寧可便走,棄此欲何所之?」景曰:「我在北打賀拔勝,破葛榮,揚名河、朔,與高王一種人。今來南渡大江,取臺城如反掌,打邵陵王於北山,破柳仲禮於南岸,皆乃所親見。今日之事,恐是天亡。乃好守城,我當復一決耳。」仰觀石闕,逡巡歎息。久之,乃以皮囊盛二子挂馬鞍,與其儀同田遷、范希榮等百餘騎東奔。王偉委臺城竄逸,侯子鑒等奔廣陵。



 王僧辯遣侯瑱率軍追景。
 景至晉陵,劫太守徐永東奔吳郡,進次嘉興,趙伯超據錢塘拒之。景退還吳郡,達松江,而侯瑱軍掩至,景眾未陣,皆舉幡乞降。景不能制,乃與腹心數十人單舸走,推墮二子於水,自滬瀆入海。至壺豆洲,前太子舍人羊鯤殺之,送屍于王僧辯,傳首西臺,曝屍於建康市。百姓爭取屠膾啖食,焚骨揚灰。曾罹其禍者,乃以灰和酒飲之。及景首至江陵,世祖命梟之於市,然後煮而漆之,付武庫。



 景長不滿七尺,而眉目疏秀。性猜忍,好殺戮。刑人或先斬手足,割舌劓鼻,經日方死。曾於石頭立大舂碓,有犯法者,皆搗殺之,其慘虐如此。自篡立後,時著白紗帽,
 而尚披青袍,或以牙梳插髻。床上常設胡床及筌蹄,著靴垂腳坐。或匹馬遊戲於宮內,及華林園彈射烏鳥。謀臣王偉不許輕出,於是鬱怏,更成失志。所居殿常有鵂鶹鳥鳴,景惡之,每使人窮山野討捕焉。普通中,童謠曰:「青絲白馬壽陽來。」後景果乘白馬,兵皆青衣。所乘馬,每戰將勝,輒躑躅嘶鳴,意氣駿逸,其奔衄,必低頭不前。



 初,中大同中,高祖嘗夜夢中原牧守皆以地來降,舉朝稱慶,寤甚悅之。旦見中書舍人朱異說所夢,異曰:「此豈宇內方一,天道前見其徵乎?」高祖曰:「吾為人少夢,昨夜感此,良足慰懷。」及太清二年,景果歸附,高祖欣然自悅,謂
 與神通,乃議納之,而意猶未決。曾夜出視事,至武德閣,獨言:「我家國猶若金甌,無一傷缺,今便受地,詎是事宜,脫致紛紜,非可悔也。」朱異接聲而對曰:「聖明御宇,上應蒼玄,北土遺黎,誰不慕仰?為無機會,未達其心。今侯景據河南十餘州,分魏土之半,輸誠送款,遠歸聖朝,豈非天誘其衷,人獎其計?原心審事,殊有可嘉。今若拒而不容,恐絕後來之望,此誠易見,願陛下無疑。」高祖深納異言,又信前夢,乃定議納景。及貞陽覆敗,邊鎮恇擾,高祖固已憂之,曰:「吾今段如此,勿作晉家事乎?」



 先是,丹陽陶弘景隱於華陽山,博學多識,嘗為詩曰:「夷甫任散誕,平
 叔坐談空。不意昭陽殿,化作單于宮。」大同末,人士競談玄理,不習武事;至是,景果居昭陽殿。天監中,有釋寶誌曰:「掘尾狗子自發狂,當死未死嚙人傷,須臾之間自滅亡,起自汝陰死三湘。」又曰:「山家小兒果攘臂,太極殿前作虎視。」掘尾狗子、山家小兒,皆猴狀。景遂覆陷都邑,毒害皇室。大同中,太醫令朱耽嘗直禁省,無何,夜夢犬羊各一在御坐,覺而惡之,告人曰:「犬羊者,非佳物也。今據御坐,將有變乎?」既而天子蒙塵,景登正殿焉。



 及景將敗,有僧通道人者,意性若狂,飲酒啖肉,不異凡等,世間遊行已數十載,姓名鄉里,人莫能知。初言隱伏,久乃方驗,人
 並呼為闍梨,景甚信敬之。景嘗於後堂與其徒共射,時僧通在坐,奪景弓射景陽山,大呼云「得奴已」。景後又宴集其黨,又召僧通。僧通取肉揾鹽以進景,問曰:「好不?」景答:「所恨太鹹。」僧通曰:「不鹹則爛臭。」果以鹽封其屍。



 王偉,陳留人。少有才學,景之表、啟、書、檄,皆其所製。景既得志,規摹篡奪,皆偉之謀。及囚送江陵,烹於市,百姓有遭其毒者,並割炙食之。



 史臣曰:夫道不恆夷,運無常泰,斯則窮通有數,盛衰相襲,時屯陽九,蓋在茲焉。若乃侯景小豎,叛換本國,識不周身,勇非出類,而王偉為其謀主,成此姦慝。驅率醜徒,
 陵江直濟,長戟強弩,淪覆宮闕,禍纏宸極,毒遍黎元,肆其恣睢之心,成其篡盜之禍。嗚呼!國之將亡,必降妖孽。雖曰人事,抑乃天時。昔夷羿亂夏,犬戎厄周,漢則莽、卓流災,晉則敦、玄構禍,方之羯賊,有逾其酷,悲夫!



\end{pinyinscope}