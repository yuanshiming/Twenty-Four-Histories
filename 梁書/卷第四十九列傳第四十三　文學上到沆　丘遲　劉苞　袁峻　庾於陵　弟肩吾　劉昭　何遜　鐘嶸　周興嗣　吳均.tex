\article{卷第四十九列傳第四十三 文學上到沆 丘遲 劉苞 袁峻 庾於陵 弟肩吾 劉昭 何遜 鐘嶸 周興嗣 吳均}

\begin{pinyinscope}

 昔司馬遷、班固書,並為《司馬相如傳》,相如不預漢廷大事,蓋取其文章尤著也。固又為《賈鄒枚路傳》,亦取其能文傳焉。范氏《後漢書》有《文苑傳》,所載之人,其詳已甚。然
 經禮樂而緯國家,通古今而述美惡,非文莫可也。是以君臨天下者,莫不敦悅其義,縉紳之學,咸貴尚其道,古往今來,未之能易。高祖聰明文思,光宅區宇,旁求儒雅,詔採異人,文章之盛,煥乎俱集。每所御幸,輒命群臣賦詩,其文善者,賜以金帛,詣闕庭而獻賦頌者,或引見焉。其在位者,則沈約、江淹、任昉,並以文採妙絕當時。至若彭城到沆、吳興丘遲、東海王僧孺、吳郡張率等,或入直文德,通宴壽光,皆後來之選也。約、淹、昉、僧孺,率別以功跡論。今綴到沆等文兼學者,至太清中人,為《文學傳》云。



 到沆,字茂瀣,彭城武原人也。曾祖彥之,宋將軍。父捴,齊
 五兵尚書。沆幼聰敏,五歲時,捴於屏風抄古詩,沆請教讀一遍,便能諷誦,無所遺失。既長勤學,善屬文,工篆隸。美風神,容止可悅。齊建武中,起家後軍法曹參軍。天監初,遷征虜主簿。高祖初臨天下,收拔賢俊,甚愛其才。東宮建,以為太子洗馬。時文德殿置學士省,召高才碩學者待詔其中,使校定墳史,詔沆通籍焉。時高祖宴華光殿,命群臣賦詩,獨詔沆為二百字,二刻使成。沆於坐立奏,其文甚美。俄以洗馬管東宮書記、散騎省優策文。三年,詔尚書郎在職清能或人才高妙者為侍郎,以沆為殿中曹侍郎。沆從父兄溉、洽,並有才名,時皆相代為殿
 中,當世榮之。四年,遷太子中舍人。沆為人不自伐,不論人長短,樂安任昉、南鄉范雲皆與友善。其年,遷丹陽尹丞,以疾不能處職事,遷北中郎諮議參軍。五年,卒官,年三十。高祖甚傷惜焉,詔賜錢二萬,布三十匹。所著詩賦百餘篇。



 丘遲,字希範,吳興烏程人也。父靈鞠,有才名,仕齊官至太中大夫。遲八歲便屬文,靈鞠常謂「氣骨似我」。黃門郎謝超宗、徵士何點並見而異之。及長,州辟從事,舉秀才,除太學博士。遷大司馬行參軍,遭父憂去職。服闋,除西中郎參軍。累遷殿中郎,以母憂去職。服除,復為殿中郎,
 遷車騎錄事參軍。高祖平京邑,霸府開,引為驃騎主簿,甚被禮遇。時勸進梁王及殊禮,皆遲文也。高祖踐阼,拜散騎侍郎,俄遷中書侍郎,領吳興邑中正,待詔文德殿。時高祖著《連珠》,詔群臣繼作者數十人,遲文最美。天監三年,出為永嘉太守,在郡不稱職,為有司所糾,高祖愛其才,寢其奏。四年,中軍將軍臨川王宏北伐,遲為諮議參軍,領記室。時陳伯之在北,與魏軍來距,遲以書喻之,伯之遂降。還拜中書郎,遷司徒從事中郎。七年,卒官,時年四十五。所著詩賦行於世。



 劉苞,字孝嘗,彭城人也。祖勔,宋司空。父愃,齊太子中庶
 子。苞四歲而父終,及年六七歲,見諸父常泣。時伯、叔父悛、繪等並顯貴,苞母謂其畏憚,怒之。苞對曰:「早孤不及有識,聞諸父多相似,故心中欲悲,無有佗意。」因而歔欷,母亦慟甚。初,苞父母及兩兄相繼亡沒,悉假瘞焉。苞年十六,始移墓所,經營改葬,不資諸父,未幾皆畢,繪常歎服之。



 少好學,能屬文。起家為司徒法曹行參軍,不就。天監初,以臨川王妃弟故,自征虜主簿仍遷王中軍功曹,累遷尚書庫部侍郎、丹陽尹丞、太子太傅丞、尚書殿中侍郎、南徐州治中,以公事免。久之,為太子洗馬,掌書記,侍講壽光殿。自高祖即位,引後進文學之士,苞及從兄
 孝綽、從弟孺、同郡到溉、溉弟洽、從弟沆、吳郡陸倕、張率並以文藻見知,多預宴坐,雖仕進有前後,其賞賜不殊。天監十年,卒,時年三十。臨終,呼友人南陽劉之遴託以喪事,務從儉率。苞居官有能名,性和而直,與人交,面折其非,退稱其美,情無所隱,士友咸以此歎惜之。



 袁峻,字孝高,陳郡陽夏人,魏郎中令渙之八世孫也。峻早孤,篤志好學,家貧無書,每從人假借,必皆抄寫,自課日五十紙,紙數不登,則不休息。訥言語,工文辭。義師剋京邑,鄱陽王恢東鎮破岡,峻隨王知管記事。天監初,鄱陽國建,以峻為侍郎,從鎮京口。王遷郢州,兼都曹參軍。
 高祖雅好辭賦,時獻文於南闕者相望焉,其藻麗可觀,或見賞擢。六年,峻乃擬揚雄《官箴》奏之。高祖嘉焉,賜束帛。除員外散騎侍郎,直文德學士省,抄《史記》、《漢書》各為二十卷。又奉敕與陸倕各製《新闕銘》,辭多不載。



 庾於陵,字子介,散騎常侍黔婁之弟也。七歲能言玄理。既長,清警博學有才思。齊隨王子隆為荊州,召為主簿,使與謝朓、宗夬抄撰群書。子隆代還,又以為送故主簿。子隆尋為明帝所害,僚吏畏避,莫有至者,唯於陵與夬獨留,經理喪事。始安王遙光為撫軍,引為行參軍,兼記室。永元末,除東陽遂安令,為民吏所稱。天監初,為建康
 獄平,遷尚書工部郎,待詔文德殿。出為湘州別駕,遷驃騎錄事參軍,兼中書通事舍人。俄領南郡邑中正,拜太子洗馬,舍人如故。舊事,東宮官屬,通為清選,洗馬掌文翰,尤其清者。近世用人,皆取甲族有才望,時於陵與周舍並擢充職,高祖曰:「官以人而清,豈限以甲族。」時論以為美。俄遷散騎侍郎,改領荊州大中正。累遷中書黃門侍郎,舍人、中正並如故。出為宣毅晉安王長史、廣陵太守,行府州事,以公事免。復起為通直郎,尋除鴻臚卿,復領荊州大中正。卒官,時年四十八。文集十卷。弟肩吾。



 肩吾,字子慎。八歲能賦詩,特為兄於陵所友愛。初為晉
 安王國常侍,仍遷王宣惠府行參軍。自是每王徙鎮,肩吾常隨府。歷王府中郎、雲麾參軍,並兼記室參軍。中大通三年,王為皇太子,兼東宮通事舍人,除安西湘東王錄事參軍,俄以本官領荊州大中正。累遷中錄事諮議參軍、太子率更令、中庶子。初,太宗在籓,雅好文章士,時肩吾與東海徐摛、吳郡陸杲、彭城劉遵、劉孝儀、儀弟孝威,同被賞接。及居東宮,又開文德省,置學士,肩吾子信、摛子陵、吳郡張長公、北地傅弘、東海鮑至等充其選。齊永明中,文士王融、謝朓、沈約文章始用四聲,以為新變,至是轉拘聲韻,彌尚麗靡,復踰於往時。時太子與湘東
 王書論之曰:吾輩亦無所遊賞,止事披閱,性既好文,時復短詠。雖是庸音,不能閣筆,有慚伎癢,更同故態。比見京師文體,懦鈍殊常,競學浮疏,急為闡緩。玄冬修夜,思所不得,既殊比興,正背《風》、《騷》。若夫六典三禮,所施則有地;吉凶嘉賓,用之則有所。未聞吟詠情性,反擬《內則》之篇;操筆寫志,更摹《酒誥》之作;遲遲春日,翻學《歸藏》;湛湛江水,遂同《大傳》。



 吾既拙於為文,不敢輕有掎摭。但以當世之作,歷方古之才人,遠則揚、馬、曹、王,近則潘、陸、顏、謝,而觀其遣辭用心,了不相似。若以今文為是,則古文為非;若昔賢可稱,則今體宜棄。俱為盍各,則未之敢許。又
 時有效謝康樂、裴鴻臚文者,亦頗有惑焉。何者?謝客吐言天拔,出於自然,時有不拘,是其糟粕;裴氏乃是良史之才,了無篇什之美。是為學謝則不屆其精華,但得其冗長;師裴則蔑絕其所長,惟得其所短。謝故巧不可階,裴亦質不宜慕。故胸馳臆斷之侶,好名忘實之類,方分肉於仁獸,逞郤克於邯鄲,入鮑忘臭,效尤致禍。決羽謝生,豈三千之可及;伏膺裴氏,懼兩唐之不傳。故玉徽金銑,反為拙目所嗤;《巴人下里》,更合郢中之聽。《陽春》高而不和,妙聲絕而不尋。竟不精討錙銖,核量文質,有異《巧心》,終愧妍手。是以握瑜懷玉之士,瞻鄭邦而知退;章甫
 翠履之人,望閩鄉而歎息。詩既若此,筆又如之。徒以煙墨不言,受其驅染;紙札無情,任其搖襞。甚矣哉,文之橫流,一至於此!



 至如近世謝朓、沈約之詩,任昉、陸倕之筆,斯實文章之冠冕,述作之楷模。張士簡之賦,周升逸之辯,亦成佳手,難可復遇。文章未墜,必有英絕;領袖之者,非弟而誰。每欲論之,無可與語,思言子建,一共商榷。辯茲清濁,使如涇、渭;論茲月旦,類彼汝南。朱丹既定,雌黃有別,使夫懷鼠知慚,濫竽自恥。譬斯袁紹,畏見子將;同彼盜牛,遙羞王烈。相思不見,我勞如何。



 太清中,侯景寇陷京都;及太宗即位,以肩吾為度支尚書。時上流諸蕃,
 並據州拒景,景矯詔遣肩吾使江州,喻當陽公大心,大心尋舉州降賊。肩吾因逃入建昌界,久之,方得赴江陵,未幾卒。文集行於世。



 劉昭,字宣卿,平原高唐人,晉太尉實九世孫也。祖伯龍,居父憂以孝聞,宋武帝敕皇太子諸王並往弔慰,官至少府卿。父彪,齊征虜晉安王記室。昭幼清警,七歲通《老》、《莊》義。既長,勤學善屬文,外兄江淹早相稱賞。天監初,起家奉朝請,累遷征北行參軍、尚書倉部郎,尋除無錫令。歷為宣惠豫章王、中軍臨川王記室。初,昭伯父肜集眾家《晉書》注干寶《晉紀》為四十卷,至昭又集《後漢》同異以注
 范曄書,世稱博悉。遷通直郎,出為剡令,卒官。《集注後漢》一百八十卷,《幼童傳》十卷,文集十卷。



 子縚,字言明。亦好學,通《三禮》。大同中,為尚書祠部郎,尋去職,不復仕。縚弟緩,字含度,少知名。歷官安西湘東王記室,時西府盛集文學,緩居其首。除通直郎,俄遷鎮南湘東王中錄事,復隨府江州,卒。



 何遜,字仲言,東海郯人也。曾祖承天,宋御史中丞。祖翼,員外郎。父詢,齊太尉中兵參軍。遜八歲能賦詩,弱冠,州舉秀才。南鄉范雲見其對策,大相稱賞,因結忘年交好。自是一文一詠,雲輒嗟賞,謂所親曰:「頃觀文人,質則過
 儒,麗則傷俗;其能含清濁,中今古,見之何生矣。」沈約亦愛其文,嘗謂遜曰:「吾每讀卿詩,一日三復,猶不能已。」其為名流所稱如此。



 天監中,起家奉朝請,遷中衛建安王水曹行參軍,兼記室。王愛文學之士,日與遊宴,及遷江州,遜猶掌書記。還為安西安成王參軍事,兼尚書水部郎,母憂去職。服闋,除仁威廬陵王記室,復隨府江州,未幾卒。東海王僧孺集其文為八卷。初,遜文章與劉孝綽並見重於世,世謂之「何劉」。世祖著論論之云:「詩多而能者沈約,少而能者謝朓、何遜。」



 時有會稽虞騫,工為五言詩,名與遜相埒,官至王國侍郎。其後又有會稽孔翁歸、
 濟陽江避,並為南平王大司馬府記室。翁歸亦工為詩,避博學有思理,更注《論語》、《孝經》。二人並有文集。



 鐘嶸,字仲偉,潁川長社人,晉侍中雅七世孫也。父蹈,齊中軍參軍。嶸與兄岏、弟嶼並好學,有思理。嶸,齊永明中為國子生,明《周易》,衛軍王儉領祭酒,頗賞接之。舉本州秀才。起家王國侍郎,遷撫軍行參軍,出為安國令。永元末,除司徒行參軍。天監初,制度雖革,而日不暇給,嶸乃言曰:「永元肇亂,坐弄天爵,勳非即戎,官以賄就。揮一金而取九列,寄片札以招六校;騎都塞市,郎將填街。服既纓組,尚為臧獲之事;職唯黃散,猶躬胥徒之役。名實淆
 紊,茲焉莫甚。臣愚謂軍官是素族士人,自有清貫,而因斯受爵,一宜削除,以懲僥競。若吏姓寒人,聽極其門品,不當因軍,遂濫清級。若僑雜傖楚,應在綏附,正宜嚴斷祿力,絕其妨正,直乞虛號而已。謹竭愚忠,不恤眾口。」敕付尚書行之。遷中軍臨川王行參軍。衡陽王元簡出守會稽,引為寧朔記室,專掌文翰。時居士何胤築室若邪山,山發洪水,漂拔樹石,此室獨存。元簡命嶸作《瑞室頌》以旌表之,辭甚典麗,選西中郎晉安王記室。



 嶸嘗品古今五言詩,論其優劣,名為《詩評》。其序曰:氣之動物,物之感人,故搖蕩性情,形諸舞詠。欲以照燭三才,輝麗萬有,
 靈祇待之以致饗,幽微藉之以昭告。動天地,感鬼神,莫近於詩。昔《南風》之辭,《卿雲》之頌,厥義夐矣。《夏歌》曰「鬱陶乎予心」,楚謠云「名餘曰正則」,雖詩體未全,然略是五言之濫觴也。逮漢李陵,始著五言之目。古詩眇邈,人代難詳,推其文體,固是炎漢之制,非衰周之倡也。自王、揚、枚、馬之徒,辭賦競爽,而吟詠靡聞。從李都尉訖班婕妤,將百年間,有婦人焉,一人而已。詩人之風,頓已缺喪。東京二百載中,唯有班固《詠史》,質木無文致。降及建安,曹公父子,篤好斯文;平原兄弟,鬱為文棟;劉楨、王粲,為其羽冀。次有攀龍託鳳,自致於屬車者,蓋將百計。彬彬之盛,
 大備於時矣!爾後陵遲衰微,訖於有晉。太康中,三張二陸,兩潘一左,勃爾復興,踵武前王,風流未沫,亦文章之中興也。永嘉時,貴黃、老,尚虛談,于時篇什,理過其辭,淡乎寡味。爰及江表,微波尚傳,孫綽、許詢、桓、庾諸公,皆平典似《道德論》,建安之風盡矣。先是郭景純用俊上之才,創變其體;劉越石仗清剛之氣,贊成厥美。然彼眾我寡,未能動俗。逮義熙中,謝益壽斐然繼作;元嘉初,有謝靈運,才高辭盛,富艷難蹤,固已含跨劉、郭,陵轢潘、左。故知陳思為建安之傑,公幹、仲宣為輔;陸機為太康之英,安仁、景陽為輔;謝客為元嘉之雄,顏延年為輔:此皆五言
 之冠冕,文辭之命世。



 夫四言文約意廣,取效《風》、《騷》,便可多得,每苦文煩而意少,故世罕習焉。五言居文辭之要,是眾作之有滋味者也,故云會於流俗。豈不以指事遣形,窮情寫物,最為詳切者邪!故《詩》有六義焉,一曰興,二曰賦,三曰比。文已盡而意有餘,興也;因物喻志,比也;直書其事,寓言寫物,賦也。弘斯三義,酌而用之,乾之以風力,潤之以丹采,使味之者無極,聞之者動心,是詩之至也。若專用比、興,則患在意深,意深則辭躓。若但用賦體,則患在意浮,意浮則文散。嬉成流移,文無止泊,有蕪漫之累矣。



 若乃春風春鳥,秋月秋蟬,夏雲暑雨,冬月祁寒,
 斯四候之感諸詩者也。嘉會寄詩以親,離群託詩以怨。至於楚臣去境,漢妾辭宮;或骨橫朔野,或魂逐飛蓬;或負戈外戍,或殺氣雄邊;塞客衣單,霜閨淚盡。又士有解佩出朝,一去忘反;女有揚蛾入寵,再盼傾國。凡斯種種,感蕩心靈,非陳詩何以展其義,非長歌何以釋其情?故曰:「《詩》可以群,可以怨。」使窮賤易安,幽居靡悶,莫尚於詩矣。故辭人作者,罔不愛好。今之士俗,斯風熾矣。裁能勝衣,甫就小學,必甘心而馳騖焉。於是庸音雜體,各為家法。至於膏腴子弟,恥文不逮,終朝點綴,分夜呻吟,獨觀謂為警策,眾視終淪平鈍。次有輕蕩之徒,笑曹、劉為古
 拙,謂鮑昭羲皇上人,謝朓今古獨步;而師鮑昭終不及「日中市朝滿」,學謝朓劣得「黃鳥度青枝」。徒自棄於高聽,無涉於文流矣。



 嶸觀王公搢紳之士,每博論之餘,何嘗不以詩為口實,隨其嗜欲,商榷不同。淄澠並汎,朱紫相奪,喧嘩競起,准的無依。近彭城劉士章,俊賞之士,疾其淆亂,欲為當世詩品,口陳標榜,其文未遂,嶸感而作焉。昔九品論人,《七略》裁士,校以賓實,誠多未值;至若詩之為技,較爾可知,以類推之,殆同博弈。方今皇帝資生知之上才,體沈鬱之幽思,文麗日月,學究天人,昔在貴遊,已為稱首;況八枿既掩,風靡雲蒸,抱玉者連肩,握珠者
 踵武。固以睨漢、魏而弗顧,吞晉、宋於胸中。諒非農歌轅議,敢致流別。嶸之今錄,庶周遊於閭里,均之於談笑耳。



 頃之,卒官。



 岏,字長岳,官至府參軍、建康平。著《良吏傳》十卷。嶼,字季望,永嘉郡丞。天監十五年,敕學士撰《遍略》,嶼亦預焉。兄弟並有文集。



 周興嗣,字思纂,陳郡項人,漢太子太傅堪後也。高祖凝,晉征西府參軍、宜都太守。興嗣世居姑孰。年十三,遊學京師,積十餘載,遂博通記傳,善屬文。嘗步自姑孰,投宿逆旅,夜有人謂之曰:「子才學邁世,初當見識貴臣,卒被知英主。」言終,不測所之。齊隆昌中,侍中謝朏為吳興太
 守,唯與興嗣談文史而已。及罷郡還,因大相稱薦。本州舉秀才,除桂陽郡丞,太守王嶸素相賞好,禮之甚厚。高祖革命,興嗣奏《休平賦》,其文甚美,高祖嘉之。拜安成王國侍郎,直華林省。其年,河南獻儛馬,詔興嗣與待詔到沆、張率為賦,高祖以興嗣為工。擢員外散騎侍郎,進直文德、壽光省。是時,高祖以三橋舊宅為光宅寺,敕興嗣與陸倕各製寺碑。及成俱奏,高祖用興嗣所製者。自是《銅表銘》、《柵塘碣》、《北伐檄》、《次韻王羲之書千字》,並使興嗣為文;每奏,高祖輒稱善,加賜金帛。九年,除新安郡丞,秩滿,復為員外散騎侍郎,佐撰國史。十二年,遷給事中,撰
 文如故。興嗣兩手先患風疽,是年又染癘疾,左目盲,高祖撫其手,嗟曰:「斯人也而有斯疾也!」手疏治疽方以賜之。其見惜如此。任昉又愛其才,常言曰:「周興嗣若無疾,旬日當至御史中丞。」十四年,除臨川郡丞。十七年,復為給事中,直西省。左衛率周捨奉敕注高祖所製歷代賦,啟興嗣助焉。普通二年,卒。所撰《皇帝實錄》、《皇德記》、《起居注》、《職儀》等百餘卷,文集十卷。



 吳均,字叔庠,吳興故鄣人也。家世寒賤,至均好學有俊才。沈約嘗見均文,頗相稱賞。天監初,柳惲為吳興,召補主簿,日引與賦詩。均文體清拔有古氣,好事者或斅之,
 謂為「吳均體」。建安王偉為揚州,引兼記室,掌文翰。王遷江州,補國侍郎,兼府城局。還除奉朝請。先是,均表求撰《齊春秋》。書成奏之,高祖以其書不實,使中書舍人劉之遴詰問數條,竟支離無對,敕付省焚之,坐免職。尋有敕召見,使撰《通史》,起三皇,訖齊代,均草本紀、世家功已畢,唯列傳未就。普通元年,卒,時年五十二。均注范曄《後漢書》九十卷,著《齊春秋》三十卷、《廟記》十卷、《十二州記》十六卷、《錢唐先賢傳》五卷、《續文釋》五卷,文集二十卷。



 先是,有廣陵高爽、濟陽江洪、會稽虞騫,並工屬文。爽,齊永明中贈衛軍王儉詩,為儉所賞,及領丹陽尹,舉爽郡孝廉。天
 監初,歷官中軍臨川王參軍。出為晉陵令,坐事系冶,作《鑊魚賦》以自況,其文甚工。後遇赦獲免,頃之,卒。洪為建陽令,坐事死。騫官至王國侍郎。並有文集。



\end{pinyinscope}