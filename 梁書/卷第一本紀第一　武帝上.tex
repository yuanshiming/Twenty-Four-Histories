\article{卷第一本紀第一 武帝上}

\begin{pinyinscope}

 高祖武皇帝,諱衍,字叔達,小字練兒,南蘭陵中都里人,漢相國何之後也。何生酂定侯延,延生侍中彪,彪生公府掾章,章生皓,皓生仰,仰生太子太傅望之,望之生光祿大夫育,育生御史中丞紹,紹生光祿勳閎,閎生濟陰太守闡,闡生吳郡太守冰,冰生中山相苞,苞生博士周,周生蛇丘長矯,矯生州從事逵,逵生孝廉休,休生廣陵郡丞
 豹,豹生太中大夫裔,裔生淮陰令整,整生濟陰太守轄,轄生州治中副子,副子生南臺治書道賜,道賜生皇考諱順之,齊高帝族弟也。參預佐命,封臨湘縣侯。歷官侍中,衛尉,太子詹事,領軍將軍,丹陽尹,贈鎮北將軍。高祖以宋孝武大明八年甲辰歲生於秣陵縣同夏里三橋宅。生而有奇異,兩胯駢骨,頂上隆起,有文在右手曰「武」。帝及長,博學多通,好籌略,有文武才幹,時流名輩咸推許焉。所居室常若雲氣,人或過者,體輒肅然。



 起家巴陵王南中郎法曹行參軍,遷衛將軍王儉東閣祭酒。儉一見,深相器異,謂廬江何憲曰:「此蕭郎三十內當作侍中,
 出此則貴不可言。」竟陵王子良開西邸,招文學,高祖與沈約、謝朓、王融、蕭琛、范雲、任昉、陸倕等並遊焉,號曰八友。融俊爽,識鑒過人,尤敬異高祖,每謂所親曰:「宰制天下,必在此人。」累遷隋王鎮西咨議參軍,尋以皇考艱去職。



 隆昌初,明帝輔政,起高祖為寧朔將軍,鎮壽春。服闋,除太子庶子、給事黃門侍郎,入直殿省。預蕭諶等定策勛,封建陽縣男,邑三百戶。建武二年,魏遣將劉昶、王肅帥眾寇司州,以高祖為冠軍將軍、軍主,隸江州刺史王廣為援。距義陽百餘里,眾以魏軍盛,趑趄莫敢前。高祖請為先啟,廣即分麾下精兵配高祖。爾夜便進,去魏軍
 數里,逕上賢首山。魏軍不測多少,未敢逼。黎明,城內見援至,因出軍攻魏柵。高祖帥所領自外進戰。魏軍表裏受敵,乃棄重圍退走。軍罷,以高祖為右軍晉安王司馬、淮陵太守。還為太子中庶子,領羽林監。頃之,出鎮石頭。



 四年,魏帝自率大眾寇雍州,明帝令高祖赴援。十月,至襄陽。詔又遣左民尚書崔慧景總督諸軍,高祖及雍州刺史曹虎等並受節度。明年三月,慧景與高祖進行鄧城,魏主帥十萬餘騎奄至。慧景失色,欲引退,高祖固止之,不從,乃狼狽自拔。魏騎乘之,於是大敗。高祖獨帥眾距戰,殺數十百人,魏騎稍卻,因得結陣斷後,至夕得下
 船。慧景軍死傷略盡,惟高祖全師而歸。俄以高祖行雍州府事。



 七月,仍授持節、都督雍梁南北秦四州郢州之竟陵司州之隨郡諸軍事、輔國將軍、雍州刺史。其月,明帝崩,東昏即位,揚州刺史始安王遙光、尚書令徐孝嗣、尚書右僕射江祏、右將軍蕭坦之、侍中江祀、衛尉劉暄更直內省,分日帖敕。高祖聞之,謂從舅張弘策曰:「政出多門,亂其階矣。《詩》云:『一國三公,吾誰適從?』況今有六,而可得乎!嫌隙若成,方相誅滅,當今避禍,惟有此地。勤行仁義,可坐作西伯。但諸弟在都,恐罹世患,須與益州圖之耳。」



 時高祖長兄懿罷益州還,仍行郢州事,乃使弘策
 詣郢,陳計於懿曰:「昔晉惠庸主,諸王爭權,遂內難九興,外寇三作。今六貴爭權,人握王憲,制主畫敕,各欲專威,睚眥成憾,理相屠滅。且嗣主在東宮本無令譽,媟近左右,蜂目忍人,一總萬機,恣其所欲,豈肯虛坐主諾,委政朝臣。積相嫌貳,必大誅戮。始安欲為趙倫,形迹已見,蹇人上天,信無此理。且性甚猜狹,徒取亂機。所可當軸,惟有江、劉而已。祏怯而無斷,暄弱而不才,折鼎覆餗,翹足可待。蕭坦之胸懷猜忌,動言相傷,徐孝嗣才非柱石,聽人穿鼻,若隙開釁起,必中外土崩。今得守外籓,幸圖身計,智者見機,不俟終日。及今猜防未生,宜召諸弟以時
 聚集。後相防疑,拔足無路。郢州控帶荊、湘,西注漢、沔;雍州士馬,呼吸數萬,虎視其間,以觀天下。世治則竭誠本朝,時亂則為國剪暴,可得與時進退,此蓋萬全之策。如不早圖,悔無及也。」懿聞之變色,心弗之許。弘策還,高祖乃啟迎弟偉及憺。是歲至襄陽。於是潛造器械,多伐竹木,沉於檀溪,密為舟裝之備。時所住齋常有五色回轉,狀若蟠龍,其上紫氣騰起,形如傘蓋,望者莫不異焉。



 永元二年冬,懿被害。信至,高祖密召長史王茂、中兵呂僧珍、別駕柳慶遠、功曹史吉士瞻等謀之。既定,以十一月乙巳召僚佐集於廳事,謂曰:「昔武王會孟津,皆曰『紂可
 伐』。今昏主惡稔,窮虐極暴,誅戮朝賢,罕有遺育,生民塗炭,天命殛之。卿等同心疾惡,共興義舉,公侯將相,良在茲日,各盡勛效,我不食言。」是日建牙。於是收集得甲士萬餘人,馬千餘匹,船三千艘,出檀溪竹木裝艦。



 先是,東昏以劉山陽為巴西太守,配精兵三千,使過荊州就行事蕭穎胄以襲襄陽。高祖知其謀,乃遣參軍王天虎、龐慶國詣江陵,遍與州府書。及山陽西上,高祖謂諸將曰:「荊州本畏襄陽人,加脣亡齒寒,自有傷弦之急,寧不闇同邪?我若總荊、雍之兵,掃定東夏,韓、白重出,不能為計。況以無算之昏主,役御刀應敕之徒哉?我能使山陽至
 荊,便即授首,諸君試觀何如。」及山陽至巴陵,高祖復令天虎齎書與穎胄兄弟。去後,高祖謂張弘策曰:「夫用兵之道,攻心為上,攻城次之,心戰為上,兵戰次之,今日是也。近遣天虎往州府,人皆有書。今段乘驛甚急,止有兩封與行事兄弟,云「天虎口具」;及問天虎而口無所說,行事不得相聞,不容妄有所道。天虎是行事心膂,彼聞必謂行事與天虎共隱其事,則人人生疑。山陽惑於眾口,判相嫌貳,則行事進退無以自明,必漏吾謀內。是馳兩空函定一州矣。山陽至江安,聞之,果疑不上。穎胄大懼,乃斬天虎,送首山陽。山陽信之,將數十人馳入,穎胄伏
 甲斬之,送首高祖。仍以南康王尊號之議來告,且曰:「時月未利,當須來年二月;遽便進兵,恐非廟算。」高祖答曰:「今坐甲十萬,糧用自竭,況所藉義心,一時驍銳,事事相接,猶恐疑怠;若頓兵十旬,必生悔吝。童兒立異,便大事不成。今太白出西方,仗義而動,天時人謀,有何不利?處分已定,安可中息?昔武王伐紂,行逆太歲,復須待年月乎?」



 竟陵太守曹景宗遣杜思沖勸高祖迎南康王都襄陽,待正尊號,然後進軍。高祖不從。王茂又私於張弘策曰:「我奉事節下,義無進退,然今者以南康置人手中,彼便挾天子以令諸侯,而節下前去為人所使,此豈歲寒
 之計?」弘策言之,高祖曰:「若使前途大事不捷,故自蘭艾同焚;若功業克建,威懾四海,號令天下,誰敢不從!豈是碌碌受人處分?待至石城,當面曉王茂、曹景宗也。」於沔南立新野郡,以集新附。



 三年二月,南康王為相國,以高祖為征東將軍,給鼓吹一部。戊申,高祖發襄陽。留弟偉守襄陽城,總州府事,弟憺守壘城,府司馬莊丘黑守樊城,功曹史吉士詢兼長史,白馬戍主黃嗣祖兼司馬,鄀令杜永兼別駕,小府錄事郭儼知轉漕。移檄京邑曰:夫道不常夷,時無永化,險泰相沿,晦明非一,皆屯困而後亨,資多難以啟聖。故昌邑悖德,孝宣聿興,海西亂政,簡
 文升歷,並拓緒開基,紹隆寶命,理驗前經,事昭往策。



 獨夫擾亂天常,毀棄君德,姦回淫縱,歲月滋甚。挺虐於剪之年,植險於髫丱之日。猜忌凶毒,觸途而著,暴戾昏荒,與事而發。自大行告漸,喜容前見,梓宮在殯,靦無哀色,懽娛遊宴,有過平常,奇服異衣,更極誇麗。至於選采妃嬪,姊妹無別,招侍巾櫛,姑姪莫辨,掖庭有稗販之名,姬姜被乾殳之服。至乃形體宣露,褻衣顛倒,斬斫其間,以為懽笑。騁肆淫放,驅屏郊邑。老弱波流,士女塗炭。行產盈路,輿尸竟道,母不及抱,子不遑哭。劫掠剽虜,以日繼夜。晝伏宵遊,曾無休息。淫酗摐肆,酣歌壚邸。寵恣愚
 豎,亂惑妖甗。梅蟲兒、茹法珍臧獲斯小,專制威柄,誅剪忠良,屠滅卿宰。劉鎮軍舅氏之尊,盡忠奉國;江僕射外戚之重,竭誠事上;蕭領軍葭莩之宗,志存柱石;徐司空、沈僕射搢紳冠冕,人望攸歸。或《渭陽》余感,或勛庸允穆,或誠著艱難,或劬勞王室,並受遺託,同參顧命,送往事居,俱竭心力。宜其慶溢當年,祚隆後裔;而一朝齏粉,孩稚無遺。人神怨結,行路嗟憤。



 蕭令君忠公幹伐,誠貫幽顯。往年寇賊遊魂,南鄭危逼,拔刃飛泉,孤城獨振。及中流逆命,憑陵京邑,謀猷禁省,指授群帥,剋剪鯨鯢,清我王度。崔慧景奇鋒迅駭,兵交象魏,武力喪魂,義夫奪膽,
 投名送款,比屋交馳,負糧影從,愚智競赴。復誓旅江甸,奮不顧身,獎厲義徒,電掩彊敵,克殲大憝,以固皇基。功出桓、文,勛超伊、呂;而勞謙省己,事昭心跡,功遂身退,不祈榮滿。敦賞未聞,禍酷遄及,預稟精靈,孰不冤痛!而群孽放命,蜂蠆懷毒,乃遣劉山陽驅扇逋逃,招逼亡命,潛圖密構,規見掩襲。蕭右軍、夏侯征虜忠斷夙舉,義形於色,奇謀宏振,應手梟懸,天道禍淫,罪不容戮。至於悖禮違教,傷化虐人,射天彈路,比之猶善,刳胎斫脛,方之非酷,盡珝縣之竹,未足紀其過,窮山澤之兔,不能書其罪。自草昧以來,圖牒所記,昏君暴后,未有若斯之甚者也。



 既人神乏主,宗稷阽危,海內沸騰,氓庶板蕩,百姓懍懍,如崩厥角,蒼生喁喁,投足無地。幕府荷眷前朝,義均休戚,上懷委付之重,下惟在原之痛,豈可臥薪引火,坐觀傾覆!至尊體自高宗,特鐘慈寵,明並日月,粹昭靈神,祥啟元龜,符驗當璧,作鎮陜籓,化流西夏,謳歌攸奉,萬有樂推。右軍蕭穎胄、征虜將軍夏侯詳並同心翼戴,即宮舊楚,三靈再朗,九縣更新,升平之運,此焉復始,康哉之盛,在乎茲日。然帝德雖彰,區宇未定,元惡未黜,天邑猶梗。仰稟宸規,率前啟路。即日遣冠軍、竟陵內史曹景宗等二十軍主,長槊五萬,驥騄為群,鶚視爭先,龍驤並驅,
 步出橫江,直指朱雀。長史、冠軍將軍、襄陽太守王茂等三十軍主,戈船七萬,乘流電激,推鋒扼險,斜趣白城。南中郎諮議參軍、軍主蕭偉等三十九軍主,巨艦迅楫,衝波噎水,旗鼓八萬,焱集石頭。南中郎諮議參軍、軍主蕭憺等四十二軍主,熊羆之士,甲楯十萬,沿波馳艓,掩據新亭。益州刺史劉季連、梁州刺史柳惔、司州刺史王僧景、魏興太守裴帥仁、上庸太守韋睿、新城太守崔僧季,並肅奉明詔,龔行天罰。蜀、漢果銳,沿流而下;淮、汝勁勇,望波遄騖。幕府總率貔貅,驍勇百萬,繕甲燕弧,屯兵冀馬,摐金沸地,鳴鞞聒天,霜鋒曜日,朱旗絳珝,方舟千里,
 駱驛係進。蕭右軍訏謨上才,兼資文武,英略峻遠,執鈞匡世。擁荊南之眾,督四方之師,宣讚中權,奉衛輿輦。旍麾所指,威棱無外,龍驤虎步,並集建業。黜放愚狡,均禮海昏,廓清神甸,掃定京宇。譬猶崩泰山而壓蟻壤,決懸河而注熛燼,豈有不殄滅者哉!



 今資斧所加,止梅蟲兒、茹法珍而已。諸君咸世胄羽儀,書勛王府,皆俯眉姦黨,受制凶威。若能因變立功,轉禍為福,並誓河、岳,永紆青紫。若執迷不悟,距逆王師,大眾一臨,刑茲罔赦,所謂火烈高原,芝蘭同泯。勉求多福,無貽後悔。賞罰之科,有如白水。



 高祖至竟陵,命長史王茂與太守曹景宗為前軍,
 中兵參軍張法安守竟陵城。茂等至漢口,輕兵濟江,逼郢城。其刺史張沖置陣據石橋浦,義師與戰不利,軍主朱僧起死之。諸將議欲併軍圍郢,分兵以襲西陽、武昌。高祖曰:「漢口不闊一里,箭道交至,房僧寄以重兵固守,為郢城人掎角。若悉眾前進,賊必絕我軍後,一朝為阻,則悔無所及。今欲遣王、曹諸軍濟江,與荊州軍相會,以逼賊壘。吾自後圍魯山,以通沔、漢。鄖城、竟陵間粟,方舟而下;江陵、湘中之兵,連旗繼至。糧食既足,士眾稍多,圍守兩城,不攻自拔,天下之事,臥取之耳。」諸將皆曰「善」。乃命王茂、曹景宗帥眾濟岸,進頓九里。其日,張沖出軍迎
 戰,茂等邀擊,大破之,皆棄甲奔走。荊州遣冠軍將軍鄧元起、軍主王世興、田安等數千人,會大軍於夏首。高祖築漢口城以守魯山,命水軍主張惠紹、朱思遠等遊遏江中,絕郢、魯二城信使。



 三月,乃命元起進據南堂西陼,田安之頓城北,王世興頓曲水故城。是時張沖死,其眾復推軍主薛元嗣及沖長史程茂為主。乙巳,南康王即帝位於江陵,改永元三年為中興元年,遙廢東昏為涪陵王。以高祖為尚書左僕射,加征東大將軍、都督征討諸軍事,假黃鉞。西臺又遣冠軍將軍蕭穎達領兵會于軍。是日,元嗣軍主沈難當率輕舸數千,亂流來戰,張惠
 紹等擊破,盡擒之。四月,高祖出沔,命王茂、蕭穎達等進軍逼郢城。元嗣戰頗疲,因不敢出。諸將欲攻之,高祖不許。五月,東昏遣寧朔將軍吳子陽、軍主光子衿等十三軍救郢州,進據巴口。



 六月,西臺遣衛尉席闡文勞軍,齎蕭穎胄等議,謂高祖曰:「今頓兵兩岸,不併軍圍郢,定西陽、武昌,取江州,此機已失;莫若請救於魏,與北連和,猶為上策。」高祖謂闡文曰:「漢口路通荊、雍,控引秦、梁,糧運資儲,聽此氣息,所以兵壓漢口,連絡數州。今若併軍圍城,又分兵前進,魯山必阻沔路,所謂扼喉。若糧運不通,自然離散,何謂持久?鄧元起近欲以三千兵往定尋陽,
 彼若懽然悟機,一酈生亦足;脫距王師,故非三千能下。進退無據,未見其可。西陽、武昌,取便得耳,得便應鎮守。守兩城不減萬人,糧儲稱是,卒無所出。脫賊軍有上者,萬人攻一城,兩城勢不得相救。若我分軍應援,則首尾俱弱;如其不遣,孤城必陷。一城既沒,諸城相次土崩,天下大事於是去矣。若郢州既拔,席卷沿流,西陽、武昌,自然風靡,何遽分兵散眾,自貽其憂!且丈夫舉動,言靜天步;況擁數州之兵以誅群豎,懸河注火,奚有不滅?豈容北面請救,以自示弱!彼未必能信,徒貽我醜聲。此之下計,何謂上策?卿為我白鎮軍:前途攻取,但以見付,事在
 目中,無患不捷,恃鎮軍靖鎮之耳。」



 吳子陽等進軍武口,高祖乃命軍主梁天惠、蔡道祐據漁湖城,唐脩期、劉道曼屯白陽壘,夾兩岸而待之。子陽又進據加湖,去郢三十里,傍山帶水,築壘柵以自固。魯山城主房僧寄死,眾復推助防孫樂祖代之。七月,高祖命王茂帥軍主曹仲宗、康絢、武會超等潛師襲加湖,將逼子陽。水涸不通艦,其夜暴長,眾軍乘流齊進,鼓噪攻之,賊俄而大潰,子陽等竄走,眾盡溺於江。王茂虜其餘而旋。於是郢、魯二城相視奪氣。



 先是,東昏遣冠軍將軍陳伯之鎮江州,為子陽等聲援。高祖乃謂諸將曰:「夫征討未必須實力,所聽
 威聲耳。今加湖之敗,誰不弭服。陳虎牙即伯之子,狼狽奔歸,彼間人情,理當忷懼,我謂九江傳檄可定也。」因命搜所獲俘囚,得伯之幢主蘇隆之,厚加賞賜,使致命焉。魯山城主孫樂祖、郢城主程茂、薛元嗣相繼請降。初,郢城之閉,將佐文武男女口十餘萬人,疾疫流腫死者十七八,及城開,高祖並加隱恤,其死者命給棺槥。



 先是,汝南人胡文超起義於灄陽,求討義陽、安陸等郡以自效,高祖又遣軍主唐脩期攻隨郡,並剋之。司州刺史王僧景遣子貞孫入質。司部悉平。



 陳伯之遣蘇隆之反命,求未便進軍。高祖曰:「伯之此言,意懷首鼠,及其猶豫,急往
 逼之,計無所出,勢不得暴。」乃命鄧元起率眾,即日沿流。八月,天子遣黃門郎蘇回勞軍。高祖登舟,命諸將以次進路,留上庸太守韋睿守郢城,行州事。鄧元起將至尋陽,陳伯之猶猜懼,乃收兵退保湖口,留其子虎牙守盆城。及高祖至,乃束甲請罪。九月,天子詔高祖平定東夏,並以便宜從事。是月,留少府、長史鄭紹叔守江州城。前軍次蕪湖,南豫州刺史申胄棄姑孰走,至是時大軍進據之,仍遣曹景宗、蕭穎達領馬步進頓江寧。東昏遣征虜將軍李居士率步軍迎戰,景宗擊走之。於是王茂、鄧元起、呂僧珍進據赤鼻邏,曹景宗、陳伯之為遊兵。是日,
 新亭城主江道林率兵出戰,眾軍擒之於陣。大軍次新林,命王茂進據越城,曹景宗據皂莢橋,鄧元起據道士墩,陳伯之據籬門。道林餘眾退屯航南,義軍迫之,因復散走,退保朱爵,憑淮以自固。時李居士猶據新亭壘,請東昏燒南岸邑屋以開戰場。自大航以西、新亭以北,蕩然矣。



 十月,東昏石頭軍主朱僧勇率水軍二千人歸降。東昏又遣征虜將軍王珍國率軍主胡虎牙等列陣於航南大路,悉配精手利器,尚十餘萬人。閹人王倀子持白虎幡督率諸軍,又開航背水,以絕歸路。王茂、曹景宗等掎角奔之,將士皆殊死戰,無不一當百,鼓噪震天地。珍
 國之眾,一時土崩,投淮死者,積屍與航等,後至者乘之以濟,於是朱爵諸軍望之皆潰。義軍追至宣陽門,李居士以新亭壘、徐元瑜以東府城降,石頭、白下諸軍並宵潰。壬午,高祖鎮石頭,命眾軍圍六門,東昏悉焚燒門內,驅逼營署、官府並入城,有眾二十萬。青州刺史桓和紿東昏出戰,因以其眾來降。高祖命諸軍築長圍。



 初,義師之逼,東昏遣軍主左僧慶鎮京口,常僧景鎮廣陵,李叔獻屯瓜步,及申胄自姑孰奔歸,又使屯破墩以為東北聲援。至是,高祖遣使曉喻,並率眾降。乃遣弟輔國將軍秀鎮京口,輔國將軍恢屯破墩,從弟寧朔將軍景鎮廣
 陵。吳郡太守蔡夤棄郡赴義師。



 十二月丙寅旦,兼衛尉張稷、北徐州刺史王珍國斬東昏,送首義師。高祖命呂僧珍勒兵封府庫及圖籍,收甗妾潘妃及凶黨王咺之以下四十一人屬吏誅之。宣德皇后令廢涪陵王為東昏侯,依漢海昏侯故事。授高祖中書監、都督揚、南徐二州諸軍事、大司馬、錄尚書、驃騎大將軍、揚州刺史,封建安郡公,食邑萬戶,給班劍四十人,黃鉞、侍中、征討諸軍事並如故;依晉武陵王遵承制故事。



 己卯,高祖入屯閱武堂。下令曰:「皇家不造,遘此昏凶,禍挻動植,虐被人鬼,社廟之危,蠢焉如綴。吾身籍皇宗,曲荷先顧,受任邊疆,
 推轂萬里,眷言瞻烏,痛心在目,故率其尊主之情,厲其忘生之志。雖寶歷重升,明命有紹,而獨夫醜縱,方煽京邑。投袂援戈,克弭多難。虐政橫流,為日既久,同惡相濟,諒非一族。仰稟朝命,任在專征,思播皇澤,被之率土。凡厥負釁,咸與惟新。可大赦天下;唯王咺之等四十一人不在赦例。」



 又令曰:「夫樹以司牧,非役物以養生;視民如傷,豈肆上以縱虐。廢主棄常,自絕宗廟。窮凶極悖,書契未有。征賦不一,苛酷滋章。緹繡土木,菽粟犬馬,徵發閭左,以充繕築。流離寒暑,繼以疫癆,轉死溝渠,曾莫救恤,朽肉枯骸,烏鳶是厭。加以天災人火,屢焚宮掖,官府臺
 寺,尺椽無遺,悲甚《黍離》,痛兼《麥秀》。遂使億兆離心,疆徼侵弱,斯人何辜,離此塗炭!今明昏遞運,大道公行,思治之氓,來蘇茲日。猥以寡薄,屬當大寵,雖運距中興,艱同草昧,思闡皇休,與之更始。凡昏制、謬賦、淫刑、濫役,外可詳檢前源,悉皆除蕩。其主守散失,諸所損耗,精立科條,咸從原例。」



 又曰:「永元之季,乾維落紐。政實多門,有殊衛文之代;權移於下,事等曹恭之時。遂使閹尹有翁媼之稱,高安有法堯之旨。鬻獄販官,錮山護澤,開塞之機,奏成小醜。直道正義,擁抑彌年,懷冤抱理,莫知誰訴。姦吏因之,筆削自己。豈直賈生流涕,許伯哭時而已哉!今理
 運惟新,政刑得所,矯革流弊,實在茲日。可通檢尚書眾曹,東昏時諸諍訟失理及主者淹停不時施行者,精加訊辨,依事議奏。」



 又下令,以義師臨陣致命及疾病死亡者,並加葬斂,收恤遺孤。又令曰:「朱爵之捷,逆徒送死者,特許家人殯葬;若無親屬,或有貧苦,二縣長尉即為埋掩。建康城內,不達天命,自取淪滅,亦同此科。」



 二年正月,天子遣兼侍中席闡文、兼黃門侍郎樂法才慰勞京邑。追贈高祖祖散騎常侍左光祿大夫,考侍中丞相。



 高祖下令曰:「夫在上化下,草偃風從,世之澆淳,恒由此作。自永元失德,書契未紀,窮凶極悖,焉可勝言。既而皞室外
 構,傾宮內積,奇技異服,殫所未見。上慢下暴,淫侈競馳。國命朝權,盡移近習。販官鬻爵,賄貨公行。並甲第康衢,漸臺廣室。長袖低昂,等和戎之賜;珍羞百品,同伐冰之家。愚民因之,浸以成俗。驕艷競爽,夸麗相高。至乃市井之家,貂狐在御;工商之子,緹繡是襲。日入之次,夜分未反,昧爽之朝,期之清旦。聖明肇運,厲精惟始,雖曰纘戎,殆同創革。且淫費之後,繼以興師,巨橋、鹿臺,凋罄不一。孤忝荷大寵,務在澄清,思所以仰述皇朝大帛之旨,俯厲微躬鹿裘之義,解而更張,斲雕為樸。自非可以奉粢盛,脩紱冕,習禮樂之容,繕甲兵之備,此外眾費,一皆禁
 絕。御府中署,量宜罷省。掖庭備御妾之數,大予絕鄭衛之音。其中有可以率先卿士,准的庶,菲食薄衣,請自孤始。加群才並軌,九官咸事,若能人務退食,競存約己,移風易俗,庶期月有成。昔毛玠在朝,士大夫不敢靡衣偷食。魏武歎曰:「孤之法不如毛尚書。」孤雖德謝往賢,任重先達,實望多士得其此心。外可詳為條格。」



 戊戌,宣德皇后臨朝,入居內殿。拜帝大司馬,解承制,百僚致敬如前。詔進高祖都督中外諸軍事,劍履上殿,入朝不趨,贊拜不名。加前後部羽葆鼓吹。置左右長史、司馬、從事中郎、掾、屬各四人,並依舊辟士,餘並如故。



 詔曰:夫日月麗
 天,高明所以表德;山岳題地,柔博所以成功。故能庶物出而資始,河海振而不洩。二象貞觀,代之者人。是以七輔、四叔,致無為於軒、昊;韋、彭、齊、晉,靖衰亂於殷、周。



 大司馬攸縱自天,體茲齊聖,文洽九功,武苞七德。欽惟厥始,徽猷早樹,誠著艱難,功參帷幙。錫賦開壤,式表厥庸。建武升歷,邊隙屢啟,公釋書輟講,經營四方。司、豫懸切,樊、漢危殆,覆彊寇於沔濱,僵胡馬於鄧汭。永元肇號,難結群醜,專威擅虐,毒被含靈,溥天惴惴,命懸晷刻。否終有期,神謨載挺,首建大策,惟新鼎祚。投袂勤王,沿流電舉,魯城雲撤,夏汭霧披,加湖群盜,一鼓殄拔,姑孰連旍,倏
 焉冰泮。取新壘其如拾芥,撲朱爵其猶掃塵。霆電外駭,省闥內傾,餘醜纖蠹,蚳蝝必盡。援彼已溺,解此倒懸,塗懽里抃,自近及遠。畿甸夷穆,方外肅寧,解茲虐網,被以寬政。積弊窮昏,一朝載廓,聲教遐漸,無思不被。雖伊尹之執茲壹德,姬旦之光於四海,方斯蔑如也。



 昔呂望翼佐聖君,猶享四履之命;文侯立功平后,尚荷二弓之錫,況於盛德元勛,超邁自古。黔首惵惵,待以為命,救其已然,拯其方斲,式閭表墓,未或能比;而大輅渠門,輟而莫授,眷言前訓,無忘終食。便宜敬升大典,式允群望。其進位相國,總百揆,揚州刺史;封十郡為梁公,備九錫之禮,
 加璽紱遠遊冠,位在諸王上,加相國綠綟綬。其驃騎大將軍如故。依舊置梁百司。



 策曰:二儀寂寞,由寒暑而代行,三才並用,資立人以為寶,故能流形品物,仰代天工。允茲元輔,應期挺秀,裁成天地之功,幽協神明之德。撥亂反正,濟世寧民,盛烈光於有道,大勳振於無外,雖伊陟之保乂王家,姬公之有此丕訓,方之蔑如也。今將授公典策,其敬聽朕命:上天不造,難鐘皇室,世祖以休明早崩,世宗以仁德不嗣,高宗襲統,宸居弗永,雖夙夜劬勞,而隆平不洽。嗣君昏暴,書契弗睹。朝權國柄,委之群甗。剿戮忠賢,誅殘台輔,含冤抱痛,噍類靡餘。實繁非一,
 並專國命。頻笑致災,睚眥及禍。嚴科毒賦,載離比屋,溥天熬熬,置身無所。冤頸引決,道樹相望,無近無遠,號天靡告。公藉昏明之期,因兆民之願,援帥群后,翊成中興。宗社之危已固,天人之望允塞,此實公紐我絕綱,大造皇家者也。



 永明季年,邊隙大啟,荊河連率,招引戎荒,江、淮擾逼,勢同履虎。公受言本朝,輕兵赴襲,縻以長算,制之環中。排危冒險,彊柔遞用,坦然一方,還成籓服。此又公之功也。在昔隆昌,洪基已謝,高宗慮深社稷,將行權道。公定策帷帳,激揚大節,廢帝立王,謀猷深著。此又公之功也。建武闡業,厥猷雖遠,戎狄內侵,憑陵關塞,司部
 危逼,淪陷指期。公治兵外討,卷甲長騖,接距交綏,電激風掃,摧堅覆銳,咽水塗原,執俘象魏,獻馘海渚,焚廬毀帳,號哭言歸。此又公之功也。樊、漢阽切,羽書續至。公星言鞠旅,稟命徂征,而軍機戎統,事非己出,善策嘉謀,抑而莫允。鄧城之役,胡馬卒至,元帥潛及,不相告報,棄甲捐師,餌之虎口。公南收散卒,北禦雕騎,全眾方軌,案路徐歸,拯我邊危,重獲安堵。此又公之功也。漢南迥弱,咫尺勍寇,兵糧蓋闕,器甲靡遺。公作籓爰始,因資靡託,整兵訓卒,蒐狩有序,俾我危城,翻為彊鎮。此又公之功也。永元紀號,瞻烏已及,雖廢昏有典,而伊、霍稱難。公首建
 大策,爰立明聖,義踰邑綸,勳高代人,易亂以化,俾昏作明。此又公之功也。文王之風,雖被江、漢,京邑蠢動,湮為洪流,句吳、於越,巢幕匪喻。公投袂萬里,事惟拯溺,義聲所覃,無思不韙。此又公之功也。魯城、夏汭,梗據中流,乘山置壘,縈川自固。公御此烏集,陵茲地險,頓兵坐甲,寒往暑移,我行永久,士忘歸願,經以遠圖,御以長策,費無遺矢,戰未窮兵,踐華之固,相望俱拔。此又公之功也。惟此群凶,同惡相濟,緣江負險,蟻聚加湖。水陸盤據,規援夏首,桴鳷一臨,應時褫潰。此又公之功也。姦甗震皇,復懷舉斧,蓄兵九派,用擬勤王。公棱威直指,勢踰風電,旌
 旆小臨,全州稽服。此又公之功也。姑孰衝要,密邇京畿,凶徒熾聚,斷塞津路。公偏師啟塗,排方繼及,兵威所震,望旗自駭,焚舟委壁,卷甲宵遁。此又公之功也。群豎猖狂,志在借一,豕突淮涘,武騎如雲。公爰命英勇,因機騁銳,氣冠版泉,勢踰洹水,追奔逐北,奄有通津,熊耳比峻,未足云擬,睢水不流,曷其能及。此又公之功也。瑯邪、石首,襟帶岨固,新壘、東墉,金湯是埒。憑險作守,兵食兼資,風激電駭,莫不震疊,城復于隍,於是乎在。此又公之功也。獨夫昏很,憑城靡懼,鼓鐘鞺鞜,慠若有餘。狎是邪甗,忌斯冠冕,凶狡因之,將逞孥戮。公奇謨密運,盛略潛通,
 忠勇之徒,得申厥效,白旗宣室,未之或比。此又公之功也。



 公有拯億兆之勳,重之以明德,爰初厲志,服道儒門,濯纓來仕,清猷映代。時運艱難,宗社危殆,昆崗已燎,玉石同焚。驅率貔貅,抑揚霆電,義等南巢,功齊牧野。若夫禹功寂漠,微管誰嗣,拯其將魚,驅其被髮,解茲亂網,理此棼絲,復禮衽席,反樂河海。永平故事,聞之者歎息;司隸舊章,見之者隕涕。請我民命,還之斗極。憫憫搢紳,重荷戴天之慶;哀哀黔首,復蒙履地之恩。德踰嵩、岱,功鄰造物,超哉邈矣,越無得而言焉。



 朕又聞之:疇庸命德,建侯作屏,咸用剋固四維,永隆萬葉。是以《二南》流化,九伯
 斯征,王道淳洽,刑措罔用。覆政弗興,歷茲永久,如毀既及,晉、鄭靡依。惟公經綸天地,寧濟區夏,道冠乎伊、稷,賞薄於桓、文,豈所以憲章齊、魯,長轡宇宙。敬惟前烈,朕甚懼焉。今進授相國,改揚州刺史為牧,以豫州之梁郡歷陽、南徐州之義興、揚州之淮南宣城吳吳興會稽新安東陽十郡,封公為梁公。錫茲白土,苴以白茅,爰定爾邦,用建塚社。在昔旦、奭,入居保佑,逮于畢、毛,亦作卿士,任兼內外,禮實宜之。今命使持節兼太尉王亮授相國揚州牧印綬,梁公璽紱;使持節兼司空王志授梁公茅土,金虎符第一至第五左,竹使符第一至第十左。相國位冠
 群后,任總百司,恒典彞數,宜與事革。其以相國總百揆,去錄尚書之號,上所假節、侍中貂蟬、中書監印、中外都督大司馬印綬,建安公印策,驃騎大將軍如故。又加公九錫,其敬聽後命:以公禮律兼修,刑德備舉,哀矜折獄,罔不用情,是用錫公大輅、戎輅各一,玄牡二駟。公勞心稼穡,念在民天,丕崇本務,惟穀是寶,是用錫公袞冕之服,赤舄副焉。公熔鈞所被,變風以雅,易俗陶民,載和邦國,是用錫公軒懸之樂,六佾之舞。公文德廣覃,義聲遠洽,椎髻髽首,夷歌請吏,是用錫公朱戶以居。公揚清抑濁,官方有序,多士聿興,《棫呻樸》流詠,是用錫公納陛以登。
 公正色御下,以身軌物,式遏不虞,折衝惟遠,是用錫公虎賁之士三百人。公威同夏日,志清姦宄,放命圮族,刑茲罔赦,是用錫公鈇、鉞各一。公跨躡嵩溟,陵厲區宇,譬諸日月,容光必至,是用錫公彤弓一,彤矢百;盧弓十,盧矢千。公永言惟孝,至感通神,恭嚴祀典,祭有餘敬,是用錫公秬鬯一卣,圭瓚副焉。梁國置丞相以下,一遵舊式。欽哉!其敬循往策,祗服大禮,對揚天眷,用膺多福,以弘我太祖之休命!



 高祖固辭。府僚勸進曰:「伏承嘉命,顯至佇策。明公逡巡盛禮,斯實謙尊之旨,未窮遠大之致。何者?嗣君棄常,自絕宗社,國命民主,剪為仇仇,折棟崩榱,
 壓焉自及,卿士懷脯斫之痛,黔首懼比屋之誅。明公亮格天之功,拯水火之切,再躔日月,重綴參辰,反龜玉於塗泥,濟斯民於坑岸,使夫匹婦童兒,羞言伊、呂,鄉校里塾,恥談五霸。而位卑乎阿衡,地狹於曲阜,慶賞之道,尚其未洽。夫大寶公器,非要非距,至公至平,當仁誰讓?明公宜祗奉天人,允膺大禮。無使後予之歌,同彼胥怨,兼濟之人,翻為獨善。」公不許。



 二月辛酉,府僚重請曰:「近以朝命蘊策,冒奏丹誠,奉被還令,未蒙虛受,搢紳顒顒,深所未達。蓋聞受金於府,通人弘致,高蹈海隅,匹夫小節,是以履乘石而周公不以為疑,贈玉璜而太公不以為
 讓。況世哲繼軌,先德在民,經綸草昧,歎深微管。加以朱方之役,荊河是依,班師振旅,大造王室。雖復累繭救宋,重胝存楚,居今觀古,曾何足云。而惑甚盜鐘,功疑不賞,皇天后土,不勝其酷。是以玉馬駿奔,表微子之去;金板出地,告龍逢之冤。明公據鞍輟哭,厲三軍之志,獨居掩涕,激義士之心,故能使海若登祗,罄圖效祉,山戎、孤竹,束馬影從,伐罪弔民,一匡靜亂。匪叨天功,實勤濡足。且明公本自諸生,取樂名教,道風素論,坐鎮雅俗,不習孫、吳,遘茲神武。驅盡誅之氓,濟必封之谷,龜玉不毀,誰之功與?獨為君子,將使伊、周何地?」於是始受相國梁公之
 命。



 是日,焚東昏淫奢異服六十二種於都街。湘東王寶晊謀反,賜死。詔追贈梁公故夫人為梁妃。



 乙丑,南兗州隊主陳文興於桓城內鑿井,得玉鏤騏驎、金鏤玉璧、水精環各二枚。又建康令羊瞻解稱鳳皇見縣之桐下里。宣德皇后稱美符瑞,歸于相國府。



 丙寅,詔:「梁國初建,宜須綜理,可依舊選諸要職,悉依天朝之制。」



 高祖上表曰:臣聞以言取士,士飾其言,以行取人,人竭其行。所謂才生於世,窮達惟時;而風流遂往,馳騖成俗,媒孽誇炫,利盡錐刀,遂使官人之門,肩摩轂擊。豈直暴蓋露冠,不避寒暑,遂乃戢屨杖策,風雨必至。良由鄉舉里選,不師古
 始,稱肉度骨,遺之管庫。加以山河梁畢,闕輿征之恩;金、張、許、史,忘舊業之替。吁,可傷哉!且夫譜牒訛誤,詐偽多緒,人物雅俗,莫肯留心。是以冒襲良家,即成冠族;妄修邊幅,便為雅士;負俗深累,遽遭寵擢;墓木已拱,方被徽榮。故前代選官,皆立選簿,應在貫魚,自有銓次。胄籍升降,行能臧否,或素定懷抱,或得之餘論,故得簡通賓客,無事掃門。頃代陵夷,九流乖失。其有勇退忘進,懷質抱真者,選部或以未經朝謁,難於進用。或有晦善藏聲,自埋衡蓽,又以名不素著,絕其階緒。必須畫刺投狀,然後彈冠,則是驅迫廉捴,獎成澆競。愚謂自今選曹宜精隱
 括,依舊立簿,使冠屨無爽,名實不違,庶人識崖涘,造請自息。



 且聞中間立格,甲族以二十登仕,後門以過立試吏,求之愚懷,抑有未達。何者?設官分職,惟才是務。若八元立年,居皂隸而見抑;四凶弱冠,處鼎族而宜甄。是則世祿之家,無意為善;布衣之士,肆心為惡。豈所以弘獎風流,希向後進?此實巨蠹,尤宜刊革。不然,將使周人有路傍之泣,晉臣興漁獵之歎。且俗長浮競,人寡退情,若限歲登朝,必增年就宦,故貌實昏童,籍已踰立,滓穢名教,於斯為甚。



 臣總司內外,憂責是任,朝政得失,義不容隱。伏願陛下垂聖淑之姿,降聽覽之末,則彞倫自穆,憲
 章惟允。



 詔依高祖表施行。



 丙戌,詔曰:嵩高惟岳,配天所以流稱;大啟南陽,霸德所以光闡。忠誠簡帝,番君膺上爵之尊;勤勞王室,姬公增附庸之地。前王令典,布諸方策,長祚字,罔不由此。



 相國梁公,體茲上哲,齊聖廣淵。文教內洽,武功外暢。推轂作籓,則威懷被於殊俗;治兵教戰,則霆雷赫於萬里。道喪時昏,讒邪孔熾。豈徒宗社如綴,神器莫主而已哉!至於兆庶殲亡,衣冠殄滅,餘類殘喘,指命崇朝,含生業業,投足無所,遂乃山川反覆,草木塗地。與夫仁被行葦之時,信及豚魚之日,何其遼夐相去之遠歟!公命師鞠旅,指景長騖。而本朝危切,樊、鄧
 遐遠,凶徒盤據,水陸相望,爰自姑孰,屈于夏首,嚴城勁卒,憑川為固。公沿漢浮江,電激風掃,舟徒水覆,地險雲傾,藉茲義勇,前無彊陣,拯危京邑,清我帝畿,撲既燎於原火,免將誅於比屋。悠悠兆庶,命不在天;茫茫六合,咸受其賜。匡俗正本,民不失職。仁信並行,禮樂同暢。伊、周未足方軌,桓、文遠有慚德。而爵後籓牧,地終秦、楚,非所以式酬光烈,允答元勳。實由公履謙為本,形於造次,嘉數未申,晦朔增佇。便宜崇斯禮秩,允副遐邇之望。可進梁公爵為王。以豫州之南譙、盧江、江州之尋陽、郢州之武昌、西陽、南徐州之南瑯邪、南東海、晉陵、揚州之臨海、
 永嘉十郡,益梁國,并前為二十郡。其相國、揚州牧、驃騎大將軍如故。



 公固辭。有詔斷表。相國左長史王瑩等率百僚敦請。



 三月辛卯,延陵縣華陽邏主戴車牒稱云:「十二月乙酉,甘露降茅山,彌漫數里。正月己酉,邏將潘道蓋於山石穴中得毛龜一。二月辛酉,邏將徐靈符又於山東見白麞一。丙寅平旦,山上雲霧四合,須臾有玄黃之色,狀如龍形,長十餘丈,乍隱乍顯,久乃從西北升天。」丁卯,兗州刺史馬元和簽:「所領東平郡壽張縣見騶虞一。」



 癸巳,受梁王之命。令曰:「孤以虛昧,任執國鈞,雖夙夜勤止,念在興治,而育德振民,邈然尚遠。聖朝永言舊式,
 隆此眷命。侯伯盛典,方軌前烈,嘉錫隆被,禮數昭崇。徒守愿節,終隔體諒。群后百司,重茲敦獎,勉茲厚顏,當此休祚。望昆、彭以長想,欽桓、文而歎息,思弘政塗,莫知津濟。邦甸初啟,籓宇惟新,思覃嘉慶,被之下國。國內殊死以下,今月十五日昧爽以前,一皆原赦。鰥寡孤獨不能自存者,賜穀五斛。府州所統,亦同蠲蕩。」



 丙午,命王冕十有二旒,建天子旌旗,出警入蹕,乘金根車,駕六馬,備五時副車,置旄頭雲罕,樂舞八佾,設鐘鋋宮縣。王妃王子王女爵命之號,一依舊儀。



 丙辰,齊帝禪位于梁王。詔曰:夫五德更始,三正迭興,馭物資賢,登庸啟聖,故帝跡所
 以代昌,王度所以改耀,革晦以明,由來尚矣。齊德淪微,危亡薦襲。隆昌凶虐,實違天地;永元昏暴,取紊人神。三光再沉,七廟如綴。鼎業幾移,含識知泯。我高、明之祚,眇焉將墜。永惟屯難,冰谷載懷。



 相國梁王,天誕睿哲,神縱靈武,德格玄祇,功均造物。止宗社之橫流,反生民之塗炭。扶傾頹構之下,拯溺逝川之中。九區重緝,四維更紐。絕禮還紀,崩樂復張。文館盈紳,戎亭息警。浹海宇以馳風,罄輪裳而稟朔。八表呈祥,五靈效祉。豈止鱗羽禎奇,雲星瑞色而已哉!勳茂於百王,道昭乎萬代,固以明配上天,光華日月者也。河獄表革命之符,圖讖紀代終之
 運。樂推之心,幽顯共積;歌頌之誠,華裔同著。昔水政既微,木德升緒,天之歷數,實有所歸,握鏡璇樞,允集明哲。



 朕雖庸蔽,闇於大道,永鑒崇替,為日已久,敢忘列代之高義,人祇之至願乎!今便敬禪於梁,即安姑孰,依唐虞、晉宋故事。



 四月辛酉,宣德皇后令曰:「西詔至,帝憲章前代,敬禪神器于梁。明可臨軒遣使,恭授璽紱,未亡人便歸于別宮。」壬戌,策曰:咨爾梁王:惟昔邃古之載,肇有生民,皇雄、大庭之辟,赫胥、尊盧之后,斯並龍圖鳥跡以前,慌忽杳冥之世,固無得而詳焉。洎乎農、軒、炎、皞之代,放勛、重華之主,莫不以大道君萬姓,公器御八枿。居之如
 執朽索,去之若捐重負。一駕汾陽,便有窅然之志;暫適箕嶺,即動讓王之心。故知戴黃屋,服玉璽,非所以示貴稱尊;乘大輅,建旂旌,蓋欲令歸趣有地。是故忘己而字兆民,殉物而君四海。及於精華內竭,畚橇外勞,則撫茲歸運,惟能是與。況兼乎笙管革文,威圖啟瑞,攝提夜朗,熒光晝發者哉!四百告終,有漢所以高揖;黃德既謝,魏氏所以樂推。爰及晉、宋,亦弘斯典。我太祖握《河》受歷,應符啟運,二葉重光,三聖係軌。嗣君喪德,昏棄紀度,毀紊天綱,凋絕地紐。茫茫九域,剪為仇仇,溥天相顧,命縣晷刻。斫涉刳孕,於事已輕;求雞徵杖,曾何足譬。是以谷滿
 川枯,山飛鬼哭,七廟已危,人神無主。



 惟王體茲上哲,明聖在躬,稟靈五緯,明並日月。彞倫攸序,則端冕而協邕熙;時難孔棘,則推鋒而拯塗炭。功踰造物,德濟蒼生,澤無不漸,仁無不被,上達蒼昊,下及川泉。文教與鵬翼齊舉,武功與日車並運。固以幽顯宅心,謳訟斯屬;豈徒桴鼓播地,卿雲叢天而已哉!至如晝睹爭明,夜飛枉矢,土淪彗刺,日既星亡,除舊之征必顯,更姓之符允集。是以義師初踐,芳露凝甘,仁風既被,素文自擾,北闕槁街之使,風車火徼之民,膜拜稽首,願為臣妾。鐘石畢變,事表於遷虞;蛟魚並出,義彰於事夏。若夫長民御眾,為之司
 牧,本同己於萬物,乃因心於百姓。寶命無常主,帝王非一族。今仰祗乾象,俯藉人願,敬禪神器,授帝位于爾躬。大祚告窮,天祿永終。於戲!王允執其中,式遵前典,以副昊天之望。禋上帝而臨億兆,格文祖而膺大業,以傳無疆之祚,豈不盛歟!



 又璽書曰:夫生者天地之大德,人者含生之通稱,並首同本,未知所以異也。而稟靈造化,賢愚之情不一;託性五常,彊柔之分或舛。群后靡一,爭犯交興,是故建君立長,用相司牧。非謂尊驕在上,以天下為私者也。兼以三正迭改,五運相遷,綠文赤字,徵《河》表《洛》。在昔勛、華,深達茲義,眷求明哲,授以蒸民。遷虞事夏,
 本因心於百姓;化殷為周,實受命於蒼昊。爰自漢、魏,罔不率由;降及晉、宋,亦遵斯典。我高皇所以格文祖而撫歸運,畏上天而恭寶歷者也。至于季世,禍亂薦臻,王度紛糾,姦回熾積。億兆夷人,刀俎為命,已然之逼,若線之危,跼天蹐地,逃形無所。群凶挾煽,志逞殘戮,將欲先殄衣冠,次移龜鼎。衡、保、周、召,並列宵人。巢幕累卵,方此非切。自非英聖遠圖,仁為己任,則鴟梟厲吻,剪焉已及。



 惟王崇高則天,博厚儀地,熔鑄六合,陶甄萬有。鋒馹交馳,振靈武以遐略;雲雷方扇,鞠義旅以勤王。揚旍旆於遠路,戮姦宄於魏闕。德冠往初,功無與二。弘濟艱難,緝熙
 王道。懷柔萬姓,經營四方。舉直措枉,較如畫一。待旦同乎殷后,日昃過於周文。風化肅穆,禮樂交暢。加以赦過宥罪,神武不殺,盛德昭於景緯,至義感於鬼神。若夫納彼大麓,膺此歸運,烈風不迷,樂推攸在。治五韙於已亂,重九鼎於既輕。自聲教所及,車書所至,革面回首,謳吟德澤。九山滅祲,四瀆安流。祥風扇起,淫雨靜息。玄甲遊於芳荃,素文馴於郊苑。躍九川於清漢,鳴六象於高崗。靈瑞雜沓,玄符昭著。至於星孛紫宮,水效孟月,飛鴻滿野,長彗橫天,取新之應既昭,革故之征必顯。加以天表秀特,軒狀堯姿;君臨之符,諒非一揆。《書》云:「天鑒厥德,用
 集大命。」《詩》云:「文王在上,於昭于天。」所以二儀乃眷,幽明允葉,豈惟宅是萬邦,緝茲謳訟而已哉!



 朕是用擁璇沉首,屬懷聖哲。昔水行告厭,我太祖既受命代終;在日天祿云謝,亦以木德而傳於梁。遠尋前典,降惟近代,百辟遐邇,莫違朕心。今遣使持節、兼太保、侍中、中書監、兼尚書令汝南縣開國侯亮,兼太尉、散騎常侍、中書令新吳縣開國侯志,奉皇帝璽紱。受終之禮,一依唐虞故事。王其陟茲元后,君臨萬方,式傳洪烈,以答上天之休命!



 高祖抗表陳讓,表不獲通。於是,齊百官豫章王元琳等八百一十九人,及梁臺侍中臣雲等一百一十七人,並上
 表勸進,高祖謙讓不受。是日,太史令蔣道秀陳天文符讖六十四條,事並明著。群臣重表固請,乃從之。



\end{pinyinscope}