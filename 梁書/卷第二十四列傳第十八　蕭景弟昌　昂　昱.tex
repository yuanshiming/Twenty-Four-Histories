\article{卷第二十四列傳第十八 蕭景弟昌 昂 昱}

\begin{pinyinscope}

 蕭景,字子昭,高祖從父弟也。父崇之字茂敬,即左光祿大夫道賜之子。道賜三子:長子尚之,字茂先;次太祖文皇帝;次崇之。初,左光祿居於鄉里,專行禮讓,為眾所推。仕歷宋太尉江夏王參軍,終于治書侍御史。齊末,追贈散騎常侍、左光祿大夫。尚之敦厚有德器,為司徒建安王中兵參軍,一府稱為長者;瑯邪王僧虔尤善之,每事
 多與議決。遷步兵校尉,卒官。天監初,追謚文宣侯。尚之子靈鈞,仕齊廣德令。高祖義師至,行會稽郡事,頃之卒。高祖即位,追封東昌縣侯,邑一千戶。子謇嗣。崇之以幹能顯,為政尚嚴厲,官至冠軍將軍、東陽太守。永明中,錢唐唐珝之反,別眾破東陽,崇之遇害。天監初,追謚忠簡侯。



 景八歲隨父在郡,居喪以毀聞。既長好學,才辯能斷。齊建武中,除晉安王國左常侍,遷永寧令,政為百城最。永嘉太守范述曾居郡,號稱廉平,雅服景為政,乃榜郡門曰:「諸縣有疑滯者,可就永寧令決。」頃之,以疾去官。永嘉人胡仲宣等千人詣闕,表請景為郡,不許。還為驃騎
 行參軍。永元二年,以長沙宣武王懿勛,除步兵校尉。是冬,宣武王遇害,景亦逃難。高祖義師至,以景為寧朔將軍、行南兗州軍事。時天下未定,江北傖楚各據塢壁。景示以威信,渠帥相率面縛請罪,旬日境內皆平。中興二年,遷督南兗州諸軍事、輔國將軍、監南兗州。高祖踐阼,封吳平縣侯,食邑一千戶,仍為使持節、都督南、北兗、青、冀四州諸軍事、冠軍將軍、南兗州刺史。詔景母毛氏為國太夫人,禮如王國太妃,假金章紫綬。景居州,清恪有威裁,明解吏職,文案無壅,下不敢欺,吏人畏敬如神。會年荒,計口賑恤,為穀粥於路以賦之,死者給棺具,人甚
 賴焉。



 天監四年,王師北伐,景帥眾出淮陽,進屠宿預。丁母憂,詔起攝職。五年,班師,除太子右衛率,遷輔國將軍、衛尉卿。七年,遷左驍騎將軍,兼領軍將軍。領軍管天下兵要,監局官僚,舊多驕侈,景在職峻切,官曹肅然。制局監皆近倖,頗不堪命,以是不得久留中。尋出為使持節、督雍、梁、南、北秦、郢州之竟陵司州之隨郡諸軍事、信武將軍、寧蠻校尉、雍州刺史。八年三月,魏荊州刺史元志率眾七萬寇潺溝,驅迫群蠻,群蠻悉渡漢水來降。議者以蠻累為邊患,可因此除之。景曰:「窮來歸我,誅之不祥。且魏人來侵,每為矛盾,若悉誅蠻,則魏軍無礙,非長策
 也。」乃開樊城受降。因命司馬朱思遠、寧蠻長史曹義宗、中兵參軍孟惠俊擊志於潺溝,大破之,生擒志長史杜景。斬首萬餘級,流屍蓋漢水,景遣中兵參軍崔繢率軍士收而瘞焉。



 景初到州,省除參迎羽儀器服,不得煩擾吏人。修營城壘,申警邊備,理辭訟,勸農桑。郡縣皆改節自勵,州內清肅,緣漢水陸千餘里,抄盜絕迹。十一年,徵右衛將軍、領石頭戍軍事。十二年,復為使持節、督南、北兗、北徐、青、冀五州諸軍事、信威將軍、南兗州刺史。十三年,徵為領軍將軍,直殿省,知十州損益事,月加祿五萬。



 景為人雅有風力,長於辭令。其在朝廷,為眾所瞻仰。於
 高祖屬雖為從弟,而禮寄甚隆,軍國大事,皆與議決。十五年,加侍中。十七年,太尉、揚州刺史臨川王宏坐法免。詔曰:「揚州應須緝理,宜得其人。侍中、領軍將軍吳平侯景才任此舉,可以安右將軍監揚州,并置佐史,侍中如故,即宅為府。」景越親居揚州,辭讓甚懇惻,至于涕泣,高祖不許。在州尤稱明斷,符教嚴整。有田舍老姥嘗訴得符,還至縣,縣吏未即發,姥語曰:「蕭監州符,火爄汝手,何敢留之!」其為人所畏敬如此。



 十八年,累表陳解,高祖未之許。明年,出為使持節、散騎常侍、都督郢、司、霍三州諸軍事、安西將軍、郢州刺史。將發,高祖幸建興苑餞別,為
 之流涕。既還宮,詔給鼓吹一部。在州復有能名。齊安、竟陵郡接魏界,多盜賊,景移書告示,魏即焚塢戍保境,不復侵略。普通四年,卒于州,時年四十七。詔贈侍中、中撫軍、開府儀同三司。謚曰忠。子勱嗣。



 昌字子建,景第二弟也。齊豫章末,為晉安王左常侍。天監初,除中書侍郎,出為豫章內史。五年,加寧朔將軍。六年,遷持節、督廣、交、越、桂四州諸軍事、輔國將軍、平越中郎將、廣州刺史。七年,進號征遠將軍。九年,分湘州置衡州,以昌為持節、督廣州之綏建湘州之始安諸軍事、信武將軍、衡州刺史,坐免。十三年,起為散騎侍郎,尋以本
 官兼宗正卿。其年,出為安右長史。累遷太子中庶子、通直散騎常侍,又兼宗正卿。昌為人亦明悟,然性好酒,酒後多過。在州郡,每醉輒徑出入人家,或獨詣草野。其於刑戮,頗無期度。醉時所殺,醒或求焉,亦無悔也。屬為有司所劾,入留京師,忽忽不樂,遂縱酒虛悸。在石頭東齋,引刀自刺,左右救之,不殊。十七年,卒,時年三十九。子伯言。



 昂字子明,景第三弟也。天監初,累遷司徒右長史,出為輕車將軍、監南兗州。初,兄景再為南兗,德惠在人,及昂來代,時人方之馮氏。徵為琅邪、彭城二郡太守,軍號如
 先。復以輕車將軍出為廣州刺史。普通二年,為散騎常侍、信威將軍。四年,轉散騎侍郎、中領軍、太子中庶子,出為吳興太守。大通二年,徵為仁威將軍、衛尉卿,尋為侍中,兼領軍將軍。中大通元年,為領軍將軍。二年,封湘陰縣侯,邑一千戶。出為江州刺史。大同元年,卒,時年五十三。謚曰恭。



 昱字子真,景第四弟也。天監初,除秘書郎,累遷太子舍人,洗馬,中書舍人,中書侍郎。每求自試,高祖以為淮南、永嘉、襄陽郡,並不就。志願邊州,高祖以其輕脫無威望,抑而不許。遷給事黃門侍郎。上表曰:「夏初陳啟,未垂採
 照,追懷慚懼,實戰胸心。臣聞暑雨祁寒,小人猶怨;榮枯寵辱,誰能忘懷!臣藉以往因,得預枝戚之重;緣報既雜,時逢坎稟之運。昔在齊季,義師之始,臣乃幼弱,粗有識慮,東西阻絕,歸赴無由,雖未能負戈擐甲,實銜淚憤懣。潛伏東境,備履艱危,首尾三年,亟移數處,雖復飢寒切身,亦不以凍餒為苦。每涉驚疑,惶怖失魄,既乖致命之節,空有項領之憂,希望開泰,冀蒙共樂;豈期二十餘年,功名無紀,畢此身骸,方填溝壑,丹誠素願,溘至長罷,俯自哀憐,能不傷歎!夫自媒自衒,誠哉可鄙;自譽自伐,實在可羞。然量己揆分,自知者審,陳力就列,寧敢空言?是
 以常願一試,屢成干請。夫上應玄象,實不易叨;錦不輕裁,誠難其製。過去業鄣,所以致乖算測。聖監既謂臣愚短,不可試用,豈容久居顯禁,徒穢黃樞。忝竊稍積,恐招物議,請解今職,乞屏退私門。伏願天照,特垂允許。臣雖叨榮兩宮,報效無地,方違省闥,伏深戀悚。」高祖手詔答曰:「昱表如此。古者用人,必前明試,皆須績用既立,乃可自退之高。昔漢光武兄子章、興二人,並有名宗室,就欲習吏事,不過章為平陰令,興為緱氏宰,政事有能,方遷郡守,非直政績見稱,即是光武猶子。昱之才地,豈得比類焉!往歲處以淮南郡,既不肯行;續用為招遠將軍、鎮
 北長史、襄陽太守,又以邊外致辭;改除招遠將軍、永嘉太守,復云內地非願;復問晉安、臨川,隨意所擇,亦復不行。解巾臨郡,事不為薄,數有致辭,意欲何在?且昱諸兄遞居連率,相繼推轂,未嘗缺歲。其同產兄景,今正居籓鎮。朕豈厚於景而薄於昱,正是朝序物議,次第若斯,於其一門,差自無愧。無論今日不得如此;昱兄弟昔在布衣,以處成長,於何取立,豈得任情反道,背天違地。孰謂朝廷無有憲章,特是未欲致之于理。既表解職,可聽如啟。」坐免官。因此杜門絕朝覲,國家慶弔不復通。



 普通五年,坐於宅內鑄錢,為有司所奏,下廷尉,得免死,徙臨海
 郡。行至上虞,有敕追還,且令受菩薩戒。昱既至,恂恂盡禮,改意蹈道,持戒又精潔,高祖甚嘉之,以為招遠將軍、晉陵太守。下車勵名跡,除煩苛,明法憲,嚴於奸吏,優養百姓,旬日之間,郡中大化。俄而暴疾卒,百姓行坐號哭,市里為之喧沸,設祭奠於郡庭者四百餘人。田舍有女人夏氏,年百餘歲,扶曾孫出郡,悲泣不自勝。其惠化所感如此。百姓相率為立廟建碑,以紀其德。又詣京師求贈謚。詔贈湘州刺史。謚曰恭。



 史臣曰:高祖光有天下,慶命傍流,枝戚屬連,咸被任遇。蕭景之才辯識斷,益政佐時,蓋梁宗室令望者矣。



\end{pinyinscope}