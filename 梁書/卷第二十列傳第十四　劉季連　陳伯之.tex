\article{卷第二十列傳第十四 劉季連 陳伯之}

\begin{pinyinscope}

 劉季連,字惠續,彭城人也。父思考,以宋高祖族弟顯於宋世,位至金紫光祿大夫。季連有名譽,早歷清官。齊高帝受禪,悉誅宋室近屬,將及季連等,太宰褚淵素善之,固請乃免。建元中,季連為尚書左丞。永明初,出為江夏內史,累遷平南長沙內史,冠軍長史、廣陵太守,並行府州事。入為給事黃門侍郎,轉太子中庶子。建武中,又出
 為平西蕭遙欣長史、南郡太守。時明帝諸子幼弱,內親則仗遙欣兄弟,外親則倚后弟劉暄、內弟江祏。遙欣之鎮江陵也,意寄甚隆;而遙欣至州,多招賓客,厚自封殖,明帝甚惡之。季連族甥琅邪王會為遙欣諮議參軍,美容貌,頗才辯,遙欣遇之甚厚。會多所慠忽,於公座與遙欣競侮季連,季連憾之,乃密表明帝,稱遙欣有異迹。明帝納焉,乃以遙欣為雍州刺史。明帝心德季連,四年,以為輔國將軍、益州刺史,令據遙欣上流。季連父,宋世為益州,貪鄙無政績,州人猶以義故,善待季連。季連下車,存問故老,撫納新舊,見父時故吏,皆對之流涕。辟遂寧
 人龔愜為府主簿。愜,龔穎之孫,累世有學行,故引焉。



 東昏即位,永元元年,徵季連為右衛將軍,道斷不至。季連聞東昏失德,京師多故,稍自驕矜。本以文吏知名,性忌而褊狹,至是遂嚴愎酷狠,土人始懷怨望。其年九月,季連因聚會,發人丁五千人,聲以講武,遂遣中兵參軍宋買率之以襲中水。穰人李託豫知之,設備守險,買與戰不利,還州,郡縣多叛亂矣。是月,新城人趙續伯殺五城令,逐始平太守。十月,晉原人樂寶稱、李難當殺其太守,寶稱自號南秦州刺史,難當益州刺史。十二月,季連遣參軍崔茂祖率眾二千討之,齎三日糧。值歲大寒,群賊
 相聚,伐樹塞路,軍人水火無所得,大敗而還,死者十七八。明年正月,新城人帛養逐遂寧太守譙希淵。三月,巴西人雍道晞率群賊萬餘逼巴西,去郡數里,道晞稱鎮西將軍,號建義。巴西太守魯休烈與涪令李膺嬰城自守,季連遣中兵參軍李奉伯率眾五千救之。奉伯至,與郡兵破擒道晞,斬之涪市。奉伯因獨進巴西之東鄉討餘賊。李膺止之曰:「卒惰將驕,乘勝履險,非良策也。不如小緩,更思後計。」奉伯不納,悉眾入山,大敗而出,遂奔還州。六月,江陽人程延期反,殺太守何法藏。魯休烈懼不自保,奔投巴東相蕭慧訓。十月,巴西人趙續伯又反,有
 眾二萬,出廣漢,乘佛輿,以五綵裹青石,誑百姓云:「天與我玉印,當王蜀。」愚人從之者甚眾。季連進討之,遣長史趙越常前驅。兵敗,季連復遣李奉伯由涪路討之。奉伯別軍自潺亭與大軍會於城,進攻其柵,大破之。



 時會稽人石文安字守休,隱居鄉里,專行禮讓,代季連為尚書左丞,出為江夏內史,又代季連入為御史中丞,與季連相善。子仲淵字欽回,聞義師起,率鄉人以應高祖。天監初,拜郢州別駕,從高祖平京邑。



 明年春,遣左右陳建孫送季連弟通直郎子淵及季連二子使蜀,喻旨慰勞。季連受命,飭還裝。高祖以西臺將鄧元起為益州刺史。元起,
 南郡人。季連為南郡之時,素薄元起。典簽朱道琛者,嘗為季連府都錄,無賴小人,有罪,季連欲殺之,逃叛以免。至是說元起曰:「益州亂離已久,公私府庫必多秏失,劉益州臨歸空竭,豈辦復能遠遣候遞。道琛請先使檢校,緣路奉迎;不然,萬里資糧,未易可得。」元起許之。道琛既至,言語不恭,又歷造府州人士,見器物輒奪之,有不獲者,語曰:「會當屬人,何須苦惜。」於是軍府大懼,謂元起至必誅季連,禍及黨與,競言之於季連。季連亦以為然;又惡昔之不禮元起也,益憤懣。司馬朱士略說季連,求為巴西郡,留三子為質,季連許之。頃之,季連遂召佐史,矯
 稱齊宣德皇后令,聚兵復反,收朱道琛殺之。書報朱士略,兼召李膺。膺、士略並不受使。使歸,元起收兵於巴西以待之,季連誅士略三子。



 天監元年六月,元起至巴西,季連遣其將李奉伯等拒戰。兵交,互有得失,久之,奉伯乃敗退還成都。季連驅略居人,閉城固守。元起稍進圍之。是冬,季連城局參軍江希之等謀以城降,不果,季連誅之。蜀中喪亂已二年矣,城中食盡,升米三千,亦無所糴,餓死者相枕。其無親黨者,又殺而食之。季連食粥累月,飢窘無計。二年正月,高祖遣主書趙景悅宣詔降季連,季連肉袒請罪。元起遷季連于城外,俄而造焉,待之
 以禮。季連謝曰:「早知如此,豈有前日之事。」元起誅李奉伯並諸渠帥,送季連還京師。季連將發,人莫之視,惟龔愜送焉。



 初,元起在道,懼事不集,無以為賞,士之至者,皆許以辟命,於是受別駕、治中檄者,將二千人。季連既至,詣闕謝,高祖引見之。季連自東掖門入,數步一稽顙,以至高祖前。高祖笑謂曰:「卿欲慕劉備而曾不及公孫述,豈無臥龍之臣乎。」季連復稽顙謝。赦為庶人。四年正月,因出建陽門,為蜀人藺道恭所殺。季連在蜀,殺道恭父,道恭出亡,至是而報復焉。



 陳伯之,濟陰睢陵人也。幼有膂力。年十三四,好著獺皮
 冠,帶刺刀,候伺鄰里稻熟,輒偷刈之。嘗為田主所見,呵之云:「楚子莫動!」伯之謂田主曰:「君稻幸多,一擔何苦?」田主將執之,伯之因杖刀而進,將刺之,曰:「楚子定何如!」田主皆反走,伯之徐擔稻而歸。及年長,在鐘離數為劫盜,嘗授面覘人船,船人斫之,獲其左耳。後隨鄉人車騎將軍王廣之,廣之愛其勇,每夜臥下榻,征伐嘗自隨。



 齊安陸王子敬為南兗州,頗持兵自衛。明帝遣廣之討子敬,廣之至歐陽,遣伯之先驅,因城開,獨入斬子敬。又頻有戰功,以勛累遷為冠軍將軍、驃騎司馬,封魚復縣伯,邑五百戶。



 義師起,東昏假伯之節、督前驅諸軍事、豫州刺
 史,將軍如故。尋轉江州,據尋陽以拒義軍。郢城平,高祖得伯之幢主蘇隆之,使說伯之,即以為安東將軍、江州刺史。伯之雖受命,猶懷兩端,偽云「大軍未須便下」。高祖謂諸將曰:「伯之此答,其心未定,及其猶豫,宜逼之。」眾軍遂次尋陽,伯之退保南湖,然後歸附。進號鎮南將軍,與眾俱下。伯之頓籬門,尋進西明門。建康城未平,每降人出,伯之輒喚與耳語。高祖恐其復懷翻覆,密語伯之曰:「聞城中甚忿卿舉江州降,欲遣刺客中卿,宜以為慮。」伯之未之信。會東昏將鄭伯倫降,高祖使過伯之,謂曰:「城中甚忿卿,欲遣信誘卿以封賞。須卿復降,當生割卿手腳;
 卿若不降,復欲遣刺客殺卿。宜深為備。」伯之懼,自是無異志矣。力戰有功。城平,進號征南將軍,封豊城縣公,邑二千戶,遣還之鎮。



 伯之不識書,及還江州,得文牒辭訟,惟作大諾而已。有事,典簽傳口語,與奪決於主者。



 伯之與豫章人鄧繕、永興人戴永忠並有舊,繕經藏伯之息英免禍,伯之尤德之。及在州,用繕為別駕,永忠記室參軍。河南褚緭,京師之薄行者,齊末為揚州西曹,遇亂居閭里;而輕薄互能自致,惟緭獨不達。高祖即位,緭頻造尚書范雲,雲不好緭,堅距之。緭益怒,私語所知曰:「建武以後,草澤底下,悉化成貴人,吾何罪而見棄。今天下草
 創,饑饉不已,喪亂未可知。陳伯之擁彊兵在江州,非代來臣,有自疑意;且熒惑守南斗,詎非為我出。今者一行,事若無成,入魏,何遽減作河南郡。」於是遂投伯之書佐王思穆,事之,大見親狎。及伯之鄉人朱龍符為長流參軍,並乘伯之愚闇,恣行姦險,刑政通塞,悉共專之。



 伯之子虎牙,時為直閣將軍,高祖手疏龍符罪,親付虎牙,虎牙封示伯之;高祖又遣代江州別駕鄧繕,伯之並不受命。答高祖曰:「龍符驍勇健兒,鄧繕事有績效,臺所遣別駕,請以為治中。」繕於是日夜說伯之云:「臺家府庫空竭,復無器仗,三倉無米,東境饑流,此萬代一時也,機不可
 失。」緭、永忠等每贊成之。伯之謂繕:「今段啟卿,若復不得,便與卿共下使反。」高祖敕部內一郡處繕,伯之於是集府州佐史謂曰:「奉齊建安王教,率江北義勇十萬,已次六合,見使以江州見力運糧速下。我荷明帝厚恩,誓死以報。今便纂嚴備辦。」使緭詐為蕭寶夤書,以示僚佐。於廳事前為壇,殺牲以盟。伯之先飲,長史已下次第歃血。緭說伯之曰:「今舉大事,宜引眾望,程元沖不與人同心;臨川內史王觀,僧虔之孫,人身不惡,便可召為長史,以代元沖。」伯之從之。仍以緭為尋陽太守,加討逆將軍;永忠輔義將軍;龍符為豫州刺史,率五百人守大雷。大雷
 戍主沈慧休,鎮南參軍李延伯。又遣鄉人孫鄰、李景受龍符節度,鄰為徐州,景為郢州。豫章太守鄭伯倫起郡兵距守。程元沖既失職,於家合率數百人,使伯之典簽呂孝通、戴元則為內應。伯之每旦常作伎,日晡輒臥,左右仗身皆休息。元沖因其解弛,從北門入,徑至廳事前。伯之聞叫聲,自率出盪,元沖力不能敵,走逃廬山。



 初,元沖起兵,要尋陽張孝季,孝季從之。既敗,伯之追孝季不得,得其母郎氏,蠟灌殺之。遣信還都報虎牙兄弟,虎牙等走盱眙,盱眙人徐安、莊興紹、張顯明邀擊之,不能禁,反見殺。高祖遣王茂討伯之。伯之聞茂來,謂緭等曰:「王
 觀既不就命,鄭伯倫又不肯從,便應空手受困。今先平豫章,開通南路,多發丁力,益運資糧,然後席卷北向,以撲饑疲之眾,不憂不濟也。」乃留鄉人唐蓋人守城,遂相率趣豫章。太守鄭伯倫堅守,伯之攻之不能下。王茂前軍既至,伯之表裏受敵,乃敗走,間道亡命出江北,與子虎牙及褚緭俱入魏。魏以伯之為使持節、散騎常侍、都督淮南諸軍事、平南將軍、光祿大夫、曲江縣侯。



 天監四年,詔太尉、臨川王宏率眾軍北討,宏命記室丘遲私與伯之書曰:陳將軍足下無恙,幸甚。將軍勇冠三軍,才為世出。棄燕雀之小志,慕鴻鵠以高翔。昔因機變化,遭逢
 明主,立功立事,開國承家,朱輪華轂,擁旄萬里,何其壯也!如何一旦為奔亡之虜,聞鳴鏑而股戰,對穹廬以屈膝,又何劣耶?尋君去就之際,非有他故,直以不能內審諸己,外受流言,沉迷猖蹶,以至於此。聖朝赦罪論功,棄瑕錄用,收赤心於天下,安反側於萬物,將軍之所知,非假僕一二談也。朱鮪涉血於友于,張繡倳刃於愛子,漢主不以為疑,魏君待之若舊。況將軍無昔人之罪,而勳重於當世。



 夫迷塗知反,往哲是與;不遠而復,先典攸高。主上屈法申恩,吞舟是漏。將軍松柏不剪,親戚安居;高臺未傾,愛妾尚在。悠悠爾心,亦何可述。今功臣名將,雁
 行有序。懷黃佩紫,贊帷幄之謀;乘軺建節,奉疆埸之任。並刑馬作誓,傳之子孫。將軍獨靦顏借命,驅馳異域,寧不哀哉!



 夫以慕容超之強,身送東市;姚泓之盛,面縛西都。故知霜露所均,不育異類;姬漢舊邦,無取雜種。北虜僭盜中原,多歷年所,惡積禍盈,理至燋爛。況偽孽昏狡,自相夷戮,部落攜離,酋豪猜貳,方當繫頸蠻邸,懸首槁街。而將軍魚游於沸鼎之中,燕巢於飛幕之上,不亦惑乎!



 暮春三月,江南草長,雜花生樹,群鶯亂飛。見故國之旗鼓,感平生於疇日,撫弦登陴,豈不愴恨。所以廉公之思趙將,吳子之泣西河,人之情也。將軍獨無情哉!想早
 勵良圖,自求多福。



 伯之乃於壽陽擁眾八千歸。虎牙為魏人所殺。伯之既至,以為使持節、都督西豫州諸軍事、平北將軍、西豫州刺史,永新縣侯,邑千戶。未之任,復以為通直散騎常侍、驍騎將軍,又為太中大夫。久之,卒於家。其子猶有在魏者。



 褚緭在魏,魏人欲擢用之。魏元會,緭戲為詩曰:「帽上著籠冠,褲上著硃衣,不知是今是,不知非昔非。」魏人怒,出為始平太守。日日行獵,墮馬死。



 史臣曰:劉季連之文吏小節,而不能以自保全,習亂然也。陳伯之小人而乘君子之器,群盜又誣而奪之,安能長久矣。



\end{pinyinscope}