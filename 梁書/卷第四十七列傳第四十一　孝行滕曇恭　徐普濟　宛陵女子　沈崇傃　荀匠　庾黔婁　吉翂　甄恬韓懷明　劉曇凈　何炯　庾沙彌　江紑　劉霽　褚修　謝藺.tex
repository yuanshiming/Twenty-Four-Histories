\article{卷第四十七列傳第四十一 孝行滕曇恭 徐普濟 宛陵女子 沈崇傃 荀匠 庾黔婁 吉翂 甄恬韓懷明 劉曇凈 何炯 庾沙彌 江紑 劉霽 褚修 謝藺}

\begin{pinyinscope}

 經云:「夫孝,德之本也。」此生民之為大,有國之所先歟!高祖創業開基,飭躬化俗,澆弊之風以革,孝治之術斯著。
 每發絲綸,遠加旌表。而淳和比屋,罕要詭俗之譽,潛晦成風,俯列踰群之迹,彰於視聽,蓋無幾焉。今採綴以備遺逸云爾。



 滕曇恭,豫章南昌人也。年五歲,母楊氏患熱,思食寒瓜,土俗所不產。曇恭歷訪不能得,銜悲哀切。俄值一桑門問其故,曇恭具以告。桑門曰:「我有兩瓜,分一相遺。」曇恭拜謝,因捧瓜還,以薦其母。舉室驚異。尋訪桑門,莫知所在。及父母卒,曇恭水漿不入口者旬日,感慟嘔血,絕而復蘇。隆冬不著繭絮,蔬食終身。每至忌日,思慕不自堪,晝夜哀慟。其門外有冬生樹二株,時忽有神光自樹而
 起,俄見佛像及夾侍之儀,容光顯著,自門而入。曇恭家人大小,咸共禮拜,久之乃滅,遠近道俗咸傳之。太守王僧度引曇恭為功曹,固辭不就。王儉時隨僧度在郡,號為滕曾子。天監元年,陸璉奉使巡行風俗,表言其狀。曇恭有子三人,皆有行業。



 時有徐普濟者,長沙臨湘人。居喪未及葬,而鄰家火起,延及其舍,普濟號慟伏棺上,以身蔽火。鄰人往救之,焚炙已悶絕,累日方蘇。



 宣城宛陵有女子與母同床寢,母為猛虎所搏,女號叫拿虎,虎毛盡落,行十數里,虎乃棄之。女抱母還,猶有氣,經時乃絕。太守蕭琛賻焉,表言其狀。有詔旌其門閭。



 沈崇傃,字思整,吳興武康人也。父懷明,宋兗州刺史。崇傃六歲丁父憂,哭踴過禮。及長,傭書以養母焉。齊建武初,起家為奉朝請。永元末,遷司徒行參軍。天監初,為前軍鄱陽王參軍事。三年,太守柳惲辟為主簿。崇傃從惲到郡,還迎其母,母卒。崇傃以不及侍疾,將欲致死,水漿不入口,晝夜號哭,旬日殆將絕氣。兄弟謂之曰:「殯葬未申,遽自毀滅,非全孝之道也。」崇傃之瘞所,不避雨雪,倚墳哀慟。每夜恒有猛獸來望之,有聲狀如歎息者。家貧無以遷窆,乃行乞經年,始獲葬焉。既而廬于墓側,自以初行喪禮不備,復以葬後更治服三年。久食麥屑,不啖
 鹽酢,坐臥於單薦,因虛腫不能起。郡縣舉其至孝。高祖聞,即遣中書舍人慰勉之,乃下詔曰:「前軍沈崇傃,少有志行,居喪踰禮。齋制不終,未得大葬,自以行乞淹年,哀典多闕,方欲以永慕之晨,更為再期之始。雖即情可矜,禮有明斷。可便令除釋,擢補太子洗馬。旌彼門閭,敦茲風教。」崇傃奉詔釋服,而涕泣如居喪,固辭不受官,苦自陳讓,經年乃得為永寧令。自以祿不及養,怛恨愈甚,哀思不自堪,至縣卒,時年三十九。



 荀匠,字文師,潁陰人,晉太保勖九世孫也。祖瓊,年十五,復父仇於成都市,以孝聞。宋元嘉末,渡淮赴武陵王義,
 為元凶追兵所殺,贈員外散騎侍郎。父法超,齊中興末為安復令,卒於官。凶問至,匠號慟氣絕,身體皆冷,至夜乃蘇。既而奔喪,每宿江渚,商旅皆不忍聞其哭聲。服未闋,兄斐起家為鬱林太守,徵俚賊,為流矢所中,死於陣。喪還,匠迎于豫章,望舟投水,傍人赴救,僅而得全。既至,家貧不得時葬。居父憂並兄服,歷四年不出廬戶。自括髮後,不復櫛沐,髮皆禿落。哭無時,聲盡則係之以泣,目眥皆爛,形體枯悴,皮骨裁連,雖家人不復識。郡縣以狀言,高祖詔遣中書舍人為其除服,擢為豫章王國左常侍。匠雖即吉,毀悴逾甚。外祖孫謙誡之曰:「主上以孝治
 天下,汝行過古人,故發明詔,擢汝此職。非唯君父之命難拒,故亦揚名後世,所顯豈獨汝身哉!」匠於是乃拜。竟以毀卒於家,時年二十一。



 庾黔婁,字子貞,新野人也。父易,司徒主簿,征不至,有高名。



 黔婁少好學,多講誦《孝經》,未嘗失色於人,南陽高士劉虯、宗測並歎異之。起家本州主簿,遷平西行參軍。出為編令,治有異績。先是,縣境多虎暴。黔婁至,虎皆渡往臨沮界,當時以為仁化所感。齊永元初,除孱陵令,到縣未旬,易在家遘疾,黔婁忽然心驚,舉身流汗,即日棄官歸家,家人悉驚其忽至。時易疾始二日,醫云:「欲知差劇,
 但嘗糞甜苦。」易泄痢,黔婁輒取嘗之,味轉甜滑,心逾憂苦。至夕,每稽顙北辰,求以身代。俄聞空中有聲曰:「徵君壽命盡,不復可延,汝誠禱既至,止得申至月末。」及晦而易亡,黔婁居喪過禮,廬於冢側。和帝即位,將起之,鎮軍蕭穎胄手書敦譬,黔婁固辭。服闋,除西臺尚書儀曹郎。



 梁臺建,鄧元起為益州刺史,表黔婁為府長史、巴西、梓潼二郡太守。及成都平,城中珍寶山積,元起悉分與僚佐,惟黔婁一無所取。元起惡其異眾,厲聲曰:「長史何獨爾為!」黔婁示不違之,請書數篋。尋除蜀郡太守,在職清素,百姓便之。元起死于蜀,部曲皆散,黔婁身營殯殮,攜
 持喪柩歸鄉里。還為尚書金部郎,遷中軍表記室參軍。東宮建,以本官侍皇太子讀,甚見知重,詔與太子中庶子殷鈞、中舍人到洽、國子博士明山賓等,遞日為太子講《五經》義。遷散騎侍郎、荊州大中正。卒,時年四十六。



 吉翂,字彥霄,馮翊蓮勺人也。世居襄陽。翂幼有孝性。年十一,遭所生母憂,水漿不入口,殆將滅性,親黨異之。天監初,父為吳興原鄉令,為姦吏所誣,逮詣廷尉。翂年十五,號泣衢路,祈請公卿,行人見者,皆為隕涕。其父理雖清白,恥為吏訊,乃虛自引咎,罪當大辟。翂乃撾登聞鼓,乞代父命。高祖異之,敕廷尉卿蔡法度曰:「吉翂請死贖
 父,義誠可嘉;但其幼童,未必自能造意。卿可嚴加脅誘,取其款實。」法度受敕還寺,盛陳徽纏,備列官司,厲色問翂曰:「爾求代父死,敕已相許,便應伏法。然刀鋸至劇,審能死不?且爾童孺,志不及此,必為人所教。姓名是誰,可具列答。若有悔異,亦相聽許。」翂對曰:「囚雖蒙弱,豈不知死可畏憚?顧諸弟稚藐,唯囚為長,不忍見父極刑,自延視息。所以內斷胸臆,上干萬乘。今欲殉身不測,委骨泉壤,此非細故,奈何受人教邪!明詔聽代,不異登仙,豈有回貳!」法度知翂至心有在,不可屈撓,乃更和顏誘語之曰:「主上知尊侯無罪,行當釋亮。觀君神儀明秀,足稱佳
 童,今若轉辭,幸父子同濟。奚以此妙年,苦求湯鑊?」翂對曰:「凡鯤鮞螻蟻,尚惜其生;況在人斯,豈願齏粉?但囚父掛深劾,必正刑書,故思殞仆,冀延父命。今瞑目引領,以聽大戮,情殫意極,無言復對。」翂初見囚,獄掾依法備加桎梏;法度矜之,命脫其二械,更令著一小者。翂弗聽,曰:「翂求代父死,死罪之囚,唯宜增益,豈可減乎?」竟不脫械。法度具以奏聞,高祖乃宥其父。丹陽尹王志求其在廷尉故事,并請鄉居,欲於歲首,舉充純孝之選。翂曰:「異哉王尹,何量翂之薄乎!夫父辱子死,斯道固然。若翂有靦面目,當其此舉,則是因父買名,一何甚辱!」拒之而止。年
 十七,應辟為本州主簿。出監萬年縣,攝官期月,風化大行。自雍還至郢,湘州刺史柳悅復召為主簿。後鄉人裴儉、丹陽尹丞臧盾、揚州中正張仄連名薦翂,以為孝行純至,明通《易》、《老》。敕付太常旌舉。初,翂以父陷罪,因成悸疾,後因發而卒。



 甄恬,字彥約,中山無極人也,世居江陵。祖欽之,長寧令。父標之,州從事。恬數歲喪父,哀感有若成人。家人矜其小,以肉汁和飯飼之,恬不肯食。年八歲,問其母,恨生不識父,遂悲泣累日,忽若有見,言其形貌,則其父也,時以為孝感。家貧,養母常得珍羞。及居喪,廬於墓側,恆有鳥
 玄黃雜色,集於廬樹,恬哭則鳴,哭止則止。又有白雀栖宿其廬。州將始興王憺表其行狀。詔曰:「朕虛己欽賢,寤寐盈想。詔彼群岳,務盡搜揚。恬既孝行殊異,聲著邦壤,敦風厲俗,弘益茲多。牧守騰聞,義同親覽。可旌表室閭,加以爵位。」恬官至安南行參軍。



 韓懷明,上黨人也,客居荊州。年十歲,母患屍疰,每發輒危殆。懷明夜於星下稽顙祈禱,時寒甚切,忽聞香氣,空中有人語曰:「童子母須臾永差,無勞自苦。」未曉,而母豁然平復。鄉里異之。十五喪父,幾至滅性,負土成墳,贈助無所受。免喪,與鄉人郭瑀俱師事南陽劉虯。虯嘗一日
 廢講,獨居涕泣。懷明竊問其故,虯家人答云:「是外祖亡日。」時虯母亦亡矣。懷明聞之,即日罷學,還家就養。虯歎曰:「韓生無虞丘之恨矣。」家貧,常肆力以供甘脆,嬉怡膝下,朝夕不離母側。母年九十一,以壽終,懷明水漿不入口一旬,號哭不絕聲。有雙白鳩巢其廬上,字乳馴狎,若家禽焉,服釋乃去。既除喪,蔬食終身,衣衾無改。天監初,刺史始興王憺表言之。州累辟不就,卒于家。



 劉曇凈,字元光,彭城莒人也。祖元真,淮南太守,居郡得罪;父慧鏡,歷詣朝士乞哀,懇惻甚至,遂以孝聞。曇凈篤行有父風。解褐安成王國左常侍。父卒於郡,曇凈奔喪,
 不食飲者累日,絕而又蘇。每哭輒嘔血。服闋,因毀瘠成疾。會有詔,士姓各舉四科,曇凈叔父慧斐舉以應孝行,高祖用為海寧令。曇凈以兄未為縣,因以讓兄,乃除安西行參軍。父亡後,事母尤淳至,身營飧粥,不以委人。母疾,衣不解帶。及母亡,水漿不入口者殆一旬。母喪,權瘞藥王寺。時天寒,曇凈身衣單布,廬於瘞所,晝夜哭泣不絕聲,哀感行路,未及期而卒。



 何炯,字士光,廬江灊人也。父撙,太中大夫。炯年十五,從兄胤受業,一期並通《五經》章句。炯白皙,美容貌,從兄求、點每稱之曰:「叔寶神清,弘治膚清。今觀此子,復見衛、杜
 在目。」炯常慕恬退,不樂進仕。從叔昌珝謂曰:「求、點皆已高蹈,爾無宜復爾。且君子出處,亦各一途。」年十九,解褐揚州主簿。舉秀才,累遷王府行參軍、尚書兵、庫部二曹郎。出為永康令,以和理稱。還為仁威南康王限內記室,遷治書侍御史。以父疾經旬,衣不解帶,頭不櫛沐,信宿之間,形貌頓改。及父卒,號慟不絕聲,枕塊藉地,腰虛腳腫,竟以毀卒。



 庾沙彌,潁陰人也。晉司空冰六世孫。父佩玉,輔國長史、長沙內史,宋昇明中坐沈攸之事誅,沙彌時始生。年至五歲,所生母為製采衣,輒不肯服。母問其故,流涕對曰:「
 家門禍酷,用是何為!」既長,終身布衣蔬食。起家臨川王國左常侍,遷中軍田曹行參軍。嫡母劉氏寢疾,沙彌晨昏侍側,衣不解帶,或應鍼灸,輒以身先試之。及母亡,水漿不入口累日,終喪不解衰絰,不出廬戶,晝夜號慟,鄰人不忍聞。墓在新林,因有旅松百餘株,自生墳側。族兄都官尚書詠表言其狀,應純孝之舉,高祖召見嘉之,以補歙令。還除輕車邵陵王參軍事,隨府會稽,復丁所生母憂。喪還都,濟浙江,中流遇風,舫將覆沒,沙彌抱柩號哭,俄而風靜,蓋孝感所致。服闋,除信威刑獄參軍,兼丹陽郡囗囗囗累遷寧遠錄事參軍,轉司馬。出為長城令,
 卒。



 江紑,字含潔,濟陽考城人也。父蒨,光祿大夫。紑幼有孝性。年十三,父患眼,紑侍疾將期月,衣不解帶。夜夢一僧云:「患眼者,飲慧眼水必差。」及覺說之,莫能解者。紑第三叔祿與草堂寺智者法師善,往訪之。智者曰:「《無量壽經》云:慧眼見真,能渡彼岸。」蒨乃因智者啟捨同夏縣界牛屯里舍為寺,乞賜嘉名。敕答云:「純臣孝子,往往感應。晉世顏含,遂見冥中送藥。近見智者,知卿第二息感夢,云飲慧眼水。慧眼則是五眼之一號,若欲造寺,可以慧眼為名。」及就創造,泄故井,井水清冽,異於常泉。依夢取水
 洗眼及煮藥,稍覺有瘳,因此遂差。時人謂之孝感。南康王為南州,召為迎主簿。紑性靜,好《老》、《莊》玄言,尤善佛義,不樂進仕。及父卒,紑廬于墓,終日號慟不絕聲,月餘卒。



 劉霽,字士烜,平原人也。祖乘民,宋冀州刺史。父聞慰,齊工員郎。霽年九歲,能誦《左氏傳》,宗黨咸異之。十四居父憂,有至性,每哭輒嘔血。家貧,與弟杳、高相篤勵學。既長,博涉多通。天監中,起家奉朝請,稍遷宣惠晉安王府參軍,兼限內記室,出補西昌相。入為尚書主客侍郎。未期,除海鹽令。霽前後宰二邑,並以和理著稱。還為建康正,非所好。頃之,以疾免。尋除建康令,不拜。母明氏寢疾,霽
 年已五十,衣不解帶者七旬,誦《觀世音經》,數至萬遍,夜因感夢,見一僧謂曰:「夫人算盡,君精誠篤至,當相為申延。」後六十餘日乃亡。霽廬于墓,哀慟過禮。常有雙白鶴馴翔廬側。處士阮孝緒致書抑譬,霽思慕不已,服未終而卒,時年五十二。著《釋俗語》八卷,文集十卷。弟杳在《文學傳》,高在《處士傳》。



 褚脩,吳郡錢唐人也。父仲都,善《周易》,為當時最。天監中,歷官《五經》博士。脩少傳父業,兼通《孝經》、《論語》,善尺牘,頗解文章。初為湘東王國侍郎,稍遷輕車湘東府行參軍,並兼國子助教。武陵王為揚州,引為宣惠參軍、限內記
 室。修性至孝,父喪毀瘠過禮,因患冷氣。及丁母憂,水漿不入口二十三日,氣絕復蘇,每號慟嘔血,遂以毀卒。



 謝藺,字希如,陳郡陽夏人也。晉太傅安八世孫。父經,中郎諮議參軍。藺五歲,每父母未飯,乳媼欲令藺先飯,藺曰:「既不覺飢。」彊食終不進。舅阮孝緒聞之,歎曰:「此兒在家則曾子之流,事君則藺生之匹。」因名之曰藺。稍受以經史,過目便能諷誦。孝緒每曰「吾家陽元也」。及丁父憂,晝夜號慟,毀瘠骨立,母阮氏常自守視譬抑之。服闋後,吏部尚書蕭子顯表其至行,擢為王府法曹行參軍,累遷外兵記室參軍。時甘露降士林館,藺獻頌,高祖嘉之,
 因有詔使製《北兗州刺史蕭楷德政碑》,又奉令製《宣城王奉述中庸頌》。太清元年,遷散騎侍郎,兼散騎常侍,使於魏。會侯景舉地入附,境上交兵,藺母慮不得還,感氣卒。及藺還入境,爾夕夢不祥,旦便投劾馳歸。既至,號慟嘔血,氣絕久之,水漿不入口。親友慮其不全,相對悲慟,彊勸以飲粥。藺初勉彊受之,終不能進,經月餘日,因夜臨而卒,時年三十八。藺所製詩賦碑頌數十篇。



 史臣曰:孔子稱「毀不滅性」,教民無以死傷生也,故制喪紀,為之節文。高柴、仲由伏膺聖教,曾參、閔損虔恭孝道,或水漿不入口,泣血終年,豈不知創鉅痛深,《蓼莪》慕切?
 所謂先王制禮,賢者俯就。至如丘、吳,終於毀滅。若劉曇凈、何炯、江紑、謝藺者,亦二子之志歟。



\end{pinyinscope}