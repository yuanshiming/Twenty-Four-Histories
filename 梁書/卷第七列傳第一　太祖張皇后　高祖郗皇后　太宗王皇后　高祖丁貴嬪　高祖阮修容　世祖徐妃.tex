\article{卷第七列傳第一 太祖張皇后 高祖郗皇后 太宗王皇后 高祖丁貴嬪 高祖阮修容 世祖徐妃}

\begin{pinyinscope}

 《易》曰:「有天地然後有萬物,有萬物然後有男女,有男女然後有夫婦。」夫婦之義尚矣哉!周禮,王者立后六宮,三夫人、九嬪、二十七世婦、八十一御妻,以聽天下之內治。故《昏義》云:「天子之與后,猶日之與月,陰之與陽,相須而成者也。」漢初因秦稱號,帝母稱皇太后,后稱皇后,而加
 以美人、良人、八子、七子之屬。至孝武制婕妤之徒凡十四等。降及魏、晉,母后之號,皆因漢法;自夫人以下,世有增損焉。高祖撥亂反正,深鑒奢逸,惡衣菲食,務先節儉。配德早終,長秋曠位,嬪嬙之數,無所改作。太宗、世祖出自儲籓,而妃並先殂,又不建椒閫。今之撰錄,止備闕云。



 太祖獻皇后張氏,諱尚柔,范陽方城人也。祖次惠,宋濮陽太守。后母蕭氏,即文帝從姑。后,宋元嘉中嬪於文帝,生長沙宣武王懿、永陽昭王敷,次生高祖。



 初,后嘗於室內,忽見庭前昌蒲生花,光彩照灼,非世中所有。后驚視,謂侍者曰:「汝見不?」對曰:「不見。」后曰:「嘗聞見者當富貴。」因
 遽取吞之。是月產高祖。將產之夜,后見庭內若有衣冠陪列焉。次生衡陽宣王暢、義興昭長公主令。宋泰始七年,殂于秣陵縣同夏里舍,葬武進縣東城里山。天監元年五月甲辰,追上尊號為皇后。謚曰獻。



 父穆之,字思靜,晉司空華六世孫。曾祖輿坐華誅,徙興古,未至召還。及過江,為丞相掾,太子舍人。穆之少方雅,有識鑒。宋元嘉中,為員外散騎侍郎。與吏部尚書江湛、太子左率袁淑善,淑薦之於始興王濬,浚深引納焉。穆之鑒其禍萌,思違其難,言於湛求外出。湛將用為東縣,固乞遠郡,久之,得為寧遠將軍、交址太守。治有異績。會刺史死,交土
 大亂,穆之威懷循拊,境內以寧。宋文帝聞之嘉焉,將以為交州刺史,會病卒。子弘籍,字真藝,齊初為鎮西參軍,卒於官。高祖踐阼,追贈穆之光祿大夫,加金章。又詔曰:「亡舅齊鎮西參軍,素風雅猷,夙肩名輩,降年不永,早世潛輝。朕少離苦辛,情地彌切,雖宅相克成,輅車靡贈,興言永往,觸目慟心。可追贈廷尉卿。」弘籍無子,從父弟弘策以第三子纘為嗣,別有傳。



 高祖德皇后郗氏,諱徽,高平金鄉人也。祖紹,國子祭酒,領東海王師。父燁,太子舍人,早卒。



 初,后母尋陽公主方娠,夢當生貴子。及生后,有赤光照于室內,器物盡明,家
 人皆怪之。巫言此女光採異常,將有所妨,乃於水濱祓除之。



 后幼而明慧,善隸書,讀史傳。女工之事,無不閑習。宋後廢帝將納為后;齊初,安陸王緬又欲婚:郗氏並辭以女疾,乃止。建元未,高祖始娉焉。生永興公主玉姚,永世公主玉婉,永康公主玉嬛。



 建武五年,高祖為雍州刺史,先之鎮,後乃迎后。至州未幾,永元元年八月殂于襄陽官舍,時年三十二。其年歸葬南徐州南東海武進縣東城里山。中興二年,齊朝進高祖位相國,封十郡,梁公,詔贈后為梁公妃。高祖踐阼,追崇為皇后。有司議謚,吏部尚書兼右僕射臣約議曰:「表號垂名,義昭不朽。先皇
 后應祥月德,比載坤靈,柔範陰化,儀形自遠。伣天作合,義先造舟,而神猷夙掩,所隔升運。宜式遵景行,用昭大典。謹按《謚法》,忠和純備曰德,貴而好禮曰德。宜崇曰德皇后。」詔從之。陵曰脩陵。



 后父燁,詔贈金紫光祿大夫。燁尚宋文帝女尋陽公主,齊初降封松滋縣君。燁子泛,中軍臨川王記室參軍。



 太宗簡皇后王氏,諱靈賓,瑯邪臨沂人也。祖儉,太尉、南昌文憲公。



 后幼而柔明淑德,叔父暕見之曰:「吾家女師也。」天監十一年,拜晉安王妃。生哀太子大器,南郡王大連,長山公主妙紘。中大通三年十月,拜皇太子妃。太清三
 年三月,薨於永福省,時年四十五。其年,太宗即位,追崇為皇后,謚曰簡。大寶元年九月,葬莊陵。先是詔曰:「簡皇后窀穸有期。昔西京霸陵,因山為藏;東漢壽陵,流水而已。朕屬值時艱,歲饑民弊,方欲以身率下,永示敦朴。今所營莊陵,務存約儉。」又詔金紫光祿大夫蕭子範為哀策文。



 父騫,字思寂,本名玄成,與齊高帝偏諱同,故改焉。以公子起家員外郎,遷太子洗馬,襲封南昌縣公,出為義興太守。還為驃騎諮議,累遷黃門郎,司徒右長史。性凝簡,不狎當世。嘗從容謂諸子曰:「吾家門戶,所謂素族,自可隨流平進,不須茍求也。」永元末,遷侍中,不拜。高祖
 霸府建,引為大司馬諮議參軍,俄遷侍中,領越騎校尉。



 高祖受禪,詔曰:「庭堅世祀,靡輟於宗周,樂毅錫壤,乃昭於洪漢。齊故太尉南昌公,含章履道,草昧興齊,謨明翊贊,同符在昔。雖子房之蔚為帝師,文若之隆比王佐,無以尚也。朕膺歷受圖,惟新寶命,莘莘玉帛,升降有典。永言前代,敬惟徽烈,匪直懋勳,義兼懷樹。可降封南昌公為侯,食邑千戶。」騫襲爵,遷度支尚書。天監四年,出為東陽太守,尋徙吳郡。八年,入為太府卿,領後軍將軍,遷太常卿。十一年,遷中書令,加員外散騎常侍。



 時高祖於鐘山造大愛敬寺,騫舊墅在寺側,有良田八十餘頃,即
 晉丞相王導賜田也。高祖遣主書宣旨就騫求市,欲以施寺。騫答旨云:「此田不賣;若是敕取,所不敢言。」酬對又脫略。高祖怒,遂付市評田價,以直逼還之。由是忤旨,出為吳興太守。在郡臥疾不視事。徵還,復為度支尚書,加給事中,領射聲校尉。以母憂去職。



 普通三年十月卒,時年四十九。詔贈侍中、金紫光祿大夫,謚曰安。子規襲爵,別有傳。



 高祖丁貴嬪,諱令光,譙國人也,世居襄陽。貴嬪生于樊城,有神光之異,紫煙滿室,故以「光」為名。相者云:「此女當大貴。」高祖臨州,丁氏因人以聞。貴嬪時年十四,高祖納
 焉。初,貴嬪生而有赤痣在左臂,治之不滅,至是無何忽失所在。事德皇后小心祗敬,嘗於供養經案之側,仿佛若見神人,心獨異之。



 高祖義師起,昭明太子始誕育,貴嬪與太子留在州城。京邑平,乃還京都。天監元年五月,有司奏為貴人,未拜;其年八月,又為貴嬪,位在三夫人上,居於顯陽殿。及太子定位,有司奏曰:禮,母以子貴。皇儲所生,不容無敬。宋泰豫元年六月,議百官以吏敬敬帝所生陳太妃,則宋明帝在時,百官未有敬。臣竊謂「母以子貴」,義著《春秋》。皇太子副貳宸極,率土咸執吏禮,既盡禮皇儲,則所生不容無敬。但帝王妃嬪,義與外隔,以
 理以例,無致敬之道也。今皇太子聖睿在躬,儲禮夙備,子貴之道,抑有舊章。王侯妃主常得通信問者,及六宮三夫人雖與貴嬪同列,並應以敬皇太子之禮敬貴嬪。宋元嘉中,始興、武陵國臣並以吏敬敬所生潘淑妃、路淑媛。貴嬪於宮臣雖非小君,其義不異,與宋泰豫朝議百官以吏敬敬帝所生,事義正同。謂宮閹施敬宜同吏禮,詣神虎門奉箋致謁;年節稱慶,亦同如此。婦人無閫外之事,賀及問訊箋什,所由官報聞而已。夫婦人之道,義無自專,若不仰繫於夫,則當俯繫於子。榮親之道,應極其所榮,未有子所行而所從不足者也。故《春秋》凡王
 命為夫人,則禮秩與子等。列國雖異於儲貳,而從尊之義不殊。前代依准,布在舊事。貴嬪載誕元良,克固大業,禮同儲君,實惟舊典。尋前代始置貴嬪,位次皇后,爵無所視;其次職者,位視相國,爵比諸侯王。此貴嬪之禮,已高朝列;況母儀春宮,義絕常算。且儲妃作配,率由盛則;以婦踰姑,彌乖從序。謂貴嬪典章,太子不異。



 於是貴嬪備典章,禮數同于太子,言則稱令。



 貴嬪性仁恕,及居宮內,接馭自下,皆得其歡心。不好華飾,器服無珍麗,未嘗為親戚私謁。及高祖弘佛教,貴嬪奉而行之,屏絕滋腴,長進蔬膳。受戒日,甘露降于殿前,方一丈五尺。高祖所
 立經義,皆得其指歸。尤精《凈名經》。所受供賜,悉以充法事。



 普通七年十一月庚辰薨,殯於東宮臨雲殿,年四十二。詔吏部郎張纘為哀策文曰:塗既啟,桂樽虛凝,龍帷已薦,象服將升。皇帝傷璧臺之永颭,悼曾城之不踐,罷鄉歌乎燕樂,廢徹齊於祀典。《風》有《采蘩》,化行南國,爰命史臣,俾流嬪德。其辭曰:軒緯之精,江漢之英;歸于君袂,生此離明。誕自厥初,時維載育;樞電繞郊,神光照屋。爰及待年,含章早穆;聲被洽陽,譽宣中谷。龍德在田,聿恭茲祀;陰化代終,王風攸始。動容諮式,出言顧史;宜其家人,刑于國紀。膺斯眷命,從此宅心;狄綴采珩,珮動雅音。
 日中思戒,月滿懷箴;如何不局,天高照臨。玄紞莫脩,褘章早缺;成物誰能,芳猷有烈。素魄貞明,紫宮照晰;逮下靡傷,思賢罔蔽。躬儉則節,昭事惟虔;金玉無玩,筐筥不捐。祥流德化,慶表親賢;甄昌軼啟,孕魯陶燕。方論婦教,明章閫席;玄池早扃,湘沅已穸。展衣委華,朱幩寢迹;慕結儲闈,哀深蕃辟。嗚呼哀哉!



 令龜兆良,葆引遷祖;具僚次列,承華接武。日杳杳以霾春,風淒淒而結緒;去曾掖以依遲,飾新宮而延佇。嗚呼哀哉!



 啟丹旗之星璟,振容車之黼裳;擬靈金而鬱楚,泛悽管而凝傷。遺備物乎營寢,掩重閽於窒皇;椒風暖兮猶昔,蘭殿幽而不陽。嗚呼
 哀哉!



 側闈高義,彤管有懌;道變虞風,功參唐跡。婉如之人,休光赤舄;施諸天地,而無朝夕。嗚呼哀哉!



 有司奏謚曰穆。太宗即位,追崇曰穆太后。



 太后父仲遷,天監初,官至兗州刺史。



 高祖阮脩容,諱令嬴,本姓石,會稽餘姚人也。齊始安王遙光納焉。遙光敗,入東昏宮。建康城平,高祖納為彩女。天監七年八月,生世祖。尋拜為脩容,常隨世祖出蕃。



 大同六年六月,薨于江州內寢,時年六十七。其年十一月,歸葬江寧縣通望山。謚曰宣。世祖即位,有司奏追崇為文宣太后。



 承聖二年,追贈太后父齊故奉朝請靈寶散
 騎常侍、左衛將軍,封武康縣侯,邑五百戶;母陳氏,武康侯夫人。



 世祖徐妃,諱昭佩,東海郯人也。祖孝嗣,太尉、枝江文忠公。父緄,侍中、信武將軍。天監十六年十二月,拜湘東王妃。生世子方等、益昌公主含貞。太清三年五月,被譴死,葬江陵瓦官寺。



 史臣曰:后妃道贊皇風,化行天下,蓋取《葛覃》、《關雎》之義焉。至於穆貴嬪,徽華早著,誕育元良,德懋六宮,美矣。世祖徐妃之無行,自致殲滅,宜哉。



\end{pinyinscope}