\article{卷第四本紀第四 簡文帝}

\begin{pinyinscope}

 太宗簡文皇帝,諱綱,字世纘,小字六通,高祖第三子,昭明太子母弟也。天監二年十月丁未,生于顯陽殿。五年,封晉安王,食邑八千戶。八年,為雲麾將軍,領石頭戍軍事,量置佐吏。九年,遷使持節、都督南北兗、青、徐、冀五州諸軍事、宣毅將軍、南兗州刺史。十二年,入為宣惠將軍、丹陽尹。十三年,出為使持節、都督荊、雍、梁、南北秦、益、寧
 七州諸軍事、南蠻校尉、荊州刺史,將軍如故。十四年,徙為都督江州諸軍事、雲麾將軍、江州刺史,持節如故。十七年,徵為西中郎將、領石頭戍軍事,尋復為宣惠將軍、丹陽尹,加侍中。普通元年,出為使持節、都督益、寧、雍、梁、南北秦、沙七州諸軍事、益州刺史;未拜,改授雲麾將軍、南徐州刺史。四年,徙為使持節、都督雍、梁、南北秦四州郢州之竟陵司州之隨郡諸軍事,平西將軍、寧蠻校尉、雍州刺史。五年,進號安北將軍。七年,權進都督荊、益、南梁三州諸軍事。是歲,丁所生穆貴嬪喪,上表陳解,詔還攝本任。中大通元年,詔依先給鼓吹一部。二年,徵為都
 督南揚、徐二州諸軍事、驃騎將軍、揚州刺史。三年四月乙巳,昭明太子薨。五月丙申,詔曰:「非至公無以主天下,非博愛無以臨四海。所以堯舜克讓,惟德是與;文王舍伯邑考而立武王,格于上下,光于四表。今岱宗牢落,天步艱難,淳風猶鬱,黎民未乂,自非克明克哲,允武允文,豈能荷神器之重,嗣龍圖之尊。晉安王綱,文義生知,孝敬自然,威惠外宣,德行內敏,群后歸美,率土宅心。可立為皇太子。」七月乙亥,臨軒策拜,以修繕東宮,權居東府。四年九月,移還東宮。



 太清三年五月丙辰,高祖崩。辛巳,即皇帝位。詔曰:「朕以不造,夙丁閔凶。大行皇帝奄棄萬
 國,攀慕號絺,厝身靡所。猥以寡德,越居民上,煢煢在疚,罔知所託,方賴籓輔,社稷用安。謹遵先旨,顧命遺澤,宜加億兆。可大赦天下。」壬午,詔曰:「育物惟寬,馭民惟惠,道著興王,本非隸役。或開奉國,便致擒虜,或在邊疆,濫被抄劫。二邦是競,黎元何罪!朕以寡昧,創承鴻業,既臨率土,化行宇宙,豈欲使彼獨為匪民。諸州見在北人為奴婢者,并及妻兒,悉可原放。」癸未,追謚妃王氏為簡皇后。六月丙戌,以南康嗣王會理為司空。丁亥,立宣城王大器為皇太子。壬辰,封當陽公大心為尋陽郡王,石城公大款為江夏郡王,寧國公大臨為南海郡王,臨城公大
 連為南郡王,西豊公大春為安陸郡王,新塗公大成為山陽郡王,臨湘公大封為宜都郡王。秋七月甲寅,廣州刺史元景仲謀應侯景,西江督護陳霸先起兵攻之,景仲自殺,霸先迎定州刺史蕭勃為刺史。戊辰,以吳郡置吳州,以安陸王大春為刺史。庚午,以司空南康嗣王會理兼尚書令,南海王大臨為揚州刺史,新興王大莊為南徐州刺史。是月,九江大饑,人相食十四五。八月癸卯,征東大將軍、開府儀同三司、南徐州刺史蕭淵藻薨。冬十月丁未,地震。十二月,百濟國遣使獻方物。



 大寶元年春正月辛亥朔,以國哀不朝會。詔曰:「蓋天下
 者,至公之神器,在昔三五,不獲已而臨蒞之。故帝王之功,聖人之餘事。軒冕之華,儻來之一物。太祖文皇帝含光大之量,啟西伯之基。高祖武皇帝道洽二儀,智周萬物。屬齊季薦瘥,彞倫剝喪,同氣離入苑之禍,元首懷無厭之欲,乃當樂推之運,因億兆之心,承彼掎角,雪茲仇恥。事非為己,義實從民。故功成弗居,卑宮菲食,大慈之業普薰,汾陽之詔屢下。于茲四紀,無得而稱。朕以寡昧,哀煢孔棘,生靈已盡,志不圖全,僶俛視陰,企承鴻緒。懸旌履薄,未足云喻。痛甚愈遲,諒闇彌切。方當玄默在躬,棲心事外。即王道未直,天步猶艱,式憑宰輔,以弘庶政。
 履端建號,仰惟舊章。可大赦天下,改太清四年為大寶元年。」丁巳,天雨黃沙。己未,太白經天,辛酉乃止。西魏寇安陸,執司州刺史柳仲禮,盡沒漢東之地。丙寅,月晝見。癸酉,前江都令祖皓起義,襲廣陵,斬賊南兗州刺史董紹先。侯景自帥水步軍擊皓。二月癸未,景攻陷廣陵,皓等並見害。丙戌,以安陸王大春為東揚州刺史。省吳州,如先為郡。詔曰:「近東垂擾亂,江陽縱逸。上宰運謀,猛士雄奮,吳、會肅清,濟、兗澄謐,京師畿內,無事戎衣。朝廷達宮,齋內左右,並可解嚴。」乙巳,以尚書僕射王克為左僕射。是月,邵陵王綸自尋陽至于夏口,郢州刺史南平王
 恪以州讓綸。丙午,侯景逼太宗幸西州。夏五月庚午,征北將軍、開府儀同三司鄱陽嗣王範薨。自春迄夏,大饑,人相食,京師尤甚。六月辛巳,以南郡王大連行揚州事。庚子,前司州刺史羊鴉仁自尚書省出奔西州。秋七月戊辰,賊行臺任約寇江州,刺史尋陽王大心以州降約。是月,以南郡王大連為江州刺史。八月甲午,湘東王繹遣領軍將軍王僧辯率眾逼郢州。乙亥,侯景自進位相國,封二十郡為漢王。邵陵王綸棄郢州走。冬十月乙未,侯景又逼太宗幸西州曲宴,自加宇宙大將軍、都督六合諸軍事。立皇子大鈞為西陽郡王,大威為武寧郡王,
 大球為建安郡王,大昕為義安郡王,大摯為綏建郡王,大圜為樂梁郡王。壬寅,景害南康嗣王會理。十一月,任約進據西陽,分兵寇齊昌,執衡陽王獻送京師,害之。湘東王繹遣前寧州刺史徐文盛督眾軍拒約。南郡王前中兵張彪起義於會稽若邪山,攻破浙東諸縣。



 二年春二月,邵陵王綸走至安陸董城,為西魏所攻,軍敗,死。三月,侯景自帥眾西寇。丁未,發京師,自石頭至新林,舳艫相接。四月,至西陽。乙亥,景分遣偽將宋子仙、任約襲郢州。丙子,執刺史蕭方諸。閏月甲子,景進寇巴陵,湘東王繹所遣領軍將軍王僧辯連戰不能剋。五月癸
 未,湘東王驛遣游擊將軍胡僧祐、信州刺史陸法和援巴陵,景遣任約帥眾拒援軍。六月甲辰,僧祐等擊破任約,擒之。乙巳,景解圍宵遁,王僧辯督眾軍追景。庚申,攻魯山城,剋之,獲魏司徒張化仁、儀同門洪慶。辛酉,進圍郢州,下之,獲賊帥宋子仙等。鄱陽王故將侯瑱起兵,襲偽儀同于慶于豫章,慶敗走。秋七月丁亥,侯景還至京師。辛丑,王僧辯軍次湓城,賊行江州事范希榮棄城走。八月丙午,晉熙人王僧振、鄭寵起兵襲郡城,偽晉州刺史夏侯威生、儀同任延遁走。戊午,侯景遣衛尉卿彭俊、廂公王僧貴率兵入殿,廢太宗為晉安王,幽于永福
 省。害皇太子大器、尋陽王大心、西陽王大鈞、武寧王大威、建平王大球、義安王大昕及尋陽王諸子二十人。矯為太宗詔,禪于豫章嗣王棟,大赦改年。遣使害南海王大臨於吳郡,南郡王大連於姑孰,安陸王大春於會稽,新興王大莊於京口。冬十月壬寅,帝謂舍人殷不害曰:「吾昨夜夢吞土,卿試為我思之。」不害曰:「昔重耳饋塊,卒還晉國。陛下所夢,得符是乎。」及王偉等進觴於帝曰:「丞相以陛下憂憤既久,使臣上壽。」帝笑曰:「壽酒,不得盡此乎?」於是並賚酒肴、曲項琵琶,與帝飲。帝知不免,乃盡酣,曰:「不圖為樂一至於斯!」既醉寢,偉乃出,俊進土囊,王脩纂坐其上,於
 是太宗崩於永福省,時年四十九。賊偽謚曰明皇帝,廟稱高宗。



 明年,三月癸丑,王僧辯率前百官奉梓宮升朝堂,世祖追崇為簡文皇帝,廟曰太宗。四月乙丑,葬莊陵。



 初,太宗見幽縶,題壁自序云:「有梁正士蘭陵蕭世纘,立身行道,終始如一,風雨如晦,雞鳴不已。弗欺暗室,豈況三光,數至於此,命也如何!」又為《連珠》二首,文甚悽愴。太宗幼而敏睿,識悟過人,六歲便屬文,高祖驚其早就,弗之信也。乃於御前面試,辭採甚美。高祖歎曰:「此子,吾家之東阿。」既長,器宇寬弘,未嘗見慍喜。方頰豊下,鬚鬢如畫,眄睞則目光燭人。讀書十行俱下。九流百氏,經目必
 記;篇章辭賦,操筆立成。博綜儒書,善言玄理。自年十一,便能親庶務,歷試蕃政,所在有稱。在穆貴嬪憂,哀毀骨立,晝夜號泣不絕聲,所坐之席,沾濕盡爛。在襄陽拜表北伐,遣長史柳津、司馬董當門,壯武將軍杜懷寶、振遠將軍曹義宗等眾軍進討,剋平南陽、新野等郡,魏南荊州刺史李志據安昌城降,拓地千餘里。及居監撫,多所弘宥,文案簿領,纖毫不可欺。引納文學之士,賞接無倦,恆討論篇籍,繼以文章。高祖所製《五經講疏》,嘗於玄圃奉述,聽者傾朝野。雅好題詩,其序云:「餘七歲有詩癖,長而不倦。」然傷於輕艷,當時號曰「宮體」。所著《昭明太子傳》
 五卷,《諸王傳》三十卷,《禮大義》二十卷,《老子義》二十卷,《莊子義》二十卷,《長春義記》一百卷,《法寶連璧》三百卷,並行於世焉。



 史臣曰:太宗幼年聰睿,令問夙標,天才縱逸,冠於今古。文則時以輕華為累,君子所不取焉。及養德東朝,聲被夷夏,洎乎繼統,實有人君之懿矣。方符文、景,運鐘《屯》、《剝》,受制賊臣,弗展所蘊,終罹懷、愍之酷,哀哉!



\end{pinyinscope}