\article{卷第六本紀第六 敬帝}

\begin{pinyinscope}

 敬皇帝,諱方智,字慧相,小字法真,世祖第九子也。太清三年,封興梁侯。承聖元年,封晉安王,邑二千戶。二年,出為平南將軍、江州刺史。三年十一月,江陵陷,太尉揚州刺史王僧辯、司空南徐州刺史陳霸先定議,以帝為太宰、承制,奉迎還京師。四年二月癸丑,至自尋陽,入居朝堂。以太尉王僧辯為中書監、錄尚書、驃騎將軍、都督中
 外諸軍事。加司空陳霸先班劍三十人。以豫州刺史侯瑱為江州刺史,儀同三司、湘州刺史蕭循為太尉,儀同三司、廣州刺史蕭勃為司徒,鎮東將軍張彪為郢州刺史。三月,齊遣其上黨王高渙送貞陽侯蕭淵明來主梁嗣,至東關,遣吳興太守裴之橫與戰,敗績,之橫死。太尉王僧辯率眾出屯姑孰。四月,司徒陸法和以郢州附于齊,遣江州刺史侯瑱討之。七月辛丑,王僧辯納貞陽侯蕭淵明,自採石濟江。甲辰,入於京師,以帝為皇太子。九月甲辰,司空陳霸先舉義,襲殺王僧辯,黜蕭淵明。丙午,帝即皇帝位。



 紹泰元年冬十月己巳,詔曰:「王室不造,嬰罹禍釁,西都失守,朝廷淪覆,先帝梓宮,播越非所,王基傾弛,率土罔戴。朕以荒幼,仍屬艱難,泣血枕戈,志復仇逆。大恥未雪,夙宵鯁憤。群公卿尹,勉以大義,越登寡闇,嗣奉洪業。顧惟夙心,念不至此。庶仰憑先靈,傍資將相,克清元惡,謝冤陵寢。今墜命載新,宗祊更祀,慶流億兆,豈予一人。可改承聖四年為紹泰元年,大赦天下,內外文武賜位一等。」以貞陽侯淵明為司徒,封建安郡公,食邑三千戶。壬子,以司空陳霸先為尚書令、都督中外諸軍事、車騎將軍、揚、南徐二州刺史司空如故。震州刺史杜龕舉兵,攻
 信武將軍陳蒨於長城,義興太守韋載據郡以應之。癸丑,進太尉蕭循為太保,新除司徒建安公淵明為太傅,司徒蕭勃為太尉。以鎮南將軍王琳為車騎將軍、開府儀同三司。戊午,尊所生夏貴妃為皇太后。立妃王氏為皇后。鎮東將軍、揚州刺史張彪進號征東大將軍。鎮北將軍、譙秦二州刺史徐嗣徽進號征北大將軍。征南將軍、南豫州刺史任約進號征南大將軍。辛未,詔司空陳霸先東討韋載。丙子,任約、徐嗣徽舉兵反,乘京師無備,竊據石頭。丁丑,韋載降,義興平。遣晉陵太守周文育率軍援長城。十一月庚辰,齊安州刺史翟子崇、楚州刺史
 劉仕榮、淮州刺史柳達摩率眾赴任約,入于石頭。庚寅,司空陳霸先旋于京師。十二月庚戌,徐嗣徽、任約又相率至採石,迎齊援。丙辰,遣猛烈將軍侯安都水軍於江寧邀之,賊眾大潰,嗣徽、約等奔于江西。庚申,翟子崇等請降,並放還北。



 太平元年春正月戊寅,大赦天下,其與任約、徐嗣徽協契同謀,一無所問。追贈簡文皇帝諸子。以故永安侯確子後襲封邵陵王,奉攜王後。癸未,鎮東將軍、震州刺史杜龕降,詔賜死,曲赦吳興郡。己亥,以太保、宜豊侯蕭循襲封鄱陽王。東揚州刺史張彪圍臨海太守王懷振於
 剡巖。二月庚戌,遣周文育、陳茜襲會稽,討彪。癸丑,彪長史謝岐、司馬沈泰、軍主吳寶真等舉城降,彪敗走。以中衛將軍臨川王大款即本號開府儀同三司,中護軍桂陽王大成為護軍將軍。丙辰,若耶村人斬張彪,傳首京師,曲赦東揚州。己未,罷震州,還復吳興郡。癸亥,賊徐嗣徽、任約襲采石戍,執戍主明州刺史張懷鈞,入于齊。甲子,以東土經杜龕、張彪抄暴,遣大使巡省。三月丙子,罷東揚州,還復會稽郡。壬午,班下遠近並雜用古今錢。戊戌,齊遣大將蕭軌出柵口,向梁山,司空陳霸先、軍主黃菆逆擊,大破之。軌退保蕪湖。遣周文育、侯安都眾軍,據
 梁山拒之。夏四月丁巳,司空陳霸先表詣梁山撫巡將帥。壬申,侯安都輕兵襲齊行臺司馬恭於歷陽,大破之,俘獲萬計。五月癸未,太傅建安公淵明薨。庚寅,齊軍水步入丹陽縣。丙申,至秣陵故冶。敕周文育還頓方丘,徐度頓馬牧,杜棱頓大桁。癸卯,齊軍進據兒塘,輿駕出頓趙建故籬門,內外纂嚴。六月甲辰,齊潛軍至蔣山龍尾,斜趨莫府山北,至玄武廟西北。乙卯,司空陳霸先授眾軍節度,與齊軍交戰,大破之,斬齊北兗州刺史杜方慶及徐嗣徽弟嗣宗,生擒徐嗣產、蕭軌、東方老、王敬寶、李希光、裴英起、劉歸義等,皆誅之。戊午,大赦天下,軍士身
 殞戰場,悉遣斂祭,其無家屬,即為瘞埋。辛酉,解嚴。秋七月丙子,車騎將軍、司空陳霸先進位司徒,加中書監,餘如故。丁亥,以開府儀同三司侯瑱為司空。八月己酉,太保鄱陽王循薨。九月壬寅,改元大赦,孝悌力田賜爵一級,殊才異行所在奏聞,饑難流移勒歸本土。進新除司徒陳霸先為丞相、錄尚書事、鎮衛大將軍、揚州牧,封義興郡公。中權將軍王沖即本號開府儀同三司。吏部尚書王通為尚書右僕射。丁巳,以郢州刺史徐度為領軍將軍。冬十一月乙卯,起雲龍、神虎門。十二月壬申,進太尉、鎮南將軍蕭勃為太保、驃騎將軍。以新除左衛將軍
 歐陽頠為安南將軍、衡州刺史。壬午,平南將軍劉法瑜進號安南將軍。甲午,以前壽昌令劉睿為汝陰王,前鎮西法曹、行參軍蕭鸑為巴陵王,奉宋、齊二代後。



 二年春正月壬寅,詔曰:「夫子降靈體哲,經仁緯義,允光素王,載闡玄功,仰之者彌高,誨之者不倦。立忠立孝,德被蒸民,制禮作樂,道冠群后。雖泰山頹峻,一簣不遺,而泗水餘瀾,千載猶在。自皇圖屯阻,祀薦不修,奉聖之門,胤嗣殲滅,敬神之寢,簠簋寂寥。永言聲烈,實兼欽愴。外可搜舉魯國之族,以為奉聖後;並繕廟堂,供備祀典,四時薦秩,一皆遵舊。」是日,又詔「諸州各置中正,依舊訪舉。
 不得輒承單狀序官,皆須中正押上,然後量授。詳依品制,務使精實。其荊、雍、青、兗雖暫為隔閡,衣冠多寓淮海,猶宜不廢司存。會計罷州,尚為大郡,人士殷曠,可別置邑居。至如分割郡縣,新號州牧,並係本邑,不勞兼置。其選中正,每求耆德,該悉以他官領之。」以車騎將軍、開府儀同三司王琳為司空、驃騎大將軍。分尋陽、太原、齊昌、高唐、新蔡五郡,置西江州,即於尋陽仍充州鎮。又詔「宗室在朝開國承家者,今猶稱世子,可悉聽襲本爵。」以尚書右僕射王通為尚書左僕射。丁巳,鎮西將軍、益州刺史長沙王韶進號征南將軍。二月庚午,領軍將軍徐度
 入東關。太保、廣州刺史蕭勃舉兵反,遣偽帥歐陽頠、傅泰、勃從子孜為前軍,南江州刺史餘孝頃以兵會之。詔平西將軍周文育、平南將軍侯安都等率眾軍南討。戊子,徐度至合肥,燒齊船三千艘。癸巳,周文育軍於巴山生獲歐陽頠。三月庚子,文育前軍丁法洪於蹠口生俘傅泰。蕭孜、餘孝頃軍退走。甲辰,以新除司空王琳為湘、郢二州刺史。甲寅,德州刺史陳法武、前衡州刺史譚世遠於始興攻殺蕭勃。夏四月癸酉,曲赦江、廣、衡三州;並督內為賊所拘逼者,並皆不問。己卯,鑄四柱錢,一準二十。齊遣使請和。壬辰,改四柱錢一準十。丙申,復閉細錢。
 蕭勃故主帥前直閣蘭敱襲殺譚世遠,敱仍為亡命夏侯明徹所殺。勃故記室李寶藏奉懷安侯蕭任據廣州作亂。戊戌,侯安都進軍,餘孝頃棄軍走,蕭孜請降,豫章平。五月乙巳,平西將軍周文育進號鎮南將軍,侯安都進號鎮北將軍,並以本號開府儀同三司。丙午,以鎮軍將軍徐度為南豫州刺史。戊辰,餘孝頃遣使詣丞相府乞降。秋八月甲午,加丞相陳霸先黃鉞,領太傅,劍履上殿,入朝不趨,贊拜不名,給羽葆、鼓吹。九月辛丑,崇丞相為相國,總百揆,封十郡為陳公,備九錫之禮,加璽紱遠遊冠,位在王公上。加相國綠綟綬。置陳國百司。冬十月
 戊辰,進陳公爵為王,增封十郡,并前為二十郡。命陳王冕十有二旒,建天子旌旂,出警入蹕,乘金根車,駕六馬,備五時副車,置旄頭雲罕,樂舞八佾,設鐘鋋宮縣。王后王子女爵命之典,一依舊儀。辛未,詔曰:五運更始,三正迭代,司牧黎庶,是屬聖賢,用能經緯乾坤,彌綸區宇,大庇黔首,闡揚洪烈。革晦以明,積代同軌,百王踵武,咸由此則。梁德湮微,禍難薦發:太清云始,用困長蛇;承聖之年,又罹封豕;爰至天成,重竊神器。三光亟改,七廟乏祀,含生已泯,鼎命斯墜,我皇之祚,眇若綴旒,靜惟《屯》、《剝》,夕惕載懷。相國陳王,有縱自天,降神惟嶽,天地合德,晷曜
 齊明。拯社稷之橫流,提億兆之塗炭。東誅叛逆,北殲獯醜,威加四海,仁漸萬國。復張崩樂,重紀絕禮,儒館聿修,戎亭虛候。雖大功在舜,盛績維禹,巍巍蕩蕩,無得而稱。來獻白環,豈直皇虞之世;入貢素雉,非止隆周之日。故效珍川陸,表瑞煙雲,玉露醴泉,旦夕凝涌,嘉禾瑞草,孳植郊甸,道昭於悠代,勛格於皇穹。明明上天,光華日月,革故著於玄象,代德彰於讖圖,獄訟有違,謳歌爰適,天之歷數,實有攸在。朕雖庸藐,闇於古昔,永稽崇替,為日已久,敢忘列代之遺典,人祇之至願乎!今便遜位別宮,敬禪于陳,一依唐虞、宋齊故事。



 陳王踐阼,奉帝為江陰王,薨于外邸,
 時年十六,追謚敬皇帝。



 史臣曰:梁季橫潰,喪亂屢臻,當此之時,天歷去矣,敬皇高讓,將同釋負焉。



 史臣侍中、鄭國公魏徵曰:「高祖固天攸縱,聰明稽古,道亞生知,學為博物,允文允武,多藝多才。爰自諸生,有不羈之度,屬昏凶肆虐,天倫及禍,收合義旅,將雪家冤。曰紂可伐,不其而會,龍躍樊、漢,電擊湘、郢,剪離德如振槁,取獨夫如拾遺。其雄才大略,固無得而稱矣。既懸白旗之首,方應皇天之眷,布德施惠,悅近來遠,開蕩蕩之王道,革靡靡之商俗,大脩文教,盛飾禮容,鼓扇玄風,闡揚
 儒業,介胄仁義,折衝樽俎,聲振寰宇,澤流遐裔,干戈載戢,凡數十年。濟濟焉,洋洋焉,魏、晉已來,未有若斯之盛。然不能息末敦本,斫彫為樸,慕名好事,崇尚浮華,抑揚孔、墨,流連釋、老。或經夜不寢,或終日不食,非弘道以利物,惟飾智以驚愚。且心未遺榮,虛廁蒼頭之伍;高談脫屣,終戀黃屋之尊。夫人之大欲,在乎飲食男女,至於軒冕殿堂,非有切身之急。高祖屏除嗜慾,眷戀軒冕,得其所難而滯於所易,可謂神有所不達,智有所不通矣。逮夫精華稍竭,鳳德已衰,惑於聽受,權在姦佞,儲后百辟,莫得盡言。險躁之心,暮年愈甚。見利而動,愎諫違卜,開
 門揖盜,棄好即仇,釁起蕭墻,禍成戎羯,身殞非命,災被億兆,衣冠敝鋒鏑之下,老幼粉戎馬之足。瞻彼《黍離》,痛深周廟;永言《麥秀》,悲甚殷墟。自古以安為危,既成而敗,顛覆之速,書契所未聞也。《易》曰:『天之所助者信,人之所助者順。』高祖之遇斯屯剝,不得其死,蓋動而之險,不由信順,失天人之所助,其能免於此乎!



 太宗聰睿過人,神彩秀發,多聞博達,富贍詞藻。然文艷用寡,華而不實,體窮淫麗,義罕疏通,哀思之音,遂移風俗,以此而貞萬國,異乎周誦、漢莊矣。我生不辰,載離多難,桀逆構扇,巨猾滔天,始自牖里之拘,終類望夷之禍。悠悠蒼天,其可問
 哉!



 昔國步初屯,兵纏魏闕,群后釋位,投袂勤王。元帝以盤石之宗,受分陜之任,屬君親之難,居連率之長,不能撫劍嘗膽,枕戈泣血,躬先士卒,致命前驅;遂乃擁眾逡巡,內懷觖望,坐觀時變,以為身幸。不急莽、卓之誅,先行昆弟之戮。又沉猜忌酷,多行無禮。騁智辯以飾非,肆忿戾以害物。爪牙重將,心膂謀臣,或顧眄以就拘囚,或一言而及菹醢。朝之君子,相顧懍然。自謂安若泰山,舉無遺策,怵於邪說,即安荊楚。雖元惡克剪,社稷未寧,而西鄰責言,禍敗旋及。上天降鑒,此焉假手,天道人事,其可誣乎!其篤志藝文,採浮淫而棄忠信;戎昭果毅,先骨肉
 而後寇仇。雖口誦《六經》,心通百氏,有仲尼之學,有公旦之才,適足以益其驕矜,增其禍患,何補金陵之覆沒,何救江陵之滅亡哉!



 敬帝遭家不造,紹茲屯運,征伐有所自出,政刑不由於己,時無伊、霍之輔,焉得不為高讓歟?」



\end{pinyinscope}