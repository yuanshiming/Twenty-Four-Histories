\article{卷第三十八列傳第三十二 硃異 賀琛}

\begin{pinyinscope}

 朱異,字彥和,吳郡錢唐人也。父巽,以義烈知名,官至齊江夏王參軍、吳平令。異年數歲,外祖顧歡撫之,謂異祖昭之曰:「此兒非常器,當成卿門戶。」年十餘歲,好群聚蒲博,頗為鄉黨所患。既長,乃折節從師,遍治《五經》,尤明《禮》、《易》,涉獵文史,兼通雜藝,博弈書算,皆其所長。年二十,詣都,尚書令沈約面試之,因戲異曰:「卿年少,何乃不廉?」異
 逡巡未達其旨。約乃曰:「天下唯有文義棋書,卿一時將去,可謂不廉也。」其年,上書言建康宜置獄司,比廷尉。敕付尚書詳議,從之。舊制,年二十五方得釋褐。時異適二十一,特敕擢為揚州議曹從事史。尋有詔求異能之士,《五經》博士明山賓表薦異曰:「竊見錢唐朱異,年時尚少,德備老成。在獨無散逸之想,處闇有對賓之色,器宇弘深,神表峰峻。金山萬丈,緣陟未登;玉海千尋,窺映不測。加以珪璋新琢,錦組初構,觸響鏗鏘,值采便發。觀其信行,非惟十室所稀,若使負重遙途,必有千里之用。」高祖召見,使說《孝經》、《周易》義,甚悅之,謂左右曰:「朱異實異。」後
 見明山賓,謂曰:「卿所舉殊得其人。」仍召異直西省,俄兼太學博士。其年,高祖自講《孝經》,使異執讀。遷尚書儀曹郎,入兼中書通事舍人,累遷鴻臚卿,太子右衛率,尋加員外常侍。



 普通五年,大舉北伐,魏徐州刺史元法僧遣使請舉地內屬,詔有司議其虛實。異曰:「自王師北討,剋獲相繼,徐州地轉削弱,咸願歸罪法僧,法僧懼禍之至,其降必非偽也。」高祖仍遣異報法僧,並敕眾軍應接,受異節度。既至,法僧遵承朝旨,如異策焉。中大通元年,遷散騎常侍。自周捨卒後,異代掌機謀,方鎮改換,朝儀國典,詔誥敕書,並兼掌之。每四方表疏,當局簿領,諮詢詳
 斷,填委於前。異屬辭落紙,覽事下議,縱橫敏贍,不暫停筆,頃刻之間,諸事便了。



 大同四年,遷右衛將軍。六年,異啟於儀賢堂奉述高祖《老子義》,敕許之。及就講,朝士及道俗聽者千餘人,為一時之盛。時城西又開士林館以延學士,異與左丞賀琛遞日述高祖《禮記中庸義》,皇太子又召異於玄圃講《易》。八年,改加侍中。太清元年,遷左衛將軍,領步兵。二年,遷中領軍,舍人如故。



 高祖夢中原平,舉朝稱慶,旦以語異,異對曰:「此宇內方一之徵。」及侯景歸降,敕召群臣議,尚書僕射謝舉等以為不可,高祖欲納之,未決;嘗夙興至武德閣,自言「我國家承平若此,
 今便受地,詎是事宜,脫致紛紜,悔無所及」。異探高祖微旨,應聲答曰:「聖明御宇,上應蒼玄,北土遺黎,誰不慕仰?為無機會,未達其心。今侯景分魏國太半,輸誠送款,遠歸聖朝,豈非天誘其衷,人獎其計!原心審事,殊有可嘉。今若不容,恐絕後來之望。此誠易見,願陛下無疑。」高祖深納異言,又感前夢,遂納之。及貞陽敗沒,自魏遣使還,述魏相高澄欲更申和睦。敕有司定議,異又以和為允,高祖果從之。其年六月,遣建康令謝挺、通直郎徐陵使北通好。是時,侯景鎮壽春,累啟絕和,及請追使。又致書與異,辭意甚切,異但述敕旨以報之。八月,景遂舉兵反,
 以討異為名。募兵得三千人,及景至,仍以其眾守大司馬門。



 初,景謀反,合州刺史鄱陽王範、司州刺史羊鴉仁並累有啟聞,異以景孤立寄命,必不應爾,乃謂使者:「鄱陽王遂不許國家有一客!」並抑而不奏,故朝廷不為之備。及寇至,城內文武咸尤之。皇太子又製《圍城賦》,其末章云:「彼高冠及厚履,並鼎食而乘肥,升紫霄之丹地,排玉殿之金扉,陳謀謨之啟沃,宣政刑之福威,四郊以之多壘,萬邦以之未綏。問豺狼其何者?訪虺蜴之為誰?」蓋以指異。異因慚憤,發病卒,時年六十七。詔曰:「故中領軍異,器宇弘通,才力優贍,諮謀帷幄,多歷年所。方贊朝經,
 永申寄任。奄先物化,惻悼兼懷。可贈侍中、尚書右僕射,給秘器一具。凶事所須,隨由資辦。」舊尚書官不以為贈,及異卒,高祖惜之,方議贈事。左右有善異者,乃啟曰:「異忝歷雖多,然平生所懷,願得執法。」高祖因其宿志,特有此贈焉。



 異居權要三十餘年,善窺人主意曲,能阿諛以承上旨,故特被寵任。歷官自員外常侍至侍中,四官皆珥貂,自右衛率至領軍,四職並驅鹵簿,近代未之有也。異及諸子自潮溝列宅至青溪,其中有臺池玩好,每暇日與賓客遊焉。四方所饋,財貨充積。性吝嗇,未嘗有散施。廚下珍羞腐爛,每月常棄十數車,雖諸子別房亦不
 分贍。所撰《禮》、《易》講疏及儀注、文集百餘篇,亂中多亡逸。



 長子肅,官至國子博士;次子閏,司徒掾。並遇亂卒。



 賀琛,字國寶,會稽山陰人也。伯父蒨,步兵校尉,為世碩儒。琛幼,蒨授其經業,一聞便通義理。蒨異之,常曰:「此兒當以明經致貴。」蒨卒後,琛家貧,常往還諸暨,販粟以自給。閑則習業,尤精《三禮》。初,蒨於鄉里聚徒教授,至是又依琛焉。



 普通中,刺史臨川王辟為祭酒從事史。琛始出都,高祖聞其學術,召見文德殿,與語悅之,謂僕射徐勉曰:「琛殊有世業。」仍補王國侍郎,俄兼太學博士,稍遷中衛參軍事、尚書通事舍人,參禮儀事。累遷通直正員郎,
 舍人如故。又征西鄱陽王中錄事,兼尚書左丞,滿歲為真。詔琛撰《新謚法》,至今施用。時皇太子議,大功之末,可以冠子嫁女。琛駮之曰:令旨以「大功之末可得冠子嫁女,不得自冠自嫁。」推以《記》文,竊猶致惑。案嫁冠之禮,本是父之所成,無父之人,乃可自冠。故稱大功小功,並以冠子嫁子為文;非關惟得為子,己身不得也。小功之末,既得自嫁娶,而亦云「冠子娶婦」,其義益明。故先列二服,每明冠子嫁子,結於後句,方顯自娶之義。既明小功自娶,即知大功自冠矣,蓋是約言而見旨。若謂緣父服大功,子服小功,小功服輕,故得為子冠嫁,大功服重,故不
 得自嫁自冠者,則小功之末,非明父子服殊,不應復云「冠子嫁子」也。若謂小功之文言己可娶,大功之文不言己冠,故知身有大功,不得自行嘉禮,但得為子冠嫁。竊謂有服不行嘉禮,本為吉凶不可相干。子雖小功之末,可得行冠嫁,猶應須父得為其冠嫁。若父於大功之末可以冠子嫁子,是於吉凶禮無礙;吉凶禮無礙,豈不得自冠自嫁?若自冠自嫁於事有礙,則冠子嫁子寧獨可通?今許其冠子而塞其自冠,是琛之所惑也。



 又令旨推「下殤小功不可娶婦,則降服大功亦不得為子冠嫁」。伏尋此旨,若謂降服大功不可冠子嫁子,則降服小功亦
 不可自冠自娶,是為凡厥降服大功小功皆不得冠娶矣。《記》文應云降服則不可,寧得惟稱下殤?今不言降服,的舉下殤,實有其義。夫出嫁出後,或有再降,出後之身,於本姊妹降為大功;若是大夫服士,又以尊降,則成小功。其於冠嫁,義無以異。所以然者,出嫁則有受我,出後則有傳重,並欲薄於此而厚於彼,此服雖降,彼服則隆。昔實期親,雖再降猶依小功之禮,可冠可嫁。若夫期降大功,大功降為小功,止是一等,降殺有倫,服末嫁冠,故無有異。惟下殤之服,特明不娶之義者,蓋緣以幼稚之故。夭喪情深,既無受厚佗姓,又異傳重彼宗,嫌其年稚
 服輕,頓成殺略,故特明不娶,以示本重之恩。是以凡厥降服,冠嫁不殊;惟在下殤,乃明不娶。其義若此,則不得言大功之降服,皆不可冠嫁也。且《記》云「下殤小功」,言下殤則不得通於中上,語小功則不得兼於大功。若實大小功降服皆不冠嫁,上中二殤亦不冠嫁者,《記》不得直云「下殤小功則不可」。恐非文意。此又琛之所疑也。



 遂從琛議。



 遷員外散騎常侍。舊尚書南坐,無貂;貂自琛始也。頃之,遷御史中丞,參禮儀事如先。琛家產既豊,買主第為宅,為有司所奏,坐免官。俄復為尚書左丞,遷給事黃門侍郎,兼國子博士,未拜,改為通直散騎常侍,領尚書
 左丞,並參禮儀事。琛前後居職,凡郊廟諸儀,多所創定。每見高祖,與語常移晷刻,故省中為之語曰:「上殿不下有賀雅。」琛容止都雅,故時人呼之。遷散騎常侍,參禮儀如故。



 是時,高祖任職者,皆緣飾姦諂,深害時政,琛遂啟陳事條封奏曰:臣荷拔擢之恩,曾不能效一職;居獻納之任,又不能薦一言。竊聞「慈父不愛無益之子,明君不畜無益之臣」,臣所以當食廢飧,中宵而歎息也。輒言時事,列之於後。非謂謀猷,寧云啟沃。獨緘胸臆,不語妻子。辭無粉飾,削槁則焚。脫得聽覽,試加省鑒。如不允合,亮其戇愚。



 其一事曰:今北邊稽服,戈甲解息,政是生聚教
 訓之時,而天下戶口減落,誠當今之急務。雖是處彫流,而關外彌甚,郡不堪州之控總,縣不堪郡之裒削,更相呼擾,莫得治其政術,惟以應赴征斂為事。百姓不能堪命,各事流移,或依於大姓,或聚於屯封,蓋不獲已而竄亡,非樂之也。國家於關外賦稅蓋微,乃至年常租課,動致逋積,而民失安居,寧非牧守之過?東境戶口空虛,皆由使命繁數。夫犬不夜吠,故民得安居。今大邦大縣,舟舸銜命者,非惟十數;復窮幽之鄉,極遠之邑,亦皆必至。每有一使,屬所搔擾;況復煩擾積理,深為民害。駑困邑宰,則拱手聽其漁獵;桀黠長吏,又因之而為貪殘。縱有
 廉平,郡猶掣肘。故邑宰懷印,類無考績,細民棄業,流冗者多,雖年降復業之詔,屢下蠲賦之恩,而終不得反其居也。



 其二事曰:聖主恤隱之心,納隍之念,聞之遐邇,至於翾飛蠕動,猶且度脫,況在兆庶。而州郡無恤民之志,故天下顒顒,惟注仰於一人,誠所謂「愛之如父母,仰之如日月,敬之如鬼神,畏之如雷霆」。茍須應痛逗藥,豈可不治之哉?今天下宰守所以皆尚貪殘,罕有廉白者,良由風俗侈靡。使之然也。淫奢之弊,其事多端,粗舉二條,言其尤者。夫食方丈於前,所甘一味。今之燕喜,相競誇豪,積果如山岳,列肴同綺繡,露臺之產,不周一燕之資,
 而賓主之間,裁取滿腹,未及下堂,已同臭腐。又歌姬儛女,本有品制,二八之錫,良待和戎。今畜妓之夫,無有等秩,雖復庶賤微人,皆盛姬姜,務在貪污,爭飾羅綺。故為吏牧民者,競為剝削,雖致貲巨億,罷歸之日,不支數年,便已消散。蓋由宴醑所費,既破數家之產;歌謠之具,必俟千金之資。所費事等丘山,為歡止在俄頃。乃更追恨向所取之少,今所費之多。如復傅翼,增其搏噬,一何悖哉!其餘淫侈,著之凡百,習以成俗,日見滋甚,欲使人守廉隅,吏尚清白,安可得邪!今誠宜嚴為禁制,道之以節儉,貶黜雕飾,糾奏浮華,使眾皆知,變其耳目,改其好惡。
 夫失節之嗟,亦民所自患,正恥不及群,故勉彊而為之,茍力所不至,還受其弊矣。今若釐其風而正其失,易於反掌。夫論至治者,必以淳素為先,正雕流之弊,莫有過儉樸者也。



 其三事曰:聖躬荷負蒼生以為任,弘濟四海以為心,不憚胼胝之勞,不辭臒瘦之苦,豈止日昃忘飢,夜分廢寢。至於百司,莫不奏事,上息責下之嫌,下無逼上之咎,斯實道邁百王,事超千載。但斗筲之人,藻棁之子,既得伏奏帷扆,便欲詭競求進,不說國之大體。不知當一官,處一職,貴使理其紊亂,匡其不及,心在明恕,事乃平章。但務吹毛求疵,擘肌分理,運挈瓶之智,徼分外
 之求,以深刻為能,以繩逐為務,迹雖似於奉公,事更成其威福。犯罪者多,巧避滋甚,曠官廢職,長弊增姦,實由於此。今誠願責其公平之效,黜其讒愚之心,則下安上謐,無僥倖之患矣。



 其四事曰:自征伐北境,帑藏空虛。今天下無事,而猶日不暇給者,良有以也。夫國弊則省其事而息其費,事省則養民,費息則財聚,止五年之中,尚於無事,必能使國豊民阜。若積以歲月,斯乃范蠡滅吳之術,管仲霸齊之由。今應內省職掌,各檢其所部。凡京師治、署、邸、肆應所為,或十條宜省其五,或三條宜除其一;及國容、戎備,在昔應多,在今宜少。雖於後應多,即事未
 須,皆悉減省。應四方屯、傳、邸、治,或舊有,或無益,或妨民,有所宜除,除之;有所宜減,減之。凡厥興造,凡厥費財,有非急者,有役民者;又凡厥討召,凡厥徵求,雖關國計,權其事宜,皆須息費休民。不息費,則無以聚財;不休民,則無以聚力。故蓄其財者,所以大用之也;息其民者,所以大役之也。若言小事不足害財,則終年不息矣;以小役不足妨民,則終年不止矣。擾其民而欲求生聚殷阜,不可得矣。耗其財而務賦斂繁興,則姦詐盜竊彌生,是弊不息而其民不可使也,則難可以語富彊而圖遠大矣。自普通以來,二十餘年,刑役薦起,民力彫流。今魏氏和親,
 疆埸無警,若不及於此時大息四民,使之生聚,減省國費,令府庫蓄積,一旦異境有虞,關河可掃,則國弊民疲,安能振其遠略?事至方圖,知不及矣。



 書奏,高祖大怒,召主書於前,口授敕責琛曰:謇謇有聞,殊稱所期。但朕有天下四十餘年,公車讜言,見聞聽覽,所陳之事,與卿不異,常欲承用,無替懷抱,每苦倥傯,更增惛惑。卿珥貂紆組,博問洽聞,不宜同於郤茸,止取名字,宣之行路。言「我能上事,明言得失,恨朝廷之不能用」。或誦《離騷》「蕩蕩其無人,遂不御乎千里」。或誦《老子》「知我者希,則我貴矣」。如是獻替,莫不能言,正旦虎樽,皆其人也。卿可分別言事,
 啟乃心,沃朕心。



 卿云「今北邊稽服,政是生聚教訓之時,而民失安居,牧守之過」。朕無則哲之知,觸向多弊,四聰不開,四明不達,內省責躬,無處逃咎。堯為聖主,四凶在朝;況乎朕也,能無惡人?但大澤之中,有龍有蛇,縱不盡善,不容皆惡。卿可分明顯出:某刺史橫暴,某太守貪殘,某官長凶虐;尚書、蘭臺,主書、舍人,某人姦猾,某人取與,明言其事,得以黜陟。向令舜但聽公車上書,四凶終自不知,堯亦永為闇主。



 卿又云「東境戶口空虛,良由使命繁多」,但未知此是何使?卿云「駑困邑宰,則拱手聽其漁獵;桀黠長吏,又因之而為貪殘」,並何姓名?廉平掣肘,復
 是何人?朝廷思賢,有如饑渴,廉平掣肘,實為異事。宜速條聞,當更擢用。凡所遣使,多由民訟,或復軍糧,諸所飆急,蓋不獲已而遣之。若不遣使,天下枉直云何綜理?事實云何濟辦?惡人日滋,善人日蔽,欲求安臥,其可得乎!不遣使而得事理,此乃佳事。無足而行,無翼而飛,能到在所;不威而伏,豈不幸甚。卿既言之,應有深見,宜陳秘術,不可懷寶迷邦。



 卿又云:守宰貪殘,皆由滋味過度。貪殘糜費,已如前答。漢文雖愛露臺之產,鄧通之錢布於天下,以此而治,朕無愧焉。若以下民飲食過差,亦復不然。天監之初,思之已甚。其勤力營產,則無不富饒;惰遊
 緩事,則家業貧窶。勤脩產業,以營盤案,自己營之,自己食之,何損於天下?無賴子弟,惰營產業,致於貧窶,無可施設,此何益於天下?且又意雖曰同富,富有不同:慳而富者,終不能設;奢而富者,於事何損?若使朝廷緩其刑,此事終不可斷;若急其制,則曲屋密房之中,云何可知?若家家搜檢,其細已甚,欲使吏不呼門,其可得乎?更相恐脅,以求財帛,足長禍萌,無益治道。若以此指朝廷,我無此事。昔之牲牢,久不宰殺,朝中會同,菜蔬而已,意粗得奢約之節。若復減此,必有《蟋蟀》之譏。若以為功德事者,皆是園中之所產育。功德之事,亦無多費,變一瓜為
 數十種,食一菜為數十味,不變瓜菜,亦無多種,以變故多,何損於事,亦豪芥不關國家。如得財如法而用,此不愧乎人。我自除公宴,不食國家之食,多歷年稔,乃至宮人,亦不食國家之食,積累歲月。凡所營造,不關材官,及以國匠,皆資雇借,以成其事。近之得財,頗有方便,民得其利,國得其利,我得其利,營諸功德。或以卿之心度我之心,故不能得知。所得財用,暴於天下,不得曲辭辯論。



 卿又云女妓越濫,此有司之責,雖然,亦有不同:貴者多畜妓樂,至於勛附若兩掖,亦復不聞家有二八,多畜女妓者。此並宜具言其人,當令有司振其霜豪。卿又云:「乃
 追恨所取為少,如復傅翼,增其搏噬,一何悖哉。」勇怯不同,貪廉各用,勇者可使進取,怯者可使守城,貪者可使捍禦,廉者可使牧民。向使叔齊守於西河,豈能濟事?吳起育民,必無成功。若使吳起而不重用,則西河之功廢。今之文武,亦復如此。取其搏噬之用,不能得不重更任,彼亦非為朝廷為之傅翼。卿以朝廷為悖,乃自甘之,當思致悖所以。卿云「宜導之以節儉」。又云「至治者必以淳素為先」。此言大善。夫子言「其身正,不令而行;其身不正,雖令不從」。朕絕房室三十餘年,無有淫佚。朕頗自計,不與女人同屋而寢,亦三十餘年。至於居處不過一床之
 地,雕飾之物不入於宮,此亦人所共知。受生不飲酒,受生不好音聲,所以朝中曲宴,未嘗奏樂,此群賢之所觀見。朕三更出理事,隨事多少,事少或中前得竟,或事多至日昃方得就食。日常一食,若晝若夜,無有定時。疾苦之日,或亦再食。昔要腹過於十圍,今之瘦削裁二尺餘,舊帶猶存,非為妄說。為誰為之?救物故也。《書》曰:「股肱惟人,良臣惟聖。」向使朕有股肱,故可得中主。今乃不免居九品之下,「不令而行」,徒虛言耳。卿今慊言,便罔知所答。



 卿又云「百司莫不奏事,詭競求進」。此又是誰?何者復是詭事?今不使外人呈事,於義可否?無人廢職,職可廢乎?職
 廢則人亂,人亂則國安乎?以咽廢飧,此之謂也。若斷呈事,誰尸其任?專委之人,云何可得?是故古人云:「專聽生姦,獨任成亂。」猶二世之委趙高,元后之付王莽。呼鹿為馬,卒有閻樂望夷之禍,王莽亦終移漢鼎。



 卿云「吹毛求疵」,復是何人所吹之疵?「擘肌分理」,復是何人乎?事及「深刻」「繩逐」,並復是誰?又云「治、署、邸、肆」,何者宜除?何者宜省?「國容戎備」,何者宜省?何者未須?「四方屯傳」,何者無益?何者妨民?何處興造而是役民?何處費財而是非急?若為「討召」?若為「征賦」?朝廷從來無有此事,靜息之方復何者?宜各出其事,具以奏聞。



 卿云「若不及於時大息其民,事
 至方圖,知無及也」。如卿此言,即時便是大役其民,是何處所?卿云「國弊民疲」,誠如卿言,終須出其事,不得空作漫語。夫能言之,必能行之。富國彊兵之術,急民省役之宜,號令遠近之法,並宜具列。若不具列,則是欺罔朝廷,空示頰舌。凡人有為,先須內省,惟無瑕者,可以戮人。卿不得歷詆內外,而不極言其事。佇聞重奏,當復省覽,付之尚書,班下海內,庶亂羊永除,害馬長息,惟新之美,復見今日。



 琛奉敕,但謝過而已,不敢復有指斥。



 久之,遷太府卿。太清二年,遷雲騎將軍、中軍宣城王長史。侯景舉兵襲京師,王移入臺內,留琛與司馬楊曒守東府。賊尋
 攻陷城,放兵殺害,琛被槍未至死,賊求得之,轝至闕下,求見僕射王克、領軍朱異,勸開城納賊。克等讓之,涕泣而止,賊復轝送莊嚴寺療治之。明年,臺城不守,琛逃歸鄉里。其年冬,賊進寇會稽,復執琛送出都,以為金紫光祿大夫。後遇疾卒,年六十九。



 琛所撰《三禮講疏》、《五經滯義》及諸儀法,凡百餘篇。



 子詡,太清初,自儀同西昌侯掾,出為巴山太守,在郡遇亂卒。



 陳吏部尚書姚察云:夏侯勝有言曰:「士患不明經術;經術明,取青紫如拾地芥耳。」朱異、賀琛並起微賤,以經術逢時,致於貴顯,符其言矣。而異遂徼寵幸,任事居權,不
 能以道佐君,茍取容媚。及延寇敗國,實異之由。禍難既彰,不明其罪,至於身死,寵贈猶殊。罰既弗加,賞亦斯濫,失於勸沮,何以為國?君子是以知太清之亂,能無及是乎。



\end{pinyinscope}