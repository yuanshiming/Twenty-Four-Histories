\article{卷第二十八列傳第二十二 裴邃兄子之高 之平 之橫 夏侯亶弟夔 魚弘附 韋放}

\begin{pinyinscope}

 裴邃,字淵明,河東聞喜人,魏襄州刺史綽之後也。祖壽孫,寓居壽陽,為宋武帝前軍長史。父仲穆,驍騎將軍。邃十歲能屬文,善《左氏春秋》。齊建武初,刺史蕭遙昌引為府主簿。壽陽有八公山廟,遙昌為立碑,使邃為文,甚見稱賞。舉秀才,對策高第,奉朝請。



 東昏踐阼,始安王蕭遙
 光為撫軍將軍、揚州刺史,引邃為參軍。後遙光敗,邃還壽陽,值刺史裴叔業以壽陽降魏,豫州豪族皆被驅掠,邃遂隨眾北徙。魏主宣武帝雅重之,以為司徒屬,中書郎,魏郡太守。魏遣王肅鎮壽陽,邃固求隨肅,密圖南歸。天監初,自拔還朝,除後軍諮議參軍。邃求邊境自效,以為輔國將軍、廬江太守。時魏將呂頗率眾五萬奄來攻郡,邃率麾下拒破之,加右軍將軍。



 五年,徵邵陽洲,魏人為長橋斷淮以濟。邃築壘逼橋,每戰輒克,於是密作沒突艦。會甚雨,淮水暴溢,邃乘艦徑造橋側,魏眾驚潰,邃乘勝追擊,大破之。進克羊石城,斬城主元康。又破霍丘
 城,斬城主甯永仁。平小峴,攻合肥。以功封夷陵縣子,邑三百戶。遷冠軍長史、廣陵太守。



 邃與鄉人共入魏武廟,因論帝王功業。其妻甥王篆之密啟高祖,云「裴邃多大言,有不臣之迹。」由是左遷為始安太守。邃志欲立功邊陲,不願閑遠,乃致書於呂僧珍曰:「昔阮咸、顏延有『二始』之歎。吾才不逮古人,今為三始,非其願也,將如之何!」未及至郡,會魏攻宿預,詔邃拒焉。行次直瀆,魏眾退。遷右軍諮議參軍、豫章王雲麾府司馬,率所領助守石頭。出為竟陵太守,開置屯田,公私便之。遷為游擊將軍、朱衣直閣,直殿省。尋遷假節、明威將軍、西戎校尉、北梁、秦二
 州刺史。復開創屯田數千頃,倉廩盈實,省息邊運,民吏獲安,乃相率餉絹千餘匹。邃從容曰:「汝等不應爾;吾又不可逆。」納其絹二匹而已。還為給事中、雲騎將軍、朱衣直閣將軍,遷大匠卿。



 普通二年,義州刺史文僧明以州叛入於魏,魏軍來援。以邃為假節、信武將軍,督眾軍討焉。邃深入魏境,從邊城道,出其不意。魏所署義州刺史封壽據檀公峴,邃擊破之,遂圍其城,壽面縛請降,義州平。除持節、督北徐州諸軍事、信武將軍、北徐州刺史。未之職,又遷督豫州、北豫、霍三州諸軍事、豫州刺史,鎮合肥。



 四年,進號宣毅將軍。是歲,大軍將北伐,以邃督征討
 諸軍事,率騎三千,先襲壽陽。九月壬戌,夜至壽陽,攻其郛,斬關而入,一日戰九合,為後軍蔡秀成失道不至,邃以援絕拔還。於是邃復整兵,收集士卒,令諸將各以服色相別。邃自為黃袍騎,先攻狄丘、甓城、黎漿等城,皆拔之。屠安成、馬頭、沙陵等戍。是冬,始修芍陂。明年,復破魏新蔡郡,略地至于鄭城,汝潁之間,所在響應。魏壽陽守將長孫稚、河間王元琛率眾五萬,出城挑戰。邃勒諸將為四甄以待之,令直閣將軍李祖憐偽遁以引稚,稚等悉眾追之,四甄競發,魏眾大敗。斬首萬餘級。稚等奔走,閉門自固,不敢復出。其年五月,卒於軍中。追贈侍中、左
 衛將軍,給鼓吹一部,進爵為侯,增邑七百戶。謚曰烈。



 邃少言笑,沉深有思略,為政寬明,能得士心。居身方正有威重,將吏憚之,少敢犯法。及其卒也,淮、肥間莫不流涕,以為邃不死,洛陽不足拔也。



 子之禮,字子義,自國子生推第,補邵陵王國左常侍、信威行參軍。王為南兗,除長流參軍,未行,仍留宿衛,補直閣將軍。丁父憂,服闋襲封,因請隨軍討壽陽,除雲麾將軍,遷散騎常侍。又別攻魏廣陵城,平之,除信武將軍、西豫州刺史,加輕車將軍,除黃門侍郎,遷中軍宣城王司馬。尋為都督北徐、仁、睢三州諸軍事、信武將軍、北徐州刺史。徵太子左衛率,兼衛
 尉卿,轉少府卿。卒,謚曰壯。子政,承聖中,官至給事黃門侍郎。江陵陷,隨例入西魏。



 之高字如山,邃兄中散大夫髦之子也。起家州從事、新都令、奉朝請,遷參軍。頗讀書,少負意氣,常隨叔父邃征討,所在立功,甚為邃所器重,戎政咸以委焉。壽陽之役,邃卒于軍所,之高隸夏侯夔,平壽陽,仍除平北豫章長史、梁郡太守,封都城縣男,邑二百五十戶。時魏汝陰來附,敕之高應接,仍除假節、飆勇將軍、潁州刺史。士民夜反,踰城而入,之高率家僮與麾下奮擊,賊乃散走。父憂還京。起為光遠將軍,合討陰陵盜賊,平之,以為譙州刺
 史。又還為左軍將軍,出為南譙太守、監北徐州,遷員外散騎常侍。尋除雄信將軍、西豫州刺史,餘如故。侯景亂,之高率眾入援,南豫州刺史、鄱陽嗣王範命之高總督江右援軍諸軍事,頓于張公洲。柳仲禮至橫江,之高遣船舸二百餘艘迎致仲禮,與韋粲等俱會青塘立營,據建興苑。及城陷,之高還合肥,與鄱陽王範西上。稍至新蔡,眾將一萬,未有所屬。元帝遣蕭慧正召之,以為侍中、護軍將軍。到江陵,承制除特進、金紫光祿大夫。卒,時年七十三。贈侍中、儀同三司,鼓吹一部。謚曰恭。子畿,累官太子右衛率、雋州刺史。西魏攻陷江陵,畿力戰死之。



 之平字如原,之高第五弟。少亦隨邃征討,以軍功封都亭侯。歷武陵王常侍、扶風、弘農二郡太守,不行,除譙州長史、陽平太守。拒侯景,城陷後,遷散騎常侍、右衛將軍、太子詹事。



 之橫字如岳,之高第十三弟也。少好賓遊,重氣俠,不事產業。之高以其縱誕,乃為狹被蔬食以激厲之。之橫歎曰:「大丈夫富貴,必作百幅被。」遂與僮屬數百人,於芍陂大營田墅,遂致殷積。太宗在東宮,聞而要之,以為河東王常侍、直殿主帥,遷直閣將軍。侯景亂,出為貞威將軍,隸鄱陽王範討景。景濟江,仍與範長子嗣入援。連營度
 淮,據東城。京都陷,退還合肥,與範溯流赴湓城。景遣任約上逼晉熙,範令之橫下援,未及至,範薨,之橫乃還。



 時尋陽王大心在江州,範副梅思立密要大心襲湓城,之橫斬思立而拒大心。大心以州降景。之橫率眾與兄之高同歸元帝,承制除散騎常侍、廷尉卿,出為河東內史。又隨王僧辯拒侯景於巴陵,景退,遷持節、平北將軍、東徐州刺史,中護軍,封豫寧侯,邑三千戶。又隨僧辯追景,平郢、魯、江、晉等州,恆為前鋒陷陣。仍至石頭,破景,景東奔,僧辯令之橫與杜掞入守臺城。及陸納據湘州叛,又隸王僧辯南討焉。於陣斬納將李賢明,遂平之。又破武
 陵王於硤口。還除吳興太守,乃作百幅被,以成其初志。



 後江陵陷,齊遣上黨王高渙挾貞陽侯攻東關,晉安王方智承制,以之橫為使持節、鎮北將軍、徐州刺史,都督眾軍,給鼓吹一部,出守蘄城。之橫營壘未周,而齊軍大至,兵盡矢窮,遂於陣沒,時年四十一。贈侍中、司空公,謚曰忠壯。子鳳寶嗣。



 夏侯亶,字世龍,車騎將軍詳長子也。齊初,起家奉朝請。永元末,詳為西中郎南康王司馬,隨府鎮荊州,亶留京師,為東昏聽政主帥。及崔慧景作亂,亶以捍禦功,除驍騎將軍。及高祖起師,詳與長史蕭穎胄協同義舉,密遣
 信下都迎亶,亶乃齎宣德皇后令,令南康王纂承大統,封十郡為宣城王,進位相國,置僚屬,選百官。建康城平,以亶為尚書吏部郎,俄遷侍中,奉璽於高祖。天監元年,出為宣城太守。尋入為散騎常侍,領右驍騎將軍。六年,出為平西始興王長史、南郡太守,父憂解職。居喪盡禮,廬于墓側,遺財悉推諸弟。八年,起為持節、督司州諸軍事、信武將軍、司州刺史,領安陸太守。服闋,襲封豊城縣公。居州甚有威惠,為邊人所悅服。十二年,以本號還朝,除都官尚書,遷給事中、右衛將軍、領豫州大中正。十五年,出為信武將軍、安西長史、江夏太守。十七年,入為通
 直散騎常侍、太子右衛率,遷左衛將軍,領前軍將軍。俄出為明威將軍、吳興太守。在郡復有惠政,吏民圖其像,立碑頌美焉。普通三年,入為散騎常侍,領右驍騎將軍,轉太府卿,常侍如故。以公事免,未幾,優詔復職。五年,遷中護軍。



 六年,大舉北伐。先遣豫州刺史裴邃帥譙州刺史湛僧智、歷陽太守明紹世、南譙太守魚弘、晉熙太守張澄,並世之驍將,自南道伐壽陽城,未克而邃卒。乃加亶使持節,馳驛代邃,與魏將河間王元琛、臨淮王元彧等相拒,頻戰克捷。尋有密敕,班師合肥,以休士馬,須堰成復進。七年夏,淮堰水盛,壽陽城將沒,高祖復遣北道
 軍元樹帥彭寶孫、陳慶之等稍進,亶帥湛僧智、魚弘、張澄等通清流澗,將入淮、肥。魏軍夾肥築城,出亶軍後,亶與僧智還襲,破之。進攻黎漿,貞威將軍韋放自北道會焉。兩軍既合,所向皆降下。凡降城五十二,獲男女口七萬五千人,米二十萬石。詔以壽陽依前代置豫州,合肥鎮改為南豫州,以亶為使持節、都督豫州緣淮南豫霍義定五州諸軍事、雲麾將軍、豫、南豫二州刺史。壽春久罹兵荒,百姓多流散,亶輕刑薄賦,務農省役,頃之民戶充復。大通二年,進號平北將軍。三年,卒於州鎮。高祖聞之,即日素服舉哀,贈車騎將軍。謚曰襄。州民夏侯簡等五
 百人表請為亶立碑置祠,詔許之。



 亶為人美風儀,寬厚有器量,涉獵文史,辯給能專對。宗人夏侯溢為衡陽內史,辭日,亶侍御坐,高祖謂亶曰:「夏侯溢於卿疏近?」稟答曰:「是臣從弟。」高祖知溢於亶已疏,乃曰:「卿傖人,好不辨族從。」亶對曰:「臣聞服屬易疏,所以不忍言族。」時以為能對。



 亶歷為六郡三州,不修產業,祿賜所得,隨散親故。性儉率,居處服用,充足而已,不事華侈。晚年頗好音樂,有妓妾十數人,並無被服姿容。每有客,常隔簾奏之,時謂簾為夏侯妓衣也。



 亶二子:誼,損。誼襲封豊城公,歷官太子舍人,洗馬。太清中,侯景入寇,誼與弟損帥部曲入城,
 並卒圍內。



 夔字季龍,亶弟也。起家齊南康王府行參軍。中興初,遷司徒屬。天監元年,為太子洗馬,中舍人,中書郎。丁父憂,服闋,除大匠卿,知造太極殿事。普通元年,為邵陵王信威長史,行府國事。其年,出為假節、征遠將軍,隨機北討,還除給事黃門侍郎。二年,副裴邃討義州,平之。三年,代兄亶為吳興太守,尋遷假節、征遠將軍、西陽、武昌二郡太守。七年,徵為衛尉,未拜,改授持節、督司州諸軍事、信武將軍、司州刺史,領安陸太守。



 八年,敕夔帥壯武將軍裴之禮、直閣將軍任思祖出義陽道,攻平靜、穆陵、陰山
 三關,克之。是時譙州刺史湛僧智圍魏東豫州刺史元慶和於廣陵,入其郛。魏將元顯伯率軍赴援,僧智逆擊破之,夔自武陽會僧智,斷魏軍歸路。慶和於內築柵以自固,及夔至,遂請降。夔讓僧智,僧智曰:「慶和志欲降公,不願降僧智,今往必乖其意;且僧智所將為烏合募人,不可御之以法。公持軍素嚴,必無犯令,受降納附,深得其宜。」於是夔乃登城拔魏幟,建官軍旗鼓,眾莫敢妄動,慶和束兵以出,軍無私焉。凡降男女口四萬餘人,粟六十萬斛,餘物稱是。顯伯聞之夜遁,眾軍追之,生擒二萬餘人,斬獲不可勝數。詔以僧智領東豫州,鎮廣陵。夔引軍
 屯安陽。夔又遣偏將屠楚城,盡俘其眾,由是義陽北道遂與魏絕。



 大通二年,魏郢州刺史元願達請降,高祖敕郢州刺史元樹往迎願達,夔亦自楚城會之,遂留鎮焉。詔改魏郢州為北司州,以夔為刺史,兼督司州。三年,遷使持節,進號仁威將軍,封保城縣侯,邑一千五百戶。中大通二年,徵為右衛將軍,丁所生母憂去職。



 時魏南兗州刺史劉明以譙城入附,詔遣鎮北將軍元樹帥軍應接,起夔為雲麾將軍,隨機北討。尋授使持節、督南豫州諸軍事、南豫州刺史。六年,轉使持節、督豫、淮、陳、潁、建、霍、義七州諸軍事、豫州刺史。豫州積歲寇戎,人頗失業,夔乃帥
 軍人於蒼陵立堰,溉田千餘頃。歲收穀百餘萬石,以充儲備,兼贍貧人,境內賴之。夔兄亶先經此任,至是夔又居焉。兄弟並有恩惠於鄉里,百姓歌之曰:「我之有州,頻仍夏侯;前兄後弟,布政優優。」在州七年,甚有聲績,遠近多附之。有部曲萬人,馬二千匹,並服習精強,為當時之盛。性奢豪,後房伎妾曳羅縠飾金翠者亦有百數。愛好人士,不以貴勢自高,文武賓客常滿坐,時亦以此稱之。大同四年,卒於州,時年五十六。有詔舉哀,賻錢二十萬,布二百匹。追贈侍中、安北將軍。謚曰桓。



 子撰嗣,官至太僕卿。撰弟譒,少麤險薄行,常停鄉里,領其父部曲,為州
 助防,刺史蕭淵明引為府長史。淵明彭城戰沒,復為侯景長史。景尋舉兵反,譒前驅濟江,頓兵城西士林館,破掠邸第及居人富室,子女財貨,盡略有之。淵明在州有四妾,章、於、王、阮,並有國色。淵明沒魏,其妾並還京第,譒至,破第納焉。



 魚弘,襄陽人。身長八尺,白皙美姿容。累從征討,常為軍鋒,歷南譙、盱眙、竟陵太守。常語人曰:「我為郡,所謂四盡:水中魚鱉盡,山中麞鹿盡,田中米穀盡,村里民庶盡。丈夫生世,如輕塵栖弱草,白駒之過隙。人生歡樂富貴幾何時!」於是恣意酣賞,侍妾百餘人,不勝金翠,服玩車
 馬,皆窮一時之絕。遷為平西湘東王司馬、新興、永寧二郡太守,卒官。



 韋放,字元直,車騎將軍睿之子。初為齊晉安王寧朔迎主簿,高祖臨雍州,又召為主簿。放身長七尺七寸,腰帶八圍,容貌甚偉。天監元年,為盱眙太守,還除通直郎,尋為輕車晉安王中兵參軍,遷鎮右始興王諮議參軍,以父憂去職。服闋,襲封永昌縣侯,出為輕車南平王長史、襄陽太守。轉假節、明威將軍、竟陵太守。在郡和理,為吏民所稱。六年,大舉北伐,以放為貞威將軍,與胡龍牙會曹仲宗進軍。七年,夏侯亶攻黎漿不克,高祖復使帥軍
 自北道會壽春城。尋遷雲麾南康王長史、尋陽太守。放累為籓佐,並著聲績。



 普通八年,高祖遣兼領軍曹仲宗等攻渦陽,又以放為明威將軍,帥師會之。魏大將費穆帥眾奄至,放軍營未立,麾下止有二百餘人。放從弟洵驍果有勇力,一軍所仗,放令洵單騎擊刺,屢折魏軍,洵馬亦被傷不能進,放胄又三貫流矢。眾皆失色,請放突去。放厲聲叱之曰:「今日唯有死耳。」乃免胄下馬,據胡床處分。於是士皆殊死戰,莫不一當百。魏軍遂退,放逐北至渦陽。魏又遣常山王元昭、大將軍李獎、乞佛寶、費穆等眾五萬來援,放率所督將陳度、趙伯超等夾擊,大破
 之。渦陽城主王緯以城降。放乃登城,簡出降口四千二百人,器仗充牛刃;又遣降人三十,分報李獎、費穆等。魏人棄諸營壘,一時奔潰,眾軍乘之,斬獲略盡。擒穆弟超,并王緯送於京師。還為太子右衛率,轉通直散騎常侍。出為持節、督梁、南秦二州諸軍事、信武將軍、梁、南秦二州刺史。中大通二年,徙督北徐州諸軍事、北徐州刺史,增封四百戶,持節、將軍如故。在鎮三年,卒,時年五十九。謚曰宜侯。



 放性弘厚篤實,輕財好施,於諸弟尤雍睦。每將遠別及行役初還,常同一室臥起,時稱為「三姜」。初,放與吳郡張率皆有側室懷孕,因指為婚姻。其後各產男女,
 未及成長而率亡,遺嗣孤弱,放常贍恤之。及為北徐州,時有勢族請姻者,放曰:「吾不失信於故友。」乃以息岐娶率女,又以女適率子,時稱放能篤舊。長子粲嗣,別有傳。



 史臣曰:裴邃之詞採早著,兼思略沉深,夏侯稟之好學辯給,夔之奢豪愛士,韋放之弘厚篤行,並遇主逢時,展其才用矣。及牧州典郡,破敵安邊,咸著功績,允文武之任,蓋梁室之名臣歟。



\end{pinyinscope}