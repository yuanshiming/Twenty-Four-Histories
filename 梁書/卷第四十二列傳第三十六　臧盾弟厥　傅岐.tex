\article{卷第四十二列傳第三十六 臧盾弟厥 傅岐}

\begin{pinyinscope}

 臧盾,字宣卿,東莞莒人。高祖燾,宋左光祿大夫。祖潭之,左民尚書。父未甄,博涉文史,有才幹,少為外兄汝南周顒所知。宋末,起家為領軍主簿,所奉即齊武帝。入齊,歷太尉祭酒、尚書主客郎、建安、廬陵二王府記室、前軍功曹史、通直郎、南徐州中正、丹陽尹丞。高祖平京邑,霸府建,引為驃騎刑獄參軍。天監初,除後軍諮議中郎、南徐
 州別駕,入拜黃門郎,遷右軍安成王長史、少府卿。出為新安太守,有能名。還為太子中庶子、司農卿、太尉長史。丁所生母憂,三年廬于墓側。服闋,除廷尉卿。出為安成王長史、江夏太守,卒官。



 盾幼從征士瑯邪諸葛璩受《五經》,通章句。璩學徒常有數十百人,盾處其間,無所狎比。璩異之,歎曰:「此生重器,王佐才也。」初為撫軍行參軍,遷尚書中兵郎。盾美風姿,善舉止,每趨奏,高祖甚悅焉。入兼中書通事舍人,除安右錄事參軍,舍人如故。



 盾有孝性,隨父宿直於廷尉,母劉氏在宅,夜暴亡,左手中指忽痛,不得寢。及曉,宅信果報凶問,其感通如此。服制未終,
 父又卒,盾居喪五年,不出廬戶,形骸枯悴,家人不復識。鄉人王端以狀聞,高祖嘉之,敕累遣抑譬。服闋,除丹陽尹丞,轉中書郎,復兼中書舍人,遷尚書左丞,為東中郎武陵王長史,行府州國事,領會稽郡丞。還除少府卿,領步兵校尉,遷御史中丞。盾性公彊,居憲臺甚稱職。



 中大通五年二月,高祖幸同泰寺開講,設四部大會,眾數萬人。南越所獻馴象,忽於眾中狂逸,乘轝羽衛及會皆駭散,惟盾與散騎郎裴之禮嶷然自若,高祖甚嘉焉。俄有詔,加散騎常侍,未拜,又詔曰:「總一六軍,非才勿授。御史中丞、新除散騎常侍盾,志懷忠密,識用詳慎,當官平允,
 處務勤恪,必能緝斯戎政。可兼領軍,常侍如故。」大同二年,遷中領軍。領軍管天下兵要,監局事多。盾為人敏贍,有風力,長於撥繁,職事甚理。天監中,吳平侯蕭景居此職,著聲稱。至是,盾復繼之。



 五年,出為仁威將軍、吳郡太守,視事未期,以疾陳解。拜光祿大夫,加金章紫綬。七年,疾愈,復為領軍將軍。九年,卒,時年六十六。即日有詔舉哀。贈侍中,領軍如故。給東園秘器,朝服一具,衣一襲,錢布各有差。謚曰忠。



 子長博,字孟弘,桂陽內史。次子仲博,曲阿令。盾弟厥。



 厥,字獻卿,亦以幹局稱。初為西中郎行參軍、尚書主客
 郎。入兼中書通事舍人,累遷正員郎、鴻臚卿,舍人如故。遷尚書右丞,未拜,出為晉安太守。郡居山海,常結聚逋逃,前二千石雖募討捕,而寇盜不止。厥下車,宣風化,凡諸凶黨,皆涘負而出,居民復業,商旅流通。然為政嚴酷少恩,吏民小事必加杖罰,百姓謂之「臧虎」。還除驃騎廬陵王諮議參軍,復兼舍人。遷員外散騎常侍,兼司農卿,舍人如故。大同八年,卒官,時年四十八。厥前後居職,所掌之局大事及蘭臺廷尉所不能決者,敕並付厥。厥辨斷精詳,咸得其理。厥卒後,有撾登聞鼓訴者,求付清直舍人。高祖曰:「臧厥既亡,此事便無可付。」其見知如此。



 子
 操,尚書三公郎。



 傅岐,字景平,北地靈州人也。高祖弘仁,宋太常。祖琰,齊世為山陰令,有治能,自縣擢為益州刺史。父翽,天監中,歷山陰、建康令,亦有能名,官至驃騎諮議。



 岐初為國子明經生,起家南康王宏常侍,遷行參軍,兼尚書金部郎。母憂去職,居喪盡禮。服闋後,疾廢久之。是時改創北郊壇,初起岐監知繕築,事畢,除如新令。縣民有因鬥相毆而死者,死家訴郡,郡錄其仇人,考掠備至,終不引咎,郡乃移獄於縣。岐即命脫械,以和言問之,便即首服。法當償死,會冬節至,岐乃放其還家,使過節一日復獄。曹掾
 固爭曰:「古者乃有此,於今不可行。」岐曰:「其若負信,縣令當坐,主者勿憂。」竟如期而反。太守深相歎異,遽以狀聞。岐後去縣,民無老小,皆出境拜送,啼號之聲,聞於數十里。至都,除廷尉正,入兼中書通事舍人,遷寧遠岳陽王記室參軍,舍人如故。出為建康令,以公事免。俄復為舍人,累遷安西中記室、鎮南諮議參軍,兼舍人如故。



 岐美容止,博涉能占對。大同中,與魏和親,其使歲中再至,常遣岐接對焉。太清元年,累遷太僕、司農卿,舍人如故。在禁省十餘年,機事密勿亞於朱異。此年冬,豫州刺史貞陽侯蕭淵明率眾伐彭城,兵敗陷魏。二年,淵明遣使還,
 述魏人欲更通和好,敕有司及近臣定議。左衛朱異曰:「高澄此意,當復欲繼好,不爽前和;邊境且得靜寇息民,於事為便。」議者並然之。岐獨曰:「高澄既新得志,其勢非弱,何事須和?此必是設間,故令貞陽遣使,令侯景自疑當以貞陽易景。景意不安,必圖禍亂。今若許澄通好,正是墮其計中。且彭城去歲喪師,渦陽新復敗退,令便就和,益示國家之弱。若如愚意,此和宜不可許。」朱異等固執,高祖遂從異議。及遣和使,侯景果有此疑,累啟請追使,敕但依違報之。至八月,遂舉兵反。十月,入寇京師,請誅朱異。三年,遷中領軍,舍人如故。二月,景於闕前通表,
 乞割江右四州,安其部下,當解圍還鎮,敕許之。乃於城西立盟,求遣宣城王出送。岐固執宣城嫡嗣之重,不宜許,遣石城公大款送之。及與景盟訖,城中文武喜躍,望得解圍。岐獨言於眾曰:「賊舉兵為逆,未遂求和,夷情獸心,必不可信,此和終為賊所詐也。」眾並怨怪之。及景背盟,莫不歎服。尋有詔,以岐勤勞,封南豊縣侯,邑五百戶,固辭不受。宮城失守,岐帶疾出圍,卒於宅。



 陳吏部尚書姚察曰:夫舉事者定於謀,故萬舉無遺策,信哉是言也。傅岐識齊氏之偽和,可謂善於謀事。是時若納岐之議,太清禍亂,固其不作。申子曰:「一言倚,天下
 靡。」此之謂乎?



\end{pinyinscope}