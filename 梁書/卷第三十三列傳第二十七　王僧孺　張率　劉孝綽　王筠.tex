\article{卷第三十三列傳第二十七 王僧孺 張率 劉孝綽 王筠}

\begin{pinyinscope}

 王僧孺,字僧孺,東海郯人,魏衛將軍肅八世孫。曾祖雅,晉左光祿大夫、儀同三司。祖準,宋司徒左長史。



 僧孺年五歲,讀《孝經》,問授者此書所載述,曰:「論忠孝二事。」僧孺曰:「若爾,常願讀之。」六歲能屬文,既長好學。家貧,常傭書以養母,所寫既畢,諷誦亦通。



 仕齊,起家王國左常侍、太學博士。尚書僕射王晏深相賞好。晏為丹陽尹,召補郡
 功曹,使僧孺撰《東宮新記》。遷大司馬豫章王行參軍,又兼太學博士。司徒竟陵王子良開西邸招文學,僧孺亦遊焉。文惠太子聞其名,召入東宮,直崇明殿。欲擬為宮僚,文惠薨,不果。時王晏子德元出為晉安郡,以僧孺補郡丞,除候官令。建武初,有詔舉士,揚州刺史始安王遙光表薦秘書丞王暕及僧孺曰:「前候官令東海王僧孺,年三十五,理尚棲約,思致悟敏,既筆耕為養,亦傭書成學。至乃照螢映雪,編蒲緝柳,先言往行,人物雅俗,甘泉遺儀,南宮故事,畫地成圖,抵掌可述;豈直鼮鼠有必對之辯,竹書無落簡之謬,訪對不休,質疑斯在。」除尚書儀
 曹郎,遷治書侍御史,出為錢唐令。



 初,僧孺與樂安任昉遇竟陵王西邸,以文學友會,及是將之縣,昉贈詩,其略曰:「惟子見知,惟餘知子。觀行視言,要終猶始。敬之重之,如蘭如芷。形應影隨,曩行今止。百行之首,立人斯著。子之有之,誰毀誰譽。修名既立,老至何遽。誰其執鞭,吾為子御。劉《略》班《藝》,虞《志》荀《錄》,伊昔有懷,交相欣勖。下帷無倦,升高有屬。嘉爾晨燈,惜餘夜燭。」其為士友推重如此。



 天監初,除臨川王後軍記室參軍,待詔文德省。尋出為南海太守。郡常有高涼生口及海舶每歲數至,外國賈人以通貨易。舊時州郡以半價就市,又買而即賣,其利數
 倍,歷政以為常。僧孺乃歎曰:「昔人為蜀部長史,終身無蜀物,吾欲遺子孫者,不在越裝。」並無所取。視事期月,有詔徵還,郡民道俗六百人詣闕請留,不許。既至,拜中書郎、領著作,復直文德省,撰《中表簿》及《起居注》。遷尚書左丞,領著作如故。俄除游擊將軍,兼御史中丞。僧孺幼貧,其母鬻紗布以自業,嘗攜僧孺至市,道遇中丞鹵簿,驅迫溝中。及是拜日,引騶清道,悲感不自勝。尋以公事降為雲騎將軍,兼職如故,頃之即真。是時高祖製《春景明志詩》五百字,敕在朝之人沈約已下同作,高祖以僧孺詩為工。遷少府卿,出監吳郡。還除尚書吏部郎,參大選,
 請謁不行。



 出為仁威南康王長史,行府、州、國事。王典簽湯道愍暱於王,用事府內,僧孺每裁抑之,道愍遂謗訟僧孺,逮詣南司。奉箋辭府曰:「下官不能避溺山隅,而正冠李下,既貽疵辱,方致徽繩,解籙收簪,且歸初服。竊以董生偉器,止相驕王;賈子上才,爰傅卑土。下官生年有值,謬仰清塵,假翼西雍,竊步東閣,多慚袨服,取亂長裾,高榻相望,直居坐右,長階如畫,獨在僚端。借其從容之詞,假以寬和之色,恩禮遠過申、白,榮望多廁應、徐。厚德難逢,小人易說。方謂離腸隕首,不足以報一言;露膽披誠,何能以酬屢顧。寧謂罻羅裁舉,微禽先落;閶闔始吹,
 細草仍墜。一辭九畹,方去五雲。縱天網是漏,聖恩可恃,亦復孰寄心骸,何施眉目。方當橫潭亂海,就魚鱉而為群;披榛捫樹,從虺蛇而相伍。豈復仰聽金聲,式瞻玉色。顧步高軒,悲如霰委;踟躕下席,淚若綆縻。」



 僧孺坐免官,久之不調。友人廬江何炯猶為王府記室,乃致書於炯,以見其意。曰:近別之後,將隔暄寒,思子為勞,未能忘弭。昔李叟入秦,梁生適越,猶懷悵恨,且或吟謠;況歧路之日,將離嚴網,辭無可憐,罪有不測。蓋畫地刻木,昔人所惡,叢棘既累,於何可聞,所以握手戀戀,離別珍重。弟愛同鄒季,淫淫承睫,吾猶復抗手分背,羞學婦人。素鐘肇
 節,金飆戒序,起居無恙,動靜履宜。子雲筆札,元瑜書記,信用既然,可樂為甚。且使目明,能祛首疾。甚善甚善。



 吾無昔人之才而有其病,癲眩屢動,消渴頻增。委化任期,故不復呼醫飲藥。但恨一旦離大辱,蹈明科,去皎皎而非自汙,抱鬱結而無誰告。丁年蓄積,與此銷亡,徒竊高價厚名,橫叨公器人爵,智能無所報,筋力未之酬,所以悲至撫膺,泣盡而繼之以血。



 顧惟不肖,文質無所底,蓋困於衣食,迫於飢寒,依隱易農,所志不過鐘庾。久為尺板斗食之吏,以從皁衣黑綬之役,非有奇才絕學,雄略高謨,吐一言可以匡俗振民,動一議可以固邦興國。全
 璧歸趙,飛矢救燕,偃息籓魏,甘臥安郢,腦日逐,髓月支,擁十萬而橫行,提五千而深入,將能執圭裂壤,功勒景鐘,錦繡為衣,朱丹被轂,斯大丈夫之志,非吾曹之所能及已。直以章句小才,蟲篆末藝,含吐緗縹之上,翩躚樽俎之側,委曲同之鍼縷,繁碎譬之米鹽,孰致顯榮,何能至到。加性疏澀,拙於進取,未嘗去來許、史,遨遊梁、竇,俯首脅肩,先意承旨。是以三葉靡遘,不與運并,十年未徙,孰非能薄。及除舊布新,清晷方旦,抱樂銜圖,訟謳有主,而猶限一吏於岑石,隔千里於泉亭,不得奉板中涓,預衣裳之會,提戈後勁,廁龍豹之謀。及其投劾歸來,恩均
 舊隸,升文石,登玉陛,一見而降顏色,再睹而接話言,非藉左右之容,無勞群公之助。又非同席共研之夙逢,笥餌卮酒之早識,一旦陪武帳,仰文陛,備聃、佚之柱下,充嚴、朱之席上,入班九棘,出專千里,據操撮之雄官,參人倫之顯職,雖古之爵人不次,取士無名,未有躡影追風,奔驟之若此者也。



 蓋基薄牆高,途遙力躓,傾蹶必然,顛匐可俟。竟以福過災生,人指鬼瞰,將均宥器,有驗傾卮,是以不能早從曲影,遂乃取疑邪徑。故司隸懍懍,思得應弦,譬縣廚之獸,如離繳之鳥,將充庖鼎,以餌鷹鸇。雖事異鑽皮,文非刺骨,猶復因茲舌杪,成此筆端,上可以
 投畀北方,次可以論輸左校,變為丹赭,充彼舂薪。幸聖主留善貸之德,紆好生之施,解網祝禽,下車泣罪,愍茲詬,憐其觳觫,加肉朽胔,布葉枯株,輟薪止火,得不銷爛。所謂還魂斗極,追氣泰山,止復除名為民,幅巾家巷,此五十年之後,人君之賜焉。木石感陰陽,犬馬識厚薄,員首方足,孰不戴天?而竊自有悲者,蓋士無賢不肖,在朝見嫉;女無美惡,入宮見妒。家貧,無苞苴可以事朋類,惡其鄉原,恥彼戚施,何以從人,何以徇物?外無奔走之友,內乏強近之親。是以構市之徒,隨相媒糵。及一朝捐棄,以快怨者之心,籲!可悲矣。



 蓋先貴後賤,古富今貧,季
 倫所以發此哀音,雍門所以和其悲曲。又迫以嚴秋殺氣,具物多悲,長夜展轉,百憂俱至。況復霜銷草色,風搖樹影。寒蟲夕叫,合輕重而同悲;秋葉晚傷,雜黃紫而俱墜。蜘蛛絡幕,熠耀爭飛,故無車轍馬聲,何聞鳴雞吠犬。俯眉事妻子,舉手謝賓遊。方與飛走為鄰,永用蓬蒿自沒。愾其長息,忽不覺生之為重。素無一廛之田,而有數口之累。豈曰匏而不食,方當長為傭保,糊口寄身,溘死溝渠,以實螻蟻。悲夫!豈復得與二三士友,抱接膝之歡,履足差肩,摛綺縠之清文,談希微之道德。唯吳馮之遇夏馥,范彧之值孔嵩,愍其留賃,憐此行乞耳。儻不以垢
 累,時存寸札,則雖先犬馬,猶松喬焉。去矣何生,高樹芳烈。裁書代面,筆淚俱下。



 久之,起為安西安成王參軍,累遷鎮右始興王中記室,北中郎南康王諮議參軍,入直西省,知撰譜事。普通三年,卒,時年五十八。



 僧孺好墳籍,聚書至萬餘卷,率多異本,與沈約、任昉家書相埒。少篤志精力,於書無所不睹。其文麗逸,多用新事,人所未見者,世重其富。僧孺集《十八州譜》七百一十卷,《百家譜集》十五卷,《東南譜集抄》十卷,文集三十卷,《兩臺彈事》不入集內為五卷,及《東宮新記》,並行於世。



 張率,字士簡,吳郡吳人。祖永,宋右光祿大夫。父瑰,齊世
 顯貴,歸老鄉邑,天監初,授右光祿,加給事中。率年十二,能屬文,常日限為詩一篇,稍進作賦頌,至年十六,向二千許首。齊始安王蕭遙光為揚州,召迎主簿,不就。起家著作佐郎。建武三年,舉秀才,除太子舍人。與同郡陸倕幼相友狎,常同載詣左衛將軍沈約,適值任昉在焉,約乃謂昉曰:「此二子後進才秀,皆南金也,卿可與定交。」由此與昉友善。遷尚書殿中郎。出為西中郎南康王功曹史,以疾不就。久之,除太子洗馬。高祖霸府建,引為相國主簿。天監初,臨川王已下並置友、學。以率為鄱陽王友,遷司徒謝朏掾,直文德待詔省。敕使抄乙部書,又使撰
 婦人事二十餘條,勒成百卷。使工書人瑯邪王深、吳郡范懷約、褚洵等繕寫,以給後宮。率又為《待詔賦》奏之,甚見稱賞。手敕答曰:「省賦殊佳。相如工而不敏,枚皋速而不工,卿可謂兼二子於金馬矣。」又侍宴賦詩,高祖乃別賜率詩曰:「東南有才子,故能服官政。餘雖慚古昔,得人今為盛。」率奉詔往返數首。其年,遷秘書丞,引見玉衡殿。高祖曰:「秘書丞天下清官,東南胄望未有為之者,今以相處,足為卿譽。」其恩遇如此。



 四年三月,禊飲華光殿。其日,河南國獻舞馬,詔率賦之,曰:臣聞「天用莫如龍,地用莫如馬。」故《禮》稱驪騵,《詩》誦騮駱。先景遺風之美,世所得
 聞;吐圖騰光之異,有時而出。洎我大梁,光有區夏,廣運自中,員照無外,日入之所,浮琛委贄,風被之域,越險效珍,軨服烏號之駿,篸駼豢龍之名。而河南又獻赤龍駒,有奇貌絕足,能拜善舞。天子異之,使臣作賦,曰:維梁受命四載,元符既臻,協律之事具舉,膠庠之教必陳,檀輿之用已偃,玉輅之御方巡。考帝文而率通,披皇圖以大觀。慶惟道而必先,靈匪聖其誰贊。見河龍之瑞唐,矚天馬之禎漢。既葉符而比德,且同條而共貫。詢國美於斯今,邁皇王於曩昔。散大明以燭幽,揚義聲而遠斥。固施之於不窮,諒無所乎朝夕。並承流以請吏,咸向風而率
 職。納奇貢於絕區,致龍媒於殊域。伊況古而赤文,爰在茲而朱翼。既效德於炎運,亦表祥於尚色。資皎月而載生,祖河房而挺授。種北唐之絕類,嗣西宛之鴻胄。稟妙足而逸倫,有殊姿而特茂。善環旋於薺夏,知蹈颻於金奏。超六種於周閑,踰八品於漢廄。伊自然之有質,寧改觀於肥瘦。豈徒服皁而養安,與進駕以馳驟。爾其挾尺縣鑿之辨,附蟬伏兔之別,十形五觀之姿,三毛八肉之勢,臣何得而稱焉,固已詳於前製。



 徒觀其神爽,視其豪異,軼跨野而忽踰輪,齊秀麒而並末駟。貶代盤而陋小華,越定單而少天驥。信無等於漏面,孰有取於決鼻。可
 以迹章、亥之所未遊,踰禹、益之所未至。將不得而屈指,亦何暇以理轡。若跡遍而忘反,非我皇之所事。方潤色於前古,邈深文而儲思。



 既而機事多暇,青春未移。時惟上巳,美景在斯。遵鎬飲之故實,陳洛宴之舊儀。漕伊川而分派,引激水以回池。集國良於民俊,列樹茂於皇枝。紛高冠以連衽,鏘鳴玉而肩隨。清輦道於上林,肅華臺之金座。望發色於綠苞,佇流芬於紫裹。聽磬寔之畢舉,聆《韶》、《夏》之咸播。承六奏之既闋,及九變之已成。均儀禽於唐序,同舞獸於虞庭。懷夏后之九代,想陳王之紫騂。乃命涓人,效良駿,經周衛,入鉤陳。言右牽之已來,寧執
 朴而後進。既傾首於律同,又蹀足於鼓振。擢龍首,回鹿軀,睨兩鏡,蹙雙鳧。既就場而雅拜,時赴曲而徐趨。敏躁中於促節,捷繁外於驚桴。騏行驥動,虎發龍驤;雀躍燕集,鵠引鳧翔。妍七盤之綽約,陵九劍之抑揚。豈借儀於褕袂,寧假器於髦皇。婉脊投頌,俯膺合雅。露沫歕紅,沾汗流赭。乃卻走於集靈,馴惠養於豊夏。鬱風雷之壯心,思展足於南野。



 若彼符瑞之富,可以臻介丘而昭卒業,搢紳群后,誠希末光,天子深穆為度,未之訪也。何則?進讓殊事,豈非帝者之彌文哉。今四衛外封,五岳內郡,宜弘下禪之規,增上封之訓,背清都而日行,指云郊而玄
 運。將絕塵而弭轍,類飛鳥與駏驢。總三才而驅騖,按五御而超攄。翳卿雲於華蓋,翼條風於屬車。無逸御於玉軫,不泛駕於金輿。飾中岳之絕軌,營奉高之舊墟。訓厚況於人神,弘施育於黎獻。垂景炎於長世,集繁祉於斯萬,在庸臣之方剛,有從軍之大願。必自茲而展採,將同畀於庖煇。悼長卿之遺書,憫周南之留恨。



 時與到洽、周興嗣同奉詔為賦,高祖以率及興嗣為工。



 其年,父憂去職。其父侍妓數十人,善謳者有色貌,邑子儀曹郎顧玩之求娉焉,謳者不願,遂出家為尼。嘗因齋會率宅,玩之乃飛書言與率姦,南司以事奏聞,高祖惜其才,寢其奏,
 然猶致世論焉。



 服闋後,久之不仕。七年,敕召出,除中權建安王中記室參軍,預長名問訊,不限日。俄有敕直壽光省,治丙丁部書抄。八年,晉安王戍石頭,以率為雲麾中記室。王遷南兗州,轉宣毅諮議參軍,並兼記室。王還都,率除中書侍郎。十三年,王為荊州,復以率為宣惠諮議,領江陵令。王為江州,以諮議領記室,出監豫章、臨川郡。率在府十年,恩禮甚篤。還除太子僕,累遷招遠將軍、司徒右長史、揚州別駕。



 率雖歷居職務,未嘗留心簿領,及為別駕奏事,高祖覽牒問之,並無對,但奉答云「事在牒中」。高祖不悅。俄遷太子家令,與中庶子陸倕、僕射劉
 孝綽對掌東宮管記,遷黃門侍郎。出為新安太守,秩滿還都,未至,丁所生母憂。大通元年,服未闋,卒,時年五十三。昭明太子遣使贈賻,與晉安王綱令曰:「近張新安又致故。其人才筆弘雅,亦足嗟惜。隨弟府朝,東西日久,尤當傷懷也。比人物零落,特可潸慨,屬有今信,乃復及之。」



 率嗜酒,事事寬恕,於家務尤忘懷。在新安,遣家僮載米三千石還吳宅,既至,遂秏太半。率問其故,答曰:「雀鼠秏也。」率笑而言曰:「壯哉雀鼠。」竟不研問。少好屬文,而《七略》及《藝文志》所載詩賦,今亡其文者,並補作之。所著《文衡》十五卷,文集三十卷,行於世。子長公嗣。



 劉孝綽,字孝綽,彭城人,本名冉。祖勔,宋司空忠昭公。父繪,齊大司馬霸府從事中郎。孝綽幼聰敏,七歲能屬文。舅齊中書郎王融深賞異之,常與同載適親友,號曰神童。融每言曰:「天下文章,若無我當歸阿士。」阿士,孝綽小字也。繪,齊世掌詔誥。孝綽年未志學,繪常使代草之。父黨沈約、任昉、范雲等聞其名,並命駕先造焉,昉尤相賞好。范雲年長繪十餘歲,其子孝才與孝綽年並十四五,及雲遇孝綽,便申伯季,乃命孝才拜之。天監初,起家著作佐郎,為《歸沐詩》以贈任昉,昉報章曰:「彼美洛陽子,投我懷秋作。詎慰耋嗟人,徒深老夫託。直史兼褒貶,轄司
 專疾惡。九折多美疹,匪報庶良藥。子其崇鋒穎,春耕勵秋獲。」其為名流所重如此。



 遷太子舍人,俄以本官兼尚書水部郎,奉啟陳謝,手敕答曰:「美錦未可便製,簿領亦宜稍習。」頃之即真。高祖雅好蟲篆,時因宴幸,命沈約、任昉等言志賦詩,孝綽亦見引。嘗侍宴,於坐為詩七首,高祖覽其文,篇篇嗟賞,由是朝野改觀焉。



 尋有敕知青、北徐、南徐三州事,出為平南安成王記室,隨府之鎮。尋補太子洗馬,遷尚書金部侍郎,復為太子洗馬,掌東宮管記。出為上虞令,遷除秘書丞。高祖謂舍人周捨曰:「第一官當用第一人。」故以孝綽居此職。公事免。尋復除秘書
 丞,出為鎮南安成王諮議,入以事免。起為安西記室,累遷安西驃騎諮議參軍,敕權知司徒右長史事,遷太府卿、太子僕,復掌東宮管記。時昭明太子好士愛文,孝綽與陳郡殷芸、吳郡陸倕、瑯邪王筠、彭城到洽等,同見賓禮。太子起樂賢堂,乃使畫工先圖孝綽焉。太子文章繁富,群才咸欲撰錄,太子獨使孝綽集而序之。遷員外散騎常侍,兼廷尉卿,頃之即真。



 初,孝綽與到洽友善,同遊東宮。孝綽自以才優於洽,每於宴坐,嗤鄙其文,洽銜之。及孝綽為廷尉卿,攜妾入官府,其母猶停私宅。洽尋為御史中丞,遣令史案其事,遂劾奏之,云:「攜少妹於華省,
 棄老母於下宅。」高祖為隱其惡,改「妹」為「姝」。坐免官。孝綽諸弟,時隨籓皆在荊、雍,乃與書論共洽不平者十事,其辭皆鄙到氏。又寫別本封呈東宮,昭明太子命焚之,不開視也。



 時世祖出為荊州,至鎮,與孝綽書曰:「君屏居多暇,差得肆意典墳,吟詠情性,比復稀數古人,不以委約而能不伎癢;且虞卿、史遷由斯而作,想摛屬之興,益當不少。洛地紙貴,京師名動,彼此一時,何其盛也。近在道務閑,微得點翰,雖無紀行之作,頗有懷舊之篇。至此已來,眾諸屑役。小生之詆,恐取辱於廬江;遮道之姦,慮興謀於從事。方且褰帷自厲,求瘼不休,筆墨之功,曾何暇
 豫。至於心乎愛矣,未嘗有歇,思樂惠音,清風靡聞。譬夫夢想溫玉,飢渴明珠,雖愧卞、隨,猶為好事。新有所製,想能示之。勿等清慮,徒虛其請。無由賞悉,遣此代懷。數路計行,遲還芳札。」孝綽答曰:「伏承自辭皇邑,爰至荊臺,未勞刺舉,且摛高麗。近雖預觀尺錦,而不睹全玉。昔臨淄詞賦,悉與楊脩,未殫寶笥,顧慚先哲。渚宮舊俗,朝衣多故,李固之薦二邦,徐珍之奏七邑,威懷之道,兼而有之。當欲使金石流功,恥用翰墨垂迹。雖乖知二,偶達聖心。爰自退居素里,卻掃窮閈,比楊倫之不出,譬張摯之杜門。昔趙卿窮愁,肆言得失;漢臣鬱志,廣敘盛衰。彼此一
 時,擬非其匹。竊以文豹何辜,以文為罪。由此而談,又何容易。故韜翰吮墨,多歷寒暑,既闕子幼南山之歌,又微敬通渭水之賦,無以自同獻笑,少酬褒誘。且才乖體物,不擬作於玄根;事殊宿諾,寧貽懼於朱亥。顧己反躬,載懷累息。但瞻言漢廣,邈若天涯,區區一心,分宵九逝。殿下降情白屋,存問相尋,食椹懷音,矧伊人矣。」



 孝綽免職後,高祖數使僕射徐勉宣旨慰撫之,每朝宴常引與焉。及高祖為《籍田詩》,又使勉先示孝綽。時奉詔作者數十人,高祖以孝綽尤工,即日有敕,起為西中郎湘東王諮議。啟謝曰:「臣不能銜珠避顛,傾柯衛足,以茲疏倖,與物
 多忤。兼逢匿怨之友,遂居司隸之官,交構是非,用成萋斐。日月昭回,俯明枉直。獄書每御,輒鑒蔣濟之冤;炙髮見明,非關陳正之辯。遂漏斯密網,免彼嚴棘,得使還同士伍,比屋唐民,生死肉骨,豈侔其施。臣誠無識,孰不戴天。疏遠畝隴,絕望高闕,而降其接引,優以旨喻,於臣微物,足為榮隕。況剛條落葉,忽沾雲露;周行所置,復齒盛流。但雕朽杇糞,徒成延獎;捕影繫風,終無效答。」又啟謝東宮曰:「臣聞之,先聖以『眾惡之,必察焉;眾好之,必察焉』。豈非孤特則積毀所歸,比周則積譽斯信?知好惡之間,必待明鑒。故晏嬰再為阿宰,而前毀後譽。後譽出於阿
 意,前毀由於直道。是以一犬所噬,旨酒貿其甘酸;一手所搖,嘉樹變其生死。又鄒陽有言,士無賢愚,入朝見嫉。至若臧文之下展季,靳尚之放靈均,絳侯之排賈生,平津之陷主父,自茲厥後,其徒實繁。曲筆短辭,不暇殫述,寸管所窺,常由切齒。殿下誨道觀書,俯同好學,前載枉直,備該神覽。臣昔因立侍,親承緒言,飄風貝錦,譬彼讒慝,聖旨殷勤,深以為歎。臣資愚履直,不能杜漸防微,曾未幾何,逢訧罹難。雖吹毛洗垢,在朝而同嗟;而嚴文峻法,肆姦其必奏。不顧賣友,志欲要君,自非上帝運超己之光,昭陵陽之虐,舞文虛謗,不取信於宸明,在縲嬰纆,
 幸得蠲於庸暗。裁下免黜之書,仍頒朝會之旨。小人未識通方,縶馬懸車,息絕朝覲。方願滅影銷聲,遂移林谷。不悟天聽罔已,造次必彰,不以距違見疵,復使引籍雲陛。降寬和之色,垂布帛之言,形之千載,所蒙已厚;況乃恩等特召,榮同起家,望古自惟,彌覺多忝。但未渝丹石,永藏輪軌,相彼工言,構茲媒諓。且款冬而生,已凋柯葉,空延德澤,無謝陽春。」



 後為太子僕,母憂去職。服闋,除安西湘東王諮議參軍,遷黃門侍郎,尚書吏部郎,坐受人絹一束,為餉者所訟,左遷信威臨賀王長史。頃之,遷秘書監。大同五年,卒官,時年五十九。



 孝綽少有盛名,而仗
 氣負才,多所陵忽,有不合意,極言詆訾。領軍臧盾、太府卿沈僧杲等,並被時遇,孝綽尤輕之。每於朝集會同處,公卿間無所與語,反呼騶卒訪道途間事,由此多忤於物。



 孝綽辭藻為後進所宗,世重其文,每作一篇,朝成暮遍,好事者咸諷誦傳寫,流聞絕域。文集數十萬言,行於世。



 孝綽兄弟及群從諸子姪,當時有七十人,並能屬文,近古未之有也。其三妹適瑯邪王叔英、吳郡張嵊、東海徐悱,並有才學;悱妻文尤清拔。悱,僕射徐勉子,為晉安郡,卒,喪還京師,妻為祭文,辭甚心妻愴。勉本欲為哀文,既睹此文,於是閣筆。



 孝綽子諒,字求信。少好學,有文才,尤
 博悉晉代故事,時人號曰「皮裏晉書」。歷官著作佐郎,太子舍人,王府主簿,功曹史,中城王記室參軍。



 王筠,字元禮,一字德柔,瑯邪臨沂人。祖僧虔,齊司空簡穆公。父楫,太中大夫。筠幼警寤,七歲能屬文。年十六,為《芍藥賦》,甚美。及長,清靜好學,與從兄泰齊名。陳郡謝覽,覽弟舉,亦有重譽,時人為之語曰:「謝有覽舉,王有養炬。」炬是泰,養即筠,並小字也。



 起家中軍臨川王行參軍,遷太子舍人,除尚書殿中郎。王氏過江以來,未有居郎署者,或勸逡巡不就,筠曰:「陸平原東南之秀,王文度獨步江東,吾得比蹤昔人,何所多恨。」乃欣然就職。尚書令沈
 約,當世辭宗,每見筠文,咨嗟吟詠,以為不逮也。嘗謂筠:「昔蔡伯喈見王仲宣稱曰:『王公之孫也,吾家書籍,悉當相與。』僕雖不敏,請附斯言。自謝朓諸賢零落已後,平生意好,殆將都絕,不謂疲暮,復逢於君。」約於郊居宅造閣齋,筠為草木十詠,書之於壁,皆直寫文詞,不加篇題。約謂人云:「此詩指物呈形,無假題署。」約製《郊居賦》,構思積時,猶未都畢,乃要筠示其草,筠讀至「雌霓連蜷」,約撫掌欣抃曰:「僕嘗恐人呼為霓。」次至「墜石磓星」,及「冰懸坎而帶坻」。筠皆擊節稱贊。約曰:「知音者希,真賞殆絕,所以相要,政在此數句耳。」筠又嘗為詩呈約,即報書
 云:「覽所示詩,實為麗則,聲和被紙,光影盈字。夔、牙接響,顧有餘慚;孔翠群翔,豈不多愧。古情拙目,每佇新奇,爛然總至,權輿已盡。會昌昭發,蘭揮玉振,克諧之義,寧比笙簧。思力所該,一至乎此,歎服吟研,周流忘念。昔時幼壯,頗愛斯文,含咀之間,倏焉疲暮。不及後進,誠非一人,擅美推能,實歸吾子。遲比閑日,清覯乃申。」筠為文能壓強韻,每公宴並作,辭必妍美。約常從容啟高祖曰:「晚來名家,唯見王筠獨步。」



 累遷太子洗馬,中舍人,並掌東宮管記。昭明太子愛文學士,常與筠及劉孝綽、陸倕、到洽、殷芸等遊宴玄圃,太子獨執筠袖撫孝綽肩而言曰:「所
 謂左把浮丘袖,右拍洪崖肩。」其見重如此。筠又與殷芸以方雅見禮焉。出為丹陽尹丞、北中郎諮議參軍,遷中書郎。奉敕製《開善寺寶誌大師碑文》,詞甚麗逸。又敕撰《中書表奏》三十卷,及所上賦頌,都為一集。俄兼寧遠湘東王長史,行府、國、郡事。除太子家令,復掌管記。



 普通元年,以母憂去職。筠有孝性,毀瘠過禮,服闋後,疾廢久之。六年,除尚書吏部郎,遷太子中庶子,領羽林監,又改領步兵。中大通二年,遷司徒左長史。三年,昭明太子薨,敕為哀策文,復見嗟賞。尋出為貞威將軍、臨海太守,在郡被訟,不調累年。大同初,起為雲麾豫章王長史,遷秘書
 監。五年,除太府卿。明年,遷度支尚書。中大同元年,出為明威將軍、永嘉太守,以疾固辭,徙為光祿大夫,俄遷雲騎將軍、司徒左長史。太清二年,侯景寇逼,筠時不入城。明年,太宗即位,為太子詹事。筠舊宅先為賊所焚,乃寓居國子祭酒蕭子雲宅,夜忽有盜攻之,驚懼墜井卒,時年六十九。家人十餘人同遇害。



 筠狀貌寢小,長不滿六尺。性弘厚,不以藝能高人,而少擅才名,與劉孝綽見重當世。其自序曰:「余少好書,老而彌篤。雖偶見瞥觀,皆即疏記,後重省覽,懽興彌深,習與性成,不覺筆倦。自年十三四,齊建武二年乙亥至梁大同六年,四十載矣。幼年
 讀《五經》,皆七八十遍。愛《左氏春秋》,吟諷常為口實,廣略去取,凡三過五抄。餘經及《周官》、《儀禮》、《國語》、《爾雅》、《山海經》、《本草》並再抄。子史諸集皆一遍。未嘗倩人假手,並躬自抄錄,大小百餘卷。不足傳之好事,蓋以備遺忘而已。」又與諸兒書論家世集云:「史傳稱安平崔氏及汝南應氏,並累世有文才,所以范蔚宗云崔氏『世擅雕龍』。然不過父子兩三世耳;非有七葉之中,名德重光,爵位相繼,人人有集,如吾門世者也。沈少傅約語人云:『吾少好百家之言,身為四代之史,自開闢已來,未有爵位蟬聯,文才相繼,如王氏之盛者也。』汝等仰觀堂構,思各努力。」筠自撰其文
 章,以一官為一集,自洗馬、中書、中庶子、吏部佐、臨海、太府各十卷,《尚書》三十卷,凡一百卷,行於世。



 史臣陳吏部尚書姚察曰:王僧孺之巨學,劉孝綽之詞藻,主非不好也,才非不用也,其拾青紫,取極貴,何難哉!而孝綽不拘言行,自躓身名,徒鬱抑當年,非不遇也。



\end{pinyinscope}