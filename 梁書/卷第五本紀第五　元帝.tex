\article{卷第五本紀第五 元帝}

\begin{pinyinscope}

 世祖孝元皇帝,諱繹,字世誠,小字七符,高祖第七子也。天監七年八月丁巳生。十三年,封湘東郡王,邑二千戶。初為寧遠將軍、會稽太守,入為侍中、宣威將軍、丹陽尹。普通七年,出為使持節、都督荊、湘、郢、益、寧、南梁六州諸軍事、西中郎將、荊州刺史。中大通四年,進號平西將軍。大同元年,進號安西將軍。三年,進號鎮西將軍。五年,入
 為安右將軍、護軍將軍,領石頭戍軍事。六年,出為使持節、都督江州諸軍事、鎮南將軍、江州刺史。太清元年,徙為使持節、都督荊、雍、湘、司、郢、寧、梁、南、北秦九州諸軍事、鎮西將軍、荊州刺史。三年三月,侯景寇沒京師。四月,太子舍人蕭歆至江陵宣密詔,以世祖為侍中、假黃鉞、大都督中外諸軍事、司徒承制,餘如故。是月,世祖征兵於湘州,湘州刺史河東王譽拒不遣。六月丙午,遣世子方等帥眾討譽,戰所敗死。七月,又遣鎮兵將軍鮑泉代討譽。九月乙卯,雍州刺史岳陽王察舉兵反,來寇江陵,世祖嬰城拒守。乙丑,察將杜掞與其兄弟及楊混,各率其
 眾來降。丙寅,察遁走。鮑泉攻湘州不克,又遣左衛將軍王僧辯代將。



 大寶元年,世祖猶稱太清四年。正月辛亥朔,左衛將軍王僧辯獲橘三十子共蒂,以獻。二月甲戌,衡陽內史周弘直表言鳳皇見郡界。夏五月辛未,王僧辯克湘州,斬河東王譽,湘州平。六月,江夏王大款、山陽王大成、宜都王大封自信安間道來奔。九月辛酉,以前郢州刺史南平王恪為中衛將軍、尚書令、開府儀同三司,中撫軍將軍世子方諸為郢州刺史,左衛將軍王僧辯為領軍將軍。改封大款為臨川郡王,大成為桂陽郡王,大封為汝
 南郡王。是月,任約進寇西陽、武昌,遣左衛將軍徐文盛、右衛將軍陰子春、太子右衛率蕭慧正、巂州刺史席文獻等下武昌拒約。以中衛將軍、尚書令、開府儀同三司南平王恪為荊州刺史,鎮武陵。十一月甲子,南平王恪、侍中臨川王大款、桂陽王大成、散騎常侍江安侯圓正、侍中左衛將軍張綰、司徒左長史曇等府州國一千人奉箋曰:竊以嵩岳既峻,山川出雲;大國有蕃,申甫惟翰。豈非皇建斯極,以位為寶;聖教辨方,慎名與器。是知太尉佐帝,重華表黃玉之符,司空相土,伯禹降玄珪之錫。伏惟明公大王殿下,命世應期,挺生將聖。忠為令德,孝
 實天經,地切應、韓,寄深旦、奭,五品斯訓,七政以齊,志存社稷,功濟屯險。夷狄內侵,枕戈泣血,鯨鯢未掃,投袂勤王,能使遊魂請盟以屈膝,醜徒銜璧而懾氣。親蕃外叛,釁均吳、楚,義討申威,兵不血刃。湘波自息,非築杜弢之壘;峴山離貳,不伐劉表之城。九江致梗,二別殊派,纔命戈船,底定灊、霍。溯流窮討,路絕窺窬,胡兵侵界,鐵馬霧合,神規獨運,皆即梟懸,翻同翅折,遂修職貢。梁、漢合契,肆犀利之兵,巴、漢俱下,竭驍勇之陣。南通五嶺,北出力原;東夷不怨,西戎即序。可謂上流千里,持戟百萬,天下之至貴,四海之所推也。今海水飛雲,崑山起燎,魏文悲
 樂推之歲,韓宣歎成禮之日,陽臺之下,獨有冠蓋相趨;夢水之傍,尚致車輿結轍。麰麥兩穗,出於南平之邦;甘露泥枝,降乎當陽之境。野蠶自績,何謝歐絲;閑田生稻,寧殊雨粟。莫非品物咸亨,是稱文明光大,豈可徽號不彰於彞典,明試不陳乎車服者哉!昔晉、鄭入周,尚作卿士;蕭、曹佐漢,且居相國。宜崇茲盛禮,顯答群望。恪等稽尋甲令,博詢惇史,謹再拜上,進位相國,總百揆,竹使符一,別準恆儀。杖金斧以剪逆暴,乘玉輅而定社稷。傍羅麗於日月,貞明合于天地。扶危翼治,豈不休哉!恪等不通大體,自昧伏奏以聞。



 世祖令答曰:「數鐘陽九,時惟百
 六,鯨鯢未剪,寤寐痛心。周粵天官,秦稱相國,東至於海,西至于河,南次朱鳶,北漸玄塞。率茲小宰,弘斯大德。將何用繼蹤曲阜,擬跡桓、文,終建一匡,肅其五拜。雖義屬隨時,事無虛紀,傳稱皆讓,《象》著鳴謙,瞻言前典,再懷哽恧。」十二月壬辰,以定州刺史蕭勃為鎮南將軍、廣州刺史。遣護軍將軍尹悅、巴州刺史王珣、定州刺史杜多安帥眾下武昌,助徐文盛。



 大寶二年,世祖猶稱太清五年。二月己亥,魏遣使來聘。三月,侯景悉兵西上,會任約軍。閏四月丙午,景遣其將宋子仙、任約襲郢州,執刺史蕭方諸。戊申,徐文盛、陰子
 春等奔歸,王珣、尹悅、杜多安並降賊。庚戌,領軍將軍王僧辯帥眾屯巴陵。甲子,景進寇巴陵。五月癸未,世祖遣遊擊將軍胡僧祐、信州刺史陸法和帥眾下援巴陵。任約敗,景遂遁走。以王僧辯為征東將軍、開府儀同三司、尚書令,胡僧祐為領軍將軍,陸法和為護軍將軍。仍令僧辯率眾軍追景,所至皆捷。八月甲辰,僧辯下次湓城。辛亥,以鎮南將軍、湘州刺史蕭方矩為中衛將軍。司空、征南將軍、南平王恪進號征南大將軍。湘州刺史,餘如故。九月己亥,以征東將軍、開府儀同三司、尚書令王僧辯為江州刺史,餘如故。盤盤國獻馴象。冬十月辛丑朔,
 有紫雲如車蓋,臨江陵城。是月,太宗崩。侍中、征東將軍、開府儀同三司、江州刺史、尚書令、長寧縣侯王僧辯等奉表曰:眾軍薄伐,塗次九水,即日獲臨城縣使人報稱:侯景弒逆皇帝,賊害太子,宗室在寇庭者,並罹禍酷。六軍慟哭,三辰改曜。哀我皇極,四海崩心。我大梁纂堯構緒,基商啟祚。太祖文皇帝徇齊作聖,肇有六州。高祖武皇帝聰明神武,奄龕天下。依日月而和四時,履至尊而制六合。麗正居貞,大橫固祉。四葉相係,三聖同基。蠢爾凶渠,遂憑天邑。閶闔受白登之辱,象魏致堯城之疑。雲扆承華,一朝俱酷。金楨玉幹,莫不同冤。悠悠彼蒼,何其罔
 極!



 臣聞喪君有君,《春秋》之茂典;以德以長,先王之通訓。少康則牧眾撫職,祀夏所以配天;平王則居正東遷,宗周所以卜世。漢光以能捕不道,故景歷重昌;中宗以不違群議,故江東可立。儔今考古,更無二謀。伏惟陛下至孝通幽,英武靈斷,當七九之厄,而應千載之期;啟殷憂之明,而居百王之會。取威定霸,嶮阻艱難,建社治兵,載循古道。家國之事,一至於斯。天祚大梁,必將有主。軒轅得姓,存者二人;高祖五王,代實居長。乘屈完而陳諸侯,拜子武而服大輅。功齊九有,道濟生民。非奉聖明,誰嗣下武!



 臣聞日月貞明,太陽不可以闕照;天地貞觀,乾道
 不可以久惕。黃屋左纛,本為億兆而尊;鸞輅龍章,蓋以郊禋而貴。寶器存乎至重,介石慎於易差。黔首豈可少選無君,宗祏豈可一日無主。伏願陛下掃地升中,柴天改物。事迫凶危,運鐘擾攘,蓋不勞宗正奉詔,博士擇時,南面即可居尊,西向無所讓德。四方既知有奉,八百始可同期。殘寇潛居,器藏社處,乾象既傾,坤儀已覆。斬莽輗車,燒卓照市,廓清函夏,正為塋陵,開雪宮圍,庶存鐘鼎,彼黍離離,伊何可言。陛下繼明闡祚,即宮舊楚。左廟右社之制,可以權宜;五禮六樂之容,歲時取備。金芝九莖,瓊茅三脊。要衛率職,尉候相望。坐廟堂以朝四夷,登
 靈臺而望雲物,禪梁甫而封泰山,臨東濱而禮日觀。然後與三事大夫,更謀都鄙。左瀍右澗,夾雒可以為居,抗殿疏龍,惟王可以在鎬,何必勤勤建業也哉。臣等不勝控款之至,謹拜表以聞。



 世祖奉諱,大臨三日,百官縞素。乃答曰:「孤以不德,天降之災,枕戈飲膽,扣心泣血。風樹之酷,萬始不追;霜露之哀,百憂總萃。甫聞伯升之禍,彌切仲謀之悲。若封豕既殲,長蛇即戮,方欲追延陵之逸軌,繼子臧之高讓,豈資秋亭之壇,安事繁陽之石。侯景,項籍也;蕭棟,殷辛也。赤泉未賞,劉邦尚曰漢王;白旗弗懸,周發猶稱太子。飛龍之位,孰謂可躋;附鳳之徒,既聞
 來議。群公卿士,其諭孤之志,無忽!」司空南平王恪率宗室五十餘人,領軍將軍胡僧祐率群僚二百餘人,江州別駕張佚率吏民三百餘人,並奉箋勸進。世祖固讓。



 十一月乙亥,王僧辯又奉表曰:紫宸曠位,赤縣無主,百靈聳動,萬國回皇。雖醉醒相扶,同歸景亳,式歌且誦,總赴唐郊,猶懼陛下俯首潸然,讓德不嗣。傳車在道,方慎宋昌之謀;法駕已陳,尚杜耿純之勸。岳牧翹首,天民累息。臣聞星回日薄,擊雷鞭電者之謂天;岳立川流,吐霧蒸雲者之謂地。苞天地之混成,洞陰陽之不測,而以裁成萬物者,其在聖人乎!故云「天地之大德曰生,聖人之大
 寶曰位。」黃屋廟堂之下,本非獲已而居;明鏡四衢之樽,蓋由應物取訓。伏惟陛下稽古文思,英雄特達。比以周旦,則文王之子;方之放勛,則帝摯之季。千年旦暮,可不在斯。庭闕湮亡,鐘鼎淪覆,嗣膺景歷,非陛下而誰?豈可使赤眉更立盆子,隗囂託置高廟。陛下方復從容高讓,用執謙光。展其矯行偽書,誣罔正朔,見機而作,斷可識矣。匪疑何卜,無待蓍龜。日者,公卿失馭,禍纏霄極,侯景憑陵,姦臣互起,率戎伐穎,無處不然,勸明誅晉,側足皆爾。刁斗夜鳴,烽火相照。中朝人士,相顧銜悲;涼州義徒,東望殞涕,惵惵黔首,將欲安歸!陛下英略緯天,沉明內
 斷,橫劍泣血,枕戈嘗膽,農山圮下之策,金匱玉鼎之謀,莫不定算扆帷,決勝千里。擊靈鼉之鼓,而建翠華之旗,驅六州之兵,而總九伯之伐,四方雖虞,一戰以霸。斬其鯨鯢,既章大戮,何校滅耳,莫匪姦回,史不絕書,府無虛月。自洞庭安波,彭蠡底定,文昭武穆,芳若椒蘭,敵國降城,和如親戚,九服同謀,百道俱進,國恥家怨,計期就雪,社稷不墜,翽在聖明。今也何時,而申帝啟之避,凶危若此,方陳泰伯之辭。國有具臣,誰敢奉詔。天下者高祖之天下,陛下者萬國之歡心,萬國豈可無君,高祖豈可廢祀。即日五星夜聚,八風通吹,雲煙紛郁,日月光華,百官
 象物而動,軍政不戒而備。飛艫巨艦,竟水浮川;鐵馬銀鞍,陵山跨谷。英傑接踵,忠勇相顧,湛宗族以酬恩,焚妻子以報主。莫不覆盾銜威,提斧擊眾,風飛電耀、志滅凶醜。所待陛下昭告后土,虔奉上帝,廣發明詔,師出以名,五行夕返,六軍曉進,便當盡司寇之威,窮蚩尤之伐,執石趙而求璽,斬姚秦而取鐘,脩掃塋陵,奉迎宗廟。陛下豈得不仰存國計,俯從民請。漢宣嗣位之後,即遣蒲類之軍;光武登極既竟,始有長安之捷。由此言之,不無前準。臣等或世受朝恩,或身荷重遇,同休等戚,自國刑家,茍有腹心,敢以死奪。不任慺慺之至,謹重奉表以聞。



 世
 祖答曰:「省示,復具一二。孤聞天生蒸民而樹之以君,所以對揚天休,司牧黔首。攝提、合雒以前,慄陸、驪連之外,書契不傳,無得稱也。自阪泉彰其武功,丹陵表其文德,有人民焉,有社稷焉,或歌謠所歸,或惟天所相。孤遭家多難,大恥未雪,國賊則蚩尤弗剪,同姓則有扈不賓,臥而思之,坐以待旦,何以應寶歷,何以嗣龍圖。庶一戎既定,罪人斯得,祀夏配天,方申來議也。」是時巨寇尚存,未欲即位,而四方表勸,前後相屬,乃下令曰:「《大壯》乘乾,《明夷》垂翼,璇度亟移,玉律屢徙,四岳頻遣勸進,九棘比者表聞。譙、沛未復,塋陵永遠,於居于處,寤寐疚懷,何心何
 顏,撫茲歸運。自今表奏,所由並斷,若有啟疏,可寫此令施行。」是日,賊司空、東南道大行臺劉神茂率儀同劉歸義、留異赴義,奉表請降。



 大寶三年,世祖猶稱太清六年。正月甲戌,世祖下令曰:「軍國多虞,戎旃未靜,青領雖熾,黔首宜安。時惟星鳥,表年祥於東秩;春紀宿龍,歌歲取於南畯。況三農務業,尚看夭桃敷水;四人有令,猶及落杏飛花。化俗移風,常在所急;勸耕且戰,彌須自許。豈直燕垂寒谷,積黍自溫,寧可墮此玄苗,坐飡紅粒,不植燕頷,空候蟬鳴。可悉深耕穊種,安堵復業,無棄民力,並分地利。班勒州郡,咸使遵
 承。」以智武將軍、南平內史王褒為吏部尚書。二月,王僧辯眾軍發自尋陽。世祖馳檄告四方曰:夫剝極生災,乃及龍戰,師貞終吉,方制獖豕。豈不以侵陽蕩薄,源之者亂階;定龕艱難,成之者忠義。故羿、澆滅於前,莽、卓誅於後。是故使桓、文之勳,復興於周代;溫、陶之績,彌盛於金行。粵若梁興五十餘載,平壹宇內,德惠悠長,仁育蒼生,義征不服。左伊右瀍,咸皆仰化;濁涇清渭,靡不向風。建翠鳳之旗,則六龍驤首;擊靈鼉之鼓,則百神警肅。風、牧、方、邵之賢,衛、霍、辛、趙之將,羽林黃頭之士,虎賁緹騎之夫,叱吒則風雲興起,鼓動則嵩、華倒拔。自桐柏以北,孤
 竹以南,碣石之前,流沙之後,延頸舉踵,交臂屈膝。胡人不敢牧馬,秦士不敢彎弓。葉和萬邦,平章百姓,十堯九舜,曷足云也。賊臣侯景,匈奴叛臣,鳴鏑餘噍。懸瓠空城,本非國寶,壽春畿要,賞不踰月。開海陵之倉,賑常平之米,檄九府之費,錫三官之錢,冒於貨賄,不知紀極。敢興逆亂,梗我王畿。賊臣正德,阻兵安忍。日者結怨江羋,遠適單于。簡牘屢彰,彭生之魂未弭;聚斂無度,景卿之誚已及。為虎傅翼,遠相招致。虔劉我生民,離散我兄弟。我是以董率皋貔,躬擐甲胄,霜戈照日,則晨離奪暉,龍騎蔽野,則平原掩色,信與江水同流,氣與寒風俱憤。凶醜
 畏威,委命下吏,乞活淮、肥,茍存徐、兗。渙汗既行,絲綸爰被。我是以班師凱歸,休牛息馬。賊猶不悛。遂復矢流王屋,兵躔象魏。總章之觀,非復聽訟之堂;甘泉之宮,永乖避暑之地。坐召憲司,臥制朝宰,矯託天命,偽作符書。重增賦斂,肆意裒剝,生者逃竄,死者暴尸,道路以目,庶僚鉗口。刑戮失衷,爵賞由心,老弱波流,士女塗炭。臧獲之人,五宗及賞;搢紳之士,三族見誅。穀粟騰踴,自相吞噬。惵惵黔首,路有銜索之哀;蠢蠢黎民,家隕桓山之泣。偃師南望,無復儲胥、露寒,河陽北臨,或有穹廬氈帳。南山之竹,未足言其愆;西山之兔,不足書其罪。外監陳瑩之
 至,伏承先帝登遐,宮車晏駕。奉諱驚號,五內摧裂,州冤本毒,無地容身。景阻饑既甚,民且狼顧,遂侵軼我彭蠡,憑凌我郢邑,窮據我江夏,掩襲我巴丘。我是以義勇爭先,忠貞盡力。斬馘凶渠,不可稱算,沙同赤岸,水若絳河。任約泥首於安南,化仁面縛於漢口,子仙乞活於鄢郢,希榮敗績於柴桑。侯景奔竄,十鼠爭穴,郭默清夷,晉熙附義,計窮力屈,反殺後主。畢、原、禜、郇、並離禍患,凡、蔣、邢、茅,皆伏鈇鑕。是可忍也,孰不可容!幕府據有上流,實惟分陜,投袂荷戈、志在畢命。昔周依晉、鄭,漢有虛、牟。彼惟末屬,猶能如此;況聯華日月,天下不賤,為臣為子,兼國
 兼家者哉!咸以義旗既建,宜須總一,共推幕府,實用主盟。粵以不佞,謬董連率,遠惟國艱,不遑寧處。中權後勁,龔行天罰,提戈蒙險,隕越以之。天馬千群,長戟百萬,驅賁獲之士,資智勇之力,大楚踰荊山,淺原度彭蠡,舳艫泛水,以掎其南,輜軿委輸,以衝其北。華夷百濮,贏糧影從。雷震風駭,直指建業。按劍而叱,江水為之倒流;抽戈而揮,皎日為之退舍。方駕長驅,百道俱入,夷山殄谷,充原蔽野。挾輈曳牛之侶,拔距磔石之夫,騎則逐日追風,弓則吟猿落鴈。捧昆崙而壓卵,傾渤海而灌熒。如駟馬之載鴻毛,若奔牛之觸魯縞。以此眾戰,誰能禦之!脫復
 蜂蠆有毒,獸窮則鬥。謂山蓋高,則四郊多壘;謂地蓋遠,則三千弗違。如彼怒蛙,譬如鼷鼠,豈費萬鈞,無勞百溢。加以日臨黃道,兵起絳宮,三門既啟,五將咸發,舉整整之旗,掃亭亭之氣,故以臨機密運,非賊所解,奉義而誅,何罪不服?今遣使持節、大都督、征東將軍、開府儀同三司、江州刺史、尚書令、長寧縣開國侯王僧辯率眾十萬,直掃金陵。鳴鼓聒天,摐金振地。朱旗夕建,如赤城之霞起;戈船夜動,若滄海之奔流。計其同惡,不盈一旅。君子在野,小人比周。何校滅耳,匪朝伊夕。舂長狄之喉,繫郅支之頸。今司寇明罰,質金夫所誅,止侯景而已。黎元何辜,
 一無所問。諸君或世樹忠貞,身荷寵爵,羽儀鼎族,書勛王府,俯眉猾豎,無由自效,豈不下慚泉壤,上愧皇天!失忠與義,難以自立。想誠南風,迺眷西顧,因變立功,轉禍為福。有能縛侯景及送首者,封萬戶開國公,絹布五萬匹。有能率動義眾,以應官軍,保全城邑,不為賊用,上賞方伯,下賞剖符,並裂山河,以紆青紫。昔由余入秦,禮同卿佐;日磾降漢,且珥金貂。必有其才,何恤無位。若執迷不反,拒逆王師,大軍一臨,刑茲罔赦。孟諸焚燎,芝艾俱盡;宣房河決,玉石同沉。信賞之科,有如皎日;黜陟之制,事均白水。檄布遠近,咸使知聞。



 三月,王僧辯等平侯景,
 傳其首於江陵。戊子,以賊平告明堂、太社。己丑,王僧辯等又奉表曰:眾軍以今月戊子總集建康。賊景鳥伏獸窮,頻擊頻挫,姦竭詐盡,深溝自固。臣等分勒武旅,百道同趣,突騎短兵,犀函鐵楯,結隊千群,持戟百萬,止紂七步,圍項三重,轟然大潰,群凶四滅。京師少長,俱稱萬歲。長安酒食,於此價高。九縣雲開,六合清朗,矧伊黔首,誰不載躍!伏惟陛下咀痛茹哀,嬰憤忍酷。自紫庭絳闕,胡塵四起,需垣好畤,冀馬雲屯,泣血治兵,嘗膽誓眾。而吳、楚一家,方與七國俱反;管、蔡流言,又以三監作亂。西涼義眾,阻強秦而不通;并州遺民,跨飛狐而見泯。豺狼當
 路,非止一人;鯨鯢不梟,倏焉五載。英武克振,怨恥並雪,永尋霜露,如何可言!臣等輒依故實,奉脩社廟,使者持節,分告塋陵。嗣后升遐,龍輴未殯,承華掩曜,梓宮莫測,並即隨由備辦,禮具凶荒。四海同哀,六軍袒哭,聖情孝友,理當感慟。日者,百司岳牧,祈仰宸鑒。以錫珪之功,既歸有道,當璧之禮,允屬聖明;而優詔謙沖,窅然凝邈。飛龍可躋,而《乾》爻在四;帝閽云叫,而閶闔未開。謳歌再馳,是用翹首。所以越人固執,熏丹穴以求君;周民樂推,踰岐山而事主。漢王不即位,無以貴功臣;光武不止戈,豈謂紹宗廟。黃帝遊於襄城,尚訪治民之道;放勛入於姑
 射,猶使樽俎有歸。伊此儻來,豈聖人所欲,帝王所應,不獲已而然。伏讀璽書,尋諷制旨,顧懷物外,未奉慈衷。陛下日角龍顏之姿,表於徇齊之日,彤雲素氣之瑞,基於應物之初。博覽則大哉無所與名,深言則曄乎昭章之觀。忠為令德,孝實動天。加以英威茂略,雄圖武算,指麾則丹浦不戰,顧眄則阪泉自蕩。地維絕而重紐,天柱傾而更植。鑿河津於孟門,百川復啟;補穹儀以五石,萬物再生。縱陛下拂袗衣而遊廣成,登泬山而去東土,群臣安得仰訴,兆庶何所歸仁。況郊祀配天,罍篚禮曠,齋宮清廟,匏竹不陳,仰望鑾輿,匪朝伊夕,瞻言法駕,載渴且
 饑。豈可久稽眾議,有曠則!舊郊既復,函、雒已平。高奴、櫟陽,宮館雖毀;濁河清渭,佳氣猶存。皋門有伉,甘泉四敞,土圭測景,仙人承露。斯蓋九州之赤縣,六合之樞機。博士捧圖書而稍還,太常定禮儀而已列。豈得不揚清駕而赴名都,具玉鑾而遊正寢!昔東周既遷,鎬京遂其不復;長安一亂,郟、洛永以為居。夏后以萬國朝諸侯,文王以六州匡天下。跡基百里,劍杖三尺。以殘楚之地,抗拒九戎;一旅之師,剪滅三叛。坦然大定,御輦東歸。解五牛於冀州,秣六馬於譙郡。緬求前古,其可得歟?對揚天命,何所讓德!有理存焉,敢重所奏。



 相國答曰:「省表,復具
 一二。群公卿士,億兆夷人,咸以皇天眷命,歸運所屬,用集寶位于予一人。文叔金吾之官,事均往願;孟德征西之位,且符前說。今淮海長鯨,雖云授首;襄陽短狐,未全革面。太平玉燭,爾乃議之。」辛卯,宣猛將軍朱買臣密害豫章嗣王棟,及其二弟橋、樛,世祖志也。



 四月乙巳,益州刺史、新除假黃鉞、太尉武陵王紀竊位於蜀,改號天正元年。世祖遣兼司空蕭泰、祠部尚書樂子雲拜謁塋陵,修復社廟。丁巳,世祖令曰:「軍容不入國,國容不入軍。雖子產獻捷,戎服從事,亞夫弗拜,義止將兵。今凶醜殲夷,逆徒殄潰,九有既截,四海乂安。漢官威儀,方陳盛禮,衛
 多君子,寄是式瞻。便可解嚴,以時宣勒。」是月,以東陽太守張彪為安東將軍。五月庚午,司空南平王恪及宗室王侯、大都督王僧辯等,復拜表上尊號,世祖猶固讓不受。庚辰,以征南將軍、湘州刺史、司空南平嗣王恪為鎮東將軍、揚州刺史,餘如故。甲申,以尚書令、征東將軍、開府儀同三司、江州刺史王僧辯為司徒、鎮衛將軍。乙酉,斬賊左僕射王偉、尚書呂季略、少卿周石珍、舍人嚴亶於江陵市。是日,世祖令曰:「君子赦過,著在周經;聖人解網,聞之湯令。自獫狁孔熾,長蛇薦食,赤縣阽危,黔黎塗炭,終宵不寐,志在雪恥。元惡稽誅,本屬侯景;王偉是其
 心膂,周石珍負背恩義,今並烹諸鼎鑊,肆之市朝。但比屯邅寇擾,為歲已積,衣冠舊貴,被逼偷生,猛士勛豪,和光茍免,凡諸惡侶,諒非一族。今特闡以王澤,削以刑書,自太清六年五月二十日昧爽以前,咸使惟新。」是月,魏遣太師潘樂、辛術等寇秦郡,王僧辯遣杜掞帥眾拒之。以陳霸先為征北大將軍、開府儀同三司、南徐州刺史。是月,魏遣使賀平侯景。



 八月,蕭紀率巴、蜀大眾連舟東下,遣護軍陸法和屯巴峽以拒之。兼通直散騎常侍、聘魏使徐陵於鄴奉表曰:臣聞封唐有聖,還承帝嚳之家;居代惟賢,終纂高皇之祚。無為稱於革舄,至治表於垂
 衣,而撥亂反正,非聞前古。至如金行重作,源出東莞;炎運猶昌,枝分南頓。豈得掩顯姓於軒轅,非才子於顓頊?莫不時因多難,俱繼神宗者也。伏惟陛下,出《震》等於勛、華,明讓同於旦、奭。握圖執鉞,將在御天,玉縢珠衡,先彰元后。神祇所命,非惟太室之祥;圖畫斯歸,何止堯門之瑞。若夫大孝聖人之心,中庸君子之德,固以作訓生民,貽風多士。一日二日,研覽萬機;允文允武,包羅群藝。擬茲三大,賓是四門,歷試諸難,咸熙庶績,斯無得而稱也。自無妄興暴,皇祚浸微,封犬希脩蛇,行災中國,靈心所宅,下武其興,望紫極而長號,瞻丹陵而殞慟。家冤將報,天
 賜黃鳥之旗;國害宜誅,神奉玄狐之籙。滕公擁樹,雄氣方嚴;張繡交兵,風神彌勇。忠誠冠於日月,孝義感於冰霜。如霆如雷,如貔如虎,前驅效命,元惡斯殲。既挂膽於西州,方燃臍於東市。蚩尤三冢,寧謂嚴誅?王莽千剸,非云明罰?青羌赤狄,同畀豺狼,胡服夷言,咸為京觀。邦畿濟濟,還見隆平;宗廟愔愔,方承多福。自氤氳渾沌之世,驪連、慄陸之君,卦起龍圖,文因鳥跡。雲師火帝,非無戰陣之風,堯誓湯征,咸用干戈之道。星躔東井,時破崤、潼;雷震南陽,初平尋、邑。未有援三靈之已墜,救四海之群飛,赫赫明明,龔行天罰,如當今之盛者也。於是卿雲似
 蓋,晨映姚鄉;甘露如珠,朝華景寢。芝房感德,咸出銅池;蓂莢伺辰,無勞銀箭。重以東漸玄菟,西踰白狼,高柳生風,扶桑盛日,莫不編名屬國,歸質鴻臚,荒服來賓,遐邇同福。其文昭武穆,跗萼也如彼;天平地成,功業也如此。久應旁求掌固,諮詢天官,斟酌繁昌,經營高邑。宗王啟霸,非勞陽武之侯;清蹕無虞,何事長安之邸。正應揚鑾旂以饗帝,仰鳳扆以承天,歷數在躬,疇與為讓!去月二十日,兼散騎常侍柳暉等至鄴,伏承聖旨謙沖,為而弗宰,或云涇陽未復,函谷無泥,旋駕金陵,方膺天眷。愚謂大庭、少昊,非有定居;漢祖、殷宗,皆無恆宅。登封岱岳,猶
 置明堂;巡狩章陵,時行司隸。何必西瞻虎據,乃建王宮;南望牛頭,方稱天闕。抑又聞之:玄圭既錫,蒼玉無陳,乃棫樸之愆期,非苞茅之不貢。雲和之瑟,久廢甘泉;孤竹之管,無聞方澤。豈不懼歟!伏願陛下因百姓之心,拯萬邦之命。豈可逡巡固讓,方求石戶之農;高謝君臨,徒引箕山之客!未知上德之不德,惟見聖人之不仁。率士翹翹,蒼生何望!昔蘇季、張儀,違鄉負俗,尚復招三方以事趙,請六國以尊秦。況臣等顯奉皇華,親承朝命,珪璋特達,通聘河陽,貂珥雍容,尋盟漳水,加牢貶館,隨勢汙隆,瞻望鄉關,誠均休戚。但輕生不造,命與時乖。忝一介之
 行人,同三危之遠擯。承閑內殿,事絕耿弇之恩;封奏邊城,私等劉琨之哭。不勝區區之至,謹拜表以聞。



 九月甲戌,司空、鎮東將軍、揚州刺史南平王恪薨。冬十月乙未,前梁州刺史蕭循自魏至於江陵,以循為平北將軍、開府儀同三司。戊申,執湘州刺史王琳於殿內,琳副將殷晏下獄死。辛酉,以子方略為湘州刺史。庚戌,琳長史陸納及其將潘烏累等舉兵反,襲陷湘州。是月,四方征鎮,王公卿士復勸世祖即尊號,猶謙讓未許。表三上,乃從之。



 承聖元年冬十一月丙子,世祖即皇帝位於江陵。詔曰:「
 夫樹之以君,司牧黔首。帝堯之心,豈貴黃屋,誠弗獲已而臨蒞之。朕皇祖太祖文皇帝積德岐、梁,化行江、漢,道映在田,具瞻斯屬。皇考高祖武皇帝明並日月,功格區宇,應天從民,惟睿作聖。太宗簡文皇帝地侔啟、誦,方符文、景。羯寇憑陵,時難孔棘。朕大拯橫流,克復宗社。群公卿士、百辟庶僚,咸以皇靈眷命,歸運斯及,天命不可以久淹,宸極不可以久曠。粵若前載,憲章令範,畏天之威,算隆寶歷,用集神器于予一人。昔虞、夏、商、周,年無嘉號,漢、魏、晉、宋,因循以久。朕雖云撥亂,且非創業,思得上繫宗祧,下惠億兆。可改太清六年為承聖元年。逋租宿責,
 並許弘貸;孝子義孫,可悉賜爵;長徒鏁士,特加原宥;禁錮奪勞,一皆曠蕩。」是日世祖不升正殿,公卿陪列而已。丁丑,以平北將軍、開府儀同三司蕭循為驃騎將軍、湘州刺史,餘如故。己卯,立王太子方矩為皇太子,改名元良。立皇子方智為晉安郡王,方略為始安郡王。追尊所生妣阮脩容為文宣太后。是月,陸納遣將潘烏累等攻破衡州刺史丁道貴於淥口,道貴走零陵。十二月壬子,陸納分兵襲巴陵,湘州刺史蕭循擊破之。是月,營州刺史李洪雅自零陵率眾出空雲灘,將下討納,納遣將吳藏等襲破洪雅,洪雅退守空雲城。



 二年春正月乙丑,詔王僧辯率眾軍士討陸納。戊寅,以吏部尚書王褒為尚書右僕射,劉+為吏部尚書。西魏遣大將尉遲迥襲益州。三月庚午,詔曰:「食乃民天,農為治本,垂之千載,貽諸百王,莫不敬授民時,躬耕帝籍。是以稼穡為寶,《周頌》嘉其樂章;禾麥不成,魯史書其方冊。秦人有農力之科,漢氏開屯田之利。頃歲屯否,多難薦臻,干戈不戢,我則未暇。廣田之令,無聞於郡國;載師之職,有陋於官方。今元惡殄殲,海內方一,其大庇黔首,庶拯橫流。一廛曠務,勞心日仄;一夫廢業,舄鹵無遺。國富刑清,家給民足。其力田之身,在所蠲免。外即宣勒,稱朕
 意焉。」辛未,李洪雅以空雲城降賊,賊執之而歸。初,丁道貴走零陵投洪雅,洪雅使收餘眾。與之俱降。洪雅既降賊,賊乃害道貴。丙子,賊將吳藏等帥兵據車輪。庚寅,有兩龍見湘州西江。夏四月丙申,僧辯軍次車輪。五月甲子,眾軍攻賊,大破之。乙丑,僧辯軍至長沙。甲戌,尉遲迥進逼巴西,潼州刺史楊虔運以城降,納迥。己丑,蕭紀軍至西陵。六月乙卯,湘州平。是月,尉遲迥圍益州。秋七月辛未,巴人苻升、徐子初斬賊城主公孫晁,舉城來降。紀眾大潰,遇兵死。乙未,王僧辯班師江陵,詔諸軍各還所鎮。八月戊戌,尉遲迥陷益州。庚子,詔曰:「夫爰始居毫,不
 廢先王之都;受命于周,無改舊邦之頌。頃戎旃既息,關柝無警。去魯興歎,有感宵分,過沛殞涕,實勞夕寐。仍以瀟、湘作亂,庸、蜀阻兵,命將授律,指期克定。今八表乂清,四郊無壘,宜從青蓋之典,言歸白水之鄉。江、湘委輸,方船連舳,巴峽舟艦,精甲百萬,先次建鄴,行實京師,然後六軍遄征,九旂揚旆,拜謁塋陵,脩復宗社。主者詳依舊典,以時宣勒。」九月庚午,司徒王僧辯旋鎮。丙子,以護軍將軍陸法和為郢州刺史。乙酉,以晉安王方智為江州刺史。是月,魏遣郭元建治舟師於合肥,又遣大將邢杲遠、步六汗薩、東方老率眾會之。冬十一月辛酉,僧辨次
 於姑孰,即留鎮焉。遣豫州刺史侯瑱據東關壘,征吳興太守裴之橫帥眾繼之。戊戌,以尚書右僕射王褒為尚書左僕射,湘東太守張綰為尚書右僕射。十二月,宿預土民東方光據城歸化,魏江西州郡皆起兵應之。



 三年春正月甲午,加南豫州刺史侯瑱征北將軍、開府儀同三司。陳霸先帥眾攻廣陵城。秦州刺史嚴超達自秦郡圍涇州,侯瑱、張彪出石梁,為其聲援。辛丑,陳霸先遣晉陵太守杜僧明率眾助東方光。三月甲辰,以司徒王僧辯為太尉、車騎大將軍。丁未,魏遣將王球率眾七百攻宿預,杜僧明逆擊,大破之。戊申,以護軍將軍、郢
 州刺史陸法和為司徒。夏四月癸酉,以征北大將軍、開府儀同三司陳霸先為司空。六月壬午,魏復遣將步六汗薩率眾救涇州。癸未,有黑氣如龍,見於殿內。秋七月甲辰,以都官尚書宗懍為吏部尚書。九月辛卯,世祖於龍光殿述《老子》義,尚書左僕射王褒為執經。乙巳,魏遣其柱國萬紐于謹率大眾來寇。冬十月丙寅,魏軍至於襄陽,蕭察率眾會之。丁卯,停講,內外戒嚴,輿駕出行都柵。是日,大風拔木,丙子,徵王僧辯等軍。十一月,以領軍胡僧祐都督城東城北諸軍事,右僕射張綰為副;左僕射王褒都督城西城南諸軍事,直殿省元景亮為副。王
 公卿士各有守備。丙戌,世祖遍行都柵,皇太子巡行城樓,使居民助運水石,諸要害所,並增兵備。丁亥,魏軍至柵下。丙申,徵廣州刺史王琳入援。丁酉,大風,城內火。以胡僧祐為開府儀同三司,巂州刺史裴畿為領軍將軍。庚子,信州刺史徐世譜、晉安王司馬任約軍次馬頭岸。戊申,胡僧祐、朱買臣等率兵出戰,買臣敗績。己酉,降左僕射王褒為護軍將軍。辛亥,魏軍大攻,世祖出枇杷門,親臨陣督戰。胡僧祐中流矢薨。六軍敗績。反者斬西門關以納魏師,城陷於西魏。世祖見執,如蕭察營,又遷還城內。十二月丙辰,徐世譜、任約退戍巴陵。辛未,西魏害
 世祖,遂崩焉,時年四十七。太子元良、始安王方略皆見害。乃選百姓男女數萬口,分為奴婢,驅入長安;小弱者皆殺之。明年四月,追尊為孝元皇帝,廟曰世祖。



 世祖聰悟俊朗,天才英發。年五歲,高祖問:「汝讀何書?」對曰:「能誦《曲禮》。」高祖曰:「汝試言之。」即誦上篇,左右莫不驚歎。初生患眼,高祖自下意治之,遂盲一目,彌加愍愛。既長好學,博綜群書,下筆成章,出言為論,才辯敏速,冠絕一時。高祖嘗問曰:「孫策昔在江東,於時年幾?」答曰:「十七。」高祖曰:「正是汝年。」賀革為府諮議,敕革講《三禮》。世祖性不好聲色,頗有高名,與裴子野、劉顯、蕭子雲、張纘及當時才秀
 為布衣之交,著述辭章,多行於世。在尋陽,夢人曰:「天下將亂,王必維之。」又背生黑子,巫媼見曰:「此大貴兆,當不可言。」初,賀革西上,意甚不悅,過別御史中丞江革,以情告之。革曰:「吾嘗夢主上遍見諸子,至湘東王,手脫帽授之。此人後必當璧,卿其行乎!」革從之。及太清之難,乃能克復,故遐邇樂推,遂膺寶命矣。所著《孝德傳》三十卷,《忠臣傳》三十卷,《丹陽尹傳》十卷。《注漢書》一百一十五卷,《周易講疏》十卷,《內典博要》一百卷,《連山》三十卷,《洞林》三卷,《玉韜》十卷,《補闕子》十卷,《老子講疏》四卷,《全德志》、《懷舊志》、《荊南志》、《江州記》、《貢職圖》、《古今同姓名錄》一卷,《筮經》十二
 卷,《式贊》三卷,文集五十卷。



 史臣曰:梁季之禍,巨寇憑壘,世祖時位長連率,有全楚之資,應身率群後,枕戈先路。虛張外援,事異勤王,在於行師,曾非百舍。後方殲夷大憝,用寧宗社,握圖南面,光啟中興,亦世祖雄才英略,紹茲寶運者也。而稟性猜忌,不隔疏近,御下無術,履冰弗懼,故鳳闕伺晨之功,火無內照之美。以世祖之神睿特達,留情政道,不怵邪說,徙蹕金陵,左鄰強寇,將何以作?是以天未悔禍,蕩覆斯生,悲夫!



\end{pinyinscope}