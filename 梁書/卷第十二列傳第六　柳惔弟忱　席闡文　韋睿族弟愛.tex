\article{卷第十二列傳第六 柳惔弟忱 席闡文 韋睿族弟愛}

\begin{pinyinscope}

 柳惔,字文通,河東解人也。父世隆,齊司空。惔年十七,齊武帝為中軍,命為參軍,轉主簿。齊初,入為尚書三公郎,累遷太子中舍人,巴東王子響友。子響為荊州,惔隨之鎮。子響暱近小人,惔知將為禍,稱疾還京。及難作,惔以先歸得免。歷中書侍郎,中護軍長史。出為新安太守,居郡,以無政績,免歸。久之,為右軍諮議參軍事。



 建武末,為
 西戎校尉、梁、南秦二州刺史。及高祖起兵,惔舉漢中應義。和帝即位,以為侍中,領前軍將軍。高祖踐阼,徵為護軍將軍,未拜,仍遷太子詹事,加散騎常侍。論功封曲江縣侯,邑千戶。高祖因宴為詩以貽惔曰:「爾實冠群后,惟余實念功。」又嘗侍座,高祖曰:「徐元瑜違命嶺南,《周書》罪不相及,朕已宥其諸子,何如?」惔對曰:「罰不及嗣,賞延于世,今復見之聖朝。」時以為知言。尋遷尚書右僕射。



 天監四年,大舉北伐,臨川王宏都督眾軍,以惔為副。軍還,復為僕射。以久疾,轉金紫光祿大夫,加散騎常侍,給親信二十人。未拜,出為使持節、安南將軍、湘州刺史。六年十
 月,卒于州,時年四十六。高祖為素服舉哀。贈侍中、撫軍將軍,給鼓吹一部。謚曰穆。惔著《仁政傳》及諸詩賦,粗有辭義。子照嗣。



 惔第四弟憕,亦有美譽,歷侍中、鎮西長史。天監十二年,卒,贈寧遠將軍、豫州刺史。



 忱字文若,惔第五弟也。年數歲,父世隆及母閻氏時寢疾,忱不解帶經年。及居喪,以毀聞。起家為司徒行參軍,累遷太子中舍人,西中郎主簿,功曹史。



 齊東昏遣巴西太守劉山陽由荊襲高祖,西中郎長史蕭穎胄計未有定,召忱及其所親席闡文等夜入議之。忱曰:「朝廷狂悖,為惡日滋。頃聞京師長者,莫不重足累息;今幸在遠,得
 假日自安。雍州之事,且藉以相斃耳。獨不見蕭令君乎?以精兵數千,破崔氏十萬眾,竟為群邪所陷,禍酷相尋。前事之不忘,後事之師也。若使彼凶心已逞,豈知使君不係踵而及?且雍州士銳糧多,蕭使君雄姿冠世,必非山陽所能擬;若破山陽,荊州復受失律之責。進退無可,且深慮之。」闡文亦深勸同高祖。穎胄乃誘斬山陽,以忱為寧朔將軍。



 和帝即位,為尚書吏部郎,進號輔國將軍、南平太守。尋遷侍中、冠軍將軍,太守如故。轉吏部尚書,不拜。郢州平,穎胄議遷都夏口,忱復固諫,以為巴硤未賓,不宜輕捨根本,搖動民志。穎胄不從。俄而巴東兵至
 硤口,遷都之議乃息。論者以為見機。



 高祖踐阼,以忱為五兵尚書,領驍騎將軍。論建義功,封州陵伯,邑七百戶。天監二年,出為安西長史、冠軍將軍、南郡太守。六年,徵為員外散騎常侍、太子右衛率。未發,遷持節、督湘州諸軍事、輔國將軍、湘州刺史。八年,坐輒放從軍丁免。俄入為秘書監,遷散騎常侍,轉祠部尚書,未拜遇疾,詔改授給事中、光祿大夫,疾篤不拜。十年,卒於家,時年四十一。追贈中書令,謚曰穆。子範嗣。



 席闡文,安定臨涇人也。少孤貧,涉獵書史。齊初,為雍州刺史蕭赤斧中兵參軍,由是與其子穎胄善。復歷西中
 郎中兵參軍,領城局。高祖之將起義也,闡文深勸之,穎胄同焉,仍遣田祖恭私報高祖,並獻銀裝刀,高祖報以金如意。和帝稱尊號,為給事黃門侍郎,尋遷衛尉卿。穎胄暴卒,州府騷擾,闡文以和帝幼弱,中流任重,時始興王憺留鎮雍部,用與西朝群臣迎王總州事,故賴以寧輯。高祖受禪,除都官尚書、輔國將軍。封山陽伯,邑七百戶。出為東陽太守,又改封湘西,戶邑如故。視事二年,以清白著稱,卒於官。詔賻錢三萬,布五十匹。謚曰威。



 韋睿,字懷文,京兆杜陵人也。自漢丞相賢以後,世為三輔著姓。祖玄,避吏隱於長安南山。宋武帝入關,以太尉
 掾征,不至。伯父祖征,宋末為光祿勛。父祖歸,寧遠長史。睿事繼母以孝聞。睿兄纂、闡,並早知名。纂、睿皆好學,闡有清操。祖征累為郡守,每攜睿之職,視之如子。時睿內兄王憕、姨弟杜惲,並有鄉里盛名。祖征謂睿曰:「汝自謂何如憕、惲?」睿謙不敢對。祖徵曰:「汝文章或小減,學識當過之;然而幹國家,成功業,皆莫汝逮也。」外兄杜幼文為梁州刺史,要睿俱行。梁土富饒,往者多以賄敗;睿時雖幼,獨用廉聞。



 宋永光初,袁抃為雍州刺史,見而異之,引為主簿。抃到州,與鄧琬起兵,睿求出為義成郡,故免抃之禍。後為晉平王左常侍,遷司空桂陽王行參軍,隨齊
 司空柳世隆守郢城,拒荊州刺史沈攸之。攸之平,遷前軍中兵參軍。久之,為廣德令。累遷齊興太守、本州別駕、長水校尉、右軍將軍。齊末多故,不欲遠鄉里,求為上庸太守,加建威將軍。俄而太尉陳顯達、護軍將軍崔慧景頻逼京師,民心遑駭,未有所定,西土人謀之於睿。睿曰:「陳雖舊將,非命世才;崔頗更事,懦而不武。其取赤族也,宜哉!天下真人,殆興於吾州矣。」乃遣其二子,自結於高祖。



 義兵檄至,睿率郡人伐竹為筏,倍道來赴,有眾二千,馬二百匹。高祖見睿甚悅,拊几曰:「他日見君之面,今日見君之心,吾事就矣。」義師剋郢、魯,平加湖,睿多建謀策,
 皆見納用。大軍發郢,謀留守將,高祖難其人;久之,顧睿曰:「棄騏驥而不乘,焉遑遑而更索?」即日以為冠軍將軍、江夏太守,行郢府事。初,郢城之拒守也,男女口垂十萬,閉壘經年,疾疫死者十七八,皆積屍於床下,而生者寢處其上,每屋輒盈滿。睿料簡隱恤,咸為營理,於是死者得埋藏,生者反居業,百姓賴之。



 梁臺建,徵為大理。高祖即位,遷廷尉,封都梁子,邑三百戶。天監二年,改封永昌,戶邑如先。東宮建,遷太子右衛率,出為輔國將軍、豫州刺史、領歷陽太守。三年,魏遣眾來寇,率州兵擊走之。



 四年,王師北伐,詔睿都督眾軍。睿遣長史王超宗、梁郡太
 守馮道根攻魏小峴城,未能拔。睿巡行圍柵,魏城中忽出數百人陳於門外,睿欲擊之,諸將皆曰:「向本輕來,未有戰備,徐還授甲,乃可進耳。」睿曰:「不然。魏城中二千餘人,閉門堅守,足以自保,無故出人於外,必其驍勇者也,若能挫之,其城自拔。」眾猶遲疑,睿指其節曰;「朝廷授此,非以為飾,韋睿之法,不可犯也。」乃進兵。士皆殊死戰,魏軍果敗走,因急攻之,中宿而城拔。遂進討合肥。先是,右軍司馬胡略等至合肥,久未能下,睿按行山川,曰:「吾聞『汾水可以灌平陽,絳水可以灌安邑』,即此是也。」乃堰肥水,親自表率,頃之,堰成水通,舟艦繼至。魏初分築東
 西小城夾合肥,睿先攻二城。既而魏援將揚靈胤帥軍五萬奄至,眾懼不敵,請表益兵。睿笑曰:「賊已至城下,方復求軍,臨難鑄兵,豈及馬腹?且吾求濟師,彼亦徵眾,猶如吳益巴丘,蜀增白帝耳。『師克在和不在眾』,古之義也。」因與戰,破之,軍人少安。



 初,肥水堰立,使軍主王懷靜築城於岸守之,魏攻陷懷靜城,千餘人皆沒。魏人乘勝至睿堤下,其勢甚盛,軍監潘靈祐勸睿退還巢湖,諸將又請走保三叉。睿怒曰:「寧有此邪!將軍死綏,有前無卻。」因令取傘扇麾幢,樹之堤下,示無動志。睿素羸,每戰未嘗騎馬,以板輿自載,督厲眾軍。魏兵來鑿堤,睿親與爭之,
 魏軍少卻,因築壘於堤以自固。睿起鬥艦,高與合肥城等,四面臨之。魏人計窮,相與悲哭。睿攻具既成,堰水又滿,魏救兵無所用。魏守將杜元倫登城督戰,中弩死,城遂潰。俘獲萬餘級,牛馬萬數,絹滿十間屋,悉充軍賞。睿每晝接客旅,夜算軍書,三更起張燈達曙,撫循其眾,常如不及,故投募之士爭歸之。所至頓舍脩立,館宇籓籬牆壁,皆應準繩。



 合肥既平,高祖詔眾軍進次東陵。東陵去魏甓城二十里,將會戰,有詔班師。去賊既近,懼為所躡,睿悉遣輜重居前,身乘小輿殿後,魏人服睿威名,望之不敢逼,全軍而還。至是遷豫州於合肥。



 五年,魏中山
 王元英寇北徐州,圍刺史昌義之於鐘離,眾號百萬,連城四十餘。高祖遣征北將軍曹景宗,都督眾軍二十萬以拒之。次邵陽洲,築壘相守,高祖詔睿率豫州之眾會焉。睿自合肥逕道由陰陵大澤行,值澗谷,輒飛橋以濟。師人畏魏軍盛,多勸睿緩行。睿曰:「鐘離今鑿穴而處,負戶而汲,車馳卒奔,猶恐其後,而況緩乎!魏人已墮吾腹中,卿曹勿憂也。」旬日而至邵陽。初,高祖敕景宗曰:「韋睿,卿之鄉望,宜善敬之。」景宗見睿,禮甚謹。高祖聞之,曰:「二將和,師必濟矣。」睿於景宗營前二十里,夜掘長塹,樹鹿角,截洲為城,比曉而營立。元英大驚,以杖擊地曰:「是何
 神也!」明旦,英自率眾來戰,睿乘素木輿,執白角如意麾軍,一日數合,英甚憚其彊。魏軍又夜來攻城,飛矢雨集,睿子黯請下城以避箭,睿不許。軍中驚,睿於城上厲聲呵之,乃定。魏人先於邵陽洲兩岸為兩橋,樹柵數百步,跨淮通道。睿裝大艦,使梁郡太守馮道根、廬江太守裴邃、秦郡太守李文釗等為水軍。值淮水暴長,睿即遣之,鬥艦競發,皆臨敵壘。以小船載草,灌之以膏,從而焚其橋。風怒火盛,煙塵晦冥,敢死之士,拔柵斫橋,水又漂疾,倏忽之間,橋柵盡壞。而道根等皆身自搏戰,軍人奮勇,呼聲動天地,無不一當百,魏人大潰。元英見橋絕,脫身遁去。
 魏軍趨水死者十餘萬,斬首亦如之。其餘釋甲稽顙,乞為囚奴,猶數十萬。所獲軍實牛馬,不可勝紀。睿遣報昌義之,義之且悲且喜,不暇答語,但叫曰:「更生!更生!」高祖遣中書郎周捨勞於淮上,睿積所獲於軍門,捨觀之,謂睿曰:「君此獲復與熊耳山等。」以功增封七百戶,進爵為侯,徵通直散騎常侍、右衛將軍。



 七年,遷左衛將軍,俄為安西長史、南郡太守,秩中二千石。會司州刺史馬仙琕北伐還軍,為魏人所躡,三關擾動,詔睿督眾軍援焉。睿至安陸,增築城二丈餘,更開大塹,起高樓,眾頗譏其示弱。睿曰:「不然,為將當有怯時,不可專勇。」是時元英復追
 仙琕,將復邵陽之恥,聞睿至,乃退。帝亦詔罷軍。明年,遷信武將軍、江州刺史。九年,徵員外散騎常侍、右衛將軍,累遷左衛將軍、太子詹事,尋加通直散騎常侍。十三年,遷智武將軍、丹陽尹,以公事免。頃之,起為中護軍。



 十四年,出為平北將軍、寧蠻校尉、雍州刺史。初,睿起兵鄉中,客陰俊光泣止睿,睿還為州,俊光道候睿,睿笑謂之曰:「若從公言,乞食於路矣。」餉耕牛十頭。睿於故舊,無所遺惜,士大夫年七十以上,多與假板縣令,鄉里甚懷之。十五年,拜表致仕,優詔不許。十七年,徵散騎常侍、護軍將軍,尋給鼓吹一部,入直殿省。居朝廷,恂恂未嘗忤視,高
 祖甚禮敬之。性慈愛,撫孤兄子過於己子,歷官所得祿賜,皆散之親故,家無餘財。後為護軍,居家無事,慕萬石、陸賈之為人,因畫之於壁以自玩。時雖老,暇日猶課諸兒以學。第三子棱,尤明經史,世稱其洽聞,睿每坐棱使說書,其所發擿,棱猶弗之逮也。高祖方銳意釋氏,天下咸從風而化;睿自以信受素薄,位居大臣,不欲與俗俯仰,所行略如他日。



 普通元年夏,遷侍中、車騎將軍,以疾未拜。八月,卒于家,時年七十九。遺令薄葬,斂以時服。高祖即日臨哭甚慟。賜錢十萬,布二百匹,東園秘器,朝服一具,衣一襲,喪事取給於官,遣中書舍人監護。贈侍中、
 車騎將軍、開府儀同三司。謚曰嚴。



 初,邵陽之役,昌義之甚德睿,請曹景宗與睿會,因設錢二十萬官賭之,景宗擲得雉,睿徐擲得盧,遽取一子反之,曰「異事」,遂作塞。景宗時與群帥爭先啟之捷,睿獨居後,其不尚勝,率多如是,世尤以此賢之。子放、正、棱、黯,放別有傳。



 正字敬直,起家南康王行參軍,稍遷中書侍郎,出為襄陽太守。初,正與東海王僧孺友善,及僧孺為尚書吏部郎,參掌大選,賓友故人莫不傾意,正獨澹然。及僧孺擯廢之後,正復篤素分,有踰曩日,論者稱焉。歷官至給事黃門侍郎。



 棱字威直,性恬素,以書史為業,博物彊記,當世之士,咸就
 質疑。起家安成王府行參軍,稍遷治書侍御史,太子僕,光祿卿。著《漢書續訓》三卷。



 黯字務直,性彊正,少習經史,有文詞。起家太子舍人,稍遷太僕卿,南豫州刺史,太府卿。侯景濟江,黯屯六門,尋改為都督城西面諸軍事。時景於城外起東西二土山,城內亦作以應之,太宗親自負土,哀太子以下躬執畚鍤。黯守西土山,晝夜苦戰,以功授輕車將軍,加持節。卒於城內,贈散騎常侍、左衛將軍。睿族弟愛。



 愛字孝友,沈靜有器局。高祖父廣,晉後軍將軍、北平太守。曾祖軌,以孝武太元之初,南遷襄陽,為本州別駕,散
 騎侍郎。祖公循,宋義陽太守。父義正,早卒。



 愛少而偏孤,事母以孝聞。性清介,不妄交遊,而篤志好學,每虛室獨坐,遊心墳素,而埃塵滿席,寂若無人。年十二,嘗遊京師,值天子出遊南苑,邑里喧嘩,老幼爭觀,愛獨端坐讀書,手不釋卷,宗族見者,莫不異焉。及長,博學有文才,尤善《周易》及《春秋左氏》義。



 袁抃為雍州刺史,辟為主簿。遭母憂,廬於墓側,負土起墳。高祖臨雍州,聞之,親往臨弔。服闋,引為中兵參軍。義師之起也,以愛為壯武將軍、冠軍南平王司馬,帶襄陽令。時京邑未定,雍州空虛,魏興太守顏僧都等據郡反,州內驚擾,百姓攜貳。愛沉敏有謀,
 素為州里信伏,乃推心撫御,曉示逆順;兼率募鄉里,得千餘人,與僧都等戰於始平郡南,大破之,百姓乃安。



 蕭穎胄之死也,和帝徵兵襄陽,愛從始興王憺赴焉。先是,巴東太守蕭璝、巴東太守魯休烈舉兵來逼荊州,及憺至,令愛書諭之,璝即日請降。



 中興二年,從和帝東下。高祖受禪,進號輔國將軍,仍為驍騎將軍,尋除寧蜀太守,與益州刺史鄧元起西上襲劉季連,行至公安,道病卒,贈衛尉卿。子乾向,官至驍騎將軍,征北長史,汝陰、鐘離二郡太守。



 陳吏部尚書姚察曰:昔竇融以河右歸漢,終為盛族;柳
 惔舉南鄭響從,而家聲弗霣,時哉!忱之謀畫,亦用有成,智矣。韋睿起上庸以附義,其地比惔則薄,及合肥、邵陽之役,其功甚盛,推而弗有,君子哉!



\end{pinyinscope}