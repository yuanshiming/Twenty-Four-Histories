\article{卷第三十七列傳第三十一 謝舉 何敬容}

\begin{pinyinscope}

 謝舉,字言揚,中書令覽之弟也。幼好學,能清言,與覽齊名。舉年十四,嘗贈沈約五言詩,為約稱賞。世人為之語曰:「王有養、炬,謝有覽、舉。」養、炬,王筠、王泰小字也。起家秘書郎,遷太子舍人,輕車功曹史,秘書丞,司空從事中郎,太子庶子,家令,掌東宮管記,深為昭明太子賞接。秘書監任昉出為新安郡,別舉詩云:「詎念耋嗟人,方深老夫
 託。」其屬意如此。嘗侍宴華林園,高祖訪舉於覽,覽對曰:「識藝過臣甚遠,惟飲酒不及於臣。」高祖大悅。轉太子中庶子,猶掌管記。



 天監十一年,遷侍中。十四年,出為寧遠將軍、豫章內史,為政和理,甚得民心。十八年,復入為侍中,領步兵校尉。普通元年,出為貞毅將軍、太尉臨川王長史。四年,入為左民尚書。其年遷掌吏部,尋以公事免。五年,起為太子中庶子,領右軍將軍。六年,復為左民尚書,領步兵校尉。俄徙為吏部尚書,尋加侍中。出為仁威將軍、晉陵太守。在郡清靜,百姓化其德,境內肅然。罷郡還,吏民詣闕請立碑,詔許之。大通二年,入為侍中、五兵
 尚書,未拜,遷掌吏部,侍中如故。舉祖莊,宋世再典選,至舉又三為此職,前代未有也。



 舉少博涉多通,尤長玄理及釋氏義。為晉陵郡時,常與義僧遞講經論,徵士何胤自虎丘山赴之。其盛如此。先是,北渡人盧廣有儒術,為國子博士,於學發講,僕射徐勉以下畢至。舉造坐,屢折廣,辭理通邁。廣深歎服,仍以所執麈尾薦之,以況重席焉。



 四年,加侍中。五年,遷尚書右僕射,侍中如故。大同三年,以疾陳解,徙為右光祿大夫,給親信二十人。其年,出為雲麾將軍、吳郡太守。先是,何敬容居郡有美績,世稱為何吳郡。及舉為政,聲跡略相比。六年,入為侍中、中書
 監,未拜,遷太子詹事、翊左將軍,侍中如故。舉父綍,齊世終此官,累表乞改授,敕不許,久之方就職。九年,遷尚書僕射,侍中、將軍如故。舉雖居端揆,未嘗肯預時務,多因疾陳解。敕輒賜假,并手敕處方,加給上藥。其恩遇如此。其年,以本官參掌選事。太清二年,遷尚書令,侍中、將軍如故。是歲,侯景寇京師,舉卒于圍內。詔贈侍中、中衛將軍、開府儀同三司,侍中、尚書令如故。文集亂中並亡逸。



 二子禧,嘏,並少知名。嘏,太清中,歷太子中庶子,出為建安太守。



 何敬容,字國禮,廬江人也。祖攸之,宋太常卿;父昌珝,齊
 吏部尚書;並有名前代。敬容以名家子,弱冠選尚齊武帝女長城公主,拜駙馬都尉。天監初,為秘書郎,歷太子舍人,尚書殿中郎,太子洗馬,中書舍人,秘書丞,遷揚州治中。出為建安內史,清公有美績,民吏稱之。還除黃門郎,累遷太子中庶子,散騎常侍,侍中,司徒左長史。普通二年,復為侍中,領羽林監,俄又領本州大中正。頃之,守吏部尚書,銓序明審,號為稱職。四年,出為招遠將軍、吳郡太守,為政勤恤民隱,辨訟如神,視事四年,治為天下第一。吏民詣闕請樹碑,詔許之。大通二年,徵為中書令,未拜,復為吏部尚書,領右軍將軍,俄加侍中。中大通元
 年,改太子中庶子。



 敬容身長八尺,白皙美鬚眉。性矜莊,衣冠尤事鮮麗,每公庭就列,容止出人。三年,遷尚書右僕射,參掌選事,侍中如故。時僕射徐勉參掌機密,以疾陳解,因舉敬容自代,故有此授焉。五年,遷左僕射,加宣惠將軍,置佐史,侍中、參掌如故。大同三年正月,朱雀門災,高祖謂群臣曰:「此門制卑狹,我始欲構,遂遭天火。」並相顧未有答。敬容獨曰:「此所謂陛下『先天而天不違』。」時以為名對。俄遷中權將軍、丹陽尹,侍中、參掌、佐史如故。五年,入為尚書令,侍中、將軍、參掌、佐史如故。



 敬容久處臺閣,詳悉舊事,且聰明識治,勤於簿領,詰朝理事,日旰
 不休。自晉、宋以來,宰相皆文義自逸,敬容獨勤庶務,為世所嗤鄙。時蕭琛子巡者,頗有輕薄才,因制卦名離合等詩以嘲之,敬容處之如初,亦不屑也。



 十一年,坐妾弟費慧明為導倉丞,夜盜官米,為禁司所執,送領軍府。時河東王譽為領軍將軍,敬容以書解慧明,譽即封書以奏。高祖大怒,付南司推劾。御史中丞張綰奏敬容挾私罔上,合棄市刑,詔特免職。初,天監中,有沙門釋寶誌者,嘗遇敬容,謂曰:「君後必貴,然終是何敗何耳」。及敬容為宰相,謂何姓當為其禍,故抑沒宗族,無仕進者,至是竟為河東所敗。



 中大同元年三月,高祖幸同泰寺講《金字
 三慧經》,敬容請預聽,敕許之。又有敕聽朔望問訊。尋起為金紫光祿大夫,未拜,又加侍中。敬容舊時賓客門生喧嘩如昔,冀其復用。會稽謝郁致書戒之曰:「草萊之人,聞諸道路,君侯已得瞻望朝夕,出入禁門,醉尉將不敢呵,灰然不無其漸,甚休,甚休!敢賀於前,又將弔也。昔流言裁作,公旦東奔;燕書始來,子孟不入。夫聖賢被虛過以自斥,未有嬰時釁而求親者也。且曝鰓之鱗,不念杯杓之水;雲霄之翼,豈顧籠樊之糧。何者?所託已盛也。昔君侯納言加首,鳴玉在腰,回豊貂以步文昌,聳高蟬而趨武帳,可謂盛矣。不以此時薦才拔士,少報聖主之恩;
 今卒如爰絲之說,受責見過,方復欲更窺朝廷,觖望萬分,竊不為左右取也。昔竇嬰、楊惲亦得罪明時,不能謝絕賓客,猶交黨援,卒無後福,終益前禍。僕之所弔,實在於斯。人人所以頗猶有踵君侯之門者,未必皆感惠懷仁,有灌夫、任安之義,乃戒翟公之大署,冀君侯之復用也。夫在思過之日,而挾復用之意,未可為智者說矣。君侯宜杜門念失,無有所通,築茅茨於鐘阜,聊優游以卒歲,見可憐之意,著待終之情。復仲尼能改之言,惟子貢更也之譬,少戢言於眾口,微自救於竹帛,所謂『失之東隅,收之桑榆』。如此,令明主聞知,尚有冀也。僕東皋鄙人,
 入穴幸無銜窶,恥天下之士不為執事道之,故披肝膽,示情素,君侯豈能鑒焉。」



 太清元年,遷太子詹事,侍中如故。二年,侯景襲京師,敬容自府移家臺內。初,景於渦陽退敗,未得審實,傳者乃云其將暴顯反,景身與眾並沒,朝廷以為憂。敬容尋見東宮,太宗謂曰:「淮北始更有信,侯景定得身免,不如所傳。」敬容對曰:「得景遂死,深是朝廷之福。」太宗失色,問其故。敬容曰:「景翻覆叛臣,終當亂國。」是年,太宗頻於玄圃自講《老》、《莊》二書,學士吳孜時寄詹事府,每日入聽。敬容謂孜曰:「昔晉代喪亂,頗由祖尚玄虛,胡賊殄覆中夏。今東宮復襲此,殆非人事,其將為
 戎乎?」俄而侯景難作,其言有徵也。三年正月,敬容卒於圍內,詔贈仁威將軍,本官並如故。



 何氏自晉司空充、宋司空尚之,世奉佛法,並建立塔寺;至敬容又舍宅東為伽藍,趨勢者因助財造構,敬容並不拒,故此寺堂宇校飾,頗為宏麗。時輕薄者因呼為「眾造寺」焉。及敬容免職出宅,止有常用器物及囊衣而已,竟無餘財貨,時亦以此稱之。



 子鷿,秘書丞,早卒。



 陳吏部尚書姚察曰:魏正始及晉之中朝,時俗尚於玄虛,貴為放誕,尚書丞郎以上,簿領文案,不復經懷,皆成於令史。逮乎江左,此道彌扇,惟卞壼以臺閣之務,頗欲
 綜理,阮孚謂之曰:「卿常無閑暇,不乃勞乎?」宋世王敬弘身居端右,未嘗省牒,風流相尚,其流遂遠。望白署空,是稱清貴;恪勤匪懈,終滯鄙俗。是使朝經廢於上,職事隳於下。小人道長,抑此之由。嗚呼!傷風敗俗,曾莫之悟。永嘉不競,戎馬生郊,宜其然矣。何國禮之識治,見譏薄俗,惜哉!



\end{pinyinscope}