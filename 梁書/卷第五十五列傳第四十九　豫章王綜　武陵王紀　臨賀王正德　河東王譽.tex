\article{卷第五十五列傳第四十九 豫章王綜 武陵王紀 臨賀王正德 河東王譽}

\begin{pinyinscope}

 豫章王綜,字世謙,高祖第二子也。天監三年,封豫章郡王,邑二千戶。五年,出為使持節、都督南徐州諸軍事、仁威將軍、南徐州刺史,尋進號北中郎將。十年,遷都督郢、司、霍三州諸軍事、雲麾將軍、郢州刺史。十三年,遷安右將軍、領石頭戍軍事。十五年,遷西中郎將,兼護軍將軍,
 又遷安前將軍、丹陽尹。十六年,復為北中郎將、南徐州刺史。普通二年,入為侍中、鎮右將軍,置佐史。



 初,其母吳淑媛自齊東昏宮得幸於高祖,七月而生綜,宮中多疑之者。及淑媛寵衰怨望,遂陳疑似之說,故綜懷之。既長,有才學,善屬文。高祖御諸子以禮,朝見不甚數,綜恒怨不見知。每出籓,淑媛恆隨之鎮。至年十五六,尚裸袒嬉戲於前,晝夜無別,內外咸有穢議。綜在徐州,政刑酷暴。又有勇力,手制奔馬。常微行夜出,無有期度。每高祖有敕疏至,輒忿恚形於顏色,群臣莫敢言者。恆於別室祠齊氏七廟,又微服至曲阿拜齊明帝陵。然猶無以自信,
 聞俗說以生者血瀝死者骨,滲,即為父子。綜乃私發齊東昏墓,出骨,瀝臂血試之。並殺一男,取其骨試之,皆有驗,自此常懷異志。



 四年,出為使持節、都督南兗、兗、徐、青、冀五州諸軍事、平北將軍、南兗州刺史,給鼓吹一部。聞齊建安王蕭寶寅在魏,遂使人入北與之相知,謂為叔父,許舉鎮歸之。會大舉北伐。六年,魏將元法僧以彭城降,高祖乃令綜都督眾軍,鎮於彭城,與魏將安豊王元延明相持。高祖以連兵既久,慮有釁生,敕綜退軍。綜懼南歸則無因復與寶寅相見,乃與數騎夜奔于延明,魏以為侍中、太尉、高平公、丹陽王,邑七千戶,錢三百萬,布
 絹三千匹,雜彩千匹,馬五十匹,羊五百口,奴婢一百人。綜乃改名纘,字德文,追為齊東昏服斬衰。於是有司奏削爵土,絕屬籍,改其姓為悖氏。俄有詔復之,封其子直為永新侯,邑千戶。大通二年,蕭寶寅在魏據長安反,綜自洛陽北遁,將赴之,為津吏所執,魏人殺之,時年四十九。



 初,綜既不得志,嘗作《聽鐘鳴》、《悲落葉》辭,以申其志。大略曰:聽鐘鳴,當知在帝城。參差定難數,歷亂百愁生。去聲懸窈窕,來響急徘徊。誰憐傳漏子,辛苦建章臺。



 聽鐘鳴,聽聽非一所。懷瑾握瑜空擲去,攀松折桂誰相許?昔朋舊愛各東西,譬如落葉不更齊。漂漂孤鴈何所棲,依
 依別鶴夜半啼。



 聽鐘鳴,聽此何窮極?二十有餘年,淹留在京域。窺明鏡,罷容色,雲悲海思徒掩抑。



 其《悲落葉》云:悲落葉,連翩下重疊。落且飛,縱橫去不歸。



 悲落葉,落葉悲。人生譬如此,零落不可持。



 悲落葉,落葉何時還?夙昔共根本,無復一相關。



 當時見者莫不悲之。



 武陵王紀,字世詢,高祖第八子也。少勤學,有文才,屬辭不好輕華,甚有骨氣。天監十三年,封為武陵郡王,邑二千戶。歷位寧遠將軍、瑯邪、彭城二郡太守、輕車將軍、丹陽尹。出為會稽太守,尋以其郡為東揚州,仍為刺史,加使持節、東中郎將。徵為侍中,領石頭戍軍事。出為宣惠
 將軍、江州刺史。徵為使持節、宣惠將軍、都督揚、南徐二州諸軍事、揚州刺史。尋改授持節、都督益、梁等十三州諸軍事、安西將軍、益州刺史,加鼓吹一部。大同十一年,授散騎常侍、征西大將軍、開府儀同三司。



 初,天監中,震太陽門,成字曰「紹宗梁位唯武王」,解者以為武王者,武陵王也,於是朝野屬意焉。及太清中,侯景亂,紀不赴援。高祖崩後,紀乃僭號於蜀,改年曰天正。立子圓照為皇太子,圓正為西陽王,圓滿竟陵王,圓普南譙王,圓肅宜都王。以巴西、梓潼二郡太守永豊侯捴為征西大將軍、益州刺史,封秦郡王。司馬王僧略、直兵參軍徐怦並固諫,紀以為貳於己,皆殺之。永豊侯捴嘆曰:「王不免矣!夫善人國之基也,今反誅之,不亡何待!」又謂所親曰:「昔桓玄年號大亨,識者謂
 之『二月了』,而玄之敗實在仲春。今年曰天正,在文為『一止』,其能久乎?」



 太清五年夏四月,紀帥軍東下至巴郡,以討侯景為名,將圖荊陜。聞西魏侵蜀,遣其將南梁州刺史譙淹回軍赴援。五月日,西魏將尉遲迥帥眾逼涪水,潼州刺史楊乾運以城降之,迥分軍據守,即趨成都。丁丑,紀次於西陵,舳艫翳川,旌甲曜日,軍容甚盛。世祖命護軍將軍陸法和於硤口夾岸築二壘,鎮江以斷之。時陸納未平,蜀軍復逼,物情恇擾,世祖憂焉。法和告急,旬日相繼。世祖乃拔任約於獄,以為晉安王司馬,撤禁兵以配之;並遣宣猛將軍劉棻共約西赴。六月,紀築連城,攻絕鐵鏁。世祖復於獄拔謝答仁為步兵校尉,配眾一旅,上赴法和。世祖與紀書曰:「皇帝敬問假黃鉞太尉武陵王:自九黎侵軼,三苗寇擾,天長喪亂,獯醜
 馮陵,虔劉象魏,黍離王室。朕枕戈東望,泣血西浮,殞愛子於二方,無諸侯之八百,身被屬甲,手貫流矢。俄而風樹之酷,萬恨始纏,霜露之悲,百憂繼集,扣心飲膽,志不圖全。直以宗社綴旒,鯨鯢未剪,嘗膽待旦,龔行天罰,獨運四聰,坐揮八柄。雖復結壇待將,褰帷納士,拒赤壁之兵,無謀於魯肅;燒烏巢之米,不訪於荀攸;才智將殫,金貝殆竭,傍無寸助,險阻備嘗。遂得斬長狄於駒門,挫蚩尤於楓木。怨恥既雪,天下無塵,經營四方,專資一力,方與岳牧,同茲清靜。隆暑炎赫,弟比何如?文武具僚,當有勞弊。今遣散騎常侍、光州刺史鄭安忠,指宣往懷。」仍令喻意於紀,許其還蜀,專制岷方。紀不從命,報書如家人禮。庚申,紀將侯睿率眾緣山將規進取,任約、謝答仁與戰,破之。既而陸納平,諸軍並西赴,世祖又與紀書曰:「甚苦大智!季月煩暑,流金爍石,聚蚊成雷,封狐千里,以茲玉體,辛苦行陣。乃眷西顧,我勞如何?自獯醜憑陵,羯胡叛換,吾年為一日之長,屬有平亂之功,膺此樂推,事歸當璧。儻遣使乎,良所遲也。如曰不然,於此投筆。友于兄弟,分形共氣。兄肥弟瘦,無復相代之期;讓棗推梨,
 長罷歡愉之日。上林靜拱,聞四鳥之哀鳴;宣室披圖,嗟萬始之長逝。心乎愛矣,書不盡言。」大智,紀之別字也。紀遣所署度支尚書樂奉業至於江陵,論和緝之計,依前旨還蜀。世祖知紀必破,遂拒而不許。丙戌,巴興民苻升、徐子初等斬紀硤口城主公孫晃,降於眾軍。王琳、宋簉、任約、謝答仁等因進攻侯睿,陷其三壘,於是兩岸十餘城遂俱降。將軍樊猛獲紀及其第三子圓滿,俱殺之於硤口,時年四十六。有司奏請絕其屬籍,世祖許之,賜姓饕餮氏。



 初,紀將僭號,妖怪非一。其最異者,內寢柏殿柱繞節生花,其莖四十有六,靃靡可愛,狀似荷花。識者曰:「王敦杖花,非佳事也。」紀年號天正,與蕭棟暗合,僉曰「天」字「二人」也,「正」字「一止」也。棟、紀僭號,各一年而滅。



 臨賀王正德,字公和,臨川靖惠王第三子也。少粗險,不拘禮節。初,高祖未有男,養之為子。及高祖踐極,便希儲貳,後立昭明太子,封正德為西豊侯,邑五百戶。自此怨望,恆懷不軌,睥睨宮扆,覬幸災變。普通六年,以黃門侍郎為輕車將軍,置佐史。頃之,遂逃奔于魏,有司奏削封
 爵。七年,又自魏逃歸,高祖不之過也。復其封爵,仍除征虜將軍。



 中大通四年,為信武將軍、吳郡太守。徵為侍中、撫軍將軍,置佐史,封臨賀郡王,邑二千戶,又加左衛將軍。而凶暴日甚,招聚亡命。侯景知其有姦心,乃密令誘說,厚相要結。遺正德書曰:「今天子年尊,姦臣亂國,憲章錯謬,政令顛倒,以景觀之,計日必敗。況大王屬當儲貳,中被廢辱,天下義士,竊所痛心,在景愚忠,能無忿慨?今四海業業,歸心大王,大王豈得顧此私情,棄茲億兆!景雖不武,實思自奮。願王允副蒼生,鑒斯誠款。」正德覽書大喜曰:「侯景意暗與我同,此天贊也。」遂許之。及景至江,
 正德潛運空舫,詐稱迎荻,以濟景焉。朝廷未知其謀,猶遣正德守朱雀航。景至,正德乃引軍與景俱進,景推正德為天子,改年為正平元年,景為丞相。臺城沒,復太清之號,降正德為大司馬。正德有怨言,景聞之,慮其為變,矯詔殺之。



 河東王譽,字重孫,昭明太子第二子也。普通二年,封枝江縣公。中大通三年,改封河東郡王,邑二千戶。除寧遠將軍、石頭戍軍事。出為瑯邪、彭城二郡太守。還除侍中、輕車將軍,置佐史。出為南中郎將、湘州刺史。



 未幾,侯景寇京邑,譽率軍入援,至青草湖,臺城沒,有詔班師,譽還湘
 鎮。時世祖軍於武城,新除雍州刺史張纘密報世祖曰:「河東起兵,岳陽聚米,共為不逞,將襲江陵。」世祖甚懼,因步道間還,遣諮議周弘直至譽所,督其糧眾。譽曰:「各自軍府,何忽隸人?」前後使三反,譽並不從。世祖大怒,乃遣世子方等征之,反為譽所敗死。又令信州刺史鮑泉討譽,並與書陳示禍福,許其遷善。譽不答,修浚城池,為拒守之計。謂鮑泉曰:「敗軍之將,勢豈語勇?欲前即前,無所多說。」泉軍于石槨寺,譽帥眾逆擊之,不利而還。泉進軍于橘洲,譽又盡銳攻之,不剋。會已暮,士卒疲弊,泉因出擊,大敗之,斬首三千級,溺死者萬餘人。譽於是焚長沙
 郭邑,驅居民於城內,鮑泉度軍圍之。譽幼而驍勇,兼有膽氣,能撫循士卒,甚得眾心。及被圍既久,雖外內斷絕,而備守猶固。後世祖又遣領軍將軍王僧辯代鮑泉攻譽,僧辯築土山以臨城內,日夕苦攻,矢石如雨,城中將士死傷者太半。譽窘急,乃潛裝海船,將潰圍而出。會其麾下將慕容華引僧辯入城,譽顧左右皆散,遂被執,謂守者曰:「勿殺我!得一見七官,申此讒賊,死亦無恨。」主者曰:「奉命不許。」遂斬之,傳首荊鎮,世祖反其首以葬焉。初,譽之將敗也,私引鏡照面,不見其頭;又見長人蓋屋,兩手據地瞰其齋;又見白狗大如驢,從城而出,不知所在。
 譽甚惡之,俄而城陷。



 史臣曰:蕭綜、蕭正德並悖逆猖狂,自致夷滅,宜矣。太清之寇,蕭紀據庸、蜀之資,遂不勤王赴難,申臣子之節;及賊景誅剪,方始起兵,師出無名,成其釁禍。嗚呼!身當管、蔡之罰,蓋自貽哉。



\end{pinyinscope}