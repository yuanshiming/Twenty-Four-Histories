\article{卷第四十六列傳第四十 胡僧祐 徐文盛 杜掞兄岸 弟幼安 兄子龕 陰子春}

\begin{pinyinscope}

 胡僧祐,字願果,南陽冠軍人。少勇決,有武幹。仕魏至銀青光祿大夫,以大通二年歸國,頻上封事,高祖器之,拜假節、超武將軍、文德主帥,使戍項城。城陷,復沒於魏。中大通元年,陳慶之送魏北海王元顥入洛陽,僧祐又得還國,除南天水、天門二郡太守,有善政。性好讀書,不解
 緝綴。然每在公宴,必彊賦詩,文辭鄙俚,多被嘲謔,僧祐怡然自若,謂己實工,矜伐愈甚。



 晚事世祖,為鎮西錄事參軍。侯景亂,西沮蠻反,世祖令僧祐討之,使盡誅其渠帥,僧祐諫,忤旨下獄。大寶二年,侯景寇荊陜,圍王僧辯於巴陵,世祖乃引僧祐於獄,拜為假節、武猛將軍,封新市縣侯,令赴援。僧祐將發,謂其子曰:「汝可開兩門,一門擬朱,一門擬白。吉則由朱門,凶則由白門。吾不捷不歸也。」世祖聞而壯之。至楊浦,景遣其將任約率銳卒五千,據白塔,遙以待之。僧祐由別路西上,約謂畏己而退,急追之,及於南安芊口,呼僧祐曰:「吳兒,何為不早降?走何
 處去。」僧祐不與之言,潛引卻,至赤砂亭,會陸法和至,乃與并軍擊約,大破之,擒約送於江陵。侯景聞之遂遁。世祖以僧祐為侍中、領軍將軍,徵還荊州。承聖二年,進為車騎將軍、開府儀同三司,餘悉如故。西魏寇至,以僧祐為都督城東諸軍事。魏軍四面起攻,百道齊舉,僧祐親當矢石,晝夜督戰,獎勵將士,明於賞罰,眾皆感之,咸為致死,所向摧殄,賊莫敢前。俄而中流矢卒,時年六十三。世祖聞之,馳往臨哭。於是內外惶駭,城遂陷。



 徐文盛,字道茂,彭城人也。世仕魏為將。父慶之,天監初,率千餘人自北歸款,未至道卒。文盛仍統其眾,稍立功
 績,高祖甚優寵之。大同末,以為持節、督寧州刺史。先是,州在僻遠,所管群蠻不識教義,貪欲財賄,劫篡相尋,前後刺史莫能制。文盛推心撫慰,示以威德,夷獠感之,風俗遂改。



 太清二年,聞國難,乃召募得數萬人來赴。世祖嘉之,以為持節、散騎常侍、左衛將軍、督梁、南秦、沙、東益、巴、北巴六州諸軍事、仁威將軍、秦州刺史,授以東討之略。於是文盛督眾軍東下,至武昌,遇侯景將任約,遂與相持。久之,世祖又命護軍將軍尹悅、平東將軍杜幼安、巴州刺史王珣等會之,並受文盛節度。擊任約於貝磯,約大敗,退保西陽。文盛進據蘆洲,又與相持。侯景聞之,
 乃率大眾西上援約,至西陽。文盛不敢戰。諸將咸曰:「景水軍輕進,又甚飢疲,可因此擊之,必大捷。」文盛不許。文盛妻石氏,先在建鄴,至是,景載以還之。文盛深德景,遂密通信使,都無戰心,眾咸憤怨。杜幼安、守簉等乃率所領獨進,與景戰,大破之,獲其舟艦以歸。會景密遣騎從間道襲陷郢州,軍中兇懼,遂大潰。文盛奔還荊州,世祖仍以為城北面都督。又聚贓污甚多,世祖大怒,下令責之,數其十罪,除其官爵。文盛既失兵權,私懷怨望,世祖聞之,乃以下獄。時任約被擒,與文盛同禁。文盛謂約曰:「汝何不早降,令我至此。」約曰:「門外不見卿馬跡,使我何
 遽得降?」文盛無以答,遂死獄中。



 杜掞,京兆杜陵人也。其先自北歸南,居於雍州之襄陽,子孫因家焉。祖靈啟,齊給事中。父懷寶,少有志節,常邀際會。高祖義師東下,隨南平王偉留鎮襄陽。天監中,稍立功績,官至驍猛將軍、梁州刺史。大同初,魏梁州刺史元羅舉州內附,懷寶復進督華州。值秦州所部武興氐王楊紹反,懷寶擊破之。五年,卒於鎮。



 掞即懷寶第七子也。幼有志氣,居鄉時以膽勇稱。釋褐廬江驃騎府中兵參軍。世祖臨荊州,仍參幕府,後為新興太守。太清二年,隨岳陽王來襲荊州,世祖以與之有舊,密邀之。掞乃與
 兄岸、弟幼安、兄子龕等夜歸于世祖,世祖以為持節、信威將軍、武州刺史。俄遷宣毅將軍,領鎮蠻護軍、武陵內史,枝江縣侯,邑千戶。令隨王僧辯東討侯景。至巴陵,會景來攻,數十日不剋而遁。加侍中、左衛將軍,進爵為公,增邑五百戶。仍隨僧辯追景至石頭,與賊相持橫嶺。及戰,景親率精銳,左右衝突,掞從嶺後橫截之,景乃大敗,東奔晉陵,掞入據城。景平,加散騎常侍、持節、督江州諸軍事、江州刺史,增邑千戶。



 是月,齊將郭元建攻秦州刺史嚴超遠於秦郡,王僧辯令掞赴援。陳霸先亦自歐陽來會,與元建大戰於士林,霸先令彊弩射,元建眾卻。掞
 因縱兵擊,大破之,斬首萬餘級,生擒千餘人,元建收餘眾而遁。時世祖執王琳於江陵,其長史陸納等遂於長沙反,世祖征掞與王僧辯討之。承聖二年,及納等戰於車輪,大敗,陷其二壘,納等走保長沙,掞等圍之。後納等降,掞又與王僧辯西討武陵王於硤口,至即破平之。於是旋鎮,遘疾卒。詔曰:「掞,京兆舊姓,元凱苗裔。家傳學業,世載忠貞。自驅傳江渚,政號廉能。推轂淺原,實聞清靜。奄致殞喪,惻愴于懷。可贈車騎將軍,加鼓吹一部。謚曰武。」



 掞兄弟九人,兄嵩、岑、旂、岌、嶷、巘、岸及弟幼安,並知名當世。



 岸,字公衡。少有武幹,好從橫之術。太清中,與掞同歸世祖,世祖以為持節、平北將軍、北梁州刺史,封江陵縣侯,邑一千戶。岸因請襲襄陽,世祖許之。岸乃晝夜兼行,先往攻其城,不剋。岳陽至,遂走依其兄巘於南陽,巘時為南陽太守。岳陽尋遣攻陷其城,岸及巘俱遇害。



 幼安性至孝,寬厚,雄勇過人。太清中,與兄掞同歸世祖,世祖以為雲麾將軍、西荊州刺史,封華容縣侯,邑一千戶。令與平南將軍王僧辯討河東王譽於長沙,平之。又命率精甲一萬,助左衛將軍徐文盛東討侯景。至貝磯,遇景將任約來逆,遂與戰,大敗之。斬其儀同叱羅子通、
 湘州刺史趙威方等,傳首江陵。乃進軍大舉,因與景相持。別攻武昌,拔之。景渡蘆洲上流以壓文盛等,幼安與眾軍攻之,景大敗,盡獲其舟艦。會景密遣襲陷郢州,執刺史方諸等以歸,人情大駭,徐文盛由漢口遁歸,眾軍大敗,幼安遂降于景。景殺之,以其多反覆故也。



 龕,掞第二兄岑之子。少驍勇,善用兵,亦太清中與諸父同歸世祖,世祖以為持節、忠武將軍、鄖州刺史,中廬縣侯,邑一千戶。與叔幼安俱隨王僧辯討河東王,平之。又隨僧辯下,繼徐文盛軍至巴陵,聞侯景襲陷郢州,西上將至,乃與僧辯等守巴陵以待之。景至,圍之數旬,不剋而
 遁。遷太府卿、安北將軍、督定州諸軍事、定州刺史,加通直散騎常侍,增邑五百戶。仍隨僧辯追景至江夏,圍其城。景將宋子仙棄城遁,龕追至楊浦,生擒之。大寶三年,眾軍至姑孰,景將侯子鑒逆戰,龕與陳霸先、王琳等率精銳擊之,大敗子鑒,遂至于石頭。景親率其黨會戰,龕與眾軍奮擊,大破景,景遂東奔。論功為最,授平東將軍、東揚州刺史,益封一千戶。



 承聖二年,又與王僧辯討陸納等於長沙,降之。又徵武陵王於西陵,亦平之。後江陵陷,齊納貞陽侯以紹梁嗣,以龕為震州刺史、吳興太守。又除鎮南將軍、都督南豫州諸軍事、南豫州刺史、溧陽
 縣侯,給鼓吹一部。又加散騎常侍、鎮東大將軍。會陳霸先襲陷京師,執王僧辯殺之。龕,僧辯之婿也,為吳興太守。以霸先既非貴素,兵又猥雜,在軍府日,都不以霸先經心;及為本郡,每以法繩其宗門,無所縱捨,霸先銜之切齒。及僧辯敗,龕乃據吳興以距之,遣軍副杜泰攻陳蒨於長城,反為蒨所敗。霸先乃遣將周文育討龕,龕令從弟北叟出距,又為文育所破,走義興,霸先親率眾圍之。會齊將柳達摩等襲京師,霸先恐,遂還與齊人連和。龕聞齊兵還,乃降,遂遇害。



 陰子春,字幼文,武威姑臧人也。晉義熙末,曾祖襲,隨宋
 高祖南遷,至南平,因家焉。父智伯,與高祖鄰居,少相友善,嘗入高祖臥內,見有異光成五色,因握高祖手曰:「公後必大貴,非人臣也。天下方亂,安蒼生者,其在君乎!」高祖曰:「幸勿多言。」於是情好轉密,高祖每有求索,如外府焉。及高祖踐阼,官至梁、秦二州刺史。



 子春,天監初,起家宣惠將軍、西陽太守。普通中,累遷至明威將軍、南梁州刺史;又遷信威將軍、都督梁、秦、華三州諸軍事、梁、秦二州刺史。太清二年,討峽中叛蠻,平之。徵為左衛將軍,又遷侍中。屬侯景亂,世祖令子春隨領軍將軍王僧辯攻邵陵王於郢州,平之。又與左衛將軍徐文盛東討侯景,
 至貝磯,與景遇,子春力戰,恆冠諸軍,頻敗景。值郢州陷沒,軍遂退敗。大寶二年,卒於江陵。



 孫顥,少知名。釋褐奉朝請,歷尚書金部郎。後入周。撰《瓊林》二十卷。



 史臣曰:胡僧祐勇干有聞,搴旗破敵者數矣;及捐軀殉節,殞身王事,雖古之忠烈,何以加焉。徐文盛始立功績,不能終其成名,為不義也。杜掞識機變之理,知向背之宜,加以身屢典軍,頻殄寇逆,勛庸顯著,卒為中興功臣。義哉!



\end{pinyinscope}