\article{卷第四十五列傳第三十九 王僧辯}

\begin{pinyinscope}

 王
 僧辯,字君才,右衛將軍神念之子也。以天監中隨父來奔。起家為湘東王國左常侍。王為丹陽尹,轉府行參軍。王出守會稽,兼中兵參軍事。王為荊州,仍除中兵,在限內。時武寧郡反,王命僧辯討平之。遷貞威將軍、武寧太守。尋遷振遠將軍、廣平太守。秩滿,還為王府中錄事,參軍如故。王被徵為護軍,僧辯兼府司馬。王為江州,仍
 除雲騎將軍司馬,守湓城。俄監安陸郡,無幾而還。尋為新蔡太守,猶帶司馬,將軍如故。王除荊州,為貞毅將軍府諮議參軍事,賜食千人,代柳仲禮為竟陵太守,改號雄信將軍。屬侯景反,王命僧辯假節,總督舟師一萬,兼糧饋赴援。纔至京都,宮城陷沒,天子蒙塵。僧辯與柳仲禮兄弟及趙伯超等,先屈膝於景,然後入朝。景悉收其軍實,而厚加綏撫。未幾,遣僧辯歸于竟陵,於是倍道兼行,西就世祖。世祖承制,以僧辯為領軍將軍。



 及荊、湘疑貳,軍師失律,世祖又命僧辯及鮑泉統軍討之,分給兵糧,剋日就道。時僧辨以竟陵部下猶未盡來,意欲待集,
 然後上頓。謂鮑泉曰:「我與君俱受命南討,而軍容若此,計將安之?」泉曰:「既稟廟算,驅率驍勇,事等沃雪,何所多慮。」僧辯曰:「不然。君之所言故是,文士之常談耳。河東少有武幹,兵刃又彊,新破軍師,養銳待敵,自非精兵一萬,不足以制之。我竟陵甲士,數經行陣,已遣召之,不久當及。雖期日有限,猶可重申,欲與卿共入言之,望相佐也。」泉曰:「成敗之舉,繫此一行,遲速之宜,終當仰聽。」世祖性嚴忌,微聞其言,以為遷延不肯去,稍已含怒。及僧辯將入,謂泉曰:「我先發言,君可見係。」泉又許之。及見世祖,世祖迎問曰:「卿已辦乎?何日當發?」僧辯具對,如向所言。世
 祖大怒,按劍厲聲曰:「卿憚行邪!」因起入內。泉震怖失色,竟不敢言。須臾,遣左右數十人收僧辯。既至,謂曰:「卿拒命不行,是欲同賊,今唯有死耳。」僧辯對曰:「僧辯食祿既深,憂責實重,今日就戮,豈敢懷恨。但恨不見老母。」世祖因斫之,中其左髀,流血至地。僧辯悶絕,久之方蘇。即送付廷尉,并收其子姪,並皆繫之。會岳陽王軍襲江陵,人情搔擾,未知其備。世祖遣左右往獄,問計於僧辯,僧辯具陳方略,登即赦為城內都督。俄而岳陽奔退,而鮑泉力不能剋長沙,世祖乃命僧辯代之。數泉以十罪,遣舍人羅重歡領齋仗三百人,與僧辯俱發。既至,遣通泉云:「
 羅舍人被令,送王竟陵來。」泉甚愕然,顧左右曰:「得王竟陵助我經略,賊不足平。」俄而重歡齎令書先入,僧辯從齋仗繼進,泉方拂席,坐而待之。僧辯既入,背泉而坐,曰:「鮑郎,卿有罪,令旨使我鏁卿,勿以故意見待。」因語重歡出令,泉即下地,鏁于床側。僧辯仍部分將帥,并力攻圍,遂平湘土。



 還復領軍將軍。侯景浮江西寇,軍次夏首。僧辯為大都督,率巴州刺史淳于量、定州刺史杜龕、宜州刺史王琳、郴州刺史裴之橫等,俱赴西陽。軍次巴陵,聞郢州已沒,僧辯因據巴陵城。世祖乃命羅州刺史徐嗣徽、武州刺史杜掞並會僧辯于巴陵。景既陷郢城,兵眾
 益廣,徒黨甚銳,將進寇荊州。乃使偽儀同丁和統兵五千守江夏,大將宋子仙前驅一萬造巴陵,景悉凶徒水步繼進。於是緣江戍邏,望風請服,賊拓邏至于隱磯。僧辯悉上江渚米糧,並沉公私船於水。及賊前鋒次江口,僧辯乃分命眾軍,乘城固守,偃旗臥鼓,安若無人。翌日,賊眾濟江,輕騎至城下,問:「城內是誰?」答曰:「是王領軍。」賊曰:「語王領軍,事勢如此,何不早降?」僧辯使人答曰:「大軍但向荊州,此城自當非礙。僧辯百口在人掌握,豈得便降。」賊騎既去,俄爾又來,曰:「我王已至,王領軍何為不出與王相見邪?」僧辯不答。頃之,又執王珣等至于城下,珣
 為書誘說城內。景帥船艦並集北寺,又分入港中,登岸治道,廣設氈屋,耀軍城東隴上,芟除草芿,開八道向城,遣五千兔頭肉薄苦攻。城內同時鼓噪,矢石雨下,殺賊既多,賊乃引退。世祖又命平北將軍胡僧祐率兵下援僧辯。是日,賊復攻巴陵,水步十處,鳴鼓吹脣,肉薄斫上。城上放木擲火爨昚石,殺傷甚多。午後賊退,乃更起長柵繞城,大列舸艦,以樓船攻水城西南角;又遣人渡洲岸,引羊柯推蝦蟆車填緌,引障車臨城,二日方止。賊又於艦上豎木桔禋,聚茅置火,以燒水柵,風勢不利,自焚而退。既頻戰挫衄,賊帥任約又為陸法和所擒,景乃燒
 營夜遁,旋軍夏首。世祖策勳行賞,以僧辯為征東將軍、開府儀同三司、江州策史,封長寧縣公。



 於是世祖命僧辯即率巴陵諸軍,沿流討景。師次郢城,步攻魯山。魯山城主支化仁,景之騎將也,率其黨力戰,眾軍大破之,化仁乃降。僧辯仍督諸軍渡江攻郢,即入羅城。宋子仙蟻聚金城拒守,攻之未剋。子仙使其黨時靈護率眾三千,開門出戰,僧辯又大破之,生擒靈護,斬首千級。子仙眾退據倉門,帶江阻險,眾軍攻之,頻戰不剋。景既聞魯山已沒,郢鎮復失羅城,乃率餘眾倍道歸建業。子仙等困蹙,計無所之,乞輸郢城,身還就景。僧辯偽許之,命給船
 百艘,以老其意。子仙謂為信然,浮舟將發,僧辯命杜龕率精勇千人,攀堞而上,同時鼓噪,掩至倉門。水軍主宋遙率樓船,暗江四面雲合;子仙行戰行走,至于白楊浦,乃大破之,生擒子仙送江陵。即率諸軍進師九水。賊偽儀同范希榮、盧暉略尚據湓城,及僧辯軍至,希榮等因挾江州刺史臨城公棄城奔走。世祖加僧辯侍中、尚書令、征東大將軍,給鼓吹一部。仍令僧辯且頓江州,須眾軍齊集,得時更進。



 頃之,世祖命江州眾軍悉同大舉,僧辯乃表皇帝凶問,告於江陵。仍率大將百餘人,連名勸世祖即位;將欲進軍,又重奉表。雖未見從,並蒙優答。事
 見本紀。



 僧辯於是發自江州,直指建業,乃先命南兗州刺史侯瑱率銳卒輕舸,襲南陵、鵲頭等戍,至即剋之。先是,陳霸先率眾五萬,出自南江,前軍五千,行至湓口。霸先倜儻多謀策,名蓋僧辯,僧辯畏之。既至湓口,與僧辯會于白茅洲,登壇盟誓。霸先為其文曰:「賊臣侯景,凶羯小胡,逆天無狀,構造姦惡;違背我恩義,破掠我國家,毒害我生民,移毀我社廟。我高祖武皇帝靈聖聰明,光宅天下,劬勞兆庶,亭育萬民,如我考妣,五十所載。哀景以窮見歸,全景將戮之首,置景要害之地,崇景非次之榮。我高祖於景何薄?我百姓於景何怨?而景長戟彊弩,陵
 蹙朝廷,鋸牙郊甸,殘食含靈。刳肝斫趾,不曈其快;曝骨焚屍,不謂為酷。高祖菲食卑宮,春秋九十,屈志凝威,憤終賊手。大行皇帝溫嚴恭默,丕守鴻名,於景何有,復加忍毒。皇枝涘抱已上,緦功以還,窮刀極俎,既屠且會。豈有率土之濱,謂為王臣,食人之禾,飲人之水,忍聞此痛,而不悼心?況臣僧辯、臣霸先等,荷稱國籓湘東王臣繹泣血銜哀之寄,摩頂至足之恩,世受先朝之德,身當將帥之任;而不能瀝膽抽腸,共誅姦逆,雪天地之痛,報君父之仇,則不可以稟靈含識,戴天履地!今日相國至孝
 玄感,靈武斯發,已破賊徒,獲其元帥,止餘景身,尚在京邑。臣僧辯與臣霸先協和將帥,同心共契,必誅凶豎,尊奉相國,嗣膺鴻業,以主郊祭。前途若有一功,獲一賞,臣僧辯等不推己讓物,先身帥眾,則天地宗廟百神之靈,共誅共責。臣僧辯、臣霸先同心共事,不相欺負,若有違戾,明神殛之。」於是升壇歃血,共讀盟文,皆淚下霑襟,辭色慷慨。



 及王師次于南洲,賊帥侯子鑒等率步騎萬餘人於岸挑戰,又以千艘並載士,兩邊悉八十棹,棹手皆越人,去來趣襲,捷過風電。僧辯乃麾細船,皆令退縮,悉使大艦夾泊兩岸。賊謂水軍欲退,爭出趨之,眾軍乃棹大艦,截其歸路,鼓噪大呼,合戰中江,賊悉赴水。僧
 辯即督諸軍沿流而下,進軍于石頭之斗城,作連營以逼賊。賊乃橫嶺上築五城拒守,侯景自出,與王師大戰於石頭城北。霸先謂僧辯曰:「醜虜遊魂,貫盈已稔,逋誅送死,欲為一決。我眾賊寡,且分其勢。」即遣彊弩二千張,攻賊西面兩城,仍使結陣以當賊。僧辯在後麾軍而進,復大破之。盧暉略聞景戰敗,以石頭城降,僧辯引軍入據之。景之退也,北走朱方,於是景散兵走告僧辯,僧辯令眾將入據臺城。其夜,軍人採梠失火,燒太極殿及東西堂等。時軍人鹵掠京邑,剝剔士庶,民為其執縛者,衵衣不免。盡驅逼居民以求購贖,自石頭至於東城,緣淮
 號叫之聲,震響京邑,於是百姓失望。



 僧辯命侯瑱、裴之橫率精甲五千,東入討景。僧辯收賊黨王偉等二十餘人,送于江陵。偽行臺趙伯超自吳松江降於侯瑱,瑱時送至僧辯。僧辯謂伯超曰:「趙公,卿荷國重恩,遂復同逆。今日之事,將欲何如?」因命送江陵。伯超既出,僧辯顧坐客曰:「朝廷昔唯知有趙伯超耳,豈識王僧辯?社稷既傾,為我所復;人之興廢,亦復何常。」賓客皆前稱歎功德。僧辯瞿然,乃謬答曰:「此乃聖上之威德,群帥之用命。老夫雖濫居戎首,何力之有焉?」於是逆寇悉平,京都剋定。世祖即帝位,以僧辯功,進授鎮衛將軍、司徒,加班劍二十
 人,改封永寧郡公,食邑五千戶,侍中、尚書令、鼓吹並如故。



 是後湘州賊陸納等攻破衡州刺史丁道貴於淥口,盡收其軍實;李洪雅又自零陵率眾出空靈灘,稱助討納。朝廷未達其心,深以為慮,乃遣中書舍人羅重歡征僧辯上就驃騎將軍宜豊侯循南征。僧辯因督杜掞等眾軍,發于建業,師次巴陵。詔僧辯為都督東上諸軍事,霸先為都督西上諸軍事。先時霸先讓都督於僧辯,僧辯不受,故世祖分為東西都督,而俱南討焉。時納等下據車輪,夾岸為城,前斷水勢,士卒驍猛,皆百戰之餘。僧辯憚之,不與輕進,於是稍作連城以逼賊。賊見不敢交
 鋒,並懷懈怠。僧辯因其無備,命諸軍水步攻之,親執旗鼓,以誡進止。於是諸軍競出,大戰於車輪,與驃騎循并力苦攻,陷其二城。賊大敗,步走歸保長沙,驅逼居民,入城拒守。僧辯追躡,乃命築壘圍之,悉令諸軍廣建圍柵,僧辯出坐壟上而自臨視。賊望,識僧辯,知不設備,賊黨吳藏、李賢明等乃率銳卒千人,開門掩出,蒙楯直進,徑趨僧辯。時杜掞、杜龕並侍左右,帶甲衛者止百餘人,因下遣人與賊交戰。李賢明乘鎧馬,從者十騎,大呼衝突,僧辯尚據胡床,不為之動。於是指揮勇敢,遂獲賢明,因即斬之。賊乃退歸城內。初,陸納阻兵內逆,以王琳為辭,
 云「朝廷若放王琳,納等自當降伏」。於時眾軍並進,未之許也。而武陵王擁眾上流,內外駭懼,世祖乃遣琳和解之。至是,湘州平。僧辯旋于江陵,因被詔會眾軍西討,督舟師二萬,輿駕出天居寺餞行。俄而武陵敗績,僧辯自枝江班師于江陵,旋鎮建業。



 是月,居少時,復回江陵。齊主高洋遣郭元建率眾二萬,大列舟艦於合肥,將謀襲建業,又遣其大將邢景遠、步六汗薩、東方老等率眾繼之。時陳霸先鎮建康,既聞此事,馳報江陵。世祖即詔僧辯次于姑孰,即留鎮焉。先命豫州刺史侯瑱率精甲三千人築壘於東關,以拒北寇;征吳郡太守張彪、吳興太守
 裴之橫會瑱於關;因與北軍戰,大敗之,僧辯率眾軍振旅于建業。承聖三年二月甲辰,詔曰:「贊俊遂賢,稱于秦典;自上安下,聞之漢制。所以仰協台曜,俯佐弘圖。使持節、侍中、司徒、尚書令、都督揚、南徐、東揚三州諸軍事、鎮衛將軍、揚州刺史、永寧郡開國公僧辯,器宇凝深,風格詳遠,行為士則,言表身文,學貫九流,武該七略。頃歲征討,自西徂東;師不疲勞,民無怨讟;王業艱難,實兼夷險。宜其燮此中台,膺茲上將;寄之經野,匡我朝猷。加太尉、車騎大將軍,餘悉如故。」



 頃之,丁母太夫人憂,世祖遣侍中謁者監護喪事,策謚曰貞敬太夫人。夫人姓魏氏。神
 念以天監初董率徒眾據東關,退保合肥漅湖西,因娶以為室,生僧辯。性甚安和,善於綏接,家門內外,莫不懷之。初,僧辯下獄,夫人流淚徒行,將入謝罪,世祖不與相見。時貞惠世子有寵於世祖,軍國大事多關領焉。夫人詣閣,自陳無訓,涕泗嗚咽,眾並憐之。及僧辯免出,夫人深相責勵,辭色俱嚴,云:「人之事君,惟須忠烈,非但保祐當世,亦乃慶流子孫。」及僧辯剋復舊京,功蓋天下,夫人恒自謙損,不以富貴驕物。朝野咸共稱之,謂為明哲婦人也。及既薨殞,甚見愍悼。且以僧辯勛業隆重,故喪禮加焉。靈柩將歸建康,又遣謁者至舟渚弔祭。命尚書左
 僕射王裒為其文曰:「維爾世基武子,族懋陽元,金相比映,玉德齊溫。既稱女則,兼循婦言。書圖鏡覽,辭章討論。教貽俎豆,訓及平原。楚發將兵,孟軻成德。盡忠資敬,自家刑國。顯允其儀,惟民之則。反命師旅,既修我戎;補茲袞職,奄有龜、蒙。母由子貴,亶爾斯崇;嘉命允集,寵章所隆。居高能降,處貴思沖;慶資善始,榮兼令終。崦嵫既夕,蒹葭早秋;奔駟難返,衝濤詎留。背龍門而西顧,過夏首而東浮;越三宮之遐岳,經三江之派流。鬱鬱增嶺,浮雲蔽虧;滔滔江、漢,逝者如斯。銘旌故旐,宇毀遺碑。即虛舟而設奠,想徂魂之有知。嗚呼哀哉!」



 其年十月,西魏相宇
 文黑泰遣兵及岳陽王眾合五萬,將襲江陵。世祖遣主書李膺征僧辯於建業,為大都督、荊州刺史。別敕僧辯云:「黑泰背盟,忽便舉斧。國家猛將,多在下流;荊陜之眾,悉非勁勇。公宜率貔虎,星言就路,倍道兼行,赴倒懸也。」僧辯因命豫州刺史侯瑱等為前軍,兗州刺史杜僧明等為後軍。處分既畢,乃謂膺云:「泰兵驍猛,難與爭銳,眾軍若集,吾便直指漢江,截其後路。凡千里饋糧,尚有飢色,況賊越數千里者乎?此孫臏剋龐涓時也。」俄而京城陷沒,宮車晏駕。及敬帝初即梁主位,僧辯預樹立之功,承制進驃騎大將軍、中書監、都督中外諸軍事、錄尚書,
 與陳霸先參謀討伐。



 時齊主高洋又欲納貞陽侯淵明以為梁嗣,因與僧辯書曰:「梁國不造,禍難相仍,侯景傾蕩建業,武陵彎弓巴、漢。卿志格玄穹,精貫白日,戮力齊心,芟夷逆醜。凡在有情,莫不嗟尚;況我鄰國,緝事言前。而西寇承間,復相掩襲。梁主不能固守江陵,殞身宗祐。王師未及,便已降敗;士民小大,皆畢寇虜。乃眷南顧,憤歎盈懷。卿臣子之情,念當鯁裂。如聞權立枝子,號令江陰,年甫十餘,極為沖藐;梁釁未已,負荷諒難。祭則衛君,政由甯氏;乾弱枝彊,終古所忌。朕以天下為家,大道濟物。以梁國淪滅,有懷舊好,存亡拯墜,義在今辰,扶危嗣
 事,非長伊德。彼貞陽侯,梁武猶子,長沙之胤,以年以望,堪保金陵,故置為梁主,納於彼國。便詔上黨王渙總攝群將,扶送江表,雷動風馳,助掃冤逆。清河王岳,前救荊城,軍度安陸,既不相及,憤惋良深。恐及西寇乘流,復躡江左。今轉次漢口,與陸居士相會。卿宜協我良規,厲彼群帥,部分舟艫,迎接今王,鳩勒勁勇,并心一力。西羌烏合,本非勍寇,直是湘東怯弱,致此淪胥。今者之師,何往不剋,善建良圖,副朕所望也。」



 貞陽承齊遣送,將屆壽陽。貞陽前後頻與僧辯書,論還國繼統之意,僧辯不納。及貞陽、高渙至于東關,散騎常侍裴之橫率眾拒戰,敗績,
 僧辯因遂謀納貞陽,仍定君臣之禮。啟曰:「自秦兵寇陜,臣便營赴援,纔及下船,荊城陷沒,即遣劉周入國,具表丹誠,左右勳豪,初並同契。周既多時不還,人情疑阻;比冊降中使,復遣諸處詢謀,物論參差,未甚決定。始得侯瑱信,示西寇權景宣書,令以真跡上呈。觀視將帥,恣欲同泰,若一朝仰違大國,臣不辭灰粉,悲梁祚永絕中興。伏願陛下便事濟江,仰藉皇齊之威,憑陛下至聖之略,樹君以長,雪報可期,社稷再輝,死且非吝。請押別使曹沖馳表齊都,續啟事以聞,伏遲拜奉在促。」貞陽答曰:「姜皓至,枉示具公忠義之懷。家國喪亂,于今積年。三后蒙
 塵,四海騰沸。天命元輔,匡救本朝。弘濟艱難,建武宗祏。至於丘園板築,尚想來儀;公室皇枝,豈不虛遲。聞孤還國,理會高懷,但近再命行人,或不宣具。公既詢謀卿士,訪逮籓維,沿溯往來,理淹旬月,使乎屆止,殊副所期。便是再立我蕭宗,重興我梁國。億兆黎庶,咸蒙此恩;社稷宗祧,曾不相愧。近軍次東關,頻遣信裴之橫處,示其可否。答對驕凶,殊駭聞矚。上黨王陳兵見衛,欲敘安危,無識之徒,忽然逆戰。前旌未舉,即自披猖,驚悼之情,彌以傷惻。上黨王深自矜嗟,不傳首級,更蒙封樹,飾棺厚殯,務從優禮。齊朝大德,信感神民。方仰藉皇威,敬憑元宰,
 討逆賊於咸陽,誅叛子於雲夢,同心協力,克定邦家。覽所示權景宣書,上流諸將,本有忠略,棄親向仇,庶當不爾,防奸定亂,終在於公。今且頓東關,更待來信,未知水陸何處見迎。夫建國立君,布在方策,入盟出質,有自來矣。若公之忠節,上感蒼旻;群帥同謀,必匪攜貳。則齊師反璟,義不陵江,如致爽言,誓以無克。韜旗側席,遲復行人。曹沖奉表齊都,即押送也。渭橋之下,惟遲敘言;汜水之陽,預有號懼。」僧辯又重啟曰:「員外常侍姜皓還,奉敕伏具動止。大齊仁義之風,曲被鄰國,恤災救難,申此大猷。皇家枝戚,莫不榮荷;江東冠冕,俱知憑賴。今歃不忘
 信,信實由衷,謹遣臣第七息顯,顯所生劉并弟子世珍,往彼充質;仍遣左民尚書周弘正至歷陽奉迎。艫舳浮江,俟一龍之渡;清宮丹陛,候六傳之入。萬國傾心,同榮晉文之反;三善克宣,方流宋昌之議。國祚既隆,社稷有奉。則群臣竭節,報厚施于大齊;戮力展愚,效忠誠於陛下。今遣吏部尚書王通奉啟以聞。」僧辯因求以敬帝為皇太子。貞陽又答曰:「王尚書通至,復枉示,知欲遣賢弟世珍以表誠質,具悉憂國之懷。復以庭中玉樹,掌內明珠,無累胸懷,志在匡救,豈非劬勞我社稷,弘濟我邦家?慚歎之懷,用忘興寢。晉安王東京貽厥之重,西都繼體
 之賢,嗣守皇家,寧非民望。但世道喪亂,宜立長君,以其蒙孽,難可承業。成、昭之德,自古希儔;沖、質之危,何代無此。孤身當否運,志不圖生。忽荷不世之恩,仍致非常之舉。自惟虛薄,兢懼已深。若建承華,本歸皇胄;心口相誓,惟擬晉安。如或虛言,神明所殛。覽今所示,深遂本懷。戢慰之情,無寄言象。但公憂勞之重,既稟齊恩;忠義之情,復及梁貳。華夷兆庶,豈不懷風?宗廟明靈,豈不相感?正爾迴璟,仍向歷陽。所期質累,便望來彼。眾軍不渡,已著盟書。斯則大齊聖主之恩規,上黨英王之然諾,得原失信,終不為也。惟遲相見,使在不賒。鄉國非遙,觸目號咽。」
 僧辯使送質于鄴。貞陽求渡衛士三千,僧辯慮其為變,止受散卒千人而已,并遣龍舟法駕往迎。貞陽濟江之日,僧辯擁楫中流,不敢就岸。後乃同會于江寧浦。



 貞陽既踐偽位,仍授僧辯大司馬,領太子太傅、揚州牧,餘悉如故。陳霸先時為司空、南徐州刺史,惡其翻覆,與諸將議,因自京口舉兵十萬,水陸俱至,襲于建康。於是水軍到,僧辯常處于石頭城,是日正視事,軍人已踰城北而入,南門又馳白有兵來。僧辯與其子頠遽走出閣,左右心腹尚數十人。眾軍悉至,僧辯計無所出,乃據南門樓乞命拜請。霸先因命縱火焚之,方共頠下就執。霸先曰:「
 我有何辜,公欲與齊師賜討?」又曰:「何意全無防備?」僧辯曰:「委公北門,何謂無備。」爾夜斬之。



 長子凱,承聖初歷官至侍中。初,僧辯平建業,遣霸先守京口,都無備防。凱屢以為言,僧辯不聽,竟及於禍。西魏寇江陵,世祖遣督城內諸軍事。荊城陷,凱隨王琳入齊,為竟陵郡守。齊遣琳鎮壽春,將圖江左。及陳平淮南,執琳殺之。凱聞琳死,乃出郡城南,登高塚上號哭,一慟而絕。



 凱弟頒,少有志節,恒隨從世祖。及荊城陷覆,沒于西魏。



 史臣曰:自侯景寇逆,世祖據有上游,以全楚之兵委僧辯將率之任。及剋平禍亂,功亦著焉,在乎策勳,當上台
 之賞。敬帝以高祖貽厥之重,世祖繼體之尊,洎渚宮淪覆,理膺寶祚。僧辯位當將相,義存伊、霍,乃受脅齊師,傍立支庶。茍欲行夫忠義,何忠義之遠矣?樹國之道既虧,謀身之計不足,自致殲滅,悲矣!



\end{pinyinscope}