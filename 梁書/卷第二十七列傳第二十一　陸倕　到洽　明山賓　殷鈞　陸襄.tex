\article{卷第二十七列傳第二十一 陸倕 到洽 明山賓 殷鈞 陸襄}

\begin{pinyinscope}

 陸倕,字佐公,吳郡吳人也。晉太尉玩六世孫。祖子真,宋東陽太守。父慧曉,齊太常卿。倕少勤學,善屬文。於宅內起兩間茅屋,杜絕往來,晝夜讀書,如此者數載。所讀一遍,必誦於口。嘗借人《漢書》,失《五行志》四卷,乃暗寫還之,略無遺脫。幼為外祖張岱所異,岱常謂諸子曰:「此兒汝家
 之陽元也。」年十七,舉本州秀才。刺史竟陵王子良開西邸延英俊,倕亦預焉。辟議曹從事參軍、廬陵王法曹行參軍。天監初,為右軍安成王外兵參軍,轉主簿。



 倕與樂安任昉友善,為《感知己賦》以贈昉,昉因此名以報之曰:「信偉人之世篤,本侯服於陸鄉。緬風流與道素,襲袞衣與繡裳。還伊人而世載,並三駿而龍光。過龍津而一息,望鳳條而曾翔。彼白玉之雖潔,此幽蘭之信芳。思在物而取譬,非斗斛之能量。匹聳峙於東岳,比凝厲於秋霜。不一飯以妄過,每三錢以投渭。匪蒙袂之敢嗟,豈溝壑之能衣。既蘊藉其有餘,又淡然而無味。得意同乎卷懷,
 違方似乎仗氣。類平叔而靡雕,似子雲之不朴。冠眾善而貽操,綜群言而名學。折高、戴於后臺,異鄒、顏乎董幄。採三《詩》於河間,訪九師於淮曲。術兼口傳之書,藝廣鏗鏘之樂。時坐睡而梁懸,裁枝梧而錐握。既文過而意深,又理勝而辭縟。咨餘生之荏苒,迫歲暮而傷情。測徂陰於堂下,聽鳴鐘於洛城。唯忘年之陸子,定一遇於班荊。餘獲田蘇之價,爾得海上之名。信落魄而無產,終長對於短生。飢虛表於徐步,逃責顯於疾行。子比我于叔則,又方餘於耀卿。心照情交,流言靡惑。萬類暗求,千里懸得。言象可廢,蹄筌自默。居非連棟,行則同車。冬日不足,
 夏日靡餘。肴核非餌,絲竹豈娛。我未捨駕,子已回輿。中飯相顧,悵然動色。邦壤既殊,離會莫測。存異山陽之居,沒非要離之側。似膠投漆中,離婁豈能識。」其為士友所重如此。



 遷驃騎臨川王東曹掾。是時禮樂制度,多所創革,高祖雅愛倕才,乃敕撰《新漏刻銘》,其文甚美。遷太子中舍人,管東宮書記。又詔為《石闕銘記》。奏之。敕曰:「太子中舍人陸倕所製《石闕銘》,辭義典雅,足為佳作。昔虞丘辨物,邯鄲獻賦,賞以金帛,前史美談,可賜絹三十匹。」遷太子庶子、國子博士,母憂去職。服闋,為中書侍郎,給事黃門侍郎,揚州別駕從事史,以疾陳解。遷鴻臚卿,入為
 吏部郎,參選事。出為雲麾晉安王長史、尋陽太守、行江州府州事。以公事免,左遷中書侍郎、司徒司馬、太子中庶子、廷尉卿。又為中庶子,加給事中、揚州大中正。復除國子博士、中庶子、中正並如故。守太常卿,中正如故。普通七年,卒,年五十七。文集二十卷,行於世。



 第四子纘,早慧,十歲通經,為童子奉車郎,卒。



 到洽,字茂+水公,彭城武原人也。宋驃騎將軍彥之曾孫。祖仲度,驃騎江夏王從事中郎。父坦,齊中書郎。洽年十八,為南徐州迎西曹行事。洽少知名,清警有才學士行。謝朓文章盛於一時,見洽深相賞好,日引與談論。每謂洽
 曰:「君非直名人,乃亦兼資文武。」朓後為吏部,洽去職,朓欲薦之,洽睹世方亂,深相拒絕。除晉安王國左常侍,不就。遂築室巖阿,幽居者積歲。樂安任昉有知人之鑒,與洽兄沼、溉並善。嘗訪洽於田舍,見之歎曰:「此子日下無雙。」遂申拜親之禮。



 天監初,沼、溉俱蒙擢用,洽尤見知賞,從弟沆亦相與齊名。高祖問待詔丘遲曰:「到洽何如沆、溉?」遲對曰:「正清過於沆,文章不減溉;加以清言,殆將難及。」即召為太子舍人。御華光殿,詔洽及沆、蕭琛、任昉侍宴,賦二十韻詩,以洽辭為工,賜絹二十匹。高祖謂昉曰:「諸到可謂才子。」昉對曰:「臣常竊議,宋得其武,梁得其文。」



 二年,遷司徒主簿,直待詔省,敕使抄甲部書。五年,遷尚書殿中郎。洽兄弟群從,遞居此職,時人榮之。七年,遷太子中舍人,與庶子陸倕對掌東宮管記。俄為侍讀,侍讀省仍置學士二人,洽復充其選。九年,遷國子博士,奉敕撰《太學碑》。十二年,出為臨川內史,在郡稱職。十四年,入為太子家令,遷給事黃門侍郎,兼國子博士。十六年,行太子中庶子。普通元年,以本官領博士。頃之,入為尚書吏部郎,請託一無所行。俄遷員外散騎常侍,復領博士,母憂去職。五年,復為太子中庶子,領步兵校尉,未拜,仍遷給事黃門侍郎,領尚書左丞。準繩不避貴戚,尚書省
 賄賂莫敢通。時鑾輿欲親戎,軍國容禮,多自洽出。六年,遷御史中丞,彈糾無所顧望,號為勁直,當時肅清。以公事左降,猶居職。舊制,中丞不得入尚書下舍,洽兄溉為左民尚書,洽引服親不應有礙,刺省詳決。左丞蕭子雲議許入溉省,亦以其兄弟素篤,不能相別也。七年,出為貞威將軍、雲麾長史、尋陽太守。大通元年,卒於郡,時年五十一。贈侍中。謚曰理子。昭明太子與晉安王綱令曰:「明北兗、到長史遂相係凋落,傷怛悲惋,不能已已。去歲陸太常殂歿,今茲二賢長謝。陸生資忠履貞,冰清玉潔,文該四始,學遍九流,高情勝氣,貞然直上。明公儒學稽
 古,淳厚篤誠,立身行道,始終如一,儻值夫子,必升孔堂。到子風神開爽,文義可觀,當官蒞事,介然無私。皆海內之俊乂,東序之秘寶。此之嗟惜,更復何論。但遊處周旋,並淹歲序,造膝忠規,豈可勝說,幸免祇悔,實二三子之力也。談對如昨,音言在耳,零落相仍,皆成異物,每一念至,何時可言。天下之寶,理當惻愴。近張新安又致故,其人文筆弘雅,亦足嗟惜,隨弟府朝,東西日久,尤當傷懷也。比人物零落,特可傷惋,屬有今信,乃復及之。」



 洽文集行于世。子伯淮、仲舉。



 明山賓,字孝若,平原鬲人也。父僧紹,隱居不仕,宋末
 國子博士徵,不就。山賓七歲能言名理,十三博通經傳,居喪盡禮。服闋,州辟從事史。起家奉朝請。兄仲璋嬰痼疾,家道屢空,山賓乃行干祿。齊始安王蕭遙光引為撫軍行參軍,後為廣陽令,頃之去官。義師至,高祖引為相府田曹參軍。梁臺建,為尚書駕部郎,遷治書侍御史,右軍記室參軍,掌治吉禮。時初置《五經》博士,山賓首膺其選。遷北中郎諮議參軍,侍皇太子讀。累遷中書侍郎、國子博士、太子率更令、中庶子,博士如故。天監十五年,出為持節、督緣淮諸軍事、征遠將軍、北兗州刺史。普通二年,徵為太子右衛率,加給事中,遷御史中丞。以公事左遷
 黃門侍郎、司農卿。四年,遷散騎常侍,領青、冀二州大中正。東宮新置學士,又以山賓居之,俄以本官兼國子祭酒。



 初,山賓在州,所部平陸縣不稔,啟出倉米以贍人。後刺史檢州曹,失簿書,以山賓為耗闕,有司追責,籍其宅入官,山賓默不自理,更市地造宅。昭明太子聞築室不就,有令曰:「明祭酒雖出撫大籓,擁旄推轂,珥金拖紫,而恒事屢空。聞構宇未成,今送薄助。」并貽詩曰:「平仲古稱奇,夷吾昔檀美。令則挺伊賢,東秦固多士。築室非道傍,置宅歸仁里。庚桑方有係,原生今易擬。必來三徑人,將招《五經》士。」



 山賓性篤實,家中嘗乏用,貨所乘牛。既售受
 錢,乃謂買主曰:「此牛經患漏蹄,治差已久,恐後脫發,無容不相語。」買主遽追取錢。處士阮孝緒聞之,歎曰:「此言足使還淳反朴,激薄停澆矣。」



 五年,又為國子博士,常侍、中正如故。其年以本官假節,權攝北兗州事。大通元年,卒,時年八十五。詔贈侍中、信威將軍。謚曰質子。昭明太子為舉哀,賻錢十萬,布百匹,并使舍人王顒監護喪事。又與前司徒左長史殷芸令曰:「北兗信至,明常侍遂至殞逝,聞之傷怛。此賢儒術該通,志用稽古,溫厚淳和,倫雅弘篤。授經以來,迄今二紀。若其上交不諂,造膝忠規,非顯外迹,得之胸懷者,蓋亦積矣。攝官連率,行當言歸,
 不謂長往,眇成疇日。追憶談緒,皆為悲端,往矣如何!昔經聯事,理當酸愴也。」



 山賓累居學官,甚有訓導之益,然性頗疏通,接於諸生,多所狎比,人皆愛之。所著《吉禮儀注》二百二十四卷,《禮儀》二十卷,《孝經喪禮服義》十五卷。



 子震,字興道,亦傳父業。歷官太學博士,太子舍人,尚書祠部郎,餘姚令。



 殷鈞,字季和,陳郡長平人也。晉太常融八世孫。父睿,有才辯,知名齊世,歷官司徒從事中郎。睿妻王奐女。奐為雍州刺史、鎮北將軍,乃言於朝,以睿為鎮北長史、河南太守。奐誅,睿并見害。鈞時年九歲,以孝聞。及長,恬靜簡
 交遊,好學有思理。善隸書,為當時楷法,南鄉范雲、樂安任昉,並稱賞之。高祖與睿少舊故,以女妻鈞,即永興公主也。



 天監初,拜駙馬都尉,起家秘書郎、太子舍人、司徒主簿、秘書丞。鈞在職,啟校定秘閣四部書,更為目錄。又受詔料檢西省法書古迹,別為品目。遷驃騎從事中郎,中書郎、太子家令、掌東宮書記。頃之,遷給事黃門侍郎、中庶子、尚書吏部郎、司徒左長史,侍中。東宮置學士,復以鈞為之。公事免。復為中庶子,領國子博士、左驍騎將軍,博士如故。出為明威將軍、臨川內史。



 鈞體羸多疾,閉閣臥治,而百姓化其德,劫盜皆奔出境。嘗禽劫帥,不加
 考掠,但和言誚責。劫帥稽顙乞改過,鈞便命遣之,後遂為善人。郡舊多山瘧,更暑必動,自鈞在任,郡境無復瘧疾。母憂去職,居喪過禮,昭明太子憂之,手書誡喻曰:「知比諸德,哀頓為過,又所進殆無一溢,甚以酸耿。迥然一身,宗奠是寄,毀而滅性,聖教所不許。宜微自遣割,俯存禮制,穀粥果蔬,少加勉彊。憂懷既深,指故有及,并令繆道臻口具。」鈞答曰:「奉賜手令,并繆道臻宣旨,伏讀感咽,肝心塗地。小人無情,動不及禮,但稟生霡劣,假推年歲,罪戾所鐘,復加橫疾。頃者綿微,守盡晷漏,目亂玄黃,心迷哀樂,惟救危苦,未能以遠理自制。姜桂之滋,實聞前
 典,不避粱肉,復忝今慈,臣亦何人,降此憂愍。謹當循復聖言,思自補續,如脫申延,實由亭造。」服闋,遷五兵尚書,猶以頓瘵經時,不堪拜受,乃更授散騎常侍、領步兵校尉,侍東宮。尋改領中庶子。昭明太子薨,官屬罷,又領右游擊,除國子祭酒,常侍如故。中大通四年,卒,時年四十九。謚曰貞子。二子:構,渥。



 陸襄,字師卿,吳郡吳人也。父閑,齊始安王遙光揚州治中。永元末,遙光據東府作亂,或勸閑去之。閑曰:「吾為人吏,何所逃死。」臺軍攻陷城,閑見執,將刑,第二子絳求代死,不獲,遂以身蔽刃,刑者俱害之。襄痛父兄之酷,喪過
 於禮,服釋後猶若居憂。



 天監三年,都官尚書范岫表薦襄,起家擢拜著作佐郎,除永寧令。秩滿,累遷司空臨川王法曹,外兵,輕車廬陵王記室參軍。昭明太子聞襄業行,啟高祖引與遊處,除太子洗馬,遷中舍人,並掌管記。出為揚州治中,襄父終此官,固辭職,高祖不許,聽與府司馬換廨居之。昭明太子敬耆老,襄母年將八十,與蕭琛、傅昭、陸杲每月常遣存問,加賜珍羞衣服。襄母嘗卒患心痛,醫方須三升粟漿,是時冬月,日又逼暮,求索無所。忽有老人詣門貨漿,量如方劑,始欲酬直,無何失之,時以襄孝感所致也。累遷國子博士,太子家令,復掌管
 記,母憂去職。襄年已五十,毀頓過禮,太子憂之,日遣使誡喻。服闋,除太子中庶子,復掌管記。中大通三年,昭明太子薨,官屬罷,妃蔡氏別居金華宮,以襄為中散大夫、領步兵校尉、金華宮家令、知金華宮事。



 七年,出為鄱陽內史。先是,郡民鮮于琛服食修道法,嘗入山採藥,拾得五色幡眊,又於地中得石璽,竊怪之。琛先與妻別室,望琛所處,常有異氣,益以為神。大同元年,遂結其門徒,殺廣晉令王筠,號上願元年,署置官屬。其黨轉相誑惑,有眾萬餘人。將出攻郡,襄先已帥民吏修城隍,為備禦,及賊至,連戰破之,生獲琛,餘眾逃散。時鄰郡豫章、安成等
 守宰,案治黨與,因求賄貨,皆不得其實,或有善人盡室離禍,惟襄郡部枉直無濫。民作歌曰:「鮮于平後善惡分,民無枉死,賴有陸君。」又有彭李二家,先因忿爭,遂相誣告,襄引入內室,不加責誚,但和言解喻之,二人感恩,深自咎悔。乃為設酒食,令其盡歡,酒罷,同載而還,因相親厚。民又歌曰:「陸君政,無怨家,鬥既罷,仇共車。」在政六年,郡中大治,民李睍等四百二十人詣闕拜表,陳襄德化,求於郡立碑,降敕許之。又表乞留襄,襄固求還,徵為吏部郎,遷秘書監,領揚州大中正。太清元年,遷度支尚書,中正如故。



 二年,侯景舉兵圍宮城,以襄直侍中省。三年
 三月,城陷,襄逃還吳。賊尋寇東境,沒吳郡。景將宋子仙進攻錢塘,會海鹽人陸黯舉義,有眾數千人,夜出襲郡,殺偽太守蘇單于,推襄行郡事。時淮南太守文成侯蕭寧逃賊入吳,襄遣迎寧為盟主,遣黯及兄子映公帥眾拒子仙。子仙聞兵起,乃退還,與黯等戰於松江,黯敗走,吳下軍聞之,亦各奔散。襄匿于墓下,一夜憂憤卒,時年七十。



 襄弱冠遭家禍,終身蔬食布衣,不聽音樂,口不言殺害五十許年。侯景平,世祖追贈侍中、雲麾將軍。以建義功,追封餘干縣侯,邑五百戶。



 陳吏部尚書姚察曰:陸倕博涉文理,到洽匪躬貞勁,明
 山賓儒雅篤實,殷鈞靜素恬和,陸襄淳深孝性,雖任遇有異,皆列於名臣矣。



\end{pinyinscope}