\article{夏侯尚傳}
\begin{pinyinscope}
 
 
 夏侯尚字伯仁,淵從子也。文帝與之親友。
 
 
\gezhu{魏書曰:尚有籌畫智略,文帝器之,與為布衣之交。}
 太祖定兾州,尚為軍司馬,將騎從征伐,後為五官將文學。魏國初建,遷黃門侍郎。代郡胡叛,遣鄢陵侯彰征討之,以尚參彰軍事,定代地,還。太祖崩於洛陽,尚持節,奉梓宮還鄴。并錄前功,封平陵亭侯,拜散騎常侍,遷中領軍。文帝踐阼,更封平陵鄉侯,遷征南將軍,領荊州刺史,假節都督南方諸軍事。尚奏:「劉備別軍在上庸,山道險難,彼不我虞,若以奇兵潛行,出其不意,則獨克之勢也。」遂勒諸軍擊破上庸,平三郡九縣,遷征南大將軍。孫權雖稱藩,尚益脩攻討之備,權後果有貳心。黃初三年,車駕幸宛,使尚率諸軍與曹真共圍江陵。權將諸葛瑾與尚軍對江,瑾渡入江中渚,而分水軍於江中。尚夜多持油舩,將步騎萬餘人,於下流潛渡,攻瑾諸軍,夾江燒其舟舩,水陸並攻,破之。城未拔,會大疫,詔勑尚引諸軍還。益封六百戶,并前千九百戶,假鉞,進為牧。荊州殘荒,外接蠻夷,而與吳阻漢水為境,舊民多居江南。尚自上庸通道,西行七百餘里,山民蠻夷多服從者,五六年間,降附數千家。五年,徙封昌陵鄉侯。
 
 
尚有愛妾嬖幸,寵奪適室;適室,曹氏女也,故文帝遣人絞殺之。尚悲感,發病怳惚,旣葬埋妾,不勝思見,復出視之。文帝聞而恚之曰:「杜襲之輕薄尚,良有以也。」然以舊臣,恩寵不衰。六年,尚疾篤,還京都,帝數臨幸,執手涕泣。尚薨,謚曰悼侯。
 \gezhu{魏書載詔曰:「尚自少侍從,盡誠竭節,雖云異姓,其猶骨肉,是以入為腹心,出當爪牙。智略深敏,謀謨過人,不幸早殞,命也柰何!贈征南大將軍、昌陵侯印綬。」}
 子玄嗣。又分尚戶三百,賜尚弟子奉爵關內侯。
 
 
玄字太初。少知名,弱冠為散騎黃門侍郎。嘗進見,與皇后弟毛曾並坐,玄恥之,不恱形之於色。明帝恨之,左遷為羽林監。正始初,曹爽輔政。玄,爽之姑子也。累遷散騎常侍、中護軍。
 \gezhu{世語曰:玄世名知人,為中護軍,拔用武官,參戟牙門,無非俊傑,多牧州典郡。立法垂教,于今皆為後式。}
 
 
 
 
 太傅司馬宣王問以時事,玄議以為:「夫官才用人,國之柄也,故銓衡專於臺閣,上之分也,孝行存乎閭巷,優劣任之鄉人,下之叙也。夫欲清教審選,在明其分叙,不使相涉而已。何者?上過其分,則恐所由之不本,而干勢馳騖之路開;下踰其叙,則恐天爵之外通,而機權之門多矣。夫天爵下通,是庶人議柄也;機權多門,是紛亂之原也。自州郡中正品度官才之來,有年載矣,緬緬紛紛,未聞整齊,豈非分叙參錯各失其要之所由哉!若令中正但考行倫輩,倫輩當行均,斯可官矣。何者?夫孝行著於家門,豈不忠恪於在官乎?仁恕稱於九族,豈不達於為政乎?義斷行於鄉黨,豈不堪於事任乎?三者之類,取於中正,雖不處其官名,斯任官可知矣。行有大小,比有高下,則所任之流,亦渙然明別矣。奚必使中正干銓衡之機於下,而執機柄者有所委仗於上,上下交侵,以生紛錯哉?且臺閣臨下,考功校否,衆職之屬,各有官長,旦夕相考,莫究於此;閭閻之議,以意裁處,而使匠宰失位,衆人驅駭,欲風俗清靜,其可得乎?天臺縣遠,衆所絕意。所得至者,更在側近,孰不脩飾以要所求?所求有路,則脩己家門者,已不如自達於鄉黨矣。自達鄉黨者,已不如自求之於州邦矣。苟開之有路,而患其飾真離本,雖復嚴責中正,督以刑罰,猶無益也。豈若使各帥其分,官長則各以其屬能否獻之臺閣,臺閣則據官長能否之第,參以鄉閭德行之次,擬其倫比,勿使偏頗。中正則唯考其行迹,別其高下,審定輩類,勿使升降。臺閣總之,如其所簡,或有參錯,則其責負自在有司。官長所第,中正輩擬,比隨次率而用之,如其不稱,責負在外。然則內外相參,得失有所,互相形檢,孰能相飾?斯則人心定而事理得,庶可以靜風俗而審官才矣。」
 
 
 
 
 又以為:「古之建官,所以濟育羣生,統理民物也,故為之君長以司牧之。司牧之主,欲一而專,一則官任定而上下安,專則職業脩而事不煩。夫事簡業脩,上下相安而不治者,未之有也。先王建萬國,雖其詳未可得而究,然分彊畫界,各守土境,則非重累羈絆之體也。下考殷、周五等之叙,徒有小大貴賤之差,亦無君官臣民而有二統互相牽制者也。夫官統不一,則職業不脩;職業不脩,則事何得而簡?事之不簡,則民何得而靜?民之不靜,則邪惡並興,而姦偽滋長矣。先王達其如此,故專其職司而一其統業。始自秦世,不師聖道,私以御職,姦以待下;懼宰官之不脩,立監牧以董之,畏督監之容曲,設司察以糾之;宰牧相累,監察相司,人懷異心,上下殊務。漢承其緒,莫能匡改。魏室之隆,日不暇及,五等之典,雖難卒復,可粗立儀準以一治制。今之長吏,皆君吏民,橫重以郡守,累以刺史。若郡所攝,唯在大較,則與州同,無為再重。宜省郡守,但任刺史;刺史職存,則監察不廢,郡吏萬數還親農業,以省煩費,豐財殖穀,一也。大縣之才,皆堪郡守,是非之訟,每生意異,順從則安,直己則爭。夫和羹之美,在於合異,上下之益,在能相濟,順從乃安,此琴瑟一聲也,蕩而除之,則官省事簡,二也。又幹郡之吏,職監諸縣,營護黨親,鄉邑舊故,如有不副,而因公掣頓,民之困弊,咎生於此,若皆并合,則亂原自塞,三也。今承衰弊,民人彫落,賢才鮮少,任事者寡,郡縣良吏往往非一,郡受縣成,其劇在下,而吏之上選,郡當先足,此為親民之吏,專得厎下,吏者民命,而常頑鄙,今如并之,吏多選清良者造職,大化宣流,民物獲寧,四也。制使萬戶之縣名之郡守,五千以上名之都尉,千戶以下令長如故,自長以上考課遷用,轉以能升,所牧亦增,此進才効功之叙也,若經制一定,則官才有次,治功齊明,五也。若省郡守,縣皆徑達,事不擁隔,官無留滯,三代之風雖未可必,簡一之化庶幾可致,便民省費在於此矣。」
 
 
 
 
 又以為:「文質之更用,猶四時之迭興也,王者體天理物,必因弊而濟通之,時彌質則文之以禮,時泰侈則救之以質。今承百王之末,秦漢餘流,世俗彌文,宜大改之以易民望。今科制自公、列侯以下,位從大將軍以上,皆得服綾錦、羅綺、紈素、金銀飾鏤之物,自是以下,雜綵之服通于賤人,雖上下等級,各示有差,然朝臣之制已得侔至尊矣,玄黃之采已得通於下矣。欲使市不鬻華麗之色,商不通難得之貨,工不作雕刻之物,不可得也。是故宜大理其本,準度古法,文質之宜,取其中則,以為禮度。車輿服章,皆從質樸,禁除末俗華麗之事,使幹朝之家,有位之室,不復有錦綺之飾,無兼采之服、纖巧之物,自上以下至於樸素之差,示有等級而已,勿使過一二之覺。若夫功德之賜,上恩所特加,皆表之有司,然後服用之。夫上之化下,猶風之靡草。樸素之教興於本朝,則彌侈之心自消於下矣。」
 
 
 
 
 宣王報書曰:「審官擇人,除重官,改服制,皆大善。禮鄉閭本行,朝廷考事,大指如所示。而中間一相承習,卒不能改。秦時無刺史,但有郡守長吏。漢家雖有刺史,奉六條而已,故刺史稱傳車,其吏言從事,居無常治,吏不成臣,其後轉更為官司耳。昔賈誼亦患服制,漢文雖身服弋綈,猶不能使上下如意。恐此三事,當待賢能然後了耳。」
 
 
 
 
 玄又書曰:「漢文雖身衣弋綈,而不革正法度,內外有僭擬之服,寵臣受無限之賜,由是觀之,似指立在身之名,非篤齊治制之意也。今公侯命世作宰,追蹤上古,將隆至治,抑末正本,若制定於上,則化行於衆矣。夫當宜改之時,留殷勤之心,令發之日,下之應也猶響尋聲耳,猶垂謙謙,曰『待賢能』,此伊周不正殷姬之典也。竊未喻焉。」
 
 
頃之,為征西將軍,假節都督雍、涼州諸軍事。
 \gezhu{魏略曰:玄旣遷,司馬景王代為護軍。護軍總統諸將,任主武官選舉,前後當此官者,不能止貨賂。故蔣濟為護軍時,有謠言「欲求牙門,當得千匹;百人督,五百匹」。宣王與濟善,聞以問濟,濟無以解之,因戲曰:「洛中市買,一錢不足則不行。」遂相對歡笑。玄代濟,故不能止絕人事。及景王之代玄,整頓法令,人莫犯者。}
 與曹爽共興駱谷之役,時人譏之。爽誅,徵玄為大鴻臚,數年徙太常。玄以爽抑絀,內不得意。中書令李豐雖宿為大將軍司馬景王所親待,然私心在玄,遂結皇后父光祿大夫張緝,謀欲以玄輔政。豐旣內握權柄,子尚公主,又與緝俱馮翊人,故緝信之。豐陰令弟兖州刺史翼求入朝,欲使將兵入,并力起。會翼求朝,不聽。嘉平六年二月,當拜貴人,豐等欲因御臨軒,諸門有陛兵,誅大將軍,以玄代之,以緝為驃騎將軍。豐密語黃門監蘇鑠、永寧署令樂敦、宂從僕射劉賢等曰:「卿諸人居內多有不法,大將軍嚴毅,累以為言,張當可以為誡。」鑠等皆許以從命。
 \gezhu{魏書曰:玄素貴,以爽故廢黜,居常怏怏不得意。中書令李豐與玄及后父光祿大夫張緝陰謀為亂,緝與豐同郡,傾巧人也,以東莞太守召,為后家,亦不得意,故皆同謀。初,豐自以身處機密,息韜又以列侯給事中,尚齊長公主,有內外之重,心不自安。密謂韜曰:「玄旣為海內重人,加以當大任,年時方壯而永見廢,又親曹爽外弟,於大將軍有嫌。吾得玄書,深以為憂。緝有才用,棄兵馬大郡,還坐家巷。各不得志,欲使汝以密計告之。」緝嘗病創卧,豐遣韜省病,韜屏人語緝曰:「韜尚公主,父子在機近,大將軍秉事,常恐不見明信,太常亦懷深憂。君侯雖有后父之尊,安危未可知,皆與韜家同慮者也,韜父欲與君侯謀之。」緝默然良乆曰:「同舟之難,吾焉所逃?此大事,不捷即禍及宗族。」韜於是往報豐。密語黃門監蘇鑠等,蘇鑠等荅豐:「惟君侯計。」豐言曰:「今拜貴人,諸營兵皆屯門。陛下臨軒,因此便共迫脅,將羣寮人兵,就誅大將軍。卿等當共密白此意。」鑠等曰:「陛下儻不從人,柰何?」豐等曰:「事有權宜,臨時若不信聽,便當劫將去耳。郍得不從?」鑠等許諾。豐曰:「此族滅事,卿等密之。事成,卿等皆當封侯常侍也。」豐復密以告玄、緝。緝遣子邈與豐相結,同謀起事。世語曰:豐遣子韜以謀報玄,玄曰「宜詳之耳」,而不以告也。}
 大將軍微聞其謀,請豐相見,豐不知而往,即殺之。
 \gezhu{世語曰:大將軍聞豐謀,舍人王羕請以命請豐:「豐若無備,情屈勢迫,必來,若不來,羕一人足以制之;若知謀泄,以衆挾輪,長戟自衞,徑入雲龍門,挾天子登陵雲臺,臺上有三千人仗,鳴鼓會衆,如此,羕所不及也」。大將軍乃遣羕以車迎之。豐見劫迫,隨羕而至。魏氏春秋曰:大將軍責豐,豐知禍及,遂正色曰:「卿父子懷姦,將傾社稷,惜吾力劣,不能相禽滅耳!」大將軍怒,使勇士以刀環築豐腰,殺之。魏略曰:豐字安國,故衞尉李義子也。黃初中,以父任召隨軍。始為白衣時,年十七八,在鄴下名為清白,識別人物,海內翕然,莫不注意。後隨軍在許昌,聲稱日隆。其父不願其然,遂令閉門,勑使斷客。初,明帝在東宮,豐在文學中。及即尊位,得吳降人,問「江東聞中國名士為誰」?降人云:「聞有李安國者是。」時豐為黃門郎,明帝問左右安國所在,左右以豐對。帝曰:「豐名乃被於吳越邪?」後轉騎都尉、給事中。帝崩後,為永寧太僕,以名過其實,能用少也。正始中,遷侍中尚書僕射。豐在臺省,常多託疾,時臺制,疾滿百日當解祿,豐疾未滿數十日,輒暫起,已復卧,如是數歲。初,豐子韜以選尚公主,豐雖外辭之,內不甚憚也。豐弟翼及偉,仕數歲閒,並歷郡守。豐嘗於人中顯誡二弟,言當用榮位為。及司馬宣王乆病,偉為二千石,荒於酒,亂新平、扶風二郡,而豐不召,衆人以為恃寵。曹爽專政,豐依違二公間,無有適莫,故于時有謗書曰:「曹爽之勢熱如湯,太傅父子冷如漿,李豐兄弟如游光。」其意以為豐雖外示清淨,而內圖事,有似於游光也。及宣王奏誅爽,住車闕下,與豐相聞,豐怖,遽氣索,足委地不能起。至嘉平四年宣王終後,中書令缺,大將軍諮問朝臣:「誰可補者?」或指向豐。豐雖知此非顯選,而自以連婚國家,思附至尊,因伏不辭,遂奏用之。豐為中書二歲,帝比每獨召與語,不知所說。景王知其議己,請豐,豐不以實告,乃殺之。其事祕。豐前後仕歷二朝,不以家計為意,仰俸廩而已。韜雖尚公主,豐常約勑不得有所侵取,時得賜錢帛,輒以外施親族;及得賜宮人,多與子弟,而豐皆以與諸外甥。及死後,有司籍其家,家無餘積。魏氏春秋曰:夜送豐尸付廷尉,廷尉鍾毓不受,曰:「非法官所治也。」以其狀告,且勑之,乃受。帝怒,將問豐死意,太后懼,呼帝入,乃止。遣使收翼。世語曰:翼後妻,散騎常侍荀廙姊,謂翼曰:「中書事發,可及書未至赴吳,何為坐取死亡!左右可共同赴水火者誰?」翼思未荅,妻曰:「君在大州,不知可與同死生者,去亦不免。」翼曰:「二兒小,吾不去。今但從坐,身死,二兒必免。」果如翼言。翼子斌,楊駿外甥也。晉惠帝初,為河南尹,與駿俱死,見晉書。}
 事下有司,收玄、緝、鑠、敦、賢等送廷尉。
 \gezhu{世語曰:玄至廷尉,不肯下辭。廷尉鍾毓自臨治玄。玄正色責毓曰:「吾當何辭?卿為令史責人也,卿便為吾作。」毓以其名士,節高不可屈,而獄當竟,夜為作辭,令與事相附,流涕以示玄。玄視,頷之而已。毓弟會,年少於玄,玄不與交,是日於毓坐狎玄,玄不受。孫盛雜語曰:玄在囹圄,會因欲狎而友玄,玄正色曰:「鍾君何相偪如此也!」}
 廷尉鍾毓奏:「豐等謀迫脅至尊,擅誅冢宰,大逆無道,請論如法。」於是會公卿朝臣廷尉議,咸以為「豐等各受殊寵,典綜機密,緝承外戚椒房之尊,玄備世臣,並居列位,而苞藏禍心,構圖凶逆,交關閹豎,授以姦計,畏憚天威,不敢顯謀,乃欲要君脅上,肆其詐虐,謀誅良輔,擅相建立,將以傾覆京室,顛危社稷。毓所正皆如科律,報毓施行」。詔書:「齊長公主,先帝遺愛,匄其三子死命。」於是豐、玄、緝、敦、賢等皆夷三族,
 \gezhu{魏書曰:豐子韜,以尚主,賜死獄中。}
 其餘親屬徙樂浪郡。玄格量弘濟,臨斬東巿,顏色不變,舉動自若,時年四十六。
 \gezhu{魏略曰:玄自從西還,不交人事,不蓄華妍。魏氏春秋曰:初,夏侯霸將奔蜀,呼玄欲與之俱。玄曰:「吾豈苟存自客於寇虜乎?」遂還京師。太傅薨,許允謂玄曰:「無復憂矣。」玄歎曰:「士宗,卿何不見事乎?此人猶能以通家年少遇我,子元、子上不吾容也。」玄嘗著樂毅、張良及本無肉刑論,辭旨通遠,咸傳於世。玄之執也,衞將軍司馬文王流涕請之,大將軍曰:「卿忘會趙司空葬乎?」先是,司空趙儼薨,大將軍兄弟會葬,賔客以百數,玄時後至,衆賔客咸越席而迎,大將軍由是惡之。臣松之案:曹爽以正始五年伐蜀,時玄已為關中都督,至十年,爽誅滅後,方還洛耳。案少帝紀,司空趙儼以六年亡,玄則無由得會儼葬,若云玄入朝,紀、傳又無其事。斯近妄不實。}
 正元中,紹功臣世,封尚從孫本為昌陵亭侯,邑三百戶,以奉尚後。
 
 
初,中領軍高陽許允與豐、玄親善。先是有詐作尺一詔書,以玄為大將軍,允為太尉,共錄尚書事。有何人天未明乘馬以詔版付允門吏,曰「有詔」,因便馳走。允即投書燒之,不以開呈司馬景王。後豐等事覺,徙允為鎮北將軍,假節督河北諸軍事。未發,以放散官物,收付廷尉,徙樂浪,道死。
 \gezhu{魏略曰:允字士宗,世冠族。父據,仕歷典農校尉、郡守。允少與同郡崔贊俱發名於兾州,召入軍。明帝時為尚書選曹郎,與陳國袁侃對,同坐職事,皆收送獄,詔旨嚴切,當有死者,正直者為重。允謂侃曰:「卿,功臣之子,法應八議,不憂死也。」侃知其指,乃為受重。允刑竟復吏,出為郡守,稍遷為侍中尚書中領軍。允聞李豐等被收,欲往見大將軍,已出門,回遑不定,中道還取袴,豐等已收訖。大將軍聞允前遽,恠之曰:「我自收豐等,不知士大夫何為忩忩乎?」是時朝臣遽者多耳,而衆人咸以為意在允也。會鎮北將軍劉靜卒,朝廷以允代靜。已受節傳,出止外舍。大將軍與允書曰:「鎮北雖少事,而都典一方,念足下震華鼓,建朱節,歷本州,此所謂著繡晝行也。」允心甚恱,與臺中相聞,欲易其鼓吹旌旗。其兄子素頗聞衆人說允前見嫌意,戒允「但當趣耳,用是為邪」!允曰:「卿俗士不解,我以榮國耳,故求之。」帝以允當出,乃詔會羣臣,羣臣皆集,帝特引允以自近;允前為侍中,顧當與帝別,涕泣歔欷。會訖,罷出,詔促允令去。會有司奏允前擅以厨錢穀乞諸俳及其官屬,故遂收送廷尉,考問竟,減死徙邊。允以嘉平六年秋徙,妻子不得自隨,行道未到,以其年冬死。魏氏春秋曰:允為吏部郎,選郡守。明帝疑其所用非次,召入,將加罪。允妻阮氏跣出,謂曰:「明主可以理奪,難以情求。」允頷之而入。帝怒詰之,允對曰:「某郡太守雖限滿文書先至,年限在後,某守雖後,日限在前。」帝前取事視之,乃釋遣出。望其衣敗,曰:「清吏也。」賜之。允之出為鎮北也,喜謂其妻曰:「吾知免矣!」妻曰:「禍見於此,何免之有?」允善相印,將拜,以印不善,使更刻之,如此者三。允曰:「印雖始成而已被辱。」問送印者,果懷之而墜於厠。相印書曰:「相印法本出陳長文,長文以語韋仲將,印工楊利從仲將受法,以語許士宗。利以法術占吉凶,十可中八九。仲將問長文『從誰得法』?長文曰:『本出漢世,有相印、相笏經,又有鷹經、牛經、馬經。印工宗養以法語程申伯,是故有一十三家相法傳於世。』」允妻阮氏賢明而醜,允始見愕然,交禮畢,無復入意。妻遣婢覘之,云「有客姓桓」,妻曰:「是必桓範,將勸使入也。」旣而範果勸之。允入,須臾便起,妻捉裾留之。允顧謂婦曰:「婦有四德,卿有其幾?」婦曰:「新婦所乏唯容。士有百行,君有其幾?」許曰:「皆備。」婦曰:「士有百行,以德為首,君好色不好德,何謂皆備?」允有慙色,知其非凡,遂雅相親重。生二子,奇、猛,少有令問。允後為景王所誅,門生走入告其婦,婦正在機,神色不變,曰:「早知爾耳。」門生欲藏其子,婦曰:「無預諸兒事。」後移居墓所,景王遣鍾會看之,若才藝德能及父,當收。兒以語母,母荅:「汝等雖佳,才具不多,率胷懷與會語,便自無憂,不須極哀,會止便止。又可多少問朝事。」兒從之。會反命,具以狀對,卒免其禍,皆母之教也。雖會之識鑒,而輸賢婦之智也。果慶及後嗣,追封子孫而已。世語曰:允二子:奇字子泰,猛字子豹,並有治理才學。晉元康中,奇為司隷校尉,猛幽州刺史。傅暢晉諸公贊曰:猛禮樂儒雅,當時最優。奇子遐,字思祖,以清尚稱,位至侍中。猛子式,字儀祖,有才幹,至濮陽內史、平原太守。}
 
 
清河王經亦與允俱稱兾州名士。甘露中為尚書,坐高貴鄉公事誅。始經為郡守,經母謂經曰:「汝田家子,今仕至二千石,物太過不祥,可以止矣。」經不能從,歷二州刺史,司隷校尉,終以致敗。
 \gezhu{世語曰:經字彥偉,初為江夏太守。大將軍曹爽附絹二十匹令交市於吳,經不發書,棄官歸。母問歸狀,經以實對。母以經典兵馬而擅去,對送吏杖經五十,爽聞,不復罪。經為司隷校尉,辟河內向雄為都官從事,王業之出,不申經,竟以及難。經刑於東市,雄哭之,感動一市。刑及經母,雍州故吏皇甫晏以家財收葬焉。漢晉春秋曰:經被收,辭母。母顏色不變,笑而應曰:「人誰不死?往所以不止汝者,恐不得其所也。以此并命,何恨之有哉?」晉武帝太始元年詔曰:「故尚書王經,雖身陷法辟,然守志可嘉。門戶堙沒,意常愍之,其賜經孫郎中。」}
 允友人同郡崔贊,亦嘗以處世太盛戒允云。
 \gezhu{荀綽兾州記曰:贊子洪,字良伯,清恪有匪躬之志,為晉吏部尚書、大司農。}
 
 
 
 
 評曰:夏侯、曹氏,世為婚姻,故惇、淵、仁、洪、休、尚、真等並以親舊肺腑,貴重於時,左右勳業,咸有効勞。爽德薄位尊,沈溺盈溢,此固大易所著,道家所忌也。玄以規格局度,世稱其名,然與曹爽中外繾綣;榮位如斯,曾未聞匡弼其非,援致良才。舉茲以論,焉能免之乎!
 
 
\end{pinyinscope}