\article{鍾繇傳子名毓}
\begin{pinyinscope}
 
 
 鍾繇字元常,頴川長社人也。
 
 
\gezhu{先賢行狀曰:鍾皓字季明,溫良篤慎,博學詩律,教授門生千有餘人,為郡功曹。時太丘長陳寔為西門亭長,皓深獨敬異。寔少皓十七歲,常禮待與同分義。會辟公府,臨辭,太守問:「誰可代君?」皓曰:「明府欲必得其人,西門亭長可用。」寔曰:「鍾君似不察人為意,不知何獨識我?」皓為司徒掾,公出,道路泥濘,導從惡其相灑,去公車絕遠。公椎軾言:「司徒今日為獨行耳!」還府向閤,鈴下不扶,令揖掾屬,公奮手不顧。時舉府掾屬皆投劾出,皓為西曹掾,即開府門分布曉語已出者,曰:「臣下不能得自直於君,若司隷舉繩墨,以公失宰相之禮,又不勝任,諸君終身何所任邪?」掾屬以故皆止。都官果移西曹掾,問空府去意,皓召都官吏,以見掾屬名示之,乃止。前後九辟三府,遷南鄉、林慮長,不之官。時郡中先輩為海內所歸者,蒼梧太守定陵陳稚叔、故黎陽令潁陰荀淑及皓。少府李膺常宗此三人,曰:「荀君清識難尚,陳、鍾至德可師。」膺之姑為皓兄之妻,生子覲,與膺年齊,並有令名。覲又好學慕古,有退讓之行。為童幼時,膺祖太尉脩言:「覲似我家性,國有道不廢,國無道免於刑戮者也。」復以膺妹妻之。覲辟州宰,未甞屈就。膺謂覲曰:「孟軻以為人無好惡是非之心,非人也。弟於人何太無皂白邪!」覲甞以膺之言白皓,皓曰:「元禮,祖公在位,諸父並盛,韓公之甥,故得然耳。國武子好昭人過,以為怨本,今豈其時!保身全家,汝道是也。」覲早亡,膺雖荷功名,位至卿佐,而卒隕身世禍。皓年六十九,終於家。皓二子迪、敷,並以黨錮不仕。繇則迪之孫。}
 甞與族父瑜俱至洛陽,道遇相者,曰:「此童有貴相,然當厄於水,努力慎之!」行未十里,度橋,馬驚,墯水幾死。瑜以相者言中,益貴繇,而供給資費,使得專學。舉孝廉,
 \gezhu{謝承漢書曰:南陽陰脩為潁川太守,以旌賢擢俊為務,舉五官掾張仲方正,察功曹鍾繇、主簿荀彧、主記掾張禮、賊曹掾杜祐、孝廉荀攸、計吏郭圖為吏,以光國朝。}
 除尚書郎、陽陵令,以疾去。辟三府,為廷尉正、黃門侍郎。是時,漢帝在西京,李傕、郭汜等亂長安中,與關東斷絕。太祖領兖州牧,始遣使上書。
 \gezhu{世語曰:太祖遣使從事王必致命天子。}
 傕、汜等以為「關東欲自立天子,今曹操雖有使命,非其至實」,議留太祖使,拒絕其意。繇說傕、汜等曰:「方今英雄並起,各矯命專制,唯曹兖州乃心王室,而逆其忠款,非所以副將來之望也。」傕、汜等用繇言,厚加荅報,由是太祖使命遂得通。太祖旣數聽荀彧之稱繇,又聞其說傕、汜,益虛心。後傕脅天子,繇與尚書郎韓斌同策謀。天子得出長安,繇有力焉。拜御史中丞,遷侍中尚書僕射,并錄前功封東武亭侯。
 
 
時關中諸將馬騰、韓遂等,各擁彊兵相與爭。太祖方有事山東,以關右為憂。乃表繇以侍中守司隷校尉,持節督關中諸軍,委之以後事,特使不拘科制。繇至長安,移書騰、遂等,為陳禍福,騰、遂各遣子入侍。太祖在官渡,與袁紹相持,繇送馬二千餘匹給軍。太祖與繇書曰:「得所送馬,甚應其急。關右平定,朝廷無西顧之憂,足下之勳也。昔蕭何鎮守關中,足食成軍,亦適當爾。」其後匈奴單于作亂平陽,繇帥諸軍圍之,未拔;而袁尚所置河東太守郭援到河東,衆甚盛。諸將議欲釋之去,繇曰:「袁氏方彊,援之來,關中陰與之通,所以未悉叛者,顧吾威名故耳。若棄而去,示之以弱,所在之民,誰非寇讎?縱吾欲歸,其得至乎!此為未戰先自敗也。且援剛愎好勝,必易吾軍,若渡汾為營,及其未濟擊之,可大克也。」張旣說馬騰會擊援,騰遣子超將精兵逆之。援至,果輕渡汾,衆止之,不從。濟水未半,擊,大破之,
 \gezhu{司馬彪戰略曰:袁尚遣高幹、郭援將兵數萬人,與匈奴單于寇河東,遣使與馬騰、韓遂等連和,騰等陰許之。傅幹說騰曰:「古人有言『順道者昌,逆德者亡』。曹公奉天子誅暴亂,法明國治,上下用命,有義必賞,無義必罰,可謂順道矣。袁氏背王命,驅胡虜以陵中國,寬而多忌,仁而無斷,兵雖彊,實失天下心,可謂逆德矣。今將軍旣事有道,不盡其力,陰懷兩端,欲以坐觀成敗,吾恐成敗旣定,奉辭責罪,將軍先為誅首矣。」於是騰懼。幹曰:「智者轉禍為福。今曹公與袁氏相持,而高幹、郭援獨制河東,曹公雖有萬全之計,不能禁河東之不危也。將軍誠能引兵討援,內外擊之,其勢必舉。是將軍一舉,斷袁氏之臂,解一方之急,曹公必重德將軍。將軍功名,竹帛不能盡載也。唯將軍審所擇!」騰曰:「敬從教。」於是遣子超將精兵萬餘人,并將遂等兵,與繇會擊援等,大破之。}
 斬援,降單于。語在旣傳。其後河東衞固作亂,與張晟、張琰及高幹等並為寇,繇又率諸將討破之。
 \gezhu{魏略曰:詔徵河東太守王邑。邑以天下未定,心不願徵,而吏民亦戀邑,郡掾衞固及中郎將范先等各詣繇求乞邑。而詔已拜杜畿為太守,畿已入界。繇不聽先等,促邑交符。邑佩印綬,徑從河北詣許自歸。繇時治在洛陽,自以威禁失督司之法,乃上書自劾曰;「臣前上言故鎮北將軍領河東太守安陽亭侯王邑巧辟治官,犯突科條,事當推劾,檢實姦詐。被詔書當如所糾。以其歸罪,故加寬赦。又臣上言吏民大小,各懷顧望,謂邑當還,拒太守杜畿,今皆反悔,共迎畿之官。謹桉文書,臣以空虛,被蒙拔擢,入充近侍,兼典機衡,忝膺重任,總統偏方。旣無德政以惠民物,又無威刑以檢不恪,至使邑違犯詔書,郡掾衞固誑迫吏民,訟訴之言,交驛道路,漸失其禮,不虔王命。今雖反悔,醜聲流聞,咎皆由繇威刑不攝。臣又疾病,前後歷年,氣力日微,尸素重祿,曠廢職任,罪明法正。謹桉侍中守司隷校尉東武亭侯鍾繇,幸得蒙恩,以斗筲之才,仍見拔擢,顯從近密,銜命督使。明知詔書深疾長吏政教寬弱,檢下無刑,乆病淹滯,衆職荒頓,法令失張。邑雖違科,當必繩正法,旣舉文書,操彈失禮,至乃使邑遠詣闕廷。隳忝使命,挫傷爪牙。而固誑迫吏民,拒畿連月,今雖反悔,犯順失正,海內兇赫,罪一由繇威刑闇弱。又繇乆病,不任所職,非繇大臣當所宜為。繇輕慢憲度,不畏詔令,不與國同心,為臣不忠,無所畏忌,大為不敬。又不承用詔書,奉詔不謹。又聦明蔽塞,為下所欺,弱不勝任。數罪謹以劾,臣請法車徵詣廷尉治繇罪,大鴻臚削爵土。臣乆嬰篤疾,涉夏盛劇,命縣呼吸,不任部官。輙以文書付功曹從事馬適議,免冠徒跣,伏須罪誅。」詔不聽。}
 自天子西遷,洛陽人民單盡,繇徙關中民,又招納亡叛以充之,數年間民戶稍實。太祖征關中,得以為資,表繇為前軍師。
 
 
魏國初建,為大理,遷相國。文帝在東宮,賜繇五熟釜,為之銘曰:「於赫有魏,作漢藩輔。厥相惟鍾,寔幹心膂。靖恭夙夜,匪遑安處。百寮師師,楷茲度矩。」
 \gezhu{魏略曰:繇為相國,以五熟釜鼎範因太子鑄之,釜成,太子與繇書曰:「昔有黃三鼎,周之九寶,咸以一體使調一味,豈若斯釜五味時芳?蓋鼎之烹餁,以饗上帝,以養聖賢,昭德祈福,莫斯之美。故非大人,莫之能造;故非斯器,莫宜盛德。今之嘉釜,有逾茲美。夫周之尸臣,宋之考父,衞之孔悝,晉之魏顆,彼四臣者,並以功德勒名鍾鼎。今執事寅亮大魏,以隆聖化。堂堂之德,於斯為盛。誠太常之所宜銘,彝器之所宜勒。故作斯銘,勒之釜口,庶可贊揚洪美,垂之不朽。」臣松之桉漢書郊祀志,孝宣時,美陽得鼎,京兆尹張敞上議曰:「按鼎有刻書曰:『王命尸臣,宕此栒邑。尸,主事之臣也;栒音荀,幽地。賜尔鸞旂,黼黻琱戈。尸臣拜首,稽首曰,敢對揚天子,丕顯休命!』此殆周之所以襃賜大臣子孫,大臣子孫刻銘其先功,藏之于宮廟也。」考父銘見左氏傳,孔悝銘在禮記,事顯故不載。國語曰:「昔克潞之役,秦來圖敗晉功,魏顆以其身追秦師于輔氏,親止杜囬;其勒銘于景鍾,至于今不遺類,其子孫不可不興也。」太子所稱四銘者也。魏略曰:後太祖征漢中,太子在孟津,聞繇有玉玦,欲得之而難。公密使臨菑侯轉因人說之,繇即送之。太子與繇書曰:「夫玉以比德君子,見美詩人。晉之垂棘,魯之璵璠,宋之結綠,楚之和璞,價越萬金,貴重都城,有稱疇昔,流聲將來。是以垂棘出晉,虞、虢雙禽;和璧入秦,相如抗節。竊見玉書,稱美玉白若截肪,黑譬純漆,赤擬雞冠,黃侔蒸栗。側聞斯語,未覩厥狀。雖德非君子,義無詩人,高山景行,私所慕仰。然四寶邈焉以遠,秦、漢未聞有良匹。是以求之曠年,未遇厥真,私願不果,饑渴未副。近見南陽宗惠叔稱君侯昔有美玦,聞之驚喜,笑與抃俱。當自白書,恐傳言未審,是以令舍弟子建因荀仲茂轉言鄙旨。乃不忽遺,厚見周稱,鄴騎旣到,寶玦初至,捧跪發匣,爛然滿目。猥以矇鄙之姿,得觀希世之寶,不煩一介之使,不損連城之價,旣有秦昭章臺之觀,而無藺生詭奪之誑。嘉貺益腆,敢不欽承!」繇報書曰:「昔忝近任,并得賜玦。尚方耆老,頗識舊物。名其符采。必得處所。以為執事有珍此者,是以鄙之,用未奉貢。幸而紆意,實以恱懌。在昔和氏,殷勤忠篤,而繇待命,是懷愧恥。」}
 數年,坐西曹掾魏諷謀反,策罷就第。
 \gezhu{魏略曰:孫權稱臣,斬送關羽。太子書報繇,繇荅書曰:「臣同郡故司空荀爽言:『人當道情,愛我者一何可愛!憎我者一何可憎!』顧念孫權,了更侮媚。」太子又書曰:「得報,知喜南方。至於荀公之清談,孫權之侮媚,執書嗢噱,不能離手。若權復黠,當折以汝南許邵月旦之評。權優游二國,俯仰荀、許,亦已足矣。」}
 文帝即王位,復為大理。及踐阼,改為廷尉,進封崇高鄉侯。遷太尉,轉封平陽鄉侯。時司徒華歆、司空王朗,並先世名臣。文帝罷朝,謂左右曰:「此三公者,乃一代之偉人也,後世殆難繼矣!」
 \gezhu{陸氏異林曰:繇嘗數月不朝會,意性異常,或問其故,云:「常有好婦來,美麗非凡。」問者曰:「必是鬼物,可殺之。」婦人後往,不即前,止戶外。繇問何以,曰:「公有相殺意。」繇曰:「無此。」乃勤勤呼之,乃入。繇意恨,有不忍之心,然猶斫之傷髀。婦人即出,以新緜拭血竟路。明日使人尋跡之,至一大冢,木中有好婦人,形體如生人,著白練衫,丹繡裲襠,傷左髀,以裲襠中緜拭血。叔父清河太守說如此。清河,陸雲也。}
 明帝即位,進封定陵侯,增邑五百,并前千八百戶,遷太傅。繇有膝疾,拜起不便。時華歆亦以高年疾病,朝見皆使載輿車,虎賁舁上殿就坐。是後三公有疾,遂以為故事。
 
 
初,太祖下令,使平議死刑可宮割者。繇以為「古之肉刑,更歷聖人,宜復施行,以代死刑。」議者以為非恱民之道,遂寢。及文帝臨饗群臣,詔謂「大理欲復肉刑,此誠聖王之法。公卿當善共議。」議未定,會有軍事,復寢。太和中,繇上疏曰:「大魏受命,繼蹤虞、夏。孝文革法,不合古道。先帝聖德,固天所縱,墳典之業,一以貫之。是以繼世,仍發明詔,思復古刑,為一代法。連有軍事,遂未施行。陛下遠追二祖遺意,惜斬趾可以禁惡,恨入死之無辜,使明習律令,與群臣共議。出本當右趾而入大辟者,復行此刑。書云:『皇帝清問下民,鰥寡有辭于苗。』此言堯當除蚩尤、有苗之刑,先審問於下民之有辭者也。若今蔽獄之時,訊問三槐、九棘、群吏、萬民,使如孝景之令,其當棄巿,欲斬右趾者許之。其黥、劓、左趾、宮刑者,自如孝文,易以髠、笞。能有姦者,率年二十至四五十,雖斬其足,猶任生育。今天下人少於孝文之世,下計所全,歲三千人。張蒼除肉刑,所殺歲以萬計。臣欲復肉刑,歲生三千人。子貢問能濟民可謂仁乎?子曰:『何事於仁,必也聖乎,堯、舜其猶病諸!』又曰:『仁遠乎哉?我欲仁,斯仁至矣。』若誠行之,斯民永濟。」書奏,詔曰:「太傅學優才高,留心政事,又於刑理深遠。此大事,公卿羣寮善共平議。」司徒王朗議,以為「繇欲輕減大辟之條,以增益刖刑之數,此即起偃為豎,化屍為人矣。然臣之愚,猶有未合微異之意。夫五刑之屬,著在科律,自有減死一等之法,不死即為減。施行已乆,不待遠假斧鑿於彼肉刑,然後有罪次也。前世仁者,不忍肉刑之慘酷,是以廢而不用。不用已來,歷年數百。今復行之,恐所減之文未彰於萬民之目,而肉刑之問已宣於寇讎之耳,非所以來遠人也。今可桉繇所欲輕之死罪,使減死之髠、刖。嫌其輕者,可倍其居作之歲數。內有以生易死不訾之恩,外無以刖易釱駭耳之聲。」議者百餘人,與朗同者多。帝以吳、蜀未平,且寢。
 \gezhu{袁宏曰:夫民心樂全而不能常全,蓋利用之物懸於外,而嗜慾之情動於內也。於是有進取貪競之行,希求放肆之事。進取不已,不能充其嗜慾,則苟且儌倖之所生也;希求無饜,無以愜其慾,則姦偽忿怒之所興也。先王知其如此,而欲救其弊,或先德化以陶其心;其心不化,然後加以刑辟。書曰:「百姓不親,五品不遜。汝作司徒而敬敷五教。蠻夷猾夏,寇賊姦宄。汝作士,五刑有服。」然則德、刑之設,參而用之者也。三代相因,其義詳焉。周禮:「使墨者守門,劓者守關,宮者守內,刖者守囿。」此肉刑之制可得而論者也。荀卿亦云,殺人者死,傷人者刑,百王之所同,未有知其所由來者也。夫殺人者死,而相殺者不已,是大辟可以懲未殺,不能使天下無殺也。傷人者刑,而害物者不息,是黥、劓可以懼未刑,不能使天下無刑也。故將欲止之,莫若先以德化。夫罪過彰著,然後入于刑辟,是將殺人者不必死,欲傷人者不必刑。縱而弗化,則陷於刑辟。故刑之所制,在於不可移之地。禮教則不然,明其善惡,所以潛勸其情,消之於未殺也;示之恥辱,所以內愧其心,治之於未傷也。故過微而不至於著,罪薄而不及於刑。終入罪辟者,非教化之所得也,故雖殘一物之生,刑一人之體,是除天下之害,夫何傷哉!率斯道也,風化可以漸淳,刑罰可以漸少,其理然也。苟不能化其心,而專任刑罰,民失義方,動罹刑網,求世休和,焉可得哉?周之成、康,豈桉三千之文而致刑錯之美乎?蓋德化漸漬,致斯有由也。漢初懲酷刑之弊,務寬厚之論,公卿大夫,相與恥言人過。文帝登朝,加以玄默。張武受賂,賜金以愧其心;吳王不朝,崇禮以訓其失。是以吏民樂業,風流篤厚,斷獄四百,幾致刑錯,豈非德刑兼用已然之効哉?世之欲言刑罰之用,不先德教之益,失之遠矣。今大辟之罪,與古同制。免死已下,不過五歲,旣釋鉗鎖,復得齒于人倫。是以民無恥惡,數為姦盜,故刑徒多而亂不治也。苟教之所去,罰當其罪,一離刀鋸,沒身不齒,鄰里且猶恥之,而況于鄉黨乎?而況朝廷乎?如此,則夙沙、趙高之儔,無施其惡矣。古者察其言,觀其行,而善惡彰焉。然則君子之去刑辟,固已遠矣。過誤不幸,則八議之所宥也。若夫卞和、史遷之冤,淫刑之所及也。苟失其道,或不免于大辟,而況肉刑哉!漢書:「斬右趾及殺人先自言告,吏坐受賕,守官物而即盜之,皆棄巿。」此班固所謂當生而令死者也。今不忍刻截之慘,而安勦絕之悲,此最治體之所先,有國所宜改者也。}
 
 
太和四年,繇薨。帝素服臨弔,謚曰成侯。
 \gezhu{魏書曰:有司議謚,以為繇昔為廷尉,辨理刑獄,決嫌明疑,民無怨者,由于、張之在漢也。詔曰:「太傅功高德茂,位為師保,論行賜謚,常先依此,兼叙廷尉于、張之德耳。」乃策謚曰成侯。}
 子毓嗣。初,文帝分毓戶邑,封繇弟演及子劭、孫豫列侯。
 
 
 
 
 毓字稚叔。年十四為散騎侍郎,機捷談笑,有父風。太和初,蜀相諸葛亮圍祁山,明帝欲西征,毓上疏曰:「夫策貴廟勝,功尚帷幄,不下殿堂之上,而決勝千里之外。車駕宜鎮守中土,以為四方威勢之援。今大軍西征,雖有百倍之威,於關中之費,所損非一。且盛暑行師,詩人所重,實非至尊動軔之時也。」遷黃門侍郎。時大興洛陽宮室,車駕便幸許昌,天下當朝正許昌。許昌偪狹,於城南以氊為殿,備設魚龍曼延,民罷勞役。毓諫,以為「水旱不時,帑藏空虛,凡此之類,可須豐年。」又上「宜復關內開荒地,使民肆力於農。」事遂施行。正始中,為散騎常侍。大將軍曹爽盛夏興軍伐蜀,蜀拒守,軍不得進。爽方欲增兵,毓與書曰:「竊以為廟勝之策,不臨矢石;王者之兵,有征無戰。誠以干戚可以服有苗,退舍足以納原寇,不必縱吳漢於江關,騁韓信於井陘也。見可而進,知難而退,蓋自古之政。惟公侯詳之!」爽無功而還。後以失爽意,徙侍中,出為魏郡太守。爽旣誅,入為御史中丞、侍中廷尉。聽君父已沒,臣子得為理謗,及士為侯,其妻不復配嫁,毓所創也。
 
 
正元中,毌丘儉、文欽反,毓持節至揚、豫州班行赦令,告諭士民,還為尚書。諸葛誕反,大將軍司馬文王議自詣壽春討誕。會吳大將孫壹率衆降,或以為「吳新有釁,必不能復出軍。東兵已多,可須後問」。毓以為「夫論事料敵,當以己度人。今誕舉淮南之地以與吳國,孫壹所率,口不至千,兵不過三百。吳之所失,蓋為無幾。若壽春之圍未解,而吳國之內轉安,未可必其不出也。」大將軍曰:「善。」遂將毓行。
 \gezhu{臣松之以為諸葛誕舉淮南以與吳,孫壹率三百人以歸魏,謂吳有釁,本非有理之言。毓之此議,蓋何足稱耳!}
 淮南旣平,為青州刺史,加後將軍,遷都督徐州諸軍事,假節,又轉都督荊州。景元四年薨,追贈車騎將軍,謚曰惠侯。子駿嗣。毓弟會,自有傳。
 
 
\end{pinyinscope}