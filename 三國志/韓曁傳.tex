\article{韓曁傳}
\begin{pinyinscope}
 
 
 韓曁字公至,南陽堵陽人也。
 
 
\gezhu{楚國先賢傳曰:曁,韓王信之後。祖術,河東太守。父純,南郡太守。}
 同縣豪右陳茂,譖曁父兄,幾至大辟。曁陽不以為言,庸賃積資,陰結死士,遂追呼尋禽茂,以首祭父墓,由是顯名。舉孝廉,司空辟,皆不就。乃變名姓,隱居避亂魯陽山中。山民合黨,欲行寇掠。曁散家財以供牛酒,請其渠帥,為陳安危。山民化之,終不為害。避袁術命召,徙居山都之山。荊州牧劉表禮辟,遂遁逃,南居孱陵界,所在見敬愛,而表深恨之。曁懼,應命,除宜城長。
 
 
太祖平荊州,辟為丞相士曹屬。後選樂陵太守,徙監冶謁者。舊時冶,作馬排,
 \gezhu{蒲拜反。為排以吹炭。}
 每一熟石用馬百匹;更作人排,又費功力;曁乃因長流為水排,計其利益,三倍於前。在職七年,器用充實。制書襃歎,就加司金都尉,班亞九卿。文帝踐阼,封宜城亭侯。黃初七年,遷太常,進封南鄉亭侯,邑二百戶。
 
 
時新都洛陽,制度未備,而宗廟主祏
 \gezhu{祏音石。春秋傳曰:命我先人典司宗祏。注曰:「宗廟所以藏主石室者。」}
 皆在鄴都。曁奏請迎鄴四廟神主,建立洛陽廟,四時蒸嘗,親奉粢盛。崇明正禮,廢去淫祀,多所匡正。在官八年,以疾遜位。景初二年春,詔曰:「太中大夫韓曁,澡身浴德,志節高絜,年踰八十,守道彌固,可謂純篤,老而益劭者也。其以曁為司徒。」夏四月薨,遺令歛以時服,葬為土藏。謚曰恭侯。
 \gezhu{楚國先賢傳曰:曁臨終遺言曰:「夫俗奢者,示之以儉,儉則節之以禮。歷見前代送終過制,失之甚矣。若爾曹敬聽吾言,斂以時服,葬以土藏,穿畢便葬,送以瓦器,慎勿有增益。」又上疏曰:「生有益於民,死猶不害於民。況臣備位台司,在職日淺,未能宣揚聖德以廣益黎庶。寢疾彌留,奄即幽冥。方今百姓農務,不宜勞役,乞不令洛陽吏民供設喪具。懼國典有常,使臣私願不得展從,謹冒以聞,惟蒙哀許。」帝得表嗟歎,乃詔曰:「故司徒韓曁,積德履行,忠以立朝,至於黃髮,直亮不虧。旣登三事,望獲毗輔之助,如何奄忽,天命不永!曾參臨沒,易簀以禮;晏嬰尚儉,遣車降制。今司徒知命,遺言卹民,必欲崇約,可謂善始令終者也。其喪禮所設,皆如故事,勿有所闕。」時賜溫明祕器,衣一稱,五時朝服,玉具劒佩。}
 子肇嗣。肇薨,子邦嗣。
 \gezhu{楚國先賢傳曰:邦字長林。少有才學。晉武帝時為野王令,有稱績。為新城太守,坐舉野王故吏為新城計吏,武帝大怒,遂殺邦。曁次子繇,高陽太守。繇子洪,侍御史。洪子壽,字德貞。晉諸公贊曰:自曁已下,世治素業,壽能敦尚家風,性尤忠厚。早歷清職,惠帝踐阼,為散騎常侍,遷守河南尹。病卒,贈驃騎將軍。壽妻賈充女。充無後,以壽子謐為嗣,弱冠為祕書監侍中,性驕佚而才出壽。少子蔚,亦有器望,並為趙王倫所誅。韓氏遂滅。}
 
 
\end{pinyinscope}