\article{虞翻傳}
\begin{pinyinscope}
 
 
 虞翻字仲翔,會稽餘姚人也,
 
 
\gezhu{吳書曰:翻少好學,有高氣。年十二,客有候其兄者,不過翻,翻追與書曰:「僕聞虎魄不取腐芥,磁石不受曲鍼,過而不存,不亦宜乎!」客得書奇之,由是見稱。}
 太守王朗命為功曹。孫策征會稽,翻時遭父喪,衰絰詣府門,朗欲就之,翻乃脫衰入見,勸朗避策。朗不能用,拒戰敗績,亡走浮海。翻追隨營護,到東部候官,候官長閉城不受,翻往說之,然後見納。
 \gezhu{吳書曰:翻始欲送朗到廣陵,朗惑王方平記,言「疾來邀我,南岳相求」,故遂南行。旣至候官,又欲投交州,翻諫朗曰:「此妄書耳,交州無南岳,安所投乎?」乃止。}
 朗謂翻曰:「卿有老母,可以還矣。」
 \gezhu{翻別傳曰:朗使翻見豫章太守華歆,圖起義兵。翻未至豫章,聞孫策向會稽,翻乃還。會遭父喪,以臣使有節,不敢過家,星行追朗至候官。朗遣翻還,然後奔喪。而傳云孫策之來,翻衰絰詣府門,勸朗避策,則為大異。}
 翻旣歸,策復命為功曹,待以交友之禮,身詣翻第。
 \gezhu{江表傳曰:策書謂翻曰:「今日之事,當與卿共之,勿謂孫策作郡吏相待也。」}
 
 
策好馳騁游獵,翻諫曰:「明府用烏集之衆,驅散附之士,皆得其死力,雖漢高帝不及也。至於輕出微行,從官不暇嚴,吏卒常苦之。夫君人者不重則不威,故白龍魚服,困於豫且,白蛇自放,劉季害之,願少留意。」策曰:「君言是也。然時有所思,端坐悒悒,有裨諶草創之計,是以行耳。」
 \gezhu{吳書曰:策討山越,斬其渠帥,悉令左右分行逐賊,獨騎與翻相得山中。翻問左右安在,策曰:「悉行逐賊。」翻曰:「危事也!」令策下馬:「此草深,卒有驚急,馬不及縈策,但牽之,執弓矢以步。翻善用矛,請在前行。」得平地,勸策乘馬。策曰:「卿無馬柰何?」荅曰:「翻能步行,日可三百里,自征討以來,吏卒無及翻者,明府試躍馬,翻能疏步隨之。」行及大道,得一鼓吏,策取角自鳴之,部曲識聲,小大皆出,遂從周旋,平定三郡。江表傳曰:策討黃祖,旋軍欲過取豫章,特請翻語曰:「華子魚自有名字,然非吾敵也。加聞其戰具甚少,若不開門讓城,金鼓一震,不得無所傷害,卿便在前具宣孤意。」翻即奉命辭行,徑到郡,請被褠葛巾與敵相見,謂歆曰:「君自料名聲之在海內,孰與鄙郡故王府君?」歆曰:「不及也。」翻曰:「豫章資糧多少?器仗精否?士民勇果孰與鄙郡?」又曰:「不如也。」翻曰:「討逆將軍智略超世,用兵如神,前走劉揚州,君所親見,南定鄙郡,亦君所聞也。今欲守孤城,自料資糧,已知不足,不早為計,悔無及也。今大軍已次椒丘,僕便還去,明日日中迎檄不到者,與君辭矣。」翻旣去,歆明旦出城,遣吏迎策。策旣定豫章,引軍還吳,饗賜將士,計功行賞,謂翻曰:「孤昔再至壽春,見馬日磾,及與中州士大夫會,語我東方人多才耳,但恨學問不博,語議之間,有所不及耳。孤意猶謂未耳。卿博學洽聞,故前欲令卿一詣許,交見朝士,以折中國妄語兒。卿不願行,便使子綱;恐子綱不能結兒輩舌也。」翻曰:「翻是明府家寶,而以示人,人儻留之,則去明府良佐,故前不行耳。」策笑曰:「然。」因曰:「孤有征討事,未得還府,卿復以功曹為吾蕭何,守會稽耳。」後三日,便遣翻還郡。臣松之以為王、華二公於擾攘之時,抗猛銳之鋒,俱非所能。歆之名德,實高於朗,而江表傳述翻說華,云「海內名聲,孰與於王」,此言非也。然王公拒戰,華逆請服,實由孫策初起,名微衆寡,故王能舉兵,豈武勝哉?策後威力轉盛,勢不可敵,華量力而止,非必用仲翔之說也。若使易地而居,亦華戰王服耳。案吳歷載翻謂歆曰:「竊聞明府與王府君齊名中州,海內所宗,雖在東垂,常懷瞻仰。」歆荅曰:「孤不如王會稽。」翻復問:「不審豫章精兵,何如會稽?」對曰:「大不如也。」翻曰:「明府言不如王會稽,謙光之譚耳;精兵不如會稽,實如尊教。」因述孫策才略殊異,用兵之奇,歆乃荅云當去。翻出,歆遣吏迎策。二說有不同,此說為勝也。}
 
 
翻出為富春長。策薨,諸長吏並欲出赴喪,翻曰:「恐鄰縣山民或有姧變,遠委城郭,必致不虞。」因留制服行喪。諸縣皆効之,咸以安寧。
 \gezhu{吳書曰:策薨,權統事。定武中郎將暠,策之從兄也,屯烏程,整帥吏士,欲取會稽。會稽聞之,使民守城以俟嗣主之命,因令人告諭暠。會稽典錄載翻說暠曰:「討逆明府,不竟天年。今攝事統衆,宜在孝廉,翻已與一郡吏士,嬰城固守,必欲出一旦之命,為孝廉除害,惟執事圖之。」於是暠退。臣松之案:此二書所說策亡之時,翻猶為功曹,與本傳不同。}
 後翻州舉茂才,漢召為侍御史,曹公為司空辟,皆不就。
 \gezhu{吳書曰:翻聞曹公辟,曰:「盜跖欲以餘財污良家邪?」遂拒不受。}
 
 
 
 
 翻與少府孔融書,并示以所著易注。融荅書曰:「聞延陵之理樂,覩吾子之治易,乃知東南之美者,非徒會稽之竹箭也。又觀象雲物,察應寒溫,原其禍福,與神合契,可謂探賾窮通者也。」會稽東部都尉張紘又與融書曰:「虞仲翔前頗為論者所侵,美寶為質,彫摩益光,不足以損。」
 
 
 
 
 孫權以為騎都尉。翻數犯顏諫爭,權不能恱,又性不協俗,多見謗毀,坐徙丹楊涇縣。呂蒙圖取關羽,稱疾還建業,以翻兼知醫術,請以自隨,亦欲因此令翻得釋也。後蒙舉軍西上,南郡太守麋芳開城出降。蒙未據郡城而作樂沙上,翻謂蒙曰:「今區區一心者麋將軍也,城中之人豈可盡信,何不急入城持其管籥乎?」蒙即從之。時城中有伏計,賴翻謀不行。關羽旣敗,權使翻筮之,得兌下坎上,節,五爻變之臨,翻曰:「不出二日,必當斷頭。」果如翻言。權曰:「卿不及伏羲,可與東方朔為比矣。」
 
 
魏將于禁為羽所獲,繫在城中,權至釋之,請與相見。他日,權乘馬出,引禁併行,翻呵禁曰:「爾降虜,何敢與吾君齊馬首乎!」欲抗鞭擊禁,權呵止之。後權于樓船會羣臣飲,禁聞樂流涕,翻又曰:「汝欲以偽求免邪?」權悵然不平。
 \gezhu{吳書曰:後權與魏和,欲遣禁還歸北,翻復諫曰:「禁敗數萬衆,身為降虜,又不能死。北習軍政,得禁必不如所規。還之雖無所損,猶為放盜,不如斬以令三軍,示為人臣有二心者。」權不聽。羣臣送禁,翻謂禁曰:「卿勿謂吳無人,吾謀適不用耳。」禁雖為翻所惡,然猶盛歎翻,魏文帝常為翻設虛坐。}
 
 
 
 
 權旣為吳王,歡宴之末,自起行酒,翻伏地陽醉,不持。權去,翻起坐。權於是大怒,手劒欲擊之,侍坐者莫不惶遽,惟大司農劉基起抱權諫曰:「大王以三爵之後,手殺善士,雖翻有罪,天下孰知之?且大王以能容賢畜衆,故海內望風,今一朝棄之,可乎?」權曰:「曹孟德尚殺孔文舉,孤於虞翻何有哉?」基曰:「孟德輕害士人,天下非之。大王躬行德義,欲與堯、舜比隆,何得自喻於彼乎?」翻由是得免。權因勑左右,自今酒後言殺,皆不得殺。
 
 
 
 
 翻甞乘船行,與麋芳相逢,芳船上人多欲令翻自避,先驅曰:「避將軍船!」翻厲聲曰:「失忠與信,何以事君?傾人二城,而稱將軍,可乎?」芳闔戶不應而遽避之。後翻乘車行,又經芳營門,吏閉門,車不得過。翻復怒曰:「當閉反開,當開反閉,豈得事宜邪?」芳聞之,有慙色。
 
 
翻性疏直,數有酒失。權與張昭論及神仙,翻指昭曰:「彼皆死人,而語神仙,世豈有仙人也!」權積怒非一,遂徙翻交州。雖處罪放,而講學不倦,門徒常數百人。
 \gezhu{翻別傳曰:權即尊號,翻因上書曰:「陛下膺明聖之德,體舜、禹之孝,歷運當期,順天濟物。奉承策命,臣獨抃舞。罪棄兩絕,拜賀無階,仰瞻宸極,且喜且悲。臣伏自刻省,命輕雀鼠,性輶毫釐,罪惡莫大,不容於誅,昊天罔極,全宥九載,退當念戮,頻受生活,復偷視息。臣年耳順,思咎憂憤,形容枯悴,髮白齒落,雖未能死,自悼終沒,不見宮闕百官之富,不覩皇輿金軒之飾,仰觀巍巍衆民之謠,傍聽鍾鼓侃然之樂,永隕海隅,棄骸絕域,不勝悲慕,逸豫大慶,恱以忘罪。」}
 又為老子、論語、國語訓注,皆傳於世。
 \gezhu{翻別傳曰:翻初立易注,奏上曰:「臣聞六經之始,莫大陰陽,是以伏羲仰天縣象,而建八卦,觀變動六爻為六十四,以通神明,以類萬物。臣高祖父故零陵太守光,少治孟氏易,曾祖父故平輿令成,纘述其業,至臣祖父鳳為之最密。臣亡考故日南太守歆,受本於鳳,最有舊書,世傳其業,至臣五世。前人通講,多玩章句,雖有祕說,於經疏闊。臣生遇世亂,長於軍旅,習經於枹鼓之間,講論於戎馬之上,蒙先師之說,依經立注。又臣郡吏陳桃夢臣與道士相遇,放髮被鹿裘,布易六爻,撓其三以飲臣,臣乞盡吞之。道士言易道在天,三爻足矣。豈臣受命,應當知經!所覽諸家解不離流俗,義有不當實,輒悉改定,以就其正。孔子曰:『乾元用九而天下治。』聖人南面,蓋取諸离,斯誠天子所宜協陰陽致麟鳳之道矣。謹正書副上,惟不罪戾。」翻又奏曰:「經之大者,莫過於易。自漢初以來,海內英才,其讀易者,解之率少。至孝靈之際,潁川荀諝號為知易,臣得其注,有愈俗儒,至所說西南得朋,東北喪朋,顛倒反逆,了不可知。孔子歎易曰:『知變化之道者,其知神之所為乎!』以美大衍四象之作,而上為章首,尤可怪笑。又南郡太守馬融,名有俊才,其所解釋,復不及諝。孔子曰『可與共學,未可與適道』,豈不其然!若乃北海鄭玄,南陽宋忠,雖各立注,忠小差玄而皆未得其門,難以示世。」又奏鄭玄解尚書違失事目:「臣聞周公制禮以辨上下,孔子曰『有君臣然後有上下,有上下然後禮義有所錯』,是故尊君卑臣,禮之大司也。伏見故徵士北海鄭玄所注尚書,以顧命康王執瑁,古『月』似『同』,從誤作『同』,旣不覺定,復訓為杯,謂之酒杯;成王疾困憑几,洮頮為濯,以為澣衣成事,『洮』字虛更作『濯』,以從其非;又古大篆『卯』字讀當為『柳』,古『柳』『卯』同字,而以為昧;『分北三苗』,『北』古『別』字,又訓北,言北猶別也。若此之類,誠可怪也。玉人職曰天子執瑁以朝諸侯,謂之酒杯;天子頮面,謂之澣衣;古篆『卯』字,反以為昧。甚違不知蓋闕之義。於此數事,誤莫大焉,宜命學官定此三事。又馬融訓注亦以為同者大同天下,今經益『金』就作『銅』字,詁訓言天子副璽,雖皆不得,猶愈於玄。然此不定,臣沒之後,而奮乎百世,雖世有知者,懷謙莫或奏正。又玄所注五經,違義尤甚者百六十七事,不可不正。行乎學校,傳乎將來,臣竊恥之。」翻放棄南方,云「自恨疏節,骨體不媚,犯上獲罪,當長沒海隅,生無可與語,死以青蠅為弔客,使天下一人知己者,足以不恨。」以典籍自慰,依易設象,以占吉凶。又以宋氏解玄頗有繆錯,更為立法,并著明楊、釋宋以理其滯。臣松之案:翻云「古大篆『卯』字讀當言『柳』,古『柳』『卯』同字」,竊謂翻言為然。故「劉」「留」「聊」「柳」同用此字,以從聲故也,與日辰「卯」字字同音異。然漢書王莽傳論卯金刀,故以為日辰之「卯」,今未能詳正。然世多亂之,故翻所說云。荀諝,荀爽之別名。}
 
 
初,山陰丁覽,太末徐陵,或在縣吏之中,或衆所未識,翻一見之,便與友善,終成顯名。
 \gezhu{會稽典錄曰:覽字孝連,八歲而孤,家又單微,清身立行,用意不苟,推財從弟,以義讓稱。仕郡至功曹,守始平長。為人精微絜淨,門無雜賔。孫權深貴待之,未及擢用,會病卒,甚見痛惜,殊其門戶。覽子固,字子賤,本名密,避滕密,改作固。固在襁褓中,闞澤見而異之,曰:「此兒後必致公輔。」固少喪父,獨與母居,家貧守約,色養致敬,族弟孤弱,與同寒溫。翻與固同僚書曰:「丁子賤塞淵好德,堂構克舉,野無遺薪,斯之為懿,其美優矣。令德之後,惟此君嘉耳。」歷顯位,孫休時固為左御史大夫,孫皓即位,遷司徒。皓悖虐,固與陸凱、孟宗同心憂國,年七十六卒。子彌,字欽遠,仕晉,至梁州刺史。孫潭,光祿大夫。徐陵字元大,歷三縣長,所在著稱,遷零陵太守。時朝廷俟以列卿之位,故翻書曰:「元大受上卿之遇,叔向在晉,未若於今。」其見重如此。陵卒,僮客土田或見侵奪,駱統為陵家訟之,求與丁覽、卜清等為比,權許焉。陵子平,字伯先,童齔知名,翻甚愛之,屢稱歎焉。諸葛恪為丹楊太守,討山越,以平威重思慮,可與效力,請平為丞,稍遷武昌左部督,傾心接物,士卒皆為盡力。初,平為恪從事,意甚薄,及恪輔政,待平益疏。恪被害,子建亡走,為平部曲所得,平使遣去,別為佗軍所獲。平兩婦歸宗,敬奉情過乎厚。其行義敦篤,皆此類也。}
 
 
在南十餘年,年七十卒。
 \gezhu{吳書曰:翻雖在徙棄,心不忘國,常憂五谿宜討,以遼東海絕,聽人使來屬,尚不足取,今去人財以求馬,旣非國利,又恐無獲。欲諫不敢,作表以示呂岱,岱不報,為愛憎所白,復徙蒼梧猛陵。江表傳曰:後權遣將士至遼東,於海中遭風,多所沒失,權悔之,乃令曰:「昔趙簡子稱諸君之唯唯,不如周舍之諤諤。虞翻亮直,善於盡言,國之周舍也。前使翻在此,此役不成。」促下問交州,翻若尚存者,給其人船,發遣還都;若以亡者,送喪還本郡,使兒子仕宦。會翻已終。}
 歸葬舊墓,妻子得還。
 \gezhu{會稽典錄曰:孫亮時,有山陰朱育,少好奇字,凡所特達,依體象類,造作異字千名以上。仕郡門下書佐。太守濮陽興正旦宴見掾吏,言次,問:「太守昔聞朱潁川問士於鄭召公,韓吳郡問士於劉聖博,王景興問士於虞仲翔,甞見鄭、劉二荅而未覩仲翔對也。欽聞國賢,思覩盛美有日矣,書佐寧識之乎?」育對曰:「往過習之。昔初平末年,王府君以淵妙之才,超遷臨郡,思賢嘉善,樂采名俊,問功曹虞翻曰:『聞玉出崑山,珠生南海,遠方異域,各生珍寶。且曾聞士人歎美貴邦,舊多英俊,徒以遠於京畿,含香未越耳。功曹雅好博古,寧識其人邪?』翻對曰:『夫會稽上應牽牛之宿,下當少陽之位,東漸巨海,西通五湖,南暢無垠,北渚浙江,南山攸居,實為州鎮,昔禹會羣臣,因以命之。山有金木鳥獸之殷,水有魚鹽珠蚌之饒,海嶽精液,善生俊異,是以忠臣係踵,孝子連閭,下及賢女,靡不育焉。』王府君笑曰:『地勢然矣,士女之名可悉聞乎?』翻對曰:『不敢及遠,略言其近者耳。往者孝子句章董黯,盡心色養,喪致其哀,單身林野,鳥獸歸懷,怨親之辱,白日報讎,海內聞名,昭然光著。太中大夫山陰陳嚻,漁則化盜,居則讓鄰,感侵退藩,遂成義里,攝養車嫗,行足厲俗,自揚子雲等上書薦之,粲然傳世。太尉山陰鄭公,清亮質直,不畏彊禦。魯相山陰鍾離意,稟殊特之姿,孝家忠朝,宰縣相國,所在遺惠,故取養有君子之謩,魯國有丹書之信。及陳宮、費齊皆上契天心,功德治狀,記在漢籍,有道山陰趙曄,徵士上虞王充,各洪才淵懿,學究道源,著書垂藻,駱驛百篇,釋經傳之宿疑,解當世之槃結,或上窮陰陽之奧祕,下攄人情之歸極。交阯刺史上虞綦毋俊,拔濟一郡,讓爵土之封。決曹掾上虞孟英,三世死義。主簿句章梁宏,功曹史餘姚駟勳,主簿句章鄭雲,皆敦終始之義,引罪免居。門下督盜賊餘姚伍隆,鄮莫候反主簿任光,章安小吏黃他,身當白刃,濟君於難。揚州從事句章王脩,委身授命,垂聲來世。河內太守上虞魏少英,遭世屯蹇,忘家憂國,列在八俊,為世英彥。尚書烏傷楊喬,桓帝妻以公主,辭疾不納。近故太尉上虞朱公,天姿聦亮,欽明神武,策無失謨,征無遺慮,是以天下義兵,思以為首。上虞女子曹娥,父溺江流,投水而死,立石碑紀,炳然著顯。』王府君曰:『是旣然矣,潁川有巢、許之逸軌,吳有太伯之三讓,貴郡雖士人紛紜,於此足矣。』翻對曰:『故先言其近者耳,若乃引上世之事,及抗節之士,亦有其人。昔越王翳讓位,逃于巫山之穴,越人薰而出之,斯非太伯之儔邪?且太伯外來之君,非其地人也。若以外來言之,則大禹亦巡於此而葬之矣。鄞大里黃公,絜己暴秦之世,高祖即阼,不能一致,惠帝恭讓,出則濟難。徵士餘姚嚴遵,王莽數聘,抗節不行,光武中興,然後俯就,矯手不拜,志陵雲日。皆著於傳籍,較然彰明,豈如巢、許,流俗遺譚,不見經傳者哉?』王府君笑曰:『善哉話言也!賢矣,非君不著。太守未之前聞也。』」濮陽府君曰:「御史所云,旣聞其人,亞斯已下,書佐寧識之乎?」育曰:「瞻仰景行,敢不識之?近者太守上虞陳業,絜身清行,志懷霜雪,貞亮之信,同操柳下,遭漢中微,委官棄祿,遁迹黟歙,以求其志,高邈妙蹤,天下所聞,故桓文林遺之尺牘之書,比竟三高。其聦明大略,忠直謇諤,則侍御史餘姚虞翻、偏將軍烏傷駱統。其淵懿純德,則太子少傅山陰闞澤,學通行茂,作帝師儒。其雄姿武毅,立功當世,則後將軍賀齊,勳成績著。其探極祕術,言合神明,則太史令上虞吳範。其文章之士,立言粲盛,則御史中丞句章任奕,鄱陽太守章安虞翔,各馳文檄,曄若春榮。處士鄮盧叙,弟犯公憲,自殺乞代。吳寧斯敦、山陰祁庚、上虞樊正,咸代父死罪。其女則松陽柳朱、永寧翟素,或一醮守節,喪身不顧,或遭寇劫賊,死不虧行。皆近世之事,尚在耳目。」府君曰:「皆海內之英也。吾聞秦始皇二十五年,以吳越地為會稽郡,治吳。漢封諸侯王,以何年復為郡,而分治於此?」育對曰:「劉賈為荊王,賈為英布所殺,又以劉濞為吳王。景帝四年,濞反誅,乃復為郡,治於吳。元鼎五年,除東越,因以其地為治,并屬於此,而立東部都尉,後徙章安。陽朔元年,又徙治鄞,或有寇害,復徙句章。到永建四年,劉府君上書,浙江之北,以為吳郡,會稽還治山陰。自永建四年歲在己巳,以至今年,積百二十九歲。」府君稱善。是歲,吳之太平三年,歲在丁丑。育後仕朝,常在臺閣,為東觀令,遙拜清河太守,加位侍中,推刺占射,文藝多通。}
 
 
翻有十一子,第四子汜最知名,永安初,從選曹郎為散騎中常侍,後為監軍使者,討扶嚴,病卒。
 \gezhu{會稽典錄曰:汜字世洪,生南海,年十六,父卒,還鄉里。孫綝廢幼主,迎立琅邪王休。休未至,綝欲入宮,圖為不軌,召百官會議,皆惶怖失色,徒唯唯而已。汜對曰:「明公為國伊周,處將相之位,擅廢立之威,將上安宗廟,下惠百姓,大小踴躍,自以伊霍復見。今迎王未至,而欲入宮,如是,羣下搖蕩,衆聽疑惑,非所以永終忠孝,揚名後世也。」綝不懌,竟立休。休初即位,汜與賀邵、王蕃、薛瑩俱為散騎中常侍。以討扶嚴功拜交州刺史、冠軍將軍、餘姚侯,尋卒。}
 汜弟忠,宜都太守;
 \gezhu{會稽典錄曰:忠字世方,翻第五子。貞固幹事,好識人物,造吳郡陸機於童齔之年,稱上虞魏遷於無名之初,終皆遠致,為著聞之士。交同縣王岐於孤宦之族,仕進先至宜都太守,忠乃代之。晉征吳,忠與夷道監陸晏、晏弟中夏督景堅守不下,城潰被害。忠子潭,字思奧。晉陽秋稱潭清貞有檢操,外如退弱,內堅正有膽幹。仕晉,歷位內外,終於衞將軍,追贈侍中左光祿大夫,開府儀同三司。}
 聳,越騎校尉,累遷廷尉,湘東、河間太守;
 \gezhu{會稽典錄曰:聳字世龍,翻第六子也。清虛無欲,進退以禮,在吳歷清官,入晉,除河間相,王素聞聳名,厚敬禮之。聳抽引人物,務在幽隱孤陋之中。時王岐難聳,以高士所達,必合秀異,聳書與族子察曰:「世之取士,曾不招未齒於丘園,索良才於總猥,所譽依已成,所毀依已敗,此吾所以歎息也。」聳疾俗喪祭無度,弟昺卒,祭以少牢,酒飯而已,當時族黨並遵行之。}
 昺,廷尉尚書,濟陰太守。
 \gezhu{會稽典錄曰:昺字世文,翻第八子也。少有倜儻之志,仕吳黃門郎,以捷對見異,超拜尚書侍中。晉軍來伐,遣昺持節都督武昌已上諸軍事,昺先上還節蓋印綬,然後歸順。在濟陰,抑彊扶弱,甚著威風。}
 
 
\end{pinyinscope}