\article{賀齊傳}
\begin{pinyinscope}
 
 
 賀齊字公苗,會稽山陰人也。
 
 
\gezhu{虞預晉書曰:賀氏本姓慶氏。齊伯父純,儒學有重名,漢安帝時為侍中、江夏太守,去官,與江夏黃瓊、漢中楊厚俱公車徵。避安帝父孝德皇帝諱,改為賀氏。齊父輔,永寧長。}
 少為郡吏,守剡長。縣吏斯從輕俠為姧,齊欲治之,主簿諫曰:「從,縣大族,山越所附,今日治之,明日寇至。」齊聞大怒,便立斬從。從族黨遂相糾合,衆千餘人,舉兵攻縣。齊率吏民,開城門突擊,大破之,威震山越。後太末、豐浦民反,轉守太末長,誅惡養善,期月盡平。
 
 
 
 
 建安元年,孫策臨郡,察齊孝廉。時王朗奔東冶,候官長商升為朗起兵。策遣永寧長韓晏領南部都尉,將兵討升,以齊為永寧長。晏為升所敗,齊又代晏領都尉事。升畏齊威名,遣使乞盟。齊因告喻,為陳禍福,升遂送上印綬,出舍求降。賊帥張雅、詹彊等不願升降,反共殺升,雅稱無上將軍,彊稱會稽太守。賊盛兵少,未足以討,齊住軍息兵。雅與女壻何雄爭勢兩乖,齊令越人因事交構,遂致疑隙,阻兵相圖。齊乃進討,一戰大破雅,彊黨震懼,率衆出降。
 
 
候官旣平,而建安、漢興、南平復亂,齊進兵建安,立都尉府,是歲八年也。郡發屬縣五千兵,各使本縣長將之,皆受齊節度。賊洪明、洪進、苑御、吳免、華當等五人,率各萬戶,連屯漢興,吳五
 \gezhu{姓吳名五}
 六千戶別屯大潭,鄒臨六千戶別屯蓋竹,同出餘汗。
 \gezhu{音干。}
 軍討漢興,經餘汗。齊以為賊衆兵少,深入無繼,恐為所斷,令松陽長丁蕃留備餘汗。蕃本與齊鄰城,恥見部伍,辭不肯留。齊乃斬蕃,於是軍中震慄,無不用命。遂分兵留備,進討明等,連大破之。臨陣斬明,其免、當、進、御皆降。轉擊蓋竹,軍向大潭,三將又降。凡討治斬首六千級,名帥盡禽,復立縣邑,料出兵萬人,拜為平東校尉。十年,轉討上饒,分以為建平縣。
 
 
十三年,遷威武中郎將,討丹陽黟、歙。時武彊、葉鄉、東陽、豐浦四鄉先降,齊表言以葉鄉為始新縣。而歙賊帥金奇萬戶屯安勤山,毛甘萬戶屯烏聊山,黟帥陳僕、祖山等二萬戶屯林歷山。林歷山四靣壁立,高數十丈,徑路危狹,不容刀楯,賊臨高下石,不可得攻。軍住經日,將吏患之。齊身出周行,觀視形便,陰募輕捷士,為作鐵弋,密於隱險賊所不備處,以弋拓斬山為緣道,夜令潛上,乃多縣布以援下人,得上百數人,四靣流布,俱鳴皷角,齊勒兵待之。賊夜聞皷聲四合,謂大軍悉已得上,驚懼惑亂,不知所為,守路備險者,皆走還依衆。大軍因是得上,大破僕等,其餘皆降,凡斬首七千。
 \gezhu{抱朴子曰:昔吳遣賀將軍討山賊,賊中有善禁者,每當交戰,官軍刀劒不得拔,弓弩射矢皆還相向,輒致不利。賀將軍長情有思,乃曰:「吾聞金有刃者可禁,蟲有毒者可禁,其無刃之物,無毒之蟲,則不可禁。彼必是能禁吾兵者也,必不能禁無刃物矣。」乃多作勁木白棓,選有力精卒五千人為先登,盡捉棓。彼山賊恃其有善禁者,了不嚴備。於是官軍以白棓擊之,彼禁者果不復行,所擊殺者萬計。}
 齊復表分歙為新定、黎陽、休陽。并黟、歙,凡六縣,權遂割為新都郡,齊為太守,立府於始新,加偏將軍。
 
 
十六年,吳郡餘杭民郎稚合宗起賊,復數千人,齊出討之,即復破稚,表言分餘杭為臨水縣。
 \gezhu{吳錄曰:晉改為臨安。}
 被命詣所在,及當還郡,權出祖道,作樂舞象。
 \gezhu{吳書曰:權謂齊曰:「今定天下,都中國,使殊俗貢珍,狡獸卒舞,非君誰與?」齊曰:「殿下以神武應期,廓開王業,臣幸遭際會,得驅馳風塵之下,佐助末行,效鷹犬之用,臣之願也。若殊俗貢珍,狡獸率舞,宜在聖德,非臣所能。」}
 賜齊軿車駿馬,罷坐住駕,使齊就車。齊辭不敢,權使左右扶齊上車,令導吏卒兵騎,如在郡儀。權望之笑曰:「人當努力,非積行累勤,此不可得。」去百餘步乃旋。
 
 
 
 
 十八年,豫章東部民彭材、李玉、王海等起為賊亂,衆萬餘人。齊討平之,誅其首惡,餘皆降服。揀其精健為兵,次為縣戶。遷奮武將軍。
 
 
二十年,從權征合肥。時城中出戰,徐盛被創失矛,齊引兵拒擊,得盛所失。
 \gezhu{江表傳曰:權征合肥還,為張遼所掩襲於津北,幾至危殆。齊時率三千兵在津南迎權。權旣入大船,會諸將飲宴,齊下席涕泣而言曰:「至尊人主,常當持重。今日之事,幾至禍敗,羣下震怖,若無天地,願以此為終身誡。」權自前收其淚曰:「大慙!謹以刻心,非但書諸紳也。」}
 
 
 
 
 二十一年,鄱陽民尤突受曹公印綬,化民為賊,陵陽、始安、涇縣皆與突相應。齊與陸遜討破突,斬首數千,餘黨震服,丹楊三縣皆降,料得精兵八千人。拜安東將軍,封山陰侯,出鎮江上,督扶州以上至皖。
 
 
 
 
 黃武初,魏使曹休來伐,齊以道遠後至,因住新市為拒。會洞口諸軍遭風流溺,所亡中分,將士失色,賴齊未濟,偏軍獨全,諸將倚以為埶。
 
 
 
 
 齊性奢綺,尤好軍事,兵甲器械極為精好,所乘船雕刻丹鏤,青蓋絳襜,干櫓戈矛,葩瓜文畫,弓弩矢箭,咸取上材,蒙衝鬬艦之屬,望之若山。休等憚之,遂引軍還。遷後將軍,假節領徐州牧。
 
 
初,晉宗為戲口將,以衆叛如魏,還為蘄春太守,圖襲安樂,取其保質。權以為恥忿,因軍初罷,六月盛夏,出其不意,詔齊督麋芳、鮮于丹等襲蘄春,遂生虜宗。後四年卒,子達及弟景皆有令名,為佳將。
 \gezhu{會稽典錄曰:景為滅賊校尉,御衆嚴而有恩,兵器精飾,為當時冠絕,早卒。達頗任氣,多所犯迕,故雖有征戰之勞,而爵位不至,然輕財貴義,膽烈過人。子質,位至虎牙將軍。景子邵,別有傳。}
 
 
\end{pinyinscope}