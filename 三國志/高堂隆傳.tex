\article{高堂隆傳}
\begin{pinyinscope}
 
 
 高堂隆字升平,泰山平陽人,魯高堂生後也。少為諸生,泰山太守薛悌命為督郵。郡督軍與悌爭論,名悌而呵之。隆按劒叱督軍曰:「昔魯定見侮,仲尼歷階;趙彈秦箏,相如進缶。臨臣名君,義之所討也。」督軍失色,悌驚起止之。後去吏,避地濟南。
 
 
 
 
 建安十八年,太祖召為丞相軍議掾,後為歷城侯徽文學,轉為相。徽遭太祖喪,不哀,反游獵馳騁;隆以義正諫,甚得輔導之節。黃初中,為堂陽長,以選為平原王傅。王即尊位,是為明帝。以隆為給事中、博士、駙馬都尉。帝初踐阼,羣臣或以為宜響會,隆曰:「唐、虞有遏密之哀,高宗有不言之思,是以至德雍熈,光于四海。」以為不宜為會,帝敬納之。遷陳留太守。犢民酉牧,年七十餘,有至行,舉為計曹掾;帝嘉之,特除郎中以顯焉。徵隆為散騎常侍,賜爵關內侯。
 
 
\gezhu{魏略曰:太史上漢歷不及天時,因更推步弦望朔晦,為太和歷。帝以隆學問優深,於天文又精,乃詔使隆與尚書郎楊偉、太史待詔駱祿參共推校。偉、祿是太史,隆故據舊歷更相劾奏,紛紜數歲,偉稱祿得日蝕而月晦不盡,隆不得日蝕而月晦盡,詔從太史。隆所爭雖不得,而遠近猶知其精微也。}
 
 
 
 
 青龍中,大治殿舍,西取長安大鍾。隆上疏曰;「昔周景王不儀刑文、武之明德,忽公旦之聖制,旣鑄大錢,又作大鍾,單穆公諫而弗聽,泠州鳩對而弗從,遂迷不反,周德以衰,良史記焉,以為永鑒。然今之小人,好說秦、漢之奢靡以盪聖心,求取亡國不度之器,勞役費損,以傷德政,非所以興禮樂之和,保神明之休也。」是日,帝幸上方,隆與卞蘭從。帝以隆表授蘭,使難隆曰:「興衰在政,樂何為也?化之不明,豈鍾之罪?」隆曰:「夫禮樂者,為治之大本也。故簫韶九成,鳳皇來儀,雷鼓六變,天神以降,政是以平,刑是以錯,和之至也。新聲發響,商辛以隕,大鍾旣鑄,周景以弊,存亡之機,恒由斯作,安在廢興之不階也?君舉必書,古之道也,作而不法,何以示後?聖王樂聞其闕,故有箴規之道;忠臣願竭其節,故有匪躬之義也。」帝稱善。
 
 
 
 
 遷侍中,猶領太史令。崇華殿災,詔問隆:「此何咎?於禮,寧有祈禳之義乎?」隆對曰:「夫災變之發,皆所以明教誡也,惟率禮脩德,可以勝之。易傳曰:『上不儉,下不節,孽火燒其室。』又曰:『君高其臺,天火為災。』此人君苟飾宮室,不知百姓空竭,故天應之以旱,火從高殿起也。上天降鑒,故譴告陛下;陛下宜增崇人道,以荅天意。昔太戊有桑穀生於朝,武丁有雊雉登於鼎,皆聞災恐懼,側身脩德,三年之後,遠夷朝貢,故號曰中宗、高宗。此則前代之明鑒也。今案舊占,災火之發,皆以臺榭宮室為誡。然今宮室之所以充廣者,實由宮人猥多之故。宜簡擇留其淑懿,如周之制,罷省其餘。此則祖乙之所以訓高宗,高宗之所以享遠號也。」詔問隆:「吾聞漢武帝時,栢梁災,而大起宮殿以厭之,其義云何?」隆對曰:「臣聞西京栢梁旣災,越巫陳方,建章是經,以厭火祥;乃夷越之巫所為,非聖賢之明訓也。五行志曰:『栢梁災,其後有江充巫蠱也,衞太子事。』如志之言,越巫建章無所厭也。孔子曰:『災者脩類應行,精祲相感,以戒人君。』是以聖主覩災責躬,退而脩德,以消復之。今宜罷散民役。宮室之制,務從約節,內足以待風雨,外足以講禮儀。清埽所災之處,不敢於此有所立作,萐莆、嘉禾必生此地,以報陛下虔恭之德。豈可疲民之力,竭民之財!實非所以致符瑞而懷遠人也。」帝遂復崇華殿,時郡國有九龍見,故改曰九龍殿。
 
 
 
 
 陵霄闕始構,有鵲巢其上,帝以問隆,對曰:「詩云『惟鵲有巢,惟鳩居之』。今興宮室,起陵霄闕,而鵲巢之,此宮室未成身不得居之象也。天意若曰,宮室未成,將有他姓制御之,斯乃上天之戒也。夫天道無親,惟與善人,不可不深防,不可不深慮。夏、商之季,皆繼體也,不欽承上天之明命,惟讒諂是從,廢德適欲,故其亡也忽焉。太戊、武丁,覩災竦懼,祗承天戒,故其興也勃焉。今若休罷百役,儉以足用,增崇德政,動遵帝則,除普天之所患,興兆民之所利,三王可四,五帝可六,豈惟殷宗轉禍為福而已哉!臣備腹心,苟可以繁祉聖躬,安存社稷,臣雖灰身破族,猶生之年也。豈憚忤逆之災,而令陛下不聞至言乎?」於是帝改容動色。
 
 
 
 
 是歲,有星孛于大辰。隆上疏曰:「凡帝王徙都立邑,皆先定天地社稷之位,敬恭以奉之。將營宮室,則宗廟為先,廄庫為次,居室為後。今圜丘、方澤、南北郊、明堂、社稷,神位未定,宗廟之制又未如禮,而崇飾居室,士民失業。外人咸云宮人之用,與興戎軍國之費,所盡略齊。民不堪命,皆有怨怒。書曰『天聦明自我民聦明,天明畏自我民明威』,輿人作頌,則嚮以五福,民怒吁嗟,則威以六極,言天之賞罰,隨民言,順民心也。是以臨政務在安民為先,然後稽古之化,格于上下,自古及今,未嘗不然也。夫采椽卑宮,唐、虞、大禹之所以垂皇風也;玉臺瓊室,夏癸、商辛之所以犯昊天也。今之宮室,實違禮度,乃更建立九龍,華飾過前。天彗章灼,始起於房心,犯帝坐而干紫微,此乃皇天子愛陛下,是以發教戒之象,始卒皆於尊位,殷勤鄭重,欲必覺寤陛下;斯乃慈父懇切之訓,宜崇孝子祗聳之禮,以率先天下,以昭示後昆,不宜有忽,以重天怒。」
 
 
 
 
 時軍國多事,用法深重。隆上疏曰:「夫拓跡垂統,必俟聖明,輔世匡治,亦須良佐,用能庶績其凝而品物康乂也。夫移風易俗,宣明道化,使四表同風,回首面內,德教光熈,九服慕義,固非俗吏之所能也。今有司務糾刑書,不本大道,是以刑用而不措,俗弊而不敦。宜崇禮樂,班叙明堂,脩三雍、大射、養老,營建郊廟,尊儒士,舉逸民,表章制度,改正朔,易服色,布愷悌,尚儉素,然後備禮封禪,歸功天地,使雅頌之聲盈于六合,緝熈之化混于後嗣。斯蓋至治之美事,不朽之貴業也。然九域之內,可揖讓而治,尚何憂哉!不正其本而救其末,譬猶棼絲,非政理也。可命羣公卿士通儒,造具其事,以為典式。」隆又以為改正朔,易服色,殊徽號,異器械,自古帝王所以神明其政,變民耳目,故三春稱王,明三統也。於是敷演舊章,奏而改焉。帝從其議,改青龍五年春三月為景初元年孟夏四月,服色尚黃,犧牲用白,從地正也。
 
 
 
 
 遷光祿勳。帝愈增崇宮殿,彫飾觀閣,鑿太行之石英,采穀城之文石,起景陽山於芳林之園,建昭陽殿於太極之北,鑄作黃龍鳳皇奇偉之獸,飾金墉、陵雲臺、陵霄闕。百役繁興,作者萬數,公卿以下至于學生,莫不展力,帝乃躬自握土以率之。而遼東不朝。悼皇后崩。天作淫雨,兾州水出,漂沒民物。隆上疏切諫曰:
 
 
 
 
 蓋「天地之大德曰生,聖人之大寶曰位;何以守位?曰仁;何以聚人?曰財」。然則士民者,乃國家之鎮也;穀帛者,乃士民之命也。穀帛非造化不育,非人力不成。是以帝耕以勸農,后桑以成服,所以昭事上帝,告虔報施也。昔在伊唐,世值陽九厄運之會,洪水滔天,使鯀治之,績用不成,乃舉文命,隨山刊木,前後歷年二十二載。災眚之甚,莫過於彼,力役之興,莫久於此,堯、舜君臣,南面而已。禹敷九州,庶士庸勳,各有等差,君子小人,物有服章。今無若時之急,而使公卿大夫並與厮徒共供事役,聞之四夷,非嘉聲也,垂之竹帛,非令名也。是以有國有家者,近取諸身,遠取諸物,嫗煦養育,故稱「愷悌君子,民之父母」。今上下勞役,疾病凶荒,耕稼者寡,饑饉荐臻,無以卒歲;宜加愍卹,以救其困。
 
 
 
 
 臣觀在昔書籍所載,天人之際,未有不應也。是以古先哲王,畏上天之明命,循陰陽之逆順,矜矜業業,惟恐有違。然後治道用興,德與神符,災異旣發,懼而脩政,未有不延期流祚者也。爰及末葉,闇君荒主,不崇先王之令軌,不納正士之直言,以遂其情志,恬忽變戒,未有不尋踐禍難,至於顛覆者也。
 
 
 
 
 天道旣著,請以人道論之。夫六情五性,同在於人,嗜欲廉貞,各居其一。及其動也,交爭于心。欲彊質弱,則縱濫不禁;精誠不制,則放溢無極。夫情之所在,非好則美,而美好之集,非人力不成,非穀帛不立。情苟無極,則人不堪其勞,物不充其求。勞求並至,將起禍亂。故不割情,無以相供。仲尼云:「人無遠慮,必有近憂。」由此觀之,禮義之制,非苟拘分,將以遠害而興治也。
 
 
 
 
 今吳、蜀二賊,非徒白地小虜、聚邑之寇,乃據險乘流,跨有士衆,僭號稱帝,欲與中國爭衡。今若有人來告,權、備並脩德政,復履清儉,輕省租賦,不治玩好,動咨耆賢,事遵禮度。陛下聞之,豈不惕然惡其如此,以為難卒討滅,而為國憂乎?若使告者曰,彼二賊並為無道,崇侈無度,役其士民,重其徵賦,下不堪命,吁嗟日甚。陛下聞之,豈不勃然忿其困我無辜之民,而欲速加之誅,其次,豈不幸彼疲弊而取之不難乎?苟如此,則可易心而度,事義之數亦不遠矣。
 
 
 
 
 且秦始皇不築道德之基,而築阿房之宮,不憂蕭墻之變,而脩長城之役。當其君臣為此計也,亦欲立萬世之業,使子孫長有天下,豈意一朝匹夫大呼,而天下傾覆哉?故臣以為使先代之君知其所行必將至於敗,則弗為之矣。是以亡國之主自謂不亡,然後至於亡;賢聖之君自謂將亡,然後至於不亡。昔漢文帝稱為賢主,躬行約儉,惠下養民,而賈誼方之,以為天下倒縣,可為痛哭者一,可為流涕者二,可為長歎息者三。況今天下彫弊,民無儋石之儲,國無終年之畜,外有彊敵,六軍暴邊,內興土功,州郡騷動,若有寇警,則臣懼版築之士不能投命虜庭矣。
 
 
又,將吏奉祿,稍見折減,方之於昔,五分居一;諸受休者又絕廩賜,不應輸者今皆出半:此為官入兼多於舊,其所出與參少於昔。而度支經用,更每不足,牛肉小賦,前後相繼。反而推之,凡此諸費,必有所在。且夫祿賜穀帛,人主所以惠養吏民而為之司命者也,若今有廢,是奪其命矣。旣得之而又失之,此生怨之府也。周禮,天府掌九伐之則以給九式之用,入有其分,出有其所,不相干乘而用各足。各足之後,乃以式貢之餘,供王玩好。又上用財,必考于司會。
 \gezhu{會音膾。}
 今陛下所與共坐廊廟治天下者,非三司九列,則臺閣近臣,皆腹心造膝,宜在無諱。若見豐省而不敢以告,從命奔走,惟恐不勝,是則具臣,非鯁輔也。昔李斯教秦二世曰:「為人主而不恣睢,命之曰天下桎梏。」二世用之,秦國以覆,斯亦滅族。是以史遷議其不正諫,而為世誡。
 
 
 
 
 書奏,帝覽焉,謂中書監、令曰:「觀隆此奏,使朕懼哉!」
 
 
 
 
 隆疾篤,口占上疏曰:
 
 
 
 
 曾子有疾,孟敬子問之。曾子曰:「鳥之將死,其鳴也哀;人之將死,其言也善。」臣寢疾病,有增無損,常懼奄忽,忠款不昭。臣之丹誠,豈惟曾子,願陛下少垂省覽!渙然改往事之過謬,勃然興來事之淵塞,使神人嚮應,殊方慕義,四靈效珍,玉衡曜精,則三王可邁,五帝可越,非徒繼體守文而已也。
 
 
 
 
 臣常疾世主莫不思紹堯、舜、湯、武之治,而蹈踵桀、紂、幽、厲之跡,莫不蚩笑季世惑亂亡國之主,而不登踐虞、夏、殷、周之軌。悲夫!以若所為,求若所致,猶緣木求魚,煎水作冰,其不可得,明矣。尋觀三代之有天下也,聖賢相承,歷載數百,尺土莫非其有,一民莫非其臣,萬國咸寧,九有有截;鹿臺之金,巨橋之粟,無所用之,仍舊南面,夫何為哉!然癸、辛之徒,恃其旅力,知足以拒諫,才足以飾非,諂諛是尚,臺觀是崇,淫樂是好,倡優是說,作靡靡之樂,安濮上之音。上天不蠲,眷然回顧,宗國為墟,下夷于隷,紂縣白旗,桀放鳴條;天子之尊,湯、武有之,豈伊異人,皆明王之冑也。且當六國之時,天下殷熾,秦旣兼之,不脩聖道,乃構阿房之宮,築長城之守,矜夸中國,威服百蠻,天下震竦,道路以目;自謂本枝百葉,永垂洪暉,豈寤二世而滅,社稷崩圮哉?近漢孝武乘文、景之福,外攘夷狄,內興宮殿,十餘年間,天下嚻然。乃信越巫,懟天遷怒,起建章之宮,千門萬戶,卒致江充妖蠱之變,至於宮室乖離,父子相殘,殃咎之毒,禍流數世。
 
 
 
 
 臣觀黃初之際,天兆其戒,異類之鳥,育長燕巢,口爪胷赤,此魏室之大異也,宜防鷹揚之臣於蕭牆之內。可選諸王,使君國典兵,往往棊跱,鎮撫皇畿,翼亮帝室。昔周之東遷,晉、鄭是依,漢呂之亂,實賴朱虛,斯蓋前代之明鑒。夫皇天無親,惟德是輔。民詠德政,則延期過歷,下有怨歎,掇錄授能。由此觀之,天下之天下,非獨陛下之天下也。臣百疾所鍾,氣力稍微,輒自輿出,歸還里舍,若遂沈淪,魂而有知,結草以報。
 
 
詔曰:「生廉侔伯夷,直過史魚,執心堅白,謇謇匪躬,如何微疾未除,退身里舍?昔邴吉以陰德,疾除而延壽;貢禹以守節,疾篤而濟愈。生其彊飯專精以自持。」隆卒,遺令薄葬,歛以時服。
 \gezhu{習鑿齒曰:高堂隆可謂忠臣矣。君侈每思諫其惡,將死不忘憂社稷,正辭動於昏主,明戒驗於身後,謇諤足以勵物,德音沒而弥彰,可不謂忠且智乎!詩云:「聽用我謀,庶無大悔。」又曰:「曾是莫聽,大命以傾。」其高堂隆之謂也。}
 
 
 
 
 初,太和中,中護軍蔣濟上疏曰「宜遵古封禪」。詔曰:「聞濟斯言,使吾汗出流足。」事寢歷歲,後遂議脩之,使隆撰其禮儀。帝聞隆沒,歎息曰:「天不欲成吾事,高堂生舍我亡也。」子琛嗣爵。
 
 
 
 
 始,景初中,帝以蘇林、秦靜等並老,恐無能傳業者。乃詔曰:「昔先聖旣沒,而其遺言餘教,著於六藝。六藝之文,禮又為急,弗可斯須離者也。末俗背本,所由來乆。故閔子譏原伯之不學,荀卿醜秦世之坑儒,儒學旣廢,則風化曷由興哉?方今宿生巨儒,並各年高,教訓之道,孰為其繼?昔伏生將老,漢文帝嗣以鼂錯;穀梁寡疇,宣帝承以十郎。其科郎吏高才解經義者三十人,從光祿勳隆、散騎常侍林、博士靜,分受四經三禮,主者具為設課試之法。夏侯勝有言:『士病不明經術,經術苟明,其取青紫如俯拾地芥耳。』今學者有能究極經道,則爵祿榮寵,不期而至。可不勉哉!」數年,隆等皆卒,學者遂廢。
 
 
初,任城棧潛,太祖世歷縣令,
 \gezhu{潛字彥皇,見應璩書林。}
 嘗督守鄴城。時文帝為太子,耽樂田獵,晨出夜還。潛諫曰:「王公設險以固其國,都城禁衞,用戒不虞。大雅云:『宗子維城,無俾城壞。』又曰:『猶之未遠,是用大簡。』若逸于遊田,晨出昏歸,以一日從禽之娛,而忘無垠之釁,愚竊惑之。」太子不恱,然自後游出差簡。黃初中,文帝將立郭貴嬪為皇后,潛上疏諫,語在后妃傳。明帝時,衆役並興,戚屬疏斥,潛上疏曰:「天生蒸民而樹之君,所以覆燾羣生,熈育兆庶,故方制四海匪為天子,裂土分疆匪為諸侯也。始自三皇,爰曁唐、虞,咸以愽濟加于天下,醇德以洽,黎元賴之。三王旣微,降逮于漢,治日益少,喪亂弘多,自時厥後,亦罔克乂。太祖濬哲神武,芟除暴亂,克復王綱,以開帝業。文帝受天明命,廓恢皇基,踐阼七載,每事未遑。陛下聖德,纂承洪緒,宜崇晏晏,與民休息。而方隅匪寧,征夫遠戍,有事海外,縣旌萬里,六軍騷動,水陸轉運,百姓舍業,日費千金。大興殿舍,功作萬計,徂來之松,刊山窮谷,怪石珷玞,浮于河、淮,都圻之內,盡為甸服,當供稾秸銍粟之調,而為苑囿擇禽之府,盛林莽之穢,豐鹿兎之藪;傷害農功,地繁茨棘,災疫流行,民物大潰,上減和氣,嘉禾不植。臣聞文王作豐,經始勿亟,百姓子來,不日而成。靈沼、靈囿,與民共之。今宮觀崇侈,彫鏤極妙,忘有虞之總期,思殷辛之瓊室,禁地千里,舉足投網,麗擬阿房,役百乾谿,臣恐民力彫盡,下不堪命也。昔秦據殽函以制六合,自以德高三皇,功兼五帝,欲號謚至萬葉,而二世顛覆,願為黔首,由枝幹旣杌,本實先拔也。蓋聖王之御世也,克明俊德,庸勳親親;俊乂在官,則功業可隆,親親顯用,則安危同憂;深根固本,並為幹翼,雖歷盛衰,內外有輔。昔成王幼沖,未能莅政,周、呂、召、畢,並在左右;今旣無衞侯、康叔之監,分陝所任,又非旦、奭。東宮未建,天下無副。願陛下留心關塞,永保無極,則海內幸甚。」後為燕中尉,辭疾不就,卒。
 
 
 
 
 評曰:辛毗、楊阜,剛亮公直,正諫匪躬,亞乎汲黯之高風焉。高堂隆學業脩明,志在匡君,因變陳戒,發於懇誠,忠矣哉!及至必改正朔,俾魏祖虞,所謂意過其通者歟!
 
 
\end{pinyinscope}