\article{孫桓傳}
\begin{pinyinscope}
 
 
 孫桓字叔武,河之子也。
 
 
\gezhu{吳書曰:河有四子。長助,曲阿長。次誼,海鹽長。並早卒。次桓,儀容端正,器懷聦朗,博學彊記,能論議應對,權常稱為宗室顏淵,擢為武衞都尉。從討關羽於華容,誘羽餘黨,得五千人,牛馬器械甚衆。}
 年二十五,拜安東中郎將,與陸遜共拒劉備。備軍衆甚盛,彌山盈谷,桓投刀奮命,與遜勠力,備遂敗走。桓斬上兜道,截其徑要。備踰山越險,僅乃得免,忿恚歎曰:「吾昔初至京城,桓尚小兒,而今迫孤乃至此也!」桓以功拜建武將軍,封丹徒侯,下督牛渚,作橫江塢,會卒。
 \gezhu{吳書曰:桓弟俊,字叔英,性度恢弘,才經文武,為定武中郎將,屯戍薄落,赤烏十三年卒。長子建襲爵,平虜將軍。少子慎,鎮南將軍。慎子丞,字顯世。文士傳曰:丞好學,有文章,作螢火賦行於世。為黃門侍郎,與顧榮俱為侍臣。歸命世內侍多得罪尤,惟榮、丞獨獲全。常使二人記事,丞荅顧問,乃下詔曰:「自今已後,用侍郎皆當如今宗室丞、顧榮疇也。」吳平赴洛,為范陽涿令,甚有稱績。永安中,陸機為成都王大都督,請丞為司馬,與機俱被害。}
 
 
 
 
 評曰:夫親親恩義,古今之常。宗子維城,詩人所稱。況此諸孫,或贊興初基,或鎮據邊垂,克堪厥任,不忝其榮者乎!故詳著云。
 
 
\end{pinyinscope}