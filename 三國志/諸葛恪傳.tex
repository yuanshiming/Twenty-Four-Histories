\article{諸葛恪傳}
\begin{pinyinscope}
 
 
 諸葛恪字元遜,瑾長子也。少知名。
 
 
\gezhu{江表傳曰:恪少有才名,發藻岐嶷,辯論應機,莫與為對。權見而奇之,謂瑾曰:「藍田生玉,真不虛也。」吳錄曰:恪長七尺六寸,少鬚眉,折頞廣額,大口高聲。}
 弱冠拜騎都尉,與顧譚、張休等侍太子登講論道藝,並為賔友。從中庶子轉為左輔都尉。
 
 
恪父瑾靣長似驢,孫權大會羣臣,使人牽一驢入,長檢其靣,題曰諸葛子瑜。恪跪曰:「乞請筆益兩字。」因聽與筆。恪續其下曰「之驢。」舉坐歡笑,乃以驢賜恪。他日復見,權問恪曰:「卿父與叔父孰賢?」對曰:「臣父為優。」權問其故,對曰:「臣父知所事,叔父不知,以是為優。」權又大噱。命恪行酒,至張昭前,昭先有酒色,不肯飲,曰:「此非養老之禮也。」權曰:「卿其能令張公辭屈,乃當飲之耳。」恪難昭曰:「昔師尚父九十,秉旄仗鉞,猶未告老也。今軍旅之事,將軍在後,酒食之事,將軍在先,何謂不養老也?」昭卒無辭,遂為盡爵。後蜀使至,群臣並會,權謂使曰:「此諸葛恪雅好騎乘,還告丞相,為致好馬。」恪因下謝,權曰:「馬未至而謝何也?」恪對曰:「夫蜀者陛下之外廄,今有恩詔,馬必至也,安敢不謝?」恪之才捷,皆此類也。
 \gezhu{恪別傳曰:權甞饗蜀使費禕,先逆勑羣臣:「使至,伏食勿起。」禕至,權為輟食,而羣下不起。禕啁之曰:「鳳皇來翔,騏驎吐哺,驢騾無知,伏食如故。」恪荅曰:「爰植梧桐,以待鳳皇,有何燕雀,自稱來翔?何不彈射,使還故鄉!」禕停食餅,索筆作麥賦,恪亦請筆作磨賦,咸稱善焉。權甞問恪:「頃何以自娛,而更肥澤?」恪對曰:「臣聞富潤屋,德潤身,臣非敢自娛,脩己而已。」又問:「卿何如滕胤?」恪荅曰:「登階躡履,臣不如胤;迴籌轉策,胤不如臣。」恪甞獻權馬,先𨪕其耳。范慎時在坐,嘲恪曰:「馬雖大畜,稟氣於天,今殘其耳,豈不傷仁?」恪荅曰:「母之於女,恩愛至矣,穿耳附珠,何傷於仁?」太子甞嘲恪:「諸葛元遜可食馬矢。」恪曰:「願太子食雞卵。」權曰:「人令卿食馬矢,卿使人食雞卵何也?」恪曰:「所出同耳。」權大笑。江表傳曰:曾有白頭鳥集殿前,權曰:「此何鳥也?」恪曰:「白頭翁也。」張昭自以坐中最老,疑恪以鳥戲之,因曰:「恪欺陛下,未甞聞鳥名白頭翁者,試使恪復求白頭母。」恪曰:「鳥名鸚母,未必有對,試使輔吳復求鸚父。」昭不能荅,坐中皆歡笑。}
 權甚異之,欲試以事,令守節度。節度掌軍糧穀,文書繁猥,非其好也。
 \gezhu{江表傳曰:權為吳王,初置節度官,使典掌軍糧,非漢制也。初用侍中偏將軍徐詳,詳死,將用恪。諸葛亮聞恪代詳,書與陸遜曰:「家兄年老,而恪性踈,今使典主糧穀,糧穀軍之要最,僕雖在遠,竊用不安。足下特為啟至尊轉之。」遜以白權,即轉恪領兵。}
 
 
 
 
 恪以丹楊山險,民多果勁,雖前發兵,徒得外縣平民而已,其餘深遠,莫能禽盡,屢自求乞為官出之,三年可得甲士四萬。衆議咸以丹楊地勢險阻,與吳郡、會稽、新都、鄱陽四郡鄰接,周旋數千里,山谷萬重,其幽邃民人,未甞入城邑,對長吏,皆仗兵野逸,白首於林莽。逋亡宿惡,咸共逃竄。山出銅鐵,自鑄甲兵。俗好武習戰,高尚氣力,其升山赴險,抵突叢棘,若魚之走淵,猨狖之騰木也。時觀閒隙,出為寇盜,每致兵征伐,尋其窟藏。其戰則蠭至,敗則鳥竄,自前世以來,不能羈也。皆以為難。恪父瑾聞之,亦以事終不逮,歎曰:「恪不大興吾家,將大赤吾族也。」恪盛陳其必捷。權拜恪撫越將軍,領丹楊太守,授棨戟武騎三百。拜畢,命恪備威儀,作鼓吹,導引歸家,時年三十二。
 
 
 
 
 恪到府,乃移書四郡屬城長吏,令各保其疆界,明立部伍,其從化平民,悉令屯居。乃分內諸將,羅兵幽阻,但繕藩籬,不與交鋒,候其穀稼將熟,輒縱兵芟刈,使無遺種。舊穀旣盡,新田不収,平民屯居,略無所入,於是山民饑窮,漸出降首。恪乃復勑下曰:「山民去惡從化,皆當撫慰,徙出外縣,不得嫌疑,有所執拘。」臼陽長胡伉得降民周遺,遺舊惡民,困迫暫出,內圖叛逆,伉縛送諸府。恪以伉違教,遂斬以徇,以狀表上。民聞伉坐執人被戮,知官惟欲出之而已,於是老幼相攜而出,歲期,人數皆如本規。恪自領萬人,餘分給諸將。
 
 
 
 
 權嘉其功,遣尚書僕射薛綜勞軍。綜先移恪等曰:「山越恃阻,不賔歷世,緩則首鼠,急則狼顧。皇帝赫然,命將西征,神策內授,武師外震。兵不染鍔,甲不沾汗。元惡旣梟,種黨歸義,蕩滌山藪,獻戎十萬。野無遺寇,邑罔殘姦。旣埽兇慝,又充軍用。藜蓧稂莠,化為善草。魑魅魍魎,更成虎士。雖實國家威靈之所加,亦信元帥臨履之所致也。雖詩羙執訊,易嘉折首,周之方、召,漢之衞、霍,豈足以談?功軼古人,勳超前世。主上歡然,遙用歎息。感四牡之遺典,思飲至之舊章。故遣中臺近官,迎致犒賜,以旌茂功,以慰劬勞。」拜恪威北將軍,封都鄉侯。恪乞率衆佃廬江、皖口,因輕兵襲舒,掩得其民而還。復遠遣斥候,觀相徑要,欲圖壽春,權以為不可。
 
 
 
 
 赤烏中,魏司馬宣王謀欲攻恪,權方發兵應之,望氣者以為不利,於是徙恪屯於柴桑。與丞相陸遜書曰:「楊敬叔傳述清論,以為方今人物彫盡,守德業者不能復幾,宜相左右,更為輔車,上熙國事,下相珍惜。又疾世俗好相謗毀,使已成之器,中有損累;將進之徒,意不歡笑。聞此喟然,誠獨擊節。愚以為君子不求備於一人,自孔氏門徒大數三千,見其異者七十二人,至于子張、子路、子貢等七十之徒,亞聖之德,然猶各有所短,師辟由喭,賜不受命,豈况下此而無所闕?且仲尼不以數子之不備而引以為友,不以人所短棄其所長也。加以當今取士,宜寬於徃古,何者?時務從橫,而善人單少,國家職司,常苦不充。苟令性不邪惡,志在陳力,便可獎就,騁其所任。若於小小宜適,私行不足,皆宜闊略,不足縷責。且士誠不可纖論苛克,苛克則彼賢聖猶將不全,況其出入者邪?故曰以道望人則難,以人望人則易,賢愚可知。自漢末以來,中國士大夫如許子將輩,所以更相謗訕,或至為禍,原其本起,非為大讎,惟坐克己不能盡如禮,而責人專以正義。夫己不如禮,則人不服。責人以正義,則人不堪。內不服其行,外不堪其責,則不得不相怨。相怨一生,則小人得容其閒。得容其閒,則三至之言,浸潤之譖,紛錯交至,雖使至明至親者處之,猶難以自定,况己為隙,且未能明者乎?是故張、陳至於血刃,蕭、朱不終其好,本由於此而已。夫不捨小過,纖微相責,乆乆至於家戶為怨,一國無復全行之士也。」恪知遜以此嫌己,故遂廣其理而贊其旨也。會遜卒,恪遷大將軍,假節,駐武昌,代遜領荊州事。
 
 
乆之,權不豫,而太子少,乃徵恪以大將軍領太子太傅,中書令孫弘領少傅。權疾困,召恪、弘及太常滕胤、將軍呂據、侍中孫峻,屬以後事。
 \gezhu{吳書曰:權寢疾,議所付託。時朝臣咸皆注意於恪,而孫峻表恪器任輔政,可付大事。權嫌恪剛很自用,峻以當今朝臣皆莫及,遂固保之,乃徵恪。後引恪等見卧內,受詔牀下,權詔曰:「吾病困矣,恐不復相見,諸事一以相委。」恪歔欷流涕曰:「臣等皆受厚恩,當以死奉詔,願陛下安精神,損思慮,無以外事為念。」權詔有司諸事一統於恪,惟殺生大事然後以聞。為治第館,設陪衞。羣官百司拜揖之儀,各有品叙。諸法令有不便者,條列以聞,權輒聽之。中外翕然,人懷歡欣。}
 
 
 
 
 翌日,權薨。弘素與恪不平,懼為恪所治,祕權死問,欲矯詔除恪。峻以告恪,恪請弘咨事,於坐中誅之,乃發喪制服。與弟公安督融書曰:「今月十六日乙未,大行皇帝委棄萬國,羣下大小,莫不傷悼。至吾父子兄弟,並受殊恩,非徒凡庸之隷,是以悲慟,肝心圮裂。皇太子以丁酉踐尊號,哀喜交并,不知所措。吾身受顧命,輔相幼主,竊自揆度,才非博陸而受姬公負圖之託,懼忝丞相輔漢之效,恐損先帝委付之明,是以憂慙惶惶,所慮萬端。且民惡其上,動見瞻觀,何時易哉?今以頑鈍之姿,處保傅之位,艱多智寡,任重謀淺,誰為脣齒?近漢之世,燕、蓋交遘,有上官之變,以身值此,何敢怡豫邪?又弟所在,與賊犬牙相錯,當於今時整頓軍具,率厲將士,警備過常,念出萬死,無顧一生,以報朝廷,無忝爾先。又諸將備守各有境界,猶恐賊虜聞諱,恣睢寇竊。邊邑諸曹,已別下約勑,所部督將,不得妄委所戍,徑來奔赴。雖懷愴怛不忍之心,公義奪私,伯禽服戎,若苟違戾,非徒小故。以親正疏,古今明戒也。」恪更拜太傅。於是罷視聽,息校官,原逋責,除關稅,事崇恩澤,衆莫不恱。恪每出入,百姓延頸,思見其狀。
 
 
 
 
 初,權黃龍元年遷都建業,二年築東興隄遏湖水。後征淮南,敗以內船,由是廢不復脩。恪以建興元年十月會衆於東興,更作大隄,左右結山俠築兩城,各留千人,使全端、留略守之,引軍而還。魏以吳軍入其疆土,恥於受侮,命大將胡遵、諸葛誕等率衆七萬,欲攻圍兩塢,圖壞隄遏。恪興軍四萬,晨夜赴救。遵等勑其諸軍作浮橋度,陣於隄上,分兵攻兩城。城在高峻,不可卒拔。恪遣將軍留贊、呂據、唐咨、丁奉為前部。時天寒雪,魏諸將會飲,見贊等兵少,而解置鎧甲,不持矛戟,但兜鍪刀楯,倮身緣遏,大笑之,不即嚴兵。兵得上,便鼓譟亂斫。魏軍驚擾散走,爭渡浮橋,橋壞絕,自投於水,更相蹈藉。樂安太守桓嘉等同時并沒,死者數萬。故叛將韓綜為魏前軍督,亦斬之。獲車乘牛馬驢騾各數千,資器山積,振旅而歸。進封恪陽都侯,加荊楊州牧,督中外諸軍事,賜金一百斤,馬二百匹,繒布各萬匹。
 
 
恪遂有輕敵之心,以十二月戰克,明年春,復欲出軍。
 \gezhu{漢晉春秋曰:恪使司馬季無往蜀說姜維,令同舉,曰:「古人有言,聖人不能為時,時至亦不可失也。今敵政在私門,外內猜隔,兵挫於外,而民怨於內,自曹操以來,彼之亡形未有如今者也。若大舉伐之,使吳攻其東,漢入其西,彼救西則東虛,重東則西輕,以練實之軍,乘虛輕之敵,破之必矣。」維從之。}
 諸大臣以為數出罷勞,同辭諫恪,恪不聽。中散大夫蔣延或以固爭,扶出。
 
 
 
 
 恪乃著論諭衆意曰:「夫天無二日,土無二王,王者不務兼并天下而欲垂祚後世,古今未之有也。昔戰國之時,諸侯自恃兵彊地廣,互有救援,謂此足以傳世,人莫能危。恣情從懷,憚於勞苦,使秦漸得自大,遂以并之,此旣然矣。近者劉景升在荊州,有衆十萬,財穀如山,不及曹操尚微,與之力競,坐觀其彊大,吞滅諸袁。北方都定之後,操率三十萬衆來向荊州,當時雖有智者,不能復為畫計,於是景升兒子交臂請降,遂為囚虜。凡敵國欲相吞,即仇讎欲相除也。有讎而長之,禍不在己,則在後人,不可不為遠慮也。昔伍子胥曰:『越十年生聚,十年教訓,二十年之外,吳其為沼乎!』夫差自恃彊大,聞此邈然,是以誅子胥而無備越之心,至於臨敗悔之,豈有及乎?越小於吳,尚為吳禍,況其彊大者邪?昔秦但得關西耳,尚以并吞六國,今賊皆得秦、趙、韓、魏、燕、齊九州之地,地悉戎馬之鄉,士林之藪。今以魏比古之秦,土地數倍;以吳與蜀比古六國,不能半之。然今所以能敵之,但以操時兵衆,於今適盡,而後生者未悉長大,正是賊衰少未盛之時。加司馬懿先誅王淩,續自隕斃,其子幼弱,而專彼大任,雖有智計之士,未得施用。當今伐之,是其厄會。聖人急於趨時,誠謂今日。若順衆人之情,懷偷安之計,以為長江之險可以傳世,不論魏之終始,而以今日遂輕其後,此吾所以長歎息者也。自古以來,務在產育,今者賊民歲月繁滋,但以尚小,未可得用耳。若復十數年後,其衆必倍於今,而國家勁兵之地,皆已空盡,唯有此見衆可以定事。若不早用之,端坐使老,復十數年,略當損半,而見子弟數不足言。若賊衆一倍,而我兵損半,雖復使伊、管圖之,未可如何。今不達遠慮者,必以此言為迂。夫禍難未至而豫憂慮,此固衆人之所迂也。及於難至,然後頓顙,雖有智者,又不能圖。此乃古今所病,非獨一時。昔吳始以伍員為迂,故難至而不可救。劉景升不能慮十年之後,故無以詒其子孫。今恪無具臣之才,而受大吳蕭、霍之任,智與衆同,思不經遠,若不及今日為國斥境,俛仰年老,而讎敵更彊,欲刎頸謝責,寧有補邪?今聞衆人或以百姓尚貧,欲務閑息,此不知慮其大危,而愛其小勤者也。昔漢祖幸已自有三秦之地,何不閉關守險,以自娛樂,空出攻楚,身被創痍,介冑生蟣蝨,將士厭困苦,豈甘鋒刃而忘安寧哉?慮於長乆不得兩存者耳!每覽荊邯說公孫述以進取之圖,近見家叔父表陳與賊爭競之計,未甞不喟然歎息也。夙夜反側,所慮如此,故聊疏愚言,以達二三君子之末。若一朝隕歿,志畫不立,貴令來世知我所憂,可思於後。」衆皆以恪此論欲必為之辭,然莫敢復難。
 
 
 
 
 丹楊太守聶友素與恪善,書諫恪曰:「大行皇帝本有遏東關之計,計未施行。今公輔贊大業,成先帝之志,寇遠自送,將士憑賴威德,出身用命,一旦有非常之功,豈非宗廟神靈社稷之福邪!宜且案兵養銳,觀釁而動。今乘此勢,欲復大出,天時未可。而苟任盛意,私心以為不安。」恪題論後,為書荅友曰:「足下雖有自然之理,然未見大數。熟省此論,可以開悟矣。」於是違衆出軍,大發州郡二十萬衆,百姓騷動,始失人心。
 
 
 
 
 恪意欲曜威淮南,驅略民人,而諸將或難之曰:「今引軍深入,疆埸之民,必相率遠遁,恐兵勞而功少,不如止圍新城。新城困,救必至,至而圖之,乃可大獲。」恪從其計,迴軍還圍新城。攻守連月,城不拔。士卒疲勞,因暑飲水,泄下流腫,病者大半,死傷塗地。諸營吏日白病者多,恪以為詐,欲斬之,自是莫敢言。恪內惟失計,而恥城不下,忿形於色。將軍朱異有所是非,恪怒,立奪其兵。都尉蔡林數陳軍計,恪不能用,策馬奔魏。魏知戰士罷病,乃進救兵。恪引軍而去。士卒傷病,流曳道路,或頓仆坑壑,或見略獲,存亡忿痛,大小呼嗟。而恪晏然自若。出住江渚一月,圖起田於潯陽,詔召相銜,徐乃旋師。由此衆庶失望,而怨黷興矣。
 
 
 
 
 秋八月軍還,陳兵導從,歸入府館。即召中書令孫嘿,厲聲謂曰:「卿等何敢妄數作詔?」嘿惶懼辭出,因病還家。恪征行之後,曹所奏署令長職司,一罷更選,愈治威嚴,多所罪責,當進見者,無不竦息。又改易宿衞,用其親近,復勑兵嚴,欲向青、徐。
 
 
 
 
 孫峻因民之多怨,衆之所嫌,構恪欲為變,與亮謀,置酒請恪。恪將見之夜,精爽擾動,通夕不寐。明將盥漱,聞水腥臭,侍者授衣,衣服亦臭。恪怪其故,易衣易水,其臭如初,意惆悵不恱。嚴畢趨出,犬銜引其衣,恪曰:「犬不欲我行乎?」還坐,頃刻乃復起,犬又銜其衣,恪令從者逐犬,遂升車。
 
 
 
 
 初,恪將征淮南,有孝子著縗衣入其閤中,從者白之,令外詰問,孝子曰:「不自覺入。」時中外守備亦悉不見,衆皆異之。出行之後,所坐廳事屋棟中折。自新城出住東興,有白虹見其船,還拜蔣陵,白虹復繞其車。
 
 
及將見,駐車宮門,峻已伏兵於帷中,恐恪不時入,事泄,自出見恪曰:「使君若尊體不安,自可須後,峻當具白主上。」欲以甞知恪。恪荅曰:「當自力入。」散騎常侍張約、朱恩等密書與恪曰:「今日張設非常,疑有他故。」恪省書而去。未出路門,逢太常滕胤,恪曰:「卒腹痛,不任入。」胤不知峻陰計,謂恪曰:「君自行旋未見,今上置酒請君,君已至門,宜當力進。」恪躊躇而還,劒履上殿,謝亮,還坐。設酒,恪疑未飲,峻因曰:「使君病未善平,當有常服藥酒,自可取之。」恪意乃安,別飲所齎酒。
 \gezhu{吳歷曰:張約、朱恩密疏告恪,恪以示滕胤,胤勸恪還,恪曰:「峻小子何能為邪!但恐因酒食中人耳。」乃以藥酒入。孫盛評曰:恪與胤親厚,約等疏,非常大事,勢應示胤,共謀安危。然恪性彊梁,加素侮峻,自不信,故入,豈胤微勸,便為之冒禍乎?吳歷為長。}
 酒數行,亮還內。峻起如廁,解長衣,著短服,出曰:「有詔收諸葛恪!」
 \gezhu{吳錄曰:峻持刀稱詔收恪,亮起立曰:「非我所為!非我所為!」乳母引亮還內。吳歷云:峻先引亮入,然後出稱詔。與本傳同。臣松之以為峻欲稱詔,宜如本傳及吳歷,不得如吳錄所言。}
 恪驚起,拔劒未得,而峻刀交下。張約從旁斫峻,裁傷左手,峻應手斫約,斷右臂。武衞之士皆趨上殿,峻曰:「所取者恪也,今已死。」悉令復刃,乃除地更飲。
 \gezhu{搜神記曰:恪入,已被殺,其妻在室使婢,語曰:「汝何故血臰?」婢曰:「不也。」有頃愈劇,又問婢曰:「汝眼目視瞻,何以不常?」婢蹷然起躍,頭至于棟,攘臂切齒而言曰:「諸葛公乃為孫峻所殺!」於是大小知恪死矣,而吏兵尋至。志林曰:初權病篤,召恪輔政。臨去,大司馬呂岱戒之曰:「世方多難,子每事必十思。」恪荅曰:「昔季文子三思而後行,夫子曰『再思可矣』,今君令恪十思,明恪之劣也。」岱無以荅,當時咸謂之失言。虞喜曰:夫託以天下至重也,以人臣行主威至難也,兼二至而管萬機,能勝之者鮮矣。自非採納羣謀,詢于芻蕘,虛己受人,常若不足,則功名不成,勳績莫著。況呂侯國之先耆,智度經遠,而甫以十思戒之,而便以示劣見拒,此元遜之踈,乃機神不俱者也。若因十思之義,廣諮當世之務,聞善速於雷動,從諫急於風移,豈得隕首殿堂,死凶豎之刃?世人奇其英辯,造次可觀,而哂呂侯無對為陋,不思安危終始之慮,是樂春藻之繁華,而忘秋實之甘口也。昔魏人伐蜀,蜀人禦之,精嚴垂發,六軍雲擾,士馬擐甲,羽檄交馳,費禕時為元帥,荷國任重,而與來敏圍碁,意無厭倦。敏臨別謂禕:「君必能辦賊者也。」言其明略內定,貌無憂色,況長寧以為君子臨事而懼,好謀而成者。且蜀為蕞爾之國,而方向大敵,所規所圖,唯守與戰,何可矜己有餘,晏然無戚?斯乃性之寬簡,不防細微,卒為降人郭脩所害,豈非兆見於彼而禍成於此哉?往聞長寧之甄文偉,今覩元遜之逆呂侯,二事體同,故並而載之,可以鏡譏于後,永為世鑒。}
 
 
先是,童謠曰:「諸葛恪,蘆葦單衣篾鉤落,於何相求成子閣。」成子閣者,反語石子岡也。建業南有長陵,名曰石子岡,葬者依焉。鉤落者,校飾革帶,世謂之鉤絡帶。恪果以葦席裹其身而篾束其腰,投之於此岡。
 \gezhu{吳錄曰:恪時年五十一。}
 
 
 
 
 恪長子綽,騎都尉,以交關魯王事,權遣付恪,令更教誨,恪鴆殺之。中子竦,長水校尉。少子建,步兵校尉。聞恪誅,車載其母而走。峻遣騎督劉承追斬竦於白都。建得渡江,欲北走魏,行數十里,為追兵所逮。恪外甥都鄉侯張震及常侍朱恩等,皆夷三族。
 
 
初,竦數諫恪,恪不從,常憂懼禍。及亡,臨淮臧均表乞收葬恪曰:「臣聞震雷電激,不崇一朝,大風衝發,希有極日,然猶繼以雲雨,因以潤物,是則天地之威,不可經日浹辰,帝王之怒,不宜訖情盡意。臣以狂愚,不知忌諱,敢冒破滅之罪,以邀風雨之會。伏念故太傅諸葛恪得承祖考風流之烈,伯叔諸父遭漢祚盡,九州鼎立,分託三方,並履忠勤,熈隆世業。爰及於恪,生長王國,陶育聖化,致名英偉,服事累紀,禍心未萌,先帝委以伊、周之任,屬以萬機之事。恪素性剛愎,矜己陵人,不能敬守神器,穆靜邦內,興功暴師,未期三出,虛耗士民,空竭府藏,專擅國憲,廢易由意,假刑劫衆,大小屏息。侍中武衞將軍都鄉侯俱受先帝囑寄之詔,見其姦虐,日月滋甚,將恐蕩搖宇宙,傾危社稷,奮其威怒,精貫昊天,計慮先於神明,智勇百於荊、聶,躬持白刃,梟恪殿堂,勳超朱虛,功越東牟。國之元害,一朝大除,馳首徇示,六軍喜踊,日月增光,風塵不動,斯實宗廟之神靈,天人之同驗也。今恪父子三首,縣市積日,觀者數萬,詈聲成風。國之大刑,無所不震,長老孩幼,無不畢見。人情之於品物,樂極則哀生,見恪貴盛,世莫與貳,身處台輔,中間歷年,今之誅夷,無異禽獸,觀訖情反,能不憯然!且已死之人,與土壤同域,鑿掘斫刺,無所復加。願聖朝稽則乾坤,怒不極旬,使其鄉邑若故吏民,收以士伍之服,惠以三寸之棺。昔項籍受殯葬之地,韓信獲收斂之恩,斯則漢高發神明之譽也。惟陛下敦三皇之仁,垂哀矜之心,使國澤加於辜戮之骸,復受不已之恩,於以揚聲遐方,沮勸天下,豈不弘哉!昔欒布矯命彭越,臣竊恨之,不先請主上,而專名以肆情,其得不誅,實為幸耳。今臣不敢章宣愚情,以露天恩,謹伏手書,冒昧陳聞,乞聖朝哀察。」於是亮、峻聽恪故吏斂葬,遂求之於石子岡。
 \gezhu{江表傳曰:朝臣有乞為恪立碑以銘其勳績者,博士盛沖以為不應。孫休曰:「盛夏出軍,士卒傷損,無尺寸之功,不可謂能;受託孤之任,死於豎子之手,不可謂智。沖議為是。」遂寢。}
 
 
始恪退軍還,聶友知其將敗,書與滕胤曰:「當人彊盛,河山可拔,一朝羸縮,人情萬端,言之悲歎。」恪誅後,孫峻忌友,欲以為鬱林太守,友發病憂死。友字文悌,豫章人也。
 \gezhu{吳錄曰:友有脣吻,少為縣吏。虞翻徙交州,縣令使友送之,翻與語而奇焉,為書與豫章太守謝斐,令以為功曹。郡時見有功曹,斐見之,問曰:「縣吏聶友,可堪何職?」對曰:「此人縣間小吏耳,猶可堪曹佐。」斐曰:「論者以為宜作功曹,君其避之。」乃用為功曹。使至都,諸葛恪友之。時論謂顧子嘿、子直,其間無所復容,恪欲以友居其間,由是知名。後為將,討儋耳,還拜丹楊太守,年三十三卒。}
 
 
\end{pinyinscope}