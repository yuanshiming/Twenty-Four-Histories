\article{楊俊傳}
\begin{pinyinscope}
 
 
 楊俊字季才,河內獲嘉人也。受學陳留邊讓,讓器異之。俊以兵亂方起,而河內處四達之衢,必為戰場,乃扶持老弱詣京、密山間,同行者百餘家。俊振濟貧乏,通共有無。宗族知故為人所略作奴僕者凡六家,俊皆傾財贖之。司馬宣王年十六七,與俊相遇,俊曰:「此非常之人也。」又司馬朗早有聲名,其族兄芝,衆未之知,惟俊言曰:「芝雖風望不及朗,實理但有優耳。」俊轉避地并州。本郡王象,少孤特,為人僕隷,年十七八,見使牧羊而私讀書,因被箠楚。俊嘉其才質,即贖象著家,聘娶立屋,然後與別。
 
 
 
 
 太祖除俊曲梁長,入為丞相掾屬,舉茂才,安陵令,遷南陽太守。宣德教,立學校,吏民稱之。徙為征南軍師。魏國旣建,遷中尉。太祖征漢中,魏諷反於鄴。俊自劾詣行在所。俊以身方罪免,牋辭太子。太子不恱,曰:「楊中尉便去,何太高遠邪!」遂被書左遷平原太守。文帝踐阼,復在南陽。時王象為散騎常侍,薦俊曰:「伏見南陽太守楊俊,秉純粹之茂質,履忠肅之弘量,體仁足以育物,篤實足以動衆,克長後進,惠訓不倦,外寬內直,仁而有斷。自初彈冠,所歷垂化,再守南陽,恩德流著,殊鄰異黨,襁負而至。今境守清靜,無所展其智能,宜還本朝,宣力輦轂,熙帝之載。」
 
 
 
 
 俊自少及長,以人倫自任。同郡審固、陳留衞恂本皆出自兵伍,俊資拔獎致,咸作佳士;後固歷位郡守,恂御史、縣令,其明鑒行義多此類也。初,臨菑侯與俊善,太祖適嗣未定,密訪羣司。俊雖並論文帝、臨菑才分所長,不適有所據當,然稱臨菑猶美,文帝常以恨之。黃初三年,車駕至宛,以巿不豐樂,發怒收俊。尚書僕射司馬宣王、常侍王象、荀緯請俊,叩頭流血,帝不許。俊曰:「吾知罪矣。」遂自殺。衆寃痛之。
 
 
\gezhu{世語曰:俊二孫:覽字公質,汝陰太守;猗字公彥,尚書:晉東海王越舅也。覽子沈,字宣弘,散騎常侍。魏略曰:王象字羲伯。旣為俊所知拔,果有才志。建安中,與同郡荀緯等俱為魏太子所禮待。及王粲、陳琳、阮瑀、路粹等亡後,新出之中,惟象才最高。魏有天下,拜象散騎侍郎,遷為常侍,封列侯。受詔撰皇覽,使象領祕書監。象從延康元年始撰集,數歲成,藏於祕府,合四十餘部,部有數十篇,通合八百餘萬字。象旣性器和厚,又文采溫雅,用是京師歸美,稱為儒宗。車駕南巡,未到宛,有詔百官不得干豫郡縣。及車駕到,而宛令不解詔旨,閉巿門。帝聞之,忿然曰:「吾是寇邪?」乃收宛令及太守楊俊。詔問尚書:「漢明帝殺幾二千石?」時象見詔文,知俊必不免。乃當帝前叩頭,流血竟面,請俊減死一等。帝不荅,欲釋入禁中。象引帝衣,帝顧謂象曰:「我知楊俊與卿本末耳。今聽卿,是無我也。卿寧無俊邪?無我邪?」象以帝言切,乃縮手。帝遂入,決俊法,然後乃出。象自恨不能濟俊,遂發病死。}
 
 
\end{pinyinscope}