\article{毛玠傳}
\begin{pinyinscope}
 
 
 毛玠字孝先,陳留平丘人也。少為縣吏,以清公稱。將避亂荊州,未至,聞劉表政令不明,遂往魯陽。太祖臨兖州,辟為治中從事。玠語太祖曰:「今天下分崩,國主遷移,生民廢業,饑饉流亡,公家無經歲之儲,百姓無安固之志,難以持乆。今袁紹、劉表,雖士民衆彊,皆無經遠之慮,未有樹基建本者也。夫兵義者勝,守位以財,宜奉天子以令不臣,脩耕植,畜軍資,如此則霸王之業可成也。」太祖敬納其言,轉幕府功曹。
 
 
 
 
 太祖為司空丞相,玠甞為東曹掾,與崔琰並典選舉。其所舉用,皆清正之士,雖於時有盛名而行不由本者,終莫得進。務以儉率人,由是天下之士莫不以廉節自勵,雖貴寵之臣,輿服不敢過度。太祖歎曰:「用人如此,使天下人自治,吾復何為哉!」文帝為五官將,親自詣玠,屬所親眷。玠荅曰:「老臣以能守職,幸得免戾,今所說人非遷次,是以不敢奉命。」大軍還鄴,議所并省。玠請謁不行,時人憚之,咸欲省東曹。乃共白曰:「舊西曹為上,東曹為次,宜省東曹。」太祖知其情,令曰:「日出於東,月盛於東,凡人言方,亦復先東,何以省東曹?」遂省西曹。初,太祖平柳城,班所獲器物,特以素屏風素馮几賜玠,曰:「君有古人之風,故賜君古人之服。」玠居顯位,常布衣蔬食,撫育孤兄子甚篤,賞賜以振施貧族,家無所餘。遷右軍師。魏國初建,為尚書僕射,復典選舉。
 
 
\gezhu{先賢行狀曰:玠雅亮公正,在官清恪。其典選舉,拔貞實,斥華偽,進遜行,抑阿黨。諸宰官治民功績不著而私財豐足者,皆免黜停廢,久不選用。于時四海翕然,莫不勵行。至乃長吏還者,垢靣羸衣,常乘柴車。軍吏入府,朝服徒行。人擬壺飡之絜,家象濯纓之操,貴者無穢欲之累,賤者絕姦貨之求,吏絜於上,俗移乎下,民到于今稱之。}
 時太子未定,而臨菑侯植有寵,玠密諫曰:「近者袁紹以嫡庶不分,覆宗滅國。廢立大事,非所宜聞。」後羣寮會,玠起更衣,太祖目指曰:「此古所謂國之司直,我之周昌也。」
 
 
崔琰旣死,玠內不恱。後有白玠者:「出見黥靣反者,其妻子沒為官奴婢,玠言曰『使天不雨者蓋此也』。」太祖大怒,收玠付獄。大理鍾繇詰玠曰:「自古聖帝明王,罪及妻子。書云:『左不共左,右不共右,予則孥戮女。』司寇之職,男子入于罪隷,女子入于舂槀。漢律,罪人妻子沒為奴婢,黥靣。漢法所行黥墨之刑,存於古典。今真奴婢祖先有罪,雖歷百世,猶有黥靣供官,一以寬良民之命,二以宥并罪之辜。此何以負於神明之意,而當致旱?案典謀,急恒寒若,舒恒燠若,寬則亢陽,所以為旱。玠之吐言,以為寬邪,以為急也?急當陰霖,何以反旱?成湯聖世,野無生草,周宣令主,旱魃為虐。亢旱以來,積三十年,歸咎黥靣,為相值不?衞人伐邢,師興而雨,罪惡無徵,何以應天?玠譏謗之言,流於下民,不恱之聲,上聞聖聽。玠之吐言,勢不獨語,時見黥靣,凡為幾人?黥靣奴婢,所識知邪?何緣得見,對之歎言?時以語誰?見荅云何?以何日月?於何處所?事已發露,不得隱欺,具以狀對。」玠曰:「臣聞蕭生縊死,困於石顯;賈子放外,讒在絳、灌;白起賜劒於杜郵;晁錯致誅於東市;伍員絕命於吳都:斯數子者,或妬其前,或害其後。臣垂齠執簡,累勤取官,職在機近,人事所竄。屬臣以私,無勢不絕,語臣以冤,無細不理。人情淫利,為法所禁,法禁於利,勢能害之。青蠅橫生,為臣作謗,謗臣之人,勢不在他。昔王叔、陳生爭正王廷,宣子平理。命舉其契,是非有宜,曲直有所,春秋嘉焉,是以書之。臣不言此,無有時、人。說臣此言,必有徵要。乞蒙宣子之辨,而求王叔之對。若臣以曲聞,即刑之日,方之安駟之贈;賜劒之來,比之重賞之惠。謹以狀對。」時桓階、和洽進言救玠。玠遂免黜,卒于家。
 \gezhu{孫盛曰:魏武於是失政刑矣。易稱「明折庶獄」,傳有「舉直措枉」,庶獄明則國無怨民,枉直當則民無不服,未有徵青蠅之浮聲,信浸潤之譖訴,可以允釐四海,惟清緝熙者也。昔者漢高獄蕭何,出復相之,玠之一責,永見擯放,二主度量,豈不殊哉!}
 太祖賜棺器錢帛,拜子機郎中。
 
 
\end{pinyinscope}