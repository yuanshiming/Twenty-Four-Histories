\article{明悼毛皇后}
\begin{pinyinscope}
 
 
 明悼毛皇后,河內人也。黃初中,以選入東宮,明帝時為平原王,進御有寵,出入與同輿輦。及即帝立,以為貴嬪。太和元年,立為皇后。后父嘉,拜騎都尉,后弟曾,郎中。
 
 
 
 
 初,明帝為王,始納河內虞氏為妃,帝即位,虞氏不得立為后,太皇后卞太后慰勉焉。虞氏曰:「曹氏自好立賤,未有能以義舉者也。然后職內事,君聽外政,其道相由而成,苟不能以善始,未能令終者也。殆必由此亡國喪祀矣!」虞氏遂絀還鄴宮。進嘉為奉車都尉,曾騎都尉,寵賜隆渥。頃之,封嘉博平鄉侯,遷光祿大夫,曾駙馬都尉。嘉本典虞車工,卒暴富貴,明帝令朝臣會其家飲宴,其容止舉動甚蚩騃,語輙自謂「侯身」,時人以為笑。
 
 
\gezhu{孫盛曰:古之王者,必求令淑以對揚至德,恢王化於關雎,致淳風於麟趾。及臻三季,並亂茲緒,義以情溺,位由寵昏,貴賤無章,下陵上替,興衰隆廢,皆是物也。魏自武王,曁于烈祖,三后之升,起自幽賤,本旣卑矣,何以長世?詩云:「絺兮綌兮,淒其以風。」其此之謂乎!}
 後又加嘉位特進,曾遷散騎侍郎。青龍三年,嘉薨,追贈光祿大夫,改封安國侯,增邑五百,并前千戶,謚曰節侯。四年,追封后母夏為野王君。
 
 
 
 
 帝之幸郭元后也,后愛寵日㢮。景初元年,帝游後園,召才人以上曲宴極樂。元后曰「宜延皇后」,帝弗許。乃禁左右,使不得宣。后知之,明日,帝見后,后曰:「昨日游宴北園,樂乎?」帝以左右泄之,所殺十餘人。賜后死,然猶加謚,葬愍陵。遷曾散騎常侍,後徙為羽林虎賁中郎將、原武典農。
 
 
\end{pinyinscope}