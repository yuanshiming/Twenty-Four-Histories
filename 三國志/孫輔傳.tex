\article{孫輔傳}
\begin{pinyinscope}
 
 
 孫輔字國儀,賁弟也,以揚武校尉佐孫策平三郡。策討丹楊七縣,使輔西屯歷陽以拒袁術,并招誘餘民,鳩合遺散。又從策討陵陽,生得祖郎等。
 
 
\gezhu{江表傳曰:策旣平定江東,逐袁胤。袁術深怨策,乃陰遣間使齎印綬與丹楊宗帥陵陽祖郎等,使激動山越,大合衆,圖共攻策。策自率將士討郎,生獲之。策謂郎曰:「爾昔襲擊孤,斫孤馬鞌,今創軍立事,除棄宿恨,惟取能用,與天下通耳。非但汝,汝莫恐怖。」郎叩頭謝罪。即破械,賜衣服,署門下賊曹。及軍還,郎與太史慈俱在前導軍,人以為榮。}
 策西襲廬江太守劉勳,輔隨從,身先士卒,有功。策立輔為廬陵太守,撫定屬城,分置長吏。遷平南將軍,假節領交州刺史。遣使與曹公相聞,事覺,權幽繫之。
 \gezhu{典略曰:輔恐權不能保守江東,因權出行東冶,乃遣人齎書呼曹公。行人以告,權乃還,偽若不知,與張昭共見輔,權謂輔曰:「兄厭樂邪,何為呼他人?」輔云無是。權因投書與昭,昭示輔,輔慙無辭。乃悉斬輔親近,分其部曲,徒輔置東。}
 數歲卒。子興、昭、偉、昕,皆歷列位。
 
 
\end{pinyinscope}