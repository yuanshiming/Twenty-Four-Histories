\article{任城威王彰傳}
\begin{pinyinscope}
 
 
 任城威王彰,字子文。少善射御,膂力過人,手格猛獸,不避險阻。數從征伐,志意慷慨。太祖常抑之曰:「汝不念讀書慕聖道,而好乘汗馬擊劒,此一夫之用,何足貴也!」課彰讀詩、書,彰謂左右曰:「丈夫一為衞、霍,將十萬騎馳沙漠,驅戎狄,立功建號耳,何能作博士邪?」太祖嘗問諸子所好,使各言其志。彰曰:「好為將。」太祖曰:「為將柰何?」對曰:「被堅執銳,臨難不顧,為士卒先;賞必行,罰必信。」太祖大笑。建安二十一年,封鄢陵侯。
 
 
 
 
 二十三年,代郡烏丸反,以彰為北中郎將,行驍騎將軍。臨發,太祖戒彰曰:「居家為父子,受事為君臣,動以王法從事,爾其戒之!」彰北征,入涿郡界,叛胡數千騎卒至。時兵馬未集,唯有步卒千人,騎數百匹。用田豫計,固守要隙,虜乃退散。彰追之,身自搏戰,射胡騎,應弦而倒者前後相屬。戰過半日,彰鎧中數箭,意氣益厲,乘勝逐北,至于桑乾,
 
 
\gezhu{臣松之案桑乾縣屬代郡,今北虜居之,號為索干之都。}
 去代二百餘里。長史諸將皆以為新涉遠,士馬疲頓,又受節度,不得過代,不可深進,違令輕敵。彰曰:「率師而行,唯利所在,何節度乎?胡走未遠,追之必破。從令縱敵,非良將也。」遂上馬,令軍中後出者斬。一日一夜與虜相及,擊,大破之,斬首獲生以千數。彰乃倍常科大賜將士,將士無不恱喜。時鮮卑大人軻比能將數萬騎觀望彊弱,見彰力戰,所向皆破,乃請服。北方悉平。時太祖在長安,召彰詣行在所。彰自代過鄴,太子謂彰曰:「卿新有功,今西見上,宜勿自伐,應對常若不足者。」彰到,如太子言,歸功諸將。太祖喜,持彰鬚曰:「黃鬚兒竟大奇也!」
 \gezhu{魏略曰:太祖在漢中,而劉備栖於山頭,使劉封下挑戰。太祖罵曰:「賣履舍兒,長使假子拒汝公乎!待呼我黃鬚來,令擊之。」乃召彰。彰晨夜進道,西到長安而太祖已還,從漢中而歸。彰鬚黃,故以呼之。}
 
 
太祖東還,以彰行越騎將軍,留長安。太祖至洛陽,得疾,驛召彰,未至,太祖崩。
 \gezhu{魏略曰:彰至,謂臨菑侯植曰:「先王召我者,欲立汝也。」植曰:「不可。不見袁氏兄弟乎!」}
 文帝即王位,彰與諸侯就國。
 \gezhu{魏略曰:太子嗣立,旣葬,遣彰之國。始彰自以先王見任有功,兾因此遂見授用,而聞當隨例,意甚不恱,不待遣而去。時以鄢陵塉薄,使治中牟。及帝受禪,因封為中牟王。是後大駕幸許昌,北州諸侯上下,皆畏彰之剛嚴;每過中牟,不敢不速。}
 詔曰:「先王之道,庸勳親親,並建母弟,開國承家,故能藩屏大宗,禦侮厭難。彰前受命北伐,清定朔土,厥功茂焉。增邑五千,并前萬戶。」黃初二年,進爵為公。三年,立為任城王。四年,朝京都,疾薨于邸,謚曰威。
 \gezhu{魏氏春秋曰:初,彰問璽綬,將有異志,故來朝不即得見。彰忿怒暴薨。}
 至葬,賜鑾輅、龍旂,虎賁百人,如漢東平王故事。子楷嗣,徙封中牟。五年,改封任城縣。太和六年,復改封任城國,食五縣二千五百戶。青龍三年,楷坐私遣官屬詣中尚方作禁物,削縣二千戶。正始七年,徙封濟南,三千戶。正元、景元初,連增邑,凡四千四百戶。
 \gezhu{楷,泰始初為崇化少府,見百官名。}
 
 
\end{pinyinscope}