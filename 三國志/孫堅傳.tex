\article{孫堅傳}
\begin{pinyinscope}
 
 
 孫堅字文臺,吳郡富春人,蓋孫武之後也。
 
 
\gezhu{吳書曰:堅世仕吳,家于富春,葬於城東。冢上數有光怪,雲氣五色,上屬於天,曼延數里。衆皆往觀視。父老相謂曰:「是非凡氣,孫氏其興矣!」及母懷姙堅,夢腸出繞吳昌門,寤而懼之,以告鄰母。鄰母曰:「安知非吉徵也。」堅生,容貌不凡,性闊達,好奇節。}
 少為縣吏。年十七,與父共載船至錢唐,會海賊胡玉等從匏里上掠取賈人財物,方於岸上分之,行旅皆住,舩不敢進。堅謂父曰:「此賊可擊,請討之。」父曰:「非爾所圖也。」堅行操刀上岸,以手東西指麾,若分部人兵以羅遮賊狀。賊望見,以為官兵捕之,即委財物散走。堅追,斬得一級以還;父大驚。由是顯聞,府召署假尉。會稽妖賊許昌起於句章,自稱陽明皇帝,
 \gezhu{靈帝紀曰:昌以其父為越王也。}
 與其子韶扇動諸縣,衆以萬數。堅以郡司馬募召精勇,得千餘人,與州郡合討破之。是歲,熹平元年也。刺史臧旻列上功狀,詔書除堅鹽瀆丞,數歲徙盱眙丞,又徙下邳丞。
 \gezhu{江表傳曰:堅歷佐三縣,所在有稱,吏民親附。鄉里知舊,好事少年,往來者常數百人,堅接撫待養,有若子弟焉。}
 
 
中平元年,黃巾賊帥張角起於魏郡,託有神靈,遣八使以善道教化天下,而潛相連結,自稱黃天泰平。三月甲子,三十六萬一旦俱發,天下響應,燔燒郡縣,殺害長吏。
 \gezhu{獻帝春秋曰:角稱天公將軍,角弟寶稱地公將軍,寶弟梁稱人公將軍。}
 漢遣車騎將軍皇甫嵩、中郎將朱儁將兵討擊之。儁表請堅為佐軍司馬,鄉里少年隨在下邳者皆願從。堅又募諸商旅及淮、泗精兵,合千許人,與儁并力奮擊,所向無前。
 \gezhu{吳書曰:堅乘勝深入,於西華失利。堅被創墮馬,卧草中。軍衆分散,不知堅所在。堅所騎騘馬馳還營,捂地呼鳴,將士隨馬於草中得堅。堅還營十數日,創少愈,乃復出戰。}
 汝、潁賊困迫,走保宛城。堅身當一面,登城先入,衆乃蟻附,遂大破之。儁具以狀聞上,拜堅別部司馬。
 \gezhu{續漢書曰:儁字公偉,會稽人,少好學,為郡功曹,察孝廉,舉進士。漢朝以討黃巾功拜車騎將軍,累遷河南尹。董卓見儁,外甚親納,而心忌之,儁亦陰備焉。關東兵起,卓議移都,儁輒止卓。卓雖憚儁,然貪其名重,乃表拜太僕以自副。儁被召不肯受拜,因進曰:「國不宜遷,必孤天下望,成山東之結,臣不見其可也。」有司詰曰:「召君受拜而君拒之,不問徙事而君陳之,何也?」儁曰:「副相國,非臣所堪也。遷都非計,臣之所急也。辭所不堪,進臣所急,臣之所宜也。」有司曰:「遷都之事,初無此計也,就有,未露,何所受聞?」儁曰:「相國董卓為臣說之,臣聞之於相國。」有司不能屈,朝廷稱服焉。後為太尉。李傕、郭汜相攻,劫質天子公卿,儁性剛,即發病而卒。}
 
 
邊章、韓遂作亂涼州。中郎將董卓拒討無功。中平三年,遣司空張溫行車騎將軍,西討章等。溫表請堅與參軍事,屯長安。溫以詔書召卓,卓良乆乃詣溫。溫責讓卓,卓應對不順。堅時在坐,前耳語謂溫曰:「卓不怖罪而鴟張大語,宜以召不時至,陳軍法斬之。」溫曰:「卓素著威名於隴蜀之間,今日殺之,西行無依。」堅曰:「明公親率王兵,威震天下,何賴於卓?觀卓所言,不假明公,輕上無禮,一罪也。章、遂跋扈經年,當以時進討,而卓云未可,沮軍疑衆,二罪也。卓受任無功,應召稽留,而軒昂自高,三罪也。古之名將,仗鉞臨衆,未有不斷斬以示威者也,是以穰苴斬莊賈,魏絳戮楊干。今明公垂意於卓,不即加誅,虧損威刑,於是在矣。」溫不忍發舉,乃曰:「君且還,卓將疑人。」堅因起出。章、遂聞大兵向至,黨衆離散,皆乞降。軍還,議者以軍未臨敵,不斷功賞,然聞堅數卓三罪,勸溫斬之,無不歎息。拜堅議郎。時長沙賊區星自稱將軍,衆萬餘人,攻圍城邑,乃以堅為長沙太守。到郡親率將士,施設方略,旬月之閒,克破星等。
 \gezhu{魏書曰:堅到郡,郡中震服,任用良吏。勑吏曰:「謹遇良善,治官曹文書,必循治,以盜賊付太守。」}
 周朝、郭石亦帥徒衆起於零、桂,與星相應。遂越境尋討,三郡肅然。漢朝錄前後功,封堅烏程侯。
 \gezhu{吳錄曰:是時廬江太守陸康從子作宜春長,為賊所攻,遣使求救於堅。堅整嚴救之。主簿進諫,堅荅曰:「太守無文德,以征伐為功,越界攻討,以全異國。以此獲罪,何媿海內乎?」乃進兵往救,賊聞而走之。}
 
 
靈帝崩,卓擅朝政,橫恣京城。諸州郡並興義兵,欲以討卓。
 \gezhu{江表傳曰:堅聞之,拊膺歎曰:「張公昔從吾言,朝廷今無此難也。」}
 堅亦舉兵。荊州刺史王叡素遇堅無禮,堅過殺之。
 \gezhu{案王氏譜,叡字通耀,晉太保祥伯父也。吳錄曰:叡先與堅共擊零、桂賊,以堅武官,言頗輕之。及叡舉兵欲討卓,素與武陵太守曹寅不相能,楊言當先殺寅。寅懼,詐作案行使者光祿大夫溫毅檄,移堅,說叡罪過,令收行刑訖,以狀上。堅即承檄勒兵襲叡。叡聞兵至,登樓望之,遣問欲何為,堅前部荅曰:「兵乆戰勞苦,所得賞,不足以為衣服,詣使君更乞資直耳。」叡曰:「刺史豈有所吝?」便開庫藏,使自入視之,知有所遺不。兵進及樓下,叡見堅,驚曰:「兵自求賞,孫府君何以在其中?」堅曰:「被使者檄誅君。」叡曰:「我何罪?」堅曰:「坐無所知。」叡窮迫,刮金飲之而死。}
 比至南陽,衆數萬人。南陽太守張咨聞軍至,晏然自若。
 \gezhu{英雄記曰:咨字子議,潁川人,亦知名。獻帝春秋曰:袁術表堅假中郎將。堅到南陽,移檄太守請軍糧。咨以問綱紀,綱紀曰:「堅鄰郡二千石,不應調發。」咨遂不與。}
 堅以牛酒禮咨,咨明日亦荅詣堅。酒酣,長沙主簿入白堅:「前移南陽,而道路不治,軍資不具,請收主簿推問意故。」咨大懼欲去,兵陳四周不得出。有頃,主簿復入白堅:「南陽太守稽停義兵,使賊不時討,請收出案軍法從事。」便牽咨於軍門斬之。郡中震慄,無求不獲。
 \gezhu{吳歷曰:初堅至南陽,咨旣不給軍糧,又不肯見堅。堅欲進兵,恐有後患,乃詐得急疾,舉軍震惶,迎呼巫醫,禱祀山川。遣所親人說咨,言病困,欲以兵付咨。咨聞之,心利其兵,即將步騎五六百人詣營省堅。堅卧與相見。無何,卒然而起,按劒罵咨,遂執斬之。此語與本傳不同。}
 前到魯陽,與袁術相見。術表堅行破虜將軍,領豫州刺史。遂治兵於魯陽城。當進軍討卓,遣長史公仇稱將兵從事還州督促軍糧。施帳幔於城東門外,祖道送稱,官屬並會。卓遣步騎數萬人逆堅,輕騎數十先到。堅方行酒談笑,勑部曲整頓行陣,無得妄動。後騎漸益,堅徐罷坐,導引入城,乃謂左右曰:「向堅所以不即起者,恐兵相蹈藉,諸君不得入耳。」卓兵見堅士衆甚整,不敢攻城,乃引還。
 \gezhu{英雄記曰:初堅討董卓,到梁縣之陽人。卓亦遣兵步騎五千迎之,陳郡太守胡軫為大督護,呂布為騎督,其餘步騎將校都督者甚衆。軫字文才,性急,預宣言曰:「今此行也,要當斬一青綬,乃整齊耳。」諸將聞而惡之。軍到廣成,去陽人城數十里。日暮,士馬疲極,當止宿,又本受卓節度宿廣成,秣馬飲食,以夜進兵,投曉攻城。諸將惡憚軫,欲賊敗其事,布等宣言「陽人城中賊已走,當追尋之;不然失之矣」,便夜進軍。城中守備甚設,不可掩襲。於是吏士饑渴,人馬甚疲,且夜至,又無壍壘。釋甲休息,而布又宣言相驚,云「城中賊出來」。軍衆擾亂奔走,皆棄甲,失鞌馬。行十餘里,定無賊,會天明,便還,拾取兵器,欲進攻城。城守已固,穿壍已深,軫等不能攻而還。}
 堅移屯梁東,大為卓軍所攻,堅與數十騎潰圍而出。堅常著赤罽幘,乃脫幘令親近將祖茂著之。卓騎爭逐茂,故堅從閒道得免。茂困迫,下馬,以幘冠冢閒燒柱,因伏草中。卓騎望見,圍繞數重,定近覺是柱,乃去。堅復相收兵,合戰於陽人,大破卓軍,梟其都督華雄等。是時,或閒堅於術,術懷疑,不運軍糧。
 \gezhu{江表傳曰:或謂術曰:「堅若得洛,不可復制,此為除狼而得虎也」,故術疑之。}
 陽人去魯陽百餘里,堅夜馳見術,畫地計校,曰:「所以出身不顧,上為國家討賊,下慰將軍家門之私讎。堅與卓非有骨肉之怨也,而將軍受譖潤之言,還相嫌疑!」
 \gezhu{江表傳載堅語曰:「大勳垂捷而軍糧不繼,此吳起所以歎泣於西河,樂毅所以遺恨於垂成也。願將軍深思之。」}
 術踧踖,即調發軍糧。堅還屯。卓憚堅猛壯,乃遣將軍李傕等來求和親,令堅列疏子弟任刺史、郡守者,許表用之。堅曰:「卓逆天無道,蕩覆王室,今不夷汝三族,縣示四海,則吾死不瞑目,豈將與乃和親邪?」復進軍大谷,拒雒九十里。
 \gezhu{山陽公載記曰:卓謂長史劉艾曰:「關東軍敗數矣,皆畏孤,無能為也。惟孫堅小戇,頗能用人,當語諸將,使知忌之。孤昔與周慎西征,慎圍邊、韓於金城。孤語張溫,求引所將兵為慎作後駐。溫不聽。孤時上言其形勢,知慎必不克。臺今有本末。事未報,溫又使孤討先零叛羌,以為西方可一時蕩定。孤皆知其不然而不得止,遂行,留別部司馬劉靖將步騎四千屯安定,以為聲勢。叛羌便還,欲截歸道,孤小擊輒開,畏安定有兵故也。虜謂安定當數萬人,不知但靖也。時又上章言狀,而孫堅隨周慎行,謂慎求將萬兵造金城,使慎以二萬作後駐,邊、韓城中無宿穀,當於外運,畏慎大兵,不敢輕與堅戰,而堅兵足以斷其運道,兒曹用必還羌谷中,涼州或能定也。溫旣不能用孤,慎又不用堅,自攻金城,壞其外垣,馳使語溫,自以克在旦夕,溫時亦自以計中也。而渡遼兒果斷蔡圍,慎棄輜重走,果如孤策。臺以此封孤都鄉侯。堅以佐軍司馬,所見與人同,自為可耳。」艾曰:「堅雖時見計,故自不如李傕、郭汜。聞在美陽亭北,將千騎步與虜合,殆死,亡失印綬,此不為能也。」卓曰:「堅時烏合義從,兵不如虜精,且戰有利鈍。但當論山東大勢,終無所至耳。」艾曰:「山東兒驅略百姓,以作寇逆,其鋒不如人,堅甲利兵彊弩之用又不如人,亦安得乆?」卓曰:「然,但殺二袁、劉表、孫堅,天下自服從孤耳。」}
 卓尋徙都西入關,焚燒雒邑。堅乃前入至雒,脩諸陵,平塞卓所發掘。
 \gezhu{江表傳曰:舊京空虛,數百里中無煙火。堅前入城,惆悵流涕。吳書曰:堅入洛,埽除漢宗廟,祠以太牢。堅軍城南甄官井上,旦有五色氣,舉軍驚恠,莫有敢汲。堅令人入井,探得漢傳國璽,文曰「受命于天,旣壽永昌」,方圜四寸,上紐交五龍,上一角缺。初,黃門張讓等作亂,劫天子出奔,左右分散,掌璽者以投井中。山陽公載記曰:袁術將僭號,聞堅得傳國璽,乃拘堅夫人而奪之。江表傳曰:案漢獻帝起居注云「天子從河上還,得六璽於閣上」,又太康之初孫皓送金璽六枚,無有玉,明其偽也。虞喜志林曰:天子六璽者,文曰「皇帝之璽」、「皇帝行璽」、「皇帝信璽」、「天子之璽」、「天子行璽」、「天子信璽」。此六璽所封事異,故文字不同。獻帝起注云「從河上還,得六玉璽於閣上」,此之謂也。傳國璽者,乃漢高祖所佩秦皇帝璽,世世傳受,號曰傳國璽。案傳國璽不在六璽之數,安得總其說乎?應氏漢官、皇甫世紀,其論六璽,文義皆符。漢宮傳國璽,文曰「受命于天,旣壽且康」。「且康」「永昌」,二字為錯,未知兩家何者為得。金玉之精,率有光氣,加以神器祕寶,輝耀益彰,蓋一代之奇觀,將來之異聞,而以不解之故,彊謂之偽,不亦誣乎!陳壽為破虜傳亦除此說,俱惑起居注,不知六璽殊名,與傳國為七者也。吳時無能刻玉,故天子以金為璽。璽雖以金,於文不異。吳降而送璽者送天子六璽,曩所得玉璽,乃古人遺印,不可施用。天子之璽,今以無有為難,不通其義者耳。臣松之以為孫堅於興義之中最有忠烈之稱,若得漢神器而潛匿不言,此為陰懷異志,豈所謂忠臣者乎?吳史欲以為國華,而不知損堅之令德。如其果然,以傳子孫,縱非六璽之數,要非常人所畜,孫皓之降,亦不得但送六璽,而寶藏傳國也。受命于天,奚取於歸命之堂,若如喜言,則此璽今尚在孫門。匹夫懷璧,猶曰有罪,而況斯物哉!}
 訖,引軍還,住魯陽。
 \gezhu{吳錄曰:是時關東州郡,務相兼并以自彊大。袁紹遣會稽周㬂為豫州刺史,來襲取州。堅慨然歎曰:「同舉義兵,將救社稷。逆賊垂破而各若此,吾當誰與戮力乎!」言發涕下。㬂字仁明,周昕之弟也。會稽典錄曰:初曹公興義兵,遣人要喁,㬂即收合兵衆,得二千人,從公征伐,以為軍師。後與堅爭豫州,屢戰失利。會次兄九江太守昂為袁術所攻,㬂往助之。軍敗,還鄉里,為許貢所害。}
 
 
初平三年,術使堅征荊州,擊劉表。表遣黃祖逆於樊、鄧之間。堅擊破之,追渡漢水,遂圍襄陽,單馬行峴山,為祖軍士所射殺。
 \gezhu{典略曰;堅悉其衆攻表,表閉門,夜遣將黃祖潛出發兵。祖將兵欲還,堅逆與戰。祖敗走,竄峴山中。堅乘勝夜追祖,祖部兵從竹木間暗射堅,殺之。吳錄曰:堅時年三十七。英雄記曰:堅以初平四年正月七日死。又云:劉表將呂公將兵緣山向堅,堅輕騎尋山討公。公兵下石。中堅頭,應時腦出物故。其不同如此也。}
 兄子賁,帥將士衆就術,術復表賁為豫州刺史。
 
 
堅四子:策、權、翊、匡。權旣稱尊號,謚堅曰武烈皇帝。
 \gezhu{吳錄曰:尊堅廟曰始祖,墓曰高陵。志林曰:堅有五子:策、權、翊、匡,吳氏所生;少子朗,庶生也,一名仁。}
 
 
\end{pinyinscope}