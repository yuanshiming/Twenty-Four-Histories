\article{崔林傳}
\begin{pinyinscope}
 
 
 崔林字德儒,清河東武城人也。少時晚成,宗族莫知,惟從兄琰異之。太祖定兾州,召除鄔長,貧無車馬,單步之官。太祖征壺關,問長吏德政最者,并州刺史張陟以林對,於是擢為兾州主簿,徙署別駕、丞相掾屬。魏國旣建,稍遷御史中丞。
 
 
 
 
 文帝踐阼,拜尚書,出為幽州刺史。北中郎將吳質統河北軍事,涿郡太守王雄謂林別駕曰:「吳中郎將,上所親重,國之貴臣也。杖節統事,州郡莫不奉牋致敬,而崔使君初不與相聞。若以邊塞不脩斬卿,使君寧能護卿邪?」別駕具以白林,林曰:「刺史視去此州如脫屣,寧當相累邪?此州與胡虜接,宜鎮之以靜,擾之則動其逆心,特為國家生北顧憂,以此為寄。」在官一期,寇竊寢息;
 
 
\gezhu{案王氏譜:雄字元伯,太保祥之宗也。魏名臣奏載安定太守孟達薦雄曰:「臣聞明君以求賢為業,忠臣以進善為效,故易稱『拔茅連茹』,傳曰『舉爾所知』。臣不自量,竊慕其義。臣昔以人乏,謬充備部職。時涿郡太守王雄為西部從事,與臣同僚。雄天性良固,果而有謀。歷試三縣,政成人和。及在近職,奉宣威恩,懷柔有術,清慎持法。臣往年出使,經過雄郡。自說特受陛下拔擢之恩,常勵節精心,思投命為效。言辭激揚,情趣款惻。臣雖愚闇,不識真偽,以謂雄才兼資文武,忠烈之性,踰越倫輩。今涿郡領戶三千,孤寡之家,參居其半,北有守兵藩衞之固,誠不足舒雄智力,展其勤幹也。臣受恩深厚,無以報國,不勝慺慺淺見之情,謹冒陳聞。」詔曰:「昔蕭何薦韓信,鄧禹進吳漢,惟賢知賢也。雄有膽智技能文武之姿,吾宿知之。今便以參散騎之選,方使少在吾門下知指歸,便大用之矣。天下之士,欲使皆先歷散騎,然後出據州郡,是吾本意也。」雄後為幽州刺史。子渾,涼州刺史。次乂,平北將軍。司徒安豐侯戎,渾之子。太尉武陵侯衍、荊州刺史澄,皆乂之子。}
 猶以不事上司,左遷河間太守,清論多為林怨也。
 \gezhu{魏名臣奏載侍中辛毗奏曰:「昔桓階為尚書令,以崔林非尚書才,遷以為河間太守。」與此傳不同。}
 
 
 
 
 遷大鴻臚。龜茲王遣侍子來朝,朝廷嘉其遠至,襃賞其王甚厚。餘國各遣子來朝,閒使連屬,林恐所遣或非真的,權取疏屬賈胡,因通使命,利得印綬,而道路護送,所損滋多。勞所養之民,資無益之事,為夷狄所笑,此曩時之所患也。乃移書燉煌喻指,并錄前世待遇諸國豐約故事,使有恒常。明帝即位,賜爵關內侯,轉光祿勳、司隷校尉。屬郡皆罷非法除過員吏。林為政推誠,簡存大體,是以去後每輒見思。
 
 
 
 
 散騎常侍劉劭作考課論,制下百僚。林議曰:「案周官考課,其文備矣,自康王以下,遂以陵遲,此即考課之法存乎其人也。及漢之季,其失豈在乎佐吏之職不密哉?方今軍旅,或猥或卒,備之以科條,申之以內外,增減無常,固難一矣。且萬目不張舉其綱,衆毛不整振其領。臯陶仕虞,伊尹臣殷,不仁者遠。五帝三王未必如一,而各以治亂。易曰:『易簡,而天下之理得矣。』太祖隨宜設辟,以遺來今,不患不法古也。以為今之制度,不為疏闊,惟在守一勿失而已。若朝臣能任仲山甫之重,式是百辟,則孰敢不肅?」
 
 
景初元年,司徒、司空並缺,散騎侍郎孟康薦林曰:「夫宰相者,天下之所瞻效,誠宜得秉忠履正本德杖義之士,足為海內所師表者。竊見司隷校尉崔林,稟自然之正性,體高雅之弘量。論其所長以比古人,忠直不回則史魚之儔,清儉守約則季文之匹也。牧守州郡,所在而治,及為外司,萬里肅齊,誠台輔之妙器,衮職之良才也。」後年遂為司空,封安陽亭侯,邑六百戶。三公封列侯,自林始也。
 \gezhu{臣松之以為漢封丞相邑,為荀恱所譏。魏封三公,其失同也。}
 頃之,又進封安陽鄉侯。
 
 
魯相上言:「漢舊立孔子廟,襃成侯歲時奉祠,辟雍行禮,必祭先師,王家出穀,春秋祭祀。今宗聖侯奉嗣,未有命祭之禮,宜給牲牢,長吏奉祀,尊為貴神。」制三府議,博士傅祗以春秋傳言立在祀典,則孔子是也。宗聖適足繼絕世,章盛德耳。至於顯立言,崇明德,則宜如魯相所上。林議以為「宗聖侯亦以王命祀,不為未有命也。周武王封黃帝、堯、舜之後,及立三恪,禹、湯之世,不列于時,復特命他官祭也。今周公已上,達於三皇,忽焉不祀,而其禮經亦存其言。今獨祀孔子者,以世近故也。以大夫之後,特受無疆之祀,禮過古帝,義踰湯、武,可謂崇明報德矣,無復重祀於非族也。」
 \gezhu{臣松之以為孟軻稱宰我之辭曰:「予以觀夫子,賢於堯舜遠矣。」又曰:「生民以來,未有盛於孔子者也。」斯非通賢之格言,商較之定準乎!雖妙極則同,萬聖猶一,然淳薄異時,質文殊用,或當時則榮,沒則已焉,是以遺風所被,寔有深淺。若乃經緯天人,立言垂制,百王莫之能違,彝倫資之以立,誠一人而已耳。周監二代,斯文為盛。然於六經之道,未能及其精致。加以聖賢不興,曠年五百,道化陵夷,憲章殆滅,若使時無孔門,則周典幾乎息矣。夫能光明先王之道,以成萬世之功,齊天地之無窮,等日月之久照,豈不有踰於羣聖哉?林曾無史遷洞想之誠,梅真慷慨之志,而守其蓬心以塞明義,可謂多見其不知量也。}
 
 
明帝又分林邑,封一子列侯。正始五年薨,謚曰孝侯。子述嗣。
 \gezhu{晉諸公贊曰:述弟隨,晉尚書僕射。為人亮濟。趙王倫篡位,隨與其事。倫敗,隨亦廢錮而卒。林孫瑋,性率而踈,至太子右衞率也。初,林識拔同郡王經於民伍之中,卒為名士,世以此稱之。}
 
 
\end{pinyinscope}