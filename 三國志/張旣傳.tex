\article{張旣傳}
\begin{pinyinscope}
 
 
 張旣字德容,馮翊高陵人也。年十六,為郡小吏。
 
 
\gezhu{魏略曰:旣世單家富,為人有容儀。少小工書疏,為郡門下小吏,而家富。自惟門寒,念無以自達,乃常畜好刀筆及版奏,伺諸大吏有乏者輒給與,以是見識焉。}
 後歷右職,舉孝廉,不行。太祖為司空,辟,未至,舉茂才,除新豐令,治為三輔第一。袁尚拒太祖於黎陽,遣所置河東太守郭援、并州刺史高幹及匈奴單于取平陽,發使西與關中諸將合從。司隷校尉鍾繇遣旣說將軍馬騰等,旣為言利害,騰等從之。騰遣子超將兵萬餘人,與繇會擊幹、援,大破之,斬援首。幹及單于皆降。其後幹復舉并州反。河內張晟衆萬餘人無所屬,冦崤、澠閒,河東衞固、弘農張琰各起兵以應之。太祖以旣為議郎,參繇軍事,使西徵諸將馬騰等,皆引兵會擊晟等,破之。斬琰、固首,幹奔荊州。封旣武始亭侯。
 
 
太祖將征荊州,而騰等分據關中。太祖復遣旣喻騰等,令釋部曲求還。騰已許之而更猶豫,旣恐為變,乃移諸縣促儲偫,二千石郊迎。騰不得已,發東。太祖表騰為衞尉,子超為將軍,統其衆。後超反,旣從太祖破超於華陰,西定關右。以旣為京兆尹,招懷流民,興復縣邑,百姓懷之。魏國旣建,為尚書,出為雍州刺史。太祖謂旣曰:「還君本州,可謂衣繡晝行矣。」從征張魯,別從散關入討叛氐,收其麥以給軍食。魯降,旣說太祖拔漢中民數萬戶以實長安及三輔。其後與曹洪破吳蘭於下辯,又與夏侯淵討宋建,別攻臨洮、狄道,平之。是時,太祖徙民以充河北,隴西、天水、南安民相恐動,擾擾不安,旣假三郡人為將吏者休課,使治屋宅,作水碓,民心遂安。太祖將拔漢中守,恐劉備北取武都氐以逼關中,問旣。旣曰:「可勸使北出就穀以避賊,前至者厚其寵賞,則先者知利,後必慕之。」太祖從其策,乃自到漢中引出諸軍,令旣之武都,徙氐五萬餘落出居扶風、天水界。
 \gezhu{三輔決錄注曰:旣為兒童,為郡功曹游殷察異之,引旣過家,旣敬諾。殷先歸,勑家具設賔饌。及旣至,殷妻笑曰:「君其悖乎!張德容童昏小兒,何異客哉!」殷曰:「卿勿怪,乃方伯之器也。」殷遂與旣論霸王之略。饗訖,以子楚託之;旣謙不受,殷固託之,旣以殷邦之宿望,難違其旨,乃許之。殷先與司隷校尉胡軫有隙,軫誣搆殺殷。殷死月餘,軫得疾患,自說但言「伏罪,伏罪,游功曹將鬼來」。於是遂死。于時關中稱曰:「生有知人之明,死有貴神之靈。」子楚字仲允,為蒲阪令。太祖定關中時,漢興郡缺,太祖以問旣,旣稱楚才兼文武,遂以為漢興太守。後轉隴西。魏略曰:楚為人慷慨,歷位宰守,所在以恩德為治,不好刑殺。太和中,諸葛亮出隴右,吏民騷動。天水、南安太守各棄郡東下,楚獨據隴西,召會吏民,謂之曰:「太守無恩德。今蜀兵至,諸郡吏民皆已應之,此亦諸卿富貴之秋也。太守本為國家守郡,義在必死,卿諸人便可取太守頭持往。」吏民皆涕淚,言「死生當與明府同,無有二心」。楚復言:「卿曹若不願,我為卿畫一計。今東二郡已去,必將寇來,但可共堅守。若國家救到,寇必去,是為一郡守義,人人獲爵寵也。若官救不到,蜀攻日急,爾乃取太守以降,未為晚也。」吏民遂城守。而南安果將蜀兵,就攻隴西。楚聞賊到,乃遣長史馬顒出門設陣,而自於城上曉謂蜀帥,言:「卿能斷隴,使東兵不上,一月之中,則隴西吏人不攻自服;卿若不能,虛自疲弊耳。」使顒鳴鼓擊之,蜀人乃去。後十餘日,諸軍上隴,諸葛亮破走。南安、天水皆坐應亮破滅,兩郡守各獲重刑,而楚以功封列侯,長史掾屬皆賜拜。帝嘉其治,詔特聽朝,引上殿。楚為人短小而大聲,自為吏,初不朝覲,被詔登階,不知儀式。帝令侍中贊引,呼「隴西太守前」,楚當言「唯」,而大應稱「諾」。帝顧之而笑,遂勞勉之。罷會,自表乞留宿衞,拜駙馬都尉。楚不學問,而性好遊遨音樂。乃畜歌者,琵琶、箏、簫,每行來將以自隨。所在樗蒲、投壺,歡欣自娛。數歲,復出為北地太守,年七十餘卒。}
 
 
是時,武威顏俊、張掖和鸞、酒泉黃華、西平麴演等並舉郡反,自號將軍,更相攻擊。俊遣使送母及子詣太祖為質,求助。太祖問旣,旣曰:「俊等外假國威,內生傲悖,計定勢足,後即反耳。今方事定蜀,且宜兩存而鬬之,猶卞莊子之刺虎,坐收其斃也。」太祖曰:「善。」歲餘,鸞遂殺俊,武威王祕又殺鸞。是時不置涼州,自三輔拒西域,皆屬雍州。文帝即王位,初置涼州,以安定太守鄒岐為刺史。張掖張進執郡守舉兵拒岐,黃華、麴演各逐故太守,舉兵以應之。旣進兵為護羌校尉蘇則聲勢,故則得以有功。旣進爵都鄉侯。涼州盧水胡伊健妓妾、治元多等反,河西大擾。帝憂之,曰:「非旣莫能安涼州。」乃召鄒岐,以旣代之。詔曰:「昔賈復請擊郾賊,光武笑曰:『執金吾擊郾,吾復何憂?』卿謀略過人,今則其時。以便宜從事,勿復先請。」遣護軍夏侯儒、將軍費曜等繼其後。旣至金城,欲渡河,諸將守以為「兵少道險,未可深入」。旣曰:「道雖險,非井陘之隘,夷狄烏合,無左車之計,今武威危急,赴之宜速。」遂渡河。賊七千餘騎逆拒軍於鸇陰口,旣揚聲軍從鸇陰,乃潛由且次出至武威。胡以為神,引還顯美。旣已據武威,曜乃至,儒等猶未達。旣勞賜將士,欲進軍擊胡。諸將皆曰:「士卒疲倦,虜衆氣銳,難與爭鋒。」旣曰:「今軍無見糧,當因敵為資。若虜見兵合,退依深山,追之則道險窮餓,兵還則出候寇鈔。如此,兵不得解,所謂『一日縱敵,患在數世』也。」遂前軍顯美。胡騎數千,因大風欲放火燒營,將士皆恐。旣夜藏精卒三千人為伏,使參軍成公英督千餘騎挑戰,勑使陽退。胡果爭奔之,因發伏截其後,首尾進擊,大破之,斬首獲生以萬數。
 \gezhu{魏略曰:成公英,金城人也。中平末,隨韓約為腹心。建安中,約從華陰破走,還湟中,部黨散去,唯英獨從。典略曰:韓遂在湟中,其壻閻行欲殺遂以降,夜攻遂,不下。遂歎息曰:「丈夫困厄,禍起婚姻乎!」謂英曰:「今親戚離叛,人衆轉少,當從羌中西南詣蜀耳。」英曰:「興軍數十年,今雖罷敗,何有棄其門而依於人乎!」遂曰:「吾年老矣,子欲何施?」英曰:「曹公不能遠來,獨夏侯爾。夏侯之衆,不足以追我,又不能乆留;且息肩於羌中,以須其去。招呼故人,綏會羌、胡,猶可以有為也。」遂從其計,時隨從者男女尚數千人。遂宿有恩於羌,羌衞護之。及夏侯淵還,使閻行留後。乃合羌、胡數萬將攻行,行欲走,會遂死,英降太祖。太祖見英甚喜,以為軍師,封列侯。從行出獵,有三鹿走過前,公命英射之,三發三中,皆應弦而倒。公抵掌謂之曰:「但韓文約可為盡節,而孤獨不可乎?」英乃下馬而跪曰:「不欺明公。假使英本主人在,實不來此也。」遂流涕哽咽。公嘉其敦舊,遂親敬之。延康、黃初之際,河西有逆謀。詔遣英佐涼州平隴右,病卒。魏略曰:閻行,金城人也,後名艷,字彥明。少有健名,始為小將,隨韓約。建安初,約與馬騰相攻擊。騰子超亦號為健。行嘗刺超,矛折,因以折矛撾超項,幾殺之。至十四年,為約所使詣太祖,太祖厚遇之,表拜犍為太守。行因請令其父入宿衞,西還見約,宣太祖教云:「謝文約:卿始起兵時,自有所逼,我所具明也。當早來,共匡輔國朝。」行因謂約曰:「行亦為將軍,興軍以來三十餘年,民兵疲瘁,所處又狹,宜早自附。是以前在鄴,自啟當令老父詣京師,誠謂將軍亦宜遣一子,以示丹赤。」約曰:「且可復觀望數歲中!」後遂遣其子,與行父母俱東。會約西討張猛,留行守舊營,而馬超等結反謀,舉約為都督。及約還,超謂約曰:「前鍾司隷任超使取將軍,關東人不可復信也。今超棄父,以將軍為父,將軍亦當棄子,以超為子。」行諫約,不欲令與超合。約謂行曰:「今諸將不謀而同,似有天數。」乃東詣華陰。及太祖與約交馬語,行在其後,太祖望謂行曰:「當念作孝子。」及超等破走,行隨約還金城。太祖聞行前意,故但誅約子孫在京師者。乃手書與行曰:「觀文約所為,使人笑來。吾前後與之書,無所不說,如此何可復忍!卿父諫議,自平安也。雖然,牢獄之中,非養親之處,且又官家亦不能乆為人養老也。」約聞行父獨在,欲使并遇害,以一其心,乃彊以少女妻行,行不獲已。太祖果疑行。會約使行別領西平郡。遂勒其部曲,與約相攻擊。行不勝,乃將家人東詣太祖。太祖表拜列侯。}
 帝甚恱,詔曰:「卿踰河歷險,以勞擊逸,以寡勝衆,功過南仲,勤踰吉甫。此勳非但破胡,乃永寧河右,使吾長無西顧之念矣。」徙封西鄉侯,增邑二百,并前四百戶。
 
 
酒泉蘇衡反,與羌豪鄰戴及丁令胡萬餘騎攻邊縣。旣與夏侯儒擊破之,衡及鄰戴等皆降。遂上疏請與儒治左城,築鄣塞,置烽候、邸閣以備胡。
 \gezhu{魏略曰:儒字俊林,夏侯尚從弟。初為鄢陵侯彰驍騎司馬,宣王為征南將軍,都督荊、豫州。正始二年,朱然圍樊城,城中守將乙脩等求救甚急。儒進屯鄧塞,以兵少不敢進,但作鼓吹,設導從,去然六七里,翱翔而還,使脩等遙見之,數數如是。月餘,及太傅到,乃俱進,然等走。時謂儒為怯,或以為曉以少疑衆,得聲救之宜。儒猶以此召還,為太僕。}
 西羌恐,率衆二萬餘落降。其後西平麴光等殺其郡守,諸將欲擊之,旣曰:「唯光等造反,郡人未必悉同。若便以軍臨之,吏民羌胡必謂國家不別是非,更使皆相持著,此為虎傅翼也。光等欲以羌胡為援,今先使羌胡鈔擊,重其賞募,所虜獲者皆以畀之。外沮其勢,內離其交,必不戰而定。」乃檄告諭諸羌,為光等所詿誤者原之;能斬賊帥送首者當加封賞。於是光部黨斬送光首,其餘咸安堵如故。
 
 
旣臨二州十餘年,政惠著聞,其所禮辟扶風龐延、天水楊阜、安定胡遵、酒泉龐淯、燉煌張恭、周生烈等,終皆有名位。
 \gezhu{魏略曰:初,旣為郡小吏,功曹徐英嘗自鞭旣三十。英字伯濟,馮翊著姓,建安初為蒲阪令。英性剛爽,自見族氏勝旣,於鄉里名行在前,加以前辱旣,雖知旣貴顯,終不肯求於旣。旣雖得志,亦不顧計本原,猶欲與英和。嘗因醉欲親狎英,英故抗意不納。英由此遂不復進用。故時人善旣不挾舊怨,而壯英之不撓。}
 黃初四年薨。詔曰:「昔荀桓子立勳翟土,晉侯賞以千室之邑;馮異輸力漢朝,光武封其二子。故涼州刺史張旣,能容民畜衆,使羣羌歸土,可謂國之良臣。不幸薨隕,朕甚愍之,其賜小子翁歸爵關內侯。」明帝即位,追謚曰肅侯。子緝嗣。
 
 
緝以中書郎稍遷東莞太守。嘉平中,女為皇后,徵拜光祿大夫,位特進,封妻向為安城鄉君。緝與中書令李豐同謀,誅。語在夏侯玄傳。
 \gezhu{魏略曰:緝字敬仲,太和中為溫令,名有治能。會諸葛亮出,緝上便宜,詔以問中書令孫資,資以為有籌略,遂召拜騎都尉,遣參征蜀軍。軍罷,入為尚書郎,以稱職為明帝所識。帝以為緝之材能,多所堪任,試呼相工相之。相者云:「不過二千石。」帝曰:「何材如是而位止二千石乎?」及在東莞,領兵數千人。緝性吝於財而矜於勢,一旦以女徵去郡,還坐里舍,悒悒躁擾。數為國家陳擊吳、蜀形勢,又嘗對司馬大將軍料諸葛恪雖得勝於邊土,見誅不乆。大將軍問其故,緝云:「威震其主,功蓋一國,欲不死可得乎?」及恪從合肥還,吳果殺之。大將軍聞恪死,謂衆人曰:「諸葛恪多輩耳!近張敬仲縣論恪,以為必見殺,今果然如此。敬仲之智為勝恪也。」緝與李豐通家,又居相側近。豐時取急出,子藐往見之,有所咨道。豐被收,事與緝連,遂收送廷尉,賜死獄中,其諸子皆并誅。緝孫殷,晉永興中為梁州刺史,見晉書。}
 
 
\end{pinyinscope}