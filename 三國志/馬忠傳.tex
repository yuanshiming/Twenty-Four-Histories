\article{馬忠傳}
\begin{pinyinscope}
 
 
 馬忠字德信,巴西閬中人也。少養外家,姓狐,名篤,後乃復姓,改名忠。為郡吏,建安末舉孝廉,除漢昌長。先主東征,敗績猇亭,巴西太守閻芝發諸縣兵五千人以補遺闕,遣忠送往。先主已還永安,見忠與語,謂尚書令劉巴曰:「雖亡黃權,復得狐篤,此為世不乏賢也。」建興元年,丞相亮開府,以忠為門下督。三年,亮入南,拜忠䍧牱太守。郡丞朱襃反。叛亂之後,忠撫育卹理,甚有威惠。八年,召為丞相參軍,副長史蔣琬署留府事。又領州治中從事。明年,亮出祁山,忠詣亮所,經營戎事。軍還,督將軍張嶷等討汶山郡叛羌。十一年,南夷豪帥劉冑反,擾亂諸郡。徵庲降都督張翼還,以忠代翼。忠遂斬冑,平南土。加忠監軍奮威將軍,封博陽亭侯。初,建寧郡殺太守正昂,縛太守張裔於吳,故都督常駐平夷縣。至忠,乃移治味縣,處民夷之間。又越嶲郡亦乆失土地,忠率將太守張嶷開復舊郡,由此就加安南將軍,進封彭鄉亭侯。延熈五年還朝,因至漢中,見大司馬蔣琬,宣傳詔旨,加拜鎮南大將軍。七年春,大將軍費禕北禦魏敵,留忠成都,平尚書事。禕還,忠乃歸南。十二年卒,子脩嗣。
 
 
\gezhu{脩弟恢。恢子義,晉建寧太守。}
 
 
 
 
 忠為人寬濟有度量,但詼啁大笑,忿怒不形於色。然處事能斷,威恩並立,是以蠻夷畏而愛之。及卒,莫不自致喪庭,流涕盡哀,為之立廟祀,迄今猶在。
 
 
張表,時名士,清望踰忠。閻宇,宿有功幹,於事精勤。繼踵在忠後,其威風稱績,皆不及忠。
 \gezhu{益部耆舊傳曰:張表,肅子也。華陽國志云:表,張松子,未詳。閻宇字文平,南郡人也。}
 
 
\end{pinyinscope}