\article{楊阜傳}
\begin{pinyinscope}
 
 
 楊阜字義山,天水兾人也。
 
 
\gezhu{魏略曰:阜少與同郡尹奉次曾、趙昂偉章俱發名,偉章、次曾與阜俱為涼州從事。}
 以州從事為牧韋端使詣許,拜安定長史。阜還,關右諸將問袁、曹勝敗孰在,阜曰:「袁公寬而不斷,好謀而少決;不斷則無威,少決則失後事,今雖彊,終不能成大業。曹公有雄才遠略,決機無疑,法一而兵精,能用度外之人,所任各盡其力,必能濟大事者也。」長史非其好,遂去官。而端徵為太僕,其子康代為刺史,辟阜為別駕。察孝廉,辟丞相府,州表留參軍事。
 
 
 
 
 馬超之戰敗渭南也,走保諸戎。太祖追至安定,而蘇伯反河閒,將引軍東還。阜時奉使,言於太祖曰:「超有信、布之勇,甚得羌、胡心,西州畏之。若大軍還,不嚴為之備,隴上諸郡非國家之有也。」太祖善之,而軍還倉卒,為備不周。超率諸戎渠帥以擊隴上郡縣,隴上郡縣皆應之,惟兾城奉州郡以固守。超盡兼隴右之衆,而張魯又遣大將楊昂以助之,凡萬餘人,攻城。阜率國士大夫及宗族子弟勝兵者千餘人,使從弟岳於城上作偃月營,與超接戰,自正月至八月拒守而救兵不至。州遣別駕閻溫循水潛出求救,為超所殺,於是刺史、太守失色,始有降超之計。阜流涕諫曰:「阜等率父兄子弟以義相勵,有死無二;田單之守,不固於此也。棄垂成之功,陷不義之名,阜以死守之。」遂號哭。刺史、太守卒遣人請和,開城門迎超。超入,拘岳於兾,使楊昂殺刺史、太守。
 
 
 
 
 阜內有報超之志,而未得其便。頃之,阜以喪妻求葬假。阜外兄姜叙屯歷城。阜少長叙家,見叙母及叙,說前在兾中時事,歔欷悲甚。叙曰:「何為乃爾?」阜曰:「守城不能完,君亡不能死,亦何面目以視息於天下!馬超背父叛君,虐殺州將,豈獨阜之憂責,一州士大夫皆蒙其恥。君擁兵專制而無討賊心,此趙盾所以書殺君也。超彊而無義,多釁易圖耳。」叙母慨然,勑叙從阜計。計定,外與鄉人姜隱、趙昂、尹奉、姚瓊、孔信、武都人李俊、王靈結謀,定討超約,使從弟謨至兾語岳,并結安定梁寬、南安趙衢、龐恭等。約誓旣明,十七年九月,與叙起兵於鹵城。超聞阜等兵起,自將出。而衢、寬等解岳,閉兾城門,討超妻子。超襲歷城,得叙母。叙母罵之曰:「汝背父之逆子,殺君之桀賊,天地豈乆容汝,而不早死,敢以面目視人乎!」超怒,殺之。阜與超戰,身被五創,宗族昆弟死者七人。超遂南奔張魯。
 
 
隴右平定,太祖封討超之功,侯者十一人,賜阜爵關內侯。阜讓曰:「阜君存無扞難之功,君亡無死節之効,於義當絀,於法當誅;超又不死,無宜苟荷爵祿。」太祖報曰:「君與羣賢共建大功,西土之人以為美談。子貢辭賞,仲尼謂之止善。君其剖心以順國命。姜叙之母,勸叙早發,明智乃爾,雖楊敞之妻蓋不過此。賢哉,賢哉!良史記錄,必不墜於地矣。」
 \gezhu{皇甫謐列女傳曰:姜叙母者,天水姜伯弈之母也。建安中,馬超攻兾,害涼州刺史韋康,州人悽然,莫不感憤。叙為撫夷將軍,擁兵屯歷。叙姑子楊阜,故為康從事,同等十餘人,皆略屬超,陰相結為康報仇,未有閒。會阜妻死,辭超寧歸西,因過至歷,候叙母,說康被害及兾中之難,相對泣良乆。姜叙舉室感悲,叙母曰:「咄!伯弈,韋使君遇難,豈一州之恥,亦汝之負,豈獨義山哉?汝無顧我,事淹變生。人誰不死?死國,忠義之大者。但當速發,我自為汝當之,不以餘年累汝也。」因勑叙與阜參議,許諾,分人使語鄉里尹奉、趙昂及安定梁寬等,令叙先舉兵叛超,超怒,必自來擊叙,寬等因從後閉門。約誓以定,叙遂進兵入鹵,昂、奉守祁山。超聞,果自出擊叙,寬等從後閉兾門,超失據。過鹵,叙守鹵。超因進至歷,歷中見超往,以為叙軍還。又傳聞超以走奔漢中,故歷無備。及超入歷,執叙母,母怒罵超。超被罵大怒,即殺叙母及其子,燒城而去。阜等以狀聞,太祖甚嘉之,手令襃揚,語如本傳。臣松之案:謐稱阜為叙姑子,而本傳云叙為阜外兄,與今名內外為不同。謐又載趙昂妻曰:趙昂妻異者,故益州刺史天水趙偉璋妻,王氏女也。昂為羌道令,留異在西。會同郡梁雙反,攻破西城,害異兩男。異女英,年六歲,獨與異在城中。異見兩男已死,又恐為雙所侵,引刀欲自刎,顧英而歎曰:「身死爾棄,當誰恃哉!吾聞西施蒙不絜之服,則人掩鼻,況我貌非西施乎?」乃以溷糞涅麻而被之,尠食瘠形,自春至冬。雙與州郡和,異竟以是免難。昂遣吏迎之,未至三十里,止謂英曰:「婦人無符信保傅,則不出房闈。昭姜沈流,伯姬待燒,每讀其傳,心壯其節。今吾遭亂不能死,將何以復見諸姑?所以偷生不死,惟憐汝耳。今官舍已近,吾去汝死矣。」遂飲毒藥而絕。時適有解毒藥良湯,撅口灌之,良乆迺蘇。建安中,昂轉參軍事,徙居兾。會馬超攻兾,異躬著布韝,佐昂守備,又悉脫所佩環、黼黻以賞戰士。及超攻急,城中饑困,刺史韋康素仁,愍吏民傷殘,欲與超和。昂諫不聽,歸以語異,異曰:「君有爭臣,大夫有專利之義;專不為非也。焉知救兵不到關隴哉?當共勉卒高勳,全節致死,不可從也。」比昂還,康與超和。超遂背約害康,又劫昂,質其嫡子月於南鄭。欲要昂以為己用,然心未甚信。超妻楊聞異節行,請與讌終日。異欲信昂於超以濟其謀,謂楊曰:「昔管仲入齊,立九合之功;由余適秦,穆公成霸。方今社稷初定,治亂在於得人,涼州士馬,迺可與中夏爭鋒,不可不詳也。」楊深感之,以為忠於己,遂與異重相接結。昂所以得信於超,全功免禍者,異之力也。及昂與楊阜等結謀討超,告異曰:「吾謀如是,事必萬全,當柰月何?」異厲聲應曰:「忠義立於身,雪君父之大恥,喪元不足為重,況一子哉?夫項託、顏淵,豈復百年,貴義存耳。」昂曰:「善。」遂共閉門逐超,超奔漢中,從張魯得兵還。異復與昂保祁山,為超所圍,三十日救兵到,乃解。超卒殺異子月。凡自兾城之難,至于祁山,昂出九奇,異輒參焉。}
 
 
 
 
 太祖征漢中,以阜為益州刺史。還,拜金城太守,未發,轉武都太守。郡濵蜀漢,阜請依龔遂故事,安之而已。會劉備遣張飛、馬超等從沮道趣下辯,而氐雷定等七部萬餘落反應之。太祖遣都護曹洪禦超等,超等退還。洪置酒大會,令女倡著羅縠之衣,蹋鼓,一坐皆笑。阜厲聲責洪曰:「男女之別,國之大節,何有於廣坐之中裸女人形體!雖桀、紂之亂,不甚於此。」遂奮衣辭出。洪立罷女樂,請阜還坐,肅然憚焉。
 
 
 
 
 及劉備取漢中以逼下辯,太祖以武都孤遠,欲移之,恐吏民戀土。阜威信素著,前後徙民、氐,使居京兆、扶風、天水界者萬餘戶,徙郡小槐里,百姓襁負而隨之。為政舉大綱而已,下不忍欺也。文帝問侍中劉曄等:「武都太守何如人也?」皆稱阜有公輔之節。未及用,會帝崩。在郡十餘年,徵拜城門校尉。
 
 
 
 
 阜常見明帝著𧛕,被縹綾半裦袖,阜問帝曰:「此於禮何法服也?」帝默然不荅,自是不法服不以見阜。
 
 
 
 
 遷將作大匠。時初治宮室,發美女以充後庭,數出入弋獵。秋,大雨震電,多殺鳥雀。阜上疏曰:「臣聞明主在上,羣下盡辭。堯、舜聖德,求非索諫;大禹勤功,務卑宮室;成湯遭旱,歸咎責己;周文刑於寡妻,以御家邦;漢文躬行節儉,身衣弋綈:此皆能昭令問,貽厥孫謀者也。伏惟陛下奉武皇帝開拓之大業,守文皇帝克終之元緒,誠宜思齊往古聖賢之善治,總觀季世放盪之惡政。所謂善治者,務儉約、重民力也;所謂惡政者,從心恣欲,觸情而發也。惟陛下稽古世代之初所以明赫,及季世所以衰弱至于泯滅,近覽漢末之變,足以動心誡懼矣。曩使桓、靈不廢高祖之法,文、景之恭儉,太祖雖有神武,於何所施其能邪?而陛下何由處斯尊哉?今吳、蜀未定,軍旅在外,願陛下動則三思,慮而後行,重慎出入,以往鑒來,言之若輕,成敗甚重。頃者天雨,又多卒暴雷電非常,至殺鳥雀。天地神明,以王者為子也,政有不當,則見災譴。克己內訟,聖人所記。惟陛下慮患無形之外,慎萌纖微之初,法漢孝文出惠帝美人,令得自嫁;頃所調送小女,遠聞不令,宜為後圖。諸所繕治,務從約節。書曰:『九族旣睦,恊和萬國。』事思厥宜,以從中道,精心計謀,省息費用。吳、蜀以定,爾乃上安下樂,九親熈熈。如此以往,祖考心歡,堯舜其猶病諸。今宜開大信於天下,以安衆庶,以示遠人。」時雍丘王植怨於不齒,藩國至親,法禁峻密,故阜又陳九族之義焉。詔報曰:「閒得密表,先陳往古明王聖主,以諷闇政,切至之辭,款誠篤實。退思補過,將順匡救,備至悉矣。覽思苦言,吾甚嘉之。」
 
 
 
 
 後遷少府。是時大司馬曹真伐蜀,遇雨不進。阜上疏曰:「昔文王有赤烏之符,而猶日仄不暇食;武王白魚入舟,君臣變色。而動得吉瑞,猶尚憂懼,況有災異而不戰竦者哉?今吳、蜀未平,而天屢降變,陛下宜深有以專精應荅,側席而坐,思示遠以德,綏邇以儉。閒者諸軍始進,便有天雨之患,稽閡山險,以積日矣。轉運之勞,擔負之苦,所費以多,若有不繼,必違本圖。傳曰:『見可而進,知難而退,軍之善政也。』徒使六軍困於山谷之間,進無所略,退又不得,非主兵之道也。武王還師,殷卒以亡,知天期也。今年凶民饑,宜發明詔損膳減服,技巧珍玩之物,皆可罷之。昔邵信臣為少府於無事之世,而奏罷浮食;今者軍用不足,益宜節度。」帝即召諸軍還。
 
 
 
 
 後詔大議政治之不便於民者,阜議以為:「致治在於任賢,興國在於務農。若舍賢而任所私,此忘治之甚者也。廣開宮館,高為臺榭,以妨民務,此害農之甚者也。百工不敦其器,而競作奇巧,以合上欲,此傷本之甚者也。孔子曰:『苛政甚於猛虎。』今守功文俗之吏,為政不通治體,苟好煩苛,此亂民之甚者也。當今之急,宜去四甚,並詔公卿郡國,舉賢良方正敦樸之士而選用之,此亦求賢之一端也。」
 
 
 
 
 阜又上疏欲省宮人諸不見幸者,乃召御府吏問後宮人數。吏守舊令,對曰:「禁密,不得宣露。」阜怒,杖吏一百,數之曰:「國家不與九卿為密,反與小吏為密乎?」帝聞而愈敬憚阜。
 
 
 
 
 帝愛女淑,未期而夭,帝痛之甚,追封平原公主,立廟洛陽,葬於南陵。將自臨送,阜上疏曰:「文皇帝、武宣皇后崩,陛下皆不送葬,所以重社稷、備不虞也。何至孩抱之赤子而可送葬也哉?」帝不從。
 
 
帝旣新作許宮,又營洛陽宮殿觀閣。阜上疏曰:「堯尚茅茨而萬國安其居,禹卑宮室而天下樂其業;及至殷、周,或堂崇三尺,度以九筵耳。古之聖帝明王,未有極宮室之高麗以彫弊百姓之財力者也。桀作琁室、象廊,紂為傾宮、鹿臺,以喪其社稷,楚靈以築章華而身受其禍;秦始皇作阿房而殃及其子,天下叛之,二世而滅。夫不度萬民之力,以從耳目之欲,未有不亡者也。陛下當以堯、舜、禹、湯、文、武為法則,夏桀、殷紂、楚靈、秦皇為深誡。高高在上,實監后德。慎守天位,以承祖考,巍巍大業,猶恐失之。不夙夜敬止,允恭卹民,而乃自暇自逸,惟宮臺是侈是飾,必有顛覆危亡之禍。易曰:『豐其屋,蔀其家,闚其戶,閴其無人。』王者以天下為家,言豐屋之禍,至於家無人也。方今二虜合從,謀危宗廟,十萬之軍,東西奔赴,邊境無一日之娛;農夫廢業,民有饑色。陛下不以是為憂,而營作宮室,無有已時。使國亡而臣可以獨存,臣又不言也;
 \gezhu{臣松之以為忠至之道,以亡己為理。是以匡救其惡,不為身計。而阜表云「使國亡而臣可以獨存,臣又不言也」,此則發憤為己,豈為國哉?斯言也,豈不傷讜烈之義,為一表之病乎!}
 君作元首,臣為股肱,存亡一體,得失同之。孝經曰:『天子有爭臣七人,雖無道不失其天下。』臣雖駑怯,敢忘爭臣之義?言不切至,不足以感寤陛下。陛下不察臣言,恐皇祖烈考之祚,將墜于地。使臣身死有補萬一,則死之日,猶生之年也。謹叩棺沐浴,伏俟重誅。」奏御,天子感其忠言,手筆詔荅。每朝廷會議,阜常侃然以天下為己任。數諫爭,不聽,乃屢乞遜位,未許。會卒,家無餘財。孫豹嗣。
 
 
\end{pinyinscope}