\article{裴潛傳}
\begin{pinyinscope}


裴潛字文行,河東聞喜人也。


\gezhu{魏略曰:潛世為著姓。父茂,仕靈帝時,歷縣令、郡守、尚書。建安初,以奉使率導關中諸將討李傕有功,封列侯。潛少不脩細行,由此為父所不禮。}
避亂荊州,劉表待以賔禮。潛私謂所親王粲、司馬芝曰:「劉牧非霸王之才,乃欲西伯自處,其敗無日矣。」遂南適長沙。太祖定荊州,以潛參丞相軍事,出歷三縣令,入為倉曹屬。太祖問潛曰:「卿前與劉備俱在荊州,卿以備才略何如?」潛曰:「使居中國,能亂人而不能為治也。若乘間守險,足以為一方主。」


時代郡大亂,以潛為代郡太守。烏丸王及其大人,凡三人,各自稱單于,專制郡事。前太守莫能治正,太祖欲授潛精兵以鎮討之。潛辭曰:「代郡戶口殷衆,士馬控弦,動有萬數。單于自知放橫日乆,內不自安。今多將兵往,必懼而拒境,少將則不見憚。宜以計謀圖之,不可以兵威迫也。」遂單車之郡。單于驚喜。潛撫之以靜。單于以下脫帽稽顙,悉還前後所略婦女、器械、財物。潛案誅郡中大吏與單于為表裏者郝溫、郭端等十餘人,北邊大震,百姓歸心。在代三年,還為丞相理曹掾,太祖襃稱治代之功,潛曰:「潛於百姓雖寬,於諸胡為峻。今計者必以潛為理過嚴,而事加寬惠;彼素驕恣,過寬必弛,旣弛又將攝之以法,此訟爭所由生也。以勢料之,代必復叛。」於是太祖深悔還潛之速。後數十日,三單于反問至,乃遣鄢陵侯彰為驍騎將軍征之。


潛出為沛國相,遷兖州刺史。太祖次摩陂,歎其軍陳齊整,特加賞賜。文帝踐阼,入為散騎常侍。出為魏郡、潁川典農中郎將,奏通貢舉,比之郡國,由是農官進仕路泰。遷荊州刺史,賜爵關內侯。明帝即位,入為尚書。出為河南尹,轉太尉軍師、大司農,封清陽亭侯,邑二百戶。入為尚書令,奏正分職,料簡名實,出事使斷官府者百五十餘條。喪父去官,拜光祿大夫。正始五年薨,追贈太常,謚曰貞侯。
\gezhu{魏略曰:時遠近皆云當為公,會病亡。始潛自感所生微賤,無舅氏,又為父所不禮,即折節仕進,雖多所更歷,清省恪然。每之官,不將妻子。妻子貧乏,織藜芘以自供。又潛為兖州時,甞作一胡牀,及其去也,留以掛柱。又以父在京師,出入薄軬車;羣弟之田廬,常步行;家人小大或并日而食;其家教上下相奉,事有似於石奮。其履檢校度,自魏興少能及者。潛為人材博,有雅容,然但如此而已,終無所推進,故世歸其絜而不宗其餘。}
子秀嗣。遺令儉葬,墓中惟置一坐,瓦器數枚,其餘一無所設。秀,咸熈中為尚書僕射。
\gezhu{文章叙錄曰:秀字季彥。弘通博濟,八歲能屬文,遂知名。大將軍曹爽辟。喪父服終,推財與兄弟。年二十五,遷黃門侍郎。爽誅,以故吏免。遷衞國相,累遷散騎常侍、尚書僕射令、光祿大夫。咸熈中,晉文王始建五等,命秀典為制度,封廣川侯。晉室受禪,進左光祿大夫,改封鉅鹿公,遷司空。著易及樂論,又畫地域圖十八篇,傳行於世。盟會圖及典治官制皆未成。年四十八,泰始七年薨,謚元公,配食宗廟。少子頠,字逸民,襲封。荀綽兾州記曰:頠為人弘雅有遠識,博學稽古,履行高整,自少知名。歷位太子中庶子、侍中尚書。元康末,為尚書左僕射。趙王倫以其望重,畏而惡之,知其不與賈氏同心,猶被枉害。}
\gezhu{臣松之案陸機惠帝起居注稱「頠雅有遠量,當朝名士也」,又曰「民之望也」。頠理具淵博,贍於論難,著崇有、貴無二論,以矯虛誕之弊,文辭精富,為世名論。子嵩,字道文。荀綽稱嵩有父祖風。為中書郎,早卒。頠從父弟邈,字景聲,有儁才,為太傅司馬越從事中郎,假節監中外營諸軍事。}
\gezhu{潛少弟徽,字文季,兾州刺史。有高才遠度,善言玄妙。事見荀粲、傅嘏、王弼、管輅諸傳。徽長子黎,字伯宗,一名演,游擊將軍。次康,字仲豫,太子左衞率。次楷,字叔則,侍中中書令、光祿大夫、開府。次綽,字季舒,黃門侍郎,早卒,追贈長水校尉。康、楷、綽皆為名士,而楷才望最重。}
\gezhu{晉諸公贊曰:康有弘量,綽以明達為稱,楷少與琅邪王戎俱為掾發名,鍾會致之大將軍司馬文王曰:「裴楷清通,王戎簡要。」文王即辟為掾,進歷顯位。謝鯤為樂廣傳,稱楷儁朗有識具,當時獨步。黎子苞,秦州刺史。康子純,黃門侍郎。次盾,徐州刺史。次郃,有器望。晉元帝為安東將軍,郃為長史,侍中王曠與司馬越書曰:「裴郃在此,雖不治事,然識量弘淹,此下人士大敬附之。」次廓,中壘將軍。楷子瓚,中書郎。次憲,豫州刺史。綽子遐,太傅主簿。瓚、遐並有盛名,早卒。}
\gezhu{晉諸公贊稱憲有清識。}
\gezhu{魏略列傳以徐福、嚴幹、李義、張旣、游楚、梁習、趙儼、裴潛、韓宣、黃朗十人共卷,其旣、習、儼、潛四人自有傳,徐福事在諸葛亮傳,游楚事在張旣傳。餘韓等四人載之於後。}
\gezhu{嚴幹字公仲,李義字孝懿,皆馮翊東縣人也。馮翊東縣舊無冠族,故二人並單家,其器性皆重厚。當中平末,同年二十餘,幹好擊劒,義好辦護喪事。馮翊甲族桓、田、吉、郭及故侍中鄭文信等,頗以其各有器實,共紀識之。會三輔亂,人多流宕,而幹、義不去,與諸知故相浮沉,採樵自活。逮建安初,關中始開。詔分馮翊西數縣為左內史郡,治高陵;以東數縣為本郡,治臨晉。義於縣分當西屬,義謂幹曰:「西縣兒曹,不可與爭坐席,今當共作方牀耳。」遂相附結,皆仕東郡為右職。司隷辟幹,不至。歲終,郡舉幹孝廉,義上計掾。義留京師,為平陵令,遷宂從僕射,遂歷顯職。逮魏封十郡,請義以為軍祭酒,又為魏尚書左僕射。及文帝即位,拜諫議大夫、執金吾衞尉,卒官。義子豐,字宣國,見夏侯玄傳。幹以孝廉拜蒲阪令,病,去官。復舉至孝,為公車司馬令。為州所請,詔拜議郎,還參州事。會以建策捕高幹,又追錄前討郭援功,封武鄉侯,遷弘農太守。及馬超反,幹郡近超,民人分散。超破,為漢陽太守。遷益州刺史,以道不通,黃初中,轉為五官中郎將。明帝時,遷永安太僕,數歲卒。始李義以直道推誠於人,故於時陳羣等與之齊好。雖無他材力,而終仕進不頓躓。幹從破亂之後,更折節學問,特善春秋公羊。司隷鍾繇不好公羊而好左氏,謂左氏為太官,而謂公羊為賣餅家,故數與幹共辯析長短。繇為人機捷,善持論,而幹訥口,臨時屈無以應。繇謂幹曰:「公羊高竟為左丘明服矣。」幹曰:「直故吏為明使君服耳,公羊未肯也。」}
\gezhu{韓宣字景然,勃海人也。為人短小。建安中,丞相召署軍謀掾,宂散在鄴。嘗於鄴出入宮,於東掖門內與臨菑侯植相遇。時天新雨,地有泥潦。宣欲避之,閡潦不得去。乃以扇自障,住於道邊。植嫌宣旣不去,又不為禮,乃駐車,使其常從問宣何官?宣云:「丞相軍謀掾也。」植又問曰:「應得唐突列侯否?」宣曰:「春秋之義,王人雖微,列于諸侯之上,未聞宰士而為下士諸侯禮也。」植又曰:「即如所言,為人父吏,見其子應有禮否?」宣又曰:「於禮,臣、子一例也,而宣年又長。」植知其枝柱難窮,乃釋去,具為太子言,以為辯。黃初中,為尚書郎,嘗以職事當受罰於殿前,已縛,束杖未行。文帝輦過,問:「此為誰?」左右對曰:「尚書郎勃海韓宣也。」帝追念前臨菑侯所說,乃寤曰:「是子建所道韓宣邪!」特原之,遂解其縛。時天大寒,宣前以當受杖,豫脫袴,纏褌靣縛;及其原,褌腰不下,乃趨而去。帝目而送之,笑曰:「此家有瞻諦之士也。」後出為清河、東郡太守。明帝時,為尚書大鴻臚,數歲卒。宣前後當官,在能否之間,然善以己恕人。始南陽韓曁以宿德在宣前為大鴻臚,曁為人賢,及宣在後亦稱職,故鴻臚中為之語曰:「大鴻臚,小鴻臚,前後治行曷相如。」案本志,宣名都不見,惟魏略有此傳,而世語列於名臣之流。}
\gezhu{黃朗字文達,沛郡人也。為人弘通有性實。父為本縣卒,朗感其如此,抗志游學,由是為方國及其郡士大夫所禮異。特與東平右姓王惠陽為碩交,惠陽親拜朗母於牀下。朗始仕黃初中,為長吏,遷長安令,會喪母不赴,復為魏令,遷襄城典農中郎將、涿郡太守。以明帝時疾病卒。始朗為君長,自以父故,常忌不呼鈴下伍伯,而呼其姓字,至於忿怒,亦終不言。朗旣仕至二千石,而惠陽亦歷長安令、酒泉太守。故時人謂惠陽外似麤疏而內堅密,能不顧朗之本末,事朗母如己母,為通度也。}
\gezhu{魚豢曰:世稱君子之德其猶龍乎,蓋以其善變也。昔長安巿儈有劉仲始者,一為巿吏所辱,乃感激,蹋其尺折之,遂行學問,經門行脩,流名海內。後以有道徵,不肯就,衆人歸其高。余以為前世偶有此耳,而今徐、嚴復參之,若皆非似龍之志也,其何能至於此哉?李推至道,張工度主,韓見識異,黃能拔萃,各著根於石上,而垂陰乎千里,亦未為易也。游翁慷慨,展布腹心,全軀保郡,見延帝王,又放陸生,優游宴戲,亦一實也。梁、趙及裴,雖張、楊不足,至於檢己,老而益明,亦難能也。}


評曰:和洽清和幹理,常林素業純固,楊俊人倫行義,杜襲溫粹識統,趙儼剛毅有度,裴潛平恒貞幹,皆一世之美士也。至林能不繫心於三司,以大夫告老,美矣哉!


\end{pinyinscope}