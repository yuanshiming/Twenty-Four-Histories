\article{王粲傳}

\begin{pinyinscope}


王粲字仲宣,山陽高平人也。曾祖父龔,祖父暢,皆為漢三公。


\gezhu{張璠漢紀曰:龔字伯宗,有高名於天下。順帝時為太尉。初,山陽太守薛勤喪妻不哭,將殯,臨之曰:「幸不為夭,復何恨哉?」及龔妻卒,龔與諸子並杖行服,時人或兩譏焉。暢字叔茂,名在八俊。靈帝時為司空,以水災免,而李膺亦免歸故郡,二人以直道不容當時。天下以暢、膺為高士,諸危言危行之徒皆推宗之,願涉其流,惟恐不及。會連有災異,而言事者皆言三公非其人,宜因其變,以暢、膺代之,則禎祥必至。由是宦豎深怨之,及膺誅死而暢遂廢,終于家。}
父謙,為大將軍何進長史。進以謙名公之冑,欲與為婚,見其二子,使擇焉。謙弗許。以疾免,卒于家。


獻帝西遷,粲徙長安,左中郎將蔡邕見而奇之。時邕才學顯著,貴重朝廷,常車騎填巷,賔客盈坐。聞粲在門,倒屣迎之。粲至,年旣幼弱,容狀短小,一坐盡驚。邕曰:「此王公孫也,有異才,吾不如也。吾家書籍文章,盡當與之。」年十七,司徒辟,詔除黃門侍郎,以西京擾亂,皆不就。乃之荊州依劉表。表以粲貌寢而體弱通侻,不甚重也。
\gezhu{臣松之曰:貌寢,謂貌負其實也。通侻者,簡易也。}
表卒。粲勸表子琮,令歸太祖。
\gezhu{文士傳載粲說琮曰:「僕有愚計,願進之於將軍,可乎?」琮曰:「吾所願聞也。」粲曰:「天下大亂,豪傑並起,在倉卒之際,彊弱未分,故人各各有心耳。當此之時,家家欲為帝王,人人欲為公侯。觀古今之成敗,能先見事機者,則恒受其福。今將軍自度,何如曹公邪?」琮不能對。粲復曰:「如粲所聞,曹公故人傑也。雄略冠時,智謀出世,摧袁氏於官渡,驅孫權於江外,逐劉備於隴右,破烏丸於白登,其餘梟夷蕩定者,往往如神,不可勝計。今日之事,去就可知也。將軍能聽粲計,卷甲倒戈,應天順命,以歸曹公,曹公必重德將軍。保己全宗,長享福祚,垂之後嗣,此萬全之策也。粲遭亂流離,託命此州,蒙將軍父子重顧,敢不盡言!」琮納其言。}
\gezhu{臣松之案:孫權自此以前,尚與中國和同,未嘗交兵,何云「驅權於江外」乎?魏武以十三年征荊州,劉備却後數年方入蜀,備身未嘗涉於關、隴。而於征荊州之年,便云逐備於隴右,旣已乖錯;又白登在平城,亦魏武所不經,北征烏丸,與白登永不相豫。以此知張隲假偽之辭,而不覺其虛之自露也。凡隲虛偽妄作,不可覆疏,如此類者,不可勝紀。}
太祖辟為丞相掾,賜爵關內侯。太祖置酒漢濵,粲奉觴賀曰:「方今袁紹起河北,杖大衆,志兼天下,然好賢而不能用,故奇士去之。劉表雍容荊楚,坐觀時變,自以為西伯可規。士之避亂荊州者,皆海內之儁傑也;表不知所任,故國危而無輔。明公定兾州之日,下車即繕其甲卒,收其豪傑而用之,以橫行天下;及平江、漢,引其賢儁而置之列位,使海內回心,望風而願治,文武並用,英雄畢力,此三王之舉也。」後遷軍謀祭酒。魏國旣建,拜侍中。博物多識,問無不對。時舊儀廢弛,興造制度,粲恒典之。
\gezhu{摯虞決疑要注曰:漢末喪亂,絕無玉珮。魏侍中王粲識舊珮,始復作之。今之玉珮,受法於粲也。}


初,粲與人共行,讀道邊碑,人問曰:「卿能闇誦乎?」曰:「能。」因使背而誦之,不失一字。觀人圍棊,局壞,粲為覆之。棊者不信,以帊蓋局,使更以他局為之。用相比校,不誤一道。其彊記默識如此。性善筭,作筭術,略盡其理。善屬文,舉筆便成,無所改定,時人常以為宿構;然正復精意覃思,亦不能加也。
\gezhu{典略曰;粲才旣高,辯論應機。鍾繇、王朗等雖各為魏卿相,至於朝廷奏議,皆閣筆不能措手。}
著詩、賦、論、議垂六十篇。建安二十一年,從征吳。二十二年春,道病卒,時年四十一。粲二子,為魏諷所引,誅。後絕。
\gezhu{文章志曰:太祖時征漢中,聞粲子死,歎曰:「孤若在,不使仲宣無後。」}


始文帝為五官將,及平原侯植皆好文學。粲與北海徐幹字偉長、廣陵陳琳字孔璋、陳留阮瑀字元瑜、汝南應瑒字德璉、
\gezhu{瑒,音徒哽反,一音暢。}
東平劉楨字公幹並見友善。


幹為司空軍謀祭酒掾屬,五官將文學。
\gezhu{先賢行狀曰:幹清玄體道,六行脩備,聦識洽聞,操翰成章,輕官忽祿,不耽世榮。建安中,太祖特加旌命,以疾休息。後除上艾長,又以疾不行。}


琳前為何進主簿。進欲誅諸宦官,太后不聽,進乃召四方猛將,並使引兵向京城,欲以劫恐太后。琳諫進曰:「易稱『即鹿無虞』。諺有『掩目捕雀』。夫微物尚不可欺以得志,況國之大事,其可以詐立乎?今將軍總皇威,握兵要,龍驤虎步,高下在心;以此行事,無異於鼓洪爐以燎毛髮。但當速發雷霆,行權立斷,違經合道,天人順之;而反釋其利器,更徵於他。大兵合聚,彊者為雄,所謂倒持干戈,授人以柄;必不成功,祇為亂階。」進不納其言,竟以取禍。琳避難兾州,袁紹使典文章。袁氏敗,琳歸太祖。太祖謂曰:「卿昔為本初移書,但可罪狀孤而已,惡惡止其身,何乃上及父祖邪?」琳謝罪,太祖愛其才而不咎。


瑀少受學於蔡邕。建安中都護曹洪欲使掌書記,瑀終不為屈。太祖並以琳、瑀為司空軍謀祭酒,管記室,
\gezhu{文士傳曰:太祖雅聞瑀名,辟之,不應,連見偪促,乃逃入山中。太祖使人焚山,得瑀,送至,召入。太祖時征長安,大延賔客,怒瑀不與語,使就技人列。瑀善解音,能鼓琴,遂撫弦而歌,因造歌曲曰:「奕奕天門開,大魏應期運。青蓋巡九州,在東西人怨。士為知己死,女為恱者玩。恩義苟敷暢,他人焉能亂?」為曲旣捷,音聲殊妙,當時冠坐,太祖大恱。}
\gezhu{臣松之案魚氏典略、摯虞文章志並云瑀建安初辭疾避役,不為曹洪屈。得太祖召,即投杖而起。不得有逃入山中,焚之乃出之事也。}
\gezhu{又典略載太祖初征荊州,使瑀作書與劉備,及征馬超,又使瑀作書與韓遂,此二書今具存。至長安之前,遂等破走,太祖始以十六年得入關耳。而張隲云初得瑀時太祖在長安,此又乖矣。瑀以十七年卒,太祖十八年策為魏公,而云瑀歌舞辭稱「大魏應期運」,愈知甚妄。又其辭云「他人焉能亂」,了不成語。瑀之吐屬,必不如此。}
軍國書檄,多琳、瑀所作也。
\gezhu{典略曰:琳作諸書及檄,草成呈太祖。太祖先苦頭風,是日疾發,卧讀琳所作,翕然而起曰:「此愈我病。」數加厚賜。太祖嘗使瑀作書與韓遂,時太祖適近出,瑀隨從,因於馬上具草,書成呈之。太祖攬筆欲有所定,而竟不能增損。}
琳徙門下督,瑀為倉曹掾屬。


瑒、楨各被太祖辟,為丞相掾屬。瑒轉為平原侯庶子,後為五官將文學。
\gezhu{華嶠漢書曰:瑒祖奉,字世叔。才敏善諷誦,故世稱「應世叔讀書,五行俱下」。著後序十餘篇,為世儒者。延熹中,至司隷校尉。子劭字仲遠,亦博學多識,尤好事。諸所撰述風俗通等,凡百餘篇,辭雖不典,世服其博聞。}
\gezhu{續漢書曰:劭又著中漢輯叙、漢官儀及禮儀故事,凡十一種,百三十六卷。朝廷制度,百官儀式,所以不亡者,由劭記之。官至泰山太守。劭弟珣,字季瑜,司空掾,即瑒之父。}
楨以不敬被刑,刑竟署吏。
\gezhu{文士傳曰:楨父名梁,字曼山,一名恭。少有清才,以文學見貴,終於野王令。}
\gezhu{典略曰:文帝嘗賜楨廓落帶,其後師死,欲借取以為像,因書嘲楨云:「夫物因人為貴。故在賤者之手,不御至尊之側。今雖取之,勿嫌其不反也。」楨荅曰:「楨聞荊山之璞,曜元后之寶;隨侯之珠,燭衆士之好;南垠之金,登窈窕之首;鼲貂之尾,綴侍臣之幘:此四寶者,伏朽石之下,潛汙泥之中,而揚光千載之上,發彩疇昔之外,亦皆未能初自接於至尊也。夫尊者所服,卑者所脩也;貴者所御,賤者所先也。故夏屋初成而大匠先立其下,嘉禾始熟而農夫先嘗其粒。恨楨所帶,無他妙飾,若實殊異,尚可納也。」楨辭旨巧妙皆如是,由是特為諸公子所親愛。其後太子嘗請諸文學,酒酣坐歡,命夫人甄氏出拜。坐中衆人咸伏,而楨獨平視。太祖聞之,乃收楨,減死輸作。}
咸著文賦數十篇。


瑀以十七年卒。幹、琳、瑒、楨二十二年卒。文帝書與元城令吳質曰:「昔年疾疫,親故多離其災,徐、陳、應、劉,一時俱逝。觀古今文人,類不護細行,鮮能以名節自立。而偉長獨懷文抱質,恬淡寡欲,有箕山之志,可謂彬彬君子矣。著中論二十餘篇,辭義典雅,足傳于後。德璉常斐然有述作意,其才學足以著書,美志不遂,良可痛惜!孔璋章表殊健,微為繁富。公幹有逸氣,但未遒耳。元瑜書記翩翩,致足樂也。仲宣獨自善於辭賦,惜其體弱,不起其文;至於所善,古人無以遠過也。昔伯牙絕絃於鍾期,仲尼覆醢于子路,痛知音之難遇,傷門人之莫逮也。諸子但為未及古人,自一時之儁也。」
\gezhu{典論曰:今之文人,魯國孔融、廣陵陳琳、山陽王粲、北海徐幹、陳留阮瑀、汝南應瑒、東平劉楨,斯七子者,於學無所遺,於辭無所假,咸自以騁騏驥於千里,仰齊足而並馳。粲長於辭賦。幹時有逸氣,然非粲匹也。如粲之初征、登樓、槐賦、征思,幹之玄猨、漏巵、圓扇、橘賦,雖張、蔡不過也,然於他文未能稱是。琳、瑀之章表書記,今之儁也。應瑒和而不壯;劉楨壯而不密。孔融體氣高妙,有過人者,然不能持論,理不勝辭,至于雜以嘲戲;及其所善,揚、班之儔也。}


自潁川邯鄲淳、
\gezhu{魏略曰:淳一名笁,字子叔。博學有才章,又善蒼、雅、蟲、篆、許氏字指。初平時,從三輔客荊州。荊州內附,太祖素聞其名,召與相見,甚敬異之。時五官將博延英儒,亦宿聞淳名,因啟淳欲使在文學官屬中。會臨菑侯植亦求淳,太祖遣淳詣植。植初得淳甚喜,延入坐,不先與談。時天暑熱,植因呼常從取水自澡訖,傅粉。遂科頭拍袒,胡舞五椎鍛,跳丸擊劒,誦俳優小說數千言訖,謂淳曰:「邯鄲生何如邪?」於是乃更著衣幘,整儀容,與淳評說混元造化之端,品物區別之意,然後論皇羲以來賢聖名臣烈士優劣之差,次頌古今文章賦誄及當官政事宜所先後,又論用武行兵倚伏之勢。乃命廚宰,酒炙交至,坐席默然,無與伉者。及暮,淳歸,對其所知歎植之材,謂之「天人」。而于時世子未立。太祖俄有意於植,而淳屢稱植材。由是五官將頗不恱。及黃初初,以淳為博士給事中。淳作投壺賦千餘言奏之,文帝以為工,賜帛千匹。}
繁欽、
\gezhu{繁,音婆。典略曰:欽字休伯,以文才機辯,少得名於汝、潁。欽旣長於書記,又善為詩賦。其所與太子書,記喉轉意,率皆巧麗。為丞相主簿。建安二十三年卒。}
陳留路粹、
\gezhu{典略曰:粹字文蔚,少學於蔡邕。初平中,隨車駕至三輔。建安初,以高才與京兆嚴像擢拜尚書郎。像以兼有文武,出為揚州刺史。粹後為軍謀祭酒,與陳琳、阮瑀等典記室。及孔融有過,太祖使粹為奏,承指數致融罪,其大略言:「融昔在北海,見王室不寧,招合徒衆,欲圖不軌,言『我大聖之後也,而滅於宋。有天下者何必卯金刀』?」又云:「融為九列,不遵朝儀,禿巾微行,唐突宮掖。又與白衣禰衡言論放蕩,衡與融更相贊揚。衡謂融曰:『仲尼不死也。』融荅曰:『顏淵復生。』」凡說融諸如此輩,辭語甚多。融誅之後,人覩粹所作,無不嘉其才而畏其筆也。至十九年,粹轉為祕書令,從大軍至漢中,坐違禁賤請驢伏法。太子素與粹善,聞其死,為之歎惜。及即帝位,特用其子為長史。}
\gezhu{魚豢曰:尋省往者,魯連、鄒陽之徒,援譬引類,以解締結,誠彼時文辯之儁也。今覽王、繁、阮、陳、路諸人前後文旨,亦何昔不若哉?其所以不論者,時世異耳。余又竊怪其不甚見用,以問大鴻臚卿韋仲將。仲將云:「仲宣傷於肥戇,休伯都無格檢,元瑜病於體弱,孔璋實自麤疏,文尉性頗忿鷙,如是彼為,非徒以脂燭自煎糜也,其不高蹈,蓋有由矣。然君子不責備于一人,譬之朱漆,雖無楨幹,其為光澤亦壯觀也。」}
沛國丁儀、丁廙、弘農楊脩、河內苟緯等,亦有文采,而不在此七人之例。
\gezhu{儀、廙、脩事,並在陳思王傳。荀勗文章叙錄曰:緯字公高。少喜文學。建安中,召署軍謀掾、魏太子庶子,稍遷至散騎常侍、越騎校尉。年四十二,黃初四年卒。}


瑒弟璩,璩子貞,咸以文章顯。璩官至侍中。貞咸熈中參相國軍事。
\gezhu{文章叙錄曰:璩字休璉,博學好屬文,善為書記。文、明帝世,歷官散騎常侍。齊王即位,稍遷侍中、大將軍長史。曹爽秉政,多違法度,璩為詩以諷焉。其言雖頗諧合,多切時要,世共傳之。復為侍中,典著作。嘉平四年卒,追贈衞尉。貞字吉甫,少以才聞,能談論。正始中,夏侯玄盛有名勢,貞嘗在玄坐作五言詩,玄嘉玩之。舉高第,歷顯位。晉武帝為撫軍大將軍,以貞參軍事。晉室踐阼,遷太子中庶子、散騎常侍。又以儒學與太尉荀顗撰定新禮,事未施行。泰始五年卒。貞弟純。純子紹,永嘉中為黃門侍郎,為司馬越所殺。純弟秀。秀子詹,鎮南大將軍、江州刺史。}


瑀子籍,才藻艷逸,而倜儻放蕩,行己寡欲,以莊周為模則。官至步兵校尉。
\gezhu{籍字嗣宗。魏氏春秋曰:籍曠達不羈,不拘禮俗。性至孝,居喪雖不率常檢,而毀幾至滅性。兖州刺史王昶請與相見,終日不得與言,昶歎賞之,自以不能測也。太尉蔣濟聞而辟之,後為尚書郎、曹爽參軍,以疾歸田里。歲餘,爽誅,太傅及大將軍乃以為從事中郎。後朝論以其名高,欲顯崇之,籍以世多故,祿仕而已,聞步兵校尉缺,廚多美酒,營人善釀酒,求為校尉,遂縱酒昏酣,遺落世事。嘗登廣武,觀楚、漢戰處,乃歎曰:「時無英才,使豎子成名乎!」時率意獨駕,不由徑路,車迹所窮,輒慟哭而反。籍少時嘗遊蘇門山,蘇門山有隱者,莫知名姓,有竹實數斛、臼杵而已。籍從之,與談太古無為之道,及論五帝三王之義,蘇門生蕭然曾不經聽。籍乃對之長嘯,清韻響亮,蘇門生逌爾而笑。籍旣降,蘇門生亦嘯,若鸞鳳之音焉。至是,籍乃假蘇門先生之論以寄所懷。其歌曰:「日沒不周西,月出丹淵中,陽精蔽不見,陰光代為雄。亭亭在須臾,厭厭將復隆。富貴俯仰間,貧賤何必終。」又歎曰:「天地解兮六合開,星辰隕兮日月頹,我騰而上將何懷?」籍口不論人過,而自然高邁,故為禮法之士何曾等深所讎疾。大將軍司馬文王常保持之,卒以壽終。子渾字長成。世語曰:渾以閑澹寡欲,知名京邑。為太子庶子。早卒。}


時又有譙郡嵇康,文辭壯麗,好言老、莊,而尚奇任俠。至景元中,坐事誅。
\gezhu{康字叔夜。案嵇氏譜:康父昭,字子遠,督軍糧治書侍御史。兄喜,字公穆,晉揚州刺史、宗正。喜為康傳曰:「家世儒學,少有儁才,曠邁不羣,高亮任性,不脩名譽,寬簡有大量。學不師授,博洽多聞,長而好老、莊之業,恬靜無欲。性好服食,嘗採御上藥。善屬文論,彈琴詠詩,自足于懷抱之中。以為神仙者,禀之自然,非積學所致。至於導養得理,以盡性命,若安期、彭祖之倫,可以善求而得也;著養生篇。知自厚者所以喪其所生,其求益者必失其性,超然獨達,遂放世事,縱意於塵埃之表。撰錄上古以來聖賢、隱逸、遁心、遺名者,集為傳贊,自混沌至于管寧,凡百一十有九人,蓋求之於宇宙之內,而發之乎千載之外者矣。故世人莫得而名焉。」}
\gezhu{虞預晉書曰:康家本姓奚,會稽人。先自會稽遷于譙之銍縣,改為嵇氏,取嵇字之上山以為姓,蓋以志其本也。一曰銍有嵇山,家于其側,遂氏焉。}
\gezhu{魏氏春秋曰:康寓居河內之山陽縣,與之游者,未嘗見其喜慍之色。與陳留阮籍、河內山濤、河南向秀、籍兄子咸、琅邪王戎、沛人劉伶相與友善,遊於竹林,號為七賢。鍾會為大將軍所昵,聞康名而造之。會,名公子,以才能貴幸,乘肥衣輕,賔從如雲。康方箕踞而鍛,會至,不為之禮。康問會曰:「何所聞而來?何所見而去?」會曰:「有所聞而來,有所見而去。」會深銜之。大將軍嘗欲辟康。康旣有絕世之言,又從子不善,避之河東,或云避世。及山濤為選曹郎,舉康自代,康荅書拒絕,因自說不堪流俗,而非薄湯、武。大將軍聞而怒焉。初,康與東平呂昭子巽及巽弟安親善。會巽淫安妻徐氏,而誣安不孝,囚之。安引康為證,康義不負心,保明其事,安亦至烈,有濟世志力。鍾會勸大將軍因此除之,遂殺安及康。康臨刑自若,援琴而鼓,旣而歎曰:「雅音於是絕矣!」時人莫不哀之。初,康採藥於汲郡共北山中,見隱者孫登。康欲與之言,登默然不對。踰時將去,康曰:「先生竟無言乎?」登乃曰:「子才多識寡,難乎免於今之世。」及遭呂安事,為詩自責曰:「欲寡其過,謗議沸騰。性不傷物,頻致怨憎。昔慙柳下。今愧孫登。內負宿心,外赧良朋。」康所著諸文論六七萬言,皆為世所玩詠。}
\gezhu{康別傳云:孫登謂康曰:「君性烈而才儁,其能免乎?」稱康臨終之言曰:「袁孝尼嘗從吾學廣陵散,吾每固之不與。廣陵散於今絕矣!」與盛所記不同。}
\gezhu{又晉陽秋云:康見孫登,登對之長嘯,踰時不言。康辭還,曰:「先生竟無言乎?」登曰:「惜哉!」此二書皆孫盛所述,而自為殊異如此。}
\gezhu{康集目錄曰:登字公和,不知何許人,無家屬,於汲縣北山土窟中得之。夏則編草為裳,冬則被髮自覆。好讀易鼓琴,見者皆親樂之。每所止家,輒給其衣服食飲,得無辭讓。}
\gezhu{世語曰:毌丘儉反,康有力,且欲起兵應之,以問山濤,濤曰:「不可。」儉亦已敗。}
\gezhu{臣松之案本傳云康以景元中坐事誅,而干寶、孫盛、習鑿齒諸書,皆云正元二年,司馬文王反自樂嘉,殺嵇康、呂安。蓋緣世語云康欲舉兵應毌丘儉,故謂破儉便應殺康也。其實不然。山濤為選官,欲舉康自代,康書告絕,事之明審者也。案濤行狀,濤始以景元二年除吏部郎耳。景元與正元相覺七八年,以濤行狀檢之,如本傳為審。又鍾會傳亦云會作司隷校尉時誅康;會作司隷,景元中也。干寶云呂安兄巽善於鍾會,巽為相國掾,俱有寵於司馬文王,故遂抵安罪。尋文王以景元四年鍾、鄧平蜀後,始授相國位;若巽為相國掾時陷安,焉得以破毌丘儉年殺嵇、呂?此又干寶疏謬,自相違伐也。}
\gezhu{康子紹,字延祖,少知名。山濤啟以為祕書郎,稱紹平簡溫敏,有文思,又曉音,當成濟者。帝曰;「紹如此,便可以為丞,不足復為郎也。」遂歷顯位。}
\gezhu{晉諸公贊曰:紹與山濤子簡、弘農楊準同好友善,而紹最有忠正之情。以侍中從惠帝北伐成都王,王師敗績,百官奔走,惟紹獨以身扞衞,遂死於帝側。故累見襃崇,追贈太尉,謚曰忠穆公。}


景初中,下邳桓威出自孤微,年十八而著渾輿經,依道以見意。從齊國門下書佐、司徒署吏,後為安成令。


吳質,濟陰人,以文才為文帝所善,官至振威將軍,假節都督河北諸軍事,封列侯。
\gezhu{魏略曰:質字季重,以才學通博,為五官將及諸侯所禮愛;質亦善處其兄弟之間,若前世樓君卿之游五侯矣。及河北平定,五官將為世子,質與劉楨等並在坐席。楨坐譴之際,質出為朝歌長,後遷元城令。其後大將軍西征,太子南在孟津小城,與質書曰:「季重無恙!途路雖局,官守有限,願言之懷,良不可任。足下所治僻左,書問致簡,益用增勞。每念昔日南皮之游,誠不可忘。旣妙思六經,逍遙百氏,彈棊間設,終以博弈,高談娛心,哀箏順耳。馳騖北塲,旅食南館,浮甘瓜於清泉,沈朱李於寒水。皦日旣沒,繼以朗月,同乘並載,以游後園,輿輪徐動,賔從無聲,清風夜起,悲笳微吟,樂往哀來,淒然傷懷。余顧而言,茲樂難常,足下之徒,咸以為然。今果分別,各在一方。元瑜長逝,化為異物,每一念至,何時可言?方今蕤賔紀辰,景風扇物,天氣和暖,衆果具繁。時駕而游,北遵河曲,從者鳴笳以啟路,文學託乘於後車,節同時異,物是人非,我勞如何!今遣騎到鄴,故使枉道相過。行矣,自愛!」二十三年,太子又與質書曰:「歲月易得,別來行復四年。三年不見,東山猶歎其遠,況乃過之,思何可支?雖書疏往反,未足解其勞結。昔年疾疫,親故多離其災,徐、陳、應、劉,一時俱逝,痛何可言邪!昔日游處,行則同輿,止則接席,何嘗須臾相失!每至觴酌流行,絲竹並奏,酒酣耳熱,仰而賦詩。當此之時,忽然不自知樂也。謂百年己分,長共相保,何圖數年之間,零落略盡,言之傷心。頃撰其遺文,都為一集。觀其姓名,已為鬼錄,追思昔游,猶在心目,而此諸子化為糞壤,可復道哉!觀古今文人,類不護細行,鮮能以名節自立。而偉長獨懷文抱質,恬淡寡欲,有箕山之志,可謂彬彬君子矣。著中論二十餘篇,成一家之業,辭義典雅,足傳于後,此子為不朽矣。德璉常斐然有述作意,才學足以著書,美志不遂,良可痛惜。間歷觀諸子之文,對之抆淚,旣痛逝者,行自念也。孔璋章表殊健,微為繁富。公幹有逸氣,但未遒耳,至其五言詩,妙絕當時。元瑜書記翩翩,致足樂也。仲宣獨自善於辭賦,惜其體弱,不足起其文,至於所善,古人無以遠過也。昔伯牙絕絃於鍾期,仲尼覆醢於子路,愍知音之難遇,傷門人之莫逮也。諸子但為未及古人,自一時之儁也,今之存者已不逮矣。後生可畏,來者難誣,然吾與足下不及見也。行年已長大,所懷萬端,時有所慮,至乃通夕不瞑。何時復類昔日!已成老翁,但未白頭耳。光武言『年已三十,在軍十年,所更非一』,吾德雖不及,年與之齊。以犬羊之質,服虎豹之文,無衆星之明,假日月之光,動見觀瞻,何時易邪?恐永不復得為昔日游也。少壯真當努力,年一過往,何可攀援?古人思秉燭夜游,良有以也。頃何以自娛?頗復有所造述不?東望於邑,裁書叙心。」臣松之以本傳雖略載太子此書,美辭多被刪落,今故悉取魏略所述以備其文。太子即王位,又與質書曰:「南皮之游,存者三人,烈祖龍飛,或將或侯。今惟吾子,棲遲下土,從我游處,獨不及門。瓶罄罍恥,能無懷愧。路不云遠,今復相聞。」初,曹真、曹休亦與質等俱在渤海游處,時休、真亦以宗親並受爵封,出為列將,而質故為長史。王顧質有望,故稱二人以慰之。始質為單家,少游遨貴戚間,蓋不與鄉里相沈浮。故雖已出官,本國猶不與之士名。及魏有天下,文帝徵質,與車駕會洛陽。到,拜北中郎將,封列侯,使持節督幽、并諸軍事,治信都。太和中,入朝。質自以不為本郡所饒,謂司徒董昭曰:「我欲溺鄉里耳。」昭曰:「君且止,我年八十,不能老為君溺攢也。」}
\gezhu{世語曰:魏王嘗出征,世子及臨菑侯植並送路側。植稱述功德,發言有章,左右屬目,王亦恱焉。世子悵然自失,吳質耳曰:「王當行,流涕可也。」及辭,世子泣而拜,王及左右咸歔欷,於是皆以植辭多華,而誠心不及也。}
\gezhu{質別傳曰:帝嘗召質及曹休歡會,命郭后出見質等。帝曰:「卿仰諦視之。」其至親如此。質黃初五年朝京師,詔上將軍及特進以下皆會質所,大官給供具。酒酣,質欲盡歡。時上將軍曹真性肥,中領軍朱鑠性瘦,質召優,使說肥瘦。真負貴,恥見戲,怒謂質曰:「卿欲以部曲將遇我邪?」驃騎將軍曹洪、輕車將軍王忠言:「將軍必欲使上將軍服肥,即自宜為瘦。」真愈恚,拔刀瞋目,言:「俳敢輕脫,吾斬爾。」遂罵坐。質案劒曰:「曹子丹,汝非屠机上肉,吳質吞爾不搖喉,咀爾不搖牙,何敢恃勢驕邪?」鑠因起曰:「陛下使吾等來樂卿耳,乃至此邪!」質顧叱之曰:「朱鑠,敢壞坐!」諸將軍皆還坐。鑠性急,愈恚,還拔劒斬地。遂便罷也。及文帝崩,質思慕作詩曰:「愴愴懷殷憂,殷憂不可居。徙倚不能坐,出入步踟躕。念蒙聖主恩,榮爵與衆殊。自謂永終身,志氣甫當舒。何意中見棄,棄我歸黃壚。煢煢靡所恃,淚下如連珠。隨沒無所益,身死名不書。慷慨自僶俛,庶幾烈丈夫。」太和四年,入為侍中。時司空陳羣錄尚書事,帝初親萬機,質以輔弼大臣,安危之本,對帝盛稱「驃騎將軍司馬懿,忠智至公,社稷之臣也。陳羣從容之士,非國相之才,處重任而不親事。」帝甚納之。明日,有切詔以督責羣,而天下以司空不如長文,即羣,言無實也。質其年夏卒。質先以怙威肆行,謚曰醜侯。質子應仍上書論枉,至正元中乃改謚威侯。應字溫舒,晉尚書。應子康,字子仲,知名於時,亦至大位。}


\end{pinyinscope}