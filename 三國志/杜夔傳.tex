\article{杜夔傳}
\begin{pinyinscope}
 
 
 杜夔字公良,河南人也。以知音為雅樂郎,中平五年,疾去官。州郡司徒禮辟,以世亂奔荊州。荊州牧劉表令與孟曜為漢主合雅樂,樂備,表欲庭觀之,夔諫曰:「今將軍號為天子合樂,而庭作之,無乃不可乎!」表納其言而止。後表子琮降太祖,太祖以夔為軍謀祭酒,參太樂事,因令創制雅樂。
 
 
 
 
 夔善鍾律,聦思過人,絲竹八音,靡所不能,惟歌舞非所長。時散郎鄧靜、尹齊善詠雅樂,歌師尹胡能歌宗廟郊祀之曲,舞師馮肅、服養曉知先代諸舞,夔總統研精,遠考諸經,近采故事,教習講肄,備作樂器,紹復先代古樂,皆自夔始也。
 
 
 
 
 黃初中,為太樂令、恊律都尉。漢鑄鍾工柴玉巧有意思,形器之中,多所造作,亦為時貴人見知。夔令玉鑄銅鍾,其聲鈞清濁多不如法,數毀改作。玉甚厭之,謂夔清濁任意,頗拒捍夔。夔、玉更相白於太祖,太祖取所鑄鍾,雜錯更試,然後知夔為精而玉之妄也,於是罪玉及諸子,皆為養馬士。文帝愛待玉,又嘗令夔與左願等於賔客之中吹笙鼓琴,夔有難色,由是帝意不恱。後因他事繫夔,使願等就學,夔自謂所習者雅,仕宦有本,意猶不滿,遂黜免以卒。
 
 
 
 
 弟子河南邵登、張泰、桑馥,各至太樂丞,下邳陳頏司律中郎將。自左延年等雖妙於音,咸善鄭聲,其好古存正莫及夔。
 
 
\gezhu{時有扶風馬鈞,巧思絕世。傅玄序之曰:「馬先生,天下之名巧也,少而游豫,不自知其為巧也。當此之時,言不及巧,焉可以言知乎?為博士居貧,乃思綾機之變,不言而世人知其巧矣。舊綾機五十綜者五十躡,六十綜者六十躡,先生患其喪功費日,乃皆易以十二躡。其奇文異變,因感而作者,猶自然之成形,陰陽之無窮,此輪扁之對不可以言言者,又焉可以言校也。先生為給事中,與常侍高堂隆、驍騎將軍秦朗爭論於朝,言及指南車,二子謂古無指南車,記言之虛也。先生曰:『古有之,未之思耳,夫何遠之有!』二子哂之曰:『先生名鈞字德衡,鈞者器之模,而衡者所以定物之輕重;輕重無準而莫不模哉!』先生曰:『虛爭空言,不如試之易効也。』於是二子遂以白明帝,詔先生作之,而指南車成。此一異也,又不可以言者也,從是天下服其巧矣。居京都,城內有地,可以為園,患無水以灌之,乃作翻車,令童兒轉之,而灌水自覆,更入更出,其巧百倍於常。此二異也。其後人有上百戲者,能設而不能動也。帝以問先生:『可動否?』對曰:『可動。』帝曰:『其巧可益否?』對曰:『可益。』受詔作之。以大木彫構,使其形若輪,平地施之,潛以水發焉。設為女樂舞象,至令木人擊鼓吹簫;作山嶽,使木人跳丸擲劒,緣絙倒立,出入自在;百官行署,舂磨鬬雞,變巧百端。此三異也。先生見諸葛亮連弩,曰:『巧則巧矣,未盡善也。』言作之可令加五倍。又患發石車,敵人之於樓邊縣濕牛皮,中之則墮,石不能連屬而至。欲作一輪,縣大石數十,以機鼓輪為常,則以斷縣石飛擊敵城,使首尾電至。嘗試以車輪縣瓴甓數十,飛之數百步矣。有裴子者,上國之士也,精通見理,聞而哂之。乃難先生,先生口屈不對。裴子自以為難得其要,言之不已。傅子謂裴子曰:『子所長者言也,所短者巧也。馬氏所長者巧也,所短者言也。以子所長,擊彼所短,則不得不屈。以子所短,難彼所長,則彼有所不解者矣。夫巧,天下之微事也,有所不解而難之不已,其相擊刺,必已遠矣。心乖於內,口屈於外,此馬氏所以不對也。』傅子見安鄉侯,言及裴子之論,安鄉侯又與裴子同。傅子曰:『聖人具體備物,取人不以一揆也:有以神取之者,有以言取之者,有以事取之者。有以神取之者,不言而誠心先達,德行顏淵之倫是也。以言取之者,以變辯是非,言語宰我、子貢是也。以事取之者,若政事冉有、季路,文學子游、子夏。雖聖人之明盡物,如有所用,必有所試,然則試冉、季以政,試游、夏以學矣。游、夏猶然,況自此而降者乎!何者?懸言物理,不可以言盡也,施之於事,言之難盡而試之易知也。今若馬氏所欲作者,國之精器,軍之要用也。費十尋之木,勞二人之力,不經時而是非定。難試易驗之事而輕以言抑人異能,此猶以己智任天下之事,不易其道以御難盡之物,此所以多廢也。馬氏所作,因變而得是,則初所言者不皆是矣。其不皆是,因不用之,是不世之巧無由出也。夫同情者相妬,同事者相害,中人所不能免也。故君子不以人害人,必以考試為衡石;廢衡石而不用,此美玉所以見誣為石,荊和所以抱璞而哭之也。』於是安鄉侯悟,遂言之武安侯,武安侯忽之,不果試也。此旣易試之事,又馬氏巧名已定,猶忽而不察,況幽深之才,無名之璞乎?後之君子其鑒之哉!馬先生之巧,雖古公輸般、墨翟、王爾,近漢世張平子,不能過也。公輸般、墨翟皆見用於時,乃有益於世。平子雖為侍中,馬先生雖給事省中,俱不典工官,巧無益於世。用人不當其才,聞賢不試以事,良可恨也。」裴子者,裴秀。安鄉侯者,曹羲。武安侯者,曹爽也。}
 
 
\end{pinyinscope}