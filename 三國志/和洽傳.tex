\article{和洽傳}
\begin{pinyinscope}
 
 
 和洽字陽士,汝南西平人也。舉孝廉,大將軍辟,皆不就。袁紹在兾州,遣使迎汝南士大夫。洽獨以「兾州土平民彊,英桀所利,四戰之地。本初乘資,雖能彊大,然雄豪方起,全未可必也。荊州劉表無他遠志,愛人樂士,土地險阻,山夷民弱,易依倚也」。遂與親舊俱南從表,表以上客待之。洽曰:「所以不從本初,辟爭地也。昏世之主,不可黷近,乆而阽危,
 
 
\gezhu{臣松之案漢書文紀曰「阽於死亡」,食貨志曰「阽危若是」,注曰「阽音鹽,如屋簷,近邊欲墮之意也。」一曰「臨危曰阽」。}
 必有讒慝閒其中者。」遂南度武陵。
 
 
太祖定荊州,辟為丞相掾屬。時毛玠、崔琰並以忠清幹事,其選用先尚儉節。洽言曰:「天下大器,在位與人,不可以一節儉也。儉素過中,自以處身則可,以此節格物,所失或多。今朝廷之議,吏有著新衣、乘好車者,謂之不清;長吏過營,形容不飾,衣裘弊壞者,謂之廉潔。至令士大夫故汙辱其衣,藏其輿服;朝府大吏,或自挈壺飱以入官寺。夫立教觀俗,貴處中庸,為可繼也。今崇一概難堪之行以檢殊塗,勉而為之,必有疲瘁。古之大教,務在通人情而已。凡激詭之行,則容隱偽矣。」
 \gezhu{孫盛曰:昔先王御世,觀民設教,雖質文因時,損益代用,至於車服禮秩,貴賤等差,其歸一揆。魏承漢亂,風俗侈泰,誠宜仰思古制,訓以約簡,使奢不陵肆,儉足中禮,進無蜉蝣之刺,退免採莫之譏;如此則治道隆而頌聲作矣。夫矯枉過正則巧偽滋生,以克訓下則民志險隘,非聖王所以陶化民物,閑邪存誠之道。和洽之言,於是允矣。}
 
 
 
 
 魏國旣建,為侍中,後有白毛玠謗毀太祖,太祖見近臣,怒甚。洽陳玠素行有本,求案實其事。罷朝,太祖令曰:「今言事者白玠不但謗吾也。乃復為崔琰觖望。此損君臣恩義,妄為死友怨歎,殆不可忍也。昔蕭、曹與高祖並起微賤,致功立勳。高祖每在屈笮,二相恭順,臣道益彰,所以祚及後世也。和侍中比求實之,所以不聽,欲重參之耳。」洽對曰:「如言事者言,玠罪過深重,非天地所覆載。臣非敢曲理玠以枉大倫也,以玠出羣吏之中,特見拔擢,顯在首職,歷年荷寵,剛直忠公,為衆所憚,不宜有此。然人情難保,要宜考覈,兩驗其實。今聖恩垂含垢之仁,不忍致之于理,更使曲直之分不明,疑自近始。」太祖曰:「所以不考,欲兩全玠及言事者耳。」洽對曰:「玠信有謗上之言,當肆之巿朝;若玠無此,言事者加誣大臣以誤主聽;二者不加檢覈,臣竊不安。」太祖曰:「方有軍事,安可受人言便考之邪?狐射姑刺陽處父於朝,此為君之誡也。」
 
 
 
 
 太祖克張魯,洽陳便宜以時拔軍徙民,可省置守之費。太祖未納,其後竟徙民棄漢中。出為郎中令。文帝踐阼,為光祿勳,封安城亭侯。明帝即位,進封西陵鄉侯,邑二百戶。
 
 
 
 
 太和中,散騎常侍高堂隆奏:「時風不至,而有休廢之氣,必有司不勤職事以失天常也。」詔書謙虛引咎,博諮異同。洽以為「民稀耕少,浮食者多。國以民為本,民以穀為命。故廢一時之農,則失育命之本。是以先王務蠲煩費,以專耕農。自春夏已來,民窮於役,農業有廢,百姓嚻然,時風不至,未必不由此也。消復之術,莫大於節儉。太祖建立洪業,奉師徒之費,供軍賞之用,吏士豐於資食,倉府衍於穀帛,由不飾無用之宮,絕浮華之費,方今之要,固在息省勞煩之役,損除他餘之務,以為軍戎之儲。三邊守禦,宜在備豫。料賊虛實,蓄士養衆,筭廟勝之策,明攻取之謀,詳詢衆庶以求厥中。若謀不素定,輕弱小敵,軍人數舉,舉而無庸,所謂『恱武無震』,古人之誡也。」
 
 
轉為太常,清貧守約,至賣田宅以自給。明帝聞之,加賜穀帛。薨,謚曰簡侯。子禽嗣。
 \gezhu{禽音離。}
 禽弟適,才爽開濟,官至廷尉、吏部尚書。
 \gezhu{晉諸公贊曰:和嶠字長輿,適之子也。少知名,以雅重稱。常慕其舅夏侯玄之為人,厚自封植,嶷然不羣。於黃門郎遷中書令,轉尚書。愍懷太子初立,以嶠為少保,加散騎常侍。家產豐富,擬於王公,而性至儉吝。嶠同母弟郁,素無名,嶠輕侮之,以此為損。卒於官,贈光祿大夫。郁以公彊當世,致位尚書令。}
 
 
洽同郡許混者,許劭子也。清醇有鑒識,明帝時為尚書。
 \gezhu{劭字子將。汝南先賢傳曰:召陵謝子微,高才遠識,見劭年十八時,乃歎息曰:「此則希世出衆之偉人也。」劭始發明樊子昭於鬻幘之肆,出虞永賢於牧豎,召李淑才鄉閭之閒,擢郭子瑜鞌馬之吏,援楊孝祖,舉和陽士,茲六賢者,皆當世之令懿也。其餘中流之士,或舉之於淹滯,或顯之乎童齒,莫不賴劭顧歎之榮。凡所拔育,顯成令德者,不可殫記。其探擿偽行,抑損虛名,則周之單襄,無以尚也。劭宗人許栩,沈沒榮利,致位司徒。舉宗莫不匍匐栩門,承風而驅,官以賄成,惟劭不過其門。廣陵徐孟本來臨汝南,聞劭高名,請為功曹。饕餮放流,絜士盈朝。袁紹公族好名,為濮陽長,棄官來還,有副車從騎,將入郡界,紹乃歎曰:「吾之輿服,豈可使許子將見之乎?」遂單車而歸。辟公府掾,拜鄢陵令,方正徵,皆不就。避亂江南,所歷之國,必翔而後集。終于豫章,時年四十六。有子曰混,顯名魏世。}
 
 
\end{pinyinscope}