\article{孫慮傳}
\begin{pinyinscope}
 
 
 孫慮字子智,登弟也。少敏惠有才藝,權器愛之。黃武七年,封建昌侯。後二年,丞相雍等奏慮性聦體達,所尚日新,比方近漢,宜進爵稱王,權未許。乆之,尚書僕射存上疏曰:「帝王之興,莫不襃崇至親,以光羣后,故魯衞於周,寵冠諸侯,高帝五王,封列于漢,所以藩屏本朝,為國鎮衞。建昌侯慮稟性聦敏,才兼文武,於古典制,宜正名號。陛下謙光,未肯如舊,羣寮大小,咸用於邑。方今姦寇恣睢,金鼓未弭,腹心爪牙,惟親與賢。輒與丞相雍等議,咸以慮宜為鎮軍大將軍,授任偏方,以光大業。」權乃許之,於是假節開府,治半州。
 
 
\gezhu{吳書載權詔曰:「期運擾亂,凶邪肆虐,威罰有序,干戈不戢。以慮氣志休懿,武略夙昭,必能為國佐定大業,故授以上將之位,至以殊特之榮,寵以兵馬之勢,委以偏方之任。外欲威振敵虜,厭難萬里,內欲鎮撫遠近,慰卹將士,誠慮建功立事竭命之秋也。慮其內脩文德,外經武訓,持盈若沖,則滿而不溢。敬慎乃心,無忝所受。」}
 慮以皇子之尊,富於春秋,遠近嫌其不能留意。及至臨事,遵奉法度,敬納師友,過於衆望。年二十,嘉禾元年卒。無子,國除。
 
 
\end{pinyinscope}