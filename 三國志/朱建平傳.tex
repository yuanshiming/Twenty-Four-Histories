\article{朱建平傳}
\begin{pinyinscope}
 
 
 朱建平,沛國人也。善相術,於閭巷之間,效驗非一。太祖為魏公,聞之,召為郎。文帝為五官將,坐上會客三十餘人,文帝問己年壽,又令徧相衆賔。建平曰:「將軍當壽八十,至四十時當有小厄,願謹護之。」謂夏侯威曰:「君四十九位為州牧,而當有厄,厄若得過,可年至七十,致位公輔。」謂應璩曰:「君六十二位為常伯,而當有厄,先此一年,當獨見一白狗,而旁人不見也。」謂曹彪曰:「君據藩國,至五十七當厄於兵,宜善防之。」
 
 
 
 
 初,潁川荀攸、鍾繇相與親善。攸先亡,子幼。繇經紀其門戶,欲嫁其妾。與人書曰:「吾與公達曾共使朱建平相,建平曰:『荀君雖少,然當以後事付鍾君。』吾時啁之曰:『惟當嫁卿阿騖耳。』何意此子竟早隕沒,戲言遂驗乎!今欲嫁阿騖,使得善處。追思建平之妙,雖唐舉、許負何以復加也!」
 
 
 
 
 文帝黃初七年,年四十,病困,謂左右曰:「建平所言八十,謂晝夜也,吾其決矣。」頃之,果崩。夏侯威為兖州刺史,年四十九,十二月上旬得疾,念建平之言,自分必死,豫作遺令及送喪之備,咸使素辦。至下旬轉差,垂以平復。三十日日昃,請紀綱大吏設酒,曰:「吾所苦漸平,明日鷄鳴,年便五十,建平之戒,真必過矣。」威罷客之後,合瞑疾動,夜半遂卒。璩六十一為侍中,直省內,欻見白狗,問之衆人,悉無見者。於是數聚會,并急游觀田里,飲宴自娛,過期一年,六十三卒。曹彪封楚王,年五十七,坐與王淩通謀,賜死。凡說此輩,無不如言,不能具詳,故粗記數事。惟相司空王昶、征北將軍程喜、中領軍王肅有蹉跌云。肅年六十二,疾篤,衆醫並以為不愈。肅夫人問以遺言,肅云:「建平相我踰七十,位至三公,今皆未也,將何慮乎!」而肅竟卒。
 
 
 
 
 建平又善相馬。文帝將出,取馬外入,建平道遇之,語曰:「此馬之相,今日死矣。」帝將乘馬,馬惡衣香,驚齧文帝膝,帝大怒,即便殺之。建平黃初中卒。
 
 
\end{pinyinscope}