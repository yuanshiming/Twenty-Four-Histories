\article{霍峻傳}
\begin{pinyinscope}
 
 
 霍峻字仲邈,南郡枝江人也。兄篤,於鄉里合部曲數百人。篤卒,荊州牧劉表令峻攝其衆。表卒,峻率衆歸先主,先主以峻為中郎將。先主自葭萌南還襲劉璋,留峻守葭萌城。張魯遣將楊帛誘峻,求共守城,峻曰:「小人頭可得,城不可得。」帛乃退去。後璋將扶禁、向存等帥萬餘人由閬水上,攻圍峻,且一年,不能下。峻城中兵纔數百人,伺其怠隙,選精銳出擊,大破之,即斬存首。先主定蜀,嘉峻之功,乃分廣漢為梓潼郡,以峻為梓潼太守、裨將軍。在官三年,年四十卒,還葬成都。先主甚悼惜,乃詔諸葛亮曰:「峻旣佳士,加有功於國,欲行酹。」遂親率群僚臨會弔祭,因留宿墓上,當時榮之。
 
 
 
 
 子弋,字紹先,先主末年為太子舍人。後主踐阼,除謁者。丞相諸葛亮北駐漢中,請為記室,使與子喬共周旋游處。亮卒,為黃門侍郎。後主立太子璿,以弋為中庶子,璿好騎射,出入無度,弋援引古義,盡言規諫,甚得切磋之體。後為參軍庲降屯副貳都督,又轉護軍,統事如前。時永昌郡夷獠恃險不賔,數為寇害,乃以弋領永昌太守,率偏軍討之,遂斬其豪帥,破壞邑落,郡界寧靜。遷監軍將軍,領建寧太守,還統南郡事。景耀六年,進號安南將軍。是歲,蜀并于魏。弋與巴東領軍襄陽羅憲各保全一方,舉以內附,咸因仍前任,寵待有加。
 
 
\gezhu{漢晉春秋曰:霍弋聞魏軍來,弋欲赴成都,後主以備敵旣定,不聽。及成都不守,弋素服號哭,大臨三日。諸將咸勸宜速降,弋曰:「今道路隔塞,未詳主之安危,大故去就,不可苟也。若主上與魏和,見遇以禮,則保境而降,不晚也。若萬一危辱,吾將以死拒之,何論遲速邪!」得後主東遷之問,始率六郡將守上表曰:「臣聞人生於三,事之如一,惟難所在,則致其命。今臣國敗主附,守死無所,是以委質,不敢有貳。」晉文王善之,又拜南中都督,委以本任。後遣將兵救援呂興,平交阯、日南、九真三郡,功封列侯,進號崇賞焉。弋孫彪,晉越嶲太守。襄陽記曰:羅憲字令則。父蒙,避亂於蜀,官至廣漢太守。憲少以才學知名,年十三能屬文。後主立太子,為太子舍人,遷庶子、尚書吏部郎,以宣信校尉再使於吳,吳人稱美焉。時黃皓預政,衆多附之,憲獨不與同,皓恚,左遷巴東太守。時右大將軍閻宇都督巴東,為領軍,後主拜憲為宇副貳。魏之伐蜀,召宇西還,留宇二千人,令憲守永安城。尋聞成都敗,城中擾動,江邊長吏皆弃城走,憲斬稱成都亂者一人,百姓乃定。得後主委質問至,乃帥所統臨于都亭三日。吳聞蜀敗,起兵西上,外託救援,內欲襲憲。憲曰:「本朝傾覆,吳為脣齒,不恤我難而徼其利,背盟違約。且漢已亡,吳何得乆,寧能為吳降虜乎!」保城繕甲,告誓將士,厲以節義,莫不用命。吳聞鍾、鄧敗,百城無主,有兼蜀之志,而巴東固守,兵不得過,使步協率衆而西。憲臨江拒射,不能禦,遣參軍楊宗突圍北出,告急安東將軍陳騫,又送文武印綬、任子詣晉王。協攻城,憲出與戰,大破其軍。孫休怒,復遣陸抗等帥衆三萬人增憲之圍。被攻凡六月日而救援不到,城中疾病大半。或說憲奔走之計,憲曰:「夫為人主,百姓所仰,危不能安,急而弃之,君子不為也,畢命於此矣。」陳騫言於晉王,遣荊州刺史胡烈救憲,抗等引退。晉王即委前任,拜憲淩江將軍,封萬年亭侯。會武陵四縣舉衆叛吳,以憲為武陵太守巴東監軍。泰始元年改封西鄂縣侯。憲遣妻子居洛陽,武帝以子襲為給事中。三年冬,入朝,進位冠軍將軍、假節。四年三月,從帝宴于華林園,詔問蜀大臣子弟,後問先輩宜時叙用者,憲薦蜀郡常忌、杜軫、壽良、巴西陳壽、南郡高軌、南陽呂雅、許國、江夏費恭、琅邪諸葛京、汝南陳裕,即皆叙用,咸顯於世。憲還,襲取吳之巫城,因上伐吳之策。憲方亮嚴正,待士不倦,輕財好施,不治產業。六年薨,贈安南將軍,謚曰烈侯。子襲,以淩江將軍領部曲,早卒,追贈廣漢太守。襲子徽,順陽內史,永嘉五年為王如所殺。此作「獻」,名與本傳不同,未詳孰是也。}
 
 
\end{pinyinscope}