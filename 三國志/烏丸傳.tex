\article{烏丸傳}
\begin{pinyinscope}
 
 
 漢末,遼西烏丸大人丘力居,衆五千餘落,上谷烏丸大人難樓,衆九千餘落,各稱王,而遼東屬國烏丸大人蘇僕延,衆千餘落,自稱峭王,右北平烏丸大人烏延,衆八百餘落,自稱汗魯王,皆有計策勇健。中山太守張純叛入丘力居衆中,自號彌天安定王,為三郡烏丸元帥,寇略青、徐、幽、兾四州,殺略吏民。靈帝末,以劉虞為幽州牧,募胡斬純首,北州乃定。後丘力居死,子樓班年小,從子蹋頓有武略,代立,總攝三王部,衆皆從其教令。袁紹與公孫瓚連戰不決,蹋頓遣使詣紹求和親,助紹擊瓚,破之。紹矯制賜蹋頓、峭王、汗魯王印綬,皆以為單于。
 
 
\gezhu{英雄記曰:紹遣使即拜烏丸三王為單于,皆安車、華蓋、羽覆、黃屋、左纛。版文曰:「使持節大將軍督幽、青、并領兾州牧阮鄉侯紹,承制詔遼東屬國率衆王頒下、烏丸遼西率衆王蹋頓、右北平率衆王汗盧維:乃祖慕義遷善,款塞內附,北捍玁狁,東拒濊貊,世守北陲,為百姓保鄣,雖時侵犯王畧,命將徂征厥罪,率不旋時,悔愆變改,方之外夷,最又聦慧者也。始有千夫長、百夫長以相統領,用能悉乃心,克有勳力於國家,稍受王侯之命。自我王室多故,公孫瓚作難,殘夷厥土之君,以侮天慢主,是以四海之內,並執干戈以衞社稷。三王奮氣裔土,忿姦憂國,控弦與漢兵為表裏,誠甚忠孝,朝所嘉焉。然而虎兕長蛇,相隨塞路,王官爵命,否而無聞。夫有勳不賞,俾勤者怠。今遣行謁者楊林,齎單于璽綬車服,以對爾勞。其各綏靜部落,教以謹慎,無使作凶作慝。世復爾祀位,長為百蠻長。厥有咎有不臧者,泯於爾祿,而喪於乃庸,可不勉乎!烏桓單于都護部衆,左右單于受其節度,他如故事。」}
 
 
後樓班大,峭王率其部衆奉樓班為單于,蹋頓為王。然蹋頓多畫計策。廣陽閻柔,少沒烏丸、鮮卑中,為其種所歸信。柔乃因鮮卑衆,殺烏丸校尉邢舉代之,紹因寵慰以安北邊。後袁尚敗奔蹋頓,憑其勢,復圖兾州。會太祖平河北,柔帥鮮卑、烏丸歸附,遂因以柔為校尉,猶持漢使節,治廣寗如舊。建安十一年,太祖自征蹋頓於柳城,潛軍詭道,未至百餘里,虜乃覺。尚與蹋頓將衆逆戰於凡城,兵馬甚盛。太祖登高望虜陣,柳軍未進,觀其小動,乃擊破其衆,臨陣斬蹋頓首,死者被野。速附丸、樓班、烏延等走遼東,遼東悉斬,傳送其首。其餘遺迸皆降。及幽州、并州柔所統烏丸萬餘落,悉徙其族居中國,帥從其侯王大人種衆與征伐。由是三郡烏丸為天下名騎。
 \gezhu{魏畧曰:景初元年秋,遣幽州刺史毌丘儉率衆軍討遼東。右北平烏丸單于寇婁敦、遼西烏丸都督率衆王護留葉,昔隨袁尚奔遼西,聞儉軍至,率衆五千餘人降。寇婁敦遣弟阿羅槃等詣闕朝貢,封其渠帥三十餘為王,賜輿馬繒采各有差。}
 
 
\end{pinyinscope}