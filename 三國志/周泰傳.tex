\article{周泰傳}
\begin{pinyinscope}
 
 
 周泰字幼平,九江下蔡人也。與蔣欽隨孫策為左右,服事恭敬,數戰有功。策入會稽,署別部司馬,授兵。權愛其為人,請以自給。策討六縣山賊,權住宣城,使士自衞,不能千人,意尚忽略,不治圍落,而山賊數千人卒至。權始得上馬,而賊鋒刃已交於左右,或斫中馬鞌,衆莫能自定。惟泰奮激,投身衞權,膽氣倍人,左右由泰並能就戰。賊旣解散,身被十二創,良乆乃蘇。是日無泰,權幾危殆。策深德之,補春穀長。後從攻皖,及討江夏,還過豫章,復補宜春長,所在皆食其征賦。
 
 
 
 
 從討黃祖有功。後與周瑜、程普拒曹公於赤壁,攻曹仁於南郡。荊州平定,將兵屯岑。曹公出濡須,泰復赴擊,曹公退,留督濡須,拜平虜將軍。時朱然、徐盛等皆在所部,並不伏也,權特為案行至濡須鄔,因會諸將,大為酣樂。權自行酒到泰前,命泰解衣,權手自指其創痕,問以所起。泰輒記昔戰鬬處以對,畢,使復服,歡讌極夜。其明日,遣使者授以御蓋。
 
 
\gezhu{江表傳曰:權把其臂,因流涕交連,字之曰:「幼平,卿為孤兄弟戰如熊虎,不惜軀命,被創數十,膚如刻畫,孤亦何心不待卿以骨肉之恩,委卿以兵馬之重乎!卿吳之功臣,孤當與卿同榮辱,等休戚。幼平意快為之,勿以寒門自退也。」即勑以己常所用御幘青縑蓋賜之。坐罷,住駕,使泰以兵馬導從出,鳴鼓角作鼓吹。}
 於是盛等乃伏。
 
 
 
 
 後權破關羽,欲進圖蜀,拜泰漢中太守、奮威將軍,封陵陽侯。黃武中卒。
 
 
 
 
 子邵以騎都尉領兵。曹仁出濡須,戰有功,又從攻破曹休,進位裨將軍,黃龍二年卒。弟承領兵襲侯。
 
 
\end{pinyinscope}