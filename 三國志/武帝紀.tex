\article{武帝紀}
\begin{pinyinscope}


太祖武皇帝,沛國譙人也,姓曹,諱操,字孟德,漢相國參之後。
\gezhu{太祖一名吉利,小字阿瞞。王沈魏書曰:其先出於黃帝。當高陽世,陸終之子曰安,是為曹姓。周武王克殷,存先世之後,封曹俠於邾。春秋之世,與於盟會,逮至戰國,為楚所滅。子孫分流,或家于沛。漢高祖之起,曹參以功封平陽侯,世襲爵土,絕而復紹,至今適嗣國於容城。}
桓帝世,曹騰為中常侍大長秋,封費亭侯。
\gezhu{司馬彪續漢書曰:騰父節,字元偉,素以仁厚稱。鄰人有亡豕者,與節豕相類,詣門認之,節不與爭;後所亡豕自還其家,豕主人大慙,送所認豕,并辭謝節,節笑而受之。由是鄉黨貴歎焉。長子伯興,次子仲興,次子叔興。騰字季興,少除黃門從官。永寧元年,鄧太后詔黃門令選中黃門從官年少溫謹者,配皇太子書,騰應其選。太子特親愛騰,飲食賞賜與衆有異。順帝即位,為小黃門,遷至中常侍大長秋。在省闥三十餘年,歷事四帝,未甞有過。好進達賢能,終無所毀傷。其所稱薦,若陳留虞放、邊韶、南陽延固、張溫、弘農張奐、潁川堂谿典等,皆致位公卿,而不伐其善。蜀郡太守因計吏修敬於騰,益州刺史种暠於函谷關搜得其牋,上太守,并奏騰內臣外交,所不當為,請免官治罪。帝曰:「牋自外來,騰書不出,非其罪也。」乃寑暠奏。騰不以介意,常稱歎暠,以為暠得事上之節。暠後為司徒,語人曰:「今日為公,乃曹常侍恩也。」騰之行事,皆此類也。桓帝即位,以騰先帝舊臣,忠孝彰著,封費亭侯,加位特進。太和三年,追尊騰曰高皇帝。}
養子嵩嗣,官至太尉,莫能審其生出本末。
\gezhu{續漢書曰:嵩字巨高。質性敦慎,所在忠孝。為司隷校尉,靈帝擢拜大司農、大鴻臚,代崔烈為太尉。黃初元年,追尊嵩曰太皇帝。吳人作曹瞞傳及郭頒世語並云:嵩,夏侯氏之子,夏侯惇之叔父。太祖於惇為從父兄弟。}
嵩生太祖。


太祖少機警,有權數,而任俠放蕩,不治行業,故世人未之奇也;
\gezhu{曹瞞傳云:太祖少好飛鷹走狗,游蕩無度,其叔父數言之於嵩。太祖患之,後逢叔父於路,乃陽敗面喎口;叔父怪而問其故,太祖曰:「卒中惡風。」叔父以告嵩。嵩驚愕,呼太祖,太祖口貌如故。嵩問曰:「叔父言汝中風,已差乎?」太祖曰:「初不中風,但失愛於叔父,故見罔耳。」嵩乃疑焉。自後叔父有所告,嵩終不復信,太祖於是益得肆意矣。}
惟梁國橋玄、南陽何顒異焉。玄謂太祖曰:「天下將亂,非命世之才不能濟也,能安之者,其在君乎!」
\gezhu{魏書曰:太尉橋玄,世名知人,覩太祖而異之,曰:「吾見天下名士多矣,未有若君者也!君善自持。吾老矣!願以妻子為託。」由是聲名益重。續漢書曰:玄字公祖,嚴明有才略,長於人物。張璠漢紀曰:玄歷位中外,以剛斷稱,謙儉下士,不以王爵私親。光和中為太尉,以乆病策罷,拜太中大夫,卒,家貧乏產業,柩無所殯。當世以此稱為名臣。世語曰:玄謂太祖曰:「君未有名,可交許子將。」太祖乃造子將,子將納焉,由是知名。孫盛異同雜語云:太祖甞私入中常侍張讓室,讓覺之;乃舞手戟於庭,踰垣而出。才武絕人,莫之能害。博覽群書,特好兵法,抄集諸家兵法,名曰接要,又注孫武十三篇,皆傳於世。甞問許子將:「我何如人?」子將不荅。固問之,子將曰:「子治世之能臣,亂世之姦雄。」太祖大笑。}
年二十,舉孝廉為郎,除洛陽北部尉,遷頓丘令,
\gezhu{曹瞞傳曰:太祖初入尉廨,繕治四門。造五色棒,縣門左右各十餘枚,有犯禁者,不避豪彊,皆棒殺之。後數月,靈帝愛幸小黃門蹇碩叔父夜行,即殺之。京師斂迹,莫敢犯者。近習寵臣咸疾之,然不能傷,於是共稱薦之,故遷為頓丘令。}
徵拜議郎。
\gezhu{魏書曰:太祖從妹夫㶏彊侯宋奇被誅,從坐免官。後以能明古學,復徵拜議郎。先是大將軍竇武、太傅陳蕃謀誅閹官,反為所害。太祖上書陳武等正直而見陷害,姦邪盈朝,善人壅塞,其言甚切;靈帝不能用。是後詔書勑三府:舉奏州縣政理無效,民為作謠言者免罷之。三公傾邪,皆希世見用,貨賂並行,彊者為怨,不見舉奏,弱者守道,多被陷毀。太祖疾之。是歲以災異博問得失,因此復上書切諫,說三公所舉奏專回避貴戚之意。奏上,天子感寤,以示三府責讓之,諸以謠言徵者皆拜議郎。是後政教日亂,豪猾益熾,多所摧毀;太祖知不可匡正,遂不復獻言。}


光和末,黃巾起。拜騎都尉,討潁川賊。遷為濟南相,國有十餘縣,長吏多阿附貴戚,贓污狼藉,於是奏免其八;禁斷淫祀,姦宄逃竄,郡界肅然。
\gezhu{魏書曰:長吏受取貪饕,依倚貴勢,歷前相不見舉;聞太祖至,咸皆舉免,小大震怖,姦宄遁逃,竄入他郡。政教大行,一郡清平。初,城陽景王劉章以有功於漢,故其國為立祠,青州諸郡轉相倣效,濟南尤盛,至六百餘祠。賈人或假二千石輿服導從作倡樂,奢侈日甚,民坐貧窮,歷世長吏無敢禁絕者。太祖到,皆毀壞祠屋,止絕官吏民不得祠祀。及至秉政,遂除姦邪鬼神之事,世之淫祀由此遂絕。}
乆之,徵還為東郡太守;不就,稱疾歸鄉里。
\gezhu{魏書曰:於是權臣專朝,貴戚橫恣。太祖不能違道取容。數數干忤,恐為家禍,遂乞留宿衞。拜議郎,常託疾病,輙告歸鄉里;築室城外,春夏習讀書傳,秋冬弋獵,以自娛樂。}


頃之,兾州刺史王芬、南陽許攸、沛國周旌等連結豪傑,謀廢靈帝,立合肥侯,以告太祖,太祖拒之。芬等遂敗。
\gezhu{司馬彪九州春秋曰:於是陳蕃子逸與術士平原襄楷會於芬坐,楷曰:「天文不利宦者,黃門、常侍貴族滅矣。」逸喜。芬曰:「若然者,芬願驅除。」於是與攸等結謀。靈帝欲北巡河間舊宅,芬等謀因此作難,上書言黑山賊攻劫郡縣,求得起兵。會北方有赤氣,東西竟天,太史上言「當有陰謀,不宜北行」,帝乃止。勑芬罷兵,俄而徵之。芬懼,自殺。}
\gezhu{魏書載太祖拒芬辭曰:「夫廢立之事,天下之至不祥也。古人有權成敗計輕重而行之者,伊尹、霍光是也。伊尹懷至忠之誠,據宰臣之勢,處官司之上,故進退廢置,計從事立。及至霍光受託國之任,藉宗臣之位,內因太后秉政之重,外有羣卿同欲之勢,昌邑即位日淺,未有貴寵,朝乏讜臣,議出密近,故計行如轉圜,事成如摧朽。今諸君徒見曩者之易,未覩當今之難。諸君自度,結衆連黨,何若七國?合肥之貴,孰若吳、楚?而造作非常,欲望必克,不亦危乎!」}


金城邊章、韓遂殺刺史郡守以叛,衆十餘萬,天下騷動。徵太祖為典軍校尉。會靈帝崩,太子即位,太后臨朝。大將軍何進與袁紹謀誅宦官,太后不聽。進乃召董卓,欲以脅太后,
\gezhu{魏書曰:太祖聞而笑之曰:「閹豎之官,古今宜有,但世主不當假之權寵,使至於此。旣治其罪,當誅元惡,一獄吏足矣,何必紛紛召外將乎?欲盡誅之,事必宣露,吾見其敗也。」}
卓未至而進見殺。卓到,廢帝為弘農王而立獻帝,京都大亂。卓表太祖為驍騎校尉,欲與計事。太祖乃變易姓名,間行東歸。
\gezhu{魏書曰:太祖以卓終必覆敗,遂不就拜,逃歸鄉里。從數騎過故人成臯呂伯奢;伯奢不在,其子與賔客共劫太祖,取馬及物,太祖手刃擊殺數人。世語曰:太祖過伯奢。伯奢出行,五子皆在,備賔主禮。太祖自以背卓命,疑其圖己,手劔夜殺八人而去。}
\gezhu{孫盛雜記曰:太祖聞其食器聲,以為圖己,遂夜殺之。旣而悽愴曰:「寧我負人,毋人負我!」遂行。}
出關,過中牟,為亭長所疑,執詣縣,邑中或竊識之,為請得解。
\gezhu{世語曰:中牟疑是亡人,見拘於縣。時掾亦已被卓書;唯功曹心知是太祖,以世方亂,不宜拘天下雄儁,因白令釋之。}
卓遂殺太后及弘農王。太祖至陳留,散家財,合義兵,將以誅卓。冬十二月,始起兵於己吾,
\gezhu{世語曰:陳留孝廉衞茲以家財資太祖,使起兵,衆有五千人。}
是歲中平六年也。


初平元年春正月,後將軍袁術、兾州牧韓馥、
\gezhu{英雄記曰:馥字文節,潁川人。為御史中丞。董卓舉為兾州牧。于時兾州民人殷盛,兵糧優足。袁紹之在勃海,馥恐其興兵,遣數部從事守之,不得動搖。東郡太守橋瑁詐作京師三公移書與州郡,陳卓罪惡,云「見逼迫,無以自救,企望義兵,解國患難。」馥得移,請諸從事問曰:「今當助袁氏邪,助董卓邪?」治中從事劉子惠曰:「今興兵為國,何謂袁、董!」馥自知言短而有慙色。子惠復言:「兵者凶事,不可為首;今宜往視他州,有發動者,然後和之。兾州於他州不為弱也,他人功未有在兾州之右者也。」馥然之。馥乃作書與紹,道卓之惡,聽其舉兵。}
豫州刺史孔伷、
\gezhu{英雄記曰:伷字公緒,陳留人。張璠漢紀,載鄭泰說卓云:「孔公緒能清談高論,噓枯吹生。」}
兖州刺史劉岱、
\gezhu{岱,劉繇之兄,事見吳志。}
河內太守王匡、
\gezhu{英雄記曰:匡字公節,泰山人。輕財好施,以任俠聞。辟大將軍何進府進符使,匡於徐州發彊弩五百西詣京師。會進敗,匡還鄉里。起家,拜河內太守。謝承後漢書曰:匡少與蔡邕善。其年為卓軍所敗,走還泰山,收集勁勇得數千人,欲與張邈合。匡先殺執金吾胡母班。班親屬不勝憤怒,與太祖并勢,共殺匡。}
勃海太守袁紹、陳留太守張邈、東郡太守橋瑁、
\gezhu{英雄記曰:瑁字元偉,玄族子。先為兖州刺史,甚有威惠。}
山陽太守袁遺、
\gezhu{遺字伯業,紹從兄。為長安令。河間張超甞薦遺于太尉朱儁,稱遺「有冠世之懿,幹時之量。其忠允亮直,固天所縱;若乃包羅載籍,管綜百氏,登高能賦,覩物知名,求之今日,邈焉靡儔。」事在超集。英雄記曰:紹後用遺為揚州刺史,為袁術所敗。太祖稱「長大而能勤學者,惟吾與袁伯業耳。」語在文帝典論。}
濟北相鮑信
\gezhu{信事見子勛傳。}
同時俱起兵,衆各數萬,推紹為盟主。太祖行奮武將軍。


二月,卓聞兵起,乃徙天子都長安。卓留屯洛陽,遂焚宮室。是時紹屯河內,邈、岱、瑁、遺屯酸棗,術屯南陽,伷屯潁川,馥在鄴。卓兵彊,紹等莫敢先進。太祖曰:「舉義兵以誅暴亂,大衆已合,諸君何疑?向使董卓聞山東兵起,倚王室之重,據二周之險,東向以臨天下;雖以無道行之,猶足為患。今焚燒宮室,劫遷天子,海內震動,不知所歸,此天亡之時也。一戰而天下定矣,不可失也。」遂引兵西,將據成臯。邈遣將衞茲分兵隨太祖。到熒陽汴水,遇卓將徐榮,與戰不利,士卒死傷甚多。太祖為流矢所中,所乘馬被創,從弟洪以馬與太祖,得夜遁去。榮見太祖所將兵少,力戰盡日,謂酸棗未易攻也,亦引兵還。


太祖到酸棗,諸軍兵十餘萬,日置酒高會,不圖進取。太祖責讓之,因為謀曰:「諸君聽吾計,使勃海引河內之衆臨孟津,酸棗諸將守成臯,據敖倉,塞轘轅、太谷,全制其險;使袁將軍率南陽之軍軍丹、析,入武關,以震三輔:皆高壘深壁,勿與戰,益為疑兵,示天下形勢,以順誅逆,可立定也。今兵以義動,持疑而不進,失天下之望,竊為諸君耻之!」邈等不能用。


太祖兵少,乃與夏侯惇等詣揚州募兵,刺史陳溫、丹楊太守周昕與兵四千餘人。還到龍亢,士卒多叛。
\gezhu{魏書曰:兵謀叛,夜燒太祖帳,太祖手劔殺數十人,餘皆披靡,乃得出營;其不叛者五百餘人。}
至銍、建平,復收兵得千餘人,進屯河內。


劉岱與橋瑁相惡,岱殺瑁,以王肱領東郡太守。


袁紹與韓馥謀立幽州牧劉虞為帝,太祖拒之。
\gezhu{魏書載太祖荅紹曰:「董卓之罪,暴於四海,吾等合大衆興義兵,遠近莫不響應,此以義動故也。今幼主微弱,制於姦臣,未有昌邑亡國之釁,而一旦改易,天下其孰安之?諸君北面,我自西向。」}
紹又甞得一玉印,於太祖坐中舉向其肘,太祖由是笑而惡焉。
\gezhu{魏書曰:太祖大笑曰:「吾不聽汝也。」紹復使人說太祖曰:「今袁公勢盛兵彊,二子已長,天下羣英,孰踰於此?」太祖不應。由是益不直紹,圖誅滅之。}


二年春,紹、馥遂立虞為帝,虞終不敢當。


夏四月,卓還長安。


秋七月,袁紹脅韓馥,取兾州。


黑山賊于毒、白繞、眭固等
\gezhu{眭,申隨反。}
十餘萬衆略魏郡、東郡,王肱不能禦,太祖引兵入東郡,擊白繞於濮陽,破之。袁紹因表太祖為東郡太守,治東武陽。


三年春,太祖軍頓丘,毒等攻東武陽。太祖乃引兵西入山,攻毒等本屯。
\gezhu{魏書曰:諸將皆以為當還自救。太祖曰:「孫臏救趙而攻魏,耿弇欲走西安攻臨菑。使賊聞我西而還,武陽自解也;不還,我能敗其本屯,虜不能拔武陽必矣。」遂乃行。}
毒聞之,棄武陽還。太祖要擊眭固,又擊匈奴於夫羅於內黃,皆大破之。
\gezhu{魏書曰:於夫羅者,南單于子也。中平中,發匈奴兵,於夫羅率以助漢。會本國反,殺南單于,於夫羅遂將其衆留中國。因天下撓亂,與西河白波賊合,破太原、河內,抄略諸郡為寇。}


夏四月,司徒王允與呂布共殺卓。卓將李傕、郭汜等殺允攻布,布敗,東出武關。傕等擅朝政。


青州黃巾衆百萬入兖州,殺任城相鄭遂,轉入東平。劉岱欲擊之,鮑信諫曰:「今賊衆百萬,百姓皆震恐,士卒無鬬志,不可敵也。觀賊衆羣輩相隨,軍無輜重,唯以鈔略為資,今不若畜士衆之力,先為固守。彼欲戰不得,攻又不能,其勢必離散,後選精銳,據其要害,擊之可破也。」岱不從,遂與戰,果為所殺。
\gezhu{世語曰:岱旣死,陳宮謂太祖曰:「州今無主,而王命斷絕,宮請說州中,明府尋往牧之,資之以收天下,此霸王之業也。」宮說別駕、治中曰:「今天下分裂而州無主;曹東郡,命世之才也,若迎以牧州,必寧生民。」鮑信等亦謂之然。}
信乃與州吏萬潛等至東郡迎太祖領兖州牧。遂進兵擊黃巾於壽張東。信力戰鬬死,僅而破之。
\gezhu{魏書曰:太祖將步騎千餘人,行視戰地,卒抵賊營,戰不利,死者數百人,引還。賊尋前進。黃巾為賊久,數乘勝,兵皆精悍。太祖舊兵少,新兵不習練,舉軍皆懼。太祖被甲嬰冑,親巡將士,明勸賞罰,衆乃復奮,承間討擊,賊稍折退。賊乃移書太祖曰:「昔在濟南,毀壞神壇,其道乃與中黃太一同,似若知道,今更迷惑。漢行已盡,黃家當立。天之大運,非君才力所能存也。」太祖見檄書,呵之罪,數開示降路;遂設奇伏,晝夜會戰,戰輙禽獲,賊乃退走。}
購求信喪不得,衆乃刻木如信形狀,祭而哭焉。追黃巾至濟北。乞降。冬,受降卒三十餘萬,男女百餘萬口,收其精銳者,號為青州兵。


袁術與紹有隙,術求援於公孫瓚,瓚使劉備屯高唐,單經屯平原,陶謙屯發干,以逼紹。太祖與紹會擊,皆破之。


四年春,軍鄄城。荊州牧劉表斷術糧道,術引軍入陳留,屯封丘,黑山餘賊及於夫羅等佐之。術使將劉詳屯匡亭。太祖擊詳,術救之,與戰,大破之。術退保封丘,遂圍之,未合,術走襄邑,追到太壽,決渠水灌城。走寧陵,又追之,走九江。夏,太祖還軍定陶。


下邳闕宣聚衆數千人,自稱天子;徐州牧陶謙與共舉兵,取泰山華、費,略任城。秋,太祖征陶謙,下十餘城,謙守城不敢出。


是歲,孫策受袁術使渡江,數年閒遂有江東。


興平元年春,太祖自徐州還,初,太祖父嵩去官後還譙,董卓之亂,避難琅邪,為陶謙所害,故太祖志在復讎東伐。
\gezhu{世語曰:嵩在泰山華縣。太祖令泰山太守應劭送家詣兖州,劭兵未至,陶謙密遣數千騎掩捕。嵩家以為劭迎,不設備。謙兵至,殺太祖弟德於門中。嵩懼,穿後垣,先出其妾,妾肥,不時得出;嵩逃于厠,與妾俱被害,闔門皆死。劭懼,棄官赴袁紹。後太祖定兾州,劭時已死。韋曜吳書曰:太祖迎嵩,輜重百餘兩。陶謙遣都尉張闓將騎二百衞送,闓於泰山華、費間殺嵩,取財物,因奔淮南。太祖歸咎於陶謙,故伐之。}
夏,使荀彧、程昱守鄄城,復征陶謙,拔五城,遂略地至東海。還過郯,謙將曹豹與劉備屯郯東,要太祖。太祖擊破之,遂攻拔襄賁,所過多所殘戮。
\gezhu{孫盛曰:夫伐罪弔民,古之令軌;罪謙之由,而殘其屬部,過矣。}


會張邈與陳宮叛迎呂布,郡縣皆應。荀彧、程昱保鄄城,范、東阿二縣固守,太祖乃引軍還。布到,攻鄄城不能下,西屯濮陽。太祖曰:「布一旦得一州,不能據東平,斷亢父、泰山之道,乘險要我,而乃屯濮陽,吾知其無能為也。」遂進軍攻之。布出兵戰,先以騎犯青州兵。青州兵奔,太祖陳亂馳突火出,墜馬,燒左手掌。司馬樓異扶太祖上馬,遂引去。
\gezhu{袁暐獻帝春秋曰:太祖圍濮陽,濮陽大姓田氏為反閒,太祖得入城。燒其東門,示無反意。及戰,軍敗。布騎得太祖而不知是,問曰:「曹操何在?」太祖曰:「乘黃馬走者是也。」布騎乃釋太祖而追黃馬者。門火猶盛,太祖突火而出。}
未至營止,諸將未與太祖相見,皆怖。太祖乃自力勞軍,令軍中促為攻具,進復攻之,與布相守百餘日。蝗蟲起,百姓大餓,布糧食亦盡,各引去。


秋九月,太祖還鄄城。布到乘氏,為其縣人李進所破,東屯山陽。於是紹使人說太祖,欲連和。太祖新失兖州,軍食盡,將許之。程昱止太祖,太祖從之。冬十月,太祖至東阿。


是歲穀一斛五十餘萬錢,人相食,乃罷吏兵新募者。陶謙死,劉備代之。


二年春,襲定陶。濟陰太守吳資保南城,未拔。會呂布至,又擊破之。夏,布將薛蘭、李封屯鉅野,太祖攻之,布救蘭,蘭敗,布走,遂斬蘭等。布復從東緍與陳宮將萬餘人來戰,時太祖兵少,設伏,縱奇兵擊,大破之。
\gezhu{魏書曰:於是兵皆出取麥,在者不能千人,屯營不固。太祖乃令婦人守陴,悉兵拒之。屯西有大隄,其南樹木幽深。布疑有伏,乃相謂曰:「曹操多譎,勿入伏中。」引軍屯南十餘里。明日復來,太祖隱兵隄裏,出半兵隄外。布益進,乃令輕兵挑戰,旣合,伏兵乃悉乘隄,步騎並進,大破之,獲其龍車,追至其營而還。}
布夜走,太祖復攻,拔定陶,分兵平諸縣。布東奔劉備,張邈從布,使其弟超將家屬保雍丘。秋八月,圍雍丘。冬十月,天子拜太祖兖州牧。十二月,雍丘潰,超自殺。夷邈三族。邈詣袁術請救,為其衆所殺,兖州平,遂東略陳地。


是歲,長安亂,天子東遷,敗于曹陽,渡河幸安邑。


建安元年春正月,太祖軍臨武平,袁術所置陳相袁嗣降。


太祖將迎天子,諸將或疑,荀彧、程昱勸之,乃遣曹洪將兵西迎,衞將軍董承與袁術將萇奴拒險,洪不得進。


汝南、潁川黃巾何儀、劉辟、黃邵、何曼等,衆各數萬,初應袁術,又附孫堅。二月,太祖進軍討破之,斬辟、邵等,儀及其衆皆降。天子拜太祖建德將軍,夏六月,遷鎮東將軍,封費亭侯。秋七月,楊奉、韓暹以天子還洛陽,
\gezhu{獻帝春秋曰:天子初至洛陽,幸城西故中常侍趙忠宅。使張楊繕治宮室,名殿曰揚安殿,八月,帝乃遷居。}
奉別屯梁。太祖遂至洛陽,衞京都,暹遁走。天子假太祖節鉞,錄尚書事。
\gezhu{獻帝紀曰:又領司隷校尉。}
洛陽殘破,董昭等勸太祖都許。九月,車駕出轘轅而東,以太祖為大將軍,封武平侯。自天子西遷,朝廷日亂,至是宗廟社稷制度始立。
\gezhu{張璠漢紀曰:初,天子敗於曹陽,欲浮河東下。侍中太史令王立曰:「自去春太白犯鎮星於牛斗,過天津,熒惑又逆行守北河,不可犯也。」由是天子遂不北渡河,將自軹關東出。立又謂宗正劉艾曰:「前太白守天關,與熒惑會;金火交會,革命之象也。漢祚終矣,晉、魏必有興者。」立後數言於帝曰:「天命有去就,五行不常盛,代火者土也,承漢者魏也,能安天下者,曹姓也,唯委任曹氏而已。」公聞之,使人語立曰:「知公忠於朝廷,然天道深遠,幸勿多言。」}


天子之東也,奉自梁欲要之,不及。冬十月,公征奉,奉南奔袁術,遂攻其梁屯,拔之。於是以袁紹為太尉,紹恥班在公下,不肯受。公乃固辭,以大將軍讓紹。天子拜公司空,行車騎將軍。是歲用棗祗、韓浩等議,始興屯田。
\gezhu{魏書曰:自遭荒亂,率乏糧穀。諸軍並起,無終歲之計,饑則寇略,飽則棄餘,瓦解流離,無敵自破者不可勝數。袁紹之在河北,軍人仰食桑椹。袁術在江、淮,取給蒲蠃。民人相食,州里蕭條。公曰:「夫定國之術,在於彊兵足食,秦人以急農兼天下,孝武以屯田定西域,此先代之良式也。」是歲乃募民屯田許下,得穀百萬斛。於是州郡例置田官,所在積穀。征伐四方,無運糧之勞,遂兼滅羣賊,克平天下。}


呂布襲劉備,取下邳。備來奔。程昱說公曰:「觀劉備有雄才而甚得衆心,終不為人下,不如早圖之。」公曰:「方今收英雄時也,殺一人而失天下之心,不可。」


張濟自關中走南陽。濟死,從子繡領其衆。


二年春正月,公到宛。張繡降,旣而悔之,復反。公與戰,軍敗,為流矢所中,長子昂、弟子安民遇害。
\gezhu{魏書曰:公所乘馬名絕影,為流矢所中,傷頰及足,并中公右臂。世語曰:昂不能騎,進馬於公,公故免,而昂遇害。}
公乃引兵還舞陰,繡將騎來鈔,公擊破之。繡奔穰,與劉表合。公謂諸將曰:「吾降張繡等,失不便取其質,以至于此。吾知所以敗。諸卿觀之,自今已後不復敗矣。」遂還許。
\gezhu{世語曰:舊制,三公領兵入見,皆交戟叉頸而前。初,公將討張繡,入覲天子,時始復此制。公自此不復朝見。}


袁術欲稱帝於淮南,使人告呂布。布收其使,上其書。術怒,攻布,為布所破。秋九月,術侵陳,公東征之。術聞公自來,棄軍走,留其將橋蕤、李豐、梁綱、樂就;公到,擊破蕤等,皆斬之。術走渡淮。公還許。


公之自舞陰還也,南陽、章陵諸縣復叛為繡,公遣曹洪擊之,不利,還屯葉,數為繡、表所侵。冬十一月,公自南征,至宛。
\gezhu{魏書曰:臨淯水,祠亡將士,歔欷流涕,衆皆感慟。}
表將鄧濟據湖陽。攻拔之,生禽濟,湖陽降。攻舞陰,下之。


三年春正月,公還許,初置軍師祭酒。三月,公圍張繡於穰。夏五月,劉表遣兵救繡,以絕軍後。
\gezhu{獻帝春秋曰:袁紹叛卒詣公云:「田豐使紹早襲許,若挾天子以令諸侯,四海可指麾而定。」公乃解繡圍。}
公將引還,繡兵來,公軍不得進,連營稍前。公與荀彧書曰:「賊來追吾,雖日行數里,吾策之,到安衆,破繡必矣。」到安衆,繡與表兵合守險,公軍前後受敵。公乃夜鑿險為地道,悉過輜重,設奇兵。會明,賊謂公為遁也,悉軍來追。乃縱奇兵步騎夾攻,大破之。秋七月,公還許。荀彧問公:「前以策賊必破,何也?」公曰:「虜遏吾歸師,而與吾死地戰,吾是以知勝矣。」


呂布復為袁術使高順攻劉備,公遣夏侯惇救之,不利。備為順所敗。九月,公東征布。冬十月,屠彭城,獲其相侯諧。進至下邳,布自將騎逆擊。大破之,獲其驍將成廉。追至城下,布恐,欲降。陳宮等沮其計,求救於術,勸布出戰,戰又敗,乃還固守,攻之不下。時公連戰,士卒罷,欲還,用荀攸、郭嘉計,遂決泗、沂水以灌城。月餘,布將宋憲、魏續等執陳宮,舉城降,生禽布、宮,皆殺之。太山臧霸、孫觀、吳敦、尹禮、昌狶各聚衆。布之破劉備也,霸等悉從布。布敗,獲霸等,公厚納待,遂割青、徐二州附于海以委焉,分琅邪、東海、北海為城陽、利城、昌慮郡。


初,公為兖州,以東平畢諶為別駕。張邈之叛也,邈劫諶母弟妻子;公謝遣之,曰:「卿老母在彼,可去。」諶頓首無二心,公嘉之,為之流涕。旣出,遂亡歸。及布破,諶生得,衆為諶懼,公曰:「夫人孝於其親者,豈不亦忠於君乎!吾所求也。」以為魯相。
\gezhu{魏書曰:袁紹宿與故太尉楊彪、大長秋梁紹、少府孔融有隙,欲使公以他過誅之。公曰:「當今天下土崩瓦解,雄豪並起,輔相君長,人懷怏怏,各有自為之心,此上下相疑之秋也,雖以無嫌待之,猶懼未信;如有所除,則誰不自危?且夫起布衣,在塵垢之間,為庸人之所陵陷,可勝怨乎!高祖赦雍齒之讎而羣情以安,如何忘之?」紹以為公外託公義,內實離異,深懷怨望。臣松之以為楊彪亦曾為魏武所困,幾至於死,孔融竟不免於誅滅,豈所謂先行其言而後從之哉!非知之難,其在行之,信矣。}


四年春二月,公還至昌邑。張楊將楊醜殺楊,眭固又殺醜,以其衆屬袁紹,屯射犬。夏四月,進軍臨河,使史渙、曹仁渡河擊之。固使楊故長史薛洪、河內太守繆尚留守,自將兵北迎紹求救,與渙、仁相遇犬城。交戰,大破之,斬固。公遂濟河,圍射犬。洪、尚率衆降,封為列侯,還軍敖倉。以魏种為河內太守,屬以河北事。


初,公舉种孝廉。兖州叛,公曰:「唯魏种且不棄孤也。」及聞种走,公怒曰:「种不南走越、北走胡,不置汝也!」旣下射犬,生禽种,公曰:「唯其才也!」釋其縛而用之。


是時袁紹旣并公孫瓚,兼四州之地,衆十餘萬,將進軍攻許,諸將以為不可敵,公曰:「吾知紹之為人,志大而智小,色厲而膽薄,忌克而少威,兵多而分畫不明,將驕而政令不一,土地雖廣,糧食雖豐,適足以為吾奉也。」秋八月,公進軍黎陽,使臧霸等入青州破齊、北海、東安,留于禁屯河上。九月,公還許,分兵守官渡。冬十一月,張繡率衆降,封列侯。十二月,公軍官渡。


袁術自敗於陳,稍困,袁譚自青州遣迎之。術欲從下邳北過,公遣劉備、朱靈要之。會術病死。程昱、郭嘉聞公遣備,言於公曰:「劉備不可縱。」公悔,追之不及。備之未東也,陰與董承等謀反,至下邳,遂殺徐州刺史車冑,舉兵屯沛。遣劉岱、王忠擊之,不克。
\gezhu{獻帝春秋曰:備謂岱等曰:「使汝百人來,其無如我何;曹公自來,未可知耳!」魏武故事曰:岱字公山,沛國人。以司空長史從征伐有功,封列侯。魏略曰:王忠,扶風人,少為亭長。三輔亂,忠饑乏噉人,隨輩南向武關。值婁子伯為荊州遣迎北方客人;忠不欲去,因率等仵逆擊之,奪其兵,聚衆千餘人以歸公。拜忠中郎將,從征討。五官將知忠甞噉人,因從駕出行,令俳取冢間髑髏繫著忠馬鞍,以為歡笑。}


廬江太守劉勳率衆降,封為列侯。


五年春正月,董承等謀泄,皆伏誅。公將自東征備,諸將皆曰:「與公爭天下者,袁紹也。今紹方來而棄之東,紹乘人後,若何?」公曰:「夫劉備,人傑也,今不擊,必為後患。
\gezhu{孫盛魏氏春秋云:荅諸將曰:「劉備,人傑也,將生憂寡人。」臣松之以為史之記言,旣多潤色,故前載所述有非實者矣,後之作者又生意改之,於失實也,不亦彌遠乎!凡孫盛製書,多用左氏以易舊文,如此者非一。嗟乎,後之學者將何取信哉?且魏武方以天下勵志,而用夫差分死之言,尤非其類。}
袁紹雖有大志,而見事遟,必不動也。」郭嘉亦勸公,遂東擊備,破之,生禽其將夏侯博。備走奔紹,獲其妻子。備將關羽屯下邳,復進攻之,羽降。昌狶叛為備,又攻破之。公還官渡,紹卒不出。


二月,紹遣郭圖、淳于瓊、顏良攻東郡太守劉延於白馬,紹引兵至黎陽,將渡河。夏四月,公北救延。荀攸說公曰:「今兵少不敵,分其勢乃可。公到延津,若將渡兵向其後者,紹必西應之,然後輕兵襲白馬,掩其不備,顏良可禽也。」公從之。紹聞兵渡,即分兵西應之。公乃引軍兼行趣白馬,未至十餘里,良大驚,來逆戰。使張遼、關羽前登,擊破,斬良。遂解白馬圍,徙其民,循河而西。紹於是渡河追公軍,至延津南。公勒兵駐營南阪下,使登壘望之,曰:「可五六百騎。」有頃,復白:「騎稍多,步兵不可勝數。」公曰:「勿復白。」乃令騎解鞍放馬。是時,白馬輜重就道。諸將以為敵騎多,不如還保營。荀攸曰:「此所以餌敵,如何去之!」紹騎將文醜與劉備將五六千騎前後至。諸將復白:「可上馬。」公曰:「未也。」有頃,騎至稍多,或分趣輜重。公曰:「可矣。」乃皆上馬。時騎不滿六百,遂縱兵擊,大破之,斬醜、良。醜、良皆紹名將也,再戰,悉禽,紹軍大震。公還軍官渡。紹進保陽武。關羽亡歸劉備。


八月,紹連營稍前,依沙塠為屯,東西數十里。公亦分營與相當,合戰不利。
\gezhu{習鑿齒漢晉春秋曰:許攸說紹曰:「公無與操相攻也。急分諸軍持之,而徑從他道迎天子,則事立濟矣。」紹不從,曰:「吾要當先圍取之。」攸怒。}
時公兵不滿萬,傷者十二三。
\gezhu{臣松之以為魏武初起兵,已有衆五千,自後百戰百勝,敗者十二三而已矣。但一破黃巾,受降卒三十餘萬,餘所吞并,不可悉紀;雖征戰損傷,未應如此之少也。夫結營相守,異於摧鋒決戰。本紀云:「紹衆十餘萬,屯營東西數十里。」魏太祖雖機變無方,略不世出,安有以數千之兵,而得逾時相抗者哉?以理而言,竊謂不然。紹為屯數十里,公能分營與相當,此兵不得甚少,一也。紹若有十倍之衆,理應當悉力圍守,使出入斷絕,而公使徐晃等擊其運車,公又自出擊淳于瓊等,揚旌往還,曾無抵閡,明紹力不能制,是不得甚少,二也。諸書皆云公坑紹衆八萬,或云七萬。夫八萬人奔散,非八千人所能縛,而紹之大衆皆拱手就戮,何緣力能制之?是不得甚少,三也。將記述者欲以少見奇,非其實錄也。按鍾繇傳云:「公與紹相持,繇為司隷,送馬二千餘匹以給軍。」本紀及世語並云公時有騎六百餘匹,繇馬為安在哉?}
紹復進臨官渡,起土山地道。公亦於內作之,以相應。紹射營中,矢如雨下,行者皆蒙楯,衆大懼。時公糧少,與荀彧書,議欲還許。彧以為「紹悉衆聚官渡,欲與公決勝敗。公以至弱當至彊,若不能制,必為所乘,是天下之大機也。且紹,布衣之雄耳,能聚人而不能用。夫以公之神武明哲而輔以大順,何向而不濟!」公從之。


孫策聞公與紹相持,乃謀襲許,未發,為刺客所殺。


汝南降賊劉辟等叛應紹,略許下。紹使劉備助辟,公使曹仁擊破之。備走,遂破辟屯。


袁紹運穀車數千乘至,公用荀攸計,遣徐晃、史渙邀擊,大破之,盡燒其車。公與紹相拒連月,雖比戰斬將,然衆少糧盡,士卒疲乏。公謂運者曰:「却十五日為汝破紹,不復勞汝矣。」冬十月,紹遣車運穀,使淳于瓊等五人將兵萬餘人送之,宿紹營北四十里。紹謀臣許攸貪財,紹不能足,來奔,因說公擊瓊等。左右疑之,荀攸、賈詡勸公。公乃留曹洪守,自將步騎五千人夜往,會明至。瓊等望見公兵少,出陳門外。公急擊之,瓊退保營,遂攻之。紹遣騎救瓊。左右或言「賊騎稍近,請分兵拒之」。公怒曰:「賊在背後,乃白!」士卒皆殊死戰,大破瓊等,皆斬之。
\gezhu{曹瞞傳曰:公聞攸來,跣出迎之,撫掌笑曰:「子卿遠來,吾事濟矣!」旣入坐,謂公曰:「袁氏軍盛,何以待之?今有幾糧乎?」公曰:「尚可支一歲。」攸曰:「無是,更言之!」又曰:「可支半歲。」攸曰:「足下不欲破袁氏邪,何言之不實也!」公曰:「向言戲之耳。其實可一月,為之柰何?」攸曰:「公孤軍獨守,外無救援而糧穀已盡,此危急之日也。今袁氏輜重有萬餘乘,在故市、烏巢,屯軍無嚴備;今以輕兵襲之,不意而至,燔其積聚,不過三日,袁氏自敗也。」公大喜,乃選精銳步騎,皆用袁軍旗幟,銜枚縛馬口,夜從間道出,人抱束薪,所歷道有問者,語之曰:「袁公恐曹操鈔略後軍,遣兵以益備。」聞者信以為然,皆自若。旣至,圍屯,大放火,營中驚亂。大破之,盡燔其糧穀寶貨,斬督將眭元進、騎督韓莒子、呂威璜、趙叡等首,割得將軍淳于仲簡鼻,未死,殺士卒千餘人,皆取鼻,牛馬割脣舌,以示紹軍。將士皆怛懼。時有夜得仲簡,將以詣麾下,公謂曰:「何為如是?」仲簡曰:「勝負自天,何用為問乎!」公意欲不殺。許攸曰:「明且鑒於鏡,此益不忘人。」乃殺之。}
紹初聞公之擊瓊,謂長子譚曰:「就彼攻瓊等,吾攻拔其營,彼固無所歸矣!」乃使張郃、高覽攻曹洪。郃等聞瓊破,遂來降。紹衆大潰,紹及譚棄軍走,渡河。追之不及,盡收其輜重圖書珎寶,虜其衆。
\gezhu{獻帝起居注曰:公上言「大將軍鄴侯袁紹前與兾州牧韓馥立故大司馬劉虞,刻作金璽,遣故任長畢瑜詣虞,為說命錄之數。又紹與臣書云:『可都鄄城,當有所立。』擅鑄金銀印,孝廉計吏,皆往詣紹。從弟濟陰太守敘與紹書云:『今海內喪敗,天意實在我家,神應有徵,當在尊兄。南兄臣下欲使即位,南兄言,以年則北兄長,以位則北兄重。便欲送璽,會曹操斷道。』紹宗族累世受國重恩,而凶逆無道,乃至於此。輙勒兵馬,與戰官渡,乘聖朝之威,得斬紹大將淳于瓊等八人首,遂大破潰。紹與子譚輕身迸走。凡斬首七萬餘級,輜重財物巨億。」}
公收紹書中,得許下及軍中人書,皆焚之。
\gezhu{魏氏春秋曰:公云:「當紹之彊,孤猶不能自保,而況衆人乎!」}
兾州諸郡多舉城邑降者。


初,桓帝時有黃星見於楚、宋之分,遼東殷馗
\gezhu{馗,古逵字,見三蒼。}
善天文,言後五十歲當有真人起於梁、沛之閒,其鋒不可當。至是凡五十年,而公破紹,天下莫敵矣。


六年夏四月,揚兵河上,擊紹倉亭軍,破之。紹歸,復收散卒,攻定諸叛郡縣。九月,公還許。紹之未破也,使劉備略汝南,汝南賊共都等應之。遣蔡揚擊都,不利,為都所破。公南征備。備聞公自行,走奔劉表,都等皆散。


七年春正月,公軍譙,令曰:「吾起義兵,為天下除暴亂。舊土人民,死喪略盡,國中終日行,不見所識,使吾悽愴傷懷。其舉義兵已來,將士絕無後者,求其親戚以後之,授上田,官給耕牛,置學師以教之。為存者立廟,使祀其先人,魂而有靈,吾百年之後何恨哉!」遂至浚儀,治睢陽渠,遣使以太牢祀橋玄。
\gezhu{襃賞令載公祀文曰:「故太尉橋公,誕敷明德,汎愛博容。國念明訓,士思令謨。靈幽體翳,邈哉晞矣!吾以幼年,逮并堂室,特以頑鄙之姿,為大君子所納。增榮益觀,皆由獎助,猶仲尼稱不如顏淵,李生之厚歎賈復。士死知己,懷此無忘。又承從容約誓之言:『殂逝之後,路有經由,不以斗酒隻雞過相沃酹,車過三步,腹痛勿怪!』雖臨時戲笑之言,非至親之篤好,胡肯為此辭乎?匪謂靈忿,能詒己疾,懷舊惟顧,念之悽愴。奉命東征,屯次鄉里,北望貴土,乃心陵墓。裁致薄奠,公其尚饗!」}
進軍官渡。


紹自軍破後,發病歐血,夏五月死。小子尚代,譚自號車騎將軍,屯黎陽。秋九月,公征之,連戰。譚、尚數敗退,固守。


八年春三月,攻其郭,乃出戰,擊,大破之,譚、尚夜遁。夏四月,進軍鄴。五月還許,留賈信屯黎陽。


己酉,令曰:「司馬法『將軍死綏』,
\gezhu{魏書云:綏,却也。有前一尺,無却一寸。}
故趙括之母,乞不坐括。是古之將者,軍破於外,而家受罪於內也。自命將征行,但賞功而不罰罪,非國典也。其令諸將出征,敗軍者抵罪,失利者免官爵。」
\gezhu{魏書載庚申令曰:「議者或以軍吏雖有功能,德行不足堪任郡國之選,所謂『可與適道,未可與權』。管仲曰:『使賢者食於能則上尊,鬬士食於功則卒輕於死,二者設於國則天下治。』未聞無能之人,不鬬之士,並受祿賞,而可以立功興國者也。故明君不官無功之臣,不賞不戰之士;治平尚德行,有事賞功能。論者之言,一似管窺虎歟!」}


秋七月,令曰:「喪亂已來,十有五年,後生者不見仁義禮讓之風,吾甚傷之。其令郡國各修文學,縣滿五百戶置校官,選其鄉之俊造而教學之,庶幾先王之道不廢,而有以益於天下。」


八月,公征劉表,軍西平。公之去鄴而南也,譚、尚爭兾州,譚為尚所敗,走保平原。尚攻之急,譚遣辛毗乞降請救。諸將皆疑,荀攸勸公許之,
\gezhu{魏書曰:公云:「我攻呂布,表不為寇,官渡之役,不救袁紹,此自守之賊也,宜為後圖。譚、尚狡猾,當乘其亂。縱譚挾詐,不終束手,使我破尚,徧收其地,利自多矣。」乃許之。}
公乃引軍還。冬十月,到黎陽,為子整與譚結婚。
\gezhu{臣松之案:紹死至此,過周五月耳。譚雖出後其伯,不為紹服三年,而於再朞之內以行吉禮,悖矣。魏武或以權宜與之約言;今云結婚,未必便以此年成禮。}
尚聞公北,乃釋平原還鄴。東平呂曠、呂翔叛尚,屯陽平,率其衆降,封為列侯。
\gezhu{魏書曰:譚之圍解,陰以將軍印綬假曠。曠受印送之,公曰:「我固知譚之有小計也。欲使我攻尚,得以其閒略民聚衆,比尚之破,可得自彊以乘我弊也。尚破我盛,何弊之乘乎?」}


九年春正月,濟河,遏淇水入白溝以通糧道。二月,尚復攻譚,留蘇由、審配守鄴。公進軍到洹水,由降。旣至,攻鄴,為土山、地道。武安長尹楷屯毛城,通上黨糧道。夏四月,留曹洪攻鄴,公自將擊楷,破之而還。尚將沮鵠守邯鄲,
\gezhu{沮音菹,河朔閒今猶有此姓。鵠,沮授子也。}
又擊拔之。易陽令韓範、涉長梁岐舉縣降,賜爵關內侯。五月,毀土山、地道,作圍壍,決漳水灌城;城中餓死者過半。秋七月,尚還救鄴,諸將皆以為「此歸師,人自為戰,不如避之」。公曰:「尚從大道來,當避之;若循西山來者,此成禽耳。」尚果循西山來,臨滏水為營。
\gezhu{曹瞞傳曰:遣候者數部前後參之,皆曰「定從西道,已在邯鄲」。公大喜,會諸將曰:「孤已得兾州,諸君知之乎?」皆曰:「不知。」公曰:「諸君方見不乆也。」}
夜遣兵犯圍,公逆擊破走之,遂圍其營。未合,尚懼。故豫州刺史陰夔及陳琳乞降,公不許,為圍益急。尚夜遁,保祁山,追擊之。其將馬延、張顗等臨陣降,衆大潰,尚走中山。盡獲其輜重,得尚印綬節鉞,使尚降人示其家,城中崩沮。八月,審配兄子榮夜開所守城東門內兵。配逆戰,敗,生禽配,斬之,鄴定。公臨祀紹墓,哭之流涕;慰勞紹妻,還其家人寶物,賜雜繒絮,廩食之。
\gezhu{孫盛云:昔者先王之為誅賞也,將以懲惡勸善,永彰鑒戒。紹因世艱危,遂懷逆謀,上議神器,下干國紀。荐社汙宅,古之制也。而乃盡哀於逆臣之冢,加恩於饕餮之室,為政之道,於斯躓矣。夫匿怨友人,前哲所恥,稅驂舊館,義無虛涕,苟道乖好絕,何哭之有!昔漢高失之於項氏,魏武遵謬於此舉,豈非百慮之一失也。}


初,紹與公共起兵,紹問公曰:「若事不輯,則方面何所可據?」公曰:「足下意以為何如?」紹曰:「吾南據河,北阻燕、代,兼戎狄之衆,南向以爭天下,庶可以濟乎?」公曰:「吾任天下之智力,以道御之,無所不可。」
\gezhu{傅子曰:太祖又云:「湯、武之王,豈同上哉?若以險固為資,則不能應機而變化也。」}


九月,令曰:「河北罹袁氏之難,其令無出今年租賦!」重豪彊兼并之法,百姓喜恱。
\gezhu{魏書載公令曰:「有國有家者,不患寡而患不均,不患貧而患不安。袁氏之治也,使豪彊擅恣,親戚兼并;下民貧弱,代出租賦,衒鬻家財,不足應命;審配宗族,至乃藏匿罪人,為逋逃主。欲望百姓親附,甲兵彊盛,豈可得邪!其收田租畝四升,戶出絹二匹、緜二斤而已,他不得擅興發。郡國守相明檢察之,無令彊民有所隱藏,而弱民兼賦也。」}
天子以公領兾州牧,公讓還兖州。


公之圍鄴也,譚略取甘陵、安平、勃海、河閒。尚敗,還中山。譚攻之,尚奔故安,遂并其衆。公遺譚書,責以負約,與之絕婚,女還,然後進軍。譚懼,拔平原,走保南皮。十二月,公入平原,略定諸縣。


十年春正月,攻譚,破之,斬譚,誅其妻子,兾州平。
\gezhu{魏書曰:公攻譚,旦及日中不決;公乃自執桴鼓,士卒咸奮,應時破陷。}
下令曰:「其與袁氏同惡者,與之更始。」令民不得復私讎,禁厚葬,皆一之於法。是月,袁熙大將焦觸、張南等叛攻熙、尚,熙、尚奔三郡烏丸。觸等舉其縣降,封為列侯。初討譚時,民亡椎冰,
\gezhu{臣松之以為討譚時,川渠水凍,使民椎冰以通舩,民憚役而亡。}
令不得降。頃之,亡民有詣門首者,公謂曰:「聽汝則違令,殺汝則誅首,歸深自藏,無為吏所獲。」民垂泣而去;後竟捕得。


夏四月,黑山賊張燕率其衆十餘萬降,封為列侯。故安趙犢、霍奴等殺幽州刺史、涿郡太守。三郡烏丸攻鮮于輔於獷平。
\gezhu{續漢書郡國志曰:獷平,縣名,屬漁陽郡。}
秋八月,公征之,斬犢等,乃渡潞河救獷平,烏丸奔走出塞。


九月,令曰:「阿黨比周,先聖所疾也。聞兾州俗,父子異部,更相毀譽。昔直不疑無兄,世人謂之盜嫂;弟五伯魚三娶孤女,謂之撾婦翁;王鳳擅權,谷永比之申伯;王商忠議,張匡謂之左道:此皆以白為黑,欺天罔君者也。吾欲整齊風俗,四者不除,吾以為羞。」冬十月,公還鄴。


初,袁紹以甥高幹領并州牧,公之拔鄴,幹降,遂以為刺史。幹聞公討烏丸,乃以州叛,執上黨太守,舉兵守壺關口。遣樂進、李典擊之,幹還守壺關城。十一年春正月,公征幹。幹聞之,乃留其別將守城,走入匈奴,求救於單于,單于不受。公圍壺關三月,拔之。幹遂走荊州,上洛都尉王琰捕斬之。


秋八月,公東征海賊管承,至淳于,遣樂進、李典擊破之,承走入海島。割東海之襄賁、郯、戚以益琅邪,省昌慮郡。
\gezhu{魏書載十月乙亥令曰:「夫治世御衆,建立輔弼,誡在面從,詩稱『聽用我謀,庶無大悔』,斯實君臣懇懇之求也。吾充重任,每懼失中,頻年已來,不聞嘉謀,豈吾開延不勤之咎邪?自今已後,諸掾屬治中、別駕,常以月旦各言其失,吾將覽焉。」}


三郡烏丸承天下亂,破幽州,略有漢民合十餘萬戶。袁紹皆立其酋豪為單于,以家人子為己女,妻焉。遼西單于蹋頓尤彊,為紹所厚,故尚兄弟歸之,數入塞為害。公將征之,鑿渠,自呼沲入泒水,
\gezhu{泒音孤。}
名平虜渠;又從泃河口
\gezhu{泃音句。}
鑿入潞河,名泉州渠,以通海。


十二月春二月,公自淳于還鄴。丁酋,令曰:「吾起義兵誅暴亂,於今十九年,所征必克,豈吾功哉?乃賢士大夫之力也。天下雖未悉定,吾當要與賢士大夫共定之;而專饗其勞,吾何以安焉!其促定功行封。」於是大封功臣二十餘人,皆為列侯,其餘各以次受封,及復死事之孤,輕重各有差。
\gezhu{魏書載公令曰:「昔趙奢、竇嬰之為將也,受賜千金,一朝散之,故能濟成大功,永世流聲。吾讀其文,未甞不慕其為人也。與諸將士大夫共從戎事,幸賴賢人不愛其謀,羣士不遺其力,是以夷險平亂,而吾得竊大賞,戶邑三萬。追思竇嬰散金之義,今分所受租與諸將掾屬及故戍於陳、蔡者,庶以疇荅衆勞,不擅大惠也。宜差死事之孤,以租穀及之。若年殷用足,租奉畢入,將大與衆人悉共饗之。」}


將北征三郡烏丸,諸將皆曰:「袁尚,亡虜耳,夷狄貪而無親,豈能為尚用?今深入征之,劉備必說劉表以襲許。萬一為變,事不可悔。」惟郭嘉策表必不能任備,勸公行。夏五月,至無終。秋七月,大水,傍海道不通,田疇請為鄉導,公從之。引軍出盧龍塞,塞外道絕不通,乃壍山堙谷五百餘里,經白檀,歷平岡,涉鮮卑庭,東指柳城。未至二百里,虜乃知之。尚、熙與蹋頓、遼西單于樓班、右北平單于能臣抵之等,將數萬騎逆軍。八月,登白狼山,卒與虜遇,衆甚盛。公車重在後,被甲者少,左右皆懼。公登高,望虜陳不整,乃縱兵擊之,使張遼為先鋒,虜衆大崩,斬蹋頓及名王已下,胡、漢降者二十餘萬口。遼東單于速僕丸及遼西、北平諸豪,棄其種人,與尚、熙奔遼東,衆尚有數千騎。初,遼東太守公孫康恃遠不服。及公破烏丸,或說公遂征之,尚兄弟可禽也。公曰:「吾方使康斬送尚、熙首,不煩兵矣。」九月,公引兵自柳城還,
\gezhu{曹瞞傳曰:時寒且旱,二百里無復水,軍又乏食,殺馬數千匹以為糧,鑿地入三十餘丈乃得水。旣還,利問前諫者,衆莫知其故,人人皆懼。公皆厚賞之,曰:「孤前行,乘危以徼倖,雖得之,天所佐也,顧不可以為常。諸君之諫,萬安之計,是以相賞,後勿難言之。」}
康即斬尚、熙及速僕丸等,傳其首。諸將或問:「公還而康斬送尚、熙,何也?」公曰:「彼素畏尚等,吾急之則并力,緩之則自相圖,其勢然也。」十一月至易水,代郡烏丸行單于普富盧、上郡烏丸行單于那樓將其名王來賀。


十三年春正月,公還鄴,作玄武池以肄舟師。
\gezhu{肄,以四反。三蒼曰:「肄,習也。」}
漢罷三公官,置丞相、御史大夫。夏六月,以公為丞相。
\gezhu{獻帝起居注曰:使太常徐璆即授印綬。御史大夫不領中丞,置長史一人。先賢行狀曰:璆字孟平,廣陵人。少履清爽,立朝正色。歷任城、汝南、東海三郡,所在化行。被徵當還,為袁術所劫。術僭號,欲授以上公之位,璆終不為屈。術死後,璆得術璽,致之漢朝,拜衞尉太常;公為丞相,以位讓璆焉。}


秋七月,公南征劉表。八月,表卒,其子琮代,屯襄陽,劉備屯樊。九月,公到新野,琮遂降,備走夏口。公進軍江陵,下令荊州吏民,與之更始。乃論荊州服從之功,侯者十五人,以劉表大將文聘為江夏太守,使統本兵,引用荊州名士韓嵩、鄧義等。
\gezhu{衞恒四體書勢序曰:上谷王次仲善隷書,始為楷法。至靈帝好書,世多能者。而師宜官為最,甚矜其能,每書,輙削焚其札。梁鵠乃益為版而飲之酒,候其醉而竊其札,鵠卒以攻書至選部尚書。於是公欲為洛陽令,鵠以為北部尉。鵠後依劉表。及荊州平,公募求鵠,鵠懼,自縛詣門,署軍假司馬,使在祕書,以勤書自効。公甞懸著帳中,及以釘壁玩之,謂勝宜官。鵠字孟皇,安定人。魏宮殿題署,皆鵠書也。皇甫謐逸士傳曰:汝南王儁,字子文,少為范滂、許章所識,與南陽岑晊善。公之為布衣,特愛儁;儁亦稱公有治世之具。及袁紹與弟術喪母,歸葬汝南,儁與公會之,會者三萬人。公於外密語儁曰:「天下將亂,為亂魁者必此二人也。欲濟天下,為百姓請命,不先誅此二子,亂今作矣。」儁曰:「如卿之言,濟天下者,舍卿復誰?」相對而笑。儁為人外靜而內明,不應州郡三府之命。公車徵,不到,避地居武陵,歸儁者一百餘家。帝之都許,復徵為尚書,又不就。劉表見紹彊,陰與紹通,儁謂表曰:「曹公,天下之雄也,必能興霸道,繼桓、文之功者也。今乃釋近而就遠,如有一朝之急,遙望漠北之救,不亦難乎!」表不從。儁年六十四,以壽終于武陵,公聞而哀傷。及平荊州,自臨江迎喪,改葬于江陵,表為先賢也。}
益州牧劉璋始受徵役,遣兵給軍。十二月,孫權為備攻合肥。公自江陵征備,至巴丘,遣張憙救合肥。權聞憙至,乃走。公至赤壁,與備戰,不利。於是大疫,吏士多死者,乃引軍還。備遂有荊州、江南諸郡。
\gezhu{山陽公載記曰:公船艦為備所燒,引軍從華容道步歸,遇泥濘,道不通,天又大風,悉使羸兵負草填之,騎乃得過。羸兵為人馬所蹈藉,陷泥中,死者甚衆。軍旣得出,公大喜,諸將問之,公曰:「劉備,吾儔也。但得計少晚;向使早放火,吾徒無類矣。」備尋亦放火而無所及。孫盛異同評曰:案吳志,劉備先破公軍,然後權攻合肥,而此記云權先攻合肥,後有赤壁之事。二者不同,吳志為是。}


十四年春三月,軍至譙,作輕舟,治水軍。秋七月,自渦入淮,出肥水,軍合肥。辛未,令曰:「自頃已來,軍數征行,或遇疫氣,吏士死亡不歸,家室怨曠,百姓流離,而仁者豈樂之哉?不得已也。其令死者家無基業不能自存者,縣官勿絕廩,長吏存卹撫循,以稱吾意。」置揚州郡縣長吏,開芍陂屯田。十二月,軍還譙。


十五年春,下令曰:「自古受命及中興之君,曷甞不得賢人君子與之共治天下者乎!及其得賢也,曾不出閭巷,豈幸相遇哉?上之人不求之耳。今天下尚未定,此特求賢之急時也。『孟公綽為趙、魏老則優,不可以為滕、薛大夫』。若必廉士而後可用,則齊桓其何以霸世!今天下得無有被褐懷玉而釣於渭濵者乎?又得無盜嫂受金而未遇無知者乎?二三子其佐我明揚仄陋,唯才是舉,吾得而用之。」冬,作銅爵臺。
\gezhu{魏武故事載公十二月己亥令曰:「孤始舉孝廉,年少,自以本非巖穴知名之士,恐為海內人之所見凡愚,欲為一郡守,好作政教,以建立名譽,使世士明知之;故在濟南,始除殘去穢,平心選舉,違迕諸常侍。以為彊豪所忿,恐致家禍,故以病還。去官之後,年紀尚少,顧視同歲中,年有五十,未名為老,內自圖之,從此却去二十年,待天下清,乃與同歲中始舉者等耳。故以四時歸鄉里,於譙東五十里築精舍,欲秋夏讀書,冬春射獵,求底下之地,欲以泥水自蔽,絕賔客往來之望,然不能得如意。後徵為都尉,遷典軍校尉,意遂更欲為國家討賊立功,欲望封侯作征西將軍,然後題墓道言『漢故征西將軍曹侯之墓』,此其志也。而遭值董卓之難,興舉義兵。是時合兵能多得耳,然常自損,不欲多之;所以然者,多兵意盛,與彊敵爭,儻更為禍始。故汴水之戰數千,後還到揚州更募,亦復不過三千人,此其本志有限也。後領兖州,破降黃巾三十萬衆。又袁術僭號於九江,下皆稱臣,名門曰建號門,衣被皆為天子之制,兩婦預爭為皇后。志計已定,人有勸術使遂即帝位,露布天下,荅言『曹公尚在,未可也』。後孤討禽其四將,獲其人衆,遂使術窮亡解沮,發病而死。及至袁紹據河北,兵勢彊盛,孤自度勢,實不敵之,但計投死為國,以義滅身,足垂於後。幸而破紹,梟其二子。又劉表自以為宗室,包藏姧心,乍前乍却,以觀世事,據有當州,孤復定之,遂平天下。身為宰相,人臣之貴已極,意望已過矣。今孤言此,若為自大,欲人言盡,故無諱耳。設使國家無有孤,不知當幾人稱帝,幾人稱王。或者人見孤彊盛,又性不信天命之事,恐私心相評,言有不遜之志,妄相忖度,每用耿耿。齊桓、晉文所以垂稱至今日者,以其兵勢廣大,猶能奉事周室也。論語云『三分天下有其二,以服事殷,周之德可謂至德矣』,夫能以大事小也。昔樂毅走趙,趙王欲與之圖燕,樂毅伏而垂泣,對曰:『臣事昭王,猶事大王;臣若獲戾,放在他國,沒世然後已,不忍謀趙之徒隷,況燕後嗣乎!』胡亥之殺蒙恬也,恬曰:『自吾先人及至子孫,積信於秦三世矣;今臣將兵三十餘萬,其勢足以背叛,然自知必死而守義者,不敢辱先人之教以忘先王也。』孤每讀此二人書,未曾不愴然流涕也。孤祖父以至孤身,皆當親重之任,可謂見信者矣,以及子桓兄弟,過於三世矣。孤非徒對諸君說此,也常以語妻妾,皆令深知此意。孤謂之言:『顧我萬年之後,汝曹皆當出嫁,欲令傳道我心,使他人皆知之。』孤此言皆肝鬲之要也。所以勤勤懇懇敘心腹者,見周公有金縢之書以自明,恐人不信之故。然欲孤便爾委捐所典兵衆以還執事,歸就武平侯國,實不可也。何者?誠恐己離兵為人所禍也。旣為子孫計,又己敗則國家傾危,是以不得慕虛名而處實禍,此所不得為也。前朝恩封三子為侯,固辭不受,今更欲受之,非欲復以為榮,欲以為外援,為萬安計。孤聞介推之避晉封。申胥之逃楚賞,未甞不捨書而歎,有以自省也。奉國威靈,仗鉞征伐,推弱以克彊,處小而禽大,意之所圖,動無違事,心之所慮,何向不濟,遂蕩平天下,不辱主命,可謂天助漢室,非人力也。然封兼四縣,食戶三萬,何德堪之!江湖未靜,不可讓位;至於邑土,可得而辭。今上還陽夏、柘、苦三縣戶二萬,但食武平萬戶,且以分損謗議,少減孤之責也。」}


十六年春正月,
\gezhu{魏書曰:庚辰,天子報:減戶五千,分所讓三縣萬五千封三子,植為平原侯,據為范陽侯,豹為饒陽侯,食邑各五千戶。}
天子命公世子丕為五官中郎將,置官屬,為丞相副。太原商曜等以大陵叛,遣夏侯淵、徐晃圍破之。張魯據漢中,三月,遣鍾繇討之。公使淵等出河東與繇會。


是時關中諸將疑繇欲自襲,馬超遂與韓遂、楊秋、李堪、成宜等叛。遣曹仁討之。超等屯潼關,公勑諸將:「關西兵精悍,堅壁勿與戰。」秋七月,公西征,
\gezhu{魏書曰:議者多言「關西兵彊,習長矛,非精選前鋒,則不可以當也」。公謂諸將曰:「戰在我,非在賊也。賊雖習長矛,將使不得以刺,諸君但觀之耳。」}
與超等夾關而軍。公急持之,而潛遣徐晃、朱靈等夜渡蒲阪津,據河西為營。公自潼關北渡,未濟,超赴船急戰。校尉丁斐因放牛馬以餌賊,賊亂取牛馬,公乃得渡,
\gezhu{曹瞞傳曰:公將過河,前隊適渡,超等奄至,公猶坐胡牀不起。張郃等見事急,共引公入船。河水急,北渡,流四五里,超等騎追射之,矢下如雨。諸將見軍敗,不知公所在,皆惶懼,至見,乃悲喜,或流涕。公大笑曰:「今日幾為小賊所困乎!」}
循河為甬道而南。賊退,拒渭口,公乃多設疑兵,潛以舟載兵入渭,為浮橋,夜,分兵結營於渭南。賊夜攻營,伏兵擊破之。超等屯渭南,遣信求割河以西請和,公不許。九月,進軍渡渭。
\gezhu{曹瞞傳曰:時公軍每渡渭,輒為超騎所衝突,營不得立,地又多沙,不可築壘。婁子伯說公曰:「今天寒,可起沙為城,以水灌之,可一夜而成。」公從之,乃多作縑囊以運水,夜渡兵作城,比明,城立,由是公軍盡得渡渭。或疑于時九月,水未應凍。臣松之案魏書:公軍八月至潼關,閏月北渡河,則其年閏八月也,至此容可大寒邪!}
超等數挑戰,又不許;固請割地,求送任子,公用賈詡計,偽許之。韓遂請與公相見,公與遂父同歲孝廉,又與遂同時儕輩,於是交馬語移時,不及軍事,但說京都舊故,拊手歡笑。旣罷,超等問遂:「公何言?」遂曰:「無所言也。」超等疑之。
\gezhu{魏書曰:公後日復與遂等會語,諸將曰:「公與虜交語,不宜輕脫,可為木行馬以為防遏。」公然之。賊將見公,悉於馬上拜,秦、胡觀者,前後重沓,公笑謂賊曰:「爾欲觀曹公邪?亦猶人也,非有四目兩口,但多智耳!」胡前後大觀。又列鐵騎五千為十重陣,精光耀日,賊益震懼。}
他日,公又與遂書,多所點竄,如遂改定者;超等愈疑遂。公乃與克日會戰,先以輕兵挑之,戰良乆,乃縱虎騎夾擊,大破之,斬成宜、李堪等。遂、超等走涼州,楊秋奔安定,關中平。諸將或問公曰:「初,賊守潼關,渭北道缺,不從河東擊馮翊而反守潼關,引日而後北渡,何也?」公曰:「賊守潼關,若吾入河東,賊必引守諸津,則西河未可渡,吾故盛兵向潼關;賊悉衆南守,西河之備虛,故二將得擅取西河;然後引軍北渡,賊不能與吾爭西河者,以有二將之軍也。連車樹柵,為甬道而南,
\gezhu{臣松之案:漢高祖二年,與楚戰滎陽京、索之間,築甬道屬河以取敖倉粟。應劭曰:「恐敵鈔輜重,故築垣牆如街巷也。」今魏武不築垣牆,但連車樹柵以扞兩面。}
旣為不可勝,且以示弱。渡渭為堅壘,虜至不出,所以驕之也;故賊不為營壘而求割地。吾順言許之,所以從其意,使自安而不為備,因畜士卒之力,一旦擊之,所謂疾雷不及掩耳,兵之變化,固非一道也。」始,賊每一部到,公輒有喜色。賊破之後,諸將問其故。公荅曰:「關中長遠,若賊各依險阻,征之,不一二年不可定也。今皆來集,其衆雖多,莫相歸服,軍無適主,一舉可滅,為功差易,吾是以喜。」


冬十月,軍自長安北征楊秋,圍安定。秋降,復其爵位,使留撫其民人。
\gezhu{魏略曰:楊秋,黃初中遷討寇將軍,位特進,封臨涇侯,以壽終。}
十二月,自安定還,留夏侯淵屯長安。


十七年春正月,公還鄴。天子命公贊拜不名,入朝不趨,劔履上殿,如蕭何故事。馬超餘衆梁興等屯藍田,使夏侯淵擊平之。割河內之蕩陰、朝歌、林慮,東郡之衞國、頓丘、東武陽、發干,鉅鹿之廮陶、曲周、南和,廣平之任城,趙之襄國、邯鄲、易陽以益魏郡。


冬十月,公征孫權。


十八年春正月,進軍濡須口,攻破權江西營,獲權都督公孫陽,乃引軍還。詔書并十四州,復為九州。夏四月,至鄴。


五月丙申,天子使御史大夫郗慮持節策命公為魏公
\gezhu{續漢書曰:慮字鴻豫,山陽高平人。少受業於鄭玄,建安初為侍中。虞溥江表傳曰:獻帝甞特見慮及少府孔融,問融曰:「鴻豫何所優長?」融曰:「可與適道,未可與權。」慮舉笏曰:「融昔宰北海,政散民流,其權安在也!」遂與融互相長短,以至不睦。公以書和解之。慮從光祿勳遷為大夫。}
曰:


朕以不德,少遭愍凶,越在西土,遷于唐、衞。當此之時,若綴旒然,
\gezhu{公羊傳曰:「君若贅旒然。贅猶綴也。」何休云:「旒,旂旒也。以旒譬者,言為下所執持東西也。」}
宗廟乏祀,社稷無位;羣凶覬覦,分裂諸夏,率土之民,朕無獲焉,即我高祖之命將墜于地。朕用夙興假寐,震悼于厥心,曰「惟祖惟父,股肱先正,
\gezhu{文侯之命曰:「亦惟先正。」鄭玄云:「先正,先臣。謂公卿大夫也。」}
其孰能恤朕躬」?乃誘天衷,誕育丞相,保乂我皇家,弘濟于艱難,朕實賴之。今將授君典禮,其敬聽朕命。


昔者董卓初興國難,羣后釋位以謀王室,
\gezhu{左氏傳曰:「諸侯釋位以間王政。」服虔曰:「言諸侯釋其私政而佐王室。」}
君則攝進,首啟戎行,此君之忠于本朝也。後及黃巾反易天常,侵我三州,延及平民,君又翦之,以寧東夏,此又君之功也。韓暹、楊奉專用威命,君則致討,克黜其難,遂遷許都,造我京畿,設官兆祀,不失舊物,天地鬼神於是獲乂,此又君之功也。袁術僭逆,肆於淮南,懾憚君靈,用丕顯謀,蘄陽之役,橋蕤授首,稜威南邁,術以隕潰,此又君之功也。迴戈東征,呂布就戮,乘轅將返,張楊殂斃,眭固伏罪,張繡稽服,此又君之功也。袁紹逆亂天常,謀危社稷,憑恃其衆,稱兵內侮,當此之時,王師寡弱,天下寒心,莫有固志,君執大節,精貫白日,奮其武怒,運其神策,致屆官渡,大殲醜類,
\gezhu{詩曰:「致天之屆,于牧之野。」鄭玄云:「屆,極也。」鴻範曰:「鯀則殛死。」}
俾我國家拯於危墜,此又君之功也。濟師洪河,拓定四州,袁譚、高幹,咸梟其首,海盜奔迸,黑山順軌,此又君之功也。烏丸三種,崇亂二世,袁尚因之,逼據塞北,束馬縣車,一征而滅,此又君之功也。劉表背誕,不供貢職,王師首路,威風先逝,百城八郡,交臂屈膝,此又君之功也。馬超、成宜,同惡相濟,濵據河、潼,求逞所欲,殄之渭南,獻馘萬計,遂定邊境,撫和戎狄,此又君之功也。鮮卑、丁零,重譯而至,箄于、白屋,請吏率職,此又君之功也。君有定天下之功,重之以明德,班敘海內,宣美風俗,旁施勤教,恤慎刑獄,吏無苛政,民無懷慝;敦崇帝族,表繼絕世,舊德前功,罔不咸秩;雖伊尹格于皇天,周公光于四海,方之蔑如也。


朕聞先王並建明德,胙之以土,分之以民,崇其寵章,備其禮物,所以藩衞王室,左右厥世也。其在周成,管、蔡不靜,懲難念功,乃使邵康公賜齊太公履,東至于海,西至于河,南至于穆陵,北至于無棣,五侯九伯,實得征之,世祚太師,以表東海;爰及襄王,亦有楚人不供王職,又命晉文登為侯伯,錫以二輅、虎賁、鈇鉞、秬鬯、弓矢,大啟南陽,世作盟主。故周室之不壞,繄二國是賴。今君稱丕顯德,明保朕躬,奉荅天命,導揚弘烈,緩爰九域,莫不率俾,
\gezhu{盤庚曰:「綏爰有衆。」鄭玄曰:「爰,於也,安隱於其衆也。」君奭曰:「海隅出日,罔不率俾。」率,循也。俾,使也。四海之隅,日出所照,無不循度而可使也。}
功高于伊、周,而賞卑於齊、晉,朕甚恧焉。朕以眇眇之身,託于兆民之上,永思厥艱,若涉淵冰,非君攸濟,朕無任焉。今以兾州之河東、河內、魏郡、趙國、中山、常山、鉅鹿、安平、甘陵、平原凡十郡,封君為魏公。錫君玄土,苴以白茅;爰契爾龜,用建冢社。昔在周室,畢公、毛公入為卿佐,周、邵師保出為二伯,外內之任,君實宜之,其以丞相領兾州牧如故。又加君九錫,其敬聽朕命。


以君經緯禮律,為民軌儀,使安職業,無或遷志,是用錫君大輅、戎輅各一,玄牡二駟。君勸分務本,穡人昏作,
\gezhu{盤庚曰:「墮農自安,不昏作勞。」鄭玄云:「昏,勉也。」}
粟帛滯積,大業惟興,是用錫君衮冕之服,赤舄副焉。君敦尚謙讓,俾民興行,少長有禮,上下咸和,是用錫君軒縣之樂,六佾之舞。君翼宣風化,爰發四方,遠人革面,華夏充實,是用錫君朱戶以居。君研其明哲,思帝所難,官才任賢,羣善必舉,是用錫君納陛以登。君秉國之鈞,正色處中,纖豪之惡,靡不抑退,是用錫君虎賁之士三百人。君糾虔天刑,章厥有罪,
\gezhu{「糾虔天刑」語出國語,韋昭注曰:「糾,察也。虔,敬也。刑,法也。」}
犯關干紀,莫不誅殛,是用錫君鈇鉞各一。君龍驤虎視,旁眺八維,掩討逆節,折衝四海,是用錫君彤弓一,彤矢百,玈弓十,玈矢千。君以溫恭為基,孝友為德,明允篤誠,感于朕思,是用錫君秬鬯一卣,珪瓚副焉。魏國置丞相已下羣卿百寮,皆如漢初諸侯王之制。往欽哉,敬服朕命!簡恤爾衆,時亮庶功,用終爾顯德,對揚我高祖之休命!
\gezhu{後漢尚書左丞潘勗之辭也。勗字元茂,陳留中牟人。魏書載公令曰:「夫受九錫,廣開土宇,周公其人也。漢之異姓八王者,與高祖俱起布衣,刱定王業,其功至大,吾何可比之?」前後三讓。於是中軍師王凌、謝亭侯荀攸、前軍師東武亭侯鍾繇、左軍師涼茂、右軍師毛玠、平虜將軍華鄉侯劉勳、建武將軍清苑亭侯劉若、伏波將軍高安侯夏侯惇、揚武將軍都亭侯王忠、奮威將軍樂鄉侯劉展、建忠將軍昌鄉亭侯鮮于輔、奮武將軍安國亭侯程昱、太中大夫都鄉侯賈詡、軍師祭酒千秋亭侯董昭、都亭侯薛洪、南鄉亭侯董蒙、關內侯王粲、傅巽、祭酒王選、袁奐、王朗、張承、任藩、杜襲、中護軍國明亭侯曹洪、中領軍萬歲亭侯韓浩、行驍騎將軍安平亭侯曹仁、領護軍將軍王圖、長史萬潛、謝奐、袁霸等勸進曰:「自古三代,胙臣以土,受命中興,封秩輔佐,皆所以襃功賞德,為國藩衞也。徃者天下崩亂,羣凶豪起,顛越跋扈之險,不可忍言。明公奮身出命以徇其難,誅二袁篡盜之逆,滅黃巾賊亂之類,殄夷首逆,芟撥荒穢,沐浴霜露二十餘年,書契已來,未有若此功者。昔周公承文、武之迹,受已成之業,高枕墨筆,拱揖羣后,商、奄之勤,不過二年,呂望因三分有二之形,據八百諸侯之勢,暫把旄鉞,一時指麾,然皆大啟土宇,跨州兼國。周公八子,並為侯伯,白牡騂剛,郊祀天地,典策備物,擬則王室,榮章寵盛如此之弘也。逮至漢興,佐命之臣,張耳、吳芮,其功至薄,亦連城開地,南面稱孤。此皆明君達主行之於上,賢臣聖宰受之於下,三代令典,漢帝明制。今比勞則周、呂逸,計功則張、吳微,論制則齊、魯重,言地則長沙多;然則魏國之封,九錫之榮,況於舊賞,猶懷玉而被褐也。且列侯諸將,幸攀龍驥,得竊微勞,佩紫懷黃,蓋以百數,亦將因此傳之萬世,而明公獨辭賞於上,將使其下懷不自安,上違聖朝歡心,下失冠帶至望,忘輔弼之大業,信匹夫之細行,攸等所大懼也。」於是公勑外為章,但受魏郡。攸等復曰:「伏見魏國初封,聖朝發慮,稽謀羣寮,然後策命;而明公乆違上指,不即大禮。今旣虔奉詔命,副順衆望,又欲辭多當少,讓九受一,是猶漢朝之賞不行,而攸等之請未許也。昔齊、魯之封,奄有東海,疆域井賦,四百萬家,基隆業廣,易以立功,故能成翼戴之勳,立一匡之績。今魏國雖有十郡之名,猶減於曲阜,計其戶數,不能參半,以藩衞王室,立垣樹屏,猶未足也。且聖上覽亡秦無輔之禍,懲曩日震蕩之艱,託建忠賢,廢墜是為,願明公恭承帝命,無或拒違。」公乃受命。魏略載公上書謝曰:「臣蒙先帝厚恩,致位郎署,受性疲怠,意望畢足,非敢希望高位,庶幾顯達。會董卓作亂,義當死難,故敢奮身出命,摧鋒率衆,遂值千載之運,奉役目下。當二袁炎沸侵侮之際,陛下與臣寒心同憂,顧瞻京師,進受猛敵,常恐君臣俱陷虎口,誠不自意能全首領。賴祖宗靈祐,醜類夷滅,得使微臣竊名其間。陛下加恩,授以上相,封爵寵祿,豐大弘厚,生平之願,實不望也。口與心計,幸且待罪,保持列侯,遺付子孫,自託聖世,永無憂責。不意陛下乃發盛意,開國備錫,以貺愚臣,地比齊、魯,禮同藩王,非臣無功所宜膺據。歸情上聞,不蒙聽許,嚴詔切至,誠使臣心俯仰偪迫。伏自惟省,列在大臣,命制王室,身非己有,豈敢自私,遂其愚意,亦將黜退,令就初服。今奉疆土,備數藩翰,非敢遠期,慮有後世;至於父子相誓終身,灰軀盡命,報塞厚恩。天威在顏,悚懼受詔。」}


秋七月,始建魏社稷宗廟。天子娉公三女為貴人,少者待年於國。
\gezhu{獻帝起居注曰:使使持節行太常大司農安陽亭侯王邑,齎璧、帛、玄纁、絹五萬匹之鄴納娉,介者五人,皆以議郎行大夫事,副介一人。}
九月,作金虎臺,鑿渠引漳水入白溝以通河。冬十月,分魏郡為東西部,置都尉。十一月,初置尚書、侍中、六卿。
\gezhu{魏氏春秋曰:以荀攸為尚書令,涼茂為僕射,毛玠、崔琰、常林、徐奕、何夔為尚書,王粲、杜襲、衞覬、和洽為侍中。}


馬超在漢陽,復因羌、胡為害,氐王千萬叛應超,屯興國。使夏侯淵討之。


十九年春正月,始耕籍田。南安趙衢、漢陽尹奉等討超,梟其妻子,超奔漢中。韓遂徙金城,入氐王千萬部,率羌、胡萬餘騎與夏侯淵戰,擊,大破之,遂走西平。淵與諸將攻興國,屠之。省安東、永陽郡。


安定太守毌丘興將之官,公戒之曰:「羌,胡欲與中國通,自當遣人來,慎勿遣人徃。善人難得,必將教羌、胡妄有所請求,因欲以自利;不從便為失異俗意,從之則無益事。」興至,遣校尉范陵至羌中,陵果教羌,使自請為屬國都尉。公曰:「吾預知當爾,非聖也,但更事多耳。」
\gezhu{獻帝起居注曰:使行太常事大司農安陽亭侯王邑與宗正劉艾,皆持節,介者五人,齎束帛駟馬,及給事黃門侍郎、掖庭丞、中常侍二人,迎二貴人于魏公國。二月癸亥,又於魏公宗廟授二貴人印綬。甲子,詣魏公宮延秋門,迎貴人升車。魏遣郎中令、少府、博士、御府乘黃廄令、丞相掾屬侍送貴人。癸酉,二貴人至洧倉中,遣侍中丹將宂從虎賁前後駱驛往迎之。乙亥,二貴人入宮,御史大夫、中二千石將大夫、議郎會殿中,魏國二卿及侍中、中郎二人,與漢公卿並升殿宴。}


三月,天子使魏公位在諸侯王上,改授金璽,赤紱、遠遊冠。
\gezhu{獻帝起居注曰:使左中郎將楊宣、亭侯裴茂持節、印授之。}


秋七月,公征孫權。
\gezhu{九州春秋曰:參軍傅幹諫曰:「治天下之大具有二,文與武也;用武則先威,用文則先德,威德足以相濟,而後王道備矣。往者天下大亂,上下失序,明公用武攘之,十平其九。今未承王命者,吳與蜀也,吳有長江之險,蜀有崇山之阻,難以威服,易以德懷。愚以為可且按甲寢兵,息軍養士,分土定封,論功行賞,若此則內外之心固,有功者勸,而天下知制矣。然後漸興學校,以導其善性而長其義節。公神武震於四海,若修文以濟之,則普天之下,無思不服矣。今舉十萬之衆,頓之長江之濵,若賊負固深藏,則士馬不能逞其能,奇變無所用其權,則大威有屈而敵心未能服矣。唯明公思虞舜舞干戚之義,全威養德,以道制勝。」公不從,軍遂無功。幹字彥材,北地人,終於丞相倉曹屬。有子曰玄。}


初,隴西宋建自稱河首平漢王,聚衆枹罕,改元,置百官,三十餘年。遣夏侯淵自興國討之。冬十月,屠枹罕,斬建,涼州平。


公自合肥還。


十一月,漢皇后伏氏坐昔與父故屯騎校尉完書,云帝以董承被誅怨恨公,辭甚醜惡,發聞,后廢黜死,兄弟皆伏法。
\gezhu{曹瞞傳曰:公遣華歆勒兵入宮收后,后閉戶匿壁中。歆壞戶發壁,牽后出。帝時與御史大夫郗慮坐,后被髮徒跣過,執帝手曰:「不能復相活邪?」帝曰:「我亦不自知命在何時也。」帝謂慮曰:「郗公,天下寧有是乎!」遂將后殺之,完及宗族死者數百人。}


十二月,公至孟津。天子命公置旄頭,宮殿設鍾虡。乙未,令曰:「夫有行之士未必能進取,進取之士未必能有行也。陳平豈篤行,蘇秦豈守信邪?而陳平定漢業,蘇秦濟弱燕。由此言之,士有偏短,庸可廢乎!有司明思此義,則士無遺滯,官無廢業矣。」又曰:「夫刑,百姓之命也,而軍中典獄者或非其人,而任以三軍死生之事,吾甚懼之。其選明達法理者,使持典刑。」於是置理曹掾屬。


二十年春正月,天子立公中女為皇后。省雲中、定襄、五原、朔方郡,郡置一縣領其民,合以為新興郡。


三月,公西征張魯,至陳倉,將自武都入氐;氐人塞道,先遣張郃、朱靈等攻破之。夏四月,公自陳倉以出散關,至河池。氐王竇茂衆萬餘人,恃險不服,五月,公攻屠之。西平、金城諸將麴演、蔣石等共斬送韓遂首。
\gezhu{典略曰:遂字文約,始與同郡邊章俱著名西州。章為督軍從事。遂奉計詣京師,何進宿聞其名,特與相見,遂說進使誅諸閹人,進不從,乃求歸。會涼州宋揚、北宮玉等反,舉章、遂為主,章尋病卒,遂為揚等所劫,不得已,遂阻兵為亂,積三十二年,至是乃死,年七十餘矣。劉艾靈帝紀曰:章,一名元。}
秋七月,公至陽平。張魯使弟衞與將楊昂等據陽平關,橫山築城十餘里,攻之不能拔,乃引軍還。賊見大軍退,其守備解散。公乃密遣解𢢼、高祚等乘險夜襲,大破之,斬其將楊任,進攻衞,衞等夜遁,魯潰奔巴中。公軍入南鄭,盡得魯府庫珍寶。
\gezhu{魏書曰:軍自武都山行千里,升降險阻,軍人勞苦;公於是大饗,莫不忘其勞。}
巴、漢皆降。復漢寧郡為漢中;分漢中之安陽、西城為西城郡,置太守;分錫、上庸郡,置都尉。


八月,孫權圍合肥,張遼、李典擊破之。


九月,巴七姓夷王朴胡、賨邑侯杜濩舉巴夷、賨民來附,
\gezhu{孫盛曰:朴音浮。濩音戶。}
於是分巴郡,以胡為巴東太守,濩為巴西太守,皆封列侯。天子命公承制封拜諸侯守相。
\gezhu{孔衍漢魏春秋曰:天子以公典任于外,臨事之賞,或宜速疾,乃命公得承制封拜諸侯守相,詔曰:「夫軍之大事,在茲賞罰,勸善懲惡,宜不旋時,故司馬法曰『賞不逾日』者,欲民速覩為善之利也。昔在中興,鄧禹入關,承制拜軍祭酒李文為河東太守,來歙又承制拜高峻為通路將軍,察其本傳,皆非先請,明臨事刻印也,斯則世祖神明,權達損益,蓋所用速示威懷而著鴻勳也。其春秋之義,大夫出疆,有專命之事,苟所以利社稷安國家而已。況君秉任二伯,師尹九有,實征夷夏,軍行蕃甸之外,失得在於斯須之間,停賞俟詔以滯世務,固非朕之所圖也。自今已後,臨事所甄,當加寵號者,其便刻印章假授,咸使忠義得相獎勵,勿有疑焉。」}


冬十月,始置名號侯至五大夫,與舊列侯、關內侯凡六等,以賞軍功。
\gezhu{魏書曰:置名號侯爵十八級,關中侯爵十七級,皆金印紫綬;又置關內外侯十六級,銅印龜紐墨綬;五大夫十五級,銅印環紐,亦墨綬,皆不食租,與舊列侯關內侯凡六等。臣松之以為今之虛封蓋自此始。}


十一月,魯自巴中將其餘衆降。封魯及五子皆為列侯。劉備襲劉璋,取益州,遂據巴中;遣張郃擊之。


十二月,公自南鄭還,留夏侯淵屯漢中。
\gezhu{是行也,侍中王粲作五言詩以美其事,曰:「從軍有苦樂,但問所從誰。所從神且武,安得乆勞師?相公征關右,赫怒振天威,一舉滅獯虜,再舉服羌夷,西牧邊地賊,忽若俯拾遺。陳賞越山嶽,酒肉踰川坻,軍中多饒飫,人馬皆溢肥,徒行兼乘還,空出有餘資。拓土三千里,往反速如飛,歌舞入鄴城,所願獲無違。」}


二十一年春二月,公還鄴。
\gezhu{魏書曰:辛未,有司以太牢告至,策勳于廟,甲午始春祠,令曰:「議者以為祠廟上殿當解履。吾受錫命,帶劔不解履上殿。今有事于廟而解履,是尊先公而替王命,敬父祖而簡君主,故吾不敢解履上殿也。又臨祭就洗,以手擬水而不盥。夫盥以絜為敬,未聞擬向不盥之禮,且『祭神如神在』,故吾親受水而盥也。又降神禮訖,下階就幕而立,須奏樂畢竟,似若不愆,烈祖遲祭,不速訖也。故吾坐俟樂闋送神乃起也。受胙納神以授侍中,此為敬恭不終實也,古者親執祭事,故吾親納于神,終抱而歸也。仲尼曰『雖違衆,吾從下』,誠哉斯言也。」}
三月壬寅,公親耕籍田。
\gezhu{魏書曰:有司奏:「四時講武於農隙。漢承秦制,三時不講,唯十月都試車馬,幸長水南門,會五營士為八陣進退,名曰乘之。今金革未偃,士民素習,自今已後,可無四時講武,但以立秋擇吉日大朝車騎,號曰治兵,上合禮名,下承漢制。」奏可。}
夏五月,天子進公爵為魏王。
\gezhu{獻帝傳載詔曰:「自古帝王,雖號稱相變,爵等不同,至乎襃崇元勳,建立功德,光啟氏姓,延于子孫,庶姓之與親,豈有殊焉。昔我聖祖受命,刱業肇基,造我區夏,鑒古今之制,通爵等之差,盡封山川以立藩屏,使異姓親戚,並列土地,據國而王,所以保乂天命,安固萬嗣。歷世承平,臣主無事。世祖中興而時有難易,是以曠年數百,無異姓諸侯王之位。朕以不德,繼序弘業,遭率土分崩,羣兇縱毒,自西徂東,辛苦卑約。當此之際,唯恐溺入于難,以羞先帝之聖德。賴皇天之靈,俾君秉義奮身,震迅神武,捍朕于艱難,獲保宗廟,華夏遺民,含氣之倫,莫不蒙焉。君勤過稷、禹,忠侔伊、周,而掩之以謙讓,守之以彌恭,是以往者初開魏國,錫君土宇,懼君之違命,慮君之固辭,故且懷志屈意,封君為上公,欲以欽順高義,須俟勳績。韓遂、宋建,南結巴、蜀,羣逆合從,圖危社稷,君復命將,龍驤虎奮,梟其元首,屠其窟栖。曁至西征,陽平之役,親擐甲冑,深入險阻,芟夷蝥賊,殄其兇醜,盪定西陲,懸旌萬里,聲教遠振,寧我區宇。蓋唐、虞之盛,三后樹功,文、武之興,旦、奭作輔,二祖成業,英豪佐命;夫以聖哲之君,事為己任,猶錫士班瑞以報功臣,豈有如朕寡德,仗君以濟,而賞典不豐,將何以荅神祇慰萬方哉?今進君爵為魏王,使使持節行御史大夫、宗正劉艾奉策璽玄土之社,苴以白茅,金虎符第一至第五,竹使符第一至十。君其正王位,以丞相領兾州牧如故。其上魏公璽綬符冊。敬服朕命,簡恤爾衆,克綏庶績,以揚我祖宗之休命。」魏王上書三辭,詔三報不許。又手詔曰:「大聖以功德為高美,以忠和為典訓,故刱業垂名,使百世可希,行道制義,使力行可效,是以勳烈無窮,休光茂著。稷、契載元首之聦明,周、邵因文、武之智用,雖經營庶官,仰歎俯思,其對豈有若君者哉?朕惟古人之功,美之如彼,思君忠勤之績,茂之如此,是以每將鏤符析瑞,陳禮命冊,寤寐慨然,自忘守文之不德焉。今君重違朕命,固辭懇切,非所以稱朕心,而訓後世也。其抑志撙節,勿復固辭。」四體書勢序曰:梁鵠以公為北部尉。曹瞞傳曰:為尚書右丞司馬建公所舉。及公為王,召建公到鄴,與歡飲,謂建公曰:「孤今日可復作尉否?」建公曰:「昔舉大王時,適可作尉耳。」王大笑。建公名防,司馬宣王之父。臣松之案司馬彪序傳,建公不為右丞,疑此不然,而王隱晉書云趙王篡位,欲尊祖為帝,博士馬平議稱京兆府君昔舉魏武帝為北部尉,賊不犯界,如此則為有徵。}
代郡烏丸行單于普富盧與其侯王來朝。天子命王女為公主,食湯沐邑。秋七月,匈奴南單于呼廚泉將其名王來朝,待以客禮,遂留魏,使右賢王去卑監其國。八月,以大理鍾繇為相國。
\gezhu{魏書曰:始置奉常宗正官。}


冬十月,治兵,
\gezhu{魏書曰:王親執金鼔以令進退。}
遂征孫權,十一月至譙。


二十二年春正月,王軍居巢,二月,進軍屯江西郝谿。權在濡須口築城拒守,遂逼攻之,權退走。三月,王引軍還,留夏侯惇、曹仁、張遼等屯居巢。


夏四月,天子命王設天子旌旗,出入稱警蹕。五月,作泮宮。六月,以軍師華歆為御史大夫。
\gezhu{魏書曰:初置衞尉官。秋八月,令曰:「昔伊摯、傅說出於賤人,管仲,桓公賊也,皆用之以興。蕭何、曹參,縣吏也,韓信、陳平負汙辱之名,有見笑之恥,卒能成就王業,聲著千載。吳起貪將,殺妻自信,散金求官,母死不歸,然在魏,奏人不敢東向,在楚則三晉不敢南謀。今天下得無有至德之人放在民間,及果勇不顧,臨敵力戰;若文俗之吏,高才異質,或堪為將守;負汙辱之名,見笑之行,或不仁不孝而有治國用兵之術:其各舉所知,勿有所遺。」}
冬十月,天子命王冕十有二旒,乘金根車,駕六馬,設五時副車,以五官中郎將丕為魏太子。


劉備遣張飛、馬超、吳蘭等屯下辯;遣曹洪拒之。


二十三年春正月,漢太醫令吉本與少府耿紀、司直韋晃等反,攻許,燒丞相長史王必營,
\gezhu{魏武故事載令曰:「領長史王必,是吾披荊棘時吏也。忠能勤事,心如鐵石,國之良吏也。蹉跌乆未辟之,捨騏驥而弗乘,焉遑遑而更求哉?故教辟之,已署所宜,便以領長史統事如故。」}
必與潁川典農中郎將嚴匡討斬之。
\gezhu{三輔決錄注曰:時有京兆金禕字德禕,自以世為漢臣,自日磾討莽何羅,忠誠顯著,名節累葉。覩漢祚將移,謂可季興,乃喟然發憤,遂與耿紀、韋晃、吉本、本子邈、邈弟穆等結謀。紀字季行,少有美名,為丞相掾,王甚敬異之,遷侍中,守少府。邈字文然,穆字思然,以禕慷慨有日磾之風,又與王必善,因以閒之,若殺必,欲挾天子以攻魏,南援劉備。時關羽彊盛,而王在鄴,留必典兵督許中事。文然等率雜人及家僮千餘人夜燒門攻必,禕遣人為內應,射必中肩。必不知攻者為誰,以素與禕善,走投禕,夜喚德禕,禕家不知是必,謂為文然等,錯應曰:「王長史已死乎?卿曹事立矣!」必乃更他路奔。一曰:必欲投禕,其帳下督謂必曰:「今日事竟知誰門而投入乎?」扶必奔南城。會天明,必猶在,文然等衆散,故敗。後十餘日,必竟以創死。獻帝春秋曰:收紀、晃等,將斬之,紀呼魏王名曰:「恨吾不自生意,竟為羣兒所誤耳!」晃頓首搏頰,以至於死。山陽公載記曰:王聞王必死,盛怒,召漢百官詣鄴,令救火者左,不救火者右。衆人以為救火者必無罪,皆附左;王以為「不救火者非助亂,救火乃實賊也」。皆殺之。}


曹洪破吳蘭,斬其將任夔等。三月,張飛、馬超走漢中,陰平氐強端斬吳蘭,傳其首。


夏四月,代郡、上谷烏丸無臣氐等叛,遣鄢陵侯彰討破之。
\gezhu{魏書載王令曰:「去冬天降疫癘,民有凋傷,軍興于外,墾田損少,吾甚憂之。其令吏民男女:女年七十已上無夫子,若年十二已下無父母兄弟,及目無所見,手不能作,足不能行,而無妻子父兄產業者,廩食終身。幼者至十二止,貧窮不能自贍者,隨口給貸。老耄須待養者,年九十已上,復不事,家一人。」}


六月,令曰:「古之葬者,必居瘠薄之地。其規西門豹祠西原上為壽陵,因高為基,不封不樹。周禮冢人掌公墓之地,凡諸侯居左右以前,卿大夫居後,漢制亦謂之陪陵。其公卿大臣列將有功者,宜陪壽陵,其廣為兆域,使足相容。」


秋七月,治兵,遂西征劉備,九月,至長安。


冬十月,宛守將侯音等反,執南陽太守,劫略吏民,保宛。初,曹仁討關羽,屯樊城,是月使仁圍宛。


二十四年春正月,仁屠宛,斬音。
\gezhu{曹瞞傳曰:是時南陽間苦繇役,音於是執太守東里襃,與吏民共反,與關羽連和。南陽功曹宗子卿往說音曰:「足下順民心,舉大事,遠近莫不望風;然執郡將,逆而無益,何不遣之。吾與子共戮力,比曹公軍來,關羽兵亦至矣。」音從之,即釋遣太守。子卿因夜踰城亡出,遂與太守收餘民圍音,會曹仁軍至,共滅之。}


夏侯淵與劉備戰於陽平,為備所殺。三月,王自長安出斜谷,軍遮要以臨漢中,遂至陽平。備因險拒守。
\gezhu{九州春秋曰:時王欲還,出令曰「雞肋」,官屬不知所謂。主簿楊脩便自嚴裝,人驚問脩:「何以知之?」脩曰:「夫雞肋,棄之如可惜,食之無所得,以比漢中,知王欲還也。」}


夏五月,引軍還長安。


秋七月,以夫人卞氏為王后。遣于禁助曹仁擊關羽。八月,漢水溢,灌禁軍,軍沒,羽獲禁,遂圍仁。使徐晃救之。


九月,相國鍾繇坐西曹掾魏諷反免。
\gezhu{世語曰:諷字子京,沛人,有惑衆才,傾動鄴都,鍾繇由是辟焉。大軍未反,諷潛結徒黨,又與長樂衞尉陳禕謀襲鄴。未及期,禕懼,告之太子,誅諷,坐死者數十人。王昶家誡曰「濟陰魏諷」,而此云沛人,未詳。}


冬十月,軍還洛陽。
\gezhu{曹瞞傳曰:王更脩治北部尉廨,令過於舊。}
孫權遣使上書,以討關羽自效。王自洛陽南征羽,未至,晃攻羽,破之,羽走,仁圍解。王軍摩陂。
\gezhu{魏略曰:孫權上書稱臣,稱說天命。王以權書示外曰:「是兒欲踞吾著爐火上邪!」侍中陳羣、尚書桓階奏曰:「漢自安帝已來,政去公室,國統數絕,至於今者,唯有名號,尺土一民,皆非漢有,期運乆已盡,歷數乆已終,非適今日也。是以桓、靈之閒,諸明圖緯者,皆言『漢行氣盡,黃家當興』。殿下應期,十分天下而有其九,以服事漢,羣生注望,遐邇怨歎,是故孫權在遠稱臣,此天人之應,異氣齊聲。臣愚以為虞、夏不以謙辭,殷、周不吝誅放,畏天知命,無所與讓也。」魏氏春秋曰:夏侯惇謂王曰:「天下咸知漢祚已盡,異代方起。自古已來,能除民害為百姓所歸者,即民主也。今殿下即戎三十餘年,功德著於黎庶,為天下所依歸,應天順民,復何疑哉!」王曰:「『施於有政,是亦為政』。若天命在吾,吾為周文王矣。」曹瞞傳及世語並云桓階勸王正位,夏侯惇以為宜先滅蜀,蜀亡則吳服,二方旣定,然後遵舜、禹之軌,王從之。及至王薨,惇追恨前言,發病卒。孫盛評曰:夏侯惇恥為漢官,求受魏印,桓階方惇,有義直之節;考其傳記,世語為妄矣。}


二十五年春正月,至洛陽。權擊斬羽,傳其首。


庚子,王崩于洛陽,年六十六。
\gezhu{世語曰:太祖自漢中至洛陽,起建始殿,伐濯龍祠而樹血出。曹瞞傳曰:王使工蘇越徙美棃,掘之,根傷盡出血。越白狀,王躬自視而惡之,以為不祥,還遂寑疾。}
遺令曰:「天下尚未安定,未得遵古也。葬畢,皆除服。其將兵屯戍者,皆不得離屯部。有司各率乃職。斂以時服,無藏金玉珍寶。」謚曰武王。二月丁卯,葬高陵。


\gezhu{魏書曰:太祖自統御海內,芟夷羣醜,其行軍用師,大較依孫、吳之法,而因事設奇,譎敵制勝,變化如神。自作兵書十萬餘言,諸將征伐,皆以新書從事。臨事又手為節度,從令者克捷,違教者負敗。與虜對陣,意思安閑,如不欲戰,然及至決機乘勝,氣勢盈溢,故每戰必克,軍無幸勝。知人善察,難眩以偽,拔于禁、樂進於行陣之間,取張遼、徐晃於亡虜之內,皆佐命立功,列為名將;其餘拔出細微,登為牧守者,不可勝數。是以刱造大業,文武並施,御軍三十餘年,手不捨書,晝則講武策,夜則思經傳,登高必賦,及造新詩,被之管絃,皆成樂章。才力絕人,手射飛鳥,躬禽猛獸,甞於南皮一日射雉獲六十三頭。及造作宮室,繕治器械,無不為之法則,皆盡其意。雅性節儉,不好華麗,後宮衣不錦繡,侍御履不二采,帷帳屏風,壞則補納,茵蓐取溫,無有緣飾。攻城拔邑,得靡麗之物,則悉以賜有功,勳勞宜賞,不吝千金,無功望施,分豪不與,四方獻御,與羣下共之。常以送終之制,襲稱之數,繁而無益,俗又過之,故預自制終亡衣服,四篋而已。}


\gezhu{傅子曰:太祖愍嫁娶之奢僭,公女適人,皆以皁帳,從婢不過十人。}


\gezhu{張華博物志曰:漢世,安平崔瑗、瑗子寔、弘農張芝、芝弟昶並善草書,而太祖亞之。桓譚、蔡邕善音樂,馮翊山子道、王九真、郭凱等善圍棊,太祖皆與埒能。又好養性法,亦解方藥,招引方術之士,廬江左慈、譙郡華他、甘陵甘始、陽城郄儉無不畢至,又習啖野葛至一尺,亦得少多飲鴆酒。}


\gezhu{傅子曰:漢末王公,多委王服,以幅巾為雅,是以袁紹、崔豹之徒,雖為將帥,皆著縑巾。魏太祖以天下凶荒,資財乏匱,擬古皮弁,裁縑帛以為帢,合于簡易隨時之義,以色別其貴賤,于今施行,可謂軍容,非國容也。}


\gezhu{曹瞞傳曰:太祖為人佻易無威重,好音樂,倡優在側,常以日達夕。被服輕綃,身自佩小鞶囊,以盛手巾細物,時或冠帢帽以見賔客。每與人談論,戲弄言誦,盡無所隱,及歡恱大笑,至以頭沒柸案中,肴膳皆沾洿巾幘,其輕易如此。然持法峻刻,諸將有計畫勝出己者,隨以法誅之,及故人舊怨,亦皆無餘。其所刑殺,輒對之垂涕嗟痛之,終無所活。初,袁忠為沛相,甞欲以法治太祖,沛國桓邵亦輕之,及在兖州,陳留邊讓言議頗侵太祖,太祖殺讓,族其家。忠、邵俱避難交州,太祖遣使就太守止燮盡族之。桓邵得出首,拜謝於庭中,太祖謂曰:「跪可解死邪!」遂殺之。甞出軍,行經麥中,令「士卒無敗麥,犯者死」。騎士皆下馬,付麥以相付,於是太祖馬騰入麥中,勑主簿議罪;主簿對以春秋之義,罰不加於尊。太祖曰:「制法而自犯之,何以帥下?然孤為軍帥,不可自殺,請自刑。」因援劔割髮以置地。又有幸姬常從晝寢,枕之卧,告之曰:「須臾覺我。」姬見太祖卧安,未即寤,及自覺,棒殺之。常討賊,廩穀不足,私謂主者曰:「如何?」主者曰:「可以小斛以足之。」太祖曰:「善。」後軍中言太祖欺衆,太祖謂主者曰:「特當借君死以猒衆,不然事不解。」乃斬之,取首題徇曰:「行小斛,盜官穀,斬之軍門。」其酷虐變詐,皆此之類也。}


評曰:漢末,天下大亂,雄豪並起,而袁紹虎眎四州,彊盛莫敵。太祖運籌演謀,鞭撻宇內,擥申、商之法術,該韓、白之奇策,官方授材,各因其器,矯情任筭,不念舊惡,終能總御皇機,克成洪業者,惟其明略最優也。抑可謂非常之人,超世之傑矣。


\end{pinyinscope}
