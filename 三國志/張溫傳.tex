\article{張溫傳}
\begin{pinyinscope}
 
 
 張溫字惠恕,吳郡吳人也。父允,以輕財重士,名顯州郡,為孫權東曹掾,卒。溫少脩節操,容貌奇偉。權聞之,以問公卿曰:「溫當今與誰為比?」大司農劉基曰:「可與全琮為輩。」太常顧雍曰:「基未詳其為人也。溫當今無輩。」權曰:「如是,張允不死也。」徵到延見,文辭占對,觀者傾竦,權改容加禮。罷出,張昭執其手曰:「老夫託意,君宜明之。」拜議郎、選曹尚書,徙太子太傅,甚見信重。
 
 
 
 
 時年三十二,以輔義中郎將使蜀。權謂溫曰:「卿不宜遠出,恐諸葛孔明不知吾所以與曹氏通意,以故屈卿行。若山越都除,便欲大搆於蜀。行人之義,受命不受辭也。」溫對曰:「臣入無腹心之規,出無專對之用,懼無張老延譽之功,又無子產陳事之效。然諸葛亮達見計數,必知神慮屈申之宜,加受朝廷天覆之惠,推亮之心,必無疑貳。」溫至蜀,詣闕拜章曰:「昔高宗以諒闇昌殷祚於再興,成王以幼沖隆周德於太平,功冒溥天,聲貫罔極。今陛下以聦明之姿,等契往古,總百揆於良佐,參列精之炳燿,遐邇望風,莫不欣賴。吳國勤任旅力,清澄江滸,願與有道平一宇內,委心協規,有如河水,軍事興煩,使役乏少,是以忍鄙倍之羞,使下臣溫通致情好。陛下敦崇禮義,未便恥忽。臣自入遠境,及即近郊,頻蒙勞來,恩詔輒加,以榮自懼,悚怛若驚。謹奉所齎函書一封。」蜀甚貴其才。還,頃之,使入豫章部伍出兵,事業未究。
 
 
 
 
 權旣陰銜溫稱美蜀政,又嫌其聲名大盛,衆庶炫惑,恐終不為己用,思有以中傷之,會曁豔事起,遂因此發舉。豔字子休,亦吳郡人也,溫引致之,以為選曹郎,至尚書。豔性狷厲,好為清議,見時郎署混濁淆雜,多非其人,欲臧否區別,賢愚異貫。彈射百僚,覈選三署,率皆貶高就下,降損數等,其守故者十未能一,其居位貪鄙,志節汙卑者,皆以為軍吏,置營府以處之。而怨憤之聲積,浸潤之譖行矣。競言豔及選曹郎徐彪,
 
 
\gezhu{吳錄曰:彪字仲虞,廣陵人也。}
 專用私情,愛憎不由公理,豔、彪皆坐自殺。溫宿與豔、彪同意,數交書疏,聞問往還,即罪溫。權幽之有司,下令曰:「昔令召張溫,虛己待之,旣至顯授,有過舊臣,何圖凶醜,專挾異心。昔曁豔父兄,附于惡逆,寡人無忌,故進而任之,欲觀豔何如。察其中間,形態果見。而溫與之結連死生,豔所進退,皆溫所為頭角,更相表裏,共為腹背,非溫之黨,即就疵瑕,為之生論。又前任溫董督三郡,指撝吏客及殘餘兵,時恐有事,欲令速歸,故授棨戟,獎以威柄。乃便到豫章,表討宿惡,寡人信受其言,特以繞帳、帳下、解煩兵五千人付之。後聞曹丕自出淮、泗,故豫勑溫有急便出,而溫悉內諸將,布於深山,被命不至。賴丕自退,不然,已往豈可深計。又殷禮者,本占候召,而溫先後乞將到蜀,扇揚異國,為之譚論。又禮之還,當親本職,而令守尚書戶曹郎,如此署置,在溫而已。又溫語賈原,當薦卿作御史,語蔣康,當用卿代賈原,專衒賈國恩,為己形勢。揆其姧心,無所不為。不忍暴於巿朝,今斥還本郡,以給厮吏。嗚呼溫也,免罪為幸!」
 
 
 
 
 將軍駱統表理溫曰:「伏惟殿下,天生明德,神啟聖心,招髦秀於四方,置俊乂於宮朝。多士旣受普篤之恩,張溫又蒙最隆之施。而溫自招罪譴,孤負榮遇,念其如此,誠可悲疚。然臣周旋之間,為國觀聽,深知其狀,故密陳其理。溫實心無他情,事無逆迹,但年紀尚少,鎮重尚淺,而戴赫烈之寵,體卓偉之才,亢臧否之譚,效襃貶之議。於是務勢者妬其寵,爭名者嫉其才,玄默者非其譚,瑕釁者諱其議,此臣下所當詳辨,明朝所當究察也。昔賈誼,至忠之臣也,漢文,大明之君也,然而絳、灌一言,賈誼遠退。何者?疾之者深,譖之者巧也。然而誤聞於天下,失彰於後世,故孔子曰『為君難,為臣不易』也。溫雖智非從橫,武非虓虎,然其弘雅之素,英秀之德,文章之采,論議之辯,卓躒冠羣,煒曄曜世,世人未有及之者也。故論溫才即可惜,言罪則可恕。若忍威烈以赦盛德,宥賢才以敦大業,固明朝之休光,四方之麗觀也。國家之於曁豔,不內之忌族,猶等之平民,是故先見用於朱治,次見舉於衆人,中見任於明朝,亦見交於義也。君臣之義,義之最重,朋友之交,交之最輕者也。國家不嫌於豔為最重之義,是以溫亦不嫌與豔為最輕之交也。時世寵之於上,溫竊親之於下也。夫宿惡之民,放逸山險,則為勁寇,將置平土,則為健兵,故溫念在欲取宿惡,以除勁寇之害,而增健兵之銳也。但自錯落,功不副言。然計其送兵,以比許晏,數之多少,溫不減之,用之彊羸,溫不下之,至於遲速,溫不後之,故得及秋冬之月,赴有警之期,不敢忘恩而遺力也。溫之到蜀,共譽殷禮,雖臣無境外之交,亦有可原也。境外之交,謂無君命而私相從,非國事而陰相聞者也;若以命行,旣脩君好,因叙己情,亦使臣之道也。故孔子使鄰國,則有私覿之禮;季子聘諸夏,亦有燕譚之義也。古人有言,欲知其君,觀其所使,見其下之明明,知其上之赫赫。溫若譽禮,能使彼歎之,誠所以昭我臣之多良,明使之得其人,顯國美於異境,揚君命於他邦。是以晉趙文子之盟于宋也,稱隨會於屈建;楚王孫圉之使于晉也,譽左史於趙鞅。亦向他國之輔,而歎本邦之臣,經傳美之以光國,而不譏之以外交也。王靖內不憂時,外不趨事,溫彈之不私,推之不假,於是與靖遂為大怨,此其盡節之明驗也。靖兵衆之勢,幹任之用,皆勝於賈原、蔣康,溫尚不容私以安於靖,豈敢賣恩以協原、康邪?又原在職不勤,當事不堪,溫數對以醜色,彈以急聲;若其誠欲賣恩作亂,則亦不必貪原也。凡此數者,校之於事旣不合,參之於衆亦不驗。臣竊念人君雖有聖哲之姿,非常之智,然以一人之身,御兆民之衆,從增宮之內,瞰四國之外,照羣下之情,求萬機之理,猶未易周也,固當聽察羣下之言,以廣聦明之烈。今者人非溫旣殷勤,臣是溫又契闊,辭則俱巧,意則俱至,各自言欲為國,誰其言欲為私,倉卒之間,猶難即別。然以殿下之聦叡,察講論之曲直,若潛神留思,纖粗研核,情何嫌而不宣,事何昧而不昭哉?溫非親臣,臣非愛溫者也。昔之君子,皆抑私忿,以增君明。彼獨行之於前,臣恥廢之於後,故遂發宿懷於今日,納愚言於聖聽,實盡心於明朝,非有念於溫身也。」權終不納。
 
 
後六年,溫病卒。二弟祗、白,亦有才名,與溫俱廢。
 \gezhu{會稽典錄曰:餘姚虞俊歎曰:「張惠恕才多智少,華而不實,怨之所聚,有覆家之禍,吾見其兆矣。」諸葛亮聞俊憂溫,意未之信,及溫放黜,亮乃歎俊之有先見。亮初聞溫敗,未知其故,思之數日,曰:「吾已得之矣,其人於清濁太明,善惡太分。」臣松之以為莊周云「名者公器也,不可以多取」,張溫之廢,豈其取名之多乎!多之為弊,古賢旣知之矣。是以遠見之士,退藏於密,不使名浮於德,不以華傷其實,旣不能被褐韞寶,杜廉逃譽,使才映一世,聲蓋人上,沖用之道,庸可暫替!溫則反之,能無敗乎?權旣疾溫名盛,而駱統方驟言其美,至云「卓躒冠羣,煒曄曜世,世人未有及之者也」。斯何異燎之方盛,又撝膏以熾之哉!文士傳曰:溫姊妹三人皆有節行,為溫事,已嫁者皆見錄奪。其中妹先適顧承,官以許嫁丁氏,成婚有日,遂飲藥而死。吳朝嘉歎,鄉人圖畫,為之贊頌云。}
 
 
\end{pinyinscope}