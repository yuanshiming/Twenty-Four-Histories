\article{周羣傳}
\begin{pinyinscope}
 
 
 周羣字仲直,巴西閬中人也。父舒,字叔布,少學術於廣漢楊厚,名亞董扶、任安。數被徵,終不詣。時人有問:「春秋讖曰代漢者當塗高,此何謂也?」舒曰:「當塗高者,魏也。」鄉黨學者私傳其語。羣少受學於舒,專心候業。於庭中作小樓,家富多奴,常令奴更直於樓上視天灾,纔見一氣,即白羣,羣自上樓觀之,不避晨夜。故凡有氣候,無不見之,是以所言多中。州牧劉璋,辟以為師友從事。
 
 
\gezhu{續漢書曰:建安七年,越嶲有男子化為女人,時羣言哀帝時亦有此,將易代之祥也。至二十五年,獻帝果封于山陽。十二年十月,有星孛于鶉尾,荊州分野,羣以為荊州牧將死而失土。明年秋,劉表卒,曹公平荊州。十七年十二月,星孛于五諸侯,羣以為西方專據土地者皆將失土。是時,劉璋據益州,張魯據漢中,韓遂據涼州,宋建據枹罕。明年冬,曹公遣偏將擊涼州。十九年,獲宋建,韓遂逃于羌中,被殺。其年秋,璋失益州。二十年秋,曹公攻漢中,張魯降。}
 
 
先主定蜀,署儒林校尉。先主欲與曹公爭漢中,問羣,羣對曰:「當得其地,不得其民也。若出偏軍,必不利,當戒慎之!」時州後部司馬蜀郡張裕亦曉占候,
 \gezhu{裕字南和。}
 而天才過羣,諫先主曰:「不可爭漢中,軍必不利。」先主竟不用裕言,果得地而不得民也。遣將軍吳蘭、雷銅等入武都,皆沒不還,悉如羣言。於是舉羣茂才。
 
 
 
 
 裕又私語人曰:「歲在庚子,天下當易代,劉氏祚盡矣。主公得益州,九年之後,寅卯之間當失之。」人密白其言。初,先主與劉璋會涪時,裕為璋從事,侍坐。其人饒鬚,先主嘲之曰:「昔吾居涿縣,特多毛姓,東西南北皆諸毛也,涿令稱曰『諸毛繞涿居乎』!」裕即荅曰:「昔有作上黨潞長,遷為涿令。涿令者,去官還家,時人與書,欲署潞則失涿,欲署涿則失潞,乃署曰『潞涿君』。」先主無鬚,故裕以此及之。先主常銜其不遜,加忿其漏言,乃顯裕諫爭漢中不驗,下獄,將誅之。諸葛亮表請其罪,先主荅曰:「芳蘭生門,不得不鉏。」裕遂弃市。後魏氏之立,先主之薨,皆如裕所刻。又曉相術,每舉鏡視面,自知刑死,未甞不撲之于地也。
 
 
 
 
 羣卒,子巨頗傳其術。
 
 
\end{pinyinscope}