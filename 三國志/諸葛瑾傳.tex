\article{諸葛瑾傳}
\begin{pinyinscope}
 
 
 諸葛瑾字子瑜,琅邪陽都人也。
 
 
\gezhu{吳書曰:其先葛氏,本琅邪諸縣人,後徙陽都。陽都先有姓葛者,時人謂之諸葛,因以為氏。瑾少游京師,治毛詩、尚書、左氏春秋。遭母憂,居喪至孝,事繼母恭謹,甚得人子之道。風俗通曰:葛嬰為陳涉將軍,有功而誅,孝文帝追錄,封其孫諸縣侯,因并氏焉。此與吳書所說不同。}
 漢末避亂江東。值孫策卒,孫權姊壻曲阿弘咨見而異之,薦之於權,與魯肅等並見賔待,後為權長史,轉中司馬。建安二十年,權遣瑾使蜀通好劉備,與其弟亮俱公會相見,退無私面。
 
 
 
 
 與權談說諫喻,未甞切愕,微見風彩,粗陳指歸,如有未合,則捨而及他,徐復託事造端,以物類相求,於是權意往往而釋。吳郡太守朱治,權舉將也,權曾有以望之,而素加敬,難自詰讓,忿忿不解。瑾揣知其故,而不敢顯陳,乃乞以意私自問,遂於權前為書,泛論物理,因以己心遙往忖度之。畢,以呈權,權喜,笑曰:「孤意解矣。顏氏之德,使人加親,豈謂此邪?」權又怪校尉殷模,罪至不測。羣下多為之言,權怒益甚,與相反覆,惟瑾默然,權曰:「子瑜何獨不言?」瑾避席曰:「瑾與殷模等遭本州傾覆,生類殄盡。棄墳墓,攜老弱,披草萊,歸聖化,在流隷之中,蒙生成之福,不能躬相督厲,陳荅萬一,至令模孤負恩惠,自陷罪戾。臣謝過不暇,誠不敢有言。」權聞之愴然,乃曰:「特為君赦之。」
 
 
後從討關羽,封宣城侯,以綏南將軍代呂蒙領南郡太守,住公安。劉備東伐吳,吳王求和,瑾與備牋曰:「奄聞旗鼓來至白帝,或恐議臣以吳王侵取此州,危害關羽,怨深禍大,不宜荅和,此用心於小,未留意於大者也。試為陛下論其輕重,及其大小。陛下若抑威損忿,蹔省瑾言者,計可立決,不復咨之於羣后也。陛下以關羽之親何如先帝?荊州大小孰與海內?俱應仇疾,誰當先後?若審此數,易於反掌。」
 \gezhu{臣松之云:以為劉后以庸蜀為關河,荊楚為維翰,關羽揚兵沔、漢,志陵上國,雖匡主定霸,功未可必,要為威聲遠震,有其經略。孫權潛包禍心,助魏除害,是為翦宗子勤王之師,行曹公移都之計,拯漢之規,於茲而止。義旗所指,宜其在孫氏矣。瑾以大義責備,荅之何患無辭;且備、羽相與,有若四體,股肱橫虧,憤痛已深,豈此奢闊之書所能迴駐哉!載之於篇,寔為辭章之費。}
 時或言瑾別遣親人與備相聞,權曰:「孤與子瑜有死生不易之誓,子瑜之不負孤,猶孤之不負子瑜也。」
 \gezhu{江表傳曰:瑾之在南郡,人有密讒瑾者。此語頗流聞於外,陸遜表保明瑾無此,宜以散其意。權報曰:「子瑜與孤從事積年,恩如骨肉,深相明究,其為人非道不行,非義不言。玄德昔遣孔明至吳,孤甞語子瑜曰:『卿與孔明同產,且弟隨兄,於義為順,何以不留孔明?孔明若留從卿者,孤當以書解玄德,意自隨人耳。』子瑜荅孤言:『弟亮以生身於人,委質定分,義無二心。弟之不留,猶瑾之不往也。』其言足貫神明。今豈當有此乎?孤前得妄語文疏,即封示子瑜,并手筆與子瑜,即得其報,論天下君臣大節,一定之分。孤與子瑜,可謂神交,非外言所間也。知卿意至,輒封來表,以示子瑜,使知卿意。」}
 黃武元年,遷左將軍,督公安,假節,封宛陵侯。
 \gezhu{吳錄曰:曹真、夏侯尚等圍朱然於江陵,又分據中州,瑾以大兵為之救援。瑾性弘緩,推道理,任計畫,無應卒倚伏之術,兵乆不解,權以此望之。及春水生,潘璋等作水城於上流,瑾進攻浮橋,真等退走。雖無大勳,亦以全師保境為功。}
 
 
 
 
 虞翻以狂直流徙,惟瑾屢為之說。翻與所親書曰:「諸葛敦仁,則天活物,比蒙清論,有以保分。惡積罪深,見忌殷重,雖有祁老之救,德無羊舌,解釋難兾也。」
 
 
瑾為人有容貌思度,于時服其弘雅。權亦重之,大事咨訪。又別咨瑾曰:「近得伯言表,以為曹丕已死,毒亂之民,當望旌瓦解,而更靜然。聞皆選用忠良,寬刑罰,布恩惠,薄賦省役,以恱民心,其患更深於操時。孤以為不然。操之所行,其惟殺伐小為過差,及離閒人骨肉,以為酷耳。至於將御,自古少有。比之於操,萬不及也。今叡之不如丕,猶丕不如操也。其所以務崇小惠,必以其父新死,自度衰微,恐困苦之民一朝崩沮,故彊屈曲以求民心,欲以自安住耳,寧是興隆之漸邪!聞任陳長文、曹子丹輩,或文人諸生,或宗室戚臣,寧能御雄才虎將以制天下乎?夫威柄不專,則其事乖錯,如昔張耳、陳餘,非不敦睦,至於秉勢,自還相賊,乃事理使然也。又長文之徒,昔所以能守善者,以操笮其頭,畏操威嚴,故竭心盡意,不敢為非耳。逮丕繼業,年已長大,承操之後,以恩情加之,用能感義。今叡幼弱,隨人東西,此曹等輩,必當因此弄巧行態,阿黨比周,各助所附。如此之日,姧讒並起,更相陷懟,轉成嫌貳。一爾已往,羣下爭利,主幼不御,其為敗也焉得乆乎?所以知其然者,自古至今,安有四五人把持刑柄,而不離刺轉相蹄齧者也!彊當陵弱,弱當求援,此亂亡之道也。子瑜,卿但側耳聽之,伯言常長於計校,恐此一事小短也。」
 \gezhu{臣松之以為魏明帝一時明主,政自己出,孫權此論,竟為無徵,而史載之者,將以主幼國疑,威柄不一,亂亡之形,有如權言,宜其存錄以為鑒戒。或當以雖失之於明帝,而事著於齊王,齊王之世,可不謂驗乎!不敢顯斥,抑足表之微辭。}
 
 
權稱尊號,拜大將軍、左都護,領豫州牧。及呂壹誅,權又有詔切磋瑾等,語在權傳。瑾輒因事以荅,辭順理正。瑾子恪,名盛當世,權深器異之;然瑾常嫌之,謂非保家之子,每以憂戚。
 \gezhu{吳書曰:初,瑾為大將軍,而弟亮為蜀丞相,二子恪、融皆典戎馬,督領將帥,族弟誕又顯名於魏,一門三方為冠蓋,天下榮之。謹才略雖不及弟,而德行尤純。妻死不改娶,有所愛妾,生子不舉,其篤慎皆如此。}
 赤烏四年,年六十八卒,遺命令素棺歛以時服,事從省約。恪已自封侯,故弟融襲爵,攝兵業駐公安。
 \gezhu{吳書曰:融字叔長,生於寵貴,少而驕樂,學為章句,博而不精,性寬容,多技藝,數以巾褐奉朝請,後拜騎都尉。赤烏中,諸郡出部伍,新都都尉陳表、吳郡都尉顧承各率所領人會佃毗陵,男女各數萬口。表病死,權以融代表,後代父瑾領攝。融部曲吏士親附之,疆外無事。}
 秋冬則射獵講武,春夏則延賔高會,休吏假卒,或不遠千里而造焉。每會輒歷問賔客,各言其能,乃合榻促席,量敵選對,或有博奕,或有摴蒱,投壺弓彈,部別類分,於是甘果繼進,清酒徐行,融周流觀覽,終日不倦。融父兄質素,雖在軍旅,身無采飾;而融錦罽文繡,獨為奢綺。孫權薨,徙奮威將軍。後恪征淮南,假融節,令引軍入沔,以擊西兵。恪旣誅,遣無難督施寬就將軍施績、孫壹、全熈等取融。融卒聞兵士至,惶懼猶豫,不能決計,兵到圍城,飲藥而死,三子皆伏誅。
 \gezhu{江表傳曰:先是,公安有靈鼉鳴,童謠曰:「白鼉鳴,龜背平,南郡城中可長生,守死不去義無成。」及恪被誅,融果刮金印龜,服之而死。}
 
 
\end{pinyinscope}