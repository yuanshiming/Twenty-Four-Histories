\article{周瑜傳}
\begin{pinyinscope}
 
 
 周瑜字公瑾,廬江舒人也。從祖父景,景子忠,皆為漢太尉。
 
 
\gezhu{謝承後漢書曰:景字仲嚮,少以廉能見稱,以明學察孝廉,辟公府。後為豫州刺史,辟汝南陳蕃為別駕,潁川李膺、荀緄、杜密、沛國朱㝢為從事,皆天下英俊之士也。稍遷至尚書令,遂登太尉。張璠漢紀曰:景父榮,章、和世為尚書令。初景歷位牧守,好善愛士,每歲舉孝廉,延請入,上後堂,與家人宴會,如此者數四。及贈送旣備,又選用其子弟,常稱曰:「移臣作子,於之何有?」先是,司徒韓縯為河內太守,在公無私,所舉一辭而已,後亦不及其門戶,曰:「我舉若可矣,不令恩偏稱一家也。」當時論者,或兩譏焉。}
 父異,洛陽令。
 
 
瑜長壯有姿貌。初,孫堅興義兵討董卓,徙家於舒。堅子策與瑜同年,獨相友善,瑜推道南大宅以舍策,升堂拜母,有無通共。瑜從父尚為丹楊太守,瑜往省之。會策將東渡,到歷陽,馳書報瑜,瑜將兵迎策。策大喜曰:「吾得卿,諧也。」遂從攻橫江、當利,皆拔之。乃渡擊秣陵,破笮融、薛禮,轉下湖孰、江乘,進入曲阿,劉繇奔走,而策之衆已數萬矣。因謂瑜曰:「吾以此衆取吳會平山越已足。卿還鎮丹楊。」瑜還。頃之,袁術遣從弟胤代尚為太守,而瑜與尚俱還壽春。術欲以瑜為將,瑜觀術終無所成,故求為居巢長,欲假塗東歸,術聽之。遂自居巢還吳。是歲,建安三年也。策親自迎瑜,授建威中郎將,即與兵二千人,騎五十匹。
 \gezhu{江表傳曰:策又給瑜鼓吹,為治館舍,贈賜莫與為比。策令曰:「周公瑾英雋異才,與孤有總角之好,骨肉之分。如前在丹楊,發衆及船糧以濟大事,論德酬功,此未足以報者也。」}
 瑜時年二十四,吳中皆呼為周郎。以瑜恩信著於廬江,出備牛渚,後領春穀長。頃之,策欲取荊州,以瑜為中護軍,領江夏太守,從攻皖,拔之。時得橋公兩女,皆國色也。策自納大橋,瑜納小橋。
 \gezhu{江表傳曰:策從容戲瑜曰:「橋公二女雖流離,得吾二人作壻,亦足為歡。」}
 復進尋陽,破劉勳,討江夏,還定豫章、廬陵,留鎮巴丘。
 \gezhu{臣松之案:孫策于時始得豫章、廬陵,尚未能得定江夏。瑜之所鎮,應在今巴丘縣也,與後所平巴丘處不同。}
 
 
五年,策薨,權統事。瑜將兵赴喪,遂留吳,以中護軍與長史張昭共掌衆事。
 \gezhu{江表傳曰:曹公新破袁紹,兵威日盛,建安七年,下書責權質任子。權召羣臣會議,張昭、秦松等猶豫不能決,權意不欲遣質,乃獨將瑜詣母前定議,瑜曰:「昔楚國初封於荊山之側,不滿百里之地,繼嗣賢能,廣土開境,立基於郢,遂據荊揚,至於南海,傳業延祚,九百餘年。今將軍承父兄餘資,六郡之衆,兵精糧多,將士用命,鑄山為銅,煑海為鹽,境內富饒,人不思亂,汎舟舉帆,朝發夕到,士風勁勇,所向無敵,有何偪迫,而欲送質?質一入,不得不與曹氏相首尾,與相首尾,則命召不得不往,便見制於人也。極不過一侯印,僕從十餘人,車數乘,馬數匹,豈與南靣稱孤同哉?不如勿遣,徐觀其變。若曹氏能率義以正天下,將軍事之未晚。若圖為暴,亂兵猶火也,不戢將自焚。將軍韜勇抗威,以待天命,何送質之有!」權母曰:「公瑾議是也。公瑾與伯符同年,小一月耳,我視之如子也,汝其兄事之。」遂不送質。}
 十一年,督孫瑜等討麻、保二屯,梟其渠帥,囚俘萬餘口,還備官亭。江夏太守黃祖遣將鄧龍將兵數千人入柴桑,瑜追討擊,生虜龍送吳。
 
 
十三年春,權討江夏,瑜為前部大督。其年九月,曹公入荊州,劉琮舉衆降,曹公得其水軍,船步兵數十萬,將士聞之皆恐。權延見羣下,問以計策。議者咸曰:「曹公豺虎也,然託名漢相,挾天子以征四方,動以朝廷為辭,今日拒之,事更不順。且將軍大勢,可以拒操者,長江也。今操得荊州,掩有其地,劉表治水軍,蒙衝鬬艦,乃以千數,操悉浮以沿江,兼有步兵,水陸俱下,此為長江之險,已與我共之矣。而勢力衆寡,又不可論。愚謂大計不如迎之。」瑜曰:「不然。操雖託名漢相,其實漢賊也。將軍以神武雄才,兼仗父兄之烈,割據江東,地方數千里,兵精足用,英雄樂業,尚當橫行天下,為漢家除殘去穢。況操自送死,而可迎之邪?請為將軍籌之:今使北土已安,操無內憂,能曠日持乆來爭疆場,又能與我校勝負於船楫可乎?今北土旣未平安,加馬超、韓遂尚在關西,為操後患。且舍鞌馬,杖舟楫,與吳越爭衡,本非中國所長。又今盛寒,馬無槀草,驅中國士衆遠涉江湖之閒,不習水土,必生疾病。此數四者,用兵之患也,而操皆冒行之。將軍禽操,宜在今日。瑜請得精兵三萬人,進住夏口,保為將軍破之。」權曰:「老賊欲廢漢自立乆矣,徒忌二袁、呂布、劉表與孤耳。今數雄已滅,惟孤尚存,孤與老賊,勢不兩立。君言當擊,甚與孤合,此天以君授孤也。」
 \gezhu{江表傳曰:權拔刀斫前奏案曰:「諸將吏敢復有言當迎操者,與此案同!」及會罷之夜,瑜請見曰:「諸人徒見操書,言水步八十萬,而各恐懾,不復料其虛實,便開此議,甚無謂也。今以實校之,彼所將中國人,不過十五六萬,且軍已久疲,所得表衆,亦極七八萬耳,尚懷狐疑。夫以疲病之卒,御狐疑之衆,衆數雖多,甚未足畏。得精兵五萬,自足制之,願將軍勿慮。」權撫背曰:「公瑾,卿言至此,甚合孤心。子布、元表諸人,各顧妻子,挾持私慮,深失所望,獨卿與子敬與孤同耳,此天以卿二人贊孤也。五萬兵難卒合,已選三萬人,船糧戰具俱辦,卿與子敬、程公便在前發,孤當續發人衆,多載資糧,為卿後援。卿能辦之者誠決,邂逅不如意,便還就孤,孤當與孟德決之。」臣松之以為建計拒曹公,實始魯肅。于時周瑜使鄱陽,肅勸權呼瑜,瑜使鄱陽還,但與肅闇同,故能共成大勳。本傳直云,權延見羣下,問以計策,瑜擺衆人之議,獨言抗拒之計,了不云肅先有謀,殆為攘肅之善也。}
 
 
時劉備為曹公所破,欲引南渡江,與魯肅遇於當陽,遂共圖計,因進住夏口,遣諸葛亮詣權,權遂遣瑜及程普等與備并力逆曹公,遇於赤壁。時曹公軍衆已有疾病,初一交戰,公軍敗退,引次江北。瑜等在南岸。瑜部將黃蓋曰:「今寇衆我寡,難與持乆。然觀操軍船艦首尾相接,可燒而走也。」乃取蒙衝鬬艦數十艘,實以薪草,膏油灌其中,裹以帷幕,上建牙旗,先書報曹公,欺以欲降。
 \gezhu{江表傳載蓋書曰:「蓋受孫氏厚恩,常為將帥,見遇不薄。然顧天下事有大勢,用江東六郡山越之人,以當中國百萬之衆,衆寡不敵,海內所共見也。東方將吏,無有愚智,皆知其不可,惟周瑜、魯肅偏懷淺戇,意未解耳。今日歸命,是其實計。瑜所督領,自易摧破。交鋒之日,蓋為前部,當因事變化,效命在近。」曹公特見行人,密問之,口勑曰:「但恐汝詐耳。蓋若信實,當授爵賞,超於前後也。」}
 又豫備走舸,各繫大船後,因引次俱前。曹公軍吏士皆延頸觀望,指言蓋降。蓋放諸船,同時發火。時風盛猛,悉延燒岸上營落。頃之,煙炎張天,人馬燒溺死者甚衆,軍遂敗退,還保南郡。
 \gezhu{江表傳曰:至戰日,蓋先取輕利艦十舫,載燥荻枯柴積其中,灌以魚膏,赤幔覆之,建旌旗龍幡於艦上。時東南風急,因以十艦最著前,中江舉帆,蓋舉火白諸校,使衆兵齊聲大叫曰:「降焉!」操軍人皆出營立觀。去北軍二里餘,同時發火,火烈風猛,往船如箭,飛埃絕爛,燒盡北船,延及岸邊營柴。瑜等率輕銳尋繼其後,雷鼓大進,北軍大壞,曹公退走。}
 備與瑜等復共追。曹公留曹仁等守江陵城,徑自北歸。
 
 
瑜與程普又進南郡,與仁相對,各隔大江。兵未交鋒,
 \gezhu{吳錄曰:備謂瑜云:「仁守江陵城,城中糧多,足為疾害。使張益德將千人隨卿,卿分二千人追我,相為從夏水入截仁後,仁聞吾入必走。」瑜以二千人益之。}
 瑜即遣甘寧前據夷陵。仁分兵騎別攻圍寧。寧告急於瑜。瑜用呂蒙計,留凌統以守其後,身與蒙上救寧。寧圍旣解,乃渡屯北岸,克期大戰。瑜親跨馬擽陣,會流矢中右脅,瘡甚,便還。後仁聞瑜卧未起,勒兵就陣。瑜乃自興,案行軍營,激揚吏士,仁由是遂退。
 
 
 
 
 權拜瑜偏將軍,領南郡太守。以下雋、漢昌、瀏陽、州陵為奉邑,屯據江陵。劉備以左將軍領荊州牧,治公安。備詣京見權,瑜上疏曰:「劉備以梟雄之姿,而有關羽、張飛熊虎之將,必非乆屈為人用者。愚謂大計宜徙備置吳,盛為築宮室,多其美女玩好,以娛其耳目,分此二人,各置一方,使如瑜者得挾與攻戰,大事可定也。今猥割土地以資業之,聚此三人,俱在疆場,恐蛟龍得雲雨,終非池中物也。」權以曹公在北方,當廣擥英雄,又恐備難卒制,故不納。
 
 
是時劉璋為益州牧,外有張魯寇侵,瑜乃詣京見權曰:「今曹操新折衂,方憂在腹心,未能與將軍道兵相事也。乞與奮威俱進取蜀,得蜀而并張魯,因留奮威固守其地,好與馬超結援。瑜還與將軍據襄陽以蹙操,北方可圖也。」權許之。瑜還江陵,為行裝,而道於巴丘病卒,
 \gezhu{臣松之案,瑜欲取蜀,還江陵治嚴,所卒之處,應在今之巴陵,與前所鎮巴丘,名同處異也。}
 時年三十六。權素服舉哀,感慟左右。喪當還吳,又迎之蕪湖,衆事費度,一為供給。後著令曰:「故將軍周瑜、程普,其有人客,皆不得問。」初瑜見友於策,太妃又使權以兄奉之。是時權位為將軍,諸將賔客為禮尚簡,而瑜獨先盡敬,便執臣節。性度恢廓,大率為得人,惟與程普不睦。
 \gezhu{江表傳曰:普頗以年長數陵侮瑜。瑜折節容下,終不與校。普後自敬服而親重之,乃告人曰:「與周公瑾交,若飲醇醪,不覺自醉。」時人以其謙讓服人如此。初曹公聞瑜年少有美才,謂可游說動也,乃密下揚州,遣九江蔣幹往見瑜。幹有儀容,以才辯見稱,獨步江、淮之閒,莫與為對。乃布衣葛巾,自託私行詣瑜。瑜出迎之,立謂幹曰:「子翼良苦,遠涉江湖為曹氏作說客邪?」幹曰:「吾與足下州里,中閒別隔,遙聞芳烈,故來叙闊,并觀雅規,而云說客,無乃逆詐乎?」瑜曰:「吾雖不及夔、曠,聞弦賞音,足知雅曲也。」因延幹入,為設酒食。畢,遣之曰:「適吾有密事,且出就館,事了,別自相請。」後三日,瑜請幹與周觀營中,行視倉庫軍資器仗訖,還宴飲,示之侍者服飾珍玩之物,因謂幹曰:「丈夫處世,遇知己之主,外託君臣之義,內結骨肉之恩,言行計從,禍福共之,假使蘇張更生,酈叟復出,猶撫其背而折其辭,豈足下幼生所能移乎?」幹但笑,終無所言。幹還,稱瑜雅量高致,非言辭所間。中州之士,亦以此多之。劉備之自京還也,權乘飛雲大船,與張昭、秦松、魯肅等十餘人共追送之,大宴會敘別。昭、肅等先出,權獨與備留語,因言次,歎瑜曰:「公瑾文武籌略,萬人之英,顧其器量廣大,恐不乆為人臣耳。」瑜之破魏軍也,曹公曰:「孤不羞走。」後書與權曰:「赤壁之役,值有疾病,孤燒船自退,橫使周瑜虛獲此名。」瑜威聲遠著,故曹公、劉備咸欲疑譖之。及卒,權流涕曰:「公瑾有王佐之資,今忽短命,孤何賴哉!」後權稱尊號,謂公卿曰:「孤非周公瑾,不帝矣。」}
 
 
 
 
 瑜少精意於音樂,雖三爵之後,其有闕誤,瑜必知之,知之必顧,故時人謠曰:「曲有誤,周郎顧。」
 
 
 
 
 瑜兩男一女。女配太子登。男循尚公主,拜騎都尉,有瑜風,早卒。循弟胤,初拜興業都尉,妻以宗女,授兵千人,屯公安。黃龍元年,封都鄉侯,後以罪徙廬陵郡。赤烏二年,諸葛瑾、步隲連名上疏曰:「故將軍周瑜子胤,昔蒙粉飾,受封為將,不能養之以福,思立功效,至縱情欲,招速罪辟。臣竊以瑜昔見寵任,入作心膂,出為爪牙,銜命出征,身當矢石,盡節用命,視死如歸,故能摧曹操於烏林,走曹仁於郢都,揚國威德,華夏是震,蠢爾蠻荊,莫不賔服,雖周之方叔,漢之信、布,誠無以尚也。夫折衝扞難之臣,自古帝王莫不貴重,故漢高帝封爵之誓曰『使黃河如帶,太山如礪,國以永存,爰及苗裔』;申以丹書,重以盟詛,藏于宗廟,傳於無窮,欲使功臣之後,世世相踵,非徒子孫,乃關苗裔,報德明功,勤勤懇懇,如此之至,欲以勸戒後人,用命之臣,死而無悔也。況於瑜身沒未乆,而其子胤降為匹夫,益可悼傷。竊惟陛下欽明稽古,隆於興繼,為胤歸訴,乞匄餘罪,還兵復爵,使失旦之雞,復得一鳴,抱罪之臣,展其後效。」權荅曰:「腹心舊勳,與孤協事,公瑾有之,誠所不忘。昔胤年少,初無功勞,橫受精兵,爵以侯將,蓋念公瑾以及於胤也。而胤恃此,酗淫自恣,前後告喻,曾無悛改。孤於公瑾,義猶二君,樂胤成就,豈有已哉?迫胤罪惡,未宜便還,且欲苦之,使自知耳。今二君勤勤援引漢高河山之誓,孤用恧然。雖德非其疇,猶欲庶幾,事亦如爾,故未順旨。以公瑾之子,而二君在中間,苟使能改,亦何患乎!」瑾、隲表比上,朱然及全琮亦俱陳乞,權乃許之。會胤病死。
 
 
 
 
 瑜兄子峻,亦以瑜元功為偏將軍,領吏士千人。峻卒,全琮表峻子護為將。權曰:「昔走曹操,拓有荊州,皆是公瑾,常不忘之。初聞峻亡,仍欲用護,聞護性行危險,用之適為作禍,故便止之。孤念公瑾,豈有已乎?」
 
 
\end{pinyinscope}