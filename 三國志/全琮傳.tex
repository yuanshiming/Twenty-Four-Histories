\article{全琮傳}
\begin{pinyinscope}
 
 
 全琮字子璜,吳郡錢唐人也。父柔,漢靈帝時舉孝廉,補尚書郎右丞,董卓之亂,棄官歸,州辟別駕從事,詔書就拜會稽東部都尉。孫策到吳,柔舉兵先附,策表柔為丹楊都尉。孫權為車騎將軍,以柔為長史,徙桂陽太守。柔甞使琮齎米數千斛到吳,有所市易。琮至,皆散用,空船而還。柔大怒,琮頓首曰:「愚以所市非急,而士大夫方有倒縣之患,故便振贍,不及啟報。」柔更以奇之。
 
 
\gezhu{徐衆評曰:禮,子事父無私財,又不敢私施,所以避尊上也。棄命專財而以邀名,未盡父子之禮。臣松之以為子路問「聞斯行諸」?子曰「有父兄在」。琮輒散父財,誠非子道,然士類縣命,憂在朝夕,權其輕重,以先人急,斯亦馮煖市義、汲黯振救之類,全謂邀名,或負其心。}
 是時中州士人避亂而南,依琮居者以百數,琮傾家給濟,與共有無,遂顯名遠近。後權以為奮威校尉,授兵數千人,使討山越。因開募召,得精兵萬餘人,出屯牛渚,稍遷偏將軍。
 
 
 
 
 建安二十四年,劉備將關羽圍樊、襄陽,琮上疏陳羽可討之計,權時已與呂蒙陰議襲之,恐事泄,故寢琮表不荅。及禽羽,權置酒公安,顧謂琮曰:「君前陳此,孤雖不相荅,今日之捷,抑亦君之功也。」於是封陽華亭侯。
 
 
 
 
 黃初元年,魏以舟軍大出洞口,權使呂範督諸將拒之,軍營相望。敵數以輕船鈔擊,琮常帶甲仗兵,伺候不休。頃之,敵數千人出江中,琮擊破之,梟其將軍尹盧。遷琮綏南將軍,進封錢唐侯。四年,假節領九江太守。
 
 
七年,權到皖,使琮與輔國將軍陸遜擊曹休,破之於石亭。是時丹楊、吳、會山民復為寇賊,攻沒屬縣,權分三郡險地為東安郡,琮領太守。
 \gezhu{吳錄曰:琮時治富春。}
 至,明賞罰,招誘降附,數年中,得萬餘人。權召琮還牛渚,罷東安郡。
 \gezhu{江表傳曰:琮還,經過錢唐,脩祭墳墓,麾幢節蓋,曜於舊里,請會邑人平生知舊、宗族六親,施散惠與,千有餘萬,本土以為榮。}
 黃龍元年,遷衞將軍、左護軍、徐州牧,
 \gezhu{吳書曰:初,琮為將甚勇決,當敵臨難,奮不顧身。及作督帥,養威持重,每御軍,常任計策,不營小利。江表傳曰:權使子登出征,已出軍,次于安樂,羣臣莫敢諫。琮密表曰:「古來太子未甞偏征也,故從曰撫軍,守曰監國。今太子東出,非古制也,臣竊憂疑。」權即從之,命登旋軍,議者咸以為琮有大臣之節也。}
 尚公主。
 
 
 
 
 嘉禾二年,督步騎五萬征六安,六安民皆散走,諸將欲分兵捕之。琮曰:「夫乘危徼倖,舉不百全者,非國家大體也。今分兵捕民,得失相半,豈可謂全哉?縱有所獲,猶不足以弱敵而副國望也。如或邂逅,虧損非小,與其獲罪,琮寧以身受之,不敢徼功以負國也。」
 
 
 
 
 赤烏九年,遷右大司馬、左軍師。為人恭順,善於承顏納規,言辭未甞切迕。初,權將圍珠崖及夷州,皆先問琮,琮曰:「以聖朝之威,何向而不克?然殊方異域,隔絕障海,水土氣毒,自古有之,兵入民出,必生疾病,轉相污染,往者懼不能反,所獲何可多致?猥虧江岸之兵,以兾萬一之利,愚臣猶所不安。」權不聽。軍行經歲,士衆疾疫死者十有八九,權深悔之。後言次及之,琮對曰:「當是時,羣臣有不諫者,臣以為不忠。」
 
 
琮旣親重,宗族子弟並蒙寵貴,賜累千金,然猶謙虛接士,貌無驕色。十二年卒,子懌嗣。後襲業領兵,救諸葛誕於壽春,出城先降,魏以為平東將軍,封臨湘侯。懌兄子禕、儀、靜等亦降魏,皆歷郡守列侯。
 \gezhu{吳書曰:琮長子緒,幼知名,奉朝請,出授兵,稍遷揚武將軍、牛渚督。孫亮即位,遷鎮北將軍。東關之役,緒與丁奉建議引兵先出,以破魏軍,封一子亭侯,年四十四卒。次子寄,坐阿黨魯王霸賜死。小子吳,孫權外孫,封都鄉侯。}
 
 
\end{pinyinscope}