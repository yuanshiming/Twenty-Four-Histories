\article{劉表傳}
\begin{pinyinscope}
 
 
 劉表字景升,山陽高平人也。少知名,號八俊。
 
 
\gezhu{張璠漢紀曰:表與同郡人張隱、薛郁、王訪、宣靖、公緒恭、劉祇、田林為八交,或謂之八顧。漢末名士錄云:表與汝南陳翔字仲麟、范滂字孟愽、魯國孔昱字世元、勃海苑康字仲真、山陽檀敷字文友、張儉字元節、南陽岑晊字公孝為八友。謝承漢書曰:表受學於同郡王暢。暢為南陽太守,行過乎儉。表時年十七,進諫曰:「奢不僭上,儉不逼下,蓋中庸之道,是故蘧伯玉恥獨為君子。府君若不師孔聖之明訓,而慕夷齊之末操,無乃皎然自遺於世!」暢荅曰:「以約失之者鮮矣。且以矯俗也。」}
 長八尺餘,姿皃甚偉。以大將軍掾為北軍中候。靈帝崩,代王叡為荊州刺史。是時山東兵起,表亦合兵軍襄陽。
 \gezhu{司馬彪戰略曰:劉表之初為荊州也,江南宗賊盛,袁術屯魯陽,盡有南陽之衆。吳人蘇代領長沙太守,具羽為華容長,各阻兵作亂。表初到,單馬入宜城,而延中廬人蒯良、蒯越、襄陽人蔡瑁與謀。表曰:「宗賊甚盛,而衆不附,袁術因之,禍今至矣!吾欲徵兵,恐不集,其策安出?」良曰:「衆不附者,仁不足也,附而不治者,義不足也;苟仁義之道行,百姓歸之如水之趣下,何患所至之不從而問興兵與策乎?」表顧問越,越曰:「治平者先仁義,治亂者先權謀。兵不在多,在得人也。袁術勇而無斷,蘇代、具羽皆武人,不足慮。宗賊帥多貪暴,為下所患。越有所素養者,使示之以利,必以衆來。君誅其無道,撫而用之。一州之人,有樂存之心,聞君盛德,必襁負而至矣。兵集衆附,南據江陵,北守襄陽,荊州八郡可傳檄而定。術等雖至,無能為也。」表曰:「子柔之言,雍季之論也。異度之計,臼犯之謀也。」遂使越遣人誘宗賊,至者五十五人,皆斬之。襲取其衆,或即授部曲。唯江夏賊張虎、陳生擁衆據襄陽,表乃使越與龐季單騎往說降之,江南遂悉平。}
 
 
袁術之在南陽也,與孫堅合從,欲襲奪表州,使堅攻表。堅為流矢所中死,軍敗,術遂不能勝表。李傕、郭汜入長安,欲連表為援,乃以表為鎮南將軍、荊州牧,封成武侯,假節。天子都許,表雖遣使貢獻,然北與袁紹相結。治中鄧羲諫表,表不聽,
 \gezhu{漢晉春秋曰:表荅羲曰:「內不失貢職,外不背盟主,此天下之達義也。治中獨何怪乎?」}
 羲辭疾而退,終表之世。張濟引兵入荊州界,攻穰城,為流矢所中死。荊州官屬皆賀,表曰:「濟以窮來,主人無禮,至於交鋒,此非牧意,牧受弔,不受賀也。」使人納其衆;衆聞之喜,遂服從。長沙太守張羨叛表,
 \gezhu{英雄記曰:張羨,南陽人。先作零陵、桂陽長,甚得江、湘間心,然性屈彊不順。表薄其為人,不甚禮也。羨由是懷恨,遂叛表焉。}
 表圍之連年不下。羨病死,長沙復立其子懌,表遂攻并懌,南收零、桂,北據漢川,地方數千里,帶甲十餘萬。
 \gezhu{英雄記曰:州界羣寇旣盡,表乃開立學官,博求儒士,使綦毋闓、宋忠等撰五經章句,謂之後定。}
 
 
太祖與袁紹方相持於官渡,紹遣人求助,表許之而不至,亦不佐太祖,欲保江漢間,觀天下變。從事中郎韓嵩、別駕劉先說表曰:「豪傑並爭,兩雄相持,天下之重在於將軍。將軍若欲有為,起乘其弊可也;若不然,固將擇所從。將軍擁十萬之衆,安坐而觀望。夫見賢而不能助,請和而不得,此兩怨必集於將軍,將軍不得中立矣。夫以曹公之明哲,天下賢俊皆歸之,其勢必舉袁紹,然後稱兵以向江漢,恐將軍不能禦也。故為將軍計者,不若舉州以附曹公,曹公必重德將軍;長享福祚,垂之後嗣,此萬全之策也。」表大將蒯越亦勸表,表狐疑,乃遣嵩詣太祖以觀虛實。嵩還,深陳太祖威德,說表遣子入質。表疑嵩反為太祖說,大怒,欲殺嵩,考殺隨嵩行者,知嵩無他意,乃止。
 \gezhu{傅子曰:初表謂嵩曰:「今天下大亂,未知所定,曹公擁天子都許,君為我觀其釁。」嵩對曰:「聖達節,次守節。嵩,守節者也。夫事君為君,君臣名定,以死守之;今策名委質,唯將軍所命,雖赴湯蹈火,死無辭也。以嵩觀之,曹公至明,必濟天下。將軍能上順天子,下歸曹公,必享百世之利,楚國實受其祐,使嵩可也;設計未定,嵩使京師,天子假嵩一官,則天子之臣,而將軍之故吏耳。在君為君,則嵩守天子之命,義不得復為將軍死也。唯將軍重思,無負嵩。」表遂使之,果如所言,天子拜嵩侍中,遷零陵太守,還稱朝廷、曹公之德也。表以為懷貳,大會寮屬數百人,陳兵見嵩,盛怒,持節將斬之,數曰:「韓嵩敢懷貳邪!」衆皆恐,欲令嵩謝。嵩不動,謂表曰:「將軍負嵩,嵩不負將軍!」具陳前言。表怒不已,其妻蔡氏諫之曰:「韓嵩,楚國之望也;且其言直,誅之無辭。」表乃弗誅而囚之。}
 表雖外皃儒雅,而心多疑忌,皆此類也。
 
 
劉備奔表,表厚待之,然不能用。
 \gezhu{漢晉春秋曰:太祖之始征柳城,劉備說表使襲許,表不從。及太祖還,謂備曰:「不用君言,故失此大會也。」備曰:「今天下分裂,日尋干戈,事會之來,豈有終極乎?若能應之於後者,則此未足為恨也。」}
 建安十三年,太祖征表,未至,表病死。
 
 
初,表及妻愛少子琮,欲以為後,而蔡瑁、張允為之支黨,乃出長子琦為江夏太守,衆遂奉琮為嗣。琦與琮遂為讎隙。
 \gezhu{典論曰:表疾病,琦還省疾。琦性慈孝,瑁、允恐琦見表,父子相感,更有託後之意,謂曰:「將軍命君撫臨江夏,為國東藩,其任至重;今釋衆而來,必見譴怒,傷親之歡心以增其疾,非孝敬也。」遂遏于戶外,使不得見,琦流涕而去。}
 越、嵩及東曹掾傅巽等說琮歸太祖,琮曰:「今與諸君據全楚之地,守先君之業,以觀天下,何為不可乎?」巽對曰:「逆順有大體,彊弱有定勢。以人臣而拒人主,逆也;以新造之楚而禦國家,其勢弗當也;以劉備而敵曹公,又弗當也。三者皆短,欲以抗王兵之鋒,必亡之道也。將軍自料何與劉備?」琮曰:「吾不若也。」巽曰:「誠以劉備不足禦曹公乎,則雖保楚之地,不足以自存也;誠以劉備足禦曹公乎,則備不為將軍下也。願將軍勿疑。」太祖軍到襄陽,琮舉州降。備走奔夏口。
 \gezhu{傅子曰:巽子公悌,瓌偉博達,有知人鑒。辟公府,拜尚書郎,後客荊州,以說劉琮之功,賜爵關內侯。文帝時為侍中,太和中卒,巽在荊州,目龐統為半英雄,證裴潛終以清行顯;統遂附劉備,見待次於諸葛亮,潛位至尚書令,並有名德。及在魏朝,魏諷以才智聞,巽謂之必反,卒如其言。巽弟子嘏,別有傳。漢晉春秋曰:王威說劉琮曰:「曹操得將軍旣降,劉備已走,必懈弛無備,輕行單進;若給威奇兵數千,徼之於險,操可獲也。獲操即威震四海,坐而虎步,中夏雖廣,可傳檄而定,非徒收一勝之功,保守今日而已。此難遇之機,不可失也。」琮不納。搜神記曰:建安初,荊州童謠曰:「八九年間始欲衰,至十三年無孑遺。」言自中平以來,荊州獨全,及劉表為牧,民又豐樂,至建安八年九年當始衰。始衰者,謂劉表妻死,諸將並零落也。十三年無孑遺者,表當又死,因以喪破也。是時,華容有女子忽啼呼云:「荊州將有大喪。」言語過差,縣以為妖言,繫獄月餘,忽於獄中哭曰:「劉荊州今日死。」華谷去州數百里,即遣馬吏驗視,而劉表果死,縣乃出之。續又歌吟曰:「不意李立為貴人。」後無幾,太祖平荊州,以涿郡李立字建賢為荊州刺史。}
 
 
太祖以琮為青州刺史、封列侯。
 \gezhu{魏武故事載令曰:「楚有江、漢山川之險,後復先疆與秦爭衡,荊州則其故地。劉鎮南乆用其民矣。身沒之後,諸子鼎峙,雖終難全,猶可引日。青州刺史琮,心高志潔,智深慮廣,輕榮重義,薄利厚德,蔑萬里之業,忽三軍之衆,篤中正之體,敦令名之譽,上耀先君之遺塵,下圖不朽之餘祚;鮑永之棄并州,竇融之離五郡,未足以喻也。雖封列侯一州之位,猶恨此寵未副其人;而比有牋求還州。監史雖尊,秩祿未優。今聽所執,表琮為諫議大夫,參同軍事。」}
 蒯越等侯者十五人。越為光祿勳;
 \gezhu{傅子曰:越,蒯通之後也,深中足智,魁傑有雄姿。大將軍何進聞其名,辟為東曹掾。越勸進誅諸閹官,進猶豫不決。越知進必敗,求出為汝陽令,佐劉表平定境內,表得以彊大。詔書拜章陵太守,封樊亭侯。荊州平,太祖與荀彧書曰:「不喜得荊州,喜得蒯異度耳。」建安十九年卒。臨終,與太祖書,託以門戶。太祖報書曰:「死者反生,生者不愧。孤少所舉,行之多矣。魂而有靈,亦將聞孤此言也。」}
 嵩,大鴻臚;
 \gezhu{先賢行狀曰:嵩字德高,義陽人。少好學,貧不改操。知世將亂,不應三公之命,與同好數人隱居於酈西山中。黃巾起,嵩避難南方,劉表逼以為別駕,轉從事中郎。表郊祀天地,嵩正諫不從,漸見違忤。奉使到許,事在前注。荊州平,嵩疾病,就在所拜授大鴻臚印綬。}
 羲,侍中;
 \gezhu{羲,章陵人。}
 先,尚書令;其餘多至大官。
 \gezhu{零陵先賢傳曰:先字始宗,博學彊記,尤好黃老言,明習漢家典故。為劉表別駕,奉章詣許,見太祖。時賔客並會,太祖問先:「劉牧如何郊天也?」先對曰:「劉牧託漢室肺腑,處牧伯之位,而遭王道未平,羣凶塞路,抱玉帛而無所聘頫,脩章表而不獲達御,是以郊天祀地,昭告赤誠。」太祖曰:「羣凶為誰?」先曰:「舉目皆是。」太祖曰:「今孤有熊羆之士,步騎十萬,奉辭伐罪,誰敢不服?」先曰:「漢道陵遲,羣生憔悴,旣無忠義之士翼戴天子,綏寧海內,使萬邦歸德,而阻兵安忍,曰莫己若,即蚩尤、智伯復見於今也。」太祖嘿然。拜先武陵太守。荊州平,先始為漢尚書,後為魏國尚書令。先甥同郡周不疑,字元直,零陵人。先賢傳稱不疑幼有異才,聦明敏達,太祖欲以女妻之,不疑不敢當。太祖愛子倉舒,夙有才智,謂可與不疑為儔。及倉舒卒,太祖心忌不疑,欲除之。文帝諫以為不可,太祖曰:「此人非汝所能駕御也。」乃遣刺客殺之。摯虞文章志曰:不疑死時年十七,著文論四首。世語曰:表死後八十餘年,至晉太康中,表冢見發。表及妻身形如生,芬香聞數里。}
 
 
評曰:董卓狼戾賊忍,暴虐不仁,自書契已來,殆未之有也。
 \gezhu{英雄記曰:昔大人見臨洮而銅人鑄,臨洮生卓而銅人毀;世有卓而大亂作,大亂作而卓身滅,抑有以也。}
 袁術奢淫放肆,榮不終己,自取之也。
 \gezhu{臣松之以為桀、紂無道,秦、莽縱虐,皆多歷年所,然後衆惡乃著。董卓自竊權柄,至于隕斃,計其日月,未盈三周,而禍崇山岳,毒流四海。其殘賊之性,實豺狼不若。「書契未有」,斯言為當。但評旣曰「賊忍」,又云「不仁」,賊忍,不仁,於辭為重。袁術無豪芒之功,纖介之善,而猖狂于時,妄自尊立,固義夫之所扼腕,人鬼之所同疾。雖復恭儉節用,而猶必覆亡不暇,而評但云「奢淫不終」,未足見其大惡。}
 袁紹、劉表咸有威容、器觀,知名當世。表跨蹈漢南,紹鷹揚河朔,然皆外寬內忌,好謀無決,有才而不能用,聞善而不能納,廢嫡立庶,舍禮崇愛,至于後嗣顛蹙,社稷傾覆,非不幸也。昔項羽背范增之謀,以喪其王業;紹之殺田豐,乃甚於羽遠矣!
 
 
\end{pinyinscope}