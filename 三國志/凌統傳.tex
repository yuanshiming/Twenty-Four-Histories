\article{凌統傳}
\begin{pinyinscope}
 
 
 凌統字公績,吳郡餘杭人也。父操,輕俠有膽氣,孫策初興,每從征伐,常冠軍履鋒。守永平長,平治山越,姦猾斂手,遷破賊校尉。及權統軍,從討江夏。入夏口,先登,破其前鋒,輕舟獨進,中流矢死。
 
 
 
 
 統年十五,左右多稱述者,權亦以操死國事,拜統別部司馬,行破賊都尉,使攝父兵。後從擊山賊,權破保屯先還,餘麻屯萬人,統與督張異等留攻圍之,克日當攻。先期,統與督陳勤會飲酒,勤剛勇任氣,因督祭酒,陵轢一坐,舉罰不以其道。統疾其侮慢,靣折不為用。勤怒詈統,及其父操,統流涕不荅,衆因罷出。勤乘酒凶悖,又於道路辱統。統不忍,引刀斫勤,數日乃死。及當攻屯,統曰:「非死無以謝罪。」乃率厲士卒,身當矢石,所次一靣,應時披壞,諸將乘勝,遂大破之。還,自拘於軍正。權壯其果毅,使得以功贖罪。
 
 
 
 
 後權復征江夏,統為前鋒,與所厚健兒數十人共乘一船,常去大兵數十里。行入右江,斬黃祖將張碩,盡獲船人。還以白權,引軍兼道,水陸並集。時呂蒙敗其水軍,而統先搏其城,於是大獲。權以統為承烈都尉,與周瑜等拒破曹公於烏林,遂攻曹仁,遷為校尉。雖在軍旅,親賢接士,輕財重義,有國士之風。
 
 
 
 
 又從破皖,拜盪寇中郎將,領沛相。與呂蒙等西取三郡,反自益陽,從往合肥,為右部督。時權徹軍,前部已發,魏將張遼等奄至津北。權使追還前兵,兵去已遠,勢不相及,統率親近三百人陷圍,扶扞權出。敵已毀橋,橋之屬者兩版,權策馬驅馳,統復還戰,左右盡死,身亦被創,所殺數十人,度權已免,乃還。橋敗路絕,統被甲潛行。權旣御船,見之驚喜。統痛親近無反者,悲不自勝。權引袂拭之,謂曰:「公績,亡者已矣,苟使卿在,何患無人?」
 
 
\gezhu{吳書曰:統創甚,權遂留統於舟,盡易其衣服。其創賴得卓氏良藥,故得不死。}
 拜偏將軍,倍給本兵。
 
 
 
 
 時有薦同郡盛暹於權者,以為梗槩大節有過於統,權曰:「且令如統足矣。」後召暹夜至,時統已卧,聞之,攝衣出門,執其手以入。其愛善不害如此。
 
 
 
 
 統以山中人尚多壯悍,可以威恩誘也,權令東占且討之,命勑屬城,凡統所求,皆先給後聞。統素愛士,士亦慕焉。得精兵萬餘人,過本縣,步入寺門,見長吏懷三版,恭敬盡禮,親舊故人,恩意益隆。事畢當出,會病卒,時年四十九。權聞之,拊牀起坐,哀不能自止,數日減膳,言及流涕,使張承為作銘誄。
 
 
二子烈、封,年各數歲,權內養於宮,愛待與諸子同,賔客進見,呼示之曰:「此吾虎子也。」及八九歲,令葛光教之讀書,十日一令乘馬,追錄統功,封烈亭侯,還其故兵。後烈有罪免,封復襲爵領兵。
 \gezhu{孫盛曰:觀孫權之養士也,傾心竭思,以求其死力,泣周泰之夷,殉陳武之妾,請呂蒙之命,育凌統之孤,卑曲苦志,如此之勤也。是故雖令德無聞,仁澤內著,而能屈彊荊吳,僭擬年歲者,抑有由也。然霸王之道,期於大者遠者,是以先王建德義之基,恢信順之宇,制經略之綱,明貴賤之叙,易簡而其親可乆,體全而其功可大,豈踒璅近務,邀利於當年哉?語曰「雖小道,必有可觀者焉,致遠恐泥」,其是之謂乎!}
 
 
\end{pinyinscope}