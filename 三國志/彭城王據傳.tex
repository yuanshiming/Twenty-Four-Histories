\article{彭城王據傳}
\begin{pinyinscope}
 
 
 彭城王據,建安十六年封范陽侯。二十二年,徙封宛侯。黃初二年,進爵為公。三年,為章陵王,其年徙封義陽。文帝以南方下溼,又以環太妃彭城人,徙封彭城。又徙封濟陰。五年,詔曰:「先王建國,隨時而制。漢祖增秦所置郡,至光武以天下損耗,并省郡縣。以今比之,益不及焉。其改封諸王,皆為縣王。」據改封定陶縣。太和六年,改封諸王,皆以郡為國,據復封彭城。景初元年,據坐私遣人詣中尚方作禁物,削縣二千戶。
 
 
\gezhu{魏書載璽書曰:「制詔彭城王:有司奏,王遣司馬董和,齎珠玉來到京師中尚方,多作禁物,交通工官,出入近署,踰侈非度,慢令違制,繩王以法。朕用憮然,不寧于心。王以懿親之重,處藩輔之位,典籍日陳於前,勤誦不輟於側。加雅素奉脩,恭肅敬慎,務在蹈道,孜孜不衰,豈忘率意正身,考終厥行哉?若然小疵,或謬于細人,忽不覺悟,以斯為失耳。書曰:『惟聖罔念作狂,惟狂克念作聖。』古人垂誥,乃至於此,故君子思心無斯須遠道焉。常慮所以累德者而去之,則德明矣;開心所以為塞者而通之,則心夷矣;慎行所以為尤者而脩之,則行全矣:三者,王之所能備也。今詔有司宥王,削縣二千戶,以彰八柄與奪之法。昔羲、文作易,著休復之語,仲尼論行,旣過能改。王其改行,茂昭斯義,率意無怠。」}
 三年,復所削戶邑。正元、景元中累增邑,并前四千六百戶。
 
 
\end{pinyinscope}