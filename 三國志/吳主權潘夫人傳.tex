\article{吳主權潘夫人傳}
\begin{pinyinscope}
 
 
 吳主權潘夫人,會稽句章人也。父為吏,坐法死。夫人與姊俱輸織室,權見而異之,召充後宮。得幸有娠,夢有似龍頭授己者,己以蔽膝受之,遂生孫亮。赤烏十三年,亮立為太子,請出嫁夫人之姊,權聽許之。明年,立夫人為皇后。性險妬容媚,自始至卒,譖害袁夫人等甚衆。
 
 
\gezhu{吳錄曰:袁夫人者,袁術女也,有節行而無子。權數以諸姬子與養之,輒不育。及步夫人薨,權欲立之。夫人自以無子,固辭不受。}
 權不豫,夫人使問中書令孫弘呂后專制故事。侍疾疲勞,因以羸疾,諸宮人伺其昏卧,共縊殺之,託言中惡。後事泄,坐死者六七人。權尋薨,合葬蔣陵。孫亮即位,以夫人姊壻譚紹為騎都尉,授兵。亮廢,紹與家屬送本郡廬陵。
 
 
\end{pinyinscope}