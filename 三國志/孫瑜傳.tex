\article{孫瑜傳}
\begin{pinyinscope}
 
 
 瑜字仲異,以恭義校尉始領兵衆。是時賔客諸將多江西人,瑜虛心綏撫,得其歡心。建安九年,領丹楊太守,為衆所附,至萬餘人。加綏遠將軍。十一年,與周瑜共討麻、保二屯,破之。後從權拒曹公於濡須,權欲交戰,瑜說權持重,權不從,軍果無功。遷奮威將軍,領郡如故,自溧陽徙屯牛渚。瑜以永安人饒助為襄安長,無錫人顏連為居巢長,使招納廬江二郡,各得降附。濟陰人馬普篤學好古,瑜厚禮之,使二府將吏子弟數百人就受業,遂立學官,臨饗講肄。是時諸將皆以軍務為事,而瑜好樂墳典,雖在戎旅,誦聲不絕。年三十九,建安二十年卒。瑜五子:彌、熈、燿、曼、紘。曼至將軍,封侯。
 
 
\end{pinyinscope}