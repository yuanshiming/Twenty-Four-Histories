\article{明帝紀}
\begin{pinyinscope}
 
 
 明皇帝諱叡,字元仲,文帝太子也。生而太祖愛之,常令在左右。
 
 
\gezhu{魏書曰:帝生數歲而有岐嶷之姿,武皇帝異之,曰:「我基於爾三世矣。」每朝宴會同,與侍中近臣並列帷幄。好學多識,特留意於法理。}
 年十五,封武德侯,黃初二年為齊公,三年為平原王。以其母誅,故未建為嗣。
 \gezhu{魏略曰:文帝以郭后無子,詔使子養帝。帝以母不以道終,意甚不平。後不獲已,乃敬事郭后,旦夕因長御問起居,郭后亦自以無子,遂加慈愛。文帝始以帝不恱,有意欲以他姬子京兆王為嗣,故久不拜太子。魏末傳曰:帝常從文帝獵,見子母鹿。文帝射殺鹿母,使帝射鹿子,帝不從,曰:「陛下已殺其母,臣不忍復殺其子。」因涕泣。文帝即放弓箭,以此深奇之,而樹立之意定。}
 七年夏五月,帝病篤,乃立為皇太子。丁巳,即皇帝位,大赦。尊皇太后曰太皇太后,皇后曰皇太后。諸臣封爵各有差。
 \gezhu{世語曰:帝與朝士素不接,即位之後,羣下想聞風采。居數日,獨見侍中劉曄,語盡日。衆人側聽,曄旣出,問「何如」?曄曰:「秦始皇、漢孝武之儔,才具微不及耳。」}
 癸未,追謚母甄夫人曰文昭皇后。壬辰,立皇弟蕤為陽平王。
 
 
 
 
 八月,孫權攻江夏郡,太守文聘堅守。朝議欲發兵救之,帝曰:「權習水戰,所以敢下船陸攻者,幾掩不備也。今已與聘相持,夫攻守勢倍,終不敢久也。」先時遣治書侍御史荀禹慰勞邊方,禹到,於江夏發所經縣兵及所從步騎千人乘山舉火,權退走。
 
 
 
 
 辛巳,立皇子冏為清河王。吳將諸葛瑾、張霸等寇襄陽,撫軍大將軍司馬宣王討破之,斬霸,征東大將軍曹休又破其別將於尋陽。論功行賞各有差。冬十月,清河王冏薨。十二月,以太尉鍾繇為太傅,征東大將軍曹休為大司馬,中軍大將軍曹真為大將軍,司徒華歆為太尉,司空王朗為司徒,鎮軍大將軍陳羣為司空,撫軍大將軍司馬宣王為驃騎大將軍。
 
 
太和元年春正月,郊祀武皇帝以配天,宗祀文皇帝於明堂以配上帝。分江夏南部,置江夏南部都尉。西平麴英反,殺臨羌令、西都長,遣將軍郝昭、鹿磐討斬之。二月辛未,帝耕于藉田。辛巳,立文昭皇后寢廟於鄴。丁亥,朝日于東郊。夏四月乙亥,行五銖錢。甲申,初營宗廟。秋八月,夕月于西郊。冬十月丙寅,治兵于東郊。焉耆王遣子入侍。十一月,立皇后毛氏。賜天下男子爵人二級,鰥寡孤獨不能自存者賜穀。十二月,封后父毛嘉為列侯。新城太守孟達反,詔驃騎將軍司馬宣王討之。
 \gezhu{三輔決錄曰:伯郎,涼州人,名不令休。其注曰:伯郎姓孟,名他,扶風人。靈帝時。中常侍張讓專朝政,讓監奴典護家事。他仕不遂,乃盡以家財賂監奴,與共結親,積年家業為之破盡。衆奴皆慙,問他所欲,他曰:「欲得卿曹拜耳。」奴被恩久,皆許諾。時賔客求見讓者,門下車常數百乘,或累日不得通。他最後到,衆奴伺其至,皆迎車而拜,徑將他車獨入。衆人悉驚,謂他與讓善,爭以珍物遺他。他得之,盡以賂讓,讓大喜。他又以蒲桃酒一斛遺讓,即拜涼州刺史。他生達,少入蜀。其處蜀事迹在劉封傳。魏略曰:達以延康元年率部曲四千餘家歸魏。文帝時初即王位,旣宿知有達,聞其來,甚恱,令貴臣有識察者往觀之,還曰「將帥之才也」,或曰「卿相之器也」,王益欽達。逆與達書曰:「近日有命,未足達旨,何者?昔伊摯背商而歸周,百里去虞而入秦,樂毅感鴟夷以蟬蛻,王遵識逆順以去就,皆審興廢之符效,知成敗之必然,故丹青畫其形容,良史載其功勳。聞卿姿度純茂,器量優絕,當騁能明時,收名傳記。今者翻然濯鱗清流,甚相嘉樂,虛心西望,依依若舊,下筆屬辭,歡心從之。昔虞卿入趙,再見取相,陳平就漢,一覲參乘,孤今於卿,情過於往,故致所御馬物以昭忠愛。」又曰:「今者海內清定,萬里一統,三垂無風塵之警,中夏無狗吠之虞,以是弛罔闊禁,與世無疑,保官空虛,初無資任。卿來相就,當明孤意,慎勿令家人繽紛道路,以親駭疏也。若卿欲來相見,且當先安部曲,有所保固,然後徐徐輕騎來東。」達旣至譙,進見閑雅,才辯過人,衆莫不屬目。又王近出,乘小輦,執達手,撫其背戲之曰:「卿得無為劉備刺客邪?」遂與同載。又加拜散騎常侍,領新城太守,委以西南之任。時衆臣或以為待之太猥,又不宜委以方任。王聞之曰:「吾保其無他,亦譬以蒿箭射蒿中耳。」達旣為文帝所寵,又與桓階、夏侯尚親善,及文帝崩,時桓、尚皆卒,達自以羈旅久在疆場,心不自安。諸葛亮聞之,陰欲誘達,數書招之,達與相報荅。魏興太守申儀與達有隙,密表達與蜀潛通,帝未之信也。司馬宣王遣參軍梁幾察之,又勸其入朝。達驚懼,遂反。干寶晉紀曰:達初入新城,登白馬塞,歎曰:「劉封、申耽,據金城千里而失之乎!」}
 
 
二年春正月,宣王攻破新城,斬達,傳其首。
 \gezhu{魏略曰:宣王誘達將李輔及達甥鄧賢,賢等開門內軍。達被圍旬有六日而敗,焚其首于洛陽四達之衢。}
 分新城之上庸、武陵、巫縣為上庸郡,錫縣為錫郡。
 
 
蜀大將諸葛亮寇邊,天水、南安、安定三郡吏民叛應亮。
 \gezhu{魏書曰:是時朝臣未知計所出,帝曰:「亮阻山為固,今者自來,旣合兵書致人之術;且亮貪三郡,知進而不知退,今因此時,破亮必也。」乃部勒兵馬步騎五萬拒亮。}
 遣大將軍曹真都督關右,並進兵。右將軍張郃擊亮於街亭,大破之。亮敗走,三郡平。丁未,行幸長安。
 \gezhu{魏略載帝露布天下并班告益州曰:「劉備背恩,自竄巴蜀。諸葛亮棄父母之國,阿殘賊之黨,神人被毒,惡積身滅。亮外慕立孤之名,而內貪專擅之實。劉升之兄弟守空城而已。亮又侮易益土,虐用其民,是以利狼、宕渠、高定、青羌莫不瓦解,為亮仇敵。而亮反裘負薪,裏盡毛殫,刖趾適屨,刻肌傷骨,反更稱說,自以為能。行兵於井底,游步於牛蹄。自朕即位,三邊無事,猶哀憐天下數遭兵革,且欲養四海之耆老,長後生之孤幼,先移風於禮樂,次講武於農隙,置亮畫外,未以為虞。而亮懷李熊愚勇之智,不思荊邯度德之戒,驅略吏民,盜利祁山。王師方振,膽破氣奪,馬謖、高祥望旗奔敗。虎臣逐北,蹈尸涉血,亮也小子,震驚朕師。猛銳踊躍,咸思長驅。朕惟率土莫非王臣,師之所處,荊棘生焉,不欲使十室之邑忠信貞良與夫淫昏之黨,同受塗炭。故先開示,以昭國誠,勉思變化,無滯亂邦。巴蜀將吏士民諸為亮所劫迫,公卿已下皆聽束手。」}
 夏四月丁酉,還洛陽宮。
 \gezhu{魏略曰:是時譌言云帝已崩,從駕羣臣迎立雍丘王植。京師自卞太后羣公盡懼。及帝還,皆私察顏色。卞太后悲喜,欲推始言者,帝曰:「天下皆言,將何所推?」}
 赦繫囚非殊死以下。乙巳,論討亮功,封爵增邑各有差。五月,大旱。六月,詔曰:「尊儒貴學,王教之本也。自頃儒官或非其人,將何以宣明聖道?其高選博士,才任侍中常侍者。申勑郡國,貢士以經學為先。」秋九月,曹休率諸軍至皖,與吳將陸議戰於石亭,敗績。乙酉,立皇子穆為繁陽王。庚子,大司馬曹休薨。冬十月,詔公卿近臣舉良將各一人。十一月,司徒王朗薨。十二月,諸葛亮圍陳倉,曹真遣將軍費曜等拒之。
 \gezhu{魏略曰:先是使將軍郝昭築陳倉城;會亮至,圍昭,不能拔。昭字伯道,太原人,為人雄壯,少入軍為部曲督,數有戰功,為雜號將軍,遂鎮守河西十餘年,民夷畏服。亮圍陳倉,使昭鄉人靳詳於城外遙說之,昭於樓上應詳曰:「魏家科法,卿所練也;我之為人,卿所知也。我受國恩多而門戶重,卿無可言者,但有必死耳。卿還謝諸葛,便可攻也。」詳以昭語告亮,亮又使詳重說昭,言人兵不敵,無為空自破滅。昭謂詳曰:「前言已定矣。我識卿耳,箭不識也。」詳乃去。亮自以有衆數萬,而昭兵纔千餘人,又度東救未能便到,乃進兵攻昭,起雲梯衝車以臨城。昭於是以火箭逆射其雲梯,梯然,梯上人皆燒死。昭又以繩連石磨壓其衝車,衝車折。亮乃更為井闌百尺以付城中,以土丸填壍,欲直攀城,昭又於內築重牆。亮足為城突,欲踊出於城裏,昭又於城內穿地橫截之。晝夜相攻拒二十餘日,亮無計,救至,引退。詔嘉昭善守,賜爵列侯。及還,帝引見慰勞之,顧謂中書令孫資曰:「卿鄉里乃有爾曹快人,為將灼如此,朕復何憂乎?」仍欲大用之。會病亡,遺令戒其子凱曰:「吾為將,知將不可為也。吾數發冢,取其木以為攻戰具,又知厚葬無益於死者也。汝必斂以時服。且人生有處所耳,死復何在邪?今去本墓遠,東西南北,在汝而已。」}
 遼東太守公孫恭兄子淵劫奪恭位,遂以淵領遼東太守。
 
 
 
 
 三年夏四月,元城王禮薨。六月癸卯,繁陽王穆薨。戊申,追尊高祖大長秋曰高皇帝,夫人吳氏曰高皇后。
 
 
 
 
 秋七月,詔曰:「禮,王后無嗣,擇建支子以繼太宗,則當纂正統而奉公義,何得復顧私親哉!漢宣繼昭帝後,加悼考以皇號;哀帝以外藩援立,而董宏等稱引亡秦,或誤時朝,旣尊恭皇,立廟京都,又寵藩妾,使比長信,敘昭穆於前殿,並四位於東宮,僭差無度,人神弗祐,而非罪師丹忠正之諫,用致丁、傅焚如之禍。自是之後,相踵行之。昔魯文逆祀,罪由夏父;宋國非度,譏在華元。其令公卿有司,深以前世行事為戒。後嗣萬一有由諸侯入奉大統,則當明為人後之義;敢為佞邪導諛時君,妄建非正之號以干正統,謂考為皇,稱妣為后,則股肱大臣,誅之無赦。其書之金策,藏之宗廟,著于今典。」
 
 
 
 
 冬十月,改平望觀曰聽訟觀。帝常言「獄者,天下之性命也」,每斷大獄,常幸觀臨聽之。
 
 
初,洛陽宗廟未成,神主在鄴廟。十一月,廟始成,使太常韓曁持節迎高皇帝、太皇帝、武帝、文帝神主于鄴,十二月己丑至,奉安神主于廟。
 \gezhu{臣松之按:黃初四年,有司奏立二廟,太皇帝大長秋與文帝之高祖共一廟,特立武帝廟,百世不毀。今此無高祖神主,蓋以親盡毀也。此則魏初唯立親廟,祀四室而已。至景初元年,始定七廟之制。孫盛曰:事亡猶存,祭如神在,迎遷神主,正斯宜矣。}
 
 
 
 
 癸卯,大月氏王波調遣使奉獻,以調為親魏大月氏王。
 
 
四年春二月壬午,詔曰:「世之質文,隨教而變。兵亂以來,經學廢絕,後生進趣,不由典謨。豈訓導未洽,將進用者不以德顯乎?其郎吏學通一經,才任牧民,博士課試,擢其高第者,亟用;其浮華不務道本者,皆罷退之。」戊子,詔太傅三公:以文帝典論刻石,立于廟門之外。癸巳,以大將軍曹真為大司馬,驃騎將軍司馬宣王為大將軍,遼東太守公孫淵為車騎將軍。夏四月,太傅鍾繇薨。六月戊子,太皇太后崩。丙申,省上庸郡。秋七月,武宣卞后祔葬于高陵。詔大司馬曹真、大將軍司馬宣王伐蜀。八月辛巳,行東巡,遣使者以特牛祠中嶽。
 \gezhu{魏書曰:行過繁昌,使執金吾臧霸行太尉事,以特牛祠受禪壇。臣松之按:漢紀章帝元和三年,詔高邑縣祠即位壇,五成陌北,臘祠門戶。此雖前代已行故事,然為壇以祀天,而壇非神也,今無事於上帝,而致祀於虛壇,求之義典,未詳所據。}
 乙未,幸許昌宮。九月,大雨,伊、洛、河、漢水溢,詔真等班師。冬十月乙卯,行還洛陽宮。庚申,令:「罪非殊死聽贖各有差。」十一月,太白犯歲星。十二月辛未,改葬文昭甄后于朝陽陵。丙寅,詔公卿舉賢良。
 
 
五年春正月,帝耕于藉田。三月,大司馬曹真薨。諸葛亮寇天水,詔大將軍司馬宣王拒之。自去冬十月至此月不雨,辛巳,大雩。夏四月,鮮卑附義王軻比能率其種人及丁零大人兒禪詣幽州貢名馬。復置護匈奴中郎將。秋七月丙子,以亮退走,封爵增位各有差。
 \gezhu{魏書曰:初,亮出,議者以為亮軍無輜重,糧必不繼,不擊自破,無為勞兵;或欲自芟上邽左右生麥以奪賊食,帝皆不從。前後遣兵增宣王軍,又勑使護麥。宣王與亮相持,賴得此麥以為軍糧。}
 乙酉,皇子殷生,大赦。
 
 
 
 
 八月,詔曰:「古者諸侯朝聘,所以敦睦親親協和萬國也。先帝著令,不欲使諸王在京都者,謂幼主在位,母后攝政,防微以漸,關諸盛衰也。朕惟不見諸王十有二載,悠悠之懷,能不興思!其令諸王及宗室公侯各將適子一人朝。後有少主、母后在宮者,自如先帝令,申明著于令。」十一月乙酉,月犯軒轅大星。戊戌晦,日有蝕之。十二月甲辰,月犯鎮星。戊午,太尉華歆薨。
 
 
 
 
 六年春二月,詔曰:「古之帝王,封建諸侯,所以藩屏王室也。詩不云乎,『懷德維寧,宗子維城』。秦、漢繼周,或彊或弱,俱失厥中。大魏創業,諸王開國,隨時之宜,未有定制,非所以永為後法也。其改封諸侯王,皆以郡為國。」三月癸酉,行東巡,所過存問高年鰥寡孤獨,賜穀帛。乙亥,月犯軒轅大星。夏四月壬寅,行幸許昌宮。甲子,初進新果于廟。五月,皇子殷薨,追封謚安平哀王。秋七月,以衞尉董昭為司徒。九月,行幸摩陂,治許昌宮,起景福、承光殿。冬十月,殄夷將軍田豫帥衆討吴將周賀於成山,殺賀。十一月丙寅,太白晝見。有星孛于翼,近太微上將星。庚寅,陳思王植薨。十二月,行還許昌宮。
 
 
青龍元年春正月甲申,青龍見郟之摩陂井中。二月丁酉,幸摩陂觀龍,於是改年;改摩陂為龍陂,賜男子爵人二級,鰥寡孤獨無出今年租賦。三月甲子,詔公卿舉賢良篤行之士各一人。夏五月壬申,詔祀故大將軍夏侯惇、大司馬曹仁、車騎將軍程昱於太祖廟庭。
 \gezhu{魏書載詔曰:「昔先王之禮,於功臣存則顯其爵祿,沒則祭于大蒸,故漢氏功臣祠於廟庭。大魏元功之臣功勳優著,終始休明者,其皆依禮祀之。」於是以惇等配饗之。}
 戊寅,北海王蕤薨。閏月庚寅朔,日有蝕之。丁酉,改封宗室女非諸王女皆為邑主。詔諸郡國山川不在祠典者勿祠。六月,洛陽宮鞠室災。
 
 
 
 
 保塞鮮卑大人步度根與叛鮮卑大人軻比能私通,并州刺史畢軌表,輒出軍以外威比能,內鎮步度根。帝省表曰:「步度根以為比能所誘,有自疑心。今軌出軍,適使二部驚合為一,何所威鎮乎?」促勑軌,以出軍者慎勿越塞過句注也。比詔書到,軌以進軍屯陰館,遣將軍蘇尚、董弼追鮮卑。比能遣子將千餘騎迎步度根部落,與尚、弼相遇,戰於樓煩,二將敗沒。步度根部落皆叛出塞,與比能合寇邊。遣驍騎將軍秦朗將中軍討之,虜乃走漠北。
 
 
 
 
 秋九月,安定保塞匈奴大人胡薄居姿職等叛,司馬宣王遣將軍胡遵等追討,破降之。
 
 
冬十月,步度根部落大人戴胡阿狼泥等詣并州降,朗引軍還。
 \gezhu{魏氏春秋曰:朗字元明,新興人。獻帝傳曰:朗父名宜祿,為呂布使詣袁術,術妻以漢宗室女。其前妻杜氏留下邳。布之被圍,關羽屢請於太祖,求以杜氏為妻,太祖疑其有色,及城陷,太祖見之,乃自納之。宜祿歸降,以為銍長。及劉備走小沛,張飛隨之,過謂宜祿曰:「人取汝妻,而為之長,何蚩蚩若是邪!隨我去乎?」宜祿從之數里,悔欲還,飛殺之。朗隨母氏畜于公宮,太祖甚愛之,每坐席,謂賔客曰:「豈有人愛假子如孤者乎?」魏略曰:朗游遨諸侯間,歷武、文之世而無尤也。及明帝即位,授以內官,為驍騎將軍、給事中,每車駕出入,朗常隨從。時明帝喜發舉,數有以輕微而致大辟者,朗終不能有所諫止,又未甞進一善人,帝亦以是親愛;每顧問之,多呼其小字阿穌,數加賞賜,為起大第於京城中。四方雖知朗無能為益,猶以附近至尊,多賂遺之,富均公侯。世語曰:朗子秀,勁厲能直言,為晉武帝博士。魏略以朗與孔桂俱在佞倖篇。桂字叔林,天水人也。建安初,數為將軍楊秋使詣太祖,太祖表拜騎都尉。桂性便辟,曉博弈、蹹鞠,故太祖愛之,每在左右,出入隨從。桂察太祖意,喜樂之時,因言次曲有所陳,事多見從,數得賞賜,人多餽遺,桂由此侯服玉食。太祖旣愛桂,五官將及諸侯亦皆親之。其後桂見太祖久不立太子,而有意於臨菑侯,因更親附臨菑侯而簡於五官將,將甚銜之。及太祖薨,文帝即王位,未及致其罪。黃初元年,隨例轉拜駙馬都尉。而桂私受西域貨賂,許為人事。事發,有詔收問,遂殺之。魚豢曰:為上者不虛授,處下者不虛受,然後外無伐檀之歎,內無尸素之刺,雍熈之美著,太平之律顯矣。而佞倖之徒,但姑息人主,至乃無德而榮,無功而祿,如是焉得不使中正日朘,傾邪滋多乎!以武皇帝之慎賞,明皇帝之持法,而猶有若此等人,而況下斯者乎?}
 
 
十二月,公孫淵斬送孫權所遣使張彌、許晏首,以淵為大司馬樂浪公。
 \gezhu{世語曰:并州刺史畢軌送漢故度遼將軍范明友鮮卑奴,年三百五十歲,言語飲食如常人。奴云:「霍顯,光後小妻。明友妻,光前妻女。」博物志曰:時京邑有一人失其姓名,食啖兼十許人,遂肥不能動。其父曾作遠方長吏,官徙送彼縣,令故義傳供食之;一二年中,一鄉中輒為之儉。傅子曰:時太原發冢破棺,棺中有一生婦人,將出與語,生人也。送之京師,問其本事,不知也。視其冢上樹木可三十歲,不知此婦人三十歲常生於地中邪?將一朝欻生,偶與發冢者會也?}
 
 
二年春二月乙未,太白犯熒惑。癸酉,詔曰:「鞭作官刑,所以糾慢怠也,而頃多以無辜死。其減鞭杖之制,著于令。」三月庚寅,山陽公薨,帝素服發哀,遣使持節典護喪事。己酉,大赦。夏四月,大疫。崇華殿災。丙寅,詔有司以太牢告祠文帝廟。追謚山陽公為漢孝獻皇帝,葬以漢禮。
 \gezhu{獻帝傳曰:帝變服,率羣臣哭之,使使持節行司徒太常和洽弔祭,又使持節行大司空大司農崔林監護喪事。詔曰:「蓋五帝之事尚矣,仲尼盛稱堯、舜巍巍蕩蕩之功者,以為禪代乃大聖之懿事也。山陽公深識天祿永終之運,禪位文皇帝以順天命。先帝命公行漢正朔,郊天祀祖以天子之禮,言事不稱臣,此舜事堯之義也。昔放勛殂落,四海如喪考妣,遏密八音,明喪葬之禮同於王者也。今有司奏喪禮比諸侯王,此豈古之遺制而先帝之至意哉?今謚公漢孝獻皇帝。」使太尉具以一太牢告祠文帝廟,曰:「叡聞夫禮也者,反本請吉,不忘厥初,是以先代之君,尊尊親親,戚有尚焉。今山陽公寢疾棄國,有司建言喪紀之禮視諸侯王。叡惟山陽公昔知天命永終於己,深觀歷數允在聖躬,傳祚禪位,尊我民主,斯乃陶唐懿德之事也。黃初受終,命公于國行漢正朔,郊天祀祖禮樂制度率乃漢舊,斯亦舜、禹明堂之義也。上考遂初,皇極攸建,允熈克讓,莫明于茲。蓋子以繼志嗣訓為孝,臣以配命欽述為忠,故詩稱『匪棘其猶,聿追來孝』,書曰『前人受命,茲不忘大功』。叡敢不奉承徽典,以昭皇考之神靈。今追謚山陽公曰孝獻皇帝,冊贈璽紱。命司徒、司空持節弔祭護喪,光祿、大鴻臚為副,將作大匠、復土將軍營成陵墓,及置百官羣吏,車旗服章喪葬禮儀,一如漢氏故事;喪葬所供羣官之費,皆仰大司農。立其後嗣為山陽公,以通三統,永為魏賔。」於是贈冊曰:「嗚呼,昔皇天降戾于漢,俾逆臣董卓,播厥凶虐,焚滅京都,劫遷大駕。于時六合雲擾,姦雄熛起。帝自西京,徂唯求定,臻茲洛邑。疇咨聖賢,聿改乘轅,又遷許昌,武皇帝是依。歲在玄枵,皇師肇征,迄于鶉尾,十有八載,羣寇殲殄,九域咸乂。惟帝念功,祚茲魏國,大啟土宇。爰及文皇帝,齊聖廣淵,仁聲旁流,柔遠能邇,殊俗向義,乾精承祚,坤靈吐曜,稽極玉衡,允膺歷數,度于軌儀,克猒帝心。乃仰欽七政,俯察五典,弗采四嶽之謀,不俟師錫之舉,幽贊神明,承天禪位。祚逮朕躬,統承洪業。蓋聞昔帝堯,元愷旣舉,凶族未流,登舜百揆,然後百揆時序,內平外成,授位明堂,退終天祿,故能冠德百王,表功高嶽。自往迄今,彌歷七代,歲曁三千,而大運來復,庸命底績,纂我民主,作建皇極。念重光,紹咸池,繼韶夏,超羣后之遐蹤,邈商、周之慙德,可謂高朗令終,昭明洪烈之懿盛者矣。非夫漢、魏與天地合德,與四時合信,動和民神,格于上下,其孰能至於此乎?朕惟孝獻享年不永,欽若顧命,考之典謨,恭述皇考先靈遺意,闡崇弘謚,奉成聖美,以章希世同符之隆,以傳億載不朽之榮。魂而有靈,嘉茲弘休。嗚呼哀哉!」八月壬申,葬于山陽國,陵曰禪陵,置園邑。葬之日,帝制錫衰弁絰,哭之慟。適孫桂氏鄉侯康,嗣立為山陽公。}
 
 
是月,諸葛亮出斜谷,屯渭南,司馬宣王率諸軍拒之。詔宣王:「但堅壁拒守以挫其鋒,彼進不得志,退無與戰,乆停則糧盡,虜略無所獲,則必走矣。走而追之,以逸待勞,全勝之道也。」
 \gezhu{魏氏春秋曰:亮旣屢遣使交書,又致巾幗婦人之飾,以怒宣王。宣王將出戰,辛毗杖節奉詔,勒宣王及軍吏已下,乃止。宣王見亮使,唯問其寢食及其事之煩簡,不問戎事。使對曰:「諸葛公夙興夜寐,罰二十已上,皆親覽焉;所啖食不過數升。」宣王曰:「亮體斃矣,其能久乎?」}
 
 
 
 
 五月,太白晝見。孫權入居巢湖口,向合肥新城,又遣將陸議、孫韶各將萬餘人入淮、沔。六月,征東將軍滿寵進軍拒之。寵欲拔新城守,致賊壽春,帝不聽,曰:「昔漢光武遣兵縣據略陽,終以破隗囂,先帝東置合肥,南守襄陽,西固祁山,賊來輒破於三城之下者,地有所必爭也。縱權攻新城,必不能拔。勑諸將堅守,吾將自往征之,比至,恐權走也。」秋七月壬寅,帝親御龍舟東征,權攻新城,將軍張頴等拒守力戰,帝軍未至數百里,權遁走,議、韶等亦退。羣臣以為大將軍方與諸葛亮相持未解,車駕可西幸長安。帝曰:「權走,亮膽破,大將軍以制之,吾無憂矣。」遂進軍幸壽春,錄諸將功,封賞各有差。八月己未,大曜兵,饗六軍,遣使者持節犒勞合肥、壽春諸軍。辛巳,行還許昌宮。
 
 
 
 
 司馬宣王與亮相持,連圍積日,亮數挑戰,宣王堅壘不應。會亮卒,其軍退還。
 
 
 
 
 冬十月乙丑,月犯鎮星及軒轅。戊寅,月犯太白。十一月,京都地震,從東南來,隱隱有聲,搖動屋瓦。十二月,詔有司刪定大辟,減死罪。
 
 
三年春正月戊子,以大將軍司馬宣王為太尉。己亥,復置朔方郡。京都大疫。丁巳,皇太后崩。乙亥,隕石于壽光縣。三月庚寅,葬文德郭后,營陵于首陽陵澗西,如終制。
 \gezhu{顧愷之啟蒙注曰:魏時人有開周王冢者,得殉葬女子,經數日而有氣,數月而能語;年可二十。送詣京師,郭太后愛養之。十餘年,太后崩,哀思哭泣,一年餘而死。}
 
 
是時,大治洛陽宮,起昭陽、太極殿,築緫章觀。百姓失農時,直臣楊阜、高堂隆等各數切諫,雖不能聽,常優容之。
 \gezhu{魏略曰:是年起太極諸殿,築緫章觀,高十餘丈,建翔鳳於其上;又於芳林園中起陂池,楫櫂越歌;又於列殿之北立八坊,諸才人以次序處其中,貴人夫人以上轉南附焉,其秩石擬百官之數。帝常游宴在內,乃選女子知書可付信者六人,以為女尚書,使典省外奏事,處當畫可,自貴人以下至尚保,及給掖庭灑掃,習伎歌者,各有千數。通引穀水過九龍殿前,為玉井綺欄,蟾蜍含受,神龍吐出。使博士馬均作司南車,水轉百戲。歲首建巨獸,魚龍曼延,弄馬倒騎,備如漢西京之制,築閶闔諸門闕外罘罳。太子舍人張茂以吳、蜀數動,諸將出征,而帝盛興宮室,留意於玩飾,賜與無度,帑藏空竭;又錄奪士女前已嫁為吏民妻者,還以配士,旣聽以生口自贖,又簡選其有姿首者內之掖庭,乃上書諫曰:「臣伏見詔書,諸士女嫁非士者,一切錄奪,以配戰士,斯誠權時之宜,然非大化之善者也。臣請論之。陛下,天之子也,百姓吏民,亦陛下之子也。禮,賜君子小人不同日,所以殊貴賤也。吏屬君子,士為小人,今奪彼以與此,亦無以異於奪兄之妻妻弟也,於父母之恩偏矣。又詔書聽得以生口年紀、顏色與妻相當者自代,故富者則傾家盡產,貧者舉假貸貰,貴買生口以贖其妻;縣官以配士為名而實內之掖庭,其醜惡者乃出與士。得婦者未必有懽心,而失妻者必有憂色,或窮或愁,皆不得志。夫君有天下而不得萬姓之懽心者,尠不危殆。且軍師在外數千萬人,一日之費非徒千金,舉天下之賦以奉此役,猶將不給,況復有宮庭非員無錄之女,椒房母后之家賞賜橫興,內外交引,其費半軍。昔漢武帝好神仙,信方士,掘地為海,封土為山,賴是時天下為一,莫敢與爭者耳。自衰亂以來,四五十載,馬不捨鞍,士不釋甲,每一交戰,血流丹野,創痍號痛之聲于今未已。猶彊寇在疆,圖危魏室。陛下不兢兢業業,念崇節約,思所以安天下者,而乃奢靡是務,中尚方純作玩弄之物,炫燿後園,建承露之盤,斯誠快耳目之觀,然亦足以騁寇讎之心矣。惜乎,舍堯舜之節儉,而為漢武之侈事,臣竊為陛下不取也。願陛下沛然下詔,萬幾之事有無益而有損者悉除去之,以所除無益之費,厚賜將士父母妻子之飢寒者,問民所疾而除其所惡,實倉廩,繕甲兵,恪恭以臨天下。如是,吳賊面縛,蜀虜輿櫬,不待誅而自服,太平之路可計日而待也。陛下可無勞神思於海表,軍師高枕,戰士備員。今羣公皆結舌,而臣所以不敢不獻瞽言者,臣昔上要言,散騎奏臣書,以聽諫篇為善,詔曰:『是也』,擢臣為太子舍人;且臣作書譏為人臣不能諫諍,今有可諫之事而臣不諫,此為作書虛妄而不能言也。臣年五十,常恐至死無以報國,是以投軀沒身,冒昧以聞,惟陛下裁察。」書通,上顧左右曰:「張茂恃鄉里故也。」以事付散騎而已。茂字彥林,沛人。}
 
 
秋七月,洛陽崇華殿災,八月庚午,立皇子芳為齊王,詢為秦王。丁巳,行還洛陽宮。命有司復崇華,改名九龍殿。冬十月己酉,中山王衮薨。壬申,太白晝見。十一月丁酉,行幸許昌宮。
 \gezhu{魏氏春秋曰:是歲張掖郡刪丹縣金山玄川溢涌,寶石負圖,狀象靈龜,廣一丈六尺,長一丈七尺一寸,圍五丈八寸,立于川西。有石馬七,其一仙人騎之,其一羈絆,其五有形而不善成。有玉匣關蓋於前,上有玉字,玉玦二,璜一。麒麟在東,鳳鳥在南,白虎在西,犧牛在北,馬自中布列四面,色皆蒼白。其南有五字,曰「上上三天王」;又曰「述大金,大討曹,金但取之,金立中,大金馬一匹在中,大吉開壽,此馬甲寅述水」。凡「中」字六,「金」字十;又有若八卦及列宿孛彗之象焉。世語曰:又有一雞象。搜神記曰:初,漢元、成之世,先識之士有言曰,魏年有和,當有開石於西三千餘里,繫五馬,文曰「大討曹」。及魏之初興也,張掖之柳谷,有開石焉,始見於建安,形成於黃初,文備於太和,周圍七尋,中高一仞,蒼質素章,龍馬、麟鹿、鳳皇、仙人之象,粲然咸著,此一事者,魏、晉代興之符也。至晉泰始三年,張掖太守焦勝上言,以留郡本國圖校今石文,文字多少不同,謹具圖上。桉其文有五馬象,其一有人平上幘,執戟而乘之,其一有若馬形而不成,其字有「金」,有「中」,有「大司馬」,有「王」,有「大吉」,有「正」,有「開壽」,其一成行,曰「金當取之」。漢晉春秋曰:氐池縣大柳谷口夜激波涌溢,其聲如雷,曉而有蒼石立水中,長一丈六尺,高八尺,白石畫之,為十三馬,一牛,一鳥,八卦玉玦之象,皆隆起,其文曰「大討曹,適水中,甲寅」。帝惡其「討」也,使鑿去為「計」,以蒼石窒之,宿昔而白石滿焉。至晉初,其文愈明,馬象皆煥徹如玉焉。}
 
 
 
 
 四年春二月,太白復晝見,月犯太白,又犯軒轅一星,入太微而出。夏四月,置崇文觀,徵善屬文者以充之。五月乙卯,司徒董昭薨。丁巳,肅慎氏獻楛矢。
 
 
 
 
 六月壬申,詔曰:「有虞氏畫象而民弗犯,周人刑錯而不用。朕從百王之末,追望上世之風,邈乎何相去之遠?法令滋章,犯者彌多,刑罰愈衆,而姦不可止。往者桉大辟之條,多所蠲除,思濟生民之命,此朕之至意也。而郡國斃獄,一歲之中尚過數百,豈朕訓導不醇,俾民輕罪,將苛法猶存,為之陷穽乎?有司其議獄緩死,務從寬簡,及乞恩者,或辭未出而獄以報斷,非所以究理盡情也。其令廷尉及天下獄官,諸有死罪具獄以定,非謀反及手殺人,亟語其親治,有乞恩者,使與奏當文書俱上,朕將思所以全之。其布告天下,使明朕意。」
 
 
 
 
 秋七月,高句驪王宮斬送孫權使胡衞等首,詣幽州。甲寅,太白犯軒轅大星。冬十月己卯,行還洛陽宮。甲申,有星孛于大辰,乙酉,又孛于東方。十一月己亥,彗星見,犯宦者天紀星。十二月癸巳,司空陳羣薨。乙未,行幸許昌宮。
 
 
景初元年春正月壬辰,山茌縣言黃龍見。
 \gezhu{茌音仕狸反。}
 於是有司奏,以為魏得地統,宜以建丑之月為正。三月,定歷改年為孟夏四月。
 \gezhu{魏書曰:初,文皇帝即位,以受禪于漢,因循漢正朔弗改。帝在東宮著論,以為五帝三王雖同氣共祖,禮不相襲,正朔自宜改變,以明受命之運。及即位,優游者久之,史官復著言宜改,乃詔三公、特進、九卿、中郎將、大夫、博士、議郎、千石、六百石博議,議者或不同。帝據古典,甲子詔曰:「夫太極運三辰五星於上,元氣轉三統五行於下,登降周旋,終則又始。故仲尼作春秋,於三微之月,每月稱王,以明三正迭相為首。今推三統之次,魏得地統,當以建丑之月為正月。考之羣藝,厥義章矣。其改青龍五年三月為景初元年四月。」}
 服色尚黃,犧牲用白,戎事乘黑首白馬,建大赤之旂,朝會建大白之旗。
 \gezhu{臣松之桉:魏為土行,故服色尚黃。行殷之時,以建丑為正,故犧牲旂旗一用殷禮。禮記云:「夏后氏尚黑,故戎事乘驪,牲用玄;殷人尚白,戎事乘翰,牲用白;周人尚赤,戎事乘騵,牲用騂。」鄭玄云:「夏后氏以建寅為正,物生色黑;殷以建丑為正,物牙色白;周以建子為正,物萌色赤。翰,白色馬也,易曰『白馬翰如』。」周禮巾車職「建大赤以朝」,大白以即戎,此則周以正色之旗以朝,先代之旗即戎。今魏用殷禮,變周之制,故建大白以朝,大赤即戎。}
 改大和歷曰景初歷。其春夏秋冬孟仲季月雖與正歲不同,至於郊祀、迎氣、礿祠、蒸甞、巡狩、蒐田、分至啟閉、班宣時令、中氣早晚、敬授民事,皆以正歲斗建為歷數之序。
 
 
 
 
 五月己巳,行還洛陽宮。己丑,大赦。六月戊申,京都地震。己亥,以尚書令陳矯為司徒,尚書右僕射衞臻為司空。丁未,分魏興之魏陽、錫郡之安富、上庸為上庸郡。省錫郡,以錫縣屬魏興郡。
 
 
有司奏:武皇帝撥亂反正,為魏太祖,樂用武始之舞。文皇帝應天受命,為魏高祖,樂用咸熈之舞。帝制作興治,為魏烈祖,樂用章武之舞。三祖之廟萬世不毀,其餘四廟親盡迭毀,如周后稷、文、武廟祧之制。
 \gezhu{孫盛曰:夫謚以表行,廟以存容,皆於旣沒然後著焉,所以原始要終,以示百世也。未有當年而逆制祖宗,未終而豫自尊顯。昔華樂以厚斂致譏,周人以豫凶違禮,魏之羣司,於是乎失正。}
 
 
 
 
 秋七月丁卯,司徒陳矯薨。孫權遣將朱然等二萬人圍江夏郡,荊州刺史胡質等擊之,然退走。初,權遣使浮海與高句驪通,欲襲遼東。遣幽州刺史毌丘儉率諸軍及鮮卑、烏丸屯遼東南界,璽書徵公孫淵。淵發兵反,儉進軍討之,會連雨十日,遼水大漲,詔儉引軍還。右北平烏丸單于寇婁敦、遼西烏丸都督王護留等居遼東,率部衆隨儉內附。己卯,詔遼東將吏士民為淵所脅略不得降者,一切赦之。辛卯,太白晝見。淵自儉還,遂自立為燕王,置百官,稱紹漢元年。
 
 
詔青、兖、幽、兾四州大作海船。九月,兾、兖、徐、豫四州民遇水,遣侍御史循行沒溺死亡及失財產者,在所開倉振救之。庚辰,皇后毛氏卒。冬十月丁未,月犯熒惑。癸丑,葬悼毛后于愍陵。乙卯,營洛陽南委粟山為圜丘。
 \gezhu{魏書載詔曰:「蓋帝王受命,莫不恭承天地以章神明,尊祀世統以昭功德,故先代之典旣著,則禘郊祖宗之制備也。昔漢氏之初,承秦滅學之後,采摭殘缺,以備郊祀,自甘泉后土、雍宮五畤,神祇兆位,多不見經,是以制度無常,一彼一此,四百餘年,廢無禘祀。古代之所更立者,遂有闕焉。曹氏繫世,出自有虞氏,今祀圜丘,以始祖帝舜配,號圜丘曰皇皇帝天;方丘所祭曰皇皇后地,以舜妃伊氏配;天郊所祭曰皇天之神,以太祖武皇帝配;地郊所祭曰皇地之祇,以武宣后配;宗祀皇考高祖文皇帝於明堂,以配上帝。」至晉泰始二年,并圜丘、方丘二至之祀於南北郊。}
 十二月壬子冬至,始祀。丁巳,分襄陽臨沮、宜城、旍陽、邔
 \gezhu{邔音其己反}
 四縣,置襄陽南部都尉。己未,有司奏文昭皇后立廟京都。分襄陽郡之鄀葉縣屬義陽郡。
 \gezhu{魏略曰:是歲,徙長安諸鐘簴、駱駝、銅人、承露盤。盤折,銅人重不可致,留于霸城。大發銅鑄作銅人二,號曰翁仲,列坐於司馬門外。又鑄黃龍、鳳皇各一,龍高四丈,鳳高三丈餘,置內殿前。起土山於芳林園西北陬,使公卿羣僚皆負土成山,樹松竹雜木善草於其上,捕山禽雜獸置其中。漢晉春秋曰:帝徙盤,盤折,聲聞數十里,金狄或泣,因留於霸城。魏略載司徒軍議掾河東董尋上書諫曰:「臣聞古之直士,盡言於國,不避死亡。故周昌比高祖於桀、紂,劉輔譬趙后於人婢。天生忠直,雖白刃沸湯,往而不顧者,誠為時主愛惜天下也。建安以來,野戰死亡,或門殫戶盡,雖有存者,遺孤老弱。若今宮室狹小,當廣大之,猶宜隨時,不妨農務,況乃作無益之物,黃龍、鳳皇,九龍、承露盤,土山、淵池,此皆聖明之所不興也,其功參倍於殿舍。三公九卿侍中尚書,天下至德,皆知非道而不敢言者,以陛下春秋方剛,心畏雷霆。今陛下旣尊羣臣,顯以冠冕,被以文繡,載以華輿,所以異於小人;而使穿方舉土,面目垢黑,沾體塗足,衣冠了鳥,毀國之光以崇無益,甚非謂也。孔子曰:『君使臣以禮,臣事君以忠。』無忠無禮,國何以立!故有君不君,臣不臣,上下不通,心懷鬱結,使陰陽不和,災害屢降,凶惡之徒因閒而起,誰當為陛下盡言是者乎?又誰當干萬乘以死為戲乎?臣知言出必死,而臣自比於牛之一毛,生旣無益,死亦何損?秉筆流涕,心與世辭。臣有八子,臣死之後,累陛下矣!」將奏,沐浴。旣通,帝曰:「董尋不畏死邪!」主者奏收尋,有詔勿問。後為貝丘令,清省得民心。}
 
 
二年春正月,詔太尉司馬宣王帥衆討遼東。
 \gezhu{干竇晉紀曰:帝問宣王:「度淵將何計以待君?」宣王對曰:「淵棄城預走,上計也;據遼水拒大軍,其次也;坐守襄平,此為成禽耳。」帝曰:「然則三者何出?」對曰:「唯明智審量彼我,乃預有所割棄,此旣非淵所及,又謂今往縣遠,不能持久,必先拒遼水,後守也。」帝曰:「往還幾日?」對曰:「往百日,攻百日;還百日,以六十日為休息,如此,一年足矣。」魏名臣奏載散騎常侍何曾表曰:「臣聞先王制法,必於全慎,故建官授任,則置假輔,陳師命將,則立監貳,宣命遣使,則設介副,臨敵交刃,則參御右,蓋以盡謀思之功,防安危之變也。是以在險當難,則權足相濟,隕缺不預,則手足相代,其為固防,至深至遠。及至漢氏,亦循舊章。韓信伐趙,張耳為貳;馬援討越,劉隆副軍。前世之迹,著在篇志。今懿奉辭誅罪,步騎數萬,道路迴阻,四千餘里,雖假天威,有征無戰,寇或潛遁,消散日月,命無常期。人非金石,遠慮詳備,誠宜有副。今北邊諸將及懿所督,皆為僚屬,名位不殊,素無定分,卒有變急,不相鎮攝。存不忘亡,聖達所戒,宜選大臣名將威重宿著者,盛其禮秩,遣詣懿軍,進同謀略,退為副佐。雖有萬一不虞之災,軍主有儲,則無患矣。」毌丘儉志記云,時以儉為宣王副也。}
 
 
二月癸卯,以大中大夫韓曁為司徒。癸丑,月犯心距星,又犯心中央大星。夏四月庚子,司徒韓曁薨。壬寅,分沛國蕭、相、竹邑、符離、蘄、銍、龍亢、山桑、洨、虹
 \gezhu{洨音胡交反。虹音絳。}
 十縣為汝陰郡。宋縣、陳郡苦縣皆屬譙郡。以沛、杼秋、公丘、彭城豐國、廣戚,并五縣為沛王國。庚戌,大赦。五月乙亥,月犯心距星,又犯中央大星。
 \gezhu{魏書載戊子詔曰:「昔漢高祖創業,光武中興,謀除殘暴,功昭四海,而墳陵崩頹,童兒牧豎踐蹈其上,非大魏尊崇所承代之意也。其表高祖、光武陵四面百步,不得使民耕牧樵採。」}
 六月,省漁陽郡之狐奴縣,復置安樂縣。
 
 
秋八月,燒當羌王芒中、注詣等叛,涼州刺史率諸郡攻討,斬注詣首。癸丑,有彗星見張宿。
 \gezhu{漢晉春秋曰:史官言於帝曰:「此周之分野也,洛邑惡之。」於是大脩禳禱之術以厭焉。魏書曰:九月,蜀陰平太守廖惇反,攻守善羌侯宕蕈營。雍州刺史郭淮遣廣魏太守王贇、南安太守游弈將兵討惇。淮上書:「贇、弈等分兵夾山東西,圍落賊表,破在旦夕。」帝曰:「兵勢惡離。」促詔淮勑弈諸別營非要處者,還令據便地。詔勑未到,弈軍為惇所破;贇為流矢所中死。}
 
 
 
 
 丙寅,司馬宣王圍公孫淵於襄平,大破之,傳淵首于京都,海東諸郡平。冬十一月,錄討淵功,太尉宣王以下增邑封爵各有差。初,帝議遣宣王討淵,發卒四萬人。議臣皆以為四萬兵多,役費難供。帝曰:「四千里征伐,雖云用奇,亦當任力,不當稍計役費。」遂以四萬人行。及宣王至遼東,霖雨不得時攻,羣臣或以為淵未可卒破,宜詔宣王還。帝曰:「司馬懿臨危制變,擒淵可計日待也。」卒皆如所策。
 
 
壬午,以司空衞臻為司徒,司隷校尉崔林為司空。閏月,月犯心中央大星。十二月乙丑,帝寢疾不豫。辛巳,立皇后。賜天下男子爵人二級,鰥寡孤獨穀。以燕王宇為大將軍,甲申免,以武衞將軍曹爽代之。
 \gezhu{漢晉春秋曰:帝以燕王宇為大將軍,使與領軍將軍夏侯獻、武衞將軍曹爽、屯騎校尉曹肈、驍騎將軍秦朗等對輔政。中書監劉放、令孫資久專權寵,為朗等素所不善,懼有後害,陰圖間之,而宇常在帝側,故未得有言。甲申,帝氣微,宇下殿呼曹肈有所議,未還,而帝少間,惟曹爽獨在。放知之,呼資與謀。資曰:「不可動也。」放曰:「俱入鼎鑊,何不可之有?」乃突前見帝,垂泣曰:「陛下氣微,若有不諱,將以天下付誰?」帝曰:「卿不聞用燕王耶?」放曰:「陛下忘先帝詔勑,藩王不得輔政。且陛下方病,而曹肈、秦朗等便與才人侍疾者言戲。燕王擁兵南面,不聽臣等入,此即豎刀、趙高也。今皇太子幼弱,未能統政,外有彊暴之寇,內有勞怨之民,陛下不遠慮存亡,而近係恩舊。委祖考之業,付二三凡士,寢疾數日,外內擁隔,社稷危殆,而己不知,此臣等所以痛心也。」帝得放言,大怒曰:「誰可任者?」放、資乃舉爽代宇,又白「宜詔司馬宣王使相參」,帝從之。放、資出,曹肈入,泣涕固諫,帝使肈勑停。肈出戶,放、資趨而往,復說止帝,帝又從其言。放曰:「宜為手詔。」帝曰:「我困篤,不能。」放即上牀,執帝手強作之,遂齎出,大言曰:「有詔免燕王宇等官,不得停省中。」於是宇、肈、獻、朗相與泣而歸第。}
 
 
 
 
 初,青龍三年中,壽春農民妻自言為天神所下,命為登女,當營衞帝室,蠲邪納福。飲人以水,及以洗創,或多愈者。於是立館後宮,下詔稱揚,甚見優寵。及帝疾,飲水無驗,於是殺焉。
 
 
三年春正月丁亥,太尉宣王還至河內,帝驛馬召到,引入卧內,執其手謂曰:「吾疾甚,以後事屬君,君其與爽輔少子。吾得見君,無所恨!」宣王頓首流涕。
 \gezhu{魏略曰:帝旣從劉放計,召司馬宣王,自力為詔,旣封,顧呼宮中常所給使者曰:「辟邪來!汝持我此詔授太尉也。」辟邪馳去。先是,燕王為帝畫計,以為關中事重,宜便道遣宣王從河內西還,事以施行。宣王得前詔,斯須復得後手筆,疑京師有變,乃馳到,入見帝。勞問訖,乃召齊、秦二王以示宣王,別指齊王謂宣王曰:「此是也,君諦視之,勿誤也!」又教齊王令前抱宣王頸。魏氏春秋曰:時太子芳年八歲,秦王九歲,在于御側。帝執宣王手,目太子曰:「死乃復可忍,朕忍死待君,君其與爽輔此。」宣王曰:「陛下不見先帝屬臣以陛下乎?」}
 即日,帝崩于嘉福殿,
 \gezhu{魏書曰:殯于九龍前殿。}
 時年三十六。
 \gezhu{臣松之桉:魏武以建安九年八月定鄴,文帝始納甄后,明帝應以十年生,計至此年正月,整三十四年耳。時改正朔,以故年十二月為今年正月,可彊名三十五年,不得三十六也。}
 癸丑,葬高平陵。
 
 
\gezhu{魏書曰:帝容止可觀,望之儼然。自在東宮,不交朝臣,不問政事,唯潛思書籍而已。即位之後,褒禮大臣,料簡功能,真偽不得相貿,務絕浮華譖毀之端,行師動衆,論決大事,謀臣將相咸服帝之大略。性特彊識,雖左右小臣官簿性行,名跡所履,及其父兄子弟,一經耳目,終不遺忘。含垢藏疾,容受直言,聽受吏民士庶上書,一月之中至數十百封,雖文辭鄙陋,猶覽省究竟,意無猒倦。孫盛曰:聞之長老,魏明帝天姿秀出,立髮垂地,口吃少言,而沉毅好斷。初,諸公受遺輔導,帝皆以方任處之,政自己出。而優禮大臣,開容善直,雖犯顏極諫,無所摧戮,其君人之量如此之偉也。然不思建德垂風,不固維城之基,至使大權偏據,社稷無衞,悲夫!}
 
 
 
 
 評曰:明帝沉毅斷識,任心而行,蓋有君人之至槩焉。于時百姓彫弊,四海分崩,不先聿脩顯祖,闡拓洪基,而遽追秦皇、漢武,宮館是營,格之遠猷,其殆疾乎!
 
 
\end{pinyinscope}