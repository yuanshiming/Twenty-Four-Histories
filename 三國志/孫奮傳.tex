\article{孫奮傳}
\begin{pinyinscope}
 
 
 孫奮字子揚,霸弟也,母曰仲姬。太元二年,立為齊王,居武昌。權薨,太傅諸葛恪不欲諸王處江濵兵馬之地,徙奮於豫章。奮怒,不從命,又數越法度。恪上牋諫曰:「帝王之尊,與天同位,是以家天下,臣父兄,四海之內,皆為臣妾。仇讎有善,不得不舉,親戚有惡,不得不誅,所以承天理物,先國後身,蓋聖人立制,百代不易之道也。昔漢初興,多王子弟至於太彊,輒為不軌,上則幾危社稷,下則骨肉相殘,其後懲戒,以為大諱。自光武以來,諸王有制,惟得自娛於宮內,不得臨民,干與政事,其與交通,皆有重禁,遂以全安,各保福祚。此則前世得失之驗也。近袁紹、劉表各有國土,土地非狹,人衆非弱,以適庶不分,遂滅其宗祀。此乃天下愚智,所共嗟痛。大行皇帝覽古戒今,防芽遏萌,慮於千載。是以寢疾之日,分遣諸王,各早就國,詔策殷勤,科禁嚴峻,其所戒勑,無所不至,誠欲上安宗廟,下全諸王,使百世相承,無凶國害家之悔也。大王宜上惟太伯順父之志,中念河間獻王、東海王彊恭敬之節,下當裁抑驕恣荒亂以為警戒。而聞頃至武昌以來,多違詔勑,不拘制度,擅發諸將兵治護宮室。又左右常從有罪過者,當以表聞,公付有司,而擅私殺,事不明白。大司馬呂岱親受先帝詔勑,輔導大王,旣不承用其言,令懷憂怖。華錡先帝近臣,忠良正直,其所陳道,當納用之,而聞怒錡,有收縛之語。又中書楊融,親受詔勑,所當恭肅,云『正自不聽禁,當如我何』?聞此之日,大小驚怪,莫不寒心。里語曰:『明鏡所以照形,古事所以知今。』大王宜深以魯王為戒,改易其行,戰戰兢兢,盡敬朝廷,如此則無求不得。若棄忘先帝法教,懷輕慢之心,臣下寧負大王,不敢負先帝遺詔,寧為大王所怨疾,豈敢忘尊主之威,而令詔勑不行於藩臣邪?此古今正義,大王所照知也。夫福來有由,禍來有漸,漸生不憂,將不可悔。向使魯王早納忠直之言,懷驚懼之慮,享祚無窮,豈有滅亡之禍哉?夫良藥苦口,惟疾者能甘之。忠言逆耳,惟達者能受之。今者恪等慺慺欲為大王除危殆於萌芽,廣福慶之基原,是以不自知言至,願蒙三思。」
 
 
 
 
 奮得牋懼,遂移南昌,游獵彌甚,官屬不堪命。及恪誅,奮下住蕪湖,欲至建業觀變。傅相謝慈等諫奮,奮殺之。
 
 
\gezhu{慈字孝宗,彭城人,見禮論,撰喪服圖及變除行於世。}
 坐廢為庶人,徙章安縣。太平三年,封為章安侯。
 \gezhu{江表傳載亮詔曰:「齊王奮前坐殺吏,廢為庶人,連有赦令,獨不見原,縱未宜復王,何以不侯?又諸孫兄弟作將,列在江渚,孤有兄獨爾云何?」有司奏可,就拜為侯。}
 
 
建衡二年,孫皓左夫人王氏卒。皓哀念過甚,朝夕哭臨,數月不出,由是民間或謂皓死,訛言奮與上虞侯奉當有立者。奮母仲姬墓在豫章,豫章太守張俊疑其或然,掃除墳塋。皓聞之,車裂俊,夷三族,誅奮及其五子,國除。
 \gezhu{江表傳曰:豫章吏十人乞代俊死,皓不聽。奮以此見疑,本在章安,徙還吳城禁錮,使男女不得通婚,或年三十四十不得嫁娶。奮上表乞自比禽獸,使男女自相配偶。皓大怒,遣察戰齎藥賜奮,奮不受藥,叩頭千下,曰:「老臣自將兒子治生求治,無豫國事,乞丐餘年。」皓不聽,父子皆飲藥死。臣松之案:建衡二年至奮之死,孫皓即位,尚猶未乆。若奮未被疑之前,兒女年二十左右,至奮死時,不得年三十四十也。若先已長大,自失時未婚娶,則不由皓之禁錮矣。此雖欲增皓之惡,然非實理。}
 
 
 
 
 評曰:孫登居心所存,足為茂美之德。慮、和並有好善之姿,規自砥礪,或短命早終,或不得其死,哀哉!霸以庶干適,奮不遵軌度,固取危亡之道也。然奮之誅夷,橫遇飛禍矣。
 
 
\end{pinyinscope}