\article{挹婁傳}
\begin{pinyinscope}
 
 
 挹婁在夫餘東北千餘里,濵大海,南與北沃沮接,未知其北所極。其土地多山險。其人形似夫餘,言語不與夫餘、句麗同。有五糓、牛、馬、麻布。人多勇力。無大君長,邑落各有大人。處山林之間,常穴居,大家深九梯,以多為好。土氣寒,劇於夫餘。其俗好養豬,食其肉,衣其皮。冬以豬膏塗身,厚數分,以禦風寒。夏則裸袒,以尺布隱其前後,以蔽形體。其人不絜,作溷在中央,人圍其表居。其弓長四尺,力如弩,矢用楛,長尺八寸,青石為鏃,古之肅慎氏之國也。善射,射人皆入。因矢施毒,人中皆死。出赤玉,好貂,今所謂挹婁貂是也。自漢已來,臣屬夫餘,夫餘責其租賦重,以黃初中叛之。夫餘數伐之,其人衆雖少,所在山險,鄰國人畏其弓矢,卒不能服也。其國便乘船寇盜,鄰國患之。東夷飲食類皆用俎豆,唯挹婁不,法俗最無綱紀也。
 
 
\end{pinyinscope}