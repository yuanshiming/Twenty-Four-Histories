\article{張紘傳}
\begin{pinyinscope}
 
 
 張紘字子綱,廣陵人。游學京都,
 
 
\gezhu{吳書曰:紘入太學,事博士韓宗,治京氏易、歐陽尚書,又於外黃從濮陽闓受韓詩及禮記、左氏春秋。}
 還本郡,舉茂才,公府辟,皆不就,
 \gezhu{吳書曰:大將軍何進、太尉朱儁、司空荀爽三府辟為掾,皆稱疾不就。}
 避難江東。孫策創業,遂委質焉。表為正議校尉,
 \gezhu{吳書曰:紘與張昭並與參謀,常令一人居守,一人從征討,後呂布襲取徐州,因為之牧,不欲令紘與策從事。追舉茂才,移書發遣紘。紘心惡布,恥為之屈。策亦重惜紘,欲以自輔。荅記不遣,曰:「海產明珠,所在為寶,楚雖有才,晉實用之。英偉君子,所游見珍,何必本州哉?」}
 從討丹楊。策身臨行陣,紘諫曰:「夫主將乃籌謨之所自出,三軍之所繫命也,不宜輕脫,自敵小寇。願麾下重天授之姿,副四海之望,無令國內上下危懼。」
 
 
建安四年,策遣紘奉章至許宮,留為侍御史。少府孔融等皆與親善。
 \gezhu{吳書曰:紘至,與在朝公卿及知舊述策才略絕異,平定三郡,風行草偃,加以忠敬款誠,乃心王室。時曹公為司空,欲加恩厚,以恱遠人,至乃優文襃崇,改號加封,辟紘為掾,舉高第,補侍御史,後以紘為九江太守。紘心戀舊恩,思還反命,以疾固辭。}
 曹公聞策薨,欲因喪伐吳。紘諫,以為乘人之喪,旣非古義,若其不克,成讎棄好,不如因而厚之。曹公從其言,即表權為討虜將軍,領會稽太守。曹公欲令紘輔權內附,出紘為會稽東部都尉。
 \gezhu{吳書曰:權初承統,春秋方富,太夫人以方外多難,深懷憂勞,數有優令辭謝,付屬以輔助之義。紘輒拜牋荅謝,思惟補察。每有異事密計及章表書記,與四方交結,常令紘與張昭草創撰作。紘以破虜有破走董卓,扶持漢室之勳;討逆平定江外,建立大業,宜有紀頌以昭公美。旣成,呈權,權省讀悲感,曰:「君真識孤家門閥閱也。」乃遣紘之部。或以紘本受北任,嫌其志趣不止於此,權不以介意。初,琅邪趙昱為廣陵太守,察紘孝廉,昱後為笮融所殺,紘甚傷憤,而力不能討。昱門戶絕滅,及紘在東部,遣主簿至琅邪設祭,并求親戚為之後,以書屬琅邪相臧宣,宣以趙宗中五歲男奉昱祀,權聞而嘉之。及討江夏,以東部少事,命紘居守,遙領所職。孔融遺紘書曰:「聞大軍西征,足下留鎮。不有居者,誰守社稷?深固折衝,亦大勳也。無乃李廣之氣,倉髮益怒,樂一當單于,以盡餘憤乎?南北並定,世將無事,孫叔投戈,絳、灌俎豆,亦在今日,但用離析,無緣會面,為愁歎耳。道直途清,相見豈復難哉?」權以紘有鎮守之勞,欲論功加賞。紘厚自挹損,不敢蒙寵,權不奪其志。每從容侍燕,微言密指,常有以規諷。江表傳曰:初,權於羣臣多呼其字,惟呼張昭曰張公,紘曰東部,所以重二人也。}
 
 
後權以紘為長史,從征合肥。
 \gezhu{吳書曰:合肥城乆不拔,紘進計曰:「古之圍城,開其一面,以疑衆心。今圍之甚密,攻之又急,誠懼并命戮力。死戰之寇,固難卒拔,及救未至,可小寬之,以觀其變。」議者不同。會救騎至,數至圍下,馳騁挑戰。}
 權率輕騎將往突敵,紘諫曰:「夫兵者凶器,戰者危事也。今麾下恃盛壯之氣,忽彊暴之虜,三軍之衆莫不寒心,雖斬將搴旗,威震敵場,此乃偏將之任,非主將之宜也。願抑賁、育之勇,懷霸王之計。」權納紘言而止。旣還,明年將復出軍,紘又諫曰:「自古帝王受命之君,雖有皇靈佐於上,文德播於下,亦賴武功以昭其勳。然而貴於時動,乃後為威耳。今麾下值四百之厄,有扶危之功,宜且隱息師徒,廣開播殖,任賢使能,務崇寬惠,順天命以行誅,可不勞而定也。」於是遂止不行。紘建計宜出都秣陵,權從之。
 \gezhu{江表傳曰:紘謂權曰:「秣陵,楚武王所置,名為金陵。地勢岡阜連石頭,訪問故老,云昔秦始皇東巡會稽經此縣,望氣者云金陵地形有王者都邑之氣,故掘斷連岡,改名秣陵。今處所具存,地有其氣,天之所命,宜為都邑。」權善其議,未能從也。後劉備之東,宿於秣陵,周觀地形,亦勸權都之。權曰:「智者意同。」遂都焉。獻帝春秋云:劉備至京,謂孫權曰:「吳去此數百里,即有驚急,赴救為難,將軍無意屯京乎?」權曰:「秣陵有小江百餘里,可以安大船,吾方理水軍,當移據之。」備曰:「蕪湖近濡須,亦佳也。」權曰:「吾欲圖徐州,宜近下也。」臣松之以為秣陵之與蕪湖,道里所校無幾,於北侵利便,亦有何異?而云欲闚徐州,貪秣陵近下,非其理也。諸書皆云劉備勸都秣陵,而此獨云權自欲都之,又為虛錯。}
 令還吳迎家,道病卒。臨困,授子靖留牋曰:「自古有國有家者,咸欲脩德政以比隆盛世,至於其治,多不馨香。非無忠臣賢佐,闇於治體也,由主不勝其情,弗能用耳。夫人情憚難而趨易,好同而惡異,與治道相反。傳曰『從善如登,從惡如崩』,言善之難也。人君承奕世之基,據自然之勢,操八柄之威,甘易同之歡,
 \gezhu{周禮太宰職曰:以八柄詔王馭羣臣。一曰爵,以馭其貴。二曰祿,以馭其富。三曰予,以馭其幸。四曰置,以馭其行。五曰生,以馭其福。六曰奪,以馭其貧。七曰廢,以馭其罪。八曰誅,以馭其過。}
 無假取於人;而忠臣挾難進之術,吐逆耳之言,其不合也,不亦宜乎!離則有釁,巧辯緣間,眩於小忠,戀於恩愛,賢愚雜錯,長幼失叙,其所由來,情亂之也。故明君寤之,求賢如饑渴,受諫而不厭,抑情損欲,以義割恩,上無偏謬之授,下無希兾之望。宜加三思,含垢藏疾,以成仁覆之大。」時年六十卒。權省書流涕。
 
 
紘著詩賦銘誄十餘篇。
 \gezhu{吳書曰:紘見柟榴枕,愛其文,為作賦。陳琳在北見之,以示人曰:「此吾鄉里張子綱所作也。」後紘見陳琳作武庫賦、應機論,與琳書深歎美之。琳荅曰:「自僕在河北,與天下隔,此間率少於文章,易為雄伯,故使僕受此過差之譚,非其實也。今景興在此,足下與子布在彼,所謂小巫見大巫,神氣盡矣。」紘旣好文學,又善楷篆,嘗與孔融書自耆。融遺紘書曰:「前勞手筆,多篆書。每舉篇見字,欣然獨笑,如復覩其人也。」}
 子玄,官至南郡太守、尚書。
 \gezhu{江表傳曰:玄清介有高行,而才不及紘。}
 玄子尚,
 \gezhu{江表傳曰「稱尚有俊才」。}
 孫皓時為侍郎,以言語辯捷見知,擢為侍中、中書令。皓使尚鼓琴,尚對曰:「素不能。」勑使學之。後晏言次說琴之精妙,尚因道「晉平公使師曠作清角,曠言吾君德簿,不足以聽之。」皓意謂尚以斯喻己,不恱。後積他事下獄,皆追以此為詰,
 \gezhu{環氏吳紀曰:皓嘗問:「詩云『汎彼柏舟』,惟柏中舟乎?」尚對曰:「詩言『檜楫松舟』,則松亦中舟也。」又問:「鳥之大者惟鶴,小者惟雀乎?」尚對曰:「大者有禿鶖,小者有鷦鷯。」皓性忌勝己,而尚談論每出其表,積以致恨。後問:「孤飲酒以方誰?」尚對曰:「陛下有百觚之量。」皓云:「尚知孔丘之不王,而以孤方之!」因此發怒收尚。尚書岑昏率公卿已下百餘人,詣宮叩頭請,尚罪得減死。}
 送建安作船。乆之,又就加誅。
 
 
 
 
 初,紘同郡秦松字文表,陳端字子正,並與紘見待於孫策,參與謀謨。各早卒。
 
 
\end{pinyinscope}