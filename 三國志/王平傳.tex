\article{王平傳}
\begin{pinyinscope}
 
 
 王平字子均,巴西宕渠人也。本養外家何氏,後復姓王。隨杜濩、朴胡詣洛陽,假校尉,從曹公征漢中,因降先主,拜牙門將、裨將軍。建興六年,屬參軍馬謖先鋒。謖舍水上山,舉措煩擾,平連規諫謖,謖不能用,大敗於街亭。衆盡星散,惟平所領千人,鳴鼓自持,魏將張郃疑其伏兵,不往偪也。於是平徐徐收合諸營遺迸,率將士而還。丞相亮旣誅馬謖及將軍張休、李盛,奪將軍黃襲等兵,平特見崇顯,加拜參軍,統五部兼當營事,進位討寇將軍,封亭侯。九年,亮圍祁山,平別守南圍。魏大將軍司馬宣王攻亮,張郃攻平,平堅守不動,郃不能克。十二年,亮卒於武功,軍退還,魏延作亂,一戰而敗,平之功也。遷後典軍、安漢將軍,副車騎將軍吳壹住漢中,又領漢中太守。十五年,進封安漢侯,代壹督漢中。延熈元年,大將軍蔣琬住沔陽,平更為前護軍,署琬府事。六年,琬還住涪,拜平前監軍、鎮北大將軍,統漢中。
 
 
 
 
 七年春,魏大將軍曹爽率步騎十餘萬向漢川,前鋒已在駱谷。時漢中守兵不滿三萬,諸將大驚。或曰:「今力不足以拒敵,聽當固守漢、樂二城,遇賊令入,比爾間,涪軍足得救關。」平曰:「不然。漢中去涪垂千里。賊若得關,便為禍也。今宜先遣劉護軍、杜參軍據興勢,平為後拒;若賊分向黃金,平率千人下自臨之,比爾間,涪軍行至,此計之上也。」惟護軍劉敏與平意同,即便施行。涪諸軍及大將軍費禕自成都相繼而至,魏軍退還,如平本策。是時,鄧芝在東,馬忠在南,平在北境,咸著名迹。
 
 
 
 
 平生長戎旅,手不能書,而所識不過十字,而口授作書,皆有意理。使人讀史、漢諸紀傳,聽之,備知其大義,往往論說不失其指。遵履法度,言不戲謔,從朝至夕,端坐徹日,㦎無武將之體,然性狹侵疑,為人自輕,以此為損焉。十一年卒,子訓嗣。
 
 
 
 
 初,平同郡漢昌句扶忠勇寬厚,
 
 
\gezhu{句古侯反}
 數有戰功,功名爵位亞平,官至左將軍,封宕渠侯。
 \gezhu{華陽國志曰:後張翼、廖化並為大將軍,時人語曰:「前有王、句,後有張、廖。」}
 
 
\end{pinyinscope}