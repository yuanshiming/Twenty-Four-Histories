\article{顧雍傳}
\begin{pinyinscope}
 
 
 顧雍字元歎,吳郡吳人也。
 
 
\gezhu{吳錄曰:雍曾祖父奉,字季鴻,潁川太守。}
 蔡伯喈從朔方還,嘗避怨於吳,雍從學琴書。
 \gezhu{江表傳曰:雍從伯喈學,專一清靜,敏而易教。伯喈貴異之,謂曰:「卿必成致,今以吾名與卿。」故雍與伯喈同名,由此也。吳錄曰:雍字元歎,言為蔡雍之所歎,因以為字焉。}
 州郡表薦,弱冠為合肥長,後轉在婁、曲阿、上虞,皆有治迹。孫權領會稽太守,不之郡,以雍為丞,行太守事,討除寇賊,郡界寧靜,吏民歸服。數年,入為左司馬。權為吳王,累遷大理奉常,領尚書令,封陽遂鄉侯,拜侯還寺,而家人不知,後聞乃驚。
 
 
黃武四年,迎母於吳。旣至,權臨賀之,親拜其母於庭,公卿大臣畢會,後太子又往慶焉。雍為人不飲酒,寡言語,舉動時當。權嘗歎曰:「顧君不言,言必有中。」至飲宴歡樂之際,左右恐有酒失而雍必見之,是以不敢肆情。權亦曰:「顧公在坐,使人不樂。」其見憚如此。是歲,改為太常,進封醴陵侯,代孫邵為丞相,平尚書事。其所選用文武將吏各隨能所任,心無適莫。時訪逮民閒,及政職所宜,輒密以聞。若見納用,則歸之於上,不用,終不宣泄。權以此重之。然於公朝有所陳及,辭色雖順而所執者正。權嘗咨問得失,張昭因陳聽采聞,頗以法令太稠,刑罰微重,宜有所蠲損。權默然,顧問雍曰:「君以為何如?」雍對曰:「臣之所聞,亦如昭所陳。」於是權乃議獄輕刑。
 \gezhu{江表傳曰:灌常令中書郎詣雍,有所咨訪。若合雍意,事可施行,即與相反覆,究而論之,為設酒食。如不合意,雍即正色改容,默然不言,無所施設,即退告。權曰:「顧公歡恱,是事合宜也;其不言者,是事未平也,孤當重思之。」其見敬信如此。江邊諸將,各欲立功自效,多陳便宜,有所掩襲。權以訪雍,雍曰:「臣聞兵法戒於小利,此等所陳,欲邀功名而為其身,非為國也,陛下宜禁制。苟不足以曜威損敵,所不宜聽也。」權從之。軍國得失,行事可不,自非面見,口未嘗言之。}
 乆之,呂壹、秦博為中書,典校諸官府及州郡文書。壹等因此漸作威福,遂造作榷酤障管之利,舉罪糾姧,纖介必聞,重以深案醜誣,毀短大臣,排陷無辜,雍等皆見舉白,用被譴讓。後壹姦罪發露,收繫廷尉。雍往斷獄,壹以囚見,雍和顏色,問其辭狀,臨出,又謂壹曰:「君意得無欲有所道?」壹叩頭無言。時尚書郎懷叙面詈辱壹,雍責叙曰:「官有正法,何至於此!」
 \gezhu{江表傳曰:權嫁從女,女顧氏甥,故請雍父子及孫譚,譚時為選曹尚書,見任貴重。是日,權極歡。譚醉酒,三起舞,舞不知止。雍內怒之。明日,召譚,訶責之曰:「君王以含垢為德,臣下以恭謹為節。昔蕭何、吳漢並有大功,何每見高帝,似不能言;漢奉光武,亦信恪勤。汝之於國,寧有汗馬之勞,可書之事邪?但階門戶之資,遂見寵任耳,何有舞不復知止?雖為酒後,亦由恃恩忘敬,謙虛不足。損吾家者必爾也。」因背向壁卧,譚立過一時,乃見遣。徐衆評曰:雍不以呂壹見毀之故,而和顏恱色,誠長者矣。然開引其意,問所欲道,此非也。壹姦險亂法,毀傷忠賢,吳國寒心,自太子登、陸遜已下,切諫不能得,是以潘濬欲因會同手劒之,以除國患,疾惡忠主,義形於色,而今乃發起令言。若壹稱枉邪,不申理,則非錄獄本旨;若承辭而奏之,吳主儻以敬丞相所言,而復原宥,伯言、承明不當悲慨哉!懷叙本無私恨,無所為嫌,故詈辱之,疾惡意耳,惡不仁者,其為仁也。季武子死,曾點倚其門而歌;子晳創發,子產催令自裁。以此言之,雍不當責懷叙也。}
 
 
雍為相十九年,年七十六,赤烏六年卒。初疾微時,權令醫趙泉視之,拜其少子濟為騎都尉。雍聞,悲曰:「泉善別死生,吾必不起,故上欲及吾目見濟拜也。」權素服臨弔,謚曰肅侯。長子邵早卒,次子裕有篤疾,少子濟嗣,無後,絕。永安元年,詔曰:「故丞相雍,至德忠賢,輔國以禮,而侯統廢絕,朕甚愍之。其以雍次子裕襲爵為醴陵侯,以明著舊勳。」
 \gezhu{吳錄曰:裕一名穆,終宜都太守。裕子榮。晉書曰:榮字彥先,為東南名士,仕吳為黃門郎,在晉歷顯位。元帝初鎮江東,以榮為軍司馬,禮遇甚重。卒,表贈侍中、驃騎將軍、儀同三司。榮兄子禺,字孟著,少有名望,為散騎侍郎,早卒。吳書曰:雍母弟徽,字子歎,少游學,有脣吻。孫權統事,聞徽有才辯,召署主簿。嘗近出行,見營軍將一男子至巿行刑,問之何罪,云盜百錢,徽語使住。須臾,馳詣闕陳啟:「方今畜養士衆以圖北虜,視此兵丁壯健兒,且所盜少,愚乞哀原。」權許而嘉之。轉東曹掾。或傳曹公欲東,權謂徽曰:「卿孤腹心,今傳孟德懷異意,莫足使揣之,卿為吾行。」拜輔義都尉,到北與曹公相見。公具問境內消息,徽應對婉順,因說江東大豐,山藪宿惡,皆慕化為善,義出作兵。公笑曰:「孤與孫將軍一結婚姻,共輔漢室,義如一家,君何為道此?」徽曰:「正以明公與主將義固磐石,休戚共之,必欲知江表消息,是以及耳。」公厚待遣還。權問定云何,徽曰:「敵國隱情,卒難探察。然徽潛采聽,方與袁譚交爭,未有他意。」乃拜徽巴東太守,欲大用之,會卒。子裕,字季則,少知名,位至鎮東將軍。雍族人悌,字子通,以孝悌廉正聞於鄉黨。年十五為郡吏,除郎中,稍遷偏將軍。權末年,嫡庶不分,悌數與驃騎將軍朱據共陳禍福,言辭切直,朝廷憚之。待妻有禮,常夜入晨出,希見其面。嘗疾篤,妻出省之,悌命左右扶起,冠幘加襲,起對,趨令妻還,其貞潔不瀆如此。悌父向歷四縣令,年老致仕,悌每得父書,常灑掃,整衣服,更設几筵,舒書其上,拜跪讀之,每句應諾,畢,復再拜。若父有疾耗之問至,則臨書垂涕,聲語哽咽。父以壽終,悌飲漿不入口五日。權為作布衣一襲,皆摩絮著之,強令悌釋服。悌雖以公議自割,猶以不見父喪,常畫壁作棺柩象,設神座於下,每對之哭泣,服未闋而卒。悌四子:彥、禮、謙、祕。秘,晉交州刺史。祕子衆,尚書僕射。}
 
 
\end{pinyinscope}