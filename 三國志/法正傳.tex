\article{法正傳}
\begin{pinyinscope}
 
 
 法正字孝直,右扶風郿人也。祖父真,有清節高名。
 
 
\gezhu{三輔決錄注曰:真字高卿,少明五經,兼通讖緯,學無常師,名有高才。常幅巾見扶風守,守曰:「哀公雖不肖,猶臣仲尼,柳下惠不去父母之邦,欲相屈為功曹何如?」真曰:「以明府見待有禮,故四時朝覲,若欲吏使之,真將在北山之北南山之南矣。」扶風守遂不敢以為吏。初,真年未弱冠,父在南郡,步往候父,已欲去,父留之待正旦,使觀朝吏會。會者數百人,真於䆫中闚其與父語。畢,問真「孰賢」?真曰:「曹掾胡廣有公卿之量。」其後廣果歷九卿三公之位,世以服真之知人。前後徵辟,皆不就,友人郭正等美之,號曰玄德先生。年八十九,中平五年卒。正父衍,字季謀,司徒掾、廷尉左監。}
 建安初,天下饑荒,正與同郡孟達俱入蜀依劉璋,乆之為新都令,後召署軍議校尉。旣不任用,又為其州邑俱僑客者所謗無行,志意不得。益州別駕張松與正相善,忖璋不足與有為,常竊歎息。松於荊州見曹公還,勸璋絕曹公而自結先主。璋曰:「誰可使者?」松乃舉正,正辭讓,不得已而往。正旣還,為松稱說先主有雄略,密謀協規,願共戴奉,而未有緣。後因璋聞曹公欲遣將征張魯之有懼心也,松遂說璋宜迎先主,使之討魯,復令正銜命。正旣宣旨,陰獻策於先主曰:「以明將軍之英才,乘劉牧之懦弱;張松,州之股肱,以響應於內;然後資益州之殷富,馮天府之險阻,以此成業,猶反掌也。」先主然之,泝江而西,與璋會涪。北至葭萌,南還取璋。
 
 
鄭度說璋曰:
 \gezhu{華陽國志曰:度,廣漢人,為州從事。}
 「左將軍縣軍襲我,兵不滿萬,士衆未附,野穀是資,軍無輜重。其計莫若盡驅巴西、梓潼民內涪水以西,其倉廩野穀一皆燒除,高壘深溝,靜以待之。彼至,請戰,勿許,乆無所資,不過百日,必將自走。走而擊之,則必禽耳。」先主聞而惡之,以問正。正曰:「終不能用,無可憂也。」璋果如正言,謂其羣下曰:「吾聞拒敵以安民,未聞動民以避敵也。」於是黜度,不用其計。及軍圍雒城,正牋與璋曰:「正受性無術,盟好違損,懼左右不明本末,必並歸咎,蒙耻沒身,辱及執事,是以捐身於外,不敢反命。恐聖聽穢惡其聲,故中間不有牋敬,顧念宿遇,瞻望悢悢。然惟前後披露腹心,自從始初以至於終,實不藏情,有所不盡,但愚闇策薄,精誠不感,以致於此耳。今國事已危,禍害在速,雖捐放於外,言足憎尤,猶貪極所懷,以盡餘忠。明將軍本心,正之所知也,實為區區不欲失左將軍之意,而卒至於是者,左右不達英雄從事之道,謂可違信黷誓,而以意氣相致,日月相選,趨求順耳恱目,隨阿遂指,不圖遠慮為國深計故也。事變旣成,又不量彊弱之勢,以為左將軍縣遠之衆,糧穀無儲,欲得以多擊少,曠日相持。而從關至此,所歷輒破,離宮別屯,日自零落。雒下雖有萬兵,皆壞陣之卒,破軍之將,若欲爭一旦之戰,則兵將勢力實不相當。各欲遠期計糧者,今此營守已固,穀米已積,而明將軍土地日削,百姓日困,敵對遂多,所供遠曠。愚意計之,謂必先竭,將不復以持乆也。空爾相守,猶不相堪,今張益德數萬之衆已定巴東,入犍為界,分平資中、德陽,三道並侵,將何以禦之?本為明將軍計者,必謂此軍縣遠無糧,饋運不及,兵少無繼。今荊州道通,衆數十倍,加孫車騎遣弟及李異、甘寧等為其後繼。若爭客主之勢,以土地相勝者,今此全有巴東,廣漢、犍為過半已定,巴西一郡,復非明將軍之有也。計益州所仰惟蜀,蜀亦破壞;三分亡二,吏民疲困,思為亂者十戶而八;若敵遠則百姓不能堪役,敵近則一旦易主矣。廣漢諸縣,是明比也。又魚復與關頭實為益州福禍之門,今二門悉開,堅城皆下,諸軍並破,兵將俱盡,而敵家數道並進,已入心腹,坐守都、雒,存亡之勢,昭然可見。斯乃大略,其外較耳,其餘屈曲,難以辭極也。以正下愚,猶知此事不可復成,况明將軍左右明智用謀之士,豈當不見此數哉?旦夕偷幸,求容取媚,不慮遠圖,莫肯盡心獻良計耳。若事窮勢迫,將各索生,求濟門戶,展轉反覆,與今計異,不為明將軍盡死難也。而尊門猶當受其憂。正雖獲不忠之謗,然心自謂不負聖德,顧惟分義,實竊痛心。左將軍從本舉來,舊心依依,實無薄意。愚以為可圖變化,以保尊門。」
 
 
十九年,進圍成都,璋蜀郡太守許靖將踰城降,事覺,不果。璋以危亡在近,故不誅靖。璋旣稽服,先主以此薄靖不用也。正說曰:「天下有獲虛譽而無其實者,許靖是也。然今主公始創大業,天下之人不可戶說,靖之浮稱,播流四海,若其不禮,天下之人以是謂主公為賤賢也。宜加敬重,以眩遠近,追昔燕王之待郭隗。」先主於是乃厚待靖。
 \gezhu{孫盛曰:夫禮賢崇德,為邦之要道,封墓式閭,先王之令軌,故必以體行英邈,高義蓋世,然後可以延視四海,振服羣黎。苟非其人,道不虛行。靖處室則友于不穆,出身則受位非所,語信則夷險易心,論識則殆為釁首,安在其可寵先而有以感致者乎?若乃浮虛是崇,偷薄斯榮,則秉直杖義之士將何以禮之?正務眩惑之術,違貴尚之風,譬之郭隗,非其倫矣。臣松之以為郭隗非賢,猶以權計蒙寵,況文休名聲夙著,天下謂之英偉,雖末年有瑕,而事不彰徹,若不加禮,何以釋遠近之惑乎?法正以靖方隗,未為不當,而盛以封墓式閭為難,何其迃哉!然則燕昭亦非,豈唯劉翁?至於友于不穆,失由子將,尋蔣濟之論,知非文休之尤。盛又譏其受任非所,將謂仕於董卓。卓初秉政,顯擢賢俊,受其策爵者森然皆是。文休為選官,在卓未至之前,後遷中丞,不為超越。以此為貶,則荀爽、陳紀之儔皆應擯棄於世矣。}
 
 
以正為蜀郡太守、揚武將軍,外統都畿,內為謀主。一飡之德,睚眦之怨,無不報復,擅殺毀傷己者數人。或謂諸葛亮曰:「法正於蜀郡太縱橫,將軍宜啟主公,抑其威福。」亮荅曰:「主公之在公安也,北畏曹公之彊,東憚孫權之逼,近則懼孫夫人生變於肘腋之下;當斯之時,進退狼跋,法孝直為之輔翼,令翻然翱翔,不可復制,如何禁止法正使不得行其意邪!」初,孫權以妹妻先主,妹才捷剛猛,有諸兄之風,侍婢百餘人,皆親執刀侍立,先主每入,衷心常凜凜;亮又知先主雅愛信正,故言如此。
 \gezhu{孫盛曰:夫威福自下,亡家害國之道,刑縱於寵,毀政亂理之源,安可以功臣而極其陵肆,嬖幸而藉其國柄者哉?故顛頡雖勤,不免違命之刑,楊干雖親,猶加亂行之戮,夫豈不愛,王憲故也。諸葛氏之言,於是乎失政刑矣。}
 
 
二十二年,正說先主曰:「曹操一舉而降張魯,定漢中,不因此勢以圖巴、蜀,而留夏侯淵、張郃屯守,身遽北還,此非其智不逮而力不足也,必將內有憂偪故耳。今策淵、郃才略,不勝國之將帥,舉衆往討,則必可克之,克之日,廣農積穀,觀釁伺隙,上可以傾覆寇敵,尊獎王室,中可以蠶食雍、涼,廣拓境土,下可以固守要害,為持乆之計。此蓋天以與我,時不可失也。」先主善其策,乃率諸將進兵漢中,正亦從行。二十四年,先主自陽平南渡沔水,緣山稍前,於定軍、興勢作營。淵將兵來爭其地。正曰:「可擊矣。」先主命黃忠乘高鼓譟攻之,大破淵軍,淵等授首。曹公西征,聞正之策,曰:「吾故知玄德不辨有此,必為人所教也。」
 \gezhu{臣松之以為蜀與漢中,其由脣齒也。劉主之智,豈不及此?將計略未展,正先發之耳。夫聽用嘉謀以成功業,霸王之主,誰不皆然?魏武以為人所教,亦豈劣哉!此蓋耻恨之餘辭,非測實之當言也。}
 
 
先主立為漢中王,以正為尚書令、護軍將軍。明年卒,時年四十五。先主為之流涕者累日。謚曰翼侯。賜子邈爵關內侯,官至奉車都尉、漢陽太守。諸葛亮與正,雖好尚不同,以公義相取。亮每奇正智術。先主旣即尊號,將東征孫權以復關羽之耻,羣臣多諫,一不從。章武二年,大軍敗績,還住白帝。亮歎曰:「法孝直若在,則能制主上,令不東行;就復東行,必不傾危矣。」
 \gezhu{先主與曹公爭,勢有不便,宜退,而先主大怒不肯退,無敢諫者。矢下如雨,正乃往當先主前,先主云:「孝直避箭。」正曰:「明公親當矢石,況小人乎?」先主乃曰:「孝直,吾與汝俱去。」遂退。}
 
 
 
 
 評曰:龐統雅好人流,經學思謀,于時荊、楚謂之高俊。法正著見成敗,有奇畫策筭,然不以德素稱也。儗之魏臣,統其荀彧之仲叔,正其程、郭之儔儷邪?
 
 
\end{pinyinscope}