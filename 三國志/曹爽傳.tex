\article{曹爽傳}
\begin{pinyinscope}

爽字昭伯,少以宗室謹重,明帝在東宮,甚親愛之。及即位,為散騎侍郎,累遷城門校尉,加散騎常侍,轉武衞將軍,寵待有殊。帝寢疾,乃引爽入卧內,拜大將軍,假節鉞,都督中外諸軍事,錄尚書事,與太尉司馬宣王並受遺詔輔少主。明帝崩,齊王即位,加爽侍中,改封武安侯,邑萬二千戶,賜劒履上殿,入朝不趨,贊拜不名。丁謐畫策,使爽白天子,發詔轉宣王為太傅,外以名號尊之,內欲令尚書奏事,先來由己,得制其輕重也。
\gezhu{魏書曰:爽使弟羲為表曰:「臣亡父真,奉事三朝,入備冢宰,出為上將。先帝以臣肺腑遺緒,獎飭拔擢,典兵禁省,進無忠恪積累之行,退無羔羊自公之節。先帝聖體不豫,臣雖奔走,侍疾嘗藥,曾無精誠翼日之應,猥與太尉懿俱受遺詔,且慙且懼,靡所厎告。臣聞虞舜序賢,以稷、契為先,成湯襃功,以伊、呂為首,審選博舉,優劣得所,斯誠輔世長民之大經,錄勳報功之令典,自古以來,未之或闕。今臣虛闇,位冠朝首,顧惟越次,中心愧惕,敢竭愚情,陳寫至實。夫天下之達道者三,謂德、爵、齒也。懿本以高明中正,處上司之位,名足鎮衆,義足率下,一也。包懷大略,允文允武,仍立征伐之勳,遐邇歸功,二也。萬里旋斾,親受遺詔,翼亮皇家,內外所向,三也。加之耆艾,紀綱邦國,體練朝政;論德則過於吉甫、樊仲;課功則踰於方叔、召虎:凡此數者,懿實兼之。臣抱空名而處其右,天下之人將謂臣以宗室見私,知進而不知退。陛下岐嶷,克明克類,如有以察臣之言,臣以為宜以懿為太傅、大司馬,上昭陛下進賢之明,中顯懿身文武之實,下使愚臣免於謗誚。」於是帝使中書監劉放令孫資為詔曰:「昔吳漢佐光武,有征定四方之功,為大司馬,名稱于今。太尉體履正直,功蓋海內,先帝本以前後欲更其位者,輒不彌乆,是以遲遲不施行耳。今大將軍薦太尉宜為大司馬,旣合先帝本旨,又放推讓,進德尚勳,乃欲明賢良、辯等列、順長少也。雖旦、奭之屬,宗師呂望,念在引領以處其下,何以過哉!朕甚嘉焉。朕惟先帝固知君子樂天知命,纖介細疑,不足為忌,當顧栢人彭亡之文,故用低佪,有意未遂耳!斯亦先帝敬重大臣,恩愛深厚之至也。昔成王建保傅之官,近漢顯宗以鄧禹為太傅,皆所以優崇儁乂,必有尊也。其以太尉為太傅。」}
爽弟羲為中領軍,訓武衞將軍,彥散騎常侍侍講,其餘諸弟皆以列侯侍從,出入禁闥,貴寵莫盛焉。南陽何晏、鄧颺、李勝、沛國丁謐、東平畢軌咸有聲名,進趣於時,明帝以其浮華,皆抑黜之;及爽秉政,乃復進叙,任為腹心。颺等欲令爽立威名於天下,勸使伐蜀,爽從其言,宣王止之不能禁。正始五年,爽乃西至長安,大發卒六七萬人,從駱谷入。是時,關中及氐、羌轉輸不能供,牛馬騾驢多死,民夷號泣道路。入谷行數百里,賊因山為固,兵不得進。爽參軍楊偉為爽陳形勢,宜急還,不然將敗。
\gezhu{世語曰:偉字世英,馮翊人。明帝治宮室,偉諫曰:「今作宮室,斬伐生民墓上松柏,毀壞碑獸石柱,辜及亡人,傷孝子心,不可以為後世之法則。」}
颺與偉爭於爽前,偉曰:「颺、勝將敗國家事,可斬也。」爽不恱,乃引軍還。
\gezhu{漢晉春秋曰:司馬宣王謂夏侯玄曰:「春秋責大德重,昔武皇帝再入漢中,幾至大敗,君所知也。今興平路勢至險,蜀已先據;若進不獲戰,退見徼絕,覆軍必矣。將何以任其責!」玄懼,言於爽,引軍退。費禕進兵據三嶺以截爽,爽爭嶮苦戰,僅乃得過。所發牛馬運轉者,死失略盡,羌、胡怨歎,而關右悉虛耗矣。}


初,爽以宣王年德並高,恒父事之,不敢專行。及晏等進用,咸共推戴,說爽以權重不宜委之於人。乃以晏、颺、謐為尚書,晏典選舉,軌司隷校尉,勝河南尹,諸事希復由宣王。宣王遂稱疾避爽。
\gezhu{初,宣王以爽魏之肺腑,每推先之,爽以宣王名重,亦引身卑下,當時稱焉。丁謐、畢軌等旣進用,數言於爽曰:「宣王有大志而甚得民心,不可以推誠委之。」由是爽恒猜防焉。禮貌雖存,而諸所興造,皆不復由宣王。宣王力不能爭,且懼其禍,故避之。}
晏等專政,共分割洛陽、野王典農部桑田數百頃,及壞湯沐地以為產業,承勢竊取官物,因緣求欲州郡。有司望風,莫敢忤旨。晏等與廷尉盧毓素有不平,因毓吏微過,深文致毓法,使主者先收毓印綬,然後奏聞。其作威如此。爽飲食車服擬於乘輿,尚方珍玩充牣其家,妻妾盈後庭,又私取先帝才人七八人,及將吏、師工、鼓吹、良家子女三十三人,皆以為伎樂。詐作詔書,發才人五十七人送鄴臺,使先帝倢伃教習為伎。擅取太樂樂器、武庫禁兵。作窟室,綺疏四周,數與晏等會其中,飲酒作樂。羲深以為大憂,數諫止之。又著書三篇,陳驕淫盈溢之致禍敗,辭旨甚切,不敢斥爽,託戒諸弟以示爽。爽知其為己發也,甚不恱。羲或時以諫喻不納,涕泣而起。宣王密為之備。九年冬,李勝出為荊州刺史,往詣宣王。宣王稱疾困篤,示以羸形。勝不能覺,謂之信然。
\gezhu{魏末傳曰:爽等令勝辭宣王,并伺察焉。宣王見勝,勝自陳無他功勞,橫蒙時恩,當為本州,詣閤拜辭,不悟加恩,得蒙引見。宣王令兩婢侍邊,持衣,衣落;復上指口,言渴求飲,婢進粥,宣王持杯飲粥,粥皆流出沾胷。勝愍然,為之涕泣,謂宣王曰:「今主上尚幼,天下恃賴明公。然衆情謂明公方舊風疾發,何意尊體乃爾!」宣王徐更寬言,才令氣息相屬,說:「年老沈疾,死在旦夕。君當屈并州,并州近胡,好善為之,恐不復相見,如何!」勝曰:「當還忝本州,非并州也。」宣王乃復陽為昏謬,曰:「君方到并州,努力自愛!」錯亂其辭,狀如荒語。勝復曰:「當忝荊州,非并州也。」宣王乃若微悟者,謂勝曰:「懿年老,意荒忽,不解君言。今還為本州刺史,盛德壯烈,好建功勳。今當與君別,自顧氣力轉微,後必不更會,因欲自力,設薄主人,生死共別。令師、昭兄弟結君為友,不可相舍去,副懿區區之心。」因流涕哽咽。勝亦長歎,荅曰:「輒當承教,須待勑命。」勝辭出,與爽等相見,說:「太傅語言錯誤,口不攝杯,指南為北。又云吾當作并州,吾荅言當還為荊州,非并州也。徐徐與語,有識人時,乃知當還為荊州耳。又欲設主人祖送。不可舍去,宜須待之。」更向爽等垂淚云:「太傅患不可復濟,令人愴然。」}


十年正月,車駕朝高平陵,爽兄弟皆從。
\gezhu{世語曰:爽兄弟先是數俱出游,桓範謂曰:「總萬機,典禁兵,不宜並出,若有閉城門,誰復內人者?」爽曰:「誰敢爾邪!」由此不復並行。至是乃盡出也。}
宣王部勒兵馬,先據武庫,遂出屯洛水浮橋。奏爽曰:「臣昔從遼東還,先帝詔陛下、秦王及臣升御牀,把臣臂,深以後事為念。臣言『二祖亦屬臣以後事為念,此自陛下所見,無所憂苦;萬一有不如意,臣當以死奉明詔』。黃門令董箕等,才人侍疾者,皆所聞知。今大將軍爽背棄顧命,敗亂國典,內則僭擬,外專威權;破壞諸營,盡據禁兵,羣官要職,皆置所親;殿中宿衞,歷世舊人皆復斥出,欲置新人以樹私計;根據槃牙,縱恣日甚。外旣如此,又以黃門張當為都監,專共交關,看察至尊,候伺神器,離間二宮,傷害骨肉。天下汹汹,人懷危懼,陛下但為寄坐,豈得乆安!此非先帝詔陛下及臣升御牀之本意也。臣雖朽邁,敢忘往言?昔趙高極意,秦氏以滅;呂、霍早斷,漢祚永世。此乃陛下之大鑒,臣受命之時也。太尉臣濟、尚書令臣孚等,皆以爽為有無君之心,兄弟不宜典兵宿衞,奏永寧宮。皇太后令勑臣如奏施行。臣輒勑主者及黃門令罷爽、羲、訓吏兵,以侯就第,不得逗留以稽車駕;敢有稽留,便以軍法從事。臣輙力疾,將兵屯洛水浮橋,伺察非常。」
\gezhu{世語曰:初,宣王勒兵從闕下趣武庫,當爽門,人逼車住。爽妻劉怖,出至廳事,謂帳下守督曰:「公在外。今兵起,如何?」督曰:「夫人勿憂。」乃上門樓,引弩注箭欲發。將孫謙在後牽止之曰:「天下事未可知!」如此者三,宣王遂得過去。}


爽得宣王奏事,不通,迫窘不知所為。
\gezhu{干寶晉紀曰:爽留車駕宿伊水南,伐木為鹿角,發屯甲兵數千人以為衞。魏末傳曰:宣王語弟孚,「陛下在外不可露宿,促送帳幔、太官食具詣行在所」。}
大司農沛國桓範聞兵起,不應太后召,矯詔開平昌門,拔取劒戟,略將門候,南奔爽。宣王知,曰:「範畫策,爽必不能用範計。」範說爽使車駕幸許昌,招外兵。爽兄弟猶豫未決,範重謂羲曰:「當今日,卿門戶求貧賤復可得乎?且匹夫持質一人,尚欲望活,今卿與天子相隨,令於天下,誰敢不應者?」羲猶不能納。侍中許允、尚書陳泰說爽,使早自歸罪。爽於是遣允、泰詣宣王,歸罪請死,乃通宣王奏事。
\gezhu{干寶晉書曰:桓範出赴爽,宣王謂蔣濟曰:「智囊往矣。」濟曰:「範則智矣,駑馬戀棧豆,爽必不能用也。」世語曰:宣王使許允、陳泰解語爽,蔣濟亦與書達宣王之旨,又使爽所信殿中校尉尹大目謂爽,唯免官而已,以洛水為誓。爽信之,罷兵。魏氏春秋曰:爽旣罷兵,曰:「我不失作富家翁。」範哭曰:「曹子丹佳人,生汝兄弟,犢耳!何圖今日坐汝等族滅矣!」}
遂免爽兄弟,以侯還第。
\gezhu{魏末傳曰:爽兄弟歸家,勑洛陽縣發民八百人,使尉部圍爽第四角,角作高樓,令人在上望視爽兄弟舉動。爽計窮愁悶,持彈到後園中,樓上人便唱言「故大將軍東南行!」爽還廳事上,與兄弟共議,未知宣王意深淺,作書與宣王曰:「賤子爽哀惶恐怖,無狀招禍,分受屠滅,前遣家人迎糧,于今未反,數日乏匱,當煩見餉,以繼旦夕。」宣王得書大驚,即荅書曰:「初不知乏糧,甚懷踧踖。令致米一百斛,并肉脯、鹽豉、大豆。」尋送。爽兄弟不達變數,即便喜歡,自謂不死。}


初,張當私以所擇才人張、何等與爽。疑其有姦,收當治罪。當陳爽與晏等陰謀反逆,並先習兵,須三月中欲發,於是收晏等下獄。會公卿朝臣廷議,以為「春秋之義,『君親無將,將而必誅』。爽以支屬,世蒙殊寵,親受先帝握手遺詔,託以天下,而包藏禍心,蔑棄顧命,乃與晏、颺及當等謀圖神器,範黨同罪人,皆為大逆不道」。於是收爽、羲、訓、晏、颺、謐、軌、勝、範、當等,皆伏誅,夷三族。
\gezhu{魏略曰:鄧颺字玄茂,鄧禹後也。少得士名於京師。明帝時為尚書郎,除洛陽令,坐事免,拜中郎,又入兼中書郎。初,颺與李勝等為浮華友,及在中書,浮華事發,被斥出,遂不復用。正始初,乃出為潁川太守,轉大將軍長史,遷侍中尚書。颺為人好貨,前在內職,許臧艾授以顯官,艾以父妾與颺,故京師為之語曰:「以官易婦鄧玄茂。」每所薦達,多如此比。故何晏選舉不得人,頗由颺之不公忠,遂同其罪,蓋由交友非其才。}
\gezhu{丁謐,字彥靖。父斐,字文侯。初,斐隨太祖,太祖以斐鄉里,特饒愛之。斐性好貨,數請求犯法,輒得原宥。為典軍校尉,總攝內外,每所陳說,多見從之。建安末,從太祖征吳。斐隨行,自以家牛羸困,乃私易官牛,為人所白,被收送獄,奪官。其後太祖問斐曰:「文侯,印綬所在?」斐亦知見戲,對曰:「以易餅耳。」太祖笑,顧謂左右曰:「東曹毛掾數白此家,欲令我重治,我非不知此人不清,良有以也。我之有斐,譬如人家有盜狗而善捕鼠,盜雖有小損,而完我囊貯。」遂復斐官,聽用如初。後數歲,病亡。謐少不肯交游,但博觀書傳。為人沈毅,頗有才略。太和中,常住鄴,借人空屋,居其中。而諸王亦欲借之,不知謐已得,直開門入。謐望見王,交脚卧而不起,而呼其奴客曰:「此何等人?促呵使去。」王怒其無禮,還具上言。明帝收謐,繫鄴獄,以其功臣子,原出。後帝聞其有父風,召拜度支郎中。曹爽宿與相親,時爽為武衞將軍,數為帝稱其可大用。會帝崩,爽輔政,乃拔謐為散騎常侍,遂轉尚書。謐為人外似踈略,而內多忌。其在臺閣,數有所彈駮,臺中患之,事不得行。又其意輕貴,多所忽略,雖與何晏、鄧颺等同位,而皆少之,唯以勢屈於爽。爽亦敬之,言無不從。故于時謗書,謂「臺中有三狗,二狗崖柴不可當,一狗憑默作疽囊。」三狗,謂何、鄧、丁也。默者,爽小字也。其意言三狗皆欲嚙人,而謐尤甚也。奏使郭太后出居別宮,及遣樂安王使北詣鄴,又遣文欽令還淮南,皆謐之計。司馬宣王由是特深恨之。}
\gezhu{畢軌,字昭先。父字子禮,建安中為典農校尉。軌以才能,少有名聲。明帝在東宮時,軌在文學中。黃初末,出為長史。明帝即位,入為黃門郎,子尚公主,居處殷富。遷并州刺史。其在并州,名為驕豪。時雜虜數為暴,害吏民,軌輒出軍擊鮮卑軻比能,失利。中護軍蔣濟表曰:「畢軌前失,旣往不咎,但恐是後難可以再。凡人材有長短,不可彊成。軌文雅志意,自為美器。今失并州,換置他州,若入居顯職,不毀其德,於國事實善。此安危之要,唯聖恩察之。」至正始中,入為中護軍,轉侍中尚書,遷司隷校尉。素與曹爽善,每言於爽,多見從之。}
\gezhu{李勝字公昭。父休字子朗,有智略。張魯前為鎮北將軍,休為司馬,家南鄭。時漢中有甘露降,子朗見張魯精兵數萬人,有四塞之固,遂建言赤氣乆衰,黃家當興,欲魯舉號,魯不聽。會魯破,太祖以其勸魯內附,賜爵關內侯,署散官騎從,詣鄴。至黃初中,仕歷上黨、鉅鹿二郡太守,後以年老還,拜議郎。勝少游京師,雅有才智,與曹爽善。明帝禁浮華,而人白勝堂有四䆫八達,各有主名。用是被收,以其所連引者多,故得原,禁錮數歲。帝崩,曹爽輔政,勝為洛陽令。夏侯玄為征西將軍,以勝為長史。玄亦宿與勝厚。駱谷之役,議從勝出,由是司馬宣王不恱於勝。累遷熒陽太守、河南尹。勝前後所宰守,未嘗不稱職,為尹歲餘,廳事前屠蘇壞,令人更治之,小材一枚激墮,正檛受符吏石虎頭,斷之。後旬日,遷為荊州刺史,未及之官而敗也。}
\gezhu{桓範字元則,世為冠族。建安末,入丞相府。延康中,為羽林左監。以有文學,與王象等典集皇覽。明帝時為中領軍尚書,遷征虜將軍、東中郎將,使持節都督青、徐諸軍事,治下邳。與徐州刺史鄭岐爭屋,引節欲斬岐,為岐所奏,不直,坐免還。復為兖州刺吏,怏怏不得意。又聞當轉為兾州牧。是時兾州統屬鎮北,而鎮北將軍呂昭才實仕進,本在範後。範謂其妻仲長曰:「我寧作諸卿,向三公長跪耳,不能為呂子展屈也。」其妻曰:「君前在東,坐欲擅斬徐州刺史,衆人謂君難為作下,今復羞為呂屈,是復難為作上也。」範忿其言觸實,乃以刀環撞其腹。妻時懷孕,遂墮胎死。範亦竟稱疾,不赴兾州。正始中拜大司農。範前在臺閣,號為曉事,及為司農,又以清省稱。範嘗抄撮漢書中諸雜事,自以意斟酌之,名曰世要論。蔣濟為太尉,嘗與範會社下,羣卿列坐有數人,範懷其所撰,欲以示濟,謂濟當虛心觀之。範出其書以示左右,左右傳之示濟,濟不肯視,範心恨之。因論他事,乃發怒謂濟曰:「我祖薄德,公輩何似邪?」濟性雖彊毅,亦知範剛毅,睨而不應,各罷。範於沛郡,仕次在曹真後。于時曹爽輔政,以範鄉里老宿,於九卿中特敬之,然不甚親也。及宣王起兵,閉城門,以範為曉事,乃指召之,欲使領中領軍。範欲應召,而其子諫之,以車駕在外,不如南出。範疑有頃,兒又促之。範欲去而司農丞吏皆止範。範不從,乃突出至平昌城門,城門已閉。門候司蕃,故範舉吏也,範呼之,舉手中版以示之,矯曰:「有詔召我,卿促開門!」蕃欲求見詔書,範呵之,言「卿非我故吏邪,何以敢爾?」乃開之。範出城,顧謂蕃曰:「太傅圖逆,卿從我去!」蕃徒行不能及,遂避側。範南見爽,勸爽兄弟以天子詣許昌,徵四方以自輔。爽疑,羲又無言。範自謂羲曰:「事昭然,卿用讀書何為邪!於今日卿等門戶倒矣!」俱不言。範又謂羲曰:「卿別營近在闕南,洛陽典農治在城外,呼召如意。今詣許昌,不過中宿,許昌別庫,足相被假;所憂當在穀食,而大司農印章在我身。」羲兄弟默然不從,中夜至五鼓,爽乃投刀於地,謂諸從駕羣臣曰:「我度太傅意,亦不過欲令我兄弟向己也。我獨有以不合於遠近耳!」遂進謂帝曰:「陛下作詔免臣官,報皇太后令。」範知爽首免而己必坐唱義也。範乃曰:「老子今茲坐卿兄弟族矣!」爽等旣免,帝還宮,遂令範隨從。到洛水浮橋北,望見宣王,下車叩頭而無言。宣王呼範姓曰:「桓大夫何為爾邪!」車駕入宮,有詔範還復位。範詣闕拜章謝,待報。會司蕃詣鴻臚自首,具說範前臨出所道。宣王乃忿然曰:「誣人以反,於法何應?」主者曰:「科律,反受其罪。」乃收範於闕下。時人持範甚急,範謂部官曰:「徐之,我亦義士耳。」遂送廷尉。}
\gezhu{世語曰:初,爽夢二虎銜雷公,雷公若二升椀,放著庭中。爽惡之,以問占者,靈臺丞馬訓曰:「憂兵。」訓退,告其妻曰:「爽將以兵亡,不出旬日。」}
\gezhu{漢晉春秋曰:安定皇甫謐以九年冬夢至洛陽,自廟出,見車騎甚衆,以物呈廟云:「誅大將軍曹爽。」寤而以告其邑人,邑人曰:「君欲作曹人之夢乎!朝無公孫彊如何?且爽兄弟典重兵,又權尚書事,誰敢謀之?」謐曰:「爽無叔振鐸之請,苟失天機則離矣,何恃於彊?昔漢之閻顯,倚母后之尊,權國威命,可謂至重矣,閹人十九人一旦尸之,況爽兄弟乎?」}
\gezhu{世語曰:初,爽出,司馬魯芝留在府,聞有事,將營騎斫津門出赴爽。爽誅,擢為御史中丞。及爽解印綬,將出,主簿楊綜止之曰:「公挾主握權,捨此以至東巿乎?」爽不從。有司奏綜導爽反,宣王曰:「各為其主也。」宥之,以為尚書郎。芝字世英,扶風人也。以後仕進至特進光祿大夫。綜字初伯,後為安東將軍司馬文王長史。}
\gezhu{臣松之案:夏侯湛為芝銘及干寶晉紀並云爽旣誅,宣王即擢芝為并州刺史,以綜為安東參軍。與世語不同。}
嘉平中,紹功臣世,封真族孫熈為新昌亭侯,邑三百戶,以奉真後。
\gezhu{干寶晉紀曰:蔣濟以曹真之勳力,不宜絕祀,故以熈為後。濟又病其言之失信於爽,發病卒。}


晏,何進孫也。母尹氏,為太祖夫人。晏長於宮省,又尚公主,少以才秀知名,好老莊言,作道德論及諸文賦著述凡數十篇。
\gezhu{晏字平叔。魏略曰:「太祖為司空時,納晏母并收養晏,其時秦宜祿兒阿蘇亦隨母在公家,並見寵如公子。蘇即朗也。蘇性謹慎,而晏無所顧憚,服飾擬於太子,故文帝特憎之,每不呼其姓字,常謂之為「假子」。晏尚主,又好色,故黃初時無所事任。及明帝立,頗為冗官。至正始初,曲合於曹爽,亦以才能,故爽用為散騎侍郎,遷侍中尚書。晏前以尚主,得賜爵為列侯,又其母在內,晏性自喜,動靜粉白不去手,行步顧影。晏為尚書,主選舉,其宿與之有舊者,多被拔擢。魏末傳曰:晏婦金鄉公主,即晏同母妹。公主賢,謂其母沛王太妃曰:「晏為惡日甚,將何保身?」母笑曰:「汝得無妬晏邪!」俄而晏死。有一男,年五六歲,宣王遣人錄之。晏母歸藏其子王宮中,向使者搏頰,乞白活之,使者具以白宣王。宣王亦聞晏婦有先見之言,心常嘉之;且為沛王故,特原不殺。魏氏春秋曰:初,夏侯玄、何晏等名盛於時,司馬景王亦預焉。晏嘗曰:「唯深也,故能通天下之志,夏侯泰初是也;唯幾也,故能成天下之務,司馬子元是也;惟神也,不疾而速,不行而至,吾聞其語,未見其人。」蓋欲以神況諸己也。初,宣王使晏與治爽等獄。晏窮治黨與,兾以獲宥。宣王曰:「凡有八族。」晏疏丁、鄧等七姓。宣王曰:「未也。」晏窮急,乃曰:「豈謂晏乎!」宣王曰:「是也。」乃收晏。臣松之案:魏末傳云晏取其同母妹為妻,此搢紳所不忍言,雖楚王之妻媦,不是甚也已。設令此言出於舊史,猶將莫之或信,況厎下之書乎!案諸王公傳,沛王出自杜夫人所生。晏母姓尹,公主若與沛王同生,焉得言與晏同母?皇甫謐烈女傳曰:爽從弟文叔,妻譙郡夏侯文寧之女,名令女。文叔早死,服闋,自以年少無子,恐家必嫁己,乃斷髮以為信。其後,家果欲嫁之,令女聞,即復以刀截兩耳,居止常依爽。及爽被誅,曹氏盡死。令女叔父上書與曹氏絕婚,彊迎令女歸。時文寧為梁相,憐其少,執義,又曹氏無遺類,兾其意沮,迺微使人諷之。令女歎且泣曰:「吾亦惟之,許之是也。」家以為信,防之少懈。令女於是竊入寢室,以刀斷鼻,蒙被而卧。其母呼與語,不應,發被視之,血流滿牀席。舉家驚惶,奔往視之,莫不酸鼻。或謂之曰:「人生世間,如輕塵棲弱草耳,何至辛苦迺爾!且夫家夷滅已盡,守此欲誰為哉?」令女曰:「聞仁者不以盛衰改節,義者不以存亡易心,曹氏前盛之時,尚欲保終,況今衰亡,何忍棄之!禽獸不行,吾豈為乎?」司馬宣王聞而嘉之,聽使乞子字養,為曹氏後,名顯於世。}


\end{pinyinscope}