\article{鄧艾傳}
\begin{pinyinscope}
 
 
 鄧艾字士載,義陽棘陽人也。少孤,太祖破荊州,徙汝南,為農民養犢。年十二,隨母至潁川,讀故太丘長陳寔碑文,言「文為世範,行為士則」,艾遂自名範,字士則。後宗族有與同者,故改焉。為都尉學士,以口吃,不得作幹佐。為稻田守叢草吏。同郡吏父怜其家貧,資給甚厚,艾初不稱謝。每見高山大澤,輙規度指畫軍營處所,時人多笑焉。後將典農綱紀,上計吏,因使見太尉司馬宣王。宣王奇之,辟之為掾,
 
 
\gezhu{世語曰:鄧艾少為襄城典農部民,與石苞皆年十二三。謁者陽翟郭玄信,武帝監軍郭誕元弈之子。建安中,少府吉本起兵許都,玄信坐被刑在家,從典農司馬求入御,以艾、苞與御,行十餘里,與語,恱之,謂二人皆當遠至為佐相。艾後為典農功曹,奉使詣宣王,由此見知,遂被拔擢。}
 遷尚書郎。
 
 
 
 
 時欲廣田畜糓,為滅賊資,使艾行陳、項已東至壽春。艾以為「田良水少,不足以盡地利,宜開河渠,可以引水澆溉,大積軍糧,又通運漕之道。」乃著濟河論以喻其指。又以為「昔破黃巾,因為屯田,積穀於許都以制四方。今三隅已定,事在淮南,每大軍征舉,運兵過半,功費巨億,以為大役。陳、蔡之間,土下田良,可省許昌左右諸稻田,并水東下。令淮北屯二萬人,淮南三萬人,十二分休,常有四萬人,且田且守。水豐常收三倍於西,計除衆費,歲完五百萬斛以為軍資。六七年間,可積三千萬斛於淮上,此則十萬之衆五年食也。以此乘吳,無往而不克矣。」宣王善之,事皆施行。正始二年,乃開廣漕渠,每東南有事,大軍興衆,汎舟而下,達于江、淮,資食有儲而無水害,艾所建也。
 
 
 
 
 出參征西軍事,遷南安太守。嘉平元年,與征西將軍郭淮拒蜀偏將軍姜維。維退,淮因西擊羌。艾曰:「賊去未遠,或能復還,宜分諸軍以備不虞。」於是留艾屯白水北。三日,維遣廖化自白水南向艾結營。艾謂諸將曰:「維今卒還,吾軍人少,法當來渡而不作橋。此維使化持吾,令不得還。維必自東襲取洮城。」洮城在水北,去艾屯六十里。艾即夜潛軍徑到,維果來渡,而艾先至據城,得以不敗。賜爵關內侯,加討寇將軍,後遷城陽太守。
 
 
 
 
 是時并州右賢王劉豹并為一部,艾上言曰:「戎狄獸心,不以義親,彊則侵暴,弱則內附,故周宣有玁狁之寇,漢祖有平城之困。每匈奴一盛,為前代重患。自單于在外,莫能牽制長卑。誘而致之,使來入侍。由是羌夷失統,合散無主。以單于在內,萬里順軌。今單于之尊日疏,外土之威寖重,則胡虜不可不深備也。聞劉豹部有叛胡,可因叛割為二國,以分其勢。去卑功顯前朝,而子不繼業,宜加其子顯號,使居鴈門。離國弱寇,追錄舊勳,此御邊長計也。」又陳:「羌胡與民同處者,宜以漸出之,使居民表崇廉恥之教,塞姦宄之路。」大將軍司馬景王新輔政,多納用焉。遷汝南太守,至則尋求昔所厚己吏父,乆已死,遣吏祭之,重遺其母,舉其子與計吏。艾所在,荒野開闢,軍民並豐。
 
 
 
 
 諸葛恪圍合肥新城,不克,退歸。艾言景王曰:「孫權已沒,大臣未附,吳名宗大族,皆有部曲,阻兵仗勢,足以建命。恪新秉國政,而內無其主,不念撫恤上下以立根基,競於外事,虐用其民,悉國之衆,頓於堅城,死者萬數,載禍而歸,此恪獲罪之日也。昔子胥、吳起、商鞅、樂毅皆見任時君,主沒而敗。况恪才非四賢,而不慮大患,其亡可待也。」恪歸,果見誅。遷兖州刺史,加振威將軍。上言曰:「國之所急,惟農與戰,國富則兵彊,兵彊則戰勝。然農者,勝之本也。孔子曰『足食足兵』,食在兵前也。上無設爵之勸,則下無財畜之功。今使考績之賞,在於積粟富民,則交游之路絕,浮華之原塞矣。」
 
 
 
 
 高貴鄉公即尊位,進封方城亭侯。毌丘儉作亂,遣健步齎書,欲疑惑大衆,艾斬之,兼道進軍,先趣樂嘉城,作浮橋。司馬景王至,遂據之。文欽以後大軍破敗於城下,艾追之至丘頭。欽奔吳。吳大將軍孫峻等號十萬衆,將渡江,鎮東將軍諸葛誕遣艾據肥陽,艾以與賊勢相遠,非要害之地,輒移屯附亭,遣泰山太守諸葛緒等於黎漿拒戰,遂走之。其年徵拜長水校尉。以破欽等功,進封方城鄉侯,行安西將軍。解雍州刺史王經圍於狄道,姜維退駐鍾提,乃以艾為安西將軍,假節、領護東羌校尉。議者多以為維力已竭,未能更出。艾曰:「洮西之敗,非小失也;破軍殺將,倉廩空虛,百姓流離,幾於危亡。今以策言之,彼有乘勝之勢,我有虛弱之實,一也。彼上下相習,五兵犀利,我將易兵新,器杖未復,二也。彼以船行,吾以陸軍,勞逸不同,三也。狄道、隴西、南安、祁山,各當有守,彼專為一,我分為四,四也。從南安、隴西,因食羌穀,若趣祁山,熟麥千頃,為之縣餌,五也。賊有黠數,其來必矣。」頃之,維果向祁山,聞艾已有備,乃回從董亭趣南安,艾據武城山以相持。維與艾爭險,不克,其夜,渡渭東行,緣山趣上邽,艾與戰於段谷,大破之。
 
 
 
 
 甘露元年詔曰:「逆賊姜維連年狡黠,民夷騷動,西土不寧。艾籌畫有方,忠勇奮發,斬將十數,馘首千計;國威震於巴、蜀,武聲揚於江、岷。今以艾為鎮西將軍、都督隴右諸軍事,進封鄧侯。分五百戶封子忠為亭侯。」二年,拒姜維於長城,維退還。遷征西將軍,前後增邑凡六千六百戶。景元三年,又破維於侯和,維却保沓中。四年秋,詔諸軍征蜀,大將軍司馬文王皆指授節度,使艾與維相綴連;雍州刺史諸葛緒要維,令不得歸。艾遣天水太守王頎等直攻維營,隴西太守牽弘等邀其前,金城太守楊欣等詣甘松。維聞鍾會諸軍已入漢中,引退還。欣等追躡於彊川口,大戰,維敗走。聞雍州已塞道屯橋頭,從孔函谷入北道,欲出雍州後。諸葛緒聞之,却還三十里。維入北道三十餘里,聞緒軍却,尋還,從橋頭過,緒趣截維,較一日不及。維遂東引,還守劒閣。鍾會攻維未能克。艾上言:「今賊摧折,宜遂乘之,從陰平由邪徑經漢德陽亭趣涪,出劒閣西百里,去成都三百餘里,奇兵衝其腹心。劒閣之守必還赴涪,則會方軌而進;劒閣之軍不還,則應涪之兵寡矣。軍志有之曰:『攻其不備,出其不意。』今掩其空虛,破之必矣。」
 
 
 
 
 冬十月,艾自陰平道行無人之地七百餘里,鑿山通道,造作橋閣。山高谷深,至為艱險,又糧運將匱,頻於危殆。艾以氊自裹,推轉而下。將士皆攀木緣崖,魚貫而進。先登至江由,蜀守將馬邈降。蜀衞將軍諸葛瞻自涪還綿竹,列陳待艾。艾遣子惠唐亭侯忠等出其右,司馬師纂等出其左。忠、纂戰不利,並退還,曰:「賊未可擊。」艾怒曰:「存亡之分,在此一舉,何不可之有?」乃叱忠、纂等,將斬之。忠、纂馳還更戰,大破之,斬瞻及尚書張遵等首,進軍到雒。劉禪遣使奉皇帝璽綬,為箋詣艾請降。
 
 
 
 
 艾至成都,禪率太子諸王及群臣六十餘人靣縛輿櫬詣軍門,艾執節解縛焚櫬,受而宥之。檢御將士,無所虜畧,綏納降附,使復舊業,蜀人稱焉。輙依鄧禹故事,承制拜禪行驃騎將軍,太子奉車、諸王駙馬都尉。蜀群司各隨高下拜為王官,或領艾官屬。以師纂領益州刺史,隴西太守牽弘等領蜀中諸郡。使於緜竹築臺以為京觀,用彰戰功。士卒死事者,皆與蜀兵同共埋藏。艾深自矜伐,謂蜀士大夫曰:「諸君賴遭某,故得有今日耳。若遇吳漢之徒,已殄滅矣。」又曰:「姜維自一時雄兒也,與某相值,故窮耳。」有識者笑之。
 
 
十二月,詔曰:「艾曜威奮武,深入虜庭,斬將搴旗,梟其鯨鯢,使僭號之主稽首係頸,歷世逋誅,一朝而平。兵不踰時,戰不終日,雲徹席卷,蕩定巴蜀。雖白起破彊楚,韓信克勁趙,吳漢禽子陽,亞夫滅七國,計功論美,不足比勳也。其以艾為太尉,增邑二萬戶,封子二人亭侯,各食邑千戶。」
 \gezhu{袁子曰:諸葛亮,重人也,而驟用蜀兵,此知小國弱民難以乆存也。今國家一舉而滅蜀,自征伐之功,未有如此之速者也。方鄧艾以萬人入江由之危險,鍾會以二十萬衆留劒閣而不得進,三軍之士已飢,艾雖戰勝克將,使劉禪數日不降,則二將之軍難以反矣。故功業如此之難也。國家前有壽春之役,後有滅蜀之勞,百姓貧而倉稟虛,故小國之慮,在於時立功以自存,大國之慮,在於旣勝而力竭,成功之後,戒懼之時也。}
 艾言司馬文王曰:「兵有先聲而後實者,今因平蜀之勢以乘吴,吴人震恐,席卷之時也。然大舉之後,將士疲勞,不可便用,且徐緩之;留隴右兵二萬人,蜀兵二萬人,煑鹽興冶,為軍農要用,並作舟船,豫順流之事,然後發使告以利害,吴必歸化,可不征而定也。今宜厚劉禪以致孫休,安士民以來遠人,若便送禪於京都,吴以為流徙,則於向化之心不勸。宜權停留,須來年秋冬,比爾吴亦足平。以為可封禪為扶風王,錫其資財,供其左右。郡有董卓塢,為之宮舍。爵其子為公侯,食郡內縣,以顯歸命之寵。開廣陵、城陽以待吴人,則畏威懷德,望風而從矣。」文王使監軍衞瓘喻艾:「事當須報,不宜輙行。」艾重言曰:「銜命征行,奉指授之策,元惡旣服;至於承制拜假,以安初附,謂合權宜。今蜀舉衆歸命,地盡南海,東接吳會,宜早鎮定。若待國命,往復道途,延引日月。春秋之義,大夫出疆,有可以安社稷,利國家,專之可也。今吴未賔;勢與蜀連,不可拘常以失事機。兵法,進不求名,退不避罪,艾雖無古人之節,終不自嫌以損于國也。」鍾會、胡烈、師纂等皆白艾所作悖逆,變釁以結。詔書檻車徵艾。
 \gezhu{魏氏春秋曰:艾仰天歎曰:「艾忠臣也,一至此乎!白起之酷,復見於今日矣。」}
 
 
艾父子旣囚,鍾會至成都,先送艾,然後作亂。會已死,艾本營將士追出艾檻車,迎還。瓘遣田續等討艾,遇於緜竹西,斬之。子忠與艾俱死,餘子在洛陽者悉誅,徙艾妻子及孫於西域。
 \gezhu{漢晉春秋曰:初艾之下江由也,以續不進,欲斬,旣而捨之。及瓘遣續,謂曰:「可以報江由之辱矣。」杜預言於衆曰:「伯玉其不免乎!身為名士,位望已高,旣無德音,又不御下以正,是小人而乘君子之器,將何以堪其責乎?」瓘聞之,不俟駕而謝。世語曰:師纂亦與艾俱死。纂性急少恩,死之日體無完皮。}
 
 
初,艾當伐蜀,夢坐山上而有流水,以問殄虜護軍爰邵。邵曰:「按易卦,山上有水曰蹇。蹇繇曰:『蹇利西南,不利東北。』孔子曰:『蹇利西南,往有功也;不利東北,其道窮也。』往必克蜀,殆不還乎!」艾憮然不樂。
 \gezhu{荀綽兾州記曰:邵起自幹吏,位至衞尉。長子翰,河東太守。中子敞,大司農。少子倩,字君幼,寬厚有器局,勤於當世,歷位兾州刺史、太子右衞率。翰子俞,字世都,清貞貴素,辯於論議,採公孫龍之辭以談微理。少有能名,辟太尉府,稍歷顯位,至侍中中書令,遷為監。臣松之按:蹇彖辭云「蹇利西南,往得中也」,不云「有功」;下云「利見大人,往有功也」。}
 
 
 
 
 泰始元年,晉室踐阼,詔曰:「昔太尉王淩謀廢齊王,而王竟不足以守位。征西將軍鄧艾,矜功失節,實應大辟。然被書之日,罷遣人衆,束手受罪,比於求生遂為惡者,誠復不同。今大赦得還,若無子孫者聽使立後,令祭祀不絕。」三年,議郎段灼上疏理艾曰:「艾心懷至忠而荷反逆之名,平定巴蜀而受夷滅之誅,臣竊悼之。惜哉,言艾之反也!艾性剛急,輕犯雅俗,不能恊同朋類,故莫肯理之。臣敢言艾不反之狀。昔姜維有斷隴右之志,艾脩治備守,積糓彊兵。值歲凶旱,艾為區種,身被烏衣,手執耒耜,以率將士。上下相感,莫不盡力。艾持節守邊,所統萬數,而不難僕虜之勞,士民之役,非執節忠勤,孰能若此?故落門、段谷之戰,以少擊多,摧破彊賊。先帝知其可任,委艾廟勝,授以長策。艾受命忘身,束馬縣車,自投死地,勇氣陵雲,士衆乘勢,使劉禪君臣靣縛,叉手屈膝。艾功名以成,當書之竹帛,傳祚萬世。七十老公,反欲何求!艾誠恃養育之恩,心不自疑,矯命承制,權安社稷;雖違常科,有合古義,原心定罪,本在可論。鍾會忌艾威名,搆成其事。忠而受誅,信而見疑,頭縣馬巿,諸子并斬,見之者垂泣,聞之者歎息。陛下龍興,闡弘大度,釋諸嫌忌,受誅之家,不拘叙用。昔秦民憐白起之無罪,吳人傷子胥之寃酷,皆為立祠。今天下民人為艾悼心痛恨,亦猶是也。臣以為艾身首分離,捐棄草土,宜收尸喪,還其田宅。以平蜀之功,紹封其孫,使闔棺定謚,死無餘恨。赦寃魂於黃泉,收信義於後世,葬一人而天下慕其行,埋一魂而天下歸其義,所為者寡而恱者衆矣。」九年,詔曰:「艾有功勳,受罪不逃刑,而子孫為民隷,朕常愍之。其以嫡孫朗為郎中。」
 
 
艾在西時,修治障塞,築起城塢。泰始中,羗虜大叛,頻殺刺史,涼州道斷。吏民安全者,皆保艾所築塢焉。
 \gezhu{世語曰:咸寧中,積射將軍樊震為西戎牙門,得見辭,武帝問震所由進,震自陳曾為鄧艾伐蜀時帳下將,帝遂尋問艾,震具申艾之忠,言之流涕。先是以艾孫朗為丹水令,由此遷為定陵令。次孫千秋有時望,光祿大夫王戎辟為掾。永嘉中,朗為新都太守,未之官,在襄陽失火,朗及母妻子舉室燒死,惟子韜子行得免。千秋先卒,二子亦燒死。}
 
 
艾州里時輩南陽州泰,亦好立功業,善用兵,官至征虜將軍、假節都督江南諸軍事。景元二年薨,追贈衞將軍,謚曰壯侯。
 \gezhu{世語曰:初,荊州刺史裴潛以泰為從事,司馬宣王鎮宛,潛數遣詣宣王,由此為宣王所知。及征孟達,泰又導軍,遂辟泰。泰頻喪考、妣、祖,九年居喪,宣王留缺待之,至三十六日,擢為新城太守。宣王為泰會,使尚書鍾繇調泰:「君釋褐登宰府,三十六日擁麾蓋,守兵馬郡;乞兒乘小車,一何駛乎?」泰曰:「誠有此。君,名公之子,少有文采,故守吏職;獼猴騎土牛,又何遲也!」衆賔咸恱。後歷兖、豫州刺史,所在有籌筭績效。}
 
 
\end{pinyinscope}