\article{呂布傳}
\begin{pinyinscope}
 
 
 呂布字奉先,五原郡九原人也。以驍武給并州。刺史丁原為騎都尉,屯河內,以布為主簿,大見親待。靈帝崩,原將兵詣洛陽。
 
 
\gezhu{英雄記曰:原子建陽。本出自寒家,為人麤略,有武勇,善騎射。為南縣吏,受使不辭難,有警急,追寇虜,輙在其前。裁知書,少有吏用。}
 與何進謀誅諸黃門,拜執金吾。進敗,董卓入京都,將為亂,欲殺原,并其兵衆。卓以布見信於原,誘布令殺原。布斬原首詣卓,卓以布為騎都尉,甚愛信之,誓為父子。
 
 
布便弓馬,膂力過人,號為飛將。稍遷至中郎將,封都亭侯。卓自以遇人無禮,恐人謀己,行止常以布自衞。然卓性剛而褊,忿不思難,嘗小失意,拔手戟擲布。布拳捷避之,
 \gezhu{詩曰:「無拳無勇,職為亂階。」注:「拳,力也。」}
 為卓顧謝,卓意亦解。由是陰怨卓。卓常使布守中閤,布與卓侍婢私通,恐事發覺,心不自安。
 
 
先是,司徒王允以布州里壯健,厚接納之。後布詣允,陳卓幾見殺狀。時允與僕射士孫瑞密謀誅卓,是以告布使為內應。布曰:「柰如父子何!」允曰:「君自姓呂,本非骨肉。今憂死不暇,何謂父子?」布遂許之,手刃刺卓。語在卓傳。允以布為奮威將軍,假節,儀比三司,進封溫侯,共秉朝政。布自殺卓後,畏惡涼州人,涼州人皆怨。由是李傕等遂相結還攻長安城。
 \gezhu{英雄記曰:郭汜在城北。布開城門,將兵就汜,言「且却兵,但身決勝負」。汜、布乃獨共對戰,布以矛刺中汜,汜後騎遂前救汜,汜、布遂各兩罷。}
 布不能拒,傕等遂入長安。卓死後六旬,布亦敗。
 \gezhu{臣松之案英雄記曰:諸書,布以四月二十三日殺卓,六月一日敗走,時又無閏,不及六旬。}
 將數百騎出武關,欲詣袁術。
 
 
布自以殺卓為術報讎,欲以德之。術惡其反覆,拒而不受。北詣袁紹,紹與布擊張燕於常山。燕精兵萬餘,騎數千。布有良馬曰赤兎。
 \gezhu{曹瞞傳曰:時人語曰:「人中有呂布,馬中有赤兎。」}
 常與其親近成廉、魏越等陷鋒突陣,遂破燕軍。而求益兵衆,將士鈔掠,紹患忌之。布覺其意,從紹求去。紹恐還為己害,遣壯士夜掩殺布,不獲。事露,布走河內,
 \gezhu{英雄記曰:布自以有功於袁氏,輕傲紹下諸將,以為擅相署置,不足貴也。布求還洛,紹假布領司隷校尉。外言當遣,內欲殺布。明日當發,紹遣甲士三十人,辭以送布。布使止於帳側,偽使人於帳中鼓箏。紹兵卧,布無何出帳去,而兵不覺。夜半兵起,亂斫布牀被,謂為已死。明日,紹訊問,知布尚在,乃閉城門。布遂引去。}
 與張楊合。紹令衆追之,皆畏布,莫敢逼近者。
 \gezhu{英雄記曰:楊及部曲諸將皆受傕、汜購募,共圖布。布聞之,謂楊曰:「布,卿州里也。卿殺布,於卿弱。不如賣布,可極得汜、傕爵寵。」楊於是外許汜、傕,內實保護布。汜、傕患之,更下大封詔書,以布為頴川太守。}
 
 
\end{pinyinscope}