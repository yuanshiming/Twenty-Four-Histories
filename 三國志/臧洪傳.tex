\article{臧洪傳}
\begin{pinyinscope}
 
 
 臧洪字子源,廣陵射陽人也。父旻,歷匈奴中郎將、中山·太原太守,所在有名。
 
 
\gezhu{謝承漢書曰:旻有幹事才,達於從政,為漢良吏。初從徐州從事辟司徒府,除盧奴令,兾州舉尤異,遷揚州刺史、丹楊太守。是時邊方有警,羌、胡出寇,三府舉能,遷旻匈奴中郎將。討賊有功,徵拜議郎,還京師。見太尉袁逢,逢問其西域諸國土地、風俗、人物、種數。旻具荅言西域本三十六國,後分為五十五,稍散至百餘國;其國大小,道里近遠,人數多少,風俗燥濕,山川、草木、鳥獸、異物名種,不與中國同者,悉口陳其狀,手畫地形。逢奇其才,歎息言:「雖班固作西域傳,何以加此?」旻轉拜長水校尉,終太原太守。}
 洪體皃魁梧,有異於人,舉孝廉為郎。時選三署郎以補縣長;琅邪趙昱為莒長,東萊劉繇下邑長,東海王朗菑丘長,洪即丘長。靈帝末,棄官還家,太守張超請洪為功曹。
 
 
董卓殺帝,圖危社稷,洪說超曰:「明府歷世受恩,兄弟並據大郡,今王室將危,賊臣未梟,此誠天下義烈報恩効命之秋也。今郡境尚全,吏民殷富,若動枹鼓,可得二萬人,以此誅除國賊,為天下倡先,義之大者也。」超然其言,與洪西至陳留,見兄邈計事。邈亦素有心,會于酸棗,邈謂超曰:「聞弟為郡守,政教威恩不由己出,動任臧洪,洪者何人?」超曰:「洪才略智數優超,超甚愛之,海內奇士也。」邈即引見洪,與語大異之。致之於劉兖州公山、孔豫州公緒,皆與洪親善。乃設壇場,方共盟誓,諸州郡更相讓,莫敢當,咸共推洪。洪乃升壇操槃歃血而盟曰:「漢室不幸,皇綱失統,賊臣董卓乘釁縱害,禍加至尊,虐流百姓,大懼淪喪社稷,剪覆四海。兖州刺史岱、豫州刺史伷、陳留太守邈、東郡太守瑁、廣陵太守超等,糾合義兵,並赴國難。凡我同盟,齊心戮力,以致臣節,殞首喪元,必無二志。有渝此盟,俾墜其命,無克遺育。皇天后土,祖宗明靈,實皆鑒之!」洪辭氣慷慨,涕泣橫下,聞其言者,雖卒伍厮養,莫不激揚,人思致節。
 \gezhu{臣松之案:于時此盟止有劉岱等五人而已。魏氏春秋橫內劉表等數人,皆非事實。表保據江、漢,身未甞出境,何由得與洪同壇而盟乎?}
 頃之,諸軍莫適先進,而食盡衆散。
 
 
超遣洪詣大司馬劉虞謀,值公孫瓚之難,至河間,遇幽、兾二州交兵,使命不達。而袁紹見洪,又奇重之,與結分合好。會青州刺史焦和卒,紹使洪領青州以撫其衆。
 \gezhu{九州春秋曰:初平中,焦和為青州刺史。是時英雄並起,黃巾寇暴,和務及同盟,俱入京畿,不暇為民保鄣,引軍踰河而西。未乆而袁、曹二公卓將戰於熒陽,敗績。黃巾遂廣,屠裂城邑。和不能禦,然軍器尚利,戰士尚衆,而耳目偵邏不設,恐動之言妄至,望寇奔走,未甞接風塵交旗鼓也。欲作陷冰丸沈河,令賊不得渡,禱祈羣神,求用兵必利,耆筮常陳於前,巫祝不去於側;入見其清談干雲,出則渾亂,命不可知。州遂蕭條,悉為丘墟也。}
 洪在州二年,羣盜奔走。紹歎其能,徙為東郡太守,治東武陽。
 
 
 
 
 太祖圍張超於雍丘,超言:「唯恃臧洪,當來救吾。」衆人以為袁、曹方睦,而洪為紹所表用,必不敗好招禍,遠來赴此。超曰:「子源,天下義士,終不背本者,但恐見禁制,不相及逮耳。」洪聞之,果徒跣號泣,並勒所領兵,又從紹請兵馬,求欲救超,而紹終不聽許。超遂族滅。洪由是怨紹,絕不與通。紹興兵圍之,歷年不下。紹令洪邑人陳琳書與洪,喻以禍福,責以恩義。洪荅曰:
 
 
 
 
 隔闊相思,發於寤寐。幸相去步武之間耳,而以趣舍異規,不得相見,其為愴恨,可為心哉!前日不遺,比辱雅貺,述叙禍福,公私切至。所以不即奉荅者,旣學薄才鈍,不足塞詰;亦以吾子攜負側室,息肩主人,家在東州,僕為仇敵。以是事人,雖披中情,墮肝膽,猶身疏有罪,言甘見怪,方首尾不救,何能恤人?且以子之才,窮該典籍,豈將闇於大道,不達余趣哉!然猶復云云者,僕以是知足下之言,信不由衷,將以救禍也。必欲筭計長短,辯諮是非,是非之論言滿天下,陳之更不明,不言無所損。又言傷告絕之義,非吾所忍行也,是以捐棄紙筆,一無所答。亦兾遙忖其心,知其計定,不復渝變也。重獲來命,援引古今,紛紜六紙,雖欲不言,焉得已哉!
 
 
僕小人也,本因行役,寇竊大州,恩深分厚,寧樂今日自還接刃!每登城勒兵,望主人之旗鼓,感故友之周旋,撫弦搦矢,不覺流涕之覆靣也。何者?自以輔佐主人,無以為悔。主人相接,過絕等倫。當受任之初,自謂究竟大事,共尊王室。豈悟天子不恱,本州見侵,郡將遘牖里之厄,陳留克創兵之謀,謀計棲遲,喪忠孝之名,杖策攜背,虧交友之分。揆此二者,與其不得已,喪忠孝之名與虧交友之道,輕重殊塗,親疏異畫,故便收淚告絕。若使主人少垂故人,住者側席,去者克己,不汲汲於離友,信刑戮以自輔,則僕抗季札之志,不為今日之戰矣。何以效之?昔張景明親登壇喢血,奉辭奔走,卒使韓牧讓印,主人得地;然後但以拜章朝主,賜爵獲傳之故,旋時之間,不蒙觀過之貸,而受夷滅之禍。
 \gezhu{臣松之案英雄記云:「袁紹使張景明、郭公則、高元才等說韓馥,使讓兾州。」然馥之讓位,景明亦有其功。其餘之事未詳。}
 呂奉先討卓來奔,請兵不獲,告去何罪?復見斫刺,濱于死亡。劉子璜奉使踰時,辭不獲命,畏威懷親,以詐求歸,可謂有志忠孝,無損霸道者也;然輙僵斃麾下,不蒙虧除。
 \gezhu{臣松之案:公孫瓚表列紹罪過云:「紹與故虎牙將軍劉勳首共造兵,勳仍有效,而以小忿枉害於勳,紹罪七也。」疑此是子璜也。}
 僕雖不敏,又素不能原始見終,覩微知著,竊度主人之心,豈謂三子宜死,罰當刑中哉?實且欲一統山東,增兵討讎,懼戰士狐疑,無以沮勸,故抑廢王命以崇承制,慕義者蒙榮,待放者被戮,此乃主人之利,非游士之願也。故僕鑒戒前人,困窮死戰。僕雖下愚,亦甞聞君子之言矣。此實非吾心也。乃主人招焉。凡吾所以背棄國民,用命此城者,正以君子之違,不適敵國故也。是以獲罪主人,見攻踰時,而足下更引此義以為吾規,無乃辭同趨異,非吾子所為休戚者哉!
 
 
 
 
 吾聞之也,義不背親,忠不違君,故東宗本州以為親援,中扶郡將以安社稷,一舉二得以徼忠孝,何以為非?而足下欲使吾輕本破家,均君主人。主人之於我也,年為吾兄,分為篤友,道乖告去,以安君親,可謂順矣。若子之言,則包胥宜致命於伍員,不當號哭於秦庭矣。苟區區於攘患,不知言乖乎道理矣。足下或者見城圍不解,救兵未至,感婚姻之義,惟平生之好,以屈節而苟生,勝守義而傾覆也。昔晏嬰不降志於白刃,南史不曲筆以求生,故身著圖象,名垂後世,況僕據金城之固,驅士民之力,散三年之畜,以為一年之資,匡困補乏,以恱天下,何圖築室反耕哉!但懼秋風揚塵,伯珪馬首南向,張楊、飛燕膂力作難,北鄙將告倒縣之急,股肱奏乞歸之誠耳。主人當鑒我曹輩,反旌退師,治兵鄴垣,何宜乆辱盛怒,暴威於吾城下哉?足下譏吾恃黑山以為救,獨不念黃巾之合從邪!加飛燕之屬悉以受王命矣。昔高祖取彭越於鉅野,光武創基兆於綠林,卒能龍飛中興,以成帝業,苟可輔主興化,夫何嫌哉!況僕親奉璽書,與之從事。
 
 
 
 
 行矣孔璋!足下徼利於境外,臧洪授命於君親;吾子託身於盟主,臧洪策名於長安。子謂余身死而名滅,僕亦笑子生死而無聞焉,悲哉!本同而末離,努力努力,夫復何言!
 
 
 
 
 紹見洪書,知無降意,增兵急攻。城中糧穀以盡,外無彊救,洪自度必不免,呼吏士謂曰:「袁氏無道,所圖不軌,且不救洪郡將。洪於大義,不得不死,念諸君無事空與此禍,可先城未敗,將妻子出。」將吏士民皆垂泣曰:「明府與袁氏本無怨隙,今為本朝郡將之故,自致殘困,吏民何忍當舍明府去也!」初尚掘鼠煑筋角,後無可復食者。主簿啟內厨米三斗,請中分稍以為糜粥,洪歎曰:「獨食此何為!」使作薄粥,衆分歠之,殺其愛妾以食將士。將士咸流涕,無能仰視者。男女七八千人相枕而死,莫有離叛。
 
 
城陷,紹生執洪。紹素親洪,盛施帷幔,大會諸將見洪,謂曰:「臧洪,何相負若此!今日服未?」洪據地瞋目曰:「諸袁事漢,四世五公,可謂受恩。今王室衰弱,無扶翼之意,欲因際會,希兾非望,多殺忠良以立姦威。洪親見呼張陳留為兄,則洪府君亦宜為弟,同共勠力,為國除害,何為擁衆觀人屠滅!惜洪力劣,不能推刃為天下報仇,何謂服乎!」紹本愛洪,意欲令屈服,原之;見洪辭切,知終不為己用,乃殺之。
 \gezhu{徐衆三國評曰:洪敦天下名義,救舊君之危,其恩足以感人情,義足以勵薄俗。然袁亦知己親友,致位州郡,雖非君臣,且實盟主,既受其命,義不應貳。袁、曹方睦,夾輔王室,呂布反覆無義,志在逆亂,而邈、超擅立布為州牧,其於王法,乃一罪人也。曹公討之,袁氏弗救,未為非理也。洪本不當就袁請兵,又不當還為怨讐。為洪計者,苟力所不足,可奔他國以求赴救,若謀力未展以待事機,則宜徐更觀釁,效死於超。何必誓守窮城而無變通,身死殄民,功名不立,良可哀也!}
 洪邑人陳容少為書生,親慕洪,隨洪為東郡丞;城未敗,洪遣出。紹令在坐,見洪當死,起謂紹曰:「將軍舉大事,欲為天下除暴,而專先誅忠義,豈合天意!臧洪發舉為郡將,柰何殺之!」紹慙,左右使人牽出,謂曰:「汝非臧洪儔,空復爾為!」容顧曰:「夫仁義豈有常,蹈之則君子,背之則小人。今日寧與臧洪同日而死,不與將軍同日而生!」復見殺。在紹坐者無不歎息,竊相謂曰:「如何一日殺二烈士!」先是,洪遣司馬二人出,求救於呂布;比還,城已陷,皆赴敵死。
 
 
 
 
 評曰:呂布有虓虎之勇,而無英奇之略,輕狡反覆,唯利是視。自古及今,未有若此不夷滅也。昔漢光武謬於龐萌,近魏太祖亦蔽於張邈。知人則哲,唯帝難之,信矣!陳登、臧洪並有雄氣壯節,登降年夙隕,功業未遂,洪以兵弱敵彊,烈志不立,惜哉!
 
 
\end{pinyinscope}