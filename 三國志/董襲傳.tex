\article{董襲傳}
\begin{pinyinscope}
 
 
 董襲字元代,會稽餘姚人,長八尺,武力過人。
 
 
\gezhu{謝承後漢書稱襲志節慷慨,武毅猛烈。}
 孫策入郡,襲迎於高遷亭,策見而偉之,到署門下賊曹。時山陰宿賊黃龍羅、周勃聚黨數千人,策自出討,襲身斬羅、勃首,還拜別部司馬,授兵數千,遷揚武都尉。從策攻皖,又討劉勳於尋陽,伐黃祖於江夏。
 
 
 
 
 策薨,權年少,初統事,太妃憂之,引見張昭及襲等,問江東可保安不,襲對曰:「江東地勢有山川之固,而討逆明府恩德在民。討虜承基,大小用命,張昭秉衆事,襲等為爪牙,此地利人和之時也,萬無所憂。」衆皆壯其言。
 
 
 
 
 鄱陽賊彭虎等衆數萬人,襲與凌統、步隲、蔣欽各別分討。襲所向輒破,虎等望見旌旗,便散走,旬日盡平,拜威越校尉,遷偏將軍。
 
 
 
 
 建安十三年,權討黃祖。祖橫兩蒙衝挾守沔口,以栟閭大紲繫石為矴,上有千人,以弩交射,飛矢雨下,軍不得前。襲與凌統俱為前部,各將敢死百人,人被兩鎧,乘大舸船,突入蒙衝裏。襲身以刀斷兩紲,蒙衝乃橫流,大兵遂進。祖便開門走,兵追斬之。明日大會,權舉觴屬襲曰:「今日之會,斷紲之功也。」
 
 
 
 
 曹公出濡須,襲從權赴之,使襲督五樓船住濡須口。夜卒暴風,五樓船傾覆,左右散走舸,乞使襲出。襲怒曰:「受將軍任,在此備賊,何等委去也,敢復言此者斬!」於是莫敢干。其夜船敗,襲死。權改服臨殯,供給甚厚。
 
 
\end{pinyinscope}