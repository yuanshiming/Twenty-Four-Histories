\article{韋曜傳}
\begin{pinyinscope}
 
 
 韋曜字弘嗣,吳郡雲陽人也。
 
 
\gezhu{曜本名昭,史為晉諱,改之。}
 少好學,能屬文,從丞相掾,除西安令,還為尚書郎,遷太子中庶子。時蔡頴亦在東宮,性好博弈,太子和以為無益,命曜論之。其辭曰:
 
 
 
 
 蓋聞君子恥當年而功不立,疾沒世而名不稱,故曰學如不及,猶恐失之。是以古之志士,悼年齒之流邁,而懼名稱之不立也。故勉精厲操,晨興夜寐,不遑寧息,經之以歲月,累之以日力,若寗越之勤,董生之篤,漸漬德義之淵,棲遲道藝之域。且以西伯之聖,姬公之才,猶有日昃待旦之勞,故能隆興周道,垂名億載,況在臣庶,而可以已乎?歷觀古今立功名之士,皆有累積殊異之迹,勞身苦體,契闊勤思,平居不墮其業,窮困不易其素,是以卜式立志於耕牧,而黃霸受道於囹圄,終有榮顯之福,以成不朽之名。故山甫勤於夙夜,而吳漢不離公門,豈有游墮哉?
 
 
 
 
 今世之人多不務經術,好翫博弈,廢事棄業,忘寢與食,窮日盡明,繼以脂燭。當其臨局交爭,雌雄未決,專精銳意,心勞體倦,人事曠而不脩,賔旅闕而不接,雖有太牢之饌,韶夏之樂,不暇存也。至或賭及衣物,徙棊易行,廉恥之意弛,而忿戾之色發,然其所志不出一枰之上,所務不過方罫之間,勝敵無封爵之賞,獲地無兼土之實,技非六藝,用非經國;立身者不階其術,徵選者不由其道。求之於戰陣,則非孫、吳之倫也;考之於道藝,則非孔氏之門也;以變詐為務,則非忠信之事也;以劫殺為名,則非仁者之意也;而空妨日廢業,終無補益。是何異設木而擊之,置石而投之哉!且君子之居室也勤身以致養,其在朝也竭命以納忠,臨事且猶旰食,而何博弈之足耽?夫然,故孝友之行立,貞純之名彰也。
 
 
 
 
 方今大吳受命,海內未平,聖朝乾乾,務在得人,勇略之士則受熊虎之任,儒雅之徒則處龍鳳之署,百行兼苞,文武並騖,博選良才,旌簡髦俊,設程試之科,垂金爵之賞,誠千載之嘉會,百世之良遇也。當世之士,宜勉思至道,愛功惜力,以佐明時,使名書史籍,勳在盟府,乃君子之上務,當今之先急也。
 
 
 
 
 夫一木之枰孰與方國之封?枯棊三百孰與萬人之將?衮龍之服,金石之樂,足以兼棊局而貿博弈矣。假令世士移博弈之力而用之於詩書,是有顏、閔之志也;用之於智計,是有良、平之思也;用之於資貨,是有猗頓之富也;用之於射御,是有將帥之備也。如此,則功名立而鄙賤遠矣。
 
 
 
 
 和廢後,為黃門侍郎。孫亮即位,諸葛恪輔政,表曜為太史令,撰吳書,華覈、薛瑩等皆與參同。孫休踐阼,為中書郎、博士祭酒。命曜依劉向故事,校定衆書。又欲延曜侍講,而左將軍張布近習寵幸,事行多玷,憚曜侍講儒士,又性精确,懼以古今警戒休意,固爭不可。休深恨布,語在休傳。然曜竟止不入。
 
 
 
 
 孫皓即位,封高陵亭侯,遷中書僕射,職省,為侍中,常領左國史。時所在承指數言瑞應。皓以問曜,曜荅曰:「此人家筐篋中物耳。」又皓欲為父和作紀,曜執以和不登帝位,宜名為傳。如是者非一,漸見責怒。曜益憂懼,自陳衰老,求去侍、史二官,乞欲成所造書,以從業別有別付,皓終不聽。時有疾病,醫藥監護,持之愈急。
 
 
 
 
 皓每饗宴,無不竟日,坐席無能否率以七升為限,雖不悉入口,皆澆灌取盡。曜素飲酒不過二升,初見禮異時,常為裁減,或密賜茶荈以當酒,至於寵衰,更見偪彊,輒以為罪。又於酒後使侍臣難折公卿,以嘲弄侵克,發摘私短以為歡。時有愆過,或誤犯皓諱,輒見收縛,至於誅戮。曜以為外相毀傷,內長尤恨,使不濟濟,非佳事也,故但示難問經義言論而已。皓以為不承用詔命,意不忠盡,遂積前後嫌忿,收曜付獄,是歲鳳皇二年也。
 
 
 
 
 曜因獄吏上辭曰:「囚荷恩見哀,無與為比,曾無芒氂有以上報,孤辱恩寵,自陷極罪。念當灰滅,長棄黃泉,愚情慺慺,竊有所懷,貪令上聞。囚昔見世間有古歷注,其所紀載旣多虛無,在書籍者亦復錯謬。囚尋按傳記,考合異同,采摭耳目所及,以作洞紀,起自庖犧,至於秦、漢,凡為三卷,當起黃武以來,別作一卷,事尚未成。又見劉熙所作釋名,信多佳者,然物類衆多,難得詳究,故時有得失,而爵位之事,又有非是。愚以官爵,今之所急,不宜乖誤。囚自忘至微,又作官職訓及辯釋名各一卷,欲表上之。新寫始畢,會以無狀,幽囚待命,泯沒之日,恨不上聞,謹以先死列狀,乞上言祕府,於外料取,呈內以聞。追懼淺蔽,不合天聽,抱怖雀息,乞垂哀省。」
 
 
 
 
 曜兾以此求免,而皓更怪其書之垢,故又以詰曜。曜對曰:「囚撰此書,實欲表上,懼有誤謬,數數省讀,不覺點污。被問寒戰,形氣吶吃。謹追辭叩頭五百下,兩手自搏。」而華覈連上疏救曜曰:「曜運值千載,特蒙哀識,以其儒學,得與史官,貂蟬內侍,承荅天問,聖朝仁篤,慎終追遠,迎神之際,垂涕勑曜。曜愚惑不達,不能敷宣陛下大舜之美,而拘繫史官,使聖趣不叙,至行不彰,實曜愚蔽當死之罪。然臣慺慺,見曜自少勤學,雖老不倦,探綜墳典,溫故知新,及意所經識古今行事,外吏之中少過曜者。昔李陵為漢將,軍敗不還而降匈奴,司馬遷不加疾惡,為陵遊說,漢武帝以遷有良史之才,欲使畢成所撰,忍不加誅,書卒成立,垂之無窮。今曜在吳,亦漢之史遷也。伏見前後符瑞彰著,神指天應,繼出累見,一統之期,庶不復乆。事平之後,當觀時設制,三王不相因禮,五帝不相沿樂,質文殊塗,損益異體,宜得曜輩依準古義,有所改立。漢氏承秦,則有叔孫通定一代之儀,曜之才學亦漢通之次也。又吳書雖已有頭角,叙贊未述。昔班固作漢書,文辭典雅,後劉珍、劉毅等作漢記,遠不及固,叙傳尤劣。今吳書當垂千載,編次諸史,後之才士論次善惡,非得良才如曜者,實不可使闕不朽之書。如臣頑蔽,誠非其人。曜年已七十,餘數無幾,乞赦其一等之罪,為終身徒,使成書業,永足傳示,垂之百世。謹通進表,叩首百下。」皓不許,遂誅曜,徙其家零陵。子隆,亦有文學也。
 
 
\end{pinyinscope}