\article{黃蓋傳}
\begin{pinyinscope}
 
 
 黃蓋字公覆,零陵泉陵人也。
 
 
\gezhu{吳書曰:故南陽太守黃子廉之後也,枝葉分離,自祖遷于零陵,遂家焉。蓋少孤,嬰丁凶難,辛苦備甞,然有壯志,雖處貧賤,不自同於凡庸,常以負薪餘閑,學書疏,講兵事。}
 初為郡吏,察孝廉,辟公府。孫堅舉義兵,蓋從之。堅南破山賊,北走董卓,拜蓋別部司馬。堅薨,蓋隨策及權,擐甲周旋,蹈刃屠城。
 
 
 
 
 諸山越不賔,有寇難之縣,輙用蓋為守長。石城縣吏,特難檢御,蓋乃署兩掾,分主諸曹。教曰:「令長不德,徒以武功為官,不以文吏為稱。今賊寇未平,有軍旅之務,一以文書委付兩掾,當檢攝諸曹,糾擿謬誤。兩掾所署,事入諾出,若有姦欺,終不加以鞭杖,宜各盡心,無為衆先。」初皆怖威,夙夜恭職;久之,吏以蓋不視文書,漸容人事。蓋亦嫌外懈怠,時有所省,各得兩掾不奉法數事。乃悉請諸掾吏,賜酒食,因出事詰問。兩掾辭屈,皆叩頭謝罪。蓋曰:「前已相勑,終不以鞭杖相加,非相欺也。」遂殺之。縣中震慄。後轉春穀長,尋陽令。凡守九縣,所在平定。遷丹楊都尉,抑彊扶弱,山越懷附。
 
 
蓋姿貌嚴毅,善於養衆,每所征討,士卒皆爭為先。建安中,隨周瑜拒曹公於赤壁,建策火攻,語在瑜傳。
 \gezhu{吳書曰:赤壁之役,蓋為流矢所中,時寒墮水,為吳軍人所得,不知其蓋也,置廁牀中。蓋自彊以一聲呼韓當,當聞之,曰:「此公覆聲也。」向之垂涕,解易其衣,遂以得生。}
 拜武鋒中郎將。武陵蠻夷反亂,攻守城邑,乃以蓋領太守。時郡兵才五百人,自以不敵,因開城門,賊半入,乃擊之,斬首數百,餘皆奔走,盡歸邑落。誅討魁帥,附從者赦之。自春訖夏,寇亂盡平,諸幽邃巴、醴、由、誕邑侯君長,皆改操易節,奉禮請見,郡境遂清。後長沙益陽縣為山賊所攻,蓋又平討。加偏將軍,病卒于官。
 
 
蓋當官決斷,事無留滯,國人思之。
 \gezhu{吳書曰:又圖畫蓋形,四時祠祭。}
 及權踐阼,追論其功,賜子柄爵關內侯。
 
 
\end{pinyinscope}