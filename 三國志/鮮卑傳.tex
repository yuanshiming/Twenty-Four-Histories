\article{鮮卑傳}
\begin{pinyinscope}
 
 
 鮮卑
 
 
\gezhu{魏書曰:鮮卑亦東胡之餘也,別保鮮卑山,因號焉。其言語習俗與烏丸同。其地東接遼水,西當西城。常以季春大會,作樂水上,嫁女娶婦,髠頭飲宴。其獸異於中國者,野馬、羱羊、端牛。端牛角為弓,世謂之角端者也。又有貂、豽、鼲子,皮毛柔蠕,故天下以為名裘。鮮卑自為冒頓所破,遠竄遼東塞外,不與餘國爭衡,未有名通於漢,而自與烏丸相接。至光武時,南北單于更相攻伐,匈奴損耗,而鮮卑遂盛。建武三十年,鮮卑大人於仇賁率種人詣闕朝貢,封於仇賁為王。永平中,祭肜為遼東太守,誘賂鮮卑,使斬叛烏丸欽志賁等首,於是鮮卑自燉煌、酒泉以東邑落大人,皆詣遼東受賞賜,青、徐二州給錢,歲二億七千萬以為常。和帝時,鮮卑大都護校尉廆帥部衆從烏丸校尉任常擊叛者,封校尉廆為率衆王。殤帝延平中,鮮卑乃東入塞,殺漁陽太守張顯。安帝時,鮮卑大人燕荔陽入朝,漢賜鮮卑王印綬,赤車參駕,止烏丸校尉所治寗下。通胡市,築南北兩部質宮,受邑落質者二十部。是後或反或降,或與匈奴、烏丸相攻擊。安帝末,發緣邊步騎二萬餘人,屯列衝要。後鮮卑八九千騎穿代郡及馬城塞入害長吏,漢遣度遼將軍鄧遵、中郎將馬續出塞追破之。鮮卑大人烏倫、其至鞬等七千餘人詣遵降,封烏倫為王,其至鞬為侯,賜采帛。遵去後,其至鞬復反,圍烏丸校尉於馬城,度遼將軍耿夔及幽州刺史救解之。其至鞬遂盛,控弦數萬騎,數道入塞,趣五原寧貊,攻匈奴南單于,殺左奧鞬日逐王。順帝時,復入塞,殺代郡太守。漢遣黎陽營兵屯中山,緣邊郡兵屯塞下,調五營弩帥令教戰射,南單于將步騎萬餘人助漢擊却之。後烏丸校尉耿曄將率衆王出塞擊鮮卑,多斬首虜,於是鮮卑三萬餘落詣遼東降。匈奴及北單于遁逃後,餘種十餘萬落詣遼東雜處,皆自號鮮卑兵。投鹿侯從匈奴軍三年,其妻在家,有子。投鹿侯歸,怪欲殺之。妻言:「嘗晝行聞雷震,仰天視而雹入其口,因吞之,遂姙身,十月而產,此子必有奇異,且長之。」投鹿侯固不信。妻乃語家,令收養焉,號檀石槐,長大勇健,智畧絕衆。年十四五,異部大人卜賁邑鈔取其外家牛羊,檀石槐策騎追擊,所向無前,悉還得所亡。由是部落畏服,施法禁,平曲直,莫敢犯者,遂推以為大人。檀石槐旣立,乃為庭於高柳北三百餘里彈汗山啜仇水上,東西部大人皆歸焉。兵馬甚盛,南鈔漢邊,北拒丁令,東却夫餘,西擊烏孫,盡據匈奴故地,東西萬二千餘里,南北七千餘里,罔羅山川、水澤、鹽池甚廣。漢患之,桓帝時使匈奴中郎將張奐征之,不克。乃更遣使者齎印綬,即封檀石槐為王,欲與和親。檀石槐拒不肯受,寇鈔滋甚。乃分其地為中東西三部。從右北平以東至遼,東接夫餘、濊貊為東部,二十餘邑,其大人曰彌加、闕機、素利、槐頭。從右北平以西至上谷為中部,十餘邑,其大人曰柯最、闕居、慕容等,為大帥。從上谷以西至燉煌,西接烏孫為西部,二十餘邑,其大人曰置鞬落羅、曰律推演、宴荔游等,皆為大帥,而制屬檀石槐。至靈帝時,大鈔畧幽、并二州。緣邊諸郡無歲不被其毒。熹平六年,遣護烏丸校尉夏育,破鮮卑中郎將田晏,匈奴中郎將臧旻與南單于出鴈門塞,三道並進,徑二千餘里征之。檀石槐帥部衆逆擊,旻等敗走,兵馬還者什一而己。鮮卑衆日多,田畜射獵,不足給食。後檀石槐乃案行烏侯秦水,廣袤數百里,停不流,中有魚而不能得。聞汗人善捕魚,於是檀石槐東擊汗國,得千餘家,徙置烏侯秦水上,使捕魚以助糧。至于今,烏侯秦水上有汗人數百戶。檀石槐年四十五死,子和連代立。和連材力不及父,而貪淫,斷法不平,衆叛者半。靈帝末年數為寇鈔,攻北地,北地庶人善弩射者射中和連,和連即死。其子騫曼小,兄子魁頭代立。魁頭旣立後,騫曼長大,與魁頭爭國,衆遂離散。魁頭死,弟步度根代立。自檀石槐死後,諸大人遂世相襲也。}
 步度根旣立,衆稍衰弱,中兄扶羅韓亦別擁衆數萬為大人。建安中,太祖定幽州,步度根與軻比能等因烏丸校尉閻柔上貢獻。後代郡烏丸能臣氐等叛,求屬扶羅韓,扶羅韓將萬餘騎迎之。到桑乾,氏等議,以為扶羅韓部威禁寬緩,恐不見濟,更遣人呼軻比能。比能即將萬餘騎到,當共盟誓。比能便於會上殺扶羅韓,扶羅韓子泄歸泥及部衆悉屬比能。比能自以殺歸泥父,特又善遇之。步度根由是怨比能。
 
 
 
 
 文帝踐阼,田豫為烏丸校尉,持節并護鮮卑,屯昌平。步度根遣使獻馬,帝拜為王。後數與軻比能更相攻擊,步度根部衆稍寡弱,將其衆萬餘落保太原、鴈門郡。步度根乃使人招呼泄歸泥曰:「汝父為比能所殺,不念報仇,反屬怨家。今雖厚待汝,是欲殺汝計也。不如還我,我與汝是骨肉至親,豈與仇等?」由是歸泥將其部落逃歸步度根,比能追之弗及。至黃初五年,步度根詣闕貢獻,厚加賞賜,是後一心守邊,不為寇害,而軻比能衆遂彊盛。明帝即位,務欲綏和戎狄,以息征伐,羈縻兩部而已。至青龍元年,比能誘步度根深結和親,於是步度根將泄歸泥及部衆悉保比能,寇鈔并州,殺略吏民。帝遣驍騎將軍秦朗征之,歸泥叛比能,將其部衆降,拜歸義王,賜幢麾、曲蓋、鼓吹,居并州如故。步度根為比能所殺。
 
 
\end{pinyinscope}