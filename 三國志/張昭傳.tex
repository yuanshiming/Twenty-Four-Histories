\article{張昭傳}
\begin{pinyinscope}
 
 
 張昭字子布,彭城人也。少好學,善隷書,從白侯子安受左氏春秋,博覽衆書,與琅邪趙昱、東海王朗俱發名友善。弱冠察孝廉,不就,與朗共論舊君諱事,州里才士陳琳等皆稱善之。
 
 
\gezhu{時汝南主簿應劭議宜為舊君諱,論者皆互有異同,事在風俗通。昭著論曰:「客有見大國之議,士君子之論,云起元建武已來,舊君名諱五十六人,以為後生不得協也。取乎經論,譬諸行事,義高辭麗,甚可嘉羨。愚意褊淺,竊有疑焉。蓋乾坤剖分,萬物定形,肇有父子君臣之經。故聖人順天之性,制禮尚敬,在三之義,君實食之,在喪之哀,君親臨之,厚莫重焉,恩莫大焉,誠臣子所尊仰,萬夫所天恃,焉得而同之哉?然親親有衰,尊尊有殺,故禮服上不盡高祖,下不盡玄孫。又傳記四世而緦麻,服之窮也;五世袒免,降殺同姓也;六世而親屬竭矣。又曲禮有不逮事之義則不諱,不諱者,蓋名之謂,屬絕之義,不拘於協,況乃古君五十六哉!邾子會盟,季友來歸,不稱其名,咸書字者,是時魯人嘉之也。何解臣子為君父諱乎?周穆王諱滿,至定王時有王孫滿者,其為大夫,是臣協君也。又厲王諱胡,及莊王之子名胡,其比衆多。夫類事建議,經有明據,傳有徵案,然後進攻退守,萬無奔北,垂示百世,永無咎失。今應劭雖上尊舊君之名,而下無所斷齊,猶歸之疑云。曲禮之篇,疑事無質,觀省上下,闕義自證,文辭可為,倡而不法,將來何觀?言聲一放,猶拾瀋也,過辭在前,悔其何追!」}
 刺史陶謙舉茂才,不應,謙以為輕己,遂見拘執。昱傾身營救,方以得免。漢末大亂,徐方士民多避難揚土,昭皆南渡江。孫策創業,命昭為長史、撫軍中郎將,升堂拜母,如比肩之舊,文武之事,一以委昭。
 \gezhu{吳書曰:策得昭甚恱,謂曰:「吾方有事四方,以士人賢者上,吾於子不得輕矣。」乃上為校尉,待以師友之禮。}
 昭每得北方士大夫書疏,專歸美於昭,昭欲嘿而不宣則懼有私,宣之則恐非宜,進退不安。策聞之,歡笑曰:「昔管子相齊,一則仲父,二則仲父,而桓公為霸者宗。今子布賢,我能用之,其功名獨不在我乎!」
 
 
策臨亡,以弟權託昭,昭率羣僚立而輔之。
 \gezhu{吳歷曰:策謂昭曰:「若仲謀不任事者,君便自取之。正復不克捷,緩步西歸,亦無所慮。」}
 上表漢室,下移屬城,中外將校,各令奉職。權悲感未視事,昭謂權曰:「夫為人後者,貴能負荷先軌,克昌堂構,以成勳業也。方今天下鼎沸,羣盜滿山,孝廉何得寢伏哀戚,肆匹夫之情哉?」乃身自扶權上馬,陳兵而出,然後衆心知有所歸。昭復為權長史,授任如前。
 \gezhu{吳書曰:是時天下分裂,擅命者衆。孫策蒞事日淺,恩澤未洽,一旦傾隕,士民狼狽,頗有同異。及昭輔權,綏撫百姓,諸侯賔旅寄寓之士,得用自安。權每出征,留昭鎮守,領幕府事。後黃巾賊起,昭討平之。權征合肥,命昭別討匡琦,又督領諸將,攻破豫章賊率周鳳等於南城。自此希復將帥,常在左右,為謀謨臣。權以昭舊臣,待遇尤重。}
 後劉備表權行車騎將軍,昭為軍師。權每田獵,常乘馬射虎,虎嘗突前攀持馬鞌。昭變色而前曰:「將軍何有當爾?夫為人君者,謂能駕御英雄,驅使羣賢,豈謂馳逐於原野,校勇於猛獸者乎?如有一旦之患,柰天下笑何?」權謝昭曰:「年少慮事不遠,以此慙君。」然猶不能已,乃作射虎車,為方目,閒不置蓋,一人為御,自於中射之。時有逸羣之獸,輒復犯車,而權每手擊以為樂。昭雖諫爭,常笑而不荅。
 
 
魏黃初二年,遣使者邢貞拜權為吳王。貞入門,不下車。昭謂貞曰:「夫禮無不敬,故法無不行。而君敢自尊大,豈以江南寡弱,無方寸之刃故乎!」貞即遽下車。拜昭為綏遠將軍,封由拳侯。
 \gezhu{吳錄曰:昭與孫紹、滕胤、鄭禮等,採周、漢,撰定朝儀。}
 權於武昌,臨釣臺,飲酒大醉。權使人以水灑羣臣曰:「今日酣飲,惟醉墮臺中,乃當止耳。」昭正色不言,出外車中坐。權遣人呼昭還,謂曰:「為共作樂耳,公何為怒乎?」昭對曰:「昔紂為糟丘酒池長夜之飲,當時亦以為樂,不以為惡也。」權默然,有慙色,遂罷酒。初,權當置丞相,衆議歸昭。權曰:「方今多事,職統者責重,非所以優之也。」後孫邵卒,百寮復舉昭,權曰:「孤豈為子布有愛乎?領丞相事煩,而此公性剛,所言不從,怨咎將興,非所以益之也。」乃用顧雍。
 
 
權旣稱尊號,昭以老病,上還官位及所統領。
 \gezhu{江表傳曰:權旣即尊位,請會百官,歸功周瑜。昭舉笏欲襃贊功德,未及言,權曰:「如張公之計,今已乞食矣。」昭大慙,伏地流汗。昭忠謇亮直,有大臣節,權敬重之,然所以不相昭者,蓋以昔駮周瑜、魯肅等議為非也。臣松之以為張昭勸迎曹公,所存豈不遠乎?夫其揚休正色,委質孫氏,誠以厄運初遘,塗炭方始,自策及權,才略足輔,是以盡誠匡弼,以成其業,上藩漢室,下保民物;鼎峙之計,本非其志也。曹公仗順而起,功以義立,兾以清一諸華,拓平荊郢,大定之機,在於此會。若使昭議獲從,則六合為一,豈有兵連禍結,遂為戰國之弊哉!雖無功於孫氏,有大當於天下矣。昔竇融歸漢,與國升降;張魯降魏,賞延于世。況權舉全吳,望風順服,寵靈之厚,其可測量哉!然則昭為人謀,豈不忠且正乎!}
 更拜輔吳將軍,班亞三司,改封婁侯,食邑萬戶。在里宅無事,乃著春秋左氏傳解及論語注。權嘗問衞尉嚴畯:「寧念小時所闇書不」?畯因誦孝經「仲尼居」。昭曰:「嚴畯鄙生,臣請為陛下誦之。」乃誦「君子之事上」,咸以昭為知所誦。
 
 
 
 
 昭每朝見,辭氣壯厲,義形於色,曾以直言逆旨,中不進見。後蜀使來,稱蜀德美,而羣臣莫拒,權歎曰:「使張公在坐,彼不折則廢,安復自誇乎?」明日,遣中使勞問,因請見昭。昭避席謝,權跪止之。昭坐定,仰曰:「昔太后、桓王不以老臣屬陛下,而以陛下屬老臣,是以思盡臣節,以報厚恩,使泯沒之後,有可稱述,而意慮淺短,違逆盛旨,自分幽淪,長棄溝壑,不圖復蒙引見,得奉帷幄。然臣愚心所以事國,志在忠益,畢命而已。若乃變心易慮,以偷榮取容,此臣所不能也。」權辭謝焉。
 
 
權以公孫淵稱藩,遣張彌、許晏至遼東拜淵為燕王,昭諫曰:「淵背魏懼討,遠來求援,非本志也。若淵改圖,欲自明於魏,兩使不反,不亦取笑於天下乎?」權與相反覆,昭意彌切。權不能堪,案刀而怒曰:「吳國士人入宮則拜孤,出宮則拜君,孤之敬君,亦為至矣,而數於衆中折孤,孤嘗恐失計。」昭孰視權曰:「臣雖知言不用,每竭愚忠者,誠以太后臨崩,呼老臣於牀下,遺詔顧命之言故在耳。」因涕泣橫流。權擲刀致地,與昭對泣。然卒遣彌、晏往。昭忿言之不用,稱疾不朝。權恨之,土塞其門,昭又於內以土封之。淵果殺彌、晏。權數慰謝昭,昭固不起,權因出過其門呼昭,昭辭疾篤。權燒其門,欲以恐之,昭更閉戶。權使人滅火,住門良乆,昭諸子共扶昭起,權載以還宮,深自克責。昭不得已,然後朝會。
 \gezhu{習鑿齒曰:張昭於是乎不臣矣!夫臣人者,三諫不從則奉身而退,身苟不絕,何忿懟之有?且秦穆違諫,卒霸西戎,晉文暫怒,終成大業。遺誓以悔過見錄,狐偃無怨絕之辭,君臣道泰,上下俱榮。今權悔往之非而求昭,後益迴慮降心,不遠而復,是其善也。昭為人臣,不度權得道,匡其後失,夙夜匪懈,以延來譽,乃追忿不用,歸罪於君,閉戶拒命,坐待焚滅,豈不悖哉!}
 
 
昭容貌矜嚴,有威風,權常曰:「孤與張公言,不敢妄也。」舉邦憚之。年八十一,嘉禾五年卒。遺令幅巾素棺,歛以時服。權素服臨弔,謚曰文侯。
 \gezhu{典略曰:余曩聞劉荊州甞自作書欲與孫伯符,以示禰正平,正平蚩之,言:「如是為欲使孫策帳下兒讀之邪,將使張子布見乎?」如正平言,以為子布之才高乎?雖然,猶自蘊藉典雅,不可謂之無筆迹也。加聞吳中稱謂之仲父,如此,其人信一時之良幹,恨其不於嵩岳等資,而乃播殖於會稽。}
 長子承已自封侯,少子休襲爵。
 
 
 
 
 昭弟子奮年二十,造作攻城大攻車,為步隲所薦。昭不願曰:「汝年尚少,何為自委於軍旅乎?」奮對曰:「昔童汪死難,子奇治阿,奮實不才耳,於年不為少也。」遂領兵為將軍,連有功效,至平州都督,封樂鄉亭侯。
 
 
承字仲嗣,少以才學知名,與諸葛瑾、步隲、嚴畯相友善。權為驃騎將軍,辟西曹掾,出為長沙西部都尉。討平山寇,得精兵萬五千人。後為濡須都督、奮威將軍,封都鄉侯,領部曲五千人,承為人壯毅忠讜,能甄識人物,拔彭城蔡款、南陽謝景於孤微童幼,後並為國士,款至衞尉,景豫章太守。
 \gezhu{吳錄曰:款字文德,歷位內外,以清貞顯於當世。後以衞尉領中書令,封留侯。二子,條、機。條孫皓時位至尚書令、太子少傅。機為臨川太守。謝景事在孫登傳。}
 又諸葛恪年少時,衆人奇其英才,承言終敗諸葛氏者元遜也。勤於長進,篤於物類,凡在庶幾之流,無不造門。年六十七,赤烏七年卒,謚曰定侯。子震嗣。初,承喪妻,昭欲為索諸葛瑾女,承以相與有好,難之,權聞而勸焉,遂為婚。
 \gezhu{臣松之案:承與諸葛瑾同以赤烏中卒,計承年小瑾四歲耳。}
 生女,權為子和納之。權數令和脩敬於承,執子壻之禮。震,諸葛恪誅時亦死。
 
 
休字叔嗣,弱冠與諸葛恪、顧譚等俱為太子登僚友,以漢書授登。
 \gezhu{吳書曰:休進授,指摘文義,分別事物,並有章條。每升堂宴飲,酒酣樂作,登輒降意與同歡樂。休為人解達,登甚愛之,常在左右。}
 從中庶子轉為右弼都尉。權嘗游獵,迨暮乃歸,休上疏諫戒,權大善之,以示於昭。及登卒後,為侍中,拜羽林都督,平三典軍事,遷揚武將軍。為魯王霸友黨所譖,與顧譚、承俱以芍陂論功事,休、承與典軍陳恂通情,詐增其伐,並徙交州。中書令孫弘佞偽險詖,休素所忿,
 \gezhu{吳錄云:弘,會稽人也。}
 弘因是譖訴,下詔書賜休死,時年四十一。
 
 
\end{pinyinscope}