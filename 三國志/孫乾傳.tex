\article{孫乾傳}
\begin{pinyinscope}
 
 
 孫乾字公祐,北海人也。先主領徐州,辟為從事,
 
 
\gezhu{鄭玄傳云:玄薦乾於州。乾被辟命,玄所舉也。}
 後隨從周旋。先主之背曹公,遣乾自結袁紹,將適荊州,乾又與麋笁俱使劉表,皆如意指。後表與袁尚書,說其兄弟分爭之變,曰:「每與劉左將軍、孫公祐共論此事,未甞不痛心入骨,相為悲傷也。」其見重如此。先主定益州,乾自從事中郎為秉忠將軍,見禮次麋笁,與簡雍同等。頃之,卒。
 
 
\end{pinyinscope}