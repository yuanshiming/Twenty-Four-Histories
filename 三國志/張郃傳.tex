\article{張郃傳}
\begin{pinyinscope}
 
 
 張郃字儁乂,河閒鄚人也。漢末應募討黃巾,為軍司馬,屬韓馥。馥敗,以兵歸袁紹。紹以郃為校尉,使拒公孫瓚。瓚破,郃功多,遷寧國中郎將。太祖與袁紹相拒於官渡,
 
 
\gezhu{漢晉春秋曰:郃說紹曰:「公雖連勝,然勿與曹公戰也,密遣輕騎鈔絕其南,則兵自敗矣。」紹不從之。}
 紹遣將淳于瓊等督運屯烏巢,太祖自將急擊之。郃說紹曰:「曹公兵精,往必破瓊等;瓊等破,則將軍事去矣,宜急引兵救之。」郭圖曰:「郃計非也。不如攻其本營,勢必還,此為不救而自解也。」郃曰:「曹公營固,攻之必不拔,若瓊等見禽,吾屬盡為虜矣。」紹但遣輕騎救瓊,而以重兵攻太祖營,不能下。太祖果破瓊等,紹軍潰。圖慙,又更譖郃曰:「郃快軍敗,出言不遜。」郃懼,乃歸太祖。
 \gezhu{臣松之案武紀及袁紹傳並云袁紹使張郃、高覽攻太祖營,郃等聞淳于瓊破,遂來降,紹衆於是大潰。是則緣郃等降而後紹軍壞也。至如此傳,為紹軍先潰,懼郭圖之譖,然後歸太祖,為參錯不同矣。}
 
 
太祖得郃甚喜,謂曰:「昔子胥不早寤,自使身危,豈若微子去殷、韓信歸漢邪?」拜郃偏將軍,封都亭侯。授以衆,從攻鄴,拔之。又從擊袁譚於渤海,別將軍圍雍奴,大破之。從討柳城,與張遼俱為軍鋒,以功遷平狄將軍。別征東萊,討管承,又與張遼討陳蘭、梅成等,破之。從破馬超、韓遂於渭南。圍安定,降楊秋。與夏侯淵討鄜賊梁興及武都氐。又破馬超,平宋建。太祖征張魯,先遣郃督諸軍討興和氐王竇茂。太祖從散關入漢中,又先遣郃督步卒五千於前通路。至陽平,魯降,太祖還,留郃與夏侯淵等守漢中,拒劉備。郃別督諸軍,降巴東、巴西二郡,徙其民於漢中。進軍宕渠,為備將張飛所拒,引還南鄭。拜盪寇將軍。劉備屯陽平,郃屯廣石。備以精卒萬餘,分為十部,夜急攻郃。郃率親兵搏戰,備不能克。其後備於走馬谷燒都圍,淵救火,從他道與備相遇,交戰,短兵接刃。淵遂沒,郃還陽平。
 \gezhu{魏略曰:淵雖為都督,劉備憚郃而易淵。及殺淵,備曰:「當得其魁,用此何為邪!」}
 當是時,新失元帥,恐為備所乘,三軍皆失色。淵司馬郭淮乃令衆曰:「張將軍,國家名將,劉備所憚;今日事急,非張將軍不能安也。」遂推郃為軍主。郃出,勒兵安陣,諸將皆受郃節度,衆心乃定。太祖在長安,遣使假郃節。太祖遂自至漢中,劉備保高山不敢戰。太祖乃引出漢中諸軍,郃還屯陳倉。
 
 
 
 
 文帝即王位,以郃為左將軍,進爵都鄉侯。及踐阼,進封鄚侯。詔郃與曹真討安定盧水胡及東羌,召郃與真並朝許宮,遣南與夏侯尚擊江陵。郃別督諸軍渡江,取洲上屯塢。明帝即位,遣南屯荊州,與司馬宣王擊孫權別將劉阿等,追至祁口,交戰,破之。諸葛亮出祁山。加郃位特進,遣督諸軍,拒亮將馬謖於街亭。謖依阻南山,不下據城。郃絕其汲道,擊,大破之。南安、天水、安定郡反應亮,郃皆破平之。詔曰:「賊亮以巴蜀之衆,當虓虎之師。將軍被堅執銳,所向克定,朕甚嘉之。益邑千戶,并前四千三百戶。」司馬宣王治水軍於荊州,欲順沔入江伐吳,詔郃督關中諸軍往受節度。至荊州,會冬水淺,大船不得行,乃還屯方城。諸葛亮復出,急攻陳倉,帝驛馬召郃到京都。帝自幸河南城,置酒送郃,遣南北軍士三萬及分遣武衞、虎賁使衞郃,因問郃曰:「遲將軍到,亮得無已得陳倉乎!」郃知亮縣軍無穀,不能乆攻,對曰:「比臣未到,亮已走矣;屈指計亮糧不至十日。」郃晨夜進至南鄭,亮退。詔郃還京都,拜征西車騎將軍。
 
 
 
 
 郃識變數,善處營陣,料戰勢地形,無不如計,自諸葛亮皆憚之。郃雖武將而愛樂儒士,嘗薦同鄉卑湛經明行脩,詔曰:「昔祭遵為將,奏置五經大夫,居軍中,與諸生雅歌投壺。今將軍外勒戎旅,內存國朝。朕嘉將軍之意,今擢湛為博士。」
 
 
諸葛亮復出祁山,詔郃督諸將西至略陽,亮還保祁山,郃追至木門,與亮軍交戰,飛矢中郃右膝,薨,
 \gezhu{魏略曰:亮軍退,司馬宣王使郃追之,郃曰:「軍法,圍城必開出路,歸軍勿追。」宣王不聽。郃不得已,遂進。蜀軍乘高布伏,弓弩亂發,矢中郃髀。}
 謚曰壯侯。子雄嗣。郃前後征伐有功,明帝分郃戶,封郃四子列侯。賜小子爵關內侯。
 
 
\end{pinyinscope}