\article{孫亮傳}
\begin{pinyinscope}
 
 
 孫亮字子明,權少子也。權春秋高,而亮最少,故尤留意。姊全公主嘗譖太子和子母,心不自安,因倚權意,欲豫自結,數稱述全尚女,勸為亮納。赤烏十三年,和廢,權遂立亮為太子,以全氏為妃。
 
 
 
 
 太元元年夏,亮母潘氏立為皇后。冬,權寢疾,徵大將軍諸葛恪為太子太傅,會稽太守滕胤為太常,並受詔輔太子。明年四月,權薨,太子即尊號,大赦,改。是歲,於魏嘉平四年也。
 
 
 
 
 閏月,以恪為帝太傅,胤為衞將軍領尚書事,上大將軍呂岱為大司馬,諸文武在位皆進爵班賞,宂官加等。冬十月,太傅恪率軍遏巢湖,
 
 
\gezhu{巢音祖了反。}
 城東興,使將軍全端守西城,都尉留略守東城。十二月朔丙申,大風雷電,魏使將軍諸葛誕、胡遵等步騎七萬圍東興,將軍王昶攻南郡,毌丘儉向武昌。甲寅,恪以大兵赴敵。戊午,兵及東興,交戰,大破魏軍,殺將軍韓綜、桓嘉等。是月,雷雨,天災武昌端門;改作端門,又災內殿。
 \gezhu{臣松之案:孫權赤烏十年,詔徙武昌宮材瓦,以繕治建康宮,而此猶有端門內殿。吳錄云:諸葛恪有遷都意,更起武昌宮。今所災者恪所新作。}
 
 
 
 
 二年春正月丙寅,立皇后全氏,大赦。庚午,王昶等皆退。二月,軍還自東興,大行封賞。三月,恪率軍伐魏。夏四月,圍新城,大疫,兵卒死者大半。秋八月,恪引軍還。冬十月,大饗。武衞將軍孫峻伏兵殺恪於殿堂。大赦。以峻為丞相,封富春侯。十一月,有大鳥五見于春申,明年改元。
 
 
五鳳元年夏,大水。秋,吳侯英謀殺峻,覺,英自殺。冬十一月,星茀于斗、牛。
 \gezhu{江表傳曰:是歲交阯稗草化為稻。}
 
 
 
 
 二年春正月,魏鎮東將軍毌丘儉、前將軍文欽以淮南之衆西入,戰于樂嘉。閏月壬辰,峻及驃騎將軍呂據、左將軍留贊率兵襲壽春,軍及東興,聞欽等敗。壬寅,兵進于橐臯,欽詣峻降,淮南餘衆數萬口來奔。魏諸葛誕入壽春,峻引軍還。二月,及魏將軍曹珍遇于高亭,交戰,珍敗績。留贊為誕別將蔣班所敗於菰陂,贊及將軍孫楞、蔣脩等皆遇害。三月,使鎮南將軍朱異襲安豐,不克。秋七月,將軍孫儀、張怡、林恂等謀殺峻,發覺,儀自殺,恂等伏辜。陽羨離里山大石自立。使衞尉馮朝城廣陵,拜將軍吳穰為廣陵太守,留略為東海太守。是歲大旱。十二月,作太廟。以馮朝為監軍使者,督徐州諸軍事,民饑,軍士怨畔。
 
 
太平元年春二月朔,
 \gezhu{吳歷曰:正月,為權立廟,稱太祖廟。}
 建業火。峻用征北大將軍文欽計,將征魏。八月,先遣欽及驃騎呂據、車騎劉纂、鎮南朱異、前將軍唐咨軍自江都入淮、泗。九月丁亥,峻卒,以從弟偏將軍綝為侍中、武衞將軍,領中外諸軍事,召還據等。據聞綝代峻,大怒。己丑,大司馬呂岱卒。壬辰,太白犯南斗。據、欽、咨等表薦衞將軍滕胤為丞相,綝不聽。癸卯,更以胤為大司馬,代呂岱駐武昌。據引兵還,欲討綝。綝遣使以詔書告喻欽、咨等,使取據。冬十月丁未,遣孫憲及丁奉、施寬等以舟兵逆據於江都,遣將軍劉丞督步騎攻胤。胤兵敗夷滅。己酉,大赦,改年。辛亥,獲呂據於新州。十一月,以綝為大將軍、假節,封永康侯。孫憲與將軍王惇謀殺綝,事覺,綝殺惇,迫憲令自殺。十二月,使五官中郎將刁玄告亂于蜀。
 
 
二年春二月甲寅,大雨,震電。乙卯,雪,大寒。以長沙東部為湘東郡,西部為衡陽郡,會稽東部為臨海郡,豫章東部為臨川郡。夏四月,亮臨正殿,大赦,始親政事。綝所表奏,多見難問,又科兵子弟年十八已下十五已上,得三千餘人,選大將子弟年少有勇力者為之將帥。亮曰:「吾立此軍,欲與之俱長。」日於菀中習焉。
 \gezhu{吳歷曰:亮數出中書視孫權舊事,問左右侍臣:「先帝數有特制,今大將軍問事,但令我書可邪!」亮後出西苑,方食生梅,使黃門至中藏取蜜漬梅,蜜中有鼠矢,召問藏吏,藏吏叩頭。亮問吏曰:「黃門從汝求蜜邪?」吏曰:「向求,實不敢與。」黃門不服,侍中刁玄、張邠啟:「黃門、藏吏辭語不同,請付獄推盡。」亮曰:「此易知耳。」令破鼠矢,矢裏燥。亮大笑謂玄、邠曰:「若矢先在蜜中,中外當俱溼,今外溼裏燥,必是黃門所為。」黃門首服,左右莫不驚悚。江表傳曰:亮使黃門以銀碗并蓋就中藏吏取交州所獻甘蔗餳。黃門先恨藏吏,以鼠矢投餳中,啟言藏吏不謹。亮呼吏持餳器入,問曰:「此器旣蓋之,且有掩覆,無緣有此,黃門將有恨於汝邪?」吏叩頭曰:「嘗從某求宮中莞席,宮席有數,不敢與。」亮曰:「必是此也。」覆問黃門,具首伏。即於目前加髠鞭,斥付外署。臣松之以為鼠矢新者,亦表裏皆溼。黃門取新矢則無以得其姦也,緣遇燥矢,故成亮之惠。然猶謂吳曆此言,不如江表傳為實也。}
 
 
 
 
 五月,魏征東大將軍諸葛誕以淮南之衆保壽春城,遣將軍朱成稱臣上疏,又遣子靚、長史吳綱諸牙門子弟為質。六月,使文欽、唐咨、全端等步騎三萬救誕。朱異自虎林率衆襲夏口,夏口督孫壹奔魏。秋七月,綝率衆救壽春,次于鑊里,朱異至自夏口,綝使異為前部督,與丁奉等將介士五萬解圍。八月,會稽南部反,殺都尉。鄱陽、新都民為亂,廷尉丁密、步兵校尉鄭冑、將軍鍾離牧率軍討之。朱異以軍士乏食引還,綝大怒,九月朔己巳,殺異於鑊里。辛未,綝自鑊里還建業。甲申,大赦。十一月,全緒子禕、儀以其母奔魏。十二月,全端、懌等自壽春城詣司馬文王。
 
 
 
 
 三年春正月,諸葛誕殺文欽。三月,司馬文王克壽春,誕及左右戰死,將吏已下皆降。秋七月,封故齊王奮為章安侯。詔州郡伐宮材。自八月沈陰不雨四十餘日。亮以綝專恣,與太常全尚,將軍劉丞謀誅綝。九月戊午,綝以兵取尚,遣弟恩攻殺丞於蒼龍門外,召大臣會宮門,黜亮為會稽王,時年十六。
 
 
\end{pinyinscope}