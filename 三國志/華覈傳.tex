\article{華覈傳}
\begin{pinyinscope}
 
 
 華覈字永先,吳郡武進人也。始為上虞尉、典農都尉,以文學入為祕府郎,遷中書丞。
 
 
 
 
 蜀為魏所并,覈詣宮門發表曰:「閒聞賊衆蟻聚向西境,西境艱險,謂當無虞。定聞陸抗表至,成都不守,臣主播越,社稷傾覆。昔衞為翟所滅而桓公存之,今道里長遠,不可救振,失委附之土,棄貢獻之國,臣以草芥,竊懷不寧。陛下聖仁,恩澤遠撫,卒聞如此,必垂哀悼。臣不勝忡悵之情,謹拜表以聞。」
 
 
 
 
 孫皓即位,封徐陵亭侯。寶鼎二年,皓更營新宮,制度弘廣,飾以珠玉,所費甚多。是時盛夏興工,農守並廢,覈上疏諫曰:
 
 
 
 
 臣聞漢文之世,九州晏然,秦民喜去慘毒之苛政,歸劉氏之寬仁,省役約法,與之更始,分王子弟以藩漢室,當此之時,皆以為泰山之安,無窮之基也。至於賈誼,獨以為可痛哭及流涕者三,可為長嘆息者六,乃曰當今之勢何異抱火於積薪之下而寢其上,火未及然而謂之安。其後變亂,皆如其言。臣雖下愚,不識大倫,竊以曩時之事,揆今之勢。
 
 
 
 
 誼曰復數年閒,諸王方剛,漢之傅相稱疾罷歸,欲以此為治,雖堯舜不能安。今大敵據九州之地,有太半之衆,習攻戰之餘術,乘戎馬之舊勢,欲與中國爭相吞之計,其猶楚漢勢不兩立,非徒漢之諸王淮南、濟北而已。誼之所欲痛哭,比今為緩,抱火卧薪之喻,於今而急。大皇帝覽前代之如彼,察今勢之如此,故廣開農桑之業,積不訾之儲,恤民重役,務養戰士,是以大小感恩,各思竭命。期運未至,早棄萬國。自是之後,彊臣專政,上詭天時,下違衆議,亡安存之本,邀一時之利,數興軍旅,傾竭府藏,兵勞民困,無時獲安。今之存者乃創夷之遺衆,哀苦之餘民耳。遂使軍資空匱,倉廩不實,布帛之賜,寒暑不周,重以失業,家戶不贍。而北積穀養民,專心東向,無復他警。蜀為西藩,土地險固,加承先主統御之術,謂其守御足以長乆,不圖一朝,奄至傾覆。脣亡齒寒,古人所懼。交州諸郡,國之南土,交阯、九真二郡已沒,日南孤危,存亡難保,合浦以北,民皆搖動,因連避役,多有離叛,而備戍減少,威鎮轉輕,常恐呼吸復有變故。昔海虜窺窬東縣,多得離民,地習海行,狃於往年,鈔盜無日,今胷背有嫌,首尾多難,乃國朝之厄會也。誠宜住建立之役,先備豫之計,勉墾殖之業,為饑乏之救。惟恐農時將過,東作向晚,有事之日,整嚴未辦。若舍此急,盡力功作,卒有風塵不虞之變,當委版築之役,應烽燧之急,驅怨苦之衆,赴白刃之難,此乃大敵所因為資也。如但固守,曠日持乆,則軍糧必乏,不待接刃,而戰士已困矣。
 
 
 
 
 昔太戊之時,桑穀生庭,懼而脩德,怪消殷興。熒惑守心,宋以為災,景公下從瞽史之言,而熒惑退舍,景公延年。夫脩德於身而感異類,言發於口而通神明,臣以愚蔽,誤忝近署,不能翼宣仁澤以感靈祇,仰慙俯愧,無所投處。退伏思惟,熒惑桑穀之異,天示二主,至於他餘錙介之妖,近是門庭小神所為,驗之天地,無有他變,而徵祥符瑞前後屢臻,明珠旣覿,白雀繼見,萬億之祚,實靈所挺,以九域為宅,天下為家,不與編戶之民轉徙同也。又今之宮室,先帝所營,卜土立基,非為不祥。又楊市土地與宮相接,若大功畢竟,輿駕遷住,門行之神,皆當轉移,猶恐長乆未必勝舊。屢遷不可,留則有嫌,此乃愚臣所以夙夜為憂灼也。臣省月令,季夏之月,不可以興土功,不可以會諸侯,不可以起兵動衆,舉大事必有大殃。今雖諸侯不會,諸侯之軍與會無異。六月戊己,土行正王,旣不可犯,加又農月,時不可失。昔魯隱公夏城中丘,春秋書之,垂為後戒。今築宮為長世之洪基,而犯天地之大禁,襲春秋之所書,廢敬授之上務,臣以愚管,竊所未安。
 
 
 
 
 又恐所召離民,或有不至,討之則廢役興事,不討則日月滋慢。若悉並到,大衆聚會,希無疾病。且人心安則念善,苦則怨叛。江南精兵,北土所難,欲以十卒當東一人。天下未定,深可憂惜之。如此宮成,死叛五千,則北軍之衆更增五萬,若到萬人,則倍益十萬,病者有死亡之損,叛者傳不善之語,此乃大敵所以歡喜也。今當角力中原,以定彊弱,正於際會,彼益我損,加以勞困,此乃雄夫智士所以深憂。
 
 
 
 
 臣聞先王治國無三年之儲,曰國非其國,安寧之世戒備如此,況敵彊大而忽農忘畜。今雖頗種殖,閒者大水沈沒,其餘存者當須耘穫,而長吏怖期,上方諸郡,身涉山林,盡力伐材,廢農棄務,士民妻孥羸小,墾殖又薄,若有水旱則永無所獲。州郡見米,當待有事,冗食之衆,仰官供濟。若上下空乏,運漕不供,而北敵犯疆,使周、召更生,良、平復出,不能為陛下計明矣。臣聞君明者臣忠,主聖者臣直,是以慺慺,昧犯天威,乞垂哀省。
 
 
 
 
 書奏,皓不納。後遷東觀令,領右國史,覈上疏辭讓,皓荅曰:「得表,以東觀儒林之府,當講校文藝,處定疑難,漢時皆名學碩儒乃任其職,乞更選英賢。聞之,以卿研精墳典,博覽多聞,可謂恱禮樂敦詩書者也。當飛翰騁藻,光贊時事,以越楊、班、張、蔡之疇,怪乃謙光,厚自菲薄,宜勉脩所職,以邁先賢,勿復紛紛。」
 
 
 
 
 時倉廩無儲,世務滋侈,覈上疏曰:「今寇虜充斥,征伐未已,居無積年之儲,出無應敵之畜,此乃有國者所宜深憂也。夫財穀所生,皆出於民,趨時務農,國之上急。而都下諸官,所掌別異,各自下調,不計民力,輒與近期。長吏畏罪。晝夜催民,委舍佃事,遑赴會日,定送到都,或蘊積不用,而徒使百姓消力失時。到秋收月,督其限入,奪其播殖之時,而責其今年之稅,如有逋懸,則籍沒財物,故家戶貧困,衣食不足。宜暫息衆役,專心農桑,古人稱一夫不耕,或受其饑,一女不織,或受其寒,是以先王治國,惟農是務。軍興以來,已向百載,農人廢南畝之務,女工停機杼之業。推此揆之,則蔬食而長饑,薄衣而履冰者,固不少矣。臣聞主之所求於民者二,民之所望於主者三。二謂求其為己勞也,求其為己死也。三謂饑者能食之,勞者能息之,有功者能賞之。民以致其二事而主失其三望者,則怨心生而功不建。今帑藏不實,民勞役猥,主之二求已備,民之三望未報。且饑者不待美饌而後飽,寒者不俟狐狢而後溫,為味者口之奇,文繡者身之飾也。今事多而役繁,民貧而俗奢,百工作無用之器,婦人為綺靡之飾,不勤麻枲,並繡文黼黻,轉相倣效,恥獨無有。兵民之家,猶復逐俗,內無儋石之儲,而出有綾綺之服,至於富賈商販之家,重以金銀,奢恣尤甚。天下未平,百姓不贍,宜一生民之原,豐穀帛之業,而棄功於浮華之巧,妨日於侈靡之事,上無尊卑等級之差,下有耗財費力之損。今吏士之家,少無子女,多者三四,少者一二,通令戶有一女,十萬家則十萬人,人織績一歲一束,則十萬束矣。使四疆之內同心戮力,數年之間,布帛必積。恣民五色,惟所服用,但禁綺繡無益之飾。且美貌者不待華采以崇好,豔姿者不待文綺以致愛,五采之飾,足以麗矣。若極粉黛,窮盛服,未必無醜婦;廢華采,去文繡,未必無美人也,若實如論,有之無益廢之無損者,何愛而不暫禁以充府藏之急乎?此救乏之上務,富國之本業也,使筦、晏復生,無以易此。漢之文、景,承平繼統,天下已定,四方無虞,猶以彫文之妨農事,錦繡之害女紅,開富國之利,杜饑寒之本。況今六合分乖,犲狼充路,兵不離疆,甲不解帶,而可以不廣生財之原,充府藏之積哉?」
 
 
 
 
 皓以覈年老,勑令草表,覈不敢。又勑作草文,停立待之。覈為文曰:「咨覈小臣,草芥凡庸。遭眷值聖,受恩特隆。越從朽壤,蟬蛻朝中。熈光紫闥,青璅是憑。毖挹清露,沐浴凱風。效無絲氂,負闕山崇。滋潤含垢,恩貸累重。穢質被榮,局命得融。欲報罔極,委之皇穹。聖恩雨注,哀棄其尤。猥命草對,潤被下愚。不敢違勑,懼速罪誅。冐承詔命,魂逝形留。」
 
 
 
 
 覈前後陳便宜,及貢薦良能,解釋罪過,書百餘上,皆有補益,文多不悉載。天冊元年以微譴免,數歲卒。曜、覈所論事章疏,咸傳於世也。
 
 
 
 
 評曰:薛瑩稱王蕃器量綽異,弘博多通;樓玄清白節操,才理條暢;賀邵厲志高潔,機理清要;韋曜篤學好古,博見羣籍,有記述之才。胡冲以為玄、邵、蕃一時清妙,略無優劣。必不得已,玄宜在先,邵當次之。華覈文賦之才,有過於曜,而典誥不及也。予觀覈數獻良規,期於自盡,庶幾忠臣矣。然此數子,處無妄之世而有名位,強死其理,得免為幸耳。
 
 
\end{pinyinscope}