\article{甘寧傳}
\begin{pinyinscope}
 
 
 甘寧字興霸,巴郡臨江人也。
 
 
\gezhu{吳書曰:寧本南陽人,其先客於巴郡。寧為吏舉計掾,補蜀郡丞,頃之,棄官歸家。}
 少有氣力,好游俠,招合輕薄少年,為之渠帥;羣聚相隨,挾持弓弩,負毦帶鈴,民聞鈴聲,即知是寧。
 \gezhu{吳書曰:寧輕俠殺人,藏舍亡命,聞於郡中。其出入,步則陳車騎,水則連輕舟,侍從被文繡,所如光道路,住止常以繒錦維舟,去或割棄,以示奢也。}
 人與相逢,及屬城長吏,接待隆厚者乃與交歡;不爾,即放所將奪其資貨,於長吏界中有所賊害,作其發負,至二十餘年。止不攻劫,頗讀諸子,乃往依劉表,因居南陽,不見進用,後轉託黃祖,祖又以凡人畜之。
 \gezhu{吳書曰:寧將僮客八百人就劉表。表儒人,不習軍事。時諸英豪各各起兵,寧觀表事勢,終必無成,恐一朝土崩,并受其禍,欲東入吳。黃祖在夏口,軍不得過,乃留依祖,三年不禮之。權討祖,祖軍敗奔走,追兵急,寧以善射,將兵在後,射殺校尉凌操。祖旣得免,軍罷還營,待寧如初。祖都督蘇飛數薦寧,祖不用,令人化誘其客,客稍亡。寧欲去,恐不獲免,獨憂悶不知所出。飛知其意,乃要寧,為之置酒,謂曰:「吾薦子者數矣,主不能用。日月逾邁,人生幾何,宜自遠圖,庶遇知己。」寧良乆乃曰:「雖有其志,未知所由。」飛曰:「吾欲白子為邾長,於是去就,孰與臨阪轉丸乎?」寧曰:「幸甚。」飛白祖,聽寧之縣。招懷亡客并義從者,得數百人。}
 
 
於是歸吳。周瑜、呂蒙皆共薦達,孫權加異,同於舊臣。寧陳計曰:「今漢祚日微,曹操彌憍,終為篡盜。南荊之地。山陵形便,江川流通,誠是國之西勢也。寧已觀劉表,慮旣不遠,兒子又劣,非能承業傳基者也。至尊當早規之,不可後操圖之。圖之之計,宜先取黃祖。祖今年老,昏耄已甚,財穀並乏,左右欺弄,務於貨利,侵求吏士,吏士心怨,舟船戰具頓廢不脩,怠於耕農,軍無法伍。至尊今往,其破可必。一破祖軍,鼓行而西,西據楚關,大勢彌廣,即可漸規巴蜀。」權深納之。張昭時在坐,難曰:「吳下業業,若軍果行,恐必致亂。」寧謂昭曰:「國家以蕭何之任付君,君居守而憂亂,奚以希慕古人乎?」權舉酒屬寧曰:「興霸,今年行討,如此酒矣,決以付卿。卿但當勉建方略,令必克祖,則卿之功,何嫌張長史之言乎。」權遂西,果禽祖,盡獲其士衆。遂授寧兵,屯當口
 \gezhu{吳書曰:初,權破祖,先作兩函,欲以盛祖及蘇飛首。飛令人告急於寧,寧曰:「飛若不言,吾豈忘之?」權為諸將置酒,寧下席叩頭,血涕交流,為權言:「飛疇昔舊恩,寧不值飛,固已損骸於溝壑,不得致命於麾下。今飛罪當夷戮,特從將軍乞其首領。」權感其言,謂曰:「今為君致之,若走去何?」寧曰:「飛免分裂之禍,受更生之恩,逐之尚必不走,豈當圖亡哉!若爾,寧頭當代入函。」權乃赦之。}
 
 
 
 
 後隨周瑜拒破曹公於烏林。攻曹仁於南郡,未拔,寧建計先徑進取夷陵,往即得其城,因入守之。時手下有數百兵,并所新得,僅滿千人。曹仁乃令五六千人圍寧。寧受攻累日,敵設高樓,雨射城中,士衆皆懼,惟寧談笑自若。遣使報瑜,瑜用呂蒙計,帥諸將解圍。後隨魯肅鎮益陽,拒關羽。羽號有三萬人,自擇選銳士五千人,投縣上流十餘里淺瀨,云欲夜涉渡。肅與諸將議。寧時有三百兵,乃曰:「可復以五百人益吾,吾往對之,保羽聞吾欬唾,不敢涉水,涉水即是吾禽。」肅便選千兵益寧,寧乃夜往。羽聞之,住不渡,而結柴營,今遂名此處為關羽瀨。權嘉寧功,拜西陵太守,領陽新、下雉兩縣。
 
 
 
 
 後從攻皖,為升城督。寧手持練,身緣城,為吏士先,卒破獲朱光。計功,呂蒙為最。寧次之,拜折衝將軍。
 
 
後曹公出濡須,寧為前部督,受勑出斫敵前營。權特賜米酒衆殽,寧乃料賜手下百餘人食。食畢,寧先以銀盌酌酒,自飲兩盌,乃酌與其都督。都督伏,不肯時持。寧引白削置膝上,呵謂之曰:「卿見知於至尊,孰與甘寧?甘寧尚不惜死,卿何以獨惜死乎?」都督見寧色厲,即起拜待酒次,通酌兵各一銀盌。至二更時,銜枚出斫敵。敵驚動,遂退。寧益貴重,增兵二千人。
 \gezhu{江表傳曰:「曹公出濡須,號步騎四十萬,臨江飲馬。權率衆七萬應之,使寧領三千人為前部督。權密勑寧,使夜入魏軍。寧乃選手下健兒百餘人,徑詣曹公營下,使拔鹿角,踰壘入營,斬得數十級。北軍驚駭鼓譟,舉火如星,寧已還入營,作鼓吹,稱萬歲。因夜見權,權喜曰:「足以驚駭老子否?聊以觀卿膽耳。」即賜絹千匹,刀百口。權曰:「孟德有張遼,孤有興霸,足相敵也。」停住月餘,北軍便退。}
 
 
寧雖麤猛好殺,然開爽有計略,輕財敬士,能厚養健兒,健兒亦樂為用命。建安二十年,從攻合肥,會疫疾,軍旅皆已引出,唯車下虎士千餘人,并呂蒙、蔣欽、凌統及寧,從權逍遙津北。張遼覘望知之,即將步騎奄至。寧引弓射敵,與統等死戰。寧厲聲問鼓吹何以不作,壯氣毅然,權尤嘉之。
 \gezhu{吳書曰:凌統怨寧殺其父操,寧常備統,不與相見。權亦命統不得讎之。甞於呂蒙舍會,酒酣,統乃以刀舞。寧起曰:「寧能雙戟舞。」蒙曰:「寧雖能,未若蒙之巧也。」因操刀持楯,以身分之。後權知統意,因令寧將兵,遂徙屯於半州。}
 
 
 
 
 寧廚下兒曾有過,走投呂蒙。蒙恐寧殺之,故不即還。後寧齎禮禮蒙母,臨當與升堂,乃出廚下兒還寧。寧許蒙不殺。斯須還船,縛置桑樹,自挽弓射殺之。畢,勑船人更增舸纜,解衣卧船中。蒙大怒,擊鼓會兵,欲就船攻寧。寧聞之,故卧不起。蒙母徒跣出諫蒙曰:「至尊待汝如骨肉,屬汝以大事,何有以私怒而欲攻殺甘寧?寧死之日,縱至尊不問,汝是為臣下非法。」蒙素至孝,聞母言,即豁然意釋,自至寧舩,笑呼之曰:「興霸,老母待卿食,急上!」寧涕泣歔欷曰:「負卿。」與蒙俱還見母,歡宴竟日。
 
 
 
 
 寧卒,權痛惜之。子瓌,以罪徙會稽,無幾死。
 
 
\end{pinyinscope}