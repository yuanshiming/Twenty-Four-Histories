\article{文帝紀}
\begin{pinyinscope}


文皇帝諱丕,字子桓,武帝太子也。中平四年冬,生于譙。
\gezhu{魏書曰:帝生時,有雲氣青色而圜如車盖當其上,終日,望氣者以為至貴之證,非人臣之氣。年八歲,能屬文。有逸才,遂博貫古今經傳諸子百家之書。善騎射,好擊劔。舉茂才,不行。}
\gezhu{獻帝起居注曰:建安十五年,為司徒趙溫所辟。太祖表「溫辟臣子弟,選舉故不以實」。使侍中守光祿勳郗慮持節奉策免溫官。}
建安十六年,為五官中郎將、副丞相。二十二年,立為魏太子。
\gezhu{魏略曰:太祖不時立太子,太子自疑。是時有高元呂者,善相人,乃呼問之,對曰:「其貴乃不可言。」問:「壽幾何?」元呂曰:「其壽,至四十當有小苦,過是無憂也。」後無幾而立為王太子,至年四十而薨。}
太祖崩,嗣位為丞相、魏王。
\gezhu{袁宏漢紀載漢帝詔曰:「魏太子丕:昔皇天授乃顯考以翼我皇家,遂攘除羣凶,拓定九州,弘功茂績,光于宇宙,朕用垂拱負扆二十有餘載。天不憖遺一老,永保余一人,早世潛神,哀悼傷切。丕奕世宣明,宜秉文武,紹熈前緒。今使使持節御史大夫華歆奉策詔授丕丞相印綬、魏王璽紱,領兾州牧。方今外有遺虜,遐夷未賔,旗鼓猶在邊境,干戈不得韜刃,斯乃播揚洪烈,立功垂名之秋也。豈得脩諒闇之禮,究曾、閔之志哉?其敬服朕命,抑弭憂懷,旁祗厥緒,時亮庶功,以稱朕意。於戲,可不勉與!」}
尊王后曰王太后。改建安二十五年為延康元年。

元年二月
\gezhu{魏書載庚戌令曰:「關津所以通商旅,池苑所以御災荒也。設禁重稅,非所以便民;其除池籞之禁,輕關津之稅,皆復什一。」辛亥,賜諸侯王將相已下大將粟萬斛,帛千匹,金銀各有差等。遣使者循行郡國,有違理掊克暴虐者,舉其罪。}
壬戌,以大中大夫賈詡為太尉,御史大夫華歆為相國,大理王朗為御史大夫。置散騎常侍、侍郎各四人,其宦人為官者不得過諸署令;為金策著令,藏之石室。


初,漢熹平五年,黃龍見譙,光祿大夫橋玄問太史令單颺:「此何祥也?」颺曰:「其國後當有王者興,不及五十年,亦當復見。天事恒象,此其應也。」內黃殷登默而記之。至四十五年,登尚在。三月,黃龍見譙,登聞之曰:「單颺之言,其驗茲乎!」
\gezhu{魏書曰:王召見登,謂之曰:「昔成風聞楚丘之繇而敬事季友,鄧晨信少公之言而自納光武。登以篤老,服膺占術,記識天道,豈有是乎!」賜登穀二百斛,遣歸家。}


已卯,以前將軍夏侯惇為大將軍。濊貃、扶餘單于、焉耆、于闐王皆各遣使奉獻。
\gezhu{魏書曰:丙戌,令史官奏修重、黎、羲、和之職,欽若昊天,歷象日月星辰以奉天時。臣松之案:魏書有是言而不聞其職也。丁亥令曰:「故尚書僕射毛玠、奉常王脩、涼茂、郎中令袁渙、少府謝奐、萬潛、中尉徐奕、國淵等,皆忠直在朝,履蹈仁義,並早即世,而子孫陵遲,惻然愍之,其皆拜子男為郎中。」}

夏四月丁巳,饒安縣言白雉見。
\gezhu{魏書曰:賜饒安田租,勃海郡百戶牛酒,大酺三日;太常以太牢祠宗廟。}
庚午,大將軍夏侯惇薨。
\gezhu{魏書曰:王素服幸鄴東城門發哀。孫盛曰:在禮,天子哭同姓於宗廟門之外。哭於城門,失其所也。}

五月戊寅,天子命王追尊皇祖太尉曰太王,夫人丁氏曰太王后,封王子叡為武德侯。
\gezhu{魏略曰:以侍中鄭稱為武德侯傅,令曰:「龍淵、太阿出昆吾之金,和氏之璧由井里之田;礱之以砥礪,錯之以他山,故能致連城之價,為命世之寶。學亦人之砥礪也。稱篤學大儒,勉以經學輔侯,宜旦夕入侍,曜明其志。」}
是月,馮翊山賊鄭甘、王照率衆降,皆封列侯。
\gezhu{魏書曰:初,鄭甘、王照及盧水胡率其屬來降,王得降書以示朝曰:「前欲有令吾討鮮卑者,吾不從而降;又有欲使吾及今秋討盧水胡者,吾不聽,今又降。昔魏武侯一謀而當,有自得之色,見譏李悝。吾今說此,非自是也,徒以為坐而降之,其功大於動兵革也。」}


酒泉黃華、張掖張進等各執太守以叛。金城太守蘇則討進,斬之。華降。
\gezhu{華後為兖州刺史,見王淩傳。}


六月辛亥,治兵于東郊,
\gezhu{魏書曰:公卿相儀,王御華蓋,視金鼓之節。}
庚午,遂南征。
\gezhu{魏略曰:王將出征,度支中郎將新平霍性上踈諫曰:「臣聞文王與紂之事,是時天下括囊無咎,凡百君子,莫肯用訊。今大王體則乾坤,廣開四聦,使賢愚各建所規。伏惟先王功無與比,而今能言之類,不稱為德。故聖人曰『得百姓之歡心』。兵書曰『戰,危事也』是以六國力戰,彊秦承弊,豳王不爭,周道用興。愚謂大王且當委重本朝而守其雌,抗威虎卧,功業可成。而今刱基,便復起兵,兵者凶器,必有凶擾,擾則思亂,亂出不意。臣謂此危,危於累卵。昔夏啟隱神三年,易有『不遠而復』,論有『不憚改』。誠願大王揆古察今,深謀遠慮,與三事大夫筭其長短。臣沐浴先王之遇,又初改政,復受重任,雖知言觸龍鱗,阿諛近福,竊感所誦,危而不持。」奏通,帝怒,遣刺姦就考,竟殺之。旣而悔之,追原不及。}


秋七月庚辰,令曰:「軒轅有明臺之議,放勛有衢室之問,皆所以廣詢于下也。
\gezhu{管子曰:黃帝立明臺之議者,上觀於兵也;堯有衢室之問者,下聽於民也;舜有告善之旌,而主不蔽也;禹立建鼓於朝,而備訴訟也;湯有總街之廷,以觀民非也;武王有靈臺之囿,而賢者進也:此古聖帝明王所以有而勿失,得而勿忘也。}
百官有司,其務以職盡規諫,將率陳軍法,朝士明制度,牧守申政事,縉紳考六藝,吾將兼覽焉。」


孫權遣使奉獻。蜀將孟達率衆降。武都氐王楊僕率種人內附,居漢陽郡。
\gezhu{魏略載王自手筆令曰:「吾前遣使宣國威靈,而達即來。吾惟春秋襃儀父,即封拜達,使還領新城太守。近復有扶老攜幼首向王化者。吾聞夙沙之民自縛其君以歸神農,豳國之衆襁負其子而入酆、鎬,斯豈驅略迫脅之所致哉?乃風化動其情而仁義感其衷,歡心內發使之然也。以此而推,西南將萬里無外,權、備將與誰守死乎?」}


甲午,軍次于譙,大饗六軍及譙父老百姓于邑東。
\gezhu{魏書曰:設伎樂百戲,令曰:「先王皆樂其所生,禮不忘其本。譙,霸王之邦,真人本出,其復譙租稅二年。」三老吏民上壽,日夕而罷。丙申,親祠譙陵。}
\gezhu{孫盛曰:昔者先王之以孝治天下也,內節天性,外施四海,存盡其敬,亡極其哀,思慕諒闇,寄政冢宰,故曰「三年之喪,自天子達于庶人」。夫然,故在三之義惇,臣子之恩篤,雍熈之化隆,經國之道固,聖人之所以通天地,厚人倫,顯至教,敦風俗,斯萬世不易之典,百王服膺之制也。是故喪禮素冠,鄶人著庶見之譏,宰予降朞,仲尼發不仁之歎,子頹忘戚,君子以為樂禍,魯侯易服,春秋知其不終,豈不以墜至痛之誠心,喪哀樂之大節者哉?故雖三季之末,七雄之弊,猶未有廢縗斬於旬朔之閒,釋麻杖於反哭之日者也。逮于漢文,變易古制,人道之紀,一旦而廢,縗素奪於至尊,四海散其遏密,義感闕於羣后,大化墜於君親;雖心存貶約,慮在經綸,至於樹德垂聲,崇化變俗,固以道薄於當年,風穨於百代矣。且武王載主而牧野不陣,晉襄墨縗而三帥為俘,應務濟功,服其焉害。魏王旣追漢制,替其大禮,處莫重之哀而設饗宴之樂,居貽厥之始而墜王化之基,及至受禪,顯納二女,忘其至恤以誣先聖之典,天心喪矣,將何以終!是以知王齡之不遐,卜世之期促也。}
八月,石邑縣言鳳皇集。


冬十一月癸卯,令曰:「諸將征伐,士卒死亡者或未收斂,吾甚哀之;其告郡國給槥
\gezhu{槥音衞}
櫝殯斂,送致其家,官為設祭。」
\gezhu{漢書高祖八月令曰:「士卒從軍死,為槥。」應劭曰:「槥,小棺也,今謂之櫝。」應璩百一詩曰:「槥車在道路,征夫不得休。」陸機大墓賦曰:「觀細木而悶遲,覩洪櫝而念槥。」}
丙午,行至曲蠡。


漢帝以衆望在魏,乃召羣公卿士,
\gezhu{袁宏漢紀載漢帝詔曰:「朕在位三十有二載,遭天下蕩覆,幸賴祖宗之靈,危而復存。然仰瞻天文,俯察民心,炎精之數旣終,行運在乎曹氏。是以前王旣樹神武之績,今王又光曜明德以應其期,是歷數昭明,信可知矣。夫大道之行,天下為公,選賢與能,故唐堯不私於厥子,而名播於無窮。朕羨而慕焉,今其追踵堯典,禪位于魏王。」}
告祠高廟。使兼御史大夫張音持節奉璽綬禪位,冊曰:「咨爾魏王:昔者帝堯禪位于虞舜,舜亦以命禹,天命不于常,惟歸有德。漢道陵遲,世失其序,降及朕躬,大亂茲昏,羣兇肆逆,宇內顛覆。賴武王神武,拯茲難于四方,惟清區夏,以保綏我宗廟,豈予一人獲乂,俾九服實受其賜。今王欽承前緒,光于乃德,恢文武之大業,昭爾考之弘烈。皇靈降瑞,人神告徵,誕惟亮采,師錫朕命,僉曰爾度克恊于虞舜,用率我唐典,敬遜爾位。於戲!天之歷數在爾躬,允執其中,天祿永終;君其祗順大禮,饗茲萬國,以肅承天命。」
\gezhu{獻帝傳載禪代衆事曰:左中郎將李伏表魏王曰:「昔先王初建魏國,在境外者聞之未審,皆以為拜王。武都李庶、姜合羈旅漢中,謂臣曰:『必為魏公,未便王也。定天下者,魏公子桓,神之所命,當合符讖,以應天人之位。』臣以合辭語鎮南將軍張魯,魯亦問合知書所出?合曰:『孔子玉版也。天子歷數,雖百世可知。』是後月餘,有亡人來,寫得冊文,卒如合辭。合長於內學,關右知名。魯雖有懷國之心,沈溺異道變化,不果寤合之言。後密與臣議策質,國人不協,或欲西通,魯即怒曰:『寧為魏公奴,不為劉備上客也。』言發惻痛,誠有由然。合先迎王師,往歲病亡於鄴。自臣在朝,每為所親宣說此意,時未有宜,弗敢顯言。殿下即位初年,禎祥衆瑞,日月而至,有命自天,昭然著見。然聖德洞達,符表豫明,實乾坤挺慶,萬國作孚。臣每慶賀,欲言合驗;事君盡禮,人以為諂。況臣名行穢賤,入朝日淺,言為罪尤,自抑而已。今洪澤被四表,靈恩格天地,海內翕習,殊方歸服,兆應並集,以揚休命,始終允臧。臣不勝喜舞,謹具表通。」王令曰:「以示外。薄德之人,何能致此,未敢當也;斯誠先王至德通于神明,固非人力也。」}
\gezhu{魏王侍中劉廙、辛毗、劉曄、尚書令桓階、尚書陳矯、陳羣、給事黃門侍郎王毖、董遇等言:「臣伏讀左中郎將李伏上事,考圖緯之言,以效神明之應,稽之古代,未有不然者也。故堯稱歷數在躬,璿璣以明天道;周武未戰而赤烏銜書;漢祖未兆而神母告符;孝宣庂微,字成木葉;光武布衣,名已勒讖。是天之所命以著聖哲,非有言語之聲,芬芳之臭,可得而知也,徒縣象以示人,微物以效意耳。自漢德之衰,漸染數世,桓、靈之末,皇極不建,曁于大亂,二十餘年。天之不泯,誕生明聖,以濟其難,是以符讖先著,以彰至德。殿下踐阼未朞,而靈象變于上,羣瑞應于下,四方不羈之民,歸心向義,唯懼在後,雖典籍所傳,未若今之盛也。臣妾遠近,莫不鳧藻。」王令曰:「犂牛之駮似虎,莠之幼似禾,事有似是而非者,今日是矣。覩斯言事,良重吾不德。」於是尚書僕射宣告官寮,使咸聞知。}
\gezhu{辛亥,太史丞許芝條魏代漢見讖緯於魏王曰:「易傳曰:『聖人受命而王,黃龍以戊己日見。』七月四日戊寅,黃龍見,此帝王受命之符瑞最著明者也。又曰:『初六,履霜,陰始凝也。』又有積蟲大穴天子之宮,厥咎然,今蝗蟲見,應之也。又曰:『聖人以德親比天下,仁恩洽普,厥應麒麟以戊己日至,厥應聖人受命。』又曰:『聖人清淨行中正,賢人福至民從命,厥應麒麟來。』春秋漢含孳曰:『漢以魏,魏以徵。』春秋玉版讖曰:『代赤者魏公子。』春秋佐助期曰:『漢以許昌失天下。』故白馬令李雲上事曰:『許昌氣見於當塗高,當塗高者當昌於許。』當塗高者,魏也;象魏者,兩觀闕是也;當道而高大者魏。魏當代漢。今魏基昌於許,漢徵絕於許,乃今效見,如李雲之言,許昌相應也。佐助期又曰:『漢以蒙孫亡。』說者以蒙孫漢二十四帝,童蒙愚昏,以弱亡。或以雜文為蒙其孫當失天下,以為漢帝非正嗣,少時為董侯,名不正,蒙亂之荒惑,其子孫以弱亡。孝經中黃讖曰:『日載東,絕火光。不橫一,聖聦明。四百之外,易姓而王。天下歸功,致太平,居八甲;共禮樂,正萬民,嘉樂家和雜。』此魏王之姓諱,著見圖讖。易運期讖曰:『言居東,西有午,兩日並光日居下。其為主,反為輔。五八四十,黃氣受,真人出。』言午,許字。兩日,昌字。漢當以許亡,魏當以許昌。今際會之期在許,是其大效也。易運期又曰:『鬼在山,禾女連,王天下。』臣聞帝王者,五行之精;易姓之符,代興之會,以七百二十年為一軌。有德者過之,至於八百,無德者不及,至四百載。是以周家八百六十七年,夏家四百數十年,漢行夏正,迄今四百二十六歲。又高祖受命,數雖起乙未,然其兆徵始於獲麟。獲麟以來七百餘年,天之歷數將以盡終。帝王之興,不常一姓。太微中,黃帝坐常明,而赤帝坐常不見,以為黃家興而赤家衰,凶亡之漸。自是以來四十餘年,又熒惑失色不明十有餘年。建安十年,彗星先除紫微,二十三年,復掃太微。新天子氣見東南以來,二十三年,白虹貫日,月蝕熒惑,比年己亥、壬子、丙午日蝕,皆水滅火之象也。殿下即位,初踐阼,德配天地,行合神明,恩澤盈溢,廣被四表,格于上下。是以黃龍數見,鳳皇仍翔,麒麟皆臻,白虎效仁,前後獻見於郊甸;甘露醴泉,奇獸神物,衆瑞並出。斯皆帝王受命易姓之符也。昔黃帝受命,風后受河圖;舜、禹有天下,鳳皇翔,洛出書;湯之王,白鳥為符;文王為西伯,赤鳥銜丹書;武王伐殷,白魚升舟;高祖始起,白虵為徵。巨跡瑞應,皆為聖人興。觀漢前後之大災,今茲之符瑞,察圖讖之期運,揆河洛之所甄,未若今大魏之最美也。夫得歲星者,道始興。昔武王伐殷,歲在鶉火,有周之分野也。高祖入秦,五星聚東井,有漢之分野也。今茲歲星在大梁,有魏之分野也。而天之瑞應,並集來臻,四方歸附,襁負而至,兆民欣戴,咸樂嘉慶。春秋大傳曰:『周公何以不之魯?蓋以為雖有繼體守文之君,不害聖人受命而王。』周公反政,尸子以為孔子非之,以為周公不聖,不為兆民也。京房作易傳曰:『凡為王者,惡者去之,弱者奪之。易姓改代,天命應常,人謀鬼謀,百姓與能。』伏惟殿下體堯舜之盛明,膺七百之禪代,當湯武之期運,值天命之移授,河洛所表,圖讖所載,怛然明白,天下學士所共見也。臣職在史官,考符察徵,圖讖效見,際會之期,謹以上聞。」王令曰:「昔周文三分天下有其二,以服事殷,仲尼歎其至德;公旦履天子之籍,聽天下之斷,終然復子明辟,書美其人。吾雖德不及二聖,敢忘高山景行之義哉?若夫唐堯、舜、禹之蹟,皆以聖質茂德處之,故能上和靈祇,下寧萬姓,流稱今日。今吾德至薄也,人至鄙也,遭遇際會,幸承先王餘業,恩未被四海,澤未及天下,雖傾倉竭府以振魏國百姓,猶寒者未盡煖,饑者未盡飽。夙夜憂懼,弗敢遑寧,庶欲保全髮齒,長守今日,以沒于地,以全魏國,下見先王,以塞負荷之責。望狹志局,守此而已。雖屢蒙祥瑞,當之戰惶,五色無主。若芝之言,豈所聞乎?心慄手悼,書不成字,辭不宣心。吾間作詩曰:『喪亂悠悠過紀,白骨從橫萬里,哀哀下民靡恃,吾將佐時整理,復子明辟致仕。』庶欲守此辭以自終,卒不虛言也。宜宣示遠近,使昭赤心。」}
\gezhu{於是侍中辛毗、劉曄、散騎常侍傅巽、衞臻、尚書令桓階、尚書陳矯、陳羣、給事中博士騎都尉蘇林、董巴等奏曰:「伏見太史丞許芝上魏國受命之符;令書懇切,允執謙讓,雖舜、禹、湯、文,義無以過。然古先哲王所以受天命而不辭者,誠急遵皇天之意,副兆民之望,弗得已也。且易曰:『觀乎天文以察時變,觀乎人文以化成天下。』又曰:『天垂象,見吉凶,聖人則之;河出圖,洛出書,聖人效之。』以為天文因人而變,至於河洛之書,著于洪範,則殷、周效而用之矣。斯言,誠帝王之明符,天道之大要也。是以由德應錄者代興於前,失道數盡者迭廢於後,傳譏萇弘欲支天之所壞,而說蔡墨『雷乘乾』之說,明神器之存亡,非人力所能逮也。今漢室衰替,帝綱墮墜,天子之詔,歇滅無聞,皇天將捨舊而命新,百姓旣去漢而為魏,昭然著明,是可知也。先王撥亂平世,將建洪基;至於殿下,以至德當歷數之運,即位以來,天應人事,粲然大備,神靈圖籍,兼仍往古,休徵嘉兆,跨越前代;是芝所取中黃、運期姓緯之讖,斯文乃著於前世,與漢並見。由是言之,天命乆矣,非殿下所得而拒之也。神明之意,候望禋享,兆民顒顒,咸注嘉願,惟殿下覽圖籍之明文,急天下之公義,輒宣令外內,布告州郡,使知符命著明,而殿下謙虛之意。」令曰:「下四方以明孤款心,是也。至於覽餘辭,豈余所謂哉?寧所堪哉?諸卿指論,未若吾自料之審也。夫虛談謬稱,鄙薄所弗當也。且聞比來東征,經郡縣,歷屯田,百姓面有饑色,衣或短褐不完,罪皆在孤;是以上慙衆瑞,下愧士民。由斯言之,德尚未堪偏王,何言帝者也!宜止息此議,無重吾不德,使逝之後,不愧後之君子。」}
\gezhu{癸丑,宣告羣寮。督軍御史中丞司馬懿、侍御史鄭渾、羊祕、鮑勛、武周等言:「令如左。伏讀太史丞許芝上符命事,臣等聞有唐世衰,天命在虞,虞氏世衰,天命在夏;然則天地之靈,歷數之運,去就之符,惟德所在。故孔子曰:『鳳鳥不至,河不出圖,吾已矣夫!』今漢室衰,自安、和、沖、質以來,國統屢絕,桓、靈荒淫,祿去公室,此乃天命去就,非一朝一夕,其所由來乆矣。殿下踐阼,至德廣被,格于上下,天人感應,符瑞並臻,考之舊史,未有若今日之盛。夫大人者,先天而天弗違,後天而奉天時,天時已至而猶謙讓者,舜、禹所不為也,故生民蒙救濟之惠,羣類受育長之施。今八方顒顒,大小注望,皇天乃眷,神人同謀,十分而九以委質,義過周文,所謂過恭也。臣妾上下,伏所不安。」令曰:「世之所不足者道義也,所有餘者苟妄也;常人之性,賤所不足,貴所有餘,故曰『不患無位,患所以立』。孤雖寡德,庶自免於常人之貴。夫『石可破而不可奪堅,丹可磨而不可奪赤』。丹石微物,尚保斯質,況吾託士人之末列,曾受教於君子哉?且於陵子仲以仁為富,栢成子高以義為貴,鮑焦感子貢之言,棄其蔬而槁死,薪者譏季札失辭,皆委重而弗視。吾獨何人?昔周武,大聖也,使叔旦盟膠鬲於四內,使公召約微子於共頭,故伯夷、叔齊相與笑之曰:『昔神農氏之有天下,不以人之壞自成,不以人之卑自高。』以為周之伐殷以恭也。吾德非周武而義慙夷、齊,庶欲遠苟妄之失道,立丹石之不奪,邁於陵之所富,蹈栢成之所貴,執鮑焦之貞至,遵薪者之清節。故曰:『三軍可奪帥,匹夫不可奪志。』吾之斯志,豈可奪哉?」}
\gezhu{乙卯,冊詔魏王禪代天下曰:「惟延康元年十月乙卯,皇帝曰:咨爾魏王,夫命運否泰,依德升降,三代卜年,著于春秋,是以天命不于常,帝王不一姓,由來尚矣。漢道陵遲,為日已乆,安、順已降,世失其序,沖、質短祚,三世無嗣,皇綱肇虧,帝典頹沮。曁于朕躬,天降之災,遭無妄厄運之會,值炎精幽昧之期。變興輦轂,禍由閹宦。董卓乘釁,惡甚澆、𤡬,劫遷省御太僕宮廟,遂使九州幅裂,彊敵虎爭,華夏鼎沸,蝮蛇塞路。當斯之時,尺土非復漢有,一夫豈復朕民?幸賴武王德膺符運,奮揚神武,芟夷兇暴,清定區夏,保乂皇家。今王纘承前緒,至德光昭,御衡不迷,布德優遠,聲教被四海,仁風扇鬼區,是以四方效珍,人神響應,天之歷數實在爾躬。昔虞舜有大功二十,而放勳禪以天下;大禹有疏導之績,而重華禪以帝位。漢承堯運,有傳聖之義,加順靈祇,紹天明命,釐降二女,以嬪于魏。使使持節行御史大夫事太常音,奉皇帝璽綬,王其永君萬國,敬御天威,允執其中,天祿永終,敬之哉?」於是尚書令桓階等奏曰:「漢氏以天子位禪之陛下,陛下以聖明之德,歷數之序,承漢之禪,允當天心。夫天命弗可得辭,兆民之望弗可得違,臣請會列侯諸將、羣臣陪隷,發璽書,順天命,具禮儀列奏。」令曰:「當議孤終不當承之意而已。猶獵,還方有令。」}
\gezhu{尚書令等又奏曰:「昔堯、舜禪於文祖,至漢氏,以師征受命,畏天之威,不敢怠遑,便即位行在所之地。今當受禪代之命,宜會百寮羣司,六軍之士,皆在行位,使咸覩天命。營中促狹,可於平敞之處設壇場,奉荅休命。臣輒與侍中常侍會議禮儀,太史官擇吉日訖,復奏。」令曰:「吾殊不敢當之,外亦何豫事也!」侍中劉廙、常侍衞臻等奏議曰:「漢氏遵唐堯公天下之議,陛下以聖德膺歷數之運,天人同忻,靡不得所,宜順靈符,速踐皇阼。問太史丞許芝,今月十七日己未宜成,可受禪命,輒治壇場之處,所當施行別奏。」令曰;「屬出見外,便設壇場,斯何謂乎?今當辭讓不受詔也。但於帳前發璽書,威儀如常,且天寒,罷作壇士使歸。」旣發璽書,王令曰:「當奉還璽綬為讓章。吾豈奉此詔承此貺邪?昔堯讓天下於許由、子州支甫,舜亦讓於善卷、石戶之農、北人無擇,或退而耕潁之陽,或辭以幽憂之疾,或遠入山林,莫知其處,或攜子入海,終身不反,或以為辱,自投深淵;且顏燭懼天撲之不完,守知足之明分,王子搜樂丹穴之潛處,被熏而不出,柳下惠不以三公之貴易其介,曾參不以晉、楚之富易其仁:斯九士者,咸高節而尚義,輕富而賤貴,故書名千載,于今稱焉。求仁得仁,仁豈在遠?孤獨何為不如哉?義有蹈東海而逝,不奉漢朝之詔也。亟為上章還璽綬,宣之天下,使咸聞焉。」己未,宣告羣寮,下魏,又下天下。}
\gezhu{輔國將軍清苑侯劉若等百二十人上書曰:「伏讀令書,深執克讓,聖意懇惻,至誠外昭,臣等有所不安。何者?石戶、北人,匹夫狂狷,行不合義,事不經見者,是以史遷謂之不然,誠非聖明所當希慕。且有虞不逆放勛之禪,夏禹亦無辭位之語,故傳曰:『舜陟帝位,若固有之。』斯誠聖人知天命不可逆,歷數弗可辭也。伏惟陛下應乾符運,至德發聞,升昭于天,是三靈降瑞,人神以和,休徵雜沓,萬國響應,雖欲勿用,將焉避之?而固執謙虛,違天逆衆,慕匹夫之微分,背上聖之所蹈,違經讖之明文,信百氏之穿鑿,非所以奉荅天命,光慰衆望也。臣等昧死以請,輒整頓壇場,至吉日受命,如前奏,分別寫令宣下。」王令曰:「昔柏成子高辭夏禹而匿野,顏闔辭魯幣而遠跡,夫以王者之重,諸侯之貴,而二子忽之,何則?其節高也。故烈士徇榮名,義夫高貞介,雖蔬食瓢飲,樂在其中。是以仲尼師王駘,而子產嘉申徒。今諸卿皆孤股肱腹心,足以明孤,而今咸若斯,則諸卿遊於形骸之內,而孤求為形骸之外,其不相知,未足多怪。亟為上章還璽綬,勿復紛紛也。」}
\gezhu{輔國將軍等一百二十人又奏曰:「臣聞符命不虛見,衆心弗可違,故孔子曰:『周公其為不聖乎?以天下讓。是天地日月輕去萬物也。』是以舜嚮天下,不拜而受命。今火德氣盡,炎上數終,帝遷明德,祚隆大魏。符瑞昭晢,受命旣固,光天之下,神人同應,雖有虞儀鳳,成周躍魚,方今之事,未足以喻。而陛下違天命以飾小行,逆人心以守私志,上忤皇穹眷命之旨,中忘聖人達節之數,下孤人臣翹首之望,非所以揚聖道之高衢,乘無窮之懿勳也。臣等聞事君有獻可替否之道,奉上有逆鱗固爭之義,臣等敢以死請。」令曰:「太古聖王之治也,至德合乾坤,惠澤均造化,禮教優乎昆蟲,仁恩洽乎草木,日月所照,戴天履地含氣有生之類,靡不被服清風,沐浴玄德;是以金革不起,苛慝不作,風雨應節,禎祥觸類而見。今百姓寒者未暖,饑者未飽,鰥者未室,寡者未嫁;權、備尚存,未可舞以干戚,方將整以齊斧;戎役未息於外,士民未安於內,耳未聞康哉之歌,目未覩擊壤之戲,嬰兒未可託於高巢,餘糧未可以宿於田畒:人事未備,至如此也。夜未曜景星,治未通真人,河未出龍馬,山未出象車,蓂莢未植階塗,萐莆未生庖廚,王母未獻白環,渠搜未見珍裘:靈瑞未效,又如彼也。昔東戶季子、容成、大庭、軒轅、赫胥之君,咸得以此就功勒名。今諸卿獨不可少假孤精心竭慮,以和天人,以格至理,使彼衆事備,羣瑞效,然後安乃議此乎,何遽相愧相迫之如是也?速為讓章,上還璽綬,無重吾不德也。」}
\gezhu{侍中劉廙等奏曰:「伏惟陛下以大聖之純懿,當天命之歷數,觀天象則符瑞著明,考圖緯則文義煥炳,察人事則四海齊心,稽前代則異世同歸;而固拒禪命,未踐尊位,聖意懇惻,臣等敢不奉詔?輒具章遣使者。」奉令曰:「泰伯三以天下讓,人無得而稱焉,仲尼歎其至德,孤獨何人?」}
\gezhu{庚申,魏王上書曰:「皇帝陛下:奉被今月乙卯璽書,伏聽冊命,五內驚震,精爽散越,不知所處。臣前上還相位,退守藩國,聖恩聽許。臣雖無古人量德度身自定之志,保己存性,實其私願。不寤陛下猥損過謬之命,發不世之詔,以加無德之臣。且聞堯禪重華,舉其克諧之德,舜授文命,采其齊聖之美,猶下咨四嶽,上觀璿璣。今臣德非虞、夏,行非二君,而承歷數之諮,應選授之命,內自揆撫,無德以稱。且許由匹夫,猶拒帝位,善卷布衣,而逆虞詔。臣雖鄙蔽,敢忘守節以當大命,不勝至願。謹拜章陳情,使行相國永壽少府糞土臣毛宗奏,并上璽綬。」}
\gezhu{辛酉,給事中博士蘇林、董巴上表曰:「天有十二次以為分野,王公之國,各有所屬,周在鶉火,魏在大梁。歲星行歷十二次國,天子受命,諸侯以封。周文王始受命,歲在鶉火,至武王伐紂十三年,歲星復在鶉火,故春秋傳曰:『武王伐紂,歲在鶉火;歲之所在,即我有周之分野也。』昔光和七年,歲在大梁,武王始受命,為時將討黃巾。是歲改年為中平元年。建安元年,歲復在大梁,始拜大將軍。十三年復在大梁,始拜丞相。今二十五年,歲復在大梁,陛下受命。此魏得歲與周文王受命相應。今年青龍在庚子,詩推度災曰:『庚者更也,子者滋也,聖命天下治。』又曰:『王者布德於子,治成於丑。』此言今年天更命聖人制治天下,布德於民也。魏以改制天下,與時協矣。顓頊受命,歲在豕韋,衞居其地,亦在豕韋,故春秋傳曰:『衞,顓頊之墟也。』今十月斗之建,則顓頊受命之分也,始魏以十月受禪,此同符始祖受命之驗也。魏之氏族,出自顓頊,與舜同祖,見于春秋世家。舜以土德承堯之火,今魏亦以土德承漢之火,於行運,會於堯舜授受之次。臣聞天之去就,固有常分,聖人當之,昭然不疑,故堯捐骨肉而禪有虞,終無恡色,舜發壠畒而君天下,若固有之,其相受授,間不替漏;天下已傳矣,所以急天命,天下不可一日無君也。今漢期運已終,妖異絕之已審,陛下受天之命,符瑞告徵,丁寧詳悉,反覆備至,雖言語相喻,無以代此。今旣發詔書,璽綬未御,固執謙讓,上逆天命,下違民望。臣謹按古之典籍,參以圖緯,魏之行運及天道所在,即尊之驗,在於今年此月,昭晰分明。唯陛下遷思易慮,以時即位,顯告天帝而告天下,然後改正朔,易服色,正大號,天下幸甚。」令曰:「凡斯皆宜聖德,故曰:『苟非其人,道不虛行。』天瑞雖彰,須德而光;吾德薄之人,胡足以當之?今讓,兾見聽許,外內咸使聞知。」}
\gezhu{壬戌,冊詔曰:「皇帝問魏王言:遣宗奉庚申書到,所稱引,聞之。朕惟漢家世踰二十,年過四百,運周數終,行祚已訖,天心已移,兆民望絕,天之所廢,有自來矣。今大命有所底止,神器當歸聖德,違天不順,逆衆不祥。王其體有虞之盛德,應歷數之嘉會,是以禎祥吉符,圖讖表錄,神人同應,受命咸宜。朕畏上帝,致位于王;天不可違,衆不可拒。且重華不逆堯命,大禹不辭舜位,若夫由、卷匹夫,不載聖籍,固非皇材帝器所當稱慕。今使音奉皇帝璽綬,王其陟帝位,無逆朕命,以祗奉天心焉。」}
\gezhu{於是尚書令桓階等奏曰:「今漢使音奉璽書到,臣等以為天命不可稽,神器不可瀆。周武中流有白魚之應,不待師期而大號已建,舜受大麓,桑蔭未移而已陟帝位,皆所以祗承天命,若此之速也。故無固讓之義,不以守節為貴,必道信於神靈,符合於天地而已。易曰:『其受命如響,無有遠近幽深,遂知來物,非天下之至賾,其孰能與於此?』今陛下應期運之數,為皇天所子,而復稽滯於辭讓,低回於大號,非所以則天地之道,副萬國之望。臣等敢以死請,輒勑有司脩治壇場,擇吉日,受禪命,發璽綬。」令曰:「兾三讓而不見聽,何汲汲於斯乎?」}
\gezhu{甲子,魏王上書曰:「奉今月壬戌璽書,重被聖命,伏聽冊告,肝膽戰悸,不知所措。天下神器,禪代重事,故堯將禪舜,納于大麓,舜之命禹,玄圭告功;烈風不迷,九州攸平,詢事考言,然後乃命,而猶執謙讓于德不嗣。況臣頑固,質非二聖,乃應天統,受終明詔;敢守微節,歸志箕山,不勝大願。謹拜表陳情,使并奉上璽綬。」侍中劉廙等奏曰:「臣等聞聖帝不違時,明主不逆人,故易稱通天下之志,斷天下之疑。伏惟陛下體有虞之上聖,承土德之行運,當亢陽明夷之會,應漢氏祚終之數,合契皇極,同符兩儀。是以聖瑞表徵,天下同應,歷運去就,深切著明;論之天命,無所與議,比之時宜,無所與爭。故受命之期,時清日晏,曜靈施光,休氣雲蒸。是乃天道恱懌,民心欣戴,而仍見閉拒,於禮何居?且羣生不可以一日無主,神器不可以斯須無統,故臣有違君以成業,下有矯上以立事,臣等敢不重以死請。」王令曰:「天下重器,王者正統,以聖德當之,猶有懼心,吾何人哉?且公卿未至乏主,斯豈小事,且宜以待固讓之後,乃當更議其可耳。」}
\gezhu{丁卯,冊詔魏王曰:「天訖漢祚,辰象著明,朕祗天命,致位于王,仍陳歷數於詔冊,喻符運於翰墨;神器不可以辭拒,皇位不可以謙讓,稽於天命,至于再三。且四海不可一日曠主,萬機不可以斯須乏統,故建大業者不拘小節,知天命者不繫細物,是以舜受大業之命而無遜讓之辭,聖人達節,不亦遠乎!今使音奉皇帝璽綬,王其欽承,以荅天下嚮應之望焉。」}
\gezhu{相國華歆、太尉賈詡、御史大夫王朗及九卿上言曰:「臣等被召到,伏見太史丞許芝、左中郎將李伏所上圖讖、符命,侍中劉廙等宣叙衆心,人靈同謀。又漢朝知陛下聖化通于神明,聖德參于虞、夏,因瑞應之備至,聽歷數之所在,遂獻璽綬,固讓尊號。能言之倫,莫不抃舞,河圖、洛書,天命瑞應,人事恊于天時,民言恊于天序。而陛下性秉勞謙,體尚克讓,明詔懇切,未肯聽許,臣妾小人,莫不伊邑。臣等聞自古及今,有天下者不常在乎一姓;考以德勢,則盛衰在乎彊弱,論以終始,則廢興在乎期運。唐、虞歷數,不在厥子而在舜、禹。舜、禹雖懷克讓之意迫,羣后執玉帛而朝之,兆民懷欣戴而歸之,率土揚謌謠而詠之,故其守節之拘,不可得而常處,達節之權,不可得而乆避;是以或遜位而不𠫤,或受禪而不辭,不𠫤者未必厭皇寵,不辭者未必渴帝祚,各迫天命而不得以已。旣禪之後,則唐氏之子為賔于有虞,虞氏之冑為客于夏代,然則禪代之義,非獨受之者實應天福,授之者亦與有餘慶焉。漢自章、和之後,世多變故,稍以陵遲,洎乎孝靈,不恒其心,虐賢害仁,聚斂無度,政在嬖豎,視民如讎,遂令上天震怒,百姓從風如歸;當時則四海鼎沸,旣沒則禍發宮庭,寵勢並竭,帝室遂卑,若在帝舜之末節,猶擇聖代而授之,荊人抱玉璞,猶思良工而刊之,況漢國旣往,莫之能匡,推器移君,委之聖哲,固其宜也。漢朝委質,旣願禪禮之速定也,天祚率土,必將有主;主率土者,非陛下其孰能任之?所謂論德無與為比,考功無推讓矣。天命不可久稽,民望不不可久違,臣等慺慺,不勝大願。伏請陛下割撝謙之志,脩受禪之禮,副人神之意,慰外內之願。」令曰:「以德則孤不足,以時則戎虜未滅。若以羣賢之靈,得保首領,終君魏國,於孤足矣。若孤者,胡足以辱四海?至乎天瑞人事,皆先王聖德遺慶,孤何有焉?是以未敢聞命。」}
\gezhu{己巳,魏王上書曰:「臣聞舜有賔于四門之勳,乃受禪于陶唐,禹有存國七百之功,乃承祿於有虞。臣以蒙蔽,德非二聖,猥當天統,不敢聞命。敢屢抗疏,略陳私願,庶章通紫庭,得全微節,情達宸極,永守本志。而音重復銜命,申制詔臣,臣實戰惕,不發璽書,而音迫於嚴詔,不敢復命。願陛下馳傳騁馹,召音還臺。不勝至誠,謹使宗奉書。」}
\gezhu{相國歆、太尉詡、御史大夫朗及九卿奏曰:「臣等伏讀詔書,於悒益甚。臣等聞易稱聖人奉天時,論語云君子畏天命,天命有去就,然後帝者有禪代。是以唐之禪虞,命在爾躬,虞之順唐,謂之受終;堯知天命去己,故不得不禪舜,舜知歷數在躬,故不敢不受;不得不禪,奉天時也,不敢不受,畏天命也。漢朝雖承季末陵遲之餘,猶務奉天命以則堯之道,是以願禪帝位而歸二女。而陛下正於大魏受命之初,抑虞、夏之達節,尚延陵之讓退,而所枉者大,所直者小,所詳者輕,所略者重,中人凡士猶為陛下陋之。沒者有靈,則重華必忿憤於蒼梧之神墓,大禹必鬱悒於會稽之山陰,武王必不恱於商陵之玄宮矣。是以臣等敢以死請。且漢政在閹宦,祿去帝室七世矣,遂集矢石于其宮殿,而二京為之丘墟。當是之時,四海蕩覆,天下分崩,武王親衣甲而冠冑,沐雨而櫛風,為民請命,則活萬國,為世撥亂,則致升平,鳩民而立長,築宮而置吏,元元無過,罔於前業,而始有造於華夏。陛下即位,光昭文德,以翊武功,勤恤民隱,視之如傷,懼者寧之,勞者息之,寒者以暖,饑者以充,遠人以恩復,寇敵以恩降,邁恩種德,光被四表;稽古篤睦,茂于放勛,網漏吞舟,弘乎周文。是以布政未朞,人神並和,皇天則降甘露而臻四靈,后土則挺芝草而吐醴泉,虎豹鹿兔,皆素其色,雉鳩燕雀,亦白其羽,連理之木,同心之瓜,五采之魚,珍祥瑞物,雜沓於其間者,無不畢備。古人有言:『微禹,吾其魚乎!』微大魏,則臣等之白骨交橫于曠野矣。伏省羣臣外內前後章奏,所以陳叙陛下之符命者,莫不條河洛之圖書,據天地之瑞應,因漢朝之款誠,宣萬方之景附,可謂信矣省矣;三王無以及,五帝無以加。民命之懸於魏邦,民心之繫於魏政,三十有餘年矣,此乃千世時至之會,萬載一遇之秋;達節廣度,宜昭於斯際,拘牽小節,不施於此時。久稽天命,罪在臣等。輒營壇場,具禮儀,擇吉日,昭告昊天上帝,秩羣神之禮,須禋祭畢,會羣寮於朝堂,議年號、正朔、服色當施行。」上復令曰:「昔者大舜飯糗茹草,將終身焉,斯則孤之前志也。及至承堯禪,被珍裘,妻二女,若固有之,斯則順天命也。羣公卿士誠以天命不可拒,民望不可違,孤亦曷以辭焉?」}
\gezhu{庚午,冊詔魏王曰:「昔堯以配天之德,秉六合之重,猶覩歷運之數,移於有虞,委讓帝位,忽如遺跡。今天旣訖我漢命,乃眷北顧,帝皇之業,實有大魏。朕守空名以竊古義,顧視前事,猶有慙德,而王遜讓至于三四,朕用懼焉。夫不辭萬乘之位者,知命達節之數也,虞、夏之君,處之不疑,故勳烈垂于萬載,美名傳於無窮。今遣守尚書令侍中覬喻,王其速陟帝位,以順天人之心,副朕之大願。」}
\gezhu{於是尚書令桓階等奏曰:「今漢氏之命已四至,而陛下前後固辭,臣等伏以為上帝之臨聖德,期運之隆大魏,斯豈數載?傳稱周之有天下,非甲子之朝,殷之去帝位,非牧野之日也,故詩序商湯,追本玄王之至,述姬周,上錄后稷之生,是以受命旣固,厥德不回。漢氏衰廢,行次已絕,三辰垂其徵,史官著其驗,耆老記先古之占,百姓協謌謠之聲。陛下應天受禪,當速即壇場,柴燎上帝,誠不宜久停神器,拒億兆之願。臣輒下太史令擇元辰,今月二十九日,可登壇受命,請詔三公羣卿,具條禮儀別奏。」令曰:「可。」}
乃為壇於繁陽。庚午,王升壇即阼,百官陪位。事訖,降壇,視燎成禮而反。改延康為黃初,大赦。
\gezhu{獻帝傳曰:辛未,魏王登壇受禪,公卿、列侯、諸將、匈奴單于、四夷朝者數萬人陪位,燎祭天地、五嶽、四瀆,曰:「皇帝臣丕敢用玄牡昭告于皇皇后帝:漢歷世二十有四,踐年四百二十有六,四海困窮,王綱不立,五緯錯行,靈祥並見,推術數者,慮之古道,咸以為天之歷數,運終茲世,凡諸嘉祥民神之意,比昭有漢數終之極,魏家受命之符。漢主以神器宜授於臣,憲章有虞,致位于丕。丕震畏天命,雖休勿休。羣公庶尹六事之人,外及將士,洎于蠻夷君長,僉曰:『天命不可以辭拒,神器不可以久曠,羣臣不可以無主,萬機不可以無統。』丕祇承皇象,敢不欽承。卜之守龜,兆有大橫,筮之三易,兆有革兆,謹擇元日,與羣寮登壇受帝璽綬,告類于爾大神;唯爾有禪,尚饗永吉,兆民之望,祚于有魏世享。」遂制詔三公:「上古之始有君也,必崇恩化以美風俗,然百姓順教而刑辟厝焉。今朕承帝王之緒,其以延康元年為黃初元年,議改正朔,易服色,殊徽號,同律度量,承土行,大赦天下;自殊死以下,諸不當得赦,皆赦除之。」魏氏春秋曰:帝升壇禮畢,顧謂羣臣曰:「舜、禹之事,吾知之矣。」干竇搜神記曰:宋大夫邢史子臣明於天道,周敬王之三十七年,景公問曰:「天道其何祥?」對曰:「後五年五月丁亥,臣將死;死後五年五月丁卯,吳將亡;亡後五年,君將終;終後四百年,邾王天下。」俄而皆如其言。所云邾王天下者,謂魏之興也。邾,曹姓,魏亦曹姓,皆邾之後。其年數則錯,未知邢史失其數邪,將年代久遠,注記者傳而有謬也?}


黃初元年十一月癸酉,以河內之山陽邑萬戶奉漢帝為山陽公,行漢正朔,以天子之禮郊祭,上書不稱臣,京都有事于太廟,致胙;封公之四子為列侯。追尊皇祖太王曰太皇帝,考武王曰武皇帝,尊王太后曰皇太后。賜男子爵人一級,為父後及孝悌力田人二級。以漢諸侯王為崇德侯,列侯為關中侯。以潁陰之繁陽亭為繁昌縣。封爵增位各有差。改相國為司徒,御史大夫為司空,奉常為太常,郎中令為光祿勳,大理為廷尉,大農為大司農。郡國縣邑,多所改易。更授匈奴南單于呼廚泉魏璽綬,賜青蓋車、乘輿、寶劔、玉玦。十二月,初營洛陽宮,戊午幸洛陽。
\gezhu{臣松之案:諸書記是時帝居北宮,以建始殿朝羣臣,門曰承明,陳思王植詩曰「謁帝承明廬」是也。至明帝時,始於漢南宮崇德殿處起太極、昭陽諸殿。魏書曰:以夏數為得天,故即用夏正,而服色尚黃。魏略曰:詔以漢火行也,火忌水,故「洛」去「水」而加「隹」。魏於行次為土,土,水之牡也,水得土而乃流,土得水而柔,故除「隹」加「水」,變「雒」為「洛」。}




是歲,長水校尉戴陵諫不宜數行弋獵,帝大怒;陵減死罪一等。


二年春正月,郊祀天地、明堂。甲戌,校獵至原陵,遣使者以太牢祠漢世祖。乙亥,朝日于東郊。
\gezhu{臣松之以為禮天子以春分朝日,秋分夕月;尋此年正月郊祀,有月無日,乙亥朝日,則有日無月,蓋文之脫也。案明帝朝日夕月,皆如禮文,故知此紀為誤者也。}
初令郡國口滿十萬者,歲察孝廉一人;其有秀異,無拘戶口。辛巳,分三公戶邑,封子弟各一人為列侯。壬午,復潁川郡一年田租。
\gezhu{魏書載詔曰:「潁川,先帝所由起兵征伐也。官渡之役,四方瓦解,遠近顧望,而此郡守義,丁壯荷戈,老弱負糧。昔漢祖以秦中為國本,光武恃河內為王基,今朕復於此登壇受禪,天以此郡翼成大魏。」}
改許縣為許昌縣。以魏郡東部為陽平郡,西部為廣平郡。
\gezhu{魏略曰:改長安、譙、許昌、鄴、洛陽為五都;立石表,西界宜陽,北循太行,東北界陽平,南循魯陽,東界郯,為中都之地。令天下聽內徙,復五年,後又增其復。}




詔曰:「昔仲尼資大聖之才,懷帝王之器,當衰周之末,無受命之運,在魯、衞之朝,教化乎洙、泗之上,悽悽焉,遑遑焉,欲屈己以存道,貶身以救世。于時王公終莫能用之,乃退考五代之禮,脩素王之事,因魯史而制春秋,就太師而正雅頌,俾千載之後,莫不宗其文以述作,仰其聖以成謀,咨!可謂命世之大聖,億載之師表者也。遭天下大亂,百祀墮壞,舊居之廟,毀而不脩,襃成之後,絕而莫繼,闕里不聞講頌之聲,四時不覩蒸甞之位,斯豈所謂崇禮報功,盛德百世必祀者哉!其以議郎孔羨為宗聖侯,邑百戶,奉孔子祀。」令魯郡脩起舊廟,置百戶吏卒以守衞之,又於其外廣為室屋以居學者。


三月,加遼東太守公孫恭為車騎將軍。初復五銖錢。夏四月,以車騎將軍曹仁為大將軍。五月,鄭甘復叛,遣曹仁討斬之。六月庚子,初祀五嶽四瀆,咸秩羣祀。
\gezhu{魏書:甲辰,以京師宗廟未成,帝親祠武皇帝于建始殿,躬執饋奠,如家人之禮。}
丁卯,夫人甄氏卒。戊辰晦,日有食之,有司奏免太尉,詔曰:「災異之作,以譴元首,而歸過股肱,豈禹、湯罪己之義乎?其令百官各虔厥職,後有天地之眚,勿復劾三公。」


秋八月,孫權遣使奉章,并遣于禁等還。丁巳,使太常邢貞持節拜權為大將軍,封吳王,加九錫。冬十月,授楊彪光祿大夫。
\gezhu{魏書曰:己亥,公卿朝朔旦,并引故漢太尉楊彪,待以客禮,詔曰:「夫先王制几杖之賜,所以賔禮黃耇襃崇元老也。昔孔光、卓茂皆以淑德高年,受茲嘉錫。公故漢宰臣,乃祖已來,世著名節,年過七十,行不踰矩,可謂老成人矣,所宜寵異以章舊德。其賜公延年杖及馮几;謁請之日,便使杖入,又可使著鹿皮冠。」彪辭讓不聽,竟著布單衣、皮弁以見。續漢書曰:彪見漢祚將終,自以累世為三公,恥為魏臣,遂稱足攣,不復行。積十餘年,帝即王位,欲以為太尉,令近臣宣旨。彪辭曰:「甞以漢朝為三公,值世衰亂,不能立尺寸之益,若復為魏臣,於國之選,亦不為榮也。」帝不奪其意。黃初四年,詔拜光祿大夫,秩中二千石,朝見位次三公,如孔光故事。彪上章固讓,帝不聽,又為門施行馬,致吏卒,以優崇之。年八十四,以六年薨。子脩,事見陳思王傳。}
以穀貴,罷五銖錢。
\gezhu{魏書曰:十一月辛未,鎮西將軍曹真命衆將及州郡兵討破叛胡治元多、蘆水、封賞等,斬首五萬餘級,獲生口十萬,羊一百一十一萬口,牛八萬,河西遂平。帝初聞胡決水灌顯美,謂左右諸將曰:「昔隗嚻灌略陽,而光武因其疲弊,進兵滅之。今胡決水灌顯美,其事正相似,破胡事今至不久。」旬日,破胡告檄到,上大笑曰:「吾策之於帷幕之內,諸將奮擊於萬里之外,其相應若合符契。前後戰克獲虜,未有如此也。」}
己卯,以大將軍曹仁為大司馬。十二月,行東巡。是歲築陵雲臺。


三年春正月丙寅朔,日有蝕之。庚午,行幸許昌宮。詔曰:「今之計考,古之貢士也;十室之邑,必有忠信,若限年然後取士,是呂尚、周晉不顯於前世也。其令郡國所選,勿拘老幼;儒通經術,吏達文法,到皆試用。有司糾故不以實者。」
\gezhu{魏書曰:癸亥,孫權上書,說:「劉備支黨四萬人,馬二三千匹,出秭歸,請往埽撲,以克捷為效。」帝報曰:「昔隗嚻之弊,禍發栒邑,子陽之禽,變起扞關,將軍其亢厲威武,勉蹈奇功,以稱吾意。」}


二月,鄯善、龜茲、于闐王各遣使奉獻,詔曰:「西戎即叙,氐、羌來王,詩、書美之。頃者西域外夷並款塞內附,
\gezhu{應劭漢書注曰:款,叩也;皆叩塞門來服從。}
其遣使者撫勞之。」是後西域遂通,置戊己校尉。




三月乙丑,立齊公叡為平原王,帝弟鄢陵公彰等十一人皆為王。初制封王之庶子為鄉公,嗣王之庶子為亭侯,公之庶子為亭伯。甲戌,立皇子霖為河東王。甲午,行幸襄邑。夏四月戊申,立鄄城侯植為鄄城王。癸亥,行還許昌宮。五月,以荊、揚、江表八郡為荊州,孫權領牧故也;荊州江北諸郡為郢州。




閏月,孫權破劉備於夷陵。初,帝聞備兵東下,與權交戰,樹柵連營七百餘里,謂羣臣曰:「備不曉兵,豈有七百里營可以拒敵者乎!『苞原隰險阻而為軍者,為敵所禽』,此兵忌也。孫權上事今至矣。」後七日,破備書到。


秋七月,兾州大蝗,民饑,使尚書杜畿持節開倉廩以振之。八月,蜀大將黃權率衆降。
\gezhu{魏書曰:權及領南郡太守史郃等三百一十八人,詣荊州刺史奉上所假印綬、棨戟、幢麾、牙門、鼓車。權等詣行在所,帝置酒設樂,引見於承光殿。權、郃等人人前自陳,帝為論說軍旅成敗去就之分,諸將無不喜恱。賜權金帛、車馬、衣裘、帷帳、妻妾,下及偏裨皆有差。拜權為侍中鎮南將軍,封列侯,即日召使驂乘;及封史郃等四十二人皆為列侯,為將軍郎將百餘人。}


九月甲午,詔曰:「夫婦人與政,亂之本也。自今以後,羣臣不得奏事太后,后族之家不得當輔政之任,又不得橫受茅土之爵;以此詔傳後世,若有背違,天下共誅之。」
\gezhu{孫盛曰:夫經國營治,必憑俊喆之輔,賢達令德,必居參亂之任,故雖周室之盛,有婦人與焉。然則坤道承天,南面罔二,三從之禮,謂之至順,至於號令自天子出,奏事專行,非古義也。昔在申、呂,實匡有周。苟以天下為心,惟德是杖,則親疏之授,至公一也,何至后族而必斥遠之哉?二漢之季世,王道陵遲,故令外戚憑寵,職為亂階。於此,自時昏道喪,運祚將移,縱無王、呂之難,豈乏田、趙之禍乎?而後世觀其若此,深懷酸毒之戒也。至於魏文,遂發一概之詔,可謂有識之爽言,非帝者之宏議。}
庚子,立皇后郭氏。賜天下男子爵人二級;鰥寡篤癃及貧不能自存者賜穀。


冬十月甲子,表首陽山東為壽陵,作終制曰:「禮,國君即位為椑,
\gezhu{椑音扶歷反。}
存不忘亡也。
\gezhu{臣松之按:禮,天子諸侯之棺,各有重數;棺之親身者曰椑。}
昔堯葬穀林,通樹之,禹葬會稽,農不易畝,
\gezhu{呂氏春秋:堯葬於穀林,通樹之;舜葬於紀,市廛不變其肆;禹葬會稽,不變人徒。}
故葬於山林,則合乎山林。封樹之制,非上古也,吾無取焉。壽陵因山為體,無為封樹,無立寑殿,造園邑,通神道。夫葬也者,藏也,欲人之不得見也。骨無痛痒之知,冢非棲神之宅,禮不墓祭,欲存亡之不黷也,為棺槨足以朽骨,衣衾足以朽肉而已。故吾營此丘墟不食之地,欲使易代之後不知其處。無施葦炭,無藏金銀銅鐵,一以瓦器,合古塗車、芻靈之義。棺但漆際會三過,飯含無以珠玉,無施珠襦玉匣,諸愚俗所為也。季孫以璵璠斂,孔子歷級而救之,譬之暴骸中原。宋公厚葬,君子謂華元、樂莒不臣,以為棄君於惡。漢文帝之不發,霸陵無求也;光武之掘,原陵封樹也。霸陵之完,功在釋之;原陵之掘,罪在明帝。是釋之忠以利君,明帝愛以害親也。忠臣孝子,宜思仲尼、丘明、釋之之言,鑒華元、樂莒、明帝之戒,存於所以安君定親,使魂靈萬載無危,斯則賢聖之忠孝矣。自古及今,未有不亡之國,亦無不掘之墓也。喪亂以來,漢氏諸陵無不發掘,至乃燒取玉匣金縷,骸骨并盡,是焚如之刑也,豈不重痛哉!禍由乎厚葬封樹。『桑、霍為我戒』,不亦明乎?其皇后及貴人以下,不隨王之國者,有終沒皆葬澗西,前又以表其處矣。蓋舜葬蒼梧,二妃不從,延陵葬子,遠在嬴、博,魂而有靈,無不之也,一澗之閒,不足為遠。若違今詔,妄有所變改造施,吾為戮尸地下,戮而重戮,死而重死。臣子為蔑死君父,不忠不孝,使死者有知,將不福汝。其以此詔藏之宗廟,副在尚書、祕書、三府。」




是月,孫權復叛。復郢州為荊州。帝自許昌南征,諸軍兵並進,權臨江拒守。十一月辛丑,行幸宛。庚申晦,日有食之。是歲,穿靈芝池。


四年春正月,詔曰:「喪亂以來,兵革未戢,天下之人,互相殘殺。今海內初定,敢有私復讎者皆族之。」築南巡臺于宛。三月丙申,行自宛還洛陽宮。癸卯,月犯心中央大星。
\gezhu{魏書載丙午詔曰:「孫權殘害民物,朕以寇不可長,故分命猛將三道並征。今征東諸軍與權黨呂範等水戰,則斬首四萬,獲船萬艘。大司馬據守濡須,其所禽獲亦以萬數。中軍、征南攻圍江陵,左將軍張郃等舳艫直渡,擊其南渚,賊赴水溺死者數千人,又為地道攻城,城中外雀鼠不得出入,此几上肉耳!而賊中癘氣疾病,夾江塗地,恐相染污。昔周武伐殷,旋師孟津,漢祖征隗嚻,還軍高平,皆知天時而度賊情也。且成湯解三面之網,天下歸仁。今開江陵之圍,以緩成死之禽。且休力役,罷省繇戍,畜養士民,咸使安息。」}
丁未,大司馬曹仁薨。是月大疫。


夏五月,有鵜鶘鳥集靈芝池,詔曰:「此詩人所謂污澤也。曹詩『刺恭公遠君子而近小人』,今豈有賢智之士處于下位乎?否則斯鳥何為而至?其博舉天下儁德茂才、獨行君子,以荅曹人之刺。」
\gezhu{魏書曰:辛酉,有司奏造二廟,立太皇帝廟,大長秋特進侯與高祖合祭,親盡以次毀;特立武皇帝廟,四時享祀,為魏太祖,萬載不毀也。}


六月甲戌,任城王彰薨于京都。甲申,太尉賈詡薨。太白晝見。是月大雨,伊、洛溢流,殺人民,壞廬宅。
\gezhu{魏書曰:七月乙未,大軍當出,使太常以特牛一告祠于郊。臣松之按:魏郊祀奏中,尚書盧毓議祀厲殃事云:「具犧牲祭器,如前後師出告郊之禮。」如此,則魏氏出師,皆告郊也。}
秋八月丁卯,以廷尉鍾繇為太尉。
\gezhu{魏書曰:有司奏改漢氏宗廟安世樂曰正世樂,嘉至樂曰迎靈樂,武德樂曰武頌樂,昭容樂曰昭業樂,雲翹舞曰鳳翔舞,育命舞曰靈應舞,武德舞曰武頌舞,文昭舞曰大昭舞,五行舞曰大武舞。}
辛未,校獵于熒陽,遂東巡。論征孫權功,諸將已下進爵增戶各有差。九月甲辰,行幸許昌宮。
\gezhu{魏書曰:十二月丙寅,賜山陽公夫人湯沐邑,公女曼為長樂郡公主,食邑各五百戶。是冬,甘露降芳林園。臣松之按:芳林園即今華林園,齊王芳即位,改為華林。}


五年春正月,初令謀反大逆乃得相告,其餘皆勿聽治;敢妄相告,以其罪罪之。三月,行自許昌還洛陽宮。夏四月,立太學,制五經課試之法,置春秋穀梁博士。五月,有司以公卿朝朔望日,因奏疑事,聽斷大政,論辨得失。秋七月,行東巡,幸許昌宮。八月,為水軍,親御龍舟,循蔡、頴,浮淮,幸壽春。揚州界將吏士民,犯五歲刑已下,皆原除之。九月,遂至廣陵,赦青、徐二州,改易諸將守。冬十月乙卯,太白晝見。行還許昌宮。
\gezhu{魏書載癸酉詔曰:「近之不綏,何遠之懷?今事多而民少,上下相弊以文法,百姓無所措其手足。昔泰山之哭者,以為苛政甚於猛虎,吾備儒者之風,服聖人之遺教,豈可以目翫其辭,行違其誡者哉?廣議輕刑,以惠百姓。」}
十一月庚寅,以兾州饑,遣使者開倉廩振之。戊申晦,日有食之。




十二月,詔曰:「先王制禮,所以昭孝事祖,大則郊社,其次宗廟,三辰五行,名山大川,非此族也,不在祀典。叔世衰亂,崇信巫史,至乃宮殿之內,戶牖之間,無不沃酹,甚矣其惑也。自今,其敢設非祀之祭,巫祝之言,皆以執左道論,著于令典。」是歲穿天淵池。


六年春二月,遣使者循行許昌以東盡沛郡,問民所疾苦,貧者振貸之。
\gezhu{魏略載詔曰:「昔軒轅建四面之號,周武稱『予有亂臣十人』,斯蓋先聖所以體國君民,亮成天工,多賢為貴也。今內有公卿以鎮京師,外設牧伯以監四方,至於元戎出征,則軍中宜有柱石之賢帥,輜重所在,又宜有鎮守之重臣,然後車駕可以周行天下,無內外之慮。吾今當征賊,欲守之積年。其以尚書令潁鄉侯陳羣為鎮軍大將軍,尚書僕射西鄉侯司馬懿為撫軍大將軍。若吾臨江授諸將方略,則撫軍當留許昌,督後諸軍,錄後臺文書事;鎮軍隨車駕,當董督衆軍,錄行尚書事;皆假節鼓吹,給中軍兵騎六百人。吾欲去江數里,築宮室,往來其中,見賊可擊之形,便出奇兵擊之;若或未可,則當舒六軍以遊獵,饗賜軍士。」}
三月,行幸召陵,通討虜渠。乙巳,還許昌宮。并州刺史梁習討鮮卑軻比能,大破之。辛未,帝為舟師東征。五月戊申,幸譙。壬戌,熒惑入太微。




六月,利成郡兵蔡方等以郡反,殺太守徐質。遣屯騎校尉任福、步兵校尉段昭與青州刺史討平之;其見脅略及亡命者,皆赦其罪。


秋七月,立皇子鑒為東武陽王。八月,帝遂以舟師自譙循渦入淮,從陸道幸徐。九月,築東巡臺。冬十月,行幸廣陵故城,臨江觀兵,戎卒十餘萬,旌旗數百里。
\gezhu{魏書載帝於馬上為詩曰:「觀兵臨江水,水流何湯湯!戈矛成山林,玄甲曜日光。猛將懷暴怒,膽氣正從橫。誰云江水廣,一葦可以航,不戰屈敵虜,戢兵稱賢良。古公宅岐邑,實始翦殷商。孟獻營虎牢,鄭人懼稽顙。充國務耕殖,先零自破亡。興農淮泗間,築室都徐方。量宜運權略,六軍咸恱康;豈如東山詩,悠悠多憂傷。」}
是歲大寒,水道冰,舟不得入江,乃引還。十一月,東武陽王鑒薨。十二月,行自譙過梁,遣使以太牢祀故漢太尉橋玄。


七年春正月,將幸許昌,許昌城南門無故自崩,帝心惡之,遂不入。壬子,行還洛陽宮。三月,築九華臺。夏五月丙辰,帝疾篤,召中軍大將軍曹真、鎮軍大將軍陳羣、征東大將軍曹休、撫軍大將軍司馬宣王,並受遺詔輔嗣主。遣後宮淑媛、昭儀已下歸其家。丁巳,帝崩于嘉福殿,時年四十。
\gezhu{魏書曰:殯於崇華前殿。}
六月戊寅,葬首陽陵。自殯及葬,皆以終制從事。
\gezhu{魏氏春秋曰:明帝將送葬,曹真、陳羣、王朗等以暑熱固諫,乃止。孫盛曰:夫窀穸之事,孝子之極痛也,人倫之道,於斯莫重。故天子七月而葬,同軌畢至。夫以義感之情,猶盡臨隧之哀,況乎天性發中,敦禮者重之哉!魏氏之德,仍世不基矣。昔華元厚葬,君子以為棄君於惡,羣等之諫,棄孰甚焉!鄄城侯植為誄曰:「惟黃初七年五月七日,大行皇帝崩,嗚呼哀哉!于時天震地駭,崩山隕霜,陽精薄景,五緯錯行,百姓呼嗟,萬國悲悼,若喪考妣,恩過墓唐,擗踊郊野,仰想穹蒼,僉曰何辜,早世殞喪,嗚呼哀哉!悲夫大行,忽焉光滅,永棄萬國,雲往雨絕。承問荒忽,惛懵哽咽,袖鋒抽刃,歎自僵斃,追慕三良,甘心同穴。感惟南風,惟以鬱滯,終於偕沒,指景自誓。考諸先記,尋之哲言,生若浮寄,惟德可論,朝聞夕逝,孔志所存。皇雖一沒,天祿永延,何以述德?表之素旃。何以詠功?宣之管絃。乃作誄曰:皓皓太素,兩儀始分,中和產物,肈有人倫,爰曁三皇,寔秉道真,降逮五帝,繼以懿純,三代製作,踵武立勳。季嗣不維,網漏于秦,崩樂滅學,儒坑禮焚,二世而殲,漢氏乃因,弗求古訓,嬴政是遵,王綱帝典,闃爾無聞。末光幽昧,道究運遷,乾坤迴歷,簡聖授賢,乃眷大行,屬以黎元。龍飛啟祚,合契上玄,五行定紀,改號革年,明明赫赫,受命于天。仁風偃物,德以禮宣;祥惟聖質,嶷在幼妍。庻幾六典,學不過庭,潛心無罔,亢志青冥。才秀藻朗,如玉之瑩,聽察無嚮,瞻覩未形。其剛如金,其貞如瓊,如氷之絜,如砥之平。爵公無私,戮違無輕,心鏡萬機,攬照下情。思良股肱,嘉昔伊呂,搜揚側陋,舉湯代禹;拔才巖穴,取士蓬戶,唯德是縈,弗拘禰祖。宅土之表,道義是圖,弗營厥險,六合是虞。齊契共遵,下以純民,恢折規矩,克紹前人。科條品制,襃貶以因。乘殷之輅,行夏之辰。金根黃屋,翠葆龍鱗,紼冕崇麗,衡紞惟新,尊肅禮容,矚之若神。方牧妙舉,欽於恤民,虎將荷節,鎮彼四鄰;朱旗所勦,九壤被震,疇克不若?孰敢不臣?縣旌海表,萬里無塵。虜備凶徹,鳥殪江岷,權若涸魚,乾腊矯鱗,肅慎納貢,越裳效珍,條支絕域,侍子內賔。德儕先皇,功侔太古。上靈降瑞,黃初叔祜:河龍洛龜,淩波游下;平鈞應繩,神鸞翔舞;數莢階除,系風扇暑;皓獸素禽,飛走郊野;神鍾寶鼎,形自舊土;雲英甘露,瀸塗被宇;靈芝冒沼,朱華陰渚。回回凱風,祁祁甘雨,稼穡豐登,我稷我黍。家佩惠君,戶蒙慈父。圖致太和,洽德全義。將登介山,先皇作儷。鐫石紀勳,兼錄衆瑞,方隆封禪,歸功天地,賔禮百靈,勳命視規,望祭四嶽,燎封奉柴,肅于南郊,宗祀上帝。三牲旣供,夏禘秋甞,元侯佐祭,獻璧奉璋。鸞輿幽藹,龍旂太常,爰迄太廟,鍾鼓鍠鍠,頌德詠功,八佾鏘鏘。皇祖旣饗,烈考來享,神具醉止,降茲福祥。天地震蕩,大行康之;三辰暗昧,大行光之;皇紘絕維,大行綱之;神器莫統,大行當之;禮樂廢弛,大行張之;仁義陸沈,大行揚之;潛龍隱鳳,大行翔之;疏狄遐康,大行匡之。在位七載,元功仍舉,將永太和,絕迹三五,宜作物師,長為神主,壽終金石,等筭東父,如何奄忽,摧身后土,俾我棾棾,靡瞻靡顧。嗟嗟皇穹,胡寧忍務?嗚呼哀哉!明監吉凶,體遠存亡,深垂典制,申之嗣皇。聖上虔奉,是順是將,乃剏玄宇,基為首陽,擬迹穀林,追堯慕唐,合山同陵,不樹不疆,塗車芻靈,珠玉靡藏。百神警侍,來賔幽堂,耕禽田獸,望魂之翔。於是,俟大隧之致功兮,練元辰之淑禎,潛華體於梓宮兮,馮正殿以居靈。顧望嗣之號咷兮,存臨者之悲聲,悼晏駕之旣修兮,感容車之速征。浮飛魂於輕霄兮,就黃墟以滅形,背三光之昭晰兮,歸玄宅之冥冥。嗟一往之不反兮,痛閟闥之長扃。咨遠臣之眇眇兮,成凶諱以怛驚,心孤絕而靡告兮,紛流涕而交頸。思恩榮以橫奔兮,閡闕塞之嶢崢,顧衰絰以輕舉兮,迫關防之我嬰。欲高飛而遙憩兮,憚天網之遠經,遙投骨於山足兮,報恩養於下庭。慨拊心而自悼兮,懼施重而命輕,嗟微驅之是效兮,甘九死而忘生,幾司命之役籍兮,先黃髮而隕零,天蓋高而察卑兮,兾神明之我聽。獨鬱伊而莫愬兮,追顧景而憐形,奏斯文以寫思兮,結翰墨以敷誠。嗚呼哀哉!」}


初,帝好文學,以著述為務,自所勒成垂百篇。又使諸儒撰集經傳,隨類相從,凡千餘篇,號曰皇覽。
\gezhu{魏書曰:帝初在東宮,疫癘大起,時人彫傷,帝深感歎,與素所敬者大理王朗書曰:「生有七尺之形,死為一棺之土,唯立德揚名,可以不朽,其次莫如著篇籍。疫癘數起,士人彫落,余獨何人,能全其壽?」故論撰所著典論、詩賦,蓋百餘篇,集諸儒於肅城門內,講論大義,侃侃無倦。常嘉漢文帝之為君,寬仁玄默,務欲以德化民,有賢聖之風。時文學諸儒,或以為孝文雖賢,其於聦明,通達國體,不如賈誼。帝由是著太宗論曰:「昔有苗不賔,重華舞以干戚,尉他稱帝,孝文撫以恩德,吳王不朝,錫之几杖以撫其意,而天下賴安;乃弘三章之教,愷悌之化,欲使曩時累息之民,得闊步高談,無危懼之心。若賈誼之才敏,籌畫國政,特賢臣之器,管、晏之姿,豈若孝文大人之量哉?」三年之中,以孫權不服,復班太宗論於天下,明示不願征伐也。他日又從容言曰:「顧我亦有所不取於漢文帝者三:殺薄昭;幸鄧通;慎夫人衣不曳地,集上書囊為帳帷。以為漢文儉而無法,舅后之家,但當養育以恩而不當假借以權,旣觸罪法,又不得不害矣。」其欲秉持中道,以為帝王儀表者如此。胡沖吳歷曰:帝以素書所著典論及詩賦餉孫權,又以紙寫一通與張昭。}


評曰:文帝天資文藻,下筆成章,博聞彊識,才藝兼該;
\gezhu{典論帝自敘曰:初平之元,董卓殺主鴆后,盪覆王室。是時四海旣困中平之政,兼惡卓之凶逆,家家思亂,人人自危。山東牧守咸以春秋之義,「衞人討州吁于濮」,言人人皆得討賊。於是大興義兵,名豪大俠,富室彊族,飄揚雲會,萬里相赴;兖、豫之師戰於滎陽,河內之甲軍於孟津。卓遂遷大駕,西都長安。而山東大者連郡國,中者嬰城邑,小者聚阡陌,以還相吞滅。會黃巾盛於海、岱,山寇暴於并、兾,乘勝轉攻,席卷而南,鄉邑望煙而奔,城郭覩塵而潰,百姓死亡,暴骨如莽。余時年五歲,上以世方擾亂,教余學射,六歲而知射,又教余騎馬,八歲而能騎射矣。以時之多故,每征,余常從。建安初,上南征荊州,至宛,張繡降。旬日而反,亡兄孝廉子脩、從兄安民遇害。時余年十歲,乘馬得脫。夫文武之道,各隨時而用,生於中平之季,長於戎旅之間,是以少好弓馬,于今不衰;逐禽輒十里,馳射常百步,日多體健,心每不猒。建安十年,始定兾州,濊、貊貢良弓,燕、代獻名馬。時歲之暮春,句芒司節,和風扇物,弓燥手柔,草淺獸肥,與族兄子丹獵於鄴西,終日手獲麞鹿九,雉兔三十。後軍南征次內蠡,尚書令荀彧奉使犒軍,見余談論之末,彧言:「聞君善左右射,此實難能。」余言:「執事未覩夫項發口縱,俯馬蹄而仰月支也。」彧喜笑曰:「乃爾!」余曰:「埒有常徑,的有常所,雖每發輒中,非至妙也。若馳平原,赴豐草,要狡獸,截輕禽,使弓不虛彎,所中必洞,斯則妙矣。」時軍祭酒張京在坐,顧彧拊手曰「善」。余又學擊劔,閱師多矣,四方之法各異,唯京師為善。桓、靈之閒,有虎賁王越善斯術,稱於京師。河南史阿言昔與越遊,具得其法,余從阿學之精熟。甞與平虜將軍劉勳、奮威將軍鄧展等共飲,宿聞展善有手臂,曉五兵,又稱其能空手入白刃。余與論劔良久,謂言將軍法非也,余顧甞好之,又得善術,因求與余對。時酒酣耳熱,方食芋蔗,便以為杖,下殿數交,三中其臂,左右大笑。展意不平,求更為之。余言吾法急屬,難相中面,故齊臂耳。展言願復一交,余知其欲突以取交中也,因偽深進,展果尋前,余却脚鄛,正截其顙,坐中驚視。余還坐,笑曰:「昔陽慶使淳于意去其故方,更授以祕術,今余亦願鄧將軍捐棄故伎,更受要道也。」一坐盡歡。夫事不可自謂己長,余少曉持複,自謂無對;俗名雙戟為坐鐵室,鑲楯為蔽木戶;後從陳國袁敏學,以單攻複,每為若神,對家不知所出,先曰若逢敏於狹路,直決耳!余於他戲弄之事少所喜,唯彈棊略盡其巧,少為之賦。昔京師先工有馬合鄉侯、東方安世、張公子,常恨不得與彼數子者對。上雅好詩書文籍,雖在軍旅,手不釋卷,每每定省從容,常言人少好學則思專,長則善忘,長大而能勤學者,唯吾與袁伯業耳。余是以少誦詩、論,及長而備歷五經、四部,史、漢、諸子百家之言,靡不畢覽。博物志曰:帝善彈棊,能用手巾角。時有一書生,又能低頭以所冠著葛巾角撇棊。}
若加之曠大之度,勵以公平之誠,邁志存道,克廣德心,則古之賢主,何遠之有哉!


\end{pinyinscope}