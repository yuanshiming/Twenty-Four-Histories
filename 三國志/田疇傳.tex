\article{田疇傳}
\begin{pinyinscope}
 
 
 田疇字子泰,右北平無終人也。好讀書、擊劒。初平元年,義兵起,董卓遷帝于長安。幽州牧劉虞歎曰:「賊臣作亂,朝廷播蕩,四海俄然,莫有固志。身備宗室遺老,不得自同於衆。今欲奉使展效臣節,安得不辱命之士乎?」衆議咸曰:「田疇雖年少,多稱其奇。」疇時年二十二矣。虞乃備禮請與相見,大恱之,遂署為從事,具其車騎。將行,疇曰:「今道路阻絕,寇虜縱橫,稱官奉使,為衆所指名。願以私行,期於得達而已。」虞從之。疇乃歸,自選其家客與年少之勇壯慕從者二十騎俱往。虞自出祖而遣之。
 
 
\gezhu{先賢行狀曰:疇將行,引虞密與議。疇因說虞曰:「今帝主幼弱,姧臣擅命,表上須報,懼失事機。且公孫瓚阻兵安忍,不早圖之,必有後悔。」虞不聽。}
 旣取道,疇乃更上西關,出塞,傍北方,直趣朔方,循閒徑去,遂至長安致命。詔拜騎都尉。疇以為天子方蒙塵未安,不可以荷佩榮寵,固辭不受。朝廷高其義。三府並辟,皆不就。得報,馳還,未至,虞已為公孫瓚所害。疇至,謁祭虞墓,陳發章表,哭泣而去。瓚聞之大怒,購求獲疇,謂曰:「汝何自哭劉虞墓,而不送章報於我也?」疇荅曰:「漢室衰穨,人懷異心,唯劉公不失忠節。章報所言,於將軍未美,恐非所樂聞,故不進也。且將軍方舉大事以求所欲,旣滅無罪之君,又讎守義之臣,誠行此事,則燕、趙之士將皆蹈東海而死耳,豈忍有從將軍者乎!」瓚壯其對,釋不誅也。拘之軍下,禁其故人莫得與通。或說瓚曰:「田疇義士,君弗能禮,而又囚之,恐失衆心。」瓚乃縱遣疇。
 
 
 
 
 疇得北歸,率舉宗族他附從數百人,埽地而盟曰:「君仇不報,吾不可以立於世!」遂人徐無山中,營深險平敞地而居,躬耕以養父母。百姓歸之,數年閒至五千餘家。疇謂其父老曰:「諸君不以疇不肖,遠來相就。衆成都邑,而莫相統一,恐非乆安之道,願推擇其賢長者以為之主。」皆曰:「善。」同僉推疇。疇曰:「今來在此,非苟安而已,將圖大事,復怨雪恥。竊恐未得其志,而輕薄之徒自相侵侮,偷快一時,無深計遠慮。疇有愚計,願與諸君共施之,可乎?」皆曰:「可。」疇乃為約束相殺傷、犯盜、諍訟之法,法重者至死,其次抵罪,二十餘條。又制為婚姻嫁娶之禮,興舉學校講授之業,班行其衆,衆皆便之,至道不拾遺。北邊翕然服其威信,烏丸、鮮卑並各遣譯使致貢遺,疇悉撫納,令不為寇。袁紹數遣使招命,又即授將軍印,因安輯所統,疇皆拒不受。紹死,其子尚又辟焉,疇終不行。
 
 
疇常忿烏丸昔多賊殺其郡冠蓋,有欲討之意而力未能。建安十二年,太祖北征烏丸,未至,先遣使辟疇,又命田預喻指。疇戒其門下趣治嚴。門人謂曰:「昔袁公慕君,禮命五至,君義不屈;今曹公使一來而君若恐弗及者,何也?」疇笑而應之曰:「此非君所識也。」遂隨使者到軍,署司空戶曹掾,引見諮議。明日出令曰:「田子泰非吾所宜吏者。」即舉茂才,拜為蓨令,不之官,隨軍次無終。時方夏水雨,而濵海洿下,濘滯不通,虜亦遮守蹊要,軍不得進。太祖患之,以問疇。疇曰:「此道,秋夏每常有水,淺不通車馬,深不載舟船,為難乆矣。舊北平郡治在平岡,道出盧龍,達于柳城;自建武以來,陷壞斷絕,垂二百載,而尚有微徑可從。今虜將以大軍當由無終,不得進而退,懈弛無備。若嘿回軍,從盧龍口越白檀之險,出空虛之地,路近而便,掩其不備,蹋頓之首可不戰而禽也。」太祖曰:「善。」乃引軍還,而署大木表於水側路傍曰:「方今暑夏,道路不通,且俟秋冬,乃復進軍。」虜候騎見之,誠以為大軍去也。太祖令疇將其衆為鄉導,上徐無山,出盧龍,歷平岡,登白狼堆,去柳城二百餘里,虜乃驚覺。單于身自臨陣,太祖與交戰,遂大斬獲,追奔逐北,至柳城。軍還入塞,論功行封,封疇亭侯,邑五百戶。
 \gezhu{先賢行狀載太祖表論疇功曰:「文雅優備,忠武又著,和於撫下,慎於事上,量時度理,進退合義。幽州始擾,胡、漢交萃,蕩析離居,靡所依懷。疇率宗人避難於無終山,北拒盧龍,南守要害,清靜隱約,耕而後食,人民化從,咸共資奉。及袁紹父子威力加於朔野,遠結烏丸,與為首尾,前後召疇,終不陷撓。後臣奉命,軍次易縣,疇長驅自到,陳討胡之勢,猶廣武之建燕策,薛公之度淮南。又使部曲持臣露布,出誘胡衆,漢民或因亡來,烏丸聞之震蕩。王旅出塞,塗由山中九百餘里,疇帥兵五百,啟導山谷,遂威烏丸,蕩平塞表。疇文武有效,節義可嘉,誠應寵賞,以旌其美。」}
 疇自以始為居難,率衆循逃,志義不立,反以為利,非本意也,固讓。太祖知其至心,許而不奪。
 \gezhu{魏書載太祖令曰:「昔伯成棄國,夏后不奪,將欲使高尚之士,優賢之主,不止於一世也。其聽疇所執。」}
 
 
遼東斬送袁尚首,令「三軍敢有哭之者斬」。疇以甞為尚所辟,乃往弔祭。太祖亦不問。
 \gezhu{臣松之以為田疇不應袁紹父子之命,以其非正也。故盡規魏祖,建盧龍之策。致使袁尚奔迸,授首遼東,皆疇之由也。旣已明其為賊,胡為復弔祭其首乎?若以甞被辟命,義在其中,則不應為人設謀,使其至此也。疇此舉止,良為進退無當,與王脩哭袁譚,貌同而心異也。}
 疇盡將其家屬及宗人三百餘家居鄴。太祖賜疇車馬穀帛,皆散之宗族知舊。從征荊州還,太祖追念疇功殊美,恨前聽疇之讓,曰:「是成一人之志,而虧王法大制也。」於是乃復以前爵封疇。
 \gezhu{先賢行狀載太祖令曰:「蓨令田疇,志節高尚,遭值州里戎夏交亂,引身深山,研精味道,百姓從之,以成都邑。袁賊之盛,命召不屈。慷慨守志,以徼真主。及孤奉詔征定河北,遂服幽都,將定胡寇,特加禮命。疇即受署,陳建攻胡蹊路所由,率齊山民,一時向化,開塞導道,供承使役,路近而便,令虜不意。斬蹋頓於白狼,遂長驅於柳城,疇有力焉。及軍入塞,將圖其功,表封亭侯,食邑五百,而疇懇惻,前後辭賞。出入三載,歷年未賜,此為成一人之高,甚違王典,失之多矣。宜從表封,無乆留吾過。」}
 疇上疏陳誠,以死自誓。太祖不聽,欲引拜之,至于數四,終不受。有司劾疇狷介違道,苟立小節,宜免官加刑。太祖重其事,依違者乆之。乃下世子及大臣博議,世子以疇同於子文辭祿,申胥逃賞,宜勿奪以優其節。尚書令荀彧、司隷校尉鍾繇亦以為可聽。
 \gezhu{魏書載世子議曰:「昔薳敖逃祿,傳載其美,所以激濁世,勵貪夫,賢於尸祿素餐飡之人也。故可得而小,不可得而毀。至於田疇,方斯近矣。免官加刑,於法為重。」魏略載教曰:「昔夷、齊棄爵而譏武王,可謂愚闇,孔子猶以為『求仁得仁』。疇之所守,雖不合道,但欲清高耳。使天下悉如疇志,即墨翟兼愛尚同之事,而老聃使民結繩之道也。外議欲為復使令司隷決之。」魏書載荀彧議,以為「君子之道,或出或處,期於為善而已。故匹夫守志,聖人各因而成之」。鍾繇以為「原思辭粟,仲尼不與,子路拒牛,謂之止善,雖可以激清勵濁,猶不足多也。疇雖不合大義,有益推讓之風,宜如世子議。」臣松之案呂氏春秋:「魯國之法,魯人有為臣妾於諸侯,有能贖之者取其金於府。子貢贖人而辭不受金,孔子曰:『賜失之矣。自今以來魯人不贖矣。』子路拯溺者,其人拜之以牛,子路受之。孔子曰:『魯人必拯溺矣。』」案此語不與繇所引者相應,未詳為繇之事誤邪,而事將別有所出?}
 太祖猶欲侯之。疇素與夏侯惇善,太祖語惇曰:「且往以情喻之,自從君所言,無告吾意也。」惇就疇宿,如太祖所戒。疇揣知其指,不復發言。惇臨去,乃拊疇背曰:「田君,主意殷勤,曾不能顧乎!」疇荅曰:「是何言之過也!疇,負義逃竄之人耳,蒙恩全活,為幸多矣。豈可賣盧龍之塞,以易賞祿哉?縱國私疇,疇獨不愧於心乎?將軍雅知疇者,猶復如此,若必不得已,請願效死刎首於前。」言未卒,涕泣橫流。惇具荅太祖。太祖喟然知不可屈,乃拜為議郎。年四十六卒。子又早死。文帝踐阼,高疇德義,賜疇從孫續爵關內侯,以奉其嗣。
 
 
\end{pinyinscope}