\article{吳主傳}
\begin{pinyinscope}
 
 
 孫權字仲謀。兄策旣定諸郡,時權年十五,以為陽羨長。
 
 
\gezhu{江表傳曰:堅為下邳丞時,權生,方頤大口,目有精光,堅異之,以為有貴象。及堅亡,策起事江東,權常隨從。性度弘朗,仁而多斷,好俠養士,始有知名,侔於父兄矣。每參同計謀,策甚奇之,自以為不及也。每請會賔客,常顧權曰:「此諸君,汝之將也。」}
 郡察孝廉,州舉茂才,行奉義校尉。漢以策遠脩職貢,遣使者劉琬加錫命。琬語人曰:「吾觀孫氏兄弟雖各才秀明達,然皆祿祚不終,惟中弟孝廉,形貌奇偉,骨體不恒,有大貴之表,年又最壽,爾試識之。」
 
 
 
 
 建安四年,從策征廬江太守劉勳。勳破,進討黃祖於沙羡。
 
 
五年,策薨,以事授權,權哭未及息。策長史張昭謂權曰:「孝廉,此寧哭時邪?且周公立法而伯禽不師,非欲違父,時不得行也。
 \gezhu{臣松之桉禮記曾子問曰:「子夏三年之喪,金革之事無避也者,禮與?初有司與?」孔子曰:「吾聞諸老聃曰,昔者魯公伯禽有為為之也。」鄭玄注曰:「周人卒哭而致事。時有徐戎作難,伯禽卒哭而征之,急王事也。」昭所云「伯禽不師」,蓋謂此也。}
 況今姦宄競逐,豺狼滿道,乃欲哀親戚,顧禮制,是猶開門而揖盜,未可以為仁也。」乃改易權服,扶令上馬,使出巡軍。是時惟有會稽、吳郡、丹楊、豫章、廬陵,然深險之地猶未盡從,而天下英豪布在州郡,賔旅寄寓之士以安危去就為意,未有君臣之固。張昭、周瑜等謂權可與共成大業,故委心而服事焉。曹公表權為討虜將軍,領會稽太守,屯吳,使丞之郡行文書事。待張昭以師傅之禮,而周瑜、程普、呂範等為將率。招延俊秀,聘求名士,魯肅、諸葛瑾等始為賔客。分部諸將,鎮撫山越,討不從命。
 \gezhu{江表傳曰:初策表用李術為廬江太守,策亡之後,不肯事權,而多納其亡叛。權移書求索,術報曰:「有德見歸,無德見叛,不應復還。」權大怒,乃以狀白曹公曰:「嚴刺史昔為公所用,又是州舉將,而李術凶惡,輕犯漢制,殘害州司,肆其無道,宜速誅滅,以懲醜類。今欲討之,進為國朝掃除鯨鯢,退為舉將報塞怨讎,此天下達義,夙夜所甘心。術必懼誅,復詭說求救。明公所居,阿衡之任,海內所瞻,願勑執事,勿復聽受。」是歲舉兵攻術於皖城。術閉門自守,求救於曹公。曹公不救。糧食乏盡,婦女或丸泥而吞之。遂屠其城,梟術首,徙其部曲三萬餘人。}
 
 
 
 
 七年,權母吳氏薨。
 
 
 
 
 八年,權西伐黃祖,破其舟軍,惟城未克,而山寇復動。還過豫章,使呂範平鄱陽、會稽,程普討樂安,太史慈領海昏,韓當、周泰、呂蒙等為劇縣令長。
 
 
九年,權弟丹楊太守翊為左右所害,以從兄瑜代翊。
 \gezhu{吳錄曰:是時權大會官僚,沈友有所是非,令人扶出,謂曰:「人言卿欲反。」友知不得脫,乃曰:「主上在許,有無君之心者,可謂非反乎?」遂殺之。友字子正,吳郡人。年十一,華歆行風俗,見而異之,因呼曰:「沈郎,可登車語乎?」友逡巡却曰:「君子講好,會宴以禮,今仁義陵遲,聖道漸壞,先生銜命,將以裨補先王之教,整齊風俗,而輕脫威儀,猶負薪救火,無乃更崇其熾乎!」歆慙曰:「自桓、靈以來,雖多英彥,未有幼童若此者。」弱冠博學,多所貫綜,善屬文辭。兼好武事,注孫子兵法。又辯於口,每所至,衆人皆默然,莫與為對,咸言其筆之妙,舌之妙,刀之妙,三者皆過絕於人。權以禮聘,旣至,論王霸之略,當時之務,權斂容敬焉。陳荊州宜并之計,納之。正色立朝,清議峻厲,為庸臣所譖,誣以謀反。權亦以終不為己用,故害之,時年二十九。}
 
 
 
 
 十年,權使賀齊討上饒,分為建平縣。
 
 
 
 
 十二年,西征黃祖,虜其人民而還。
 
 
十三年春,權復征黃祖,祖先遣舟兵拒軍,都尉呂蒙破其前鋒,而淩統、董襲等盡銳攻之,遂屠其城。祖挺身亡走,騎士馮則追梟其首,虜其男女數萬口。是歲,使賀齊討黟、歙,
 \gezhu{黟音伊。歙音攝。}
 分歙為始新、新定、
 \gezhu{吳錄曰:晉改新定為遂安。}
 犂陽、休陽縣,
 \gezhu{吳錄曰:晉改休陽為海寧。}
 以六縣為新都郡。荊州牧劉表死,魯肅乞奉命弔表二子,且以觀變。肅未到,而曹公已臨其境,表子琮舉衆以降。劉備欲南濟江,肅與相見,因傳權旨,為陳成敗。備進住夏口,使諸葛亮詣權,權遣周瑜、程普等行。是時曹公新得表衆,形勢甚盛,諸議者皆望風畏懼,多勸權迎之。
 \gezhu{江表傳載曹公與權書曰:「近者奉辭伐罪,旄麾南指,劉琮束手。今治水軍八十萬衆,方與將軍會獵於吳。」權得書以示羣臣,莫不嚮震失色。}
 惟瑜、肅執拒之議,意與權同。瑜、普為左右督,各領萬人,與備俱進,遇於赤壁,大破曹公軍。公燒其餘船引退,士卒饑疫,死者大半。備、瑜等復追至南郡,曹公遂北還,留曹仁、徐晃於江陵,使樂進守襄陽。時甘寧在夷陵,為仁黨所圍,用呂蒙計,留凌統以拒仁,以其半救寧,軍以勝反。權自率衆圍合肥,使張昭攻九江之當塗。昭兵不利,權攻城踰月不能下。曹公自荊州還,遣張喜將騎赴合肥。未至,權退。
 
 
 
 
 十四年,瑜、仁相守歲餘,所殺傷甚衆。仁委城走。權以瑜為南郡太守。劉備表權行車騎將軍,領徐州牧。備領荊州牧,屯公安。
 
 
 
 
 十五年,分豫章為鄱陽郡;分長沙為漢昌郡,以魯肅為太守,屯陸口。
 
 
 
 
 十六年,權徙治秣陵。明年,城石頭,改秣陵為建業。聞曹公將來侵,作濡須塢。
 
 
十八年正月,曹公攻濡須,權與相拒月餘。曹公望權軍,歎其齊肅,乃退。
 \gezhu{吳歷曰:曹公出濡須,作油船,夜渡洲上。權以水軍圍取,得三千餘人,其沒溺者亦數千人。權數挑戰,公堅守不出。權乃自來,乘輕船,從灞須口入公軍。諸將皆以為是挑戰者,欲擊之。公曰:「此必孫權欲身見吾軍部伍也。」勑軍中皆精嚴,弓弩不得妄發。權行五六里,迴還作鼓吹。公見舟船器仗軍伍整肅,喟然歎曰:「生子當如孫仲謀,劉景升兒子若豚犬耳!」權為牋與曹公,說:「春水方生,公宜速去。」別紙言:「足下不死,孤不得安。」曹公語諸將曰:「孫權不欺孤。」乃徹軍還。魏略曰:權乘大船來觀軍,公使弓弩亂發,箭著其船,船偏重將覆,權因迴船,復以一面受箭,箭均船平,乃還。}
 初,曹公恐江濵郡縣為權所略,徵令內移。民轉相驚,自廬江、九江、蘄春、廣陵戶十餘萬皆東渡江,江西遂虛,合肥以南惟有皖城。
 
 
十九年五月,權征皖城。閏月,克之,獲廬江太守朱光及參軍董和,男女數萬口。是歲劉備定蜀。權以備已得益州,令諸葛瑾從求荊州諸郡。備不許,曰:「吾方圖涼州,涼州定,乃盡以荊州與吳耳。」權曰:「此假而不反,而欲以虛辭引歲。」遂置南三郡長吏,關羽盡逐之。權大怒,乃遣呂蒙督鮮于丹、徐忠、孫規等兵二萬取長沙、零陵、桂陽三郡,使魯肅以萬人屯巴丘
 \gezhu{巴丘今曰巴陵。}
 以禦關羽。權住陸口,為諸軍節度。蒙到,二郡皆服,惟零陵太守郝普未下。會備到公安,使關羽將三萬兵至益陽,權乃召蒙等使還助肅。蒙使人誘普,普降,盡得三郡將守,因引軍還,與孫皎、潘璋并魯肅兵並進,拒羽於益陽。未戰,會曹公入漢中,備懼失益州,使使求和。權令諸葛瑾報,更尋盟好,遂分荊州長沙、江夏、桂陽以東屬權,南郡、零陵、武陵以西屬備。備歸,而曹公已還。權反自陸口,遂征合肥。合肥未下,徹軍還。兵皆就路,權與凌統、甘寧等在津北為魏將張遼所襲,統等以死扞權,權乘駿馬越津橋得去。
 \gezhu{獻帝春秋曰:張遼問吳降人:「向有紫髯將軍,長上短下,便馬善射,是誰?」降人荅曰:「是孫會稽。」遼及樂進相遇,言不早知之,急追自得,舉軍歎恨。江表傳曰:權乘駿馬上津橋,橋南已見徹,丈餘無版。谷利在馬後,使權持鞍緩控,利於後著鞭,以助馬勢,遂得超渡。權旣得免,即拜利都亭侯。谷利者,本左右給使也,以謹直為親近監,性忠果亮烈,言不苟且,權愛信之。}
 
 
 
 
 二十一年冬,曹公次于居巢,遂攻濡須。
 
 
 
 
 二十二年春,權令都尉徐詳詣曹公請降,公報使脩好,誓重結婚。
 
 
二十三年十月,權將如吳,親乘馬射虎於庱亭。
 \gezhu{庱音攄陵反。}
 馬為虎所傷,權投以雙戟,虎却廢,常從張世擊以戈,獲之。
 
 
二十四年,關羽圍曹仁於襄陽,曹公遣左將軍于禁救之。會漢水暴起,羽以舟兵盡虜禁等步騎三萬送江陵,惟城未拔。權內憚羽,外欲以為己功,牋與曹公,乞以討羽自效。曹公且欲使羽與權相持以鬬之,驛傳權書,使曹仁以弩射示羽。羽猶豫不能去。閏月,權征羽,先遣呂蒙襲公安,獲將軍士仁。蒙到南郡,南郡太守麋芳以城降。蒙據江陵,撫其老弱,釋于禁之囚。陸遜別取宜都,獲秭歸、枝江、夷道,還屯夷陵,守峽口以備蜀。關羽還當陽,西保麥城。權使誘之。羽偽降,立幡旗為象人於城上,因遁走,兵皆解散,尚十餘騎。權先使朱然、潘璋斷其徑路。十二月,璋司馬馬忠獲羽及其子平、都督趙累等於章鄉,遂定荊州。是歲大疫,盡除荊州民租稅。曹公表權為驃騎將軍,假節、領荊州牧,封南昌侯。權遣校尉梁寓奉貢于漢,及令王惇市馬,又遣朱光等歸。
 \gezhu{魏略曰:梁寓字孔儒,吳人也。權遣寓觀望曹公,曹公因以為掾,尋遣還南。}
 
 
二十五年春正月,曹公薨,太子丕代為丞相魏王,改年為延康。秋,魏將梅敷使張儉求見撫納。南陽陰、酇、筑陽、
 \gezhu{筑音逐。}
 山都、中盧五縣民五千家來附。冬,魏嗣王稱尊號,改元為黃初。二年四月,劉備稱帝於蜀。
 \gezhu{魏略曰:權聞魏文帝受禪而劉備稱帝,乃呼問知星者,己分野中星氣何如,遂有僭意。而以位次尚少,無以威衆,又欲先卑而後踞之,為卑則可以假寵,後踞則必致討,致討然後可以怒衆,衆怒然後可以自大,故深絕蜀而專事魏。}
 權自公安都鄂,改名武昌,以武昌、下雉、尋陽、陽新、柴桑、沙羨六縣為武昌郡。五月,建業言甘露降。八月,城武昌,下令諸將曰:「夫存不忘亡,安必慮危,古之善教。昔儁不疑漢之名臣,於安平之世而刀劒不離於身,蓋君子之於武備,不可以已。況今處身疆畔,豺狼交接,而可輕忽不思變難哉?頃聞諸將出入,各尚謙約,不從人兵,甚非備慮愛身之謂。夫保己遺名,以安君親,孰與危辱?宜深警戒,務崇其大,副孤意焉。」自魏文帝踐阼,權使命稱藩,及遣于禁等還。十一月,策命權曰:「蓋聖王之法,以德設爵,以功制祿;勞大者祿厚,德盛者禮豐。故叔旦有夾輔之勳,太公有鷹揚之功,並啟土宇,并受備物,所以表章元功,殊異賢哲也。近漢高祖受命之初,分裂膏腴以王八姓,斯則前世之懿事,後王之元龜也。朕以不德,承運革命,君臨萬國,秉統天機,思齊先代,坐而待旦。惟君天資忠亮,命世作佐,深覩歷數,達見廢興,遠遣行人,浮于潛漢。
 \gezhu{禹貢曰:沲、潛旣道,注曰:「水自江出為沲,漢為潛。」}
 望風影附,抗疏稱藩,兼納纖絺南方之貢,普遣諸將來還本朝,忠肅內發,款誠外昭,信著金石,義蓋山河,朕甚嘉焉。今封君為吳王,使使持節太常高平侯貞,授君璽綬策書、金虎符第一至第五、左竹使符第一至第十,以大將軍使持節督交州,領荊州牧事,錫君青土,苴以白茅,對揚朕命,以尹東夏。其上故驃騎將軍南昌侯印綬符策。今又加君九錫,其敬聽後命。以君綏安東南,綱紀江外,民夷安業,無或攜貳,是用錫君大輅、戎輅各一,玄牡二駟。君務財勸農,倉庫盈積,是用錫君衮冕之服,赤舄副焉。君化民以德,禮教興行,是用錫君軒縣之樂。君宣導休風,懷柔百越,是用錫君朱戶以居。君運其才謀,官方任賢,是用錫君納陛以登。君忠勇並奮,清除姦慝,是用錫君虎賁之士百人。君振威陵邁,宣力荊南,梟滅凶醜,罪人斯得,是用錫君鈇鉞各一。君文和於內,武信于外,是用錫君彤弓一、彤矢百、玈弓十、玈矢千。君以忠肅為基,恭勤為德,是用錫君秬鬯一卣,圭瓚副焉。欽哉!敬敷訓典,以服朕命,以勗相我國家,永終爾顯烈。」
 \gezhu{江表傳曰:權羣臣議,以為宜稱上將軍九州伯,不應受魏封。權曰:「九州伯,於古未聞也。昔沛公亦受項羽拜為漢王,此蓋時宜耳,復何損邪?」遂受之。孫盛曰:「昔伯夷、叔齊不屈有周,魯仲連不為秦民。夫以匹夫之志,猶義不辱,況列國之君參分天下,而可二三其節,或臣或否乎?余觀吳、蜀,咸稱奉漢,至於漢代,莫能固秉臣節,君子是以知其不能克昌厥後,卒見吞於大國也。向使權從羣臣之議,終身稱漢將,豈不義悲六合,仁感百世哉!」}
 是歲,劉備帥軍來伐,至巫山、秭歸,使使誘導武陵蠻夷,假與印傳,許之封賞。於是諸縣及五谿民皆反為蜀。權以陸遜為督,督朱然、潘璋等以拒之。遣都尉趙咨使魏。魏帝問曰:「吳王何等主也?」咨對曰:「聦明仁智,雄略之主也。」帝問其狀,咨曰:「納魯肅於凡品,是其聦也;拔呂蒙於行陣,是其明也;獲于禁而不害,是其仁也;取荊州而兵不血刃,是其智也;據三州虎視於天下,是其雄也;屈身於陛下,是其略也。」
 \gezhu{吳書曰:咨字德度,南陽人,博聞多識,應對辯捷,權為吳王,擢至中大夫,使魏。魏文帝善之,嘲咨曰:「吳王頗知學乎?」荅曰:「吳王浮江萬艘,帶甲百萬,任賢使能,志存經略,雖有餘閑,博覽書傳歷史,藉採奇異,不效書生尋章擿句而已。」帝曰:「吳可征不?」咨對曰:「大國有征伐之兵,小國有備禦之固。」又曰:「吳難魏不?」咨曰:「帶甲百萬,江、漢為池,何難之有?」又曰:「吳如大夫者幾人?」咨曰:「聦明特達者八九十人,如臣之比,車載斗量,不可勝數。」咨頻載使,北人敬異。權聞而嘉之,拜騎都尉。咨言曰:「觀北方終不能守盟,今日之計,朝廷承漢四百之際,應東南之運,宜改年號,正服色,以應天順民。」權納之。}
 帝欲封權子登,權以登年幼,上書辭封,重遣西曹掾沈珩陳謝,并獻方物。
 \gezhu{吳書曰:珩字仲山,吳郡人,少綜經藝,尤善春秋內、外傳。權以珩有智謀,能專對,乃使至魏。魏文帝問曰:「吳嫌魏東向乎?」珩曰:「不嫌。」曰:「何以?」曰:「信恃舊盟,言歸于好,是以不嫌。若魏渝盟,自有豫備。」又問:「聞太子當來,寧然乎?」珩曰:「臣在東朝,朝不坐,宴不與,若此之議,無所聞也。」文帝善之,乃引珩自近,談語終日。珩隨事響應,無所屈服。珩還言曰:「臣密參侍中劉曄,數為賊設姦計,終不乆愨。臣聞兵家舊論,不恃敵之不我犯,恃我之不可犯,今為朝廷慮之。且當省息他役,惟務農桑以廣軍資;脩繕舟車,增作戰具,令皆兼盈;撫養兵民,使各得其所;擥延英俊,獎勵將士,則天下可圖矣。」以奉使有稱,封永安鄉侯,官至少府。}
 立登為王太子。
 \gezhu{江表傳曰:是歲魏文帝遣使求雀頭香、大貝、明珠、象牙、犀角、瑇瑁、孔雀、翡翠、鬪鴨、長鳴雞。羣臣奏曰:「荊、揚二州,貢有常典,魏所求珍玩之物非禮也,宜勿與。」權曰:「昔惠施尊齊為王,客難之曰:『公之學去尊,今王齊,何其倒也?』惠子曰:『有人於此,欲擊其愛子之頭,而石可以代之,子頭所重而石所輕也,以輕代重,何為不可乎?』方有事於西北,江表元元,恃主為命,非我愛子邪?彼所求者,於我瓦石耳,孤何惜焉?彼在諒闇之中,而所求若此,寧可與言禮哉!」皆具以與之。}
 
 
黃武元年春正月,陸遜部將軍宋謙等攻蜀五屯,皆破之,斬其將。三月,鄱陽言黃龍見。蜀軍分據險地,前後五十餘營,遜隨輕重以兵應拒,自正月至閏月,大破之,臨陣所斬及投兵降首數萬人。劉備奔走,僅以身免。
 \gezhu{吳歷曰:權以使聘魏,具上破備獲印綬及首級、所得土地,并表將吏功勤宜加爵賞之意。文帝報使,致鼲子裘、明光鎧、騑馬,又以素書所作典論及詩賦與權。魏書載詔荅曰:「老虜邊窟,越險深入,曠日持乆,內迫罷弊,外困智力,故見身於雞頭,分兵擬西陵,其計不過謂可轉足前迹以搖動江東。根未著地,摧折其支,雖未刳備五臟,使身首分離,其所降誅,亦足使虜部衆兇懼。昔吳漢先燒荊門,後發夷陵,而子陽無所逃其死;來歙始襲略陽,文叔喜之,而知隗嚻無所施其巧。今討此虜,正似其事,將軍勉建方略,務全獨克。」}
 
 
初權外託事魏,而誠心不款。魏乃遣侍中辛毗、尚書桓階往與盟誓,并徵任子,權辭讓不受。秋九月,魏乃命曹休、張遼、臧霸出洞口,曹仁出濡須,曹真、夏侯尚、張郃、徐晃圍南郡。權遣呂範等督五軍,以舟軍拒休等,諸葛瑾、潘璋、楊粲救南郡,朱桓以濡須督拒仁。時楊、越蠻夷多未平集,內難未弭,故權卑辭上書,求自改厲,「若罪在難除,必不見置,當奉還土地民人,乞寄命交州,以終餘年。」文帝報曰:「君生於擾攘之際,本有從橫之志,降身奉國,以享茲祚。自君策名已來,貢獻盈路。討備之功,國朝仰成。埋而掘之,古人之所恥。
 \gezhu{國語曰:狸埋之,狸掘之,是以無成功。}
 朕之與君,大義已定,豈樂勞師遠臨江漢?廊廟之議,王者所不得專;三公上君過失,皆有本末。朕以不明,雖有曾母投杼之疑,猶兾言者不信,以為國福。故先遣使者犒勞,又遣尚書、侍中踐脩前言,以定任子。君遂設辭,不欲使進,議者恠之。
 \gezhu{魏略載魏三公奏曰:「臣聞枝大者披心,尾大者不掉,有國有家之所慎也。昔漢承秦弊,天下新定,大國之王,臣節未盡,以蕭、張之謀不備錄之,至使六王前後反叛,已而伐之,戎車不輟。又文、景守成,忘戰戢役,驕縱吳、楚,養虺成虵,旣為社稷大憂,蓋前事之不忘,後事之師也。吳王孫權,幼豎小子,無尺寸之功,遭遇兵亂,因父兄之緒,少蒙翼卵煦伏之恩,長含鴟梟反逆之性,背棄天施,罪惡積大。復與關羽更相覘伺,逐利見便,挾為卑辭。先帝知權姦以求用,時以于禁敗於水災,等當討羽,因以委權。先帝委裘下席,權不盡心,誠在惻怛,欲因大喪,寡弱王室,希託董桃傳先帝令,乘未得報許,擅取襄陽,及見驅逐,乃更折節。邪辟之態,巧言如流,雖重驛累使,發遣禁等,內包隗嚻顧望之姦,外欲緩誅,支仰蜀賊。聖朝含弘,旣加不忍,優而赦之,與之更始,猥乃割地王之,使南面稱孤,兼官累位,禮備九命,名馬百駟,以成其勢,光寵顯赫,古今無二。權為犬羊之姿,橫被虎豹之文,不思靜力致死之節,以報無量不世之恩。臣每見所下權前後章表,又以愚意採察權旨,自以阻帶江湖,負固不服,狃忲累世,詐偽成功,上有尉他、英布之計,下誦伍被屈彊之辭,終非不侵不叛之臣。以為晁錯不發削弱王侯之謀,則七國同衡,禍乆而大;蒯通不決襲歷下之策,則田橫自慮,罪深變重。臣謹考之周禮九伐之法,平權凶惡,逆節萌生,見罪十五。昔九黎亂德,黃帝加誅;項羽罪十,漢祖不捨。權所犯罪釁明白,非仁恩所養,宇宙所容。臣請免權官,鴻臚削爵土,捕治罪。敢有不從,移兵進討,以明國典好惡之常,以靜三州元元之苦。」其十五條,文多不載。}
 又前都尉浩周勸君遣子,乃實朝臣交謀,以此卜君,君果有辭,外引隗嚻遣子不終,內喻竇融守忠而已。世殊時異,人各有心。浩周之還,口陳指麾,益令議者發明衆嫌,終始之本,無所據仗,故遂俛仰從羣臣議。今省上事,款誠深至,心用慨然,悽愴動容。即日下詔,勑諸軍但深溝高壘,不得妄進。若君必效忠節,以解疑議,登身朝到,夕召兵還。此言之誠,有如大江!」
 \gezhu{魏略曰:浩周字孔異,上黨人。建安中仕為蕭令,至徐州刺史。後領護于禁軍,軍沒,為關羽所得。權襲羽,并得周,甚禮之。及文帝即王位,權乃遣周,為牋魏王曰:「昔討關羽,獲于將軍,即白先王,當發遣之。此乃奉款之心,不言而發。先王未深留意,而謂權中間復有異圖,愚情慺慺,用未果決。遂值先王委離國祚,殿下承統,下情始通。公私契闊,未獲備舉,是令本誓未即昭顯。梁寓傳命,委曲周至,深知殿下以為意望。權之赤心,不敢有他,願垂明恕,保權所執。謹遣浩周、東里衮,至情至實,皆周等所具。」又曰:「權本性空薄,文武不昭,昔承父兄成軍之緒,得為先王所見獎飾,遂因國恩,撫綏東土。而中閒寡慮,庶事不明,畏威忘德,以取重戾。先王恩仁,不忍遐棄,旣釋其宿罪,且開明信。雖致命虜庭,梟獲關羽,功效淺薄,未報萬一。事業未究,先王即世。殿下踐阼,威仁流邁,私懼情願未蒙昭察。梁寓來到,具知殿下不遂疏遠,必欲撫錄,追本先緒。權之得此,欣然踊躍,心開目明,不勝其慶。權世受寵遇,分義深篤,今日之事,永執一心,惟察慺慺,重垂含覆。」又曰:「先王以權推誠已驗,軍當引還,故除合肥之守,著南北之信,令權長驅不復後顧。近得守將周泰、全琮等白事,過月六日,有馬步七百,徑到橫江,又督將馬和復將四百人進到居巢,琮等聞有兵馬渡江,視之,為兵馬所擊,臨時交鋒,大相殺傷。卒得此問,情用恐懼。權實在遠,不豫聞知,約勑無素,敢謝其罪。又聞張征東、朱橫海今復還合肥,先王盟要,由來未乆,且權自度未獲罪釁,不審今者何以發起,牽軍遠次?事業未訖,甫當為國討除賊備,重聞斯問,深使失圖。凡遠人所恃,在於明信,願殿下克卒前分,開示坦然,使權誓命,得卒本規。凡所願言,周等所當傳也。」初東里衮為于禁軍司馬,前與周俱沒,又俱還到,有詔皆見之。帝問周等,周以為權必臣服,而東里衮謂其不可必服。帝恱周言,以為有以知之。是歲冬,魏王受漢禪,遣使以權為吳王,詔使周與使者俱往。周旣致詔命,時與權私宴,謂權曰:「陛下未信王遣子入侍也,周以闔門百口明之。」權因字謂周曰:「浩孔異,卿乃以舉家百口保我,我當何言邪?」遂流涕沾襟。及與周別,又指天為誓。周還之後,權不遣子而設辭,帝乃乆留其使。到八月,權上書謝,又與周書曰:「自道路開通,不忘脩意。旣新奉國命,加知起居,假歸河北,故使情問不獲果至。望想之勞,曷云其已。孤以空闇,分信不昭,中間招罪,以取棄絕,幸蒙國恩,復見赦宥,喜乎與君克卒本圖。傳不云乎,雖不能始,善終可也。」又曰:「昔君之來,欲令遣子入侍,于時傾心歡以承命,徒以登年幼,欲假年歲之閒耳。而赤情未蒙昭信,遂見討責,常用慙怖。自頃國恩復加開導,忘其前愆,取其後效,喜得因此尋竟本誓。前以有表具說遣子之意,想君假還,已知之也。」又曰:「今子當入侍,而未有妃耦,昔君念之,以為可上連綴宗室若夏侯氏,雖中閒自棄,常奉戢在心。當垂宿念,為之先後,使獲攀龍附驥,永自固定。其為分惠,豈有量哉!如是欲遣孫長緒與小兒俱入,奉行禮聘,成之在君。」又曰:「小兒年弱,加教訓不足,念當與別,為之緬然,父子恩情,豈有已邪!又欲遣張子布追輔護之。孤性無餘,凡所欲為,今盡宣露。惟恐赤心不先暢達,是以具為君說之,宜明所以。」於是詔曰:「權前對浩周,自陳不敢自遠,樂委質長為外臣,又前後辭旨,頭尾擊地,此鼠子自知不能保爾許地也。又今與周書,請以十二月遣子,復欲遣孫長緒、張子布隨子俱來,彼二人皆權股肱心腹也。又欲為子於京師求婦,此權無異心之明效也。」帝旣信權甘言,且謂周為得其真,而權但華偽,竟無遣子意。自是之後,帝旣彰權罪,周亦見踈遠,終身不用。}
 權遂改年,臨江拒守。冬十一月,大風,範等兵溺死者數千,餘軍還江南。曹休使臧霸以輕船五百、敢死萬人襲攻徐陵,燒攻城車,殺略數千人。將軍全琮、徐盛追斬魏將尹盧,殺獲數百。十二月,權使太中大夫鄭泉聘劉備于白帝,始復通也。
 \gezhu{江表傳曰:權云:「近得玄德書,已深引咎,求復舊好。前所以名西為蜀者,以漢帝尚存故耳,今漢已廢,自可名為漢中王也。」吳書曰:鄭泉字文淵,陳郡人。博學有奇志,而性嗜酒,其閑居每曰:「願得美酒滿五百斛船,以四時甘脆置兩頭,反覆沒飲之,憊即住而啖肴膳。酒有斗升減,隨即益之,不亦快乎!」權以為郎中。嘗與之言:「卿好於衆中面諫,或失禮敬,寧畏龍鱗乎?」對曰:「臣聞君明臣直,今值朝廷上下無諱,實恃洪恩,不畏龍鱗。」後侍讌,權乃怖之,使提出付有司促治罪。臨出屢顧,權呼還,笑曰:「卿言不畏龍鱗,何以臨出而顧乎?」對曰:「實侍恩覆,知無死憂,至當出閤,感惟威靈,不能不顧耳。」使蜀,劉備問曰:「吳王何以不荅吾書,得無以吾正名不宜乎?」泉曰:「曹操父子陵轢漢室,終奪其位。殿下託為宗室,有維城之責,不荷戈執殳為海內率先,而於是自名,未合天下之議,是以寡君未復書耳。」備甚慙恧。泉臨卒,謂同類曰:「必葬我陶家之側,庶百歲之後化而成土,幸見取為酒壺,實獲我心矣。」}
 然猶與魏文帝相往來,至後年乃絕。是歲改夷陵為西陵。
 
 
二年春正月,曹真分軍據江陵中州。是月,城江夏山。改四分,用乾象歷。
 \gezhu{江表傳曰:權推五德之運,以為土行用未祖辰臘。志林曰:土行以辰臘,得其數矣。土盛於戌,而以未祖,其義非也。土生於未,故未為坤初。是以月令:建未之月,祀黃精於郊,祖用其盛。今祖用其始,豈應運乎?}
 三月,曹仁遣將軍常彫等,以兵五千,乘油舩,晨渡濡須中州。仁子泰因引軍急攻朱桓,桓兵拒之,遣將軍嚴圭等擊破彫等。是月,魏軍皆退。夏四月,權羣臣勸即尊號,權不許。
 \gezhu{江表傳曰:權辭讓曰:「漢家堙替,不能存救,亦何心而競乎?」羣臣稱天命符瑞,固重以請。權未之許,而謂將相曰:「往年孤以玄德方向西鄙,故先命陸遜選衆以待之。聞北部分,欲以助孤,孤內嫌其有挾,若不受其拜,是相折辱而趣其速發,便當與西俱至,二處受敵,於孤為劇,故自抑桉,就其封王。低屈之趣,諸君似未之盡,今故以此相解耳。」}
 劉備薨于白帝。
 \gezhu{吳書曰:權遣立信都尉馮熙聘于蜀,弔備喪也。熙字子柔,潁川人,馮異之後也。權之為車騎,熙歷東曹掾,使蜀還,為中大夫。後使于魏,文帝問曰:「吳王若欲脩宿好,宜當厲兵江關,縣旍巴蜀,而聞復遣脩好,必有變故。」熙曰:「臣聞西使直報問,且以觀釁,非有謀也。」又曰:「聞吳國比年災旱,人物彫損,以大夫之明,觀之何如?」熙對曰:「吳王體量聦明,善於任使,賦政施役,每事必咨,敬養賔旅,親賢愛士,賞不擇怨仇,而罰必加有罪,臣下皆感恩懷德,惟忠與義。帶甲百萬,穀帛如山,稻田沃野,民無饑歲,所謂金城湯池,彊富之國也。以臣觀之,輕重之分,未可量也。」帝不恱,以陳羣與熙同郡,使羣誘之,啗以重利。熙不為迴。送至摩陂,欲困苦之。後又召還,未至,熙懼見迫不從,必危身辱命,乃引刀自刺。御者覺之,不得死。權聞之,垂涕曰:「此與蘇武何異?」竟死於魏。}
 五月,曲阿言甘露降。先是戲口守將晉宗殺將王直,以衆叛如魏,魏以為蘄春太守,數犯邊境。六月,權令將軍賀齊督糜芳、劉邵等襲蘄春,邵等生虜宗。冬十一月,蜀使中郎將鄧芝來聘。
 \gezhu{吳歷曰:蜀致馬二百匹,錦千端,及方物。自是之後,聘使往來以為常。吳亦致方土所出,以荅其厚意焉。}
 
 
三年夏,遣輔義中郎將張溫聘于蜀。秋八月,赦死罪。九月,魏文帝出廣陵,望大江,曰「彼有人焉,未可圖也」,乃還。
 \gezhu{干寶晉紀曰:魏文帝之在廣陵,吳人大駭,乃臨江為疑城,自石頭至于江乘,車以木楨,衣以葦席,加采飾焉,一夕而成。魏人自江西望,甚憚之,遂退軍。權令趙達筭之,曰:「曹丕走矣,雖然,吳衰庚子歲。」權曰:「幾何?」達屈指而計之,曰:「五十八年。」權曰:「今日之憂,不暇及遠,此子孫事也。」吳錄曰:是歲蜀主又遣鄧芝來聘,重結盟好。權謂芝曰:「山民作亂,江邊守兵多徹,慮曹丕乘空弄態,而反求和。議者以為內有不暇,幸來求和,於我有利,宜當與通,以自辨定。恐西州不能明孤赤心,用致嫌疑。孤土地邊外,閒隙萬端,而長江巨海,皆當防守。丕觀釁而動,惟不見便,寧得忘此,復有他圖。」}
 
 
四年夏五月,丞相孫邵卒。
 \gezhu{吳錄曰:邵字長緒,北海人,長八尺。為孔融功曹,融稱曰「廊廟才也」。從劉繇於江東。及權統事,數陳便宜,以為應納貢聘,權即從之。拜廬江太守,遷車騎長史。黃武初為丞相,威遠將軍,封陽羨侯。張溫、曁豔奏其事,邵辭位請罪,權釋令復職,年六十三卒。志林曰:吳之創基,邵為首相,史無其傳,竊常恠之。嘗問劉聲叔。聲叔,博物君子也,云:「推其名位,自應立傳。項竣、吴孚時已有注記,此云與張惠恕不能。後韋氏作史,蓋惠恕之黨,故不見書。」}
 六月,以太常顧雍為丞相。
 \gezhu{吳書曰:以尚書令陳化為太常。化字元耀,汝南人,博覽衆書,氣幹剛毅,長七尺九寸,雅有威容。為郎中令使魏,魏文帝因酒酣,嘲問曰:「吳、魏峙立,誰將平一海內者乎?」化對曰:「易稱帝出乎震,加聞先哲知命,舊說紫蓋黃旗,運在東南。」帝曰:「昔文王以西伯王天下,豈復在東乎?」化曰:「周之初基,太伯在東,是以文王能興於西。」帝笑,無以難,心奇其辭。使畢當還,禮送甚厚。權以化奉命光國,拜犍為太守,置官屬。頃之,遷太常,兼尚書令。正色立朝,勑子弟廢田業,絕治產,仰官廩祿,不與百姓爭利。妻早亡,化以古事為鑒,乃不復娶。權聞而貴之,以其年壯,勑宗正妻以宗室女,化固辭以疾,權不違其志。年出七十,乃上疏乞骸骨,遂爰居章安,卒於家。長子熾,字公熙,少有志操,能計筭。衞將軍全琮表稱熾任大將軍,赴召,道卒。}
 皖口言木連理。冬十二月,鄱陽賊彭綺自稱將軍,攻沒諸縣,衆數萬人。是歲地連震。
 \gezhu{吳錄曰:是冬魏文帝至廣陵,臨江觀兵,兵有十餘萬,旌旗彌數百里,有渡江之志。權嚴設固守。時大寒冰,舟不得入江。帝見波濤洶涌,歎曰:「嗟乎!固天所以隔南北也!」遂歸。孫韶又遣將高壽等率敢死之士五百人於徑路夜要之,帝大驚,壽等獲副車羽蓋以還。}
 
 
五年春,令曰:「軍興日乆,民離農畔,父子夫婦,不能相卹,孤甚愍之。今北虜縮竄,方外無事,其下州郡,有以寬息。」是時陸遜以所在少穀,表令諸將增廣農畒。權報曰:「甚善。今孤父子親自受田,車中八牛以為四耦,雖未及古人,亦欲與衆均等其勞也。」秋七月,權聞魏文帝崩,征江夏,圍石陽,不克而還。蒼梧言鳳皇見。分三郡惡地十縣置東安郡,
 \gezhu{吳錄曰:郡治富春也。}
 以全琮為太守,平討山越。冬十月,陸遜陳便宜,勸以施德緩刑,寬賦息調。又云:「忠讜之言,不能極陳,求容小臣,數以利聞。」權報曰:「夫法令之設,欲以遏惡防邪,儆戒未然也,焉得不有刑罰以威小人乎?此為先令後誅,不欲使有犯者耳。君以為太重者,孤亦何利其然,但不得已而為之耳。今承來意,當重諮謀,務從其可。且近臣有盡規之諫,親戚有補察之箴,所以匡君正主明忠信也。書載『予違汝弼,汝無面從』,孤豈不樂忠言以自裨補邪?而云「不敢極陳」,何得為忠讜哉?若小臣之中,有可納用者,寧得以人廢言而不採擇乎?但諂媚取容,雖闇亦所明識也。至於發調者,徒以天下未定,事以衆濟。若徒守江東,脩崇寬政,兵自足用,復用多為?顧坐自守可陋耳。若不豫調,恐臨時未可便用也。又孤與君分義特異,榮戚實同,來表云不敢隨衆容身苟免,此實甘心所望於君也。」於是令有司盡寫科條,使郎中褚逢齎以就遜及諸葛瑾,意所不安,令損益之。是歲,分交州置廣州,俄復舊。
 \gezhu{江表傳曰:權於武昌新裝大船,名為長安,試泛之釣臺沂。時風大盛,谷利令柂工取樊口。權曰:「當張頭取羅州。」利拔刀向柂工曰:「不取樊口者斬。」工即轉柂入樊口,風遂猛不可行,乃還。權曰:「阿利畏水何怯也?」利跪曰:「大王萬乘之主,輕於不測之淵,戲於猛浪之中,船樓裝高,邂逅顛危,柰社稷何?是以利輒敢以死爭。」權於是貴重之,自此後不復名之,常呼曰谷。}
 
 
 
 
 六年春正月,諸將獲彭綺。閏月,韓當子綜以其衆降魏。
 
 
七年春三月,封子慮為建昌侯。罷東安郡。夏五月,鄱陽太守周魴偽叛,誘魏將曹休。秋八月,權至皖口,使將軍陸遜督諸將大破休於石亭。大司馬呂範卒。是歲,改合浦為珠官郡。
 \gezhu{江表傳曰:是歲將軍翟丹叛如魏。權恐諸將畏罪而亡,乃下令曰:「自今諸將有重罪三,然後議。」}
 
 
黃龍元年春,公卿百司皆勸權正尊號。夏四月,夏口、武昌並言黃龍、鳳凰見。丙申,南郊即皇帝位,
 \gezhu{吳錄載權告天文曰:「皇帝臣權敢用玄牡昭告于皇皇后帝:漢享國二十有四世,歷年四百三十有四,行氣數終,祿祚運盡,普天弛絕,率土分崩。孽臣曹丕遂奪神器,丕子叡繼世作慝,淫名亂制。權生於東南,遭值期運,承乾秉戎,志在平世,奉辭行罰,舉足為民。羣臣將相,州郡百城,執事之人,咸以為天意已去於漢,漢氏已絕祀於天,皇帝位虛,郊祀無主。休徵嘉瑞,前後雜沓,歷數在躬,不得不受。權畏天命,不敢不從,謹擇元日,登壇燎祭,即皇帝位。惟爾有神饗之,左右有吳,永終天祿。」}
 是日大赦,改年。追尊父破虜將軍堅為武烈皇帝,母吳氏為武烈皇后,兄討逆將軍策為長沙桓王。吳王太子登為皇太子。將吏皆進爵加賞。初,興平中,吳中童謠曰:「黃金車,班蘭耳,闓昌門,出天子。」
 \gezhu{昌門,吳西郭門,夫差所作。}
 五月,使校尉張剛、管篤之遼東。六月,蜀遣衞尉陳震慶權踐位。權乃參分天下,豫、青、徐、幽屬吳,兖、兾、并、涼屬蜀。其司州之土,以函谷關為界,造為盟曰:「天降喪亂,皇綱失叙,逆臣承釁,劫奪國柄,始於董卓,終於曹操,窮凶極惡,以覆四海,至令九州幅裂,普天無統,民神痛怨,靡所戾止。及操子丕,桀逆遺醜,荐作姦回,偷取天位,而叡么麼,尋丕凶蹟,阻兵盜土,未伏厥誅。昔共工亂象而高辛行師,三苗干度而虞舜征焉。今日滅叡,禽其徒黨,非漢與吳,將復誰在?夫討惡翦暴,必聲其罪,宜先分製,奪其土地,使士民之心,各知所歸。是以春秋晉侯伐衞,先分其田以畀宋人,斯其義也。且古建大事,必先盟誓,故周禮有司盟之官,尚書有告誓之文,漢之與吳,雖信由中,然分土裂境,宜有盟約。諸葛丞相德威遠著,翼戴本國,典戎在外,信感陰陽,誠動天地,重復結盟,廣誠約誓,使東西士民咸共聞知。故立壇殺牲,昭告神明,再歃加書,副之天府。天高聽下,靈威棐諶,司慎司盟,羣臣羣祀,莫不臨之。自今日漢、吳旣盟之後,勠力一心,同討魏賊,救危恤患,分災共慶,好惡齊之,無或攜貳。若有害漢,則吳伐之;若有害吳,則漢伐之。各守分土,無相侵犯。傳之後葉,克終若始。凡百之約。皆如載書。信言不豔,實居于好。有渝此盟,創禍先亂,違貳不恊,慆慢天命,明神上帝是討是督,山川百神是糾是殛,俾墜其師,無克祚國。于爾大神,其明鑒之!」秋九月,權遷都建業,因故府不改館,徵上大將軍陸遜輔太子登,掌武昌留事。
 
 
 
 
 二年春正月,魏作合肥新城。詔立都講祭酒,以教學諸子。遣將軍衞溫、諸葛直將甲士萬人浮海求夷洲及亶洲。亶洲在海中,長老傳言秦始皇帝遣方士徐福將童男童女數千人入海,求蓬萊神山及仙藥,止此洲不還。世相承有數萬家,其上人民,時有至會稽貨布,會稽東縣人海行,亦有遭風流移至亶洲者。所在絕遠,卒不可得至,但得夷洲數千人還。
 
 
 
 
 三年春二月,遣太常潘濬率衆五萬討武陵蠻夷。衞溫、諸葛直皆以違詔無功,下獄誅。夏,有野蠶成繭,大如卵。由拳野稻自生,改為禾興縣。中郎將孫布詐降以誘魏將王淩,淩以軍迎布。冬十月,權以大兵潛伏於阜陵俟之,淩覺而走。會稽南始平言嘉禾生。十二月丁卯,大赦,改明年元也。
 
 
嘉禾元年春正月,建昌侯慮卒。三月,遣將軍周賀、校尉裴潛乘海之遼東。秋九月,魏將田豫要擊,斬賀于成山。冬十月,魏遼東太守公孫淵遣校尉宿舒、閬中令孫綜稱藩於權,并獻貂馬。權大恱,加淵爵位。
 \gezhu{江表傳曰:是冬,羣臣以權未郊祀,奏議曰:「頃者嘉瑞屢臻,遠國慕義,天意人事,前後備集,宜脩郊祀,以承天意。」權曰:「郊祀當於土中,今非其所,於何施此?」重奏曰:「普天之下,莫非王土;王者以天下為家。昔周文、武郊於酆、鎬,非必土中。」權曰:「武王伐紂,即阼於鎬京,而郊其所也。文王未為天子,立郊於酆,見何經典?」復奏曰:「伏見漢書郊祀志,匡衡奏從甘泉河東郊於酆。」權曰:「文王性謙讓,處諸侯之位,明未郊也。經傳無明文,匡衡俗儒意說,非典籍正義,不可用也。」志林曰:吳王糾駮郊祀之奏,追貶匡衡,謂之俗儒。凡在見者,莫不慨然,以為統盡物理,達於事宜。至於稽之典籍,乃更不通。毛氏之說云:「堯見天因邰而生后稷,故國之於邰,命使事天。」故詩曰:「后稷肇祀,庶無罪悔,以迄于今。」言自后稷以來皆得祭天,猶魯人郊祀也。是以棫樸之作,有積燎之薪。文王郊酆,經有明文,匡衡豈俗,而枉之哉?文王雖未為天子,然三分天下而有其二,伐崇戡黎,祖伊奔告。天旣棄殷,乃眷西顧,太伯三讓,以有天下。文王為王,於義何疑?然則匡衡之奏,有所未盡。桉世宗立甘泉、汾陰之祠,皆出方士之言,非據經典者也。方士以甘泉、汾陰黃帝祭天地之處,故孝武因之,遂立二畤。漢治長安,而甘泉在北,謂就乾位,而衡云:「武帝居甘泉,祭於南宮」,此旣誤矣。祭汾陰在水之脽,呼為澤中,而衡云「東之少陽」,失其本意。此自吳事,於傳無非,恨無辯正之辭,故矯之云。脽,音誰,見漢書音義。}
 
 
二年春正月,詔曰:「朕以不德,肇受元命,夙夜兢兢,不遑假寢。思平世難,救濟黎庶,上荅神祇,下慰民望。是以眷眷,勤求俊傑,將與勠力,共定海內,苟在用心,與之偕老。今使持節督幽州領青州牧遼東太守燕王,乆脅賊虜,隔在一方,雖乃心於國,其路靡緣。今因天命,遠遣二使,款誠顯露,章表殷勤,朕之得此,何喜如之!雖湯遇伊尹,周獲呂望,世祖未定而得河右,方之今日,豈復是過?普天一統,於是定矣。書不云乎,『一人有慶,兆民賴之』。其大赦天下,與之更始,其明下州郡,咸使聞知。特下燕國,奉宣詔恩,令普天率土備聞斯慶。」三月,遣舒、綜還,使太常張彌、執金吾許晏、將軍賀達等將兵萬人,金寶珍貨,九錫備物,乘海授淵。
 \gezhu{江表傳載權詔曰:「故魏使持節車騎將軍遼東太守平樂侯:天地失序,皇極不建,元惡大憝,作害于民,海內分崩,羣生堙滅,雖周餘黎民,靡有孑遺,方之今日,亂有甚焉。朕受歷數,君臨萬國,夙夜戰戰,念在弭難,若涉淵水,罔知攸濟。是以把旄杖鉞,翦除凶虐,自東徂西,靡遑寧處,苟力所及,民無災害。雖賊虜遺種,未伏辜誅,猶繫囚枯木,待時而斃。惟將軍天姿特達,兼包文武,觀時覩變,審於去就,踰越險阻,顯致赤心,肇建大計,為天下先,元勳巨績,侔於古人。雖昔竇融背棄隴右,卒占西河,以定光武,休名美實,豈復是過?欽嘉雅尚,朕實欣之。自古聖帝明王,建化垂統,以爵襃德,以祿報功;功大者祿厚,德盛者禮崇。故周公有挾輔之勞,太師有鷹揚之功,並啟土宇,兼受備物。今將軍規萬年之計,建不世之略,絕僭逆之虜,順天人之肅,濟成洪業,功無與比,齊魯之事,奚足言哉!詩不云乎,『無言不讎,無德不報』。今以幽、青二州十七郡七十縣,封君為燕王,使持節守太常張彌授君璽綬策書、金虎符第一至第五、竹使符第一至第十。錫君玄土,苴以白茅,爰契爾龜,用錫冢社。方有戎事,典統兵馬,以大將軍曲蓋麾幢,督幽州、青州牧遼東太守如故。今加君九錫,其敬聽後命。以君三世相承,保綏一方,寧集四郡,訓及異俗,民夷安業,無或攜貳,是用錫君大輅、戎輅、玄牡二駟。君務在勸農,嗇人成功,倉庫盈積,官民俱豐,是用錫君衮冕之服,赤舄副焉。君正化以德,敬下以禮,敦義崇謙,內外咸和,是用錫君軒縣之樂。君宣導休風,懷保邊遠,遠人迴面,莫不影附,是用錫君朱戶以居,君運其才略,官方任賢,顯直措枉,羣善必舉,是用錫君虎賁之士百人。君戎馬整齊,威震遐方,糾虔天刑,彰厥有罪,是用錫君鈇鉞各一。君文和於內,武信於外,禽討逆節,折衝掩難,是用錫君彤弓一、彤矢百、玈弓十、玈矢千。君忠勤有効,溫恭為德,明允篤誠,感于朕心,是用錫君秬鬯一卣,珪瓚副焉。欽哉!敬茲訓典,寅亮天工,相我國家,永終爾休。」}
 舉朝大臣,自丞相雍已下皆諫,以為淵未可信,而寵待太厚,但可遣吏兵數百護送舒、綜,權終不聽。
 \gezhu{臣松之以為權愎諫違衆,信淵意了,非有攻伐之規,重複之慮。宣達錫命,乃用萬人,是何不愛其民,昏虐之甚乎?此役也,非惟闇塞,寔為無道。}
 淵果斬彌等,送其首于魏,沒其兵資。權大怒,欲自征淵,
 \gezhu{江表傳載權怒曰:「朕年六十,世事難易,靡所不嘗,近為鼠子所前却,令人氣踴如山。不自截鼠子頭以擲于海,無顏復臨萬國。就令顛沛,不以為恨。」}
 尚書僕射薛綜等切諫乃止。是歲,權向合肥新城,遣將軍全琮征六安,皆不克還。
 \gezhu{吳書曰:初,張彌、許晏等俱到襄平,官屬從者四百許人。淵欲圖彌、晏,先分其人衆,置遼東諸縣,以中使秦旦、張羣、杜德、黃彊等及吏兵六十人,置玄菟郡。玄菟郡在遼東北,相去二百里,太守王贊領戶二百,兼重可三四百人。旦等皆舍於民家,仰其飲食。積四十許日,旦與彊等議曰:「吾人遠辱國命,自棄於此,與死亡何異?今觀此郡,形勢甚弱。若一旦同心,焚燒城郭,殺其長吏,為國報恥,然後伏死,足以無恨。孰與偷生苟活長為囚虜乎?」彊等然之。於是陰相約結,當用八月十九日夜發。其日中時,為部中張松所告,贊便會士衆閉城門。旦、羣、德、彊等皆踰城得走。時羣病疽創著膝,不及輩旅,德常扶接與俱,崎嶇山谷。行六七百里,創益困,不復能前,卧草中,相守悲泣。羣曰:「吾不幸創甚,死亡無日,卿諸人宜速進道,兾有所達。空相守,俱死於窮谷之中,何益也?」德曰:「萬里流離,死生共之,不忍相委。」於是推旦、彊使前,德獨留守羣,捕菜果食之。旦、彊別數日,得達句驪王宮,因宣詔於句驪王宮及其主簿,詔言有賜為遼東所攻奪。宮等大喜,即受詔,命使人隨旦還迎羣、德。其年,宮遣皁衣二十五人送旦等還,奉表稱臣,貢貂皮千枚,鶡雞皮十具。旦等見權,悲喜不能自勝。權義之,皆拜校尉。間一年,遣使者謝宏、中書陳恂拜宮為單于,加賜衣物珍寶。恂等到安平口,先遣校尉陳奉前見宮,而宮受魏幽州刺史諷旨,令以吳使自效。奉聞之,倒還。宮遣主簿笮咨、帶固等出安平,與宏相見。宏即縛得三十餘人質之,宮於是謝罪,上馬數百匹。宏乃遣咨、固奉詔書賜物歸與宮。是時宏舩小,載馬八十匹而還。}
 
 
 
 
 三年春正月,詔曰:「兵乆不輟,民困於役,歲或不登。其寬諸逋,勿復督課。」夏五月,權遣陸遜、諸葛瑾等屯江夏、沔口,孫韶、張承等向廣陵、淮陽,權率大衆圍合肥新城。是時蜀相諸葛亮出武功,權謂魏明帝不能遠出,而帝遣兵助司馬宣王拒亮,自率水軍東征。未至壽春,權退還,孫韶亦罷。秋八月,以諸葛恪為丹陽太守,討山越。九月朔,隕霜傷穀。冬十一月,太常潘濬平武陵蠻夷,事畢,還武昌。詔復曲阿為雲陽,丹徒為武進。廬陵賊李桓、羅厲等為亂。
 
 
 
 
 四年夏,遣呂岱討桓等。秋七月,有雹。魏使以馬求易珠璣、翡翠、瑇瑁,權曰:「此皆孤所不用,而可得馬,何苦而不聽其交易?」
 
 
 
 
 五年春,鑄大錢,一當五百。詔使吏民輸銅,計銅畀直。設盜鑄之科。三月,武昌言甘露降於禮賔殿。輔吳將軍張昭卒。中郎將吾粲獲李桓,將軍唐咨獲羅厲等。自十月不雨,至於夏。冬十月,彗星見于東方。鄱陽賊彭旦等為亂。
 
 
 
 
 六年春正月,詔曰:「夫三年之喪,天下之達制,人情之極痛也;賢者割哀以從禮,不肖者勉而致之。世治道泰,上下無事,君子不奪人情,故三年不逮孝子之門。至於有事,則殺禮以從宜,要絰而處事。故聖人制法,有禮無時則不行。遭喪不奔非古也,蓋隨時之宜,以義斷恩也。前故設科,長吏在官,當須交代,而故犯之,雖隨糾坐,猶已廢曠。方事之殷,國家多難,凡在官司,宜各盡節,先公後私,而不恭承,甚非謂也。中外羣僚,其更平議,務令得中,詳為節度。」顧譚議,以為「奔喪立科,輕則不足以禁孝子之情,重則本非應死之罪,雖嚴刑益設,違奪必少。若偶有犯者,加其刑則恩所不忍,有減則法廢不行。愚以為長吏在遠,苟不告語,勢不得知。比選代之間,若有傳者,必加大辟,則長吏無廢職之負,孝子無犯重之刑。」將軍胡綜議,以為「喪紀之禮,雖有典制,苟無其時,所不得行。方今戎事軍國異容,而長吏遭喪,知有科禁,公敢干突,苟念聞憂不奔之恥,不計為臣犯禁之罪,此由科防本輕所致。忠節在國,孝道立家,出身為臣,焉得兼之?故為忠臣不得為孝子。宜定科文,示以大辟,若故違犯,有罪無赦。以殺止殺,行之一人,其後必絕。」丞相雍奏從大辟。其後吳令孟宗喪母奔赴,已而自拘於武昌以聽刑。陸遜陳其素行,因為之請,權乃減宗一等,後不得以為比,因此遂絕。二月,陸遜討彭旦等,其年,皆破之。冬十月,遣衞將軍全琮襲六安,不克。諸葛恪平山越事畢,北屯廬江。
 
 
赤烏元年春,鑄當千大錢。夏,呂岱討廬陵賊,畢,還陸口。秋八月,武昌言麒麟見。有司奏言麒麟者太平之應,宜改年號。詔曰:「間者赤烏集於殿前,朕所親見,若神靈以為嘉祥者,改年宜以赤烏為元。」羣臣奏曰:「昔武王伐紂,有赤烏之祥,君臣觀之,遂有天下,聖人書策載述最詳者,以為近事旣嘉,親見又明也。」於是改年。步夫人卒,追贈皇后。初,權信任校事呂壹,壹性苛慘,用法深刻。太子登數諫,權不納,大臣由是莫敢言。後壹姦罪發露伏誅,權引咎責躬,乃使中書郎袁禮告謝諸大將,因問時事所當損益。禮還,復有詔責數諸葛瑾、步隲、朱然、呂岱等曰:「袁禮還,云與子瑜、子山、義封、定公相見,並以時事當有所先後,各自以不掌民事,不肯便有所陳,悉推之伯言、承明。伯言、承明見禮,泣涕懇惻,辭旨辛苦,至乃懷執危怖,有不自安之心。聞此悵然,深自刻怪。何者?夫惟聖人能無過行,明者能自見耳。人之舉厝,何能悉中,獨當己有以傷拒衆意,忽不自覺,故諸君有嫌難耳;不爾,何緣乃至於此乎?自孤興軍五十年,所役賦凡百皆出於民。天下未定,孽類猶存,士民勤苦,誠所貫知。然勞百姓,事不得已耳。與諸君從事,自少至長,髮有二色,以謂表裏足以明露,公私分計,足用相保。盡言直諫,所望諸君;拾遺補闕,孤亦望之。昔衞武公年過志壯,勤求輔弼,每獨歎責。
 \gezhu{江表傳曰:權又云:「天下無粹白之狐,而有粹白之裘,衆之所積也。夫能以駮致純,不惟積乎?故能用衆力,則無敵於天下矣;能用衆智,則無畏於聖人矣。」}
 且布衣韋帶,相與交結,分成好合,尚汚垢不異。今日諸君與孤從事,雖君臣義存,猶謂骨肉不復是過。榮福喜戚,相與共之。忠不匿情,智無遺計,事統是非,諸君豈得從容而已哉!同船濟水,將誰與易?齊桓諸侯之霸者耳,有善管子未嘗不歎,有過未嘗不諫,諫而不得,終諫不止。今孤自省無桓公之德,而諸君諫諍未出於口,仍執嫌難。以此言之,孤於齊桓良優,未知諸君於管子何如耳?乆不相見,因事當笑。共定大業,整齊天下,當復有誰?凡百事要所當損益,樂聞異計,匡所不逮。」
 
 
二年春三月,
 \gezhu{江表傳載權正月詔曰:「郎吏者,宿衞之臣,古之命士也。間者所用頗非其人。自今選三署皆依四科,不得以虛辭相飾。」}
 遣使者羊衜、鄭冑、將軍孫怡之遼東,擊魏守將張持、高慮等,虜得男女。
 \gezhu{文士傳曰:冑字敬先,沛國人。父札,才學博達,權為驃騎將軍,以札為從事中郎,與張昭、孫邵共定朝儀。冑其少子,有文武姿局,少知名,舉賢良,稍遷建安太守。呂壹賔客於郡犯法,冑收付獄,考竟。壹懷恨,後密譖冑。權大怒,召冑還,潘濬、陳表並為請,得釋。後拜宣信校尉,往救公孫淵,已為魏所破,還遷執金吾。子豐,字曼季,有文學操行,與陸雲善,與雲詩相往反。司空張華辟,未就,卒。臣松之聞孫怡者,東州人,非權之宗也。}
 零陵言甘露降。夏五月,城沙羡。冬十月,將軍蔣祕南討夷賊。祕所領都督廖式殺臨賀太守嚴綱等,自稱平南將軍,與弟潛共攻零陵、桂陽,及搖動交州、蒼梧、鬱林諸郡,衆數萬人。遣將軍呂岱、唐咨討之,歲餘皆破。
 
 
 
 
 三年春正月,詔曰:「蓋君非民不立,民非穀不生。頃者以來,民多征役,歲又水旱,年穀有損,而吏或不良,侵奪民時,以致饑困。自今以來,督軍郡守,其謹察非法,當農桑時,以役事擾民者,舉正以聞。」夏四月,大赦,詔諸郡縣治城郭,起譙樓,穿壍發渠,以備盜賊。冬十一月,民饑,詔開倉廩以振貧窮。
 
 
四年春正月,大雪,平地深三尺,鳥獸死者大半。夏四月,遣衞將軍全琮略淮南,決芍陂,燒安城邸閣,收其人民。威北將軍諸葛恪攻六安。琮與魏將王淩戰于芍陂,中郎將秦晃等十餘人戰死。車騎將軍朱然圍樊,大將軍諸葛瑾取柤中。
 \gezhu{漢晉春秋曰:零陵太守殷禮言於權曰:「今天棄曹氏,喪誅累見,虎爭之際而幼童莅事。陛下身自御戎,取亂侮亡,宜滌荊、揚之地,舉彊羸之數,使彊者執戟,羸者轉運,西命益州軍于隴右,授諸葛瑾、朱然大衆指事襄陽,陸遜、朱桓別征壽春,大駕入淮陽,歷青、徐。襄陽、壽春困於受敵,長安以西務對蜀軍,許、洛之衆勢必分離;掎角瓦解,民必內應,將帥對向,或失便宜;一軍敗績,則三軍離心,便當秣馬脂車,陵蹈城邑,乘勝逐北,以定華夏。若不悉軍動衆,循前輕舉,則不足大用,易於屢退。民疲威消,時往力竭,非出兵之策也。」權弗能用之。}
 五月,太子登卒。是月,魏太傅司馬宣王救樊。六月,軍還。閏月,大將軍瑾卒。秋八月,陸遜城邾。
 
 
 
 
 五年春正月,立子和為太子,大赦,改禾興為嘉興。百官奏立皇后及四王,詔曰:「今天下未定,民物勞瘁,且有功者或未錄,饑寒者尚未恤,猥割土壤以豐子弟,崇爵位以寵妃妾,孤甚不取。其釋此議。」三月,海鹽縣言黃龍見。夏四月,禁進獻御,減太官膳。秋七月,遣將軍聶友、校尉陸凱以兵三萬討珠崖、儋耳。是歲大疫,有司又奏立后及諸王。八月,立子霸為魯王。
 
 
 
 
 六年春正月,新都言白虎見。諸葛恪征六安,破魏將謝順營,收其民人。冬十一月,丞相顧雍卒。十二月,扶南王范旃遣使獻樂人及方物。是歲,司馬宣王率軍入舒,諸葛恪自皖遷于柴桑。
 
 
七年春正月,以上大將軍陸遜為丞相。秋,宛陵言嘉禾生。是歲,步隲、朱然等各上疏云:「自蜀還者,咸言背盟與魏交通,多作舟舩,繕治城郭。又蔣琬守漢中,聞司馬懿南向,不出兵乘虛以掎角之,反委漢中,還近成都。事已彰灼,無所復疑,宜為之備。」權揆其不然,曰:「吾待蜀不薄,聘享盟誓,無所負之,何以致此?又司馬懿前來入舒,旬日便退,蜀在萬里,何知緩急而便出兵乎?昔魏欲入漢川,此閒始嚴,亦未舉動,會聞魏還而止,蜀寧可復以此有疑邪?又人家治國,舟船城郭,何得不獲?今此閒治軍,寧復欲以禦蜀邪?人言苦不可信,朕為諸君破家保之。」蜀竟自無謀,如權所籌。
 \gezhu{江表傳載權詔曰:「督將亡叛而殺其妻子,是使妻去夫,子棄父,甚傷義教,自今勿殺也。」}
 
 
八年春二月,丞相陸遜卒。夏,雷霆犯宮門柱,又擊南津大橋楹。茶陵縣鴻水溢出,流漂居民二百餘家。秋七月,將軍馬茂等圖逆,夷三族。
 \gezhu{吳歷曰:茂本淮南鍾離長,而為王淩所失,叛歸吳,吳以為征西將軍、九江太守、外部督,封侯,領千兵。權數出苑中,與公卿諸將射。茂與兼符節令朱貞、無難督虞欽、牙門將朱志等合計,伺權在苑中,公卿諸將在門未入,令貞持節稱詔,悉收縛之;茂引兵入苑擊權,分據宮中及石頭塢,遣人報魏。事覺,皆族之。}
 八月,大赦。遣校尉陳勳將屯田及作士三萬人鑿句容中道,自小其至雲陽西城,通會巿,作邸閣。
 
 
九年春二月,車騎將軍朱然征魏柤中,斬獲千餘。夏四月,武昌言甘露降。秋九月,以驃騎步隲為丞相,車騎朱然為左大司馬,衞將軍全琮為右大司馬,鎮南呂岱為上大將軍,威北將軍諸葛恪為大將軍。
 \gezhu{江表傳曰:是歲,權詔曰:「謝宏往日陳鑄大錢,云以廣貨,故聽之。今聞民意不以為便,其省息之,鑄為器物,官勿復出也。私家有者,勑以輸藏,計畀其直,勿有所枉也。」}
 
 
十年春正月,右大司馬全琮卒。
 \gezhu{江表傳曰:是歲權遣諸葛壹偽叛以誘諸葛誕,誕以步騎二萬迎壹於高山。權出涂中,遂至高山,潛軍以待之。誕覺而退。}
 二月,權適南宮。三月,改作太初宮,諸將及州郡皆義作。
 \gezhu{江表傳載權詔曰:「建業宮乃朕從京來所作將軍府寺耳,材柱率細,皆以腐朽,常恐損壞。今未復西,可徙武昌宮材瓦,更繕治之。」有司奏言曰:「武昌宮已二十八歲,恐不堪用,宜下所在通更伐致。」權曰:「大禹以卑宮為美,今軍事未已,所在多賦,若更通伐,妨損農桑。徙武昌材瓦,自可用也。」}
 夏五月,丞相步隲卒。冬十月,赦死罪。
 
 
十一年春正月,朱然城江陵。二月,地仍震。
 \gezhu{江表傳載權詔曰:「朕以寡德,過奉先祀,莅事不聦,獲譴靈祇,夙夜祗戒,若不終日。羣僚其各厲精,思朕過失,勿有所諱。」}
 三月,宮成。夏四月,雨雹,雲陽言黃龍見。五月,鄱陽言白虎仁。
 \gezhu{瑞應圖曰:白虎仁者,王者不暴虐,則仁虎不害也。}
 詔曰:「古者聖王積行累善,脩身行道,以有天下,故符瑞應之,所以表德也。朕以不明,何以臻茲?書云『雖休勿休』,公卿百司,其勉脩所職,以匡不逮。」
 
 
十二年春三月,左大司馬朱然卒。四月,有兩烏銜鵲墮東館。丙寅,驃騎將軍朱據領丞相,燎鵲以祭。
 \gezhu{吳錄曰:六月戊戌,寶鼎出臨平湖。八月癸丑,白鳩見於章安。}
 
 
十三年夏五月,日至,熒惑入南斗,秋七月,犯魁第二星而東。八月,丹楊、句容及故鄣、寧國諸山崩,鴻水溢。詔原逋責,給貸種食。廢太子和,處故鄣。魯王霸賜死。冬十月,魏將文欽偽叛以誘朱異,權遣呂據就異以迎欽。異等持重,欽不敢進。十一月,立子亮為太子。遣軍十萬,作堂邑涂塘以淹北道。十二月,魏大將軍王昶圍南郡,荊州刺史王基攻西陵,遣將軍戴烈、陸凱往拒之,皆引還。
 \gezhu{庾闡揚都賦注曰:烽火以炬置孤山頭,皆緣江相望,或百里,或五十、三十里,寇至則舉以相告,一夕可行萬里。孫權時合暮舉火於西陵,鼓三竟,達吳郡南沙。}
 是歲,神人授書,告以改年、立后。
 
 
太元元年夏五月,立皇后潘氏,大赦,改年。初臨海羅陽縣有神,自稱王表。
 \gezhu{吳錄曰:羅陽今安固縣。}
 周旋民閒,語言飲食與人無異,然不見其形。又有一婢,名紡績。是月,遣中書郎李崇齎輔國將軍羅陽王印綬迎表。表隨崇俱出,與崇及所在郡守令長談論,崇等無以易。所歷山川,輒遣婢與其神相聞。秋七月,崇與表至,權於蒼龍門外為立第舍,數使近臣齎酒食往。表說水旱小事,往往有驗。
 \gezhu{孫盛曰:盛聞國將興,聽於民;國將亡,聽於神。權年老志衰,讒臣在側,廢適立庶,以妾為妻,可謂多涼德矣。而偽設符命,求福妖邪,將亡之兆,不亦顯乎!}
 秋八月朔,大風,江海涌溢,平地深八尺,吳高陵松柏斯拔,郡城南門飛落。冬十一月,大赦。權祭南郊還,寢疾。
 \gezhu{吳錄曰:權得風疾。}
 十二月,驛徵大將軍恪,拜為太子太傅。詔省徭役,減征賦,除民所患苦。
 
 
二年春正月,立故太子和為南陽王,居長沙;子奮為齊王,居武昌;子休為琅邪王,居虎林。二月,大赦,改元為神鳳。皇后潘氏薨。諸將吏數詣王表請福,表亡去。夏四月,權薨,時年七十一,謚曰大皇帝。秋七月,葬蔣陵。
 \gezhu{傅子曰:孫策為人明果獨斷,勇蓋天下,以父堅戰死,少而合其兵將以報讎,轉鬬千里,盡有江南之地,誅其名豪,威行鄰國。及權繼其業,有張子布以為腹心,有陸議、諸葛瑾、步隲以為股肱,有呂範、朱然以為爪牙,分任受職,乘閒伺隙,兵不妄動,故戰少敗而江南安。}
 
 
評曰:孫權屈身忍辱,任才尚計,有句踐之奇英,人之傑矣。故能自擅江表,成鼎峙之業。然性多嫌忌,果於殺戮,曁臻末年,彌以滋甚。至于讒說殄行,胤嗣廢斃,
 \gezhu{馬融注尚書曰:殄,絕也,絕君子之行。}
 豈所謂貽厥孫謀以燕翼子者哉?其後葉陵遲,遂致覆國,未必不由此也。
 \gezhu{臣松之以為孫權橫廢無罪之子,雖為兆亂,然國之傾覆,自由暴皓。若權不廢和,皓為世適,終至滅亡,有何異哉?此則喪國由於昏虐,不在於廢黜也。設使亮保國祚,休不早死,則皓不得立。皓不得立,則吳不亡矣。}
 
 
\end{pinyinscope}