\article{龐悳傳}
\begin{pinyinscope}
 
 
 龐悳字令明,南安狟道人也。
 
 
\gezhu{狟音桓。}
 少為郡吏州從事。初平中,從馬騰擊反羌叛氐。數有功,稍遷至校尉。建安中,太祖討袁譚、尚於黎陽,譚遣郭援、高幹等略取河東,太祖使鍾繇率關中諸將討之。悳隨騰子超拒援、幹於平陽,悳為軍鋒,進攻援、幹,大破之,親斬援首。
 \gezhu{魏略曰:悳手斬一級,不知是援。戰罷之後,衆人皆言援死而不得其首。援,鍾繇之甥。悳晚後於鞬中出一頭,繇見之而哭。悳謝繇,繇曰:「援雖我甥,乃國賊也。卿何謝之?」}
 拜中郎將,封都亭侯。後張白騎叛於弘農,悳復隨騰征之,破白騎於兩殽閒。每戰,常陷陣却敵,勇冠騰軍。後騰徵為衞尉,悳留屬超。太祖破超於渭南,悳隨超亡入漢陽,保兾城。後復隨超奔漢中,從張魯。太祖定漢中,悳隨衆降。太祖素聞其驍勇,拜立義將軍,封關門亭侯,邑三百戶。
 
 
侯音、衞開等以宛叛,悳將所領與曹仁共攻拔宛,斬音、開,遂南屯樊,討關羽。樊下諸將以悳兄在漢中,頗疑之。
 \gezhu{魏略曰:悳從兄名柔,時在蜀。}
 悳常曰:「我受國恩,義在效死。我欲身自擊羽。今年我不殺羽,羽當殺我。」後親與羽交戰,射羽中額。時悳常乘白馬,羽軍謂之白馬將軍,皆憚之。仁使悳屯樊北十里,會天霖雨十餘日,漢水暴溢,樊下平地五六丈,悳與諸將避水上隄。羽乘船攻之,以大船四面射隄上。悳被甲持弓,箭不虛發。將軍董衡、部曲將董超等欲降,悳皆收斬之。自平旦力戰至日過中,羽攻益急,矢盡,短兵接戰。悳謂督將成何曰:「吾聞良將不怯死以苟免,烈士不毀節以求生,今日,我死日也。」戰益怒,氣愈壯,而水浸盛,吏士皆降。悳與麾下將一人,伍伯二人,彎弓傅矢,乘小船欲還仁營。水盛船覆,失弓矢,獨抱船覆水中,為羽所得,立而不跪。羽謂曰:「卿兄在漢中,我欲以卿為將,不早降何為?」悳罵羽曰:「豎子,何謂降也!魏王帶甲百萬,威振天下。汝劉備庸才耳,豈能敵邪!我寧為國家鬼,不為賊將也。」遂為羽所殺。太祖聞而悲之,為之流涕,封其二子為列侯。文帝即王位,乃遣使就悳墓賜謚,策曰:「昔先軫喪元,王蠋絕脰,隕身徇節,前代美之。惟侯戎昭果毅,蹈難成名,聲溢當時,義高在昔,寡人愍焉,謚曰壯侯。」又賜子會等四人爵關內侯,邑各百戶。會勇烈有父風,官至中尉將軍,封列侯。
 \gezhu{王隱蜀記曰:鍾會平蜀,前後鼓吹,迎悳屍喪還葬鄴,冢中身首如生。臣松之案悳死於樊城,文帝即位,又遣使至悳墓所,則其屍喪不應在蜀。此王隱之虛說也。}
 
 
\end{pinyinscope}