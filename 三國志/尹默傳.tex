\article{尹默傳}
\begin{pinyinscope}
 
 
 尹默字思潛,梓潼涪人也。益部多貴今文而不崇章句,默知其不博,乃遠游荊州,從司馬德操、宋仲子等受古學。皆通諸經史,又專精於左氏春秋,自劉歆條例,鄭衆、賈逵父子、陳元方、服虔注說,咸略誦述,不復桉本。先主定益州,領牧,以為勸學從事,及立太子,以默為僕射,以左氏傳授後主。後主踐阼,拜諫議大夫。丞相亮住漢中,請為軍祭酒。亮卒,還成都,拜太中大夫,卒。子宗傳其業,為博士。
 
 
\gezhu{宋仲子後在魏。魏略曰:其子與魏諷謀反,伏誅。魏太子荅王朗書曰:「昔石厚與州吁游,父碏知其與亂;韓子昵田蘇,穆子知其好仁:故君子游必有方,居必就士,誠有以也。嗟乎!宋忠無石子先識之明,老罹此禍。今雖欲願行滅親之誅,立純臣之節,尚可得邪!」}
 
 
\end{pinyinscope}