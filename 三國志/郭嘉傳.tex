\article{郭嘉傳}
\begin{pinyinscope}
 
 
 郭嘉字奉孝,潁川陽翟人也。
 
 
\gezhu{傅子曰:嘉少有遠量。漢末天下將亂。自弱冠匿名迹,密交結英儁,不與俗接,故時人多莫知,惟識達者奇之。年二十七,辟司徒府。}
 初,北見袁紹,謂紹謀臣辛評、郭圖曰:「夫智者審於量主,故百舉百全而功名可立也。袁公徒欲效周公之下士,而未知用人之機。多端寡要,好謀無決,欲與共濟天下大難,定霸王之業,難矣!」於是遂去之。先是時,潁川戲志才,籌畫士也,太祖甚器之。早卒。太祖與荀彧書曰:「自志才亡後,莫可與計事者。汝、潁固多奇士,誰可以繼之?」彧薦嘉。召見,論天下事。太祖曰:「使孤成大業者,必此人也。」嘉出,亦喜曰:「真吾主也。」表為司空軍祭酒。
 \gezhu{傅子曰:太祖謂嘉曰:「本初擁冀州之衆,青、并從之,地廣兵彊,而數為不遜。吾欲討之,力不敵,如何?」對曰:「劉、項之不敵,公所知也。漢祖唯智勝;項羽雖彊,終為所禽。嘉竊料之,紹有十敗,公有十勝,雖兵彊,無能為也。紹繁禮多儀,公體任自然,此道勝一也。紹以逆動,公奉順以率天下,此義勝二也。漢末政失於寬,紹以寬濟寬,故不攝,公糾之以猛而上下知制,此治勝三也。紹外寬內忌,用人而疑之,所任唯親戚子弟,公外易簡而內機明,用人無疑,唯才所宜,不間遠近,此度勝四也。紹多謀少決,失在後事,公策得輒行,應變無窮,此謀勝五也。紹因累世之資,高議揖讓以收名譽,士之好言飾外者多歸之,公以至心待人,推誠而行,不為虛美,以儉率下,與有功者無所吝,士之忠正遠見而有實者皆願為用,此德勝六也。紹見人飢寒,恤念之形於顏色,其所不見,慮或不及也,所謂婦人之仁耳,公於目前小事,時有所忽,至於大事,與四海接,恩之所加,皆過其望,雖所不見,慮之所周,無不濟也,此仁勝七也。紹大臣爭權,讒言惑亂,公御下以道,浸潤不行,此明勝八也。紹是非不可知,公所是進之以禮,所不是正之以法,此文勝九也。紹好為虛勢,不知兵要,公以少克衆,用兵如神,軍人恃之,敵人畏之,此武勝十也。」太祖笑曰:「如卿所言,孤何德以堪之也!」嘉又曰:「紹方北擊公孫瓚,可因其遠征,東取呂布。不先取布,若紹為寇,布為之援,此深害也。」太祖曰:「然。」}
 
 
征呂布,三戰破之,布退固守。時士卒疲倦,太祖欲引軍還,嘉說太祖急攻之,遂禽布。語在荀攸傳。
 \gezhu{傅子曰:太祖欲引軍還,嘉曰:「昔項籍七十餘戰,未嘗敗北,一朝失勢而身死國亡者,恃勇無謀故也。今布每戰輒破,氣衰力盡,內外失守。布之威力不及項籍,而困敗過之,若乘勝攻之,此成禽也。」太祖曰:「善。」魏書曰:劉備來奔,以為豫州牧。或謂太祖曰:「備有英雄志,今不早圖,後必為患。」太祖以問嘉,嘉曰:「有是。然公提劒起義兵,為百姓除暴,推誠杖信以招俊桀,猶懼其未也。今備有英雄名,以窮歸己而害之,是以害賢為名,則智士將自疑,回心擇主,公誰與定天下?夫除一人之患,以沮四海之望,安危之機,不可不察!」太祖笑曰:「君得之矣。」傅子曰:初,劉備來降,太祖以客禮待之,使為豫州牧。嘉言於太祖曰:「備有雄才而甚得衆心。張飛、關羽者,皆萬人之敵也,為之死用。嘉觀之,備終不為人下,其謀未可測也。古人有言:『一日縱敵,數世之患。』宜早為之所。」是時,太祖奉天子以號令天下,方招懷英雄以明大信,未得從嘉謀。會太祖使備要擊袁術,嘉與程昱俱駕而諫太祖曰:「放備,變作矣!」時備已去,遂舉兵以叛。太祖恨不用嘉之言。案魏書所云,與傅子正反也。}
 
 
孫策轉鬬千里,盡有江東,聞太祖與袁紹相持於官渡,將渡江北襲許。衆聞皆懼,嘉料之曰:「策新并江東,所誅皆英豪雄桀,能得人死力者也。然策輕而無備,雖有百萬之衆,無異於獨行中原也。若刺客伏起,一人之敵耳。以吾觀之,必死於匹夫之手。」策臨江未濟,果為許貢客所殺。
 \gezhu{傅子曰:太祖欲速征劉備,議者懼軍出,袁紹擊其後,進不得戰而退失所據。語在武紀。太祖疑,以問嘉。嘉勸太祖曰:「紹性遲而多疑,來必不速。備新起,衆心未附,急擊之必敗。此存亡之機,不可失也。」太祖曰:「善。」遂東征備。備敗奔紹,紹果不出。臣松之案武紀,決計征備,量紹不出,皆出自太祖。此云用嘉計,則為不同。又本傳稱嘉料孫策輕佻,必死於匹夫之手,誠為明於見事。然自非上智,無以知其死在何年也。今正以襲許年死,此蓋事之偶合。}
 
 
從破袁紹,紹死,又從討譚、尚於黎陽,連戰數克。諸將欲乘勝遂攻之,嘉曰:「袁紹愛此二子,莫適立也。有郭圖、逢紀為之謀臣,必交鬬其閒,還相離也。急之則相持,緩之而後爭心生。不如南向荊州若征劉表者,以待其一變成而後擊之,可一舉定也。」太祖曰:「善。」乃南征。軍至西平,譚、尚果爭兾州。譚為尚軍所敗,走保平原,遣辛毗乞降。太祖還救之,遂從定鄴。又從攻譚於南皮,兾州平。封嘉洧陽亭侯。
 \gezhu{傅子曰:河北旣平,太祖多辟召青、兾、幽、并知名之士,漸臣事之,以為省事掾屬。皆嘉之謀也。}
 
 
 
 
 太祖將征袁尚及三郡烏丸,諸下多懼劉表使劉備襲許以討太祖,嘉曰:「公雖威震天下,胡恃其遠,必不設備。因其無備,卒然擊之,可破滅也。且袁紹有恩於民夷,而尚兄弟生存。今四州之民,徒以威附,德施未加,舍而南征,尚因烏丸之資,招其死主之臣,胡人一動,民夷俱應,以生蹋頓之心,成覬覦之計,恐青、兾非己之有也。表,坐談客耳,自知才不足以御備,重任之則恐不能制,輕任之則備不為用,雖虛國遠征,公無憂矣。」太祖遂行。至易,嘉言曰:「兵貴神速。今千里襲人,輜重多,難以趨利,且彼聞之,必為備;不如留輜重,輕兵兼道以出,掩其不意。」太祖乃密出盧龍塞,直指單于庭。虜卒聞太祖至,惶怖合戰。大破之,斬蹋頓及名王已下。尚及兄熈走遼東。
 
 
嘉深通有筭略,達於事情。太祖曰:「唯奉孝為能知孤意。」年三十八,自柳城還,疾篤,太祖問疾者交錯。及薨,臨其喪,哀甚,謂荀攸等曰:「諸君年皆孤輩也,唯奉孝最少。天下事竟,欲以後事屬之,而中年夭折,命也夫!」乃表曰:「軍祭酒郭嘉,自從征伐,十有一年。每有大議,臨敵制變。臣策未決,嘉輙成之。平定天下,謀功為高。不幸短命,事業未終。追思嘉勳,實不可忘。可增邑八百戶,并前千戶。」
 \gezhu{魏書載太祖表曰:「臣聞襃忠寵賢,未必當身,念功惟績,恩隆後嗣。是以楚宗孫叔,顯封厥子;岑彭旣沒,爵及支庶。故軍祭酒郭嘉,忠良淵淑,體通性達。每有大議,發言盈庭,執中處理,動無遺策。自在軍旅,十有餘年,行同騎乘,坐共幄席,東禽呂布,西取眭固,斬袁譚之首,平朔土之衆,踰越險塞,盪定烏丸,震威遼東,以梟袁尚。雖假天威,易為指麾,至於臨敵,發揚誓命,凶逆克殄,勳實由嘉。方將表顯,短命早終。上為朝廷悼惜良臣,下自毒恨喪失奇佐。宜追增嘉封,并前千戶,襃亡為存,厚往勸來也。」}
 謚曰貞侯。子弈嗣。
 \gezhu{魏書稱弈通達見理。弈字伯益,見王昶家誡。}
 
 
後太祖征荊州還,於巴丘遇疾疫,燒船,歎曰:「郭奉孝在,不使孤至此。」
 \gezhu{傅子曰:太祖又云:「哀哉奉孝!痛哉奉孝!惜哉奉孝!」}
 初,陳羣非嘉不治行檢,數廷訴嘉,嘉意自若。太祖愈益重之,然以羣能持正,亦恱焉。
 \gezhu{傅子曰:太祖與荀彧書,追傷嘉曰:「郭奉孝年不滿四十,相與周旋十一年,阻險艱難,皆共罹之。又以其通達,見世事無所凝滯,欲以後事屬之,何意忽爾失之,悲痛傷心。今表增其子滿千戶,然何益亡者,追念之感深。且奉孝乃知孤者也;天下人相知者少,又以此痛惜。柰何柰何!」又與彧書曰:「追惜奉孝,不能去心。其人見時事兵事,過絕於人。又人多畏病,南方有疫,常言『吾往南方,則不生還』。然與共論計,云當先定荊。此為不但見計之忠厚,必欲立功分,棄命定。事人心乃爾,何得使人忘之!」}
 弈為太子文學,早薨。子深嗣。深薨,子獵嗣。
 \gezhu{世語曰:嘉孫敞,字泰中,有才識,位散騎常侍。}
 
 
\end{pinyinscope}