\article{陳矯傳}
\begin{pinyinscope}
 
 
 陳矯字季弼,廣陵東陽人也。避亂江東及東城,辭孫策、袁術之命,還本郡。太守陳登請為功曹,使矯詣許,謂曰:「許下論議,待吾不足;足下相為觀察,還以見誨。」矯還曰:「聞遠近之論,頗謂明府驕而自矜。」登曰:「夫閨門雍穆,有德有行,吾敬陳元方兄弟;淵清玉絜,有禮有法,吾敬華子魚;清脩疾惡,有識有義,吾敬趙元達;博聞彊記,奇逸卓犖,吾敬孔文舉;雄姿傑出,有王霸之略,吾敬劉玄德:所敬如此,何驕之有!餘子瑣瑣,亦焉足錄哉?」登雅意如此,而深敬友矯。
 
 
 
 
 郡為孫權所圍於匡奇,登令矯求救於太祖。矯說太祖曰:「鄙郡雖小,形便之國也,若蒙救援,使為外藩,則吳人剉謀,徐方永安,武聲遠震,仁愛滂流,未從之國,望風景附,崇德養威,此王業也。」太祖奇矯,欲留之。矯辭曰:「本國倒縣,本奔走告急,縱無申胥之效,敢忘弘演之義乎?」
 
 
\gezhu{劉向新序曰:齊桓公求婚於衞,衞不與,而嫁於許。衞為狄所伐,桓公不救,至於國滅君死。懿公屍為狄人所食,惟有肝在。懿公有臣曰弘演,適使反,致命於肝曰:「君為其內,臣為其外。」乃刳腹內肝而死。齊桓公曰:「衞有臣若此而尚滅,寡人無有,亡無日矣!」乃救衞,定其君。}
 太祖乃遣赴救。吳軍旣退,登多設閒伏,勒兵追奔,大破之。
 
 
太祖辟矯為司空掾屬,除相令,征南長史,彭城、樂陵太守,魏郡西部都尉。曲周民父病,以牛禱,縣結正棄市。矯曰:「此孝子也。」表赦之。遷魏郡太守。時繫囚千數,至有歷年,矯以為周有三典之制,漢約三章之法,今惜輕重之理,而忽乆繫之患,可謂謬矣。悉自覽罪狀,一時論決。大軍東征,入為丞相長史。軍還,復為魏郡,轉西曹屬。從征漢中,還為尚書。行前未到鄴,太祖崩洛陽,羣臣拘常,以為太子即位,當須詔命。矯曰:「王薨于外,天下惶懼。太子宜割哀即位,以繫遠近之望。且又愛子在側,彼此生變,則社稷危矣。」即具官備禮,一日皆辦。明旦,以王后令,策太子即位,大赦蕩然。文帝曰:「陳季弼臨大節,明略過人,信一時之俊傑也。」帝旣踐阼,轉署吏部,封高陵亭侯,遷尚書令。明帝即位,進爵東鄉侯,邑六百戶。車駕嘗卒至尚書門,矯跪問帝曰:「陛下欲何之?」帝曰:「欲案行文書耳。」矯曰:「此自臣職分,非陛下所宜臨也。若臣不稱其職,則請就黜退。陛下宜還。」帝慙,回車而反。其亮直如此。
 \gezhu{世語曰:劉曄以先進見幸,因譖矯專權。矯懼,以問長子本,本不知所出。次子騫曰:「主上明聖,大人大臣,今若不合,不過不作公耳。」後數日,帝見矯,矯又問二子,騫曰:「陛下意解,故見大人也。」旣入,盡日,帝曰:「劉曄構君,朕有以迹君;朕心故已了。」以金五鉼授之,矯辭。帝曰:「豈以為小惠?君已知朕心,顧君妻子未知故也。」帝憂社稷,問矯:「司馬公忠正,可謂社稷之臣乎?」矯曰:「朝廷之望;社稷,未知也。」}
 加侍中光祿大夫,遷司徒。景初元年薨,謚曰貞侯。
 \gezhu{魏氏春秋曰:矯本劉氏子,出嗣舅氏而婚于本族。徐宣每非之,庭議其闕。太祖惜矯才量,欲擁全之,乃下令曰:「喪亂已來,風教彫薄,謗議之言,難用襃貶。自建安五年已前,一切勿論。其以斷前誹議者,以其罪罪之。」}
 
 
子本嗣,歷位郡守、九卿。所在操綱領,舉大體,能使羣下自盡。有統御之才,不親小事,不讀法律而得廷尉之稱,優於司馬岐等,精練文理。遷鎮北將軍,假節都督河北諸軍事。薨,子粲嗣。本弟騫,咸熈中為車騎將軍。
 \gezhu{案晉書曰:騫字休淵,為晉佐命功臣,至太傅,封高平郡公。}
 
 
初,矯為郡功曹,使過泰山。泰山太守東郡薛悌異之,結為親友。戲謂矯曰:「以郡吏而交二千石,鄰國君屈從陪臣游,不亦可乎!」悌後為魏郡及尚書令,皆承代矯云。
 \gezhu{世語曰:悌字孝威。年二十二,以兖州從事為泰山太守。初,太祖定兾州,以悌及東平王國為左右長史,後至中領軍,並悉忠貞練事,為世吏表。}
 
 
\end{pinyinscope}