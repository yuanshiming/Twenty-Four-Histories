\article{鍾離牧傳}
\begin{pinyinscope}
 
 
 鍾離牧字子幹,會稽山陰人,漢魯相意七世孫也。
 
 
\gezhu{會稽典錄曰:牧父緒,樓船都尉,兄駰,上計吏,少與同郡謝贊、吳郡顧譚齊名。牧童齔時號為遲訥,駰常謂人曰:「牧必勝我,不可輕也。」時人皆以為不然。}
 少爰居永興,躬自墾田,種稻二十餘畝。臨熟,縣民有識認之,牧曰:「本以田荒,故墾之耳。」遂以稻與縣人。縣長聞之,召民繫獄,欲繩以法,牧為之請。長曰:「君慕承宮,自行義事,
 \gezhu{續漢書曰:宮字少子,琅邪人,甞在蒙陰山中耕種禾黍,臨熟,人就認之,宮便推與而去,由是發名,位至左中郎將、侍中。}
 僕為民主,當以法率下,何得寢公憲而從君邪?」牧曰:「此是郡界,緣君意顧,故來蹔住。今以少稻而殺此民,何心復留?」遂出裝,還山陰,長自往止之,為釋繫民。民慙懼,率妻子舂所取稻得六十斛米,送還牧,牧閉門不受。民輸置道旁,莫有取者。牧由此發名。
 \gezhu{徐衆評曰:牧蹈長者之規。問者曰:「如牧所行,犯而不校,又從而救之,直而不有,又還而不受,可不謂之仁讓乎哉?」荅曰:「異乎吾所聞。原憲之問於孔子曰:『克伐怨欲不行焉,可以為仁乎?』孔子曰:『可以為難矣,仁則吾不知也。』『惡不仁者,其為仁矣。』今小民不展四體,而認人之稻,不仁甚矣,而牧推而與之,又救其罪,斯為讓非其義,所救非人,非所謂惡不仁者。苟不惡不仁,安得為仁哉!蒼梧澆娶妻而美,讓於其兄;尾生篤信,水至不去而死;直躬好直,證父攘羊;申鳴奉法,盡忠於君而執其父。忠信直讓,此四行者,聖賢之所貴也。然不貴蒼梧之讓,非讓道也;不取尾生之信,非信所也;不許直躬之直,非直體也;不嘉申鳴之忠,非忠意也。今牧犯而不校,還而不取,可以為難矣,未得為仁讓也。夫聖人以德報德,以直報怨,而牧欲以德報怨,非也。必不得已,二者何從?吾從孔子也。」}
 
 
赤烏五年,從郎中補太子輔義都尉,遷南海太守。
 \gezhu{會稽典錄曰:高涼賊率仍弩等破略百姓,殘害吏民,牧越界撲討,旬日降服。又揭陽縣賊率曾夏等衆數千人,歷十餘年,以侯爵雜繒千匹,下書購募,絕不可得。牧遣使慰譬,登皆首服,自改為良民。始興太守羊衜與太常滕胤書曰:「鍾離子幹吾昔知之不熟,定見其在南海,威恩部伍,智勇分明,加操行清純,有古人之風。」其見貴如此。在郡四年,以疾去職。}
 還為丞相長史,轉司直,遷中書令。會建安、鄱陽、新都三郡山民作亂,出牧為監軍使者,討平之。賊帥黃亂、常俱等出其部伍,以充兵役。封秦亭侯,拜越騎校尉。
 
 
永安六年,蜀并于魏,武陵五谿夷與蜀接界,時論懼其叛亂,乃以牧為平魏將軍,領武陵太守,往之郡。魏遣漢葭縣長郭純試守武陵太守,率涪陵民入蜀遷陵界,屯于赤沙,誘致諸夷邑君,或起應純,又進攻酉陽縣,郡中震懼。牧問朝吏曰:「西蜀傾覆,邊境見侵,何以禦之?」皆對曰:「今二縣山險,諸夷阻兵,不可以軍驚擾,驚擾則諸夷盤結。宜以漸安,可遣恩信吏宣教慰勞。」牧曰:「不然。外境內侵,誑誘人民,當及其根柢未深而撲取之,此救火貴速之勢也。」勑外趣嚴,掾史沮議者便行軍法。撫夷將軍高尚說牧曰:「昔潘太常督兵五萬,然後以討五谿夷耳。又是時劉氏連和,諸夷率化,今旣無往日之援,而郭純已據遷陵,而明府以三千兵深入,尚未見其利也。」牧曰:「非常之事,何得循舊?」即率所領,晨夜進道,緣山險行,垂二千里,從塞上,斬惡民懷異心者魁帥百餘人及其支黨凡千餘級,純等散,五谿平。遷公安督、揚武將軍,封都鄉侯,徙濡須督。
 \gezhu{會稽典錄曰:牧之在濡須,深以進取可圖,而不敢陳其策,與侍中東觀令朱育宴,慨然歎息。育謂牧恨於策爵未副,因謂牧曰:「朝廷諸君,以際會坐取高官,亭侯功無與比,不肯在人下,見顧者猶以於邑,況於侯也!」牧笑而荅曰:「卿之所言,未獲我心也。馬援有言,人當功多而賞薄。吾功不足錄,而見寵已過當,豈以為恨?國家不深相知,而見害朝人,是以默默不敢有所陳。若其不然,當建進取之計,以報所受之恩,不徒自守而已,憤歎以此也。」育復曰:「國家已自知侯,以侯之才,無為不成。愚謂自可陳所懷。」牧曰:「武安君謂秦王云:『非成業難,得賢難;非得賢難,用之難;非用之難,任之難。』武安君欲為秦王并兼六國,恐授事而不見任,故先陳此言。秦王旣許而不能,卒隕將成之業,賜劒杜郵。今國家知吾,不如秦王之知武安,而害吾者有過范睢。大皇帝時,陸丞相討鄱陽,以二千人授吾,潘太常討武陵,吾又有三千人,而朝廷下議,棄吾於彼,使江渚諸督,不復發兵相繼。蒙國威靈自濟,今日何為常。向使吾不料時度宜,苟有所陳,至見委以事,不足兵勢,終有敗績之患,何無不成之有?」}
 復以前將軍假節,領武陵太守。卒官。家無餘財,士民思之。子禕嗣,代領兵。
 \gezhu{會稽典錄曰:牧次子盛,亦履恭讓,為尚書郎。弟徇領兵為將,拜偏將軍,戍西陵,與監軍使者唐盛論地形勢,謂宜城、信陵為建平援,若不先城,敵將先入。盛以施績、留平,智略名將,屢經於彼,無云當城之者,不然徇計。後半年,晉果遣將脩信陵城。晉軍平吳,徇領水軍督,臨陣戰死。}
 
 
 
 
 評曰:山越好為叛亂,難安易動,是以孫權不遑外禦,卑詞魏氏。凡此諸臣,皆克寧內難,綏靜邦域者也。呂岱清恪在公;周魴譎略多奇;鍾離牧蹈長者之規;全琮有當世之才,貴重於時,然不檢姧子,獲譏毀名云。
 
 
\end{pinyinscope}