\article{孫翊傳}
\begin{pinyinscope}
 
 
 孫翊字叔弼,權弟也,驍悍果烈,有兄策風。太守朱治舉孝廉,司空辟。
 
 
\gezhu{典略曰:翊名儼,性似策。策臨卒,張昭等謂策當以兵屬儼,而策呼權,佩以印綬。}
 建安八年,以偏將軍領丹楊太守,時年二十。後年為左右邊鴻所殺,鴻亦即誅。
 \gezhu{吳歷載翊妻徐節行,宜與媯覽等事相次,故列於後孫韶傳中。}
 
 
子松為射聲校尉、都鄉侯。
 \gezhu{吳錄曰:松善與人交,輕財好施。鎮巴丘,數咨陸遜以得失。嘗有小過,遜面責松,松意色不平,遜觀其少釋,謂曰:「君過聽不以其鄙,數見訪及,是以承來意進盡言,便變色,何也?」松笑曰:「屬亦自忿行事有此,豈有望邪!」}
 黃龍三年卒。蜀丞相諸葛亮與兄瑾書曰:「旣受東朝厚遇,依依於子弟。又子喬良器,為之惻愴。見其所與亮器物,感用流涕。」其悼松如此,由亮養子喬咨述故云。
 
 
\end{pinyinscope}