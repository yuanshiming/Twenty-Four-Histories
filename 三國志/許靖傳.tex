\article{許靖傳}
\begin{pinyinscope}
 
 
 許靖字文休,汝南平輿人。少與從弟劭俱知名,並有人倫臧否之稱,而私情不協。劭為郡功曹,排擯靖不得齒叙,以馬磨自給。潁川劉翊為汝南太守,乃舉靖計吏,察孝廉,除尚書郎,典選舉。靈帝崩,董卓秉政,以漢陽周毖為吏部尚書,與靖共謀議,進退天下之士,沙汰穢濁,顯拔幽滯。進用潁川荀爽、韓融、陳紀等為公、卿、郡守,拜尚書韓馥為兾州牧,侍中劉岱為兖州刺史,潁川張咨為南陽太守,陳留孔伷為豫州刺史,東郡張邈為陳留太守,而遷靖巴郡太守,不就,補御史中丞。馥等到官,各舉兵還向京都,欲以誅卓。卓怒毖曰:「諸君言當拔用善士,卓從君計,不欲違天下人心。而諸君所用人,至官之日,還來相圖。卓何用相負!」叱毖令出,於外斬之。靖從兄陳相瑒,又與伷合規,靖懼誅,奔伷。
 
 
\gezhu{蜀記云:靖後自表曰:「黨賊求生,情所不忍;守官自危,死不成義。竊念古人當難詭常,權以濟其道。」}
 伷卒,依揚州刺史陳禕。禕死,吳郡都尉許貢、會稽太守王朗素與靖有舊,故往保焉。靖收恤親理經紀振贍,出於仁厚。
 
 
孫策東渡江,皆走交州以避其難,靖身坐岸邊,先載附從,踈親悉發,乃從後去,當時見者莫不歎息。旣至交阯,交阯太守士爕厚加敬待。陳國袁徽以寄寓交州,徽與尚書令荀彧書曰:「許文休英才偉士,智略足以計事。自流宕已來,與羣士相隨,每有患急,常先人後己,與九族中外同其饑寒。其紀綱同類,仁恕惻隱,皆有效事,不能復一二陳之耳。」鉅鹿張翔
 \gezhu{萬機論云:翔字元鳳。}
 銜王命使交部,乘勢募靖,欲與誓要,靖拒而不許。靖與曹公書曰:
 
 
世路戎夷,禍亂遂合,駑怯偷生,自竄蠻貊,成闊十年,吉凶禮廢。昔在會稽,得所貽書,辭旨款密,乆要不忘。迫於袁術放命圮族,扇動羣逆,津塗四塞,雖縣心北風,欲行靡由。正禮師退,術兵前進,會稽傾覆,景興失據,三江五湖皆為虜庭。臨時困厄,無所控告。便與袁沛、鄧子孝等浮涉滄海,南至交州。經歷東歐、閩、越之國,行經萬里,不見漢地,漂薄風波,絕糧茹草,饑殍荐臻,死者大半。旣濟南海,與領守兒孝德相見,知足下忠義奮發,整飭元戎,西迎大駕,巡省中嶽。承此休問,且悲且憙,即與袁沛及徐元賢復共嚴裝,欲北上荊州。會蒼梧諸縣夷、越蠭起,州府傾覆,道路阻絕,元賢被害,老弱並殺。靖尋循渚岸五千餘里,復遇疾癘,伯母隕命,并及羣從,自諸妻子,一時略盡。復相扶侍,前到此郡,計為兵害及病亡者,十遺一二。生民之艱,辛苦之甚,豈可具陳哉!
 \gezhu{臣松之以為孔子稱「賢者避世,其次避地」,蓋貴其識見安危,去就得所也。許靖羇客會稽,閭閻之士,孫策之來,於靖何為?而乃汎萬里之海,入疫癘之鄉,致使尊弱塗炭,百罹備經,可謂自貽矣。謀臣若斯,難以言智。孰若安時處順,端拱吳、越,與張昭、張紘之儔同保元吉者哉?}
 懼卒顛仆,永為亡虜,憂悴慘慘,忘寢與食。欲附奉朝貢使,自獲濟通,歸死闕庭,而荊州水陸無津,交部驛使斷絕。欲上益州,復有峻防,故官長吏,一不得入。前令交阯太守士威彥,深相分託於益州兄弟,又靖亦自與書,辛苦懇惻,而復寂寞,未有報應。雖仰瞻光靈,延頸企踵,何由假翼自致哉?
 
 
知聖主允明,顯授足下專征之任,凡諸逆節,多所誅討,想力競者一心,順從者同規矣。又張子雲昔在京師,志匡王室,今雖臨荒域,不得參與本朝,亦國家之藩鎮,足下之外援也。
 \gezhu{子雲名津,南陽人,為交州刺史。見吳志。}
 若荊、楚平和,王澤南至,足下忽有聲命於子雲,勤見保屬,令得假途由荊州出,不然,當復相紹介於益州兄弟,使相納受。儻天假其年,人緩其禍,得歸死國家,解逋逃之負,泯軀九泉,將復何恨!若時有險易,事有利鈍,人命無常,隕沒不達者,則永銜罪責,入於裔土矣。
 
 
昔營丘翼周,杖鉞專征,博陸佐漢,虎賁警蹕。
 \gezhu{漢書霍光傳曰:「光出都肄郎羽林,道上稱警蹕。」未詳虎賁所出也。}
 今日足下扶危持傾,為國柱石,秉師望之任,兼霍光之重。五侯九伯,制御在手,自古及今,人臣之尊未有及足下者也。夫爵高者憂深,祿厚者責重,足下據爵高之任,當責重之地,言出於口,即為賞罰,意之所存,便為禍福。行之得道,即社稷用寧;行之失道,即四方散亂。國家安危,在於足下;百姓之命,縣於執事。自華及夷,顒顒注望。足下任此,豈可不遠覽載籍廢興之由,榮辱之機,棄忘舊惡,寬和羣司,審量五材,為官擇人?苟得其人,雖讎必舉;苟非其人,雖親不授。以寧社稷,以濟下民,事立功成,則繫音於管絃,勒勳於金石,願君勉之!為國自重,為民自愛。
 
 
 
 
 翔恨靖之不自納,搜索靖所寄書疏,盡投之于水。
 
 
後劉璋遂使使招靖,靖來入蜀。璋以靖為巴郡、廣漢太守。南陽宋仲子於荊州與蜀郡太守王商書曰:「文休倜儻瑰瑋,有當世之具,足下當以為指南。」
 \gezhu{益州耆舊傳曰:商字文表,廣漢人,以才學稱,聲問著於州里。劉璋辟為治中從事。是時王塗隔絕,州之牧伯猶七國之諸侯也,而璋懦弱多疑,不能黨信大臣。商奏記諫璋,璋頗感悟。初,韓遂與馬騰作亂關中,數與璋父焉交通信,至騰子超復與璋相聞,有連蜀之意。商謂璋曰:「超勇而不仁,見得不思義,不可以為脣齒。老子曰:『國之利器,不可以示人。』今之益部,土美民豐,寶物所出,斯乃狡夫所欲傾覆,超等所以西望也。若引而近之,則由養虎,將自遺患矣。」璋從其言,乃拒絕之。荊州牧劉表及儒者宋忠咸聞其名,遺書與商切致殷勤。許靖號為臧否,至蜀,見商而稱之曰:「設使商生於華夏,雖王景興無以加也。」璋以商為蜀郡太守。成都禽堅有至孝之行,商表其墓,追贈孝廉。又與嚴君平、李弘立祠作銘,以旌先賢。脩學廣農,百姓便之。在郡十載,卒於官,許靖代之。}
 建安十六年,轉在蜀郡。
 \gezhu{山陽公載記曰:建安十七年,漢立皇子熈為濟陰王,懿為山陽王,敦為東海王。靖聞之曰:「『將欲歙之,必固張之;將欲取之,必固與之』。其孟德之謂乎!」}
 十九年,先主克蜀,以靖為左將軍長史。先主為漢中王,靖為太傅。及即尊號,策靖曰:「朕獲奉洪業,君臨萬國,夙宵惶惶,懼不能綏。百姓不親,五品不遜,汝作司徒,其敬敷五教,五教在寬。君其勗哉!秉德無怠,稱朕意焉。」
 
 
靖雖年逾七十,愛樂人物,誘納後進,清談不倦。丞相諸葛亮皆為之拜。章武二年卒。子欽,先靖夭沒。欽子游,景耀中為尚書。始靖兄事潁川陳紀,與陳郡袁渙、平原華歆、東海王朗等親善,歆、朗及紀并子羣,魏初為公輔大臣,咸與靖書,申陳舊好,情義欵至,文多故不載。
 \gezhu{魏略:王朗與文休書曰:「文休足下:消息平安,甚善甚善。豈意脫別三十餘年而無相見之緣乎!詩人比一日之別於歲月,豈況悠悠歷累紀之年者哉!自與子別,若沒而復浮,若絕而復連者數矣。而今而後,居升平之京師,攀附於飛龍之聖主;儕輩略盡,幸得老與足下並為遺種之叟,而相去數千里,加有邅蹇之隔,時聞消息於風聲,託舊情於思想,眇眇異處,與異世無以異也。往者隨軍到荊州,見鄧子孝、桓元將,粗聞足下動靜,云夫子旣在益州,執職領郡,德素規矩,老而不墯。是時侍宿武皇帝於江陵劉景升聽事之上,共道足下於通夜,拳拳飢渴,誠無已也。自天子在東宮,及即位之後,每會羣賢,論天下髦儁之見在者,豈獨人盡為英士鮮易取最,故乃猥以原壤之朽質,感夫子之情聽;每叙足下,以為謀首,豈其注意,乃復過於前世,書曰『人惟求舊』,易稱『同聲相應,同氣相求』,劉將軍之與大魏,兼而兩之,總此二義。前世邂逅,以同為睽,非武皇帝之旨;頃者蹉跌,其泰而否,亦非足下之意也。深思書、易之義,利結分於宿好,故遣降者送吳所獻致名馬、貂、罽,得因無嫌。道初開通,展叙舊情,以達聲問。乆闊情慉,非夫筆墨所能寫陳,亦想足下同其志念。今者,親生男女凡有幾人?年並幾何?僕連失一男一女,今有二男:大兒名肅,年二十九,生於會稽;小兒裁歲餘。臨書愴恨,有懷緬然。」又曰:「過聞『受終於文祖』之言於尚書。又聞『歷數在躬,允執其中』之文於論語。豈自意得於老耄之齒,正值天命受於聖主之會,親見三讓之弘辭,觀衆瑞之總集,覩升堂穆穆之盛禮,瞻燔燎焜曜之青烟;于時忽自以為處唐、虞之運,際於紫微之天庭也。徒慨不得攜子之手,共列於世。有二子之數,以聽有唐『欽哉』之命也。子雖在裔土,想亦極目而迴望,側耳而遐聽,延頸而鶴立也。昔汝南陳公初拜,不依故常,讓上卿於李元禮。以此推之,吾宜退身以避子位也。苟得避子以竊讓名,然後綬帶委質,游談於平、勃之間,與子共陳往時避地之艱辛,樂酒酣宴,高談大噱,亦足遺憂而忘老。捉筆陳情,隨以喜笑。」又曰:「前夏有書而未達,今重有書,而并致前問。皇帝旣深悼劉將軍之早世,又愍其孤之不易,又惜使足下孔明等士人氣類之徒,遂沈溺於羌夷異種之間,永與華夏乖絕,而無朝聘中國之期緣,瞻睎故土桑梓之望也,故復運慈念而勞仁心,重下明詔以發德音,申勑朗等,使重為書與足下等。以足下聦明,揆殷勤之聖意,亦足悟海岱之所常在,知百川之所宜注矣。昔伊尹去夏而就殷,陳平違楚而歸漢,猶曜德於阿衡,著功於宰相。若足下能弼人之遺孤,定人之猶豫,去非常之偽號,事受命之大魏,客主兼不世之榮名,上下蒙不朽之常耀,功與事並,聲與勳著,考績效足,以超越伊、呂矣。旣承詔旨,且服舊之情,情不能已。若不言足下之所能,陳足下之所見,則無以宣明詔命,弘光大之恩,叙宿昔夢想之思。若天啟衆心,子導蜀意,誠此意有攜手之期。若險路未夷,子謀不從,則懼聲問或否,復面何由!前後二書,言每及斯,希不切然有動於懷。足下周游江湖,以曁南海,歷觀夷俗,可謂徧矣;想子之心,結思華夏,可謂深矣。為身擇居,猶願中土;為主擇居,安豈可以不繫意於京師,而持疑於荒裔乎?詳思愚言,速示還報也。」}
 
 
\end{pinyinscope}