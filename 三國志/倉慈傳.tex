\article{倉慈傳}
\begin{pinyinscope}
 
 
 倉慈字孝仁,淮南人也。始為郡吏。建安中,太祖開募屯田於淮南,以慈為綏集都尉。黃初末,為長安令,清約有方,吏民畏而愛之。太和中,遷燉煌太守。郡在西陲,以喪亂隔絕,曠無太守二十歲,大姓雄張,遂以為俗。前太守尹奉等,循故而已,無所匡革。慈到,抑挫權右,撫恤貧羸,甚得其理。舊大族田地有餘,而小民無立錐之土;慈皆隨口割賦,稍稍使畢其本直。先是屬城獄訟衆猥,縣不能決,多集治下;慈躬往省閱,料簡輕重,自非殊死,但鞭杖遣之,一歲決刑曾不滿十人。又常日西域雜胡欲來貢獻,而諸豪族多逆斷絕;旣與貿遷,欺詐侮易,多不得分明。胡常怨望,慈皆勞之。欲詣洛者,為封過所,欲從郡還者,官為平取,輙以府見物與共交市,使吏民護送道路,由是民夷翕然稱其德惠。數年卒官,吏民悲感如喪親戚,圖畫其形,思其遺像。及西域諸胡聞慈死,悉共會聚於戊己校尉及長吏治下發哀,或有以刀畫面,以明血誠,又為立祠,遙共祠之。
 
 
\gezhu{魏略曰:天水王遷,承代慈,雖循其迹,不能及也。金城趙基承遷後,復不如遷。至嘉平中,安定皇甫隆代基為太守。初,燉煌不甚曉田,常灌溉滀水,使極濡洽,然後乃耕。又不曉作耬犂,用水,及種,人牛功力旣費,而收穀更少。隆到,教作耬犂,又教衍溉,歲終率計,其所省庸力過半,得穀加五。又燉煌俗,婦人作裙,攣縮如羊腸,用布一匹;隆又禁改之,所省復不訾。故燉煌人以為隆剛斷嚴毅不及於慈,至於勤恪愛惠,為下興利,可以亞之。}
 
 
自太祖迄于咸熈,魏郡太守陳國吳瓘、清河太守樂安任燠、京兆太守濟北顏斐、弘農太守太原令狐邵、濟南相魯國孔乂,或哀矜折獄,或推誠惠愛,或治身清白,或擿姦發伏,咸為良二千石。
 \gezhu{瓘、燠事行無所見。魏略曰:顏斐字文林。有才學。丞相召為太子洗馬,黃初初轉為黃門侍郎,後為京兆太守。始,京兆從馬超破後,民人多不專於農殖,又歷數四二千石,取解目前,亦不為民作乆遠計。斐到官,乃令屬縣整阡陌,樹桑果。是時民多無車牛。斐又課民以閑月取車材,使轉相教匠作車。又課民無牛者,令畜豬狗,賣以買牛。始者民以為煩,一二年閒,家家有丁車、大牛。又起文學,聽吏民欲讀書者,復其小傜。又於府下起菜園,使吏役閑鉏治。又課民當輸租時,車牛各因便致薪兩束,為冬寒冰炙筆硯。於是風化大行,吏不煩民,民不求吏。京兆與馮翊、扶風接界,二郡道路旣穢塞,田疇又荒萊,人民饑凍,而京兆皆整頓開明,豐富常為雍州十郡最。斐又清己,仰奉而已,於是吏民恐其遷轉也。至青龍中,司馬宣王在長安立軍市,而軍中吏士多侵侮縣民,斐以白宣王。宣王乃發怒召軍市候,便於斐前杖一百。時長安典農與斐共坐,以為斐宜謝,乃私推築斐。斐不肯謝,良乆乃曰:「斐意觀明公受分陝之任,乃欲一齊衆庶,必非有所左右也。而典農竊見推築,欲令斐謝;假令斐謝,是更為不得明公意也。」宣王遂嚴持吏士。自是之後,軍營、郡縣各得其分。後數歲,遷為平原太守,吏民啼泣遮道,車不得前,步步稽留,十餘日乃出界,東行至崤而疾困。斐素心戀京兆,其家人從者見斐病甚,勸之,言:「平原當自勉勵作健。」斐曰:「我心不願平原,汝曹等呼我,何不言京兆邪?」遂卒,還平原。京兆聞之,皆為流涕,為立碑,于今稱頌之。令狐邵字孔叔。父仕漢,為烏丸校尉。建安初,袁氏在兾州,邵去本郡家居鄴。九年,暫出到武安毛城中。會太祖破鄴,遂圍毛城。城破,執邵等輩十餘人,皆當斬。太祖閱見之,疑其衣冠也,問其祖考,而識其父,乃解放,署軍謀掾。仍歷宰守,後徙丞相主簿,出為弘農太守。所在清如冰雪,妻子希至官省;舉善而教,恕以待人,不好獄訟,與下無忌。是時,郡無知經者,乃歷問諸吏,有欲遠行就師,輒假遣,令詣河東就樂詳學經,粗明乃還,因設文學。由是弘農學業轉興。至黃初初,徵拜羽林郎,遷虎賁中郎將,三歲,病亡。始,邵族子愚,為白衣時,常有高志,衆人謂愚必榮令狐氏,而邵獨以為「愚性倜儻,不脩德而願大,必滅我宗」。愚聞邵言,其心不平。及邵為虎賁郎將,而愚仕進已多所更歷,所在有名稱。愚見邵,因從容言次,微激之曰:「先時聞大人謂愚為不繼,愚今竟云何邪?」邵熟視而不荅也。然私謂其妻子曰:「公治性度猶如故也。以吾觀之,終當敗滅。但不知我乆當坐之不邪?將逮汝曹耳!」邵沒之後,十餘年間,愚為兖州刺史,果與王淩謀廢立,家屬誅滅。邵子華,時為弘農郡丞,以屬疏得不坐。案孔氏譜:孔乂字元儁,孔子之後。曾祖疇,字元矩,陳相。漢桓帝立老子廟於苦縣之賴鄉,畫孔子像於壁;疇為陳相,立孔子碑於像前,今見存。乂父祖皆二千石,乂為散騎常侍,上疏規諫。語在三少帝紀。至大鴻臚。子恂字士信,晉平東將軍衞尉也。}
 
 
 
 
 評曰:任峻始興義兵,以歸太祖,闢土殖穀,倉庾盈溢,庸績致矣。蘇則威以平亂,旣政事之良,又矯矯剛直,風烈足稱。杜畿寬猛克濟,惠以康民。鄭渾、倉慈,恤理有方。抑皆魏代之名守乎!恕屢陳時政,經論治體,蓋有可觀焉。
 
 
\end{pinyinscope}