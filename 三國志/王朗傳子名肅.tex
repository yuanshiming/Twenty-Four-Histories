\article{王朗傳子名肅}
\begin{pinyinscope}


王朗字景興,東海郡人也。以通經,拜郎中,除菑丘長。師太尉楊賜,賜薨,棄官行服。舉孝廉,辟公府,不應。徐州刺史陶謙察朗茂才。時漢帝在長安,關東兵起,朗為謙治中,與別駕趙昱等說謙曰:「春秋之義,求諸侯莫如勤王。今天子越在西京,宜遣使奉承王命。」謙乃遣昱奉章至長安。天子嘉其意,拜謙安東將軍。以昱為廣陵太守,朗會稽太守。


\gezhu{朗家傳曰:會稽舊祀秦始皇,刻木為像,與夏禹同廟。朗到官,以為無德之君不應見祀,於是除之。居郡四年,惠愛在民。}
孫策渡江略地。朗功曹虞翻以為力不能拒,不如避之。朗自以身為漢吏,宜保城邑,遂舉兵與策戰,敗績,浮海至東冶。策又追擊,大破之。朗乃詣策。策以朗儒雅,詰讓而不害。
\gezhu{獻帝春秋曰:孫策率軍如閩、越討朗。朗泛舟浮海,欲走交州,為兵所逼,遂詣軍降。策令使者詰朗曰:「問逆賊故會稽太守王朗:朗受國恩當官,云何不惟報德,而阻兵安忍?大軍征討,幸免梟夷,不自埽屏,復聚黨衆,屯住郡境。遠勞王誅,卒不悟順。捕得云降,庶以欺詐,用全首領,得爾與不,具以狀對。」朗稱禽虜,對使者曰:「朗以瑣才,誤竊朝私,受爵不讓,以遘罪網。前見征討,畏死苟免。因治人物,寄命須臾。又迫大兵,惶怖北引。從者疾患,死亡略盡。獨與老母共乘一欐,流矢始交,便棄欐就俘,稽顙自首於征役之中。朗惶惑不達,自稱降虜。緣前迷謬,被詰慙懼。朗愚淺駑怯,畏威自驚。又無良介,不早自歸。於破亡之中,然後委命下隷。身輕罪重,死有餘辜。申脰就鞅,蹴足入絆,叱咤聽聲,東西惟命。」}
雖流移窮困,朝不謀夕,而收卹親舊,分多割少,行義甚著。


太祖表徵之,朗自曲阿展轉江海,積年乃至。
\gezhu{朗被徵未至。孔融與朗書曰:「世路隔塞,情問斷絕,感懷增思。前見章表,知尋湯武罪己之迹,自投東裔同鯀之罰,覽省未周,涕隕潸然。主上寬仁,貴德宥過。曹公輔政,思賢並立。策書屢下,殷勤款至。知櫂舟浮海,息駕廣陵,不意黃能突出羽淵也。談笑有期,勉行自愛!」漢晉春秋曰:孫策之始得朗也,譴讓之。使張昭私問朗,朗誓不屈,策忿而不敢害也,留置曲阿。建安三年,太祖表徵朗,策遣之。太祖問曰:「孫策何以得至此邪?」朗曰:「策勇冠一世,有儁才大志。張子布,民之望也,北面而相之。周公瑾,江淮之傑,攘臂而為其將。謀而有成,所規不細,終為天下大賊,非徒狗盜而已。」}
拜諫議大夫,參司空軍事。
\gezhu{朗家傳曰:朗少與沛國名士劉陽交友。陽為莒令,年三十而卒,故後世鮮聞。初,陽以漢室漸衰,知太祖有雄才,恐為漢累,意欲除之而事不會。及太祖貴,求其嗣子甚急。其子惶窘,走伏無所。陽親舊雖多,莫敢藏者。朗乃納受積年,及從會稽還,又數開解。太祖乆乃赦之,陽門戶由是得全。}
魏國初建,以軍祭酒領魏郡太守,遷少府、奉常、大理。務在寬恕,罪疑從輕。鍾繇明察當法,俱以治獄見稱。
\gezhu{魏略曰:太祖請同會,啁朗曰:「不能效君昔在會稽折秔米飯也。」朗仰而歎曰:「宜適難值!」太祖問:「云何?」朗曰:「如朗昔者,未可折而折;如明公今日,可折而不折也。」太祖以孫權稱臣遣貢諮朗,朗荅曰:「孫權前牋,自詭躬討虜以補前愆,後疏稱臣,以明無二。牙獸屈膝,言鳥告歡,明珠、南金,遠珍必至。情見乎辭,效著乎功。三江五湖,為沼于魏,西吳東越,化為國民。鄢、郢旣拔,荊門自開。席卷巴、蜀,形勢已成。重休累慶,雜沓相隨。承旨之日,撫掌擊節。情之畜者,辭不能宣。」}




文帝即王位,遷御史大夫,封安陵亭侯。上疏勸育民省刑曰:「兵起已來三十餘年,四海盪覆,萬國殄瘁。賴先王芟除寇賊,扶育孤弱,遂令華夏復有綱紀。鳩集兆民,于茲魏土,使封鄙之內,雞鳴狗吠達於四境,蒸庶欣欣,喜遇升平。今遠方之寇未賔,兵戎之役未息,誠令復除足以懷遠人,良宰足以宣德澤,阡陌咸脩,四民殷熾,必復過於曩時而富於平日矣。易稱勑法,書著祥刑,一人有慶,兆民賴之,慎法獄之謂也。昔曹相國以獄市為寄,路溫舒疾治獄之吏。夫治獄者得其情,則無冤死之囚;丁壯者得盡地力,則無饑饉之民;窮老者得仰食倉廩,則無餧餓之殍;嫁娶以時,則男女無怨曠之恨;胎養必全,則孕者無自傷之哀;新生必復,則孩者無不育之累;壯而後役,則幼者無離家之思;二毛不戎,則老者無頓伏之患。醫藥以療其疾,寬繇以樂其業,威罰以抑其彊,恩仁以濟其弱,振貸以贍其乏。十年之後,旣笄者必盈巷。二十年之後,勝兵者必滿野矣。」


及文帝踐阼,改為司空,進封樂平鄉侯。
\gezhu{魏名臣奏載朗節省奏曰:「詔問所宜損益,必謂東京之事也。若夫西京雲陽、汾陰之大祭,千有五百之羣,祀通天之臺,入阿房之宮,齋必百日,養犧五載,牛則三千,其重玉則七千;其器,文綺以飾重席,童女以蹈舞綴;釀酎必貫三時而後成,樂人必三千四百而後備;內宮美人數至近千,學官博士七千餘人;中廄則騑騄駙馬六萬餘匹,外牧則扈養三萬而馬十之;執金吾從騎六百,走卒倍焉;太常行陵赤車千乘,太官賜官奴婢六千,長安城內治民為政者三千,中二千石蔽罪斷刑者二十有五獄。政充事猥,威儀繁富,隆於三代,近過禮中。夫所以極奢吝,大抵多受之於秦餘。旣違繭栗愨誠之本,埽地簡易之指,又失替質而損文、避泰而從約之趣。豈夫當今隆興盛明之時,祖述堯舜之際,割奢務儉之政,除繁崇省之令,詳刑慎罰之教,所宜希慕哉?及夫寢廟日一太牢之祀,郡國並立宗廟之法,丞相御史大夫官屬吏從之數,若此之輩,旣已屢改於哀、平之前,不行光武之後矣。謹桉圖牒,所改秦在天地及五帝、六宗、宗廟、社稷,旣已因前代之兆域矣。夫天地則埽地而祭,其餘則皆壇而埒之矣。明堂所以祀上帝,靈臺所以觀天文,辟雍所以脩禮樂,太學所以集儒林,高禖所以祈休祥,又所以察時務,揚教化。稽古先民,開誕慶祚,舊時皆在國之陽,並高棟夏屋,足以肆饗射,望雲物。七郊雖尊祀尚質,猶皆有門宇便坐,足以避風雨。可須軍罷年豐,以漸脩治。舊時虎賁羽林五營兵,及衞士并合,雖且萬人,或商賈墯游子弟,或農野謹鈍之人;雖有乘制之處,不講戎陣,旣不簡練,又希更寇,雖名實不副,難以備急。有警而後募兵,軍行而後運粮,或乃兵旣乆屯,而不務營佃,不脩器械,無有貯聚,一隅馳羽檄,則三靣並荒擾,此亦漢氏近世之失而不可式者也。當今諸夏已安,而巴蜀在畫外。雖未得偃武而弢甲,放馬而戢兵,宜因年之大豐,遂寄軍政於農事。吏士小大,並勤稼穡,止則成井里於廣野,動則成校隊於六軍,省其暴繇,贍其衣食。易稱『恱以使民,民忘其勞;恱以犯難,民忘其死』,今之謂矣。粮畜於食,勇畜於勢,雖坐曜烈威而衆未動,畫外之蠻,必復稽顙以求改往而效用矣。若畏威效用,不戰而定,則賢於交兵而後威立,接刃而後功成遠矣。若姦凶不革,遂迷不反,猶欲以其所虐用之民,待大魏投命報養之士,然後徐以前歌後舞樂征之衆,臨彼倒戟折矢樂服之羣,伐腐摧枯,未足以為喻也。」}
時帝頗出游獵,或昏夜還宮。朗上疏曰:「夫帝王之居,外則飾周衞,內則重禁門,將行則設兵而後出幄,稱警而後踐墀,張弧而後登輿,清道而後奉引,遮列而後轉轂,靜室而後息駕,皆所以顯至尊,務戒慎,垂法教也。近日車駕出臨捕虎,日昃而行,及昏而反,違警蹕之常法,非萬乘之至慎也。」帝報曰:「覽表,雖魏絳稱虞箴以諷晉悼,相如陳猛獸以戒漢武,未足以喻。方今二寇未殄,將帥遠征,故時入原野以習戎備。至於夜還之戒,已詔有司施行。」
\gezhu{王朗集載朗為大理時上主簿趙郡張登:「昔為本縣主簿,值黑山賊圍郡,登與縣長王儁帥吏兵七十二人直往赴救,與賊交戰,吏兵散走。儁殆見害,登手格二賊,以全儁命。又守長夏逸,為督郵所枉,登身受考掠,理逸之罪。義濟二君。宜加顯異。」太祖以所急者多,未遑擢叙。至黃初初,朗又與太尉鍾繇連名表聞,兼稱登在職勤勞。詔曰:「登忠義彰著,在職功勤。名位雖卑,直亮宜顯。饔膳近任,當得此吏。今以登為太官令。」}




初,建安末,孫權始遣使稱藩,而與劉備交兵。詔議「當興師與吳并取蜀不」?朗議曰:「天子之軍,重於華、岱,誠宜坐曜天威,不動若山。假使權親與蜀賊相持,搏戰曠日,智均力敵,兵不速決,當須軍興以成其勢者,然後宜選持重之將,承寇賊之要,相時而後動,擇地而後行,一舉可無餘事。今權之師未動,則助吳之軍無為先征。且雨水方盛,非行軍動衆之時。」帝納其計。黃初中,鵜鶘集靈芝池,詔公卿舉獨行君子。朗薦光祿大夫楊彪,且稱疾,讓位於彪。帝乃為彪置吏卒,位次三公。詔曰:「朕求賢於君而未得,君乃翻然稱疾,非徒不得賢,更開失賢之路,增玉鉉之傾。無乃居其室出其言不善,見違於君子乎!君其勿有後辭。」朗乃起。


孫權欲遣子登入侍,不至。是時車駕徙許昌,大興屯田,欲舉軍東征。朗上疏曰:「昔南越守善,嬰齊入侍,遂為冢嗣,還君其國。康居驕黠,情不副辭,都護奏議以為宜遣侍子,以黜無禮。且吳濞之禍,萌於子入,隗嚻之叛,亦不顧子。往者聞權有遣子之言而未至,今六軍戒嚴,臣恐輿人未暢聖旨,當謂國家慍於登之逋留,是以為之興師。設師行而登乃至,則為所動者至大,所致者至細,猶未足以為慶。設其傲很,殊無入志,懼彼輿論之未暢者,並懷伊邑。臣愚以為宜勑別征諸將,各明奉禁令,以慎守所部。外曜烈威,內廣耕稼,使泊然若山,澹然若淵,勢不可動,計不可測。」是時,帝以成軍遂行,權子不至,車駕臨江而還。
\gezhu{魏書曰:車駕旣還,詔三公曰:「三世為將,道家所忌。窮兵黷武,古有成戒。況連年水旱,士民損耗,而功作倍於前,勞役兼於昔,進不滅賊,退不和民。夫屋漏在上,知之在下,然迷而知反,失道不遠,過而能改,謂之不過。今將休息,棲備高山,沈權九淵,割除擯棄,投之畫外。車駕當以今月中旬到譙,淮、漢衆軍,亦各還反,不臘西歸矣。」}


明帝即位,進封蘭陵侯,增邑五百,并前千二百戶。使至鄴省文昭皇后陵,見百姓或有不足。是時方營脩宮室,朗上疏曰:「陛下即位已來,恩詔屢布,百姓萬民莫不欣欣。臣頃奉使北行,往反道路,聞衆傜役,其可得蠲除省減者甚多。願陛下重留日昃之聽,以計制寇。昔大禹將欲拯天下之大患,故乃先卑其宮室,儉其衣食,用能盡有九州,弼成五服。句踐欲廣其禦兒之疆,
\gezhu{禦兒,吳界邊戍之地名。}
馘夫差於姑蘇,故亦約其身以及家,儉其家以施國,用能囊括五湖,席卷三江,取威中國,定霸華夏。漢之文、景亦欲恢弘祖業,增崇洪緒,故能割意於百金之臺,昭儉於弋綈之服,內減太官而不受貢獻,外省傜賦而務農桑,用能號稱升平,幾致刑錯。孝武之所以能奮其軍勢,拓其外境,誠因祖考畜積素足,故能遂成大功。霍去病,中才之將,猶以匈奴未滅,不治第宅。明卹遠者略近,事外者簡內。自漢之初及其中興,皆於金革略寢之後,然後鳳闕猥閌,德陽並起。今當建始之前足用列朝會,崇華之後足用序內官,華林、天淵足用展游宴,若且先成閶闔之象魏,使足用列遠人之朝貢者,脩城池,使足用絕踰越,成國險,其餘一切,且須豐年。一以勤耕農為務,習戎備為事,則國無怨曠,戶口滋息,民充兵彊,而寇戎不賔,緝熙不足,未之有也。」轉為司徒。


時屢失皇子,而後宮就館者少,朗上疏曰:「昔周文十五而有武王,遂享十子之祚,以廣諸姬之胤。武王旣老而生成王,成王是以鮮於兄弟。此二王者,各樹聖德,無以相過,比其子孫之祚,則不相如。蓋生育有早晚,所產有衆寡也。陛下旣德祚兼彼二聖,春秋高於姬文育武之時矣,而子發未舉於椒蘭之奧房,藩王未繁於掖庭之衆室。以成王為喻,雖未為晚,取譬伯邑,則不為夙。周禮六宮內官百二十人,而諸經常說,咸以十二為限,至於秦漢之末,或以千百為數矣。然雖彌猥,而就時於吉館者或甚鮮,明『百斯男』之本,誠在於一意,不但在於務廣也。老臣慺慺,願國家同祚於軒轅之五五,而未及周文之二五,用為伊邑。且少小常苦被褥泰溫,泰溫則不能便柔膚弱體,是以難可防護,而易用感慨。若常令少小之縕袍,不至於甚厚,則必咸保金石之性,而比壽於南山矣。」帝報曰:「夫忠至者辭篤,愛重者言深。君旣勞思慮,又手筆將順,三復德音,欣然無量。朕繼嗣未立,以為君憂,欽納至言,思聞良規。」朗著易、春秋、孝經、周官傳,奏議論記,咸傳於世。
\gezhu{魏略曰:朗本名嚴,後改為朗。魏書曰:朗高才博雅,而性嚴整慷慨,多威儀,恭儉節約,自婚姻中表禮贄無所受。常譏世俗有好施之名,而不卹窮賤,故用財以周急為先。}
太和二年薨,謚曰成侯。子肅嗣。初,文帝分朗戶邑,封一子列侯,朗乞封兄子詳。


肅字子雍。年十八,從宋忠讀太玄,而更為之解。
\gezhu{肅父朗與許靖書云:肅生於會稽。}
黃初中,為散騎黃門侍郎。太和三年,拜散騎常侍。四年,大司馬曹真征蜀,肅上疏曰:「前志有之,『千里饋糧,士有饑色,樵蘇後爨,師不宿飽』,此謂平塗之行軍者也。又況於深入阻險,鑿路而前,則其為勞必相百也。今又加之以霖雨,山坂峻滑,衆逼而不展,糧縣而難繼,實行軍者之大忌也。聞曹真發已踰月而行裁半谷,治道功夫,戰士悉作。是賊偏得以逸而待勞,乃兵家之所憚也。言之前代,則武王伐紂,出關而復還;論之近事,則武、文征權,臨江而不濟。豈非所謂順天知時,通於權變者哉!兆民知聖上以水雨艱劇之故,休而息之,後日有釁,乘而用之,則所謂恱以犯難,民忘其死者矣。」於是遂罷。又上疏:「宜遵舊禮,為大臣發哀,薦果宗廟。」事皆施行。又上疏陳政本曰:「除無事之位,損不急之祿,止因食之費,并從容之官;使官必有職,職任其事,事必受祿,祿代其耕,乃往古之常式,當今之所宜也。官寡而祿厚,則公家之費鮮,進仕之志勸。進仕之志勸,各展才力,莫相倚仗。敷奏以言,明試以功,能之與否,簡在帝心。是以唐、虞之設官分職,申命公卿,各以其事,然後惟龍為納言,猶今尚書也,以出內帝命而已。夏、殷不可得而詳。甘誓曰『六事之人』,明六卿亦典事者也。周官則備矣,五日視朝,公卿大夫並進,而司士辨其位焉。其記曰:『坐而論道,謂之王公;作而行之,謂之士大夫。』及漢之初,依擬前代,公卿皆親以事升朝。故高祖躬追反走之周昌,武帝遙可奉奏之汲黯,宣帝使公卿五日一朝,成帝始置尚書五人。自是陵遲,朝禮遂闕。可復五日視朝之儀,使公卿尚書各以事進。廢禮復興,光宣聖緒,誠所謂名美而實厚者也。」


青龍中,山陽公薨,漢主也。肅上疏曰:「昔唐禪虞,虞禪夏,皆終三年之喪,然後踐天子之尊。是以帝號無虧,君禮猶存。今山陽公承順天命,允荅民望,進禪大魏,退處賔位。公之奉魏,不敢不盡節。魏之待公,優崇而不臣。旣至其薨,櫬斂之制,輿徒之飾,皆同之於王者,是故遠近歸仁,以為盛美。且漢緫帝皇之號,號曰皇帝。有別稱帝,無別稱皇,則皇是其差輕者也。故當高祖之時,土無二王,其父見在而使稱皇,明非二王之嫌也。況今以贈終,可使稱皇以配其謚。」明帝不從使稱皇,乃追謚曰漢孝獻皇帝。
\gezhu{孫盛曰:化合神者曰皇,德合天者曰帝。是故三皇創號,五帝次之。然則皇之為稱,妙於帝矣。肅謂為輕,不亦謬乎!臣松之以為上古謂皇皇后帝,次言三、五,先皇後帝,誠如盛言。然漢氏諸帝,雖尊父為皇,其實則貴而無位,高而無民,比之於帝,得不謂之輕乎!魏因漢禮,名號無改。孝獻之崩,豈得遠考古義?肅之所云,蓋就漢制而為言耳。謂之為謬,乃是譏漢,非難肅也。}




後肅以常侍領祕書監,兼崇文觀祭酒。景初間,宮室盛興,民失農業,期信不敦,刑殺倉卒。肅以疏曰:「大魏承百王之極,生民無幾,干戈未戢,誠宜息民而惠之以安靜遐邇之時也。夫務畜積而息疲民,在於省傜役而勤稼穡。今宮室未就,功業未訖,運漕調發,轉相供奉。是以丁夫疲於力作,農者離其南畒,種穀者寡,食穀者衆,舊穀旣沒,新穀莫繼。斯則有國之大患,而非備豫之長策也。今見作者三四萬人,九龍可以安聖體,其內足以列六宮,顯陽之殿,又向將畢,惟泰極已前,功夫尚大,方向盛寒,疾疢或作。誠願陛下發德音,下明詔,深愍役夫之疲勞,厚矜兆民之不贍,取常食廩之士,非急要者之用,選其丁壯,擇留萬人,使一朞而更之,咸知息代有日,則莫不恱以即事,勞而不怨矣。計一歲有三百六十萬夫,亦不為少。當一歲成者,聽且三年。分遣其餘,使皆即農,無窮之計也。倉有溢粟,民有餘力:以此興功,何功不立?以此行化,何化不成?夫信之於民,國家大寶也。仲尼曰:『自古皆有死,民非信不立。』夫區區之晉國,微微之重耳,欲用其民,先示以信,是故原雖將降,顧信而歸,用能一戰而霸,于今見稱。前車駕當幸洛陽,發民為營,有司命以營成而罷。旣成,又利其功力,不以時遣。有司徒營其目前之利,不顧經國之體。臣愚以為自今已後,儻復使民,宜明其令,使必如期。若有事以次,寧復更發,無或失信。凡陛下臨時之所行刑,皆有罪之吏,宜死之人也。然衆庶不知,謂為倉卒。故願陛下下之於吏而暴其罪。鈞其死也,無使汙于宮掖而為遠近所疑。且人命至重,難生易殺,氣絕而不續者也,是以聖賢重之。孟軻稱殺一無辜以取天下,仁者不為也。漢時有犯蹕驚乘輿馬者,廷尉張釋之奏使罰金,文帝恠其輕,而釋之曰:『方其時,上使誅之則已。今下廷尉。廷尉,天下之平也,一傾之,天下用法皆為輕重,民安所措其手足?』臣以為大失其義,非忠臣所宜陳也。廷尉者,天子之吏也,猶不可以失平,而天子之身,反可以惑謬乎?斯重於為己,而輕於為君,不忠之甚也。周公曰:『天子無戲言;言則史書之,工誦之,士稱之。』言猶不戲,而况行之乎?故釋之之言不可不察,周公之戒不可不法也。」又陳「諸鳥獸無用之物,而有芻穀人徒之費,皆可蠲除。」




帝嘗問曰:「漢桓帝時,白馬令李雲上書言:『帝者,諦也。是帝欲不諦。』當何得不死?」肅對曰:「但為言失逆順之節。原其本意,皆欲盡心,念存補國。且帝者之威,過於雷霆,殺一匹夫,無異螻蟻。寬而宥之,可以示容受切言,廣德宇於天下。故臣以為殺之未必為是也。」帝又問:「司馬遷以受刑之故,內懷隱切,著史記非貶孝武,令人切齒。」對曰:「司馬遷記事,不虛美,不隱惡。劉向、揚雄服其善叙事,有良史之才,謂之實錄。漢武帝聞其述史記,取孝景及己本紀覽之,於是大怒,削而投之。於今此兩紀有錄無書。後遭李陵事,遂下遷蠶室。此為隱切在孝武,而不在於史遷也。」




正始元年,出為廣平太守。公事徵還,拜議郎。頃之,為侍中,遷太常。時大將軍曹爽專權,任用何晏、鄧颺等。肅與太尉蔣濟、司農桓範論及時政,肅正色曰:「此輩即弘恭、石顯之屬,復稱說邪!」爽聞之,戒何晏等曰:「當共慎之!公卿已比諸君前世惡人矣。」坐宗廟事免。後為光祿勳。時有二魚長尺,集于武庫之屋,有司以為吉祥。肅曰:「魚生於淵而亢於屋,介鱗之物失其所也。邊將其殆有棄甲之變乎?」其後果有東關之敗。徙為河南尹。


嘉平六年,持節兼太常,奉法駕,迎高貴鄉公于元城。是歲,白氣經天,大將軍司馬景王問肅其故,肅荅曰:「此蚩尤之旗也,東南其有亂乎?君若脩己以安百姓,則天下樂安者歸德,唱亂者先亡矣。」明年春,鎮東將軍毌丘儉、揚州刺史文欽反,景王謂肅曰:「霍光感夏侯勝之言,始重儒學之士,良有以也。安國寧主,其術焉在?」肅曰:「昔關羽率荊州之衆,降于禁於漢濵,遂有北向爭天下之志。後孫權襲取其將士家屬,羽士衆一旦瓦解。今淮南將士父母妻子皆在內州,但急往禦衞,使不得前,必有關羽土崩之勢矣。」景王從之,遂破儉、欽。後遷中領軍,加散騎常侍,增邑三百,并前二千二百戶。甘露元年薨,門生縗絰者以百數。追贈衞將軍,謚曰景侯。子惲嗣。惲薨,無子,國絕。景元四年,封肅子恂為蘭陵侯。咸熈中,開建五等,以肅著勳前朝,改封恂為氶子。
\gezhu{世語曰:恂字良夫,有通識,在朝忠正。歷河南尹、侍中,所居有稱。乃心存公,有匪躬之節。鬲令袁毅餽以駿馬,知其貪財,不受。毅竟以黷貨而敗。建立二學,崇明五經,皆恂所建。卒時年四十餘,贈車騎將軍。肅女適司馬文王,即文明皇后,生晉武帝、齊獻王攸。晉諸公贊曰:恂兄弟八人。其達者,虔字恭祖,以功幹見稱,位至尚書。弟愷,字君夫,少有才力而無行檢,與衞尉石崇友善,俱以豪侈競於世,終於後將軍。虔子康、隆,仕亦宦達,為後世所重。}


初,肅善賈、馬之學,而不好鄭氏,采會同異,為尚書、詩、論語、三禮、左氏解,及撰定父朗所作易傳,皆列於學官。其所論駮朝廷典制、郊祀、宗廟、喪紀、輕重,凡百餘篇。時樂安孫叔然,
\gezhu{臣松之桉叔然與晉武帝同名,故稱其字。}
授學鄭玄之門,人稱東州大儒。徵為秘書監,不就。肅集聖證論以譏短玄,叔然駮而釋之,及作周易、春秋例,毛詩、禮記、春秋三傳、國語、爾雅諸注,又著書十餘篇。自魏初徵士燉煌周生烈,
\gezhu{臣松之桉此人姓周生,名烈。何晏論語集解有烈義例,餘所著述,見晉武帝中經簿。}
明帝時大司農弘農董遇等,亦歷注經傳,頗傳於世。
\gezhu{魏略曰:遇字季直,性質訥而好學。興平中,關中擾亂,與兄季中依將軍段煨。采梠負販,而常挾持經書,投閑習讀。其兄笑之而遇不改。及建安初,王綱小設,郡舉孝廉,稍遷黃門侍郎。是時,漢帝委政太祖,遇旦夕侍講,為天子所愛信。至二十二年,許中百官矯制,遇雖不與謀,猶被錄詣鄴,轉為冗散。常從太祖西征,道由孟津,過弘農王冢。太祖疑欲謁,顧問左右,左右莫對,遇乃越第進曰:「春秋之義,國君即位未踰年而卒,未成為君。弘農王即阼旣淺,又為暴臣所制,降在藩國,不應謁。」太祖乃過。黃初中,出為郡守。明帝時,入為侍中、大司農。數年,病亡。初,遇善治老子,為老子作訓注。又善左氏傳,更為作朱墨別異。人有從學者,遇不肯教,而云「必當先讀百徧」。言「讀書百徧而義自見」。從學者云:「苦渴無日。」遇言「當以三餘」。或問三餘之意,遇言「冬者歲之餘,夜者日之餘,陰雨者時之餘也」。由是諸生少從遇學,無傳其朱墨者。}
\gezhu{世語曰:遇子綏,位至祕書監,亦有才學。齊王冏功臣董艾,即綏之子也。}
\gezhu{魏略以遇及賈洪、邯鄲淳、薛夏、隗禧、蘇林、樂詳等七人為儒宗,其序曰:「從初平之元,至建安之末,天下分崩,人懷苟且,綱紀旣衰,儒道尤甚。至黃初元年之後,新主乃復,始掃除太學之灰炭,補舊石碑之缺壞,備博士之員錄,依漢甲乙以考課。申告州郡,有欲學者,皆遣詣太學。太學始開,有弟子數百人。至太和、青龍中,中外多事,人懷避就。雖性非解學,多求詣太學。太學諸生有千數,而諸博士率皆麄踈,無以教弟子。弟子本亦避役,竟無能習學,冬來春去,歲歲如是。又雖有精者,而臺閣舉格太高,加不念統其大義,而問字指墨法點注之間,百人同試,度者未十。是以志學之士遂復陵遲,而末求浮虛者各競逐也。正始中,有詔議圜丘,普延學士。時郎官及司徒領吏二萬餘人,雖復分布,見在京師者尚且萬人,而應書與議者略無幾人。又是時朝堂公卿以下四百餘人,其能操筆者未有十人,多皆相從飽食而退。嗟夫!學業沈隕,乃至於此。是以私心常區區貴乎數公者,各處荒亂之際,而能守志彌敦者也。」}
\gezhu{賈洪字叔業,京兆新豐人也。好學有才,而特精於春秋左傳。建安初,仕郡,舉計掾,應州辟。時州中自參軍事以下百餘人,唯洪與馮翊嚴苞交通,材學最高。洪歷守三縣令,所在輒開除廄舍,親授諸生。後馬超反,超劫洪,將詣華陰,使作露布。洪不獲已,為作之。司隷鍾繇在東,識其文,曰:「此賈洪作也。」及超破走,太祖召洪署軍謀掾。猶以其前為超作露布文,故不即叙。晚乃出為陰泉長。延康中,轉為白馬王相。善能談戲。王彪亦雅好文學,常師宗之,過於三卿。數歲病亡,亡時年五十餘,時人為之恨仕不至二千石。而嚴苞亦歷守二縣,黃初中,以高才入為秘書丞,數奏文賦,文帝異之。出為西平太守,卒官。}
\gezhu{薛夏字宣聲,天水人也。博學有才。天水舊有姜、閻、任、趙四姓,常推於郡中,而夏為單家,不為降屈。四姓欲共治之,夏乃游逸,東詣京師。太祖宿聞其名,甚禮遇之。後四姓又使囚遙引夏,關移潁川,收捕係獄。時太祖已在兾州,聞夏為本郡所質,撫掌曰:「夏無罪也。漢陽兒輩直欲殺之耳!」乃告潁川使理出之,召署軍謀掾。文帝又嘉其才,黃初中為祕書丞,帝每與夏推論書傳,未嘗不終日也。每呼之不名,而謂之薛君。夏居甚貧,帝又顧其衣薄,解所御服袍賜之。其後征東將軍曹休來朝,時帝方與夏有所咨論,而外啟休到,帝引入。坐定,帝顧夏言之於休曰:「此君,祕書丞天水薛宣聲也,宜共談。」其見遇如此。尋欲用之,會文帝崩。至太和中,嘗以公事移蘭臺。蘭臺自以臺也,而祕書署耳,謂夏為不得移也,推使當有坐者。夏報之曰:「蘭臺為外臺,祕書為內閣,臺、閣,一也,何不相移之有?」蘭臺屈無以折。自是之後,遂以為常。後數歲病亡,勑其子無還天水。隗禧字子牙,京兆人也。世單家。少好學。初平中,三輔亂,禧南客荊州,不以荒擾,擔負經書,每以採梠餘日,則誦習之。太祖定荊州,召署軍謀掾。黃初中,為譙王郎中。王宿聞其儒者,常虛心從學。禧亦敬恭以授王,由是大得賜遺。以病還,拜郎中。年八十餘,以老處家,就之學者甚多。禧旣明經,又善星官,常仰瞻天文,歎息謂魚豢曰:「天下兵戈尚猶未息,如之何?」豢又嘗從問左氏傳,禧荅曰:「欲知幽微莫若易,人倫之紀莫若禮,多識山川草木之名莫若詩,左氏直相斫書耳,不足精意也。」豢因從問詩,禧說齊、韓、魯、毛四家義,不復執文,有如諷誦。又撰作諸經解數十萬言,未及繕寫而得聾,後數歲病亡也。}
\gezhu{其邯鄲淳事在王粲傳,蘇林事在劉邵、高堂隆傳,樂詳事在杜畿傳。}
\gezhu{魚豢曰:學之資於人也,其猶藍之染於素乎!故雖仲尼,猶曰「吾非生而知之者」,況凡品哉!且世人所以不貴學者,必見夫有「誦詩三百而不能專對於四方」故也。余以為是則下科耳,不當顧中庸以上,材質適等,而加之以文乎!今此數賢者,略余之所識也。檢其事能,誠不多也。但以守學不輟,乃上為帝王所嘉,下為國家名儒,非由學乎?由是觀之,學其胡可以已哉!}


評曰:鍾繇開達理幹,華歆清純德素,王朗文博富贍,誠皆一時之俊偉也。魏氏初祚,肇登三司,盛矣夫!王肅亮直多聞,能析薪哉!
\gezhu{劉寔以為肅方於事上而好下佞己,此一反也。性嗜榮貴而不求苟合,此二反也。吝惜財物而治身不穢,此三反也。}


\end{pinyinscope}