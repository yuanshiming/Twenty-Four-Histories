\article{曹休傳}
\begin{pinyinscope}
 
 
 曹休字文烈,太祖族子也。天下亂,宗族各散去鄉里。休年十餘歲,喪父,獨與一客擔喪假葬,攜將老母,渡江至吳。
 
 
\gezhu{魏書曰:休祖父嘗為吳郡太守。休於太守舍見壁上祖父畫像,下榻拜涕泣,同坐者皆嘉歎焉。}
 以太祖舉義兵,易姓名轉至荊州,閒行北歸,見太祖。太祖謂左右曰:「此吾家千里駒也。」使與文帝同止,見待如子。常從征伐,使領虎豹騎宿衞。
 
 
 
 
 劉備遣將吳蘭屯下辯,太祖遣曹洪征之,以休為騎都尉,參洪軍事。太祖謂休曰:「汝雖參軍,其實帥也。」洪聞此令,亦委事於休。備遣張飛屯固山,欲斷軍後。衆議狐疑,休曰:「賊實斷道者,當伏兵潛行。今乃先張聲勢,此其不能也。宜及其未集,促擊蘭,蘭破則飛自走矣。」洪從之,進兵擊蘭,大破之,飛果走。
 
 
太祖拔漢中,諸軍還長安,拜休中領軍。文帝即王位,為領軍將軍,錄前後功,封東陽亭侯。夏侯惇薨,以休為鎮南將軍,假節都督諸軍事,車駕臨送,上乃下輿執手而別。孫權遣將屯歷陽,休到,擊破之,又別遣兵渡江,燒賊蕪湖營數千家。遷征東將軍,領揚州刺史,進封安陽鄉侯。
 \gezhu{魏書曰:休喪母至孝。帝使侍中奪喪服,使飲酒食肉,休受詔而形體益憔悴。乞歸譙葬母,帝復遣越騎校尉薛喬奉詔節其憂哀,使歸家治喪,一宿便葬,葬訖詣行在所。帝見,親自寬慰之。其見愛重如此。}
 
 
帝征孫權,以休為征東大將軍,假黃鉞,督張遼等及諸州郡二十餘軍,擊權大將呂範等於洞浦,破之。拜揚州牧。明帝即位,進封長平侯。吳將審悳屯皖,休擊破之,斬悳首,吳將韓綜、翟丹等前後率衆詣休降。增邑四百,并前二千五百戶,遷大司馬,都督揚州如故。太和二年,帝為二道征吳,遣司馬宣王從漢水下,督休諸軍向尋陽。賊將偽降,休深入,戰不利,退還宿石亭。軍夜驚,士卒亂,棄甲兵輜重甚多。休上書謝罪,帝遣屯騎校尉楊曁慰諭,禮賜益隆。休因此癰發背薨,謚曰壯侯。子肇嗣。
 \gezhu{世語曰:肇字長思。}
 
 
肇有當世才度,為散騎常侍、屯騎校尉。明帝寢疾,方與燕王宇等屬以後事。帝意尋變,詔肇以侯歸第。正始中薨。追贈衞將軍。子興嗣。初,文帝分休戶三百封肇弟纂為列侯,後為殄吳將軍,薨,追贈前將軍。
 \gezhu{張隱文士傳曰:肇孫攄,字顏遠,少厲志操,博學有才藻。仕晉,辟公府,歷洛陽令,有能名。大司馬齊王冏輔政,攄與齊人左思俱為記室督。從中郎出為襄陽太守、征南司馬。值天下亂,攄討賊向吳,戰敗死。}
 
 
\end{pinyinscope}