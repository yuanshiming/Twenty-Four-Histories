\article{陸瑁傳}
\begin{pinyinscope}
 
 
 陸瑁字子璋,丞相遜弟也。少好學篤義。陳國陳融、陳留濮陽逸、沛郡蔣纂、廣陵袁迪等,皆單貧有志,就瑁游處,
 
 
\gezhu{迪孫曄,字思光,作獻帝春秋,云迪與張紘等俱過江,迪父綏為太傅掾,張超之討董卓,以綏領廣陵事。}
 瑁割少分甘,與同豐約。及同郡徐原,爰居會稽,素不相識,臨死遺書,託以孤弱,瑁為起立墳墓,收導其子。又瑁從父績早亡,二男一女,皆數歲以還,瑁迎攝養,至長乃別。州郡辟舉,皆不就。
 
 
 
 
 時尚書曁豔盛明臧否,差斷三署,頗揚人闇昧之失,以顯其讁。瑁與書曰:「夫聖人嘉善矜愚,忘過記功,以成美化。加今王業始建,將一大統,此乃漢高棄瑕錄用之時也,若令善惡異流,貴汝潁月旦之評,誠可以厲俗明教,然恐未易行也。宜遠模仲尼之汎愛,中則郭泰之弘濟,近有益於大道也。」豔不能行,卒以致敗。
 
 
 
 
 嘉禾元年,公車徵瑁,拜議郎、選曹尚書。孫權忿公孫淵之巧詐反覆,欲親征之,瑁上疏諫曰:「臣聞聖王之御遠夷,羈縻而已,不常保有,故古者制地,謂之荒服,言慌惚無常,不可保也。今淵東夷小醜,屏在海隅,雖託人面,與禽獸無異。國家所為不愛貨寶遠以加之者,非嘉其德義也,誠欲誘納愚筭,以規其馬耳。淵之驕黠,恃遠負命,此乃荒貊常態,豈足深怪?昔漢諸帝亦嘗銳意以事外夷,馳使散貨,充滿西域,雖時有恭從,然其使人見害,財貨并沒,不可勝數。今陛下不忍悁悁之忿,欲越巨海,身踐其土,羣臣愚議,竊謂不安。何者?北寇與國,壤地連接,苟有間隙,應機而至。夫所以越海求馬,曲意於淵者,為赴目前之急,除腹心之疾也,而更棄本追末,捐近治遠,忿以改規,激以動衆,斯乃猾虜所願聞,非大吳之至計也。又兵家之術,以功役相疲,勞逸相待,得失之間,所覺輒多。且沓渚去淵,道里尚遠,今到其岸,兵勢三分,使彊者進取,次當守船,又次運糧,行人雖多,難得悉用;加以單步負糧,經遠深入,賊地多馬,邀截無常。若淵狙詐,與北未絕,動衆之日,脣齒相濟。若實孑然無所憑賴,其畏怖遠迸,或難卒滅。使天誅稽於朔野,山虜承間而起,恐非萬安之長慮也。」權未許。
 
 
 
 
 瑁重上疏曰:「夫兵革者,固前代所以誅暴亂,威四夷也,然其役皆在姦雄已除,天下無事,從容廟堂之上,以餘議議之耳。至於中夏鼎沸,九域槃牙之時,率須深根固本,愛力惜費,務自休養,以待鄰敵之闕,未有正於此時,舍近治遠,以疲軍旅者也。昔尉他叛逆,僭號稱帝,于時天下乂安,百姓殷阜,帶甲之數,糧食之積,可謂多矣,然漢文猶以遠征不易,重興師旅,告喻而已。今凶桀未殄,疆埸猶警,雖蚩尤、鬼方之亂,故當以緩急差之,未宜以淵為先。願陛下抑威任計,暫寧六師,潛神嘿規,以為後圖,天下幸甚。」權再覽瑁書,嘉其詞理端切,遂不行。
 
 
 
 
 初,瑁同郡聞人敏見待國邑,優於宗脩,惟瑁以為不然,後果如其言。
 
 
赤烏二年,瑁卒。子喜亦涉文籍,好人倫,孫皓時為選曹尚書。
 \gezhu{吳錄曰:喜字文仲,瑁第二子也,入晉為散騎常侍。瑁孫曄,字士光,至車騎將車、儀同三司。曄弟玩,字士瑤。晉陽秋稱玩器量淹雅,位至司空,追贈太尉。}
 
 
\end{pinyinscope}