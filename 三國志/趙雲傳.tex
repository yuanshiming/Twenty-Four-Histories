\article{趙雲傳}
\begin{pinyinscope}
 
 
 趙雲字子龍,常山真定人也。本屬公孫瓚,瓚遣先主為田楷拒袁紹,雲遂隨從,為先主主騎。
 
 
\gezhu{雲別傳曰:雲身長八尺,姿顏雄偉,為本郡所舉,將義從吏兵詣公孫瓚。時袁紹稱兾州牧,瓚深憂州人之從紹也,善雲來附,嘲雲曰:「聞貴州人皆願袁氏,君何獨迴心,迷而能反乎?」雲荅曰:「天下訩訩,未知孰是,民有倒縣之厄,鄙州論議,從仁政所在,不為忽袁公私明將軍也。」遂與瓚征討。時先主亦依託瓚,每接納雲,雲得深自結託。雲以兄喪,辭瓚暫歸,先主知其不反,捉手而別,雲辭曰:「終不背德也。」先主就袁紹,雲見於鄴。先主與雲同床眠卧,密遣雲合募得數百人,皆稱劉左將軍部曲,紹不能知。遂隨先主至荊州。}
 及先主為曹公所追於當陽長阪,棄妻子南走,雲身抱弱子,即後主也,保護甘夫人,即後主母也,皆得免難。遷為牙門將軍。先主入蜀,雲留荊州。
 \gezhu{雲別傳曰:初,先主之敗,有人言雲已北去者,先主以手戟擿之曰:「子龍不棄我走也。」頃之,雲至。從平江南,以為偏將軍,領桂陽太守,代趙範。範寡嫂曰樊氏,有國色,範欲以配雲。雲辭曰:「相與同姓,卿兄猶我兄。」固辭不許。時有人勸雲納之,雲曰:「範迫降耳,心未可測;天下女不少。」遂不取。範果逃走,雲無纖介。先是,與夏侯惇戰於博望,生獲夏侯蘭。蘭是雲鄉里人,少小相知,雲白先主活之,薦蘭明於法律,以為軍正。雲不用自近,其慎慮類如此。先主入益州,雲領留營司馬。此時先主孫夫人以權妹驕豪,多將吳吏兵,縱橫不法。先主以雲嚴重,必能整齊,特任掌內事。權聞備西征,大遣舟船迎妹,而夫人內欲將後主還吳,雲與張飛勒兵截江,乃得後主還。}
 
 
先主自葭萌還攻劉璋,召諸葛亮。亮率雲與張飛等俱泝江西上,平定郡縣。至江州,分遣雲從外水上江陽,與亮會于成都。成都旣定,以雲為翊軍將軍。
 \gezhu{雲別傳曰:益州旣定,時議欲以成都中屋舍及城外園地桑田分賜諸將。雲駮之曰:「霍去病以匈奴未滅,無用家為,令國賊非但匈奴,未可求安也。須天下都定,各反桑梓,歸耕本土,乃其宜耳。益州人民,初罹兵革,田宅皆可歸還,今安居復業,然後可役調,得其歡心。」先主即從之。夏侯淵敗,曹公爭漢中地,運米北山下,數千萬囊。黃忠以為可取,雲兵隨忠取米。忠過期不還,雲將數十騎輕行出圍,迎視忠等。值曹公揚兵大出,雲為公前鋒所擊,方戰,其大衆至,勢逼,遂前突其陣,且鬬且却。公軍散,已復合,雲陷敵,還趣圍。將張著被創,雲復馳馬還營迎著。公軍追至圍,此時沔陽長張翼在雲圍內,翼欲閉門拒守,而雲入營,更大開門,偃旗息鼓。公軍疑雲有伏兵,引去。雲雷鼓震天,惟以戎弩於後射公軍,公軍驚駭,自相蹂踐,墮漢水中死者甚多。先主明旦自來至雲營圍視昨戰處,曰:「子龍一身都是膽也。」作樂飲宴至暝,軍中號雲為虎威將軍。孫權襲荊州,先主大怒,欲討權。雲諫曰:「國賊是曹操,非孫權也,且先滅魏,則吳自服。操身雖斃,子丕篡盜,當因衆心,早圖關中,居河、渭上流以討凶逆,關東義士必裹糧策馬以迎王師。不應置魏,先與吳戰;兵勢一交,不得卒解也。」先主不聽,遂東征,留雲督江州。先主失利於秭歸,雲進兵至永安,吳軍已退。}
 建興元年,為中護軍、征南將軍,封永昌亭侯,遷鎮東將軍。五年,隨諸葛亮駐漢中。明年,亮出軍,揚聲由斜谷道,曹真遣大衆當之。亮令雲與鄧芝往拒,而身攻祁山。雲、芝兵弱敵彊,失利於箕谷,然歛衆固守,不至大敗。軍退,貶為鎮軍將軍。
 \gezhu{雲別傳曰:亮曰:「街亭軍退,兵將不復相錄,箕谷軍退,兵將初不相失,何故?」芝荅曰:「雲身自斷後,軍資什物略無所棄,兵將無緣相失。」雲有軍資餘絹,亮使分賜將士,雲曰:「軍事無利,何為有賜?其物請悉入赤岸府庫,須十月為冬賜。」亮大善之。}
 
 
 
 
 七年卒,追謚順平侯。
 
 
初,先主時,惟法正見謚;後主時,諸葛亮功德蓋世,蔣琬、費禕荷國之重,亦見謚;陳祗寵待,特加殊獎,夏侯霸遠來歸國,故復得謚;於是關羽、張飛、馬超、龐統、黃忠及雲乃追謚,時論以為榮。
 \gezhu{雲別傳載後主詔曰:「雲昔從先帝,功績旣著。朕以幼沖,涉塗艱難,賴恃忠順,濟於危險。夫謚所以叙元勳也,外議雲宜謚。」大將軍姜維等議,以為雲昔從先帝,勞績旣著,經營天下,遵奉法度,功效可書。當陽之役,義貫金石,忠以衞上,君念其賞,禮以厚下,臣忘其死。死者有知,足以不溺;生者感恩,足以殞身。謹按謚法,柔賢慈惠曰順,執事有班曰平,克定禍亂曰平,應謚雲曰順平侯。}
 雲子統嗣,官至虎賁中郎,督行領軍。次子廣,牙門將,隨姜維沓中,臨陣戰死。
 
 
 
 
 評曰:關羽、張飛皆稱萬人之敵,為世虎臣。羽報效曹公,飛義釋嚴顏,並有國士之風。然羽剛而自矜,飛暴而無恩,以短取敗,理數之常也。馬超阻戎負勇,以覆其族,惜哉!能因窮致泰,不猶愈乎!黃忠、趙雲彊摯壯猛,並作爪牙,其灌、滕之徒歟?
 
 
\end{pinyinscope}