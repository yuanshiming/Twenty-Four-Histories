\article{蕭懷王熊傳}
\begin{pinyinscope}
 
 
 蕭懷王熊,早薨。黃初二年追封謚蕭懷公。太和三年,又追封爵為王。青龍二年,子哀王炳嗣,食邑二千五百戶。六年薨,無子,國除。
 
 
 
 
 評曰:任城武藝壯猛,有將領之氣。陳思文才富豔,足以自通後葉,然不能克讓遠防,終致攜隙。傳曰「楚則失之矣。而齊亦未為得也」,其此之謂歟!
 
 
\gezhu{魚豢曰:諺言「貧不學儉,卑不學恭」,非人性分也,勢使然耳。此實然之勢,信不虛矣。假令太祖防遏植等,在於疇昔,此賢之心,何緣有窺望乎?彰之挾恨,尚無所至。至於植者,豈能興難?乃令楊脩以倚注遇害,丁儀以希意族滅,哀夫!余每覽植之華采,思若有神。以此推之,太祖之動心,亦良有以也。}
 
 
\end{pinyinscope}