\article{孫韶傳}
\begin{pinyinscope}
 
 
 孫韶字公禮。伯父河,字伯海,本姓俞氏,亦吳人也。孫策愛之,賜姓為孫,列之屬籍。
 
 
\gezhu{吳書曰:河,堅族子也,出後姑俞氏,後復姓為孫。河質性忠直,訥言敏行,有氣幹,能服勤。少從堅征討,常為前驅,後領左右兵,典知內事,待以腹心之任。又從策平定吳、會,從權討李術,術破,拜威寇中郎將,領廬江太守。}
 後為將軍,屯京城。
 
 
初,孫權殺吳郡太守盛憲,
 \gezhu{會稽典錄曰:憲字孝章,器量雅偉,舉孝廉,補尚書郎,稍遷吳郡太守,以疾去官。孫策平定吳、會,誅其英豪,憲素有高名,策深忌之。初,憲與少府孔融善,融憂其不免禍,乃與曹公書曰:「歲月不居,時節如流,五十之年,忽焉已至。公為始滿,融又過二,海內知識,零落殆盡,惟會稽盛孝章尚存。其人困於孫氏,妻孥湮沒,單孑獨立,孤危愁苦,若使憂能傷人,此子不得復永年矣。春秋傳曰:『諸侯有相滅亡者,桓公不能救,則桓公恥之。』今孝章實丈夫之雄也,天下譚士依以揚聲,而身不免於幽執,命不期於旦夕,是吾祖不當復論損益之友,而朱穆所以絕交也。公誠能馳一介之使,加咫尺之書,則孝章可致,友道可弘也。今之少年,喜謗前輩,或能譏平皮柄反。孝章;孝章要為有天下大名,九牧之民所共稱歎。燕君市駿馬之骨,非欲以騁道里,乃當以招絕足也。惟公匡復漢室,宗社將絕,又能正之,正之之術,實須得賢。珠玉無脛而自至者,以人好之也,況賢者之有足乎?昭王築臺以尊郭隗,隗雖小才,而逢大遇,竟能發明主之至心,故樂毅自魏往,劇辛自趙往,鄒衍自齊往。嚮使郭隗倒縣而王不解,臨溺而王不拯,則士亦將高翔遠引,莫有北首燕路者矣。凡所稱引,自公所知,而有云者,欲公崇篤斯義也,因表不悉。」由是徵為騎都尉。制命未至,果為權所害。子匡奔魏,位至征東司馬。}
 憲故孝廉媯覽、戴員亡匿山中,孫翊為丹楊,皆禮致之。覽為大都督督兵,員為郡丞。及翊遇害,河馳赴宛陵,責怒覽、員,以不能全權,令使姧變得施。二人議曰:「伯海與將軍踈遠,而責我乃耳。討虜若來,吾屬無遺矣。」遂殺河,使人北迎揚州刺史劉馥,令住歷陽,以丹楊應之。會翊帳下徐元、孫高、傅嬰等殺覽、員。
 \gezhu{吳歷曰:媯覽、戴員親近邊洪等,數為翊所困,常欲叛逆,因吳主出征,遂其姧計。時諸縣令長並會見翊,翊以妻徐氏頗曉卜,翊入語徐:「吾明日欲為長吏作主人,卿試卜之。」徐言:「卦不能佳,可須異日。」翊以長吏來久,宜速遣,乃大請賔客。翊出入常持刀,爾時有酒色,空手送客,洪從後斫翊,郡中擾亂,無救翊者,遂為洪所殺,迸走入山。徐氏購募追捕,中宿乃得,覽、員歸罪殺洪。諸將皆知覽、員所為,而力不能討。覽入居軍府中,悉取翊嬪妾及左右侍御,欲復取徐。恐逆之見害,乃紿之曰:「乞須晦日設祭除服。」時月垂竟,覽聽須祭畢。徐潛使所親信語翊親近舊將孫高、傅嬰等,說:「覽已虜略婢妾,今又欲見偪,所以外許之者,欲安其意以免禍耳。欲立微計,願二君哀救。」高、嬰涕泣荅言:「受府君恩遇,所以不即死難者,以死無益,欲思惟事計,事計未立,未敢啟夫人耳。今日之事,實夙夜所懷也。」乃密呼翊時侍養者二十餘人,以徐意語之,共盟誓,合謀。到晦日,設祭,徐氏哭泣盡哀畢,乃除服,薰香沐浴,更於他室,安施幃帳,言笑歡恱,示無戚容。大小悽愴,怪其如此。覽密覘視,無復疑意。徐呼高、嬰與諸婢羅住戶內,使人報覽,說已除凶即吉,惟府君勑命。覽盛意入,徐出戶拜。覽適得一拜,徐便大呼:「二君可起!」高、嬰俱出,共得殺覽,餘人即就外殺員。夫人乃還縗絰,奉覽、員首以祭翊墓。舉軍震駭,以為神異。吳主續至,悉族誅覽、員餘黨,擢高、嬰為牙門,其餘皆加賜金帛,殊其門戶。}
 
 
韶年十七,收河餘衆,繕治京城,起樓櫓,脩器備以禦敵。權聞亂,從椒丘還,過定丹楊,引軍歸吳。夜至京城下營,試攻驚之,兵皆乘城傳檄備警,讙聲動地,頗射外人,權使曉喻乃止。明日見韶,甚器之,即拜承烈校尉,統河部曲,食曲阿、丹徒二縣,自置長吏,一如河舊。後為廣陵太守、偏將軍。權為吳王,遷揚威將軍,封建德侯。權稱尊號,為鎮北將軍。韶為邊將數十年,善養士卒,得其死力。常以警疆埸遠斥候為務,先知動靜而為之備,故鮮有負敗。青、徐、汝、沛頗來歸附,淮南濵江屯候皆撤兵遠徙,徐、泗、江、淮之地,不居者各數百里。自權西征,還都武昌,韶不進見者十餘年。權還建業,乃得朝覲。權問青、徐諸屯要害,遠近人馬衆寡,魏將帥姓名,盡具識之,所問咸對。身長八尺,儀貌都雅。權歡恱曰:「吾乆不見公禮,不圖進益乃爾。」加領幽州牧、假節。赤烏四年卒。子越嗣,至右將軍。越兄楷武衞大將軍、臨成侯,代越為京下督。楷弟異至領軍將軍,弈宗正卿,恢武陵太守。天璽元年,徵楷為宮下鎮驃騎將軍。初永安賊施但等劫皓弟謙,襲建業,或白楷二端不即赴討者,皓數遣詰楷。楷常惶怖,而卒被召,遂將妻子親兵數百人歸晉,晉以為車騎將軍,封丹楊侯。
 \gezhu{晉諸公贊曰:吳平,降為渡遼將軍,永安元年卒。吳錄曰:楷處事嚴整不如孫秀,而人聞知名,過也。}
 
 
\end{pinyinscope}