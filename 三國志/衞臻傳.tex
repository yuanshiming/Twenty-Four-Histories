\article{衞臻傳}
\begin{pinyinscope}
 
 
 衞臻字公振,陳留襄邑人也。父茲,有大節,不應三公之辟。太祖之初至陳留,茲曰:「平天下者,必此人也。」太祖亦異之,數詣茲議大事。從討董卓,戰于熒陽而卒。太祖每涉郡境,輒遣使祠焉。
 
 
\gezhu{先賢行狀曰:茲字子許。不為激詭之行,不徇流俗之名;明慮淵深,規略宏遠。為車騎將軍何苗所辟,司徒楊彪再加旌命。董卓作亂,漢室傾蕩,太祖到陳留,始與茲相見,遂同盟,計興武事。茲荅曰:「亂生乆矣,非兵無以整之。」且言「兵之興者,自今始矣」。深見廢興,首讚弘謀。合兵三千人,從太祖入熒陽,力戰終日,失利,身歿。郭林宗傳曰:「茲弱冠與同郡圈文生俱稱盛德。林宗與二人共至市,子許買物,隨價讎直,文生訾呵,減價乃取。林宗曰:「子許少欲,文生多情,此二人非徒兄弟,乃父子也。」後文生以穢貨見損,茲以烈節垂名。}
 夏侯惇為陳留太守,舉臻計吏,命婦出宴,臻以為「末世之俗,非禮之正」。惇怒,執臻,旣而赦之。後為漢黃門侍郎。東郡朱越謀反,引臻。太祖令曰:「孤與卿君同共舉事,加欽令問。始聞越言,固自不信。及得荀令君書,具亮忠誠。」會奉詔命,聘貴人于魏,因表留臻參丞相軍事。追錄臻父舊勳,賜爵關內侯,轉為戶曹掾。文帝即王位,為散騎常侍。及踐阼,封安國亭侯。時羣臣並頌魏德,多抑損前朝。臻獨明禪授之義,稱揚漢美。帝數目臻曰:「天下之珍,當與山陽共之。」遷尚書,轉侍中吏部尚書。帝幸廣陵,行中領軍,從。征軍大將軍曹休表得降賊辭,「孫權已在濡須口」。臻曰:「權恃長江,未敢亢衡,此必畏怖偽辭耳。」考核降者,果守將詐所作也。
 
 
 
 
 明帝即位,進封康鄉侯,後轉為右僕射,典選舉,如前加侍中。中護軍蔣濟遺臻書曰:「漢祖遇亡虜為上將,周武拔漁父為太師;布衣厮養,可登王公,何必守文,試而後用?」臻荅曰:「古人遺智慧而任度量,須考績而加黜陟;今子同牧野於成、康,喻斷蛇於文、景,好不經之舉,開拔奇之津,將使天下馳騁而起矣。」諸葛亮寇天水,臻奏:「宜遣奇兵入散關,絕其糧道。」乃以臻為征蜀將軍,假節督諸軍事,到長安,亮退。還,復職,加光祿大夫。是時,帝方隆意於殿舍,臻數切諫。及殿中監擅收蘭臺令史,臻奏案之。詔曰:「殿舍不成,吾所留心,卿推之何?」臻上疏曰:「古制侵官之法,非惡其勤事也,誠以所益者小,所墮者大也。臣每察校事,類皆如此,懼羣司將遂越職,以至陵遲矣。」亮又出斜谷;征南上:「朱然等軍已過荊城。」臻曰:「然,吳之驍將,必下從權,且為勢以綴征南耳。」權果召然入居巢,進攻合肥。帝欲自東征,臻曰:「權外示應亮,內實觀望。且合肥城固,不足為慮。車駕可無親征,以省六軍之費。」帝到尋陽而權竟退。
 
 
 
 
 幽州刺史毌丘儉上疏曰:「陛下即位已來,未有可書。吳、蜀恃險,未可卒平,聊可以此方無用之士克定遼東。」臻曰:「儉所陳皆戰國細術,非王者之事也。吳頻歲稱兵,寇亂邊境,而猶案甲養士,未果尋致討者,誠以百姓疲勞故也。且淵生長海表,相承三世,外撫戎夷,內脩戰射,而儉欲以偏軍長驅,朝至夕卷,知其妄矣。」儉行軍遂不利。
 
 
臻遷為司空,徙司徒。正始中,進爵長垣侯,邑千戶,封一子列侯。初,太祖乆不立太子,而方奇貴臨菑侯。丁儀等為之羽翼,勸臻自結,臻以大義拒之。及文帝即位,東海王霖有寵,帝問臻:「平原侯何如?」臻稱明德美而終不言。曹爽輔政,使夏侯玄宣指,欲引臻入守尚書令,及為弟求婚,皆不許。固乞遜位。詔曰:「昔干木偃息,義壓彊秦;留侯頤神,不忘楚事。讜言嘉謀,望不吝焉。」賜宅一區,位特進,秩如三司。薨,追贈太尉,謚曰敬侯。子烈嗣,咸熈中為光祿勳。
 \gezhu{臣松之案舊事及傅咸集,烈終於光祿勳。烈二弟京、楷,皆二千石。楷子權,字伯輿。晉大司馬汝南王亮輔政,以權為尚書郎。傅咸與亮牋曰:「衞伯輿貴妃兄子,誠有才章,應作臺郎,然未得東宮官屬。東宮官屬,前患楊駿,親理塞路,今有伯輿,復越某作郎。一犬吠形,羣犬吠聲,懼於羣犬,遂至回聽。」權作左思吳都賦叙及注,叙粗有文辭,至於為注,了無所發明,直為塵穢紙墨,不合傳寫也。}
 
 
\end{pinyinscope}