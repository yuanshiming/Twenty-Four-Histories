\article{王脩傳}
\begin{pinyinscope}
 
 
 王脩字叔治,北海營陵人也。年七歲喪母。母以社日亡,來歲鄰里社,脩感念母,哀甚。鄰里聞之,為之罷社。年二十,游學南陽,止張奉舍。奉舉家得疾病,無相視者,脩親隱恤之,病愈乃去。初平中,北海孔融召以為主簿,守高密令。高密孫氏素豪俠,人客數犯法。民有相劫者,賊入孫氏,吏不能執。脩將吏民圍之,孫氏拒守,吏民畏憚不敢近。脩令吏民:「敢有不攻者與同罪。」孫氏懼,乃出賊。由是豪彊懾服。舉孝廉,脩讓邴原,融不聽。
 
 
\gezhu{融集有融荅脩教曰:「原之賢也,吾已知之矣。昔高陽氏有才子八人,堯不能用,舜實舉之。原可謂不患無位之士。以遺後賢,不亦可乎!」脩重辭,融荅曰:「掾清身絜己,歷試諸難,謀而鮮過,惠訓不倦。余嘉乃勳,應乃懿德,用升爾于王庭,其可辭乎!」}
 時天下亂,遂不行。頃之,郡中有反者。脩聞融有難,夜往奔融。賊初發,融謂左右曰:「能冒難來,唯王脩耳!」言終而脩至。復署功曹。時膠東多賊寇,復令脩守膠東令。膠東人公沙盧宗彊,自為營塹,不肯應發調。脩獨將數騎徑入其門,斬盧兄弟,公沙氏驚愕莫敢動。脩撫慰其餘,由是寇少止。融每有難,脩雖休歸在家,無不至。融常賴脩以免。
 
 
袁譚在青州,辟脩為治中從事,別駕劉獻數毀短脩。後獻以事當死,脩理之,得免。時人益以此多焉。袁紹又辟脩除即墨令,後復為譚別駕。紹死,譚、尚有隙。尚攻譚,譚軍敗,脩率吏民往救譚。譚喜曰:「成吾軍者,王別駕也。」譚之敗,劉詢起兵漯陰,諸城皆應。譚歎息曰:「今舉州背叛,豈孤之不德邪!」脩曰:「東萊太守管統雖在海表,此人不反。必來。」後十餘日,統果棄其妻子來赴譚,妻子為賊所殺,譚更以統為樂安太守。譚復欲攻尚,脩諫曰:「兄弟還相攻擊,是敗亡之道也。」譚不恱,然知其忠節。後又問脩:「計安出?」脩曰:「夫兄弟者,左右手也。譬人將鬬而斷其右手,而曰『我必勝』,若是者可乎?夫棄兄弟而不親,天下其誰親之!屬有讒人,固將交鬬其間,以求一朝之利,願明使君塞耳勿聽也。若斬佞臣數人,復相親睦,以禦四方,可以橫行天下。」譚不聽,遂與尚相攻擊,請救於太祖。太祖旣破兾州,譚又叛。太祖遂引軍攻譚於南皮。脩時運糧在樂安,聞譚急,將所領兵及諸從事數十人往赴譚。至高密,聞譚死,下馬號哭曰:「無君焉歸?」遂詣太祖,乞收葬譚屍。太祖欲觀脩意,默然不應。脩復曰:「受袁氏厚恩,若得收歛譚屍,然後就戮,無所恨。」太祖嘉其義,聽之。
 \gezhu{傅子曰:太祖旣誅袁譚,梟其首,令曰:「敢哭之者戮及妻子。」於是王叔治、田子泰相謂曰:「生受辟命,亡而不哭,非義也。畏死忘義,何以立世?」遂造其首而哭之,哀動三軍。軍正白行其戮,太祖曰:「義士也。」赦之。臣松之案田疇傳,疇為袁尚所辟,不被譚命。傅子合而言之,有違事實。}
 以脩為督軍糧,還樂安。譚之破,諸城皆服,唯管統以樂安不從命。太祖命脩取統首,脩以統亡國之忠臣,因解其縛,使詣太祖。太祖恱而赦之。袁氏政寬,在職勢者多畜聚。太祖破鄴,籍沒審配等家財物貲以萬數。及破南皮,閱脩家,穀不滿十斛,有書數百卷。太祖歎曰:「士不妄有名。」乃禮辟為司空掾,行司金中郎將,遷魏郡太守。為治,抑彊扶弱,明賞罰,百姓稱之。
 \gezhu{魏略曰:脩為司金中郎將,陳黃白異議,因奏記曰:「脩聞枳棘之林,無梁柱之質;涓流之水,無洪波之勢。是以在職七年,忠讜不昭於時,功業不見於事,欣於所受,俯慙不報,未嘗不長夜起坐,中飯釋餐。何者?力少任重,不堪而懼也。謹貢所議如左。」太祖甚然之,乃與脩書曰:「君澡身浴德,流聲本州,忠能成績,為世美談,名實相副,過人甚遠。孤以心知君,至深至孰,非徒耳目而已也。察觀先賢之論,多以鹽鐵之利,足贍軍國之用。昔孤初立司金之官,念非屈君,餘無可者。故與君教曰:『昔遏父陶正,民賴其器用,及子媯滿,建侯于陳;近桑弘羊,位至三公。此君元龜之兆先告者也』,是孤用君之本言也,或恐衆人未曉此意。自是以來,在朝之士,每得一顯選,常舉君為首,及聞袁軍師衆賢之議,以為不宜越君。然孤執心將有所厎,以軍師之職,閑於司金,至於建功,重於軍師。孤之精誠,足以達君;君之察孤,足以不疑。但恐傍人淺見,以蠡測海,為蛇畫足,將言前後百選,輙不用之,而使此君沈滯冶官。張甲李乙,尚猶先之,此主人意待之不優之效也。孤懼有此空聲冒實,淫鼃亂耳。假有斯事,亦庶鍾期不失聽也;若其無也,過備何害?昔宣帝察少府蕭望之才任宰相,故復出之,令為馮翊。從正卿往,似於左遷。上使侍中宣意曰:『君守平原日淺,故復試君三輔,非有所閒也。』孤揆先主中宗之意,誠備此事。旣君崇勳業以副孤意。公叔文子與臣俱升,獨何人哉!」後無幾而遷魏郡太守。}
 
 
魏國旣建,為大司農郎中令。太祖議行肉刑,脩以為時未可行,太祖採其議。徙為奉常。其後嚴才反,與其徒屬數十人攻掖門。脩聞變,召車馬未至,便將官屬步至宮門。太祖在銅爵臺望見之,曰:「彼來者必王叔治也。」相國鍾繇謂脩:「舊,京城有變,九卿各居其府。」脩曰:「食其祿,焉避其離?居府雖舊,非赴難之義。」頃之,病卒官。子忠,官至東萊太守、散騎常侍。初,脩識高柔於弱冠,異王基於幼童,終皆遠至,世稱其知人。
 \gezhu{王隱晉書曰:脩一子,名儀,字朱表,高亮雅直。司馬文王為安東,儀為司馬。東關之敗,文王曰:「近日之事,誰任其咎?」儀曰:「責在軍師。」文王怒曰:「司馬欲委罪於孤邪?」遂殺之。子襃,字偉元。少立操尚,非禮不動。身長八尺四寸,容貌絕異。痛父不以命終,絕世不仕。立屋墓側,以教授為務。旦夕常至墓前拜,輙悲號斷絕。墓前有一柏樹,襃常所攀援,涕泣所著,樹色與凡樹不同。讀詩至「哀哀父母,生我勞悴」,未曾不反覆流涕,泣下沾衿。家貧躬耕,計口而田,度身而蠶。諸生有密為襃刈麥者,襃遂棄之;自是莫敢復佐刈者。襃門人為本縣所役,求襃為屬,襃曰:「卿學不足以庇身,吾德薄不足以蔭卿,屬之何益?且吾不捉筆已四十年。」乃步擔乾飯,兒負鹽豉,門徒從者千餘人。安丘令以為見己,整衣出迎之於門。襃乃下道至土牛,磬折而立。云:「門生為縣所役,故來送別。」執手涕泣而去。令即放遣諸生,一縣以為恥。同縣管彥,少有才力,未知名,襃獨以為當自達,常友愛之;男女各始生,共許為婚。彥果為西夷校尉。襃後更以女嫁人,彥弟馥問襃,襃曰:「吾薄志畢願,山藪自處,姊妹皆遠,吉凶斷絕,以此自誓。賢兄子葬父於帝都,此則洛陽之人也,豈吾欲婚之本指邪?」馥曰:「嫂,齊人也。當還臨菑。」襃曰:「安有葬父河南,隨妻還齊!用意如此,何婚之有?」遂不婚。邴春者,根矩之後也。少立志操,寒苦自居,負笈遊學,身不停家,鄉邑翕然,以為能係其先也。襃以為春性險狹,慕名意多,終必不成,及後春果無學業,流離遠外,有識以此歸之。襃常以為人所行,其當歸於善道,不可以己所能而責人所不能也。有致遺者,皆不受。及洛都傾覆,寇賊蠭起,襃宗親悉欲移江東,襃戀墳壠。賊大盛,乃南達泰山郡。襃思土不肯去,賊害之。漢晉春秋曰:襃與濟南劉兆字延世,俱以不仕顯名。襃以父為文王所濫殺,終身不應徵聘,未甞西向坐,以示不臣於晉也。魏略純固傳以脂習、王脩、龐淯、文聘、成公英、郭憲、單固七人為一傳。其脩、淯、聘三人自各有傳,成公英別見張旣傳,單固見王淩傳,餘習、憲二人列於脩傳後也。脂習字元升,京兆人也。中平中仕郡,公府辟,舉高第,除太醫令。天子西遷及東詣許昌,習常隨從。與少府孔融親善。太祖為司空,威德日盛,而融故以舊意,書疏倨傲。習常責融,欲令改節,融不從。會融被誅,當時許中百官先與融親善者,莫敢收恤,而習獨往撫而哭之曰:「文舉,卿捨我死,我當復與誰語者?」哀歎無已。太祖聞之,收習,欲理之,尋以其事直見原,徙許東土橋下。習後見太祖,陳謝前愆。太祖呼其字曰:「元升,卿故慷慨!」因問其居處,以新移徙,賜穀百斛。至黃初,詔欲用之,以其年老,然嘉其敦舊,有欒布之節,賜拜中散大夫。還家,年八十餘卒。郭憲字幼簡,西平人,為其郡右姓。建安中為郡功曹,州辟不就,以仁篤為一郡所歸。至十七年,韓約失衆,從羌中還,依憲。衆人多欲取約以徼功,而憲皆責怒之,言:「人窮來歸我,云何欲危之?」遂擁護厚遇之。其後約病死,而田樂、陽逵等就斬約頭,當送之。逵等欲條疏憲名,憲不肯在名中,言我尚不忍生圖之,豈忍取死人以要功乎?逵等乃止。時太祖方攻漢中,在武都,而逵等送約首到。太祖宿聞憲名,及視條疏,怪不在中,以問逵等,逵具以情對。太祖歎其志義,乃并表列與逵等並賜爵關內侯,由是名震隴右。黃初元年病亡。正始初,國家追嘉其事,復賜其子爵關內侯。}
 
 
\end{pinyinscope}