\article{辰韓傳}
\begin{pinyinscope}
 
 
 辰韓在馬韓之東,其耆老傳世,自言古之亡人避秦役來適韓國,馬韓割其東界地與之。有城柵。其言語不與馬韓同,名國為邦,弓為弧,賊為寇,行酒為行觴。相呼皆為徒,有似秦人,非但燕、齊之名物也。名樂浪人為阿殘;東方人名我為阿,謂樂浪人本其殘餘人。今有名之為秦韓者。始有六國,稍分為十二國。
 
 
\end{pinyinscope}