\article{黃權傳}
\begin{pinyinscope}
 
 
 黃權字公衡,巴西閬中人也。少為郡吏,州牧劉璋召為主簿。時別駕張松建議,宜迎先主,使伐張魯。權諫曰:「左將軍有驍名,今請到,欲以部曲遇之,則不滿其心,欲以賔客禮待,則一國不容二君。若客有泰山之安,則主有累卵之危。可但閉境,以待河清。」璋不聽,竟遣使迎先主,出權為廣漢長。及先主襲取益州,將帥分下郡縣,郡縣望風景附,權閉城堅守,須劉璋稽服,乃詣降先主。先主假權偏將軍。
 
 
\gezhu{徐衆評曰:權旣忠諫於主,又閉城拒守,得事君之禮。武王下車,封比干之墓,表商容之閭,所以大顯忠賢之士,而明示所貴之旨。先主假權將軍,善矣,然猶薄少,未足彰忠義之高節,而大勸為善者之心。}
 及曹公破張魯,魯走入巴中,權進曰:「若失漢中,則三巴不振,此為割蜀之股臂也。」於是先主以權為護軍,率諸將迎魯。魯已還南鄭,北降曹公,然卒破杜濩、朴胡,殺夏侯淵,據漢中,皆權本謀也。
 
 
先主為漢中王,猶領益州牧,以權為治中從事。及稱尊號,將東伐吳,權諫曰:「吳人悍戰,又水軍順流,進易退難,臣請為先驅以甞寇,陛下宜為後鎮。」先主不從,以權為鎮北將軍,督江北軍以防魏師;先主自在江南。及吳將軍陸議乘流斷圍,南軍敗績,先主引退。而道隔絕,權不得還,故率將所領降于魏。有司執法,白收權妻子。先主曰:「孤負黃權,權不負孤也。」待之如初。
 \gezhu{臣松之以為漢武用虛罔之言,滅李陵之家,劉主拒憲司所執,宥黃權之室,二主得失縣邈遠矣。詩云「樂只君子,保乂爾後」,其劉主之謂也。}
 
 
魏文帝謂權曰:「君捨逆効順,欲追蹤陳、韓邪?」權對曰:「臣過受劉主殊遇,降吳不可,還蜀無路,是以歸命。且敗軍之將,免死為幸,何古人之可慕也!」文帝善之,拜為鎮南將軍,封育陽侯,加侍中,使之陪乘。蜀降人或云誅權妻子,權知其虛言,未便發喪,
 \gezhu{漢魏春秋曰:文帝詔令發喪,權荅曰:「臣與劉、葛推誠相信,明臣本志。疑惑未實,請須後問。」}
 後得審問,果如所言。及先主薨問至,魏羣臣咸賀而權獨否。文帝察權有局量,欲試驚之,遣左右詔權,未至之間,累催相屬,馬使奔馳,交錯於道,官屬侍從莫不碎魄,而權舉止顏色自若。後領益州刺史,徙占河南。大將軍司馬宣王深器之,問權曰:「蜀中有卿輩幾人?」權笑而荅曰:「不圖明公見顧之重也!」宣王與諸葛亮書曰:「黃公衡,快士也,每坐起歎述足下,不去口實。」景初三年,蜀延熈二年,權遷車騎將軍、儀同三司。
 \gezhu{蜀記曰:魏明帝問權:「天下鼎立,當以何地為正?」權對曰:「當以天文為正。往者熒惑守心而文皇帝崩,吳、蜀二主平安,此其徵也。」}
 明年卒,謚曰景侯。子邕嗣。邕無子,絕。
 
 
 
 
 權留蜀子崇,為尚書郎,隨衞將軍諸葛瞻拒鄧艾。到涪縣,瞻盤桓未進,崇屢勸瞻宜速行據險,無令敵得入平地。瞻猶與未納,崇至于流涕。會艾長驅而前,瞻却戰至緜竹,崇帥厲軍士,期於必死,臨陣見殺。
 
 
\end{pinyinscope}