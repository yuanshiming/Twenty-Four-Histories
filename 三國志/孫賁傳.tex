\article{孫賁傳}
\begin{pinyinscope}
 
 
 孫賁字伯陽。父羌字聖壹,堅同產兄也。賁早失二親,弟輔嬰孩,賁自贍育,友愛甚篤。為郡督郵守長。堅於長沙舉義兵,賁去吏從征伐。堅薨,賁攝帥餘衆,扶送靈柩。後袁術徙壽春,賁又依之。術從兄紹用會稽周昂為九江太守,紹與術不恊,術遣賁攻破昂於陰陵。術表賁領豫州刺史,轉丹楊都尉,行征虜將軍,討平山越。為揚州刺史劉繇所迫逐,因將士衆還住歷陽。頃之,術復使賁與吳景共擊樊能、張英等,未能拔。及策東渡,助賁、景破英、能等,遂進擊劉繇。繇走豫章。策遣賁、景還壽春報術,值術僭號,署置百官,除賁九江太守。賁不就,棄妻孥還江南。
 
 
\gezhu{江表傳曰:袁術以吳景守廣陵,策族兄香亦為術所用,作汝南太守,而令賁為將軍,領兵在壽春。策與景等書曰:「今征江東,未知二三君意云何耳?」景即棄守歸,賁困而後免,香以道遠獨不得還。吳書曰:香字文陽。父孺,字仲孺,堅再從弟也,仕郡主簿功曹。香從堅征伐有功,拜郎中。後為袁術驅馳,加征南將軍,死於壽春。}
 時策已平吳、會二郡,賁與策征廬江太守劉勳、江夏太守黃祖,軍旋,聞繇病死,過定豫章,上賁領太守,
 \gezhu{江表傳曰:時丹楊僮芝自署廬陵太守,策留賁弟輔領兵住南昌,策謂賁曰:「兄今據豫章,是扼僮芝咽喉而守其門戶矣。但當伺其形便,因令國儀杖兵而進,使公瑾為作勢援,一舉可定也。」後賁聞芝病,即如策計。周瑜到巴丘,輔遂得進據廬陵。}
 後封都亭侯。建安十三年,使者劉隱奉詔拜賁為征虜將軍,領郡如故。在官十一年卒。子鄰嗣。
 
 
鄰年九歲,代領豫章,進封都鄉侯。
 \gezhu{吳書曰:鄰字公達,雅性精敏,幼有令譽。}
 在郡垂二十年,討平叛賊,功績脩理。召還武昌,為繞帳督。時太常潘濬掌荊州事,重安長陳留舒燮有罪下獄,濬嘗失燮,欲致之於法。論者多為有言,濬猶不釋。鄰謂濬曰:「舒伯膺兄弟爭死,海內義之,以為美譚,仲膺又有奉國舊意。今君殺其子弟,若天下一統,青蓋北巡,中州士人必問仲膺繼嗣,荅者云潘承明殺燮,於事何如?」濬意即解,燮用得濟。
 \gezhu{博物志曰:仲膺名邵。初,伯膺親友為人所殺,仲膺為報怨。事覺,兄弟爭死,皆得免。袁術時,邵為阜陵長。亦見江表傳。}
 鄰遷夏口沔中督、威遠將軍,所居任職。赤烏十二年卒。子苗嗣。苗弟旅及叔父安、熈、績,皆歷列位。
 \gezhu{吳歷曰:鄰又有子曰述,為武昌督,平荊州事。震,無難督。諧,城門校尉。歆,樂鄉督。震後禦晉軍,與張悌俱死。賁曾孫惠,字德施。惠別傳曰:惠好學有才智,晉永寧元年,赴齊王冏義,以功封晉興侯,辟大司馬賊曹屬。冏驕矜僭侈,天下失望。惠獻言於冏,諷以五難、四不可,勸令委讓萬機,歸藩青岱,辭甚深切。冏不能納,頃之果敗。成都王穎召為大將軍參軍。是時穎將有事於長沙,以陸機為前鋒都督。惠與機鄉里親厚,憂其致禍,謂之曰:「子盍讓都督於王粹乎?」機曰:「將謂吾避賊首鼠,更速其害。」機尋被戮,二弟雲、耽亦見殺,惠甚傷恨之。永興元年,乘輿幸鄴,司空東海王越治兵下邳,惠以書干越,詭其姓名,自稱南岳逸民秦祕之,勉以勤王匡世之略,辭義甚美。越省其書,牓題道衢,招求其人。惠乃出見,越即以為記室參軍,專掌文疏,豫參謀議。每造書檄,越或驛馬催之,應命立成,皆有辭旨。累遷顯職,後為廣武將軍、安豐內史。年四十七卒。惠文翰凡數十首。}
 
 
\end{pinyinscope}