\article{公孫瓚傳}
\begin{pinyinscope}
 
 
 公孫瓚字伯珪,遼西令支人也。
 
 
\gezhu{令音郎定反。支音其兒反。}
 為郡門下書佐。有姿儀,大音聲,侯太守器之,以女妻焉,
 \gezhu{典略曰:瓚性辯慧,每白事不肯梢入,常總說數曹事,無有忘誤,太守奇其才。}
 適詣涿郡盧植讀經。後復為郡吏。劉太守坐事徵詣廷尉,瓚為御車,身執徒養。及劉徙日南,瓚具米肉,於北芒上祭先人,舉觴祝曰:「昔為人子,今為人臣,當詣日南。日南鄣氣,或恐不還,與先人辭於此。」再拜慷慨而起,時見者莫不歔欷。劉道得赦還。瓚以孝廉為郎,除遼東屬國長史。甞從數十騎出行塞,見鮮卑數百騎,瓚乃退入空亭中,約其從騎曰:「今不衝之,則死盡矣。」瓚乃自持矛,兩頭施刃,馳出刺胡,殺傷數十人,亦亡其從騎半,遂得免。鮮卑懲艾,後不敢復入塞。遷為涿令。光和中,涼州賊起,發幽州突騎三千人,假瓚都督行事傳,使將之。軍到薊中,漁陽張純誘遼西烏丸丘力居等叛,劫略薊中,自號將軍,
 \gezhu{九州春秋曰:純自號彌天將軍、安定王。}
 略吏民攻右北平、遼西屬國諸城,所至殘破。瓚將所領,追討純等有功,遷騎都尉。屬國烏丸貪至王率種人詣瓚降。遷中郎將,封都亭侯,進屯屬國,與胡相攻擊五六年。丘力居等鈔略青、徐、幽、兾,四州被其害,瓚不能禦。
 
 
朝議以宗正東海劉伯安旣有德義,昔為幽州刺史,恩信流著,戎狄附之,若使鎮撫,可不勞衆而定,乃以劉虞為幽州牧。
 \gezhu{吳書曰:虞,東海恭王之後也。遭世衰亂,又與時主疏遠,仕縣為戶曹吏。以能治身奉職,召為郡吏,以孝廉為郎,累遷至幽州刺史,轉甘陵相,甚得東土戎狄之心。後以疾歸家,常降身隱約,與邑黨州閭同樂共卹,等齊有無,不以名位自殊,鄉曲咸共宗之。時鄉曲有所訴訟,不以詣吏,自投虞平之;虞以情理為之論判,皆大小敬從,不以為恨。嘗有失牛者,骨體毛色與虞牛相似,因以為是,虞便推與之;後主自得本牛,乃還謝罪。會甘陵復亂,吏民思虞治行,復以為甘陵相,甘陵大治。徵拜尚書令、光祿勳,以公族有禮,更為宗正。英雄記曰:虞為博平令,治正推平,高尚純樸,境內無盜賊,災害不生。時隣縣接壤,蝗蟲為害,至博平界,飛過不入。魏書曰:虞在幽州,清靜儉約,以禮義化民。靈帝時,南宮災,吏遷補州郡者,皆責助治宮錢,或一千萬,或二千萬,富者以私財辨,或發民錢以備之,貧而清慎者無以充調,或至自殺。靈帝以虞清貧,特不使出錢。}
 虞到,遣使至胡中,告以利害,責使送純首。丘力居等聞虞至,喜,各遣譯自歸。瓚害虞有功,乃陰使人徼殺胡使。胡知其情,間行詣虞。虞上罷諸屯兵,但留瓚將步騎萬人屯右北平。純乃棄妻子,逃入鮮卑,為其客王政所殺,送首詣虞。封政為列侯。虞以功即拜太尉,封襄賁侯。
 \gezhu{英雄記曰:虞讓太尉,因薦衞尉趙謨、益州牧劉焉、豫州牧黃琬、南陽太守羊續,並任為公。}
 會董卓至洛陽,遷虞大司馬,瓚奮武將軍,封薊侯。
 
 
關東義兵起,卓遂劫帝西遷,徵虞為太傅,道路隔塞,信命不得至。袁紹、韓馥議,以為少帝制於姦臣,天下無所歸心。虞,宗室知名,民之望也,遂推虞為帝。遣使詣虞,虞終不肯受。紹等復勸虞領尚書事,承制封拜,虞又不聽,然猶與紹等連和。
 \gezhu{九州春秋曰:紹、馥使故樂浪太守甘陵張岐齎議詣虞,使即尊號。虞厲聲呵岐曰:「卿敢出此言乎!忠孝之道,旣不能濟。孤受國恩,天下擾亂,未能竭命以除國恥,望諸州郡烈義之士勠力西靣,援迎幼主,而乃妄造逆謀,欲塗汚忠臣邪!」吳書曰:馥以書與袁術,云帝非孝靈子,欲依絳、灌誅廢少主,迎立代王故事;稱虞功德治行,華夏少二,當今公室枝屬皆莫能及。又云:「昔光武去定王五世,以大司馬領河北,耿弇、馮異勸即尊號,卒代更始。今劉公自恭王枝別,其數亦五,以大司馬領幽州牧,此其與光武同。」是時有四星會于箕尾,馥稱讖云神人將在燕分。又言濟陰男子王定得玉印,文曰「虞為天子」。又見兩日出於代郡,謂虞當代立。紹又別書報術。是時術陰有不臣之心,不利國家有長主,外託公義以荅拒之。紹亦使人私報虞,虞以國有正統,非人臣所宜言,固辭不許;乃欲圖奔匈奴以自絕,紹等乃止。虞於是奉修職貢,愈益恭肅;諸外國羌、胡有所貢獻,道路不通,皆為傳送,致之京師。}
 虞子和為侍中,在長安。天子思東歸,使和偽逃卓,潛出武關詣虞,令將兵來迎。和道經袁術,為說天子意。術利虞為援,留和不遣,許兵至俱西,令和為書與虞。虞得和書,乃遣數千騎詣和。瓚知術有異志,不欲遣兵,止虞,虞不可。瓚懼術聞而怨之,亦遣其從弟越將千騎詣術以自結,而陰教術執和,奪其兵。由是虞、瓚益有隙。和逃術來北,復為紹所留。
 
 
是時,術遣孫堅屯陽城拒卓,紹使周昂奪其處。術遣越與堅攻昂,不勝,越為流矢所中死。瓚怒曰:「余弟死,禍起於紹。」遂出軍屯磐河,將以報紹。紹懼,以所佩勃海太守印綬授瓚從弟範,遣之郡,欲以結援。範遂以勃海兵助瓚,破青、徐黃巾,兵益盛;進軍界橋。
 \gezhu{典略載瓚表紹罪狀曰:「臣聞皇、羲以來,始有君臣上下之事,張化以導民,刑罰以禁暴。今行車騎將軍袁紹,託其先軌,寇竊人爵,旣性暴亂,厥行淫穢。昔為司隷校尉,會值國家喪禍之際,太后承攝,何氏輔政,紹專為邪媚,不能舉直,至令丁原焚燒孟津,招來董卓,造為亂根,紹罪一也。卓既入雒而主見質,紹不能權譎以濟君父,而棄置節傳,迸竄逃亡,忝辱爵命,背上不忠,紹罪二也。紹為勃海太守,默選戎馬,當攻董卓,不告父兄,至使太傅門戶、太僕母子一旦而斃,不仁不孝,紹罪三也。紹旣興兵,涉歷二年,不卹國難,廣自封殖,乃多以資糧專為不急,割剥富室,收考責錢,百姓吁嗟,莫不痛怨,紹罪四也。韓馥之迫,竊其虛位,矯命詔恩,刻金印玉璽,每下文書,皁囊施檢,文曰『詔書一封,邟口浪反鄉侯印』。昔新室之亂,漸以即真,今紹所施,擬而方之,紹罪五也。紹令崔巨業候視星日,財貨賂遺,與共飲食,克期會合,攻鈔郡縣,此豈大臣所當宜為?紹罪六也。紹與故虎牙都尉劉勳首共造兵,勳仍有効,又降伏張楊,而以小忿枉害於勳,信用讒慝,殺害有功,紹罪七也。紹又上故上谷太守高焉、故甘陵相姚貢,橫責其錢,錢不備畢,二人并命,紹罪八也。春秋之義,子以母貴。紹母親為婢使,紹實微賤,不可以為人後,以義不宜,乃據豐隆之重任,忝汚王爵,損辱袁宗,紹罪九也。又長沙太守孫堅,前領豫州刺史,驅走董卓,掃除陵廟,其功莫大;紹令周昂盜居其位,斷絕堅糧,令不得入,使卓不被誅,紹罪十也。臣又每得後將軍袁術書,云紹非術類也。紹之罪戾,雖南山之竹不能載。昔姬周政弱,王道陵遲,天子遷都,諸侯背叛,於是齊桓立柯亭之盟,晉文為踐土之會,伐荊楚以致菁茅,誅曹、衞以彰無禮。臣雖闒茸,名非先賢,蒙被朝恩,當此重任,職在鈇鉞,奉辭伐罪,輒與諸將州郡兵討紹等。若事克捷,罪人斯得,庶續桓、文忠誠之効,攻戰形狀,前後續上。」遂舉兵與紹對戰,紹不勝。}
 以嚴綱為兾州,田楷為青州,單經為兖州,置諸郡縣。紹軍廣川,令將麴義先登與瓚戰,生禽綱。瓚軍敗走勃海,與範俱還薊,於大城東南築小城,與虞相近,稍相恨望。
 
 
虞懼瓚為變,遂舉兵襲瓚。虞為瓚所敗,出奔居庸。瓚攻拔居庸,生獲虞,執虞還薊。會卓死,天子遣使者段訓增虞邑,督六州;瓚遷前將軍,封易侯。瓚誣虞欲稱尊號,脅訓斬虞。
 \gezhu{魏氏春秋曰:初,劉虞和輯戎狄,瓚以胡夷難禦,當因不賔而討之,今加財賞,必益輕漢,效一時之名,非乆長深慮。故虞所賞賜,瓚輒鈔奪。虞數請會,稱疾不往。至是戰敗,虞欲討之,告東曹掾右北平人魏攸。攸曰:「今天下引領,以公為歸,謀臣爪牙,不可無也。瓚,文武才力足恃,雖有小惡,固宜容忍。」乃止。後一年,攸病死。虞又與官屬議,密令衆襲瓚。瓚部曲放散在外,自懼敗,掘東城門欲走。虞兵無部伍,不習戰,又愛民屋,勑令勿燒。故瓚得放火,因以精銳衝突。虞衆大潰,奔居庸城。瓚攻及家屬以還,殺害州府,衣冠善士殆盡。典略曰:瓚曝虞於市而祝曰:「若應為天子者,天當降雨救之。」時盛暑熱,竟日不雨,遂殺虞。英雄記曰:虞之見殺,故常山相孫瑾、掾張逸、張瓚等忠義奮發,相與就虞,罵瓚極口,然後同死。}
 瓚上訓為幽州刺史。瓚遂驕矜,記過忘善,多所賊害。
 \gezhu{英雄記曰:瓚統內外,衣冠子弟有材秀者,必抑困使在窮苦之地。或問其故,荅曰:「今取衣冠家子弟及善士富貴之,皆自以為職當得之,不謝人善也。」所寵遇驕恣者,類多庸兒,若故卜數師劉緯臺、販繒李移子、賈人樂何當等三人,與之定兄弟之誓,自號為伯,謂三人者為仲叔季,富皆巨億,或取其女以配己子,常稱古者曲周、灌嬰之屬以譬也。}
 虞從事漁陽鮮于輔、齊周、騎都尉鮮于銀等率州兵欲報瓚,以燕國閻柔素有恩信,共推柔為烏丸司馬。柔招誘烏丸、鮮卑,得胡、漢數萬人,與瓚所置漁陽太守鄒丹戰于潞北,大破之,斬丹。袁紹又遣麴義及虞子和,將兵與輔合擊瓚。瓚軍數敗,乃走還易京固守。
 \gezhu{英雄記曰:先是有童謠曰:「燕南垂,趙北際,中央不合大如礪,惟有此中可避世。」瓚以易當之,乃築京固守。瓚別將有為敵所圍,義不救也。其言曰:「救一人,使後將恃救不力戰;今不救此,後將當念在自勉。」是以袁紹始北擊之時,瓚南界上別營自度守則不能自固,又知必不見救,是以或自殺其將帥,或為紹兵所破,遂令紹軍徑至其門。臣松之以為童謠之言無不皆驗;至如此記,似若無徵。謠言之作,蓋令瓚終始保易,無事遠略。而瓚因破黃巾之威,意志張遠,遂置三州刺史,圖滅袁氏,所以致敗也。}
 為圍塹十重,於塹裏築京,皆高五六丈,為樓其上;中塹為京,特高十丈,自居焉,積穀三百萬斛。
 \gezhu{英雄記曰:瓚諸將家家各作高樓,樓以千計。瓚作鐵門,居樓上,屏去左右,婢妾侍側,汲上文書。}
 瓚曰:「昔謂天下事可指麾而定,今日視之,非我所決,不如休兵,力田畜穀。兵法,百樓不攻。今吾樓櫓千重,食盡此穀,足知天下之事矣。」欲以此弊紹。紹遣將攻之,連年不能拔。
 \gezhu{漢晉春秋曰:袁紹與瓚書曰:「孤與足下,旣有前盟舊要,申以討亂之誓,愛過夷、叔,分著丹青,謂為流力同仇,足踵齊、晉,故解印釋紱,以北帶南,分割膏腴,以奉執事,此非孤赤情之明驗邪?豈寤足下棄烈士之高義,尋禍亡之險蹤,輟而改慮,以好易怨,盜遣士馬,犯暴豫州。始聞甲卒在南,親臨戰陣,懼於飛矢迸流,狂刃橫集,以重足下之禍,徒增孤子之咎釁也,故為薦書懇惻,兾可改悔。而足下超然自逸,矜其威詐,謂天罔可吞,豪雄可滅,果令貴弟殞於鋒刃之端。斯言猶在於耳,而足下曾不尋討禍源,克心罪己,苟欲逞其無疆之怒,不顧逆順之津,匿怨害民,聘於余躬。遂躍馬控弦,處我祗上,毒徧生民,辜延白骨。孤辭不獲已,以登界橋之役。是時足下兵氣霆震,駿馬電發;僕師徒肇合,機械不嚴,彊弱殊科,衆寡異論,假天之助,小戰大克,遂陵躡奔背,因壘館穀,此非天威棐諶,福豐有禮之符表乎?足下志猶未厭,乃復糾合餘燼,率我蛑賊,以焚爇勃海。孤又不獲寧,用及龍河之師。羸兵前誘,大軍未濟,而足下膽破衆散,不鼓而敗,兵衆擾亂,君臣並奔。此又足下之為,非孤之咎也。自此以後,禍隙彌深,孤之師旅不勝其忿,遂至積尸為京,頭顱滿野,愍彼無辜,未嘗不慨然失涕也。後比得足下書,辭意婉約,有改往脩來之言。僕旣欣於舊好克復,且愍兆民之不寧,每輒引師南駕,以順簡書。弗盈一時,而北邊羽檄之文,未嘗不至。孤是用痛心疾首,靡所錯情。夫處三軍之帥,當列將之任,宜令怒如嚴霜,喜如時雨,臧否好惡,坦然可觀。而足下二三其德,彊弱易謀,急則曲躬,緩則放逸,行無定端,言無質要,為壯士者固若此乎!旣乃殘殺老弱,幽土憤怨,衆叛親離,孑然無黨。又烏丸、濊貊,皆足下同州,僕與之殊俗,各奮迅激怒,爭為鋒銳;又東西鮮卑,舉踵來附。此非孤德所能招,乃足下驅而致之也。夫當荒危之世,處干戈之險,內違同盟之誓,外失戎狄之心,兵興州壤,禍發蕭牆,將以定霸,不亦難乎!前以西山陸梁,出兵平討,會麴義餘殘,畏誅逃命,故遂住大軍,分兵撲蕩,此兵孤之前行,乃界橋搴旗拔壘,先登制敵者也。始聞足下鐫金紆紫,命以元帥,謂當因茲奮發,以報孟明之耻,是故戰夫引領,竦望旌斾,怪遂含光匿影,寂爾無聞,卒臻屠滅,相為惜之。夫有平天下之怒,希長世之功,權御師徒,帶養戎馬,叛者無討,服者不收,威懷並喪,何以立名?今舊京克復,天罔云補,罪人斯亡,忠幹翼化,華夏儼然,望於穆之作,將戢干戈,放散牛馬,足下獨何守區區之士,保軍內之廣,甘惡名以速朽,亡令德之乆長?壯而籌之,非良策也。宜釋憾除嫌,敦我舊好。若斯言之玷,皇天是聞。」瓚不荅,而增脩戎備。謂關靖曰:「當今四方虎爭,無有能坐吾城下相守經年者明矣。袁本初其若我何!」}
 建安四年,紹悉軍圍之。瓚遣子求救於黑山賊,復欲自將突騎直出,傍西南山,擁黑山之衆,陸梁兾州,橫斷紹後。長史關靖說瓚曰:「今將軍將士皆已土崩瓦解,其所以能相守持者,顧戀其居處老小,以將軍為主耳。將軍堅守曠日,袁紹要當自退;自退之後,四方之衆必復可合也。若將軍今舍之而去,軍無鎮重,易京之危,可立待也。將軍失本,孤在草野,何所成邪!」瓚遂止不出。
 \gezhu{英雄記曰:關靖字士起,太原人。本酷吏也,諂而無大謀,特為瓚所信幸。}
 救至,欲內外擊紹。遣人與子書,克期兵至,舉火為應。
 \gezhu{典略曰:瓚遣行人文則齎書告子續曰:「袁氏之攻,似若神鬼,鼓角鳴於地中,梯衝舞吾樓上。日窮月踧,無所聊賴。汝當碎首於張燕,速致輕騎,到者當起烽火於北,吾當從內出。不然,吾亡之後,天下雖廣,汝欲求安足之地,其可得乎!」獻帝春秋曰:瓚夢薊城崩,知必敗,乃遣間使與續書。紹候者得之,使陳琳更其書曰:「蓋聞在昔衰周之世,僵戶流血,以為不然,豈意今日身當其衝!」其餘語與典略所載同。}
 紹候者得其書,如期舉火。瓚以為救兵至,遂出欲戰。紹設伏擊,大破之,復還守。紹為地道,突壞其樓,稍至中京。
 \gezhu{英雄記曰:袁紹分部攻者掘地為道,穿穴其樓下,稍稍施木柱之,度足達半,便燒所施之柱,樓輒傾倒。}
 瓚自知必敗,盡殺其妻子,乃自殺。
 \gezhu{漢晉春秋曰:關靖曰:「吾聞君子陷人於危,必同其難,豈可獨生乎!」乃策馬赴紹軍而死。紹悉送其首於許。}
 
 
鮮于輔將其衆奉王命。以輔為建忠將軍,督幽州六郡。太祖與袁紹相拒於官渡,閻柔遣使詣太祖受事,遷護烏丸校尉。而輔身詣太祖,拜左渡遼將軍,封亭侯,遣還鎮撫本州。
 \gezhu{魏略曰:輔從太祖於官渡。袁紹破走,太祖喜,顧謂輔曰:「如前歲本初送公孫瓚頭來,孤自視忽然耳,而今克之。此旣天意,亦二三子之力。」}
 太祖破南皮,柔將部曲及鮮卑獻名馬以奉軍,從征三郡烏丸,以功封關內侯。
 \gezhu{魏略曰:太祖甚愛閻柔,每謂之曰:「我視卿如子,亦欲卿視我如父也。」柔由此自託於五官將,如兄弟。}
 輔亦率其衆從。文帝踐阼,拜輔虎牙將軍,柔渡遼將軍,皆進封縣侯。位特進。
 
 
\end{pinyinscope}