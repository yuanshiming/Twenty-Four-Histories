\article{吳主權徐夫人傳}
\begin{pinyinscope}
 
 
 吳主權徐夫人,吳郡富春人也。祖父真,與權父堅相親,堅以妹妻真,生琨。琨少仕州郡,漢末擾亂,去吏,隨堅征伐有功,拜偏將軍。堅薨,隨孫策討樊能、于麋等於橫江,擊張英於當利口,而船少,欲駐軍更求。琨母時在軍中,謂琨曰:「恐州家多發水軍來逆人,則不利矣,如何可駐邪?宜伐蘆葦以為泭,佐船渡軍。」
 
 
\gezhu{泭音敷。郭璞注方言曰:「泭,水中𥱼也。」}
 琨具啟策,策即行之,衆悉俱濟,遂破英,擊走笮融、劉繇,事業克定。策表琨領丹楊太守,會吳景委廣陵來東,復為丹楊守,
 \gezhu{江表傳曰:初,袁術遣從弟胤為丹楊,策令琨討而代之。會景還,以景前在丹楊,寬仁得衆,吏民所思,而琨手下兵多,策嫌其太重,且方攻伐,宜得琨衆,乃復用景,召琨還吳。}
 琨以督軍中郎將領兵,從破廬江太守李術,封廣德侯,遷平虜將軍。後從討黃祖,中流矢卒。
 
 
 
 
 琨生夫人,初適同郡陸尚。尚卒,權為討虜將軍在吳,聘以為妃,使母養子登。後權遷移,以夫人妬忌,廢處吳。積十餘年,權為吳王及即尊號,登為太子,羣臣請立夫人為后,權意在步氏,卒不許。後以疾卒。兄矯,嗣父琨侯,討平山越,拜偏將軍,先夫人卒,無子。弟祚襲封,亦以戰功至于蕪湖督、平魏將軍。
 
 
\end{pinyinscope}