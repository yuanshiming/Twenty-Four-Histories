\article{劉巴傳}
\begin{pinyinscope}
 
 
 劉巴字子初,零陵烝陽人也。少知名,
 
 
\gezhu{零陵先賢傳曰:巴祖父曜,蒼梧太守。父祥,江夏太守、盪寇將軍。時孫堅舉兵討董卓,以南陽太守張咨不給軍糧,殺之。祥與同心,南陽士民由此怨祥,舉兵攻之,與戰,敗亡。劉表亦素不善祥,拘巴,欲殺之,數遣祥故所親信人密詐謂巴曰:「劉牧欲相危害,可相隨逃之。」如此再三,巴輒不應。具以報表,表乃不殺巴。年十八,郡署戶曹史主記主簿。劉先主欲遣周不疑就巴學,巴荅曰:「昔游荊北,時涉師門,記問之學,不足紀名,內無楊朱守靜之術,外無墨翟務時之風,猶天之南箕,虛而不用。賜書乃欲令賢甥摧鸞鳳之豔,游燕雀之宇,將何以啟明之哉?愧於『有若無,實若虛』,何以堪之!」}
 荊州牧劉表連辟,及舉茂才,皆不就。表卒,曹公征荊州。先主奔江南,荊、楚羣士從之如雲,而巴北詣曹公。曹公辟為掾,使招納長沙、零陵、桂陽。
 \gezhu{零陵先賢傳曰:曹公敗於烏林,還北時,欲遣桓階,階辭不如巴。巴謂曹公曰:「劉備據荊州,不可也。」公曰:「備如相圖,孤以六軍繼之也。」}
 會先主略有三郡,巴不得反使,遂遠適交阯,
 \gezhu{零陵先賢傳云:巴往零陵,事不成,欲游交州,道還京師。時諸葛亮在臨烝,巴與亮書曰:「乘危歷險,到值思義之民,自與之衆,承天之心,順物之性,非余身謀所能勸動。若道窮數盡,將託命於滄海,不復顧荊州矣。」亮追謂曰:「劉公雄才蓋世,據有荊土,莫不歸德,天人去就,已可知矣。足下欲何之?」巴曰:「受命而來,不成當還,此其宜也。足下何言邪!」}
 先主深以為恨。
 
 
巴復從交阯至蜀。
 \gezhu{零陵先賢傳曰:巴入交阯,更姓為張。與交阯太守士爕計議不合,乃由牂牁道去。為益州郡所拘留,太守欲殺之。主簿曰:「此非常人,不可殺也。」主簿請自送至州,見益州牧劉璋,璋父焉昔為巴父祥所舉孝廉,見巴驚喜,每大事輒以咨訪。臣松之案:劉焉在漢靈帝時已經宗正太常,出為益州牧,祥始以孫堅作長沙時為江夏太守,不得舉焉為孝廉,明也。}
 俄而先主定益州,巴辭謝罪負,先主不責。
 \gezhu{零陵先賢傳曰:璋遣法正迎劉備,巴諫曰:「備,雄人也,入必為害,不可內也。」旣入,巴復諫曰:「若使備討張魯,是放虎於山林也。」璋不聽。巴閉門稱疾。備攻成都,令軍中曰:「其有害巴者,誅及三族。」及得巴,甚喜。}
 而諸葛孔明數稱薦之,先主辟為左將軍西曹掾。
 \gezhu{零陵先賢傳曰:張飛甞就巴宿,巴不與語,飛遂忿恚。諸葛亮謂巴曰:「張飛雖實武人,敬慕足下。主公今方收合文武,以定大事;足下雖天素高亮,宜少降意也。」巴曰:「大丈夫處世,當交四海英雄,如何與兵子共語乎?」備聞之,怒曰:「孤欲定天下,而子初專亂之。其欲還北,假道於此,豈欲成孤事邪?」備又曰:「子初才智絕人,如孤,可任用之,非孤者難獨任也。」亮亦曰:「運籌策於帷幄之中,吾不如子初遠矣!若提枹鼓,會軍門,使百姓喜勇,當與人議之耳。」初攻劉璋,備與士衆約:「若事定,府庫百物,孤無預焉。」及拔成都,士衆皆捨干戈,赴諸藏競取寶物。軍用不足,備甚憂之。巴曰:「易耳,但當鑄直百錢,平諸物賈,令吏為官巿。」備從之,數月之間,府庫充實。}
 建安二十四年,先主為漢中王,巴為尚書,後代法正為尚書令。躬履清儉,不治產業,又自以歸附非素,懼見猜嫌,恭默守靜,退無私交,非公事不言。
 \gezhu{零陵先賢傳曰:是時中夏人情未一,聞備在蜀,四方延頸。而備銳意欲即真,巴以為如此示天下不廣,且欲緩之。與主簿雍茂諫備,備以他事殺茂,由是遠人不復至矣。}
 先主稱尊號,昭告于皇天上帝后土神祇,凡諸文誥策命,皆巴所作也。章武二年卒。卒後,魏尚書僕射陳羣與丞相諸葛亮書,問巴消息,稱曰劉君子初,甚敬重焉。
 \gezhu{零陵先賢傳曰:輔吳將軍張昭甞對孫權論巴褊阨,不當拒張飛太甚。權曰:「若令子初隨世沈浮,容恱玄德,交非其人,何足稱為高士乎?」}
 
 
\end{pinyinscope}