\article{盧毓傳}
\begin{pinyinscope}
 
 
 盧毓字子家,涿郡涿人也。父植,有名於世。
 
 
\gezhu{續漢書曰:植字子幹。少事馬融,與鄭玄同門相友。植剛毅有大節,常喟然有濟世之志,不苟合取容,不應州郡命召。建寧中,徵博士,出補九江太守,以病去官。作尚書章句、禮記解詁。稍遷侍中、尚書。張角起,以植為北中郎將征角,失利抵罪。頃之,復以為尚書。張讓劫少帝奔小平津,植手劒責數讓等,讓等皆放兵,垂泣謝罪,遂自殺。董卓議欲廢帝,衆莫敢對,植獨正言,語在卓傳。植以老病去位,隱居上谷軍都山,初平三年卒。太祖北征柳城,過涿郡,令告太守曰:「故北中郎將盧植,名著海內,學為儒宗,士之楷模,乃國之楨幹也。昔武王入殷,封商容之閭,鄭喪子產而仲尼隕涕。孤到此州,嘉其餘風。春秋之義,賢者之後,有異於人。敬遣丞掾脩墳墓,并致薄醊,以彰厥德。」植有四子,毓最小。}
 毓十歲而孤,遇本州亂,二兄死難。當袁紹、公孫瓚交兵,幽兾饑荒,養寡嫂孤兄子,以學行見稱。文帝為五官將,召毓署門下賊曹。崔琰舉為兾州主簿。時天下草創,多逋逃,故重士亡法,罪及妻子。亡士妻白等,始適夫家數日,未與夫相見,大理奏棄巿。毓駮之曰:「夫女子之情,以接見而恩生,成婦而義重。故詩云『未見君子,我心傷悲;亦旣見止,我心則夷』。又禮『未廟見之婦而死,歸葬女氏之黨,以未成婦也』。今白等生有未見之悲,死有非婦之痛,而吏議欲肆之大辟,則若同牢合𢀿之後,罪何所加?且記曰『附從輕』,言附人之罪,以輕者為比也。又書云『與其殺不辜,寧失不經』,恐過重也。苟以白等皆受禮聘,已入門庭,刑之為可,殺之為重。」太祖曰:「毓執之是也。又引經典有意,使孤歎息。」由是為丞相法曹議令史,轉西曹議令史。
 
 
 
 
 魏國旣建,為吏部郎。文帝踐阼,徙黃門侍郎,出為濟陰相,梁、譙二郡太守。帝以譙舊鄉,故大徙民充之,以為屯田。而譙土地墝瘠,百姓窮困,毓愍之,上表徙民於梁國就沃衍,失帝意。雖聽毓所表,心猶恨之,遂左遷毓,使將徙民為睢陽典農校尉。毓心在利民,躬自臨視,擇居美田,百姓賴之。遷安平、廣平太守,所在有惠化。
 
 
 
 
 青龍二年,入為侍中。先是,散騎常侍劉劭受詔定律,未就。毓上論古今科律之意,以為法宜一正,不宜有兩端,使姦吏得容情。及侍中高堂隆數以宮室事切諫,帝不恱,毓進曰:「臣聞君明則臣直,古之聖王恐不聞其過,故有敢諫之鼓。近臣盡規,此乃臣等所以不及隆。隆諸生,名為狂直,陛下宜容之。」在職三年,多所駮爭。詔曰:「官人秩才,聖帝所難,必須良佐,進可替否。侍中毓禀性貞固,心平體正,可謂明試有功,不懈于位者也。其以毓為吏部尚書。」使毓自選代,曰:「得如卿者乃可。」毓舉常侍鄭冲,帝曰:「文和,吾自知之,更舉吾所未聞者。」乃舉阮武、孫邕,帝於是用邕。
 
 
 
 
 前此諸葛誕、鄧颺等馳名譽,有四䆫八達之誚,帝疾之。時舉中書郎,詔曰:「得其人與否,在盧生耳。選舉莫取有名,名如畫地作餅,不可啖也。」毓對曰:「名不足以致異人,而可以得常士。常士畏教慕善,然後有名,非所當疾也。愚臣旣不足以識異人,又主者正以循名案常為職,但當有以驗其後。故古者敷奏以言,明試以功。今考績之法廢,而以毀譽相進退,故真偽渾雜,虛實相蒙。」帝納其言,即詔作考課法。會司徒缺,毓舉處士管寧,帝不能用。更問其次,毓對曰:「敦篤至行,則太中大夫韓曁;亮直清方,則司隷校尉崔林;貞固純粹,則太常常林。」帝乃用曁。毓於人及選舉,先舉性行,而後言才。黃門李豐甞以問毓,毓曰:「才所以為善也,故大才成大善,小才成小善。今稱之有才而不能為善,是才不中器也。」豐等服其言。
 
 
齊王即位,賜爵關內侯。時曹爽秉權,將樹其黨,徙毓僕射,以侍中何晏代毓。頃之,出毓為廷尉,司隷畢軌又枉奏免官,衆論多訟之,乃以毓為光祿勳。爽等見收,太傅司馬宣王使毓行司隷校尉,治其獄。復為吏部尚書,加奉車都尉,封高樂亭侯,轉為僕射,故典選舉,加光祿大夫。高貴鄉公即位,進封大梁鄉侯。封一子高亭侯。毌丘儉作亂,大將軍司馬景王出征,毓綱紀後事,加侍中。正元三年,疾病,遜位。遷為司空,固推驃騎將軍王昶、光祿大夫王觀、司隷校尉王祥。詔使使者即授印綬,進爵封容城侯,邑二千三百戶。甘露二年薨,謚曰成侯。孫藩嗣。毓子欽、珽,咸熈中欽為尚書,珽泰山太守。
 \gezhu{世語曰:欽字子若,珽字子笏。欽泰始中為尚書僕射,領選,咸寧四年卒,追贈衞將軍,開府。虞預晉書曰:欽少居名位,不顧財利,清虛淡泊,動脩禮典。同郡張華,家單少孤,不為鄉邑所知,惟欽貴異焉。欽子浮,字子雲。晉諸公贊曰:張華博識多聞,無物不知。浮高朗經博,有美於華,起家太子舍人,病疽,截手,遂廢。朝廷器重之,就家以為國子博士,遷祭酒。永平中為祕書監。珽及子皓、志並至尚書。志子諶,字子諒。溫嶠表稱諶清出有文思。諶別傳曰:諶善著文章。洛陽傾覆,北投劉琨,琨以為司空從事中郎。琨敗,諶歸段末波。元帝之初,累召為散騎中書侍郎,不得南赴。永和六年,卒於胡。胡中子孫過江。妖賊帥盧循,諶之曾孫。}
 
 
 
 
 評曰:桓階識覩成敗,才周當世。陳羣動仗名義,有清流雅望;泰弘濟簡至,允克堂構矣。魏世事統臺閣,重內輕外,故八座尚書,即古六卿之任也。陳、徐、衞、盧,乆居斯位,矯、宣剛斷骨鯁,臻、毓規鑒清理,咸不忝厥職云。
 
 
\end{pinyinscope}