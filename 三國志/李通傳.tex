\article{李通傳}
\begin{pinyinscope}
 
 
 李通字文達,江夏平春人也。
 
 
\gezhu{魏略曰:通小字萬億。}
 以俠聞於江、汝之間。與其郡人陳恭共起兵於朗陵,衆多歸之。時有周直者,衆二千餘家,與恭、通外和內違。通欲圖殺直而恭難之。通知恭無斷,乃獨定策,與直克會,酒酣殺直。衆人大擾,通率恭誅其黨帥,盡并其營。後恭妻弟陳郃,殺恭而據其衆。通攻破郃軍,斬郃首以祭恭墓。又生禽黃巾大帥吳霸而降其屬。遭歲大饑,通傾家振施,與士分糟糠,皆爭為用,由是盜賊不敢犯。
 
 
 
 
 建安初,通舉衆詣太祖于許。拜通振威中郎將,屯汝南西界。太祖討張繡,劉表遣兵以助繡,太祖軍不利。通將兵夜詣太祖,太祖得以復戰,通為先登,大破繡軍。拜裨將軍,封建功侯。分汝南二縣,以通為陽安都尉。
 
 
 
 
 通妻伯父犯法,朗陵長趙儼收治,致之大辟。是時殺生之柄,決於牧守,通妻子號泣以請其命。通曰:「方與曹公勠力,義不以私廢公。」嘉儼執憲不阿,與為親交。太祖與袁紹相拒於官渡。紹遣使拜通征南將軍,劉表亦陰招之,通皆拒焉。通親戚部曲流涕曰:「今孤危獨守,以失大援,亡可立而待也,不如亟從紹。」通按劒以叱之曰:「曹公明哲,必定天下。紹雖彊盛,而任使無方,終為之虜耳。吾以死不貳。」即斬紹使,送印綬詣太祖。又擊郡賊瞿恭、江宮、沈成等,皆破殘其衆,送其首。遂定淮、汝之地。改封都亭侯,拜汝南太守。
 
 
時賊張赤等五千餘家聚桃山,通攻破之。劉備與周瑜圍曹仁於江陵,別遣關羽絕北道。通率衆擊之,下馬拔鹿角入圍,且戰且前,以迎仁軍,勇冠諸將。通道得病薨,時年四十二。追增邑二百戶,并前四百戶。文帝踐阼,謚曰剛侯。詔曰:「昔袁紹之難,自許、蔡以南,人懷異心。通秉義不顧,使攜貳率服,朕甚嘉之。不幸早薨,子基雖已襲爵,未足醻其庸勳。基兄緒,前屯樊城,又有功。世篤其勞,以基為奉義中郎將,緒平虜中郎將,以寵異焉。」
 \gezhu{王隱晉書曰:緒子秉,字玄冑,有儁才,為時所貴,官至秦州刺史。秉嘗荅司馬文王問,因以為家誡曰:「昔侍坐於先帝,時有三長吏俱見。臨辭出,上曰:『為官長當清,當慎,當勤,脩此三者,何患不治乎?』並受詔。旣出,上顧謂吾等曰:『相誡勑正當爾不?』侍坐衆賢,莫不贊善。上又問曰:『必不得已,於斯三者何先?』或對曰:『清固為本。』次復問吾,對曰:『清慎之道,相須而成,必不得已,慎乃為大。夫清者不必慎,慎者必自清,亦由仁者必有勇,勇者不必有仁,是以易稱括囊無咎,藉用白茅,皆慎之至也。』上曰:『卿言得之爾。可舉近世能慎者誰乎?』諸人各未知所對,吾乃舉故太尉荀景倩、尚書董仲連、僕射王公仲並可謂為慎。上曰:『此諸人者,溫恭朝夕,執事有恪,亦各其慎也。然天下之至慎,其惟阮嗣宗乎!每與之言,言及玄遠,而未曾評論時事,臧否人物,真可謂至慎矣。』吾每思此言,亦足以為明誡。凡人行事,年少立身,不可不慎,勿輕論人,勿輕說事,如此則悔吝何由而生,患禍無從而至矣。」秉子重,字茂曾。少知名,歷位吏部郎、平陽太守。晉諸公贊曰:重以清尚稱。相國趙王倫以重望取為右司馬。重以倫將為亂,辭疾不就。倫逼之不已,重遂不復自活,至於困篤,扶曳受拜,數日卒,贈散騎常侍。重二弟,尚字茂仲,矩字茂約,永嘉中並典郡;矩至江州刺史。重子式,字景則,官至侍中。}
 
 
\end{pinyinscope}