\article{辛毗傳}
\begin{pinyinscope}
 
 
 辛毗字佐治,潁川陽翟人也。其先建武中,自隴西東遷。毗隨兄評從袁紹。太祖為司空,辟毗,毗不得應命。及袁尚攻兄譚於平原,譚使毗詣太祖求和。
 
 
\gezhu{英雄記曰:譚、尚戰於外門,譚軍敗奔北。郭圖說譚曰:「今將軍國小兵少,糧匱勢弱,顯甫之來,久則不敵。愚以為可呼曹公來擊顯甫。曹公至,必先攻鄴,顯甫還救。將軍引兵而西,自鄴以北皆可虜得。若顯甫軍破,其兵奔亡,又可斂取以拒曹公。曹公遠僑而來,糧餉不繼,必自逃去。比此之際,趙國以北皆我之有,亦足與曹公為對矣。不然,不諧。」譚始不納,後遂從之。問圖:「誰可使?」圖荅:「辛佐治可。」譚遂遣毗詣太祖。}
 太祖將征荊州,次于西平。毗見太祖致譚意,太祖大恱。後數日,更欲先平荊州,使譚、尚自相弊。他日置酒,毗望太祖色,知有變,以語郭嘉。嘉白太祖,太祖謂毗曰:「譚可信?尚必可克不?」毗對曰:「明公無問信與詐也,直當論其勢耳。袁氏本兄弟相伐,非謂他人能閒其間,乃謂天下可定於己也。今一旦求救於明公,此可知也。顯甫見顯思困而不能取,此力竭也。兵革敗於外,謀臣誅於內,兄弟讒鬩,國分為二;連年戰伐,而介冑生蟣蝨,加以旱蝗,饑饉並臻,國無囷倉,行無裹糧,天災應於上,人事困於下,民無愚智,皆知土崩瓦解,此乃天亡尚之時也。兵法稱有石城湯池帶甲百萬而無粟者,不能守也。今往攻鄴,尚不還救,即不能自守。還救,即譚踵其後。以明公之威,應困窮之敵,擊疲弊之寇,無異迅風之振秋葉矣。天以袁尚與明公,明公不取而伐荊州。荊州豐樂,國未有釁。仲虺有言:『取亂侮亡。』方今二袁不務遠略而內相圖,可謂亂矣;居者無食,行者無糧,可謂亡矣。朝不謀夕,民命靡繼,而不綏之,欲待他年;他年或登,又自知亡而改脩厥德,失所以用兵之要矣。今因其請救而撫之,利莫大焉。且四方之寇,莫大於河北;河北平,則六軍盛而天下震。」太祖曰:「善。」乃許譚平,次于黎陽。明年攻鄴,克之,表毗為議郎。
 
 
 
 
 乆之,太祖遣都護曹洪平下辯,使毗與曹休參之,令曰:「昔高祖貪財好色,而良、平匡其過失。今佐治、文烈憂不輕矣。」軍還,為丞相長史。
 
 
 
 
 文帝踐阼,遷侍中,賜爵關內侯。時議改正朔。毗以魏氏遵舜、禹之統,應天順民;至於湯、武,以戰伐定天下,乃改正朔。孔子曰「行夏之時」,左氏傳曰「夏數為得天正」,何必期於相反。帝善而從之。
 
 
 
 
 帝欲徙兾州士家十萬戶實河南。時連蝗民饑,群司以為不可,而帝意甚盛。毗與朝臣俱求見,帝知其欲諫,作色以見之,皆莫敢言。毗曰:「陛下欲徙士家,其計安出:」帝曰:「卿謂我徙之非邪?」毗曰:「誠以為非也。」帝曰:「吾不與卿共議也。」毗曰:「陛下不以臣不肖,置之左右,廁之謀議之官,安得不與臣議邪!臣所言非私,乃社稷之慮也,安得怒臣!」帝不荅,起入內;毗隨而引其裾,帝遂奮衣不還,良乆乃出,曰:「佐治,卿持我何太急邪?」毗曰:「今徙,旣失民心,又無以食也。」帝遂徙其半。嘗從帝射雉,帝曰:「射雉樂哉!」毗曰:「於陛下甚樂,而於羣下甚苦。」帝默然,後遂為之稀出。
 
 
 
 
 上軍大將軍曹真征朱然于江陵,毗行軍師。還,封廣平亭侯。帝欲大興軍征吳,毗諫曰:「吳、楚之民,險而難禦,道隆後服,道洿先叛,自古患之,非徒今也。今陛下祚有海內,夫不賔者,其能乆乎?昔尉佗稱帝,子陽僭號,歷年未幾,或臣或誅。何則,違逆之道不乆全,而大德無所不服也。方今天下新定,土廣民稀。夫廟筭而後出軍,猶臨事而懼,況今廟筭有闕而欲用之,臣誠未見其利也。先帝屢起銳師,臨江而旋。今六軍不增於故,而復循之,此未易也。今日之計,莫若脩范蠡之養民,法管仲之寄政,則充國之屯田,明仲尼之懷遠;十年之中,彊壯未老,童齓勝戰,兆民知義,將士思奮,然後用之,則役不再舉矣。」帝曰:「如卿意,更當以虜遺子孫邪?」毗對曰:「昔周文王以紂遺武王,惟知時也。苟時未可,容得已乎!」帝竟伐吳,至江而還。
 
 
 
 
 明帝即位,進封潁鄉侯,邑三百戶。時中書監劉放、令孫資見信於主,制斷時政,大臣莫不交好,而毗不與往來。毗子敞諫曰:「今劉、孫用事,衆皆影附,大人宜小降意,和光同塵;不然必有謗言。」毗正色曰:「主上雖未稱聦明,不為闇劣。吾之立身,自有本未。就與劉、孫不平,不過令吾不作三公而已,何危害之有?焉有大丈夫欲為公而毀其高節者邪?」宂從僕射畢軌表言:「尚書僕射王思精勤舊吏,忠亮計略不如辛毗,毗宜代思。」帝以訪放、資,放、資對曰:「陛下用思者,誠欲取其効力,不貴虛名也。毗實亮直,然性剛而專,聖慮所當深察也。」遂不用。出為衞尉。
 
 
帝方脩殿舍,百姓勞役,毗上疏曰:「竊聞諸葛亮講武治兵,而孫權巿馬遼東,量其意指,似欲相左右。備豫不虞,古之善政,而今者宮室大興,加連年穀麥不收。詩云:『民亦勞止,迄可小康,惠此中國,以綏四方。』唯陛下為社稷計。」帝報曰:「二虜未滅而治宮室,直諫者立名之時也。夫王者之都,當及民勞兼辦,使後世無所復增,是蕭何為漢規摹之略也。今卿為魏重臣,亦宜解其大歸。」帝又欲平北芒,令於其上作臺觀,則見孟津。毗諫曰:「天地之性,高高下下,今而反之,旣非其理;加以損費人功,民不堪役。且若九河盈溢,洪水為害,而丘陵皆夷,將何以禦之?」帝乃止。
 \gezhu{魏略曰:諸葛亮圍祁山,不克,引退。張郃追之,為流矢所中死。帝惜郃,臨朝而歎曰:「蜀未平而郃死,將若之何!」司空陳羣曰:「郃誠良將,國所依也。」毗心以為郃雖可惜,然已死,不當內弱主意,而示外以不大也。乃持羣曰:「陳公,是何言歟!當建安之末,天下不可一日無武皇帝也,及委國祚,而文皇帝受命,黃初之世,亦謂不可無文皇帝也,及委棄天下,而陛下龍興。今國內所少,豈張郃乎?」陳羣曰:「亦誠如辛毗言。」帝笑曰:「陳公可謂善變矣。」臣松之以為擬人必於其倫,取譬宜引其類,故君子於其言,無所苟而已矣。毗欲弘廣主意,當舉若張遼之疇,安有於一將之死而可以祖宗為譬哉?非所宜言,莫過於茲,進違其類,退似諂佞,佐治剛正之體,不宜有此。魏略旣已難信,習氏又從而載之,竊謂斯人受誣不少。}
 
 
青龍二年,諸葛亮率衆出渭南。先是,大將軍司馬宣王數請與亮戰,明帝終不聽。是歲恐不能禁,乃以毗為大將軍軍師,使持節;六軍皆肅,準毗節度,莫敢犯違。
 \gezhu{魏略曰:宣王數數欲進攻,毗禁不聽。宣王雖能行意,而每屈於毗。}
 亮卒,復還為衞尉。薨,謚曰肅侯。子敞嗣,咸熈中為河內太守。
 \gezhu{世語曰:敞字泰雍,官至衞尉。毗女憲英,適太常泰山羊耽,外孫夏侯湛為其傳曰:「憲英聦明有才鑒。初文帝與陳思王爭為太子,旣而文帝得立,抱毗頸而喜曰:『辛君知我喜不?』毗以告憲英,憲英歎曰:『太子代君主宗廟社稷者也。代君不可以不戚,主國不可以不懼,宜戚而喜,何以能乆?魏其不昌乎!』弟敞為大將軍曹爽參軍。司馬宣王將誅爽,因爽出,閉城門。大將軍司馬魯芝將爽府兵,犯門斬關,出城門赴爽,來呼敞俱去。敞懼,問憲英曰:『天子在外,太傅閉城門,人云將不利國家,於事可得爾乎?』憲英曰:『天下有不可知,然以吾度之,太傅殆不得不爾!明皇帝臨崩,把太傅臂,以後事付之,此言猶在朝士之耳。且曹爽與太傅俱受寄託之任,而獨專權勢,行以驕奢,於王室不忠,於人道不直,此舉不過以誅曹爽耳。』敞曰:『然則事就乎?』憲英曰:『得無殆就!爽之才非太傅之偶也。』敞曰:『然則敞可以無出乎?』憲英曰:『安可以不出。職守,人之大義也。凡人在難,猶或卹之;為人執鞭而棄其事,不祥,不可也。且為人死,為人任,親昵之職也,從衆而已。』敞遂出。宣王果誅爽。事定之後,敞歎曰:『吾不謀於姊,幾不獲於義。』逮鍾會為鎮西將軍,憲英謂從子羊祜曰:『鍾士季何故西出?』祜曰:『將為滅蜀也。』憲英曰:『會在事縱恣,非持乆處下之道,吾畏其有他志也。』祜曰:『季母勿多言。』其後會請子琇為參軍,憲英憂曰:『他日見鍾會之出,吾為國憂之矣。今日難至吾家,此國之大事,必不得止也。』琇固請司馬文王,文王不聽。憲英語琇曰:『行矣,戒之!古之君子,入則致孝於親,出則致節於國,在職思其所司,在義思其所立,不遺父母憂患而已。軍旅之間,可以濟者,其惟仁恕乎!汝其慎之!』琇竟以全身。憲英年至七十有九,泰始五年卒。」}
 
 
\end{pinyinscope}