\article{劉曄傳}
\begin{pinyinscope}
 
 
 劉曄字子揚,淮南成悳人,
 
 
\gezhu{悳音德。}
 漢光武子阜陵王延後也。父普,母脩,產渙及曄。渙九歲,曄七歲,而母病困。臨終,戒渙、曄以「普之侍人,有諂害之性。身死之後,懼必亂家。汝長大能除之,則吾無恨矣。」曄年十三,謂兄渙曰:「亡母之言,可以行矣。」渙曰:「𨚗可爾!」曄即入室殺侍者,徑出拜墓。舍內大駕,白普。普怒,遣人追曄。曄還拜謝曰:「亡母顧命之言,敢受不請擅行之罰。」普心異之,遂不責也。汝南許劭名知人,避地揚州,稱曄有佐世之才。
 
 
 
 
 揚士多輕俠狡桀,有鄭寶、張多、許乾之屬,各擁部曲。寶最驍果,才力過人,一方所憚。欲驅略百姓越赴江表,以曄高族名人,欲彊逼曄使唱導此謀。曄時年二十餘,心內憂之,而未有緣。會太祖遣使詣州,有所案問。曄往見,為論事勢,要將與歸,駐止數日。寶果從數百人齎牛酒來候使,曄令家僮將其衆坐中門外,為設酒飯;與寶於內宴飲。密勒健兒,令因行觴而斫寶。寶性不甘酒,視候甚明,觴者不敢發。曄因自引取佩刀斫殺寶,斬其首以令其軍,云:「曹公有令,敢有動者,與寶同罪。」衆皆驚怖,走還營。營有督將精兵數千,懼其為亂,曄即乘寶馬,將家僮數人,詣寶營門,呼其渠帥,喻以禍福,皆叩頭開門內曄。曄撫慰安懷,咸悉恱服,推曄為主。曄覩漢室漸微,己為支屬,不欲擁兵,遂委其部曲與廬江太守劉勳。勳怪其故,曄曰:「寶無法制,其衆素以鈔略為利,僕宿無資,而整齊之,必懷怨難乆,故相與耳。」時勳兵彊於江、淮之間。孫策惡之,遣使卑辭厚幣,以書說勳曰:「上繚宗民,數欺下國,忿之有年矣。擊之,路不便,願因大國伐之。上繚甚實,得之可以富國,請出兵為外援。」勳信之,又得策珠寶、葛越,喜恱。外內盡賀,而曄獨否。勳問其故,對曰:「上繚雖小,城堅池深,攻難守易,不可旬日而舉,則兵疲於外,而國內虛。策乘虛而襲我,則後不能獨守。是將軍進屈於敵,退無所歸。若軍必出,禍今至矣。」勳不從。興兵伐上繚,策果襲其後。勳窮踧,遂奔太祖。
 
 
太祖至壽春,時廬江界有山賊陳策,衆數萬人,臨險而守。先時遣偏將致誅,莫能禽克。太祖問羣下,可伐與不?咸云:「山峻高而谿谷深隘,守易攻難;又無之不足為損,得之不足為益。」曄曰:「策等小豎,因亂赴險,遂相依為彊耳,非有爵命威信相伏也。往者偏將資輕,而中國未夷,故策敢據險以守。今天下略定,後伏先誅。夫畏死趨賞,愚智所同,故廣武君為韓信畫策,謂其威名足以先聲後實而服鄰國也。豈況明公之德,東征西怨,先開賞募,大兵臨之,令宣之日,軍門啟而虜自潰矣。」太祖笑曰:「卿言近之!」遂遣猛將在前,大軍在後,至則克策,如曄所度。太祖還,辟曄為司空倉曹掾。
 \gezhu{傅子曰:太祖徵曄及蔣濟、胡質等五人,皆揚州名士。每舍亭傳,未曾不講,所以見重;內論國邑先賢、禦賊固守、行軍進退之宜,外料敵之變化、彼我虛實、戰爭之術,夙夜不解。而曄獨卧車中,終不一言。濟怪而問之,曄荅曰:「對明主非精神不接,精神可學而得乎?」及見太祖,太祖果問揚州先賢,賊之形勢。四人爭對,待次而言,再見如此,太祖每和恱,而曄終不一言。四人笑之。後一見太祖止無所復問,曄乃設遠言以動太祖,太祖適知便止。若是者三。其旨趣以為遠言宜徵精神,獨見以盡其機,不宜於猥坐說也。太祖已探見其心矣,坐罷,尋以四人為令,而授曄以心腹之任;每有疑事,輒以函問曄,至一夜數十至耳。}
 
 
太祖征張魯,轉曄為主簿。旣至漢中,山峻難登,軍食頗乏。太祖曰:「此妖妄之國耳,何能為有無?吾軍少食,不如速還。」便自引歸,令曄督後諸軍,使以次出。曄策魯可克,加糧道不繼,雖出,軍猶不能皆全,馳白太祖:「不如致攻。」遂進兵,多出弩以射其營。魯奔走,漢中遂平。曄進曰:「明公以步卒五千,將誅董卓,北破袁紹,南征劉表,九州百郡,十并其八,威震天下,勢慴海外。今舉漢中,蜀人望風,破膽失守,推此而前,蜀可傳檄而定。劉備,人傑也,有度而遲,得蜀日淺,蜀人未恃也。今破漢中,蜀人震恐,其勢自傾。以公之神明,因其傾而壓之,無不克也。若小緩之,諸葛亮明於治而為相,關羽、張飛勇冠三軍而為將,蜀民旣定,據險守要,則不可犯矣。今不取,必為後憂。」太祖不從,
 \gezhu{傅子曰:居七日,蜀降者說:「蜀中一日數十驚,備雖斬之而不能安也。」太祖延問曄曰:「今尚可擊不?」曄曰:「今已小定,未可擊也。」}
 大軍遂還。曄自漢中還,為行軍長史,兼領軍。延康元年,蜀將孟達率衆降。達有容止才觀,文帝甚器愛之,使達為新城太守,加散騎常侍。曄以為「達有苟得之心,而恃才好術,必不能感恩懷義。新城與吳、蜀接連,若有變態,為國生患。」文帝竟不易,後達終於叛敗
 \gezhu{傅子曰:初,太祖時,魏諷有重名,自卿相已下皆傾心交之。其後孟達去劉備歸文帝,論者多稱有樂毅之量。曄一見諷、達而皆云必反,卒如其言。}
 
 
黃初元年,以曄為侍中,賜爵關內侯。詔問羣臣令料劉備當為關羽出報吳不。衆議咸云:「蜀,小國耳,名將唯羽。羽死軍破,國內憂懼,無緣復出。」曄獨曰:「蜀雖狹弱,而備之謀欲以威武自彊,勢必用衆以示其有餘。且關羽與備,義為君臣,恩猶父子;羽死不能為興軍報敵,於終始之分不足。」後備果出兵擊吳。吳悉國應之,而遣使稱藩。朝臣皆賀,獨曄曰:「吳絕在江、漢之表,無內臣之心乆矣。陛下雖齊德有虞,然醜虜之性未有所感。因難求臣,必難信也。彼必外迫內困,然後發此使耳,可因其窮,襲而取之。夫一日縱敵,數世之患,不可不察也。」備軍敗退,吳禮敬轉廢,帝欲興衆伐之,曄以為「彼新得志,上下齊心,而阻帶江湖,必難倉卒。」帝不聽。
 \gezhu{傅子曰:孫權遣使求降,帝以問曄。曄對曰:「權無故求降,必內有急。權前襲殺關羽,取荊州四郡,備怒,必大興師伐之。外有彊寇,衆心不安,又恐中國承其釁而伐之,故委地求降,一以却中國之兵,二則假中國之援,以彊其衆而疑敵人。權善用兵,見策知變,其計必出於此。今天下三分,中國十有其八。吳、蜀各保一州,阻山依水,有急相救,此小國之利也。今還自相攻,天亡之也。宜大興師,徑渡江襲其內。蜀攻其外,我襲其內,吳之亡不出旬月矣。吳亡則蜀孤。若割吳半,蜀固不能乆存,況蜀得其外,我得其內乎!」帝曰:「人稱臣降而伐之,疑天下欲來者心,必以為懼,其殆不可!孤何不且受吳降,而襲蜀之後乎?」對曰:「蜀遠吳近,又聞中國伐之,便還軍,不能止也。今備已怒,故興兵擊吳,聞我伐吳,知吳必亡,必喜而進與我爭割吳地,必不改計抑怒救吳,必然之勢也。」帝不聽,遂受吳降,即拜權為吳王。曄又進曰:「不可。先帝征伐,天下兼其八,威震海內,陛下受禪即真,德合天地,聲曁四遠,此實然之勢,非卑臣頌言也。權雖有雄才,故漢驃騎將軍南昌侯耳,官輕勢卑。士民有畏中國心,不可彊迫與成所謀也。不得已受其降,可進其將軍號,封十萬戶侯,不可即以為王也。夫王位,去天子一階耳,其禮秩服御相亂也。彼直為侯,江南士民未有君臣之義也。我信其偽降,就封殖之,崇其位號,定其君臣,是為虎傅翼也。權旣受王位,却蜀兵之後,外盡禮事中國,使其國內皆聞之,內為無禮以怒陛下。陛下赫然發怒,興兵討之,乃徐告其民曰:『我委身事中國,不愛珍貨重寶,隨時貢獻,不敢失臣禮也,無故伐我,必欲殘我國家,俘我民人子女以為僮隷僕妾。』吳民無緣不信其言也。信其言而感怒,上下同心,戰加十倍矣。」又不從。遂即拜權為吳王。權將陸議大敗劉備,殺其兵八萬餘人,備僅以身免。權外禮愈卑,而內行不順,果如曄言。}
 五年,幸廣陵泗口,命荊、揚州諸軍並進。會羣臣,問:「權當自來不?」咸曰:「陛下親征,權恐怖,必舉國而應。又不敢以大衆委之臣下,必自將而來。」曄曰:「彼謂陛下欲以萬乘之重牽己,而超越江湖者在於別將,必勒兵待事,未有進退也。」大駕停住積日,權果不至,帝乃旋師。云:「卿策之是也。當念為吾滅二賊,不可但知其情而已。」
 
 
 
 
 明帝即位,進爵東亭侯,邑三百戶。詔曰:「尊嚴祖考,所以崇孝表行也;追本敬始,所以篤教流化也。是以成湯、文、武,寔造商、周,詩、書之義,追尊稷、契,歌頌有娀、姜嫄之事,明盛德之源流,受命所由興也。自我魏室之承天序,旣發迹於高皇、太皇帝,而功隆於武皇、文皇帝。至於高皇之父處士君,潛脩德讓,行動神明,斯乃乾坤所福饗,光靈所從來也。而精神幽遠,號稱罔記,非所謂崇孝重本也。其令公卿已下,會議號謚。」曄議曰:「聖帝孝孫之欲襃崇先祖,誠無量已。然親疏之數,遠近之降,蓋有禮紀,所以割斷私情,克成公法,為萬世式也。周王所以上祖后稷者,以其佐唐有功,名在祀典故也。至於漢氏之初,追謚之義,不過其父。上比周室,則大魏發迹自高皇始;下論漢氏,則追謚之禮不及其祖。此誠往代之成法,當今之明義也。陛下孝思中發,誠無已已,然君舉必書,所以慎於禮制也。以為追尊之義,宜齊高皇而已。」尚書衞臻與曄議同,事遂施行。
 
 
 
 
 遼東太守公孫淵奪叔父位,擅自立,遣使表狀。曄以為公孫氏漢時所用,遂世官相承,水則由海,陸則阻山,故胡夷絕遠難制,而世權日乆。今若不誅,後必生患。若懷貳阻兵,然後致誅,於事為難。不如因其新立,有黨有仇,先其不意,以兵臨之,開設賞募,可不勞師而定也。後淵竟反。
 
 
曄在朝,略不交接時人。或問其故,曄荅曰:「魏室即阼尚新,智者知命,俗或未咸。僕在漢為支葉,於魏備腹心,寡偶少徒,於宜未失也。」太和六年,以疾拜太中大夫。有間,為大鴻臚,在位二年遜位,復為太中大夫,薨。謚曰景侯。子㝢嗣。
 \gezhu{傅子曰:曄事明皇帝,又大見親重。帝將伐蜀,朝臣內外皆曰「不可」。曄入與帝議,因曰「可伐」;出與朝臣言,因曰「不可伐」。曄有膽智,言之皆有形。中領軍楊曁,帝之親臣,又重曄,持不可伐蜀之議最堅,每從內出,輒過曄,曄講不可之意。後曁從駕行天淵池,帝論伐蜀事,曁切諫。帝曰:「卿書生,焉知兵事!」曁謙謝曰:「臣出自儒生之末,陛下過聽,拔臣羣萃之中,立之六軍之上,臣有微心,不敢不盡言。臣言誠不足采,侍中劉曄先帝謀臣,常曰蜀不可伐。」帝曰:「曄與吾言蜀可伐。」曁曰:「曄可召質也。」詔召曄至,帝問曄,終不言。後獨見,曄責帝曰:「伐國,大謀也,臣得與聞大謀,常恐眯夢漏泄以益臣罪,焉敢向人言之?夫兵,詭道也,軍事未發,不猒其密也。陛下顯然露之,臣恐敵國已聞之矣。」於是帝謝之。曄見出,責曁曰:「夫釣者中大魚,則縱而隨之,須可制而後牽,則無不得也。人主之威,豈徒大魚而已!子誠直臣,然計不足采,不可不精思也。」曁亦謝之。曄能應變持兩端如此。或惡曄於帝曰:「曄不盡忠,善伺上意所趨而合之。陛下試與曄言,皆反意而問之,若皆與所問反者,是曄常與聖意合也。復每問皆同者,曄之情必無所逃矣。」帝如言以驗之,果得其情,從此疏焉。曄遂發狂,出為大鴻臚,以憂死。諺曰「巧詐不如拙誠」,信矣。以曄之明智權計,若居之以德義,行之以忠信,古之上賢,何以加諸?獨任才智,不與世士相經緯,內不推心事上,外困於俗,卒不能自安於天下,豈不惜哉!}
 少子陶,亦高才而薄行,官至平原太守。
 \gezhu{王弼傳曰:淮南人劉陶,善論縱橫,為當時所推。傅子曰:陶字季冶,善名稱,有大辯。曹爽時為選部郎,鄧颺之徒稱之以為伊呂。當此之時,其人意陵青雲,謂玄曰:「仲尼不聖。何以知其然?智者圖國;天下羣愚,如弄一丸於掌中,而不能得天下。」玄以其言大惑,不復詳難也。謂之曰:「天下之質,變無常也。今見卿窮!」爽之敗,退居里舍,乃謝其言之過。干寶晉紀曰:毌丘儉之起也,大將軍以問陶,陶荅依違。大將軍怒曰:「卿平生與吾論天下事,至於今日而更不盡乎?」乃出為平原太守,又追殺之。}
 
 
\end{pinyinscope}