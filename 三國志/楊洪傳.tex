\article{楊洪傳}
\begin{pinyinscope}
 
 
 楊洪字季休,犍為武陽人也。劉璋時歷部諸郡。先主定蜀,太守李嚴命為功曹。嚴欲徙郡治舍,洪固諫不聽,遂辭功曹,請退。嚴欲薦洪於州,為蜀部從事。先主爭漢中,急書發兵,軍師將軍諸葛亮以問洪,洪曰:「漢中則益州咽喉,存亡之機會,若無漢中則無蜀矣,此家門之禍也。方今之事,男子當戰,女子當運,發兵何疑?」時蜀郡太守法正從先主北行,亮於是表洪領蜀郡太守,衆事皆辦,遂使即真。頃之,轉為益州治中從事。
 
 
 
 
 先主旣稱尊號,征吳不克,還住永安。漢嘉太守黃元素為諸葛亮所不善,聞先主疾病,懼有後患,舉郡反,燒臨邛城。時亮東行省疾,成都單虛,是以元益無所憚。洪即啟太子,遣其親兵,使將軍陳曶、鄭綽討元。衆議以為元若不能圍成都,當由越嶲據南中,洪曰:「元素性凶暴,無他恩信,何能辦此?不過乘水東下,兾主上平安,面縛歸死;如其有異,奔吳求活耳。勑曶、綽但於南安峽口遮即便得矣。」曶、綽承洪言,果生獲元。洪建興元年賜爵關內侯,復為蜀郡太守、忠節將軍,後為越騎校尉,領郡如故。
 
 
 
 
 五年,丞相亮北住漢中,欲用張裔為留府長史,問洪何如?洪對曰:「裔天姿明察,長於治劇,才誠堪之,然性不公平,恐不可專任,不如留向朗。朗情偽差少,裔隨從目下,效其器能,於事兩善。」初,裔少與洪親善。裔流放在吳,洪臨裔郡,裔子郁給郡吏,微過受罰,不特原假。裔後還聞之,深以為恨,與洪情好有損。及洪見亮出,至裔許,具說所言。裔荅洪曰:「公留我了矣,明府不能止。」時人或疑洪意自欲作長史,或疑洪知裔自嫌,不願裔處要職,典後事也。後裔與司鹽校尉岑述不和,至于忿恨。亮與裔書曰:「君昔在栢下,營壞,吾之用心,食不知味;後流迸南海,相為悲歎,寢不安席;及其來還,委付大任,同獎王室,自以為與君古之石交也。石交之道,舉讎以相益,割骨肉以相明,猶不相謝也,況吾但委意於元儉,而君不能忍邪?」論者由是明洪無私。
 
 
 
 
 洪少不好學問,而忠清款亮,憂公如家,事繼母至孝。六年卒官。始洪為李嚴功曹,嚴未至犍為而洪已為蜀郡。洪迎門下書佐何祗,有才策功幹,舉郡吏,數年為廣漢太守,時洪亦尚在蜀郡。是以西土咸服諸葛亮能盡時人之器用也。
 
 
\gezhu{益部耆舊傳雜記曰:每朝會,祗次洪坐。嘲祗曰:「君馬何駛?」祗曰:「故吏馬不敢駛,但明府未著鞭耳。」傳之以為笑。祗字君肅,少寒貧,為人寬厚通濟,體甚壯大,又能飲食,好聲色,不持節儉,故時人少貴之者。甞夢井中生桑,以問占夢趙直,直曰:「桑非井中之物,會當移植;然桑字四十下八,君壽恐不過此。」祗笑言「得此足矣」。初仕郡,後為督軍從事。時諸葛亮用法峻密,陰聞祗游戲放縱,不勤所職,常奄往錄獄。衆人咸為祗懼。祗密聞之,夜張灯火見囚,讀諸解狀。諸葛晨往,祗悉已闇誦,荅對解釋,無所凝滯,亮甚異之。出補成都令,時郫縣令缺,以祗兼二縣。二縣戶口猥多,切近都治,饒諸姦穢,每比人,常眠睡,值其覺寤,輒得姦詐,衆咸畏祗之發摘,或以為有術,無敢欺者。使人投筭,祗聽其讀而心計之,不差升合,其精如此。汶山夷不安,以祗為汶山太守,民夷服信。遷廣漢。後夷反叛,辭:「令得前何府君,乃能安我耳!」時難屈祗,拔祗族人為汶山,復得安。轉祗為犍為。年四十八卒,如直所言。後有廣漢王離,字伯元,亦以才幹顯。為督軍從事,推法平當,稍遷,代祗為犍為太守,治有美績,雖聦明不及祗,而文采過之也。}
 
 
\end{pinyinscope}