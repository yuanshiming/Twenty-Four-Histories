\article{薛綜傳}
\begin{pinyinscope}
 
 
 薛綜字敬文,沛郡竹邑人也。
 
 
\gezhu{吳錄曰:其先齊孟嘗君封於薛。秦滅六國,而失其祀,子孫分散。漢祖定天下,過齊,求孟嘗後,得其孫陵、國二人,欲復其封。陵、國兄弟相推,莫適受,乃去之竹邑,因家焉,故遂氏薛。自國至綜,世典州郡,為著姓。綜少明經,善屬文,有秀才。}
 少依族人避地交州,從劉熈學。士燮旣附孫權,召綜為五官中郎,除合浦、交阯太守。時交土始開,刺吏呂岱率師討伐,綜與俱行,越海南征,及到九真。事畢還都,守謁者僕射。西使張奉於權前列尚書闞澤姓名以嘲澤,澤不能荅。綜下行酒,因勸酒曰:「蜀者何也?有犬為獨,無犬為蜀,橫目苟身,虫入其腹。」
 \gezhu{臣松之見諸書本「苟身」或作「句身」,以為旣云「橫目」,則宜曰「句身」。}
 奉曰:「不當復列君吳邪?」綜應聲曰:「無口為天,有口為吳,君臨萬邦,天子之都。」於是衆坐喜笑,而奉無以對。其樞機敏捷,皆此類也。
 \gezhu{江表傳曰:費禕聘于吳,陛見,公卿侍臣皆在坐。酒酣,禕與諸葛恪相對嘲難,言及吳、蜀。禕問曰:「蜀字云何?」恪曰:「有水者濁,無水者蜀。橫目苟身,虫入其腹。」禕復問:「吳字云何?」恪曰:「無口者天,有口者吳,下臨滄海,天子帝都。」與本傳不同。}
 
 
 
 
 呂岱從交州召出,綜懼繼岱者非其人,上疏曰:「昔帝舜南巡,卒於蒼梧。秦置桂林、南海、象郡,然則四國之內屬也,有自來矣。趙佗起番禺,懷服百越之君,珠官之南是也。漢武帝誅呂嘉,開九郡,設交阯刺史以鎮監之。山川長遠,習俗不齊,言語同異,重譯乃通,民如禽獸,長幼無別,椎結徒跣,貫頭左袵,長吏之設,雖有若無。自斯以來,頗徙中國罪人雜居其間,稍使學書,粗知言語,使驛往來,觀見禮化。及後錫光為交阯,任延為九真太守,乃教其耕犂,使之冠履;為設媒官,始知聘娶;建立學斆,導之經義。由此已降,四百餘年,頗有似類。自臣昔客始至之時,珠崖除州縣嫁娶,皆須八月引戶,人民集會之時,男女自相可適,乃為夫妻,父母不能止。交阯糜泠、九真都龐二縣,皆兄死弟妻其嫂,世以此為俗,長吏恣聽,不能禁制。日南郡男女倮體,不以為羞。由此言之,可謂蟲豸,有靦面目耳。然而土廣人衆,阻險毒害,易以為亂,難使從治。縣官羈縻,示令威服,田戶之租賦,裁取供辦,貴致遠珍名珠、香藥、象牙、犀角、瑇瑁、珊瑚、琉璃、鸚鵡、翡翠、孔雀、奇物、充備寶玩,不必仰其賦入,以益中國也。然在九甸之外,長吏之選,類不精覈。漢時法寬,多自放恣,故數反違法。珠崖之廢,起於長吏覩其好髮,髠取為髲。及臣所見,南海黃蓋為日南太守,下車以供設不豐,檛殺主簿,仍見驅逐。九真太守儋萌為妻父周京作主人,并請大吏,酒酣作樂,功曹番歆起舞屬京,京不肯起,歆猶迫彊,萌忿杖歆,亡於郡內。歆弟苗帥衆攻府,毒矢射萌,萌至物故。交阯太守士燮遣兵致討,卒不能克。又故刺史會稽朱符,多以鄉人虞襃、劉彥之徒分作長吏,侵虐百姓,彊賦於民,黃魚一枚收稻一斛,百姓怨叛,山賊並出,攻州突郡。符走入海,流離喪亡。次得南陽張津,與荊州牧劉表為隙,兵弱敵彊,歲歲興軍,諸將厭患,去留自在。津小檢攝,威武不足,為所陵侮,遂至殺沒。後得零陵賴恭,先輩仁謹,不曉時事。表又遣長沙吳巨為蒼梧太守。巨武夫輕悍,不為恭服所取,相怨恨,逐出恭,求步隲。是時津故將夷廖、錢博之徒尚多,隲以次鉏治,綱紀適定,會仍召出。呂岱旣至,有士民之變。越軍南征,平討之日,改置長吏,章明王綱,威加萬里,大小承風。由此言之,綏邊撫裔,實有其人。牧伯之任,旣宜清能,荒流之表,禍福尤甚。今日交州雖名粗定,尚有高涼宿賊;其南海、蒼梧、鬱林、珠官四郡界未綏,依作寇盜,專為亡叛逋逃之藪。若岱不復南,新刺史宜得精密檢攝八郡,方略智計能稍稍以漸;能治高涼者,假以威寵,借之形勢,責其成効,庶幾可補復。如但中人,近守常法,無奇數異術者,則羣惡日滋,乆遠成害。故國之安危,在於所任,不可不察也。竊懼朝廷忽輕其選,故敢竭愚情,以廣聖思。」
 
 
 
 
 黃龍三年,建昌侯慮為鎮軍大將軍,屯半州,以綜為長史,外掌衆事,內授書籍。慮卒,入守賊曹尚書,遷尚書僕射。時公孫淵降而復叛,權盛怒,欲自親征。綜上疏諫曰:「夫帝王者,萬國之元首,天下之所繫命也。是以居則重門擊柝以戒不虞,行則清道案節以養威嚴,蓋所以存萬安之福,鎮四海之心。昔孔子疾時,託乘桴浮之語,季由斯喜,拒以無所取才。漢元帝欲御樓船,薛廣德請刎頸以血染車。何則?水火之險至危,非帝王所宜涉也。諺曰:『千金之子,坐不垂堂。』況萬乘之尊乎?今遼東戎貊小國,無城池之固,備禦之術,器械銖鈍,犬羊無政,往必禽克,誠如明詔。然其方土寒埆,穀稼不殖,民習鞌馬,轉徙無常。卒聞大軍之至,自度不敵,鳥驚獸駭,長驅奔竄,一人匹馬,不可得見,雖獲空地,守之無益,此不可一也。加又洪流滉瀁,有成山之難,海行無常,風波難免,倏忽之間,人船異勢。雖有堯舜之德,智無所施,賁育之勇,力不得設,此不可二也。加以鬱霧冥其上,鹹水蒸其下,善生流腫,轉相洿染,凡行海者,稀無斯患,此不可三也。天生神聖,顯以符瑞,當乘平喪亂,康此民物;嘉祥日集,海內垂定,逆虜凶虐,滅亡在近。中國一平,遼東自斃,但當拱手以待耳。今乃違必然之圖,尋至危之阻,忽九州之固,肆一朝之忿,旣非社稷之重計,又開闢以來所未嘗有,斯誠羣僚所以傾身側息,食不甘味,寢不安席者也。惟陛下抑雷霆之威,忍赫斯之怒,遵乘橋之安,遠履冰之險,則臣子賴祉,天下幸甚。」時羣臣多諫,權遂不行。
 
 
正月乙未,權勑綜祝祖不得用常文,綜承詔,卒造文義,信辭粲爛。權曰:「復為兩頭,使滿三也。」綜復再祝,辭令皆新,衆咸稱善。赤烏三年,徙選曹尚書。五年,為太子少傅,領選職如故。
 \gezhu{吳書曰:後權賜綜紫綬囊,綜陳讓紫色非所宜服,權曰:「太子年少,涉道日淺,君當博之以文,約之以禮,茅土之封,非君而誰?」是時綜以名儒居師傅之位,仍兼選舉,甚為優重。}
 六年春,卒。凡所著詩賦難論數萬言,名曰私載,又定五宗圖述、二京解,皆傳於世。
 
 
子珝,官及威南將軍,征交阯還,道病死。
 \gezhu{漢晉春秋曰:孫休時,珝為五官中郎將,遣至蜀求馬。及還,休問蜀政得失,對曰:「主闇而不知其過,臣下容身以求免罪,入其朝不聞正言,經其野民皆菜色。臣聞燕雀處堂,子母相樂,自以為安也,突決棟焚,而燕雀怡然不知禍之將及,其是之謂乎!」}
 珝弟瑩,字道言,初為祕府中書郎,孫休即位,為散騎中常侍。數年,以病去官。孫皓初,為左執法,遷選曹尚書,及立太子,又領少傅。建衡三年,皓追歎瑩父綜遺文,且命瑩繼作。瑩獻詩曰:「惟臣之先,昔仕于漢,奕世緜緜,頗涉臺觀。曁臣父綜,遭時之難,卯金失御,邦家毀亂。適茲樂土,庶存孑遺,天啟其心,東南是歸。厥初流隷,困于蠻垂。大皇開基,恩德遠施。特蒙招命。拯擢泥汙,釋放巾褐,受職剖符。作守合浦,在海之隅,遷入京輦,遂升機樞。枯瘁更榮,絕統復紀,自微而顯,非願之始。亦惟寵遇,心存足止。重值文皇,建號東宮,乃作少傅,光華益隆。明明聖嗣,至德謙崇,禮遇兼加,惟渥惟豐。哀哀先臣,念竭其忠,洪恩未報,委世以終。嗟臣蔑賤,惟昆及弟,幸生幸育,託綜遺體。過庭旣訓,頑蔽難啟。堂構弗克,志存耦耕。豈悟聖朝,仁澤流盈。追錄先臣,愍其無成,是濟是拔,被以殊榮。珝忝千里,受命南征,旌旗備物,金革揚聲。及臣斯陋,實闇實微,旣顯前軌,人物之機;復傅東宮,繼世荷煇,才不逮先,是忝是違。乾德博好,文雅是貴,追悼亡臣,兾存遺類。如何愚胤,曾無髣髴!瞻彼舊寵,顧此頑虛,孰能忍媿,臣實與居。夙夜反側,克心自論,父子兄弟,累世蒙恩,死惟結草,生誓投身,雖則灰隕,無報萬分。」
 
 
是歲,何定建議鑿聖谿以通江淮,皓令瑩督萬人往,遂以多盤石難施功,罷還,出為武昌左部督。後定被誅,皓追聖谿事,下瑩獄,徙廣州。右國史華覈上疏曰:「臣聞五帝三王皆立史官,叙錄功美,垂之無窮。漢時司馬遷、班固,咸命世大才,所撰精妙,與六經俱傳。大吳受命,建國南土。大皇帝末年,命太史令丁孚、郎中項峻始撰吳書。孚、峻俱非史才,其所撰作,不足紀錄。至少帝時,更差韋曜、周昭、薛瑩、梁廣及臣五人,訪求往事,所共撰立,備有本末。昭、廣先亡,曜負恩蹈罪,瑩出為將,復以過徙,其書遂委滯,迄今未撰奏。臣愚淺才劣,適可為瑩等記注而已,若使撰合,必襲孚、峻之跡,懼墜大皇帝之元功,損當世之盛美。瑩涉學旣博,文章尤妙,同寮之中,瑩為冠首。今者見吏,雖多經學,記述之才,如瑩者少,是以慺慺為國惜之。實欲使卒垂成之功,編於前史之末。奏上之後,退填溝壑,無所復恨。」皓遂召瑩還,為左國史。頃之,選曹尚書同郡繆禕以執意不移,為羣小所疾,左遷衡陽太守。旣拜,又追以職事見詰責,拜表陳謝。因過詣瑩,復為人所白,云禕不懼罪,多將賔客會聚瑩許。乃收禕下獄,徙桂陽,瑩還廣州。未至,召瑩還,復職。是時法政多謬,舉措煩苛,瑩每上便宜,陳緩刑簡役,以濟育百姓,事或施行。遷光祿勳。天紀四年,晉軍征皓,皓奉書於司馬伷、王渾、王濬請降,其文,瑩所造也。瑩旣至洛陽,特先見叙,為散騎常侍,荅問處當,皆有條理。
 \gezhu{干寶晉紀曰:武帝從容問瑩曰:「孫皓之所以亡者何也?」瑩對曰:「歸命侯臣皓之君吳也,昵近小人,刑罰妄加,大臣大將,無所親信,人人憂恐,各不自保,危亡之釁,實由於此。」帝遂問吳士存亡者之賢愚,瑩各以狀對。}
 太康三年卒。著書八篇,名曰新議。
 \gezhu{王隱晉書曰:瑩子兼,字令長,清素有器宇,資望故如上國,不似吳人。歷位二宮丞相長史。元帝踐阼,累遷丹楊尹、尚書,又為太子少傅。自綜至兼,三世傅東宮。}
 
 
 
 
 評曰:張紘文理意正,為世令器,孫策待之亞於張昭,誠有以也。嚴、程、闞生,一時儒林也。至畯辭榮濟舊,不亦長者乎!薛綜學識規納,為吳良臣。及瑩纂蹈,允有先風,然於暴酷之朝,屢登顯列,君子殆諸。
 
 
\end{pinyinscope}