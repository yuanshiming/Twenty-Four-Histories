\article{文聘傳}
\begin{pinyinscope}
 
 
 文聘字仲業,南陽宛人也,為劉表大將,使禦北方。表死,其子琮立。太祖征荊州,琮舉州降,呼聘欲與俱,聘曰:「聘不能全州,當待罪而已。」太祖濟漢,聘乃詣太祖,太祖問曰:「來何遲邪?」聘曰:「先日不能輔弼劉荊州以奉國家,荊州雖沒,常願據守漢川,保全土境,生不負於孤弱,死無愧於地下,而計不得已,以至於此。實懷悲慙,無顏早見耳。」遂欷歔流涕。太祖為之愴然曰:「仲業,卿真忠臣也。」厚禮待之。授聘兵,使與曹純追討劉備於長阪。太祖先定荊州,江夏與吳接,民心不安,乃以聘為江夏太守,使典北兵,委以邊事,賜爵關內侯。
 
 
\gezhu{孫盛曰:資父事君,忠孝道一。臧霸少有孝烈之稱,文聘著垂泣之誠,是以魏武一面,委之以二方之任,豈直壯武見知於倉卒之間哉!}
 與樂進討關羽於尋口,有功,進封延壽亭侯,加討逆將軍。又攻羽輜重於漢津,燒其船於荊城。文帝踐阼,進爵長安鄉侯,假節。與夏侯尚圍江陵,使聘別屯沔口,止石梵,自當一隊,禦賊有功,遷後將軍,封新野侯。孫權以五萬衆自圍聘於石陽,甚急,聘堅守不動,權住二十餘日乃解去。聘追擊破之。
 \gezhu{魏略曰:孫權嘗自將數萬衆卒至。時大雨,城柵崩壞,人民散在田野,未及補治。聘聞權到,不知所施,乃思惟莫若潛默可以疑之。乃勑城中人使不得見,又自卧舍中不起。權果疑之,語其部黨曰:「北方以此人忠臣也,故委之以此郡,今我至而不動,此不有密圖,必當有外救。」遂不敢攻而去。魏略此語,與本傳反。}
 增邑五百戶,并前千九百戶。
 
 
 
 
 聘在江夏數十年,有威恩,名震敵國,賊不敢侵。分聘戶邑封聘子岱為列侯,又賜聘從子厚爵關內侯。聘薨,謚曰壯侯。岱又先亡,聘養子休嗣。卒,子武嗣。
 
 
 
 
 嘉平中,譙郡桓禺為江夏太守,清儉有威惠,名亞於聘。
 
 
\end{pinyinscope}