\article{韓當傳}
\begin{pinyinscope}
 
 
 韓當字義公,遼西令支人也。
 
 
\gezhu{令音郎定反。支音巨兒反。}
 以便弓馬,有膂力,幸於孫堅,從征伐周旋,數犯危難,陷敵擒虜,為別部司馬。
 \gezhu{吳書曰:當勤苦有功,以軍旅陪隷,分於英豪,故爵位不加。終於堅世,為別部司馬。}
 及孫策東渡,從討三郡,遷先登校尉,授兵二千,騎五十匹。從征劉勳,破黃祖,還討鄱陽,領樂安長,山越畏服。後以中郎將與周瑜等拒破曹公,又與呂蒙襲取南郡,遷偏將軍,領永昌太守。宜都之役,與陸遜、朱然等共攻蜀軍於涿鄉,大破之,徙威烈將軍,封都亭侯。曹真攻南郡,當保東南。在外為帥,厲將士同心固守,又敬望督司,奉遵法令,權善之。黃武二年,封石城侯,遷昭武將軍,領冠軍太守,後又加都督之號。將敢死及解煩兵萬人,討丹楊賊,破之。會病卒,子綜襲侯領兵。
 
 
其年,權征石陽,以綜有憂,使守武昌,而綜淫亂不軌。權雖以父故不問,綜內懷懼,
 \gezhu{吳書曰:綜欲叛,恐左右不從,因諷使劫略,示欲饒之,轉相放效,為行旅大患。後因詐言被詔,以部曲為寇盜見詰讓,云「將吏以下,當並收治」,又言恐罪自及。左右因曰:「惟當去耳。」遂共圖計,以當葬父,盡呼親戚姑姊,悉以嫁將吏,所幸婢妾,皆賜與親近,殺牛飲酒歃血,與共盟誓。}
 載父喪,將母家屬部曲男女數千人奔魏。魏以為將軍,封廣陽侯。數犯邊境,殺害人民,權常切齒。東興之役,綜為前鋒,軍敗身死,諸葛恪斬送其首,以白權廟。
 
 
\end{pinyinscope}