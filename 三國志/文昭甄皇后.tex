\article{文昭甄皇后}
\begin{pinyinscope}
 
 
 文昭甄皇后,中山無極人,明帝母,漢太保甄邯後也,世吏二千石。父逸,上蔡令。后三歲失父。
 
 
\gezhu{魏書曰:逸娶常山張氏,生三男五女:長男豫,早終;次儼,舉孝廉,大將軍掾、曲梁長;次堯,舉孝廉;長女姜,次脫,次道,次榮,次即后。后以漢光和五年十二月丁酉生。每寑寐,家中髣髴見如有人持玉衣覆其上者,常共怪之。逸薨,加號慕,內外益奇之。後相者劉良相后及諸子,良指后曰:「此女貴乃不可言。」后自少至長,不好戲弄。年八歲,外有立騎馬戲者,家人諸姊皆上閣觀之,后獨不行。諸姊怪問之,后荅言:「此豈女人之所觀邪?」年九歲,喜書,視字輙識,數用諸兄筆硯,兄謂后言:「汝當習女工。用書為學,當作女博士邪?」后荅言:「聞古者賢女,未有不學前世成敗,以為己誡。不知書,何由見之?」}
 後天下兵亂,加以饑饉,百姓皆賣金銀珠玉寶物,時后家大有儲穀,頗以買之。后年十餘歲,白母曰:「今世亂而多買寶物,匹夫無罪,懷璧為罪。又左右皆饑乏,不如以穀振給親族鄰里,廣為恩惠也。」舉家稱善,即從后言。
 \gezhu{魏略曰:后年十四,喪中兄儼,悲哀過制,事寡嫂謙敬,事處其勞,拊養儼子,慈愛甚篤。后母性嚴,待諸婦有常,后數諫母:「兄不幸早終,嫂年少守節,顧留一子,以大義言之,待之當如婦,愛之宜如女。」母感后言流涕,便令后與嫂共止,寢息坐起常相隨,恩愛益密。}
 
 
建安中,袁紹為中子熈納之。熈出為幽州,后留養姑。及兾州平,文帝納后於鄴,有寵,生明帝及東鄉公主。
 \gezhu{魏略曰:熈出在幽州,后留侍姑。及鄴城破,紹妻及后共坐皇堂上。文帝入紹舍,見紹妻及后,后怖,以頭伏姑膝上,紹妻兩手自搏。文帝謂曰:「劉夫人云何如此?令新婦舉頭!」姑乃捧后令仰,文帝就視,見其顏色非凡,稱歎之。太祖聞其意,遂為迎取。世語曰:太祖下鄴,文帝先入袁尚府,有婦人被髮垢靣,垂涕立紹妻劉後,文帝問之,劉荅「是熈妻」,顧擥髮髻,以巾拭面,姿貌絕倫。旣過,劉謂后「不憂死矣」!遂見納,有寵。魏書曰:后寵愈隆而彌自挹損,後宮有寵者勸勉之,其無寵者慰誨之,每因閑宴,常勸帝,言「昔黃帝子孫蕃育,蓋由妾媵衆多,乃獲斯祚耳。所願廣求淑媛,以豐繼嗣。」帝心嘉焉。其後帝欲遣任氏,后請於帝曰:「任旣鄉黨名族,德、色,妾等不及也,如何遣之?」帝曰:「任性狷急不婉順,前後忿吾非一,是以遣之耳。」后流涕固請曰:「妾受敬遇之恩,衆人所知,必謂任之出,是妾之由。上懼有見私之譏,下受專寵之罪,願重留意!」帝不聽,遂出之。十六年十月,太祖征關中,武宣皇后從,留孟津,帝居守鄴。時武宣皇后體小不安,后不得定省,憂怖,晝夜泣涕;左右驟以差問告,后猶不信,曰:「夫人在家,故疾每動,輙歷時,今疾便差,何速也?此欲慰我意耳!」憂愈甚。後得武宣皇后還書,說疾已平復,后乃懽恱。十七年正月,大軍還鄴,后朝武宣皇后,望幄座悲喜,感動左右。武宣皇后見后如此,亦泣,且謂之曰:「新婦謂吾前病如昔時困邪?吾時小小耳,十餘日即差,不當視我顏色乎!」嘆嗟曰:「此真孝婦也。」二十一年,太祖東征,武宣皇后、文帝及明帝、東鄉公主皆從,時后以病留鄴。二十二年九月,大軍還,武宣皇后左右侍御見后顏色豐盈,怪問之曰:「后與二子別乆,下流之情,不可為念,而后顏色更盛,何也?」后笑荅之曰:「叡等自隨夫人,我當何憂!」后之賢明以禮自持如此。}
 延康元年正月,文帝即王位,六月,南征,后留鄴。黃初元年十月,帝踐阼。踐阼之後,山陽公奉二女以嬪于魏,郭后、李、陰貴人並愛幸,后愈失意,有怨言。帝大怒,二年六月,遣使賜死,葬于鄴。
 \gezhu{魏書曰:有司奏建長秋宮,帝璽書迎后,詣行在所,后上表曰:「妾聞先代之興,所以饗國乆長,垂祚後嗣,無不由后妃焉。故必審選其人,以興內教。令踐阼之初,誠宜登進賢淑,統理六宮。妾自省愚陋,不任粢盛之事,加以寢疾,敢守微志。」璽書三至而后三讓,言甚懇切。時盛暑,帝欲須秋涼乃更迎后。會后疾遂篤,夏六月丁卯,崩于鄴。帝哀痛咨嗟,策贈皇后璽綬。臣松之以為春秋之義,內大惡諱,小惡不書。文帝之不立甄氏,及加殺害,事有明審。魏史若以為大惡邪,則宜隱而不言,若謂為小惡邪,則不應假為之辭,而崇飾虛文乃至於是,異乎所聞於舊史。推此而言,其稱卞、甄諸后言行之善,皆難以實論。陳氏刪落,良有以也。}
 
 
明帝即位,有司奏請追謚,使司空王朗持節奉策以太牢告祠于陵,又別立寢廟。
 \gezhu{魏書載三公奏曰:「蓋孝敬之道,篤乎其親,乃四海所以承化,天地所以明察,是謂生則致其養,歿則光其靈,誦述以盡其美,宣揚以顯其名者也。今陛下以聖懿之德,紹承洪業,至孝烝烝,通於神明,遭離殷憂,每勞謙讓。先帝遷神山陵,大禮旣備,至於先后,未有顯謚。伏惟先后恭讓著於幽微,至行顯於不言,化流邦國,德侔二南,故能膺神靈嘉祥,為大魏世妃。雖夙年登遐,萬載之後,永播融烈,后妃之功莫得而尚也。案謚法:『聖聞周達曰昭。德明有功曰昭。』昭者,光明之至,盛乆而不昧者也。宜上尊謚曰文昭皇后。」是月,三公又奏曰:「自古周人始祖后稷,又特立廟以祀姜嫄。今文昭皇后之於萬嗣,聖德至化,豈有量哉!夫以皇家世妃之尊,而克讓允恭,固推盛位,神靈遷化,而無寢廟以承享禮,非所以報顯德,昭孝敬也。稽之古制,宜依周禮,先妣別立寢廟。」並奏可之。}
 太和元年三月,以中山魏昌之安城鄉戶千,追封逸,謚曰敬侯;適孫像襲爵。四月,初營宗廟,掘地得玉璽,方一寸九分,其文曰「天子羨思慈親」,明帝為之改容,以太牢告廟。又甞夢見后,於是差次舅氏親疏高下,叙用各有差,賞賜累鉅萬;以像為虎賁中郎將。是月,后母薨,帝制緦服臨喪,百僚陪位。四年十一月,以后舊陵庳下,使像兼太尉,持節詣鄴,昭告后土,十二月,改葬朝陽陵。像還,遷散騎常侍。青龍二年春,追謚后兄儼曰安城鄉穆侯。夏,吳賊寇揚州,以像為伏波將軍,持節監諸將東征,還,復為射聲校尉。三年薨,追贈衞將軍,改封魏昌縣,謚曰貞侯;子暢嗣。又封暢弟溫、𩋾、豔皆為列侯。四年,改逸、儼本封皆曰魏昌侯,謚因故。封儼世婦劉為東鄉君,又追封逸世婦張為安喜君。
 
 
 
 
 景初元年夏,有司議定七廟。冬,又奏曰:「蓋帝王之興,旣有受命之君,又有聖妃恊于神靈,然後克昌厥世,以成王業焉。昔高辛氏卜其四妃之子皆有天下,而帝摯、陶唐、商、周代興。周人上推后稷,以配皇天,追述王初,本之姜嫄,特立宮廟,世世享甞,周禮所謂『奏夷則,歌中呂,舞大濩,以享先妣』者也。詩人頌之曰:『厥初生民,時維姜嫄。』言王化之本,生民所由。又曰:『閟宮有侐,實實枚枚,赫赫姜嫄,其德不回。』詩、禮所稱姬宗之盛,其美如此。大魏期運,繼于有虞,然崇弘帝道,三世彌隆,廟祧之數,實與周同。今武宣皇后、文德皇后各配無窮之祚,至於文昭皇后膺天靈符,誕育明聖,功濟生民,德盈宇宙,開諸後嗣,乃道化之所興也。寢廟特祀,亦姜嫄之閟宮也,而未著不毀之制,懼論功報德之義,萬世或闕焉,非所以昭孝示後世也。文昭廟宜世世享祀奏樂,與祖廟同,永著不毀之典,以播聖善之風。」於是與七廟議並勒金策,藏之金匱。
 
 
帝思念舅氏不已。暢尚幼,景初末,以暢為射聲校尉,加散騎常侍,又特為起大第,車駕親自臨之。又於其後園為像母起觀廟,名其里曰渭陽里,以追思母氏也。嘉平三年正月,暢薨,追贈車騎將軍,謚曰恭侯;子紹嗣。太和六年,明帝愛女淑薨,追封謚淑為平原懿公主,為之立廟。取后亡從孫黃與合葬,追封黃列侯,以夫人郭氏從弟德為之後,承甄氏姓,封德為平原侯,襲公主爵。
 \gezhu{孫盛曰:於禮,婦人旣無封爵之典,況於孩末,而可建以大邑乎?德自異族,援繼非類,匪功匪親,而襲母爵,違情背典,於此為甚。陳羣雖抗言,楊阜引事比並,然皆不能極陳先王之禮,明封建繼嗣之義,忠至之辭,猶有闕乎!詩云:「赫赫師尹,民具爾瞻。」宰輔之職,其可略哉!晉諸公贊曰:德字彥孫。司馬景王輔政,以女妻德。妻早亡,文王復以女繼室,即京兆長公主。景、文二王欲自結於郭后,是以頻繁為婚。德雖無才學,而恭謹謙順。甄溫字仲舒,與郭建及德等皆后族,以事宜見寵。咸熈初,封郭建為臨渭縣公,德廣安縣公,邑皆千八百戶。溫本國侯,進為輔國大將軍,加侍中,領射聲校尉,德鎮軍大將軍。泰始元年,晉受禪,加建、德、溫三人位特進。德為人貞素,加以世祖姊夫,是以遂貴當世。德暮年官更轉為宗正,遷侍中。太康中,大司馬齊王攸當之藩,德與左衞將軍王濟共諫請,時人嘉之。世祖以此望德,由此出德為大鴻臚,加侍中、光祿大夫,尋疾薨,贈中軍大將軍,開府侍中如故,謚恭公,子喜嗣。喜精粹有器美,歷中書郎、右衞將軍、侍中,位至輔國大將軍,加散騎常侍。喜與國姻親,而經趙王倫、齊王冏事故,能不豫際會,良由其才短,然亦以退靜免之。}
 青龍中,又封后從兄子毅及像弟三人,皆為列侯。毅數上疏陳時政,官至越騎校尉。嘉平中,復封暢子二人為列侯。后兄儼孫女為齊王皇后,后父已沒,封后母為廣樂鄉君。
 
 
\end{pinyinscope}