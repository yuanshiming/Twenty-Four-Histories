\article{公孫度傳}

\begin{pinyinscope}

公孫度字升濟,本遼東襄平人也。度父延,避吏居玄菟,任度為郡吏。時玄菟太守公孫琙,子豹,年十八歲,早死。度少時名豹,又與琙子同年,琙見而親愛之,遣就師學,為取妻。後舉有道,除尚書郎,稍遷兾州刺史,以謠言免。同郡徐榮為董卓中郎將,薦度為遼東太守。度起玄菟小吏,為遼東郡所輕。先時,屬國公孫昭守襄平令,召度子康為伍長。度到官,收昭,笞殺於襄平市。郡中名豪大姓田韶等宿遇無恩,皆以法誅,所夷滅百餘家,郡中震慄。東伐高句驪,西擊烏丸,威行海外。初平元年,度知中國擾攘,語所親吏柳毅、陽儀等曰:「漢祚將絕,當與諸卿圖王耳。」
\gezhu{魏書曰:度語毅、儀:「讖書云孫登當為天子,太守姓公孫,字升濟,升即登也。」}
時襄平延里社生大石,長丈餘,下有三小石為之足。或謂度曰:「此漢宣帝冠石之祥,而里名與先君同。社主土地,明當有土地,而三公為輔也。」度益喜。故河內太守李敏,郡中知名,惡度所為,恐為所害,乃將家屬入于海。度大怒,掘其父冢,剖棺焚屍,誅其宗族。
\gezhu{晉陽秋曰:敏子追求敏,出塞,越二十餘年不娶。州里徐邈責之曰:「不孝莫大於無後,何可終身不娶乎!」乃娶妻,生子胤而遣妻,常如居喪之禮,不勝憂,數年而卒。胤生不識父母,及有識,蔬食哀戚亦如三年之喪。以祖父不知存亡,設主奉之。由是知名,仕至司徒。}
\gezhu{臣松之案:本傳云敏將家入海,而復與子相失,未詳其故。}
分遼東郡為遼西、中遼郡,置太守。越海收東萊諸縣,置營州刺史。自立為遼東侯、平州牧,追封父延為建義侯。立漢二祖廟,承制設壇墠於襄平城南,郊祀天地,籍田,治兵,乘鸞路,九旒,旄頭羽騎。太祖表度為武威將軍,封永寧鄉侯,度曰:「我王遼東,何永寧也!」藏印綬武庫。度死,子康嗣位,以永寧鄉侯封弟恭。是歲建安九年也。

十二年,太祖征三郡烏丸,屠柳城。袁尚等奔遼東,康斬送尚首。語在武紀。封康襄平侯,拜左將軍。康死,子晃、淵等皆小,衆立恭為遼東太守。文帝踐阼,遣使即拜恭為車騎將軍、假節,封平郭侯;追贈康大司馬。

初,恭病陰消為閹人,劣弱不能治國。太和二年,淵脅奪恭位。明帝即拜淵揚烈將軍、遼東太守。淵遣使南通孫權,往來賂遺。
\gezhu{吳書載淵表權曰:「臣伏惟遭天地反易,遇無妄之運;王路未夷,傾側擾攘。自先人以來,歷事漢、魏,階緣際會,為國效節,繼世享任,得守藩表,猶知符命未有攸歸。每感厚恩,頻辱顯使,退念人臣交不越境,是以固守所執,拒違前使。雖義無二信,敢忘大恩!陛下鎮撫,長存小國,前後裴校尉、葛都尉等到,奉被勑誡,聖旨彌密,重紈累素,幽明備著,所以申示之事,言提其耳。臣晝則謳吟,宵則發夢,終身誦之,志不知足。季末凶荒,乾坤否塞,兵革未戢,人民蕩析。仰此天命將有眷顧,私從一隅永瞻雲日。今魏家不能採錄忠善,襃功臣之後,乃令讒譌得行其志,聽幽州刺史、東萊太守誑誤之言,猥興州兵,圖害臣郡。臣不負魏,而魏絕之。蓋聞人臣有去就之分;田饒適齊,樂毅走趙,以不得事主,故保有道之君;陳平、耿況亦覩時變,卒歸於漢,勒名帝籍。伏惟陛下德不再出,時不世遇,是以慺慺懷慕自納,望遠視險,有如近易。誠願神謨蚤定洪業,奮六師之勢,收河、洛之地,為聖代宗。天下幸甚!」}
\gezhu{魏略曰:國家知淵兩端,而思遼東吏民為淵所誤。故公文下遼東,因赦之曰:「告遼東、玄菟將校吏民:逆賊孫權遭遇亂階,因其先人劫略州郡,遂成羣凶,自擅江表,含垢藏疾。兾其可化,故割地王權,使南面稱孤,位以上將,禮以九命。權親叉手,北向稽顙。假人臣之寵,受人臣之榮,未有如權者也。狼子野心,告令難移,卒歸反覆,背恩叛主,滔天逆神,乃敢僭號。恃江湖之險阻,王誅未加。比年已來,復遠遣船,越渡大海,多持貨物,誑誘邊民。邊民無知,與之交關。長吏以下莫肯禁止,至使周賀浮舟百艘,沈滯津岸,貿遷有無。旣不疑拒,齎以名馬,又使宿舒隨賀通好。十室之邑,猶有忠信,陷君於惡,春秋所書也。今遼東、玄菟奉事國朝,紆青拖紫,以千百為數,戴纚垂纓,咸佩印綬,曾無匡正納善之言。龜玉毀於匵,虎兕出於匣,是誰之過歟?國朝為子大夫羞之!昔狐突有言:『父教子貳,何以事君?策名委質,貳乃辟也。』今乃阿順邪謀,脅從姦惑,豈獨父兄之教不詳,子弟之舉習非而已哉!若苗穢害田,隨風烈火,芝艾俱焚,安能自別乎?且又此事固然易見,不及鑒古成敗,書傳所載也。江南海北有萬里之限,遼東君臣無怵惕之患,利則利所不利,貴則義所不貴,此為厭安樂之居,求危亡之禍,賤忠貞之節,重背叛之名。蠻、貊之長,猶知愛禮,以此事人,亦難為顏!且又宿舒無罪,擠使入吳,奉不義之使,始與家訣,涕泣而行。及至賀死之日,覆衆成山,舒雖脫死,魂魄離身。何所逼迫,乃至於此!今忠臣烈將,咸忿遼東反覆攜貳,皆欲乘桴浮海,期於肆意。朕為天下父母,加念天下新定,旣不欲勞動干戈,遠涉大川,費役如彼,又悼邊陲遺餘黎民,迷誤如此,故遣郎中衞慎、邵瑁等且先奉詔示意。若股肱忠良,能效節立信以輔時君,反邪就正以建大功,福莫大焉。儻恐自嫌已為惡逆所見染汙,不敢倡言,永懷伊戚。其諸與賊使交通,皆赦除之,與之更始。」}
權遣使張彌、許晏等,齎金玉珍寶,立淵為燕王。淵亦恐權遠不可恃,且貪貨物,誘致其使,悉斬送彌、晏等首,
\gezhu{魏略載淵表曰:「臣前遣校尉宿舒、郎中令孫綜,甘言厚禮,以誘吳賊。幸賴天道福助大魏,使此賊虜暗然迷惑,違戾羣下,不從衆諫,承信臣言,遠遣船使,多將士卒,來致封拜。臣之所執,得如本志,雖憂罪釁,私懷幸甚。賊衆本號萬人,舒、綜伺察,可七八千人,到沓津。偽使者張彌、許晏與中郎將萬泰、校尉裴潛將吏兵四百餘人,齎文書命服什物,下到臣郡。泰、潛別齎致遺貨物,欲因市馬。軍將賀達、虞咨領餘衆在船所。臣本欲須涼節乃取彌等,而彌等人兵衆多,見臣不便承受吳命,意有猜疑。懼其死作,變態妄生,即進兵圍取,斬彌、晏、泰、潛等首級。其吏從兵衆,皆士伍小人,給使東西,不得自由,面縛乞降,不忍誅殺,輒聽納受,徙充邊城。別遣將韓起等率將三軍,馳行至沓。使領長史柳遠設賔主禮誘請達、咨,三軍潛伏以待其下,又驅羣馬貨物,欲與交市。達、咨懷疑不下,使諸市買者五六百人下,欲交市。起等金鼓始震,鋒矢亂發,斬首三百餘級,被創赴水沒溺者可二百餘人,其散走山谷,來歸降及藏竄饑餓死者,不在數中。得銀印、銅印、兵器、資貨,不可勝數。謹遣西曹掾公孫珩奉送賊權所假臣節、印綬、符策、九錫、什物,及彌等偽節、印綬、首級。」又曰:「宿舒、孫綜前到吳,賊權問臣家內小大,舒、綜對臣有三息,脩別屬亡弟。權敢姦巧,便擅拜命。謹封送印綬、符策。臣雖無昔人洗耳之風,慙為賊權汙損所加,旣行天誅,猶有餘忿。」又曰:「臣父康,昔殺權使,結為讎隙。今乃譎欺,遣使誘致,令權傾心,虛國竭祿,遠命上卿,寵授極位,震動南土,備盡禮數。又權待舒、綜,契闊委曲,君臣上下,畢歡竭情。而令四使見殺,梟示萬里,士衆流離,屠戮津渚,慙恥遠布,痛辱彌天。權之怨疾,將刻肌骨。若天衰其業,使至喪隕,權將內傷憤激而死。若期運未訖,將播毒螫,必恐長虵來為寇害。徐州諸屯及城陽諸郡,與相接近,如有船衆後年向海門,得其消息,乞速告臣,使得備豫。」又曰:「臣門戶受恩,實深實重,自臣承攝即事以來,連被榮寵,殊特無量,分當隕越,竭力致死。而臣狂愚,意計迷闇,不即禽賊,以至見疑。前章表所陳情趣事勢,實但欲罷弊此賊,使困自絕,誠不敢背累世之恩,附僭盜之虜也。而後愛憎之人,緣事加誣,偽生節目,卒令明聽疑於市虎,移恩改愛,興動威怒,幾至沈沒,長為負忝。幸賴慈恩,猶垂三宥,使得補過,解除愆責。如天威遠加,不見假借,早當麋碎,辱先廢祀,何緣自明,建此微功。臣旣喜於事捷,得自申展,悲於疇昔,至此變故,餘怖踊躍,未敢便寧。唯陛下旣崇春日生全之仁,除忿塞隙,抑弭纖介,推今亮往,察臣本心,長令抱戴,銜分三泉。」又曰:「臣被服光榮,恩情未報,而以罪釁,自招譴怒,分當即戮,為衆社戒。所以越典詭常,偽通於吳,誠自念窮迫,報效未立,而為天威督罰所加,長恐奄忽,不得自洗。故敢自闕替廢於一年,遣使誘吳,知其必來,權之求郡,積有年歲,初無倡荅一言之應,今權得使,來必不疑,至此一舉,果如所規,上卿大衆,翕赫豐盛,財貨賂遺,傾國極位,到見禽取,流離死亡千有餘人,滅絕不反。此誠暴猾賊之鋒,摧矜夸之巧,昭示天下,破損其業,足以慙之矣。臣之慺慺念效於國,雖有非常之過,亦有非常之功,願陛下原其踰闕之愆,采其豪毛之善,使得國恩保全終始矣。」}
明帝於是拜淵大司馬,封樂浪公,持節、領郡如故。
\gezhu{魏名臣奏載中領軍夏侯獻表曰:「公孫淵昔年敢違王命,廢絕計貢者,實挾兩端。旣恃阻險,又怙孫權。故敢跋扈,恣睢海外。宿舒親見賊權軍衆府庫,知其弱少不足憑恃,是以決計斬賊之使。又高句麗、濊貊與淵為仇,並為寇鈔。今外失吳援,內有胡寇,心知國家能從陸道,勢不得不懷惶懼之心。因斯之時,宜遣使示以禍福。奉車都尉鬷弘,武皇帝時始奉使命,開通道路。文皇帝即位,欲通使命,遣弘將妻子還歸鄉里,賜其車、牛,絹百匹。弘以受恩,歸死國朝,無有還意,乞留妻子,身奉使命。公孫康遂稱臣妾。以弘奉使稱意,賜爵關內侯。弘性果烈,乃心於國,夙夜拳拳,念自竭效。冠族子孫,少好學問,博通書記,多所關涉,口論速捷,辯而不俗,附依典誥若出胷臆,加仕本郡常在人右,彼方士人素所敬服。若當遣使,以為可使弘行。弘乃自舊土,習其國俗,為說利害,辯足以動其意,明足以見其事,才足以行之,辭足以見信。若其計從,雖酈生之降齊王,陸賈之說尉他,亦無以遠過也。欲進遠路,不宜釋騏驥;將已篤疾,不宜廢扁鵲。願察愚言也。」}
使者至,淵設甲兵為軍陣,出見使者,又數對國中賔客出惡言。
\gezhu{吳書曰:魏遣使者傅容、聶夔拜淵為樂浪公。淵計吏從洛陽還,語淵曰:「使者左駿伯,使皆擇勇力者,非凡人也。」淵由是疑怖。容、夔至,住學館中。淵先以步騎圍之,乃入受拜。容、夔大怖,由是還洛言狀。}
景初元年,乃遣幽州刺史毌丘儉等齎璽書徵淵。淵遂發兵,逆於遼隧,與儉等戰。儉等不利而還。淵遂自立為燕王,置百官有司。遣使者持節,假鮮卑單于璽,封拜邊民,誘呼鮮卑,侵擾北方。
\gezhu{魏書曰:淵知此變非獨出儉,遂為備。遣使謝吳,自稱燕王,求為與國。然猶令官屬上書自直於魏曰:「大司馬長史臣郭昕、參軍臣柳浦等七百八十九人言:奉被今年七月己卯詔書,伏讀懇切,精魄散越,不知身命所當投措!昕等伏自惟省,螻蟻小醜,器非時用,遭值千載,被受公孫淵祖考以來光明之德,惠澤沾渥,滋潤榮華,無尺寸之功,有負乘之累;遂蒙襃獎,登名天府,並以駑蹇附龍託驥,紆青拖紫,飛騰雲梯,感恩惟報,死不擇地。臣等聞明君在上,聽政采言,人臣在下,得無隱情,是以因緣訴讓,冐犯愬寃。郡在藩表,密邇不羈,平昔三州,轉輸費調,以供賞賜,歲用累億,虛耗中國。然猶跋扈,虔劉邊陲,烽火相望,羽檄相逮,城門晝閉,路無行人,州郡兵戈,奔散覆沒。淵祖父度初來臨郡,承受荒殘,開日月之光,建神武之略,聚烏合之民,埽地為業,威震燿于殊俗,德澤被于羣生。遼土之不壞,實度是賴。孔子曰:『微管仲,吾其被髮左衽。』向不遭度,則郡早為丘墟,而民係於虜廷矣。遺風餘愛,永存不朽。度旣薨殂,吏民感慕,欣戴子康,尊而奉之。康踐統洪緒,克壯徽猷,文昭武烈,邁德種仁;乃心京輦,翼翼虔恭,佐國平亂,效績紛紜,功隆事大,勳藏王府。度、康當值武皇帝休明之會,合策名之計,夾輔漢室,降身委質,卑己事魏。匪處小厭大,畏而服焉,乃慕託高風,懷仰盛懿也。武皇帝亦虛心接納,待以不次,功無巨細,每不見忘。又命之曰:『海北土地割以付君,世世子孫實得有之。』皇天后土,實聞德音。臣庶小大豫在下風,奉以周旋,不敢失墜。淵生有蘭石之姿,少含愷悌之訓,允文允武,忠惠且直;生民欽仰,莫弗懷愛。淵纂戎祖考,君臨萬民,為國以禮,淑化流行,獨見先覩,羅結遐方,勤王之義,視險如夷,世載忠亮,不隕厥名。孫權慕義,不遠萬里,連年遣使,欲自結援,雖見絕殺,不念舊怨,纖纖往來,求成恩好。淵執節彌固,不為利迴,守志匪石,確乎彌堅。猶懼丹心未見保明,乃卑辭厚幣,誘致權使,梟截獻馘,以示無二。吳雖在遠,水道通利,舉帆便至,無所隔限。淵不顧敵讎之深,念存人臣之節,絕彊吳之歡,昭事魏之心,靈祇明鑒,普天咸聞。陛下嘉美洪烈,懿茲武功,誕錫休命,寵亞齊、魯,下及陪臣,普受介福。誠以天覆之恩,當卒終始,得竭股肱,永保祿位,不虞一旦,橫被殘酷。惟育養之厚,念積累之效,悲思不遂,痛切見棄,舉國號咷,拊膺泣血。夫三軍所伐,蠻夷戎狄驕逸不虔,於是致武,不聞義國,反受誅討。蓋聖王之制,五服之域,有不供職,則脩文德,而又不至,然後征伐。淵小心翼翼,恪恭于位,勤事奉上,可謂勉矣。盡忠竭節,還被患禍。小弁之作,離騷之興,皆由此也。就或佞邪,盜言孔甘,猶當清覽,憎而知善;讒巧似直,惑亂聖聽,尚望文告,使知所由。若信有罪,當垂三宥;若不改寤,計功減降,當在八議。而潛軍伺襲,大兵奄至,舞戈長驅,衝擊遼土。犬馬惡死,況於人類!吏民昧死,挫辱王師。淵雖冤枉,方臨危殆,猶恃聖恩,悵然重奔,兾必姦臣矯制,妄肆威虐,乃謂臣等曰:『漢安帝建光元年,遼東屬國都尉龐奮,受二月乙未詔書,曰收幽州刺史馮煥、玄菟太守姚光。推案無乙未詔書,遣侍御史幽州牧考姦臣矯制者。今刺史或儻謬承矯制乎?』臣等議:以為刺史興兵,搖動天下,殆非矯制,必是詔命。淵乃俛仰歎息,自傷無罪。深惟土地所以養人,竊慕古公杖策之岐,乃欲投冠釋紱,逝歸林麓。臣等維持,誓之以死,屯守府門,不聽所執。而七營虎士,五部蠻夷,各懷素飽,不謀同心,奮臂大呼,排門遁出。近郊農民釋其耨鎛,伐薪制梃,改案為櫓,奔馳赴難,軍旅行成,雖蹈湯火,死不顧生。淵雖見孤棄,怨而不怒,比遣勑軍,勿得干犯,及手書告語,懇惻至誠。而吏士凶悍,不可解散,期於畢命,投死無悔。淵懼吏士不從教令,乃躬馳騖,自往化解,僅乃止之。一飯之惠,匹夫所死,況淵累葉信結百姓,恩著民心。自先帝初興,爰曁陛下,榮淵累葉,豐功懿德,策名襃揚,辯著廊廟,勝衣舉履,誦詠明文,以為口實。埋而掘之,古人所恥。小白、重耳衰世諸侯,猶慕著信,以隆霸業。詩美文王作孚萬邦,論語稱仲尼去食存信;信之為德,固亦大矣。今吳、蜀共帝,鼎足而居,天下搖蕩,無所統一,臣等每為陛下懼此危心。淵據金城之固,仗和睦之民,國殷兵強,可以橫行。策名委質,守死善道,忠至義盡,為九州表。方今二敵闚𨵦,未知孰定,是之不戒,而淵是害。茹柔吐剛,非王者之道也。臣等雖鄙,誠竊恥之。若無天乎,臣一郡吉凶,尚未可知;若云有天,亦何懼焉!臣等聞仕於家者,二世則主之,三世則君之。臣等生於荒裔之土,出於圭竇之中,無大援於魏,世隷於公孫氏,報生與賜,在於死力。昔蒯通言直,漢祖赦其誅;鄭詹辭順,晉文原其死。臣等頑愚,不達大節,苟執一介,披露肝膽,言逆龍鱗,罪當萬死。惟陛下恢崇撫育,亮其控告,使疏遠之臣永有保恃。」}


二年春,遣太尉司馬宣王征淵。六月,軍至遼東。
\gezhu{漢晉春秋曰:公孫淵自立,稱紹漢元年。聞魏人將討,復稱臣於吳,乞兵北伐以自救。吳人欲戮其使,羊衜曰:「不可,是肆匹夫之怒而捐霸王之計也。不如因而厚之,遣奇兵潛往以要其成。若魏伐淵不克,而我軍遠赴,是恩結遐夷,義蓋萬里,若兵連不解,首尾離隔,則我虜其傍郡,驅略而歸,亦足以致天之罰,報雪曩事矣。」權曰:「善」。乃勒兵大出。謂淵使曰:「請俟後問,當從簡書,必與弟同休戚,共存亡,雖隕于中原,吾所甘心也。」又曰:「司馬懿所向無前,深為弟憂也。」}
淵遣將軍卑衍、楊祚等步騎數萬屯遼隧,圍塹二十餘里。宣王軍至,令衍逆戰。宣王遣將軍胡遵等擊破之。宣王令軍穿圍,引兵東南向,而急東北,即趨襄平。衍等恐襄平無守,夜走。諸軍進至首山,淵復遣衍等迎軍殊死戰。復擊,大破之,遂進軍造城下,為圍塹。會霖雨三十餘日,遼水暴長,運船自遼口徑至城下。雨霽,起土山、脩櫓,為發石連弩射城中。淵窘急。糧盡,人相食,死者甚多。將軍楊祚等降。八月丙寅夜,大流星長數十丈,從首山東北墜襄平城東南。壬午,淵衆潰,與其子脩將數百騎突圍東南走,大兵急擊之,當流星所墜處斬淵父子。城破,斬相國以下首級以千數,傳淵首洛陽,遼東、帶方、樂浪、玄菟悉平。


初,淵家數有怪,犬冠幘絳衣上屋,炊有小兒蒸死甑中。襄平北巿生肉,長圍各數尺,有頭目口喙,無手足而動搖。占曰:「有形不成,有體無聲,其國滅亡。」始度以中平六年據遼東,至淵三世,凡五十年而滅。
\gezhu{魏略曰:始淵兄晃為恭任子,在洛,聞淵劫奪恭位,謂淵終不可保,數自表聞,欲令國家討淵。帝以淵已秉權,故因而撫之。及淵叛,遂以國法繫晃。晃雖有前言,兾不坐,然內以骨肉,知淵破則己從及。淵首到,晃自審必死,與其子相對啼哭。時上亦欲活之,而有司以為不可,遂殺之。}


\end{pinyinscope}