\article{甘皇后傳}
\begin{pinyinscope}
 
 
 先主甘皇后,沛人也。先主臨豫州,住小沛,納以為妄。先主數喪嫡室,常攝內事。隨先主於荊州,產後主。值曹公軍至,追及先主於當陽長阪,于時困偪,棄后及後主,賴趙雲保護,得免於難。后卒,葬于南郡。章武二年,追謚皇思夫人,遷葬於蜀,未至而先主殂隕。丞相亮上言:「皇思夫人履行脩仁,淑慎其身。大行皇帝昔在上將,嬪配作合,載育聖躬,大命不融。大行皇帝存時,篤義垂恩,念皇思夫人神柩在遠飄颻,特遣使者奉迎。會大行皇帝崩,今皇思夫人神柩以到,又梓宮在道,園陵將成,安厝有期。臣輒與太常臣賴恭等議:禮記曰:『立愛自親始,教民孝也;立敬自長始,教民順也。』不忘其親,所由生也。春秋之義,母以子貴。昔高皇帝追尊太上昭靈夫人為昭靈皇后,孝和皇帝改葬其母梁貴人,尊號曰恭懷皇后,孝愍皇帝亦改葬其母王夫人,尊號曰靈懷皇后。今皇思夫人宜有尊號,以慰寒泉之思,輒與恭等案謚法,宜曰昭烈皇后。詩曰:『穀則異室,死則同穴。』
 
 
\gezhu{禮云:上古無合葬,中古後因時方有。}
 故昭烈皇后宜與大行皇帝合葬,臣請太尉告宗廟,布露天下,具禮儀別奏。」制曰可。
 
 
\end{pinyinscope}