\article{張遼傳}
\begin{pinyinscope}
 
 
 張遼字文遠,鴈門馬邑人也。本聶壹之後,以避怨變姓。少為郡吏。漢末,并州刺史丁原以遼武力過人,召為從事,使將兵詣京都。何進遣詣河北募兵,得千餘人。還,進敗,以兵屬董卓。卓敗,以兵屬呂布,遷騎都尉。布為李傕所敗,從布東奔徐州,領魯相,時年二十八。
 
 
 
 
 太祖破呂布於下邳,遼將其衆降,拜中郎將,賜爵關內侯。數有戰功,遷裨將軍。袁紹破,別遣遼定魯國諸縣。與夏侯淵圍昌豨於東海,數月糧盡,議引軍還,遼謂淵曰:「數日已來,每行諸圍,豨輒屬目視遼。又其射矢更稀,此必豨計猶豫,故不力戰。遼欲挑與語,儻可誘也?」乃使謂豨曰:「公有命,使遼傳之。」豨果下與遼語,遼為說「太祖神武,方以德懷四方,先附者受大賞」。豨乃許降。遼遂單身上三公山,入豨家,拜妻子。豨歡喜,隨詣太祖。太祖遣豨還,責遼曰:「此非大將法也。」遼謝曰:「以明公威信著於四海,遼奉聖旨,豨必不敢害故也。」
 
 
 
 
 從討袁譚、袁尚於黎陽,有功,行中堅將軍。從攻尚於鄴,尚堅守不下。太祖還許,使遼與樂進拔陰安,徙其民河南。復從攻鄴,鄴破,遼別徇趙國、常山,招降緣山諸賊及黑山孫輕等。從攻袁譚,譚破,別將徇海濵,破遼東賊柳毅等。還鄴,太祖自出迎遼,引共載,以遼為盪寇將軍。復別擊荊州,定江夏諸縣,還屯臨潁,封都亭侯。從征袁尚於柳城,卒與虜遇,遼勸太祖戰,氣甚奮,太祖壯之,自以所持麾授遼。遂擊,大破之,斬單于蹋頓。
 
 
\gezhu{傅子曰:太祖將征柳城,遼諫曰:「夫許,天子之會也。今天子在許,公遠北征,若劉表遣劉備襲許,據之以號令四方,公之勢去矣。」太祖策表必不能任備,遂行也。}
 
 
 
 
 時荊州未定,復遣遼屯長社。臨發,軍中有謀反者,夜驚亂起火,一軍盡擾。遼謂左右曰:「勿動。是不一營盡反,必有造變者,欲以動亂人耳。」乃令軍中,其不反者安坐。遼將親兵數十人,中陣而立。有頃定,即得首謀者殺之。陳蘭、梅成以氐六縣叛,太祖遣于禁、臧霸等討成,遼督張郃、牛蓋等討蘭。成偽降禁,禁還。成遂將其衆就蘭,轉入灊山。灊中有天柱山,高峻二十餘里,道險狹,步徑裁通,蘭等壁其上。遼欲進,諸將曰:「兵少道險,難用深入。」遼曰:「此所謂一與一,勇者得前耳。」遂進到山下安營,攻之,斬蘭、成首,盡虜其衆。太祖論諸將功,曰:「登天山,履峻險,以取蘭、成,盪寇功也。」增邑,假節。
 
 
太祖旣征孫權還,使遼與樂進、李典等將七千餘人屯合肥。太祖征張魯,教與護軍薛悌,署函邊曰「賊至乃發」。俄而權率十萬衆圍合肥,乃共發教,教曰:「若孫權至者,張、李將軍出戰;樂將軍守,護軍勿得與戰。」諸將皆疑。遼曰:「公遠征在外,比救至,彼破我必矣。是以教指及其未合逆擊之,折其盛勢,以安衆心,然後可守也。成敗之機,在此一戰,諸君何疑?」李典亦與遼同。於是遼夜募敢從之士,得八百人,椎牛饗將士,明日大戰。平旦,遼被甲持戟,先登陷陣,殺數十人,斬二將,大呼自名,衝壘入,至權麾下。權大驚,衆不知所為,走登高冢,以長戟自守。遼叱權下戰,權不敢動,望見遼所將衆少,乃聚圍遼數重。遼左右麾圍,直前急擊,圍開,遼將麾下數十人得出,餘衆號呼曰:「將軍棄我乎!」遼復還突圍,拔出餘衆。權人馬皆披靡,無敢當者。自旦戰至日中,吳人奪氣,還脩守備,衆心乃安,諸將咸服。權守合肥十餘日,城不可拔,乃引退。遼率諸軍追擊,幾復獲權。太祖大壯遼,拜征東將軍。
 \gezhu{孫盛曰:夫兵固詭道,奇正相資,若乃命將出征,推轂委權,或賴率然之形,或憑掎角之勢,羣帥不和,則棄師之道也。至於合肥之守,縣弱無援,專任勇者則好戰生患,專任怯者則懼心難保。且彼衆我寡,必懷貪墯;以致命之兵,擊貪墯之卒,其勢必勝;勝而後守,守則必固。是以魏武推選方圓,參以同異,為之密教,節宣其用;事至而應,若合符契妙矣!}
 
 
 
 
 建安二十一年,太祖復征孫權,到合肥,循行遼戰處,歎息者良乆。乃增遼兵,多留諸軍,徙屯居巢。關羽圍曹仁於樊,會權稱藩,召遼及諸軍悉還救仁。遼未至,徐晃已破關羽,仁圍解。遼與太祖會摩陂。遼軍至,太祖乘輦出勞之,還屯陳郡。
 
 
文帝即王位,轉前將軍。
 \gezhu{魏書曰:王賜遼帛千匹,穀萬斛。}
 分封兄汎及一子列侯。孫權復叛,遣遼還屯合肥,進遼爵都鄉侯。給遼母輿車,及兵馬送遼家詣屯,勑遼母至,導從出迎。所督諸軍將吏皆羅拜道側,觀者榮之。文帝踐阼,封晉陽侯,增邑千戶,并前二千六百戶。黃初二年,遼朝洛陽宮,文帝引遼會建始殿,親問破吳意狀。帝歎息顧左右曰:「此亦古之邵虎也。」為起第舍,又特為遼母作殿,以遼所從破吳軍應募步卒,皆為虎賁。孫權復稱藩。遼還屯雍丘,得疾。帝遣侍中劉曄將太醫視疾,虎賁問消息,道路相屬。疾未瘳,帝迎遼就行在所,車駕親臨,執其手,賜以御衣,太官日送御食。疾小差,還屯。孫權復叛,帝遣遼乘舟,與曹休至海陵,臨江。權甚憚焉,勑諸將:「張遼雖病,不可當也,慎之!」是歲,遼與諸將破權將呂範。遼病篤,遂薨於江都。帝為流涕,謚曰剛侯。子虎嗣。六年,帝追念遼、典在合肥之功,詔曰:「合肥之役,遼、典以步卒八百,破賊十萬,自古用兵,未之有也。使賊至今奪氣,可謂國之爪牙矣。其分遼、典邑各百戶,賜一子爵關內侯。」虎為偏將軍,薨。子統嗣。
 
 
\end{pinyinscope}