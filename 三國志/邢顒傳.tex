\article{邢顒傳}
\begin{pinyinscope}
 
 
 邢顒,字子昂,河間鄚人也。舉孝廉,司徒辟,皆不就。易姓字,適右北平,從田疇游。積五年,而太祖定兾州。顒謂疇曰:「黃巾起來二十餘年,海內鼎沸,百姓流離。今聞曹公法令嚴。民厭亂矣,亂極則平。請以身先。」遂裝還鄉里。田疇曰:「邢顒,民之先覺也。」乃見太祖,求為鄉導以克柳城。
 
 
 
 
 太祖辟顒為兾州從事,時人稱之曰:「德行堂堂邢子昂。」除廣宗長,以故將喪棄官。有司舉正,太祖曰:「顒篤於舊君,有一致之節。勿問也。」更辟司空掾,除行唐令,勸民農桑,風化大行。入為丞相門下督,遷左馮翊,病,去官。是時,太祖諸子高選官屬,令曰:「侯家吏,宜得淵深法度如邢顒輩。」遂以為平原侯植家丞。顒防閑以禮,無所屈撓,由是不合。庶子劉楨書諫植曰:「家丞邢顒,北土之彥,少秉高節,玄靜澹泊,言少理多,真雅士也。楨誠不足同貫斯人,並列左右。而楨禮遇殊特,顒反疏簡,私懼觀者將謂君侯習近不肖,禮賢不足,採庶子之春華,忘家丞之秋實。為上招謗,其罪不小,以此反側。」後參丞相軍事,轉東曹掾。初,太子未定,而臨菑侯植有寵,丁儀等並贊翼其美。太祖問顒,顒對曰:「以庶代宗,先世之戒也。願殿下深重察之!」太祖識其意,後遂以為太子少傅,遷太傅。文帝踐阼,為侍中尚書僕射,賜爵關內侯,出為司隷校尉,徙太常。黃初四年薨。子友嗣。
 
 
\gezhu{晉諸公贊曰:顒曾孫喬,字曾伯。有體量局幹,美於當世。歷清職。元康中,與劉渙俱為尚書吏部郎,稍遷至司隷校尉。}
 
 
\end{pinyinscope}