\article{潘璋傳}
\begin{pinyinscope}
 
 
 潘璋字文珪,東郡發干人也。孫權為陽羨長,始往隨權。性博蕩嗜酒,居貧,好賒酤,債家至門,輒言後豪富相還。權奇愛之,因使召募,得百餘人,遂以為將。討山賊有功,署別部司馬。後為吳大巿刺姧,盜賊斷絕,由是知名,遷豫章西安長。劉表在荊州,民數被寇,自璋在事,寇不入境。比縣建昌起為賊亂,轉領建昌,加武猛校尉,討治惡民,旬月盡平,召合遺散,得八百人,將還建業。
 
 
 
 
 合肥之役,張遼奄至,諸將不備,陳武鬬死,宋謙、徐盛皆披走,璋身次在後,便馳進,橫馬斬謙、盛兵走者二人,兵皆還戰。權甚壯之,拜偏將軍,遂領百校,屯半州。
 
 
 
 
 權征關羽,璋與朱然斷羽走道,到臨沮,住夾石。璋部下司馬馬忠禽羽,并羽子平、都督趙累等。權即分宜都巫、秭歸二縣為固陵郡,拜璋為太守、振威將軍,封溧陽侯。甘寧卒,又并其軍。劉備出夷陵,璋與陸遜并力拒之,璋部下斬備護軍馮習等,所殺傷甚衆,拜平北將軍、襄陽太守。
 
 
 
 
 魏將夏侯尚等圍南郡,分前部三萬人作浮橋,渡百里洲上,諸葛瑾、楊粲並會兵赴救,未知所出,而魏兵日渡不絕。璋曰:「魏勢始盛,江水又淺,未可與戰。」便將所領,到魏上流五十里,伐葦數百萬束,縛作大筏,欲順流放火,燒敗浮橋。作筏適畢,伺水長當下,尚便引退。璋下備陸口。權稱尊號,拜右將軍。
 
 
 
 
 璋為人麤猛,禁令肅然,好立功業,所領兵馬不過數千,而其所在常如萬人。征伐止頓,便立軍巿,他軍所無,皆仰取足。然性奢泰,末年彌甚,服物僭擬。吏兵富者,或殺取其財物,數不奉法。監司舉奏,權惜其功而輒原不問。嘉禾三年卒。子平,以無行徙會稽。璋妻居建業,賜田宅,復客五十家。
 
 
\end{pinyinscope}