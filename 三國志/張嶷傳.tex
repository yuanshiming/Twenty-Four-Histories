\article{張嶷傳}
\begin{pinyinscope}
 
 
 張嶷字伯岐,巴郡南充國人也。
 
 
\gezhu{益部耆舊傳曰:嶷出自孤微,而少有通壯之節。}
 弱冠為縣功曹。先主定蜀之際,山寇攻縣,縣長捐家逃亡,嶷冒白刃,攜負夫人,夫人得免。由是顯名,州召為從事。時郡內士人龔祿、姚伷位二千石,當世有聲名,皆與嶷友善。建興五年,丞相亮北住漢中,廣漢、緜竹山賊張慕等鈔盜軍資,劫掠吏民,嶷以都尉將兵討之。嶷度其鳥散,難以戰禽,乃詐與和親,克期置酒。酒酣,嶷身率左右,因斬慕等五十餘級,渠帥悉殄。尋其餘類,旬日清泰。後得疾病困篤,家素貧匱,廣漢太守蜀郡何祗,名為通厚,嶷宿與踈闊,乃自轝詣祗,託以治疾。祗傾財醫療,數年除愈。其黨道信義皆此類也。拜為牙門將,屬馬忠,北討汶山叛羌,南平四郡蠻夷,輙有籌畫戰克之功。
 \gezhu{益部耆舊傳曰:嶷受兵三百人,隨馬忠討叛羌。嶷別督數營在先,至他里。邑所在高峻,嶷隨山立上四五里。羌於要厄作石門,於門上施牀,積石於其上,過者下石槌擊之,無不糜爛。嶷度不可得攻,乃使譯告曉之曰:「汝汶山諸種反叛,傷害良善,天子命將討滅惡類。汝等若稽顙過軍,資給糧費,福祿永隆,其報百倍。若終不從,大兵致誅,雷擊電下,雖追悔之,亦無益也。」耆帥得命,即出詣嶷,給糧過軍。軍前討餘種,餘種聞他里已下,悉恐怖失所,或迎軍出降,或奔竄山谷,放兵攻擊,軍以克捷。後南夷劉冑又反,以馬忠為督庲降討冑,嶷復屬焉,戰鬬常冠軍首,遂斬冑。平南事訖,䍧牱興古獠種復反,忠令嶷領諸營往討,嶷內招降得二千人,悉傳詣漢中。}
 十四年,武都氐王符健請降,遣將軍張尉往迎,過期不到,大將軍蔣琬深以為念。嶷平之曰:「符健求附款至,必無他變,素聞健弟狡黠,又夷狄不能同功,將有乖離,是以稽留耳。」數日,問至,健弟果將四百戶就魏,獨健來從。
 
 
 
 
 初,越嶲郡自丞相亮討高定之後,叟夷數反,殺太守龔祿、焦璜,是後太守不敢之郡,只住安定縣,去郡八百餘里,其郡徒有名而已。時論欲復舊郡,除嶷為越嶲太守,嶷將所領往之郡,誘以恩信,蠻夷皆服,頗來降附。北徼捉馬最驍勁,不承節度,嶷乃往討,生縛其帥魏狼,又解縱告喻,使招懷餘類。表拜狼為邑侯,種落三千餘戶皆安土供職。諸種聞之,多漸降服,嶷以功賜爵關內侯。
 
 
 
 
 蘇祁邑君冬逢、逢弟隗渠等,已降復反。嶷誅逢。逢妻,旄牛王女,嶷以計原之。而渠逃入西徼。渠剛猛捷悍,為諸種深所畏憚,遣所親二人詐降嶷,實取消息。嶷覺之,許以重賞,使為反間,二人遂合謀殺渠。渠死,諸種皆安。又斯都耆帥李求承,昔手殺龔祿,嶷求募捕得,數其宿惡而誅之。
 
 
 
 
 始嶷以郡郛宇頹壞,更築小塢。在官三年,徙還故郡,繕治城郭,夷種男女莫不致力。
 
 
 
 
 定莋、臺登、卑水三縣去郡三百餘里,舊出鹽鐵及漆,而夷徼乆自固食。嶷率所領奪取,署長吏焉。嶷之到定莋,定莋率豪狼岑,槃木王舅,甚為蠻夷所信任,忿嶷自侵,不自來詣。嶷使壯士數十直往收致,撻而殺之,持尸還種,厚加賞賜,喻以狼岑之惡,且曰:「無得妄動,動即殄矣!」種類咸靣縛謝過。嶷殺牛饗宴,重申恩信,遂獲鹽鐵,器用周贍。
 
 
 
 
 漢嘉郡界旄牛夷種類四千餘戶,其率狼路,欲為姑壻冬逢報怨,遣叔父離將逢衆相度形勢。嶷逆遣親近齎牛酒勞賜,又令離姊逆逢妻宣暢意旨。離旣受賜,并見其姊,姊弟歡恱,悉率所領將詣嶷,嶷厚加賞待,遣還。旄牛由是輙不為患。
 
 
 
 
 郡有舊道,經旄牛中至成都,旣平且近;自旄牛絕道,已百餘年,更由安上,旣險且遠。嶷遣左右齎貨幣賜路,重令路姑喻意,路乃率兄弟妻子悉詣嶷,嶷與盟誓,開通舊道,千里肅清,復古亭驛。奏封路為旄牛㽛毗王,遣使將路朝貢。後主於是加嶷撫戎將軍,領郡如故。
 
 
 
 
 嶷初見費禕為大將軍,恣性汎愛,待信新附太過,嶷書戒之曰:「昔岑彭率師,來歙杖節,咸見害於刺客,今明將軍位尊權重,宜鑒前事,少以為警。」後禕果為魏降人郭脩所害。
 
 
 
 
 吳太傅諸葛恪以初破魏軍,大興兵衆以圖攻取。侍中諸葛瞻,丞相亮之子,恪從弟也,嶷與書曰:「東主初崩,帝實幼弱,太傅受寄託之重,寄託之重,亦何容易!親以周公之才,猶有管、蔡流言之變,霍光受任,亦有燕、蓋、上官逆亂之謀,賴成、昭之明,以免斯難耳。昔每聞東主殺生賞罰,不任下人,又今以垂沒之命,卒召太傅,屬以後事,誠實可慮。加吳、楚剽急,乃昔所記,而太傅離少主,履敵庭,恐非良計長筭之術也。雖云東家綱紀肅然,上下輯睦,百有一失,非明者之慮邪?取古則今,今則古也,自非郎君進忠言於太傅,誰復有盡言者也!旋軍廣農,務行德惠,數年之中,東西並舉,實為不晚,願深採察。」恪竟以此夷族。嶷識見多如是類。
 
 
在郡十五年,邦域安穆。屢乞求還,乃徵詣成都。夷民戀慕,扶轂泣涕,過旄牛邑,邑君襁負來迎,及追尋至蜀郡界,其督相率隨嶷朝貢者百餘人。嶷至,拜盪寇將軍,慷慨壯烈,士人咸多貴之,然放蕩少禮,人亦以此譏焉,
 \gezhu{益部耆舊傳曰:時車騎將軍夏侯霸謂嶷曰:「雖與足下踈闊,然託心如舊,宜明此意。」嶷荅曰:「僕未知子,子未知我,大道在彼,何云託心乎!願三年之後徐陳斯言。」有識之士以為美談。}
 是歲延熈十七年也。魏狄道長李簡密書請降,衞將軍姜維率嶷等因簡之資以出隴西。
 \gezhu{益部耆舊傳曰:嶷風溼固疾,至都寖篤,扶杖然後能起。李簡請降,衆議狐疑,而嶷曰必然。姜維之出,時論以嶷初還,股疾不能在行中,由是嶷自乞肆力中原,致身敵庭。臨發,辭後主曰:「臣當值聖明,受恩過量,加以疾病在身,常恐一朝隕沒,辜負榮遇。天不違願,得豫戎事。若涼州克定,臣為藩表守將;若有未捷,殺身以報。」後主慨然為之流涕。}
 旣到狄道,簡悉率城中吏民出迎軍。軍前與魏將徐質交鋒,嶷臨陣隕身,然其所殺傷亦過倍。旣亡,封長子瑛西鄉侯,次子護雄襲爵。南土越嶲民夷聞嶷死,無不悲泣,為嶷立廟,四時水旱輙祀之。
 \gezhu{益部耆舊傳曰:余觀張嶷儀貌辭令,不能駭人,而其策略足以入筭,果烈足以立威,為臣有忠誠之節,處類有亮直之風,而動必顧典,後主深崇之。雖古之英士,何以遠踰哉!蜀世譜曰:嶷孫弈,晉梁州刺史。}
 
 
評曰:黃權弘雅思量,李恢公亮志業,呂凱守節不回,馬忠擾而能毅,
 \gezhu{尚書曰:擾而毅。鄭玄注曰:擾,馴也。致果曰毅。}
 王平忠勇而嚴整,張嶷識斷明果,咸以所長,顯名發迹,遇其時也。
 
 
\end{pinyinscope}