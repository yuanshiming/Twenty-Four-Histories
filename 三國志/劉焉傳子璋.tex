\article{劉焉傳子璋}
\begin{pinyinscope}
 
 
 劉焉字君郎,江夏竟陵人也,漢魯恭王之後裔,章帝元和中徙封竟陵,支庶家焉。焉少仕州郡,以宗室拜中郎,後以師祝公喪去官。
 
 
\gezhu{臣松之案:祝公,司徒祝恬也。}
 居陽城山,積學教授,舉賢良方正,辟司徒府,歷雒陽令、兾州刺史、南陽太守、宗正、太常。焉覩靈帝政治衰缺,王室多故,乃建議言:「刺史、太守,貨賂為官,割剥百姓,以致離叛。可選清名重臣以為牧伯,鎮安方夏。」焉內求交阯牧,欲避世難。議未即行,侍中廣漢董扶私謂焉曰:「京師將亂,益州分野有天子氣。」焉聞扶言,意更在益州。會益州刺史郤儉賦斂煩擾,
 \gezhu{儉,郤正祖也。}
 謠言遠聞,而并州殺刺史張壹,涼州殺刺史耿鄙,焉謀得施。出為監軍使者,領益州牧,封陽城侯,當收儉治罪;
 \gezhu{續漢書曰:是時用劉虞為幽州,劉焉為益州,劉表為荊州,賈琮為兾州。虞等皆海內清名之士,或從列卿尚書以選為牧伯,各以本秩居任。舊典:傳車參駕,施赤為帷裳。臣松之案:靈帝崩後,義軍起,孫堅殺荊州刺史王叡,然後劉表為荊州,不與焉同時也。漢靈帝紀曰:帝引見焉,宣示方畧,加以賞賜,勑焉為益州刺史。前刺史劉儁、郤儉皆貪殘放濫,取受狼藉,元元無聊,呼嗟充野,焉到便收攝行法,以示萬姓,勿令漏露,使癰疽決潰,為國生梗。焉受命而行,以道路不通,住荊州東界。}
 扶亦求為蜀郡西部屬國都尉,及太倉令會巴西趙韙去官,俱隨焉。
 \gezhu{陳壽益部耆舊傳曰:董扶字茂安。少從師學,兼通數經,善歐陽尚書,又事聘士楊厚,究極圖讖。遂至京師,游覽太學,還家講授,弟子自遠而至。永康元年,日有蝕之,詔舉賢良方正之士,策問得失。左馮翊趙謙等舉扶,扶以病不詣,遙於長安上封事,遂稱疾篤歸家。前後宰府十辟,公車三徵,再舉賢良方正、博士、有道皆不就,名稱尤重。大將軍何進表薦扶曰:「資游、夏之德,述孔氏之風,內懷焦、董消復之術。方今并、涼騷擾,西戎蠢叛,宜勑公車特召,待以異禮,諮謀奇策。」於是靈帝徵扶,即拜侍中。在朝稱為儒宗,甚見器重。求為蜀郡屬國都尉。扶出一歲而靈帝崩,天下大亂。後去官,年八十二卒于家。始扶發辭抗論,益部少雙,故號曰致止,言人莫能當,所至而談止也。後丞相諸葛亮問秦宓以扶所長,宓曰:「董扶襃秋毫之善,貶纖芥之惡。」}
 
 
是時梁州逆賊馬相、趙祗等於緜竹縣自號黃巾,合聚疲役之民,一二日中得數千人,先殺緜竹令李升,吏民翕集合萬餘人,便前破雒縣,攻益州殺儉,又到蜀郡、犍為,旬月之間,破壞三郡。相自稱天子,衆以萬數。州從事賈龍素領兵數百人在犍為東界,攝斂吏民,得千餘人,攻相等,數日破走,州界清靜。龍乃選吏卒迎焉。焉徙治緜竹,撫納離叛,務行寬惠,陰圖異計。張魯母始以鬼道,又有少容,常往來焉家,故焉遣魯為督義司馬,住漢中,斷絕谷閣,殺害漢使。焉上書言米賊斷道,不得復通,又託他事殺州中豪強王咸、李權等十餘人,以立威刑。
 \gezhu{益部耆舊雜記曰:李權字伯豫,為臨邛長。子福。見犍為楊戲輔臣贊。}
 犍為太守任岐及賈龍由此反攻焉,焉擊殺岐、龍。
 \gezhu{英雄記曰:劉焉起兵,不與天下討董卓,保州自守。犍為太守任岐自稱將軍,與從事陳超舉兵擊焉,焉擊破之。董卓使司徒趙謙將兵向州,說校尉賈龍,使引兵還擊焉,焉出青羌與戰,故能破殺。岐、龍等皆蜀郡人。}
 
 
焉意漸盛,造作乘輿車具千餘乘。荊州牧劉表表上焉有似子夏在西河疑聖人之論。時焉子範為左中郎將,誕治書御史,璋為奉車都尉,皆從獻帝在長安,
 \gezhu{英雄記曰:範父焉為益州牧,董卓所徵發,皆不至。收範兄弟三人,鏁械於郿塢,為陰獄以繫之。}
 惟小子別部司馬瑁素隨焉。獻帝使璋曉諭焉,焉留璋不遣。
 \gezhu{典略曰:時璋為奉車都尉,在京師。焉託疾召璋,璋自表省焉,焉遂留璋不還。}
 時征西將軍馬騰屯郿而反,焉及範與騰通謀,引兵襲長安。範謀泄,奔槐里,騰敗,退還涼州,範應時見殺,於是收誕行刑。
 \gezhu{英雄記曰:範從長安亡之馬騰營,從焉求兵。焉使校尉孫肇將兵往助之,敗於長安。}
 議郎河南龐羲與焉通家,乃募將焉諸孫入蜀。時焉被天火燒城,車具蕩盡,延及民家。焉徙治成都,旣痛其子,又感祅灾,興平元年,癕疽發背而卒。州大吏趙韙等貪璋溫仁,共上璋為益州刺史,詔書因以為監軍使者,領益州牧,以韙為征東中郎將,率衆擊劉表。
 \gezhu{英雄記曰:焉死,子璋代為刺史。會長安拜潁川扈瑁為刺史,入漢中。荊州別駕劉闔,璋將沈彌、婁發、甘寧反,擊璋不勝,走入荊州。璋使趙韙進攻荊州,屯朐䏰。上蠢,下如振反。}
 
 
璋,字季玉,旣襲焉位,而張魯稍驕恣,不承順璋,璋殺魯母及弟,遂為讎敵。璋累遣龐羲等攻魯,數為所破。魯部曲多在巴西,故以羲為巴西太守,領兵禦魯。
 \gezhu{英雄記曰:龐羲與璋有舊,又免璋諸子於難,故璋厚德羲,以羲為巴西太守,遂專權勢。}
 後羲與璋情好攜隙,趙韙稱兵內向,衆散見殺,皆由璋明斷少而外言入故也。
 \gezhu{英雄記曰:先是,南陽、三輔人流入益州數萬家,收以為兵,名曰東州兵。璋性寬柔,無威略,東州人侵暴舊民,璋不能禁,政令多闕,益州頗怨。趙韙素得人心,璋委任之。韙因民怨謀叛,乃厚賂荊州請和,陰結州中大姓,與俱起兵,還擊璋。蜀郡、廣漢、犍為皆應韙。璋馳入成都城守,東州人畏韙,咸同心并力助璋,皆殊死戰,遂破反者,進攻韙於江州。韙將龐樂、李異反殺韙軍,斬韙。漢獻帝春秋曰:漢朝聞益州亂,遣五官中郎將牛亶為益州刺史;徵璋為卿,不至。}
 璋聞曹公征荊州,已定漢中,遣河內陰溥致敬於曹公。加璋振威將軍,兄瑁平寇將軍。瑁狂疾物故。
 \gezhu{臣松之案:魏臺訪「物故」之義,高堂隆荅曰:「聞之先師:物,無也;故,事也;言無復所能於事也。」}
 璋復遣別駕從事蜀郡張肅送叟兵三百人并雜御物於曹公,曹公拜肅為廣漢太守。璋復遣別駕張松詣曹公,曹公時已定荊州,走先主,不復存錄松,松以此怨。會曹公軍不利於赤壁,兼以疫死。松還,疵毀曹公,勸璋自絕,
 \gezhu{漢書春秋曰:張松見曹公,曹公方自矜伐,不存錄松。松歸,乃勸璋自絕。習鑿齒曰:昔齊桓一矜其功而叛者九國,曹操暫自驕伐而天下三分,皆勤之於數十年之內而棄之於俯仰之頃,豈不惜乎!是以君子勞謙日仄,慮以下人,功高而居之以讓,勢尊而守之以卑。情近於物,故雖貴而人不猒其重;德洽羣生,故業廣而天下愈欣其慶。夫然,故能有以富貴,保其功業,隆顯當時,傳福百世,何驕矜之有哉!君子是以知曹操之不能遂兼天下者也。}
 因說璋曰:「劉豫州,使君之肺腑,可與交通。」璋皆然之,遣法正連好先主,尋又令正及孟達送兵數千助先主守禦,正遂還。後松復說璋曰:「今州中諸將龐羲、李異等皆恃功驕豪,欲有外意,不得豫州,則敵攻其外,民攻其內,必敗之道也。」璋又從之,遣法正請先主。璋主簿黃權陳其利害,從事廣漢王累自倒縣於州門以諫,璋一無所納,勑在所供奉先主,先主入境如歸。先主至江州北,由墊江水
 \gezhu{墊音徒協反。}
 詣涪,
 \gezhu{音浮。}
 去成都三百六十里,是歲建安十六年也。璋率步騎三萬餘人,車乘帳幔,精光耀日,往就與會;先主所將將士,更相之適,歡飲百餘日。璋資給先主,使討張魯,然後分別。
 \gezhu{吳書曰:璋以米二十萬斛,騎千匹,車千乘,繒絮錦帛,以資送劉備。}
 
 
明年,先主至葭萌,還兵南向,所在皆克。十九年,進圍成都數十日,城中尚有精兵三萬人,穀帛支一年,吏民咸欲死戰。璋言:「父子在州二十餘年,無恩德以加百姓。百姓攻戰三年,肌膏草野者,以璋故也,何心能安!」遂開城出降,羣下莫不流涕。先主遷璋于南郡公安,盡歸其財物、故佩振威將軍印綬。孫權殺關羽,取荊州,以璋為益州牧,駐秭歸。璋卒,南中豪率雍闓據益郡反,附於吳。權復以璋子闡為益州刺史,處交、益界首。丞相諸葛亮平南土,闡還吳,為御史中丞。
 \gezhu{吳書云闡一名緯,為人恭恪,輕財愛義,有仁讓之風,後疾終於家。}
 
 
 
 
 初,璋長子循妻,龐羲女也。先主定蜀,羲為左將軍司馬,璋時從羲啟留循,先主以為奉車中郎將。是以璋二子之後,分在吳、蜀。
 
 
評曰:昔魏豹聞許負之言則納薄姬於室,
 \gezhu{孔衍漢魏春秋曰:許負,河內溫縣之婦人,漢高祖封為明雌亭侯。臣松之以為今東人呼母為負,衍以許負為婦人,如為有似,然漢高祖時封皆列侯,未有鄉亭之爵,疑此封為不然。}
 劉歆見圖讖之文則名字改易,終於不免其身,而慶鍾二主。此則神明不可虛要,天命不可妄兾,必然之驗也。而劉焉聞董扶之辭則心存益土,聽相者之言則求婚吳氏,遽造輿服,圖竊神器,其惑甚矣。璋才非人雄,而據土亂世,負乘致寇,自然之理,其見奪取,非不幸也。
 \gezhu{張璠曰:劉璋愚弱而守善言,斯亦宋襄公、徐偃王之徒,未為無道之主也。張松、法正,雖有君臣之義不正,然固已委名附質,進不顯陳事勢,若韓嵩、劉先之說劉表,退不告絕奔亡,若陳平、韓信之去項羽,而兩端攜貳,為謀不忠,罪之次也。}
 
 
\end{pinyinscope}