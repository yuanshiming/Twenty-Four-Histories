\article{濮陽興傳}
\begin{pinyinscope}
 
 
 濮陽興字子元,陳留人也。父逸,漢末避亂江東,官至長沙太守。
 
 
\gezhu{逸事見陸瑁傳。}
 興少有士名,孫權時除上虞令,稍遷至尚書左曹,以五官中郎將使蜀,還為會稽太守。時琅邪王休居會稽,興深與相結。及休即位,徵興為太常衞將軍、平軍國事,封外黃侯。
 
 
 
 
 永安三年,都尉嚴密建丹楊湖田,作浦里塘。詔百官會議,咸以為用功多而田不保成,唯興以為可成。遂會諸兵民就作,功傭之費不可勝數,士卒死亡,或自賊殺,百姓大怨之。
 
 
 
 
 興遷為丞相。與休寵臣左將軍張布共相表裹,邦內失望。
 
 
 
 
 七年七月,休薨。左典軍萬彧素與烏程侯孫皓善,乃勸興、布,於是興、布廢休適子而迎立皓,皓旣踐阼,加興侍中,領青州牧。俄彧譖興、布追悔前事。十一月朔入朝,皓因收興、布,徙廣州,道追殺之,夷三族。
 
 
 
 
 評曰:諸葛恪才氣幹略,邦人所稱,然驕且吝,周公無觀,況在於恪?矜己陵人,能無敗乎!若躬行所與陸遜及弟融之書,則悔吝不至,何尤禍之有哉?滕胤厲脩士操,遵蹈規矩,而孫峻之時猶保其貴,必危之理也。峻、綝凶豎盈溢,固無足論者。濮陽興身居宰輔,慮不經國,協張布之邪,納萬彧之說,誅夷其宜矣。
 
 
\end{pinyinscope}