\article{呂虔傳}
\begin{pinyinscope}
 
 
 呂虔字子恪,任城人也。太祖在兖州,聞虔有膽策,以為從事,將家兵守湖陸。襄陵校尉杜松部民炅母等作亂,與昌豨通。太祖以虔代松。虔到,招誘炅母渠率及同惡數十人,賜酒食。簡壯士伏其側,虔察炅母等皆醉,使伏兵盡格殺之。撫其餘衆,羣賊乃平。太祖以虔領泰山太守。郡接山海,世亂,聞民人多藏竄。袁紹所置中郎將郭祖、公孫犢等數十輩,保山為寇,百姓苦之。虔將家兵到郡,開恩信,祖等黨屬皆降服,諸山中亡匿者盡出安土業。簡其彊者補戰士,泰山由是遂有精兵,冠名州郡。濟南黃巾徐和等,所在劫長吏,攻城邑。虔引兵與夏侯淵會擊之,前後數十戰,斬首獲生數千人。太祖使督青州諸郡兵以討東萊羣賊李條等,有功。太祖令曰:「夫有其志,必成其事,蓋烈士之所徇也。卿在郡以來,禽姦討暴,百姓獲安,躬蹈矢石,所征輒克。昔寇恂立名於汝、潁,耿弇建策於青、兖,古今一也。」舉茂才,加騎都尉,典郡如故。虔在泰山十數年,甚有威惠。文帝即王位,加裨將軍,封益壽亭侯,遷徐州刺史,加威虜將軍。請琅邪王祥為別駕,民事一以委之,世多其能任賢。
 
 
\gezhu{孫盛雜語曰:祥字休徵。性至孝,後母苛虐,每欲危害祥,祥色養無怠。盛寒之月,後母曰:「吾思食生魚。」祥脫衣,將剖冰求之,有少,堅冰解,下有魚躍出,因奉以供,時人以為孝感之所致也。供養三十餘年,母終乃仕,以淳誠貞粹見重於時。王隱晉書曰:祥始出仕,年過五十矣,稍遷至司隷校尉。高貴鄉公入學,以祥為三老,遷司空太尉。司馬文王初為晉王,司空荀顗要祥盡敬,祥不從。語在三少帝紀。晉武踐阼,拜祥為太保,封雎陵公。泰始四年,年八十九薨。祥弟覽,字玄通,光祿大夫。晉諸公贊稱覽率素有至行。覽子孫繁衍,頗有賢才相係,弈世之盛,古今少比焉。}
 討利城叛賊,斬獲有功。明帝即位,徙封萬年亭侯,增邑二百,并前六百戶。虔薨,子翻嗣。翻薨,子桂嗣。
 
 
\end{pinyinscope}