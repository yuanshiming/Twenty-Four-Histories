\article{滿寵}
\begin{pinyinscope}
 
 
 滿寵字伯寧,山陽昌邑人也。年十八,為郡督郵。時郡內李朔等各擁部曲,害于平民,太守使寵糾焉。朔等請罪,不復鈔略。守高平令。縣人張苞為郡督郵,貪穢受取,干亂吏政。寵因其來在傳舍,率吏卒出收之,詰責所犯,即日考竟,遂棄官歸。
 
 
 
 
 太祖臨兖州,辟為從事。及為大將軍,辟署西曹屬,為許令。時曹洪宗室親貴,有賔客在界,數犯法,寵收治之。洪書報寵,寵不聽。洪白太祖,太祖召許主者。寵知將欲原,乃速殺之。太祖喜曰:「當事不當耳邪?」故太尉楊彪收付縣獄,尚書令荀彧、少府孔融等並屬寵:「但當受辭,勿加考掠。」寵一無所報,考訊如法。數日,求見太祖,言之曰:「楊彪考訊無他辭語。當殺者宜先彰其罪;此人有名海內,若罪不明,必大失民望,竊為明公惜之。」太祖即日赦出彪。初,彧、融聞考掠彪,皆怒,及因此得了,更善寵。
 
 
\gezhu{臣松之以為楊公積德之門,身為名臣,縱有愆負,猶宜保祐,況淫刑所濫,而可加其楚掠乎?若理應考訊,荀、孔二賢豈其妄有相請屬哉?寵以此為能,酷吏之用心耳。雖有後善,何解前虐?}
 
 
 
 
 時袁紹盛於河朔,而汝南紹之本郡,門生賔客布在諸縣,擁兵拒守。太祖憂之,以寵為汝南太守。寵募其服從者五百人,率攻下二十餘壁,誘其未降渠帥,於坐上殺十餘人,一時皆平。得戶二萬,兵二千人,令就田業。
 
 
 
 
 建安十三年,從太祖征荊州。大軍還,留寵行奮威將軍,屯當陽。孫權數擾東陲,復召寵還為汝南太守,賜爵關內侯。關羽圍襄陽,寵助征南將軍曹仁屯樊城拒之,而左將軍于禁等軍以霖雨水長為羽所沒。羽急攻樊城,樊城得水,往往崩壞,衆皆失色。或謂仁曰:「今日之危,非力所支。可及羽圍未合,乘輕船夜走,雖失城,尚可全身。」寵曰:「山水速疾,兾其不久。聞羽遣別將已在郟下,自許以南,百姓擾擾,羽所以不敢遂進者,恐吾軍掎其後耳。今若遁去,洪河以南,非復國家有也;君宜待之。」仁曰:「善。」寵乃沈白馬,與軍人盟誓。會徐晃等救至,寵力戰有功,羽遂退。進封安昌亭侯。文帝即王位,遷揚武將軍。破吳於江陵有功,更拜伏波將軍,屯新野。大軍南征,到精湖,寵帥諸軍在前,與賊隔水相對。寵勑諸將曰:「今夕風甚猛,賊必來燒軍,宜為其備。」諸軍皆警。夜半,賊果遣十部伏夜來燒,寵掩擊破之,進封南鄉侯。黃初三年,假寵節鉞。五年,拜前將軍。
 
 
明帝即位,進封昌邑侯。太和二年,領豫州刺史。三年春,降人稱吳大嚴,揚聲欲詣江北獵,孫權欲自出。寵度其必襲西陽而為之備,權聞之,退還。秋,使曹休從廬江南入合肥,令寵向夏口。寵上疏曰:「曹休雖明果而希用兵,今所從道,背湖旁江,易進難退,此兵之窪地也。若入無彊口,宜深為之備。」寵表未報,休遂深入。賊果從無彊口斷夾石,要休還路。休戰不利,退走。會朱靈等從後來斷道,與賊相遇。賊驚走,休軍乃得還。是歲休薨,寵以前將軍代都督揚州諸軍事。汝南兵民戀慕,大小相率,奔隨道路,不可禁止。護軍表上,欲殺其為首者。詔使寵將親兵千人自隨,其餘一無所問。四年,拜寵征東將軍。其冬,孫權揚聲欲至合肥,寵表召兖、豫諸軍,皆集。賊尋退還,被詔罷兵。寵以為今賊大舉而還,非本意也,此必欲偽退以罷吾兵,而倒還乘虛,掩不備也,表不罷兵。後十餘日,權果更來,到合肥城,不克而還。其明年,吳將孫布遣人詣揚州求降,辭云:「道遠不能自致,乞兵見迎。」刺史王淩騰布書,請兵馬迎之。寵以為必詐,不與兵,而作報書曰:「知識邪正,欲避禍就順,去暴歸道,甚相嘉尚。今欲遣兵相迎,然計兵少則不足相衞,多則事必遠聞。且先密計以成本志,臨時節度其宜。」寵會被書當入朝,勑留府長史:「若淩欲往迎,勿與兵也。」淩於後索兵不得,乃單遣一督將步騎七百人往迎之。布夜掩擊,督將迸走,死傷過半。初,寵與淩共事不平,淩支黨毀寵疲老悖謬,故明帝召之。旣至,體氣康彊,見而遣還。
 \gezhu{世語曰:王淩表寵年過耽酒,不可居方任。帝將召寵,給事中郭謀曰:「寵為汝南太守、豫州刺史二十餘年,有勳方岳。及鎮淮南,吳人憚之。若不如所表,將為所闚。可令還朝,問以方事以察之。」帝從之。寵旣至,進見,飲酒至一石不亂。帝慰勞之,遣還。}
 寵屢表求留,詔報曰:「昔廉頗彊食,馬援據鞌,今君未老而自謂已老,何與廉、馬之相背邪?其思安邊境,惠此中國。」
 
 
 
 
 明年,吴將陸遜向廬江,論者以為宜速赴之。寵曰:「廬江雖小,將勁兵精,守則經時。又賊舍船二百里來,後尾空縣,尚欲誘致,今宜聽其遂進,但恐走不可及耳。」整軍趨楊宜口。賊聞大兵東下,即夜遁。時權歲有來計。青龍元年,寵上疏曰:「合肥城南臨江湖,北遠壽春,賊攻圍之,得據水為勢;官兵救之,當先破賊大輩,然後圍乃得解。賊往甚易,而兵往救之甚難,宜移城內之兵,其西三十里,有奇險可依,更立城以固守,此為引賊平地而掎其歸路,於計為便。」護軍將軍蔣濟議,以為:「旣示天下以弱,且望賊煙火而壞城,此為未攻而自拔。一至於此,劫略無限,必以淮北為守。」帝未許。寵重表曰:「孫子言,兵者,詭道也。故能而示之以弱不能,驕之以利,示之以懾。此為形實不必相應也。又曰『善動敵者形之』。今賊未至而移城却內,此所謂形而誘之也。引賊遠水,擇利而動,舉得於外,則福生於內矣。」尚書趙咨以寵策為長,詔遂報聽。其年,權自出,欲圍新城,以其遠水,積二十日不敢下舡。寵謂諸將曰:「權得吾移城,必於其衆中有自大之言,今大舉來欲要一切之功,雖不敢至,必當上岸耀兵以示有餘。」乃潛遣步騎六千,伏肥城隱處以待之。權果上岸耀兵,寵伏軍卒起擊之,斬首數百,或有赴水死者。明年,權自將號十萬,至合肥新城。寵馳往赴,募壯士數十人,折松為炬,灌以麻油,從上風放火,燒賊攻具,射殺權弟子孫泰。賊於是引退。三年春,權遣兵數千家佃於江北。至八月,寵以為田向收孰,男女布野,其屯衞兵去城遠者數百里,可掩擊也。遣長史督三軍循江東下,摧破諸屯,焚燒穀物而還。詔羙之,因以所獲盡為將士賞。
 
 
景初二年,以寵年老徵還,遷為太尉。寵不治產業,家無餘財。詔曰:「君典兵在外,專心憂公,有行父、祭遵之風。賜田十頃,糓五百斛,錢二十萬,以明清忠儉約之節焉。」寵前後增邑,凡九千六百戶,封子孫二人亭侯。正始三年薨,謚曰景侯。子偉嗣。偉以格度知名,官至衞尉。
 \gezhu{世語曰:偉字公衡。偉子長武,有寵風,年二十四,為大將軍掾。高貴鄉公之難,以掾守閶闔掖門,司馬文王弟安陽亭侯幹欲入。幹妃,偉妹也。長武謂幹曰:「此門近,公且來,無有入者,可從東掖門。」幹遂從之。文王問幹入何遲,幹言其故。參軍王羨亦不得入,恨之。旣而羨因王左右啟王,滿掾斷門不內人,宜推劾。壽春之役,偉從文王至許,以疾不進。子從,求還省疾,事定乃從歸,由此內見恨。收長武考死杖下,偉免為庶人。時人冤之。偉弟子奮,晉元康中至尚書令、司隷校尉。寵、偉、長武、奮,皆長八尺。荀綽兾州記曰:奮性清平,有識檢。晉諸公讚曰:奮體量通雅,有寵風也。}
 
 
\end{pinyinscope}