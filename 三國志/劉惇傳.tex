\article{劉惇傳}
\begin{pinyinscope}
 
 
 劉惇字子仁,平原人也。遭亂避地,客遊廬陵,事孫輔。以明天官達占數顯於南土。每有水旱寇賊,皆先時處期,無不中者。輔異焉,以為軍師,軍中咸敬事之,號曰神明。
 
 
 
 
 建安中,孫權在豫章,時有星變,以問惇,惇曰:「災在丹楊。」權曰:「何如?」曰:「客勝主人,到其日當得問。」是時邊鴻作亂,卒如惇言。
 
 
 
 
 惇於諸術皆善,尤明太一,皆能推演其事,窮盡要妙,著書百餘篇,名儒刁玄稱以為奇。惇亦寶愛其術,不以告人,故世莫得而明也。
 
 
\end{pinyinscope}