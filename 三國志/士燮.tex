\article{士燮}
\begin{pinyinscope}
 
 
 士燮字威彥,蒼梧廣信人也。其先本魯國汶陽人,至王莽之亂,避地交州。六世至燮父賜,桓帝時為日南太守。燮少游學京師,事潁川劉子奇,治左氏春秋。察孝廉,補尚書郎,公事免官。父賜喪闋後,舉茂才,除巫令,遷交阯太守。
 
 
 
 
 弟壹,初為郡督郵。刺史丁宮徵還京都,壹侍送勤恪,宮感之,臨別謂曰:「刺史若待罪三事,當相辟也。」後宮為司徒,辟壹。比至,宮已免,黃琬代為司徒,甚禮遇壹。董卓作亂,壹亡歸鄉里。
 
 
\gezhu{吳書曰:琬與卓相害,而壹盡心於琬,甚有聲稱。卓惡之,乃署教曰:「司徒掾士壹,不得除用。」故歷年不遷。會卓入關,壹乃亡歸。}
 交州刺史朱符為夷賊所殺,州郡擾亂。燮乃表壹領合浦太守,次弟徐聞令䵋領九真太守,
 \gezhu{䵋音于鄙反,見字林。}
 䵋弟武,領南海太守。
 
 
 
 
 燮體器寬厚,謙虛下士,中國士人往依避難者以百數。耽玩春秋,為之注解。陳國袁徽與尚書令荀彧書曰:「交阯士府君旣學問優博,又達於從政,處大亂之中,保全一郡,二十餘年疆埸無事,民不失業,羈旅之徒,皆蒙其慶,雖竇融保河西,曷以加之?官事小闋,輒玩習書傳,春秋左氏傳尤簡練精微,吾數以咨問傳中諸疑,皆有師說,意思甚密。又尚書兼通古今,大義詳備。聞京師古今之學是非忿爭,今欲條左氏、尚書長義上之。」其見稱如此。
 
 
燮兄弟並為列郡,雄長一州,偏在萬里,威尊無上。出入鳴鍾磬,備具威儀,笳簫鼓吹,車騎滿道,胡人夾轂焚燒香者常有數十。妻妾乘輜軿,子弟從兵騎,當時貴重,震服百蠻,尉他不足踰也。
 \gezhu{葛洪神仙傳曰:燮甞病死,已三日,仙人董奉以一丸藥與服,以水含之,捧其頭,搖捎之,食頃,即開目動手,顏色漸復,半日能起坐,四日復能語,遂復常。奉字昌異,侯官人也。}
 武先病沒。
 
 
 
 
 朱符死後,漢遣張津為交州刺史,津後又為其將區景所殺,而荊州牧劉表遣零陵賴恭代津。是時蒼梧太守史璜死,表又遣吳巨代之,與恭俱至。漢聞張津死,賜燮璽書曰:「交州絕域,南帶江海,上恩不宣,下義壅隔,知逆賊劉表又遣賴恭闚看南土,今以燮為綏南中郎將,董督七郡,領交阯太守如故。」後燮遣吏張旻奉貢詣京都,是時天下喪亂,道路斷絕,而燮不廢貢職,特復下詔拜安遠將軍,封龍度亭侯。後巨與恭相失,舉兵逐恭,恭走還零陵。
 
 
 
 
 建安十五年,孫權遣步隲為交州刺史。隲到,燮率兄弟奉承節度。而吳巨懷異心,隲斬之。權加燮為左將軍。建安末年,燮遣子廞入質,權以為武昌太守,燮、壹諸子在南者,皆拜中郎將。燮又誘導益州豪姓雍闓等,率郡人民使遙東附,權益嘉之,遷衞將軍,封龍編侯,弟壹偏將軍,都鄉侯。燮每遣使詣權,致雜香細葛,輒以千數,明珠、大貝、流離、翡翠、瑇瑁、犀、象之珍,奇物異果,蕉、邪、龍眼之屬,無歲不至。壹時貢馬凡數百匹。權輒為書,厚加寵賜,以荅慰之。燮在郡四十餘歲,黃武五年,年九十卒。
 
 
權以交阯縣遠,乃分合浦以北為廣州,呂岱為刺史;交阯以南為交州,戴良為刺史。又遣陳時代燮為交阯太守。岱留南海,良與時俱前行到合浦,而燮子徽自署交阯太守,發宗兵拒良。良留合浦。交阯桓鄰,燮舉吏也,叩頭諫徽使迎良,徽怒,笞殺鄰。鄰兄治子發又合宗兵擊徽,徽閉門城守,治等攻之數月不能下,乃約和親,各罷兵還。而呂岱被詔誅徽,自廣州將兵晝夜馳入,過合浦,與良俱前。壹子中郎將匡與岱有舊,岱署匡師友從事,先移書交阯,告喻禍福,又遣匡見徽,說令服罪,雖失郡守,保無他憂。岱尋匡後至,徽兄祗,弟幹、頌等六人肉袒奉迎。岱謝令復服,前至郡下。明旦早施帳幔,請徽兄弟以次入,賔客滿坐。岱起,擁節讀詔書,數徵罪過,左右因反縛以出,即皆伏誅,傳首詣武昌。
 \gezhu{孫盛曰:夫柔遠能邇,莫善於信;保大定功,莫善於義。故齊桓創基,德彰於柯會;晉文始伯,義顯於伐原。故能九合一匡,世主夏盟,令問長世,貽範百王。呂岱師友士匡,使通信誓,徽兄弟肉袒,推心委命,岱因滅之,以要功利,君子是以知孫權之不能遠略,而呂氏之祚不延者也。}
 壹、䵋、匡後出,權原其罪,及燮質子廞,皆免為庶人。數歲,壹、䵋坐法誅。廞病卒,無子,妻寡居,詔在所月給俸米,賜錢四十萬。
 
 
 
 
 評曰:劉繇藻厲名行,好尚臧否,至於擾攘之時,據萬里之土,非其長也。太史慈信義篤烈,有古人之分。士燮作守南越,優游終世,至子不慎,自貽凶咎,蓋庸才玩富貴而恃阻險,使之然也。
 
 
\end{pinyinscope}