\article{陸績傳}
\begin{pinyinscope}
 
 
 陸績字公紀,吳郡吳人也。父康,漢末為廬江太守。
 
 
\gezhu{謝承後漢書曰:康字季寧,少惇孝悌,勤脩操行,太守李肅察孝廉。肅後坐事伏法,康斂尸送喪還潁川,行服,禮終,舉茂才,歷三郡太守,所在稱治,後拜廬江太守。}
 績年六歲,於九江見袁術。術出橘,績懷三枚,去,拜辭墮地,術謂曰:「陸郎作賔客而懷橘乎?」績跪荅曰:「欲歸遺母。」術大奇之。孫策在吳,張昭、張紘、秦松為上賔,共論四海未泰,須當用武治而平之,績年少末坐,遙大聲言曰:「昔管夷吾相齊桓公,九合諸侯,一匡天下,不用兵車。孔子曰:『遠人不服,則脩文德以來之。』今論者不務道德懷取之術,而惟尚武,績雖童蒙,竊所未安也。」昭等異焉。
 
 
績容貌雄壯,博學多識,星歷筭數無不該覽。虞翻舊齒名盛,龐統荊州令士,年亦差長,皆與績友善。孫權統事,辟為奏曹掾,以直道見憚,出為鬱林太守,加偏將軍,給兵二千人。績旣有躄疾,又意在儒雅,非其志也。雖有軍事,著述不廢,作渾天圖,注易釋玄,皆傳於世。豫自知亡日,乃為辭曰:「有漢志士吳郡陸績,幼敦詩、書,長玩禮、易,受命南征,遘疾遇厄,遭命不幸,嗚呼悲隔!」又曰:「從今已去,六十年之外,車同軌,書同文,恨不及見也。」年三十二卒。長子宏,會稽南部都尉,次子叡,長水校尉。
 \gezhu{績於鬱林所生女,名曰鬱生,適張溫弟白。姚信集有表稱之曰:「臣聞唐、虞之政,舉善而教,旌德擢異,三王所先,是以忠臣烈士,顯名國朝,淑婦貞女,表迹家閭。蓋所以闡崇化業,廣殖清風,使苟有令性,幽明俱著,苟懷懿姿,士女同榮。故王蠋建寒松之節而齊王表其里,義姑立殊絕之操而魯侯高其門。臣切見故鬱林太守陸績女子鬱生,少履貞特之行,幼立匪石之節,年始十三,適同郡張白。侍廟三月,婦禮未卒,白遭罹家禍,遷死異郡。鬱生抗聲昭節,義形於色,冠蓋交橫,誓而不許,奉白姊妹嶮巇之中,蹈履水火,志懷霜雪,義心固於金石,體信貫於神明,送終以禮,邦士慕則。臣聞昭德以行,顯行以爵,苟非名爵,則勸善不嚴,故士之有誄,魯人志其勇,杞婦見書,齊人哀其哭。乞蒙聖朝,斟酌前訓,上開天聦,下垂坤厚,襃鬱生以義姑之號,以厲兩髦之節,則皇風穆暢,士女改視矣。」}
 
 
\end{pinyinscope}