\article{崔琰傳}
\begin{pinyinscope}


崔琰字季珪,清河東武城人也。少樸訥,好擊劒,尚武事。年二十三,鄉移為正,始感激,讀論語、韓詩。至年二十九,乃結公孫方等就鄭玄受學。學未朞,徐州黃巾賊攻破北海,玄與門人到不其山避難。時穀糴縣乏,玄罷謝諸生。琰旣受遣,而寇盜充斥,西道不通。於是周旋青、徐、兖、豫之郊,東下壽春,南望江、湖。自去家四年乃歸,以琴書自娛。




大將軍袁紹聞而辟之。時士卒橫暴,掘發丘壠,琰諫曰:「昔孫卿有言:『士不素教,甲兵不利,雖湯武不能以戰勝。』今道路暴骨,民未見德,宜勑郡縣掩骼埋胔,示憯怛之愛,追文王之仁。」紹以為騎都尉。後紹治兵黎陽,次于延津,琰復諫曰:「天子在許,民望助順,不如守境述職,以寧區宇。」紹不聽,遂敗于官渡。及紹卒,二子交爭,爭欲得琰。琰稱疾固辭,由是獲罪,幽於囹圄,賴陰夔、陳琳營救得免。




太祖破袁氏,領兾州牧,辟琰為別駕從事,謂琰曰:「昨案戶籍,可得三十萬衆,故為大州也。」琰對曰:「今天下分崩,九州幅裂,二袁兄弟親尋干戈,兾方蒸庶暴骨原野。未聞王師仁聲先路,存問風俗,救其塗炭,而校計甲兵,唯此為先,斯豈鄙州士女所望於明公哉!」太祖改容謝之。于時賔客皆伏失色。




太祖征并州,留琰傅文帝於鄴。世子仍出田獵,變易服乘,志在驅逐。琰書諫曰:「蓋聞盤于游田,書之所戒,魯隱觀魚,春秋譏之,此周、孔之格言,二經之明義。殷鑒夏后,詩稱不遠,子卯不樂,禮以為忌,此又近者之得失,不可不深察也。袁族富彊,公子寬放,盤游滋侈,義聲不聞,哲人君子,俄有色斯之志,熊羆壯士,墯於吞噬之用,固所以擁徒百萬,跨有河朔,無所容足也。今邦國殄瘁,惠康未洽,士女企踵,所思者德。况公親御戎馬,上下勞慘,世子宜遵大路,慎以行正,思經國之高略,內鑒近戒,外揚遠節,深惟儲副,以身為寶。而猥襲虞旅之賤服,忽馳騖而陵險,志雉兎之小娛,忘社稷之為重,斯誠有識所以惻心也。唯世子燔翳捐褶,以塞衆望,不令老臣獲罪於天。」世子報曰:「昨奉嘉命,惠示雅數,欲使燔翳捐褶,翳已壞矣,褶亦去焉。後有此比,蒙復誨諸。」




太祖為丞相,琰復為東西曹掾屬徵事。初授東曹時,教曰:「君有伯夷之風,史魚之直,貪夫慕名而清,壯士尚稱而厲,斯可以率時者已。故授東曹,往踐厥職。」魏國初建,拜尚書。時未立太子,臨菑侯植有才而愛。太祖狐疑,以函令密訪於外。唯琰露板荅曰:「蓋聞春秋之義,立子以長,加五官將仁孝聦明,宜承正統。琰以死守之。」植,琰之兄女壻也。太祖貴其公亮,喟然歎息,


\gezhu{世語曰:植妻衣繡,太祖登臺見之,以違制命,還家賜死。}
遷中尉。


琰聲姿高暢,眉目疏朗,鬚長四尺,甚有威重,朝士瞻望,而太祖亦敬憚焉。
\gezhu{先賢行狀曰:琰清忠高亮,雅識經遠,推方直道,正色於朝。魏氏初載,委授銓衡,總齊清議,十有餘年。文武羣才,多所明拔。朝廷歸高,天下稱平。}
琰甞薦鉅鹿楊訓,雖才好不足,而清貞守道,太祖即禮辟之。後太祖為魏王,訓發表稱贊功伐,襃述盛德。時人或笑訓希世浮偽,謂琰為失所舉。琰從訓取表草視之,與訓書曰:「省表,事佳耳!時乎時乎,會當有變時。」琰本意譏論者好譴呵而不尋情理也。有白琰此書傲世怨謗者,太祖怒曰:「諺言『生女耳』,『耳』非佳語。『會當有變時』,意指不遜。」於是罰琰為徒隷,使人視之,辭色不撓。太祖令曰:「琰雖見刑,而通賔客,門若市人,對賔客虬鬚直視,若有所瞋。」遂賜琰死。
\gezhu{魏略曰:人得琰書,以裹幘籠,行都道中。時有與琰宿不平者,遙見琰名著幘籠,從而視之,遂白之。太祖以為琰腹誹心謗,乃收付獄,髠刑輸徒。前所白琰者又復白之云:「琰為徒,虬鬚直視,心似不平。」時太祖亦以為然,遂欲殺之。乃使清公大吏往經營琰,勑吏曰:「三日期消息。」琰不悟,後數日,吏故白琰平安。公忿然曰:「崔琰必欲使孤行刀鋸乎!」吏以是教告琰,琰謝吏曰:「我殊不宜,不知公意至此也!」遂自殺。}


始琰與司馬朗善,晉宣王方壯,琰謂朗曰:「子之弟,聦哲明允,剛斷英跱,殆非子之所及也。」
\gezhu{臣松之案:「跱」或作「特」,竊謂「英特」為是也。}
朗以為不然,而琰每秉此論。琰從弟林,少無名望,雖姻族猶多輕之,而琰常曰:「此所謂大器晚成者也,終必遠至。」涿郡孫禮、盧毓始入軍府,琰又名之曰:「孫疏亮亢烈,剛簡能斷,盧清警明理,百鍊不消,皆公才也。」後林、禮、毓咸至鼎輔。及琰友人公孫方、宋階早卒,琰撫其遺孤,恩若己子。其鑒識篤義,類皆如此。
\gezhu{魏略曰:明帝時,崔林甞與司空陳羣共論兾州人士,稱琰為首。羣以「智不存身」貶之。林曰:「大丈夫為有邂逅耳,即如卿諸人,良足貴乎!」}


初,太祖性忌,有所不堪者,魯國孔融、
\gezhu{融字文舉。續漢書:融,孔子二十世孫也。高祖父尚,鉅鹿太守。父宙,太山都尉。融幼有異才。時河南尹李膺有重名,勑門下簡通賔客,非當世英賢及通家子孫弗見也。融年十餘歲,欲觀其為人,遂造膺門,語門者曰:「我,李君通家子孫也。」膺見融,問曰:「高明父祖,甞與僕周旋乎?」融曰:「然。先君孔子與君先人李老君,同德比義而相師友,則融與君累世通家也。」衆坐奇之,僉曰:「異童子也。」太中大夫陳煒後至,同坐以告煒,煒曰:「人小時了了者,大亦未必奇也。」融荅曰:「即如所言,君之幼時,豈實慧乎!」膺大笑,顧謂曰:「高明長大,必為偉器。」山陽張儉,以中正為中常侍侯覽所忿疾,覽為刊章下州郡捕儉。儉與融兄襃有舊,亡投襃。遇襃出,時融年十六,儉以其少不告也。融知儉長者,有窘迫色,謂曰:「吾獨不能為君主邪!」因留舍藏之。後事泄,相國以下密就掩捕,儉得脫走,登時收融及襃送獄。融曰:「保納藏舍者融也,融當坐之。」襃曰:「彼來求我,罪我之由,非弟之過,我當坐之。」兄弟爭死,郡縣疑不能決,乃上讞,詔書令襃坐焉。融由是名震遠近,與平原陶丘洪、陳留邊讓,並以俊秀,為後進冠蓋。融持論經理不及讓等,而逸才宏博過之。司徒大將軍辟舉高第,累遷北軍中候、虎賁中郎將、北海相,時年二十八。承黃巾殘破之後,脩復城邑,崇學校,設庠序,舉賢才,顯儒士。以彭璆為方正,邴原為有道,王脩為孝廉。告高密縣為鄭玄特立一鄉,名為鄭公鄉。又國人無後,及四方游士有死亡者,皆為棺木而殯葬之。郡人甄子然孝行知名,早卒,融恨不及之,乃令配食縣社。其禮賢如此。在郡六年,劉備表融領青州刺史。建安元年,徵還為將作大匠,遷少府。每朝會訪對,輙為議主,諸卿大夫寄名而已。}
\gezhu{司馬彪九州春秋曰:融在北海,自以智能優贍,溢才命世,當時豪俊皆不能及。亦自許大志,且欲舉軍曜甲,與羣賢要功,自於海岱結殖根本,不肯碌碌如平居郡守,事方伯、赴期會而已。然其所任用,好奇取異,皆輕剽之才。至於稽古之士,謬為恭敬,禮之雖備,不與論國政也。高密鄭玄,稱之鄭公,執子孫禮。及高談教令,盈溢官曹,辭氣溫雅,可玩而誦。論事考實,難可悉行。但能張磔網羅,其自理甚疏。租賦少稽,一朝殺五部督郵。姦民汙吏,猾亂朝市,亦不能治。幽州精兵亂,至徐州,卒到城下,舉國皆恐。融直出說之,令無異志。遂與別校謀夜覆幽州,幽州軍敗,悉有其衆。無幾時,還復叛亡。黃巾將至,融大飲醇酒,躬自上馬,禦之淶水之上。寇令上部與融相拒,兩翼徑涉水,直到所治城。城潰,融不得入,轉至南縣,左右稍叛。連年傾覆,事無所濟,遂不能保鄣四境,棄郡而去。後徙徐州,以北海相自還領青州刺史,治郡北陲。欲附山東,外接遼東,得戎馬之利,建樹根本,孤立一隅,不與共也。于時曹、袁、公孫共相首尾,戰士不滿數百,穀不至萬斛。王子法、劉孔慈凶辯小才,信為腹心。左丞祖、劉義遜清儁之士,備在坐席而已,言此民望,不可失也。丞祖勸融自託彊國,融不聽而殺之。義遜棄去。遂為袁譚所攻,自春至夏,城小寇衆,流矢雨集。然融憑几安坐,讀書論議自若。城壞衆亡,身奔山東,室家為譚所虜。}
\gezhu{張璠漢紀曰:融在郡八年,僅以身免。帝初都許,融以為宜略依舊制,定王畿,正司隷所部為千里之封,乃引公卿上書言其義。是時天下草創,曹、袁之權未分,融所建明,不識時務。又天性氣爽,頗推平生之意,狎侮太祖。太祖制酒禁,而融書啁之曰:「天有酒旗之星,地列酒泉之郡,人有旨酒之德,故堯不飲千鍾,無以成其聖。且桀紂以色亡國,今令不禁婚姻也。」太祖外雖寬容,而內不能平。御史大夫郗慮知旨,以法免融官。歲餘,拜太中大夫。雖居家失勢,而賔客日滿其門,愛才樂酒,常歎曰:「坐上客常滿,鐏中酒不空,吾無憂矣。」虎賁士有貌似蔡邕者,融每酒酣,輙引與同坐,曰:「雖無老成人,尚有典刑。」其好士如此。}
\gezhu{續漢書曰:太尉楊彪與袁術婚姻,術僭號,太祖與彪有隙,因是執彪,將殺焉。融聞之,不及朝服,往見太祖曰:「楊公累世清德,四葉重光,周書『父子兄弟,罪不相及』,況以袁氏之罪乎?易稱『積善餘慶』,但欺人耳。」太祖曰:「國家之意也。」融曰:「假使成王欲殺召公,則周公可得言不知邪?今天下纓緌縉紳之士所以瞻仰明公者,以明公聦明仁智,輔相漢朝,舉直措枉,致之雍熈耳。今橫殺無辜,則海內觀聽,誰不解體?孔融魯國男子,明日便當褰衣而去,不復朝矣。」太祖意解,遂理出彪。}
\gezhu{魏氏春秋曰:袁紹之敗也,融與太祖書曰:「武王伐紂,以妲己賜周公。」太祖以融學博,謂書傳所紀見。後問之,對曰:「以今度之,想其當然耳!」十三年,融對孫權使,有訕謗之言,坐棄市。二子年八歲,時方弈棊,融被收,端坐不起。左右曰:「而父見執,不起何也?」二子曰:「安有巢毀而卵不破者乎!」遂俱見殺。融有高名清才,世多哀之。太祖懼遠近之議也,乃令曰:「太中大夫孔融旣伏其罪矣,然世人多採其虛名,少於核實,見融浮豔,好作變異,眩其誑詐,不復察其亂俗也。此州人說平原禰衡受傳融論,以為父母與人無親,譬若缻器,寄盛其中,又言若遭饑饉,而父不肖,寧贍活餘人。融違天反道,敗倫亂理,雖肆市朝,猶恨其晚。更以此事列上,宣示諸軍將校掾屬,皆使聞見。」}
\gezhu{世語曰:融二子,皆齠齓。融見收,顧謂二子曰:「何以不辭?」二子俱曰:「父尚如此,復何所辭!」以為必俱死也。臣松之以為世語云融二子不辭,知必俱死,猶差可安。如孫盛之言,誠所未譬。八歲小兒,能玄了禍福,聦明特達,卓然旣遠,則其憂樂之情,宜其有過成人,安有見父收執而曾無變容,弈棊不起,若在暇豫者乎?昔申生就命,言不忘父,不以己身將死而廢念父之情也。父安猶尚若茲,而況於顛沛哉?盛以此為美談,無乃賊夫人之子與!蓋由好奇情多,而不知言之傷理。}
南陽許攸、
\gezhu{魏略曰:攸字子遠,少與袁紹及太祖善。初平中隨紹在兾州,甞在坐席言議。官渡之役,諫紹勿與太祖相攻,語在紹傳。紹自以彊盛,必欲極其兵勢。攸知不可為謀,乃亡詣太祖。紹破走,及後得兾州,攸有功焉。攸自恃勳勞,時與太祖相戲,每在席,不自限濟,至呼太祖小字,曰:「某甲,卿不得我,不得兾州也。」太祖笑曰:「汝言是也。」然內嫌之。其後從行出鄴東門,顧謂左右曰:「此家非得我,則不得出入此門也。」人有白者,遂見收之。}
婁圭,皆以恃舊不虔見誅。
\gezhu{魏略曰:婁圭字子伯,少與太祖有舊。初平中在荊州北界合衆,後詣太祖。太祖以為大將,不使典兵,常在坐席言議。及河北平定,隨在兾州。其後太祖從諸子出游,子伯時亦隨從。子伯顧謂左右曰:「此家父子,如今日為樂也。」人有白者,太祖以為有腹誹意,遂收治之。吳書曰:子伯少有猛志,甞歎息曰:「男兒居世,會當得數萬兵千匹騎著後耳!」儕輩笑之。後坐藏亡命,被繫當死,得踰獄出,捕者追之急,子伯乃變衣服如助捕者,吏不能覺,遂以得免。會天下義兵起,子伯亦合衆與劉表相依。後歸曹公,遂為所待,軍國大計常與焉。劉表亡,曹公向荊州。表子琮降,以節迎曹公,諸將皆疑詐,曹公以問子伯。子伯曰:「天下擾攘,各貪王命以自重,今以節來,是必至誠。」曹公曰:「大善。」遂進兵。寵秩子伯,家累千金,曰:「婁子伯富樂於孤,但勢不如孤耳!」從破馬超等,子伯功為多。曹公常歎曰:「子伯之計,孤不及也。」後與南郡習授同載,見曹公出,授曰:「父子如此,何其快耳!」子伯曰:「居世間,當自為之,而但觀他人乎!」授乃白之,遂見誅。魚豢曰:古人有言曰:「得鳥者,羅之一目也,然張一目之羅,終不得鳥矣。鳥能遠飛,遠飛者,六翮之力也,然無衆毛之助,則飛不遠矣。」以此推之,大魏之作,雖有功臣,亦未必非茲輩胥附之由也。}
而琰最為世所痛惜,至今冤之。
\gezhu{世語曰:琰兄孫諒,字士文,以簡素稱,仕晉為尚書大鴻臚。荀綽兾州記云諒即琰之孫也。}

\end{pinyinscope}