\article{郤正傳}
\begin{pinyinscope}
 
 
 郤正字令先,河南偃師人也。祖父儉,靈帝末為益州刺史,為盜賊所殺。會天下大亂,故正父揖因留蜀。揖為大將軍孟達營都督,隨達降魏,為中書令史。正本名纂。少以父死母嫁,單煢隻立,而安貧好學,博覽墳籍。弱冠能屬文,入為祕書吏,轉為令史,遷郎,至令。性澹於榮利,而尤耽意文章,自司馬、王、楊、班、傅、張、蔡之儔遺文篇賦,及當世美書善論,益部有者,則鑽鑿推求,略皆寓目。自在內職,與宦人黃皓比屋周旋,經三十年,皓從微至貴,操弄威權,正旣不為皓所愛,亦不為皓所憎,是以官不過六百石,而免於憂患。
 
 
 
 
 依則先儒,假文見意,號曰釋譏,其文繼於崔駰達旨。其辭曰:
 
 
 
 
 或有譏余者曰:『聞之前記,夫事與時並,名與功偕,然則名之與事,前哲之急務也。是故創制作範,匪時不立,流稱垂名,匪功不記,名必須功而乃顯,事亦俟時以行止,身沒名滅,君子所耻。是以達人研道,探賾索微,觀天運之符表,考人事之盛衰,辯者馳說,智者應機,謀夫演略,武士奮威,雲合霧集,風激電飛,量時揆宜,用取世資,小屈大申,存公忽私,雖尺枉而尋直,終揚光以發輝也。今三方鼎跱,九有未乂,悠悠四海,嬰丁禍敗,嗟道義之沈塞,愍生民之顛沛,此誠聖賢拯救之秋,烈士樹功之會也。吾子以高朗之才,珪璋之質,兼覽博闚,留心道術,無遠不致,無幽不悉;挺身取命,幹茲奧祕,躊躇紫闥,喉舌是執,九考不移,有入無出,
 
 
\gezhu{尚書曰:三載考績,三考黜陟幽明。九考則二十七年。}
 究古今之真偽,計時務之得失。雖時獻一策,偶進一言,釋彼官責,慰此素飱,固未能輸竭忠款,盡瀝胷肝,排方入直,惠彼黎元,俾吾徒草鄙並有聞焉也。盍亦綏衡緩轡,回軌易塗,輿安駕肆,思馬斯徂,審厲揭以投濟,要夷庚之赫憮,播秋蘭以芳世,副吾徒之彼圖,不亦盛與!』
 
 
 
 
 余聞而歎曰:「嗚呼,有若云乎邪!夫人心不同,實若其面,子雖光麗,旣美且豔,管闚筐舉,守厥所見,未可以言八紘之形埒,信萬事之精練也。』
 
 
 
 
 或人率爾,抑而揚衡曰:『是何言與!是何言與!』
 
 
 
 
 余應之曰:『虞帝以面從為戒,孔聖以恱己為尤,若子之言,良我所思,將為吾子論而釋之。昔在鴻荒,矇昧肈初,三皇應籙,五帝承符,爰曁夏、商,前典攸書。姬衰道缺,霸者翼扶,嬴氏慘虐,吞嚼八區,於是從橫雲起,狙詐如星,奇邪蠭動,智故萌生;或飾真以讎偽,或挾邪以干榮,或詭道以要上,或鬻技以自矜;背正崇邪,弃直就佞,忠無定分,義無常經。故鞅法窮而慝作,斯義敗而姦成,呂門大而宗滅,韓辯立而身刑。夫何故哉?利回其心,寵耀其目,赫赫龍章,鑠鑠車服,媮幸苟得,如反如仄,淫邪荒迷,恣睢自極,和鸞未調而身在轅側,庭宁未踐而棟折榱覆。天收其精,地縮其澤,人弔其躬,鬼芟其頟。初升高岡,終隕幽壑,朝含榮潤,夕為枯魄。是以賢人君子,深圖遠慮,畏彼咎戾,超然高舉,寧曳尾於塗中,穢濁世之休譽。彼豈輕主慢民,而忽於時務哉?蓋易著行止之戒,詩有靖恭之歎,乃神之聽之而道使之然也。
 
 
 
 
 自我大漢,應天順民,政治之隆,皓若陽春,俯憲坤典,仰式乾文。播皇澤以熈世,揚茂化之醲醇,君臣履度,各守厥真;上垂詢納之弘,下有匡救之責,士無虛華之寵,民有一行之迹,粲乎斖斖,尚此忠益。然而道有隆窳,物有興廢,有聲有寂,有光有翳。朱陽否於素秋,玄陰抑於孟春,羲和逝而望舒係,運氣匿而耀靈陳。沖、質不永,桓、靈墜敗,英雄雲布,豪傑盖世,家挾殊議,人懷異計,故從橫者歘披其胷,狙詐者暫吐其舌也。
 
 
 
 
 今天綱已綴,德樹西鄰,丕顯祖之宏規,縻好爵於士人,興五教以訓俗,豐九德以濟民,肅明祀以礿祭,幾皇道以輔真。雖跱者未一,偽者未分,聖人垂戒,蓋均無貧;故君臣恊美於朝,黎庶欣戴於野,動若重規,靜若疊矩。濟濟偉彥,元凱之倫也,有過必知,顏子之仁也,侃侃庶政,冉、季之治也,鷹楊鷙騰,伊、望之事也;緫羣俊之上略,含薛氏之三計,敷張、陳之祕策,故力征以勤世,援華英而不遑,豈暇脩枯籜於榛穢哉!
 
 
 
 
 然吾不才,在朝累紀,託身所天,心焉是恃。樂滄海之廣深,歎嵩嶽之高跱,聞仲尼之贊商,感鄉校之益己,彼平仲之和羹,亦進可而替否;故矇冒瞽說,時有攸獻,譬遒人之有采於市閭,游童之吟詠乎疆畔,庶以增廣福祥,輸力規諫。若其合也,則以闇恊明,進應靈符;如其違也,自我常分,退守己愚。進退任數,不矯不誣,循性樂天,夫何恨諸?此其所以旣入不出,有而若無者也。狹屈氏之常醒,濁漁父之必醉,溷柳季之卑辱,褊夷叔之高懟。合不以得,違不以失,得不克詘,失不慘悸;不樂前以顧軒,不就後以慮輊,不粥譽以干澤,不辭愆以忌絀。何責之釋?何飱之卹?何方之排?何責之入?九考不移,固其所執也。
 
 
方今朝士山積,髦俊成羣,猶鱗介之潛乎巨海,毛羽之集乎鄧林,游禽逝不為之尠,浮魴臻不為之殷。且陽靈幽於唐葉,陰精應於商時,陽盱請而洪灾息,桑林禱而甘澤茲。
 \gezhu{淮南子曰:禹為水,以身請於陽盱之河,湯苦旱,以身禱於桑林之際,聖人之憂民,如此其明也。呂氏春秋曰:昔殷湯克夏桀而天下大旱,三年不收,湯乃以身禱於桑林曰:「余一人有罪,無及萬方,萬方有罪,在余一人,無以一人之不敏,使上帝鬼神傷民之大命。」湯於是剪其髮,攦其爪,自以為犧牲,用祈福於上帝。民乃甚恱。雨乃大至。}
 行止有道,啟塞有期。我師遺訓,不怨不尤,委命恭己,我又何辭?辭窮路單,將反初節,綜墳典之流芳,尋孔氏之遺藝,綴微辭以存道,憲先軌而投制,韙叔肹之優游,美踈氏之遐逝,收止足以言歸,汎皓然以容裔,欣環堵以恬娛,免咎悔於斯世,顧茲心之未泰,懼末塗之泥滯,仍求激而增憤,肆中懷以告誓。昔九方考精於至貴,秦牙沉思於殊形;
 \gezhu{淮南子曰:秦穆公謂伯樂曰:「子之年長矣,子姓有可使求馬者乎?」對曰:「良馬者,可以形容筋骨相也。相天下之馬者,若滅若沒,若失若亡,其一若此馬者,絕塵却轍。臣之子皆下才也,可告以良馬而不可告以天下之馬。天下之馬,臣有所與共儋纏采薪九方堙,此其相馬,非臣之下也,請見之。」穆公見之,使之求馬,三月而反,報曰:「已得馬矣,在於沙丘。」穆公曰:「何馬也?」對曰:「牝而黃。」使人往取之,牡而驪。穆公不說,召伯樂而問之曰:「敗矣,子之所使求馬者也!毛物牝牡尚弗能知,又何馬之能知?」伯樂喟然太息曰:「一至此乎!是乃所以千萬里馬而無數者也。若堙之所觀者天機也,得其精而忘其麄,在其內而忘其外,見其所見而不見其所不見,視其所視而遺其所不視,若彼之所相者,乃有貴乎馬者。」馬至,而果天下之馬也。淮南子又曰:伯樂、寒風、秦牙、葛青,所相各異,其知馬一也;蓋九方觀其精,秦牙察其形。}
 薛燭察寶以飛譽,
 \gezhu{越絕書曰:昔越王勾踐有寶劒五枚,聞於天下。客有能相劒者名薛燭,王召而問之:「吾有寶劒五,請以示子。」乃取豪曹、臣闕,薛燭曰:「皆非也。」又取純鈎、湛盧,薛燭曰:「觀其劒鈔,爛爛如列宿之行,觀其光,渾渾如水之將溢於塘,觀其文,渙渙如冰將釋,此所謂純鈎邪?」王曰:「是也。」王曰:「客有直之者,有市之鄉三,駿馬千匹,千戶之都二,可乎?」薛燭曰:「不可。當造此劒之時,赤堇之山破而出錫,若邪之溪涸而出銅,雨師掃灑,雷公擊鼓,太一下觀,天精下之,歐冶乃因天之精,悉其技巧,一曰純鈎,二曰湛盧。今赤堇之山已合,若邪之溪深而不測,歐冶子已死,雖傾城量金,珠玉竭河,獨不得此一物。有市之鄉三,駿馬千匹,千戶之都二,亦何足言與!」}
 瓠梁託弦以流聲;
 \gezhu{淮南子曰:瓠巴鼓瑟而鱏魚聽之。又曰:瓠梁之歌可隨也,而以歌者不可為也。}
 齊隷拊髀以濟文,
 \gezhu{臣松之曰:按此謂孟甞君田文下坐客,能作雞鳴以濟其厄者也。凡作雞鳴,必先拊髀,以傚雞之拊翼也。}
 楚客潛寇以保荊;
 \gezhu{淮南子曰:楚將子發好求技道之士。楚有善為偷者,往見曰:「聞君求技道之士,臣偷也,願以技備一卒。」子發聞之,衣不及帶,冠不暇正,出見而禮之。左右諫曰:「偷者,天下之盜也,何為禮之?」君曰:「此非左右之所得與。」後無幾何,齊興兵伐楚。子發將師以當之,兵三却。楚賢大夫皆盡其計而悉其誠,齊師愈彊。於是卒偷進請曰:「臣有薄技,願為君行之。」君曰「諾」。偷即夜出,解齊將軍之幬帳,而獻之子發。子發使人歸之,曰:「卒有出採薪者,得將軍之帳,使使歸於執事。」明日又復往取枕,子發又使歸之。明日又復往取簪,子發又使歸之。齊師聞之大駭,將軍與軍吏謀曰:「今日不去,楚軍恐取吾頭矣!」即旋師而去。}
 雍門援琴而挾說,
 \gezhu{桓譚新論曰:雍門周以琴見,孟甞君曰:「先生鼓琴,亦能令文悲乎?」對曰:「臣之所能令悲者,先貴而後賤,昔富而今貧,擯壓窮巷,不交四鄰;不若身材高妙,懷質抱真,逢讒罹謗,怨結而不得信;不若交歡而結髮,愛無怨而生離,遠赴絕國,無相見期;不若幼無父母,壯無妻兒,出以野澤為鄰,入用堀穴為家,困於朝夕,無所假貸:若此人者,但聞飛烏之號,秋風鳴條,則傷心矣,臣一為之援琴而長太息,未有不悽惻而涕泣者也。今若足下,居則廣廈高堂,連闥洞房,下羅帷,來清風;倡優在前,諂諛侍側,揚激楚,舞鄭妾,流聲以娛耳,練色以淫目;水戲則舫龍舟,建羽旗,鼓釣乎不測之淵;野游則登平原,馳廣囿,彊弩下高鳥,勇士格猛獸;置酒娛樂,沈醉忘歸:方此之時,視天地曾不若一指,雖有善鼓琴,未能動足下也。」孟甞君曰:「固然!」雍門周曰:「然臣竊為足下有所常悲。夫角帝而困秦者君也,連五國而伐楚者又君也。天下未甞無事,不從即衡;從成則楚王,衡成則秦帝。夫以秦、楚之彊而報弱薛,猶磨蕭斧而伐朝菌也,有識之士,莫不為足下寒心。天道不常盛,寒暑更進退,千秋萬歲之後,宗廟必不血食;高臺旣已傾,曲池又已平,墳墓生荊棘,狐狸穴其中,游兒牧豎躑躅其足而歌其上曰:『孟甞君之尊貴,亦猶若是乎!』」於是孟甞君喟然太息,涕淚承睫而未下。雍門周引琴而鼓之,徐動宮徵,叩角羽,終而成曲,孟甞君遂歔欷而就之曰:「先生鼓琴,令文立若亡國之人也。」}
 韓哀秉轡而馳名;
 \gezhu{呂氏春秋曰:韓哀作御。王褒聖主得賢臣頌曰:及至駕齧膝,參乘旦,王良執靶,韓哀附輿,縱馳騁騖,忽如景靡,過都越國,蹶如歷塊,追奔電,逐遺風,周流八極,萬里一息,何其遼哉!人馬相得也。}
 盧敖翱翔乎玄闕,若士竦身於雲清。
 \gezhu{淮南子曰:盧敖游乎北海,經乎太陰,入乎玄闕,至於蒙轂之上,見一士焉,深目而玄準,戾頸而鳶肩,豐上而殺下,軒軒然方迎風而舞,顧見盧敖慢然下其臂,遯逃乎碑下。盧敖俯而視之,方卷龜殼而食合梨。盧敖乃與之語曰:「惟敖為背羣離黨,窮觀於六合之外者,非敖而已乎!敖幼而好游,長不喻解,周行四極,惟北陰之不闚,今卒睹夫子於是,子殆可與敖為交乎!」若士者齤然而笑曰:「嘻乎!子中州民,寧肯而遠至此?此猶光乎日月而戴列星,陰陽之所行,四時之所生,此其比夫不名之地,猶突奧也。若我南游乎罔䍚之野,北息乎沈墨之鄉,西窮冥冥之黨,東貫鴻濛之光,此其下無地而上無天,聽焉無聞,視焉則眴,此其外猶有沈沈之汜,其餘一舉而千萬里,吾猶未能之在。今子游始至於此,乃語窮觀,豈不亦遠哉!然子處矣,吾與汗漫期於九垓之上,吾不可以乆。」若士舉臂而竦身,遂入雲中。盧敖仰而視之,弗見乃止,曰:「吾比夫子也,猶黃鵠之與壤蟲,終日行不離咫尺,自以為遠,不亦悲哉!」}
 余實不能齊技於數子,故乃靜然守己而自寧。』
 
 
 
 
 景耀六年,後主從譙周之計,遣使請降於鄧艾,其書,正所造也。明年正月,鍾會作亂成都,後主東遷洛陽,時擾攘倉卒,蜀之大臣無翼從者,惟正及殿中督汝南張通,捨妻子單身隨侍。後主賴正相導宜適,舉動無闕,乃慨然歎息,恨知正之晚。時論嘉之。賜爵關內侯。泰始中,除安陽令,遷巴西太守。泰始八年詔曰:「正昔在成都,顛沛守義,不違忠節,及見受用,盡心幹事,有治理之績,其以正為巴西太守。」咸寧四年卒。凡所著述詩論賦之屬,垂百篇。
 
 
評曰:杜微脩身隱靜,不役當世,庶幾夷、皓之槩。周羣占天有徵,杜瓊沈默慎密,諸生之純也。許、孟、來、李,愽涉多聞,尹默精於左氏,雖不以德業為稱,信皆一時之學士。譙周詞理淵通,為世碩儒,有董、楊之規,郤正文辭燦爛,有張、蔡之風,加其行止,君子有取焉。二子處晉事少,在蜀事多,故著于篇。
 \gezhu{張璠以為譙周所陳降魏之策,蓋素料劉禪懦弱,心無害戾,故得行也。如遇忿肆之人,雖無他筭,然矜殉鄙恥,或發怒妄誅,以立一時之威,快其斯須之意者,此亦夷滅之禍云。}
 
 
\end{pinyinscope}