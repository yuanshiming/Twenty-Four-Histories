\article{涼茂傳}
\begin{pinyinscope}
 
 
 涼茂字伯方,山陽昌邑人也。少好學,論議常據經典,以處是非。太祖辟為司空掾,舉高第,補侍御史。時泰山多盜賊,以茂為泰山太守,旬月之閒,襁負而至者千餘家。
 
 
\gezhu{博物記曰:襁,織縷為之,廣八寸,長尺二,以約小兒於背上,負之而行。}
 轉為樂浪太守。公孫度在遼東,擅留茂,不遣之官,然茂終不為屈。度謂茂及諸將曰:「聞曹公遠征,鄴無守備,今吾欲以步卒三萬,騎萬匹,直指鄴,誰能禦之?」諸將皆曰:「然。」
 \gezhu{臣松之案此傳云公孫度聞曹公遠征,鄴無守備,則太祖定鄴後也。案度傳,度以建安九年卒,太祖亦以此年定鄴,自後遠征,唯有北征柳城耳。征柳城之年,度已不復在矣。}
 又顧謂茂曰:「於君意何如?」茂荅曰:「比者海內大亂,社稷將傾,將軍擁十萬之衆,安坐而觀成敗,夫為人臣者,固若是邪!曹公憂國家之危敗,愍百姓之苦毒,率義兵為天下誅殘賊,功高而德廣,可謂無二矣。以海內初定,民始安集,故未責將軍之罪耳!而將軍乃欲稱兵西向,則存亡之效,不崇朝而決。將軍其勉之!」諸將聞茂言,皆震動。良乆,度曰:「涼君言是也。」後徵遷為魏郡太守、甘陵相,所在有績。文帝為五官將,茂以選為長史,遷左軍師。魏國初建,遷尚書僕射,後為中尉奉常。文帝在東宮,茂復為太子太傅,甚見敬禮。卒官。
 \gezhu{英雄記曰:茂名在八友中。}
 
 
\end{pinyinscope}