\article{東夷傳}
\begin{pinyinscope}
 
 
 書稱「東漸于海,西被于流沙」。其九服之制,可得而言也。然荒域之外,重譯而至,非足跡車軌所及,未有知其國俗殊方者也。自虞曁周,西戎有白環之獻,東夷有肅慎之貢,皆曠世而至,其遐遠也如此。及漢氏遣張騫使西域,窮河源,經歷諸國,遂置都護以總領之,然後西域之事具存,故史官得詳載焉。魏興,西域雖不能盡至,其大國龜茲、于寘、康居、烏孫、踈勒、月氏、鄯善、車師之屬,無歲不奉朝貢,略如漢氏故事。而公孫淵仍父祖三世有遼東,天子為其絕域,委以海外之事,遂隔斷東夷,不得通於諸夏。景初中,大興師旅,誅淵,又潛軍浮海,收樂浪、帶方之郡,而後海表謐然,東夷屈服。其後高句麗背叛,又遣偏師致討,窮追極遠,踰烏丸、骨都,過沃沮,踐肅慎之庭,東臨大海。長老說有異靣之人,近日之所出,遂周觀諸國,采其法俗,小大區別,各有名號,可得詳紀。雖夷狄之邦,而俎豆之象存。中國失禮,求之四夷,猶信。故撰次其國,列其同異,以接前史之所未備焉。
 
 
\end{pinyinscope}