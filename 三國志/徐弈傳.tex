\article{徐弈傳}
\begin{pinyinscope}
 
 
 徐弈字季才,東莞人也。避難江東,孫策禮命之。弈改姓名,微服還本郡。太祖為司空,辟為掾屬,從西征馬超。超破,軍還。時關中新服,未甚安,留弈為丞相長史,鎮撫西京,西京稱其威信。轉為雍州刺史,復還為東曹屬。丁儀等見寵於時,並害之,而弈終不為動。
 
 
\gezhu{魏書曰:或謂弈曰:「夫以史魚之直,孰與蘧伯玉之智?丁儀方貴重,宜思所以下之。」弈曰:「以公明聖,儀豈得乆行其偽乎!且姧以事君者,吾所能禦也,子寧以他規我。」傅子曰:武皇帝,至明也。崔琰、徐弈,一時清賢,皆以忠信顯於魏朝;丁儀間之,徐弈失位而崔琰被誅。}
 出為魏郡太守。太祖征孫權,徙為留府長史,謂奕曰:「君之忠亮,古人不過也,然微太嚴。昔西門豹佩韋以自緩,夫能以柔弱制剛彊者,望之於君也。今使君統留事,孤無復還顧之憂也。」魏國旣建,為尚書,復典選舉,遷尚書令。
 
 
太祖征漢中,魏諷等謀反,中尉楊俊左遷。太祖歎曰:「諷所以敢生亂心,以吾爪牙之臣無遏姦防謀者故也。安得如諸葛豐者,使代俊乎!」桓階曰:「徐弈其人也。」太祖乃以弈為中尉,手令曰:「昔楚有子玉,文公為之側席而坐;汲黯在朝,淮南為之折謀。詩稱『邦之司直』,君之謂與!」在職數月,疾篤乞退,拜諫議大夫,卒。
 \gezhu{魏書曰:文帝每與朝臣會同,未甞不嗟歎,思弈之為人。弈無子,詔以其族子統為郎,以奉弈後。}
 
 
\end{pinyinscope}