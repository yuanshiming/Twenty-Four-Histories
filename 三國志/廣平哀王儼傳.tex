\article{廣平哀王儼傳}
\begin{pinyinscope}

廣平哀王儼,黃初三年封。四年薨。無子。國除。

評曰:魏氏王公,旣徒有國土之名,而無社稷之實,又禁防壅隔,同於囹圄;位號靡定,大小歲易;骨肉之恩乖,常棣之義廢。為法之弊,一至于此乎!

\gezhu{袁子曰:魏興,承大亂之後,民人損減,不可則以古始。於是封建侯王,皆使寄地,空名而無其實。王國使有老兵百餘人,以衞其國。雖有王侯之號,而乃儕於匹夫。縣隔千里之外,無朝聘之儀,鄰國無會同之制。諸侯游獵不得過三十里,又為設防輔監國之官以伺察之。王侯皆思為布衣而不能得。旣違宗國藩屏之義,又虧親戚骨肉之恩。}

\gezhu{魏氏春秋載宗室曹冏上書曰:「臣聞古之王者,必建同姓以明親親,必樹異姓以明賢賢。故傳曰『庸勳親親,昵近尊賢』;書曰『克明俊德,以親九族』;詩云『懷德維寧,宗子維城』。由是觀之,非賢無與興功,非親無與輔治。夫親親之道,專用則其漸也微弱;賢賢之道,偏任則其弊也劫奪。先聖知其然也,故博求親疎而並用之;近則有宗盟藩衞之固,遠則有仁賢輔弼之助,盛則有與共其治,衰則有與守其土,安則有與享其福,危則有與同其禍。夫然,故能有其國家,保其社稷,歷紀長乆,本枝百世也。今魏尊尊之法雖明,親親之道未備。詩不云乎,『鶺鴒在原,兄弟急難』。以斯言之,明兄弟相救於喪亂之際,同心於憂禍之間,雖有鬩牆之忿,不忘禦侮之事。何則?憂患同也。今則不然,或任而不重,或釋而不任,一旦疆埸稱警,關門反拒,股肱不扶,胷心無衞。臣竊惟此,寢不安席,思獻丹誠,貢策朱闕。謹撰合所聞,叙論成敗。}
\gezhu{論曰:昔夏、殷、周歷世數十,而秦二世而亡。何則?三代之君,與天下共其民,故天下同其憂。秦王獨制其民,故傾危而莫救。夫與民共其樂者,人必憂其憂;與民同其安者,人必拯其危。先王知獨治之不能乆也,故與人共治之;知獨守之不能固也,故與人共守之。兼親踈而兩用,參同異而並建。是以輕重足以相鎮,親踈足以相衞,并兼路塞,逆節不生。及其衰也,桓、文帥禮;苞茅不貢,齊師伐楚;宋不城周,晉戮其宰。王綱弛而復張,諸侯傲而復肅。二霸之後,浸以陵遲。吳、楚憑江,負固方城,雖心希九鼎,而畏迫宗姬,姦情散於胷懷,逆謀消於脣吻;斯豈非信重親戚,任用賢能,枝葉碩茂,本根賴之與?自此之後,轉相攻伐;吳并於越,晉分為三,魯滅於楚,鄭兼於韓。曁于戰國,諸姬微矣,惟燕、衞獨存,然皆弱小,西迫彊秦,南畏齊、楚,憂懼滅亡,匪遑相恤。至於王赧,降為庶人,猶枝幹相持,得居虛位,海內無主,四十餘年。秦據勢勝之地,騁譎詐之術,征伐關東,蠶食九國,至於始皇,乃定天位。曠日若彼,用力若此,豈非深固根蔕不拔之道乎?}
\gezhu{易曰;『其亡其亡,繫于苞桑。』周德其可謂當之矣。秦觀周之弊,以為小弱見奪,於是廢五等之爵,立郡縣之官,弃禮樂之教,任苛刻之政;子弟無尺寸之封,功臣無立錐之地,內無宗子以自毗輔,外無諸侯以為藩衞,仁心不加於親戚,惠澤不流於枝葉;譬猶芟刈股肱,獨任胷腹,浮舟江海,捐棄楫櫂,觀者為之寒心,而始皇晏然自以為關中之固,金城千里,子孫帝王萬世之業也,豈不悖哉!是時淳于越諫曰:『臣聞殷、周之王,封子弟功臣千有餘城。今陛下君有海內而子弟為匹夫,卒有田常六卿之臣,而無輔弼,何以相救?事不師古而能長乆者,非所聞也。』始皇聽李斯偏說而絀其議,至於身死之日,無所寄付,委天下之重於凡夫之手,託廢立之命於姧臣之口,至令趙高之徒,誅鉏宗室。胡亥少習刻薄之教,長遵凶父之業,不能改制易法,寵任兄弟,而乃師譚申、商,諮謀趙高;自幽深宮,委政讒賊,身殘望夷,求為黔首,豈可得哉?遂乃郡國離心,衆庶潰叛,勝、廣倡之於前,劉、項弊之於後。向使始皇納淳于之策,抑李斯之論,割裂州國,分王子弟,封三代之後,報功臣之勞,士有常君,民有定主,枝葉相扶,首尾為用,雖使子孫有失道之行,時人無湯、武之賢,姧謀未發,而身已屠戮,何區區之陳、項而復得措其手足哉?故漢祖奮三尺之劒,驅烏集之衆,五年之中,遂成帝業。自開闢以來,其興立功勳,未有若漢祖之易也。夫伐深根者難為功,摧枯朽者易為力,理勢然也。漢監秦之失,封殖子弟,及諸呂擅權,圖危劉氏,而天下所以不傾動,百姓所以不易心者,徒以諸侯彊大,盤石膠固,東牟、朱虛受命於內,齊、代、吳、楚作衞於外故也。}
\gezhu{向使高祖踵亡秦之法,忽先王之制,則天下已傳,非劉氏有也。然高祖封建,地過古制,大者跨州兼郡,小者連城數十,上下無別,權侔京室,故有吳、楚七國之患。賈誼曰:『諸侯彊盛,長亂起姧。夫欲天下之治安,莫若衆建諸侯而少其力,令海內之勢,若身之使臂,臂之使指,則下無背叛之心,上無誅伐之事。』文帝不從。至於孝景,猥用晁錯之計,削黜諸侯,親者怨恨,踈者震恐,吳、楚倡謀,五國從風。兆發高帝,釁鍾文、景,由寬之過制,急之不漸故也。所謂末大必折,尾大難掉。尾同於體,猶或不從,況乎非體之尾,其可掉哉?武帝從主父之策,下推恩之令,自是之後,齊分為七,趙分為六,淮南三割,梁、代五分,遂以陵遲,子孫微弱,衣食租稅,不預政事,或以酎金免削,或以無後國除。至於成帝,王氏擅朝。劉向諫曰:『臣聞公族者,國之枝葉;枝葉落則本根無所庇蔭。方今同姓疏遠,母黨專政,排擯宗室,孤弱公族,非所以保守社稷,安固國嗣也。』其言深切,多所稱引,成帝雖悲傷歎息而不能用。至于哀、平,異姓秉權,假周公之事,而為田常之亂,高拱而竊天位,一朝而臣四海。漢宗室王侯,解印釋紱,貢奉社稷,猶懼不得為臣妾,或乃為之符命,頌莽恩德,豈不哀哉!由斯言之,非宗子獨忠孝於惠、文之間,而叛逆於哀、平之際也,徒權輕勢弱,不能有定耳。賴光武皇帝挺不世之姿,禽王莽於已成,紹漢嗣於旣絕,斯豈非宗子之力也?而曾不監秦之失策,襲周之舊制,踵王國之法,而徼倖無疆之期。至於桓、靈,閹豎執衡,朝無死難之臣,外無同憂之國,君孤立於上,臣弄權於下,本末不能相御,身首不能相使。由是天下鼎沸,姧凶並爭,宗廟焚為灰燼,宮室變為榛藪,居九州之地,而身無所安處,悲夫!}
\gezhu{魏太祖武皇帝躬聖明之賢,兼神武之畧,恥王綱之廢絕,愍漢室之傾覆,龍飛譙、沛,鳳翔兖、豫,掃除凶逆,翦滅鯨鯢,迎帝西京,定都潁邑,德動天地,義感人神。漢氏奉天,禪位大魏。大魏之興,于今二十有四年矣,觀五代之存亡而不用其長策,覩前車之傾覆而不改於轍迹;子弟王空虛之地,君有不使之民,宗室竄於閭閻,不聞邦國之政,權均匹夫,勢齊凡庶;內無深根不拔之固,外無盤石宗盟之助,非所以安社稷,為萬世之業也。且今之州牧、郡守,古之方伯、諸侯,皆跨有千里之土,兼軍武之任,或比國數人,或兄弟並據;而宗室子弟曾無一人閒厠其閒,與相維持,非所以彊幹弱枝,備萬一之虞也。今之用賢,或超為名都之主,或為偏師之帥,而宗室有文者必限小縣之宰,有武者必置百人之上,使夫廉高之士,畢志於衡軛之內,才能之人,恥與非類為伍,非所以勸進賢能襃異宗室之禮也。夫泉竭則流涸,根朽則葉枯;枝繁者蔭根,條落者本孤。故語曰『百足之蟲,至死不僵』,以扶之者衆也。此言雖小,可以譬大。且墉基不可倉卒而成,威名不可一朝而立,皆為之有漸,建之有素。譬之種樹,久則深固其本根,茂盛其枝葉,若造次徙於山林之中,植於宮闕之下,雖壅之以黑墳,煖之以春日,猶不救於枯槁,而何暇繁育哉?夫樹猶親戚,土猶士民,建置不乆,則輕下慢上,平居猶懼其離叛,危急將若之何?是以聖王安而不逸,以慮危也,存而設備,以懼亡也。故疾風卒至而無摧拔之憂,天下有變而無傾危之患矣。」}
\gezhu{冏,中常侍兄叔興之後,少帝族祖也。是時天子幼稚,冏兾以此論感悟曹爽,爽不能納。}


\end{pinyinscope}