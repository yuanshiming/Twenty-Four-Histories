\article{中山恭王衮傳}
\begin{pinyinscope}
 
 
 中山恭王衮,建安二十一年封平鄉侯。少好學,年十餘歲能屬文。每讀書,文學左右常恐以精力為病,數諫止之,然性所樂,不能廢也。二十二年,徙封東鄉侯,其年又改封贊侯。黃初二年,進爵為公,官屬皆賀,衮曰:「夫生深宮之中,不知稼穡之艱難,多驕逸之失。諸賢旣慶其休,宜輔其闕。」每兄弟游娛,衮獨譚思經典。文學防輔相與言曰:「受詔察公舉錯,有過當奏,及有善,亦宜以聞,不可匿其美也。」遂共表稱陳衮美。衮聞之,大驚懼,責讓文學曰:「脩身自守,常人之行耳,而諸君乃以上聞,是適所以增其負累也。且如有善,何患不聞,而遽共如是,是非益我者。」其誡慎如此。三年,為北海王。其年,黃龍見鄴西漳水,衮上書贊頌。詔賜黃金十斤,詔曰:「昔唐叔歸禾,東平獻頌,斯皆骨肉贊美,以彰懿親。王研精墳典,耽味道真,文雅煥炳,朕甚嘉之。王其克慎明德,以終令問。」四年,改封贊王。七年,徙封濮陽。太和二年就國,尚約儉,教勑妃妾紡績織絍,習為家人之事。五年冬,入朝。六年,改封中山。
 
 
 
 
 初,衮來朝,犯京都禁。青龍元年,有司奏衮。詔曰:「王素敬慎,邂逅至此,其以議親之典議之。」有司固執。詔削縣二,戶七百五十。
 
 
\gezhu{魏書載璽書曰:「制詔中山王:有司奏,王乃者來朝,犯交通京師之禁。朕惟親親之恩,用寢吏議。然法者,所與天下共也,不可得廢。今削王縣二,戶七百五十。夫克己復禮,聖人稱仁,朝過夕改,君子與之。王其戒諸,無貳咎悔也。」}
 衮憂懼,戒勑官屬愈謹。帝嘉其意,二年,復所削縣。三年秋,衮得疾病,詔遣太醫視疾,殿中、虎賁齎手詔、賜珍膳相屬,又遣太妃、沛王林並就省疾。衮疾困,勑令官屬曰:「吾寡德忝寵,大命將盡。吾旣好儉,而聖朝著終誥之制,為天下法。吾氣絕之日,自殯及葬,務奉詔書。昔衞大夫蘧瑗葬濮陽,吾望其墓,常想其遺風,願託賢靈以弊髮齒,營吾兆域,必往從之。禮:男子不卒婦人之手。亟以時成東堂。」堂成,名之曰遂志之堂,輿疾往居之。又令世子曰:「汝幼少,未聞義方,早為人君,但知樂,不知苦;不知苦,必將以驕奢為失也。接大臣,務以禮。雖非大臣,老者猶宜荅拜。事兄以敬,恤弟以慈;兄弟有不良之行,當造膝諫之。諫之不從,流涕喻之;喻之不改,乃白其母。若猶不改,當以奏聞,并辭國土。與其守寵罹禍,不若貧賤全身也。此亦謂大罪惡耳,其微過細故,當掩覆之。嗟爾小子,慎脩乃身,奉聖朝以忠貞,事太妃以孝敬。閨闈之內,奉令于太妃;閫閾之外,受教於沛王。無怠乃心,以慰予靈。」其年薨。詔沛王林留訖葬,使大鴻臚持節典護喪事,宗正弔祭,贈賵甚厚。凡所著文章二萬餘言,才不及陳思王而好與之侔。子孚嗣。景初、正元、景元中,累增邑,并前三千四百戶。
 
 
\end{pinyinscope}