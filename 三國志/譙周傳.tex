\article{譙周傳}
\begin{pinyinscope}
 
 
 譙周字允南,巴西西充國人也。父򠐃,字榮始,治尚書,兼通諸經及圖、緯。州郡辟請,皆不應,州就假師友從事。周幼孤,與母兄同居。旣長,耽古篤學,家貧未甞問產業,誦讀典籍,欣然獨笑,以忘寢食。研精六經,尤善書札。頗曉天文,而不以留意;諸子文章非心所存,不悉徧視也。身長八尺,體貌素朴,性推誠不飾,無造次辯論之才,然潛識內敏。
 
 
 
 
 建興中,丞相亮領益州牧,命周為勸學從事。
 
 
\gezhu{蜀記曰:周初見亮,左右皆笑。旣出,有司請推笑者,亮曰:「孤尚不能忍,況左右乎!」}
 亮卒於敵庭,周在家聞問,即便奔赴,尋有詔書禁斷,惟周以速行得達。大將軍蔣琬領刺史,徙為典學從事,緫州之學者。
 
 
 
 
 後主立太子,以周為僕,轉家令。時後主頗出游觀,增廣聲樂。周上疏諫曰:「昔王莽之敗,豪傑並起,跨州據郡,欲弄神器,於是賢才智士思望所歸,未必以其勢之廣狹,惟其德之薄厚也。是故於時更始、公孫述及諸有大衆者多已廣大,然莫不快情恣欲,怠於為善,游獵飲食,不恤民物。世祖初入河北,馮異等勸之曰:『當行人所不能為。』遂務理寃獄,節儉飲食,動遵法度,故北州歌歎,聲布四遠。於是鄧禹自南陽追之,吳漢、寇恂未識世祖,遙聞德行,遂以權計舉漁陽、上谷突騎迎于廣阿。其餘望風慕德者邳肜、耿純、劉植之徒,至于輿病齎棺,繈負而至者,不可勝數,故能以弱為彊,屠王郎,吞銅馬,折赤眉而成帝業也。及在洛陽,甞欲小出,車駕已御,銚期諫曰:『天下未寧,臣誠不願陛下細行數出。』即時還車。及征隗囂,潁川盜起,世祖還洛陽,但遣寇恂往,恂曰:『潁川以陛下遠征,故姦猾起叛,未知陛下還,恐不時降;陛下自臨,潁川賊必即降。』遂至潁川,竟如恂言。故非急務,欲小出不敢,至於急務,欲自安不為,故帝者之欲善也如此!故傳曰『百姓不徒附』,誠以德先之也。今漢遭厄運,天下三分,雄哲之士思望之時也。陛下天姿至孝,喪踰三年,言及隕涕,雖曾閔不過也。敬賢任才,使之盡力,有踰成康。故國內和一,大小勠力,臣所不能陳。然臣不勝大願,願復廣人所不能者。夫輓大重者,其用力苦不衆,拔大艱者,其善術苦不廣,且承事宗廟者,非徒求福祐,所以率民尊上也。至於四時之祀,或有不臨,池苑之觀,或有仍出,臣之愚滯,私不自安。夫憂責在身者,不暇盡樂,先帝之志,堂構未成,誠非盡樂之時。願省減樂官、後宮所增造,但奉脩先帝所施,下為子孫節儉之教。」徙為中散大夫,猶侍太子。
 
 
 
 
 于時軍旅數出,百姓彫瘁,周與尚書令陳祗論其利害,退而書之,謂之仇國論。其辭曰:「因餘之國小,而肇建之國大,並爭於世而為仇敵。因餘之國有高賢卿者,問於伏愚子曰:『今國事未定,上下勞心,往古之事,能以弱勝彊者,其術何如?』伏愚子曰:『吾聞之,處大無患者恒多慢,處小有憂者恒思善;多慢則生亂,思善則生治,理之常也。故周文養民,以少取多,句踐卹衆,以弱斃彊,此其術也。』賢卿曰:『曩者項彊漢弱,相與戰爭,無日寧息,然項羽與漢約分鴻溝為界,各欲歸息民;張良以為民志旣定,則難動也,尋帥追羽,終斃項氏,豈必由文王之事乎?肇建之國方有疾疢,我因其隙,陷其邊陲,覬增其疾而斃之也。』伏愚子曰:『當殷、周之際,王侯世尊,君臣乆固,民習所專;深根者難拔,據固者難遷。當此之時,雖漢祖安能杖劒鞭馬而取天下乎?當秦罷侯置守之後,民疲秦役,天下土崩,或歲改主,或月易公,鳥驚獸駭,莫知所從,於是豪彊並爭,虎裂狼分,疾博者獲多,遲後者見吞。今我與肇建皆傳國易世矣,旣非秦末鼎沸之時,實有六國並據之勢,故可為文王,難為漢祖。夫民疲勞則搔擾之兆生,上慢下暴則瓦解之形起。諺曰:「射幸數跌,不如審發。」是故智者不為小利移目,不為意似改步,時可而後動,數合而後舉,故湯、武之師不再戰而克,誠重民勞而度時審也。如遂極武黷征,土崩勢生,不幸遇難,雖有智者將不能謀之矣。若乃奇變從橫,出入無間,衝波截轍,超谷越山,不由舟楫而濟盟津者,我愚子也,實所不及。』」
 
 
 
 
 後遷光祿大夫,位亞九列。周雖不與政事,以儒行見禮,時訪大議,輒據經以對,而後生好事者亦咨問所疑焉。
 
 
 
 
 景耀六年冬,魏大將軍鄧艾克江由,長驅而前。而蜀本謂敵不便至,不作城守調度,及聞艾已入陰平,百姓擾擾,皆迸山野,不可禁制。後主使羣臣會議,計無所出。或以為蜀之與吳,本為和國,宜可奔吳;或以為南中七郡,阻險斗絕,易以自守,宜可奔南。惟周以為:「自古已來,無寄他國為天子者也,今若入吳,固當臣服。且政理不殊,則大能吞小,此數之自然也。由此言之,則魏能并吳,吳不能并魏明矣。等為小稱臣,孰與為大,再辱之恥,何與一辱?且若欲奔南,則當早為之計,然後可果;今大敵以近,禍敗將及,羣小之心,無一可保?恐發足之日,其變不測,何至南之有乎!」羣臣或難周曰:「今艾以不遠,恐不受降,如之何?」周曰:「方今東吳未賔,事勢不得不受之,受之後,不得不禮。若陛下降魏,魏不裂土以封陛下者,周請身詣京都,以古義爭之。」衆人無以易周之理。
 
 
後主猶疑於入南,周上疏曰:「或說陛下以北兵深入,有欲適南之計,臣愚以為不安。何者?南方遠夷之地,平常無所供為,猶數反叛,自丞相亮南征,兵勢偪之,窮乃幸從。是後供出官賦,取以給兵,以為愁怨,此患國之人也。今以窮迫,欲往依恃,恐必復反叛,一也。北兵之來,非但取蜀而已,若奔南方,必因人勢衰,及時赴追,二也。若至南方,外當拒敵,內供服御,費用張廣,他無所取,耗損諸夷必甚,甚必速叛,三也。昔王郎以邯鄲僭號,時世祖在信都,畏偪於郎,欲棄還關中。邳肜諫曰:『明公西還,則邯鄲城民不肯捐父母,背城主,而千里送公,其亡叛可必也。』世祖從之,遂破邯鄲。今北兵至,陛下南行,誠恐邳肜之言復信於今,四也。願陛下早為之圖,可獲爵土;若遂適南,勢窮乃服,其禍必深。易曰:『亢之為言,知得而不知喪,知存而不知亡;知得失存亡而不失其正者,其惟聖人乎!』言聖人知命而不苟必也。故堯、舜以子不善,知天有授,而求授人;子雖不肖,禍尚未萌,而迎授與人,況禍以至乎!故微子以殷王之昆,面縛銜璧而歸武王,豈所樂哉,不得已也。」於是遂從周策。劉氏無虞,一邦蒙賴,周之謀也。
 \gezhu{孫綽評曰:譙周說後主降魏,可乎?曰:自為天子而乞降請命,何恥之深乎!夫為社稷死則死之,為社稷亡則亡之。先君正魏之篡,不與同天矣。推過於其父,俛首而事讎,可謂苟存,豈大居正之道哉!孫盛曰:春秋之義,國君死社稷,卿大夫死位,況稱天子而可辱於人乎!周謂萬乘之君偷生苟免,亡禮希利,要兾微榮,惑矣。且以事勢言之,理有未盡。何者?禪雖庸主,實無桀、紂之酷,戰雖屢北,未有土崩之亂,縱不能君臣固守,背城借一,自可退次東鄙以思後圖。是時羅憲以重兵據白帝,霍弋以彊卒鎮夜郎。蜀土險狹,山水峻隔,絕巘激湍,非步卒所涉。若悉取舟檝,保據江州,徵兵南中,乞師東國,如此則姜、廖五將自然雲從,吳之三師承命電赴,何投寄之無所而慮於必亡邪?魏師之來,褰國大舉,欲追則舟楫靡資,欲留則師老多虞。且屈伸有會,情勢代起,徐因思奮之民,以攻驕惰之卒,此越王所以敗闔閭,田單所以摧騎劫也,何為怱怱遽自囚虜,下堅壁於敵人,致斫石之至恨哉?葛生有云:「事之不濟則已耳,安能復為之下!」壯哉斯言,可以立懦夫之志矣。觀古燕、齊、荊、越之敗,或國覆主滅,或魚縣鳥竄,終能建功立事,康復社稷,豈曰天助,抑亦人謀也。向使懷苟存之計,納譙周之言,何邦基之能構,令名之可獲哉?禪旣闇主,周實駑臣,方之申包、田單、范蠡、大夫種,不亦遠乎!}
 
 
時晉文王為魏相國,以周有全國之功,封陽城亭侯。又下書辟周,周發至漢中,困疾不進。咸熈二年夏,巴郡文立從洛陽還蜀,過見周。周語次,因書版示立曰:「典午忽兮,月酉沒兮。」典午者謂司馬也,月酉者謂八月也,至八月而文王果崩。
 \gezhu{華陽國志曰:文立字廣休,少治毛詩、三禮,兼通羣書。刺史費禕命為從事,入為尚書郎,復辟禕大將軍東曹掾,稍遷尚書。蜀并于魏,梁州建,首為別駕從事,舉秀才。晉泰始二年,拜濟陰太守,遷太子中庶子。立上言:「故蜀大官及盡忠死事者子孫,雖仕郡國,或有不才,同之齊民為劇;又諸葛亮、蔣琬、費禕等子孫流徙中畿,各宜量才叙用,以慰巴、蜀之心,傾吳人之望。」事皆施行。轉散騎常侍,獻可替否,多所補納。稍遷衞尉,中朝服其賢雅,為時名卿。咸寧末卒。立章奏詩詩賦論頌凡數十篇。}
 晉室踐阼,累下詔所在發遣周。周遂輿疾詣洛,泰始三年至。以疾不起,就拜騎都尉,周乃自陳無功而封,求還爵土,皆不聽許。
 
 
五年,予甞為本郡中正,清定事訖,求休還家,往與周別。周語予曰:「昔孔子七十二、劉向、揚雄七十一而沒,今吾年過七十,庶慕孔子遺風,可與劉、揚同軌,恐不出後歲,必便長逝,不復相見矣。」疑周以術知之,假此而言也。六年秋,為散騎常侍,疾篤不拜,至冬卒。
 \gezhu{晉陽秋載詔曰:「朕甚悼之,賜朝服一具,衣一襲,錢十五萬。」周息熈上言,周臨終屬熈曰:「乆抱疾,未曾朝見,若國恩賜朝服衣物者,勿以加身。當還舊墓,道險行難,豫作輕棺。殯斂已畢,上還所賜。」詔還衣服,給棺直。}
 凡所著述,撰定法訓、五經論、古史考書之屬百餘篇。
 \gezhu{益部耆舊傳曰:益州刺史董榮圖畫周像於州學,命從事李通頌之曰:「抑抑譙侯,好古述儒,寶道懷真,鑒世盈虛,雅名美迹,終始是書。我后欽賢,無言不譽,攀諸前哲,丹青是圖。嗟爾來葉,鋻茲顯模。」}
 周三子,熈、賢、同。少子同頗好周業,亦以忠篤質素為行,舉孝廉,除錫令、東宮洗馬,召不就。
 
 
周長子熈。熈子秀,字元彥。
 \gezhu{晉陽秋曰:秀性清靜,不交於世,知將大亂,豫絕人事,從兄弟及諸親里不與相見。州郡辟命,及李雄盜蜀,安車徵秀,又雄叔父驤、驤子壽辟命,皆不應。常冠鹿皮,躬耕山藪。永和三年,安西將軍桓溫平蜀,表薦秀曰:「臣聞大朴旣虧,則高尚之標顯;道喪時昏,則忠貞之義彰。故有洗耳投淵以振玄邈之風,亦有秉心矯迹以惇在三之節。是以上代之君,莫不崇重斯軌,所以篤俗訓民,靜一流競。伏惟大晉應符御世,運無常通,時有屯蹇,神州丘墟,三方圮裂,兔罝絕響於中林,白駒無聞於空谷,斯有識之所悼心,大雅之所歎息者也。陛下聖德嗣興,方恢天緒。臣昔奉役,有事西土,鯨鯢旣縣,思宣大化;訪諸故老,搜楊潛逸,庶武羅於羿、浞之墟,想王蠋於亡齊之境。切聞巴西譙秀,植操貞固,抱德肥遁,揚清渭波。于時皇極遘道消之會,群黎蹈顛沛之艱,中華有顧瞻之哀,幽谷無遷喬之望;凶命屢招,姦威仍偪,身寄虎吻,危同朝露,而能抗節玉立,誓不降辱,杜門絕跡,不面偽庭,進免龔勝亡身之禍,退無薛方詭對之譏;雖園、綺之棲商、洛,管寧之默遼海,方之於秀,殆無以過。于今西土,以為美談。夫旌德禮賢,化道之所先,崇表殊節,聖哲之上務。方今六合未康,豺狼當路,遺黎偷薄,義聲弗聞,益宜振起道義之徒,以敦流遁之弊。若秀蒙蒲帛之徵,足以鎮靜頹風,軌訓嚻俗;幽遐仰流,九服知化矣。」及蕭敬叛亂,避難宕渠川中,鄉人宗族馮依者以百數。秀年八十,衆人以其篤老,欲代之負擔,秀拒曰:「各有老弱,當先營救。吾氣力自足堪此,不以垂朽之年累諸君也。」後十餘年,卒於家。}
 
 
\end{pinyinscope}