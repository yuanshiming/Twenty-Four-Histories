\article{鄭渾傳}
\begin{pinyinscope}
 
 
 鄭渾字文公,河南開封人也。高祖父衆,衆父興,皆為名儒。
 
 
\gezhu{續漢書曰:興字少贛,諫議大夫。衆字子師,大司農。}
 渾兄泰,與荀攸等謀誅董卓,為揚州刺史,卒。
 \gezhu{張璠漢紀曰:泰字公業。少有才略,多謀計,知天下將亂,陰交結豪傑。家富於財,有田四百頃,而食常不足,名聞山東。舉孝廉,三府辟,公車徵,皆不就。何進輔政,徵用名士,以泰為尚書侍郎,加奉車都尉。進將誅黃門,欲召董卓為助,泰謂進曰:「董卓彊忍寡義,志欲無饜,若借之朝政,授之大事,將肆其心以危朝廷。以明公之威德,據阿衡之重任,秉意獨斷,誅除有罪,誠不待卓以為資援也。且事留變生,其鑒不遠。」又為陳時之要務,進不能用,乃棄官去。謂潁川人荀攸曰:「何公未易輔也。」進尋見害,卓果專權,廢帝。關東義兵起,卓會議大發兵,羣寮咸憚卓,莫敢忤旨。泰恐其彊,益將難制,乃曰:「夫治在德,不在兵也。」卓不恱曰:「如此,兵無益邪?」衆人莫不變容,為泰震慄。泰乃詭辭對曰:「非以無益,以山東不足加兵也。今山東議欲起兵,州郡相連,人衆相動,非不能也。然中國自光武以來,無雞鳴狗吠之驚,百姓忘戰日乆;仲尼有言『不教民戰,是謂棄之』,雖衆不能為害,一也。明公出自西州,少為國將,閑習軍事,數踐戰場,名稱當世;以此威民,民懷懾服,二也。袁本初公卿子弟,生處京師,體長婦人;張孟卓東平長者,坐不窺堂;孔公緒能清談高論,噓枯吹生,無軍帥之才,負霜露之勤;臨鋒履刃,決敵雌雄,皆非明公敵,三也。察山東之士,力能跨馬控弦,勇等孟賁,捷齊慶忌,信有聊城之守,策有良平之謀;可任以偏師,責以成功,未聞有其人者,四也。就有其人,王爵不相加,婦姑位不定,各恃衆怙力,將人人棊跱,以觀成敗,不肯同心共膽,率徒旅進,五也。關西諸郡,北接上黨、太原、馮翊、扶風、安定,自頃以來,數與胡戰,婦女載戟挾矛,弦弓負矢,況其悍夫;以此當山東忘戰之民,譬驅羣羊向虎狼,其勝可必,六也。且天下之權勇,今見在者不過并、涼、匈奴、屠各、湟中、義從、八種西羌,皆百姓素所畏服,而明公權以為爪牙,壯夫震慄,況小醜乎!七也。又明公之將帥,皆中表腹心,周旋日乆,自三原、硤口以來,恩信醇著,忠誠可遠任,智謀可特使,以此當山東解合之虛誕,實不相若,八也。夫戰有三亡:以亂攻治者亡,以邪攻正者亡,以逆攻順者亡。今明公秉國政平,討夷凶宦,忠義克立;以三德待於三亡,奉辭伐罪,誰人敢禦?九也。東州有鄭康成,學該古今,儒生之所以集;北海邴根矩,清高直亮,羣士之楷式。彼諸將若詢其計畫,案典校之彊弱,燕、趙、齊、梁非不盛,終見滅於秦,吳、楚七國非不衆,而不敢踰滎陽,況今德政之赫赫,股肱之邦良,欲造亂以徼不義者,必不相然讚,成其凶謀,十也。若十事少有可采,無事徵兵以驚天下,使患役之民,相聚為非,棄德恃衆,以輕威重。」卓乃恱,以泰為將軍,統諸軍擊關東。或謂卓曰:「鄭泰智略過人,而結謀山東,今資之士馬,使就其黨,切為明公懼之。」卓收其兵馬,留拜議郎。後又與王允謀共誅卓,泰脫身自武關走,東歸。後將軍袁術以為揚州刺史,未至官,道卒,時年四十二。}
 渾將泰小子袤避難淮南,袁術賔禮甚厚。渾知術必敗。時華歆為豫章太守,素與泰善,渾乃渡江投歆。太祖聞其篤行,召為掾,復遷下蔡長、邵陵令。天下未定,民皆剽輕,不念產殖;其生子無以相活,率皆不舉。渾所在奪其漁獵之具,課使耕桑,又兼開稻田,重去子之法。民初畏罪,後稍豐給,無不舉贍;所育男女,多以鄭為字。辟為丞相掾屬,遷左馮翊。
 
 
 
 
 時梁興等略吏民五千餘家為寇鈔,諸縣不能禦,皆恐懼,寄治郡下。議者悉以為當移就險,渾曰:「興等破散,竄在山阻。雖有隨者,率脅從耳。今當廣開降路,宣喻恩信。而保險自守,此示弱也。」乃聚歛吏民,治城郭,為守禦之備。遂發民逐賊,明賞罰,與要誓,其所得獲,十以七賞。百姓大恱,皆願捕賊,多得婦女、財物。賊之失妻子者,皆還求降。渾責其得他婦女,然後還其妻子,於是轉相寇盜,黨與離散。又遣吏民有恩信者,分布山谷告喻,出者相繼,乃使諸縣長吏各還本治以安集之。興等懼,將餘衆聚鄜城。太祖使夏侯淵就助郡擊之,渾率吏民前登,斬興及其支黨。又賊靳富等,脅將夏陽長、邵陵令并其吏民入磑山,渾復討擊破富等,獲二縣長吏,將其所略還。及趙青龍者,殺左內史程休,渾聞,遣壯士就梟其首。前後歸附四千餘家,由是山賊皆平,民安產業。轉為上黨太守。
 
 
太祖征漢中,以渾為京兆尹。渾以百姓新集,為制移居之法,使兼複者與單輕者相伍,溫信者與孤老為比,勤稼穡,明禁令,以發姦者。由是民安於農,而盜賊止息。及大軍入漢中,運轉軍糧為最。又遣民田漢中,無逃亡者。太祖益嘉之,復入為丞相掾。文帝即位,為侍御史,加駙馬都尉,遷陽平、沛郡二太守。郡界下溼,患水澇,百姓饑乏。渾於蕭、相二縣界,興陂遏,開稻田。郡人皆以為不便,渾曰:「地勢洿下,宜溉灌,終有魚稻經乆之利,此豐民之本也。」遂躬率吏民,興立功夫,一冬閒皆成。比年大收,頃畒歲增,租入倍常,民賴其利,刻石頌之,號曰鄭陂。轉為山陽、魏郡太守,其治放此。又以郡下百姓,苦乏材木,乃課樹榆為籬,並益樹五果;榆皆成藩,五果豐實。入魏郡界,村落齊整如一,民得財足用饒。明帝聞之,下詔稱述,布告天下,遷將作大匠。渾清素在公,妻子不免於饑寒。及卒,以子崇為郎中。
 \gezhu{晉陽秋曰:泰子袤,字林叔。泰與華歆、荀攸善。見袤曰:「鄭公業為不亡矣。」初為臨菑侯文學,稍遷至光祿大夫。泰始七年,以袤為司空,固辭不受,終於家。子默,字思元。晉諸公贊曰:默遵守家業,以篤素稱,位至太常。默弟質、舒、詡,皆為卿。默子球,清直有理識,尚書右僕射、領選。球弟豫,為尚書。}
 
 
\end{pinyinscope}