\article{彭羕傳}
\begin{pinyinscope}
 
 
 彭羕字永年,廣漢人。身長八尺,容貌甚偉。姿性驕傲,多所輕忽,惟敬同郡秦子勑,薦之於太守許靖曰:「昔高宗夢傅說,周文求呂尚,爰及漢祖,納食其於布衣,此乃帝王之所以倡業垂統,緝熈厥功也。今明府稽古皇極,允執神靈,體公劉之德,行勿翦之惠,清廟之作於是乎始,襃貶之義於是乎興,然而六翮未之備也。伏見處士緜竹秦宓,膺山甫之德,履雋生之直,枕石漱流,吟詠縕袍,偃息於仁義之途,恬惔於浩然之域,高概節行,守真不虧,雖古人潛遁,蔑以加旃。若明府能招致此人,必有忠讜落落之譽,豐功厚利,建跡立勳,然後紀功於王府,飛聲於來世,不亦美哉!」
 
 
 
 
 羕仕州,不過書佐,後又為衆人所謗毀於州牧劉璋,璋髠鉗羕為徒隷。會先主入蜀,泝流北行。羕欲納說先主,乃往見龐統。統與羕非故人,又適有賔客,羕徑上統牀卧,謂統曰:「須客罷當與卿善談。」統客旣罷,往就羕坐,羕又先責統食,然後共語,因留信宿,至于經日。統大善之,而法正宿自知羕,遂並致之先主。先主亦以為奇,數令羕宣傳軍事,指授諸將,奉使稱意,識遇日加。成都旣定,先主領益州牧,拔羕為治中從事。羕起徒步,一朝處州人之上,形色嚻然,自矜得遇滋甚。諸葛亮雖外接待羕,而內不能善。屢密言先主,羕心大志廣,難可保安。先主旣敬信亮,加察羕行事,意以稍踈,左遷羕為江陽太守。
 
 
 
 
 羕聞當遠出,私情不恱,往詣馬超。超問羕曰:「卿才具秀拔,主公相待至重,謂卿當與孔明、孝直諸人齊足並驅,寧當外授小郡,失人本望乎?」羕曰:「老革荒悖,可復道邪!」
 
 
\gezhu{揚雄方言曰:滅、鰓、乾、都、耆、革,老也。郭璞注曰:皆老者皮色枯瘁之形也。臣松之以為皮去毛曰革。古者以革為兵,故語稱兵革,革猶兵也。羕罵備為老革,猶言老兵也。}
 又謂超曰:「卿為其外,我為其內,天下不足定也。」超羇旅歸國,常懷危懼,聞羕言大驚,默然不荅。羕退,具表羕辭,於是收羕付有司。
 
 
羕於獄中與諸葛亮書曰:「僕昔有事於諸侯,以為曹操暴虐,孫權無道,振威闇弱,其惟主公有霸王之器,可與興業致治,故乃翻然有輕舉之志。會公來西,僕因法孝直自衒鬻,龐統斟酌其間,遂得詣公於葭萌,指掌而譚,論治世之務,講霸王之業,建取益州之策,公亦宿慮明定,即相然贊,遂舉事焉。僕於故州不免凡庸,憂於罪罔,得遭風雲激矢之中,求君得君,志行名顯,從布衣之中擢為國士,盜竊茂才。分子之厚,誰復過此。
 \gezhu{臣松之以為「分子之厚」者,羕言劉主分兒子厚恩,施之於己,故其書後語云「負我慈父,罪有百死」也。}
 羕一朝狂悖,自求葅醢,為不忠不義之鬼乎!先民有言,左手據天下之圖,右手刎咽喉,愚夫不為也。況僕頗別菽麥者哉!所以有怨望意者,不自度量,苟以為首興事業,而有投江陽之論,不解主公之意,意卒感激,頗以被酒,侻失『老』語。此僕之下愚薄慮所致,主公實未老也。且夫立業,豈在老少,西伯九十,寧有衰志,負我慈父,罪有百死。至於內外之言,欲使孟起立功北州,勠力主公,共討曹操耳,寧敢有他志邪?孟起說之是也,但不分別其間,痛人心耳。昔每與龐統共相誓約,庶託足下末蹤,盡心於主公之業,追名古人,載勳竹帛。統不幸而死,僕敗以取禍。自我惰之,將復誰怨!足下,當世伊、呂也,宜善與主公計事,濟其大猷。天明地察,神祇有靈,復何言哉!貴使足下明僕本心耳。行矣努力,自愛,自愛!」羕竟誅死,時年三十七。
 
 
\end{pinyinscope}