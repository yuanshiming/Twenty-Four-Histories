\article{高句麗傳}
\begin{pinyinscope}
 
 
 高句麗在遼東之東千里,南與朝鮮、濊貊,東與沃沮,北與夫餘接。都於丸都之下,方可二千里,戶三萬。多大山深谷,無原澤。隨山谷以為居,食澗水。無良田,雖力佃作,不足以實口腹。其俗節食,好治宮室,於所居之左右立大屋,祭鬼神,又祠靈星、社稷。其人性凶急,善寇鈔。其國有王,其官有相加、對盧、沛者、古雛加、主簿、優台丞、使者、皁衣先人,尊卑各有等級。東夷舊語以為夫餘別種,言語諸事,多與夫餘同,其性氣衣服有異。本有五族,有涓奴部、絕奴部、順奴部、灌奴部、桂婁部。本涓奴部為王,稍微弱,今桂婁部代之。漢時賜鼓吹伎人,常從玄菟郡受朝服衣幘,高句麗令主其名籍。後稍驕恣,不復詣郡,於東界築小城,置朝服衣幘其中,歲時來取之,今胡猶名此城為幘溝漊。溝漊者,句麗名城也。其置官,有對盧則不置沛者,有沛者則不置對盧。王之宗族,其大加皆稱古雛加。涓奴部本國主,今雖不為王,適統大人,得稱古雛加,亦得立宗廟,祠靈星、社稷。絕奴部世與王婚,加古雛之號。諸大加亦自置使者、皁衣先人,名皆達於王,如卿大夫之家臣,會同坐起,不得與王家使者、皁衣先人同列。其國中大家不佃作,坐食者萬餘口,下戶遠擔米糧魚鹽供給之。其民喜歌舞,國中邑落,暮夜男女群聚,相就歌戲。無大倉庫,家家自有小倉,名之為桴京。其人絜清自喜,善藏釀。跪拜申一脚,與夫餘異,行步皆走。以十月祭天,國中大會,名曰東盟。其公會,衣服皆錦繡金銀以自飾。大加主簿頭著幘,如幘而無餘,其小加著折風,形如弁。其國東有大穴,名隧穴,十月國中大會,迎隧神還於國東上祭之,置木隧於神坐。無牢獄,有罪諸加評議,便殺之,沒入妻子為奴婢。其俗作婚姻,言語已定,女家作小屋於大屋後,名壻屋,壻暮至女家戶外,自名跪拜,乞得就女宿,如是者再三,女父母乃聽使就小屋中宿,傍頓錢帛,至生子已長大,乃將婦歸家。其俗淫。男女已嫁娶,便稍作送終之衣。厚葬,金銀財幣,盡於送死,積石為封,列種松栢。其馬皆小,便登山。國人有氣力,習戰鬪,沃沮、東濊皆屬焉。又有小水貊。句麗作國,依大水而居,西安平縣北有小水,南流入海,句麗別種依小水作國,因名之為小水貊,出好弓,所謂貊弓是也。
 
 
 
 
 王莽初發高句麗兵以伐胡,欲不行,彊迫遣之,皆亡出塞為寇盜。遼西大尹田譚追擊之,為所殺。州郡縣歸咎於句麗侯騊,嚴尤奏言:「貊人犯法,罪不起於騊,且宜安慰。今猥被之大罪,恐其遂反。」莽不聽,詔尤擊之。尤誘期句麗侯騊至而斬之,傳送其首詣長安。莽大恱,布告天下,更名高句麗為下句麗。當此時為侯國,漢光武帝八年,高句麗王遣使朝貢,始見稱王。
 
 
 
 
 至殤、安之間,句麗王宮數寇遼東,更屬玄菟。遼東太守蔡風、玄菟太守姚光以宮為二郡害,興師伐之。宮詐降請和,二郡不進。宮密遣軍攻玄菟,焚燒候城,入遼隧,殺吏民。後宮復犯遼東,蔡風輕將吏士追討之,軍敗沒。
 
 
 
 
 宮死,子伯固立。順、桓之間,復犯遼東,寇新安、居鄉,又攻西安平,於道上殺帶方令,略得樂浪太守妻子。靈帝建寧二年,玄菟太守耿臨討之,斬首虜數百級,伯固降,屬遼東。熹平中,伯固乞屬玄菟。公孫度之雄海東也,伯固遣大加優居、主簿然人等助度擊富山賊,破之。
 
 
 
 
 伯固死,有二子,長子拔奇,小子伊夷模。拔奇不肖,國人便共立伊夷模為王。自伯固時,數寇遼東,又受亡胡五百餘家。建安中,公孫康出軍擊之,破其國,焚燒邑落。拔奇怨為兄而不得立,與涓奴加各將下戶三萬餘口詣康降,還住沸流水。降胡亦叛伊夷模,伊夷模更作新國,今日所在是也。拔奇遂往遼東,有子留句麗國,今古雛加駮位居是也。其後復擊玄菟,玄菟與遼東合擊,大破之。
 
 
 
 
 伊夷模無子,淫灌奴部,生子名位宮。伊夷模死,立以為王,今句麗王宮是也。其曾祖名宮,生能開目視,其國人惡之,及長大,果凶虐,數寇鈔,國見殘破。今王生墯地,亦能開目視人。句麗呼相似為位,似其祖,故名之為位宮。位宮有力勇,便鞌馬,善獵射。景初二年,太尉司馬宣王率衆討公孫淵,宮遣主簿大加將數千人助軍。正始三年,宮寇西安平,其五年,為幽州刺吏毌丘儉所破。語在儉傳。
 
 
\end{pinyinscope}