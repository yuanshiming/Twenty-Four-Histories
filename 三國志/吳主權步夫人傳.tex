\article{吳主權步夫人傳}
\begin{pinyinscope}
 
 
 吳主權步夫人,臨淮淮陰人也,與丞相隲同族。漢末,其母攜將徙廬江,廬江為孫策所破,皆東渡江,以美麗得幸於權,寵冠後庭。生二女,長曰魯班,字大虎,前配周瑜子循,後配全琮;少曰魯育,字小虎,前配朱據,後配劉纂。
 
 
\gezhu{吳歷曰:纂先尚權中女,早卒,故又以小虎為繼室。}
 
 
 
 
 夫人性不妬忌,多所推進,故乆見愛待。權為王及帝,意欲以為后,而羣臣議在徐氏,權依違者十餘年,然宮內皆稱皇后,親戚上疏稱中宮。及薨,臣下緣權指,請追正名號,乃贈印綬,策命曰:「惟赤烏元年閏月戊子,皇帝曰:嗚呼皇后,惟后佐命,共承天地。虔恭夙夜,與朕均勞。內教脩整,禮義不愆。寬容慈惠,有淑懿之德。民臣縣望,遠近歸心。朕以世難未夷,大統未一,緣后雅志,每懷謙損。是以于時未授名號,亦必謂后降年有永,永與朕躬對揚天休。不寤奄忽,大命近止。朕恨本意不早昭顯,傷后殂逝,不終天祿。愍悼之至,痛于厥心。今使使持節丞相醴陵亭侯雍,奉策授號,配食先后。魂而有靈,嘉其寵榮。嗚呼哀哉!」葬於蔣陵。
 
 
\end{pinyinscope}