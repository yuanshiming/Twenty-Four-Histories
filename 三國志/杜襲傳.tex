\article{杜襲傳}
\begin{pinyinscope}
 
 
 杜襲字子緒,潁川定陵人也。曾祖父安,祖父根,著名前世。
 
 
\gezhu{先賢行狀曰:安年十歲,名稱鄉黨。至十五,入太學,號曰神童。旣名知人,清高絕俗。洛陽令周紆數候安,安常逃避不見。時貴戚慕安高行,多有與書者,輒不發,以慮後患,常鑿壁藏書。後諸與書者果有大罪,推捕所與交通者,吏至門,安乃發壁出書,印封如故,當時皆嘉其慮遠。三府並辟,公車特徵,拜宛令。先是宛有報讎者,其令不忍致理,將與俱亡。縣中豪彊有告其處者,致捕得。安深疾惡之,到官治戮,肆之於巿。懼有司繩彈,遂自免。後徵拜巴郡太守,率身正下,以禮化俗。以病卒官,時服薄斂,素器不漆,子自將車。州郡賢之,表章墳墓。根舉孝廉,除郎中。時和熹鄧后臨朝,外戚橫恣,安帝長大,猶未歸政。根乃與同時郎上書直諫,鄧后怒,收根等伏誅。誅者皆絹囊盛,於殿上撲地。執法者以根德重事公,默語行事人,使不加力。誅訖,車載城外,根以撲輕得蘇息,遂閉目不動搖。經三日,乃密起逃竄,為宜城山中酒家客,積十五年,酒家知其賢,常厚敬待。鄧后崩,安帝謂根乆死。以根等忠直,普下天下,錄見誅者子孫。根乃自出,徵詣公車,拜符節令。或問根:「往日遭難,天下同類知故不少,何至自苦歷年如此?」根荅曰:「周旋人間,非絕迹之處。邂逅發露,禍及親知,故不為也。」遷濟陰太守,以德讓為政,風移俗改。年七十八以壽終,棺不加漆,斂以時服。長吏下車,常先詣安、根墓致祠。}
 襲避亂荊州,劉表待以賔禮。同郡繁欽數見奇於表,襲喻之曰:「吾所以與子俱來者,徒欲龍蟠幽藪,待時鳳翔。豈謂劉牧當為撥亂之主,而規長者委身哉?子若見能不已,非吾徒也。吾其與子絕矣!」欽慨然曰:「請敬受命。」襲遂南適長沙。
 
 
建安初,太祖迎天子都許。襲逃還鄉里,太祖以為西鄂長。縣濵南境,寇賊縱橫。時長吏皆歛民保城郭,不得農業。野荒民困,倉庾空虛。襲自知恩結於民,乃遣老弱各分散就田業,留丁彊備守,吏民歡恱。會荊州出步騎萬人來攻城,襲乃悉召縣吏民任拒守者五十餘人,與之要誓。其親戚在外欲自營護者,恣聽遣出;皆叩頭願致死。於是身執矢石,率與戮力。吏民感恩,咸為用命。臨陣斬數百級,而襲衆死者三十餘人,其餘十八人盡被創,賊得入城。襲帥傷痍吏民決圍得出,死喪略盡,而無反背者。遂收散民,徙至摩陂營,吏民慕而從之如歸。
 \gezhu{九州春秋曰:建安六年,劉表攻西鄂,西鄂長杜子緒帥縣男女嬰城而守。時南陽功曹栢孝長亦在城中,聞兵攻聲,恐懼,入室閉戶,牽被覆頭。相攻半日,稍敢出面。其明,側立而聽。二日,往出戶問消息。至四五日,乃更負楯親鬬,語子緒曰:「勇可習也。」}
 
 
 
 
 司隷鍾繇表拜議郎參軍事。荀彧又薦襲,太祖以為丞相軍祭酒。魏國旣建,為侍中,與王粲、和洽並用。粲彊識博聞,故太祖游觀出入,多得驂乘,至其見敬不及洽、襲。襲嘗獨見,至于夜半。粲性躁競,起坐曰:「不知公對杜襲道何等也?」洽笑荅曰:「天下事豈有盡邪?卿晝侍可矣,悒悒於此,欲兼之乎!」後襲領丞相長史,隨太祖到漢中討張魯。太祖還,拜襲駙馬都尉,留督漢中軍事。綏懷開導,百姓自樂出徙洛、鄴者,八萬餘口。夏侯淵為劉備所沒,軍喪元帥,將士失色。襲與張郃、郭淮糾攝諸軍事,權宜以郃為督,以一衆心,三軍遂定。太祖東還,當選留府長史,鎮守長安,主者所選多不當,太祖令曰:「釋騏驥而不乘,焉皇皇而更索?」遂以襲為留府長史,駐關中。
 
 
 
 
 時將軍許攸擁部曲,不附太祖而有慢言。太祖大怒,先欲伐之。羣臣多諫:「可招懷攸,共討彊敵。」太祖橫刀於膝,作色不聽。襲入欲諫,太祖逆謂之曰:「吾計以定,卿勿復言。」襲曰:「若殿下計是邪,臣方助殿下成之;若殿下計非邪,雖成宜改之。殿下逆臣,令勿言之,何待下之不闡乎?」太祖曰:「許攸慢吾,如何可置乎?」襲曰:「殿下謂許攸何如人邪?」太祖曰:「凡人也。」襲曰:「夫惟賢知賢,惟聖知聖,凡人安能知非凡人邪?方今犲狼當路而狐狸是先,人將謂殿下避彊攻弱,進不為勇,退不為仁。臣聞千鈞之弩不為鼷鼠發機,萬石之鍾不以莛撞起音,今區區之許攸,何足以勞神武哉?」太祖曰:「善。」遂厚撫攸,攸即歸服。時夏侯尚暱於太子,情好至密。襲謂尚非益友,不足殊待,以聞太祖。文帝初甚不恱,後乃追思。語在尚傳。其柔而不犯,皆此類也。
 
 
 
 
 文帝即王位,賜爵關內侯。及踐阼,為督軍糧御史,封武平亭侯,更為督軍糧執法,入為尚書。明帝即位,進封平陽鄉侯。諸葛亮出秦川,大將軍曹真督諸軍拒亮,徙襲為大將軍軍師,分邑百戶賜兄基爵關內侯。真薨,司馬宣王代之,襲復為軍師,增邑三百,并前五百五十戶。以疾徵還,拜太中大夫。薨,追贈少府,謚曰定侯。子會嗣。
 
 
\end{pinyinscope}