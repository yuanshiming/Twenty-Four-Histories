\article{董和傳}
\begin{pinyinscope}
 
 
 董和字幼宰,南郡枝江人也,其先本巴郡江州人。漢末,和率宗族西遷,益州牧劉璋以為牛鞞、
 
 
\gezhu{音髀。}
 江原長、成都令。蜀土富實,時俗奢侈,貨殖之家,侯服玉食,婚姻葬送,傾家竭產。和躬率以儉,惡衣蔬食,防遏踰僭,為之軌制,所在皆移風變善,畏而不犯。然縣界豪彊憚和嚴法,說璋轉和為巴東屬國都尉。吏民老弱相攜乞留和者數千人,璋聽留二年,還遷益州太守,其清約如前。與蠻夷從事,務推誠心,南土愛而信之。
 
 
先主定蜀,徵和為掌軍中郎將,與軍師將軍諸葛亮並署左將軍大司馬府事,獻可替否,共為歡交。自和居官食祿,外牧殊域,內幹機衡,二十餘年,死之日家無儋石之財。亮後為丞相,教與羣下曰:「夫參署者,集衆思廣忠益也。若遠小嫌,難相違覆,曠闕損矣。違覆而得中,猶弃弊蹻而獲珠玉。然人心苦不能盡,惟徐元直處茲不惑,又董幼宰參署七年,事有不至,至于十反,來相啟告。苟能慕元直之十一,幼宰之殷勤,有忠於國,則亮可少過矣。」又曰:「昔初交州平,屢聞得失,後交元直,勤見啟誨,前參事於幼宰,每言則盡,後從事於偉度,數有諫止;雖恣性鄙暗,不能悉納,然與此四子終始好合,亦足以明其不疑於直言也。」其追思和如此。
 \gezhu{偉度者,姓胡,名濟,義陽人。為亮主簿,有忠盡之效,故見襃述。亮卒,為中典軍,統諸軍,封成陽亭侯,遷中監軍前將軍,督漢中,假節領兖州刺史,至右驃騎將軍。濟弟博,歷長水校尉尚書。}
 
 
\end{pinyinscope}