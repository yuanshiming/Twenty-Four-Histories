\article{劉繇傳}
\begin{pinyinscope}
 
 
 劉繇字正禮,東萊牟平人也。齊孝王少子封牟平侯,子孫家焉。繇伯父寵,為漢太尉。
 
 
\gezhu{續漢書曰:繇祖父本,師受經傳,博學羣書,號為通儒。舉賢良方正,為般長,卒官。寵字祖榮,受父業,以經明行脩,舉孝廉,光祿大夫察四行,除東平陵令。視事數年,以母病棄官,百姓士民攀輿拒輪,充塞道路,車不得前,乃止亭,輕服潛遁,歸脩供養。後辟大將軍府,稍遷會稽太守,正身率下,郡中大治。徵入為將作大匠。山陰縣民去治數十里有若邪中在山谷間,五六老翁年皆七八十,聞寵遷,相率共送寵,人齎百錢。寵見,勞來曰:「父老何乃自苦遠來!」皆對曰:「山谷鄙老,生未嘗至郡縣。他時吏發求不去,民閒或夜不絕狗吠,竟夕民不得安。自明府下車以來,狗不夜吠,吏稀至民閒,年老遭值聖化,今聞當見棄去,故勠力來送。」寵謝之,為選受一大錢,故會稽號寵為取一錢太守。其清如是。寵前後歷二郡,八居九列,四登三事。家不藏賄,無重寶器,恒菲飲食,薄衣服,弊車羸馬,號為窶陋。三去相位,輒歸本土。往來京師,常下道脫驂過,人莫知焉。寵嘗欲止亭,亭吏止之曰:「整頓傳舍,以待劉公,不可得止。」寵因過去。其廉儉皆此類也。以老病卒于家。}
 繇兄岱,字公山,歷位侍中,兖州刺吏。
 \gezhu{續漢書曰:繇父輿,一名方,山陽太守。岱、繇皆有雋才。英雄記稱岱孝悌仁恕,以虛己受人。}
 
 
繇十九,從父韙為賊所劫質,繇篡取以歸,由是顯名。舉孝廉,為郎中,除下邑長。時郡守以貴戚託之,遂棄官去。州辟部濟南,濟南相中常侍子,貪穢不循,繇奏免之。平原陶丘洪薦繇,欲令舉茂才。刺史曰:「前年舉公山,柰何復舉正禮乎?」洪曰:「若明使君用公山於前,擢正禮於後,所謂御二龍於長塗,騁騏驥於千里,不亦可乎!」會辟司空掾,除侍御史,不就。避亂淮浦,詔書以為揚州刺史。時袁術在淮南,繇畏憚,不敢之州。欲南渡江,吳景、孫賁迎置曲阿。術圖為僭逆,攻沒諸郡縣。繇遣樊能、張英屯江邊以拒之。以景、賁術所授用,乃迫逐使去。於是術乃自置揚州刺史,與景、賁并力攻英、能等,歲餘不下。漢命加繇為牧,振武將軍,衆數萬人,孫策東渡,破英、能等。繇奔丹徒,
 \gezhu{袁宏漢紀曰:劉繇將奔會稽,許子將曰:「會稽富實,策之所貪,且窮在海隅,不可往也。不如豫章,北連豫壤,西接荊州。若收合吏民,遣使貢獻,與曹兖州相聞,雖有袁公路隔在其間,其人豺狼,不能乆也。足下受王命,孟德、景升必相救濟。」繇從之。}
 遂泝江南保豫章,駐彭澤。笮融先至,
 \gezhu{笮音壯力反。}
 殺太守朱皓,
 \gezhu{獻帝春秋曰:是歲,繇屯彭澤,又使融助皓討劉表所用太守諸葛玄。許子將謂繇曰:「笮融出軍,不顧命名義者也。朱文明善推誠以信人,宜使密防之。」融到,果詐殺皓,代領郡事。}
 入居郡中。繇進討融,為融所破,更復招合屬縣,攻破融。融敗走入山,為民所殺,繇尋病卒,時年四十二。
 
 
 
 
 笮融者,丹楊人,初聚衆數百,往依徐州牧陶謙。謙使督廣陵、彭城運漕,遂放縱擅殺,坐斷三郡委輸以自入。乃大起浮圖祠,以銅為人,黃金塗身,衣以錦采,垂銅槃九重,下為重樓閣道,可容三千餘人,悉課讀佛經,令界內及旁郡人有好佛者聽受道,復其他役以招致之,由此遠近前後至者五千餘人戶。每浴佛,多設酒飯,布席於路,經數十里,民人來觀及就食且萬人,費以巨億計。曹公攻陶謙,徐土搔動,融將男女萬口,馬三千匹,走廣陵,廣陵太守趙昱待以賔禮。先是,彭城相薛禮為陶謙所偪,屯秣陵。融利廣陵之衆,因酒酣殺昱,放兵大略,因載而去。過殺禮,然後殺皓。
 
 
 
 
 後策西伐江夏,還過豫章,收載繇喪,善遇其家。王朗遺策書曰:「劉正禮昔初臨州,未能自達,實賴尊門為之先後,用能濟江成治,有所處定。踐境之禮,感分結意,情在終始。後以袁氏之嫌,稍更乖剌。更以同盟,還為讎敵,原其本心,實非所樂。康寧之後,常願渝平更成,復踐宿好。一爾分離,款意不昭,奄然殂隕,可為傷恨!知敦以厲薄,德以報怨,收骨育孤,哀亡愍存,捐旣往之猜,保六尺之託,誠深恩重分,美名厚實也。昔魯人雖有齊怨,不廢喪紀,春秋善之,謂之得禮,誠良史之所宜藉,鄉校之所歎聞。正禮元子,致有志操,想必有以殊異。威盛刑行,施之以恩,不亦優哉!」
 
 
繇長子基,字敬輿,年十四,居繇喪盡禮,故吏餽餉,皆無所受。
 \gezhu{吳書曰:基遭多難,嬰丁困苦,潛處味道,不以為戚。與羣弟居,常夜卧早起,妻妾希見其面。諸弟敬憚,事之猶父。不妄交游,門無雜賔。}
 姿容美好,孫權愛敬之。權為驃騎將軍,辟東曹掾,拜輔義校尉、建忠中郎。權為吳王,遷基大農。權嘗宴飲,騎都尉虞翻醉酒犯忤,權欲殺之,威怒甚盛,由基諫爭,翻以得免。權大暑時,嘗於舩中宴飲,於船樓上值雷雨,權以蓋自覆,又命覆基,餘人不得也。其見待如此。徙郎中令。權稱尊號,改為光祿勳,分平尚書事。年四十九卒。後權為子霸納基女,賜第一區,四時寵賜,與全、張比。基二弟,鑠、尚,皆騎都尉。
 
 
\end{pinyinscope}