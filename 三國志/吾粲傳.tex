\article{吾粲傳}
\begin{pinyinscope}
 
 
 吾粲字孔休,吳郡烏程人也。
 
 
\gezhu{吳錄曰:粲生數歲,孤城嫗見之,謂其母曰:「是兒有卿相之骨。」}
 孫河為縣長,粲為小吏,河深奇之。河後為將軍,得自選長吏,表粲為曲阿丞,遷為長史,治有名迹。雖起孤微,與同郡陸遜、卜靜等比肩齊聲矣。孫權為車騎將軍,召為主簿,出為山陰令,還為參軍校尉。
 
 
 
 
 黃武元年,與呂範、賀齊等俱以舟師拒魏將曹休於洞口。值天大風,諸船綆紲斷絕,漂沒著岸,為魏軍所獲,或覆沒沈溺,其大船尚存者,水中生人皆攀緣號呼,他吏士恐船傾沒,皆以戈矛撞擊不受。粲與黃淵獨令船人以承取之,左右以為船重必敗,粲曰:「船敗,當俱死耳!人窮,柰何棄之。」粲、淵所活者百餘人。
 
 
 
 
 還,遷會稽太守,召處士謝譚為功曹,譚以疾不詣,粲教曰:「夫應龍以屈伸為神,鳳皇以嘉鳴為貴,何必隱形於天外,潛鱗於重淵者哉?」粲募合人衆,拜昭義中郎將,與呂岱討平山越,入為屯騎校尉、少府,遷太子太傅。遭二宮之變,抗言執正,明嫡庶之分,欲使魯王霸出駐夏口,遣楊笁不得令在都邑。又數以消息語陸遜,遜時駐武昌,連表諫爭。由此為霸、笁等所譖害,下獄誅。
 
 
\end{pinyinscope}