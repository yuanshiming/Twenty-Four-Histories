\article{費詩傳}
\begin{pinyinscope}
 
 
 費詩字公舉,犍為南安人也。劉璋時為緜竹令,先主攻緜竹時,詩先舉城降。成都旣定,先主領益州牧,以詩為督軍從事,出為䍧牱太守,還為州前部司馬。先主為漢中王,遣詩拜關羽為前將軍,羽聞黃忠為後將軍,羽怒曰:「大丈夫終不與老兵同列!」不肯受拜。詩謂羽曰:「夫立王業者,所用非一。昔蕭、曹與高祖少小親舊,而陳、韓亡命後至,論其班列,韓最居上,未聞蕭、曹以此為怨。今漢王以一時之功,隆崇於漢室,然意之輕重,寧當與君侯齊乎!且王與君侯,譬猶一體,同休等戚,禍福共之,愚為君侯,不宜計官號之高下,爵祿之多少為意也。僕一介之使,銜命之人,君侯不受拜,如是便還,但相為惜此舉動,恐有後悔耳!」羽大感悟,遽即受拜。
 
 
 
 
 後羣臣議欲推漢中王稱尊號,詩上疏曰:「殿下以曹操父子偪主篡位,故乃羇旅萬里,糾合士衆,將以討賊。今大敵未克,而先自立,恐人心疑惑。昔高祖與楚約,先破秦者王。及屠咸陽,獲子嬰,猶懷推讓,況今殿下未出門庭,便欲自立邪!愚臣誠不為殿下取也。」由是忤指,左遷部永昌從事。
 
 
\gezhu{習鑿齒曰:夫創本之君,須大定而後正己,纂統之主,俟速建以係衆心,是故惠公朝虜而子圉夕立,更始尚存而光武舉號,夫豈忘主徼利,社稷之故也。今先主糾合義兵,將以討賊。賊彊禍大,主沒國喪,二祖之廟,絕而不祀,苟非親賢,孰能紹此?嗣祖配天,非咸陽之譬,杖正討逆,何推讓之有?於此時也,不知速尊有德以奉大統,使民欣反正,世覩舊物,杖順者齊心,附逆者同懼,可謂闇惑矣。其黜降也宜哉!臣松之以為鑿齒論議,惟此議最善。}
 建興三年,隨諸葛亮南行,歸至漢陽縣,降人李鴻來詣亮,亮見鴻,時蔣琬與詩在坐。鴻曰:「閒過孟達許,適見王沖從南來,言往者達之去就,明公切齒,欲誅達妻子,賴先主不聽耳。達曰:『諸葛亮見顧有本末,終不爾也。』盡不信沖言,委仰明公,無復已已。」亮謂琬、詩曰:「還都當有書與子度相聞。」詩進曰:「孟達小子,昔事振威不忠,後又背叛先主,反覆之人,何足與書邪!」亮默然不荅。亮欲誘達以為外援,竟與達書曰:「往年南征,歲未及還,適與李鴻會於漢陽,承知消息,慨然永歎,以存足下平素之志,豈徒空託名榮,貴為乖離乎!嗚呼孟子,斯實劉封侵陵足下,以傷先主待士之義。又鴻道王沖造作虛語,云足下量度吾心,不受沖說。尋表明之言,追平生之好,依依東望,故遣有書。」達得亮書,數相交通,辭欲叛魏。魏遣司馬宣王征之,即斬滅達。亮亦以達無款誠之心,故不救助也。蔣琬秉政,以詩為諫議大夫,卒於家。
 
 
王沖者,廣漢人也。為牙門將,統屬江州督李嚴。為嚴所疾,懼罪降魏。魏以沖為樂陵太守。
 \gezhu{孫盛蜀世譜曰:詩子立,晉散騎常侍。自後益州諸費有名位者,多是詩之後也。}
 
 
 
 
 評曰:霍峻孤城不傾,王連固節不移,向朗好學不倦,張裔膚敏應機,楊洪乃心忠公,費詩率意而言,皆有可紀焉。以先主之廣濟,諸葛之準繩,詩吐直言,猶用陵遲,況庸后乎哉!
 
 
\end{pinyinscope}