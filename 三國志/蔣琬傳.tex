\article{蔣琬傳}
\begin{pinyinscope}
 
 
 蔣琬字公琰、零陵湘鄉人也。弱冠與外弟泉陵劉敏俱知名。琬以州書佐隨先主入蜀,除廣都長。先主甞因游觀奄至廣都,見琬衆事不理,時又沈醉,先主大怒,將加罪戮。軍師將軍諸葛亮請曰:「蔣琬,社稷之器,非百里之才也。其為政以安民為本,不以脩飾為先,願主公重加察之。」先主雅敬亮,乃不加罪,倉卒但免官而已。
 
 
 
 
 琬見推之後,夜夢有一牛頭在門前,流血滂沲,意甚惡之,呼問占夢趙直。直曰:「夫見血者,事分明也。牛角及鼻,『公』字之象,君位必當至公,大吉之徵也。」頃之,為什邡令。先主為漢中王,琬入為尚書郎。建興元年,丞相亮開府,辟琬為東曹掾。舉茂才,琬固讓劉邕、陰化、龐延、廖淳,亮教荅曰:「思惟背親捨德,以殄百姓,衆人旣不隱於心,實又使遠近不解其義,是以君宜顯其功舉,以明此選之清重也。」遷為參軍。五年,亮住漢中,琬與長史張裔統留府事。八年,代裔為長史,加撫軍將軍。亮數外出,琬常足食足兵以相供給。亮每言:「公琰託志忠雅,當與吾共贊王業者也。」密表後主曰:「臣若不幸,後事宜以付琬。」
 
 
 
 
 亮卒,以琬為尚書令,俄而加行都護,假節,領益州刺史,遷大將軍,錄尚書事,封安陽亭侯。時新喪元帥,遠近危悚。琬出類拔萃,處羣僚之右,旣無戚容,又無喜色,神守舉止,有如平日,由是衆望漸服,延熈元年,詔琬曰:「寇難未弭,曹叡驕凶,遼東三郡苦其暴虐,遂相糾結,與之離隔。叡大興衆役,還相攻伐。曩秦之亡,勝、廣首難,今有此變,斯乃天時。君其治嚴,總帥諸軍屯住漢中,須吳舉動,東西掎角,以乘其釁。」又命琬開府,明年就加為大司馬。
 
 
 
 
 東曹掾楊戲素性簡略,琬與言論,時不應荅。或欲搆戲於琬曰:「公與戲語而不見應,戲之慢上,不亦甚乎!」琬曰:「人心不同,各如其靣;靣從後言,古人之所誡也。戲欲贊吾是耶,則非其本心,欲反吾言,則顯吾之非,是以默然,是戲之快也。」又督農楊敏曾毀琬曰:「作事憒憒,誠非及前人。」或以白琬,主者請推治敏,琬曰:「吾實不如前人,無可推也。」主者重據聽不推,則乞問其憒憒之狀。琬曰:「苟其不如,則事不當理,事不當理,則憒憒矣。復何問邪?」後敏坐事繫獄,衆人猶懼其必死,琬心無適莫,得免重罪。其好意存道,皆此類也。
 
 
 
 
 琬以為昔諸葛亮數闚秦川,道險運艱,竟不能克,不若乘水東下。乃多作舟舩,欲由漢、沔襲魏興、上庸。會舊疾連動,未時得行。而衆論咸謂如不克捷,還路甚難,非長策也。於是遣尚書令費禕、中監軍姜維等喻指。琬承命上疏曰:「芟穢弭難,臣職是掌。自臣奉辭漢中,已經六年,臣旣闇弱,加嬰疾疢,規方無成,夙夜憂慘。今魏跨帶九州,根蔕滋蔓,平除未易。若東西并力,首尾掎角,雖未能速得如志,且當分裂蠶食,先摧其支黨。然吳期二三,連不克果,俯仰惟艱,實忘寢食。輙與費禕等議,以涼州胡塞之要,進退有資,賊之所惜;且羌、胡乃心思漢如渴,又昔偏軍入羌,郭淮破走,筭其長短,以為事首,宜以姜維為涼州刺史。若維征行,銜持河右,臣當帥軍為維鎮繼。今涪水陸四通,惟急是應,若東北有虞,赴之不難。」由是琬遂還住涪。疾轉增劇,至九年卒,謚曰恭。
 
 
 
 
 子斌嗣,為綏武將軍、漢城護軍。魏大將軍鍾會至漢城,與斌書曰:「巴蜀賢智文武之士多矣。至於足下、諸葛思遠,譬諸草木,吾氣類也。桑梓之敬,古今所敦。西到,欲奉瞻尊大君公侯墓,當洒埽墳塋,奉祠致敬。願告其所在!」斌荅書曰:「知惟臭味意眷之隆,雅託通流,未拒來謂也。亡考昔遭疾疢,亡於涪縣,卜云其吉,遂安厝之。知君西邁,乃欲屈駕脩敬墳墓。視予猶父,顏子之仁也,聞命感愴,以增情思。」會得斌書報,嘉歎意義,及至涪,如其書云。
 
 
 
 
 後主旣降鄧艾,斌詣會於涪,待以交友之禮。隨會至成都,為亂兵所殺。斌弟顯,為太子僕,會亦愛其才學,與斌同時死。
 
 
 
 
 劉敏,左護軍、揚威將軍,與鎮北大將軍王平俱鎮漢中。魏遣大將軍曹爽襲蜀時,議者或謂但可守城,不出拒敵,必自引退。敏以為男女布野,農糓栖畒,若聽敵入,則大事去矣。遂帥所領與平據興勢,多張旗幟,彌亘百餘里。會大將軍費禕從成都至,魏軍即退,敏以功封雲亭侯。
 
 
\end{pinyinscope}