\article{王蕃傳}
\begin{pinyinscope}
 
 
 王蕃字永元,廬江人也。愽覽多聞,兼通術藝。始為尚書郎,去官。孫休即位,與賀邵、薛瑩、虞汜俱為散騎中常侍,皆加駙馬都尉。時論清之。遣使至蜀,蜀人稱焉,還為夏口監軍。
 
 
 
 
 孫皓初,復入為常侍,與萬彧同官。彧與皓有舊,俗王挾侵,謂蕃自輕。又中書丞陳聲,皓之嬖臣,數譖毀蕃。蕃體氣高亮,不能承顏順指,時或迕意,積以見責。
 
 
 
 
 甘露二年,丁忠使晉還,皓大會羣臣,蕃沈醉頓伏,皓疑而不恱,轝蕃出外。頃之請還,酒亦不解。蕃性有威嚴,行止自若,皓大怒,呵左右於殿下斬之。衞將軍滕牧、征西將軍留平請,不能得。
 
 
\gezhu{江表傳曰:皓用巫史之言,謂建業宮不利,乃西巡武昌,仍有遷都之意,恐羣臣不從,乃大請會,賜將吏。問蕃「射不主皮,為力不同科,其義云何」?蕃思惟未荅,即於殿上斬蕃。出登來山,使親近將跳蕃首,作虎跳狼爭咋齧之,頭皆碎壞,欲以示威,使衆不敢犯也。此與本傳不同。吳錄曰:皓每於會,因酒酣,輒令侍臣嘲謔公卿,以為笑樂。萬彧旣為左丞相,蕃嘲彧曰:「魚潛於淵,出水煦沫。何則?物有本性,不可橫處非分也。彧出自谿谷,羊質虎皮,虛受光赫之寵,跨越三九之位,犬馬猶能識養,將何以報厚施乎!」彧曰:「唐虞之朝無謬舉之才,造父之門無駑蹇之質,蕃上誣明選,下訕楨幹,何傷於日月,適多見其不知量耳。」臣松之按本傳云丁忠使晉還,皓為大會,於會中殺蕃,檢忠從北還在此年之春,彧時尚未為丞相,至秋乃為相耳。吳錄所言為乖互不同。}
 
 
 
 
 丞相陸凱上疏曰:「常侍王蕃黃中通理,知天知物,處朝忠蹇,斯社稷之重鎮,大吳之龍逢也。昔事景皇,納言左右,景皇欽嘉,歎為異倫。而陛下忿其苦辭,惡其直對,梟之殿堂,尸骸暴棄,邦內傷心,有識悲悼。」其痛蕃如此。蕃死時年三十九,皓徙蕃家屬廣州。二弟著、延皆作佳器,郭馬起事,不為馬用,見害。
 
 
\end{pinyinscope}