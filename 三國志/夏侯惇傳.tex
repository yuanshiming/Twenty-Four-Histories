\article{夏侯惇傳}
\begin{pinyinscope}
 
 
 夏侯惇字元讓,沛國譙人,夏侯嬰之後也。年十四,就師學,人有辱其師者,惇殺之,由是以烈氣聞。太祖初起,惇常為裨將,從征伐。太祖行奮武將軍,以惇為司馬,別屯白馬,遷折衝校尉,領東郡太守。太祖征陶謙,留惇守濮陽。張邈叛迎呂布,太祖家在鄄城,惇輕軍往赴,適與布會,交戰。布退還,遂入濮陽,襲得惇軍輜重。遣將偽降,共執持惇,責以寶貨,惇軍中震恐。惇將韓浩乃勒兵屯惇營門,召軍吏諸將,皆案甲當部不得動,諸營乃定。遂詣惇所,叱持質者曰:「汝等凶逆,乃敢執劫大將軍,復欲望生邪!且吾受命討賊,寧能以一將軍之故,而縱汝乎?」因涕泣謂惇曰:「當柰國法何!」促召兵擊持質者。持質者惶遽叩頭,言「我但欲乞資用去耳」!浩數責,皆斬之。惇旣免,太祖聞之,謂浩曰:「卿此可為萬世法。」乃著令,自今已後有持質者,皆當并擊,勿顧質。由是劫質者遂絕。
 
 
\gezhu{孫盛曰:案光武紀,建武九年,盜劫陰貴人母弟,吏以不得拘質迫盜,盜遂殺之也。然則合擊者,乃古制也。自安、順已降,政教陵遲,劫質不避王公,而有司莫能遵奉國憲者,浩始復斬之,故魏武嘉焉。}
 
 
太祖自徐州還,惇從征呂布,為流矢所中,傷左目。
 \gezhu{魏略曰:時夏侯淵與惇俱為將軍,軍中號惇為盲夏侯。惇惡之,每照鏡,恚怒,輒撲鏡於地。}
 復領陳留、濟陰太守,加建武將軍,封高安鄉侯。時大旱,蝗蟲起,惇乃斷太壽水作陂,身自負土,率將士勸種稻,民賴其利。轉領河南尹。太祖平河北,為大將軍後拒。鄴破,遷伏波將軍,領尹如故,使得以便宜從事,不拘科制。建安十二年,錄惇前後功,增封邑千八百戶,并前二千五百戶。二十一年,從征孫權還,使惇都督二十六軍,留居巢。賜伎樂名倡,令曰:「魏絳以和戎之功,猶受金石之樂,況將軍乎!」二十四年,太祖軍於摩陂,召惇常與同載,特見親重,出入卧內,諸將莫得比也。拜前將軍,
 \gezhu{魏書曰:時諸將皆受魏官號,惇獨漢官,乃上疏自陳不當不臣之禮。太祖曰:「吾聞太上師臣,其次友臣。夫臣者,貴德之人也,區區之魏,而臣足以屈君乎?」惇固請,乃拜為前將軍。}
 督諸軍還壽春,徙屯召陵。文帝即王位,拜惇大將軍,數月薨。
 
 
惇雖在軍旅,親迎師受業。性清儉,有餘財輒以分施,不足資之於官,不治產業。謚曰忠侯。子充嗣。帝追思惇功,欲使子孫畢侯,分惇邑千戶,賜惇七子二孫爵皆關內侯。惇弟廉及子楙素自封列侯。初,太祖以女妻楙,即清河公主也。楙歷位侍中尚書、安西鎮東將軍,假節。
 \gezhu{魏略曰:楙字子林,惇中子也。文帝少與楙親,及即位,以為安西將軍、持節,承夏侯淵處都督關中。楙性無武略,而好治生。至太和二年,明帝西征,人有白楙者,遂召還為尚書。楙在西時,多畜伎妾,公主由此與楙不和。其後羣弟不遵禮度,楙數切責,弟懼見治,乃共構楙以誹謗,令主奏之,有詔收楙。帝意欲殺之,以問長水校尉京兆段默,默以為「此必清河公主與楙不睦,出於譖構,兾不推實耳。且伏波與先帝有定天下之功,宜加三思」。帝意解,曰:「吾亦以為然。」乃發詔推問為公主作表者,果其羣弟子臧、子江所構也。}
 充薨,子廙嗣。廙薨,子劭嗣。
 \gezhu{晉陽秋曰:泰始二年,高安鄉侯夏侯佐卒,惇之孫也,嗣絕。詔曰:「惇,魏之元功,勳書竹帛。昔庭堅不祀,猶或悼之,況朕受禪於魏,而可以忘其功臣哉!宜擇惇近屬紹封之。」}
 
 
韓浩者,河內人,及沛國史渙與浩,俱以忠勇顯。浩至中護軍,渙至中領軍,皆掌禁兵,封列侯。
 \gezhu{魏書曰:韓浩字元嗣。漢末起兵,縣近山藪,多寇,浩聚徒衆為縣藩衞。太守王匡以為從事,將兵拒董卓於盟津。時浩舅杜陽為河陰令,卓執之,使招浩,浩不從。袁術聞而壯之,以為騎都尉。夏侯惇聞其名,請與相見,大奇之,使領兵從征伐。時大議損益,浩以為當急田。太祖善之,遷護軍。太祖欲討柳城,領軍史渙以為道遠深入,非完計也,欲與浩共諫。浩曰:「今兵勢彊盛,威加四海,戰勝攻取,無不如志,不以此時遂除天下之患,將為後憂。且公神武,舉無遺策,吾與君為中軍主,不宜沮衆。」遂從破柳城,改其官為中護軍,置長史、司馬。從討張魯,魯降。議者以浩智略足以綏邊,欲留使都督諸軍,鎮漢中。太祖曰:「吾安可以無護軍?」乃與俱還。其見親任如此。及薨,太祖愍惜之。無子,以養子榮嗣。史渙字公劉。少任俠,有雄氣。太祖初起,以客從,行中軍校尉,從征伐,常監諸將,見親信,轉拜中領軍。十四年薨。子靜嗣。}
 
 
\end{pinyinscope}