\article{傅嘏傳}
\begin{pinyinscope}
 
 
 傅嘏字蘭石,北地泥陽人,傅介子之後也。伯父巽,黃初中為侍中尚書。
 
 
\gezhu{傅子曰:嘏祖父睿,代郡太守。父充,黃門侍郎。}
 嘏弱冠知名,
 \gezhu{傅子曰:是時何晏以材辯顯於貴戚之間,鄧颺好變通,合徒黨,鬻聲名於閭閻,而夏侯玄以貴臣子少有重名,為之宗主,求交於嘏而不納也。嘏友人荀粲,有清識遠心,然猶怪之。謂嘏曰:「夏侯泰初一時之傑,虛心交子,合則好成,不合則怨至。二賢不睦,非國之利,此藺相如所以下廉頗也。」嘏荅之曰:「泰初志大其量,能合虛聲而無實才。何平叔言遠而情近,好辯而無誠,所謂利口覆邦國之人也。鄧玄茂有為而無終,外要名利,內無關鑰,貴同惡異,多言而妬前;多言多釁,妬前無親。以吾觀此三人,皆敗德也。遠之猶恐禍及,況昵之乎?」}
 司空陳羣辟為掾。時散騎常侍劉劭作考課法,事下三府。嘏難劭論曰:「蓋聞帝制宏深,聖道奧遠,苟非其才,則道不虛行,神而明之,存乎其人。曁乎王略虧頹而曠載罔綴,微言旣沒,六籍泯玷。何則?道弘致遠而衆才莫晞也。案劭考課論,雖欲尋前代黜陟之文,然其制度略以闕亡。禮之存者,惟有周典,外建侯伯,藩屏九服,內立列司,筦齊六職,土有恒貢,官有定則,百揆均在,四民殊業,故考績可理而黜陟易通也。大魏繼百王之末,承秦、漢之烈,制度之流,靡所脩采。自建安以來,至于青龍,神武撥亂,肇基皇祚,掃除凶逆,芟夷遺寇,旌旗卷舒,日不暇給。及經邦治戎,權法並用,百官羣司,軍國通任,隨時之宜,以應政機。以古施今,事雜義殊,難得而通也。所以然者,制宜經遠,或不切近,法應時務,不足垂後。夫建官均職,清理民物,所以立本也;循名考實,糾勵成規,所以治末也。本綱末舉而造制未呈,國略不崇而考課是先,懼不足以料賢愚之分,精幽明之理也。昔先王之擇才,必本行於州閭,講道於庠序,行具而謂之賢,道脩則謂之能。鄉老獻賢能于王,王拜受之,舉其賢者,出使長之,科其能者,入使治之,此先王收才之義也。方今九州之民,爰及京城,未有六鄉之舉,其選才之職,專任吏部。案品狀則實才未必當,任薄伐則德行未為叙,如此則殿最之課,未盡人才。述綜王度,敷贊國式,體深義廣,難得而詳也。」
 
 
正始初,除尚書郎,遷黃門侍郎。時曹爽秉政,何晏為吏部尚書,嘏謂爽弟羲曰:「何平叔外靜而內銛巧,好利,不念務本。吾恐必先惑子兄弟,仁人將遠,而朝政廢矣。」晏等遂與嘏不平,因微事以免嘏官。起家拜熒陽太守,不行。太傅司馬宣王請為從事中郎。曹爽誅,為河南尹,
 \gezhu{傅子曰:河南尹內掌帝都,外統京畿,兼古六鄉六遂之士。其民異方雜居,多豪門大族,商賈胡貊,天下四方會利之所聚,而姦之所生。前尹司馬芝,舉其綱而太簡,次尹劉靜,綜其目而太密,後尹李勝,毀常法以收一時之聲。嘏立司馬氏之綱統,裁劉氏之綱目以經緯之,李氏所毀以漸補之。郡有七百吏,半非舊也。河南俗黨五官掾功曹典選職,皆授其本國人,無用異邦人者,嘏各舉其良而對用之,官曹分職,而後以次考核之。其治以德教為本,然持法有恒,簡而不可犯,見理識情,獄訟不加檟楚而得其實。不為小惠,有所薦達及大有益於民事,皆隱其端迹,若不由己出。故當時無赫赫之名,吏民乆而後安之。}
 遷尚書。嘏常以為「秦始罷侯置守,設官分職,不與古同。漢、魏因循,以至于今。然儒生學士,咸欲錯綜以三代之禮,禮弘致遠,不應時務,事與制違,名實未附,故歷代而不至於治者,蓋由是也。欲大改定官制,依古正本,今遇帝室多難,未能革易」。
 
 
時論者議欲自伐吳,三征獻策各不同。詔以訪嘏,嘏對曰:「昔夫差陵齊勝晉,威行中國,終禍姑蘇;齊閔兼土拓境,闢地千里,身蹈顛覆。有始不必善終,古之明效也。孫權自破關羽并荊州之後,志盈欲滿,凶宄以極,是以宣文侯深建宏圖大舉之策。今權以死,託孤於諸葛恪。若矯權苛暴,蠲其虐政,民免酷烈,偷安新惠,外內齊慮,有同舟之懼,雖不能終自保完,猶足以延期挺命於深江之外矣。而議者或欲汎舟徑濟,橫行江表;或欲四道並進,攻其城壘;或欲大佃疆埸,觀釁而動:誠皆取賊之常計也。然自治兵以來,出入三載,非掩襲之軍也。賊之為寇,幾六十年矣,君臣偽立,吉凶共患,又喪其元帥,上下憂危,設令列船津要,堅城據險,橫行之計,其殆難捷。惟進軍大佃,最差完牢。兵出民表,寇鈔不犯;坐食積穀,不煩運士;乘釁討襲,無遠勞費:此軍之急務也。昔樊噲願以十萬之衆,橫行匈奴,季布面折其短。今欲越長江,涉虜庭,亦向時之喻也。未若明法練士,錯計於全勝之地,振長策以禦敵之餘燼,斯必然之數也。」
 \gezhu{司馬彪戰略載嘏此對,詳於本傳,今悉載之以盡其意。彪曰:嘉平四年四月,孫權死。征南大將軍王昶、征東將軍胡遵、鎮南將軍毌丘儉等表請征吳。朝廷以三征計異,詔訪尚書傅嘏,嘏對曰:「昔夫差勝齊陵晉,威行中國,不能以免姑蘇之禍;齊閔辟土兼國,開地千里,不足以救顛覆之敗:有始不必善終,古事之明效也。孫權自破蜀兼平荊州之後,志盈欲滿,罪戮忠良,殊及胤嗣,元凶已極。相國宣文侯先識取亂侮亡之義,深建宏圖大舉之策。今權已死,託孤於諸葛恪。若矯權苛暴,蠲其虐政,民免酷烈,偷安新惠,外內齊慮,有同舟之懼,雖不能終自保完,猶足以延期挺命於深江之表矣。昶等或欲汎舟徑渡,橫行江表,收民略地,因糧於寇;或欲四道並進,臨之以武,誘間攜貳,待其崩壞;或欲進軍大佃,偪其項領,積穀觀釁,相時而動:凡此三者,皆取賊之常計也。然施之當機,則功成名立,苟不應節,必貽後患。自治兵已來,出入三載,非掩襲之軍也。賊喪元帥,利存退守,若撰飾舟楫,羅船津要,堅城清野,以防卒攻,橫行之計,殆難必施。賊之為寇,幾六十年,君臣偽立,吉凶同患,若恪蠲其弊,天去其疾,崩潰之應,不可卒待。今邊壤之守,與賊相遠,賊設羅落,又持重密,間諜不行,耳目無聞。夫軍無耳目,校察未詳,而舉大衆以臨巨險,此為希幸徼功,先戰而後求勝,非全軍之長策也。唯有進軍大佃,最差完牢。可詔昶、遵等擇地居險,審所錯置,及令三方一時前守。奪其肥壤,使還耕塉土,一也;兵出民表,寇鈔不犯,二也;招懷近路,降附日至,三也;羅落遠設,間構不來,四也;賊退其守,羅落必淺,佃作易之,五也;坐食積穀,士不運輸,六也;釁隙時聞,討襲速決,七也:凡此七者,軍事之急務也。不據則賊擅便資,據之則利歸於國,不可不察也。夫屯壘相偪,形勢已交,智勇得陳,巧拙得用,策之而知得失之計,角之而知有餘不足,虜之情偽,將焉所逃?夫以小敵大,則役煩力竭,以貧敵富,則斂重財匱。故『敵逸能勞之,飽能飢之』,此之謂也。然後盛衆厲兵以震之,參惠倍賞以招之,多方廣似以疑之。由不虞之道,以間其不戒;比及三年,左提右挈,虜必冰散瓦解,安受其弊,可坐筭而得也。昔漢氏歷世常患匈奴,朝臣謀士早朝晏罷,介冑之將則陳征伐,搢紳之徒咸言和親,勇奮之士思展搏噬。故樊噲願以十萬之衆橫行匈奴,季布面折其短。李信求以二十萬獨舉楚人,而果辱秦軍。今諸將有陳越江陵險,獨步虜庭,即亦向時之類也。以陛下聖德,輔相忠賢,法明士練,錯計於全勝之地,振長策以禦之,虜之崩潰,必然之數。故兵法曰:『屈人之兵,而非戰也;拔人之城,而非攻也。』若釋廟勝必然之理,而行萬一不必全之路,誠愚臣之所慮也。故謂大佃而偪之計最長。」時不從嘏言。其年十一月,詔昶等征吳。五年正月,諸葛恪拒戰,大破衆軍於東關。}
 後吳大將諸葛恪新破東關,乘勝揚聲欲向青、徐,朝廷將為之備。嘏議以為「淮海非賊輕行之路,又昔孫權遣兵入海,漂浪沈溺,略無孑遺,恪豈敢傾根竭本,寄命洪流,以徼乾沒乎?
 \gezhu{漢書張湯傳曰:湯始為小吏,乾沒,與長安富賈田甲、魚翁叔之屬交私。服虔說曰:「乾沒,射成敗也。」如淳曰:「得利為乾,失利為沒。」臣松之以虔直以乾沒為射成敗,而不說乾沒之義,於理猶為未暢。淳以得利為乾,又不可了。愚謂乾讀宜為干燥之干。蓋謂有所徼射,不計干燥之與沈沒而為之。}
 恪不過遣偏率小將素習水軍者,乘海泝淮,示動青、徐,恪自并兵來向淮南耳」。後恪果圖新城,不克而歸。
 
 
嘏常論才性同異,鍾會集而論之。
 \gezhu{傅子曰:嘏旣達治好正,而有清理識要,好論才性,原本精微,尠能及之。司隷校尉鍾會年甚少,嘏以明智交會。臣松之案:傅子前云嘏了夏侯之必敗,不與之交,而此云與鍾會善。愚以為夏侯玄以名重致患,釁由外至;鍾會以利動取敗,禍自己出。然則夏侯之危兆難覩,而鍾氏之敗形易照也。嘏若了夏侯之必危,而不見鍾會之將敗,則為識有所蔽,難以言通;若皆知其不終,而情有彼此,是為厚薄由于愛憎,奚豫於成敗哉?以愛憎為厚薄,又虧於雅體矣。傅子此論,非所以益嘏也。}
 嘉平末,賜爵關內侯。高貴鄉公即尊位,進封武鄉亭侯。正元二年春,毌丘儉、文欽作亂。或以司馬景王不宜自行,可遣太尉孚往,惟嘏及王肅勸之。景王遂行。
 \gezhu{漢晉春秋曰:嘏固勸景王行,景王未從。嘏重言曰:「淮、楚兵勁,而儉等負力遠鬬,其鋒未易當也。若諸將戰有利鈍,大勢一失,則公事敗矣。」是時景王新割目瘤,創甚,聞嘏言,蹶然而起曰:「我請輿疾而東。」}
 以嘏守尚書僕射,俱東。儉、欽破敗,嘏有謀焉。及景王薨,嘏與司馬文王徑還洛陽,文王遂以輔政。語在鍾會傳。
 \gezhu{世語曰:景王疾甚,以朝政授傅嘏,嘏不敢受。及薨,嘏祕不發喪,以景王命召文王於許昌,領公軍焉。孫盛評曰:晉宣、景、文王之相魏也,權重相承,王業基矣。豈蕞爾傅嘏所宜間厠?世語所云,斯不然矣。}
 會由是有自矜色,嘏戒之曰:「子志大其量,而勳業難為也,可不慎哉!」嘏以功進封陽鄉侯,增邑六百戶,并前千二百戶。是歲薨,時年四十七,追贈太常,謚曰元侯。
 \gezhu{傅子曰:初,李豐與嘏同州,少有顯名,早歷大官,內外稱之,嘏又不善也。謂同志曰:「豐飾偽而多疑,矜小失而昧於權利,若處庸庸者可也,自任機事,遭明者必死。」豐後為中書令,與夏侯玄俱禍,卒如嘏言。嘏自少與兾州刺史裴徽、散騎常侍荀甝善,徽、甝早亡。又與鎮北將軍何曾、司空陳泰、尚書僕射荀顗、後將軍鍾毓並善相友綜朝事,俱為名臣。}
 子祗嗣。咸熈中開建五等,以嘏著勳前朝,改封祗涇原子。
 \gezhu{晉諸公贊曰:祗字子莊,嘏少子也。晉永嘉中至司空。祗子宣,字世弘。世語稱宣以公正知名,位至御史中丞。宣弟暢,字世道,祕書丞,沒在胡中。著晉諸公贊及晉公卿禮秩故事。}
 
 
評曰:昔文帝、陳王以公子之尊,博好文采,同聲相應,才士並出,惟粲等六人最見名目。而粲特處常伯之官,興一代之制,然其沖虛德宇,未若徐幹之粹也。衞覬亦以多識典故,相時王之式。劉劭該覽學籍,文質周洽。劉廙以清鑒著,傅嘏用才達顯云。
 \gezhu{臣松之以為傅嘏識量名輩,寔當時高流。而此評但云「用才達顯」,旣於題目為拙,又不足以見嘏之美也。}
 
 
\end{pinyinscope}