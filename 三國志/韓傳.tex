\article{韓傳}
\begin{pinyinscope}
 
 
 韓在帶方之南,東西以海為限,南與倭接,方可四千里。有三種,一曰馬韓,二曰辰韓,三曰弁韓。辰韓者,古之辰國也。馬韓在西。其民土著,種植,知蠶桑,作緜布。各有長帥,大者自名為臣智,其次為邑借,散在山海間,無城郭。有爰襄國、牟水國、桑外國、小石索國、大石索國、優休牟涿國、臣濆沽國、伯濟國、速盧不斯國、日華國、古誕者國、古離國、怒藍國、月支國、咨離牟盧國、素謂乾國、古爰國、莫盧國、卑離國、占離卑國、臣釁國、支侵國、狗盧國、卑彌國、監奚卑離國、古蒲國、致利鞠國、冉路國、兒林國、駟盧國、內卑離國、感奚國、萬盧國、辟卑離國、臼斯烏旦國、一離國、不彌國、支半國、狗素國、捷盧國、牟盧卑離國、臣蘇塗國、莫盧國、古臘國、臨素半國、臣雲新國、如來卑離國、楚山塗卑離國、一難國、狗奚國、不雲國、不斯濆邪國、爰池國、乾馬國、楚離國,凡五十餘國。大國萬餘家,小國數千家,總十餘萬戶。辰王治月支國。臣智或加優呼臣雲遣支報安邪踧支濆臣離兒不例拘邪秦支廉之號。其官有魏率善、邑君、歸義侯、中郎將、都尉、伯長。
 
 
 
 
 侯準旣僭號稱王,為燕亡人衞滿所攻奪,
 
 
\gezhu{魏略曰:昔箕子之後朝鮮侯,見周衰,燕自尊為王,欲東略地,朝鮮侯亦自稱為王,欲興兵逆擊燕以尊周室。其大夫禮諫之,乃止。使禮西說燕,燕止之,不攻。後子孫稍驕虐,燕乃遣將秦開攻其西方,取地二千餘里,至滿潘汗為界,朝鮮遂弱。及秦并天下,使蒙恬築長城,到遼東。時朝鮮王否立,畏秦襲之,略服屬秦,不肯朝會。否死,其子準立。二十餘年而陳、項起,天下亂,燕、齊、趙民愁苦,稍稍亡往準,準乃置之於西方。及漢以盧綰為燕王,朝鮮與燕界於浿水。及綰反,入匈奴,燕人衞滿亡命,為胡服,東度浿水,詣準降,說準求居西界,收中國亡命為朝鮮藩屏。準信寵之,拜為博士,賜以圭,封之百里,令守西邊。滿誘亡黨,衆稍多,乃詐遣人告準,言漢兵十道至,求入宿衞,遂還攻準。準與滿戰,不敵也。}
 將其左右宮人走入海,居韓地,自號韓王。
 \gezhu{魏略曰:其子及親留在國者,因冒姓韓氏。準王海中,不與朝鮮相往來。}
 其後絕滅,今韓人猶有奉其祭祀者。漢時屬樂浪郡,四時朝謁。
 \gezhu{魏略曰:初,右渠未破時,朝鮮相歷谿卿以諫右渠不用,東之辰國,時民隨出居者二千餘戶,亦與朝鮮貢蕃不相往來。至王莽地皇時,廉斯鑡為辰韓右渠帥,聞樂浪土地美,人民饒樂,亡欲來降。出其邑落,見田中驅雀男子一人,其語非韓人。問之,男子曰:「我等漢人,名戶來,我等輩千五百人伐材木,為韓所擊得,皆斷髮為奴,積三年矣。」鑡曰:「我當降漢樂浪,汝欲去不?」戶來曰:「可。」鑡因將戶來,來出詣含資縣,縣言郡,郡即以鑡為譯,從芩中乘大船入辰韓,逆取戶來。降伴輩尚得千人,其五百人已死。鑡時曉謂辰韓:「汝還五百人。若不者,樂浪當遣萬兵乘船來擊汝。」辰韓曰:「五百人已死,我當出贖直耳。」乃出辰韓萬五千人,弁韓布萬五千匹,鑡收取直還。郡表鑡功義,賜冠幘、田宅,子孫數世,至安帝延光四年時,故受復除。}
 
 
 
 
 桓、靈之末,韓濊彊盛,郡縣不能制,民多流入韓國。建安中,公孫康分屯有縣以南荒地為帶方郡,遣公孫模、張敞等收集遺民,興兵伐韓濊,舊民稍出,是後倭韓遂屬帶方。景初中,明帝密遣帶方太守劉昕、樂浪太守鮮于嗣越海定二郡,諸韓國臣智加賜邑君印綬,其次與邑長。其俗好衣幘,下戶詣郡朝謁,皆假衣幘,自服印綬衣幘千有餘人。部從事吳林以樂浪本統韓國,分割辰韓八國以與樂浪,吏譯轉有異同,臣智激韓忿,攻帶方郡崎離營。時太守弓遵、樂浪太守劉茂興兵伐之,遵戰死,二郡遂滅韓。
 
 
 
 
 其俗少綱紀,國邑雖有主帥,邑落雜居,不能善相制御。無跪拜之禮。居處作草屋土室,形如冢,其戶在上,舉家共在中,無長幼男女之別。其葬有棺無槨,不知乘牛馬,牛馬盡於送死。以瓔珠為財寶,或以綴衣為飾,或以縣頸垂耳,不以金銀錦繡為珍。其人性彊勇,魁頭露紒,如炅兵,衣布袍,足履革蹻蹋。其國中有所為及官家使築城郭,諸年少勇健者,皆鑿脊皮,以大繩貫之,又以丈許木鍤之,通日嚾呼作力,不以為痛,旣以勸作,且以為健。常以五月下種訖,祭鬼神,群聚歌舞,飲酒晝夜無休。其舞,數十人俱起相隨,踏地低昂,手足相應,節奏有似鐸舞。十月農功畢,亦復如之。信鬼神,國邑各立一人主祭天神,名之天君。又諸國各有別邑。名之為蘇塗。立大木,縣鈴鼓,事鬼神。諸亡逃至其中,皆不還之,好作賊。其立蘇塗之義,有似浮屠,而所行善惡有異。其北方近郡諸國差曉禮俗,其遠處直如囚徒奴婢相聚。無他珍寶。禽獸草木略與中國同。出大栗,大如梨。又出細尾雞,其尾皆長五尺餘。其男子時時有文身。又有州胡在馬韓之西海中大島上,其人差短小,言語不與韓同,皆髠頭如鮮卑,但衣韋,好養牛及豬。其衣有上無下,略如裸勢。乘船往來,巿買韓中。
 
 
\end{pinyinscope}