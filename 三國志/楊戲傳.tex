\article{楊戲傳}
\begin{pinyinscope}
 
 
 楊戲字文然,犍為武陽人也。少與巴西程祁公弘、巴郡楊汰季儒、蜀郡張表伯達並知名。戲每推祁以為冠首,丞相亮深識之。戲年二十餘,從州書佐為督軍從事,職典刑獄,論法決疑,號為平當,府辟為屬主簿。亮卒,為尚書右選部郎,刺史蔣琬請為治中從事史。琬以大將軍開府,又辟為東曹掾,遷南中郎參軍,副貳庲降都督,領建寧太守。以疾徵還成都,拜護軍監軍,出領梓潼太守,入為射聲校尉,所在清約不煩。延熈二十年,隨大將軍姜維出軍至芒水。戲素心不服維,酒後言笑,每有慠弄之辭。維外寬內忌,意不能堪,軍還,有司承旨奏戲,免為庶人。後景耀四年卒。
 
 
 
 
 戲性雖簡惰省略,未甞以甘言加人,過情接物。書符指事,希有盈紙。然篤於舊故,居誠存厚。與巴西韓儼、黎韜童幼相親厚,後儼痼疾廢頓,韜無行見捐,戲經紀振卹,恩好如初。又時人謂譙周無當世才,少歸敬者,唯戲重之,常稱曰:「吾等後世,終自不如此長兒也。」有識以此貴戲。
 
 
 
 
 張表有威儀風觀,始名位與戲齊,後至尚書,督庲降後將軍,先戲沒。祁、汰各早死。
 
 
\gezhu{戲同縣後進有李密者,字令伯。華陽國志曰:密祖父光,朱提太守。父早亡。母何氏,更適人。密見養於祖母。治春秋左氏傳,博覽多所通涉,機警辯捷。事祖母以孝聞,其侍疾則泣涕側息,日夜不解帶,膳飲湯藥,必自口甞。本郡禮命不應,州辟從事尚書郎,大將軍主簿,太子洗馬,奉使聘吳。吳主問蜀馬多少,對曰:「官用有餘,人間自足。」吳主與羣臣汎論道義,謂寧為人弟,密曰:「願為人兄矣。」吳主曰:「何以為兄?」密曰:「為兄供養之日長。」吳主及羣臣皆稱善。蜀平後,征西將軍鄧艾聞其名,請為主簿,及書招,欲與相見,皆不往。以祖母年老,心在色養。晉武帝立太子,徵為太子洗馬,詔書累下,郡縣偪遣,於是密上書曰:「臣以險舋,夙遭閔凶,生孩六月,慈父見背,行年四歲,舅奪母志。祖母劉,愍臣孤弱,躬見撫養。臣少多疾病,九歲不行,零丁孤苦,至於成立,旣無伯叔,終鮮兄弟,門衰祚薄,晚有兒息。外無朞功彊近之親,內無應門五尺之童,煢煢孑立,形影相弔。而劉早嬰疾病,常在牀蓐,臣侍湯藥,未曾廢離。逮奉聖朝,沐浴清化,前太守臣逵察臣孝廉,後刺史臣榮舉臣秀才,臣以供養無主,辭不赴命。詔書特下,拜臣郎中,尋蒙國恩,除臣洗馬,猥以微賤,當侍東宮,非臣隕首所能上報。臣具表聞,辭不就職。詔書切峻,責臣逋慢,郡縣偪迫,催臣上道,州司臨門,急於星火。臣欲奉詔奔馳,則劉病日篤,苟順私情,則告訴不許,臣之進退,實為狼狽。伏惟聖朝以孝治天下,凡在故老,猶蒙矜愍,況臣孤苦,特為尤甚。且臣少仕偽朝,歷職郎署,本圖宦達,不矜名節。今臣亡國賤俘,至微至陋,猥蒙拔擢,寵命優渥,豈敢盤桓,有所希兾?但以劉日薄西山,氣息奄奄,人命危淺,朝不慮夕。臣無祖母,無以至今日,祖母無臣,亦無以終餘年,母孫二人,更相為命,是以區區不敢廢遠。臣今年四十有四,祖母劉今年九十有六,是臣盡節於陛下之日長,報養劉之日短也。烏鳥私情,願乞終養。臣之辛苦,非徒蜀之人士及二州牧伯所見明知,皇天后土,實所共鑒。願陛下矜愍愚誠,聽臣微志,庶劉僥倖,保卒餘年。臣生當隕首,死當結草,臣不勝犬馬怖懼之情!」武帝覽表曰:「密不空有名也。」嘉其誠欵,賜奴婢二人,下郡縣供養其祖母奉膳。及祖母卒,服終,從尚書郎為河內溫縣令,政化嚴明。中山諸王每過溫縣,必責求供給,溫吏民患之。及密至,中山王過縣,欲求芻茭薪蒸,密牋引高祖過沛,賔禮老幼,桑梓之供,一無煩擾,「伏惟明王孝思惟則,動識先戒,本國望風,式歌且舞,誅求之碎,所未聞命。」自後諸王過,不敢有煩。隴西王司馬子舒深敬友密,而貴勢之家憚其公直。密去官,為州大中正,性方直,不曲意勢位。後失荀勗、張華指,左遷漢中太守,諸王多以為冤。一年去官,年六十四卒。著述理論十篇,安東將軍胡熊與皇甫士安並善之。}
 
 
 
 
 戲以延熈四年著季漢輔臣贊,其所頌述,今多載于蜀書,是以記之於左。自此之後卒者,則不追謚,故或有應見稱紀而不在乎篇者也。其戲之所贊而今不作傳者,余皆注疏本末於其辭下,可以觕知其髣髴云爾。
 
 
 
 
 昔文王歌德,武王歌興,夫命世之主,樹身行道,非唯一時,亦由開基植緒,光于來世者也。自我中漢之末,王綱棄柄,雄豪並起,役殷難結,生人塗地。於是世主感而慮之,初自燕、代則仁聲洽著,行自齊、魯則英風播流,寄業荊、郢則臣主歸心,顧援吳、越則賢愚賴風,奮威巴、蜀則萬里肅震,厲師庸、漢則元寇歛迹,故能承高祖之始兆,復皇漢之宗祀也。然而姦凶懟險,天征未加,猶孟津之翔師,復須戰於鳴條也。天祿有終,奄忽不豫。雖攝歸一統,萬國合從者,當時儁乂扶攜翼戴,明德之所懷致也,蓋濟濟有可觀焉。遂乃並述休風,動于後聽。其辭曰:
 
 
 
 
 皇帝遺植,爰滋八方,別自中山,靈精是鍾,順期挺生,傑起龍驤。始于燕、代,伯豫君荊,吳、越憑賴,望風請盟,挾巴跨蜀,庸漢以并。乾坤復秩,宗祀惟寧,躡基履迹,播德芳聲。華夏思美,西伯其音,開慶來世,歷載攸興。--贊昭烈皇帝
 
 
 
 
 忠武英高,獻策江濵,攀吳連蜀,權我世真。受遺阿衡,整武齊文,敷陳德教,理物移風,賢愚競心,僉忘其身。誕靜邦內,四裔以綏,屢臨敵庭,實耀其威,研精大國,恨於未夷。--贊諸葛丞相
 
 
 
 
 司徒清風,是咨是臧,識愛人倫,孔音鏘鏘。--贊許司徒
 
 
 
 
 關、張赳赳,出身匡世,扶翼攜上,雄壯虎烈。藩屏左右,飜飛電發,濟于艱難,贊主洪業,侔迹韓、耿,齊聲雙德。交待無禮,並致姦慝,悼惟輕慮,隕身匡國。--贊關雲長、張益德
 
 
 
 
 驃騎奮起,連橫合從,首事三秦,保據河、潼。宗計於朝,或異或同,敵以乘舋,家破軍亡。乖道反德,託鳳攀龍。--贊馬孟起
 
 
 
 
 翼侯良謀,料世興衰,委質于主,是訓是諮,暫思經筭,覩事知機。--贊法孝直
 
 
 
 
 軍師美至,雅氣曄曄,致命明主,忠情發臆,惟此義宗,亡身報德。--贊龐士元
 
 
 
 
 將軍敦壯,摧峰登難,立功立事,于時之幹。--贊黃漢升
 
 
 
 
 掌軍清節,亢然恒常,讜言惟司,民思其綱。--贊董幼宰
 
 
安遠彊志,允休允烈,輕財果壯,當難不惑,以少禦多,殊方保業。--贊鄧孔山
 \gezhu{孔山名方,南郡人也。以荊州從事隨先主入蜀。蜀旣定,為犍為屬國都尉,因易郡名,為朱提太守,遷為安遠將軍、庲降都督,住南昌縣。章武二年卒。失其行事,故不為傳。}
 
 
揚威才幹,欷歔文武,當官理任,衎衎辯舉,圖殖財施,有義有叙。--贊費賔伯
 \gezhu{賔伯名觀,江夏鄳人也。劉璋母,觀之族姑,璋又以女妻觀。觀建安十八年參李嚴軍,拒先主於緜竹,與嚴俱降,先主旣定益州,拜為裨將軍,後為巴郡太守、江州都督,建興元年封都亭侯,加振威將軍。觀為人善於交接。都護李嚴性自矜高,護軍輔匡等年位與嚴相次,而嚴不與親褻;觀年少嚴二十餘歲,而與嚴通狎如時輩云。年三十七卒。失其行事,故不為傳。}
 
 
 
 
 屯騎主舊,固節不移,旣就初命,盡心世規,軍資所恃,是辨是裨。--贊王文儀
 
 
 
 
 尚書清尚,勑行整身,抗志存義,味覽典文,倚其高風,好侔古人。--贊劉子初
 
 
 
 
 安漢雍容,或婚或賔,見禮當時,是謂循臣。--贊麋子仲
 
 
少府修慎,
 \gezhu{王元泰名謀,漢嘉人也。有容止操行。劉璋時,為巴郡太守,還為州治中從事。先主定益州,領牧,以為別駕。先主為漢中王,用荊楚宿士零陵賴恭為太常,南陽黃柱為光祿勳,謀為少府;建興初,賜爵關內侯,後代賴恭為太常。恭、柱、謀皆失其行事,故不為傳。恭子厷,為丞相西曹令史,隨諸葛亮於漢中,早夭,亮甚惜之,與留府長史參軍張裔、蔣琬書曰:「令史失賴厷,掾屬喪楊顒,為朝中損益多矣。」顒亦荊州人也。後大將軍蔣琬問張休曰:「漢嘉前輩有王元泰,今誰繼者?」休對曰:「至於元泰,州里無繼,況鄙都乎!」其見重如此。襄陽記曰:楊顒字子昭,楊儀宗人也。入蜀,為巴郡太守,丞相諸葛亮主簿。亮甞自校簿書,顒直入諫曰:「為治有體,上下不可相侵,請為明公以作家譬之。今有人使奴執耕稼,婢典炊爨,雞主司晨,犬主吠盜,牛負重載,馬涉遠路,私業無曠,所求皆足,雍容高枕,飲食而已,忽一旦盡欲以身親其役,不復付任,勞其體力,為此碎務,形疲神困,終無一成。豈其智之不如奴婢雞狗哉?失為家主之法也。是故古人稱坐而論道謂之王公,作而行之謂之士大夫。故邴吉不問橫道死人而憂牛喘,陳平不肯知錢穀之數,云自有主者,彼誠達於位分之體也。今明公為治,乃躬自校簿書,流汗竟日,不亦勞乎!」亮謝之。後為東曹屬典選舉。顒死,亮垂泣三日。}
 鴻臚明真,
 \gezhu{何彥英名宗,蜀郡郫人也。事廣漢任安學,精究安術,與杜瓊同師而名問過之。劉璋時,為犍為太守。先主定益州,領牧,辟為從事祭酒。後援引圖、讖,勸先主即尊號。踐阼之後,遷為大鴻臚。建興中卒。失其行事,故不為傳。子雙,字漢偶。滑稽談笑,有淳于髠、東方朔之風。為雙柏長。早卒。}
 諫議隱行,儒林天文。宣班大化,或首或林。--贊王元泰、何彥英、杜輔國、周仲直
 
 
車騎高勁,惟其泛愛,以弱制彊,不陷危墜。--贊吳子遠
 \gezhu{子遠名壹,陳留人也。隨劉焉入蜀。劉璋時,為中郎將,將兵拒先主於涪,詣降。先主定益州,以壹為護軍討逆將軍,納壹妹為夫人。章武元年,為關中都督。建興八年,與魏延入南安界,破魏將費瑤,徙亭侯,進封高陽鄉侯,遷左將軍。十二年,丞相亮卒,以壹督漢中,車騎將軍,假節,領雍州刺史,進封濟陽侯。十五年卒。失其行事,故不為傳。壹族弟班,字元雄,大將軍何進官屬吳匡之子也。以豪俠稱,官位常與壹相亞。先主時,為領軍。後主世,稍遷至驃騎將軍,假節,封綿竹侯。}
 
 
 
 
 安漢宰南,奮擊舊鄉,翦除蕪穢,惟刑以張,廣遷蠻、濮,國用用彊。--贊李德昂
 
 
 
 
 輔漢惟聦,旣機且惠,因言遠思,切問近對,贊時休美,和我業世。--贊張君嗣
 
 
 
 
 鎮北敏思,籌畫有方,導師禳穢,遂事成章。偏任東隅,末命不祥,哀悲本志,放流殊疆。--贊黃公衡
 
 
 
 
 越騎惟忠,厲志自祗,職于內外,念公忘私。--贊楊季休
 
 
征南厚重,征西忠克,統時選士,猛將之烈。--贊趙子龍、陳叔至
 \gezhu{叔至名到,汝南人也。自豫州隨先主,名位常亞趙雲,俱以忠勇稱。建興初,官至永安都督、征西將軍,封亭侯。}
 
 
鎮南粗彊,
 \gezhu{輔元弼名匡,襄陽人也。隨先主入蜀。益州旣定,為巴郡太守。建興中,徙鎮南,為右將軍,封中鄉侯。}
 監軍尚篤,
 \gezhu{劉南和名邕,義陽人也。隨先主入蜀。益州旣定,為江陽太守。建興中,稍遷至監軍後將軍,賜爵關內侯,卒。子式嗣。少子武,有文,與樊建齊名,官亦至尚書。}
 並豫戎任,任自封裔。--贊輔元弼、劉南和
 
 
 
 
 司農性才,敷述允章,藻麗辭理,斐斐有光。--贊秦子敕
 
 
 
 
 正方受遺,豫聞後綱,不陳不僉,造此異端,斥逐當時,任業以喪。--贊李正方
 
 
 
 
 文長剛粗,臨難受命,折衝外禦,鎮保國境。不協不和,忘節言亂,疾終惜始,實惟厥性。--贊魏文長
 
 
 
 
 威公狷狹,取異衆人;閑則及理,逼則傷侵,舍順入凶,大易之云。--贊楊威公
 
 
季常良實,文經勤類,士元言規,處仁聞計,
 \gezhu{文經、士元,皆失其名實、行事、郡縣。處仁本名存,南陽人也。以荊州從事隨先主入蜀,南次至雒,以為廣漢太守。存素不服龐統,統中矢卒,先主發言嘉歎,存曰:「統雖盡忠可惜,然違大雅之義。」先主怒曰:「統殺身成仁,更為非也?」免存官。頃之,病卒。失其行事,故不為傳。}
 孔休、文祥,或才或臧,
 \gezhu{孔休名觀,為荊州主簿別駕從事,見先主傳。失其郡縣。文祥名禎,襄陽人也。隨先主入蜀,歷雒、郫令、南廣漢太守。失其行事。子忠,官至尚書郎。襄陽記曰:習禎有風流,善談論,名亞龐統,而在馬良之右。子忠,亦有名。忠子隆,為步兵校尉,掌校秘書。}
 播播述志,楚之蘭芳。--贊馬季常、衞文經、韓士元、張處仁、殷孔林、習文祥
 
 
國山休風,
 \gezhu{國山名甫,廣漢郪人也。好人流言議。劉璋時,為州書佐。先主定蜀後,為緜竹令,還為荊州議曹從事。隨先主征吳,軍敗於秭歸,遇害。子祐,有父風,官至尚書右選郎。}
 永南耽思;
 \gezhu{永南名邵,廣漢郪人也。先主定蜀後,為州書佐部從事。建興元年,丞相亮辟為西曹掾。亮南征,留邵為治中從事,是歲卒。華陽國志曰:邵兄邈,字漢南,劉璋時為牛鞞長。先主領牧,為從事,正旦命行酒,得進見,讓先主曰:「振威以將軍宗室肺腑,委以討賊,元功未效,先寇而滅;邈以將軍之取鄙州,甚為不宜也。」先主曰:「知其不宜,何以不助之?」邈曰:「匪不敢也,力不足耳。」有司將殺之,諸葛亮為請,得免。乆之,為犍為太守、丞相參軍、安漢將軍。建興六年,亮西征。馬謖在前敗績,亮將殺之,邈諫以「秦赦孟明,用霸西戎,楚誅子玉,二世不競」,失亮意,還蜀。十二年,亮卒,後主素服發哀三日,邈上疏曰:「呂祿、霍、禹未必懷反叛之心,孝宣不好為殺臣之君,直以臣懼其偪,主畏其威,故姦萌生。亮身杖彊兵,狼顧虎視,五大不在邊,臣常危之。今亮殞沒,蓋宗族得全,西戎靜息,大小為慶。」後主怒,下獄誅之。}
 盛衡、承伯,言藏言時;
 \gezhu{盛衡名勳,承伯名齊,皆巴西閬中人也。勳,劉璋時為州書佐,先主定蜀,辟為左將軍屬,後轉州別駕從事,卒。齊為太守張飛功曹。飛貢之先主,為尚書郎。建興中,從事丞相掾,遷廣漢太守,復為飛參軍。亮卒,為尚書。勳、齊皆以才幹自顯見;歸信於州黨,不如姚伷。伷字子緒,亦閬中人。先主定益州後,為功曹書佐。建興元年,為廣漢太守。丞相亮北駐漢中,辟為掾。並進文武之士,亮稱曰:「忠益者莫大於進人,進人者各務其所尚;今姚掾並存剛柔,以廣文武之用,可謂博雅矣,願諸掾各希此事,以屬其望。」遷為參軍。亮卒,稍遷為尚書僕射。時人服其真誠篤粹。延熈五年卒,在作贊之後。}
 孫德果銳,
 \gezhu{孫德名福,梓潼涪人也。先主定益州後,為書佐、西充國長、成都令。建興元年,徙巴西太守,為江州督、楊威將軍,入為尚書僕射,封平陽亭侯。延熈初,大將軍蔣琬出征漢中,福以前監軍領司馬,卒。益部耆舊雜記曰:諸葛亮於武功病篤,後主遣福省侍,遂因諮以國家大計。福往具宣聖旨,聽亮所言,至別去數日,忽馳思未盡其意,遂却騎馳還見亮。亮語福曰:「孤知君還意。近日言語,雖彌日有所不盡,更來亦決耳。君所問者,公琰其宜也。」福謝:「前實失不諮請公,如公百年後,誰可任大事者?故輒還耳。乞復請,蔣琬之後,誰可任者?」亮曰:「文偉可以繼之。」又復問其次,亮不荅。福還,奉使稱旨。福為人精識果銳,敏於從政。子驤,字叔龍,亦有名,官至尚書郎、廣漢太守。}
 偉南篤常;
 \gezhu{偉南名朝,永南兄。郡功曹,舉孝廉,臨邛令,入為別駕從事。隨先主東征吳,章武二年卒於永安。益部耆舊雜記曰:朝又有一弟,早亡,各有才望,時人號之李氏三龍。華陽國志曰:羣下上先主為漢中王;其文,朝所造也。臣松之案耆舊所記,以朝、邵及早亡者為三龍。邈之狂直,不得在此數。}
 德緒義彊,志壯氣剛。
 \gezhu{德緒名祿,巴西安漢人也。先主定益州,為郡從事牙門將。建興三年,為越嶲太守,隨丞相亮南征,為蠻夷所害,時年三十一。弟衡,景耀中為領軍。義彊名士,廣漢郪人,國山從兄也。從先主入蜀後,舉孝廉,為符節長,遷牙門將,出為宕渠太守,徙在犍為。會丞相亮南征,轉為益州太守,將南行,為蠻夷所害。}
 濟濟脩志,蜀之芬香。--贊王國山、李永南、馬盛衡、馬承伯、李孫德、李偉南,龔德緒、王義彊
 
 
休元輕寇,損時致害,
 \gezhu{休元名習,南郡人。隨先主入蜀。先主東征吳,習為領軍,統諸軍,大敗於猇亭。}
 文進奮身,同此顛沛,
 \gezhu{文進名南,亦自荊州隨先主入蜀,領兵從先主征吳,與習俱死。時又有義陽傅肜,先主退軍,斷後拒戰,兵人死盡,吳將語肜令降,肜罵曰:「吳狗!何有漢將軍降者!」遂戰死。拜子僉為左中郎,後為關中都督,景耀六年,又臨危授命。論者嘉其父子弈世忠義。蜀記載晉武帝詔曰:「蜀將軍傅僉,前在關城,身拒官軍,致死不顧。僉父肜,復為劉備戰亡。天下之善一也,豈由彼此以為異?」僉息著、募,後沒入奚官,免為庶人。}
 患生一人,至於弘大。--贊馮休元、張文進
 
 
江陽剛烈,立節明君,兵合遇寇,不屈其身,單夫隻役,隕命於軍。--贊程季然
 \gezhu{季然名畿,巴西閬中人也。劉璋時為漢昌長。縣有賨人,種類剛猛,昔高祖以定關中。巴西太守龐羲以天下擾亂,郡宜有武衞,頗招合部曲。有讒於璋,說羲欲叛者,璋陰疑之。羲聞,甚懼,將謀自守,遣畿子郁宣旨,索兵自助。畿報曰:「郡合部曲,本不為叛,雖有交搆,要在盡誠;若必以懼,遂懷異志,非畿之所聞。」并敕郁曰:「我受州恩,當為州牧盡節。汝為郡吏,當為太守効力,不得以吾故有異志也。」羲使人告畿曰:「爾子在郡,不從太守,家將及禍!」畿曰:「昔樂羊為將,飲子之羹,非父子無恩,大義然也。今雖復羹子,吾必飲之。」羲知畿必不為己,厚陳謝於璋以致無咎。璋聞之,遷畿江陽太守。先主領益州牧,辟為從事祭酒。後隨先主征吳,遇大軍敗績,泝江而還,或告之曰:「後追已至,解船輕去,乃可以免。」畿曰:「吾在軍,未曾為敵走,況從天子而見危哉!」追人遂及畿船,畿身執戟戰,敵船有覆者。衆大至,共擊之,乃死。}
 
 
公弘後生,卓爾奇精,夭命二十,悼恨未呈--。贊程公弘
 \gezhu{公弘,名祁,季然之子也。}
 
 
古之奔臣,禮有來偪,怨興司官,不顧大德。靡有匡救,倍成奔北,自絕于人,作笑二國。--贊糜芳、士仁、郝普、潘濬
 \gezhu{糜芳字子方,東海人也,為南郡太守。士仁字君義,廣陽人也,為將軍,住公安,統屬關羽;與羽有隙,叛迎孫權。郝普字子太,義陽人。先主自荊州入蜀,以普為零陵太守。為吳將呂蒙所譎,開城詣蒙。潘濬字承明,武陵人也。先主入蜀,以為荊州治中,典留州事,亦與關羽不穆。孫權襲羽,遂入吳。普至廷尉,濬至太常,封侯。}
 
 
\gezhu{益部耆舊雜記載王嗣、常播、衞繼三人,皆劉氏王蜀時人,故錄于篇。}
 
 
\gezhu{王嗣字承宗,犍為資中人也。其先,延熈世以功德顯著。舉孝廉,稍遷西安圍督、汶山太守,加安遠將軍。綏集羌、胡,咸悉歸服,諸種素桀惡者皆來首降,嗣待以恩信,時北境得以寧靜。大將軍姜維每出北征,羌、胡出馬牛羊氈毦及義穀裨軍糧,國賴其資。遷鎮軍,故領郡。後從維北征,為流矢所傷,數月卒。戎夷會葬,贈送數千人,號呼涕泣。嗣為人美厚篤至,衆所愛信。嗣子及孫,羌、胡見之如骨肉,或結兄弟,恩至於此。}
 
 
\gezhu{常播字文平,蜀郡江原人也。播仕縣主簿功曹。縣長廣都朱游,建興十五年中被上官誣劾以逋沒官穀,當論重罪。播詣獄訟爭,身受數千杖,肌膚刻爛,毒痛慘至,更歷三獄,幽閉二年有餘。每將考掠,吏先驗問,播不荅,言「但急行罰,無所多問」!辭終不撓,事遂分明。長免刑戮。時唯主簿楊玩亦證明其事,與播言同。衆咸嘉播忘身為君,節義抗烈。舉孝廉,除郪長,年五十餘卒。書於舊德傳,後縣令潁川趙敦圖其像,贊頌之。}
 
 
\gezhu{衞繼字子業,漢嘉嚴道人也。兄弟五人。繼父為縣功曹。繼為兒時,與兄弟隨父游戲庭寺中,縣長蜀郡成都張君無子,數命功曹呼其子省弄,甚憐愛之。張因言宴之間,語功曹欲乞繼,功曹即許之,遂養為子。繼敏達夙成,學識通博,進仕州郡,歷職清顯。而其餘兄弟四人,各無堪當世者,父恒言己之將衰,張明府將盛也。時法禁以異姓為後,故復為衞氏。屢遷拜奉車都尉、大尚書,忠篤信厚,為衆所敬。鍾會之亂,遇害成都。}
 
 
 
 
 評曰:鄧芝堅貞簡亮,臨官忘家,張翼亢姜維之銳,宗預禦孫權之嚴,咸有可稱。楊戲商略,意在不羣,然智度有短,殆罹世難云。
 
 
\end{pinyinscope}