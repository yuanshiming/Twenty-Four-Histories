\article{曹仁傳}
\begin{pinyinscope}
 
 
 曹仁字子孝,太祖從弟也。
 
 
\gezhu{魏書曰:仁祖襃,潁川太守。父熾,侍中、長水校尉。}
 少好弓馬弋獵。後豪傑並起,仁亦陰結少年,得千餘人,周旋淮、泗之間,遂從太祖,為別部司馬,行厲鋒校尉。太祖之破袁術,仁所斬獲頗多。從征徐州,仁常督騎,為軍前鋒。別攻陶謙將呂由,破之,還與大軍合彭城,大破謙軍。從攻費、華、即墨、開陽,謙遣別將救諸縣,仁以騎擊破之。太祖征呂布,仁別攻勾陽,拔之,生獲布將劉何。太祖平黃巾,迎天子都許,仁數有功,拜廣陽太守。太祖器其勇略,不使之郡,以議郎督騎。太祖征張繡,仁別徇旁縣,虜其男女三千餘人。太祖軍還,為繡所追,軍不利,士卒喪氣,仁率厲將士甚奮,太祖壯之,遂破繡。
 
 
 
 
 太祖與袁紹乆相持於官渡,紹遣劉備徇㶏彊諸縣,多舉衆應之。自許以南,吏民不安,太祖以為憂。仁曰:「南方以大軍方有目前急,其勢不能相救,劉備以彊兵臨之,其背叛固宜也。備新將紹兵,未能得其用,擊之可破也。」太祖善其言,遂使將騎擊備,破走之,仁盡復收諸叛縣而還。紹遣別將韓荀鈔斷西道,仁擊荀於雞洛山,大破之。由是紹不敢復分兵出。復與史渙等鈔紹運車,燒其糧穀。
 
 
 
 
 河北旣定,從圍壺關。太祖令曰:「城拔,皆坑之。」連月不下。仁言於太祖曰:「圍城必示之活門,所以開其生路也。今公告之必死,將人自為守。且城固而糧多,攻之則士卒傷,守之則引日乆;今頓兵堅城之下,以攻必死之虜,非良計也。」太祖從之,城降。於是錄仁前後功,封都亭侯。
 
 
 
 
 從平荊州,以仁行征南將軍,留屯江陵,拒吳將周瑜。瑜將數萬衆來攻,前鋒數千人始至,仁登城望之,乃募得三百人,遣部曲將牛金逆與挑戰。賊多,金衆少,遂為所圍。長史陳矯俱在城上,望見金等垂沒,左右皆失色。仁意氣奮怒甚,謂左右取馬來,矯等共援持之。謂仁曰:「賊衆盛,不可當也。假使棄數百人何苦,而將軍以身赴之!」仁不應,遂被甲上馬,將其麾下壯士數十騎出城。去賊百餘步,迫溝,矯等以為仁當住溝上,為金形勢也,仁徑渡溝直前,衝入賊圍,金等乃得解。餘衆未盡出,仁復直還突之,拔出金兵,亡其數人,賊衆乃退。矯等初見仁出,皆懼,及見仁還,乃歎曰:「將軍真天人也!」三軍服其勇。太祖益壯之,轉封安平亭侯。
 
 
 
 
 太祖討馬超,以仁行安西將軍,督諸將拒潼關,破超渭南。蘇伯、田銀反,以仁行驍騎將軍,都督七軍討銀等,破之。復以仁行征南將軍,假節,屯樊,鎮荊州。侯音以宛叛,略傍縣衆數千人,仁率諸軍攻破音,斬其首,還屯樊,即拜征南將軍。關羽攻樊,時漢水暴溢,于禁等七軍皆沒,禁降羽。仁人馬數千人守城,城不沒者數板。羽乘船臨城,圍數重,外內斷絕,糧食欲盡,救兵不至。仁激厲將士,示以必死,將士感之皆無二。徐晃救至,水亦稍減,晃從外擊羽,仁得潰圍出,羽退走。
 
 
 
 
 仁少時不脩行檢,及長為將,嚴整奉法令,常置科於左右,案以從事。鄢陵侯彰北征烏丸,文帝在東宮,為書戒彰曰:「為將奉法,不當如征南邪!」及即王位,拜仁車騎將軍,都督荊、揚、益州諸軍事,進封陳侯,增邑二千,并前三千五百戶。追賜仁父熾謚曰陳穆侯,置守冢十家。
 
 
後召還屯宛。孫權遣將陳邵據襄陽,詔仁討之。仁與徐晃攻破邵,遂入襄陽,使將軍高遷等徙漢南附化民於漢北,文帝遣使即拜仁大將軍。又詔仁移屯臨潁,遷大司馬,復督諸軍據烏江,還屯合淝。黃初四年薨,謚曰忠侯。
 \gezhu{魏書曰:仁時年五十六。傅子曰:曹大司馬之勇,賁、育弗加也。張遼其次焉。}
 子泰嗣,官至鎮東將軍,假節,轉封寗陵侯。泰薨,子初嗣。又分封泰弟楷、範,皆為列侯,而牛金官至後將軍。
 
 
仁弟純,
 \gezhu{英雄記曰:純字子和。年十四而喪父,與同產兄仁別居。承父業,富於財,僮僕人客以百數,純綱紀督御,不失其理,鄉里咸以為能。好學問,敬愛學士,學士多歸焉,由是為遠近所稱。年十八,為黃門侍郎。二十,從太祖到襄邑募兵,遂常從征戰。}
 初以議郎參司空軍事,督虎豹騎從圍南皮。袁譚出戰,士卒多死。太祖欲緩之,純曰:「今千里蹈敵,進不能克,退必喪威;且懸師深入,難以持乆。彼勝而驕,我敗而懼,以懼敵驕,必可克也。」太祖善其言,遂急攻之,譚敗。純麾下騎斬譚首。及北征三郡,純部騎獲單于蹹頓。以前後功封高陵亭侯,邑三百戶。從征荊州,追劉備於長坂,獲其二女輜重,收其散卒。進降江陵,從還譙。建安十五年薨。文帝即位,追謚曰威侯。
 \gezhu{魏書曰:純所督虎豹騎,皆天下驍銳,或從百人將補之,太祖難其帥。純以選為督,撫循甚得人心。及卒,有司白選代,太祖曰:「純之比,何可復得!吾獨不中督邪?」遂不選。}
 子演嗣,官至領軍將軍,正元中進封平樂鄉侯。演薨,子亮嗣。
 
 
\end{pinyinscope}