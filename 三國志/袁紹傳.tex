\article{袁紹傳}
\begin{pinyinscope}
 
 
 袁紹字本初,汝南汝陽人也。高祖父安,為漢司徒。自安以下四世居三公位,由是勢傾天下。
 
 
\gezhu{華嶠漢書曰:安字邵公,好學有威重。明帝時為楚郡太守,治楚王獄,所申理者四百餘家,皆蒙全濟,安遂為名臣。章帝時至司徒,生蜀郡太守京。京弟敞為司空。京子陽,太尉。陽四子:長子平,平弟成,左中郎將,並早卒;成弟逢,逢弟隗,皆為公。魏書曰:自安以下,皆博愛容衆,無所揀擇;賔客入其門,無賢愚皆得所欲,為天下所歸。紹即逢之庶子,術異母兄也,出後成為子。英雄記曰:成字文開,壯健有部分,貴戚權豪自大將軍梁兾以下皆與結好,言無不從。故京師為作諺曰:「事不諧,問文開。」}
 紹有姿皃威容,能折節下士,士多附之,太祖少與交焉。以大將軍掾為侍御史,
 \gezhu{英雄記曰:紹生而父死,二公愛之。幼使為郎,弱冠除濮陽長,有清名。遭母喪,服竟,又追行父服,凡在冢廬六年。禮畢,隱居洛陽,不妄通賔客,非海內知名,不得相見。又好游俠,與張孟卓、何伯求、吳子卿、許子遠、伍德瑜等皆為奔走之友。不應辟命。中常侍趙忠謂諸黃門曰:「袁本初坐作聲價,不應呼召而養死士,不知此兒欲何所為乎?」紹叔父隗聞之,責數紹曰:「汝且破我家!」紹於是乃起應大將軍之命。臣松之案:魏書云「紹,逢之庶子,出後伯父成」。如此記所言,則似實成所生。夫人追服所生,禮無其文,況於所後而可以行之!二書未詳孰是。}
 稍遷中軍校尉,至司隷。
 
 
靈帝崩,太后兄大將軍何進與紹謀誅諸閹官,
 \gezhu{續漢書曰:紹使客張津說進曰:「黃門、常侍秉權日乆,又永樂太后與諸常侍專通財利,將軍宜整頓天下,為海內除患。」進以為然,遂與紹結謀。}
 太后不從。乃召董卓,欲以脅太后。常侍、黃門聞之,皆詣進謝,唯所錯置。時紹勸進便可於此決之,至于再三,而進不許,令紹使洛陽方略武吏檢司諸宦者,又令紹弟虎賁中郎將術選溫厚虎賁二百人,當入禁中,代持兵黃門陛守門戶。中常侍段珪等矯太后命,召進入議,遂殺之,宮中亂。
 \gezhu{九州春秋曰:初紹說進曰:「黃門、常侍累世太盛,威服海內,前竇武欲誅之而反為所害,但坐言語漏泄,以五營士為兵故耳。五營士生長京師,服畏中人,而竇氏反用其鋒,遂果叛走歸黃門,是以自取破滅。今將軍以元舅之尊,二府並領勁兵,其部曲將吏皆英雄名士,樂盡死力,事在掌握,天贊其時也。今為天下誅除貪穢,功勳顯著,垂名後世,雖周之申伯,何足道哉?今大行在前殿,將軍以詔書領兵衞守,可勿入宮。」進納其言,後更狐疑。紹懼進之改變,脅進曰:「今交搆已成,形勢已露,將軍何為不早決之?事留變生,後機禍至。」進不從,遂敗。}
 術將虎賁燒南宮嘉德殿青瑣門,欲以迫出珪等。珪等不出,劫帝及帝弟陳留王走小平津。紹旣斬宦者所署司隷校尉許相,遂勒兵捕諸閹人,無少長皆殺之。或有無鬚而誤死者,至自發露形體而後得免,宦者或有行善自守而猶見及。其濫如此。死者二千餘人。急追珪等,珪等悉赴河死。帝得還宮。
 
 
董卓呼紹,議欲廢帝,立陳留王。是時紹叔父隗為太傅,紹偽許之,曰:「此大事,出當與太傅議。」卓曰:「劉氏種不足復遺。」紹不應,橫刀長揖而去。
 \gezhu{獻帝春秋曰:卓欲廢帝,謂紹曰:「皇帝沖闇,非萬乘之主。陳留王猶勝,今欲立之。人有小智,大或癡,亦知復何如,為當且爾;卿不見靈帝乎?念此令人憤毒!」紹曰;「漢家君天下四百許年,恩澤深渥,兆民戴之來乆。今帝雖幼沖,未有不善宣聞天下,公欲廢適立庶,恐衆不從公議也。」卓謂紹曰:「豎子!天下事豈不決我?我今為之,誰敢不從?爾謂董卓刀為不利乎!」紹曰:「天下健者,豈唯董公?」引佩刀橫揖而出。臣松之以為紹于時與卓未搆嫌隙,故卓與之諮謀。若但以言議不同,便罵為豎子,而有推刃之心,及紹復荅,屈彊為甚,卓又安能容忍而不加害乎?且如紹此言,進非亮正,退違詭遜,而顯其競爽之旨,以觸哮闞之鋒,有志功業者,理豈然哉!此語,妄之甚矣。}
 紹旣出,遂亡奔兾州。侍中周毖、城門校尉伍瓊、議郎何顒等,皆名士也,卓信之,而陰為紹,乃說卓曰:「夫廢立大事,非常人所及。紹不達大體,恐懼故出奔,非有他志也。今購之急,勢必為變。袁氏樹恩四世,門世故吏徧於天下,若収豪傑以聚徒衆,英雄因之而起,則山東非公之有也。不如赦之,拜一郡守,則紹喜於免罪,必無患矣。」卓以為然,乃拜紹勃海太守,封邟鄉侯。
 
 
紹遂以勃海起兵,將以誅卓。語在武紀。紹自號車騎將軍,主盟,與兾州牧韓馥立幽州牧劉虞為帝,遣使奉章詣虞,虞不敢受。後馥軍安平,為公孫瓚所敗。瓚遂引兵入兾州,以討卓為名,內欲襲馥。馥懷不自安。
 \gezhu{英雄記曰:逢紀說紹曰:「將軍舉大事而仰人資給,不據一州,無以自全。」紹荅云:「兾州兵彊,吾士饑乏,設不能辦,無所容立。」紀曰:「可與公孫瓚相聞,導使來南,擊取兾州。公孫必至而馥懼矣,因使說利害,為陳禍福,馥必遜讓。於此之際,可據其位。」紹從其言而瓚果來。}
 會卓西入關,紹還軍延津,因馥惶遽,使陳留高幹、頴川荀諶等說馥曰:「公孫瓚乘勝來向南,而諸郡應之,袁車騎引軍東向,此其意不可知,竊為將軍危之。」馥曰:「為之柰何?」諶曰:「公孫提燕、代之卒,其鋒不可當。袁氏一時之傑,必不為將軍下。夫兾州,天下之重資也,若兩雄并力,兵交於城下,危亡可立而待也。夫袁氏,將軍之舊,且同盟也,當今為將軍計,莫若舉兾州以讓袁氏。袁氏得兾州,則瓚不能與之爭,必厚德將軍。兾州入於親交,是將軍有讓賢之名,而身安於泰山也。願將軍勿疑!」馥素恇怯,因然其計。馥長史耿武、別駕閔純、治中李歷諫馥曰:「兾州雖鄙,帶甲百萬,穀支十年。袁紹孤客窮軍,仰我鼻息,譬如嬰兒在股掌之上,絕其哺乳,立可餓殺。柰何乃欲以州與之?」馥曰:「吾,袁氏故吏,且才不如本初,度德而讓,古人所貴,諸君獨何病焉!」從事趙浮、程奐請以兵拒之,馥又不聽。乃讓紹,
 \gezhu{九州春秋曰:馥遣都督從事趙浮、程奐將彊弩萬張屯河陽。浮等聞馥欲以兾州與紹,自孟津馳東下。時紹尚在朝歌清水口,浮等從後來,船數百艘,衆萬餘人,整兵鼓夜過紹營,紹甚惡之。浮等到,謂馥曰:「袁本初軍無斗糧,各己離散,雖有張楊、於扶羅新附,未肯為用,不足敵也。小從事等請自以見兵拒之,旬日之間,必土崩瓦解;明將軍但當開閤高枕,何憂何懼!」馥不從,乃避位,出居趙忠故舍。遣子齎兾州印綬於黎陽與紹。}
 紹遂領兾州牧。
 
 
從事沮授
 \gezhu{沮音葅。}
 說紹曰:「將軍弱冠登朝,則播名海內;值廢立之際,則忠義奮發;單騎出奔,則董卓懷怖;濟河而北,則勃海稽首。振一郡之卒,撮兾州之衆,威震河朔,名重天下。雖黃巾猾亂,黑山跋扈,舉軍東向,則青州可定;還討黑山,則張燕可滅;回衆北首,則公孫必喪;震脅戎狄,則匈奴必從。橫大河之北,合四州之地,收英雄之才,擁百萬之衆,迎大駕於西京,復宗廟於洛邑,號令天下,以討未復,以此爭鋒,誰能敵之?比及數年,此功不難。」紹喜曰:「此吾心也。」即表授為監軍、奮威將軍。
 \gezhu{獻帝傳曰:沮授,廣平人,少有大志,多權略。仕州別駕,舉茂才,歷二縣令,又為韓馥別駕,表拜騎都尉。袁紹得兾州,又辟焉。英雄記曰:是時年號初平,紹字本初,自以為年與字合,必能克平禍亂。}
 卓遣執金吾胡母班、將作大匠吳脩齎詔書喻紹,紹使河內太守王匡殺之。
 \gezhu{漢末名士錄曰:班字季皮,太山人,少與山陽度尚、東平張邈等八人並輕財赴義,振濟人士,世謂之八廚。謝承後漢書曰:班,王匡之妹夫,董卓使班奉詔到河內,解釋義兵。匡受袁紹旨,收班繫獄,欲殺之以徇軍。班與匡書云:「自古已來,未有下土諸侯舉兵向京師者。劉向傳曰『擲鼠忌器』,器猶忌之,況卓今處宮闕之內,以天子為藩屏,幼主在宮,如何可討?僕與太傅馬公、太僕趙岐、少府陰脩俱受詔命。關東諸郡雖實嫉卓,猶以銜奉王命,不敢玷辱。而足下獨囚僕於獄,欲以釁鼓,此悖暴無道之甚者也。僕與董卓有何親戚,義豈同惡?而足下張虎狼之口,吐長蛇之毒,恚卓遷怒,何甚酷哉!死,人之所難,然恥為狂夫所害。若亡者有靈,當訴足下於皇天。夫婚姻者禍福之機,今日著矣。曩為一體,今為血讎。亡人子二人,則君之甥,身沒之後,慎勿令臨僕尸骸也。」匡得書,抱班二子而泣。班遂死於獄。班甞見太山府君及河伯,事在搜神記,語多不載。}
 卓聞紹得關東,乃悉誅紹宗族太傅隗等。當是時,豪俠多附紹,皆思為之報,州郡鋒起,莫不假其名。馥懷懼,從紹索去,往依張邈。
 \gezhu{英雄記曰:紹以河內朱漢為都官從事。漢先時為馥所不禮,內懷怨恨,且欲邀迎紹意,擅發城郭兵圍守馥第,拔刃登屋。馥走上樓,收得馥大兒,槌折兩脚。紹亦立收漢,殺之。馥猶憂怖,故報紹索去。}
 後紹遣使詣邈,有所計議,與邈耳語。馥在坐上,謂見圖構,無何起至溷自殺。
 \gezhu{英雄記曰:公孫瓚擊青州黃巾賊,大破之,還屯廣宗,改易守令,兾州長吏無不望風響應,開門受之。紹自往征瓚,合戰於界橋南二十里。瓚步兵三萬餘人為方陣,騎為兩翼,左右各五千餘匹,白馬義從為中堅,亦分作兩校,左射右,右射左,旌旗鎧甲,光照天地。紹令麴義以八百兵為先登,彊弩千張夾承之,紹自以步兵數萬結陣於後。義乆在涼州,曉習羌鬬,兵皆驍銳。瓚見其兵少,便放騎欲陵蹈之。義兵皆伏楯下不動,未至數十步,乃同時俱起,揚塵大叫,直前衝突,彊弩雷發,所中必倒,臨陣斬瓚所署兾州刺史嚴綱甲首千餘級。瓚軍敗績,步騎奔走,不復還營。義追至界橋;瓚殿兵還戰橋上,義復破之,遂到瓚營,拔其牙門,營中餘衆皆復散走。紹在後,未到橋十數里,下馬發鞍,見瓚已破,不為設備,惟帳下彊弩數十張,大戟士百餘人自隨。瓚部迸騎二千餘匹卒至,便圍紹數重,弓矢雨下。別駕從事田豐扶紹欲郤入空垣,紹以兜鍪撲地曰:「大丈夫當前鬬死,而入牆間,豈可得活乎?」彊弩乃亂發,多所殺傷。瓚騎不知是紹,亦稍引郤;會麴義來迎,乃散去。瓚每與虜戰,常乘白馬,追不虛發,數獲戎捷,虜相告云「當避白馬」。因虜所忌,簡其白馬數千匹,選騎射之士,號為白馬義從;一曰胡夷健者常乘白馬,瓚有健騎數千,多乘白馬,故以號焉。紹旣破瓚,引軍南到薄落津,方與賔客諸將共會,聞魏郡兵反,與黑山賊于毒共覆鄴城,遂殺太守栗成。賊十餘部,衆數萬人,聚會鄴中。坐上諸客有家在鄴者,皆憂怖失色,或起啼泣,紹容皃不變,自若也。賊陶升者,故內黃小吏也,有善心,獨將部衆踰西城入,閉守州門,不內他賊,以車載紹家及諸衣冠在州內者,身自扞衞,送到斥丘乃還。紹到,遂屯斥丘,以陶升為建義中郎將。乃引軍入朝歌鹿場山蒼巖谷討于毒,圍攻五日,破之,斬毒及長安所署兾州牧壺壽。遂尋山北行,薄擊諸賊左髭丈八等,皆斬之。又擊劉石、青牛角、黃龍、左校、郭大賢、李大目、于氐根等,皆屠其屯壁,奔走得脫,斬首數萬級。紹復還屯鄴。初平四年,天子使太傅馬日磾、太僕趙岐和解關東。岐別詣河北,紹出迎於百里上,拜奉帝命。岐住紹營,移書告瓚。瓚遣使具與紹書曰:「趙太僕以周召之德,銜命來征,宣揚朝恩,示以和睦,曠若開雲見日,何喜如之?昔賈復、寇恂亦爭士卒,欲相危害,遇光武之寬,親俱陛見,同輿共出,時人以為榮。自省邊鄙,得與將軍共同此福,此誠將軍之眷,而瓚之幸也。」麴義後恃功而驕恣,紹乃殺之。}
 
 
初,天子之立非紹意,及在河東,紹遣潁川郭圖使焉。圖還說紹迎天子都鄴,紹不從。
 \gezhu{獻帝傳云:沮授說紹曰:「將軍累葉輔弼,世濟忠義。今朝廷播越,宗廟毀壞,觀諸州郡外託義兵,內圖相滅,未有存主卹民者。且今州城粗定,宜迎大駕,安宮鄴都,挾天子而令諸侯,畜士馬以討不庭,誰能禦之!」紹恱,將從之。郭圖、淳于瓊曰:「漢室陵遲,為日乆矣,今欲興之,不亦難乎!且今英雄據有州郡,衆動萬計,所謂秦失其鹿,先得者王。若迎天子以自近,動輙表聞,從之則權輕,違之則拒命,非計之善者也。」授曰:「今迎朝廷,至義也,又於時宜大計也,若不早圖,必有先人者也。夫權不失機,功在速捷,將軍其圖之!」紹弗能用。案此書稱郭圖之計,則與本傳違也。}
 會太祖迎天子都許,收河南地,關中皆附。紹悔,欲令太祖徙天子都鄄城以自密近,太祖拒之。天子以紹為太尉,轉為大將軍,封鄴侯,
 \gezhu{獻帝春秋曰:紹恥班在太祖下,怒曰;「曹操當死數矣,我輒救存之,今乃背恩,挾天子以令我乎!」太祖聞,而以大將軍讓於紹。}
 紹讓侯不受。頃之。擊破瓚於易京,井其衆。
 \gezhu{典略曰:自此紹貢御希慢,私使主薄耿苞密白曰:「赤德衰盡,袁為黃胤,宜順天意。」紹以苞密白事示軍府將吏。議者咸以苞為妖妄宜誅,紹乃殺苞以自解。九州春秋曰:紹延徵北海鄭玄而不禮,趙融聞之曰:「賢人者,君子之望也。不禮賢,是失君子之望也。夫有為之君,不敢失萬民之歡心,況於君子乎?失君子之望,難乎以有為也。」英雄記載太祖作董卓歌,辭云:「德行不虧缺,變故自難常。鄭康成行酒,伏地氣絕,郭景圖命盡於園桑。」如此之文,則玄無病而卒。餘書不見,故載錄之。}
 出長子譚為青州,沮授諫紹:「必為禍始。」紹不聽,曰:「孤欲令諸兒各據一州也。」
 \gezhu{九州春秋載授諫辭曰:「世稱一兔走衢,萬人逐之,一人獲之,貪者悉止,分定故也。且年均以賢,德均則卜,古之制也。願上惟先代成敗之戒,下思逐兔分定之義。」紹曰:「孤欲令四兒各據一州,以觀其能。」授出曰:「禍其始此乎!」譚始至青州,為都督,未為刺史,後太祖拜為刺史。其土自河而西,蓋不過平原而已。遂北排田楷,東攻孔融,曜兵海隅,是時百姓無主,欣戴之矣。然信用羣小,好受近言,肆志奢淫,不知稼穡之艱難。華彥、孔順皆姦佞小人也,信以為腹心;王脩等備官而已。然能接待賔客,慕名敬士。使婦弟領兵在內,至令草竊,巿井而外,虜掠田野;別使兩將募兵下縣,有賂者見免,無者見取,貧弱者多,乃至於竄伏丘野之中,放兵捕索如獵鳥獸。邑有萬戶者,著籍不盈數百,收賦納稅,參分不入一。招命賢士,不就;不彊棄軍期,安居族黨,亦不能罪也。}
 又以中子熈為幽州,甥高幹為并州。衆數十萬,以審配、逢紀統軍事,田豐、荀諶、許攸為謀主,顏良、文醜為將率,簡精卒十萬,騎萬匹,將攻許。
 \gezhu{世語曰:紹步卒五萬,騎八千。孫盛評曰:案魏武謂崔琰曰「昨案貴州戶籍,可得三十萬衆」。由此推之,但兾州勝兵已如此,況兼幽、并及青州乎?紹之大舉,必悉師而起,十萬近之矣。獻帝傳曰:紹將南師,沮授、田豐諫曰:「師出歷年,百姓疲弊,倉庾無積,賦役方殷,此國之深憂也。宜先遣使獻捷天子,務農逸民;若不得通,乃表曹氏隔我王路,然後進屯黎陽,漸營河南,益作舟船,繕治器械,分遣精騎,鈔其邊鄙,令彼不得安,我取其逸。三年之中,事可坐定也。」審配、郭圖曰:「兵書之法,十圍五攻,敵則能戰。今以明公之神武,跨河朔之彊衆,以伐曹氏。譬若覆手,今不時取,後難圖也。」授曰:「蓋救亂誅暴,謂之義兵;恃衆憑彊,謂之驕兵。兵義無敵,驕者先滅。曹氏迎天子安宮許都,今舉兵南向,於義則違。且廟勝之策,不在彊弱。曹氏法令旣行,士卒精練,非公孫瓚坐受圍者也。今棄萬安之術,而興無名之兵,竊為公懼之!」圖等曰:「武王伐紂,不曰不義,況兵加曹氏而云無名!且公師武臣竭力,將士憤怒,人思自騁,而不及時早定大業,慮之失也。夫天與弗取,反受其咎,此越之所以霸,吳之所以亡也。監軍之計,計在持牢,而非見時知機之變也。」紹從之。圖等因是譖授「監統內外,威震三軍,若其浸盛,何以制之?夫臣與主不同者昌,主與臣同者亡,此黃石之所忌也。且御衆於外,不宜知內。」紹疑焉。乃分監軍為三都督,使授及郭圖、淳于瓊各典一軍,遂合而南。}
 
 
先是,太祖遣劉備詣徐州拒袁術。術死,備殺刺史車冑,引軍屯沛。紹遣騎佐之。太祖遣劉岱、王忠擊之,不克。建安五年,太祖自東征備。田豐說紹襲太祖後,紹辭以子疾,不許,豐舉杖擊地曰:「夫遭難遇之機,而以嬰兒之病失其會,惜哉!」太祖至,擊破備;備奔紹。
 \gezhu{魏氏春秋載紹檄州郡文曰:「蓋聞明主圖危以制變,忠臣慮難以立權。曩者彊秦弱主,趙高執柄,專制朝命,威福由己,終有望夷之禍,汙辱至今。及臻呂后,祿、產專政,擅斷萬機,決事省禁,下陵上替,海內寒心。於是絳侯、朱虛興威奮怒,誅夷逆亂,尊立太宗,故能道化興隆,光明顯融,此則大臣立權之明表也。司空曹操,祖父騰,故中常侍,與左悺、徐璜並作妖孽,饕餮放橫,傷化虐民。父嵩,乞匄攜養,因贓假位,輿金輦璧,輸貨權門,竊盜鼎司,傾覆重器。操贅閹遺醜,本無令德,僄狡鋒俠,好亂樂禍。幕府昔統鷹揚,掃夷凶逆。續遇董卓侵官暴國,於是提劒揮鼓,發命東夏,方收羅英雄,棄瑕錄用,故遂與操參咨策略,謂其鷹犬之才,爪牙可任。至乃愚佻短慮,輕進易退,傷夷折衂,數喪師徒。幕府輙復分兵命銳,脩完補輯,表行東郡太守、兖州刺史,被以虎文,授以偏師,獎蹙威柄,兾獲秦師一克之報。而操遂乘資跋扈,肆行酷裂,割剥元元,殘賢害善。故九江太守邊讓,英才俊逸,天下知名,以直言正色,論不阿諂,身被梟縣之戮,妻孥受灰滅之咎。自是士林憤痛,民怨彌重,一夫奮臂,舉州同聲,故躬破於徐方,地奪於呂布,仿偟東裔,蹈據無所。幕府唯彊幹弱枝之義,且不登叛人之黨,故復援旌擐甲,席卷赴征,金鼓響震,布衆破沮,拯其死亡之患,復其方伯之任,是則幕府無德於兖土之民,而有大造於操也。後會鑾駕東反,羣虜亂政。時兾州方有北鄙之警,匪遑離局,故使從事中郎徐勛就發遣操,使繕脩郊廟,翼衞幼主。而便放志專行,脅遷省禁,卑侮王官,敗法亂紀,坐召三臺,專制朝政,爵賞由心,刑戮在口,所愛光五宗,所惡滅三族,羣談者蒙顯誅,腹議者蒙隱戮,道路以目,百寮鉗口,尚書記朝會,公卿充員品而已。故太尉楊彪,歷典三司,享國極位,操因睚眥,被以非罪,榜楚并兼,五毒俱至,觸情放慝,不顧憲章。又議郎趙彥,忠諫直言,議有可納,故聖朝含聽,改容加錫,操欲迷奪時權,杜絕言路,擅收立殺,不俟報聞。又梁孝王,先帝母弟,墳陵尊顯,松栢桑梓,猶宜恭肅,而操率將校吏士親臨發掘,破棺裸尸,略取金寶,至令聖朝流涕,士民傷懷。又署發丘中郎將、摸金校尉,所過墮突,無骸不露。身處三公之官,而行桀虜之態,殄國虐民,毒流人鬼。加其細政苛慘,科防互設,繒繳充蹊,坑穽塞路,舉手挂網羅,動足蹈機陷,是以兖、豫有無聊之民,帝都有吁嗟之怨。歷觀古今書籍,所載貪殘虐烈無道之臣,於操為甚。幕府方詰外姦,未及整訓,加意含覆,兾可彌縫。而操豺狼野心,潛苞禍謀,乃欲撓折棟梁,孤弱漢室,除滅中正,專為梟雄。往歲伐鼓北征,討公孫瓚,彊禦桀逆,拒圍一年。操因其未破,陰交書命,欲託助王師,以相掩襲,故引兵造河,方舟北濟。會其行人發路,瓚亦梟夷,故使鋒芒挫縮,厥圖不果。屯據敖倉,阻河為固,乃欲以螳蜋之斧,禦隆車之隧。幕府奉漢威靈,折衝宇宙,長戟百萬,胡騎千羣,奮中黃、育、獲之材,騁良弓勁弩之勢,并州越太行,青州涉濟、漯,大軍汎黃河以角其前,荊州下宛、葉而掎其後,雷震虎步,並集虜庭,若舉炎火以焫飛蓬,覆滄海而沃熛炭,有何不消滅者哉?當今漢道陵遲,綱弛紀絕。操以精兵七百,圍守宮闕,外稱陪衞,內以拘執,懼其篡逆之禍,因斯而作。乃忠臣肝腦塗地之秋,烈士立功之會也,可不勗哉!」此陳琳之辭。}
 
 
紹進軍黎陽,遣顏良攻劉延於白馬。沮授又諫紹:「良性促狹,雖驍勇不可獨任。」紹不聽。太祖救延,與良戰,破斬良。
 \gezhu{獻帝傳曰:紹臨發,沮授會其宗族,散資財以與之曰:「夫勢在則威無不加,勢亡則不保一身,哀哉!」其弟宗曰:「曹公士馬不敵,君何懼焉!」授曰:「以曹兖州之明略,又挾天子以為資,我雖克公孫,衆實疲弊,而將驕主忲,軍之破敗,在此舉也。揚雄有言,『六國蚩蚩,為嬴弱姬』,今之謂也。」}
 紹渡河,壁延津南,使劉備、文醜挑戰。太祖擊破之,斬醜,再戰,禽紹大將。紹軍大震。
 \gezhu{獻帝傳曰:紹將濟河,沮授諫曰:「勝負變化,不可不詳。今宜留屯延津,分兵官渡,若其克獲,還迎不晚,設其有難,衆弗可還。」紹弗從。授臨濟歎曰:「上盈其志,下務其功,悠悠黃河,吾其反乎?」遂以疾辭。紹恨之,乃省其所部兵屬郭圖。}
 太祖還官渡。沮授又曰:「北兵數衆而果勁不及南,南穀虛少而貨財不及北;南利在於急戰,北利在於緩搏。宜徐持乆,曠以日月。」紹不從。連營稍前,逼官渡,合戰,太祖軍不利,復壁。紹為高櫓,起土山,射營中,營中皆蒙楯,衆大懼。太祖乃為發石車,擊紹樓,皆破,紹衆號曰霹靂車。
 \gezhu{魏氏春秋曰:以古有矢石,又傳言「旝動而鼓」,說文曰「旝,發石也」,於是造發石車。}
 紹為地道,欲襲太祖營。太祖輙於內為長塹以拒之,又遣奇兵襲擊紹運車,大破之,盡焚其穀。太祖與紹相持日乆,百姓疲乏,多叛應紹,軍食乏。會紹遣淳于瓊等將兵萬餘人北迎運車,沮授說紹:「可遣將蔣奇別為支軍於表,以斷曹公之鈔。」紹復不從。瓊宿烏巢,去紹軍四十里。太祖乃留曹洪守,自將步騎五千候夜潛往攻瓊。紹遣騎救之,敗走。破瓊等,悉斬之。太祖還,未至營,紹將高覽、張郃等率其衆降。紹衆大潰,紹與譚單騎退渡河。餘衆偽降,盡坑之。
 \gezhu{張璠漢紀云:殺紹卒凡八萬人。}
 沮授不及紹渡,為人所執,詣太祖,
 \gezhu{獻帝傳云:授大呼曰:「授不降也,為軍所執耳!」太祖與之有舊,逆謂授曰:「分野殊異,遂用圮絕,不圖今日乃相禽也!」授對曰:「兾州失策,以取奔北。授智力俱困,宜其見禽耳。」太祖曰:「本初無謀,不用君計,今喪亂過紀,國家未定,當相與圖之。」授曰:「叔父、母、弟,縣命袁氏,若蒙公靈,速死為福。」太祖歎曰:「孤早相得,天下不足慮。」}
 太祖厚待之。後謀還袁氏,見殺。
 
 
初,紹之南也,田豐說紹曰:「曹公善用兵,變化無方,衆雖少,未可輕也,不如以乆持之。將軍據山河之固,擁四州之衆,外結英雄,內脩農戰,然後簡其精銳,分為奇兵,乘虛迭出,以擾河南,救右則擊其左,救左則擊其右,使敵疲於奔命,民不得安業;我未勞而彼已困,不及二年,可坐克也。今釋廟勝之策,而決成敗於一戰,若不如志,悔無及也。」紹不從。豐懇諫,紹怒甚,以為沮衆,械繫之。紹軍旣敗,或謂豐曰:「君必見重。」豐曰:「若軍有利,吾必全,今軍敗,吾其死矣。」紹還,謂左右曰:「吾不用田豐言,果為所笑。」遂殺之。
 \gezhu{先賢行狀曰:豐字元皓,鉅鹿人,或云勃海人。豐天姿瓌傑,權略多奇,少喪親,居喪盡哀,日月雖過,笑不至矧。博覽多識,名重州黨。初辟太尉府,舉茂才,遷侍御史。閹宦擅朝,英賢被害,豐乃棄官歸家。袁紹起義,卑辭厚幣以招致豐,豐以王室多難,志存匡救,乃應紹命,以為別駕。勸紹迎天子,紹不納。紹後用豐謀,以平公孫瓚。逢紀憚豐亮直,數讒之於紹,紹遂忌豐。紹軍之敗也,土崩奔北,師徒略盡,軍皆拊膺而泣曰:「向令田豐在此,不至於是也。」紹謂逢紀曰:「兾州人聞吾軍敗,皆當念吾,唯田別駕前諫止吾,與衆不同,吾亦慙見之。」紀復曰:「豐聞將軍之退,拊手大笑,喜其言之中也。」紹於是有害豐之意。初,太祖聞豐不從戎,喜曰:「紹必敗矣。」及紹奔遁,復曰:「向使紹用田別駕計,尚未可知也。」孫盛曰:觀田豐、沮授之謀,雖良、平何以過之?故君貴審才,臣尚量主;君用忠良,則霸王之業隆,臣奉闇后,則覆亡之禍至:存亡榮辱,常必由茲。豐知紹將敗,敗則己必死,甘冒虎口以盡忠規,烈士之於所事,慮不存己。夫諸侯之臣,義有去就,況豐與紹非純臣乎!詩云「逝將去汝,適彼樂土」,言去亂邦,就有道可也。}
 紹外寬雅,有局度,憂喜不形於色,而內多忌害,皆此類也。
 
 
 
 
 兾州城邑多叛,紹復擊定之。自軍敗後發病,七年,憂死。
 
 
紹愛少子尚,皃美,欲以為後而未顯。
 \gezhu{典論曰:譚長而惠,尚少而美。紹妻劉氏愛尚,數稱其才,紹亦奇其皃,欲以為後,未顯而紹死。劉氏性酷妬,紹死,僵尸未殯,寵妾五人,劉盡殺之。以為死者有知,當復見紹於地下,乃髠頭墨靣以毀其形。尚又為盡殺死者之家。}
 審配、逢紀與辛評、郭圖爭權,配、紀與尚比,評、圖與譚比。衆以譚長,欲立之。配等恐譚立而評等為己害,緣紹素意,乃奉尚代紹位。譚至,不得立,自號車騎將軍。由是譚、尚有隙。太祖北征譚、尚。譚軍黎陽,尚少與譚兵,而使逢紀從譚。譚求益兵,配等議不與。譚怒,殺紀。
 \gezhu{英雄記曰:紀字元圖。初,紹去董卓出奔,與許攸及紀俱詣兾州,紹以紀聦達有計策,甚親信之,與共舉事。後審配任用,與紀不睦。或有讒配於紹,紹問紀,紀稱「配天性烈直,古人之節,不宜疑之」。紹曰:「君不惡之邪?」紀荅曰:「先日所爭者私情,今所陳者國事。」紹善之,卒不廢配。配由是更與紀為親善。}
 太祖渡河攻譚,譚告急於尚。尚欲分兵益譚,恐譚遂奪其衆,乃使審配守鄴,尚自將兵助譚,與太祖相拒於黎陽。自二月至九月,大戰城下,譚、尚敗退,入城守。太祖將圍之,乃夜遁。追至鄴,收其麥,拔陰安,引軍還許。太祖南征荊州,軍至西平。譚、尚遂舉兵相攻,譚敗奔平原。尚攻之急,譚遣辛毗詣太祖請救。太祖乃還救譚,十月至黎陽。
 \gezhu{魏氏春秋載劉表遺譚書曰:「天篤降害,禍難殷流,尊公殂殞,四海悼心。賢胤承統,遐邇屬望,咸欲展布旅力,以投盟主,雖亡之日,猶存之願也。何寤青蠅飛於干旍,無極游於二壘,使股肱分為二體,背膂絕為異身!昔三王五伯,下及戰國,父子相殘,蓋有之矣;然或欲以成王業,或欲以定伯功,或欲以顯宗主,或欲以固冢嗣,未有棄親即異,抏其本根,而能崇業濟功,垂祚後世者也。若齊襄復九世之讎,士匄卒荀偃之事,是故春秋美其義,君子稱其信。夫伯游之恨於齊,未若太公之忿曹;宣子之承業,未若仁君之繼統也。且君子之違難不適讎國,豈可忘先君之怨,棄至親之好,為萬世之戒,遺同盟之恥哉!兾州不弟之慠,旣已然矣;仁君當降志辱身,以匡國為務;雖見憎於夫人,未若鄭莊之於姜氏,兄弟之嫌,未若重華之於象傲也。然莊公有大隧之樂,象受有鼻之封。願棄捐前忿,遠思舊義,復為母子昆弟如初。」又遺尚書曰:「知變起辛、郭,禍結同生,追閼伯、實沈之蹤,忘常棣死喪之義,親尋干戈,僵尸流血,聞之哽咽,雖存若亡。昔軒轅有涿鹿之戰,周武有商、奄之師,皆所以翦除穢害而定王業,非彊弱之事爭,喜怒之忿也。故雖滅親不為尤,誅兄不傷義。今二君初承洪業,纂繼前軌,進有國家傾危之慮,退有先公遺恨之負,當唯義是務,唯國是康。何者?金木水火以剛柔相濟,然後克得其和,能為民用。今青州天性峭急,迷於曲直。仁君度數弘廣,綽然有餘,當以大包小,以優容劣,先除曹操以卒先公之恨,事定之後,乃議曲直之計,不亦善乎!若留神遠圖,克己復禮,當振斾長驅,共獎王室,若迷而不反,違而無改,則胡夷將有誚讓之言,況我同盟,復能戮力為君之役哉?此韓盧、東郭自困於前而遺田父之獲者也。憤踊鶴望,兾聞和同之聲。若其泰也,則袁族其與漢升降乎!如其否也,則同盟永無望矣。」譚、尚盡不從。漢晉春秋載審配獻書於譚曰:「春秋之義,國君死社稷,忠臣死王命。苟有圖危宗廟,敗亂國家,王綱典律,親踈一也。是以周公垂泣而蔽管、蔡之獄,季友歔欷而行鍼叔之鴆。何則?義重人輕,事不得已也。昔衞靈公廢蒯聵而立輙,蒯聵為不道,入戚以篡,衞師伐之。春秋傳曰:『以石曼姑之義,為可以拒之。』是以蒯聵終獲叛逆之罪,而曼姑永享忠臣之名。父子猶然,豈況兄弟乎!昔先公廢絀將軍以續賢兄,立我將軍以為適嗣,上告祖靈,下書譜牒,先公謂將軍為兄子,將軍謂先公為叔父,海內遠近,誰不備聞?且先公即世之日,我將軍斬衰居廬,而將軍齋于堊室,出入之分,於斯益明。是時凶臣逢紀,妄畫蛇足,曲辭諂媚,交亂懿親,將軍奮赫然之怒,誅不旋時,我將軍亦奉命承旨,加以淫刑。自是之後,癰疽破潰,骨肉無絲髮之嫌,自疑之臣,皆保生全之福。故悉遣彊胡,簡命名將,料整器械,選擇戰士,殫府庫之財,竭食土之實,其所以供奉將軍,何求而不備?君臣相率,共衞旌麾,戰為鴈行,賦為幣主,雖傾倉覆庫,翦剥民物,上下欣戴,莫敢告勞。何則?推戀戀忠赤之情,盡家家肝腦之計,脣齒輔車,不相為賜。謂為將軍心合意同,混齊一體,必當并威偶勢,禦寇寧家。何圖凶險讒慝之人,造飾無端,誘導姦利,至令將軍翻然改圖,忘孝友之仁,聽豺狼之謀,誣先公廢立之言,違近者在喪之位,悖紀綱之理,不顧逆順之節,橫易兾州之主,欲當先公之繼。遂放兵鈔撥,屠城殺吏,交尸盈原,裸民滿野,或有髠𩮜髮膚,割截支體,冤魂痛於幽冥,創痍號於草棘。又乃圖獲鄴城,許賜秦、胡財物婦女,豫有分界。或聞告令吏士云:『孤雖有老母,輙使身體完具而已。』聞此言者,莫不驚愕失氣,悼心揮涕,使太夫人憂哀憤懣於堂室,我州君臣士友假寐悲歎,無所措其手足;念欲靜師拱默以聽執事之圖,則懼違春秋死命之節,貽太夫人不測之患,隕先公高世之業。且三軍憤慨,人懷私怒,我將軍辭不獲已,以及館陶之役。是時外為禦難,內實乞罪,旣不見赦,而屠各二三其心,臨陣叛戾。我將軍進退無功,首尾受敵,引軍奔避,不敢告辭。亦謂將軍當少垂親親之仁,貺以緩追之惠,而乃尋蹤躡軌,無所逃命。困獸必鬬,以干嚴行,而將軍師旅土崩瓦解,此非人力,乃天意也。是後又望將軍改往修來,克己復禮,追還孔懷如初之愛;而縱情肆怒,趣破家門,企踵鶴立,連結外讎,散鋒於火,播增毒螫,烽煙相望,涉血千里,遺城厄民,引領悲怨,雖欲勿救,惡得已哉!故遂引軍東轅,保正疆埸,雖近郊壘,未侵境域,然望旌麾,能不永歎?配等備先公家臣,奉廢立之命。而圖等干國亂家,禮有常刑。故奮弊州之賦,以除將軍之疾,若乃天啟于心,早行其誅,則我將軍匍匐悲號於將軍股掌之上,配等亦袒躬布體以待斧鉞之刑。若必不悛,有以國斃,圖頭不縣,軍不旋踵。願將軍詳度事宜,錫以環玦。」典略曰:譚得書悵然,登城而泣。旣劫於郭圖,亦以兵鋒累交,遂戰不解。}
 尚聞太祖北,釋平原還鄴。其將呂曠、呂翔叛尚歸太祖,譚復陰刻將軍印假曠、翔。太祖知譚詐,與結婚以安之,乃引軍還。尚使審配、蘇由守鄴,復攻譚平原。太祖進軍將攻鄴,到洹水,去鄴五十里,由欲為內應,謀泄,與配戰城中,敗,出奔太祖。太祖遂進攻之,為地道,配亦於內作塹以當之。配將馮禮開突門,內太祖兵三百餘人,配覺之,從城上以大石擊突中柵門,柵門閉,入者皆沒。太祖遂圍之,為塹,周四十里,初令淺,示若可越。配望而笑之,不出爭利。太祖一夜掘之,廣深二丈,決漳水以灌之,自五月至八月,城中餓死者過半。尚聞鄴急,將兵萬餘人還救之,依西山來,東至陽平亭,去鄴十七里,臨滏水,舉火以示城中,城中亦舉火相應。配出兵城北,欲與尚對決圍。太祖逆擊之,敗還,尚亦破走,依曲漳為營,太祖遂圍之。未合,尚懼,遣陰夔、陳琳乞降,不聽。尚還走濫口,進復圍之急,其將馬延等臨陣降,衆大潰,尚奔中山。盡收其輜重,得尚印綬、節鉞及衣物,以示其家,城中崩沮。配兄子榮守東門,夜開門內太祖兵,與配戰城中,生禽配。配聲氣壯烈,終無撓辭,見者莫不歎息。遂斬之。
 \gezhu{先賢行狀曰:配字正南,魏郡人,少忠烈慷慨,有不可犯之節。袁紹領兾州,委以腹心之任,以為治中別駕,并總幕府。初,譚之去,皆呼辛毗、郭圖家得出,而辛評家獨被收。及配兄子開城門內兵,時配在城東南角樓上,望見太祖兵入,忿辛、郭壞敗兾州,乃遣人馳詣鄴獄,指殺仲治家。是時,辛毗在軍,聞門開,馳走詣獄,欲解其兄家,兄家已死。是日生縛配,將詣帳下,辛毗等逆以馬鞭擊其頭,罵之曰:「奴,汝今日真死矣!」配顧曰:「狗輩,正由汝曹破我兾州,恨不得殺汝也!且汝今日能殺生我邪?」有頃,公引見,謂配:「知誰開卿城門?」配曰:「不知也。」曰:「自卿子榮耳。」配曰:「小兒不足用乃至此!」公復謂曰:「曩日孤之行圍,何弩之多也?」配曰:「恨其少耳!」公曰:「卿忠於袁氏父子,亦自不得不爾也。」有意欲活之。配旣無撓辭,而辛毗等號哭不已,乃殺之。初,兾州人張子謙先降,素與配不善,笑謂配曰:「正南,卿竟何如我?」配厲聲曰:「汝為降虜,審配為忠臣,雖死,豈若汝生邪!」臨行刑,叱持兵者令北向,曰:「我君在北。」樂資山陽公載記及袁暐獻帝春秋並云太祖兵入城,審配戰於門中,旣敗,逃于井中,於井獲之。臣松之以為配一代之烈士,袁氏之死臣,豈當數窮之日,方逃身於井,此之難信,誠為易了。不知資、暐之徒竟為何人,未能識別然否,而輕弄翰墨,妄生異端,以行其書。如此之類,正足以誣罔視聽,疑誤後生矣。寔史籍之罪人,達學之所不取者也。}
 高幹以并州降,復以幹為刺史。
 
 
太祖之圍鄴也,譚略取甘陵、安平、勃海、河間,攻尚於中山。尚走故安從熈,譚悉收其衆。太祖將討之,譚乃拔平原,并南皮,自屯龍湊。十二月,太祖軍其門,譚不出,夜遁奔南皮,臨清河而屯。十年正月,攻拔之,斬譚及圖等。熈、尚為其將焦觸、張南所攻,奔遼西烏丸。觸自號幽州刺史,驅率諸郡太守令長,背袁向曹,陳兵數萬,殺白馬盟,令曰:「違命者斬!」衆莫敢語,各以次歃。至別駕韓珩,曰:「吾受袁公父子厚恩,今其破亡,智不能救,勇不能死,於義闕矣;若乃北靣於曹氏,所弗能為也。」一坐為珩失色。觸曰:「夫興大事,當立大義,事之濟否,不待一人,可卒珩志,以勵事君。」高幹叛,執上黨太守,舉兵守壺口關。遣樂進、李典擊之,未拔。十一年,太祖征幹。幹乃留其將夏昭、鄧升守城,自詣匈奴單于求救,不得,獨與數騎亡,欲南奔荊州,上洛都尉捕斬之。
 \gezhu{典略曰:上洛都尉王琰獲高幹,以功封侯;其妻哭於室,以為琰富貴將更娶妾媵而奪己愛故也。}
 十二年,太祖至遼西擊烏丸。尚、熈與烏丸逆軍戰,敗走奔遼東,公孫康誘斬之,送其首。
 \gezhu{典略曰:尚為人有勇力,欲奪取康衆,與熈謀曰:「今到,康必相見,欲與兄手擊之,有遼東猶可以自廣也。」康亦心計曰:「今不取熈、尚,無以為說於國家。」乃先置其精勇於廄中,然後請熈、尚。熈、尚入,康伏兵出,皆縛之,坐於凍地。尚寒,求席,熈曰:「頭顱方行萬里,何席之為!」遂斬首。譚,字顯思。熈,字顯弈。尚,字顯甫。吳書曰:尚有弟名買,與尚俱走遼東。曹瞞傳云:買,尚兄子。未詳。}
 太祖高韓珩節,屢辟不至,卒於家。
 \gezhu{先賢行狀曰:珩字子佩,代郡人,清粹有雅量。少喪父母,奉養兄姊,宗族稱孝悌焉。}
 
 
\end{pinyinscope}