\article{陳武傳}
\begin{pinyinscope}
 
 
 陳武字子烈,廬江松滋人。孫策在壽春,武往脩謁,時年十八,長七尺七寸,因從渡江,征討有功,拜別部司馬。策破劉勳,多得廬江人,料其精銳,乃以武為督,所向無前。及權統事,轉督五校。仁厚好施,鄉里遠方客多依託之。尤為權所親愛,數至其家。累有功勞,進位偏將軍。建安二十年,從擊合肥,奮命戰死。權哀之,自臨其葬。
 
 
\gezhu{江表傳曰:權命以其愛妾殉葬,復客二百家。孫盛曰:昔三良從秦穆師以之不征;魏妾旣出,杜回以之僵仆。禍福之報,如此之效也。權仗計任術,以生從死,世祚之促,不亦宜乎!}
 
 
 
 
 子脩有武風,年十九,權召見獎厲,拜別部司馬,授兵五百人。時諸新兵多有逃叛,而脩撫循得意,不失一人。權奇之,拜為校尉。建安末,追錄功臣後,封脩都亭侯,為解煩督。黃龍元年卒。
 
 
 
 
 弟表,字文奧,武庶子也,少知名,與諸葛恪、顧譚、張休等並侍東宮,皆共親友。尚書曁豔亦與表善,後豔遇罪,時人咸自營護,信厚言薄,表獨不然,士以此重之。徙太子中庶子,拜翼正都尉。兄脩亡後,表母不肯事脩母,表謂其母曰:「兄不幸早亡,表統家事,當奉嫡母。母若能為表屈情,承順嫡母者,是至願也;若母不能,直當出別居耳。」表於大義公正如此。由是二母感寤雍穆。表以父死敵場,求用為將,領兵五百人。表欲得戰士之力,傾意接待,士皆愛附,樂為用命。時有盜官物者,疑無難士施明。明素壯悍,收考極毒,惟死無辭,廷尉以聞。權以表能得健兒之心,詔以明付表,使自以意求其情實。表便破械沐浴,易其衣服,厚設酒食,歡以誘之。明乃首服,具列支黨。表以狀聞。權奇之,欲全其名,特為赦明,誅戮其黨。遷表為無難右部督,封都亭侯,以繼舊爵。表皆陳讓,乞以傳脩子延,權不許。嘉禾三年,諸葛恪領丹楊太守,討平山越,以表領新安都尉,與恪參勢。初,表所受賜復人得二百家,在會稽新安縣。表簡視其人,皆堪好兵,乃上疏陳讓,乞以還官,充足精銳。詔曰:「先將軍有功於國,國家以此報之,卿何得辭焉?」表乃稱曰:「今除國賊,報父之仇,以人為本。空枉此勁銳以為僮僕,非表志也。」皆輒料取以充部伍。所在以聞,權甚嘉之。下郡縣,料正戶羸民以補其處。表在官三年,廣開降納,得兵萬餘人。事捷當出,會鄱陽民吳遽等為亂,攻沒城郭,屬縣搖動,表便越界赴討,遽以破敗,遂降。陸遜拜表偏將軍,進封都鄉侯,北屯章阬。年三十四卒。家財盡於養士,死之日,妻子露立,太子登為起屋宅。子敖年十七,拜別部司馬,授兵四百人。敖卒,脩子延復為司馬代敖。延弟永,將軍,封侯。始施明感表,自變行為善,遂成健將,致位將軍。
 
 
\end{pinyinscope}