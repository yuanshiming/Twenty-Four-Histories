\article{費禕傳}
\begin{pinyinscope}
 
 
 費禕字文偉,江夏鄳人也。
 
 
\gezhu{鄳音盲。}
 少孤,依族父伯仁。伯仁姑,益州牧劉璋之母也。璋遣使迎仁,仁將禕遊學入蜀。會先主定蜀,禕遂留益土,與汝南許叔龍、南郡董允齊名。時許靖喪子,允與禕欲共會其葬所。允白父和請車,和遣開後鹿車給之。允有難載之色,禕便從前先上。及至喪所,諸葛亮及諸貴人悉集,車乘甚鮮,允猶神色未泰,而禕晏然自若。持車人還,和問之,知其如此,乃謂允曰:「吾常疑汝於文偉優劣未別也,而今而後,吾意了矣。」
 
 
先主立太子,禕與允俱為舍人,遷庶子。後主踐位,為黃門侍郎。丞相亮南征還,羣寮於數十里逢迎,年位多在禕右,而亮特命禕同載,由是衆人莫不易觀。亮以初從南歸,以禕為昭信校尉使吳。孫權性旣滑稽,嘲啁無方,諸葛恪、羊衜等才愽果辯,論難鋒至,禕辭順義篤,據理以荅,終不能屈。
 \gezhu{禕別傳曰:孫權每別酌好酒以飲禕,視其已醉,然後問以國事,并論當世之務,辭難累至。禕輙辭以醉,退而撰次所問,事事條荅,無所遺失。}
 權甚器之,謂禕曰:「君天下淑德,必當股肱蜀朝,恐不能數來也。」
 \gezhu{禕別傳曰:權乃以手中常所執寶刀贈之,禕荅曰:「臣以不才,何以堪明命?然刀所以討不庭、禁暴亂者也,但願大王勉建功業,同獎漢室,臣雖闇弱,終不負東顧。」}
 還,遷為侍中。亮北住漢中,請禕為參軍。以奉使稱旨,頻煩至吳。建興八年,轉為中護軍,後又為司馬。值軍師魏延與長史楊儀相憎惡,每至並坐爭論,延或舉刃擬儀,儀泣涕橫集。禕常入其坐間,諫喻分別,終亮之世,各盡延、儀之用者,禕匡救之力也。亮卒,禕為後軍師。頃之,代蔣琬為尚書令。
 \gezhu{禕別傳曰:于時戰國多事,公務煩猥,禕識悟過人,每省讀書記,舉目暫視,已究其意旨,其速數倍於人,終亦不忘。常以朝晡聽事,其間接納賔客,飲食嬉戲,加之博弈,每盡人之歡,事亦不廢。董允代禕為尚書令,欲斆禕之所行,旬日之中,事多愆滯。允乃歎曰:「人才力相縣若此甚遠,此非吾之所及也。聽事終日,猶有不暇爾。」}
 琬自漢中還涪,禕遷大將軍,錄尚書事。
 
 
延熈七年,魏軍次于興勢,假禕節,率衆往禦之。光祿大夫來敏至禕許別,求共圍棊。于時羽檄交馳。人馬擐甲,嚴駕已訖,禕與敏留意對戲,色無厭倦。敏曰:「向聊觀試君耳!君信可人,必能辨賊者也。」禕至,敵遂退,封成鄉侯。
 \gezhu{殷基通語曰:司馬懿誅曹爽,禕設甲乙論平其是非。甲以為曹爽兄弟凡品庸人,苟以宗子枝屬,得蒙顧命之任,而驕奢僭逸,交非其人,私樹朋黨,謀以亂國。懿奮誅討,一朝殄盡,此所以稱其任,副士民之望也。乙以為懿感曹仲付己不一,豈爽與相干?事勢不專,以此陰成疵瑕。初無忠告侃爾之訓,一朝屠戮,讒其不意,豈大人經國篤本之事乎!若爽信有謀主之心,大逆已搆,而發兵之日,更以芳委爽兄弟。懿父子從後閉門舉兵,蹙而向芳,必無悉寧,忠臣為君深慮之謂乎?以此推之,爽無大惡明矣。若懿以爽奢僭,廢之刑之可也,滅其尺口,被以不義,絕子丹血食,及何晏子魏之親甥,亦與同戮,為僭濫不當矣。}
 琬固讓州職,禕復領益州刺史。禕當國功名,略與琬比。
 \gezhu{禕別傳曰:禕雅性謙素,家不積財。兒子皆令布衣素食,出入不從車騎,無異凡人。}
 十一年,出住漢中。自琬及禕,雖自身在外,慶賞威刑,皆遙先諮斷,然後乃行,其推任如此。後十四年夏,還成都,成都望氣者云都邑無宰相位,故冬復北屯漢壽。延熈十五年,命禕開府。十六年歲首大會,魏降人郭循在坐。禕歡飲沈醉,為循手刃所害,謚曰敬侯。子承嗣,為黃門侍郎。承弟恭,尚公主。
 \gezhu{禕別傳曰:恭為尚書郎,顯名當世,早卒。}
 禕長女配太子璿為妃。
 
 
\end{pinyinscope}