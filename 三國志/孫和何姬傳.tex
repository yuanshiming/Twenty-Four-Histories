\article{孫和何姬傳}
\begin{pinyinscope}
 
 
 孫和何姬,丹楊句容人也。父遂,本騎士。孫權甞游幸諸營,而姬觀於道中,權望見異之,命宦者召入,以賜子和。生男,權喜,名之曰彭祖,即皓也。太子和旣廢,後為南陽王,居長沙。孫亮即位,孫峻輔政。峻素媚事全主,全主與和母有隙,遂勸峻徙和居新都,遣使賜死,嫡妃張氏亦自殺。何姬曰:「若皆從死,誰當養孤?」遂拊育皓,及其三弟。皓即位,尊和為昭獻皇帝,
 
 
\gezhu{吳錄曰:皓初尊和為昭獻皇帝,俄改曰文皇帝。}
 何姬為昭獻皇后,稱升平宮,月餘,進為皇太后。封弟洪永平侯,蔣溧陽侯,植宣城侯。洪卒,子邈嗣,為武陵監軍,為晉所殺。植官至大司徒。吳末昬亂,何氏驕僭,子弟橫放,百姓患之。故民譌言「皓乆死,立者何氏子」云。
 \gezhu{江表傳曰:皓以張布女為美人,有寵,皓問曰:「汝父所在?」荅曰:「賊以殺之。」皓大怒,棒殺之。後思其顏色,使巧工刻木作美人形象,恒置座側。問左右:「布復有女否?」荅曰:「布大女適故衞尉馮朝子純。」即奪純妻入宮,大有寵,拜為左夫人,晝夜與夫人房宴,不聽朝政,使尚方以金作華燧、步搖、假髻以千數。令宮人著以相撲,朝成夕敗,輒出更作,工匠因緣偷盜,府藏為空。會夫人死,皓哀愍思念,葬于苑中,大作冢,使工匠刻柏作木人,內冢中以為兵衞,以金銀珍玩之物送葬,不可稱計。已葬之後,皓治喪於內,半年不出。國人見葬太奢麗,皆謂皓已死,所葬者是也。皓舅子何都顏狀似皓,云都代立。臨海太守奚熈信譌言,舉兵欲還秣陵誅都,都叔父植時為備海督,擊殺熈,夷三族,譌言乃息,而人心猶疑。}
 
 
\end{pinyinscope}