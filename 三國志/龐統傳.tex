\article{龐統傳}
\begin{pinyinscope}
 
 
 龐統字士元,襄陽人也。少時樸鈍,未有識者。潁川司馬徽清雅有知人鑒,統弱冠往見徽,徽採桑於樹上,坐統在樹下,共語自晝至夜。徽甚異之,稱統當為南州士之冠冕,由是漸顯。
 
 
\gezhu{襄陽記曰:諸葛孔明為卧龍,龐士元為鳳雛,司馬德操為水鏡,皆龐德公語也。德公,襄陽人。孔明每至其家,獨拜牀下,德公初不令止。德操甞造德公,值其渡沔,上祀先人墓,德操徑入其室,呼德公妻子,使速作黍,「徐元直向云有客當來就我與龐公譚。」其妻子皆羅列拜於堂下,奔走供設。須臾,德公還,直入相就,不知何者是客也。德操年小德公十歲,兄事之,呼作龐公,故世人遂謂龐公是德公名,非也。德公子山民,亦有令名,娶諸葛孔明小姊,為魏黃門吏部郎,早卒。子渙,字世文,晉太康中為牂牁太守。統,德公從子也,少未有識者,惟德公重之,年十八,使往見德操。德操與語,旣而歎曰:「德公誠知人,此實盛德也。」}
 
 
 
 
 後郡命為功曹。性好人倫,勤於長養。每所稱述,多過其才,時人怪而問之,統荅曰:「當今天下大亂,雅道陵遲,善人少而惡人多。方欲興風俗,長道業,不美其譚即聲名不足慕企,不足慕企而為善者少矣。今拔十失五,猶得其半,而可以崇邁世教,使有志者自勵,不亦可乎?」
 
 
吳將周瑜助先主取荊州,因領南郡太守。瑜卒,統送喪至吳,吳人多聞其名。及當西還,並會昌門,陸績、顧劭、全琮皆往。統曰:「陸子可謂駑馬有逸足之力,顧子可謂駑牛能負重致遠也。」
 \gezhu{張勃吳錄曰:或問統曰:「如所目,陸子為勝乎?」統曰:「駑馬雖精,所致一人耳。駑牛一日行三百里,所致豈一人之重哉!」劭就統宿,語,因問:「卿名知人,吾與卿孰愈?」統曰:「陶冶世俗,甄綜人物,吾不及卿;論帝王之秘策,攬倚伏之要最,吾似有一日之長。」劭安其言而親之。}
 謂全琮曰:「卿好施慕名,有似汝南樊子昭。
 \gezhu{蔣濟萬機論云許子將襃貶不平,以拔樊子昭而抑許文休。劉曄曰:「子昭拔自賈豎,年至耳順,退能守靜,進能不苟。」濟荅曰:「子昭誠自長幼皃潔,然觀其臿齒牙,樹頰胲,吐唇吻,自非文休敵也。」胲音改。}
 雖智力不多,亦一時之佳也。」績、劭謂統曰:「使天下太平,當與卿共料四海之士。」深與統相結而還。
 
 
先主領荊州,統以從事守耒陽令,在縣不治,免官。吳將魯肅遺先主書曰:「龐士元非百里才也,使處治中、別駕之任,始當展其驥足耳。」諸葛亮亦言之於先主,先主見與善譚,大器之,以為治中從事。
 \gezhu{江表傳曰:先主與統從容宴語,問曰:「卿為周公瑾功曹,孤到吳,聞此人密有白事,勸仲謀相留,有之乎?在君為君,卿其無隱。」統對曰:「有之。」備歎息曰:「孤時危急,當有所求,故不得不往,殆不免周瑜之手!天下智謀之士,所見略同耳。時孔明諫孤莫行,其意獨篤,亦慮此也。孤以仲謀所防在北,當賴孤為援,故決意不疑。此誠出於險塗,非萬全之計也。」}
 親待亞於諸葛亮,遂與亮並為軍師中郎將。
 \gezhu{九州春秋曰:統說備曰:「荊州荒殘,人物殫盡,東有吳孫,北有曹氏,鼎足之計,難以得志。今益州國富民彊,戶口百萬,四部兵馬,所出必具,寶貨無求於外,今可權借以定大事。」備曰:「今指與吾為水火者,曹操也,操以急,吾以寬;操以暴,吾以仁;操以譎,吾以忠;每與操反,事乃可成耳。今以小故而失信義於天下者,吾所不取也。」統曰:「權變之時,固非一道所能定也。兼弱攻昧,五伯之事。逆取順守,報之以義,事定之後,封以大國,何負於信?今日不取,終為人利耳。」備遂行。}
 亮留鎮荊州。統隨從入蜀。
 
 
益州牧劉璋與先主會涪,統進策曰:「今因此會,便可執之,則將軍無用兵之勞,而坐定一州也。」先主曰:「初入他國,恩信未著,此不可也。」璋旣還成都,先主當為璋北征漢中,統復說曰:「陰選精兵,晝夜兼道,徑襲成都;璋旣不武,又素無預備,大軍卒至,一舉便定,此上計也。楊懷、高沛,璋之名將,各杖彊兵,據守關頭,聞數有牋諫璋,使發遣將軍還荊州。將軍未至,遣與相聞,說荊州有急,欲還救之,並使裝束,外作歸形;此二子旣服將軍英名,又喜將軍之去,計必乘輕騎來見,將軍因此執之,進取其兵,乃向成都,此中計也。退還白帝,連引荊州,徐還圖之,此下計也。若沈吟不去,將致大困,不可乆矣。」先主然其中計,即斬懷、沛,還向成都,所過輒克。於涪大會,置酒作樂,謂統曰:「今日之會,可謂樂矣。」統曰:「伐人之國而以為歡,非仁者之兵也。」先主醉,怒曰:「武王伐紂,前歌後舞,非仁者邪?卿言不當,宜速起出!」於是統逡巡引退。先主尋悔,請還。統復故位,初不顧謝,飲食自若。先主謂曰:「向者之論,阿誰為失?」統對曰:「君臣俱失。」先主大笑,宴樂如初。
 \gezhu{習鑿齒曰:夫霸王者,必體仁義以為本,杖信順以為宗,一物不具,則其道乖矣。今劉備襲奪璋土,權以濟業,負信違情,德義俱愆,雖功由是隆,宜大傷其敗,譬斷手全軀,何樂之有?龐統懼斯言之泄宣,知其君之必悟,故衆中匡其失,而不脩常謙之道,矯然太當,盡其謇諤之風。夫上失而能正,是有臣也,納勝而無執,是從理也;有臣則陛隆堂高,從理則羣策畢舉;一言而三善兼明,暫諫而義彰百代,可謂達乎大體矣。若惜其小失而廢其大益,矜此過言,自絕遠讜,能成業濟務者,未之有也。臣松之以為謀襲劉璋,計雖出於統,然違義成功,本由詭道,心旣內疚,則歡情自戢,故聞備稱樂之言,不覺率爾而對也。備宴酣失時,事同樂禍,自比武王,曾無愧色,此備有非而統無失,其云「君臣俱失」,蓋分謗之言耳。習氏所論,雖大旨無乖,然推演之辭,近為流宕也。}
 
 
進圍雒縣,統率衆攻城,為流矢所中,卒,時年三十六。先主痛惜,言則流涕。拜統父議郎,遷諫議大夫,諸葛亮親為之拜。追賜統爵關內侯,謚曰靖侯。統子宏,字巨師,剛簡有臧否,輕傲尚書令陳袛,為袛所抑,卒於涪陵太守。統弟林,以荊州治中從事參鎮北將軍黃權征吳,值軍敗,隨權入魏,魏封列侯,至鉅鹿太守。
 \gezhu{襄陽記云:林婦,同郡習禎姉。禎事在楊戲輔臣贊。曹公之破荊州,林婦與林分隔,守養弱女十有餘年,後林隨黃權降魏,始復集聚。魏文帝聞而賢之,賜牀帳衣服,以顯其義節。}
 
 
\end{pinyinscope}