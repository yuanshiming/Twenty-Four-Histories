\article{徐盛傳}
\begin{pinyinscope}
 
 
 徐盛字文嚮,琅邪莒人也。遭亂,客居吳,以勇氣聞。孫權統事,以為別部司馬,授兵五百人,守柴桑長,拒黃祖。祖子射,甞率數千人下攻盛。盛時吏士不滿二百,與相拒擊,傷射吏士千餘人。已乃開門出戰,大破之。射遂絕迹不復為寇。權以為校尉、蕪湖令。復討臨城南阿山賊有功,徙中郎將,督校兵。
 
 
 
 
 曹公出濡須,從權禦之。魏甞大出橫江,盛與諸將俱赴討。時乘蒙衝,遇迅風,船落敵岸下,諸將恐懼,未有出者,盛獨將兵,上突斫敵,敵披退走,有所傷殺,風止便還,權大壯之。
 
 
 
 
 及權為魏稱藩,魏使邢貞拜權為吳王。權出都亭候貞,貞有驕色,張昭旣怒,而盛忿憤,顧謂同列曰:「盛等不能奮身出命,為國家并許洛,吞巴蜀,而令吾君與貞盟,不亦辱乎!」因涕泣橫流。貞聞之,謂其旅曰:「江東將相如此,非乆下人者也。」
 
 
 
 
 後遷建武將軍,封都亭侯,領廬江太守,賜臨城縣為奉邑。劉備次西陵,盛攻取諸屯,所向有功。曹休出洞口,盛與呂範、全琮渡江拒守。遭大風,船人多喪,盛收餘兵,與休夾江。休使兵將就船攻盛,盛以少禦多,敵不能克,各引軍退。遷安東將軍,封蕪湖侯。
 
 
 
 
 後魏文帝大出,有渡江之志,盛建計從建業築圍,作薄落,圍上設假樓,江中浮船。諸將以為無益,盛不聽,固立之。文帝到廣陵,望圍愕然,彌漫數百里,而江水盛長,便引軍退。諸將乃伏。
 
 
\gezhu{干寶晉紀所云疑城,已注孫權傳。魏氏春秋云:文帝歎曰:「魏雖有武騎千羣,無所用也。」}
 
 
 
 
 黃武中卒。子楷,襲爵領兵。
 
 
\end{pinyinscope}