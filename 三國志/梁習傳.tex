\article{梁習傳}
\begin{pinyinscope}
 
 
 梁習字子虞,陳郡柘人也,為郡綱紀。太祖為司空,辟召為漳長,累轉乘氏、海西、下邳令,所在有治。還為西曹令史,遷為屬。并土新附,習以別部司馬領并州刺史。時承高幹荒亂之餘,胡狄在界,張雄跋扈,吏民亡叛,入其部落;兵家擁衆,作為寇害,更相扇動,往往棊跱。習到官,誘喻招納,皆禮召其豪右,稍稍薦舉,使詣幕府;豪右已盡,乃次發諸丁彊以為義從;又因大軍出征,分請以為勇力。吏兵已去之後,稍移其家,前後送鄴,凡數萬口;其不從命者,興兵致討,斬首千數,降附者萬計。單于恭順,名王稽顙,部曲服事供職,同於編戶。邊境肅清,百姓布野,勤勸農桑,令行禁止。貢達名士,咸顯於世,語在常林傳。太祖嘉之,賜爵關內侯,更拜為真。長老稱詠,以為自所聞識,刺史未有及習者。建安十八年,州并屬兾州,更拜議郎、西部都督從事,統屬兾州,總故部曲。又使於上黨取大材供鄴宮室。習表置屯田都尉二人,領客六百夫,於道次耕種菽粟,以給人牛之費。後單于入侍,西北無虞,習之績也。
 
 
\gezhu{魏略曰:鮮卑大人育延,常為州所畏,而一旦將其部落五千餘騎詣習,求互市。習念不聽則恐其怨,若聽到州下,又恐為所略,於是乃許之往與會空城中交市。遂勑郡縣,自將治中以下軍往就之。市易未畢,市吏收縛二胡。延騎皆驚,上馬彎弓圍習數重,吏民惶怖不知所施。習乃徐呼市吏,問縛胡意,而胡實侵犯人。習乃使譯呼延,延到,習責延曰:「汝胡自犯法,吏不侵汝,汝何為使諸騎驚駭邪?」遂斬之,餘胡破膽不敢動。是後無寇虜。至二十二年,太祖拔漢中,諸軍還到長安,因留騎督太原烏丸王魯昔,使屯池陽,以備盧水。昔有愛妻,住在晉陽。昔旣思之,又恐遂不得歸,乃以其部五百騎叛還并州,留其餘騎置山谷間,而單騎獨入晉陽,盜取其妻。已出城,州郡乃覺;吏民又畏昔善射,不敢追。習乃令從事張景,募鮮卑使逐昔。昔馬負其妻,重騎行遲,未及與其衆合,而為鮮卑所射死。始太祖聞昔叛,恐其為亂於北邊;會聞已殺之,大喜,以習前後有策略,封為關內侯。}
 文帝踐阼,復置并州,復為刺史,進封申門亭侯,邑百戶;政治常為天下最。太和二年,徵拜大司農。習在州二十餘年,而居處貧窮,無方面珍物,明帝異之,禮賜甚厚。四年,薨,子施嗣。
 
 
初,濟陰王思與習俱為西曹令史。思因直日白事,失太祖指。太祖大怒,教召主者,將加重辟。時思近出,習代往對,已被收執矣,思乃馳還,自陳己罪,罪應受死。太祖歎習之不言、思之識分,曰:「何意吾軍中有二義士乎?」
 \gezhu{臣松之以為習與王思,同寮而已,親非骨肉,義非刎頸,而以身代思,受不測之禍。以之為義,無乃乖先哲之雅旨乎!史遷云:「死有重於太山,有輕於鴻毛」,故君子不為苟存,不為苟亡。若使思不引分,主不加恕,則所謂自經於溝瀆而莫之知也。習之死義者,豈其然哉!}
 後同時擢為刺史,思領豫州。思亦能吏,然苛碎無大體,官至九卿,封列侯。
 \gezhu{魏略苛吏傳曰:思與薛悌、郤嘉俱從微起,官位略等。三人中,悌差挾儒術,所在名為閑省。嘉與思事行相似。文帝詔曰:「薛悌駁吏,王思、郤嘉純吏也,各賜關內侯,以報其勤。」思為人雖煩碎,而曉練文書,敬賢禮士,傾意形勢,亦以是顯名。正始中,為大司農,年老目瞑,瞋怒無度,下吏嗷然不知何據。性少信,時有吏父病篤,近在外舍,自白求假。思疑其不實,發怒曰:「世有思婦病母者,豈此謂乎!」遂不與假。吏父明日死,思無恨意。其為刻薄類如此。思又性急,甞執筆作書,蠅集筆端,驅去復來,如是再三。思恚怒,自起逐蠅不能得,還取筆擲地,蹋壞之。時有丹陽施畏、魯郡倪覬、南陽胡業亦為刺史、郡守,時人謂之苛暴。又有高陽劉類,歷位宰守,苛慝尤其,以善脩人事,不廢於世。嘉平中,為弘農太守。吏二百餘人,不與休假,專使為不急。過無輕重,輒捽其頭,又亂杖撾之,牽出復入,如是數四。乃使人掘地求錢,所在市里,皆有孔穴。又外託簡省,每出行,陽勑督郵不得使官屬曲脩禮敬,而陰識不來者,輙發怒中傷之。性又少信,每遣大吏出,輒使小吏隨覆察之,白日常自於牆壁閒闚閃,夜使幹廉察諸曹,復以幹不足信,又遣鈴下及奴婢使轉相檢驗。嘗案行,宿止民家。民家二狗逐豬,豬驚走,頭插柵間,號呼良乆。類以為外之吏擅共飲食,不復徵察,便使伍伯曳五官掾孫弼入,頓頭責之。弼以實對,類自愧不詳,因託問以他事。民尹昌,年垂百歲,聞類出行,當經過,謂其兒曰:「扶我迎府君,我欲陳恩。」兒扶昌在道左,類望見,呵其兒曰:「用是死人,使來見我。」其視人無禮,皆此類也。舊俗,民謗官長者有三不肯,謂遷、免與死也。類在弘農,吏民患之,乃題其門曰:「劉府君有三不肯。」類雖聞之,猶不能自改。其後安東將軍司馬文王西征,路經弘農,弘農人告類荒耄不任宰郡,乃召入為五官中郎將。}
 
 
\end{pinyinscope}