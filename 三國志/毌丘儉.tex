\article{毌丘儉}
\begin{pinyinscope}
 
 
 毌丘儉字仲恭,河東聞喜人也。父興,黃初中為武威太守,伐叛柔服,開通河右,名次金城太守蘇則。討賊張進及討叛胡有功,封高陽鄉侯。
 
 
\gezhu{魏名臣奏載雍州刺史張旣表曰:「河右遐遠,喪亂彌乆,武威當諸郡路道喉轄之要,加民夷雜處,數有兵難。領太守毌丘興到官,內撫吏民,外懷羌、胡,卒使柔附,為官効用。黃華、張進初圖逆亂,扇動左右,興志氣忠烈,臨難不顧,為將校民夷陳說禍福,言則涕泣。於時男女萬口咸懷感激,形毀髮亂,誓心致命。尋率精兵踧脅張掖,濟拔領太守杜通、西海太守張睦。張掖番和、驪靬二縣吏民及郡雜胡棄惡詣興,興皆安恤,使盡力田。興每所歷,盡竭心力,誠國之良吏。殿下即位,留心萬機,苟有毫毛之善,必有賞錄,臣伏緣聖旨,指陳其事。」}
 入為將作大匠。儉襲父爵,為平原侯文學。明帝即位,為尚書郎,遷羽林監。以東宮之舊,甚見親待。出為洛陽典農。時取農民以治宮室,儉上疏曰:「臣愚以為天下所急除者二賊,所急務者衣食。誠使二賊不滅,士民飢凍,雖崇美宮室,猶無益也。」遷荊州刺史。
 
 
 
 
 青龍中,帝圖討遼東,以儉有幹策,徙為幽州刺史,加渡遼將軍,使持節,護烏丸校尉。率幽州諸軍至襄平,屯遼隧。右北平烏丸單于寇婁敦、遼西烏丸都督率衆王護留等,昔隨袁尚奔遼東者,率衆五千餘人降。寇婁敦遣弟阿羅槃等詣闕朝貢,封其渠率二十餘人為侯、王,賜輿馬繒綵各有差。公孫淵逆與儉戰,不利,引還。明年,帝遣太尉司馬宣王統中軍及儉等衆數萬討淵,定遼東。儉以功進封安邑侯,食邑三千九百戶。
 
 
正始中,儉以高句驪數侵叛,督諸軍步騎萬人出玄菟,從諸道討之。句驪王宮將步騎二萬人,進軍沸流水上,大戰梁口,
 \gezhu{梁音渴。}
 宮軍破走。儉遂束馬縣車,以登丸都,屠句驪所都,斬獲首虜以千數。句驪沛者名得來,數諫宮,
 \gezhu{臣松之案東夷傳:沛者,句驪國之官名。}
 宮不從其言。得來歎曰:「立見此地將生蓬蒿。」遂不食而死,舉國賢之。儉令諸軍不壞其墓,不伐其樹,得其妻子,皆放遣之。宮單將妻子逃竄。儉引軍還。六年,復征之,宮遂奔買溝。儉遣玄菟太守王頎追之,
 \gezhu{世語曰:頎字孔碩,東萊人,晉永嘉中大賊王弥,頎之孫。}
 過沃沮千有餘里,至肅慎氏南界,刻石紀功,刊丸都之山,銘不耐之城。諸所誅納八千餘口,論功受賞,侯者百餘人。穿山溉灌,民賴其利。
 
 
 
 
 遷左將軍,假節監豫州諸軍事,領豫州刺史,轉為鎮南將軍。諸葛誕戰於東關,不利,乃令誕、儉對換。誕為鎮南,都督豫州。儉為鎮東,都督楊州。吳太傅諸葛恪圍合肥新城,儉與文欽禦之,太尉司馬孚督中軍東解圍,恪退還。
 
 
初,儉與夏侯玄、李豊等厚善。揚州刺史前將軍文欽,曹爽之邑人也,驍果麄猛,數有戰功,好增虜獲,以徼寵賞,多不見許,怨恨日甚。儉以計厚待欽,情好歡洽。欽亦感戴,投心無二。正元二年正月,有彗星數十丈,西北竟天,起於吴、楚之分。儉、欽喜,以為己祥。遂矯太后詔,罪狀大將軍司馬景王,移諸郡國,舉兵反。迫脅淮南將守諸別屯者,及吏民大小,皆入壽春城,為壇於城西,歃血稱兵為盟,分老弱守城,儉、欽自將五六萬衆渡淮,西至項。儉堅守,欽在外為游兵。
 \gezhu{儉、欽等表曰:「故相國懿,匡輔魏室,歷事忠貞,故烈祖明皇帝授以寄託之任。懿勠力盡節,以寧華夏。又以齊王聦明,無有穢德,乃心勤盡忠以輔上,天下賴之。懿欲討滅二虜以安宇內,始分軍糧,克時同舉,未成而薨。齊王以懿有輔己大功,故遂使師承統懿業,委以大事。而師以盛年在職,無疾託病,坐擁彊兵,無有臣禮,朝臣非之,義士譏之,天下所聞,其罪一也。懿造計取賊,多舂軍糧,克期有日。師為大臣,當除國難,又為人子,當卒父業。哀聲未絕而便罷息,為臣不忠,為子不孝,其罪二也。賊退過東關,坐自起衆,三征同進,喪衆敗績,歷年軍實,一旦而盡,致使賊來,天下騷動,死傷流離,其罪三也。賊舉國悉衆,號五十萬,來向壽春,圖詣洛陽,會太尉孚與臣等建計,乃杜塞要險,不與爭鋒,還固新城。淮南將士,衝鋒履刃,晝夜相守,勤瘁百日,死者塗地,自魏有軍已來,為難苦甚,莫過於此。而師遂意自由,不論封賞,權勢自在,無所領錄,其罪四也。故中書令李豊等,以師無人臣節,欲議退之。師知而請豊,其夕拉殺,載尸埋棺。豊等為大臣,帝王腹心,擅加酷暴,死無罪名,師有無君之心,其罪五也。懿每歎說齊王自堪人主,君臣之義定。奉事以來十有五載,始欲歸政,按行武庫,詔問禁兵不得妄出。師自知姦慝,人神所不祐,矯廢君主,加之以罪。孚,師之叔父,性甚仁孝,追送齊王,悲不自勝。群臣皆怒而師懷忍,不顧大義,其罪六也。又故光祿大夫張緝,無罪而誅,夷其妻子,并及母后,逼恐至尊,彊催督遣,臨時哀愕,莫不傷痛;而師稱慶,反以歡喜,其罪七也。陛下踐阼,聦明神武,事經聖心,欲崇省約,天下聞之,莫不歡慶;而師不自改悔,脩復臣禮,而方徵兵募士,毀壞宮內,列侯自衞。陛下即阼,初不朝覲。陛下欲臨幸師舍以省其疾,復拒不通,不奉法度,其罪八也。近者領軍許允當為鎮北,以厨錢給賜,而師舉奏加辟,雖云流徙,道路餓殺,天下聞之,莫不哀傷,其罪九也。三方之守,一朝闕廢,多選精兵,以自營衞,五營領兵,闕而不補,多載器杖,充聚本營,天下所聞,人懷憤怨,譌言盈路,以疑海內,其罪十也。多休守兵,以占高第,以空虛四表,欲擅彊勢,以逞姦心,募取屯田,加其復賞,阻兵安忍,壞亂舊法。合聚諸藩王公以著鄴,欲悉誅之,一旦舉事廢主。天不長惡,使目腫不成,其罪十一也。臣等先人皆隨從太祖武皇帝征討凶暴,獲成大功,與高祖文皇帝即受漢禪,開國承家,猶堯舜相傳也。臣與安豐護軍鄭翼、廬江護軍呂宣、太守張休、淮南太守丁尊、督守合肥護軍王休等議,各以累世受恩,千載風塵,思尽軀命,以完全社稷安主為効。斯義苟立,雖焚妻子,吞炭漆身,死而不恨也。按師之罪,宜加大辟,以彰姦慝。春秋之義,一世為善,十世宥之。懿有大功,海內所書,依古典議,廢師以侯就第。弟昭,忠肅寬明,樂善好士,有高世君子之度,忠誠為國,不與師同。臣等碎首所保,可以代師輔導聖躬。太尉孚,忠孝小心,所宜親寵,授以保傅。護軍散騎常侍望,忠公親事,當官稱能,奉迎乘輿,有宿衞之舊,可為中領軍。春秋之義,大義滅親,故周公誅弟,石碏戮子,季友鴆兄,上為國計,下全宗族。殛鯀用禹,聖人明典,古今所稱。乞陛下下臣等所奏,朝堂博議。臣言當道,使師遜位避賢者,罷兵去備,如三皇舊法,則天下協同。若師負勢恃衆不自退者,臣等率將所領,晝夜兼行,惟命是授。臣等今日所奏,惟欲使大魏永存,使陛下得行君意,遠絕亡之禍,百姓安全,六合一躰,使忠臣義士不愧於三皇五帝耳。臣恐兵起,天下擾亂,臣輙上事,移三征及州郡國典農,各安慰所部吏民,不得妄動,謹具以狀聞。惟陛下愛養精神,明慮危害,以寧海內。師專權用勢,賞罰自由,聞臣等舉衆,必下詔禁絕關津,使驛書不通,擅復徵調,有所收捕。此乃師詔,非陛下詔書,在所皆不得復承用。臣等道遠,懼文書不得皆通,輙臨時賞罰,以便宜從事,須定集上也。」}
 
 
大將軍統中外軍討之,別使諸葛誕督豫州諸軍從安風津擬壽春,征東將軍胡遵督青、徐諸軍出於譙、宋之間,絕其歸路。大將軍屯汝陽,使監軍王基督前鋒諸軍據南頓以待之。令諸軍皆堅壁勿與戰。儉、欽進不得鬬,退恐壽春見襲,不得歸,計窮不知所為。淮南將士,家皆在北,衆心沮散,降者相屬,惟淮南新附農民為之用。大將軍遣兖州刺史鄧艾督泰山諸軍萬餘人至樂嘉,示弱以誘之,大將軍尋自洙至。欽不知,果夜來欲襲艾等,會明,見大軍兵馬盛,乃引還。
 \gezhu{魏氏春秋曰:欽中子俶,小名鴦。年尚幼,勇力絕人,謂欽曰:「及其未定,擊之可破也。」於是分為二隊,夜夾攻軍。俶率壯士先至,大呼大將軍,軍中震擾。欽後期不應。會明,俶退,欽亦引還。魏末傳曰:殿中人姓尹,字大目,小為曹氏家奴,常侍在帝側,大將軍將俱行。大目知大將軍一目已突出,啟云:「文欽本是明公腹心,但為人所誤耳,又天子鄉里。大目昔為文欽所信,乞得追解語之,令還與公復好。」大將軍聽遣大目單身往,乘大馬,被鎧冑,追文欽,遙相與語。大目心實欲曹氏安,謬言:「君侯何若若不可復忍數日中也!」欲使欽解其旨。欽殊不悟,乃更厲聲罵大目:「汝先帝家人,不念報恩,而反與司馬師作逆;不顧上天,天不祐汝!」乃張弓傅矢欲射大目,大目涕泣曰:「世事敗矣,善自努力也。」}
 大將軍縱驍騎追擊,大破之,欽遁走。是日,儉聞欽戰敗,恐懼夜走,衆潰。北至鎮縣,左右人兵稍棄儉去,儉獨與小弟秀及孫重藏水邊草中。安風津都尉部民張屬就射殺儉,傳首京都。屬封侯。秀、重走入吴。將士諸為儉、欽所迫脅者,悉歸降。
 \gezhu{欽與郭淮書曰:「大將軍昭伯與太傅俱受顧命,登牀把臂,託付天下,此遠近所知。後以勢利,乃絕其祀,及其親黨,皆一時之俊,可為痛心,柰何柰何!公侯恃與大司馬公恩親分著,義貫金石,當此之時,想益毒痛,有不可堪也。王太尉嫌其專朝,潛欲舉兵,事竟不捷,復受誅夷,害及楚王,想甚追恨。太傅旣亡,然其子師繼承父業,肆其虐暴,日月滋甚,放主殺后,殘戮忠良,包藏禍心,遂至篡弒。此可忍也,孰不可忍?欽以名義大故,事君有節,忿憤內發,忘寢與食,無所吝顧也。會毌丘子邦自與父書,騰說公侯,盡事主之義,欲奮白髮,同符太公,惟須東問,影響相應,聞問之日,能不慷慨!是以不顧妻孥之痛,即與毌丘鎮東舉義兵三萬餘人,西趨京師,欲扶持王室,埽除姦逆,企踵西望,不得聲問,魯望高子,不足喻急。夫當仁不讓,況救君之難,度道遠艱,故不果期要耳。然同舟共濟,安危勢同,禍痛已連,非言飾所解,自公侯所明也。共事曹氏,積信魏朝,行道之人,皆所知見。然在朝之士,冒利偷生,烈士所恥,公侯所賤,賈豎所不忍為也,況當塗之士邪?軍屯住項,小人以閏月十六日別進兵,就於樂嘉城討師,師之徒衆尋時崩潰,其所斬截,不復訾原,但當長驅徑至京師,而流言先至,毌丘不復詳之,更謂小人為誤,諸軍便尔瓦解。毌丘還走,追尋釋解,無所及。小人還項,復遇王基等十二軍,追尋毌丘,進兵討之,即時克破,所向全勝,要那後無繼何?孤軍梁昌,進退失所,還據壽春,壽春復走,狼狽躓閡,無復他計,惟當歸命大吳,借兵乞食,繼踵伍員耳。不若僕隷,如何快心,復君之讎,永使曹氏少享血食,此亦大國之所祐念也。想公侯不使程嬰、杵臼擅名於前代,而使大魏獨無鷹揚之士與?今大吳敦崇大義,深見愍悼。然僕於國大分連接,遠同一勢,曰欲俱舉,瓜分中國,不願偏取以為己有。公侯必欲共忍帥胷懷,宜廣大勢,恐秦川之卒不可孤舉。今者之計,宜屈己伸人,託命歸漢,東西俱舉尔,乃可克定師黨耳。深思鄙言,若愚計可從,宜使漢軍克制期要,使六合校考,與周、召同封,以託付兒孫。此亦非小事也,大丈夫寧處其落落,是以遠逞忠心,時望嘉應。」時郭淮已卒,欽未知,故有此書。世語曰:毌丘儉之誅,黨與七百餘人,傳侍御史杜友治獄,惟舉首事十人,餘皆奏散。友字季子,東郡人,仕晉兾州刺史、河南尹。子默,字世玄,歷吏部郎,衞尉。}
 
 
儉子甸為治書侍御史,先時知儉謀將發,私出將家屬逃走新安靈山上。別攻下之,夷儉三族。
 \gezhu{世語曰:甸字子邦,有名京邑。齊王之廢也,甸謂儉曰:「大人居方獄重任,國傾覆而晏然自守,將受四海之責。」儉然之。大將軍惡其為人也。及儉起兵,問屈𩑺所在,云不來無能為也。儉初起兵,遣子宗四人入吳。太康中,吳平,宗兄弟皆還中國。宗字子仁,有儉風,至零陵太守。宗子奧,巴東監軍、益州刺史。習鑿齒曰:毌丘儉感明帝之顧命,故為此役。君子謂毌丘儉事雖不成,可謂忠臣矣。夫竭節而赴義者我也,成之與敗者時也,我苟無時,成何可必乎?忘我而不自必,乃所以為忠也。古人有言:「死者復生,生者不愧。」若毌丘儉可謂能不愧也。}
 
 
欽亡入吴,吴以欽為都護、假節、鎮北大將軍、幽州牧、譙侯。
 \gezhu{欽降吳表曰:「稟命不幸,常隷魏國,兩絕於天。雖側伏偶都,自知無路。司馬師滔天作逆,廢害二主,辛、癸、高、莽,惡不足喻。欽累世受魏恩,烏鳥之情,竊懷憤踊,在三之義,期於弊仆。前與毌丘儉、郭淮等俱舉義兵,當共討師,埽除凶孽,誠臣慺慺愚管所執。智慮淺薄,微節不騁,進無所依,悲痛切心。退惟不能扶翼本朝,抱愧俛仰,靡所自厝。冒緣古義,固有所歸,庶假天威,得展萬一,僵仆之日,亦所不恨。輙相率將,歸命聖化,慙偷苟生,非辭所陳。謹上還所受魏使持節、前將軍、山桑侯印綬。臨表惶惑,伏須罪誅。」魏書曰:欽字仲若,譙郡人。父稷,建安中為騎將,有勇力。欽少以名將子,材武見稱。魏諷反,欽坐與諷辭語相連,及下獄,掠笞數百,當死,太祖以稷故赦之。太和中,為五營校督,出為牙門將。欽性剛暴無禮,所在倨傲陵上,不奉官法,輙見奏遣,明帝抑之。後復為淮南牙門將,轉為廬江太守、鷹揚將軍。王淩奏欽貪殘,不宜撫邊,求免官治罪,由是徵欽還。曹爽以欽鄉里,厚養待之,不治欽事。復遣還廬江,加冠軍將軍,貴寵踰前。欽以故益驕,好自矜伐,以壯勇高人,頗得虛名於三軍。曹爽誅後,進欽為前將軍以安其心,後代諸葛誕為揚州刺史。自曹爽之誅,欽常內懼,與諸葛誕相惡,無所與謀。會誕去兵,毌丘儉往,乃陰共結謀。戰敗走,晝夜間行,追者不及,遂得入吳,孫峻厚待之。欽雖在他國,不能屈節下人,自呂據、朱異等諸大將皆憎疾之,惟峻常左右之。}
 
 
\end{pinyinscope}