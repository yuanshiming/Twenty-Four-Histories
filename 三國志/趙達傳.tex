\article{趙達傳}
\begin{pinyinscope}
 
 
 趙達,河南人也。少從漢侍中單甫受學,用思精密,謂東南有王者氣,可以避難,故脫身渡江。治九宮一筭之術,究其微旨,是以能應機立成,對問若神,至計飛蝗,射隱伏,無不中效。或難達曰:「飛者固不可校,誰知其然,此殆妄耳。」達使其人取小豆數斗,播之席上,立處其數,驗覆果信。甞過知故,知故為之具食。食畢,謂曰:「倉卒乏酒,又無嘉肴,無以叙意,如何?」達因取盤中隻箸,再三從橫之,乃言:「卿東壁下有美酒一斛,又有鹿肉三斤,何以辭無?」時坐有他賔,內得主人情,主人慙曰:「以卿善射有無,欲相試耳,竟效如此。」遂出酒酣飲。又有書簡上作千萬數,著空倉中封之,令達筭之。達處如數,云:「但有名無實。」其精微若是。
 
 
 
 
 達寶惜其術,自闞澤、殷禮皆名儒善士,親屈節就學,達祕而不告。太史丞公孫滕少師事達,勤苦累年,達許教之者有年數矣,臨當喻語而輒復止。滕他日齎酒具,候顏色,拜跪而請,達曰:「吾先人得此術,欲圖為帝王師,至仕來三世,不過太史郎,誠不欲復傳之。且此術微妙,頭乘尾除,一筭之法,父子不相語。然以子篤好不倦,今真以相授矣。」飲酒數行,達起取素書兩卷,大如手指,達曰:「當寫讀此,則自解也。吾乆廢,不復省之,今欲思論一過,數日當以相與。」滕如期往,至乃陽求索書,驚言失之,云:「女壻昨來,必是渠所竊。」遂從此絕。
 
 
 
 
 初孫權行師征伐,每令達有所推步,皆如其言。權問其法,達終不語,由此見薄,祿位不至。
 
 
\gezhu{吳書曰:初,權即尊號,令達筭作天子之後,當復幾年?達曰:「高祖建元十二年,陛下倍之。」權大喜,左右稱萬歲。果如達言。}
 
 
達常笑謂諸星氣風術者曰:「當迴筭帷幕,不出戶牖以知天道,而反晝夜暴露以望氣祥,不亦難乎!」閒居無為,引筭自校,乃歎曰:「吾筭訖盡某年月日,其終矣。」達妻數見達效,聞而哭泣。達欲弭妻意,乃更步筭,言:「向者謬誤耳,尚未也。」後如期死。權聞達有書,求之不得,乃錄問其女,及發棺無所得,法術絕焉。
 \gezhu{吳錄曰:皇象字休明,廣陵江都人。幼工書。時有張子並、陳梁甫能書。甫恨逋,並恨峻,象斟酌其閒,甚得其妙,中國善書者不能及也。嚴武字子卿,衞尉畯再從子也,圍棊莫與為輩。宋壽占夢,十不失一。曹不興善畫,權使畫屏風,誤落筆點素,因就以作蠅。旣進御,權以為生蠅,舉手彈之。孤城鄭嫗能相人,及範、惇、達八人,世皆稱妙,謂之八絕云。晉陽秋曰:吳有葛衡字思真,明達天官,能為機巧,作渾天,使地居于中,以機動之,天轉而地止,以上應晷度。}
 
 
評曰:三子各於其術精矣,其用思妙矣,然君子等役心神,宜於大者遠者,是以有識之士,舍彼而取此也。
 \gezhu{孫盛曰:夫玄覽未然,逆鑒來事,雖裨竈、梓慎其猶病諸,況術之下此者乎?吳史書達知東南當有王氣,故輕舉濟江。魏承漢緒,受命中畿,達不能豫覩兆萌,而流竄吳越。又不知吝術之鄙,見薄於時,安在其能逆覩天道而審帝王之符瑞哉?昔聖王觀天地之文,以畫八卦之象,故亹亹成於蓍策,變化形乎六爻,是以三易雖殊,卦繇理一,安有迴轉一籌,可以鉤深測隱,意對逆占,而能遂知來物者乎?流俗好異,妄設神奇,不幸之中,仲尼所棄,是以君子志其大者,無所取諸。臣松之以為盛云「君子志其大者,無所取諸」,故評家之旨,非新聲也。其餘所譏,則皆為非理。自中原酷亂,至于建安,數十年間,生民殆盡,比至小康,皆百死之餘耳。江左雖有兵革,不能如中國之甚也,焉知達不筭其安危,知禍有多少,利在東南,以全其身乎?而責不知魏氏將興,流播吳越,在京房之籌,猶不能自免刑戮,況達但以祕術見薄,在悔吝之間乎!古之道術,蓋非一方,探賾之功,豈惟六爻,苟得其要,則可以易而知之矣,迴轉一籌,胡足怪哉?達之推筭,窮其要妙以知幽測隱,何愧於古!而以裨、梓限之,謂達為妄,非篤論也。抱朴子曰:時有葛仙公者,每飲酒醉,常入人家門前陂水中卧,竟日乃出。曾從吳主別,到洌州,還遇大風,百官船多沒,仙公船亦沒淪,吳主甚悵恨。明日使人鉤求公船,而登高以望焉。乆之,見公步從水上來,衣履不沾,而有酒色。旣見而言曰:「臣昨侍從而伍子胥見請,暫過設酒,忽忽不得,即委之。」又有姚光者,有火術。吳主身臨試之,積荻數千束,使光坐其上,又以數千束荻裹之,因猛風而燔之。荻了盡,謂光當以化為燼,而光端坐灰中,振衣而起,把一卷書。吳主取其書視之,不能解也。又曰:吳景帝有疾,求覡視者,得一人。景帝欲試之,乃殺鵝而埋於苑中,架小屋,施牀几,以婦人屐履服物著其上,乃使覡視之。告曰:「若能說此冢中鬼婦人形狀者,當加賞而即信矣。」竟日盡夕無言,帝推問之急,乃曰:「實不見有鬼,但見一頭白鵝立墓上,所以不即白之,疑是鬼神變化作此相,當候其真形而定。無復移易,不知何故,不敢不以實上聞。」景帝乃厚賜之。然則鵝死亦有鬼也。葛洪神仙傳曰:仙人介象,字元則,會稽人,有諸方術。吳主聞之,徵象到武昌,甚敬貴之,稱為介君,為起宅,以御帳給之,賜遺前後累千金,從象學蔽形之術。試還後宮,及出殿門,莫有見者。又使象作變化,種瓜菜百果,皆立生可食。吳主共論鱠魚何者最美,象曰:「鯔魚為上。」吳主曰:「論近道魚耳,此出海中,安可得邪?」象曰:「可得耳。」乃令人於殿庭中作方埳,汲水滿之,并求鉤。象起餌之,垂綸於埳中。須臾,果得鯔魚。吳主驚喜,問象曰:「可食不?」象曰:「故為陛下取以作生鱠,安敢取不可食之物!」乃使廚下切之。吳主曰:「聞蜀使來,得蜀薑作韲甚好,恨爾時無此。」象曰:「蜀薑豈不易得,願差所使者,并付直。」吳主指左右一人,以錢五十付之。象書一符,以著青竹杖中,使行人閉目騎杖,杖止,便買薑訖,復閉目。此人承其言騎杖,須臾止,已至成都,不知是何處,問人,人言是蜀市中,乃買薑。于時吳使張溫先在蜀,旣於市中相識,甚驚,便作書寄其家。此人買薑畢,捉書負薑,騎杖閉目,須臾已還到吳,廚下切鱠適了。臣松之以為葛洪所記,近為惑衆,其書文頗行世,故撮取數事,載之篇末也。神仙之術,詎可測量,臣之臆斷,以為惑衆,所謂夏蟲不知冷氷耳。}
 
 
\end{pinyinscope}