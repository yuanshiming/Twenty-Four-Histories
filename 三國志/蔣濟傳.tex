\article{蔣濟傳}
\begin{pinyinscope}
 
 
 蔣濟字子通,楚國平阿人也。仕郡計吏、州別駕。建安十三年,孫權率衆圍合肥。時大軍征荊州,遇疾疫,唯遣將軍張喜單將千騎,過領汝南兵以解圍,頗復疾疫。濟乃密白刺史偽得喜書,云步騎四萬已到雩婁,遣主簿迎喜。三部使齎書語城中守將,一部得入城,二部為賊所得。權信之,遽燒圍走,城用得全。明年使於譙,太祖問濟曰:「昔孤與袁本初對官渡,徙燕、白馬民,民不得走,賊亦不敢鈔。今欲徙淮南民,何如?」濟對曰:「是時兵弱賊彊,不徙必失之。自破袁紹,北拔柳城,南向江、漢,荊州交臂,威震天下,民無他志。然百姓懷土,實不樂徙,懼必不安。」太祖不從,而江、淮間十餘萬衆,皆驚走吳。後濟使詣鄴,太祖迎見大笑曰:「本但欲使避賊,乃更驅盡之。」拜濟丹楊太守。大軍南征還,以溫恢為揚州刺史,濟為別駕。令曰:「季子為臣,吳宜有君。今君還州,吾無憂矣。」民有誣告濟為謀叛主率者,太祖聞之,指前令與左將軍于禁、沛相封仁等曰:「蔣濟寧有此事!有此事,吾為不知人也。此必愚民樂亂,妄引之耳。」促理出之。辟為丞相主簿西曹屬。令曰:「舜舉臯陶,不仁者遠;臧否得中,望於賢屬矣。」關羽圍樊、襄陽。太祖以漢帝在許,近賊,欲徙都。司馬宣王及濟說太祖曰:「于禁等為水所沒,非戰攻之失,於國家大計未足有損。劉備、孫權,外親內踈,關羽得志,權必不願也。可遣人勸躡其後,許割江南以封權,則樊圍自解。」太祖如其言。權聞之,即引兵西襲公安、江陵。羽遂見禽。
 
 
 
 
 文帝即王位,轉為相國長史。及踐阼,出為東中郎將。濟請留,詔曰:「高祖歌曰『安得猛士守四方』!天下未寧,要須良臣以鎮邊境。如其無事,乃還鳴玉,未為後也。」濟上萬機論,帝善之。入為散騎常侍。時有詔,詔征南將軍夏侯尚曰:「卿腹心重將,時當任使。恩施足死,惠愛可懷。作威作福,殺人活人。」尚以示濟。濟旣至,帝問曰;「卿所聞見天下風教何如?」濟對曰:「未有他善,但見亡國之語耳。」帝忿然作色而問其故,濟具以荅,因曰:「夫『作威作福』,書之明誡。『天子無戲言』,古人所慎。惟陛下察之!」於是帝意解,遣追取前詔。
 
 
 
 
 黃初三年,與大司馬曹仁征吳,濟別襲羨溪。仁欲攻濡須洲中,濟曰:「賊據西岸,列船上流,而兵入洲中,是為自內地獄,危亡之道也。」仁不從,果敗。仁薨,復以濟為東中郎將,代領其兵。詔曰:「卿兼資文武,志節忼愾,常有超越江湖吞吳會之志,故復授將率之任。」頃之,徵為尚書。車駕幸廣陵,濟表水道難通,又上三州論以諷帝。帝不從,於是戰船數千皆滯不得行。議者欲就留兵屯田,濟以為東近湖,北臨淮,若水盛時,賊易為寇,不可安屯。帝從之,車駕即發。還到精湖,水稍盡,盡留船付濟。船本歷適數百里中,濟更鑿地作四五道,蹴船令聚;豫作土豚遏斷湖水,皆引後船,一時開遏入淮中。帝還洛陽,謂濟曰:「事不可不曉。吾前決謂分半燒舩於山陽池中,卿於後致之,略與吾俱至譙。又每得所陳,實入吾意。自今討賊計畫,善思論之。」
 
 
 
 
 明帝即位,賜爵關內侯。大司馬曹休帥軍向皖,濟表以為「深入虜地,與權精兵對,而朱然等在上流,乘休後,臣未見其利也。」軍至皖,吳出兵安陸,濟又上疏曰:「今賊示形於西,必欲并兵圖東,宜急詔諸軍往救之。」會休軍已敗,盡棄器仗輜重退還。吳欲塞夾口,遇救兵至,是以官軍得不沒。遷為中護軍。時中書監、令號為專任,濟上疏曰:「大臣太重者國危,左右太親者身蔽,古之至戒也。往者大臣秉事,外內扇動。陛下卓然自覽萬機,莫不祗肅。夫大臣非不忠也,然威權在下,則衆心慢上,勢之常也。陛下旣已察之於大臣,願無忘於左右。左右忠正遠慮,未必賢於大臣,至於便辟取合,或能工之。今外所言,輒云中書,雖使恭慎不敢外交,但有此名,猶惑世俗。況實握事要,日在目前,儻因疲倦之間有所割制,衆臣見其能推移於事,即亦因時而向之。一有此端,因當內設自完,以此衆語,私招所交,為之內援。若此,臧否毀譽,必有所興,功負賞罰,必有所易;直道而上者或壅,曲附左右者反達。因微而入,緣形而出,意所狎信,不復猜覺。此宜聖智所當早聞,外以經意,則形際自見。或恐朝臣畏言不合而受左右之怨,莫適以聞。臣竊亮陛下潛神默思,公聽並觀,若事有未盡於理而物有未周於用,將改曲易調,遠與黃、唐角功,近昭武、文之迹,豈近習而已哉!然人君猶不可悉天下事以適己明,當有所付。三官任一臣,非周公旦之忠,又非管夷吾之公,則有弄機敗官之弊。當今柱石之士雖少,至於行稱一州,智效一官,忠信竭命,各奉其職,可並驅策,不使聖明之朝有專吏之名也。」詔曰:「夫骨鯁之臣,人主之所仗也。濟才兼文武,服勤盡節,每軍國大事,輒有奏議,忠誠奮發,吾甚壯之。」就遷為護軍將軍,加散騎常侍。
 
 
\gezhu{司馬彪戰略曰:太和六年,明帝遣平州刺史田豫乘海渡,幽州刺史王雄陸道,并攻遼東。蔣濟諫曰:「凡非相吞之國,不侵叛之臣,不宜輕伐。伐之而不制,是驅使為賊。故曰『虎狼當路,不治狐狸。先除大害,小害自已』。今海表之地,累世委質,歲選計考,不乏職貢。議者先之,正使一舉便克,得其民不足益國,得其財不足為富;儻不如意,是為結怨失信也。」帝不聽,豫行竟無成而還。}
 
 
景初中,外勤征役,內務宮室,怨曠者多,而年糓饑儉。濟上疏曰:「陛下方當恢崇前緒,光濟遺業,誠未得高枕而治也。今雖有十二州,至於民數,不過漢時一大郡。二賊未誅,宿兵邊陲,且耕且戰,怨曠積年。宗廟宮室,百事草創,農桑者少,衣食者多,今其所急,唯當息耗百姓,不至甚弊。弊攰之民,儻有水旱,百萬之衆,不為國用。凡使民必須農隙,不奪其時。夫欲大興功之君,先料其民力而燠休之。句踐養胎以待用,昭王恤病以雪仇,故能以弱燕服彊齊,羸越滅勁吳。今二敵不攻不滅,不事即侵,當身不除,百世之責也。以陛下聖明神武之略,舍其緩者,專心討賊,臣以為無難矣。又歡娛之躭,害於精爽;神太用則竭,形太勞則弊。願大簡賢妙,足以充『百斯男』者。其宂散未齒,且悉分出,務在清靜。」詔曰:「微護軍,吾弗聞斯言也。」
 \gezhu{漢晉春秋曰:公孫淵聞魏將來討,復稱臣於孫權,乞兵自救。帝問濟:「孫權其救遼東乎?」濟曰:「彼知官備以固,利不可得,深入則非力所能,淺入則勞而無獲;權雖子弟在危,猶將不動,況異域之人,兼以往者之辱乎!今所以外揚此聲者,譎其行人疑於我,我之不克,兾折後事已耳。然沓渚之間,去淵尚遠,若大軍相持,事不速決,則權之淺規,或能輕兵掩襲,未可測也。」}
 
 
齊王即位,徙為領軍將軍,進爵昌陵亭侯,
 \gezhu{列異傳曰:濟為領軍,其婦夢見亡兒涕泣曰:「死生異路,我生時為卿相子孫,今在地下為泰山五伯,憔悴困辱,不可復言。今太廟西謳士孫阿,今見召為泰山令,願母為白侯,屬阿令轉我得樂處。」言訖,母忽然驚寤,明日以白濟。濟曰:「夢為爾耳,不足恠也。」明日暮,復夢曰:「我來迎新君,止在廟下。未發之頃,暫得來歸。新君明日日中當發,臨發多事,不復得歸,永辭於此。侯氣彊,難感悟,故自訴於母,願重啟侯,何惜不一試驗之?」遂道阿之形狀,言甚備悉。天明,母重啟侯:「雖云夢不足恠,此何太適?適亦何惜不一驗之?」濟乃遣人詣太廟下,推問孫阿,果得之,形狀證驗悉如兒言。濟涕泣曰:「幾負吾兒!」於是乃見孫阿,具語其事。阿不懼當死,而喜得為泰山令,惟恐濟言不信也。曰:「若如節下言,阿之願也。不知賢子欲得何職?」濟曰:「隨地下樂者與之。」阿曰:「輒當奉教。」乃厚賞之,言訖遣還。濟欲速知其驗,從領軍門至廟下,十步安一人,以傳消息。辰時傳阿心痛,巳時傳阿劇,日中傳阿亡。濟泣曰:「雖哀吾兒之不幸,自喜亡者有知。」後月餘,兒復來語母曰:「已得轉為錄事矣。」}
 遷太尉。初,侍中高堂隆論郊祀事,以魏為舜後,推舜配天。濟以為舜本姓媯,其苗曰田,非曹之先,著文以追詰隆。
 \gezhu{臣松之案蔣濟立郊議稱曹騰碑文云「曹氏族出自邾」,魏書述曹氏胤緒亦如之。魏武作家傳,自云曹叔振鐸之後。故陳思王作武帝誄曰:「於穆武王,冑稷胤周。」此其不同者也。及至景初,明帝從高堂隆議,謂魏為舜後,後魏為禪晉文,稱「昔我皇祖有虞」,則其異彌甚。尋濟難隆,及與尚書繆襲往反,並有理據,文多不載。濟亦未能定氏族所出,但謂「魏非舜後而橫祀非族,降黜太祖,不配正天,皆為繆妄」。然于時竟莫能正。濟又難:鄭玄注祭法云「有虞以上尚德,禘郊祖宗,配用有德,自夏已下,稍用其姓氏」。濟曰:「夫虯龍神於獺,獺自祭其先,不祭虯龍也。騏驎白虎仁於豺,豺自祭其先,不祭騏虎也。如玄之說,有虞已上,豺獺之不若邪?臣以為祭法所云,見疑學者乆矣,鄭玄不考正其違而就通其義。」濟豺獺之譬,雖似俳諧,然其義旨,有可求焉。}
 是時,曹爽專政,丁謐、鄧颺等輕改法度。會有日蝕變,詔群臣問其得失,濟上疏曰:「昔大舜佐治,戒在比周;周公輔政,慎於其朋;齊侯問災,晏嬰對以布惠;魯君問異,臧孫荅以緩役。應天塞變,乃實人事。今二賊未滅,將士暴露已數十年,男女怨曠,百姓貧苦。夫為國法度,惟命世大才,乃能張其綱維以垂于後,豈中下之吏所宜改易哉?終無益於治,適足傷民,望宜使文武之臣各守其職,率以清平,則和氣祥瑞可感而致也。」
 
 
以隨太傅司馬宣王屯洛水浮橋,誅曹爽等,進封都鄉侯,邑七百戶。濟上疏曰:「臣忝寵上司,而爽敢苞藏禍心,此臣之無任也。太傅奮獨斷之策,陛下明其忠節,罪人伏誅,社稷之福也。夫封寵慶賞,必加有功。今論謀則臣不先知,語戰則非臣所率,而上失其制,下受其弊。臣備宰司,民所具瞻,誠恐冒賞之漸自此而興,推讓之風由此而廢。」固辭,不許。
 \gezhu{孫盛曰:蔣濟之辭邑,可謂不負心矣。語曰「不為利回,不為義疚」,蔣濟其有焉。}
 是歲薨,謚曰景侯。
 \gezhu{世語曰:初,濟隨司馬宣王屯洛水浮橋,濟書與曹爽,言宣王旨「惟免官而已」,爽遂誅滅。濟病其言之失信,發病卒。}
 子秀嗣。秀薨,子凱嗣。咸熈中,開建五等,以濟著勳前朝,改封凱為下蔡子。
 
 
\end{pinyinscope}