\article{劉廙傳}
\begin{pinyinscope}
 
 
 劉廙字恭嗣,南陽安衆人也。年十歲,戲於講堂上,潁川司馬德操拊其頭曰:「孺子,孺子,『黃中通理』,寧自不知不?」廙兄望之,有名於世,荊州牧劉表辟為從事。而其友二人,皆以讒毀,為表所誅。望之又以正諫不合,投傳告歸。廙謂望之曰:「趙殺鳴、犢,仲尼回輪。
 
 
\gezhu{劉向新序曰:趙簡子欲專天下,謂其相曰:「趙有犢犨,晉有鐸鳴,魯有孔丘,吾殺三人者,天下可王也。」於是乃召犢犨、鐸鳴而問政焉,已即殺之。使使者聘孔子於魯,以胖牛肉迎於河上。使者謂船人曰:「孔子即上船,中河必流而殺之。」孔子至,使者致命,進胖牛之肉。孔子仰天而歎曰:「美哉水乎,洋洋乎,使丘不濟此水者,命也夫!」子路趨而進曰:「敢問何謂也?」孔子曰:「夫犢犨、鐸鳴,晉國之賢大夫也,趙簡子未得意之時,須而後從政,及其得意也,殺之。黃龍不反于涸澤,鳳皇不離其罻羅。故刳胎焚林,則麒麟不臻;覆巢破卵,則鳳皇不翔;竭澤而漁,則龜龍不見。鳥獸之於不仁,猶知避之,況丘乎?故虎嘯而谷風起,龍興而景雲見,擊庭鍾於外,而黃鍾應於內。夫物類之相感,精神之相應,若響之應聲,影之象形,故君子違傷其類者。今彼已殺吾類矣,何為之此乎?」於是遂回車,不渡而還。}
 今兄旣不能法栁下惠和光同塵於內,則宜模范蠡遷化於外。坐而自絕於時,殆不可也!」望之不從,尋復見害。廙懼,奔揚州,
 \gezhu{廙別傳載廙道路為牋謝劉表曰:「考匊過蒙分遇榮授之顯,未有管、狐、桓、文之烈,孤德隕命,精誠不遂。兄望之見禮在昔,旣無堂構昭前之績,中規不密,用墜禍辟。斯乃明神弗祐,天降之災。悔吝之負,哀號靡及。廙之愚淺,言行多違,懼有浸潤三至之間。考匊之愛已衰,望之之責猶存,必傷天慈旣往之分,門戶殪滅,取笑明哲。是用迸竄,永涉川路,即日到廬江尋陽。昔鍾儀有南音之操,椒舉有班荊之思,雖遠猶邇,敢忘前施?」傅子曰:表旣殺望之,荊州士人皆自危也。夫表之本心,於望之不輕也,以直迕情,而讒言得入者,以無容直之度也。據全楚之地,不能以成功者,未必不由此也。夷、叔迕武王以成名,丁公順高祖以受戮,二主之度遠也。若不遠其度,惟褊心是從,難乎以容民畜衆矣。}
 遂歸太祖。太祖辟為丞相掾屬,轉五官將文學。文帝器之,命廙通草書。廙荅書曰:「初以尊卑有踰,禮之常分也。是以貪守區區之節,不敢脩草。必如嚴命,誠知勞謙之素,不貴殊異若彼之高,而惇白屋如斯之好,苟使郭隗不輕於燕,九九不忽於齊,樂毅自至,霸業以隆。
 \gezhu{戰國策曰:有以九九求見齊桓公,桓公不納。其人曰;「九九小術,而君納之,況大於九九者乎?」於是桓公設庭燎之禮而見之。居無幾,隰朋自遠而至,齊遂以霸。}
 虧匹夫之節,成巍巍之美,雖愚不敏,何敢以辭?」魏國初建,為黃門侍郎。
 
 
 
 
 太祖在長安,欲親征蜀,廙上疏曰:「聖人不以智輕俗,王者不以人廢言。故能成功於千載者,必以近察遠,智周於獨斷者,不恥於下問,亦欲博采必盡於衆也。且韋弦非能言之物,而聖賢引以自匡。臣才智闇淺,願自比於韋弦。昔樂毅能用弱燕破大齊,而不能以輕兵定即墨者,夫自為計者雖弱必固,欲自潰者雖彊必敗也。自殿下起軍以來,三十餘年,敵無不破,彊無不服。今以海內之兵,百勝之威,而孫權負險於吳,劉備不賔於蜀。夫夷狄之臣,不當兾州之卒,權、備之籍,不比袁紹之業,然本初以亡,而二寇未捷,非闇弱於今而智武於昔也。斯自為計者,與欲自潰者異勢耳。故文王伐崇,三駕不下,歸而脩德,然後服之。秦為諸侯,所征必服,及兼天下,東向稱帝,匹夫大呼而社稷用隳。是力斃於外,而不卹民於內也。臣恐邊寇非六國之敵,而世不乏才,土崩之勢,此不可不察也。天下有重得,有重失:勢可得而我勤之,此重得也;勢不可得而我勤之,此重失也。於今之計,莫若料四方之險,擇要害之處而守之,選天下之甲卒,隨方面而歲更焉。殿下可高枕於廣夏,潛思於治國;廣農桑,事從節約,脩之旬年,則國富民安矣。」太祖遂進前而報廙曰:「非但君當知臣,臣亦當知君。今欲使吾坐行西伯之德,恐非其人也。」
 
 
魏諷反,廙弟偉為諷所引,當相坐誅。太祖令曰:「叔向不坐弟虎,古之制也。」特原不問,
 \gezhu{廙別傳曰:初,廙弟偉與諷善,廙戒之曰;「夫交友之美,在於得賢,不可不詳。而世之交者,不審擇人,務合黨衆,違先聖人交友之義,此非厚己輔仁之謂也。吾觀魏諷,不脩德行,而專以鳩合為務,華而不實,此直攪世治名者也。卿其慎之,勿復與通。」偉不從,故及於難。}
 徙署丞相倉曹屬。廙上疏謝曰:「臣罪應傾宗,禍應覆族。遭乾坤之靈,值時來之運,揚湯止沸,使不燋爛;起煙於寒灰之上,生華於已枯之木。物不荅施於天地,子不謝生於父母,可以死效,難用筆陳。」
 \gezhu{廙別傳載廙表論治道曰:「昔者周有亂臣十人,有婦人焉,九人而已,孔子稱『才難,不其然乎』!明賢者難得也。況亂弊之後,百姓凋盡,士之存者蓋亦無幾。股肱大職,及州郡督司,邊方重任,雖備其官,亦未得人也。此非選者之不用意,蓋才匱使之然耳。況於長吏以下,羣職小任,能皆簡練備得其人也?其計莫如督之以法。不爾而數轉易,往來不已,送迎之煩,不可勝計。轉易之間,輒有姦巧,旣於其事不省,而為政者亦以其不得乆安之故,知惠益不得成於己,而苟且之可免於患,皆將不念盡心於卹民,而夢想於聲譽,此非所以為政之本意也。今之所以為黜陟者,近頗以州郡之毀譽,聽往來之浮言耳。亦皆得其事實而課其能否也?長吏之所以為佳者,奉法也,憂公也,卹民也。此三事者,或州郡有所不便,往來者有所不安。而長吏執之不已,於治雖得計,其聲譽未為美;闕而從人,於治雖失計,其聲譽必集也。長吏皆知黜陟之在於此也,亦何能不去本而就末哉?以為長吏皆宜使小乆,足使自展。歲課之能,三年總計,乃加黜陟。課之皆當以事,不得依名。事者,皆以戶口率其墾田之多少,及盜賊發興,民之亡叛者,為得負之計。如此行之,則無能之吏,脩名無益;有能之人,無名無損。法之一行,雖無部司之監,姦譽妄毀,可得而盡。」事上,太祖甚善之。}
 廙著書數十篇,及與丁儀共論刑禮,皆傳於世。文帝即王位,為侍中,賜爵關內侯。黃初二年卒。
 \gezhu{廙別傳云:時年四十二。}
 無子。帝以弟子阜嗣。
 \gezhu{案劉氏譜:阜字伯陵,陳留太守。阜子喬,字仲彥。晉陽秋曰:喬有贊世志力。惠帝末,為豫州刺史。喬冑胤丕顯,貴盛至今。}
 
 
\end{pinyinscope}