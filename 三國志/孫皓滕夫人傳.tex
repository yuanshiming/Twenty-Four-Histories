\article{孫皓滕夫人傳}
\begin{pinyinscope}
 
 
 孫皓滕夫人,故太常胤之族女也。胤夷滅,夫人父牧,以踈遠徙邊郡。孫休即位,大赦,得還,以牧為五官中郎。皓旣封烏程侯,聘牧女為妃。皓即位,立為皇后,封牧高密侯,拜衞將軍,錄尚書事。後朝士以牧尊戚,頗推令諫爭。而夫人寵漸衰,皓滋不恱,皓母何恒左右之。又太史言,於運歷,后不可易,皓信巫覡,故得不廢,常供養升平宮。牧見遣居蒼梧郡,雖爵位不奪,其實裔也,遂道路憂死。長秋官僚備員而已,受朝賀表疏如故。而皓內諸寵姬,佩皇后璽紱者多矣。
 
 
\gezhu{江表傳曰:皓又使黃門備行州郡,科取將吏家女。其二千石大臣子女,皆當歲歲言名,年十五六一簡閱,簡閱不中,乃得出嫁。後宮千數,而採擇無已。}
 天紀四年,隨皓遷于洛陽。
 
 
 
 
 評曰:易稱「正家而天下定」。詩云:「刑于寡妻,至于兄弟,以御于家邦。」誠哉,是言也!遠觀齊桓,近察孫權,皆有識士之明,傑人之志,而嫡庶不分,閨庭錯亂,遺笑古今,殃流後嗣。由是論之,惟以道義為心、平一為主者,然後克免斯累邪!
 
 
\end{pinyinscope}