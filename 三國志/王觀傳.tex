\article{王觀傳}
\begin{pinyinscope}
 
 
 王觀字偉臺,東郡廩丘人也。少孤貧勵志,太祖召為丞相文學掾,出為高唐、陽泉、酇、任令,所在稱治。文帝踐阼,入為尚書郎、廷尉監,出為南陽、涿郡太守。涿北接鮮卑,數有寇盜,觀令邊民十家已上,屯居,築京候。時或有不願者,觀乃假遣朝吏,使歸助子弟,不與期會,但勑事訖各還。於是吏民相率不督自勸,旬日之中,一時俱成。守禦有備,寇鈔以息。明帝即位,下詔書使郡縣條為劇、中、平者。主者欲言郡為中平,觀教曰:「此郡濵近外虜,數有寇害,云何不為劇邪?」主者曰:「若郡為外劇,恐於明府有任子。」觀曰:「夫君者,所以為民也。今郡在外劇,則於役條當有降差。豈可為太守之私而負一郡之民乎?」遂言為外劇郡,後送任子詣鄴。時觀但有一子而又幼弱。其公心如此。觀治身清素,帥下以儉,僚屬承風,莫不自勵。
 
 
 
 
 明帝幸許昌,召觀為治書侍御史,典行臺獄。時多有倉卒喜怒,而觀不阿意順指。太尉司馬宣王請觀為從事中郎,遷為尚書,出為河南尹,徙少府。大將軍曹爽使材官張達斫家屋材,及諸私用之物,觀聞知,皆錄奪以沒官。少府統三尚方御府內藏玩弄之寶,爽等奢放,多有干求,憚觀守法,乃徙為太僕。司馬宣王誅爽,使觀行中領軍,據爽弟羲營,賜爵關內侯,復為尚書,加駙馬都尉。高貴鄉公即位,封中鄉亭侯。頃之,加光祿大夫,轉為右僕射。常道鄉公即位,進封陽鄉侯,增邑千戶,并前二千五百戶。遷司空,固辭,不許,遣使即第拜授。就官數日,上送印綬,輒自輿歸里舍。薨于家,遺令藏足容棺,不設明器,不封不樹。謚曰肅侯。子悝嗣。咸熈中,開建五等,以觀著勳前朝,改封悝膠東子。
 
 
 
 
 評曰:韓曁處以靜居行化,出以任職流稱;崔林簡樸知能;高柔明於法理;孫禮剛斷伉厲;王觀清勁貞白:咸克致公輔。及曁年過八十,起家就列;柔保官二十年,元老終位:比之徐邈、常林,於茲為疚矣。
 
 
\end{pinyinscope}