\article{來敏傳}
\begin{pinyinscope}
 
 
 來敏字敬達,義陽新野人,來歙之後也。父豔,為漢司空。
 
 
\gezhu{華嶠後漢書曰:豔好學下士,開館養徒衆。少歷顯位,靈帝時位至司空。}
 漢末大亂,敏隨姊奔荊州,姊夫黃琬是劉璋祖母之姪,故璋遣迎琬妻,敏遂俱與姊入蜀,常為璋賔客。涉獵書籍,善左氏春秋,尤精於倉、雅訓詁,好是正文字。先主定益州,署敏典學校尉,及立太子,以為家令。後主踐阼,為虎賁中郎將。丞相亮住漢中,請為軍祭酒、輔軍將軍,坐事去職。
 \gezhu{亮集有教曰:「將軍來敏對上官顯言『新人有何功德而奪我榮資與之邪?諸人共憎我,何故如是』?敏年老狂悖,生此怨言。昔成都初定,議者以為來敏亂羣,先帝以新定之際,故遂含容,無所禮用。後劉子初選以為太子家令,先帝不恱而不忍拒也。後主旣位,吾闇於知人,遂復擢為將軍祭酒,違議者之審見,背先帝所疏外,自謂能以敦厲薄俗,帥之以義。今旣不能,表退職,使閉門思愆。」}
 亮卒後,還成都為大長秋,又免,後累遷為光祿大夫,復坐過黜。前後數貶削,皆以語言不節,舉動違常也。時孟光亦以樞機不慎,議論干時,然猶愈於敏,俱以其耆宿學士見禮於世。而敏荊楚名族,東宮舊臣,特加優待,是故廢而復起。後以敏為執慎將軍,欲令以官重自警戒也。年九十七,景耀中卒。子忠,亦博覽經學,有敏風,與尚書向充等並能協贊大將軍姜維。維善之,以為參軍。
 
 
\end{pinyinscope}