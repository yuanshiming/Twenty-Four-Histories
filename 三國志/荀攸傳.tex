\article{荀攸傳}
\begin{pinyinscope}
 
 
 荀攸字公達,彧從子也。祖父曇,廣陵太守。
 
 
\gezhu{荀氏家傳曰:曇字元智。兄昱,字伯脩。張璠漢紀稱昱、曇並傑俊有殊才。昱與李膺、王暢、杜密等號為八俊,位至沛相。攸父彝,州從事。彝於彧為從祖兄弟。}
 攸少孤。及曇卒,故吏張權求守曇墓。攸年十三,疑之,謂叔父衢曰:「此吏有非常之色,殆將有姦!」衢寤,乃推問,果殺人亡命。由是異之。
 \gezhu{魏書曰:攸年七八歲,衢曾醉,誤傷攸耳;而攸出入遊戲,常避護不欲令衢見。衢後聞之,乃驚其夙智如此。荀氏家傳曰:衢子祈,字伯旗,與族父愔俱著名。祈與孔融論肉刑,愔與孔融論聖人優劣,並在融集。祈位至濟陰太守;愔後徵有道,至丞相祭酒。}
 
 
何進秉政,徵海內名士攸等二十餘人。攸到,拜黃門侍郎。董卓之亂,關東兵起,卓徙都長安。攸與議郎鄭泰、何顒、侍中种輯、越騎校尉伍瓊等謀曰:「董卓無道,甚於桀紂,天下皆怨之,雖資彊兵,實一匹夫耳。今直刺殺之以謝百姓,然後據殽、函,輔王命,以號令天下,此桓文之舉也。」事垂就而覺,收顒、攸繫獄,顒憂懼自殺,
 \gezhu{張璠漢紀曰:顒字伯求,少與郭泰、賈彪等遊學洛陽,泰等與同風好。顒顯名太學,於是中朝名臣太傅陳蕃、司隷李膺等皆深接之。及黨事起,顒亦名在其中,乃變名姓亡匿汝南間,所至皆交結其豪桀。顒旣奇太祖而知荀彧,袁紹慕之,與為奔走之友。是時天下士大夫多遇黨難,顒常歲再三私入洛陽,從紹計議,為諸窮窘之士解釋患禍。而袁術亦豪俠,與紹爭名。顒未常造術,術深恨之。漢末名士錄曰:術嘗於衆坐數顒三罪,曰:「王德彌先覺儁老,名德高亮,而伯求踈之,是一罪也。許子遠凶淫之人,性行不純,而伯求親之,是二罪也。郭、賈寒窶,無他資業,而伯求肥馬輕裘,光曜道路,是三罪也。」陶丘洪曰:「王德彌大賢而短於濟時,許子遠雖不純而赴難不憚濡足。伯求舉善則以德彌為首,濟難則以子遠為宗。且伯求嘗為虞偉高手刃復仇,義名奮發。其怨家積財巨萬,文馬百駟,而欲使伯求羸牛疲馬,頓伏道路,此為披其胷而假仇敵之刃也。」術意猶不平。後與南陽宗承會於闕下,術發怒曰:「何伯求,凶德也,吾當殺之。」承曰:「何生英俊之士,足下善遇之,使延令名於天下。」術乃止。後黨禁除解,辟司空府。每三府掾屬會議,顒策謀有餘,議者皆自以為不及。遷北軍中候,董卓以為長史。後荀彧為尚書令,遣人迎叔父司空爽喪,使并置顒尸,而葬之於爽冢傍。}
 攸言語飲食自若,會卓死得免。
 \gezhu{魏書云攸使人說卓得免,與此不同。}
 棄官歸,復辟公府,舉高第,遷任城相,不行。攸以蜀漢險固,人民殷盛,乃求為蜀郡太守,道絕不得至,駐荊州。
 
 
 
 
 太祖迎天子都許,遺攸書曰:「方今天下大亂,智士勞心之時也,而顧觀變蜀漢,不已乆乎!」於是徵攸為汝南太守,入為尚書。太祖素聞攸名,與語大恱,謂荀彧,鍾繇曰:「公達,非常人也,吾得與之計事,天下當何憂哉!」以為軍師。建安三年,從征張繡。攸言於太祖曰:「繡與劉表相恃為彊,然繡以遊軍仰食於表,表不能供也,勢必離。不如緩軍以待之,可誘而致也;若急之,其勢必相救。」太祖不從,遂進軍之穰,與戰。繡急,表果救之。軍不利。太祖謂攸曰:「不用君言至是。」乃設奇兵復戰,大破之。
 
 
是歲,太祖自宛征呂布,
 \gezhu{魏書曰:議者云表、繡在後而還襲呂布,其危必也。攸以為表、繡新破,勢不敢動。布驍猛,又恃袁術,若從橫淮、泗間,豪傑必應之。今乘其初叛,衆心未一,往可破也。太祖曰:「善。」比行,布以敗劉備,而臧霸等應之。}
 至下邳,布敗退固守,攻之不拔,連戰,士卒疲,太祖欲還。攸與郭嘉說曰:「呂布勇而無謀,今三戰皆北,其銳氣衰矣。三軍以將為主,主衰則軍無奮意。夫陳宮有智而遲,今及布氣之未復,宮謀之未定,進急攻之,布可拔也。」乃引沂、泗灌城,城潰,生禽布。
 
 
後從救劉延於白馬,攸畫策斬顏良。語在武紀。太祖拔白馬還,遣輜重循河而西。袁紹渡河追,卒與太祖遇。諸將皆恐,說太祖還保營,攸曰:「此所以禽敵,柰何去之!」太祖目攸而笑。遂以輜重餌賊,賊競奔之,陣亂。乃縱步騎擊,大破之,斬其騎將文醜,太祖遂與紹相拒於官渡。軍食方盡,攸言於太祖曰:「紹運車旦暮至,其將韓𦳣銳而輕敵,擊可破也。」
 \gezhu{臣松之案諸書,韓𦳣或作韓猛,或云韓若,未詳孰是。}
 太祖曰:「誰可使?」攸曰:「徐晃可。」乃遣晃及史渙邀擊破走之,燒其輜重。會許攸來降,言紹遣淳于瓊等將萬餘兵迎運糧,將驕卒惰,可要擊也。衆皆疑。唯攸與賈詡勸太祖。太祖乃留攸及曹洪守。太祖自將攻破之,盡斬瓊等。紹將張郃、高覽燒攻櫓降,紹遂棄軍走。郃之來,洪疑不敢受,攸謂洪曰:「郃計不用,怒而來,君何疑?」乃受之。
 
 
七年,從討袁譚、尚於黎陽。明年,太祖方征劉表,譚、尚爭兾州。譚遣辛毗乞降請救,太祖將許之,以問羣下。羣下多以為表彊,宜先平之,譚、尚不足憂也。攸曰:「天下方有事,而劉表坐保江、漢之閒,其無四方志可知矣。袁氏據四州之地,帶甲十萬,紹以寬厚得衆,借使二子和睦以守其成業,則天下之難未息也。今兄弟遘惡,其勢不兩全。若有所并則力專,力專則難圖也。及其亂而取之,天下定矣,此時不可失也。」太祖曰:「善。」乃許譚和親,遂還擊破尚。其後譚叛,從斬譚於南皮。兾州平,太祖表封攸曰:「軍師荀攸,自初佐臣,無征不從,前後克敵,皆攸之謀也。」於是封陵樹亭侯。十二年,下令大論功行封,太祖曰:「忠正密謀,撫寧內外,文若是也。公達其次也。」增邑四百,并前七百戶,
 \gezhu{魏書曰:太祖自柳城還,過攸舍,稱述攸前後謀謨勞勳,曰:「今天下事略已定矣,孤願與賢士大夫共饗其勞。昔高祖使張子房自擇邑三萬戶,今孤亦欲君自擇所封焉。」}
 轉為中軍師。魏國初建,為尚書令。
 
 
攸深密有智防,自從太祖征伐,常謀謩帷幄,時人及子弟莫知其所言。
 \gezhu{魏書曰:攸姑子辛韜曾問攸說太祖取兾州時事。攸曰:「佐治為袁譚乞降,王師自往平之,吾何知焉?」自是韜及內外莫敢復問軍國事也。}
 太祖每稱曰:「公達外愚內智,外怯內勇,外弱內彊,不伐善,無施勞,智可及,愚不可及,雖顏子、寗武不能過也。」文帝在東宮,太祖謂曰:「荀公達,人之師表也,汝當盡禮敬之。」攸曾病,世子問病,獨拜牀下,其見尊異如此。攸與鍾繇善,繇言:「我每有所行,反覆思惟,自謂無以易;以咨公達,輒復過人意。」公達前後凡畫奇策十二,唯繇知之。繇撰集未就,會薨,故世不得盡聞也。
 \gezhu{臣松之案:攸亡後十六年,鍾繇乃卒,撰攸奇策,亦有何難?而年造八十,猶云未就,遂使攸從征機策之謀不傳於世,惜哉!}
 攸從征孫權,道薨。太祖言則流涕。
 \gezhu{魏書曰:時建安十九年,攸年五十八。計其年大彧六歲。魏書載太祖令曰:「孤與荀公達周遊二十餘年,無毫毛可非者。」又曰:「荀公達真賢人也,所謂『溫良恭儉讓以得之』。孔子稱『晏平仲善與人交,乆而敬之』,公達即其人也。」傅子曰:或問近世大賢君子,荅曰:「荀令君之仁,荀軍師之智,斯可謂近世大賢君子矣。荀令君仁以立德,明以舉賢,行無諂黷,謀能應機。孟軻稱『五百年而有王者興,其間必有命世者』,其荀令君乎!太祖稱『荀令君之進善,不進不休,荀軍師之去惡,不去不止』也。」}
 
 
 
 
 長子緝,有攸風,早沒。次子適嗣,無子,絕。黃初中,紹封攸孫彪為陵樹亭侯,邑三百戶,後轉封丘陽亭侯。正始中,追謚攸曰敬侯。
 
 
\end{pinyinscope}