\article{呂範傳}
\begin{pinyinscope}
 
 
 呂範字子衡,汝南細陽人也。少為縣吏,有容觀姿貌。邑人劉氏,家富女美,範求之。女母嫌,欲勿與,劉氏曰:「觀呂子衡寧當乆貧者邪?」遂與之婚。後避亂壽春,孫策見而異之,範遂自委昵,將私客百人歸策。時太妃在江都,策遣範迎之。徐州牧陶謙謂範為袁氏覘候,諷縣掠考範,範親客健兒篡取以歸。時唯範與孫河常從策,跋涉辛苦,危難不避,策亦親戚待之,每與升堂,飲宴於太妃前。
 
 
 
 
 後從策攻破廬江,還俱東渡,到橫江、當利,破張英、于麋,下小丹楊、湖熟,領湖熟相。策定秣陵、曲阿,收笮融、劉繇餘衆,增範兵二千,騎五十匹。後領宛陵令,討破丹楊賊,還吳,遷都督。
 
 
\gezhu{〕江表傳曰:策從容獨與範棊,範曰:「今將軍事業日大,士衆日盛,範在遠,聞綱紀猶有不整者,範願蹔領都督,佐將軍部分之。」策曰:「子衡,卿旣士大夫,加手下已有大衆,立功於外,豈宜復屈小職,知軍中細碎事乎!」範曰:「不然。今捨本土而託將軍者,非為妻子也,欲濟世務。猶同舟涉海,一事不牢,即俱受其敗。此亦範計,非但將軍也。」策笑,無以荅。範出,便釋褠,著袴褶,執鞭,詣閤下啟事,自稱領都督,策乃授傳,委以衆事。由是軍中肅睦,威禁大行。}
 
 
是時下邳陳瑀自號吳郡太守,住海西,與彊族嚴白虎交通。策自將討虎,別遣範與徐逸攻瑀於海西,梟其大將陳牧。
 \gezhu{九州春秋曰:初平三年,揚州刺史陳禕死,袁術使瑀領揚州牧。後術為曹公所敗於封丘,南人叛瑀,瑀拒之。術走陰陵,好辭以下瑀,瑀不知權,而又怯,不即攻術。術於淮北集兵向壽春。瑀懼,使其弟公琰請和於術。術執之而進,瑀走歸下邳。}
 又從攻祖郎於陵陽,太史慈於勇里。七縣平定,拜征虜中郎將,征江夏,還平鄱陽。
 
 
 
 
 策薨,奔喪于吳。後權復征江夏,範與張昭留守。
 
 
 
 
 曹公至赤壁,與周瑜等俱拒破之,拜裨將軍,領彭澤太守,以彭澤、柴桑、歷陽為奉邑。劉備詣京見權,範密請留備。後遷平南將軍,屯柴桑。
 
 
 
 
 權討關羽,過範館,謂曰:「昔早從卿言,無此勞也。今當上取之,卿為我守建業。」權破羽還,都武昌,拜範建威將軍,封宛陵侯,領丹楊太守,治建業,督扶州以下至海,轉以溧陽、懷安、寧國為奉邑。
 
 
 
 
 曹休、張遼、臧霸等來伐,範督徐盛、全琮、孫韶等,以舟師拒休等於洞口。遷前將軍,假節,改封南昌侯。時遭大風,船人覆溺,死者數千,還軍,拜揚州牧。
 
 
性好威儀,州民如陸遜、全琮及貴公子,皆脩敬虔肅,不敢輕脫。其居處服飾,於時奢靡,然勤事奉法,故權恱其忠,不怪其侈。
 \gezhu{江表傳曰:人有白範與賀齊奢麗夸綺,服飾僭擬王者,權曰:「昔管仲踰禮,桓公優而容之,無損於霸。今子衡、公苗,身無夷吾之失,但其器械精好,舟車嚴整耳,此適足作軍容,何損於治哉?」告者乃不敢復言。}
 
 
 
 
 初策使範典主財計,權時年少,私從有求,範必關白,不敢專許,當時以此見望。權守陽羨長,有所私用,策或料覆,功曹周谷輒為傅著簿書,使無譴問。權臨時恱之,及後統事,以範忠誠,厚見信任,以谷能欺更簿書,不用也。
 
 
黃武七年,範遷大司馬,印綬未下,疾卒。權素服舉哀,遣使者追贈印綬。及還都建業,權過範墓呼曰:「子衡!」言及流涕,祀以太牢。
 \gezhu{江表傳曰:初,權移都建業,大會將相文武,時謂嚴畯曰:「孤昔歎魯子敬比鄧禹,呂子衡方吳漢,閒卿諸人未平此論,今定云何?」畯退席曰:「臣未解指趣,謂肅、範受饒,襃歎過實。」權曰:「昔鄧仲華初見光武,光武時受更始使,撫河北,行大司馬事耳,未有帝王志也。禹勸之以復漢業,是禹開初議之端矣。子敬英爽有殊略,孤始與一語,便及大計,與禹相似,故比之。呂子衡忠篤亮直,性雖好奢,然以憂公為先,不足為損,避袁術自歸於兄,兄作大將,別領部曲,故憂兄事,乞為都督,辦護脩整,加之恪勤,與吳漢相類,故方之。皆有旨趣,非孤私之也。」畯乃服。}
 
 
 
 
 範長子先卒,次子據嗣。據字世議,以父任為郎,後範寢疾,拜副軍校尉,佐領軍事。範卒,遷安軍中郎將。數討山賊,諸深惡劇地,所擊皆破。隨太常潘濬討五谿,復有功。朱然攻樊,據與朱異破城外圍,還拜偏將軍,入補馬閑右部督,遷越騎校尉。太元元年,大風,江水溢流,漸淹城門,權使視水,獨見據使人取大船以備害。權嘉之,拜盪魏將軍。權寢疾,以據為太子右部督。太子即位,拜右將軍。魏出東興,據赴討有功。明年,孫峻殺諸葛恪,遷據為驃騎將軍,平西宮事。五鳳二年,假節,與峻等襲壽春,還遇魏將曹珍,破之於高亭。太平元年,帥師侵魏,未及淮,聞孫峻死,以從弟綝自代,據大怒,引軍還,欲廢綝。綝聞之,使中書奉詔,詔文欽、劉纂、唐咨等使取據,又遣從兄憲以都下兵逆據於江都。左右勸據降魏,據曰:「恥為叛臣。」遂自殺。夷三族。
 
 
\end{pinyinscope}