\article{陸遜傳}
\begin{pinyinscope}
 
 
 陸遜字伯言,吳郡吳人也。本名議,世江東大族。
 
 
\gezhu{陸氏世頌曰:遜祖紆,字叔盤,敏淑有思學,守城門校尉。父駿,字季才,淳懿信厚,為邦族所懷,官至九江都尉。}
 遜少孤,隨從祖廬江太守康在官。袁術與康有隙,將攻康,康遣遜及親戚還吳。遜年長於康子績數歲,為之綱紀門戶。
 
 
孫權為將軍,遜年二十一,始仕幕府,歷東西曹令史,出為海昌屯田都尉,并領縣事。
 \gezhu{陸氏祠堂像贊曰:海昌,今鹽官縣也。}
 縣連年亢旱,遜開倉穀以振貧民,勸督農桑,百姓蒙賴。時吳、會稽、丹楊多有伏匿,遜陳便宜,乞與募焉。會稽山賊大帥潘臨,舊為所在毒害,歷年不禽。遜以手下召兵,討治深險,所向皆服,部曲已有二千餘人。鄱陽賊帥尤突作亂,復往討之,拜定威校尉,軍屯利浦。
 
 
 
 
 權以兄策女配遜,數訪世務,遜建議曰:「方今英雄棊跱,豺狼闚望,克敵寧亂,非衆不濟。而山寇舊惡,依阻深地。夫腹心未平,難以圖遠,可大部伍,取其精銳。」權納其策,以為帳下右部督。會丹楊賊帥費棧受曹公印綬,扇動山越,為作內應,權遣遜討棧。棧支黨多而往兵少,遜乃益施牙幢,分布鼓角,夜潛山谷間,鼓譟而前,應時破散。遂部伍東三郡,彊者為兵,羸者補戶,得精卒數萬人,宿惡盪除,所過肅清,還屯蕪湖。
 
 
 
 
 會稽太守淳于式表遜枉取民人,愁擾所在。遜後詣都,言次,稱式佳吏,權曰:「式白君而君薦之,何也?」遜對曰:「式意欲養民,是以白遜。若遜復毀式以亂聖聽,不可長也。」權曰:「此誠長者之事,顧人不能為耳。」
 
 
 
 
 呂蒙稱疾詣建業,遜往見之,謂曰:「關羽接境,如何遠下,後不當可憂也?」蒙曰:「誠如來言,然我病篤。」遜曰:「羽矜其驍氣,陵轢於人。始有大功,意驕志逸,但務北進,未嫌於我,有相聞病,必益無備。今出其不意,自可禽制。下見至尊,宜好為計。」蒙曰:「羽素勇猛,旣難為敵,且已據荊州,恩信大行,兼始有功,膽勢益盛,未易圖也。」蒙至都,權問:「誰可代卿者?」蒙對曰:「陸遜意思深長,才堪負重,觀其規慮,終可大任。而未有遠名,非羽所忌,無復是過。若用之,當令外自韜隱,內察形便,然後可克。」權乃召遜,拜偏將軍右部督代蒙。
 
 
 
 
 遜至陸口,書與羽曰:「前承觀釁而動,以律行師,小舉大克,一何巍巍!敵國敗績,利在同盟,聞慶拊節,想遂席卷,共獎王綱。近以不敏,受任來西,延慕光塵,思稟良規。」又曰:「于禁等見獲,遐邇欣歎,以為將軍之勳足以長世,雖昔晉文城濮之師,淮陰拔趙之略,蔑以尚茲。聞徐晃等步騎駐旌,闚望麾葆。操猾虜也,忿不思難,恐潛增衆,以逞其心。雖云師老,猶有驍悍。且戰捷之後,常苦輕敵,古人杖術,軍勝彌警,願將軍廣為方計,以全獨克。僕書生疏遲,忝所不堪,喜鄰威德,樂自傾盡,雖未合策,猶可懷也。儻明注仰,有以察之。」羽覽遜書,有謙下自託之意,意大安,無復所嫌。遜具啟形狀,陳其可禽之要。權乃潛軍而上,使遜與呂蒙為前部,至即克公安、南郡。遜徑進,領宜都太守,拜撫邊將軍,封華亭侯。備宜都太守樊友委郡走,諸城長吏及蠻夷君長皆降。遜請金銀銅印,以假授初附。是歲建安二十四年十一月也。
 
 
遜遣將軍李異、謝旌等將三千人,攻蜀將詹晏、陳鳳。異將水軍,旌將步兵,斷絕險要,即破晏等,生降得鳳。又攻房陵太守鄧輔、南鄉太守郭睦,大破之。秭歸大姓文布、鄧凱等合夷兵數千人,首尾西方。遜復部旌討破布、凱。布、凱脫走,蜀以為將。遜令人誘之,布帥衆還降。前後斬獲招納,凡數萬計。權以遜為右護軍、鎮西將軍,進封婁侯。
 \gezhu{吳書曰:權嘉遜功德,欲殊顯之,雖為上將軍列侯,猶欲令歷本州舉命,乃使揚州牧呂範就辟別駕從事,舉茂才。}
 
 
 
 
 時荊州士人新還,仕進或未得所,遜上疏曰:「昔漢高受命,招延英異,光武中興,羣俊畢至,苟可以熙隆道教者,未必遠近。今荊州始定,人物未達,臣愚慺慺,乞普加覆載抽拔之恩,令並獲自進,然後四海延頸,思歸大化。」權敬納其言。
 
 
黃武元年,劉備率大衆來向西界,權命遜為大都督、假節,督朱然、潘璋、宋謙、韓當、徐盛、鮮于丹、孫桓等五萬人拒之。備從巫峽、建平、連平、連圍至夷陵界,立數十屯,以金錦爵賞誘動諸夷,使將軍馮習為大督,張南為前部,輔匡、趙融、廖淳、傅肜等各為別督,先遣吳班將數千人於平地立營,欲以挑戰。諸將皆欲擊之,遜曰:「此必有譎,且觀之。」
 \gezhu{吳書曰:諸將並欲迎擊備,遜以為不可,曰:「備舉軍東下,銳氣始盛,且乘高守險,難可卒攻,攻之縱下,猶難盡克,若有不利,損我大勢,非小故也。今但且獎厲將士,廣施方略,以觀其變。若此間是平原曠野,當恐有顛沛交馳之憂,今緣山行軍,勢不得展,自當罷於木石之間,徐制其弊耳。」諸將不解,以為遜畏之,各懷憤恨。}
 備知其計不可,乃引伏兵八千,從谷中出。遜曰:「所以不聽諸君擊班者,揣之必有巧故也。」遜上疏曰:「夷陵要害,國之關限,雖為易得,亦復易失。失之非徒損一郡之地,荊州可憂。今日爭之,當令必諧。備干天常,不守窟穴,而敢自送。臣雖不材,憑奉威靈,以順討逆,破壞在近。尋備前後行軍,多敗少成,推此論之,不足為戚。臣初嫌之,水陸俱進,今反舍船就步,處處結營,察其布置,必無他變。伏願至尊高枕,不以為念也。」諸將並曰:「攻備當在初,今乃令入五六百里,相銜持經七八月,其諸要害皆以固守,擊之必無利矣。」遜曰:「備是猾虜,更甞事多,其軍始集,思慮精專,未可干也。今住已乆,不得我便,兵疲意沮,計不復生,犄角此寇,正在今日。」乃先攻一營,不利。諸將皆曰:「空殺兵耳。」遜曰:「吾已曉破之之術。」乃勑各持一把茅,以火攻拔之。一爾勢成,通率諸軍同時俱攻,斬張南、馮習及胡王沙摩柯等首,破其四十餘營。備將杜路、劉寧等窮逼請降。備升馬鞍山,陳兵自繞。遜督促諸軍四面蹙之,土崩瓦解,死者萬數。備因夜遁,驛人自擔,燒鐃鎧斷後,僅得入白帝城。其舟船器械,水步軍資,一時略盡,尸骸漂流,塞江而下。備大慙恚,曰:「吾乃為遜所折辱,豈非天邪!」
 
 
 
 
 初,孫桓別討備前鋒於夷道,為備所圍,求救於遜。遜曰:「未可。」諸將曰:「孫安東公族,見圍已困,柰何不救?」遜曰:「安東得士衆心,城牢糧足,無可憂也。待吾計展,欲不救安東,安東自解。」及方略大施,備果奔潰。桓後見遜曰:「前實怨不見救,定至今日,乃知調度自有方耳。」
 
 
 
 
 當禦備時,諸將軍或是孫策時舊將,或公室貴戚,各自矜恃,不相聽從。遜案劒曰:「劉備天下知名,曹操所憚,今在境界,此彊對也。諸君並荷國恩,當相輯睦,共翦此虜,上報所受,而不相順,非所謂也。僕雖書生,受命主上。國家所以屈諸君使相承望者,以僕有尺寸可稱,能忍辱負重故也。各在其事,豈復得辭!軍令有常,不可犯矣。」及至破備,計多出遜,諸將乃服。權聞之,曰:「君何以初不啟諸將違節度者邪?」遜對曰:「受恩深重,任過其才。又此諸將或任腹心,或堪爪牙,或是功臣,皆國家所當與共克定大事者。臣雖駑懦,竊慕相如、寇恂相下之義,以濟國事。」權大笑稱善,加拜遜輔國將軍,領荊州牧,即改封江陵侯。
 
 
又備旣住白帝,徐盛、潘璋、宋謙等各競表言備必可禽,乞復攻之。權以問遜,遜與朱然、駱統以為曹丕大合士衆,外託助國討備,內實有姦心,謹決計輒還。無幾,魏軍果出,三方受敵也。
 \gezhu{吳錄曰:劉備聞魏軍大出,書與遜云:「賊今已在江陵,吾將復東,將軍謂其能然不?」遜荅曰:「但恐軍新破,創痍未復,始求通親,且當自補,未暇窮兵耳。若不惟筭,欲復以傾覆之餘,遠送以來者,無所逃命。」}
 
 
 
 
 備尋病亡,子禪襲位,諸葛亮秉政,與權連和。時事所宜,權輒令遜語亮,并刻權印,以置遜所。權每與禪、亮書,常過示遜,輕重可否,有所不安,便令改定,以印封行之。
 
 
七年,權使鄱陽太守周魴譎魏大司馬曹休,休果舉衆入皖,乃召遜假黃鉞,為大都督,逆休。
 \gezhu{陸機為遜銘曰:魏大司馬曹休侵我北鄙,乃假公黃鉞,統御六師及中軍禁衞而攝行王事,主上執鞭,百司屈膝。吳錄曰:假遜黃鉞,吳王親執鞭以見之。}
 休旣覺知,恥見欺誘,自恃兵馬精多,遂交戰。遜自為中部,令朱桓、全琮為左右翼,三道俱進,果衝休伏兵,因驅走之,追亡逐北,徑至夾石,斬獲萬餘,牛馬騾驢車乘萬兩,軍資器械略盡。休還,疽發背死。諸軍振旅過武昌,權令左右以御蓋覆遜,入出殿門,凡所賜遜,皆御物上珍,於時莫與為比。遣還西陵。
 
 
 
 
 黃龍元年,拜上大將軍、右都護。是歲,權東巡建業,留太子、皇子及尚書九官,徵遜輔太子,并掌荊州及豫章三郡事,董督軍國。時建昌侯慮於堂前作鬬鴨欄,頗施小巧,遜正色曰:「君侯宜勤覽經典以自新益,用此何為?」慮即時毀徹之。射聲校尉松於公子中最親,戲兵不整,遜對之髠其職吏。南陽謝景善劉廙先刑後禮之論,遜呵景曰:「禮之長於刑乆矣,廙以細辯而詭先聖之教,皆非也。君今侍東宮,宜遵仁義以彰德音,若彼之談,不須講也。」
 
 
 
 
 遜雖身在外,乃心於國,上疏陳時事曰:「臣以為科法嚴峻,下犯者多。頃年以來,將吏罹罪,雖不慎可責,然天下未一,當圖進取,小宜恩貸,以安下情。且世務日興,良能為先,自不姦穢入身,難忍之過,乞復顯用,展其力效。此乃聖王忘過記功,以成王業。昔漢高舍陳平之愆,用其奇略,終建勳祚,功垂千載。夫峻法嚴刑,非帝王之隆業;有罰無恕,非懷遠之弘規也。」
 
 
 
 
 權欲遣偏師取夷州及朱崖,皆以諮遜,遜上疏曰:「臣愚以為四海未定,當須民力,以濟時務。今兵興歷年,見衆損減,陛下憂勞聖慮,忘寢與食,將遠規夷州,以定大事,臣反覆思惟,未見其利,萬里襲取,風波難測,民易水土,必致疾疫,今驅見衆,經涉不毛,欲益更損,欲利反害。又珠崖絕險,民猶禽獸,得其民不足濟事,無其兵不足虧衆。今江東見衆,自足圖事,但當畜力而後動耳。昔桓王創基,兵不一旅,而開大業。陛下承運,拓定江表。臣聞治亂討逆,須兵為威,農桑衣食,民之本業,而干戈未戢,民有饑寒。臣愚以為宜育養士民,寬其租賦,衆克在和,義以勸勇,則河渭可平,九有一統矣。」權遂征夷州,得不補失。
 
 
 
 
 及公孫淵背盟,權欲往征,遜上疏曰:「淵憑險恃固,拘留大使,名馬不獻,實可讎忿。蠻夷猾夏,未染王化,鳥竄荒裔,拒逆王師,至令陛下爰赫斯怒,欲勞萬乘汎輕越海,不慮其危而涉不測。方今天下雲擾,羣雄虎爭,英豪踊躍,張聲大視。陛下以神武之姿,誕膺期運,破操烏林,敗備西陵,禽羽荊州,斯三虜者當世雄傑,皆摧其鋒。聖化所綏,萬里草偃,方蕩平華夏,總一大猷。今不忍小忿,而發雷霆之怒,違垂堂之戒,輕萬乘之重,此臣之所惑也。臣聞志行萬里者,不中道而輟足;圖四海者,匪懷細以害大。彊寇在境,荒服未庭,陛下乘桴遠征,必致闚𨵦,慼至而憂,悔之無及。若使大事時捷,則淵不討自服;今乃遠惜遼東衆之與馬,柰何獨欲捐江東萬安之本業而不惜乎?乞息六師,以威大虜,早定中夏,垂曜將來。」權用納焉。
 
 
嘉禾五年,權北征,使遜與諸葛瑾攻襄陽。遜遣親人韓扁齎表奉報,還,遇敵於沔中,鈔邏得扁。瑾聞之甚懼,書與遜云:「大駕已旋,賊得韓扁,具知吾闊狹。且水乾,宜當急去。」遜未荅,方催人種葑豆,與諸將弈棊射戲如常。瑾曰:「伯言多智略,其當有以。」自來見遜,遜曰:「賊知大駕以旋,無所復慼,得專力於吾。又已守要害之處,兵將意動,且當自定以安之,施設變術,然後出耳。今便示退,賊當謂吾怖,仍來相蹙,必敗之勢也。」乃密與瑾立計,令瑾督舟船,遜悉上兵馬,以向襄陽城。敵素憚遜,遽還赴城。瑾便引船出,遜徐整部伍,張拓聲勢,步趨船,敵不敢干。軍到白圍,託言住獵,潛遣將軍周峻、張梁等擊江夏新市、安陸、石陽,石陽市盛,峻等奄至,人皆捐物入城。城門噎不得關,敵乃自斫殺己民,然後得闔。斬首獲生,凡千餘人。
 \gezhu{臣松之以為遜慮孫權已退,魏得專力於己,旣能張拓形勢,使敵不敢犯,方舟順流,無復怵惕矣,何為復潛遣諸將,奄襲小縣,致令市人駭奔,自相傷害?俘馘千人,未足損魏,徒使無辜之民橫罹荼酷,與諸葛渭濵之師,何其殊哉!用兵之道旣違,失律之凶宜應,其祚無三世,及孫而滅,豈此之餘殃哉!}
 其所生得,皆加營護,不令兵士干擾侵侮。將家屬來者,使就料視。若亡其妻子者,即給衣糧,厚加慰勞,發遣令還,或有感慕相攜而歸者。鄰境懷之,
 \gezhu{臣松之以為此無異殘林覆巢而全其遺𪅏,曲惠小仁,何補大虐?}
 江夏功曹趙濯、弋陽備將斐生及夷王梅頤等,並帥支黨來附遜。遜傾財帛,周贍經恤。
 
 
又魏江夏太守逯式
 \gezhu{逯音錄。}
 兼領兵馬,頗作邊害,而與北舊將文聘子休宿不協。遜聞其然,即假作荅式書云:「得報懇惻,知與休乆結嫌隙,勢不兩存,欲來歸附,輒以密呈來書表聞,撰衆相迎。宜潛速嚴,更示定期。」以書置界上,式兵得書以見式,式惶懼,遂自送妻子還洛。由是吏士不復親附,遂以免罷。
 \gezhu{臣松以為邊將為害,蓋其常事,使逯式得罪,代者亦復如之,自非狡焉思肆,將成大患,何足虧損唯慮,尚為小詐哉?以斯為美,又所不取。}
 
 
 
 
 六年,中郎將周祗乞於鄱陽召募,事下問遜。遜以為此郡民易動難安,不可與召,恐致賊寇。而祗固陳取之,郡民吳遽等果作賊殺祗,攻沒諸縣。豫章、廬陵宿惡民,並應遽為寇。遜自聞,輒討即破,遽等相率降,遜料得精兵八千餘人,三郡平。
 
 
 
 
 時中書典校呂壹,竊弄權柄,擅作威福,遜與太常潘濬同心憂之,言至流涕。後權誅壹,深以自責,語在權傳。
 
 
時謝淵、謝厷等各陳便宜,欲興利改作,
 \gezhu{會稽典錄曰:謝淵字休德,少脩德操,躬秉耒耜,旣無慼容,又不易慮,由是知名。舉孝廉,稍遷至建武將軍,雖在戎旅,猶垂意人物。駱統子名秀,被門庭之謗,衆論狐疑,莫能證明。淵聞之歎息曰:「公緒早夭,同盟所哀。聞其子志行明辯,而被闇昧之謗,望諸夫子烈然高斷,而各懷遲疑,非所望也。」秀卒見明,無復瑕玷,終為顯士,淵之力也。吳歷稱云,謝厷才辯有計術。}
 以事下遜。遜議曰:「國以民為本,彊由民力,財由民出。夫民殷國弱,民瘠國彊者,未之有也。故為國者,得民則治,失之則亂,若不受利,而令盡用立效,亦為難也。是以詩歎『宜民宜人,受祿于天』。乞垂聖恩,寧濟百姓,數年之間,國用小豐,然後更圖。」
 
 
 
 
 赤烏七年,代顧雍為丞相,詔曰:「朕以不德,應期踐運,王塗未一,姦宄充路,夙夜戰懼,不惶鑒寐。惟君天資聦叡,明德顯融,統任上將,匡國弭難。夫有超世之功者,必應光大之寵;懷文武之才者,必荷社稷之重。昔伊尹隆湯,呂尚翼周,內外之任,君實兼之。今以君為丞相,使使持節守太常傅常授印綬。君其茂昭明德,脩乃懿績,敬服王命,綏靖四方。於乎!總司三事,以訓羣寮,可不敬與,君其勗之!其州牧都護領武昌事如故。」
 
 
 
 
 先是,二宮並闕,中外職司多遣子弟給侍。全琮報遜,遜以為子弟苟有才,不憂不用,不宜私出以要榮利;若其不佳,終為取禍。且聞二宮勢敵,必有彼此,此古人之厚忌也。琮子寄,果阿附魯王,輕為交構。遜書與琮曰:「卿不師日磾,而宿留阿寄,終為足下門戶致禍矣。」琮旣不納,更以致隙。及太子有不安之議,遜上疏陳:「太子正統,宜有盤石之固,魯王藩臣,當使寵秩有差,彼此得所,上下獲安。謹叩頭流血以聞。」書三四上,及求詣都,欲口論適庶之分,以匡得失。旣不聽許,而遜外生顧譚、顧承、姚信,並以親附太子,枉見流徙。太子太傅吾粲坐數與遜交書,下獄死。權累遣中使責讓遜,遜憤恚致卒,時年六十三,家無餘財。
 
 
 
 
 初,曁豔造營府之論,遜諫戒之,以為必禍。又謂諸葛恪曰:「在我前者,吾必奉之同升;在我下者,則扶持之。今觀君氣陵其上,意蔑乎下,非安德之基也。」又廣陵楊笁少獲聲名,而遜謂之終敗,勸笁兄穆令與別族。其先覩如此。長子延早夭,次子抗襲爵。孫休時,追謚遜曰昭侯。
 
 
 
 
 抗字幼節,孫策外孫也。遜卒時,年二十,拜建武校尉,領遜衆五千人,送葬東還,詣都謝恩。孫權以楊笁所白遜二十事問抗,禁絕賔客,中使臨詰,抗無所顧問,事事條荅,權意漸解。赤烏九年,遷立節中郎將,與諸葛恪換屯柴桑。抗臨去,皆更繕完城圍,葺其墻屋,居廬桑果,不得妄敗。恪入屯,儼然若新。而恪柴桑故屯,頗有毀壞,深以為慙。太元元年,就都治病。病差當還,權涕泣與別,謂曰:「吾前聽用讒言,與汝父大義不篤,以此負汝。前後所問,一焚滅之,莫令人見也。」建興元年,拜奮威將軍。太平二年,魏將諸葛誕舉壽春降,拜抗為柴桑督,赴壽春,破魏牙門將偏將軍,遷征北將軍。永安二年,拜鎮軍將軍,都督西陵,自關羽至白帝。三年,假節。孫皓即位,加鎮軍大將軍,領益州牧。建衡二年,大司馬施績卒,拜抗都督信陵、西陵、夷道、樂鄉,公安諸軍事,治樂鄉。
 
 
 
 
 抗聞都下政令多闕,憂深慮遠,乃上疏曰:「臣聞德均則衆者勝寡,力侔則安者制危,蓋六國所以兼并於彊秦,西楚所以北面於漢高也。今敵跨制九服,非徒關右之地;割據九州,豈但鴻溝以西而已。國家外無連國之援,內非西楚之彊,庶政陵遲,黎民未乂,而議者所恃,徒以長川峻山,限帶封域,此乃書傳之末事,非智者之所先也。臣每遠惟戰國存亡之符,近覽劉氏傾覆之釁,考之典籍,驗之行事,中夜撫枕,臨餐忘食。昔匈奴未滅,去病辭館;漢道未純,賈生哀泣。況臣王室之出,世荷光寵,身名否泰,與國同慼,死生契闊,義無苟且,夙夜憂怛,念至情慘。夫事君之義犯而勿欺,人臣之節匪躬是殉,謹陳時宜十七條如左。」十七條失本,故不載。
 
 
 
 
 時何定弄權,閹官預政;抗上疏曰:「臣聞開國承家,小人勿用,靖譖庸回,唐書攸戒,是以雅人所以怨刺,仲尼所以歎息也。春秋已來,爰及秦、漢,傾覆之釁,未有不由斯者也。小人不明理道,所見旣淺,雖使竭情盡節,猶不足任,況其姦心素篤,而憎愛移易哉?苟患失之,無所不至。今委以聦明之任,假以專制之威,而兾雍熈之聲作,肅清之化立,不可得也。方今見吏,殊才雖少,然或冠冕之冑,少漸道教,或清苦自立,資能足用,自可隨才授職,抑黜羣小,然後俗化可清,庶政無穢也。」
 
 
鳳皇元年,西陵督步闡據城以叛,遣使降晉。抗聞之,日部分諸軍,令將軍左弈、吾彥、蔡貢等徑赴西陵,勑軍營更築嚴圍,自赤谿至故市,內以圍闡,外以禦寇,晝夜催切,如敵以至,衆甚苦之。諸將咸諫曰:「今及三軍之銳,亟以攻闡,比晉救至,闡必可拔。何事於圍,而以弊士民之力乎?」抗曰:「此城處勢旣固,糧穀又足,且所繕脩備禦之具,皆抗所宿規。今反身攻之,旣非可卒克,且北救必至,至而無備,表裏受難,何以禦之?」諸將咸欲攻闡,抗每不許。宜都太守雷譚言至懇切,抗欲服衆,聽令一攻。攻果無利,圍備始合。晉車騎將軍羊祜率師向江陵,諸將咸以抗不宜上,抗曰:「江陵城固兵足,無所憂患。假令敵沒江陵,必不能守,所損者小。如使西陵槃結,則南山羣夷皆當擾動,則所憂慮,難可竟言也。吾寧棄江陵而赴西陵,況江陵牢固乎?」初,江陵平衍,道路通利,抗勑江陵督張咸作大堰遏水,漸漬平中,以絕寇叛。祜欲因所遏水,浮船運糧,揚聲將破堰以通步車。抗聞,使咸亟破之。諸將皆惑,屢諫不聽。祜至當陽,聞堰敗,乃改船以車運,大費損功力。晉巴東監軍徐胤率水軍詣建平,荊州刺史楊肇至西陵。抗令張咸固守其城;公安督孫遵巡南岸禦祜;水軍督留慮、鎮西將軍朱琬拒胤;身率三軍,憑圍對肇。將軍朱喬、營都督俞贊亡詣肇。抗曰:「贊軍中舊吏,知吾虛實者,吾常慮夷兵素不簡練,若敵攻圍,必先此處。」即夜易夷民,皆以舊將充之。明日,肇果攻故夷兵處,抗命旋軍擊之,矢石雨下,肇衆傷死者相屬。肇至經月,計屈夜遁。抗欲追之,而慮闡畜力項領,伺視間隙,兵不足分,於是但鳴鼓戒衆,若將追者。肇衆兇懼,悉解甲挺走,抗使輕兵躡之,肇大破敗,祜等皆引軍還。抗遂陷西陵城,誅夷闡族及其大將吏,自此以下,所請赦者數萬口。脩治城圍,東還樂鄉,貌無矜色,謙沖如常,故得將士歡心。
 \gezhu{晉陽秋曰:抗與羊祜推僑、札之好。抗嘗遺祜酒,祜飲之不疑。抗有疾,祜饋之藥,抗亦推心服之。于時以為華元、子反復見於今。漢晉春秋曰:羊祜旣歸,增脩德信,以懷吳人。陸抗每告其邊戍曰:「彼專為德,我專為暴,是不戰而自服也。各保分界,無求細益而已。」於是吳、晉之間,餘糧栖畝而不犯,牛馬逸而入境,可宣告而取也。沔上獵,吳獲晉人先傷者,皆送而相還。抗嘗疾,求藥於祜,祜以成合與之,曰:「此上藥也,近始自作,未及服,以君疾急,故相致。」抗得而服之,諸將或諫,抗不荅。孫皓聞二境交和,以詰於抗,抗曰:「夫一邑一鄉,不可以無信義之人,而況大國乎?臣不如是,正足以彰其德耳,於祜無傷也。」或以祜、抗為失臣節,兩譏之。習鑿齒曰:夫理勝者天下之所保,信順者萬人之所宗,雖大猷旣喪,義聲乆淪,狙詐馳於當塗,權略周乎急務,負力從橫之人,臧獲牧豎之智,未有不憑此以創功,捨茲而獨立者也。是故晉文退舍,而原城請命;穆子圍鼓,訓之以力;冶夫獻策,而費人斯歸;樂毅緩攻,而風烈長流。觀其所以服物制勝者,豈徒威力相詐而已哉!自今三家鼎足四十有餘年矣,吳人不能越淮、沔而進取中國,中國不能陵長江以爭利者,力均而智侔,道不足以相傾也。夫殘彼而利我,未若利我而無殘;振武以懼物,未若德廣而民懷。匹夫猶不可以力服,而況一國乎?力服猶不如以德來,而況不制乎?是以羊祜恢大同之略,思五兵之則,齊其民人,均其施澤,振義網以羅彊吳,明兼愛以革暴俗,易生民之視聽,馳不戰乎江表。故能德音恱暢,而襁負雲集,殊鄰異域,義讓交弘,自吳之遇敵,未有若此者也。抗見國小主暴,而晉德彌昌,人積兼己之善,而己無固本之規,百姓懷嚴敵之德,闔境有棄主之慮,思所以鎮定民心,緝寧外內,奮其危弱,抗權上國者,莫若親行斯道,以侔其勝。使彼德靡加吾,而此善流聞,歸重邦國,弘明遠風,折衝於枕席之上,校勝於帷幄之內,傾敵而不以甲兵之力,保國而不浚溝池之固,信義感於寇讎,丹懷體於先日。豈設狙詐以危賢,徇己身之私名,貪外物之重我,闇服之而不備者哉!由是論之,苟守局而保疆,一卒之所能;協數以相危,小人之近事;積詐以防物,臧獲之餘慮;威勝以求安,明哲之所賤。賢人君子所以拯世垂範,舍此而取彼者,其道良弘故也。}
 
 
 
 
 加拜都護。聞武昌左部督薛瑩徵下獄,抗上疏曰:「夫俊乂者,國家之良寶,社稷之貴資,庶政所以倫叙,四門所以穆清也。故大司農樓玄、散騎中常侍王蕃、少府李勗,皆當世秀頴,一時顯器,旣蒙初寵,從容列位,而並旋受誅殛,或圮族替祀,或投棄荒裔。蓋周禮有赦賢之辟,春秋有宥善之義,書曰:『與其殺不辜,寧失不經。』而蕃等罪名未定,大辟以加,心經忠義,身被極刑,豈不痛哉!且已死之刑,固無所識,至乃焚爍流漂,棄之水濵,懼非先王之正典,或甫侯之所戒也。是以百姓哀聳,士民同慼。蕃、勗永已,悔亦靡及,誠望陛下赦召玄出,而頃聞薛瑩卒見逮錄。瑩父綜納言先帝,傅弼文皇,及瑩承基,內厲名行,今之所坐,罪在可宥。臣懼有司未詳其事,如復誅戮,益失民望,乞垂天恩,原赦瑩罪,哀矜庶獄,清澄刑網,則天下幸甚!」
 
 
 
 
 時師旅仍動,百姓疲弊,抗上疏曰:「臣聞易貴隨時,傳美觀釁,故有夏多罪而殷湯用師,紂作淫虐而周武授鉞。苟無其時,玉臺有憂傷之慮,孟津有反旆之軍。今不務富國彊兵,力農畜穀,使文武之才效展其用,百揆之署無曠厥職,明黜陟以厲庶尹,審刑賞以示勸沮,訓諸司以德,而撫百姓以仁,然後順天乘運,席卷宇內,而聽諸將徇名,窮兵黷武,動費萬計,士卒彫瘁,寇不為衰,而我已大病矣!今爭帝王之資,而昧十百之利,此人臣之姦便,非國家之良策也。昔齊魯三戰,魯人再克而亡不旋踵。何則?大小之勢異也。況今師所克獲,不補所喪哉?且阻兵無衆,古之明鑒,誠宜蹔息進取小規,以畜士民之力,觀釁伺隙,庶無悔吝。」
 
 
 
 
 二年春,就拜大司馬、荊州牧。三年夏,疾病,上疏曰:「西陵、建平,國之蕃表,旣處下流,受敵二境。若敵汎舟順流,舳艫千里,星奔電邁,俄然行至,非可恃援他部以救倒縣也。此乃社稷安危之機,非徒封疆侵陵小害也。臣父遜昔在西垂陳言,以為西陵國之西門,雖云易守,亦復易失。若有不守,非但失一郡,則荊州非吳有也。如其有虞,當傾國爭之。臣往在西陵,得涉遜迹,前乞精兵三萬,而主者循常,未肯差赴。自步闡以後,益更損耗。今臣所統千里,受敵四處,外禦彊對,內懷百蠻,而上下見兵財有數萬,羸弊日乆,難以待變。臣愚以為諸王幼沖,未統國事,可且立傅相,輔導賢姿,無用兵馬,以妨要務。又黃門豎宦,開立占募,兵民怨役,逋逃入占。乞特詔簡閱,一切料出,以補疆埸受敵常處,使臣所部足滿八萬,省息衆務,信其賞罰,雖韓、白復生,無所展巧。若兵不增,此制不改,而欲克諧大事,此臣之所深慼也。若臣死之後,乞以西方為屬。願陛下思覽臣言,則臣死且不朽。」
 
 
秋遂卒,子晏嗣。晏及弟景、玄、機、雲分領抗兵。晏為裨將軍、夷道監。天紀四年,晉軍伐吳,龍驤將軍王濬順流東下,所至輒克,終如抗慮。景字士仁,以尚公主拜騎都尉,封毗陵侯,旣領抗兵,拜偏將軍、中夏督,澡身好學,著書數十篇也。
 \gezhu{文士傳曰:陸景母張承女,諸葛恪外生。恪誅,景母坐見黜。景少為祖母所育養,及祖母亡,景為之心喪三年。}
 二月壬戌,晏為王濬別軍所殺。癸亥,景亦遇害,時年三十一。景妻,孫皓適妹,與景俱張承外孫也。
 \gezhu{景弟機,字士衡,雲字士龍。機雲別傳曰:晉太康末,俱入洛,造司空張華,華一見而奇之,曰:「伐吳之役,利在獲二儁。」遂為之延譽,薦之諸公。太傅楊駿辟機為祭酒,轉太子洗馬、尚書著作郎。雲為吳王郎中令,出宰浚儀,甚有惠政,吏民懷之,生為立祠。後並歷顯位。機天才綺練,文藻之美,獨冠於時。雲亦善屬文,清新不及機,而口辯持論過之。于時朝廷多故,機、雲並自結於成都王頴。頴用機為平原相,雲清河內史。尋轉雲右司馬,甚見委仗。無幾而與長沙王搆隙,遂舉兵攻洛,以機行後將軍,督王粹、牽秀等諸軍二十萬,士龍著南征賦以美其事。機吳人,羇旅單宦,頓居羣士之右,多不厭服。機屢戰失利,死散過半。初,宦人孟玖,頴所嬖幸,乘寵豫權,雲數言其短,頴不能納,玖又從而毀之。是役也,玖弟超亦領衆配機,不奉軍令。機繩之以法,超宣言曰陸機將反。及牽秀等譖機於頴,以為持兩端,玖又搆之於內,頴信之,遣收機,并收雲及弟耽,並伏法。機兄弟旣江南之秀,亦著名諸夏,並以無罪夷滅,天下痛惜之。機文章為世所重,雲所著亦傳於世。初,抗之克步闡也,誅及嬰孩,識道者尤之曰:「後世必受其殃!」及機之誅,三族無遺,孫惠與朱誕書曰:「馬援擇君,凡人所聞,不意三陸相攜暴朝,殺身傷名,可為悼歎。」事亦並在晉書。}
 
 
 
 
 評曰:劉備天下稱雄,一世所憚,陸遜春秋方壯,威名未著,摧而克之,罔不如志。予旣奇遜之謀略,又歎權之識才,所以濟大事也。及遜忠誠懇至,憂國亡身,庶幾社稷之臣矣。抗貞亮籌幹,咸有父風,弈世載美,具體而微,可謂克構者哉!
 
 
\end{pinyinscope}