\article{孫峻傳}
\begin{pinyinscope}
 
 
 孫峻字子遠,孫堅弟靜之曾孫也。靜生暠。暠生恭,為散騎侍郎。恭生峻。少便弓馬,精果膽決。孫權末,徙武衞都尉,為侍中。權臨薨,受遺輔政,領武衞將軍,故典宿衞,封都鄉侯。旣誅諸葛恪,遷丞相大將軍,督中外諸軍事,假節,進封富春侯。滕胤以恪子竦妻父辭位,峻曰:「鯀禹罪不相及,滕侯何為?」峻、胤雖內不沾洽,而外相苞容,進胤爵高密侯,共事如前。
 
 
\gezhu{吳錄曰:羣臣上奏,共推峻為太尉,議胤為司徒。時有媚峻者,以為大統宜在公族,若滕胤為亞公,聲名素重,衆心所附,不可貳也。乃表以峻為丞相,又不置御史大夫,士人皆失望矣。}
 
 
 
 
 峻素無重名,驕矜險害,多所刑殺,百姓囂然。又姦亂宮人,與公主魯班私通。五鳳元年,吳侯英謀殺峻,英事泄死。
 
 
二年,魏將毌丘儉、文欽以衆叛,與魏人戰于樂嘉,峻帥驃騎將軍呂據、左將軍留贊襲壽春,會欽敗降,軍還。
 \gezhu{吳書曰:留贊字正明,會稽長山人。少為郡吏,與黃巾賊帥吳桓戰,手斬得桓。贊一足被創,遂屈不伸。然性烈,好讀兵書及三史,每覽古良將戰攻之勢,輒對書獨歎,因呼諸近親謂曰:「今天下擾亂,英豪並起,歷觀前世,富貴非有常人,而我屈躄在閭巷之閒,存亡無以異。今欲割引吾足,幸不死而足申,幾復見用,死則已矣。」親戚皆難之。有閒,贊乃以刀自割其筋,血流滂沱,氣絕良乆。家人驚怖,亦以旣爾,遂引申其足。足申創愈,以得蹉步。淩統聞之,請與相見,甚奇之,乃表薦贊,遂被試用。累有戰功,稍遷屯騎校尉。時事得失,每常規諫,好直言不阿旨,權以此憚之。諸葛恪征東興,贊為前部,合戰先陷陣,大敗魏師,遷左將軍。孫峻征淮南,授贊節,拜左護軍。未至壽春,道路病發,峻令贊將車重先還。魏將蔣班以步騎四千追贊。贊病困,不能整陣,知必敗,乃解曲蓋印綬付弟子以歸,曰:「吾自為將,破敵搴旗,未甞負敗。今病困兵羸,衆寡不敵,汝速去矣,俱死無益於國,適所以快敵耳。」弟子不肯受,拔刀欲斫之,乃去。初,贊為將,臨敵必先被髮叫天,因抗音而歌,左右應之,畢乃進戰,戰無不克。及敗,歎曰:「吾戰有常術,今病困若此,固命也!」遂被害,時年七十三,衆庶痛惜焉。二子略、平,並為大將。}
 是歲,蜀使來聘,將軍孫儀、張怡、林恂等欲因會殺峻。事泄,儀等自殺,死者數十人,并及公主魯育。
 
 
 
 
 峻欲城廣陵,朝臣知其不可城,而畏之莫敢言。唯滕胤諫止,不從,而功竟不就。
 
 
 
 
 其明年,文欽說峻征魏,峻使欽與呂據、車騎劉纂、鎮南朱異、前將軍唐咨自江都入淮、泗,以圖青、徐。峻與胤至石頭,因餞之,領從者百許人入據營。據御軍齊整,峻惡之,稱心痛去,遂夢為諸葛恪所擊,恐懼發病死,時年三十八,以後事付綝。
 
 
\end{pinyinscope}