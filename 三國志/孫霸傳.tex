\article{孫霸傳}
\begin{pinyinscope}
 
 
 孫霸字子威,和弟也。和為太子。霸為魯王,寵愛崇特,與和無殊。頃之,和、霸不穆之聲聞於權耳,權禁斷往來,假以精學。督軍使者羊衜上疏曰:「臣聞古之有天下者,皆先顯別適庶,封建子弟,所以尊重祖宗,為國藩表也。二宮拜授,海內稱宜,斯乃大吳興隆之基。頃聞二宮並絕賔客,遠近悚然,大小失望。竊從下風,聽採衆論,咸謂二宮智達英茂,自正名建號,於今三年,德行內著,美稱外昭,西北二隅,乆所服聞。謂陛下當副順遐邇所以歸德,勤命二宮賔延四遠,使異國聞聲,思為臣妾。今旣未垂意於此,而發明詔,省奪備衞,抑絕賔客,使四方禮敬,不復得通,雖實陛下敦尚古義,欲令二宮專志於學,不復顧慮觀聽小宜,期於溫故博物而已,然非臣下傾企喁喁之至願也。或謂二宮不遵典式,此臣所以寢息不寧。就如所嫌,猶宜補察,密加斟酌,不使遠近得容異言。臣懼積疑成謗,乆將宣流,而西北二隅,去國不遠,異同之語,易以聞達。聞達之日,聲論當興,將謂二宮有不順之愆,不審陛下何以解之?若無以解異國,則亦無以釋境內。境內守疑,異國興謗,非所以育巍巍,鎮社稷也。願陛下早發優詔,使二宮周旋禮命如初,則天清地晏,萬國幸甚矣。」
 
 
 
 
 時全寄、吳安、孫奇、楊笁等陰共附霸,圖危太子。譖毀旣行,太子以敗,霸亦賜死。流笁屍于江,兄穆以數諫戒笁,得免大辟,猶徙南州。霸賜死後,又誅寄、安、奇等,咸以黨霸搆和故也。
 
 
 
 
 霸二子,基、壹。五鳳中,封基為吳侯,壹宛陵侯。基侍孫亮在內,太平二年,盜乘御馬,收付獄。亮問侍中刁玄曰:「盜乘御馬罪云何?」玄對曰:「科應死。然魯王早終,惟陛下哀原之。」亮曰:「法者,天下所共,何得阿以親親故邪?當思惟可以釋此者,柰何以情相迫乎?」玄曰:「舊赦有大小,或天下,亦有千里、五百里赦,隨意所及。」亮曰:「解人不當爾邪!」乃赦宮中,基以得免。孫皓即位,追和、霸舊隙,削基、壹爵土,與祖母謝姬俱徙會稽烏傷縣。
 
 
\end{pinyinscope}