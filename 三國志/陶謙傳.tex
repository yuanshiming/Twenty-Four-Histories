\article{陶謙傳}
\begin{pinyinscope}
 
 
 陶謙字恭祖,丹楊人。
 
 
\gezhu{吳書曰:謙父,故餘姚長。謙少孤,始以不羈聞於縣中。年十四,猶綴帛為幡,乘竹馬而戲,邑中兒僮皆隨之。故蒼梧太守同縣甘公出遇之塗,見其容貌,異而呼之,住車與語,甚恱,因許妻以女。甘公夫人聞之,怒曰:「妾聞陶家兒敖戲無度,如何以女許之?」公曰:「彼有奇表,長必大成。」遂妻之。}
 少好學,為諸生,仕州邵,舉茂才,除盧令,
 \gezhu{吳書曰:謙性剛直,有大節,少察孝廉,拜尚書郎,除舒令。郡守張磐,同郡先輩,與謙父友意殊親之,而謙恥為之屈。與衆還城,因以公事進見,坐罷,磐常私還入與謙飲宴,或拒不為留。常以舞屬謙,謙不為起,固彊之;及舞,又不轉。磐曰:「不當轉邪?」曰:「不可轉,轉則勝人。」由是不樂,卒以搆隙。謙在官清白,無以糾舉,祠靈星,有贏錢五百,欲以臧之。謙委官而去。}
 遷幽州剌史,徵拜議郎,參車騎將軍張溫軍事,西討韓遂。
 \gezhu{吳書曰:會西羌寇邊,皇甫嵩為征西將軍,表請武將。召拜謙楊武都尉,與嵩征羌,大破之。後邊章、韓遂為亂,司空張溫銜命征討;又請謙為參軍事,接遇甚厚,而謙輕其行事,心懷不服。及軍罷還,百寮高會,溫屬謙行酒,謙衆辱溫。溫怒,徙謙於邊。或說溫曰:「陶恭祖本以材略見重於公,一朝以醉飲過失,不蒙容貸,遠棄不毛,厚德不終,四方人士安所歸望!不如釋憾除恨,克復初分,於以遠聞德美。」溫然其言,乃追還謙。謙至,或又謂謙曰:「足下輕辱三公,罪自己作,今蒙釋宥,德莫厚矣;宜降志卑辭以謝之。」謙曰:「諾。」又謂溫曰:「陶恭祖今深自罪責,思在變革。謝天子禮畢,必詣公門。公宜見之,以慰其意。」時溫於宮門見謙,謙仰曰:「謙自謝朝廷,豈為公邪?」溫曰:「恭祖癡病尚未除邪?」遂為之置酒,待之如初。}
 會徐州黃巾起,以謙為徐州剌史,擊黃巾,破走之。董卓之亂,州郡起兵,天子都長安,四方斷絕,謙遣使間行致貢獻,遷安東將軍、徐州牧,封溧陽侯。是時,徐州百姓殷盛,穀米封贍,流民多歸之。而謙背道任情:廣陵太守琅邪趙昱,徐方名士也,以忠直見疏;
 \gezhu{謝承漢書曰:昱年十三,母嘗病,經涉三月。昱慘戚消瘠,至目不交睫,握粟出卜,祈禱泣血,鄉黨稱其孝。就處士東莞綦毌君受公羊傳,兼該羣業。至歷年潛志,不闚園圃,親疏希見其面。時入定省父母,須臾即還。高絜廉正,抱禮而立,清英儼恪,莫干其志;旌善以興化,彈邪以矯俗。州郡請召,常稱病不應。國相檀謨、陳遵比召,不起;或興盛怒,終不迴意。舉孝廉,除莒長,宣揚五教,政為國表。會黃巾作亂,陸梁五郡,郡縣發兵,以為先辦。徐州刺史巴祗表功第一,當受遷賞,昱深以為恥,委官還家。徐州牧陶謙初辟別駕從事,辭疾遜遁。謙重令楊州從事會稽吳範宣旨,昱守意不移;欲威以刑罰,然後乃起。舉茂才,遷廣陵太守。賊笮融從臨淮見討,迸入郡界,昱將兵拒戰,敗績,見害。}
 曹宏等,讒慝小人也,謙親任之。刑政失和,良善多被其害,由是漸亂。下邳闕宣自稱天子,謙初與合從寇鈔,後遂殺宣,并其衆。
 
 
初平四年,太祖征謙,攻拔十餘城,至彭城大戰。謙兵敗走,死者萬數,泗水為之不流。謙退守剡。太祖以糧少引軍還。
 \gezhu{吳書曰:曹公父於泰山被殺,歸咎於謙。欲伐謙而畏其彊,乃表令州郡一時罷兵。詔曰:「今海內擾攘,州郡起兵,征夫勞瘁,寇難未弭,或將吏不良,因緣討捕,侵侮黎民,離害者衆;風聲流聞,震蕩城邑,丘牆懼於橫暴,貞良化為羣惡,此何異乎抱薪救焚,扇火止沸哉!今四民流移,託身佗方,攜白首於山野,棄稚子於溝壑,顧故鄉而哀歎,向阡陌而流涕,饑厄困苦,亦已甚矣。雖悔往者之迷謬,思奉教於今日,然兵連衆結,鋒鏑布野,恐一朝解散,夕見係虜,是以阻兵屯據,欲止而不敢散也。詔書到,其各罷遣甲士,還親農桑,惟留常員吏以供官署,慰示遠近,咸使聞知。」謙被詔,乃上書曰:「臣聞懷遠柔服,非德不集;克難平亂,非兵不濟。是以涿鹿、版泉、三苗之野有五帝之師,有扈、鬼方、商、奄四國有王者之伐,自古在昔,未有不揚威以弭亂,震武以止暴者也。臣前初以黃巾亂治,受策長驅,匪遑啟處。雖憲章勑戒,奉宣威靈,敬行天誅,每伐輒克,然妖寇類衆,殊不畏死,父兄殲殪,子弟羣起,治屯連兵,至今為患。若承命解甲,弱國自虛,釋武備以資亂,損官威以益寇,今日兵罷,明日難必至,上忝朝廷寵授之本,下令羣凶日月滋蔓,非所以彊幹弱枝遏惡止亂之務也。臣雖愚蔽,忠恕不昭,抱恩念報,所不忍行。輒勒部曲,申令警備。出芟彊寇,惟力是視,入宣德澤,躬奉職事,兾效微勞,以贖罪負。」又曰:「華夏沸擾,于今未弭,包茅不入,職貢多闕,寤寐憂歎,無日敢寧。誠思貢獻必至,薦羞獲通,然後銷鋒解甲,臣之願也。臣前調穀百萬斛,已在水次,輒勑兵衞送。」曹公得謙上事,知不罷兵。乃進攻彭城,多殺人民。謙引兵擊之,青州刺史田楷亦以兵救謙。公引兵還。臣松之案:此時天子在長安,曹公尚未秉政。罷兵之詔,不得由曹氏出。}
 
 
興平元年,復東征,略定琅邪、東海諸縣。謙恐,欲走歸丹楊。會張邈叛迎呂布,太祖還擊布。是歲,謙病死。
 \gezhu{吳書曰:謙死時,年六十三,張昭等為之哀辭曰:「猗歟使君,君侯將軍,膺秉懿德,允武允文,體足剛直,守以溫仁。令舒及盧,遺愛于民;牧幽曁徐,甘棠是均。憬憬夷貊,賴侯以清;蠢蠢妖寇,匪侯不寧。唯帝念績,爵命以章,旣牧且侯,啟土溧陽。遂升上將,受號安東,將平世難,社稷是崇。降年不永,奄忽殂薨,喪覆失恃,民知困窮。曾不旬日,五郡潰崩,哀我人斯,將誰仰憑?追思靡及,仰叫皇穹。嗚呼哀哉!」謙二子:商、應,皆不仕。}
 
 
\end{pinyinscope}