\article{常林傳}
\begin{pinyinscope}
 
 
 常林字伯槐,河內溫人也。年七歲,有父黨造門,問林:「伯先在否?汝何不拜!」林曰:「雖當下客,臨子字父,何拜之有?」於是咸共嘉之。
 
 
\gezhu{魏略曰:林少單貧。雖貧,自非手力,不取之於人。性好學,漢末為諸生,帶經耕鉏。其妻常自餽餉之,林雖在田野,其相敬如賔。}
 太守王匡起兵討董卓,遣諸生於屬縣微伺吏民罪負,便收之,考責錢穀贖罪,稽遲則夷滅宗族,以崇威嚴。林叔父檛客,為諸生所白,匡怒收治。舉宗惶怖,不知所責多少,懼繫者不救。林往見匡同縣胡母彪曰:「王府君以文武高才,臨吾鄙郡。鄙郡表裏山河,土廣民殷,又多賢能,惟所擇用。今主上幼沖,賊臣虎據,華夏震慄,雄才奮用之秋也。若欲誅天下之賊,扶王室之微,智者望風,應之若響,克亂在和,何征不捷。苟無恩德,任失其人,覆亡將至,何暇匡翼朝廷,崇立功名乎?君其藏之!」因說叔父見拘之意。彪即書責匡,匡原林叔父。林乃避地上黨,耕種山阿。當時旱蝗,林獨豐收,盡呼比鄰,升斗分之。依故河閒太守陳延壁。陳、馮二姓,舊族冠冕。張楊利其婦女,貪其資貨。林率其宗族,為之策謀。見圍六十餘日,卒全堡壁。
 
 
 
 
 并州刺史高幹表為騎都尉,林辭不受。後刺史梁習薦州界名士林及楊俊、王淩、王象、荀緯,太祖皆以為縣長。林宰南和,治化有成,超遷博陵太守、幽州刺史,所在有績。文帝為五官將,林為功曹。太祖西征,田銀、蘇伯反,幽、兾扇動。文帝欲親自討之,林曰:「昔忝博陵,又在幽州,賊之形勢,可料度也。北方吏民,樂安厭亂,服化已乆,守善者多。銀、伯犬羊相聚,智小謀大,不能為害。方今大軍在遠,外有彊敵,將軍為天下之鎮也,輕動遠舉,雖克不武。」文帝從之,遣將往伐,應時克滅。
 
 
出為平原太守、魏郡東部都尉,入為丞相東曹屬。魏國旣建,拜尚書。文帝踐阼,遷少府,封樂陽亭侯,
 \gezhu{魏略曰:林性旣清白,當官又嚴。少府寺與鴻臚對門,時崔林為鴻臚。崔性闊達,不與林同,數數聞林撾吏聲,不以為可。林夜撾吏,不勝痛,叫呼敖敖徹曙。明日,崔出門,與林車相遇,乃啁林曰:「聞卿為廷尉,爾邪?」林不覺荅曰:「不也。」崔曰:「卿不為廷尉,昨夜何故考囚乎?」林大慙,然不能自止。}
 轉大司農。明帝即位,進封高陽鄉侯,徙光祿勳太常。晉宣王以林鄉邑耆德,每為之拜。或謂林曰:「司馬公貴重,君宜止之。」林曰:「司馬公自欲敦長幼之叙,為後生之法。貴非吾之所畏,拜非吾之所制也。」言者踧踖而退。
 \gezhu{魏略曰:初,林少與司馬京兆善。太傅每見林,輒欲跪。林止之曰:「公尊貴矣,止也!」及司徒缺,太傅有意欲以林補之。案魏略此語,與本傳反。臣松之以為林之為人,不畏權貴者也。論其然否,謂本傳為是。}
 時論以林節操清峻,欲致之公輔,而林遂稱疾篤。拜光祿大夫。年八十三,薨,追贈驃騎將軍,葬如公禮,謚曰貞侯。子峕嗣,為泰山太守,坐法誅。峕弟靜紹封。
 \gezhu{案晉書,諸葛誕反,大將軍東征,峕坐稱疾,為司馬文王所法。魏略以林及吉茂、沐並、時苗四人為清介傳。吉茂字叔暢,馮翊池陽人也,世為著姓。好書,不耻惡衣惡食,而耻一物之不知。建安初,關中始平,茂與扶風蘇則共入武功南山,隱處精思數歲。州舉茂才,除臨汾令,居官清靜,吏民不忍欺。轉為武德侯庶子。二十二年,坐其宗人吉本等起事被收。先是科禁內學及兵書,而茂皆有,匿不送官。及其被收,不知當坐本等,顧謂其左右曰:「我坐書也。」會鍾相國證茂、本服弟已絕,故得不坐。後以茂為武陵太守,不之官。轉酇相,以國省,拜議郎。景初中病亡。自茂脩行,從少至長,冬則被裘,夏則裋褐,行則步涉,食則茨藿,臣役妻子,室如懸磬。其或饋遺,一不肯受。雖不以此高人,亦心疾不義而貴且富者。先時國家始制九品,各使諸郡選置中正,差叙自公卿以下,至于郎吏,功德材行所任。茂同郡護羌校尉王琰,前數為郡守,不名為清白。而琰子嘉仕歷諸縣,亦復為通人。嘉時還為散騎郎,馮翊郡移嘉為中正。嘉叙茂雖在上第,而狀甚下,云:「德優能少。」茂慍曰:「痛乎,我效汝父子冠幘劫人邪!」初,茂同產兄黃,以十二年中從公府掾為長陵令。是時科禁長吏擅去官,而黃聞司徒趙溫薨,自以為故吏,違科奔喪,為司隷鍾繇所收,遂伏法。茂時為白衣,始有清名於三輔,以為兄坐追義而死,怨怒不肯哭。至歲終,繇舉茂。議者以為茂必不就,及舉旣到而茂就之,故時人或以茂為畏繇,或以茂為髦士也。沐並字德信,河間人也。少孤苦,袁紹父子時,始為名吏。有志介,嘗過姊,姊為殺雞炊黍而不留也。然為人公果,不畏彊禦,丞相召署軍謀掾。黃初中,為成臯令。校事劉肇出過縣,遣人呼縣吏,求索槀穀。是時蝗旱,官無有見。未辦之間,肇人從入並之閤下,呴呼罵吏。並怒,因躧履提刀而出,多從吏,並欲收肇。肇覺知驅走,具以狀聞。有詔:「肇為牧司爪牙吏,而並欲收縛,無所忌憚,自恃清名邪?」遂收欲殺之。髠決減死,刑竟復吏,由是放散十餘年。至正始中,為三府長史。時吳使朱然、諸葛瑾攻圍樊城,遣船兵於峴山東斫材,牂牁人兵作食,有先熟者呼後熟者,言:「共食來。」後熟者荅言:「不也。」呼者曰:「汝欲作沐德信邪?」其名流布,播於異域如此。雖自華夏,不知者以為前世人也。為長史八年,晚出為濟陰太守,召還,拜議郎。年六十餘,自慮身無常,豫作終制,誡其子以儉葬,曰:「告雲、儀等:夫禮者,生民之始教,而百世之中庸也。故力行者則為君子,不務者終為小人,然非聖人莫能履其從容也。是以富貴者有驕奢之過,而貧賤者譏於固陋,於是養生送死,苟切非禮。由斯觀之,陽虎璵璠,甚於暴骨,桓魋石椁,不如速朽。此言儒學撥亂反正、鳴鼓矯俗之大義也,未是夫窮理盡性、陶冶變化之實論也。若能原始要終,以天地為一區,萬物為芻狗,該覽玄通,求形景之宗,同禍福之素,一死生之命,吾有慕於道矣。夫道之為物,惟怳惟忽,壽為欺魄,夭為鳧沒,身淪有無,與神消息,含恱陰陽,甘夢太極。奚以棺椁為牢,衣裳為纏?屍繫地下,長幽桎梏,豈不哀哉!昔莊周闊達,無所適莫;又楊王孫裸體,貴不乆容耳。至夫末世,緣生怨死之徒,乃有含珠鱗柙,玉牀象袵,殺人以徇;壙穴之內,錮以紵絮,藉以蜃炭,千載僵燥,託類神仙。於是大教陵遲,競於厚葬,謂莊子為放蕩,以王孫為戮屍,豈復識古有衣薪之鬼,而野有狐狸之胔乎哉?吾以材質滓濁,汙於清流。昔忝國恩,歷試宰守,所在無效,代匠傷指,狼跋首尾,無以雪耻。如不可求,從吾所好。今年過耳順,奄忽無常,苟得獲沒,即以吾身襲於王孫矣。上兾以贖巿朝之逋罪,下以親道化之靈祖。顧爾幼昏,未知臧否,若將逐俗,抑廢吾志,私稱從令,未必為孝;而犯魏顆聽治之賢,爾為棄父之命,誰或矜之!使死而有知,吾將屍視。」至嘉平中,病甚。臨困,又勑豫掘埳。戒氣絕,令二人舉屍即埳,絕哭泣之聲,止婦女之送,禁弔祭之賔,無設摶治粟米之奠。又戒後亡者不得入藏,不得封樹。妻子皆遵之。時苗字德冑,鉅鹿人也。少清白,為人疾惡。建安中,入丞相府。出為壽春令,令行風靡。揚州治在其縣,時蔣濟為治中。苗以初至往謁濟,濟素嗜酒,適會其醉,不能見苗。苗恚恨還,刻木為人,署曰「酒徒蔣濟」,置之牆下,旦夕射之。州郡雖知其所為不恪,然以其履行過人,無若之何。又其始之官,乘薄軬音飯車,黃牸牛,布被囊。居官歲餘,牛生一犢。及其去,留其犢,謂主簿曰:「令來時本無此犢,犢是淮南所生有也。」羣吏曰:「六畜不識父,自當隨母。」苗不聽,時人皆以為激,然由此名聞天下。還為太官令,領其郡中正,定九品,於叙人才不能寬,然紀人之短,雖在乆遠,銜之不置。如所忿蔣濟者,仕進至太尉,濟不以苗前毀己為嫌,苗亦不以濟貴更屈意。為令數歲,不肅而治。遷典農中郎將。年七十餘,以正始中病亡也。}
 
 
\end{pinyinscope}