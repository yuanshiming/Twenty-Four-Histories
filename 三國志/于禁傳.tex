\article{于禁傳}
\begin{pinyinscope}
 
 
 于禁字文則,泰山鉅平人也。黃巾起,鮑信招合徒衆,禁附從焉。及太祖領兖州,禁與其黨俱詣為都伯,屬將軍王朗。朗異之,薦禁才任大將軍。太祖召見與語,拜軍司馬,使將兵詣徐州,攻廣威,拔之,拜陷陣都尉。從討呂布於濮陽,別破布二營於城南,又別將破高雅於須昌。從攻壽張、定陶、離狐,圍張超於雍丘,皆拔之。從征黃巾劉辟、黃邵等,屯版梁,邵等夜襲太祖營,禁帥麾下擊破之,斬辟、邵等,盡降其衆。遷平虜校尉。從圍橋蕤於苦,斬蕤等四將。從至宛,降張繡。繡復叛,太祖與戰不利,軍敗,還舞陰。是時軍亂,各間行求太祖,禁獨勒所將數百人,且戰且引,雖有死傷不相離。虜追稍緩,禁徐整行隊,鳴鼓而還。未至太祖所,道見十餘人被創裸走,禁問其故,曰:「為青州兵所劫。」初,黃巾降,號青州兵,太祖寬之,故敢因緣為略。禁怒,令其衆曰:「青州兵同屬曹公,而還為賊乎!」乃討之,數之以罪。青州兵遽走詣太祖自訴。禁旣至,先立營壘,不時謁太祖。或謂禁:「青州兵已訴君矣,宜促詣公辨之。」禁曰:「今賊在後,追至無時,不先為備,何以待敵?且公聦明,譖訴何緣!」徐鑿塹安營訖,乃入謁,具陳其狀。太祖恱,謂禁曰:「淯水之難,吾其急也,將軍在亂能整,討暴堅壘,有不可動之節,雖古名將,何以加之!」於是錄禁前後功,封益壽亭侯。復從攻張繡於穰,禽呂布於下邳,別與史渙、曹仁攻眭固於射犬,破斬之。
 
 
 
 
 太祖初征袁紹,紹兵盛,禁願為先登。太祖壯之,乃選步騎二千人,使禁將,守延津以拒紹,太祖引軍還官渡。劉備以徐州叛,太祖東征之。紹攻禁,禁堅守,紹不能拔。復與樂進等將步騎五千,擊紹別營,從延津西南緣河至汲、獲嘉二縣,焚燒保聚三十餘屯,斬首獲生各數千,降紹將何茂、王摩等二十餘人。太祖復使禁別將屯原武,擊紹別營於杜氏津,破之。遷裨將軍,後從還官渡。太祖與紹連營,起土山相對。紹射營中,士卒多死傷,軍中懼。禁督守土山,力戰,氣益奮。紹破,遷偏將軍。兾州平。昌豨復叛,遣禁征之。禁急進攻豨;豨與禁有舊,詣禁降。諸將皆以為豨已降,當送詣太祖,禁曰:「諸君不知公常令乎!圍而後降者不赦。夫奉法行令,事上之節也。豨雖舊友,禁可失節乎!」自臨與豨決,隕涕而斬之。是時太祖軍淳于,聞而歎曰:「豨降不詣吾而歸禁,豈非命耶!」益重禁。
 
 
\gezhu{臣松之以為圍而後降,法雖不赦;囚而送之,未為違命。禁曾不為舊交希兾萬一,而肆其好殺之心,以戾衆人之議,所以卒為降虜,死加惡謚,宜哉。}
 東海平,拜禁虎威將軍。後與臧霸等攻梅成,張遼、張郃等討陳蘭。禁到,成舉衆三千餘人降。旣降復叛,其衆奔蘭。遼等與蘭相持,軍食少,禁運糧前後相屬,遼遂斬蘭、成。增邑二百戶,并前千二百戶。是時,禁與張遼、樂進、張郃、徐晃俱為名將,太祖每征伐,咸遞行為軍鋒,還為後拒;而禁持軍嚴整,得賊財物,無所私入,由是賞賜特重。然以法御下,不甚得士衆心。太祖常恨朱靈,欲奪其營。以禁有威重,遣禁將數十騎,齎令書,徑詣靈營奪其軍,靈及其部衆莫敢動;乃以靈為禁部下督,衆皆震服,其見憚如此。遷左將軍,假節鉞,分邑五百戶,封一子列侯。
 
 
建安二十四年,太祖在長安,使曹仁討關羽於樊,又遣禁助仁。秋,大霖雨,漢水溢,平地水數丈,禁等七軍皆沒。禁與諸將登高望水,無所回避,羽乘大船就攻禁等,禁遂降,惟龐悳不屈節而死。太祖聞之,哀歎者乆之,曰:「吾知禁三十年,何意臨危處難,反不如龐悳邪!」會孫權禽羽,獲其衆,禁復在吳。文帝踐阼,權稱藩,遣禁還。帝引見禁,鬚髮皓白,形容憔顇,泣涕頓首。帝慰喻以荀林父、孟明視故事,
 \gezhu{魏書載制曰:「昔荀林父敗績於邲,孟明喪師於殽,秦、晉不替,使復其位。其後晉獲狄土,秦霸西戎,區區小國,猶尚若斯,而況萬乘乎?樊城之敗,水災暴至,非戰之咎,其復禁等官。」}
 拜為安遠將軍。欲遣使吳,先令北詣鄴謁高陵。帝使豫於陵屋畫關羽戰克、龐悳憤怒、禁降服之狀。禁見,慙恚發病薨。子圭嗣,封益壽亭侯。謚禁曰厲侯。
 
 
\end{pinyinscope}