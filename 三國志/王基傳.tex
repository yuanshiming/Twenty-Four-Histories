\article{王基傳}
\begin{pinyinscope}
 
 
 王基字伯輿,東萊曲城人也。少孤,與叔父翁居。翁撫養甚篤,基亦以孝稱。年十七,郡召為吏,非其好也,遂去,入琅邪界游學。黃初中,察孝廉,除郎中。是時青土初定,刺史王凌特表請基為別駕,後召為秘書郎,凌復請還。頃之,司徒王朗辟基,淩不遣。朗書劾州曰:「凡家臣之良,則升于公輔,公臣之良,則入于王職,是故古者侯伯有貢士之禮。今州取宿衞之臣,留秘閣之吏,所希聞也。」淩猶不遣。淩流稱青土,蓋亦由基恊和之輔也。大將軍司馬宣王辟基,未至,擢為中書侍郎。
 
 
 
 
 明帝盛脩宮室,百姓勞瘁。基上疏曰:「臣聞古人以水喻民,曰『水所以載舟,亦所以覆舟』。故在民上者,不可以不戒懼。夫民逸則慮易,苦則思難,是以先王居之以約儉,俾不至於生患。昔顏淵云東野子之御,馬力盡矣而求進不已,是以知其將敗。今事役勞苦,男女離曠,願陛下深察東野之弊,留意舟水之喻,息奔駟於未盡,節力役於未困。昔漢有天下,至孝文時唯有同姓諸侯,而賈誼憂之曰:『置火積薪之下而寢其上,因謂之安也。』今寇賊未殄,猛將擁兵,檢之則無以應敵,乆之則難以遺後,當盛明之世,不務以除患,若子孫不競,社稷之憂也。使賈誼復起,必深切於曩時矣。」
 
 
 
 
 散騎常侍王肅著諸經傳解及論定朝儀,改易鄭玄舊說,而基據持玄義,常與抗衡。遷安平太守,公事去官。大將軍曹爽請為從事中郎,出為安豐太守。郡接吳寇,為政清嚴有威惠,明設防備,敵不敢犯。加討寇將軍。吳嘗大發衆集建業,揚聲欲入攻揚州,刺史諸葛誕使基策之。基曰:「昔孫權再至合肥,一至江夏,其後全琮出廬江,朱然寇襄陽,皆無功而還。今陸遜等已死,而權年老,內無賢嗣,中無謀主。權自出則懼內釁卒起,癕疽發潰;遣將則舊將已盡,新將未信。此不過欲補定支黨,還自保護耳。」後權竟不能出。時曹爽專柄,風化陵遲,基著時要論以切世事。以疾徵還,起家為河南尹,未拜,爽伏誅,基甞為爽官屬,隨例罷。
 
 
 
 
 其年為尚書,出為荊州刺史,加揚烈將軍,隨征南王昶擊吳。基別襲步恊於夷陵,恊閉門自守。基示以攻形,而實分兵取雄父邸閣,收米三十餘萬斛,虜安北將軍譚正,納降數千口。於是移其降民,置夷陵縣。賜爵關內侯。基又表城上昶,徙江夏治之,以偪夏口,由是賊不敢輕越江。明制度,整軍農,兼脩學校,南方稱之。時朝廷議欲伐吳,詔基量進趣之宜。基對曰:「夫兵動而無功,則威名折於外,財用窮於內,故必全而後用也。若不資通川聚糧水戰之備,則雖積兵江內,無必渡之勢矣。今江陵有沮、漳二水,溉灌膏腴之田以千數。安陸左右,陂池沃衍。若水陸並農,以實軍資,然後引兵詣江陵、夷陵,分據夏口,順沮、漳,資水浮糓而下。賊知官兵有經乆之勢,則拒天誅者意沮,而向王化者益固。然後率合蠻夷以攻其內,精卒勁兵以討其外,則夏口以上必拔,而江外之郡不守。如此,吳、蜀之交絕,交絕而吳禽矣。不然,兵出之利,未可必矣。」於是遂止。
 
 
 
 
 司馬景王新統政,基書戒之曰:「天下至廣,萬機至猥,誠不可不矜矜業業,坐而待旦也。夫志正則衆邪不生,心靜則衆事不躁,思慮審定則教令不煩,親用忠良則遠近恊服。故知和遠在身,定衆在心。許允、傅嘏、袁侃、崔贊皆一時正士,有直質而無流心,可與同政事者也。」景王納其言。
 
 
 
 
 高貴鄉公即尊位,進封常樂亭侯。毌丘儉、文欽作亂,以基為行監軍、假節,統許昌軍,適與景王會於許昌。景王曰:「君籌儉等何如?」基曰:「淮南之逆,非吏民思亂也,儉等誑脅迫懼,畏目下之戮,是以尚群聚耳。若大兵臨偪,必土崩瓦解,儉、欽之首,不終朝而縣於軍門矣。」景王曰:「善。」乃令基居軍前。議者咸以儉、欽慓悍,難與爭鋒。詔基停駐。基以為:「儉等舉軍足以深入,而乆不進者,是其詐偽已露,衆心疑沮也。今不張示威形以副民望,而停軍高壘,有似畏懦,非用兵之勢也。若或虜略民人,又州郡兵家為賊所得者,更懷離心;儉等所迫脅者,自顧罪重,不敢復還,此為錯兵無用之地,而成姦宄之源。吳寇因之,則淮南非國家之有,譙、沛、汝、豫危而不安,此計之大失也。軍宜速進據南頓,南頓有大邸閣,計足軍人四十日糧。保堅城,因積穀,先人有奪人之心,此平賊之要也。」基屢請,乃聽進據㶏水。旣至,復言曰:「兵聞拙速,未覩工遟之乆。方今外有彊寇,內有叛臣,若不時決,則事之深淺未可測也。議者多欲將軍持重。將軍持重是也,停軍不進非也。持重非不行之謂也,進而不可犯耳。今據堅城,保壁壘,以積實資虜,縣運軍糧,甚非計也。」景王欲須諸軍集到,猶尚未許。基曰:「將在軍,君令有所不受。彼得則利,我得亦利,是謂爭城,南頓是也。」遂輙進據南頓,儉等從項亦爭欲往,發十餘里,聞基先到,復還保項。時兖州刺史鄧艾屯樂嘉,儉使文欽將兵襲艾。基知其勢分,進兵偪項,儉衆遂敗。欽等已平,遷鎮南將軍,都督豫州諸軍事,領豫州刺史,進封安樂鄉侯。上疏求分戶二百,賜叔父子喬爵關內侯,以報叔父拊育之德。有詔特聽。
 
 
 
 
 諸葛誕反,基以本官行鎮東將軍,都督揚、豫諸軍事。時大軍在項,以賊兵精,詔基斂軍堅壘。基累啟求進討。會吳遣朱異來救誕,軍於安城。基又被詔引諸軍轉據北山,基謂諸將曰:「今圍壘轉固,兵馬向集,但當精修守備以待越逸,而更移兵守險,使得放縱,雖有智者不能善後矣。」遂守便宜上疏曰:「今與賊家對敵,當不動如山。若遷移依險,人心搖蕩,於勢大損。諸軍並據深溝高壘,衆心皆定,不可傾動,此御兵之要也。」書奏,報聽。大將軍司馬文王進屯丘頭,分部圍守,各有所統。基督城東城南二十六軍,文王勑軍吏入鎮南部界,一不得有所遣。城中食盡,晝夜攻壘,基輙拒擊,破之。壽春旣拔,文王與基書曰:「初議者云云,求移者甚衆,時未臨履,亦謂宜然。將軍深筭利害,獨秉固志,上違詔命,下拒衆議,終至制敵禽賊,雖古人所述,不是過也。」文王欲遣諸將輕兵深入,招迎唐咨等子弟,因釁有蕩覆吳之勢。基諫曰:「昔諸葛恪乘東關之勝,竭江表之兵,以圍新城,城旣不拔,而衆死者太半。姜維因洮上之利,輕兵深入,糧餉不繼,軍覆上邽。夫大捷之後,上下輕敵,輕敵則慮難不深。今賊新敗於外,又內患未弭,是其脩備設慮之時也。且兵出踰年,人有歸志,今俘馘十萬,罪人斯得,自歷代征伐,未有全兵獨克如今之盛者也。武皇帝克袁紹於官渡,自以所獲已多,不復追奔,懼挫威也。」文王乃止。以淮南初定,轉基為征東將軍,都督揚州諸軍事,進封東武侯。基上疏固讓,歸功參佐,由是長史司馬等七人皆侯。
 
 
 
 
 是歲,基母卒,詔祕其凶問,迎基父豹喪合葬洛陽,追贈豹北海太守。甘露四年,轉為征南將軍,都督荊州諸軍事。常道鄉公即尊位,增邑千戶,并前五千七百戶。前後封子二人亭侯、關內侯。
 
 
 
 
 景元二年,襄陽太守表吳賊鄧由等欲來歸化,基被詔,當因此震蕩江表。基疑其詐,馳驛陳狀。且曰:「嘉平以來,累有內難,當今之務,在於鎮安社稷,綏寧百姓,未宜動衆以求外利。」文王報書曰:「凡處事者,多曲相從順,鮮能確然共盡理實。誠感忠愛,每見規示,輙敬依來指。」後由等竟不降。
 
 
\gezhu{司馬彪戰畧載基此事,詳於本傳。曰:「景元二年春三月,襄陽太守胡烈表上『吳賊鄧由、李光等,同謀十八屯,欲來歸化,遣將張吳、鄧生,并送質任。克期欲令郡軍臨江迎拔』。大將軍司馬文王啟聞。詔征南將軍王基部分諸軍,使烈督萬人徑造沮水,荊州、義陽南屯宜城,承書夙發。若由等如期到者,便當因此震蕩江表。基疑賊詐降,誘致官兵,馳驛止文王,說由等可疑之狀。『且當清澄,未宜便舉重兵深入應之』。又曰:『夷陵東道,當由車御,至赤岸乃得渡坦,西道當出箭谿口,乃趣平土,皆山險狹,竹木叢蔚,卒有要害,弩馬不陳。今者筋角弩弱,水潦方降,廢盛農之務,徼難必之利,此事之危者也。昔子午之役,兵行數百里而值霖雨,橋閣破壞,後糧腐敗,前軍縣乏。姜維深入,不待輜重,士衆飢餓,覆軍上邽。文欽、唐咨,舉吳重兵,昧利壽春,身歿不反。此皆近事之鑒戒也。嘉平以來,累有內難。當今之宜,當鎮安社稷,撫寧上下,力農務本,懷柔百姓,未宜動衆以求外利也。得之未足為多,失之傷損威重。』文王累得基書,意疑。尋勑諸軍已上道者,且權停住所在,須後節度。基又言於文王曰:『昔漢祖納酈生之說,欲封六國,寤張良之謀,而趣銷印。基謀慮淺短,誠不及留侯,亦懼襄陽有食其之謬。』文王於是遂罷軍嚴,後由等果不降。」}
 
 
 
 
 是歲基薨,追贈司空,謚曰景侯。子徽嗣,早卒。咸熈中,開建五等,以基著勳前朝,改封基孫廙,而以東武餘邑賜一子爵關內侯。晉室踐阼,下詔曰:「故司空王基旣著德立勳,又治身清素,不營產業,乆在重任,家無私積,可謂身沒行顯,足用勵俗者也。其以奴婢二人賜其家。」
 
 
 
 
 評曰:徐邈清尚弘通,胡質素業貞粹,王昶開濟識度,王基學行堅白,皆掌統方任,垂稱著績。可謂國之良臣,時之彥士矣。
 
 
\end{pinyinscope}