\article{孫綝傳}
\begin{pinyinscope}
 
 
 孫綝字子通,與峻同祖。綝父綽為安民都尉。綝始為偏將軍,及峻死,為侍中武衞將軍,領中外諸軍事,代知朝政。呂據聞之大恐,與諸督將連名,共表薦滕胤為丞相,綝更以胤為大司馬,代呂岱駐武昌。據引兵還,使人報胤,欲共廢綝。綝聞之,遣從兄慮將兵逆據於江都,使中使勑文欽、劉纂、唐咨等合衆擊據,遣侍中左將軍華融、中書丞丁晏告胤取據,并喻胤宜速去意。胤自以禍反,因留融、晏,勒兵自衞,召典軍楊崇、將軍孫咨,告以綝為亂,迫融等使有書難綝。綝不聽,表言胤反,許將軍劉丞以封爵,使率兵騎急攻圍胤。胤又劫融等,使詐詔發兵。融等不從,胤皆殺之。
 
 
\gezhu{文士傳曰:華融字德蕤,廣陵江都人。祖父避亂,居山陰蕋山下。時皇象亦寓居山陰,吳郡張溫來就象學,欲得所舍。或告溫曰:「蕋山下有華德蕤者,雖年少,美有令志,可舍也。」溫遂止融家,朝夕談講。俄而溫為選部尚書,乃擢融為太子庶子,遂知名顯達。融子諝,黃門郎,與融并見害。次子譚,以才辯稱,晉祕書監。}
 胤顏色不變,談笑若常。或勸胤引兵至蒼龍門,將士見公出,必皆委綝就公。時夜已半,胤恃與據期,又難舉兵向宮,乃約令部典,說呂侯以在近道,故皆為胤盡死,無離散者。時大風,比曉,據不至。綝兵大會,遂殺胤及將士數十人,夷胤三族。
 \gezhu{臣松之以為孫綝雖凶虐,與滕胤宿無嫌隙,胤若且順綝意,出鎮武昌,豈徒免當時之禍,仍將永保元吉,而犯機觸害,自取夷滅,悲夫!}
 
 
 
 
 綝遷大將軍,假節,封永寧侯,負貴倨傲,多行無禮。初,峻從弟慮與誅諸葛恪之謀,峻厚之,至右將軍、無難督,授節蓋,平九官事。綝遇慮薄於峻時,慮怒,與將軍王惇謀殺綝。綝殺惇,慮服藥死。
 
 
 
 
 魏大將軍諸葛誕舉壽春叛,保城請降。吳遣文欽、唐咨、全端、全懌等帥三萬人救之。魏鎮南將軍王基圍誕,欽等突圍入城。魏悉中外軍二十餘萬增誕之圍。朱異帥三萬人屯安豐城,為文欽勢。魏兖州刺史州泰拒異於陽淵,異敗退,為泰所追,死傷二千人。綝於是大發卒出屯鑊里,復遣異率將軍丁奉、黎斐等五萬人攻魏,留輜重於都陸。異屯黎漿,遣將軍任度、張震等募勇敢六千人,於屯西六里為浮橋夜渡,築偃月壘。為魏監軍石苞及州泰所破,軍却退就高。異復作車箱圍趣五木城。苞、泰攻異,異敗歸,而魏太山太守胡烈以奇兵五千詭道襲都陸,盡焚異資糧。綝授兵三萬人使異死戰,異不從,綝斬之於鑊里,而遣弟恩救,會誕敗引還。綝旣不能拔出誕,而喪敗士衆,自戮名將,莫不怨之。
 
 
綝以孫亮始親政事,多所難問,甚懼。還建業,稱疾不朝,築室于朱雀橋南,使弟威遠將軍據入蒼龍宿衞,弟武衞將軍恩、偏將軍幹、長水校尉闓分屯諸營,欲以專朝自固。亮內嫌綝,乃推魯育見殺本末,責怒虎林督朱熊、熊弟外部督朱損不匡正孫峻,乃令丁奉殺熊於虎林,殺損於建業。綝入諫不從,亮遂與公主魯班、太常全尚、將軍劉承議誅綝。亮妃,綝從姊女也,以其謀告綝。綝率衆夜襲全尚,遣弟恩殺劉承於蒼龍門外,遂圍宮。
 \gezhu{江表傳曰:亮召全尚息黃門侍郎紀密謀,曰:「孫綝專勢,輕小於孤。孤見勑之,使速上岸,為唐咨等作援,而留湖中,不上岸一步。又委罪朱異,擅殺功臣,不先表聞。築第橋南,不復朝見。此為自在,無復所畏,不可乆忍。今規取之,卿父作中軍都督,使密嚴整士馬,孤當自出臨橋,帥宿衞虎騎、左右無難一時圍之。作版詔勑綝所領皆解散,不得舉手,正爾自得之。卿去,但當使密耳。卿宣詔語卿父,勿令卿母知之,女人旣不曉大事,且綝同堂姊,邂逅泄漏,誤孤非小也。」紀承詔,以告尚,尚無遠慮,以語紀母。母使人密語綝。綝夜發嚴兵廢亮,比明,兵已圍宮。亮大怒,上馬,帶鞬執弓欲出,曰:「孤大皇帝之適子,在位已五年,誰敢不從者?」侍中近臣及乳母共牽攀止之,乃不得出,歎咤二日不食,罵其妻曰:「爾父憒憒,敗我大事!」又呼紀,紀曰:「臣父奉詔不謹,負上,無面目復見。」因自殺。孫盛曰:亮傳稱亮少聦慧,勢當先與紀謀,不先令妻知也。江表傳說漏泄有由,於事為詳矣。}
 使光祿勳孟宗告廟廢亮,召羣司議曰:「少帝荒病昏亂,不可以處大位,承宗廟,以告先帝廢之。諸君若有不同者,下異議。」皆震怖,曰:「唯將軍令。」綝遣中書郎李崇奪亮璽綬,以亮罪狀班告遠近。尚書桓彝不肯署名,綝怒殺之。
 \gezhu{漢晉春秋曰:彝,魏尚書令階之弟。吳錄曰:晉武帝問薛瑩吳之名臣,瑩對稱彝有忠貞之節。}
 
 
 
 
 典軍施正勸綝徵立琅邪王休,綝從之,遣宗正楷奉書於休曰:「綝以薄才,見授大任,不能輔導陛下。頃月以來,多所造立,親近劉承,恱於美色,發吏民婦女,料其好者,留於宮內,取兵子弟十八已下三千餘人,習之苑中,連日續夜,大小呼嗟,敗壞藏中矛戟五千餘枚,以作戲具。朱據先帝舊臣,子男熊、損皆承父之基,以忠義自立,昔殺小主,自是大主所創,帝不復精其本末,便殺熊、損,諫不見用,諸下莫不側息。帝於宮中作小船三百餘艘,成以金銀,師工晝夜不息。太常全尚,累世受恩,不能督諸宗親,而全端等委城就魏。尚位過重,曾無一言以諫陛下,而與敵往來,使傳國消息,懼必傾危社稷。推案舊典,運集大王,輒以今月二十七日擒尚斬承。以帝為會稽王,遣楷奉迎。百寮喁喁,立住道側。」
 
 
 
 
 綝遣將軍孫耽送亮之國,徙尚於零陵,遷公主於豫章。綝意彌溢,侮慢民神,遂燒大橋頭伍子胥廟,又壞浮屠祠,斬道人。休旣即位,稱草莽臣,詣闕上書曰:「臣伏自省,才非幹國,因緣肺腑,位極人臣,傷錦敗駕,罪負彰露,尋愆惟闕,夙夜憂懼。臣聞天命棐諶,必就有德,是以幽厲失度,周宣中興,陛下聖德,纂承大統,宜得良輔,以協雍熈,雖堯之盛,猶求稷契之佐,以協明聖之德。古人有言:『陳力就列,不能者止。』臣雖自展竭,無益庶政,謹上印綬節鉞,退還田里,以避賢路。」休引見慰喻。又下詔曰:「朕以不德,守潘于外,值茲際會,羣公卿士,曁于朕躬,以奉宗廟。朕用憮然,若涉淵水。大將軍忠計內發,扶危定傾,安康社稷,功勳赫然。昔漢孝宣踐阼,霍光尊顯,襃德賞功,古今之通義也。其以大將軍為丞相、荊州牧,食五縣。」恩為御史大夫、衞將軍,據右將軍,皆縣侯。幹雜號將軍、亭侯,闓亦封亭侯。綝一門五侯,皆典禁兵,權傾人主,自吳國朝臣未甞有也。
 
 
綝奉牛酒詣休,休不受,齎詣左將軍張布;酒酣,出怨言曰:「初廢少主時,多勸吾自為之者。吾以陛下賢明,故迎之。帝非我不立,今上禮見拒,是與凡臣無異,當復改圖耳。」布以言聞休,休銜之,恐其有變,數加賞賜,又復加恩侍中,與綝分省文書。或有告綝懷怨侮上欲圖反者,休執以付綝,綝殺之,由是愈懼,因孟宗求出屯武昌,休許焉,盡勑所督中營精兵萬餘人,皆令裝載,所取武庫兵器,咸令給與。
 \gezhu{吳歷曰:綝求中書兩郎,典知荊州諸軍事,主者奏中書不應外出,休特聽之,其所請求,一皆給與。}
 將軍魏邈說休曰「綝居外必有變」,武衞士施朔又告「綝欲反有徵」。休密問張布,布與丁奉謀於會殺綝。
 
 
 
 
 永安元年十二月丁卯,建業中謠言明會有變,綝聞之,不恱。夜大風發木揚沙,綝益恐。戊辰臘會,綝稱疾。休彊起之,使者十餘輩,綝不得已,將入,衆止焉。綝曰:「國家屢有命,不可辭。可豫整兵,令府內起火,因是可得速還。」遂入,尋而火起,綝求出,休曰:「外兵自多,不足煩丞相也。」綝起離席,奉、布目左右縛之。綝叩首曰:「願徙交州。」休曰:「卿何以不徙滕胤、呂據?」綝復曰:「願沒為官奴。」休曰:「何不以胤、據為奴乎!」遂斬之。以綝首令其衆曰:「諸與綝同謀皆赦。」放仗者五千人。闓乘船欲北降,追殺之。夷三族。發孫峻棺,取其印綬,斲其木而埋之,以殺魯育等故也。
 
 
 
 
 綝死時年二十八。休恥與峻、綝同族。特除其屬籍,稱之曰故峻、故綝云。休又下詔曰:「諸葛恪、滕胤、呂據蓋以無罪為峻、綝兄弟所見殘害,可為痛心,促皆改葬,各為祭奠。其罹恪等事見遠徙者,一切召還。」
 
 
\end{pinyinscope}