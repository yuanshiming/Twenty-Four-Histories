\article{孫策傳}
\begin{pinyinscope}
 
 
 策字伯符。堅初興義兵,策將母徙居舒,與周瑜相友,收合士大夫,江、淮間人咸向之。
 
 
\gezhu{江表傳曰:堅為朱儁所表,為佐軍,留家著壽春。策年十餘歲,已交結知名,聲譽發聞。有周瑜者,與策同年,亦英達夙成,聞策聲聞,自舒來造焉。便推結分好,義同斷金,勸策徙居舒,策從之。}
 堅薨,還葬曲阿。已乃渡江居江都。
 \gezhu{魏書曰:策當嗣侯,讓與弟匡。}
 
 
徐州牧陶謙深忌策。策舅吳景,時為丹楊太守,策乃載母徙曲阿,與呂範、孫河俱就景,因緣召募得數百人。興平元年,從袁術。術甚奇之,以堅部曲還策。
 \gezhu{吳歷曰:初策在江都時,張紘有母喪。策數詣紘,咨以世務,曰:「方今漢祚中微,天下擾攘,英雄儁傑各擁衆營私,未有能扶危濟亂者也。先君與袁氏共破董卓,功業未遂,卒為黃祖所害。策雖暗稚,竊有微志,欲從袁揚州求先君餘兵,就舅氏於丹楊,收合流散,東據吳會,報讎雪恥,為朝廷外藩。君以為何如?」紘荅曰:「旣素空劣,方居衰絰之中,無以奉贊盛略。」策曰:「君高名播越,遠近懷歸。今日事計,決之於君,何得不紆慮啟告,副其高山之望?若微志得展,血讎得報,此乃君之勳力,策心所望也。」因涕泣橫流,顏色不變。紘見策忠壯內發,辭令慷慨,感其志言,乃荅曰:「昔周道陵遲,齊、晉並興;王室已寧,諸侯貢職。今君紹先侯之軌,有驍武之名,若投丹楊,收兵吳會,則荊、揚可一,讎敵可報。據長江,奮威德,誅除羣穢,匡輔漢室,功業侔於桓、文,豈徒外藩而已哉?方今世亂多難,若功成事立,當與同好俱南濟也。」策曰:「一與君同符合契,同有永固之分,今便行矣,以老母弱弟委付於君,策無復回顧之憂。」江表傳曰:策徑到壽春見袁術,涕泣而言曰:「亡父昔從長沙入討董卓,與明使君會於南陽,同盟結好;不幸遇難,勳業不終。策感惟先人舊恩,欲自憑結,願明使君垂察其誠。」術甚貴異之,然未肯還其父兵。術謂策曰:「孤始用貴舅為丹楊太守,賢從伯陽為都尉,彼精兵之地,可還依召募。」策遂詣丹楊依舅,得數百人,而為涇縣大帥祖郎所襲,幾至危殆。於是復往見術,術以堅餘兵千餘人還策。}
 
 
太傅馬日磾杖節安集關東,在壽春以禮辟策,表拜懷義校尉,術大將喬蕤、張勳皆傾心敬焉。術常歎曰:「使術有子如孫郎,死復何恨!」策騎士有罪,逃入術營,隱於內廄。策指使人就斬之,訖,詣術謝。術曰:「兵人好叛,當共疾之,何為謝也?」由是軍中益畏憚之。術初許策為九江太守,已而更用丹楊陳紀。後術欲攻徐州,從廬江太守陸康求米三萬斛。康不與,術大怒。策昔曾詣康,康不見,使主簿接之。策常銜恨。術遣策攻康,謂曰:「前錯用陳紀,每恨本意不遂。今若得康,廬江真卿有也。」策攻康,拔之,術復用其故吏劉勳為太守,策益失望。先是,劉繇為揚州刺史,州舊治壽春。壽春,術已據之,繇乃渡江治曲阿。時吳景尚在丹楊,策從兄賁又為丹楊都尉,繇至,皆迫逐之。景、賁退舍歷陽。繇遣樊能、于麋屯橫江津,張英屯當利口,以距術。術自用故吏琅邪惠衢為揚州刺史,更以景為督軍中郎將,與賁共將兵擊英等,連年不克。策乃說術,乞助景等平定江東。
 \gezhu{江表傳曰:策說術云:「家有舊恩在東,願助舅討橫江;橫江拔,因投本土召募,可得三萬兵,以佐明使君匡濟漢室。」術知其恨,而以劉繇據曲阿,王朗在會稽,謂策未必能定,故許之。}
 術表策為折衝校尉,行殄寇將軍,兵財千餘,騎數十匹,賔客願從者數百人。比至歷陽,衆五六千。策母先自曲阿徙於歷陽,策又徙母阜陵,渡江轉鬬,所向皆破,莫敢當其鋒,而軍令整肅,百姓懷之。
 \gezhu{江表傳曰:策渡江攻繇牛渚營,盡得邸閣糧穀、戰具,是歲興平二年也。時彭城相薛禮、下邳相笮融依繇為盟主,禮據秣陵城,融屯縣南。策先攻融,融出兵交戰,斬首五百餘級,融即閉門不敢動。因渡江攻禮,禮突走,而樊能、于麋等復合衆襲奪牛渚屯。策聞之,還攻破能等,獲男女萬餘人。復下攻融,為流矢所中,傷股,不能乘馬,因自輿還牛渚營。或叛告融曰:「孫郎被箭已死。」融大喜,即遣將於茲鄉。策遣步騎數百挑戰,設伏於後,賊出擊之,鋒刃未接而偽走,賊追入伏中,乃大破之,斬首千餘級。策因往到融營下,令左右大呼曰:「孫郎竟云何!」賊於是驚怖夜遁。融聞策尚在,更深溝高壘,繕治守備。策以融所屯地勢險固,乃舍去,攻破繇別將於海陵,轉攻湖孰、江乘,皆下之。}
 
 
策為人,美姿顏,好笑語,性闊達聽受,善於用人,是以士民見者,莫不盡心,樂為致死。劉繇棄軍遁逃,諸郡守皆捐城郭奔走。
 \gezhu{江表傳曰:策時年少,雖有位號,而士民皆呼為孫郎。百姓聞孫郎至,皆失魂魄;長吏委城郭,竄伏山草。及至,軍士奉令,不敢虜略,雞犬菜茹,一無所犯,民乃大恱,競以牛酒詣軍。劉繇旣走,策入曲阿勞賜將士,遣將陳寶詣阜陵迎母及弟。發恩布令,告諸縣:「其劉繇、笮融等故鄉部曲來降首者,一無所問;樂從軍者,一身行,復除門戶;不樂者,勿彊也。」旬日之間,四面雲集,得見兵二萬餘人,馬千餘匹,威震江東,形勢轉盛。}
 吳人嚴白虎等衆各萬餘人,處處屯聚。吳景等欲先擊破虎等,乃至會稽。策曰:「虎等羣盜,非有大志,此成禽耳。」遂引兵渡浙江,據會稽,屠東冶,乃攻破虎等。
 \gezhu{吳錄曰:時有烏程鄒他、錢銅及前合浦太守嘉興王晟等,各聚衆萬餘或數千。引兵撲討,皆攻破之。策母吳氏曰:「晟與汝父有升堂見妻之分,今其諸子兄弟皆已梟夷,獨餘一老翁,何足復憚乎?」乃舍之,餘咸族誅。策自討虎,虎高壘堅守,使其弟輿請和。許之。輿請獨與策會面約。旣會,策引白刃斫席,輿體動,策笑曰:「聞卿能坐躍,勦捷不常,聊戲卿耳!」輿曰:「我見刃乃然。」策知其無能也,乃以手戟投之,立死。輿有勇力,虎衆以其死也,甚懼。進攻破之。虎奔餘杭,投許昭於虜中。程普請擊昭,策曰:「許昭有義於舊君,有誠於故友,此丈夫之志也。」乃舍之。臣松之案:許昭有義於舊君,謂濟盛憲也,事見後注。有誠於故友,則受嚴白虎也。}
 盡更置長吏,策自領會稽太守,復以吳景為丹楊太守,以孫賁為豫章太守;分豫章為廬陵郡,以賁弟輔為廬陵太守,丹楊朱治為吳郡太守。彭城張昭、廣陵張紘、秦松、陳端等為謀主。
 \gezhu{江表傳曰:策遣奉正都尉劉由、五官掾高承奉章詣許,拜獻方物。}
 
 
時袁術僭號,策以書責而絕之。
 \gezhu{吳錄載策使張紘為書曰:「蓋上天垂司過之星,聖王建敢諫之鼓,設非謬之備,急箴闕之言,何哉?凡有所長,必有所短也。去冬傳有大計,無不悚懼;旋知供備貢獻,萬夫解惑。頃聞建議,復欲追遵前圖,即事之期,便有定月。益使憮然,想是流妄;設其必爾,民何望乎?曩日之舉義兵也,天下之士所以響應者,董卓擅廢置,害太后、弘農王,略烝宮人,發掘園陵,暴逆至此,故諸州郡雄豪聞聲慕義。神武外振,卓遂內殲。元惡旣斃,幼主東顧,俾保傅宣命,欲令諸軍振旅,然河北通謀黑山,曹操放毒東徐,劉表稱亂南荊,公孫瓚炰烋北幽,劉繇決力江滸,劉備爭盟淮隅,是以未獲承命櫜弓戢戈也。今備、繇旣破,操等饑餒,謂當與天下合謀,以誅醜類。捨而不圖,有自取之志,非海內所望,一也。昔成湯伐桀,稱有夏多罪;武王伐紂,曰殷有罪罰重哉。此二王者,雖有聖德,宜當君世;如使不遭其時,亦無由興矣。幼主非有惡於天下,徒以春秋尚少,脅於彊臣,若無過而奪之,懼未合於湯、武之事,二也。卓雖狂狡,至廢主自與,亦猶未也,而天下聞其桀虐,攘臂同心而疾之,以中土希戰之兵,當邊地勁悍之虜,所以斯須游魂也。今四方之人,皆玩敵而便戰鬬矣,可得而勝者,以彼亂而我治,彼逆而我順也。見當世之紛若,欲大舉以臨之,適足趣禍,三也。天下神器,不可虛干,必須天贊與人力也。殷湯有白鳩之祥,周武有赤烏之瑞,漢高有星聚之符,世祖有神光之徵,皆因民困瘁於桀、紂之政,毒苦於秦、莽之役,故能芟去無道,致成其志。今天下非患於幼主,未見受命之應驗,而欲一旦卒然登即尊號,未之或有,四也。天子之貴,四海之富,誰不欲焉?義不可,勢不得耳。陳勝、項籍、王莽、公孫述之徒,皆南面稱孤,莫之能濟。帝王之位,不可橫兾,五也。幼主岐嶷,若除其偪,去其鯁,必成中興之業。夫致主於周成之盛,自受旦、奭之美,此誠所望於尊明也。縱使幼主有他改異,猶望推宗室之譜屬,論近親之賢良,以紹劉統,以固漢宗。皆所以書功金石,圖形丹青,流慶無窮,垂聲管弦。捨而不為,為其難者,想明明之素,必所不忍,六也。五世為相,權之重,勢之盛,天下莫得而比焉。忠貞者必曰宜夙夜思惟,所以扶國家之躓頓,念社稷之危殆,以奉祖考之志,以報漢室之恩。其忽履道之節而彊進取之欲者,將曰天下之人非家吏則門生也,孰不從我?四方之敵非吾匹則吾役也,誰能違我?盍乘累世之勢,起而取之哉?二者殊數,不可不詳察,七也。所貴於聖哲者,以其審於機宜,慎於舉措。若難圖之事,難保之勢,以激羣敵之氣,以生衆人之心,公義故不可,私計又不利,明哲不處,八也。世人多惑於圖緯而牽非類,比合文字以恱所事,苟以阿上惑衆,終有後悔者,自往迄今,未嘗無之,不可不深擇而熟思,九也。九者,尊明所見之餘耳,庶備起予,補所遺忘。忠言逆耳,幸留神聽!」典略云張昭之辭。臣松之以為張昭雖名重,然不如紘之文也,此書必紘所作。}
 曹公表策為討逆將軍,封為吳侯。
 \gezhu{江表傳曰:建安二年夏,漢朝遣議郎王誧奉戊辰詔書曰:「董卓逆亂,凶國害民。先將軍堅念在平討,雅意未遂,厥美著聞。策遵善道,求福不回。今以策為騎都尉,襲爵烏程侯,領會稽太守。」又詔勑曰:「故左將軍袁術不顧朝恩,坐創凶逆,造合虛偽,欲因兵亂,詭詐百姓,始聞其言以為不然。定得使持節平東將軍領徐州牧溫侯布上術所造惑衆妖妄,知術鴟梟之性,遂其無道,脩治王宮,署置公卿,郊天祀地,殘民害物,為禍深酷。布前後上策乃心本朝,欲還討術,為國效節,乞加顯異。夫縣賞俟功,惟勤是與,故便寵授,承襲前邑,重以大郡,榮耀兼至,是策輸力竭命之秋也。其亟與布及行吳郡太守安東將軍陳瑀勠力一心,同時赴討。」策自以統領兵馬,但以騎都尉領郡為輕,欲得將軍號,及使人諷誧,誧便承制假策明漢將軍。是時,陳瑀屯海西,策奉詔治嚴,當與布、瑀參同形勢。行到錢唐,瑀陰圖襲策,遣都尉萬演等密渡江,使持印傳三十餘紐與賊丹楊、宣城、涇、陵陽、始安、黟、歙諸險縣大帥祖郎、焦已及吳郡烏程嚴白虎等,使為內應,伺策軍發,欲攻取諸郡。策覺之,遣呂範、徐逸攻瑀於海西,大破瑀,獲其吏士妻子四千人。山陽公載記曰:瑀單騎走兾州,自歸袁紹,紹以為故安都尉。吳錄載策上表謝曰:「臣以固陋,孤特邊陲。陛下廣播高澤,不遺細節,以臣襲爵,兼典名郡。仰榮顧寵,所不克堪。興平二年十二月二十日,於吳郡曲阿得袁術所呈表,以臣行殄寇將軍;至被詔書,乃知詐擅。雖輒捐廢,猶用悚悸。臣年十七,喪失所怙,懼有不任堂構之鄙,以忝析薪之戒,誠無去病十八建功,世祖列將弱冠佐命。臣初領兵,年未弱冠,雖駑懦不武,然思竭微命。惟術狂惑,為惡深重。臣憑威靈,奉辭伐罪,庶必獻捷,以報所受。」臣松之案:本傳云孫堅以初平三年卒,策以建安五年卒,策死時年二十六,計堅之亡,策應十八,而此表云十七,則為不符。張璠漢紀及吳歷並以堅初平二年死,此為是而本傳誤也。江表傳曰:建安三年,策又遣使貢方物,倍於元年所獻。其年,制書轉拜討逆將軍,改封吳侯。}
 後術死,長史楊弘、大將張勳等將其衆欲就策,廬江太守劉勳要擊,悉虜之,收其珍寶以歸。策聞之,偽與勳好盟。勳新得術衆,時豫章上繚宗民萬餘家在江東,策勸勳攻取之。勳旣行,策輕軍晨夜襲拔廬江,勳衆盡降,勳獨與麾下數百人自歸曹公。
 \gezhu{江表傳曰:策被詔勑,與司空曹公、衞將軍董承、益州牧劉璋等并力討袁術、劉表。軍嚴當進,會術死,術從弟胤、女壻黃猗等畏懼曹公,不敢守壽春,乃共舁術棺柩,扶其妻子及部曲男女,就劉勳於皖城。勳糧食少,無以相振,乃遣從弟偕告糴於豫章太守華歆。歆郡素少穀,遣吏將偕就海昏上繚,使諸宗帥共出三萬斛米以與偕。偕往歷月,纔得數千斛。偕乃報勳,具說形狀,使勳來襲取之。勳得偕書,便潛軍到海昏邑下。宗帥知之,空壁逃匿,勳了無所得。時策西討黃祖,行及石城,聞勳輕身詣海昏,便分遣從兄賁、輔率八千人於彭澤待勳,自與周瑜率二萬人步襲皖城,即克之,得術百工及鼓吹部曲三萬餘人,并術、勳妻子。上用汝南李術為廬江太守,給兵三千人以守皖,皆徙所得人東詣吳。賁、輔又於彭澤破勳。勳走入楚江,從尋陽步上到置馬亭,聞策等已克皖,乃投西塞。至沂,築壘自守,告急於劉表,求救於黃祖。祖遣太子射船軍五千人助勳。策復就攻,大破勳。勳與偕北歸曹公,射亦遁走。策收得勳兵二千餘人,船千艘,遂前進夏口攻黃祖。時劉表遣從子虎、南陽韓晞將長矛五千,來為黃祖前鋒。策與戰,大破之。吳錄載策表曰:「臣討黃祖,以十二月八日到祖所屯沙羨縣。劉表遣將助祖,並來趣臣。臣以十一日平旦部所領江夏太守行建威中郎將周瑜、領桂陽太守行征虜中郎將呂範、領零陵太守行蕩寇中郎將程普、行奉業校尉孫權、行先登校尉韓當、行武鋒校尉黃蓋等同時俱進。身跨馬櫟陳,手擊急鼓,以齊戰勢。吏士奮激,踊躍百倍,心精意果,各競用命。越渡重壍,迅疾若飛。火放上風,兵激煙下,弓弩並發,流矢雨集,日加辰時,祖乃潰爛。鋒刃所截,猋火所焚,前無生寇,惟祖迸走。獲其妻息男女七人,斬虎、韓晞已下二萬餘級,其赴水溺者二萬餘口,船六千餘艘,財物山積。雖表未禽,祖宿狡猾,為表腹心,出作爪牙,表之鴟張,以祖氣息,而祖家屬部曲埽地無餘,表孤特之虜,成鬼行尸。誠皆聖朝神武遠振,臣討有罪,得效微勤。」}
 是時哀紹方彊,而策并江東,曹公力未能逞,且欲撫之。
 \gezhu{吳歷曰:曹公聞策平定江南,意甚難之,常呼「猘兒難與爭鋒也」。}
 乃以弟女配策小弟匡,又為子章取賁女,皆禮辟策弟權、翊,又命揚州刺史嚴象舉權茂才。
 
 
建安五年,曹公與袁紹相拒於官渡,策陰欲襲許,迎漢帝,
 \gezhu{吳錄曰:時有高岱者,隱於餘姚,策命出使會稽丞陸昭逆之,策虛己候焉。聞其善左傳,乃自玩讀,欲與論講。或謂之曰:「高岱以將軍但英武而已,無文學之才,若與論傳而或云不知者,則某言符矣。」又謂岱曰:「孫將軍為人,惡勝己者,若每問,當言不知,乃合意耳。如皆辨義,此必危殆。」岱以為然,及與論傳,或荅不知。策果怒,以為輕己,乃囚之。知友及時人皆露坐為請。策登樓,望見數里中填滿。策惡其收衆心,遂殺之。岱字孔文,吳郡人也。受性聦達,輕財貴義。其友士拔奇,取於未顯,所友八人,皆世之英偉也。太守盛憲以為上計,舉孝廉。許貢來領郡,岱將憲避難於許昭家,求救於陶謙。謙未即救,岱憔悴泣血,水漿不入口。謙感其忠壯,有申包胥之義,許為出軍,以書與貢。岱得謙書以還,而貢已囚其母。吳人大小皆為危竦,以貢宿忿,往必見害。岱言在君則為君,且母在牢獄,期於當往,若得入見,事自當解。遂通書自白,貢即與相見。才辭敏捷,好自陳謝,貢登時出其母。岱將見貢,語友人張允、沈䁕令豫具船,以貢必悔,當追逐之。出便將母乘船易道而逃。貢須臾遣人追之,令追者若及於船,江上便殺之,已過則止。使與岱錯道,遂免。被誅時,年三十餘。江表傳曰:時有道士琅邪于吉,先寓居東方,往來吳會,立精舍,燒香讀道書,制作符水以治病,吳會人多事之。策嘗於郡城門樓上集會諸將賔客,吉乃盛服杖小函,漆畫之,名為仙人鏵,趨度門下。諸將賔客三分之二下樓迎拜之,掌賔者禁呵不能止。策即令收之。諸事之者,悉使婦女入見策母,請救之。母謂策曰:「于先生亦助軍作福,醫護將士,不可殺之。」策曰:「此子妖妄,能幻惑衆心,遠使諸將不復相顧君臣之禮,盡委策下樓拜之,不可不除也。」諸將復連名通白事陳乞之,策曰:「昔南陽張津為交州刺史,舍前聖典訓,廢漢家法律,甞著絳帕頭,鼓琴燒香,讀邪俗道書,云以助化,卒為南夷所殺。此甚無益,諸君但未悟耳。今此子已在鬼籙,勿復費紙筆也。」即催斬之,縣首於巿。諸事之者,尚不謂之死而云尸解焉,復祭祀求福。志林曰:初順帝時,琅邪宮崇詣闕上師于吉所得神書於曲陽泉水上,白素朱界,號太平青領道,凡百餘卷。順帝至建安中,五六十歲,于吉是時近已百年,年在耄悼,禮不加刑。又天子巡狩,問百年者,就而見之,敬齒以親愛,聖王之至教也。吉罪不及死,而暴加酷刑,是乃謬誅,非所以為美也。喜推考桓王之薨,建安五年四月四日。是時曹、袁相攻,未有勝負。案夏侯元讓與石威則書「袁紹破後也。書授孫賁以長沙,業張津以零、桂。」此為桓王於前亡,張津於後死,不得相讓,譬言津之死意矣。臣松之案:太康八年,廣州大中正王範上交廣二州春秋。建安六年,張津猶為交州牧。江表傳之虛如志林所云。搜神記曰:策欲渡江襲許,與吉俱行。時大旱,所在熇厲。策催諸將士使速引船,或身自早出督切,見將吏多在吉所,策因此激怒,言:「我為不如于吉邪,而先趨務之?」便使收吉。至,呵問之曰:「天旱不雨,道塗艱澁,不時得過,故自早出,而卿不同憂戚,安坐船中作鬼物態,敗吾部伍,今當相除。」令人縛置地上暴之,使請雨,若能感天日中雨者,當原赦,不爾行誅。俄而雲氣上蒸,膚寸而合,比至日中,大雨總至,溪澗盈溢。將士喜恱,以為吉必見原,並往慶慰。策遂殺之。將士哀惜,共藏其尸。天夜,忽更興雲覆之;明旦往視,不知所在。案江表傳、搜神記于吉事不同,未詳孰是。}
 密治兵,部署諸將。未發,會為故吳郡太守許貢客所殺。先是,策殺貢,貢小子與客亡匿江邊。策單騎出,卒與客遇,客擊傷策。
 \gezhu{江表傳曰:廣陵太守陳登治射陽,登即瑀之從兄子也。策前西征,登陰復遣間使,以印綬與嚴白虎餘黨,圖為後害,以報瑀見破之辱。策歸,復討登。軍到丹徒,須待運糧。策性好獵,將步騎數出。策驅馳逐鹿,所乘馬精駿,從騎絕不能及。初,吳郡太守許貢上表於漢帝曰:「孫策驍雄,與項籍相似,宜加貴寵,召還京邑。若被詔不得不還,若放於外必作世患。」策候吏得貢表,以示策。策請貢相見,以責讓貢。貢辭無表,策即令武士絞殺之。貢奴客潛民間,欲為貢報讎。獵日,卒有三人即貢客也。策問:「爾等何人?」荅云:「是韓當兵,在此射鹿耳。」策曰:「當兵吾皆識之,未嘗見汝等。」因射一人,應弦而倒。餘二人怖急,便舉弓射策,中頰。後騎尋至,皆刺殺之。九州春秋曰:策聞曹公北征柳城,悉起江南之衆,自號大司馬,將北襲許,恃其勇,行不設備,故及於難。孫盛異同評曰:凡此數書,各有所失。孫策雖威行江外,略有六郡,然黃祖乘其上流,陳登間其心腹,且深險彊宗未盡歸復,曹、袁虎爭,勢傾山海,策豈暇遠師汝、潁,而遷帝於吳、越哉?斯蓋庸人之所鑒見,況策達於事勢者乎?又案袁紹以建安五年至黎陽,而策以四月遇害,而志云策聞曹公與紹相拒於官渡,謬矣。伐登之言,為有證也。又江表傳說策悉識韓當軍士,疑此為詐,便射殺一人。夫三軍將士或有新附,策為大將,何能悉識?以所不識,便射殺之,非其論也,又策見殺在五年,柳城之役在十二年,九州春秋乖錯尤甚矣。臣松之案:傅子亦云曹公征柳城,將襲許。記述若斯,何其踈哉!然孫盛所譏,未為悉是。黃祖始被策破,魂氣未反,且劉表君臣本無兼并之志,雖在上流,何辦規擬吳會?策之此舉,理應先圖陳登,但舉兵所在,不止登而已。于時彊宗驍帥,祖郎、嚴虎之徒,禽滅已盡,所餘山越,蓋何足慮?然則策之所規,未可謂之不暇也。若使策志獲從,大權在手,淮、泗之間,所在皆可都,何必畢志江外,其當遷帝於揚、越哉?案魏武紀,武帝以建安四年已出屯官渡,乃策未死之前,乆與袁紹交兵,則國志所云不為謬也。許貢客,無聞之小人,而能感識恩遇,臨義忘生,卒然奮發,有侔古烈矣。詩云:「君子有徽猷,小人與屬。」貢客其有焉。}
 創甚,請張昭等謂曰:「中國方亂,夫以吳、越之衆,三江之固,足以觀成敗。公等善相吾弟!」呼權佩以印綬,謂曰:「舉江東之衆,決機於兩陳之間,與天下爭衡,卿不如我;舉賢任能,各盡其心,以保江東,我不知卿。」至夜卒,時年二十六。
 \gezhu{吳歷曰:策旣被創,醫言可治,當好自將護,百日勿動。策引鏡自照,謂左右曰:「面如此,尚可復建功立事乎?」椎几大奮,創皆分裂,其夜卒。搜神記曰:策旣殺于吉,每獨坐,彷彿見吉在左右,意深惡之,頗有失常。後治創方差,而引鏡自照,見吉在鏡中,顧而弗見,如是再三,因撲鏡大叫,創皆崩裂,須臾而死。}
 
 
 
 
 權稱尊號,追謚策曰長沙桓王,封子紹為吳侯,後改封上虞侯。紹卒,子奉嗣。孫皓時,訛言謂奉當立,誅死。
 
 
評曰:孫堅勇摯剛毅,孤微發迹,導溫戮卓,山陵杜塞,有忠壯之烈。策英氣傑濟,猛銳冠世,覽奇取異,志陵中夏。然皆輕佻果躁,隕身致敗。且割據江東,策之基兆也,而權尊崇未至,子止侯爵,於義儉矣。
 \gezhu{孫盛曰:孫氏兄弟皆明略絕羣。創基立事,策之由也,自臨終之日,顧命委權。夫意氣之間,猶有刎頸,況天倫之篤愛,豪達之英鑒,豈吝名號於旣往,違情本之至實哉?抑將遠思虛盈之數,而慎其名器者乎?夫正本定名,為國之大防;杜絕疑貳,消釁之良謨。是故魯隱矜義,終致羽父之禍;宋宣懷仁,卒有殤公之哀。皆心存小善,而不達經綸之圖;求譽當年,而不思貽厥之謀。可謂輕千乘之國,蹈道則未也。孫氏因擾攘之際,得奮其縱橫之志,業非積德之基,邦無磐石之固,勢一則祿祚可終,情乖則禍亂塵起,安可不防微於未兆,慮難於將來?壯哉!策為首事之君,有吳開國之主;將相在列,皆其舊也,而嗣子弱劣,析薪弗荷,奉之則魯桓、田巿之難作,崇之則與夷、子馮之禍興。是以正名定本,使貴賤殊邈,然後國無陵肆之責,後嗣罔猜忌之嫌,羣情絕異端之論,不逞杜覬覦之心;於情雖違,於事雖儉,至於括囊遠圖,永保維城,可謂為之于其未有,治之于其未亂者也。陳氏之評,其未達乎!}
 
 
\end{pinyinscope}