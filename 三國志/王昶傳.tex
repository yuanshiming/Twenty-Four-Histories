\article{王昶傳}
\begin{pinyinscope}
 
 
 王昶字文舒,太原晉陽人也。
 
 
\gezhu{案王氏譜:昶伯父柔,字叔優;父澤,字季道。郭林宗傳曰:叔優、季道幼少之時,聞林宗有知人之鑒,共往候之,請問才行所宜,以自處業。林宗笑曰:「卿二人皆二千石才也,雖然,叔優當以仕宦顯,季道宜以經術進,若違才易務,亦不至也。」叔優等從其言。叔優至北中郎將,季道代郡太守。}
 少與同郡王淩俱知名。淩年長,昶兄事之。文帝在東宮,昶為太子文學,遷中庶子。文帝踐阼,徙散騎侍郎,為洛陽典農。時都畿樹木成林,昶斫開荒萊,勤勸百姓,墾田特多。遷兖州刺史。明帝即位,加揚烈將軍,賜爵關內侯。昶雖在外任,心存朝廷,以為魏承秦、漢之弊,法制苛碎,不大釐改國典以準先王之風,而望治化復興,不可得也。乃著治論,略依古制而合於時務者二十餘篇,又著兵書十餘篇,言奇正之用,
 \gezhu{孫子兵法曰:兵以正合,以奇勝;奇正還相生,若循環之無端。}
 青龍中奏之。
 
 
 
 
 其為兄子及子作名字,皆依謙實,以見其意,故兄子默字處靜,沈字處道,其子渾字玄冲,深字道冲。遂書戒之曰:
 
 
夫人為子之道,莫大於寶身全行,以顯父母。此三者人知其善,而或危身破家,陷於滅亡之禍者,何也?由所祖習非其道也。夫孝敬仁義,百行之首,行之而立,身之本也。孝敬則宗族安之,仁義則鄉黨重之,此行成於內,名著於外者矣。人若不篤於至行,而背本逐末,以陷浮華焉,以成朋黨焉;浮華則有虛偽之累,朋黨則有彼此之患。此二者之戒,昭然著明,而循覆車滋衆,逐末彌甚,皆由惑當時之譽,昧目前之利故也。夫富貴聲名,人情所樂,而君子或得而不處,何也?惡不由其道耳。患人知進而不知退,知欲而不知足,故有困辱之累,悔吝之咨。語曰:「如不知足,則失所欲。」故知足之足常足矣。覽往事之成敗,察將來之吉凶,未有干名要利,欲而不厭,而能保世持家,永全福祿者也。欲使汝曹立身行己,遵儒者之教,履道家之言,故以玄默冲虛為名,欲使汝曹顧名思義,不敢違越也。古者盤杅有銘,几杖有誡,俯仰察焉,用無過行;況在己名,可不戒之哉!夫物速成則疾亡,晚就則善終。朝華之草,夕而零落;松栢之茂,隆寒不衰。是以大雅君子惡速成,戒闕黨也。若范匄對秦客而武子擊之折其委笄,惡其掩人也。
 \gezhu{國語曰:范文子暮退於朝,武子曰:「何暮也?」對曰:「有秦客廋辭於朝,大夫莫之能對也,吾知三焉。」武子怒曰:「大夫非不能也,讓父兄也。尔童子而三掩人於朝,吾不在,晉國亡無日也。」擊之以杖,折其委笄。臣松之案:對秦客者,范燮也。此云范匄,蓋誤也。}
 夫人有善鮮不自伐,有能者寡不自矜;伐則掩人,矜則陵人。掩人者人亦掩之,陵人者人亦陵之。故三郤為戮於晉,王叔負罪於周,不惟矜善自伐好爭之咎乎?故君子不自稱,非以讓人,惡其蓋人也。夫能屈以為伸,讓以為得,弱以為彊,鮮不遂矣。夫毀譽,愛惡之原而禍福之機也,是以聖人慎之。孔子曰:「吾之於人,誰毀誰譽;如有所譽,必有所試。」又曰:「子貢方人。賜也賢乎哉,我則不暇。」以聖人之德,猶尚如此,況庸庸之徒而輕毀譽哉?
 
 
昔伏波將軍馬援戒其兄子,言:「聞人之惡,當如聞父母之名;耳可得而聞,口不可得而言也。」斯戒至矣。
 \gezhu{臣松之以為援之此誡,可謂切至之言,不刊之訓也。凡道人過失,蓋謂居室之愆,人未之知,則由己而發者也。若乃行事,得失已暴於世,因其善惡,即以為誡,方之於彼,則有愈焉。然援誡稱龍伯高之美,言杜季良之惡,致使事徹時主,季良以敗。言之傷人,孰大於此?與其所誡,自相違伐。}
 人或毀己,當退而求之於身。若己有可毀之行,則彼言當矣;若己無可毀之行,則彼言妄矣。當則無怨於彼,妄則無害於身,又何反報焉?且聞人毀己而忿者,惡醜聲之加人也,人報者滋甚,不如默而自脩己也。諺曰:「救寒莫如重裘,止謗莫如自脩。」斯言信矣。若與是非之士,凶險之人,近猶不可,況與對校乎?其害深矣。夫虛偽之人,言不根道,行不顧言,其為浮淺較可識別;而世人惑焉,猶不檢之以言行也。近濟陰魏諷、山陽曹偉皆以傾邪敗沒,熒惑當世,挾持姦慝,驅動後生。雖刑於鈇鉞,大為烱戒,然所汙染,固以衆矣。可不慎與!
 \gezhu{世語曰:黃初中,孫權通章表。偉以白衣登江上,與權交書求賂,欲以交結京師,故誅之。}
 
 
若夫山林之士,夷、叔之倫,甘長飢於首陽,安赴火於緜山,雖可以激貪勵俗,然聖人不可為,吾亦不願也。今汝先人世有冠冕,惟仁義為名,守慎為稱,孝悌於閨門,務學於師友。吾與時人從事,雖出處不同,然各有所取。頴川郭伯益,好尚通達,敏而有知。其為人弘曠不足,輕貴有餘;得其人重之如山,不得其人忽之如草。吾以所知親之昵之,不願兒子為之。
 \gezhu{伯益名弈,郭嘉之子。}
 北海徐偉長,不治名高,不求苟得,澹然自守,惟道是務。其有所是非,則託古人以見其意,當時無所褒貶。吾敬之重之,願兒子師之。東平劉公幹,博學有高才,誠節有大意,然性行不均,少所拘忌,得失足以相補。吾愛之重之,不願兒子慕之。
 \gezhu{臣松之以為文舒復擬則文淵,顯言人之失。魏諷、曹偉,事陷惡逆,著以為誡,差無可尤。至若郭伯益、劉公幹,雖其人皆往,善惡有定;然旣交之於昔,不宜復毀之於今,而乃形于翰墨,永傳後葉,於舊交則違久要之義,於子孫則揚人前世之惡。於夫鄙懷,深所不取。善乎東方之誡子也,以首陽為拙,柳下為工,寄旨古人,無傷當時。方之馬、王,不亦遠哉!}
 樂安任昭先,淳粹履道,內敏外恕,推遜恭讓,處不避洿,怯而義勇,在朝忘身。吾友之善之,願兒子遵之。
 \gezhu{昭先名嘏。別傳曰:嘏,樂安博昌人。世為著姓,夙智性成,故鄉人為之語曰:「蔣氏翁,任氏童。」父旐,字子旟,以至行稱。漢末,黃巾賊起,天下飢荒,人民相食。寇到博昌,聞旐姓字,乃相謂曰:「宿聞任子旟,天下賢人也。今雖作賊,那可入其鄉邪?」遂相帥而去。由是聲聞遠近,州郡並招舉孝廉,歷酸棗、祝阿令。嘏八歲喪母,號泣不絕聲,自然之哀,同於成人,故幼以至性見稱。年十四始學,疑不再問,三年中誦五經,皆究其義,兼包群言,無不綜覽,於時學者號之神童。遂遇荒亂,家貧賣魚,會官稅魚,魚貴數倍,嘏取直如常。又與人共買生口,各雇八匹。後生口家來贖,時價直六十匹。共買者欲隨時價取贖,嘏自取本價八匹。共買者慙,亦還取本價。比居者擅耕嘏地數十畝種之,人以語嘏,嘏曰:「我自以借之耳。」耕者聞之,慙謝還地。及邑中爭訟,皆詣嘏質之,然後意厭。其子弟有不順者,父兄竊數之曰:「汝所行,豈可令任君知邪!」其禮教所化,率皆如此。會太祖創業,召海內至德,嘏應其舉,為臨菑侯庶子、相國東曹屬、尚書郎。文帝時,為黃門侍郎。每納忠言,輒手書懷本,自在禁省,歸書不封。帝嘉其淑慎,累遷東郡、趙郡、河東太守,所在化行,有遺風餘教。嘏為人淳粹凱弟,虛己若不足,恭敬如有畏。其脩身履義,皆沈默潛行,不顯其美,故時人少得稱之。著書三十八篇,凡四萬餘言。嘏卒後,故吏東郡程威、趙國劉固、河東上官崇等,錄其事行及所著書奏之。詔下祕書,以貫群言。}
 若引而伸之,觸類而長之,汝其庶幾舉一隅耳。及其用財先九族,其施舍務周急,其出入存故老,其論議貴無貶,其進仕尚忠節,其取人務實道,其處勢戒驕淫,其貧賤慎無戚,其進退念合宜,其行事加九思,如此而已。吾復何憂哉?
 
 
 
 
 青龍四年,詔「欲得有才智文章,謀慮淵深,料遠若近,視昧而察,籌不虛運,策弗徒發,端一小心,清脩密靜,乹乹不解,志尚在公者,無限年齒,勿拘貴賤,卿校已上各舉一人」。太尉司馬宣王以昶應選。正始中,轉在徐州,封武觀亭侯,遷征南將軍,假節都督荊、豫諸軍事。昶以為國有常衆,戰無常勝;地有常險,守無常勢。今屯宛,去襄陽三百餘里,諸軍散屯,船在宣池,有急不足相赴,乃表徙治新野,習水軍於二州,廣農墾殖,倉糓盈積。
 
 
 
 
 嘉平初,太傅司馬宣王旣誅曹爽,乃奏愽問大臣得失。昶陳治略五事:其一,欲崇道篤學,抑絕浮華,使國子入太學而脩庠序;其二,欲用考試,考試猶準繩也,未有舍準繩而意正曲直,廢黜陟而空論能否也;其三,欲令居官者乆於其職,有治績則就增位賜爵;其四,欲約官實祿,勵以廉恥,不使與百姓爭利;其五,欲絕侈靡,務崇節儉,令衣服有章,上下有叙,儲糓畜帛,反民於樸。詔書襃讚。因使撰百官考課事,昶以為唐虞雖有黜陟之文,而考課之法不垂。周制冢宰之職,大計群吏之治而誅賞,又無校比之制。由此言之,聖主明於任賢,略舉黜陟之體,以委達官之長,而總其統紀,故能否可得而知也。其大指如此。
 
 
二年,昶奏:「孫權流放良臣,適庶分爭,可乘釁而制吳、蜀;白帝、夷陵之間,黔、巫、秭歸、房陵皆在江北,民夷與新城郡接,可襲取也。」乃遣新城太守州泰襲巫、秭歸、房陵,荊州刺史王基詣夷陵,昶詣江陵,兩岸引竹絙為橋,渡水擊之。賊奔南岸,鑿七道並來攻。於是昶使積弩同時俱發,賊大將施績夜遁入江陵城,追斬數百級。昶欲引致平地與合戰,乃先遣五軍案大道發還,使賊望見以喜之,以所獲鎧馬甲首,馳環城以怒之,設伏兵以待之。績果追軍,與戰,克之。績遁走,斬其將鍾離茂、許旻,收其甲首旗鼓珍寶器仗,振旅而還。王基、州泰皆有功。於是遷昶征南大將軍、儀同三司,進封京陵侯。毌丘儉、文欽作亂,引兵拒儉、欽有功,封二子亭侯、關內侯,進位驃騎將軍。諸葛誕反,昶據夾石以逼江陵,持施績、全熈使不得東。誕旣誅,詔曰:「昔孫臏佐趙,直湊大梁。西兵驟進,亦所以成東征之勢也。」增邑千戶,并前四千七百戶,遷司空,持節、都督如故。甘露四年薨,謚曰穆侯。子渾嗣,咸熈中為越騎校尉。
 \gezhu{案晉書:渾自越騎入晉,累居方任,平吳有功,封一子江陵侯,位至司徒。渾子濟,字武子,有儁才令望,為河南尹、太僕。早卒,追贈驃騎將軍。渾弟深,兾州刺史。深弟湛,字處冲,汝南太守。湛子承,字安期,東海內史。承子述,字懷祖,尚書令、衞將軍。述子坦之,字文度,北中郎將,徐、兖二州刺史。昶諸子中,湛最有悳譽,而承亦自為名士,述及坦之並顯重於世,為時盛門云。自湛已下事,見晉陽秋也。}
 
 
\end{pinyinscope}