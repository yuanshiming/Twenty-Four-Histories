\article{關羽傳}
\begin{pinyinscope}
 
 
 關羽字雲長,本字長生,河東解人也。亡命奔涿郡。先主於鄉里合徒衆,而羽與張飛為之禦侮。先主為平原相,以羽、飛為別部司馬,分統部曲。先主與二人寢則同牀,恩若兄弟。而稠人廣坐,侍立終日,隨先主周旋,不避艱險。
 
 
\gezhu{蜀記曰:曹公與劉備圍呂布於下邳,關羽啟公,布使秦宜祿行求救,乞娶其妻,公許之。臨破,又屢啟於公。公疑其有異色,先遣迎看,因自留之,羽心不自安。此與魏氏春秋所說無異也。}
 先主之襲殺徐州刺史車冑,使羽守下邳城,行太守事,
 \gezhu{魏書云:以羽領徐州。}
 而身還小沛。
 
 
建安五年,曹公東征,先主奔袁紹。曹公禽羽以歸,拜為偏將軍,禮之甚厚。紹遣大將軍顏良攻東郡太守劉延於白馬,曹公使張遼及羽為先鋒擊之。羽望見良麾蓋,策馬刺良於萬衆之中,斬其首還,紹諸將莫能當者,遂解白馬圍。曹公即表封羽為漢壽亭侯。初,曹公壯羽為人,而察其心神無乆留之意,謂張遼曰:「卿試以情問之。」旣而遼以問羽,羽歎曰:「吾極知曹公待我厚,然吾受劉將軍厚恩,誓以共死,不可背之。吾終不留,吾要當立效以報曹公乃去。」遼以羽言報曹公,曹公義之。
 \gezhu{傅子曰:遼欲白太祖,恐太祖殺羽,不白,非事君之道,乃歎曰:「公,君父也;羽,兄弟耳。」遂白之。太祖曰:「事君不忘其本,天下義士也。度何時能去?」遼曰:「羽受公恩,必立效報公而後去也。」}
 及羽殺顏良,曹公知其必去,重加賞賜。羽盡封其所賜,拜書告辭,而奔先主於袁軍。左右欲追之,曹公曰:「彼各為其主,勿追也。」
 \gezhu{臣松之以為曹公知羽不留而心嘉其志,去不遣追以成其義,自非有王霸之度,孰能至於此乎?斯實曹氏之休美。}
 
 
從先主就劉表。表卒,曹公定荊州,先主自樊將南渡江,別遣羽乘船數百艘會江陵。曹公追至當陽長阪,先主斜趣漢津,適與羽船相值,共至夏口。
 \gezhu{蜀記曰:初,劉備在許,與曹公共獵。獵中,衆散,羽勸備殺公,備不從。及在夏口,飄颻江渚,羽怒曰:「往日獵中,若從羽言,可無今日之困。」備曰:「是時亦為國家惜之耳;若天道輔正,安知此不為福邪!」臣松之以為備後與董承等結謀,但事泄不克諧耳,若為國家惜曹公,其如此言何!羽若果有此勸而備不肯從者,將以曹公腹心親戚,寔繁有徒,事不宿構,非造次所行;曹雖可殺,身必不免,故以計而止,何惜之有乎!旣往之事,故託為雅言耳。}
 孫權遣兵佐先主拒曹公,曹公引軍退歸。先主收江南諸郡,乃封拜元勳,以羽為襄陽太守、盪寇將軍,駐江北。先主西定益州,拜羽董督荊州事。羽聞馬超來降,舊非故人,羽書與諸葛亮,問超人才可誰比類。亮知羽護前,乃荅之曰:「孟起兼資文武,雄烈過人,一世之傑,黥、彭之徒,當與益德並驅爭先,猶未及髯之絕倫逸羣也。」羽美鬚髯,故亮謂之髯。羽省書大恱,以示賔客。
 
 
 
 
 羽甞為流矢所中,貫其左臂,後創雖愈,每至陰雨,骨常疼痛,醫曰:「矢鏃有毒,毒入于骨,當破臂作創,刮骨去毒,然後此患乃除耳。」羽便伸臂令醫劈之。時羽適請諸將飲食相對,臂血流離,盈於盤器,而羽割炙引酒,言笑自若。
 
 
二十四年,先主為漢中王,拜羽為前將軍,假節鉞。是歲,羽率衆攻曹仁於樊。曹公遣于禁助仁。秋,大霖雨,漢水汎溢,禁所督七軍皆沒。禁降羽,羽又斬將軍龐德。梁郟、陸渾羣盜或遙受羽印號,為之支黨,羽威震華夏。曹公議徙許都以避其銳,司馬宣王、蔣濟以為關羽得志,孫權必不願也。可遣人勸權躡其後,許割江南以封權,則樊圍自解。曹公從之。先是,權遣使為子索羽女,羽罵辱其使,不許婚,權大怒。
 \gezhu{典略曰:羽圍樊,權遣使求助之,勑使莫速進,又遣主簿先致命於羽。羽忿其淹遲,又自已得于禁等,乃罵曰:「狢子敢爾,如使樊城拔,吾不能滅汝邪!」權聞之,知其輕己,偽手書以謝羽,許以自往。臣松之以為荊、吳雖外睦,而內相猜防,故權之襲羽,潛師密發。按呂蒙傳云:「伏精兵於𦩷𦪇之中,使白衣搖櫓,作商賈服。」以此言之,羽不求助於權,權必不語羽當往也。若許相援助,何故匿其形迹乎?}
 又南郡太守麋芳在江陵,將軍傅士仁屯公安,素皆嫌羽自輕己。羽之出軍,芳、仁供給軍資不悉相救。羽言「還當治之」,芳、仁咸懷懼不安。於是權陰誘芳、仁,芳、仁使人迎權。而曹公遣徐晃救曹仁,
 \gezhu{蜀記曰:羽與晃宿相愛,遙共語,但說平生,不及軍事。須臾,晃下馬宣令:「得關雲長頭,賞金千斤。」羽驚怖,謂晃曰:「大兄,是何言邪!」晃曰:「此國之事耳。」}
 羽不能克,引軍退還。權已據江陵,盡虜羽士衆妻子,羽軍遂散。權遣將逆擊羽,斬羽及子平于臨沮。
 \gezhu{蜀記曰:權遣將軍擊羽,獲羽及子平。權欲活羽以敵劉、曹,左右曰:「狼子不可養,後必為害。曹公不即除之,自取大患,乃議徙都。今豈可生!」乃斬之。臣松之桉吳書:孫權遣將潘璋逆斷羽走路,羽至即斬,且臨沮去江陵二三百里,豈容不時殺羽,方議其生死乎?又云「權欲活羽以敵劉、曹」,此之不然,可以絕智者之口。吳歷曰:權送羽首於曹公,以諸侯禮葬其屍骸。}
 
 
追謚羽曰壯繆侯。
 \gezhu{蜀記曰:羽初出軍圍樊,夢豬嚙其足,語子平曰:「吾今年衰矣,然不得還!」江表傳云:羽好左氏傳,諷誦略皆上口。}
 子興嗣。興字安國,少有令問,丞相諸葛亮深器異之。弱冠為侍中、中監軍,數歲卒。子統嗣,尚公主,官至虎賁中郎將。卒,無子,以興庶子彝續封。
 \gezhu{蜀記曰:龐德子會,隨鍾、鄧伐蜀,蜀破,盡滅關氏家。}
 
 
\end{pinyinscope}