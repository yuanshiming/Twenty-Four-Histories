\article{烏丸鮮卑東夷傳}
\begin{pinyinscope}
 
 
 書載「蠻夷猾夏」,詩稱「玁狁孔熾」,乆矣其為中國患也。秦、漢以來,匈奴乆為邊害。孝武雖外事四夷,東平兩越、朝鮮,西討貳師、大宛,開卭苲、夜郎之道,然皆在荒服之外,不能為中國輕重。而匈奴最逼於諸夏,胡騎南侵則三邊受敵,是以屢遣衞、霍之將,深入北伐,窮追單于,奪其饒衍之地。後遂保塞稱藩,世以衰弱。建安中,呼厨泉南單于入朝,遂留內侍,使右賢王撫其國,而匈奴折節,過於漢舊。然烏丸、鮮卑稍更彊盛,亦因漢末之亂,中國多事,不遑外討,故得擅漢南之地,寇暴城邑,殺畧人民,北邊仍受其困。會袁紹兼河北,乃撫有三郡烏丸,寵其名王而收其精騎。其後尚、熈又逃于蹋頓。蹋頓又驍武,邊長老皆比之冒頓,恃其阻遠,敢受亡命,以雄百蠻。太祖潛師北伐,出其不意,一戰而定之,夷狄懾服,威振朔土。遂引烏丸之衆服從征討,而邊民得用安息。後鮮卑大人軻比能復制御群狄,盡收匈奴故地,自雲中、五原以東抵遼水,皆為鮮卑庭。數犯塞寇邊,幽、并苦之。田豫有馬城之圍,畢軌有陘北之敗。青龍中,帝乃聽王雄,遣劒客刺之。然後種落離散,互相侵伐,彊者遠遁,弱者請服。由是邊陲差安,漢南少事,雖時頗鈔盜,不能復相扇動矣。烏丸、鮮卑即古所謂東胡也。其習俗、前事,撰漢記者已錄而載之矣。故但舉漢末魏初以來,以備四夷之變云。
 
 
\gezhu{魏書曰:烏丸者,東胡也。漢初,匈奴冒頓滅其國,餘類保烏丸山,因以為號焉。俗善騎射,隨水草放牧,居無常處,以穹廬為宅,皆東向。日弋獵禽獸,食肉飲酪,以毛毳為衣。貴少賤老,其性悍驁,怒則殺父兄,而終不害其母,以母有族類,父兄以己為種,無復報者故也。常推募勇健能理決鬬訟相侵犯者為大人,邑落各有小帥,不世繼也。數百千落自為一部,大人有所召呼,刻木為信,邑落傳行,無文字,而部衆莫敢違犯。氏姓無常,以大人健者名字為姓。大人已下,各自畜牧治產,不相徭役。其嫁娶皆先私通,畧將女去,或半歲百日,然後遣媒人送馬牛羊以為聘娶之禮。婿隨妻歸,見妻家無尊卑,旦起皆拜,而不自拜其父母。為妻家僕役二年,妻家乃厚遣送女,居處財物,一出妻家。故其俗從婦人計,至戰鬪時,乃自決之。父子男女,相對蹲踞,悉髠頭以為輕便。婦人至嫁時乃養髮,分為髻,著句決,飾以金碧,猶中國有冠步搖也。父兄死,妻後母執嫂;若無執嫂者,則己子以親之次妻伯叔焉,死則歸其故夫。俗識鳥獸孕乳,時以四節,耕種常用布穀鳴為候。地宜青穄、東牆,東牆似蓬草,實如葵子,至十月熟。能作白酒,而不知作麴糱。米常仰中國。大人能作弓矢鞌勒,鍛金鐵為兵器,能刺韋作文繡,織縷氊𣮷。有病,知以艾灸,或燒石自熨,燒地卧上,或隨痛病處,以刀決脉出血,及祝天地山川之神,無鍼藥。貴兵死,斂屍有棺,始死則哭,葬則歌舞相送。肥養犬,以采繩嬰牽,并取亡者所乘馬、衣物、生時服飾,皆燒以送之。特屬累犬,使護死者神靈歸乎赤山。赤山在遼東西北數千里,如中國人以死之魂神歸泰山也。至葬日,夜聚親舊員坐,牽犬馬歷位,或歌哭者,擲肉與之。使二人口誦呪文,使死者魂神徑至,歷嶮阻,勿令橫鬼遮護,達其赤山,然後殺犬馬衣物燒之。敬鬼神,祠天地日月星辰山川,及先大人有健名者,亦同祠以牛羊,祠畢皆燒之。飲食必先祭。其約法,違大人言死,盜不止死。其相殘殺,令都落自相報,相報不止,詣大人平之,有罪者出其牛羊以贖死命,乃止。自殺其父兄無罪。其亡叛為大人所捕者,諸邑落不肯受,皆逐使至雍狂地。地無山,有沙漠、流水、草木,多蝮虵,在丁令之西南,烏孫之東北,以窮困之。自其先為匈奴所破之後,人衆孤弱,為匈奴臣服,常歲輸牛馬羊,過時不具,輒虜其妻子。至匈奴壹衍鞮單于時,烏丸轉彊,發掘匈奴單于冢,將以報冒頓所破之恥。壹衍鞮單于大怒,發二萬騎以擊烏丸。大將軍霍光聞之,遣度遼將軍范明友將三萬騎出遼東追擊匈奴。比明友兵至,匈奴已引去。烏丸新被匈奴兵,乘其衰弊,遂進擊烏丸,斬首六千餘級,獲三王首還。後數復犯塞,明友輒征破之。至王莽末,並與匈奴為寇。光武定天下,遣伏波將軍馬援將三千騎,從五原關出塞征之,無利,而殺馬千餘匹。烏丸遂盛,鈔擊匈奴,匈奴轉徙千里,漠南地空。建武二十五年,烏丸大人郝且等九千餘人率衆詣闕,封其渠帥為侯王者八十餘人,使居塞內,布列遼東屬國、遼西、右北平、漁陽、廣陽、上谷、代郡、鴈門、太原、朔方諸郡界,招來種人,給其衣食,置校尉以領護之,遂為漢偵備,擊匈奴、鮮卑。至永平中,漁陽烏丸大人欽志賁帥種人叛,鮮卑還為寇害,遼東太守祭肜募殺志賁,遂破其衆。至安帝時,漁陽、右北平、鴈門烏丸率衆王無何等復與鮮卑、匈奴合,鈔畧代郡、上谷、涿郡、五原,乃以大司農何熈行車騎將軍,左右羽林五營士,發緣邊七郡黎陽營兵合二萬人擊之。匈奴降,鮮卑、烏丸各還塞外。是後,烏丸稍復親附,拜其大人戎末廆為都尉。至順帝時,戎末廆率將王侯咄歸、去延等從烏丸校尉耿曄出塞擊鮮卑有功,還皆拜為率衆王,賜束帛。}
 
 
\end{pinyinscope}