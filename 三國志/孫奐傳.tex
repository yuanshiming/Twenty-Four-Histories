\article{孫奐傳}
\begin{pinyinscope}
 
 
 孫奐字季明。兄皎旣卒,代統其衆,以揚武中郎將領江夏太守。在事一年,遵皎舊迹,禮劉靖、李允、吳碩、張梁及江夏閭舉等,並納其善。奐訥於造次而敏於當官,軍民稱之。黃武五年,權攻石陽,奐以地主,使所部將軍鮮于丹帥五千人先斷淮道,自帥吳碩、張梁五千人為軍前鋒,降高城,得三將。大軍引還,權詔使在前往,駕過其軍,見奐軍陣整齊,權歎曰:「初吾憂其遲鈍,今治軍,諸將少能及者,吾無憂矣。」拜揚威將軍,封沙羨侯。吳碩、張梁皆裨將軍,賜爵關內侯。
 
 
\gezhu{江表傳曰:初權在武昌,欲還都建業,而慮水道泝流二千里,一旦有警,不相赴及,以此懷疑。及至夏口,於塢中大會百官議之,詔曰:「諸將吏勿拘位任,其有計者,為國言之。」諸將或陳宜立柵柵夏口,或言宜重設鐵鎖者,權皆以為非計。時梁為小將,未有知名,乃越席而進曰:「臣聞香餌引泉魚,重幣購勇士,今宜明樹賞罰之信,遣將入沔,與敵爭利,形勢旣成,彼不敢干也。使武昌有精兵萬人,付智略者任將,常使嚴整。一旦有警,應聲相赴。作甘水城,輕艦數十,諸所宜用,皆使備具。如此開門延敵,敵自不來矣。」權以梁計為最得,即超增梁位。後稍以功進至沔中督。}
 奐亦愛樂儒生,復命部曲子弟就業,後仕進朝廷者數十人。年四十,嘉禾三年卒。子承嗣,以昭武中郎將代統兵,領郡。赤烏六年卒,無子,封承庶弟壹奉奐後,襲業為將。孫峻之誅諸葛恪也,壹與全熈、施績攻恪弟公安督融,融自殺。壹從鎮南遷鎮軍,假節督夏口。及孫綝誅滕胤、呂據,據、胤皆壹之妹夫也,壹弟封又知胤、據謀,自殺。綝遣朱異潛襲壹。異至武昌,壹知其攻己,率部曲千餘口過將胤妻奔魏。魏以壹為車騎將軍、儀同三司,封吳侯,以故主芳貴人邢氏妻之。邢美色妬忌,下不堪命,遂共殺壹及邢氏。壹入魏,黃初三年死。
 
 
\end{pinyinscope}