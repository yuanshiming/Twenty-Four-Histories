\article{孫皎傳}
\begin{pinyinscope}
 
 
 皎字叔朗,始拜護軍校尉,領衆二千餘人。是時曹公數出濡須,皎每赴拒,號為精銳。遷都護征虜將軍,代程普督夏口。黃蓋及兄瑜卒,又并其軍。賜沙羨、雲杜、南新市、竟陵為奉邑,自置長吏。輕財能施,善於交結,與諸葛瑾至厚,委廬江劉靖以得失,江夏李允以衆事,廣陵吳碩、河南張梁以軍旅,而傾心親待,莫不自盡。皎嘗遣兵候獲魏邊將吏美女以進皎,皎更其衣服送還之,下令曰:「今所誅者曹氏,其百姓何罪?自今以往,不得擊其老弱。」由是江淮間多歸附者。嘗以小故與甘寧忿爭,或以諫寧,寧曰:「臣子一例,征虜雖公子,何可專行侮人邪!吾值明主,但當輸効力命,以報所天,誠不能隨俗屈曲矣。」權聞之,以書讓皎曰:「自吾與北方為敵,中間十年,初時相遲年小,今者且三十矣。孔子言『三十而立』,非但謂五經也。授卿以精兵,委卿以大任,都護諸將於千里之外,欲使如楚任昭奚恤,揚威於北境,非徒相使逞私志而已。近聞卿與甘興霸飲,因酒發作,侵陵其人,其人求屬呂蒙督中。此人雖麤豪,有不如人意時,然其較略大丈夫也。吾親之者,非私之也。吾親愛之,卿踈憎之;卿所為每與吾違,其可乆乎?夫居敬而行簡,可以臨民;愛人多容,可以得衆。二者尚不能知,安可董督在遠,禦寇濟難乎?卿行長大,特受重任,上有遠方瞻望之觀,下有部曲朝夕從事,何可恣意有盛怒邪?人誰無過,貴其能改,宜追前愆,深自咎責。今故煩諸葛子瑜重宣吾意。臨書摧愴,心悲淚下。」皎得書,上疏陳謝,遂與寧結厚。後呂蒙當襲南郡,權欲令皎與蒙為左右部大督,蒙說權曰:「若至尊以征虜能,宜用之;以蒙能,宜用蒙。昔周瑜、程普為左右部督,共攻江陵,雖事決於瑜,普自恃乆將,且俱是督,遂共不睦,幾敗國事,此目前之戒也。」權寤,謝蒙曰:「以卿為大督,命皎為後繼。」禽關羽,定荊州,皎有力焉。建安二十四年卒。權追錄其功,封子胤為丹楊侯。胤卒,無子。弟晞嗣,領兵,有罪自殺,國除。弟咨、彌、儀皆將軍,封侯。咨羽林督,儀無難督。咨為滕胤所殺,儀為孫峻所害。
 
 
\end{pinyinscope}