\article{蘇則傳}
\begin{pinyinscope}
 
 
 蘇則字文師,扶風武功人也。少以學行聞,舉孝廉茂才,辟公府,皆不就。起家為酒泉太守,轉安定、武都,
 
 
\gezhu{魏書曰:則剛直疾惡,常慕汲黯之為人。魏略曰:則世為著姓,興平中,三輔亂,饑窮,避難北地。客安定,依富室師亮。亮待遇不足,則慨然歎曰:「天下會安,當不乆爾,必還為此郡守,折庸輩士也。」後與馮翊吉茂等隱於郡南太白山中,以書籍自娛。及為安定太守,而師亮等皆欲逃走。則聞之,豫使人解語,以禮報之。}
 所在有威名。太祖征張魯,過其郡,見則恱之,使為軍導。魯破,則綏安下辯諸氐,通河西道,徙為金城太守。是時喪亂之後,吏民流散饑窮,戶口損耗,則撫循之甚謹。外招懷羌胡,得其牛羊,以養貧老。與民分糧而食,旬月之閒,流民皆歸,得數千家。乃明為禁令,有干犯者輒戮,其從教者必賞。親自教民耕種,其歲大豐收,由是歸附者日多。
 
 
李越以隴西反,則率羌胡圍越,越即請服。太祖崩,西平麴演叛,稱護羌校尉。則勒兵討之。演恐,乞降。文帝以其功,加則護羌校尉,賜爵關內侯。
 \gezhu{魏名臣奏載文帝令問雍州刺史張旣曰:「試守金城太守蘇則,旣有綏民平夷之功,聞又出軍西定湟中,為河西作聲勢,吾甚嘉之。則之功効,為可加爵邑未邪?封爵重事,故以問卿。密白意,且勿宣露也。」旣荅曰:「金城郡,昔為韓遂所見屠剥,死喪流亡,或竄戎狄,或陷寇亂,戶不滿五百。則到官,內撫彫殘,外鳩離散,今見戶千餘。又梁燒雜種羌,昔與遂同惡,遂斃之後,越出障塞。則前後招懷,歸就郡者三千餘落,皆卹以威恩,為官效用。西平麴演等唱造邪謀,則尋出軍,臨其項領,演則歸命送質,破絕賊糧。則旣有卹民之效,又能和戎狄,盡忠效節。遭遇聖明,有功必錄。若則加爵邑,誠足以勸忠臣,勵風俗也。」}
 
 
 
 
 後演復結旁郡為亂,張掖張進執太守杜通,酒泉黃華不受太守辛機,進、華皆自稱太守以應之。又武威三種胡並寇鈔,道路斷絕。武威太守毌丘興告急於則。時雍、涼諸豪皆驅略羌胡以從進等,郡人咸以為進不可當。又將軍郝昭、魏平先是各屯守金城,亦受詔不得西度。則乃見郡中大吏及昭等與羌豪帥謀曰:「今賊雖盛,然皆新合,或有脅從,未必同心;因釁擊之,善惡必離,離而歸我,我增而彼損矣。旣獲益衆之實,且有倍氣之勢,率以進討,破之必矣。若待大軍,曠日持乆,善人無歸,必合於惡,善惡旣合,勢難卒離。雖有詔命,違而合權,專之可也。」於是昭等從之,乃發兵救武威,降其三種胡,與興擊進於張掖。演聞之,將步騎三千迎則,辭來助軍,而實欲為變。則誘與相見,因斬之,出以徇軍,其黨皆散走。則遂與諸軍圍張掖,破之,斬進及其支黨,衆皆降。演軍敗,華懼,出所執乞降,河西平。乃還金城。進封都亭侯,邑三百戶。
 
 
徵拜侍中,與董昭同寮。昭嘗枕則膝卧,則推下之,曰:「蘇則之膝,非佞人之枕也。」初,則及臨菑侯植聞魏氏代漢,皆發服悲哭,文帝聞植如此,而不聞則也。帝在洛陽,嘗從容言曰:「吾應天而禪,而聞有哭者,何也?」則謂為見問,鬚髯悉張,欲正論以對。侍中傅巽掐
 \gezhu{苦洽反。}
 則曰:「不謂卿也。」於是乃止。
 \gezhu{魏略曰:舊儀,侍中親省起居,故俗謂之執虎子。始則同郡吉茂者,是時仕甫歷縣令,遷為宂散。茂見則,嘲之曰:「仕進不止執虎子。」則笑曰:「我誠不能效汝蹇蹇驅鹿車馳也。」初,則在金城,聞漢帝禪位,以為崩也,乃發喪;後聞其在,自以不審,意頗默然。臨菑侯植自傷失先帝意,亦怨激而哭。其後文帝出游,追恨臨菑,顧謂左右曰:「人心不同,當我登大位之時,天下有哭者。」時從臣知帝此言,有為而發也,而則以為為己,欲下馬謝。侍中傅巽目之,乃悟。孫盛曰:夫士不事其所非,不非其所事,趣舍出處,而豈徒哉!則旣策名新朝,委質異代,而方懷二心生忿,欲奮爽言,豈大雅君子去就之分哉?詩云:「士也罔極,二三其德。」士之二三,猶喪妃偶,況人臣乎?}
 文帝問則曰:「前破酒泉、張掖,西域通使,燉煌獻徑寸大珠,可復求市益得不?」則對曰:「若陛下化洽中國,德流沙漠,即不求自至;求而得之,不足貴也。」帝嘿然。後則從行獵,槎桎拔,失鹿,帝大怒,踞胡牀拔刀,悉收督吏,將斬之。則稽首曰:「臣聞古之聖王不以禽獸害人,今陛下方隆唐堯之化,而以獵戲多殺羣吏,愚臣以為不可。敢以死請!」帝曰:「卿,直臣也。」遂皆赦之。然以此見憚。黃初四年,左遷東平相。未至,道病薨,謚曰剛侯。子怡嗣。怡薨,無子,弟愉襲封。愉,咸熈中為尚書。
 \gezhu{愉字休豫,歷位太常光祿大夫,見晉百官名。山濤啟事稱愉忠篤有智意。臣松之案愉子紹,字世嗣,為吳王師。石崇妻,紹之兄女也。紹有詩在金谷集。紹弟慎,左衞將軍。}
 
 
\end{pinyinscope}