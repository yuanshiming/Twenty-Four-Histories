\article{孫和傳}
\begin{pinyinscope}
 
 
 孫和字子孝,慮弟也。少以母王有寵見愛,年十四,為置宮衞,使中書令闞澤教以書藝。好學下士,甚見稱述。赤烏五年,立為太子,時年十九。闞澤為太傅,薛綜為少傅,而蔡穎、張純、封俌、嚴維等皆從容侍從。
 
 
\gezhu{吳書曰:和少岐嶷有智意,故權尤愛幸,常在左右,衣服禮秩雕玩珍異之賜,諸子莫得比焉。好文學,善騎射,承師涉學,精識聦敏,尊敬師傅,愛好人物。穎等每朝見進賀,和常降意,歡以待之。講校經義,綜察是非,及訪諮朝臣,考績行能,以知優劣,各有條貫。後諸葛壹偽叛以誘魏將諸葛誕,權潛軍待之。和以權暴露外次,又戰者凶事,常憂勞憯怛,不復會同飲食,數上諫,戒令持重,務在全勝,權還,然後敢安。張純字元基,敦之子。吳錄曰:純少厲操行,學博才秀,切問捷對,容止可觀。拜郎中,補廣德令,治有異績,擢為太子輔義都尉。}
 
 
 
 
 是時有司頗以條書問事,和以為姦妄之人,將因事錯意,以生禍心,不可長也,表宜絕之。又都督劉寶白庶子丁晏,晏亦白寶,和謂晏曰:「文武在事,當能幾人,因隙構簿,圖相危害,豈有福哉?」遂兩釋之,使之從厚。常言當世士人宜講脩術學,校習射御,以周世務,而但交游博弈以妨事業,非進取之謂。後羣寮侍宴,言及博弈,以為妨事費日而無益於用,勞精損思而終無所成,非所以進德脩業,積累功緒者也。且志士愛日惜力,君子慕其大者,高山景行,恥非其次。夫以天地長乆,而人居其間,有白駒過隙之喻,年齒一暮,榮華不再。凡所患者,在於人情所不能絕,誠能絕無益之欲以奉德義之塗,棄不急之務以脩功業之基,其於名行,豈不善哉?夫人情猶不能無嬉娛,嬉娛之好,亦在於飲宴琴書射御之間,何必博弈,然後為歡。乃命侍坐者八人,各著論以矯之。於是中庶子韋曜退而論奏,和以示賔客。時蔡穎好弈,直事在署者頗斆焉,故以此諷之。
 
 
是後王夫人與全公主有隙。權嘗寢疾,和祠祭於廟,和妃叔父張休居近廟,邀和過所居。全公主使人覘視,因言太子不在廟中,專就妃家計議;又言王夫人見上寢疾,有喜色。權由是發怒,夫人憂死,而和寵稍損,懼於廢黜。魯王霸覬覦滋甚,陸遜、吾粲、顧譚等數陳適庶之義,理不可奪,全寄、楊笁為魯王霸支黨,譖愬日興。粲遂下獄誅,譚徙交州。權沈吟者歷年,
 \gezhu{殷基通語曰:初權旣立和為太子,而封霸為魯王,初拜猶同宮室,禮秩未分。羣公之議,以為太子、國王上下有序,禮秩宜異,於是分宮別僚,而隙端開矣。自侍御賔客造為二端,仇黨疑貳,滋延大臣。丞相陸遜、大將軍諸葛恪、太常顧譚、驃騎將軍朱據、會稽太守滕胤、大都督施績、尚書丁密等奉禮而行,宗事太子,驃騎將軍步隲、鎮南將軍呂岱、大司馬全琮、左將軍呂據、中書令孫弘等附魯王,中外官僚將軍大臣舉國中分。權患之,謂侍中孫峻曰:「子弟不睦,臣下分部,將有袁氏之敗,為天下笑。一人立者,安得不亂?」於是有改嗣之規矣。臣松之以為袁紹、劉表謂尚、琮為賢,本有傳後之意,異於孫權旣以立和而復寵霸,坐生亂階,自構家禍,方之袁、劉,昏悖甚矣。步隲以德度著稱,為吳良臣,而阿附於霸,事同楊笁,何哉?和旣正位,適庶分定,就使才德不殊,猶將義不黨庶,況霸實無聞,而和為令嗣乎?夫邪僻之人,豈其舉體無善,但一為不善,衆美皆亡耳。隲若果有此事,則其餘不足觀矣!呂岱、全琮之徒,蓋所不足論耳。}
 後遂幽閉和。於是驃騎將軍朱據、尚書僕射屈晃率諸將吏泥頭自縛,連日詣闕請和。權登白爵觀見,甚惡之,勑據、晃等無事忩忩。權欲廢和立亮,無難督陳正、五營督陳象上書,稱引晉獻公殺申生,立奚齊,晉國擾亂,又據、晃固諫不止。權大怒,族誅正、象,據、晃牽入殿,杖一百,
 \gezhu{吳歷曰:晃入,口諫曰:「太子仁明,顯聞四海。今三方鼎跱,實不宜搖動太子,以生衆心。願陛下少垂聖慮,老臣雖死,猶生之年。」叩頭流血,辭氣不撓。權不納晃言,斥還田里。孫皓即位,詔曰:「故僕射屈晃,志匡社稷,忠諫亡身。封晃子緒為東陽亭侯,弟幹、恭為立義都尉。」緒後亦至尚書僕射。晃,汝南人,見胡沖荅問。吳書曰:張純亦盡言極諫,權幽之,遂棄市。}
 竟徙和於故鄣,羣司坐諫誅放者十數。衆咸冤之。
 \gezhu{吳書曰:權寢疾,意頗感寤,欲徵和還立之,全公主及孫峻、孫弘等固爭之,乃止。}
 
 
太元二年正月,封和為南陽王,遣之長沙。
 \gezhu{吳書曰:和之長沙,行過蕪湖,有鵲巢于帆檣,故官寮聞之皆憂慘,以為檣末傾危,非乆安之象。或言鵲巢之詩有「積行累功以致爵位」之言,今王至德茂行,復受國土,儻神靈以此告寤人意乎?}
 四月,權薨,諸葛恪秉政。恪即和妃張之舅也。妃使黃門陳遷之建業上疏中宮,并致問於恪。臨去,恪謂遷曰:「為我達妃,期當使勝他人。」此言頗泄。又恪有徙都意,使治武昌宮,民間或言欲迎和。及恪被誅,孫峻因此奪和璽綬,徙新都,又遣使者賜死。和與妃張辭別,張曰:「吉凶當相隨,終不獨生活也。」亦自殺,舉邦傷焉。
 
 
孫休立,封和子皓為烏程侯,自新都之本國。休薨,皓即阼,其年追謚父和曰文皇帝,改葬明陵,置園邑二百家,令、丞奉守。後年正月,又分吳郡、丹楊九縣為吳興郡,治烏程,置太守,四時奉祠。有司奏言,宜立廟京邑。寶鼎二年七月,使守大匠薛珝營立寢堂,號曰清廟。十二月,遣守丞相孟仁、太常姚信等備官寮中軍步騎二千人,以靈輿法駕,東迎神於明陵。皓引見仁,親拜送於庭。
 \gezhu{吳書曰:比仁還,中使手詔,日夜相繼,奉問神靈起居動止。巫覡言見和被服,顏色如平生日,皓悲喜涕淚,悉召公卿尚書詣闕門下受賜。}
 靈輿當至,使丞相陸凱奉三牲祭於近郊,皓於金城外露宿。明日,望拜於東門之外。其翌日,拜廟薦祭,歔欷悲感。比七日三祭,倡技晝夜娛樂。有司奏言「祭不欲數,數則黷,宜以禮斷情」,然後止。
 \gezhu{吳歷曰:和四子:皓、德、謙、俊。孫休即位,封德錢唐侯,謙永安侯,俊拜騎都尉。皓在武昌,吳興施但因民之不堪命,聚萬餘人,劫謙,將至秣陵,欲立之。未至三十里住,擇吉日,但遣使以謙命詔丁固、諸葛靚。靚即斬其使。但遂前到九里,固、靚出擊,大破之。但兵裸身無鎧甲,臨陣皆披散。謙獨坐車中,遂生獲之。固不敢殺,以狀告皓,皓酖之,母子皆死。俊,張承外孫,聦明辨惠,為遠近所稱,皓又殺之。}
 
 
\end{pinyinscope}