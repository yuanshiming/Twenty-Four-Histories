\article{趙儼傳}
\begin{pinyinscope}
 
 
 趙儼字伯然,潁川陽翟人也。避亂荊州,與杜襲、繁欽通財同計,合為一家。太祖始迎獻帝都許,儼謂欽曰:「曹鎮東應期命世,必能匡濟華夏,吾知歸矣。」建安二年,年二十七,遂扶持老弱詣太祖,太祖以儼為朗陵長。縣多豪猾,無所畏忌。儼取其尤甚者,收縛案驗,皆得死罪。儼旣囚之,乃表府解放,自是威恩並著。時袁紹舉兵南侵,遣使招誘豫州諸郡,諸郡多受其命。惟陽安郡不動,而都尉李通急錄戶調。儼見通曰:「方今天下未集,諸郡並叛,懷附者復收其緜絹,小人樂亂,能無遺恨!且遠近多虞,不可不詳也。」通曰:「紹與大將軍相持甚急,左右郡縣背叛乃爾。若緜絹不調送,觀聽者必謂我顧望,有所須待也。」儼曰:「誠亦如君慮;然當權其輕重,小緩調,當為君釋此患。」乃書與荀彧曰:「今陽安郡當送緜絹,道路艱阻,必致寇害。百姓困窮,鄰城並叛,易用傾蕩,乃一方安危之機也。且此郡人執守忠節,在險不貳。微善必賞,則為義者勸。善為國者,藏之於民。以為國家宜垂慰撫,所歛緜絹,皆俾還之。」彧報曰:「輒白曹公,公文下郡,緜絹悉以還民。」上下歡喜,郡內遂安。
 
 
 
 
 入為司空掾屬主簿。
 
 
\gezhu{魏略曰:太祖北拒袁紹,時遠近無不私遺牋記,通意於紹者。儼與領陽安太守李通同治,通亦欲遣使。儼為陳紹必敗意,通乃止。及紹破走,太祖使人搜閱紹記室,惟不見通書疏,陰知儼必為之計,乃曰:「此必趙伯然也。」臣松之案魏武紀:破紹後,得許下軍中人書,皆焚之。若故使人搜閱,知其有無,則非所以安人情也。疑此語為不然。}
 時于禁屯潁陰,樂進屯陽翟,張遼屯長社,諸將任氣,多共不協;使儼并參三軍,每事訓喻,遂相親睦。太祖征荊州,以儼領章陵太守,徙都督護軍,護于禁、張遼、張郃、朱靈、李典、路招、馮楷七軍。復為丞相主簿,遷扶風太守。太祖徙出故韓遂、馬超等兵五千餘人,使平難將軍殷署等督領,以儼為關中護軍,盡統諸軍。羌虜數來寇害,儼率署等追到新平,大破之。屯田客呂並自稱將軍,聚黨據陳倉,儼復率署等攻之,賊即破滅。
 
 
時被書差千二百兵往助漢中守,署督送之。行者卒與室家別,皆有憂色。署發後一日,儼慮其有變,乃自追至斜谷口,人人慰勞,又深戒署。還宿雍州刺史張旣舍。署軍復前四十里,兵果叛亂,未知署吉凶。而儼自隨步騎百五十人,皆與叛者同部曲,或婚姻,得此問,各驚,被甲持兵,不復自安。儼欲還,旣等以為「今本營黨已擾亂,一身赴之無益,可須定問」。儼曰:「雖疑本營與叛者同謀,要當聞行者變,乃發之。又有欲善不能自定,宜及猶豫,促撫寧之。且為之元帥,旣不能安輯,身受禍難,命也。」遂去。行三十里止,放馬息,盡呼所從人,喻以成敗,慰勵懇切。皆慷慨曰:「死生當隨護軍,不敢有二。」前到諸營,各召料簡諸姦結叛者八百餘人,散在原野,惟取其造謀魁率治之,餘一不問。郡縣所收送,皆放遣,乃即相率還降。儼密白:「宜遣將詣大營,請舊兵鎮守關中。」太祖遣將軍劉柱將二千人,當須到乃發遣,而事露,諸營大駭,不可安喻。儼謂諸將曰:「舊兵旣少,東兵未到,是以諸營圖為邪謀。若或成變,為難不測。因其狐疑,當令早決。」遂宣言當差留新兵之溫厚者千人鎮守關中,其餘悉遣東。便見主者,內諸營兵名籍,案累重,立差別之。留者意定,與儼同心。其當去者亦不敢動,儼一日盡遣上道,因使所留千人,分布羅落之。東兵尋至,乃復脅喻,并徙千人,令相及共東,凡所全致二萬餘口。
 \gezhu{孫盛曰:盛聞為國以禮,民非信不立。周成不棄桐葉之言,晉文不違伐原之誓,故能隆刑措之道,建一匡之功。儼旣詐留千人,使效心力,始雖權也,宜以信終。兵威旣集,而又逼徙。信義喪矣,何以臨民?}
 
 
 
 
 關羽圍征南將軍曹仁於樊。儼以議郎參仁軍事南行,與平寇將軍徐晃俱前。旣到,羽圍仁遂堅,餘救兵未到。晃所督不足解圍,而諸將呵責晃促救。儼謂諸將曰:「今賊圍素固,水潦猶盛。我徒卒單少,而仁隔絕不得同力,此舉適所以弊內外耳。當今不若前軍偪圍,遣諜通仁,使知外救,以勵將士。計北軍不過十日,尚足堅守。然後表裏俱發,破賊必矣。如有緩救之戮,余為諸軍當之。」諸將皆喜,便作地道,箭飛書與仁,消息數通,北軍亦至,并勢大戰。羽軍旣退,舟船猶據沔水,襄陽隔絕不通,而孫權襲取羽輜重,羽聞之,即走南還。仁會諸將議,咸曰:「今因羽危懼,必可追禽也。」儼曰:「權邀羽連兵之難,欲掩制其後,顧羽還救,恐我承其兩疲,故順辭求效,乘釁因變,以觀利鈍耳。今羽已孤迸,更宜存之以為權害。若深入追北,權則改虞於彼,將生患於我矣。王必以此為深慮。」仁乃解嚴。太祖聞羽走,恐諸將追之,果疾勑仁,如儼所策。
 
 
文帝即王位,為侍中。頃之,拜駙馬都尉,領河東太守,典農中郎將。黃初三年,賜爵關內侯。孫權寇邊,征東大將軍曹休統五州軍禦之,徵儼為軍師。權衆退,軍還,封宜土亭侯,轉為度支中郎將,遷尚書。從征吳,到廣陵,復留為征東軍師。明帝即位,進封都鄉侯,邑六百戶,監荊州諸軍事,假節。會疾,不行,復為尚書,出監豫州諸軍事,轉大司馬軍師,入為大司農。齊王即位,以儼監雍、涼諸軍事,假節,轉征蜀將軍,又遷征西將軍,都督雍、涼。正始四年,老疾求還,徵為驃騎將軍,
 \gezhu{魏略曰:舊故四征有官厨財籍,遷轉之際,無不因緣。而儼义手上車,發到霸上,忘持其常所服藥。雍州聞之,乃追送雜藥材數箱,儼笑曰:「人言語殊不易,我偶問所服藥耳,何用是為邪?」遂不取。}
 遷司空。薨,謚曰穆侯。子亭嗣。初,儼與同郡辛毗、陳羣、杜襲並知名,號曰辛、陳、杜、趙云。
 
 
\end{pinyinscope}