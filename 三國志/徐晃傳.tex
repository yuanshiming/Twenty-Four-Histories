\article{徐晃傳}
\begin{pinyinscope}
 
 
 徐晃字公明,河東楊人也。為郡吏,從車騎將軍楊奉討賊有功,拜騎都尉。李傕、郭汜之亂長安也,晃說奉,令與天子還洛陽,奉從其計。天子渡河至安邑,封晃都亭侯。及到洛陽,韓暹、董承日爭鬬,晃說奉令歸太祖;奉欲從之,後悔。太祖討奉於梁,晃遂歸太祖。
 
 
 
 
 太祖授晃兵,使擊卷、
 
 
\gezhu{卷音墟權反。}
 原武賊,破之,拜裨將軍。從征呂布,別降布將趙庶、李鄒等。與史渙斬眭固於河內。從破劉備,又從破顏良,拔白馬,進至延津,破文醜,拜偏將軍。與曹洪擊㶏彊賊祝臂,破之,又與史渙擊袁紹運車於故市,功最多,封都亭侯。太祖旣圍鄴,破邯鄲,易陽令韓範偽以城降而拒守,太祖遣晃攻之。晃至,飛矢城中,為陳成敗。範悔,晃輙降之。旣而言於太祖曰:「二袁未破,諸城未下者傾耳而聽,今日滅易陽,明日皆以死守,恐河北無定時也。願公降易陽以示諸城,則莫不望風。」太祖善之。別討毛城,設伏兵掩擊,破三屯。從破袁譚於南皮,討平原叛賊,克之。從征蹋頓,拜橫野將軍。從征荊州,別屯樊,討中廬、臨沮、宜城賊。又與滿寵討關羽於漢津,與曹仁擊周瑜於江陵。十五年,討太原反者,圍大陵,拔之,斬賊帥商曜。韓遂、馬超等反關右,遣晃屯汾陰以撫河東,賜牛酒,令上先人墓。太祖至潼關,恐不得渡,召問晃。晃曰:「公盛兵於此,而賊不復別守蒲阪,知其無謀也。今假臣精兵
 \gezhu{臣松之云:案晃于時未應稱臣,傳寫者誤也。}
 渡蒲阪津,為軍先置,以截其裏,賊可擒也。」太祖曰:「善。」使晃以步騎四千人渡津。作塹柵未成,賊梁興夜將步騎五千餘人攻晃,晃擊走之,太祖軍得渡。遂破超等,使晃與夏侯淵平隃麋、汧諸氐,與太祖會安定。太祖還鄴,使晃與夏侯淵平鄜、夏陽餘賊,斬梁興,降三千餘戶。從征張魯。別遣晃討攻櫝、仇夷諸山氐,皆降之。遷平寇將軍。解將軍張順圍。擊賊陳福等三十餘屯,皆破之。
 
 
 
 
 太祖還鄴,留晃與夏侯淵拒劉備於陽平。備遣陳式等十餘營絕馬鳴閣道,晃別征破之,賊自投山谷,多死者。太祖聞,甚喜,假晃節,令曰:「此閣道,漢中之險要咽喉也。劉備欲斷絕外內,以取漢中。將軍一舉,克奪賊計,善之善者也。」太祖遂自至陽平,引出漢中諸軍。復遣晃助曹仁討關羽,屯宛。會漢水暴隘,于禁等沒。羽圍仁於樊,又圍將軍呂常於襄陽。晃所將多新卒,以羽難與爭鋒,遂前至陽陵陂屯。太祖復還,遣將軍徐商、呂建等詣晃,令曰:「須兵馬集至,乃俱前。」賊屯偃城。晃到,詭道作都塹,示欲截其後,賊燒屯走。晃得偃城,兩面連營,稍前,去賊圍三丈所。未攻,太祖前後遣殷署、朱蓋等凡十二營詣晃。賊圍頭有屯,又別屯四冢。晃揚聲當攻圍頭屯,而密攻四冢。羽見四冢欲壞,自將步騎五千出戰,晃擊之,退走,遂追陷與俱入圍,破之,或自投沔水死。太祖令曰:「賊圍塹鹿角十重,將軍致戰全勝,遂陷賊圍,多斬首虜。吾用兵三十餘年,及所聞古之善用兵者,未有長驅徑入敵圍者也。且樊、襄陽之在圍,過於莒、即墨,將軍之功,踰孫武、穰苴。」晃振旅還摩陂,太祖迎晃七里,置酒大會。太祖舉巵酒勸晃,且勞之曰:「全樊、襄陽,將軍之功也。」時諸軍皆集,太祖案行諸營,士卒咸離陣觀,而晃軍營整齊,將士駐陣不動。太祖歎曰:「徐將軍可謂有周亞夫之風矣。」
 
 
 
 
 文帝即王位,以晃為右將軍,進封逯鄉侯。及踐阼,進封楊侯。與夏侯尚討劉備於上庸,破之。以晃鎮陽平,徙封陽平侯。明帝即位,拒吳將諸葛瑾於襄陽。增邑二百,并前三千一百戶。病篤,遺令歛以時服。
 
 
 
 
 性儉約畏慎,將軍常遠斥候,先為不可勝,然後戰,追奔爭利,士不暇食。常歎曰:「古人患不遭明君,今幸遇之,常以功自效,何用私譽為!」終不廣交援。太和元年薨,謚曰壯侯。子蓋嗣。蓋薨,子霸嗣。明帝分晃戶,封晃子孫二人列侯。
 
 
初,清河朱靈為袁紹將。太祖之征陶謙,紹使靈督三營助太祖,戰有功。紹所遣諸將各罷歸,靈曰:「靈觀人多矣,無若曹公者,此乃真明主也。今以遇,復何之?」遂留不去。所將士卒慕之,皆隨靈留。靈後遂為好將,名亞晃等,至後將軍,封高唐亭侯。
 \gezhu{九州春秋曰:初,清河季雍以鄃叛袁紹而降公孫瓚,瓚遣兵衞之。紹遣靈攻之。靈家在城中,瓚將靈母弟置城上,誘呼靈。靈望城涕泣曰:「丈夫一出身與人,豈復顧家耶!」遂力戰拔之,生擒雍而靈家皆死。魏書曰:靈字文博。太祖旣平兾州,遣靈將新兵五千人騎千匹守許南。太祖戒之曰:「兾州新兵,數承寬緩,暫見齊整,意尚怏怏。卿名先有威嚴,善以道寬之,不然即有變。」靈至陽翟,中郎將程昂等果反,即斬昂,以狀聞。太祖手書曰:「兵中所以為危險者,外對敵國,內有姦謀不測之變。昔鄧禹中分光武軍西行,而有宗歆、馮愔之難,後將二十四騎還洛陽,禹豈以是減損哉?來書懇惻,多引咎過,未必如所云也。」文帝即位,封靈鄃侯,增其戶邑。詔曰:「將軍佐命先帝,典兵歷年,威過方、邵,功踰絳、灌。圖籍所美,何以加焉?朕受天命,帝有海內,元功之將,社稷之臣,皆朕所與同福共慶,傳之無窮者也。今封隃侯。富貴不歸故鄉,如夜行衣繡。若平常所志,願勿難言。」靈謝曰:「高唐,宿所願。」於是更封高唐侯,薨,謚曰威侯。子術嗣。}
 
 
 
 
 評曰:太祖建茲武功,而時之良將,五子為先。于禁最號毅重,然弗克其終。張郃以巧變為稱,樂進以驍果顯名,而鑒其行事,未副所聞。或注記有遺漏,未如張遼、徐晃之備詳也。
 
 
\end{pinyinscope}