\article{胡綜傳}
\begin{pinyinscope}
 
 
 胡綜字偉則,汝南固始人也。少孤,母將避難江東。孫策領會稽太守,綜年十四,為門下循行,留吳與孫權共讀書。策薨,權為討虜將軍,以綜為金曹從事,從討黃祖,拜鄂長。權為車騎將軍,都京,召綜還,為書部,與是儀、徐詳俱典軍國密事。劉備下白帝,權以見兵少,使綜料諸縣,得六千人,立解煩兩部,詳領左部、綜領右部督。吳將晉宗叛歸魏,魏以宗為蘄春太守,去江數百里,數為寇害。權使綜與賀齊輕行掩襲,生虜得宗,加建武中郎將。魏拜權為吳王,封綜、儀、詳皆為亭侯。
 
 
 
 
 黃武八年夏,黃龍見夏口,於是權稱尊號,因瑞改元。又作黃龍大牙,常在中軍,諸軍進退,視其所向,命綜作賦曰:
 
 
 
 
 乾坤肇立,三才是生。狼弧垂象,實惟兵精。聖人觀法,是效是營,始作器械,爰求厥成。黃、農創代,拓定皇基,上順天心,下息民災。高辛誅共,舜征有苗,啟有甘師,湯有鳴條。周之牧野,漢之垓下,靡不由兵,克定厥緒。明明大吳,實天生德,神武是經,惟皇之極。乃自在昔,黃、虞是祖,越歷五代,繼世在下。應期受命,發迹南土,將恢大繇,革我區夏。乃律天時,制為神軍,取象太乙,五將三門;疾則如電,遲則如雲,進止有度,約而不煩。四靈旣布,黃龍處中,周制日月,實曰太常,桀然特立,六軍所望。仙人在上,鑒觀四方,神寔使之,為國休祥。軍欲轉向,黃龍先移,金鼓不鳴,寂然變施,闇謨若神,可謂祕奇。在昔周室,赤烏銜書,今也大吳,黃龍吐符。合契河洛,動與道俱,天贊人和,僉曰惟休。
 
 
 
 
 蜀聞權踐阼,遣使重申前好。綜為盟文,文義甚美,語在權傳。
 
 
 
 
 權下都建業,詳、綜並為侍中,進封鄉侯,兼左右領軍。時魏降人或云魏都督河北振威將軍吳質,頗見猜疑,綜乃偽為質作降文三條:
 
 
 
 
 其一曰:「天綱弛絕,四海分崩,羣生憔悴,士人播越,兵寇所加,邑無居民,風塵煙火,往往而處,自三代以來,大亂之極,未有若今時者也。臣質志薄,處時無方,繫於土壤,不能飜飛,遂為曹氏執事戎役,遠處河朔,天衢隔絕,雖望風慕義,思託大命,媿無因緣,得展其志。每往來者,竊聽風化,伏知陛下齊德乾坤,同明日月,神武之姿,受之自然,敷演皇極,流化萬里,自江以南,戶受覆燾。英雄俊傑,上達之士,莫不心歌腹詠,樂在歸附者也。今年六月末,奉聞吉日,龍興踐阼,恢弘大繇,整理天綱,將使遺民,覩見定主。昔武王伐殷,殷民倒戈;高祖誅項,四面楚歌。方之今日,未足以喻。臣質不勝昊天至願,謹遣所親同郡黃定恭行奉表,及託降叛,間關求達,其欲所陳,載列于左。」
 
 
 
 
 其二曰:「昔伊尹去夏入商,陳平委楚歸漢,書功竹帛,遺名後世,世主不謂之背誕者,以為知天命也。臣昔為曹氏所見交接,外託君臣,內如骨肉,恩義綢繆,有合無離,遂受偏方之任,總河北之軍。當此之時,志望高大,永與曹氏同死俱生,惟恐功之不建,事之不成耳。及曹氏之亡,後嗣繼立,幼沖統政,讒言彌興。同儕者以勢相害,異趣者得間其言,而臣受性簡略,素不下人,視彼數子,意實迫之,此亦臣之過也。遂為邪議所見搆會,招致猜疑,誣臣欲叛。雖識真者保明其心,世亂讒勝,餘嫌猶在,常懼一旦橫受無辜,憂心孔疚,如履冰炭。昔樂毅為燕昭王立功於齊,惠王即位,疑奪其任,遂去燕之趙,休烈不虧。彼豈欲二三其德,蓋畏功名不建,而懼禍之將及也。昔遣魏郡周光以賈販為名,託叛南詣,宣達密計。時以倉卒,未敢便有章表,使光口傳而已。以為天下大歸可見,天意所在,非吳復誰?此方之民,思為臣妾,延頸舉踵,惟恐兵來之遲耳。若使聖恩少加信納,當以河北承望王師,欵心赤實,天日是鑒。而光去經年,不聞咳唾,未審此意竟得達不?瞻望長歎,日月以幾,魯望高子,何足以喻!又臣今日見待稍薄,蒼蠅之聲,緜緜不絕,必受此禍,遲速事耳。臣私度陛下未垂明慰者,必以臣質貫穿仁義之道,不行若此之事,謂光所傳,多虛少實,或謂此中有他消息,不知臣質搆讒見疑,恐受大害也。且臣質若有罪之,且自當奔赴鼎鑊,束身待罪,此蓋人臣之宜也。今日無罪,橫見譖毀,將有商鞅、白起之禍。尋惟事勢,去亦宜也。死而弗義,不去何為!樂毅之出,吳起之走,君子傷其不遇,未有非之者也。願陛下推古況今,不疑怪於臣質也。又念人臣獲罪,當如伍員奉己自效,不當徼幸因事為利。然今與古,厥勢不同,南北悠遠,江湖隔絕,自不舉事,何得濟免!是以忘志士之節,而思立功之義也。且臣質又以曹氏之嗣,非天命所在,政弱刑亂,柄奪於臣,諸將專威於外,各自為政,莫或同心,士卒衰耗,帑藏空虛,綱紀毀廢,上下並昏,想前後數得降叛,具聞此問。兼弱攻昧,宜應天時,此實陛下進取之秋,是以區區敢獻其計。今若內兵淮、泗,據有下邳,荊、揚二州,聞聲響應,臣從河北席卷而南,形勢一連,根牙永固。關西之兵繫於所衞,青、徐二州不敢徹守,許、洛餘兵衆不滿萬,誰能來東與陛下爭者?此誠千載一會之期,可不深思而熟計乎!及臣所在,旣自多馬,加以羗胡常以三四月中美草時,驅馬來出,隱度今者,可得三千餘匹。陛下出軍,當投此時,多將騎士來就馬耳。此皆先定所一二知。凡兩軍不能相究虛實,今此間實羸,易可克定,陛下舉動,應者必多。上定洪業,使普天一統,下令臣質建非常之功,此乃天也。若不見納,此亦天也。願陛下思之,不復多陳。」
 
 
 
 
 其三曰:「昔許子遠舍袁就曹,規畫計較,應見納受,遂破袁軍,以定曹業。向使曹氏不信子遠,懷疑猶豫,不決於心,則今天下袁氏有也。願陛下思之。間聞界上將閻浮、趙楫欲歸大化,唱和不速,以取破亡。今臣欵欵,遠授其命,若復懷疑,不時舉動,令臣孤絕,受此厚禍,即恐天下雄夫烈士欲立功者,不敢復託命陛下矣。願陛下思之。皇天后土,實聞其言。」
 
 
 
 
 此文旣流行,而質已入為侍中矣。
 
 
 
 
 二年,青州人隱蕃歸吳,上書曰:「臣聞紂為無道,微子先出;高祖寬明,陳平先入。臣年二十二,委棄封域,歸命有道,賴蒙天靈,得自全致。臣至止有日,而主者同之降人,未見精別,使臣微言妙旨,不得上達。於邑三歎,曷惟其已。謹詣闕拜章,乞蒙引見。」權即召入。蕃謝荅問,及陳時務,甚有辭觀。綜時侍坐,權問何如,綜對曰:「蕃上書,大語有似東方朔,巧捷詭辯有似禰衡,而才皆不及。」權又問可堪何官,綜對曰:「未可以治民,且試以都輦小職。」權以蕃盛論刑獄,用為廷尉監。左將軍朱據、廷尉郝普稱蕃有王佐之才,普尤與之親善,常怨歎其屈。後蕃謀叛,事覺伏誅,
 
 
\gezhu{吳錄曰:蕃有口才,魏明帝使詐叛如吳,令求作廷尉職,重案大臣以離間之。旣為廷尉監,衆人以據、普與蕃親善,常車馬雲集,賔客盈堂。及至事覺,蕃亡走,捕得,考問黨與,蕃無所言。吳主使將入,謂曰:「何乃以肌肉為人受毒乎?」蕃曰:「孫君,丈夫圖事,豈有無伴!烈士死,不足相牽耳。」遂閉口而死。吳歷曰:權問普:「卿前盛稱蕃,又為之怨望朝廷,使蕃反叛,皆卿之由。」}
 普見責自殺。據禁止,歷時乃解。拜綜偏將軍,兼左執法,領辭訟。遼東之事,輔吳將軍張昭以諫權言辭切至,權亦大怒,其和協彼此,使之無隙,綜有力焉。
 
 
 
 
 性嗜酒,酒後歡呼極意,或推引杯觴,搏擊左右。權愛其才,弗之責也。
 
 
 
 
 凡自權統事,諸文誥策命,鄰國書符,略皆綜之所造也。初以內外多事,特立科,長吏遭喪,皆不得去,而數有犯者。權患之,使朝臣下議。綜議以為宜定科文,示以大辟,行之一人,其後必絕。遂用綜言,由是奔喪乃斷。
 
 
赤烏六年卒,子冲嗣。冲平和有文幹,天紀中為中書令
 \gezhu{吳錄曰:冲後仕晉尚書郎、吳郡太守。}
 
 
 
 
 徐詳者字子明,吳郡烏程人也,先綜死。
 
 
 
 
 評曰:是儀、徐詳、胡綜,皆孫權之時幹興事業者也。儀清恪貞素,詳數通使命,綜文采才用,各見信任,譬之廣夏,其榱椽之佐乎!
 
 
\end{pinyinscope}