\article{鮑勛傳}
\begin{pinyinscope}
 
 
 鮑勛字叔業,泰山平陽人也,漢司隷校尉鮑宣九世孫。宣後嗣有從上黨徙泰山者,遂家焉。勛父信,靈帝時為騎都尉,大將軍何進遣東募兵。後為濟北相,恊規太祖,身以遇害。語在董卓傳、武帝紀。
 
 
\gezhu{魏書曰:信父丹,官至少府侍中,世以儒雅顯。少有大節,寬厚愛人,沈毅有謀。大將軍何進辟拜騎都尉,遣歸募兵,得千餘人,還到成皐而進已遇害。信至京師,董卓亦始到。信知卓必為亂,勸袁紹襲卓,紹畏卓不敢發。語在紹傳。信乃引軍還鄉里,收徒衆二萬,騎七百,輜重五千餘乘。是歲,太祖始起兵於己吾,信與弟韜以兵應太祖。太祖與袁紹表信行破虜將軍,韜裨將軍。時紹衆最盛,豪傑多向之。信獨謂太祖曰:「夫略不世出,能總英雄以撥亂反正者,君也。苟非其人,雖彊必斃。君殆天之所啟!」遂深自結納,太祖亦親異焉。汴水之敗,信被瘡,韜在陣戰亡。紹劫奪韓馥位,遂據兾州。信言於太祖曰:「姧臣乘釁,蕩覆王室,英雄奮節,天下嚮應者,義也。今紹為盟主,因權專利,將自生亂,是復有一卓也。若抑之,則力不能制,祗以遘難,又何能濟?且可規大河之南,以待其變。」太祖善之。太祖為東郡太守,表信為濟北相。會黃巾大衆入州界,劉岱欲與戰,信止之,岱不從,遂敗。語在武紀。太祖以賊恃勝而驕,欲設奇兵挑擊之於壽張。先與信出行戰地,後步軍未至,而卒與賊遇,遂接戰。信殊死戰,以救太祖,太祖僅得潰圍出,信遂沒,時年四十一。雖遭亂起兵,家本脩儒,治身至儉,而厚養將士,居無餘財,士以此歸之。}
 建安十七年,太祖追錄信功,表封勛兄邵新都亭侯。
 \gezhu{魏書曰:邵有父風,太祖嘉之,加拜騎都尉,使持節。邵薨,子融嗣。}
 辟勛丞相掾。
 \gezhu{魏書曰:勛清白有高節,知名於世。}
 
 
 
 
 二十二年,立太子,以勛為中庶子。徙黃門侍郎,出為魏郡西部都尉。太子郭夫人弟為曲周縣吏,斷盜官布,法應棄市。太祖時在譙,太子留鄴,數手書為之請罪。勛不敢擅縱,具列上。勛前在東宮,守正不撓,太子固不能恱,及重此事,恚望滋甚。會郡界休兵有失期者,密勑中尉奏免勛官。乆之,拜侍御史。延康元年,太祖崩,太子即王位,勛以駙馬都尉兼侍中。
 
 
 
 
 文帝受禪,勛每陳「今之所急,唯在軍農,寬惠百姓。臺榭苑囿,宜以為後。」文帝將出游獵,勛停車上疏曰:「臣聞五帝三王,靡不明本立教,以孝治天下。陛下仁聖惻隱,有同古烈。臣兾當繼蹤前代,令萬世可則也。如何在諒闇之中,脩馳騁之事乎!臣冒死以聞,唯陛下察焉。」帝手毀其表而競行獵,中道頓息,問侍臣曰:「獵之為樂,何如八音也?」侍中劉曄對曰:「獵勝於樂。」勛抗辭曰:「夫樂,上通神明,下和人理,隆治致化,萬邦咸乂。故移風易俗莫善於樂。況獵,暴華蓋於原野,傷生育之至理,櫛風沐雨,不以時隙哉?昔魯隱觀漁於棠,春秋譏之。雖陛下以為務,愚臣所不願也。」因奏:「劉曄佞諛不忠,阿順陛下過戲之言。昔梁丘據取媚於遄臺,曄之謂也。請有司議罪以清皇朝。」帝怒作色,罷還,即出勛為右中郎將。
 
 
 
 
 黃初四年,尚書令陳羣、僕射司馬宣王並舉勛為宮正,宮正即御史中丞也。帝不得已而用之,百寮嚴憚,罔不肅然。六年秋,帝欲征吳,羣臣大議,勛面諫曰:「王師屢征而未有所克者,蓋以吳、蜀唇齒相依,憑阻山水,有難拔之勢故也。往年龍舟飄蕩,隔在南岸,聖躬蹈危,臣下破膽。此時宗廟幾至傾覆,為百世之戒。今又勞兵襲遠,日費千金,中國虛耗,令黠虜玩威,臣竊以為不可。」帝益忿之,左遷勛為治書執法。
 
 
 
 
 帝從壽春還,屯陳留郡界。太守孫邕見,出過勛。時營壘未成,但立摽埒,邕邪行不從正道,軍營令史劉曜欲推之,勛以塹壘未成,解止不舉。大軍還洛陽,曜有罪,勛奏絀遣,而曜密表勛私解邕事。詔曰:「勛指鹿作馬,收付廷尉。」廷尉法議:「正刑五歲。」三官駮:「依律罰金二斤。」帝大怒曰:「勛無活分,而汝等敢縱之!收三官已下付刺姦,當令十鼠同穴。」太尉鍾繇、司徒華歆、鎮軍大將軍陳羣、侍中辛毗、尚書衞臻、守廷尉高柔等並表「勛父信有功於太祖」,求請勛罪。帝不許,遂誅勛。勛內行旣脩,廉而能施,死之日,家無餘財。後二旬,文帝亦崩,莫不為勛歎恨。
 
 
\end{pinyinscope}