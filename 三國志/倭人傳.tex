\article{倭人傳}
\begin{pinyinscope}
 
 
 倭人在帶方東南大海之中,依山島為國邑。舊百餘國,漢時有朝見者,今使譯所通三十國。從郡至倭,循海岸水行,歷韓國,乍南乍東,到其北岸狗邪韓國,七千餘里,始度一海,千餘里至對海國。其大官曰卑狗,副曰卑奴母離。所居絕島,方可四百餘里,土地山險,多深林,道路如禽鹿徑。有千餘戶,無良田,食海物自活,乖船南北巿糴。又南渡一海千餘里,名曰瀚海,至一大國,官亦曰卑狗,副曰卑奴母離。方可三百里,多竹木叢林,有三千許家,差有田地,耕田猶不足食,亦南北巿糴。又渡一海,千餘里至末盧國,有四千餘戶,濵山海居,草木茂盛,行不見前人。好捕魚鰒,水無深淺,皆沉沒取之。東南陸行五百里,到伊都國,官曰爾支,副曰泄謨觚、柄渠觚。有千餘戶,世有王,皆統屬女王國,郡使往來常所駐。東南至奴國百里,官曰兕馬觚,副曰卑奴母離,有二萬餘戶。東行至不彌國百里,官曰多模,副曰卑奴母離,有千餘家。南至投馬國,水行二十日,官曰彌彌,副曰彌彌那利,可五萬餘戶。南至邪馬壹國,女王之所都,水行十日,陸行一月。官有伊支馬,次曰彌馬升,次曰彌馬獲支,次曰奴佳鞮,可七萬餘戶。自女王國以北,其戶數道里可得略載,其餘旁國遠絕,不可得詳。次有斯馬國,次有已百支國,次有伊邪國,次有都支國,次有彌奴國,次有好古都國,次有不呼國,次有姐奴國,次有對蘇國,次有蘇奴國,次有呼邑國,次有華奴蘇奴國,次有鬼國,次有為吾國,次有鬼奴國,次有邪馬國,次有躬臣國,次有巴利國,次有支惟國,次有烏奴國,次有奴國,此女王境界所盡。其南有狗奴國,男子為王,其官有狗古智卑狗,不屬女王。自郡至女王國萬二千餘里。
 
 
 
 
 男子無大小皆黥面文身。自古以來,其使詣中國,皆自稱大夫。夏后少康之子封於會稽,斷髮文身以避蛟龍之害。今倭水人好沉沒捕魚蛤,文身亦以厭大魚水禽,後稍以為飾。諸國文身各異,或左或右,或大或小,尊卑有差。計其道里,當在會稽、東冶之東。其風俗不淫,男子皆露紒,以木緜招頭。其衣橫幅,但結束相連,略無縫。婦人被髮屈紒,作衣如單被,穿其中央,貫頭衣之。種禾稻、紵麻,蠶桑、緝績,出細紵、縑緜。其地無牛馬虎豹羊鵲。兵用矛、楯、木弓。木弓短下長上,竹箭或鐵鏃或骨鏃,所有無與儋耳、朱崖同。倭地溫暖,冬夏食生菜,皆徒跣。有屋室,父母兄弟卧息異處,以朱丹塗其身體,如中國用粉也。食飲用籩豆,手食。其死,有棺無槨,封土作冢。始死停喪十餘日,當時不食肉,喪主哭泣,他人就歌舞飲酒。已葬,舉家詣水中澡浴,以如練沐。其行來渡海詣中國,恒使一人,不梳頭,不去蟣蝨,衣服垢污,不食肉,不近婦人,如喪人,名之為持衰。若行者吉善,共顧其生口財物;若有疾病,遭暴害,便欲殺之,謂其持衰不謹。出真珠、青玉。其山有丹,其木有柟、杼、豫樟、楺櫪、投橿、烏號、楓香,其竹篠簳、桃支。有薑、橘、椒、蘘荷,不知以為滋味。有獮猴、黑雉。其俗舉事行來,有所云為,輒灼骨而卜,以占吉凶,先告所卜,其辭如令龜法,視火坼占兆。其會同坐起,父子男女無別,人性嗜酒。
 
 
\gezhu{魏略曰:其俗不知正歲四節,但計春耕秋收為年紀。}
 見大人所敬,但搏手以當跪拜。其人壽考,或百年,或八九十年。其俗,國大人皆四五婦,下戶或二三婦。婦人不淫,不妬忌。不盜竊,少諍訟。其犯法,輕者沒其妻子,重者沒其門戶。及宗族尊卑,各有差序,足相臣服。收租賦。有邸閣國,國有市,交易有無,使大倭監之。自女王國以北,特置一大率檢察,諸國畏憚之。常治伊都國,於國中有如刺史。王遣使詣京都、帶方郡、諸韓國,及郡使倭國,皆臨津搜露,傳送文書賜遺之物詣女王,不得差錯。下戶與大人相逢道路,逡巡入草。傳辭說事,或蹲或跪,兩手據地,為之恭敬。對應聲曰噫,比如然諾。
 
 
 
 
 其國本亦以男子為王,住七八十年,倭國亂,相攻伐歷年,乃共立一女子為王,名曰卑彌呼,事鬼道,能惑衆,年已長大,無夫壻,有男弟佐治國。自為王以來,少有見者。以婢千人自侍,唯有男子一人給飲食,傳辭出入。居處宮室樓觀,城柵嚴設,常有人持兵守衞。
 
 
 
 
 女王國東渡海千餘里,復有國,皆倭種。又有侏儒國在其南,人長三四尺,去女王四千餘里。又有裸國、黑齒國復在其東南,船行一年可至。參問倭地,絕在海中洲島之上,或絕或連,周旋可五千餘里。
 
 
景初二年六月,倭女王遣大夫難升米等詣郡,求詣天子朝獻,太守劉夏遣吏將送詣京都。其年十二月,詔書報倭女王曰:「制詔親魏倭王卑彌呼:帶方太守劉夏遣使送汝大夫難升米、次使都巿牛利奉汝所獻男生口四人,女生口六人、班布二匹二丈,以到。汝所在踰遠,乃遣使貢獻,是汝之忠孝,我甚哀汝。今以汝為親魏倭王,假金印紫綬,裝封付帶方太守假授汝。其綏撫種人,勉為孝順。汝來使難升米、牛利涉遠,道路勤勞,今以難升米為率善中郎將,牛利為率善校尉,假銀印青綬,引見勞賜遣還。今以絳地交龍錦五匹、
 \gezhu{臣松之以為地應為綈,漢文帝著皁衣謂之弋綈是也。此字不體,非魏朝之失,則傳寫者誤也。}
 絳地縐粟𦋺十張、蒨絳五十匹、紺青五十匹,荅汝所獻貢直。又特賜汝紺地句文錦三匹、細班華𦋺五張、白絹五十匹、金八兩、五尺刀二口、銅鏡百枚、真珠、鈆丹各五十斤,皆裝封付難升米、牛利還到錄受。悉可以示汝國中人,使知國家哀汝,故鄭重賜汝好物也。」
 
 
 
 
 正始元年,太守弓遵遣建中校尉梯儁等奉詔書印綬詣倭國,拜假倭王,并齎詔賜金、帛、錦𦋺、刀、鏡、采物,倭王因使上表荅謝恩詔。其四年,倭王復遣使大夫伊聲耆、掖邪狗等八人,上獻生口、倭錦、絳青縑、緜衣、帛布、丹木、򠐂、短弓矢。掖邪狗等壹拜率善中郎將印綬。其六年,詔賜倭難升米黃幢,付郡假授。其八年,太守王頎到官。倭女王卑彌呼與狗奴國男王卑彌弓呼素不和,遣倭載斯、烏越等詣郡說相攻擊狀。遣塞曹掾史張政等因齎詔書、黃幢,拜假難升米為檄告喻之。卑彌呼以死,大作冢,徑百餘步,徇葬者奴婢百餘人。更立男王,國中不服,更相誅殺,當時殺千餘人。復立卑彌呼宗女壹與,年十三為王,國中遂定。政等以檄告喻壹與,壹與遣倭大夫率善中郎將掖邪狗等二十人送政等還,因詣臺,獻上男女生口三十人,貢白珠五千,孔青大句珠二枚,異文雜錦二十匹。
 
 
 
 
 評曰:史、漢著朝鮮、兩越,東京撰錄西羗。魏世匈奴遂衰,更有烏丸、鮮卑,爰及東夷,使譯時通,記述隨事,豈常也哉!
 
 
\gezhu{魏略西戎傳曰:氐人有王,所從來乆矣。自漢開益州,置武都郡,排其種人,分竄山谷間,或在福祿,或在汧、隴左右。其種非一,稱槃瓠之後,或號青氐,或號白氐,或號蚺氐,此蓋蟲之類而處中國,人即其服色而名之也。其自相號曰盍稚,各有王侯,多受中國封拜。近去建安中,興國氐王阿貴、白項氐王千萬各有部落萬餘,至十六年,從馬超為亂。超破之後,阿貴為夏侯淵所攻滅,千萬西南入蜀,其部落不能去,皆降。國家分徙其前後兩端者,置扶風、美陽,今之安夷、撫夷二部護軍所典是也。其太守善,分留天水、南安界,今之廣平魏郡所守是也。其俗,語不與中國同,及羌雜胡同,各自有姓,姓如中國之姓矣。其衣服尚青絳。俗能織布,善田種,畜養豕牛馬驢騾。其婦人嫁時著衽露,其緣飾之制有似羌,衽露有似中國袍。皆編髮。多知中國語,由與中國錯居故也。其自還種落間,則自氐語。其嫁娶有似於羌,此蓋乃昔所謂西戎在於街、兾、豲道者也。今雖都統於郡國,然故自有王侯在其虛落間。又故武都地陰平街左右,亦有萬餘落。貲虜,本匈奴也,匈奴名奴婢為貲。始建武時,匈奴衰,分去其奴婢,亡匿在金城、武威、酒泉北黑水、西河東西,畜牧逐水草,抄盜涼州,部落稍多,有數萬,不與東部鮮卑同也。其種非一,有大胡,有丁令,或頗有羌雜處,由本亡奴婢故也。當漢、魏之際,其大人有檀柘,死後,其枝大人南近在廣魏、令居界,有禿瑰來數反,為涼州所殺。今有劭提,或降來,或遁去,常為西州道路患也。}
 
 
\gezhu{燉煌西域之南山中,從婼羌西至葱領數千里,有月氏餘種葱茈羌、白馬、黃牛羌,各有酋豪,北與諸國接,不知其道里廣狹。傳聞黃牛羌各有種類,孕身六月生,南與白馬羌鄰。西域諸國,漢初開其道,時有三十六,後分為五十餘。從建武已來,更相吞滅,於今有二十道。從燉煌玉門關入西域,前有二道,今有三道。從玉門關西出,經婼羌轉西,越葱領,經縣度,入大月氏,為南道。從玉門關西出,發都護井,回三隴沙北頭,經居盧倉,從沙西井轉西北,過龍堆,到故樓蘭,轉西詣龜茲,至葱領,為中道。從玉門關西北出,經橫坑,辟三隴沙及龍堆,出五舩北,到車師界戊己校尉所治高昌,轉西與中道合龜茲,為新道。凡西域所出,有前史已具詳,今故略說。南道西行,且志國、小宛國、精絕國、樓蘭國皆并屬鄯善也。戎盧國、扞彌國、渠勒國、皮穴國皆并屬于寘。罽賔國、大夏國、高附國、天笁國皆并屬大月氏。}
 
 
\gezhu{臨兒國,浮屠經云其國王生浮屠。浮屠,太子也。父曰屑頭邪,母云莫邪。浮屠身服色黃,髮青如青絲,乳青毛,蛉赤如銅。始莫邪夢白象而孕,及生,從母左脅出,生而有結,墮地能行七步。此國在天笁城中。天笁又有神人,名沙律。昔漢哀帝元壽元年,博士弟子景盧受大月氏王使伊存口受浮屠經曰復立者其人也。浮屠所載臨蒲塞、桑門、伯聞、疏問、白疏間、比丘、晨門,皆弟子號也。浮屠所載與中國老子經相出入,蓋以為老子西出關,過西域之天笁、教胡。浮屠屬弟子別號,合有二十九,不能詳載,故略之如此。}
 
 
\gezhu{車離國一名禮惟特,一名沛隷王,在天笁東南三千餘里,其地卑溼暑熱。其王治沙奇城,有別城數十,人民怯弱,月氏、天笁擊服之。其地東西南北數千里,人民男女皆長一丈八尺,乘象、橐駞以戰,今月氏役稅之。盤越國一名漢越王,在天笁東南數千里,與益部相近,其人小與中國人等,蜀人賈似至焉。南道而西極轉東南盡矣。中道西行尉犂國、危須國、山王國皆并屬焉耆,姑墨國、溫宿國、尉頭國皆并屬龜茲也。楨中國、莎車國、竭石國、渠沙國、西夜國、依耐國、蒲犁國、億若國、榆令國、捐毒國、休脩國、琴國皆并屬疏勒。自是以西,大宛、安息、條支、烏弋。烏弋一名排特,此四國次在西,本國也,無增損。前世謬以為條支在大秦西,今其實在東。前世又謬以為彊於安息,今更役屬之,號為安息西界。前世又謬以為弱水在條支西,今弱水在大秦西。前世又謬以為從條支西行二百餘日,近日所入,今從大秦西近日所入。大秦國一號犂靬,在安息、條支西大海之西,從安息界安谷城乘船,直截海西,遇風利二月到,風遲或一歲,無風或三歲。其國在海西,故俗謂之海西。有河出其國,西又有大海。海西有遲散城,從國下直北至烏丹城,西南又渡一河,乘船一日乃過。西南又渡一河,一日乃過。凡有大都三,却從安谷城陸道直北行之海北,復直西行之海西,復直南行經之烏遲散城,渡一河,乘船一日乃過。周回繞海,凡當渡大海六日乃到其國。國有小城邑合四百餘,東西南北數千里。其王治濵側河海,以石為城郭。其土地有松、栢、槐、梓、竹、葦、楊柳、胡桐、百草。民俗,田種五穀,畜乘有馬、騾、驢、駱駞。桑蠶。俗多奇幻,口中出火,自縛自解,跳十二丸巧妙。其國無常主,國中有災異,輒更立賢人以為王,而生放其故王,王亦不敢怨。其俗人長大平正,似中國人而胡服。自云本中國一別也,常欲通使於中國,而安息圖其利,不能得過。其俗能胡書。其制度,公私宮室為重屋,旌旗擊鼓,白蓋小車,郵驛亭置如中國。從安息繞海北到其國,人民相屬,十里一亭,三十里一置,終無盜賊。但有猛虎、師子為害,行道不羣則不得過。其國置小王數十,其王所治城周回百餘里,有官曹文書。王有五宮,一宮間相去十里,其王平旦之一宮聽事,至日暮一宿,明日復至一宮,五日一周。置三十六將,每議事,一將不至則不議也。王出行,常使從人持一韋囊自隨,有白言者,受其辭投囊中,還宮乃省為決理。以水精作宮柱及器物。作弓矢。其別枝封小國,曰澤散王,曰驢分王,曰且蘭王,曰賢督王,曰汜復王,曰于羅王,其餘小王國甚多,不能一一詳之也。國出細絺。作金銀錢,金錢一當銀十。有織成細布,言用水羊毳,名曰海西布。此國六畜皆出水,或云非獨用羊毛也,亦用木皮或野蠒絲作,織成氍毹、毾㲪、𦋺帳之屬皆好,其色又鮮於海東諸國所作也。又常利得中國絲,解以為胡綾,故數與安息諸國交市於海中。海水苦不可食,故往來者希到其國中。山出九色次玉石,一曰青,二曰赤,三曰黃,四曰白,五曰黑,六曰綠,七曰紫,八曰紅,九曰紺。今伊吾山中有九色石,即其類。陽嘉三年時,踈勒王臣槃獻海西青石、金帶各一。又今西域舊圖云𦋺賔、條支諸國出琦石,即次玉石也。大秦多金、銀、銅、鐵、鈆、錫、神龜、白馬、朱髦、駭雞犀、瑇瑁、玄熊、赤螭、辟毒鼠、大貝、車渠、馬腦、南金、翠爵、羽翮、象牙、符采玉、明月珠、夜光珠、真白珠、虎魄、珊瑚、赤白黑綠黃青紺縹紅紫十種流離、璆琳、琅玕、水精、玫瑰、雄黃、雌黃、碧、五色玉、黃白黑綠紫紅絳紺金黃縹留黃十種氍毹、五色毾㲪、五色九色首下毾㲪、金縷繡、雜色綾、金塗布、緋持布、發陸布、緋持渠布、火浣布、阿羅得布、巴則布、度伐布、溫宿布、五色桃布、絳地金織帳、五色斗帳、一微木、二蘇合、狄提、迷迷、兜納、白附子、薰陸、鬱金、芸膠、薰草木十二種香。大秦道旣從海北陸通,又循海而南,與交阯七郡外夷比,又有水道通益州、永昌、故永昌出異物。前世但論有水道,不知有陸道,今其略如此,其民人戶數不能備詳也。自葱領西,此國最大,置諸小王甚多,故錄其屬大者矣。}
 
 
\gezhu{澤散王屬大秦,其治在海中央,北至驢分,水行半歲,風疾時一月到,最與安息安谷城相近,西南詣大秦都不知里數。驢分王屬大秦,其治去大秦都二千里。從驢分城西之大秦渡海,飛橋長二百三十里,渡海道西南行,繞海直西行。且蘭王屬大秦。從思陶國直南渡河,乃直西行之且蘭三千里。道出河南,乃西行,從且蘭復直西行之汜復國六百里。南道會汜復,乃西南之賢督國。且蘭、汜復直南,乃有積石,積石南乃有大海,出珊瑚,真珠。且蘭、汜復、斯賔阿蠻北有一山,東西行。大秦、海西東各有一山,皆南北行。賢督王屬大秦,其治東北去汜復六百里。汜復王屬大秦,其治東北去於羅三百四十里渡海也。於羅屬大秦,其治在汜復東北,渡河,從於羅東北又渡河,斯羅東北又渡河。斯羅國屬安息,與大秦接也。大秦西有海水,海水西有河水,河水西南北行有大山,西有赤水,赤水西有白王山,白玉山有西王母,西王母西有脩流沙,流沙西有大夏國、堅沙國、屬繇國、月氏國,四國西有黑水,所傳聞西之極矣。}
 
 
\gezhu{北新道西行,東至且彌國、西且彌國、單桓國、畢陸國、蒲陸國、烏貪國,皆并屬車師後部王。王治于賴城,魏賜其王壹多雜守魏侍中,號大都尉,受魏王印。轉西北則烏孫、康居,本國無增損也。北烏伊別國在康居北,又有柳國,又有巖國,又有奄蔡國一名阿蘭,皆與康居同俗。西與大秦,東南與康居接。其國多名貂,畜牧逐水草,臨大澤,故時羈屬康居,今不屬也。呼得國在葱嶺北,烏孫西北,康居東北,勝兵萬餘人,隨畜牧,出好馬,有貂。堅昆國在康居西北,勝兵三萬人,隨畜牧,亦多貂,有好馬。}
 
 
\gezhu{丁令國在康居北,勝兵六萬人,隨畜牧,出名鼠皮,白昆子、青昆子皮。此上三國,堅昆中央,俱去匈奴單于庭安習水七千里,南去車師六國五千里,西南去康居界三千里,西去康居王治八千里。或以為此丁令即匈奴北丁令也,而北丁令在烏孫西,似其種別也。又匈奴北有渾窳國,有屈射國,有丁令國,有隔昆國,有新棃國,明北海之南自復有丁令,非此烏孫之西丁令也。烏孫長老言北丁令有馬脛國,其人音聲似鴈騖,從膝以上身頭,人也,膝以下生毛,馬脛馬蹄,不騎馬而走疾馬,其為人勇健敢戰也。短人國在康居西北,男女皆長三尺,人衆甚多,去奄蔡諸國甚遠。康居長老傳聞常有商度此國,去康居可萬餘里。}
 
 
\gezhu{魚豢議曰:俗以為營廷之魚不知江海之大,浮游之物不知四時之氣,是何也?以其所在者小與其生之短也。余今氾覽外夷大秦諸國,猶尚曠若發矇矣,況夫鄒衍之所推出,大易、太玄之所測度乎!徒限處牛蹄之涔,又無彭祖之年,無緣託景風以迅游,載騕褭以遐觀,但勞眺乎三辰,而飛思乎八荒耳。}
 
 
\end{pinyinscope}