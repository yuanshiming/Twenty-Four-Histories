\article{武宣卞皇后}
\begin{pinyinscope}
 
 
 武宣卞皇后,琅邪開陽人,文帝母也。本倡家,
 
 
\gezhu{魏書曰:后以漢延熹三年十二月己巳生齊郡白亭,有黃氣滿室移日。父敬侯怪之,以問卜者王旦,旦曰:「此吉祥也。」}
 年二十,太祖於譙納后為妾。後隨太祖至洛。及董卓為亂,太祖微服東出避難。袁術傳太祖凶問,時太祖左右至洛者皆欲歸,后止之曰:「曹君吉凶未可知,今日還家,明日若在,何靣目復相見也?正使禍至,共死何苦!」遂從后言。太祖聞而善之。建安初,丁夫人廢,遂以后為繼室。諸子無母者,太祖皆令后養之。
 \gezhu{魏略曰:太祖始有丁夫人,又劉夫人生子脩及清河長公主。劉早終,丁養子脩。子脩亡於穰,丁常言:「將我兒殺之,都不復念!」遂哭泣無節。太祖忿之,遣歸家,欲其意折。後太祖就見之,夫人方織,外人傳云「公至」,夫人踞機如故。太祖到,撫其背曰:「顧我共載歸乎!」夫人不顧,又不應。太祖却行,立於戶外,復云:「得無尚可邪!」遂不應,太祖曰:「真訣矣。」遂與絕,欲其家嫁之,其家不敢。初,丁夫人旣為嫡,加有子脩,丁視后母子不足。后為繼室,不念舊惡,因太祖出行,常四時使人饋遺,又私迎之,延以正坐而己下之,迎來送去,有如昔日。丁謝曰:「廢放之人,夫人何能常爾邪!」其後丁亡,后請太祖殯葬,許之,乃葬許城南。後太祖病困,自慮不起,歎曰:「我前後行意,於心未曾有所負也。假令死而有靈,子脩若問『我母所在』,我將何辭以荅!」魏書曰:后性約儉,不尚華麗,無文繡珠玉,器皆黑漆。太祖常得名璫數具,命后自選一具,后取其中者,太祖問其故,對曰:「取其上者為貪,取其下者為偽,故取其中者。」}
 文帝為太子,左右長御賀后曰:「將軍拜太子,天下莫不歡喜,后當傾府藏賞賜。」后曰:「王自以丕年大,故用為嗣,我但當以免無教導之過為幸耳,亦何為當重賜遺乎!」長御還,具以語太祖。太祖恱曰:「怒不變容,喜不失節,故是最為難。」
 
 
二十四年,拜為王后,策曰:「夫人卞氏,撫養諸子,有母儀之德。今進位王后,太子諸侯陪位,羣卿上壽,減國內死罪一等。」二十五年,太祖崩,文帝即王位,尊后曰王太后,及踐阼,尊后曰皇太后,稱永壽宮。
 \gezhu{魏書曰:后以國用不足,滅損御食,諸金銀器物皆去之。東阿王植,太后少子,最愛之。後植犯法,為有司所奏,文帝令太后弟子奉車都尉蘭持公卿議白太后,太后曰:「不意此兒所作如是,汝還語帝,不可以我故壞國法。」及自見帝,不以為言。臣松之案:文帝夢磨錢,欲使文滅而更愈明,以問周宣。宣荅曰:「此陛下家事,雖意欲爾,而太后不聽。」則太后用意,不得如此書所言也。魏書又曰:太后每隨軍征行,見高年白首,輙住車呼問,賜與絹帛,對之涕泣曰:「恨父母不及我時也。」太后每見外親,不假以顏色,常言「居處當務節儉,不當望賞賜,念自佚也。外舍當怪吾遇之太薄,吾自有常度故也。吾事武帝四五十年,行儉日乆,不能自變為奢,有犯科禁者,吾且能加罪一等耳,莫望錢米恩貸也。」帝為太后弟秉起第,第成,太后幸第請諸家外親,設下廚,無異膳。太后左右菜食粟飯,無魚肉。其儉如此。}
 明帝即位,尊太后曰太皇太后。
 
 
 
 
 黃初中,文帝欲追封太后父母,尚書陳羣奏曰:「陛下以聖德應運受命,創業革制,當永為後式。案典籍之文,無婦人分土命爵之制。在禮典,婦因夫爵。秦違古法,漢氏因之,非先王之令典也。」帝曰:「此議是也,其勿施行。以作著詔下藏之臺閣,永為後式。」至太和四年春,明帝乃追謚太后祖父廣曰開陽恭侯,父遠曰敬侯,祖母周封陽都君及敬侯夫人,皆贈印綬。其年五月,后崩。七月,合葬高陵。
 
 
初,太后弟秉,以功封都鄉侯,黃初七年進封開陽侯,邑千二百戶,為昭烈將軍。
 \gezhu{魏略曰:初,卞后弟秉,當建安時得為別部司馬,后常對太祖怨言,太祖荅言:「但得與我作婦弟,不為多邪?」后又欲太祖給其錢帛,太祖又曰:「但汝盜與,不為足邪?」故訖太祖世,秉官不移,財亦不益。}
 秉薨,子蘭嗣。少有才學,
 \gezhu{魏略曰:蘭獻賦贊述太子德美,太子報曰:「賦者,言事類之所附也,頌者,美盛德之形容也,故作者不虛其辭,受者必當其實。蘭此賦,豈吾實哉?昔吾丘壽王一陳寶鼎,何武等徒以歌頌,猶受金帛之賜,蘭事雖不諒,義足嘉也。今賜牛一頭。」由是遂見親敬。}
 為奉車都尉、游擊將軍,加散騎常侍。蘭薨,子暉嗣。
 \gezhu{魏略曰:明帝時,蘭見外有二難,而帝留意於宮室,常因侍從,數切諫。帝雖不能從,猶納其誠欵。後蘭苦酒消渴,時帝信巫女用水方,使人持水賜蘭,蘭不肯飲。詔問其意?蘭言治病自當以方藥,何信於此?帝為變色,而蘭終不服。後渴稍甚,以至於亡。故時人見蘭好直言,謂帝靣折之而蘭自殺,其實不然。}
 又分秉爵,封蘭弟琳為列侯,官至步兵校尉。蘭子隆女為高貴鄉公皇后,隆以后父為光祿大夫,位特進,封睢陽鄉侯,妻王為顯陽鄉君。追封隆前妻劉為順陽鄉君,后親母故也。琳女又為陳留王皇后,時琳已沒,封琳妻劉為廣陽鄉君。
 
 
\end{pinyinscope}