\article{張魯傳}
\begin{pinyinscope}
 
 
 張魯字公祺,沛國豐人也。祖父陵,客蜀,學道鵠鳴山中,造作道書以惑百姓,從受道者出五斗米,故世號米賊。陵死,子衡行其道。衡死,魯復行之。益州牧劉焉以魯為督義司馬,與別部司馬張脩將兵擊漢中太守蘇固,魯遂襲脩殺之,奪其衆。焉死,子璋代立,以魯不順,盡殺魯母家室。魯遂據漢中,以鬼道教民,自號「師君」。其來學道者,初皆名「鬼卒」。受本道已信,號「祭酒」。各領部衆,多者為治頭大祭酒。皆教以誠信不欺詐,有病自首其過,大都與黃巾相似。諸祭酒皆作義舍,如今之亭傳。又置義米肉,懸於義舍,行路者量腹取足;若過多,鬼道輒病之。犯法者,三原,然後乃行刑。不置長吏,皆以祭酒為治,民夷便樂之。雄據巴、漢垂三十年。
 
 
\gezhu{典略曰:熹平中,妖賊大起,三輔有駱曜。光和中,東方有張角,漢中有張脩。駱曜教民緬匿法,角為太平道,脩為五斗米道。太平道者,師持九節杖為符祝,教病人叩頭思過,因以符水飲之,得病或日淺而愈者,則云此人信道,其或不愈,則云不信道。脩法略與角同,加施靜室,使病者處其中思過。又使人為姦令祭酒,祭酒主以老子五千文,使都習,號為姦令。為鬼吏,主為病者請禱。請禱之法,書病人姓名,說服罪之意。作三通,其一上之天,著山上,其一埋之地,其一沈之水,謂之三官手書。使病者家出米五斗以為常,故號曰五斗米師。實無益於治病,但為淫妄,然小人昏愚,競共事之。後角被誅,脩亦亡。及魯在漢中,因其民信行脩業,遂增飾之。教使作義舍,以米肉置其中以止行人;又教使自隱,有小過者,當治道百步,則罪除;又依月令,春夏禁殺;又禁酒。流移寄在其地者,不敢不奉。臣松之謂張脩應是張衡,非典略之失,則傳寫之誤。}
 漢末,力不能征,遂就寵魯為鎮民中郎將,領漢寧太守,通貢獻而已。民有地中得玉印者,羣下欲尊魯為漢寧王。魯功曹巴西閻圃諫魯曰:「漢川之民,戶出十萬,財富土沃,四面險固;上匡天子,則為桓、文,次及竇融,不失富貴。今承制署置,勢足斬斷,不煩於王。願且不稱,勿為禍先。」魯從之。韓遂、馬超之亂,關西民從子午谷奔之者數萬家。
 
 
建安二十年,太祖乃自散關出武都征之,至陽平關。魯欲舉漢中降,其弟衞不肯,率衆數萬人拒關堅守。太祖攻破之,遂入蜀。
 \gezhu{魏名臣奏載董昭表曰:「武皇帝承涼州從事及武都降人之辭,說張魯易攻,陽平城下南北山相遠,不可守也,信以為然。及往臨履,不如所聞,乃歎曰:『他人商度,少如人意。』攻陽平山上諸屯,旣不時拔,士卒傷夷者多。武皇帝意沮,便欲拔軍截山而還,遣故大將軍夏侯惇、將軍許褚呼山上兵還。會前軍未還,夜迷惑,誤入賊營,賊便退散。侍中辛毗、劉曄等在兵後,語惇、褚,言『官兵已據得賊要屯,賊已散走』。猶不信之。惇前自見,乃還白武皇帝,進兵定之,幸而克獲。此近事,吏士所知。」又楊曁表曰:「武皇帝始征張魯,以十萬之衆,身親臨履,指授方略,因就民麥以為軍糧。張衞之守,蓋不足言。地險守易,雖有精兵虎將,勢不能施。對兵三日,欲抽軍還,言『作軍三十年,一朝持與人,如何』。此計已定,天祚大魏,魯守自壞,因以定之。」世語曰:魯遣五官掾降,弟衞橫山築陽平城以拒,王師不得進。魯走巴中。軍糧盡,太祖將還。西曹掾東郡郭諶曰:「不可。魯已降,留使旣未反,衞雖不同,偏攜可攻。縣軍深入,以進必克,退必不免。」太祖疑之。夜有野麋數千突壞衞營,軍大驚。夜,高祚等誤與衞衆遇,祚等多鳴鼓角會衆。衞懼,以為大軍見掩,遂降。}
 魯聞陽平已陷,將稽顙,圃又曰:「今以迫往,功必輕;不如依杜濩赴朴胡相拒,然後委質,功必多。」於是乃奔南山入巴中。左右欲悉燒寶貨倉庫,魯曰:「本欲歸命國家,而意未達。今之走,避銳鋒,非有惡意。寶貨倉庫,國家之有。」遂封藏而去。太祖入南鄭,甚嘉之。又以魯本有善意,遣人慰喻。魯盡將家出,太祖逆拜魯鎮南將軍,待以客禮,封閬中侯,邑萬戶。封魯五子及閻圃等皆為列侯。
 \gezhu{臣松之以為張魯雖有善心,要為敗而後降,今乃寵以萬戶,五子皆封侯,過矣。習鑿齒曰:魯欲稱王,而閻圃諫止之,今封圃為列侯。夫賞罰者,所以懲惡勸善也,苟其可以明軌訓於物,無遠近幽深矣。今閻圃諫魯勿王,而太祖追封之,將來之人孰不思順!塞其本源而末流自止,其此之謂與!若乃不明於此而重燋爛之功,豐爵厚賞止於死戰之士,則民利於有亂,俗競於殺伐,阻兵杖力,干戈不戢矣。太祖之此封,可謂知賞罰之本,雖湯武居之,無以加也。魏略曰:黃初中,增圃爵邑,在禮請中。後十餘歲病死。晉書云:西戎司馬閻纘,圃孫也。}
 為子彭祖取魯女。魯薨,諡之曰原侯。子富嗣。
 \gezhu{魏略曰:劉雄鳴者,藍田人也。少以采藥射獵為事,常居覆車山下,每晨夜,出行雲霧中,以識道不迷,而時人因謂之能為雲霧。郭、李之亂,人多就之。建安中,附屬州郡,州郡表薦為小將。馬超等反,不肯從,超破之。後詣太祖,太祖執其手謂之曰:「孤方入關,夢得一神人,即卿邪!」乃厚禮之,表拜為將軍,遣令迎其部黨。部黨不欲降,遂劫以反,諸亡命皆往依之,有衆數千人,據武關道口。太祖遣夏侯淵討破之,雄鳴南奔漢中。漢中破,窮無所之,乃復歸降。太祖捉其鬚曰:「老賊,真得汝矣!」復其官,徙勃海。時又有程銀、侯選、李堪,皆河東人也,興平之亂,各有衆千餘家。建安十六年,並與馬超合。超破走,堪臨陣死。銀、選南入漢中,漢中破,詣太祖降,皆復官爵。}
 
 
 
 
 評曰:公孫瓚保京,坐待夷滅。度殘暴而不節,淵仍業以載凶,祇足覆其族也。陶謙昬亂而憂死,張楊授首於臣下,皆擁據州郡,曾匹夫之不若,固無可論者也。燕、繡、魯舍羣盜,列功臣,去危亡,保宗祀,則於彼為愈焉。
 
 
\end{pinyinscope}