\article{溫恢傳}
\begin{pinyinscope}
 
 
 溫恢字曼基,太原祁人也。父恕,為涿郡太守,卒。恢年十五,送葬還歸鄉里,內足於財。恢曰:「世方亂,安以富為?」一朝盡散,振施宗族。州里高之,比之郇越。舉孝廉,為廩丘長,鄢陵、廣川令,彭城、魯相,所在見稱。入為丞相主簿,出為揚州刺史。太祖曰:「甚欲使卿在親近,顧以為不如此州事大。故書云:『股肱良哉!庶事康哉!』得無當得蔣濟為治中邪?」時濟見為丹楊太守,乃遣濟還州。又語張遼、樂進等曰:「揚州刺史曉達軍事,動靜與共咨議。」
 
 
 
 
 建安二十四年,孫權攻合肥,是時諸州皆屯戍。恢謂兖州刺史裴潛曰:「此閒雖有賊,不足憂,而畏征南方有變。今水生而子孝縣軍,無有遠備。關羽驍銳,乘利而進,必將為患。」於是有樊城之事。詔書召潛及豫州刺史呂貢等,潛等緩之。恢密語潛曰:「此必襄陽之急欲赴之也。所以不為急會者,不欲驚動遠衆。一二日必有密書促卿進道,張遼等又將被召。遼等素知王意,後召前至,卿受其責矣!」潛受其言,置輜重,更為輕裝速發,果被促令。遼等尋各見召,如恢所策。
 
 
 
 
 文帝踐阼,以恢為侍中,出為魏郡太守。數年,遷涼州刺史,持節領護羌校尉。道病卒,時年四十五。詔曰:「恢有柱石之質,服事先帝,功勤明著。及為朕執事,忠於王室,故授之以萬里之任,任之以一方之事。如何不遂,吾甚愍之!」賜恢子生爵關內侯。生早卒,爵絕。
 
 
 
 
 恢卒後,汝南孟建為涼州刺史,有治名,官至征東將軍。
 
 
\gezhu{魏略曰:建字公威,少與諸葛亮俱遊學。亮後出祁山,荅司馬宣王書,使杜子緒宣意於公威也。}
 
 
\end{pinyinscope}