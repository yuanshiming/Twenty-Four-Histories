\article{文德郭皇后}
\begin{pinyinscope}
 
 
 文德郭皇后,安平廣宗人也。祖世長吏。
 
 
\gezhu{魏書曰:父永,官至南郡太守,謚敬侯。母姓董氏,即堂陽君,生三男二女:長男浮,高唐令,次女昱,次即后,后弟都,弟成。后以漢中平元年三月乙卯生,生而有異常。}
 后少而父永奇之曰:「此乃吾女中王也。」遂以女王為字。早失二親,喪亂流離,沒在銅鞮侯家。太祖為魏公時,得入東宮。后有智數,時時有所獻納。文帝定為嗣,后有謀焉。太子即王位,后為夫人,及踐阼,為貴嬪。甄后之死,由后之寵也。黃初三年,將登后位,文帝欲立為后,中郎棧潛上疏曰:「在昔帝王之治天下,不唯外輔,亦有內助,治亂所由,盛衰從之。故西陵配黃,英娥降媯,並以賢明,流芳上世。桀奔南巢,禍階末喜;紂以炮烙,怡恱妲己。是以聖哲慎立元妃,必取先代世族之家,擇其令淑以統六宮,虔奉宗廟,陰教聿脩。易曰:『家道正而天下定。』由內及外,先王之令典也。春秋書宗人釁夏云,無以妾為夫人之禮。齊桓誓命于葵丘,亦曰『無以妾為妻』。今後宮嬖寵,常亞乘輿。若因愛登后,使賤人暴貴,臣恐後世下陵上替,開張非度,亂自上起也。」文帝不從,遂立為皇后。
 \gezhu{魏書曰:后上表謝曰:「妾無皇、英釐降之節,又非姜、任思齊之倫,誠不足以假充女君之盛位,處中饋之重任。」后自在東宮,及即尊位,雖有異寵,心愈恭肅,供養永壽宮,以孝聞。是時柴貴人亦有寵,后教訓獎導之。後宮諸貴人時有過失,常彌覆之,有譴讓,輙為帝言其本末,帝或大有所怒,至為之頓首請罪,是以六宮無怨。性儉約,不好音樂,常慕漢明德馬后之為人。}
 
 
后蚤喪兄弟,以從兄表繼永後,拜奉車都尉。后外親劉斐與他國為婚,后聞之,勑曰:「諸親戚嫁娶,自當與鄉里門戶匹敵者,不得因勢,彊與他方人婚也。」后姊子孟武還鄉里,求小妻,后止之。遂勑諸家曰:「今世婦女少,當配將士,不得因緣取以為妾也。宜各自慎,無為罰首。」
 \gezhu{魏書曰:后常勑戒表、武等曰:「漢氏椒房之家,少能自全者,皆由驕奢,可不慎乎!」}
 
 
 
 
 五年,帝東征,后留許昌永始臺。時霖雨百餘日,城樓多壞,有司奏請移止。后曰:「昔楚昭王出游,貞姜留漸臺,江水至,使者迎而無符,不去,卒沒。今帝在遠,吾幸未有是患,而便移止,柰何?」羣臣莫敢復言。六年,帝東征吳,至廣陵,后留譙宮。時表留宿衞,欲遏水取魚。后曰:「水當通運漕,又少材木,奴客不在目前,當復私取官竹木作梁遏。今奉車所不足者,豈魚乎?」
 
 
明帝即位,尊后為皇太后,稱永安宮。太和四年,詔封表安陽亭侯,又進爵鄉侯,增邑并前五百戶,遷中壘將軍。以表子詳為騎都尉。其年,帝追謚太后父永為安陽鄉敬侯,母董為都鄉君。遷表昭德將軍,加金紫,位特進,表第二子訓為騎都尉。及孟武母卒,欲厚葬,起祠堂,太后止之曰:「自喪亂以來,墳墓無不發掘,皆由厚葬也;首陽陵可以為法。」青龍三年春,后崩于許昌,以終制營陵,三月庚寅,葬首陽陵西。
 \gezhu{魏略曰:明帝旣嗣立,追痛甄后之薨,故太后以憂暴崩。甄后臨沒,以帝屬李夫人。及太后崩,夫人乃說甄后見譖之禍,不獲大斂,被髮覆靣,帝哀恨流涕,命殯葬太后,皆如甄后故事。漢晉春秋曰:初,甄后之誅,由郭后之寵,及殯,令被髮覆面,以糠塞口,遂立郭后,使養明帝。帝知之,心常懷忿,數泣問甄后死狀。郭后曰:「先帝自殺,何以責問我?且汝為人子,可追讎死父,為前母枉殺後母邪?」明帝怒,遂逼殺之,勑殯者使如甄后故事。魏書載哀策曰:「維青龍三年三月壬申,皇太后梓宮啟殯,將葬于首陽之西陵。哀子皇帝叡親奉冊祖載,遂親遣奠,叩心擗踊,號咷仰訴,痛靈魂之遷幸,悲容車之向路,背三光以潛翳,就黃壚而安厝。嗚呼哀哉!昔二女妃虞,帝道以彰,三母嬪周,聖善彌光,旣多受祉,享國延長。哀哀慈妣,興化閏房,龍飛紫極,作合聖皇,不虞中年,暴離災殃。愍予小子,煢煢摧傷,魂雖永逝,定省曷望?嗚呼哀哉!」}
 帝進表爵為觀津侯,增邑五百,并前千戶。遷詳為駙馬都尉。四年,追改封永為觀津敬侯,世婦董為堂陽君。追封謚后兄浮為梁里亭戴侯,都為武城亭孝侯,成為新樂亭定侯,皆使使者奉策,祠以太牢。表薨,子詳嗣,又分表爵封詳弟述為列侯。詳薨,子釗嗣。
 
 
\end{pinyinscope}