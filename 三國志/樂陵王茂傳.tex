\article{樂陵王茂傳}
\begin{pinyinscope}
 
 
 樂陵王茂,建安二十二年封萬歲亭侯。二十三年,改封平輿侯。黃初三年,進爵,徙封乘氏公。七年,徙封中丘。茂性慠佷,少無寵於太祖。及文帝世,又獨不王。太和元年,徙封聊城公,其年為王。詔曰:「昔象之為虐至甚,而大舜猶侯之有鼻。近漢氏淮南、阜陵,皆為亂臣逆子,而猶或及身而復國,或至子而錫土。有虞建之於上古,漢文、明、章行之乎前代,斯皆敦叙親親之厚義也。聊城公茂少不閑禮教,長不務善道。先帝以為古之立諸侯也,皆命賢者,故姬姓有未必侯者,是以獨不王茂。太皇太后數以為言。如聞茂頃來少知悔昔之非,欲脩善將來。君子與其進,不保其往也。今封茂為聊城王,以慰太皇太后下流之念。」六年,改封曲陽王。正始三年,東平靈王薨,茂稱嗌痛,不肯發哀,居處出入自若。有司奏除國土,詔削縣一,戶五百。五年,徙封樂陵,詔以茂租奉少,諸子多,復所削戶,又增戶七百。嘉平、正元、景元中,累增邑,并前五千戶。
 
 
 
 
 文皇帝九男:甄氏皇后生明帝,李貴人生贊哀王恊,潘淑媛生北海悼王蕤,朱淑媛生東武陽懷王鑒,仇昭儀生東海定王霖,徐姬生元城哀王禮,蘇姬生邯鄲懷王邕,張姬生清河悼王貢,宋姬生廣平哀王儼。
 
 
\end{pinyinscope}