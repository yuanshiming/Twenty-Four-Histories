\article{孫匡傳}
\begin{pinyinscope}
 
 
 孫匡字季佐,翊弟也。舉孝廉茂才,未試用,卒,時年二十餘。
 
 
\gezhu{江表傳曰:曹休出洞口,呂範率軍禦之。時匡為定武中郎將,遣範令放火,燒損茅芒以乏軍用,範即啟送匡還吳。權別其族為丁氏,禁固終身。臣松之案本傳曰:「匡未試用卒,時年二十餘。」而江表傳云呂範在洞口,匡為定武中郎將。旣為定武,非為未試用。且孫堅以初平二年卒,洞口之役在黃初三年,堅卒至此合三十一年,匡時若尚在,本傳不得云卒時年二十餘也。此蓋權別生弟朗,江表傳誤以為匡也。朗之名位見三朝錄及虞喜志林也。}
 子泰,曹氏之甥也,為長水校尉。嘉禾三年,從權圍新城,中流矢死。泰子秀為前將軍、夏口督。秀公室至親,捉兵在外,皓意不能平。建衡二年,皓遣何定將五千人至夏口獵。先是,民間僉言秀當見圖,而定遠獵,秀遂驚,夜將妻子親兵數百人奔晉。晉以秀為驃騎將軍、儀同三司,封會稽公。
 \gezhu{江表傳曰:皓大怒,追改秀姓曰厲。干寶晉紀曰:秀在晉朝,初聞皓降,羣臣畢賀,秀稱疾不與,南向流涕曰:「昔討逆弱冠以一校尉創業,今後主舉江南而棄之,宗廟山陵,於此為墟。悠悠蒼天,此何人哉!」朝廷美之。晉諸公贊曰:吳平,降為伏波將軍,開府如故。永寧中卒,追贈驃騎、開府。子儉,字仲節,給事中。}
 
 
\end{pinyinscope}