\article{顧譚傳}
\begin{pinyinscope}
 
 
 譚字子默,弱冠與諸葛恪等為太子四友,從中庶子轉輔正都尉。
 
 
\gezhu{陸機為譚傳曰:宣太子正位東宮,天子方隆訓導之義,妙簡俊彥,講學左右。時四方之傑畢集,太傅諸葛恪以雄奇蓋衆,而譚以清識絕倫,獨見推重。自太尉范慎、謝景、羊徽之徒,皆以秀稱其名,而悉在譚下。}
 赤烏中,代恪為左節度。
 \gezhu{吳書曰:譚初踐官府,上疏陳事,權輟食稱善,以為過於徐詳。雅性高亮,不脩意氣,或以此望之。然權鑒其能,見待甚隆,數蒙賞賜,特見召請。}
 每省簿書,未嘗下籌,徒屈指心計,盡發疑謬,下吏以此服之。加奉車都尉。薛綜為選曹尚書,固讓譚曰:「譚心精體密,貫道達微,才照人物,德允衆望,誠非愚臣所可越先。」後遂代綜。祖父雍卒數月,拜太常,代雍平尚書事。是時魯王霸有盛寵,與太子和齊衡,譚上疏曰:「臣聞有國有家者,必明嫡庶之端,異尊卑之禮,使高下有差,階級踰邈,如此則骨肉之恩生,覬覦之望絕。昔賈誼陳治安之計,論諸侯之勢,以為勢重,雖親必有逆節之累,勢輕,雖踈必有保全之祚。故淮南親弟,不終饗國,失之於勢重也;吳芮踈臣,傳祚長沙,得之於勢輕也。昔漢文帝使慎夫人與皇后同席,袁盎退夫人之座,帝有怒色,及盎辨上下之儀,陳人彘之戒,帝旣恱懌,夫人亦悟。今臣所陳,非有所偏,誠欲以安太子而便魯王也。」由是霸與譚有隙。時長公主壻衞將軍全琮子寄為霸賔客,寄素傾邪,譚所不納。先是,譚弟承與張休俱北征壽春,全琮時為大都督,與魏將王淩戰於芍陂,軍不利,魏兵乘勝陷沒五營將秦晃軍,休、承奮擊之。遂駐魏師。時琮羣子緒、端亦並為將,因敵旣住,乃進擊之,淩軍用退。時論功行賞,以為駐敵之功大,退敵之功小,休、承並為雜號將軍,緒、端偏裨而已。寄父子益恨,共構會譚。
 \gezhu{吳錄曰:全琮父子屢言芍陂之役為典軍陳恂詐增張休、顧承之功,而休、承與恂通情。休坐繫獄,權為譚故,沈吟不決,欲令譚謝而釋之。及大會,以問譚,譚不謝,而曰:「陛下,讒言其興乎!」江表傳曰:有司奏譚誣罔大不敬,罪應大辟。權以雍故,不致法,皆徙之。}
 譚坐徙交州,幽而發憤,著新言二十篇。其知難篇蓋以自悼傷也。見流二年,年四十二,卒於交阯。
 
 
\end{pinyinscope}