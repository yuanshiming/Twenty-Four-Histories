\article{田豫}
\begin{pinyinscope}
 
 
 田豫字國讓,漁陽雍奴人也。劉備之奔公孫瓚也,豫時年少,自託於備,備甚奇之。備為豫州刺史,豫以母老求歸,備涕泣與別,曰:「恨不與君共成大事也。」
 
 
 
 
 公孫瓚使豫守東州令,瓚將王門叛瓚,為袁紹將萬餘人來攻。衆懼欲降。豫登城謂門曰:「卿為公孫所厚而去,意有所不得已也;今還作賊,乃知卿亂人耳。夫挈瓶之智,守不假器,吾旣受之矣;何不急攻乎?」門慙而退。瓚雖知豫有權謀而不能任也。瓚敗而鮮于輔為國人所推,行太守事,素善豫,以為長史。時雄傑並起,輔莫知所從。豫謂輔曰:「終能定天下者,必曹氏也。宜速歸命,無後禍期。」輔從其計,用受封寵。太祖召豫為丞相軍謀掾,除頴陰、朗陵令,遷弋陽太守,所在有治。
 
 
 
 
 鄢陵侯彰征代郡,以豫為相。軍次易北,虜伏騎擊之,軍人擾亂,莫知所為。豫因地形,回車結圜陣,弓弩持滿於內,疑兵塞其隙。胡不能進,散去。追擊,大破之,遂前平代,皆豫策也。
 
 
 
 
 遷南陽太守。先時,郡人侯音反,衆數千人在山中為群盜,大為郡患。前太守收其黨與五百餘人,表奏皆當死。豫悉見諸繫囚,慰喻,開其自新之路,一時破械遣之。諸囚皆叩頭,願自效,即相告語,群賊一朝解散,郡內清靜。具以狀上,太祖善之。
 
 
 
 
 文帝初,北狄彊盛,侵擾邊塞,乃使豫持節護烏丸校尉,牽招、解儁并護鮮卑。自高柳以東,濊貊以西,鮮卑數十部,比能、彌加、素利割地統御,各有分界;乃共誓要,皆不得以馬與中國市。豫以戎狄為一,非中國之利,乃先搆離之,使自為讎敵,互相攻伐。素利違盟,出馬千匹與官,為比能所攻,求救於豫。豫恐遂相兼并,為害滋深,宜救善討惡,示信衆狄。單將銳卒,深入虜庭,胡人衆多,鈔軍前後,斷截歸路。豫乃進軍,去虜十餘里結屯營,多聚牛馬糞然之,從他道引去。胡見煙火不絕,以為尚在,去,行數十里乃知之。追豫到馬城,圍之十重,豫密嚴,使司馬建旌旗,鳴鼓吹,將步騎從南門出,胡人皆屬目往赴之。豫將精銳自北門出,鼓譟而起,兩頭俱發,出虜不意,虜衆散亂,皆棄弓馬步走,追討二十餘里,僵尸蔽地。又烏丸王骨進桀黠不恭,豫因出塞案行,單將麾下百餘騎入進部。進逆拜,遂使左右斬進,顯其罪惡以令衆。衆皆怖慴不敢動,便以進弟代進。自是胡人破膽,威震沙漠。山賊高艾,衆數千人,寇鈔,為幽、兾害,豫誘使鮮卑素利部斬艾,傳首京都。封豫長樂亭侯。為校尉九年,其御夷狄,恒摧抑兼并,乖散彊猾。凡逋亡姦宄,為胡作計不利官者,豫皆構刺攪離,使凶邪之謀不遂,聚居之類不安。事業未究,而幽州刺史王雄支黨欲令雄領烏丸校尉,毀豫亂邊,為國生事。遂轉豫為汝南太守,加殄夷將軍。
 
 
 
 
 太和末,公孫淵以遼東叛,帝欲征之而難其人,中領軍楊曁舉豫應選。
 
 
\gezhu{臣松之案:曁字休先,熒陽人,事見劉曄傳。曁子肇,晉荊州刺史。山濤啟事稱肇有才能。肇子潭字道元,次歆字公嗣,潭子彧字長文,次經字仲武,皆見潘岳集。}
 乃使豫以本官督青州諸軍,假節,往討之。會吳賊遣使與淵相結,帝以賊衆多,又以渡海,詔豫使罷軍。豫度賊船垂還,歲晚風急,必畏漂浪,東隨無岸,當赴成山。成山無藏船之處,輙便循海,案行地形,及諸山島,徼截險要,列兵屯守。自入成山,登漢武之觀。賊還,果遇惡風,船皆觸山沈沒,波蕩著岸,無所蒙竄,盡虜其衆。初,諸將皆笑於空地待賊,及賊破,競欲與謀,求入海鉤取浪舡。豫懼窮虜死戰,皆不聽。初,豫以太守督青州,青州刺史程喜內懷不服,軍事之際,多相違錯。喜知帝寶愛明珠,乃密上:「豫雖有戰功而禁令寬弛,所得器仗珠金甚多,放散皆不納官。」由是功不見列。
 
 
 
 
 後孫權號十萬衆攻新城,征東將軍滿寵欲率諸軍救之。豫曰:「賊悉衆大舉,非徒投射小利,欲質新城以致大軍耳。宜聽使攻城,挫其銳氣,不當與爭鋒也。城不可拔,衆必罷怠;罷怠然後擊之,可大克也。若賊見計,必不攻城,勢將自走。若便進兵,適入其計。又大軍相向,當使難知,不當使自畫也。」豫輙上狀,天子從之。會賊遁走。後吳復來寇,豫往拒之,賊即退。諸軍夜驚,云:「賊復來!」豫卧不起,令衆「敢動者斬」。有頃,竟無賊。
 
 
景初末,增邑三百,并前五百戶。正始初,遷使持節護匈奴中郎將,加振威將軍,領并州刺史。外胡聞其威名,相率來獻。州界寧肅,百姓懷之。徵為衞尉。屢乞遜位,太傅司馬宣王以為豫克壯,書喻未聽。豫書荅曰:「年過七十而以居位,譬猶鍾鳴漏盡而夜行不休,是罪人也。」遂固稱疾篤。拜太中大夫,食卿祿。年八十二薨。子彭祖嗣。
 \gezhu{魏畧曰:豫罷官歸,居魏縣。會汝南遣健步詣征北,感豫宿恩,過拜之。豫為殺雞炊黍,送詣至陌頭,謂之曰:「罷老,苦汝來過。無能有益,若何?」健步愍其貧羸,流涕而去,還為故吏民說之。汝南為具資數千匹,遣人餉豫,豫一不受。會病,立戒其妻子曰:「葬我必於西門豹邊。」妻之難之,言:「西門豹古之神人,那可葬於其邊乎?」豫言:「豹所履行與我敵等耳,使死而有靈,必與我善。」妻子從之。汝南聞其死也,悲之,旣為畫像,又就為立碑銘。}
 
 
豫清儉約素,賞賜皆散之將士。每胡、狄私遺,悉簿藏官,不入家;家常貧匱。雖殊類,咸高豫節。
 \gezhu{魏畧曰:鮮卑素利等數來客見,多以牛馬遺豫;豫轉送官。胡以為前所與豫物顯露,不如持金。乃密懷金三十斤,謂豫曰:「願避左右,我欲有所道。」豫從之,胡因跪曰:「我見公貧,故前後遺公牛馬,公輒送官,今密以此上公,可以為家資。」豫張袖受之,荅其厚意。胡去之後,皆悉付外,具以狀聞。於是詔褒之曰:「昔魏絳開懷以納戎賂,今卿舉袖以受狄金,朕甚嘉焉。」乃即賜絹五百匹。豫得賜,分以其半藏小府,後胡復來,以半與之。}
 嘉平六年,下詔襃揚,賜其家錢糓。語在徐邈傳。
 
 
\end{pinyinscope}