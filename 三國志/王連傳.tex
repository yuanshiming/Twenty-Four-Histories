\article{王連傳}
\begin{pinyinscope}
 
 
 王連字文儀,南陽人也。劉璋時入蜀,為梓潼令。先主起事葭萌,進軍來南,連閉城不降,先主義之,不彊偪也。及成都旣平,以連為什邡令,轉在廣都,所居有績。遷司鹽校尉,較鹽鐵之利,利入甚多,有裨國用,於是簡取良才以為官屬,若呂乂、杜祺、劉幹等,終皆至大官,自連所拔也。遷蜀郡太守、興業將軍,領鹽府如故。建興元年,拜屯騎校尉,領丞相長史,封平陽亭侯。時南方諸郡不賔,諸葛亮將自征之,連諫以為「此不毛之地,疫癘之鄉,不宜以一國之望,冒險而行」。亮慮諸將才不及己,意欲必往,而連言輒懇至,故停留者乆之。會連卒。子山嗣,官至江陽太守。
 
 
\end{pinyinscope}