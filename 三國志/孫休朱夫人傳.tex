\article{孫休朱夫人傳}
\begin{pinyinscope}
 
 
 孫休朱夫人,朱據女,休姊公主所生也。
 
 
\gezhu{臣松之以為休妻其甥,事同漢惠。荀恱譏之已當,故不復廣言。}
 赤烏末,權為休納以為妃。休為琅邪王,隨居丹楊。建興中,孫峻專政,公族皆患之。全尚妻即峻姊,故惟全主祐焉。初,孫和為太子時,全主譖害王夫人,欲廢太子,立魯王,朱主不聽,由是有隙。五鳳中,孫儀謀殺峻,事覺被誅。全主因言朱主與儀同謀,峻枉殺朱主。休懼,遣夫人還建業,執手泣別。旣至,峻遣還休。太平中,孫亮知朱主為全主所害,問朱主死意?全主懼曰:「我實不知,皆據二子熊、損所白。」亮殺熊、損。損妻是峻妹也,孫綝益忌亮,遂廢亮,立休。永安五年,立夫人為皇后。休卒,羣臣尊夫人為皇太后。孫皓即位月餘,貶為景皇后,稱安定宮。甘露元年七月,見逼薨,合葬定陵。
 \gezhu{搜神記曰:孫峻殺朱主,埋於石子岡。歸命即位,將欲改葬之。冢墓相亞,不可識別,而宮人頗識主亡時所著衣服,乃使兩巫各住一處以伺其靈,使察鑒之,不得相近。乆時,二人俱白:見一女人年可三十餘,上著青錦束頭,紫白袷裳,丹綈絲履,從石子岡上半岡,而以手抑膝長太息,小住須臾,進一冢上便住,徘徊良乆,奄然不見。二人之言,不謀而同,於是開冢,衣服如之。}
 
 
\end{pinyinscope}