\article{李典傳}
\begin{pinyinscope}
 
 
 李典字曼成,山陽鉅野人也。典從父乾,有雄氣,合賔客數千家在乘氏。初平中,以衆隨太祖,破黃巾於壽張,又從擊袁術,征徐州。呂布之亂,太祖遣乾還乘氏,慰勞諸縣。布別駕薛蘭、治中李封招乾,欲俱叛,乾不聽,遂殺乾。太祖使乾子整將乾兵,與諸將擊蘭、封。蘭、封破,從平兖州諸縣有功,稍遷青州刺史。整卒,典徙潁陰令,為中郎將,將整軍,
 
 
\gezhu{魏書曰:典少好學,不樂兵事,乃就師讀春秋左氏傳,博觀羣書。太祖善之,故試以治民之政。}
 遷離狐太守。
 
 
 
 
 時太祖與袁紹相拒官渡,典率宗族及部曲輸穀帛供軍。紹破,以典為裨將軍,屯安民。太祖擊譚、尚於黎陽,使典與程昱等以船運軍糧。會尚遣魏郡太守高蕃將兵屯河上,絕水道,太祖勑典、昱:「若船不得過,下從陸道。」典與諸將議曰:「蕃軍少甲而恃水,有懈怠之心,擊之必克。軍不內禦;苟利國家,專之可也,宜亟擊之。」昱亦以為然。遂北渡河,攻蕃,破之,水道得通。劉表使劉備北侵,至葉,太祖遣典從夏侯惇拒之。備一旦燒屯去,惇率諸軍追擊之,典曰:「賊無故退,疑必有伏。南道窄狹,草木深,不可追也。」惇不聽,與于禁追之,典留守。惇等果入賊伏裏,戰不利,典往救,備望見救至,軍散退。從圍鄴,鄴定,與樂進圍高幹於壺關,擊管承於長廣,皆破之。遷捕虜將軍,封都亭侯。
 
 
 
 
 典宗族部曲三千餘家,居乘氏,自請願徙詣魏郡。太祖笑曰:「卿欲慕耿純邪?」典謝曰:「典駑怯功微,而爵寵過厚,誠宜舉宗陳力;加以征伐未息,宜實郊遂之內,以制四方,非慕純也。」遂徙部曲宗族萬三千餘口居鄴。太祖嘉之,遷破虜將軍。與張遼、樂進屯合肥,孫權率衆圍之,遼欲奉教出戰。進、典、遼皆素不睦,遼恐其不從,典慨然曰:「此國家大事,顧君計何如耳,吾不可以私憾而忘公義乎!」乃率衆與遼破走權。增邑百戶,并前三百戶。
 
 
 
 
 典好學問,貴儒雅,不與諸將爭功。敬賢士大夫,恂恂若不及,軍中稱其長者。年三十六薨,子禎嗣。文帝踐阼,追念合肥之功,增禎邑百戶,賜典一子爵關內侯,邑百戶;謚典曰愍侯。
 
 
\end{pinyinscope}