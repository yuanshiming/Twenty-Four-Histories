\article{張燕}
\begin{pinyinscope}
 
 
 張燕,常山真定人也,本姓褚。黃巾起,燕合聚少年為羣盜,在山澤間轉攻,還真定,衆萬餘人。博陵張牛角亦起衆,自號將兵從事,與燕合。燕推牛角為帥,俱攻癭陶。牛角為飛矢所中,被創且死,令衆奉燕,告曰:「必以燕為帥。」牛角死,衆奉燕,故改姓張。燕剽捍捷速過人,故軍中號曰飛燕。其後人衆寖廣,常山、趙郡、中山、上黨、河內諸山谷皆相通,其小帥孫輕、王當等,各以部衆從燕,衆至百萬,號曰黑山。靈帝不能征,河北諸郡被其害。燕遣人至京都乞降,拜燕平難中郎將。
 
 
\gezhu{九州春秋曰:張角之反也,黑山、白波、黃龍、左校、牛角、五鹿、羝根、苦蝤、劉石、平漢、大洪、司隷、緣城、羅市、雷公、浮雲、飛燕、白爵、楊鳳、于毒等各起兵,大者二三萬,小者不減數千。靈帝不能討,乃遣使拜楊鳳為黑山校尉,領諸山賊,得舉孝廉計吏。後遂彌漫,不可復數。典略曰:黑山、黃巾諸帥,本非冠蓋,自相號字,謂騎白馬者為張白騎,謂輕捷者為張飛燕,謂聲大者為張雷公,其饒鬚者則自稱于羝根,其眼大者自稱李大目。張璠漢紀云:又有左校、郭大賢、左髭丈八三部也。}
 是後,董卓遷天子於長安,天下兵數起,燕遂以其衆與豪傑相結。袁紹與公孫瓚爭兾州,燕遣將杜長等助瓚,與紹戰,為紹所敗,人衆稍散,太祖將定兾州,燕遣使求佐王師,拜平北將軍;率衆詣鄴,封安國亭侯,邑五百戶。燕薨,子方嗣。方薨,子融嗣。
 \gezhu{陸機晉惠帝起居注曰:門下通事令史張林,飛燕之曾孫。林與趙王倫為亂,未及周年,位至尚書令、衞將軍,封郡公。尋為倫所殺。}
 
 
\end{pinyinscope}