\article{呂蒙傳}
\begin{pinyinscope}
 
 
 呂蒙字子明,汝南富陂人也。少南渡,依姊夫鄧當。當為孫策將,數討山越。蒙年十五六,竊隨當擊賊,當顧見大驚,呵叱不能禁止。歸以告蒙母,母恚欲罰之,蒙曰:「貧賤難可居,脫誤有功,富貴可致。且不探虎穴,安得虎子?」母哀而舍之。時當職吏以蒙年小輕之,曰:「彼豎子何能為?此欲以肉餧虎耳。」他日與蒙會,又蚩辱之。蒙大怒,引刀殺吏,出走,逃邑子鄭長家。出因校尉袁雄自首,承間為言,策召見奇之,引置左右。
 
 
 
 
 數歲,鄧當死,張昭薦蒙代當,拜別部司馬。權統事,料諸小將兵少而用薄者,欲并合之。蒙陰賒貰,為兵作絳衣行縢,及簡日,陳列赫然,兵人練習,權見之大恱,增其兵。從討丹楊,所向有功,拜平北都尉,領廣德長。
 
 
 
 
 從征黃祖,祖令都督陳就逆以水軍出戰。蒙勒前鋒,親梟就首,將士乘勝,進攻其城。祖聞就死,委城走,兵追禽之。權曰:「事之克,由陳就先獲也。」以蒙為橫野中郎將,賜錢千萬。
 
 
 
 
 是歲,又與周瑜、程普等西破曹公於烏林,圍曹仁於南郡。益州將襲肅舉軍來附,瑜表以肅兵益蒙,蒙盛稱肅有膽用,且慕化遠來,於義宜益不宜奪也。權善其言,還肅兵。瑜使甘寧前據夷陵,曹仁分衆攻寧,寧困急,使使請救。諸將以兵少不足分,蒙謂瑜、普曰:「留淩公績,蒙與君行,解圍釋急,勢亦不乆,蒙保公績能十日守也。」又說瑜分遣三百人柴斷險道,賊走可得其馬。瑜從之。軍到夷陵,即日交戰,所殺過半。敵夜遁去,行遇柴道,騎皆舍馬步走。兵追蹙擊,獲馬三百匹,方船載還。於是將士形勢自倍,乃渡江立屯,與相攻擊,曹仁退走,遂據南郡,撫定荊州。還,拜偏將軍,領尋陽令。
 
 
 
 
 魯肅代周瑜,當之陸口,過蒙屯下。肅意尚輕蒙,或說肅曰:「呂將軍功名日顯,不可以故意待也,君宜顧之。」遂往詣蒙。酒酣,蒙問肅曰:「君受重任,與關羽為鄰,將何計略,以備不虞?」肅造次應曰:「臨時施宜。」蒙曰:「今東西雖為一家,而關羽實熊虎也,計安可不豫定?」因為肅畫五策。肅於是越席就之,拊其背曰:「呂子明,吾不知卿才略所及乃至於此也。」遂拜蒙母,結友而別。
 
 
\gezhu{江表傳曰:初,權謂蒙及蔣欽曰:「卿今並當塗掌事,宜學問以自開益。」蒙曰:「在軍中常苦多務,恐不容復讀書。」權曰:「孤豈欲卿治經為博士邪?但當令涉獵見往事耳。卿言多務孰若孤,孤少時歷詩、書、禮記、左傳、國語,惟不讀易。至統事以來,省三史、諸家兵書,自以為大有所益。如卿二人,意性朗悟,學必得之,寧當不為乎?宜急讀孫子、六韜、左傳、國語及三史。孔子言『終日不食,終夜不寢以思,無益,不如學也』。光武當兵馬之務,手不釋卷。孟德亦自謂老而好學。卿何獨不自勉勗邪?」蒙始就學,篤志不倦,其所覽見,舊儒不勝。後魯肅上代周瑜,過蒙言議,常欲受屈。肅拊蒙背曰:「吾謂大弟但有武略耳,至於今者,學識英博,非復吳下阿蒙。」蒙曰:「士別三日,即更刮目相待。大兄今論,何一稱穰侯乎。兄今代公瑾,旣難為繼,且與關羽為鄰。斯人長而好學,讀左傳略皆上口,梗亮有雄氣,然性頗自負,好陵人。今與為對,當有單複以鄉待之。」。密為肅陳三策,肅敬受之,祕而不宣。權常歎曰:「人長而進益,如呂蒙、蔣欽,蓋不可及也。富貴榮顯,更能折節好學,耽恱書傳,輕財尚義,所行可迹,並作國士,不亦休乎!」}
 
 
 
 
 時蒙與成當、宋定、徐顧屯次比近,三將死,子弟幼弱,權悉以兵并蒙。蒙固辭,陳啟顧等皆勤勞國事,子弟雖小,不可廢也。書三上,權乃聽。蒙於是又為擇師,使輔導之,其操心率如此。
 
 
魏使廬江謝奇為蘄春典農,屯皖田鄉,數為邊寇。蒙使人誘之,不從,則伺隙襲擊,奇遂縮退,其部伍孫子才、宋豪等,皆攜負老弱,詣蒙降。後從權拒曹公於濡須,數進奇計,又勸權夾水口立塢,所以備御甚精,
 \gezhu{吳錄曰:權欲作塢,諸將皆曰:「上岸擊賊,洗足入船,何用塢為?」呂蒙曰:「兵有利鈍,戰無百勝,如有邂逅,敵步騎蹙人,不暇及水,其得入船乎?」權曰:「善。」遂作之。}
 曹公不能下而退。
 
 
曹公遣朱光為廬江太守,屯皖,大開稻田,又令間人招誘鄱陽賊帥,使作內應。蒙曰:「皖田肥美,若一收孰,彼衆必增,如是數歲,操態見矣,宜早除之。」乃具陳其狀。於是權親征皖,引見諸將,問以計策。
 \gezhu{吳書曰:諸將皆勸作土山,添攻具,蒙趨進曰:「治攻具及土山,必歷日乃成,城備旣脩,外救必至,不可圖也。且乘雨水以入,若留經日,水必向盡,還道艱難,蒙竊危之。今觀此城,不能甚固,以三軍銳氣,四面並攻,不移時可拔,及水以歸,全勝之道也。」權從之。}
 蒙乃薦甘寧為升城督,督攻在前,蒙以精銳繼之。侵晨進攻,蒙手執枹鼓,士卒皆騰踊自升,食時破之。旣而張遼至夾石,聞城已拔,乃退。權嘉其功,即拜廬江太守,所得人馬皆分與之,別賜尋陽屯田六百戶,官屬三十人。蒙還尋陽,未期而廬陵賊起,諸將討擊不能禽,權曰:「鷙鳥累百,不如一鶚。」復令蒙討之。蒙至,誅其首惡,餘皆釋放,復為平民。
 
 
 
 
 是時劉備令關羽鎮守,專有荊土,權命蒙西取長沙、零、桂三郡。蒙移書二郡,望風歸服,惟零陵太守郝普城守不降。而備自蜀親至公安,遣羽爭三郡。權時住陸口,使魯肅將萬人屯益陽拒羽,而飛書召蒙,使捨零陵,急還助肅。初,蒙旣定長沙,當之零陵,過酃,載南陽鄧玄之,玄之者郝普之舊也,欲令誘普。及被書當還,蒙祕之,夜召諸將,授以方略,晨當攻城,顧謂玄之曰:「郝子太聞世間有忠義事,亦欲為之,而不知時也。左將軍在漢中,為夏侯淵所圍。關羽在南郡,今至尊身自臨之。近者破樊本屯,救酃,逆為孫規所破。此皆目前之事,君所親見也。彼方首尾倒縣,救死不給,豈有餘力復營此哉?今吾士卒精銳,人思致命,至尊遣兵,相繼於道。今予以旦夕之命,待不可望之救,猶牛蹄中魚,兾賴江漢,其不可恃亦明矣。若子太必能一士卒之心,保孤城之守,尚能稽延旦夕,以待所歸者,可也。今吾計力度慮,而以攻此,曾不移日,而城必破,城破之後,身死何益於事,而令百歲老母戴白受誅,豈不痛哉?度此家不得外問,謂援可恃,故至於此耳。君可見之,為陳禍福。」玄之見普,具宣蒙意,普懼而聽之。玄之先出報蒙,普尋後當至。蒙豫勑四將,各選百人,普出,便入守城門。須臾普出,蒙迎執其手,與俱下船。語畢,出書示之,因拊手大笑,普見書,知備在公安,而羽在益陽,慙恨入地。蒙留孫河委以後事。即日引軍赴益陽。劉備請盟,權乃歸普等,割湘水,以零陵還之。以尋陽、陽新為蒙奉邑。
 
 
 
 
 師還,遂征合肥,旣徹兵,為張遼等所襲,蒙與淩統以死扞衞。後曹公又大出濡須,權以蒙為督,據前所立塢,置彊弩萬張於其上,以拒曹公。曹公前鋒屯未就,蒙攻破之,曹公引退。拜蒙左護軍、虎威將軍。
 
 
 
 
 魯肅卒,蒙西屯陸口,肅軍人馬萬餘盡以屬蒙。又拜漢昌太守,食下雋、劉陽、漢昌、州陵。與關羽分土接境,知羽驍雄,有并兼心,且居國上流,其勢難乆。初,魯肅等以為曹公尚存,禍難始搆,宜相輔協,與之同仇,不可失也,蒙乃密陳計策曰:「令征虜守南郡,潘璋住白帝,蔣欽將游兵萬人,循江上下,應敵所在,蒙為國家前據襄陽,如此,何憂於操,何賴於羽?且羽君臣,矜其詐力,所在反覆,不可以腹心待也。今羽所以未便東向者,以至尊聖明,蒙等尚存也。今不於彊壯時圖之,一旦僵仆,欲復陳力,其可得邪?」權深納其策,又聊復與論取徐州意,蒙對曰:「今操遠在河北,新破諸袁,撫集幽、兾,未暇東顧。徐土守兵,聞不足言,往自可克。然地勢陸通,驍騎所騁,至尊今日得徐州,操後旬必來爭,雖以七八萬人守之,猶當懷憂。不如取羽,全據長江,形勢益張。」權尤以此言為當。及蒙代肅,初至陸口,外倍脩恩厚,與羽結好。
 
 
後羽討樊,留兵將備公安、南郡。蒙上疏曰:「羽討樊而多留備兵,必恐蒙圖其後故也。蒙常有病,乞分士衆還建業,以治疾為名。羽聞之,必撤備兵,盡赴襄陽。大軍浮江,晝夜馳上,襲其空虛,則南郡可下,而羽可禽也。」遂稱病篤,權乃露檄召蒙還,陰與圖計。羽果信之,稍撤兵以赴樊。魏使于禁救樊,羽盡禽禁等,人馬數萬,託以糧乏,擅取湘關米。權聞之,遂行,先遣蒙在前。蒙至尋陽,盡伏其精兵𦩷𦪇中,使白衣搖櫓,作商賈人服,晝夜兼行,至羽所置江邊屯候,盡收縛之,是故羽不聞知。遂到南郡,士仁、麋芳皆降。
 \gezhu{吳書曰:將軍士仁在公安拒守,蒙令虞翻說之。翻至城門,謂守者曰:「吾欲與汝將軍語。」仁不肯相見。乃為書曰:「明者防禍於未萌,智者圖患於將來,知得知失,可與為人,知存知亡,足別吉凶。大軍之行,斥候不及施,烽火不及舉,此非天命,必有內應。將軍不先見時,時至又不應之,獨守縈帶之城而不降,死戰則毀宗滅祀,為天下譏笑。呂虎威欲徑到南郡,斷絕陸道,生路一塞,案其地形,將軍為在箕舌上耳,奔走不得免,降則失義,竊為將軍不安,幸孰思焉。」仁得書,流涕而降。翻謂蒙曰:「此譎兵也,當將仁行,留兵備城。」遂將仁至南郡。南郡太守麋芳城守,蒙以仁示之,遂降。吳錄曰:初,南郡城中失火,頗焚燒軍器。羽以責芳,芳內畏懼,權聞而誘之,芳潛相和。及蒙攻之,乃以牛酒出降。}
 蒙入據城,盡得羽及將士家屬,皆撫慰,約令軍中不得干歷人家,有所求取。蒙麾下士,是汝南人,取民家一笠,以覆官鎧,官鎧雖公,蒙猶以為犯軍令,不可以鄉里故而廢法,遂垂涕斬之。於是軍中震慄,道不拾遺。蒙旦暮使親近存恤耆老,問所不足,疾病者給醫藥,饑寒者賜衣糧。羽府藏財寶,皆封閉以待權至。羽還,在道路,數使人與蒙相聞,蒙輒厚遇其使,周游城中,家家致問,或手書示信。羽人還,私相參訊,咸知家門無恙,見待過於平時,故羽吏士無鬬心。會權尋至,羽自知孤窮,乃走麥城,西至漳鄉,衆皆委羽而降。權使朱然、潘璋斷其徑路,即父子俱獲,荊州遂定。
 
 
以蒙為南郡太守,封孱陵侯,
 \gezhu{江表傳曰:權於公安大會,呂蒙以疾辭,權笑曰:「禽羽之功,子明謀也,今大功已捷,慶賞未行,豈邑邑邪?」乃增給步騎鼓吹,勑選虎威將軍官屬,并南郡、廬江二郡威儀。拜畢還營,兵馬導從,前後鼓吹,光耀于路。}
 賜錢一億,黃金五百斤。蒙固辭金錢,權不許。封爵未下,會蒙疾發,權時在公安,迎置內殿,所以治護者萬方,募封內有能愈蒙疾者,賜千金。時有鍼加,權為之慘慼,欲數見其顏色,又恐勞動,常穿壁瞻之,見小能下食則喜,顧左右言笑,不然則咄唶,夜不能寐。病中瘳,為下赦令,羣臣畢賀。後更增篤,權自臨視,命道士於星辰下為之請命。年四十二,遂卒於內殿。時權哀痛甚,為之降損。蒙未死時,所得金寶諸賜盡付府藏,勑主者命絕之日皆上還,喪事務約。權聞之,益以悲感。
 
 
 
 
 蒙少不脩書傳,每陳大事,常口占為牋疏。常以部曲事為江夏太守蔡遺所白,蒙無恨意。及豫章太守顧邵卒,權問所用,蒙因薦遺奉職佳吏,權笑曰:「君欲為祁奚耶?」於是用之。甘寧麤暴好殺,旣常失蒙意,又時違權令,權怒之,蒙輒陳請:「天下未定,鬬將如寧難得,宜容忍之。」權遂厚寧,卒得其用。
 
 
 
 
 蒙子霸襲爵,與守冢三百家,復田五十頃。霸卒,兄琮襲侯。琮卒,弟睦嗣。
 
 
 
 
 孫權與陸遜論周瑜、魯肅及蒙曰:「公瑾雄烈,膽略兼人,遂破孟德,開拓荊州,邈焉難繼,君今繼之。公瑾昔要子敬來東,致達於孤,孤與宴語,便及大略帝王之業,此一快也。後孟德因獲劉琮之勢,張言方率數十萬衆水步俱下。孤普請諸將,咨問所宜,無適先對,至子布、文表,俱言宜遣使脩檄迎之,子敬即駮言不可,勸孤急呼公瑾,付任以衆,逆而擊之,此二快也。且其決計策,意出張蘇遠矣;後雖勸吾借玄德地,是其一短,不足以損其二長也。周公不求備於一人,故孤忘其短而貴其長,常以比方鄧禹也。又子明少時,孤謂不辭劇易,果敢有膽而已;及身長大,學問開益,籌略奇至,可以次於公瑾,但言議英發不及之耳。圖取關羽,勝於子敬。子敬荅孤書云:『帝王之起,皆有驅除,羽不足忌。』此子敬內不能辨,外為大言耳,孤亦恕之,不苟責也。然其作軍,屯營不失,令行禁止,部界無廢負,路無拾遺,其法亦美也。」
 
 
 
 
 評曰:曹公乘漢相之資,挾天子而掃羣桀,新盪荊城,仗威東夏,于時議者莫不疑貳。周瑜、魯肅建獨斷之明,出衆人之表,實奇才也。呂蒙勇而有謀斷,識軍計,譎郝普,禽關羽,最其妙者。初雖輕果妄殺,終於克己,有國士之量,豈徒武將而已乎!孫權之論,優劣允當,故載錄焉。
 
 
\end{pinyinscope}