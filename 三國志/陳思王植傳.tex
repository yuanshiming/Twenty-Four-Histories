\article{陳思王植傳}

\begin{pinyinscope}

陳思王植字子建。年十歲餘,誦讀詩、論及辭賦數十萬言,善屬文。太祖嘗視其文,謂植曰:「汝倩人邪?」植跪曰:「言出為論,下筆成章,顧當面試,柰何倩人?」時鄴銅爵臺新城,太祖悉將諸子登臺,使各為賦。植援筆立成,可觀,太祖甚異之。
\gezhu{陰澹魏紀載植賦曰「從明后而嬉游兮,登層臺以娛情。見太府之廣開兮,觀聖德之所營。建高門之嵯峨兮,浮雙闕乎太清。立中天之華觀兮,連飛閣乎西城。臨漳水之長流兮,望園果之滋榮。仰春風之和穆兮,聽百鳥之悲鳴。天雲垣其旣立兮,家願得而獲逞。揚仁化於宇內兮,盡肅恭於上京。惟桓文之為盛兮,豈足方乎聖明!休矣美矣!惠澤遠揚。翼佐我皇家兮,寧彼四方。同天地之規量兮,齊日月之暉光。永貴尊而無極兮,等年壽於東王」云云。太祖深異之。}
性簡易,不治威儀。輿馬服飾,不尚華麗。每進見難問,應聲而對,特見寵愛。建安十六年,封平原侯。十九年,徙封臨菑侯。太祖征孫權,使植留守鄴,戒之曰:「吾昔為頓丘令,年二十三。思此時所行,無悔於今。今汝年亦二十三矣,可不勉與!」植旣以才見異,而丁儀、丁廙、楊脩等為之羽翼。太祖狐疑,幾為太子者數矣。而植任性而行,不自彫勵,飲酒不節。文帝御之以術,矯情自飾,宮人左右並為之說,故遂定為嗣。二十二年,增植邑五千,并前萬戶。

植嘗乘車行馳道中,開司馬門出。太祖大怒,公車令坐死。由是重諸侯科禁,而植寵日衰。
\gezhu{魏武故事載令曰:「始者謂子建,兒中最可定大事。」又令曰:「自臨菑侯植私出,開司馬門至金門,令吾異目視此兒矣。」又令曰:「諸侯長史及帳下吏,知吾出輒將諸侯行意否?從子建私開司馬門來,吾都不復信諸侯也。恐吾適出,便復私出,故攝將行。不可恒使,吾爾誰為心腹也!」}
太祖旣慮終始之變,以楊脩頗有才策,而又袁氏之甥也,於是以罪誅脩。植益內不自安。
\gezhu{典略曰:楊脩字德祖,太尉彪子也。謙恭才博。建安中,舉孝廉,除郎中,丞相請署倉曹屬主簿。是時,軍國多事,脩總知外內,事皆稱意。自魏太子已下,並爭與交好。又是時臨菑侯植以才捷愛幸,來意投脩,數與脩書,書曰:「數日不見,思子為勞;想同之也。僕少好辭賦,迄至于今二十有五年矣。然今世作者,可略而言也。昔仲宣獨步於漢南,孔璋鷹揚於河朔,偉長擅名於青土,公幹振藻於海隅,德璉發迹於大魏,足下高視於上京。當此之時,人人自謂握靈蛇之珠,家家自謂抱荊山之玉也。吾王於是設天網以該之,頓八紘以掩之,今盡集茲國矣。然此數子,猶不能飛翰絕迹,一舉千里也。以孔璋之才,不閑辭賦,而多自謂與司馬長卿同風,譬畫虎不成還為狗者也。前為書啁之,反作論盛道僕贊其文。夫鍾期不失聽,于今稱之。吾亦不敢妄歎者,畏後之嗤余也。世人著述,不能無病。僕常好人譏彈其文;有不善者,應時改定。昔丁敬禮嘗作小文,使僕潤飾之,僕自以才不能過若人,辭不為也。敬禮云:『卿何所疑難乎!文之佳麗,吾自得之。後世誰相知定吾文者邪?』吾常歎此達言,以為美談。昔尼父之文辭,與人通流;至於制春秋,游、夏之徒不能錯一字。過此而言不病者,吾未之見也。蓋有南威之容,乃可以論於淑媛;有龍淵之利,乃可以議於割斷。劉季緒才不逮於作者,而好詆呵文章,掎摭利病。昔田巴毀五帝,罪三王,呰五伯於稷下,一旦而服千人,魯連一說,使終身杜口。劉生之辯未若田氏,今之仲連求之不難,可無歎息乎!人各有所好尚。蘭茞蓀蕙之芳,衆人之所好,而海畔有逐臭之夫;咸池、六英之發,衆人所樂,而墨翟有非之之論:豈可同哉!今往僕少小所著辭賦一通相與。夫街談巷說,必有可采,擊轅之歌,有應風雅,匹夫之思,未易輕棄也。辭賦小道,固未足以揄揚大義,彰示來世也。昔揚子雲,先朝執戟之臣耳,猶稱『壯夫不為』也;吾雖薄德,位為藩侯,猶庶幾勠力上國,流惠下民,建永世之業,流金石之功,豈徒以翰墨為勳績,辭頌為君子哉?若吾志不果,吾道不行,亦將採史官之實錄,辨時俗之得失,定仁義之衷,成一家之言,雖未能藏之名山,將以傳之同好,此要之白首,豈可以今日論乎!其言之不怍,恃惠子之知我也。明早相迎,書不盡懷。」脩荅曰:「不侍數日,若彌年載,豈獨愛顧之隆,使係仰之情深邪!損辱來命,蔚矣其文。誦讀反覆,雖諷雅、頌,不復過也。若仲宣之擅江表,陳氏之跨兾域,徐、劉之顯青、豫,應生之發魏國,斯皆然矣。至如脩者,聽采風聲,仰德不暇,自周章於省覽,何惶駭於高視哉?伏惟君侯,少長貴盛,體旦、發之質,有聖善之教。遠近觀者,徒謂能宣昭懿德,光贊大業而已,不謂復能兼覽傳記,留思文章。今乃含王超陳,度越數子;觀者駭視而拭目,聽者傾首而聳耳;非夫體通性達,受之自然,其誰能至於此乎?又嘗親見執事握牘持筆,有所造作,若成誦在心,借書於手,曾不斯須少留思慮。仲尼日月,無得踰焉。脩之仰望,殆如此矣。是以對鶡而辭,作暑賦彌日而不獻,見西施之容,歸憎其貌者也。伏想執事不知其然,猥受顧賜,教使刊定。春秋之成,莫能損益。呂氏、淮南,字直千金;然而弟子鉗口,市人拱手者,聖賢卓犖,固所以殊絕凡庸也。今之賦頌,古詩之流,不更孔公,風雅無別耳。脩家子雲,老不曉事,彊著一書,悔其少作。若此,仲山、周旦之徒,則皆有愆乎!君侯忘聖賢之顯迹,述鄙宗之過言,竊以為未之思也。若乃不忘經國之大美,流千載之英聲,銘功景鍾,書名竹帛,此自雅量素所蓄也,豈與文章相妨害哉?輒受所惠,竊備矇瞍誦歌而已。敢忘惠施,以忝莊氏!季緒瑣瑣,何足以云。」其相往來,如此甚數。植後以驕縱見疏,而植故連綴脩不止,脩亦不敢自絕。至二十四年秋,公以脩前後漏泄言教,交關諸侯,乃收殺之。脩臨死,謂故人曰:「我固自以死之晚也。」其意以為坐曹植也。脩死後百餘日而太祖薨,太子立,遂有天下。初,脩以所得王髦劒奉太子,太子常服之。及即尊位,在洛陽,從容出宮,追思脩之過薄也,撫其劒,駐車顧左右曰:「此楊德祖昔所說王髦劒也。髦今焉在?」及召見之,賜髦穀帛。}
\gezhu{摯虞文章志曰:劉季緒名脩,劉表子。官至東安太守。著詩、賦、頌六篇。}
\gezhu{臣松之案呂氏春秋曰:「人有臭者,其兄弟妻子皆莫能與居,其人自若而居海上。海上人有恱其臭者,晝夜隨之而不能去。」此植所云「逐臭之夫」也。田巴事出魯連子,亦見皇覽,文多故不載。}
\gezhu{世語曰:脩年二十五,以名公子有才能,為太祖所器,與丁儀兄弟,皆欲以植為嗣。太子患之,以車載廢簏,內朝歌長吳質與謀。脩以白太祖,未及推驗。太子懼,告質,質曰:「何患?明日復以簏受絹車內以惑之,脩必復重白,重白必推,而無驗,則彼受罪矣。」世子從之,脩果白,而無人,太祖由是疑焉。脩與賈逵、王淩並為主簿,而為植所友。每當就植,慮事有闕,忖度太祖意,豫作荅教十餘條,勑門下,教出以次荅。教裁出,荅已入,太祖怪其捷,推問始泄。太祖遣太子及植各出鄴城一門,密勑門不得出,以觀其所為。太子至門,不得出而還。脩先戒植:「若門不出侯,侯受王命,可斬守者。」植從之。故脩遂以交搆賜死。脩子嚻,嚻子準,皆知名於晉世。嚻,泰始初為典軍將軍,受心膂之任,早卒。準字始丘,惠帝末為兾州刺史。}
\gezhu{荀綽兾州記曰:準見王綱不振,遂縱酒,不以官事為意,逍遙卒歲而已。成都王知準不治,猶以其為名士,惜而不責,召以為軍謀祭酒。府散停家,關東諸侯議欲以準補三事,以示懷賢尚德之舉。事未施行而卒。準子嶠字國彥,髦字士彥,並為後出之俊。準與裴頠、樂廣善,遣往見之。頠性弘方,愛嶠之有高韻,謂準曰:「嶠當及卿,然髦小減也。」廣性清淳,愛髦之有神檢,謂準曰:「嶠自及卿,然髦尤精出。」準歎曰:「我二兒之優劣,乃裴、樂之優劣也。」評者以為嶠雖有高韻,而神檢不逮,廣言為得。傅暢云:「嶠似準而踈。」嶠弟俊,字惠彥,最清出。嶠、髦皆為二千石。俊,太傅掾。}
二十四年,曹仁為關羽所圍。太祖以植為南中郎將,行征虜將軍。欲遣救仁,呼有所勑戒。植醉不能受命,於是悔而罷之。
\gezhu{魏氏春秋曰:植將行,太子飲焉,偪而醉之。王召植,植不能受王命,故王怒也。}


文帝即王位,誅丁儀、丁廙并其男口。
\gezhu{魏略曰:丁儀字正禮,沛郡人也。父沖,宿與太祖親善,時隨乘輿。見國家未定,乃與太祖書曰:「足下平生常喟然有匡佐之志,今其時矣。」是時張楊適還河內,太祖得其書,乃引軍迎天子東詣許,以沖為司隷校尉。後數來過諸將飲,酒美不能止,醉爛腸死。太祖以沖前見開導,常德之。聞儀為令士,雖未見,欲以愛女妻之,以問五官將。五官將曰:「女人觀貌,而正禮目不便,誠恐愛女未必恱也。以為不如與伏波子楙。」太祖從之。尋辟儀為掾,到與論議,嘉其才朗,曰:「丁掾,好士也,即使其兩目盲,尚當與女,何況但眇?是吾兒誤我。」時儀亦恨不得尚公主,而與臨菑侯親善,數稱其奇才。太祖旣有意欲立植,而儀又共贊之。及太子立,欲治儀罪,轉儀為右刺姦掾,欲儀自裁而儀不能。乃對中領軍夏侯尚叩頭求哀,尚為涕泣而不能救。後遂因職事收付獄,殺之。}
\gezhu{廙字敬禮,儀之弟也。文士傳曰:廙少有才姿,博學洽聞。初辟公府,建安中為黃門侍郎。廙嘗從容謂太祖曰:「臨菑侯天性仁孝,發於自然,而聦明智達,其殆庶幾。至於博學淵識,文章絕倫。當今天下之賢才君子,不問少長,皆願從其游而為之死,實天下所以鍾福於大魏,而永授無窮之祚也。」欲以勸動太祖。太祖荅曰:「植,吾愛之,安能若卿言!吾欲立之為嗣,何如?」廙曰:「此國家之所以興衰,天下之所以存亡,非愚劣瑣賤者所敢與及。廙聞知臣莫若於君,知子莫若於父。至於君不論明闇,父不問賢愚,而能常知其臣子者何?蓋由相知非一事一物,相盡非一旦一夕。況明公加之以聖哲,習之以人子。今發明達之命,吐永安之言,可謂上應天命,下合人心,得之於須臾,垂之於萬世者也。廙不避斧鉞之誅,敢不盡言!」太祖深納之。}
植與諸侯並就國。黃初二年,監國謁者灌均希指,奏「植醉酒悖慢,劫脅使者」。有司請治罪,帝以太后故,貶爵安鄉侯。
\gezhu{魏書載詔曰:「植,朕之同母弟。朕於天下無所不容,而況植乎?骨肉之親,捨而不誅,其改封植。」}
其年改封鄄城侯。三年,立為鄄城王,邑二千五百戶。四年,徙封雍丘王。其年,朝京都。上疏曰:


臣自抱釁歸藩,刻肌刻骨,追思罪戾,晝分而食,夜分而寢。誠以天罔不可重離,聖恩難可再恃。竊感相鼠之篇,無禮遄死之義,形影相弔,五情愧赧。以罪棄生,則違古賢「夕改」之勸,忍活苟全,則犯詩人「胡顏」之譏。伏惟陛下德象天地,恩隆父母,施暢春風,澤如時雨。是以不別荊棘者,慶雲之惠也;七子均養者,尸鳩之仁也;舍罪責功者,明君之舉也;矜愚愛能者,慈父之恩也:是以愚臣徘徊於恩澤而不能自棄者也。


前奉詔書,臣等絕朝,心離志絕,自分黃耇無復執珪之望。不圖聖詔猥垂齒召,至止之日,馳心輦轂。僻處西館,未奉闕廷,踊躍之懷,瞻望反仄。謹拜表獻詩二篇,其辭曰:「於穆顯考,時惟武皇,受命于天,寧濟四方。朱旗所拂,九土披攘,玄化滂流,荒服來王。超商越周,與唐比蹤。篤生我皇,弈世載聦,武則肅烈,文則時雍,受禪炎漢,臨君萬邦。萬邦旣化,率由舊則;廣命懿親,以藩王國。帝曰爾侯,君茲青土,奄有海濵,方周于魯,車服有輝,旗章有叙,濟濟儁乂,我弼我輔。伊予小子,恃寵驕盈,舉挂時網,動亂國經。作藩作屏,先軌是墮,傲我皇使,犯我朝儀。國有典刑,我削我絀,將寘于理,元兇是率。明明天子,時篤同類,不忍我刑,暴之朝肆,違彼執憲,哀予小子。改封兖邑,于河之濵,股肱弗置,有君無臣,荒淫之闕,誰弼予身?煢煢僕夫,于彼兾方,嗟予小子,乃罹斯殃。赫赫天子,恩不遺物,冠我玄冕,要我朱紱。朱紱光大,使我榮華,剖符授玉,王爵是加。仰齒金璽,俯執聖策,皇恩過隆,祗承怵惕。咨我小子,頑凶是嬰,逝慙陵墓,存愧闕廷。匪敢慠德,寔恩是恃,威靈改加,足以沒齒。昊天罔極,性命不圖,常懼顛沛,抱罪黃壚。願蒙矢石,建旗東嶽,庶立豪氂,微功自贖。危軀授命,知足免戾,甘赴江、湘,奮戈吳、越。天啟其衷,得會京畿,遲奉聖顏,如渴如饑。心之云慕,愴矣其悲,天高聽卑,皇肯照微!」又曰:「肅承明詔,應會皇都,星陳夙駕,秣馬脂車。命彼掌徒,肅我征旅,朝發鸞臺,夕宿蘭渚。芒芒原隰,祁祁士女,經彼公田,樂我稷黍。爰有樛木,重陰匪息;雖有糇糧,饑不遑食。望城不過,面邑匪游,僕夫警策,平路是由。玄駟藹藹,揚鑣㵱沫;流風翼衡,輕雲承蓋。涉澗之濵,緣山之隈,遵彼河滸,黃阪是階。西濟關谷,或降或升;騑驂倦路,再寢再興。將朝聖皇,匪敢晏寧;弭節長騖,指日遄征。前驅舉燧,後乘抗旌;輪不輟運,鸞無廢聲。爰曁帝室,稅此西墉;嘉詔未賜,朝覲莫從。仰瞻城閾,俯惟闕廷;長懷永慕,憂心如酲。」


帝嘉其辭義,優詔荅勉之。
\gezhu{魏略曰:初植未到關,自念有過,宜當謝帝。乃留其從官著關東,單將兩三人微行,入見清河長公主,欲因主謝。而關吏以聞,帝使人逆之,不得見。太后以為自殺也,對帝泣。會植科頭負鈇鑕,徒跣詣闕下,帝及太后乃喜。及見之,帝猶嚴顏色,不與語,又不使冠履。植伏地泣涕,太后為不樂。詔乃聽復王服。}
\gezhu{魏氏春秋曰:是時待遇諸國法峻。任城王暴薨。諸王旣懷友于之痛。植及白馬王彪還國,欲同路東歸,以叙隔闊之思,而監國使者不聽。植發憤告離而作詩曰:「謁帝承明廬,逝將歸舊疆。清晨發皇邑,日夕過首陽。伊、洛曠且深,欲濟川無梁。汎舟越洪濤,怨彼東路長。回顧戀城闕,引領情內傷。大谷何寥廓,山樹鬱蒼蒼。霖雨泥我塗,流潦浩從橫。中田絕無軌,改轍登高岡。脩阪造雲日,我馬玄以黃。玄黃猶能進,我思鬱以紆。鬱紆將何念?親愛在離居。本圖相與偕,中更不克俱。鴟梟鳴衡軛,豺狼當路衢;蒼蠅閒白黑,讒巧反親踈。欲還絕無蹊,攬轡止踟蹰。踟蹰亦何留,相思無終極。秋風發微涼,寒蟬鳴我側。原野何蕭條,白日忽西匿。孤獸走索羣,銜草不遑食。歸鳥赴高林,翩翩厲羽翼。感物傷我懷,撫心長歎息。歎息亦何為,天命與我違。柰何念同生,一往形不歸!孤魂翔故域,靈柩寄京師。存者勿復過,亡沒身自衰。人生處一世,忽若朝露晞。年在桑榆間,影嚮不能追。自顧非金石,咄咤令心悲。心悲動我神,棄置莫復陳。丈夫志四海,萬里猶比鄰。恩愛苟不虧,在遠分日親。何必同衾幬,然後展殷勤。倉卒骨肉情,能不懷苦辛?苦辛何慮思,天命信可疑。虛無求列仙,松子乆吾欺。變故在斯須,百年誰能持?離別永無會,執手將何時?王其愛玉體,俱享黃髮期。收涕即長塗,援筆從此辭。」}


六年,帝東征,還過雍丘,幸植宮,增戶五百。太和元年,徙封浚儀。二年,復還雍丘。植常自憤怨,抱利器而無所施,上疏求自試曰:


臣聞士之生世,入則事父,出則事君;事父尚於榮親,事君貴於興國。故慈父不能愛無益之子,仁君不能畜無用之臣。夫論德而授官者,成功之君也;量能而受爵者,畢命之臣也。故君無虛授,臣無虛受;虛授謂之謬舉,虛受謂之尸祿,詩之「素餐」所由作也。昔二虢不辭兩國之任,其德厚也;旦、奭不讓燕、魯之封,其功大也。今臣蒙國重恩,三世于今矣。正值陛下升平之際,沐浴聖澤,潛潤德教,可謂厚幸矣。而竊位東藩,爵在上列,身被輕煖,口厭百味,目極華靡,耳倦絲竹者,爵重祿厚之所致也。退念古之授爵祿者,有異於此,皆以功勤濟國,輔主惠民。今臣無德可述,無功可紀,若此終年無益國朝,將挂風人「彼己」之譏。是以上慙玄冕,俯愧朱紱。


方今天下一統,九州晏如,而顧西有違命之蜀,東有不臣之吳,使邊境未得脫甲,謀士未得高枕者,誠欲混同宇內以致太和也。故啟滅有扈而夏功昭,成克商、奄而周德著。今陛下以聖明統世,將欲卒文、武之功,繼成、康之隆,簡賢授能,以方叔、邵虎之臣鎮御四境,為國爪牙者,可謂當矣。然而高鳥未挂於輕繳,淵魚未縣於鉤餌者,恐釣射之術或未盡也。昔耿弇不俟光武,亟擊張步,言不以賊遺於君父。故車右伏劒於鳴轂,雍門刎首於齊境,若此二士,豈惡生而尚死哉?誠忿其慢主而陵君也。
\gezhu{劉向說苑曰:越甲至齊,雍門狄請死之。齊王曰:「鼓鐸之聲未聞,矢石未交,長兵未接,子何務死?知為人臣之禮邪?」雍門狄對曰:「臣聞之,昔者王田於囿,左轂鳴,車右請死之,王曰:『子何為死?』車右曰:『為其鳴吾君也。』王曰:『左轂鳴者,此工師之罪也。子何事之有焉?』車右對曰:『吾不見工師之乘,而見其鳴吾君也。』遂刎頸而死。有是乎?」王曰:「有之。」雍門狄曰:「今越甲至,其鳴吾君,豈左轂之下哉?車右可以死左轂,而臣獨不可以死越甲邪?」遂刎頸而死。是日,越人引軍而退七十里,曰:「齊王有臣,鈞如雍門狄,疑使越社稷不血食。」遂歸。齊王葬雍門狄以上卿之禮。}
夫君之寵臣,欲以除患興利;臣之事君,必以殺身靜亂,以功報主也。昔賈誼弱冠,求試屬國,請係單于之頸而制其命;終軍以妙年使越,欲得長纓纓其王,羈致北闕。此二臣,豈好為誇主而燿世哉?志或鬱結,欲逞其才力,輸能於明君也。昔漢武為霍去病治第,辭曰:「匈奴未滅,臣無以家為!」固夫憂國忘家,捐軀濟難,忠臣之志也。今臣居外,非不厚也,而寢不安席,食不遑味者,伏以二方未克為念。


伏見先武皇帝武臣宿將,年耆即世者有聞矣。雖賢不乏世,宿將舊卒,猶習戰陣,竊不自量,志在效命,庶立毛髮之功,以報所受之恩。若使陛下出不世之詔,效臣錐刀之用,使得西屬大將軍,當一校之隊,若東屬大司馬,統偏舟之任,必乘危蹈險,騁舟奮驪,突刃觸鋒,為士卒先。雖未能禽權馘亮,庶將虜其雄率,殲其醜類,必効須臾之捷,以滅終身之愧,使名挂史筆,事列朝策。雖身分蜀境,首縣吳闕,猶生之年也。如微才弗試,沒世無聞,徒榮其軀而豐其體,生無益於事,死無損於數,虛荷上位而忝重祿,禽息鳥視,終於白首,此徒圈牢之養物,非臣之所志也。流聞東軍失備,師徒小衂,輟食棄餐,奮袂攘衽,撫劒東顧,而心已馳於吳會矣。


臣昔從先武皇帝南極赤岸,東臨滄海,西望玉門,北出玄塞,伏見所以行軍用兵之勢,可謂神妙矣。故兵者不可豫言,臨難而制變者也。志欲自效於明時,立功於聖世。每覽史籍,觀古忠臣義士,出一朝之命,以徇國家之難,身雖屠裂,而功銘著於鼎鍾,名稱垂於竹帛,未嘗不拊心而歎息也。臣聞明主使臣,不廢有罪。故奔北敗軍之將用,秦、魯以成其功;
\gezhu{臣松之案:秦用敗軍之將,事顯,故不注。魯連與燕將書曰:「曹子為魯將,三戰三北而亡地五百里,向使曹子計不反顧,義不旋踵,刎頸而死,則亦不免為敗軍之將矣。曹子棄三北之恥,而退與魯君計。桓公朝天子,會諸侯,曹子以一劒之任,披桓公之心於壇坫之上,顏色不變,辭氣不悖。三戰之所亡,一朝而復之。天下震動,諸侯驚駭,威加吳、越。」若此二士者,非不能成小廉而行小節也。}
絕纓盜馬之臣赦,楚、趙以濟其難。
\gezhu{臣松之案:楚莊掩絕纓之罪,事亦顯,故不書。秦穆公有赦盜馬事,趙則未聞。蓋以秦亦趙姓,故互文以避上「秦」字也。}
臣竊感先帝早崩,威王棄世,臣獨何人,以堪長乆!常恐先朝露,填溝壑,墳土未乾,而身名並滅。臣聞騏驥長鳴,則伯樂照其能;盧狗悲號,則韓國知其才。是以效之齊、楚之路,以逞千里之任;試之狡兔之捷,以驗搏噬之用。今臣志狗馬之微功,竊自惟度,終無伯樂、韓國之舉,是以於邑而竊自痛者也。


夫臨搏而企竦,聞樂而竊抃者,或有賞音而識道也。昔毛遂,趙之陪隷,猶假錐囊之喻,以寤主立功,何況巍巍大魏多士之朝,而無慷慨死難之臣乎!夫自衒自媒者,士女之醜行也。干時求進者,道家之明忌也。而臣敢陳聞於陛下者,誠與國分形同氣,憂患共之者也。兾以塵霧之微補益山海,熒燭末光增輝日月,是以敢冐其醜而獻其忠。
\gezhu{魏略曰:植雖上此表,猶疑不見用,故曰「夫人貴生者,非貴其養體好服,終竟年壽也,貴在其代天而理物也。夫爵祿者,非虛張者也,有功德然後應之,當矣。無功而爵厚,無德而祿重,或人以為榮,而壯夫以為恥。故太上立德,其次立功,蓋功德者所以垂名也。名者不滅,士之所利,故孔子有夕死之論,孟軻有棄生之義。彼一聖一賢,豈不願乆生哉?志或有不展也。是用喟然求試,必立功也。嗚呼!言之未用,欲使後之君子知吾意者也。}


三年,徙封東阿。五年,復上疏求存問親戚,因致其意曰:


臣聞天稱其高者,以無不覆;地稱其廣者,以無不載;日月稱其明者,以無不照;江海稱其大者,以無不容。故孔子曰:「大哉堯之為君!惟天為大,惟堯則之。」夫天德之於萬物,可謂弘廣矣。蓋堯之為教,先親後疏,自近及遠。其傳曰:「克明俊德,以親九族;九族旣睦,平章百姓。」及周之文王亦崇厥化,其詩曰:「刑于寡妻,至于兄弟,以御于家邦。」是以雍雍穆穆。風人詠之。昔周公弔管、蔡之不咸,廣封懿親以藩屏王室,傳曰:「周之宗盟,異姓為後。」誠骨肉之恩爽而不離,親親之義寔在敦固,未有義而後其君,仁而遺其親者也。


伏惟陛下資帝唐欽明之德,體文王翼翼之仁,惠洽椒房,恩昭九族,羣后百寮,番休遞上,執政不廢於公朝,下情得展於私室,親理之路通,慶弔之情展,誠可謂恕己治人,推惠施恩者矣。至於臣者,人道絕緒,禁錮明時,臣竊自傷也。不敢乃望交氣類,脩人事,叙人倫。近且婚媾不通,兄弟乖絕,吉凶之問塞,慶弔之禮廢,恩紀之違,甚於路人,隔閡之異,殊於胡越。今臣以一切之制,永無朝覲之望,至於注心皇極,結情紫闥,神明知之矣。然天實為之,謂之何哉!退惟諸王常有戚戚具爾之心,願陛下沛然垂詔,使諸國慶問,四節得展,以叙骨肉之歡恩。全怡怡之篤義。妃妾之家,膏沐之遺,歲得再通,齊義於貴宗,等惠於百司,如此,則古人之所歎,風雅之所詠,復存於聖世矣。


臣伏自惟省,無錐刀之用。及觀陛下之所拔授,若以臣為異姓,竊自料度,不後於朝士矣。若得辭遠遊,戴武弁,解朱組,佩青紱,駙馬、奉車,趣得一號,安宅京室,執鞭珥筆,出從華蓋,入侍輦轂,承荅聖問,拾遺左右,乃臣丹誠之至願,不離於夢想者也。遠慕鹿鳴君臣之宴,中詠常棣匪他之誡,下思伐木友生之義,終懷蓼莪罔極之哀;每四節之會,塊然獨處,左右惟僕隷,所對惟妻子,高談無所與陳,發義無所與展,未嘗不聞樂而拊心,臨觴而歎息也。臣伏以為犬馬之誠不能動人,譬人之誠不能動天。崩城、隕霜,臣初信之,以臣心況,徒虛語耳。若葵藿之傾葉,太陽雖不為之回光,然向之者誠也。竊自比葵藿,若降天地之施,垂三光之明者,實在陛下。


臣聞文子曰:「不為福始,不為禍先。」今之否隔,友于同憂,而臣獨倡言者,竊不願於聖世使有不蒙施之物。有不蒙施之物,必有慘毒之懷,故栢舟有「天只」之怨,谷風有「棄予」之歎。故伊尹恥其君不為堯舜,孟子曰:「不以舜之所以事堯事其君者,不敬其君者也。」臣之愚蔽,固非虞、伊,至於欲使陛下崇光被時雍之美,宣緝熈章明之德者,是臣慺慺之誠,竊所獨守,寔懷鶴立企佇之心。敢復陳聞者,兾陛下儻發天聦而垂神聽也。


詔報曰:「蓋教化所由,各有隆弊,非皆善始而惡終也,事使之然。故夫忠厚仁及草木,則行葦之詩作;恩澤衰薄,不親九族,則角弓之章刺。今令諸國兄弟,情理簡怠,妃妾之家,膏沐疏略,朕縱不能敦而睦之,王援古喻義備悉矣,何言精誠不足以感通哉?夫明貴賤,崇親親,禮賢良,順少長,國之綱紀,本無禁固諸國通問之詔也,矯枉過正,下吏懼譴,以至於此耳。已勑有司,如王所訴。」


植復上疏陳審舉之義,曰:


臣聞天地協氣而萬物生,君臣合德而庶政成;五帝之世非皆智,三季之末非皆愚,用與不用,知與不知也。旣時有舉賢之名,而無得賢之實,必各援其類而進矣。諺曰:「相門有相,將門有將。」夫相者,文德昭者也;將者,武功烈者也。文德昭,則可以匡國朝,致雍熈,稷、契、夔、龍是也;武功烈,則所以征不庭,威四夷,南仲、方叔是矣。昔伊尹之為媵臣,至賤也,呂尚之處屠釣,至陋也,及其見舉於湯武、周文,誠道合志同,玄謨神通,豈復假近習之薦,因左右之介哉。書曰:「有不世之君,必能用不世之臣;用不世之臣,必能立不世之功。」殷周二王是矣。若夫齷齪近步,遵常守故,安足為陛下言哉?故陰陽不和,三光不暢,官曠無人,庶政不整者,三司之責也。疆埸騷動,方隅內侵,沒軍喪衆,干戈不息者,邊將之憂也。豈可虛荷國寵而不稱其任哉?故任益隆者負益重,位益高者責益深,書稱「無曠庶官」,詩有「職思其憂」,此其義也。


陛下體天真之淑聖,登神機以繼統,兾聞康哉之歌,偃武行文之美。而數年以來,水旱不時,民困衣食,師徒之發,歲歲增調,加東有覆敗之軍,西有殪沒之將,至使蚌蛤浮翔於淮、泗,鼲鼬讙譁於林木。臣每念之,未嘗不輟食而揮餐,臨觴而搤腕矣。昔漢文發代,疑朝有變,宋昌曰:「內有朱虛、東牟之親,外有齊、楚、淮南、琅邪,此則磐石之宗,願王勿疑。」臣伏惟陛下遠覽姬文二虢之援,中慮周成召、畢之輔,下存宋昌磐石之固。昔騏驥之於吳阪,可謂困矣,及其伯樂相之,孫郵御之,形體不勞而坐取千里。蓋伯樂善御馬,明君善御臣;伯樂馳千里,明君致太平;誠任賢使能之明效也。若朝司惟良,萬機內理,武將行師,方難克弭。陛下可得雍容都城,何事勞動鑾駕,暴露於邊境哉?


臣聞羊質虎皮,見草則恱,見豺則戰,忘其皮之虎也。今置將不良,有似於此。故語曰:「患為之者不知,知之者不得為也。」昔樂毅奔趙,心不忘燕;廉頗在楚,思為趙將。臣生乎亂,長乎軍,又數承教于武皇帝,伏見行師用兵之要,不必取孫、吳而闇與之合。竊揆之於心,常願得一奉朝覲,排金門,蹈玉陛,列有職之臣,賜須臾之問,使臣得一散所懷,攄舒蘊積,死不恨矣。


被鴻臚所下發士息書,期會甚急。又聞豹尾已建,戎軒騖駕,陛下將復勞玉躬,擾挂神思。臣誠竦息,不遑寧處。願得策馬執鞭,首當塵露,撮風后之奇,接孫、吳之要,追慕卜商起予左右,效命先驅,畢命輪轂,雖無大益,兾有小補。然天高聽遠,情不上通,徒獨望青雲而拊心,仰高天而歎息耳。屈平曰:「國有驥而不知乘,焉皇皇而更索!」昔管、蔡放誅,周、召作弼;叔魚陷刑,叔向匡國。三監之釁,臣自當之;二南之輔,求必不遠。華宗貴族,藩王之中,必有應斯舉者。故傳曰:「無周公之親,不得行周公之事。」唯陛下少留意焉。


近者漢氏廣建藩王,豐則連城數十,約則饗食祖祭而已,未若姬周之樹國,五等之品制也。若扶蘇之諫始皇,淳于越之難周青臣,可謂知時變矣。夫能使天下傾耳注目者,當權者是矣,故謀能移主,威能懾下。豪右執政,不在親戚;權之所在,雖疏必重,勢之所去,雖親必輕,蓋取齊者田族,非呂宗也。分晉者趙、魏,非姬姓也。惟陛下察之。苟吉專其位,凶離其患者,異姓之臣也。欲國之安,祈家之貴,存共其榮,沒同其禍者,公族之臣也。今反公族疏而異姓親,臣竊惑焉。


臣聞孟子曰:「君子窮則獨善其身,達則兼善天下。」今臣與陛下踐冰履炭,登山浮澗,寒溫燥濕,高下共之,豈得離陛下哉?不勝憤懣,拜表陳情。若有不合,乞且藏之書府,不便滅棄,臣死之後,事或可思。若有豪氂少挂聖意者,乞出之朝堂,使夫博古之士,糾臣表之不合義者。如是,則臣願足矣。


帝輒優文荅報。
\gezhu{魏略曰:是後大發士息,及取諸國士。植以近前諸國士息已見發,其遺孤稚弱,在者無幾,而復被取,乃上書曰:「臣聞古者聖君,與日月齊其明,四時等其信,是以戮凶無重,賞善無輕,怒若驚霆,喜若時雨,恩不中絕,教無二可,以此臨朝,則臣下知所死矣。受任在萬里之外,審主之所以受官,必以之所投命,雖有構會之徒,泊然不以為懼者,蓋君臣相信之明效也。昔章子為齊將,人有告之反者,威王曰:『不然。』左右曰:『王何以明之?』王曰:『聞章子改葬死母;彼尚不欺死父,顧當叛生君乎?』此君之信臣也。昔管仲親射桓公,後幽囚從魯檻車載,使少年挽而送齊。管仲知桓公之必用己,懼魯之悔,謂少年曰:『吾為汝唱,汝為和,聲和聲,宜走。』於是管仲唱之,少年走而和之,日行數百里,宿昔而至。至則相齊,此臣之信君也。臣初受封,策書曰:『植受茲青社,封于東土,以屏翰皇家,為魏藩輔。』而所得兵百五十人,皆年在耳順,或不踰矩,虎賁官騎及親事凡二百餘人。正復不老,皆使年壯,備有不虞,檢校乘城,顧不足以自救,況皆復耄耋罷曳乎?而名為魏東藩,使屏翰王室,臣竊自羞矣。就之諸國,國有士子,合不過五百人。伏以為三軍益損,不復賴此。方外不定,必當須辨者,臣願將部曲倍道奔赴,夫妻負襁,子弟懷糧,蹈鋒履刃,以徇國難,何但習業小兒哉?愚誠以揮涕增河,鼷鼠飲海,於朝萬無損益,於臣家計甚有廢損。又臣士息前後三送,兼人已竭。惟尚有小兒,七八歲已上,十六七已還,三十餘人。今部曲皆年耆,卧在牀席,非糜不食,眼不能視,氣息裁屬者,凡三十七人;疲瘵風靡,疣盲聾聵者,二十三人。惟正須此小兒,大者可備宿衞,雖不足以禦寇,粗可以警小盜;小者未堪大使,為可使耘鉏穢草,驅護鳥雀。休候人則一事廢,一日獵則衆業散,不親自經營則功不攝;常自躬親,不委下吏而已。陛下聖仁,恩詔三至,士子給國,長不復發。明詔之下,有若皦日,保金石之恩,必明神之信,畫然自固,如天如地。定習業者並復見送,晻若晝晦,悵然失圖。伏以為陛下旣爵臣百寮之右,居藩國之任,為置卿士,屋名為宮,冢名為陵,不使其危居獨立,無異於凡庶。若柏成欣於野耕,子仲樂於灌園;蓬戶茅牖,原憲之宅也;陋巷單瓢,顏子之居也:臣才不見效用,常慨然執斯志焉。若陛下聽臣悉還部曲,罷官屬,省監官,使解璽釋紱,追柏成、子仲之業,營顏淵、原憲之事,居子臧之廬,宅延陵之室。如此,雖進無成功,退有可守,身死之日,猶松、喬也。然伏度國朝終未肯聽臣之若是,固當羈絆於世繩,維繫於祿位,懷屑屑之小憂,執無已之百念,安得蕩然肆志,逍遙於宇宙之外哉?此願未從,陛下必欲崇親親,篤骨肉,潤白骨而榮枯木者,惟遂仁德以副前恩詔。」皆遂還之。}


其年冬,詔諸王朝六年正月。其二月,以陳四縣封植為陳王,邑三千五百戶。植每欲求別見獨談,論及時政,幸兾試用,終不能得。旣還,悵然絕望。時法制,待藩國旣自峻迫,寮屬皆賈豎下才,兵人給其殘老,大數不過二百人。又植以前過,事復減半,十一年中而三徙都,常汲汲無歡,遂發疾薨,時年四十一。
\gezhu{植嘗為琴瑟調歌,辭曰:「吁嗟此轉蓬,居世何獨然!長去本根逝,夙夜無休閑。東西經七陌,南北越九阡,卒遇回風起,吹我入雲閒。自謂終天路,忽焉下沈淵。驚飈接我出,故歸彼中田。當南而更北,謂東而反西,宕宕當何依,忽亡而復存。飄颻周八澤,連翩歷五山,流轉無恒處,誰知吾苦艱?願為中林草,秋隨野火燔,糜滅豈不痛,願與林葉連。」}
\gezhu{孫盛曰:異哉,魏氏之封建也!不度先王之典,不思藩屏之術,違敦睦之風,背維城之義。漢初之封,或權侔人主,雖云不度,時勢然也。魏氏諸侯,陋同匹夫,雖懲七國,矯枉過也。且魏之代漢,非積德之由,風澤旣微,六合未一,而彫翦枝幹,委權異族,勢同瘣木,危若巢幕,不嗣忽諸,非天喪也。五等之制,萬世不易之典。六代興亡,曹冏論之詳矣。}
遺令薄葬。以小子志,保家之主也,欲立之。初,植登魚山,臨東阿,喟然有歸焉之心,遂營為墓。子志嗣,徙封濟北王。景初中詔曰:「陳思王昔雖有過失,旣克己慎行,以補前闕,且自少至終,篇籍不離于手,誠難能也。其收黃初中諸奏植罪狀,公卿已下議尚書、中書、祕書三府、大鴻臚者皆削除之。撰錄植前後所著賦頌詩銘雜論凡百餘篇,副藏內外。」志累增邑,并前九百九十戶。
\gezhu{志別傳曰:志字允恭,好學有才行。晉武帝為中撫軍,迎常道鄉公于鄴,志夜與帝相見,帝與語,從暮至旦,甚器之。及受禪,改封鄄城公。發詔以志為樂平太守,歷章武、趙郡,遷散騎常侍、國子博士,後轉博士祭酒。及齊王攸當之藩,下禮官議崇錫之典,志歎曰:「安有如此之才,如此之親,而不得樹本助化,而遠出海隅者乎?」乃建議以諫,辭旨甚切。帝大怒,免志官。後復為散騎常侍。志遭母憂,居喪盡哀,因得疾病,喜怒失常,太康九年卒,謚曰定公。}


\end{pinyinscope}