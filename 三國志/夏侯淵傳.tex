\article{夏侯淵傳}
\begin{pinyinscope}
 
 
 夏侯淵字妙才,惇族弟也。太祖居家,曾有縣官事,淵代引重罪,太祖營救之,得免。
 
 
\gezhu{魏略曰:時兖、豫大亂,淵以饑乏,棄其幼子,而活亡弟孤女。}
 太祖起兵,以別部司馬、騎都尉從,遷陳留、潁川太守。及與袁紹戰於官渡,行督軍校尉。紹破,使督兖、豫、徐州軍糧;時軍食少,淵傳饋相繼,軍以復振。昌狶反,遣于禁擊之,未拔,復遣淵與禁并力,遂擊狶,降其十餘屯,狶詣禁降。淵還,拜典軍校尉。
 \gezhu{魏書曰:淵為將,赴急疾,常出敵之不意,故軍中為之語曰:「典軍校尉夏侯淵,三日五百,六日一千。」}
 濟南、樂安黃巾徐和、司馬俱等攻城,殺長吏,淵將泰山、齊、平原郡兵擊,大破之,斬和,平諸縣,收其糧穀以給軍士。
 
 
 
 
 十四年,以淵為行領軍。太祖征孫權還,使淵督諸將擊廬江叛者雷緒,緒破,又行征西護軍,督徐晃擊太原賊,攻下二十餘屯,斬賊帥商曜,屠其城。從征韓遂等,戰於渭南。又督朱靈平隃糜、汧氐。與太祖會安定,降楊秋。
 
 
 
 
 十七年,太祖乃還鄴,以淵行護軍將軍,督朱靈、路招等屯長安,擊破南山賊劉雄,降其衆。圍遂、超餘黨梁興於鄠,拔之,斬興,封博昌亭侯。馬超圍涼州刺史韋康於兾,淵救康,未到,康敗。去兾二百餘里,超來逆戰,軍不利。汧氐反,淵引軍還。
 
 
 
 
 十九年,趙衢、尹奉等謀討超,姜叙起兵鹵城以應之。衢等譎說超,使出擊叙,於後盡殺超妻子。超奔漢中,還圍祁山。叙等急求救,諸將議者欲須太祖節度。淵曰:「公在鄴,反覆四千里,比報,叙等必敗,非救急也。」遂行,使張郃督步騎五千在前,從陳倉狹道入,淵自督糧在後。郃至渭水上,超將氐羌數千逆郃。未戰,超走,郃進軍收超軍器械。淵到,諸縣皆已降。韓遂在顯親,淵欲襲取之,遂走。淵收遂軍糧,追至略陽城,去遂二十餘里,諸將欲攻之,或言當攻興國氐。淵以為遂兵精,興國城固,攻不可卒拔,不如擊長離諸羌。長離諸羌多在遂軍,必歸救其家。若捨羌獨守則孤,救長離則官兵得與野戰,可必虜也。淵乃留督將守輜重,輕兵步騎到長離,攻燒羌屯,斬獲甚衆。諸羌在遂軍者,各還種落。遂果救長離,與淵軍對陣。諸將見遂衆,惡之,欲結營作塹乃與戰。淵曰:「我轉鬬千里,今復作營塹,則士衆罷弊,不可乆。賊雖衆,易與耳。」乃鼔之,大破遂軍,得其旌麾,還略陽,進軍圍興國。氐王千萬逃奔馬超,餘衆降。轉擊高平屠各,皆散走,收其糧穀牛馬。乃假淵節。
 
 
 
 
 初,枹罕宋建因涼州亂,自號河首平漢王。太祖使淵帥諸將討建。淵至,圍枹罕,月餘拔之,斬建及所置丞相已下。淵別遣張郃等平河關,渡河入小湟中,河西諸羌盡降,隴右平。太祖下令曰:「宋建造為亂逆三十餘年,淵一舉滅之,虎步關右,所向無前。仲尼有言:『吾與爾不如也。』」
 
 
 
 
 二十一年,增封三百戶,并前八百戶。還擊武都氐羌下辯,收氐穀十餘萬斛。太祖西征張魯,淵等將涼州諸將侯王已下,與太祖會休亭。太祖每引見羌、胡,以淵畏之。會魯降,漢中平,以淵行都護將軍,督張郃、徐晃等平巴郡。太祖還鄴,留淵守漢中,即拜淵征西將軍。
 
 
 
 
 二十三年,劉備軍陽平關,淵率諸將拒之,相守連年。二十四年正月,備夜燒圍鹿角。淵使張郃護東圍,自將輕兵護南圍。備挑郃戰,郃軍不利。淵分所將兵半助郃,為備所襲,淵遂戰死。謚曰愍侯。
 
 
 
 
 初,淵雖數戰勝,太祖常戒曰:「為將當有怯弱時,不可但恃勇也。將當以勇為本,行之以智計;但知任勇,一匹夫敵耳。」
 
 
淵妻,太祖內妹。長子衡,尚太祖弟海陽哀侯女,恩寵特隆。衡襲爵,轉封安寧亭侯。黃初中,賜中子霸,太和中,賜霸四弟爵皆關內侯。霸,正始中為討蜀護軍右將軍,進封愽昌亭侯,素為曹爽所厚。聞爽誅,自疑,亡入蜀。以淵舊勳赦霸子,徙樂浪郡。
 \gezhu{魏略曰:霸字仲權。淵為蜀所害,故霸常切齒,欲有報蜀意。黃初中為偏將軍。子午之役,霸召為前鋒,進至興世圍,安營在曲谷中。蜀人望知其是霸也,指下兵攻之。霸手戰鹿角間,賴救至,然後解。後為右將軍,屯隴西,其養士和戎,並得其歡心。至正始中,代夏侯儒為征蜀護軍,統屬征西。時征西將軍夏侯玄,於霸為從子,而玄於曹爽為外弟。及司馬宣王誅曹爽,遂召玄,玄來東。霸聞曹爽被誅而玄又徵,以為禍必轉相及,心旣內恐;又霸先與雍州刺史郭淮不和,而淮代玄為征西,霸尤不安,故遂奔蜀。南趣陰平而失道,入窮谷中,糧盡,殺馬步行,足破,卧巖石下,使人求道,未知何之。蜀聞之,乃使人迎霸。初,建安五年,時霸從妹年十三四,在本郡,出行樵採,為張飛所得。飛知其良家女,遂以為妻,產息女,為劉禪皇后。故淵之初亡,飛妻請而葬之。及霸入蜀,禪與相見,釋之曰:「卿父自遇害於行間耳,非我先人之手刃也。」指其兒子以示之曰:「此夏侯氏之甥也。」厚加爵寵。}
 霸弟威,官至兖州刺史。
 \gezhu{世語曰:威字季權,任俠。貴歷荊、兖二州刺史。子駿,并州刺史。次莊,淮南太守。莊子湛,字孝若,以才愽文章,至南陽相、散騎常侍。莊,晉景陽皇后姊夫也。由此一門侈盛於時。}
 威弟惠,樂安太守。
 \gezhu{文章叙錄曰:惠字稚權,幼以才學見稱,善屬奏議。歷散騎黃門侍郎,與鍾毓數有辯駮,事多見從。遷燕相、樂安太守。年三十七卒。}
 惠弟和,河南尹。
 \gezhu{世語曰:和字義權,清辯有才論。歷河南尹、太常。淵弟三子稱,弟五子榮。從孫湛為其序曰:「稱字叔權。自孺子而好合聚童兒,為之渠帥,戲必為軍旅戰陳之事,有違者輒嚴以鞭捶,衆莫敢逆。淵陰奇之,使讀項羽傳及兵書,不肯,曰:『能則自為耳,安能學人?』年十六,淵與之田,見奔虎,稱驅馬逐之,禁之不可,一箭而倒。名聞太祖,太祖把其手,喜曰:『我得汝矣!』與文帝為布衣之交,每讌會,氣陵一坐,辯士不能屈。世之高名者多從之游。年十八卒。弟榮,字幼權。幼聦惠,七歲能屬文,誦書日千言,經目輒識之。文帝聞而請焉。賔客百餘人,人一奏刺,悉書其鄉邑名氏,世所謂爵里刺也,客示之,一寓目,使之遍談,不謬一人。帝深奇之。漢中之敗,榮年十三,左右提之走,不肯,曰:『君親在難,焉所逃死!』乃奮劒而戰,遂沒陣。」}
 衡薨,子績嗣,為虎賁中郎將。績薨,子襃嗣。
 
 
\end{pinyinscope}