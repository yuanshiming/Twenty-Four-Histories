\article{司馬朗傳}
\begin{pinyinscope}
 
 
 司馬朗字伯達,河內溫人也。
 
 
\gezhu{司馬彪序傳曰:朗祖父儁,字元異,博學好古,倜儻有大度。長八尺三寸,腰帶十圍,儀狀魁岸,與衆有異,鄉黨宗族咸景附焉。位至潁川太守。父防,字建公,性質直公方,雖閑居宴處,威儀不忒。雅好漢書名臣列傳,所諷誦者數十萬言。少仕州郡,歷官洛陽令、京兆尹,以年老轉拜騎都尉。養志閭巷,闔門自守。諸子雖冠成人,不命曰進不敢進,不命曰坐不敢坐,不指有所問不敢言,父子之間肅如也。年七十一,建安二十四年終。有子八人,朗最長,次即晉宣皇帝也。}
 九歲,人有道其父字者,朗曰:「慢人親者,不敬其親者也。」客謝之。十二,試經為童子郎,監試者以其身體壯大,疑朗匿年,劾問。朗曰:「朗之內外,累世長大,朗雖穉弱,無仰高之風,損年以求早成,非志所為也。」監試者異之。後關東兵起,故兾州刺史李邵家居野王,近山險,欲徙居溫。朗謂邵曰:「脣齒之喻,豈唯虞、虢,溫與野王即是也;今去彼而居此,是為避朝亡之期耳。且君,國人之望也,今冦未至而先徙,帶山之縣必駭,是搖動民之心而開姦宄之原也,切為郡內憂之。」邵不從。邊山之民果亂,內徙,或為冦鈔。
 
 
是時董卓遷天子都長安,卓因留洛陽。朗父防為治書御史,當徙西,以四方雲擾,乃遣朗將家屬還本縣。或有告朗欲逃亡者,執以詣卓,卓謂朗曰:「卿與吾亡兒同歲,幾大相負!」朗因曰:「明公以高世之德,遭陽九之會,清除群穢,廣舉賢士,此誠虛心垂慮,將興至治也。威德以隆,功業以著,而兵難日起,州郡鼎沸,郊境之內,民不安業,損棄居產,流亡藏竄,雖四關設禁,重加刑戮,猶不絕息,此朗之所以於邑也。願明公監觀往事,少加三思,即榮名並於日月,伊、周不足侔也。」卓曰:「吾亦悟之,卿言有意!」
 \gezhu{臣松之案朗此對,但為稱述卓功德,未相箴誨而已。了不自申釋,而卓便云「吾亦悟之,卿言有意」!客主之辭如為不相酬塞也。}
 
 
 
 
 朗知卓必亡,恐見留,即散財物以賂遺卓用事者,求歸鄉里。到謂父老曰;「董卓悖逆,為天下所讎,此忠臣義士奮發之時也。郡與京都境壤相接,洛東有成臯,北界大河,天下興義兵者若未得進,其勢必停於此。此乃四分五裂戰爭之地,難以自安,不如及道路尚通,舉宗東到黎陽。黎陽有營兵,趙威孫鄉里舊婚,為監營謁者,統兵馬,足以為主。若後有變,徐復觀望未晚也。」父老戀舊,莫有從者,惟同縣趙咨將家屬俱與朗往焉。後數月,關東諸州郡起兵,衆數十萬,皆集熒陽及河內。諸將不能相一,縱兵鈔略,民人死者且半。乆之,關東兵散,太祖與呂布相持於濮陽,朗乃將家還溫。時歲大饑,人相食,朗收恤宗族,教訓諸弟,不為衰世解業。
 
 
年二十二,太祖辟為司空掾屬,除成臯令,以病去,復為堂陽長。其治務寬惠,不行鞭杖,而民不犯禁。先時,民有徙充都內者,後縣調當作船,徙民恐其不辨,乃相率私還助之,其見愛如此。遷元城令,入為丞相主簿。朗以為天下土崩之勢,由秦滅五等之制,而郡國無蒐狩習戰之備故也。今雖五等未可復行,可令州郡並置兵,外備四夷,內威不軌,於策為長。又以為宜復井田。往者以民各有累世之業,難中奪之,是以至今。今承大亂之後,民人分散,土業无主,皆為公田,宜及此時復之。議雖未施行,然州郡領兵,朗本意也。遷兖州刺史,政化大行,百姓稱之。雖在軍旅,常麤衣惡食,儉以率下。雅好人倫典籍,鄉人李覿等盛得名譽,朗常顯貶下之;後覿等敗,時人服焉。鍾繇、王粲著論云:「非聖人不能致太平。」朗以為「伊、顏之徒雖非聖人,使得數世相承,太平可致」。
 \gezhu{魏書曰:文帝善朗論,命祕書錄其文。孫盛曰:繇旣失之,朗亦未為得也。昔「湯舉伊尹,而不仁者遠矣」。易稱「顏氏之子,其殆庶幾乎!有不善未嘗不知,知之未嘗復行」。由此而言,聖人之與大賢,行藏道一,舒卷斯同,御世垂風,理無降異;升泰之美,豈俟積世哉?「善人為邦百年,亦可以勝殘去殺」。又曰「不踐跡,亦不入于室」。數世之論,其在斯乎!方之大賢,固有間矣。}
 建安二十二年,與夏侯惇、臧霸等征吳。到居巢,軍士大疫,朗躬巡視,致醫藥。遇疾卒,時年四十七。遺命布衣幅巾,歛以時服,州人追思之。
 \gezhu{魏書曰:朗臨卒,謂將士曰:「刺史蒙國厚恩,督司萬里,微功未效,而遭此疫癘,旣不能自救,孤負國恩。身沒之後,其布衣幅巾,歛以時服,勿違吾志也。」}
 明帝即位,封朗子遺昌武亭侯,邑百戶。朗弟孚又以子望繼朗後。遺薨,望子洪嗣。
 \gezhu{晉諸公贊曰:望字子初,孚之長子。有才識,早知名。咸熈中位至司徒,入晉封義陽王,遷太尉、大司馬。時孚為太宰,父子居上公位,自中代已來未之有也。洪字孔業,封河間王。}
 
 
初朗所與俱徙趙咨,官至太常,為世好士。
 \gezhu{咨字君初。子酆字子仲,晉驃騎將軍,封東平陵公。並見百官名志。}
 
 
\end{pinyinscope}