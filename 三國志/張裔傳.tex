\article{張裔傳}
\begin{pinyinscope}
 
 
 張裔字君嗣,蜀郡成都人也。治公羊春秋,博涉史、漢。汝南許文休入蜀,謂裔幹理敏捷,是中夏鍾元常之倫也。劉璋時,舉孝廉,為魚復長,還州署從事,領帳下司馬。張飛自荊州由墊江入,璋授裔兵,拒張飛於德陽陌下,軍敗,還成都。為璋奉使詣先主,先主許以禮其君而安其人也,裔還,城門乃開。先主以裔為巴郡太守,還為司金中郎將,典作農戰之器。先是,益州郡殺太守正昂,耆率雍闓恩信著於南土,使命周旋,遠通孫權。乃以裔為益州太守,徑往至郡。闓遂趑趄不賔,假鬼教曰:「張府君如瓠壺,外雖澤而內實麤,不足殺,令縛與吳。」於是遂送裔於權。
 
 
 
 
 會先主薨,諸葛亮遣鄧芝使吳,亮令芝言次可從權請裔。裔自至吳數年,流徙伏匿,權未之知也,故許芝遣裔。裔臨發,權乃引見,問裔曰:「蜀卓氏寡女,亡奔司馬相如,貴土風俗何以乃爾乎?」裔對曰:「愚以卓氏之寡女,猶賢於買臣之妻。」權又謂裔曰:「君還,必用事西朝,終不作田父於閭里也,將何以報我?」裔對曰:「裔負罪而歸,將委命有司。若蒙徼倖得全首領,五十八已前父母之年也,自此已後大王之賜也。」權言笑歡恱,有器裔之色。裔出閤,深悔不能陽愚,即便就船,倍道兼行。權果追之,裔已入永安界數十里,追者不能及。
 
 
 
 
 旣至蜀,丞相亮以為參軍,署府事,又領益州治中從事。亮出駐漢中,裔以射聲校尉領留府長史,常稱曰:「公賞不遺遠,罰不阿近,爵不可以無功取,刑不可以貴勢免,此賢愚之所以僉忘其身者也。」其明年,北詣亮諮事,送者數百,車乘盈路,裔還書與所親曰:「近者涉道,晝夜接賔,不得寧息,人自敬丞相長史,男子張君嗣附之,疲倦欲死。」其談啁流速,皆此類也。
 
 
\gezhu{臣松之以為談啁貴於機捷,書疏可容留意。今因書疏之巧,以著談啁之速,非其理也。}
 少與犍為楊恭友善,恭早死,遺孤未數歲,裔迎留,與分屋而居,事恭母如母。恭之子息長大,為之娶婦,買田宅產業,使立門戶。撫恤故舊,振贍衰宗,行義甚至。加輔漢將軍,領長史如故。建興八年卒。子毣嗣,
 \gezhu{毣音忙角反,見字林,曰「毣,思貌也」。}
 歷三郡守監軍。毣弟郁,太子中庶子。
 
 
\end{pinyinscope}