\article{嚴畯傳}
\begin{pinyinscope}
 
 
 嚴畯字曼才,彭城人也。少耽學,善詩、書、三禮,又好說文。避亂江東,與諸葛瑾、步隲齊名友善。性質直純厚,其於人物,忠告善道,志存補益。張昭進之於孫權,權以為騎都尉、從事中郎。及橫江將軍魯肅卒,權以畯代肅,督兵萬人,鎮據陸口。衆人咸為畯喜,畯前後固辭:「樸素書生,不閑軍事,非才而據,咎悔必至。」發言慷慨,至于流涕,
 
 
\gezhu{志林曰:權又試畯騎,上馬墮鞍。}
 權乃聽焉。世嘉其能以實讓。權為吳王,及稱尊號,畯嘗為衞尉,使至蜀,蜀相諸葛亮深善之。不畜祿賜,皆散之親戚知故,家常不充。廣陵劉穎與畯有舊,穎精學家巷,權聞徵之,以疾不就。其弟略為零陵太守,卒官,穎往赴喪,權知其詐病,急驛收錄。畯亦馳語穎,使還謝權。權怒廢畯,而穎得免罪。乆之,以畯為尚書令,後卒。
 \gezhu{吳書曰:晙時年七十八,二子凱、爽。凱官至升平少府。}
 
 
 
 
 畯著孝經傳、潮水論,又與裴玄、張承論管仲、季路,皆傳於世。玄字彥黃,下邳人也,亦有學行,官至太中大夫。問子欽齊桓、晉文、夷、惠四人優劣,欽荅所見,與玄相反覆,各有文理。欽與太子登游處,登稱其翰采。
 
 
\end{pinyinscope}