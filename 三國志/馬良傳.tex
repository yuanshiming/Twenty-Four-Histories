\article{馬良傳}
\begin{pinyinscope}
 
 
 馬良字季常,襄陽宜城人也。兄弟五人,並有才名,鄉里為之諺曰:「馬氏五常,白眉最良。」良眉中有白毛,故以稱之。先主領荊州,辟為從事。及先主入蜀,諸葛亮亦從往,良留荊州,與亮書曰:「聞雒城已拔,此天祚也。尊兄應期贊世,配業光國,魄兆見矣。
 
 
\gezhu{臣松之以為良蓋與亮結為兄弟,或相與有親;亮年長,良故呼亮為尊兄耳。}
 夫變用雅慮,審貴垂明,於以簡才,宜適其時。若乃和光恱遠,邁德天壤,使時閑於聽,世服於道,齊高妙之音,正鄭、衞之聲,並利於事,無相奪倫,此乃管絃之至,牙、曠之調也。雖非鍾期,敢不擊節!」先主辟良為左將軍掾。
 
 
 
 
 後遣使吳,良謂亮曰:「今銜國命,恊穆二家,幸為良介於孫將軍。」亮曰:「君試自為文。」良即為草曰:「寡君遣掾馬良通聘繼好,以紹昆吾、豕韋之勳。其人吉士,荊楚之令,鮮於造次之華,而有克終之美,願降心存納,以慰將命。」權敬待之。
 
 
 
 
 先主稱尊號,以良為侍中。及東征吳,遣良入武陵招納五溪蠻夷,蠻夷渠帥皆受印號,咸如意指。會先主敗績於夷陵,良亦遇害。先主拜良子秉為騎都尉。
 
 
良弟謖,字幼常,以荊州從事隨先主入蜀,除緜竹成都令、越嶲太守。才器過人,好論軍計,丞相諸葛亮深加器異。先主臨薨謂亮曰:「馬謖言過其實,不可大用,君其察之!」亮猶謂不然,以謖為參軍,每引見談論,自晝達夜。
 \gezhu{襄陽記曰:建興三年,亮征南中,謖送之數十里。亮曰:「雖共謀之歷年,今可更惠良規。」謖對曰:「南中恃其險遠,不服乆矣,雖今日破之,明日復反耳。今公方傾國北伐以事彊賊。彼知官勢內虛,其叛亦速。若殄盡遺類以除後患,旣非仁者之情,且又不可倉卒也。夫用兵之道,攻心為上,攻城為下,心戰為上,兵戰為下,願公服其心而已。」亮納其策,赦孟獲以服南方。故終亮之世,南方不敢復反。}
 
 
建興六年,亮出軍向祁山,時有宿將魏延、吳壹等,論者皆言以為宜令為先鋒,而亮違衆拔謖,統大衆在前,與魏將張郃戰于街亭,為郃所破,士卒離散。亮進無所據,退軍還漢中。謖下獄物故,亮為之流涕。良死時年三十六,謖年三十九。
 \gezhu{襄陽記曰:謖臨終與亮書曰:「明公視謖猶子,謖視明公猶父,願深惟殛鯀興禹之義,使平生之交不虧於此,謖雖死無恨於黃壤也。」于時十萬之衆為之垂涕。亮自臨祭,待其遺孤若平生。蔣琬後詣漢中,謂亮曰:「昔楚殺得臣,然後文公喜可知也。天下未定而戮智計之士,豈不惜乎!」亮流涕曰:「孫武所以能制勝於天下者,用法明也。是以楊干亂法,魏絳戮其僕。四海分裂,兵交方始,若復廢法,何用討賊邪!」習鑿齒曰:諸葛亮之不能兼上國也,豈不宜哉!夫晉人規林甫之後濟,故廢法而收功;楚成闇得臣之益己,故殺之以重敗。今蜀僻陋一方,才少上國,而殺其俊傑,退收駑下之用,明法勝才,不師三敗之道,將以成業,不亦難乎!且先主誡謖之不可大用,豈不謂其非才也?亮受誡而不獲奉承,明謖之難廢也。為天下宰匠,欲大收物之力,而不量才節任,隨器付業;知之大過,則違明主之誡,裁之失中,即殺有益之人,難乎其可與言智者也。}
 
 
\end{pinyinscope}