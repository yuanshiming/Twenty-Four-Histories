\article{朱據傳}
\begin{pinyinscope}
 
 
 朱據字子範,吳郡吳人也。有姿貌膂力,又能論難。黃武初,徵拜五官郎中,補侍御史。是時選曹尚書曁豔,疾貪汙在位,欲沙汰之。據以為天下未定,宜以功覆過,棄瑕取用,舉清厲濁,足以沮勸,若一時貶黜,懼有後咎。豔不聽,卒敗。
 
 
 
 
 權咨嗟將率,發憤歎息,追思呂蒙、張溫,以為據才兼文武,可以繼之,自是拜建義校尉,領兵屯湖孰。黃龍元年,權遷都建業,徵據尚公主,拜左將軍,封雲陽侯。謙虛接士,輕財好施,祿賜雖豐而常不足用。嘉禾中,始鑄大錢,一當五百。後據部曲應受三萬緡,工王遂詐而受之,典校呂壹疑據實取,考問主者,死於杖下,據哀其無辜,厚棺斂之。壹又表據吏為據隱,故厚其殯。權數責問據,據無以自明,藉草待罪。數月,典軍吏劉助覺,言王遂所取,權大感寤,曰:「朱據見枉,況吏民乎?」乃窮治壹罪,賞助百萬。
 
 
 
 
 赤烏九年,遷驃騎將軍。遭二宮搆爭,據擁護太子,言則懇至,義形於色,守之以死,
 
 
\gezhu{殷基通語載據爭曰:「臣聞太子國之本根,雅性仁孝,天下歸心,今卒責之,將有一朝之慮。昔晉獻用驪姬而申生不存,漢武信江充而戾太子冤死。臣竊懼太子不堪其憂,雖立思子之宮,無所復及矣。」}
 遂左遷新都郡丞。未到,中書令孫弘譖潤據,因權寢疾,弘為昭書追賜死,時年五十七。孫亮時,二子熊、損各復領兵,為全公主所譖,皆死。永安中,追錄前功,以熊子宣襲爵雲陽侯,尚主。孫皓時,宣至驃騎將軍。
 
 
 
 
 評曰:虞翻古之狂直,固難免乎末世,然權不能容,非曠宇也。陸績之於楊玄,是仲尼之左丘明,老聃之嚴周矣;以瑚璉之器,而作守南越,不亦賊夫人歟!張溫才藻俊茂,而智防未備,用致艱患。駱統抗明大義,辭切理至,值權方閉不開。陸瑁篤義規諫,君子有稱焉。吾粲、朱據遭罹屯蹇,以正喪身,悲夫!
 
 
\end{pinyinscope}