\article{東平靈王徽傳}
\begin{pinyinscope}
 
 
 東平靈王徽,奉叔公朗陵哀侯王後。建安二十二年,封歷城侯。黃初二年,進爵為公。三年,為廬江王。四年,徙封壽張王。五年,改封壽張縣。太和六年,改封東平。青龍二年,徽使官屬檛壽張縣吏,為有司所奏。詔削縣一,戶五百。其年復所削縣。正始三年薨。子翕嗣。景初、正元、景元中,累增邑,并前三千四百戶。
 
 
\gezhu{臣松之案:翕入晉,封廩丘公。魏宗室之中,名次鄄城公。至泰始二年,翕遣世子琨奉表來朝。詔曰:「翕秉德履道,魏宗之良。今琨遠至,其假世子印綬,加騎都尉,賜朝服一具,錢十萬,隨才叙用。」翕撰解寒食散方,與皇甫謐所撰並行於世。}
 
 
\end{pinyinscope}