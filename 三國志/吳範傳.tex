\article{吳範傳}
\begin{pinyinscope}
 
 
 吳範字文則,會稽上虞人也。以治歷數,知風氣,聞於郡中。舉有道,詣京都,世亂不行。會孫權起於東南,範委身服事,每有災祥,輒推數言狀,其術多效,遂以顯名。
 
 
 
 
 初,權在吳,欲討黃祖,範曰:「今茲少利,不如明年。明年戊子,荊州劉表亦身死國亡。」權遂征祖,卒不能克。明年,軍出,行及尋陽,範見風氣,因詣船賀,催兵急行,至即破祖,祖得夜亡。權恐失之,範曰:「未遠,必生禽祖。」至五更中,果得之。劉表竟死,荊州分割。
 
 
 
 
 及壬辰歲,範又白言:「歲在甲午,劉備當得益州。」後呂岱從蜀還,遇之白帝,說備部衆離落,死亡且半,事必不克。權以難範,範曰:「臣所言者天道也,而岱所見者人事耳。」備卒得蜀。
 
 
 
 
 權與呂蒙謀襲關羽,議之近臣,多曰不可。權以問範,範曰:「得之。」後羽在麥城,使使請降。權問範曰:「竟當降否?」範曰:「彼有走氣,言降詐耳。」權使潘璋邀其徑路,覘候者還,白羽已去。範曰:「雖去不免。」問其期,曰:「明日日中。」權立表下漏以待之。及中不至,權問其故,範曰:「時尚未正中也。」頃之,有風動帷,範拊手曰:「羽至矣。」須臾,外稱萬歲,傳言得羽。
 
 
 
 
 後權與魏為好,範曰:「以風氣言之,彼以貌來,其實有謀,宜為之備。」劉備盛兵西陵,範曰:「後當和親。」終皆如言。其占驗明審如此。
 
 
 
 
 權以範為騎都尉,領太史令,數從訪問,欲知其決。範祕惜其術,不以至要語權。權由是恨之。
 
 
\gezhu{吳錄曰:範獨心計,所以見重者術,術亡則身棄矣,故終不言。}
 
 
 
 
 初,權為將軍時,範甞白言江南有王氣,亥子之間有大福慶。權曰:「若終如言,以君為侯。」及立為吳王,範時侍宴,曰:「昔在吳中,甞言此事,大王識之邪?」權曰:「有之。」因呼左右,以侯綬帶範。範知權欲以厭當前言,輒手推不受。及後論功行封,以範為都亭侯。詔臨當出,權恚其愛道於己也,削除其名。
 
 
範為人剛直,頗好自稱,然與親故交接有終始。素與魏滕同邑相善。滕甞有罪,權責怒甚嚴,敢有諫者死,範謂滕曰:「與汝偕死。」滕曰:「死而無益,何用死為?」範曰:「安能慮此,坐觀汝邪?」乃髠頭自縛詣門下,使鈴下以聞。鈴下不敢,曰:「必死,不敢白。」範曰:「汝有子邪?」曰:「有。」曰:「使汝為吳範死,子以屬我。」鈴下曰:「諾。」乃排閤入。言未卒,權大怒,欲便投以戟。逡巡走出,範因突入,叩頭流血,言與涕並。良乆,權意釋,乃免滕。滕見範謝曰:「父母能生長我,不能免我於死。丈夫相知,如汝足矣,何用多為!」
 \gezhu{會稽典錄曰:滕字周林,祖父河內太守朗,字少英,列在八俊。滕性剛直,行不苟合,雖遭困偪,終不迴撓。初亦迕策,幾殆,賴太妃救得免,語見妃嬪傳。歷歷陽、鄱陽、山陰三縣令,鄱陽太守。}
 
 
黃武五年,範病卒。長子先死,少子尚幼,於是業絕。權追思之,募三州有能舉知術數如吳範、趙達者,封千戶侯,卒無所得。
 \gezhu{吳錄曰:範先知其死日,謂權曰:「陛下某日當喪軍師。」權曰:「吾無軍師,焉得喪之?」範曰:「陛下出軍臨敵,須臣言而後行,臣乃陛下之軍師也。」至其日果卒。臣松之案,範死時,權未稱帝,此云陛下,非也。}
 
 
\end{pinyinscope}