\article{楚王彪傳}
\begin{pinyinscope}
 
 
 楚王彪字朱虎。建安二十一年,封壽春侯。黃初二年,進爵,徙封汝陽公。三年,封弋陽王。其年徙封吳王。五年,改封壽春縣。七年,徙封白馬。太和五年冬,朝京都。六年,改封楚。初,彪來朝,犯禁,青龍元年,為有司所奏,詔削縣三,戶千五百。二年,大赦,復所削縣。景初三年,增戶五百,并前三千戶。嘉平元年,兖州刺史令狐愚與太尉王淩謀迎彪都許昌。語在淩傳。乃遣傅及侍御史就國案驗,收治諸相連及者。廷尉請徵彪治罪。於是依漢燕王旦故事,使兼廷尉大鴻臚持節賜彪璽書切責之,使自圖焉。
 
 
\gezhu{孔衍漢魏春秋載璽書曰:「夫先王行賞不遺仇讎,用戮不違親戚,至公之義也。故周公流涕而決二叔之罪,孝武傷懷而斷昭平之獄,古今常典也。惟王,國之至親,作藩于外,不能祗奉王度,表率宗室,而謀於姧邪,乃與太尉王淩、兖州刺史令狐愚構通逆謀,圖危社稷,有悖忒之心,無忠孝之意。宗廟有靈,王其何面目以見先帝?朕深痛王自陷罪辜,旣得王情,深用憮然。有司奏王當就大理,朕惟公族甸師之義,不忍肆王市朝,故遣使者賜書。王自作孽,匪由於他,燕剌之事,宜足以觀。王其自圖之!」}
 彪乃自殺。妃及諸子皆免為庶人,徙平原。彪之官屬以下及監國謁者,坐知情無輔導之義,皆伏誅。國除為淮南郡。正元元年詔曰:「故楚王彪,背國附姧,身死嗣替,雖自取之,猶哀矜焉。夫含垢藏疾,親親之道也,其封彪世子嘉為常山真定王。」景元元年,增邑,并前二千五百戶。
 \gezhu{臣松之案:嘉入晉,封高邑公。元康中,與石崇俱為國子博士。嘉後為東莞太守,崇為征虜將軍,監青、徐軍事,屯於下邳,嘉以詩遺崇曰:「文武應時用,兼才在明哲。嗟嗟我石生,為國之俊傑。入侍於皇闥,出則登九列。威檢肅青、徐,風發宣吳裔。疇昔謬同位,情至過魯、衞。分離踰十載,思遠心增結。願子鑒斯誠,寒暑不踰契。」崇荅曰:「昔常接羽儀,俱游青雲中,敦道訓冑子,儒化渙以融,同聲無異響,故使恩愛隆。豈惟敦初好,款分在令終。孔不陋九夷,老氏適西戎。逍遙滄海隅,可以保王躬。世事非所務,周公不足夢。玄寂令神王,是以守至沖。」王隱晉書載吏部郎李重啟云:「魏氏宗室屈滯,每聖恩所存。東莞太守曹嘉,才幹學義,不及志、翕,而良素脩潔,性業踰之;又已歷二郡。臣以為優先代之後,可以嘉為員外散騎侍郎。」}
 
 
\end{pinyinscope}