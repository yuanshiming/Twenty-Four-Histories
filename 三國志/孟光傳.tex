\article{孟光傳}
\begin{pinyinscope}
 
 
 孟光字孝裕,河南洛陽人,漢太尉孟郁之族。
 
 
\gezhu{續漢書曰:郁,中常侍孟賁之弟。}
 靈帝末為講部吏。獻帝遷都長安,遂逃入蜀,劉焉父子待以客禮。博物識古,無書不覽,尤銳意三史,長於漢家舊典。好公羊春秋而譏呵左氏,每與來敏爭此二義,光常譊譊讙咋。
 \gezhu{譊音奴交反。讙音休袁反。咋音徂格反。}
 先主定益州,拜為議郎,與許慈等並掌制度。後主踐阼,為符節令、屯騎校尉、長樂少府,遷大司農。延熈九年秋,大赦,光於衆中責大將軍費禕曰:「夫赦者,偏枯之物,非明世所宜有也。衰弊窮極,必不得已,然後乃可權而行之耳。今主上仁賢,百僚稱職,有何旦夕之危,倒懸之急,而數施非常之恩,以惠姦宄之惡乎?又鷹隼始擊,而更原宥有罪,上犯天時,下違人理。老夫耄朽,不達治體,竊謂斯法難以經乆,豈具瞻之高美,所望於明德哉!」禕但顧謝踧踖而已。光之指摘痛癢,多如是類,故執政重臣,心不能恱,爵位不登;每直言無所囬避,為代所嫌。太常廣漢鐔承、
 \gezhu{華陽國志曰:承字公文,歷郡守少府。}
 光祿勳河東裴儁等,年資皆在光後,而登據上列,處光之右,蓋以此也。
 \gezhu{傅暢裴氏家記曰:儁字奉先,魏尚書令潛弟也。儁姊夫為蜀中長史,儁送之,時年十餘歲,遂遭漢末大亂,不復得還。旣長知名,為蜀所推重也。子越,字令緒,為蜀督軍。蜀破,遷還洛陽,拜議郎。}
 
 
 
 
 後進文士祕書郎郤正數從光諮訪,光問正太子所習讀并其情性好尚,正荅曰:「奉親虔恭,夙夜匪懈,有古世子之風;接待羣僚,舉動出於仁恕。」光曰:「如君所道,皆家戶所有耳;吾今所問,欲知其權略智調何如也。」正曰:「世子之道,在於承志竭歡,旣不得妄有所施為,且智調藏於胷懷,權略應時而發,此之有無,焉可豫設也?」光解正慎宜,不為放談,乃曰:「吾好直言,無所回避,每彈射利病,為世人所譏嫌疑;省君意亦不甚好吾言,然語有次。今天下未定,智意為先,智意雖有自然,然不可力彊致也。此儲君讀書,寧當傚吾等竭力博識以待訪問,如傅士探策講試以求爵位邪!當務其急者。」正深謂光言為然。後光坐事免官,年九十餘卒。
 
 
\end{pinyinscope}