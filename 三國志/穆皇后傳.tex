\article{穆皇后傳}
\begin{pinyinscope}
 
 
 先主穆皇后,陳留人也。兄吳壹,少孤,壹父素與劉焉有舊,是以舉家隨焉入蜀。焉有異志,而聞善相者相后當大貴。焉時將子瑁自隨,遂為瑁納后。瑁死,后寡居。先主旣定益州,而孫夫人還吳,
 
 
\gezhu{漢晉春秋云:先主入益州,吳遣迎孫夫人。夫人欲將太子歸吳,諸葛亮使趙雲勒兵斷江留太子,乃得止。}
 羣下勸先主聘后,先主疑與瑁同族,法正進曰:「論其親踈,何與晉文之於子圉乎?」於是納后為夫人。
 \gezhu{習鑿齒曰:夫婚姻,人倫之始,王化之本,匹夫猶不可以無禮,而況人君乎?晉文廢禮行權,以濟其業,故子犯曰,有求於人,必先從之,將奪其國,何有於妻,非無故而違禮教者也。今先主無權事之偪,而引前失以為譬,非導其君以堯、舜之道者。先主從之,過矣。}
 建安二十四年,立為漢中王后。章武元年夏五月,策曰:「朕承天命,奉至尊,臨萬國。今以后為皇后,遣使持節丞相亮授璽綬,承宗廟,母天下,皇后其敬之哉!」建興元年五月,後主即位,尊后為皇太后,稱長樂宮。壹官至車騎將軍,封縣侯。延熈八年,后薨,合葬惠陵。
 \gezhu{孫盛蜀世譜曰:壹孫喬,沒李雄中三十年,不為雄屈也。}
 
 
\end{pinyinscope}