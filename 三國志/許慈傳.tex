\article{許慈傳}
\begin{pinyinscope}
 
 
 許慈字仁篤,南陽人也。師事劉熈,善鄭氏學,治易、尚書、三禮、毛詩、論語。建安中,與許靖等俱自交州入蜀。時又有魏郡胡潛,字公興,不知其所以在益土。潛雖學不沾洽,然卓犖彊識,祖宗制度之儀,喪紀五服之數,皆指掌畫地,舉手可采。先主定蜀,承喪亂歷紀,學業衰廢,乃鳩合典籍,沙汰衆學,慈、潛並為博士,與孟光、來敏等典掌舊文。值庶事草創,動多疑議,慈、潛更相克伐,謗讟忿爭,形於聲色;書籍有無,不相通借,時尋楚撻,以相震攇。
 
 
\gezhu{攇,虛晚反。}
 其矜己妬彼,乃至於此。先主愍其若斯,羣僚大會,使倡家假為二子之容。傚其訟䦧之狀,酒酣樂作,以為嬉戲,初以辭義相難,終以刀杖相屈,用感切之。潛先沒,慈後主世稍遷至大長秋,卒。
 \gezhu{孫盛曰:蜀少人士,故慈、潛等並見載述。}
 子勛傳其業,復為博士。
 
 
\end{pinyinscope}