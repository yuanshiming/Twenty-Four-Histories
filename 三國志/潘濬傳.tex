\article{潘濬傳}
\begin{pinyinscope}
 
 
 潘濬字承明,武陵漢壽人也。弱冠從宋仲子受學。
 
 
\gezhu{吳書曰:濬為人聦察,對問有機理,山陽王粲見而貴異之。由是知名,為郡功曹。}
 年未三十,荊州牧劉表辟為部江夏從事。時沙羡長贓穢不脩,濬按殺之,一郡震竦。後為湘鄉令,治甚有名。劉備領荊州,以濬為治中從事。備入蜀,留典州事。
 
 
孫權殺關羽,并荊土,拜濬輔軍中郎將,授以兵。
 \gezhu{江表傳曰:權克荊州,將吏悉皆歸附,而濬獨稱疾不見。權遣人以牀就家輿致之,濬伏面著牀席不起,涕泣交橫,哀哽不能自勝。權慰勞與語,呼其字曰:「承明,昔觀丁父,鄀俘也,武王以為軍帥;彭仲爽,申俘也,文王以為令尹。此二人,卿荊國之先賢也,初雖見囚,後皆擢用,為楚名臣。卿獨不然,未肯降意,將以孤異古人之量邪?」使親近以手巾拭其面,濬起下地拜謝。即以為治中,荊州諸軍事一以諮之。武陵郡從事樊伷誘導諸夷,圖以武陵屬劉備,外白差督督萬人往討之。權不聽,特召問濬,濬荅:「以五千兵往,足可以擒伷。」權曰:「卿何以輕之?」濬曰:「伷是南陽舊姓,頗能弄脣吻,而實無辯論之才。臣所以知之者,伷昔甞為州人設饌,比至日中,食不可得,而十餘自起,此亦侏儒觀一節之驗也。」權大笑而納其言,即遣濬將五千往,果斬平之。}
 遷奮威將軍,封常遷亭侯。
 \gezhu{吳書曰:芮玄卒,濬并領玄兵,屯夏口。玄字文表,丹楊人。父祉,字宣嗣,從孫堅征伐有功,堅薦祉為九江太守,後轉吳郡,所在有聲。玄兄良,字文鸞,隨孫策平定江東,策以為會稽東部都尉,卒,玄領良兵,拜奮武中郎將,以功封溧陽侯。權為子登揀擇淑媛,羣臣咸稱玄父祉兄良並以德義文武顯名三世,故遂娉玄女為妃焉。黃武五年卒,權甚愍惜之。}
 權稱尊號,拜為少府,進封劉陽侯,
 \gezhu{江表傳曰:權數射雉,濬諫權,權曰:「相與別後,時時蹔出耳,不復如往日之時也。」濬曰:「天下未定,萬機務多,射雉非急,弦絕括破,皆能為害,乞特為臣故息置之。」濬出,見雉翳故在,乃手自撤壞之。權由是自絕,不復射雉。}
 遷太常。五谿蠻夷叛亂盤結,權假濬節,督諸軍討之。信賞必行,法不可干,斬首獲生,蓋以萬數,自是羣蠻衰弱,一方寧靜。
 \gezhu{吳書曰:驃騎將軍步隲屯漚口,求召募諸郡以增兵。權以問濬,濬曰:「豪將在民間,耗亂為害,加隲有名勢,在所所媚,不可聽也。」權從之。中郎將豫章徐宗,有名士也,甞到京師,與孔融交結,然儒生誕節,部曲寬縱,不奉節度,為衆作殿,濬遂斬之。其奉法不憚私議,皆此類也。歸義隱蕃,以口辯為豪傑所善,濬子翥亦與周旋,饋餉之。濬聞大怒,疏責翥曰:「吾受國厚恩,志報以命,爾輩在都,當念恭順,親賢慕善,何故與降虜交,以糧餉之?在遠聞此,心震面熱,惆悵累旬。疏到,急就往使受杖一百,促責所餉。」當時人咸怪濬,而蕃果圖叛誅夷,衆乃歸服。江表傳曰:時濬姨兄零陵蔣琬為蜀大將軍,或有間濬於武陵太守衞旌者,云濬遣密使與琬相聞,欲有自託之計。旌以啟權,權曰:「承明不為此也。」即封旌表以示於濬,而召旌還,免官。}
 
 
 
 
 先是,濬與陸遜俱駐武昌,共掌留事,還復故。時校事呂壹操弄威柄,奏桉丞相顧雍、左將軍朱據等,皆見禁止。黃門侍郎謝厷語次問壹:「顧公事何如?」壹荅:「不能佳。」厷又問:「若此公免退,誰當代之?」壹未荅厷,厷曰:「得無潘太常得之乎?」壹良乆曰:「君語近之也。」厷謂曰:「潘太常常切齒於君,但道遠無因耳。今日代顧公,恐明日便擊君矣。」壹大懼,遂解散雍事。濬求朝,詣建業,欲盡辭極諫。至,聞太子登已數言之而不見從,濬乃大請百寮,欲因會手刃殺壹,以身當之,為國除患。壹密聞知,稱疾不行。濬每進見,無不陳壹之姦險也。由此壹寵漸衰,後遂誅戮。權引咎責躬,因誚讓大臣,語在權傳。
 
 
赤烏二年,濬卒,子翥嗣。濬女配建昌侯孫慮。
 \gezhu{吳書曰:翥字文龍,拜騎都尉,後代領兵,早卒。翥弟祕,權以姊陳氏女妻之,調湘鄉令。襄陽記曰:襄陽習溫為荊州大公平。大公平,今之州都。祕過辭於溫,問曰:「先君昔曰君侯當為州里議主,今果如其言,不審州里誰當復相代者?」溫曰:「無過於君也。」後祕為尚書僕射,代溫為公平,甚得州里之譽。}
 
 
\end{pinyinscope}