\article{呂乂傳}
\begin{pinyinscope}
 
 
 呂乂字季陽,南陽人也。父常,送故將軍劉焉入蜀,值王路隔塞,遂不得還。乂少孤,好讀書鼔琴。初,先主定益州,置鹽府校尉,較鹽鐵之利,後校尉王連請乂及南陽杜祺、南鄉劉幹等並為典曹都尉。乂遷新都、緜竹令,乃心隱卹,百姓稱之,為一州諸城之首。遷巴西太守。丞相諸葛亮連年出軍,調發諸郡,多不相救,乂募取兵五千人詣亮,慰喻檢制,無逃竄者。徙為漢中太守,兼領督農,供繼軍糧。亮卒,累遷廣漢、蜀郡太守。蜀郡一都之會,戶口衆多,又亮卒之後,士伍亡命,更相重冒,姦巧非一。乂到官,為之防禁,開喻勸導,數年之中,漏脫自出者萬餘口。後入為尚書,代董允為尚書令,衆事無留,門無停賔。乂歷職內外,治身儉約,謙靖少言,為政簡而不煩,號為清能;然持法刻深,好用文俗吏,故居大官,名聲損於郡縣。延熈十四年卒。子辰,景耀中為成都令。辰弟雅,謁者。雅清厲有文才,著格論十五篇。
 
 
 
 
 杜祺歷郡守監軍大將軍司馬,劉幹官至巴西太守,皆與乂親善,亦有當時之稱,而儉素守法,不及於乂。
 
 
 
 
 評曰:董和蹈羔羊之素,劉巴履清尚之節,馬良貞實,稱為令士,陳震忠恪,老而益篤,董允匡主,義形於色,皆蜀臣之良矣。呂乂臨郡則垂稱,處朝則被損,亦黃、薛之流亞矣。
 
 
\end{pinyinscope}