\article{鄧哀王沖傳}
\begin{pinyinscope}
 
 
 鄧哀王沖字倉舒。少聦察岐嶷,生五六歲,智意所及,有若成人之智。時孫權曾致巨象,太祖欲知其斤重,訪之羣下,咸莫能出其理。沖曰:「置象大船之上,而刻其水痕所至,稱物以載之,則校可知矣。」太祖大恱,即施行焉。時軍國多事,用刑嚴重。太祖馬鞌在庫,而為鼠所齧,庫吏懼必死,議欲面縛首罪,猶懼不免。沖謂曰:「待三日中,然後自歸。」沖於是以刀穿單衣,如鼠齧者,謬為失意,貌有愁色。太祖問之,沖對曰:「世俗以為鼠齧衣者,其主不吉。今單衣見齧,是以憂戚。」太祖曰:「此妄言耳,無所苦也。」俄而庫吏以齧鞌聞,太祖笑曰:「兒衣在側,尚齧,況鞌縣柱乎?」一無所問。沖仁愛識達,皆此類也。凡應罪戮,而為沖微所辨理,賴以濟宥者,前後數十。
 
 
\gezhu{魏書曰:沖每見當刑者,輒探覩其冤枉之情而微理之。及勤勞之吏,以過誤觸罪,常為太祖陳說,宜寬宥之。辨察仁愛,與性俱生,容貌姿美,有殊於衆,故特見寵異。臣松之以「容貌姿美」一類之言,而分以為三,亦叙屬之一病也。}
 太祖數對羣臣稱述,有欲傳後意。年十三,建安十三年疾病,太祖親為請命。及亡,哀甚。文帝寬喻太祖,太祖曰:「此我之不幸,而汝曹之幸也。」
 \gezhu{孫盛曰:春秋之義,立嫡以長不以賢。沖雖存也猶不宜立,況其旣沒,而發斯言乎?詩云:「無易由言。」魏武其易之也。}
 言則流涕,為聘甄氏亡女與合葬,贈騎都尉印綬,命宛侯據子琮奉沖後。二十二年,封琮為鄧侯。黃初二年,追贈謚沖曰鄧哀侯,又追加號為公。
 \gezhu{魏書載策曰:「惟黃初二年八月丙午,皇帝曰:咨爾鄧哀侯沖,昔皇天鍾美於爾躬,俾聦哲之才,成於弱年。當永享顯祚,克成厥終。如何不祿,早世夭昏!朕承天序,享有四海,並建親親,以藩王室,惟爾不逮斯榮,且葬禮未備。追悼之懷,愴然攸傷。今遷葬于高陵,使使持節兼謁者僕射郎中陳承,追賜號曰鄧公,祠以太牢。魂而有靈,休茲寵榮。嗚呼哀哉!」魏略曰:文帝常言「家兄孝廉,自其分也。若使倉舒在,我亦無天下。」}
 三年,進琮爵,徙封冠軍公。四年,徙封己氏公。太和五年,加沖號曰鄧哀王。景初元年,琮坐於中尚方作禁物,削戶三百,貶爵為都鄉侯。三年,復為己氏公。正始七年,轉封平陽公。景初、正元、景元中,累增邑,并前千九百戶。
 
 
\end{pinyinscope}