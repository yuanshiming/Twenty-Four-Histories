\article{劉馥傳}
\begin{pinyinscope}
 
 
 劉馥字元穎,沛國相人也。避亂揚州,建安初,說袁術將戚寄、秦翊,使率衆與俱詣太祖。太祖恱之,司徒辟為掾。後孫策所置廬江太守李述攻殺揚州刺史嚴象,廬江梅乾、雷緒、陳蘭等聚衆數萬在江、淮間,郡縣殘破。太祖方有袁紹之難,謂馥可任以東南之事,遂表為揚州刺史。
 
 
 
 
 馥旣受命,單馬造合肥空城,建立州治,南懷緒等,皆安集之,貢獻相繼。數年中恩化大行,百姓樂其政,流民越江山而歸者以萬數。於是聚諸生,立學校,廣屯田,興治芍陂及茹陂、七門、吳塘諸堨以溉稻田,官民有畜。又高為城壘,多積木石,編作草苫數千萬枚,益貯魚膏數千斛,為戰守備。
 
 
 
 
 建安十三年卒。孫權率十萬衆攻圍合肥城百餘日,時天連雨,城欲崩,於是以苫蓑覆之,夜然脂照城外,視賊所作而為備,賊以破走。揚州士民益追思之,以為雖董安于之守晉陽,不能過也。及陂塘之利,至今為用。
 
 
 
 
 馥子靖,黃初中從黃門侍郎遷廬江太守,詔曰:「卿父昔為彼州,今卿復據此郡,可謂克負荷者也。」轉在河內,遷尚書,賜爵關內侯,出為河南尹。散騎常侍應璩書與靖曰:「入作納言,出臨京任。富民之術,日引月長。藩落高峻,絕穿窬之心。五種別出,遠水火之災。農器必具,無失時之闕。蠶麥有苫備之用,無雨濕之虞。封符指期,無流連之吏。鰥寡孤獨,蒙廩振之實。加之以明擿幽微,重之以秉憲不撓;有司供承王命,百里垂拱仰辨。雖昔趙、張、三王之治,未足以方也。」靖為政類如此。初雖如碎密,終於百姓便之,有馥遺風。母喪去官,後為大司農衞尉,進封廣陸亭侯,邑三百戶。上疏陳儒訓之本曰:「夫學者,治亂之軌儀,聖人之大教也。自黃初以來,崇立太學二十餘年,而寡有成者,蓋由愽士選輕,諸生避役,高門子弟,恥非其倫,故無學者。雖有其名而無其人,雖設其教而無其功。宜高選博士,取行為人表,經任人師者,掌教國子。依遵古法,使二千石以上子孫,年從十五,皆入太學。明制黜陟榮辱之路;其經明行脩者,則進之以崇德;荒教廢業者,則退之以懲惡;舉善而教不能則勸,浮華交游,不禁自息矣。闡弘大化,以綏未賔;六合承風,遠人來格。此聖人之教,致治之本也。」後遷鎮北將軍,假節都督河北諸軍事。靖以為「經常之大法,莫善於守防,使民夷有別」。遂開拓邊守,屯據險要。又脩廣戾陵渠大堨,水溉灌薊南北;三更種稻,邊民利之。嘉平六年薨,追贈征北將軍,進封建成鄉侯,謚曰景侯。子熈嗣。
 
 
\gezhu{晉陽秋曰:劉弘字叔和,熈之弟也。弘與晉世祖同年,居同里,以舊恩屢登顯位。自靖至弘,世不曠名,而有政事才。晉西朝之末,弘為車騎大將軍開府,荊州刺史,假節都督荊、交、廣州諸軍事,封新城郡公。其在江、漢,值王室多難,得專命一方,盡其器能。推誠羣下,厲以公義,簡刑獄,務農桑。每有興發,手書郡國,丁寧款密,故莫不感恱,顛倒奔赴,咸曰「得劉公一紙書,賢於十部從事也」。時帝在長安,命弘得選用宰守。徵士武陵伍朝高尚其事,牙門將皮初有勳江漢,弘上朝為零陵太守,初為襄陽太守。詔書以襄陽顯郡,初資名輕淺,以弘壻夏侯陟為襄陽。弘曰:「夫統天下者當與天下同心,治一國者當與一國推實。吾統荊州十郡,安得十女壻,然後為治哉!」乃表「陟姻親,舊制不得相監臨事,初勳宜見酬」。報聽之,衆益服其公當。廣漢太守辛冉以天子蒙塵,四方雲擾,進從橫計於弘。弘怒斬之,時人莫不稱善。晉諸公贊曰:于時天下雖亂,荊州安全。弘有劉景升保有江漢之志,不附太傅司馬越。越甚銜之。會弘病卒。子璠,北中郎將。}
 
 
\end{pinyinscope}