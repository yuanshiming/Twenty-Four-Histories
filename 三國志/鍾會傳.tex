\article{鍾會傳}
\begin{pinyinscope}
 
 
 鍾會字士季,潁川長社人,太傅繇小子也。少敏惠夙成。
 
 
\gezhu{會為其母傳曰:「夫人張氏,字昌蒲,太原茲氏人,太傅定陵成侯之命婦也。世長吏二千石。夫人少喪父母,充成侯家,脩身正行,非禮不動,為上下所稱述。貴妾孫氏,攝嫡專家,心害其賢,數讒毀无所不至。孫氏辨博有智巧,言足以飾非成過,然竟不能傷也。及姙娠,愈更嫉妬,乃置藥食中,夫人中食,覺而吐之,瞑眩者數日。或曰:『何不向公言之?』荅曰:『嫡庶相害,破家危國,古今以為鑒誡。假如公信我,衆誰能明其事?彼以心度我,謂我必言,固將先我;事由彼發,顧不快耶!』遂稱疾不見。孫氏果謂成侯曰:『妾欲其得男,故飲以得男之藥,反謂毒之!』成侯曰:『得男藥佳事,闇於食中與人,非人情也。』遂訊侍者具服,孫氏由是得罪出。成侯問夫人何能不言,夫人言其故,成侯大驚,益以此賢之。黃初六年,生會,恩寵愈隆。成侯旣出孫氏,更納正嫡賈氏。」臣松之案:鍾繇于時老矣,而方納正室。蓋禮所云宗子雖七十无无主婦之義也。魏氏春秋曰:會母見寵於繇,繇為之出其夫人。卞太后以為言,文帝詔繇復之。繇恚憤,將引鴆,弗獲,餐椒致噤,帝乃止。}
 中護軍蔣濟著論,謂「觀其眸子,足以知人。」會年五歲,繇遣見濟,濟甚異之,曰:「非常人也。」及壯,有才數技藝,而愽學精練名理,以夜續晝,由是獲聲譽。正始中,以為秘書郎,遷尚書中書侍郎。
 \gezhu{世語曰:司馬景王命中書令虞松作表,再呈輒不可意,命松更定。以經時,松思竭不能改,心苦之,形於顏色。會察其有憂,問松,松以實荅。會取視,為定五字。松恱服,以呈景王,王曰:「不當爾邪,誰所定也?」松曰:「鍾會。向亦欲啟之,會公見問,不敢饕其能。」王曰:「如此,可大用,可令來。」會問松王所能,松曰:「博學明識,無所不貫。」會乃絕賔客,精思十日,平旦入見,至鼓二乃出。出後,王獨拊手歎息曰:「此真王佐材也!」松字叔茂,陳留人,九江太守邊讓外孫。松弱冠有才,從司馬宣王征遼東,宣王命作檄,及破賊,作露布。松從還,宣王辟為掾,時年二十四,遷中書郎,遂至太守。松子濬,字顯弘,晉廷尉。臣松之以為鍾會名公之子,聲譽夙著,弱冠登朝,已歷顯仕,景王為相,何容不悉,而方於定虞松表然後乃蒙接引乎?設使先不相識,但見五字而便知可大用,雖聖人其猶病諸,而況景王哉?}
 高貴鄉公即尊位,賜爵關內侯。
 
 
 
 
 毌丘儉作亂,大將軍司馬景王東征,會從,典知密事,衞將軍司馬文王為大軍後繼。景王薨於許昌,文王緫統六軍,會謀謨帷幄。時中詔勑尚書傅嘏,以東南新定,權留衞將軍屯許昌為內外之援,令嘏率諸軍還。會與嘏謀,使嘏表上,輒與衞將軍俱發,還到雒水南屯住。於是朝廷拜文王為大將軍、輔政,會遷黃門侍郎,封東武亭侯,邑三百戶。
 
 
甘露二年,徵諸葛誕為司空,時會喪寧在家,策誕必不從命,馳白文王。文王以事已施行,不復追改。
 \gezhu{會時遭所生母喪。其母傳曰:「夫人性矜嚴,明於教訓,會雖童稚,勤見規誨。年四歲授孝經,七歲誦論語,八歲誦詩,十歲誦尚書,十一誦易,十二誦春秋左氏傳、國語,十三誦周禮、禮記,十四誦成侯易記,十五使入太學問四方奇文異訓。謂會曰:『學猥則倦,倦則意怠;吾懼汝之意怠,故以漸訓汝,今可以獨學矣。』雅好書籍,涉歷衆書,特好易、老子,每讀易孔子說鳴鶴在陰、勞謙君子、藉用白茅、不出戶庭之義,每使會反覆讀之,曰:『易三百餘爻,仲尼特說此者,以謙恭慎密,樞機之發,行己至要,榮身所由故也,順斯術已往,足為君子矣。』正始八年,會為尚書郎,夫人執會手而誨之曰:『汝弱冠見叙,人情不能不自足,則損在其中矣,勉思其戒!』是時大將軍曹爽專朝政,日縱酒沈醉,會兄侍中毓宴還,言其事。夫人曰:『樂則樂矣,然難乆也。居上不驕,制節謹度,然後乃無危溢之患。今奢僭若此,非長守富貴之道。』嘉平元年,車駕朝高平陵,會為中書郎,從行。相國宣文侯始舉兵,衆人恐懼,而夫人自若。中書令劉放、侍郎衞瓘、夏侯和等家皆怪問:『夫人一子在危難之中,何能無憂?』荅曰:『大將軍奢僭無度,吾常疑其不安。太傅義不危國,必為大將軍舉耳。吾兒在帝側何憂?聞且出兵無他重器,其勢必不乆戰。』果如其言,一時稱明。會歷機密十餘年,頗豫政謀。夫人謂曰:『昔范氏少子為趙簡子設伐邾之計,事從民恱,可謂功矣。然其母以為乘偽作詐,末業鄙事,必不能乆。其識本深遠,非近人所言,吾常樂其為人。汝居心正,吾能免矣。但當脩所志以輔益時化,不忝先人耳。常言人誰能皆體自然,但力行不倦,抑亦其次。雖接鄙賤,必以言信。取與之間,分畫分明。』或問:『此無乃小乎?』荅曰:『君子之行,皆積小以致高大,若以小善為無益而弗為,此乃小人之事耳。希通慕大者,吾所不好。』會自幼少,衣不過青紺,親營家事,自知恭儉。然見得思義,臨財必讓。會前後賜錢帛數百萬計,悉送供公家之用,一無所取。年五十有九,甘露二年二月暴疾薨。比葬,天子有手詔,命大將軍高都侯厚加賵贈,喪事無巨細,一皆供給。議者以為公侯有夫人,有世婦,有妻,有妾,所謂外命婦也。依春秋成風、定姒之義,宜崇典禮,不得緫稱妾名,於是稱成侯命婦。殯葬之事,有取於古制,禮也。」}
 及誕反,車駕住項,文王至壽春,會復從行。
 
 
 
 
 初,吳大將全琮,孫權之婚親重臣也,琮子懌、孫靜、從子端、翩、緝等,皆將兵來救誕。懌兄子輝、儀留建業,與其家內爭訟,携其母,將部曲數十家渡江,自歸文王。會建策,密為輝、儀作書,使輝、儀所親信齎入城告懌等,說吳中怒懌等不能拔壽春,欲盡誅諸將家,故逃來歸命。懌等恐懼,遂將所領開東城門出降,皆蒙封寵,城中由是乖離。壽春之破,會謀居多,親待日隆,時人謂之子房。軍還,遷為太僕,固辭不就。以中郎在大將軍府管記室事,為腹心之任。以討諸葛誕功,進爵陳侯,屢讓不受。詔曰:「會典綜軍事,參同計策,料敵制勝,有謀謨之勳,而推寵固讓,辭指款實,前後累重,志不可奪。夫成功不處,古人所重,其聽會所執,以成其美。」遷司隷校尉。雖在外司,時政損益,當世與奪,無不綜與。嵇康等見誅,皆會謀也。
 
 
 
 
 文王以蜀大將姜維屢擾邊陲,料蜀國小民疲,資力單竭,欲大舉圖蜀。惟會亦以為蜀可取,豫共籌度地形,考論事勢。景元三年冬,以會為鎮西將軍、假節都督關中諸軍事。文王勑青、徐、兖、豫、荊、揚諸州,並使作船,又令唐咨作浮海大船,外為將伐吳者。四年秋,乃下詔使鄧艾、諸葛緒各統諸軍三萬餘人,艾趣甘松、沓中連綴維,緒趣武街、橋頭絕維歸路。會統十餘萬衆,分從斜谷、駱谷入。先命牙門將許儀在前治道,會在後行,而橋穿,馬足陷,於是斬儀。儀者,許褚之子,有功王室,猶不原貸。諸軍聞之,莫不震竦。蜀令諸圍皆不得戰,退還漢、樂二城守。魏興太守劉欽趣子午谷,諸軍數道平行,至漢中。蜀監軍王含守樂城,護軍蔣斌守漢城,兵各五千。會使護軍荀愷、前將軍李輔各統萬人,愷圍漢城,輔圍樂城。會徑過,西出陽安口,遣人祭諸葛亮之墓。使護軍胡烈等行前,攻破關城,得庫藏積糓。姜維自沓中還,至陰平,合集士衆,欲赴關城。未到,聞其已破,退趣白水,與蜀將張翼、廖化等合守劒閣拒會。會移檄蜀將吏士民曰:
 
 
 
 
 往者漢祚衰微,率土分崩,生民之命,幾於泯滅。太祖武皇帝神武聖哲,撥亂反正,拯其將墜,造我區夏。高祖文皇帝應天順民,受命踐阼。烈祖明皇帝奕世重光,恢拓洪業。然江山之外異政殊俗,率土齊民未蒙王化,此三祖所以顧懷遺恨也。今主上聖德欽明,紹隆前緒,宰輔忠肅明允,劬勞王室,布政垂惠而萬邦恊和,施德百蠻而肅慎致貢。悼彼巴蜀,獨為匪民,愍此百姓,勞役未已。是以命授六師,龔行天罰,征西、雍州、鎮西諸軍,五道並進。古之行軍,以仁為本,以義治之;王者之師,有征無戰;故虞舜舞干戚而服有苗,周武有散財、發廩、表閭之義。今鎮西奉辭銜命,攝統戎重,庶弘文告之訓,以濟元元之命,非欲窮武極戰,以快一朝之政,故畧陳安危之要,其敬聽話言。
 
 
 
 
 益州先主以命世英才,興兵朔野,困躓兾、徐之郊,制命紹、布之手,太祖拯而濟之,與隆大好。中更背違,棄同即異,諸葛孔明仍規秦川,姜伯約屢出隴右,勞動我邊境,侵擾我氐、羌,方國家多故,未遑脩九伐之征也。今邊境乂清,方內無事,畜力待時,并兵一向,而巴蜀一州之衆,分張守備,難以禦天下之師。段谷、侯和沮傷之氣,難以敵堂堂之陳。比年以來,曾無寕歲,征夫勤瘁,難以當子來之民。此皆諸賢所親見也。蜀相牡見禽於秦,公孫述授首於漢,九州之險,是非一姓。此皆諸賢所備聞也。明者見危於無形,智者規禍於未萌,是以微子去商,長為周賔,陳平背項,立功於漢。豈晏安酖毒,懷祿而不變哉?今國朝隆天覆之恩,宰輔弘寬恕之德,先惠後誅,好生惡殺。往者吳將孫壹舉衆內附,位為上司,寵秩殊異。文欽、唐咨為國大害,叛主讎賊,還為戎首。咨困逼禽獲,欽二子還降,皆將軍、封侯;咨與聞國事。壹等窮踧歸命,猶加盛寵,况巴蜀賢知見機而作者哉!誠能深鑒成敗,邈然高蹈,投跡微子之蹤,錯身陳平之軌,則福同古人,慶流來裔,百姓士民,安堵舊業,農不易畒,巿不回肆,去累卵之危,就永安之福,豈不美與!若偷安旦夕,迷而不反,大兵一發,玉石皆碎,雖欲悔之,亦無及已。其詳擇利害,自求多福,各具宣布,咸使聞知。
 
 
鄧艾追姜維到陰平,簡選精銳,欲從漢德陽入江由、左儋道詣緜竹,趣成都,與諸葛緒共行。緒以本受節度邀姜維,西行非本詔,遂進軍前向白水,與會合。會遣將軍田章等從劒閣西,徑出江由。未至百里,章先破蜀伏兵三校,艾使章先登。遂長駈而前。會與緒軍向劒閣,會欲專軍勢,密白緒畏懦不進,檻車徵還。軍悉屬會,
 \gezhu{按百官名:緒入晉為太常崇禮衞尉。子沖,廷尉。荀綽兖州記曰:沖子詮,字德林,玫字仁林,並知名顯達。詮,兖州刺史。玫,侍中御史中丞。}
 進攻劒閣,不克,引退,蜀軍保險拒守。艾遂至緜竹,大戰,斬諸葛瞻。維等聞瞻已破,率其衆東入于巴。會乃進軍至涪,遣胡烈、田續、龐會等追維。艾進軍向成都,劉禪詣艾降,遣使勑維等令降於會。維至廣漢郪縣,令兵悉放器仗,送節傳於胡烈,便從東道詣會降。會上言曰:「賊姜維、張翼、廖化、董厥等逃死遁走,欲趣成都。臣輒遣司馬夏侯咸、護軍胡烈等,徑從劒閣,出新都、大渡截其前,參軍爰𩇕、將軍句安等躡其後,參軍皇甫闓、將軍王買等從涪南出衝其腹,臣據涪縣為取西勢援。維等所統步騎四五萬人,擐甲厲兵,塞川填谷,數百里中首尾相繼,憑恃其衆,方軌而西。臣勑咸、闓等令分兵據勢,廣張羅罔,南杜走吳之道,西塞成都之路,北絕越逸之徑,四靣雲集,首尾並進,蹊路斷絕,走伏無地。臣又手書申喻,開示生路,群寇困逼,知命窮數盡,解甲投戈,靣縛委質,印綬萬數,資器山積。昔舜舞干戚,有苗自服;牧野之師,商旅倒戈:有征無戰,帝王之盛業。全國為上,破國次之;全軍為上,破軍次之:用兵之令典。陛下聖德,侔蹤前代,翼輔忠明,齊軌公旦,仁育群生,義征不譓,殊俗向化,無思不服,師不踰時,兵不血刃,萬里同風,九州共貫。臣輙奉宣詔命,導揚恩化,復其社稷,安其閭伍,舍其賦調,弛其征役,訓之德禮以移其風,示之軌儀以易其俗,百姓欣欣,人懷逸豫,后來其蘇,義無以過。」會於是禁檢士衆不得鈔略,虛己誘納,以接蜀之群司,與維情好歡甚。
 \gezhu{世語曰:夏侯霸奔蜀,蜀朝問「司馬公如何德」?霸曰:「自當作家門。」「京師俊士」?曰:「有鍾士季,其人管朝政,吳、蜀之憂也。」漢晉春秋曰:初,夏侯霸降蜀,姜維問之曰:「司馬懿旣得彼政,當復有征伐之志不?」霸曰:「彼方營立家門,未遑外事。有鍾士季者,其人雖少,終為吳、蜀之憂,然非常之人亦不能用也。」後十五年而會果滅蜀。按習鑿齒此言,非出他書,故採用世語而附益也。}
 十二月詔曰:「會所向摧弊,前無彊敵,緘制衆城,罔羅迸逸。蜀之豪帥,靣縛歸命,謀無遺策,舉無廢功。凡所降誅,動以萬計,全勝獨克,有征無戰。拓平西夏,方隅清晏。其以會為司徒,進封縣侯,增邑萬戶。封子二人亭侯,邑各千戶。」
 
 
會內有異志,因鄧艾承制專事,密白艾有反狀,
 \gezhu{世語曰:會善效人書,於劒閣要艾章表白事,皆易其言,令辭指悖傲,多自矜伐。又毀文王報書,手作以疑之也。}
 於是詔書檻車徵艾。司馬文王懼艾或不從命,勑會並進軍成都,監軍衞瓘在會前行,以文王手筆令宣喻艾軍,艾軍皆釋仗,遂收艾入檻車。會所憚惟艾,艾旣禽而會尋至,獨統大衆,威震西土。自謂功名蓋世,不可復為人下,加猛將銳卒皆在己手,遂謀反。欲使姜維等皆將蜀兵出斜谷,會自將大衆隨其後。旣至長安,令騎士從陸道,步兵從水道順流浮渭入河,以為五日可到孟津,與騎會洛陽,一旦天下可定也。會得文王書云:「恐鄧艾或不就徵,今遣中護軍賈充將步騎萬人徑入斜谷,屯樂城,吾自將十萬屯長安,相見在近。」會得書,驚呼所親語之曰:「但取鄧艾,相國知我能獨辦之;今來大重,必覺我異矣,便當速發。事成,可得天下;不成,退保蜀漢,不失作劉備也。我自淮南以來,畫無遣策,四海所共知也。我欲持此安歸乎!」會以五年正月十五日至,其明日,悉請護軍、郡守、牙門騎督以上及蜀之故官,為太后發喪於蜀朝堂。矯太后遺詔,使會起兵廢文王,皆班示坐上人,使下議訖,書版署置,更使所親信代領諸軍。所請群官,悉閉著益州諸曹屋中,城門宮門皆閉,嚴兵圍守。會帳下督丘建本屬胡烈,烈薦之文王,會請以自隨,任愛之。建愍烈獨坐,啟會,使聽內一親兵出取飲食,諸牙門隨例各內一人。烈紿語親兵及疏與其子曰:「丘建密說消息,會已作大坑,白棓
 \gezhu{棓與棒同。}
 數千,欲悉呼外兵入,人賜白㡊,
 \gezhu{苦洽反。}
 拜為散將,以次棓殺坑中。」諸牙門親兵亦咸說此語,一夜傳相告,皆徧。或謂會:「可盡殺牙門騎督以上。」會猶豫未決。十八日日中,烈軍兵與烈兒雷鼓出門,諸軍兵不期皆鼓譟出,曾無督促之者,而爭先赴城。時方給與姜維鎧杖,白外有匈匈聲,似失火,有頃,白兵走向城。會驚,謂維曰:「兵來似欲作惡,當云何?」維曰:「但當擊之耳。」會遣兵悉殺所閉諸牙門郡守,內人共舉机以柱門,兵斫門,不能破。斯須,門外倚梯登城,或燒城屋,蟻附亂進,矢下如雨,牙門、郡守各緣屋出,與其卒兵相得。姜維率會左右戰,手殺五六人,衆旣格斬維,爭赴殺會。會時年四十,將士死者數百人。
 \gezhu{晉諸公贊曰:胡烈兒名淵,字世元,遵之孫也。遵,安定人,以才兼文武,累居藩鎮,至車騎將軍。子奮,字玄威,亦歷方任。女為晉武帝貴人,有寵。太康中,以奮為尚書僕射,加鎮軍大將軍、開府。弟廣,字宣祖,少府。次烈,字玄武,秦州刺史。次岐,宇玄嶷,并州刺史。廣子喜,涼州刺史。淵小字鷂鴟,時年十八,旣殺會救父,名震遠近。後趙王倫篡位,三王興義,倫使淵與張泓將兵禦齊王,屢破齊軍。會成都戰克,淵乃歸降伏法。}
 
 
初,艾為太尉,會為司徒,皆持節、都督諸軍如故,咸未受命而斃。會兄毓,以四年冬薨,會竟未知問。會兄子邕,隨會與俱死,會所養兄子毅及峻、辿
 \gezhu{勑連反。}
 等下獄,當伏誅。司馬文王表天子下詔曰:「峻等祖父繇,三祖之世,極位台司,佐命立勳,饗食廟庭。父毓,歷職內外,幹事有績。昔楚思子文之治,不滅鬪氏之祀。晉錄成宣之忠,用存趙氏之後。以會、邕之罪,而絕繇、毓之類,吾有愍然!峻、辿兄弟特原,有官爵者如故。惟毅及邕息伏法。」或曰,毓曾密啟司馬文王,言會挾術難保,不可專任,故宥峻等云。
 \gezhu{漢晉春秋曰:文王嘉其忠亮,笑荅毓曰:「若如卿言,必不以及宗矣。」}
 
 
初,文王欲遣會伐蜀,西曹屬邵悌求見曰:「今遣鍾會率十餘萬衆伐蜀,愚謂會單身無重任,不若使餘人行。」文王笑曰:「我寧當復不知此耶?蜀為天下作患,使民不得安息,我今伐之如指掌耳,而衆人皆言蜀不可伐。夫人心豫怯則智勇並竭,智勇並竭而彊使之,適為敵禽耳。惟鍾會與人意同,今遣會伐蜀,必可滅蜀。滅蜀之後,就如卿所慮,當何所能一辦耶?凡敗軍之將不可以語勇,亡國之大夫不可與圖存,心膽以破故也。若蜀以破,遺民震恐,不足與圖事;中國將士各自思歸,不肯與同也。若作惡,祗自滅族耳。卿不須憂此,慎莫使人聞也。」及會白鄧艾不軌,文王將西,悌復曰:「鍾會所統,五六倍於鄧艾,但可勑會取艾,不足自行。」文王曰:「卿忘前時所言邪,而更云可不須行乎?雖爾,此言不可宣也。我要自當以信義待人,但人不當負我,我豈可先人生心哉!近日賈護軍問我,言:『頗疑鍾會不?』我荅言:『如今遣卿行,寧可復疑卿邪?』賈亦無以易我語也。我到長安,則自了矣。」軍至長安,會果已死,咸如所策。
 \gezhu{按咸熈元年百官名:邵悌字元伯,陽平人。漢晉春秋曰:文王聞鍾會功曹向雄之收葬會也,召而責之曰:「往者王經之死,卿哭於東市而我不問,今鍾會躬為叛逆而又輒收葬,若復相容,其如王法何!」雄曰:「昔先王掩骼埋胔,仁流朽骨,當時豈先卜其功罪而後收葬哉?今王誅旣加,於法已備,雄感義收葬,教亦無闕。法立於上,教弘於下,以此訓物,雄曰可矣!何必使雄背死違生,以立於時。殿下讎對枯骨,捐之中野,百歲之後,為臧獲所笑,豈仁賢所掩哉?」王恱,與宴談而遣之。習鑿齒曰:向伯茂可謂勇於蹈義也,哭王經而哀感市人,葬鍾會而義動明主,彼皆忠烈奮勁,知死而往,非存生也。況使經、會處世,或身在急難,而有不赴者乎?故尋其奉死之心,可以見事生之情,覽其忠貞之節,足以愧背義之士矣。王加禮而遣,可謂明達。}
 
 
會常論易無玄體、才性同異。及會死後,於會家得書二十篇,名曰道論,而實刑名家也,其文似會。初,會弱冠與山陽王弼並知名。弼好論儒道,辭才逸辯,注易及老子,為尚書郎,年二十餘卒。
 \gezhu{弼字輔嗣。何劭為其傳曰:弼幼而察惠,年十餘,好老氏,通辨能言。父業,為尚書郎。時裴徽為吏部郎,弼未弱冠,往造焉。徽一見而異之,問弼曰:「夫無者誠萬物之所資也,然聖人莫肯致言,而老子申之無已者何?」弼曰:「聖人體無,無又不可以訓,故不說也。老子是有者也,故恒言無所不足。」尋亦為傅嘏所知。于時何晏為吏部尚書,甚奇弼,歎之曰:「仲尼稱後生可畏,若斯人者,可與言天人之際乎!」正始中,黃門侍郎累缺。晏旣用賈充、裴秀、朱整,又議用弼。時丁謐與晏爭衡,致高邑王黎於曹爽,爽用黎。於是以弼補臺郎。初除,覲爽,請間,爽為屏左右,而弼與論道,移時無所他及,爽以此嗤之。時爽專朝政,黨與共相進用,弼通儁不治名高。尋黎無幾時病亡,爽用王沈代黎,弼遂不得在門下,晏為之歎恨。弼在臺旣淺,事功亦雅非所長,益不留意焉。淮南人劉陶善論縱橫,為當時所推。每與弼語,常屈弼。弼天才卓出,當其所得,莫能奪也。性和理,樂游宴,解音律,善投壺。其論道賦會文辭,不如何晏,自然有所拔得,多晏也,頗以所長笑人,故時為士君子所疾。弼與鍾會善,會論議以校練為家,然每服弼之高致。何晏以為聖人無喜怒哀樂,其論甚精,鍾會等述之。弼與不同,以為聖人茂於人者神明也,同於人者五情也,神明茂故能體沖和以通無,五情同故不能無哀樂以應物,然則聖人之情,應物而無累於物者也。今以其無累,便謂不復應物,失之多矣。弼注易,潁川人荀融難弼大衍義。弼荅其意,白書以戲之曰:「夫明足以尋極幽微,而不能去自然之性。顏子之量,孔父之所預在,然遇之不能無樂,喪之不能無哀。又常狹斯人,以為未能以情從理者也,而今乃知自然之不可革。足下之量,雖已定乎胷懷之內,然而隔踰旬朔,何其相思之多乎?故知尼父之於顏子,可以無大過矣。」弼注老子,為之指略,致有理統。著道畧論,注易,往往有高麗言。太原王濟好談,病老、莊,常云:「見弼易注,所悟者多。」然弼為人淺而不識物情,初與王黎、荀融善,黎奪其黃門郎,於是恨黎,與融亦不終。正始十年,曹爽廢,以公事免。其秋遇癘疾亡,時年二十四,無子絕嗣。弼之卒也,晉景王聞之,嗟歎者累日,其為高識所惜如此。孫盛曰:易之為書,窮神知化,非天下之至精,其孰能與於此?世之注解,殆皆妄也。況弼以賦會之辨而欲籠統玄旨者乎?故其叙浮義則麗辭溢目,造陰陽則妙賾無間,至於六爻變化,群象所效,日時歲月,五氣相推,弼皆擯落,多所不關。雖有可觀者焉,恐將泥夫大道。博物記曰:初,王粲與族兄凱俱避地荊州,劉表欲以女妻粲,而嫌其形陋而用率,以凱有風貌,乃以妻凱。凱生業,業即劉表外孫也。蔡邕有書近萬卷,末年載數車與粲,粲亡後,相國掾魏諷謀反,粲子與焉,旣被誅,邕所與書悉入業。業字長緒,位至謁者僕射。子宏字正宗,司隷校尉。宏,弼之兄也。魏氏春秋曰:文帝旣誅粲二子,以業嗣粲。}
 
 
評曰:王淩風節格尚,毌丘儉才識拔幹,諸葛誕嚴毅威重,鍾會精練策數,咸以顯名,致茲榮任,而皆心大志迂,不慮禍難,變如發機,宗族塗地,豈不謬惑邪!鄧艾矯然彊壯,立功立事,然闇於防患,咎敗旋至,豈遠知乎諸葛恪而不能近自見,此蓋古人所謂目論者也。
 \gezhu{史記曰:越王無疆與中國爭彊,當楚威王時,越北伐齊,齊威王使人說越云,越王不納。齊使者曰:「幸也,越之不亡也。吾不貴其用智之如目,目見毫毛而不自見其睫也。今王知晉之失計,不自知越之過,是目論也。」}
 
 
\end{pinyinscope}