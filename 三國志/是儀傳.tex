\article{是儀傳}
\begin{pinyinscope}
 
 
 是儀字子羽,北海營陵人也。本姓氏,初為縣吏,後仕郡,郡相孔融嘲儀,言「氏」字「民」無上,可改為「是」,乃遂改焉。
 
 
\gezhu{徐衆評曰:古之建姓,或以所生,或以官號,或以祖名,皆有義體,以明氏族。故曰胙之以土而命之氏,此先王之典也,所以明本重始,彰示功德,子孫不忘也。今離文析字,橫生忌諱,使儀易姓,忘本誣祖,不亦謬哉!教人易姓,從人改族,融旣失之,儀又不得也。}
 後依劉繇,避亂江東。繇軍敗,儀徙會稽。
 
 
 
 
 孫權承攝大業,優文徵儀。到見親任,專典機密,拜騎都尉。
 
 
 
 
 呂蒙圖襲關羽,權以問儀,儀善其計,勸權聽之。從討羽,拜忠義校尉。儀陳謝,權令曰:「孤雖非趙簡子,卿安得不自屈為周舍邪?」
 
 
 
 
 旣定荊州,都武昌,拜裨將軍,後封都亭侯,守侍中。欲復授兵,儀自以非材,固辭不受。黃武中,遣儀之皖就將軍劉邵,欲誘致曹休。休到,大破之,遷偏將軍,入闕省尚書事,外緫平諸官,兼領辭訟,又令教諸公子書學。
 
 
大駕東遷,太子登留鎮武昌,使儀輔太子。太子敬之,事先諮詢,然後施行。進封都鄉侯。後從太子還建業,復拜侍中、中執法,平諸官事、領辭訟如舊。典校郎呂壹誣白故江夏太守刁嘉謗訕國政,權怒,收嘉繫獄,悉驗問。時同坐人皆怖畏壹,並言聞之,儀獨云無聞。於是見窮詰累日,詔旨轉厲,羣臣為之屏息。儀對曰:「今刀鋸已在臣頸,臣何敢為嘉隱諱,自取夷滅,為不忠之鬼!顧以聞知當有本末。」據實荅問,辭不傾移。權遂舍之,嘉亦得免。
 \gezhu{徐衆評曰:是儀以羈旅異方,客仕吳朝,值讒邪殄行,當嚴毅之威,命縣漏刻,禍急危機,不雷同以害人,不苟免以傷義,可謂忠勇公正之士,雖祁奚之免叔向,慶忌之濟朱雲,何以尚之?忠不諂君,勇不懾聳,公不存私,正不黨邪,資此四德,加之以文敏,崇之以謙約,履之以和順,保傅二宮,存身愛名,不亦宜乎!}
 
 
 
 
 蜀相諸葛亮卒,權垂心西州,遣儀使蜀申固盟好。奉使稱意,後拜尚書僕射。
 
 
 
 
 南、魯二宮初立,儀以本職領魯王傅。儀嫌二宮相近切,乃上疏曰:「臣竊以魯王天挺懿德,兼資文武,當今之宜,宜鎮四方,為國藩輔。宣揚德美,廣耀威靈,乃國家之良規,海內所瞻望,。但臣言辭鄙野,不能究盡其意。愚以二宮宜有降殺,正上下之序,明教化之本。」書三四上。為傅盡忠,動輒規諫;事上勤,與人恭。
 
 
 
 
 不治產業,不受施惠,為屋舍財足自容。鄰家有起大宅者,權出望見,問起大室者誰,左右對曰:「似是儀家也。」權曰:「儀儉,必非也。」問果他家。其見知信如此。
 
 
 
 
 服不精細,食不重膳,拯贍貧困,家無儲畜。權聞之,幸儀舍,求視蔬飯,親甞之,對之歎息,即增俸賜,益田宅。儀累辭讓,以恩為戚。
 
 
 
 
 時時有所進達,未甞言人之短。權常責儀以不言事,無所是非,儀對曰:「聖主在上,臣下守職,懼於不稱,實不敢以愚管之言,上干天聽。」
 
 
 
 
 事國數十年,未甞有過。呂壹歷白將相大臣,或一人以罪聞者數四,獨無以白儀。權歎曰:「使人盡如是儀,當安用科法為?」
 
 
 
 
 及寢疾,遺令素棺,斂以時服,務從省約,年八十一卒。
 
 
\end{pinyinscope}