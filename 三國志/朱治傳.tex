\article{朱治傳}
\begin{pinyinscope}
 
 
 朱治字君理,丹楊故鄣人也。初為縣吏,後察孝廉,州辟從事,隨孫堅征伐。中平五年,拜司馬,從討長沙、零、桂等三郡賊周朝、蘇馬等,有功,堅表治行都尉。從破董卓於陽人,入洛陽。表治行督軍校尉,特將步騎,東助徐州牧陶謙討黃巾。
 
 
 
 
 會堅薨,治扶翼策,依就袁術。後知術政德不立,乃勸策還平江東。時太傅馬日磾在壽春,辟治為掾,遷吳郡都尉。是時吳景已在丹楊,而策為術攻廬江,於是劉繇恐為袁、孫所并,遂搆嫌隙。而策家門盡在州下,治乃使人於曲阿迎太妃及權兄弟,所以供奉輔護,甚有恩紀。治從錢唐欲進到吳,吳郡太守許貢拒之於由拳,治與戰,大破之。貢南就山賊嚴白虎,治遂入郡,領太守事。策旣走劉繇,東定會稽。
 
 
 
 
 權年十五,治舉為孝廉。後策薨,治與張昭等共尊奉權。建安七年,權表治為九真太守,行扶義將軍,割婁、由拳、無錫、毗陵為奉邑,置長吏。征討夷越,佐定東南,禽截黃巾餘類陳敗、萬秉等。黃武元年,封毗陵侯,領郡如故。二年,拜安國將軍,金印紫綬,徙封故鄣。
 
 
 
 
 權歷位上將,及為吳王,治每進見,權常親迎,執版交拜,饗宴贈賜,恩敬特隆,至從行吏,皆得奉贄私覿,其見異如此。
 
 
 
 
 初,權弟翊,性峭急,喜怒快意,治數責數,諭以道義。權從兄豫章太守賁,女為曹公子婦,及曹公破荊州,威震南土,賁畏懼,欲遣子入質。治聞之,求往見賁,為陳安危,
 
 
\gezhu{江表傳載治說賁曰:「破虜將軍昔率義兵入討董卓,聲冠中夏,義士壯之。討逆繼世,廓定六郡,特以君侯骨肉至親,器為時生,故表漢朝,剖符大郡,兼建將校,仍關綜兩府,榮冠宗室,為遠近所瞻。加討虜聦明神武,繼承洪業,攬結英雄,周濟世務,軍衆日盛,事業日隆,雖昔蕭王之在河北,無以加也,必克成王基,應運東南。故劉玄德遠布腹心,求見拯救,此天下所共知也。前在東聞道路之言,云將軍有異趣,良用憮然。今曹公阻兵,傾覆漢室,幼帝流離,百姓元元未知所歸。而中國蕭條,或百里無煙,城邑空虛,道殣相望,士歎於外,婦怨乎室,加之以師旅,因之以饑饉,以此料之,豈能越長江與我爭利哉?將軍當斯時也,而欲背骨肉之親,違萬安之計,割同氣之膚,啖虎狼之口,為一女子,改慮易圖,失機毫釐,差以千里,豈不惜哉!」}
 賁由此遂止。
 
 
 
 
 權常歎治憂勤王事。性儉約,雖在富貴,車服惟供事。權優異之,自令督軍御史典屬城文書,治領四縣租稅而已。然公族子弟及吳四姓多出仕郡,郡吏常以千數,治率數年一遣詣王府,所遣數百人,每歲時獻御,權荅報過厚。是時丹楊深地,頻有姧叛,亦以年向老,思戀土風,自表屯故鄣,鎮撫山越。諸父老故人,莫不詣門,治皆引進,與共飲宴,鄉黨以為榮。在故鄣歲餘,還吳。黃武三年卒,在郡三十一年,年六十九。
 
 
子才,素為校尉領兵,旣嗣父爵,遷偏將軍。
 \gezhu{吳書曰:才字君業,為人精敏,善騎射,權愛異之,常侍從游戲。少以父任為武衞校尉,領兵隨從征伐,屢有功捷。本郡議者以才少處榮貴,未留意於鄉黨,才乃歎曰:「我初為將,謂跨馬蹈敵,當身履鋒,足以揚名,不知鄉黨復追迹其舉措乎!」於是更折節為恭,留意於賔客,輕財尚義,施不望報,又學兵法,名聲始聞於遠近。會疾卒。}
 才弟紀,權以策女妻之,亦以校尉領兵。紀弟緯、萬歲,皆早夭。才子琬,襲爵為將,至鎮西將軍。
 
 
\end{pinyinscope}