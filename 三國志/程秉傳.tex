\article{程秉傳}
\begin{pinyinscope}
 
 
 程秉字德樞,汝南南頓人也。逮事鄭玄,後避亂交州,與劉熈考論大義,遂博通五經。士燮命為長史。權聞其名儒,以禮徵秉,旣到,拜太子太傅。黃武四年,權為太子登娉周瑜女,秉守太常,迎妃於吳,權親幸秉船,深見優禮。旣還,秉從容進說登曰:「婚姻人倫之始,王教之基,是以聖王重之,所以率先衆庶,風化天下,故詩美關雎,以為稱首。願太子尊禮教於閨房,存周南之所詠,則道化隆於上,頌聲作於下矣。」登笑曰:「將順其美,匡救其惡,誠所賴於傅君也。」
 
 
 
 
 病卒官。著周易摘、尚書駮、論語弼,凡三萬餘言。秉為傅時,率更令河南徵崇亦篤學立行云。
 
 
\gezhu{吳錄曰:崇字子和,治易、春秋左氏傳,兼善內術。本姓李,遭亂更姓,遂隱於會稽,躬耕以求其志。好尚者從學,所教不過數人輒止,欲令其業必有成也。所交結如丞相步隲等,咸親焉。嚴畯薦崇行足以厲俗,學足以為師。初見太子登,以疾賜不拜。東宮官僚皆從諮詢。太子數訪以異聞。年七十而卒。}
 
 
\end{pinyinscope}