\article{孫亮全夫人傳}
\begin{pinyinscope}
 
 
 孫亮全夫人,全尚女也。尚從祖母公主愛之,每進見輒與俱。及潘夫人母子有寵,全主自以與孫和母有隙,乃勸權為潘氏男亮納夫人,亮遂為嗣。夫人立為皇后,以尚為城門校尉,封都亭侯,代滕胤為太常、衞將軍,進封永平侯,錄尚書事。時全氏侯有五人,並典兵馬,其餘為侍郎、騎都尉,宿衞左右,自吳興,外戚貴盛莫及。及魏大將諸葛誕以壽春來附,而全懌、全端、全煒、全儀等並因此際降魏,全熈謀泄見殺,由是諸全衰弱。會孫綝廢亮為會稽王,後又黜為候官侯,夫人隨之國,居候官,尚將家屬徙零陵,追見殺。
 
 
\gezhu{吳錄曰:亮妻惠解有容色,居候官,吳平乃歸,永寧中卒。}
 
 
\end{pinyinscope}