\article{簡雍傳}
\begin{pinyinscope}
 
 
 簡雍字憲和,涿郡人也。少與先主有舊,隨從周旋。先主至荊州,雍與麋笁、孫乾同為從事中郎,常為談客,往來使命。先主入益州,劉璋見雍,甚愛之。後先主圍成都,遣雍往說璋,璋遂與雍同輿而載,出城歸命。先主拜雍為昭德將軍。優游風儀,性簡傲跌宕,在先主坐席,猶箕踞傾倚,威儀不肅,自縱適;諸葛亮已下則獨擅一榻,項枕卧語,無所為屈。時天旱禁酒,釀者有刑。吏於人家索得釀具,論者欲令與作酒者同罰。雍與先主游觀,見一男女行道,謂先主曰:「彼人欲行淫,何以不縛?」先主曰:「卿何以知之?」雍對曰:「彼有其具,與欲釀者同。」先主大笑,而原欲釀者。雍之滑稽,皆此類也。
 
 
\gezhu{或曰:雍本姓耿,幽州人語謂耿為簡,遂隨音變之。}
 
 
\end{pinyinscope}