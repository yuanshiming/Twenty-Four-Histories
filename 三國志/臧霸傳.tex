\article{臧霸傳}
\begin{pinyinscope}
 
 
 臧霸字宣高,泰山華人也。父戒,為縣獄掾,據法不聽太守欲所私殺。太守大怒,令收戒詣府,時送者百餘人。霸年十八,將客數十人徑於費西山中要奪之,送者莫敢動,因與父俱亡命東海,由是以勇壯聞。黃巾起,霸從陶謙擊破之,拜騎都尉。遂收兵於徐州,與孫觀、吳敦、尹禮等並聚衆,霸為帥,屯於開陽。太祖之討呂布也,霸等將兵助布。旣禽布,霸自匿。太祖募索得霸,見而恱之,使霸招吳敦、尹禮、孫觀、觀兄康等,皆詣太祖。太祖以霸為琅邪相,敦利城、禮東莞、觀北海、康城陽太守,割青、徐二州,委之於霸。太祖之在兖州,以徐翕、毛暉為將。兖州亂,翕、暉皆叛。後兖州定,翕、暉亡命投霸。太祖語劉備,令語霸送二人首。霸謂備曰:「霸所以能自立者,以不謂此也。霸受公生全之恩,不敢違命。然王霸之君可以義告,願將軍為之辭。」備以霸言白太祖,太祖歎息,謂霸曰:「此古人之事而君能行之,孤之願也。」乃皆以翕、暉為郡守。時太祖方與袁紹相拒,而霸數以精兵入青州,故太祖得專事紹,不以東方為念。太祖破袁譚於南皮,霸等會賀。霸因求遣子弟及諸將父兄家屬詣鄴,太祖曰:「諸君忠孝,豈復在是!昔蕭何遣子弟入侍,而高祖不拒,耿純焚室輿櫬以從,而光武不逆,吾將何以易之哉!」東州擾攘,霸等執義征暴,清定海岱,功莫大焉,皆封列侯。霸為都亭侯,加威虜將軍。又與于禁討昌豨,與夏侯淵討黃巾餘賊徐和等,有功,遷徐州刺史。沛國公武周為下邳令,霸敬異周,身詣令舍。部從事𧩪詷不法,周得其罪,便收考竟,霸益以善周。從討孫權,先登,再入巢湖,攻居巢,破之。張遼之討陳蘭,霸別遣至皖,討吳將韓當,使權不得救蘭。當遣兵逆霸,霸與戰於逢龍,當復遣兵邀霸於夾石,與戰破之,還屯舒。權遣數萬人乘舩屯舒口,分兵救蘭,聞霸軍在舒,遁還。霸夜追之,比明,行百餘里,邀賊前後擊之。賊窘急,不得上舩,赴水者甚衆。由是賊不得救蘭,遼遂破之。霸從討孫權於濡須口,與張遼為前鋒,行遇霖雨,大軍先及,水遂長,賊舩稍進,將士皆不安。遼欲去,霸止之曰:「公明於利鈍,寧肯捐吾等邪?」明日果有令。遼至,以語太祖。太祖善之,拜揚威將軍,假節。後權乞降,太祖還,留霸與夏侯惇等屯居巢。
 
 
 
 
 文帝即王位,遷鎮東將軍,進爵武安鄉侯,都督青州諸軍事。及踐阼,進封開陽侯,徙封良成侯。與曹休討吳賊,破呂範於洞浦,徵為執金吾,位特進。每有軍事,帝常咨訪焉。
 
 
\gezhu{魏略曰:霸一名奴寇。孫觀名嬰子。吳敦名黯奴。尹禮名盧兒。建安二十四年,霸遣別軍在洛。會太祖崩,霸所部及青州兵,以為天下將亂,皆鳴鼓擅去。文帝即位,以曹休都督青、徐,霸謂休曰:「國家未肯聽霸耳!若假霸步騎萬人,必能橫行江表。」休言之於帝,帝疑霸軍前擅去,今意壯乃爾!遂東巡,因霸來朝而奪其兵。}
 明帝即位,增邑五百,并前三千五百戶。薨,謚曰威侯。子艾嗣。
 \gezhu{魏書曰:艾少以才理稱,為黃門郎,歷位郡守。}
 艾官至青州刺史、少府。艾薨,謚曰恭侯。子權嗣。霸前後有功,封子三人列侯,賜一人爵關內侯。
 \gezhu{霸一子舜,字太伯,晉散騎常侍,見武帝百官名。此百官名,不知誰所撰也,皆有題目,稱舜「才頴條暢,識贊時宜」也。}
 
 
而孫觀亦至青州刺史,假節,從太祖討孫權,戰被創,薨。子毓嗣,亦至青州刺史。
 \gezhu{魏書曰:孫觀字仲臺,泰山人。與臧霸俱起,討黃巾,拜騎都尉。太祖破呂布,使霸招觀兄弟,皆厚遇之。與霸俱戰伐,觀常為先登,征定青、徐羣賊,功次於霸,封呂都亭侯。康亦以功封列侯。與太祖會南皮,遣子弟入居鄴,拜觀偏將軍,遷青州刺史。從征孫權於濡須口,假節。攻權,為流矢所中,傷左足,力戰不顧,太祖勞之曰:「將軍被創深重,而猛氣益奮,不當為國愛身乎?」轉振威將軍,創甚,遂卒。}
 
 
\end{pinyinscope}