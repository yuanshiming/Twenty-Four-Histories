\article{鄧芝傳}
\begin{pinyinscope}
 
 
 鄧芝字伯苗,義陽新野人,漢司徒禹之後也。漢末入蜀,未見知待。時益州從事張裕善相,芝往從之,裕謂芝曰:「君年過七十,位至大將軍,封侯。」芝聞巴西太守龐羲好士,往依焉。先主定益州,芝為郫邸閣督。先主出至郫,與語,大奇之,擢為郫令,遷廣漢太守。所在清嚴有治績,入為尚書。
 
 
 
 
 先主薨於永安。先是,吳王孫權請和,先主累遣宋瑋、費禕等與相報荅。丞相諸葛亮深慮權聞先主殂隕,恐有異計,未知所如。芝見亮曰:「今主上幼弱,初在位,宜遣大使重申吳好。」亮荅之曰:「吾思之乆矣,未得其人耳,今日始得之。」芝問其人為誰?亮曰:「即使君也。」乃遣芝脩好於權。權果狐疑,不時見芝,芝乃自表請見權曰:「臣今來亦欲為吳,非但為蜀也。」權乃見之,語芝曰:「孤誠願與蜀和親,然恐蜀主幼弱,國小勢偪,為魏所乘,不自保全,以此猶豫耳。」芝對曰:「吳、蜀二國四州之地,大王命世之英,諸葛亮亦一時之傑也。蜀有重險之固,吳有三江之阻,合此二長,共為脣齒,進可并兼天下,退可鼎足而立,此理之自然也。大王今若委質於魏,魏必上望大王之入朝,下求太子之內侍,若不從命,則奉辭伐叛,蜀必順流見可而進,如此,江南之地非復大王之有也。」權默然良乆曰:「君言是也。」遂自絕魏,與蜀連和,遣張溫報聘於蜀。蜀復令芝重往,權謂芝曰:「若天下太平,二主分治,不亦樂乎!」芝對曰:「夫天無二日,土無二王,如并魏之後,大王未深識天命者也,君各茂其德,臣各盡其忠,將提枹鼓,則戰爭方始耳。」權大笑曰:「君之誠欵,乃當爾邪!」權與亮書曰:「丁厷掞張,
 
 
\gezhu{掞音夷念反,或作豔。臣松之案漢書禮樂志曰「長離前掞光耀明」。左思蜀都賦「摛藻掞天庭」。孫權蓋謂丁厷之言多浮豔也。}
 陰化不盡;和合二國,唯有鄧芝。」及亮北住漢中,以芝為中監軍、揚武將軍。亮卒,遷前軍師前將軍,領兖州刺史,封陽武亭侯,頃之為督江州。權數與芝相聞,饋遺優渥。延熈六年,就遷為車騎將軍,後假節。十一年,涪陵國人殺都尉反叛,芝率軍征討,即梟其渠帥,百姓安堵。
 \gezhu{華陽國志曰:芝征涪陵,見玄猿緣山。芝性好弩,手自射猿,中之。猿拔其箭,卷木葉塞其創。芝曰:「嘻,吾違物之性,其將死矣!」一曰:芝見猿抱子在樹上,引弩射之,中猿母,其子為拔箭,以木葉塞創。芝乃歎息,投弩水中,自知當死。}
 十四年卒。
 
 
 
 
 芝為大將軍二十餘年,賞罰明斷,善卹卒伍。身之衣食資仰於官,不苟素儉,然終不治私產,妻子不免饑寒,死之日家無餘財。性剛簡,不飾意氣,不得士類之和。於時人少所敬貴,唯器異姜維云。子良,襲爵,景耀中為尚書左選郎,晉朝廣漢太守。
 
 
\end{pinyinscope}