\article{李譔傳}
\begin{pinyinscope}
 
 
 李譔字欽仲,梓潼涪人也。父仁,字德賢,與同縣尹默俱游荊州,從司馬徽、宋忠等學。譔具傳其業,又從默講論義理,五經、諸子,無不該覽,加博好技藝,筭術、卜數、醫藥、弓弩、機械之巧,皆致思焉。始為州書佐、尚書令史。延熈元年,後主立太子,以譔為庶子,遷為僕射,轉中散中大夫、右中郎將,猶侍太子。太子愛其多知,甚恱之。然體輕脫,好戲啁,故世不能重也。著古文易、尚書、毛詩、三禮、左氏傳、太玄指歸,皆依準賈、馬,異於鄭玄。與王氏殊隔,初不見其所述,而意歸多同。景耀中卒。時又有漢中陳術,字申伯,亦博學多聞,著釋問七篇、益部耆舊傳及志,位歷三郡太守。
 
 
\end{pinyinscope}