\article{孫皓傳}
\begin{pinyinscope}

孫皓字元宗,權孫,和子也,一名彭祖,字皓宗。孫休立,封皓為烏程侯,遣就國。西湖民景養相皓當大貴,皓陰喜而不敢泄。休薨,是時蜀初亡,而交阯攜叛,國內震懼,貪得長君。左典軍萬彧昔為烏程令,與皓相善,稱皓才識明斷,是長沙桓王之疇也,又加之好學,奉遵法度,屢言之於丞相濮陽興、左將軍張布。興、布說休妃太后朱,欲以皓為嗣。朱曰:「我寡婦人,安知社稷之慮,苟吳國無損,宗廟有賴可矣。」於是遂迎立皓,時年二十三。改元,大赦。是歲,於魏咸熈元年也。


元興元年八月,以上大將軍施績、大將軍丁奉為左右大司馬,張布為驃騎將軍,加侍中,諸增位班賞,一皆如舊。九月,貶太后為景皇后,追謚父和曰文皇帝,尊母何為太后。十月,封休太子𩅦為豫章王,次子汝南王,次子梁王,次子陳王,立皇后滕氏。


\gezhu{江表傳曰:皓初立,發優詔,恤士民,開倉廩,振貧乏,科出宮女以配無妻,禽獸擾於苑者皆放之。當時翕然稱為明主。}
皓旣得志,麤暴驕盈,多忌諱,好酒色,大小失望。興、布竊悔之。或以譖皓,十一月,誅興、布。十二月,孫休葬定陵。封后父滕牧為高密侯,
\gezhu{吳歷曰:牧本名密,避丁密,改名牧,丁密避牧,改名為固。}
舅何洪等三人皆列侯。是歲,魏置交阯太守之郡。晉文帝為魏相國,遣昔吳壽春城降將徐紹、孫彧銜命齎書,陳事勢利害,以申喻皓。
\gezhu{漢晉春秋載晉文王與皓書曰:「聖人稱有君臣然後有上下禮義,是故大必字小,小必事大,然後上下安服,羣生獲所。逮至末塗,純德旣毀,勦民之命,以爭彊於天下,違禮順之至理,則仁者弗由也。方今主上聖明,覆燾無外,僕備位宰輔,屬當國重。唯華夏乖殊,方隅圮裂,六十餘載,金革亟動,無年不戰,暴骸喪元,困悴罔定,每用悼心,坐以待旦。將欲止戈興仁,為百姓請命,故分命偏師,平定蜀漢,役未經年,全軍獨克。于時猛將謀夫,朝臣庶士,咸以奉天時之宜,就旣征之軍,藉吞敵之勢,宜遂回旗東指,以臨吳境。舟師汎江,順流而下,陸軍南轅,取徑四郡,兼成都之械,漕巴漢之粟,然後以中軍整旅,三方雲會,未及浹辰,可使江表厎平,南夏順軌。然國朝深惟伐蜀之舉,雖有靜難之功,亦悼蜀民獨罹其害,戰於緜竹者,自元帥以下並受斬戮,伏尸蔽地,血流丹野。一之於前,猶追恨不忍,況重之於後乎?是故旋師案甲,思與南邦共全百姓之命。夫料力忖勢,度資量險,遠考古昔廢興之理,近鑒西蜀安危之効,隆德保祚,去危即順,屈己以寧四海者,仁哲之高致也;履危偷安,隕德覆祚,而不稱於後世者,非智者之所居也。今朝廷遣徐紹、孫彧獻書喻懷,若書御於前,必少留意,回慮革筭,結歡弭兵,共為一家,惠矜吳會,施及中土,豈不泰哉!此昭心之大願也,敢不承受。若不獲命,則普天率土,期於大同,雖重干戈,固不獲已也。」}


甘露元年三月,皓遣使隨紹、彧報書曰:「知以高世之才,處宰輔之任,漸導之功,勤亦至矣。孤以不德,階承統緒,思與賢良共濟世道,而以壅隔未有所緣,嘉意允著,深用依依。今遣光祿大夫紀陟、五官中郎將弘璆宣明至懷。」
\gezhu{江表傳曰:皓書兩頭言白,稱名言而不著姓。}
\gezhu{吳錄曰:陟字子上,丹楊人。初為中書郎,孫峻使詰南陽王和,令其引分。陟密使令正辭自理,峻怒。陟懼,閉門不出。孫休時,父亮為尚書令,而陟為中書令,每朝會,詔以屏風隔其座。出為豫章太守。}
\gezhu{干寶晉紀曰:陟、璆奉使如魏,入境而問諱,入國而問俗。壽春將王布示之馬射,旣而問之曰:「吳之君子亦能斯乎?」陟曰:「此軍人騎士肄業所及,士大夫君子未有為之者矣。」布大慙。旣至,魏帝見之,使儐問曰:「來時吳王何如?」陟對曰:「來時皇帝臨軒,百寮陪位,御膳無恙。」晉文王饗之,百寮畢會,使儐者告曰:「某者安樂公也,某者匈奴單于也。」陟曰:「西主失土,為君王所禮,位同三代,莫不感義,匈奴邊塞難羈之國,君王懷之,親在坐席,此誠威恩遠著。」又問:「吳之戍備幾何?」對曰:「自西陵以至江都,五千七百里。」又問曰:「道里甚遠,難為堅固?」對曰:「疆界雖遠,而其險要必爭之地,不過數四,猶人雖有八尺之軀靡不受患,其護風寒亦數處耳。」文王善之,厚為之禮。}
\gezhu{臣松之以為人有八尺之體靡不受患,防護風寒豈唯數處?取譬若此,未足稱能。若曰譬如金城萬雉,所急防者四門而已。方陟此對,不猶愈乎!}
\gezhu{吳錄曰:皓以諸父與和相連及者,家屬皆徙東冶,唯陟以有密旨,特封子孚都亭侯。孚弟瞻,字思遠,入仕晉驃騎將軍。弘璆,曲阿人,弘咨之孫,權外甥也。璆後至中書令、太子少傅。}
紹行到濡須,召還殺之,徙其家屬建安,始有白紹稱美中國者故也。夏四月,蔣陵言甘露降,於是改年大赦。秋七月,皓逼殺景后朱氏,亡不在正殿,於苑中小屋治喪,衆知其非疾病,莫不痛切。又送休四子於吳小城,尋復追殺大者二人。九月,從西陵督步闡表,徙都武昌,御史大夫丁固、右將軍諸葛靚鎮建業。陟、璆至洛,遇晉文帝崩,十一月,乃遣還。皓至武昌,又大赦。以零陵南部為始安郡,桂陽南部為始興郡。十二月,晉受禪。


寶鼎元年正月,遣大鴻臚張儼、五官中郎將丁忠弔祭晉文帝。及還,儼道病死。
\gezhu{吳錄曰:儼字子節,吳人也。弱冠知名,歷顯位,以博聞多識,拜大鴻臚。使于晉,皓謂儼曰:「今南北通好,以君為有出境之才,故相屈行。」對曰:「皇皇者華,蒙其榮耀,無古人延譽之美,磨厲鋒鍔,思不辱命。」旣至,車騎將軍賈充、尚書令裴秀、侍中荀勗等欲傲以所不知而不能屈。尚書僕射羊祜、尚書何楨並結縞帶之好。}
忠說皓曰:「北方守戰之具不設,弋陽可襲而取。」皓訪羣臣,鎮西大將軍陸凱曰:「夫兵不得已而用之耳,且三國鼎立已來,更相侵伐,無歲寧居。今彊敵新并巴蜀,有兼土之實,而遣使求親,欲息兵役,不可謂其求援於我。今敵形勢方彊,而欲徼幸求勝,未見其利也。」車騎將軍劉纂曰:「天生五才,誰能去兵?譎詐相雄,有自來矣。若其有闕,庸可棄乎?宜遣閒諜,以觀其勢。」皓陰納纂言,且以蜀新平,故不行,然遂自絕。八月,所在言得大鼎,於是改年,大赦。以陸凱為左丞相,常侍萬彧為右丞相。冬十月,永安山賊施但等聚衆數千人,
\gezhu{吳錄曰:永安今武康縣也。}
劫皓庶弟永安侯謙出烏程,取孫和陵上鼓吹曲蓋。比至建業,衆萬餘人。丁固、諸葛靚逆之於牛屯,大戰,但等敗走。獲謙,謙自殺。
\gezhu{漢晉春秋曰:初望氣者云荊州有王氣破揚州而建業宮不利,故皓徙武昌,遣使者發民掘荊州界大臣名家冢與山岡連者以厭之。旣聞但反,自以為徙土得計也。使數百人鼓譟入建業,殺但妻子,云天子使荊州兵來破揚州賊,以厭前氣。}
分會稽為東陽郡,分吳、丹楊為吳興郡。
\gezhu{皓詔曰:「古者分土建國,所以襃賞賢能,廣樹藩屏。秦毀五等為三十六郡,漢室初興,闓立乃至百五,因事制宜,蓋無常數也。今吳郡陽羨、永安、餘杭、臨水及丹楊故鄣、安吉、原鄉、於潛諸縣,地勢水流之便,悉注烏程,旣宜立郡以鎮山越,且以藩衞明陵,奉承大祭,不亦可乎!其亟分此九縣為吳興郡,治烏程。」}
以零陵北部為邵陵郡。十二月,皓還都建業,衞將軍滕牧留鎮武昌。


二年春,大赦。右丞相萬彧上鎮巴丘。夏六月,起顯明宮,
\gezhu{太康三年地記曰:吳有太初宮,方三百丈,權所起也。昭明宮方五百丈,皓所作也。避晉諱,故曰顯明。}
\gezhu{吳歷云:顯明在太初之東。}
\gezhu{江表傳曰:皓營新宮,二千石以下皆自入山督攝伐木。又破壞諸營,大開園囿,起土山樓觀,窮極伎巧,功役之費以億萬計。陸凱固諫,不從。}
冬十二月,皓移居之。是歲,分豫章、廬陵、長沙為安成郡。


三年春二月,以左右御史大夫丁固、孟仁為司徒、司空。
\gezhu{吳書曰:初,固為尚書,夢松樹生其腹上,謂人曰:「松字十八公也,後十八歲,吾其為公乎!」卒如夢焉。}
秋九月,皓出東關,丁奉至合肥。是歲,遣交州刺史劉俊、前部督脩則等入擊交阯,為晉將毛炅等所破,皆死,兵散還合浦。


建衡元年春正月,立子瑾為太子,及淮陽、東平王。冬十月,改年,大赦。十一月,左丞相陸凱卒。遣監軍虞汜、威南將軍薛珝、蒼梧太守陶璜由荊州,監軍李勗、督軍徐存從建安海道,皆就合浦擊交阯。


二年春。萬彧還建業。李勗以建安道不通利,殺導將馮斐,引軍還。三月,天火燒萬餘家,死者七百人。夏四月,左大司馬施績卒。殿中列將何定曰:「少府李勗枉殺馮斐,擅徹軍退還。」勗及徐存家屬皆伏誅。秋九月,何定將兵五千人上夏口獵。都督孫秀奔晉。是歲大赦。


三年春正月晦,皓舉大衆出華里,皓母及妃妾皆行,東觀令華覈等固爭,乃還。
\gezhu{江表傳曰:初丹楊刁玄使蜀,得司馬徽與劉廙論運命歷數事。玄詐增其文以誑國人曰:「黃旗紫蓋見於東南,終有天下者,荊、揚之君乎!」又得中國降人,言壽春下有童謠曰「吳天子當上」。皓聞之,喜曰:「此天命也。」即載其母妻子及後宮數千,從牛渚陸道西上,云青蓋入洛陽,以順天命。行遇大雪,道塗陷壞,兵士被甲持仗,百人共引一車,寒凍殆死。兵人不堪,皆曰:「若遇敵便當倒戈耳。」皓聞之,乃還。}
是歲,汜、璜破交阯,禽殺晉所置守將,九真、日南皆還屬。
\gezhu{漢晉春秋曰:初霍弋遣楊稷、毛炅等戍,與之誓曰:「若賊圍城,未百日而降者,家屬誅;若過百日而城沒者,刺史受其罪。」稷等日未滿而糧盡,乞降於璜。璜不許,而給糧使守。吳人並諫,璜曰:「霍弋已死,無能來者,可須其糧盡,然後乃受,使彼來無罪,而我取有義,內訓吾民,外懷鄰國,不亦可乎!」稷、炅糧盡,救不至,乃納之。}
\gezhu{華陽國志曰:稷,犍為人。炅,建寧人。稷等城中食盡,死亡者半,將軍王約反降,吳人得入城,獲稷、炅,皆囚之。孫皓使送稷下都,稷至合浦,歐血死。晉追贈交州刺史。初,毛炅與吳軍戰,殺前部督脩則。陶璜等以炅壯勇,欲赦之。而則子允固求殺炅,炅亦不為璜等屈,璜等怒,靣縛炅詰之,曰:「晉兵賊!」炅厲聲曰:「吳狗,何等為賊?」吳人生剖其腹,允割其心肝,罵曰:「庸復作賊?」炅猶罵不止,曰:「尚欲斬汝孫皓,汝父何死狗也!」乃斬之。晉武帝聞而哀矜,即詔使炅長子襲爵,餘三子皆關內侯。此與漢晉春秋所說不同。}
大赦,分交阯為新昌郡。諸將破扶嚴,置武平郡。以武昌督范慎為太尉。右大司馬丁奉、司空孟仁卒。
\gezhu{吳錄曰:仁字恭武,江夏人也,本名宗,避皓字,易焉。少從南陽李肅學。其母為作厚褥大被,或問其故,母曰:「小兒無德致客,學者多貧,故為廣被,庶可得與氣類接也。」其讀書夙夜不懈,肅奇之,曰:「卿宰相器也。」初為驃騎將軍朱據軍吏,將母在營。旣不得志,又夜雨屋漏,因起涕泣,以謝其母,母曰:「但當勉之,何足泣也?」據亦稍知之,除為塩池司馬。自能結網,手以捕魚,作鮓寄母,母因以還之,曰:「汝為魚官,而以鮓寄我,非避嫌也。」遷吳令。時皆不得將家之官,每得時物,來以寄母,常不先食。及聞母亡,犯禁委官,語在權傳。特為減死一等,復使為官,蓋優之也。}
\gezhu{楚國先賢傳曰:宗母嗜筍,冬節將至。時筍尚未生,宗入竹林哀歎,而筍為之出,得以供母,皆以為至孝之所致感。累遷光祿勳,遂至公矣。}
西苑言鳳皇集,改明年元。


鳳皇元年秋八月,徵西陵督步闡。闡不應,據城降晉。遣樂鄉都督陸抗圍取闡,闡衆悉降。闡及同計數十人皆夷三族。大赦。是歲右丞相萬彧被譴憂死,徙其子弟於廬陵。
\gezhu{江表傳曰:初皓游華里,彧與丁奉、留平密謀曰:「此行不急,若至華里不歸,社稷事重,不得不自還。」此語頗泄。皓聞知,以彧等舊臣,且以計忍而陰銜之。後因會,以毒酒飲彧,傳酒人私減之。又飲留平,平覺之,服他藥以解,得不死。彧自殺。平憂懣,月餘亦死。}
何定姦穢發聞,伏誅。皓以其惡似張布,追改定名為布。
\gezhu{江表傳曰:定,汝南人,本孫權給使也,後出補吏。定佞邪僭媚,自表先帝舊人,求還內侍,皓以為樓下都尉,典知酤糴事,專為威福。而皓信任,委以衆事。定為子求少府李勗女,不許。定挾忿譖勗於皓,皓尺口誅之,焚其尸。定又使諸將各上好犬,皆千里遠求,一犬至直數千匹。御犬率具纓,直錢一萬。一犬一兵,養以捕兎供廚。所獲無幾。吳人皆歸罪於定,而皓以為忠勤,賜爵列侯。}
\gezhu{吳歷曰:中書郎奚熈譖宛陵令賀惠。惠,劭弟也。遣使者徐粲訊治,熈又譖粲顧護不即決斷。皓遣使就宛陵斬粲,收惠付獄。會赦得免。}


二年春三月,以陸抗為大司馬。司徒丁固卒。秋九月,改封淮陽為魯,東平為齊,又封陳留、章陵等九王,凡十一王,王給三千兵。大赦。皓愛妾或使人至市劫奪百姓財物,司市中郎將陳聲,素皓幸臣也,恃皓寵遇,繩之以法。妾以愬皓,皓大怒,假他事燒鋸斷聲頭,投其身於四望之下。是歲,太尉范慎卒。


三年,會稽妖言章安侯奮當為天子。臨海太守奚熈與會稽太守郭誕書,非論國政。誕但白熈書,不白妖言,送付建安作船。
\gezhu{會稽邵氏家傳曰:邵疇字溫伯,時為誕功曹。誕被收,惶遽無以自明。疇進曰:「疇今自在,疇之事,明府何憂?」遂詣吏自列,云不白妖言,事由於己,非府君罪。吏上疇辭,皓怒猶盛。疇慮誕卒不免,遂自殺以證之。臨亡,置辭曰:「疇生長邊陲,不閑教道,得以門資,厠身本郡,踰越儕類,位極朝右,不能贊揚盛化,養之以福。今妖訛橫興,干國亂紀,疇以噂𠴲之語,本非事實,雖家誦人詠,不足有慮。天下重器,而匹夫橫議,疾其醜聲,不忍聞見,欲含垢藏疾,不彰之翰筆,鎮躁歸靜,使之自息。愚心勤勤,每執斯旨,故誕屈其所是,默以見從。此之為愆,實由於疇。謹不敢逃死,歸罪有司,唯乞天鑒,特垂清察。」吏收疇喪,得辭以聞,皓乃免誕大刑,送付建安作船。疇亡時,年四十。皓嘉疇節義,詔郡縣圖形廟堂。}
遣三郡督何植收熈,熈發兵自衞,斷絕海道。熈部曲殺熈,送首建業,夷三族。秋七月,遣使者二十五人分至州郡,科出亡叛。大司馬陸抗卒。自改年及是歲,連大疫。分鬱林為桂林郡。


天冊元年,吳郡言掘地得銀,長一尺,廣三分,刻上有年月字,於是大赦,改年。


天璽元年,吳郡言臨平湖自漢末草穢壅塞,今更開通。長老相傳,此湖塞,天下亂,此湖開,天下平。又於湖邊得石函,中有小石,青白色,長四寸,廣二寸餘,刻上作皇帝字,於是改年,大赦。會稽太守車浚、湘東太守張詠不出筭緡,就在所斬之,徇首諸郡。
\gezhu{江表傳曰:浚在公清忠,值郡荒旱,民無資糧,表求振貸。皓謂浚欲樹私恩,遣人梟首。又尚書熊睦見皓酷虐,微有所諫,皓使人以刀環撞殺之,身無完肌。}
秋八月,京下督孫楷降晉。鄱陽言歷陽山石文理成字,凡二十,云「楚九州渚,吳九州都,揚州士,作天子,四世治,太平始」。
\gezhu{江表傳曰:歷陽縣有石山臨水,高百丈,其三十丈所,有七穿駢羅,穿中色黃赤,不與本體相似,俗相傳謂之石印。又云,石印封發,天下當太平。下有祠屋,巫祝言石印神有三郎。時歷陽長表上言石印發,皓遣使以太牢祭歷山。巫言,石印三郎說「天下方太平」。使者作高梯,上看印文,詐以朱書石作二十字,還以啟皓。皓大喜曰:「吳當為九州作都、渚乎!從大皇帝逮孤四世矣,太平之主,非孤復誰?」重遣使,以印綬拜三郎為王,又刻石立銘,襃贊靈德,以荅休祥。}
又吳興陽羨山有空石,長十餘丈,名曰石室,在所表為大瑞。乃遣兼司徒董朝、兼太常周處至陽羨縣,封襌國山。明年改元,大赦,以恊石文。


天紀元年夏,夏口督孫慎出江夏、汝南,燒略居民。初,騶子張俶多所譖白,累遷為司直中郎將,封侯,甚見寵愛,是歲姦情發聞,伏誅。
\gezhu{江表傳曰:俶父,會稽山陰縣卒也,知俶不良,上表云:「若用俶為司直,有罪乞不從坐。」皓許之。俶表正彈曲二十人,專糾司不法,於是愛惡相攻,互相謗告。彈曲承言,收繫囹圄,聽訟失理,獄以賄成。人民窮困,無所措手足。俶奢淫無厭,取小妻三十餘人,擅殺無辜,衆姦並發,父子俱見車裂。}


二年秋七月,立成紀、宣威等十一王,王給三千兵,大赦。


三年夏,郭馬反。馬本合浦太守脩允部曲督。允轉桂林太守,疾病,住廣州,先遣馬將五百兵至郡安撫諸夷。允死,兵當分給,馬等累世舊軍,不樂離別。皓時又科實廣州戶口,馬與部曲將何典、王族、吳述、殷興等因此恐動兵民,合聚人衆,攻殺廣州督虞授。馬自號都督交廣二州諸軍事、安南將軍,興廣州刺史,述南海太守。典攻蒼梧,族攻始興。
\gezhu{漢晉春秋曰:先是,吳有說讖者曰:「吳之敗,兵起南裔,亡吳者公孫也。」皓聞之,文武職位至于卒伍有姓公孫者,皆徙於廣州,不令停江邊。及聞馬反,大懼曰:「此天亡也。」}
八月,以軍師張悌為丞相,牛渚都督何植為司徒。執金吾滕循為司空,未拜,轉鎮南將軍,假節領廣州牧,率萬人從東道討馬,與族遇於始興,未得前。馬殺南海太守劉略,逐廣州刺史徐旗。皓又遣徐陵督陶濬將七千人從西道,命交州牧陶璜部伍所領及合浦、鬱林諸郡兵,當與東西軍共擊馬。


有鬼目菜生工人黃耇家,依緣棗樹,長丈餘,莖廣四寸,厚三分。又有買菜生工人吳平家,高四尺,厚三分,如枇杷形,上廣尺八寸,下莖廣五寸,兩邊生葉綠色。東觀案圖,名鬼目作芝草,買菜作平慮草,遂以耇為侍芝郎,平為平慮郎,皆銀印青綬。


冬,晉命鎮東大將軍司馬伷向涂中,安東將軍王渾、揚州刺史周浚向牛渚,建威將軍王戎向武昌,平南將軍胡奮向夏口,鎮南將軍杜預向江陵,龍驤將軍王濬、廣武將軍唐彬浮江東下,太尉賈充為大都督,量宜處要,盡軍勢之中。陶濬至武昌,聞北軍大出,停駐不前。


初,皓每宴會羣臣,無不咸令沈醉。置黃門郎十人,特不與酒,侍立終日,為司過之吏。宴罷之後,各奏其闕失,迕視之咎,謬言之愆,罔有不舉。大者即加威刑,小者輒以為罪。後宮數千,而採擇無已。又激水入宮,宮人有不合意者,輒殺流之。或剥人之面,或鑿人之眼。岑昬險諛貴幸,致位九列,好興功役,衆所患苦。是以上下離心,莫為皓盡力,蓋積惡已極,不復堪命故也。
\gezhu{吳平後,晉侍中庾峻等問皓侍中李仁曰:「聞吳主披人面,刖人足,有諸乎?」仁曰:「以告者過也。君子惡居下流,天下之惡皆歸焉。蓋此事也,若信有之,亦不足能恠。昔唐、虞五刑,三代七辟,肉刑之制,未為酷虐。皓為一國之主,秉殺生之柄,罪人陷法,加之以懲,何足多罪!夫受堯誅者不能無怨,受桀賞者不能無慕,此人情也。」又問曰:「云歸命侯乃惡人橫睛逆視,皆鑿其眼,有諸乎?」仁曰:「亦無此事,傳之者謬耳。曲禮曰視天子由袷以下,視諸侯由頤以下,視大夫由衡,視士則平面,得游目五步之內;視上於衡則傲,下於帶則憂,旁則邪。以禮視瞻,高下不可不慎,況人君乎哉?視人君相迕,是乃禮所謂傲慢;傲慢則無禮,無禮則不臣,不臣則犯罪,犯罪則陷不測矣。正使有之,將有何失?」凡仁所荅,峻等皆善之,文多不悉載。}


四年春,立中山、代等十一王,大赦。濬、彬所至,則土崩瓦解,靡有禦者。預又斬江陵督伍延,渾復斬丞相張悌、丹楊太守沈瑩等,所在戰克。
\gezhu{干寶晉紀曰:吳丞相軍師張悌、護軍孫震、丹楊太守沈瑩帥衆三萬濟江,圍成陽都尉張喬於楊荷橋,衆才七千,閉柵自守,舉白接告降。吳副軍師諸葛靚欲屠之,悌曰:「彊敵在前,不宜先事其小;且殺降不祥。」靚曰:「此等以救兵未至而力少,故且偽降以緩我,非來伏也。因其無戰心而盡阬之,可以成三軍之氣。若舍之而前,必為後患。」悌不從,撫之而進。與討吳護軍張翰、揚州刺史周浚成陣相對。沈瑩領丹楊銳卒刀楯五千,號曰青巾兵,前後屢陷堅陣,於是以馳淮南軍,三衝不動。退引亂,薛勝、蔣班因其亂而乘之,吳軍以次土崩,將帥不能止,張喬又出其後,大敗吳軍于阪橋,獲悌、震、瑩等。}
\gezhu{襄陽記曰:悌字巨先,襄陽人,少有名理,孫休時為屯騎校尉。魏伐蜀,吳人問悌曰:「司馬氏得政以來,大難屢作,智力雖豐,而百姓未服也。今又竭其資力,遠征巴蜀,兵勞民疲而不知恤,敗於不暇,何以能濟?昔夫差伐齊,非不克勝,所以危亡,不憂其本也,況彼之爭地乎!」悌曰:「不然。曹操雖功蓋中夏,威震四海,崇詐杖術,征伐無已,民畏其威,而不懷其德也。丕、叡承之,係以慘虐,內興宮室,外懼雄豪,東西驅馳,無歲獲安,彼之失民,為日乆矣。司馬懿父子,自握其柄,累有大功,除其煩苛而布其平惠,為之謀主而救其疾,民心歸之,亦已乆矣。故淮南三叛而腹心不擾,曹髦之死,四方不動,摧堅敵如折枯,蕩異同如反掌,任賢使能,各盡其心,非智勇兼人,孰能如之?其威武張矣,本根固矣,羣情服矣,姦計立矣。今蜀閹宦專朝,國無政令,而玩戎黷武,民勞卒弊,競於外利,不脩守備。彼彊弱不同,智筭亦勝,因危而伐,殆其克乎!若其不克,不過無功,終無退北之憂,覆軍之慮也,何為不可哉?昔楚劒利而秦昭懼,孟明用而晉人憂,彼之得志,故我之大患也。」吳人笑其言,而蜀果降于魏。晉來伐吳,皓使悌督沈瑩、諸葛靚,率衆三萬渡江逆之。至牛渚,沈瑩曰:「晉治水軍於蜀乆矣,今傾國大舉,萬里齊力,必悉益州之衆浮江而下。我上流諸軍,無有戒備,名將皆死,幼少當任,恐邊江諸城,盡莫能禦也。晉之水軍,必至于此矣!宜畜衆力,待來一戰。若勝之日,江西自清,上方雖壞,可還取之。今渡江逆戰,勝不可保,若或摧喪,則大事去矣。」悌曰:「吳之將亡,賢愚所知,非今日也。吾恐蜀兵來至此,衆心必駭懼,不可復整。今宜渡江,可用決戰力爭。若其敗喪,則同死社稷,無所復恨。若其克勝,則北敵奔走,兵勢萬倍,便當乘威南上,逆之中道,不憂不破也。若如子計,恐行散盡,相與坐待敵到,君臣俱降,無復一人死難者,不亦辱乎!」遂渡江戰,吳軍大敗。諸葛靚與五六百人退走,使過迎悌,悌不肯去,靚自往牽之,謂曰:「且夫天下存亡有大數,豈卿一人所知,如何故自取死為?」悌垂涕曰:「仲思,今日是我死日也。且我作兒童時,便為卿家丞相所拔,常恐不得其死,負名賢知顧。今以身徇社稷,復何遁邪?莫牽曳之如是。」靚流涕放之,去百餘步,已見為晉軍所殺。}
\gezhu{吳錄曰:悌少知名,及處大任,希合時趣,將護左右,清論譏之。}
\gezhu{搜神記曰:臨海松陽人柳榮從悌至楊府,榮病死船中二日,時軍已上岸,無有埋之者,忽然大呼,言「人縛軍師!人縛軍師!」聲激揚,遂活。人問之,榮曰:「上天北斗門下卒見人縛張悌,意中大愕,不覺大呼,言『何以縛張軍師。』門下人怒榮,叱逐使去。榮便去,怖懼,口餘聲發揚耳。」其日,悌戰死。榮至晉元帝時猶在。}


三月丙寅,殿中親近數百人叩頭請皓殺岑昏,皓惶憒從之。
\gezhu{干寶晉紀曰:皓殿中親近數百人叩頭請皓曰:「北軍日近,而兵不舉刃,陛下將如之何!」皓曰:「何故?」對曰:「坐岑昏。」皓獨言:「若爾,當以奴謝百姓。」衆因曰:「唯!」遂並起收昏。皓駱驛追止,已屠之也。}


戊辰,陶濬從武昌還,即引見,問水軍消息,對曰:「蜀船皆小,今得二萬兵,乘大船戰,自足擊之。」於是合衆,授濬節鉞。明日當發,其夜衆悉逃走。而王濬順流將至,司馬伷、王渾皆臨近境。皓用光祿勳薛瑩、中書令胡沖等計,分遣使奉書於濬、伷、渾曰:「昔漢室失統,九州分裂,先人因時,略有江南,遂分阻山川,與魏乖隔。今大晉龍興,德覆四海。闇劣偷安,未喻天命。至於今者,猥煩六軍,衡蓋路次,遠臨江渚,舉國震惶,假息漏刻。敢緣天朝含弘光大,謹遣私署太常張夔等奉所佩印綬,委質請命,惟垂信納,以濟元元。」
\gezhu{江表傳載皓將敗,與舅何植書曰:「昔大皇帝以神武之略,奮三千之卒,割據江南,席卷交、廣,開拓洪基,欲祚之萬世。至孤末德,嗣守成緒,不能懷集黎元,多為咎闕,以違天度。闇昧之變,反謂之祥,致使南蠻逆亂,征討未克。聞晉大衆,遠來臨江,庶竭勞瘁,衆皆摧退,而張悌不反,喪軍過半。孤甚愧悵,于今無聊。得陶濬表云武昌以西並復不守。不守者,非糧不足,非城不固,兵將背戰耳。兵之背戰,豈怨兵邪?孤之罪也。天文縣變於上,士民憤歎於下,觀此事勢,危如累卵,吳祚終訖,何其局哉!天匪亡吳,孤所招也。瞑目黃壤,當復何顏見四帝乎!公其勗勉奇謨,飛筆以聞。」皓又遺羣臣書曰:「孤以不德,忝繼先軌。處位歷年,政教凶勃,遂令百姓乆困塗炭,至使一朝歸命有道,社稷傾覆,宗廟無主,慙愧山積,沒有餘罪。自惟空薄,過偷尊號,才瑣質穢,任重王公,故周易有折鼎之誡,詩人有彼其之譏。自居宮室,仍抱篤疾,計有不足,思慮失中,多所荒替。邊側小人,因生酷虐,虐毒橫流,忠順被害。闇昧不覺,尋其壅蔽,孤負諸君,事已難圖,覆水不可收也。今大晉平治四海,勞心務於擢賢,誠是英俊展節之秋也。管仲極讎,桓公用之,良、平去楚,入為漢臣,舍亂就理,非不忠也。莫以移朝改朔,用損厥志。嘉勗休尚,愛敬動靜。夫復何言,投筆而已!」}


壬申,王濬最先到,於是受皓之降,解縛焚櫬,延請相見。
\gezhu{晉陽秋曰:濬收其圖籍,領州四,郡四十三,縣三百一十三,戶五十二萬三千,吏三萬二千,兵二十三萬,男女口二百三十萬,米穀二百八十萬斛,舟船五千餘艘,後宮五千餘人。}
伷以皓致印綬於己,遣使送皓。皓舉家西遷,以太康元年五月丁亥集于京邑。四月甲申,詔曰:「孫皓窮迫歸降,前詔待之以不死,今皓垂至,意猶愍之,其賜號為歸命侯。進給衣服車乘,田三十頃,歲給穀五千斛,錢五十萬,絹五百匹,緜五百斤。」皓太子瑾拜中郎,諸子為王者,拜郎中。
\gezhu{搜神記曰:吳以草創之國,信不堅固,邊屯守將皆質其妻子,名曰保質。童子少年,以類相與嬉游者,日有十數。永安二年三月,有一異兒,長四尺餘,年可六七歲,衣青衣,來從羣兒戲,諸兒莫之識也。皆問曰:「爾誰家小兒,今日忽來?」荅曰:「見爾羣戲樂,故來耳。」詳而視之,眼有光芒,爚爚外射。諸兒畏之,重問其故。兒乃荅曰:「爾惡我乎?我非人也,乃熒惑星也。將有以告爾:三公鉏,司馬如。」諸兒大驚,或走告大人,大人馳往觀之。兒曰:「舍爾去乎!」竦身而躍,即以化矣。仰而視之,若引一匹練以登天。大人來者,猶及見焉,飄飄漸高,有頃而沒。時吳政峻急,莫敢宣也。後五年而蜀亡,六年而晉興,至是而吳滅,司馬如矣。}
\gezhu{干寶晉紀曰:王濬治船於蜀,吳彥取其流柹以呈孫皓,曰:「晉必有攻吳之計,宜增建平兵。建平不下,終不敢渡江。」皓弗從。陸抗之克步闡,皓意張大,乃使尚廣筮并天下,遇同人之頤,對曰:「吉。庚子歲,青蓋當入洛陽。」故皓不脩其政,而恒有窺上國之志。是歲也,實在庚子。}
五年,皓死于洛陽。
\gezhu{吳錄曰:皓以四年十二月死,時年四十二,葬河南縣界。}


評曰:孫亮童孺而無賢輔,其替位不終,必然之勢也。休以舊愛宿恩,任用興、布,不能拔進良才,改絃易張,雖志善好學,何益救亂乎?又使旣廢之亮不得其死,友于之義薄矣。皓之淫刑所濫,隕斃流黜者,蓋不可勝數。是以羣下人人惴恐,皆日日以兾,朝不謀夕。其熒惑、巫祝,交致祥瑞,以為至急。昔舜、禹躬稼,至聖之德,猶或矢誓衆臣,予違女弼,或拜昌言,常若不及。況皓凶頑,肆行殘暴,忠諫者誅,讒諛者進,虐用其民,窮淫極侈,宜腰首分離,以謝百姓。旣蒙不死之詔,復加歸命之寵,豈非曠蕩之恩,過厚之澤也哉!


\gezhu{孫盛曰:夫古之立君,所以司牧羣黎,故必仰協乾坤,覆燾萬物;若乃淫虐是縱,酷被羣生,則天殛之,勦絕其祚,奪其南面之尊,加其獨夫之戮。是故湯、武抗鉞,不犯不順之譏;漢高奮劒,而無失節之議。何者?誠四海之酷讎,而人神之所擯故也。況皓罪為逋寇,虐過辛、癸,梟首素旗,猶不足以謝冤魂,洿室荐社,未足以紀暴迹,而乃優以顯命,寵錫仍加,豈龔行天罰,伐罪弔民之義乎?是以知僭逆之不懲,而凶酷之莫戒。詩云:「取彼譖人,投畀豺虎。」聊譖猶然,矧僭虐乎?且神旗電掃,兵臨偽窟,理窮勢迫,然後請命,不赦之罪旣彰,三驅之義又塞,極之權道,亦無取焉。}
\gezhu{陸機著辨亡論,言吳之所以亡,其上篇曰:「昔漢氏失御,姦臣竊命,禍基京畿,毒徧宇內,皇綱弛紊,王室遂卑。於是羣雄蜂駭,義兵四合,吳武烈皇帝慷慨下國,電發荊南,權略紛紜,忠勇伯世。威稜則夷羿震蕩,兵交則醜虜授馘,遂掃清宗祊,蒸禋皇祖。於時雲興之將帶州,飈起之師跨邑,哮闞之羣風驅,熊羆之族霧集,雖兵以義合,同盟勠力,然皆包藏禍心,阻兵怙亂,或師無謀律,喪威稔寇,忠規武節,未有若此其著者也。武烈旣沒,長沙桓王逸才命世。弱冠秀發,招擥遺老,與之述業。神兵東驅,奮寡犯衆,攻無堅城之將,戰無交鋒之虜。誅叛柔服而江外厎定,飭法脩師而威德翕赫,賔禮名賢而張昭為之雄,交御豪俊而周瑜為之傑。彼二君子,皆弘敏而多奇,雅達而聦哲,故同方者以類附,等契者以氣集,而江東蓋多士矣。將北伐諸華,誅鉏干紀,旋皇輿於夷庚,反帝座于紫闥,挾天子以令諸侯,清天步而歸舊物。戎車旣次,羣凶側目,大業未就,中世而隕。用集我大皇帝,以奇蹤襲於逸軌,叡心發乎令圖,從政咨於故實,播憲稽乎遺風,而加之以篤固,申之以節儉,疇咨俊茂,好謀善斷,束帛旅於丘園,旌命交于塗巷。故豪彥尋聲而響臻,志士希光而影騖,異人輻湊,猛士如林。於是張昭為師傅,周瑜、陸公、魯肅、呂蒙之疇入為腹心,出作股肱;甘寧、淩統、程普、賀齊、朱桓、朱然之徒奮其威,韓當、潘璋、黃蓋、蔣欽、周泰之屬宣其力;風雅則諸葛瑾、張承、步隲以聲名光國,政事則顧雍、潘濬、呂範、呂岱以器任幹職,奇偉則虞翻、陸績、張溫、張惇以諷議舉正,奉使則趙咨、沈珩以敏達延譽,術數則吳範、趙達以禨祥協德,董襲、陳武殺身以衞主,駱統、劉基彊諫以補過,謀無遺筭,舉不失策。故遂割據山川,跨制荊、吳,而與天下爭衡矣。魏氏嘗藉戰勝之威,率百萬之師,浮鄧塞之舟,下漢陰之衆,羽楫萬計,龍躍順流,銳騎千旅,虎步原隰,謀臣盈室,武將連衡,喟然有吞江滸之志,一宇宙之氣。而周瑜驅我偏師,黜之赤壁,喪旗亂轍,僅而獲免,收迹遠遁。漢王亦馮帝王之號,率巴、漢之民,乘危騁變,結壘千里,志報關羽之敗,圖收湘西之地。而我陸公亦挫之西陵,覆師敗績,困而後濟,絕命永安。續以濡須之寇,臨川摧銳,蓬籠之戰,孑輪不反。由是二邦之將,喪氣挫鋒,勢衄財匱,而吳藐然坐乘其弊,故魏人請好,漢氏乞盟,遂躋天號,鼎峙而立。西屠庸蜀之郊,北裂淮漢之涘,東苞百越之地,南括羣蠻之表。於是講八代之禮,蒐三王之樂,告類上帝,拱揖羣后。虎臣毅卒,循江而守,長戟勁鎩,望飇而奮。庶尹盡規於上,四民展業于下,化協殊裔,風衍遐圻。乃俾一介行人,撫巡外域,臣象逸駿,擾於外閑,明珠瑋寶,輝於內府,珍瑰重跡而至,奇玩應響而赴,輶軒騁於南荒,衝輣息於朔野,齊民免干戈之患,戎馬無晨服之虞,而帝業固矣。大皇旣歿,幼主莅朝,姦回肆虐。景皇聿興,虔修遺憲,政無大闕,守文之良主也。降及歸命之初,典刑未滅,故老猶存。大司馬陸公以文武熈朝,左丞相陸凱以謇諤盡規,而施績、范慎以威重顯,丁奉、鍾離斐以武毅稱,孟宗、丁固之徒為公卿,樓玄、賀劭之屬掌機事,元首雖病,股肱猶良。爰及末葉,羣公旣喪,然後黔首有瓦解之志,皇家有土崩之釁,歷命應化而微,王師躡運而發,卒散於陣,民奔于邑,城池無藩籬之固,山川無溝阜之勢,非有工輸雲梯之械,智伯灌激之害,楚子築室之圍,燕人濟西之隊,軍未浹辰而社稷夷矣。雖忠臣孤憤,烈士死節,將奚救哉?夫曹、劉之將非一世之選,向時之師無曩日之衆,戰守之道抑有前符,險阻之利俄然未改,而成敗貿理,古今詭趣,何哉?彼此之化殊,授任之才異也。」}
\gezhu{其下篇曰:「昔三方之王也,魏人據中夏,漢氏有岷、益,吳制荊、揚而奄交、廣。曹氏雖功濟諸華,虐亦深矣,其民怨矣。劉公因險飾智,功已薄矣,其俗陋夫。吳桓王基之以武,太祖成之以德,聦明睿達,懿度深遠矣。其求賢如不及,恤民如稚子,接士盡盛德之容,親仁罄丹府之愛。拔呂蒙於戎行,識潘濬於係虜。推誠信士,不恤人之我欺;量能授器,不患權之我逼。執鞭鞠躬,以重陸公之威;悉委武衞,以濟周瑜之師。卑宮菲食,以豐功臣之賞;披懷虛己,以納謨士之筭。故魯肅一面而自託,士燮蒙險而效命。高張公之德而省游田之娛,賢諸葛之言而割情欲之歡,感陸公之規而除刑政之煩,奇劉基之議而作三爵之誓,屏氣跼蹐以伺子明之疾,分滋損甘以育淩統之孤,登壇慷慨歸魯肅之功,削投惡言信子瑜之節。是以忠臣競盡其謀,志士咸得肆力,洪規遠略,固不厭夫區區者也。故百官苟合,庶務未遑。初都建業,羣臣請備禮秩,天子辭而不許,曰:『天下其謂朕何!』宮室輿服,蓋慊如也。爰及中葉,天人之分旣定,百度之缺粗修,雖醲化懿綱,未齒乎上代,抑其體國經民之具,亦足以為政矣。地方幾萬里,帶甲將百萬,其野沃,其民練,其財豐,其器利,東負滄海,西阻險塞,長江制其區宇,峻山帶其封域,國家之利,未見有弘於茲者矣。借使中才守之以道,善人御之有術,敦率遺憲,勤民謹政,循定策,守常險,則可以長世永年,未有危亡之患。或曰,吳、蜀脣齒之國,蜀滅則吳亡,理則然矣,夫蜀蓋藩援之與國,而非吳人之存亡也。何則?其郊境之接,重山積險,陸無長轂之徑;川阨流迅,水有驚波之艱。雖有銳師百萬,啟行不過千夫;軸艫千里,前驅不過百艦。故劉氏之伐,陸公喻之長虵,其勢然也。昔蜀之初亡,朝臣異謀,或欲積石以險其流,或欲機械以御其變。天子緫羣議而諮之大司馬陸公,陸公以四瀆天地之所以節宣其氣,固無可遏之理,而機械則彼我之所共,彼若棄長伎以就所屈,即荊、揚而爭舟楫之用,是天贊我也,將謹守峽口以待禽耳。逮步闡之亂,憑保城以延彊寇,重資幣以誘羣蠻。于時大邦之衆,雲翔電發,縣旌江介,築壘遵渚,襟帶要害,以止吳人之西,而巴漢舟師沿江東下。陸公以偏師三萬,北據東坑,深溝高壘,案甲養威。反虜踠跡待戮,而不敢北闚生路,彊寇敗績宵遁,喪師大半,分命銳師五千,西禦水軍,東西同捷,獻俘萬計。信哉賢人之謀,豈欺我哉!自是烽燧罕警,封域寡虞。陸公沒而潛謀兆,吳釁深而六師駭。夫太康之役,衆未盛乎曩日之師,廣州之亂,禍有愈乎向時之難,而邦家顛覆,宗廟為墟。嗚呼!人之云亡,邦國殄瘁,不其然與!易曰『湯武革命順乎天』,玄曰『亂不極則治不形』,言帝王之因天時也。古人有言,曰『天時不如地利』,易曰『王侯設險以守其國』,言為國之恃險也。又曰:『地利不如人和』,『在德不在險』,言守險之由人也。吳之興也,參而由焉,孫卿所謂合其參者也。及其亡也,恃險而已,又孫卿所謂舍其參者也。夫四州之氓非無衆也,大江之南非乏俊也,山川之嶮易守也,勁利之器易用也,先政之業易循也,功不興而禍遘者何哉?所以用之者失也。故先王達經國之長規,審存亡之至數,恭己以安百姓,敦惠以致人和,寬沖以誘俊乂之謀,慈和以結士民之愛。是以其安也,則黎元與之同慶;及其危也,則兆庶與之共患。安與衆同慶,則其危不可得也;危與下共患,則其難不足卹也。夫然,故能保其社稷而固其土宇,麥秀無悲殷之思,黍離無愍周之感矣。」}


\end{pinyinscope}