\article{卷五十七雜傳第四十五}

\begin{pinyinscope}

 李崧李崧,深州饒陽人也。崧幼聰敏,能文章,為鎮州參軍。唐魏王繼岌為興聖宮使,領鎮州節度使,以推官李蕘掌書記。崧謂掌書呂柔曰:「魏王皇子,天下之望,書奏之職,非蕘所當。」柔私使崧代為之,以示盧質、馮道,道等皆以為善。乃以崧為興聖宮巡官,拜協律郎。繼岌與郭崇韜代蜀,以崧掌書記。繼岌已破蜀,劉皇后聰讒者言,陰遣人之蜀,教繼岌殺崇韜,人情不安。崧入見繼岌曰:「王何
 為作此危事?誠不能容崇韜,至洛誅之何晚?今遠軍五千里,不見咫尺之詔殺大臣,動搖人情,是召亂也。」繼岌曰:「吾亦悔之,奈何?」崧乃召書吏三四人,登樓去梯,夜以黃紙作詔書,倒用都統印,明旦告諭諸軍,人心乃定。



 師還,繼岌死於道。崧至京師,任圜判三司,用崧為鹽鐵判官,以內憂去職還鄉里。服除,范延光居鎮州,辟崧掌書記。延光為樞密使,崧拜拾遺,直樞密院。



 累遷戶部侍郎、端明殿學士。長興中,明宗春秋高,秦王從榮多不法,晉高祖為六軍副使,懼禍及,求出外籓。是時,契丹入鴈門,明宗選將以捍太原,晉高祖欲之。



 樞密使范延光、趙
 延壽等議將,久不決,明宗怒甚,責延壽等,延壽等惶恐,欲以康義誠應選,崧獨曰:「太原,國之北門,宜得重臣,非石敬瑭不可也!」由是從崧議。晉高祖深德之,陰遣人謝崧曰:「為浮屠者,必合其尖。」蓋欲使崧終始成己事也。其後晉高祖以兵入京師,崧竄匿伊闕民家,晉高祖召為戶部侍郎,拜中書侍郎、同中書門下平章事兼樞密使。丁內艱,起復。



 高祖崩,出帝即位,以崧兼判三司,與馮玉對掌樞密。是時,晉兵敗契丹於陽城,趙延壽在幽州,詐言思歸以誘晉兵,崧等信之。初,漢高祖在晉,掌親軍,為侍衛都指揮使,與杜重威同制加平章事,漢高祖恥之,
 怒不肯謝,晉高祖遣和凝諭之,乃謝。其後漢高祖出居太原,重威代為侍衛使,崧亦數稱重威之材,於是漢高祖以崧為排己,深恨之。崧又信延壽之詐以為然,卒以重威將大兵,其後敗于中渡,晉遂以亡。



 契丹耶律德光犯京師,德光素聞延壽等稱崧為人,及入京師,謂人曰:「吾破南朝,得崧一人而已!」乃拜崧太子太師。契丹北還,命崧以族俱行,留之鎮州。



 其後麻荅棄鎮州,崧與馮道等得還。高祖素不悅崧,又為怨者譖之,言崧為契丹所厚,故崧遇漢權臣,常惕惕為謙謹,莫敢有所忤。



 漢高祖入京師,以崧第賜蘇逢吉,崧家遭亂,多埋金寶,逢吉
 悉有之。而崧弟嶼、義與逢吉子弟同舍,酒酣,出怨言,以為奪我第。崧又以宅券獻逢吉,逢吉尤不喜。漢法素嚴,楊邠、史弘肇多濫弄法。嶼僕葛延遇為嶼商賈,多乾沒其貲,嶼笞責之。延遇夜宿逢吉部曲李澄家,以情告澄。是時,高祖將葬睿陵,河中李守貞反。澄乃教延遇告變,言崧與其甥王凝謀因山陵放火焚京師,又以蠟丸書通守貞。



 逢吉遣人召崧至第,從容告之,崧知不免,乃以幼女託逢吉。逢吉送崧侍衛獄。崧出乘馬,從者去,無一人,崧恚曰:「自古豈有不死之人,然亦豈有不亡之國乎!」



 乃自誣伏,族誅。



 崧素與翰林學士徐台符相善,後周太
 祖入立,台符告宰相馮道,請誅葛延遇,道以延遇數經赦宥,難之。樞密使王峻聞之,多台符有義,乃奏誅延遇。



 李鏻李鏻,唐宗室子也。其伯父陽事唐,咸通間為給事中。鏻少舉進士,累不中,客河朔間,自稱清海軍掌書記,謁定州王處直,處直不為禮。乃易其綠衣,更為緋衣,謁常山李弘規,弘規進之趙王王鎔,鎔留為從事。其後張文禮弒鎔自立,遣鏻聘唐莊宗於太原。鏻為人利口敢言,乃陰為莊宗畫文禮可破之策。後文禮敗,莊宗以鏻為支使。



 莊宗即位,拜鏻宗正卿,以李瓊為少卿。獻祖、懿祖墓在趙州昭慶縣,唐國初建,鏻、瓊上言:「獻祖宣皇帝建初
 陵,懿祖光皇帝啟運陵,請置臺令。」縣中無賴子自稱宗子者百餘人,宗正無譜牒,莫能考按。有民詣寺自言世為丹陽竟陵臺令,厚賂宗正吏,鏻、瓊不復詳考,遂補為令。民即持絳幡招置部曲,侵奪民田百餘頃,以謂陵園需地。民訴於官,不能決,以聞。莊宗下公卿博士,問故唐諸帝陵寢所在。公卿博士言:「丹陽在今潤州,而竟陵非唐事。鏻不學無知,不足以備九卿。」



 坐貶司農少卿,出為河中節度副使。



 明宗即位,以鏻故人,召還,累遷戶部尚書。鏻意頗希大用,嘗謂馮道、趙鳳曰:「唐家故事,宗室皆為宰相。今天祚中興,宜按舊典,鏻雖不才,嘗事莊宗霸
 府,識今天子於籓邸,論才較業,何後眾人?而久置班行,於諸君安乎?」道等惡其言。後楊溥諜者見鏻言事,鏻謂安重誨曰:「楊溥欲歸國久矣,若朝廷遣使諭之,可以召也。」重誨信之,以玉帶與諜者使為信,久而無效,由是貶鏻兗州行軍司馬。



 鏻與廢帝有舊,愍帝時,為兵部尚書,奉使湖南,聞廢帝立,喜,以謂必用己為相。還過荊南,謂高從誨曰:「士固有否泰,吾不為時用久矣。今新天子即位,我將用矣!」乃就從誨求寶貨入獻以為賀,從誨與馬紅裝拂二、猓犬然皮一,因為鏻置酒,問其副使馬承翰:「今朝廷之臣,孰有公輔之望?」承翰曰:「尚書崔居儉、左丞姚
 顗,其次太常盧文紀也。」從誨笑顧左右,取進奏官報狀示鏻顗與文紀皆拜平章事矣。鏻慚失色。還,遂獻其皮、拂,廢帝終不用。



 初,李愚自太常卿作相,而盧文紀代之,及文紀作相,鏻乃求為太常卿。及拜命,中謝曰:「臣叨入相之資。」朝士傳以為笑。



 鏻事晉累遷太子太保。漢高祖即位,拜鏻司徒,居數月卒,年八十八,贈太傅。



 賈緯賈緯,鎮州獲鹿人也。少舉進士不中,州辟參軍。唐天成中,范延光鎮成德,辟趙州軍事判官,遷石邑令。緯長於史學。唐自武宗已後無實錄,史官之職廢,緯採次傳聞,為《唐年補錄》六十五卷。當唐之末,王室微弱,諸侯強盛,
 征伐擅出,天下多事,故緯所論次多所闕誤。而喪亂之際,事迹粗存,亦有補於史氏。晉天福中,為太常博士,非其好也,數求為史職,改屯田員外郎、起居郎、史館脩撰,與脩《唐書》。丁內艱,服除,知制誥。累遷中書舍人、諫議大夫、給事中,復為脩撰。漢隱帝時,詔與王伸、竇儼等同脩晉高祖、出帝、漢高祖實錄。初,桑維翰為相,常惡緯為人,待之甚薄。緯為維翰傳,言「繼翰死,有銀八千鋌。」翰林學士徐台符以為不可,數以非緯,緯不得已,更為數千鋌。廣順元年,實錄成,緯求遷官不得,由是怨望。是時,宰相王峻監脩國史,緯書日歷,多言當時大臣過失,峻見之,
 怒曰:「賈給事子弟仕宦亦要門閥,奈何歷詆當朝之士,使其子孫何以仕進?」



 言之太祖,貶平盧軍行軍司馬。明年卒于青州。



 段希堯段希堯,河內人也。晉高祖為河東節度使,以希堯為判官。高祖軍屯忻州,軍中有擁高祖呼萬歲者,高祖惶惑,不知所為。希堯勸高祖斬其亂首,乃止。高祖將舉兵太原,與其賓佐謀,希堯以為不可,高祖雖不聽,然重其為人,不責之也。高祖入立,希堯比諸將吏,恩澤最薄。久之,稍遷諫議大夫,使于吳越。是時,江、淮不通,凡使吳越者皆泛海,而多風波之患。希堯過海,遭大風,左右皆恐
 懼,希堯曰:「吾平生不欺,汝等恃吾,可無恐也。」已而風亦止。歷萊、懷、棣三州刺史。出帝時,為吏部侍郎,判東、西銓事,累遷禮部尚書。卒,年七十九,贈太子少保。



 張允張允,鎮州人也。少事鎮州為張文禮參軍。唐莊宗討張文禮,允脫身降,莊宗系之獄,文禮敗,乃出之為魏州功曹。趙在禮辟節度推官,歷滄、兗二鎮掌書記。



 入為監察御史,累遷水部員外郎,知制誥。廢帝皇子重美為河南尹,掌六軍,以允剛介,乃拜允給事中,為六軍判官。罷,遷左散騎常侍。晉高祖即位,屢赦天下,允為《駮赦論》以獻曰:「管子曰:『凡赦者小利而大害,久而不勝其禍;無赦者
 小害而大利,久而不勝其福。』又漢之吳漢疾篤,帝問漢所欲言。漢曰:『惟願陛下無赦爾!』蓋行赦不以為恩,不行赦不以為無恩,罰有罪故也。自古皆以水旱則降德音而宥過,開狴牢而出囚,冀感天心以救其災者,非也。假有二人之訟者,一有罪而一無罪,若有罪者見捨,則無罪者銜冤。此乃致災之道,非救災之術也。至使小人遇天災,則皆喜而相勸以為惡,曰:『國將赦矣,必捨我以救災。』如此,則是教民為惡也。夫天之為道,福善而禍淫。若捨惡人而變災為福,則是天又喜人為惡也。凡天之降災,所以警戒人主節嗜慾,務勤儉,恤鰥寡,正刑罰而已。」
 是時,晉高祖方好臣下有言,覽之大喜。允事漢為吏部侍郎,隱帝誅戮大臣,京師皆恐,允常退朝不敢還家,止于相國寺。周太祖以兵入京師,允匿于佛殿承塵,墜而卒,年六十五。



 王松王松,父徽,為唐僖宗宰相。松舉進士,後唐時,歷刑部郎中,唐末,從事方鎮。晉高祖鎮太原,辟松節度判官。晉高祖即位,拜右諫議大夫,累拜工部尚書。



 出帝北遷,蕭翰立許王從益於京師,以松為左丞相。漢高祖入洛,先遣人馳詔東京百官嘗授偽命者皆焚之,使勿自疑,由是御史臺悉斂百官偽敕焚之。松以手指其胸,引郭子儀
 自誚,以語人曰:「此乃二十四考中書令也。」聞者笑之。後松子仁寶為李守貞河中支使,守貞反,松以子故上書自陳,高祖憐之,但使解職而已。松有田城東,歲時往來京師,以疾卒。



 裴皞裴皞,字司東,河東人也。裴氏自晉、魏以來,世為名族,居燕者號「東眷」,居涼者號「西眷」,居河東者號「中眷」。皞出於名家,而容止端秀,性剛急,直而無隱。少好學,唐光化中舉進士,拜校書郎、拾遺、補闕。事梁為翰林學士、中書舍人。事後唐為禮部侍郎。皞喜論議,每陳朝廷闕失,多斥權臣。改太子賓客,以老拜兵部尚書致仕。晉高祖起
 為工部尚書,復以老告,拜右僕射致仕。卒,年八十五,贈太子太保。



 皞以文學在朝廷久,宰相馬胤孫、桑維翰,皆皞禮部所放進士也。後胤孫知舉,放榜,引新進士詣皞,皞喜作詩曰:「門生門下見門生。」世傳以為榮。維翰已作相,嘗過皞,皞不迎不送。人或問之,皞曰:「我見桑公於中書,庶寮也;桑公見我於私第,門生也。何送迎之有?」人亦以為當。



 王仁裕王仁裕,字德輦,天水人也。少不知書,以狗馬彈射為樂,年二十五始就學,而為人俊秀,以文辭知名秦、隴間。秦帥辟為秦州節度判官。秦州入于蜀,仁裕因事蜀為中書舍人、
 翰林學士。唐莊宗平蜀,仁裕事唐,復為秦州節度判官。王思同鎮興元,辟為從事。思同留守西京,以為判官。廢帝舉兵鳳翔,思同戰敗,廢帝得仁裕,聞其名不殺,置之軍中。自廢帝起事,至其入立,馳檄諸鎮,詔書、告命皆仁裕為之。久之,以都官郎中充翰林學士。晉高祖入立,罷職為郎中,歷司封左司郎中、諫議大夫。漢高祖時,復為翰林學士承旨,累遷戶部尚書,罷為兵部尚書、太子少保。顯德三年卒,年七十七,贈太子少師。



 仁裕性曉音律,晉高祖初定雅樂,宴群臣於永福殿,奏黃鐘,仁裕聞之曰:「音不純肅而無和聲,當有爭者起於禁中。」已而兩軍
 校斗升龍門外,聲聞於內,人以為神。喜為詩。其少也,嘗夢剖其腸胃,以西江水滌之,顧見江中沙石皆為篆籀之文,由是文思益進。乃集其平生所作詩萬餘首為百卷,號《西江集》。仁裕與和凝於五代時皆以文章知名,又嘗知貢舉,仁裕門生王溥、凝門生范質,皆至宰相,時稱其得人。



 裴羽裴羽,字用化,其父贄,相唐僖宗,官至司空。羽以一品子為河南壽安尉。事梁為御史臺主簿,改監察御史。唐明宗時,為吏部郎中,與右散騎常侍陸崇使于閩,為海風所飄至錢塘。是時,吳越王錢鏐與安重誨有隙,唐方絕
 鏐朝貢,羽等被留經歲,而崇以疾卒。後鏐遣羽還,羽求載崇尸與俱歸。鏐初不許,羽以語感動鏐,鏐惻然許之,因附羽表自歸。明宗得鏐表大喜,由是吳越復通於中國。羽護崇喪至京師,及其橐裝還其家,士人皆多羽之義。羽,周太祖時為左散騎常侍,卒,贈戶部尚書。



 王延王延,字世美,鄭州長豐人也。少好學,嘗以賦謁梁相李琪,琪為之稱譽,薦為即墨縣令。馮道作相,與延故人,召拜左補闕。遷水部員外郎,知制誥。拜中書舍人,權知貢舉。吏部尚書盧文紀與故相崔協有隙。是時,協子頎方舉進士,文紀謂延曰:「吾嘗譽子于朝,貢舉選士,當求實
 效,無以虛名取人。昔有越人善泅,生子方晬,其母浮之水上。人怪而問之,則曰:『其父善泅,子必能之。』若是可乎?」延退而笑曰:「盧公之言,為崔協也,恨其父遂及其子邪!」明年,選頎甲科,人皆稱其公。累遷刑部尚書,以太子少保致仕。卒,年七十三。



 延為人重然諾,與其弟規相友愛,五代之際,稱其家法焉。



 馬重績馬重績,字洞微,其先出於北狄,而世事軍中。重績少學數術,明太一、五紀、八象、《三統大歷》,居于太原。唐莊宗鎮太原,每用兵征伐,必以問之,重績所言無不中,拜大理司直。明宗時,廢不用。晉高祖以太原拒命,廢帝遣兵圍
 之,勢甚危急,命重績筮之,遇《同人》,曰:「天火之象,乾建而離明。健者君之德也,明者南面而嚮之,所以治天下也。同人者人所同也,必有同我者焉。《易》曰:『戰乎乾。』乾,西北也。又曰:『相見乎離。』離,南方也。其同我者自北而南乎?乾,西北也,戰而勝,其九月十月之交乎?」是歲九月,契丹助晉擊敗唐軍,晉遂有天下。拜重績太子右贊善大夫,遷司天監。明年,張從賓反,命重績筮之,遇《隨》,曰:「南瞻析木,木不自續,虛而動之,動隨其覆。歲將秋矣,無能為也!」七月而從賓敗。高祖大喜,賜以良馬、器幣。



 天福三年,重績上言:「歷象,王者所以正一氣之元,宣萬邦之命。而古今
 所紀,考審多差,《宣明》氣朔正而星度不驗,《崇玄》五星得而歲差一日,以《宣明》之氣朔,合《崇玄》之五星,二歷相參,然後符合。自前世諸歷,皆起天正十一月為歲首,用太古甲子為上元,積歲愈多,差闊愈甚。臣輒合二歷,創為新法,以唐天寶十四載乙未為上元,雨水正月中氣為氣首。」詔下司天監趙仁錡、張文皓等考覈得失。仁錡等言:「明年庚子正月朔,用重績歷考之,皆合無舛。」乃下詔班行之,號《調元歷》。行之數歲輒差,遂不用。重績又言:「漏刻之法,以中星考晝夜為一百刻,八刻六十分刻之二十為一時,時以四刻十分為正,此自古所用也。



 今失其
 傳,以午正為時始,下侵未四刻十分而為午。由是晝夜昏曉,皆失其正,請依古改正。」從之。重績卒年六十四。



 趙延義趙延義,字子英,秦州人也。曾祖省躬通數術,避亂於蜀。父溫珪,事蜀王建為司天監,每為建占吉兇,小不中,輒加詰責。溫珪臨卒,戒其子孫曰:「數術,吾世業,然吾仕亂國,得罪而幾死者數矣!子孫能以佗道仕進者,不必為也。」然延義少亦以此仕蜀為司天監。蜀亡,仕唐為星官。延義兼通三式,頗善相人。契丹滅晉,延義隨虜至鎮州。李筠、白再榮謀逐麻答歸漢,猶豫未決,延義假述數術贊成之。周太祖自魏以兵入京師,太祖召延義問:「漢祚
 短促者,天數邪?」延義言:「王者撫天下,當以仁恩德澤,而漢法深酷,刑罰枉濫,天下稱冤,此其所以亡也!」



 是時,太祖方以兵圍蘇逢吉、劉銖第,欲誅其族,聞延義言悚然,因貸其族,二家獲全。延義事周為太府卿、判司天監,以疾卒。



\end{pinyinscope}