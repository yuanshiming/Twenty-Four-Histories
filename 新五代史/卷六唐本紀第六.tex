\article{卷六唐本紀第六}

\begin{pinyinscope}

 明宗
 聖德和武欽孝皇帝,世本夷狄,無姓氏。父霓,為雁門部將,生子邈佶烈,以騎射事太祖,為人質厚寡言,執事恭謹,太祖養以為子,賜名嗣源。



 梁攻兗、鄆,朱宣、朱瑾來乞師,太祖遣李存信將兵三萬救之。存信留莘縣不進,使嗣源別以兵三乾先擊梁兵,梁兵解去。存信留莘縣久之,為羅弘信所襲,存信敗走,嗣源獨殿而還,太祖以嗣源所將騎五百號「橫衝都」。光化三年,李嗣昭攻梁
 邢、洺,出青山,遇葛從周兵,嗣昭大敗走,梁兵追之。嗣源從間道後至,謂嗣昭曰:「為公一戰。」乃解鞍礪鏃,憑高為陣,左右指畫,梁追兵望之莫測。嗣源急呼曰:「吾取葛公,士卒可無動!」乃馳騎犯之,出入奮擊。嗣昭繼進,梁兵解去。嗣源身中四矢,太祖解衣賜藥以勞之,由是李橫衝名重四方。



 梁、晉相拒於柏鄉,梁龍驤軍以赤、白馬為兩陣,旗幟鎧仗皆如馬色,晉兵望之皆懼。莊宗舉鐘以飲嗣源曰:「卿望梁家赤、白馬懼乎?雖吾亦怯也。」嗣源笑曰:「有其表爾,翌日歸吾廄也。」莊宗大喜曰:「卿當以氣吞之。」因引鐘飲酹,奮楇馳騎,犯其白馬,挾二裨將而還。梁兵
 敗,以功拜代州刺史。



 莊宗攻劉守光,嗣源及李嗣昭將兵三萬別出飛狐,定山後,取武、媯、儒三州。



 莊宗已平魏州,因徇下磁、相,拜相州刺史、昭德軍節度使。久之,徙鎮安國。契丹攻幽州,莊宗遣嗣源與閻寶等擊走之。



 同光元年,徙鎮橫海。是時,梁、唐相拒于河上,李繼韜以潞州叛降梁,莊宗有憂色,召嗣源帳中,謂曰:「繼韜以上黨降梁,而梁方急攻澤州,吾出不意襲鄆州,以斷梁右臂,可乎?」嗣源對曰:「夾河之兵久矣,茍非出奇,則大計不決,臣請獨當之。」乃以步騎五千涉濟,至鄆州,鄆人無備,遂襲破之,即拜天平軍節度使、蕃漢馬步軍副都總管。



 梁
 軍攻破德勝南柵,莊宗退保楊劉。王彥章急攻鄆州,莊宗悉軍救之,嗣源為前鋒,擊梁軍。追至中都,擒彥章及梁監軍張漢傑。彥章雖敗,而段凝悉將梁兵屯河上,莊宗未知所嚮,諸將多言乘勝以取青、齊,嗣源曰:「彥章之敗,凝猶未知,使其聞之,遲疑定計,亦須三日。縱使料吾所向,亟發救兵,必渡黎陽,數萬之眾,舟楫非一日具也。此去汴州,不數百里,前無險阻,方陣而行,信宿可至,汴州已破,段凝豈足顧哉!」而郭崇韜亦勸莊宗入汴,莊宗以為然,遣嗣源以千騎先至汴州,攻封丘門,王瓚開門降。莊宗後至,見嗣源大喜,手攬其衣,以頭觸之曰:「天
 下與爾共之。」拜中書令。



 二年,莊宗祀天南郊,賜以鐵券。五月,破楊立於潞州。六月,徙鎮宣武,兼蕃漢內外馬步軍總管。冬,契丹侵漁陽,嗣源敗之于涿州。



 三年,徙鎮成德。莊宗幸鄴,請朝行在,不許。貞簡太后疾,請入省,又不許。



 太后崩,請赴山陵,許之,而契丹侵邊,乃止。十二月,遂朝于洛陽。



 天成元年,郭崇韜、朱友謙皆以讒死,嗣源以名位高,亦見疑忌。趙在禮反於魏,大臣皆請遣嗣源討賊,莊宗不許。群臣屢請,莊宗不得已,遣之。三月壬子,嗣源至
 魏,屯御河南,在禮登樓謝罪。甲寅,軍變,嗣源入于魏,與在禮合,夕出,止魏縣。丁巳,以其兵南,遣石敬瑭將三百騎為先鋒。嗣源行過鉅鹿,掠小坊馬二千匹以益軍。壬申,入汴州。



 四月丁亥,莊宗崩。己丑,入洛陽。甲午,監國,朝群臣于興聖宮。乙未,中門使安重誨為樞密使。殺元行欽及租庸使孔謙。壬寅,左驍衛大將軍孔循為樞密使。



 丙午,始奠於西宮,皇帝即位於柩前,易斬縗以袞冕。壬
 子,魏王繼岌薨。甲寅,大赦,改元。渤海國王大諲撰使大陳林來。是月,張居翰罷。



 五月丙辰朔,太子賓客鄭玨、工部尚書任圜為中書侍郎:同中書門下平章事。



 戊辰,趙在禮為義成軍節度使。六月丁酉,汴州控鶴軍亂,指揮使張諫殺其權知州事高逖。己亥,諫伏誅。秋七月庚申,安重誨殺殿直馬延于御史臺門。契丹使梅老述骨來,渤海使大昭佐來。己卯,貶豆盧革為辰州刺史,韋說敘州刺史。甲申,流革于陵州,說于合州。八月乙酉朔,陜州硤石縣民高存妻一產三男子。丁
 酉,以象笏三十二賜百官之無笏者。閱稼於冷泉宮。己亥,契丹寇邊。丁未,平盧軍節度使霍彥威殺其登州刺史王公儼。甲寅,醫官張志忠為太原少尹。九月己未,幸至德宮及袁建豐第。冬十月丁亥,雲南山後兩林百蠻都鬼主、右武衛大將軍李卑晚使大鬼主傅能何華來。辛丑,契丹使沒骨餒來告阿保機哀,廢朝三日。旱,辛亥雨。



 二年春正月癸丑朔,更名亶。癸亥,端明殿學士兵部侍郎馮道、太常卿崔協為中書侍郎:同中書門下平章事。二月壬午朔,新羅使張芬來。西川節度使孟知祥殺其
 兵馬都監李嚴。丙申,赦京師囚。郭從謙為景州刺史,既而殺之。戊戌,山南東道節度使劉訓為南面招討使,以伐荊南。三月壬子朔,幸會節園,群臣買宴。盧臺軍亂,殺其將寫震。新羅使林彥來。夏四月庚寅,盧臺軍將龍晊等伏誅。夏四月庚寅,盧臺軍將龍晊等伏誅。六月丙戌,任圜罷。庚子,幸白司馬坡,祭突厥神。秋七月甲子,隨州刺史西方鄴取夔、忠、萬州。癸酉,殺豆盧革、韋說。八月乙酉,牂牁使宋朝化及昆明
 使者來。九月庚午,黨項使如連山來。壬申,契丹使梅老來。



 冬十月乙酉,如汴州。宣武軍節度使朱守殷反,馬步軍都指揮使馬彥超死之。己丑,守殷自殺。乙未,殺太子少保致仕任圜。辛丑,德音釋輕繫囚。是月,傳箭于霍彥威。十一月乙亥,契丹使梅老來。十二月己丑,回鶻西界吐蕃遣使者來。甲辰,畋于東郊。丙午,追尊祖考為皇帝,妣為皇后;高祖聿謚曰孝恭,廟號惠祖,祖妣劉氏謚曰孝恭昭;曾祖敖謚曰孝質,廟號毅祖,祖妣張氏謚曰孝質
 順;祖琰謚曰孝靖,廟號烈祖,祖妣何氏謚曰孝靖穆;考謚曰孝成,廟號德祖,妣劉氏謚曰孝成懿。立廟于應州。



 三年春正月丁巳,契丹陷平州。二月辛巳,吐渾都督李紹虜來。乙未,孔循罷。



 戊戌,回鶻使李阿山來。三月丁未朔,御札求直言。己未,鄭玨罷。癸亥,成德軍節度使王建立為尚書右僕射、同中書門下平章事。西方鄴克歸州。戊辰,宣徽南院使范延光為樞密使。夏四月戊寅,延光罷。乙酉,達靼遣使者來。義武軍節度使王都反。壬寅,歸德軍節度使王晏球為北面行營招討使。五月,契丹禿餒入于定州。



 辛酉,右衛上將軍趙敬怡為樞密使。封回
 鶻可汁王仁裕為順化可汗。秋七月己未,殺齊州防禦使曹廷隱。八月,盧龍軍節度使趙德鈞執契丹首領惕隱赫邈。慶州防禦使竇廷琬反。冬十月,靜難軍節度使李敬周討之。丁巳,突厥使張慕晉來。十一月壬午,吐渾使念九來。甲午,王建立罷。十二月,李敬周克慶州,竇廷琬伏誅。幸亥,幸康義誠第。



 四年春正月壬辰,回鶻使掣撥都督來。二月癸卯,王晏球克定州。辛酉,晏球獻馘俘。趙敬怡薨。丁卯,崔協薨。庚午,至自汴州。三月丙戌,殺姪從璨。夏四月,契丹寇雲州。癸丑,契丹使撩括梅里來求禿餒,殺之。甲
 寅,端明殿學士、尚書兵部侍郎趙鳳為門下侍郎兼工部尚書、同中書門下平章事。五月己巳,朝群臣賀朔。乙酉,追謚少帝曰昭宣光烈孝皇帝。契丹寇雲州。秋七月壬申,殺右金吾衛上將軍毛璋。八月乙巳,黑水使骨至來。丁未,吐渾首領念公山來。乙卯,黨項折遇明來。己未,高麗王建使張彬來。九月癸巳,殺供奉官烏昭遇。冬十二月辛丑,殺西平縣令李商。



 長興元年春正月丁卯,閱馬於苑。辛卯,宣徽南院使朱弘昭為大內留守。二月,戊戌,黑水兀兒遣使者來。乙巳,
 天雄軍節度使石敬瑭為御營使。癸丑,朝獻於太微宮。甲寅,享于太廟。乙卯,有事于南郊,大赦,改元。三月庚寅,立淑妃曹氏為皇后。夏四月戊戌,安重誨使河中衙內指揮使楊彥溫逐其節度使從珂。壬寅,西京留守索自通、侍衛步軍指揮使藥彥稠討之。辛亥,自通執彥溫殺之。戊午,群臣上尊號曰聖明神武文德恭孝皇帝。辛酉,吐蕃首領於撥葛來。五月丁丑,回鶻使孽栗祖來。庚辰,回鶻使安黑連來。秋七月壬午,訪莊宗子孫瘞所。八月乙未,忠武軍節度使張延朗為
 三司使。壬寅,殺捧聖都軍使李行德、大將張儉,滅其族。吐渾來附。封子從榮為秦王。戊申,海州將王傳極殺其刺史陳宣,叛于吳來降。乙卯,吐渾康合畢來。丙辰,封子從厚為宋王。九月壬戌,吐蕃使王滿儒來。東川節度使董璋反。甲申,成德軍節度使范延光為樞密使。丁亥,石敬瑭為東川行營都招討使。



 冬十月丁酉,始藏冰。甲辰,驍衛上將軍致仕張筠進助軍粟。乙巳,董璋陷閬州,殺節度使李仁矩,指揮使姚洪死之。孟知祥反。十一月庚申朔,秦王從榮受冊,謁于太廟。丙戌,契丹東丹王突欲來奔。
 十二月丁未,二王後、秘書丞、酅國公楊仁矩卒,廢朝一日。丁巳,回鶻順化可汗王仁裕使翟末斯來。安重誨討董璋。沙州曹義金遣使者來。



 二年春正月戊辰,黨項使折七移來。庚辰,達靼使列六薛娘居來。二月丁酉,幸安元信第。戊戌,突厥使杜阿熟、吐渾使康萬琳來。辛丑,安重誨罷。三月,趙鳳罷。丁亥,太常卿李愚為中書侍郎、同中書門下平章事。夏四月甲辰,宣徽北院使趙延壽為樞密使。甲寅,董璋陷遂州,武信軍節度使夏魯奇死之。乙卯,以旱赦流罪以下囚。閏五月丁酉,殺太子太師致仕安重誨及其妻張氏、子崇
 贊崇緒。秋八月己未,契丹使邪姑兒來。九月丁亥,放五坊鷹隼。冬十一月戊申,吐蕃遣使者來。



 辛丑,旌表棣州民邢釗門閭。十二月甲寅朔,除鐵禁,初稅農具錢。己未,西涼府遣使者來。己巳,回鶻使安永思來。辛未,渤海使文成角來。黨項寇方渠。



 三年春正月庚子,契丹使拽骨來。己酉,渤海、回鶻皆遣使者來。二月己卯,靜難軍節度使藥彥稠及黨項戰于牛兒谷,敗之。三月甲申,契丹遣使者來。夏四月庚申,新羅遣使者來。五月己丑,二王後詹事司直楊延紹襲封
 酅國公。丙午,孟知祥攻董璋,陷綿州。六月甲寅,封王建為高麗國王、大義軍使孟知祥殺董璋,陷東川。達靼首領頡哥以其族來附。秋八月己卯,吐蕃遣使者來。冬十月庚申,幸石敬瑭第。



 四年春正月庚寅,端明殿學士、兵部侍郎劉昫為中書侍郎、同中書門下平章事。



 二月戊午,孟知祥使朱滉來。三月甲辰,追冊晉國夫人夏氏為皇后。夏五月戊寅,封子從珂為潞王,從益許王,姪從溫兗王,從璋洋王,從敏涇王。丙戌,契丹使述
 骨卿來。秋七月乙未,回鶻都督李末來,獻白鶻,命放之。八月戊申,大赦。九月戊戌,趙延壽罷。山南東道節度使硃弘昭為樞密使。冬十月庚申,范延光罷。三司使馮贇為樞密使。壬申,幸士和亭,得疾。十一月壬辰,秦王從榮以兵入興聖宮,不克,伏誅。乙未,侍衛親軍都指揮使康義誠殺三司使孫岳。戊戌,皇帝崩於雍和殿。



 嗚呼,自古治世少而亂世多!三代之王有天下者,皆數百年,其可道者,數君而已,況於後世邪!況於五代邪!予
 聞長老為予言:「明宗雖出夷狄,而為人純質,寬仁愛人。」於五代之君,有足稱也。嘗夜焚香,仰天而祝曰:「臣本蕃人,豈足治天下!世亂久矣,願天早生聖人。」自初即位,減罷宮人、伶官;廢內藏庫,四方所上物,悉歸之有司。廣壽殿火災,有司理之,請加丹雘,喟然歎曰:「天以火戒我,豈宜增以侈邪!」歲嘗旱,已而雪,暴出庭中,詔武德司宮中無得掃雪,曰:「此天所以賜我也。」數問宰相馮道等民間疾苦,聞道等言穀帛賤,民無疾疫,則欣然曰:「吾何以堪之,當與公等作好事,以報上天。」吏有犯贓,輒置之死,曰:「此民之蠹也!」以詔書褒廉吏孫岳等,以風示天下。其愛
 人恤物,蓋亦有意於治矣。其即位時,春秋已高,不邇聲色,不樂遊畋。在位七年,於五代之君,最為長世,兵革粗息,年屢豐登,生民實賴以休息。然夷狄性果,仁而不明,屢以非辜誅殺臣下。至於從榮父子之間,不能慮患為防,而變起倉卒,卒陷之以大惡,帝亦由此飲恨而終。當是時,大理少卿康澄上疏言時事,其言曰:「為國家者有不足懼者五,深可畏者六:三辰失行不足懼,天象變見不足懼,小人訛言不足懼,山崩川竭不足懼,水旱蟲蝗不足懼也;賢士藏匿深可畏,四民遷業深可畏,上下相徇深可畏,廉恥道消深可畏,毀譽亂真深可畏,直言不
 聞深可畏也。」識者皆多澄言切中時病。



 若從榮之變,任圜、安重誨等之死,可謂上下相徇,而毀譽亂真之敝矣。然澄之言,豈止一時之病,凡為國者,可不戒哉!



\end{pinyinscope}