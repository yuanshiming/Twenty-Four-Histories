\article{卷十五唐明宗家人傳第三}

\begin{pinyinscope}

 明宗和武憲皇後曹氏昭懿皇后夏氏
 明宗三后一妃:和武憲皇后曹氏生晉國公主;昭懿皇后夏氏生秦王從榮、愍帝;宣憲皇后魏氏,潞王從珂母也;淑妃王氏,許王從益之慈母也。曹氏、夏氏皆不見其世家。夏氏無封爵,明宗未即位前卒。明宗天成元年,封楚國夫人曹氏為淑妃,追封夏氏晉國夫人。長興元年,立淑妃為皇后,而夏氏所生二子皆已王,乃追冊為皇后,謚曰昭懿。



 宣憲皇後魏氏魏氏,鎮州平山人也。初適平山民王氏,生子十歲矣。明宗為騎將,掠平山,得其子母以歸。居數年,魏氏卒,葬太原。其子是為潞王從珂。明宗時,從珂已王,乃追封魏氏為魯國夫人。廢帝即位,追尊魏氏為皇太后,議建陵寢,而太原石敬瑭反,乃於京師河南府東立寢宮。清泰三年六月丙寅,遣工部尚書崔儉奉上皇太后寶冊,謚曰宣憲。



 淑妃王氏淑妃王氏,邠州餅家子也,有美色,號「花見羞」。少賣梁故將劉鄩為侍兒,鄩卒,王氏無所歸。是時,明宗夏夫人已卒,方求別室,有言王氏於安重誨者,重誨以告明宗而
 納之。王氏素得鄩金甚多,悉以遣明宗左右及諸子婦,人人皆為王氏稱譽,明宗益愛之。而夫人曹氏為人簡質,常避事,由是王氏專寵。



 明宗即位,議立皇后,而曹氏當立,曹氏謂王氏曰:「我素多病,而性不耐煩,妹當代我。」王氏曰:「后,帝匹也,至尊之位,誰敢干之!」乃立曹氏為皇后,王氏為淑妃。妃事皇后亦甚謹,每帝晨起,盥櫛服御,皆妃執事左右,及罷朝,帝與皇后食,妃侍,食徹乃退,未嘗少懈,皇后心亦益愛之。然宮中之事,皆主於妃。



 明宗病,妃與宦者孟漢瓊出納左右,遂專用事,殺安重誨、秦王從榮,皆與焉。劉鄩諸子,皆以妃故封拜官爵。愍帝即
 位,冊尊皇后為皇太后,妃為皇太妃。初,明宗後宮有生子者,命妃母之,是為許王從益。從益乳母司衣王氏,見明宗已老而秦王握兵,心欲自託為後計,乃曰:「兒思秦王。」是時從益已四歲,又數教從益自言求見秦王。明宗遣乳嫗將兒往來秦府,遂與從榮私通,從榮因使王氏伺察宮中動靜。從榮已死,司衣王氏以謂秦王實以兵入宮衛天子,而以反見誅,出怨言。愍帝聞之,大怒,賜司衣王氏死,而事連太妃,由是心不悅,欲遷之至德宮,以太后素善妃,懼傷其意而止,然待之甚薄。



 廢帝入立,嘗置酒妃院,妃舉酒曰:「願辭皇帝為比丘尼。」帝驚,問其故,
 曰「小兒處偶得命,若大人不容,則死之日,何面見先帝!」因泣下。廢帝亦為之悽然,待之頗厚。石敬瑭兵犯京師,廢帝聚族將自焚。妃謂太后曰:「事急矣,宜少回避,以俟姑夫。」太后曰:「我家至此,何忍獨生,妹自勉之!」太后乃與帝俱燔死,而妃與許王從益及其妹匿於鞠院以免。



 晉高祖立,妃自請為尼,不可,乃遷于至德宮。晉遷都汴,以妃子母俱東,置於宮中,高祖皇后事妃如母。天福四年九月癸未,詔以郇國三千戶封唐許王從益為郇國公,以奉唐祀,服色、旌旗一依舊制。太常議立莊宗、明宗、愍帝三室,以至德宮為廟;詔立高祖、太宗,為五廟,使從
 益歲時主祠。



 出帝即位,妃母子俱還洛陽。契丹犯京師,趙延壽所尚明宗公主已死,耶律德光乃為延壽娶從益妹,是為永安公主。公主不知其母為誰,素亦養於妃,妃至京師主婚禮。德光見明宗畫像,焚香再拜,顧妃曰:「明宗與我約為弟兄,爾吾嫂也。」



 已而靳之曰:「今日乃吾婦也。」乃拜從益為彰信軍節度使,從益辭,不之官,與妃俱還洛陽。



 德光北歸,留蕭翰守汴州。漢高祖起太原,翰欲北去,乃使人召從益,委以中國。從益子母逃於徽陵域中,以避使者,使者迫之以東,遂以從益權知南朝軍國事。



 從益御崇元殿,翰率契丹諸將拜殿上,晉群臣拜
 殿下。群臣入謁太妃,妃曰:「吾家子母孤弱,為翰所迫,此豈福邪?禍行至矣!」乃以王松、趙上交為左右丞相,李式、翟光鄴為樞密使,燕將劉祚為侍衛親軍都指揮使。翰留契丹兵千人屬祚而去。



 漢高祖擁兵而南,從益遣人召高行周、武行德等為拒,行周等皆不至,乃與王松謀以燕兵閉城自守。妃曰:「吾家亡國之餘,安敢與人爭天下!」乃遣人上書迎漢高祖。高祖聞其嘗召行周而不至,遣郭從義先入京師殺妃母子。妃臨死呼曰:「吾家母子何罪?何不留吾兒,使每歲寒食持一盂飯灑明宗墳上。」聞者悲之。從益死時年十七。



 愍帝哀皇后孔氏愍帝哀皇后孔氏,父循,橫海軍節度使。后有賢行,生四子。愍帝即位,立為皇后,未及冊命而難作。愍帝出奔,后病子幼,皆不能從。廢帝入立,后及四子皆見殺。晉高祖立,追謚曰哀。



 明宗二子明宗四子,曰從璟、從榮、從厚、從益。



 從璟初名從審,為人驍勇善戰,而謙退謹敕。從莊宗戰,數有功,為金槍指揮使。明宗軍變于魏,莊宗謂從璟曰:「爾父於國有大功,忠孝之心,朕自明信。今為亂軍所逼,爾宜自往宣朕意,毋使自疑。」從璟馳至衛州,為元行欽所執,將殺之,從璟呼曰:「我父為亂軍所逼,公等不亮其心,我亦不能至魏,願
 歸衛天子。」



 行欽釋之。莊宗憐其言,賜名繼璟,以為己子。



 從莊宗如汴州,將士多亡於道,獨從璟不去,左右或勸其逃禍,從璟不聽。莊宗聞明宗已渡黎陽,復欲遣從璟通問。行欽以為不可,遂殺之。明宗即位,贈太保。



 嗚呼!無父烏生,無君烏以為生?而世之言曰:「為忠孝者不兩全。」夫豈然哉?君父,人倫之大本;忠孝,臣子之大節。豈其不相為用,而又相害者乎?抑私與義而已耳。蓋以其私則兩害,以其義則兩得。其父以兵攻其君,為其子者,從父乎?從君乎?曰:「身從其居,志從其義,可也。」身居君所則從君,居父所則從父。其從於君者,必辭其君曰:「子不
 可以射父,願無與兵焉!」則又號泣而呼其父曰:「盍捨兵而歸我君乎!」君敗則死之,父敗則終喪而事君。其從於父者,必告之曰:「君不可以射也,盍捨兵而歸吾君乎!」君敗則死之,父敗則待罪於君,赦己則終喪而事之。古之知孝者莫如舜,知義者莫如孔、孟,其於君臣父子之際詳矣,使其不幸而遭焉,其亦如是而已矣!從璟之於莊宗,知所從而得其死矣。哀哉!



 秦王從榮,天成元年,以檢校司徒兼御史大夫,拜天雄軍節度使、同中書門下平章事。三年,徙鎮河東。長興元年,拜河南尹,兼判六軍諸衛事。從璟死,從榮於諸皇子
 次最長,又握兵柄。然其為人輕雋而鷹視,頗喜儒,學為歌詩,多招文學之士,賦詩飲酒,故後生浮薄之徒,日進諛佞以驕其心。自將相大臣皆患之,明宗頗知其非而不能裁制。從榮嘗侍側,明宗問曰:「爾軍政之餘,習何事業?」對曰:「有暇讀書,與諸儒講論經義爾。」明宗曰:「經有君臣父子之道,然須碩儒端士,乃可親之。吾見先帝好作歌詩,甚無謂也。汝將家子,文章非素習,必不能工,傳於人口,徒取笑也。吾老矣,於經義雖不能曉,然尚喜屢聞之,甚餘不足學也。」



 是歲秋,封從榮秦王。故事,諸王受封不朝廟,而有司希旨,欲重其禮,乃建議曰:「古者因禘、嘗而
 發爵祿,所以示不敢專。今受大封而不告廟,非敬順之道也。」於是從榮朝服,乘輅車,具鹵簿,至朝堂受冊,出,載冊以車,朝于太廟,京師之人皆以為榮。三年,加兼中書令。有司又言:「故事,親王班宰相下,今秦王位高而班下,不稱。」於是與宰相分班而居右。



 四年,加尚書令,食邑萬戶。太僕少卿何澤上書,請立從榮為皇太子。是時明宗已病,得澤書不悅,顧左右曰:「群臣欲立太子,吾當養老於河東。」乃召大臣議立太子事,大臣皆莫敢可否。從榮入白曰:「臣聞姦人言,欲立臣為太子,臣實不願也。」明宗曰:「此群臣之欲爾。」從榮出,見范延光、趙延壽等曰:「諸公
 議欲立吾為太子,是欲奪吾兵柄而幽之東宮耳。」延光等患之,乃加從榮天下兵馬大元帥。有司又言:「元帥或統諸道,或專一面,自前世無天下大元帥之名,其禮無所考按。請自節度使以下,凡領兵職者,皆具橐鞬以軍禮庭參;其兼同中書門下平章事者,初見亦如之,其後許如客禮。凡元帥府文符行天下,皆用帖。又升班在宰相上。」從榮大宴元帥府,諸將皆有頒給:控鶴、奉聖、嚴衛指揮使,人馬一匹、絹十匹;其諸軍指揮使,人絹十匹;都頭已下,七匹至三匹。又請嚴衛、捧聖千人為牙兵,每入朝,以數百騎先後,張弓挾矢,馳走道上,見者皆震懾。從榮
 又命其寮屬及四方游士試作《征淮檄》,陳己所以平一天下之意。



 言事者請為諸王擇師傅,以加訓導。宰相難其事,因請從榮自擇。從榮乃請翰林學士崔棁、刑部侍郎任贊為元帥判官。明宗曰:「學士代予言,不可也。」從榮出而恚曰:「任以元帥而不得請屬寮,非吾所諭也。」將相大臣見從榮權位益隆,而輕脫如此,皆知其禍而莫敢言者。惟延光、延壽陰有避禍意,數見明宗,涕泣求解樞密,二人皆引去,而從榮之難作。



 十一月戊子,雪,明宗幸宮西士和亭,得傷寒疾。己丑,從榮與樞密使朱弘昭、馮贇入問起居於廣壽殿,帝不能知人。王淑妃告曰「從榮
 在此。」又曰:「弘昭等在此。」皆不應。從榮等去,乃遷於雍和殿,宮中皆慟哭。至夜半後,帝蹶然自興於榻,而侍疾者皆去,顧殿上守漏宮女曰:「夜漏幾何?」對曰:「四更矣!」帝即唾肉如肺者數片,溺涎液斗餘。守漏者曰:「大家省事乎?」曰:「吾不知也。」



 有頃,六宮皆至,曰:「大家還魂矣!」因進粥一器。至旦,疾少愈,而從榮稱疾不朝。



 初,從榮常忌宋王從厚賢於己,而懼不為嗣。其平居驕矜自得,及聞人道宋王之善,則愀然有不足之色。其入問疾也,見帝已不知人,既去,而聞宮中哭聲,以謂帝已崩矣,乃謀以兵入宮。使其押衙馬處鈞告弘昭等,欲以牙兵入宿衛,問何
 所可以居者。弘昭等對曰:「宮中皆王所可居,王自擇之。」因私謂處鈞曰:「聖上萬福,王宜竭力忠孝,不可草草。」處鈞具以告從榮,從榮還遣處鈞語弘昭等曰:「爾輩不念家族乎?」弘昭、贇及宣徽使孟漢瓊等入告王淑妃以謀之,曰:「此事須得侍衛兵為助。」乃召侍衛指揮使康義誠,謀於竹林之下。義誠有子在秦王府,不敢決其謀,謂弘昭曰:「僕為將校,惟公所使爾!」弘昭大懼。



 明日,從榮遣馬處鈞告馮贇曰:「吾今日入居興聖宮。」又告義誠,義誠許諾。



 贇即馳入內,見義誠及弘昭、漢瓊等坐中興殿閣議事,贇責義誠曰:「主上所以畜養吾徒者,為今日爾!今
 安危之機,間不容髮,奈何以子故懷顧望,使秦王得至此門,主上安所歸乎?吾輩復有種乎?」漢瓊曰:「賤命不足惜,吾自率兵拒之。」



 即入見曰:「從榮反,兵已攻端門。」宮中相顧號泣。明宗問弘昭等曰:「實有之乎?」對曰:「有之。」明宗以手指天泣下,良久曰:「義誠自處置,毋令震動京師。」潞王子重吉在側,明宗曰:「吾與爾父起微賤,至取天下,數救我危窘。從榮得何氣力,而作此惡事!爾亟以兵守諸門。」重吉即以控鶴兵守宮門。



 是日,從榮自河南府擁兵千人以出。從榮寮屬甚眾,而正直之士多見惡,其尤所惡者劉贊、王居敏,而所暱者劉陟、高輦。從榮兵出,與
 陟、輦並轡耳語,行至天津橋南,指日景謂輦曰:「明日而今,誅王居敏矣!」因陣兵橋北,下據胡床而坐,使人召康義誠。而端門已閉,叩左掖門,亦閉,而於門隙中見捧聖指揮使朱弘實率騎兵從北來,即馳告從榮。從榮驚懼,索鐵厭心,自調弓矢。皇城使安從益率騎兵三百衝之,從榮兵射之,從益稍卻。弘實騎兵五百自左掖門出,方渡河,而後軍來者甚眾,從榮乃走歸河南府,其判官任贊已下皆走出定鼎門,牙兵劫嘉善坊而潰。從榮夫妻匿床下,從益殺之。



 明宗聞從榮已死,悲咽幾墮于榻,絕而蘇者再。馮道率百寮入見,明宗曰:「吾家事若此,慚見
 群臣!」君臣相顧,泣下沾襟。從榮二子尚幼,皆從死。後六日而明宗崩。



 明宗四侄明宗兄弟皆不見于世家,而有姪四人,曰從璨、從璋、從溫、從敏。



 從璨初為右衛大將軍,安重誨用事,自諸王將相皆下之,從璨為人剛猛,不能少屈,而性倜儻,輕財好施,重誨忌之。明宗幸汴州,以從璨為大內皇城使。嘗於會節園飲,酒酣,戲登御榻,重誨奏其事,貶房州司戶參軍,賜死。重誨見誅,詔復其官,贈太保。



 從璋字子良,少善騎射。莊宗時,將兵戍常山,聞明宗兵變于魏,乃亦起兵據刑州。明宗即位,以為捧聖左廂都
 指揮使,改皇城使,領饒州刺史,拜彰國軍節度使,徙鎮義成。明宗幸汴州,從璋欲率民為貢獻,其從事諫以為不可,從璋怒,引弓欲射之,坐罷為右驍衛上將軍。居久之,出鎮保義,徙河中。長興四年夏,封洋王。晉高祖立,徙鎮威勝,降封隴西郡公。從璋為人貪鄙,自鎮保義,始折節自修,在南陽頗有遺愛。天福二年卒,年五十一。



 從溫字德基,初為北京副留守。歷安國、忠武、義武、成德、武寧五節度使,封兗王。晉高祖立,復為忠武軍節度使。從溫為人貪鄙,多作天子器服以自僭,宗族、賓客諫之,不聽,其妻關氏大呼於牙門曰:「從溫欲反,而造天子服
 器。」從溫大恐,乃悉毀之。



 明宗諸子八人,至晉出帝時六已亡歿,惟從溫、從敏在,太后常曰:「吾惟有一兄,豈可繩之以法!」從溫由此益驕。嘗誣親吏薛仁嗣為盜,悉籍沒其家貲數千萬。仁嗣等詣闕自訴,事下有司,從溫具伏。出帝懼傷太后意,釋之而不問。開運二年,徙河陽三城,卒于官。



 是時從璋子重俊為虢州刺史,坐臟,亦以太后故,罪其判官高獻而已。重俊復為商州刺史。坐與其妹姦及殺其僕孫漢榮掠其妻,賜死。



 從敏字叔達,為人沉厚寡言,善騎射。初從莊宗為馬步軍都指揮使兼行軍司馬,明宗入立,遷皇城使、保義軍
 節度使,與討王都。歷鎮橫海、義武、成德、歸德、保義、昭義、河陽,封涇王。漢高祖時,為西京留守,封秦國公。周廣順元年卒,贈中書令,謚曰恭惠。



\end{pinyinscope}