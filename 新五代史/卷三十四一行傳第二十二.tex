\article{卷三十四一行傳第二十二}

\begin{pinyinscope}

 嗚呼,五代之亂極矣,《傳》所謂「天地閉,賢人隱」之時歟!當此之時,臣弒其君,子弒其父,而搢紳之士安其祿而立其朝,充然無復廉恥之色者皆是也。吾以謂自古忠臣義士多出於亂世,而怪當時可道者何少也,豈果無其人哉?雖曰干戈興,學校廢,而禮義衰,風俗隳壞,至於如此,然自古天下未嘗無人也,吾意必有潔身自負之士,嫉世遠去而不可見者。自古材賢有韞于中而不見於
 外,或窮居陋巷,委身草莽,雖顏子之行,不遇仲尼而名不彰,況世變多故,而君子道消之時乎!吾又以謂必有負材能,修節義,而沉淪于下,泯沒而無聞者。求之傳記,而亂世崩離,文字殘缺,不可復得,然僅得者四五人而已。



 處乎山林而群麋鹿,雖不足以為中道,然與其食人之祿,俯首而包羞,孰若無愧於心,放身而自得,吾得二人焉,曰鄭遨、張薦明。勢利不屈其心,去就不違其義,吾得一人焉,曰石昂。茍利於君,以忠獲罪,而何必自明,有至死而不言者,此古之義士也,吾得一人焉,曰程福贇。五代之亂,君不君,臣不臣,父不父,子不子,至於兄弟、夫
 婦人倫之際,無不大壞,而天理幾乎其滅矣。於此之時,能以孝悌自修於一鄉,而風行於天下者,猶或有之,然其事跡不著,而無可紀次,獨其名氏或因見於書者,吾亦不敢沒,而其略可錄者,吾得一人焉,曰李自倫。作《一行傳》。



 鄭遨張薦明附鄭遨,字雲叟,滑州白馬人也。唐明宗祖廟諱遨,故世行其字。遨少好學,敏於文辭。唐昭宗時,舉進士不中,見天下已亂,有拂衣遠去之意,欲攜其妻、子與俱隱,其妻不從,遨及入少室山為道士。其妻數以書勸遨還家,輒投之於火,後聞其妻、子卒,一慟而止。遨與李振故善,振後
 事梁貴顯,欲以祿遨,遨不顧,後振得罪南竄,遨徒步千里往省之,由是聞者益高其行。其後,遨聞華山有五粒松,脂淪入地,千歲化為藥,能去三尸,因徙居華陰,欲求之。與道士李道殷、羅隱之友善,世目以為三高士。遨種田,隱之賣藥以自給,道殷有釣魚術,鉤而不餌,又能化石為金,遨嘗驗其信然,而不之求也。節度使劉遂凝數以寶貨遺之,遨一不受。



 唐明宗時以左拾遺、晉高祖時以諫議大夫召之,皆不起,即賜號為逍遙先生。天福四年卒,年七十四。



 遨之節高矣,遭亂世不污於榮利,至棄妻、子不顧而去,豈非與世自絕而篤愛其身者歟?然遨
 好飲酒弈棋,時時為詩章落人間,人間多寫以縑素,相贈遺以為寶,至或圖寫其形,玩于屋壁,其跡雖遠而其名愈彰,與乎石門、荷之徒異矣。



 與遨同時張薦明者,燕人也。少以儒學遊河朔,後去為道士,通老子、莊周之說。高祖召見,問「道家可以治國乎?」對曰:「道也者,妙萬物而為言,得其極者,尸居袵席之間可以治天地也。」高祖大其言,延入內殿講《道德經》,拜以為師。薦明聞宮中奉時鼓,曰:「陛下聞鼓乎?其聲一而已。五音十二律,鼓無一焉,然和之者鼓也。夫一,萬事之本也,能守一者可以治天下。」高祖善之,賜號通玄先生,後不知其所終。



 石昂石昂,青州臨淄人也。家有書數千卷,喜延四方之士,士無遠近,多就昂學問,食其門下者或累歲,昂未嘗有怠色。而昂不求仕進。節度使符習高基行,召以為臨淄令。習入朝京師,監軍楊彥朗知留後事,昂以公事至府上謁,贊者以彥朗諱「石」,更其姓曰「右」。昂趨于庭,仰責彥朗曰:「內侍奈何以私害公!昂姓『石』,非『右』也。」彥朗大怒,拂衣起去,昂即趨出。解官還於家,語其子曰:「吾本不欲仕亂世,果為刑人所辱,子孫其以我為戒!」



 昂父亦好學,平生不喜拂說,父死,昂於柩前誦《尚書》,曰:「此吾先人之所欲聞也。」禁其家不可以佛事污吾先人。



 晉高祖時,詔天下
 求孝悌之士,戶部尚書王權、宗正卿石光贊、國子祭酒田敏、兵部侍郎王延等相與詣東上閣門,上昂行義可以應詔。詔昂至京師,召見便殿,以為宗正丞。遷少卿。出帝即位,晉政日壞,昂數上疏極諫,不聽,乃稱疾東歸,以壽終于家。昂既去,而晉室大亂。



 程福贇程福贇者,不知其世家。為人沉厚寡言而有勇。少為軍卒,以戰功累遷洺州團練使。晉出帝時,為奉國右廂都指揮使。開運中,契丹入寇,出帝北征,奉國軍士乘間夜縱火焚營,欲因以為亂,福贇身自救火被傷,火滅而亂者不得發。福贇以為契丹且大至,而天子在軍,京師虛
 空,不宜以小故動搖人聽,因匿其事不以聞。軍將李殷位次福贇下,利其去而代之,因誣福贇與亂者同謀,不然何以不奏。出帝下福贇獄,人皆以為冤,福贇終不自辨以見殺。



 李自倫李自倫者,深州人也。天福四年正月,尚書戶部奏:「深州司功參軍李自倫六世同居,奉敕準格。按格,孝義旌表,必先加按驗,孝者復其終身,義門仍加旌表。



 得本州審到鄉老程言等稱,自倫高祖訓,訓生粲,粲生則,則生忠,忠生自倫,自倫生光厚,六世同居不妄。」敕以所居飛鳧鄉為孝義鄉,匡聖里為仁和里,准式旌表門閭。九月丙
 子,戶部復奏:「前登州義門王仲昭六世同居,其旌表有聽事、步欄,前列屏,樹烏頭正門,閥閱一丈二尺,烏頭二柱端冒以瓦桶,築雙闕一丈,在烏頭之南三丈七尺,夾樹槐柳,十有五步,請如之。」敕曰:「此故事也,令式無之。其量地之宜,高其外門,門安綽楔,左右建臺,高一丈二尺,廣狹方正稱焉,圬以白而赤其四角,使不孝不義者見之,可以悛心而易行焉。」



\end{pinyinscope}