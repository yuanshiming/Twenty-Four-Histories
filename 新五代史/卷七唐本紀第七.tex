\article{卷七唐本紀第七}

\begin{pinyinscope}

 愍皇帝,明宗第五子從厚也。為人形質豐厚,寡言好禮,明宗以其貌類己,特愛之。天成二年,以檢校司徒拜河南尹、判六軍諸衛事,加檢校太保、同中書門下平章事。從厚妃,孔循女也,安重誨怒循以女妻從厚,三年,罷循樞密使,出從厚為宣武軍節度使。明年,徙鎮河東。長興元年,封從厚宋王,徙鎮成德。二年,徙鎮天雄,累加兼中書令。



 四年十一月,秦王從榮伏誅。明宗病甚,遣宦者孟
 漢瓊召王於鄴,而明宗崩,秘其喪六日。十二月癸卯朔,發喪于西宮,皇帝即位于柩前,群臣見於東階,復于喪位。丙午,成服于西宮。庚戌,登光政門樓,存問軍民。辛亥,殺司衣王氏。癸丑,始聽政。乙卯,殺司儀康氏。丁巳,馮道為大行皇帝山陵使,戶部尚書韓彥惲為副,中書舍人王延為判官,禮部尚書王權為禮儀使,兵部尚書李鏻為鹵簿使,御史中丞龍敏為儀仗使,左僕射權判河南府盧質為橋道頓遞使。丁卯,示覃。



 應順元年春正月壬申朔,視朝于廣壽殿。乙亥,契
 丹使都督沒辣於來。戊寅,大赦,改元,用樂。回鶻可汗王仁美遣使者來。沙州、瓜州遣使者來。乙未,朱弘昭、馮贇獻錢助作山陵。閏月丙午,冊皇太后。甲寅,冊太妃王氏。北京留守石敬瑭獻銀絹助作山陵。二月庚寅,視作山陵。鳳翔節度使潞王從珂反。辛卯,西京留守王思同為西面行營都部署,靜難軍節度使藥彥稠為副。三月丙辰,思同兵潰,嚴衛指揮使尹暉、羽林指揮使楊思權以其軍叛降于從珂。辛酉,殺侍衛親軍都指揮使朱弘實。癸亥,河陽三城節度使康義誠為鳳翔行營都招討使,王思同為副。西
 京副留守劉遂雍叛降於從珂,思同奔歸于京師,不克,死之。丁卯,京城巡檢使安從進叛,殺馮贇,朱弘昭自殺,從進傳其二首于從珂。戊辰,如衛州。



 廢帝,鎮州平山人也。本姓王氏,其世微賤,母魏氏,少寡,明宗為騎將,過平山,掠得之。魏氏有子阿三,已十餘歲,明宗養以為子,名曰從珂。及長,狀貌雄偉,謹信寡言,而驍勇善戰,明宗甚愛之。自晉兵戰梁于河上,從珂常立戰功,莊宗呼其小字曰:「阿三不徒與我同年,其敢戰亦類我。」同光二年,為衛州刺史突騎指揮使,戍於石門。明
 宗討趙在禮,自魏反兵而南,從珂率戍兵自曲陽、孟縣馳出常山以追明宗。明宗之南也,兵少,得從珂兵在後,而軍聲大振。明宗入立,拜從珂河中節度使,封潞王。是時,明宗春秋已高,王於諸子次最長,樞密使安重誨患之,乃矯詔河中裨將楊彥溫使圖之。王閱馬于黃龍莊,彥溫即閉門拒之,王止于虞鄉以聞。明宗召王還京師,居之清化里第。重誨數請行軍法,明宗不聽,後重誨見殺,乃起王為左衛大將軍、西京留守。長興三年,為鳳翔節度使。王子重吉自明宗時典禁兵,為控鶴指揮使,愍帝即位,朱弘昭、馮贇用事,乃罷重吉兵職,出為亳州團
 練使。又徙王為北京留守,不降制書而宣授,又以李從璋為代。初,安重誨得罪罷河中,以從璋為代,而重誨見殺,故王益自疑,遂據城反。愍帝遣王思同會諸鎮兵討之,思同戰敗走,諸鎮兵皆潰。



 清泰元年三月丁巳,王以兵東。庚申,次長安,西京副留守劉遂雍叛于唐,來降。甲子,次華州,執藥彥稠。丙寅,次靈寶,河中安彥威、陜州康思立叛于唐,來降。己巳,次陜州。康義誠叛于唐,來降。殺宣徽使孟漢瓊。愍帝出居于衛州。



 夏四月壬申,入京師,馮道率百官迎王于蔣橋,王辭不見。入哭于西宮,遂見群臣,道拜,王答拜。入居于至
 德宮。癸酉,以太后令降天子為鄂王,命王監國。乙亥,皇帝即位。丙子,率河南民財以賞軍。丁丑,借民房課五月以賞軍。戊寅,弒鄂王,慈州刺史宋令詢死之。乙酉,大赦,改元。戊子,殺康義誠及藥彥稠。五月丙午,端明殿學士、左諫議大夫韓昭胤為樞密使,莊宅使劉延朗為樞密副使。庚戌,馮道罷。天雄軍節度使范延光為樞密使。甲寅,賜勸進選人、宗子官。六月庚辰,幸范延光及索自通第。秋七月辛亥,太常卿盧文紀為中書侍郎、同中書門下平章事。丁巳,立沛國夫人劉氏為皇后。八月辛未,尚書左丞姚顗為中
 書侍郎、同中書門下平章事。許御署官選。九月,契丹寇邊。冬十月戊寅,李愚、劉昫罷。十二月己亥,雄武軍節度使張延郎為中書侍郎、同中書門下平章事。契丹寇雲州。庚寅,幸龍門。



 旱。



 二年春二月甲戌,范延光罷。己丑,追尊魯國太夫人魏氏為皇太后。三月辛丑,忠武軍節度使趙延壽為樞密使。夏五月辛卯,宣徽南院使劉延皓為樞密使。契丹寇邊。六月癸未,群臣獻添都馬。秋七月丁酉,回鶻可汗王仁美使其都督陳福海來。



 劉延皓罷。
 九月己酉,刑部尚書房暠為樞密使。乙卯,渤海遣使者來。



 三年春正月乙未,百濟遣使者來。丁未,封子重美為雍王。三月丙午,翰林學士、禮部侍郎馬胤孫為中書侍郎,同中書門下平章事。河東節度使石敬瑭反。夏五月乙卯,建雄軍節度使張敬達為太原四面都招討使,義武軍節度使楊光遠為副。戊申,先鋒指揮使安審信叛降於石敬瑭。己酉,振武戍將安重榮叛降於石敬瑭。壬子,天雄軍屯駐捧聖都虞候張令昭逐其節度使劉延皓。六月癸亥,以令昭為右千牛衛將軍,權知天雄軍事。
 甲戌,宣武軍節度使范延光為天雄軍四面招討使。秋七月戊申,克魏州。壬子,張令昭伏誅。癸丑,彰聖指揮使張萬迪叛降于石敬瑭。八月戊午,契丹使梅里來。九月甲辰,張敬達及契丹戰于太原,敗績,契丹圍敬達于晉安。戊申,如河陽。冬十月壬戌,括馬,籍民為兵。十一月戊子,盧龍軍節度使趙德鈞為行營都統。丁酉,契丹立晉。閏月甲子,楊光遠殺張敬達,以其軍叛降于契丹。甲戌,契丹及晉人至于潞州。丁丑,至自河陽。辛巳,皇帝崩。



 嗚呼,君臣之際,可謂難哉!蓋明者慮於未萌而前知,暗者告以將及而不懼,故先事而言,則雖忠而不信,事至而悔,其可及乎?重誨區區獨見潞王之禍,而謀之不臧,至於殞身赤族,其隙自茲。及愍帝之亡也,穴於徽陵,其土一壟,路人見者,皆為之悲。使明宗為有知,其有愧於重誨矣,哀哉!



\end{pinyinscope}