\article{卷十八漢家人傳第六}

\begin{pinyinscope}

 高祖皇后李氏高祖
 皇后李氏,晉陽人也,其父為農。高祖少為軍卒,牧馬晉陽,夜入其家劫取之。高祖已貴,封魏國夫人,生隱帝。開運四年,高祖起兵太原,賞軍士,帑藏不足充,欲斂於民。后諫曰:「方今起事,號為義兵,民未知惠而先奪其財,殆非新天子所以救民之意也。今後宮所有,請悉出之,雖其不足,士亦不以為怨也。」



 高祖為改容謝之。高祖即位,立為皇后。高祖崩,隱帝冊尊為皇太后。



 帝年少,
 數與小人郭允明、後贊、李業等游戲宮中,后數切責之。帝曰:「國家之事,外有朝廷,非太后所宜言也。」太常卿張昭聞之,上疏諫帝,請:「親近師傅,延問正人,以開聰明。」帝益不省。其後,帝卒與允明等謀議,遂至於亡。



 初,帝與允明等謀誅楊邠、史弘肇等,議已定,入白太后。太后曰:「此大事也,當與宰相議之。」李業從旁對曰:「先皇帝平生言,朝廷大事,勿問書生。」



 太后深以為不可,帝拂衣而去,曰:「何必謀於閨門!」邠等死,周太祖起兵嚮京師,慕容彥超敗於劉子陂,帝欲出自臨兵,太后止之曰:「郭威本吾家人,非其危疑,何肯至此!今若按兵無動,以詔諭威,威必
 有說,則君臣之際,庶幾尚全。」



 帝不從以出,遂及於難。



 周太祖入京師,舉事皆稱太后誥。已而議立湘陰公贇為天子,贇未至,太祖乃請太后臨朝。已而太祖出征契丹,軍士擁之以還。太祖請事太后為母,太后誥曰:「侍中功烈崇高,德聲昭著,剪除禍亂,安定邦家,謳歌有歸,歷數攸屬,所以軍民推戴,億兆同歡。老身未終殘年,屬此多難,唯以衰朽,託於始終。載省來箋,如母見待,感認深意,涕泗橫流。」於是迂后於太平宮,上尊號曰昭聖皇太后。顯德元年春崩。



 高祖弟子侄高祖二弟三子:弟曰崇、曰信,子曰承訓、承祐、承勛。崇子
 曰贇,高祖愛之,以為己子。乾祐元年,拜贇徐州節度使。承訓早卒,追封魏王。承祐次立,是謂隱帝。承勳為開封尹。



 周太祖已敗漢兵于北郊,隱帝遇弒。太祖入京師,以謂漢大臣必相推戴,及見宰相馮道等,道殊無意,太祖不得已,見道猶下拜,道受太祖拜如平時,徐勞之曰:「公行良苦!」太祖意色皆沮,以謂漢大臣未有推立己意,又難於自立,因白漢太后擇立漢嗣。而宗室河東節度使崇等在者四人,乃為太后誥曰:「河東節度使崇,許州節度使信,皆高祖之弟,徐州節度使贇,開封尹承勳,皆高祖之子,文武百辟,其擇嗣君以承天統。」於是周太祖與
 王峻入見太后,言:「開封尹承勳,高祖皇帝之子,宜立。」太后以承勳久病,不任為嗣。太祖與群臣請見承勳視起居,太后命以臥榻舁承勳出見群臣,群臣視之信然,乃共奏曰:「徐州節度使贇,高祖愛之,以為子,宜立為嗣。」乃遣太師馮道率群臣迎贇。道揣周太祖意不在贇,謂太祖曰:「公此舉由衷乎?」太祖指天為誓。道既行,謂人曰:「吾平生不為謬語人,今謬語矣!」道見贇,傳太后意召之。



 贇行至宋州,太祖自澶州為兵士擁還京師,王峻慮贇左右生變,遣侍衛馬軍指揮使郭崇以兵七百騎衛贇。崇至宋州,贇登樓問崇所以來之意,崇曰:「澶州軍變,懼未
 察之,遣崇護衛,非惡意也。」贇召崇,崇不敢進,馮道出與崇語,崇乃登樓見贇,已而奪贇部下兵。



 太祖以書召道先歸,留其副趙上交、王度奉贇入朝太后。道乃先還,贇謂道曰:「寡人此來,所恃者以公三十年舊相,是以不疑。」道默然。贇客將賈正等數目道,欲圖之。贇曰:「勿草草,事豈出於公邪!」道已去,郭崇幽贇於外館,殺賈正及判官董裔、牙內都虞候劉福、孔目官夏昭度等。



 太祖已監國,太后乃下誥曰:「此者樞密使郭威,志安宗社,議立長君,以徐州節度使贇高祖近親,立為漢嗣,乃自籓鎮召赴京師。雖誥命已行,而軍情不附,天道在北,人心靡東。適
 當改卜之初,俾膺分土之命,贇可降授開府儀同三司,檢校太師、上柱國,封湘陰公。」贇以幽死。



 初,贇自徐州入也,以都押衙鞏庭美、教練使楊溫守徐州。庭美等聞贇不得立,乃閉城拒命。太祖拜王彥超徐州節度使,下詔諭庭美等許以刺史,并詔贇赦庭美等。



 廣順元年三月,彥超克徐州,庭美等皆見殺。



 承勳,廣順元年以病卒,追封陳王。



 嗚呼!予既悲湘陰公贇之事,又嘉庭美、楊溫之所為。贇於漢非嫡長,特以周氏移國,畏天下而難之,故假贇以伺間爾。當是之時,天下皆知贇之必不立也,然庭美、
 溫之區區為贇守孤城以死,其始終之迹,何愧於死節之士哉!然予考於實錄,二人之死狀不明。夫二人之事,固知其無所成,其所重者死爾,然史氏不著,不知其何以死也。當王彥超之攻徐州也,周嘗遣人招庭美等,予得其詔書四,皆言庭美等嘗已送款於周,後懼罪而復叛,然庭美等款狀亦不見,是皆不可知也。夫史之闕文,要不慎哉。其疑以傳疑,則信者信矣。予固嘉二人之忠而悲其志,然不得列於死節之士者,惜哉!



 高祖從弟信蔡王信,高祖之從弟也。高祖鎮太原,以信為興捷軍都指揮使領義成軍節度使,徙領許州。高祖寢疾,隱帝當
 立為嗣,楊邠等受顧命,不欲信在京師,乃遣信就鎮,信涕泣而去。信所至黷貨,好行殺戮。軍士有犯法者,信召其妻子,對之刲剔支解,使自食其肉,血流盈前,信命樂飲酒自如也。楊邠等死,信大喜,謂其寮佐曰:「吾嘗為天無眼,而使我鬱鬱於此者三年矣!主上孤立,幾落賊手。諸公可以勸我一杯矣。」已而聞難作,信憂不能食。周太祖軍變於澶州,王峻遣前申州刺史馬鐸以兵巡檢許
 州,信乃自殺。周太祖即位,追封蔡王。



\end{pinyinscope}