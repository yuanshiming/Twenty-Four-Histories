\article{卷十三梁家人傳第一}

\begin{pinyinscope}

 新五代史·傳嗚呼,
 梁之惡極矣!自其起盜賊,至於亡唐,其遺毒流于天下。



 天下豪傑,四面並起,孰不欲戡刃於胸,然卒不能少挫其鋒以得志。梁之無敵於天下,可謂虎狼之彊矣。及其敗也,因於一二女子之娛,至於洞胸流腸,刲若羊豕,禍生父子之間,乃知女色之能敗人矣。自古女禍,大者亡天下,其次亡家,其次亡身,身茍免矣,猶及其子孫,雖遲速不回,未有無禍者也。然原其本末,未始不起於忽微。《易·坤》之初六曰:「履霜,堅冰至。」《家人》之初九曰:「閑有
 家,悔亡。」其言至矣,可不戒哉!梁之家事,《詩》所謂「不可道」者。至於唐、晉以後,親疏嫡庶亂矣!作《家人傳》。



 文惠皇后王氏梁太祖母曰文惠皇后王氏,單州單父人也。其生三子:長曰廣王全昱,次曰朗王存,其次太祖。后少寡,攜其三子傭食蕭縣人劉崇家。太祖壯而無賴,縣中皆厭苦之。崇患太祖慵墮不作業,數加笞責,獨崇母憐之,時時自為櫛沐,戒家人曰:「朱三非常人也,宜善遇之!」黃巢起,太祖與存俱亡為盜,從黃巢攻廣州,存戰死。居數歲,太祖背巢降唐,反以破巢,遂鎮宣武。乃遣人以車馬之蕭縣,迎
 后於崇家。使者至門,后惶恐走避,謂劉氏曰:「朱三落魄無行,作賊死矣,何以至此邪!」使者具道太祖所以然,后乃驚喜泣下,與崇母俱載以歸,封晉國太夫人。



 太祖置酒太夫人前,舉觴為壽,歡甚。太祖啟曰:「朱五經平生讀書,不登一第,有子為節度使,無忝於先人也。」后惻然良久曰:「汝能至此,可謂英特,然行義未必得如先人也!」太祖莫知其故,后曰:「朱二與汝俱從黃巢,獨死蠻嶺,其孤皆在午溝,汝今富貴,獨不念之乎?」太祖泣涕謝罪,乃悉召存諸子以歸。太祖剛暴多殺戮,后每誡之,多賴以全活。



 大順二年秋,后疾,卜者曰:「宜還故鄉。」乃歸。卒於午溝。太
 祖即位,立四廟,追尊皇考為穆皇帝,后曰文惠皇后。



 元貞皇后張氏太祖元貞皇后張氏,單州碭山縣渠亭里富家子也。太祖少以婦聘之,生末帝。



 太祖貴,封魏國夫人。后賢明精悍,動有禮法,雖太祖剛暴,亦嘗畏之。太祖每以外事訪之,後言多中。太祖時時暴怒殺戮,后嘗救護,人賴以獲全。太祖嘗出兵,行至中途,后意以為不然,馳一介召之,如期而至。



 郴王友裕攻徐州,破朱瑾於石佛山,瑾走,友裕不追,太祖大怒,奪其兵。友裕惶恐,與數騎亡山中,久之,自匿於廣王。后陰使人教友裕脫身自歸,友裕晨馳入見太祖,拜伏庭中,泣涕請死,太祖怒甚,使左右捽出,
 將斬之。后聞之,不及履,走庭中持友裕泣曰:「汝束身歸罪,豈不欲明非反乎?」太祖意解,乃免。



 太祖已破朱瑾,納其妻以歸,后迎太祖於封丘,太祖告之。后遽見瑾妻,瑾妻再拜,后亦拜,悽然泣下曰:「兗鄆與司空同姓之國,昆仲之間,以小故興干戈,而使吾姒至此;若不幸汴州失守,妾亦如此矣!」言已又泣。太祖為之感動,乃送瑾妻為尼,后嘗給其衣食。司空,太祖時檢校官也。



 天祐元年,后以疾卒。太祖即位,追冊為賢妃。初葬開封縣潤色鄉,末帝立,追謚曰元貞皇太后,祔於宣陵。后已死,太祖始為荒淫,卒以及禍云。



 陳昭儀昭儀陳氏,宋州人也,少以色進。太祖已貴,嬪妾數百,而昭儀專寵。太祖嘗疾,昭儀與尼數十人晝夜為佛法,未嘗少懈,太祖以為愛己,尤寵之。開平三年,度為尼,居宋州佛寺。



 李昭容昭容李氏,亦以色進。尤謹愿,未嘗去左右。太祖病,晝寢方寐,忽棟折,獨李氏侍側,遽牽太祖衣,太祖驚走,棟折寢上,太祖德之,拜昭容。皆不知其所終。



 末帝德妃張氏末帝德妃張氏,其父歸霸,事太祖為梁功臣。帝為王時,以婦聘之。帝即位,將冊妃為后,妃請待帝郊天,而帝卒不得郊。貞明五年,妃病甚,帝遽冊為德妃,其夕薨,年二
 十四。



 次妃郭氏次妃郭氏,父歸厚,事梁為登州刺史。妃少以色進。梁亡,唐莊宗入汴,梁故妃妾,皆號泣迎拜。賀王友雍妃石氏有色,莊宗召之,石氏慢罵,莊宗殺之。次以召妃,妃懼而聽命。已而度為尼,賜名誓正,居于洛陽。



 初,莊宗之入汴也,末帝登建國樓,謂控鶴指揮使皇甫麟曰:「吾,晉世仇也,不可俟彼刀鋸,卿可盡我命,無使我落仇人之手!麟與帝相持慟哭。是夕,進刃於帝,麟亦自剄。莊宗入汴,命河南張全義葬其尸,藏其首於太社。晉天福三年,詔太社先藏罪人首級,許親屬收葬,乃出末帝首,遣右衛將軍
 安崇阮與妃同葬之。妃卒洛陽。



 太祖兄子太祖二兄:曰全昱,曰存。八子:長曰友裕,次曰友珪、友璋、友貞、友雍、友徽、友孜,其一養子曰友文。開平元年五月乙酉,封友文為博王、友珪郢王、友璋福王、友貞均王、友雍賀王、友徽建王。友裕前即位卒,追封郴王,而康王友孜,末帝即位封。友璋初為壽州團練使、押左右番殿直、監豐德庫,友珪時,為鄆州留後,末帝時,為忠武軍節度使,徙鎮武寧,及友雍、友徽皆不知其所終。



 兄廣王全昱全昱子友諒友能友誨廣王全昱,太祖即位封。太祖與仲兄存俱亡為盜,全昱獨與其母猶寄食劉崇家。



 太祖已貴,乃與其母俱歸宣
 武,領嶺南西道節度使。以太師致仕。



 太祖將受禪,有司備禮前殿,全昱視之,顧太祖曰:「朱三,爾作得否?」太祖宴居宮中,與王飲博,全昱酒酣,取骰子擊盆而迸之,呼太祖曰:「朱三,爾碭山一百姓,遭逢天子用汝為四鎮節度使,於汝何負?而滅他唐家三百年社稷,吾將見汝赤其族矣,安用博為!」太祖不悅,罷會。全昱亦不樂在京師,常居碭山故里。



 三子皆封王:友諒衡王,友能惠王,友誨邵王。



 乾化元年,升宋州為宣武軍,以友諒為節度使。友諒進瑞麥一莖三穗,太祖怒曰:「今年宋州大水,何用此為!」乃罷友諒,居京師。太祖臥病,全昱來視疾,與太祖相持
 慟哭;太祖為釋友諒,使與東歸。貞明二年,全昱以疾薨。徙衡王友諒嗣封廣王。



 友能為宋、滑二州留後、陳州刺史,所至為不法,姦人多依倚之。而陳俗好淫祠左道,其學佛者,自立一法,號曰「上乘」,晝夜伏聚,男女雜亂。妖人母乙、董乙聚眾稱天子,建置官屬,友能初縱之,乙等攻劫州縣,末帝發兵擊滅之。自康王友孜謀反伏誅,末帝始疏斥宗室,宗室皆反仄。貞明四年,友能以陳州兵反,犯京師,至陳留,兵敗,還走陳州,後數月降,末帝赦之,降為房陵侯。



 友誨為陜州節度使,欲以州兵為亂,末帝召還京師,與友諒、友能皆被幽囚。



 梁亡,莊宗入汴,皆見殺。



 兄朗王存存子友寧友倫朗王存,初與太祖俱從黃巢攻廣州,存戰死。存子友寧、友倫。



 友寧字安仁,幼聰敏,喜慍不形於色。太祖以為軍校,善用弓劍。遷衙內制勝都指揮使、龔州刺史。太祖圍鳳翔,遣友寧東備宣武。王師範襲梁,圍齊州,友寧引兵擊之,奪馬千匹,斬首數千級。太祖奉昭宗還京師,拜友寧建武軍節度使,賜號「迎鑾毅勇功臣」。太祖復遣攻師範,圍博昌,屠之,清河為之不流。戰于石樓,兵敗,友寧墮馬見殺。



 友倫幼亦明敏,通《論語》、小學,曉音律。存已死,太祖以友倫為元從馬軍指揮使,表右威武將軍。燕人攻魏內黃,友倫以前鋒夜渡河,奪馬千匹。李罕之以潞州
 降梁,晉人攻潞,友倫以兵入潞州,取罕之以歸。累遷檢校司空,領藤州刺史。


太祖圍鳳翔,晉人襲梁,友倫以兵三萬至
 \gezhu{
  樊石}
 山,晉人乃卻,友倫西會太祖於鳳翔。昭宗還長安,拜友倫寧遠軍節度使。太祖東歸,留友倫宿衛,伺察昭宗所為。



 友倫擊鞠墜馬死,太祖大怒,以兵七萬至河中。昭宗涕泣,不知所為,將奔太原,不果。宰相崔胤遣人止太祖,太祖以為友倫胤等殺之,奏請誅胤等,昭宗未從,乃遣友諒至京師,以兵圍開化坊,殺胤及京兆尹鄭元規、皇城使王建勛、飛龍使陳班、閣門使王建襲、客省使王建乂、前左僕射張濬。



 太祖即位,已封宗室,中書
 上議,故皇兄存,皇姪建武軍節度使友寧、寧遠軍節度使友倫,皆當封。於是追封存朗王、友寧安王、友倫密王。



 子郴王友裕郴王友裕,字端夫,幼善騎射,從太祖征伐,能以寬厚得士卒心。太祖與晉圍黃鄴於西華,鄴卒荷槊登城罵敵,晉王使胡騎連射不能中。太祖顧友裕,一發中之,軍中皆大歡呼,晉王喜,遣友裕良弓百矢。太祖鎮宣武,以為衙內都指揮使。景福元年,太祖攻鄆,友裕以先鋒次斗門,鄆兵夜擊之,友裕敗走。太祖從後來,不知友裕之敗也,前軍遇敵多死。太祖至村落間,始與友裕相得。是時,朱宣在濮州,太祖乃遣友裕先以二百騎前,太祖後至,
 與友裕相失。太祖卒與敵遇,敗而走。敵兵追之甚急,前至大溝,幾不免,賴溝中有積薪,馬乃得過,梁將李璠等死者十餘人。



 冬,友裕取濮州,遂圍時溥於徐州。朱瑾以兵二萬救溥,友裕敗瑾於石佛山,瑾走。都虞候朱友恭讒之太祖,以為瑾可追而友裕不追。太祖大怒,奪其兵屬龐師古,以友裕屬吏,使者誤致書於友裕,友裕惶恐,不知所為,賴張皇后教之,得免。



 權知許州。許州近蔡,苦於大寇,居民殘破,友裕招撫流散,增戶三萬餘。



 遷諸軍都指揮使,與平兗、鄆,還領許州。崔洪奔淮南,友裕引兵定蔡州,市不易肆。太祖兼鎮護國軍,以友裕為留後。遷
 忠武軍節度使。太祖攻鳳翔,未下,去攻邠州。友裕破靈臺、良原,下隴州,楊崇本以邠州降。後崇本復叛,太祖遣友裕攻之,屯于永壽。友裕以疾卒。



 子博王友文博王友文,字德明,本姓康名勤。幼美風姿,好學,善談論,頗能為詩,太祖養以為子。太祖領四鎮,以友文為度支鹽鐵制置使。太祖用兵四方,友文征賦聚斂以供軍實。太祖即位,以故所領宣武、宣義、天平、護國四鎮征賦,置建昌宮總之,以友文為使,封博王。太祖幸西都,友文留守東京。



 子庶人友珪庶人友珪者,太祖初鎮宣武,略地宋、亳間,與逆旅婦人野合而生也。長而辯黠多智。博王友文多材藝,
 太祖愛之,而年又長,太祖即位,嫡嗣未立,心嘗獨屬友文。太祖自張皇后崩,無繼室,諸子在鎮,皆邀其婦入侍。友文妻王氏有色,尤寵之。太祖病久,王氏與友珪妻張氏,常專房侍疾。太祖病少間,謂王氏曰:「吾知終不起,汝之東都,召友文來,吾與之決。」蓋心欲以後事屬之。乃謂敬翔曰:「友珪可與一郡,趣使之任。」乃以友珪為萊州刺史。



 太祖素剛暴,既病,而喜怒難測,是時左降者,必有後命,友珪大懼。其妻張氏曰:「官家以傳國寶與王氏,使如東都召友文,君今受禍矣!」夫婦相對而泣。



 左右勸友珪曰:「事急計生,何不早自為圖?」友珪乃易衣服,微行入左龍虎
 軍,見統軍韓勍計事,勍以牙兵五百隨友珪,雜控鶴衛士而入。夜三鼓,斬關入萬春門,至寢中,侍疾者皆走。太祖惶駭起呼曰:「我疑此賊久矣,恨不早殺之,逆賊忍殺父乎!」友珪親吏馮廷諤以劍犯太祖,太祖旋柱而走,劍擊柱者三,太祖憊,仆于床,廷諤以劍中之,洞其腹,腸胃皆流。友珪以裀褥裹之寢中,祕喪四日。乃出府庫,大齎群臣及諸軍。遣受旨丁昭浦矯詔馳至東都,殺友文。又下詔曰:「朕艱難創業,踰三十年。託於人上,忽焉六載,中外協力,期于小康。豈意友文陰畜異圖,將行大逆。昨二日夜,甲士突入大內,賴友珪忠孝,領兵剿
 戮,保全朕躬。然而疾恙震驚,彌所危殆。友珪克平兇逆,厥功靡倫,宜委權主軍國。」然後發喪。乾化二年六月既望,友珪於柩前即皇帝位,拜韓勍忠武軍節度使,以末帝為汴州留後,河中朱友謙為中書令。友謙不受命。而懷州龍驤軍三千,劫其將劉重霸,據懷州,自言討賊。三年正月,友珪祀天於洛陽南郊,改元曰鳳歷。



 太祖外孫袁象先與駙馬都尉趙巖等,謀與末帝討賊。二月,象先以禁兵入宮,友珪與妻張氏趨北垣樓下,將踰城以走,不果,使馮廷諤進刃其妻及己,廷諤亦自殺。末帝即位,復友文官爵,廢友珪為庶人。



 子康王友孜康王友孜,目重瞳子,嘗竊自負,以為當為天子。貞明元年,末帝德妃薨,將葬,友孜使刺客夜入寢中。末帝方寐,夢人害己,既寤,聞榻上寶劍鎗然有聲,躍起,抽劍曰:「將有變邪!」乃索寢中,得刺客,手殺之,遂誅友孜。明日,謂趙巖、張漢傑曰:「幾與卿輩不相見。」由此遂疏弱宗室,而信任趙、張,以至於敗亡。



 嗚呼,《春秋》之法,是非與奪之際,難矣哉!或問:「梁太祖以臣弒君,友珪以子弒父,一也。與弒即位,踰年改元,《春秋》之法,皆以君書,而友珪不得列于本紀,何也?且父子之惡均,而奪其子,是與其父也,豈《春秋》之旨哉?」



 予應之曰:「
 梁事著矣!其父之惡,不待與奪其子而後彰,然末帝之志,不可以不伸也。《春秋》之法,君弒而賊不討者,國之臣子任其責。予於友珪之事,所以伸討賊者之志也。」



\end{pinyinscope}