\article{卷二十五唐臣傳第十三}

\begin{pinyinscope}

 周德威周德威,字鎮遠,朔州馬邑人也。為人勇而多智,能望塵以知敵數。其狀貌雄偉,笑不改容,人見之,凜如也。事晉王為騎將,稍遷鐵林軍使,從破王行瑜,以功遷衙內指揮使。其小字陽五,當梁、晉之際,周陽五之勇聞天下。梁軍圍晉太原,令軍中曰:「能生得周陽五者為刺史。」有驍將陳章者,號陳野義,常乘白馬被朱甲以自異,出入陣中,求周陽五,欲必生致之。晉王戒德威曰:「陳野義欲得
 汝以求刺史,見白馬朱甲者,宜善備之!」德威笑曰:「陳章好大言耳,安知刺史非臣作邪?」因戒其部兵曰:「見白馬朱甲者,當佯走以避之。」兩軍皆陣,德威微服雜卒伍中。陳章出挑戰,兵始交,德威部下見白馬朱甲者,因退走,章果奮槊急追之,德威伺章已過,揮鐵槌擊之,中章墮馬,遂生擒之。



 梁攻燕,晉遣德威將五萬人為燕攻梁,取潞州,遷代州刺史、內外蕃漢馬步軍都指揮使。梁軍捨燕攻潞,圍以夾城,潞州守將李嗣昭閉城拒守,而德威與梁軍相持於外踰年。嗣昭與德威素有隙,晉王病且革,語莊宗曰:「梁軍圍潞,而德威與嗣昭有隙,吾甚憂之!」
 王喪在殯,莊宗新立,殺其叔父克寧,國中未定,而晉之重兵,悉屬德威于外,晉人皆恐。莊宗使人以喪及克寧之難告德威,且召其軍。德威聞命,即日還軍太原,留其兵城外,徒步而入,伏梓宮前慟哭幾絕,晉人乃安。



 遂從莊宗復擊梁軍,破夾城,與李嗣昭歡如初。以破夾城功,拜振武節度使、同中書門下平章事。



 天祐七年秋,梁遣王景仁將魏、滑、汴、宋等兵七萬人擊趙。趙王王熔乞師於晉,晉遣德威先屯趙州。冬,梁軍至柏鄉,趙人告急,莊宗自將出贊皇,會德威于石橋,進距柏鄉五里,營於野河北。晉兵少,而景仁所將神威、龍驤、拱宸等軍,皆梁
 精兵,人馬鎧甲飾以組繡金銀,其光耀日,晉軍望之色動。德威勉其眾曰:「此汴、宋傭販兒,徒飾其外耳,其中不足懼也!其一甲直數十千,擒之適足為吾資,無徒望而愛之,當勉以往取之。」退而告莊宗曰:「梁兵甚銳,未可與爭,宜少退以待之。」莊宗曰:「吾提孤軍出千里,其利速戰。今不乘勢急擊之,使敵知吾之眾寡,則吾無所施矣!」德威曰:「不然,趙人能城守而不能野戰。吾之取勝,利在騎兵,平川廣野,騎兵之所長也。今吾軍於河上,迫賊營門,非吾用長之地也。」



 莊宗不悅,退臥帳中,諸將無敢入見。德威謂監軍張承業曰:「王怒老兵。不速戰者,非怯也。
 且吾兵少而臨賊營門,所恃者,一水隔耳。使梁得舟筏渡河,吾無類矣!不如退軍鄗邑,誘敵出營,擾而勞之,可以策勝也。」承業入言曰:「德威老將知兵,願無忽其言!」莊宗遽起曰:「吾方思之耳。」已而德威獲梁游兵,問景仁何為,曰:「治舟數百,將以為浮梁。」德威引與俱見,莊宗笑曰:「果如公所料。」乃退軍鄗邑。德威晨遣三百騎叩梁營挑戰,自以勁兵三千繼之。景仁怒,悉其軍以出,與德威轉鬥數十里,至于鄗南。兩軍皆陣,梁軍橫亙六七里,汴、宋之軍居西,魏、滑之軍居東。莊宗策馬登高,望而喜曰:「平原淺草,可前可卻,真吾之勝地!」乃使人告德威曰:「吾當
 為公先,公可繼進。」德威諫曰:「梁軍輕出而遠來,與吾轉戰,其來必不暇齎糧糗,縱其能齎亦不暇食,不及日午,人馬俱飢,因其將退而擊之勝。」諸將亦皆以為然。至未申時,梁軍東偏塵起,德威鼓噪而進,麾其西偏曰:「魏、滑軍走矣!」又麾其東偏曰:「梁軍走矣!」梁陣動,不可復整,乃皆走,遂大敗。自鄗追至于柏鄉,橫尸數十里,景仁以十餘騎僅而免。



 自梁與晉爭,凡數十戰,其大敗未嘗如此。



 劉守光僭號於燕,晉遣德威將三萬出飛狐以擊之。德威入祁溝關,取涿州,遂圍守光於幽州,破其外城,守光閉門距守。而晉軍盡下燕諸州縣,獨幽州不下,
 圍之踰年乃破之,以功拜盧龍軍節度使。德威雖為大將,而常身與士卒馳騁矢石之間。



 守光驍將單廷珪,望見德威於陣,曰:「此周陽五也!」乃挺槍馳騎追之。德威佯走,度廷珪垂及,側身少卻,廷珪馬方馳,不可止,縱其少過,奮槌擊之,廷珪墜馬,遂見擒。莊宗與劉掞相持於魏,掞夜潛軍出黃澤關以襲太原,德威自幽州以千騎入土門以躡之。掞至樂平,遇雨不得進而還。德威與掞俱東,爭趨臨清。臨清有積粟,且晉軍餉道也,德威先馳據之,以故莊宗卒能困掞軍而敗之。



 莊宗勇而好戰,尤銳於見敵。德威老將,常務持重以挫人之鋒,故其用兵,
 常伺敵之隙以取勝。十五年,德威將燕兵三萬人,與鎮、定等軍從莊宗于河上,自麻家渡進軍臨濮,以趨汴州。軍宿胡柳陂,黎明,候騎報曰:「梁軍至矣!」莊宗問戰於德威,德威對曰:「此去汴州,信宿而近,梁軍父母妻子皆在其中,而梁人家國繫此一舉。吾以深入之兵,當其必死之戰,可以計勝,而難與力爭也。且吾軍先至此,糧爨具而營柵完,是謂以逸待勞之師也。王宜按軍無動,而臣請以騎軍擾之,使其營柵不得成,樵爨不暇給,因其勞乏而乘之,可以勝也。」莊宗曰:「吾軍河上,終日俟敵,今見敵不擊,復何為乎?」顧李存審曰:「公以輜重先,吾為公殿。」



 遽督軍而出。德威謂其子曰:「吾不知死所矣!」前遇梁軍而陣:王居中,鎮、定之軍居左,德威之軍居右,而輜重次右之西。兵已接,莊宗率銀槍軍馳入梁陣,梁軍小敗,犯晉輜重,輜重見梁朱旗,皆驚走入德威軍,德威軍亂,梁軍乘之,德威父子皆戰死。莊宗與諸將相持而哭曰:「吾不聽老將之言,而使其父子至此!」莊宗即位,贈德威太師。明宗時,加贈太尉,配享莊宗廟。晉高祖追封德威燕王。子光輔,官至刺史。



 符存審子彥超彥饒彥卿符存審,字德詳,陳州宛丘人也。初名存,少微賤,嘗犯法當死,臨刑,指旁壞垣顧主者曰:「願就死于彼,冀得垣土
 覆尸。」主者哀而許之,為徙垣下。而主將方飲酒,顧其愛妓,思得善歌者佐酒,妓言:「有符存常為妾歌,甚善。」主將馳騎召存審,而存審以徙垣下故,未加刑,因往就召,使歌而悅之,存審因得不死。



 其後事李罕之,從罕之歸晉,晉王以為義兒軍使,賜姓李氏,名存審。



 從晉王擊李匡儔,為前鋒,破居庸關。又從擊王行瑜,破龍泉寨,以功遷檢校左僕射。從李嗣昭攻汾州,執李瑭,遷左右廂步軍指揮使。又從嗣昭攻潞州,降丁會。從周德威破梁夾城,遷忻州刺史、蕃漢馬步軍指揮使。晉、趙攻燕,梁救燕,擊趙深州,圍蓚縣,存審與史建瑭軍下博,擊走梁軍,
 遷領邢州團練使。魏博叛梁降晉,存審為前鋒,屯臨清。莊宗入魏,存審殿軍魏縣,與劉掞相距於莘西。從莊宗敗掞於故元城,閻寶以邢州降,乃以存審為安國軍節度使。毛璋以滄州降,徙存審橫海,加同中書門下平章事。



 契丹圍幽州,是時晉與梁相持河上,欲發兵,兵少,欲勿救,懼失之。莊宗疑,以問諸將,而存審獨以為當救,曰:「願假臣騎兵五千足矣!」乃遣存審分兵救之,卒擊走契丹。從戰胡柳陂,晉軍晨敗,亡周德威,存審與其子彥圖力戰,暮復敗梁軍於土山,遂取德勝,築河南北為兩城,晉人謂之「夾寨」。遷內外蕃漢馬步軍總管。



 梁朱友謙以
 河中同州降晉,梁遣劉掞攻同州,友謙求救,乃遣存審與李嗣昭救之。河中兵少而弱,梁人素易之,且不虞晉軍之速至也。存審選精騎二百雜河中兵出擊掞壘,陽敗而走,掞兵追之,晉騎反擊,獲其騎兵五十,梁人知其晉軍也,皆大驚。然河中糧少而新降,人心頗持兩端,晉軍屯朝邑,諸將皆欲速戰,存審曰:「使梁軍知吾利於速戰,則將夾渭而營,斷我餉道,以持久困我,則進退不可,敗之道也。不若緩師示弱,伺隙出奇,可以取勝。」乃按軍不動。居旬日,望氣者言:「有黑氣,狀如鬥雞。」存審曰:「可以一戰矣!」乃進軍擊掞,大敗之,掞閉壁不復出。存審曰:「
 掞兵已敗,不如逸之。」乃休士卒,遣裨將王建及牧馬於沙苑,掞以謂晉軍且懈,乃夜遁去,存審追擊于渭河,又大敗之。張文禮弒趙王王鎔,晉遣閻寶、李嗣昭等攻之,至輒戰死,最後遣存審破之。



 存審為將有機略,大小百餘戰,未嘗敗衄,與周德威齊名。德威死,晉之舊將獨存審在。契丹攻遮虜,乃以存審為盧龍軍節度使。時存審已病,辭不肯行,莊宗使人慰諭,彊遣之。



 莊宗滅梁入洛,存審自以身為大將,不得與破梁之功,怏怏,疾益甚,因請朝京師。是時,郭崇韜權位已重,然其名望素出存審下,不樂其來而加己上,因沮其事,存審妻郭氏泣訴于
 崇韜曰:「吾夫於國有功,而於公鄉里之舊,奈何忍令死棄窮野!」崇韜愈怒。存審章累上,輒不許,存審伏枕歎曰:「老夫事二主四十年,今日天下一家,四夷遠俗,至於亡國之將、射鉤斬袪之人,皆得親見天子,奉觴為壽,而獨予棄死於此,豈非命哉!」崇韜度存審病已亟,乃請許其來朝。徙存審宣武軍節度使,卒於幽州。臨終,戒其子曰:「吾少提一劍去鄉里,四十年間取將相,然履鋒冒刃出死入生而得至此也。」因出其平生身所中矢鏃百餘而示之曰:「爾其勉哉!」存審三子:彥超、彥饒、彥卿。



 彥超為汾州刺史。郭從謙弒莊宗,明宗入洛陽,是時,彥超為北京
 巡檢,永王存霸奔於太原,彥超見留守張憲謀之。憲,儒者,事莊宗最久,不忍背恩,欲納之,彥超不從,存霸遂見殺。明宗即位,彥超來朝,明宗德之,勞曰:「河東無事,賴爾之力也。」以為建雄軍留後。遷北京留守,徙鎮昭義,罷為上將軍,復為泰寧軍節度使,又徙安遠。彥超主藏奴王希全盜其貲,彥超稍責之,奴懼,夜叩其門,言有急,彥超出,見殺,贈太尉。



 次子彥饒,為汴州馬步軍都指揮使。天成元年,發汴兵三千戍瓦橋關,控鶴指揮使張諫為亂,殺權知州高逖,迫彥饒為帥。彥饒陽許之曰:「欲吾為帥,當止焚掠,明日以軍禮見吾於南衙。」乃陰與拱衙指揮
 使龐起伏甲於衙內。明日,諫等皆集,伏兵發,誅諫等,殺四百餘人,即日牒州事與推官韋儼。明宗下詔褒其忠略。



 其後累遷彰聖都指揮使,歷曹、沂、饒三州刺史。清泰三年,自饒州刺史拜忠正軍節度使、侍衛馬步軍都指揮使。晉高祖起太原,彥饒以侍衛兵從廢帝至河陽。廢帝敗,晉高祖以楊光遠代彥饒將親軍,徙彥饒義成軍節度使。范延光反,白奉進以侍衛兵三千屯滑州。兵士犯法,奉進捕得五人,其三人義成兵也,因並斬之,彥饒怒。



 明日,奉進從數騎過彥饒謝不先告而殺,彥饒曰:「軍士各有部分,義成兵卒豈公所得斬邪?何無主客之禮
 也!」奉進怒曰:「軍士犯法,安有彼此!且僕已自謝過,而公怒不息,欲與延光同反邪!」拂衣而起,彥饒不復留之,其麾下大噪,追奉進殺之,彥饒不之止也。已而屯駐軍將馬萬等聞亂,以兵擒彥饒送之京師,遂以彥饒應延光反聞。行至赤岡,高祖使人殺之,下詔削奪在身官爵。彥饒與晉初無釁隙,以一旦之忿,不能馭其軍,殺奉進已非其本意,以反見誅,非其罪也!



 史建瑭子匡翰史建瑭,雁門人也。晉王為雁門節度使,其父敬思為九府都督,從晉王入關破黃巢,復京師,擊秦宗權于陳州,嘗將騎兵為先鋒。晉王東追黃巢于冤朐,還過梁,軍其
 城北。梁王置酒上源驛,獨敬思與薛鐵山、賀回鶻等十餘人侍。晉王醉,留宿梁驛,梁兵夜圍而攻之。敬思登驛樓,射殺梁兵十餘人,會天大雨,晉王得與從者俱去,縋尉氏門以出。而敬思為梁追兵所得,見殺。



 建瑭少事軍中為裨校,自晉降丁會,與梁相距於潞州,建瑭已為晉兵先鋒。梁兵數為建瑭所殺,相戒常避史先鋒。梁遣王景仁攻趙,晉軍救趙,建瑭以先鋒兵出井陘,戰於柏鄉。梁軍為方陣,分其兵為二:汴、宋之軍居左,魏、滑之軍居右。



 周德威擊其左,建瑭擊其右,梁軍皆走,遂大敗之。以功加檢校左僕射。



 天祐九年,晉攻燕,燕王劉守光乞師
 於梁,梁太祖自將擊趙,圍棗彊、蓚縣。



 是時晉精兵皆北攻燕,獨符存審與建瑭以三千騎屯趙州。梁軍已破棗彊,存審扼下博橋。建瑭分其麾下五百騎為五隊:一之衡水,一之南宮,一之信都,一之阜城,而自將其一,約各取梁芻牧者十人會下博。至暮,擒梁兵數十,皆殺之,各留其一人,縱使逸去,告之曰:「晉王軍且大至。」明日,建瑭率百騎為梁旗幟,雜其芻牧者,暮叩梁營,殺其守門卒,縱火大呼,斬擊數十百人。而梁芻牧者所出,各遇晉兵,有所亡失,其縱而不殺者,歸而皆言晉軍且至。梁太祖夜拔營去,蓚縣人追擊之,梁軍棄其輜重鎧甲不可勝
 計。梁太祖方病,由是增劇,而晉軍以故得並力以收燕者,二人之力也。後從莊宗入魏博,敗劉掞於故元城,累以功歷貝、相二州刺史。十八年,晉軍討張文禮於鎮州,建瑭以先鋒兵下趙州,執其刺史王珽。兵傅鎮州,建瑭攻其城門,中流矢卒,年四十二。



 建瑭子匡翰,尚晉高祖女,是為魯國長公主。匡翰為將,沉毅有謀,而接下以禮,與部曲語未嘗不名。歷天雄軍步軍都指揮使、彰聖馬軍都指揮使。事晉為懷和二州刺史、鄭州防禦使、義成軍節度使,所至兵民稱慕之。史氏世為將,而匡翰好讀書,尤喜《春秋三傳》,與學者講論,終日無倦。義成軍從事
 關澈尤嗜酒,嘗醉罵匡翰曰:「近聞張彥澤臠張式,未見史匡翰斬關澈,天下談者未有偶爾!」匡翰不怒,引滿自罰而慰勉之,人皆服其量。卒年四十。



 王建及王建及,許州人也。少事李罕之,從罕之奔晉,為匡衛指揮使。梁、晉戰柏鄉,相距鄗邑野河上,鎮、定兵扼河橋,梁兵急擊之。莊宗登高臺望見鎮、定兵將敗,顧建及曰:「橋為梁奪,則吾軍危矣,奈何?」建及選二百人馳擊梁兵,梁兵敗,解去。從戰莘縣、故元城,皆先登陷陣,以功累拜遼州刺史,將銀槍效節軍。



 晉攻楊劉,建及躬自負葭葦堙塹,先登拔之。從戰胡柳,晉兵已敗,與梁爭土山,梁兵先
 至,登山而陣。莊宗至山下望梁陣堅而整,呼其軍曰:「今日之戰,得山者勝。」因馳騎犯之,建及以銀槍軍繼進,梁兵下走,陣山西,晉兵遂得土山。



 諸將皆言:「潰兵未集,日暮不可戰。」閻寶曰:「彼陣山上,吾在其下,尚能擊之,況以高而擊下,不可失也。」建及以為然,因白莊宗曰:「請登高望臣破敵!」



 即呼眾曰:「今日所失輜重皆在山西,盍往取之!」即馳犯梁陣,梁兵大敗。晉遂軍德勝,為南北城于河上。梁將賀瑰攻其南城,以竹笮維戰艦於河,晉兵不得渡,南城危甚。莊宗積金帛於軍門,募能破梁戰艦者,至於吐火禁咒莫不皆有。建及重鎧執槊呼曰:「梁、晉一
 水間爾,何必巧為!吾今破之矣。」即以大甕積薪,自上流縱火焚梁戰艦,建及以二舟載甲士隨之,斧其竹笮,梁兵皆走。晉軍乃得渡。救南城,瑰圍解去。



 自莊宗得魏博,建及將銀槍效節軍。建及為將,喜以家貲散士卒。莊宗遣宦官韋令圖監其軍,令圖言:「建及得士心,懼有異志,不可令典牙兵。」即以為代州刺史。建及怏怏而卒,年五十七。



 元行欽元行欽,幽州人也。為劉守光裨將,守光篡其父仁恭,使行欽以兵攻仁恭於大安山而囚之,又使行欽害諸兄弟。其後晉攻幽州,守光使行欽募兵雲、朔間。是時明宗
 掠地山北,與行欽相拒廣邊軍,凡八戰,明宗七射中行欽,行欽拔矢而戰,亦射明宗中股。行欽屢敗,乃降。明宗撫其背而飲以酒曰:「壯士也!」因養以為子。



 常從明宗戰,數立功。莊宗已下魏,益選驍將自衛,聞行欽驍勇,取之為散員都部署,賜姓名曰李紹榮。



 莊宗好戰而輕敵,與梁軍戰潘張,軍敗而潰,莊宗得三四騎馳去,梁兵數百追及,攢槊圍之。行欽望其旗而識之,馳一騎,奪劍斷其二矛,斬首一級,梁兵解去。



 莊宗還營,持行欽泣曰:「富貴與卿共之!」由是寵絕諸將。拜忻州刺史,遷武寧軍節度使。莊宗宴群臣於內殿,酒酣樂作,道平生戰陣事以
 為笑樂,而怪行欽不在,因左右顧視曰:「紹榮安在?」所司奏曰:「奉敕宴使相,紹榮散官,不得與也。」



 莊宗罷會不樂。明日,即拜行欽同中書門下平章事。自此不召群臣入內殿,但宴武臣而已。



 趙在禮反於魏,莊宗方選大將擊之,劉皇后曰:「此小事,可趣紹榮指揮。」



 乃以為鄴都行營招撫使,將二千人討之。行欽攻鄴南門,以詔書招在禮。在禮送羊酒犒軍,登城謂行欽曰:「將士經年離去父母,不取敕旨奔歸,上貽聖憂,追悔何及?若公善為之辭,尚能改過自新。」行欽曰:「天子以汝等有社稷之功,小過必當赦宥。」在禮再拜,以詔書示諸軍。皇甫暉從旁奪
 詔書壞之,軍士大噪。行欽具以聞,莊宗大怒,敕行欽:「破城之日,無遺種!」乃益召諸鎮兵,皆屬行欽。行欽屯澶州,分諸鎮兵為五道,毀民車輪、門扉、屋椽為筏,渡長慶河攻冠氏門,不克。



 是時,邢、洺諸州,相繼皆叛,而行欽攻鄴無功,莊宗欲自將以往,群臣皆諫止,乃遣明宗討之。明宗至魏,軍城西,行欽軍城南。而明宗軍變,入于魏,與在禮合。行欽聞之,退屯衛州,以明宗反聞。莊宗遣金槍指揮使李從璟馳詔明宗計事。



 從璟,明宗子也。行至衛州,而明宗已反,行欽乃繫從璟,將殺之,從璟請還京師,乃許之。明宗自魏縣引兵南,行欽率兵趨還京師。從莊宗
 幸汴州,行至滎澤,聞明宗已渡黎陽,莊宗復遣從璟通問于明宗,行欽以為不可,因擊殺從璟。



 明宗入汴州,莊宗至萬勝鎮,不得進,與行欽登道旁冢,置酒,相顧泣下。有野人獻雉,問其冢名,野人曰:「愁臺也。」莊宗益不悅,因罷酒去。西至石橋,置酒野次,莊宗謂行欽曰:「卿等從我久,富貴急難無不同也。今茲危蹙,而默默無言,坐視成敗。我至滎澤,欲單騎渡河,自求總管,卿等各陳利害。今日俾我至此,卿等何如?」行欽泣而對曰:「臣本小人,蒙陛下撫養,位至將相。危難之時,不能報國,雖死無以塞責。」因與諸將百餘人,皆解髻斷髮,置之于地,誓以死報,君
 臣相持慟哭。



 莊宗還洛陽,數日,復幸汜水。郭從謙反,莊宗崩,行欽出奔。行至平陸,為野人所執,送虢州,刺史石潭折其兩足,載以檻車,送京師。明宗見之,罵曰:「我兒何負於爾!」行欽真目直視曰:「先皇帝何負於爾!」乃斬於洛陽市,市人皆為之流涕。



 嗚呼!死之所以可貴者,以其義不茍生爾。故曰:主在與在,主亡與亡者,社稷之臣也。方明宗之兵變于魏,諸將未知去就,而行欽獨以反聞,又殺其子從璟,至於斷髮自誓,其誠節有足嘉矣。及莊宗之崩,不能自決,而反逃死以求生,終於被執而見殺。其言雖不屈,而死非其志
 也,烏足貴哉!



 安金全安金全,代北人也。為人驍果,工騎射,號能擒生踏伏。事晉為騎將,數從莊宗用兵有功,官至刺史,以疾居于太原。莊宗已下魏博,與梁相距河上。梁將王檀襲太原,晉兵皆從莊宗于河上,太原無備,監軍張承業大恐,率諸司工匠登城扞禦,而外攻甚急。金全彊起謂承業曰:「太原,晉之根本也。一旦不守,則大事去矣!



 老夫誠憊矣,然尚能為公破賊。」承業喜,授以甲兵。金全被甲跨馬,召率子弟及故將吏得百餘人,夜出北門,擊檀於羊馬城中,檀軍驚潰,而晉救兵稍至。然莊宗不以金全為能,終其
 世不錄其功。金全與明宗有舊,明宗即位,拜金全振武軍節度使、同中書門下平章事。在鎮二年,召還京師,以疾卒。



 袁建豐袁建豐,不知其世家也。晉王討黃巢至華陰,闌得之,時方九歲,愛其俊爽,收養之。長習騎射,為鐵林都虞候,從擊王行瑜、李匡威,以功遷突陣指揮使。從莊宗破夾城,戰柏鄉,遷左廂馬軍指揮使。明宗為衙內指揮使,建豐為副使,從莊宗入魏,取衛、磁、洺三州,拜洺州刺史。擊梁將王千,斬首千餘級,獲其將校七十餘人。遷相州刺史。從戰胡柳,指揮使孟謙據相州叛,建豐還討平之。徙隰
 州刺史,病風廢。明宗即位,以舊恩召還京師,親幸其第,撫慰甚厚,加檢校太尉,遙領鎮南軍節度使,俾食其俸以卒,贈太尉。



 西方鄴西方鄴,定州滿城人也。父再遇,為汴州軍校,鄴居軍中,以勇力聞。年二十,南渡河遊梁,不見用,復歸莊宗于河上,莊宗以為孝義指揮使,數從征伐有功,同光中為曹州刺史,以州兵屯汴州。明宗自魏反,兵南渡河,而莊宗東幸汴州,汴州節度使孔循懷二志,使北門迎明宗,西門迎莊宗,所以供帳委積如一,曰:「先至者入之。」鄴因責循曰:「主上破梁而得公,有不殺之恩,奈何欲納總管而
 負國!」



 循不答。鄴度循不可爭,而石敬瑭妻,明宗女也,時方在汴,鄴欲殺之,以堅人心。



 循知其謀,取藏其家,鄴無如之何。而明宗已及汴,乃將五百騎西迎莊宗於汜水,嗚咽泣下,莊宗亦為之噓唏,乃使以兵為先鋒。慶宗至汴西,不得入,還洛陽,遇弒。明宗入洛,鄴請死於馬前,明宗嘉歎久之。明年,荊南高季興叛,明宗遣襄州節度使劉訓等招討,而以東川董璋為西南面招討使,乃拜鄴夔州刺史,副璋以兵出三峽。已而訓等無功見黜,諸將皆罷,璋亦嘗出兵,惟鄴獨取三州,乃以夔州為寧江軍,拜鄴節度使。已而又取歸州,數敗季興之
 兵。鄴武人,所為多不中法度,判官譚善達數以諫。鄴怒,遣人告善達受人金,下獄。善達素剛,辭益不遜,遂死於獄中。鄴病,見善達為祟,卒於鎮。



\end{pinyinscope}