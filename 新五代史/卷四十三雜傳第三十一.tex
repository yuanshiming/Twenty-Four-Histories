\article{卷四十三雜傳第三十一}

\begin{pinyinscope}

 氏叔琮氏叔琮,開封尉氏人也。為梁騎兵伍長,梁兵擊黃巢陳、許間,叔琮戰數有功,太祖壯之,使將後院馬軍,從攻徐、兗,表宿州刺史。使攻襄陽,戰數敗,降為陽翟鎮遏使。久之,遷曹州刺史。太祖下河中,取晉、絳,晉王遣使致書太祖求成,太祖以晉書詞嫚,乃遣叔琮與賀德倫等攻之。叔琮自太行入,取澤、潞,出石會,營于洞渦,久之糧盡,乃旋。表晉州刺史。晉人復取絳州,攻臨汾,叔琮選壯士二
 人深目而胡鬚者,牧馬襄陵道旁,晉人以為晉兵,雜行道中,伺其怠,擒晉二人而歸。晉人大驚,以為有伏兵,乃退屯於蒲縣。太祖遣友寧兵萬人會叔琮禦晉,友寧欲休兵以待,叔琮曰:「敵聞救至必走,走則何功邪?」乃夜擊之,晉人大敗,逐之至於太原。太祖大喜曰:「破太原非氏老不可。」已而兵大疫,叔琮班師,令曰:「病不能行者焚之。」病者懼,皆言無恙,乃以精卒為殿而還至石會,留數騎,以大將旗幟立于高岡,晉兵疑其有伏,乃不敢追。久之,徙保大軍節度使。昭宗遷洛,拜右龍武統軍。太祖遣叔琮與李彥威等弒昭宗,已而殺之。



 李彥威李彥威,壽州人也。少事梁太祖,為人穎悟,善揣人意,太祖憐之,養以為子,冒姓朱氏,名友恭。歷汝、潁二州刺史。昭宗遷洛,拜右龍武統軍。初,劉季述廢昭宗,立皇太子裕為天子。昭宗反正,以為太子幼,為賊所立,赦之,復其始封為德王。昭宗自岐還,太祖見裕眉目疏秀,惡之,謂宰相崔胤曰:「德王嘗為季述所立,安得猶在乎?公白天子殺之。」胤奏之,昭宗不許,佗日以問太祖,太祖曰:「臣安敢及之,胤欲賣臣爾。」昭宗遷洛,謂蔣玄暉曰:「德王,朕愛子也,全忠何為欲殺之?」因泣下,嚙指流血。玄暉具以白太祖,太祖益惡之。是時,昭宗改元天祐,遷于東都,為梁
 所迫,而晉人、蜀人以為天祐之號非唐所建,不復稱之,但稱天復。王建亦傳檄天下,舉兵誅梁。太祖大懼,恐昭宗奔佗鎮,以兵七萬如河中,陰遣敬翔至洛,告彥威與氏叔琮等,使行弒逆。八月壬辰,彥威、叔琮以龍武兵宿禁中,夜二鼓,以兵百人叩宮門奏事,夫人裴正一開門問曰:「奏事安得以兵入?」龍武牙官史太殺之,趨椒蘭殿,問昭宗所在,昭宗方醉,起走,太持劍逐之,昭宗單衣旋柱而走,太劍及之,昭宗崩。訃至河中,太祖陽為驚駭,投地號哭,罵曰:「奴輩負我,俾我被惡名於後世邪!」太祖至洛,流彥威、叔琮嶺南,使張廷範殺之。彥威臨刑大呼曰:「
 賣我以滅口,其如神理何?」顧廷範曰:「勉之,公行自及。」遂見殺。已而還其姓名。



 莊宗時,得故唐內人景奼,言當彥威等弒昭宗時,諸王宗屬數百人皆遇害,而同為一坑,瘞于龍興寺北,請合為一冢而改葬之。詔以故濮王為首,葬以一品禮云。



 李振李振,字興緒,其祖抱真,唐潞州節度使。振為唐金吾衛將軍,拜台州刺史。



 盜起浙東,不果行,乃西歸。過梁,以策干太祖,太祖留之。太祖兼領鄆州,表振節度副使。



 振奏事長安,舍梁邸。宦官劉季述謀廢昭宗,遣其姪希正因梁邸吏程巖見振曰:「今主上嚴急,誅殺不辜,中尉懼及
 禍,將行廢立,請與諸邸吏協力以定中外,如何?」振駭然曰:「百歲奴事三歲主,而敢爾邪!今梁王百萬之師,方仗大義尊天子,君等無為此不祥也!」振還,季述卒與巖等廢昭宗,幽之東宮,號太上皇,立皇太子裕為天子。是時,太祖用兵在邢、洺間,季述詐為太上皇誥告太祖,太祖猶豫,未知所為,振曰:「夫豎刁、伊戾之亂,所以為霸者資也。今閹宦作亂,天子危辱,此正仗義立功之時。」太祖大悟,乃囚季述使者,遣振詣京師見崔胤,謀出昭宗。昭宗返正,太祖大喜,執振手曰:「卿謀得之矣!」



 王師範以青州降梁,遣振往代師範,師範疑懼,不知所為,振曰:「獨不聞
 漢張繡乎?繡與曹公為敵,然不歸袁紹而歸曹公者,知其志大,不以私讎殺人也。今梁王方欲成大事,豈以故怨害忠臣乎?」師範洗然自釋,乃西歸梁。



 昭宗遷洛,振往來京師,朝臣皆側目,振視之若無人。有所小怒,必加譴謫。



 故振一至京師,朝廷必有貶降。時人目振為鴟梟。太祖之弒昭宗也,遣振至京師與朱友恭、氏叔琮謀之。昭宗崩,太祖問振所以待友恭等宜如何?振曰:「昔晉司馬氏殺魏君而誅成濟,不然,何以塞天下口?」太祖乃歸罪友恭等而殺之。



 振嘗舉進士咸通、乾符中,連不中,尤憤唐公卿,及裴樞等七人賜死白馬驛,振謂太祖曰:「此輩
 嘗自言清流,可投之河,使為濁流也。」太祖笑而從之。



 太祖即位,累遷戶部尚書。友珪時,以振代敬翔為崇政院使。莊宗滅梁入汴,振謁見郭崇韜,崇韜曰:「人言李振一代奇才,吾今見之,乃常人爾!」已而伏誅。



 裴迪裴迪,字升之,河東聞喜人也。為人明敏,善治財賦,精於簿書。唐司空裴璩判度支,辟為出使巡官。都統王鐸鎮滑州,奏迪汴、宋、鄆等州供軍院使。鐸為租庸使,辟租庸招納使。梁太祖鎮宣武,辟節度判官。太祖用兵四方,常留迪以調兵賦。太祖乃榜門,以兵事自處,而以貨財獄訟一切任迪。太祖西攻岐,王師範謀襲汴,遣健卒苗
 公立持書至汴,陰伺虛實。迪召公立問東事,公立色動,乃屏人密詰之,具得其事。迪不暇啟,遣朱友寧以兵巡兗、鄆,以故師範雖竊發而事卒不成。



 太祖自岐還,將吏皆賜「迎鑾葉贊功臣」,將吏入見,太祖目迪曰:「葉贊之功,惟裴公有之,佗人不足當也。」迪入唐,累遷太常卿。太祖即位,召拜右僕射,居一歲告老,以司空致仕,卒于家。



 韋震韋震,字東卿,雍州萬年人也。初名肇。為人彊敏,有口辯。事梁太祖為都統判官。申叢執秦宗權,欲送于太祖,又欲自獻於京師,又欲挾宗權奪其兵。太祖遣震入蔡州視之,叢遣騎兵三百迎震,欲殺之,震以計得免。還白太
 祖曰:「叢不足慮,為其謀者牙將裴涉,妄庸人也。」叢後果為郭璠所殺。璠以宗權歸於太祖,太祖欲大其事,請獻俘于唐,唐以時溥破黃巢,獻馘而已,宗權不足俘,左拾遺徐彥樞亦疏請所在斬決。太祖遣震奏事京師,往復論列,卒俘宗權。太祖德之,表為節度副使。昭宗幸石門,太祖遣震由虢略間道奉表行在,昭宗賜其名震。太祖已破兗、鄆,遂攻吳,大敗于清口。太祖懼諸鎮乘間圖己,乃諷杜洪、鐘傳、王師範、錢鏐等薦己為元帥,且求兼領鄆州。昭宗初不許,震彊辯,敢大言,語數不遜,昭宗卒許梁以鄆州,太祖遂兼四鎮,表震鄆州留後。昭宗遷洛,震
 入為河南尹、六軍諸衛副使,以病喑,守太子太保致仕。太祖受禪,改太子太傅。末帝即位,加太師,卒。



 孔循孔循,不知其家世何人也。少孤,流落於汴州,富人李讓闌得之,養以為子。



 梁太祖鎮宣武,以李讓為養子,循乃冒姓朱氏。稍長,給事太祖帳下,太祖諸兒乳母有愛之者,養循為子,乳母之夫姓趙,循又冒姓為趙氏,名殷衡。昭宗東遷洛陽,太祖盡去天子左右,悉以梁人代之,以王殷為宣徽使,循為副使。



 循與蔣玄暉、張廷範等共與弒昭宗之謀,其後循與玄暉有隙,哀帝即位,將有事于南郊,循因與王殷讒于太祖曰:「玄暉私侍何太后,
 與廷範等奉天子郊天,冀延唐祚。」太祖大怒。是時,梁兵攻壽春,大敗而歸,哀帝遣裴迪勞軍,太祖見迪,怒甚,迪還,哀帝不敢郊。封太祖魏王,備九錫,太祖拒而不受。玄暉與宰相柳璨相次馳至梁自解,璨曰:「自古王者之興,必有封國,而唐所以不即遜位者,當先建國備九錫,然後禪也。」太祖曰:「我不由九錫作天子可乎?」璨懼,馳去。太祖遣循與王殷弒何皇后,因殺璨及玄暉、廷範等,以循為樞密副使。



 唐亡,事梁為汝州防禦使、左衛大將軍、租庸使,始改姓孔,名循。莊宗時,權知汴州。明宗自魏兵反而南,莊宗東出汜水,循持兩端,遣迎明宗於北門,迎莊
 宗於西門,供帳牲餼,其禮如一,而戒其人曰:「先至者入之。」明宗先至,遂納之。



 明宗即位,以為樞密使。明宗幸汴州,循留守東都,民有犯曲者,循族殺其家,明宗知其冤,因詔天下除曲禁,許民得造曲。循為人柔佞而險猾,安重誨尤親信之,凡循所言,無不聽用。明宗嘗欲以皇子娶重誨女,重誨以問循,循曰:「公為機密之臣,不宜與皇子婚。」重誨信之,乃止。而循陰使人白明宗,求以女妻皇子,明宗即以宋王從厚娶循女。重誨始惡其為人,出循為忠武軍節度使,徙鎮橫海,卒於鎮,年四十八,贈太尉。



 孫德昭孫德昭,鹽州五原人也。其父惟最,有材略。黃巢陷長安,
 惟最率其鄉里子弟,得義兵千人,南攻巢于咸陽,興平州將壯其所為,益以州兵二千。與破賊功,拜右金吾衛大將軍。朱玫亂京師,僖宗幸興元,惟最率兵擊賊。累遷鄜州節度使,留京師宿衛。鄜州將吏詣闕請惟最之鎮,京師民數萬與神策軍復遮留不得行,改荊南節度使,在京制置,分判神策軍,號「扈駕都」。是時,京師數亂,民皆賴以為保。



 德昭以父任為神策軍指揮使。光化三年,劉季述廢昭宗,幽之東宮,宰相崔胤謀反正,陰使人求義士可共成事者,德昭乃與孫承誨、董從實應胤,胤裂衣襟為書以盟。天復元年正月朔,未旦,季述將朝,德昭伏
 甲士道旁,邀其輿斬之,承誨等分索餘黨皆盡。昭宗聞外喧嘩,大恐。德昭馳至,扣門曰:「季述誅矣,皇帝當反正!」何皇后呼曰:「汝可進逆首!」德昭擲其首入。已而承誨等悉取餘黨首以獻,昭宗信之。德昭破鎖出昭宗,御丹鳳樓反正,以功拜靜海軍節度使,賜姓李,號「扶傾濟難忠烈功臣」,與承誨等皆拜節度使、同中書門下平章事,圖形凌煙閣,俱留京師,號「三使相」,恩寵無比。



 是時,崔胤方欲誅唐宦官,外交梁以為恃,而宦官亦倚李茂貞為扞蔽,梁、岐交爭。冬十月,宦者韓全誨劫昭宗幸鳳翔,承誨、從實皆從,而德昭獨與梁,乃率兵衛胤及百官保東街,
 趣梁兵以西,梁太祖頗德其附己,以龍鳳劍、鬥雞紗遺之。



 太祖至華州,德昭以軍禮迎謁道旁。太祖至京師,表同州留後,將行,京師民復請留,遂為兩街制置使。梁兵圍鳳翔,德昭以其兵八千屬太祖,太祖益德之,使先之洛陽,賜甲第一區。昭宗東遷,拜左威衛上將軍,以疾免。太祖即位,以烏銀帶、袍、笏、名馬賜之。疾少間,以為左衛大將軍。末帝立,拜左金吾大將軍以卒。承誨、從實至鳳翔,與宦者俱見殺。



 王敬蕘王敬蕘,潁州汝陰人也。事州為牙將。唐末,王仙芝等攻劫汝、潁間,刺史不能拒,敬蕘遂代之,即拜刺史。敬蕘為
 人狀貌魁傑,而沈勇有力,善用鐵槍,重三十斤。潁州與淮西為鄰境,數為秦宗權所攻,力戰拒之,宗權悉陷河南諸州,獨敬蕘不可下,由是潁旁諸州民,皆依敬蕘避賊。是時,所在殘破,獨潁州戶二萬。梁太祖攻淮南,道過潁州,敬蕘供饋梁兵甚厚,太祖大喜,表敬蕘沿淮指揮使。其後梁兵攻吳,龐師古死清口,敗兵亡歸,過潁,大雪,士卒飢凍,敬蕘乃沿淮積薪為作糜粥餔之,亡卒多賴以全活,太祖表敬蕘武寧軍留後,遂拜節度使。天祐三年,為左衛上將軍。太祖即位,敬蕘以疾致仕,後卒于家。



 蔣殷蔣殷,幼為王重盈養子,冒姓王氏。梁太祖取河中,以王氏舊恩錄其子孫,表殷牙將,太祖尤愛之。唐遷洛陽,殷為宣徽北院使。太祖已下襄陽,轉攻淮南,還屯正陽,哀帝遣殷勞軍。是時,哀帝方卜郊,殷與樞密使蔣玄暉等有隙,因譖之太祖,言玄暉等教天子卜郊祈天,且待諸侯助祭者以謀興復,太祖大怒,哀帝為改卜郊。是時,太祖將有篡弒之謀,何太后嘗泣涕叩頭為玄暉等言:「梁王禪位後,願全唐家子母。」殷乃誣玄暉嘗私侍太后,太祖斬玄暉及張廷範、柳璨等,遣殷弒太后於積善宮。哀帝下詔慚愧,自言以母后故無以奉天,乃卒不郊。庶人
 友珪與殷善,友珪弒太祖自立,拜殷武寧軍節度使。末帝即位,以福王友璋代殷,殷不受代。王瓚亦王氏子,懼為殷所累,乃言殷非王氏子,本姓蔣。末帝詔削官爵,還其姓,遣牛存節討之,殷舉族自燔死。



\end{pinyinscope}