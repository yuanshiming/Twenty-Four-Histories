\article{卷二十一梁臣傳第九}

\begin{pinyinscope}

 嗚呼!孟子謂「春秋無義戰」,予亦以謂五代無全臣。無者,非無一人,蓋僅有之耳,餘得死節之士三人焉。其仕不及于二代者,各以其國繫之,作梁、唐、晉、漢、周臣傳。其餘仕非一代,不可以國繫之者,作《雜傳》。夫入于雜,誠君子之所羞,而一代之臣,未必皆可貴也,覽者詳其善惡焉。



 敬翔敬翔,字子振,同州馮翊人也,自言唐平陽王暉之後。少好學,工書檄。乾符中舉進士不中,乃客大梁。翔同里人
 王發為汴州觀察支使,遂往依焉。久之,發無所薦引,翔客益窘,為人作箋刺,傳之軍中。太祖素不知書,翔所作皆俚俗語,太祖愛之,謂發曰:「聞君有故人,可與俱來。」翔見太祖,太祖問曰:「聞子讀《春秋》,《春秋》所記何等事?」翔曰:「諸侯爭戰之事耳。」太祖曰:「其用兵之法可以為吾用乎?」翔曰:「兵者,應變出奇以取勝,《春秋》古法,不可用於今。」太祖大喜,補以軍職,非其所好,乃以為館驛巡官。太祖與蔡人戰汴郊,翔時時為太祖謀畫,多中,太祖欣然,以謂得翔之晚,動靜輒以問之。太祖奉昭宗自岐還長安,昭宗召翔與李振升延喜樓勞之,拜翔太府卿。



 初,太祖常
 侍殿上,昭宗意衛兵有能擒之者,乃佯為鞋結解,以顧太祖,太祖跪而結之,而左右無敢動者,太祖流汗浹背,由此稀復進見。昭宗遷洛陽,宴崇勛殿,酒半起,使人召太祖入內殿,將有所託。太祖益懼,辭以疾。昭宗曰:「卿不欲來,可使敬翔來。」太祖遽麾翔出,翔亦佯醉去。



 太祖已破趙匡凝,取荊、襄,遂攻淮南。翔切諫,以謂新勝之兵,宜持重以養威。太祖不聽。兵出光州,遭大雨,幾不得進,進攻壽州,不克,而多所亡失,太祖始大悔恨。歸而忿躁,殺唐大臣幾盡,然益以翔為可信任。梁之篡弒,翔之謀為多。太祖即位,以唐樞密院故用宦者,乃改為崇政院,以
 翔為使。遷兵部尚書、金鑾殿大學士。



 翔為人深沉有大略,從太祖用兵三十餘年,細大之務必關之。翔亦盡心勤勞,晝夜不寐,自言惟馬上乃得休息。而太祖剛暴難近,有所不可,翔亦未嘗顯言,微開其端,太祖意悟,多為之改易。



 太祖破徐州,得時溥寵姬劉氏,愛幸之,劉氏故尚讓妻也,乃以妻翔。翔已貴,劉氏猶侍太祖,出入臥內如平時,翔頗患之。劉氏誚翔曰:「爾以我嘗失身於賊乎?



 尚讓,黃家宰相;時溥,國之忠臣。以卿門地,猶為辱我,請從此決矣!」翔以太祖故,謝而止之。劉氏車服驕侈,別置典謁,交結籓鎮,權貴往往附之,寵信言事不下於翔。當
 時貴家,往往效之。



 太祖崩,友珪立,以翔先帝謀臣,懼其圖己,不欲翔居內職,乃以李振代翔為崇政使,拜翔中書侍郎、同中書門下平章事。翔以友珪畏己,多稱疾,未嘗省事。



 末帝即位,趙巖等用事,頗離間舊臣,翔愈鬱鬱不得志。其後,梁盡失河北,與晉相拒楊劉,翔曰:「故時河朔半在,以先帝之武,御貔虎之臣,猶不得志於晉。



 今晉日益彊,梁日益削,陛下處深宮之中,所與計事者,非其近習,則皆親戚之私,而望成事乎?臣聞晉攻楊劉,李亞子負薪渡水,為士卒先。陛下委蛇守文,以儒雅自喜,而遣賀瑰為將,豈足當彼之餘鋒乎?臣雖憊矣,受國恩深,
 若其乏材,願得自效。」巖等以翔為怨言,遂不用。



 其後,王彥章敗于中都,末帝懼,召段凝於河上。是時,梁精兵悉在凝軍,凝有異志,顧望不來。末帝遽呼翔曰:「朕居常忽卿言,今急矣,勿以為懟,卿其教我當安歸?」翔曰:「臣從先帝三十餘年,今雖為相,實朱氏老奴爾,事陛下如郎君,以臣之心,敢有所隱?陛下初用段凝,臣已爭之,今凝不來,敵勢已近,欲為陛下謀,則小人間之,必不見聽。請先死,不忍見宗廟之亡!」君臣相向慟哭。



 翔與李振俱為太祖所信任,莊宗入汴,詔赦梁群臣,李振喜謂翔曰:「有詔洗滌,將朝新君。」邀翔欲俱入見。翔夜止高頭車坊,將旦,左
 右報曰:「崇政李公入朝矣!」翔歎曰:「李振謬為丈夫矣!復何面目入梁建國門乎?」乃自經而卒。



 朱珍李唐賓附硃珍,徐州豐人也。少與龐師古等俱從梁太祖為盜。珍為將,善治軍選士,太祖初鎮宣武,珍為太祖創立軍制,選將練兵甚有法。太祖得諸將所募兵及佗降兵,皆以屬珍,珍選將五十餘人,皆可用。梁敗黃巢、破秦宗權、東並兗鄆,未嘗不在戰中,而常勇出諸將。太祖與晉王東逐黃巢,還過汴,館之上源驛,太祖使珍夜以兵攻之,晉王亡去,珍悉殺其麾下兵。義成軍亂,逐安師儒,師儒奔梁。太祖遣珍以兵趨滑州,道遇大雪,珍趣兵疾馳,一夕
 至城下,遂乘其城。義成軍以為方雪,不意梁兵來,不為備,遂下之。



 秦宗權遣盧瑭、張晊等攻梁,是時梁兵尚少,數為宗權所困。太祖乃拜珍淄州刺史,募兵於淄青。珍偏將張仁遇白珍曰:「軍中有犯令者,請先斬而後白。」珍曰:「偏將欲專殺邪?」立斬仁遇以徇軍,軍中皆感悅。珍得所募兵萬餘以歸,太祖大喜曰:「賊在吾郊,若踐吾麥,奈何!今珍至,吾事濟矣!且賊方息兵養勇,度吾兵少,而未知珍來,謂吾不過堅守而已,宜出其不意以擊之。」乃出兵擊敗晊等,宗權由此敗亡,而梁軍威大振,以得珍兵故也。



 珍從太祖攻朱宣,取曹州,執其刺史丘弘禮。又
 取濮州,刺史朱裕奔于鄆州。



 太祖乃還汴,留珍攻鄆州。珍去鄆二十里,遣精兵挑之,鄆人不出。朱裕詐為降書,陰使人召珍,約開門為內應。珍信之,夜率其兵叩鄆城門,朱裕登陴,開門內珍軍,珍軍已入甕城而垂門發,鄆人從城上磔石以投之,珍軍皆死甕城中,珍僅以身免,太祖不之責也。



 魏博軍亂,囚樂彥貞。太祖遣珍救魏,珍破黎陽、臨河、李固,分遣聶金、范居實等略澶州,殺魏豹子軍二千於臨黃。珍威振河朔。魏人殺彥貞,珍乃還。梁攻徐州,遣珍先攻下豐縣,又敗時溥於吳康,與李唐賓等屯蕭縣。



 唐賓者,陜人也。初為尚讓偏將,與太祖
 戰尉氏門,為太祖所敗,唐賓乃降梁。



 梁兵攻掠四方,唐賓常與珍俱,與珍威名略等,而驍勇過之,珍戰每小卻,唐賓佐之乃大勝。珍嘗私迎其家置軍中,太祖疑珍有異志,遣唐賓伺察之。珍與唐賓不協,唐賓不能忍,夜走還宣武,珍單騎追之,交訴太祖前。太祖兩惜其材,為和解之。



 珍屯蕭縣,聞太祖將至,戒軍中治館廄以待。唐賓部將嚴郊治廄失期,軍吏督之,郊訴于唐賓,唐賓以讓珍,珍怒,拔劍而起,唐賓拂衣就珍,珍即斬之,遣使者告唐賓反。使者晨至梁,敬翔恐太祖暴怒不可測,乃匿使者,至夜而見之,謂雖有所發,必須明旦,冀得少緩其事
 而圖之。既夕,乃引珍使者入見,太祖大驚,然已夜矣,不能有所發,翔因從容為太祖畫。明日,佯收唐賓妻子下獄。因如珍軍,去蕭一舍,珍迎謁,太祖命武士執之。諸將霍存等十餘人叩頭救珍,太祖大怒,舉胡床擲之曰:「方珍殺唐賓時,獨不救之邪!」存等退,珍遂縊死。



 龐師古龐師古,曹州南華人也,初名從。梁太祖鎮宣武,初得馬五百匹為騎兵,乃以師古將之,從破黃巢、秦宗權,皆有功。太祖攻時溥未下,留兵屬師古守之,師古取其宿遷,進屯呂梁。溥以兵二萬出戰,師古敗之,斬首二千級。孫儒逐楊行密,取揚州,淮南大亂,太祖遣師古渡淮攻儒,
 為儒所敗。是時,朱珍、李唐賓已死,師古與霍存分將其兵。郴王友裕攻徐州,朱瑾以兵救時溥,友裕敗溥於石佛山,瑾收餘兵去。太祖以友裕可追而不追,奪其兵以屬師古。師古攻破徐州,斬溥,太祖表師古徐州留後。梁兵攻鄆州,臨濟水,師古徹木為橋,夜以中軍先濟。朱宣走中都,見殺。



 太祖已下兗、鄆,乃遣師古與葛從周攻楊行密于淮南,師古出清口,從周出安豐。師古自其微時事太祖,為人謹甚,未嘗離左右,及為將出兵,必受方略以行,軍中非太祖命,不妄動。師古營清口,地勢卑,或請就高為柵,師古以非太祖命不聽。淮人決水浸之,請者
 告曰:「淮人決河,上流水至矣!」師古以為搖動士卒,立斬之。已而水至,兵不能戰,遂見殺。



 嗚呼,兵之勝敗,豈易言哉!梁兵彊於天下,而吳人號為輕弱,然師古再舉擊吳,輒再敗以死。其後太祖自將出光山,攻壽春,然亦敗也。蓋自高駢死,唐以梁兼統淮南,遂與孫、楊爭,凡三十年間,三舉而三敗。以至彊遭至弱而如此,此其不可以理得也。兵法固有以寡而敗眾、以弱而勝彊者,顧吳豈足以知之哉!豈非適與其機會邪?故曰:「兵者凶器,戰者危事也。」可不慎哉!



 葛從周葛從周,字通美,濮州甄城人也。少從黃巢,敗降梁。從
 太祖攻蔡州,太祖墜馬,從周扶太祖復騎,與敵步鬥傷面,身被數瘡,偏將張延壽從旁擊之,從周得與太祖俱去。太祖盡黜諸將,獨用從周、延壽為大將。



 秦宗權掠地潁、亳,及梁兵戰于焦夷,從周獲其將王涓一人。從朱珍收兵淄青,遇東兵輒戰,珍得兵歸,從周功為多。張全義襲李罕之於河陽,罕之奔晉,召晉兵以攻全義,全義乞兵於梁,太祖遣從周、丁會等救之,敗晉兵於沇河。潞州馮霸殺晉守將李克恭以降梁,太祖遣從周入潞州,晉兵攻之,從周不能守,走河陽。太祖攻魏,從周與丁會先下黎陽、臨河,會太祖於內黃,敗魏兵於永定橋。從丁會攻宿
 州,以水浸其城,遂破之。太祖攻朱瑾於兗州,未下,留從周圍之,瑾閉壁不出,從周詐言救兵至,陽避之高吳,夜半潛還城下,瑾以謂從周已去,乃出兵收外壕,從周掩擊之,殺千餘人。



 晉攻魏,魏人求救,太祖遣侯言救魏,言築壘于洹水。太祖怒言不出戰,遣從周代言。從周至軍,益閉壘不出,而鑿三闇門以待,晉兵攻之,從周以精兵自闇門出擊,敗晉王兵。晉王怒,自將擊從周,從周雖大敗,而梁兵擒其子落落,送于魏,斬之。遂徙攻鄆州,擒朱宣於中都,又攻兗州,走朱瑾。太祖表從周兗州留後,以兗、鄆兵攻淮南,出安豐,會龐師古于清口。從周行至濠州,
 聞師古死,遽還,至渒河將渡而淮兵追之,從周亦大敗。是時,晉兵出山東攻相、衛,太祖遣從周略地山東,下洺州,斬其刺史邢善益;又下邢州,走其刺史馬師素;又下磁州,殺其刺史袁奉滔。五日而下三州。太祖乃表從周兼邢州留後。



 劉仁恭攻魏,已屠貝州,羅紹威求救于梁,從周會太祖救魏,入于魏州。燕兵攻館陶門,從周以五百騎出戰,曰:「大敵在前,何可返顧!」使閉門而後戰。破其八柵,燕兵走,追至於臨清,擁之御河,溺死者甚眾。太祖以從周為宣義行軍司馬。



 太祖遣從周攻劉守文于滄州,以蔣暉監其軍。守文求救于其父仁恭,仁恭以燕兵
 救之,暉語諸將曰:「吾王以我監諸將,今燕兵來,不可迎戰,宜縱其入城,聚食倉廩,使兩困而後取之。」諸將頗以為然。從周怒曰:「兵在上將,豈監軍所得言!且暉之言乃常談爾,勝敗之機在吾心,暉豈足以知之!」乃勒兵逆仁恭于乾寧,戰于老鴉堤,仁恭大敗,斬首三萬餘級,獲其將馬慎交等百餘人,馬三千匹。是時,守文亦求救於晉,晉為攻邢、洺以牽之,從周遽還,敗晉兵于青山。遂從太祖攻鎮州,下臨城,王熔乞盟,太祖表從周泰寧軍節度使。



 從氏叔琮攻晉太原,不克。梁兵西攻鳳翔,青州王師範遣其將劉掞襲兗州,從周家屬為掞所得,厚遇之而
 不殺。太祖還自鳳翔,乃遣從周攻掞,從周卒招降掞。



 太祖即位,拜左金吾衛上將軍,以疾致仕,拜右衛上將軍,居于偃師。末帝即位,拜昭義軍節度使、陳留郡王,食其俸于家。卒,贈太尉。



 霍存霍存,洺州曲周人也。少從黃巢,巢敗,存乃降梁。存為將驍勇,善騎射。秦宗權攻汴,存以三千人夜破張晊柵,又以騎兵破秦賢,殺三千人,敗晊於赤岡。從朱珍掠淄青、龐師古攻時溥,皆有功。朱珍與李唐賓俱死,乃以龐師古代珍、存代唐賓以攻溥,破碭山,存獲其將石君和等五十人。梁攻宿州,葛從周引水浸之,丁會與存戰城
 下,遂下之。從攻潞州,與晉人遇,戰馬牢川,存入則當其前,出則為其殿,晉人卻,遂東攻魏,取淇門,殺三千人。梁得曹州,太祖以存為刺史,兼諸軍都指揮使。梁攻鄆州,硃瑾來救,梁諸將或勸太祖縱瑾入鄆,耗其食,堅圍勿戰,以此可俱弊。太祖曰:「瑾來必與時溥俱,不若遣存邀之。」存伏兵蕭縣,已而瑾果與溥俱出迷離,存發伏擊之,遂敗瑾等於石佛山,存中流矢卒。太祖已即位,閱騎兵於繁臺,顧諸將曰:「使霍存在,豈勞吾親閱邪!諸君寧復思之乎?」佗日語又如此。



 張存敬張存敬,譙郡人也。為人剛直有膽勇,少事梁太祖為將,
 善因危窘出奇計。李罕之與晉人攻張全義於河陽,太祖遣存敬與丁會等救之,罕之解圍去。太祖以存敬為諸軍都虞候。太祖攻徐、兗,以存敬為行營都指揮使。從葛從周攻滄州,敗劉仁恭於老鴉堤。還攻王熔於鎮州,入其城中,取其馬牛萬計。遷宋州刺史。復從諸將攻幽州,存敬取其瀛、莫、祁、景四州。梁攻定州,與王處直戰懷德驛,大敗之,枕尸十餘里。梁已下鎮、定,乃遣存敬攻王珂于河中,存敬出含山,下晉、絳二州,珂降于梁。太祖表存敬護國軍留後,復徙宋州刺史,未至,卒于河中,贈太傅。



 存敬子仁穎、仁愿。仁愿有孝行,存敬卒,事其兄仁穎,
 出必告,反必面,如事父之禮。仁愿曉法令,事梁、唐、晉,常為大理卿,卒,贈祕書監。



 符道昭符道昭,蔡州人也。為秦宗權騎將,宗權敗,道昭流落無所依,後依鳳翔李茂貞,茂貞愛之,養以為子,名繼遠。梁攻茂貞,道昭與梁兵戰,屢敗,乃歸梁,太祖表道昭秦州節度使,以亂不果行。太祖為元帥,初開府,而李周彞以鄜州降,以為左司馬,擇右司馬難其人,及得道昭,乃授之。羅紹威將誅其牙兵,惡魏兵彊,未敢發,求梁為助。太祖乃悉發魏兵使攻燕,而遣馬嗣勳助紹威誅牙兵。牙兵已誅,魏兵在外者聞之皆亂,魏將左行遷據歷亭、史
 仁遇據高唐以叛,道昭等從太祖悉破之。道昭為將,勇於犯敵而少成算,每戰先發,多敗,而周彞等繼之乃勝。開平元年與康懷英等攻潞州,築夾城為蚰蜒塹以圍之,逾年不能下,晉兵攻破夾城,道昭戰死。



 劉捍劉捍,開封人也。為人明敏有威儀,善擯贊。太祖初鎮宣武,以為客將,使從朱珍募兵淄青。太祖北攻鎮州,與王鎔和,遣捍見鎔,鎔軍未知梁意,方嚴兵,捍馳一騎入城中,諭鎔以太祖意,鎔乃聽命。梁兵攻定州,降王處直,捍復以一騎入慰城中。太祖圍鳳翔,遣捍入見李茂貞計事。唐昭宗召見,問梁軍中事,稱旨,賜以錦袍,拜登州刺
 史,賜號「迎鑾毅勇功臣。」梁兵攻淮南,遣捍先之淮口,築馬頭下浮橋以渡梁兵。太祖出光山攻壽州,又使捍作浮橋於淮北,以渡歸師。拜宋州刺史。太祖即位,遷左天武指揮使、元從親軍都虞候、左龍虎統軍,出為佑國軍留後。同州劉知俊反,以賂誘捍將吏,執捍而去,知俊械之,送於李茂貞,見殺。太祖哀之,贈捍太傅。



 寇彥卿寇彥卿,字俊臣,開封人也。世事宣武軍為牙將。太祖初就鎮,以為通引官,累遷右長直都指揮使,領洺州刺史。羅紹威將誅牙軍,太祖遣彥卿之魏計事,彥卿陰為紹威計畫,乃悉誅牙軍。



 彥卿身長八尺,隆準方面,語音如
 鐘,工騎射,好書史,善伺太祖意,動作皆如旨。太祖嘗曰:「敬翔、劉捍、寇彥卿皆天為我生之。」其愛之如此。賜以所乘愛馬「一丈烏」。太祖圍鳳翔,以彥卿為都排陣使,彥卿乘烏馳突陣前,太祖目之曰:「真神將也!」



 初,太祖與崔胤謀,欲遷都洛陽,而昭宗不許。其後昭宗奔於鳳翔,太祖以兵圍之,昭宗既出,明年,太祖以兵至河中,遣彥卿奉表迫請遷都。彥卿因悉驅徙長安居人以東,人皆拆屋為筏,浮渭而下,道路號哭,仰天大罵曰:「國賊崔胤、朱溫使我至此!」昭宗亦顧瞻陵廟,傍徨不忍去,謂其左右為俚語云:「紇干山頭凍死雀,何不飛去生處樂。」相與泣下
 沾襟。昭宗行至華州,遣人告太祖以何皇后有娠,願留華州待冬而行。太祖大怒,顧彥卿曰:「汝往趣官家來,不可一日留也。」



 彥卿復馳至華,即日迫昭宗上道。



 太祖即位,拜彥卿感化軍節度使。歲餘,召為左金吾衛大將軍,充金吾衙仗使。



 彥卿晨朝至天津橋,民梁現不避道,前驅捽現投橋上石欄以死。彥卿見太祖自首,太祖惜之,詔彥卿以錢償現家以贖罪。御史司憲崔沂劾奏彥卿,請論如法,太祖不得已,責授彥卿左衛中郎將。復拜相州防禦使,遷河陽節度使。



 太祖遇弒,彥卿出太祖畫像事之如生,嘗對客語先朝,必涕泗交下。末帝即位,徙鎮
 威勝。彥卿明敏善事人,而怙寵作威,好誅殺,多猜忌。卒於鎮,年五十七。



\end{pinyinscope}