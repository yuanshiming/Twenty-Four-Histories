\article{卷五十五雜傳第四十三}

\begin{pinyinscope}

 劉昫劉昫,涿州歸義人也。昫為人美風儀,與其兄暄、弟皞,皆以好學知名燕、薊之間。後為定州王處直觀察推官。處直為子都所囚,昫兄暄亦為怨家所殺,昫乃避之滄州。唐莊宗即位,拜昫太常博士,以為翰林學士,明宗時,累遷兵部侍郎居職。



 明宗素重昫而愛其風韻,遷端明殿學士。長興三年,拜中書侍郎兼刑部尚書、同中書門下平章事,昫詣中興殿門謝,是日大祠不坐,句入謝端
 明殿。昫自端明殿學士拜相,當時以此為榮。廢帝入立,遷吏部尚書、門下侍郎,監修國史。



 初,廢帝入,問三司使王玫:「帑廩之數幾何?」玫言:「其數百萬。」及責以賞軍而無十一,廢帝大怒,罷玫,命昫兼判三司。昫性察,而嫉三司蠹敝尤甚,乃句計文簿,核其虛實,殘租積負悉蠲除之。往時吏幸積年之負蓋而不發,因以把持州縣求賄賂,及昫一切蠲除,民間歡然以為德,而三司吏皆沮怨。先是,馮道與昫為姻家而同為相,道罷,李愚代之。愚素惡道為人,凡事有稽失者,必指以誚昫曰:「此公親家翁所為也!」昫性少容恕,而愚特剛介,遂相詆詬。相府史吏惡
 此兩人剛直,因共揚言,其事聞,廢帝並罷之,以昫為右僕射。是時,三司諸吏提印聚立月華門外,聞宣麻罷昫相,皆歡呼相賀曰:「自此我曹快活矣!」



 昫在相位,不習典故。初,明宗崩,太常卿崔居儉以故事當為禮儀使,居儉辭以祖諱蠡。馮道改居儉祕書監,居儉怏怏失職。中書舍人李詳為居儉誥詞,有「聞名心懼」之語,昫輒易曰「有恥且格」。居儉訴曰:「名諱有令式,予何罪也?」



 當時聞者皆傳以為笑。及為僕射,入朝遇雨,移班廊下,御史臺吏引僕射立中丞御史下,昫詰吏以故事,自宰相至臺省皆不能知。是時,馮道罷相為司空。自隋、唐以來,三公無職
 事,不特置,及道為司空,問有司班次,亦皆不能知,由是不入朝堂,俟臺官、兩省入而後入,宰相出則隨而出。至昫為僕射,自以由宰相罷,與道同,乃隨道出入,有司不能彈正,而議者多竊笑之。



 晉高祖時,張從賓反,殺皇子重乂於洛陽,乃以昫為東都留守,判鹽鐵。開運中,拜司空、同中書門下平章事,復判三司。契丹犯京師,昫以目疾罷為太保,是歲卒,年六十。



 盧文紀盧文紀,字子持,其祖簡求,為唐太原節度使,父嗣業,官至右補闕。文紀舉進士,事梁為刑部侍郎、集賢殿學士。唐明宗時,為御史中丞。初上事,百官臺參,吏白諸道進
 奏官賀,文紀問:「當如何?」吏對曰:「朝廷在長安時,進奏官見大夫、中丞如胥史。自唐衰,天子微弱,諸侯強盛,貢奉不至,朝廷姑息方鎮,假借邸吏,大夫、中丞上事,進奏官至客次通名,勞以茶酒而不相見,相傳以為故事。」



 文紀曰:「吾雖德薄,敢隳舊制?」因遣吏諭之。進奏官奮臂喧然欲去,不得已入見,文紀據床端笏,臺吏通名贊拜,既出,恚怒不自勝,訴於樞密使安重誨。重誨曰:「吾不知故事,可上訴于朝。」即相率詣閣門求見以壯訴。明宗問宰相趙鳳:「進奏吏比外何官?」鳳曰:「州縣發遞知後之流也。」明宗怒曰:「乃吏卒爾,安得慢吾法官!」皆杖而遣之。文紀又
 請悉復中外官校考法,將相天子自書之,詔雖施行,而官卒不考。歲餘,遷工部尚書。



 文紀素與宰相崔協有隙,協除工部郎中於鄴,文紀以鄴與其父名同音,大怒,鄴赴省參上,文紀不見之,因請連假。已而鄴奉使未行,文紀即出視事,鄴因醉忿自經死,文紀坐貶石州司馬。久之,為秘書監、太常卿。奉使于蜀,過鳳翔。時廢帝為鳳翔節度使,文紀為人形貌魁偉、語音瑯然,廢帝奇之。後廢帝入立,欲擇宰相,問於左右,左右皆言:「文紀及姚顗有人望。」廢帝因悉書清望官姓名內琉璃瓶中,夜焚香祝天,以箸挾之,首得文紀,欣然相之,乃拜中書侍郎、同中
 書門下平章事。



 是時,天下多事,廢帝數以責文紀。文紀因請罷五日起居,復唐故事,開延英,冀得從容奏議天下事。廢帝以謂五日起居,明宗所以見群臣也,不可罷,而便殿論事,可以從容,何必延英。因詔宰相有事,不以時詣閣門請對。晉高祖起太原,廢帝北征,過拜徽陵,休仗舍,顧文紀曰:「吾自鳳翔識卿,不以常人為待,自卿為相,詢于輿議,皆云可致太平,今日使吾至此,卿宜如何?」文紀皇恐謝罪。廢帝至河陽,文紀勸帝扼橋自守,不聽。晉高祖入立,罷為吏部尚書,累遷太子太師,致仕。周太祖入立,即拜司空于家。卒,年七十六,贈司徒。



 馬胤孫馬胤孫,字慶先,棣州商河人也。為人懦暗,少好學,學韓愈為文章。舉進士,為唐潞王從珂河中觀察支使。從珂為楊彥溫所逐,罷居於京師里第,胤孫從而不去。



 從珂為京兆尹,徙鎮鳳翔,胤孫常從之,以為觀察判官。潞王將舉兵反,與將吏韓昭胤等謀議已定,召胤孫告之曰:「受命移鎮,路出京師,何向為便?」胤孫曰:「君命召,不俟駕。今大王為國宗屬,而先帝新棄天下,臨喪赴鎮,臣子之忠也。」



 左右皆笑其愚,然從珂心獨重之。廢帝入立,以為戶部郎中、翰林學士。久之,拜中書侍郎、同中書門下平章事。



 胤孫不通世務,故事多壅塞。是時,馮道罷匡國軍
 節度使,拜司空。司空自唐已來無特拜者,有司不知故事,朝廷議者紛然,或曰司空三公,宰相職也,當參與大政,而宰相盧文紀獨以謂司空之職,祭祀掃除而已。胤孫皆不能決。時劉昫亦罷相為僕射,右散騎常侍孔昭序建言:「常侍班當在僕射前。」胤孫責御史臺檢例,臺言:「故事無所見,據今南北班位,常侍在前。」胤孫即判臺狀施行,劉昫大怒。



 崔居儉揚言於朝曰:「孔昭序解語,是朝廷無解語人也!且僕射師長百寮,中丞、大夫就班脩敬,而常侍在南宮六卿之下,況僕射乎?昭序癡兒,豈識事體?」朝士聞居儉言,流議稍息。胤孫臨事多不能決,當時
 號為「三不開」,謂其不開口以論議,不開印以行事,不開門以延士大夫也。晉兵起太原,廢帝至河陽,是時勢已危迫,胤孫自洛來朝行在,人皆冀其有所建言,胤孫獻綾三百匹而已。晉高祖入立,罷歸田里。



 胤孫既學韓愈為文,故多斥浮屠氏之說,及罷歸,乃反學佛,撰《法喜集》、《佛國記》行於世。時人誚之曰:「佞清泰不徹,乃來佞佛。」清泰,廢帝年號也。



 人有戲胤孫曰:「公素慕韓愈為人,而常誦傅奕之論,今反佞佛,是佛佞公邪,公佞佛邪?」胤孫答曰:「豈知非佛佞我也?」時人傳以為笑。後以太子賓客分司居于洛陽,周廣順中卒。胤孫卒後,其家婢有為胤孫
 語者。初,崔協為明宗相,在位無所發明,既死,而有降語其家,胤孫又然。時人嘲之曰:「生不能言,死而後語」



 云。



 姚顗姚顗,字百真,京兆長安人也。少蠢,不脩容止,時人莫之知,中條山處士司空圖一見以為奇,以其女妻之。舉進士,事梁為翰林學士、中書舍人。唐莊宗滅梁,貶復州司馬,已而以為左散騎常侍兼吏部侍郎、尚書左丞。廢帝欲擇宰相,選當時清望官知名於世者,得盧文紀及顗,乃拜顗中書侍郎、同中書門下平章事。顗為人仁恕,不知錢陌銖兩之數,御家無法,在相位齪齪無所為。唐制吏部分為三銓,尚書一人曰尚書銓,侍郎二人曰中銓、
 東銓。每歲集以孟冬三旬,而選盡季春之月。



 天成中,馮道為相,建言:「天下未一,選人歲纔數百,而吏部三銓分注,雖曰故事,其實徒繁而無益。」始詔三銓合為一,而尚書、侍郎共行選事。至顗與盧文紀為相,復奏分銓為三。而循資、長定舊格,歲久多舛,因增損之。選人多不便之,往往邀遮宰相,喧訴不遜,顗等無如之何,廢帝為下詔書禁止。晉高祖立,罷顗為戶部尚書。卒,年七十五,卒之日,家無餘貲,尸不能斂,官為賵贈乃能斂,聞者哀憐之。



 劉岳劉岳,字昭輔,洛陽人也。唐民部尚書政會之八代孫,崇龜、崇望其諸父也。



 岳名家子,好學,敏於文辭,善談論。舉
 進士,事梁為左拾遺、侍御史。末帝時,為翰林學士,累官至兵部侍郎。梁亡,貶均州司馬,復用為太子詹事。唐明宗時,為吏部侍郎。故事,吏部文武官告身,皆輸朱膠紙軸錢然後給,其品高者則賜之,貧者不能輸錢,往往但得敕牒而無告身。五代之亂,因以為常,官卑者無復給告身,中書但錄其制辭,編為敕甲。岳建言,以謂「制辭或任其材能,或褒其功行,或申以訓誡,而受官者既不給告身,皆不知受命之所以然,非王言所以告詔也。請一切賜之。」由是百官皆賜告身,自岳始也。



 宰相馮道世本田家,狀貌質野,朝士多笑其陋。道旦入朝,兵部侍郎任
 贊與岳在其後,道行數反顧,贊問岳:「道反顧何為?」岳曰:「遺下《兔園冊》爾。」



 《兔園冊》者,鄉校俚儒教田夫牧子之所誦也,故岳舉以誚道。道聞之大怒,徙岳秘書監。其後李愚為相,遷岳太常卿。



 初,鄭餘慶嘗採唐士庶吉凶書疏之式,雜以當時家人之禮,為《書儀》兩卷。



 明宗見其起復、冥昏之制,歎曰:』儒者所以隆孝悌而敦風俗,且無金革之事,起復可乎?婚,吉禮也,用於死者可乎?」乃詔岳選文學通知古今之士,共刪定之。



 岳與太常博士段顒、田敏等增損其書,而其事出鄙俚,皆當時家人女子傳習所見,往往轉失其本,然猶時有《禮》之遺制。其後亡失,愈
 不可究其本末,其婚禮親迎,有女坐婿鞍合髻之說,尤為不經。公卿之家,頗遵用之。至其久也,又益訛謬可笑,其類甚多。岳卒于官,年五十六,贈吏部尚書。子溫叟。



 嗚呼,甚矣,人之好為禮也!在上者不以禮示之,使人不見其本,而傳其習俗之失者,尚拳拳而行之。五代干戈之亂,不暇於禮久矣!明宗武君,出於夷狄,而不通文字,乃能有意使民知禮。而岳等皆當時儒者,卒無所發明,但因其書增損而已。然其後世士庶吉凶,皆取岳書以為法,而十又轉失其三四也,可勝歎哉!



 馬縞馬縞,不知其世家,少舉明經,又舉宏詞。事梁為太常少
 卿,以知禮見稱于世。



 唐莊宗時,累遷中書舍人、刑部侍郎、權判太常卿。明宗入立,繼唐太祖、莊宗而不立親廟。縞言:「漢諸侯王入繼統者,必別立親廟,光武皇帝立四廟于南陽,請如漢故事,立廟以申孝享。」明宗下其議,禮部尚書蕭頃等請如縞議。宰相鄭玨等議引漢桓、靈為比,以謂靈帝尊其祖解瀆亭侯淑為孝元皇,父萇為孝仁皇,請下有司定謚四代祖考為皇,置園陵如漢故事。事下太常,博士王丕議漢桓帝尊祖為孝穆皇帝,父為孝崇皇帝。縞以謂孝穆、孝崇有皇而無帝,惟吳孫皓尊其父和為文皇帝,不可以為法。右僕射李琪等議與
 縞同。明宗詔曰:「五帝不相襲禮,三王不相沿樂,惟皇與帝,異世殊稱。爰自嬴秦,已兼厥號,朕居九五之位,為億兆之尊,奈何總二名於眇躬,惜一字於先世。」乃命宰臣集百官於中書,各陳所見。李琪等請尊祖禰為皇帝,曾高為皇。宰相鄭玨合群議奏曰:「禮非天降而本人情,可止可行,有損有益。今議者引古,以漢為據,漢之所制,夫復何依?開元時,尊皋陶為德明皇帝,涼武昭王為興聖皇帝,皆立廟京師,此唐家故事也。臣請四代祖考皆加帝如詔旨,而立廟京師。」詔可其加帝,而立廟應州。



 劉岳脩《書儀》,其所增損,皆決於縞。縞又言:「縗麻喪紀,所以別
 親疏,辨嫌疑。《禮》,叔嫂無服,推而遠之也。唐太宗時,有司議為兄之妻服小功五月,今有司給假為大功九月,非是。」廢帝下其議,太常博士段顒議「嫂服給假以大功者,令文也,令與禮異者非一,而喪服之不同者五。《禮》,姨舅皆服小功,令皆大功。妻父母婿外甥皆服緦,令皆小功。禮、令之不可同如此。」右贊善大夫趙咸又議曰:「喪,與其易也,寧戚。《儀禮》五服,或以名加,或因尊制,推恩引義,各有所當。據《禮》為兄之子妻服大功,今為兄之子母服小功,是輕重失其倫也。



 以名則兄子之妻疏,因尊則嫂非卑,嫂服大功,其來已久。令,國之典,不可滅也。」



 司封郎中
 曹琛,請下其議,并以《禮》、令之違者定議。詔尚書省集百官議。左僕射劉昫等議曰:「令於喪服無正文,而嫂服給大功假,乃假寧附令,而敕無年月,請凡喪服皆以《開元禮》為定,下太常具五服制度,附于令。」令有五服,自縞始也。



 縞明宗時嘗坐覆獄不當,貶綏州司馬。復為太子賓客,遷戶部、兵部侍郎。盧文紀作相,以其迂儒鄙之,改國子祭酒。卒,年八十,贈兵部尚書。



 崔居儉崔居儉,清河人也。祖蠡、父蕘皆為唐名臣。居儉美文辭,風骨清秀,少舉進士。梁貞明中,為中書舍人、翰林學士、御史中丞。唐莊宗時,為刑部侍郎、太常卿。崔氏自後魏、
 隋、唐與盧、鄭皆為甲族,吉凶之事,各著家禮。至其後世子孫,專以門望自高,為世所嫉。明宗崩,居儉以故事為禮儀使,居儉以祖諱蠡,辭不受,宰相馮道即徙居儉為秘書監。居儉歷兵、吏部侍郎、尚書左丞、戶部尚書。晉天福四年卒,年七十,贈右僕射。居儉拙於為生,居顯官,衣常乏,死之日貧不能葬,聞者哀之。



 崔棁崔棁,字子文,深州安平人也。父涿,唐末為刑部郎中。棁少好學,頗涉經史,工於文辭。遭世亂,寓居于滑臺,不遊里巷者十餘年,人罕識其面。梁貞明三年,舉進士甲科,開封尹王瓚辟掌奏記。棁性至孝,其父涿病,不肯服藥,
 曰:「死生有命,何用藥為?」棁屢進醫藥,不納。每賓客問疾者,棁輒迎拜門外,泣涕而告之,涿終不服藥而卒。棁居喪哀毀,服除,唐明宗以為監察御史,不拜,踰年再命,乃拜。累遷都官郎中、翰林學士。



 晉高祖時,以戶部侍郎為學士承旨,權知天福二年貢舉。初,棁為學士,嘗草制,為宰相桑維翰所改。棁以唐故事學士草制有所改者當罷職,乃引經據爭之,維翰頗不樂。而棁少專於文學,不能蒞事,維翰乃命棁知貢舉,棁果不能舉職。時有進士孔英者,素有醜行,為當時所惡。棁既受命,往見維翰,維翰素貴,嚴尊而語簡,謂棁曰:「孔英來矣。」棁不諭其意,以
 謂維翰以孔英為言,乃考英及第,物議大以為非,即罷學干,拜尚書左丞,遷太常卿。



 五年,高祖詔太常復文武二舞,詳定正、冬朝會禮及樂章。自唐末之亂,禮樂制度亡失已久,棁與御史中丞竇貞固、刑部侍郎呂琦、禮部侍郎張允等草定之。其年冬至,高祖會朝崇元殿,廷設宮縣,二舞在北,登歌在上。文舞郎八佾,六十有四人,冠進賢,黃紗袍,白中單,白練衣蓋襠,白布大口褲,革帶履。左執籥,右秉翟。執籥引者二人。武舞郎八佾,六十有四人,服平巾幘,緋絲布大袖、繡襠甲金飾,白練衣蓋,錦騰蛇起梁帶,豹文大口褲,烏靴。左執干,右執戚。執旌引者二人。
 加鼓吹十二按,負以熊釣,以象百獸率舞。按設羽葆鼓一,大鼓一,金錞一。



 歌、簫、笳各二人。王公上壽,天子舉爵,奏《玄同》;三舉,登歌奏《文同》;舉食,文舞舞《昭德》,武舞舞《成功》之曲。禮畢,高祖大悅,賜棁金帛,群臣左右睹者皆嗟歎之。然禮樂廢久,而制作簡繆,又繼以龜茲部《霓裳法曲》,參亂雅音,其樂工舞郎,多教坊伶人、百工商賈、州縣避役之人,又無老師良工教習。



 明年正旦,復奏于廷,而登歌發聲悲離煩慝,如《薤露》、《虞殯》之音,舞者行列進退,皆不應節,聞者皆悲憤。其年高祖崩。棁以風痺改太子賓客分司西京以卒。



 開運二年,太常少卿陶穀奏廢二
 舞。明年,契丹滅晉,耶律德光入京師,太常請備法駕奉迎,樂工教習鹵簿鼓吹,都人聞者為之流涕焉。



 李懌李懌,京兆人也。少好學,頗工文辭。唐末舉進士,為秘書省校書郎、集賢校理。唐亡,事梁為監察御史,累遷中書舍人、翰林學士。梁亡,責授懷州司馬,遇赦量移,稍遷衛尉少卿。天成中,復為中書舍人、翰林學士,累遷尚書右丞承旨。



 時右散騎常侍張文寶知貢舉,所放進士,中書有覆落者,乃請下學士院作詩賦為貢舉格,學士竇夢徵、張礪等所作不工,乃命懌為之,懌笑曰:「年少舉進士登科,蓋偶然爾。後生可畏,來者未可量,假令予復就禮
 部試,未必不落第,安能與英俊為準格?」聞者多其知體。後遷刑部尚書分司洛陽,卒,年七十餘。



\end{pinyinscope}