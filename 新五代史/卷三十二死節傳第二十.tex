\article{卷三十二死節傳第二十}

\begin{pinyinscope}

 語曰:「世亂識忠臣。」誠哉!五代之際,不可以為無人,吾得全節之士三人焉,作《死節傳》。



 王彥章裴約劉仁贍附王彥章,字子明,鄆州壽張人也。少為軍卒,事梁太祖,為開封府押衙、左親從指揮使、行營先鋒馬軍使。末帝即位,遷濮州刺史,又徙澶州刺史。彥章為人驍勇有力,能跣足履棘行百步。持一鐵槍,騎而馳突,奮疾如飛,而佗人莫能舉也,軍中號王鐵槍。



 梁、晉爭天下為勁敵,獨彥
 章心常輕晉王,謂人曰:「亞次鬥雞小兒耳,何足懼哉!」梁分魏、相六州為兩鎮,懼魏軍不從,遣彥章將五百騎入魏,屯金波亭以虞變。魏軍果亂,夜攻彥章。彥章南走,魏人降晉。晉軍攻破澶州,虜彥章妻子歸之太原,賜以第宅,供給甚備,間遣使者招彥章,彥章斬其使者以自絕。然晉人畏彥章之在梁也,必欲招致之,待其妻子愈厚。



 自梁失魏、博,與晉夾河而軍,彥章常為先鋒。遷汝鄭二州防禦使、匡國軍節度使、北面行營副招討使,又徙宣義軍節度使。是時,晉已盡有河北,以鐵鎖斷德勝口,築河南、北為兩城,號「夾寨」。而梁末帝昏亂,小人趙巖、張漢
 傑等用事,大臣宿將多被讒間,彥章雖為招討副使,而謀不見用。龍德三年夏,晉取鄆州,梁人大恐,宰相敬翔顧事急,以繩內靴中,入見末帝,泣曰:「先帝取天下,不以臣為不肖,所謀無不用。今彊敵未滅,陛下棄忽臣言,臣身不用,不如死!」乃引繩將自經。末帝使人止之,問所欲言。翔曰:「事急矣,非彥章不可!」末帝乃召彥章為招討使,以段凝為副。末帝問破敵之期,彥章對曰:「三日。」左右皆失笑。



 彥章受命而出,馳兩日至滑州,置酒大會,陰遣人具舟於楊村,命甲士六百人皆持巨斧,載冶者,具鞁炭,乘流而下。彥章會飲,酒半,佯起更衣,引精兵數千,沿河
 以趨德勝。舟兵舉鎖燒斷之,因以巨斧斬浮橋,而彥章引兵急擊南城。浮橋斷,南城遂破,蓋三日矣。是時莊宗在魏,以朱守殷守夾寨,聞彥章為招討使,驚曰:「彥章驍勇,吾嘗避其鋒,非守殷敵也。然彥章兵少,利於速戰,必急攻我南城。」



 即馳騎救之,行二十里,而得夾寨報者曰:「彥章兵已至。」比至,而南城破矣。



 莊宗徹北城為筏,下楊劉,與彥章俱浮于河,各行一岸,每舟抃相及輒戰,一日數十接。彥章至楊劉,攻之幾下。晉人築壘博州東岸,彥章引兵攻之,不克,還擊楊劉,戰敗。



 是時,段凝已有異志,與趙巖、張漢傑交通,彥章素剛,憤梁日削,而嫉巖等所
 為,嘗謂人曰:「俟吾破賊還,誅姦臣以謝天下。」巖等聞之懼,與凝葉力傾之。



 其破南城也,彥章與凝各為捷書以聞,凝遣人告巖等匿彥章書而上己書,末帝初疑其事,已而使者至軍,獨賜勞凝而不及彥章,軍士皆失色。及楊劉之敗也,凝乃上書言:「彥章使酒輕敵而至於敗。」趙巖等從中日夜毀之,乃罷彥章,以凝為招討使。彥章馳至京師入見,以笏畫地,自陳勝敗之迹,巖等諷有司劾彥章不恭,勒還第。



 唐兵攻兗州,末帝召彥章使守捉東路。是時,梁之勝兵皆屬段凝,京師只有保鑾五百騎,皆新捉募之兵,不可用,乃以屬彥章,而以張漢傑監之。彥章
 至遞坊,以兵少戰敗,退保中都;又敗,與其牙兵百餘騎死戰。唐將夏魯奇素與彥章善,識其語音,曰:「王鐵槍也!」舉槊刺之,彥章傷重,馬踣,被擒。莊宗見之,曰:「爾常以孺子待我,今日服乎?」又曰:「爾善戰者,何不守兗州而守中都?中都無壁壘,何以自固?」彥章對曰:「大事已去,非人力可為!」莊宗惻然,賜藥以封其創。彥章武人不知書,常為俚語謂人曰:「豹死留皮,人死留名。」其於忠義,蓋天性也。莊宗愛其驍勇,欲全活之,使人慰諭彥章,彥章謝曰:「臣與陛下血戰十餘年,今兵敗力窮,不死何待?且臣受梁恩,非死不能報,豈有朝事梁而暮事晉,生何面目見天
 下之人乎!」莊宗又遣明宗往諭之,彥章病創,臥不能起,仰顧明宗,呼其小字曰:「汝非邈佶烈乎?我豈茍活者?」遂見殺,年六十一。晉高祖時,追贈彥章太師。



 與彥章同時有裴約者,潞州之牙將也。莊宗以李嗣昭為昭義軍節度使,約以裨將守澤州。嗣昭卒,其子繼韜以澤、潞叛降於梁,約召其州人泣而諭曰:「吾事故使二十餘年,見其分財饗士,欲報梁仇,不幸早世。今郎君父喪未葬,違背君親,吾能死于此,不能從以歸梁也!」眾皆感泣。



 梁遣董璋率兵圍之,約與州人拒守,求救於莊宗。是時,莊宗方與梁人戰河上,而已建大號,聞繼韜叛降梁,頗有憂色,
 及聞約獨不叛,喜曰:「吾於繼韜何薄?



 於約何厚?而約能分逆順邪!」顧符存審曰:「吾不惜澤州與梁,一州易得,約難得也。爾識機便,為我取約來。」存審以五千騎馳至遼州,而梁兵已破澤州,約見殺。



 至周世宗時,又有劉仁贍者焉。仁贍字守惠,彭城人也。父金,事楊行密,為濠、滁二州刺史,以驍勇知名。仁贍為將,輕財重士,法令嚴肅,少略通兵書。事南唐,為左監門衛將軍、黃袁二州刺史,所至稱治。李景使掌親軍,以為武昌軍節度使。周師征淮,先遣李穀攻自壽春,景遣將劉彥貞拒周兵,以仁贍為清淮軍節度使,鎮壽州。李穀退守正陽浮橋,彥貞見周
 兵之卻,意其怯,急追之。仁贍以為不可,彥貞不聽,仁贍獨按兵城守。彥貞果敗於正陽。



 世宗攻壽州,圍之數重,以方舟載炮,自淝河中流擊其城;又束巨竹數十萬竿,上施版屋,號為「竹龍」,載甲士以攻之,又決其水砦入于淝河。攻之百端,自正月至于四月不能下,而歲大暑,霖雨彌旬,周兵營寨水深數尺,淮、淝暴漲,炮舟竹龍皆飄南岸,為景兵所焚,周兵多死。世宗東趨濠梁,以李重進為廬、壽都招討使。景亦遣其元帥齊王景達等列砦紫金山下,為夾道以屬城中。而重進與張永德兩軍相疑不協,仁贍屢請出戰,景達不許,由是憤惋成疾。



 明年
 正月,世宗復至淮上,盡破紫金山砦,壞其夾道,景兵大敗,諸將往往見擒,而景之守將廣陵馮延魯、光州張紹、舒州周祚、泰州方訥、泗州范再遇等,或走或降,皆不能守,雖景君臣亦皆震懾,奉表稱臣,願割土地、輸貢賦,以效誠款,而仁贍獨堅守,不可下。世宗使景所遣使者孫晟等至城下示之,仁贍子崇諫幸其父病,謀與諸將出降,仁贍立命斬之,監軍使周廷構哭于中門救之,不得,於是士卒皆感泣,願以死守。



 三月,仁贍病甚,已不知人,其副使孫羽詐為仁贍書,以城降。世宗命舁仁贍至帳前,嘆嗟久之,賜以玉帶、御馬,復使入城養疾,是日卒。制
 曰:「劉仁贍盡忠所事,抗節無虧,前代名臣,幾人可比!予之南伐,得爾為多。」乃拜仁贍檢校太尉兼中書令、天平軍節度使。仁贍不能受命而卒,年五十八。



 世宗遣使弔祭,喪事官給,追封彭城郡王,以其子崇贊為懷州刺史,賜莊宅各一區。李景聞仁贍卒,亦贈太師。壽州故治壽春,世宗以其難剋,遂徙城下蔡,而復其軍曰忠正軍,曰:「吾以旌仁贍之節也。」



 嗚呼,天下惡梁久矣!然士之不幸而生其時者,不為之臣可也,其食人之祿者,必死人之事,如彥章者,可謂得其死哉!仁贍既殺其子以自明矣,豈有垂死而變節者
 乎?今《周世宗實錄》載仁贍降書,蓋其副使孫羽等所為也。當世宗時,王環為蜀守秦州,攻之久不下,其力屈而降,世宗頗嗟其忠,然止於為大將軍。視世宗待二人之薄厚而考其制書,乃知仁贍非降者也。自古忠臣義士之難得也!五代之亂,三人者,或出於軍卒,或出於偽國之臣,可勝嘆哉!可勝嘆哉!



\end{pinyinscope}