\article{卷六十二南唐世家第二}

\begin{pinyinscope}

 李昪,字正倫,徐州人也。世本微賤,父榮,遇唐末之亂,不知其所終。昪少孤,流寓濠、泗間,楊行密攻濠州,得之,奇其狀貌,養以為子。而楊氏諸子不能容,行密以乞徐溫,乃冒姓徐氏,名知誥。及壯,身長七尺,廣顙隆準。為人溫厚有謀。為吳樓船軍使,以舟兵屯金陵。柴再用攻宣州,用其兵殺李遇,昪以功拜昇州刺史。時江淮初定,州、縣吏多武夫,務賦斂為戰守,昪獨好學,接禮儒者,能自勵
 為勤儉,以寬仁為政,民稍譽之。徐溫鎮潤州,以昇、池等六州為屬,溫聞昪理昇州有善政,往視之,見其府庫充實,城壁修整,乃徙治之,而遷昪潤州刺史。



 昪初不欲往,屢求宣州,溫不與。既而徐知訓為朱瑾所殺,溫居金陵,未及聞。昪居潤州,近廣陵,得先聞,即日以州兵渡江定亂,遂得政。



 昪事徐溫甚孝謹,溫嘗罵其諸子不如昪,諸子頗不能容,而知訓尤甚,嘗召昪飲酒,伏劍士欲害之,行酒吏刁彥能覺之,酒至昪,以手爪掐之,昪悟起走,乃免。



 後昪自潤州入覲,知訓與飲於山光寺,又欲害之,徐知諫以其謀告昪,昪起遁去。



 知訓以劍授刁彥能,使追
 殺之,及於中途而還,紿以不及,由是得免。後昪貴,以彥能為撫州節度使。



 知訓之用事也,嘗凌弱楊氏而驕侮諸將,遂以見殺。及昪秉政,欲收人心,乃寬刑法、推恩信,起延賓亭以待四方之士,引宋齊丘、駱知祥、王令謀等為謀客,士有羈旅於吳者,皆齒用之。嘗陰使人察視民間有婚喪匱乏者,往往賙給之。盛暑未嘗張蓋、操扇,左右進蓋,必卻之,曰:「士眾尚多暴露,我何用此?」以故溫雖遙秉大政,而吳人頗已歸昪。



 武義元年,拜左僕射,參知政事。溫行軍司馬徐玠數勸溫以己子代昪,溫遣子知詢入廣陵,謀代昪秉政。會溫病卒,知詢奔還金陵,玠反
 為昪謀,誣知詢以罪,斬其客將周廷望,以知詢為右統軍。楊溥僭號,拜昪太尉、中書令。大和三年,出鎮金陵,如溫之制,留其子景通為司徒同平章事,以王令謀、宋齊丘為左、右僕射同平章事。四年,封昪東海郡王。



 昪照鑒見白須,顧其吏周宗歎曰:「功業已就,而吾老矣,奈何?」宗知其意,馳詣廣陵見宋齊丘,謀禪代。齊丘以為未可,請斬宗以謝吳人,昪黜宗為池州刺史。



 吳臨江王濛者,怨徐氏捨己而立溥,心嘗不平,及昪將謀篡國,先廢濛為歷陽公,使吏以兵守之。濛殺守者,奔廬州節度使周本。本,吳舊將也,聞濛至,欲納之,為其子祚所止。本曰:「此吾
 故主家郎君也,何忍拒之!」遽自出迎,祚閉門遮本不得出,縛濛送金陵,見殺。



 五年,昪封齊王。已而閩、越諸國皆遣使勸進,昪謂人望已歸。天祚三年,建齊國,置宗廟社稷,以宋齊丘、徐玠為左、右丞相。十月,溥遣攝太尉楊璘傳位於昪,國號齊,改元升元。昪以冊尊溥曰:「受禪老臣知誥,謹上冊皇帝為高尚思玄弘古讓皇帝。」追尊徐溫為忠武皇帝,封子景為吳王,封徐氏子知證江王,知諤饒王。周本與諸將至金陵勸進,歸而歎曰:「吾不誅篡國者以報楊氏,今老矣,豈能事二姓乎!」憤惋而死。



 二年四月,遷楊溥於潤州丹陽宮。以王輿為浙西節度使、
 馬思讓為丹陽宮使,以嚴兵守之。



 徐氏諸子請昪復姓,昪謙抑不敢忘徐氏恩,下其議百官,百官皆請,然後復姓李氏,改名曰昪。自言唐憲宗子建王恪生超,超生志,為徐州判司;志生榮。乃自以為建王四世孫,改國號曰唐。立唐高祖、太宗廟,追尊四代祖恪為孝靜皇帝,廟號定宗;曾祖超為孝平皇帝,廟號成宗;祖志孝安皇帝,廟號惠宗;考榮孝德皇帝,廟號慶宗。奉徐溫為義父,徐氏子孫皆封王、公,女封郡、縣主。以門下侍郎張居詠、中書侍郎李建勳、右僕射張延翰同平章事。十一月,以步騎八萬講武於銅橋。



 楊溥卒於丹陽宮。溥子璉為吳太
 子時,昪以女妻之,及昪篡國,封其女永興公主。女聞人呼公主,則嗚咽流涕而辭,宮中皆憐之。溥卒,以璉為康化軍節度使,已而以疾卒。



 三年四月,昪郊祀昊天上帝於圓丘,禮畢,群臣請上尊號。昪曰:「尊號非古也。」不許。州、縣言民孝悌五代同居者七家,皆表門閭,復其徭役;其尤盛者江州陳氏,宗族七百口,每食設廣席,長幼以次坐而共食,有畜犬百餘,共一牢食,一犬不至,諸犬為之不食。



 四年六月,晉安州節度使李金全叛,送款於昪,昪遣鄂州屯營使李承裕迎之。



 承裕與晉將馬全節、安審暉戰安陸南,三戰皆敗,承裕與裨將段處恭皆死,都監
 杜光鄴及其兵五百人被執,送於京師,高祖厚賜之,遣還。昪致書高祖,復送光鄴等,請以敗軍行法,高祖又遣之,昪以甲士臨淮拒之,乃止。



 六年,吳越國火,焚其宮室、府庫,甲兵皆盡,群臣請乘其弊攻之,昪不許,遣使弔問,厚賙其乏。錢氏自吳時素為敵國,昪見天下亂久,常厭用兵,及將篡國,先與錢氏約和,歸其所執將士,錢氏亦歸吳敗將,遂通好不絕。



 昪客馮延巳好論兵大言,嘗誚昪曰:「田舍翁安能成大事!」而昪志在守吳舊地而已,無復經營之略也,然吳人亦賴以休息。



 七年,昪卒,年五十六,謚曰光文肅武孝高皇帝,廟號烈祖,陵曰永陵。子景
 立。



 景,初名景通,昪長子也。既立,又改名璟。徐溫死,昪專政,以為兵部尚書、參知政事。明年,昪鎮金陵,留景為司徒、同平章事,與宋齊丘、王令謀居廣陵,輔楊溥。昪將篡國,召景歸金陵為副都統。昪立,封齊王。昪卒,嗣位,改元保大。



 尊母宋氏為皇太后,妃鐘氏為皇后。封弟壽王景遂為燕王,宣城王景達鄂王,景逷前未王,為保寧王。秋,改封景遂齊王、諸道兵馬元帥、太尉、中書令,景達為燕王、副元帥,盟於昪柩前,約兄弟世世繼立。封其子冀南昌王、江都尹。



 冬十月,破虔州妖賊張遇賢。遇賢,循州羅
 縣小吏也。初,有神降羅縣民家,與人言禍福輒中。遇賢禱之,神曰:「遇賢是羅漢,可留事我。」是時,南海劉死,子玢初立,嶺南盜賊起,群盜千餘人未有所統,問神當為主者,神言遇賢,遂共推為帥。遇賢自號中天八國王,改元永樂,置官屬,群賊盜皆絳衣,攻剽嶺外,問神所嚮,神曰:「當過嶺取虔州。」遂襲南康,節度賈浩不能禦。遇賢據白雲洞,造宮室,有眾十餘萬,連陷諸縣。景遣洪州營屯虞候嚴思、通事舍人邊鎬率兵攻之。



 遇賢問神,神不復語,群盜皆懼,遂執遇賢以降。



 景以馮延巳、常夢錫為翰林學士,馮延魯為中書舍人,陳覺為樞密使,魏岑、查文
 徽為副使。夢錫直宣政殿,專掌密命,而延巳等皆以邪佞用事,吳人謂之「五鬼」。夢錫屢言五人者不可用,景不納。十二月,景下令中外庶政委齊王景遂參決,惟陳覺、查文徽得奏事,群臣非召見者不得入。給事中蕭儼上疏切諫,不報。侍衛軍都虞候賈崇詣閣求見景,曰:「臣事先朝三十年,見先帝所以成功業者,皆用眾賢之謀,故延接疏遠,未嘗壅隔,然下情猶有不達者。今陛下新即位,所信用者何人?奈何頓與臣下隔絕!臣老即死,恐無復一見顏色。」因泣下嗚咽,景為之動容,引與坐,賜食而慰之,遂寢所下令。



 初,宋齊丘為昪謀篡楊氏最有
 力,及事成,乃陽入九華山,昪屢招之,乃出。



 昪僭號,未幾,齊丘以病罷相,出為洪州節度使。景立,復召為相,而陳覺、魏岑等皆為齊丘所引用。而岑與覺有隙,譖覺於景,左遷少府監。齊丘亦罷相為浙西節度使。齊丘不得意,願復歸九華山,賜號九華先生,封青陽公,食青陽一縣。



 二年二月,閩人連重遇、朱文進弒其君王延羲,文進自立。是時,延羲弟延政亦自立於建州,國號殷。王氏兄弟連兵累年,閩大亂,景因其亂遣查文徽及待詔臧循發兵攻建州。延政聞唐且攻之,遣人紿福州曰:「唐兵助我討賊矣。」福州信之,共殺文進等以降,延政遣其從子繼昌
 守福州。文徽軍屯建陽,福州將李仁達殺王繼昌自稱留後,泉州將留從效亦殺其刺史黃紹頗,皆送款於文徽。



 四年八月,文徽乘勝克建、汀、泉、漳四州,景分延平、劍浦、富沙三縣,置劍州,遷王延政之族於金陵。以延政為饒州節度使、李仁達為福州節度使、留從效為清源軍節度使。景遂欲罷兵,而查文徽、陳覺等皆言:「仁達等餘孽猶在,不若乘勝盡取之。」陳覺自言可不用尺兵致仁達等。景以覺為宣諭使,召仁達朝金陵,仁達不從。覺慚,還至建州,矯命發汀、建、信、撫州兵攻仁達。時魏岑安撫漳、泉,聞覺起兵,亦擅發兵會覺。景大怒,馮延巳等為言:「兵業
 行,不可止。」乃以王崇文為招討使、王建封為副使,益兵以會之,以延魯、魏岑、陳覺皆為監軍使。



 仁達送款於吳越,吳越以兵三萬應仁達。覺等爭功,進退不相應,延魯與吳越兵先戰,大敗而走,諸軍皆潰歸。景怒,遣使者鎖覺、延魯至金陵。而馮延巳方為宰相,宋齊丘復自九華召為太傅,為稍解之,乃流覺蘄州、延魯舒州。韓熙載上書切諫,請誅覺等,齊丘惡之,貶熙載和州司馬。是歲,契丹陷京師,中國無主,而景方以覺等疲兵東南,不暇北顧。御史中丞江文蔚劾奏宰相馮延巳、諫議大夫魏岑亂政,與覺等同罪而不見貶黜,言甚切直。景大怒,自答
 其疏,貶文蔚江州司士參軍,亦罷延巳為少傅、岑為太子洗馬。



 五年,以景遂為太弟;景達為元帥,封齊王;南昌王冀為副元帥,封燕王。契丹遣使來聘,以兵部尚書賈潭報聘。



 六年,漢李守貞反河中,遣其客將朱元來求援,景以潤州節度使李金全為北面行營招撫使,兵攻沭陽,聞守貞已敗,乃還。是時,漢隱帝少,中國衰弱,淮北群盜多送款於景,景遣皇甫暉出海、泗諸州招納之。



 八年,福州詐言「吳越戍兵亂,殺李仁達而遁」,遣人請建州節度使查文徽,文徽與劍州刺史陳誨下舟閩江趨應之。福州以兵出迎。誨曰:「閩人多詐難信,宜駐江岸徐圖之。」
 文徽曰:「久則生變,乘其未定,亟取之。」留誨屯江口,進至西門,伏兵發,文徽被擒。誨與越人戰,大敗之,獲其將馬先進。景送先進還越,越亦歸景文徽。是歲,楚王馬希廣為其弟希萼所弒,希萼自立。



 九年秋,楚人囚希萼於衡山,立其弟希崇,附于景,楚國大亂。景遣信州刺史邊鎬攻楚,破潭州,盡遷馬氏之族于金陵。景以希萼為洪州節度使,希崇舒州節度使,以邊鎬為湖南節度使。



 十年,分洪州高安、清江、萬載、上高四縣,置筠州。以馮延巳、孫忌為左、右僕射同平章事。廣州劉晟乘楚之亂,取桂管,景遣將軍張巒出兵爭之,不克。楚地新定,其府庫空虛,
 宰相馮延巳以克楚為功,不欲取費於國,乃重斂其民以給軍,楚人皆怨而叛,其將劉言攻邊鎬,鎬不能守,遁歸。



 十一年,金陵大火踰月。



 十二年,大饑,民多疫死。



 十三年十一月,周師南征,詔曰:「蠢爾淮甸,敢拒大邦,盜據一方,僭稱偽號。晉、漢之代,寰海未寧,而乃招納叛亡,朋助兇逆。金全之據安陸,守貞之叛河中,大起師徒,來為應援。迫奪閩、越,塗炭湘、潭,至於應接慕容,憑陵徐部,沭陽之役,曲直可知。勾誘契丹,入為邊患,結連并壘,實我世仇。罪惡難名,人神共憤。」乃拜李穀為行營都部署,攻自壽州始。是時,宋齊丘為洪州節度使,景召齊丘還金陵,
 以劉彥貞為神武統軍,劉仁贍為清淮軍節度使,以距周師。李穀曰:「吾無水戰之具,而使淮兵斷正陽浮橋,則我背腹受敵。」乃焚其芻糧,退屯正陽。



 是時世宗親征,行至圉鎮,聞穀退軍,曰:「吾軍卻,唐兵必追之。」遣李重進急趨正陽,曰:「唐兵且至,宜急擊之。」劉彥貞等聞穀退軍,果以為怯,急追之。



 比及正陽,而重進先至,軍未及食而戰,彥貞等遂敗。彥貞之兵施利刃於拒馬,維以鐵索;又刻木為獸,號「捷馬牌」;以皮囊布鐵蒺藜于地。周兵見而知其怯,一鼓敗之。世宗營于淝水之陽,徙浮橋于下蔡。景遣林仁肇等爭之不得,而周師取滁州。景懼,遣泗州牙
 將王知朗至徐州,稱唐皇帝奉書,願效貢賦,陳兄事之禮,世宗不答。景東都副留守馮延魯、光州刺史張紹、舒州刺史周祚、泰州刺史方訥皆棄城走;延魯削髮為僧,為周兵所獲。蘄州裨將李福殺其刺史王承雋降周。景益懼,始改名璟以避周廟諱,遣其翰林學士鐘謨、文理院學士李德明奉表稱臣,獻犒軍牛五百頭、酒二千石、金銀羅綺數千,請割壽、濠、泗、楚、光、海六州,以求罷兵。



 世宗不報,分兵襲下揚、泰。景遣人懷蠟丸書走契丹求救,為邊將所執。光州刺史張承翰降周。



 十四年三月,景又遣司空孫晟、禮部尚書王崇質奉表,辭益卑服,世宗猶
 不答,前遣鐘謨等並晟、崇質皆留行在。而謨等請歸取景表,盡獻江北地,世宗許之,遣崇質、德明等還,始賜景書曰:「自有唐失御,天步方艱,六紀于茲,瓜分鼎峙。



 自為聲教,各擅蒸黎,交結四夷,憑凌上國。華風不競,否運所鐘,凡百有心,敦不興憤?朕擅一百州之富庶,握三十萬之甲兵,農戰交修,士卒樂用,茍不能恢復內地,申畫邊疆,便議班旋,真同戲劇。至於削去尊稱,願輸臣節,孫權事魏,蕭詧奉周,古也雖然,今則不取。但存帝號,何爽歲寒?倘堅事大之心,必不迫人於險。」德明等還,盛稱世宗英武,景不悅。宋齊丘、陳覺等皆以割地無益,而德明賣
 國以圖利。景怒,斬德明。遣元帥齊王景達與陳覺、邊鎬、許文縝率兵趣壽春,景達將朱元等復得舒、蘄、泰三州。夏,大雨,周師在揚、滁、和者皆卻,諸將請要其險隘擊之。宋齊丘曰:「擊之怨深,不如縱之以為德。」誡諸將閉壁,無得要戰,故周師皆集於壽州。世宗屯于渦口,欲再幸揚州,宰相范質以師老泣諫,乃班師,以李重進攻廬、壽,向訓守揚州。訓請棄揚州,併力以攻壽春,乃封府庫付主者,遣景舊將按巡城中,秋毫不犯而去,淮人大悅,皆負糗糧,以送周師。



 十五年,景達遣朱元等屯紫金山,築甬道以餉壽州。二月,世宗復南征,徙下蔡浮橋于渦口,為
 鎮淮軍,築二城以夾淮。周師連破紫金諸寨。景達雖為元帥,兵事皆決於陳覺。覺與朱元素有隙,以元李守貞客,反覆難信,景遣大將楊守忠代元,且召之。元憤怒,叛降于周,諸軍皆潰,許文縝、邊鎬皆被執,景達以舟兵奔還金陵。劉仁贍病且死,其副使孫羽等以壽州降於周。世宗班師。景遣人焚揚州,驅其士庶而去。冬十月,世宗復南征,遂圍濠州,刺史郭廷謂告於周曰:「臣不能守一州以抗王師,然願請命于唐而後降。」世宗為之緩攻,廷謂遣人請命于景,景許其降,乃降。又取泗州。周師步騎數萬,水陸齊進,軍士作《檀來》之歌,聲聞數十里。
 十二月,屯于楚州之北門。



 交泰元年正月,大赦改元。周師攻楚州,守將張彥卿、鄭昭業城守甚堅,攻四十日不可破。世宗親督兵以洞屋穴城而焚之,城壞,彥卿、昭業戰死,周兵怒甚,殺戮殆盡。周師復取海、泰、揚州。世宗幸迎鑾以臨大江,景知不能支,而恥自屈身去其名號,乃遣陳覺奉表,請傳國與其世子而聽命。



 初,周師南征,無水戰之具,已而屢敗景兵,獲水戰卒,乃造戰艦數百艘,使降卒教之水戰,命王環將以下淮。景之水軍多敗,長淮之舟,皆為周師所得。又造齊雲船數百艘,世宗至楚州北神堰,齊雲舟大,不能過,乃開老鸛河以通之,遂至
 大江。景初自恃水戰,以周兵非敵,且未能至江。及覺奉使,見舟師列于江次甚盛,以為自天而下,乃請曰:「臣願還國取景表,盡獻江北諸州,如約。」世宗許之,始賜景書曰「皇帝恭問江南國主」,勞其良苦而已。是時,揚、泰、滁、和、壽、濠、泗、楚、光、海等州,已為周得,景遂獻廬、舒、蘄、黃,畫江以為界。五月,景下令去帝號,稱國主,奉周正朔,時顯德五年也。



 初,孫晟使于周,留不遣,而世宗問晟江南虛實,不對,世宗怒,殺晟。周已罷兵,景乃贈劉仁贍太師,追封晟魯國公。世宗遣鐘謨、馮延魯歸國。景復遣謨等朝京師,手自書表,稱天地父母之恩不可報,又請降詔書同
 籓鎮,遣謨面陳願傳位世子。世宗遣謨等還國,優詔以勞安之。景以謨為禮部侍郎、延魯戶部侍郎。



 景為太子時;延魯等皆出入東宮,禮部尚書常夢錫自昪世屢言不可使延魯等近太子,及景立,延魯用事,夢錫每排斥之。景既割地稱臣,有語及朝廷為大朝者,夢錫大笑曰:「君等嘗欲致君如堯、舜,今日自為小朝邪?」鐘謨素善李德明,既歸,而聞德明由宋齊丘等見殺,欲報其冤,未能發。陳覺,齊丘黨也,與嚴續素有隙。覺嘗奉使周,還言世宗以江南不即聽命者,嚴續之謀,勸景誅續以謝罪。景疑之,謨因請使于周,驗其事。景已割地稱臣,乃
 遣謨入朝謝罪,言不即割地者,非續謀,願赦之。世宗大驚,曰:「續能為謀,是忠其主也,朕豈殺忠臣乎?」謨還,言覺姦詐,景怒,流覺饒州,殺之,宋齊丘坐覺黨與,放還青陽,賜死。以太弟景遂為洪州節度使,燕王冀為太子。



 景困於用兵,鐘謨請鑄大錢以一當十,文曰「永通泉貨」。謨嘗得罪,而大錢廢。韓熙載又鑄鐵錢,以一當二。



 九月,太子冀卒,次子從嘉封吳王,居東宮。鐘謨言從嘉輕肆,請立紀國公從善,景怒,貶謨國子司業,立從嘉為太子。世宗使人謂景曰:「吾與江南,大義已定,然慮後世不能容汝,可及吾世修城隍、治要害為子孫計。」景因營緝諸城,謀
 遷其都于洪州,群臣皆不欲遷,惟樞密使唐鎬贊之,乃升洪州為南昌,建南都。建隆二年,留太子從嘉監國,景遷于南都。而洪州迫隘,宮府營廨,皆不能容,群臣日夕思歸,景悔怒不已。唐鎬慚懼,發疾卒。



 六月,景卒,年六十四。從嘉嗣立,以喪歸金陵,遣使入朝,願復景帝號,太祖皇帝許之,乃謚曰明道崇德文宣孝皇帝,廟號元宗,陵曰順陵。



 煜字重光,初名從嘉,景第六子也。煜為人仁孝,善屬文,工書畫,而豐額駢齒,一目重瞳子。自太子冀已上,五子皆早亡,煜以次封吳王。建隆二年,景遷南都,立煜為太
 子,留監國。景卒,煜嗣立於金陵。母鐘氏,父名泰章。煜尊母曰聖尊后;立妃周氏為國後;封弟從善韓王,從益鄭王,從謙宜春王,從度昭平郡公,從信文陽郡公。大赦境內。遣中書侍郎馮延魯修貢于朝廷,令諸司四品已下無職事者,日二員待制於內殿。



 三年,泉州留從效卒。景之稱臣於周也,從效亦奉表貢獻于京師,世宗以景故,不納。從效聞景遷洪州,懼以為襲己,遣其子紹基納貢于金陵,而從效病卒,泉人因並送其族于金陵,推立副使張漢思。漢思老不任事,州人陳洪進逐之,自稱留後,煜即以洪進為節度使。乾德二年,始用鐵錢,民間多藏
 匿舊錢,舊錢益少,商賈多以十鐵錢易一銅錢出境,官不可禁,煜因下令以一當十。拜韓熙載中書侍郎、勤政殿學士。封長子仲遇清源公,次子仲儀宣城公。



 五年,命兩省侍郎、給事中、中書舍人、集賢勤政殿學士,分夕於光政殿宿直,煜引與談論。煜嘗以熙載盡忠,能直言,欲用為相,而熙載後房妓妾數十人,多出外舍私侍賓客,煜以此難之,左授熙載右庶子,分司南都。熙載盡斥諸妓,單車上道,煜喜留之,復其位。已而諸妓稍稍復還,煜曰:「吾無如之何矣!」是歲,熙載卒,煜歎曰:「吾終不得熙載為相也。」欲以平章事贈之,問前世有此比否,群臣對曰:「
 昔劉穆之贈開府儀同三司。」遂贈熙載平章事。熙載,北海將家子也,初與李穀相善。明宗時,熙載南奔吳,穀送至正陽,酒酣臨訣,熙載謂穀曰:「江左用吾為相,當長驅以定中原。」穀曰:「中國用吾為相,取江南如探囊中物爾。」



 及周師之征淮也,命穀為將,以取淮南,而熙載不能有所為也。



 開寶四年,煜遣其弟韓王從善朝京師,遂留不遣。煜手疏求從善還國,太祖皇帝不許。煜嘗怏怏以國蹙為憂,日與臣下酣宴,愁思悲歌不已。



 五年,煜下令貶損制度。下書稱教,改中書、門下省為左、右內史府,尚書省為司會府,御史臺為司憲府,翰林為文館,樞密院為
 光政院,諸王皆為國公,以尊朝廷。煜性驕侈,好聲色,又喜浮圖,為高談,不恤政事。



 六年,內史舍人潘佑上書極諫,煜收下獄,佑自縊死。



 七年,太祖皇帝遣使詔煜赴闕,煜稱疾不行,王師南征,煜遣徐鉉、周惟簡等奉表朝廷求緩師,不答。八年十二月,王師克金陵。九年,煜俘至京師,太祖赦之,封煜違命侯,拜左千牛衛將軍。其後事具國史。



 予世家江南,其故老多能言李氏時事,云太祖皇帝之出師南征也,煜遣其臣徐鉉朝於京師。鉉居江南,以名臣自負,其來也,欲以口舌馳說存其國,其日夜計謀思慮言語應對之際詳矣。及其將見也,大臣亦先入請,
 言鉉博學有材辯,宜有以待之。太祖笑曰:「第去,非爾所知也。」明日,鉉朝于廷,仰而言曰:「李煜無罪,陛下師出無名。」太祖徐召之升,使畢其說。鉉曰:「煜以小事大,如子事父,未有過失,奈何見伐?」其說累數百言。太祖曰:「爾謂父子者為兩家可乎?」鉉無以對而退。嗚呼,大哉,何其言之簡也!蓋王者之興,天下必歸于一統。其可來者來之,不可者伐之;僭偽假竊,期於掃蕩一平而後已。予讀周世宗《征淮南詔》,怪其區區攈摭前事,務較曲直以為辭,何其小也!然世宗之英武有足喜者,豈為其辭者之過歟?



 據湯悅所撰《江南錄》云:「景以保大十五年正月,改元交泰,是歲盡獻淮南十四州,畫江為界。」保大十五年,乃周顯德
 四年也。案《五代舊史》及《世宗實錄》,顯德四年十月壬申,世宗方復南征,五年正月丙午,始克楚州。二月己亥,景始盡獻淮南諸州,畫江為界,當是保大十六年也。悅等南唐故臣,記其目見之事,何其差繆?而《九國志》、《紀年通譜》之類,但以悅書為正,不復參校,遂皆差一年。至於景滅閩國,是保大四年,《江南錄》書於三年,亦差一年,已具《閩世家》注。或疑景立逾年而改元,則滅閩國當為三年,周取淮南當為十五年不差,但《江南錄》誤於景立之年改元保大,所以常差一年也。今知不然者,以諸書參校,閩人殺王延羲,當晉開運元年,周師始伐南唐當顯德二年。據景以初立之年即改元,則開運元年為保大二年,顯德二年為保大十三年。今《江南錄》書延羲被殺於二年,周師始伐於十三年,則是景立之年改元,不誤,而悅等書滅王氏、割淮南自各差一年爾。昪自晉天福二年建國,至皇朝開寶八年國滅,凡三十九年。



\end{pinyinscope}