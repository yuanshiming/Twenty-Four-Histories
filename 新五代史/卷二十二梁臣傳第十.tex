\article{卷二十二梁臣傳第十}

\begin{pinyinscope}

 康懷英康懷英,兗州人也。事硃瑾為牙將,梁兵攻瑾,瑾出略食豐、沛間,留懷英守城,懷英即以城降梁,瑾遂奔于吳。太祖得懷英,大喜。後從氏叔琮攻趙匡凝,下鄧州。梁兵攻李茂貞于岐,以懷英為先鋒,至武功,擊殺岐兵萬餘人,太祖喜曰:「邑名武功,真武功也。」以名馬賜之。是時,李周彞以鄜坊兵救岐,屯于三原界,懷英擊走之,因取其翟州而還。岐兵屯奉天,懷英柵其東北。夜半,岐兵攻之,懷英
 以為夜中不欲驚它軍,獨以三千人出戰,遲明,岐兵解去,身被十餘瘡。李茂貞與梁和,昭宗還京師,賜懷英「迎鑾毅勇功臣」。



 楊行密攻宿州,太祖遣懷英擊走之,表宿州刺史。遷保義軍節度使。丁會以潞州叛梁降晉,太祖命懷英為招討使,將行,太祖戒之,語甚切,懷英惶恐,以謂潞州期必得,乃築夾城圍之。晉遣周德威屯于亂柳,數攻夾城,懷英不敢出戰,太祖乃以李思安代懷英將,降懷英為都虞候。久之,思安亦無功,太祖大怒,罷思安,以同州劉知俊為招討使。知俊未至軍,太祖自至澤州,為懷英等軍援,且督之。已而晉王李克用卒,莊宗召周
 德威還。太祖聞晉有喪,德威去,亦歸洛陽,而諸將亦少弛。莊宗謂德威曰:「晉之所以能敵梁,而彼所憚者,先王也。今聞吾王之喪,謂我新立,未能出兵,其意必怠,宜出其不意以擊之,非徒解圍,亦足以定霸也。」



 乃與德威等疾馳六日至北黃碾,會天大昏霧,伏兵三垂岡,直趨夾城,攻破之。懷英大敗,亡大將三百人,懷英以百騎遁歸,詣闕請死。太祖曰:「去歲興兵,太陰虧食,占者以為不利,吾獨違之而致敗,非爾過也。」釋之,以為右衛上將軍。



 劉知俊叛,奔于岐,以懷英為保義軍節度使、西路副招討使。知俊以岐兵圍靈武,太祖遣懷英攻邠寧以牽
 之。懷英取寧、慶、衍三州,還至昇平,知俊掩擊之,懷英大敗。徙鎮感化。其後朱友謙叛附于晉,以懷英討之,與晉人戰白徑嶺,懷英又大敗。徙鎮永平,卒于鎮。



 劉掞劉掞,密州安丘人也。少事青州王敬武。敬武卒,子師範立。棣州刺史張蟾叛,師範遣指揮使盧洪討蟾,洪亦叛。師範偽為好辭召洪,洪至,迎於郊外,命掞斬之座上,因使掞攻張蟾,破之。師範表掞登州刺史,以為行軍司馬。



 梁太祖西攻鳳翔,師範乘梁虛,陰遣人分襲梁諸州縣,它遣者謀多漏洩,事不成。獨掞素好兵書,有機略。是時,梁已破朱瑾等,悉有兗、鄆,以葛從周為兗州節度使,從
 周將兵在外,掞乃使人負油鬻城中,悉視城中虛實出入之所。油者得羅城下水竇可入,掞乃以步兵五百從水竇襲破之,徙從周家屬外第,親拜其母,撫之甚有恩禮。



 太祖已出昭宗于鳳翔,引兵東還,遣朱友寧攻師範、從周攻掞。掞以版輿置從周母城上,母呼從周曰:「劉將軍待我甚厚,無異於汝。人臣各為其主,汝可察之!」



 從周為之緩攻。掞乃悉簡婦人及民之老疾不足當敵者出之,獨與少壯者同辛苦,分衣食,堅守以待。外援不至,人心頗離,副使王彥溫踰城而奔,守陴者多逸。掞乃遣人陽語彥溫曰:「副使勿多以人出,非吾素遣
 者,皆勿以行。」又下令城中曰:「吾遣從副使者得出,否者皆族。」城中皆惑,奔者乃止。已而梁兵聞之,果疑彥溫非實降者,斬之城下,由是城守益堅。



 師範兵已屈,從周以禍福諭掞,掞報曰:「俟吾主降,即以城還梁。」師範敗,降梁,掞乃亦降。從周為具齎裝,送掞歸梁,掞曰:「降將蒙梁恩不誅,幸矣,敢乘馬而衣裘乎!」乃素服乘驢歸梁。太祖賜之冠帶,飲之以酒,掞辭以量小,太祖曰:「取兗州,量何大乎?」以為元從都押衙。是時,太祖已領四鎮,將吏皆功臣舊人,掞一旦以降將居其上,及諸將見掞,皆用軍禮,掞居自如,太祖益奇之。



 太祖即位,累遷左龍武統
 軍。劉知俊叛,陷長安,太祖遣掞與牛存節討之,知俊走鳳翔,太祖乃以長安為永平軍,拜掞節度使。末帝即位,領鎮南軍節度使,為開封尹。



 楊師厚卒,分相、魏為兩鎮,末帝恐魏兵亂,遣掞以兵屯于魏縣。魏兵果亂,劫賀德倫降晉。莊宗入魏,掞以謂晉兵悉從莊宗赴魏,而太原可襲,乃結草為人,執以旗幟,以驢負之往來城上,而潛軍出黃澤關襲太原。晉兵望梁壘旗幟往來,不知其去也,以故不追。掞至樂平,遇雨,不克進而旋,急趨臨清,爭魏積粟,而周德威已先至,掞乃屯於莘縣,築甬道及河以饋軍。



 久之,末帝以書責掞曰:「閫外之事全付將軍,河
 朔諸州一旦淪沒。今倉儲已竭,飛輓不充,將軍與國同心,宜思良畫!」掞報曰:「晉兵甚銳,未可擊,宜待之。」末帝復遣問掞必勝之策,掞曰:「臣無奇術,請人給米十斛,米盡則敵破矣!」



 末帝大怒,誚掞曰:「將軍蓄米,將療饑乎?將破敵乎?」乃遣使者監督其軍。掞召諸將謀曰:「主上深居禁中,與白面兒謀,必敗人事。今敵盛,未可輕動,諸君以為如何?」諸將皆欲戰,掞乃悉召諸將坐之軍門,人以河水一杯飲之,諸將莫測,或飲或辭,掞曰:「一杯之難猶若此,滔滔河流可盡乎?」諸將皆失色。



 是時,莊宗在魏,數以勁兵壓掞營,掞不肯出,而末帝又數促掞,使出戰。莊宗與
 諸將謀曰:「劉掞學《六韜》,喜以機變用兵,本欲示弱以襲我,今其見迫,必求速戰。」乃聲言歸太原,命符存審守魏,陽為西歸,而潛兵貝州。掞果報末帝曰:「晉王西歸,魏無備,可擊。」乃以兵萬人攻魏城東,莊宗自貝州返趨擊之。



 掞忽見晉軍,驚曰:「晉王在此邪!」兵稍卻,追至故元城,莊宗與符存審為兩方陣夾之,掞為圓陣以禦晉人。兵再合,掞大敗,南奔,自黎陽濟河,保滑州。末帝以為義成軍節度使。明年,河朔皆入于晉,降掞亳州團練使。



 兗州張萬進反,拜掞兗州安撫制置使。萬進敗死,乃拜掞泰寧軍節度使。朱友謙叛,陷同州,末帝以掞為河東道招討
 使,行次陜州,掞為書以招友謙,友謙不報,留月餘待之。尹皓、段凝等素惡掞,乃譖之,以為掞與友謙親家,故其逗留以養賊。



 已而掞兵數敗,乃罷掞歸洛陽,酖殺之,年六十四,贈中書令。



 子遂凝、遂雍,事唐皆為刺史。掞妾王氏有美色,掞卒後,入明宗宮中,是為王淑妃。明宗晚年,淑妃用事,掞二子皆被恩寵。



 潞王從珂反於鳳翔,時遂雍為西京副留守,留守王思同率諸鎮兵討鳳翔,戰敗東歸,遂雍閉門不內,悉封府庫以待潞王。潞王前軍至者,悉以金帛給之。潞王見遂雍,握手流涕,由是事無大小皆與圖議。廢帝入立,拜遂雍淄州刺史,以掞兄琪之
 子遂清代遂雍為西京副留守。



 遂清歷易、棣等五州刺史,皆有善政,遷鳳州防禦使、宣徽北院使,判三司。



 晉開運中為安州防禦使以卒。遂清性至孝,居父喪哀毀,鄉里稱之。嘗為淄州刺史,迎其母,母及郊,遂清為母執轡行數十里,州人咸以為榮。



 牛存節牛存節,字贊正,青州博昌人也。初名禮,事諸葛爽於河陽,爽卒,存節顧其徒曰:「天下洶洶,當得英雄事之。」乃率其徒十餘人歸梁太祖。存節為人木彊忠謹,太祖愛之,賜之名字,以為小校。張晊攻汴,存節破其二寨。梁攻濮州,戰南劉橋、范縣,存節功多。李罕之圍張全義於河陽,
 全義乞兵於梁,太祖以存節故事河陽,知其間道,使以兵為前鋒。是時歲饑,兵行乏食,存節以金帛就民易乾葚以食軍,擊走罕之。太祖攻魏,存節下魏黎陽、臨河,殺魏萬二千人,與太祖會內黃。



 遷滑州牢城遏後指揮使。



 梁兵攻鄆,存節使都將王言藏船鄆西北隅濠中,期以日午渡兵踰濠急攻之。會營中火起,鄆人登城望火,言伏不敢動,與存節失期,存節獨破鄆西甕城門,奪其濠橋,梁兵得俱進,遂破朱宣。從葛從周攻淮南,從周敗渒河,存節收其散卒八千以歸。拜亳、宿二州刺史。朱瑾走吳,召吳兵攻徐、宿,存節謀曰:「淮兵必不先攻宿,然宿溝
 壘素固,可以禦敵。」乃夜以兵急趣徐州,比傅徐城下,瑾兵方至,望其塵起,驚曰:「梁兵已來,何其速也!」不能攻而去。已而太祖使者至,授存節軍機,悉與存節意合,由是諸將益服其能。遷潞州都指揮使。太祖攻鳳翔,使召存節。存節為將,法令嚴整而善得士心,潞人送者皆號泣。累拜邢州團練使、元帥府左都押衙。



 太祖即位,拜右千牛衛上將軍。從康懷英攻潞州,為行營排陣使。晉兵已破夾城,存節等以餘兵歸,行至天井關,聞晉兵攻澤州,存節顧諸將曰:「吾行雖不受命,然澤州要害,不可失也。」諸將皆不欲救之。存節戒士卒熟息,已而謂曰:「事急不
 赴,豈曰勇乎!」舉策而先,士卒隨之。比至澤州,州人已焚外城,將降晉,聞存節至,乃稍定。存節入城,助澤人守,晉人穴地道以攻之,存節選勇士數十,亦穴地以應之,戰於隧中,敵不得入,晉人解去。遷左龍虎統軍、六軍都指揮使、絳州刺史,遷鄜州留後。



 同州劉知俊叛,奔鳳翔,乃遷存節匡國軍節度使。友珪立,朱友謙叛附于晉,西連鳳翔,存節東西受敵。同州水鹹而無井,知俊叛梁,以渴不能守而走,故友謙與岐兵合圍持久,欲以渴疲之,存節禱而擇地鑿井八十,水皆甘可食,友謙卒不能下。



 末帝立,加同中書門下平章事,徙鎮天平。蔣殷反徐州,遣
 存節攻破之,以功加太尉。梁、晉相距於河上,存節病痟,而梁、晉方苦戰,存節忠憤彌激,治軍督士,未嘗言病。病革,召歸京師,將卒,語其子知業曰:「忠孝,吾子也。」不及其佗。贈太師。



 張歸霸弟歸厚歸弁張歸霸,清河人也。末帝娶其女,是為德妃。歸霸少與其弟歸厚、歸弁俱從黃巢,巢敗東走,歸霸兄弟乃降梁。秦宗權攻汴,歸霸戰數有功。張晊軍赤岡,以騎兵挑戰,矢中歸霸,歸霸拔之,反以射賊,一發而斃,奪其馬而歸。太祖從高丘望見,甚壯之,賞以金帛,并以其馬賜之。使以弓手五百人伏湟中,太祖以騎數百為遊兵,過晊柵,晊
 出兵追太祖,歸霸發伏,殺晊兵千人,奪馬數十匹。



 太祖攻蔡州,蔡將蕭顥急擊太祖營,歸霸不暇請,與徐懷玉分出東南壁門,合擊敗之,太祖得拔營去。太祖攻兗、鄆,取曹州,使歸霸以兵數千守之,與朱瑾逆戰金鄉,大敗之。又破濮州。晉人攻魏,歸霸從葛從周救魏,戰洹水,歸霸擒克用子落落以與魏人。又破劉仁恭於內黃,功出諸將右。光化二年,權知邢州。遷萊州刺史,拜左衛上將軍、曹州刺史。開平元年,拜右龍虎統軍、左驍衛上將軍。二年,拜河陽節度使,以疾卒。



 子漢傑,事末帝為顯官,以張德妃故用事。梁亡,唐莊宗入汴,遂族誅。



 弟歸厚,字德
 坤。為將善用弓槊,能以少擊眾。張晊屯赤岡,歸厚與晊獨戰陣前,晊憊而卻,諸將乘之,晊遂大敗。太祖大悅,以為騎長。梁攻時溥,歸厚以麾下先進九里山,遇徐兵而戰,梁故將陳璠叛在徐,歸厚望見識之,瞋目大罵,馳騎直往取之,矢中其左目。郴王友裕攻鄆,屯濮州,太祖從後至,友裕徙柵,與太祖相失。太祖卒與鄆兵遇,太祖登高望之,鄆兵纔千人,太祖與歸厚以子軍直衝之,戰已合,鄆兵大至,歸厚度不能支,以數十騎衛太祖先還。歸厚馬中矢僵,乃持槊步斗。太祖還軍中,遣張筠馳騎第取之,以為必死矣。歸厚體被十餘箭,得筠馬乃歸,太
 祖見之,泣曰:「爾在,喪軍何足計乎!」使舁歸宣武。遷右神武統軍,歷洺、晉、絳三州刺史。與晉人屢戰未嘗屈。乾化元年,拜鎮國軍節度使,以疾卒。



 子漢卿。



 歸弁,為將亦善戰,開平初為滑州長劍指揮使。子漢融。梁亡,皆族誅。



 王重師王重師,許州長社人也。為人沈嘿多智,善劍槊。秦宗權陷許州,重師脫身歸梁,從太祖平蔡,攻兗、鄆,為拔山軍指揮使。重師苦戰齊、魯間,威震鄰敵。遷潁州刺史。太祖攻濮州,已破,濮人積草焚之,梁兵不得入。是時,重師方病金瘡,臥帳中,諸將強之,重師遽起,悉取軍中氈毯沃以水,蒙之火上,率精卒以短兵突入,梁兵隨之皆入,遂
 取濮州。重師身被八九瘡,軍士負之而還。太祖聞之,驚曰:「柰何使我得濮州而失重師乎!」使醫理之,逾月乃愈。王師範降,表重師青州留後,累遷佑國軍節度使、同中書門下平章事。居數年,甚有威惠。重師與劉捍故有隙,捍嘗構之太祖,太祖疑之。重師遣其將張君練西攻邠、鳳而不先請,君練兵小敗,太祖以其擅發兵,挫失國威,將召而罪之,遣劉捍代重師。重師不知太祖怒己,捍至,重師不出迎,見之青門,禮又倨,捍因馳白太祖,言重師有二志。太祖益怒,貶重師溪州刺史,再貶崖州司戶參軍,未行,賜死。



 徐懷玉徐懷玉,亳州焦夷人也。少事梁太祖,與太祖俱起微賤。懷玉為將,以雄豪自任,而勇於戰陣。從太祖鎮宣武,為永城鎮將。秦宗權攻梁,壁金隄、靈昌、酸棗,懷玉以輕騎連擊破之,俘殺五千餘人,遷左長劍都虞候。又破宗權於板橋、赤岡,拔其八柵。從太祖東攻兗、鄆,破徐、宿。懷玉金創被體,戰必克捷,所得賞賚,往往以分士卒,為梁名將。本名琮,太祖賜名懷玉。從太祖攻魏,敗魏兵黎陽,遂東攻兗,破朱瑾於金鄉。又從龐師古攻楊行密,師古敗清口,懷玉獨完一軍,行收散卒萬餘人以歸。遷沂州刺史,屬歲屢豐,乃繕兵治壁,為戰守具。已而王師範叛梁,
 攻東境,懷玉屢以州兵擊破之。遷齊州防禦使。天復四年,以州兵遼昭宗都洛陽,遷華州觀察留後,以兵屯雍州。遷右羽林統軍,屯於澤州,晉人攻之,為隧以入,懷玉擊之隧中,晉人乃卻。太祖時,歷曹、晉二州刺史,晉數攻之,懷玉堅守,敗晉兵於洪洞。拜保大軍節度使。太祖崩,友珪自立,硃友謙附於晉,以襲鄜州,執懷玉殺之。



\end{pinyinscope}