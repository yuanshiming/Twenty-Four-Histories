\article{卷四十八雜傳第三十六}

\begin{pinyinscope}

 盧文進盧文進,字大用,范陽人也。為劉守光騎將。唐莊宗攻范陽,文進以先降拜壽州刺史,莊宗以屬其弟存矩。存矩為新州團練使,統山後八軍。莊宗與劉鄩相拒於莘,召存矩會兵擊鄩。存矩募山後勁兵數千人,課民出馬,民以十牛易一馬,山後之人皆怨,而兵又不樂南行,行至祁溝關,聚而謀為亂。文進有女幼而美,存矩求之為側室,文進以其大將不敢拒,雖與,心常歉之也,因與亂
 軍殺存矩反。攻新州,不克,攻武州,又不克,遂奔于契丹,契丹使守平州。



 明宗即位,文進自平州率眾數萬歸唐,明宗得之,喜甚,以為義成軍節度使。



 居歲餘,徙鎮威勝,加同平章事,入為上將軍,出鎮昭義,徙安遠。晉高祖立,與契丹約為父子,文進懼不自安。天福元年冬,殺其行軍司馬馮知兆、副使杜重貴,送款於李昪,昪遣兵迎之。文進居數鎮,頗有善政,兵民愛之。其將行也,從數騎,自至營中別其將士,告以避契丹之意,將士皆再拜為訣,乃南奔。昪以文進為天雄統軍、宣潤節度使。



 文進身長七尺,狀貌偉然。自其奔契丹也,數引契丹攻掠幽、薊之
 間,虜其人民,教契丹以中國織紝工作無不備,契丹由此益彊。同光中,契丹數以奚騎出入塞上,攻掠燕、趙,人無寧歲。唐兵屯涿州,歲時饋運,自瓦橋關至幽州,嚴兵斥候,常苦鈔奪,為唐患者十餘年,皆文進為之也。及其南奔,始屈身晦迹,務為恭謹,禮接文士,謙謙若不足,其所談論,近代朝廷儀制、臺閣故事而已,未嘗言兵。後以左衛上將軍卒于金陵。



 李金全李金全,其先出於吐渾。金全少為唐明宗廝養,以驍勇善騎射,常從明宗戰伐,以功為刺史。天成中,為彰義軍節度使,在鎮務為貪暴。罷歸,獻馬數十匹,居數日,又以
 獻,明宗謂曰:「卿患馬多邪,何進獻之數也?且卿在涇州治狀如何,無乃以馬為事乎?」金全慚不能對。徙鎮橫海。久之,罷為右衛上將軍。



 晉高祖時,安州屯防指揮使王暉殺節度使周瑰,高祖遣金全將騎兵千人以往,下詔書招暉曰:「暉降,以為唐州刺史。」又以信箭諭安州,不戮一人,且戒金全曰:「無失吾信。」金全未至,襄州安從進意暉必走江南,以精兵遮其要路。暉聞金全來,果南走,為從進兵所殺。金全後至,得暉餘黨數百人,皆送京師。暉之亂也,大掠城中三日,金全利其所掠貲,因擒其將武克和等十餘人殺之,克和呼曰:「王暉首亂,猶賜之信誓,
 以為刺史;我等何罪,反見殺邪?若朝廷之命,何以示信?茍將軍違詔而殺降,亦將不免也!」高祖不能詰。即以金全為安遠軍節度使。



 金全左都押衙明漢榮用事,所為不法,高祖患之,不欲因漢榮以累功臣,為選廉吏賈仁沼代之,且召漢榮。漢榮教金全留己而不遣,金全客龐令圖諫曰:「仁沼昔事王晏球,晏球攻王都於中山,都遣善射者登城射晏球,中兜牟,仁沼從後引弓,射善射者,一發而斃,晏球求其人,欲厚賞之,仁沼退而不言,此天下之忠臣也。



 都敗,晏球遣仁沼獻捷于京師,凡所賜與甚厚,悉以分故人、親戚之貧者,此天下之廉士也。為人如
 此,豈有為人謀而不善者乎?宜納仁沼而遣漢榮。」漢榮聞之,夜使人殺令圖而仁沼,仁沼舌壞而死。



 天福五年夏,高祖以馬全節代金全。而仁沼二子欲詣京師訴其父冤,漢榮大懼,紿金全曰:「前日天子召漢榮,公違詔而不遣。仁沼之死,其二子將訴于朝。今以全節代公,是召公對獄也。」金全信之,遂叛,送款于李昪。高祖發兵三萬授全節討之。昪遣其將李承裕入安州,金全遂南奔,行至泌川,引頸北望,涕泣而去。昪以金全為天威統軍。漢隱帝時,李守貞反河中,乞兵於昪,金全為昪潤州節度使,與查文徽等出沐陽。昪之諸將皆銳於攻取,金全
 獨以謂遠不相及,不可行,乃止。



 其後亦不復用,不知其所終。



 楊思權楊思權,邠州新平人也。事梁為控鶴右第一軍使。唐莊宗滅梁,以為夾馬都指揮使。明宗時,秦王從榮為河東節度使,以馮贇為副,思權為北京步軍都指揮使以佐佑之。從榮素驕,所為多不法。是時,宋王從厚為河南尹。從厚年少,謙恭好禮。



 明宗陰遣人從容語從厚之善,以諷勉之。從榮不悅,告思權曰:「天下共賢河南而非我,我將廢矣,奈何?」思權曰:「公有甲士,而思權在,何患也!」乃勸從榮招募死士、增利器械以為備。馮贇患之,以其事聞。
 明宗召思權還京師,以從榮故,亦不之責也。後為右羽林都指揮使,將兵戍興元。潞王從珂反鳳翔,興元張虔釗會諸鎮兵討賊。諸鎮兵圍鳳翔,思權攻城西,嚴衛指揮使尹暉攻城東,破其兩關城。



 從珂登城呼外兵,告以己非反者,其語甚哀,外兵聞者皆悲之,而虔釗督戰甚急,軍士反兵逐虔釗,思權因呼其眾曰:「潞王真吾主也!」即擁軍士入城降。暉聞思權已降,亦麾其軍使解甲,由是諸鎮之兵皆潰。思權與暉入見從珂,思權前曰:「臣以赤心奉殿下,殿下事成,願不以防禦、團練使處臣。」乃出一紙於懷中曰:「願志臣姓名以為驗。」從珂即書曰:「可
 邠寧節度使。」廢帝入立,拜思權靜難軍節度使。後為右龍武統軍、左衛上將軍。天福八年,卒于京師,贈太傅。



 尹暉尹暉者,魏州大名人也。從廢帝入洛陽,而晉高祖來朝,與暉遇于道。暉時猶為嚴衛指揮使,恃先降功,不為高祖屈,馬上橫鞭揖之,高祖怒,白廢帝暉不可與名籓。乃以為應州節度使。晉高祖入立,罷為右衛大將軍。范延光反,以書招暉,暉懼,出奔淮南,為人所殺,有子勛。



 王弘贄王弘贄,不知其世家何人也。唐明宗時,為合階二州刺史、右千牛衛將軍、衛州刺史。潞王從珂反於鳳翔,擁兵東至陜。愍帝懼,夜以百餘騎出奔,至衛州東七八里,遇
 晉高祖將朝於京師,騶呵前導者不避,愍帝遣左右叱之,對曰:「成德軍節度使石敬瑭也。」愍帝即下馬慟哭,謂敬瑭曰:「潞王反,康義誠等皆叛我,我無所依,長公主教我逆爾于路。」高祖曰:「衛州刺史王弘贄,宿將也,且多知時事,請就圖之。」即馳騎前見弘贄曰:「主上危迫,吾戚屬也,何以圖全?」弘贄曰:「天子避狄,自古有之,然將相大臣從乎?」曰:「無也。」「國寶、乘輿、法物從乎?」曰:「無也。」弘贄歎曰:「所謂大木將顛,非一繩所維。今萬乘之主,以百騎出奔,而將相大臣無一人從者,則人心去就可知也。雖欲興復,其可得乎!」即從高祖上謁於驛舍。高祖且以弘贄語白愍
 帝。弓箭庫使沙守榮、奔弘進前謂高祖曰:「主上,明宗愛子,公,愛婿也,公於此時不能報國,而反問大臣、國寶所在,公亦助賊反邪?」乃抽佩刀刺高祖,高祖親將陳暉扞之,守榮與暉戰死,弘進亦自刎。高祖因盡殺帝從兵,獨留帝於驛而去。弘贄奉帝居於州廨。弘贄有子巒,為殿直,廢帝入立,遣巒持鴆與弘贄。初,愍帝在衛州,弘贄令市中酒家獻酒,愍帝見之,大驚,遽殞于地,久而蘇,弘贄曰:「此酒家也,願獻酒以慰無憀。」



 愍帝受之,由是日獻一觴。及巒持鴆至,因使酒家獻之,愍帝飲而不疑,遂崩。弘贄後事晉為鳳翔行軍司馬,以光祿卿致仕,卒,贈太傅。



 劉審交劉審交,字求益,幽州文安人也。少略知書,通於吏事,為唐興令,補范陽牙校。劉守光僭號,以審交為兵部尚書,守光敗,歸于太原,唐莊宗以為從事。其後趙德鈞鎮范陽,北面轉運使馬紹宏辟審交判官。王晏球討王都,以為轉運供軍使。



 定州平,拜遼州刺史。復為北面轉運使,改慈州刺史,以母老去官。母喪,哀毀過禮,不調累年。晉高祖即位,楊光遠討范延光於魏州,審交復為供軍使。是時,晉高祖分戶部、度支、鹽鐵為三使,歲餘,三司益煩弊,乃復合為一,拜審交三司使。



 議者請檢天下民田,宜得益租,審交曰:「租有定額,而天下比年無閑田,民之苦
 樂,不可等也。」遂止不檢,而民賴以不擾。遷右衛上將軍、陳州防禦使。出視民田,見民耕器薄陋,乃取河北耕器為範,為民更鑄。安從進平,徙審交襄州,又徙青州,皆有善政。罷還。契丹犯京師,留蕭翰而去,翰復以審交為三司使。已而翰召許王從益守京師。漢高祖起義太原,從益召高行周以拒高祖,行周不至。從益母王淑妃與群臣謀迎高祖,或以謂燕兵在京師者猶數千,可以城守而待行周,淑妃不從,議未決。審交進曰:「餘燕人也,今為燕守城,當為燕謀,然事勢不可為也。



 太妃語是。」從益乃罷不設備,遣人西迎高祖。高祖至,罷審交不用。隱帝時,
 為汝州防禦使,有能名。乾祐三年卒,年七十四。州人聚哭柩前,上疏乞留葬近郊,使民得歲時祠祭。詔特贈太尉,起祠立碑。



 王周王周,魏州人也。少以勇力從軍,事唐莊宗、明宗,為裨校,以力戰有功拜刺史。晉天福中,從楊光遠討范延光於魏州,又從杜重威討安重榮於鎮州,皆有功。



 歷貝州、涇州節度使。涇州張彥澤為政苛虐,民多流亡,周乃更為寬恕,問民疾苦,去其苛弊二十餘事,民皆復歸。歷遷武勝、保義、義武、成德四鎮,皆有善政。定州橋壞,覆民租車,周曰:「橋梁不修,刺史過也。」乃償民粟,為治其橋。杜重威
 降契丹,契丹兵過鎮州,臨城呼周使出降,周泣曰:「受晉厚恩,不能死戰而以城降,何面目面行見人主與士大夫乎!」乃劇飲,求刀欲自引決,家人止之,迫以出降。契丹以周為武勝軍節度使。漢高祖入立,徙鎮武寧。卒于鎮,贈中書令。



 高行周行珪附高行周,字尚質,媯州人也。世為懷戎戍將。父思繼。思繼兄弟皆以武勇雄於北邊,為幽州節度使李匡威戍將。匡威為其弟匡儔所篡,晉王將討其亂,謀曰:「高思繼兄弟在孔領關,有兵三千,此後患也,不如遣人招之。思繼為吾用,則事無不成。」克用遣人招思繼兄弟。燕俗重氣
 義,思繼等聞晉兵為匡威報仇,乃欣然從之,為晉兵前鋒。匡儔聞思繼兄弟皆叛,乃棄城走。克用以劉仁恭守幽州,以其兄某為先鋒都指揮使,思繼為中軍都指揮使,弟某為後軍都指揮使,高氏兄弟分掌燕兵。克用臨決謂仁恭曰:「思繼兄弟,勢傾一方,為燕患者,必高氏也,宜善為防。」克用留晉兵千人為仁恭衛。而晉兵多犯法,思繼等數誅殺之。克用以責仁恭,仁恭以高氏為訴,由是晉盡誅思繼兄弟。



 仁恭以其兄某之子行珪為牙將,而思繼子行周年十餘歲,亦收之帳下,稍長,補以軍職。仁恭被囚,守光立,以行珪為武州刺史。其後守光背晉,
 晉兵攻之。守光將元行欽牧馬山後,聞守光且見圍,即率所牧馬赴援,而麾下兵叛于道,推行欽為幽州留後,行欽曰:「吾所憚者行珪也。」乃遣人之懷戎,得行珪子繫之。兵過武州,招行珪曰:「守光可取而代也。當從我行,不然,且殺公子。」行珪謝曰:「與君俱劉公將,而忍叛之?吾當為劉氏也,尚何顧吾子耶!」行欽即以兵圍行珪。



 月餘,行珪城中食盡,召其州人告曰:「吾非不為父老守也,今劉公救兵不至,奈何?可殺吾以降晉。」父老皆泣,願以死守。是時,行周適從行珪在武州,即夜縋行周馳入晉見莊宗,莊宗因遣明宗救武州。比至,行欽已解去,行珪乃降
 晉。莊宗時,歷朔忻嵐三州刺史、大同軍節度使。明宗入立,徙鎮威勝、安遠。



 行珪性貪鄙,所為多不法,副使范延策,為人剛直,數規諫之,行珪不聽,銜之。已而戍兵有謀叛者,行珪先覺之,因潛徙庫兵于佗所。戍兵叛,趨庫劫兵無所得,乃潰去,行珪追而殺之。因誣奏延策同反,并其子皆見殺,天下冤之。行珪卒於鎮,贈太尉。



 當行珪之降晉也,行周隸明宗帳下,初為裨將,趙德鈞識之,謂明宗曰:「此子貌厚而小心,佗日必大貴,宜善待之。」梁、晉軍河上,莊宗遣明宗東襲鄆州,行周將前軍,夜遇雨,軍中皆欲止不進,行周曰:「此天贊我也!鄆人恃雨,不備吾來,
 宜出其不意。」即夜馳涉濟,入其城,鄆人方覺,遂取之。莊宗滅梁,以功領端州刺史,遷絳州。明宗時,從平朱守殷,克王都,遷潁州團練使、振武軍節度使。歷鎮彰武、昭義。晉高祖時,為西京留守,徙鎮天雄。安從進叛,以行周為襄州行營都部署,討平之,徙鎮歸德。出帝時,代景延廣為侍衛親軍都指揮使。是時,李彥韜、馮玉等用事,乃求歸鎮。契丹滅晉,留蕭翰守汴,翰又棄去,召唐故許王從益入汴。而漢高祖起太原,從益遣人召行周,將以拒漢,行周嘆曰:「衰世難輔,況兒戲乎!」乃不從。漢高祖入京師,加行周守中書令,徙鎮天平軍,封臨清王。



 周太祖入立,封齊
 王。卒,贈尚書令,追封秦王。有子懷德。



 白再榮白再榮,不知其世家何人也。少為軍卒。唐、晉之間,為護聖指揮使。契丹犯京師,再榮從契丹北歸,至鎮州,契丹留麻荅守鎮州而去,晉人從者多留焉。居未幾,李筠、何福進等謀逐麻荅,使人召再榮,再榮遲疑不欲往,軍士迫之,乃往,共攻之。麻荅走,諸將以再榮名次最高,乃推為留後。再榮出於行伍,貪而無謀。



 是時,李崧、和凝等皆隨契丹留鎮州,再榮以兵環其居,迫而求物,又欲害崧取其貲。李穀謂曰:「公等親被契丹之苦,憂死不暇,然逐麻荅者,乃眾人所為,非獨公力也。今纔得生路,而遽殺
 宰相,此契丹尚或不為,然它日至京師,天子問宰相何在,何以對之?」再榮默然,乃止。而悉拘嘗事麻荅者取其財,鎮人謂之「白麻荅」。漢高祖即位,拜再榮為留後,遷義成軍節度使。罷還京師。周太祖以兵入京師,軍士攻再榮於第,悉取其財。已而前啟曰:「士卒嘗事公隸麾下,一旦無禮如此,亦復何面見公乎!」乃斬之,攜其首而去,家人以帛贖而葬之。



 安叔千安叔千,字胤宗,沙陀三部落人也。少善騎射,事唐莊宗,以為奉安指揮使。



 明宗時與討王都,拜秦州刺史。從擊契丹,為先鋒都指揮使,以功拜昭武軍節度使。



 歷靜難、橫
 海、安國、建雄四鎮。叔千狀貌堂堂,而不通文字,所為鄙陋,人謂之「沒字碑。」晉出帝時,為左金吾衛上將軍。契丹犯京師,晉百官迎見耶律德光於赤岡,叔千出班夷言,德光勞曰:「是安沒字否?汝在邢州,已通誠款,吾今至此,當與汝一吃飯處。」叔千再拜。乃以為鎮國軍節度使。漢高祖入立,罷歸京師,自以常私附契丹,頗懷愧懼。以太子太師致仕。周太祖兵入京師,軍士大掠,叔千家貲已盡,而軍士意其有所藏者,箠掠不已。傷重,歸於洛陽,卒,年七十二。



\end{pinyinscope}