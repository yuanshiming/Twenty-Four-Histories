\article{卷十一周本紀第十一}

\begin{pinyinscope}

 太祖聖神恭肅文武孝皇帝,姓郭氏,邢州堯山人也。父簡,事晉為順州刺史。



 劉仁恭攻破順州,簡見殺,子威少孤,依潞州人常氏。潞州留後李繼韜募勇敢士為軍卒,威年十八,以勇力應募。為人負氣,好使酒,繼韜特奇之。威嘗游于市,市有屠者,常以勇服其市人。威醉,呼屠者,使進幾割肉,割不如法,叱之。屠者披其腹示之曰:「爾勇者,能殺我乎?」威即前取刀刺殺之。一市皆驚,威頗自如。



 為吏所執,繼韜惜其勇,陰縱之使亡,已而復召置麾下。繼韜叛晉附於梁,後莊宗滅梁,繼韜誅死,其麾下兵悉隸從馬直,威以通書算補為軍吏。好讀《閫外春秋》,略知兵法,後為侍衛軍吏。漢高祖為侍衛親軍都虞候,尤親愛之。後高祖所臨鎮,嘗以威從。契丹滅晉,漢高祖起兵太原,即皇帝位,拜威樞密副使。



 乾祐元年正月,高祖疾大漸,以隱帝託威及史弘肇等。隱帝即位,拜威樞密使。



 是歲三月,河中李守貞、永興趙思綰、鳳翔王景崇相次反,隱帝遣白文珂、郭從義、常思等分討之,久皆無功。隱帝謂威曰:「吾欲煩公可乎?」威對
 曰:「臣不敢請,亦不敢辭,惟陛下命。」乃加拜威同中書門下平章事,使西督諸將。威居軍中,延見賓客,褒衣博帶,及臨陣行營,幅巾短後,與士卒無異;上所賜予,與諸將會射,恣其所取,其餘悉以分賜士卒,將士皆懽樂。威至河中,自柵其城東,思柵其南,文珂柵其西,調五縣丁二萬人築連壘以護三柵。諸將皆謂守貞窮寇,破在旦夕,不宜勞人如此,威不聽。已而守貞數出兵擊壞連壘,威輒補之,守貞輒復出擊,每出必有亡失。久之,城中兵食俱盡,威曰:「可矣!」乃治攻具,為期日,四面攻之,破其羅城,守貞與妻子自焚死,思綰、景崇相次降。



 隱帝勞威以
 玉帶,加檢校太師兼侍中,威辭曰:「臣事先帝,見功臣多矣,未嘗以玉帶賜之。」因言:「臣幸得率行伍,假漢威靈以破賊者,豈特臣之功,皆將相之賢,有以安朝廷,撫內外,而饋餉以時,故臣得以專事征伐。」隱帝以威為賢,於是悉召楊邠、史弘肇、蘇逢吉、禹珪、竇貞固、王章等皆賜以玉帶,威乃受。威又推功大臣,請加爵賞,於是加貞固司空,逢吉司徒,禹珪、邠左右僕射。已而又曰:「此特漢廷親近之臣耳。漢諸宗室、天下方鎮,外暨荊、浙、湖南,皆未及也。」



 由是濫賞遍于天下。



 是冬,契丹寇邊,威以樞密使北伐,至魏州,契丹遁。三年二月,師還。四月,拜威鄴都留守、天
 雄軍節度使,仍以樞密使之鎮。宰相蘇逢吉以謂樞密使不可以籓鎮兼領,與史弘肇等固爭。久之,卒以樞密使行,詔河北諸州皆聽威節度。



 隱帝與李業等謀,已殺史弘肇等,詔鎮寧軍節度使李弘義殺侍衛步軍指揮使王殷于澶州,又詔侍衛馬軍指揮使郭崇殺威及宣徽使王峻於魏。詔書先至澶州,弘義恐事不果,反以詔書示殷,殷與弘義遣人告威。已而詔殺威、峻使者亦馳騎至,威匿詔書,召樞密使院吏魏仁浦謀於臥內。仁浦勸威反,教威倒用留守印,更為詔書,詔威誅諸將校以激怒之,將校皆憤然效用。



 十一月丁丑,威遂舉兵渡河。
 隱帝遣開封尹侯益、保大軍節度使張彥超、客省使閻晉卿等率兵拒威,又遣內養脫覘威所嚮。脫為威所得,威乃附脫奏請縛李業等送軍中。隱帝得威奏,以示業等,業等皆言威反狀已白,乃悉誅威家屬于京師。



 庚辰,威至滑州,義成軍節度使宋延渥叛于漢來降。壬午,犯封丘。甲辰,及泰寧軍節度使慕容彥超戰于劉子陂,彥超敗,奔于兗州。郭允明反,弒隱帝于趙村。丙戌,威入京師,縱火大掠。戊子,率百官朝太后於明德門,請立嗣君。太后下令:文武百寮、六軍將校,議擇賢明,以承大統。庚寅,威率百官詣明德門,請立武寧軍節度使贇為
 嗣。遣太師馮道迎贇于徐州。辛卯,請太后臨朝聽政,以王峻為樞密使,翰林學士、尚書兵部侍郎范質為副使。



 十二月甲午朔,威北伐契丹,軍于滑州。癸丑,至澶州而旋。王峻遣郭崇以騎七百逆劉贇于宋州,殺之,其將鞏廷美、楊溫為贇守徐州。戊午,次皋門,漢宰相竇貞固、蘇禹珪來勸進。庚申,太后制以威監國。



 廣順元年春正月丁卯,皇帝即位,大赦,改元,國號周。己巳,上漢太后尊號曰昭聖皇太后。戊寅,漢劉崇自立于太原。己卯,馮道為中書令。二月辛丑,西州回鶻使都督來。丁未,契丹兀欲遣使裊骨
 支來。癸丑,寒食,望祭於蒲池。丁巳,尚書左丞田敏使于契丹。回鶻使摩尼來。三月甲戌,武寧軍節度使王彥超克徐州。



 夏四月甲午,立夫人董氏為德妃。五月辛未,追尊祖考為皇帝,妣為皇后:高祖璟謚曰睿和,廟號信祖,祖妣張氏謚曰睿恭;曾祖諶謚曰明憲,廟號僖祖,祖妣申氏謚曰明孝;祖蘊謚曰翼順,廟號義祖,祖妣韓氏謚曰翼敬;考謚曰章肅,廟號慶祖,妣王氏謚曰章德。六月辛亥,范質及戶部侍郎判三司李穀為中書侍郎、同中書門下平章事。竇貞固、蘇禹珪罷。癸丑,范質參知樞密院事。丁巳,宣徽北院使翟光
 鄴為樞密副使。秋七月戊寅,幸王峻第。八月壬寅,契丹來歸趙瑩之喪。冬十月丙午,漢人來討,攻自晉州。十一月,王峻及建雄軍節度使王彥超拒之。十二月,慕容彥超反。



 二年春正月甲子,侍衛步軍都指揮使曹英為兗州行營都部署。庚午,高麗王昭使其廣評侍郎徐逢來。二月庚寅,府州防禦使折德扆克岢嵐軍。三月丁巳朔,寒食,望祭於郊。戊辰,內客省使鄭仁誨為樞密副使,翟光鄴罷。夏五月庚申,東征,李穀留守東都,鄭仁誨為大內都
 點檢。癸亥,次曹州,赦流罪以下囚。乙亥,克兗州。



 壬午,赦兗州。六月乙酉朔,幸曲阜,祠孔子。庚子,至自兗州。秋九月乙丑,太僕少卿王演使于高麗。契丹寇邊。



 三年春正月乙卯,麟州刺史楊重訓叛於漢,來附。閏月丙戌,回鶻使獨呈相溫來。二月甲子,貶王峻為商州司馬。三月甲申,封榮為晉王。丙戌,鄭仁誨罷。己丑,棣州團練使王仁鎬為左衛大將軍、樞密副使。夏六月,大雨,水。秋七月,契丹盧臺軍使張藏英來奔。九月,吐渾黨富達等來。冬十月庚
 申,馮道為奉迎神主使。



 十一月癸未,黨項使吳帖磨五等來。十二月戊申,四廟神主至自西京,迎之於西郊,祔於太廟。壬申,殺天雄軍節度使王殷。乙亥,享於太廟。



 顯德元年春正月丙子朔,有事於南郊,大赦,改元,群臣上尊號曰聖明文武仁德皇帝。戊寅,罷鄴都。丙戌,鎮寧軍節度使鄭仁誨為樞密使。壬辰,端明殿學士、戶部侍郎王溥為中書侍郎、同中書門下平章事,王仁鎬罷。是日,皇帝崩於滋德殿。



\end{pinyinscope}