\article{卷五十九司天考第二}

\begin{pinyinscope}

 昔孔子作《春秋》而天人備。予述本紀,書人而不書天,予何敢異於聖人哉!



 其文雖異,其意一也。



 自堯、舜、三代以來,莫不稱天以舉事,孔子刪《詩》、《書》不去也。蓋聖人不絕天於人,亦不以天參人。絕天於人則天道廢,以天參人則人事惑,故常存而不究也。《春秋》雖書日食、星變之類,孔子未嘗道其所以然者,故其弟子之徒,莫得有所述於後世也。然則天果與於人乎?果不與乎?曰:天,吾
 不知,質諸聖人之言可也。《易》曰:「天道虧盈而益謙,地道變盈而流謙,鬼神害盈而福謙,人道惡盈而好謙。」此聖人極論天人之際,最詳而明者也。其於天地鬼神,以不可知為言,其可知者人而已。夫日中則昃,盛衰必復。天,吾不知,吾見其虧益於物者矣。草木之成者,變而衰落之;物之下者,進而流行之。地,吾不知,吾見其變流於物者矣。人之貪滿者多禍,其守約者多福。鬼神,吾不知,吾見人之禍福者矣。



 天地鬼神,不可知其心,則因其著於物者以測之。故據其迹之可見者以為言,曰虧益,曰變流,曰害福。若人則可知者,故直言其情曰好惡。其知與
 不知,異辭也,參而會之,與人無以異也。其果與於人乎,不與於人乎,則所不知也。以其不可知,故常尊而遠之;以其與人無所異也,則修吾人事而已。人事者,天意也。《書》曰:「天視自我民視,天聽自我民聽。」未有人心悅於下而天意怒於上者,未有人理逆於下而天道順於上者。然則王者君天下,子生民,布德行政,以順人心,是之謂奉天。至於三辰五星常動而不息,不能無盈縮差忒之變,而占之有中有不中,不可以為常者,有司之事也。本紀所述人君行事詳矣,其興亡治亂可以見。至於三辰五星逆順變見,有司之所占者,故以其官志之,以備司
 天之所考。



 嗚呼,聖人既沒,而異端起。自秦、漢以來,學者惑於災異矣,天文五行之說,不勝其繁也。予之所述,不得不異乎《春秋》也,考者可以知焉。



 開平二年夏四月辛丑,熒惑犯上將。甲寅,地震。四年十二月庚午,月有食之。



 乾化元年春正月丙戌朔,日有食之。五月,客星犯帝坐。二年正月丙申,熒惑犯房第二星。戊申,月犯心大星。四月甲寅,月掩心大星。壬申,彗出於張;甲戌,彗出靈臺。



 同光元年十月辛未朔,日有食之。二年六月甲申,眾星交流;丙戌,眾星交流。



 八月戊子,熒惑犯星。十一月丁巳,地震。三年三月丙申,熒惑犯上相。戊
 申,月有食之。四月癸亥朔,日有食之。甲子,熒惑犯左執法。六月甲子,太白晝見。丙寅,歲犯右執法。己巳,太白晝見。庚寅,眾星流,自二更盡三更而止。辛卯,眾小星流於西南。九月甲辰,月有食之。丁未,天狗墮,有聲如雷,野雉皆雊。丙辰,太白、歲相犯。十一月甲寅,地震。



 天成元年三月,惡星入天庫,流星犯天棓。四月庚戌,金犯積尸。六月乙未,眾小星交流。七月己未,月犯太白。庚申,太白晝見。乙丑,月入南斗魁。八月乙酉朔,日有食之。癸卯,太白犯心大星。乙巳,月犯五諸侯。辛亥,熒惑犯上將。



 九月丁巳,月犯心大星。己巳,月犯昴。庚午,熒惑犯右執法;己卯,熒
 惑犯左執法。十月戊子,熒惑犯上相。己丑至于庚子,日月赤而無光。丙午,月掩左執法。



 十一月丁丑,月暈匝火、木。戊寅,月犯金、木、土。十二月戊戌,熒惑犯氐。乙巳,月掩庶子。二年正月甲戌,熒惑、歲相犯。二月辛卯,熒惑犯鍵閉。三月戊午,月掩鬼。庚申,眾小星流于西北。己巳,熒惑犯上相。乙亥,月入羽林。四月丁亥,月犯右執法;癸卯,月入羽林。六月辛丑,熒惑犯房。八月己卯朔,日有食之。庚子,月犯五諸侯。九月壬子,歲犯房。庚申,月入羽林;壬申,月犯上將。十月壬午,月犯五諸侯。癸未,地震。十一月乙卯,月入羽林。辛未,地震;壬申,地震。



 十二月癸未,地震。三
 年春正月壬申,金、火合于奎。二月丁丑朔,日有食之。四月丁酉,月犯五諸侯;五月丁巳,月掩房距星;六月乙酉,月掩心庶子;癸巳,月入羽林。自正月至于是月,宗人、宗正搖不止。七月乙卯,月入南斗魁。閏八月癸卯朔,熒惑犯上將。戊申,月犯南斗。乙卯,熒惑犯右執法。庚戌,太白犯右執法。



 九月庚辰,土、木合于箕。辛巳,金、火合于軫。十月庚午,彗出西南。十一月戊子,月掩軒轅大星。乙未,太白犯鎮,月掩房。十二月壬寅朔,熒惑犯房,金、木相犯于斗。乙卯,月有食之。四年正月癸巳,月入南斗魁。二月辛酉,月及火、土合于斗。三月壬辰,歲犯牛。六月癸丑,月有
 食之,既。七月丁丑,月入南斗。九月丙子,熒惑入哭星。十二庚戌,月有食之,既。



 長興元年六月癸巳朔,日有食之。乙卯,太白犯天鐏。八月己亥,月犯南斗。



 乙卯,月犯積尸。九月辛酉朔,眾小星交流而殞。十一月壬戌,熒惑犯氐。十二月丙辰,熒惑犯天江。二年正月乙亥,太白犯羽林。庚辰,月犯心距星;二月丁未,月犯房。四月甲寅,熒惑犯羽林。五月癸亥,太白晝見。閏五月乙巳,歲晝見。六月壬午,地震。八月丁巳,辰犯端門。九月丙戌,眾星交流;丁亥,眾星交流而殞。



 戊子,太白晝見。丁未,雷。十一月甲申朔,日有食之。丙戌,太白犯鍵。三年四月庚辰,熒惑犯積
 尸。九月庚寅,太白犯哭星。十月壬申,太白晝見。十一月己亥,太白犯壁壘。四年五月癸卯,太白晝見。六月庚午,眾星交流。七月乙亥朔,眾星交流。九月辛巳,太白犯右執法。乙未,雷。



 應順元年二月丁酉,眾星流于西北。四月戊寅,白虹貫日。是月改元。



 清泰元年五月己未,太白晝見。六月甲戌,太白犯右執法。九月辛丑,眾星交流。壬寅,雨雹於京師。冬十一月丁未,彗出虛、危,掃天壘及哭星。



 天福元年三月壬子,熒惑犯積尸。二年正月乙卯,日有食之。七月丙寅,月有食之。十二月己卯朔,日有白虹二。三年三月壬子,日有白虹二。五月壬子,月犯上
 將。四年四月辛巳,太白犯東井北轅;甲午,太白犯五諸侯;五月丁未,太白犯輿鬼中星。七月庚子朔,日有食之。九月癸未,月掩畢。五年十一月丁丑,月有食之。六年八月辛卯,太白犯軒轅。九月己卯,熒惑犯上將。壬子,彗出於西,掃天市垣。八年四月戊申朔,日有食之。八月丙子,熒惑犯右掖。十月庚戌,彗出東方。



 丙辰,熒惑犯進賢。十一月庚子,月犯房。



 開運元年二月辛亥,日有白虹二。壬戌,太白犯昴。己巳,熒惑犯天鑰。三月戊子,月有食之。四月丁巳,太白犯五諸侯。七月庚辰,月犯熒惑;壬午,月入南斗。甲申,太白犯東井。八月甲辰,熒惑入南斗。九月庚
 午朔,日有食之。丙子,月入南斗;乙酉,月食昴。丙戌,月有食之。庚寅,月犯五諸侯;十月癸卯,月入南斗;十一月辛巳,月犯昴。十二月癸丑,太白犯辰。二年七月乙未朔,月犯角;壬寅,月犯心前大星。庚戌,歲犯井鉞。八月甲子朔,日有食之。甲戌,歲犯東井。



 九月己酉,月犯昴。甲寅,太白犯南斗魁。十一月甲午朔,太白犯哭星。癸丑,月掩角距星;戊午,月犯心後星。三年二月壬戌朔,日有食之。



 天福十二年四月丙子,太白晝見。十月己丑,太白犯亢距星。十一月壬子,雨木冰。辛酉,雨木冰。壬戌,月犯昴。癸酉,雨木冰。乙亥,月掩心大星;乙卯,月犯南斗。十二月乙未,月
 有食之。



 乾祐元年四月甲午,月犯南斗。六月戊寅朔,日有食之。乙未,月入南斗。七月甲寅,月掩心庶子星。八月乙酉,鎮犯太微西垣。戊戌,歲犯右執法。九月丁卯,月掩鬼。十月丁丑,歲犯左執法。二年四月壬午,太白晝見。六月癸酉朔,日有食之。壬午,月犯心;丙戌,月犯天關;八月乙亥,月犯房次將。九月壬寅,太白犯右執法。庚戌,太白犯鎮。辛酉,鎮犯右執法。丁卯,太白犯歲。鎮自元年八月己丑入太微垣,犯上將、執法、內屏、謁者,勾己往來,至是歲十一月辛亥而出。甲寅,月犯昴。三年二月甲戌,月犯昴。六月乙卯,鎮犯左掖。七月甲申,熒惑
 犯司怪。八月癸卯,太白犯房;庚戌,太白犯心大星。十月辛酉,月犯心大星,太白犯木。十一月甲子朔,日有食之。



 廣順元年二月丁巳,歲犯咸池。己未,熒惑犯五諸侯。三月甲子,歲守心。己卯,熒惑犯鬼;壬午,熒惑犯天尸。四月甲午,歲犯鉤鈐。二年二月庚寅,太白經天。四月丙戌朔,日有食之。七月乙丑,熒惑犯井鉞;八月乙未,熒惑犯天鐏。九月辛酉,熒惑犯鬼。庚辰,太白掩右執法。十月壬辰,太白犯進賢。三年四月乙丑,熒惑犯靈臺;五月辛巳,熒惑犯上將;丙申,熒惑犯右執法。七月乙酉,月犯房。



 十二月戊申,雨木冰。



 顯德元年正月庚寅,有大星墜,有聲如
 雷,牛馬皆逸,京城以為曉鼓,皆伐鼓以應之。三年正月壬戌,有星孛于參。十二月庚午,白虹貫日。癸酉,月有食之。



 五代亂世,文字不完,而史官所記亦有詳略,其日、月、五星之變,大者如此。



 至於氣祲之象,出沒銷散不常,尤難占據。而五代之際,日有冠珥、環暈、纓紐、負抱、戴履、背氣,十日之中常七八,其繁不可以勝書,而背氣尤多。天福八年正月丙戌,黃霧四塞。九年正月乙未,大霧中二白虹相偶。四月庚戌,大霧中有蒼白二虹。廣順元年十一月甲子,白虹竟天。此其尤異者也。至於吳火出楊林江
 水中、閩天雨豆之類,皆非中國耳目所及者,不可得而悉書矣。



\end{pinyinscope}