\article{卷五十三雜傳第四十一}

\begin{pinyinscope}

 王景崇王景崇,邢州人也。為人明敏巧辯,善事人。唐明宗鎮邢州,以為牙將,其後嘗從明宗,隸麾下。明宗即位,拜通事舍人,歷引進閣門使,馳詔方鎮、監軍征伐,必用景崇。後事晉,累拜左金吾衛大將軍,常怏怏人主不能用其材。晉亡,蕭翰據京師,景崇厚賂其將高牟翰以求用。已而翰北歸,許王從益居京師,用景崇為宣徽使、監左藏庫。漢高祖起太原,景崇取庫金奔迎高祖。高祖至京師,拜
 景崇右衛大將軍,未之奇也。高祖攻鄴,景崇不得從,乃求留守起居表,詣行在見高祖,願留軍中效用,為高祖畫攻戰之策,甚有辯,高祖乃奇其材。



 是時,漢方新造,鳳翔侯益、永興趙贊皆嘗受命契丹,高祖立,益等內顧自疑,乃陰召蜀人為助,高祖患之。及已破鄴,益等懼,皆請入朝。會回鶻入貢,言為黨項所隔不得通,願得漢兵為援,高祖遣景崇以兵迎回鶻。景崇將行,高祖已疾,召入臥內戒之曰:「益等已來,善矣,若猶遲疑,則以便宜圖之。」景崇行至陜,趙贊已東入朝,而蜀兵方寇南山,景崇擊破蜀兵,追至大散關而還。高祖乃詔景崇兼鳳翔巡檢
 使。



 景崇至鳳翔,侯益未有行意,而高祖崩,或勸景崇可速誅益,景崇念獨受命先帝而少主莫知,猶豫未決。益從事程渥,與景崇同鄉里,有舊,往說景崇曰:「吾與子為故人,吾位不過賓佐,而子已貴矣,奈何欲以陰狡害人而取之乎?侯公父子爪牙數百,子毋妄發,禍行及矣!非吾,誰為子言之。」於是景崇頗不欲殺益,益乃亡去,景崇大悔失不殺之。



 益至京師,隱帝新立,史弘肇、楊邠等用事,益乃厚賂邠等,陰以事中景崇。



 已而益拜開封尹,景崇心不自安,諷鳳翔將吏求己領府事。朝廷患之,拜景崇邠州留後,以趙暉為鳳翔節度使。景崇乃叛,盡殺侯
 益家屬,與趙思綰共推李守貞為秦王,隱帝即以趙暉討之。景崇西招蜀人為助,蜀兵至寶雞,為暉將藥元福、李彥從所敗。暉攻鳳翔,塹而圍之,數以精兵挑戰,景崇不出。暉乃令千人潛之城南一舍,偽為蜀兵旗幟,循南山而下,聲言蜀救兵至矣,須臾塵起,景崇以為然,乃令數千人潰圍而出以為應。暉設伏以待之,景崇兵大敗,由是不敢復出。



 明年,守貞、思綰相次皆敗,景崇客周璨謂景崇曰:「公能守此者,以有河中、京兆也。今皆敗矣,何所恃乎?不如降也。」景崇曰:「誠累君等,然事急矣,吾欲為萬有一得之計可乎?吾聞趙暉精兵皆在城北,今使公
 孫輦等燒城東門偽降,吾以牙兵擊其城北兵,脫使不成而死,猶勝於束手屯。」璨等皆然之。遲明,輦燒東門將降,而府中火起,景崇自焚矣,輦乃降暉。



 趙思綰趙思綰,魏州人也。為河中節度使趙贊牙將。漢高祖即位,徙贊鎮永興,贊入朝京師,留思綰兵數百人於永興。高祖遣王景崇至永興,與齊藏珍以兵迎回鶻,陰以西事屬之。景崇至永興,贊雖入朝,而其所召蜀兵已據子午谷,景崇用思綰兵擊走之。遂與思綰俱西,然以非己兵,懼思綰等有二心,意欲黥其面以自隨,而難言之,乃稍微風其旨。思綰厲聲請先黥以率眾,齊藏珍惡之,竊
 勸景崇殺思綰,景崇不聽,與俱西。



 高祖遣使者召思綰等,是時侯益來朝,思綰以兵從益東歸。思綰謂其下常彥卿曰:「趙公已入人手,吾屬至,并死矣,奈何?」彥卿曰:「事至而變,勿預言也。」



 益行至永興,永興副使安友規出迎益,飲于郊亭,思綰前曰:「兵館城東,然將士家屬皆居城中,願縱兵入城挈其家屬。」益信之以為然。思綰與部下入城,有州校坐於城門,思綰毆之,奪其佩刀斬之,并斬門者十餘人,遂閉門劫庫兵以叛。



 高祖遣郭從義、王峻討之,經年莫能下,而王景崇亦叛,與思綰俱送款於李守貞,守貞以思綰為晉昌軍節度使。隱帝遣郭威西督
 諸將兵,先圍守貞於河中。居數月,思綰城中食盡,殺人而食,每犒宴,殺人數百,庖宰一如羊豕。思綰取其膽以酒吞之,語其下曰:「食膽至千,則勇無敵矣!」思綰計窮,募人為地道,將走蜀,其判官陳讓能謂思綰曰:「公比於國無嫌,但懼死而為此爾!今國家用兵三方,勞敝不已,誠能翻然效順,率先自歸,以功補過,庶幾有生;若坐守窮城,待死而已。」



 思綰然之,乃遣教練使劉珪詣從義乞降,而遣其將劉筠奉表朝廷。拜思綰鎮國軍留後,趣使就鎮,思綰遲留不行。蜀陰遣人招思綰,思綰將奔蜀,而從義亦疑之,乃遣人白郭威,威命從義圖之。從義因入城
 召思綰,趣之上道,至則擒之。思綰問曰:「何以用刑?」告者曰:「立釘也。」思綰厲聲曰:「為吾告郭公,吾死未足塞責,然釘磔之醜,壯夫所恥,幸少假之。」從義許之,父子俱斬於市。



 慕容彥超慕容彥超,吐谷渾部人,漢高祖同產弟也。嘗冒姓閻氏,彥超黑色胡髯,號閻崑倉。少事唐明宗為軍校,累遷刺史。唐、晉之間,歷磁、單、濮、棣四州,坐濮州造麴受賕,法當死,漢高祖自太原上章論救,得減死,流于房州。契丹滅晉,漢高祖起太原,彥超自流所逃歸漢,拜鎮寧軍節度使。杜重威反于魏,高祖以天平軍節度使高行周為都
 部署以討之,以彥超為副。彥超與行周謀議多不協,行周用兵持重,兵至城下,久之不進。彥超欲速進戰,而行周不許。行周有女嫁重威子,彥超揚言行周以女故,惜賊城而不攻,行周大怒。高祖聞二人不相得,懼有佗變,由是遽親征。彥超數以事凌辱行周,行周不能忍,見宰相涕泣,以屎塞口以自訴。高祖知曲在彥超,遣人慰勞行周,召彥超責之,又遣詣行周謝過,行周意稍解。



 是時,漢兵頓魏城下已久,重威守益堅,諸將皆知未可圖,方伺其隙,而彥超獨言可速攻,高祖以為然,因自督士卒急攻,死傷者萬餘人,由是不敢復言攻。後重威出降,高祖
 以行周為天雄軍節度使,行周辭不敢受,高祖遣蘇逢吉諭之曰:「吾當為爾徙彥超。」行周乃受,而彥超徙鎮泰寧。



 隱帝已殺史弘肇等,又遣人之魏殺周太祖及王峻等,懼事不果,召諸將入衛京師。使者至兗,彥超方食,釋匕箸而就道。周兵犯京師,開封尹侯益謂隱帝曰:「北兵之來,其家屬皆在京師,宜閉門以挫其銳,遣其妻子登陴以招北兵,可使解甲。」彥超誚益曰:「益老矣!此懦夫之計也。」隱帝乃遣彥超副益,將兵于北郊。



 周兵至,益夜叛降于周。彥超力戰于七里,隱帝出勞軍,太后使人告彥超善衛帝,彥超大言報曰:「北兵何能為?當於陣上喝坐
 使歸營。」又謂隱帝曰:「官家宮中無事,明日可出觀臣戰。」明日隱帝復出勞軍,彥超戰敗奔兗州,隱帝遇弒于北郊。



 周太祖入立,彥超不自安,數有所獻,太祖報以玉帶,又賜詔書安慰之,呼彥超為弟而不名,又遣翰林學士魯崇諒往慰諭之,彥超心益疑懼。已而劉旻自立于太原,出兵攻晉、絳,太祖遣王峻用兵西方,彥超乘間亦謀反,遣押衙鄭麟至京師求入朝,太祖知其詐,手詔許之。彥超復稱管內多盜而止,又為高行周所與書以進,其辭皆指斥周過失,若欲共反者。太祖驗其印文偽,以書示行周。彥超又遣人南結李昪,昪為出兵攻沐陽,為周兵
 所敗,而劉旻攻晉、絳不克,解去。太祖乃遣侍衛步軍指揮使曹英、客省使向訓討之,彥超閉城自守。



 初,彥超之反也,判官崔周度諫曰:「魯,詩書之國也,自伯禽以來未有能霸者,然以禮義守之而長世者多矣。今公英武,一代之豪傑也,若量力相時而動,可以保富貴終身。李河中、安襄陽、鎮陽杜令公,近歲之龜鑒也。」彥超大怒,未有以害之。已而見圍,因大括城中民貲以犒軍,前陜州司馬閻弘魯懼其鞭撲,乃悉家貲以獻。彥超以為未盡,又欲並罪周度,乃令周度監括弘魯家。周度謂弘魯曰:「公命之死生,繫財之多少,願無隱也。」弘魯遣家僮與周
 度掘搜索無所得。



 彥超又遣鄭麟持刃迫之,弘魯惶恐拜其妻妾,妻妾皆言無所隱。周度入白彥超,彥超不信,下弘魯及周度于獄。弘魯乳母於泥中得金纏臂獻彥超,欲贖出弘魯,彥超大怒,遣軍校笞弘魯夫婦肉爛而死,遂斬周度于市。



 是歲,鎮星犯角、亢,占曰:「角、亢,鄭分,兗州當焉。」彥超即率軍府將吏步出西門三十里致祭,迎於開元寺,塑像以事之,日常一至,又使民家立黃幡以禳之。



 彥超為人多智詐而好聚斂,在鎮嘗置庫質錢,有奸民為偽銀以質者,主吏久之乃覺。彥超陰教主吏夜穴庫垣,盡徙其金帛于佗所而以盜告。彥超即榜于市,
 使民自占所質以償之,民皆爭以所質物自言,已而得質偽銀者,置之深室,使教十餘人日夜為之,皆鐵為質而包雙銀,號「鐵胎銀」。其被圍也,勉其城守者曰:「吾有銀數千鋌,當悉以賜汝。」軍士私相謂曰:「此鐵胎爾,復何用哉!」皆不為之用。



 明年五月,太祖親征,城破,彥超夫妻皆投井死,其子繼勛率其徒五百人出奔被擒,遂滅其族。兗州平,太祖詔贈閻弘魯左驍衛大將軍、崔周度祕書監。



\end{pinyinscope}