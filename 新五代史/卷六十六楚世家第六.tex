\article{卷六十六楚世家第六}

\begin{pinyinscope}

 馬殷,字霸圖,許州鄢陵人也。唐中和三年,蔡州秦宗權遣孫儒、劉建峰將兵萬人屬其弟宗衡,略地淮南,殷初為儒裨將。宗衡等攻楊行密於揚州,未克,梁兵方急攻宗權,宗權數召儒等,儒不欲還,宗衡屢趨之,儒怒,殺宗衡,自將其兵取高郵,遂逐行密。行密據宣州,儒以兵圍之,久不克,遣殷與建峰掠食旁縣。儒戰敗死,殷等無所歸,乃推建峰為帥,殷為先鋒,轉攻豫章,略虔、吉,有眾數
 萬。



 乾寧元年,入湖南,次醴陵。潭州刺史鄧處訥發邵州兵戍龍回關,建峰等至關,降其戍將蔣勛。建峰取勛鎧甲被先鋒兵,張其旗幟,直趨潭州,至東門,東門守者以為關兵戍還,開門內之,遂殺處訥,建峰自稱留後。僖宗授建峰湖南節度使、殷為馬步軍都指揮使。蔣勛求為邵州刺史,建峰不與,勛率兵攻湘鄉,建峰遣殷擊勛於邵州。



 建峰庸人,不能帥其下,常與部曲飲酒讙呼。軍卒陳贍妻有色,建峰私之,贍怒,以鐵楇擊殺建峰。軍中推行軍司馬張佶為帥,佶將入府,乘馬輒踶齧,傷佶髀。



 佶臥病,語諸將曰:「吾非汝主也,馬公英勇,可共立之。」諸將
 乃共殺贍,磔其尸,遣姚彥章迎殷於邵州。殷至,佶乘肩輿入府,殷拜謁於廷中,佶召殷上,乃率將吏下,北面再拜,以位與之,時乾寧三年也。



 唐拜殷潭州刺史。殷遣其將秦彥暉、李瓊等攻連、邵、郴、衡、道、永六州,皆下之。桂管劉士政懼,遣其將陳可璠、王建武等率兵守全義嶺。殷遣使聘于士政,使者至境上,可璠等不納。殷怒,遣瓊等以兵七千攻之,擒可璠等及其兵二千餘人,悉坑之,遂圍桂管,虜士政,盡取其屬州。殷表瓊桂管觀察使。四年,拜殷武安軍節度使。



 初,孫儒敗於宣州,殷弟賨為楊行密所執,行密收儒餘兵為「黑雲都」,以賨為指揮使。賨從
 行密攻戰,數有功,為人質重,未嘗自矜,行密愛之,問賨誰家子,賨曰:「馬殷弟也。」行密大驚曰:「汝兄貴矣,吾今歸汝可乎?」賨不對。他日又問之,賨謝曰:「臣,孫儒敗卒也,幸公待以不死,非殺身不足報。湖南鄰境,朝夕聞殷動靜足矣,不願去也。」行密歎曰:「昔吾愛子之貌,今吾得子之心矣。



 然勉為吾合二國之懽,通商賈、易有無以相資,亦所以報我也!」乃厚禮遣賨歸。



 殷大喜,表賨節度副使。



 行密遣將劉存等攻杜洪,圍鄂州,殷遣秦彥暉、許德勛以舟兵救之,已而杜洪敗死,存等遂攻殷。殷遣彥暉拒於上流,偏將黃璠以舟三百伏瀏陽口。存等屢戰不
 勝,乃致書于殷以求和,殷欲許之,彥暉曰:「淮人多詐,將怠我師,不可信。」



 急擊之,存等退走,黃璠以瀏陽舟截江合擊,大敗之,劉存及陳知新戰死,彥暉取岳州。



 梁太祖即位,殷遣使修貢,太祖拜殷侍中兼中書令,封楚王。荊南高季昌以兵斷漢口,邀殷貢使,殷遣許德勳攻其沙頭,季昌求和,乃止。楊行密袁州刺史呂師周來奔。師周,勇健豪俠,頗通緯候、兵書,自言五世將家,懼不能免,常與酒徒聚飲,醉則起舞,悲歌慷慨泣下。行密聞之,疑其有異志,使人察其動靜。師周益懼,謂其裨將綦毋章曰:「吾與楚人為敵境,吾常望其營上雲氣甚佳,未易敗也。



 吾聞馬公仁者,待士有禮,吾欲逃死於楚可乎?」章曰:「公自圖之,章舌可斷,語不泄也。」師周以兵獵境上,乃奔於楚,綦毋章縱其家屬隨之。殷聞師周至,大喜曰:「吾方南圖嶺表而得此人,足矣。」以為馬步軍都指揮使,率兵攻嶺南,取昭、賀、梧、蒙、龔、富等州。殷表師周昭州刺史。



 朗州雷彥恭召吳人攻平江,許德勳擊敗之。殷遣秦彥暉攻朗州,彥恭奔于吳,執其弟彥雄等七人送于梁。於是澧州向瑰、辰州宋鄴、漵州昌師益等率溪洞諸蠻皆附于殷。殷請升朗州為永順軍,表張佶節度使。殷乃請依唐太宗故事,開天冊府,置官屬。太祖拜殷天冊上將軍,殷
 以其弟賨為左相,存為右相,廖光圖等十八人為學士。末帝時,加殷武昌、靜江、寧遠等軍節度使,洪、鄂四面行營都統。



 唐莊宗滅梁,殷遣其子希範修貢京師,上梁所授都統印。莊宗問洞庭廣狹,希範對曰:「車駕南巡,纔堪飲馬爾。」莊宗嘉之。莊宗平蜀,殷大懼,表求致仕,莊宗下璽書慰勞之。明宗即位,遣使修貢,并賀明年正月,荊南高季昌執其貢使史光憲。殷遣袁詮、王環等攻之,至其城下,季昌求和,乃止。



 殷初兵力尚寡,與楊行密、成汭、劉等為敵國,殷患之,問策於其將高郁,郁曰:「成汭地狹兵寡,不足為吾患,而劉志在五管而已,楊行密,
 孫儒之仇,雖以萬金交之,不能得其懽心。然尊王仗順,霸者之業也,今宜內奉朝廷以求封爵而外誇鄰敵,然後退修兵農,畜力而有待爾。」於是殷始修貢京師,然歲貢不過所產茶茗而已。乃自京師至襄、唐、郢、復等州置邸務以賣茶,其利十倍。郁又諷殷鑄鉛鐵錢,以十當銅錢一。又令民自造茶以通商旅,而收其算,歲入萬計。由是地大力完,數邀封爵。



 天成二年,請建行臺。明宗封殷楚國王,有司言無封國王禮,請如三公用竹冊,乃遣尚書右丞李序持節以竹冊封之。殷以潭州為長沙府,建國承制,自置官屬,以其弟賨為靜江軍節度使,子希振
 武順軍節度使,次子希聲判內外諸軍事,姚彥章為左相,許德勳為右相,李鐸為司徒,崔穎為司空,拓拔常為僕射,馬珙為尚書,文武皆進位。謚其曾祖筠曰文肅、祖正曰莊穆、父元豐曰景莊,立三廟于長沙。長興元年,殷卒,年七十九,詔曰「馬殷官爵俱高,無以為贈,謚曰武穆」而已。子希聲立。



 希聲字若訥,殷次子也。殷建國,以希聲判內外諸軍事。荊南高季昌聞殷將高郁素教殷以計策而楚以彊,患之,嘗使諜者行間於殷,殷不聽。希聲用事,諜者語希聲曰:「季昌聞楚用高郁,大喜,以為亡馬氏者必郁也。」希聲
 素愚,以為然,遽奪郁兵職,郁怒曰:「吾事君王久矣,亟營西山,將老焉,犬子漸大,能咋人矣!」



 希聲聞之,矯殷令殺郁。殷老不復省事,莫知郁死,是日大霧四塞,殷怪之,語左右曰:「吾嘗從孫儒,儒每殺不辜,天必大霧,豈馬步獄有冤死乎?」明日,吏以狀白,殷拊膺大哭曰:「吾荒耄如此,而殺吾勳舊!」顧左右曰:「吾亦不久於此矣!」明年殷薨。



 希聲立,授武安、靜江等軍節度使。希聲嘗聞梁太祖好食雞,慕之,乃日烹五十雞以供膳。葬殷上潢,希聲不哭泣,頓食雞肉數器而起。其禮部侍郎潘起譏之曰:「昔阮籍居喪而食蒸豚,世豈乏賢邪!」長興三年,希聲卒,追封衡
 陽王。弟希範立。



 希範字寶規,殷第四子也。殷子十餘人,嫡子希振長而賢,其次希聲與希範同日生,而希聲母袁夫人有美色,希聲以母寵得立,而希振棄官為道士,居于家。希聲卒,而希範以次立,襲殷官爵,封楚王。清泰二年,賜以弓矢冠劍。天福四年,加希範天冊上將軍,開府承制如殷故事。



 希範好學,善詩,文士廖光圖、徐仲雅、李皋、拓拔常等十八人皆故殷時學士,希範性奢侈,光圖等皆薄徒,飲博讙呼,獨常沉厚長者,上書切諫,光圖等惡之。



 襄州安從進、安州李金全叛,晉高祖詔希範出兵。希範遣張
 少敵以舟兵趨漢陽,漕米五萬斛以饋軍,金全等敗,少敵乃旋。



 溪州刺史彭士愁率錦、獎諸蠻攻澧州,希範遣劉勍、劉全明等以步卒五千擊之,士愁大敗。勍等攻溪州,士愁走獎州,遣其子師暠率諸蠻酋降于勍。溪州西接牂柯、兩林,南通桂林、象郡,希範乃立銅柱以為表,命學士李皋銘之。於是,南寧州酋長莫彥殊率其本部十八州、都雲酋長尹懷昌率其昆明等十二部、牂柯張萬濬率其夷、播等七州皆附於希範。



 希範作會春園、嘉宴堂,其費巨萬,始加賦於國中,拓拔常切諫以為不可。希範又作九龍殿,以八龍繞柱,自言身一龍也。是時,契丹滅
 晉,中國大亂,希範牙將丁思覲廷諫希範曰:「先王起卒伍,以攻戰而得此州,倚朝廷以制鄰敵,傳國三世,有地數千里,養兵十萬人。今天子囚辱,中國無主,真霸者立功之時。誠能悉國之兵出荊、襄以趨京師,倡義於天下,此桓文、之業也。奈何耗國用而窮土木,為兒女之樂乎?」希範謝之,思覲瞋目視希範曰:「孺子終不可教也!」乃扼喉而死。開運四年,希範卒,年四十九,謚曰文昭。希廣立。



 希廣字德丕,希範同母弟也。希範平生惡拓拔常諫諍,常入謁,希範呼閽者指常曰:「吾不欲見此人,勿復內也。」乃謝絕之。及臥病,始思常言,以為忠,召之託以希廣。希
 範卒,常數勸希廣以位奉其兄希萼,希廣不從。



 希萼為朗州節度使,希範之卒,希萼自朗州來奔喪。希廣將劉彥瑫謀曰:「武陵之來,其意不善,宜出兵迎之,以備非常,使其解甲釋兵而後入。」張少敵、周廷誨曰:「王能與之則已,不然宜早除之。」希廣泣曰:「吾兄也,焉忍殺之,分國而治可也。」乃以兵迎希萼於砆石,止之於碧湘宮,厚賂以遣之。希萼憤然而去,乃遣使詣京師求封爵,請置邸稱籓。漢隱帝不許,降璽書慰勞講解之。希萼怒,送款於李景,舉兵攻長沙。希廣遣劉彥瑫、許可瓊等御之。



 彥瑫敗希萼於僕射洲。希萼去,誘溪洞諸蠻寇益陽。希廣遣崔
 珙璉以步卒七千屯湘鄉玉潭以遏諸蠻。劉彥瑫以舟兵趨武陵,攻希萼。彥瑫敗於湄洲,希廣大懼,遣使請兵於京師,漢隱帝不能出師。希萼舟兵沿江而上,自號「順天將軍」,攻岳州,刺史王贇堅城不戰,希萼呼贇曰:「吾昔約君同行,今何異心乎?」贇曰:「君王兄弟不相容,而責將吏異心乎?願君王入長沙,不傷同氣,臣不敢不盡節。」



 希萼引兵去,下湘鄉,止長沙,屯水西。劉彥瑫、許可瓊屯水東。



 彭師暠登城望水西軍,入白希廣曰:「武陵兵驕,雜以蠻蜑,其勢易破。請令可瓊等陣山前,臣以步兵三千自巴溪渡江趨岳麓,候夜擊之。」希廣以為可,而可瓊已
 陰送款於希萼,遂沮其議。明日,師暠詣可瓊計事,瞋目叱之曰:「視汝反文在面,豈欲投賊乎!」拂衣而出,急白希廣,請殺之,希廣不聽。希萼攻長樂門,牙將吳宏、楊滌戰於門中,希萼少衄,已而許可瓊奔于希萼,宏、滌聞之皆潰。希廣率妻子匿于慈堂。明日擒之。希萼見之惻然曰:「此鈍夫也,豈能為惡?左右惑之爾。」顧其下曰:「吾欲活之,如何?」其下皆不對,遂縊死之。



 乾祐三年,希萼自立。明年,漢隱帝崩,京師大亂,希萼遂臣於李景,景冊封希萼楚王,希萼悉以軍政事任其弟希崇。希崇與楚舊將徐威、陸孟俊、魯綰等謀作亂。希萼置酒端陽門,希崇辭以
 疾,威等縱惡馬十餘匹,以壯士執楇隨之,突入其府,劫庫兵,縛希萼,迎希崇以立。希崇遣彭師暠、廖偃囚希萼於衡山,師暠奉希萼為衡山王,臣於李景。希崇懼,亦請命於景。景遣邊鎬入楚,盡遷馬氏之族于金陵,時周廣順元年也。封希萼楚王,居洪州;希崇領舒州節度使,居揚州。



 顯德三年,世宗征淮,下揚州,下詔撫安馬氏子孫。已而揚州復入于景,希崇率其兄弟十七人歸京師,拜右羽林統軍,希能左屯衛大將軍,希貫右千牛衛大將軍,希隱、希濬、希知、希朗皆為節度行軍司馬。



 劉言,吉州廬陵人也。王進逵,武陵人也。言,初事刺史彭
 玕,從玕奔楚,言事希範為長州刺史。進逵少為靜江軍卒,事希萼為指揮使。



 希萼攻希廣,以進逵為先鋒,陷長沙。長沙遭亂殘毀,希萼使進逵以靜江兵營緝之,兵皆愁怨,進逵因擁之,夜以長柯巨斧斫關,奔歸武陵。希萼方醉,不能省,明日遣將唐翥追之,及于武陵,翥戰大敗而還。進逵乃逐出留後馬光惠,迎言於辰州以為帥,進逵自為副。已而希萼將徐威等作亂,縛希萼而立希崇,湖南大亂。李景遣邊鎬入楚,遷馬氏于金陵,因并召言。言不從,遣進逵與行軍司馬何景真等攻鎬於長沙,鎬敗走。



 周廣順三年,言奉表京師,以邀封爵。又言長沙殘
 破,不可居,請移治所於武陵。周太祖皆從之,乃升朗州為武平軍,在武安軍上,以言為節度使,因以武安授進逵,進逵自以言己所迎立,不為之下。言患之,二人始有隙,欲相圖。進逵謀曰:「言將可用者不過何景真、朱全琇爾,召而殺之,言可取也。」是時,劉晟取楚梧、桂、宜、蒙等州,進逵因白言召景真等會兵攻晟。言信之,遣景真、全琇往,至皆見殺,乃舉兵襲武陵,執言殺之,奉表京師,周太祖即以進逵為武平軍節度使。



 世宗征淮南,授進逵南面行營都統。進逵攻鄂州,過岳州,岳州刺史潘叔嗣,進逵故時同列,待進逵甚謹。進逵左右就叔嗣求賂,叔嗣
 不與,左右讒其短,進逵面罵之,叔嗣慚恨,語其下曰:「進逵戰勝而還,吾無遺類矣。」進逵入鄂州,方攻下長山,叔嗣以兵襲武陵。進逵聞之,輕舟而歸,與叔嗣戰武陵城外,進逵敗,見殺。



 周行逢,武陵人也。與王進逵俱為靜江軍卒,事希萼為軍校。進逵攻邊鎬,行逢別破益陽,殺李景兵二千餘人,擒其將李建期。進逵為武安軍節度使,拜行逢集州刺史,為進逵行軍司馬。進逵與劉言有隙,行逢為畫謀策襲殺言。進逵據武陵,行逢據潭州。



 顯德元年,拜行逢武清軍節度使,權知潭州軍府事。潘叔嗣殺進逵,或勸
 其入武陵,叔嗣曰:「吾殺進逵,救死而已,武陵非吾利也。」乃還岳州,遣其客將李簡率武陵人迎行逢於潭州。行逢入武陵,或請以潭州與叔嗣,行逢曰:「叔嗣殺主帥,罪當死,以其迎我,未忍殺爾。若與武安,是吾使之殺王公也。」召以為行軍司馬。叔嗣怒,稱疾不至,行逢怒曰:「是又欲殺我矣!」乃陽以武安與之,召使至府受命,至則殺之。



 行逢故武陵農家子,少貧賤無行,多慷慨大言。及居武陵,能儉約自勉勵,而性勇敢,果於殺戮,麾下將吏素恃功驕慢者,一以法繩之。大將十餘人謀為亂,行逢召宴諸將,酒半,以壯士擒下斬之,一境皆畏服。民過無大小
 皆死,夫人嚴氏諫曰:「人情有善惡,安得一概殺之乎!」行逢怒曰:「此外事,婦人何知!」嚴氏不悅,紿曰:「家田佃戶,以公貴,頗不力農,多恃勢以侵民,請往視之。」至則營居以老,歲時衣青裙押佃戶送租入城。行逢往就見之,勞曰:「吾貴矣,夫人何自苦邪!」嚴氏曰:「公思作戶長時乎?民租後時,常苦鞭撲,今貴矣,宜先期以率眾,安得遂忘壟畝間乎!」行逢彊邀之,以群妾擁升肩輿,嚴氏卒無留意,因曰:「公用法太嚴而失人心,所以不欲留者,一旦禍起,田野間易為逃死爾。」行逢為少損。



 建隆三年,行逢病,召其將吏,以其子保權屬之曰:「吾起隴畝為團兵,同時十人,
 皆以誅死,惟衡州刺史張文表獨存,然常怏怏不得行軍司馬。吾死,文表必叛,當以楊師璠討之。如其不能,則嬰城勿戰,自歸於朝廷。」



 行逢卒,子保權立。文表聞之,怒曰:「行逢與我起微賤而立功名,今日安能北面事小兒乎!」遂舉兵叛,攻下潭州。保權乞師於朝廷,亦命楊師璠討文表,告以先人之言,感激涕泣,師璠亦泣,顧其軍曰:「汝見郎君乎?年未成人而賢若此。」



 軍士奮然,皆思自效。師璠至平津亭,文表出戰,大敗之。初,保權之乞師也,太祖皇帝遣慕容延釗討文表,未至而文表為師璠所執。延釗兵入朗州,保權舉族朝于京師,其後事具國史。



 殷
 自唐乾寧三年入湖南,至周廣順元年,凡五十七年,餘具《年譜》注。



\end{pinyinscope}