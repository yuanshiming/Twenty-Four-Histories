\article{卷七十四四夷附錄第三}

\begin{pinyinscope}

 奚,
 本
 匈奴之別種。當唐之末,居陰涼川,在營府之西,幽州之西南,皆數百里。有人馬二萬騎。分為五部:一曰阿薈部,二曰啜米部,三曰粵質部,四曰奴皆部,五曰黑訖支部。後徙居琵琶川,在幽州東北數百里。地多黑羊,馬嵒前蹄堅善走,其登山逐獸,下上如飛。



 契丹阿保機彊盛,室韋、奚、霫皆服屬之。奚人常為契丹守界上,而苦其苛虐,奚王去諸怨叛,以別部西徙媯州,依北山射獵,常
 採北山麝香、仁參賂劉守光以自托。其族至數千帳,始分為東、西奚。去諸之族,頗知耕種,歲借邊民荒地種穄,秋熟則來獲,窖之山下,人莫知其處。爨以平底瓦鼎,煮穄為粥,以寒水解之而飲。



 去諸卒,子掃剌立。莊宗破劉守光,賜掃剌姓李,更其名紹威。紹威卒,子拽剌立。同光以後,紹威父子數遣使朝貢。初,紹威娶契丹女舍利逐不魯之姊為妻,後逐不魯叛亡入西奚,紹威納之。晉高祖入立,割幽州鴈門以北入于契丹,是時紹威與逐不魯皆已死,耶律德光已立晉北歸,拽剌迎謁馬前,德光曰:「非爾罪也。



 負我者,掃剌與逐不魯爾。」乃發其墓,粉其
 骨而揚之。後德光滅晉,拽剌常以兵從。其後不復見於中國。



 自去諸徙媯州,自別為西奚,而東奚在琵琶川者,亦為契丹所并,不復能自見云。



 吐渾,本號吐谷渾,或曰乞伏乾歸之苗裔。自後魏以來,名見中國,居於青海之上。當唐至德中,為吐蕃所攻,部族分散,其內附者,唐處之河西。其大姓有慕容、拓拔、赫連等族。懿宗時,首領赫連鐸為陰山府都督,與討龐勛,以功拜大同軍節度使。為晉王所破,其部族益微,散處蔚州界中。莊宗時,有首領白承福者,依中山北石門為柵,莊宗為置寧朔、奉化兩府,以承福為都督,賜其姓名
 為李紹魯。



 終唐時,常遣使朝貢中國。



 晉高祖立,割鴈門以北入于契丹,於是吐渾為契丹役屬,而苦其苛暴。是時,安重榮鎮成德,有異志,陰遣人招吐渾入塞,承福等乃自五臺入處中國。契丹耶律德光大怒,遣使者責誚高祖,高祖恐懼,遣供奉官張澄率兵搜索并、鎮、忻、代等州山谷中吐渾驅出之。然晉亦苦契丹,思得吐渾為緩急之用,陰遣劉知遠鎮太原慰撫之。終高祖時,承福數遣使者朝貢。後出帝與契丹絕盟,召承福入朝,拜大同軍節度使,待之甚厚。契丹與晉相距于河,承福以其兵從出帝禦虜。是歲大熱,吐渾多疾死,乃遣承福歸太
 原,居之嵐、石之間。劉知遠稍侵辱之,承福謀復亡出塞,知遠以兵圍其族,殺承福及其大姓赫連海龍、白可久、白鐵匱等,其羊馬貲財巨萬計,皆籍沒之,其餘眾以其別部王義宗主之。吐渾遂微,不復見。



 初,唐以承福之族為熟吐渾。長興中,又有生吐渾杜每兒來朝貢。每兒,不知其國地、部族。至漢乾祐二年,又有吐渾何戛剌來朝貢,不知為生、熟吐渾,蓋皆微,不足考錄。



 達靼,靺鞨之遺種,本在奚、契丹之東北,後為契丹所攻,而部族分散,或屬契丹,或屬渤海,別部散居陰山者,自號達靼。當唐末,以名見中國。有每相溫、于越相溫,咸通
 中,從朱邪赤心討龐勛。其後李國昌、克用父子為赫連鐸等所敗,嘗亡入達靼。後從克用入關破黃巢,由是居雲、代之間。其俗善騎射,畜多駝、馬。



 其君長、部族名字,不可究見,惟其嘗通於中國者可見云。



 同光中,都督折文逋數自河西來貢駝、馬。明宗討王都於定州,都誘契丹入寇,明宗詔達靼入契丹界,以張軍勢,遣宿州刺史薛敬忠以所獲契丹團牌二百五十及弓箭數百賜雲州生界達靼,蓋唐常役屬之。長興三年,首領頡哥率其族四百餘人來附。



 訖于顯德,常來不絕。



 黨項,西羌之遺種。其國在《禹貢》析支之地,東至松州,西
 接葉護,南界春桑,北鄰吐渾,有地三千餘里。無城邑而有室屋,以毛罽覆之。其人喜盜竊而多壽,往往百五六十歲。其大姓有細封氏、費聽氏、折氏、野利氏,拓拔氏為最彊。唐德宗時,黨項諸部相率內附,居慶州者號東山部落,居夏州者號平夏部落。部有大姓而無君長,不相統一,散處邠寧、鄜延、靈武、河西,東至麟、府之間。自同光以後,大姓之彊者各自來朝貢。



 明宗時,詔沿邊置場市馬,諸夷皆入市中國,而回鶻、黨項馬最多。明宗招懷遠人,馬來無駑壯皆售,而所售常過直,往來館給,道路倍費。其每至京師,明宗為御殿見之,勞以酒食,既醉,連
 袂歌呼,道其土風以為樂,去又厚以賜賚,歲耗百萬計。唐大臣皆患之,數以為言。乃詔吏就邊場售馬給直,止其來朝,而黨項利其所得,來不可止。其在靈、慶之間者,數犯邊為盜。自河西回鶻朝貢中國,道其部落,輒邀劫之,執其使者,賣之他族,以易牛馬。明宗遣靈武康福、邠州藥彥稠等出兵討之。福等擊破阿埋韋悉褒勒彊賴埋廝骨尾及其大首領連香李八薩王、都統悉那埋摩、侍御乞埋嵬悉逋等族,殺數千人,獲其牛羊巨萬計,及其所劫外國寶玉等,悉以賜軍士。由是黨項之患稍息。



 至周太祖時,府州黨項尼也六泥香王子、拓拔山等皆
 來朝貢。廣順三年,慶州刺史郭彥欽貪其羊馬,侵擾諸部,獨野雞族彊不可近,乃誣其族犯邊。太祖遣使招慰之。野雞族苦彥欽,不肯聽命,太祖遣邠州折從阮、寧州刺史張建武等討之。建武勇於立功,不能通夷情,馳軍擊野雞族,殺數百人。而喜玉、折思、殺牛三族聞建武擊破野雞族,各以牛酒犒軍,軍士利其物,反劫掠之。三族共誘建武軍至包山,度險,三族共擊之,軍投崖谷,死傷甚眾。太祖怒,罪建武等,選良吏為慶州刺史以招撫之。



 其他諸族,散處尚邊界上者甚眾,然其無國地、君長,故莫得而紀次云。



 突厥,國地、君世、部族、名號、物俗,見於唐著矣。至唐之末。為諸夷所侵,部族微散。五代之際,嘗來朝貢。同光三年,渾解樓來。天成二年,首領張慕晉來。



 長興二年,首領杜阿熟來。天福六年,遣使者薛同海等來。凡四至,其後不復來。



 然突厥於時最微,又來不數,故其君長史皆失不能紀。



 吐蕃,國地、君世、部族、名號、物俗,見於唐著矣。當唐之盛時,河西、隴右三十三州,涼州最大,土沃物繁而人富樂。其地宜馬,唐置八監,牧馬三十萬匹。



 以安西都護府羈縻西域三十六國。唐之軍、鎮、監、務,三百餘城,常以中國
 兵更戍,而涼州置使節度之。安祿山之亂,肅宗起靈武,悉召河西兵赴難,而吐蕃乘虛攻陷河西、隴右,華人百萬皆陷于虜。文宗時,嘗遣使者至西域,見甘、涼、瓜、沙等州城邑如故,而陷虜之人見唐使者,夾道迎呼,涕泣曰:「皇帝猶念陷蕃人民否?」其人皆天寶時陷虜者子孫,其語言稍變,而衣服猶不改。



 至五代時,吐蕃已微弱,回鶻、黨項諸羌夷分侵其地,而不有其人民。值中國衰亂,不能撫有,惟甘、涼、瓜、沙四州常自通於中國。甘州為回鶻牙,而涼、瓜、沙三州將吏,猶稱唐官,數來請命。自梁太祖時,嘗以靈武節度使兼領河西節度,而觀察甘、肅、威等
 州。然雖有其名,而涼州自立守將。唐長興四年,涼州留後孫超遣大將拓拔承謙及僧、道士、耆老楊通信等至京師求旌節,明宗問孫超等世家,承謙曰:「吐蕃陷涼州,張掖人張義朝募兵擊走吐蕃,唐因以義朝為節度使,發鄆州兵二千五百人戍之。唐亡,天下亂,涼州以東為突厥、黨項所隔,鄆兵遂留不得返。今涼州漢人,皆其戍人子孫也。」明宗乃拜孫超節度使。清泰元年,留後李文謙來請命。後數年,涼州人逐出文謙,靈武馮暉遣牙將吳繼勳代文謙為留後,是時天福七年。明年,晉高祖遣涇州押牙陳延暉齎詔書安撫涼州,涼州人共劫留延
 暉,立以為刺史。至漢隱帝時,涼州留後折逋嘉施來請命,漢即以為節度使。嘉施,土豪也。周廣順二年,嘉施遣人市馬京師,因來請命帥。是時,樞密使王峻用事。峻故人申師厚者,少起盜賊,為兗州牙將,與峻相友善,後峻貴,師厚敝衣蓬首,日候峻出,拜馬前,訴以飢寒,峻未有以發。而嘉施等來請帥,峻即建言:「涼州深入夷狄,中國未嘗命吏,請募率府率、供奉官能往者。」月餘,無應募者,乃奏起師厚為左衛將軍,已而拜河西節度使。師厚至涼州,奏薦押衙副使崔虎心、陽妃谷首領沈念般等及中國留人子孫王廷翰、溫崇樂、劉少英為將吏。又自安
 國鎮至涼州,立三州以控扼諸羌,用其酋豪為刺史。然涼州夷夏雜處,師厚小人,不能撫有。至世宗時,師厚留其子而逃歸,涼州遂絕於中國。獨瓜、沙二州,終五代常來。沙州,梁開平中有節度使張奉,自號「金山白衣天子」。至唐莊宗時,回鶻來朝,沙州留後曹義金亦遣使附回鶻以來,莊宗拜義金為歸義軍節度使、瓜沙等州觀察處置等使。



 晉天福五年,義金卒,子元德立。至七年,沙州曹元忠、瓜州曹元深皆遣使來。周世宗時,又以元忠為歸義軍節度使,元恭為瓜州團練使。其所貢:硇砂、羚羊角、波斯錦、安西白灊、金星礬、胡桐律、大鵬砂、毦褐、玉團。
 皆因其來者以名見,而其卒立、世次,史皆失其紀。



 而吐蕃不見於梁世。唐天成三年,回鶻王仁喻來朝,吐蕃亦遣使附以來,自此數至中國。明宗嘗御端明殿見其使者,問其牙帳所居,曰:「西去涇州二千里。」



 明宗賜以虎皮,人一張,皆披以拜,委身宛轉,落其氈帽,亂髮如蓬,明宗及左右皆大笑。至漢隱帝時猶來朝,後遂不復至,史亦失其君世云。



 回鶻,為唐患尤甚。其國地、君世、物俗,見於唐著矣。唐嘗以女妻之,故其世以中國為舅。其國本在娑陵水上、後為黠戛斯所侵,徙天德、振武之間,又為石雄、張仲武所
 破,其餘眾西徙,役屬吐蕃。是時吐蕃已陷河西、隴右,乃以回鶻散處之。



 當五代之際,有居甘州、西州者嘗見中國,而甘州回鶻數至,猶呼中國為舅,中國答以詔書亦呼為甥。梁乾化元年,遣都督周易言等來,而史不見其君長名號,梁拜易言等官爵,遣左監門衛上將軍楊沼押領還蕃。至唐莊宗時,王仁美遣使者來,貢玉、馬,自稱「權知可汗」,莊宗遣司農卿鄭續持節冊仁美為英義可汗。是歲,仁美卒,其弟狄銀立,遣都督安千想等來。同光四年,狄銀卒,阿咄欲立。天成二年,權知國事王仁裕遣李阿山等來朝,明宗遣使者冊仁裕為順化可汗。晉高
 祖時又冊為奉化可汗。阿咄欲,不知其為狄銀親疏,亦不知其立卒;而仁裕訖五代常來朝貢,史亦失其紀。其地出玉、犛、綠野馬、獨峰駝、白貂鼠、羚羊角、硇砂、膃肭臍、金剛鑽、紅鹽、罽氎、騊駼之革。其地宜白麥、青騑麥、黃麻、葱韭、胡荽,以橐駝耕而種。其可汗常樓居,妻號天公主,其國相號媚祿都督。見可汗,則去帽被髮而入以為禮。婦人總髮為髻,高五六寸,以紅絹囊之;既嫁,則加氈帽。又有別族號龍家,其俗與回紇小異。長興四年,回鶻來獻白鶻一聯,明宗命解緤放之。



 自明宗時,常以馬市中國,其所齎寶玉皆屬縣官,而民犯禁為市者輒罪之。
 周太祖時除其禁,民得與回鶻私市,玉價由此倍賤。顯德中,來獻玉,世宗曰:「玉雖寶而無益。」卻之。



 于闐,國地、君世、物俗見於唐。五代亂世,中國多故,不能撫來四夷。其嘗自通於中國者僅以名見,其君世、終始,皆不可知。而于闐尤遠,去京師萬里外。



 其國西南近葱嶺,與婆羅門為鄰國,而相去猶三千餘里,南接吐蕃,西北至疏勒二千餘里。



 晉天福三年,于闐國王李聖天遣使者馬繼榮來貢紅鹽、鬱金、氂牛尾、玉灊等,晉遣供奉官張匡鄴假鴻臚卿,彰武軍節度判官高居誨為判官,冊聖天為大寶于闐國王。是歲冬十二月,匡鄴等自靈
 州行二歲至于闐,至七年冬乃還。而居誨頗記其往復所見山川諸國,而不能道聖天世次也。



 居誨記曰:「自靈州過黃河,行三十里,始涉沙入黨項界,曰細腰沙、神點沙。



 至三公沙,宿月支都督帳。自此沙行四百餘里,至黑堡沙,沙尤廣,遂登沙嶺。沙嶺,黨項牙也,其酋曰捻崖天子。渡白亭河至涼州,自涼州西行五百里至甘州。甘州,回鶻牙也。其南,山百餘里,漢小月支之故地也,有別族號鹿角山沙陀,云朱耶氏之遺族也。自甘州西,始涉磧。磧無水,載水以行。甘州人教晉使者作馬蹄木澀,木澀四竅,馬蹄亦鑿四竅而綴之,駝蹄則包以氂皮乃可行。
 西北五百里至肅州,渡金河,西百里出天門關,又西百里出玉門關,經吐蕃界。吐蕃男子冠中國帽,婦人辮髮,戴瑟瑟珠,云珠之好者,一珠易一良馬。西至瓜州、沙州,二州多中國人,聞晉使者來,其刺史曹元深等郊迎,問使者天子起居。瓜州南十里鳴沙山,云冬夏殷殷有聲如雷,云《禹貢》流沙也。又東南十里三危山,云三苗之所竄也。其西,渡都鄉河曰陽關。沙州西曰仲雲,其牙帳居胡盧磧。云仲雲者,小月支之遺種也,其人勇而好戰,瓜、沙之人皆憚之。胡盧磧,漢明帝時征匈奴,屯田於吾盧,蓋其地也。地無水而嘗寒多雪,每天暖雪銷,乃得水。匡
 鄴等西行入仲雲界,至大屯城,仲雲遣宰相四人、都督三十七人候晉使者,匡鄴等以詔書慰諭之,皆東向拜。自仲雲界西,始涉兼磧,無水,掘地得濕沙,人置之胸以止渴。又西,渡陷河,伐檉置水中乃渡,不然則陷。又西,至紺州。紺州,于闐所置也,在沙州西南,云去京師九千五百里矣。又行二日至安軍州,遂至于闐。聖天衣冠如中國,其殿皆東向,曰金冊殿,有樓曰七鳳樓。以蒲桃為酒,又有紫酒、青酒,不知其所釀,而味尤美。



 其食,粳沃以蜜,粟沃以酪。其衣布帛。有園圃花木。俗喜鬼神而好佛。聖天居處,嘗以紫衣僧五十人列侍,其年號同慶二十九
 年。其國東南曰銀州、盧州、湄州,其南千三百里曰玉州,雲漢張騫所窮河源出于闐,而山多玉者此山也。」其河源所出,至于闐分為三:東曰白玉河,西曰綠玉河,又西曰烏玉河。三河皆有玉而色異,每歲秋水涸,國王撈玉於河,然後國人得撈玉。



 自靈州渡黃河至于闐,往往見吐蕃族帳,而于闐常與吐蕃相攻劫。匡鄴等至于闐,聖天頗責誚之,以邀誓約。匡鄴等還,聖天又遣都督劉再升獻玉千斤及玉印、降魔杵等。漢乾祐元年,又遣使者王知鐸來。



 高麗,本扶餘人之別種也。其國地、君世見於唐,比他夷
 狄有姓氏,而其官號略可曉其義。當唐之末,其王姓高氏。同光元年,遣使廣評侍郎韓申一、副使春部少卿朴巖來,而其國王姓名,史失不紀。至長興三年,權知國事王建遣使者來,明宗乃拜建玄菟州都督,充大義軍使,封高麗國王。建,高麗大族也。開運二年,建卒,子武立。乾祐四年,武卒,子昭立。王氏三世,終五代常來朝貢,其立也必請命中國,中國常優答之。其地產銅、銀,周世宗時,遣尚書水部員外郎韓彥卿以帛數千匹市銅於高麗以鑄鐵。六年,昭遣使者貢黃銅五萬斤。高麗俗知文字,喜讀書,昭進《別敘孝經》一卷、《越王新義》八卷、《皇靈孝經》
 一卷、《孝經雌圖》一卷。《別敘》,敘孔子所生及弟子事迹;《越王新義》,以「越王」為問目,若今「正義」;《皇靈》,述延年辟穀;《雌圖》,載日食、星變。皆不經之說。



 渤海,本號靺鞨,高麗之別種也。唐高宗滅高麗,徙其人散處中國,置安東都護府於平壤以統治之。武后時,契丹攻北邊,高麗別種大乞乞仲象與靺鞨酋長乞四比羽走遼東,分王高麗故地,武后遣將擊殺乞四比羽,而乞乞仲象亦病死。仲象子祚榮立,因并有比羽之眾,其眾四十萬人,據挹婁,臣于唐。至中宗時,置忽汗州,以祚榮為都督,封渤海郡王,其後世遂號渤海。其貴族姓大
 氏,開平元年,國王大諲撰遣使者來,訖顯德常來朝貢。其國土物產,與高麗同。諲撰世次、立卒,史失其紀。



 新羅,弁韓之遺種也。其國地、君世、物俗見於唐。其大族曰金氏、朴氏,自唐高祖時封金真為樂浪郡王,其後世常為君長。同光元年,新羅國王金朴英遣使者來朝貢。長興四年,權知國事金溥遣使來。朴英、溥世次、卒立,史皆失其紀。自晉已後不復至。



 黑水靺鞨,本號勿吉。當後魏時見中國。其國,東至海,南界高麗,西接突厥,北鄰室韋,蓋肅慎氏之地也。其眾分為數十部,而黑水靺鞨最處其北,尤勁悍,無文字之記。
 其兵,角弓、楛矢。同光二年,黑水兀兒遣使者來,其後常來朝貢,自登州泛海出青州。明年,黑水胡獨鹿亦遣使來。兀兒、胡獨鹿若其兩部酋長,各以使來。而其部族、世次、立卒,史皆失其紀。至長興三年,胡獨鹿卒,子桃李花立,嘗請命中國,後遂不復見云。



 南詔蠻,見於唐。其國在漢故永昌郡之東、姚州之西。僖宗幸蜀,募能使南詔者,得宗室子李龜年及徐虎、虎姪藹,乃以龜年為使,虎為副,藹為判官,使南詔。



 南詔所居曰苴絺城,龜年等不至苴絺,至善闡,得其要約與唐為甥舅。僖宗許以安化公主妻之,南詔大喜,遣人隨龜年
 求公主。已而黃巢敗,收復長安,僖宗東還,乃止。



 同光三年,魏王繼岌及郭崇韜等破蜀,得王衍時所俘南詔蠻數十人,又得徐藹,自言嘗使南詔,乃矯詔還其所俘,遣藹等持金帛招撫南詔,諭以威德,南詔不納。



 至明宗時,巂州山後兩林百蠻都鬼主、右武衛大將軍李卑晚,遣大鬼主傅能何華來朝貢,明宗拜卑晚寧遠將軍,又以大渡河南山前邛州六姓都鬼主懷安郡王勿定摽莎為定遠將軍。明年遣左金吾衛將軍烏昭遠為入蠻國信使,昭遠不能達而還。



 牂牁蠻,在辰州西千五百里,以耕植為生,而無城郭聚
 落,有所攻擊,則相屯聚。刻木為契。其首領姓謝氏,其名見於唐。至天成二年嘗一至,其使者曰清州八郡刺史宋朝化,冠帶如中國,貢草豆蔻二萬箇、朱砂五百兩、蠟二百斤。



 昆明,在黔州西南三千里外,地產羊馬。其人椎髻、跣足、披氈,其首領披虎皮。天成二年,嘗一至,其首領號昆明大鬼主,羅殿王、普露靜王九部落,各遣使者來,使者號若土,附牂牁以來。



 占城,在西南海上。其地方千里,東至海,西至雲南,南鄰真腦,北抵驩州。



 其人,俗與大食同。其乘,象、馬;其食,稻米、
 水兕、山羊。鳥獸之奇,犀、孔雀。自前世未嘗通中國。顯德五年,其國王因德漫遣使者莆訶散來,貢猛火油八十四瓶、薔薇水十五瓶,其表以貝多葉書之,以香木為函。猛火油以灑物,得水則出火。薔薇水,雲得自西域,以灑衣,雖敝而香不滅。



 五代,四夷見中國者,遠不過於闐、占城。史之所紀,其西北頗詳,而東南尤略,蓋其遠而罕至,且不為中國利害云。



\end{pinyinscope}