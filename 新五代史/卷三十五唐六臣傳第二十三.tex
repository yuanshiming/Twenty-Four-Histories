\article{卷三十五唐六臣傳第二十三}

\begin{pinyinscope}

 甚哉,白馬之禍,悲夫,可為流涕者矣!然士之生死,豈其一身之事哉?初,唐天祐三年,梁王欲以嬖吏張廷範為太常卿,唐宰相裴樞以謂太常卿唐常以清流為之,廷範乃梁客將,不可。梁王由此大怒,曰:「吾常語裴樞純厚不陷浮薄,今亦為此邪!」是歲四月,彗出西北,掃文昌、軒轅、天市,宰相柳璨希梁王旨,歸其譴於大臣,於是左僕射裴樞、獨孤損、右僕射崔遠、守太保致仕趙崇、兵部侍
 郎王贊、工部尚書王溥、吏部尚書陸扆皆以無罪貶,同日賜死于白馬驛。凡搢紳之士與唐而不與梁者,皆誣以朋黨,坐貶死者數百人,而朝廷為之一空。



 明年三月,唐哀帝遜位于梁,遣中書侍郎、同中書門下平章事張文蔚為冊禮使,禮部尚書蘇循為副;中書侍郎、同中書門下平章事楊涉為押傳國寶使,翰林學士、中書舍人張策為副;御史大夫薛貽矩為押金寶使,尚書左丞趙光逢為副。四月甲子,文蔚等自上源驛奉冊寶,乘輅車,導以金吾仗衛、太常鹵簿,朝梁于金祥殿。王袞冕南面,臣文蔚、臣循奉冊升殿,進讀已,臣涉、臣策奉傳國璽,
 臣貽矩、臣光逢奉金寶,以次升,進讀已,降,率文武百官北面舞蹈再拜賀。



 夫一太常卿與社稷孰為重?使樞等不死,尚惜一卿,其肯以國與人乎?雖樞等之力未必能存唐,然必不亡唐而獨存也。嗚呼!唐之亡也,賢人君子既與之共盡,其餘在者皆庸懦不肖、傾險獪猾、趨利賣國之徒也。不然,安能蒙恥忍辱於梁庭如此哉!作《唐六臣傳》。



 張文蔚張文蔚,字右華,河間人也。初以文行知名,舉進士及第。唐昭宗時,為翰林學士承旨。是時,天子微弱,制度已隳,文蔚居翰林,制詔四方,獨守大體。昭宗遷洛,拜中書侍
 郎、同中書門下平章事。柳璨殺裴樞等七人,蔓引朝士,輒加誅殺,縉紳相視以目,皆不自保,文蔚力講解之,朝士多賴以全活。梁太祖立,仍以文蔚為相,梁初制度皆文蔚所裁定。文蔚居家亦孝悌。開平二年,太祖北巡,留文蔚西都,以暴疾卒,贈右僕射。



 楊涉楊涉,祖收,唐懿宗時宰相;父嚴,官至兵部侍郎。涉舉進士,昭宗時為吏部尚書。哀帝即位,拜中書侍郎、同中書門下平章事。涉,唐名家,世守禮法,而性特謹厚,不幸遭唐之亂,拜相之日,與家人相對泣下,顧謂其子凝式曰:「吾不能脫此網羅,禍將至矣,必累爾等。」唐亡,事梁為門
 下侍郎、同中書門下平章事,在位三年,俯首無所施為,罷為左僕射,知貢舉,後數年卒。



 子凝式,有文詞,善筆札,歷事梁、唐、晉、漢、周,常以心疾致仕,居于洛陽,官至太子太保。



 張策張策,字少逸,河西敦煌人也。父同,為唐容管經略使。策少聰悟好學,通章句。父同,居洛陽敦化里,浚井得古鼎,銘曰:「魏黃初元年春二月,匠吉千。」



 同以為奇,策時年十三,居同側,啟曰:「漢建安二十五年,曹公薨,改元延康。



 是歲十月,文帝受禪,又改黃初,是黃初元年無二月也,銘何謬邪?」同大驚異之。



 策少好浮圖之說,乃落髮為僧,居
 長安慈恩寺。黃巢犯長安,策乃返初服,奉父母以避亂,居田里十餘年。召拜廣文館博士。邠州王行瑜辟觀察支使。晉王李克用攻行瑜,策與婢肩輿其母東歸,行積雪中,行者憐之。梁太祖兼四鎮,辟鄭、滑支使,以母喪解職。服除,入唐為膳部員外郎。華州韓建辟判官,建徙許州,以為掌書記,建遣策聘于太祖,太祖見而喜曰:「張夫子至矣。」遂留以為掌書記,薦之于朝,累拜中書舍人、翰林學士。太祖即位,遷工部侍郎奉旨。開平二年,拜刑部侍郎、同中書門下平章事,遷中書侍郎。以風恙罷為刑部尚書,致仕,卒于洛陽。



 趙光逢趙光逢,字延吉,父隱,唐左僕射。光逢在唐以文行知名,時人稱其方直溫潤,謂之「玉界尺。」昭宗時為翰林學士承旨、御史中丞,以世亂棄官,居洛陽,杜門絕人事者五六年。柳璨為相,與光逢有舊恩,起光逢為吏部侍郎、太常卿。唐亡,事梁為中書侍郎、同中書門下平章事,累遷左僕射,以太子太保致仕。末帝即位,起為司空、同中書門下平章事,復以司徒致仕。唐天成中,即其家拜太保,封齊國公,卒,贈太傅。



 薛貽矩薛貽矩,字熙用,河東聞喜人也,仕唐為兵部侍郎、翰林學士承旨。昭宗自岐還長安,大誅宦者,貽矩時為中尉
 韓全誨等作畫像贊,坐左遷。貽矩乃自結於梁太祖,太祖言之於朝,拜吏部尚書,遷御史大夫。天祐三年,太祖自長蘆還軍,哀帝遣貽矩來勞,貽矩以臣禮見,太祖揖之升階,貽矩曰:「殿下功德及人,三靈改卜,皇帝方行舜、禹之事,臣安敢違?」乃稱臣拜舞,太祖側身以避之。貽矩還,遂趣哀帝遜位。太祖即位,拜貽矩中書侍郎、同中書門下平章事,累拜司空。貽矩為梁相五年,卒,贈侍中。



 蘇循杜曉附蘇循,不知何許人也。為人巧佞,阿諛無廉恥,惟利是趨。事唐為禮部尚書。



 是時,梁太祖已弒昭宗,立哀帝,唐之舊臣皆憤惋切齒,或俯首畏禍,或去不仕,而循特附會
 梁以希進用。梁兵攻楊行密,大敗于珝河,太祖躁忿,急於禪代,欲邀唐九錫,群臣莫敢當其議,獨循倡言:「梁王功德,天命所歸,宜即受禪。」明年,梁太祖即位,循為冊禮副使。



 循有子楷,乾寧中舉進士及第,昭宗遣學士陸扆覆落之,楷常慚恨。及昭宗遇弒,唐政出於梁,楷為起居郎,與柳璨、張廷範等相結,因謂廷範曰:「夫謚者,所以易名而貴信也。前有司謚先帝曰『昭』,名實不稱,公為太常卿,予史官也,不可以不言。」乃上疏駁議。而廷範本梁客將,嘗求太常卿不得者,廷範亦以此怨唐,因下楷疏廷範,廷範議曰:「臣聞執事堅固之謂恭,亂而不損之謂靈,
 武而不遂之謂莊,在國逢難之謂閔,因事有功之謂襄,請改謚昭宗皇帝曰恭靈莊閔皇帝,廟號襄宗。」



 梁太祖已即位,置酒玄德殿,顧群臣自陳德薄不足以當天命,皆諸公推戴之力。



 唐之舊臣楊涉、張文蔚等皆慚懼俯伏不能對,獨循與張禕、薛貽矩盛稱梁王功德,所以順天應人者。循父子皆自以附會梁得所託,旦夕引首,希見進用,敬翔尤惡之,謂太祖曰:「梁室新造,宜得端士以厚風俗,循父子皆無行,不可立於新朝。」於是父子皆勒歸田里,乃依朱友謙於河中。其後,友謙叛梁降晉,晉王將即帝位,求唐故臣在者,以備百官之闕,友謙遣循至魏
 州。是時梁未滅,晉諸將相多不欲晉王即位。晉王之意雖銳,將相大臣未有贊成其議者。循始至魏州,望州廨聽事即拜,謂之「拜殿」。及入謁,蹈舞呼萬歲而稱臣,晉王大悅。明日又獻「畫日筆」三十管,晉王益喜,因以循為節度副使。已而病卒。莊宗即位,贈左僕射。



 楷,同光中為尚書員外郎。明宗即位,大臣欲理其駮謚之罪,以憂死。



 當唐之亡也,又有杜曉者,字明遠。祖審權,父讓能,皆為唐相。昭宗時,王行瑜、李茂貞兵犯京師,昭宗殺讓能於臨皋以自解。曉以父死無罪,居喪哀毀;服除,布衣幅巾,自廢十餘年。崔胤判鹽鐵,辟巡官,除畿縣尉,直昭文館,
 皆不起。



 崔遠判戶部,又辟巡官,或謂曉曰:「嵇康死,子紹自廢不出仕,山濤以物理責之,乃仕。吾子忍令杜氏歲時鋪席祭其先人同匹庶乎?」曉乃為之起。累遷膳部郎中、翰林學士。梁太祖即位,遷工部侍郎奉旨。開平二年,拜中書侍郎、同中書門下平章事。友珪立,遷禮部尚書、集賢殿大學士。袁象先等討賊,兵大掠,曉為亂兵所殺,贈右僕射。



 嗚呼!始為朋黨之論者誰歟?甚乎作俑者也,真可謂不仁之人哉!予嘗至繁城,讀《魏受禪碑》,見漢之群臣稱魏功德,而大書深刻,自列其姓名,以夸耀于世。



 又讀《梁實
 錄》,見文蔚等所為如此,未嘗不為之流涕也。夫以國予人而自夸耀,及遂相之,此非小人,孰能為也?漢、唐之末,舉其朝皆小人也,而其君子者何在哉!當漢之亡也,先以朋黨禁錮天下賢人君子,而立其朝者,皆小人也,然後漢從而亡。及唐之亡也,又先以朋黨盡殺朝廷之士,而其餘存者,皆庸懦不肖傾險之人也,然後唐從而亡。夫欲空人之國而去其君子者,必進朋黨之說;欲孤人主之勢而蔽其耳目者,必進朋黨之說;欲奪國而與人者,必進朋黨之說。夫為君子者,故嘗寡過,小人欲加之罪,則有可誣者,有不可誣者,不能遍及也。至欲舉天下
 之善,求其類而盡去之,惟指以為朋黨耳。故其親戚故舊,謂之朋黨可也;交游執友,謂之朋黨可也;宦學相同,謂之朋黨可也;門生故吏,謂之朋黨可也。是數者,皆其類也,皆善人也。故曰:欲空人之國而去其君子者,惟以朋黨罪之,則無免者矣。



 夫善善之相樂,以其類同,此自然之理也。故聞善者必相稱譽,稱譽則謂之朋黨,得善者必相薦引,薦引則謂之朋黨,使人聞善不敢稱譽,人主之耳不聞有善于下矣,見善不敢薦引,則人主之目不得見善人矣。善人日遠,而小人日進,則為人主者,倀倀然誰與之圖治安之計哉?故曰:欲孤人主之勢而蔽
 其耳目者,必用朋黨之說也。



 一君子存,群小人雖眾,必有所忌,而有所不敢為,惟空國而無君子,然後小人得肆志於無所不為,則漢魏、唐梁之際是也。故曰:可奪國而予人者,由其國無君子,空國而無君子,由以朋黨而去之也。嗚呼,朋黨之說,人主可不察哉!《傳》曰「一言可以喪邦」者,其是之謂與!可不鑒哉!可不戒哉!



\end{pinyinscope}