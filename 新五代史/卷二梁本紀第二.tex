\article{卷二梁本紀第二}

\begin{pinyinscope}

 開平元年春正月壬寅,天子使御史大夫薛貽矩來勞軍。宰相張文蔚率百官來勸進。夏四月壬戌,更名晃。甲子,皇帝即位。戊辰,大赦,改元,國號梁。封唐主為濟陰王。升汴州為開封府,建為東都,以唐東都為西都。廢京兆府為雍州。賜東都酺一日。契丹阿保機使袍
 笏梅老來。



 五月丁丑朔,以唐相張文蔚楊涉為門下侍郎,御史大夫薛貽矩為中書侍郎;同中書門下平章事。戊寅,渤海、契丹遣使者來。乙酉,封兄全昱為廣王,子友文博王,友珪郢王,友璋福王,友貞均王,友徽建王,姪友諒衡王,友能惠王,友誨邵王。甲午,改樞密院為崇政院,太府卿敬翔為使。是月,潞州行營都指揮使李思安及晉人戰,敗績。六月甲寅,平盧軍節度使韓建守司徒,同中書門下
 平章事。秋七月己亥,追尊祖考為皇帝,妣為皇后;皇高祖黯謚曰宣元,廟號肅祖,祖妣范氏謚曰宣僖;曾祖茂琳謚曰光獻,廟號敬祖,祖妣楊氏謚曰光孝;祖信謚曰昭武,廟號憲祖,祖妣劉氏謚曰昭懿;考誠謚曰文穆,廟號烈祖,妣王氏謚曰文惠。八月丁卯,同州虸蚄蟲生。隰州黃河清。九月,括馬。冬十月己未,講武于繁臺。十一月壬寅,赦亡命背軍髡黥刑徒。



 二年春正月丁酉,渤海遣使者來。己亥,卜郊於西都。弒濟陰王。二月辛未,契丹阿保
 機遣使者來。三月壬申朔,如西都。丙子,如懷州。丁丑,如澤州。戊寅,封鴻臚卿李崧介國公,為二王後。壬午,匡國軍節度使劉知俊為潞州行營招討使。



 癸巳,改卜郊。張文蔚薨。夏四月癸卯,楊涉罷,吏部侍郎於兢為中書侍郎,翰林學士承旨禮部侍郎張策為刑部侍郎:同中書門下平章事。壬子,至澤州。五月己丑,潞州行營都虞候康懷英及晉人戰于夾城,敗績。戊戌,立唐三廟。契丹遣使者來。



 六月壬寅,忠武軍節度使
 劉知俊為西路行營招討使,以伐岐。己酉,殺右金吾衛上將軍王師範,滅其族。丙辰,劉知俊及岐人戰于漠谷,敗之。秋九月丁丑,如陜州,博王友文留守東都。冬十月丁未,至自陜州。十一月癸巳,張策罷,左僕射楊涉同中書門下平章事。十二月己亥,以介國公為三恪,酅國公、萊國公為二王後。



 三年春正月甲戌,如西都。復然燈以祈福。庚寅,享于太廟。辛卯,有事于南郊,大赦。丙申,群臣上尊
 號,曰睿文聖武廣孝皇帝。二月壬戌,講武于西杏園。



 甲子,延州高萬興叛于岐,來降。三月辛未,渤海國王大諲譔遣使者來。甲戌,如河中。山南東道節度使楊師厚為潞州四面行營招討使。劉知俊取丹州。夏四月丙午,知俊克延、鄜、坊三州。五月己卯,至自河中,殺佑國軍節度使王重師。六月庚戌,劉知俊執祐國軍節度使劉捍,叛附于岐。辛亥,如陜州。乙卯,冀王朱友謙為同州東面行營招討使。劉
 知俊奔于岐。丹州軍亂,逐其刺史宋知誨。秋七月,商州軍亂,逐其刺史李稠,稠奔于岐。乙丑,克丹州,執其首惡王行思。乙亥,至自陜州。甲申,襄州軍亂,殺其留後王班。房州刺史楊虔叛附于蜀。八月辛亥,降死罪囚。辛酉,均州刺史張敬方克房州,執楊虔。閏月癸酉,契丹遣使者來。己卯,閱稼于西苑。九月壬寅,行營招討使左衛上將軍陳暉克襄州,執其首惡李洪。丁未,保義軍節度使王檀為潞州東面
 行營招討使。辛亥,韓建、楊涉罷。太常卿趙光逢為中書侍郎、翰林學士承旨工部侍郎杜曉為戶部侍郎:同中書門下平章事。辛酉,李洪、楊虔伏誅。冬十一月甲午,日南至,告謝于南郊。己酉,搜訪賢良。鎮國軍節度使康懷英伐岐。十二月,懷英克寧、慶、衍三州。及劉知俊戰于升平,敗績。



 四年春正月壬辰朔,始用樂。丁未,講武于榆林。二月己丑,閱稼于穀水。秋八月丙寅,如陜州。河南尹張宗奭留守西都。辛未,護國軍
 節度使楊師厚為西路行營招討使以伐岐。九月己丑,至自陜州。辛亥,搜訪賢良。冬十一月己丑,寧國軍節度使王景仁為北面行營招討使以伐趙。趙王王熔、北平王王處直叛附于晉,晉人救趙。十二月癸酉,頒律令格式。



 乾化元年春正月丁亥,王景仁及晉人戰于柏鄉,敗績。庚寅,赦流罪以下囚,求危言正諫。癸巳,天雄軍節度使楊師厚為北面行營招討使。夏四月壬申,契丹阿保機遣使者來。五月甲申朔,大赦,改元。癸巳,幸張宗奭第。秋八月戊辰,閱稼于榆林。渤海遣使者來。戊寅,大閱于興
 安鞠場。九月辛巳朔,御文明殿,入閣。



 庚子,如魏州。張宗奭留守西都。冬十月丙子,大閱于魏東郊。十一月,高萬興取鹽州。壬辰,至自魏州。乙未,回鶻、吐蕃遣使者來。二年春二月丁巳,光祿卿盧玭使于蜀。甲子,如魏州,張宗奭留守西都。次白馬,殺左散騎常侍孫騭、右諫議大夫張衍、兵部郎中張俊。戊寅,如貝州。三月丙戌,屠棗彊。丁未,復如魏州。



 夏四月己巳,至自魏州。戊寅,如西都。五月丁亥,德音
 降死罪已下囚。罷役徒,禁屠及捕生。渤海遣使者來。是月,薛貽矩薨。六月,疾革,郢王友珪反。戊寅,皇帝崩。



 嗚呼,天下之惡梁久矣!自後唐以來,皆以為偽也。至予論次五代,獨不偽梁,而議者或譏予大失《春秋》之旨,以謂「梁負大惡,當加誅絕,而反進之,是獎篡也,非《春秋》之志也。」予應之曰:「是《春秋》之志爾。魯桓公弒隱公而自立者,宣公弒子赤而自立者,鄭厲公逐世子忽而自立者,
 衛公孫剽逐其君衎而自立者,聖人於《春秋》,皆不絕其為君。此予所以不偽梁者,用《春秋》之法也。」「然則《春秋》亦獎篡乎?」曰:「惟不絕四者之為君,於此見《春秋》之意也。聖人之於《春秋》,用意深,故能勸戒切,為言信,然後善惡明。夫欲著其罪於後世,在乎不沒其實。其實嘗為君矣,書其為君。其實篡也,書其篡。各傳其實,而使後世信之,則四君之罪,不可得而掩爾。使為君者不得掩其惡,然後人知惡名不可逃,則為惡者庶乎其息矣。是謂用意深而勸戒切,為言信而善惡明也。桀、紂,不待貶其王,而萬世所共惡者也。《春秋》於大惡之君不誅絕之者,不害其
 褒善貶惡之旨也,惟不沒其實以著其罪,而信乎後世,與其為君而不得掩其惡,以息人之為惡。



 能知《春秋》之此意,然後知予不偽梁之旨也。」



\end{pinyinscope}