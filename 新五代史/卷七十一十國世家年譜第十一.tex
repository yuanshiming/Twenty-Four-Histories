\article{卷七十一十國世家年譜第十一}

\begin{pinyinscope}

 嗚呼,
 堯、舜盛矣!三代之王,功有餘而德不足,故皆更始以自新,由是改正朔矣,至於後世,遂名年以建元。及僭竊交興,而稱號紛雜,則不可以不別也。五代十國,稱帝改元者七。吳越、荊、楚,常行中國年號。然予聞於故老,謂吳越亦嘗稱帝改元,而求其事迹不可得,頗疑吳越後自諱之。及旁采閩、楚、南漢諸國之書,與吳越往來者多矣,皆無稱帝之事。獨得其封落星石為寶石山制書,稱
 寶正六年辛卯,則知其嘗改元矣。辛卯,長興二年,乃鏐之末世也,然不見其終始所因,故不得而備列。錢氏訖五代,嘗外尊中國,豈其張軌之比乎。十國皆非中國有也,其稱帝改元與不,未足較其得失,故並列之。作《十國世家年譜》。



 以下表略或問:十國固非中國有也,然猶命以封爵,而稱中國年號來朝貢者,亦有之矣,本紀之不書,何也?曰:封爵之不書,所以見其非中國有也。其朝貢之來如夷狄,以夷狄書之則甚矣。問者曰:四夷、十國,皆非中國之有也,四夷之封爵朝貢則書,而十國之不書何也?曰:以中國而視夷狄,夷狄之可也。以五代之君而視十國,夷狄之則未可
 也。故十國之封爵、朝貢,不如夷狄,則無以書之。書如夷狄,則五代之君未可以夷狄之也。是以外而不書,見其自絕於中國焉爾。問者曰:外而不書,則東漢之立何以書?曰:吾於東漢,常異其辭於九國也。《春秋》因亂世而立治法,本紀以治法而正亂君。世亂則疑難之事多,正疑處難,敢不慎也。周、漢之事,可謂難矣哉!或謂:劉旻嘗致書于周,求其子贇不得而後自立,然則旻之志不以忘漢為仇,而以失子為仇也。曰:漢嘗詔立贇為嗣,則贇為漢之國君,不獨為旻子也。



 旻之大義,宜不為周屈,其立雖未必是,而義當不屈於周,此其可以異乎九國矣。



 終
 旻之世,猶稱乾祐,至承鈞立,然後改元,則旻之志豈不可哀也哉!



 十國年世,惟楚、閩、東漢三國,諸家之說不同,而互有得失,最難考正。今略其諸說而正其是者,庶幾博覽者不惑,而一以《年譜》為正也。馬氏,據《湖湘故事》、《九國志》、《運歷圖》,並云殷以長興元年卒,是歲,子希聲立,長興三年卒。而《五代舊史》殷列傳云,殷長興二年卒,享年七十八,子希聲立,不周歲而卒;明宗本紀長興元年,書希聲除節度使,起復,三年八月,又書希聲卒。今據《九國志》,殷以大中六年歲在壬申生,享年七十九。蓋自大中壬申至長興元年庚寅,實七十九年,為得其實。而希聲,據《湖湘故事》、《九國志》、《運歷圖》皆以三年卒,與明宗本紀皆合,不疑。惟《舊史》書殷卒二年,及年七十八,希聲立不周歲卒為繆爾。希萼、希崇之亂,南唐盡遷馬氏之族歸於金陵。《五代舊史》云,時廣順元年也。而《運歷圖》云乾祐二年馬氏滅者,繆也。初,殷入湖南,掘地得石,讖云:「龍起頭,豬掉尾。」蓋殷以乾寧三年歲在丙辰,自立於湖南,至廣順
 元年辛亥而滅。《九國志》以乾祐三年為辛亥,《湖湘故事》以顯德元年為辛亥者,皆繆也。惟《五代舊史》得其實。王氏世次,曰潮、曰審知、曰延翰、曰璘、曰昶、曰曦、曰延政,凡七主。而潮以唐景福元年歲在壬子始入福州,至開運三年丙午而滅,實五十五年。當云七主五十五年,為得其實。而《運歷圖》云五十六年,《九國志》、《五代舊史》、《紀年通譜》、《閩中實錄》、《閩王列傳》皆云七主六十年者,皆繆也。審知,《五代舊史》本傳云,同光元年十二月卒,《九國志》亦云同光元年卒。《運歷圖》同光三年卒。今檢《五代舊史》莊宗本紀,同光二年五月丙午,審知加檢校太師守中書令,豈得卒於元年也?又至四年二月庚子,福建副使王延翰奏稱權知軍府事,三月辛亥,遂除延翰威武軍節度使。以此推之,審知卒當在同光三年十二月,蓋閩去京師遠,明年二月延翰之奏始至京師,理當然也。



 又據《閩王列傳》、《九國志》,皆云審知在位二十九年。審知以唐乾寧四年嗣位,是歲丁巳,至同光三年乙酉,實二十九年。則《運
 歷圖》為是,而《舊史》、《九國志》元年卒者,皆繆也。璘本名延鈞,《五代舊史》本傳云在位十二年。《九國志》云在位十一年。《閩王列傳》、《紀年通譜》皆云在位十年。蓋璘以天成元年殺延翰自立,是歲丙戌,至清泰二年乙未,實十年而卒,與《閩王列傳》合,而《舊史》、《九國志》皆繆也。璘以清泰二年改元永和,是歲見殺,而《舊史》、《九國志》、《運歷圖》皆無永和之號,又《運歷圖》書鏻見殺在天福元年丙申者,皆繆也。劉旻,《九國志》云,乾祐七年十一月旻卒,享年六十,子承鈞立,時年二十九。乾祐七年,乃顯德元年也。而《五代舊史》、《周世宗實錄》、《運歷圖》、《紀年通譜》皆云顯德二年冬旻卒。又有旻偽中書舍人王保衡《晉陽見聞要錄》云,旻乙卯生,卒年六十一,子承鈞立。承鈞丙戌生,立時年二十九。保衡是旻之臣,其親所見聞,所得最實,然而頗為轉寫差誤爾。按保衡書旻乙卯生,若享年六十一,當於乙卯歲卒,則是顯德二年也。又書承鈞丙戌生,立時年二十九,則當是顯德元年甲寅歲也。豈有旻卒於二年,承鈞以元年嗣位?理必不然。以《九國志》參較,旻
 享年六十,顯德元年卒,承鈞以是歲嗣位,時年二十九,為得其實,但《見聞要錄》衍「一」字爾。其雲二年卒者,皆繆也。《九國志》又云,承鈞立,服喪三年,至乾祐九年服除,改十年為天會元年,當是顯德四年。而《紀年通譜》以顯德三年為天會元年者,繆也。晉與梁為敵國,自稱天祐者二十年,故首列
 于《年譜》,其後遂滅梁而為唐,故不列於世家。



\end{pinyinscope}