\article{卷六十四後蜀世家第四}

\begin{pinyinscope}

 孟
 知祥,字保胤,邢州龍岡人也。其叔父遷,當唐之末,據邢、洺、磁三州,為晉所虜。晉王以遷守澤潞,梁兵攻晉,遷以澤潞降梁。知祥父道,獨留事晉而不顯。及知祥壯,晉王以其弟克讓女妻之,以為左教練使。莊宗為晉王,以知祥為中門使。前此為中門使者多以罪誅,知祥懼,求他職,莊宗命知祥薦可代己者,知祥因薦郭崇韜自代,崇韜德之,知祥遷馬步軍都虞候。莊宗建號,以太原為
 北京,以知祥為太原尹、北京留守。



 魏王繼岌代蜀,郭崇韜為招討使,崇韜臨訣,白曰:「即臣等平蜀,陛下擇帥以守西川,無如孟知祥者。」已而唐兵破蜀,莊宗遂以知祥為成都尹、劍南西川節度副大使。知祥馳至京師,莊宗戒有司盛供帳,多出內府珍奇諸物以宴勞之。酒酣,語及平昔,以為笑樂,歎曰:「繼岌前日乳臭兒爾,乃能為吾平定兩川,吾徒老矣,孺子可喜,然益令人悲爾!吾憶先帝棄世時,疆土侵削,僅保一隅,豈知今日奄有天下,九州四海,珍奇異產,充牣吾府!」因指以示知祥,曰:「吾聞蜀土之富,無異於此,以卿親賢,故以相付。」



 同光四年正月
 戊辰,知祥至成都,而崇韜已死。魏王繼岌引軍東歸,先鋒康延孝反,攻破漢州。知祥遣大將李仁罕會任圜、董璋等兵擊破延孝,知祥得其將李肇、侯弘實及其兵數千以歸。而莊宗崩,魏王繼岌死,明宗入立。知祥乃訓練兵甲,陰有王蜀之志。益置義勝、定遠、驍銳、義寧、飛棹等軍七萬餘人,命李仁罕、趙廷隱、張業等分將之。



 初,魏王之班師也,知祥率成都富人及王氏故臣家,得錢六百萬緡以犒軍,其餘者猶二百萬。任圜自蜀入為相,兼判三司,素知蜀所餘錢。是冬,知祥拜侍中,乃以太僕卿趙季良齎官告賜之,因以為三川制置使,督蜀犒軍餘
 錢送京師,且制置兩川征賦。知祥怒,不奉詔。然知祥與季良有舊,遂留之。



 樞密使安重誨頗疑知祥有異志,思有以制之。初,知祥鎮蜀,莊宗以宦者焦彥賓為監軍,明宗入立,悉誅宦者,罷諸道監軍。彥賓已罷,重誨復以客省使李嚴為監軍。嚴前使蜀,既歸而獻策伐蜀,蜀人皆惡之,而知祥亦怒曰:「焦彥賓以例罷,而諸道皆廢監軍,獨吾軍置之,是嚴欲以蜀再為功也。」掌書記母昭裔及諸將吏皆請止嚴而無內,知祥曰:「吾將有以待其來!」嚴至境上,遣人持書候知祥,知祥盛兵見之,冀嚴懼而不來,嚴聞之自若。天成二年正月,嚴至成都,知祥置酒召嚴。



 是時,焦彥賓雖罷,猶在蜀,嚴於懷中出詔示知祥以誅彥賓,知祥不聽,因責嚴曰:「今諸方鎮已罷監軍,公何得來此?」目客將王彥銖執嚴下,斬之。明宗不能詰。



 初,知祥鎮蜀,遣人迎其家屬于太原,行至鳳翔,鳳翔節度使李從嚴聞知祥殺李嚴,以為知祥反矣,遂留之。明宗既不能詰,而欲以恩信懷之,乃遣客省使李仁矩慰諭知祥,并送瓊華公主及其子昶等歸之。



 知祥因請趙季良為節度副使,事無大小,皆與參決。三年,唐徙季良為果州團練使,以何瓚為節度副使。知祥得制書匿之,表留季良,不許。乃遣其將雷廷魯至京師論請,明宗不得已
 而從之。是時,瓚行至綿谷,懼不敢進,知祥乃奏瓚為行軍司馬。



 是歲,唐師伐荊南,詔知祥以兵下峽,知祥遣毛重威率兵三千戍夔州。已而荊南高季興死,其子從誨請命,知祥請罷戍兵,不許。知祥諷重威以兵鼓噪,潰而歸,唐以詔書劾重威,知祥奏請無劾,由是唐大臣益以知祥為必反。



 四年,明宗將有事於南郊,遣李仁矩責知祥助禮錢一百萬緡。知祥覺唐謀欲困己,辭不肯出。久之,請獻五十萬而已。初,魏王繼岌東歸,留精兵五千戍蜀。自安重誨疑知祥有異志,聽言事者,用己所親信分守兩川管內諸州,每除守將,則以精兵為其牙隊,多者
 二三千,少者不下五百人,以備緩急。是歲,以夏魯奇為武信軍節度使;分東川之閬州為保寧軍,以李仁矩為節度使;又以武虔裕為綿州刺史。



 仁矩與東川董璋有隙,而虔裕重誨表兄,由是璋與知祥皆懼,以謂唐將致討。自璋鎮東川,未嘗與知祥通問,於是璋始遣人求婚以自結。而知祥心恨璋,欲不許,以問趙季良,季良以為宜合從以拒唐,知祥乃許。於是連表請罷還唐所遣節度使、刺史等。明宗優詔慰諭之。



 長興元年二月,明宗有事於南郊,加拜知祥中書令。初,知祥與璋俱有異志,而重誨信言事者,以璋盡忠於國,獨知祥可疑,重誨猶
 欲倚璋以圖知祥。是歲九月,董璋先反,攻破閬州,擒李仁矩殺之。是月應聖節,知祥開宴,東北望再拜,俯伏嗚咽,泣下沾襟,士卒皆為之歔欷,明日遂舉兵反。



 是秋,明宗改封瓊華公主為福慶長公主,有司言前世公主受封,皆未出降,無遣使就籓冊命之儀。詔有司草具新儀,乃遣秘書監劉岳為冊使。岳行至鳳翔,聞知祥反,乃旋。明宗下詔削奪知祥官爵,命天雄軍節度使石敬瑭為都招討使,夏魯奇為副。知祥遣李仁罕、張業、趙廷隱將兵三萬人會璋攻遂州,別遣侯弘實將四千人助璋守東川,又遣張武下峽取渝州。唐師攻劍門,殺璋守兵三
 千人,遂入劍門。璋來告急,知祥大駭,遣廷隱分兵萬人以東,已而聞唐軍止劍州不進,喜曰:「使唐軍急趨東川,則遂州解圍,吾勢沮而兩川搖矣。今其不進,吾知易與爾。」十二月,敬瑭及廷隱戰于劍門,唐師大敗。張武已取渝州,武病卒,其副將袁彥超代將其軍,又取黔州。二年正月,李仁罕克遂州,夏魯奇死之,知祥以仁罕為武信軍留後,遣人馳魯奇首示敬瑭軍,敬瑭乃班師。利州李彥珂聞唐軍敗東歸,乃棄城走,知祥以趙廷隱為昭武軍留後。李仁罕進攻夔州,刺史安崇阮棄城走,以趙季良為留後。



 是時,唐軍涉險,以餉道為艱,自潼關以西,民
 苦轉饋,每費一石不能致一斗,道路嗟怨,而敬瑭軍亦旋,所在守將又皆棄城走。明宗憂之,以責安重誨。重誨懼,遽自請行。而重誨亦以被讒得罪死。明宗謂致知祥等反,由重誨失策,及重誨死,乃遣西川進奏官蘇願、進奉軍將杜紹本西歸招諭知祥,具言知祥家屬在京師者皆無恙。



 知祥聞重誨誅死,而唐厚待其家屬,乃邀璋欲同謝罪,璋曰:「孟公家屬皆存,而我子孫獨見殺,我何謝為!」知祥三遣使往見璋,璋不聽,乃遣觀察判官李昊說璋,璋益疑知祥賣己,因發怒,以語侵昊。昊乃勸知祥攻之。而璋先襲破知祥漢州,知祥遣趙廷隱率兵三
 萬,自將擊之,陣雞距橋。知祥得璋降卒,衣以錦袍,使持書招降璋,璋曰:「事已及此,不可悔也!」璋軍士皆噪曰:「徒曝我於日中,何不速戰?」璋即麾軍以戰。兵始交,璋偏將張守進來降,知祥乘之,璋遂大敗,走。



 過金鴈橋,麾其子光嗣使降,以保家族,光嗣哭曰:「自古豈有殺父以求生者乎,寧俱就死!」因與璋俱走。知祥遣趙廷隱追之,不及,璋走至梓州見殺,光嗣自縊死,知祥遂並有東川。然自璋死,知祥卒不遣使謝唐。



 唐樞密使范延光曰:「知祥雖已破璋,必借朝廷之勢,以為兩川之重,自非屈意招之,彼亦不能自歸也。」明宗曰:「知祥,吾故人也,本因間諜致
 此危疑,撫吾故人,何屈意之有?」先是,克寧妻孟氏,知祥妹也。莊宗已殺克寧,孟氏歸於知祥,其子瑰,留事唐為供奉官。明宗即遣瑰歸省其母,因賜知祥詔書招慰之。知祥兼據兩川,以趙季良為武泰軍留後、李仁罕武信軍留後、趙廷隱保寧軍留後、張業寧江軍留後、李肇昭武軍留後。季良等因請知祥稱王,以墨制行事,議未決而瑰至蜀。知祥見瑰倨慢。九月,瑰自蜀還,得知祥表,請除趙季良等為五鎮節度,其餘刺史已下,得自除授。又請封蜀王,且言福慶公主已卒。明宗為之發哀,遣閣門使劉政恩為宣諭使。政恩復命,知祥始遣其將朱滉來
 朝。



 四年二月癸亥,制以知祥檢校太尉兼中書令,行成都尹、劍南東西兩川節度,管內觀察處置、統押近界諸蠻,兼西山八國雲南安撫制置等使。遣工部尚書盧文紀冊封知祥為蜀王,而趙季良等五人皆拜節度使。唐兵先在蜀者數萬人,知祥皆厚給其衣食,因請送其家屬,明宗詔諭不許。十一月,明宗崩。明年閏正月,知祥乃即皇帝位,國號蜀。以趙季良為司空、同中書門下平章事,中門使王處回為樞密使,李昊為翰林學士。



 三月,唐潞王舉兵於鳳翔,愍帝遣王思同等討之,思同兵潰,山南西道節度使張虔釗、武定軍節度使孫漢韶皆以其
 地附于蜀。四月,知祥改元曰明德。六月,虔釗等至成都,知祥宴勞之,虔釗奉觴起為壽,知祥手緩不能舉觴,遂病,以其子昶為皇太子監國。知祥卒,謚為文武聖德英烈明孝皇帝,廟號高祖,陵曰和陵。



 昶,知祥第三子也。知祥為兩川節度使,昶為行軍司馬。知祥僭號,以昶為東川節度使、同中書門下平章事。知祥病,昶監國。知祥已卒而祕未發,王處回夜過趙季良,相對泣涕不已,季良正色曰:「今彊侯握兵,專伺時變,當速立嗣君以絕非望,泣無益也。」處回遂與季良立昶,而後發喪。昶立,不改元,仍稱明德,至五年始改元曰慶
 政。



 明德三年三月,熒惑犯積尸,昶以謂積尸蜀分也,懼,欲禳之,以問司天少監胡韞,韞曰:「按十二次,起井五度至柳八度,為鶉首之次,鶉首,秦分也,蜀雖屬秦,乃極南之表爾。前世火入鬼,其應在秦。晉咸和九年三月,火犯積尸,四月,雍州刺史郭權見殺。義熙四年,火犯鬼,明年,雍州刺史朱齡石見殺。而蜀皆無事。」



 乃止。



 昶好打球走馬,又為方士房中之術,多採良家子以充後宮。樞密副使韓保貞切諫,昶大悟,即日出之,賜保貞金數斤。有上書者,言臺省官當擇清流,昶歎曰:「何不言擇其人而任之?」左右請以其言詰上書者,昶曰:「吾見唐太宗
 初即位,獄吏孫伏伽上書言事,皆見嘉納,奈何勸我拒諫耶!」



 然昶年少不親政事,而將相大臣皆知祥故人,知祥寬厚,多優縱之,及其事昶,益驕蹇,多踰法度,務廣第宅,奪人良田,發其墳墓,而李仁罕、張業尤甚。昶即位數月,執仁罕殺之,并族其家。是時,李肇自鎮來朝,杖而入見,稱疾不拜,及聞仁罕死,遽釋杖而拜。



 廣政九年,趙季良卒,張業益用事。業,仁罕甥也。仁罕被誅時,業方掌禁兵,昶懼其反,乃用以為相,業兼判度支,置獄于家,務以酷法厚斂蜀人,蜀人大怨。



 十一年,昶與匡聖指揮使安思謙謀,執而殺之。王處回、趙廷隱相次致仕,由是故
 將舊臣殆盡。昶始親政事,於朝堂置匭以通下情。



 是時,契丹滅晉,漢高祖起于太原,中國多故,雄武軍節度使何建以秦、成、階三州附于蜀,昶因遣孫漢韶攻下鳳州,於是悉有王衍故地。漢將趙思綰據永興、王景崇據鳳翔反,皆送款于昶。昶遣張虔釗出大散關,何建出隴右,李廷珪出子午谷,以應思綰。昶相母昭裔切諫,以為不可,然昶志欲窺關中甚銳,乃遣安思謙益兵以東。已而漢誅思綰、景崇,虔釗等皆罷歸,而思謙恥於無功,多殺士卒以威眾。



 昶與翰林使王藻謀殺思謙,而邊吏有急奏,藻不以時聞,輒啟其封,昶怒之。其殺思謙也,藻方侍
 側,因并擒藻斬之。



 十二年,置吏部三銓、禮部貢舉。



 十三年,昶加號睿文英武仁聖明孝皇帝。封子玄喆秦王,判六軍事;次子玄玨褒王;弟仁毅夔王,仁贄雅王,仁裕嘉王。



 十八年,周世宗伐蜀,攻自秦州。昶以韓繼勛為雄武軍節度,聞周師來伐,歎曰:「繼勳豈足以當周兵邪!」客省使趙季札請行,乃以季札為秦州監軍使。季札行至德陽,聞周兵至,遽馳還奏事。昶問之,季札惶懼不能道一言,昶怒殺之,乃遣高彥儔、李廷珪出堂倉以拒周師。彥儔大敗,走青泥,於是秦、成、階、鳳復入于周。昶懼,分遣使者聘于南唐、東漢,以張形勢。



 二十年,世宗以所得蜀
 俘歸之,昶亦歸所獲周將胡立于京師,因寓書於世宗,世宗怒昶無臣禮,不答。



 二十一年,周兵伐南唐,取淮南十四州,諸國皆懼。荊南高保融以書招昶使歸周,昶以前嘗致書世宗不答,乃止。昶幼子玄寶,生七歲而卒,太常言無服之殤無贈典,昶問李昊,昊曰:「昔唐德宗皇子評生四歲而卒,贈揚州大都督,封肅王,此故事也。」昶乃贈玄寶青州大都督,追封遂王。



 二十五年,立秦王玄喆為皇太子。昶幸晉、漢之際,中國多故,而據險一方,君臣務為奢侈以自娛,至於溺器,皆以七寶裝之。宋興,已下荊、潭,昶益懼,遣大程官孫遇以蠟丸書間行東漢,約出
 兵以撓中國,遇為邊吏所得。太祖皇帝遂詔伐蜀,遣王全斌、崔彥進等出鳳州,劉光乂、曹彬等出歸州;詔八作司度右掖門南、臨汴水為昶治第一區,凡五百餘間,供帳什物皆具,以待昶。



 昶遣王昭遠、趙彥韜等拒命。昭遠,成都人也,年十三,事東郭禪師智諲為童子。知祥嘗飯僧於府,昭遠執巾履從智諲以入,知祥見之,愛其惠黠。時昶方就學,即命昭遠給事左右,而見親狎。昶立,以為捲簾使。樞密使王處回致仕,昶以樞密使權重難制,乃以昭遠為通奏使知樞密使事,然事無大小,一以委之,府庫金帛恣其所取不問。昶母李太后常為昶言昭遠
 不可用,昶不聽。昭遠好讀兵書,以方略自許。兵始發成都,昶遣李昊等餞之,昭遠手執鐵如意,指揮軍事,自比諸葛亮,酒酣,謂昊曰:「吾之是行,何止克敵,當領此二三萬雕面惡少兒,取中原如反掌爾!」



 昶又遣子玄喆率精兵數萬守劍門。玄喆輦其愛姬,攜樂器、伶人數十以從,蜀人見者皆竊笑。全斌至三泉,遇昭遠,擊敗之。昭遠焚吉柏江浮橋,退守劍門。軍頭向韜得蜀降卒言:「來蘇小路,出劍門南清彊店,與大路合。」全斌遣偏將史延德分兵出來蘇,北擊劍門,與全斌夾攻之,昭遠、彥韜敗走,皆見擒。玄喆聞昭遠等敗,亦逃歸。



 劉光乂攻夔州,守
 將高彥儔戰敗,閉牙城拒守,判官羅濟勸其走,彥儔曰:「吾昔不能守秦川,今又奔北,雖人主不殺我,我何面目見蜀人乎!」又勸其降,彥儔不許,乃自焚死。而蜀兵所在奔潰,將帥多被擒獲。昶問計於左右,老將石頵以謂東兵遠來,勢不能久,宜聚兵堅守以敝之。昶歎曰:「吾與先君以溫衣美食養士四十年,一旦臨敵,不能為吾東向放一箭,雖欲堅壁,誰與吾守者邪!」乃命李昊草表以降,時乾德三年正月也。自興師至昶降,凡六十六日。初,昊事王衍為翰林學士,衍之亡也,昊為草降表,至是又草焉,蜀人夜表其門曰「世修降表李家」,當時傳以為笑。



 昶
 至京師,拜檢校太師兼中書令,封秦國公,七日而卒,追封楚王。其母李氏,為人明辯,甚見優禮,詔書呼為「國母」,嘗召見勞之曰:「母善自愛,無戚戚思蜀,他日當送母歸。」李氏曰:「妾家本太原,倘得歸老故鄉,不勝大願。」是時劉鈞尚在。太祖大喜曰:「俟平劉鈞,當如母願。」昶之卒也,李氏不哭,以酒酹地祝曰:「汝不能死社稷,茍生以取羞。吾所以忍死者,以汝在也。吾今何用生為!」



 因不食而卒。其餘事具國史。



 知祥興滅年數甚明,諸書皆同,蓋自同光三年乙酉入蜀,至皇朝乾德三年乙丑國滅,凡四十一年。惟《舊五代史》,雲同光三年丙戌至乾德三年乙丑,四十年者,繆也。



\end{pinyinscope}