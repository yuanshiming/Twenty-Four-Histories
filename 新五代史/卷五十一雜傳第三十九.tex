\article{卷五十一雜傳第三十九}

\begin{pinyinscope}

 朱守殷硃守殷,少事唐莊宗為奴,名曰會兒。莊宗讀書,會兒常侍左右。莊宗即位,以其廝養為長直軍,以守殷為軍使,故未嘗經戰陣之用。然好言人陰私長短以自結,莊宗以為忠,遷蕃漢馬步軍都虞候,使守德勝。王彥章攻德勝,守殷無備,遂破南城,莊宗罵曰:「駑才,果誤予事!」明宗請以守殷行軍法,莊宗不聽。同光二年,領鎮武軍節度使。是時,莊宗初入洛,守殷巡檢校京師,恃恩驕恣,凌侮勳
 舊,與伶人景進相為表裏。魏王繼岌已殺郭崇韜,進誣朱友謙與崇韜謀反,莊宗遣守殷圍其第而殺之。是時,明宗自鎮州來朝,居于私第。莊宗方惑群小,疑忌大臣,遣守殷伺察明宗動靜。守殷陰使人告明宗曰:「位高人臣者身危,功蓋天下者不賞,公可謂位高而功著矣。宜自圖歸籓,無與禍會也!」明宗曰:「吾洛陽一匹夫爾,何能為也!」既而明宗卒反于魏。莊宗東討,守殷將騎軍陣宣仁門外以俟駕。郭從謙作亂,犯興教門以入,莊宗亟召守殷等軍,守殷按軍不動。莊宗獨與諸王宦官百餘人射賊,守殷等終不至,方移兵憩北邙山下,聞莊宗
 已崩,即馳入宮中,選載嬪御、寶貨以歸,縱軍士劫掠,遣人趣明宗入洛。



 明宗即位,拜守殷同中書門下平章事、河南尹、判六軍諸衛事。明年,遷宣武軍節度使。九月,明宗詔幸汴州,議者喧然,或以為征吳,或以為東諸侯有屈強者,將制置之。守殷尤不自安,乃殺都指揮使馬彥超,閉城反。明宗行至京水,聞守殷反,遣范延光馳兵傅其城。汴人開門納延光,守殷自殺其族,乃引頸命左右斬之。



 明宗至汴州,命鞭其尸,梟首于市七日,傳徇洛陽。



 守殷之將反也,召都指揮使馬彥超與計事,彥超不從,守殷殺之。明宗憐彥超之死,以其子承祚為洺州長史。



 董璋董璋,不知其世家何人也。少與高季興、孔循俱為汴州富人李讓家僮。梁太祖鎮宣武,養讓為子,是為朱友讓。其僮奴以友讓故,皆得事梁太祖,璋以軍功為指揮使。晉李繼韜以潞州叛降梁,末帝遣璋攻下澤州,即以璋為刺史。



 梁亡,璋事唐為邠寧節度使,與郭崇韜相善。崇韜伐蜀,以璋為行營右廂馬步軍都虞候,軍事大小,皆與參決。蜀平,以為劍南東川節度使,孟知祥鎮西川。其後,二人有異志。安重誨居中用事,議者多言知祥必不為唐用,而能制知祥者璋也,往往稱璋忠義,重誨以為然,頗優寵之,以故璋益橫。



 天成四年,明宗祀天南
 郊,詔兩川貢助南郊物五十萬,使李仁矩齎安重誨書往諭璋,璋訴不肯出,只出十萬而已。又因事欲殺仁矩,仁矩涕泣而免,歸言璋必反。



 其後使者至東川,璋益倨慢,使者還,多言璋欲反狀。重誨患之,乃稍擇將吏為兩川刺史,以精兵為其牙衛,分布其諸州。又分閬州置保寧軍,以仁矩為節度使,遣姚洪將兵千人從仁矩戍閬州。璋及知祥覺唐疑己,且削其地,遂連謀以反。璋因為其子娶知祥女以相結。又遣其將李彥釗扼劍門關為七砦,於關北增置關,號永定。



 凡唐戍兵東歸者,皆遮留之,獲其逃者,覆以鐵籠,火炙之,或刲肉釘面,割心而啖。
 長興元年九月,知祥攻陷遂州,璋攻陷閬州執李仁矩、姚洪,皆殺之。



 初,璋等反,唐獨誅璋家屬,知祥妻子皆在成都,其疏屬留京師者皆不誅。石敬瑭討璋等,兵久無功,而自關以西饋運不給,遠近勞敝,明宗患之。安重誨自往督軍,敬瑭不納,重誨遂得罪死,敬瑭亦還。明宗乃遣西川進奏官蘇愿、東川軍將劉澄西歸,諭璋等使改過。知祥遣人告璋,欲與俱謝過自歸,璋曰:「唐不殺孟公家族,於西川恩厚矣。我子孫何在?何謝之有!」璋由此疑知祥賣己。三年四月,以兵萬人攻知祥,戰于彌牟,璋大敗,還走梓州。初,唐陵州刺史王暉代還過璋,璋邀留之。
 至是,暉執璋殺之,傳其首於知祥。



 范延光範延光,字子瑰,相州臨漳人也。唐明宗為節度使,置延光麾下,而未之奇也。



 明宗破鄆州,梁兵方扼楊劉,其先鋒將康延孝陰送款於明宗。明宗求可以通延孝款於莊宗者,延光輒自請行,乃懷延孝蠟丸書,西見莊宗致之,且曰:「今延孝雖有降意,而梁兵扼楊劉者甚盛,未可圖也,不如築壘馬家口以通汶陽。」莊宗以為然。



 壘成,梁遣王彥章急攻新壘。明宗使延光間行求兵,夜至河上,為梁兵所得,送京師,下延光獄,搒掠數百,脅以白刃,延光終不肯言晉事。繫之數月,稍為獄吏所獲。莊宗入汴,
 獄吏去其桎梏,拜而出之。莊宗見延光,喜,拜檢校工部尚書。



 明宗時,為宣徽南院使。明宗行幸汴州,至滎陽,朱守殷反,延光曰:「守殷反迹始見,若緩之使得為計,則城堅而難近。故乘人之未備者,莫若急攻,臣請騎兵五百,馳至城下,以神速駭之。」乃以騎兵五百,自暮疾馳至半夜,行二百里,戰于城下。遲明,明宗亦馳至,汴兵望見天子乘輿,乃開門,而延光先入,猶巷戰,殺傷甚眾,守殷死,汴州平。



 明年,遷樞密使,出為成德軍節度使。安重誨死,復召延光與趙延壽並為樞密使。明宗問延光馬數幾何,對曰「騎軍三萬五千。」明宗撫髀歎曰:「吾兵間四十
 年,自太祖在太原時,馬數不過七千,莊宗取河北,與梁家戰河上,馬纔萬匹。今有馬三萬五千而不能一天下,吾老矣,馬多奈何!」延光因曰:「臣嘗計,一馬之費,可養步卒五人,三萬五千匹馬,十五萬兵之食也。」明宗曰:「肥戰馬而瘠吾人,此吾所媿也!」



 夏州李仁福卒,其子彞超自立而邀旄節。明宗遣安從進代之,彞超不受代。以兵攻之,久不克。隰州刺史劉遂凝馳驛入見獻策,言綏、銀二州之人,皆有內向之意,請除二刺史以招降之。延光曰:「王師問罪,本在彞超,夏州已破,綏、銀豈足顧哉!若不破夏州,雖得綏、銀,不能守也。」遂凝又請自馳入說彞超使出
 降,延光曰:「一遂凝,萬一失之不足惜,所惜者朝廷大體也。」是時,王淑妃用事,遂凝兄弟與淑妃有舊,方倚以蒙恩寵,所言無不聽,而大臣以妃故,多不敢爭,獨延光從容沮止之。明宗有疾,不能視朝,京師之人,洶洶異議,藏竄山谷,或寄匿於軍營,有司不能禁。或勸延光以嚴法制之,延光曰:「制動當以靜,宜少待之。」



 已而明宗疾少間,京師乃定。



 是時,秦王握兵驕甚,宋王弱而且在外,議者多屬意於潞王。延光懼禍之及也,乃求罷去。延壽陰察延光有避禍意,亦遽求罷。明宗再三留之,二人辭益懇至,繼之以泣。明宗不得已,乃皆罷之,延光復鎮成德,而
 用朱弘昭、馮贇為樞密使。已而秦王舉兵見誅,明宗崩,潞王反,殺愍帝,唐室大亂,弘昭、贇皆及禍以死。末帝復詔延光為樞密使,拜宣武軍節度使。天雄軍亂,逐節度使劉延皓,遣延光討平之,即以為天雄軍節度使。



 延光常夢大蛇自臍入其腹,半入而掣去之,以問門下術士張生,張生贊曰:「蛇,龍類也,入腹內,王者之兆也。」張生自延光微時,言其必貴,延光素神之,常置門下。言多輒中,遂以其言為然,由是頗畜異志。當晉高祖起太原,末帝遣延光以兵二萬屯遼州,與趙延壽掎角。既而延壽先降,延光獨不降。高祖即位,延光賀表又頗後諸侯至,
 又其女為末帝子重美妃,以此遂懷反側。高祖封延光臨清王以慰其心。



 有平山人祕瓊者,為成德軍節度使董溫其衙內指揮使,後溫其為契丹所虜,瓊乃悉殺溫其家族,瘞之一穴,而取其家貲巨萬計,晉高祖入立,以瓊為齊州防御使,橐其貲裝,道出于魏。延光陰遣人以書招之,瓊不納,延光怒,選兵伏境上,伺瓊過,殺之于夏津,悉取其貲,以戍邏者誤殺聞。由是高祖疑其必為亂,乃幸汴州。



 天福二年六月,延光遂反,遣其牙將孫銳、澶州刺史馮暉,以兵二萬距黎陽,掠滑、衛。高祖以楊光遠為招討使,引兵自滑州渡胡梁攻之。銳輕脫無謀,兵
 行以娼女十餘自隨,張蓋操扇,酣歌飲食自若。軍士苦大熱,皆不為用。光遠得諜者,詢得其謀,誘銳等渡河,半濟而擊之,兵多溺死,銳、暉退走入魏,閉壁不復出。



 初,延光反意未決,而得暴疾不能興,銳乃陰召暉入城。迫延光反,延光惶惑,遂從之。高祖聞延光用銳等以反,笑曰:「吾雖不武,然嘗從明宗取天下,攻堅破強多矣。如延光已非我敵,況銳等兒戲邪?行取孺子爾!」乃決意討之。



 延光初無必反意,及銳等敗,延光遣牙將王知新齎表自歸,高祖不見,以知新屬武德司。延光又附楊光遠表請降,不報,延光遂堅守。晉以箭書二百射城中,悉赦魏
 人,募能斬延光者。然魏城堅難下,攻之逾年不克,師老糧匱。宗正丞石帛上書極諫,請赦延光,願以單車入說而降之。高祖亦悔悟。三年九月,使謁者入魏赦延光,延光乃降,冊封東平郡王、天平軍節度使,賜鐵券。居數月來朝,因慚請老,以太子太師致仕。



 初,高祖赦降延光,語使者謂之曰:「許卿不死矣,若降而殺之,何以享國?」



 延光謀於副使李式,式曰:「主上敦信明義,許之不死,則不死矣。」乃降。乃致仕居京師,歲時宴見,高祖待之與群臣無間,然心不欲使在京師。歲餘,使宣徽使劉處讓載酒夜過延光,謂曰:「上遣處讓來時,適有契丹使至,北朝皇
 帝問晉魏博反臣何在,恐晉不能制,當鎖以來,免為中國後患。」延光聞之泣下,莫知所為。



 處讓曰:「當且之洛陽,以避契丹使者。」延光曰:「楊光遠留守河南,吾之仇也。



 吾有田宅在河陽,可以往乎?」處讓曰:「可也。」乃挈其帑歸河陽。其行輜重盈路,光遠利其貲,果圖之。因奏曰:「延光反覆姦臣,若不圖之,非北走胡則南走吳越,請拘之洛陽。」高祖猶豫未決。光遠兼鎮河陽,其子承勳知州事,乃遣承勛以兵脅之使自裁。延光曰:「天子賜我鐵券,許之不死,何得及此?」乃以壯士驅之上馬,行至浮橋,推墮水溺死,以延光自投水死聞,因盡取其貲。高祖以適會其意,
 不問,為之輟朝,贈太傅。水運軍使曹千獲其流尸于繆家灘,詔許歸葬相州。



 已葬,墓輒崩,破其棺槨,頭顱皆碎。初,祕瓊殺董溫其取其貲,延光又殺瓊而取之,而終以貲為光遠所殺,而光遠亦不能免也。



 當延光反時,有李彥珣者,為河陽行軍司馬,張從賓反河陽,彥珣附之,從賓敗,彥珣奔于魏,延光以為步軍都監,使之守城。招討使楊光遠知彥珣邢州人也,其母尚在,乃遣人之邢州,取其母至城下,示彥珣以招之,彥珣望見,自射殺之。



 及延光出降,晉高祖拜彥珣房州刺史,大臣言彥珣殺母當誅,高祖以謂赦令已行,不可失信。後以坐贓誅。



 嗚呼,甚哉,人性之慎於習也!故聖人於仁義深矣,其為教也,勤而不怠,緩而不迫,欲民漸習而自趨之,至於久而安以成俗也。然民之無知,習見善則安於為善,習見惡則安於為惡。五代之亂,其來遠矣。自唐之衰,干戈饑饉,父不得育其子,子不得養其親。其始也,骨肉不能相保,蓋出于不幸,因之禮義日以廢,恩愛日以薄,其習久而遂以大壞,至於父子之間,自相賊害。五代之際,其禍害不可勝道也。夫人情莫不共知愛其親,莫不共知惡於不孝,然彥珣彎弓射其母,高祖從而赦之,非徒彥珣不自知為大惡,而高祖亦安焉不以為怪也,豈非積
 習之久而至於是歟!《語》曰:「性相近,習相遠。」至其極也,使人心不若禽獸,可不哀哉!若彥珣之惡,而恬然不以為怪,則晉出帝之絕其父,宜其舉世不知為非也。



 婁繼英婁繼英,不知何許人也。歷梁、唐,為絳、冀二州刺史、北面水陸轉運使、耀州團練使。晉高祖時,為左監門衛上將軍。繼英子婦,溫延沼女也,自明宗時誅其父韜,延沼兄弟廢居于許,心常怨望。及范延光反,繼英有弟為魏州子城都虞候,延光遣人以蠟書招繼英,繼英乃遣延沼入魏見延光,延光大喜,與之信箭,使陰圖許。延沼與其弟延浚、延袞募不逞之徒千人,期以攻許。而許州節
 度使萇從簡以延光之反,疑有應者,為備甚嚴。延沼未及發,延光蠟書事泄於京師,繼英惶恐不自安,乃出奔許。高祖下詔招慰之,使復位,繼英懼不敢出。溫氏兄弟謀殺繼英以自歸,延沼以其女故不忍。張從賓反於洛陽,延沼兄弟乃與繼英俱投從賓於汜水。繼英知溫氏之初欲殺己也,反譖延沼兄弟於從賓,從賓殺之。從賓敗,繼英為杜重威所殺。



 安重榮安重榮,小字鐵胡,朔州人也。祖從義,利州刺史。父全,勝州刺史、振武馬步軍都指揮使。重榮有力,善騎射,為振武巡邊指揮使。晉高祖起太原,使張穎陰招重榮,其母
 與兄皆以為不可,重榮業已許穎,母、兄謀共殺穎以止之,重榮曰:「未可,吾當為母卜之。」乃立一箭,百步而射之,曰:「石公為天子則中。」一發輒中;又立一箭而射之,曰:「吾為節度使則中。」一發又中,其母、兄乃許,重榮以巡邊千騎叛入太原。高祖即位,拜重榮成德軍節度使。



 重榮雖武夫,而曉吏事,其下不能欺。有夫婦訟其子不孝者,重榮拔劍授其父,使自殺之,其父泣曰:「不忍也!」其母從傍詬罵,奪其劍而逐之,問之,乃繼母也,重榮叱其母出,後射殺之。



 重榮起於軍卒,暴至富貴,而見唐廢帝、晉高祖皆自籓侯得國,嘗謂人曰:「天子寧有種邪?兵強馬
 壯者為之爾!」雖懷異志,而未有以發也。是時,高祖與契丹約為父子,契丹驕甚,高祖奉之愈謹,重榮憤然,以謂「詘中國以尊夷狄,困已敝之民,而充無厭之欲,此晉萬世恥也!」數以此非誚高祖。契丹使者往來過鎮州,重榮箕踞慢罵,不為之禮,或執殺之。是時,吐渾白氏役屬契丹,苦其暴虐,重榮誘之入塞。契丹數遣使責高祖,并求使者,高祖對使者鞠躬俯首,受責愈謹,多為好辭以自解,而姑息重榮不能詰。乃遣供奉官張澄以兵二千搜索并、鎮、忻、代山谷中吐渾,悉驅出塞。吐渾去而復來,重榮卒納之,因招集亡命,課民種稗,食馬萬匹,所為益驕。
 因怒殺指揮使賈章,誣之以反。章女尚幼,欲捨之,女曰:「吾家三十口皆死於兵,存者特吾與父爾,今父死,吾何忍獨生,願就死!」遂殺之。鎮人於是高賈女之烈,而知重榮之必敗也。重榮既僭侈,以為金魚袋不足貴,刻玉為魚佩之。娶二妻,高祖因之並加封爵。



 天福六年夏,契丹使者拽剌過鎮,重榮侵辱之,拽剌言不遜,重榮怒,執拽剌,以輕騎掠幽州南境之民,處之博野。上表曰:「臣昨據熟吐渾白承福、赫連功德等領本族三萬餘帳自應州來奔,又據生吐渾、渾、契苾、兩突厥三部南北將沙陀、安慶、九府等各領其族、牛羊、車帳、甲馬七八路來奔,具
 言契丹殘害,掠取生口羊馬,自今年二月已後,號令諸蕃,點閱強壯,辦具軍裝,期以上秋南向。諸蕃部誠恐上天不祐,敗滅家族,願先自歸,其諸部勝兵眾可十萬。又據沿河黨項、山前後逸越利諸族首領皆遣人送契丹所授告身、敕牒、旗幟來歸款,皆號泣告勞,願治兵甲以報怨。又據朔州節度副使趙崇殺節度使劉山,以城來歸。竊以諸蕃不招呼而自至,朔州不攻伐而自歸,雖繫人情,盡由天意。又念陷蕃諸將等,本自勳勞,久居富貴,沒身虜塞,酷虐不勝,企足朝廷,思歸可諒,茍聞傳檄,必盡倒戈。」其表數千言。又為書以遺朝廷大臣、四方籓鎮,
 皆以契丹可取為言。高祖患之,為之幸鄴,報重榮曰:「前世與虜和親,皆所以為天下計,今吾以天下臣之,爾以一鎮抗之,大小不等,無自辱焉!」重榮謂晉無如我何,反意乃決。重榮雖以契丹為言,反陰遣人與幽州節度使劉晞相結。契丹亦利晉多事,幸重榮之亂,期兩敝之,欲因以窺中國,故不加怒於重榮。



 重榮將反也,其母又以為不可,重榮曰:「為母卜之。」指其堂下幡竿龍口仰射之,曰:「吾有天下則中之。」一發而中,其母乃許。饒陽令劉巖獻水鳥五色,重榮曰:「此鳳也。」畜之後潭。又使人為大鐵鞭以獻,誑其民曰:「鞭有神,指人,人輒死。」號「鐵鞭郎君」,
 出則以為前驅。鎮之城門抱關鐵胡人,無故頭自落,鐵胡,重榮小字,雖甚惡之,然不悟也。其冬,安從進反襄陽,重榮聞之,乃亦舉兵。是歲,鎮州大旱、蝗,重榮聚飢民數萬,驅以向鄴,聲言入覲。行至宗城破家堤,高祖遣杜重威朔之,兵已交,其將趙彥之與重榮有隙,臨陣卷旗以奔晉軍,其鎧甲鞍轡皆裝以銀,晉國不知其來降,爭殺而分之。重榮聞彥之降晉,大懼,退入于輜重中,其兵二萬皆潰去。是冬大寒,潰兵飢凍及見殺無孑遺,重榮獨與十餘騎奔還,以牛馬革為甲,驅州人守城以待。重威兵至城下,重榮裨將自城西水碾門引官軍以入,殺守
 城二萬餘人。重榮以吐渾數百騎守牙城,重威使人擒之,斬首以獻,高祖御樓受馘,命漆其首送于契丹。改成德軍為順德,鎮州曰恆州,常山曰恆山云。



 安從進安從進,振武索葛部人也。祖、父皆事唐為騎將。從進初從莊宗於兵間,為護駕馬軍都指揮使,領貴州刺史。明宗時,為保義、彰武軍節度使,未嘗將兵征伐。



 李彞超自立於夏州,從進嘗一以兵往,卒亦無功。愍帝即位,徙領順化,為侍衛馬軍都指揮使。潞王反鳳翔,從進巡檢京城,殺樞密使馮贇,送款于從珂。愍帝出奔,從珂將至京師,從進率百官班迎于郊。清泰中,徙鎮山南東道。晉高
 祖即位,加同中書門下平章事。



 高祖取天下不順,常以此慚,籓鎮多務,過為姑息,而籓鎮之臣,或不自安,或心慕高祖所為,謂舉可成事,故在位七年,而反者六起,從進最後反,然皆不免也。自范延光反鄴,從進已畜異志,恃江為險,招集亡命,益置軍兵。南方貢輸道出襄陽者,多擅留之,邀遮商旅,皆黥以充軍。與安重榮陰相結託,期為表裹。高祖患之,謀徙從進,使人謂曰:「東平王建立來朝,願還鄉里,已徙上黨。朕虛青州以待卿,卿誠樂行,朕即降制。」從進報曰:「移青州在漢江南,臣即赴任。」



 高祖亦優容之。其子弘超為宮苑副使,居京師,從進請賜告歸,
 遂不遣。王令謙、潘知麟者,皆從進牙將也,常從從進最久,知其必敗,切諫之。從進遣子弘超與令謙遊南山,酒酣,令人推墮崖死。



 天福六年,安重榮執殺契丹使者,反迹見,高祖為之幸鄴,鄭王重貴留守京師。



 宰相和凝曰:「陛下且北,從進必反,何以制之?」高祖曰:「卿意奈何?」凝曰:「臣聞兵法,先人者奪人,願為空名宣敕十數通授鄭王,有急則命將以往。」從進聞高祖北,遂殺知麟以反。鄭王以空名敕授李建崇、郭金海等討之,從進引兵攻鄧州,不克,進至湖陽,遇建崇等,大駭,以為神速,復為野火所燒,遂大敗。從進以數十騎奔還襄陽。高祖遣高行周圍
 之,踰年糧盡,從進自焚死。執其子弘受及其將佐四十三人送京師,高祖御樓受俘,徇於市而斬之。降襄陽為防禦,贈令謙忠州刺史,知麟順州刺史。



 楊光遠楊光遠,字德明,其父曰阿登啜,蓋沙陀部人也。光遠初名阿檀,為唐莊宗騎將,從周德威戰契丹於新州,折其一臂,遂廢不用。久之,以為幽州馬步軍都指揮使,戍瓦橋關。光遠為人病禿折臂,不通文字,然有辨智,長於吏事。明宗時,為媯、瀛、冀、易四州刺史,以治稱。



 初,唐兵破王都於中山,得契丹大將荝剌等十餘人。已而契丹與中國通和,遣使者求荝剌等,明宗與大臣議,皆欲歸之,獨
 光遠不可,曰:「荝剌皆北狄善戰者,彼失之如去手足;且居此久,熟知中國事,歸之豈吾利也!」明宗曰:「蕃人重盟誓,已與吾好,豈相負也?」光遠曰:「臣恐後悔不及爾!」明宗嘉其說,卒不遣荝剌等。光遠自易州刺史拜振武軍節度使。清泰二年,徙鎮中山,兼北面行營都虞候,禦契丹於雲、應之間。



 晉高祖起太原,末帝以光遠佐張敬達為太原四面招討副使,為契丹所敗,退守晉安寨。契丹圍之數月,人馬食盡,殺馬而食,馬盡,乃殺敬達出降。耶律德光見之,靳曰:「爾輩大是惡漢兒。」光遠與諸將初不知其誚己,猶為謙言以對,德光曰:「不用鹽酪,食一萬
 匹戰馬,豈非惡漢兒邪!」光遠等大慚伏,德光問曰:「懼否?」皆曰:「甚懼。」曰:「何懼?」曰:「懼皇帝將入蕃。」德光曰:「吾國無土地官爵以居汝,汝等勉事晉。」晉高祖以光遠為宣武軍節度使、侍衛馬步軍都指揮使。光遠進見,佯為悒悒之色,常如有所恨者,高祖疑其有所不足,使人問之,對曰:「臣於富貴無不足也,惟不及張生鐵死得其所,此常為愧爾!」由是高祖以為忠,頗親信之。



 范延光反,以為魏府都招討使,久之不能下,高祖卒用佗計降延光。而光遠自以握重兵在外,謂高祖畏己,始為恣橫。高祖每優容之,為選其子承祚尚長安公主,其次子承信等皆超拜
 官爵,恩寵無比。樞密使桑維翰惡之,數以為言。光遠自魏來朝,屢指維翰擅權難制。高祖不得已,罷出維翰於相州,亦徙光遠西京留守,兼鎮河陽,奪其兵職。光遠始大怨望,陰以寶貨奉契丹,訴己為晉疏斥。所養部曲千人,撓法犯禁河、洛之間,甚於寇盜。天福五年,徙鎮平盧,封東平王。光遠請其子以行,乃拜承祚單州刺史,承勛萊州防禦使,父子俱東,車騎連屬數十里。出帝即位,拜太師,封壽王。



 是時,晉馬少,括天下馬以佐軍,景延廣請取光遠前所借官馬三百匹,光遠怒曰:「此馬先帝賜我,安得復取,是疑我反也!」遂謀為亂。而承祚自單州逃歸,
 出帝即以承祚為淄州刺史,遣使者賜以玉帶、御馬以慰安之,光遠益驕,乃反。召契丹入寇,陷貝州。博州刺史周儒亦叛降契丹。



 是時,出帝與耶律德光相距澶、魏之間,鄆州觀察判官竇儀計事軍中,謀曰:「今不以重兵大將守博州渡,使儒得引契丹東過河與光遠合,則河南危矣!」出帝乃遣李守貞、皇甫遇以兵萬人沿河而下。儒果引契丹自馬家渡濟河,方築壘,守貞等急擊之,契丹大敗,遂與光遠隔絕。德光聞河上兵大敗,與晉決戰戚城,亦敗。



 契丹已北,出帝復遣守貞、符彥卿東討,光遠嬰城固守,自夏至冬,城中人相食幾盡。光遠北望契丹,稽
 首以呼德光曰:「皇帝誤光遠耶!」其子承勳等勸光遠出降,光遠曰:「我在代北時,嘗以紙錢祭天池,投之輒沒,人言我當作天子,宜且待時,毋輕議也。」承勳知不可,乃殺節度判官丘濤、親將杜延壽、楊瞻、白延祚等,劫光遠幽之,遣人奉表待罪。承信、承祚皆詣闕自歸,而光遠亦上章請死。



 出帝以其二子為侍衛將軍,賜光遠昭書,許以不死,群臣皆以為不可,乃敕李守貞便宜處置。守貞遣客省副使何延祚殺之于其家。延祚至其第,光遠方閱馬于廄,延祚使一都將入謂之曰:「天使在門,欲歸報天子,未有以藉手。」光遠曰:「何謂也?」曰:「願得大王頭爾!」光遠
 罵曰:「我有何罪?昔我以晉安寨降契丹,使爾家世世為天子,我亦望以富貴終身,而反負心若此!」遂見殺,以病卒聞。



 承勳事晉為鄭州防禦使,德光滅晉,使人召承勛至京師,責其劫父,臠而食之,乃以承信為平盧節度使。漢高祖贈光遠尚書令,封齊王,命中書舍人張正撰光遠碑銘文賜承信,使刻石于青州。碑石既立,天大雷電,擊折之。



 阿登啜初非姓氏,其後改名瑊而姓楊氏。光遠初名檀,清泰二年,有司言明宗廟諱犯偏傍者皆易之,乃賜名光遠云。光遠既病禿,而妻又跛其足也,人為之語曰:「自古豈有禿瘡天子、跛腳皇后邪?」相傳以為笑。然而
 召夷狄為天下首禍,卒滅晉氏,瘡痍中國者三十餘年,皆光遠為之也。



\end{pinyinscope}