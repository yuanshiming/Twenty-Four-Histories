\article{卷三十七伶官傳第二十五}

\begin{pinyinscope}

 嗚呼,盛衰之理,雖曰天命,豈非人事哉!原莊宗之所以得天下,與其所以失之者,可以知之矣。世言晉王之將終也,以三矢賜莊宗而告之曰:「梁,吾仇也,燕王吾所立,契丹與吾約為兄弟,而皆背晉以歸梁。此三者,吾遺恨也。與爾三矢,爾其無忘乃父之志!」莊宗受而藏之於廟。其後用兵,則遣從事以一少牢告廟,請其矢,盛以錦囊,負而前驅,及凱旋而納之。方其係燕父子以組,函梁君
 臣之首,入于太廟,還矢先王而告以成功,其意氣之盛,可謂壯哉!及仇讎已滅,天下已定,一夫夜呼,亂者四應,蒼皇東出,未及見賊而士卒離散,君臣相顧,不知所歸,至於誓天斷髮,泣下沾襟,何其衰也!豈得之難而失之易歟?抑本其成敗之迹而皆自於人歟?《書》曰:「滿招損,謙得益。」憂勞可以興國,逸豫可以亡身,自然之理也。故方其盛也,舉天下之豪傑莫能與之爭;及其衰也,數十伶人困之,而身死國滅,為天下笑。夫禍患常積於忽微,而智勇多困於所溺,豈獨伶人也哉!作《伶官傳》。



 莊宗既好俳優,又知音,能度曲,至今汾、晉之俗,往往能
 歌其聲,謂之「御製」者皆是也。其小字亞子,當時人或謂之亞次。又別為優名以自目,曰李天下。



 自其為王,至於為天子,常身與俳優雜戲于庭,伶人由此用事,遂至於亡。



 皇后劉氏素微,其父劉叟,賣藥善卜,號劉山人。劉氏性悍,方與諸姬爭寵,常自恥其世家,而特諱其事。莊宗乃為劉叟衣服,自負蓍囊藥笈,使其子繼岌提破帽而隨之,造其臥內,曰:「劉山人來省女。」劉氏大怒,笞繼岌而逐之。宮中以為笑樂。



 其戰於胡柳也,嬖伶周匝為梁人所得。其後滅梁入汴,周匝謁於馬前,莊宗得之喜甚,賜以金帛,勞其良苦。周匝對曰:「身陷仇人,而得不死以
 生者,教坊使陳俊、內園栽接使儲德源之力也。願乞二州以報此兩人。」莊宗皆許以為刺史。郭崇韜諫曰:「陛下所與共取天下者,皆英豪忠勇之士。今大功始就,封賞未及於一人,而先以伶人為刺史,恐失天下心。不可!」因格其命。踰年,而伶人屢以為言,莊宗謂崇韜曰:「吾已許周匝矣,使吾慚見此三人。公言雖正,然當為我屈意行之。」



 卒以俊為景州刺史、德源為憲州刺史。



 莊宗好畋獵,獵于中牟,踐民田。中牟縣令當馬切諫,為民請,莊宗怒,叱縣令去,將殺之。伶人敬新磨知其不可,乃率諸伶走追縣令,擒至馬前責之曰:「汝為縣令,獨不知吾天子好
 獵邪?奈何縱民稼穡以供稅賦!何不飢汝縣民而空此地,以備吾天子之馳騁?汝罪當死!」因前請亟行刑,諸伶共唱和之。莊宗大笑,縣令乃得免去。莊宗嘗與群優戲于庭,四顧而呼曰:「李天下,李天下何在?」新磨遽前以手批其頰。莊宗失色,左右皆恐,群伶亦大驚駭,共持新磨詰曰:「汝奈何批天子頰?」新磨對曰:「李天下者,一人而已,復誰呼邪!」於是左右皆笑,莊宗大喜,賜與新磨甚厚。新磨嘗奏事殿中,殿中多惡犬,新磨去,一犬起逐之,新磨倚柱而呼曰:「陛下毋縱兒女齧人!」莊宗家世夷狄,夷狄之人諱狗,故新磨以此譏之。莊宗大怒,彎弓注矢將
 射之,新磨急呼曰:「陛下無殺臣!臣與陛下為一體,殺之不祥!」莊宗大驚,問其故,對曰:「陛下開國,改元同光,天下皆謂陛下同光帝。且同,銅也,若殺敬新磨,則同無光矣。」莊宗大笑,乃釋之。



 然時諸伶,獨新磨尤善俳,其語最著,而不聞其佗過惡。其敗政亂國者,有景進、史彥瓊、郭門高三人為最。



 是時,諸伶人出入宮掖,侮弄縉紳,群臣憤嫉,莫敢出氣,或反相附託,以希恩倖,四方籓鎮,貨賂交行,而景進最居中用事。莊宗遣進等出訪民間,事無大小皆以聞。每進奏事殿中,左右皆屏退,軍機國政皆與參決,三司使孔謙兄事之,呼為「八哥」。莊宗初入洛,居唐
 故宮室,而嬪御未備。閹宦希旨,多言宮中夜見鬼物,相驚恐,莊宗問所以禳之者,因曰:「故唐時,後宮萬人,今空宮多怪,當實以人乃息。」莊宗欣然。其後幸鄴,乃遣進等採鄴美女千人,以充後宮。而進等緣以為姦,軍士妻女因而逃逸者數千人。莊宗還洛,進載鄴女千人以從,道路相屬,男女無別。魏王繼岌已破蜀,劉皇后聽宦者讒言,遣繼岌賊殺郭崇韜。崇韜素嫉伶人,常裁抑之,伶人由此皆樂其死。皇弟存乂,崇韜之婿也,進讒於莊宗曰:「存乂且反,為婦翁報仇。」乃囚而殺之。朱友謙,以梁河中降晉者,及莊宗入洛,伶人皆求賂於友謙,友謙不能給
 而辭焉。進乃讒友謙曰:「崇韜且誅,友謙不自安,必反,宜並誅之。」於是及其將五六人皆族滅之,天下不勝其冤。進,官至銀青光祿大夫、檢校左散騎常侍兼御史大夫,上柱國。



 史彥瓊者,為武德使,居鄴都,而魏博六州之政皆決彥瓊,自留守王正言而下,皆俯首承事之。是時,郭崇韜以無罪見殺于蜀,天下未知其死也,第見京師殺其諸子,因相傳曰:「崇韜殺魏王繼岌而自王於蜀矣,以故族其家。」鄴人聞之,方疑惑。已而朱友謙又見殺。友謙子廷徽為澶州刺史,有詔彥瓊使殺之,彥瓊秘其事,夜半馳出城。鄴人見彥瓊無故夜馳出,因驚傳曰:「劉皇后
 怒崇韜之殺繼岌也,已弒帝而自立,急召彥瓊計事。」鄴都大恐。貝州人有來鄴者,傳引語以歸。戍卒皇甫暉聞之,由此劫趙在禮作亂。在禮已至館陶,鄴都巡檢使孫鐸,見彥瓊求兵禦賊,彥瓊不肯與,曰:「賊未至,至而給兵豈晚邪?」已而賊至,彥瓊以兵登北門,聞賊呼聲,大恐,棄其兵而走,單騎歸于京師。在禮由是得入於鄴以成其叛亂者,由彥瓊啟而縱之也。



 郭門高者,名從謙,門高其優名也。雖以優進,而嘗有軍功,故以為從馬直指揮使。從馬直,蓋親軍也。從謙以姓郭,拜崇韜為叔父,而皇弟存乂又以從謙為養子。崇韜死,存乂見囚,從謙置酒軍
 中,憤然流涕,稱此二人之冤。是時,從馬直軍士王溫宿衛禁中,夜謀亂,事覺被誅。莊宗戲從謙曰:「汝黨存乂、崇韜負我,又教王溫反。復欲何為乎?」從謙恐,退而激其軍士曰:「罄爾之貲,食肉而飲酒,無為後日計也。」軍士問其故,從謙因曰:「上以王溫故,俟破鄴,盡坑爾曹。」



 軍士信之,皆欲為亂。李嗣源兵反,嚮京師,莊宗東幸汴州,而嗣源先入。莊宗至萬勝,不得進而還,軍士離散,尚有二萬餘人。居數日,莊宗復東幸汜水,謀扼關以為拒。四月丁亥朔,朝群臣於中興殿,宰相對三刻罷。從駕黃甲馬軍陣於宣仁門、步軍陣於五鳳門以俟。莊宗入食內殿,從謙
 自營中露刃注矢,馳攻興教門,與黃甲軍相射。莊宗聞亂,率諸王衛士擊亂兵出門。亂兵縱火焚門,緣城而入,莊宗擊殺數十百人。亂兵從樓上射帝,帝傷重,踣於絳霄殿廊下,自皇后、諸王左右皆奔走。



 至午時,帝崩,五坊人善友聚樂器而焚之。嗣源入洛,得其骨,葬新安之雍陵。以從謙為景州刺史,已而殺之。



 《傳》曰:「君以此始,必以此終。」莊宗好伶,而弒於門高,焚以樂器。可不信哉!可不戒哉!



\end{pinyinscope}