\article{卷二十三梁臣傳第十一}

\begin{pinyinscope}

 楊師厚楊師厚,潁州斤溝人也。少事河陽李罕之,罕之降晉,選其麾下勁卒百人獻于晉王,師厚在籍中。師厚在晉,無所知名,後以罪奔于梁,太祖以為宣武軍押衙、曹州刺史。梁攻王師範,師厚戰臨朐,擒其偏將八十餘人,取棣州,以功拜齊州刺史。太祖攻趙匡凝於襄陽,遣師厚為先鋒。師厚取穀城西童山木為浮橋,渡漢水,擊匡凝,敗之,匡凝棄城走。師厚進攻荊南,又走匡凝弟匡明,功
 為多,拜山南東道節度使、同中書門下平章事。



 劉知俊叛,攻陷長安,劉掞、牛存節等攻之,久不克。師厚以奇兵出,旁南山入其西門,降其守者,遂克之。晉周德威攻晉州以應知俊,師厚敗之于蒙坑,以功遷保義軍節度使,徙鎮宣義。是時,梁兵攻趙久無功,太祖病臥洛陽,少間,乃自將北擊趙。師厚從太祖至洹水,夜行迷失道,明旦,次魏縣,聞敵將至,梁兵潰亂不可止,久之無敵,乃定。已而太祖疾作,乃還。明年少間,而晉軍攻燕,燕王劉守光求援於梁,太祖為之擊趙以牽晉,屯於龍花,遣師厚攻棗彊,三月不能下。太祖怒,自往督兵戰,乃破,屠之,進圍
 蓚縣。晉史建瑭以輕兵夜擊梁軍,梁軍大擾,太祖與師厚皆棄輜重南走。太祖還東都,師厚留屯魏州。明年,太祖遇弒,友珪自立,師厚乘間殺魏牙將潘晏、臧延範等,逐出節度使羅周翰,友珪因以師厚為天雄軍節度使。



 自太祖與晉戰河北,師厚常為招討使,悉領梁之勁兵。太祖崩,師厚遂逐其帥,而稍矜倨難制。時魏恃牙兵,其帥得以倔彊。羅紹威時,牙兵盡死,魏勢孤,始為梁所制。師厚已得志,乃復置銀槍效範軍。友珪陰欲圖之,召師厚入計事。其吏田溫等勸師厚勿行,師厚曰:「吾二十年不負朱家,今若不行,則見疑而生事,然吾知上為人,
 雖往,無如我何也。」乃以勁兵二萬朝京師,留其兵城外,以十餘人自從,入見友珪,友珪益恐懼,賜與巨萬而還。



 已而末帝謀討友珪,問於趙巖,巖曰:「此事成敗,在招討楊公爾。得其一言諭禁軍,吾事立辦。」末帝乃遣馬慎交陰見師厚,布腹心。師厚猶豫未決,謂其下曰:「方郢王弒逆時,吾不能即討。今君臣之分已定,無故改圖,人謂我何?」其下或曰:「友珪弒父與君,乃天下之惡,均王仗大義以誅賊,其事易成。彼若一朝破賊,公將何以自處?」師厚大悟,乃遣其將王舜賢至洛陽,見袁象先計事,使朱漢賓以兵屯滑州為應。末帝卒與象先殺友珪。



 末帝即位,
 封師厚鄴王,詔書不名,事無巨細皆以諮之,然心益忌而畏之。已而師厚瘍發卒,末帝為之受賀於宮中。由是始分相、魏為兩鎮。魏軍亂,以魏博降晉,梁失河北自此始。



 王景仁王景仁,廬州合淝人也。初名茂章,少從楊行密起淮南。景仁為將驍勇剛悍,質略無威儀,臨敵務以身先士卒,行密壯之。梁太祖遣子友寧攻王師範于青州,師範乞兵於行密,行密遣景仁以步騎七千救師範。師範以兵背城為兩柵,友寧夜擊其一柵,柵中告急,趣景仁出戰,景仁按兵不動。友寧已破一柵,連戰不已。遲明,景仁度
 友寧兵已困,乃出戰,大敗之,遂斬友寧,以其首報行密。



 是時,梁太祖方攻鄆州,聞子友寧死,以兵二十萬倍道而至,景仁閉壘示怯,伺梁兵怠,毀柵而出,驅馳疾戰,戰酣退坐,召諸將飲酒,已而復戰。太祖登高望見之,得青州降人,問:「飲酒者為誰?」曰:「王茂章也。」太祖歎曰:「使吾得此人為將,天下不足平也!」梁兵又敗。景仁軍還,梁兵急追之,景仁度不可走,遣裨將李虔裕以眾一旅設覆於山下以待之,留軍不行,解鞍而寢。虔裕疾呼曰:「追兵至矣,宜速走,虔裕以死遏之!」景仁曰:「吾亦戰於此也。」虔裕三請,景仁乃行,而虔裕卒戰死,梁兵以故不能及,而景
 仁全軍以歸。



 景仁事行密,為潤州團練使。行密死,子渥自宣州入立,以景仁代守宣州。渥已立,反求宣州故物,景仁惜不與,渥怒,以兵攻之。景仁奔于錢鏐,鏐表景仁領宣州節度使。梁太祖素識景仁,乃遣人召之,景仁間道歸梁,仍以為寧國軍節度使,加同中書門下平章事。久之,未有以用,使參宰相班,奉朝請而已。



 開平四年,以景仁為北面招討使,將韓勍、李思安等兵伐趙;行至魏州,司天監言:「太陰虧,不利行師。」太祖亟召景仁等還,已而復遣之。景仁已去,太祖思術者言,馳使者止景仁於魏以待。景仁已過邢、洺,使者及之,景仁不奉詔,進營
 於柏鄉。乾化元年正月庚寅,日有食之,崇政使敬翔白太祖曰:「兵可憂矣!」



 太祖為之旰食。是日,景仁及晉人戰,大敗於柏鄉,景仁歸訴於太祖,太祖曰:「吾亦知之,蓋韓勍、李思安輕汝為客,不從節度爾。」乃罷景仁就第,後數月,悉復其官爵。



 末帝立,以景仁為淮南招討使,攻廬、壽,軍過獨山,山有楊行密祠,景仁再拜號泣而去。戰于霍山,梁兵敗走,景仁殿而力戰,以故梁兵不甚敗。景仁歸京師,病疽卒,贈太尉。



 賀瑰賀瑰,字光遠,濮州人也。事鄆州朱宣為都指揮使。梁太祖攻朱瑾于兗州,宣遣瑰與何懷寶、柳存等以兵萬人
 救兗州,瑰趨待賓館,欲絕梁餉道。梁太祖略地至中都,得降卒,言瑰等兵趨待賓館矣!以六壬占之,得「斬關」,以為吉,乃選精兵夜疾馳百里,期先至待賓以逆瑰,而夜黑,兵失道,旦至鉅野東,遇瑰兵,擊之,瑰等大敗。瑰走,梁兵急追之,瑰顧路窮,登塚上大呼曰:「我賀瑰也,可勿殺我!」



 太祖馳騎取之,并取懷寶等數十人,降其卒三千餘人。是日,大風揚沙蔽天,太祖曰:「天怒我殺人少邪?」即盡殺降卒三千人,而系瑰及懷寶等至兗城下以招瑾,瑾不納,因斬懷寶等十餘人,而獨留瑰。瑰感太祖不殺,誓以身自效。從太祖平青州,以為曹州刺史。太祖即位,
 累遷相州刺史。末帝時,遷左龍虎統軍,宣義軍節度使。



 貞明元年,魏兵亂,賀德倫降晉,晉王入魏州。劉掞敗于故元城,走黎陽,貝、衛、洺、磁諸州皆入于晉。晉軍取楊劉,末帝乃以瑰為招討使,與謝彥章等屯于行臺。晉軍迫瑰十里而柵,相持百餘日。瑰與彥章有隙,伏甲殺之,莊宗喜曰:「將帥不和,梁亡無日矣!」乃令軍中歸其老疾於鄴,以輕兵襲濮州。瑰自行臺躡之,戰于胡柳陂,晉人輜重在陣西,瑰軍將薄之,晉軍亂,斬其將周德威,盡取其輜重。



 軍已勝,陣無石山,日暮,晉兵仰攻之,瑰軍下山擊晉軍,瑰大敗,晉遂取濮州,城德勝,夾河為柵。瑰以舟兵
 攻南柵,不能得,還軍行臺,以疾卒,年六十二,贈侍中。有子光圖。



 王檀王檀,字眾美,京兆人也。少事梁太祖為小校,尚讓攻梁,戰尉氏門,檀勇出諸將,太祖奇之,遷踏白副指揮使。從朱珍募兵東方,戰數有功。梁與蔡兵戰板橋,李重裔馬踣,為蔡兵所擒,檀馳取之,并獲其將一人。從太祖破魏內黃,遷衝山都虞候。復從朱珍攻徐州,檀獲其將一人。梁兵攻王師範,檀以一軍破其密州,拜密州刺史。太祖即位,遷保義軍節度使,潞州東北面招討使。王景仁敗於柏鄉,晉兵圍邢州,太祖大懼,欲自將救之,檀止太祖,
 請自拒敵,力戰,卒全邢州,以功加同中書門下平章事,進封瑯琊郡王。友珪立,徙鎮宣化。貞明元年又徙匡國。是時,莊宗取魏博,檀以謂晉兵悉在河北,乃以奇兵西出陰地襲太原,不克而還。徙鎮天平,檀嘗招納亡盜居帳下,帳下兵亂,入殺檀,年五十八,贈太師,謚曰忠毅。



 馬嗣勳馬嗣勛,濠州鐘離人也,少事州為客將,為人材武有辯。梁太祖攻濠州,刺史張遂遣嗣勛持牌印降梁。楊行密攻遂,遂又使嗣勛乞兵於太祖。梁兵未至,濠州已沒,嗣勛無所歸,乃留事梁,太祖以為宣武軍元從押衙。太祖西攻鳳翔,行至華州,遣嗣勳入說韓建,建即時出降。天
 祐二年,羅紹威將誅牙軍,乞兵於梁,梁女嫁魏,適死,太祖乃遣嗣勳以長直千人為彩輿入魏,致兵器於輿中,聲言助葬。嗣勳館銅臺,夜與魏新鄉鎮兵攻石柱門,入迎紹威家屬,衛之。乃益取魏甲兵攻牙軍,牙軍不知兵所從來,莫能為備,殺其八千餘人,遲明皆盡。嗣勳中重瘡卒。太祖即位,贈太保。



 王虔裕王虔裕,瑯琊臨沂人也。為人健勇善騎射,以弋獵為生。少從諸葛爽起青、棣間,其後爽為汝州防禦使,率兵北擊沙陀,還入長安攻黃巢。爽兵敗降巢,巢以爽為河陽節度使。中和三年,孫儒陷河陽,虔裕隨爽奔于梁。是時,
 太祖新就鎮,黃巢、秦宗權等兵方盛,太祖數為所窘,而梁未有佗將,乃以虔裕將騎兵,常為先鋒擊巢陳、蔡間,拔其數柵。巢走,梁兵躡之,戰于萬勝戍,賊敗而東,虔裕功為多,乃表虔裕義州刺史。黃巢已去,秦宗權攻許、鄭,與梁為敵境,大小百餘戰,虔裕常有功。秦賢攻汴南境,太祖遣虔裕拒賢於尉氏,戰敗,失一裨將,太祖怒,拘虔裕於軍中。邢州孟遷降梁,為晉人所圍,太祖遣虔裕以精兵百人疾馳,夜破晉圍,入邢州,遲明,立梁旗幟於城上,晉人以為救兵至,乃退。已而晉兵復來,遷執虔裕降于晉,見殺。



 謝彥章謝彥章,許州人也。幼事葛從周,從周憐其敏惠,養以為子,授之兵法,從周以千錢置大盤中,為行陣偏伍之狀,示以出入進退之節,彥章盡得之。及壯,事梁太祖為騎將。是時,賀瑰善用步卒,而彥章與孟審澄、侯溫裕皆善將騎兵,審澄、溫裕所將不過三千,彥章多而益辦。彥章事末帝,累遷匡國軍節度使。貞明四年,晉攻河北,賀瑰為北面招討使,彥章為排陣使,屯於行臺。彥章為將,好禮儒士,雖居軍中,嘗儒服,或臨敵御眾,肅然有將帥之威,左右馳驟,疾若風雨。晉人望其行陣齊整,相謂曰:「謝彥章必在此也!」其名重敵中如此。瑰心忌之。彥章與瑰
 行視郊外,瑰指一地語彥章曰:「此地岡阜隆起,其中坦然,營柵之地也。」已而晉兵柵之,瑰疑彥章陰以告晉,益惡之。彥章故與馬步都虞候硃珪有隙,瑰欲速戰,彥章請持重以老敵,珪乃誣彥章以為將反。瑰旦享士,使珪伏甲殺之,審澄、溫裕皆見害。



\end{pinyinscope}