\article{卷四十一雜傳第二十九}

\begin{pinyinscope}

 盧光稠譚全播盧
 光稠、譚全播,皆南康人也。光稠狀貌雄偉,無佗材能,而全播勇敢有識略,然全播常奇光稠為人。唐末,群盜起南方,全播謂光稠曰:「天下洶洶,此真吾等之時,無徒守此貧賤為也!」乃相與聚兵為盜。眾推全播為主,全播曰:「諸君徒為賊乎?而欲成功乎?若欲成功,當得良帥,盧公堂堂,真君等主也。」眾陽諾之,全播怒,拔劍擊木三,斬之,曰:「不從令者如此木!」眾懼,乃立光稠為帥。



 是時,王
 潮攻陷嶺南,全播攻潮,取其虔、韶二州,又遣光稠弟光睦攻潮州。



 光睦好勇而輕進,全播戒其持重,不聽,度其必敗,乃為奇兵伏其歸路。光睦果敗走,潮人追之,全播以伏兵邀擊,大敗之,遂取潮州。是時,劉巖起南海,擊走光睦,以兵數萬攻虔州。光稠大懼,謂全播曰:「虔、潮皆公取之,今日非公不能守也。」全播曰:「吾知劉巖易與爾!」乃選精兵萬人,伏山谷中,陽治戰地於城南,告巖戰期。以老弱五千出戰,戰酣,偽北,巖急追之,伏兵發,巖遂大敗。光稠第戰功,全播悉推諸將,光稠心益賢之。



 梁初,江南、嶺表悉為吳與南漢分據,而光稠獨以虔、韶二州請命
 於京師,願通道路,輸貢賦。太祖為置百勝軍,以光稠為防禦使、兼五嶺開通使,又建鎮南軍,以為留後。



 開平五年,光稠病,以符印屬全播,全播不受。光稠卒,全播立其子延昌而事之。延昌好遊獵,其將黎求閉門拒延昌,延昌見殺。求因謀殺全播,全播懼,稱疾不出。求乃自立,請命於梁。乾化元年,拜求防禦使。求暴病死,其將李彥圖自立,全播益懼,遂稱疾篤,杜門自絕。彥圖疑之,使人覘其動靜,全播應覘為狀以自免。



 彥圖死,州人相率詣全播第,扣門請之,全播乃起,遣使請命于梁,拜防禦使。全播治虔州七年,有善政,楊隆演遣劉信攻破虔州,以全
 播歸廣陵,卒年八十五。當盧氏時,劉已取韶州,及全播被執,虔州遂入于吳。



 雷滿雷滿,武陵人也。為人兇悍獢勇,文身斷髮。唐廣明中,湖南饑,盜賊起,滿與同里人區景思、周岳等聚諸蠻數千,獵於大澤中,乃擊鮮釃酒,擇坐中豪者,補置伍長,號土團軍,諸蠻從之,推滿為帥。是時,高駢鎮荊南,召滿隸麾下,使以蠻軍擊賊。駢徙淮南,滿從至廣陵,逃歸,殺刺史崔翥,遂據朗州,請命於唐。昭宗以澧、朗為武貞軍,拜滿節度使。



 是時,澧陽人向瑰殺刺史呂自牧據澧州,而溪洞諸蠻宋鄴昌、師益等,皆起兵剽掠湖外,滿亦以輕舟
 上下荊江,攻劫州縣。楊行密攻杜洪于鄂州,荊南成汭出兵救洪,汭戰敗,溺死於君山。滿襲破荊南,不能守,焚掠殆盡而去。



 滿嘗鑿深池於府中,客有過者,召宴池上,指其水曰:「蛟龍水怪皆窟於此,蓋水府也。」酒酣,取坐上器擲池中,因裸而入,取器嬉水上,久之乃出,治衣復坐,意氣自若。



 滿居朗州,引沅水塹其城,上為長橋,為不可攻之計。天祐中,滿卒,子彥恭自立。彥恭附于楊行密,亦嘗攻劫為荊、湖患。開平元年,馬殷發兵攻彥恭,恃塹為阻,逾年不能破。三年,彥恭奔于楊行密,馬殷擒其弟彥雄等七人送于梁,斬于汴市,彥恭卒于淮南,澧、
 朗遂入于楚。



 鐘傳鐘傳,洪州高安人也。事州為小校,黃巢攻掠江淮,所在盜起,往往據州縣。



 傳以州兵擊賊,頻勝,遂逐觀察使,自稱留後。唐以洪州為鎮南軍,拜傳節度使。



 江夏伶人杜洪者,亦據鄂州,楊行密屢攻之,洪頗倚傳為首尾。久之,洪敗死。是時,危全諷、韓師德等分據撫、吉諸州,傳皆不能節度,以兵攻之,稍聽命,獨全諷不能下,乃自率兵圍之。城中夜火起,諸將請急攻之,傳曰:「吾聞君子不迫人之危。」乃掃地祭天,嚮城再拜,祝曰:「全諷不降,非民之罪,願天止火。」全諷聞之,明日乃亦聽命,請以女妻傳子
 匡時。傳居江西三十餘年,累拜太保、中書令,封南平王。天祐三年,傳卒,子匡時自稱留後,請命于唐。全諷曰:「聽鐘郎為節度使三年,吾將自為之。」已而傳養子延規與匡時爭立,乞兵於楊渥,渥遣秦裴等攻匡時,匡時敗,被執歸廣陵。開平三年,全諷等起兵江西,謀復鐘氏故地,全諷為楊隆演將周本所敗,江西遂入於吳。



 趙匡凝趙匡凝,字光儀,蔡州人也。其父德諲事秦宗權,為申州刺史。宗權反,德諲攻下襄陽。梁太祖攻蔡州,宗權屢敗,德諲乃以山南東道七州降。梁太祖初鎮宣武,嘗為宗權所困,聞德諲降,大喜,表為行營副都統,河陽、保義、義
 昌三節度行軍司馬。會其兵以攻蔡州,破之,德諲功多。德諲卒,子匡凝自立。是時,成汭死,雷彥恭襲取荊南,匡凝遣其弟匡明逐彥恭,太祖表匡凝荊襄節度使,以匡明為荊南留後。是時,唐衰,籓鎮不復奉朝廷,獨匡凝兄弟貢賦不絕。



 匡凝為人氣貌甚偉,性方嚴,喜自修飾,頗好學問,聚書數千卷,為政有威惠。



 太祖攻兗州,朱瑾求救於晉,晉遣史儼等將兵數千救瑾,瑾敗,與儼等奔于淮南。



 晉王李克用遣人以書幣假道於匡凝,以聘于楊行密,求歸儼等。晉王使者為梁得,太祖大怒。是時,梁已破兗、鄆,遣氏叔琮、康懷英等攻匡凝,叔琮取泌、隨二州,
 懷英取鄧州,匡凝懼,請盟,乃止。



 太祖弒昭宗,將謀代唐,畏匡凝兄弟不從,遣使告之,匡凝對使者流涕答曰:「受唐恩深,不敢妄有佗志。」太祖遣楊師厚攻之,太祖以兵殿漢北,匡凝戰敗,以輕舟奔於楊行密。師厚進攻荊南,匡明奔於蜀。匡凝至廣陵,行密見之,戲曰:「君在鎮時,輕車重馬,歲輸於梁,今敗乃歸我乎?」匡凝曰:「僕世為唐臣,歲時職貢,非輸賊也。今以不從賊之故,力屈歸公,惟公生死之耳!」行密厚遇之。



 其後行密死,楊渥稍不禮之。渥方宴,食青梅,匡凝顧渥曰:「勿多食,發小兒熱。」



 諸將以為慢,渥遷匡凝海陵,後為徐溫所殺。匡明卒於蜀。



\end{pinyinscope}