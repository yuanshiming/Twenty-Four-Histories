\article{卷四十二雜傳第三十}

\begin{pinyinscope}

 朱宣弟瑾硃宣,宋州下邑人也。少從其父販鹽為盜,父抵法死,宣乃去事青州節度使王敬武為軍校,敬武以隸其將曹全晟。中和二年,敬武遣全晟入關與破黃巢。還過鄆州,鄆州節度使薛崇卒,其將崔君預自稱留後。全晟攻殺君預,遂據鄆州。宣以戰功,為鄆州馬步軍都指揮使。已而全晟死,軍中推宣為留後,唐僖宗即拜宣天平軍節度使。



 梁太祖鎮宣武,以兄事宣。太祖新就鎮,兵力尚少,
 數為秦宗權所困,太祖乞兵於宣。宣與其弟瑾以兗、鄆之兵救汴,大破蔡兵,走宗權。是時,太祖已襲取滑州,稍欲并吞諸鎮,宣、瑾既還,乃馳檄兗、鄆,言宣、瑾多誘宣武軍卒亡以東,乃發兵收亡卒,因攻之,遂為敵國,苦戰曹、濮間。是時,梁又東攻徐州,西有蔡賊,北敵強晉,宣、瑾兄弟自相首尾,然卒為梁所滅。乾寧四年,宣敗,走中都,為葛從周所執,斬于汴橋下。



 瑾,宣從父弟也。從宣居鄆州,補軍校。少倜儻,有大志,兗州節度使齊克讓愛其為人,以女妻之。瑾行親迎,乃選壯士為輿夫,伏兵器輿中。夜至兗州,兵發,遂虜克讓,自
 稱留後。僖宗即拜瑾泰寧軍節度使。



 瑾與宣已破秦宗權於汴州,梁太祖責瑾誘宣武軍卒以歸,遣朱珍攻瑾,取曹州,又攻濮州,而太祖自攻鄆。瑾兄弟往來相救,凡十餘年,大小數十戰,與太祖屢相勝敗。太祖得宣將賀瑰、何懷寶及瑾兄瓊,乃將瓊等至兗城下,告瑾曰:「汝兄敗矣!今瓊等已降,不如早自歸。」瑾偽曰:「諾。」乃遣牙將胡規持書幣詣軍門請降。太祖大喜,至延壽門與瑾交語,瑾曰:「願得瓊送符印。」太祖信之,遣客將劉捍送瓊往。瑾伏壯士橋下,單騎迎瓊,揮手語捍曰:「請瓊獨來!」瓊前,壯士擒之,遂閉門,責瓊先降,斬之,擲其首城外。太祖度
 不可下,乃留兵圍之而去。



 瑾嬰城自守,而與葛從周等戰城下,瑾兵屢敗,宣亦敗於鄆州,乃乞兵於晉,晉遣李承嗣、史儼等以騎兵五千救之。太祖已破宣,乃急趨兗。瑾城中食盡,與承嗣等掠食豐、沛間,梁兵奄至,瑾將康懷英等以城降梁。瑾等將麾下兵走沂州,沂州刺史尹處賓不納。又走海州,梁兵急追之,乃奔於淮南。楊行密聞瑾來,大喜,解其玉帶贈之,表瑾領武寧軍節度使,以為行軍副使。其後,梁遣龐師古、葛從周等攻淮南,行密用瑾,大破梁兵於清口,斬師古。行密累表瑾東南諸道行營副都統、領平盧軍節度使、同中書門下平章事。



 行
 密死,渥及隆演相繼立,皆年少,徐溫與其子知訓專政,畏瑾,欲除之,瑾乃謀殺知訓。嘗以月旦遣愛妾候知訓家,知訓強通之,妾歸自訴,瑾益不平。屢勸隆演誅徐氏,以去國患,隆演不能為。既而知訓以泗州建靜淮軍,出瑾為節度使。



 將行,召之夜飲。明日,知訓過瑾謝,延之升堂,出其妻陶氏,知訓方拜,瑾以笏擊踣之,伏兵自戶突出,殺之。初,瑾以二惡馬擊庭中,知訓入而釋馬,使相踶嗚,故外人莫聞其變。瑾攜其首馳示隆演曰:「今日為吳除患矣!」隆演曰:「此事非吾敢知!」遽起入內。瑾忿然以首擊柱,提劍而出,府門已闔,因踰垣,折其足。



 瑾顧路窮,大
 呼曰:「吾為萬人去害,而以一身死之!」遂自刎。潤州徐知誥聞亂,以兵趨廣陵,族瑾家。瑾妻陶氏臨刑而泣,其妾曰:「何為泣乎?今行見公矣!」



 陶氏收淚,欣然就戮,聞者哀之。



 瑾名重江淮,人畏之。其死也,尸之廣陵北門,路人私共瘞之。是時,民多病瘧,皆取其墓上土,以水服之,云病輒愈,更益新土,漸成高墳。徐溫等惡之,發其尸,投於雷公塘。後溫病,夢瑾挽弓射之。溫懼,網其骨,葬塘側,立祠其上。



 初,瑾嘗病疽,醫者視之,色懼,瑾曰:「但理之,吾非以病死者。」於是果然。



 卒年五十二。



 王師範王師範,青州人也。其父敬武,為平盧軍牙將。唐廣明元
 年,無棣人洪霸郎為盜齊、棣間,平盧節度使安師儒遣敬武率兵擊破之。敬武反,兵逐師儒,自稱留後,都統王鐸承制拜敬武節度使。敬武卒,師範尚幼,其棣州刺史張蟾叛。昭宗以為師範年少,其下不服從,乃拜太子少師崔安潛為平盧節度使。師範不受代,蟾迎安潛入棣州。師範遣其將盧洪攻蟾,洪以兵返襲青州,師範陽為好辭,遣人迎語洪曰:「吾幼未能任事,賴諸將共持之爾。不然,聽公所為也。」洪以師範無能為,遽還,不為備。師範伏兵於道,語其僕劉鄩曰:「洪來,為我斬之!用爾為牙將。」明日,洪來,師範出迎,鄩於坐上斬之,伏兵發,盡
 殺其餘兵,乃急攻棣州,破張蟾,安潛奔歸于京師。昭宗乃拜師範節度使。



 師範頗好儒學,聚書至萬卷,為政有威愛。梁太祖圍昭宗於鳳翔,宦官韓全誨等矯詔召諸鎮兵以擊梁。詔至青州,師範泣曰:「諸鎮有兵,所以籓扞天子,今天子危辱,而諸鎮反以兵自衛;吾雖力不足,當成敗以之。」乃遣使乞兵於楊行密。



 是時,梁已東下兗、鄆,師範乃遣劉鄩與其弟師魯分攻兗、密諸州。遣張居厚以壯士二百為輿夫,伏兵輿中,西馳梁軍,稱師範使者聘梁,因欲劫殺太祖。居厚至華州東城,華州將婁敬思疑其有異,剖輿視之,見其兵。居厚遂擊殺敬思,以兵攻
 西城,不克而反。劉鄩逐葛從周取兗州,而平盧諸州皆起兵攻梁。



 其後,梁太祖自鳳翔東還,遣朱友寧攻師範,友寧戰死。復遣楊師厚攻之,屯于臨朐。師範以兵迫之,師厚陽為怯不敢出,間遣人陽言曰:「梁兵少,方乞兵於鳳翔,今糧且絕,當還軍。」師範以為然,乃遣師魯悉兵攻之,師厚拒而不戰。師魯兵卻,師厚追擊至聖王山,師魯大敗,遂傅其城,而梁別將劉重霸下其棣州,師範乃請降,太祖許之。師範素服乘驢詣太祖請罪,太祖待以客禮。久之,表師範河陽節度使。



 太祖即位,召為右金吾衛上將軍,居于洛陽。太祖心欲誅之,未有以發。太祖諸子
 已封王,宴於宮中,友寧妻泣謂太祖曰:「陛下化家為國,諸子人人皆得封,而妾夫獨以戰死,奈何仇人猶在朝廷!」太祖奮然戟手曰:「吾亦幾忘此賊!」乃遣人就洛陽族滅之。使者至,先掘坑於外,乃入告之。師範設席為具,與諸宗族飲酒,謂使者曰:「死,人之所不免,況有罪乎?然懼少長失序,下愧於先人。」酒半,令少長以次起,就戮於坑所,聞者皆哀憐之。同光三年,贈師範太尉。



 李罕之李罕之,陳州項城人也。為人驍勇,力兼數人。少學,讀書不成,去為僧,以其無賴,所往皆不容。乃乞食酸棗市中,市中人皆不與,罕之擲器于地,裂其衣,又去為盜。是
 時,黃巢起曹、濮,乃往依之。巢北渡江,罕之與其麾下走淮南,自歸於高駢,駢表光州刺史。歲餘,秦宗權急攻光州,罕之不能守,還走項城,收其餘眾,依諸葛爽於河陽,爽以罕之為懷州刺史。巢已敗走,爽降唐,僖宗拜爽東南面招討使,以攻宗權,爽表罕之副使,以兵屯宋州,又表河南尹、東都留守。秦宗權遣孫儒攻河南,罕之兵少,西走澠池,儒燒宮闕,剽掠而去。罕之壁澠池。



 歲餘,諸葛爽死,其將劉經立爽子仲方。仲方年少,事皆任經,經慮罕之兇勇難制,以兵攻之,罕之返擊走經。罕之追至鞏縣,陳舟于汜水,將渡河,經遣張言拒之河上,言反背
 經,與罕之合攻河陽,為經所敗,退保懷州。已而孫儒陷河陽,仲方奔于梁。梁兵擊走儒,罕之襲取河陽,言取河南,皆附于梁。



 罕之與言皆爽叛將,事已成,乃相與交臂為盟,誓同休戚不相忘。罕之御眾無法,性苛暴,頗失士心。而言善治軍旅,教民播殖,務為積聚。罕之用兵,言嘗供給其乏。罕之求取無已,言頗苦之,不能輸,罕之召言軍吏笞責之,言益不平。罕之悉兵攻晉、絳,言夜襲河陽,罕之奔晉。晉表罕之澤州刺史,使李存孝以兵三萬助罕之攻言。言求救於梁。罕之敗於沇河,乃歸太原,李克用延之帳中。罕之留其子頎事晉,乃之澤州,日以兵鈔
 懷、孟間,啖人為食。居民屯聚摩雲山,罕之悉攻殺之,立柵其上,時人號曰李摩雲。是時,晉方徇地山東,頗倚罕之為扞蔽。李茂貞等犯京師,克用以兵至渭北,僖宗以克用為邠州四面行營都統,表罕之為副。破王行瑜,加檢校太尉,食邑千戶。



 罕之自以功多於晉,私謂蓋寓曰:「自吾脫身河陽,賴晉容我,未能有以報之;今行老矣,無能為也。若吾王見憐,與一小鎮,使休兵養疾而後歸老,幸也!」寓為言之,克用不對。佗日,諸鎮擇守將,未嘗及罕之,罕之心益怏怏。寓告克用,懼罕之有佗心,克用曰:「吾於罕之,豈惜一鎮,然鷹鳥之性,飽則揚矣!」



 光化元年,
 潞州薛志勤卒,罕之遽入潞州,使人啟晉王曰:「志勤且死,新帥未至,所以然者,備佗盜耳!」克用大怒,遣李嗣昭攻之。罕之執晉守將馬溉、伊鐔等,遣子顥送于梁以乞兵。梁太祖遣丁會守潞州,以罕之為河陽節度使,行至懷州,以疾卒,年五十八。



 罕之初背梁而歸晉,晉王以罕之守澤州,罕之留其子頎與莊宗遊,甚狎。後罕之背晉以歸梁,晉王怒,欲殺頎,莊宗與之駿馬,使奔于梁。太祖得頎父子大喜,使與友倫將兵以衛昭宗,故頎當太祖時,常掌禁兵。末帝誅友珪,頎與其謀,拜右羽林統軍、澶州刺史。事唐,歷衛、衍二州刺史,累遷右領軍衛上將
 軍。天福中卒,年七十,贈太尉。



 孟方立孟方立,邢州平鄉人也。少為軍卒,以勇力選為隊將。唐廣明中,潞州節度使高潯攻諸葛爽于河陽,遣方立將兵出天井關為先鋒。潯為其將劉廣所逐,廣為亂軍所殺。方立聞亂,引兵自天井入據潞州,唐因以為昭義軍節度使。昭義所節制澤、潞、邢、洺、磁五州,而治潞州。方立以謂潞州山川高險,而人俗勁悍,自劉積以來嘗逐其帥;且己邢人也,因徙其軍于邢州。而潞人怨方立之徙也,因以澤、潞二州歸于晉。晉遣李克修為澤潞節度使,方立以邢、洺、磁三州自為昭義軍。



 晉數遣李存孝等出
 兵以窺山東,三州之人俘掠殆盡,赤地數千里,無復耕桑者累年。方立以孤城自守,求救于梁,梁方東事兗、鄆,不能救也。文德元年,方立乞兵于王鎔以攻晉,鎔許之。方立乃遣其將奚忠信攻晉遼州,而鎔以佗故不能出兵。



 兵既失約,忠信大敗,而晉兵乘勝攻之。



 方立將石元佐者,善兵而多智,方立嘗信用之。忠信之敗也,元佐為晉將安金俊所得,金俊厚遇之,問以攻邢之策,元佐曰:「方立善守而邢城堅,若攻之,必不得志。宜急攻其磁州,方立來救,可以敗也。」金俊以為然。軍於滏水之西,方立果帥兵來救,為金俊所敗,馳入邢州,閉壁不復出。外無
 救兵,城中食且盡,方立夜出巡城,號令守者,守者皆不應,方立知不可,乃歸飲鴆而卒。



 軍中以其弟洺州刺史遷為留後,求救於梁。梁太祖遣王虔裕將騎兵三百助遷守,遷執虔裕降晉。晉徙遷族于太原,以為汾州刺史,後以為澤潞節度使。天復元年,梁遣氏叔琮攻晉,出天井關,遷開門降,為梁兵鄉道以攻太原,不克。叔琮軍還過潞,以遷歸于梁,梁太祖惡其返覆,殺之。



 王珂王珂,河中人也。其仲父重榮,以河中兵破黃巢,有功於唐,拜河中節度使。



 重榮無子,以其兄重簡子珂為後。重榮卒,弟重盈立,重盈卒,軍中乃以珂重榮子,立之。重盈
 子陜州節度使珙、絳州刺史瑤,與珂爭立,珙、瑤以書與梁太祖,言珂故王氏蒼頭,小字忠兒,不應得立。珂亦求援於晉,晉人言之朝,昭宗以晉故,許之。而珙、瑤亦西結王行瑜、韓建、李茂貞為援,行瑜等交章論列,昭宗報以重榮與晉於唐嘗有大功,業許之,不可易。行瑜等怒,以兵犯京師,殺宰相李磎等而去。



 珙、瑤連兵攻珂河中,珂求援於晉,晉兵西討三鎮,行下絳州,斬瑤而過,至于渭北,擊破行瑜。昭宗卒以珂為河中節度使。晉以女妻之,遣李嗣昭將兵助珂攻珙陜州。珙為人慘刻,嘗斬人擲其首於前,言笑自若,其下苦之。偏將李璠因珙戰敗,殺
 珙,自稱留後。



 是時,梁已下鎮、定,將移兵西,而昭宗為劉季述所廢,京師大亂。崔胤陰召梁以兵西,梁太祖以珂在河中,懼為患,乃顧張存敬、侯言,以一大繩與之,曰:「為我持縛珂來!」存敬等兵出含山,破晉、絳二州,遣何絪以兵守之,絕晉援。



 存敬圍河中,珂告急於晉,晉以絪故不得前。珂乃遣其妻以書告晉王曰:「賊勢如此,朝夕乞食於梁矣!大人何忍而不救邪?」晉王報之曰:「梁兵為阻,眾寡不敵,救之則并晉俱亡,不若與王郎自歸朝廷。」珂乃為書與李茂貞曰:「天子初返正,詔籓鎮無相侵以安王室。今朱公棄約以見攻,其勢不止於弊邑;若弊邑朝
 亡,則西北諸鎮非諸君所能守也!願與華州出兵潼關以為應。」茂貞不報。珂計窮,乃治舟于河,將歸于京師。珂夜登城諭守陴者,守陴者皆不應。牙將劉訓夜入珂寢白事,珂叱之曰:「兵欲反邪!」訓乃解衣自索而入曰:「公茍懷疑,請先斷臂!」珂曰:「事急矣!計安出乎?」訓曰:「公若攜家夜濟,人必爭舟,一夫鴟張,大事即去。



 不若遲明以情諭軍中,願從者猶得其半。不然,且為款狀以緩梁兵,徐圖向背。」



 珂以為然。



 梁太祖自同州降唐,即依重榮,以母王氏,故事重榮為舅。珂乃登城呼存敬曰:「吾於梁王有家世之舊,兵當退舍,俟梁王來,吾將聽命。」存敬乃退舍,使
 馳詣太祖於洛陽。太祖至河中,先之城東,哭於重榮之墓而後入。珂欲面縛牽羊以見太祖,太祖謂曰:「太師阿舅之恩何時可忘,郎君若以亡國之禮見,太師其謂我何?」



 珂迎於路,握手噓唏,乃徙珂於汴。太祖以珂晉婿也,疑其貳己,使珂西入覲,行至華州,使人殺之傳舍。



 瓚,重盈之諸子也,梁太祖已執珂,自領河中節度使,以瓚為吏。瓚事梁,為諸衛大將軍,泰寧、鎮國軍節度使。末帝時,為開封尹。貞明五年,代賀瑰為北面行營招討使。是時,晉已城德勝,瓚自黎陽渡河攻澶州,不克,退屯楊村,扼河上流,與晉人相持經年,大小百餘戰,瓚卒無
 功,末帝遣戴思遠代,瓚復為開封尹。



 莊宗自鄆入京師,末帝聞唐兵且至,日夜涕泣,不知所為,自持國寶,指其宮室謂瓚曰:「使吾保此者,繫卿之畫如何耳!」唐兵已過宛朐,瓚驅率市人登城拒守。



 唐兵攻封丘門,瓚開門迎降,伏地請死,莊宗勞而起之曰:「朕與卿家世婚姻,然人臣各為主耳,復何罪邪!」因以為開封尹,遷宣武軍節度使。已而故梁臣趙巖、張漢傑等相次誅死,瓚以憂卒。贈太子太師。



 趙犨趙犨,其先青州人也。世為陳州牙將。犨幼與群兒戲道中,部分行伍,指顧如將帥,雖諸大兒皆聽其節度,其父
 叔文見之,驚曰:「大吾門者,此兒也!」及壯,善用弓劍,為人勇果,重氣義,刺史聞其材,召置麾下。累遷忠武軍馬步軍都虞候。



 王仙芝寇河南,陷汝州,將犯東都,犨引兵擊敗之,仙芝乃南去。已而黃巢起,所在州縣,往往陷賊。陳州豪傑數百人,相與詣忠武軍,求得犨為刺史以自保,忠武軍表犨陳州刺史。已而巢陷長安,犨語諸將吏曰:「以吾計,巢若不為長安市人所誅,必驅其眾東走,吾州適當其衝矣!」乃治城池為守備,遷民六十里內者皆入城中,選其子弟,配以兵甲,以其弟昶、珝為將。巢敗,果東走,先遣孟楷據項城,昶擊破之,執楷以歸。巢從後至,聞
 楷被執,大怒。



 既而秦宗權以蔡州附巢,巢勢甚盛,乃悉眾圍犨,置舂磨,糜人之肉以為食。



 陳人恐,犨語其下曰:「吾家三世陳將,必能保此。爾曹男子,當於死中求生,建功立業,未必不因此時。」陳人皆踴躍。巢柵城北三里為八仙營,起宮闕,置百官,聚糧餉,欲以久弊之,其兵號二十萬。陳人舊有巨弩數百,皆廢壞,後生弩工皆不識其器。珝創意理之,弩矢激五百步,人馬皆洞,以故巢不敢近。圍凡三百日,犨食將盡,乃乞兵於梁。梁太祖與李克用皆自將會陳,擊敗巢將黃鄴于西華。西華有積粟,巢恃以為餉,及鄴敗,巢乃解圍去。



 梁太祖入陳州,
 犨兄弟迎謁馬首甚恭。然犨陰識太祖必成大事,乃降心屈迹,為自託之計。以梁援己恩,為太祖立生祠,朝夕拜謁。以其子巖尚太祖女,是謂長樂公主。黃巢已去,秦宗權復亂淮西,陷旁二十餘州,而陳去蔡最近,犨兄弟力拒之,卒不能下。後巢、宗權皆敗死,唐昭宗即以陳州為忠武軍,拜犨節度使。犨已病,乃以位與其弟昶,後數月卒。



 昶乘大寇新滅,乃休兵課農,事梁尤謹。梁兵攻戰四方,昶饋輓供億,未嘗少懈。昶卒,珝代立。



 珝頗知書,乃求鄧艾故迹,決翟王陂溉民田。兄弟居陳二十餘年,陳人大賴之。



 梁太祖已降韓建,取同、華,徙珝為同州留後。
 入唐,為右金吾衛上將軍。歲餘,以疾免官歸,卒于家,陳人為之罷市。



 犨次子巖,梁末帝時為戶部尚書、租庸使,與張漢傑、漢倫等居中用事。梁自太祖以暴虐殺戮為事,而末帝為人特和柔恭謹,然性庸愚,以漢傑婦家,而巖婿也,故親信之,大臣老將皆切齒,末帝獨不悟,以至於亡。



 初,友珪殺太祖自立,以末帝為東都留守。巖如東都,末帝與之飲酒,從容以誠款告之。巖為末帝謀,遣人召楊師厚兵起事。巖還西都,卒與袁象先以禁兵誅友珪,取傳國寶以授末帝。



 末帝立,巖自以有功於梁,又尚公主,聞唐駙馬杜悰位至將相,自奉甚豐,恥其
 不及。乃占天下良田大宅,裒刻商旅,其門如市,租庸之物,半入其私,巖飲食必費萬錢。



 故時,魏州牙兵驕,數為亂,羅紹威盡誅之。太祖崩,楊師厚逐羅氏,據魏州,復置牙兵二千,末帝患之。師厚死,巖與租庸判官邵贊議曰:「魏為唐患,百有餘年,自先帝時,嘗切齒紹威,以其前恭而後倨。今先帝新棄天下,師厚復為陛下憂,所以然者,以魏地大而兵多也。陛下不以此時制之,寧知後人不為師厚也?不若分相、魏為兩鎮,則無北顧之憂矣。」末帝以為然,乃分相、澶、衛為昭德軍。牙兵亂,以魏博降晉,梁由是盡失河北。



 是時,梁將劉鄩等與莊宗相距澶、
 魏之間,兵數敗。巖曰:「古之王者必郊祀天地,陛下即位猶未郊天,議者以為朝廷無異籓鎮,如此何以威重天下?今河北雖失,天下幸安,願陛下力行之。」敬翔以為不可,曰:「今府庫虛竭,箕斂供軍,若行效禋,則必賞賚;是取虛名而受實弊也。」末帝不聽,乃備法駕幸西京,而莊宗取楊劉,或傳:「晉兵入東都矣!」或曰:「扼汜水矣!」或曰:「下鄆、濮矣!」



 京師大風拔木,末帝大懼,從官相顧而泣,末帝乃還東都,遂不果郊。



 鎮州張文禮殺王鎔,使人告梁曰:「臣已北召契丹,願梁以兵萬人出德、棣州,則晉兵憊矣。」敬翔以為然,巖與漢傑皆以為不可,乃止。其後黜王彥章
 用段凝,皆巖力也。



 莊宗兵將至汴,末帝惶惑不知所為,登建國樓以問群臣,或曰:「晉以孤軍遠來,勢難持久,雖使入汴,不能守也。宜幸洛陽,保險以召天下兵,徐圖之,勝負未可知也。」末帝猶豫,巖曰:「勢已如此,一下此樓,何人可保!」末帝卒死於樓上。



 當巖用事時,許州溫韜尤曲事巖,巖因顧其左右曰:「吾常待韜厚,今以急投之,必不幸吾為利。」乃走投韜,韜斬其首以獻。莊宗已滅梁,巖素所善段凝奏請誅巖家屬,乃族滅之。



 嗚呼,禍福之理,豈可一哉!君子小人之禍福異也。老子曰:「禍兮福所倚,福兮禍所伏。」後世之談禍福者,皆以其
 言為至論也。夫為善而受福,焉得禍?為惡而受禍,焉得福?惟君子之罹非禍者,未必不為福;小人求非福者,未嘗不及禍,此自然之理也。始,犨自以先見之明,深結梁太祖,及其子孫皆享其祿利,自謂知所託矣,安知其族卒與梁俱滅也?犨之求福於梁,蓋老氏之所謂福也,非君子之所求也,不可戒哉!



 馮行襲馮行襲,字正臣,均州人也。唐末,山南盜孫喜以眾千人襲均州刺史呂燁,燁不能禦。行襲為州校,乃陰選勇士伏江南,獨乘小舟逆喜,告曰:「州人聞公至,皆欲歸矣。然知公兵多,民懼虜掠,恐其驚擾,請留兵江北,獨與腹心
 數人從行,願為前導,以慰安州民,事可立定。」喜以為然,乃留其兵江北,獨與行襲渡江。



 軍吏前謁,行襲擊喜仆地,斬之,伏兵發,盡殺從行者。餘兵在江北,聞喜死,皆潰。山南節度使劉巨容表行襲均州刺史。



 是時,僖宗在蜀,諸鎮貢獻行在者皆道山南,盜賊多據州西長山以邀劫之,行襲盡破諸賊。洋州葛佐辟行襲行軍司馬,使以兵鎮谷口,通秦、蜀道,行襲由此知名。李茂貞兼領山南,遣子繼臻守金州,行襲逐之,遂據金州。昭宗乃以金州為戎昭軍,拜行襲節度使。昭宗在岐,梁太祖引兵而西,中尉韓全誨遣中官卻文晏等二十餘人召兵江淮,以
 拒太祖,行襲已附梁,乃盡殺文晏等。太祖攻趙匡凝於襄陽,行襲遣子勖以舟兵會均、房,以功遷匡國軍節度使。行襲為人嚴酷少恩,而所至輒天幸,境旱有蝗,則飛鳥食之,歲兇,田中鹵穀自生。唐衰,知梁必興,尤盡心傾附事梁,官至司空,封長樂郡王,卒贈太傅,謚曰忠敬。



\end{pinyinscope}