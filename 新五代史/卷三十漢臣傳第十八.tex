\article{卷三十漢臣傳第十八}

\begin{pinyinscope}

 蘇逢吉蘇逢吉,京兆長安人也。漢高祖鎮河東,父悅為高祖從事,逢吉常代悅作奏記,悅乃言之高祖。高祖召見逢吉,精神爽秀,憐之,乃以為節度判官。高祖性素剛嚴,賓佐稀得請見,逢吉獨入,終日侍立高祖書閣中。兩使文簿盈積,莫敢通,逢吉輒取內之懷中,伺高祖色可犯時以進之,高祖多以為可,以故甚愛之。然逢吉為人貪詐無行,喜為殺戮。高祖嘗以生日遣逢吉疏理獄囚以祈福,
 謂之「靜獄。」逢吉入獄中閱囚,無輕重曲直悉殺之,以報曰:「獄靜矣。」



 高祖建號,拜逢吉中書侍郎、同中書門下平章事。是時,制度草創,朝廷大事皆出逢吉,逢吉以為己任。然素不學問,隨事裁決,出其意見,是故漢世尤無法度,而不施德政,民莫有所稱焉。高祖既定京師,逢吉與蘇禹珪同在中書,除吏多違舊制。逢吉尤納貨賂,市權鬻官,謗者喧嘩。然高祖方倚信二人,故莫敢有告者。鳳翔李永吉初朝京師,逢吉以永吉故秦王從嚴子,家世王侯,當有奇貨,使人告永吉,許以一州,而求其先王玉帶,永吉以無為解,逢吉乃使人市一玉帶,直數千緡,責
 永吉償之;前客省使王筠自晉末使楚,至是還,逢吉意筠得楚王重賂,遣人求之,許以一州,筠怏怏,以其橐裝之半獻之。而皆不得州。



 晉相李崧從契丹以北,高祖入京師,以崧第賜逢吉,而崧別有田宅在西京,逢吉遂皆取之。崧自北還,因以宅券獻逢吉,逢吉不悅,而崧子弟數出怨言。其後,逢吉乃誘人告崧與弟嶼、義等,下獄,崧款自誣伏:「與家僮二十人,謀因高祖山陵為亂。」獄上中書,逢吉改「二十人」為「五十人」,遂族崧家。



 是時,天下多盜,逢吉自草詔書下州縣,凡盜所居本家及鄰保皆族誅。或謂逢吉曰:「為盜族誅,已非王法,況鄰保乎!」逢吉甗以
 為是,不得已,但去族誅而已。於是鄆州捕賊使者張令柔盡殺平陰縣十七村民數百人。衛州刺史葉仁魯聞部有盜,自帥兵捕之。時村民十數共逐盜,入于山中,盜皆散走。仁魯從後至,見民捕盜者,以為賊,悉擒之,斷其腳筋,暴之山麓,宛轉號呼,累日而死。聞者不勝其冤,而逢吉以仁魯為能,由是天下因盜殺人滋濫。



 逢吉已貴,益為豪侈,謂中書堂食為不可食,乃命家廚進羞,日極珍善。繼母死,不服喪。妻武氏卒,諷百官及州鎮皆輸綾絹為喪服。武氏未期,除其諸子為官。



 有庶兄自外來,未白逢吉而見其諸子,逢吉怒,託以佗事告於高祖,杖殺
 之。



 逢吉嘗從高祖征鄴,數使酒辱周太祖於軍中,太祖恨之。其後隱帝立,逢吉素善李濤,諷濤請罷太祖與楊邠樞密。李太后怒濤離間大臣,罷濤相,以楊邠兼平章事,事悉關決。逢吉、禹珪由是備位而已。乾祐二年,加拜司空。



 周太祖鎮鄴,不落樞密使,逢吉以謂樞密之任,方鎮帶之非便,與史弘肇爭,於是卒如弘肇議。弘肇怨逢吉異己,已而會飲王章第,使酒坐中,弘肇怒甚。逢吉謀求出鎮以避之,既而中輟,人問其故,逢吉曰:「茍捨此而去,史公一處分,吾齏粉矣!」



 是時,隱帝少年,小人在側。弘肇等威制人主,帝與左右李業、郭允明等皆患之。逢吉
 每見業等,以言激之,業等卒殺弘肇,即以逢吉權知樞密院。方命草麻,聞周太祖起兵,乃止。逢吉夜宿金祥殿東閣,謂司天夏官正王處訥曰:「昨夕未暝,已見李崧在側,生人接死者,無吉事也。」周太祖至北郊,官軍敗于劉子陂。逢吉宿七里,夜與同舍酣飲,索刀將自殺,為左右所止。明日與隱帝走趙村,自殺於民舍。周太祖定京師,梟其首,適當李崧被刑之所。廣順初,賜其子西京莊并宅一區。



 史弘肇史弘肇,字化元,鄭州滎澤人也。為人驍勇,走及奔馬。梁末,調民七戶出一兵,弘肇為兵,隸開道指揮,選為禁兵。
 漢高祖典禁兵,弘肇為軍校。其後,漢高祖鎮太原,使將武節左右指揮,領雷州刺史。高祖建號於太原,代州王暉拒命,弘肇攻破之,以功拜忠武軍節度使、侍衛步軍都指揮使。



 是時契丹北歸,留耿崇美攻王守恩於潞州。高祖遣弘肇前行擊之,崇美敗走,守恩以城歸漢。而河陽武行德、澤州翟令奇等,皆迎弘肇自歸。弘肇入河陽,高祖從後至,遂入京師。



 弘肇為將,嚴毅寡言,麾下嘗少忤意,立楇殺之,軍中為股心慄,以故高祖起義之初,弘肇行兵所至,秋亳無犯,兩京帖然。遷侍衛親軍馬步軍都指揮使,領歸德軍節度使、同中書門下平章事。高祖
 疾大漸,與楊邠、蘇逢吉等同授顧命。



 隱帝時,河中李守貞、鳳翔王景崇、永興趙思綰等皆反,關西用兵,人情恐懼,京師之民,流言以相驚恐。弘肇出兵警察,務行殺戮,罪無大小皆死。是時太白晝見,民有仰觀者,輒腰斬于市。市有醉者忤一軍卒,誣其訛言,坐棄市。凡民抵罪,吏以白弘肇,但以三指示之,吏即腰斬之。又為斷舌、決口、斮筋、折足之刑。李崧坐奴告變族誅,弘肇取其幼女以為婢。於是前資故將失職之家,姑息僮奴,而廝養之輩,往往脅制其主。侍衛孔目官解暉狡酷,因緣為姦,民抵罪者,莫敢告訴。



 燕人何福進有玉枕,直
 錢十四萬,遣僮賣之淮南以鬻茶。僮隱其錢,福進笞責之,僮乃誣告福進得趙延壽玉枕,以遺吳人。弘肇捕治,福進棄市,帳下分取其妻子,而籍其家財。弘肇不喜賓客,嘗言:「文人難耐,呼我為卒。」



 弘肇領歸德,其副使等月率私錢千緡為獻。潁州麴場官麴溫與軍將何拯爭官務,訟之三司,三司直溫。拯訴之弘肇,弘肇以謂潁己屬州,而溫不先白己,乃追溫殺之,連坐者數十人。



 周太祖平李守貞,推功群臣,弘肇拜中書令。隱帝自關西罷兵,漸近小人,與後贊、李業等嬉遊無度,而太后親族頗行干託,弘肇與楊邠稍裁抑之。太后有故人子求補軍職,
 弘肇輒斬之。帝始聽樂,賜教坊使等玉帶、錦袍,往謝弘肇,弘肇怒曰:「健兒為國征行者未有偏賜,爾曹何功,敢當此乎!」悉取所賜還官。



 周太祖出鎮魏州,弘肇議帶樞密行,蘇逢吉、楊邠以為不可,弘肇恨之。明日,會飲竇貞固第,弘肇厲聲舉爵屬太祖曰:「昨日廷論,何為異同?今日與公飲此。」



 逢吉與邠亦舉大爵曰:「此國家事也,何必介意乎!」遂俱飲爵。弘肇曰:「安朝廷,定禍亂,直須長槍大劍,若『毛錐子』安足用哉?」三司使王章曰:「無『毛錐子』,軍賦何從集乎?」「毛錐子」,蓋言筆也。弘肇默然。他日,會飲章第,酒酣,為手勢令,弘肇不能為,客省使閻晉卿坐次弘
 肇,屢教之。蘇逢吉戲曰:「坐有姓閻人,何憂罰爵!」弘肇妻閻氏,酒家倡,以為譏己,大怒,以醜語詬逢吉,逢吉不校。弘肇欲毆之,逢吉先出。弘肇起索劍欲追之,楊邠泣曰:「蘇公,漢宰相,公若殺之,致天子何地乎?」弘肇馳馬去,邠送至第而還。由是將相如水火。隱帝遣王峻置酒公子亭和解之。



 是時,李業、郭允明、後贊、聶文進等用事,不喜執政。而隱帝春秋漸長,為大臣所制,數有忿言,業等乘間譖之,以謂弘肇威震人主,不除必為亂。隱帝頗欲除之。夜聞作坊鍛甲聲,以為兵至,達旦不寐。由是與業等密謀禁中。乾祐三年冬十月十三日,弘肇與楊邠、王章
 等入朝,坐廣政殿東廡,甲士數十人自內出,擒弘肇、邠、章斬之,并族其三家。



 弘肇已死,帝坐崇元殿召君臣,告以弘肇等謀反,君臣莫能對。又召諸軍校見於萬歲殿,帝曰:「弘肇等專權,使汝曹常憂橫死,今日吾得為汝主矣!」軍校皆拜。周太祖即位,追封弘肇鄭王,以禮歸葬。



 楊邠楊邠,魏州冠氏人也。少為州掌籍吏,租庸使孔謙領度支,補邠勾押官,歷孟、華、鄆三州糧料院使。事漢高祖為右都押衙,高祖即位,拜樞密使。邠出於小吏,不喜文士,與蘇逢吉等內相排忌。逢吉諷李濤上疏罷邠與周太祖樞密使,邠泣訴李太后前,太后怒,罷濤相,加邠中書
 侍郎兼吏部尚書、同平章事。是時,逢吉、禹珪頗以私賄除吏,多繆。邠為相,事無大小,必先示邠,邠以為可,乃入白,而深革逢吉所為,凡門蔭出身,諸司補吏者,一切罷之。邠雖長於吏事,而不知大體,以謂為國家者,帑廩實、甲兵完而已,禮樂文物皆虛器也。以故秉大政而務苛細,凡前資官不得居外,而天下行旅,皆給過所然後得行。旬日之間,人情大擾,邠度不可行而止。



 邠常與王章論事帝前,帝曰:「事行之後,勿使有言也!」邠遽曰:「陛下但禁聲,有臣在。」聞者為之戰心慄。李太后弟業求為宣徽使,帝與太后私以問邠,邠止以為不可。帝欲立所愛耿夫
 人為后,邠又以為不可;夫人死,將以后禮葬之,邠又以為不可。由是隱帝大怒,而左右乘間構之,與史弘肇等同日見殺。



 邠為人頗儉靜,四方之賂雖不卻,然往往以獻於帝。居家謝絕賓客,晚節稍通縉紳,延客門下。知史傳有用,乃課吏傳寫。未幾,及於禍。周太祖即位,追封弘農王。



 王章王章,魏州南樂人也。為州孔目官,張令昭逐節度使劉延皓,章事令昭。令昭敗,章婦翁白文珂與副招討李周善,乃以章託周。周匿章褚中,以橐駝負之洛陽,藏周第。唐滅,章乃出,為河陽糧料使。漢高祖典禁兵,補章孔
 目官,從之太原。



 高祖即位,拜三司使、檢校太尉。高祖崩,隱帝即位,加太尉、同中書門下平章事。



 是時,漢方新造,承契丹之後,京師空乏,而關西三叛作,周太祖用兵西方,章供饋軍旅,未嘗乏絕。然征利剝下,民甚苦之。往時民租一石輸二升為「雀鼠耗」,章乃增一石輸二斗為「省耗」;緡錢出入,皆以八十為陌,章減其出者陌三;州縣民訴田者,必全州縣覆之,以括其隱田。天下由此重困。然尤不喜文士,嘗語人曰:「此輩與一把算子,未知顛倒,何益於國邪!」百官俸廩,皆取供軍之餘不堪者,命有司高估其價,估定又增,謂之「抬估」,章猶意不能滿,往往復增
 之。民有犯鹽、礬、酒曲者,無多少皆抵死,吏緣為姦,民莫堪命。已而與史弘肇等同日見殺。



 劉銖劉銖,陜州人也。少為梁邵王牙將,與漢高祖有舊,高祖鎮太原,以為左都押衙。銖為人慘酷好殺戮,高祖以為勇斷類己,特信用之。高祖即位,拜永興軍節度使,徙鎮平盧,加檢校太師、同平章事,又加侍中。



 是時,江淮不通,吳越錢鏐使者常泛海以至中國。而濱海諸州皆置博易務,與民貿易。民負失期者,務吏擅自攝治,置刑獄,不關州縣。而前為吏者,納其厚賂,縱之不問。民頗為苦,銖乃一切禁之。然銖用法,亦自為刻深。民有過者,問其年
 幾何,對曰若干,即隨其數杖之,謂之「隨年杖」。每杖一人,必兩杖俱下,謂之「合歡杖」。又請增民租,畝出錢三十以為公用,民不堪之。隱帝患銖剛暴,召之,懼不至。是時,沂州郭淮攻南唐還,以兵駐青州,隱帝乃遣符彥卿往代銖。銖顧禁兵在,莫敢有異意,乃受代還京師。



 銖嘗切齒於史弘肇、楊邠等,已而弘肇等死,銖謂李業等曰:「諸君可謂僂儸兒矣。」權知開封府,周太祖兵犯京師,銖悉誅太祖與王峻等家屬。太祖入京師,銖妻裸露以席自蔽,與銖俱見執。銖謂其妻曰:「我則死矣,汝應與人為婢。」太祖使人責銖曰:「與公共事先帝,獨無故人之情乎?吾家
 屠滅,雖有君命,加之酷毒,一何忍也。今公亦有妻子,獨念之乎?」銖曰:「為漢誅叛臣爾,豈知其佗。」



 是時,太祖方欲歸人心,乃與群臣議曰:「劉侍中墜馬傷甚,而軍士逼辱,迨有微生,吾欲奏太后,貸其家屬,何如?」群臣皆以為善。乃止殺銖,與李業等梟首於市,赦其妻子。太祖即位,賜陜州莊宅各一區。



 李業李業,高祖皇后之弟也。后昆弟七人,業最幼,故尤憐之。高祖時,以為武德使。隱帝即位,業以皇太后故,益用事,無顧憚。時天下旱、蝗,黃河決溢,京師大風拔木,壞城門,宮中數見怪物投瓦石、撼門扉。隱帝召司天趙延乂問
 禳除之法,延乂對曰:「臣職天象日時,察其變動,以考順逆吉凶而已,禳除之事,非臣所知也。然臣所聞,殆山魈也。」皇太后乃召尼誦佛書以禳之,一尼如廁,既還,悲泣不知人者數日,及醒訊之,莫知其然。而帝方與業及聶文進、後贊、郭允明等狎暱,多為廋語相誚戲,放紙鳶于宮中。太后數以災異戒帝,不聽。時宣徽使闕,業欲得之,太后亦遣人諷大臣。大臣楊邠、史弘肇等皆以為不可。業由此怨望,謀殺邠等。



 邠等已死。又遣供奉官孟業以詔書殺郭威于魏州。威舉兵反,隱帝遣左神武統軍袁泬、侍衛馬軍都指揮使閻晉卿等率兵拒威于澶
 淵。兵未出,威已至滑州,帝大懼,謂大臣曰:「昨太草草耳。」業請出府庫以賚軍,宰相蘇禹珪以為未可,業拜禹珪於帝前曰:「相公且為官家勿惜府庫。」乃詔賜京師兵及魏兵從威南者錢人十千,督其子弟作書,以告北兵之來者。及漢兵敗于北郊,業取內庫金寶,懷之以奔其兄保義軍節度使洪信,洪信拒而不納。業走至絳州,為人所殺。



 聶文進聶文進,并州人也。少為軍卒,善書算,給事漢高祖帳中。高祖鎮太原,以為押司官。高祖即位,歷拜領軍屯衛將軍、樞密院承旨。周太祖為樞密使,頗親信之,文進稍橫
 恣。遷右領軍大將軍,入謝,召諸將軍設食朝堂,儀鸞、翰林、御廚供帳飲食,文進自如,有司不敢劾。周太祖鎮鄴,文進等用事居中,及謀殺楊邠等,文進夜作詔書,制置中外。邠等已死,文進點閱兵籍,指麾殺戮,以為己任。周太祖在鄴聞邠等遇害,初以為文進不與,及發詔書,皆文進手跡,乃大詬之。



 周兵至京師,隱帝敗于北郊,太后懼,使謂文進善衛帝,對曰:「臣在此,百郭威何害!」慕容彥超敗走,帝宿于七里,文進夜與其徒飲酒,歌呼自若。明旦,隱帝遇弒,文進亦自殺。



 後贊後贊,兗州瑕丘人。其母,倡也。贊幼善謳,事張延朗。延朗
 死,贊更事漢高祖,高祖愛之,以為牙將。高祖即位,拜飛龍使,隱帝尤愛幸之。楊邠等執政,贊久不得遷,乃共謀殺邠等。邠等死,隱帝悔之,贊與允明等番休侍帝,不欲左右言已短。隱帝兵敗北郊,贊奔兗州,慕容彥超執送京師,梟首于市。



 郭允明郭允明,少為漢高祖廝養,高祖愛之,以為翰林茶酒使。隱帝尤狎愛之,允明益驕橫無顧避,大臣不能禁。允明使荊南高保融,車服導從如節度使,保融待之甚厚。允明乃陰使人步測其城池高下,若為攻取之計者以動之。荊人皆恐,保融厚賂以遺之。遷飛龍使。已而李業與
 允明謀殺楊邠等,是日無雲而昏,霧雨如泣,日中,載邠等十餘尸暴之市中。允明手殺邠等諸子於朝堂西廡,王章婿張貽肅血流逆注。



 隱帝敗於北郊,還至封丘門,不得入,帝走趙村,允明從後追之,弒帝於民舍,乃自殺。



\end{pinyinscope}