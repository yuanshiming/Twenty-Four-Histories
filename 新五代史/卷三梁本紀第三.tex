\article{卷三梁本紀第三}

\begin{pinyinscope}

 末帝,太祖第三子友貞也。為人美容貌,沈厚寡言,雅好儒士。太祖即位,封均王,為左天興軍使、東京馬步軍都指揮使。乾化二年六月,太祖遇弒,友珪自立,殺博王友文,以弒帝之罪歸之。以王為東京留守、開封尹,敬翔為中書侍郎、同中書門下平章事,戶部尚書李振為崇政院使。



 明年,友珪改元曰鳳曆。二月,駙馬都尉趙巖至東都,王私與之謀,遣馬慎交之魏州,見楊師厚
 計事。師厚遣小校王舜賢至洛陽,告左龍虎統軍袁象先使討賊。是時懷州龍驤屯兵叛,方捕索之,王乃偽為友珪詔書,發左右龍驤在東都者皆還洛陽,因激怒之曰:「天子以懷州屯兵叛,追汝等欲盡坑之。」諸將皆泣,莫知所為。王曰:「先皇帝經營王業三十餘年,今日尚為友珪所殺,汝等安所逃死乎!」因出太祖畫像示諸將而泣曰:「汝能趨洛陽擒逆賊,則轉禍為福矣。」軍士皆呼萬歲,請王為主,王乃遣人趣象先等。庚寅,象先等以禁兵討賊,友珪死,杜曉見殺。象先遣趙巖持傳國寶至東都,請王入洛陽,王報曰:「夷門,太祖所以興王業也,北拒并汾,
 東至淮海,國家藩鎮,多在東方,命將出師,利于便近。」



 是月,皇帝即位於東都,復稱乾化三年,復博王友文官爵。



 三月丁未,更名鎤。



 夏五月,楊師厚取滄州。



 秋九月甲辰,御史大夫姚洎為中書侍郎、同中書門下平章事。



 冬十二月,晉人取幽州。



 四年夏四月丁丑,貶于兢為萊州司馬。武寧軍節度使蔣殷反,天平軍節度使牛存節討之。



 貞明元年春正月,存節克徐州。



 三月丁卯,趙光逢罷。平盧軍節度使賀德倫為天雄軍節度使,分其相、澶、衛州為昭德軍,宣
 徽使張筠為節度使。己丑,天雄軍亂,賀德倫叛附于晉。邠州李保衡叛于岐,來附。



 夏六月庚寅朔,晉王李存勖入於魏州,遂取德州。



 冬十月辛亥,康王友孜反,伏誅。



 十一月乙丑,改元。耀州溫昭圖叛于岐,來附。



 是歲,更名瑱。



 二年春二月丙申,楊涉罷。



 三月,鎮南軍節度使劉鄩及晉人戰于故元城,敗績,奔于滑州。晉人取衛州、惠州。捉生都將李霸反,伏誅。



 夏六月,捉生都將張溫叛降于晉。



 秋七月,晉人取相州,張筠奔於京師,安國軍節度使使閻
 寶叛附于晉。



 八月丁酉,太子太保致仕趙光逢為司空兼門下侍郎、同中書門下平章事。



 九月,晉人取滄州,橫海軍節度使戴思遠奔于京師。晉人克貝州,守將張源德死之。



 冬十月丁酉,中書侍郎鄭玨同中書門下平章事。



 三年夏四月辛卯,右千牛衛大將軍劉璩使于契丹。



 冬十二月,宣義軍節度使賀環為北面行營招討使。己巳,如西都卜郊。晉人取楊劉。



 四年正月,不克郊,己卯,至自西都。



 夏四月己酉,尚書吏部侍郎蕭頃為中書侍郎、同中書門下平章事。己巳,趙
 光逢罷。



 冬十二月庚子朔,賀環殺其將謝彥章、孟審澄、侯溫裕。癸亥,環及晉人戰于胡柳,敗績。



 是歲,泰寧軍節度使張守進叛附于晉,亳州團練使劉鄩為袞州安撫制置使以討之。



 五年春正月,晉軍于德勝。



 秋八月乙未朔,開封尹王瓚為北面行營招討使。



 冬十月,劉鄩克袞州,張守進伏誅。



 十二月,晉人取濮陽。天平軍節度使霍彥威為北面行營招討使。



 六年夏四月己亥,降死罪以下囚。乙巳,尚書左丞李琪為中書侍郎、同中書門下平章事。河中節度使朱友謙
 襲同州,殺其節度使程全暉,叛附于晉,泰寧軍節度使劉蚑討之。



 秋七月,陳州妖賊毋乙自稱天子。



 九月庚寅,供奉官郎公遠為契丹歡好使。



 冬十月,毋乙伏誅。



 龍德元年春,趙將張文禮殺其君熔來乞師,不許。



 三月丁亥朔,禁私度僧尼。陳州刺史惠王友能反。



 夏五月丙戌朔,德音改元,降流罪已下囚。



 秋,赦友能,降封房陵侯。天平軍節度使戴思遠為北面行營招討使。



 冬十月,思遠及晉人戰于戚城,敗績。



 二年春正月,思遠襲魏州,取成安。



 秋八月,滑州兵馬留後段凝攻衛州,執其刺史李存儒。戴思遠克淇門、共城、
 新鄉。



 三年春三月,潞州李繼韜叛於晉,來附。



 夏閏四月,唐人鄆州。



 五月庚申,宣義軍節度使王彥章為北面行營招討使,取德勝南城。



 秋八月,段凝為北面行營招討使。先鋒將康延孝叛降於唐。



 冬十月甲戌,宣義軍節度使王彥章及唐人戰於中都,敗績,死之。唐人取曹州。盜竊傳國寶奔於唐。戊寅,皇帝崩。梁亡。



\end{pinyinscope}