\article{卷九晉本紀第九}

\begin{pinyinscope}

 出帝父敬儒,高祖兄也,為唐莊宗騎將,早卒,高祖以其子重貴為子。高祖六子,五皆早死,而重睿幼,故重貴得立。



 重貴少而謹厚,善騎射。高祖使博士王震教以《禮記》,久之,不能通大義,謂震曰:「此非我家事也。」高祖為契丹所立,謀以一子留守太原。契丹使盡出諸子自擇之,指重貴曰:「此眼大者可也。」遂拜金紫光祿大夫,行太原尹、北京留守,知河東節度事。



 天福二年九月,召拜左金吾
 衛上將軍。三年冬,為開封尹,封鄭王,加太尉,同中書門下平章事。六年,高祖幸鄴,留守東京。已而為廣晉尹,徙封齊王。



 七年六月乙丑,高祖崩,皇帝即位于柩前。庚午,使左驍衛將軍石德超以御馬二撲祭于相州之西山。如京使李仁廓使於契丹,契丹使梅李來。丙子,馮道為大行皇帝山陵使,門下侍郎竇貞固為副,太常卿崔棁為禮儀使,戶部侍郎呂琦為鹵簿使,御史中丞王易簡為儀仗使。己卯,四方館使硃崇範、右金吾衛大將軍梁言使于契丹。



 秋七月壬
 辰,皇祖母劉氏崩,輟視朝三日。丁酉,使石德超撲馬于相州之西山。庚子,大赦。甲辰,契丹使通事來。八月戊午,高行周克襄州。庚申,天平軍節度使景延廣、義成軍節度使李守貞、彰德軍節度使郭謹,進錢粟助作山陵。甲子,契丹使郎五來。庚午,葬皇祖母於魏縣。癸酉,契丹使其客省使張九思來。九月辛丑,李守貞為大行皇帝山陵都部署。冬十月己未,契丹使舍利來。庚午,回鶻遣使者來。



 十一月,契丹使大卿來。庚寅,葬聖文章武孝皇帝於顯陵。己
 亥,牛羊使董殷使于契丹。庚子,祔高祖神主於太廟。辛丑,蠲高祖靈車所過民租之半。十二月庚午,北京留守劉知遠進百頭穹廬。契丹于越使令骨支來。辛未,又使野里巳來。丙子,于闐使都督劉再昇來,沙州曹元深、瓜州曹元忠皆遣使附再昇以來。旱,蝗。八年春正月,契丹于越使烏多奧來。二月壬子,景延廣為御營使。己未,如東京,赦廣晉府囚。庚申,次澶州,赦囚。乙丑,至自鄴都。庚午,寒食,望祭顯陵於南莊,焚御衣、紙
 錢。三月己卯朔,趙瑩罷。晉昌軍節度使桑維翰為侍中。辛丑,引進使、太府卿孟承誨使于契丹。蝗。夏四月庚午,董殷使於契丹。供奉官張福率威順軍捕蝗于陳州。五月,泰寧軍節度使安審信捕蝗于中都。丁亥,追封皇伯敬儒為宋王。



 癸卯,馮道罷。甲辰,以旱、蝗大赦。六月庚戌,祭蝗於皋門。癸亥,供奉官七人帥奉國軍捕蝗於京畿。辛未,括借民粟,殺藏粟者。秋七月甲午,冊皇太后。丁酉,射于南莊。契丹使梅里等來。甲辰,供奉官李漢超帥奉國軍捕蝗于京畿。八月丁未朔,募民捕蝗,易以粟。辛亥,檢民青苗。九月戊寅,尊秦
 國夫人安氏為皇太妃。



 丙申,幸大年莊及景延廣第。冬十月戊申,立馮氏為皇后。壬子,畋于近郊,幸沙臺。丙寅,契丹使通事劉胤來。庚午,括借民粟。十一月己卯,董殷使於契丹。甲申,幸八角,閱馬牧。乙未,契丹使梅里來。戊戌,齊州刺史楊承祚奔于青州。辛丑,高麗使其廣評侍郎金仁逢來。十二月癸丑,給事中邊光範、登州刺史郭彥威使于契丹。甲寅,高麗使太相來。平盧軍節度使楊光遠反,淄州刺史翟進宗死之。



 開運元年春正月甲戌朔,契丹寇滄州。己卯,陷貝州。庚辰,歸德軍節度使高行周為北面行營都部署。契丹入
 鴈門,寇代州。辛巳,殿直王班使於契丹,至于鄴都,不得進而復。大饑。壬午,前靜難軍節度使李周留守東京,景延廣為御營使。



 乙酉,北征。丙戌,契丹寇黎陽。辛卯,講武於澶州。契丹屯于元城,趙延壽寇南樂。甲午,劉知遠為幽州道行營招討使。括馬。丙申,契丹寇黎陽。辛丑,劉知遠及契丹偉王戰于秀容,敗之。博州刺史周儒叛降於契丹。二月戊申,前軍都虞候李守貞及契丹戰于馬家渡,敗之。癸丑,北面行營都虞候馬全節及契丹戰于北平,敗之。三月癸酉,及契丹戰于戚城,契
 丹去。己丑,冀州刺史白從暉及契丹戰于衡水,敗之。癸巳,籍民為武定軍。夏四月,契丹陷德州,沿河巡檢使梁進敗之,取德州。



 甲寅,至自澶州,赦京師。己未,馬全範及契丹戰于定豐,敗之。辛酉,率借民財。



 五月戊寅,李守貞討楊光遠。丁亥,鄴都留守張從恩為貝州行營都部署。辛卯,李守貞為青州行營都部署。六月,克淄州。丙午,復置樞密使。丁未,侍中桑維翰為中書令,充樞密使。丙辰,河決滑州,環梁山,入于汶、濟。秋七月辛未朔,大赦,改元。己丑,太子太傅劉昫守司空兼門下侍郎、同中書門下平章事。八月辛丑朔,劉知遠
 為北面行營都統,順德軍節度使杜威為都招討使。戊辰,旌表陳州項城民史仁詡門閭。九月丙子,契丹寇遂城、樂壽,代州刺史白文珂及契丹戰于七里烽,敗之。冬十月庚戌,武寧軍節度使趙在禮為北面行營副都統,鄴都留守馬全節為副招討使。十二月己亥朔,射兔于皋門。丁巳,楊承勳囚其父光遠以降,殺之。閏月乙酉,德音赦青州囚。契丹寇恒州。



 二年春正月,契丹陷泰州。壬子,馬全節及契丹戰于榆林,兩軍皆潰。戊午,幸南莊,張從恩留守東都。辛酉,高行周為御營使。乙丑,北征,契丹去。二月己巳,幸黎陽。橫海
 軍節度使田武為東北面行營都部署,以備契丹。丙子,大閱于戚城。丙戌,閱馬於鐵丘。丙申,端明殿學士、尚書戶部侍郎馮玉為戶部尚書、樞密使。三月戊戌,契丹陷祁州,刺史沈斌死之。丁未,畋于戚城。庚戌,馬全節克泰州。辛亥,易州戍將孫方諫及契丹諧里戰于狼山,敗之。甲寅,杜威克滿城。乙卯,克遂城。庚申,杜威及契丹戰于陽城,敗之,追奔至于衛村,又敗之。夏四月戊寅,勞旋于戚城。己卯,勞旋于王莽河。甲申,至自澶州,赦左右軍囚。庚寅,大賞軍功。五月丙申朔,大赦。丙午,幸南莊。六月丁卯,射于繁臺,幸杜威第。旱。秋八月甲
 子朔,廢二舞。丙寅,和凝罷。馮玉為中書侍郎、同中書門下平章事。辛未,閱馬于茂澤陂。丁丑,括馬。九月己亥,閱馬于萬龍岡,幸李守貞第。冬十月丁丑,高麗使其廣評侍郎韓玄珪、禮賓卿金廉等來。戊寅,射兔于硯臺。戊子,高麗使其兵部侍郎劉崇珪、內軍卿朴藝言來。十一月戊戌,封王武為高麗國王。己巳,射兔于皋門,幸沙臺。十二月丁丑,臘,畋于郊。丁亥,桑維翰罷。開封尹趙瑩為中書令,李崧守侍中、樞密使。



 三年春二月丙子,回鶻使突厥陸來。壬午,射鴨于板橋,幸南莊。夏六月,孫方諫以狼山叛附于契丹。丙寅,契丹
 寇邊。己丑,李守貞為行營都部署,義成軍節度使皇甫遇為副。河決漁池。大饑,群盜起。秋七月,大雨,水,河決楊劉、朝城、武德。八月辛酉,河溢歷亭。九月,河決澶、滑、懷州。辛丑,行營馬軍排陣使張彥澤及契丹戰於新興,敗之。癸卯,劉知遠及契丹戰于朔州,敗之。大雨霖,河決臨黃。冬十月,河決衛州,丙寅,河決原武。辛未,杜威為北面行營都招討使,李守貞為兵馬都監。十一月,永靜軍節度使梁漢璋及契丹戰於瀛州,敗績。契丹寇鎮、定。十二月己未,杜威軍于中渡。壬戌,奉國都指揮使王清及契丹
 戰于滹沱,敗績,死之。杜威、李守貞、張彥澤以其軍叛降于契丹。庚午,射兔于沙臺。壬申,張彥澤犯京師,殺開封尹桑維翰。契丹滅晉。



 嗚呼,餘書「封子重貴為鄭王」,又書「追封皇伯敬儒為宋王」者,豈無意哉!



 《禮》:「兄弟之子,猶子也。」重貴書子可也,敬儒出帝父也,書曰皇伯者何哉?



 出帝立不以正,而絕其所生也。蓋出帝於高祖,得為子而不得為後者,高祖自有子也。方高祖疾病,抱其子重睿置於馮道懷中而託之,出帝豈得立邪?晉之大臣,既違禮廢命而立之,以謂出帝為高祖子則得立,為敬儒子則不得立,於是深諱
 其所生而絕之,以欺天下為真高祖子也。《禮》曰:「為人後者,為其父母服。」使高祖無子,出帝得為後而立以正,則不待絕其所生以為欺也。故餘書曰「追封皇伯敬儒為宋王」者,以見其立不以正,而滅絕天性,臣其父而爵之,以欺天下也。



\end{pinyinscope}