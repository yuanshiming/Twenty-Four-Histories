\article{卷十四唐家人傳第二}

\begin{pinyinscope}

 太祖劉太妃貞簡皇後曹氏太祖正室劉氏,代北人也;其次妃曹氏,太原人也。太祖封晉王,劉氏封秦國夫人。自太祖起兵代北,劉氏常從征伐。為人明敏多智略,頗習兵機,常教其侍妾騎射,以佐太祖。太祖東追黃巢,還軍過梁,館于封禪寺。梁王邀太祖入城,置酒上源驛,夜半以兵攻之。太祖左右有先脫歸者,以難告夫人,夫人神色不動,立斬告者,陰召大將謀保軍以還。遲明,太祖還,與夫人相嚮慟哭,因欲
 舉兵擊梁。夫人曰:「公本為國討賊,今梁事未暴,而遽反兵相攻,天下聞之,莫分曲直。不若斂軍還鎮,自訴于朝。」太祖從之。



 其後,太祖擊劉仁恭,敗歸。梁遣氏叔琮、康懷英等連歲攻晉,圍太原,晉兵屢敗,太祖憂窘,不知所為。大將李存信等勸太祖亡入北邊,收兵以圖再舉,太祖然之。入以語夫人,夫人問誰為此謀者,曰:「存信也。」夫人罵曰:「存信,代北牧羊兒耳,安足與計成敗邪!且公嘗笑王行瑜棄邠州走,卒為人擒,今乃自為此乎?昔公亡在達靼,幾不能自脫,賴天下多故,乃得南歸。今屢敗之兵,散亡無幾,一失其守,誰肯從公?北邊其可至乎?」太祖
 大悟而止。已而亡兵稍稍復集。



 夫人無子,性賢,不妒忌,常為太祖言:「曹氏相,當生貴子,宜善待之。」



 而曹氏亦自謙退,因相得甚歡。



 曹氏封晉國夫人,後生子,是謂莊宗,太祖奇之,曹氏由是專寵。太祖性暴,怒多殺人,左右無敢言者,惟曹氏從容諫譬,往往見聽。及莊宗立,事曹氏尤謹,其救趙破燕取魏博,與梁戰河上十餘歲,歲嘗馳省其母至三四,人皆稱其孝。莊宗即位,冊尊曹氏為皇太后,而以嫡母劉氏為皇太妃。太妃往謝太后,太后有慚色。



 太妃曰:「願吾兒享國無窮,使吾獲沒于地以從先君,幸矣,復何言哉!」



 莊宗滅梁入汴,使人迎太后歸洛,
 居長壽宮,而太妃獨留晉陽。同光三年五月,太妃薨。七月,太后薨,謚曰貞簡,葬于坤陵。而太妃無謚,葬魏縣。太妃與太后甚相愛,其送太后于洛也,涕泣而別,歸而相思慕,遂至不起。太后聞之,欲馳至晉陽視疾,及其卒也,又欲自往葬之,莊宗泣諫,群臣交章請留,乃止。而太后自太妃卒,悲哀不飲食,逾月亦崩。



 莊宗敬皇後劉氏莊宗神閔敬皇后劉氏,魏州成安人也。莊宗正室曰衛國夫人韓氏,其次燕國夫人伊氏,其次后也,初封魏國夫人。后父劉叟,黃鬚,善醫卜,自號劉山人。后生五六歲,晉王攻魏,掠成安,裨將袁建豐得后,納之晉宮,貞簡太
 后教以吹笙歌舞。



 既笄,甚有色,莊宗見而悅之。莊宗已為晉王,太后幸其宮,置酒為壽,自起歌舞,太后歡甚,命劉氏吹笙佐酒,酒罷去,留劉氏以賜莊宗。先時,莊宗攻梁軍於夾城,得符道昭妻侯氏,寵專諸宮,宮中謂之「夾寨夫人」。莊宗出兵四方,常以侯氏從軍。其後,劉氏生子繼岌,莊宗以為類己,愛之,由是劉氏寵益專,自下魏博、戰河上十餘年,獨以劉氏從。劉氏多智,善迎意承旨,其他嬪御莫得進見。



 其父聞劉氏已貴,詣魏宮上謁。莊宗召袁建豐問之,建豐曰:「臣始得劉氏於成安北塢,時有黃須丈人護之。」乃出劉叟示建豐,建豐曰:「是也。」然劉氏
 方與諸夫人爭寵,以門望相高,因大怒曰:「妾去鄉時,略可記憶,妾父不幸死於亂兵,妾時環尸慟哭而去。此田舍翁安得至此!」因命笞劉叟于宮門。



 莊宗已即皇帝位,欲立劉氏為皇后,而韓夫人正室也,伊夫人位次在劉氏上,以故難其事而未發。宰相豆盧革、樞密使郭崇韜希旨,上章言劉氏當立,莊宗大悅。



 同光二年四月已卯,皇帝御文明殿,遣使冊劉氏為皇后。皇后受冊,乘重翟車,鹵簿、鼓吹,見於太廟。韓夫人等皆不平之,乃封韓氏為淑妃,伊氏為德妃。



 莊宗自滅梁,志意驕怠,宦官、伶人亂政,后特用事於中。自以出於賤微,踰次得立,以為佛
 力。又好聚斂,分遣人為商賈,至於市肆之間,薪芻果茹,皆稱中宮所賣。四方貢獻,必分為二,一以上天子,一以入中宮,宮中貨賄山積。惟寫佛書,饋賂僧尼,而莊宗由此亦佞佛。有胡僧自于闐來,莊宗率皇后及諸子迎拜之。



 僧遊五臺山,遣中使供頓,所至傾動城邑。又有僧誠惠,自言能降龍。嘗過鎮州,王鎔不為之禮,誠惠怒曰:「吾有毒龍五百,當遣一龍揭片石,常山之人,皆魚鱉也。」會明年滹沱河大水,壞鎮州關城,人皆以為神。莊宗及后率諸子、諸妃拜之,誠惠端坐不起,由是士無貴賤皆拜之,獨郭崇韜不拜也。



 是時,皇太后及皇后交通籓鎮,太后
 稱「誥令」,皇后稱「教命」,兩宮使者旁午於道。許州節度使溫韜以后佞佛,因請以私第為佛寺,為后薦福。莊宗數幸郭崇韜、元行欽等私第,常與后俱。其後幸張全義第,酒酣,命后拜全義為養父。全義日遣姬妾出入中宮,問遺不絕。



 莊宗有愛姬,甚有色而生子,后心患之。莊宗燕居宮中,元行欽侍側,莊宗問曰:「爾新喪婦,其復娶乎?吾助爾聘。」后指愛姬請曰:「帝憐行欽,何不賜之?」



 莊宗不得已,陽諾之。后趣行欽拜謝,行欽再拜,起顧愛姬,肩輿已出宮矣。莊宗不樂,稱疾不食者累日。



 同光三年秋大水,兩河之民,流徙道路,京師賦調不充,六軍之士,往往殍
 踣,乃預借明年夏、秋租稅,百姓愁苦,號泣于路,莊宗方與后荒於畋遊。十二月己卯臘,畋于白沙,后率皇子、後宮畢從,歷伊闕,宿龕澗,癸未乃還。是時大雪,軍士寒凍,金槍衛兵萬騎,所至責民供給,壞什器,徹廬舍而焚之,縣吏畏懼,亡竄山谷。



 明年三月,客星犯天庫,有星流于天棓。占星者言:「御前當有急兵,宜散積聚以禳之。」宰相請出庫物以給軍,莊宗許之,后不肯,曰:「吾夫婦得天下,雖因武功,蓋亦有天命。命既在天,人如我何!」宰相論于延英,后於屏間耳屬之,因取妝奩及皇幼子滿喜置帝前曰:「諸侯所貢,給賜已盡,宮中所有惟此耳,請鬻以給
 軍!」宰相惶恐而退。及趙在禮作亂,出兵討魏,始出物以齎軍,軍士負而詬曰:「吾妻子已饑死,得此何為!」



 莊宗東幸汴州,從駕兵二萬五千,及至萬勝,不得進而還,軍士離散,所亡太半。至罌子谷,道路隘狹,莊宗見從官執兵仗者,皆以好言勞之曰:「適報魏王平蜀,得蜀金銀五十萬,當悉給爾等。」對曰:「陛下與之太晚,得者亦不感恩。」



 莊宗泣下,因顧內庫使張容哥索袍帶以賜之,容哥對曰:「盡矣。」軍士叱容哥曰:「致吾君至此,皆由爾輩!」因抽刀逐之,左右救之而免。容哥曰:「皇后惜物,不以給軍,而歸罪於我。事若不測,吾身萬段矣!」乃投水而死。



 郭從謙反,莊
 宗中流矢,傷甚,臥絳霄殿廓下,渴欲得飲,后令宦官進飧酪,不自省視。莊宗崩,后與李存渥等焚嘉慶殿,擁百騎出師子門。后於馬上以囊盛金器寶帶,欲於太原造寺為尼。在道與存渥姦,及至太原,乃削髮為尼。明宗入立,遣人賜后死。晉天福五年,追謚曰神閔敬皇后。



 自唐末喪亂,后妃之制不備,至莊宗時,後宮之數尤多,有昭容、昭儀、昭媛、出使、御正、侍真、懿才、咸一、瑤芳、懿德、宣一等,其餘名號,不可勝紀。莊宗遇弒,後宮散走,朱守殷入宮,選得三十餘人。虢國夫人夏氏以嘗幸於莊宗,守殷不敢留。明宗立,悉放莊宗時宮人還其家,獨夏氏
 無所歸,乃以河陽節度使夏魯奇同姓也,因以歸之,後嫁契丹突欲李贊華。贊華性酷毒,喜殺人,婢妾微過,常加刲灼。夏氏懼,求離婚,及削髮為尼以卒。而韓淑妃、伊德妃皆居太原,晉高祖反時,為契丹所虜。



 太祖四弟唐自朱邪得姓而為李氏,得國而為晉,得天下而為唐。其始出於夷狄,而終以亂亡,故其世次不可詳見。其可見者,曰太祖四弟、八子、五孫,三世而絕。太祖四弟,曰克讓、克脩、克恭、克寧,皆不知其父母名號。



 克讓,少善騎射,為振武軍校,從討王仙芝,以功拜金吾衛將軍,留京師。李氏自憲宗時以部族歸唐,唐處之河
 西,嘗遣一子宿衛京師,賜第於親仁坊。其後太祖起兵雲中,殺唐守將段文楚。唐發兵討太祖,遣王處存以兵圍親仁坊,捕宿衛子克讓。克讓與其僕何相溫、石的歷等十餘騎,彎弧躍馬,突圍而出。處存以千餘人追至渭橋,克讓等射殺百餘人,追兵乃止,克讓奔于鴈門。明年,太祖復歸唐,克讓還宿衛京師。黃巢犯長安,克讓守潼關,為賊所敗,奔于南山,匿佛寺,為寺僧所殺。



 克脩字崇遠,從討龐勛,以功拜朔州刺史。太祖鎮鴈門,以為奉誠軍使。從入關,討黃巢,為先鋒,遷左營軍使。潞州孟方立遷于邢州,晉取潞州,表克脩昭義軍節度使。
 數出山東擊方立,又與李罕之攻寇懷、孟之間。其後,太祖自將擊方立,還軍過潞,克脩性儉嗇,供饋甚薄,太祖大怒,詬而擊笞之。克脩慚憤,發疾卒。



 二子:嗣弼、嗣肱。嗣弼為涿州刺史,天祐十九年,契丹攻破涿州,嗣弼歿于虜。



 嗣肱少有膽略,從周德威數立戰功,為馬步軍都虞候。李存審敗梁軍于胡壁,嗣肱獲梁將一人。梁太祖圍蓚縣,嗣肱從存審救蓚,梁軍解去,嗣肱功為多,超拜蔚州刺史、鴈門以北都知兵馬使。累遷澤、代二州刺史。新州王郁叛晉,亡入契丹,山後諸州皆叛,嗣肱取媯、儒、武三州,拜新州刺史、山北都團練使。同光元年春,卒於官。



 克恭,初為決勝軍使。克脩卒,以克恭代為昭義軍節度使。克脩為人簡儉,潞人素安其政,且哀其見笞以死。克恭橫暴不法,又不習軍事,由是潞人皆怨。克恭選後院勁兵五百人,獻于太祖,行至銅鞮,其將馮霸以其徒叛。太祖遣李元審討之,戰于沁水,元審大敗被傷,奔入潞州。牙將安居受亦叛,殺克恭及元審,使人召霸,霸不受命,居受懼而出奔,行至長子,為野人所殺,傳首于霸。霸乃入潞州,自稱留後,以附于梁。



 克寧,為人仁孝,居諸兄弟中最賢,事太祖小心不懈。太祖與赫連鐸、李可舉戰雲、蔚間,後奔達靼,入破黃巢,克
 寧未嘗不從行。太祖鎮太原,以為內外制置蕃漢都知兵馬使,檢校太保、振武軍節度使,軍中之事,無大小皆決克寧。



 太祖病,召莊宗侍側,屬張承業與克寧曰:「以亞子屬公等。」太祖崩,莊宗告於克寧曰:「兒年孤稚,未通庶政,雖有先王之命,恐不足以當大事。叔父勳德俱高,先王嘗任以政矣,敢以軍府煩季父,以待兒之有立。」克寧曰:「吾兄之命,以兒屬我,誰敢易之!」因下而北面再拜稱賀,莊宗乃即晉王位。



 初,太祖起於雲、朔之間,所得驍勇之士,多養以為子,而與英豪戰爭,卒就霸業,諸養子之功為多,故尤寵愛之,衣服禮秩如嫡。諸養子麾下皆
 有精兵,恃功自恣,自先王時常見優假。及新王立,年少,或託疾不朝,或見而不拜。養子存顥、存實告克寧曰:「兄亡弟及,古之道也。以叔拜姪,理豈安乎?人生富貴,當自取之。」克寧曰:「吾家三世,父慈子孝,先王土宇,茍有所歸,吾復何求也!」



 克寧妻孟氏素剛悍,存顥等各遣其妻入說孟氏,孟氏數以迫克寧。克寧仁而無斷,惑於群言,遂至於禍。都虞候李存質得罪於克寧,克寧殺之,而與張承業,李存璋有隙,又求兼領大同軍節度使。於是幸臣史敬熔見太后,告克寧與存顥謀執王及太后以降梁。莊宗召承業、存璋告之曰:「季父所為如此,奈何?然骨肉不可
 自相魚肉,吾當避賢路以紓禍於吾家。」承業等請誅克寧。乃伏兵於府,置酒大會,克寧既至,執而殺之。



 太祖七子太祖子八人:莊宗長子也,次曰存美、存霸、存禮、存渥、存乂、存確、存紀。



 同光三年十二月辛亥,詔封存美等七人為王。蓋存霸、存渥、存紀與莊宗同母也,存美、存乂、存確、存禮不知其母名氏號位。存美封邕王,存霸永王,存禮薛王,存渥申王,存乂睦王,存確通王,存紀雅王。



 存乂歷建雄、保大二軍節度使。娶郭崇韜女。是時,魏州妖人楊千郎用事,自言有墨子術,能役使鬼神,化丹砂、水銀。莊宗頗神之,拜千郎檢校尚書郎,賜紫,其妻出入宮禁,承恩
 寵,而士或因之以求官爵,存乂及存渥等往往朋淫于其家。及崇韜被族,莊宗遣宦官陰察外議以為如何,而宦官因欲盡誅崇韜親黨以絕後患,乃誣言:「存乂過千郎,酒酣,攘臂號泣,為婦翁稱冤,言甚怨望。」莊宗大怒,以兵圍其第而誅之,并誅千郎。



 存霸歷昭義、天平、河中三軍節度使,存渥義成、天平二軍節度使,皆居京師,食其俸祿而已。趙在禮作亂,乃遣存霸於河中。李嗣源兵反,嚮京師,莊宗再幸汜水,徙存霸北京留守,存渥河中節度使,宣麻未訖,郭從謙反,攻興教門,存渥從莊宗拒賊。莊宗中流矢崩,存渥與劉皇后同奔于太原,行至風
 谷,為部下所殺。存霸聞京師亂,亦自河中奔太原,比至,麾下皆散走,惟使下康從弁不去。存霸乃剪髮、衣僧衣,謁符彥超曰:「願為山僧,冀公庇護。」彥超欲留之,為軍眾所殺。



 存紀、存確聞郭從謙反,奔於南山,匿民家。明宗詔河南府及諸道:「諸王出奔,所至送赴闕;如不幸物故者,收瘞以聞。」存紀等所匿民家以告安重誨,重誨謂霍彥威曰:「二王逃難,主上尋求,恐其所失。今上既監國典喪,此禮如何?」



 彥威曰:「上性仁慈,不可聞奏。宜密為之所,以安人情。」乃即民家殺之。



 存美素病風,居太原,與存禮皆不知其所終。



 莊宗五子莊宗五子、長曰繼岌,其次繼潼、繼嵩、繼蟾、繼嶢。繼岌母曰劉皇后,其四皆不著其母名號。



 莊宗即位,繼岌為北都留守,判六軍諸衛事。遷檢校太尉、同中書門下平章事。



 豆盧革為相,建言:唐故事,皇子皆為宮使。因以鄴宮為興聖宮,以繼岌為使。同光三年,封魏王。是歲伐蜀,以繼岌為西南面行營都統,郭崇韜為都招討使,工部尚書任圜、翰林學士李愚皆參軍事。九月戊申,將兵六萬自鳳翔入大散關,軍無十日之糧,而所至州鎮皆迎降,遂食其粟。至興州,蜀將程奉璉以五百騎降,因以其兵脩閣道,以過唐軍。王衍將兵萬人屯利州,分其半逆戰
 于三泉,為先鋒康延孝所敗,衍懼,斷吉柏江浮橋,奔歸成都。唐軍自文州間道以入。十月己酉,繼岌至綿州,衍上箋請降。丙辰,入成都。王衍乘竹輿至昇仙橋,素衣、牽羊,草索繫首,肉袒、銜璧、輿櫬,群臣衰絰,徒跣以降。繼岌下而取璧,崇韜解縛,焚櫬。自出師至降衍,凡七十五日,兵不血刃,自古用兵之易,未有如此。然繼岌雖為都統,而軍政號令一出崇韜。



 初,莊宗遣宦者供奉官李從襲監中軍,高品李廷安、呂知柔為典謁。從襲等素惡崇韜,又見崇韜專任軍事,益不平之。及破蜀,蜀之貴臣大將,自王宗弼已下,皆爭以蜀寶貨,妓樂奉崇韜父子,而魏
 王所得,匹馬、束帛、唾壺、麈柄而已;崇韜日決軍事,將吏賓客趨走盈庭,而都統府惟大將晨謁,牙門闐然。由是從襲等不勝其憤。已而宗弼率蜀人見繼岌,請留崇韜鎮蜀,從襲等因言崇韜有異志,勸繼岌為備。繼岌謂崇韜曰:「陛下倚侍中如衡、華,尊之廟堂之上,期以一天下而制四夷,必不棄元老於蠻夷之地。此事非予敢知也。」



 莊宗聞崇韜欲留蜀,亦不悅,遣宦者向延嗣趣繼岌班師。延嗣至成都,崇韜不出迎,及見,禮益慢,延嗣怒,從襲等因告延嗣崇韜有異志,恐危魏王。延嗣還,具言之。劉皇后涕泣請保全繼岌,莊宗遣宦官馬彥珪往視崇韜
 去就。是時,兩川新定,孟知祥未至,所在盜賊聚山林,崇韜方遣任圜等分出招集,恐後生變,故師未即還。而彥珪將行,見劉皇后曰:「臣見延嗣言蜀中事勢已不可,禍機之作,間不容髮,安能三千里往復廩命乎!」劉皇后以彥珪語告莊宗,莊宗曰:「傳言未審,豈可便令果決?」皇后不得請,因自為教與繼岌,使殺崇韜。明年正月,崇韜留任圜守蜀,以待知祥之至,崇韜期班師有日。彥珪至蜀,出皇后教示繼岌,繼岌曰:「今大軍將發,未有釁端,豈可作此負心事!」從襲等泣曰:「今有密敕,王茍不行,使崇韜知之,則吾屬無類矣!」繼岌曰:「上無詔書,徒以皇后手
 教,安能殺招討使?」從襲等力爭,繼岌不得已而從之。詰旦,從襲以都統命召崇韜,繼岌登樓以避之。崇韜入,升階,繼岌從者李環撾碎其首。



 繼岌遂班師。二月,軍至泥溪,先鋒康延孝叛,據漢州,繼岌遣任圜討平之。



 四月辛卯,至興平,聞明宗反,兵入京師,繼岌欲退保鳳翔。至武功,李從襲勸繼岌馳趨京師,以救內難。行至渭河,西都留守張抃斷浮橋,繼岌不得度,乃循河而東,至渭南,左右皆潰。從襲謂繼岌曰:「大事已去,福不可再,王宜自圖。」繼岌徘徊泣下,謂李環曰:「吾道盡途窮,子當殺我。」環遲疑久之,謂繼岌乳母曰:「吾不忍見王,王若無路求生,當
 踣面以俟。」繼岌面榻而臥,環縊殺之。任圜從後至,葬繼岌華州之西南。繼岌少病閹,無子。明宗已即位,圜率征蜀之師二萬至京師,明宗撫慰久之,問圜繼岌何在,圜具言繼岌死狀。



 同光三年,詔以皇子繼嵩、繼潼、繼蟾、繼嶢皆為光祿大夫,檢校司徒。蓋其皆幼,故不封。當莊宗遇弒時,太祖子孫在者十有一人,明宗入立,其四人見殺,其餘皆不知所終,太祖之後遂絕。



\end{pinyinscope}