\article{卷八晉本紀第八}

\begin{pinyinscope}

 高祖聖文章武明德孝皇帝,其父臬捩雞,本出於西夷,自朱邪歸唐,從朱邪入居陰山。其後,晉王李克用起於雲、朔之間,臬捩雞以善騎射,常從晉王征伐有功,官至洺州刺史。臬捩雞生敬瑭,其姓石氏,不知其得姓之始也。



 敬瑭為人沈厚寡言,明宗愛之,妻以女,是為永寧公主,由是常隸明宗帳下,號左射軍。莊宗已得魏,梁將劉掞急攻清平,莊宗馳救之。兵未及陣,為掞所掩,敬瑭以
 十餘騎橫槊馳擊,取之以旋。莊宗拊其背而壯之,手啗以酥,啗酥,夷狄所重,由是名動軍中。十五年,莊宗戰於胡柳,前鋒周德威戰死,敬瑭以左射軍從明宗復擊敗梁兵。明宗戰胡盧套、楊村,為梁兵所敗,敬瑭常脫明宗於危。



 趙在禮之亂,明宗討之,至魏而兵變,明宗初欲自歸於天子,明己所以不反者。



 敬瑭獻計曰:「豈有軍變於外,上將獨無事者乎?且猶豫者兵家大忌,不如速行。



 願得騎兵三百先攻汴州,夷門天下之要害也,得之可以成事。」明宗然之,與之驍騎三百,渡黎陽為前鋒,明宗遂入汴。莊宗自洛後至,不得入,而兵皆潰去。莊宗西還,明
 宗以敬瑭為前鋒趣汜水,且收其散卒。莊宗遇弒,明宗入立,拜敬瑭保義軍節度使,賜號「竭忠建策興復功臣」,兼六軍諸衛副使。在陜為政以廉聞。是時,諸侯多不奉法,鄧州陶、亳州李鄴皆以贓污論死,明宗下詔書褒廉吏普州安崇阮、洺州張萬進、耀州孫岳等以諷天下,而以敬瑭為首。



 天成二年十月,從幸汴州,為御營使,拜宣武軍節度使、侍衛親軍馬步軍都指揮使,六軍副使如故,改賜「耀忠匡定保節功臣。」三年四月,徙鎮天雄,拜同中書門下平章事、興唐尹。五月,拜駙馬都尉。董璋反東川,為行營都招討使,不克而還。復兼六軍諸衛副使。
 徙鎮河陽三城,未行,而契丹、吐渾、突厥皆入寇,是時,秦王從榮統六軍,敬瑭疑其必及禍,不欲為其副,乃自請行。及制出,不落副使,輒復辭行。明宗數責大臣問誰可行者,范延光、趙延壽等卒以敬瑭為請,乃拜河東節度使、大同彰國振武威塞等軍蕃漢馬步軍總管,落六軍副使,乃行。



 明年,明宗崩,愍帝即位,加中書令。三月,徙鎮成德。清泰元年五月,復鎮太原,來朝京師。潞王從珂反於鳳翔,愍帝出奔,遇敬瑭于道。敬瑭殺帝從者百餘人,幽帝于衛州而去。廢帝即位,疑敬瑭必反。



 天福元年五月,徙鎮天平,敬瑭果不受命,謂其屬曰:「先
 帝授吾太原使老焉,今無故而遷,是疑吾反也。且太原地險而粟多,吾當內檄諸鎮,外求援於契丹,可乎?」桑維翰、劉知遠等共以為然。乃上表論廢帝不當立,請立許王從益為明宗嗣。



 廢帝下詔削奪敬瑭官爵,命張敬達等討之,敬瑭求援於契丹。九月,契丹耶律德光入自雁門,與唐兵戰,敬達大敗。敬瑭夜出北門見耶律德光,約為父子。



 十一月丁酉,皇帝即位,國號晉。以幽、涿、薊、檀、順、瀛、莫、蔚、朔、雲、應、新、媯、儒、武、寰州入於契丹。己亥,大赦,改元。掌書記桑維翰為翰林學士、尚
 書禮部侍郎,知樞密使事。閏月丙寅,翰林學士承旨、尚書戶部侍郎趙瑩為門下侍郎,桑維翰為中書侍郎:同中書門下平章事,兼樞密使。甲戌,趙德鈞及其子延壽叛于唐來降,契丹鎖之以歸。己卯,次河陽,節度使萇從簡叛于唐來降。



 辛巳,至自太原。盧文紀、姚顗罷。甲申,大赦,殺張延朗、劉延朗,赦房暠。十二月乙酉,如河陽。追降王從珂為庶人。丁亥,司空馮道兼門下侍郎、同中書門下平章事。己丑,曹州指揮使石重立殺其刺史鄭玩。辛卯,御札求直言。癸巳,鎮州牙內都虞候祕瓊逐其節度副使李彥琦。同州裨將門鐸殺其將
 楊漢賓。庚子,天平軍節度使王建立殺其副使李彥贇。旱。



 二年春正月癸亥,安遠軍節度使盧文進叛降於吳。丁卯,天雄軍節度使范延光殺齊州防禦使祕瓊。戊寅,兵部侍郎李崧為中書侍郎、同中書門下平章事,樞密使。



 封唐宗室子為公,及隋酅公為二王後,以周介公備三恪。二月丁西,契丹使太子解里來。三月庚辰,如汴州。夏四月丁亥,赦囚,蠲民租賦。趙瑩使于契丹。辛卯,宣武軍節度使楊光遠進助國錢。契丹使宮苑使李可興來。五月壬戌,御札求直
 言。



 丁丑,追尊祖考為皇帝,妣為皇后:高祖璟謚曰孝安,廟號靖祖,祖妣秦氏謚曰孝安元;曾祖彬謚曰孝簡,廟號肅祖,祖妣安氏謚曰孝簡恭;祖昱謚曰孝平,廟號睿祖,祖妣來氏謚曰孝平獻;考紹雍謚曰孝元,廟號獻祖,妣何氏謚曰孝元懿。六月癸未,契丹使夷離畢來。天雄軍節度使范延光反。丁酉,傳箭于義成軍節度使符彥饒。丁未,楊光遠為魏府四面行營都部署。東都巡檢張從賓反,留守判官李遐死之,奉國都指揮使侯益、護聖都指揮使杜重威討之。從賓寇河陽,殺皇子重乂;寇河南,殺皇子重信。秋七月,從賓陷汜水關,殺巡檢使宋廷
 浩。壬子,右衛大將軍尹暉叛奔于吳,不克,伏誅。右監門衛大將軍婁繼英叛降於張從賓。義成軍亂,殺戍將侍衛馬步軍都指揮使白奉進。甲寅,戍將奉國指揮使馬萬執符彥饒歸於京師,命殺之于赤岡。乙卯,楊光遠為魏府行營都招討使。辛酉,杜重威克汜水關。壬申,楊光遠克博州。丙子,安州屯防指揮使王暉殺其節度使周瑰,右衛上將軍李金全討之。



 八月丙申,靜難軍節度使安叔千進添都馬。乙巳,赦非死罪囚及張從賓、符彥饒、王暉餘黨。九月,楊光遠進粟。
 冬十月辛巳,禁造甲兵。



 三年春二月戊戌,諸鎮皆進物以助國。三月壬戌,回鶻可汗王仁美使翟全福來。



 丁丑,禁私造銅器。秋七月辛酉,以皇業錢作受命寶。八月戊寅,馮道及左僕射劉昫為契丹冊禮使。壬午,澶州刺史馮暉降。丙戌,許御署官選。己丑,蠲水旱民稅。



 辛丑,歸伶官於契丹。九月己酉,赦范延光。己未,歸靜鞭官劉守威,金吾勘契官王殷、司天雞叫學生殷暉于契丹。于闐
 使馬繼榮來,回鶻使李萬金來。己巳,赦魏州,蠲民稅。是月,宣徽南院使劉處讓為樞密使。冬十月戊寅,契丹使中書令韓頻來奉冊曰英武明義皇帝。庚辰,升汴州為東京,以洛陽為西京,雍州為晉昌軍。戊子,右金吾衛大將軍馬從斌使于契丹。己未,契丹使梅里來。戊戌,大赦。庚子,封李聖天為大寶于闐國王。十一月辛亥,升廣晉府為鄴都。壬戌,除鑄錢令。十二月丙子,封子重貴為鄭王。



 四年春正月,盜發唐愍皇帝墓。辛亥,澶州防禦使張
 從恩為樞密副使。旌表深州民李自倫門閭。三月乙巳,回鶻使其都督拽里敦來。丙辰,頒《調元歷》。靈州戍將王彥忠以懷遠城反。己未,彥忠降,供奉官齊延祚殺之。夏四月辛巳,封回鶻可汗王仁美為奉化可汗。甲申,廢樞密使。秋七月丙辰,復禁鑄錢。閏月壬申,桑維翰罷。八月己亥朔,河決博平。西戎寇涇州,彰義軍節度使張彥澤敗之,執其首領野離羅蝦獨。九月丁丑,契丹使粘木孤來。癸未,封李從益為郇國公以奉唐後。



 丙戌,高麗王建使其廣評侍郎邢順來。冬十一月乙亥,立唐高祖、太宗、
 莊宗、明宗、愍帝廟于西京。戊子,契丹使遙折來,吐蕃罷延族來附。



 五年春正月丁卯朔,德音除民公私債。己丑,回鶻使石海金來。夏四月甲子,契丹興化王來。五月丙戌,安遠軍節度使李金全叛附于唐。六月癸卯,李昪遣其將李承裕入于安州,金全奔于唐,安遠軍節度使馬全節及承裕戰,敗之。丁巳,克安州,承裕奔于雲夢,全節執而殺之。秋八月丁酉,閱稼于西郊。己未,西京留守楊光遠殺太子太師范延光。九月丁卯,翰林學士承旨、戶部侍郎和凝為中書侍郎、同中書門下平章事。辛巳,閱稼于沙臺。
 冬十月丁未,契丹使舍利來。十一月丙子,冬至,始用二舞。



 六年春正月戊寅,封唐叔虞為興安王,臺駘為昌寧公。二月戊申,停買宴錢。



 三月,除民二年至四年以前稅。夏四月己未,契丹使述括來。五月,吐渾首領白承福來。秋七月壬午,突厥使薛同海來。八月壬辰,如鄴都,開封尹鄭王重貴留守東京,宣徽南院使張從恩東京內外兵馬都監。壬寅,大赦。甲寅,光祿卿張澄使于契丹。九月乙亥,前安國軍節度使楊彥詢使于契丹。丁丑,吐渾使白可久來。河決中都,入于沓河。冬十月,河決滑、
 濮、鄆、澶州。山南東道節度使安從進反。十一月丁丑,西京留守高行周為南面軍前都部署以討之。十二月丙戌朔,鄭王重貴為廣晉尹,徙封齊王。先鋒都指揮使郭金海及安從進戰於唐州,敗之。成德軍節度使安重榮反。天平節度使杜重威為鎮州行營招討使。丙申,契丹遣使者來。戊戌,杜重威及安重榮戰於宗城,敗之。



 七年春正月丁巳,克鎮州,安重榮伏誅,赦廣晉。庚午,契丹使達剌來。三月,歸德軍節度使安彥威塞決河於滑州。閏月,天興蝗食麥。夏五月乙巳,尊皇太妃劉氏為太后。六月丙辰,吐渾使念丑漢來。乙丑,皇帝崩
 於保昌殿。



\end{pinyinscope}