\article{卷三十八宦者傳第二十六}

\begin{pinyinscope}

 嗚呼,自古宦、女之禍深矣!明者未形而知懼,暗者患及而猶安焉,至於亂亡而不可悔也。雖然,不可以不戒。作《宦者傳》。



 張承業張承業,字繼元,唐僖宗時宦者也。本姓康,幼閹,為內常侍張泰養子。晉王兵擊王行瑜,承業數往來兵間,晉王喜其為人。及昭宗為李茂貞所迫,將出奔太原,乃先遣承業使晉以道意,因以為河東監軍。其後崔胤誅宦官,
 宦官在外者,悉詔所在殺之。晉王憐承業,不忍殺,匿之斛律寺。昭宗崩,乃出承業,復為監軍。



 晉王病且革,以莊宗屬承業曰:「以亞子累公等。」莊宗常兄事承業,歲時升堂拜母,甚親重之。莊宗在魏,與梁戰河上十餘年,軍國之事,皆委承業,承業亦盡心不懈。凡所以畜積金粟,收市兵馬,勸課農桑,而成莊宗之業者,承業之功為多。自貞簡太后、韓德妃、伊淑妃及諸公子在晉陽者,承業一切以法繩之,權貴皆斂手畏承業。



 莊宗歲時自魏歸省親,須錢蒲博、賞賜伶人,而承業主藏,錢不可得。莊宗乃置酒庫中,酒酣,使子繼岌為承業起舞,舞罷,承業出寶
 帶、幣、馬為贈,莊宗指錢積呼繼岌小字以語承業曰:「和哥乏錢,可與錢一積,何用帶、馬為也?」承業謝曰:「國家錢,非臣所得私也。」莊宗以語侵之,承業怒曰:「臣,老敕使,非為子孫計,惜此庫錢,佐王成霸業爾!若欲用之,何必問臣?財盡兵散,豈獨臣受禍也?」莊宗顧元行欽曰:「取劍來!」承業起,持莊宗衣而泣,曰:「臣受先王顧託之命,誓雪家國之讎。今日為王惜庫物而死,死不愧於先王矣!」閻寶從旁解承業手令去,承業奮拳毆寶踣,罵曰:「閻寶,朱溫之賊,蒙晉厚恩,不能有一言之忠,而反諂諛自容邪!」太后聞之,使召莊宗。莊宗性至孝,聞太后召,甚懼,乃酌兩
 卮謝承業曰:「吾杯酒之失,且得罪太后。願公飲此,為吾分過。」承業不肯飲。莊宗入內,太后使人謝承業曰:「小兒忤公,已笞之矣。」明日,太后與莊宗俱過承業第,慰勞之。



 盧質嗜酒傲忽,自莊宗及諸公子多見侮慢,莊宗深嫉之。承業乘間請曰:「盧質嗜酒無禮,臣請為王殺之。」莊宗曰:「吾方招納賢才以就功業,公何言之過也!」



 承業起賀曰:「王能如此,天下不足平也!」質因此獲免。



 天祐十八年,莊宗已諾諸將即皇帝位。承業方臥病,聞之,自太原肩輿至魏,諫曰:「大王父子與梁血戰三十年,本欲雪家國之讎,而復唐之社稷。今元兇未滅,而遽以尊名自居,非
 王父子之初心,且失天下望,不可。」莊宗謝曰:「此諸將之所欲也。」承業曰:「不然,梁,唐、晉之仇賊,而天下所共惡也。今王誠能為天下去大惡,復列聖之深讎,然後求唐後而立之。使唐之子孫在,孰敢當之?使唐無子孫,天下之士,誰可與王爭者?臣,唐家一老奴耳,誠願見大王之成功,然後退身田里,使百官送出洛東門,而令路人指而歎曰『此本朝敕使,先王時監軍也』,豈不臣主俱榮哉?」莊宗不聽。承業知不可諫,乃仰天大哭曰:「吾王自取之!誤我奴矣。」肩輿歸太原,不食而卒,年七十七。同光元年,贈左武衛上將軍,謚曰正憲。



 張居翰張居翰,字德卿,故唐掖廷令張從玫之養子。昭宗時,為范陽軍監軍,與節度使劉仁恭相善。天復中,大誅宦者,仁恭匿居翰大安山之北谿以免。其後,梁兵攻仁恭,仁恭遣居翰從晉王攻梁潞州以牽其兵,晉遂取潞州,以居翰為昭義監軍。莊宗即位,與郭崇韜並為樞密使。莊宗滅梁而驕,宦官因以用事,郭崇韜又專任政,居翰默默,茍免而已。魏王破蜀,王衍朝京師,行至秦川,而明宗軍變于魏。莊宗東征,慮衍有變,遣人馳詔魏王殺之。詔書已印畫,而居翰發視之,詔書言「誅衍一行」,居翰以謂殺降不祥,乃以詔傅柱,揩去「行」字,改為一「家」。時蜀降人
 與衍俱東者千餘人,皆獲免。莊宗遇弒,居翰見明宗于至德宮,求歸田里。天成三年,卒於長安,年七十一。



 五代文章陋矣,而史官之職廢於喪亂,傳記小說多失其傳,故其事迹,終始不完,而雜以訛繆。至於英豪奮起,戰爭勝敗,國家興廢之際,豈無謀臣之略,辯士之談?而文字不足以發之,遂使泯然無傳於後世。然獨張承業事卓卓在人耳目,至今故老猶能道之。其論議可謂傑然歟!殆非宦者之言也。



 自古宦者亂人之國,其源深於女禍。女,色而已;宦者之害,非一端也。蓋其用事也近而習,其為心也專而忍。能以小善中人之意,小信固人之
 心,使人主必信而親之。待其已信,然後懼以禍福而把持之。雖有忠臣碩士列於朝廷,而人主以為去己疏遠,不若起居飲食、前後左右之親為可恃也。故前後左右者日益親,則忠臣碩士日益疏,而人主之勢日益孤。勢孤,則懼禍之心日益切,而把持者日益牢。安危出其喜怒,禍患伏於帷闥,則嚮之所謂可恃者,乃所以為患也。患已深而覺之,欲與疏遠之臣圖左右之親近,緩之則養禍而益深,急之則挾人主以為質,雖有聖智不能與謀,謀之而不可為,為之而不可成,至其甚,則俱傷而兩敗。故其大者亡國,其次亡身,而使姦豪得借以為資而
 起,至抉其種類,盡殺以快天下之心而後已。此前史所載宦者之禍常如此者,非一世也。夫為人主者,非欲養禍於內而疏忠臣碩士於外,蓋其漸積而勢使之然也。夫女色之惑,不幸而不悟,則禍斯及矣,使其一悟,捽而去之可也。宦者之為禍,雖欲悔悟,而勢有不得而去也,唐昭宗之事是已。故曰深於女禍者,謂此也。可不戒哉!昭宗信狎宦者,由是有東宮之幽。既出而與崔胤圖之,胤為宰相,顧力不足為,乃召兵於梁。梁兵且至,而宦者挾天子走之岐。



 梁兵圍之三年,昭宗既出,而唐亡矣。



 初,昭宗之出也,梁王悉誅唐宦者第五可範等七百餘人,
 其在外者,悉詔天下捕殺之,而宦者多為諸鎮所藏匿而不殺。是時,方鎮僭擬,悉以宦官給事,而吳越最多。及莊宗立,詔天下訪求故唐時宦者悉送京師,得數百人,宦者遂復用事,以至於亡。此何異求已覆之車,躬駕而履其轍也?可為悲夫!



 莊宗未滅梁時,承業已死。其後居翰雖為樞密使,而不用事。有宣徽使馬紹宏者,嘗賜姓李,頗見信用。然誣殺大臣,黷貨賂,專威福,以取怨於天下者,左右狎暱,黃門內養之徒也。是時,明宗自鎮州入覲,奉朝請於京師。莊宗頗疑其有異志,陰遣紹宏伺其動靜,紹宏反以情告明宗。明宗自魏而反,天下皆知禍
 起於魏,孰知其啟明宗之二心者,自紹宏始也!郭崇韜已破蜀,莊宗信宦者言而疑之。然崇韜之死,莊宗不知,皆宦者為之也。當此之時,舉唐之精兵皆在蜀,使崇韜不死,明宗入洛,豈無西顧之患?其能晏然取唐而代之邪?及明宗入立,又詔天下悉捕宦者而殺之。宦者亡竄山谷,多削髮為浮圖。其亡至太原者七十餘人,悉捕而殺之都亭驛,流血盈庭。



 明宗晚而多病,王淑妃專內以干政,宦者孟漢瓊因以用事。秦王入視明宗疾已革,既出而聞哭聲,以謂帝崩矣,乃謀以兵入宮者,懼不得立也。大臣朱弘昭等方圖其事,議未決,漢瓊遽入見明宗,
 言秦王反,即以兵誅之,陷秦王大惡,而明宗以此飲恨而終。後愍帝奔于衛州,漢瓊西迎廢帝于路,廢帝惡而殺之。



 嗚呼!人情處安樂,自非聖哲,不能久而無驕怠。宦、女之禍非一日,必伺人之驕怠而浸入之。明宗非佚君,而猶若此者,蓋其在位差久也。其餘多武人崛起,及其嗣續,世數短而年不永,故宦者莫暇施為。其為大害者,略可見矣。獨承業之論,偉然可愛,而居翰更一字以活千人。君子之於人也,茍有善焉,無所不取,吾於斯二人者,有所取焉。取其善而戒其惡,所謂「愛而知其惡,憎而知其
 善」也。



 故並述其禍敗之所以然者著於篇。



\end{pinyinscope}