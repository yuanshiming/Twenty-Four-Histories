\article{卷六十九南平世家第九}

\begin{pinyinscope}

 高季興,字貽孫,陜州
 硤
 石人也。本名季昌,避後唐獻祖廟諱,更名季興。季興少為汴州富人李讓家僮。梁太祖初鎮宣武,讓以入貲得幸,養為子,易其姓名曰朱友讓。季興以友讓故得進見,太祖奇其材,命友讓以子畜之,因冒姓朱氏,補制勝軍使,遷毅勇指揮使。



 天復二年,梁兵攻鳳翔,李茂貞堅壁不出,太祖議欲收軍還河中,季興獨進曰:「天下豪傑窺此舉者一歲矣,今岐人已憊,破
 在旦夕,而大王之所慮者,閉壁以老我師,此可以誘致之也。」太祖壯其言,命季興募勇敢士,得騎士馬景,季興授以計,引見太祖。景曰:「此行無還理,願錄其後嗣。」太祖惻然止之,景固請,乃行。景以數騎馳叩城門告曰:「梁兵將東,前鋒去矣。」岐人以為然,開門出追梁軍,梁兵隨景後以進,殺其九千餘人,景死之。茂貞後與梁和,昭宗出,贈景官,謚曰忠壯。季興由是知名。明年,拜宋州刺史。從破青州,徙潁州防禦使,復姓高氏。



 當唐之末,襄州趙匡凝襲破雷彥恭於荊南,以其弟匡明為留後。梁兵攻破襄州,匡凝奔於吳,匡明奔于蜀,乃以季興為荊南節度
 觀察留後。開平元年,拜季興節度使。二年,加同中書門下平章事。荊南節度十州,當唐之末,為諸道所侵,季興始至,江陵一城而已,兵火之後,井邑凋零。季興招緝綏撫,人士歸之,乃以倪可福、鮑唐為將帥,梁震、司空熏、王保義等為賓客。



 太祖崩,季興見梁日以衰弱,乃謀阻兵自固,治城隍,設樓櫓。以兵攻歸、峽,為蜀將王宗壽所敗。又發兵聲言助梁擊晉,以侵襄州,為孔勍所敗,乃絕貢賦累年。



 梁末帝優容之,封季興渤海王,賜以袞冕劍佩。貞明三年,始復修貢。



 梁亡,唐莊宗入洛,下詔慰諭季興,司空薰等皆勸季興入朝京師,梁震以為不可,曰:「梁、唐
 世為仇敵,夾河血戰垂二十年,今主上新滅梁,而大王梁室故臣,握彊兵,居重鎮,以身入朝,行為虜爾。」季興不聽,留其二子,以騎士三百為衛,朝于洛陽。莊宗果欲留之,郭崇韜諫曰:「唐新滅梁得天下,方以大信示人,今四方諸侯相繼入貢,不過遣子弟將吏,而季興以身述職,為諸侯率,宜加恩禮,以諷動來者。而反縻之,示天下以不廣,且絕四方內向之意,不可。」莊宗乃止,厚禮而遣之。莊宗嘗問季興曰:「吾已滅梁,欲征吳、蜀,何者為先?」季興曰:「宜先蜀,臣請以本道兵先進。」莊宗大悅,以手拊其背,季興因命工繡其手迹於衣,歸以為榮耀。季興已去,莊
 宗心悔遣之,密詔襄州劉訓圖之。季興行至襄州,心動,夜斬關而出。已去,而詔書夜至。季興歸而謂梁震曰:「不聽子言,幾不免。」因曰:「吾行有二失;來朝一失,放還一失。且主上百戰以取河南,對功臣誇手抄《春秋》;又曰:『我於手指上得天下。』其自矜伐如此。而荒于遊畋,政事多廢,吾可無慮矣。」同光三年,封南平王。魏王繼岌已破蜀,得蜀金帛四十餘萬,自峽而下,而莊宗之難作。季興聞京師有變,乃悉邀留蜀物,而殺其使者韓珙等十餘人。



 初,唐兵伐蜀,季興請以本道兵自取夔、忠、萬、歸、峽等州,乃以季興為峽路東南面招討使,而季興未嘗出兵。魏王
 已破蜀,而明宗入立,季興因請夔、忠等州為屬郡,唐大臣以為季興請自取之,而兵出無功,不與。季興屢請,雖不得已而與之,而唐猶自除刺史,季興拒而不納。明宗乃以襄州劉訓為招討使,攻之,不克,而唐別將西方鄴克其夔、忠、萬三州,季興遂以荊、歸、峽三州臣於吳,吳冊季興秦王。天成三年冬卒,年七十一,謚曰武信。季興子九人,長子從誨立。



 從誨字遵聖。季興時,入梁為供奉官,累遷鞍轡庫使,賜告歸寧,季興遂留為馬步軍都指揮使、行軍司馬。季興卒,吳以從誨為荊南節度使。從誨以父自絕于唐,懼復
 見討,乃遣使者聘于楚,楚王馬殷為之請命于唐,而從誨亦遣押衙劉知謙奉表自歸,進贖罪銀三千兩,明宗納之。長興元年正月,拜從誨節度使,追封季興楚王,謚曰武信。三年,封從誨渤海王。應順元年,封南平王。



 從誨為人明敏,多權詐。晉高祖遣翰林學士陶穀為從誨生辰國信使,從誨宴穀望沙樓,大陳戰艦于樓下,謂穀曰:「吳、蜀不賓久矣,願修武備,習水戰,以待師期。」穀還,具道其語,晉高祖大喜,復遣使賜以甲馬百匹。襄州安從進反,結從誨為援,從誨外為拒絕,陰與之通。晉師致討,從誨遣將李端以舟師為應,從進誅,從誨求郢州為屬郡,
 高祖不許。



 契丹滅晉,漢高祖起太原,從誨遣人間道奉表勸進,且言漢得天下,願乞郢州為屬,漢高祖陽諾之。高祖入汴,從誨遣使朝貢,因求郢州,高祖不與。從誨怒,發兵攻郢州,為刺史尹實所敗。漢遣國子祭酒田敏使于楚,假道荊南,從誨問敏中國虛實,以為契丹之後,兵食皆殫,意欲以誚敏,敏為言:「杜重威悉以晉戈甲降虜,虜置之鎮州,未嘗以北,而晉兵皆漢有也。」從誨不悅。敏以印本《五經》遺從誨,從誨謝曰:「予之所識,不過《孝經》十八章爾。」敏曰:「至德要道,於此足矣。」敏因誦《諸侯章》曰:「在上不驕,高而不危,制節謹度,滿而不溢。」



 從誨以為譏己,
 即以大卮罰敏。



 荊南地狹兵弱,介於吳、楚,為小國。自吳稱帝,而南漢、閩、楚皆奉梁正朔,歲時貢奉,皆假道荊南。季興、從誨常邀留其使者,掠取其物,而諸道以書責誚,或發兵加討,即復還之而無媿。其後南漢與閩、蜀皆稱帝,從誨所嚮稱臣,蓋利其賜予。俚俗語謂奪攘茍得無愧恥者為賴子,猶言無賴也,故諸國皆目為「高賴子」。



 從誨自求郢州不得,遂自絕於漢。逾年,復通朝貢。乾祐元年十月卒,年五十八,贈尚書令,謚曰文獻。子保融立。從誨十五子,長曰保勳,次保正,保融第三子也,不知其得立之因。



 保融字德長。從誨時,為節度副使,兼峽州刺史。從誨卒,拜節度使。廣順元年,封渤海郡王。顯德元年,進封南平王。世宗征淮,保融遣指揮使魏璘率兵三千,出夏口以為應。又遣客將劉扶奉箋南唐,勸其內附。李景稱臣,世宗得保融所與箋,大喜,賜以絹百匹。荊南自後唐以來,常數歲一貢京師,而中間兩絕。及世宗時,無歲不貢矣。保融以謂器械金帛,皆土地常產,不足以效誠節,乃遣其弟保紳來朝,世宗益嘉之。



 初,季興之鎮,梁以兵五千為牙兵,衣食皆給於梁。至明宗時,歲給以鹽萬三千石,後不復給。及世宗平淮,故命泰州給之。



 保融性迂緩,無材
 能,而事無大小,皆委其弟保勖。其從叔從義謀為亂,為其徒高知訓所告,徙之松滋而殺之。宋興,保融懼,一歲之間三入貢。建隆元年,以疾卒,年四十一,贈太尉,謚曰貞懿。弟保勖立。



 保勖字省躬,從誨第十子也。保融卒,拜節度使。三年,保勖疾,謂其將梁延嗣曰:「我疾遂不起,兄弟孰可付之後事者?」延嗣曰:「公不念貞懿王乎?先王寢疾,以軍府付公,今先王子繼沖長矣。」保勖曰:「子言是也。」即以繼沖判內外兵馬。十一月,保勖卒,年三十九,贈侍中。保融之子繼沖立。



 繼沖字成和。保勖卒,拜節度使。湖南周行逢卒,子保權立,其將張文表作亂,建隆四年,太祖命慕容延釗等討之。延釗假道荊南,約以兵過城外。繼沖大將李景威曰:「兵尚權譎,城外之約,不可信也。宜嚴兵以待之。」判官孫光憲叱之曰:「汝峽江一民爾,安識成敗!且中國自周世宗時,已有混一天下之志,況聖宋受命,真主出邪!王師豈易當也!」因勸繼沖去斥候,封府庫以待,繼沖以為然。景威出而歎曰:「吾言不用,大事去矣,何用生為!」因扼吭而死。延釗軍至,繼沖出逆于郊,而前鋒遽入其城。繼沖亟歸,見旌旗甲馬,布列衢巷,大懼,即詣延釗納牌印,太
 祖優詔復命繼沖為節度使。



 乾德元年,有事於南郊,繼沖上書願陪祠。九月,具文告三廟,率其將吏宗族五百餘人朝於京師,拜武寧軍節度使以卒。光憲拜黃州刺史,其後事具國史。



 季興興滅年世甚明,諸書皆同,蓋自梁開平元年鎮荊南,至皇朝乾德元年國除,凡五
 十
 七年。



\end{pinyinscope}