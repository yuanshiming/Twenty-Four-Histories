\article{卷四唐本紀第四}

\begin{pinyinscope}

 莊宗光聖神閔孝皇
 帝,其
 先本號朱邪,蓋出於西突厥,至其後世,別自號曰沙陀,而朱邪為姓。唐德宗時,有朱邪盡忠者,居於北庭之金滿州。貞元中,吐蕃贊普攻陷北庭,徙盡忠於甘州而役屬之。其後贊普為回鶻所敗,盡忠與其子執宜東走。



 贊普怒,追之,及于石門關。盡忠戰死,執宜獨走歸唐,居之鹽州,以隸河西節度使范希朝。希朝徙鎮太原,執宜從之,居之定襄神武川之新
 城。其部落萬騎,皆驍勇善騎射,號「沙陀軍」。執宜死,其子曰赤心。懿宗咸通十年,神策大將軍康承訓統十八將討龐勛於徐州,以朱邪赤心為太原行營招討沙、陀三部落軍使。以從破勛功,拜單于大都護、振武軍節度使,賜姓名曰李國昌,以之屬籍。



 沙陀素強,而國昌恃功益橫恣,懿宗患之。十三年,徙國昌雲州刺史、大同軍防禦使,國昌稱疾拒命。國昌子克用,尤善騎射,能仰中雙鳧,為雲州守捉使。國昌已拒命,克用乃殺大同軍防御使段文楚,據雲州,自稱留後。唐以太僕卿盧簡方為振武節度使,會幽、並兵討之。簡方行至嵐州,軍潰,由是沙陀侵
 掠代北,為邊患矣。明年,僖宗即位,以謂前太原節度使李業遇沙陀有恩,而業已死,乃以其子鈞為靈武節度使、宣慰沙陀六州三部落使,以招緝之。拜克用大同軍防禦使。



 居久之,國昌出擊黨項,吐渾赫連鐸襲破振武。克用聞之,自雲州往迎國昌,而雲州人亦閉關拒之。國昌父子無所歸,因掠蔚、朔間,得兵三千,國昌入保蔚州,克用還據新城。僖宗乃拜鐸大同軍使,以李鈞為代北招討使,以討沙陀。乾符五年,沙陀破遮虜軍,又破苛嵐軍,而唐兵數敗,沙陀由此益熾,北據蔚、朔,南侵忻、代、嵐、石,至于太谷焉。廣明元年,招
 討使李琢會幽州李可舉、雲州赫連鐸擊沙陀,克用與可舉相拒雄武軍。其叔父友金以蔚、朔州降于琢,克用聞之,遽還。可舉追至藥兒嶺,大敗之,琢軍夾擊,又敗之于蔚州。沙陀大潰,克用父子亡入達靼。



 克用少驍勇,軍中號曰「李鴉兒」;其一目眇,及其貴也,又號「獨眼龍」,其威名蓋於代北。其在達靼,久之,鬱鬱不得志,又常懼其圖己,因時時從其群豪射獵,或掛針於木,或立馬鞭,百步射之輒中,群豪皆服以為神。



 黃巢已陷京師,中和元年,代北起軍使陳景思發沙陀先所降者,與吐渾、安慶等萬人赴京師,行至絳州,沙陀軍亂,大掠而還。景思念沙
 陀非克用不可將,乃以詔書召克用於達靼,承制以為代州刺史、鴈門以北行營節度使。率蕃漢萬人出石嶺關,過太原,求發軍錢。節度使鄭從讜與之錢千緡、米千石,克用怒,縱兵大掠而還。二年十一月,景思、克用復以步騎萬七千赴京師。三年正月,出于河中,進屯乾坑。巢黨驚曰:「鴉兒軍至矣!」二月,敗巢將黃鄴於石隄谷;三月,又敗趙璋、尚讓於良田坡,橫尸三十里。是時,諸鎮兵皆會長安,大戰渭橋,賊敗走入城,克用乘勝追之,自光泰門先入,戰望春宮升陽殿,巢敗,南走出藍田關。京師平,克用功第一。天子拜克用檢校司空、同中書門下平章
 事、河東節度使,以國昌為鴈門以北行營節度使。十月,國昌卒。



 十一月,遣其弟克修攻昭義孟方立,取其澤、潞二州。方立走山東,以邢、洺、磁三州自別為昭義軍。黃巢南走至蔡州,降秦宗權,遂攻陳州。四年,克用以兵五萬救陳州,出天井關,假道河陽,諸葛爽不許,乃自河中渡河。四月,敗尚讓於太康,又敗黃鄴于西華。巢且走且戰,至中牟,臨河未渡,而克用追及之,賊眾驚潰。



 比至封丘,又敗之,巢脫身走,克用追之,一日夜馳三百里,至於冤朐,不及而還。



 洺、磁孟氏據之,故當時有兩昭義。



 過汴州,休軍封禪寺,朱全忠饗克用於上源驛。夜,酒
 罷,克用醉臥,伏兵發,火起,侍者郭景銖滅燭,匿克用床下,以水醒面而告以難。會天大雨滅火,克用得從者薛鐵山、賀回鶻等,隨電光,縋尉氏門出,還軍中。七月,至於太原,訟其事于京師,請加兵於汴,遣弟克修將兵萬人屯于河中以待。僖宗和解之,用破巢功,封克用隴西郡王。



 光啟元年,河中王重榮與宦者田令孜有隙,徙重榮兗州,以定州王處存為河中節度使,詔克用以兵護處存之鎮。重榮使人紿克用曰:「天子詔重榮,俟克用至,與處存共誅之。」因偽為詔書示克用曰:「此朱全忠之謀也。」克用信之,八上表請討全忠,僖宗不許,克
 用大怒。重榮既不肯徙,僖宗遣邠州硃玫、鳳翔李昌符討之。克用反以兵助重榮,敗玫於沙苑,遂犯京師,縱火大掠。天子出居於興元,克用退屯河中。朱玫亦反以兵追天子,不及,得襄王煴,迫之稱帝,屯于鳳翔。僖宗念獨克用可以破玫而不能使也,當破黃巢長安時,天下兵馬都監楊復恭與克用善,乃遣諫議大夫劉崇望以詔書召克用,且道復恭意,使進兵討玫等。克用陽諾而不行。



 明年,孟方立死,其弟遷立。大順元年,克用擊破孟遷,取邢、洺、磁三州,乃遣安金俊攻赫連鐸於雲州。幽州李匡威救鐸,戰於蔚州,金俊大敗。於是匡威、鐸及朱全忠
 皆請因其敗伐之。昭宗以克用破黃巢功高,不可伐,下其事臺、省四品官議,議者多言不可。宰相張濬獨以謂沙陀前逼僖宗幸興元,罪當誅,可伐。軍容使楊復恭,克用所善也,亦極諫以為不可,昭宗然之,詔諭全忠等。全忠陰賂濬,使持其議益堅,昭宗不得已,以濬為太原四面行營兵馬都統,韓建為副使。是時,潞州將馮霸叛降于梁,梁遣葛從周入潞州。唐以京兆尹孫揆為昭義軍節度使,克用遣李存孝執揆于長子,又遣康君立取潞州。十一月,濬及克用戰于陰地,濬軍三戰三敗,濬、建遁歸。克用兵大掠晉、絳,至於河中,赤地千里。克用上表
 自訴,其辭慢侮,天子為之引咎,優詔答之。



 二年二月,復拜克用河東節度使、隴西郡王,加檢校太師兼中書令。四月,攻赫連鐸于雲州,圍之百餘日,鐸走吐渾。八月,大搜于太原,出晉、絳,掠懷、孟,至于邢州,遂攻王鎔于鎮州。克用柵常山西,以十餘騎渡滹沱覘敵,遇大雨,平地水深數尺。鎮人襲之,克用匿林中,禱其馬曰:「吾世有太原者馬不嘶。」馬偶不嘶以免。前軍李存孝取臨城,進攻元氏。李匡威救鎔,克用還軍邢州。景福元年,王鎔攻邢州,李存信、李嗣勛等敗鎔于堯山。二月,會王處存攻鎔,戰于新市,為鎔所敗。八月,李匡威攻雲州,以牽克用之兵,
 克用潛入于雲州,返出擊匡威,匡威敗走。十月,李存孝以邢州叛。二年,存孝求援於王鎔,克用出兵井陘擊鎔,且以書招鎔,而急攻其平山,鎔懼,遂與克用通和,獻帛五十萬匹,出兵助攻邢州。



 乾寧元年三月,執存孝,殺之。冬,攻幽州,李匡儔棄城走,追至景城,見殺,以劉仁恭為留後。



 二年,河中王重盈卒,其諸子珂、珙爭立,克用請立珂,鳳翔李茂貞、邠寧王行瑜、華州韓建請立珙。昭宗初兩難之,乃以宰相崔胤為河中節度使,既而許克用立珂。茂貞等怒,三鎮兵犯京師,聞克用亦起兵,乃皆罷去。六月,克用攻絳州,斬刺史王瑤。瑤,珙弟,助珙以爭者。七
 月,至於河中,同州王行約奔於京師,陽言曰:「沙陀十萬至矣!」謀奉天子幸邠州,茂貞假子閻圭亦謀劫幸鳳翔,京師大亂,昭宗出居于石門。克用軍留月餘不進,昭宗遣延王戒丕、丹王允兄事克用,且告急。八月,克用進軍渭橋,以為邠寧四面行營都統。昭宗還京師。十一月,克用擊破邠州,王行瑜走至慶州,見殺。克用還軍雲陽,請擊茂貞,昭宗慰勞克用,使與茂貞解仇以紓難,拜克用「忠正平難功臣」,封晉王。是時,晉軍渭北,遇雨六十日,或勸克用入朝,克用未決,都押衙蓋寓曰:「天子還自石門,寢未安席,若晉兵渡渭,人情豈復能安?勤王而已,何必
 朝哉?」克用笑曰:「蓋寓猶不信我,況天下乎!」乃收軍而還。



 三年正月,昭宗復以張濬為相,克用曰:「此朱全忠之謀也。」乃上表曰:「若陛下朝以濬為相,則臣將暮至闕廷!」京師大恐,浚命遽止。朱全忠之攻兗、鄆也,克用遣李存信假道魏州以救朱宣等。存信屯於莘縣,軍士侵掠魏境,羅弘信伏兵攻之,存信敗走洺州。克用自將擊魏,戰於洹水,亡其子落落。六月,破魏成安、洹水、臨漳等十餘邑。十月,又敗魏人於白龍潭,進攻觀音門,全忠救至,乃解。



 四年,劉仁恭叛晉,克用以兵五萬擊仁恭,戰于安塞,克用大敗。



 光化元年,朱全忠遣葛從周攻下邢、洺、磁三州。
 克用遣周德威出青山口,遇從周於張公橋,德威大敗。冬,潞州守將薛志勤卒,李罕之據潞州,叛附於朱全忠。



 二年,全忠遣氏叔琮攻破承天軍,又破遼州,至于榆次,周德威敗之于洞渦。



 秋,李嗣昭復取澤、潞。三年,嗣昭敗汴軍于沙河,復取洺州,朱全忠自將圍之,嗣昭走,至青山口,遇汴伏兵,嗣昭大敗。秋,嗣昭取懷州。是歲,汴人攻鎮、定,鎮、定皆絕晉以附于朱全忠。



 天復元年,全忠封梁王。梁攻下晉、絳、河中,執王珂以歸。晉失三與國,乃下意為書幣聘梁以求和。梁王以為晉弱可取,乃曰:「晉雖請盟,而書辭慢。」因大舉擊晉。四月,氏叔琮入天井,張文敬
 入新口,葛從周入土門,王處直入飛狐,侯言入陰地。叔琮取澤、潞,其別將白奉國破承天軍,遼州守將張鄂、汾州守將李瑭皆迎梁軍降,晉人大懼。會天大雨霖,梁兵多疾,皆解去。五月,晉復取汾州,誅李瑭。六月,周德威、李嗣昭取慈、隰。二年,進攻晉、絳,大敗於蒲縣,梁軍乘勝破汾、慈、隰三州,遂圍太原。克用大懼,謀出奔雲州,又欲奔匈奴,未決,梁軍大疫,解去,周德威復取汾、慈、隰三州。



 四年,梁遷唐都於洛陽,改元曰天祐。克用以謂劫天子以遷都者梁也,天祐非唐號,不可稱,乃仍稱天復。



 五年,會契丹阿保機於雲中,約為兄弟。



 六年,梁攻燕滄州,燕王
 劉仁恭來乞師。克用恨仁恭反覆,欲不許,其子存勖諫曰:「此吾復振之時也。今天下之勢,歸梁者十七八,彊如趙、魏、中山,莫不聽命。是自河以北,無為梁患者,其所憚者惟我與仁恭耳,若燕、晉合勢,非梁之福也。夫為天下者不顧小怨,且彼常困我而我急其難,可因以德而懷之,是謂一舉而兩得,此不可失之機也。」克用以為然,乃為燕出兵攻破潞州,梁圍乃解去,以李嗣昭為潞州留後。



 七年,梁兵十萬攻潞州,圍以夾城。遣周德威救潞州,軍于亂柳。冬,克用疾,是歲,梁滅唐,克用復稱天祐四年。



 五年正月辛卯,克用卒,年五十三。子存勖立,葬克用
 於雁門。



 嗚呼,世久而失其傳者多矣,豈獨史官之繆哉!李氏之先,蓋出於西突厥,本號朱邪,至其後世,別自號曰沙陀,而以朱邪為姓,拔野古為始祖。其自序云:沙陀者,北庭之磧也,當唐太宗時,破西突厥諸部,分同羅、僕骨之人於此磧,置沙陀府,而以其始祖拔野古為都督;其傳子孫,數世皆為沙陀都督,故其後世因自號沙陀。然予考於傳記,其說皆非也。夷狄無姓氏,朱邪,部族之號耳,拔野古與朱邪同時人,非其始祖,而唐太宗時,未嘗有沙陀府也。唐太宗破西突厥,分其諸部,置十三州,以同羅
 為龜林都督府,僕骨為金微都督府,拔野古為幽陵都督府,未嘗有沙陀府也。當是時,西突厥有鐵勒,延陀、阿史那之類為最大;其別部有同羅、僕骨、拔野古等以十數,蓋其小者也;又有處月、處密諸部,又其小者也。朱邪者,處月別部之號耳。太宗二十二年,已降拔野古,其明年,阿史那賀魯叛。至高宗永徽二年,處月朱邪孤注從賀魯戰于牢山,為契苾何力所敗,遂沒不見。後百五六十年,憲宗時,有朱邪盡忠及子執宜見於中國,而自號沙陀,以朱邪為姓矣。蓋沙陀者,大磧也,在金莎山之陽,蒲類海之東,自處月以來居此磧,號沙陀突厥,而夷
 狄無文字傳記,硃邪又微不足錄,故其後世自失其傳。至盡忠孫始賜姓李氏,李氏後大,而夷狄之人遂以沙陀為貴種云。



\end{pinyinscope}