\article{卷五十雜傳第三十八}

\begin{pinyinscope}

 王
 峻王峻,字秀峰,相州安陽人也。父豐,為樂營將。峻少以善歌事梁節度使張筠。



 唐莊宗已下魏博,筠棄相州,走歸京師。租庸使趙巖過筠家,筠命峻歌佐酒,巖見而悅之。是時巖方用事,筠因以峻遣巖。梁亡,巖族誅,峻流落民間。久之,事三司使張延朗,延朗不甚愛之。晉高祖滅唐,殺延朗,是時漢高祖從晉起兵,因悉以延朗貲產賜之,峻因得事漢高祖。高祖鎮河東,峻為客將。高祖即位,拜
 峻客省使。



 漢遣郭從義討趙思綰,以峻監其軍。累遷宣徽北院使。



 周太祖鎮天雄軍,峻為監軍。漢隱帝已殺大臣史弘肇等,又遣人殺周太祖及峻等,峻等遂與太祖舉兵犯京師。太祖監國,以漢太后命拜峻樞密使。太祖將兵北出,至澶州,返軍向京師。是時,太祖已遣馮道迎湘陰公贇于徐州,而漢宗室蔡王信在許州。峻與王殷謀,遣侍衛馬軍指揮使郭崇率兵之宋州、前申州刺史馬鐸之許州以伺變,崇、鐸遂幽贇而殺信。



 太祖入立,拜峻右僕射、門下侍郎、同中書門下平章事,監修國史。劉旻攻晉州,峻為行營都部署,得以便宜從事。別遣陳思
 讓、康延沼自烏嶺出絳州與峻會。



 峻至陜州,留不進。太祖遣使者翟守素馳至陜州,諭峻欲親征。峻屏左右謂守素曰:「晉州城堅不可近,而劉旻兵銳亦未可當,臣所以留此者,非怯也,蓋有待爾。且陛下新即位,四方籓鎮,未有威德以加之,豈宜輕舉!而兗州慕容彥超反迹已露,若陛下出汜水,則彥超入京師,陛下何以待之?」守素馳還,具道峻言。是時,太祖已下詔西幸,聞峻語,遽自提其耳曰:「幾敗吾事!」乃止不行。峻軍出自絳州,前鋒報過蒙坑,峻喜,謂其屬曰:「蒙坑,晉、絳之險也,旻不分兵扼之,使吾過此,可知其必敗也。」峻軍去晉州一舍,旻聞周兵
 大至,即解去。諸將皆欲追之,峻猶豫不決。明日,遣騎兵追旻,不及而還。從討慕容彥超,為隨駕都部署,率眾先登。



 峻與太祖俱起于魏,自謂佐命之功,以天下為己任。凡所論請,事無大小,期於必得,或小不如志,言色輒不遜,太祖每優容之。峻年長於太祖二歲,往往呼峻為兄,或稱其字,峻由是益橫。鄭仁誨、李重進、向訓等,皆太祖故時偏裨,太祖初即位,謙抑未欲進用,而峻心忌之。自破慕容彥超還,即求解樞密以探上意,太祖慰勞之。峻多發書諸鎮,求為保薦,居數日,諸鎮皆馳騎上峻書,太祖大駭。峻連章求解,因不視事,太祖遣近臣召之曰:「卿
 若不出,吾當自往候卿。」峻曰:「車駕若來,是致臣有不測也。」然殊無出意。樞密直學士陳同與峻相善,太祖即遣同召峻。同還奏曰:「峻意少解,然請陛下聲言嚴駕,若將幸之,則峻必出矣。」



 太祖僶俛從之。峻聞太祖且來,遂馳入謁。



 峻於樞密院起事,極其華侈,邀太祖臨幸,賜予甚厚。太祖於內園起一小殿,峻輒奏曰:「宮室已多,何用此為?」太祖曰:「樞密院屋不少,卿亦何必有作?」



 峻慚不能對。峻為樞密使兼宰相,又求兼領平盧。已受命,暫之鎮,又請借左藏庫綾萬匹,太祖皆勉從之。又請用顏衎、陳同代李穀、范質為相,太祖曰:「進退宰相,豈可倉卒?當
 徐思之。」峻論請不已,語漸不遜。日亭午,太祖未食,峻爭不已,是時寒食假,太祖曰:「俟假開,當為卿行。」峻乃退。太祖遂不能忍,明日御便殿,召百官皆入,即幽峻於別所。太祖見馮道,泣曰:「峻凌朕,不能忍!」



 即貶商州司馬,卒於貶所。



 峻已被黜,太祖以峻監修國史,意其所書不實,因召史官取日曆讀之,史官以禁中事非外所知,懼以漏落得罪。峻貶後,李穀監修,因請命近臣錄禁中事付史館,乃命樞密直學士就樞密院錄送史館,自此始。



 王殷王殷,大名人也。少為軍卒,以軍功累遷靈武馬步軍都指揮使。唐廢帝時,從范延光討張令昭于魏,以功拜祁
 州刺史。晉天福中,徙原州刺史。



 殷事母以孝聞,欲與人游,必先白母,母所不可者,未嘗取往。及為刺史,政事有小失,母責之,殷即取杖授婢僕,自笞於母前。母亡服喪,晉高祖詔殷起復,以為憲州刺史,殷乞終喪。服除,出帝以為奉國右廂都指揮使。



 後從漢高祖討杜重威,先登力戰,矢中其腦,鏃自口出而不死,高祖嘉之,以為侍衛步軍都指揮使,領寧江軍節度使。契丹犯邊,漢遣殷以兵屯澶州。隱帝已殺楊邠等,詔鎮寧軍節度使李弘義殺殷于澶州,又詔郭崇殺周太祖于魏。詔書至澶州,弘義恐事不果,反以告殷,殷遣人馳至魏告周太祖,遂起
 兵反。太祖入立,拜侍衛親軍都指揮使,出為天雄軍節度使、同中書門下平章事,仍領親軍,自河以北皆受殷節度。殷頗務聚斂,太祖聞而惡之,遣人謂之曰:「吾起魏時,帑廩儲畜豈少邪?



 汝為國家用,足矣。」殷不聽。



 殷與王峻俱從太祖起自魏,後峻得罪,殷不自安。廣順三年秋九月永壽節,殷求入為壽,太祖許之,而懼其疑也,復遣使止之。明年,太祖有事于南郊。是冬,殷來朝,殷握兵柄,職當警衛,出入多以兵從,又求兵甲,以備非常。是時,太祖臥疾,疑殷有異志,乃力疾御滋德殿,殷入起居,即命執之,削奪在身官爵,長流登州。已而殺之,徙其家屬于
 登州。



 劉詞劉詞,字好謙,大名元城人也。少事楊師厚,以勇悍知名。唐莊宗下魏博,與梁戰夾河,詞以軍功為效節軍使,遷長劍指揮使,坐事左遷汝州十餘年。廢帝時,詔諸州鎮選驍勇者充禁軍,詞得選為禁軍校。從破張從賓、楊光遠,以功遷奉國第一軍都虞候。從馬全節破安州,以功遷指揮使。從杜重威破鎮州,以先登功拜泌州刺史。晉軍討安從進,為襄州行營都虞候,以功遷泌州團練使。徙房州,歲餘,為政不苛撓,人頗便之。詞居暇日,常被甲枕戈而臥,謂人曰:「我以此取富貴,豈可一日輒忘之?且
 人情易習,若一墮其筋力,有事何以報國!」漢高祖時,復為奉國右廂都指揮使。漢軍討李守貞于河中,詞以侍衛步軍都指揮使領寧江軍節度使,為行營都虞候,以功拜鎮國軍節度使。周太祖入立,加同中書門下平章事。歷鎮安國、河陽三城。世宗戰高平,樊愛能等軍敗南走,遇詞而止之曰:「軍敗矣,可無前也。」詞不聽,輒趣兵以進,世宗嘉之,以為隨駕都部署。及班師,以為河東行營副都部署,徙鎮永興。明年卒于鎮,年六十五,贈侍中,謚忠惠。



 王環王環,鎮州真定人也。以勇力事孟知祥為御者,及知祥
 僭號于蜀,使典衛兵。



 晉開運之亂,秦、鳳、階、成入于蜀,孟昶以環為鳳州節度使。周世宗即位,明年,遣王景、向訓攻秦、鳳州,數為環所敗,大臣皆請罷兵。世宗曰:「吾欲一天下以為家,而聲教不及秦、鳳,今兵已出,無功而返,吾有慚焉。」乃決意攻之。周兵糧道頗艱,昶遣兵五千出堂倉抵黃花谷以爭糧道。景、訓先知其來,命排陣使張建雄以兵二千當谷口,別遣裨將以勁兵千人出其後,伏堂倉以待其歸。蜀兵前遇建雄,戰不勝,退走堂倉,伏發,盡殪之,由是蜀兵守諸城堡者皆潰。



 初,昶遣其秦州節度使高處儔以兵援環,未至,聞堂倉兵敗,亦潰歸,處儔
 判官趙玭閉城不內,處儔遂奔成都,玭乃以城降,成、階二州相繼亦降,獨環堅守百餘日,然後克之。世宗召見環,歎曰:「三州已降,環獨堅守,吾數以書招之,而環不答,至於力屈就擒,雖不能死,亦忠其所事也,用之可勸事君者。」乃拜環右驍衛將軍。是時,周師已征淮,即以環佐侯章為攻取賊城水砦副部署。初,周師南征,李景陳兵於淮,舟楫甚盛,周師無水戰之具,世宗患之,乃置造船務於京城之西,為戰艦數百艘,得景降卒,教之水戰。明年,世宗再徵淮,使環將水戰卒數千,自蔡河以入淮。環居軍中,未嘗有戰功。蜀卒與環俱擒者,世宗不殺,悉以
 從軍,後多南奔於景,世宗待環益不疑。已而景將許文縝、邊鎬等皆被擒,世宗悉以為將軍,與環等列第京師,歲時賜與甚厚。明年又幸淮南,又以環從,遇疾,卒於泗州。



 折從阮折從阮,字可久,初名從遠,避漢高祖名,改為阮,雲中人也。其父嗣倫,為麟州刺史。從阮為人,溫恭長者,居父喪,以孝聞。唐莊宗鎮太原,以為牙將,後以為府州刺史。晉出帝與契丹敗盟,從阮以兵攻契丹,取其城堡十餘,遷本州團練使,兼領朔州刺史、安北都護、振武軍節度使、契丹西南面行營馬步軍都虞候。漢高祖入立,於府州
 建永安軍,以從阮為節度使。明年,以其族朝京師,徙鎮武勝,即拜從阮子德扆為府州團練使。周太祖入立,從阮歷徙宣義、保義、靜難三鎮。顯德二年,罷還京師,行至洛陽卒,贈中書令。



\end{pinyinscope}