\article{卷六十八閩世家第八}

\begin{pinyinscope}

 王審知,字信通,光州固始人也。父恁,世為農。兄潮,為縣史。唐末群盜起,壽州人王緒攻陷固始,緒聞潮兄弟材勇,召置軍中,以潮為軍校。是時,蔡州秦宗權方募士以益兵,乃以緒為光州刺史,召其兵會擊黃巢。緒遲留不行,宗權發兵攻緒。緒率眾南奔,所至剽掠,自南康入臨汀,陷漳浦,有眾數萬。緒性猜忌,部將有材能者,多因事殺之,潮頗自懼。軍次南安,潮說其前鋒將曰:「吾屬棄墳墓、妻子而為盜者,為緒所脅爾,豈其本心哉!今緒雄猜,將吏之材能者必死,
 吾屬不自保朝夕,況欲圖成事哉!」前鋒將大悟,與潮相持而泣。乃選壯士數十人,伏篁竹間,伺緒至,躍出擒之,囚之軍中。緒後自殺。



 緒已見廢,前鋒將曰:「生我者潮也。」乃推
 潮為
 主。是時,泉州刺史廖彥若為政貪暴,泉人苦之,聞潮略地至其境,而軍行整肅,其耆老相率遮道留之,潮即引兵圍彥若,逾年克之。光啟二年,福建觀察使陳巖表潮泉州刺史。景福元年巖卒,其婿范暉自稱留後。潮遣審知攻暉,久不克,士卒傷死甚眾,審知請班師,潮不許。



 又請潮自臨軍,且益兵,潮
 報曰:「兵與將俱盡,吾當自往。」審知乃親督士卒攻破之,暉見殺。唐即以潮為福建觀察使,潮以審知為副使。



 審知為人狀兒雄偉,隆準方口,常乘白馬,軍中號「白馬三郎」。乾寧四年,潮卒,審知代立。唐以福州為威武軍,拜審知節度使,累遷同中書門下平章事,封瑯琊王。唐亡,梁太祖加拜審知中書令,封閩王,升福州為大都督府。是時,楊行密據有江淮,審知歲遣使汎海,自登、萊朝貢于梁,使者入海,覆溺常十三四。



 審知雖起盜賊,而為人儉約,好禮下士。王淡,唐相溥之子;楊沂,唐相涉從弟;徐寅,唐時知名進士,皆依審知仕宦。又建學四門,以
 教閩士之秀者。招來海中蠻夷商賈。海上黃崎,波濤為陰,一夕風雨雷電震擊,開以為港,閩人以為審知德政所致,號為甘棠港。



 審知同光三年卒,年六十四,謚曰忠懿。子延翰立。



 延翰字子逸,審知長子也。同光四年,唐拜延翰節度使。是歲,莊宗遇弒,中國多故,延翰乃取司馬遷《史記》閩越王無諸傳示其將吏曰:「閩,自古王國也,吾今不王,何待之有?」於是軍府將吏上書勸進。十月,延翰建國稱王,而猶稟唐正朔。



 延翰為人長大,美皙如玉,其妻崔氏陋而淫,延翰不能制。審知喪未期,徹其几筵,又多選良家子
 為妾。崔氏性石,良家子之美者,輒幽之別室,繫以大械,刻木為人手以擊頰,又以鐵錐刺之,一歲中死者八十四人。崔氏後病,見以為祟而卒。



 審知養子建州刺史延稟,本姓周氏,自審知時與延翰不葉。延翰立,以其弟延鈞為泉州刺史,延鈞怒。二人因謀作亂。十二月,延稟、延鈞皆以兵入,執延翰殺之。而延鈞立,更名鏻。



 鏻,審知次子也。唐即拜鏻節度使,累加檢校太師、中書令,封閩王。初,延稟與鏻之謀殺延翰也,延稟之兵先至,已執延翰而殺之,膽日鏻兵始至,延稟自以養子,推鏻而立之。延稟還建州,鏻餞于郊,延稟臨訣謂鏻曰:「善繼
 先志,毋煩老兄復來!」鏻銜之。長興二年,延稟率兵擊鏻,攻其西門,使其子繼雄轉海攻其南門,鏻遣王仁達拒之。仁達伏甲舟中,偽立白幟請降,繼雄信之,登舟,伏兵發,刺殺之,梟其首西門,其兵見之皆潰去,延稟見執。鏻誚之曰:「予不能繼先志,果煩老兄復來!」延廩不能對,遂殺之。延稟子繼昇守建州,聞敗,奔于錢塘。



 長興三年,鏻上書言:「楚王馬殷、吳越王錢鏐皆為尚書令,今皆已薨,請授臣尚書令。」唐不報,鏻遂絕朝貢。



 鏻好鬼神、道家之說,道士陳守元以左道見信,建寶皇宮以居之。守元謂鏻曰:「寶皇命王少避其位,後當為六十年天子。」鏻欣然
 遜位,命其子繼鵬權主府事。



 既而復位,遣守元問寶皇:「六十年後將安歸?」守元傳寶皇語曰:「六十年後,當為大羅仙人。」鏻乃即皇帝位,受冊於寶皇,以黃龍見真封宅,改元為龍啟,國號閩。追謚審知為昭武孝皇帝,廟號太祖,立五廟,置百官,以福州為長樂府。而閩地狹,國用不足,以中軍使薛文傑為國計使。文傑多察民間陰事,致富人以罪,而籍沒其貲以佐用,閩人皆怨。又薦妖巫徐彥,曰:「陛下左右多姦臣,不質諸鬼神,將為亂。」鏻使彥視鬼於宮中。



 文傑與內樞密使吳英有隙,英病在告,文傑謂英曰:「上以公居近密,而屢以疾告,將罷公。」英曰:「奈何?」
 文傑因教英曰:「即上遣人問公疾,當言『頭痛而已,無他苦也。』」英以為然。明日,諷鏻使巫視英疾,巫言:「入北廟,見英為崇順王所訊,曰:『汝何敢謀反?』以金槌擊其首。」鏻以語文傑,文傑曰:「未可信也,宜問其疾如何。」鏻遣人問之,英曰:「頭痛。」鏻以為然,即以英下獄,命文傑劾之,英自誣伏,見殺。英嘗主閩兵,得其軍士心,軍士聞英死,皆怒。是歲,吳人攻建州,鏻遣其將王延宗救之,兵士在道不肯進,曰:「得文傑乃進。」鏻惜之不與,其子繼鵬請與之以紓難,乃以檻車送文傑軍中。文傑善數術,自占云:「過三日可無患。」送者聞之,疾馳二日而至,軍士踴躍,磔文傑於
 市,閩人爭以瓦石投之,臠食立盡。明日,鏻使者至,赦之,已不及。初,文傑為鏻造檻車,以謂古制疏闊,乃更其制,令上下通,中以鐵芒內嚮,動輒觸之,既成,首被其毒。



 龍啟三年,改元永和。王仁達為鏻殺延稟有功,而典親兵,鏻心忌之,嘗問仁達曰:「趙高指鹿為馬,以愚二世,果有之邪?」仁達曰:「秦二世愚,故高指鹿為馬,非高能愚二世也。今陛下聰明,朝廷官不滿百,起居動靜,陛下皆知之,敢有作威福者,族滅之而已。」鏻慚,賜與金帛慰安之。退而謂人曰:「仁達智略,在吾世可用,不可遺後世患。」卒誣以罪殺之。



 鏻妻早卒,繼室金氏賢而不見答。審知婢
 金鳳,姓陳氏,鏻嬖之,遂立以為后。



 初,鏻有嬖吏歸守明者,以色見倖,號歸郎,鏻後得風疾,陳氏與歸郎姦。又有百工院使李可殷,因歸郎以通陳氏。鏻命錦工作九龍帳,國人歌曰:「誰謂九龍帳,惟貯一歸郎!」



 鏻婢春燕有色,其子繼鵬蒸之,鏻已病,繼鵬因陳氏以求春燕,鏻怏怏與之。



 其次子繼韜怒,謀殺繼鵬,繼鵬懼,與皇城使李仿圖之。是歲十月,金粦饗軍于大酺殿,坐中昏然,言見延稟來,仿以為鏻病已甚,乃令壯士先殺李可殷于家。明日晨朝,鏻無恙,問仿殺可殷何罪,仿懼而出,與繼鵬率皇城衛士而入。鏻聞鼓噪聲,走匿九龍帳中,衛士刺之不
 殂,宮人不忍其苦,為絕之。繼韜及陳後、歸郎皆為仿所殺。鏻立十年見殺,謚曰惠皇帝,廟號太宗。



 繼鵬,鏻長子也。既立,更名昶,改元通文,以李仿判六軍諸衛事。仿有弒君之罪,既立昶,而心常自疑,多養死士以為備。昶患之,因大享軍,伏甲擒仿殺之,梟其首于市。仿部曲千人叛,燒啟聖門,奪仿首,奔於錢塘。



 晉天福二年,昶遣使朝貢京師,高祖遣散騎常侍盧損冊昶閩王,拜其子繼恭臨海郡王。損至閩,昶稱疾不見,令繼恭主之。又遣中書舍人劉乙勞損于館,乙衣冠偉然,騶僮甚盛。他日損遇乙于途,布衣芒屩而已,損使人誚之曰:「
 鳳閣舍人,何偪下之甚也!」乙羞愧,以手掩面而走。昶聞之,怒損侵辱之,損還,昶無所答。



 而其子繼恭遣其佐鄭元弼隨損至京師貢方物,致書晉大臣,述昶意求以敵國禮相往來。高祖怒其不遜,下詔暴其罪,歸其貢物不納。兵部員外郎李知損上書請籍沒其物而禁錮使者,於是以元弼下獄。獄具引見,元弼俯伏曰:「昶,夷貊之君,不知禮義,陛下方示大信,以來遠人,臣將命無狀,願伏斧金質,以贖昶罪。」高祖乃赦元弼,遣歸。



 昶亦好巫,拜道士譚紫霄為正一先生,又拜陳守元為天師,而妖人林興以巫見幸,事無大小,興輒以寶皇語命之而後行。守
 元教昶起三清臺三層,以黃金數千斤鑄寶皇及元始天尊、太上老君像,日焚龍腦、薰陸諸香數斤,作樂于臺下,晝夜聲不輟,云如此可求大還丹。三年夏,虹見其宮中,林興傳神言:「此宗室將為亂之兆也。」乃命興率壯士殺審知子延武、延望及其子五人。後興事敗,亦被殺。而昶愈惑亂,立父婢春燕為淑妃,後立以為皇后。又遣醫人陳究以空名堂牒賣官。



 昶弟繼嚴判六軍諸衛事,昶疑而罷之,代以季弟繼鏞,而募勇士為宸衛都以自衛,其賜予給賞,獨厚於他軍。控鶴都將連重遇、拱宸都將朱文進,皆以此怒激其軍。是歲夏,術者言昶宮中當有
 災,昶徙南宮避災,而宮中火,昶疑重遇軍士縱火。



 內學士陳郯素以便佞為昶所親信,昶以火事語之,郯反以告重遇。重遇懼,夜率衛士縱火焚南宮,昶挾愛姬、子弟、黃門衛士斬關而出,宿於野次。重遇迎延義立之。



 延義令其子繼業率兵襲昶,及之;射殺數人,昶知不免,擲弓于地,繼業執而殺之,及其妻、子皆死無遺類。延義立,謚昶曰康宗。



 延義,審知少子也。既立,更名曦,遣使者朝貢於晉,改元永隆。鑄大鐵錢,以一當十。曦自昶世倔彊難制,昶相王倓每抑折之,曦亦憚倓,不敢有所發。新羅遣使聘閩以寶
 劍,昶舉以示倓曰:「此將何為?」倓曰:「不忠不孝者,斬之。」



 曦居旁色變。曦既立,而新羅復獻劍,曦思倓前言,而倓已死,命發塚戮其尸,倓面如生,血流被體。



 泉州刺史餘廷英嘗矯曦命掠取良家子,曦怒,召下御史劾之。廷英進買宴錢千萬,曦曰:「皇后土貢何在?」廷英又獻皇后錢千萬,乃得不劾。曦嘗嫁女,朝士有不賀者笞之。御史中丞劉贊坐不糾舉,將加笞,諫議大夫鄭元弼切諫,曦謂元弼曰:「卿何如魏鄭公,乃敢彊諫!」元弼曰:「陛下似唐太宗,臣為魏鄭公可矣。」



 曦喜,乃釋贊不笞。



 曦弟延政為建州節度使,封富沙王,自曦立,不葉,數舉兵相攻,曦由此惡
 其宗室,多以事誅之。諫議大夫黃峻舁櫬詣朝堂極諫,曦怒,貶峻漳州司戶參軍。校書郎陳光逸上書疏曦過惡五十餘事,曦命衛士鞭之百而不死,以繩係頸,掛於木,久而乃絕。國計使陳匡範增算商之法以獻,曦曰:「匡範人中寶也。」已而歲入不登其數,乃借於民以足之,匡範以憂死。其後知其借於民也,剖棺斷尸,棄之水中。



 曦性既淫虐,而妻李氏悍而酗酒,賢妃尚氏有色而寵。李仁遇,曦甥也,以色嬖之,用以為相。曦常為牛飲,群臣侍酒,醉而不勝,有訴及私棄酒者輒殺之。諸子繼柔棄酒,並殺其贊者一人。連重遇殺昶,懼為國人所討,與朱文
 進連姻以自固。



 曦心疑之,常以語誚重遇等,重遇等流涕自辨。李氏石尚妃之寵,欲圖曦而立其子亞澄,乃使人謂重遇等曰:「上心不平於二公,奈何?」重遇等懼。六年三月,曦出遊,醉歸,重遇等遣壯士拉於馬上而殺之,謚曰景宗。



 延政,審知子也。曦立,為淫虐,延政數貽書諫之。曦怒,遣杜建崇監其軍,延政逐之,曦乃舉兵攻延政,為延政所敗。延政乃以建州建國稱殷,改元天德。



 明年,連重遇已殺曦,集閩群臣告曰:「昔太祖武皇帝親冒矢石,遂啟有閩,及其子孫,淫虐不道。今天厭王氏,百姓與能,當求有
 德,以安此土。」群臣皆莫敢議,乃掖朱文進升殿,率百官北面而臣之。文進以重遇判六軍諸衛事,王氏子弟在福州者無少長皆殺之。以黃紹頗守泉州,程贇守漳州,許文縝守汀州,稱晉年號,時開運元年也。泉州軍將留從效詐其州人曰:「富沙王兵收福州矣,吾屬世為王氏臣,安能交臂而事賊乎?」州人共殺紹頗,迎王繼勛為刺史,漳州聞之,亦殺贇,迎王繼成為刺史,皆王氏之諸子也。文縝懼,以汀州降于延政。延政已得三州,重遇亦殺文進,傳首建州以自歸。福州裨將林仁翰又殺重遇,謀迎延政都福州。



 是時,南唐李景聞閩亂,發兵攻之,延政
 遣其從子繼昌守福州,而南唐兵方急攻延政,福州將李仁達謂其徒曰:「唐兵攻建州,富沙王不能自保,其能有此土也?」



 乃擒繼昌殺之。欲自立,懼眾不附,以雪峰寺僧卓儼明示眾曰:「此非常人也。」



 被以袞冕,率諸將吏北面而臣之。已而又殺儼明,乃自立,送款于李景,景以仁達為威武軍節度使,更其名曰弘義。而景兵攻破建州,遷延政族於金陵,封鄱陽王。



 是歲,景保大四年也。



 留從效聞延政降唐,執王繼勛送于金陵,李景以泉州為清源軍,以從效為節度使。景已破延政,遣人召李仁達入朝,仁達不從,遂降于吳越。而留從效亦逐景守兵,
 據泉、漳二州,景猶封從效晉江王。周世宗時,從效遣牙將蔡仲興為商人,間道至京師,求置邸內屬。是時,世宗與李景畫江為界,遂不納,從效仍臣於南唐。



 其後事具國史。



 晉開運三年丙午,南唐保大四年也。是歲,李景兵破建州,王氏滅。《江南錄》云:「保大三年,虜王氏之族,遷於金陵。」謬也。據王潮實以唐景福元年入福州,拜觀察使,而後人紀錄者,乃用「騎馬來、騎馬去」之讖以為據,遂以王潮光啟二年歲在丙午拜泉州刺史為始年,至保大四年,歲復在丙午而滅,故為六十一年。然其奄有閩國,則當自景福元年為始,實五十五年也。今諸家記其國滅丙午是也。其始年則牽於讖書,謬矣。惟《江南錄》又差其末年也。



\end{pinyinscope}