\article{卷十七晉家人傳第五}

\begin{pinyinscope}

 高祖皇后李氏高祖皇后李氏,唐明宗皇帝女也。后初號永寧公主,清泰二年封魏國長公主。



 自廢帝立,常疑高祖必反。三年,公主自太原入朝千春節,辭歸,留之不得,廢帝醉,語公主曰:「爾歸何速,欲與石郎反邪?」既醒,左右告之,廢帝大悔。公主歸,以語高祖,高祖由是益不自安。高祖即位,公主當為皇后。天福二年三月,有司言:「皇太妃尊號已正,請上寶冊。」太妃,高祖庶母劉氏也。高祖以宗廟未立,謙
 抑未皇。七年夏五月,高祖已病,乃詔尊太妃為皇太后,然卒不奉冊而高祖崩,故后訖高祖世亦無冊命。出帝天福八年七月,冊尊皇后為皇太后。太后為人彊敏,高祖常嚴憚之。出帝馮皇后用事,太后數訓戒之,出帝不從,乃及於敗。



 開運三年十二月,耶律德光已降晉兵,遣張彥澤先犯京師,以書遺太后,具道已降晉軍,且曰:「吾有梳頭妮子竊一藥囊以奔於晉,今皆在否?吾戰陽城時,亡奚車一乘,在否?」又問契丹先為晉獲者及景延廣、桑維翰等所在。太后與帝聞彥澤至,欲自焚,嬖臣薛超勸止之。及得德光所與書,乃滅火,出上苑中。帝召當直
 學士范質,謂曰:「杜郎一何相負!昔先帝起太原時,欲擇一子留守,謀之北朝皇帝,皇帝以屬我,我素以為其所知,卿為我草奏具言之,庶幾活我子母。」質為帝草降表曰:孫男臣重貴言:頃者唐運告終,中原失馭,數窮否極,天缺地傾。先人有田一成,有眾一旅,兵連禍結,力屈勢孤。翁皇帝救患摧剛,興利除害,躬擐甲胃,深入寇場。犯露蒙霜,度鴈門之險;馳風擊電,行中冀之誅。黃鉞一麾,天下大定,勢凌宇宙,義感神明。功成不居,遂興晉祚,則翁皇帝有大造於石氏也。



 旋屬天降鞠凶,先君即世,臣遵承遺旨,篡紹前基。諒闇之初,荒迷失次,凡有軍國重
 事,皆委將相大臣。至於擅繼宗祧,既非廩命;輕發文字,輒敢抗尊。自啟釁端,果貽赫怒,禍至神惑,運盡天亡。十萬師徒,望風束手;億兆黎庶,延頸歸心。臣負義包羞,貪生忍恥,自貽顛覆,上累祖宗,偷度朝昏,茍存視息。翁皇帝若惠顧疇昔,稍霽雷霆,未賜靈誅,不絕先祀,則百口荷更生之德,一門銜無報之恩,雖所願焉,非敢望也。臣與太后、妻馮氏於郊野面縛俟罪次。



 又為太后表曰:晉室皇太后新婦李氏妾言:張彥澤、傅住兒等至,伏蒙皇帝阿翁降書安撫者。



 妾伏念先皇帝頃在並、汾,適逢屯難,危同累卵,急若倒懸,智勇俱窮,朝夕不保。



 皇帝阿翁
 發自冀北,親抵河東,跋履山川,踰越險阻。立平巨孽,遂定中原,救石氏之覆亡,立晉朝之社稷。不幸先帝厭代,嗣子承祧,不能繼好息民,而反虧恩辜義。兵戈屢動,駟馬難追,戚實自貽,咎將誰執!今穹旻震怒,中外攜離,上將牽羊,六師解甲。妾舉宗負釁,視景偷生,惶惑之中,撫問斯至,明宣恩旨,典示含容,慰諭丁寧,神爽飛越。豈謂已垂之命,忽蒙更生之恩,省罪責躬,九死未報。



 今遣孫男延煦、延寶,奉表請罪,陳謝以聞。



 德光報曰:「可無憂,管取一吃飯處。」



 四年正月丁亥朔,德光入京師,帝與太后肩輿至郊外,德光不見,館於封禪寺,遣其將崔延勛以
 兵守之。是時雨雪寒凍,皆苦飢。太后使人謂寺僧曰:「吾嘗於此飯僧數萬,今日豈不相憫邪?」寺僧辭以虜意難測,不敢獻食。帝陰祈守者,乃稍得食。



 辛卯,德光降帝為光祿大夫、檢校太尉,封「負義侯」,遷於黃龍府。德光使人謂太后曰:「吾聞重貴不從母教而至於此,可求自便,勿與俱行。」太后答曰:「重貴事妾甚謹。所失者,違先君之志,絕兩國之歡。然重貴此去,幸蒙大惠,全生保家,母不隨子,欲何所歸!」於是太后與馮皇后、皇弟重睿、皇子延煦、延寶等舉族從帝而北,以宮女五十、宦者三十、東西班五十、醫官一、控鶴官四、御廚七、茶酒司三、儀鸞司三、六軍
 士二十人從,衛以騎兵三百。所經州縣,皆故晉將吏,有所供饋,不得通。路傍父老,爭持羊酒為獻,衛兵推隔不使見帝,皆涕泣而去。



 自幽州行十餘日,過平州,出榆關,行砂磧中,飢不得食,遣宮女、從官,採木實、野蔬而食。又行七八日,至錦州,虜人迫帝與太后拜阿保機畫像。帝不勝其辱,泣而呼曰:「薛超誤我,不令我死!」又行五六日,過海北州,至東丹王墓,遣延煦拜之。又行十餘日,渡遼水,至渤海國鐵州。又行七八日,過南海府,遂至黃龍府。



 是歲六月,契丹國母徙帝、太后於懷密州,州去黃龍府西北一千五百里。行過遼陽二百里,而國母為永康王
 所囚,永康王遣帝、太后還止遼陽,稍供給之。明年四月,永康王至遼陽,帝白衣紗帽,與太后、皇后詣帳中上謁,永康王止帝以常服見。帝伏地雨泣,自陳過咎。永康王使人扶起之,與坐,飲酒奏樂。而永康王帳下伶人、從官,望見故主,皆泣下,悲不自勝,爭以衣服藥餌為遺。



 五月,永康王上陘,取帝所從行宦者十五人、東西班十五人及皇子延煦而去。



 永康王妻兄禪奴愛帝小女,求之,帝辭以尚幼。永康王馳一騎取之,以賜禪奴。陘,虜地,尤高涼,虜人常以五月上陘避暑,八月下陘。至八月,永康王下陘,太后自馳至霸州見永康王,求於漢兒城側賜地
 種牧以為生。永康王以太后自從,行十餘日,遣與延煦俱還遼陽。



 明年乃漢乾祐二年,其二月,徙帝、太后於建州。自遼陽東南行千二百里至建州,節度使趙延暉避正寢以館之。去建州數十里外得地五十餘頃,帝遣從行者耕而食之。



 明年三月,太后寢疾,無醫藥,常仰天而泣,南望戟手罵杜重威、李守貞等曰:「使死者無知則已,若其有知,不赦爾於地下!」八月疾亟,謂帝曰:「我死,焚其骨送范陽佛寺,無使我為虜地鬼也!」遂卒。帝與皇后、宮人、宦者、東西班,皆被髮徙跣,扶舁其柩至賜地,焚其骨,穿地而葬焉。



 周顯德中,有中國人自契丹亡歸者,言見
 帝與皇后諸子皆無恙。後不知其所終。



 太妃安氏安太妃,代北人也,不知其世家,為敬儒妻,生出帝,封秦國夫人。出帝立,尊為皇太妃。妃老而失明,從出帝北遷,自遼陽徙建州,卒於道中。臨卒謂帝曰:「當焚我為灰,南向揚之,庶幾遺魂得反中國也。」既卒,砂磧中無草木,乃毀奚車而焚之,載其燼骨至建州。李太后亦卒,遂並葬之。



 出帝皇后馮氏出帝皇后馮氏,定州人也。父蒙,為州進奏吏,居京師,以巧佞為安重誨所喜,以為鄴都副留守。高祖留守鄴都,得蒙懽甚,乃為重胤娶濛女,後封吳國夫人。重胤早卒,
 后寡居,有色,出帝悅之。高祖崩,梓宮在殯,出帝居喪中,納之以為后。



 是日,以六軍仗衛、太常鼓吹,命后至西御莊,見于高祖影殿。群臣皆賀。帝顧謂馮道等曰:「皇太后之命,與卿等不任大慶。」群臣出,帝與皇后酣飲歌舞,過梓宮前,酹而告曰:「皇太后之命,與先帝不任大慶。」左右皆失笑,帝亦自絕倒,顧謂左右曰:「我今日作新女婿,何似?」后與左右皆大笑,聲聞于外。后既立,專內寵,封拜宮官尚宮、知客等皆為郡夫人,又用男子李彥弼為皇后宮都押衙。其兄玉執政,內外用事,晉遂以亂。契丹犯京師,暴帝之惡于天下曰:「納叔母於中宮,亂人倫之大
 典。」后隨帝北遷,哀帝之辱,數求毒藥,欲與帝俱飲以死,而藥不可得。後不知其所終。



 高祖叔父兄弟晉氏始出夷狄而微,終為夷狄所滅,故其宗室次序本末不能究見。其可見者,曰高祖二叔父,一兄六弟,七子二孫,而有略有詳,非惟禍亂多故而失其事實,抑亦無足稱焉者。然粗存其見者,以備其闕云。二叔父曰萬友、萬詮,兄曰敬儒,弟曰敬威、敬德、敬殷、敬贇、敬暉、重胤,子曰重貴、重信、重乂、重英、重進、重睿、重杲,孫曰延煦、延寶。孝平皇帝生孝元皇帝、萬友、萬詮,孝元皇帝生高祖,萬友生敬威、敬贇,萬詮生敬暉,而敬儒、敬德、敬殷、重胤皆
 不知其於高祖為親疏也。



 高祖,孝元皇帝第二子也,而敬儒為兄,疑其長子也,則於高祖屬長而親,然贈官反最後於諸弟,而高祖世獨不得追封,此又可疑也。重胤,高祖弟也,亦不知其為親疏,然高祖愛之,養以為子,故於名加「重」而下齒諸子。高祖叔、兄與弟敬殷、子重進,皆前即位卒,而敬威、敬德、重胤、重英,高祖反時死。高祖少子曰馮六,未名而卒,而舊說以重睿為幼子者,非也。



 石氏世事軍中,萬友、萬詮職卑不見。天福二年正月,萬友自故金紫光祿大夫、檢校司徒兼御史大夫、上柱國贈太師。萬詮亦自金紫光祿大夫、檢校司空兼御史大夫、
 上柱國贈太傅。出帝天福八年五月,追封皇叔祖萬友為秦王,萬詮加贈太師,追封趙王。



 從弟敬威敬威字奉信,唐廢帝時為彰聖右第三都指揮使,領常州刺史。聞高祖舉兵太原,謂人曰「生而有死,人孰能免?吾兄方舉大事,吾不可偷生取辱,見笑一時。」遂自殺。敬德時為沂州馬步軍指揮使,以高祖反誅。天福二年正月,贈敬威、敬德皆為太傅,並贈敬殷以檢校太子賓客,亦贈太傅,而不及敬儒。七年正月,追封敬威廣王,敬德福王,敬殷通王,皆贈太尉。敬儒始以故金紫光祿大夫、檢校尚書左僕射兼御史大夫、上柱國贈太傅,而獨不
 得封。出帝天福八年五月,加贈三皇叔皆為太師,而皇伯敬儒始追封宋王,亦加贈太師。



 從兄敬贇敬贇字德和,少無賴,竄身民間。高祖使人求得之,補太原牙將。即位,以為飛龍皇城使,累遷曹州防禦使。天福五年冬,拜河陽三城節度使。敬贇性貪暴,高祖為擇賢佐吏輔之,而敬贇亦憚高祖嚴,未嘗敢犯法。歲餘,徙鎮保義。出帝時,加同中書門下平章事,始漸驕恣。帝嘗遣使者至,必問曰:「小姪安否?」陜人苦其暴虐,召還京師,以其皇叔不能責也,斥其元從都押衙蘇彥存、鄭溫遇以警之。



 契丹犯邊,敬贇從出帝幸澶淵,使以兵備汶陽,守
 麻家渡,未嘗見敵,皆無功。開運元年七月,復出為威勝軍節度使。歲餘,出帝以曹州為威信軍,授敬贇節度使。



 在曹貪暴尤甚,久之,召還。張彥澤兵犯京師,敬贇夜走,踰城東垣,墮沙濠溺死,時年四十九。



 從弟韓王敬暉韓王敬暉字德昭,為人厚重剛直,勇而多智,高祖尤愛之。高祖時為曹州防禦使,以廉儉見稱,卒於官,贈太傅。天福八年,加贈太師,追封韓王。子曦嗣。



 高祖諸子孫高祖李皇后生楚王重信,其諸子皆不知其母。當高祖起太原,重英為右衛將軍,重胤為皇城副使,居京師。聞高祖舉事,匿民家井中,捕得誅之,并族民家。天福二年
 正月,高祖為二子發哀,皆贈為太保;並贈重進以故左金吾衛將軍贈太保。七年正月,皆加贈太傅,追封重英虢王,重胤郯王,重進夔王。出帝天福八年五月,皆加贈太師。



 子楚王重信楚王重信字守孚,為人敏悟多智而好禮。天福二年二月,以左驍衛上將軍拜河陽三城節度使,有善政,高祖下詔褒之。是歲范延光反,詔前靈武節度使張從賓發河陽兵討延光,從賓亦反,重信見殺,時年二十。高祖欲贈重信太尉,大臣引漢故事,皇子無為三公者。高祖曰:「此兒為善被禍,吾哀之甚,自我而已,豈有例邪!」



 乃贈太
 尉。七年正月,加贈太師,追封沂王。出帝天福八年五月,易封楚王。



 子壽王重乂壽王重乂字弘理,為人好學,頗知兵法。高祖即位,拜左驍衛大將軍。高祖幸汴州,以為東都留守。張從賓反,攻河南,見殺,時年十九,贈太傅。天福七年正月,加贈太尉,追封壽王。出帝天福八年五月,加贈太師。皆無子。



 子重睿重睿為人貌類高祖。高祖臥疾,宰相馮道入見臥內,重睿尚幼,高祖呼出使拜道於前,因以宦者抱持寘道懷中,高祖雖不言,左右皆知其以重睿託道也。高祖崩,晉大臣以國家多事,議立長君,而景延廣已陰許立出帝,
 重睿遂不得立。出帝以重睿為檢校太保、開封尹,以左散騎常侍邊蔚權知開封府事。開運二年五月,拜重睿雄武軍節度使,歲餘,徙鎮忠武,皆不之鎮。契丹滅晉,重睿從出帝北遷,後不知其所終。



 子重杲陳王重杲,高祖幼子也。小字馮六,未名而卒,贈太傅,追封陳王,賜名重杲。



 出帝天福八年五月,加贈太師。



 孫延煦延寶延煦、延寶,高祖諸孫也,出帝以為子。



 開運二年秋,以延煦為鄭州刺史。延煦少,不能視事,以一宦者從之,又選尚書郎路航參知州事。宦者遂專政事,每詬辱航,出帝召航還。已而徙延煦齊州防禦使。三年,拜鎮寧軍節度
 使。是時,河北用兵,天下旱蝗,民餓死者百萬計,而諸鎮爭為聚斂,趙在禮所積鉅萬,為諸侯王最。出帝利其貲,乃以延煦娶在禮女,在禮獻絹三千匹,前後所獻不可勝數。三年五月,遣宗正卿石光贊以聘幣一百五十床迎于其第,出帝宴在禮萬歲殿,所以賜予甚厚,君臣窮極奢侈,時人以為榮。在禮謂人曰:「吾此一婚,其費十萬。」十一月,徙延煦鎮保義。



 自延煦為齊州防禦使,而延寶代為鄭州刺史。及契丹滅晉,出帝與太后遣延煦、延寶齎降表、玉璽、金印以歸契丹,而延寶時亦為威信軍節度使矣。契丹得璽,以為製作非工,與前史所傳者
 異,命延煦等還報求真璽。出帝以狀答曰:「頃潞王從珂自焚於洛陽,玉璽不知所在,疑已焚之。先帝受命,命玉工製此璽,在位群臣皆知之。」乃已。後延煦等從出帝北遷,不知其所終。



 嗚呼!古之不幸無子,而以其同宗之子為後者,聖人許之,著之《禮》經而不諱也。而後世閭閻鄙俚之人則諱之,諱則不勝其欺與偽也。故其茍偷竊取嬰孩襁褓,諱其父母,而自欺以為我生之子,曰:「不如此,則不能得其一志盡愛於我,而其心必二也。」而為其子者,亦自諱其所生,而絕其天性之親,反視以為叔伯父,以此欺其九族,
 而亂其人鬼親疏之屬。凡物生而有知,未有不愛其父母者。使是子也,能忍而真絕其天性歟,曾禽獸之不若也。使其不忍而外陽絕之,是大偽也。夫閭閻鄙俚之人之慮於事者,亦已深矣!然而茍竊欺偽不可以為法者,小人之事也。惟聖人則不然,以謂人道莫大於繼絕,此萬世之通制而天下之公行也,何必諱哉!所謂子者,未有不由父母而生者也,故為人後者,必有所生之父,有所後之父,此理之自然也,何必諱哉!其簡易明白,不茍不竊,不欺不偽,可以為通制而公行者,聖人之法也。又以謂為人之後者所承重,故加其服以斬。而不絕其所生
 之親者,天性之不可絕也,然而恩有屈於義,故降其服以期。服,外物也,可以降,而父母之名不可改,故著於經曰:「為人後者,為其父母報。」自三代以來,有天下國家者莫不用之,而晉氏不用也。出帝之於敬儒,絕其父道,臣而爵之,非特以其義不當立,不得已而絕之,蓋亦習見閭閻鄙俚之所為也。五代,干戈賊亂之世也,禮樂崩壞,三綱五常之道絕,而先王之制度文章掃地而盡於是矣!如寒食野祭而焚紙錢,天子而為閭閻鄙俚之事者多矣!而晉氏起於夷狄,以篡逆而得天下,高祖以耶律德光為父,而出帝於德光則以為祖而稱孫,於其所生
 父則臣而名之,是豈可以人理責哉!



\end{pinyinscope}