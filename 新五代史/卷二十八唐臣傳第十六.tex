\article{卷二十八唐臣傳第十六}

\begin{pinyinscope}

 豆盧革豆盧革,父瓚,唐舒州刺史。豆盧為世名族,唐末天下亂,革避地之中山,唐亡,為王處直掌書記。莊宗在魏,議建唐國,而故唐公卿之族遭亂喪亡且盡,以革名家子,召為行臺左丞相。莊宗即位,拜同中書門下平章事。革雖唐名族,而素不學問,除拜官吏,多失其序,常為尚書郎蕭希甫駁正,革頗患之。莊宗已滅梁,革乃薦韋說為相。說,唐末為殿中侍御史,坐事貶南海,後事梁為禮部
 侍郎。革以說能知前朝事,故引以佐己,而說亦無學術,徒以流品自高。



 是時,莊宗內畏劉皇后,外惑宦官、伶人,郭崇韜雖盡忠於國,而亦無學術,革、說俯仰默默無所為,唯諾崇韜而已。唐、梁之際,仕宦遭亂奔亡,而吏部銓文書不完,因緣以為姦利,至有私鬻告敕,亂易昭穆,而季父、母舅反拜姪、甥者,崇韜請論以法。是時唐新滅梁,朝廷紀綱未立,議者以為宜革以漸,而崇韜疾惡太甚,果於必行,說、革心知其未可,而不能有所建言。是歲冬,選人吳延皓改亡叔告身行事,事發,延皓及選吏尹玫皆坐死,尚書左丞判吏部銓崔沂等皆貶,說、革詣閣門
 待罪。由是一以新法從事,往往以偽濫駮放而斃踣羈旅、號哭道路者,不可勝數。及崇韜死,說乃教門人上書言其事,而議者亦以罪之。



 是歲,大水,四方地連震,流民殍死者數萬人,軍士妻子皆採穭以食。莊宗日以責三司使孔謙,謙不知所為。樞密小吏段徊曰:「臣嘗見前朝故事,國有大故,則天子以朱書御札問宰相。水旱,宰相職也。」莊宗乃命學士草詔,手自書之,以問革、說。革、說不能對,第曰:「陛下威德著於四海,今西兵破蜀,所得珍寶億萬,可以給軍。水旱,天之常道,不足憂也。」革自為相,遭天下多故,而方服丹砂煉氣以求長生,嘗嘔血數日,幾
 死。二人各以其子為拾遺,父子同省,人以為非,遽改佗官,而革以說子為弘文館學士,說以革子為集賢院學士。



 莊宗崩,革為山陵使,莊宗已祔廟,革以故事當出鎮,乃還私第,數日未得命,而故人賓客趣使入朝。樞密使安重誨詬之于朝曰:「山陵使名尚在,不俟改命,遽履新朝,以我武人可欺邪!」諫官希旨,上疏誣革縱田客殺人,說坐與鄰人爭井,遂俱罷。革貶辰州刺史,說漵州刺史,所在馳驛發遣。宰相鄭玨、任圜三上章,請毋行後命,不報。革復坐請俸私自入,說賣官與選人,責授革費州司戶參軍,說夷州司戶參軍,皆員外置同正員。已而竄革
 陵州,說合州,皆長流百姓。



 初,說嘗以罪竄之南海,遇赦,還寓江陵,與高季興相知,及為相,常以書幣相問遺。唐兵伐蜀,季興請以兵入三峽,莊宗許之,使季興自取夔、忠、萬、歸、峽等州為屬郡。及破蜀,季興無功,而唐用佗將取五州。明宗初即位,季興數請五州,以謂先帝所許,朝廷不得已而與之。及革、說再貶,因以其事歸罪二人。天成二年夏,詔陵、合州刺史監賜自盡。



 革子升,說子濤,皆官至尚書郎,坐其父廢。至晉天福初,濤為尚書膳部員外郎,卒。



 盧程盧程,不知其世家何人也。唐昭宗時,程舉進士,為鹽鐵
 出使巡官。唐亡,避亂燕、趙,變服為道士,遊諸侯間。豆盧革為王處直判官,盧汝弼為河東節度副使,二人皆故唐時名族,與程門地相等,因共薦之以為河東節度推官。莊宗嘗召程草文書,程辭不能。其後戰胡柳,掌書記王誠歿于陣,莊宗還軍太原,置酒謂監軍張承業曰:「吾以卮酒辟一書記於坐。」因舉卮屬巡官馮道。程位在道上,以嘗辭不能,故不用,而遷程支使。程大恨曰:「用人不以門閥而先田舍兒邪!」



 莊宗已即位,議擇宰相,而盧汝弼、蘇循已死,次節度判官盧質當拜,而質不樂行事,乃言豆盧革與程皆故唐時名族,可以為相,莊宗以程為
 中書侍郎、同平章事。是時,朝廷新造,百度未備,程、革拜命之日,肩輿導從,喧呼道中。莊宗聞其聲以問左右,對曰:「宰相簷子入門。」莊宗登樓視之,笑曰:「所謂似是而非者也。」



 程奉皇太后冊,自魏至太原,上下山險,所至州縣,驅役丁夫,官吏迎拜,程坐肩輿自若,少忤其意,必加笞辱。人有假驢夫於程者,程帖興唐府給之,府吏啟無例,程怒笞吏背。少尹任圜,莊宗姊婿也,詣程訴其不可。程戴華陽巾,衣鶴氅,據几決事,視圜罵曰:「爾何蟲豸,恃婦家力也!宰相取給州縣,何為不可!」圜不對而去,夜馳至博州見莊宗。莊宗大怒,謂郭崇韜曰:「朕誤相此癡
 物,敢辱予九卿!」趣令自盡,崇韜亦欲殺之,賴盧質力解之,乃罷為右庶子。莊宗入洛,程於路墜馬,中風卒,贈禮部尚書。



 任圜任圜,京兆三原人也。為人明敏,善談辯,見者愛其容止,及聞其論議縱橫,益皆悚動。李嗣昭節度昭義,辟圜觀察支使。梁兵築夾城圍潞州,踰年而晉王薨,晉兵救潞者皆解去。嗣昭危甚,問圜去就之計,圜勸嗣昭堅守以待,不可有二心。



 已而莊宗攻破梁夾城,聞圜為嗣昭畫守計,甚嘉之,由是益知名。其後嗣昭與莊宗有隙,圜數奉使往來,辨釋讒構,嗣昭卒免於禍,圜之力也。嗣昭從
 莊宗戰胡柳,擊敗梁兵,圜頗有功,莊宗勞之曰:「儒士亦破體邪?仁者之勇,何其壯也!」



 張文禮弒王鎔,莊宗遣嗣昭討之。嗣昭戰歿,圜代將其軍,號令嚴肅。既而文禮子處球等閉城堅守,不可下,圜數以禍福諭鎮人,鎮人信之。圜嘗擁兵至城下,處球登城呼圜曰:「城中兵食俱盡,而久抗王師,若泥首自歸,懼無以塞責,幸公見哀,指其生路。」圜告之曰:「以子先人,固難容貸,然罰不及嗣,子可從輕。



 其如拒守經年,傷吾大將,一朝困竭,方布款誠,以此計之,子亦難免。然坐而待斃,曷若伏而俟命?」處球流涕曰:「公言是也!」乃遣子送狀乞降,人皆稱圜其言不欺。
 既而佗將攻破鎮州,處球雖見殺,而鎮之吏民以嘗乞降,故得保其家族者甚眾。



 其後以鎮州為北京,拜圜工部尚書,兼真定尹、北京副留守知留守事,為政有惠愛。明年,郭崇韜兼領成德軍節度使,改圜行軍司馬,仍知真定府事。圜與崇韜素相善,又為其司馬,崇韜因以鎮州事託之,而圜多所違異。初,圜推官張彭為人傾險貪黷,圜不能察,信任之,多為其所賣。及崇韜領鎮,彭為圜謀隱其公廨錢。



 莊宗遣宦者選故趙王時宮人百餘,有許氏者尤有色,彭賂守者匿之。後事覺,召彭詣京師,將罪之,彭懼,悉以前所隱公錢簿書獻崇韜,崇韜深
 德彭,不殺,由是與圜有隙。同光三年,圜罷司馬,守工部尚書。



 魏王繼岌暨崇韜伐蜀,懼圜攻己於後,乃辟圜參魏王軍事。蜀滅,表圜黔南節度使,圜懇辭不就。繼岌殺崇韜,以圜代將其軍而旋。康延孝反,繼岌遣圜將三千人,會董璋、孟知祥等兵,擊敗延孝於漢州,而魏王先至渭南,自殺,圜悉將其軍以東。明宗嘉其功,拜圜同中書門下平章事,兼判三司。是時,明宗新誅孔謙,圜選辟才俊,抑絕僥倖,公私給足,天下便之。



 是秋,韋說、豆盧革罷相,圜與安重誨、鄭玨、孔循議擇當為相者,圜意屬李琪,而玨、循雅不欲琪為相,謂重誨曰:「李琪非無文藝,但不
 廉耳!宰相,端方有器度者足以為之,太常卿崔協可也。」重誨以為然。佗日,明宗問誰可相者,重誨即以協對。圜前爭曰:「重誨未諳朝廷人物,為人所賣。天下皆知崔協不識文字,而虛有儀表,號為『沒字碑』。臣以陛下誤加採擢,無功幸進,此不知書,以臣一人取笑足矣,相位有幾,豈容更益笑端?」明宗曰:「宰相重位,卿等更自詳審。



 然吾在籓時,識易州刺史韋肅,世言肅名家子,且待我甚厚,置之此位可乎?肅或未可,則馮書記先朝判官,稱為長者,可以相矣!」馮書記者,道也。議未決,重誨等退休於中興殿廓下,孔循不揖,拂衣而去,行且罵曰:「天下事一則
 任圜,二則任圜,圜乃何人!」圜謂重誨曰:「李琪才藝,可兼時輩百人,而讒夫巧沮,忌害其能,若舍琪而相協,如棄蘇合之丸而取蜣良之轉也!」重誨笑而止。然重誨終以循言為信,居月餘,協與馮道皆拜相。協在相位數年,人多嗤其所為,然圜與重誨交惡自協始。



 故事,使臣出四方,皆自戶部給券,重誨奏請自內出,圜以故事爭之,不能得,遂與重誨辨於帝前,圜聲色俱厲。明宗罷朝,後宮嬪御迎前問曰:「與重誨論者誰?」



 明宗曰:「宰相也。」宮人奏曰:「妾在長安,見宰相奏事,未嘗如此,蓋輕大家耳!」明宗由是不悅,而使臣給券卒自內出,圜益憤沮。重誨嘗
 過圜,圜出妓,善歌而有色,重誨欲之,圜不與,由是二人益相惡。而圜遽求罷職,乃罷為太子少保。



 圜不自安,因請致仕,退居于磁州。



 朱守殷反于汴州,重誨誣圜與守殷連謀,遣人矯制殺之。圜受命怡然,聚族酣飲而死。明宗知而不問,為下詔,坐圜與守殷通書而言涉怨望。愍帝即位,贈圜太傅。



 趙鳳趙鳳,幽州人也,少以儒學知名。燕王劉守光時,悉黥燕人以為兵,鳳懼,因髡為僧,依燕王弟守奇自匿。守奇奔梁,梁以守奇為博州刺史,鳳為其判官。守奇卒,鳳去為鄆州節度判官。晉取鄆州,莊宗聞鳳名,得之喜,以為扈
 鑾學士。莊宗即位,拜鳳中書舍人、翰林學士。



 莊宗及劉皇后幸河南尹張全義第,酒酣,命皇后拜全義為父。明日,遣宦者命學士作箋上全義,以父事之,鳳上書極言其不可。全義養子郝繼孫犯法死,宦官、伶人冀其貲財,固請籍沒,鳳又上書言:「繼孫為全義養子,不宜有別籍之財,而於法不至籍沒,刑人利財,不可以示天下。」是時,皇后及群小用事,鳳言皆不見納。



 明宗武君,不通文字,四方章奏,常使安重誨讀之。重誨亦不知書,奏讀多不稱旨。孔循教重誨求儒者置之左右,而兩人皆不知唐故事,於是置端明殿學士,以馮道及鳳為之。



 鳳好直言
 而性剛強,素與任圜善,自圜為相,頗薦進之。初,端明殿學士班在翰林學士下,而結銜又在官下。明年,鳳遷禮部侍郎,因諷圜升學士於官上,又詔班在翰林學士上。圜為重誨所殺,而誣以謀反。是時,重誨方用事,雖明宗不能詰也,鳳獨號哭呼重誨曰:「任圜天下義士,豈肯謀反!而公殺之,何以示天下?」



 重誨慚不能對。



 術士周玄豹以相法言人事多中,莊宗尤信重之,以為北京巡官。明宗為內衙指揮使,重誨欲試玄豹,乃使佗人與明宗易服,而坐明宗於下坐,召玄豹相之,玄豹曰:「內衙,貴將也,此不足當之。」乃指明宗於下坐曰:「此是也!」因為明宗言
 其後貴不可言。明宗即位,思玄豹以為神,將召至京師,鳳諫曰:「好惡,上所慎也。今陛下神其術而召之,則傾國之人,皆將奔走吉凶之說,轉相惑亂,為患不細。」明宗遂不復召。



 朱守殷反,明宗幸汴州,守殷已誅,又詔幸鄴。是時,從駕諸軍方自河南徙家至汴,不欲北行,軍中為之洶洶。而定州王都以為天子幸汴州誅守殷,又幸鄴以圖己,因疑不自安。宰相率百官詣閣,請罷幸鄴,明宗不聽,人情大恐,群臣不復敢言。鳳手疏責安重誨,言甚切直,重誨以白,遂罷幸。



 有僧遊西域,得佛牙以獻,明宗以示大臣。鳳言:「世傳佛牙水火不能傷,請驗其真偽。」因
 以斧斫之,應手而碎。是時,宮中施物已及數千,因鳳碎之乃止。



 天成四年夏,拜門下侍郎、同中書門下平章事。秘書少監于嶠者,自莊宗時與鳳俱為翰林學士,而嶠亦訐直敢言,與鳳素善。及鳳已貴,而嶠久不遷,自以材名在鳳上而不用,因與蕭希甫數非斥時政,尤詆訾鳳,鳳心銜之,未有以發。而嶠與鄰家爭水竇,為安重誨所怒,鳳即左遷嶠祕書少監。嶠因被酒往見鳳,鳳知其必不遜,乃辭以沐髮,嶠詬直吏,又溺於從者直盧而去。省吏白鳳,嶠溺於客次,且詬鳳。鳳以其事聞,明宗下詔奪嶠官,長流武州百姓,又流振武,天下冤之。



 其後安重誨
 為邊彥溫等告變,明宗詔彥溫等廷詰,具伏其詐,即斬之。後數日,鳳奏事中興殿,啟曰:「臣聞姦人有誣重誨者。」明宗曰:「此閑事,朕已處置之,卿可無問也。」鳳曰:「臣所聞者,繫國家利害,陛下不可以為閑。」因指殿屋曰:「此殿所以尊嚴宏壯者,棟梁柱石之所扶持也,若折其一棟,去其一柱,則傾危矣。



 大臣,國之棟梁柱石也,且重誨起微賤,歷艱危,致陛下為中興主,安可使姦人動搖!」明宗改容謝之曰:「卿言是也。」遂族彥溫等三家。



 其後重誨得罪,群臣無敢言者,獨鳳數言重誨盡忠。明宗以鳳為朋黨,罷為安國軍節度使。鳳在鎮所得俸祿,悉以分將校賓
 客。廢帝入立,召為太子太保。病足居于家,疾篤,自筮,投蓍而歎曰:「吾家世無五十者,又皆窮賤,吾今壽過其數而富貴,復何求哉!」清泰二年卒于家。



 李襲吉李襲吉,父圖,洛陽人,或曰唐相林甫之後也。乾符中,襲吉舉進士,為河中節度使李都搉鹽判官。後去之晉,晉王以為榆次令,遂為掌書記。襲吉博學,多知唐故事。遷節度副使,官至諫議大夫。晉王與梁有隙,交兵累年,後晉王數困,欲與梁通和,使襲吉為書諭梁,辭甚辨麗。梁太祖使人讀之,至於「毒手尊拳,交相於暮夜,金戈鐵馬,蹂踐於明時」,歎曰:「李公僻處一隅,有士如此,使吾得之,
 傅虎以翼也!」顧其從事敬翔曰:「善為我答之。」及翔所答,書辭不工,而襲吉之書,多傳於世。襲吉為人恬淡,以文辭自娛,天祐三年卒。以盧汝弼代為副使。



 汝弼工書畫,而文辭不及襲吉。其父簡求為河東節度使,為唐名家,故汝弼亦多知唐故事。晉王薨,莊宗嗣為晉王,承制封拜官爵皆出汝弼。十八年,卒。



 莊宗即位,贈襲吉禮部尚書、汝弼兵部尚書。



 張憲張憲,字允中,晉陽人也。為人沈靜寡欲,少好學,能鼓琴飲酒。莊宗素知其文辭,以為天雄軍節度使掌書記。莊宗即位,拜工部侍郎、租庸使,遷刑部侍郎、判吏部銓、東
 都副留守。憲精於吏事,甚有能政。



 莊宗幸東都,定州王都來朝,莊宗命憲治鞠場,與都擊鞠。初,莊宗建號於東都,以鞠場為即位壇,於是憲言:「即位壇,王者所以興也。漢鄗南、魏繁陽壇,至今皆在,不可毀。」乃別治宮西為鞠場,場未成,莊宗怒,命兩虞候亟毀壇以為場。憲退而歎曰:「此不祥之兆也!」



 初,明宗北伐契丹,取魏鎧仗以給軍,有細鎧五百,憲遂給之而不以聞。莊宗至魏,大怒,責憲馳自取之,左右諫之乃止。又問憲庫錢幾何。憲上庫簿有錢三萬緡,莊宗益怒,謂其嬖伶史彥瓊曰:「我與群臣飲博,須錢十餘萬,而憲以故紙給我。我未渡河時,庫錢
 常百萬緡,今復何在?」彥瓊為憲解之乃已。



 郭崇韜伐蜀,薦憲可任為相,而宦官、伶人不欲憲在朝廷,樞密承旨段徊曰:「宰相在天子面前,事有非是,尚可改作,一方之任,茍非其人,則為患不細。憲材誠可用,不如任以一方。」乃以為太原尹、北京留守。



 趙在禮作亂,憲家在魏州,在禮善待其家,遣人以書招憲,憲斬其使,不發其書而上之。莊宗遇弒,明宗入京師,太原猶未知,而永王存霸奔于太原。左右告憲曰:「今魏兵南嚮,主上存亡未可知,存霸之來無詔書,而所乘馬斷其鞦,豈非戰敗者乎!宜拘之以俟命。」憲曰:「吾本書生,無尺寸之功,而人主遇我甚
 厚,豈有懷二心以幸變,第可與之俱死爾!」憲從事張昭遠教憲奉表明宗以勸進,憲涕泣拒之。已而存霸削發,見北京巡檢符彥超,願為僧以求生,彥超麾下兵大噪,殺存霸。憲出奔沂州,亦見殺。



 嗚呼!予於死節之士,得三人而失三人焉。鞏廷美、楊溫之死,予既已哀之。



 至於張憲之事,尤為之痛惜也。予於舊史考憲事實,而永王存霸、符彥超與憲傳所書始末皆不同,莫得而考正。蓋方其變故倉卒之時。傳者失之爾。然要其大節,亦可以見也,憲之志誠可謂忠矣。當其不顧其家,絕在禮而斬其使,涕泣以拒昭遠之說,其志
 甚明。至其欲與存霸俱死,及存霸被殺,反棄太原而出奔,然猶不知其心果欲何為也。而舊史書憲坐棄城而賜死,予亦以為不然。予之於憲固欲成其美志,而要在憲失其官守而其死不明,故不得列于死節也。



 蕭希甫蕭希甫,宋州人也。為人有機辯,多矯激,少舉進士,為梁開封尹袁象先掌書記。象先為青州節度使,以希甫為巡官。希甫不樂,乃棄其母妻,變姓名,亡之鎮州,自稱青州掌書記,謁趙王王鎔。鎔以希甫為參軍,尤不樂,居歲餘,又亡之易州,削髮為僧,居百丈山。莊宗將建國于魏,置百官,求天下隱逸之士,幽州李紹宏薦希甫為魏州
 推官。



 莊宗即帝位,欲以知制誥,有詔定內宴儀,問希甫:「樞密使得坐否?」希甫以為不可。樞密使張居翰聞之怒,謂希甫曰:「老夫歷事三朝天子,見內宴數百,子本田舍兒,安知宮禁事?」希甫不能對。由是宦官用事者皆切齒。宰相豆盧革等希宦官旨,共排斥之,以為駕部郎中,希甫失志,尤怏怏。



 莊宗滅梁,遣希甫宣慰青齊,希甫始知其母已死,而妻袁氏亦改嫁矣。希甫乃發哀服喪,居于魏州,人有引漢李陵書以譏之曰:「老母終堂,生妻去室。」時皆傳以為笑。



 明宗即位,召為諫議大夫。是時,復置匭函,以希甫為使,希甫建言:「自兵亂相乘,王綱大壞,侵欺
 凌奪,有力者勝。凡略人之妻女,占人之田宅,姦臟之吏,刑獄之冤者,何可勝紀?而匭函一出,投訴必多,至於功臣貴戚,有不得繩之以法者。」乃自天成元年四月二十八日昧爽已前,大辟已上,皆赦除之,然後出匭函以示眾。初,明宗欲以希甫為諫議大夫,豆盧革、韋說頗沮難之。其後革、說為安重誨所惡,希甫希旨,誣奏:「革縱田客殺人,而說與鄰人爭井,井有寶貨。」有司推劾,井中惟破釜而已,革、說終皆貶死。明宗賜希甫帛百匹、粟麥三百石,拜左散騎常侍。



 希甫性褊而躁進,嘗遣人夜叩宮門上變,言河堰牙官李筠告本軍謀反,詰旦,追問無狀,斬
 筠,軍士詣安重誨求希甫啖之。是時,明宗將有事於南郊,前齋一日,群臣習儀于殿廷,宰相馮道、趙鳳,河南尹秦王從榮,樞密使安重誨候班于月華門外。希甫與兩省班先入,道等坐廓下不起,既出,希甫召堂頭直省朝堂驅使官,責問宰相、樞密見兩省官何得不起,因大詬詈。是夜,託疾還第。月餘,坐告李筠事動搖軍眾,貶嵐州司戶參軍,卒於貶所。



 劉贊劉贊,魏州人也。父玭為縣令,贊始就學,衣以青布衫襦,每食則玭自肉食,而別以蔬食食贊於床下,謂之曰:「肉食,君之祿也,爾欲之,則勤學問以干祿;吾肉非爾之食
 也。」由是贊益力學,舉進士,為羅紹威判官,去為租庸使趙巖巡官,又為孔謙鹽鐵判官。明宗時,累遷中書舍人、御史中丞、刑部侍郎。守官以法,權豪不可干以私。



 是時,秦王從榮握兵而驕,多過失,言事者請置師傅以輔道之。大臣畏王,不敢決其事,因請王得自擇,秦王即請贊,乃拜贊秘書監,為秦王傅。贊泣曰:「禍將至矣!」秦王所請王府元帥官屬十餘人,類多浮薄傾險之徒,日獻諛諂以驕王,獨贊從容諷諫,率以正道。秦王嘗命賓客作文於坐中,贊自以師傅,恥與群小比伍,雖操筆勉彊,有不悅之色。秦王惡之,後戒左右贊來不得通,贊亦不往,月
 一至府而已,退則杜門不交人事。



 已而秦王果敗死,唐大臣議王屬官當坐者,馮道曰:「元帥判官任贊與秦王非素好,而在職不逾月,詹事王居敏及劉贊皆以正直為王所惡,河南府判官司徒詡病告家居久,皆宜不與其謀。而諮議參軍高輦與王最厚,輦法當死,其餘可次第原減。」



 朱弘昭曰:「諸公不知其意爾,使秦王得入光政門,當待贊等如何?吾徒復有家族邪!且法有首從,今秦王夫婦男女皆死,而贊等止其一身幸矣!」道等難之。而馮贇亦爭不可,贊等乃免死。於是論高輦死,而任贊等十七人皆長流。



 初,贊聞秦王敗,即白衣駕驢以
 俟,人有告贊奪官而已,贊曰:「豈有天子塚嗣見殺,而賓僚奪官者乎,不死幸矣!」已而贊長流嵐州百姓。清泰二年,詔歸田里,行至石會關,病卒。



 何瓚何瓚,閩人也,唐末舉進士及第。莊宗為太原節度使,辟為判官。莊宗每出征伐,留張承業守太原,承業卒,瓚代知留守事。瓚為人明敏,通於吏事,外若疏簡而內頗周密。莊宗建大號于鄴都,拜瓚諫議大夫,瓚慮莊宗事不成,求留守北京。



 瓚與明宗有舊,明宗即位,召還,見於內殿,勞問久之,已而以瓚為西川節度副使。



 是時,孟知祥已有二志,方以副使趙季良為心腹,聞瓚代之,亟奏留
 季良,遂改瓚行軍司馬。瓚恥於自辭,不得已而往,明宗賜予甚厚。初,知祥在北京為馬步軍都虞候,而瓚留守太原,知祥以軍禮事瓚,瓚常繩以法,知祥初不樂,及瓚為司馬,猶勉待之甚厚。知祥反,罷瓚司馬,置之私第,瓚飲恨而卒。



\end{pinyinscope}