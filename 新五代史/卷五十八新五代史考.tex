\article{卷五十八新五代史考}

\begin{pinyinscope}

 嗚呼,五代禮樂文章,吾無取焉。其後世有欲知之者,不可以遺也。作《司天職方考》。



 司天考第一司天掌日月星辰之象。周天一歲,四時,二十四氣,七十二候,行十日十二辰,以為歷。而謹察其變者,以為占。占者,非常之兆也,以驗吉凶,以求天意,以覺人事,其術藏於有司。歷者,有常之數也,以推寒暑,以先天道,以勉人事,其法信於天下。術有時而用,法不可一日而差。差之
 毫釐,則亂天人之序,乖百事之時,蓋有國之所重也。然自堯命羲、和見於《書》,中星閏餘,略存其大法。而三代中間千有餘歲,遺文曠廢,《六經》無所述。而孔子之徒,亦未嘗道也。至於後世,其學一出於陰陽之家,其事則重,其學則末。夫天人之際,遠哉微矣,而使一藝之士,布算積分,上求數千萬歲之前,必得甲子朔旦夜半冬至,而日、月、五星皆會于子,謂之上元,以為歷始。蓋自漢而後,其說始詳見於世,其源流所自止於如此。



 是果堯、舜、三代之法歟?皆不可得而考矣。然自是以來,歷家之術,雖世多不同,而未始不本於此。



 五代之初,因唐之故,用《崇玄
 歷》。至晉高祖時,司天監馬重績始更造新歷,不復推古上元甲子冬至七曜之會,而起唐天寶十四載乙未為上元,用正月雨水為氣首。初,唐建中時,術者曹士蒍始變古法,以顯慶五年為上元,雨水為歲首,號《符天歷》。然世謂之小歷,只行於民間。而重績乃用以為法,遂施於朝廷,賜號《調元歷》。然行之五年,輒差不可用,而復用《崇玄歷》。周廣順中,國子博士王處訥私撰《明玄歷》于家。民間又有《萬分歷》,而蜀有《永昌歷》、《正象歷》,南唐有《齊政歷》。五代之際,歷家可考見者止於此。而《調元歷》法既非古,《明玄》又止藏其家,《萬分》止行於民間,其法皆不足紀。而《
 永昌》《正象》《齊政歷》,皆止用於其國,今亦亡,不復見。



 世宗即位,外伐僭叛,內修法度。端明殿學士王朴,通於歷數,乃詔朴撰定。



 歲餘,朴奏曰:臣聞聖人之作也,在乎知天人之變者也。人情之動,則可以言知之;天道之動,則當以數知之。數之為用也,聖人以之觀天道焉。歲月日時,由斯而成;陰陽寒暑,由斯而節;四方之政,由斯而行。夫為國家者,履端立極,必體其元;布政考績,必因其歲;禮動樂舉,必正其朔;三農百工,必順其時;五刑九伐,必順其氣;庶務有為,必從其日月。是以聖人受命,必治歷數。故五紀有常度,庶徵有常應,正朔行之於天下也。



 自唐之
 季,凡歷數朝,亂日失天,垂將百載,天之歷數,汨陳而已。陛下順考古道,寅畏上天,咨詢庶官,振舉墜典。臣雖非能者,敢不奉詔。乃包萬象以為法,齊七政以立元,測圭箭以候氣,審朓朒以定朔,明九道以步月,校遲疾以推星,考黃道之斜正,辨天勢之昇降,而交蝕詳焉。



 夫立天之道,曰陰與陽。陰陽各有數,合則化成矣。陽之策三十六,陰之策二十四。奇偶相命,兩陽三陰,同得七十二。同則陰陽之數合。七十二者,化成之數也。化成則謂之五行之數。五之,得期數。過之者謂之氣盈,不及者謂之朔虛。至於應變分用,無所不通。故以七十二為經法。經
 者,常用之法也。百者,數之節也,隨法進退,不失舊位,故謂之通法。以通法進經法,得七千二百,謂之統法。自元入經,先用此法,統歷之諸法也。以通法進統法,得七十二萬。氣朔之下,收分必盡,謂之全率。以通法進全率,得七千二百萬,謂之大率,而元紀生焉。元者,歲、月、日、時皆甲子;日、月、五星合在子;當盈縮、先後之中,所謂七政齊矣。



 古者植圭於陽城,以其近洛也。蓋尚慊其中,乃在洛之東偏。開元十二年,遣使天下候影,南距林邑,北距橫野,中得浚儀之岳臺,應南北弦,居地之中。大周建國,定都於汴。樹圭置箭,測岳臺晷漏,以為中數。晷漏正,則日
 之所至,氣之所應,得之矣。



 日月皆有盈縮。日盈月縮,則後中而朔。月盈日縮,則先中而朔。自古朓朒之法,率皆平行之數;入歷既有前次,而又衰稍不倫。《皇極》舊術,則迂迴而難用。降及諸歷,則疏遠而多失。今以月離朓朒,隨歷校定,日躔朓朒,臨用加減。



 所得者,入離定日也。一日之中,分為九限。每限損益,衰稍有倫。朓朒之法,可謂審矣。



 赤道者,天之紘帶也。其勢圜而平,紀宿度之常數焉。黃道者,日軌也。其半在赤道內,半在赤道外,去極二十四度。當與赤道近,則其勢斜;當與赤道遠,則其勢直。當斜,則日行宜遲;當直,則日行宜速。故二分前後加其
 度,二至前後減其度。九道者,月軌也。其半在黃道內,半在黃道外,去極遠六度。出黃道,謂之正交;入黃道,謂之中交。若正交在秋分之宿,中交在春分之宿,則比黃道益斜。



 若正交在春分之宿,中交在秋分之宿,則比黃道反直。若正交、中交在二至之宿,則其勢差斜。故校去二至二分遠近,以考斜正,乃得加減之數。自古雖有九道之說,蓋亦知而未詳,徒有祖述之文,而無推步之用。今以黃道一周,分為八節;一節之中,分為九道;盡七十二道,而使日月無所隱其斜正之勢焉。九道之法,可謂明矣。



 星之行也,近日而疾,遠日而遲。去日極遠,勢盡而留。自
 古諸歷,分段失實,隆降無準;今日行分尚多,次日便留;自留而退,惟用平行,仍以入段行度為入歷之數;皆非本理,遂至乖戾。今校逐日行分積,以為變段。然後自疾而漸遲,勢盡而留。自留而行,亦積微而後多。別立諸段變歷,以推變差,俾諸段變差,際會相合。星之遲疾,可得而知之矣。



 自古相傳,皆謂去交十五度以下,則日月有蝕。殊不知日月之相掩,與暗虛之所射,其理有異。今以日月徑度之大小,校去交之遠近,以黃道之斜正,天勢之升降,度仰視、旁視之分數,則交虧得其實矣。



 臣考前世,無食神首尾之文。近自司天卜祝小術,不能舉其
 大體,遂為等接之法。蓋從假用,以求徑捷,於是乎交有逆行之數。後學者不能詳知,因言歷有九曜,以為注歷之常式。今並削而去之。謹以《步日》、《步月》、《歲星》、《步發斂》為四篇,合為《歷經》一卷,《歷》十一卷,《草》三卷,顯德三年《七政細行歷》一卷,以為《欽天歷》。



 昔在帝堯,欽若昊天。陛下考歷象日月星辰,唐堯之道也。天道玄遠,非微臣之所盡知。



 世宗嘉之。詔司天監用之,以明年正月朔旦為始。



 《顯德欽天歷》演紀上元甲子,距今顯德三年丙辰,積七千二百六十九萬八千四百五十二算外。



 《
 欽天》統法:七千二百。



 《欽天》經法:七十二。



 《欽天》通法:一百。



 《欽天》步日躔術歲率:二百六十二萬九千七百六十,四十。



 軌率:二百六十二萬九千八百四十四,八十。



 朔率:二十一萬二千六百二十,二十八。



 歲策:三百六十五,一千七百六十,四十。



 軌策:三百六十五,一千八百四十四,八十。



 歲中:一百八十二,四千四百八十,二十。



 軌中:一百八十二,四千五百二十二,四十。



 朔策:二十九,三千八百二十,二十八。



 氣策:一十五,一千五百七十三,三十五。



 象策:七,二千七百五十五,七。



 周紀:六十。



 歲差:八十四,四十。



 辰則:六百;八刻二十四分。



 赤道宿次鬥:二十六度。牛:八度。女:十二度。虛:一十度少。危:十七度。室:十六度。壁:九度。北方七宿九十八度少。



 奎:十六度。婁:十二度。胃:十四度。昴:十一度。畢:十七度。觜:一度。



 參:一十度。西方七宿八十一度。



 井:三十三度。鬼:三度。柳:十五度。星:七度。張:十八度。翼:十八度。



 軫:十七度。南方七宿一百一十一度。



 角:十二度。亢:九度。氐:十五度。房:五度。心:五度。尾:十八度。箕:十一度。東方七宿七十五度。



 中節
 置歲率,以演紀上元距所求積年乘之,為氣積。統法而一,為日。盈周紀去之,命甲子算外,即天正中氣日辰及分秒也。以氣策累加之,秒盈通法從分,分盈統法從日,日盈周紀去之,即各得次氣日辰及分秒也。



 朔弦望置氣積,以朔率去之,不盡為閏餘。用減氣積,為朔積。統法而一,為日。盈周紀去之,命甲子算外,即天正常朔日辰及分秒也。以象策累加之,即各得弦望及次朔也。



 日躔入歷置歲率,以閏餘減之,統法而一,為日。歲中以下為盈;以
 上,減去歲中為縮,即天正常朔加時所入也。累加象策,滿歲中去之,盈縮互命,即四象所入也。



 日躔朓朒置加時入歷分秒,以其日損益率乘之,統法而一,損益其日朓朒數,為日躔朓朒定數。



 赤道日度置氣積,以軌率去之,餘統法而一,為度;命赤道虛八算外,即天正中氣加時日躔赤道宿度及分秒也。加歲中,以次命之,即夏至之宿也。



 黃道宿次
 置二至日躔赤道宿度。距前後每五度為限,初率八,每限減一,蓋九限,末率空,乃一度少彊,亦限率空。其半當四立之宿。自後亦五度為限,初率空,每限增一,盡九限,末率八,殷二分之宿。自二分至二至,亦如之。各以限率乘所入限度,為分。經法而一,為度。二至前後各九限以減、二分前後各九限以加赤道宿,為黃道宿及分。就其分為少、太、半之數。



 黃道日度置天正中氣加時日躔赤道宿度。各與所入限率相乘,皆以統法通之;所入限率乘其分,以從之。經法而一,
 為分;盈統法,為度。用減赤道所躔,即天正中氣加時日躔黃道宿度及分也。加歲中,以黃道宿次命之,即夏至加時日度及分也。



 午中日躔置二至分,減去半法,為午後分;不足,反減,為午前分。以乘初日躔分,經法而一,午前以加、午後以減加時黃道日度,為午中日度及分也。各以次日躔分加之,滿統法從度。依宿次命之,即次日午中日躔也。



 午中日躔入歷置天正中氣午前分,便為午中入盈歷日分。其在午後
 者,以午後分減歲中,為午中入縮歷日分。累加一日,滿歲中即去之,盈縮互命,為每日午中入歷也。



 岳臺中晷置午中入歷分,以其日損益率乘之,如統法而一,為分;分十為寸。用損益其下中晷數,為定數也。



 晨昏分各置入歷分,以其日損益率乘之,如統法而一,用損益其下晨分,即所求晨定分也。用損加、益減其下昏分,即所求昏定分也。



 日出入辰刻
 置晨昏分,以一百八十加晨、減昏,為日出入分。各以辰則除,為辰數;餘滿經法,為刻;命辰數子正算外,則日出入辰刻也。



 晝夜刻置日入分,以日出分減之,為晝分。用減統法,為夜分。各滿經法,為晝夜刻。



 五夜辰刻置昏分,以辰則除,為辰數;經法除,為刻數。命辰數子正算外,即甲夜辰刻也。倍晨分,五約之,為更用分。又五約之,為籌用分。用累加甲夜,滿辰則為辰,滿經法為刻,即
 各得五夜辰刻也。



 昏曉中星置昏分,減去半統,用乘軌率,統法除之,為距中分。盈統法,為度。加午中日躔,為昏中星;減之,為曉中星。



 赤道內外數置入歷分,以其日損益率乘之,如統法而一,用損益其下內外數;如不足損,則反損之;內外互命,即得所求赤道內外定數也。



 九服距軌數置距岳臺南北里數,以三百六十通之,為步。一千七百
 五十六除之,用北加、南減二千五百一十三,為其地戴中數以赤道內外定數,內減、外加之,即九服距軌數也。



 九服中晷置距軌數,二十五乘之,一百三十七除,為天用分。置之,以二十二乘,六約之,用減四千,為晷法。又以天用分自相乘,如晷法而一,為地用分。相從為晷分,分十為寸,即得其地中晷也。



 九服刻漏經法通軌中而半之,用自相乘,如其地戴中數而一;以乘二百六十三,經法除之,為漏法。通軌中於上,置赤道
 內外數於下,以下減上,餘用乘之;盈漏法,為漏分。赤道內以減、赤道外以加一千六百二十,為其地晨分。減統法,為昏分。置晨昏分,各如岳臺術入之,即得其地日出入辰刻、五夜辰刻、昏曉中星也。



 《欽天》步月離術離率:一十九萬八千三百九十三,九。



 交率:一十九萬五千九百二十七,九十七,五十六。



 離策:二十七,三千九百九十三,九。



 交策:二十七,一千五百二十七,九十七,五十六。



 望策:一十四,五千五百一十,一十四。



 交中:一十三,四千三百六十三,九十八,七十八。



 離朔:一,七千二十七,一十九。



 交朔:二,二千二百九十二,三十,四十四。



 中準:一千七百三十六。



 中限:四千七百八十。



 平離:九百六十三。



 程節:八百。



 月離入歷置朔積,以離率去之,餘滿統法為日,即天正常朔加時入歷也。累加象策,盈離策去之,即弦望及次朔入歷也。



 月離朓朒置入歷分,以日躔朓朒定數,朓減、朒加之,程節除之,為限數。餘乘所入限損益率,程節而一,用損益其限朓朒為定數。



 朔弦望定日各以日躔月離朓朒定數,朓減、朒加朔弦望常分,為定日。定朔加時日入後,則進一日;有交見初則不進。弦望加時日未出,則退一日,日雖出有交見初亦如之。



 元日有交,則消息定之。定朔與後朔乾同者,大;不同者,小;無中氣者,為閏。



 朔望加時日度各置日躔入歷,以日躔月離朓朒定數,朓減、朒加之,為定朔加時入歷。以歷分乘其日損益率,統法而一,損益其下盈縮數,為定數。置定朔歷分,通法約之,以定數盈加、縮減之。各命以冬夏至之宿算外,即所求也。



 月離入交置朔積,以交率去之,餘滿統法為日,即天正常朔入交泛日也。以望策累加之,盈交策去之,即望及次朔所入也。各以日躔朓朒定數,朓減、朒加之,為入交常日。



 置月離朓朒定數,經法乘之,平離而一,朓減、朒加常分,即入
 交定日也。



 黃道正交月度統法通朔交定日,以二百五十四乘之,十九而一。復以統法除,為入交度。用減其朔加時日度,即朔前月離正交黃道宿度也。



 九道宿次月離出入黃道六度。變從八節,斜正不同。故月有九道。黃道八節,各有九限。



 若正交起,八節後第一限之宿,為月行其節第一道。起第二限之宿,為月行其節第二道,即以所起限為正交後第一限。初率八,每限減一,盡九
 限,末率空。又九限,初率空,每限增一,末率八,殷半交之宿。自後亦九限,初率八,每限減一,末率空。又九限,初率空,每限增一,末率八,復與黃道相會,謂之中交。自中交至正交,亦如之。各置所入限度,以限率乘之,為泛差。其正交、中交前後各九限,以距二至之宿限數乘之。半交前後各九限,以距二分之宿限數乘之:皆如經法而一,為黃道差。在冬至之宿後,正交前後各九限為減,中交前後各九限為加。在夏至之宿後,正交前後各九限為加,中交前後各九限為減。凡月正交後出黃道外,中交後入黃道內。其半交前後各九限,在春分之宿後,出黃
 道外,秋分之宿後,入黃道內:皆以差為加;在春分之宿後,入黃道內,秋分之宿後,出黃道外:皆以差為減。四約泛差,以黃道差減之,為赤道差。正交、中交前後各九限,皆以差為加。半交前後各九限,皆以差為減。以黃赤二差加減黃道,為九道宿次;就其分為少、太、半之數。八節各九道,七十二道周焉。



 九道正交月度置月離正交黃道宿度;各以所入限率乘之,亦乘其分,經法約之,為泛差。用求黃赤二差,以加減之,即月離正交九道宿度也。



 九道朔月度置月離正交九道宿度,以入交度加之,命以九道宿次,即其朔加時月離九道宿度也。



 九道望月度置朔望加時日相距之度,以軌中加之,為加時象積。用加其朔九道月度,命以其道宿次,既所求也。自望推朔,亦如之。



 月離午中入歷置朔望月離入歷,加半統,減去定分,各以日躔月離朓朒定數,朓減、朒加之,即所求也。



 晨昏月度置其日晨昏分,以定分減之,為前;不足,返減,為後。用乘其日離程,統法而一,滿經法為度,為晨昏前後度。前加、後減加時月,為晨昏月度。



 晨昏象積置加時象積,以前象前後度,前減、後加,又以後象前後度,前加、後減之,即所求也。



 每日晨昏月度累計距後象離度,以減晨昏象積,為加;不足,反減之,為減。以距後象日數除之,用加減每日離度,為定度。累加
 晨昏月度,命以九道宿次,即所求。



 月去黃道度置入交定日。交中以下,月行陽道;以上,去之,月行陰道:皆以經法通之。



 用減九百八十,餘以乘之,五百五十六而一,為分;滿經法為度。行陽道,在黃道外;行陰道,在黃道內,即所求月去黃道內外度也。



 日月食限置定交行陰陽道日。半交中以下,為交後;以上,用減交中,為交前:皆以統法通之,為距交分。朔視距交分,陽道四千二百一十九、陰道一萬三百八十三以下,日入食
 限。望視距交分陰陽道皆六千九百九十五以下,月入蝕限。



 日月食甚加時定分置朔定分。半統以上,以半統減之;半統以下,用減半統:為距午分。十一乘之,經法而一。半統以下,以減半統;以上,以加朔定分:為日食加時定分。望以其日晨分與一千六百二十相減,餘以二百四十五乘之,三百一十三而一;用減二百四十五,餘以損益望定分,為月食加時定分。



 日食常準
 置中準;與其日赤道內外數相乘,二千五百一十三除,為黃道出入食差。以距午分減半晝分以乘之,半晝分而一;赤道內以減、赤道外以加中準,為日食常準。



 日食定準置日躔入歷,以經法通之,三千二百八十七以下,用減三千二百八十七,為二至後;以上,減去三千二百八十七,為二分前。六千五百七十四以上,用減九千八百六十一,為二分後;以上,減去九千八百六十一,為二至前。各三約之,二至前後用減、二分前後用加二千七百七十二,為黃道斜正食差。以距午分乘之,半晝分而一,以
 加常準,為定準。



 日食分以定準加中限,為陰道定準;減中限,為陽道定限。不足減者,反減之,為限外分。視陰道距交分,定準以上,定限以下,為陰道食;即置定限,以距交分減之,為距食分。定準以下,雖曰陰道,亦為陽道食;即加陽道定限,為距食分。其有限外分者,即減去限外分,為距食分。不足減者,不食。其陽道距交分,定限以下,為入定食限;即用減陽道定限,為距食分。各置距食分,皆以四百七十八除,為日食之大分;餘為小分。命大分以十為限;命小分以半
 及彊弱。



 月食分視距交分,中準以下,皆既;以上,用減食限,為距食分。置之,以五百二十六除,為月食之大分;餘為小分。命大分以十為限;命小分以半及彊弱。



 月食泛用分置距食分,一千九百一十二以上,用減四千七百八十;餘自相乘,六萬三千二百七十二除之;以減六百四十七,為泛用分。九百五十六以下,用減一千九百一十二,餘以通法乘之,七百三十五而一;以減五百一十七,為
 泛用分。九百五十六以上,以距食分自相乘,二千三百六十二除之;用減三百八十七,為泛用分。



 月食泛用分置距食分,二千一百四以上,用減五千二百六十;餘自相乘,六萬九千一百六十九除之;以減七百一十一,為泛用分。一千五十二以上,用減二千一百四十;餘,七除之;以減五百六十七,為泛用分。一千五十二以下,以距食分減之;餘自相乘,二千六百五十四而一;用減四百一十七,為泛用分。



 日月初末加時定分
 各置泛用分,以平離乘之,其日離程而一,為定用分。以減朔望定分,為虧初。



 加之,為復末。加時常分,如食甚術推之,得虧初、復末定分。置初、甚、末定分,各以辰則除之,為辰;經法除之,為刻:即初、甚、末之辰刻也。



 虧食所起日食起虧自西,月食起虧自東。其食分少者,月行陽道,則日食偏南,月食偏北;陰道,則日食偏北,月食偏南:此常數也。立春後,立夏前,食分多,則日食偏南,月食偏北;立秋後,立冬前,食分多,則日食偏北,月食偏南:此黃道斜正也。陽道交前,陰道交後,食分多,則日食偏南,月食
 偏北;陽道交後,陰道交前,食分多,則日食偏北,月食偏南:此九道斜正也。黃道比常數所偏差少,九道比黃道所偏又四分之一:皆據午而言之。若午前午後,一理偏南,一理偏北,及消息所食分數多少,以定初、甚、末之方,即各得所求也。



 帶食出入分視其日出入分,在虧初定分已上,復末定分已下,即帶食出入。食甚在出入分已下者,以出入分減復末定分,為帶食差。食甚在出入分已上者,以虧初定分減出入分,為帶食差。各置帶食差,以距食分乘之,定用分而一,
 日以四百七十八、月以五百二十六除,為帶食之大分;餘為小分。



 食入更籌各置初、甚、末定分。晨分已下,以昏分加之;昏分已上,昏分減之:皆更用分而一,為更數。餘,籌用分而一,為籌數。



 《欽天》步五星術歲星周率:二百八十七萬一千九百七十六,六。



 變率:二十四萬二千二百一十五,六十六。



 歷率:二百六十二萬九千七百六十一,七十八。



 周策:三百九十八,六千三百七十六,六。



 歷中:一百八十二,四千四百八十,八十九。



 變段變日變度變歷晨見一十七三三十七二二十四順疾九十一十六六十三一十一一十三順遲二十五二九一二十九前留二十六三十二退遲一十四一一十二空二十八退疾二十七四三十八一三十七
 退疾二十七四三十八一三十七退遲一十四一一十二空二十八後留二十六三十二順遲二十五二九一二十九順疾九十一十六六十三一十一一十三夕伏一十七三三十七二二十四熒惑周率:五百六十一萬五千四百二十二,一十一。



 變率:二百九十八萬五千六百六十一,七十一。



 歷率:二百六十二萬九千七百六十,空。



 周策:七百七十九,六千六百二十二,一十一。



 歷中:一百八十二,四千四百八十,空。



 變段變日變度變歷晨見七十三五十三六十八五十五十八順疾七十三五十一一四十八三次疾七十一四十六六十九四十四一十七次遲七十一四十五三十三四十二五十八順遲六十二一十九二十九一十八二十前留八六十九退遲一十一五十八空四十
 四退疾二十一七四十六二四十退疾二十一七四十六二四十退遲一十一五十八空四十四後留八六十九順遲六十二一十九二十九一十八二十次遲七十一四十五三十三四十二五十八次疾七十一四十六六十九四十四一十七順疾七十三五十一一四十八三夕伏七十三五十三六十八五十五十八鎮星
 周率:二百七十二萬二千一百七十六,九十。



 變率:九萬二千四百一十六,五十。



 歷率:二百六十二萬九千七百五十九,八十。



 周策:三百七十八,五右七十六,九十。



 歷中:一百八十二,四千四百七十九,九十。



 變段變日變度變歷晨見一十九二七一一十四順疾六十五六三十八三五十一順遲一十九空六十三空三十五前留三十七三退遲一
 十六空四十三空一十四退疾三十三二三十五空六十退疾三十三二三十五空六十退遲一十六空四十三空一十四後留三十七三順遲一十九空六十三空三十五順疾六十五六三十八三五十一夕伏一十九二七一一十四太白周率:四百二十萬四千一百四十三,九十六。



 變率:四百二十萬四千一百四十三,九十六。



 歷率:二百六十二萬九千七百五十,五十六。



 周策:五百八十三,六千五百四十三,九十六。



 歷中:一百八十二,四千四百七十五,二十八。



 變段變日變度變歷夕見四十二五十三四十五十一一十七順疾九十六一百二十一五十七一百一十六三十九次疾七十三八十三十七七十七二次遲三十三三十四一三十二四十順遲二十四一十一六十一一
 十一二十四前留六六十九退遲四一二十二空三十一退疾六三六十五一二十二夕伏七四四十一三十七晨見七四四十一三十七退疾六三六十五一二十二退遲四一二十二空三十一後留六六十九順遲二十四一十一六十一一十一二十四次遲三十三三十四一三十二四
 十次疾七十三八十三十七七十七二順疾九十六一百二十一五十七一百一十六三十九晨伏四十二五十三四十五十一一十七辰星周率:八十三萬四千三百三十五,五十二。



 變率:八十三萬四千三百三十五,五十二。



 歷率:二百六十二萬九千七百六十,四十四。



 周策:一百一十五,六千三百三十五,五十二。



 歷中:一百八十二,四千四百八十,二十二。



 變段變日變度變歷
 夕見一十七三十四一二十九五十四順疾一十一一十八二十四一十六四順遲一十六一十一四十三一十一十前留二六十八夕伏一十一六二晨見一十一六二後留二六十八順遲一十六一十一四十三一十一十順疾一十一一十八二十四一十六四晨伏一十七三十四一二十九五十
 四中日中星置氣積,以其星周率除之,為周數;不盡為天正中氣積前合。用減歲率,為前年天正中氣後合。如不足減,則加歲率以減之,為次前年天正中氣後合。各以統法約之,為日、為度,即所求平合中日、中星也。置中日,以逐段變日累加之,即逐段中日也。置中星,以逐段變度順加、退減之,即得逐段中星。金水夕伏晨見,皆退變也。



 入歷置變率。以周數乘之,以歷率去之,餘滿統法為度。歷中以下,為先;以上,減去歷中,為後:即所求平合入歷。以逐
 段變歷累加之,得逐段入歷也。



 後後定數置入歷分,以其度損益率乘之,經法而一,用損益其下先後數,即所求也。



 常日定星置中日中星,各以先後定數,先加、後減之,留用前段先後數,太白順伏見及前順疾次疾後次遲次疾、辰星順伏見及前疾後遲,並先減、後加之,即各為其段常日定星。置定星,以其年天正中氣日躔黃道宿次加而命之,得逐段末日加時宿度也。



 盈縮定數置常日,如歲中以下,為在盈;以上,減去歲中,餘為在縮:即常日入盈縮歷也。置歷分。以其日損益率乘之,經法而一,用損益其下盈縮數,即得所求也。



 定日置常日,以盈縮定數盈減、縮加之,為定日。以其年天正中氣加而命之,即逐段末日加時日辰也。



 入中節置定日,以氣策除之,命起冬至,即所入氣日數也。



 平行分
 置定日,以前段定日減之,為日率;定星與前段定星相減,為度率。通度率,以經法乘之,通日率而一,為平行分。



 初末行分近伏段與伏段平行分,合而半之,為其段近伏行分。以平行分減之,餘減平行分,為其段遠伏行分。近留段近留行分空。倍平行分為其段遠留行分。其不近伏留段,皆以順行二段平行分,合而半之,為前段末日、後段初日行分。各與其段平行分相減,平行分多,則加平行分;平行分少,則減平行分,即前段初日、後段末日行分。其不近伏留段,退行則以遲段近疾行分,為疾段近遲行
 分,所得與平行分相減,平行分多,則加之,少則減之:皆為遠遲行分也。



 初行夜半宿次置經法,以前段末日加時分減之:餘乘前段末日行分,經法而一;用順加、退減前段末日加時宿度,為其段初行昏後夜半宿度也。



 每日行分初末行分相減,為差率。累計其段初行昏後夜半距後段初行昏後夜半日數除之,為日差。半日差,以減多、加少為其段初末定行分。置初定行分,用日差末多則累
 加、末少則累減,為每日行分。以每日行分順加、退減初行昏後夜半宿度,為每日昏後夜半星所至宿度也。



 先定日昏後夜半宿次自初日累計距所求日數,以乘其段日差;末多用加、末少用減初日行分,為其日行分。合初日而半之,以所累計日乘之,用順加、退減其段初行昏後夜半宿次,即所求也。



 《欽天》步發斂術候策:五,五百二十四,四十五。



 卦策:六,六百二十九,三十四。



 外策:三,三百一十四,六十七。



 維策:一十二,一千二百五十八,六十八。



 氣盈:一千五百七十三,三十五。



 朔虛:三千三百九十九,七十二。



 氣候圖冬至十一月中蚯蚓結麋角解水泉動小寒十二月節鴈北鄉鵲始巢雉始雊大寒十二月中雞始乳鷙鳥厲疾水澤腹堅立春正月節東風解凍蟄蟲始振魚上冰雨水正月中獺祭魚鴻鴈來草木萌動
 驚蟄二月節桃始華倉庚鳴鷹化為鳩春分二月中玄鳥至雷乃發聲始電清明三月節桐始華田鼠化為鴽虹始見穀雨三月中萍始生鳴鳩拂其羽戴勝降於桑立夏四月節螻蟈鳴蚯蚓出王瓜生小滿四月中苦菜秀靡草死小暑至芒種五月節螗螂生鵙始鳴反舌無聲夏至五月中鹿角解蜩始鳴半夏生小暑六月節溫風至蟋蟀居壁鷹乃學習大暑六月中腐草為螢土潤溽暑大雨時行
 立秋七月節涼風至白露降寒蟬鳴處暑七月中鷹祭鳥天地始肅禾乃登白露八月節鴻鴈來玄鳥歸群鳥養羞秋分八月中雷乃收聲蟄蟲坯戶水始涸寒露九月節鴻鴈來賓雀入水為蛤菊有黃華霜降九月中豺祭獸草木黃落蟄蟲咸俯立冬十月節水始冰地始凍雉入水為蜃小雪十月中虹藏不見天氣上騰地氣下降閉塞成冬大雪十一月節鶡鳥不鳴虎始交荔挺出爻象圖
 冬至《坎》初六公《中孚》辟《復》侯《屯》內小寒《坎》九二侯《屯》外大夫《謙》卿《睽》大寒《坎》六三公《升》辟《臨》侯《小過》內立春《坎》六四侯《小過》外大夫《蒙》卿《益》雨水《坎》九五公《漸》辟《泰》侯《需》內驚蟄《坎》上六侯《需》外大夫《隨》卿《晉》春分《震》初九公《解》辟《大壯》侯《豫》內清明《震》六二侯《豫》外大夫《訟》卿《蠱》穀雨《震》六三公《革》闢《夬》侯《旅》內立夏《震》九四侯《旅》外大夫《師》卿《比》
 小滿《震》六五公《小畜》辟《乾》侯《大有》內芒種《震》上六侯《大有》外大夫《家人》卿《井》夏至《離》初九公《咸》闢《姤》侯《鼎》內小暑《離》六二侯《鼎》外大夫《豐》卿《渙》大暑《離》九三公《履》辟《遁》侯《恒》內立秋《離》九四侯《恆》外大夫《節》卿《同人》處暑《離》六五公《損》辟《否》侯《巽》內白露《離》上九侯《巽》外大夫《萃》卿《大畜》秋分《兌》初九公《賁》辟《觀》侯《歸妹》內寒露《兌》九二侯《歸妹》外大夫《無妄》卿《明夷》
 霜降《兌》六三公《困》辟《剝》侯《艮》內立冬《兌》九四侯《艮》外大夫《既濟》卿《噬嗑》小雪《兌》九五公《大過》辟《坤》侯《未濟》內大雪《兌》上六侯《未濟》外大夫《蹇》卿《頤》七十二候各置中節,即初候也。以候策累加之,即次候也。



 六十四卦置中氣,即公卦也。以卦策累加之,即次卦也。置侯卦,以外策加之,即外卦也。



 五行用事
 置四立之節而命之,即春木、夏火、秋金、冬水用事之初也。置四季之節,各以維策加之,即土用事也。



 沒日中節分五千六百二十六秒六十五已上者,用減統法,為有沒分。通氣策以乘之,氣盈而一,滿統法為日;用加其氣而命之,即所求沒日也。



 滅日常朔分朔虛已下者,為滅分。以朔率乘之,朔虛而一,盈統法為日;用加其朔而命之,即所求滅日也。



 右朴所撰《欽天歷經》四篇。《舊史》亡其《步發斂》一篇,而在
 者三篇,簡略不完,不足為法。



 朴歷世既罕傳,予嘗問於著作佐郎劉羲候叟,羲叟為予求得其本經,然後朴之歷大備。羲叟好學知書史,尤通於星歷,嘗謂予曰:「前世造歷者,其法不同而多差。至唐一行始以天地之中數作《大衍歷》,最為精密。後世善治歷者,皆用其法,惟寫分擬數而已。至朴亦能自為一家。朴之歷法,總日躔差為盈縮二歷,分月離為遲疾二百四十八限,以考衰殺之漸,以審朓朒,而朔望正矣。校赤道九限,更其率數,以步黃道,使日躔有常度;分黃道八節,辨其內外,以揆九道,使月行如循環,而二曜協矣。觀天勢之升降,察軌道之斜
 正,以制食差,而交會密矣。測嶽臺之中晷,以辨二至之日夜,而軌漏實矣。推星行之逆順、伏留,使舒亟有漸,而五緯齊矣。然不能宏深簡易,而徑急是取。至其所長,雖聖人出不能廢也。」羲叟之言蓋如此,覽者得以考焉。



\end{pinyinscope}