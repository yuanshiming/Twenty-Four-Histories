\article{卷三十九雜傳第二十七}

\begin{pinyinscope}

 王鎔王鎔,其先回鶻阿布思之遺種,曰沒諾干,為鎮州王武俊騎將,武俊錄以為子,遂冒姓王氏。沒諾干子曰末垣活,末垣活子曰昇,昇子曰廷湊,廷湊子曰元達,元達子曰紹鼎、紹懿,紹鼎子曰景崇。自升以上三世,常為鎮州騎將,自景崇以上四世五人,皆為成德軍節度使。景崇官至守太尉,封常山郡王,唐中和二年卒。子鎔立,年十歲。是時,晉新有太原,李匡威據幽州,王處存據中山,赫
 連鐸據大同,孟方立據邢臺,四面豪傑並起而交爭。鎔介於其間,而承祖父百年之業,士馬彊而畜積富,為唐累世籓臣。故鎔年雖少,藉其世家以取重,四方諸鎮廢立承繼,有請於唐者,皆因鎔以聞。



 自晉兵出山東,已破孟遷,取邢、洺、磁三州,景福元年,乃大舉擊趙,下臨城。鎔求救於李匡威,匡威來救,晉軍解去。明年,晉會王處存攻鎔堅固、新市。



 晉王與處存皆自將,而鎔未嘗臨軍,遣追風都團練使段亮、翦寇都團練使馬珂等,以兵屬匡威而已。匡威戰磁河,晉軍大敗。明年春,晉攻天長軍,鎔出兵救之,敗于叱日嶺,晉軍遂出井陘。鎔又求救於
 匡威,晉軍解去。



 初,匡威悅其弟匡儔之婦美而淫之,匡儔怒,及其救鎔也,誘其軍亂而自立。



 匡威內慚不敢還,乃以符印歸其弟,而將奔于京師。行至深州,鎔德匡威救己,使人邀之,館于梅子園,以父事之。



 匡威客李正抱者,少游燕、趙間,每徘徊常山,愛之不能去。正抱、匡威皆失國無聊,相與登城西高閣,顧覽山川,泫然而泣,乃與匡威謀劫牛而代之。因詐為忌日,鎔去衛從,晨詣館慰,坐定,甲士自幕後出,持鎔兩袖,鎔曰:「吾國賴公而存,誠無以報厚德,今日之事,是所甘心。」因叩頭以位與匡威。匡威素少鎔,以謂無能為也,因與鎔方轡詣府,將代其
 位。行過親事營,軍士閉門大噪,天雨震電,暴風拔木,屋瓦皆飛。屠者墨君和望見鎔,識之,從缺垣中躍出,挾鎔於馬,負之而走,亂軍擊殺匡威、正抱,燕人皆死。匡儔雖憾其兄,而陽以大義責鎔甚急。



 鎔既失燕援,而晉軍急攻平山,劫鎔以盟,鎔遂與晉和。



 其後梁太祖下晉邢、洺、磁三州,乃為書詔鎔,使絕晉而歸梁,鎔依違不決。



 晉將李嗣昭復取洺州,梁太祖擊敗嗣昭,嗣昭棄洺州走。梁獲其輜重,得鎔與嗣昭書,多道梁事,太祖怒,因移兵常山,顧謂葛從周曰:「得鎮州以與爾,爾為我先鋒。」從周至臨城,中流矢,臥輿中,梁軍大沮。梁太祖自將
 傅城下,焚其南關,鎔懼,顧其屬曰:「事急矣!奈何?」判官周式,辨士也,對曰:「此難與力爭,而可以理奪也。」式與梁太祖有舊,因請入梁軍。太祖望見式,罵曰:「吾常以書招鎔不來,今吾至此,而爾為說客,晚矣!且晉吾仇也,而鎔附之,吾知李嗣昭在城中,可使先出。」乃以所得鎔與嗣昭書示式,式進曰:「梁欲取一鎮州而止乎,而欲成霸業於天下也?且霸者責人以義而不私,今天子在上,諸侯守封睦鄰,所以息爭,且休民也。昔曹公破袁紹,得魏將吏與紹書,悉焚之,此英雄之事乎!今梁知兵舉無名,而假嗣昭以為辭。且王氏五世六公撫有此士,豈無死士,而
 待嗣昭乎?」



 太祖大喜,起牽式衣而撫之曰:「吾言戲耳。」因延式於上坐,議與鎔和。鎔以子昭祚為質,梁太祖以女妻之。太祖即位,封鎔趙王。



 鎔祖母喪,諸鎮皆弔,梁使者見晉使在館,還言趙王有二志。是時,魏博羅紹威卒,梁因欲盡取河北,開平四年冬,遣供奉官杜廷隱監魏博將夏諲,以兵三千襲深、冀二州,以王景仁為北面行營招討使。鎔懼,乞兵于晉。晉人擊敗景仁於柏鄉,梁遂失鎮、定,而莊宗由此益彊,北破幽、燕,南并魏博,鎔常以兵從。鎔德晉甚。



 明年,會莊宗於承天軍,奉觴為壽,莊宗以鎔父友,尊禮之,酒酣為鎔歌,拔佩刀斷衣而盟,許以
 女妻鎔子昭誨。



 鎔為人仁而不武,未嘗敢為兵先,佗兵攻趙,常藉鄰兵為救。當是時,諸鎮相弊於戰爭,而趙獨安,樂王氏之無事,都人士女褒衣博帶,務夸侈為嬉游。鎔尤驕於富貴,又好左道,煉丹藥,求長生,與道士王若訥留游西山,登王母祠,使婦人維錦繡牽持而上。每出,逾月忘歸,任其政於宦者。宦者石希蒙與鎔同臥起。天祐十八年冬,鎔自西山宿鶻營莊,將還府,希蒙止之。宦者李弘規諫曰:「今晉王身自暴露以親矢石,而大王竭軍國之用為游畋之資,開城空宮,逾月不返,使一失閉門不納從者,大王欲何歸乎?」鎔懼,促駕,希蒙固止之。弘
 規怒,遣親事軍將蘇漢衡率兵擐甲露刃於帳前曰:「軍士勞矣!願從王歸。」弘規繼而進曰:「惑王者希蒙也,請殺之以謝軍士!」鎔不答,弘規呼鎔甲士斬希蒙首,擲於鎔前,鎔懼,遽歸。使其子昭祚與大將張文禮族弘規、漢衡,收其偏將下獄,窮究反狀,親軍皆懼。文禮誘以為亂,夜半,親軍千餘人踰垣而入,鎔方與道士焚香受籙,軍士斬鎔首,袖之而出,因縱火焚其宮室,遂滅王氏之族。



 鎔小子昭誨,年十歲,其軍士有德鎔者,藏之穴中,亂定,髡其髮,被以僧衣,遇湖南人李震,匿昭誨於茶籠中,載之湖南,依南嶽為浮圖,易名崇隱。明宗時,昭誨已
 長,思歸,而鎔故將符習為宣武軍節度使,震以歸習,習表於朝。昭誨自稱前成德軍中軍使以見,拜考功郎中、司農少卿。周顯德中,猶為少府監云。



 張文禮者,狡獪人也,鎔惑愛之,以為子,號王德明。鎔已死,文禮自為留後。



 莊宗初納之,後知其通於梁也,遣趙故將符習與閻寶擊之。文禮家鬼夜哭,野河水變為血,游魚皆死,文禮懼,病疽卒。子處瑾秘喪拒守,擊敗習等。以李嗣昭代之,嗣昭中流矢卒,以李存進代之,存進輒復戰歿,乃以符存審為招討使,遂破之。執文禮妻及子處瑾、處球、處琪等,折足歸于晉。趙人請而醢之,磔文禮尸于市。



 羅紹威羅紹威,字端己,其先長沙人。祖讓,北遷為魏州貴鄉人。父弘信,為牧監卒。



 文德元年,魏博牙軍亂,遂殺其帥樂彥貞,立其將趙文建為留後,已而又殺之。牙將未知所立,乃聚呼曰:「孰能為我帥者?」弘信從眾中出應曰:「我可為君等帥也。」弘信狀貌奇怪,面色青黑,軍中異之,共立為留後。唐昭宗即位,拜弘信節度使。



 梁太祖將攻晉,乞糴于弘信,弘信不與,由是有隙。梁兵攻魏,取黎陽、淇門、衛縣。戰于內黃,魏兵五戰五敗,弘信懼,請盟,乃止。是時,梁方東攻兗、鄆,北敵晉,晉遣李存信救朱宣,假道于魏。太祖聞,遣使語弘信曰:「晉人志在河朔,兵
 還滅魏矣。」弘信以為然,乃發兵擊存信於莘縣,太祖遣葛從周助之。梁兵擒晉王子落落,送于魏,弘信殺之,乃與晉絕。太祖猶疑弘信有二心,乃以兄事弘信,常為卑辭厚幣以聘魏。魏使者至梁,太祖北面拜而受幣,謂使者曰:「六兄於我有倍年之長,吾何敢慢之。」弘信大喜,以為厚己。以故太祖往來燕、趙之間,卒有河北者,魏不為之患也。弘信死,紹威立。



 紹威好學工書,頗知屬文,聚書數萬卷,開館以延四方之士。弘信在唐,以其先長沙人,故封長沙郡王,紹威襲父爵長沙。紹威新立,幽州劉仁恭以兵十萬攻魏,屠貝州,紹威求救於梁,大敗燕軍
 於內黃。明年,梁太祖遣葛從周會魏兵攻滄州,取其德州,遂敗燕軍於老鴉隄,紹威以故德梁助己。



 魏博自田承嗣始有牙軍,牙軍歲久益驕,至紹威時已二百年,父子世相婚姻以自結。前帥史憲誠、何全皞、韓君雄、樂彥貞等,皆由牙軍所立,怒輒遂殺之。紹威為人精悍明敏,通習吏事,為政有威嚴,然其家世由牙軍所立。天祐二年,魏州城中地陷,紹威懼有變。已而牙校李公牷作亂,紹威誅之,乃間遣使告梁乞兵,欲盡誅牙軍。梁太祖許之,為遣李思安等攻滄州,召兵於魏,紹威因悉發魏兵以從,獨牙軍在。



 紹威子廷規娶梁女,會梁女卒,
 太祖陰遣客將馬嗣勛選良兵實輿中,以長直軍千人雜輿夫入魏,詐為助葬,太祖以兵繼其後。紹威夜以奴兵數百,會嗣勛兵擊牙軍,并其家屬盡殺之。太祖自內黃馳至魏,魏兵從攻滄州者行至歷亭,聞之皆反,入澶、博諸州,魏境大亂,數月,太祖為悉平之。牙軍死,魏兵悉叛,紹威勢益孤,太祖乃欲奪其地,紹威始大悔。是歲,太祖復攻滄州,宿兵長蘆,紹威饋給梁兵,自滄至魏五百里,起亭堠,供帳什物自具,梁兵數十萬皆取足,紹威以此重困。昭宗東遷洛陽,詔諸鎮繕理京師,紹威營太廟成,加拜守侍中,進封鄴王。



 太祖圍滄州未下,劉守光
 會晉軍破梁潞州。太祖自長蘆歸,過魏,疾作,臥府中,諸將莫得見,紹威懼太祖終襲己,乃乘間入見曰:「今四方稱兵,為梁患者,以唐在故也;唐家天命已去,不如早自取之。」太祖大喜,乃急歸。太祖即位,將都洛陽,紹威取魏良材為五鳳樓、朝元前殿,浮河而上,立之京師。太祖嘆曰:「吾聞蕭何守關中,為漢起未央宮,豈若紹威越千里而為此,若神化然,功過蕭何遠矣!」賜以寶帶名馬。



 燕王劉守光囚其父仁恭,與其兄守文有隙,紹威馳書勸守光等降梁。太祖聞之笑曰:「吾常攻燕不能下,今紹威折簡,乃勝用兵十萬。」太祖每有大事,多遣使者問之,紹威
 時亦馳簡入白,使者相遇道中,其事往往相合。



 紹威自以魏久不用兵,願伐木安陽淇門為船,自河入洛,歲漕穀百萬石,以供京師。太祖益以紹威盡忠,遣將程厚、盧凝督其役。舟未成而紹威病,乃表言:「魏故大鎮,多外兵,願得梁一有功重臣臨之,請以骸骨就第。」太祖亟命其子周翰監府事,語使者曰:「亟行,語而主,為我彊飯,如有不諱,當世世貴爾子孫。



 今使周翰監府事,尚冀卿復愈耳。」紹威仕梁,累拜太師兼中書令,卒年三十四,贈尚書令,謚曰貞壯。



 子三人,廷規,官至司農卿卒。周翰襲父位,乾化二年八月為楊師厚所逐,徙為宣義軍節度使,
 卒于官,年十四。周敬代為宣義軍節度使,年十歲,徙鎮忠武。



 明年,為秘書監、駙馬都尉、光祿卿。唐莊宗時為金吾大將軍,明宗以為匡國軍節度使,罷為上將軍。晉天福二年卒,年三十二。廷規娶梁太祖二女,一曰安陽公主,一曰金華公主。周翰娶末帝女,曰壽春公主,周敬亦娶末帝女,曰晉安公主。



 王處直王處直,字允明,京兆萬年人也。父宗,善殖財貨,富擬王侯,為唐神策軍吏,官至金吾大將軍,領興元節度使,子處存、處直。處存以父任為驍衛將軍、定州已來制置內閑廄宮苑等使。乾符六年,即拜義武軍節度使。黃巢陷
 長安,處存感憤流涕,率鎮兵入關討賊。巢敗第功,而收城擊賊,李克用為第一;勤王倡義,處存為第一。乾寧二年,處存卒于鎮,三軍以河朔故事,推處存子郜為留後,即拜節度使,加檢校司空、同中書門下平章事。處直為後院中軍都知兵馬使。



 光化三年,梁兵攻定州,郜遣處直率兵拒之,戰于沙河,為梁兵所敗。兵返入城逐郜,郜出奔晉,亂兵推處直為留後。梁兵圍之,處直遣人告梁,請絕晉而事梁,出絹十萬匹犒軍,乃與梁盟。梁太祖表處直義武軍節度使,累封太原王。太祖即位,封處直北平王。其後梁兵攻王鎔,鎔求救于晉,處直亦遣人至
 晉,願絕梁以自效。



 晉兵救鎔,處直以兵五千從,破梁軍於柏鄉。其後晉北破燕,南取魏博,與梁戰河上,十餘年,處直未嘗不以兵從。



 處直好巫,而客有李應之者,妖妄人也。處直有疾,應之以左道治之而愈,處直益以為神,使衣道士服,以為行營司馬,軍政無大小,咸取決焉。初,應之於陘邑闌得小兒劉雲郎,養以為子,而處直未有子,乃以雲郎與處直,而紿曰:「此子生而有異。」處直養以為子,更名曰都,甚愛之。應之由此益橫,乃籍管內丁壯,別立新軍,自將之,治第博陵坊,四面開門,皆用左道。處直將吏知其必為患,而莫能諫也。是時,幽州李匡儔假
 道中山以如京師,處直伏甲城外,以備不虞。匡儔已去,甲士入城圍應之第,執而殺之,因詣處直請殺都,處直不與。明日,第功行賞,因陰疏甲士姓名,自隊長已上藏于別籍,其後因事誅之,凡二十年,無一人免者,而處直終為都所殺。



 都為人狡佞多謀,處直以為節度副使。張文禮弒王鎔,莊宗發兵討文禮,處直與左右謀曰:「鎮,定之蔽也,文禮雖有罪,然鎮亡定不獨存。」乃遣人請莊宗毋發兵,莊宗取所獲文禮與梁蠟書示處直曰:「文禮負我,師不可止。」處直有孽子郁,當郜之亡于晉也,郁亦奔焉,晉王以女妻之,為新州防禦使。處直見莊宗必討文
 禮,益自疑,乃陰與郁交通,使郁北招契丹入塞以牽晉兵,且許召郁為嗣,都聞之不說。而定人皆言契丹不可召,恐自貽患,處直不聽。郁自奔晉,常恐處直不容,因此大喜,以為乘其隙可取之,乃以厚賂誘契丹阿保機。阿保機舉國入寇,定人皆不欲契丹之舉,小吏和昭訓勸都舉事,都因執處直,囚之西宅,自為留後,凡王氏子孫及處直將校殺戮殆盡。明年正月朔旦,都拜處直於西宅,處直奮起揕其胸而呼曰:「逆賊!吾何負爾?」然左右無兵,遂欲齧其鼻,都掣袖而走,處直遂見殺。



 初,有黃蛇見于碑樓,處直以為龍,藏而祠之,又有野鵲數百,巢麥田
 中,處直以為己德所致,而定人皆知其不祥,曰:「蛇穴山澤,而處人室,鵲巢烏,降而田居,小人竊位,而在上者失其所居之象也。」已而處直果被廢死。



 莊宗已敗契丹於沙河,追奔過定州,與都相得懽甚,以其子繼岌娶都女,以都為義武軍節度使。同光二年,莊宗幸鄴,都來朝,賜與巨萬。莊宗以繼岌故,待都甚厚,所請無不從。及明宗立,頗惡都為人,而安重誨每以法繩之,都始有異志。



 是時,唐兵擊契丹,數往來定州,都供饋多闕,益不自安。和昭訓為都謀曰:「天子新立,四方未附,其勢易離,可為自安之計。」已而朱守殷反於汴州,都遂亦反,遣人以蠟書
 招青、徐、岐、潞、梓五鎮,約皆舉兵,而五鎮不應。明宗遣王晏球討之。都復與王郁招契丹為援,契丹遣禿餒將萬騎救都。都遣指揮使鄭季璘、龍泉鎮將杜弘壽以二千人迎契丹,為晏球所敗。季璘、弘壽被執,晏球責曰:「吾嘗使人招汝,何故不降?」弘壽對曰:「受恩中山兩世矣,不敢有二心。」遂見殺,弘壽臨刑,神色自若。晏球屯軍望都,與都及契丹戰,大敗之曲陽,都及禿餒得數騎遁去,閉城不復出。



 初,莊宗軍中闌得一男子,愛之,使冒姓李,名繼陶,養於宮中以為子。明宗即位,安重誨出以乞段徊,徊亦惡而逐之。都使人求得之。至是,紿其眾曰:「此莊宗太
 子也。」被以天子之服,使巡城上,以示晏球軍,軍士識者曰:「繼陶也。」



 共詬之。都居城中,兵少,惟以契丹二千人守城,呼禿餒為餒王,屈身事之。諸將有欲出降者,都伺察嚴密,殺戮無虛日,以故堅守經年。天成四年二月,城破,都與家屬皆自焚死,王氏遂絕于中山。而處存有子鄴,鄴子廷胤,與莊宗連外姻,為人驍勇,自為軍校,能與士卒同辛苦,明宗時,歷貝、忻、密、澶、隰州刺史。范延光反於鄴,晉高祖以廷胤為楊光遠行營中軍使。破延光有功,拜彰德軍節度使。



 初,處直為都所囚,幼子威北走契丹。契丹謂晉高祖曰:「吾欲使威襲其先人爵士,如何?」高祖
 對曰:「中國之法,自將校為刺史,升團練防禦而至節度使,請送威歸中國,漸進之。」契丹怒曰:「爾自諸侯為天子,豈有漸乎?」高祖聞之,遽徙廷胤鎮義武,曰:「此亦王氏之後也。」後徙鎮海而卒。



 劉守光劉守光,深州樂壽人也。其父仁恭,事幽州李可舉,能穴地為道以攻城,軍中號「劉窟頭。」稍以功遷軍校。仁恭為人有勇,好大言。可舉死,子匡威惡其為人,不欲使居軍中,徙為瀛州景城縣令。瀛州軍亂,殺刺史,仁恭募縣中得千人,討平之,匡威喜,復以為將,使戍蔚州。戍兵過期不得代,皆思歸,出怨言。匡威為弟匡儔所逐,仁恭聞亂,
 乃擁戍兵攻幽州,行至居庸關,戰敗,奔晉、晉以為壽陽鎮將。



 仁恭多智詐,善事人,事晉王愛將蓋寓尤謹,每對寓涕泣,自言:「居燕無罪,以讒見逐。」因道燕虛實,陳可取之謀,晉王益信而愛之。乾寧元年,晉擊破匡儔,乃以仁恭為幽州留後,留其親信燕留得等十餘人監其軍,為之請命于唐,拜檢校司空、盧龍軍節度使。



 其後晉攻羅弘信,求兵於仁恭,仁恭不與,晉王以書微責誚之,仁恭大怒,執晉使者,殺燕留得等以叛。晉王自將討之,戰于安塞,晉王大敗。光化元年,遣其子守文襲滄州,逐節度使盧彥威,遂取滄、景、德三州。為其子請命於唐,昭宗
 遲之,未即從,仁恭怒,語唐使者曰:「為我語天子,旌節吾自有,但要長安本色爾,何屢求而不得邪!」昭宗卒以守文為橫海軍節度使。



 仁恭父子率兩鎮兵十萬,號稱三十萬以擊魏,屠貝州。羅紹威求救於梁,梁遣李思安救魏,大敗守文於內黃,斬首五萬。仁恭走,梁軍追擊之,自魏至長河,橫尸數百里。梁軍自是連歲攻之,破其瀛、漠二州,仁恭懼,復附晉。



 天祐三年,梁攻滄州,仁恭調其境內凡男子年十五已上、七十已下,皆黥其面,文曰「定霸都」,得二十萬人,兵糧自具,屯于瓦橋。梁軍壁長蘆,深溝高壘,仁恭不能近。滄州被圍百餘日,城中食盡,人自相
 食,析骸而爨,或丸墐土而食,死者十六七。仁恭求救於晉,晉王為之攻潞州以牽梁圍,晉破潞州,梁軍乃解去。



 然仁恭幸世多故,而驕於富貴,築宮大安山,窮極奢侈,選燕美女充其中。又與道士煉丹藥,冀可不死。令燕人用墐土為錢,悉斂銅錢,鑾山而藏之,已而殺其工以滅口,後人皆莫知其處。



 仁恭有愛妾羅氏,其子守光烝之,仁恭怒,笞守光,逐之。梁開平元年,遣李思安攻仁恭,仁恭在大安,守光自外將兵以入,擊走思安,乃自稱盧龍節度使,遣李小喜、元行欽以兵攻大安山,執仁恭而幽之。其兄守文聞父且囚,即率兵討守光,至于盧臺,為
 守光所敗,進戰玉田,又敗,乃乞兵于契丹。明年,守文將契丹、吐渾兵四萬人戰于雞蘇,守光兵敗,守文陽為不忍,出於陣而呼其眾曰:「毋殺吾弟!」



 守光將元行欽識守文,躍馬而擒之,又囚之於別室,既而殺之。守文將吏孫鶴、呂兗等,立守文子延祚以距守光,守光圍之百餘日,城中食盡,米斛直錢三萬,人相殺而食,或食墐土,馬相食其駿尾,兗等率城中飢民食以麴,號「宰務」,日殺以餉軍。久之,延祚力窮,遂降。



 守光素庸愚,由此益驕,為鐵籠、鐵刷,人有過者,坐之籠中,外燎以火,或刷剔其皮膚以死,燕之士逃禍于佗境。守光身衣赭黃,謂其將吏曰:「
 我衣此而南面,可以帝天下乎?」孫鶴切諫以為不可。梁攻趙,趙王王鎔求救於守光,孫鶴曰:「今趙無罪,而梁伐之,諸侯救趙之兵,先至者霸,臣恐燕軍未出,而晉已先破梁矣,此不可失之時也。」守光曰:「趙王嘗與我盟而背之,今急乃來歸我;且兩虎方鬥,可待之,吾當為卞莊子也。」遂不出兵。晉王果救趙,大敗梁兵於柏鄉,進掠邢、洺,至於黎陽。守光聞晉空國深入梁,乃治兵戒嚴,遣人以語動鎮、定曰:「燕有精兵三十萬,率二鎮以從晉,然誰當主此盟者?」晉人患之,謀曰:「昔夫差爭黃池之會,而越入吳;項羽貪伐齊之利,而漢敗楚。今吾越千里以伐人,
 而彊燕在其後,此腹心之患也。」乃為之班師。



 守光益以為諸鎮畏其強,乃諷諸鎮共推尊己,於是晉王率天德宋瑤、振武周德威、昭義李嗣昭、義武王處直、成德王鎔等,以墨制冊尊守光為尚書令、尚父。守光又遣告于梁,請授己河北兵馬都統,以討鎮、定、河東。梁遣閣門使王瞳拜守光河北採訪使。有司白守光,尚父受冊,用唐冊太尉禮儀,守光問曰:「此儀注何不郊天改元?」有司曰:「此天子之禮也,尚父雖尊,乃人臣耳。」守光怒曰:「我為尚父,誰當帝者乎?且今天下四分五裂,大者稱帝,小者稱王,我以二千里之燕,獨不能帝一方乎?」乃械梁、晉使者
 下獄,置斧鑕于其庭,令曰:「敢諫者死!」



 孫鶴進曰:「滄州之敗,臣蒙王不殺之恩,今日之事,不敢不諫。」守光怒,推之伏鑕,令軍士割而啖之。鶴呼曰:「不出百日,大兵當至!」命窒其口而醢之。



 守光遂以梁乾化元年八月自號大燕皇帝,改元曰應天,以王瞳、齊涉為左右相。晉遣太原少尹李承勳賀冊尚父,至燕,而守光已僭號。有司迫承勳稱臣,承勛不屈,以列國交聘禮入見,守光怒,殺之。



 明年,晉遣周德威將三萬人,會鎮、定之兵以攻燕,自祈溝關入,其澶、涿、武、順諸州皆迎降。守光被圍經年,累戰常敗,乃遣客將王遵化致書於德威曰:「予得罪於晉,迷而不
 復,今其病矣,公善為我辭焉。」德威謂遵化曰:「大燕皇帝尚未郊天,何至此邪?予受命以討僭亂,不知其佗也。」守光益窘,乃獻絹千匹、銀千兩、錦百段,遣其將周遵業謂德威曰:「吾王以情告公,富貴成敗,人之常理;錄功宥過,霸者之事也。守光去歲妄自尊崇,本不能為朱溫下耳,豈意大國暴師經年,幸少寬之。」德威不許。守光登城呼德威曰:「公三晉賢士,獨不急人之危乎?」



 遣人以所乘馬易德威馬而去,因告曰:「俟晉王至則降。」晉王乃自臨軍,守光登城見晉王,晉王問將如何?守光曰:「今日俎上肉耳,惟王所為也!」守光有嬖者李小喜,勸其毋降,守光因
 請俟佗日。是夕,小喜叛降于晉軍。明旦,晉軍攻破其城,執仁恭及其家族三百口。



 守光與其妻李氏、祝氏,子繼珣、繼方、繼祚等,南走滄州,迷失道,至燕樂界中,數日不得食,遣其妻祝氏乞食于田家,田家怪而詰之,祝氏以實告,乃被擒送幽州。晉王方大饗軍,客將引守光見,晉王戲之曰:「主人何避客之遽也?」守光叩頭請死,命械守光並其父仁恭以從軍。軍還過趙,趙王王鎔會晉王,置酒,酒酣請曰:「願見仁恭父子。」晉王命破械出之,引置下坐。飲食自若,皆無慚色。



 晉王至太原,仁恭父子曳以組練,獻于太廟。守光將死,泣曰:「臣死無恨,然教臣不降者,
 李小喜也,罪人不死,臣將訴於地下。」晉王使召小喜,小喜真目曰:「囚父弒兄,蒸其骨肉,亦小喜教爾邪?」晉王怒,命先斬小喜。守光知不免,呼曰:「王將復唐室以成霸業,何不赦臣使自效?」其二婦從旁罵曰:「事已至此,生復何為?願先死!」乃俱死。晉王命李存霸執仁恭至雁門,刺其心血以祭先王墓,然後斬之。



\end{pinyinscope}