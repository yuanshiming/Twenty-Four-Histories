\article{卷七十五五代史記序}

\begin{pinyinscope}

 孟子曰:三代之得天下也以仁,其失天下也以不仁。自生民以來,一治一亂,旋相消長,未有去仁而興、積仁而亡者。甚哉,五代不仁之極也,其禍敗之復,殄滅剝喪之威,亦其效耳。夫國之所以存者以有民,民之所以生者以有君。方是時,上之人以慘烈自任,刑戮相高,兵革不休,夷滅構禍,置君猶易吏,變國若傳舍,生民膏血塗草野,骸骼暴原隰,君民相視如髦蠻草木,幾何其不胥為夷也!逮皇天悔禍,真人出寧,易暴以仁,轉禍以德,民咸保其首領,收其族屬,各正性命,豈非天邪!方夷夏相蹂,兵連亂結,非無忠良豪傑之士竭謀殫智,以緩民之死,乃湮沒而無聞矣。否閉極而泰道升,聖人作而萬物睹,指揮中原,兵不頓刃,向之滔天巨猾,搖毒煽禍以害斯人者,蹈鼎鑊斧金質之不暇,豈非人邪!天與人相為表裏,和同於無間。聖人知
 天之所助,人之所歸,國之所恃以為固者,仁而已,非特三代然也。堯舜之盛,漢
 唐之興,秦隋之失,魏晉之亡,南北之亂,莫不由此也。五代距今百有餘年,故老遺俗,往往垂絕,無能道說者,史官秉筆之士,或文採不足以耀無窮,道學不足以繼述作,使五十有餘年間,廢興存亡之跡,奸臣賊子之罪,忠臣義士之節,不傳於後世,來者無所考焉。惟廬陵歐陽公,慨然以自任,蓋潛心累年而後成書,其事跡實錄,詳於舊記,而褒貶義例,仰師《春秋》,由遷、固而來,未之有也。至於論
 朋黨宦女,忠孝兩全,義子降服,豈小補哉,豈小補哉!



\end{pinyinscope}