\article{卷五十二雜傳第四十}

\begin{pinyinscope}

 杜重威杜重威,朔州人也。其妻石氏,晉高祖之女弟。高祖即帝位,封石氏為公主,拜重威舒州刺史,以典禁兵。從侯益攻破張從賓於汜水,以功拜潞州節度使。范延光反於鄴,重威從高祖攻降延光,徙領忠武,加同平章事。又徙領天平,遷侍衛親軍都指揮使。



 安重榮反,重威逆戰於宗城,重榮為偃月陣,重威擊之不動。重威欲少卻以伺之,偏將王重胤曰:「兩兵方交,退者先敗。」乃分兵為三,重
 威先以左右隊擊其兩翼,戰酣,重胤以精兵擊其中軍,重榮將趙彥之來奔,重榮遂大敗,走還鎮州,閉壁不敢出。重威攻破之,以功拜重威成德軍節度使。



 重威出於武卒,無行而不知將略。破鎮州,悉取府庫之積及重榮之貲,皆沒之家,高祖知而不問。及出帝與契丹絕好,契丹連歲入寇,重威閉城自守,屬州城邑多所屠戮。胡騎驅其人民千萬過其城下,重威登城望之,未嘗出救。



 開運元年,加重威北面行營招討使。明年,引兵攻泰州,破滿城、遂城。契丹已去至古北,還兵擊之,重威等南走,至陽城,為虜所困,賴符彥卿、張彥澤等因大風奮擊,契丹
 大潰。諸將欲追之,重威為俚語曰:「逢賊得命,更望復子乎?」



 乃收馬馳歸。



 重威居鎮州,重斂其民,戶口凋敝,又懼契丹之至,乃連表乞還京師。未報,亟上道,朝廷莫能止,即拜重威為鄴都留守。而鎮州所留私粟十餘萬斛,殿中監王欽祚和市軍儲,乃錄以聞,給絹數萬匹以償之,重威大怒曰:「吾非反者,安得籍沒邪!」



 三年秋,契丹高牟翰詐以瀛州降,復以重威為北面行營招討使。是秋,天下大水,霖雨六十餘日,飢殍盈路,居民拆木以供爨,剉槁席以秣馬牛,重威兵行泥潦中,調發供饋,遠近愁苦。重威至瀛州,牟翰已棄城去,重威退屯武強。契丹寇鎮、
 定,重威西趨中渡橋,與虜夾滹沱河而軍。偏將宋彥筠、王清渡水力戰,而重威按軍不動,彥筠遂敗,清戰死。轉運使李穀教重威以三腳木為橋,募敢死士過河擊賊,諸將皆以為然,獨重威不許。



 契丹遣騎兵夜並西山擊欒城,斷重威軍後。是時,重威已有異志,而糧道隔絕,乃陰遣人詣契丹請降。契丹大悅,許以中國與重威為帝,重威信以為然,乃伏甲士,召諸將告以降虜。諸將愕然,以上將先降,乃皆聽命。重威出降表使諸將書名,乃令軍士陣於柵外,軍士猶喜躍以為決戰,重威告以糧盡出降,軍士解甲大哭,聲震原野。契丹賜重威赭袍,使衣
 以示諸軍,拜重威太傅。



 契丹犯京師,重威以晉兵屯陳橋,士卒飢凍,不勝其苦。重威出入道中,市人隨而詬之,重威俯首不敢仰顧。契丹據京師,率城中錢帛以賞軍,將相皆不免,重威當率萬緡,乃訴於契丹曰:「臣以晉軍十萬先降,乃獨不免率乎?」契丹笑而免之,遣還鄴都。明年,契丹北歸,重威與其妻石氏詣虜帳中為別。



 漢高祖定京師,拜重威太尉、歸德軍節度使,重威懼,不受命。遣高行周攻之,不克,高祖乃自將攻之。遣給事中陳同以詔書召之,重威不聽命,而漢兵數敗,圍之百餘日。初,契丹留燕兵千五百人在京師,高祖自太原入,告者言其
 將反,高祖悉誅於繁臺,其亡者奔于鄴。燕將張璉先以兵二千在鄴,聞燕兵見殺,乃勸重威固守。高祖已殺燕兵,悔之,數遣人招璉等,璉登城呼曰:「繁臺之誅,燕兵何罪?



 既無生理,請以死守!」重威食盡,屑麴而食,民多逾城出降,皆無人色。重威乃遣判官王敏及其妻相次請降,高祖許之。重威素服出見高祖,高祖赦重威,拜檢校太師、守太傅、兼中書令。悉誅璉及重威將吏,而錄其私帑,以重威歸京師。



 高祖病甚,顧大臣曰:「善防重威!」高祖崩,祕不發喪,大臣乃共誅之,及其子弘璋、弘璨、弘璲尸於市,市人蹴而詬之,吏不能禁,支裂蹈踐,斯須而盡。



 李守貞李守貞,河陽人也。晉高祖鎮河陽,以為客將,其後嘗從高祖,高祖即位,拜客省使。監馬全節軍破李金全於安州,以功拜宣徽使。



 出帝即位,楊光遠反,召契丹入寇。守貞領義成軍節度使,為侍衛親軍都虞候,從出帝幸澶州。麻荅以奇兵入鄆州,渡馬家口,柵於河東。守貞馳往破之,契丹兵多溺死,獲馬數百匹,裨將七十餘人。徙領泰寧軍節度使,以兵二萬討之。光遠降,其故吏宋顏悉取光遠寶貨、名姬、善馬獻之守貞,守貞德之,陰置顏麾下。是時,凡出師破賊,必有德音赦其餘類。而光遠黨與十餘人皆亡命,捕之甚急,樞密使桑維翰緩其制書,久
 而不下。言事者告顏匿守貞所,詔取顏殺之,守貞大怒,乃與維翰有隙。



 賊平行賞,守貞悉以黦茶染木給之,軍中大怒,以帛裹之為人首,梟於木間,曰:「守貞首也。」守貞以功拜同平章事,賜以光遠舊第,守貞取旁官民舍大治之,為京師之甲。出帝臨幸,燕錫恩禮,出於諸將。



 契丹入寇,出帝再幸澶州,杜重威為北面招討使,守貞為都監。晉兵素驕,而守貞、重威為將皆無節制,行營所至,居民豢圉一空,至於草木皆盡。其始發軍也,有賜賚,曰「掛甲錢」,及班師,又加賞勞,曰「卸甲錢」,出入之費,常不下三十萬,由此晉之公私重困。守貞與重威等攻下泰州,破
 滿城,殺二千餘人。還,為侍衛親軍都指揮使,領天平軍節度使,又領歸德。



 是時,出帝遣人以書招趙延壽使歸國,延壽詐言思歸,願得晉兵為應,而契丹高牟翰亦詐以瀛州降,出帝以為然,命杜重威等將兵應之。初,晉大臣皆言重威不忠,有怨望之心,不可用,乃用守貞。是時,重威鎮魏州,守貞嘗將兵往來過魏,重威待之甚厚,多以戈甲金帛奉之。出帝嘗謂守貞曰:「卿常以家財散士卒,可謂忠於國者乎!」守貞謝曰:「皆重威與臣者。」因請與重威俱北。於是卒以重威為招討使,守貞為都監,屯于武強。契丹寇鎮、定,守貞等軍於中渡,遂與重威降於契
 丹。契丹以守貞為司徒。契丹犯京師,拜守貞天平軍節度使。



 漢高祖入京師,守貞來朝,拜太保、河中節度使。高祖崩,杜重威死,守貞懼,不自安,以謂漢室新造,隱帝初立,天下易以圖,而門下僧總倫以方術陰乾守貞,為言有非常之相,守貞乃決計反。而趙思綰先以京兆反,遣人以赭黃衣遺守貞,守貞大喜,以為天人皆應,乃發兵西據潼關,招誘草寇,所在竊發。漢遣白文珂、常思等出軍擊之。已而王景崇又以鳳翔反,景崇與思綰遣人推守貞為秦王,守貞拜景崇等官爵。又遣人間以蠟丸書遺吳、蜀、契丹,使出兵以牽漢。



 文珂等攻景崇、思綰等久無
 功,隱帝乃遣樞密使郭威率禁兵將文珂等督攻之。



 諸將皆請先擊思綰、景崇,威計未知所向。行至華州,節度使扈彥珂謂威曰:「三叛連衡,以守貞為主,守貞先敗,則思綰、景崇可傳聲而破矣。若捨近圖遠,使守貞出兵於後,思綰、景崇拒戰于前,則漢兵屈矣。」威以為然,遂先擊守貞。



 是時,馮道罷相居河陽,威初出兵,過道家問策,道曰:「君知博乎?」威少無賴,好蒲博,以為道譏之,艴然而怒。道曰:「凡博者錢多則多勝,錢少則多敗,非其不善博,所以敗者,勢也。今合諸將之兵以攻一城,較其多少,勝敗可知。」



 威大悟,謀以遲久困之,乃與諸將分為三柵,柵
 其城三面,而闕其南,發五縣丁夫築長城以連三柵。守貞出其兵壞長城,威輒補其所壞,守貞輒出爭之,守貞兵常失十三四,如此逾年,守貞城中兵無幾,而食又盡,殺人而食。威曰:「可矣。」乃為期日,督兵四面攻而破之。初,守貞召總倫問以濟否,總倫曰:「王當自有天下,然分野方災,俟殺人垂盡,則王事濟矣。」守貞以為然。嘗會將吏大飲,守貞指畫虎圖曰:「吾有天命者中其掌。」引弓一發中之,將吏皆拜賀,守貞益以自負。城破,守貞與妻子自焚,漢軍入城,於煙燼中斬其首,傳送京師,梟於南市,其餘黨皆磔之。



 張彥澤張彥澤,其先突厥部人也。後徙居陰山,又徙太原。彥澤為人驍悍殘忍,目睛黃而夜有光,顧視如猛獸。以善射為騎將,數從莊宗、明宗戰伐。與晉高祖連姻,高祖時,已為護聖右廂都指揮使、曹州刺史。與討范延光,拜鎮國軍節度使,歲中,徙鎮彰義。



 為政暴虐,常怒其子,數笞辱之。子逃至齊州,州捕送京師,高祖以歸彥澤。



 彥澤上章請殺之,其掌書記張式不肯為作章,屢諫止之。彥澤怒,引弓射式,式走而免。式素為彥澤所厚,多任以事,左右小人皆素嫉之,因共讒式,且迫之曰:「不速去,當及禍。」式乃出奔。彥澤遣指揮使李興以二十騎追之,戒曰:「式不
 肯來,當取其頭以來!」式至衍州,刺史以兵援之門邠州,節度使李周留式,馳騎以聞,詔流式商州。彥澤遣司馬鄭元昭詣闕論請,期必得式,且曰:「彥澤若不得張式,患在不測。」高祖不得已,與之。彥澤得式,剖心、決口、斷手足而斬之。



 高祖遣王周代彥澤,以為右武衛大將軍。周奏彥澤所為不法者二十六條,并述涇人殘敝之狀,式父鐸詣闕訴冤,諫議大夫鄭受益、曹國珍,尚書刑部郎中李濤、張麟,員外郎麻麟、王禧伏閣上疏,論彥澤殺式之冤,皆不省。濤見高祖切諫,高祖曰:「彥澤功臣,吾嘗許其不死。」濤厲聲曰:「彥澤罪若可容,延光鐵券何在!」



 高祖怒,起
 去,濤隨之諫不已,高祖不得已,召式父鐸、弟守貞、子希範等,皆拜以官,為蠲涇州民稅,免其雜役一年,下詔罪己,然彥澤止削階、降爵而已。於是國珍等復與御史中丞王易簡率三院御史詣閣門連疏論之,不報。



 出帝時,彥澤為左龍武軍大將軍,遷右武衛上將軍,又遷右神武統軍。自契丹與晉戰河北,彥澤在兵間,數立戰功,拜彰國軍節度使。與契丹戰陽城,為契丹所圍,而軍中無水,鑿井輒壞,又天大風,契丹順風揚塵奮擊甚銳,軍中大懼。彥澤以問諸將,諸將皆曰:「今虜乘上風,而吾居其下,宜待風回乃可戰。」彥澤以為然。諸將皆去,偏將藥
 元福獨留,謂彥澤曰:「今軍中飢渴已甚,若待風回,吾屬為虜矣!且逆風而戰,敵人謂我必不能,所謂出其不意。」彥澤即拔拒馬力戰,契丹奔北二十餘里,追至衛村,又大敗之,契丹遁去。



 開運三年秋,杜重威為都招討使,李守貞兵馬都監,彥澤馬軍都排陣使。彥澤往來鎮、定之間,敗契丹于泰州,斬首二千級。重威、守貞攻瀛州不克,退及武彊,聞契丹空國入寇,惶惑不知所之,而彥澤適至,言虜可破之狀,乃與重威等西趨鎮州。彥澤為先鋒,至中渡橋,已為虜所據,彥澤猶力戰爭橋,燒其半,虜小敗卻,乃夾河而寨。



 十二月丙寅,重威、守貞叛降契丹,彥
 澤亦降。耶律德光犯闕,遣彥澤與傅住兒以二千騎先入京師,彥澤倍道疾驅,至河,銜枚夜渡。壬申夜五鼓,自封丘門斬關而入。有頃,宮中火發,出帝以劍擁後宮十餘人將赴火,為小吏薛超所持。彥澤自寬仁門傳德光與皇太后書入,乃滅火。大內都點檢康福全宿衛寬仁門,登樓覘賊,彥澤呼而下之,諸門皆啟。彥澤頓兵明德樓前,遣傅住兒入傳戎王宣語,帝脫黃袍,素服再拜受命。使人召彥澤,彥澤謝曰:「臣無面目見陛下。」復使召之,彥澤笑而不答。



 明日,遷帝於開封府,帝與太后、皇后肩輿,宮嬪、宦者十餘人皆步從。彥澤遣控鶴指揮使李筠以
 兵監守,內外不通。帝與太后所上德光表章,皆先示彥澤乃敢遣。帝取內庫帛數段,主者曰:「此非帝有也。」不與。又使求酒於李崧,崧曰:「臣家有酒非敢惜,慮陛下憂躁,飲之有不測之虞,所以不敢進。」帝姑烏氏公主私賂守門者,得入與帝訣,歸第自經死。德光渡河,帝欲郊迎,彥澤不聽,遣白德光,德光報曰:「天無二日,豈有兩天子相見於道路邪!」乃止。



 初,彥澤至京師,李濤謂人曰:「吾禍至矣!與其逃於溝竇而不免,不若往見之。」濤見彥澤,為俚語以自投死,彥澤笑而厚待之。



 彥澤自以有功於契丹,晝夜酣飲自娛,出入騎從常數百人,猶題其旗幟曰「赤
 心為主」。迫遷出帝,遂輦內庫,輸之私第,因縱軍士大掠京師。軍士邏獲罪人,彥澤醉不能問,真目視之,出三手指,軍士即驅出斷其腰領。皇子延煦母楚國夫人丁氏有色,彥澤使人求於皇太后,太后遲疑未與,即劫取之。彥澤與閣門使高勳有隙,乘醉入其家,殺數人而去。



 耶律德光至京師,聞彥澤劫掠,怒,鎖之。高勳亦自訴於德光,德光以其狀示百官及都人,問:「彥澤當誅否?」百官皆請不赦,而都人爭投狀疏其惡,乃命高勳監殺之。彥澤前所殺士大夫子孫,皆縗絰杖哭,隨而詬詈,以杖朴之,彥澤俯首無一言。行至北市,斷腕出鎖,然後用刑,勳剖其
 心祭死者,市人爭破其腦,取其髓,臠其肉而食之。



 嗚呼,晉之事醜矣,而惡亦極也!其禍亂覆亡之不暇,蓋必然之理爾。使重威等雖不叛以降虜,亦未必不亡;然開虜之隙,自一景延廣,而卒成晉禍者,此三人也。視重威、彥澤之死,而晉人所以甘心者,可以知其憤疾怨怒於斯人者,非一日也。至於爭已戮之尸,臠其肉,剔其髓而食之,撦裂蹈踐,斯須而盡,何其甚哉!



 此自古未有也。然當是時,舉晉之兵皆在北面,國之存亡,系此三人之勝敗,則其任可謂重矣。蓋天下惡之如彼,晉方任之如此,而終以不悟,豈非所謂「臨亂之君,各賢其臣」者歟?



\end{pinyinscope}