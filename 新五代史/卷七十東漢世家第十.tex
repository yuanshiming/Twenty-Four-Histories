\article{卷七十東漢世家第十}

\begin{pinyinscope}

 劉旻,漢高祖母弟也。初名崇,為人美鬚髯,目重瞳子。少無賴,嗜酒好博,嘗黥為卒。高祖事晉為河東節度使,以旻為都指揮使。高祖即帝位,以為太原尹、北京留守、同中書門下平章事。隱帝時,累加中書令。



 隱帝少,政在大臣,周太祖為樞密使,新討三叛,立大功,而與旻素有隙,旻頗不自安,謂判官鄭珙曰:「主上幼弱,政在權臣,而吾與郭公不葉,時事如何?」



 珙曰:「漢政將亂矣!晉陽兵雄天
 下,而地形險固,十州征賦足以自給。公為宗室,不以此時為計,後必為人所制。」旻曰:「子言,乃吾意也。」乃罷上供征賦,收豪傑,籍丁民以益兵。三年,周太祖起魏,隱帝遇弒,旻乃謀舉兵。



 周太祖之自魏入也,反狀已白,而漢大臣不即推尊之,故未敢即立,乃白漢太后,立旻子贇為漢嗣,遣宰相馮道迎贇于徐州。當是時,人皆知太祖之非實意也,旻獨喜曰:「吾兒為帝矣,何患!」乃罷兵,遣人至京師。周太祖少賤,黥其頸上為飛雀,世謂之郭雀兒。太祖見旻使者,具道所以立贇之意,因自指其頸以示使者曰:「自古豈有雕青天子?幸公無以我為疑。」旻喜,益
 信以為然。太原少尹李驤曰:「郭公舉兵犯順,其勢不能為漢臣,必不為劉氏立後。」因勸旻以兵下太行,控孟津以俟變,庶幾贇得立,贇立而罷兵可也。旻大罵曰:「驤腐儒,欲離間我父子!」命左右牽出斬之。驤臨刑歎曰:「吾為愚人畫計,死誠宜矣!然吾妻病,不可獨存,願與之俱死。」旻聞之,即並戮其妻于市,以其事白漢,以明無他。已而周太祖果代漢,降封贇湘陰公。旻遣牙將李鋋奉書周太祖,求贇歸太原,而贇已死。



 旻慟哭,為李驤立祠,歲時祠之。



 乃以周廣順元年正月戊寅即皇帝位于太原,以子承鈞為太原尹,判官鄭珙、趙華為宰相,都押衙陳
 光裕為宣徽使,遣通事舍人李鋋間行使于契丹。契丹永康王兀欲與旻約為父子之國,旻乃遣宰相鄭珙致書兀欲,稱姪皇帝,以叔父事之而已。兀欲遣燕王述軋、政事令高勳以冊尊旻為大漢神武皇帝,並冊旻妻為皇后。兀欲性豪俊,漢使者至,輒以酒肉困之,珙素有疾,兀欲彊之飲,一夕而以醉卒。然兀欲聞旻自立,頗幸中國多故,乃遣其貴臣述軋、高勳以自愛黃騮、九龍十二稻玉帶報聘。



 已而兀欲為述軋所弒,述律代立。旻遣樞密直學士王得中聘于述律,求兵以攻周。述律遣蕭禹厥率兵五萬助旻。旻出陰地攻晉州,為王峻所敗。是歲
 大寒,旻軍凍餒,亡失過半。明年,又攻府州,為折德扆所敗,德扆因取岢嵐軍。



 周太祖崩,旻聞之喜,遣使乞兵于契丹。契丹遣楊袞將鐵馬萬騎及奚諸部兵五六萬人號稱十萬以助旻。旻以張元徽為先鋒,自將騎兵三萬攻潞州。潞州李筠遣穆令鈞以步騎三千拒元徽于太平驛,元徽擊敗之,益圍潞州。



 是時,世宗新即位,以謂旻幸周有大喪,而天子新立,必不能出兵,宜自將以擊其不意。自宰相馮道等多言不可,世宗意甚銳。顯德元年三月親征,甲午,戰于高平,李重進、白重贊將左,樊愛能、何徽將右,向訓、史彥超居中軍,張永德以禁兵衛蹕。旻
 亦列為三陣,張元徽居東偏,楊袞居西偏,旻居其中。袞望周師謂旻曰:「勍敵也,未可輕動。」旻奮髯曰:「時不可失,無妄言也!」袞怒而去。旻號令東偏先進,王得中叩馬諫曰:「南風甚急,非北軍之利也,宜少待之。」旻怒曰:「老措大,毋妄沮吾軍!」即麾元徽,元徽擊周右軍,兵始交,愛能、徽退走,其騎軍亂,步卒數千棄甲叛降元徽,呼萬歲聲振川谷。世宗大駭,躬督戰士,士皆奮命爭先,而風勢愈盛,旻自麾赤幟收軍,軍不可遏,旻遂敗。日暮,旻收餘兵萬人阻澗而止。



 是時,周之後軍,劉詞將之,在後未至,而世宗銳於速戰,戰已勝,詞軍繼至,因乘勝追擊之,旻又大敗,
 輜重器甲、乘輿服御物皆為周師所獲。旻獨乘契丹黃騮,自雕窠嶺間道馳去,夜失道山谷間,得村民為鄉導,誤趨平陽,得他道以歸,而張元徽戰歿於陣。楊袞怒旻,按兵西偏不戰,故獨全軍而返。旻歸,為黃騮治廄,飾以金銀,食以三品料,號「自在將軍」。



 世宗休軍潞州,大宴將士,斬敗將樊愛能、何徽等七十餘人,軍威大振。進攻太原,遣符彥卿、史彥超北控忻口,以斷契丹援路。太原城方四十里,周師去城三百步,圍之匝,自四月至於六月,攻之不克,而彥卿等為契丹所敗,彥超戰歿,世宗遽班師。



 初,周師圍城也,旻遣王得中送楊袞以歸,因乞援
 兵於契丹,契丹發數萬騎助旻,遣得中先還。至代州,代州將桑珪殺防禦使鄭處謙,以城降周,并送得中于周。



 世宗召問得中虜助兵多少,得中言送袞歸,無所求也,世宗信之。已而契丹敗符彥卿於忻口,得中遂見殺。



 旻自敗於高平,已而被圍,以憂得疾,明年十一月卒,年六十,子承鈞立。



 承鈞,旻次子也。少頗好學,工書。旻卒,承鈞遣人奉表契丹,自稱男。述律答之以詔,呼承鈞為兒,許其嗣位。初,旻常謂張元徽等曰:「吾以高祖之業,贇之冤,義不為郭公屈爾,期與公等勉力以復家國之仇。至於稱帝一方,豈
 獲已也,顧我是何天子,爾亦是何節度使?」故其僭號仍稱乾祐,不改元,不立宗廟,四時之祭,用家人禮。承鈞既立,始赦境內,改乾祐十年曰天會元年,立七廟於顯聖宮。



 契丹遣高勳助承鈞,承鈞遣李存瑰與勳攻上黨,無所得而還。明年,世宗北伐契丹,下三關,契丹使來告急,承鈞將發兵,而世宗班師,乃已。



 宋興,昭義節度使李筠叛命,遣其將劉繼沖、判官孫孚奉表稱臣,執其監軍周光遜、李廷玉送于太原,乞兵為援。承鈞欲謀於契丹,繼沖道筠意,請無用契丹兵。



 承鈞即率其國兵自將出團柏谷,群臣餞之汾水。僕射趙華曰:「李筠舉事輕易,
 陛下不圖成敗,空國興師,臣實憂之。」承鈞至太平驛,封筠隴西郡王。筠見承鈞儀衛不備,非如王者,悔臣之,筠因自陳受周氏恩,不忍背德。而承鈞與周世仇也,聞筠言亦不悅。遣宣徽使盧贊監其軍,筠心益不平,與贊多不葉,承鈞遣宰相衛融和解之。



 已而筠敗死,衛融被執至京師,太祖皇帝問融承鈞所以助筠反狀,融言不遜,太祖命以鐵楇擊其首,流血被面,融呼曰:「臣得死所矣!」太祖顧左右曰:「此忠臣也。」釋之,命以良藥傅其瘡。遣融致書于承鈞,求周光遜等,約亦歸融太原。



 承鈞不報,融遂留京師。承鈞謂趙華曰:「不聽公言,幾至於敗。然失衛
 融、盧贊,吾以為恨爾。」



 承鈞由此益重儒者,以抱腹山人郭無為參議國政。無為,棣州人,方顙鳥喙,好學多聞,善談辯。嘗衣褐為道士,居武當山。周太祖討李守貞于河中,無為詣軍門上謁,詢以當世之務,太祖奇之。或謂太祖曰:「公為漢大臣,握重兵居外,而延縱橫之士,非所以防微慮遠之道也。」由是太祖不納。無為去,隱抱腹山。承鈞內樞密使段常識之,薦其材,承鈞以諫議大夫召之,遂以為相。五年,宿衛殿直行首王隱、劉紹、趙鸞等謀作亂,事覺被誅,其詞連段常,乃罷常樞密為汾州刺史,縊殺之。



 自旻世凡舉事必稟契丹,而承鈞之立多略。契丹
 遣使者責承鈞改元、援李筠、殺段常不以告,承鈞惶恐謝罪。使者至契丹輒見留,承鈞奉之愈謹,而契丹待承鈞益薄。承鈞自李筠敗而失契丹之援,無復南侵之意。地狹產薄,以歲輸契丹,故國用日削,乃拜五臺山僧繼顒為鴻臚卿。繼顒,故燕王劉守光之子,守光之死,以孽子得不殺,削髮為浮圖,後居五臺山,為人多智,善商財利,自旻世頗以賴之。繼顒能講《華嚴經》,四方供施,多積畜以佐國用。五臺當契丹界上,繼顒常得其馬以獻,號「添都馬」,歲率數百匹。又於柏谷置銀冶,募民鑿山取礦,烹銀以輸,劉氏仰以足用,即其冶建寶興軍。繼顒後累
 官至太師、中書令,以老病卒,追封定王。



 太祖皇帝嘗因界上諜者謂承鈞曰:「君家與周氏為世仇,宜其不屈,今我與爾無所間,何為困此一方之人也?若有志於中國,宜下太行以決勝負。」承鈞遣諜者復命曰:「河東土地兵甲,不足以當中國之十一;然承鈞家世非叛者,區區守此,蓋懼漢氏之不血食也。」太祖哀其言,笑謂諜者曰:「為我語承鈞,開爾一路以為生。」故終其世不加兵。



 承鈞立十三年病卒,其養子繼恩立。



 繼恩本姓薛氏,父釗為卒,旻以女妻之,生繼恩。漢高祖以釗婿也,除其軍籍,置之門下。釗無材能,高祖衣食之
 而無所用。妻以旻女常居中,釗罕得見,釗常怏怏,因醉拔佩刀刺之,傷而不死,釗即自裁。旻女後適何氏,生子繼元,而何氏及旻女皆卒。旻以其子承鈞無子,乃以二子命承鈞養為子。承鈞立,以繼恩為太原尹。



 承鈞嘗謂郭無為曰:』繼恩純孝,然非濟世之才,恐不能了我家事。」無為不對。



 承鈞病臥勤政閣,召無為,執手以後事付之。



 承鈞卒,繼恩告哀於契丹而後立。繼恩服縗裳視事,寢處皆居勤政閣,而承鈞故執事百司宿衛者皆在太原府廨。九月,繼恩置酒會諸大臣宗子,飲罷,臥閣中。



 供奉官侯霸榮率十餘人挺刃入閣,閉戶而殺之。郭無為遣
 人以梯登屋入,殺霸榮並其黨。



 初,承鈞之語郭無為也,繼恩怨無為不助己,及立,欲逐之而未果,故霸榮之亂,人皆以謂無為之謀,霸榮死,口滅而無知者。無為迎繼元而立之。



 繼元為人忍。旻子十餘人,皆無可稱者。當繼元時,有鎬、鍇、錡、錫、銑,於繼元為諸父,皆為繼元所殺,獨銑以佯愚獲免。承鈞妻郭氏,繼元兄弟自少母之。



 繼元妻段氏,嘗以小過為郭氏所責,既而以它疾而卒,繼元疑其殺之。及立,遣嬖者范超圖殺郭氏,郭氏方縗服哭承鈞于柩前,超執而縊殺之,於是劉氏之子孫無遺類矣。



 繼元立,改
 元曰廣運。王師北征,繼元閉城拒守,太祖皇帝以詔書招繼元出降,許以平盧軍節度使,郭無為安國軍節度使。無為捧詔色動,而並人及繼元左右皆欲堅守以拒命。無為仰天慟哭,拔佩刀欲自裁,為左右所持。繼元自下執其手,延之上坐,無為曰:「奈何以孤城拒百萬之王師?」蓋欲搖動並人,而並人守意益堅。



 宦者衛德貴察無為有異志,以告繼元,繼元遣人縊殺之。



 初,太祖皇帝命引汾水浸其城,水自城門入,而有積草自城中飄出塞之。是時,王師頓兵甘草地中,會歲暑雨,軍士多疾,乃班師。王師已去,繼元決城下水注之臺駘澤,水已落而城多摧圮。契
 丹使者韓知璠時在太原,嘆曰:「王師之引水浸城也,知其一而不知其二,若先浸而後涸,則並人無類矣!」



 太平興國四年,王師復北征,繼元窮窘,而並人猶欲堅守。其樞密副使馬峰老疾居於家,舁入見繼元,流涕以興亡諭之,繼元乃降。太宗皇帝御城北高臺受降,以繼元為右衛上將軍,封彭城公。其後事具國史。



 旻年世興滅,諸書皆同,自周廣順元年建號,至皇朝太平興國四年滅,凡二十八年,餘具年譜注。



\end{pinyinscope}