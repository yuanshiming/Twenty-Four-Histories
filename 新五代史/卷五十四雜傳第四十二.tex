\article{卷五十四雜傳第四十二}

\begin{pinyinscope}

 《傳》曰:「禮義廉恥,國之四維;四維不張,國乃滅亡。」善乎,管生之能言也!禮義,治人之大法;廉恥,立人之大節。蓋不廉,則無所不取;不恥,則無所不為。人而如此,則禍亂敗亡,亦無所不至,況為大臣而無所不取,無所不為,則天下其有不亂,國家其有不亡者乎!予讀馮道《長樂老敘》,見其自述以為榮,其可謂無廉恥者矣,則天下國家可從而知也。



 予於五代,得全節之士三,死事之臣十有五,
 而怪士之被服儒者以學古自名,而享人之祿、任人之國者多矣,然使忠義之節,獨出於武夫戰卒,豈於儒者果無其人哉?豈非高節之士惡時之亂,薄其世而不肯出歟?抑君天下者不足顧,而莫能致之歟?孔子以謂:「十室之邑,必有忠信。」豈虛言也哉!



 予嘗得五代時小說一篇,載王凝妻李氏事,以一婦人猶能如此,則知世固嘗有其人而不得見也。凝家青、齊之間,為虢州司戶參軍,以疾卒于官。凝家素貧,一子尚幼,李氏攜其子,負其遺骸以歸。東過開封,止旅舍,旅舍主人見其婦人獨攜一子而疑之,不許其宿。李氏顧天已暮,不肯去,主人牽其
 臂而出之。李氏仰天長慟曰:「我為婦人,不能守節,而此手為人執邪?不可以一手並污吾身!」即引斧自斷其臂。路人見者,環聚而嗟之,或為彈指,或為之泣下。開封尹聞之,白其事于朝,官為賜藥封瘡,厚恤李氏,而笞其主人者。嗚呼,士不自愛其身而忍恥以偷生者,聞李氏之風,宜少知愧哉!



 馮道馮道,字可道,瀛州景城人也。事劉守光為參軍,守光敗,去事宦者張承業。



 承業監河東軍,以為巡官,以其文學薦之晉王,為河東節度掌書記。莊宗即位,拜戶部侍郎,充翰林學士。道為人,能自刻苦為儉約。當晉與梁夾河
 而軍,道居軍中,為一茅庵,不設床席,臥一束芻而已。所得俸祿,與僕廝同器飲食,意恬如也。諸將有掠得人之美女者以遺道,道不能卻,置之別室,訪其主而還之。其解學士居父喪於景城,遇歲饑,悉出所有以賙鄉里,而退耕于野,躬自負薪。有荒其田不耕者與力不能耕者,道夜往,潛為之耕。其人後來愧謝,道殊不以為德。服除,復召為翰林學士。行至汴州,遇趙在禮作亂,明宗自魏擁兵還,犯京師。孔循勸道少留以待,道曰:「吾奉詔赴闕,豈可自留!」乃疾趨至京師。莊宗遇弒,明宗即位,雅知道所為,問安重誨曰:「先帝時馮道何在?」重誨曰:「為學士也。」
 明宗曰:「吾素知之,此真吾宰相也。」拜道端明殿學士,遷兵部侍郎。歲餘,拜中書侍郎、同中書門下平章事。



 天成、長興之間,歲屢豐熟,中國無事。道嘗戒明宗曰:「臣為河東掌書記時,奉使中山,過井陘之險,懼馬蹶失,不敢怠於銜轡;及至平地,謂無足慮,遽跌而傷。凡蹈危者慮深而獲全,居安者患生於所忽,此人情之常也。」明宗問曰:「天下雖豐,百姓濟否?」道曰:「穀貴餓農,穀賤傷農。」因誦文士聶夷中《田家詩》,其言近而易曉。明宗顧左右錄其詩,常以自誦。水運軍將於臨河縣得一玉杯,有文曰「傳國寶萬歲杯」,明宗甚愛之,以示道,道曰:「此前世有形之寶
 爾,王者固有無形之寶也。」明宗問之,道曰:「仁義者,帝王之寶也。故曰:『大寶曰位,何以守位曰仁。』」明宗武君,不曉其言,道已去,召侍臣講說其義,嘉納之。



 道相明宗十餘年,明宗崩,相愍帝。潞王反於鳳翔,愍帝出奔衛州,道率百官迎潞王入,是為廢帝,遂相之。廢帝即位,愍帝猶在衛州,後三日,愍帝始遇弒崩。



 已而廢帝出道為同州節度使,踰年,拜司空。晉滅唐,道又事晉,晉高祖拜道守司空、同中書門下平章事,加司徒,兼侍中,封魯國公。高祖崩,道相出帝,加太尉,封燕國公,罷為匡國軍節度使,徙鎮威勝。契丹滅晉,道又事契丹,朝耶律德光於京
 師。德光責道事晉無狀,道不能對。又問曰:「何以來朝?」對曰:「無城無兵,安敢不來。」德光誚之曰:「爾是何等老子?」對曰:「無才無德癡頑老子。」德光喜,以道為太傅。德光北歸,從至常山。漢高祖立,乃歸漢,以太師奉朝請。周滅漢,道又事周,周太祖拜道太師,兼中書令。道少能矯行以取稱於世,及為大臣,尤務持重以鎮物,事四姓十君,益以舊德自處。然當世之士無賢愚皆仰道為元老,而喜為之稱譽。



 耶律德光嘗問道曰:「天下百姓如何救得?」道為俳語以對曰:「此時佛出救不得,惟皇帝救得。」人皆以謂契丹不夷滅中國之人者,賴道一言之善也。周兵反,犯
 京師,隱帝已崩,太祖謂漢大臣必行推戴,及見道,道殊無意。太祖素拜道,因不得已拜之,道受之如平時,太祖意少沮,知漢未可代,遂陽立湘陰公贇為漢嗣,遣道迎贇于徐州。贇未至,太祖將兵北至澶州,擁兵而反,遂代漢。議者謂道能沮太祖之謀而緩之,終不以晉、漢之亡責道也。然道視喪君亡國亦未嘗以屑意。



 當是時,天下大亂,戎夷交侵,生民之命,急於倒懸,道方自號「長樂老」,著書數百言,陳己更事四姓及契丹所得階勳官爵以為榮。自謂:「孝於家,忠於國,為子、為弟、為人臣、為師長、為夫、為父,有子、有孫。時開一卷,時飲一杯,食味、別聲、被
 色,老安於當代,老而自樂,何樂如之?」蓋其自述如此。



 道前事九君,未嘗諫諍。世宗初即位,劉旻攻上黨,世宗曰:「劉旻少我,謂我新立而國有大喪,必不能出兵以戰。且善用兵者出其不意,吾當自將擊之。」道乃切諫,以為不可。世宗曰:「吾見唐太宗平定天下,敵無大小皆親征。」道曰:「陛下未可比唐太宗。」世宗曰:「劉旻烏合之眾,若遇我師,如山壓卵。」道曰:「陛下作得山定否?」世宗怒,起去,卒自將擊旻,果敗旻于高平。世宗取淮南,定三關,威武之振自高平始。其擊旻也,鄙道不以從行,以為太祖山陵使。葬畢而道卒,年七十三,謚曰文懿,追封瀛王。道既卒,時
 人皆共稱歎,以謂與孔子同壽,其喜為之稱譽蓋如此。道有子吉。



 李琪兄掞李琪,字台秀,河西敦煌人也。其兄珽,唐末舉進士及第,為監察御史。丁內艱,貧無以葬,乞食而後葬。珽飢臥廬中,聞者哀憐之。服除,還拜御史。荊南成汭辟掌書記。吳兵圍杜洪,梁太祖遣汭與馬殷等救洪。汭以大舟載兵數萬,珽為汭謀曰:「今一舟容甲士千人,糗糧倍之,緩急不可動,若為敵人縻之,則武陵、武安必為公之後患。不若以勁兵屯巴陵,壁不與戰,吳兵糧盡,則圍解矣。」汭不聽,汭果敗,溺死。趙匡凝鎮襄陽,又辟掌書記。太祖破匡凝,
 得珽,喜曰:「此真書記也。」太祖即位,除考功員外郎、知制誥。珽度太祖不欲先用故吏,固辭不拜,出知曹州。曹州素劇難理,前刺史十餘輩,皆坐廢,珽至,以治聞。遷兵部郎中、崇政院直學士。許州馮行襲病,行襲有牙兵二千,皆故蔡卒,太祖懼為變。行襲為人嚴酷,從事魏峻切諫,行襲怒,誣以贓下獄,欲誅之。乃遣珽代行襲為留後。珽至許州,止傳舍,慰其將吏,行襲病甚,欲使人代受詔,珽曰:「東首加朝服,禮也。」乃即臥內見行襲,道太祖語,行襲感泣,解印以授珽。珽乃理峻冤,立出之,還報太祖,太祖喜曰:「珽果辦吾事。」會歲饑,盜劫汴、宋間,曹州尤甚,太
 祖復遣珽治之。珽至索賊,得大校張彥珂、珽甥李郊等,及牙兵百餘人,悉誅之。召拜左諫議大夫。太祖幸河北,至內黃,顧珽曰:「何謂內黃?」珽曰:「河南有外黃、下黃,故此名內黃。」太祖曰:「外黃、下黃何在?」珽曰:「秦有外黃都尉,今在雍丘;下黃為北齊所廢,今在陳留。」太祖平生不愛儒者,聞珽語大喜。友珪立,除右散騎常侍,侍講。袁象先討賊,珽為亂兵所殺。



 琪少舉進士、博學宏辭,累遷殿中侍御史,與其兄珽皆以文章知名。唐亡,事梁太祖為翰林學士。梁兵征伐四方,所下詔書,皆琪所為,下筆輒得太祖意。末帝時,為御史中丞、尚書左丞,拜同中書門下平
 章事,與蕭頃同為宰相。頃性畏慎周密,琪倜儻負氣,不拘小節,二人多所異同。琪內結趙巖、張漢傑等為助,以故頃言多沮。頃嘗掎摭其過。琪所私吏當得試官,琪改試為守,為頃所發,末帝大怒,欲竄逐之,而巖等救解,乃得罷為太子少保。



 唐莊宗滅梁,得琪,欲以為相,而梁之舊臣多嫉忌之,乃以為太常卿。遷吏部尚書。同光三年秋,天下大水,京師乏食尤甚,莊宗以硃書御札詔百僚上封事。琪上書數千言,其說漫然無足取,而莊宗獨稱重之,遂以為國計使。方欲以為相,而莊宗崩。明宗入洛陽,群臣勸進,有司具儀,用柩前即位故事。霍彥威、孔
 循等請改國號,絕土德。明宗武君,不曉其說,問何謂改號,對曰:「莊宗受唐錫姓為宗屬,繼昭宗以立,而號國曰唐。今唐天命已絕,宜改號以自新。」明宗疑之,下其事群臣,群臣依違不決。琪議曰:「殿下宗室之賢,立功三世,今興兵向闕,以赴難為名,而欲更易統號,使先帝便為路人,則煢然梓宮,何所依往!」明宗以為然,乃發喪成服,而後即位。以琪為御史中丞。



 自唐末喪亂,朝廷之禮壞,天子未嘗視朝,而入閣之制亦廢。常參之官日至正衙者,傳聞不坐即退,獨大臣奏事,日一見便殿,而侍從內諸司,日再朝而已。明宗初即位,乃詔群臣,五日一隨宰
 相入見內殿,謂之起居。琪以謂非唐故事,請罷五日起居,而復朔望入閣。明宗曰:「五日起居,吾思所以數見群臣也,不可罷。



 而朔望入閣可復。」然唐故事,天子日御殿見群臣,曰常參;朔望薦食諸陵寢,有思慕之心,不能臨前殿,則御便殿見群臣,曰入閣。宣政,前殿也,謂之衙,衙有仗。紫宸,便殿也,謂之閣。其不御前殿而御紫宸也,乃自正衙喚仗,由閣門而入,百官俟朝于衙者,因隨以入見,故謂之入閣。然衙,朝也,其禮尊;閣,宴見也,其事殺。自乾符已後,因亂禮闕,天子不能日見群臣而見朔望,故正衙常日廢仗,而朔望入閣有仗,其後習見,遂以入閣
 為重。至出御前殿,猶謂之入閣,其後亦廢,至是而復。然有司不能講正其事。凡群臣五日一入見中興殿,便殿也,此入閣之遺制,而謂之起居,朔望一出御文明殿,前殿也,反謂之入閣,琪皆不能正也。琪又建言:「入閤有待制、次對官論事,而內殿起居,一見而退,欲有言者,無由自陳,非所以數見群臣之意也。」明宗乃詔起居日有言事者,許出行自陳。又詔百官以次轉對。



 是時,樞密使安重誨專權用事,重誨前驅過御史臺門,殿直馬延誤沖之,重誨即臺門斬延而後奏。琪為中丞,畏重誨不敢彈糾,又懼諫官論列,乃託宰相任圜先白重誨而後糾,然
 猶依違不敢正言其事。豆盧革等罷相,住圜議欲以琪為相,而孔循、鄭玨沮之,乃止。遷尚書右僕射。琪以狀申中書,言《開元禮》「僕射上事日,中書、門下率百官送上。」中書下太常,禮院言無送上之文,而琪已落新授,復舉上儀,皆不可。



 明宗討王都,已破定州,自汴還洛,琪當率百官至上東門,而請至偃師奉迎。



 其奏章言「敗契丹之兇黨,破真定之逆城」,坐誤以定州為真定,罰俸一月。霍彥威卒,詔琪撰神道碑文。彥威故梁將,而琪故梁相也,敘彥威在梁事不曰偽,為馮道所駁。



 琪為人重然諾,喜稱人善。少以文章知名,亦以此自負。既貴,乃刻牙版為金
 字曰「前鄉貢進士李琪」,常置之坐側。為人少持重,不知進退,故數為當時所沮。



 以太子太傅致仕,卒,年六十。



 鄭玨鄭玨,唐宰相綮之諸孫也。其父徽,為河南尹張全義判官。玨少依全義居河南,舉進士數不中,全義以玨屬有司,乃得及第。昭宗時,為監察御史。梁太祖即位,拜左補闕。梁諸大臣以全義故數薦之,累拜中書舍人、翰林學士奉旨。末帝時,拜中書侍郎、同中書門下平章事。



 唐莊宗自鄆州入汴,末帝聞唐兵且至,惶恐不知所為,與李振、敬翔等相持慟哭,因召玨問計安出,玨曰:「臣有一策,不知陛下能行否?」末帝問其策如何,玨曰:「願得陛下傳
 國寶馳入唐軍,以緩其行,而待救兵之至。」帝曰:「事急矣,寶固不足惜,顧卿之行,能了事否?」玨俯首徐思曰:「但恐不易了。」於是左右皆大笑。莊宗入汴,玨率百官迎謁道左。貶萊州司戶參軍,量移曹州司馬。張全義為言於郭崇韜,復召為太子賓客。明宗即位,欲用任圜為相,而安重誨以圜新進,不欲獨相之,以問樞密使孔循。循嘗事梁,與玨善,因言玨故梁相,性謹慎而長者,乃拜玨平章事。



 明宗幸汴州,六軍家屬自洛遷汴,而明宗又欲幸鄴都,軍士愁怨,大臣頗以為言。明宗不省,上下洶洶,轉相動搖,獨玨稱贊,以為當行。趙鳳極言於安重誨,重誨
 驚懼,入見明宗切諫,乃詔罷其行。而玨又稱贊之,以為宜罷。玨在相位既碌碌無所為,又病聾,孔循罷樞密使,玨不自安,亟以疾求去職。明宗數留之,玨章四上,乃拜左僕射致仕,賜鄭州莊一區。卒,贈司空。



 李愚李愚,字子晦,渤海無棣人也。愚為人謹重寡言,好學,為古文。滄州節度使盧彥威以愚為安陵主簿,丁母憂解去。後遊關中,劉季述幽昭宗於東內,愚以書說韓建,使圖興復,其言甚壯。建不能用,乃去之洛陽。舉進士、宏詞,為河南府參軍。白馬之禍,愚復去之山東,與李延光相善,延光以經術事梁末帝為侍講,數稱薦愚,愚由此得
 召。久之,拜左拾遺、崇政院直學士。



 衡王友諒,末帝兄也,梁大臣李振等皆拜之,獨愚長揖,末帝以責愚曰:「衡王朕拜之,卿獨揖,可乎?」愚曰:「陛下以家人禮見之,則拜宜也。臣於王無所私,豈宜妄有所屈?」坐言事忤旨,罷為鄧州觀察判官。



 唐莊宗滅梁,愚朝京師,唐諸公卿素聞愚學古,重之,拜主客郎中、翰林學士。



 魏王繼岌伐蜀,辟愚都統判官。蜀道阻險,議者以謂宜緩師待變而進,招討使郭崇韜以決於愚,愚曰:「王衍荒怠,亂國之政,其人厭之。乘其倉卒,擊其無備,其利在速,不可緩也。」崇韜以為然,而所至迎降,遂以滅蜀。初,軍行至寶雞,招討判官陳
 乂稱疾請留,愚厲聲曰:「陳乂見利則進,知難則止。今大軍涉險,人心易搖,正可斬之以徇。」由是軍中無敢言留者。



 明宗即位,累遷兵部侍郎承旨。明宗祀天南郊,愚為宰相馮道、趙鳳草加恩制,道鄙其辭,罷為太常卿。任圜罷相,乃拜愚中書侍郎、同平章事。愚為相,不治第宅,借延賓館以居。愚有疾,明宗遣宦官視之,見其敗毯敝席,四壁蕭然,明宗嗟嘆,命以供帳物賜之。



 潞王反,犯京師,愍帝夜出奔。明日,愚與馮道至端門,聞帝已出,而朱弘昭、馮贇皆已死,愚欲至中書候太后進止,道曰:「潞王已處處張榜招安,今即至矣,何可俟太后旨也?」乃相與
 出迎。廢帝入立,罷道出鎮同州,以劉昫為相。昫性褊急,而愚素剛介,動輒違戾。昫與馮道姻家,愚數以此誚昫,兩人遂相誼詬,乃俱罷。愚守左僕射。



 是時,兵革方興,天下多事,而愚為相,欲依古以創理,乃請頒《唐六典》示百司,使各舉其職,州縣貢士,作鄉飲酒禮,時以其迂闊不用。愍帝即位,有意於治,數召學士,問以時事,而以愚為迂,未嘗有所問。廢帝亦謂愚等無所事,常目宰相曰:「此粥飯僧爾!」以謂飽食終日,而無所用心也。清泰二年,以疾卒。



 盧導盧導,字熙化,范陽人也。唐末舉進士,為監察御史。唐亡
 事梁,累遷左司郎中、侍御史知雜事,以病免。唐明宗時,召拜右諫議大夫,遷中書舍人。潞王從珂自鳳翔以兵犯京師,愍帝出奔于衛州。宰相馮道、李愚集百官於天宮寺,將出迎潞王于郊,京師大恐,都人藏竄,百官久而不集,惟導與舍人張昭先至。馮道請導草箋勸進,導曰:「潞王入朝,郊迎可也,若勸進之事,豈可輕議哉!」道曰:「勸進其可已乎?」導曰:「今天子蒙塵于外,遽以大位勸人,若潞王守節不回,以忠義見責,其將何辭以對?且上與潞王,皆太后子也,不如率百官詣宮門,取太后進止。」語未終,有報曰:「潞王至矣。」京城巡檢使安從進催百官班
 迎,百官紛然而去。潞王止于正陽門外,道又促導草箋,導對如初。李愚曰:「吾輩罪人,盧舍人言是也。」導終不草箋。導後事晉為吏部侍郎。天福六年卒,年七十六。



 司空頲司空頲,貝州清陽人也。唐僖宗時,舉進士不中,後去為羅紹威掌書記。紹威卒,入梁為太府少卿。楊師厚鎮天雄,頲解官往依之。師厚卒,賀德倫代之。張彥之亂,命判官王正言草奏詆斥梁君臣,正言素不能文辭,又為兵刃所迫,流汗浹背,不能下筆。彥怒,推正言下榻,詬曰:「鈍漢辱我!」顧書吏問誰可草奏者,吏即言頲羅王時書記,乃馳騎召之。頲為亂兵劫其衣,以敝服蔽形而至,見彥
 長揖,神氣自若,揮筆成文,而言甚淺鄙,彥以其易曉,甚喜,即給以衣服僕馬,遂以為德倫判官。德倫以魏博降晉,晉王兼領天雄,仍以頲為判官。梁、晉相距河上,常以頲權軍府事。頲為郭崇韜所惡,崇韜數言其受賂。都虞候張裕多過失,頲屢以法繩之。頲有侄在梁,遣家奴召之,裕擒其家奴,以謂通書於梁。莊宗族殺之。



\end{pinyinscope}