\article{卷六十三前蜀世家第三}

\begin{pinyinscope}

 王
 建,字光圖,許州舞陽人也。隆眉廣顙,狀貌偉然。少無賴,以屠牛、盜驢、販私鹽為事,里人謂之「賊王八」。後為忠武軍卒,稍遷隊將。



 黃巢陷長安,僖宗在蜀,忠武軍將鹿晏弘以兵八千屬楊復光討賊,巢敗走,復光以其兵為八都,都將千人,建與晏弘皆為一都頭。復光死,晏弘率八都西迎僖宗于蜀,所過剽略。行至興元,逐節度使牛叢,自稱留後。僖宗即以晏弘為節度使,晏弘以建
 等八都頭皆領屬州刺史。已而晏弘擁眾東歸,陷陳、許,建與晉暉、韓建、張造、李師泰等各率一都,西奔于蜀。僖宗得之大喜,號「隨駕五都」,以屬十軍觀軍容使田令孜,令孜以建等為養子。僖宗還長安,使建與晉暉等將神策軍宿衛。



 光啟元年,河王重榮與令孜爭鹽池,重榮召晉兵犯京師,僖宗幸鳳翔。二年三月,移幸興元,以建為清道使,負玉璽以從。行至當塗驛,李昌符焚棧道,棧道幾斷,建控僖宗馬,冒煙焰中過,宿阪下,僖宗枕建膝寢,既覺,涕泣,解御衣賜之。



 僖宗已至興元,令孜以謂天子播越,由己致之,懼且得罪,西川節度使陳敬瑄,令
 孜同母弟也,令孜因求為西川監軍,楊復恭代為軍容使。復恭出建為壁州刺史,建乃招集亡命及溪洞夷落,有眾八千,以攻閬州,執其刺史楊行遷。又攻利州,利州刺史王珙棄城走。敬瑄患之,以問令孜,令孜曰:「王八吾兒也,以一介召之,可置麾下。」乃使人招建。



 東川顧彥朗與建有舊,建聞令孜召己,大喜,因至梓州,謂彥朗曰:「十軍阿父召我,我欲至成都見陳公,以求一鎮。」即以其家屬託彥朗,選精兵二千,馳之成都。行至鹿頭關,敬瑄悔召建,使人止之。建大怒,擊破鹿頭關,取漢州。彥朗聞之,出兵助建,軍于學射。敬瑄遣將句惟立逆建,建擊敗之,
 遂攻彭州。敬瑄遣眉州刺史山行章將兵五萬屯新繁,建又擊敗之,虜獲萬餘人,橫尸四十里。敬瑄發兵七萬益行章,與建相持濛陽、新都百餘日。昭宗遣左諫議大夫李洵為兩川宣諭和協使,詔彥朗等罷兵。彥朗請以大臣鎮蜀,因為建求旌節。文德元年六月,以宰相韋昭度為西川節度使。分邛、蜀、黎、雅為永平軍,拜建節度使。



 敬瑄不受代,昭宗即命昭度將彥朗等兵討之。昭宗以建為招討牙內都指揮使。



 久之,不克,建謂昭度曰:「公以數萬之眾,困兩川之人,而師久無功,奈何?且唐室多故,東方諸鎮,兵接都畿,公當歸相天子,靜中原以固根本,
 此蠻夷之國,不足以留公。」昭度遲疑未決,建遣軍士擒昭度親吏於軍門,臠而食之,建入白曰:「軍士飢,須此為食爾!」昭度大恐,即留符節與建而東。昭度已去,建即以兵扼劍門,兩川由是阻絕。



 山行章屯廣都,建擊敗之,行章走眉州,以州降建。建引兵攻成都,而資、簡、戎、茂、嘉、邛諸州皆殺刺史降建。建攻成都甚急,田令孜登城呼建曰:「老夫與公相厚,何嫌而至此!」建曰:「軍容父子之恩,心何可忘!然兵討不受代者,天子命也。」令孜夜入建軍,以節度觀察牌印授建。明日,敬瑄開門迎建。建將入城,以張勍為都虞候,戒其軍士曰:「吾以張勍為虞候矣,汝等
 無犯其令,幸勍執而見我,我尚活汝,使其殺而後白,吾亦不能詰也。」建入城,軍士剽略,勍殺百人而止。後建遷敬瑄于雅州,使人殺之;復以令孜為監軍,既而亦殺之。大順二年十月,唐以建為檢校司徒、成都尹、劍南西川節度副大使知節度事、管內觀察處置雲南八國招撫等使。



 東川顧彥朗卒,其弟彥暉立。唐遣宦者宗道弼賜彥暉東川旌節,綿州刺史常厚執道弼以攻梓州,建遣李簡、王宗滌等討厚。自彥朗死,建欲圖并東川而未有以發,及李簡等討厚,戒曰:「兵已破厚,彥暉必出犒師,即與俱來,無煩吾再舉也。」



 簡等擊厚,敗之鐘陽,厚走還綿
 州,以唐旌節還道弼而出之。彥暉已得節,辭疾不出犒軍。乾寧二年,建遣王宗滌攻之。十二月,宗滌敗彥暉於楸林,斬其將羅璋,遂圍梓州。三年五月,昭宗遣宦者袁易簡詔建罷兵,建收兵還成都。黔南節度使王肇以其地降於建。



 四年,宗滌復攻東川,別遣王宗侃、宗阮等出峽,取渝、瀘州。五月,建自將攻東川,昭宗遣諫議大夫李洵、判官韋莊宣諭兩川,詔建罷兵。建不奉詔,乃責授建南州刺史,以郯王為鳳翔節度使,李茂貞代建為西川節度使。茂貞拒命,乃復建官爵。冬十月,建攻破梓州,彥暉自殺。彥暉將顧彥瑤顧城已危,謂諸將吏曰:「事公
 當生死以之!」指其所佩賓鐵劍曰:「事急而有叛者,當齒此劍!」及城將破,彥瑤與彥暉召集將吏飲酒,遂與之俱死。建以王宗滌為東川留後,唐即以宗滌為節度使,於是并有兩川之地。



 是時,鳳翔李茂貞兼據梁、洋、秦、隴,數以兵侵建。天復元年,梁太祖兵誅宦者,宦者韓全誨等劫天子幸鳳翔,梁兵圍之,茂貞閉城拒守經年,力窘,求與梁和。建間遣人聘茂貞,許以出兵為援,勸其堅壁勿和。遣王宗滌將兵五萬,聲言迎駕,以攻興元,執其節度使李繼業,而武定節度使拓拔思敬遂以其地降于建,於是並有山南西道。是時,荊南成汭死,襄州趙匡凝遣其弟
 匡明襲據之,建乘其間,攻下夔、施、忠、萬四州。三年八月,唐封建蜀王。四年,唐遷都洛陽,改元天祐,建與唐隔絕而不知,故仍稱天復。六年,又取歸州,於是并有三峽。



 七年,梁滅唐,遣使者諭建,建拒而不納。建因馳檄四方,會兵討梁,四方知其非誠實,皆不應。



 是歲正月,巨人見青城山。六月,鳳凰見萬歲縣,黃龍見嘉陽江,而諸州皆言甘露、白鹿、白雀、龜、龍之瑞。秋九月己亥,建乃即皇帝位。封其諸子為王,以王宗佶為中書令,韋莊為左散騎常侍判中書門下事,唐襲為樞密使,鄭騫為御史中丞,張格、王鍇皆為翰林學士,周博雅為成都尹。蜀恃險而
 富,當唐之末,士人多欲依建以避亂。建雖起盜賊,而為人多智詐,善待士,故其僭號,所用皆唐名臣世族;莊,見素之孫;格,浚之子也。建謂左右曰:「吾為神策軍將時,宿衛禁中,見天子夜召學士,出入無間,恩禮親厚如寮友,非將相可比也。」故建待格等恩禮尤異,其餘宋玭等百餘人,並見信用。



 武成元年正月,祀天南郊,大赦,改元,以王宗佶為太師。宗佶本姓甘氏,建為忠武軍卒時掠得之,養以為子,後以軍功累遷武信軍節度使。後建所生子元懿等稍長,宗佶以養子心不自安,與鄭騫等謀,求為大司馬,總六軍,開元帥府,凡軍事便宜行而後聞。
 建以宗佶創業功多,優容之。唐襲本以舞僮見幸於建,宗佶尤易之,後為樞密使,猶名呼襲,襲雖內恨,而外奉宗佶愈謹。建聞之,怒曰:「宗佶名呼我樞密使,是將反也。」宗佶求大司馬,章三上,建以問襲,襲因激怒建曰:「宗佶功臣,其威望可以服人心,陛下宜即與之。」建心益疑。宗佶入奏事,自請不已,建叱衛士撲殺之,并賜騫死。六月,以遂王宗懿為皇太子。建加尊號英武睿聖皇帝。七月,騶虞見武定。



 二年,頒《永昌歷》。廣都嘉禾合穗。



 三年八月,有龍五十見洵陽水中。十月,麟見壁州。十二月,大赦,改明年為永平元年。岐王李茂貞自為梁所圍,而山南入
 於蜀,地狹勢孤,遂與建和,以其子娶建女,因求山南故地。建怒,不與,以王宗侃為北路都統,宗佑、宗賀、唐襲為三面招討使以攻岐。戰于青泥,宗侃敗績,退保西縣,為茂貞兵所圍。建自將擊之,岐兵敗,解去,建至興元而還。加尊號曰英武睿聖光孝皇帝。



 二年,又加號曰英武睿聖神功文德光孝皇帝。初,田令孜之為監軍也,盜唐傳國璽入于蜀而埋之,二月,尚食使歐陽柔治令孜故第,穿地而得之,以獻。五月,梁遣光祿卿盧玭來聘,推建為兄,其印文曰「大梁入蜀之印」。宰相張格曰:「唐故事,奉使四夷,其印曰『大唐入某國之印』,今梁已兄事陛下,奈
 何卑我如夷狄?」



 建怒,欲殺梁使者,格曰:「此梁有司之過爾,不可以絕兩國之懽。」已而梁太祖崩,建遣將作監李紘弔之,遂刻其印文曰「大蜀入梁之印」。劍州木連理。六月,麟見文州。十二月,黃龍見富義江。



 三年正月,麟見永泰。五月,騶虞見壁山,有二鹿隨之。秋七月,皇太子元膺殺太子少保唐襲。元膺,建次子也,初名宗懿,後更名宗坦,建得銅牌子于什仿,有文二十餘字,建以為符讖,因取之以名諸子,故又更曰元膺。元膺為人犬叚喙齲齒,多材藝,能射錢中孔,嘗自抱畫球擲馬上,馳而射之,無不中。年十七,為皇太子,判六軍,創天武神機營,開永和府,置
 官屬。建以元膺年少任重,以記事戒之,令「一切學朕所為,則可以保國」。又命道士廣成先生杜光庭為之師。唐襲,建之嬖也,元膺易之,屢謔于朝,建懼其交惡,乃罷襲樞密使,出為興元節度使。



 已而襲罷歸,元膺廷疏其過失,建益不悅。是月七夕,元膺召諸王大臣置酒,而集王宗翰、樞密使潘峭、翰林學士毛文錫不至,元膺怒曰:「集王不來,峭與文錫教之耳!」明日,元膺白建峭及文錫離間語。建怒,將罪之。元膺出而襲入,建以問之,襲曰:「太子謀作亂,欲召諸將、諸王以兵錮之,然後舉事爾!」建疑之,襲請召營兵入衛。元膺初不為備,聞襲召兵,以為誅
 己,乃與伶人安悉香、軍將喻全殊率天武兵自衛,遣人擒峭及文錫而笞之,幽於其家;召大將徐瑤、常謙率兵出拒襲,與襲戰神武門,襲中流矢,墜馬死。建遣王宗賀以兵討之,元膺兵敗皆潰去,元膺匿躍龍池檻中。明日,出而丐食,蜀人識之,以告,建遣宗翰招諭之,宗翰未至,為衛兵所殺。建乃立其幼子鄭王宗衍為太子。白龍見邛州江。



 四年,荊南高季昌侵蜀巫山,遣嘉王宗壽敗之於瞿唐。八月,殺黔南節度使王宗訓。冬,南蠻攻掠界上,建遣夔王宗範擊敗之於大渡河。麟見昌州。



 五年,起壽昌殿於龍興宮,畫建像於壁;又起扶天閣,畫諸功臣像。
 十一月,大火,焚其宮室。遣王宗儔等攻岐,取其秦、鳳、階、成四州,至大散關。梁叛將劉知俊在岐,於是特以其族來。



 通正元年,遣王宗綰等率兵十二萬出大散關攻岐,取隴州。八月,起文思殿,以清資五品正員官購群書以實之,以內樞密使毛文錫為文思殿大學士。黃龍見大昌池。十月,大赦。改明年元曰天漢,國號漢。



 天漢元年,殺劉知俊。十二月,大赦,改明年元曰光天,復國號蜀。



 光天元年六月,建卒,年七十二。建晚年多內寵,賢妃徐氏與妹淑妃,皆以色進,專房用事,交結宦者唐文扆等干與外政。建年老昏耄,文扆判六軍,事無大小,
 皆決文扆。及建疾,以兵入宿衛,謀盡去建故將。故將聞建疾,皆不得入見,久之,宗弼等排闥入,言文扆欲為變,乃殺之。建因以老將大臣多許昌故人,必不為太子用,思擇人未得而疾亟,乃以宦者宋光嗣為樞密使判六軍而建卒。太子立,去「宗」



 名衍。



 衍字化源。建十一子,曰衛王宗仁,簡王元膺,趙王宗紀,豳王宗輅,韓王宗智,莒王宗特,信王宗傑,魯王宗鼎,興王宗澤,薛王宗平。而鄭王宗衍最幼,其母徐賢妃也,以母寵得立為皇太子,開崇賢府,置官屬,後更曰天策府。衍為人方頤大口,垂手過膝,顧目見耳,頗知學問,能為
 浮艷之辭。元膺死,建以豳王宗輅貌類己,而信王宗傑於諸子最材賢,欲於兩人擇立之。而徐妃專寵,建老昏耄,妃與宦者唐文扆教相者上言衍相最貴,又諷宰相張格贊成之,衍由是得為太子。



 建卒,衍立,謚建曰神武聖文孝德明惠皇帝,廟號高祖,陵曰永陵。建正室周氏號昭聖皇后,後建數日而卒,衍因尊其母徐氏為皇太后,后妹淑妃為皇太妃。太后、太妃以教令賣官,自刺史以下,每一官闕,必數人並爭,而入錢多者得之;通都大邑起邸店,以奪民利。



 衍年少荒淫,委其政於宦者宋光嗣、光葆、景潤澄、王承休、歐陽晃、田魯儔等;以韓昭、潘在迎、
 顧在珣、嚴旭等為狎客;起宣華苑,有重光、太清、延昌、會真之殿,清和、迎仙之宮,降真、蓬萊、丹霞之亭,飛鸞之閣,瑞獸之門;又作怡神亭,與諸狎客、婦人日夜酣飲其中。嘗以九日宴宣華苑,嘉王宗壽以社稷為言,言發泣涕。韓昭等曰:「嘉王酒悲爾!」諸狎客共以慢言謔嘲之,坐上喧然。衍不能省也。



 蜀人富而喜遨,當王氏晚年,俗競為小帽,僅覆其頂,俯首即墮,謂之「危腦帽」。衍以為不祥,禁之。而衍好戴大帽,每微服出遊民間,民間以大帽識之,因令國中皆戴大帽。又好裹尖巾,其狀如錐。而後宮皆戴金蓮花冠,衣道士服,酒酣免冠,其髻髽然,更施朱
 粉,號「醉妝」,國中之人皆效之。嘗與太后、太妃游青城山,宮人衣服,皆畫雲霞,飄然望之若仙。衍自作《甘州曲》,述其仙狀,上下山谷,衍常自歌,而使宮人皆和之。衍立之明年,改元乾德。



 乾德元年正月,祀天南郊,大赦,加尊號為聖德明孝皇帝。



 二年冬,北巡,至于西縣,旌旗戈甲,連亙百餘里。其還也,自閬州浮江而上,龍舟畫舸,昭耀江水,所在供億,人不堪命。



 三年正月,還成都。



 五年,起上清宮,塑王子晉像,尊以為聖祖至道玉宸皇帝,又塑建及衍像,侍立於其左右;又於正殿塑玄元皇帝及唐諸帝,備法駕而朝之。



 六年,以王承休為天雄節度使。天雄
 軍,秦州也。承休以宦者得幸,為宣徽使,承休妻嚴氏,有絕色,衍通之。是時,唐莊宗滅梁,蜀人皆懼。莊宗遣李嚴聘蜀,衍與俱朝上清,而蜀都士庶,簾帷珠翠,夾道不絕。嚴見其人物富盛,而衍驕淫,歸乃獻策伐蜀。明年,唐魏王繼岌、郭崇韜伐蜀。是歲,衍改元曰咸康。衍自立,歲常獵于子來山。是歲,又幸彭州陽平化、漢州三學山。以王承休妻嚴氏故,十月,幸秦州,群臣切諫,衍不聽。行至梓潼,大風發屋拔木,太史曰:「此貪狼風也,當有敗軍殺將者。」衍不省。衍至綿谷而唐師入其境,衍懼,遽還。唐師所至,州縣皆迎降。衍留王宗弼守綿谷,遣王宗勛、宗儼、宗
 昱率兵以拒唐師。宗勛等至三泉,望風退走。衍詔宗弼誅宗勳等,宗弼反與宗勳等合謀,送款於唐師。衍自綿谷還成都,百官及後宮迎謁七里亭,衍雜宮人作回鶻隊以入。明日,御文明殿,與其群臣相對涕泣。而宗弼亦自綿谷馳歸,登太玄門,收成都尹韓昭、宦者宋光嗣、景潤澄、歐陽晃等殺之,函首送於繼岌。衍即上表乞降,宗弼遷衍于天啟宮。魏王繼岌至成都,衍君臣面縛輿櫬,出降于七里亭。



 莊宗召衍入洛,賜衍詔曰:「固當列土而封,必不薄人於險,三辰在上,一言不欺!」衍捧詔忻然就道,率其宗族及偽宰相王鍇、張格、瘐傳素、許寂、翰林
 學士李旻等,及諸將佐家族數千人以東。同光四年四月,行至秦川驛,莊宗用伶人景進計,遣宦者向延嗣誅其族。衍母徐氏臨刑呼曰:「吾兒以一國迎降,反以為戮,信義俱棄,吾知其禍不旋踵矣!」衍妾劉氏,鬒髮如雲而有色,行刑者將免之,劉氏曰:「家國喪亡,義不受辱!」遂就死。



 宗弼,本姓魏,名弘夫,建錄為養子。建攻顧彥暉,宗弼常以建語泄之彥暉者,彥暉敗,建待之如初。建病且卒,宗弼守太師兼中書令、判六軍,輔政。衍已降,宗弼以蜀珍寶奉魏王及郭崇韜,求為西川節度使,魏王曰:「此我家物也,何用獻為?」居數日,為崇韜所殺。



 宗壽,許州民家
 子也。建以同姓,錄之為子。宗壽好學,工琴奕,為人恬退,喜道家之術,事建時為鎮江軍節度使。衍既立,宗壽為太子太保奉朝請,以煉丹養氣自娛。衍為淫亂,獨宗壽常切諫之,後為武信軍節度使。唐師伐蜀,所在迎降,魏王嘗以書招之,獨宗壽不降。聞衍已銜璧,大慟,從衍東遷,至岐陽,以賂賂守者,得入見衍,衍泣下霑襟,曰:「早從王言,豈有今日!」衍死,宗壽走澠池,聞莊宗遇弒,亡入熊耳山。天成二年,出詣京師,上書求衍宗族葬之。明宗嘉其忠,以為保義軍行軍司馬,封衍順正公,許以諸侯禮葬之。宗壽得王氏十八喪,葬之長安南三趙村。



 嗚呼,自秦、漢以來,學者多言祥瑞,雖有善辨之士,不能祛其惑也。予讀《蜀書》,至於龜、龍、麟、鳳、騶虞之類世所謂王者之嘉瑞,莫不畢出於其國,異哉!然考王氏之所以興亡成敗者,可以知之矣。或以為一王氏不足以當之,則視時天下治亂,可以知之矣。



 龍之為物也,以不見為神,以升雲行天為得志。今偃然暴露其形,是不神也;不上于天而下見於水中,是失職也。然其一何多歟,可以為妖矣!鳳凰,鳥之遠人者也。昔舜治天下,政成而民悅,命夔作樂,樂聲和,鳥獸聞之皆鼓舞。當是之時,鳳凰適至,舜之史因并記以為美,後世因以鳳來為有道之應。
 其後鳳凰數至,或出於庸君繆政之時,或出於危亡大亂之際,是果為瑞哉?麟,獸之遠人者也。昔魯哀公出獵,得之而不識,蓋索而獲之,非其自出也。故孔子書於《春秋》曰「西狩獲麟」者,譏之也。「西狩」,非其遠也;「獲麟」,惡其盡取也。狩必書地,而哀公馳騁所涉地多,不可遍以名舉,故書「西」以包眾地,謂其舉國之西皆至也。麟,人罕識之獸也,以見公之窮山竭澤而盡取,至於不識之獸,皆搜索而獲之,故曰「譏之也」。聖人已沒,而異端之說興,乃以麟為王者之瑞,而附以符命、讖緯詭怪之言。鳳嘗出於舜,以為瑞,猶有說也,及其後出於亂世,則可以知其非
 瑞矣。



 若麟者,前有治世如堯、舜、禹、湯、文、武、周公之世,未嘗一出,其一出而當亂世,然則孰知其為瑞哉?龜,玄物也,污泥川澤,不可勝數,其死而貴於卜官者,用適有宜爾。而《戴氏禮》以其在宮沼為王者難致之瑞,《戴禮》雜出於諸家,其失亦以多矣。騶虞,吾不知其何物也。《詩》曰:「吁嗟乎騶虞!」賈誼以謂騶者,文王之囿;虞,虞官也。當誼之時,其說如此,然則以之為獸者,其出於近世之說乎?



 夫破人之惑者,難與爭於篤信之時,待其有所疑焉,然後從而攻之可也。麟、鳳、龜、龍,王者之瑞,而出於五代之際,又皆萃于蜀,此雖好為祥瑞之說者亦可疑也。因其可
 疑者而攻之,庶幾惑者有以思焉。



 據《前蜀書》、《運歷圖》、《九國志》,皆云建以唐大順二年入成都為西川節度使,天復七年九月建號,明年正月改元武成,今以為定。惟《舊五代史》云「龍紀元年入成都,天祐五年建號改元」者繆也。至後唐同光三年蜀滅,則諸書皆同。自大順二年至同光三年,凡三十五年。



\end{pinyinscope}