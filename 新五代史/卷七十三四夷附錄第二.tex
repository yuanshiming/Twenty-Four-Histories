\article{卷七十三四夷附錄第二}

\begin{pinyinscope}

 兀欲,東丹王突欲子也。突欲奔于唐,兀欲留不從,號永康王。契丹好飲人血,突欲左右姬妾,多刺其臂吮之,其小過輒挑目、刲灼,不勝其毒。然喜賓客,好飲酒,工畫,頗知書。其自契丹歸中國,載書數千卷,樞密使趙延壽每假其異書、醫經,皆中國所無者。明宗時,自滑州朝京師,遙領武信軍節度使,食其俸,賜甲第一區,宮女數人。契丹兵助晉于太原,唐廢帝遣宦者秦繼旻、皇城使李彥
 紳殺突欲于其第。晉高祖追封突欲為燕王。



 德光滅晉,兀欲從至京師。德光殺繼旻、彥紳,籍其家貲,悉以賜兀欲。德光死欒城,兀欲與趙延壽及諸大將等俱入鎮州。延壽自稱權知軍國事,遣人求鎮州管鑰於兀欲,兀欲不與。延壽左右曰:「契丹大人聚而謀者洶洶,必有變,宜備之。



 今中國之兵,猶有萬人,可以擊虜;不然,事必不成。」延壽猶豫不決。兀欲妻,延壽以為妹,五月朔旦,兀欲召延壽及張礪、李崧、馮道等置酒,酒數行,兀欲謂延壽曰:「妹自上國來,當一見之。」延壽欣然與兀欲俱入。食頃,兀欲出坐,笑謂礪等曰:「燕王謀反,鎖之矣。諸君可無慮也。」
 又曰:「先帝在汴州與我算子一莖,許我知南朝軍國事,昨聞寢疾,無遺命,燕王安得自擅邪?」礪等罷去。兀欲召延壽廷立而詰之,延壽不能對。乃遣人監之,而籍其家貲。兀欲宣德光遺制曰:「永康王,大聖皇帝之嫡孫,人皇王之長子,可於中京即皇帝位。」中京,契丹謂鎮州也。遣使者告哀於諸鎮。蕭翰聞德光死,棄汴州而北,至鎮州,兀欲已去。翰以騎圍張礪宅,執礪而責曰:「汝教先帝勿用胡人為節度使,何也?」礪對不屈,翰鎖之。是夕,礪卒。



 兀欲為人俊偉,亦工畫,能飲酒,好禮士,德光嘗賜以絹數千匹,兀欲散之,一日而盡。兀欲已立,先遣人報其祖母
 述律。述律怒曰:「我兒平晉取天下,有大功業,其子在我側者當立,而人皇王背我歸中國,其子豈得立邪?」乃率兵逆兀欲,將廢之。兀欲留其將麻答守鎮州,晉諸將相隨德光在鎮州者皆留之而去。以翰林學士徐台符、李汗從行,與其祖母述律相距于石橋。述律所將兵多亡歸兀欲。兀欲乃幽述律於祖州。祖州,阿保機墓所也。



 述律為人多智而忍。阿保機死,悉召從行大將等妻,謂曰:「我今為寡婦矣,汝等豈宜有夫。」乃殺其大將百餘人,曰:「可往從先帝。」左右有過者,多送木葉山,殺於阿保機墓隊中,曰:「為我見先帝于地下。」大將趙思溫,本中國人也,
 以材勇為阿保機所寵,述律後以事怒之,使送木葉山,思溫辭不肯行。述律曰;「爾,先帝親信,安得不往見之?」思溫對曰:「親莫如后,后何不行?」述律曰:「我本欲從先帝於地下,以子幼,國中多故,未能也。然可斷吾一臂以送之。」左右切諫之,乃斷其一腕,而釋思溫不殺。初,德光之擊晉也,述律常非之,曰:「吾國用一漢人為主可乎?」德光曰:「不可也。」述律曰:「然則汝得中國不能有,後必有禍,悔無及矣。」德光死,載其尸歸,述律不哭而撫其尸曰:「待我國中人畜如故,然後葬汝。」已而兀欲囚之,後死于木葉山。



 兀欲更名阮,號天授皇帝,改元曰天祿。是歲八月,葬德
 光於木葉山,遣人至鎮州召馮道、和凝等會葬。使者至鎮州,鎮州軍亂,大將白再榮等逐出麻答。據定州,已而悉其眾以北。麻答者,德光之從弟也。德光滅晉,以為邢州節度使,兀欲立,命守鎮州。麻答尤酷虐,多略中國人,剝面,抉目,拔髮,斷腕而殺之,出入常以鉗鑿挑割之具自隨,寢處前後掛人肝、脛、手、足,言笑自若,鎮、定之人不勝其毒。麻答已去,馮道等乃南歸。



 漢乾祐元年,兀欲率萬騎攻邢州,陷內丘。契丹入寇,常以馬嘶為候。其來也,馬不嘶鳴,而矛戟夜有光,又月食,虜眾皆懼,以為凶,雖破內丘,而人馬傷死者太半。兀欲立五年,會諸部酋長,
 復謀入寇,諸部大人皆不欲,兀欲彊之。燕王述軋與太寧王嘔里僧等率兵殺兀欲於大神澱。德光子齊王述律聞亂,走南山。契丹擊殺述軋、嘔里僧,而迎述律以立。



 述律立,改元慶歷,號天順皇帝,後更名璟。述律有疾,不能近婦人,左右給事,多以宦者。然畋獵好飲酒,不恤國事,每酣飲,自夜至旦,晝則常睡,國人謂之「睡王」。



 初,兀欲常遣使聘漢,使者至中國而周太祖入立。太祖復遣將軍朱憲報聘,憲還而兀欲死。述律立,遂不復南寇。顯德六年夏,世宗北伐,以保大軍節度使田景咸為淤口關部署,右神武統軍李洪信為合流口部署,前鳳翔節度
 使王晏為益津關部署、侍衛親軍馬步都虞候韓通為陸路都部署。世宗自乾寧軍御龍舟,艛船戰艦,首尾數十里,至益津關,降其守將,而河路漸狹,舟不能進,乃捨舟陸行。瓦橋淤口關、瀛莫州守將,皆迎降。方下令進攻幽州,世宗遇疾,乃置雄州於瓦橋關、霸州於益津關而還。周師下三關、瀛、莫,兵不血刃。述律聞之,謂其國人曰:「此本漢地,今以還漢,又何惜耶?」述律後為庖者因其醉而殺之。



 嗚呼!自古夷狄服叛,雖不繫中國之盛衰,而中國之制夷狄,則必因其彊弱。



 予讀周《日歷》,見世宗取瀛、莫、定三
 關,兵不血刃,而史官譏其以王者之師,馳千里而襲人,輕萬乘之重於萑葦之間,以僥倖一勝。夫兵法,決機因勢,有不可失之時。世宗南平淮甸,北伐契丹,乘其勝威,擊其昏殆,世徒見周師之出何速,而不知述律有可取之機也。是時,述律以謂周之所取,皆漢故地,不足顧也。然則十四州之故地,皆可指麾而取矣。不幸世宗遇疾,功志不就。然瀛、莫、三關,遂得復為中國之人,而十四州之俗,至今陷於夷狄。彼其為志豈不可惜,而其功不亦壯哉!夫兵之變化屈伸,豈區區守常談者所可識也!



 初,蕭翰聞德光死,北歸,有同州郃陽縣令胡嶠為翰掌書
 記,隨入契丹。而翰妻爭石,告翰謀反,翰見殺,嶠無所依,居虜中七年。當周廣順三年,亡歸中國,略能道其所見。云:「自幽州西北入居庸關,明日,又西北入石門關,關路崖狹,一夫可以當百,此中國控扼契丹之險也。又三日,至可汗州,南望五臺山,其一峰最高者,東臺也。又三日,至新武州,西北行五十里有雞鳴山,云唐太宗北伐聞雞鳴于此,因以名山。明日,入永定關,此唐故關也。又四日,至歸化州。又三日,登天嶺,嶺東西連亙,有路北下,四顧冥然,黃雲白草,不可窮極。契丹謂嶠曰:『此辭鄉嶺也,可一南望而為永訣。』同行者皆慟哭,往往絕而復蘇。又
 行三四日,至黑榆林,時七月,寒如深冬。又明日,入斜谷,谷長五十里,高崖峻谷,仰不見日,而寒尤甚。已出谷,得平地,氣稍溫。又行二日,渡湟水。又明日,渡黑水。



 又二日,至湯城澱,地氣最溫,契丹若大寒,則就溫于此。其水泉清冷,草軟如茸,可藉以寢。而多異花,記其二種:一曰旱金,大如掌,金色爍人;一曰青囊,如中國金燈,而色類藍可愛。又二日,至儀坤州,渡麝香河。自幽州至此無里候,其所向不知為南北。又二日,至赤崖。翰與兀欲相及,遂及述律戰于沙河。述律兵敗而北,兀欲追至獨樹渡,遂囚述律於撲馬山。又行三日,遂至上京,所謂西樓也。西
 樓有邑屋市肆,交易無錢而用布。有綾錦諸工作、宦者、翰林、伎術、教坊、角牴、秀才、僧、尼、道士等,皆中國人,而並、汾、幽、薊之人尤多。自上京東去四十里,至真珠寨,始食菜。明日,東行,地勢漸高,西望平地松林鬱然數十里。遂入平川,多草木,始食西瓜,云契丹破回紇得此種,以牛糞覆棚而種,大如中國冬瓜而味甘。又東行,至褭潭,始有柳,而水草豐美,有息雞草尤美,而本大,馬食不過十本而飽。自褭潭入大山,行十餘日而出,過一大林,長二三里,皆蕪荑,枝葉有芒刺如箭羽,其地皆無草。兀欲時卓帳于此,會諸部人葬德光。自此西南行,日六十里,行
 七日,至大山門,兩高山相去一里,而長松豐草,珍禽野卉,有屋室碑石,曰:『陵所也。』兀欲入祭,諸部大人惟執祭器者得入。入而門闔。明日開門,曰『拋盞』,禮畢。問其禮,皆秘不肯言。」嶠所目見囚述律、葬德光等事,與中國所記差異。



 已而翰得罪被鎖,嶠與部曲東之福州。福州,翰所治也。嶠等東行,過一山,名十三山,云此西南去幽州二千里。又東行,數日,過衛州,有居人三十餘家,蓋契丹所虜中國衛州人,築城而居之。嶠至福州而契丹多憐嶠,教其逃歸,嶠因得其諸國種類遠近。云:「距契丹國東至于海,有鐵甸,其族野居皮帳,而人剛勇。其地少草木,水
 鹹濁,色如血,澄之久而後可飲。又東,女真,善射,多牛、鹿、野狗。其人無定居,行以牛負物,遇雨則張革為屋。常作鹿鳴,呼鹿而射之,食其生肉。能釀糜為酒,醉則縛之而睡,醒而後解,不然,則殺人。又東南,渤海,又東,遼國,皆與契丹略同。其南海曲,有魚鹽之利。又南,奚,與契丹略同,而人好殺戮。又南至于榆關矣,西南至儒州,皆故漢地。西則突厥、回紇。西北至嫗厥律,其人長大,髦頭,酋長全其髮,盛以紫囊。地苦寒,水出大魚,契丹仰食。又多黑、白、黃貂鼠皮,北方諸國皆仰足。其人最勇,鄰國不敢侵。又其西,轄戛,又其北,單于突厥,皆與嫗厥律略同。又北,黑
 車子,善作車帳,其人知孝義,地貧無所產。



 云契丹之先,常役回紇,後背之走黑車子,始學作車帳。又北,牛蹄突厥,人身牛足,其地尤寒,水曰瓠河,夏秋冰厚二尺,春冬冰徹底,常燒器銷冰乃得飲。東北,至韈劫子,其人髦首,披布為衣,不鞍而騎,大弓長箭,尤善射,遇人輒殺而生食其肉,契丹等國皆畏之。契丹五騎遇一韈劫子,則皆散走。其國三面皆室韋,一曰室韋,二曰黃頭室韋,三曰獸室韋。其地多銅、鐵、金、銀,其人工巧,銅鐵諸器皆精好,善織毛錦。地尤寒,馬溺至地成冰堆。又北,狗國,人身狗首,長毛不衣,手捕猛獸,語為犬嗥,其妻皆人,能漢語,
 生男為狗,女為人,自相婚嫁,穴居食生,而妻女人食。云嘗有中國人至其國,其妻憐之使逃歸,與其箸十餘隻,教其每走十餘里遺一箸,狗夫追之,見其家物,必銜而歸,則不能追矣。」其說如此。又曰:「契丹嘗選百里馬二十匹,遣十人齎乾食少北行,窮其所見。其人自黑車子,歷牛蹄國以北,行一年,經四十三城,居人多以木皮為屋,其語言無譯者,不知其國地、山川、部族、名號。其地氣,遇平地則溫和,山林則寒冽。至三十三城,得一人,能鐵甸語,其言頗可解,云地名頡利烏于邪堰。云『自此以北,龍蛇猛獸、魑魅群行,不可往矣』。其人乃還。此北荒之極也。」



 契
 丹謂嶠曰:「夷狄之人豈能勝中國?然晉所以敗者,主暗而臣不忠。」因具道諸國事,曰:「子歸悉以語漢人,使漢人努力事其主,無為夷狄所虜,吾國非人境也。」嶠歸,錄以為《陷虜記》云。



 契丹年號,諸家所記,舛謬非一,莫可考正,惟嘗見於中國者可據也。據耶律德光《立晉高祖冊文》云:「惟天顯九年,歲次丙申。」是歲乃晉天福元年,推而上之,得唐天成三年戊子為天顯元年。按《契丹附錄》,德光與唐明宗同年而立,立三年改元天顯,與此正合矣。又據開運四年德光滅晉入汴,肆赦,稱會同十年。



 推而上之,得天福三年為會同元年,是天顯盡十年,而十一年改為會同矣。惟此二者,其據甚明。餘皆不足考也。《附錄》所載夷狄年號,多略不書,蓋無所用,故不必備也。



\end{pinyinscope}