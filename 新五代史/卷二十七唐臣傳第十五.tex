\article{卷二十七唐臣傳第十五}

\begin{pinyinscope}

 硃弘昭馮鷿附硃弘昭,太原人也。少事明宗為客將,明宗即位,為文思使。與安重誨有隙,故常使于外。董璋為東川節度使,乃以弘昭為副使。西川孟知祥殺其監軍李嚴,弘昭大懼,求還京師,璋不許,遂相猜忌,弘昭益開懷待之不疑,璋頗重其為人。後璋有軍事,遣弘昭入朝,弘昭乃免。遷左衛大將軍內客省使、宣徽南院使、鳳翔節度使。孟知祥反,石敬瑭伐蜀,久無功,明宗遣安重誨督軍。是時重誨
 已有間。重誨至鳳翔,弘昭迎謁,禮甚恭,延重誨于家,使其妻妾侍飲食。重誨以弘昭厚己,酒酣,具言蒙天子厚恩,而所以讒間之端,因泣下。弘昭即奏言重誨怨望,又陰遣人馳告敬瑭,使拒重誨。會敬瑭以糧餉不繼,遽燒營返軍。重誨亦以被讒召還,過鳳翔,弘昭閉門不納,重誨由此得罪死。樞密使范延光尤惡弘昭為人,罷為左武衛上將軍、宣徽南院使。久之,為山南東道節度使。是時,明宗已病,而秦王從榮禍起有端,唐諸大臣皆欲引去以避禍。樞密使范延光、趙延壽日夕更見,涕泣求去,明宗怒而不許。延壽使其妻興平公主入言於中,延光
 亦因孟漢瓊、王淑妃進說,故皆得罷。以弘昭及馮贇代延壽、延光,弘昭入見,辭曰:「臣廝養之才,不足當大任。」明宗叱之曰:「公等皆不欲在吾目前邪?吾養公等安用!」弘昭惶恐,乃視事。



 馮贇者,亦太原人也。其父璋,事明宗為閽者。贇為兒時,以通黠為明宗所愛。



 明宗為節度使,以贇為進奏官。明宗即位,即為客省使、宣徽北院使。歷河東忠武節度使、三司使。明宗病甚,大臣稀復進見,而孟漢瓊、王淑妃用事,弘昭及贇並掌機務於中,大事皆決此四人。及殺秦王而立愍帝,益自以為功。又其所用多非其人,給事中陳乂,為人險譎,好陰謀,嘗事梁張漢傑,
 又事郭崇韜,兩人皆輒敗死,弘昭乃引以為樞密直學士,而用其謀。是時,弘昭、贇遣漢瓊至魏,召愍帝入立,而留漢瓊權知後事。明年正月,漢瓊請入朝,弘昭、贇乃議徙成德范延光代漢瓊,北京留守石敬瑭代延光,鳳翔潞王從珂代敬瑭。三人者皆唐大臣,以漢瓊故,輕易其地,又不降制書,第遣使者監其上道,從珂由此遂反。從珂兵已東,愍帝大懼,遣人召弘昭計事。弘昭謂其客穆延暉曰:「上召我急,將罪我也。吾兒婦,君之女也,其以歸,無使及禍。」乃拔劍大哭,欲自裁,而家人止之。使者促弘昭入見甚急,弘昭呼曰:「窮至此邪!」乃自投于井以死。
 安從進聞之,亦殺贇于家,贇母新死,子母棄尸于道,妻子皆見殺。贇有子三歲,其故吏張守素匿之以免。漢高祖即位,贈弘昭尚書令,贇中書令。



 劉延朗劉延朗,宋州虞城人也。初,廢帝起於鳳翔,與共事者五人:節度判官韓昭胤,掌書記李專美,牙將宋審虔,客將房暠,而延朗為孔目官。初,愍帝即位,徙廢帝為北京留守,不降制書,遣供奉官趙處愿促帝上道。帝疑惑,召昭胤等計議,昭胤等皆勸帝反,由是事無大小,皆此五人謀之。而暠又喜鬼神巫祝之說,有瞽者張濛,自言事太白山神,神,魏崔浩也,其言吉凶無不中,暠素信之。嘗引
 濛見帝,聞其語聲,驚曰:「此非人臣也!」暠使蒙問於神,神傳語曰:「三珠併一珠,驢馬沒人驅。歲月甲庚午,中興戊己土。」暠不曉其義,使問濛,濛曰:「神言如此,我能傳之,不能解也。」帝即以濛為館驛巡官。



 帝將反,而兵少,又乏食,由此甚懼,使暠問蒙,濛傳神語曰:「王當有天下,可無憂!」於是決反,使專美作檄書,言:「朱弘昭、馮贇幸明宗病,殺秦王而立愍帝。帝年少,小人用事,離間骨肉,將問罪於朝!」遣使者馳告諸鎮,皆不應,獨隴州防禦使相里金遣其判官薛文遇計事。帝得文遇,大喜。而延朗調率城中民財以給軍。王思同率諸鎮兵圍鳳翔,廢帝懼,又遣暠
 問神,神曰:「王兵少,東兵來,所以迎王也。」已而東兵果叛降于帝。帝入京師,即位之日,受冊明宗柩前。冊曰:「維應順元年,歲次甲午,四月庚午朔。」帝回顧贇曰:「張濛神言,豈不驗哉!」



 由是贇益見親信,而專以巫祝用事。



 帝既立,以昭胤為左諫議大夫、端明殿學士,專美為比部郎中、樞密院直學士,審虔為皇城使,暠為宣徽北院使,延朗為莊宅使。久之,昭胤、暠為樞密使,延朗為副使,審虔為侍衛步軍都指揮使,而薛文遇亦為職方郎中、樞密院直學士。由是審虔將兵,專美、文遇主謀議,而昭胤、暠及延朗掌機密。



 初,帝與晉高祖俱事明宗,而心不相悅。
 帝既入立,高祖不得已來朝,而心頗自疑,欲求歸鎮,且難言之,乃陽為羸疾,灸灼滿身,冀帝憐而遣之。延朗等多言敬瑭可留京師,昭胤、專美曰:「敬瑭與趙延壽皆尚唐公主,不可獨留。」乃復授高祖河東而遣之。是時,契丹數寇北邊,以高祖為大同、振武、威塞、彰國等軍蕃漢馬步軍都總管,屯于忻州。而屯兵忽變,擁高祖呼「萬歲」,高祖懼,斬三十餘人而後止。於是帝益疑之。



 是時,高祖悉握精兵在北,饋運芻糧,遠近勞弊。帝與延朗等日夕謀議,而專美、文遇迭宿中興殿盧,召見訪問,常至夜分而罷。是時,高祖弟重胤為皇城副使,而石氏公主母曹太
 后居中,因得伺帝動靜言語以報高祖,高祖益自危懼。每帝遣使者勞軍,即陽為羸疾不自堪,因數求解總管以探帝心。是時,帝母魏氏追封宣憲皇太后,而墓在太原,有司議立寢宮。高祖建言陵與民家墓相雜,不可立宮。帝疑高祖欲毀民墓,為國取怨,帝由此發怒,罷高祖總管,徙鄆州。延朗等多言不可,而司天趙延義亦言天象失度,宜安靜以弭災,其事遂止。



 後月餘,文遇獨直,帝夜召之,語罷敬瑭事,文遇曰:「臣聞『作舍道邊,三年不成』。國家之事,斷在陛下。且敬瑭徙亦反,不徙亦反,遲速爾,不如先事圖之。」帝大喜曰:「術者言朕今年當得一賢
 佐以定天下,卿其是邪!」乃令文遇手書除目,夜半下學士院草制。明日宣制,文武兩班皆失色。居五六日,敬瑭以反聞。



 敬瑭上書,言帝非明宗子,而許王從益次當立。帝得書大怒,手壞而投之,召學士馬胤孫為答詔,曰:「宜以惡語詆之。」



 延朗等請帝親征,帝心憂懼,常惡言敬瑭事,每戒人曰:「爾無說石郎,令我心膽墮地!」由此不欲行。而延朗等屢迫之,乃行。至懷州,帝夜召李崧問以計策。



 文遇不知而繼至,帝見之色變,崧躡其足,文遇乃出。帝曰:「我見文遇肉顫,欲抽刀刺之。」崧曰:「文遇小人,致誤大事,刺之益醜。」乃已。是時,契丹已立敬瑭為天子,以兵
 而南,帝惶惑不知所之。遣審虔將千騎至白馬坡踏戰地,審虔曰:「何地不堪戰?雖有其地,何人肯立于此?不如還也。」帝遂還,自焚。高祖入京師,延朗等六人皆除名為民。



 初,延朗與暠並掌機密,延朗專任事,諸將當得州者,不以功次為先後,納賂多者得善州,少及無賂者得惡州,或久而不得,由是人人皆怨。暠心患之,而不能爭也,但日飽食高枕而已。每延朗議事,則垂頭陽睡不省。及晉兵入,延朗以一騎走南山,過其家,指而歎曰:「吾積錢三十萬于此,不知何人取之!」遂為追兵所殺。晉高祖聞暠常不與延朗事,哀之,後復以為將。歲餘卒。專美事
 晉為大理卿,開運中卒。當晉之將起,廢帝以昭胤為中書侍郎、同中書門下平章事,出為河陽節度使,與審虔、文遇皆不知其所終。



 嗚呼,禍福成敗之理,可不戒哉!張濛神言驗矣,然焉知其不為禍也!予之所記,大抵如此,覽者可以深思焉。廢帝之起,所與圖議者,此五六人而已。考其逆順之理,雖有智者為之謀,未必能不敗,況如此五六人者哉!故并述以附延朗,見其始終之際云。



 康思立康思立,本山陰諸部人也。少為騎將,從莊宗破梁夾城,戰柏鄉,累以功遷突騎指揮使。明宗即位,歷應嵐二州
 刺史、宿州團練使、昭武軍節度使,徙鎮保義,皆有善政。潞王從珂反於鳳翔,愍帝遣王思同等討之,思立有捧聖、羽林屯兵千五百人,乃以羽林千人屬思同。思同至鳳翔,軍叛,降于從珂。思立聞之,欲盡誅羽林千人家屬,未及,而從珂兵已至,思立乃以捧聖兵城守,從珂兵傅其城,呼曰:「西兵七萬策新天子,爾五百人其能拒邪?徒陷陜人於死耳!」捧聖兵聞之,皆解甲,思立遂開門迎從珂。廢帝即位,以思立初無降意,頗不悅之,徙安遠,又徙安國,以年老罷為右神武統軍。石敬瑭反太原,廢帝以思立為北面行營馬軍都指揮使。



 廢帝幸懷州,遣思立
 將從駕騎兵出團柏谷救張敬達,未至,而敬達死,楊光遠降晉,思立疾,卒于道。晉高祖入立,贈太子少師。



 康義誠康義誠,字信臣,代北三部落人也。以騎射事晉王,莊宗時為突騎指揮使。從明宗討趙在禮,至魏而軍變,義誠前陳莊宗過失,勸明宗南嚮。明宗即位,遷捧聖指揮使,領汾州刺史。從破朱守殷,遷侍衛親軍馬步軍都指揮使,領河陽三城節度使。出為山南東道節度使,復為親軍都指揮使,領河陽,加同中書門下平章事。



 秦王從榮素驕,自為河南尹,典六軍,拜大元帥,唐諸大臣皆懼禍及,思自脫,獨義誠心結之,遣其子事秦王府。明宗病,從
 榮謀以兵入宮,唐大臣朱弘昭、馮贇等皆以為不可,而義誠獨持兩端。從榮已舉兵,至天津橋,弘昭等入,以反白,明宗涕泣召義誠,使自處置,而義誠卒不出兵。馬軍指揮使朱弘實以兵擊從榮,從榮敗走,見殺。



 三司使孫岳嘗為馮贇言從榮必敗之狀,義誠聞而不悅。及從榮死,義誠始引兵入河南府,召岳檢閱從榮家貲。岳至,義誠乘亂,使人射之,岳走至通利坊見殺,明宗不能詰。義誠已殺岳,又以從榮故,與弘實有隙。愍帝即位,弘實常以誅從榮功自負,義誠心益不平。



 潞王從珂反鳳翔,王思同率諸鎮兵圍之,興元張虔釗兵叛降從珂,思同走,
 諸鎮兵皆潰。愍帝大怒,謂朱弘昭等曰:「朕新即位,天下事皆出諸公,然於事兄,未有失範,諸公以大計見迫,不能獨違,事一至此,何方轉禍?吾當率左右往迎吾兄,遜以位,茍不吾信,死其所也!」弘昭等惶恐不能對,義誠前曰:「西師驚潰,主將怯耳。今京師兵尚多,臣請盡將以西,扼關而守,招集亡散,以為後圖。」愍帝以為然,幸左藏庫,親給將士人絹二十匹,錢五千。是時,明宗山陵未畢,帑藏空虛。軍士負物揚言曰:「到鳳翔更請一分。」朱弘實見軍士無鬥志,而義誠盡將以西,疑其二心,謂義誠曰:「今西師小衄,而無一騎東者,人心可知。不如以見兵守
 京師以自固,彼雖幸勝,特得虔釗一軍耳。諸鎮之兵在後,其敢徑來邪!」義誠怒曰:「如此言,弘實反矣!」弘實曰:「公謂誰欲反邪?」其聲厲而聞。愍帝召兩人,爭於前,帝不能決,遂斬弘實,以義誠為招討使,悉將禁軍以西。愍帝奔衛州。義誠行至新安,降于從珂。清泰元年四月,斬于興教門外,夷其族。



 嗚呼!五代為國,興亡以兵,而其軍制,後世無足稱焉。惟侍衛親軍之號,今猶因之而甚重,此五代之遺制也。然原其始起微矣,及其至也,可謂盛哉!當唐之末,方鎮之兵多矣,凡一軍有指揮使一人,而合一州之諸軍,又有
 馬步軍都指揮使一人,蓋其卒伍之長也。自梁以宣武軍建國,因其舊制,有在京馬步軍都指揮使,後唐因之,至明宗時,始更為侍衛親軍馬步軍都指揮使。當是時,天子自有六軍諸衛之職,六軍有統軍,諸衛有將軍,而又以大臣宗室一人判六軍諸衛事,此朝廷大將天子國兵之舊制也。而侍衛親軍者,天子自將之私兵也,推其名號可知矣。天子自為將,則都指揮使乃其卒伍之都長耳。然自漢、周以來,其職益重,漢有侍衛司獄,凡朝廷大事皆決侍衛獄。是時,史弘肇為都指揮使,與宰相、樞密使並執國政,而弘肇尤專任,以至於亡。語曰:「涓涓
 不絕,流為江河。熒熒不滅,炎炎奈何?」



 可不戒哉!然是時,方鎮各自有兵,天子親軍猶不過京師之兵而已。今方鎮名存而實亡,六軍諸衛又益以廢,朝廷無大將之職,而舉天下內外之兵皆屬侍衛司矣。則為都指揮使者,其權豈不益重哉!親軍之號,始於明宗,其後又有殿前都指揮使,亦親軍也,皆不見其更置之始。今天下之兵,分屬此兩司矣。



 藥彥稠藥彥稠,沙陀三部落人也。初為騎將,明宗即位,拜澄州刺史。從王晏球破王都定州,遷侍衛步軍都虞候,領壽州節度使。安重誨矯詔遣河中指揮使楊彥溫逐其節
 度使潞王從珂。以彥稠為招討使,明宗疑彥溫有所說,戒彥稠得彥溫毋殺,將訊之。彥稠希重誨旨,殺彥溫以滅口,明宗大怒,然不之罪也。長興中,為靜難軍節度使,黨項阿埋、屈悉保等族抄掠方渠,邀殺回鶻使者,明宗遣彥稠與靈武康福會兵擊之,阿埋等亡竄山谷。明宗以謂黨項知懼,可加約束而綏撫之。使者未至,彥稠等自牛兒族入白魚谷,盡誅其族,獲其大首領連香等,遣人上捷。明宗謂其使者曰:「吾誅黨項,非有所利也。凡軍中所獲,悉與士卒分之,毋以進奉為名,重斂軍士也。」已而彥稠以黨項所掠回鶻進奉玉兩團及遺秦王金裝
 胡錄等來獻,明宗曰:「吾已語彥稠矣,不可失信。」因悉以賜彥稠。又逐鹽州諸戎,取其所掠男女千餘人。



 潞王從珂反,彥稠為招討副使。王思同兵潰,彥稠與思同俱東走,為潞王兵所得,囚之華州獄,已而殺之。晉高祖立,贈侍中。



\end{pinyinscope}