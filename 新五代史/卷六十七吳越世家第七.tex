\article{卷六十七吳越世家第七}

\begin{pinyinscope}

 錢鏐,字具美,杭州臨安人也。臨安里中有大木,鏐幼時與群兒戲木下,鏐坐大石指麾群兒為隊伍,號令頗有法,群兒皆憚之。及壯,無賴,不喜事生業,以販鹽為盜。縣錄事鐘起有子數人,與鏐飲博,起嘗禁其諸子,諸子多竊從之遊。豫章人有善術者,望牛斗間有王氣。牛斗,錢塘分也,因遊錢塘。占之在臨安,乃之臨安,以相法隱市中,陰求其人。起與術者善,術者私謂起曰:「占君縣有貴
 人,求之市中不可得,視君之相貴矣,然不足當之。」起乃為置酒,悉召賢豪為會,陰令術者遍視之,皆不足當。術者過起家,鏐適從外來,見起,反走,術者望見之,大驚曰:「此真貴人也!」起笑曰:「此吾旁舍錢生爾。」術者召鏐至,熟視之,顧起曰:「君之貴者,因此人也。」乃慰鏐曰:「子骨法非常,願自愛。」因與起訣曰:「吾求其人者,非有所欲也,直欲質吾術爾。」明日乃去。起始縱其子等與鏐遊,時時貸其窮乏。



 鏐善射與槊,稍通圖緯諸書。唐乾符二年,浙西裨將王郢作亂,石鑑鎮將董昌募鄉兵討賊,表鏐偏將,擊郢破之。是時,黃巢眾已數千,攻掠浙東,至臨安,鏐
 曰:「今鎮兵少而賊兵多,難以力禦,宜出奇兵邀之。」乃與勁卒二十人伏山谷中,巢先鋒度險皆單騎,鏐伏弩射殺其將,巢兵亂,鏐引勁卒蹂之,斬首數百級。鏐曰:「此可一用爾,大眾至何可敵邪!」乃引兵趨八百里,八百里,地名也,告道旁媼曰:「後有問者,告曰:『臨安兵屯八百里矣。』」巢眾至,聞媼語,不知其地名,曰:「嚮十餘卒不可敵,況八百里乎!」遂急引兵過。都統高駢聞巢不敢犯臨安,壯之,召董昌與鏐俱至廣陵。久之,駢無討賊意,昌等不見用,辭還,駢表昌杭州刺史。是時,天下已亂,昌乃團諸縣兵為八都,以鏐為都指揮使,成及為靖江都將。



 中和
 二年,越州觀察使劉漢宏與昌有隙,漢宏遣其弟漢宥、都虞候辛約,屯兵西陵。鏐率八都兵渡江,竊取軍號,斫其營,營中驚擾,因焚之,漢宥等皆走。漢宏復遣將黃珪、何肅屯諸暨、蕭山,鏐皆攻破之。與漢宏遇,戰,大敗之,殺何肅、辛約。漢宏易服持膾刀以遁,追者及之,漢宏曰:「我宰夫也。」舉刀示之,乃免。



 四年,僖宗遣中使焦居璠為杭、越通和使,詔昌及漢宏罷兵,皆不奉詔。漢宏遣其將朱褒、韓公玫、施堅實等以舟兵屯望海。鏐出平水,成及夜率奇兵破褒等於曹娥埭,進屯豐山,施堅實等降,遂攻破越州。漢宏走台州,臺州刺史執漢宏送於鏐,斬于會
 稽,族其家。鏐乃奏昌代漢寵,而自居杭州。



 光啟三年,拜鏐左衛大將軍、杭州刺史,昌越州觀察使。是歲,畢師鐸囚高駢,淮南大亂,六合鎮將徐約攻取蘇州。潤州牙將劉浩逐其帥周寶,寶奔常州,浩推度支催勘官薛朗為帥。鏐遣都將成及、杜棱等攻常州,取周寶以歸,鏐具軍禮郊迎,館寶於樟亭,寶病卒。棱等進攻潤州,逐劉浩,執薛朗,剖其心以祭寶。然後遣其弟金求攻徐約,約敗走入海,追殺之。



 昭宗拜鏐杭州防禦使。是時,楊行密、孫儒爭淮南,與鏐戰蘇、常間。久之,儒為行密所殺,行密據淮南,取潤州,鏐亦取蘇、常。唐升越州威勝軍,以董昌為節
 度使,封隴西郡王;杭州武勝軍,拜鏐都團練使,以成及為副使。及字弘濟,與鏐同事攻討,謀多出於及,而鏐以女妻及子仁琇。鏐乃以杜棱、阮結、顧全武等為將校,沈崧、皮光業、林鼎、羅隱為賓客。



 景福二年,拜鏐鎮海軍節度使、潤州刺史。乾寧元年,加同中書門下平章事。



 二年,越州董昌反。昌素愚,不能決事,臨民訟,以骰子擲之,而勝者為直。妖人應智王溫、巫韓媼等,以妖言惑昌,獻鳥獸為符瑞。牙將倪德儒謂昌曰:「曩時謠言有羅平鳥主越人禍福,民間多圖其形禱祠之,視王書名與圖類。」因出圖以示昌。



 昌大悅,乃自稱皇帝,國號羅平,改元順天。
 分其兵為兩軍,中軍衣黃,外軍衣白,銘其衣曰「歸義」。副使黃竭切戒昌以為不可,昌大怒,使人斬竭,持其首至,罵曰:「此賊負我好聖,明時三公不肯作,乃自求死邪!」投之圊中。昌乃以書告鏐,鏐以昌反狀聞。



 昭宗下詔削昌官爵,封鏐彭城郡王,浙江東道招討使。鏐曰:「董氏於吾有恩,不可遽伐。」以兵三萬屯迎恩門,遣其客沈滂諭昌使改過。昌以錢二百萬犒軍,執應智等送軍中,自請待罪,鏐乃還兵。昌復拒命,遣其將陳郁、崔溫等屯香嚴、石侯,乞兵於楊行密,行密遣安仁義救昌。鏐遣顧全武攻昌,斬崔溫。昌所用諸將徐珣、湯臼、袁邠皆庸人,不
 知兵,遇全武輒敗。昌兄子真,驍勇善戰,全武等攻之,逾年不能克。真與其裨將刺羽有隙,羽譖之,昌殺真,兵乃敗。全武執昌歸杭州,行至西小江,昌顧左右曰:「吾與錢公俱起鄉里,吾嘗為大將,今何面復見之乎!」



 左右相對泣下,因真目大呼,投水死。



 昭宗以宰相王溥鎮越州,溥請授鏐,乃改威勝軍為鎮東軍,拜鏐鎮海、鎮東軍節度使、加檢校太尉、中書令,賜鐵券,恕九死。鏐如越州受命,還治錢塘,號越州為「東府」。光化元年,移鎮海軍於杭州,加鏐檢校太師,改鏐鄉里曰廣義鄉勳貴里,鏐素所居營曰衣錦營。婺州刺史王壇叛附于淮南,楊行密
 遣其將康儒應壇,因攻睦州。鏐遣其弟金求敗儒於軒渚,壇奔宣州。昭宗詔鏐圖形凌煙閣,升衣錦營為衣錦城,石鑒山曰衣錦山,大官山曰功臣山。鏐游衣錦城,宴故老,山林皆覆以錦,號其幼所嘗戲大木曰「衣錦將軍」。



 天復二年,封鏐越王。鏐巡衣錦城,武勇右都指揮使徐綰與左都指揮使許再思叛,焚掠城郭,攻內城,鏐子傳瑛及其將馬綽、陳為等閉門拒之。鏐歸,至北郭門不得入。成及代鏐與綰戰,斬首百餘級,綰屯龍興寺。鏐微服踰城而入,遣馬綽、王榮、杜建徽等分屯諸門,使顧全武備東府,全武曰:「東府不足慮,可慮者淮南爾,綰急,必召淮
 兵至,患不細矣。楊公大丈夫,今以難告,必能閔我。」鏐以為然。全武曰:「獨行,事必不濟,請擇諸公子可行者。」鏐曰:「吾嘗欲以元鳷婚楊氏。」乃使隨全武如廣陵。綰果召田頵於宣州。全武等至廣陵,行密以女妻元鳷,亟召頵還。頵取鏐錢百萬,質鏐子元瓘而歸。



 天祐元年,封鏐吳王,鏐建功臣堂,立碑紀功,列賓佐將校名氏於碑陰者五百人。四年,升衣錦城為安國衣錦軍。



 梁太祖即位,封鏐吳越王兼淮南節度使。客有勸鏐拒梁命者,鏐笑曰:「吾豈失為孫仲謀邪!」遂受之。太祖嘗問吳越進奏吏曰:「錢鏐平生有所好乎?」吏曰:「好玉帶、名馬。」太祖笑曰:「真
 英雄也。」乃以玉帶一匣、打球御馬十匹賜之。



 江西危全諷等為楊渥所敗,信州危仔倡奔於鏐,鏐惡其姓,改曰元。開平二年,加鏐守中書令,改臨安縣為安國縣,廣義鄉為衣錦鄉。三年,加守太保。



 楊渥將周本、陳章圍蘇州,鏐遣其弟鋸、鏢救之。淮兵為水柵環城,以銅鈴繫網沈水中,斷潛行者。水軍卒司馬福,多智而善水行,乃先以巨竹觸網,淮人聞鈴聲遂舉網,福乃過,入城中,其出也亦然。乃取其軍號,內外夾攻,號令相應,淮人以為神,遂大敗之,本等走,擒其將閭丘直、何明等。



 四年,鏐游衣錦軍,作《還鄉歌》曰:「三節還鄉兮掛錦衣,父老遠來相追隨。



 牛斗無孛人無欺,吳越一王駟馬歸。」乾化元年,加鏐守尚書令,兼淮南、宣潤等道四面行營都統。立生祠於衣錦軍。鏐弟鏢居湖州,擅殺戍將潘長,懼罪奔于淮南。



 二年,梁郢王友珪立,冊尊鏐尚父。末帝貞明三年,加鏐天下兵馬都元帥,開府置官屬。四年,楊隆演取虔州,鏐始由海路入貢京師。龍德元年,賜鏐詔書不名。



 唐莊宗入洛,鏐遣使貢獻,求玉冊。莊宗下其議於有司,群臣皆以謂非天子不得用玉冊,郭崇韜尤為不可,既而許之,乃賜鏐玉冊金印。鏐因以鎮海等軍節度授其子元瓘,自稱吳越國王,更名所居曰宮殿、府曰朝,官屬皆稱臣,起
 玉冊、金券、詔書三樓於衣錦軍,遣使冊新羅、渤海王,海中諸國,皆封拜其君長。



 明宗即位,安重誨用事,鏐致書重誨,書辭嫚,重誨大怒。是時,供奉官烏昭遇、韓玫使吳越,既還,玫誣昭遇稱臣舞蹈,重誨乃奏削鏐王爵、元帥、尚父,以太師致仕。元瓘等遣人以絹表間道自陳。安重誨死,明宗乃復鏐官爵。長興三年,鏐卒,年八十一,謚曰武肅。子元瓘立。



 元瓘字明寶,少為質於田頵。頵叛於吳,楊行密會越兵攻之,頵每戰敗歸,即欲殺元瓘,頵母嘗蔽護之。後頵將出,語左右曰:「今日不勝,必斬錢郎。」是日頵戰死,元瓘得
 歸。



 鏐臥病,召諸大將告之曰:「吾子皆愚懦,不足任後事,吾死,公等自擇之。」



 諸將泣下,皆曰:「元瓘從王征伐最有功,諸子莫及,請立之。」鏐乃出管鑰數篋,召元瓘與之曰:「諸將許爾矣。」鏐卒,元瓘立,襲封吳越國王,玉冊、金印,皆如鏐故事。



 王延政自立於建州,閩中大亂,元瓘遣其將仰詮、薛萬忠等攻之,逾年,大敗而歸。元瓘亦善撫將士,好儒學,善為詩,使其國相沈崧置擇能院,選吳中文士錄用之。然性尤奢僭,好治宮室。天福六年,杭州大火,燒其宮室迨盡,元瓘避之,火輒隨發。元瓘大懼,因病狂。是歲卒,年五十五,謚曰文穆。子佐立。



 佐字祐,立時年十三,諸將皆少佐,佐初優容之,諸將稍不法,佐乃黜其大將章德安於明州、李文慶於睦州,殺內都監杜昭達、統軍使闞璠,由是國中皆畏恐。



 王延義、延政兄弟相攻,卓儼明、朱文進、李仁達等自相篡殺,連兵不解者數年。仁達附于李景,已而又叛,景兵攻之,仁達求救於佐。佐召諸將計事,諸將皆不欲行,佐奮然曰:「吾為元帥,而不能舉兵邪?諸將吾家素畜養,獨不肯以身先我乎?有異吾議者斬!」乃遣其統軍使張筠、趙承泰等率兵三萬,水陸赴之。遣將誓軍,號令齊整。筠等大敗景兵,俘馘萬計,獲其將楊業、蔡遇等,遂取福州而還,
 由是諸將皆服。



 佐立七年,襲封吳越國王,玉冊、金印,皆如元瓘。開運四年,佐卒,年二十,謚曰忠獻。弟俶立。



 俶字文德。佐卒,弟倧以次立。初,元瓘質於宣州,以胡進思、戴惲等自隨,元瓘立,用進思等為大將。佐既年少,進思以舊將自待,甚見尊禮,及倧立,頗卑侮之,進思不能平。倧大閱兵於碧波亭,方第賞,進思前諫以賞太厚,倧怒擲筆水中曰:「以物與軍士,吾豈私之,何見咎也!」進思大懼。歲除,畫工獻《鐘馗擊鬼圖》,倧以詩題圖上,進思見之大悟,知倧將殺己。是夕擁衛兵廢倧,囚於義和院,迎俶立之,遷倧於東府。俶歷漢、周,襲封吳越國王,賜玉冊、
 金印。



 世宗征淮南,詔俶攻常、宣二州以牽李景,俶治國中兵以待。景聞周師將大舉,乃遣使安撫,境上皆戒嚴。蘇州候吏陳滿不知景使,以謂朝廷已克諸州,遣使安撫矣,亟言於俶,請舉兵以應。俶相國吳程遽調兵以出,相國元德昭以為王師必未渡淮,與程爭於俶前,不可奪。程等攻常州,果為景將柴克宏所敗,程裨將邵可遷力戰,可遷子死馬前,猶戰不顧,程等僅以身免。周師渡淮,俶乃盡括國中丁民益兵,使邵可遷等以戰船四百艘、水軍萬七千人至于通州以會期。



 吳越自唐末有國,而楊行密、李昪據有江淮。吳越貢賦,朝廷遣使,皆由登、
 萊泛海,歲常飄溺其使。顯德四年,詔遣左諫議大夫尹日就、吏部郎中崔頌等使于俶,世宗諭之曰:「朕此行決平江北,卿等還當陸來也。」五年,王師征淮,正月克靜海軍,而日就等果陸還。世宗已平淮南,遣使賜俶兵甲旗幟、橐駝羊馬。



 錢氏兼有兩浙幾百年,其人比諸國號為怯弱,而俗喜淫侈,偷生工巧,自鏐世常重斂其民以事奢僭,下至雞魚卵鷇,必家至而日取。每笞一人以責其負,則諸案史各持其簿列于廷;凡一簿所負,唱其多少,量為笞數,以次唱而笞之,少者猶積數十,多者至笞百餘,人尤不勝其苦。又多掠得嶺海商賈寶貨。當五
 代時,常貢奉中國不絕。及世宗平淮南,宋興,荊、楚諸國相次歸命,俶勢益孤,始傾其國以事貢獻。太祖皇帝時,俶嘗來朝,厚禮遣還國,俶喜,益以器服珍奇為獻,不可勝數。



 太祖曰:「此吾帑中物爾,何用獻為!」太平興國三年,詔俶來朝,俶舉族歸于京師,國除。其後事具國史。



 嗚呼!天人之際,為難言也。非徒自古術者好奇而幸中,至於英豪草竊亦多自託於妖祥,豈其欺惑愚眾,有以用之歟?蓋其興也,非有功德漸積之勤,而黥髡盜販,倔起於王侯,而人亦樂為之傳歟?考錢氏之始終,非有德澤施其一方,百年之際,虐用其人甚矣,其動於氣象
 者,豈非其孽歟?是時四海分裂,不勝其暴,又豈皆然歟?是皆無所得而推歟?術者之言,不中者多,而中者少,而人特喜道其中者歟?



 鏐世興滅,諸書皆同,蓋自唐乾寧二年為鎮海、鎮東軍節度使,兼有兩浙,至皇朝太平興國三年國除,凡八十四年。



\end{pinyinscope}