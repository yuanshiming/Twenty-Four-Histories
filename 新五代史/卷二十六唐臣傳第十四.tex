\article{卷二十六唐臣傳第十四}

\begin{pinyinscope}

 符習符習,趙州昭慶人也。少事趙王王鎔為軍校,自晉救趙,破梁軍柏鄉,趙常遣習將兵從晉。晉軍德勝,張文禮弒趙王王鎔,上書莊宗,求習歸趙。莊宗遣之,習號泣曰:「臣世家趙,受趙王恩,王嘗以一劍與臣使自效,今聞王死,欲以劍自裁,念卒無益,請擊趙破賊,報王冤。」莊宗壯之,乃遣閻寶、史建瑭等助習討文禮,以習為鎮州兵馬留後。習攻文禮不克,莊宗用佗將破之。拜習成德軍節度
 使,習辭不敢受,乃以相、衛二州為義寧軍,以習為節度使,習辭曰:「魏博六州,霸王之府也,不宜分割以示弱,願授臣河南一鎮,得自攻取之。」乃拜習天平軍節度使、東南面招討使,習亦未嘗攻取。後徙鎮安國,又徙平盧。



 趙在禮作亂,遣習以鎮兵討賊。習未至魏,而明宗兵變,習不敢進。明宗遣人招之,習見明宗於胙縣,而以明宗舉兵不順,去就之意未決,霍彥威紿習曰:「主上所殺者十人,公居其四,復何猶豫乎?」習意乃決。平盧監軍楊希望聞習為明宗所召,乃以兵圍習家屬,將殺之。指揮使王公儼素為希望所信,紿希望曰:「內侍盡忠朝廷,誅反者
 家族,孰敢不效命!宜分兵守城,以虞外變,習家不足慮也。」



 希望信之,乃悉分其兵守城,公儼因擒希望斬之,習家屬由是獲免。而公儼宣言青人不便習之嚴急,不欲習復來,因自求為節度使。明宗乃以房知溫代習鎮平盧,拜公儼登州刺史。公儼不時承命,知溫擒而殺之。習復鎮天平,徙鎮宣武。



 習素為安重誨所不悅,希其旨者上言習厚斂汴人,乃以太子太師致仕,歸昭慶故里,明宗以其子令謙為趙州刺史以奉養之。習以無罪,怏怏失職,縱獵劇飲以自娛。居歲餘,中風卒,贈太師。



 習二子:令謙、蒙。令謙有勇力,善騎射,以父任為將,官至趙州刺
 史,有善政,卒于州,州人號泣送葬者數千人,當時號為良刺史。蒙少好學,性剛鯁,為成德軍節度副使。後事晉,官至禮部侍郎。



 烏震烏震,冀州信都人也。少事趙王王鎔為軍卒,稍以功遷裨校,隸符習軍。習從莊宗于河上,而鎔為張文禮所弒,震從習討文禮,而家在趙,文禮執震母妻及子十餘人以招震,震不顧。文禮乃皆斷其手鼻,割而不誅,縱至習軍,軍中皆不忍正視。



 震一慟而止,憤激自勵,身先士卒。晉軍攻破鎮州,震以功拜刺史,歷深、趙二州。



 震為人純質,少好學,通《左氏春秋》,喜作詩,善書。及為刺史,以廉平
 為政有聲,遷冀州刺史,兼北面水陸轉運使。明宗聞其名,擢拜河北道副招討使,領寧國軍節度使,代房知溫戍于盧臺軍。始至而戍兵龍晊等作亂,見殺,贈太師。



 嗚呼!忠孝以義則兩得,吾既已言之矣,若烏震者,可謂忠乎?甚矣,震之不思也。夫食人之祿而任人之事,事有任,專其責,而其國之利害,由己之為不為,為之雖利於國,而有害於其親者,猶將辭其祿而去之。矧其事眾人所皆可為,而任不專己,又其為與不為,國之利害不繫焉者,如是而不顧其親,雖不以為利,猶曰不孝,況因而利之乎!夫能事其親以孝,然後能事其君以忠,若烏震
 者,可謂大不孝矣,尚何有於忠哉!



 孔謙孔謙,魏州人也,為魏州孔目官。魏博入于晉,莊宗以為度支使。謙為人勤敏,而傾巧善事人,莊宗及其左右皆悅之。自少為吏,工書算,頗知金穀聚斂之事。晉與梁相拒河上十餘年,大小百餘戰,謙調發供饋,未嘗闕乏,所以成莊宗之業者,謙之力為多,然民亦不勝其苦也。



 莊宗初建大號,謙自謂當為租庸使,而郭崇韜用魏博觀察使判官張憲為使,以謙為副。謙已怏怏。既而莊宗滅梁,謙從入汴,謂崇韜曰:「鄴,北都也,宜得重人鎮之,非張憲不可。」崇韜以為然,因以憲留守北都,而以宰相豆盧革
 判租庸。



 謙益失望,乃陰求革過失,而革嘗以手書假租庸錢十萬,謙因以書示崇韜,而微泄其事,使革聞之。革懼,遂求解職以讓崇韜,崇韜亦不肯當。莊宗問:「誰可者?」



 崇韜曰:「孔謙雖長於金穀,而物議未可居大任,不若復用張憲。」乃趣召憲。憲為人明辯,人頗忌之,謙因乘間謂革曰:「租庸錢穀,悉在目前,委一小吏可辦。



 鄴都天下之重,不可輕以任人。」革以語崇韜,崇韜罷憲不召,以興唐尹王正言為租庸使。謙益憤憤,因求解職。莊宗怒其避事,欲寘之法,賴伶官景進救解之,乃止。已而正言病風,不任事,景進數以為言,乃罷正言,以謙為租庸使,賜「豐
 財贍國功臣」。



 謙無佗能,直以聚斂為事。莊宗初即位,推恩天下,除百姓田租,放諸場務課利欠負者,謙悉違詔督理。故事:觀察使所治屬州事,皆不得奪達,上所賦調,亦下觀察使行之。而謙直以租庸帖調發諸州,不關觀察,觀察使交章論理,以謂:「制敕不下支郡,刺史不專奏事,唐制也。租庸直帖,沿偽梁之弊,不可為法。今唐運中興,願還舊制。」詔從其請,而謙不奉詔,卒行直帖。又請減百官俸錢,省罷節度觀察判官、推官等員數。以至鄣塞天下山谷徑路,禁止行人,以收商旅征算;遣大程官放豬羊柴炭,占庇人戶;更制括田竿尺;盡率州使公廨錢。
 由是天下皆怨苦之。明宗立,下詔暴謙罪,斬于洛市,籍沒其家。遂罷租庸使額,分鹽鐵、度支、戶部為三司。



 張延朗張延朗,汴州開封人也。事梁,以租庸吏為鄆州糧料使。明宗克鄆州,得延朗,復以為糧料使,後徙鎮宣武、成德,以為元從孔目官。明宗即位,為莊宅使、宣徽北院使、忠武軍節度使。長興元年,拜三司使。唐制:戶部度支以本司郎中、侍郎判其事,而有鹽鐵轉運使。其後用兵,以國計為重,遂以宰相領其職。乾符已後,天下喪亂,國用愈空,始置租庸使,用兵無常,隨時調斂,兵罷則止。梁興,始置租庸使,領天下錢穀,廢鹽鐵、戶部、度支之官。莊宗滅
 梁,因而不改。明宗入立,誅租庸使孔謙而廢其使職,以大臣一人判戶部、度支、鹽鐵,號曰判三司。延朗因請置三司使,事下中書。中書用唐故事,拜延朗特進、工部尚書,充諸道鹽鐵轉運等使,兼判戶部度支事。詔以延朗充三司使,班在宣徽使下。三司置使自此始。



 延朗號為有心計,以三司為己任,而天下錢穀亦無所建明。明宗常出遊幸,召延朗共食,延朗不至,附使者報曰:「三司事忙,無暇。」聞者笑之。歷泰寧、雄武軍節度使。廢帝以為吏部尚書兼中書門下平章事,判三司。



 晉高祖有異志,三司財貨在太原者,延朗悉調取之,高祖深以為恨。晉兵
 起,廢帝欲親征,而心畏高祖,遲疑不決,延朗與劉延朗等勸帝必行。延朗籍諸道民為丁及括其馬,丁馬未至,晉兵入京師,高祖得延朗,殺之。



 李嚴李嚴,幽州人也,初名讓坤。事劉守光為刺史,後事莊宗為客省使。嚴為人明敏多藝能,習騎射,頗知書而辯。同光三年,使于蜀,為王衍陳唐興復功德之盛,音辭清亮,蜀人聽之皆竦動。衍樞密使宋光嗣召嚴置酒,從容問中國事。嚴對曰:「前年天子建大號于鄴宮,自鄆趨汴,定天下不旬日,而梁之降兵猶三十萬,東漸于海,西極甘涼,北懾幽陵,南踰閩嶺,四方萬里,莫不臣妾。而淮南楊
 氏承累世之彊,鳳翔李公恃先朝之舊,皆遣子入侍,稽首稱籓。至荊、湖、吳越,修貢賦,效珍奇,願自比於列郡者,至無虛月。天子方懷之以德,而震之以威,天下之勢,不得不一也。」光嗣曰:「荊、湖、吳越非吾所知,若鳳翔則蜀之姻親也,其人反覆,其可信乎?又聞契丹日益彊盛,大國其可無慮乎?」嚴曰:「契丹之彊,孰與偽梁?」光嗣曰:「比梁差劣爾!」嚴曰:「唐滅梁如拉朽,況其不及乎!唐兵布天下,發一鎮之眾,可以滅虜使無類。然而天生四夷,不在九州之內,自前古王者,皆存而不論,蓋不欲窮兵黷武也。」蜀人聞嚴應對,愈益奇之。



 是時,蜀之君臣皆庸暗,而恃
 險自安,窮極奢僭。嚴自蜀還,具言可取之狀。



 初,莊宗遣嚴以名馬入蜀,市珍奇以充後宮,而蜀法嚴禁以奇貨出劍門,其非奇物而出者,名曰「入草物」,由是嚴無所得而還,惟得金二百兩、地衣、毛布之類。



 莊宗聞之,大怒曰:「物歸中國,謂之『入草』,王衍其能免為『入草人』乎?」



 於是決議伐蜀。



 冬,魏王繼岌西伐,以嚴為三川招討使,與康延孝以兵五千先行,所過州縣皆迎降。延孝至漢州,王衍告曰:「得李嚴來即降。」眾皆以伐蜀之謀自嚴始,而衍怨嚴深,不宜往。嚴聞之喜,即馳騎入益州。衍見嚴,以妻母為託,即日以蜀降。



 嚴還,明宗以為泗州防禦使,客省使
 如故。



 其後孟知祥屈彊於蜀,安重誨稍裁抑之,思有以制知祥者,嚴乃求為西川兵馬都監。將行,其母曰:「汝前啟破蜀之謀,今行,其以死報蜀人矣!」嚴不聽。初,嚴與知祥同事莊宗,時知祥為中門使,嚴嘗有過,莊宗怒甚,命斬之,知祥戒行刑者少緩,入白莊宗曰:「嚴小過,不宜以喜怒殺人,恐失士大夫心。」莊宗怒稍解,命知祥監笞嚴二十而釋之。知祥雖與嚴有舊恩,而惡其來。蜀人聞嚴來,亦皆惡之。



 嚴至,知祥置酒從容問嚴曰:「朝廷以公來邪?公意自欲來邪?」嚴曰:「君命也。」



 知祥發怒曰:「天下籓鎮皆無監軍,安得爾獨來此?此乃孺子熒惑朝廷爾!」即擒
 斬之,明宗不能詰也,知祥由此遂反。



 李仁矩李仁矩,不知其世家。少事明宗為客將,明宗即位,以為客省使、左衛大將軍。



 明宗祀天南郊,東、西川當進助禮錢,使仁矩趣之。仁矩恃恩驕恣,見籓臣不以禮。



 東川節度使董璋置酒召仁矩,仁矩辭醉不往,於傳舍與倡妓飲。璋怒,率衙兵露刃之傳舍,仁矩惶恐,不襪而靴走庭中,璋責之曰:「爾以西川能斬李嚴,謂我獨不能斬爾邪!」顧左右牽出斬之。仁矩涕泣拜伏謝罪,乃止。明日,璋置酒召仁矩,見其妻子,以厚謝之。仁矩還,言璋必反。仁矩素為安重誨所親信,自璋有異志,重誨思有以制之,乃
 分東川之閬州為保寧軍,以仁矩為節度使,遣姚洪將兵戍之。



 璋以書至京師告其子光業曰:「朝廷割我支郡,分建節髦,又以兵戍之,是將殺我也。若唐復遣一騎入斜谷,吾反必矣!與汝自此而決。」光業私以書示樞密承旨李虔徽,使白重誨,重誨不省。仁矩至鎮,伺璋動靜必以聞,璋益疑懼,遂決反。重誨又遣荀咸乂將兵益戍閬州,光業亟言以為不可,重誨不聽。咸乂未至,璋已反,攻閬州,仁矩召將校問策,皆曰:「璋有二心久矣,常以利啖吾兵,兵未可用,而賊鋒方銳,宜堅壁以挫之。守旬日,大軍必至,賊當自退。」仁矩曰:「蜀懦,安能當我精銳之師!」
 即驅之出戰,兵未交而潰,仁矩被擒,並其家屬皆見殺。



 毛璋毛璋,滄州人也。梁末,戴思遠為橫海軍節度使,璋事思遠為軍校。晉已下魏博,思遠棄滄州出奔,璋以滄州降晉,以功為貝州刺史。璋為人有膽勇,自晉與梁相拒河上,璋累戰有功。莊宗滅梁,拜璋華州節度使。在鎮多不法,議者疑其有異志,乃徙璋鎮昭義。璋初欲拒命,其判官邊蔚切諫諭之,乃聽命。璋累歷籓鎮,又在華州得魏王繼岌伐蜀餘貲,既富而驕,益為淫侈。嘗服赭袍飲酒,使其所得蜀奴為王衍宮中之戲於前。明宗聞而惡之,召為金吾上將軍。東川董璋上書言璋遣子廷贇持書
 往西川,疑其有姦。明宗乃遣人追還廷贇,並璋下御史獄。廷贇款稱實璋假子,有叔父在蜀,欲往省之,而無私書。璋無罪名,有司議:「璋前任籓鎮,陰畜異圖,及處班行,不慎行止。」乃停璋見任官,勒還私第。



 初,廷贇之蜀,與其客趙延祚俱,及召下獄,延祚多捃璋陰事欲言之,璋許延祚重賂以滅口。既出而責賂於璋,不與,延祚乃詣臺自言,並璋復下獄,鞫之無狀。



 中丞呂夢奇議曰:「璋前經推劾,已蒙昭雪,而延祚以責賂之故,復加織羅。」乃稍宥璋。璋款上,有告者言夢奇受賂而劾獄不盡,乃移軍巡獄。獄吏希旨,鍛煉其事,璋具伏:許賂延祚而未與,嘗
 以馬借夢奇而無受賂。璋坐長流儒州,已而令所在賜自盡。



\end{pinyinscope}