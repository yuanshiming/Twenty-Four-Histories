\article{卷十二周本紀第十二}

\begin{pinyinscope}

 世宗睿武孝文皇帝,本姓柴氏,邢州龍岡人也。柴氏女適太祖,是為聖穆皇后。



 后兄守禮子榮,幼從姑長太祖家,以謹厚見愛,太祖遂以為子。太祖後稍貴,榮亦壯,而器貌英奇,善騎射,略通書史黃老,性沈重寡言。太祖為漢樞密使,榮為左監門衛大將軍。太祖鎮天雄,榮領貴州刺史、天雄軍牙內都指揮使。



 乾祐三年冬,周兵起魏,犯京師,留榮守魏。太祖入立,拜澶州刺史、鎮寧軍節度使,
 檢校太傅、同中書門下平章事。榮素為樞密使王峻所忌,廣順三年正月來朝,不得留。既而峻有罪誅,三月,拜榮開封尹,封晉王。是冬,卜以來年正月朔旦有事于南郊,而太祖遇疾,不能視朝者久之。



 顯德元年正月丙子,郊,僅而成禮,即以王判內外兵馬事。壬辰,太祖崩,秘不發喪。丙申,發喪,皇帝即位於柩前。右監門衛大將軍魏仁浦為樞密副使。二月庚戌,回鶻遣使者來。丁卯,馮道為大行皇帝山陵使,太常卿田敏為禮儀使,兵部尚書張昭為鹵簿使,御史中丞張煦為儀仗使,開封少
 尹權判府事王敏為橋道頓遞使。



 漢人來討,攻自潞州。三月辛巳,大赦。癸未,鄭仁誨留守東京。乙酉,如潞州以攻漢。壬辰,次澤州,閱兵于北郊。癸巳,及劉旻戰于高原,敗之,追及于高平,又敗之。丁酉,幸潞州。己亥,侍衛馬軍都指揮使樊愛能、步軍都指揮使何徽伏誅。



 壬寅,天雄軍節度使符彥卿為河東行營都部署。夏四月乙卯,葬神聖文武恭肅孝皇帝於嵩陵。汾州防禦使董希顏叛于漢來附。丙辰,遼州刺史張漢超叛於漢來附。辛酉,取嵐、憲州。壬戌,立衛國夫人符氏為皇后。取石、泌州。乙丑,
 馮道薨。庚午,赦潞州流罪以下囚。如太原。忻州監軍李勍殺其刺史趙皋,叛于漢來附。五月丙子,代州守將鄭處謙叛于漢來附,契丹救漢。丁酉,回鶻使因難敵略來。符彥卿及契丹戰于忻口,敗績,先鋒都指揮使史彥超死之。六月乙巳,班師。乙丑,次新鄭,前拜嵩陵。庚午,至自太原。秋七月庚辰,閱稼于南御莊。癸巳,樞密院直學士、工部侍郎景範為中書侍郎、同中書門下平章事,魏仁浦為樞密使。冬十月甲辰,殺左羽林大將軍孟漢卿。



 二年春二月,
 御札求直言。夏五月辛未,宣徽南院使向訓、鳳翔節度使王景伐蜀。甲戌,大毀佛寺,禁民親無侍養而為僧尼及私自度者。秋九月丙寅朔,頒銅禁。



 閏月癸丑,向訓克秦州。冬十月辛未,取成州。戊寅,高麗使王子太相融來。取階州。十一月乙未朔,李穀為淮南道行營都部署以伐唐。戊申,王景克鳳州。十二月丙戌,鄭仁誨薨。



 三年春正月,增築京城。庚子,向訓留守東京。壬寅,南征。辛亥,侍衛親軍都指揮使李重進及唐人戰于正陽,敗之。甲寅,重進為淮南道行營都招討使。二月丙寅,幸下蔡浮橋。壬申,克滁州。甲戌,李景來求成,不答。壬午,景使
 其臣鐘謨來奉表。丙戌,取揚州。辛卯,取泰州。三月庚子,內外馬步軍都軍頭袁彥為竹龍都部署。是月,取光、舒、常州。夏四月,常、泰州復入於唐。五月乙卯,至自淮南,赦京師囚。六月壬申,德音赦淮南囚。秋七月,皇后崩。揚、光、舒、滁州復入于唐。八月乙丑,課民種禾及韭。九月丙午,端明殿學士、左散騎常侍王朴為尚書戶部侍郎、樞密副使。冬十月辛酉,葬宣懿皇后于懿陵。十一月庚寅,廢諸祠不在祀典者。乙巳,殺李景之臣孫晟。



 四年春正月己丑朔,赦非死罪囚。二月甲戌,王朴留守
 東京。乙亥,南征。三月丁未,克壽州。夏四月己巳,至自壽州。己卯,放降卒八百歸于蜀。癸未,追冊彭城郡夫人劉氏為皇后。五月丙申,殺密州防禦使侯希進。秋八月乙亥,李穀罷,王朴為樞密使。癸未,蜀人來歸我濮州刺史胡立。冬十月己巳,王朴留守東京,三司使張美為大內都點檢。壬申,南征。十二月乙卯,泗州守將范再遇叛於唐,以其州來降。庚申,濠州團練使郭廷謂以其州來降。丁丑,取泰州。



 五年春正月丁亥,取海州。壬辰,取靜海軍。丁未,克楚州,守將張彥卿、鄭昭業死之。二月甲寅,取雄州。丁卯,如揚州。癸酉,如瓜洲。三月壬午朔,如泰州。丁亥,復如揚州。辛卯,幸迎鑾。己亥,克淮南十有四州,以江為界。三月辛亥,李景來買宴。四月庚申,祔五室神主于新廟。壬申,至自淮南,回鶻、達靼遣使者來。六月辛未,放降卒四千六百于唐。秋七月乙酉,水部員外郎韓彥卿市銅于高麗。丁亥,頒《均田圖》。九月,占城國王釋利因德
 縵使莆訶散來。冬十月丁酉,括民租。十一月庚戌,作《通禮》、《正樂》。十二月丙戌,罷州縣課戶、俸戶。



 六年春正月,高麗王昭遣使者來。辛酉,女真使阿辨來。三月己酉,甘州回鶻來獻玉,卻之。庚申,王朴薨。丙寅,宣徽南院使吳延祚留守東京。癸酉,停給銅魚。甲戌,北征。是月,吳延祚為左驍衛上將軍、樞密使。夏四月壬辰,取乾寧軍。



 辛丑,取益津關,以為霸州。癸卯,取瓦橋關,以為雄州。五月乙巳朔,取瀛州。



 甲戌,至自雄州。六月癸未,立皇后符氏,封子宗
 訓為梁王、宗誼燕國公。戊子,占城使莆訶散來。己丑,范質、王溥參知樞密院事,魏仁浦同中書門下平章事。癸巳,皇帝崩于滋德殿。



 恭皇帝,世宗第四子宗訓也。世宗即位,大臣請封皇子為王,世宗謙抑久之。



 及北取三關,遇疾還京師,始封宗訓梁王,時年七歲。



 顯德六年六月癸巳,世宗崩。甲午,皇帝即位於柩前。癸卯,范質為大行皇帝山陵使,翰林學士竇儼為禮儀使,兵部尚書張昭為鹵簿使,御史中丞邊歸讜為儀仗使,
 宣徽南院使,判開封府事昝居潤為橋道頓遞使。秋七月丁未,戶部尚書李濤為山陵副使,度支郎中盧億為判官。八月庚寅,封弟熙讓為曹王,熙謹紀王,熙誨蘄王。壬寅,高麗遣使者來。九月丙寅,左驍衛大將軍戴交使于高麗。冬十一月壬寅,葬睿武孝文皇帝于慶陵。高麗遣使者來。



 七年春正月甲辰,遜于位。宋興。



 嗚呼,五代本紀備矣,君臣之際,可勝道哉!梁之友珪反,唐戕克寧而殺存乂、從璨,則父子骨肉之恩,幾何其不絕矣。太妃薨而輟朝,立劉氏、馮氏為皇后,則夫婦之倫幾何其不乖而不至於禽獸矣。寒食野祭而焚紙錢,居喪改元而用樂,殺馬延及任圜,則禮樂刑政幾何其不壞矣。至於賽雷山、傳箭而撲馬,則中國幾何其不夷狄矣。可謂亂世也歟!而世宗區區五六年間,取秦隴,平淮右,復三關,威武之聲震懾夷夏,而方內延儒學文章之士,考制度、修《通禮》、定《正樂》、議《刑統》,其制作
 之法皆可施於後世。其為人明達英果,論議偉然。即位之明年,廢天下佛寺三千三百三十六。是時中國乏錢,乃詔悉毀天下銅佛像以鑄錢,嘗曰:「吾聞佛說以身世為妄,而以利人為急,使其真身尚在,茍利於世,猶欲割截,況此銅像,豈其所惜哉?」由是群臣皆不敢言。嘗夜讀書,見唐元稹《均田圖》,慨然歎曰:「此致治之本也,王者之政自此始!」乃詔頒其圖法,使吏民先習知之,期以一歲,大均天下之田,其規為志意豈小哉!其伐南唐,問宰相李穀以計策;後克淮南,出穀疏,使學士陶穀為贊,而盛以錦囊,嘗置之坐側。其英武之材可謂雄傑,及其虛心
 聽納,用人不疑,豈非所謂賢主哉!其北取三關,兵不血刃,而史家猶譏其輕社稷之重,而僥幸一勝於倉卒,殊不知其料強弱、較彼我而乘述律之殆,得不可失之機,此非明於決勝者,孰能至哉?誠非史氏之所及也!



\end{pinyinscope}