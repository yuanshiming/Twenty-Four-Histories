\article{卷二十九晉臣傳第十七}

\begin{pinyinscope}

 桑維翰桑維翰,字國僑,河南人也。為人醜怪,身短而面長,常臨鑒以自奇曰:「七尺之身,不如一尺之面。」慨然有志於公輔。初舉進士,主司惡其姓,以「桑」



 「喪」同音。人有勸其不必舉進士,可以從佗求仕者,維翰慨然,乃著《日出扶桑賦》以見志。又鑄鐵硯以示人曰:「硯弊則改而佗仕。」卒以進士及第。晉高祖辟為河陽節度掌書記,其後常以自從。



 高祖自太原徙天平,不受命,而有異謀,以問將佐,將
 佐皆恐懼不敢言,獨維翰與劉知遠贊成之,因使維翰為書求援於契丹。耶律德光已許諾,而趙德鈞亦以重賂啖德光,求助己以篡唐。高祖懼事不果,乃遣維翰往見德光,為陳利害甚辯,德光意乃決,卒以滅唐而興晉,維翰之力也。高祖即位,以維翰為翰林學士、禮部侍郎、知樞密院事,遷中書侍郎、同中書門下平章事,兼樞密使。天福四年,出為相州節度使,歲餘,徙鎮泰寧。



 吐渾白承福為契丹所迫,附鎮州安重榮以歸晉,重榮因請與契丹絕好,用吐渾以攻之。高祖重違重榮,意未決。維翰上疏言契丹未可與爭者七,高祖召維翰使者至臥內,
 謂曰:「北面之事,方撓吾胸中,得卿此疏,計已決矣,可無憂也。」維翰又勸高祖幸鄴都。七年,高祖在鄴,維翰來朝,徙鎮晉昌。



 出帝即位,召拜侍中。而景延廣用事,與契丹絕盟,維翰言不能入,乃陰使人說帝曰:「制契丹而安天下,非用維翰不可。」乃出延廣於河南,拜維翰中書令,復為樞密使,封魏國公,事無巨細,一以委之。數月之間,百度浸理。初,李瀚為翰林學士,好飲而多酒過,高祖以為浮薄。天福五年九月,詔廢翰林學士,按《唐六典》歸其職於中書舍人,而端明殿學士、樞密院學士皆廢。及維翰為樞密使,復奏置學士,而悉用親舊為之。



 維翰權勢既
 盛,四方賂遺,歲積巨萬。內客省使李彥韜、端明殿學士馮玉用事,共讒之。帝欲驟黜維翰,大臣劉昫、李崧皆以為不可,卒以玉為樞密使,既而以為相,維翰日益見疏。帝飲酒過度得疾,維翰遣人陰白太后,請為皇弟重睿置師傅。



 帝疾愈,知之,怒,乃罷維翰以為開封尹。維翰遂稱足疾,稀復朝見。



 契丹屯中渡,破欒城,杜重威等大軍隔絕,維翰曰:「事急矣!」乃見馮玉等計事,而謀不合。又求見帝,帝方調鷹於苑中,不暇見,維翰退而嘆曰:「晉不血食矣!」



 自契丹與晉盟,始成於維翰,而終敗於景延廣,故自兵興,契丹凡所書檄,未嘗不以此兩人為言。耶律德
 光犯京師,遣張彥澤遺太后書,問此兩人在否,可使先來。而帝以繼翰嘗議毋絕盟而己違之也,不欲使維翰見德光,因諷彥澤圖之,而彥澤亦利其貲產。維翰狀貌既異,素以威嚴自持,晉之老將大臣,見者無不屈服,彥澤以驍捍自矜,每往候之,雖冬月未嘗不流汗。初,彥澤入京師,左右勸維翰避禍,維翰曰:「吾為大臣,國家至此,安所逃死邪!」安坐府中不動。彥澤以兵入,問:「維翰何在?」維翰厲聲曰:「吾,晉大臣,自當死國,安得無禮邪!」彥澤股栗不敢仰視,退而謂人曰:「吾不知桑維翰何如人,今日見之,猶使人恐懼如此,其可再見乎?」乃以帝命召維
 翰。維翰行,遇李崧,立馬而語,軍吏前白維翰,請赴侍衛司獄。維翰知不免,顧崧曰:「相公當國,使維翰獨死?」崧慚不能對。是夜,彥澤使人縊殺之,以帛加頸,告德光曰:「維翰自縊。」德光曰:「我本無心殺維翰,維翰何必自致。」德光至京師,使人檢其尸,信為縊死,乃以尸賜其家,而貲財悉為彥澤所掠。



 景延廣景延廣,字航川,陜州人也。父建善射,嘗教延廣曰:「射不入鐵,不如不發。」



 由是延廣以挽彊見稱。事梁邵王友誨,友誨謀反被幽,延廣亡去。後從王彥章戰中都,彥章敗,延廣身被數創,僅以身免。



 明宗時,朱守殷以汴州反,晉
 高祖為六軍副使,主誅從守殷反者。延廣為汴州軍校當誅,高祖惜其才,陰縱之使亡,後錄以為客將。高祖即位,以為侍衛步軍都指揮使,領果州團練使,從領寧江軍節度使。天福四年,出鎮義成,又徙保義,復召為侍衛馬步軍都虞候,徙鎮河陽三城,遷馬步軍都指揮使,領天平。



 高祖崩,出帝立,延廣有力,頗伐其功。初,出帝立,晉大臣議告契丹,致表稱臣,延廣獨不肯,但致書稱孫而已,大臣皆知其不可而不能奪。契丹果怒,數以責晉,延廣謂契丹使者喬瑩曰:「先皇帝北朝所立,今衛子中國自冊,可以為孫,而不可為臣。且晉有橫磨大劍十萬口,
 翁要戰則來,佗日不禁孫子,取笑天下。」



 瑩知其言必起兩國之爭,懼後無以取信也,因請載于紙,以備遺忘。延廣敕吏具載以授瑩,瑩藏其書衣領中以歸,具以延廣語告契丹,契丹益怒。



 天福八年秋,出帝幸大年莊還,置酒延廣第。延廣所進器服、鞍馬、茶床、椅榻皆裹金銀,飾以龍鳳。又進帛五千匹,綿一千四百兩,馬二十二匹,玉鞍、衣襲、犀玉、金帶等,請賜從官,自皇弟重睿,下至伴食刺史、重睿從者各有差。帝亦賜延廣及其母、妻、從事、押衙、孔目官等稱是。時天下旱、蝗,民餓死者歲十數萬,而君臣窮極奢侈以相誇尚如此。



 明年春,契丹入寇,延廣
 從出帝北徵為御營使,相拒澶、魏之間。先鋒石公霸遇虜於戚城,高行周、符彥卿兵少不能救,馳騎促延廣益兵,延廣按兵不動。三將被圍數重,帝自御軍救之,三將得出,皆泣訴。然延廣方握親兵,恃功恣橫,諸將皆由其節度,帝亦不能制也。契丹嘗呼晉人曰:「景延廣喚我來,何不速戰?」是時,諸將皆力戰,而延廣未嘗見敵。契丹已去,延廣獨閉壁不敢出。自延廣一言而契丹與晉交惡,凡號令征伐一出延廣,晉大臣皆不得與,故契丹凡所書檄,未嘗不以延廣為言。契丹去,出帝還京師,乃出延廣為河南尹,留守西京。明年,出帝幸澶淵,以延廣從,皆
 無功。



 延廣居洛陽,鬱鬱不得志。見晉日削,度必不能支契丹,乃為長夜之飲,大治第宅,園置妓樂,惟意所為。後帝亦追悔,遣供奉官張暉奉表稱臣以求和,德光報曰:「使桑維翰、景延廣來,而割鎮、定與我,乃可和。」晉知其不可,乃止。契丹至中渡,延廣屯河陽,聞杜重威降,乃還。



 德光犯京師,行至相州,遣騎兵數千雜晉軍渡河趨洛,以取延廣,戒曰:「延廣南奔吳,西走蜀,必追而取之。」而延廣顧慮其家,未能引決,虜騎奄至,乃與從事閻丕馳騎見德光於封丘,并丕見鎖。延廣曰:「丕,臣從事也,以職相隨,何罪而見鎖?」丕乃得釋。德光責延廣曰:「南北失懽,皆因
 爾也。」召喬瑩質其前言,延廣初不服,瑩從衣領中出所藏書,延廣乃服。因以十事責延廣,每服一事,授一牙籌,授至八籌,延廣以面伏地,不能仰視,遂叱而鎖之。將送之北行,至陳橋,止民家。夜分,延廣伺守者殆,引手扼吭而死,時年五十六。漢高祖時,贈侍中。



 嗚呼,自古禍福成敗之理,未有如晉氏之明驗也!其始以契丹而興,終為契丹所滅。然方其以逆抗順,大事未集,孤城被圍,外無救援,而徒將一介之命,持片舌之彊,能使契丹空國興師,應若符契,出危解難,遂成晉氏,當是之時,維翰之力為多。及少主新立,釁結兵連,敗約起
 爭,發自延廣。然則晉氏之事,維翰成之,延廣壞之,二人之用心者異,而其受禍也同,其故何哉?蓋夫本末不順而與夷狄共事者,常見其禍,未見其福也。可不戒哉!可不戒哉!



 吳巒吳巒,字寶川,鄆州盧縣人也。少舉明經不中,清泰中為大同沙彥珣節度判官。



 晉高祖起太原,召契丹為援,契丹過雲州,彥珣出城迎謁,為契丹所虜。城中推巒主州事,巒即閉門拒守,契丹以兵圍之。高祖入立,以雲州入于契丹,而巒猶守城不下,契丹圍之凡七月。高祖義巒所為,乃以書告契丹,使解兵去。高祖召巒,以為武寧軍
 節度副使、諫議大夫、復州防禦使。



 出帝即位,與契丹絕盟,河北諸州皆警,以謂貝州水陸之衝,緩急可以轉餉,乃積芻粟數十萬,以王令溫為永清軍節度使。令溫牙將邵珂,素驕很難制,令溫奪其職。珂閑居無憀,乃陰使人亡入契丹,言貝州積粟多而無兵守,可取。令溫以事朝京師,心頗疑珂,乃質其子崇範以自隨。晉大臣以巒前守雲州七月,契丹不能下,乃遣巒馳驛代令溫守貝州。巒善撫士卒,會天大寒,裂其帷幄以衣士卒,士卒皆愛之。珂因求見蠻,願自效,巒推心信之。開運元年正月,契丹南寇,圍貝州,巒命珂守南門。契丹圍三日,四面急
 攻之,巒從城上投薪草焚其梯沖殆盡。已而珂自南門引契丹入,巒守東門方戰,而左右報珂反,巒顧城中已亂,即投井死。而令溫家屬為契丹所虜,出帝憫之,以令溫為武勝軍節度使,後累歷方鎮,周顯德中卒。令溫,瀛州河間人也。



\end{pinyinscope}