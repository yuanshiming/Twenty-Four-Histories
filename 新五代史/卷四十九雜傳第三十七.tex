\article{卷四十九雜傳第三十七}

\begin{pinyinscope}

 翟光鄴
 翟光鄴,字化基,濮州鄄城人也。其父景珂,倜儻有膽氣。梁、晉相距于河上,景珂率聚邑人守永定驛,晉人攻之,踰年不能下,景珂卒戰死。光鄴時年十歲,為晉兵所掠,明宗愛其穎悟,常以自隨。光鄴事唐,官至耀州團練使。晉高祖時,歷棣沂二州刺史、西京副留守。出帝已破楊光遠,以光鄴為青州防禦使。光鄴招輯兵民,甚有恩意。契丹滅晉,遣光鄴知曹州。許王從益入汴,以為樞密使。
 漢高祖入京師,改右領軍衛大將軍、左金吾大將軍,充街使。周太祖入立,拜宣徽使、樞密副使,出知永興軍,卒于官。



 光鄴為人沈默多謀,事繼母以孝聞。雖貴,不營財產,常假官舍以居,蕭然僅蔽風雨。雍睦親族,粗衣糲食,與均有無,而光鄴處之晏然,日與賓客飲酒聚書為樂。其所臨政,務以寬靜休息為意。病亟,戒其左右,氣絕以尸歸洛,無久留以煩軍府。既卒,州人上書乞留葬立祠,不許。



 馮暉馮暉,魏州人也。為效節軍卒,以功遷隊長。唐莊宗入魏,與梁相距於河上,暉以隊長亡入梁軍,王彥章以暉驍
 勇,隸之麾下。梁亡,莊宗赦暉不問。從明宗討楊立、魏王繼岌平蜀,累遷夔、興二州刺史。董璋反東川,暉從晉高祖討璋,軍至劍門,劍門兵守,不得入,暉從佗道出其左,擊蜀守兵殆盡。會晉高祖班師,拜暉澶州刺史。



 天福中,範延光反魏州,遣暉襲滑州,不克,遂入於魏,為延光守。已而出降,拜義成軍節度使,徙鎮靈武。靈武自唐明宗已後,市馬糴粟,招來部族,給賜軍士,歲用度支錢六千萬,自關以西,轉輸供給,民不堪役,而流亡甚眾。青岡、土橋之間,氐、羌剽掠道路,商旅行必以兵。暉始至,則推以恩信,部族懷惠,止息侵奪,然後廣屯田以省轉餉,治倉
 庫、亭館千餘區,多出俸錢,民不加賦,管內大治,晉高祖下詔書褒美。



 黨項拓拔彥超最為大族,諸族嚮背常以彥超為去就。暉之至也,彥超來謁,遂留之,為起第於城中,賜予豐厚,務足其意。彥超既留,而諸部族爭以羊馬為市易,期年有馬五千匹。晉見暉馬多而得夷心,反以為患,徙鎮靜難,又徙保義。歲中,召為侍衛步軍都指揮使,領河陽節度使,暉於是始覺晉有患己意。是時,出帝昏亂,馮玉、李彥韜等用事,暉曲意事之,因得復鎮靈武。時王令溫鎮靈武,失夷落心,大為邊患。暉即請曰:「今朝廷多事,必不能以兵援臣,願得自募兵以為衛。」乃募得
 兵千餘人,行至梅戍,蕃夷稍稍來謁,暉顧首領一人,指其佩劍曰:「此板橋王氏劍邪?吾聞王氏劍天下利器也。」俯而取諸腰間,若將玩之,因擊殺首領者,其從騎十餘人皆殺之。裨將藥元福曰:「今去靈武尚五六百里,奈何?」暉笑曰:「此夷落之豪,部族之所恃也,吾能殺之,其餘豈敢動哉!」已而諸族皆以兵扼道路,暉以言譬諭之,獨所殺首領一族求戰,即與之戰而敗走,諸族遂不敢動。暉至靈武,撫綏邊部,凡十餘年,恩信大著。官至中書令,封陳留王。廣順三年卒,追封衛王。子繼業。



 皇甫暉皇甫暉,魏州人也。為魏軍卒,戍瓦橋關,歲滿當代歸,而
 留屯貝州。是時,唐莊宗已失政,天下離心。暉為人驍勇無賴,夜博軍中,不勝,乃與其徒謀為亂,劫其部將楊仁晟曰:「唐能破梁而得天下者,以先得魏而盡有河北兵也。魏軍甲不去體、馬不解鞍者十餘年,今天下已定,而天子不念魏軍久戍之勞,去家咫尺,不得相見。今將士思歸不可遏,公當與我俱行。不幸天子怒吾軍,則坐據一州,足以起事。」仁晟曰:「公等何計之過也!今英主在上,天下一家,精甲銳兵,不下數十萬,公等各有家屬,何故出此不祥之言?」軍士知不可強,遂斬之,推一小校為主,不從,又斬之,乃攜二首以詣裨將趙在禮,在禮從之,
 乃夜焚貝州以入于魏,在禮以暉為馬步軍都指揮使。暉擁甲士數百騎,大掠城中,至一民家,問其姓,曰:「姓國。」暉曰:「吾當破國!」遂盡殺之。又至一家,問其姓,曰:「姓萬。」



 暉曰:「吾殺萬家足矣。」又盡殺之。及明宗入魏,遂與在禮合謀,莊宗之禍自暉始。明宗即位,暉自軍卒擢拜陳州刺史,終唐世常為刺史。



 晉天福中,以衛將軍居京師。在禮已秉旄節,罷鎮來朝,暉往候之曰:「與公俱起甘陵,卒成大事,然由我發也,公今富貴,能恤我乎?不然,禍起坐中!」在禮懼,遽出器幣數千與之,而飲以酒,暉飲自若,不謝而去。久之,為密州刺史。



 契丹犯闕,暉率其州人奔于江
 南,李景以為歙州刺史、奉化軍節度使,鎮江州。周師征淮,景以暉為北面行營應援使,屯清流關,為周師所敗,並其都監姚鳳皆被擒。



 世宗召見,暉金瘡被體,哀之,賜以金帶、鞍馬,後數日卒。拜鳳左屯衛上將軍。



 唐景思唐景思,秦州人也。幼善角牴,以屠狗為生。後去為軍卒,累遷指揮使。唐魏王繼岌伐蜀,景思為蜀守固鎮。繼岌兵至,景思以城降,拜興州刺史。晉高祖時,為貝州行軍司馬。出帝時,契丹攻陷貝州,景思為趙延壽所得,以為壕砦使。契丹滅晉,拜景思亳州防禦使。漢高祖時,為鄧州行軍司馬,後為沿淮巡檢。



 漢法酷,而史弘肇用事,喜
 以告訐殺人。景思有奴,嘗有所求不如意,即馳見弘肇,言景思與李景交通,而私畜兵甲。弘肇遣吏將三十騎往收景思,奴謂吏曰:「景思勇者也,得則殺之,不然將失之也。」吏至,景思迎前,以兩手抱吏呼冤,請詣獄自理。吏引奴與景思驗,景思曰:「我家在此,請索之。有錢十千,為受外賂。有甲一屬,為私畜兵。」吏索之,惟一衣笥,軍籍、糧簿而已。吏閔而寬之,景思請械送京師以自明。景思有僕王知權在京師,聞景思被告,乃見弘肇,願先下獄明景思不反,弘肇憐之,送知權獄中,日勞以酒食。景思既械就道,潁、亳之人隨至京師共明之。弘肇乃鞫其奴,具
 伏,即奏斬奴而釋景思。後從世宗戰高平,世宗以所得漢降兵數千為效順指揮,以景思為指揮使,復戍淮上。周師伐淮南,以功領饒州刺史,遷濠州刺史,兵攻濠州,以戰傷重卒,贈武清軍節度使。



 王進王進,幽州良鄉人也。為人勇悍,走及奔馬。少聚徒為盜,鄉里患之,符彥超遣人以賂招置麾下。彥超鎮安遠軍,軍中有變,遣進馳奏京師,明宗怪其來速,嘉其足力,以隸寧衛指揮。漢高祖為侍衛親軍指揮使,以進為軍校。高祖鎮河東,因以之從,每有急,遣進馳至京師,往返不過五六日,由是愈親愛之,累遷奉國軍都指揮使。從周
 太祖起魏,遷虎捷右廂都指揮使。歷汝、鄭二州防禦使、彰德軍節度使。顯德元年秋,以疾卒,贈太師。



 嗚呼!予述舊史,至於王進之事,未嘗不廢書而歎曰:甚哉,五代之君,皆武人崛起,其所與俱勇夫悍卒,各裂土地封侯王,何異豺狼之牧斯人也!雖其附託遭遇,出於一時之幸,然猶必皆橫身陣敵,非有百夫之勇,則必一日之勞。至如進者,徒以疾足善走而秉旄節,何其甚歟!豈非名器之用,隨世而輕重者歟?世治則君子居之而重,世亂則小人易得而輕歟?抑因緣僥倖,未始不有,而尤多於亂世,既其極也,遂至於是歟?豈其又有甚於是
 者歟?當此之時,為國長者不過十餘年,短者三四年至一二年。天下之人,視其上易君代國,如更戍長無異,蓋其輕如此,況其下者乎!如進等者,豈足道哉!《易》否泰消長,君子小人常相上下,視在上者如進等,則其在下者可知矣。予書進事,所以哀斯人之亂,而見當時賢人君子之在下者,可勝道哉!可勝道哉!



 常思常思,字克恭,太原人也。初從唐莊宗為卒,後為長劍指揮使。歷唐、晉為六軍都虞候。漢高祖為河東節度使,以思為牢城指揮使。高祖入立,領武勝軍節度使,徙鎮昭義。思起軍卒,未嘗有戰功,徒以幸會漢興,遂秉旄節。在
 潞州五年,以聚斂為事,而性鄙儉。初,思微時,周太祖方少孤無依,衣食于思家,以思為叔,後思與周太祖俱遭漢以取富貴。周太祖已即位,每呼思為常叔,拜其妻,如家人禮。



 廣順三年,徙鎮歸德,居三年來朝,又徙平盧,思因啟曰:「臣居宋,宋民負臣絲息十萬兩,願以券上進。」太祖頷之,即焚其券,詔宋州悉蠲除之。思居青州,踰年得疾,歸于洛陽,卒,贈中書令。



 孫方諫孫方諫,鄭州清苑人也。初,定州西北有狼山堡,定人常保以避契丹,有尼深意居其中,以佛法誘民,民多歸之。後尼死,堡人言其尸不朽,因奉而事之。尼姓孫氏,方諫
 自以為尼族人,即繼行其法,堡人推以為主。晉出帝時,義武軍節度使惡方諫聚徒山中,恐為邊患,因表以為游奕使。方諫因有所求不得,乃北通契丹。



 契丹後滅晉,以方諫為義武軍節度使。已而徙方諫於雲中,方諫不受命,率其徒復入狼山。漢高祖起,契丹縱火燒定州,虜其人民北去。方諫聞之,自狼山入,據之以歸漢,高祖嘉之,即拜方諫義武軍節度使。周太祖時,徙鎮鎮國,以其弟行友為定州留後。世宗攻太原,方諫朝於行在,從還京,至洛得疾,徙鎮匡國,卒於洛陽,年六十二,贈太師。



\end{pinyinscope}