\article{卷一梁本紀第一}

\begin{pinyinscope}

 太祖神武元聖孝皇帝,姓朱氏,宋州碭山午溝里人也。其父誠,以《五經》教授鄉里,生三子,曰全昱、存、溫。誠卒,三子貧,不能為生,與其母傭食蕭縣人劉崇家。全昱無他材能,然為人頗長者。存、溫勇有力,而溫尤兇悍。



 唐僖宗乾符四年,黃巢起曹、濮,存、溫亡入賊中。
 巢攻嶺南,存戰死。巢陷京師,以溫為東南面行營先鋒使。攻陷同州,以為同州防禦使。是時,天子在蜀,諸鎮會兵討賊。溫數為河中王重榮所敗,屢請益兵於巢,巢中尉孟楷抑而不通。溫客謝瞳說溫曰:「黃家起於草莽,幸唐衰亂,直投其隙而取之爾,非有功德興王之業也,此豈足與共成事哉!今天子在蜀,諸鎮之兵日集,以謀興復,是唐德未厭於人也。且將軍力戰於外,而庸人制之於內,此章邯所以背秦而歸楚也。」溫以為然,乃殺其監軍嚴實,自歸於河中,因王重榮以降。都統王鐸承制拜溫左金吾衛大將
 軍、河中行營招討副使,天子賜溫名全忠。



 中和三年三月,拜全忠汴州刺史、宣武軍節度使。四月,諸鎮兵破巢,復京師,巢走藍田。七月丁卯,全忠歸于宣武。是歲,黃巢出藍田關,陷蔡州。節度使秦宗權叛附于巢,遂圍陳州。徐州時溥為東南面行營兵馬都統,會東諸鎮兵以救陳。陳州刺史趙犨亦乞兵于全忠。溥雖為都統而不親兵,四年,全忠乃自將救犨,率諸鎮兵擊敗巢將黃鄴、尚讓等。犨以全忠為德,始附屬焉。是時,河東李克用下兵太行,度河,出洛陽,與東兵會,擊巢。巢已敗去,全忠及克用追敗之于郾城。巢走中牟,又敗之于
 王滿。巢走封丘,又大敗之。巢挺身東走,至泰山狼虎谷,為時溥追兵所殺。九月,天子以全忠為檢校司徒、同中書門下平章事,封沛郡侯。光啟二年三月,進爵王。義成軍亂,逐其節度使安師儒,推牙將張驍為留後,師儒來奔,殺之。遣硃珍、李唐賓陷滑州,以胡真為留後。十二月,徙封吳興郡王。



 自黃巢死,秦宗權稱帝,陷陜、洛、懷、孟、唐、許、汝、鄭州,遣其將秦賢、盧瑭、張晊攻汴。賢軍板橋,晊軍北郊,瑭軍萬勝,環汴為三十六柵。王顧兵少,不敢出。乃遣朱珍募兵於東方,而求救於兗、鄆。三年春,珍得萬人、馬數百匹以歸。乃擊賢
 板橋,拔其四柵。又擊瑭萬勝,瑭敗,投水死。宗權聞瑭等敗,乃自將精兵數千,柵北郊。五月,兗州朱瑾、鄆州硃宣來赴援。王置酒軍中,中席,王陽起如廁,以輕兵出北門襲晊,而樂聲不輟。晊不意兵之至也,兗、鄆之兵又從而合擊,遂大敗之,斬首二萬餘級。宗權與晊夜走,過鄭,屠其城而去。宗權至蔡,復遣張晊攻汴。王聞晊復來,登封禪寺後岡,望晊兵過,遣朱珍躡之,戒曰:「晊見吾兵必止。望其止,當速返,毋與之斗也。」已而晊見珍在後,果止。珍即馳還。



 王令珍引兵蔽大林,而自率精騎出其東,伏大塚間。晊止而食,食畢,拔旗幟,馳擊珍。珍兵小
 卻,王引伏兵橫出,斷晊軍為三而擊之。晊大敗,脫身走。宗權怒,斬晊。而河陽、陜、洛之兵為宗權守者,聞蔡精兵皆已殲於汴,因各潰去。故諸葛爽將李罕之取河陽、張全義取洛陽以來附。十月,天子使來,賜王紀功碑。朱宣、朱瑾兵助汴,已破宗權東歸,王移檄兗、鄆,誣其誘汴亡卒以東,乃發兵攻之,取其曹州、濮州。遂遣朱珍攻鄆州,大敗而還。十二月,天子使來,賜王鐵券及德政碑。



 淮南節度使高駢死,楊行密入揚州,天子以王兼淮南節度使。王乃表行密為副使,以行軍司馬李璠為留後。璠之揚州,行密不納。文德元年正月,王如淮南,至宋州而還。
 是時,秦宗權陷襄州,以趙德諲為節度使。德諲叛于宗權以來附。天子因以王為蔡州四面行營都統,以德諲為副。



 三月庚子,僖宗崩。天雄軍亂,囚其節度使樂彥貞。其子相州刺史從訓攻魏,來乞兵。遣朱珍助從訓攻魏。而魏軍殺彥貞,從訓戰死,魏人立羅弘信,珍乃還。



 張全義取河陽,逐李罕之。罕之奔于河東。李克用遣兵圍河陽,全義來求救,遣丁會、牛存節救之,擊敗河東兵于沇河。



 五月,行營討蔡州,圍之百餘日,不克。是時,時溥已為東南面都統,又以王統行營,而溥猶稱都統。王乃上書,論溥討蔡無功而不落都統,且欲激怒溥以起兵端。初,
 高駢死,淮南亂,楚州刺史劉瓚來奔,納之,及王兵攻蔡不克,還,欲攻徐,乃遣硃珍將兵數千以東,聲言送瓚還楚州。溥怒論己,又聞珍以兵來,果出兵拒之。珍戰于吳康,大敗之,取其豐、蕭二縣。遂攻宿州,下之。珍屯蕭縣,別遣龐師古攻徐州。龍紀元年正月,師古敗溥于呂梁。淮西牙將申叢執秦宗權,折其足,將檻送京師;別將郭璠殺叢,篡宗權以來獻。王遣行軍司馬李璠獻俘于京師,表郭璠淮西留後。三月,天子封王為東平王。七月,朱珍殺李唐賓,王如蕭縣,執珍殺之,遂攻徐州。冬,大雨,水,不能軍而旋。



 初,秦宗權遣其弟宗衡掠地淮南。是歲,宗
 衡為其將孫儒所殺,儒攻楊行密于揚州。淮南大亂,行密走宣州,儒入揚州。大順元年春,遣龐師古攻孫儒于淮南,大敗而還。四月,宿州將張筠以宿州復歸于時溥,王自將攻之,不克。



 初,黃巢敗走,李克用追之,至於冤朐,不及而旋。過汴,駐軍于北郊,王邀克用置酒上源驛,夜以兵攻之。克用踰城而免,訟其事于京師,天子知曲在汴而和解之。至是,宰相張濬私與汴交,王厚之以賂,濬為汴請伐河東。唐諸大臣皆以為不可興師。濬挾汴力,請益堅。天子不得已,許之。五月,以濬為太原四面行營都統,王為東南面招討使。然王不親兵,以兵三千屬濬而
 已。濬屯于陰地。河東叛將馮霸殺潞州守將李克恭來降,遣葛從周入潞州。李克用遣康君立攻之,從周走河陽。



 九月,王如河陽。十月,天子以王兼宣義軍節度使,遂如滑州,假道于魏,以攻河東,且責其軍須,亦所以怒魏為兵端也。魏人果以謂非兵所當出,而辭以糧乏,皆不許。於是攻魏。十一月,張濬之師大敗于陰地。二年正月,王及魏人戰于內黃,大敗之,屠故元城,羅弘信來送款。十月,克宿州。十一月,曹州將郭紹賓殺其刺史郭饒來降。十二月,丁會敗朱瑾于金鄉。景福元年二月,攻鄆州,前軍朱友裕敗于斗門,王軍後至,又敗而還。冬,友裕取
 濮州,遂攻徐州。二年四月,龐師古克徐州,殺時溥。王如徐州,以師古為留後,遂攻兗、鄆。



 乾寧元年二月,王及朱宣戰于漁山,大敗之。二年八月,又敗宣於梁山。十一月,又敗之于巨野。兗、鄆求救于河東,李克用發兵救之,假道于魏。既而魏人擊之,克用怒,大舉攻魏。羅弘信來求救,遣葛從周救魏。是歲,李克用封晉王。三年五月,戰于洹水,擒克用子落落,送于魏,殺之。七月,鳳翔李茂貞犯京師,天子出居於華州。王請以兵赴難,天子優詔止之。又請遷都洛陽,不許。四年正月,龐師古克鄆州,王如鄆州,以朱友裕為留後。遂攻兗州。朱瑾奔于淮南,以葛從
 周為兗州留後。九月,攻淮南,龐師古出清口,葛從周出安豐,王軍屯于宿州。楊行密遣硃瑾先擊清口,師古敗死。從周亟返兵,至於渒河,瑾又敗之。王懼,馳歸。



 光化元年三月,天子以王兼天平軍節度使。四月,遣葛從周攻晉之山東,取邢、洺、磁三州。襄州趙匡凝自其父德諲時來附,匡凝又與楊行密、李克用通,而其事泄。七月,遣氏叔琮、康懷英攻匡凝,取其泌、隨、鄧三州。匡凝請和,乃止。十二月,李罕之以潞州來降。二年,幽州劉仁恭攻魏,羅紹威來求救。王救魏,
 敗仁恭于內黃。四月,遣氏叔琮攻晉太原,不克。七月,李克用取澤、潞。十一月,保義軍亂,殺其節度使王珙,推其牙將李璠為留後,其將朱簡殺璠來降。以簡為保義軍節度使。三年四月,遣葛從周攻劉仁恭之滄州,取其德州,及仁恭戰于老鴉堤,大敗之。八月,晉取洺州。王如洺州,復取之。是時,鎮、定皆附于晉。遂攻鎮州,破臨城,王鎔來送款。進攻定州,王郜奔於晉,其將王處直以定州降。



 唐宦者劉季述作亂,天子幽于東宮。天復元年正月,護駕都頭孫德昭誅季述,天子復位。封王為梁王。遣張存敬攻王珂於河中,出含山,下晉、絳二州。王珂求救于晉,
 晉不能救,乃來降。三月,大舉攻晉。氏叔琮出太行,取澤、潞。葛從周、張存敬、侯言、張歸厚及鎮、定之兵,皆會于太原,圍之,不克,遇雨而還。五月,天子以王兼河中尹、護國軍節度使。六月,晉取慈、隰。



 自劉季述等已誅,宰相崔胤外與梁交,欲假梁兵盡誅宦者。而鳳翔李茂貞、邠寧王行瑜等,皆遣子弟以精兵宿衛天子,宦官韓全誨等亦因恃以為助。天子與胤計事,宦者屬耳,頗聞之。乃選美女,內之宮中,陰令伺察其實。久之,果得胤奏謀所以誅宦者之說,全誨等大懼,日夜相與涕泣,思圖胤以求全。胤知謀泄,事急,即矯為制,召梁兵入誅宦者。十月,王以
 宣武、宣義、天平、護國兵七萬,至于河中,取同州,遂攻華州,韓建出降。全誨等聞梁王兵且至,即以岐、邠宿衛兵劫天子奔于鳳翔。王乃上書言胤所以召之之意。天子怒,罷胤相,責授工部尚書,詔梁兵還鎮。王引兵去,攻邠州,屯于三原。邠州節度使楊崇本以邠、寧、慶、衍四州降。崔胤奔於華州。二年春,王退軍于河中。晉攻晉、絳。遣朱友寧擊敗晉軍于蒲縣,取汾、慈、隰,遂圍太原,不克而還,汾、慈、隰復入於晉。四月,友寧引兵西,至興平,及李茂貞戰于武功,大敗之。王兵犯鳳翔,茂貞數出戰,輒敗,遂圍之。十一月,鄜坊李周彞以兵救鳳翔,王遣孔勍襲鄜州,
 虜周彞之族,徙于河中,周彞乃降。是時,岐兵屢敗,而圍久,城中食盡,自天子至後宮,皆凍餒。三年正月,茂貞殺韓全誨等二十人,囊其首,示梁軍,約出天子以為解。甲子,天子出幸梁軍。遣使者馳召崔胤,胤託疾不至。王使人戲胤曰:「吾未識天子,懼其非是,子來為我辨之。」天子還至興平,胤率百官奉迎。王自為天子執轡,且泣且行,行十餘里,止之。人見者,咸以為忠。己巳,天子至自鳳翔,素服哭于太廟而後入,殺宦者七百餘人。二月甲戌,天子賜王「回天再造竭忠守正功臣」,以輝王祚為諸道兵馬元帥,王為副元帥。王乃留子友倫為護駕指揮使,
 以為天子衛,引兵東歸。



 天子餞于延喜樓,賜《楊柳枝》五曲。



 初,梁兵已西,青州王師範遣其將劉鄩襲據梁兗州。王已還梁,四月,如鄆州,遣朱友寧攻青州。師範敗之于石樓,友寧死。九月,楊師厚敗青人于臨朐,取其棣州,師範以青州降,而鄩亦降。友倫擊鞠,墜馬死。王怒,以為崔胤殺之,遣朱友謙殺胤于京師。其與友倫擊鞠者,皆殺之。



 自天子奔華州,王請遷都洛陽,雖不許,而王命河南張全義修洛陽宮以待。天祐元年正月,王如河中,遣牙將寇彥卿如京師,請遷都洛陽,并徙長安居人以
 東。



 天子行至陜州,王朝于行在,先如東都。是時,六軍諸衛兵已散亡,其從以東者,小黃門十數人,打球供奉、內園小兒等二百餘人。行至穀水,王教醫官許昭遠告其謀亂,悉殺而代之,然後以聞。由是天子左右皆梁人矣。四月甲辰,天子至自西都。



 是時,晉王李克用、岐王李茂貞、楚王趙匡凝、蜀王王建、吳王楊行密聞梁遷天子洛陽,皆欲舉兵討梁,王大懼。六月,楊崇本復附于岐。王乃以兵如河中,聲言攻崇本,遣朱友恭、氏叔琮、蔣玄暉等行弒,昭宗崩。十月,王朝于京師,殺朱友恭、氏叔琮。十
 一月,攻淮南,取其光州,攻壽州,不克而旋。二年二月,遣蔣玄暉殺德王裕等九王于九曲池。六月,殺司空裴贄等百餘人。七月,天子使來,賜王「迎鑾紀功碑」。



 王欲代唐,使人諭諸鎮,襄州趙匡凝以為不可。遣楊師厚攻之,取其唐、鄧、復、郢、隨、均、房七州。王如襄州,軍于漢北。九月,師厚破襄州,匡凝奔于淮南。師厚取荊南,荊南留後趙匡明奔于蜀。遂出光州,以攻壽州,不克。天子卜祀天於南郊,王怒,以為蔣玄暉等欲祈天以延唐。天子懼,改卜郊。十一月辛巳,天子封王為魏王、相國,總百揆。以宣武、宣義、天平、護國、天雄、武順、祐國、河陽、義武、昭義、武寧、保義、
 忠義、武昭、武定、泰寧、平盧、匡國、鎮國、荊南、忠武二十一軍為魏國,備九錫。王怒,不受。十二月,天子以王為天下兵馬元帥。王益怒,遣人告樞密使蔣玄暉與何太后私通,殺玄暉而焚之,遂弒太后於積善宮。又殺宰相柳璨,太常卿張延範車裂以徇。天子詔以太后故停郊。



 三年春,魏州羅紹威謀殺其牙軍,來假兵以虞變,王為發兵北攻劉仁恭之滄州,兵過魏而紹威已殺牙軍,其兵之在外者果皆叛,據貝、衛、澶、博州,王以兵悉殺之。遂攻滄州,軍於長蘆。劉仁恭求救於晉。晉人取潞州,王乃旋軍。



\end{pinyinscope}