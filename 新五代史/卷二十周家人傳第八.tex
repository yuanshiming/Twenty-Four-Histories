\article{卷二十周家人傳第八}

\begin{pinyinscope}

 柴守禮周太祖聖穆皇后柴氏,無子,養后兄守禮之子以為子,是為世宗。守禮字克讓,以后族拜銀青光祿大夫、檢校吏部尚書、兼御史大夫。世宗即位,加金紫光祿大夫、檢校司空、光祿卿。致仕,居於洛陽,終世宗之世,未嘗至京師,而左右亦莫敢言,第以元舅禮之,而守禮亦頗恣橫,嘗殺人於市,有司有聞,世宗不問。是時,王溥、汪晏、王彥超、韓令坤等同時將相,皆有父在洛陽,與守禮朝夕往
 來,惟意所為,洛陽人多畏避之,號「十阿父。」守禮卒年七十二,官至太傅。



 嗚呼,父子之恩至矣!孟子言:舜為天子,而瞽叟殺人,則棄天下,竊負之而逃。以謂天下可無舜,不可無至公,舜可棄天下,不可刑其父,此為世立言之說也。



 然事固有不得如其意者多矣!蓋天子有宗廟社稷之重、百官之衛、朝廷之嚴,其不幸有不得竊而逃,則如之何而可?予讀周史,見守禮殺人,世宗寢而不問,蓋進任天下重矣,而子於其父亦至矣,故寧受屈法之過,以申父子之道,其所以合於義者,蓋知權也。君子之於事,擇其輕重而
 處之耳。失刑輕,不孝重也。刑者所以禁人為非,孝者所以教人為善,其意一也,孰為重?刑一人,未必能使天下無殺人,而殺其父,滅天性而絕人道,孰為重?權其所謂輕重者,則天下雖不可棄,而父亦不可刑也。然則為舜與世宗者,宜如何無使瞽叟、守禮至於殺人,則可謂孝矣!然而有不得如其意,則擇其輕重而處之焉。世宗之知權,明矣夫!



 世宗貞惠皇后劉氏世宗三皇后。貞惠皇后劉氏,不知其世家,蓋微時所娶也,世宗為左監門衛將軍,得封彭城縣君。世宗從太祖於魏,后留京師,太祖舉兵,漢誅其族家屬,后見殺。太祖
 即位,追封彭城郡夫人。世宗顯德四年夏四月,始詔彭城郡夫人劉氏追冊為皇后,有司謚曰貞惠,陵曰惠陵。



 宣懿皇后符氏宣懿皇后符氏,其祖秦王存審,父魏王彥卿。后世王家,出於將相之貴,為人明果有大志。初適李守貞子崇訓。守貞事漢為河中節度使,已挾異志。有術者善聽人聲以知吉凶,守貞出其家人使聽之,術者聞后聲,驚曰:「此天下之母也!」守貞益自負,曰:「吾婦猶為天下母,吾取天下復何疑哉!」於是決反。而漢遣周太祖討之,逾年,攻破其城。崇訓知不免,手自殺其家人,次以及后,后走匿,以帷幔自蔽,崇訓惶遽求后不得,遂自殺。漢兵入其家,后
 儼然坐堂上,顧軍士曰:「郭公與吾王父有舊,汝輩無犯我!」軍士見之不敢迫。太祖聞之,以謂一女子能使亂兵不敢犯,奇之,為加慰勉,以歸彥卿。后感太祖不殺,拜太祖為父。其母以后夫家滅亡,而獨脫死兵刃之間,以為天幸,欲使削髮為尼,后不肯曰:「死生有命,天也。何必妄毀形髮為!」太祖於后有恩,而世宗性特英銳,聞后如此,益奇之。及劉夫人卒,遂納以為繼室。世宗即位,冊為皇后。世宗辦急多暴怒,而後嘗追悔,每怒左右,后必從容伺顏色,漸為解說,世宗意亦隨解,由是益重之。世宗征淮,后以帝不宜親行,切諫止之,世宗不聽。師久無功,遭
 大暑雨,后以憂成疾而崩。議者以方用兵,請殺喪禮,於是百官朝臨于西宮,三日而釋服,帝亦七日而釋,葬於新鄭,陵曰懿陵。



 後立皇后符氏。后妹也。國初,遷西宮,號周太后。



 世宗七子世宗子七人:長曰宜哥,次二皆未名,次曰恭皇帝,次曰熙讓,次曰熙謹,次曰熙誨,皆不知其母為誰氏。宜哥與其二,皆為漢誅。太祖即位,詔賜皇孫名誼,贈左驍衛大將軍;誠,左武衛大將軍;諴,左屯衛大將軍。



 顯德三年,群臣請封宗室,世宗以謂為國日淺,恩信未及於人,而須功德大成,慶流于世,而後議之可也。明年夏四月癸未,
 先封太祖諸子。又詔曰:「父子之道,聖賢不忘,再思天閼之端,愈動悲傷之抱。故皇子左驍衛大將軍誼、左武衛大將軍諴、左屯衛大將軍誠等,載惟往事,有足傷懷,宜增一字之封,仍贈三台之秩。誼可贈太尉,追封越王;誠太傅,吳王;諴太保,韓王。」而皇子在者皆不封。



 六年,北復三關,遇疾,還京師。六月癸未,皇子宗訓,特進左衛上將軍,封梁王;而宗讓亦拜左驍衛上將軍,封燕國公。後十日而世宗崩,梁王即位,是為恭皇帝。其年八月,宗讓更名熙讓,封曹王。熙謹、熙誨皆前未封爵,遂拜熙謹右武衛大將軍,封紀王;熙誨左領軍衛大將軍,蘄王。皇朝
 乾德二年十月,熙謹卒。熙讓、熙誨,不知其所終。



 嗚呼!至公,天下之所共也。其是非曲直之際,雖父愛其子,亦或有所不得私焉。當周太祖舉兵于魏,漢遣劉銖誅其家族於京師,酷毒備至;後太祖入立,遣人責銖,銖辭不屈,太祖雖深恨之,然以銖辭直,終不及其家也。及追封妻子之被殺者,其言深自隱痛之而已,不敢有非漢之辭焉,蓋知其曲在己也。故略存其辭,以見周之有愧於其心者矣!



\end{pinyinscope}