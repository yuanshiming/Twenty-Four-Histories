\article{卷四十七雜傳第三十五}

\begin{pinyinscope}

 華溫琪華溫琪,字德潤,宋州下邑人也。世本農家。溫琪身長七尺。少從黃巢為盜,巢陷長安,以溫琪為供奉官都知。巢敗,溫琪走滑州,顧其狀貌魁偉,懼不自容,乃投白馬河,流數十里,不死,河上人援而出之。又自經於桑林,桑輒枝折。乃之胙縣,有田父見之曰:「子狀貌堂堂,非常人也!」乃匿于家。後歲餘,聞濮州刺史朱裕募士為兵,乃往依之。



 後事梁,為開道指揮使,累以戰功為絳、棣二州刺史。
 棣州苦河水為患,溫琪徙于新州以避之,民賴其利。歷齊、晉二州。莊宗攻晉州,踰月不能破,梁末帝嘉溫琪善守,升晉州為定昌軍,以溫琪為節度使。坐掠部民妻,為其夫所訟,罷為金吾衛大將軍、左龍武統軍。朱友謙以河中叛附於晉,末帝拜溫琪汝州防御使、河中行營排陣使。遷耀州觀察留後。



 莊宗滅梁,見溫琪,曰:「此為梁守平陽者也。」嘉之,因以耀州為順義軍,拜溫琪節度使,徙鎮雄武。明宗時來朝,願留闕下,以為左驍衛上將軍。踰年,明宗謂樞密使安重誨曰:「溫琪舊人,宜與一重鎮。」重誨意不欲與,對以無員闕。



 佗日,明宗語又及之,重誨
 曰:「可代者惟樞密使耳。」明宗曰:「可。」重誨不能答。溫琪聞之懼,稱疾不出者累月。已而以為鎮國軍節度使。廢帝時,以太子太保致仕。天福元年卒,贈太子太傅。



 萇從簡萇從簡,陳州人也。世本屠羊。從簡去事晉為軍校,力敵數人,善用槊。莊宗用兵攻城,從簡多為梯頭,莊宗愛其勇,以功累遷步軍都指揮使。莊宗與梁軍對陣,梁軍有執大旗出入陣間者,莊宗登高丘望見之,嘆曰:「彼猛士,誰能為我取之者?」



 從簡因前請往,莊宗惜之,不許。從簡潛率數騎,馳入梁軍,奪其旗而還,軍中皆鼓噪,莊宗壯之,賜與甚厚。



 從簡嘗中流矢,鏃入髀骨,命工取之。工無
 良藥,欲鑿其骨,人皆以為不可。



 從簡遽使鑿之,工遲疑不忍下,從簡叱其亟鑿,左右視者,皆若不勝其毒,而從簡言笑自若。然其為人剛暴難制,莊宗每屈法優容之。累遷蔡州防禦使。明宗時,歷麟、汝、汾、金四州防禦使。明宗嘗戒之曰:「富貴可惜,然汝不能守也。先帝能貸爾,吾恐不能。」從簡性不可悛,明宗亦不之責。



 廢帝舉兵於鳳翔,從簡與諸鎮兵圍之,已而兵潰,從簡東走,被執。廢帝責其不降,從簡曰:「事主不敢二心。」廢帝釋之,拜潁州團練使。晉高祖起兵太原,廢帝將親征,召為招討副使,從至河陽,拜河陽三城節度使。廢帝還洛陽,從簡即降晉。
 歷鎮忠武、武寧,入為左金吾衛上將軍。卒年六十五,贈太師。



 從簡好食人肉,所至多潛捕民間小兒以食。許州富人有玉帶,欲之而不可得,遣二卒夜入其家,殺而取之。卒夜踰垣,隱木間,見其夫婦相待如賓,二卒歎曰:「吾公欲奪其寶,而害斯人,吾必不免。」因躍出而告之,使其速以帶獻,遂踰垣而去,不知其所之。



 張筠弟抃張筠,海州人也。世以貲為商賈。筠事節度使時溥為宿州刺史。梁兵攻溥取宿州,得筠,愛其辯惠,以為四鎮客將、長直軍使,累拜宣徽使。末帝分相、澶、衛三州為昭德軍,以筠為節度使,由是魏博軍叛附於晉。晉王攻相州,
 筠棄城走。後以為永平軍節度使。梁亡事唐,仍為京兆尹。從郭崇韜伐蜀,為劍南兩川安撫使。



 蜀平,拜河南尹,徙鎮興元。筠嘗有疾,不見將吏,副使符彥琳入問疾,筠又辭不見。彥琳疑筠已死,即請出牌印。筠怒,命左右收彥琳下獄,以其反聞。明宗知彥琳無反狀,召彥琳釋之,陽徙筠為西京留守,戒守者不內,筠至長安不得入,乃朝京師,以為左驍衛上將軍。



 筠弟抃,當筠為京兆尹時,以為牙內指揮使、三白渠營田制置使。筠西伐蜀,留抃守京兆。蜀平,魏王繼岌班師,至興平,而明宗自魏起,京師大亂,抃乃斷咸陽浮橋以拒繼岌,繼岌乃自殺。初,筠
 代康懷英為永平軍節度使,而懷英死,筠即掠其家貲。又於唐故宮掘地,多得金玉。有偏將侯莫陳威者,嘗與溫韜發唐諸陵,分得寶貨,筠因以事殺威而取之。魏王繼岌死渭南,抃悉取其行橐。而王衍自蜀行至秦川,莊宗遣宦者向延嗣殺之,延嗣因盡得衍蜀中珍寶。明宗即位,即遣人捕誅宦者,延嗣亡命,而蜀之珍寶抃又取之。由是兄弟貲皆巨萬。然筠為人好施予,以其富,故所至不為聚斂,民賴以安。而抃嗜酒貪鄙,歷沂、密二州刺史。晉出帝時,以將軍市馬於回鶻,坐馬不中式,有司理其價直,抃性鄙,因鬱鬱而卒。



 筠居洛陽,擁其貲,以酒色聲
 妓自娛足者十餘年,人謂之「地仙。」天福二年,徙居長安。是歲,張從賓作亂,入洛陽,筠遂以免。卒,贈太子太師。



 嗚呼,五代反者多矣,吾於明宗獨難其辭。至於魏王繼岌薨,然後終其事也。



 莊宗遇弒,繼岌以元子握重兵,死于外而不得立,此大事也,而前史不書其所以然。



 夫繼岌之存亡,於張抃無所利害,抃何為而拒之不使之東乎?豈其有所使而為之乎?



 然明宗於符彥超深以為德,而待抃無所厚,此其又可疑也。不然,好亂之臣,望風而饗應乎?使抃不斷浮橋,而繼岌得以兵東,明宗未必能自立。則繼岌之死,由抃之拒,其所系者豈小哉!



 楊彥詢楊彥詢,字成章,河中寶鼎人也。少事青州王師範,師範好學,聚書萬卷,使彥詢掌之。彥詢為人聰悟,遂見親信。師範降梁,後見殺,彥詢無所歸,乃之魏,事楊師厚為客將。魏博叛梁入於晉,彥詢因留事晉。莊宗滅梁,以彥詢為引進副使,奉使吳、蜀,常稱旨。歷德州刺史、羽林將軍晉高祖鎮太原,廢帝疑其有貳志,擇諸將之謹厚者佐之,乃以彥詢為太原節度副使。其後晉高祖以疑見徙,欲拒命不行,以問彥詢,彥詢不敢正言,因曰:「太原之力,能與唐敵否?公其審計之!」高祖反意已決,彥詢亦不復敢言。高祖左右以彥詢異議,欲殺之,高祖遽止之,曰:「惟
 副使一人,我自保之。」乃免。是時,高祖乞兵於契丹,契丹耶律德光立高祖于太原,以兵送至河上。彥詢為宣徽使,數往來虜帳中,德光亦愛其為人。明年,拜威德軍節度使,復入為宣徽使,又拜安國軍節度使。天福七年,徙鎮鎮國,遭歲大饑,為政有惠愛。以病風罷為右金吾衛上將軍。卒年七十四,贈太子太師。



 李周李周,字通理,邢州內丘人,唐昭義軍節度使抱真之後也。父矩,遭世亂不仕,嘗謂周曰:「邯鄣用武之地,今世道未平,汝當從軍旅以興吾門。」周年十六,為內丘捕賊將,以勇聞。是時,梁、晉兵爭山東,群盜充斥道路,行者必以
 兵衛。內丘人盧岳將徙家太原,舍逆旅,傍徨不敢進,周意憐之,為送至西山。有盜從林中射岳,中其馬,周大呼曰:「吾在此,孰敢爾邪?」盜聞其聲,曰:「此李周也。」



 因各潰去。周送岳至太原,岳謂之曰:「吾少學星曆,且工相人。子方頤隆準,眉目疏徹,身長七尺,真將相也。吾占天象,晉必有天下,子宜留事晉,以圖富貴。」



 周以母老辭歸。



 是時,梁遣葛從周攻下邢、洺,晉王柵兵青山口,周未知所歸,乃思岳言,至青山歸晉,晉王以周為萬勝黃頭軍使。後從征伐常有功。從戰柏鄉,先登,遷匡霸指揮使,守楊劉。周為將甚勇,其於用兵,善守,能與士卒同甘苦。梁兵攻周,
 周堅守。久之,周聞母喪奔歸,莊宗遣佗將代周守,幾為梁兵所破,莊宗遽追周還守之,乃得不破。其後梁人已破德勝,因東擊楊劉,以巨艦絕河,斷晉餉援。周遣人馳趨莊宗求救,請日行百里以赴急,莊宗笑曰:「周為我守,何憂!」日行六十里,且行且獵,曰:「周非梁將可敵也。」比至,周已絕糧三日。莊宗以巨筏積薪沃油,順流縱火焚梁艦,梁兵解去。莊宗見周勞曰:「微公,諸將為梁擒矣!」歷相、蔡二州刺史。明宗時,拜武信軍節度使,徙鎮靜難,歷武寧、安遠、永興、宣武四鎮,所至多善政。晉高祖時,復鎮靜難,罷還。出帝幸澶淵,以周留守東京,還,拜開封尹。卒年
 七十四,贈太師。



 劉處讓劉處讓,字德謙,滄州人也。少為張萬進親吏,萬進入梁,為泰寧軍節度使,以處讓為牙將。萬進叛梁附晉,梁遣劉鄩討之。萬進遣處讓求救於晉,晉王方與梁相拒,未能出兵,處讓乃於軍門截耳而訴曰:「萬進所以見圍者,以附晉故也,奈何不顧其急?茍不出兵,願請死!」晉王壯之,曰:「義士也!」為之發兵。未渡河,而萬進為梁兵所敗,處讓因留事晉。莊宗即位,為客省使,常使四方,多稱旨。



 天成中,遷引進使,累遷左驍衛大將軍。廢帝時,魏州軍亂,逐其帥劉延皓,遣范延光招討,以處讓為河北都轉
 運使。晉高祖立,歷宣徽南院使。范延光反,高祖命楊光遠為招討使,以處讓參其軍事。已而副招討使張從賓叛于河陽,處讓分兵擊破從賓。還,與光遠攻鄴,逾年不能下。其後延光有降意而遲疑,處讓入城,譬以禍福,延光乃出降。



 唐制,樞密使常以宦者為之,自梁用敬翔、李振,至莊宗始用武臣,而權重將相,高祖時,以宰相桑維翰、李崧兼樞密使,處讓與諸宦者心不平之。光遠之討延光也,以晉重兵在己掌握,舉動多驕恣,其所求請,高祖頗裁抑之。處讓為光遠言:「此非上意,皆維翰、崧等嫉公耳!」光遠大怒。及兵罷,光遠見高祖,訴以維翰等沮己,
 高祖不得已,罷維瀚等,以處讓為樞密使。處讓在職,凡所陳述,多不稱旨。處讓丁母憂,高祖遂不復拜樞密使,以其印付中書而廢其職。處讓居喪期年,起復為彰德軍節度使、右金吾衛上將軍。以疾卒,年六十三,累贈太師。



 李承約李承約,字德儉,薊門人也。少事劉仁恭,為山後八軍巡檢使,將騎兵二千人。



 仁恭為其子守光所囚,承約以其騎兵奔晉,晉王以為匡霸指揮使。從破夾寨,戰臨清,以功累遷洺汾二州刺史、潁州團練使。天成中,邠州節度使毛璋有異志,明宗拜承約涇州節度副使,使往伺璋
 動靜。承約見璋,諭以禍福。後明宗遣人代璋,璋即時受代。明宗大喜,即拜承約黔南節度使。承約以恩信撫諸夷落,勸民農桑,興起學校。居數年,當代,黔南人詣京師乞留,為許留一年。召為左衛上將軍,改左龍武統軍,拜昭義軍節度使,復為左龍武統軍。天福二年,遷左驍衛上將軍。數請老,不許。卒年七十五,贈太子太師。



 張希崇張希崇,字德峰,幽州薊人也。少好學,通《左氏春秋》。劉守光不喜儒士,希崇因事軍中為偏將,將兵戍平州。其後契丹攻陷平州,得希崇,知其儒者也,以為盧龍軍行軍司馬。明宗時,盧文進自平州亡歸,契丹因以希崇代文
 進為平州節度使,遣其親將以三百騎監之。居歲餘,虜將喜其為人,監兵稍怠,希崇因與其麾下謀走南歸。其麾下皆言兵我,不可俱亡,懼不得脫,因勸希崇獨去。希崇曰:「虜兵守我者三百騎爾,烹其將,其兵必散走。且平州去虜帳千餘里,使其聞亂而呼兵,則吾與汝等在漢界矣!」眾皆曰善。乃先為阱,置以石灰。明日,虜將謁希崇,希崇飲之以酒,殺之阱中,兵皆潰去,希崇率其麾下,得生口二萬南歸。明宗嘉之,拜汝州防禦使。遷靈武節度使。靈州地接戎狄,戍兵餉道,常苦抄掠,希崇乃開屯田,教士耕種,軍以足食,而省轉饋,明宗下詔褒美。希崇撫養
 士卒,招輯夷落,自回鶻、瓜、沙皆遣使入貢。居四歲,上書求還內地,徙鎮邠寧。晉高祖入立,復拜靈武節度使,希崇歎曰:「吾當老死邊徼,豈非命邪!」希崇事母至孝,朝夕母食,必侍立左右,徹饌乃敢退。為將不喜聲色。好讀書,頗知星歷。天福三年,月掩畢口大星,希崇嘆曰:「畢口大星,邊將也,我其當之乎!」明年正月卒,贈太師。有子仁謙。



 相里金相裏金,字奉金,并州人也。為人勇悍,而能折節下士。事晉王,為五院軍隊長。梁、晉戰柏鄉、胡柳,皆有功,遷黃甲指揮使。同光中,拜忻州刺史。是時諸州皆用武人,多以部曲主場務,漁蠹公私,以利自入,金獨禁部曲不與事,
 厚其給養,使掌家事而已。遷隴州防禦使。廢帝起兵鳳翔,馳檄四鄰,四鄰未有應者,獨金首遣判官薛文遇見廢帝,往來計事。廢帝即位,德之,拜保義軍節度使。晉高祖起太原,廢帝以金為太原四面步軍都指揮使。高祖入立,徙鎮建雄,罷為上將軍。



 天福五年卒,贈太師。



 張廷蘊張廷蘊,開封襄邑人也。少為宣武軍卒,去事晉,稍遷軍校。常從莊宗征伐,先登力戰,金瘡滿體,莊宗壯之,以為帳前黃甲二十指揮步軍都虞候、魏博三城巡檢使。是時,莊宗在魏,以劉皇后從行,劉氏多縱其下擾人為不法,人無敢言者,廷蘊輒收而斬之。李繼韜叛于潞州,莊
 宗遣明宗為招討使,元行欽為都部署,廷蘊為馬步軍都指揮使,將兵為前鋒。廷蘊至潞,日已暮,即率兵百餘踰濠登城,城守者不能御,遂破潞州。明旦,明宗與行欽後至,明宗心頗慊之。廷蘊以功遷羽林都指揮使、申懷沂三州刺史、金潁隴絳四州防禦團練使、左監門衛上將軍。開運中,以疾卒。



 廷蘊武人,所識不過數字,而平生重文士。嘗從明宗破梁鄆州,獲判官趙鳳,廷蘊謂曰:「吾視汝貌必儒人,可無隱也。」鳳以實對,廷蘊亟薦於明宗。後鳳貴為相,數薦廷蘊於安重誨,重誨屢言之,明宗以廷蘊破潞之隙,終恨之,故終不秉髦節。廷蘊素廉,歷七
 州,卒之日,家無餘貲。



 馬全節馬全節,字大雅,大名元城人也。唐同光中,全節為捉生指揮使。趙在禮反鄴都,以全節為馬步軍指揮使。明宗即位,歷博單郢沂四州刺史、金州防禦使。廢帝時,蜀人攻金州,州兵才數百,全節散家財,與士卒堅守,蜀人去,廢帝召全節,以為滄州留後。晉高祖入立,即拜全節橫海軍節度使,徙鎮安遠,代李金全。金全叛附于李昪,高祖發兵三萬,使全節與安審暉討之,金全南奔。昪將李承裕守安州,全節與承裕戰州南,大敗承裕,斬首三千級,生擒千餘人。承裕棄城去,審暉追至雲夢,執承裕及
 其兵二千人,全節斬千五百人,以其餘兵并承裕獻於京師。承裕謂全節曰:「吾掠城中,所得百萬計,將軍皆取之矣。吾見天子,必訴此而後就刑。」



 全節懼,因殺承裕,高祖置而不問,徙全節鎮昭義。又徙安國。從杜重威討安重榮,以功徙鎮義武。自出帝與契丹交惡,全節未嘗不在兵間。開運元年,為行營都虞候,契丹與晉大軍相距澶、魏之間,全節別攻白團城,破之,虜七百人。克泰州,虜二千人,降其守將晉廷謙。四月,契丹敗于戚城,引兵分道而北,全節敗之于定豐,執其將安暉。七月,徙廣晉尹,留守鄴都。十月,杜重威為招討使,以全節為副,大敗契
 丹于衛村。



 全節為人謙謹,事母至孝,其臨政決事,必問法如何。初,徙廣晉,過元城,衣白襴謁其縣令,州里以為榮。開運二年,徙鎮順國,未至而卒,年五十五,贈中書令。



 皇甫遇皇甫遇,常山真定人也。為人有勇力,虯髯善射。少從唐明宗征伐,事唐為武勝軍節度使,所至苛暴,以誅斂為務,賓佐多解官逃去,以避其禍。晉高祖時,歷義武、昭義、建雄、河陽四鎮,罷為神武統軍。契丹入寇,陷貝州,出帝以高行周為北面行營都部署,遇為馬軍右廂排陣使。是時,青州楊光遠據城反,出帝乃遣李守貞及遇分兵守鄆州。遇等至馬家渡,契丹方將渡河助光遠,遇等擊
 敗之,以功拜義成軍節度使、馬軍都指揮使。



 開運二年,契丹寇西山,遣先鋒趙延壽圍鎮州,杜重威不敢出戰。延壽分兵大掠,攻破欒城、柏鄉等九縣,南至邢州。是時歲除,出帝與近臣飲酒過量,得疾,不能出征,乃遣北面行營都監張從恩會馬全節、安審琦及遇等禦之。從恩等至相州,陣安陽河南,遣遇與慕容彥超率數千騎前視虜。遇渡漳河,逢虜數萬,轉戰十餘里,至榆林,為虜所圍,遇馬中箭而踣,得其僕杜知敏馬,乘之以戰。知敏為虜所擒,遇謂彥超曰:「知敏,義士也,豈可失之!」即與彥超躍馬入虜,取之而還。虜兵與遇戰,自午至未,解而復合,
 益出生兵,勢甚盛。遇戒彥超曰:「今日之勢,戰與走爾,戰尚或生,走則死也。等死,死戰,猶足以報國。」張從恩與諸將怪遇視虜無報,皆謂遇已陷虜矣。已而有馳騎報遇被圍,安審琦率兵將赴之,從恩疑報者詐,不欲往,審琦曰:「成敗天也,當與公共之,雖虜不南來,吾屬失皇甫遇,復何面目見天子!」即引騎渡河,諸軍皆從而北,拒虜十餘里,虜望見救兵來,即解去。遇與審琦等收軍而南,契丹亦皆北去。是時,契丹兵已深入,人馬俱乏,其還也,諸將不能追,而從恩率遇等退保黎陽,虜因得解去。



 三年冬,以杜重威為都招討使,遇為馬軍右廂都指揮使,屯於
 中渡。重威已陰送款契丹,伏兵幕中,悉召諸將列坐,告以降虜,遇與諸將愕然不能對。重威出降表,遇等俯首以次自畫其名,即麾兵解甲出降。契丹遣遇與張彥澤先入京師,遇行至平棘,絕吭而死。



 嗚呼,梁亡而敬翔死,不得為死節;晉亡而皇甫遇死,不得為死事,吾豈無意哉!梁之篡唐,用翔之謀為多,由子佐其父而弒其祖,可乎?其不戮於斧鉞,為幸免矣。方晉兵之降虜也,士卒初不知,及使解甲,哭聲震天,即降豈其欲哉!使遇奮然攘臂而起,殺重威於坐中,雖不幸不免而見害,猶為得其死矣,其義烈豈不凜然哉!既俯首
 聽命,相與亡人之國矣,雖死不能贖也,豈足貴哉!君子之於人,或推以恕,或責以備。恕,故遷善自新之路廣;備則難得,難得,故可貴焉。然知其所可恕,與其所可貴,豈不又難哉!



 安彥威安彥威,字國俊,代州崞縣人也。少以軍卒隸唐明宗麾下。彥威善射,頗知兵法。明宗鎮天平、宣武、成德,以彥威常為牙將,以謹厚見信。明宗入立,皇子從榮鎮鄴,彥威為護聖指揮使。以從榮判六軍,彥威遷捧聖指揮使,領寧國軍節度使。



 晉高祖入立,拜彥威北京留守,徙鎮歸德。是時,河決滑州,命彥威塞之,彥威出私錢募民治隄。
 遷西京留守,遭歲大饑,彥威賑撫飢民,民有犯法,皆寬貸之,飢民愛之,不忍流去。丁母憂,哀毀過制。出帝與契丹隳盟,拜彥威北面行營副都統,彥威悉以家財佐軍用。以疾卒於京師。



 彥威與安太妃同宗,出帝事以為舅,彥威未嘗以為言。及卒,太妃臨哭,人始知同宗也,當時益稱其慎重。



 李瓊李瓊,滄州饒安人也。少為騎將,與晉高祖隸唐明宗麾下。同光二年,契丹犯塞,明宗出涿州,遇契丹,與戰不勝,諸將各稍引去,而晉高祖獨戰不已,契丹漸合而圍之。瓊引高祖衣與俱遁,至劉李河而追兵且及,瓊浮水
 先至南岸,高祖至河中流,馬踣,瓊以長矛援出之,又以所乘馬與高祖,而步護之,走十餘里,乃得免。



 明宗兵變于魏而南,瓊從高祖以三百騎先趨汴州。高祖為保義軍節度使,以為牙隊指揮使。高祖建國,以為護聖都虞候,賜與金帛甚厚,而不與之官爵,瓊亦鬱鬱。



 久之,拜相、中二州刺史。出帝時,為棣州刺史。楊光遠反,以書招瓊,瓊拒而不納。遷洺州團練使,又為護聖右廂都指揮使。晉亡,契丹入京師,以瓊為威州刺史,行至鄭州,遇盜見殺。



 劉景巖劉景巖,延州人也。其家素富,能以貲交游豪俊。事高萬
 金,為部曲,其後為丹州刺史。晉高祖起兵太原,唐廢帝調民七戶出一卒為義兵。延州節度使楊漢章發鄉民赴京師,將行,景巖遣人激怒之,義兵亂,殺漢章,迎景巖為留後。晉高祖即位,即拜景巖節度使。景巖從事熊皦,為人多智,陰察景巖跋扈難制,懼其有異心,欲以利愚之,因語景巖,以謂邊地不可以久安,為陳保名享利之策,言邠、涇多善田,其利百倍,宜多市田射利以自厚。景巖信之,歲餘,其獲甚多。景巖使皦朝京師,皦乃言:「景巖不宜在邊,可徙之內地。」乃移景巖邠州,皦入拜補闕,而景巖又徙鎮保義,居未幾,又徙武勝。景巖乃悟皦為賣
 己,遂誣奏皦隱己玉帶,皦坐貶商州上津令。皦懼景巖邀害之,道亡,匿山中。開運三年,景巖罷武勝,以太子太師致仕,居華州。契丹犯京師,以周密鎮延州,景巖乃還故里。而州人逐密,立高允權,允權妻劉氏,景巖孫女子也。景巖良田甲第、僮僕甚盛,黨項司家族畜牧近郊,尤富強,景巖與之往來,允權頗患之。允權妻歲時歸省,景巖謂曰:「高郎一縣令,而有此州,其可保乎?」允權益惡之,而心又利其田宅,乃誣其反而殺之,年八十餘。



 長子行琮,德州刺史,罷,留京師,亦被誅。次子行謙,允權婦翁也,為奏言非劉氏子,遂免不誅。



\end{pinyinscope}