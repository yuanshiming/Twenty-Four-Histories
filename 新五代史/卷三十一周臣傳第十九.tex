\article{卷三十一周臣傳第十九}

\begin{pinyinscope}

 王樸
 王朴,字文伯,東平人也。少舉進士,為校書郎,依漢樞密使楊邠。邠與王章、史弘肇等有隙,朴見漢興日淺,隱帝年少孱弱,任用小人,而邠為大臣,與將相交惡,知其必亂,乃去邠東歸。後李業等教隱帝誅權臣,邠與章、弘肇皆見殺,三家之客多及,而朴以故獨免。



 周世宗鎮澶州,朴為節度掌書記。世宗為開封尹,拜朴右拾遺,為推官。世宗即位,遷比部郎中,獻《平邊策》,曰:唐失道而失吳、蜀,
 晉失道而失幽、并。觀所以失之之由,知所以平之之術。



 當失之時,君暗政亂,兵驕民困,近者姦於內,遠者叛於外,小不制而至于僭,大不制而至于濫,天下離心,人不用命,吳、蜀乘其亂而竊其號,幽、并乘其間而據其地。平之之術,在乎反唐、晉之失而已。必先進賢退不肖,以清其時;用能去不能,以審其材;恩信號令,以結其心;賞功罰罪,以盡其力;恭儉節用,以豐其財;徭役以時,以阜其民。俟其倉廩實、器用備、人可用而舉之。彼方之民,知我政化大行,上下同心,力彊財足,人安將和,有必取之勢,則知彼情狀者願為之間諜,知彼山川者願為之先導。
 彼民與此民之心同,是與天意同;與天意同,則無不成之功。



 攻取之道,從易者始。當今惟吳易圖,東至海,南至江,可撓之地二千里。從少備處先撓之,備東則撓西,備西則撓東,彼必奔走以救其弊,奔走之間,可以知彼之虛實、眾之彊弱,攻虛擊弱,則所向無前矣。勿大舉,但以輕兵撓之。彼人怯弱,知我師入其地,必大發以來應,數大發則民困而國竭,一不大發則我獲其利。



 彼竭我利,則江北諸州乃國家之所有也。既得江北,則用彼之民,揚我之兵,江之南亦不難而平之也。如此,則用力少而收功多。得吳,則桂、廣皆為內臣,岷、蜀可飛書而召之。如不
 至,則四面並進,席卷而蜀平矣。吳、蜀平,幽可望風而至。



 唯并必死之寇,不可以恩信誘,必須以彊兵攻,力已竭,氣已喪,不足以為邊患,可為後圖。方今兵力精練,器用具備,群下知法,諸將用命,一稔之後,可以平邊。



 臣書生也,不足以講大事,至於不達大體,不合機變,惟陛下寬之。



 遷左諫議大夫,知開封府事。歲中,遷左散騎常侍,充端明殿學士。是時,世宗新即位,銳意征伐,已撓群議,親敗劉旻於高平,歸而益治兵,慨然有平一天下之志。數顧大臣問治道,選文學之士徐台符等二十人,使作《為君難為臣不易論》及《平邊策》,朴在選中。而當時文士皆
 不欲上急於用武,以謂平定僭亂,在修文德以為先。惟翰林學士陶穀竇儀、御史中丞楊昭儉與朴皆言用兵之策,朴謂江淮為可先取。世宗雅已知朴,及見其議論偉然,益以為奇,引與計議天下事,無不合,遂決意用之。顯德三年,征淮,以僕為東京副留守。還,拜戶部侍郎、樞密副使,遷樞密使。四年,再徵淮,以朴留守京師。



 世宗之時,外事征伐,而內修法度。朴為人明敏多材智,非獨當世之務,至於陰陽律歷之法,莫不通焉。顯德二年,詔朴校定大歷,乃削去近世符天流俗不經之學,設通、經、統三法,以歲軌離交朔望周變率策之數,步日月五星,為《
 欽天歷》。



 六年,又詔朴考正雅樂,朴以謂十二律管互吹,難得其真,乃依京房為律準,以九尺之絃十三,依管長短寸分設柱,用七聲為均,樂成而和。



 朴性剛果,又見信於世宗,凡其所為,當時無敢難者,然人亦莫能加也。世宗征淮,朴留京師,廣新城,通道路,莊偉宏闊,今京師之制,多其所規為。其所作樂,至今用之不可變。其陳用兵之略,非特一時之策。至言諸國興滅次第云:「淮南可最先取,并必死之寇,最後亡。」其後宋興,平定四方,惟并獨後服,皆如朴言。



 六年春,世宗遣朴行視汴口,作斗門,還,過故相李穀第,疾作,仆于坐上,舁歸而卒,年五十四。世
 宗臨其喪,以玉鉞叩地,大慟者數四。贈侍中。



 鄭仁誨鄭仁誨,字日新,太原晉陽人也。初,事唐將陳紹光。紹光為人驍勇而好使酒,嘗因醉怒仁誨,拔劍欲殺之,左右皆奔走,仁誨植立不動,無懼色,紹光擲劍于地,撫仁誨曰:「汝有器量,必富貴,非吾所及也。」仁誨後棄紹光去,還鄉里,事母以孝聞。漢高祖為河東節度使,周太祖居帳下,時時往過仁誨,與語甚懽。每事有疑,即從仁誨質問,仁誨所對不阿,周太祖益奇之。漢興,周太祖為樞密使,乃召仁誨用之,累官至內客省使。太祖破李守貞於河中,軍中機畫,仁誨多所參決。太祖入立,以仁誨為大內
 都點檢、恩州團練使、樞密副使,累遷宣徽北院使,出為鎮寧軍節度使。顯德元年,拜樞密使。世宗攻河東,仁誨留守東都。明年冬,以疾卒。



 世宗將臨其喪,有司言歲不利臨喪,世宗不聽,乃先以桃荝而臨之。



 仁誨自其微時,常為太祖謀畫,及居大位,未嘗有所聞,而太祖、世宗皆親重之,然亦能謙謹好禮,不自矜伐,為士大夫所稱。贈中書令,追封韓國公,謚曰忠正。



 扈載扈載,字仲熙,北燕人也。少好學,善屬文。廣順初,舉進士高第,拜校書郎,直史館。再遷監察御史。其為文章,以辭多自喜。常次歷代有國廢興治亂之跡為《運源賦》,甚詳。
 又因遊相國寺,見庭竹可愛,作《碧鮮賦》,題其壁,世宗聞之,遣小黃門就壁錄之,覽而稱善,因拜水部員外郎、知制誥。遷翰林學士,賜緋,而載已病,不能朝謝。居百餘日,乃力疾入直學士院。世宗憐之,賜告還第,遣太醫視疾。



 初,載以文知名一時,樞密使王朴尤重其才,薦於宰相李穀,久而不用,朴以問穀曰:「扈載不為舍人,何也?」穀曰:「非不知其才,然載命薄,恐不能勝。」



 朴曰:「公為宰相,以進賢退不肖為職,何言命邪?」已而召拜知制誥。及為學士,居歲中病卒,年三十六。議者以穀能知人而朴能薦士。



 是時,天子英武,樂延天下奇才,而尤禮文士,載與張昭、
 竇儼、陶穀、徐台符等俱被進用。穀居數人中,文辭最劣,尤無行。昭、儼數與論議,其文粲然,而穀徒能先意所在,以進諛取合人主,事無大小,必稱美頌贊,至於廣京城,為木偶耕人、紫芝白兔之類,皆為頌以獻,其辭大抵類俳優。而載以不幸早卒,論議雖不及昭、儼,而不為穀之諛也。



 嗚呼!作器者,無良材而有良匠;治國者,無能臣而有能君。蓋材待匠而成,臣待君而用。故曰,治國譬之於奕,知其用而置得其處者勝,不知其用而置非其處者敗。敗者臨棋注目,終日而勞心,使善奕者視焉,為之易置其
 處則勝矣。勝者所用,敗者之棋也;興國所用,亡國之臣也。王朴之材,誠可謂能矣。不遇世宗,何所施哉?世宗之時,外事征伐,攻取戰勝;內修制度,議刑法,定律歷,講求禮樂之遺文,所用者五代之士也,豈皆愚怯於晉、漢,而材智於周哉?惟知所用爾。夫亂國之君,常置愚不肖於上,而彊其不能,以暴其短惡,置賢智於下,而泯沒其材能,使君子、小人皆失其所,而身蹈危亡。治國之君,能置賢智於近,而置愚不肖於遠,使君子、小人各適其分,而身享安榮。治亂相去雖遠甚,而其所以致之者不多也,反其所置而已。嗚呼,自古治君少而亂君多,況於五代,
 士之遇不遇者,可勝嘆哉!



\end{pinyinscope}