\article{卷四十五雜傳第三十三}

\begin{pinyinscope}

 張全義張全義,字國維,濮州臨濮人也。少以田家子役于縣,縣令數困辱之,全義因亡入黃巢賊中。巢陷長安,以全義為吏部尚書、水運使。巢賊敗,去事諸葛爽於河陽。爽死,事其子仲方。仲方為孫儒所逐,全義與李罕之分據河陽、洛陽以附于梁,二人相得甚歡。然罕之性貪暴,日以寇鈔為事。全義勤儉,御軍有法,督民耕殖。



 以故,罕之常乏食,而全義常有餘。罕之仰給全義,全義不能給,二人因
 有隙。罕之出兵攻晉、絳,全義襲取河陽,罕之奔晉,晉遣兵助罕之,圍全義甚急。全義乞兵于梁,梁遣牛存節、丁會等以兵萬人自九鼎渡河,擊敗罕之於沇水,晉軍解去。



 梁以丁會守河陽,全義還為河南尹。全義德梁出己,由是盡心焉。



 是時,河南遭巢、儒兵火之後,城邑殘破,戶不滿百,全義披荊棘,勸耕殖,躬載酒食,勞民畎畝之間,築南、北二城以居之。數年,人物完盛,民甚賴之。及梁太祖劫唐昭宗東遷,繕理宮闕、府廨、倉庫,皆全義之力也。全義初名言,唐昭宗賜名全義。唐亡,全義事梁,又請改名,太祖賜名宗奭。太祖猜忌,晚年尤甚,全義奉事益謹,
 卒以自免。



 自梁與晉戰河北,兵數敗亡,全義輒搜卒伍鎧馬,月獻之以補其缺。太祖兵敗蓚縣,道病,還洛,幸全義會節園避暑,留旬日,全義妻女皆迫淫之。其子繼祚憤恥不自勝,欲剚刃太祖,全義止之曰:「吾為李罕之兵圍河陽,啖木屑以為食,惟有一馬,欲殺以餉軍,死在朝夕,而梁兵擊之,得至今日,此恩不可忘也。」繼祚乃止。



 嘗有言全義於太祖者,太祖召全義,其意不測。全義妻儲氏明敏有口辯,遽入見,厲聲曰:「宗奭,種田叟爾!守河南三十年,開荒斫土,捃拾財賦,助陛下創業,今年齒衰朽,已無能為,而陛下疑之,何也?」太祖笑曰:「我無惡心,嫗勿
 多言。」全義事梁,累拜中書令,食邑至萬三千戶,兼領忠武陜虢鄭滑河陽節度使、判六軍諸衛事、天下兵馬副元帥,封魏王。



 初,全義為李罕之所敗,其弟全武及其家屬為晉兵所得,晉王給以田宅,待之甚厚,全義常陰遣人通問於太原。及梁亡,莊宗入汴,全義自洛來朝,泥首待罪,莊宗勞之曰:「卿家弟姪,幸復相見。全義俯伏感涕。年老不能進趨,遣人掖扶而登,宴犒盡歡,命皇子繼岌、皇弟存紀等皆兄事之。全義因去梁所賜名,請復其故名。而全義猶不自安,乃厚賂劉皇后以自託。



 初,梁末帝幸洛陽,將祀天於南郊而不果,其儀仗法物猶在,全義
 因請幸洛陽,白南郊儀物已具。莊宗大悅,加拜全義太師、尚書令。明年十一月,莊宗幸洛陽,南郊而禮物不具,因改用來年二月,然不以前語責全義。以皇后故,待之愈厚,數幸其第,命皇后拜全義為父,改封齊王。



 初,莊宗滅梁,欲掘梁太祖墓,斫棺戮尸。全義以謂梁雖仇敵,今已屠滅其家,足以報怨,剖棺之戮,非王者以大度示天下也。莊宗以為然,鏟去墓闕而已。



 全義監軍嘗得李德裕平泉醒酒石,德裕孫延古,因託全義復求之。監軍忿然曰:「自黃巢亂後,洛陽園宅無復能守,豈獨平泉一石哉!」全義嘗在巢賊中,以為譏己,固大怒,奏笞殺監軍者,
 天下冤之。其聽訟,以先訴者為直,民頗以為苦。



 同光四年,趙在禮反於魏,元行欽討賊無功,莊宗欲自將討之,大臣皆諫以為不可,因言明宗可將。是時,郭崇韜、朱友謙皆已見殺,明宗自鎮州來朝,處之私第,莊宗疑之,不欲遣也。群臣固請,不從;最後全義力以為言,莊宗乃從。已而明宗至魏果反,全義以憂卒,年七十五,謚曰忠肅。



 子繼祚,官至上將軍。晉高祖時,與張從賓反於河陽,當族誅。而宰相桑維翰以其父珙嘗事全義有恩,乞全活之,不許,止誅繼祚及其妻子而已。



 朱友謙硃友謙,字德光,許州人也。初名簡,以卒隸澠池鎮,有罪
 亡去,為盜石濠、三鄉之間,商旅行路皆苦之。久之,去,為陜州軍校。陜州節度使王珙,為人嚴酷,與其弟珂爭河中,戰敗,其牙將李璠與友謙謀,共殺珙,附于梁,太祖表璠代珙。



 璠立,友謙復以兵攻之,璠得逃去,梁太祖又表友謙代璠。梁兵西攻李茂貞,太祖往來過陜,友謙奉事尤謹,因請曰:「僕本無功,而富貴至此,元帥之力也。且幸同姓,願更名以齒諸子。」太祖益憐之,乃更其名友謙,錄以為子。太祖即位,徙鎮河中,累遷中書令,封冀王。



 太祖遇弒,友珪立,加友謙侍中,友謙雖受命,而心常不平。已而友珪使召友謙入覲,友謙不行,乃附于晉。友珪遣
 招討使韓勍將康懷英等兵五萬擊友謙。晉王出澤、潞以救之,遇懷英于解縣,大敗之,追至白逕嶺,夜秉炬擊之,懷英又敗,梁兵乃解去。友謙醉寢晉王帳中,晉王視之,顧左右曰:「冀王雖甚貴,然恨其臂短耳!」



 末帝即位,友謙復臣于梁而不絕晉也。貞明六年,友謙遣其子令德襲同州,逐節度使程全暉,因求兼鎮。末帝初不許,已而許之,制命未至,友謙復叛,始絕梁而附晉矣。末帝遣劉鄩等討之,鄩為李存審所敗。晉封友謙西平王,加守太尉,以其子令德為同州節度使。



 莊宗滅梁入洛,友謙來朝,賜姓名曰李繼麟,賜予巨萬。明
 年,加守太師、尚書令,賜鐵券恕死罪。以其子令德為遂州節度使,令錫忠武軍節度使,諸子及其將校為刺史者十餘人,恩寵之盛,時無與比。是時宦官、伶人用事,多求賂于友謙,友謙不能給而辭焉,宦官、伶人皆怒。唐兵伐蜀,友謙閱其精兵,命其子令德將以從軍。及郭崇韜見殺,伶人景進言:「唐兵初出時,友謙以為討己,閱兵自備。」



 又言:「與崇韜謀反。」且曰:「崇韜所以反於蜀者,以友謙為內應。友謙見崇韜死,謀與存乂為郭氏報冤。」莊宗初疑其事,群伶、宦官日夜以為言。友謙聞之大恐,將入朝以自明,將吏皆勸其毋行。友謙曰:「郭公有大功於國,而
 以讒死,我不自明,誰為我言者!」乃單車入朝。景進使人詐為變書,告友謙反。莊宗惑之,乃徙友謙義成軍節度使,遣朱守殷夜以兵圍其館,驅友謙出徽安門外,殺之,復其姓名。詔魏王繼岌殺令德於遂州,王思同殺令錫於許州,夏魯奇族其家屬于河中。



 魯奇至其家,友謙妻張氏率其宗族二百餘口見魯奇曰:「朱氏宗族當死,願無濫及平人。」乃別其婢僕百人,以其族百口就刑。張氏入室取其鐵券示魯奇曰:「此皇帝所賜也,不知為何語!」魯奇亦為之慚。



 友謙死,其將史武等七人皆坐友謙族誅,天下冤之。



 袁象先袁象先,宋州下邑人,唐南陽王恕己之後也。父敬初,梁太府卿、駙馬都尉,尚太祖妹,是為萬安大長公主。象先以梁甥為宣武軍內外馬步軍都指揮使,歷宿、洺、陳三州刺史。太祖即位,累遷左龍武統軍、在京馬步軍都指揮使。



 太祖遇弒,友珪立。末帝留守東都,以大事謀於趙巖,巖曰:「此事如反掌耳,但得招討楊令公一言諭禁軍,則事可成。」末帝即遣人之魏州,以謀告楊師厚,師厚遣裨將王舜賢至洛陽與象先謀,象先許諾。是時,龍驤軍將劉重遇戍于懷州,以其軍作亂,友珪遣霍彥威擊敗于鄢陵,其餘兵奔散,捕之甚急。末帝即召龍驤軍在東京者
 告之曰:「上以重遇故,欲盡召龍驤軍至洛而誅之。」乃偽為友珪詔書示之,龍驤軍恐懼,不知所為,因告之曰:「友珪弒父與君,天下之賊也!爾能趨洛陽擒之,以其首祭先帝,則所謂轉禍而為福也。」軍士踴躍曰:「王言是也。」末帝即馳奏,言:「龍驤軍反。」象先聞之,即引禁軍千人入宮攻友珪,友珪死。末帝即位,拜象先鎮南軍節度使、同中書門下平章事、開封尹、判在京馬步軍諸軍事。貞明四年,為平盧軍節度使,徙鎮宣武。



 象先為梁將,未嘗有戰功,徒以甥故掌親軍。及誅友珪,有功於末帝。在宋州十餘年,誅斂其民,積貸千萬。莊宗滅梁,象先來朝洛陽,輦
 其資數十萬,賂唐將相、伶官、宦者及劉皇后等,由是內外翕然稱其為人。莊宗待之甚厚,賜姓名為李紹安,改宣武軍為歸德軍,曰:「歸德之名,為卿設也。」遣之還鎮。是歲卒,年六十,贈太師。



 象先二子,正辭官至刺史,泬周世宗時為橫海軍節度使。象先平生所積財產數千萬,邸舍四千間,其卒也,不以分諸子,而悉與正辭。正辭初以父任為飛龍副使。



 唐廢帝時,獻錢五萬緡,領衢州刺史。晉高祖入立,復獻五萬緡,求為真刺史。拜雄州刺史,州在靈武之西,吐蕃界中。正辭憚,不欲行,復獻錢數萬,乃得免。正辭不勝其忿,以衣帶自經,其家人救之而止。
 出帝時,又獻錢三萬緡、銀萬兩,出帝憐之,欲與一內郡,未及而卒。



 正辭積錢盈室,室中嘗有聲如牛,人以為妖,勸其散積以禳之。正辭曰:「吾聞物之有聲,求其同類爾,宜益以錢,聲必止。」聞者傳以為笑。



 朱漢賓硃漢賓,字績臣,亳州譙人也。其父元禮為軍校,從梁軍戰,歿于清口。漢賓為人有膽力,梁太祖以其父死戰,憐之,以為養子。是時,梁方東攻兗、鄆,鄆州朱瑾募其軍中驍勇者,黥雙鴈于其頰,號「鴈子都」。太祖聞之,乃更選勇士數百人,號「落雁都」,以漢賓為指揮使。及漢賓貴,人猶以為「朱落鴈」。漢賓事梁為天威軍使,歷磁滑宋亳曹五
 州刺史、安遠軍節度使。莊宗滅梁,罷漢賓為右龍武統軍,待之頗薄。後莊宗因出游幸其第,漢賓妻有色而惠,因侍左右,進酒食,奏歌舞,莊宗懽甚,留至夜漏二更而去,漢賓自此有寵。初,漢賓在梁也,與朱友謙俱為太祖養子,而友謙年長,漢賓以兄事之。其後梁亡,漢賓數寓書友謙,友謙不答,漢賓銜之。其後友謙見族,人皆以為漢賓有力。明宗入立,以漢賓為莊宗所厚,惡之,以為右衛上將軍。安重誨用事,漢賓依附之,相為婚姻,由是復得為昭義軍節度使。重誨死,漢賓罷為上將軍,遂以太子少保致仕。漢賓為將,未嘗有戰功,而臨政能守法,好
 施惠,人頗愛之。清泰二年卒,年六十四。晉高祖時,贈太子少傅,謚曰貞惠。



 段凝段凝,開封人也。初名明遠,後更名凝。為澠池主簿。其父事梁太祖,以事坐徙。後凝棄官,亦事太祖,為軍巡使。又以其妹內太祖,妹有色,後為美人。凝為人憸巧,善窺迎人意,又以妹故,太祖漸親信之,常使監諸軍。為懷州刺史,梁太祖北征,過懷州,凝獻饋甚豐,太祖大悅。過相州,相州刺史李思安獻饋如常禮,比凝為薄,太祖怒,思安因以得罪死。遷凝鄭州刺史,使監兵於河上。李振亟請罷之,太祖曰:「凝未有罪。」振曰:「待其有罪,則社稷亡矣!」然
 終不罷也。



 莊宗已下魏博,與梁相距河上。梁以王彥章為招討使,凝為副。是時,末帝昏亂,小人趙巖、張漢傑等用事,凝依附巖等為姦。彥章為招討使,三日,用奇計破唐德勝南城。而凝與彥章各自上其功,巖等從中匿彥章功狀,悉歸其功於凝。凝因納金巖等,求代彥章,末帝惑巖等言,卒以凝為招討使,軍于王村。是時,唐已下鄆州,凝乃自酸棗決河東注鄆,以隔絕唐軍,號「護駕水」。莊宗自鄆趨汴,汴兵悉已屬凝,京師無備,乃遣張漢倫馳馹召凝于河上,漢倫中道墜馬,傷不能進。已而梁亡,凝率精兵五萬降唐,莊宗賜以錦袍、御馬。明日,凝奏:「故梁
 姦人趙巖、張漢傑等十餘人侮弄權柄,殘害生靈,請皆族之。」凝出入唐朝無隗色,見唐將相若倡優,因伶人景進納賂劉皇后,以求恩寵。莊宗甚親愛之,賜姓名曰李紹欽,以為泰寧軍節度使。居月餘,用庫錢數十萬,有司請責其償,莊宗釋之。郭崇韜固請,以為不可,莊宗怒曰:「朕為卿所制,都不自由!」終釋之。



 莊宗遣李紹宏監諸將備契丹,凝軍瓦橋關,以諂事紹宏,紹宏數薦凝可大用,郭崇韜每以為不可。遷武勝軍節度使。趙在禮反,紹宏請以凝招討,莊宗使凝條奏方略,凝所請偏裨,皆其故黨,莊宗疑之,乃止。明宗即位,勒歸田里。明年,長流遼州,
 賜死。



 劉玘劉玘,汴州雍丘人也,世為宣武軍牙將。梁太祖鎮宣武,玘以軍卒補隊長,稍以戰功遷牙將,為襄州都指揮使。山南節度使王班為亂軍所殺,亂軍推玘為留後,玘偽許之,明日饗士于庭,伏甲幕中,酒半,擒為亂者殺之。會梁遣陳暉兵亦至,襄州平,以功拜復州刺史,徙亳、安二州。末帝時,為晉州觀察留後,凡八年,日與晉人交戰。莊宗滅梁,玘來朝,莊宗勞之曰:「劉侯亡恙,爾居晉陽之南鄙久矣,不早相聞,今日見訪不其晚邪?」玘頓首謝罪,遣還鎮,遂以為節度使,徙鎮安遠。



 天成元年,以史敬鎔代
 之,玘還京師,未至,拜武勝軍節度使,以疾卒于道中,贈侍中。



 周知裕周知裕,字好問,幽州人也。為劉仁恭騎將,仁恭為其子守光所囚,知裕去事守光兄守文。守光又攻殺守文,乃與張萬進立守文子延祚而事之。守光又殺廷祚,以其子繼威代之。萬進殺繼威,與知裕俱奔于梁。梁太祖得知裕喜甚,為置歸化軍,以知裕為指揮使,凡與晉戰所得,及兵背晉而歸梁者,皆以隸知裕。梁、晉相拒河上十餘年,其摧堅陷陣,歸化一軍為最,然知裕位不過刺史。莊宗入汴,知裕與段凝軍河上,聞梁已亡,欲自殺,為賓
 客故人止之,乃降唐。莊宗尤寵待之,諸將嫉其寵,因獵射之,知裕走以免。莊宗為殺射者,以知裕為房州刺史。明宗時,歷絳、淄二州刺史,遷宿州團練使、安州留後。所居皆有善政。安州近淮,俗惡病者,父母有疾,置之佗室,以竹竿繫飲食委之,至死不近。知裕深患之,加以教道,由是稍革。罷為右神武統軍。應順中卒,贈太傅。



 陸思鐸陸思鐸,澶州臨黃人也。少事梁為宣武軍卒,以善射知名。累遷拱辰左廂都指揮使,領恩州刺史。梁、晉相拒河上,思鐸鏤其姓名於箭筈以射晉軍,而矢中莊宗馬鞍,莊宗拔矢,見思鐸姓名,奇之。其後滅梁,思鐸謁見,莊宗
 出其矢以示之,思鐸伏地請死,莊宗慰而起之,拜龍武右廂都指揮使。晉高祖時,為陳、蔡二州刺史。卒年五十四。思鐸在陳州,有善政,臨終戒其子曰:「陳人愛我,我死則葬焉。」



 遂葬於陳
 州。



\end{pinyinscope}