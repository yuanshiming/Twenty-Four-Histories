\article{卷三十六義兒傳第二十四}

\begin{pinyinscope}

 嗚呼!世道衰,人倫壞,而親疏之理反其常,干戈起於骨肉,異類合為父子。



 開平、顯德五十年間,天下五代而實八姓,其三出於丐養。蓋其大者取天下,其次立功名、位將相,豈非因時之隙,以利合而相資者邪!唐自號沙陀,起代北,其所與俱皆一時雄傑虣武之士,往往養以為兒,號「義兒軍」,至其有天下,多用以成功業,及其亡也亦由焉。太祖養子多矣,其可紀者九人,其一是為明宗,其
 次曰嗣昭、嗣本、嗣恩、存信、存孝、存進、存璋、存賢。作《義兒傳》。



 李嗣昭李嗣昭,本姓韓氏,汾州太谷縣民家子也。太祖出獵,至其家,見其林中鬱鬱有氣,甚異之,召其父問焉。父言家適生兒,太祖因遺以金帛而取之,命其弟克柔養以為子。初名進通,後更名嗣昭。嗣昭為人短小,而膽勇過人。初喜嗜酒,太祖嘗微戒之,遂終身不飲。太祖愛其謹厚,常從用兵,為衙內指揮使。



 陜州王珙與其兄珂爭立於河中,遣嗣昭助珂,敗珙於猗氏,獲其將三人。梁軍救珙,嗣昭又敗之于胡壁堡,執其將一人。光化元年,澤州李
 罕之襲潞州以降梁,梁遣丁會應罕之,嗣昭與會戰含山,執其將一人,斬首三千級,遂取澤州。二年,晉遣李君慶攻梁潞州,君慶為梁所敗,太祖鴆殺君慶,嗣昭攻克之。三年,出山東,取梁洺州,梁太祖自將攻之,遣葛從周設伏於青山口。嗣昭聞梁太祖自來,棄城走,前遇伏兵,因大敗。



 天復元年,梁破河中,執王珂,取晉、絳、慈、隰,因大舉擊晉,圍太原。嗣昭日以精騎出擊梁兵,會大雨,梁軍解去。晉汾州刺史李瑭叛降梁軍,梁軍已去,嗣昭復取汾州,斬瑭。遂出陰地,取慈州,降其刺史唐禮。又取隰州,降其刺史張瑰。是歲,梁軍西犯京師,圍鳳翔,嗣昭乘間
 攻梁晉、絳,戰平陽,執梁將一人。



 進攻蒲縣。梁朱友寧、氏叔琮以兵十萬迎擊之,嗣昭等敗走,友寧追之,晉遣李存信率兵迎嗣昭,存信又敗。梁軍遂圍太原,而慈、隰、汾州復入于梁。太祖大恐,謀走雲州,李存信等勸太祖奔于契丹,嗣昭力爭以為不可,賴劉太妃亦言之,乃止。



 嗣昭晝夜出奇兵擊梁軍,梁軍解去,嗣昭復取汾、慈、隰。是歲,鎮、定皆已絕晉而附梁。晉外失大國之援,內亡諸州,仍歲之間,孤城被圍者再。於此時,嗣昭力戰之功為多。



 天祐三年,與周德威攻梁潞州,降丁會,以嗣昭為昭義軍節度使。梁遣李思安將兵十萬攻潞,築夾城以圍之。梁
 太祖遣人招降嗣昭,嗣昭斬其使者,閉城拒守,踰年,莊宗始攻破夾城。嗣昭完緝兵民,撫養甚有恩意。梁、晉戰胡柳,晉軍敗,周德威戰死。莊宗懼,欲收兵還臨濮,嗣昭曰:「梁軍已勝,旦暮思歸。吾若收軍,使彼休息,整而復出,何以當之?宜以精騎撓之,因其勞乏,可以勝也。」莊宗然之。是時,梁軍已登無石山,莊宗遣嗣昭轉擊山北,而自以銀槍軍趨而曰:「今日之戰,得山者勝!」晉軍皆爭登山,梁軍遽下,陣於山西,晉軍從上急擊,大敗之。



 於是晉城德勝矣。周德威死,嗣昭權知幽州,居數月,以李紹宏代之。嗣昭將去,幽州人皆號哭閉關遮留之,嗣昭夜
 遁,乃得去。



 十九年,從莊宗擊契丹於望都,莊宗為契丹圍之數十重,嗣昭以三百騎決圍,取莊宗以出。是時,晉遣閻寶攻張文禮於鎮州,寶為鎮人所敗,乃以嗣昭代之。鎮兵出掠九門,嗣昭以奇兵擊之,鎮軍且盡,餘三人匿破垣中,嗣昭馳馬射之,反為賊射中腦,嗣昭顧箙中矢盡,拔矢于腦,射殺一人,還營而卒。



 嗣昭諸子,繼儔長而懦,其弟繼韜囚之以自立,莊宗方與梁兵相持河上,不暇究其事,因即以為昭義軍留後。繼韜委其政於魏琢、申蒙,琢等常教繼韜反,繼韜未決。莊宗在魏,以事召監軍張居翰、節度判官任圜。琢等以謂莊宗召居翰等
 問繼韜事,繼韜且見誅,因以語趣之,繼韜乃遣其弟繼遠入梁,梁末帝即拜繼韜同中書門下平章事。居數月,莊宗滅梁,繼韜將走契丹,會赦至,乃已,因隨其母朝于京師,繼遠諫曰:「兄為臣子,以反為名,復何面以見天子?且潞城堅而倉廩實,不如閉城坐食積粟,以延歲月,愈於往而就戮也。」繼韜不聽。繼韜母楊氏,善畜財,平生居積行販,至貲百萬。當嗣昭為梁圍以夾城彌年,軍用乏絕,楊氏之積,蓋有助焉。至是,乃齎銀數十萬兩至京師,厚賂宦官、伶人,宦官、伶人皆言:「繼韜初無惡意,為奸人所誤耳。」楊夫人亦以賂謁劉皇后,劉皇后為言:「嗣昭功
 臣,宜蒙恩貸。」由是莊宗釋繼韜。嘗從獵,寵倖無間。李存渥尤切齒,數詆責之,繼韜懷不自安,復賂宦官、伶人求歸鎮,莊宗不許。繼韜陰使人告繼遠,令起變於軍中,冀天子遣己往安緝之,事泄,斬于天津橋。其二子嘗為質于梁,莊宗破梁得之,撫其背曰:「爾幼,猶能佐其父反,長復何為乎?」至是因并誅之。即遣人斬繼遠,以繼儔知潞州事。已而召繼儔還京師,繼儔悉取繼韜妓妾珍玩,而不時即路。其弟繼達怒曰:「吾兄父子誅死,而大兄不仁,利其貲財,淫其妻妾,吾所不忍也!」



 乃服縗麻,引數百騎坐戟門,使人入殺繼儔。節度副使李繼珂募市
 人千餘攻繼達,繼達走城外,自剄死。



 嗣昭七子,至明宗時,子繼能坐笞殺其母主藏婢,婢家告變,言繼能反,與其弟繼襲皆見殺,惟一子繼忠僅免。繼忠家於晉陽,楊氏所積餘貲猶巨萬,晉高祖自太原起兵,召契丹為援,契丹求賂,高祖貸于繼忠以取足。高祖入立,甚德之,以為沂、棣、單三州刺史,開運中卒。楊氏平生積產,嗣昭父子三人賴之。



 嗣本嗣本,本姓張氏,鴈門人也。世為銅冶鎮將。嗣本少事太祖,太祖愛之,賜以姓名,養為子。從擊居庸關,以功遷義兒軍使。從破王行瑜,遷威遠軍使。從攻羅弘信,以先鋒
 兵破湯陰。從莊宗破潞州夾城。累以戰功遷代州刺史、雲州防禦使、振武節度使,號威信可汗。天祐十三年,從莊宗擊劉鄩於故元城,下洺、磁諸州,六月,還軍振武。契丹入代北,攻蔚州,嗣本戰歿。



 嗣恩嗣恩,本姓駱,吐谷渾部人也。少事太祖,能騎射,為鐵林軍將,稍以戰功遷突陣指揮使,賜姓名,以為子。從敗康懷英於河西,遷左廂馬軍都指揮使。從李嗣昭援朱友謙於河中,與梁兵力戰,槊中其口,戰不已。遷遼州刺史。從莊宗入魏,遷天雄軍馬步都指揮使。劉鄩攻太原,兵趣樂平,嗣恩從後追之,自佗道先入太原以守。鄩兵去,
 嗣恩亦以兵會莊宗于魏,從戰于莘。遷代州刺史、石嶺關已北都知兵馬使、振武節度使。天祐十五年,卒于太原。追贈太尉。



 存信存信,本姓張氏,其父君政,回鶻李思忠之部人也。存信少善騎射,能四夷語,通六蕃書。從太祖起代北,入關破黃巢,累以功為馬步軍都指揮使,遂賜姓名,以為子。存信與存孝俱為養子,材勇不及存孝,而存信不為之下,由是交惡,存孝所為,存信每沮激之,存孝卒得罪死。而存信數從征伐,以功領郴州刺史。太祖遣將兵救朱宣,存信屯于莘縣,為羅弘信所擊,存信敗,亡太祖子落
 落。後從太祖討劉仁恭,大敗于安塞。太祖大怒,顧存信曰:「昨日吾醉,公不能為我戰邪?古人三敗,公已二矣。」將殺之,存信叩頭謝罪而免。由是大懼,常稱疾,天復二年卒,年四十一。



 存孝存孝,代州飛狐人也。本姓安,名敬思。太祖掠地代北得之,給事帳中,賜姓名,以為子,常從為騎將。文德元年,河南張言襲破河陽,李罕之來歸晉,晉處罕之于澤州,遣存孝與薛阿檀、安休休等以兵七千助罕之還擊河陽。梁亦遣丁會、牛存節等助言。戰于溫縣,梁軍先扼太行,存孝大敗,安休休被執。是時,晉已得澤、潞,歲出山東,與
 孟方立爭邢、洺、磁,存孝未嘗不在兵間。方立死,晉取三州,存孝功為多。



 明年,潞州軍亂,殺李克恭以歸唐,梁遣李讜攻李罕之于澤州,存孝以騎兵五千救之。梁軍呼罕之曰:「公常恃太原以為命,今上黨已歸唐,唐兵大集,圍太原,沙陀將無穴以自處,公復誰恃而不降乎?」存孝以精騎五百繞梁柵而呼曰:「我沙陀之求穴者,待爾肉以食軍,可令肥者出鬥!」梁驍將鄧季筠引軍出戰,存孝舞槊擒之,李讜敗走,追擊至馬牢關。還攻潞州,唐以孫揆為潞州節度使,揆儒者,以梁卒三千為衛,褒衣大蓋,擁節先驅。存孝以三百騎伏長子西崖谷間,伺揆軍過,
 橫擊斷之,擒揆以歸。初,梁遣葛從周、朱崇節守潞州以待揆,聞揆見執,皆棄去,晉遂復取潞州。是時,張浚、韓建伐晉,擊陰地關,晉以李存信、薛阿檀等當濬,別遣存孝軍於趙城。唐軍戰敗于陰地關,濬退保晉州,韓建走絳州。存孝攻晉州,濬兵出戰,輒復敗,因閉壁不敢出。存孝去,攻絳州。濬、建皆走。



 存孝猨臂善射,身被重鎧,櫜弓坐槊,手舞鐵楇,出入陣中,以兩騎自從,戰酣易騎,上下如飛。初,存孝取潞州功為多,而太祖別以大將康君立為潞州留後,存孝為汾州刺史,存孝負其功,不食者數日。及走張浚,遷邠州刺史。大順二年,徙邢州留後。是時,晉
 軍連歲攻趙常山,存孝常為先鋒,下趙臨城、元氏。趙王求救於幽州李匡威,匡威兵至,晉軍輒引去。存孝素與存信有隙,存信譖之曰:「存孝有二心,常避趙不擊。」存孝不自安,乃附梁通趙,自歸于唐,因請會兵以代晉。



 唐命趙王王鎔援之。明年,趙與幽州有隙,懼而與晉和,反以兵三萬助晉擊存孝。



 存孝嬰城自守。太祖自將兵傅其城,掘塹以圍之,存孝出兵衝擊,塹不得成。裨將袁奉韜使人說存孝曰:「公所畏者,晉王爾。王俟塹成,且留兵去,諸將非公敵也,雖塹何為?」存孝以為然,縱兵成塹。塹成,深溝高壘,不可近,存孝遂窘。城中食盡,登城呼曰:「兒蒙
 王恩,位至將相,豈欲舍父子而附仇讎,乃存信構陷之耳。



 願生見王一言而死。」太祖哀之,遣劉夫人入城慰諭之。劉夫人引與俱來,存孝泥首請罪曰:「兒於晉有功而無過,所以至此,由存信為之耳!」太祖叱曰:「爾為書檄,罪我百端,亦存信為之邪?」縛載後車,至太原,車裂之以徇。然太祖惜其材,悵然恨諸將之不能容也,為之不視事者十餘日。



 康君立素與存信相善,方二人之交惡也,君立每左右存信以傾之。存孝已死,太祖與諸將博,語及存孝,流涕不已,君立以為不然,太祖怒,鴆殺君立。君立初為雲州牙將,唐僖宗時,逐段文楚,與太祖俱起雲中,
 蓋君立首事。其後累立戰功,表昭義節度使,以存孝故殺之。



 存進存進,振武人也,本姓孫,名重進。太祖攻破朔州得之,賜以姓名,養為子。



 從太祖入關破黃巢,以為義兒軍使。從莊宗戰柏鄉,遷行營馬步軍都虞候,歷慈、沁二州刺史。莊宗初得魏博,以為天雄軍都部署,治梁亂軍,一切以法,人有犯者,輒梟首磔尸於市,魏人屏息畏之。從戰河上,以功遷振武軍節度使。是時,晉軍德勝,為南北寨,每以舟兵來往,頗以為勞,而河北無竹石,存進乃以葦笮維大艦為浮梁。莊宗大喜,解衣以賜之。



 晉討張文禮於
 鎮州,久不克,而史建瑭、閻寶、李嗣昭相次戰歿,乃以存進代嗣昭為招討使,軍于東垣渡。東垣土惡,築壘不能就,存進伐木為柵。晉軍晨出芻牧,文禮子處球以兵千餘逼存進柵,存進出戰橋上,殺處球兵殆盡,而存進亦歿於陣。追贈太尉。



 子漢韶,明宗時復本姓,為洋州節度使。潞王從珂以鳳翔反,漢韶與張虔釗會唐軍討之,唐軍皆降于從珂,獨漢韶與虔釗軍不降,俱奔于蜀。事蜀,歷永平、興元、武信節度使。年七十餘,卒於蜀。



 存璋存璋,字德璜,初與康群立、恭志勤等從太祖入關,破黃巢,累遷義兒軍使。



 太祖病革,存璋與張承業等受顧命,
 立莊宗為晉王,晉王以存璋為河東馬步軍使。



 晉自先王時,嘗優假軍士,軍士多犯法踰禁,莊宗新立,尤患之,存璋一切繩之以法,境內為之清肅。從攻夾城,戰柏鄉,以功遷汾州刺史。莊宗與劉鄩戰於魏博,梁遣王檀來,乘虛襲太原,存璋以汾州兵入太原距守,以功遷大同軍防禦使,遂為節度使。天祐十九年以疾卒。追贈太尉。



 存賢存賢,許州人也,本姓王名賢。少為軍卒,善角牴,太祖擊黃巢于陳州,得之,賜以姓名,養為子。後為義兒軍副兵馬使,遷沁州刺史。先時,沁州當敵衝,徙其南百餘里,據險立柵而寓居。至存賢為刺史,曰:「徙城避敵,豈勇者所
 為?」乃復城故州。梁兵屢攻之,存賢力自距守,卒不能近。遷武州刺史、山北團練使,又遷慈州。天祐十八年,梁兵攻朱友謙于河中,莊宗遣存賢援友謙。是時,友謙新叛梁歸晉,而河中食少,人心多貳,諜者因謂存賢曰:「河中人欲殺子以歸梁,宜亟去。」存賢曰:「死王事,吾志也。復何恨哉!」卒擊走梁兵。



 莊宗即位,拜右武衛上將軍。莊宗亦好角牴,嘗與王較而屢勝,頗以自矜,因顧存賢曰:「爾能勝我,與爾一鎮。」存賢博而勝之。同光二年春,幽州符存審病,莊宗置酒宮中,歎曰:「吾創業故人,零落殆盡,其所存者惟存審耳。今又病篤,北方之事誰可代之?」因
 顧存賢曰:「無以易卿。角牴之勝,吾不食言。」即日以為盧龍軍節度使。是歲,卒於幽州,年六十五。贈太傅。



\end{pinyinscope}