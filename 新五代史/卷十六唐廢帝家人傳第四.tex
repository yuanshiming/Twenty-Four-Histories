\article{卷十六唐廢帝家人傳第四}

\begin{pinyinscope}

 廢帝皇后劉氏廢帝皇後劉氏,父茂威,應州渾元人也。后為人彊悍,廢帝素憚之。初封沛國夫人,廢帝即位,立為皇后。其弟延皓,少事廢帝為牙將,廢帝即位,拜宮苑使、宣徽南院使。清泰二年,為樞密使、天雄軍節度使。延皓為人素謹厚,及貴而改節,以后故用事,受賕,掠人園宅,在鄴不恤軍士,軍士皆怒。捧聖都虞候張令昭以其屯駐兵逐延皓,延皓走相州。是時,石敬瑭已反,方用兵,而令昭之亂作。
 令昭乃閉城,遣其副使邊仁嗣請己為節度使。廢帝以令昭為右千牛衛將軍,權知天雄軍府事。已而遣范延光討之,令昭敗走邢州,追至沙河,斬之,屯駐諸軍亂者三千餘人皆死。有司請以延皓行軍法,廢帝以后故,削其官爵而已。



 廢帝二子廢帝二子,曰重吉、重美,一女為尼,號幼澄,皆不知其所生。



 廢帝鎮鳳翔,重吉為控鶴指揮使,與尼俱留京師。控鶴,親兵也。愍帝即位,不欲重吉掌親兵,乃出重吉為亳州團練使,居幼澄於禁中,又徙廢帝北京。廢帝自疑,乃反。愍帝遣人殺重吉於宋州,幼澄亦死。



 重美,幼而明敏如成人。廢帝即位,自左衛上將軍領成德軍節度使、兼河南尹、判六軍諸衛事,改領天雄軍節度使、同中書門下平章事,封雍王。石敬瑭反,廢帝欲北征,重美謂宜持重,固請毋行。廢帝心憚敬瑭,初不欲往,聞重美言,以為然,而劉延皓與劉延朗等迫之不已,廢帝遂如河陽,留重美守京師。京師震恐,居民皆出城以藏竄,門者禁止之。重美曰:「國家多難,不能與民為主,而欲禁其避禍,可乎?」因縱民出。及晉兵將至,劉皇后積薪于地,將焚其宮室,重美曰:「新天子至,必不露坐,但佗日重勞民力,取怨身後耳!」后以為然。廢帝自焚,后及重美
 與俱死。



 嗚呼!家人之道,不可不正也。夫禮者,所以別嫌而明微也。甚矣,五代之際,君君臣臣父父子子之道乖,而宗廟、朝廷,人鬼皆失其序,斯可謂亂世者歟!自古未之有也。唐一號而三姓,周一號而二姓。唐太祖、莊宗為一家,明宗、愍帝為一家,廢帝為一家;周太祖為一家,世宗為一家。別其家而同其號者,何哉?唐從其號,見其盜而有也;周從其號,與之也。而別其家者,昭穆親疏之不可亂也。號可同,家不可以不別,所以別嫌而明微也。梁博王友文之不別,何哉?著禍本也,梁太祖之禍,自友文始,存
 之所以戒也。



\end{pinyinscope}