\article{卷四十雜傳第二十八}

\begin{pinyinscope}

 李茂貞李茂貞,深州博野人也。本姓宋,名文通,為博野軍卒,戍鳳翔。黃巢犯京師,鄭畋以博野軍擊賊,茂貞以功自隊長遷軍校。光啟元年,朱玫反,僖宗出居興元。



 玫遣王行瑜攻大散關,茂貞與保鑾都將李金延等敗行瑜於大唐峰。明年,玫遂敗死。



 茂貞以功自扈蹕都頭拜武定軍節度使,賜以姓名。扈蹕東歸,至鳳翔,鳳翔節度使李昌符與天威都頭楊守立爭道,以兵相攻,昌符不勝,走隴州。
 僖宗遣茂貞擊殺昌符,以功拜鳳翔隴右節度使。大順元年,封隴西郡王。



 二年,樞密使楊復恭得罪,奔於興元,興元節度使楊守亮,復恭之養子也,納之。茂貞乃上書言復恭父子罪皆當誅,因自請為山南招討使。昭宗以宦者故,難之,未許。茂貞擅發兵攻破興元,復恭父子見殺。茂貞表其子繼密權知興元軍府事,昭宗乃徙茂貞山南西道節度使,以宰相徐彥若鎮鳳翔。茂貞不奉詔,上表自論曰:「但慮軍情忽變,戎馬難羈。徒令甸服生靈,因茲受幣;未審乘輿播越,自此何之?」



 昭宗以茂貞表辭不遜,不能忍,以問宰相杜讓能,讓能以謂:「茂貞地
 大兵彊,而唐力未可以致討;鳳翔又近京師,易以自危而難於後悔,佗日雖欲誅晁錯以謝諸侯,恐不能也。」昭宗怒曰:「吾不能孱孱坐受凌弱!」乃責讓能治兵,而以覃王嗣周為京西招討使。令下,京師市人皆知不可,相與聚承天門,遮宰相請無舉兵,爭投瓦石擊宰相,宰相下輿而走,亡其堂印,人情大恐,昭宗意益堅。覃王率扈駕軍五十四都戰于盩厔,唐軍敗潰,茂貞遂犯京師,屯于三橋。昭宗御安福門,殺兩樞密以謝茂貞,使罷兵。茂貞素與讓能有隙,因曰:「謀舉兵者非兩樞密,乃讓能也。」



 陳兵臨皋驛,請殺讓能。讓能曰:「臣故先言之矣,惟殺臣可
 以紓國難。」昭宗泣下沾襟,貶讓能雷州司戶參軍,賜死,茂貞乃罷兵。



 明年,河中節度使王重盈卒,其諸子珂、珙爭立。晉王李克用請立珂,茂貞與韓建、王行瑜請立珙,昭宗不許。茂貞等怒,率三鎮兵犯京師,謀廢昭宗,立吉王保。未果,而晉王亦舉兵,茂貞懼,乃殺宰相韋昭度、李磎,留其養子繼鵬以兵二千宿衛而去。晉兵至河中,繼鵬與行瑜弟行實等爭劫昭宗出奔,京師大亂,昭宗出居於石門。茂貞以兵至鄠縣,斬繼鵬自贖。晉兵已破王行瑜,還軍渭北,請擊茂貞。



 昭宗以謂晉遠而茂貞近,因欲庇之以為德,而冀緩急之可恃也;且茂貞已殺其子
 自贖矣,乃詔罷歸晉軍。克用歎曰:「唐不誅茂貞,憂未已也!」



 昭宗自石門還,益募安聖、捧宸等軍萬餘人,以諸王將之。茂貞謂唐將討己,亦治兵請覲,京師大恐,居人亡入山谷。茂貞遂犯京師,昭宗遣覃王拒之,覃王至三橋,軍潰,昭宗出居于華州。遣宰相孫偓以兵討茂貞,韓建為茂貞請,乃已。久之,加拜茂貞尚書令,封岐王。其後,昭宗為宦者所廢,既反正,宰相崔胤欲借梁兵誅諸宦者,陰與梁太祖謀之。中尉韓全誨等,亦倚茂貞之彊,以為外援,茂貞遣其子繼筠以兵數千宿衛京師,宦者恃岐兵,益驕不可制。



 天復元年,胤召梁太祖以西,梁軍至
 同州,全誨等懼,與繼筠劫昭宗幸鳳翔。



 梁軍圍之逾年,茂貞每戰輒敗,閉壁不敢出。城中薪食俱盡,自冬涉春,雨雪不止,民凍餓死者日以千數。米斗直錢七千,至燒人屎煮尸而食。父自食其子,人有爭其肉者,曰:「此吾子也,汝安得而食之!」人肉斤直錢百,狗肉斤直錢五百。父甘食其子,而人肉賤於狗。天子於宮中設小磨,遣宮人自屑豆麥以供御,自後宮、諸王十六宅,凍餒而死者日三四。城中人相與邀遮茂貞,求路以為生。茂貞窮急,謀以天子與梁以為解。昭宗謂茂貞曰:「朕與六宮皆一日食粥,一日食不托,安能不與梁和乎?」三年正月,茂貞與
 梁約和,斬韓全誨等二十餘人,傳首梁軍,梁圍解。



 天子雖得出,然梁遂劫東遷而唐亡,茂貞非惟亡唐,亦自困矣。



 及梁太祖即位,諸侯之彊者皆相次稱帝,獨茂貞不能,但稱岐王,開府置官屬,以妻為皇后,鳴梢羽扇視朝,出入擬天子而已。茂貞居岐,以寬仁愛物,民頗安之,嘗以地狹賦薄,下令搉油,因禁城門無內松薪,以其可為炬也,有優者誚之曰:「臣請並禁月明。」茂貞笑而不怒。



 初,茂貞破楊守亮取興元,而邠、寧、鄜坊皆附之,有地二十州,其被梁圍也,興元入于蜀;開平已後,邠、寧、鄜、坊入于梁,秦、鳳、階、成又入于蜀;當梁末年,所有七州而已。



 莊宗已破梁,茂貞稱岐王,上箋以季父行自處。及聞入洛,乃上表稱臣,遣其子從嚴來朝。莊宗以其耆老,甚尊禮之,改封秦王,詔書不名。同光二年,以疾卒,年六十九,謚曰忠敬。



 從嚴為人柔而善書畫,茂貞承制拜從嚴彰義軍節度使。茂貞卒,拜鳳翔節度使。魏王繼岌征蜀,為供軍轉運應接使。蜀平,繼岌遣從嚴部送王衍,行至鳳翔,監軍使柴重厚拒而不納,從嚴遂東至華州,聞莊宗之難乃西歸。明宗入立,聞重厚嘗拒從嚴,遣人誅之。從嚴上書,言重厚守鳳翔,軍民無所擾,願貸其過。雖不許,士人以此多之。歷鎮宣武、
 天平。從嚴有田千頃、竹千畝在鳳翔,懼侵民利,未嘗省理,鳳翔人愛之。廢帝起鳳翔,將行,鳳翔人叩馬乞從嚴。廢帝入立,復以從嚴為鳳翔節度使,卒年四十九。



 韓建韓建,字佐時,許州長社人也。少為蔡州軍校,隸忠武軍將鹿晏弘。從楊復光攻黃巢於長安,巢已破,復光亦死,晏弘與建等無所屬,乃以麾下兵西迎僖宗於蜀,所過攻劫。行至興元,逐牛叢,據山南。已而不能守,晏弘東走許州,建乃奔於蜀,拜金吾衛將軍。僖宗還長安,建為潼關防禦使、華州刺史。華州數經大兵,戶口流散,建少賤,習農事,乃披荊棘,督民耕植,出入閭里,問其疾苦。建初
 不知書,乃使人題其所服器皿床榻,為其名目以視之,久乃漸通文字。見《玉篇》,喜曰:「吾以類求之,何所不得也。」因以通音韻聲偶,暇則課學書史。是時,天下已亂,諸鎮皆武夫,獨建撫緝兵民,又好學。荊南成汭時冒姓郭,亦善緝荊楚。當時號為「北韓南郭」。



 大順元年,以兵屬張濬伐晉,濬敗,建自含山遁歸。河中王重盈死,諸子珂、珙爭立,晉人助珂,建與王行瑜、李茂貞助珙。昭宗不許,建等大怒,以三鎮兵犯京師。昭宗見建等責之,行瑜、茂貞惶恐戰汗不能語,獨建前自陳述。乃殺宰相韋昭度、李磎等,謀廢昭宗。會晉舉兵且至,建等懼,乃還。晉兵問罪三
 鎮,兵傅華州,建登城呼曰:「弊邑未常失禮於大國,何為見攻?」晉人曰:「君以兵犯天子,殺大臣,是以討也。」已而與晉和。



 乾寧三年,李茂貞復犯京師,昭宗將奔太原,次渭北,建遣子允請幸華州。昭宗又欲如鄜州,建追及昭宗於富平,泣曰:「籓臣倔彊,非止茂貞,若捨近畿而巡極塞,乘輿渡河,不可復矣!」昭宗亦泣,遂幸華州。



 是時,天子孤弱,獨有殿後軍及定州三都將李筠等兵千餘人為衛,以諸王將之。



 建已得昭宗幸其鎮,遂欲制之,因請罷諸王將兵,散去殿後諸軍,累表不報。昭宗登齊雲樓,西北顧望京師,作《菩薩蠻辭》三章以思歸,其卒章曰:「野煙生
 碧樹,陌上行人去。安得有英雄,迎歸大內中?」酒酣,與從臣悲歌泣下,建與諸王皆屬和之。建心尤不悅,因遣人告諸王謀殺建、劫天子幸佗鎮。昭宗召建,將辨之,建稱疾不出,乃遣諸王自詣,建不見。請送諸王十六宅,昭宗難之。建乃率精兵數千圍行宮,請誅李筠。昭宗大懼,遽詔斬筠,悉散殿後及三都衛兵,幽諸王於十六宅。



 昭宗益悔幸華,遣延王戒丕使于晉,以謀興復。戒丕還,建與中尉劉季述誣諸王謀反,以兵圍十六宅,諸王皆登屋叫呼,遂見殺。昭宗無如之何,為建立德政碑以慰安之。



 建已殺諸王,乃營南莊,起樓閣,欲邀昭宗游幸,
 因以廢之而立德王裕。其父叔豐謂建曰:「汝陳、許間一田夫爾,遭時之亂,蒙天子厚恩至此,欲以兩州百里之地行大事,覆族之禍,吾不忍見,不如先死!」因泣下歔欷。李茂貞、梁太祖皆欲發兵迎天子,建稍恐懼,乃止。光化元年,昭宗還長安,自為建畫像,封建潁川郡王,賜以鐵券。建辭王爵,乃封建許國公。



 梁太祖以兵嚮長安,遣張存敬攻同州,建判官司馬鄴以城降,太祖使鄴召建,建乃出降。太祖責建背己,建曰:「判官李巨川之謀也。」太祖怒,即殺巨川,以建從行。



 昭宗東遷,建從至洛,昭宗舉酒屬太祖與建曰:「遷都之後,國步小康,社稷安危,繫卿兩
 人。」次何皇后舉觴,建躡太祖足,太祖乃陽醉去。建出,謂太祖曰:「天子與宮人眼語,幕下有兵仗聲,恐公不免也。」太祖以故尤德之,表建平盧軍節度使。



 太祖即位,拜司徒同中書門下平章事。太祖性剛暴,臣下莫敢諫諍,惟建時有言,太祖亦優容之。太祖郊於洛,建為大禮使。罷相,出鎮許州,太祖崩,許州軍亂,見殺,年五十八。



 李仁福李仁福,不知其世家。當唐僖宗時,有拓拔思敬者,為夏州偏將,後以與破黃巢功,賜姓李氏,拜夏州節度使。思敬卒,乾寧二年,以其弟思諫為節度使。



 自唐末天下大亂,史官實錄多闕,諸鎮因時倔起,自非有大善惡暴著
 於世者,不能紀其始終。是時,興元、鳳翔、邠寧、鄜坊、河中、同華諸鎮之兵,四面並起而交爭,獨靈夏未嘗為唐患,而亦無大功。朱玫之亂,思敬與鄜州李思孝皆以兵屯渭橋。其後,黃巢陷京師,王重榮、李克用等會諸鎮兵討賊,思敬與破巢復京師,然皆未嘗有所可稱,故思敬之世次、功過不顯而無傳。



 梁開平二年,思諫卒,軍中立其子彝昌為留後,梁即拜彝昌節度使。明年,其將高宗益作亂,殺彞昌。是時,仁福為蕃部指揮使,戍兵於外,軍中乃迎仁福立之,不知其於思諫為親疏也。是歲四月,拜仁福檢校司空、定難軍節度使。終梁之世,奉正朔而
 已。是時,岐王李茂貞,晉王李克用,數會兵攻仁福,梁輒出兵救之。仁福累官至檢校太師兼中書令,封朔方王。長興四年三月卒,其子彞超自立為留後。



 自仁福時,邊將多言仁福通於契丹,恐為邊患。明宗因其卒,乃以彞超為延州刺史、彰武軍節度使,而徙彰武安從進代之。恐彝超不受代,遣邠州藥彥稠以兵五萬送從進之鎮。彞超果不受代,從進與彥稠以兵圍之,百餘日不克。夏州城壁素堅,故老傳言赫連勃勃蒸土築之,從進等穴地道,至城下堅如鐵石,鑿不能入。彞超外招黨項,抄掠從進等糧道,自陜以西,民運斗粟束芻,其費數千,人
 不堪命,道路愁苦。明宗遂釋不攻,以彞超為定難軍節度使。清泰二年卒。



 其弟彞興,累官檢校太師兼侍中,周顯德中,封西平王,其後事具國史。



 韓遜韓遜,不知其世家。初為靈武軍校,當唐末之亂,據有靈鹽,唐即以為節度使,而史失不錄,不見其事。梁開平三年,封朔方節度使韓遜為潁川王,始見於史。是時,邠寧楊崇本、鄜延李周彝、鳳翔李茂貞,皆與梁爭戰,獨遜與夏州李思諫臣屬於梁,未嘗以兵爭。李茂貞嘗遣劉知俊攻遜,不能克,遜亦善撫其部,人皆愛之,為遜立生祠。



 貞明中,遜卒,軍中立其子洙為留後,梁即以為節度
 使。至莊宗時,又以洙兼河西節度。天成四年,洙卒,即以洙子澄為朔方軍留後。其將李賓作亂,澄乃上章請師於朝,明宗以康福為朔方河西節度使以代澄,由是命吏而相代矣。韓氏自遜有靈武,傳世皆無所稱述,澄後不知其所終。



 楊崇本楊崇本,幼事李茂貞,養以為子,冒姓李,名曰繼徽,茂貞表崇本靜難軍節度使。梁太祖攻岐未下,乃移兵攻邠州,崇本迎降,太祖使復其姓,賜名崇本,遷其家於河中以為質。崇本妻有美色,太祖用兵,往來河中,嘗幸之。崇本妻頗愧恥,間遣人誚崇本曰:「大丈夫不能庇其伉
 儷,我已為朱公婦矣,無面視君,有刀繩而已!」崇本涕泣憤怒。其後梁兵解岐圍,崇本妻得歸,崇本乃復背梁歸茂貞。茂貞西連蜀兵會崇本攻雍、華,關西大震。太祖以兵西至河中,遣郴王友裕擊之,友裕至永壽而卒,梁兵乃旋。崇本屯美原,太祖復遣劉知俊、康懷英等擊之,崇本大敗,自此不復東。乾化四年,為其子彥魯所殺。崇本養子李保衡,殺彥魯以降梁。



 高萬興高萬興,河西人也。唐末,河西屬李茂貞,茂貞將胡敬璋為延州刺史,萬興與其弟萬金俱事敬璋為騎將。敬璋死,其將劉萬子代為刺史。梁開平二年,葬於州南,萬子
 在會,其將許從實殺萬子,自為延州刺史。是時,萬興兄弟皆將兵戍境上,聞萬子死,以其部下數千人,降梁。



 梁太祖兵屯河中,遣同州劉知俊以兵應萬興,攻丹州,執其刺史崔公實。進攻延州,執許從實。鄜州李彥容、坊州李彥昱皆棄城走。梁太祖乃以萬興為延州刺史、忠義軍節度使,以牛存節為保大軍節度使。已而劉知俊叛,乃徙存節守同州,以萬金為保大軍節度使。萬興累遷檢校太師兼中書令,渤海郡王。貞明四年,萬金卒,乃以萬興為鄜延節度使,進封延安郡王,徙封北平王。梁亡,莊宗入洛,萬興嘗一來朝。同光三年,卒於鎮。



 萬興兄
 弟皆驍勇,而未嘗立戰功,然以戍兵降梁,梁取鄜、坊、丹、延自萬興始,故其兄弟世守其土。



 萬興死,子允韜代立,長興元年徙鎮安國,又徙義成,清泰中卒。



 萬金子允權,開運中為膚施令,罷居於家。是時,周密為彰信軍節度使,契丹滅晉,延州軍亂,逐密,密守東城,而西城之兵以允權為留後。聞漢高祖起太原,遂歸漢,即拜節度使,廣順三年卒。



 溫韜溫韜,京兆華原人也。少為盜,後事李茂貞,為華原鎮將,冒姓李,名彥韜。



 茂貞以華原縣為耀州,以韜為刺史。梁太祖圍茂貞於鳳翔,韜以耀州降梁,已而復叛歸茂貞。
 茂貞又以美原縣為鼎州,建義勝軍,以韜為節度使。末帝時,韜復叛茂貞降梁,梁改耀州為崇州,鼎州為裕州,義勝為靜勝軍,即以韜為節度使,復其姓溫,更其名曰昭圖。



 韜在鎮七年,唐諸陵在其境內者,悉發掘之,取其所藏金寶,而昭陵最固,韜從埏道下,見宮室制度閎麗,不異人間,中為正寢,東西廂列石床,床上石函中為鐵匣,悉藏前世圖書,鐘、王筆迹,紙墨如新,韜悉取之,遂傳人間,惟乾陵風雨不可發。



 其後朱友謙叛梁,取同州,晉王以兵援友謙而趨華原,韜懼,求徙佗鎮,遂徙忠武。莊宗滅梁,韜自許來朝,因伶人景進納賂劉皇后,皇后為
 言之,莊宗待韜甚厚,賜姓名曰李紹沖。郭崇韜曰:「此劫陵賊爾,罪不可赦!」莊宗曰:「已宥之矣,不可失信。」遽遣還鎮。明宗入洛,與段凝俱收下獄,已而赦之,勒歸田里。



 明年,流於德州,賜死。



 嗚呼,厚葬之弊,自秦漢已來,率多聰明英偉之主,雖有高談善說之士,極陳其禍福,有不能開其惑者矣!豈非富貴之欲,溺其所自私者篤,而未然之禍,難述於無形,不足以動其心歟?然而聞溫韜之事者,可以少戒也!五代之君,往往不得其死,何暇顧其後哉!獨周太祖能鑒韜之禍,其將終也,為書以遺世宗,使以瓦棺、紙衣而斂。
 將葬,開棺示人,既葬,刻石以告後世,毋作下宮,毋置守陵妾。其意丁寧切至,然實錄不書其葬之薄厚也。又使葬其平生所服袞冕、通天冠、絳紗袍各二,其一於京師,其一於澶州;又葬其劍、甲各二,其一於河中,其一於大名者,莫能原其旨也。



\end{pinyinscope}