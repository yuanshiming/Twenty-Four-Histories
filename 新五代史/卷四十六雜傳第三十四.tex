\article{卷四十六雜傳第三十四}

\begin{pinyinscope}

 趙在禮趙在禮,字幹臣,涿州人也。少事劉仁恭為軍校,仁恭遣佐其子守文襲取滄州,其後守文為其弟守光所殺,在禮乃奔於晉。莊宗時,為效節指揮使,將魏兵戍瓦橋關。還至貝州,軍士皇甫暉作亂,推其將楊仁晟為首,仁晟不從,殺之;又推一小校,小校不從,又殺之;乃攜二首詣在禮。在禮聞亂,衣不及帶,方踰垣而走,暉曳其足而下之,環以白刃,示之二首,曰:「不從我者如此首!」在禮從之,
 遂反。



 在禮自貝州還攻魏,縱軍大掠。是時,興唐尹王正言年老病昏,聞在禮至,呼吏草奏,吏已奔散,正言猶不知,方據案大怒,左右告曰:「賊已市中殺人,吏民皆走,欲誰呼邪?」正言大驚曰:「吾初不知此。」即索馬將去,廄吏曰:「公妻子為虜矣,安得馬乎?」正言惶恐,步出府門,見在禮,望而下拜,在禮呼正言曰:「公何自屈之甚邪!此軍士之情,非予志也。」在禮即自稱兵馬留後。



 莊宗遣元行欽討之,行欽攻魏不克,乃遣明宗代行欽。明宗至鄴,軍變,因入城與在禮合。明宗兵反嚮京師,在禮留于魏。明宗即位,拜在禮義成軍節度使,在禮不受命,遂拜鄴都留守、
 興唐尹。久之,皇甫暉等皆去,在禮獨在魏,患魏軍之驕,懼及禍,乃求徙鎮橫海。歷鎮泰寧、匡國、天平、忠武、武寧、歸德、晉昌,所至邸店羅列,積貲巨萬。



 晉出帝時,以在禮為北面行營馬步都虞候,以擊契丹,未嘗有戰功。在禮在宋州,人尤苦之;已而罷去,宋人喜而相謂曰:「眼中拔釘,豈不樂哉!」既而復受詔居職,乃籍管內,口率錢一千,自號「拔釘錢。」晉亡,契丹入汴,在禮自宋馳至洛陽,遇契丹拽剌等,拜於馬首,拽剌等兵共侵辱之,誅責貨財,在禮不勝其憤。



 行至鄭州,聞晉大臣多為契丹所鎖,中夜惶惑,解衣帶就馬櫪自經而卒,年六十二。



 漢高祖立,
 贈中書令。



 霍彥威霍彥威,字子重,洺州曲周人也。少遭兵亂,梁將霍存掠得之,愛其俊爽,養以為子。嘗從存戰,中矢,眇其一目。後事梁太祖,太祖亦愛之,稍遷左龍驤軍使、右監門衛上將軍。預誅友珪,以功拜洺州刺史,遷邠寧節度使。李茂貞遣梁叛將劉知俊攻邠州,彥威固守踰年,每獲知俊兵,必縱還之,知俊德之,後不復攻。徙鎮義成,又徙天平,兼北面行營招討使,與晉軍相持河上,彥威屢敗,降為陜州留後。



 莊宗滅梁,彥威自陜來朝,莊宗置酒故梁崇元殿,彥威與梁將段凝、袁象先等皆在。莊宗酒酣,指彥
 威等舉酒屬明宗曰:「此皆前日之勍敵,今侍吾飲,乃卿功也。」彥威等惶恐伏地請死,莊宗勞之曰:「吾與總管戲爾,卿無畏也。」賜姓名曰李紹真。明年,徙鎮武寧,從明宗擊契丹,明宗愛其為人,甚親厚之。



 其後趙在禮反,彥威別討趙太於邢州,破之,還以兵屬明宗討在禮。明宗軍變,從馬直軍吏張破敗率眾殺將校,縱火焚營噪呼,明宗叱之曰:「自吾為帥十有餘年,何負爾輩!今賊城破在旦夕,乃爾輩立功名、取富貴之時。況爾天子親軍,返效賊耶!」軍士對曰:「城中之人何罪,戍卒思歸而不得耳!天子不垂原宥,志在巢除。



 且聞破魏之後,欲盡坑魏博諸
 軍,某等初無叛心,直畏死耳!今宜與城中合勢,擊退諸鎮之兵,請天子帝河南,令公鎮河北。」明宗涕泣諭之,亂兵環列而呼曰:「令公不欲帝河北,則佗人有之,我輩狼虎,豈識尊卑!」彥威與安重誨勸明宗許之,乃擁兵入城,與在禮合,彥威獨不入。明宗入城,與在禮置酒大會,而部兵在外者聞明宗反,皆潰去,獨彥威所將五千人營城西北隅不動。居二日,明宗復出,得彥威兵,乃之魏縣,謀欲還鎮州,彥威、重誨勸明宗以兵南向。



 莊宗崩,彥威從明宗入洛陽,首率群臣勸進,內外機事,皆決彥威。彥威素與段凝、溫韜有隙,因擅捕凝、韜下獄,將殺之,安重
 誨曰:「凝、韜之惡,天下所知,然主上方平內難,以恩信示人,豈公報仇之時?」彥威乃止。明宗即位,乃赦凝、韜,放歸田里,已而卒賜死。



 彥威徙鎮平盧。朱守殷反,伏誅,彥威遣使者馳騎獻兩箭為賀,明宗賜兩箭以報之。夷狄之法,起兵令眾,以傳箭為號令,然非下得施於上也。明宗本出夷狄,而彥威武人,君臣皆不知禮,動多此類。然彥威客有淳于晏者,登州人也,少舉明經及第,遭世亂,依彥威,自彥威為偏裨時已從之。彥威嘗戰敗脫身走,麾下兵無從者,獨晏徒步以一劍從之榛棘間以免。彥威高其義,所歷方鎮,常辟以自從,至其家事無大小,皆決
 於晏,彥威以故得少過失。當時諸鎮辟召寮屬,皆以晏為法。



 天成三年冬,彥威卒於鎮。是時,明宗方獵于近郊,青州馳騎奏彥威卒,明宗涕泣還宮,輟朝,仍終其月不舉樂,贈彥威太師,謚曰忠武。



 房知溫房知溫,字伯玉,兗州瑕丘人也。少以勇力為赤甲都官健,後隸魏州馬鬥軍,稍遷親隨軍指揮使。莊宗取魏博,得知溫,賜姓李氏,名曰紹英,以為澶州刺史,歷曹、貝二州刺史,戍瓦橋關。明宗自魏反兵南向,知溫首馳赴之。天成元年,拜泰寧軍節度使。明年,為北面招討使,屯於盧臺。明宗遣烏震往代知溫還鎮,其戍卒效節軍將龍
 晊等攻震殺之。效節,魏州軍也。魏州自羅紹威誅衙軍,楊師厚為節度使,復置銀槍效節軍。當梁末帝時,師厚幾為梁患。師厚卒,以賀德倫代之。末帝患魏軍彊難制,與趙巖等謀分相、魏為兩鎮,魏軍由此作亂,劫德倫叛梁而降晉,梁遂失河北。莊宗自得魏兵,與梁戰河上,數有功,許其軍以滅梁而厚賞。及梁亡,魏軍雖數賜與,而驕縱無厭,常懷怨望;皇甫暉之亂,劫趙在禮入魏,皆此軍也。



 明宗入立,在禮鎮天雄軍,以魏軍素驕,常懼禍,不遑居,陰遣人訴于明宗,求解去。明宗乃以皇子從榮代在禮,而遣魏效節九指揮北戍盧臺。軍發之日,不給兵
 甲,惟以長竿繫旗幟以表隊伍,軍士頗自疑惑。明年,明宗遣烏震代知溫戍,而知溫意尤不樂。盧臺戍軍夾水東西為兩寨,震初至,與知溫會東寨,方博,效節軍亂,噪於門外,知溫即乘馬而出。亂軍擊殺震,執轡留知溫,知溫紿曰:「騎兵皆在西寨,今獨步軍,恐無能為也。」知溫即躍馬登舟渡河入西寨,以騎軍盡殺亂者。明宗下詔,悉誅其家屬於魏州,凡九指揮三千餘家數萬口,驅至漳水上殺之,漳水為之變色。魏之驕兵,於是而盡。明宗知變自知溫起,釋而不問,徙鎮武寧,加兼侍中,歷鎮天平、平盧。



 初,明宗為北面招討使,而知溫為副使,廢帝時以
 裨將事知溫甚謹,後因杯酒失意。及廢帝起兵鳳翔,愍帝出奔,知溫乘間有窺覦之意,謂其司馬李沖曰:「吾有錢數屋,養兵數千,因時建義,功必有成。」沖曰:「今天子孱弱,上下離心,潞王兵威甚盛,事未可知,沖請懷表而西以覘之。」及沖至京師,廢帝已入立,沖即奉表稱賀,還勸知溫入朝,廢帝慰勞之甚厚。知溫還鎮,封東平王。太常上言:「策拜王公,皇帝臨軒遣冊。其在外者,正衙命使,而鹵簿、鼓吹、輅車、法物不出都城,考之故事無明文。今北平王德鈞、東平王知溫受封遣策,請下兵部、太常、太僕,給鹵簿、鼓吹、輅車、法物赴本道,禮畢還有司。」知溫在鎮,
 常厚斂其民,積貲巨萬,治第青州南城,出入以聲妓,游嬉不恤政事。天福元年卒于宮,贈太尉。



 知溫卒後,其子彥儒獻其父錢三萬緡、絹布三萬匹、金百兩、銀千兩、茶千五百斤、絲十萬兩,拜沂州刺史。其將吏分其餘貲者,皆為富家云。



 王晏球王晏球,字瑩之,洛陽人也。少遇亂,為盜所掠,汴州富人杜氏得之,養以為子,冒姓杜氏。梁太祖鎮宣武,選富家子之材武者置之帳下,號「子都」。晏球為人倜儻有大節,為子都指揮使。太祖即位,為右千牛衛將軍。友珪立,龍驤戍卒反,自懷州趣京師,遣晏球擊敗之于河陽,
 以功遷龍驤第一指揮使。



 末帝即位,遷龍驤四軍指揮使。梁遣捉生軍將李霸將千人戍楊劉,霸夜作亂,自水門入,縱火大噪,以長竿縛布沃油,仰燒建國門。晏球聞亂,不俟命,率龍驤五百騎擊之,賊勢稍卻。末帝登樓見之,呼曰:「此非吾龍驤軍邪!」晏球奏曰:「亂者,李霸一部爾,陛下嚴守宮城,而責臣破賊。」遲明盡殺之,以功拜澶州刺史。



 梁、晉軍河上,以晏球為行營馬步軍都指揮使。莊宗入汴,晏球以兵追之,行至封丘,聞末帝已崩,即解甲降唐,莊宗賜姓名曰李紹虔,拜齊州防禦使,戍瓦橋關。明宗兵變,自鄴而南,遣人招晏球,晏球從至洛陽,拜
 歸德軍節度使。定州王都反,以晏球為招討使,與宣徽南院使張延朗等討之。都遣人北招契丹,契丹遣禿餒將萬騎救都。晏球聞禿餒等兵且來,留張延朗屯新樂,自逆於望都。而契丹從他道入定州,與都出不意擊延朗軍,延朗大敗,收餘兵會晏球趨曲陽,都乘勝追之。



 晏球先至水次,方坐胡床指麾,而都眾掩至,晏球與左右十餘人連矢射之,都眾稍卻,而後軍亦至。晏球立高岡,號令諸將皆橐弓矢、用短兵,回顧者斬。符彥卿以左軍攻其左,高行珪以右軍攻其右,中軍騎士抱馬項馳入都軍,都遂大敗,自曲陽至定州,橫尸棄甲六十餘里。都
 與禿餒入城,不敢復出。契丹又遣惕隱以七千騎益都,晏球遇之唐河,追擊至滿城,斬首二千級,獲馬千匹。契丹自中國多故,彊於北方,北方諸夷無大小皆畏伏,而中國之兵遭契丹者,未嘗少得志。自晏球擊敗禿餒,又走惕隱,其餘眾奔潰投村落,村落之人以鋤櫌白梃所在擊殺之,無復遺類。



 惕隱與數十騎走至幽州西,為趙德鈞擒送京師。明宗下詔責誚契丹。契丹後數遣使至中國,求歸惕隱等,辭甚卑遜,輒斬其使以絕之。於是時,中國之威幾於大震,而契丹少衰伏矣,自晏球始也。



 晏球攻定州,久不克,明宗數遣人促其破賊,晏球以謂未
 可急攻。其偏將朱弘昭、張虔釗等宣言曰:「晏球怯耳!」乃驅兵以進,兵果敗,殺傷三千餘人,由是諸將不敢復言攻。晏球乃休養士卒,食其三州之賦,悉以俸祿所入具牛酒,日與諸將高會。久之,都城中食盡,先出其民萬餘人,數與禿餒謀決圍以走,不果,都將馬讓能以城降,都自焚死。



 晏球為將有機略,善撫士卒。其擊禿餒,既因敗以為功,而諸將皆欲乘勝取都,晏球返,獨不動,卒以持久弊之。自天成三年四月都反,明年二月始克之,軍中未嘗戮一人。以破都功,拜天平軍節度使。又徙平盧,累官至兼中書令。是歲卒,年六十二,贈太尉。



 安重霸安重霸,雲州人也,初與明宗俱事晉王。重霸得罪奔于梁,又奔于蜀。重霸為人狡譎多智,善事人。蜀王建以為親將。王衍立,少年,宦者王承休用事,重霸深結承休以自託。梁末,蜀取李茂貞秦、成、階三州,重霸勸承休求鎮秦州,衍以承休為節度使,重霸為其副使。重霸與承休多取秦州花木獻衍,請衍東遊。唐魏王兵伐蜀,承休大恐,以問重霸,重霸曰:「劍門天下之險,雖有精兵,不可過也。然公受國恩,聞難不可不赴,願與公俱西。」承休素親信之,以為然。承休整軍將發,秦人送之,帳飲城外。酒罷,承休上道,重霸立承休馬前,辭曰:「秦、隴不可失,願留為
 公守。」承休業已上道,無如之何。唐軍已破蜀,重霸亦以秦、成、階三州降唐,明宗以為閬州團練使。罷為左衛大將軍。久之,以為匡國軍節度使。廢帝時,為京兆尹、西京留守,徙鎮大同,以病罷還,卒于潞州。



 王建立王建立,遼州榆社人也。唐明宗為代州刺史,以建立為虞候將。莊宗嘗遣女奴之代州祭墓,女奴侵擾代人,建立捕而笞之。莊宗怒,欲殺之,明宗為庇讓之以免。



 明宗自魏反,犯京師,曹皇后、王淑妃皆在常山,建立殺常山監軍并其守兵,明宗家屬因得無患,由是明宗益愛之。明宗即位,以為成德軍節度副使,已而拜節度使、檢校
 太尉、同中書門下平章事。



 建立與安重誨素不協,定州王都有二志,數以書通建立,約為兄弟,重誨知之以為言。明宗不欲傷建立,亟召還京師。建立入見,亦多言重誨過失。明宗大怒,欲亟罷重誨,群臣左右諷解之,乃止。然卒以建立為右僕射、同中書門下平章事、判三司事。居歲餘,自言不識文字,願解三司,明宗不許。久之,建立稱疾,明宗笑曰:「人固有詐疾而得疾者。」乃出為平盧節度使,又徙上黨。建立怏怏不得志,遂求解職,乃以太子少保致仕。



 建立數請朝見,不許,乃自詣京師,闌至後樓見明宗,涕泣言己無罪,為重誨所擯,明宗曰:「汝為節
 度使,不作好事,豈獨重誨讒汝邪!」賜以茶藥而遣之。



 廢帝立,復起為天平軍節度使。晉高祖時,徙鎮平盧。天福五年來朝,高祖勞之曰:「三十年前老兄,可毋拜!」賜以肩輿入朝,給二宦者掖而升殿,宴見甚渥。又徙昭義,賜以玉斧、蜀馬。累封韓王。建立好殺人,其晚節始惑浮圖法,戒殺生,所至人稍安之。卒年七十,贈尚書令。



 子守恩,以廕補,稍遷諸衛將軍。建立已卒,家于潞,守恩自京師得告歸,而契丹滅晉。昭義節度使張從恩與守恩姻家,乃以守恩權巡檢使,以守潞州,而從恩入見契丹。從恩既去,守恩因剽劫從恩家貲,以潞州降漢。漢高祖即位,以
 守恩為昭義軍節度使,徙鎮靜難、西京留守,加同中書門下平章事。



 守恩性貪鄙,人甚苦之。時周太祖以樞密使將白文珂等軍西平三叛,還過洛陽,守恩以使相自處,肩輿出迎。太祖怒,即日以頭子命文珂代守恩為留守,而守恩方詣館謁,坐於客次以俟見,而吏馳報新留守視事於府矣。守恩大驚,不知所為,遂罷去,奉朝請于京師。



 後隱帝殺史弘肇等,召群臣上殿慰諭之,群臣恐懼,無敢言者,獨守恩前對曰:「陛下始睡覺矣。」聞者皆縮頭。顯德中,為左金吾衛上將軍以卒。



 嗚呼!道德仁義,所以為治,而法制綱紀,亦所以維持之
 也。自古亂亡之國,必先壞其法制而後亂從之。亂與壞相乘,至蕩然無復綱紀,則必極於大亂而後返,此勢之然也,五代之際是已。若文珂、守恩皆位兼將相,漢大臣也,而周太祖以一樞密使頭子易置之,如更戍卒。是時,太祖與漢未有間隙之端,其無君叛上之志,宜未萌於心,而其所為如此者,何哉?蓋其習為常事,故特發於喜怒頤指之間,而文珂不敢違,守恩不得拒。太祖既處之不疑,而漢廷君臣亦置而不問,其上下安然而不怪者,豈非朝廷法制綱紀壞亂相乘,其來也遠,既極而至於此歟!是以善為天下慮者,不敢忽於微,而常杜其漸也,
 可不戒哉!



 康福康福,蔚州人也,世為軍校。福以騎射事晉王為偏將。莊宗嘗曰:「吾家以羊馬為生,福狀貌類胡人而豐厚,胡宜羊馬。」乃令福牧馬于相州,為小馬坊使,逾年馬大蕃滋。明宗自魏反,兵過相州,福以小坊馬二千匹歸命,明宗軍勢由是益盛。



 明宗入立,拜飛龍使,領磁州刺史、襄州兵馬都監。從劉訓討荊南,無功而還。福為將無佗能,善諸戎語,明宗嘗召入便殿,訪以外事,福輒為蕃語以對。樞密使安重誨惡之,常戒福曰:「無妄奏事,當斬汝!」福懼,求外任。



 靈武韓洙死,第六澄立,而偏將李從賓作亂。澄
 表請朝廷命帥,而重誨以謂靈武深入夷境,為帥者多遇害,乃拜福涼州刺史,朔方、河西軍節度使。福入見明宗,涕泣言為重誨所擠。明宗召重誨為福更佗鎮,重誨曰:「福為刺史無功效而建節旄,其敢有所擇邪!」明宗怒,謂福曰:「重誨遣汝,非吾意也。吾當遣兵護汝,可無憂。」乃令將軍牛知柔以兵衛福。行至方渠,而羌夷果出邀福,福以兵擊走之。至青岡峽,遇雪,福登山望見川谷中煙火,有吐蕃數千帳,不覺福至,福分其兵馬三道,出其不意襲之。吐蕃大駭,棄車帳而走,殺之殆盡,獲其玉璞、綾錦、羊馬甚眾,由是威聲大振。



 福居靈武三歲,歲常豐稔,有
 馬千駟,蕃夷畏服。言事者疑福有異志,重誨亦言福必負朝廷。明宗遣人謂福曰:「我何少汝而欲負我!」福言:「受國恩深,有死無二。」因乞還朝,不許。福章再上,即隨而至,明宗不之罪,徙鎮彰義。歷靜難、雄武,充西面都部署。晉高祖時,徙鎮河中,代還,卒于京師,贈太師,謚曰武安。



 福世本夷狄,夷狄貴沙陀,故常自言沙陀種也。福嘗有疾臥閣中,寮佐入問疾,見其錦衾,相顧竊戲曰:「錦衾爛兮!」福聞之,怒曰:「我沙陀種也,安得謂我為奚?」聞者笑之。



 郭延魯郭延魯,沁州綿上人也。父饒,以驍勇事晉,數立軍功,為沁州刺史者九年,為政有惠愛,州人思之。延魯以善槊
 為將,累遷神武都知兵馬使。朱守殷反,從攻汴州,以先登功為汴州馬步軍都指揮使,累遷復州刺史。延魯嘆曰:「吾先君為沁州者九年,民到于今思之。吾今幸得為刺史,其敢忘吾先君之志!」由是益以廉平自勵,民甚賴之。秩滿,州人乞留,不許,皆遮道攀號。天福中,拜單州刺史,卒于官。



 當是時,刺史皆以軍功拜,言事者多以為言,以謂方天下多事,民力困敝之時,不宜以刺史任武夫,恃功縱下,為害不細。而延魯父子特以善政著聞焉。



 嗚呼,五代之民其何以堪之哉!上輸兵賦之急,下困剝斂之苛。自莊宗以來,方鎮進獻之事稍作,至於晉而不
 可勝紀矣。其「添都」、「助國」之物,動以千數計。至於來朝、奉使、買宴、贖罪,莫不出於進獻。而功臣大將,不幸而死,則其子孫率以家貲求刺史,其物多者得大州善地。蓋自天子皆以賄賂為事矣,則為其民者其何以堪之哉!於此之時,循廉之吏如延魯之徒者,誠難得而可貴也哉!



\end{pinyinscope}