\article{卷七十二四夷附錄第一}

\begin{pinyinscope}

 嗚
 呼,夷狄居處飲食,隨水草寒暑徙遷,有君長部號而無世族、文字記別,至於弦弓毒矢,彊弱相并,國地大小,興滅不常,是皆烏足以考述哉!惟其服叛去來,能為中國利害者,此不可以不知也。自古夷狄之於中國,有道未必服,無道未必不來,蓋自因其衰盛。雖嘗置之治外,而羈縻制馭恩威之際,不可失也。其得之未必為利,失之有足為患,可不慎哉!作《四夷附錄》。



 新五代史·附錄夷狄,種號多矣。其大者自以名通中國,其次小遠者附見,又其次微不足錄者,不可勝數。其地環列九州之外,而西北常彊,為中國患。三代獫狁,見於《詩》、《書》。秦、漢以來,匈奴著矣。隋、唐之間,突厥為大。其後有吐蕃、回鶻之彊。五代之際,以名見中國者十七八,而契丹最盛。



 契丹自後魏以來,名見中國。或曰與庫莫奚同類而異種。其居曰梟羅個沒里。



 沒里者,河也。是謂黃水之南,黃龍之北,得鮮卑之故地,故又以為鮮卑之遺種。



 當唐之世,其地北接室韋,東鄰高麗,西界奚國,而南至營州。其部族之大者曰大賀氏,後分為八部,其一曰伹皆利部,二曰乙
 室活部,三曰實活部,四曰納尾部,五曰頻沒部,六曰內會雞部,七曰集解部,八曰奚枿部。部之長號大人,而常推一大人建旗鼓以統八部。至其歲久,或其國有災疾而畜牧衰,則八部聚議,以旗鼓立其次而代之。被代者以為約本如此,不敢爭。某部大人遙輦次立,時劉仁恭據有幽州,數出兵摘星嶺攻之,每歲秋霜落,則燒其野草,契丹馬多飢死,即以良馬賂仁恭求市牧地,請聽盟約甚謹。八部之人以為遙輦不任事,選於其眾,以阿保機代之。



 阿保機,亦不知其何部人也,為人多智勇而善騎射。是時,劉守光暴虐,幽、涿之人多亡入契丹。阿保機
 乘間入塞,攻陷城邑,俘其人民,依唐州縣置城以居之。



 漢人教阿保機曰:「中國之王無代立者。」由是阿保機益以威制諸部而不肯代。其立九年,諸部以其久不代,共責誚之。阿保機不得已,傳其旗鼓,而謂諸部曰:「吾立九年,所得漢人多矣,吾欲自為一部以治漢城,可乎?」諸部許之。漢城在炭山東南灤河上,有鹽鐵之利,乃後魏滑鹽縣也。其地可植五穀,阿保機率漢人耕種,為治城郭邑屋廛市,如幽州制度,漢人安之,不復思歸。阿保機知眾可用,用其妻述律策,使人告諸部大人曰:「我有鹽池,諸部所食。然諸部知食鹽之利,而不知鹽有主人,可乎?
 當來犒我。」諸部以為然,共以牛酒會鹽池。阿保機伏兵其旁,酒酣伏發,盡殺諸部大人,遂立,不復代。



 梁將篡唐,晉王李克用使人聘于契丹,阿保機以兵三十萬會克用於雲州東城。



 置酒。酒酣,握手約為兄弟。克用贈以金帛甚厚,期共舉兵擊梁。阿保機遺晉馬千匹。既歸而背約,遣使者袍笏梅老聘梁。梁遣太府卿高頃、軍將郎公遠等報聘。逾年,頃還,阿保機遣使者解裏隨頃,以良馬、貂裘、朝霞錦聘梁,奉表稱臣,以求封冊。梁復遣公遠及司農卿渾特以詔書報勞,別以記事賜之,約共舉兵滅晉,然後封冊為甥舅之國,又使以子弟三百騎入衛京
 師。克用聞之,大恨。是歲克用病,臨卒,以一箭屬莊宗,期必滅契丹。渾特等至契丹,阿保機不能如約,梁亦未嘗封冊。



 而終梁之世,契丹使者四至。



 莊宗天祐十三年,阿保機攻晉蔚州,執其振武節度使李嗣本。是時,莊宗已得魏博,方南向與梁爭天下,遣李存矩發山北兵。存矩至祁溝關,兵叛,擁偏將盧文進擊殺存矩,亡入契丹。契丹攻破新州,以文進部將劉殷守之。莊宗遣周德威擊殷,而文進引契丹數十萬大至,德威懼,引軍去,為契丹追及,大敗之。德威走幽州,契丹圍之。幽、薊之間,虜騎遍滿山谷,所得漢人,以長繩連頭繫之於木,漢人夜多自
 解逃去。文進又教契丹為火車、地道、起土山以攻城。城中熔銅鐵汁揮之,中者輒爛墮。德威拒守百餘日,莊宗遣李嗣源、閻寶、李存審等救之。契丹數為嗣源等所敗,乃解去。



 契丹比他夷狄尤頑傲,父母死,以不哭為勇,載其尸深山,置大木上,後三歲往取其骨焚之,酹而咒曰:「夏時向陽食,冬時向陰食,使我射獵,豬鹿多得。」



 其風俗與奚、靺鞨頗同。至阿保機,稍并服旁諸小國,而多用漢人,漢人教之以隸書之半增損之,作文字數千,以代刻木之約。又制婚嫁,置官號。乃僭稱皇帝,自號天皇王。以其所居橫帳地名為姓,曰世里。世里,譯者謂之耶律。名
 年曰天贊。



 以其所居為上京,起樓其間,號西樓,又於其東千里起東樓,北三百里起北樓,南木葉山起南樓,往來射獵四樓之間。契丹好鬼而貴日,每月朔旦,東向而拜日,其大會聚、視國事,皆以東向為尊,四樓門屋皆東向。



 莊宗討張文禮,圍鎮州。定州王處直懼鎮且亡,晉兵必并擊己,遣其子郁說契丹,使入塞以牽晉兵。郁謂阿保機曰:「臣父處直使布愚款曰:故趙王王鎔,王趙六世,鎮州金城湯池,金帛山積,燕姬趙女,羅綺盈廷。張文禮得之而為晉所攻,懼死不暇,故皆留以待皇帝。」阿保機大喜。其妻述律不肯,曰:「我有羊馬之富,西樓足以娛樂,
 今捨此而遠赴人之急,我聞晉兵彊天下,且戰有勝敗,後悔何追?」



 阿保機躍然曰:「張文禮有金玉百萬,留待皇后,可共取之。」於是空國入寇。郁之召契丹也,定人皆以為後患不可召,而處直不聽。郁已去,處直為其子都所廢。



 阿保機攻幽州不克,又攻涿州,陷之。聞處直廢而都立,遂攻中山,渡沙河。都告急於莊宗。莊宗自將鐵騎五千,遇契丹前鋒於新城,晉兵自桑林馳出,人馬精甲,光明燭日。虜騎愕然,稍卻,晉軍乘之,虜遂散走,而沙河冰薄,虜皆陷沒。阿保機退保望都。會天大雪,契丹人馬飢寒,多死,阿保機顧盧文進以手指天曰:「天未使
 我至此。」乃引兵去。莊宗躡其後,見其宿處,環秸在地,方隅整然,雖去而不亂,歎曰:「虜法令嚴,蓋如此也!」



 契丹雖無所得而歸,然自此頗有窺中國之志,患女真、渤海等在其後,欲擊渤海,懼中國乘其虛,乃遣使聘唐以通好。同光之間,使者再至。莊宗崩,明宗遣供奉官姚坤告哀於契丹。坤至西樓而阿保機方東攻渤海,坤追至慎州見之。阿保機錦袍大帶垂後,與其妻對坐穹廬中,延坤入謁。阿保機問曰:「聞爾河南、北有兩天子,信乎?」坤曰:「天子以魏州軍亂,命總管令公將兵討之,而變起洛陽,凶問今至矣。總管返兵河北,赴難京師,為眾所推,已副人
 望。」阿保機仰天大哭曰:「晉王與我約為兄弟,河南天子,即吾兒也。昨聞中國亂,欲以甲馬五萬往助我兒,而渤海未除,志願不遂。」又曰:「我兒既沒,理當取我商量,新天子安得自立?」



 坤曰:「新天子將兵二十年,位至大總管,所領精兵三十萬,天時人事,其可得違?」



 其子突欲在側曰:「使者無多言,蹊田奪牛,豈不為過!」坤曰:「應天順人,豈比匹夫之事。至如天皇王得國而不代,豈彊取之邪?」阿保機即慰勞坤曰:「理正當如是爾!」又曰:「吾聞此兒有宮婢二千人,樂官千人,放鷹走狗,嗜酒好色,任用不肖,不惜人民,此其所以敗也。我自聞其禍,即舉家斷酒,解放鷹
 犬,罷散樂官。我亦有諸部樂官千人,非公宴不用。我若所為類吾兒,則亦安能長久?」又謂坤曰:「吾能漢語,然絕口不道於部人,懼其效漢而怯弱也。」因戒坤曰:「爾當先歸,吾以甲馬三萬會新天子幽、鎮之間,共為盟約,與我幽州,則不復侵汝矣。」



 阿保機攻渤海,取其扶餘一城,以為東丹國,以其長子人皇王突欲為東丹王。已而阿保機病死,述律護其喪歸西樓,立其次子元帥太子耀屈之。坤從至西樓而還。



 當阿保機時,有韓延徽者,幽州人也,為劉守光參軍,守光遣延徽聘於契丹。



 延徽見阿保機不拜,阿保機怒,留之不遣,使牧羊馬。久之,知其材,召
 與語,奇之,遂用以為謀主。阿保機攻黨項、室韋,服諸小國,皆延徽謀也。延徽後逃歸,事莊宗,莊宗客將王緘譖之,延徽懼,求歸幽州省其母。行過常山,匿王德明家。



 居數月,德明問其所向,延徽曰:「吾欲復走契丹。」德明以為不可,延徽曰:「阿保機失我,如喪兩目而折手足,今復得我,必喜。」乃復走契丹。阿保機見之,果大喜,以謂自天而下。阿保機僭號,以延徽為相,號「政事令」,契丹謂之「崇文令公」,後卒于虜。



 耀屈之後更名德光。葬阿保機木葉山,謚曰大聖皇帝,後更其名曰億。德光立三年,改元曰天顯,遣使者以名馬聘唐,并求碑石為阿保機刻銘。明宗
 厚禮之,遣飛勝指揮使安念德報聘。定州王都反,唐遣王晏球討之。都以蠟丸書走契丹求援,德光遣禿餒、荝剌等以騎五千救都,都及禿餒擊晏球於曲陽,為晏球所敗。德光又遣惕隱赫邈益禿餒以騎七千,晏球又敗之于唐河。赫邈與數騎返走,至幽州,為趙德鈞所執,而晏球攻破定州,擒禿餒、荝剌,皆送京師。明宗斬禿餒等六百餘人,而赦赫邈,選其壯健者五十餘人為「契丹直」。



 初,阿保機死,長子東丹王突欲當立,其母述律遣其幼子安端少君之扶餘代之,將立以為嗣。然述律尤愛德光。德光有智勇,素已服其諸部,安端已去,而諸部希述
 律意,共立德光。突欲不得立,長興元年,自扶餘泛海奔於唐。明宗因賜其姓為東丹,而更其名曰慕華。以其來自遼東,乃以瑞州為懷化軍,拜慕華懷化軍節度、瑞慎等州觀察處置等使。其部曲五人皆賜姓名,罕只曰罕友通,穆葛曰穆順義,撒羅曰羅賓德,易密曰易師仁,蓋禮曰蓋來賓,以為歸化、歸德將軍郎將。又賜前所獲赫邈姓名曰狄懷惠,抯列曰列知恩,荝剌曰原知感,福郎曰服懷造,竭失訖曰訖懷宥。其餘為「契丹直」者,皆賜姓名。二年,更賜突欲姓李,更其名曰贊華。三年,以贊華為義成軍節度使。



 契丹自阿保機時侵滅諸國,稱雄北方。
 及救王都,為王晏球所敗,喪其萬騎,又失赫邈等,皆名將,而述律尤思念突欲,由是卑辭厚幣數遣使聘中國,因求歸赫邈、荝剌等,唐輒斬其使而不報。當此之時,中國之威幾振。



 距幽州北七百里有榆關,東臨海,北有兔耳、覆舟山。山皆斗絕,並海東北,僅通車,其旁地可耕植。唐時置東西狹石、淥疇、米磚、長揚、黃花、紫蒙、白狼等戍,以扼契丹於此。戍兵常自耕食,惟衣絮歲給幽州,久之皆有田宅,養子孫,以堅守為己利。自唐末幽、薊割據,戍兵廢散,契丹因得出陷平、營,而幽、薊之人歲苦寇鈔。自涿州至幽州百里,人跡斷絕,轉餉常以兵護
 送,契丹多伏兵鹽溝以擊奪之。莊宗之末,趙德鈞鎮幽州,於鹽溝置良鄉縣,又於幽州東五十里築城,皆戍以兵。及破赫邈等,又於其東置三河縣。由是幽、薊之人,始得耕牧,而輸餉可通。德光乃西徙橫帳居揆剌泊,出寇雲、朔之間。明宗患之,以石敬瑭鎮河東,總大同、彰國、振武、威塞等軍禦之。應順、清泰之間,調發饋餉,遠近勞敝。



 德光事其母甚謹,常侍立其側,國事必告而後行。石敬瑭反,唐遣張敬達等討之。敬瑭遣使求救於德光。德光白其母曰:「吾嘗夢石郎召我,而使者果至,豈非天邪!」母召胡巫問吉凶,巫言吉,乃許。是歲九月,契丹出鴈
 門,車騎連亙數十里,將至太原,遣人謂敬瑭曰:「吾為爾今日破敵可乎?」敬瑭報曰:「皇帝赴難,要在成功,不在速,大兵遠來,而唐軍甚盛,願少待之。」使者未至,而兵已交。



 敬達大敗。敬瑭夜出北門見德光,約為父子,問曰:「大兵遠來,戰速而勝者,何也?」德光曰:「吾謂唐兵能守鴈門而扼諸險要,則事未可知。今兵長驅深入而無阻,吾知大事必濟。且吾兵多難久,宜以神速破之。此其所以勝也。」敬達敗,退保晉安寨,德光圍之。唐遣趙德鈞、延壽救敬達,而德鈞父子按兵團柏谷不救。德光謂敬瑭曰:「吾三千里赴義,義當徹頭。」乃築壇晉城南,立敬瑭為皇帝,自
 解衣冠被之,冊曰:「咨爾子晉王,予視爾猶子,爾視予猶父。」已而楊光遠殺張敬達降晉。晉高祖自太原入洛陽,德光送至潞州,趙德鈞、延壽出降。德光謂晉高祖曰:「大事已成。吾命大相溫從爾渡河,吾亦留此,俟爾入洛而後北。」臨訣,執手噓戲,脫白貂裘以衣高祖,遺以良馬二十匹,戰馬千二百匹,戒曰:「子子孫孫無相忘!」時天顯九年也。



 高祖已入洛,德光乃北,執趙德鈞、延壽以歸。德鈞,幽州人也,事劉守光、守文為軍校,莊宗伐燕得之,賜姓名曰李紹斌。其子延壽,本姓劉氏,常山人也,其父邧為蓚縣令,劉守文攻破蓚縣,德鈞得延壽并其母種氏而
 納之,因以延壽為子。



 延壽為人,姿質妍柔,稍涉書史,明宗以女妻之,號興平公主。莊、明之世,德鈞鎮幽州十餘年,以延壽故,尤見信任。延壽明宗時為樞密使,罷,至廢帝立,復以為樞密使。晉高祖起太原,廢帝遣延壽將兵討之。而德鈞亦請以鎮兵討賊,廢帝察其有異志,使自飛狐出擊其後,而德鈞南出吳,會延壽於西唐,延壽因以兵屬之。



 廢帝以德鈞為諸道行營都統,延壽為太原南面招討使。德鈞為延壽求鎮州節度使。



 廢帝怒曰:「德鈞父子握彊兵,求大鎮,茍能敗契丹而破太原,雖代予亦可。若玩寇要君,但恐犬兔俱斃。」因遣使者趣德鈞
 等進軍。德鈞陰遣人聘德光,求立己為帝。德光指穹廬前巨石謂德鈞使者曰:「吾已許石郎矣。石爛,可改也。」德光至潞州,鎖德鈞父子而去。德光母述律見之,問曰:「汝父子自求為天子何邪?」德鈞慚不能對,悉以田宅之籍獻之。述律問何在,曰:「幽州。」述律曰:「幽州屬我矣,何獻之為?」明年,德鈞死,德光以延壽為幽州節度使,封燕王。



 契丹當莊宗、明宗時攻陷營、平二州,及已立晉,又得雁門以北幽州節度管內,合一十六州。乃以幽州為燕京,改天顯十一年為會同元年,更其國號大遼,置百官,皆依中國,參用中國之人。晉高祖每遣使聘問,奉表稱臣,
 歲輸絹三十萬匹,其餘寶玉珍異,下至中國飲食諸物,使者相屬於道,無虛日。德光約高祖不稱臣,更表為書,稱「兒皇帝」,如家人禮。德光遣中書令韓熲奉冊高祖為英武明義皇帝。高祖復遣趙瑩、馮道等以太常鹵簿奉冊德光及其母尊號。終其世,奉之甚謹。



 高祖崩,出帝即位,德光怒其不先以告,而又不奉表,不稱臣而稱孫,數遣使者責晉。晉大臣皆恐,而景延廣對契丹使者語,獨不遜。德光益怒。楊光遠反青州,招之。開運元年春,德光傾國南寇,分其眾為三:西出鴈門,攻並、代,劉知遠擊敗之于秀容;東至于河,陷博州,以應光遠;德光與延壽南,
 攻陷貝州。德光屯元城,兵及黎陽。晉出帝親征,遣李守貞等東馳馬家渡,擊敗契丹。而德光與晉相距於河,月餘,聞馬家渡兵敗,乃引眾擊晉,戰于戚城。德光臨陣,望見晉軍旗幟光明,而士馬嚴整,有懼色,謂其左右曰:「楊光遠言晉家兵馬半已餓死,何其盛也!」



 兵既交,殺傷相半,陣間斷箭遺鏃,布厚寸餘。日暮,德光引去,分其兵為二,一出滄州,一出深州以歸。二年正月,德光復傾國入寇,圍鎮州,分兵攻下鼓城等九縣。杜重威守鎮州,閉壁不敢出。契丹南掠邢、洺、磁,至于安陽河,千里之內,焚剽殆盡。契丹見大桑木,罵曰:「吾知紫披襖出自汝身,吾豈
 容汝活邪!」束薪於木而焚之。是時,出帝病,不能出征,遣張從恩、安審琦、皇甫遇等禦之。遇前渡漳水,遇契丹,戰於榆林,幾為所虜。審琦從後救之,契丹望見塵起,謂救兵至,引去。而從恩畏怯,不敢追,亦引兵南走黎陽。契丹已北,而出帝疾少間,乃下詔親征,軍于澶州,遣杜重威等北伐。契丹歸至古北,聞晉軍且至,即復引而南,及重威戰於陽城、衛村。晉軍飢渴,鑿井輒壞,絞泥汁而飲。德光坐奚車中,呼其眾曰:「晉軍盡在此矣,可生擒之,然後平定天下。」會天大風,晉軍奮死擊之,契丹大敗。德光喪車,騎一白橐駝而走。至幽州,其首領大將各笞數百,獨
 趙延壽免焉。是時,天下旱蝗,晉人苦兵,乃遣開封府軍將張暉假供奉官聘于契丹,奉表稱臣,以修和好。德光語不遜。然契丹亦自厭兵。德光母述律嘗謂晉人曰:「南朝漢兒爭得一向臥邪?自古聞漢來和蕃,不聞蕃去和漢,若漢兒實有回心,則我亦何惜通好!」晉亦不復遣使,然數以書招趙延壽。



 延壽見晉衰而天下亂,嘗有意窺中國,而德光亦嘗許延壽滅晉而立之。延壽得晉書,偽為好辭報晉,言身陷虜思歸,約晉發兵為應。而德光將高牟翰亦詐以瀛州降晉,晉君臣皆喜。三年七月,遣杜重威、李守貞、張彥澤等出兵,為延壽應。兵趨瀛州,牟翰
 空城而去。晉軍至城下,見城門皆啟,疑有伏兵,不敢入。遣梁漢璋追牟翰及之,漢璋戰死。重威等軍屯武強。德光聞晉出兵,乃入寇鎮州。重威西屯中渡,與德光夾水而軍。德光分兵,並西山出晉軍後,攻破欒城縣,縣有騎軍千人,皆降於虜。德光每獲晉人,刺其面,文曰「奉敕不殺」,縱以南歸。重威等被圍糧絕,遂舉軍降。德光喜,謂趙延壽曰:「所得漢兒皆與爾。」因以龍鳳赭袍賜之,使衣以撫晉軍,亦以赭袍賜重威。遣傅住兒監張彥澤將騎二千,先入京師。晉出帝與太后為降表,自陳過咎。德光遣解里以手詔賜帝曰:「孫兒但勿憂,管取一吃飲處。」德光
 將至京師,有司請以法駕奉迎,德光曰:「吾躬擐甲胃,以定中原,太常之儀,不暇顧也。」止而不用。出帝與太后出郊奉迎,德光辭不見,曰:「豈有兩天子相見於道路邪!」四年正月丁亥朔旦,晉文武百官班于都城北,望帝拜辭,素服紗帽以待。德光被甲衣貂帽,立馬于高岡,百官俯伏待罪。德光入自封丘門,登城樓,遣通事宣言諭眾曰:「我亦人也,可無懼。我本無心至此,漢兵引我來爾。」



 遂入晉宮,宮中嬪妓迎謁,皆不顧,夕出宿于赤岡。封出帝負義侯,遷於黃龍府。



 癸巳,入居晉宮,以契丹守諸門,門廡殿廷皆磔犬掛皮,以為厭勝。甲午,德光胡服視朝于
 廣政殿。乙未,被中國冠服,百官常參,起居如晉儀,而氈裘左衣任,胡馬奚車,羅列階陛,晉人俯首,不敢仰視。二月丁巳朔,金吾六軍、殿中省仗、太常樂舞陳於廷,德光冠通天冠,服絳紗袍,執大珪以視朝,大赦,改晉國為大遼國,開運四年為會同十年。



 德光嘗許趙延壽滅晉而立以為帝,故契丹擊晉,延壽常為先鋒,虜掠所得,悉以奉德光及其母述律。德光已滅晉而無立延壽意,延壽不敢自言,因李崧以求為皇太子。德光曰:「吾於燕王無所愛惜,雖我皮肉,可為燕王用者,吾可割也。吾聞皇太子是天子之子,燕王豈得為之?」乃命與之遷秩。翰林學士
 張礪進擬延壽中京留守、大丞相、錄尚書事、都督中外諸軍事。德光索筆,塗其錄尚書事、都督中外諸軍事,止以為中京留守、大丞相,而延壽前為樞密使、封燕王皆如故。又以礪為右僕射兼門下侍郎、同中書門下平章事,與故晉相和凝並為宰相。礪,明宗時翰林學士,晉高祖起太原,唐廢帝遣礪督趙延壽進軍於團柏谷,已而延壽為德光所鎖,並礪遷于契丹。德光重其文學,仍以為翰林學士。礪常思歸,逃至境上,為追者所得,德光責之,礪曰:「臣本漢人,衣服飲食言語不同,今思歸而不得,生不如死。」



 德光顧其通事高唐英曰:「吾戒爾輩善待引
 人,致其逃去,過在爾也。」因笞唐英一百而待礪如故,其愛之如此。德光將視朝,有司給延壽貂蟬冠,礪三品冠服,延壽與礪皆不肯服。而延壽別為王者冠以自異。礪曰:「吾在上國時,晉遣馮道奉冊北朝,道齎二貂冠,其一宰相韓延徽冠之,其一命我冠之。今其可降服邪!」卒冠貂蟬以朝。三月丙戌朔,德光服靴、袍,御崇元殿,百官入閣,德光大悅,顧其左右曰:「漢家儀物,其盛如此。我得於此殿坐,豈非真天子邪!」其母述律遣人齎書及阿保機明殿書賜德光。明殿,若中國陵寢下宮之制,其國君死,葬,則於其墓側起屋,謂之明殿,置官屬職司,歲時奉表
 起居如事生,置明殿學士一人掌答書詔,每國有大慶弔,學士以先君之命為書以賜國君,其書常曰報兒皇帝云。



 德光已滅晉,遣其部族酋豪及其通事為諸州鎮刺史、節度使,括借天下錢帛以賞軍。胡兵人馬不給糧草,遣數千騎分出四野,劫掠人民,號為「打草穀」,東西二三千里之間,民被其毒,遠近怨嗟。漢高祖起太原,所在州鎮多殺契丹守將歸漢,德光大懼。又時已熱,乃以蕭翰為宣武軍節度使。翰,契丹之大族,其號阿缽,翰之妹亦嫁德光,而阿缽本無姓氏,契丹呼翰為國舅,及將以為節度使,李崧為製姓名曰蕭翰,於是始姓蕭。德光
 已留翰守汴,乃北歸,以晉內諸司伎術、宮女、諸軍將卒數千人從。自黎陽渡河,行至湯陰,登愁死岡,謂其宣徽使高勳曰:「我在上國,以打圍食肉為樂,自入中國,心常不快,若得復吾本土,死亦無恨。」勳退而謂人曰:「虜將死矣。」相州梁暉殺契丹守將,閉城距守。德光引兵破之,城中男子無少長皆屠之,婦女悉驅以北。後漢以王繼弘鎮相州,得髑髏十數萬枚,為大冢葬之。德光至臨洺,見其井邑荒殘,笑謂晉人曰:「致中國至此,皆燕王為罪首。」



 又顧張礪曰:「爾亦有力焉。」德光行至欒城,得疾,卒于殺胡林。契丹破其腹,去其腸胃,實之以鹽,載而北,晉人
 謂之「帝羓」焉。永康王兀欲立,謚德光為嗣聖皇帝,號阿保機為太祖,德光為太宗。



\end{pinyinscope}