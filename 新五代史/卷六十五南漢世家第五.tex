\article{卷六十五南漢世家第五}

\begin{pinyinscope}

 劉隱,其祖安仁,上蔡人也,後徙閩中,商賈南海,因家焉。父謙,為廣州牙將。唐乾符五年,黃巢攻破廣州,去略湖、湘間,廣州表謙封州刺史、賀江鎮遏使,以禦梧、桂以西。歲餘,有兵萬人,戰艦百餘艘。謙三子,曰隱、臺、巖。



 謙卒,廣州表隱代謙封州刺史。乾寧中,節度使劉崇龜死,嗣薛王知柔代為帥,行至湖南,廣州將盧琚、覃玘作亂,知柔不敢進。隱以封州兵攻殺琚、玘,迎知柔,知柔辟隱行軍
 司馬。其後徐彥若代知柔,表隱節度副使,委以軍政。彥若卒,軍中推隱為留後。天祐二年,拜隱節度使。梁開平元年,加檢校太尉、兼侍中。二年,兼靜海軍節度、安南都護。三年,加檢校太師、兼中書令,封南平王。



 隱父子起封州,遭世多故,數有功於嶺南,遂有南海。隱復好賢士。是時,天下已亂,中朝士人以嶺外最遠,可以避地,多游焉。唐世名臣謫死南方者往往有子孫,或當時仕宦遭亂不得還者,皆客嶺表。王定保、倪曙、劉濬、李衡、周傑、楊洞潛、趙光裔之徒,隱皆招禮之。定保容管巡官,曙唐太學博士,濬崇望之子,以避亂往;衡德裕之孫,唐右補闕,以
 奉使往。皆辟置幕府,待以賓客。傑善星歷,唐司農少卿,因避亂往,隱數問以災變,傑恥以星術事人,常稱疾不起,隱亦客之。



 洞潛初為邕管巡官,秩滿客南海,隱常師事之,後以為節度副使,及僭號,為陳吉凶禮法。為國制度,略有次序,皆用此數人焉。



 乾化元年,進封隱南海王。是歲卒,年三十八。弟立。



 ,初名巖,謙庶子也。其母段氏生於外舍,謙妻韋氏素妒,聞之怒,拔劍而出,命持至,將殺之。及見而悸,劍輒墮地,良久曰:「此非常兒也!」後三日,卒殺段氏,養為己子。及長,善騎射,身長七尺,垂手過膝。



 隱為行軍司馬,
 亦辟薛王府諮議參軍。隱鎮南海,為副使。隱卒,代立。



 乾化二年,除清海節度使,檢校太保、同平章事。三年,加檢校太傅。末帝即位,悉以隱官爵授,襲封南海王。



 唐末,南海最後亂,僖宗以後,大臣出鎮者,天下皆亂,無所之,惟除南海而已,自隱始亦自立。是時,交州曲顥、桂州劉士政、邕州葉廣略、容州龐巨昭,分據諸管;盧光稠據虔州以攻嶺上,其弟光睦據潮州,子延昌據韶州;高州刺史劉昌魯、新州刺史劉潛及江東七十餘寨,皆不能制。隱攻韶州,曰:「韶州所賴者光稠,擊之,虔人必應,應則首尾受敵,此不宜直攻而可以計取。」隱不聽,
 果敗而歸,因盡以兵事付。悉平諸寨,遂殺昌魯等,更置刺史,卒出兵攻敗盧氏,取潮、韶。又西與馬殷爭容、桂,殷取桂管,虜士政;取容管,逐巨昭,又取邕管。



 隱、自梁初受封爵,稟正朔而已。



 貞明三年,即皇帝位,國號大越,改元曰乾亨。追尊安仁文皇帝,謙聖武皇帝,隱襄皇帝,立三廟。置百官,以楊洞潛為兵部侍郎,李衡禮部侍郎,倪曙工部侍郎,趙光胤兵部尚書,皆平章事。光胤自以唐甲族,恥事偽國,常怏怏思歸。乃習為光胤手書,遣使間道至洛陽,召其二子損、益並其家屬皆至。光胤驚喜,為盡心焉。



 性聰悟而苛酷,為刀鋸、支解、刳剔
 之刑,每視殺人,則不勝其喜,不覺朵頤,垂涎呀呷,人以為真蛟蜃也。又好奢侈,悉聚南海珍寶,以為玉堂珠殿。



 二年,祀天南郊,大赦境內,改國號漢。初欲僭號,憚王定保不從,遣定保使荊南,及還,懼其非己,使倪曙勞之,告以建國。定保曰:「建國當有制度,吾入南門,清海軍額猶在,四方其不取笑乎!」笑曰:「吾備定保久矣,而不思此,宜其譏也。」



 三年,冊越國夫人馬氏為皇后。馬氏,楚王殷女也。



 四年春,置選部貢舉,放進士、明經十餘人,如唐故事,歲以為常。



 七年,唐莊宗入汴,懼,遣宮苑使何詞入詢中國虛實,稱大漢國主致書大唐皇帝。詞還,言唐
 必亂,不足憂,大喜。又性好夸大,嶺北商賈至南海者,多召之,使升宮殿,示以珠玉之富。自言家本咸秦,恥王蠻夷,呼唐天子為「洛州刺史」。



 是歲,雲南驃信鄭旻遣使致朱鬃白馬以求婚,使者自稱皇親母弟、清容布燮兼理、賜金錦袍虎綾紋攀金裝刀、封歸仁慶侯、食邑一千戶、持節鄭昭淳。昭淳好學有文辭,與游宴賦詩,及群臣皆不能逮,遂以隱女增城縣主妻旻。



 八年,作南宮,王定保獻《南宮七奇賦》以美之。初名巖,又更曰陟。



 九年,白龍見南宮三清殿,改元曰白龍,又更名龔,以應龍見之祥。有胡僧言:「讖書『滅劉氏者龔也。』」乃採《周易》「飛龍
 在天」之義為「」字,音「儼」,以名焉。



 四年,楚人以舟師攻封州,封州兵敗於賀江,懼,以《周易》筮之,遇《大有》,遂赦境內,改元曰大有。遣將蘇章以神弩軍三千救封州,章以兩鐵索沈賀江中,為巨輪於岸上,築堤以隱之,因輕舟迎戰,陽敗而奔,楚人逐之,章舉巨輪挽索鎖楚舟,以彊弩夾江射之,盡殺楚人。



 三年,遣將李守鄘、梁克貞攻交趾,擒曲承美等。承美至南海,登義鳳樓受俘,謂承美曰:「公常以我為偽廷,今反面縛,何也?」承美頓首伏罪,乃赦之。



 承美,顥子也。克貞又攻占城,掠其寶貨而歸。



 四年,愛州楊廷藝叛,攻交州刺史李進,進遁歸。遣承旨程寶攻
 廷藝,寶戰死。



 五年,封子耀樞邕王,龜圖康王,洪度秦王,洪熙晉王,洪昌越王,洪弼齊王,洪雅韶王,洪澤鎮王,洪操萬王,洪杲循王,洪暐息王,洪邈高王,洪簡同王,洪建益王,洪濟辨王,洪道貴王,洪昭宣王,洪政通王,洪益定王。



 九年,遣將軍孫德晟攻蒙州,不克。



 十年,交州牙將皎公羨殺楊廷藝自立,廷藝故將吳權攻交州,公羨來乞師。封洪操交王,出兵白藤以攻之。以兵駐海門,權已殺公羨,逆戰海口,植鐵橛海中,權兵乘潮而進,洪操逐之,潮退舟還,轢橛者皆覆,洪操戰死,收餘眾而還。



 十五年,卒,年五十四,謚天皇大帝,廟號高祖,陵曰康
 陵。子玢立。



 玢,初名洪度,封秦王。子耀樞、龜圖皆早死,玢次當立。病臥寢中,召右僕射王翻與語,呼洪度、洪熙小字曰:「壽、俊雖長,然皆不足任吾事,惟洪昌類我,吾欲立之。奈何吾子孫不肖,後世如鼠入牛角,勢當漸小爾!」因泣下歔欷。



 翻為謀,出洪度以邕州,洪熙容州,然後立洪昌為太子。議已定,崇文使蕭益入問疾,以告之,益諫曰:「少者得立,長者爭之,禍始此矣!」由是洪度卒得立。



 更名玢,改元曰光天,尊母趙昭儀為皇太妃,以晉王洪熙輔政。



 玢立,果不能任事。在殯,召伶人作樂,飲酒宮中,裸
 男女以為樂,或衣墨縗與倡女夜行,出入民家。由是山海間盜賊競起。妖人張遇賢,自稱中天八國王,攻陷循州。玢遣越王洪昌、循王洪杲攻之,遇賢圍洪昌等於錢帛館,裨將萬景忻、陳道庠力戰,挾二王潰圍而走。玢莫能省,嶺東皆亂。



 洪熙日益進聲妓誘玢為荒恣。玢亦頗疑諸弟圖己,敕宦官守宮門,入者皆露索。



 洪熙、洪杲、洪昌陰遣陳道庠養勇士劉思潮、譚令禋、林少彊、少良、何昌廷等,習為角牴以獻玢。玢宴長春宮以閱之,玢醉起,道庠與思潮等隨至寢門拉殺之,盡殺其左右。玢立二年,年二十四,謚曰殤。弟晟立。



 晟,
 初名洪熙,封晉王。既弒玢,遂自立,改元曰應乾,以洪昌為兵馬元帥,知政事,洪杲副元帥,劉思潮等封功臣。晟既殺兄,立不順,懼眾不伏,乃益峻刑法以威眾。已而洪杲屢請討賊,陰勸晟誅思潮等以止外議。晟大怒,使使者夜召洪杲。洪杲知不免,乃留使者,入具沐浴,詣佛前祝曰:「洪杲誤念,來生王宮,今見殺矣。後世當生民家,以免屠害。」涕泣與家人訣別,然後赴召,至則殺之。冬,晟祀天南郊,改元曰乾和,群臣上尊號曰大聖文武大明至道大光孝皇帝。



 二年夏,遣洪昌祠襄帝陵於海曲,至昌華宮,晟使盜刺殺之。晟自殺洪杲,由是與諸弟有
 隙,而洪昌最賢,素所欲立者,晟尤忌之,故先及害。鎮王洪澤居邕州,有善政,是歲鳳皇見邕州,晟怒,使人酖殺之。而諸弟相次見殺。



 三年,殺其弟洪雅,又殺劉思潮等五人。思潮等死,陳道庠懼,不自安,其友鄧伸以荀悅《漢紀》遺之,道庠莫能曉,伸罵曰:「憝獠!韓信誅而彭越醢,皆在此書矣!」道庠悟,益懼。晟聞之大怒,以道庠、伸下獄,皆斬之於市,夷其族。



 以右僕射王翻為英州刺史,使人殺之於路。



 五年,晟弟洪弼、洪道、洪益、洪濟、洪簡、洪建、洪暐、洪昭,同日皆見殺。



 六年,遣工部郎中、知制誥鐘允章聘楚以求婚,楚不許。允章還,晟曰:「馬公復能經略南土
 乎?」是時,馬希廣新立,希萼起兵武陵,湖南大亂,允章具言楚可攻之狀。晟乃遣巨象指揮使吳珣、內侍吳懷恩攻賀州,已克之,楚人來救,珣鑿大阱於城下,覆箔於上,以土傅之,楚兵迫城,悉陷阱中,死者數千,楚人皆走。



 珣等攻桂州及連、宜、嚴、梧、蒙五州,皆克之。掠全州而還。



 九年冬,又遣內侍潘崇徹攻郴州,李景兵亦在,與崇徹遇,戰,大敗景兵於宜章,遂取郴州。晟益得志,遣巨艦指揮使暨彥贇以兵入海,掠商人金帛作離宮游獵,故時劉氏有南宮、大明、昌華、甘泉、玩華、秀華、玉清、太微諸宮,凡數百,不可悉紀。宦者林延遇、宮人盧瓊仙,內外專恣
 為殺戮,晟不復省。常夜飲大醉,以瓜置伶人尚玉樓項,拔劍斬之以試劍,因并斬其首。明日酒醒,復召玉樓侍飲,左右白已殺之,晟歎息而已。



 十年,湖南王進逵以兵五萬率溪洞蠻攻郴州,潘崇徹敗進逵於豪石,斬首萬餘級。



 十一年,晟病甚,封其子繼興衛王,璇興桂王,慶興荊王,保興祥王,崇興梅王。



 十二年,晟親耕藉田。交州吳昌浚遣使稱臣,求節鉞。昌濬者,權子也。權自時據交州,遣洪操攻之,洪操戰死,遂棄不復攻。權死,子昌岌立,昌岌卒,弟昌浚立,始稱臣於晟。晟遣給事中李璵以旌節招之,璵至白州,濬使人止璵曰:「海賊為亂,道路不
 通。」璵不果行。晟殺其弟洪邈。



 十三年,又殺其弟洪政,於是之諸子盡矣。顯德三年,世宗平江北,晟始惶恐,遣使修貢於京師,為楚人所隔,使者不得行,晟憂形於色。又嘗自言知星,末年,月食牛女間,出書占之,歎曰:「吾當之矣!」因為長夜之飲。



 十六年,卜葬域於城北,運甓為壙,晟親臨視之。是秋卒,年三十九,謚曰文武光聖明孝皇帝,廟號中宗,陵曰昭陵。子鋹立。



 鋹,初名繼興,封衛王。晟卒,以長子立,改元曰大寶。晟性剛忌,不能任臣下,而獨任其嬖倖宦官、宮婢延遇、瓊仙等。至鋹尤愚,以謂群臣皆自有家室,顧子孫,不能盡忠,
 惟宦者親近可任,遂委其政於宦者龔澄樞、陳延壽等,至其群臣有欲用者,皆閹然後用。澄樞等既專政,鋹乃與宮婢波斯女等淫戲後宮,不復出省事。延壽又引女巫樊胡子,自言玉皇降胡子身。鋹於內殿設帳幄,陳寶貝,胡子冠遠遊冠,衣紫霞裾,坐帳中宣禍福,呼鋹為太子皇帝,國事皆決於胡子,盧瓊仙、龔澄樞等爭附之。胡子乃為鋹言:「澄樞等皆上天使來輔太子,有罪不可問。」尚書左丞鐘允章參政事,深嫉之,數請誅宦官,宦官皆仄目。



 二年,鋹祀天南郊,前三日,允章與禮官登壇,四顧指麾,宦者許彥真望見之曰:「此謀反爾!」乃拔劍升壇,允
 章迎叱之,彥真馳走,告允章反。鋹下允章獄,遣禮部尚書薛用丕治之。允章與用丕有舊,因泣下曰:「吾今無罪,自誣以死固無恨,然吾二子皆幼,不知父冤,俟其長,公可告之。」彥真聞之,罵曰:「反賊欲使而子報仇邪?」復入白鋹,并捕二子繫獄,遂族誅之。陳延壽謂鋹曰:「先帝所以得傳陛下者,由盡殺群弟也。」勸鋹稍誅去諸王。鋹以為然,殺其弟桂王璇興。



 是歲,建隆元年也。鋹將邵廷琄言於鋹曰:「漢乘唐亂,居此五十年,幸中國有故,干戈不及,而漢益驕於無事,今兵不識旗鼓,而人主不知存亡。夫天下亂久矣,亂久而治,自然之勢也。今聞真主已出,必將
 盡有海內,其勢非一天下不能已。」勸鋹修兵為備,不然,悉珍寶奉中國,遣使以通好。鋹懵然莫以為慮,惡廷琄言直,深恨之。



 四年,芝菌生宮中,野獸觸寢門,苑中羊吐珠,井旁石自立,行百餘步而仆,樊胡子皆以符瑞諷群臣入賀。



 五年,鋹以宦者李托養女為貴妃,專寵。托為內太師,居中專政。許彥真既殺鐘允章,惡龔澄樞等居己上,謀殺之。澄樞使人告彥真反,族誅之。



 七年,王師南伐,克郴州,晟所遣將暨彥贇與其刺史陸光圖皆戰死,餘眾退保韶州。鋹始思廷琄言,遣廷琄以舟兵出洸口抗王師。會王師退舍,廷琄訓士卒,修戰備,嶺人倚以為
 良將。有譖者投無名書言廷琄反,鋹遣使者賜死;士卒排軍門見使者,訴廷琄無反狀,不能救,為立祠於洸口。



 八年,交州吳昌文卒,其佐呂處玶與峰州刺史喬知祐爭立,交趾大亂,驩州丁璉舉兵擊破之,鋹授璉交州節度。



 九年,南海民妻生子兩首四臂。是時,太祖皇帝詔李煜諭鋹使稱臣,鋹怒,囚煜使者龔慎儀。



 十三年,詔潭州防禦使潘美出師,師次白霞。鋹遣龔澄樞守賀州、郭崇岳守桂州、李托守韶州以備。是歲秋,潘美平賀州,十月平昭州,又平桂州,十一月平連州。鋹喜曰:「昭、桂、連、賀,本屬湖南,今北師取之,足矣,其不復南也。」



 其愚如此。十二
 月平韶州。開寶四年正月,平英、雄二州,鋹將潘崇徹先降。師次瀧頭,鋹遣使請和,求緩師。二月,師度馬逕,鋹遣其右僕射蕭漼奉表降。漼行,鋹惶迫,復令整兵拒命。美等進師,鋹遣其弟祥王保興率文武詣美軍降,不納。龔澄樞、李托等謀曰:「北師之來,利吾國寶貨爾,焚為空城,師不能駐,當自還也。」



 乃盡焚其府庫、宮殿。鋹以海舶十餘,悉載珍寶、嬪御,將入海,宦官樂範竊其舟以逃歸。師次白田,鋹素衣白馬以降。獻俘京師,赦鋹為左千牛衛大將軍,封恩赦侯。其後事具國史。



 隱興滅年世,諸書皆同。蓋自唐天祐二年隱為廣州節度使,至皇朝開寶四年國滅,凡六十七年。《舊五代史》以梁貞明三年僭號為始,故曰五十五年爾。



\end{pinyinscope}