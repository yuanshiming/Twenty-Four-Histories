\article{卷四十四雜傳第三十二}

\begin{pinyinscope}

 劉知俊劉知俊,字希賢,徐州沛人也。少事時溥,溥與梁相攻,知俊與其麾下二千人降梁,太祖以為左開道指揮使。知俊姿貌雄傑,能被甲上馬,輪劍入敵,勇出諸將。



 當是時,劉開道名重軍中。歷海、懷、鄭三州刺史,從破青州,以功表匡國軍節度使。



 邠州楊崇本以兵六萬攻雍州,屯于美原。是時,太祖方與諸將攻滄州,知俊不俟命,與康懷英等擊敗崇本,斬馘二萬,獲馬三千匹,執其偏裨百人。
 李思安為夾城攻潞州,久不下,太祖罷思安,拜知俊行營招討使,未至潞,夾城已破,徙西路行營招討使,敗邠、岐兵於幕谷。是時,延州高萬興叛楊崇本降梁,太祖遣知俊會萬興,攻下丹、延、鄜、坊四州,加檢校太尉兼侍中,封大彭郡王。知俊功益高,太祖性多猜忌,屢殺諸將,王重師無罪見殺,知俊益懼,不自安。太祖已下鄜、坊,遣知俊復攻邠州,知俊以軍食不給未行。



 太祖幸河中,使宣徽使王殷召知俊。其弟知浣為親軍指揮使,間遣人告知俊以不宜來。知俊遂叛,臣於李茂貞,以兵攻雍、華,執劉捍送于鳳翔。太祖使人謂知俊曰:「朕待卿至矣,何相
 負邪?」知俊報曰:「王重師不負陛下而族滅,臣非背德,但畏死爾!」太祖復使語曰:「朕固知卿以此,吾誅重師,乃劉捍誤我,致卿至此,吾豈不恨之邪?今捍已死,未能塞責。」知俊不報,以兵斷潼關。



 太祖遣劉鄩、牛存節攻知俊,知俊遂奔于茂貞,茂貞地狹,無以處之,使之西攻靈武。韓遜告急,太祖遣康懷英、寇彥卿等攻邠寧以牽之。知俊大敗懷英於升平,殺梁將許從實。茂貞大喜,以知俊為涇州節度使,使攻興元,取興、鳳,圍西縣。



 已而茂貞左右忌知俊功,以事間之,茂貞奪其軍。知俊乃奔于蜀,王建以為武信軍節度使,使返攻茂貞,取秦、鳳、階、成四州。建
 雖待知俊甚厚,然亦陰忌其材,嘗謂左右曰:「吾老矣,吾且死,知俊非爾輩所能制,不如早圖之!」而蜀人亦共嫉之。知俊為人色黑,而其生歲在丑。建之諸子,皆以「宗」、「承」為名,乃於里巷構為謠言曰:「黑牛出圈棕繩斷。」建益惡之,遂見殺。



 丁會丁會,字道隱,壽州壽春人也。少工挽喪之歌,尤能悽愴其聲以自喜。後去為盜,與梁太祖俱從黃巢。梁太祖鎮宣武,以為宣武都押衙。光啟四年,東都張全義襲破河陽,逐李罕之,罕之召晉兵圍河陽,全義告急。是時,梁軍在魏,乃遣會及葛從周等將萬人救之。會等行至河陰,
 謀曰:「罕之料吾不敢渡九鼎,以吾兵少而來遠,且不虞吾之速至也。出其不意,掩其不備者,兵家之勝策也。」乃渡九鼎,直趨河陽,戰于沇水,罕之大敗,河陽圍解。大順元年,梁軍擊魏,會及葛從周破黎陽、臨河,遂敗羅弘信于內黃。梁軍攻時溥於徐州,遣會別攻宿州,刺史張筠閉城距守,會堰汴水浸其東,城壞,筠降。兗州朱瑾以兵萬餘擊單父,會及瑾戰于金鄉,大敗之。光化二年,李罕之叛晉,以潞州降梁。會自河陽攻晉澤州,下之。乃以會為昭義軍留後,會畏梁太祖雄猜,常稱疾者累年。天復元年,太祖復起會為昭義軍節度使。昭宗遇弒,會與
 三軍縞素發哀。梁軍攻燕滄州,燕王守光乞師于晉,晉人為攻潞州,會乃降晉。晉王以會歸于太原,賜以甲第,位在諸將上。莊宗立,以會為都招討使。天祐七年,以疾卒于太原。唐興,追贈太師。



 賀德倫賀德倫,河西人也。少為滑州牙將。梁太祖兼領宣義,德倫從太祖征伐,以功累遷平盧軍節度使。貞明元年,魏州楊師厚卒,末帝以魏兵素驕難制,乃分相、澶、衛三州建昭德軍,以張筠為節度使;魏、博、貝三州仍為天雄軍,以德倫為節度使。



 遣劉鄩以兵六萬渡河,聲言攻鎮定,王彥章以騎兵五百入魏州,屯金波亭以虞變;分魏牙
 兵之半入昭德。租庸使遣孔目吏閱魏兵籍,檢校府庫。德倫促牙兵上道,牙兵親戚相決別,哭聲盈途。效節軍將張彥謀於其眾曰:「朝廷以我軍府彊盛,設法殘破之。況我六州舊為籓府,未嘗遠出河門,一旦離親戚,去鄉里,生不如死。」



 乃相與夜攻金波亭,彥章走出。遲明,魏兵攻牙城,殺五百餘人,執德倫致之樓上,縱兵大掠。



 末帝遣供奉官扈異馳至魏諭彥,許以刺史。彥謂異曰:「為我報皇帝,三軍不負朝廷,朝廷負三軍,割隸無名,所以亂耳。但以六州還魏,而詔劉鄩反兵,皇帝可以高枕。」異還,言彥狂蹶不足畏,宜促鄩兵擊之。末帝使人諭彥,以制
 置已定,不可復易。使者三反,彥怒曰:「傭保兒敢如是邪!」乃召羅紹威故吏司空頲曰:「為我作奏,若復依違,則渡河虜之耳!」末帝優詔答之,言:「王鎔死,鎮人請降,遣鄩以兵定鎮州,非有佗也,若魏不便之,即召鄩還。」戒彥勿為朝廷生事。



 彥乃以楊師厚鎮魏州嘗帶招討使,逼德倫論列之,末帝不許,諭以詔書,彥裂詔書抵于地,曰:「愚主聽人穿鼻,難與共事矣!」乃迫德倫降晉,德倫惶恐曰:「惟將軍命。乃遣牙將曹廷隱奉書莊宗。



 莊宗入魏,德倫以彥逼己,遣人陰訴於莊宗,莊宗斬彥於臨清而後入。徙德倫為大同軍節度使。行至太原,監軍張承業留之。王
 檀攻太原,德倫麾下多奔檀,承業懼德倫為變,殺之。



 閻寶閻寶,字瓊美,鄆州人也。少為朱瑾牙將,瑾走淮南,寶降於梁。梁太祖時,為諸軍都虞候,常從諸將征伐,未嘗獨立戰功。至末帝時,以寶為保義軍節度使。



 貞明三年,賀德倫以魏博降晉,晉軍攻下洺、磁、相、衛,移兵圍邢州。末帝遣捉生都指揮使張溫將五百騎救寶,溫至內黃,遇晉軍,乃降晉。晉遣溫將所降梁軍至城下招寶,寶遂降晉。晉王拜寶檢校太尉、同中書門下平章事,領天平軍節度使、東南面招討使,位在諸將上。梁、晉戰胡柳,晉軍敗。莊宗欲引兵退保臨濮,寶曰:「夫決勝料勢,決戰料情,
 情勢既得,斷在不疑。今梁兵窘蹙,其勢可破;勝而驕怠,其情可知。此不可失之時也。」莊宗謝曰:「微公,幾敗吾事。」乃整軍復戰,遂敗梁兵。十八年,晉軍討張文禮於鎮州,以寶為招討使。明年三月,寶戰敗,退保趙州。慚憤發疽卒,追贈太師。晉天福中,追封太原王。



 康延孝康延孝,代北人也。為太原軍卒,有罪亡命于梁。末帝遣段凝軍于河上,以延孝為左右先鋒指揮使。延孝見梁末帝任用群小,知其必亡,乃以百騎奔於唐。見莊宗于朝城,莊宗解御衣、金帶以賜之。拜延孝博州刺史、捧日軍使兼南面招討指揮使。莊宗屏人問延孝梁事,延孝
 具言:「末帝懦弱。趙巖婿也,張漢傑婦家,皆用事。段凝奸邪,以入金多為大將,自其父時故將皆出其下。王彥章,驍將也,遣漢傑監其軍而制之。小人進任,而忠臣勇士皆見疏斥,此其必亡之勢也。」莊宗又問梁計如何,曰:「臣在梁時,竊聞其議:期以仲冬大舉,遣董璋以陜虢、澤潞之眾出石會以攻太原;霍彥威以關西、汝、洛之兵掠邢洺以趨鎮定;王彥章以京師禁衛擊鄆州;段凝以河上之軍當陛下。」莊宗初聞延孝言梁必亡,喜,及聞其大舉也,懼,曰:「其將何以禦之?」延孝曰:「梁兵雖眾,分則無餘。臣請待其既分,以鐵騎五千自鄆趨汴,出其不意,搗其空
 虛,不旬日,天下定矣。」莊宗甚壯其言。



 後董璋等雖不出兵,而梁兵悉屬段凝于河上,京師無備,莊宗卒用延孝策,自鄆入汴,凡八日而滅梁。以功拜鄭州防禦使,賜姓名曰李紹琛。二年,遷保義軍節度使。



 三年,征蜀,以延孝為先鋒排陣斬斫使,破鳳州,取固鎮,降興州。與王衍戰三泉,衍敗走,斷吉柏江浮橋,延孝造舟以渡,進取綿州。衍復斷綿江浮橋。延孝謂招撫使李嚴曰:「吾遠軍千里,入人之國,利在速戰。乘衍破膽之時,但得百騎過鹿頭關,彼將迎降不暇。若修繕橋梁,必留數日,使衍得閉關為備,則勝負未可知也。」因與嚴乘馬浮江,軍士隨之濟
 者千餘人,遂入鹿頭關,下漢州,居三日,後軍始至。衍弟宗弼果以蜀降。延孝屯漢州,以俟魏王繼岌。



 蜀平,延孝功為多。左廂馬步軍都指揮使董璋位在延孝下,然特見重於郭崇韜。



 崇韜有軍事,獨召璋與計議,而不問延孝,延孝大怒,責璋曰:「吾有平蜀之功,公等僕遬相從,反俯首郭公之門,吾為都將,獨不能以軍法斬公邪?」璋訴於崇韜,崇韜解璋軍職,表為東川節度使,延孝愈怒曰:「吾冒白刃,犯險阻,以定兩川,璋有何功而得旄節!」因見崇韜言其不可。崇韜曰:「紹琛反邪?敢違吾節度!」



 延孝懼而退。明年崇韜死,延孝謂璋曰:「公復俯首何門邪?」璋求
 哀以免。



 繼岌班師,命延孝以萬二千人為殿,行至武連,聞朱友謙無罪見殺。友謙有子令德在遂州,莊宗遣使者詔繼岌即誅之。繼岌不遣延孝,而遣董璋,延孝已自疑,及璋過延孝軍,又不謁,延孝大怒,謂其下曰:「南平梁,西取蜀,其謀盡出於郭公,而汗馬之勞,攻城破敵者我也。今郭公已死,我豈得存?而友謙與我俱背梁以歸唐者,友謙之禍次及我矣!」延孝部下皆友謙舊將,知友謙被族,皆號哭訴于軍門曰:「朱公無罪,二百口被誅,舊將往往從死,我等死必矣!」延孝遂擁其眾自劍州返入蜀,自稱西川節度、三川制置等使。馳檄蜀人,數日之間,眾
 至五萬。繼岌遣任圜以七千騎追之,及於漢州,會孟知祥夾攻之,延孝戰敗,被擒,載以檻車。



 圜置酒軍中,引檻車至坐上,知祥酌大卮從車中飲之而謂曰:「公自梁朝脫身歸命,遂擁節旄。今平蜀之功,何患富貴,而入此檻車邪?」延孝曰:「郭崇韜佐命之臣,功在第一,兵不血刃而取兩川,一旦無罪,闔門受戮。顧如延孝,何保首領。以此不敢歸朝耳!」任圜東還,延孝檻車至鳳翔,莊宗遣宦者殺之。



\end{pinyinscope}