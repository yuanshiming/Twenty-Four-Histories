\article{卷十漢本紀第十}

\begin{pinyinscope}

 高
 祖睿文聖武昭肅孝皇帝,姓劉氏,初名知遠,其先沙陀部人也,其後世居于太原。知遠弱不好弄,嚴重寡言,面紫色,目多白睛,凜如也。與晉高祖俱事明宗,為偏將。明宗及梁人戰德勝,晉高祖馬甲斷,梁兵幾及,知遠以所乘馬授之,復取高祖馬殿而還,高祖德之。高祖留守北京,以知遠為押衙。



 潞王從珂反,愍帝出奔,高祖自鎮州朝京師,遇愍帝於衛州,止傳舍,知遠遣勇士石敢袖
 鐵槌侍高祖,以虞變。高祖與愍帝議事未決,左右欲兵之,知遠擁高祖入室,敢與左右格鬥而死,知遠即率兵盡殺愍帝左右,留帝傳舍而去。



 廢帝入立,高祖復鎮河東,已而有隙,高祖將舉兵,知遠與桑維翰密為高祖謀畫,贊成之。高祖即位於太原,以知遠為侍衛親軍都虞候,領保義軍節度使。契丹耶律德光送高祖至潞州,臨決,指知遠曰:「此都軍甚操剌,無大故勿棄之。」



 天福二年,遷侍衛馬步軍都指揮使,領忠武軍節度使。已而以杜重威代知遠領忠武,徙知遠領歸德,知遠恥與重威同制,杜門不出。高祖怒,欲罷其兵職,宰相
 趙瑩以為不可,高祖乃遣端明殿學士和凝就第宣諭,知遠乃受命。五年,徙鄴都留守。九月,朝京師,高祖幸其第。六年,拜河東節度使、北京留守。七年,高祖崩。



 知遠從高祖起太原,有佐命功,自出帝立,與契丹絕盟,用兵北方,常疑知遠勳位已高,幸晉多故而有異志,每優尊之。拜中書令,封太原王、幽州道行營招討使,又拜北面行營都統。開運二年四月,封北平王,三年五月,加守太尉,然王未嘗出兵。契丹寇澶州,別遣偉王攻鴈門,敗之于秀容。八月,殺吐渾白承福等族,取其貲鉅萬,良馬數千。



 四年,契丹犯京師,出帝北遷,王遣牙將王峻奉表契丹,耶律德光呼之為兒,賜以木拐一,木拐,虜法貴之如中國几杖,非優大臣不可得。峻持拐歸,虜人望之皆避道。峻還,為王言契丹必不能有中國,乃議建國。



 二月戊辰,河東行軍司馬張彥威等上箋勸進。辛未,皇帝即位,稱天福十二年。



 磁州賊首梁暉取相州來歸。武節都指揮使史弘肇取代州,殺其刺史王暉。晉州將藥可儔殺守將駱從
 朗及括錢使、諫議大夫趙熙來歸。辛巳,陜州留後趙暉、潞州留後王守恩來歸。三月丙戌朔,蠲河東雜稅。辛卯,延州軍亂,逐其節度使周密。壬辰,丹州指揮使高彥詢以其州來歸。壬寅,契丹遁,以其將蕭翰為宣武軍節度使守汴州。



 夏四月己未,右都押衙楊邠為樞密使,蕃漢兵馬都孔目官郭威權樞密副使。契丹陷相州,殺梁暉。癸亥,立魏國夫人李氏為皇后。甲子,河東節度判官蘇逢吉、觀察推官蘇禹珪為中書侍郎、同中書門下平章事。乙丑,侍衛親軍步軍都指揮使史弘肇取潞州。戊辰,奉國指揮使武行德以河
 陽來歸。史弘肇取澤州。丙子,契丹耶律德光卒於欒城,契丹入于鎮州。五月甲午,太原尹劉崇為北京留守。丙申,如東京。



 蕭翰遁歸于契丹,以郇國公李從益知南朝軍國事。戊申,次絳州,刺史李從朗來歸。



 六月丙辰,次河陽,殺李從益及其母于京師。甲子,至自太原。戊辰,改國號漢,赦罪人、蠲民稅。於闐遣使者來。是夏,劉昫薨。秋閏七月乙丑,禁造契丹服器。



 天雄軍節度使杜重威反,天平軍節度使高行周為鄴都行營都部署以討之。庚
 辰,追尊祖考為皇帝,妣為皇后:高祖湍謚曰明元,廟號文祖,祖妣李氏謚曰明貞;曾祖昂謚曰恭僖,廟號德祖,祖妣楊氏謚曰恭惠;祖僎謚曰昭憲,廟號翼祖,祖妣李氏謚曰昭穆;考琠謚曰章聖,廟號顯祖,妣安氏謚曰章懿。以漢高皇帝為高祖,光武皇帝為世祖,皆不祧。八月,護聖指揮使白再榮逐契丹,以鎮州來歸。丙申,安國軍節度使薛懷讓殺契丹之將劉鐸,入于邢州。九月甲戌,吏部尚書竇貞固守司空兼門下侍郎,翰林學士、中書舍人李濤為中書侍郎:同中書門下平章事。庚辰,北征。



 冬十月甲申,次韋城,赦河北。十一月壬申,杜重威降。十
 二月癸巳,至自鄴都。



 乾祐元年春正月乙卯,大赦,改元。己未,更名暠。丁丑,皇帝崩于萬歲殿。



 隱皇帝,高祖第二子承祐也。高祖即位,拜右衛上將軍、大內都點檢。魏王承訓長而賢,高祖愛之,方屬以為嗣,承訓薨,高祖不豫,悲哀疾劇,乃以承祐屬諸將相。宰相蘇逢吉曰:「皇子承祐未封王,請亟封之。」未及封而高祖崩,秘不發喪,殺杜重威。



 乾祐元年二月辛巳,封承祐周王。是日,皇帝即位于柩前。壬辰,右衛大將軍、鳳翔巡檢使王景崇及蜀人戰于大散關,敗之。癸巳,大赦。三月壬戌,竇貞固為大行皇帝
 山陵使,吏部侍郎段希堯為副,太常卿張昭為禮儀使,兵部侍郎盧價為鹵簿使,御史中丞邊蔚為儀仗使。丁丑,李濤罷。護國軍節度使李守貞反,陷潼關。夏四月辛巳,陜州兵馬都監王玉克潼關。壬午,永興軍將趙思綰叛附于李守貞,客省使王峻帥師屯于關西。楊邠為中書侍郎兼吏部尚書、同中書門下平章事,郭威為樞密使,鎮寧軍節度使郭從義為永興軍兵馬都部署。戊子,保義軍節度使白文珂為河中兵馬都部署。河決原武。五月己未,回鶻遣使者來。乙亥,魏州內黃民武進妻一產三男子。河決滑州魚
 池。旱,蝗。秋七月戊申朔,彰德軍節度使王繼弘殺其判官張易。翽鵒食蝗。丙辰,禁捕翽鵒。庚申,郭威同中書門下平章事。癸亥,契丹鄚州刺史王彥徽來奔。庚午,殺成德軍副使張鵬。乙亥,王景崇叛附于李守貞。八月壬午,郭威討李守貞。九月,西面行營都虞候尚弘遷及趙思綰戰,敗績。冬十月甲申,吐蕃使斯漫篤蘭氈藥斯來。十一月甲寅,殺太子太傅李崧,滅其族。壬申,葬睿文聖武昭肅孝皇帝于睿陵。十二月己卯,彰武軍節度使高允權殺太子太師致仕劉景巖。



 二年春正月乙巳朔,赦囚。二月丙子,蠲民紐配租。夏五月,李守貞之將周光遜降。乙丑,趙思綰降。六月辛卯,回鶻首領楊彥珣來。西涼府遣使者來。蝗。秋七月丁巳,郭威殺華州留後趙思綰于京兆。甲子,克河中。八月,郭從義殺前永興巡檢喬守溫。丙戌,郭威使來獻俘。冬十月,契丹寇趙、魏,群臣進添都馬。契丹陷內丘。己丑,郭威及宣徽南院使王峻伐契丹。十一月,契丹遁。



 三年春正月,西面行營都部署趙暉克鳳翔。丙午,郭威進添都馬。壬子,趙暉獻馘俘。二月甲戌,旌
 表潁州汝陰民麴溫門閭。三月己酉,寒食,望祭于南御園。



 夏四月壬午,郭威以樞密使為天雄軍節度使。六月癸卯,河決原武。秋八月,達靼來附。冬十一月丙子,殺楊邠及侍衛親軍都指揮使史弘肇、三司使王章,皆滅其族。



 郭威反。庚辰,義成軍節度使宋延渥叛附于威。壬午,威犯封丘,泰寧軍節度使慕容彥超軍於七里店。癸未,勞軍于北郊。甲申,勞軍于劉子陂。慕容彥超及郭威戰,敗績,開封尹侯益叛降于威。郭允明反。乙酉,皇帝崩,蘇逢吉自殺。漢亡。



 嗚呼!
 人君即位稱元年,常事爾,古不以為重也。孔子未修《春秋》,其前固已如此,雖暴君昏主,妄庸之史,其記事先後遠近,莫不以歲月一二數之,乃理之自然也。其謂一為元,亦未嘗有法,蓋古人之語爾。及後世曲學之士,始謂孔子書「元年」為《春秋》大法,遂以改元為重事。自漢以後,又名年以建元,而正偽紛雜,稱號遂多,不勝其紀也。五代,亂世也,其事無法而不合於理者多矣,皆不足道也。至其年號乖錯以惑後世,則不可以不
 明。初,梁太祖以乾化二年遇弒,明年,末帝已誅友珪,黜其鳳曆之號,復稱乾化三年,尚為有說。至漢高祖建國,黜晉出帝開運四年,復稱天福十二年者,何哉?蓋以其愛憎之私爾。方出帝時,漢高祖居太原,常憤憤下視晉,而晉亦陽優禮之,幸而未見其隙。及契丹滅晉,漢未嘗有赴難之意。出帝已北遷,方陽以兵聲言追之,至土門而還。及其即位改元,而黜開運之號,則其用心可知矣。蓋其於出帝無復君臣之義,而幸禍以為利者,其素志也,可勝歎哉!夫所謂有諸中必形於外者,其見於是乎!



\end{pinyinscope}