\article{卷二十四唐臣傳第十二}

\begin{pinyinscope}

 郭崇韜郭崇韜,代州鴈門人也,為河東教練使。為人明敏,能應對,以材幹見稱。莊宗為晉王,孟知祥為中門使,崇韜為副使。中門之職,參管機要,先時,吳珙、張虔厚等皆以中門使相繼獲罪。知祥懼,求外任,莊宗曰:「公欲避事,當舉可代公者。」知祥乃薦崇韜為中門使,甚見親信。



 晉兵圍張文禮於鎮州,久不下,而定州王都引契丹入寇。契丹至新樂,晉人皆恐,欲解圍去,莊宗未決,崇韜曰:「契丹之
 來,非救文禮,為王都以利誘之耳,且晉新破梁軍,宜乘已振之勢,不可遽自退怯。」莊宗然之,果敗契丹。莊宗即位,拜崇韜兵部尚書、樞密使。



 梁王彥章擊破德勝,唐軍東保楊劉,彥章圍之。莊宗登壘,望見彥章為重塹以絕唐軍,意輕之,笑曰:「我知其心矣,其欲持久以弊我也。」即引短兵出戰,為彥章伏兵所射,大敗而歸。莊宗問崇韜:「計安出?」是時,唐已得鄆州矣,崇韜因曰:「彥章圍我於此,其志在取鄆州也。臣願得兵數千,據河下流,築壘於必爭之地,以應鄆州為名,彥章必來爭,既分其兵,可以圖也。然板築之功難卒就,陛下日以精兵挑戰,使彥章
 兵不得東,十日壘成矣。」莊宗以為然,乃遣崇韜與毛璋將數千人夜行,所過驅掠居人,毀屋伐木,渡河築壘於博州東,晝夜督役,六日壘成。彥章果引兵急攻之,時方大暑,彥章兵熱死,及攻壘不克,所失太半,還趨楊劉,莊宗迎擊,遂敗之。



 康延孝自梁奔唐,先見崇韜,崇韜延之臥內,盡得梁虛實。是時,莊宗軍朝城,段凝軍臨河。唐自失德勝,梁兵日掠澶、相,取黎陽、衛州,而李繼韜以澤潞叛入于梁,契丹數犯幽、涿,又聞延孝言梁方召諸鎮兵欲大舉,唐諸將皆憂惑,以謂成敗未可知。莊宗患之,以問諸將,諸將皆曰:「唐得鄆州,隔河難守,不若棄鄆與梁,而
 西取衛州、黎陽,以河為界,與梁約罷兵,毋相攻,庶幾以為後圖。」莊宗不悅,退臥帳中,召崇韜問計,崇韜曰:「陛下興兵仗義,將士疲戰爭、生民苦轉餉者,十餘年矣。況今大號已建,自河以北,人皆引首以望成功而思休息。今得一鄆州,不能守而棄之,雖欲指河為界,誰為陛下守之?且唐未失德勝時,四方商賈,徵輸必集,薪芻糧餉,其積如山。自失南城,保楊劉,道路轉徙,耗亡太半。而魏、博五州,秋稼不稔,竭民而斂,不支數月,此豈按兵持久之時乎?臣自康延孝來,盡得梁之虛實,此真天亡之時也。願陛下分兵守魏,固楊劉,而自鄆長驅搗其巢穴,不出
 半月,天下定矣!」莊宗大喜曰:「此大丈夫之事也!」因問司天,司天言:「歲不利用兵。」崇韜曰:「古者命將,鑿凶門而出。況成算已決,區區常談,豈足信也!」莊宗即日下令軍中,歸其家屬於魏,夜渡楊劉,從鄆州入襲汴,八日而滅梁。莊宗推功,賜崇韜鐵券,拜侍中、成德軍節度使,依前樞密使。莊宗與諸將以兵取天下,而崇韜未嘗居戰陣,徒以謀議居佐命第一之功,位兼將相,遂以天下為己任,遇事無所迴避。而宦官、伶人用事,特不便也。



 初,崇韜與宦者馬紹宏俱為中門使,而紹宏位在上。及莊宗即位,二人當為樞密使,而崇韜不欲紹宏在己上,乃以張
 居翰為樞密使,紹宏為宣徽使。紹宏失職怨望,崇韜因置內勾使,以紹宏領之。凡天下錢穀出入于租庸者,皆經內勾。既而文簿繁多,州縣為弊,遽罷其事,而紹宏尤側目。崇韜頗懼,語其故人子弟曰:「吾佐天子取天下,今大功已就,而群小交興,吾欲避之,歸守鎮陽,庶幾免禍,可乎?」



 故人子弟對曰:「俚語曰:『騎虎者,勢不得下。』今公權位已隆,而下多怨嫉,一失其勢,能自安乎?」崇韜曰:「奈何?」對曰:「今中宮未立,而劉氏有寵,宜請立劉氏為皇后,而多建天下利害以便民者,然後退而乞身。天子以公有大功而無過,必不聽公去。是外有避權之名,而內有中
 宮之助,又為天下所悅,雖有讒間,其可動乎?」崇韜以為然,乃上書請立劉氏為皇后。



 崇韜素廉,自從入洛,始受四方賂遺,故人子弟或以為言,崇韜曰:「吾位兼將相,祿賜巨萬,豈少此邪?今籓鎮諸侯,多梁舊將,皆主上斬袪射鉤之人也。今一切拒之,豈無反側?且藏於私家,何異公帑?」明年,天子有事南郊,乃悉獻其所藏,以佐賞給。



 莊宗已郊,遂立劉氏為皇后。崇韜累表自陳,請依唐舊制,還樞密使於內臣,而並辭鎮陽,優詔不允。崇韜又曰:「臣從陛下軍朝城,定計破梁,陛下撫臣背而約曰:『事了,與卿一鎮。』今天下一家,俊賢並進,臣憊矣,願乞身如約。」莊
 宗召崇韜謂曰:「朝城之約,許卿一鎮,不許卿去。欲捨朕,安之乎?」崇韜因建天下利害二十五事,施行之。



 李嗣源為成德軍節度使,徙崇韜忠武。崇韜因自陳權位已極,言甚懇至。莊宗曰:「豈可朕居天下之尊,使卿無尺寸之地?」崇韜辭不已,遂罷其命,仍為侍中、樞密使。



 同光三年夏,霖雨不止,大水害民田,民多流死。莊宗患宮中暑濕不可居,思得高樓避暑。宦官進曰:「臣見長安全盛時,大明、興慶宮樓閣百數。今大內不及故時卿相家。」莊宗曰:「吾富有天下,豈不能作一樓?」乃遣宮苑使王允平營之。



 宦官曰:「郭崇韜眉頭不伸,常為租庸惜財用,陛下雖欲
 有作,其可得乎?」莊宗乃使人問崇韜曰:「昔吾與梁對壘於河上,雖祁寒盛暑,被甲跨馬,不以為勞。今居深宮,蔭廣廈,不勝其熱,何也?」崇韜對曰:「陛下昔以天下為心,今以一身為意,艱難逸豫,為慮不同,其勢自然也。願陛下無忘創業之難,常如河上,則可使繁暑坐變清涼。」莊宗默然。終遣允平起樓,崇韜果切諫。宦官曰:「崇韜之第,無異皇居,安知陛下之熱!」由是讒間愈入。



 河南縣令羅貫,為人彊直,頗為崇韜所知。貫正身奉法,不受權豪請託,宦官、伶人有所求請,書積几案,一不以報,皆以示崇韜。崇韜數以為言,宦官、伶人由此切齒。河南自故唐時張
 全義為尹,縣令多出其門,全義廝養畜之。及貫為之,奉全義不屈,縣民恃全義為不法者,皆按誅之。全義大怒,嘗使人告劉皇后,從容為白貫事,而左右日夜共攻其短。莊宗未有以發。皇太后崩,葬坤陵,陵在壽安,莊宗幸陵作所,而道路泥塗,橋壞。莊宗止輿問:「誰主者?」宦官曰:「屬河南。」



 因亟召貫,貫至,對曰:「臣初不奉詔,請詰主者。」莊宗曰:「爾之所部,復問何人!」即下貫獄,獄吏榜掠,體無完膚。明日,傳詔殺之。崇韜諫曰:「貫罪無佗,橋道不修,法不當死。」莊宗怒曰:「太后靈駕將發,天子車輿往來,橋道不修,卿言無罪,是朋黨也!」崇韜曰:「貫雖有罪,當具獄行法
 於有司。陛下以萬乘之尊,怒一縣令,使天下之人,言陛下用法不公,臣等之過也。」莊宗曰:「貫,公所愛,任公裁決!」因起入宮,崇韜隨之,論不已。莊宗自闔殿門,崇韜不得入。



 貫卒見殺。



 明年征蜀,議擇大將。時明宗為總管,當行。而崇韜以讒見危,思立大功為自安之計,乃曰:「契丹為患北邊,非總管不可禦。魏王繼岌,國之儲副,而大功未立,且親王為元帥,唐故事也。」莊宗曰:「繼岌,小子,豈任大事?必為我擇其副。」崇韜未及言,莊宗曰:「吾得之矣,無以易卿也。」乃以繼岌為西南面行營都統,崇韜為招討使,軍政皆決崇韜。



 唐軍入蜀,所過迎降。王衍弟宗弼,陰送
 款于崇韜,求為西川兵馬留後,崇韜以節度使許之。軍至成都,宗弼遷衍于西宮,悉取衍嬪妓、珍寶奉崇韜及其子廷誨。



 又與蜀人列狀見魏王,請崇韜留鎮蜀。繼岌頗疑崇韜,崇韜無以自明,因以事斬宗弼及其弟宗渥、宗勳,沒其家財。蜀人大恐。



 崇韜素嫉宦官,嘗謂繼岌曰:「王有破蜀功,師旋,必為太子,俟主上千秋萬歲後,當盡去宦官,至於扇馬,亦不可騎。」繼岌監軍李從襲等見崇韜專任軍事,心已不平,及聞此言,遂皆切齒,思有以圖之。莊宗聞破蜀,遣宦官向延嗣勞軍,崇韜不郊迎,延嗣大怒,因與從襲等共構之。延嗣還,上蜀簿,得兵三十萬,
 馬九千五百匹,兵器七百萬,糧二百五十三萬石,錢一百九十二萬緡,金銀二十二萬兩,珠玉犀象二萬,文錦綾羅五十萬匹。莊宗曰:「人言蜀天下之富國也,所得止於此邪?」延嗣因言蜀之寶貨皆入崇韜,且誣其有異志,將危魏王。莊宗怒,遣宦官馬彥珪至蜀,視崇韜去就。彥珪以告劉皇后,劉皇后教彥珪矯詔魏王殺之。



 崇韜有子五人,其二從死於蜀,餘皆見殺。其破蜀所得,皆籍沒。明宗即位,詔許歸葬,以其太原故宅賜其二孫。



 當崇韜用事,自宰相豆盧革、韋悅等皆傾附之,崇韜父諱弘,革等即因佗事,奏改弘文館為崇文館。以其姓郭,因以
 為子儀之後,崇韜遂以為然。其伐蜀也,過子儀墓,下馬號慟而去,聞者頗以為笑。然崇韜盡忠國家,有大略。其已破蜀,因遣使者以唐威德風諭南詔諸蠻,欲因以綏來之,可謂有志矣!



 安重誨安重誨,應州人也。其父福遷,事晉為將,以驍勇知名。梁攻朱宣于鄆州,晉兵救宣,宣敗,福遷戰死。重誨少事明宗,為人明敏謹恪。明宗鎮安國,以為中門使,及兵變於魏,所與謀議大計,皆重誨與霍彥威決之。明宗即位,以為左領軍衛大將軍、樞密使,兼領山南東道節度使。固辭不拜,改兵部尚書,使如故。在位六年,累加侍中兼中
 書令。



 重誨自為中門使,已見親信,而以佐命功臣,處機密之任,事無大小,皆以參決,其勢傾動天下。雖其盡忠勞心,時有補益,而恃功矜寵,威福自出,旁無賢人君子之助,其獨見之慮,禍釁所生,至於臣主俱傷,幾滅其族,斯其可哀者也。



 重誨嘗出,過御史臺門,殿直馬延誤衝其前導,重誨怒,即臺門斬延而後奏。



 是時,隨駕子軍士桑弘遷,毆傷相州錄事參軍;親從兵馬使安虔,走馬衝宰相前導。弘遷罪死,虔決杖而已。重誨以斬延,乃請降敕處分,明宗不得已從之,由是御史、諫官無敢言者。



 宰相任圜判三司,以其職事與重誨爭,不能得,圜怒,辭
 疾,退居於磁州。朱守殷以汴州反,重誨遣人矯詔馳至其家,殺圜而後白,誣圜與守殷通謀,明宗皆不能詰也。而重誨恐天下議己因取三司積欠二百餘萬,請放之,冀以悅人而塞責,明宗不得已,為下詔蠲除之。其威福自出,多此類也。



 是時,四方奏事,皆先白重誨然後聞。河南縣獻嘉禾,一莖五穗,重誨視之曰:「偽也。」笞其人而遣之。夏州李仁福進白鷹,重誨卻之,明日,白曰:「陛下詔天下毋得獻鷹鷂,而仁福違詔獻鷹,臣已卻之矣。」重誨出,明宗陰遣人取之以入。



 佗日,按鷹於西郊,戒左右:「無使重誨知也!」宿州進白兔,重誨曰:「兔陰且狡,雖白何為!」遂
 卻而不白。



 明宗為人雖寬厚,然其性夷狄,果於殺人。馬牧軍使田令方所牧馬,瘠而多斃,坐劾當死,重誨諫曰:「使天下聞以馬故,殺一軍使,是謂貴畜而賤人。」令方因得減死。明宗遣回鶻侯三馳傳至其國。侯三至醴泉縣,縣素僻,無驛馬,其令劉知章出獵,不時給馬,侯三遽以聞。明宗大怒,械知章至京師,將殺之,重誨從容為言,知章乃得不死。其盡忠補益,亦此類也。



 重誨既以天下為己任,遂欲內為社稷之計,而外制諸侯之彊。然其輕信韓玫之譖,而絕錢鏐之臣;徒陷彥溫於死,而不能去潞王之患;李嚴一出而知祥貳,仁矩未至而董璋叛;四方
 騷動,師旅並興,如投膏止火,適足速之。此所謂獨見之慮,禍釁所生也。



 錢鏐據有兩浙,號兼吳越而王,自梁及莊宗,常異其禮,以羈縻臣屬之而已。



 明宗即位,鏐遣使朝京師,寓書重誨,其禮慢。重誨怒,未有以發,乃遣其嬖吏韓玫、副供奉官烏昭遇復使於鏐。而玫恃重誨勢,數凌辱昭遇,因醉使酒,以馬箠擊之。鏐欲奏其事,昭遇以為辱國,固止之。及玫還,返譖於重誨曰:「昭遇見鏐,舞蹈稱臣,而以朝廷事私告鏐。」昭遇坐死御史獄,乃下制削奪鏐官爵,以太師致仕,於是錢氏遂絕於唐矣。



 潞王從珂為河中節度使,重誨以謂從珂非李氏子,後必為國
 家患,乃欲陰圖之。



 從珂閱馬黃龍莊,其牙內指揮使楊彥溫閉城以叛。從珂遣人謂彥溫曰:「我遇汝厚,何苦而反邪?」報曰:「彥溫非叛也,得樞密院宣,請公趨歸朝廷耳!」從珂走虞鄉,馳騎上變。明宗疑其事不明,欲究其所以,乃遣殿直都知范氳以金帶襲衣、金鞍勒馬賜彥溫,拜彥溫絳州刺史,以誘致之。重誨固請用兵,明宗不得已,乃遣侍衛指揮使藥彥稠、西京留守索自通率兵討之,而誡曰:「為我生致彥溫,吾將自訊其事。」彥稠等攻破河中,希重誨旨,斬彥溫以滅口。重誨率群臣稱賀,明宗大怒曰:「朕家事不了,卿等不合致賀!」從珂罷鎮,居清化里
 第。重誨數諷宰相,言從珂失守,宜得罪,馮道因白請行法。明宗怒曰:「吾兒為姦人所中,事未辨明,公等出此言,是不欲容吾兒人間邪?」趙鳳因言:「《春秋》責帥之義,所以勵為臣者。」明宗曰:「皆非公等意也!」道等惶恐而退。居數日,道等又以為請,明宗顧左右而言他。明日,重誨乃自論列,明宗曰:「公欲如何處置,我即從公!」



 重誨曰:「此父子之際,非臣所宜言,惟陛下裁之。」明宗曰:「吾為小校時,衣食不能自足,此兒為我擔石灰,拾馬糞,以相養活,今貴為天子,獨不能庇之邪!



 使其杜門私第,亦何與公事!」重誨由是不復敢言。



 孟知祥鎮西川,董璋鎮東川,二人皆
 有異志,重誨每事裁抑,務欲制其姦心,凡兩川守將更代,多用己所親信,必以精兵從之,漸令分戍諸州,以虞緩急。二人覺之,以為圖己,益不自安。既而遣李嚴為西川監軍,知祥大怒,斬嚴;又分閬州為保寧軍,以李仁矩為節度使以制璋,且削其地,璋以兵攻殺仁矩。二人遂皆反。



 唐兵戍蜀者,積三萬人,其後知祥殺璋,兼據兩川,而唐之精兵皆陷蜀。



 初,明宗幸汴州,重誨建議,欲因以伐吳,而明宗難之。其後戶部尚書李金粦得吳諜者言:「徐知誥欲舉吳國以稱籓,願得安公一言以為信。」鏻即引諜者見重誨,重誨大喜以為然,乃以玉帶與諜者,
 使遺知誥為信,其直千緡。初不以其事聞,其後逾年,知誥之問不至,始奏貶鏻行軍司馬。已而捧聖都軍使李行德、十將張儉告變,言:「樞密承旨李虔徽語其客邊彥溫云:『重誨私募士卒,繕治甲器,欲自伐吳。又與諜者交私。』」明宗以問重誨,重誨惶恐,請究其事。明宗初頗疑之,大臣左右皆為之辨,既而少解,始告重誨以彥溫之言,因廷詰彥溫,具伏其詐,於是君臣相顧泣下。彥溫、行德、儉皆坐族誅。重誨因求解職,明宗慰之曰:「事已辨,慎無措之胸中。」重誨論請不已,明宗怒曰:「放卿去,朕不患無人!」顧武德使孟漢瓊至中書,趣馮道等議代重誨者。馮
 道曰:「諸公茍惜安公,使得罷去,是紓其禍也。」趙鳳以為大臣不可輕動。遂以范延光為樞密使,而重誨居職如故。



 董璋等反,遣石敬瑭討之,而川路險阻,糧運甚艱,每費一石,而致一斗。自關以西,民苦輸送,往往亡聚山林為盜賊。明宗謂重誨曰:「事勢如此,吾當自行。」



 重誨曰:「此臣之責也。」乃請行。關西之人聞重誨來,皆已恐動,而重誨日馳數百里,遠近驚駭。督趣糧運,日夜不絕,斃踣道路者,不可勝數。重誨過鳳翔,節度使朱弘昭延之寢室,使其妻子奉事左右甚謹。重誨酒酣,為弘昭言:「昨被讒構,幾不自全,賴人主明聖,得保家族。」因感歎泣下。重誨
 去,弘昭馳騎上言:「重誨怨望,不可令至行營,恐其生事。」而宣徽使孟漢瓊自行營使還,亦言西人震駭之狀,因述重誨過惡。重誨行至三泉,被召還。過鳳翔,弘昭拒而不納,重誨懼,馳趨京師。未至,拜河中節度使。



 重誨已罷,希旨者爭求其過。宦者安希倫,坐與重誨交私,常與重誨陰伺宮中動息,事發棄市。重誨益懼,因上章告老。以太子太師致仕;而以李從璋為河中節度使,遣藥彥稠率兵如河中虞變。重誨子崇緒、崇贊,宿衛京師,聞制下,即日奔其父。重誨見之,驚曰:「渠安得來!」已而曰:「此非渠意,為人所使耳。吾以一死報國,餘復何言!」乃械送
 二子于京師,行至陜州,下獄。明宗又遣翟光業至河中,視重誨去就,戒曰:「有異志,則與從璋圖之。」又遣宦者使於重誨。使者見重誨,號泣不已,重誨問其故,使者曰:「人言公有異志,朝廷遣藥彥稠率師至矣!」重誨曰:「吾死未塞責,遽勞朝廷興師,以重明主之憂。」光業至,從璋率兵圍重誨第,入拜于庭。重誨降而答拜,從璋以楇擊其首,重誨妻走抱之而呼曰:「令公死未晚,何遽如此!」又擊其首,夫妻皆死,流血盈庭。從璋檢責其家貲,不及數千緡而已。明宗下詔,以其絕錢鏐,致孟知祥、董璋反,及議伐吳,以為罪。



 並殺其二子,其餘子孫皆免。



 重誨得罪,知其
 必死,歎曰:「我固當死,但恨不與國家除去潞王!」此其恨也。



 嗚呼,官失其職久矣!予讀梁宣底,見敬翔、李振為崇政院使,凡承上之旨,宣之宰相而奉行之。宰相有非其見時而事當上決者,與其被旨而有所復請者,則具記事而入,因崇政使聞,得旨則復宣而出之。梁之崇政使,乃唐樞密之職,蓋出納之任也,唐常以宦者為之,至梁戒其禍,始更用士人,其備顧問、參謀議于中則有之,未始專行事于外也。至崇韜、重誨為之,始復唐樞密之名,然權侔於宰
 相矣。



 從世因之,遂分為二,文事任宰相,武事任樞密。樞密之任既重,而宰相自此失其職也。



\end{pinyinscope}