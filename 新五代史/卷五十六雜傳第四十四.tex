\article{卷五十六雜傳第四十四}

\begin{pinyinscope}

 和凝和凝,字成績,鄆州須昌人也。其九世祖逢堯為唐監察御史,其後世遂不復宦學。凝父矩,性嗜酒,不拘小節,然獨好禮文士,每傾貲以交之,以故凝得與之游。



 而凝幼聰敏,形神秀發。舉進士,梁義成軍節度使賀瑰辟為從事。瑰與唐莊宗戰于胡柳,瑰戰敗,脫身走,獨凝隨之,反顧見凝,麾之使去。凝曰:「丈夫當為知己死,吾恨未得死所爾,豈可去也!」已而一騎追瑰幾及,凝叱之不止,即
 引弓射殺之,瑰由此得免。瑰歸,戒其諸子曰:「和生,志義之士也,後必富貴,爾其謹事之!」因妻之以女。天成中,拜殿中侍御史,累遷主客員外郎,知制誥,翰林學士,知貢舉。是時,進士浮薄,喜為喧嘩以動主司。主司每放榜,則圍之以棘,閉省門,絕人出入以為常。凝徹棘開門,而士皆肅然無嘩,所取皆一時之秀,稱為得人。晉初,拜端明殿學士,兼判度支,為翰林學士承旨。高祖數召之,問以時事,凝所對皆稱旨。天福五年,拜中書侍郎、同中書門下平章事。



 高祖將幸鄴,而襄州安從進反迹已見。凝曰:「陛下幸鄴,從進必因此時反,則將奈何?」高祖曰:「卿
 將何以待之?」凝曰:「先人者,所以奪人也。請為宣敕十餘通,授之鄭王,有急則命將擊之。」高祖以為然。是時,鄭王為開封尹,留不從幸,乃授以宣敕。高祖至鄴,從進果反,鄭王即以宣敕命騎將李建崇、焦繼勛等討之。從進謂高祖方幸鄴,不意晉兵之速也,行至花山,遇建崇等兵,以為神,遂敗走。出帝即位,加右僕射,歲餘,罷平章事,遷左僕射。漢高祖時,拜太子太傅,封魯國公。顯德二年卒,年五十八,贈侍中。



 凝好飾車服,為文章以多為富,有集百餘卷,嘗自鏤板以行于世,識者多非之。



 然性樂善,好稱道後進之士。唐故事,知貢舉者所放進士,以己及第
 時名次為重。



 凝舉進士及第時第五,後知舉,選範質為第五。後質位至宰相,封魯國公,官至太子太傅,皆與凝同,當時以為榮焉。



 趙瑩趙瑩,字玄輝,華州華陰人也。為人純厚,美風儀。事梁將康延孝為從事。晉高祖為保義軍節度使,以瑩掌書記,自是徙鎮常以瑩從。高祖將起兵太原,以問諸將吏,將吏或贊成之,瑩獨懼形于色,勸高祖毋反。高祖雖不用其言,心甚愛之。



 高祖即位,拜翰林學士承旨、戶部侍郎、同中書門下平章事。累拜中書令。出為晉昌軍節度使、開封尹。是時,出帝童昏,馮玉、李彥韜等用事,與桑維翰
 爭權,乃共譖去之,以瑩柔而易制,故復引以為相。契丹滅晉,瑩從出帝北徙虜中,瑩事兀欲為太子太保。周太祖時,與契丹通好,遣尚書左丞田敏使于契丹,遇瑩于幽州,瑩見敏悲不自勝。瑩子易則、易從。當其徙而北也,與易從俱,而易則留事漢,官至刑部郎中。後瑩病將卒,告于契丹,願以尸還中國,契丹許之。及卒,遣易從護其喪南歸。太祖憐之,贈瑩太傅,葬于華陰。



 馮玉馮玉,字璟臣,定州人也。少舉進士不中。馮贇為河東節度使,辟為推官。入拜監察御史,累遷禮部郎中,為鹽鐵判官。晉出帝納玉姊為后,玉以后戚知制誥,拜中書舍
 人。玉不知書,而與殷鵬同為舍人,制誥常遣鵬代作。頃之,玉出為潁州團練使,拜端明殿學士、戶部侍郎,遷樞密使、中書侍郎、同中書門下平章事。是時,出帝童昏,馮皇后用事,軍國大務,一決於玉。玉嘗有疾在告,自刺史已上,宰相不敢除授,以俟玉決。玉除中書舍人盧價為工部侍郎,桑維翰以價資望淺為不可,由是與維翰有隙,維翰由此罷相。玉為相,四方賄,積貲巨萬。契丹滅晉,張彥澤先以兵入京師,兵士爭先入玉家,其貲一夕而盡。明日見彥澤,猶諂笑,自言願得持晉玉璽獻契丹,以冀恩獎。彥澤不納。出帝之北,玉從入契丹,契丹以為
 太子太保。周廣順三年,其子傑自契丹逃歸,玉懼,以憂卒。



 盧質盧質,字子徵,河南人也。父望,唐司勳郎中。質幼聰惠,善屬文。事唐為祕書郎,丁母憂,解職。後去遊太原,晉王以為河東節度掌書記。質與張承業等定議立莊宗為嗣。莊宗將即位,以質為大禮使,拜行臺禮部尚書。莊宗即位,欲以質為相。質性疏逸,不欲任責,因固辭不受。拜太原尹、北京留守,遷戶部尚書、翰林學士。從平梁,權判租庸,遷兵部尚書,後為學士承旨,仍賜「論思匡佐功臣」。



 天成元年,拜匡國軍節度使。三年,拜兵部尚書,判太常卿
 事。歷鎮河陽、橫海。



 初,梁已篡唐,封哀帝為濟陰王,既而酖殺之,瘞于曹州。同光三年,莊宗將議改葬,而曹太后崩,乃止。因其故壟,稍廣其封,以時薦饗而已。質乃建議立廟追謚,謚曰昭宣光烈孝皇帝,廟號景宗。天成四年八月戍申,明宗御文明殿,遣質奉冊立廟于曹州。而議者以謂輝王不幸為賊臣所立,而昭宗、何皇后皆為梁所弒,遂以亡國,而「昭宣光烈」非所宜稱,且立廟稱宗而不入太廟,皆非是。共以此非質,大臣亦知其不可,乃奏去廟號。



 秦王從榮坐謀反誅,質以右僕射權知河南府事。廢帝反鳳翔,愍帝發兵誅之,竭帑藏以厚賞,而兵至
 鳳翔皆叛降。廢帝悉將而東,事成許以重賞,而軍士皆過望。



 廢帝入立,有司獻籍數甚少,廢帝暴怒。自諸鎮至刺史,皆進錢帛助國用,猶不足,三司使王玫請率民財以佐用。乃使質與玫等共議配率,而貧富不均,怨訟並起,囚繫滿獄。六七日間,所得不滿十萬。廢帝患之,乃命質等借民屋課五月,由是民大咨怨。晉高祖入立,質以疾分司西京,拜太子太保。卒,年七十六,贈太子太師,謚曰文忠。



 呂琦呂琦,字輝山,幽州安次人也。父兗,為橫海軍節度判官。節度使劉守文與其弟守光以兵相攻,守文敗死,其吏
 民立其子延祚而事之,以兗為謀主。已而延祚又為守光所敗,兗見殺。守光怒兗,并族其家。琦年十五,見執,將就刑,兗故客趙玉紿其監者曰:「此吾弟也。」監者信之,縱琦去。玉與琦得俱走,琦足弱不能行,玉負之而行,逾數百里,變姓名,乞食于道,以免。



 琦為人美風儀,重節概,少喪其家,游學汾、晉之間。唐莊宗鎮太原,以為代州軍事推官。後為橫海趙德鈞節度推官,入為殿中侍御史。明宗時,為駕部員外郎,兼侍御史知雜事。河陽主藏吏盜所監物,下軍巡獄,獄吏尹訓納賂反其獄,其冤家訴于朝,下御史臺按驗,得訓贓狀,奏攝訓赴臺。訓為安重誨
 所庇,不與,琦請不已,訓懼自殺,獄乃辨,蒙活者甚眾。歲餘,遷禮部郎中、史館修撰。



 長興中,廢帝失守河中,罷居清化坊,與琦同巷,琦數往過之。後廢帝入立,待琦甚厚,拜知制誥、給事中、樞密院直學士、端明殿學士。是時,晉高祖鎮河東,有二志,廢帝患之,琦與李崧俱備顧問,多所裨畫。琦言:「太原之患,必引契丹為助,不如先事制之。」自明宗時王都反定州,契丹遣禿餒、荝剌等助都,而為趙德鈞、王晏球所敗,禿餒見殺,荝剌等皆送京師。其後契丹數遣使者求荝剌等,其辭甚卑恭,明宗輒斬其使者不報。而東丹王又亡入中國,契丹由此數欲求和。琦
 因言:「方今之勢,不如與契丹通和,如漢故事,歲給金帛,妻之以女,使彊籓大鎮顧外無所引援,可弭其亂心。」崧以琦語語三司使張延朗,延朗欣然曰:「茍能紓國患,歲費縣官十數萬緡,責吾取足可也!」因共建其事。廢帝大喜,佗日以琦等語問樞密直學士薛文遇,文遇大以為非,因誦戎昱「社稷依明主,安危託婦人」之詩,以誚琦等。廢帝大怒,急召崧、琦等問和戎計如何。琦等察帝色怒,亟曰:「臣等為國計,非與契丹求利於中國也。」帝即發怒曰:「卿等佐朕欲致太平而若是邪?朕一女尚幼,欲棄之夷狄,金帛所以養士而扞國也,又輸以資虜,可乎?」



 崧等
 惶恐拜謝,拜無數,琦足力乏不能拜而先止。帝曰:「呂琦彊項,肯以人主視我邪!」琦曰:「臣素病羸,拜多而乏,容臣少息。」頃之喘定,奏曰:「陛下以臣等言非,罪之可也,雖拜何益?」帝意稍解,曰:「勿拜。」賜酒一卮而遣之,其議遂寢。因遷琦御史中丞,居數月,復為端明殿學士。其後晉高祖起太原,果引契丹為助,遂以亡唐。琦事晉為秘書監,累遷兵部侍郎。天福八年卒。



 趙玉仕至職方員外郎,琦事之如父,玉疾,親嘗藥扶侍,及卒,為其家主辦喪葬。玉子文度幼孤,琦教以學,如己子,後舉進士及第云。琦有子餘慶、端。



 薛融薛融,汾州平遙人也。少以儒學知名,唐明宗時為右補闕,直弘文館。晉高祖鎮太原,融為觀察判官。高祖徙鄆,欲據太原拒命,延見賓佐,問以可否,而坐中或贊成之,或恐懼不敢言,融獨從容對曰:「融本儒生爾,軍旅之事,未嘗學也,進退存亡之理,豈易言哉!」高祖不之責也。高祖入立,拜吏部郎中,兼侍御史知雜事。累拜左諫議大夫,遷中書舍人。融曰:「文辭非臣所長也。」遂辭不拜。時詔修洛陽大內,融上疏切諫,高祖褒納其言,即詔罷其役。遷御史中丞,改尚書右丞,分司西京。卒,年六十。



 何澤何澤,廣州人也。父鼎,唐末為容管經略使。澤少好學,長
 於歌詩。舉進士,為洛陽令。唐莊宗好畋獵,數踐民田,澤乃潛身伏草間伺莊宗,當馬諫曰:「陛下未能一天下以休兵,而暴斂疲民以給軍食。今田將熟,奈何恣畋游以害多稼?使民何以出租賊,吏以何督民耕?陛下不聽臣言,願賜臣死於馬前,使後世知陛下之過。」



 莊宗大笑,為之止獵。拜倉部郎中。明宗時,數上書言事。明宗幸汴州,又欲幸鄴,而人情不便,大臣屢言不聽;澤伏閤切諫,明宗嘉之,拜吏部郎中、史館修撰。澤外雖直言,而內實邪佞,嘗於內殿起居,班退,獨留,以笏叩顙,北望而呼曰:「明主,明主!」聞者皆哂之。



 五代之際,民苦於兵,往往因親疾
 以割股,或既喪而割乳廬墓,以規免州縣賦役。戶部歲給蠲符,不可勝數,而課州縣出紙,號為「蠲紙」。澤上書言其敝,明宗下詔悉廢戶部蠲紙。



 澤與宰相趙鳳有舊,數私於鳳,求為給諫。鳳薄其為人,以為太常少卿。敕未出而澤先知之,即稱新官上章自訴。章下中書,鳳等言:「澤未拜命而稱新官,輕侮朝廷,請坐以法。」乃以太僕少卿致仕,居于河陽。澤時年已七十,尚希仕進,即遣婢宜子詣匭上章言事,請立秦王為皇太子。秦王素驕,多不軌,遂成其禍,由澤而始。晉高祖入立,召為太常少卿,以疾卒于家。



 王權王權,字秀山,太原人也。唐左僕射起之曾孫。父蕘,官至右司郎中。權舉進士,為右補闕。唐亡,事梁為職方員外郎、知制誥、翰林學士,累遷御史中丞。唐莊宗滅梁,貶權隨州司馬。起為右庶子,累遷戶部尚書。晉高祖時為兵部尚書。是時,高祖以父事契丹,權當奉使,歎曰:「我雖不才,安能稽顙於穹廬乎?」因辭不行,坐是停任。踰年以太子少傅致仕。卒,年七十入,贈左僕射。



 史圭史圭,常山石邑人也。為人明敏好學。為寧晉、樂壽縣令,有善政,縣人立碑以頌之。郭崇韜鎮成德,辟為從事。明宗時,為尚書郎。安重誨為樞密使,薦圭直學士。故事,直
 學士職雖清,而承領文書,參掌庶務,與判官無異。重誨素不知書,倚圭以備顧問,始白許圭升殿侍立。樞密直學士升殿自圭始。改尚書右丞,判吏部銓事。重誨敗死,圭出為貝州刺史。罷歸常山,閉門絕人事,出入閭里乘輜軿車。



 晉高祖立,召拜刑部侍郎、鹽鐵副使,遷吏部侍郎,分知銓事,有能名。以疾罷,卒于常山。



 龍敏龍敏,字欲訥,幽州永清人也。少仕州,攝參軍。劉守光亂,敏避之滄州,遂客於梁,久不調。敏素善馮道,道為唐莊宗從事,乃潛往依之。監軍張承業謂道曰:「聞子有客,可與俱來。」道以敏見承業,承業辟敏監軍巡官,使掌奏記。
 莊宗即位,召拜司門員外郎。敏父咸式,年七十餘,而其祖父年九十餘,皆在鄴,敏乃求為興唐尹,事祖、父以孝聞。丁母憂,去職。趙在禮反,逼敏起視事。明宗即位,在禮鎮滄州,敏乃復得居喪。服除,累拜兵部侍郎。馮贇留守北京,辟敏副留守。



 贇入為樞密使,敏拜吏部侍郎。是時,晉高祖起太原,乞兵契丹。唐廢帝在懷州,趙德鈞父子有異志,張敬達屯于晉安,勢甚危急。廢帝問計從臣,敏曰:「晉所恃者契丹也。東丹王失國之君,今在京師,若以兵送東丹自幽州而入西樓,契丹且有內顧之憂,何暇助晉?晉失契丹,大事去矣。」又謂李懿曰:「敏,燕人也,能知
 德鈞。德鈞為將,守城嬰塹,篤勵健兒而已。使其當大敵,奮不顧身,非其能也。



 況有異志乎?今聞駕前之馬,猶有五千,願得壯者千匹,健兵千人,與勇將郎萬金,自平遙沿山冒虜中而趨官砦,且戰且行,得其半達,則事濟矣!」懿為言之廢帝,廢帝莫能用。然人皆壯其大言。歷晉為太常卿,使于吳越。是時,使吳越者,見吳越王皆下拜,敏獨揖之。還,遷工部侍郎。乾祐元年,瘍發於首卒,贈右僕射。



\end{pinyinscope}