\article{卷十九周家人傳第七}

\begin{pinyinscope}

 聖穆皇后柴氏
 太祖一后三妃。聖穆皇后柴氏,邢州堯山人也,與太祖同里,遂以歸焉。太祖微時,喜飲博任俠,不拘細行,后常諫止之。太祖狀貌奇偉,后心知其貴人也,事之甚謹。及太祖即位,后已先卒,乃下詔:「故夫人柴氏,追冊為皇后,謚曰聖穆。」



 淑妃楊氏淑妃楊氏,鎮州真定人也。父弘裕,真定少尹。妃幼以色選入趙王宮,事王熔。



 熔為張文禮所殺,鎮州亂,妃亦流
 寓民間,後嫁里人石光輔。居數年,光輔死。太祖柴夫人卒,聞妃有色而賢,遂娶之為繼室。太祖方事漢高祖於太原,天福中妃卒,遂葬太原之近郊。太祖即位,廣順元年九月,追冊為淑妃。拜妃弟廷璋為右飛龍使,廷璋辭曰:「臣父老矣,願以授之。」太祖曰:「吾方思之,豈忘爾父邪!」即召弘裕,弘裕老不能行,乃就其家拜金紫光祿大夫、真定少尹。太祖崩,葬嵩陵,一后三妃皆當陪葬,而太原未克,世宗詔有司營嵩陵之側為虛墓以俟。顯德元年,世宗已敗劉旻於高平,遂攻太原,太原閉壁被圍,乃遷妃喪而葬之。



 貴妃張氏貴妃張氏,鎮州真定人也。祖記,成德軍節度判官、檢校兵部尚書。父同芝,事趙王王熔為諮呈官,官至檢校工部尚書。熔死,鎮州亂,莊宗遣幽州符存審以兵討張文禮,裨將武從諫館於妃家,見妃尚幼,憐之,而從諫家在太原,遂以妃歸,為其子婦。久之,太祖事漢高祖於太原,楊夫人卒,而武氏子亦卒,乃納妃為繼室。



 太祖貴,累封吳國夫人。太祖以兵入京師,漢遣劉銖戮其家,妃與諸子皆死。太祖即位,追冊為貴妃。



 德妃董氏太祖子侗信侄守願奉超遜德妃董氏,鎮州靈壽人也。祖文廣,唐深州錄事參軍。父光嗣,趙州昭慶尉。



 妃幼穎悟,始能言,聞樂聲知其律呂。
 年七歲,鎮州亂,其家失之,為潞州牙將所得,置諸褚中以歸。潞將妻嘗生女,輒不育,得妃憐之,養以為子,過於所生。居五六年,妃家悲思,其兄瑀求之人間,莫知所在。潞將仕於京師,遇瑀,欣然歸之,時年十三。瑀以嫁里人劉進超,進超亦仕晉為內職。契丹犯闕,進超歿于虜中,妃嫠居洛陽。漢高祖由太原入京師,太祖從,過洛陽,聞妃有賢行,聘之。太祖建國,中宮虛位,遂冊為德妃。廣順三年卒,年三十九。



 妃兄三人:瑀官至太子右贊善大夫,玄之、自明皆至刺史。



 初,帝舉兵于魏,漢以兵圍帝第,時張貴妃與諸子青哥、意哥,姪守筠、奉超、定哥,皆被誅。青
 哥、意哥,不知其母誰氏。太祖即位,詔故第二子青哥贈太尉,賜名侗;第三子意哥贈司空,賜名信;皇姪守筠贈左領軍衛將軍,以筠聲近榮,為世宗避,更名守愿;奉超贈左監門衛將軍;定哥贈左千年衛將軍,賜名遜。世宗顯德四年夏四月癸未,詔曰:「禮以緣情,恩以悼往,矧在友于之列,尤鐘惻愴之情。



 故皇弟贈太保侗、贈司空信,景邊初啟,大年不登,俾予終鮮,實勤予懷。侗可贈太傅,追封郯王;信司徒,杞王。」又詔曰:「故皇從弟贈左領軍衛將軍守愿、贈左監門衛將軍奉超、贈左千牛衛將軍遜等,頃因季世,不享遐齡,每念非辜,難忘有慟。守愿可贈
 左衛大將軍,奉超右衛大將軍,遜右武衛大將軍。」



\end{pinyinscope}