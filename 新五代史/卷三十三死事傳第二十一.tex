\article{卷三十三死事傳第二十一}

\begin{pinyinscope}

 嗚
 呼甚哉!自開平訖于顯德,終始五十三年,而天下五代。士之不幸而生其時,欲全其節而不二者,固鮮矣。於此之時,責士以死與必去,則天下為無士矣。然其習俗,遂以茍生不去為當然。至於儒者,以仁義忠信為學,享人之祿,任人之國者,不顧其存亡,皆恬然以茍生為得,非徒不知愧,而反以其得為榮者,可勝數哉!故吾於死事之臣,有所取焉。君子之於人也,樂成其美而不求其
 備,況死者人之所難乎?吾於五代,得全節之士三人而已。其初無卓然之節,而終以死人之事者,得十有五人焉,而戰沒者不得與也。然吾取王清、史彥超者,其有旨哉!其有旨哉!作《死事傳》。



 張源德張源德者,不知其世家,或曰本晉人也。少事晉,無所稱。從李罕之以潞州叛晉降梁,罕之遣源德見梁太祖。太祖時,源德自金吾衛將軍為蔡州刺史。梁貞明三年,魏博節度使楊師厚卒,末帝分魏、相等六州為兩鎮,懼魏軍不從,乃遣劉鄩將兵萬人,屯於魏以虞變。魏軍果叛,
 迫其節度使賀德倫以魏、博二州降晉。當是時,源德為鄩守貝州。晉王入魏,諸將欲先擊貝州,晉王曰:「貝城小而堅,攻之難卒下。且源德雖恃劉鄩之兵,然與滄州相首尾,今德州居其中而無備,不如先取之,則滄、貝之勢分而易圖也。」乃先襲破德州,然後以兵五千攻源德,源德堅守不下,晉軍塹而圍之。



 已而劉鄩大敗于故元城,南走黎陽,晉軍攻破洺州,而衛州刺史來昭、邢州節度使閻寶皆以城降晉,磁州刺史靳昭、相州張筠、滄州戴思遠皆棄城走。當此時,晉已先下全燕,而鎮、定皆附于晉,自河以北、山以東,四面千里,六鎮數十州之地皆歸
 晉,獨貝一州,圍之踰年不可下。源德守既堅,而貝人聞晉已盡有河北,城中食且盡,乃勸源德出降,源德不從,遂見殺。



 源德已死,貝人謀曰:「晉圍吾久,吾窮而後降,懼皆不免也。」乃告于晉曰:「吾欲被甲執兵而降,得赦而後釋之,如何?」晉軍許諾,貝人三千出降,已釋甲,晉兵四面圍而盡殺之。



 夏魯奇夏魯奇,字邦傑,青州人也。唐莊宗時,賜姓名曰李紹奇,其後莊宗賜姓名者,皆復其故。魯奇初事梁為宣武軍校,後奔于晉,為衛護指揮使。從周德威攻劉守光於幽州,守光將單廷珪、元行欽以驍勇自負,魯奇每與二將
 鬥,輒不能解,兩軍皆釋兵而觀之。晉已下魏博,梁將劉鄩軍于洹水,莊宗以百騎覘敵,遇尋阜伏兵,圍之數重,幾不得脫,魯奇力戰,手殺百餘人,身被二十餘創,與莊宗決圍而出。莊宗益奇之,以為磁州刺史。從戰中都,擒王彥章,莊宗壯之,賜絹千疋,拜鄭州防禦使。遷河陽節度使,為政有惠愛。徙鎮忠武,河陽之人遮留不得行,父老詣京師乞留,明宗遣中使往諭之,魯奇乃得去。唐師伐荊南,以魯奇為招討副使,無功而還。徙鎮武信,東川董璋反,攻遂州,魯奇閉城拒之,旬月救兵不至,城中食盡,魯奇自刎死,年四十九。



 姚洪姚洪,本梁之小校也。自董璋為梁將,洪嘗事璋,後事唐為指揮使。長興中,遣洪將千人戍閬州。董璋反,遣人以書招洪,洪得璋書,輒投廁中。後璋兵攻破閬州,執洪,璋曰:「爾為健兒,我遇汝厚,奈何負我邪?」洪罵曰:「老賊!爾昔為李七郎奴,掃馬糞,得一臠殘炙,感恩不已。今天子用爾為節度使,何苦反邪?



 吾能為國家死,不能從人奴以生!」璋怒,然鑊于前,令壯士十人刲其肉而食,洪至死大罵。明宗聞之泣下,錄其二子,而厚恤其家。



 王思同王思同,幽州人也。其父敬柔,娶劉仁恭女,生思同。思同事仁恭為銀胡錄指揮使,仁恭為其子守光所囚,思同
 奔晉,以為飛勝指揮使。梁、晉相距于莘,遣思同築壘楊劉,以功遷神武十軍都指揮使,累遷鄭州防禦使。思同為人敢勇,善騎射,好學,頗喜為詩,輕財重義,多禮文士,然未嘗有戰功。



 明宗時,以久次為匡國軍節度使,徙鎮雄武。是時,吐蕃數為寇,而秦州無亭障,思同列四十餘柵以禦之。居五年,來朝,明宗問以邊事,思同指畫山川,陳其利害。思同去,明宗顧左右曰:「人言思同不管事,能若是邪?」於是始知其材,以為右武衛上將軍、京兆尹、西京留守。石敬瑭討董璋,思同為先鋒指揮使,兵入劍門,而後軍不繼,思同與璋戰,不勝而卻。敬瑭兵罷,思同徙
 鎮山南西道,已而復為京兆尹、西京留守。



 應順元年二月,潞王從珂反鳳翔,馳檄四鄰,言姦臣幸先帝疾病,賊殺秦王而立幼嗣,侵弱宗室,動搖籓方,陳己所以興兵討亂之狀。因遣伶奴安十十以五絃謁思同,欲因其懽以通意。是時,諸鎮皆懷嚮背,所得潞王書檄,雖以上聞,而不絕其使。獨思同執十十及從珂所使推官郝詡等送京師。愍帝嘉其忠,即以思同為西面行營馬步軍都部署。三月,會諸鎮兵圍鳳翔,破東西關城。從珂兵弱而守甚堅,外兵傷死者眾,從珂登城呼外兵而泣曰:「吾從先帝二十年,大小數百戰,甲不解體,金瘡滿身,士卒固
 嘗從我矣。今先帝新棄天下,而朝廷信用姦人,離間骨肉,我實何罪而見伐乎?」因慟哭。士卒聞者,皆悲憐之。興元張虔釗攻城西,督戰甚急,士卒苦之,反兵攻虔釗,虔釗走。羽林指揮使楊思權呼曰:「潞王,吾主也!」乃引軍自西門入降從珂。而思同硃知,猶督戰。嚴衛指揮使尹暉麾其眾曰:「城西軍入城受賞矣!何用戰邪?」士卒解甲棄仗,聲聞數里,遂皆入城降。諸鎮之兵皆潰。



 思同挺身走,至長安,西京副留守劉遂雍閉門不納,乃走潼關。從珂引兵東,至昭應,前鋒追執思同。從珂責曰:「罪可逃乎?」思同曰:「非不知從王而得生,恐終死不能見先帝於地下。」
 從珂愧其言,乃殺之。漢高祖即位,贈侍中。



 張敬達張敬達,字志通,代州人也,小字生鐵。少以騎射事唐莊宗為直軍使。明宗時,為河東馬步軍都指揮使,領欽州刺史,累遷彰國、大同軍節度使,徙鎮武信、晉昌。清泰二年,契丹數犯邊,廢帝以河東節度使石敬瑭兼大同、彰國、振武、威塞等軍蕃漢馬步軍都總管,屯於忻州,屯兵聚噪遮敬瑭呼「萬歲」,敬瑭斬三十餘人以止之。廢帝疑敬瑭有異志,乃以敬達為北面副總管,以分其兵。明年夏,徙敬瑭鎮天平,遂以敬達為大同、彰國、振武、威塞
 等軍蕃漢馬步軍都部署,敬瑭因此遂反。即以敬達為太原四面招討使。六月,兵圍太原,敬達為長城連柵,雲梯飛炮以攻之,所為城柵將成,輒有大風雨水暴至以壞之。



 敬瑭求救於契丹。九月,契丹耶律德光自鴈門入,旌旗相屬五十餘里。德光先遣人告敬瑭曰:「吾欲今日破敵可乎?」敬瑭報曰:「大兵遠來,而賊勢方盛,要在成功,不必速也。」使者未復命,而兵已交。敬達陣於西山,契丹以羸騎三千,革鞭木登,人馬皆不甲胄,以趨唐軍。唐軍爭馳之,契丹兵走。追至汾曲,伏發,斷唐軍為二,其在北者皆死,死者萬餘人。敬達收軍柵晉安,契丹圍之。廢帝
 遣趙延壽、范延光等救之。延壽屯團柏谷,延光屯遼州,相去皆百餘里。契丹兵圍敬達者,自晉安寨南,長百餘里,闊五十里,敬達軍中望之,但見穹廬連屬如岡阜,四面亙以毛索,掛鈴為警,縱犬往來。敬達軍中有夜出者,輒為契丹所得,由是閉壁不敢復出。延壽等皆有二心,無救敬達意。敬達猶有兵五萬人、馬萬匹,久之食盡,削木篩糞以飼其馬,馬死者食之,已而馬盡。副招討使楊光遠勸敬達降晉,敬達自以不忍背唐,而救兵且至,光遠促之不已,敬達曰:「諸公何相迫邪!何不殺我而降?」光遠即斬敬達降。契丹耶律德光聞敬達死,哀其忠,遣人
 收葬之。



 翟進宗張萬迪附翟進宗、張萬迪者,皆不知其何人也。初皆事唐,後事晉,進宗為淄州刺史,萬迪為登州刺史。楊光遠反,以騎兵數百脅取二刺史至青州,萬迪聽命,而進宗獨不屈,光遠遂殺進宗。出帝贈進宗左武衛上將軍。及光遠平,曲赦青州,雖光遠子孫皆見慰釋,而獨不赦萬迪,暴其罪而斬之。詔求進宗尸,加禮歸葬,葬事官給,以其子仁欽為東頭供奉官。



 沈斌沈斌,字安時,徐州下邳人也。少為軍卒,事梁為拱辰都
 指揮使。後事唐,從魏王繼岌破蜀,平康延孝,以功為虢州刺史,歷隨、趙等八州刺史。晉開運元年,為祁州刺史。契丹犯塞至於榆林,過祁州,斌以謂契丹深入晉地而歸兵羸乏可擊,即以州兵邀之。契丹以精騎劃門,斌兵多死,城中無備,虜將趙延壽留兵急攻之,延壽招斌降,斌從城上罵延壽曰:「公父子誤計,陷于腥膻,忍以犬羊之眾,殘賊父母之邦,斌能為國死爾,不能效公所為也!」已而城陷,斌自盡,其家屬皆沒于虜。



 王清王清,字去瑕,洺州曲周人也。初事唐為寧衛指揮使。後事晉為奉國都虞候。



 安從進叛襄州,從高行周攻之,逾
 年不能下,清謂行周曰:「從進閉孤城以自守,其勢豈得久邪?」因請先登,遂攻破之。開運二年冬,從杜重威戰陽城,清以力戰功為步軍之最,加檢校司徒。是冬,重威軍中渡橋南,虜軍其北以相拒,而虜以精騎並西山出晉軍後,南擊欒城,斷晉餉道。清謂重威曰:「晉軍危矣!今去鎮州五里,而守死于此,營孤食盡,將若之何?請以步兵二千為先鋒,奪橋開路,公率諸軍繼進以入鎮州,可以守也。」重威許之,遣與宋彥筠俱前,清與虜戰,敗之,奪其橋。是時,重威已有二志,猶豫不肯進,彥筠亦退走,清曰:「吾獨死於此矣!」



 因力戰而死。年五十三。漢高祖立,贈清
 太傅。



 史彥超史彥超,雲州人也。為人勇悍驍捷。周太祖起魏時,彥超為漢龍捷都指揮使,以兵從。太祖入立,遷虎捷都指揮使,戍于晉州。劉旻攻晉州,州無主帥,知州王萬敢不能拒,彥超以戍兵堅守月餘,太祖遣王峻救之,旻兵解去。以功遷龍捷右廂都指揮使,領鄭州防禦使。周、漢戰高平,彥超為前鋒,先登陷陣,以功拜感德軍節度使。周兵圍漢太原,契丹救漢,出忻、代。世宗遣符彥卿拒之,以彥超為先鋒,戰忻口,彥超勇憤俱發,左右馳擊,解而復合者數四,遂歿于陣。



 是時,世宗敗漢高平,乘勝而進,圍城
 之役,諸將議不一,故久無成功。世宗欲解去而未決,聞彥超戰死,遽班師,倉卒之際,亡失甚眾。世宗既惜彥超而憤無成功,憂忿不食者數日。贈彥超太師,優恤其家焉。



 孫晟孫晟,初名鳳,又名忌,密州人也。好學,有文辭,尤長於詩。少為道士,居廬山簡寂宮。常畫唐詩人賈島像置於屋壁,晨夕事之。簡寂宮道士惡晟,以為妖,以杖驅出之。乃儒服北之趙、魏,謁唐莊宗于鎮州,莊宗以晟為著作佐郎。天成中,朱守殷鎮汴州,辟為判官。守殷反,伏誅,晟乃棄其妻子,亡命陳、宋之間。安重誨惡晟,以謂教守殷反
 者晟也,畫其像購之,不可得,遂族其家。



 晟奔于吳。是時,李昪方篡楊氏,多招四方之士,得晟,喜其文辭,使為教令,由是知名。晟為人口吃,遇人不能道寒暄,已而坐定,談辯鋒生,聽者忘倦。昪尤愛之,引與計議,多合意,以為右僕射,與馮延巳並為昪相。晟輕延巳為人,常曰:「金碗玉盃而盛狗屎可乎?」晟事昪父子二十餘年,官至司空,家益富驕,每食不設几案,使眾妓各執一器,環立而侍,號「肉臺盤」,時人多效之。



 周世宗征淮,李景懼,始遣泗州牙將王知朗至徐州,奉書以求和,世宗不答。



 又遣翰林學士鐘謨、文理院學士李德明奉表稱臣,不答。乃遣禮
 部尚書王崇質副晟奉表,謨與晟等皆言景願割壽、濠、泗、楚、光、海六州之地,歲貢百萬以佐軍。



 而世宗已取滁、揚、濠、泗諸州,欲盡取淮南乃止,因留使者不遣,而攻壽州益急。



 謨等見世宗英武非景敵,而師甚盛,壽春且危,乃曰:「願陛下寬臣五日之誅,容臣還取景表,盡獻淮北諸州。」世宗許之,遣供奉官安弘道押德明、崇質南還,而謨與晟皆見留。德明等既還,景悔,不肯割地。世宗亦以暑雨班師,留李重進、張永德等分攻廬、壽,周兵所得揚、泰諸州,皆不能守,景兵復振。重進與永德兩軍相疑,有隙,永德上書言重進反,世宗不聽。景知二將之相疑也,
 乃以蠟丸書遺重進,勸其反。



 初,晟之奉使也,語崇質曰:「吾行必不免,然吾終不負永陵一抷土也。」永陵者,昪墓也。及崇質還,而晟與鐘謨俱至京師,館于都亭驛,待之甚厚,每朝會入閣,使班東省官後,召見必飲以醇酒。已而周兵數敗,盡失所得諸州,世宗憂之,召晟問江南事,晟不對,世宗怒,未有以發。會重進以景蠟丸書來上,多斥周過惡以為言,由是發怒曰:「晟來使我,言景畏吾神武,願得北面稱臣,保無二心,安得此指斥之言乎?」亟召待衛軍虞候韓通收晟下獄,及其從者二百餘人皆殺之。晟臨死,世宗猶遣近臣問之,晟終不對,神色怡然,正
 其衣冠南望而拜曰:「臣惟以死報國爾!」乃就刑。晟既死,鐘謨亦貶耀州司馬。其後,世宗怒解,憐晟忠,悔殺之,召拜鐘謨衛尉少卿。景已割江北,遂遣謨還,而景聞晟死,亦贈魯國公。



\end{pinyinscope}