\article{卷六十職方考第三}

\begin{pinyinscope}

 嗚
 呼,自三代以上,莫不分土而治也。後世鑒古矯失,始郡縣天下。而自秦、漢以來,為國孰與三代長短?及其亡也,未始不分,至或無地以自存焉。蓋得其要,則雖萬國而治,失其所守,則雖一天下不能以容,豈非一本於道德哉!唐之盛時,雖名天下為十道,而其勢未分。既其衰也,置軍節度,號為方鎮,鎮之大者連州十餘,小者猶兼三四,故其兵驕則逐帥,帥彊則叛上,土地為其世有,干
 戈起而相侵,天下之勢,自茲而分。然唐自中世多故矣,其興衰救難,常倚鎮兵扶持,而侵凌亂亡,亦終以此。豈其利害之理然歟?自僖、昭以來,日益割裂。梁初,天下別為十一國,南有吳、浙、荊、湖、閩、漢,西有岐、蜀,北有燕、晉,而朱氏所有七十八州以為梁。莊宗初起并、代,取幽、滄,有州三十五,其後又取梁魏、博等十有六州,合五十一州以滅梁。岐王稱臣,又得其州七。同光破蜀,已而復失,惟得秦、鳳、階、成四州,而營、平二州陷于契丹,其增置之州一,合一百二十三州以為唐。



 石氏入立,獻十有六州于契丹,而得蜀金州,又增置之州一,合百九州以為晉。劉
 氏之初,秦、鳳、階、成復入於蜀,隱帝時增置之州一,合一百六州以為漢。郭氏代漢,十州入于劉旻,世宗取秦、鳳、階、成、瀛、莫及淮南十四州,又增置之州五而廢者三,合一百一十八州以為周。宋興因之。此中國之大略也。其餘外屬者,強弱相並,不常其得失。至於周末,閩已先亡,而在者七國。自江以南二十一州為南唐,自劍以南及山南西道四十六州為蜀,自湖南北十州為楚,自浙東西十三州為吳越,自嶺南北四十七州為南漢,自太原以北十州為東漢,而荊、歸、峽三州為南平。合中國所有,二百六十八州,而軍不在焉。唐之封疆遠矣,前史備載,
 而羈縻寄治虛名之州在其間。五代亂世,文字不完,而時有廢省,又或陷於夷狄,不可考究其詳。其可見者,具之如譜。



 以下表略汴州,唐故曰宣武軍。梁以汴州為開封府,建為東都。後唐滅梁,復為宣武軍。



 晉天福三年升為東京。漢、周因之。



 洛陽,梁、唐、晉、漢、周常以為都。唐故為東都。梁為西都。後唐為洛京。



 晉為西京,漢、周因之。



 雍州,唐故上都,昭宗遷洛,廢為佑國軍。梁初改京兆府曰大安,佑國軍曰永平。唐滅梁,復為西京。晉廢為晉昌軍。漢改曰永興,周因之。



 曹州,故屬宣
 武軍節度。晉開運二年置威信軍。漢初,軍廢。周廣順二年復置彰信軍。



 宋州,故屬宣武軍節度。梁初徙置宣武軍。唐滅梁,改曰歸德。



 陳州,故屬忠武軍節度。晉開運二年置鎮安軍。漢初,軍廢。周廣順二年復之。



 許州,唐故曰忠武。梁改曰匡國。唐滅梁,復曰忠武。



 滑州,唐故曰義成。以避
 梁王父諱改曰宣義。唐滅梁,復其故。



 襄州,唐故曰山南東道。唐、梁之際改曰忠義軍。後以延州為忠義,襄州復曰山南東道。



 鄧州,故屬山南東道節度。梁破趙匡凝,分鄧州置宣化軍。唐改曰威勝。周改曰武勝。



 安州,梁置宣威軍。唐改曰安遠,晉罷,漢復曰安遠,周又罷。



 晉州,故屬護國軍節度。梁開平四年置定昌軍,貞明三年改曰建寧。唐改曰建雄。



 金州,故屬山南東道節度。唐末置戎昭軍,已而廢之,遂入於蜀。至晉高祖時,又置懷德軍,尋罷。



 陜州,唐故曰保義,梁改曰鎮國,後唐復曰
 保義。



 華州,唐故曰鎮國,梁改曰感化,後唐復曰鎮國。



 同州,唐故曰匡國,梁改曰忠武,後唐復曰匡國。



 耀州,本華原縣,唐末屬李茂貞,建為耀州,置義勝軍。梁末帝時,茂貞養子溫韜以州降梁,梁改耀州為崇州,義勝曰靜勝。後唐復為耀州,改曰順義。



 延州,故屬保大軍節度。梁置忠
 義軍,唐改曰彰武。



 魏州,唐故曰大名府,
 置天雄軍,五代皆因之。後唐建鄴都,晉、漢因之,至周罷。大名府,後唐曰興唐,晉曰廣晉,漢、周復曰大名。



 澶州,故屬天雄軍節度。晉天福九年置鎮寧軍。



 相州,故屬天雄軍節度。梁末帝分置昭德軍,而天雄軍亂,遂入于晉。莊宗滅梁,復屬天雄。晉高祖置彰德軍。



 邢州,故屬昭義軍節度。昭義所統澤、潞、邢、洺、磁五州。唐末孟方立為昭義軍節度使,徙其軍額于邢州,而澤、潞二州入于晉。方立但有邢、洺、磁三州。



 故當唐末有兩昭義軍。梁、晉之爭,或入于梁,或入于晉。梁以邢、洺、磁三州為保義軍。莊宗滅梁,改曰安國。



 鎮州,故曰成德軍。梁初以成音犯廟諱,改曰武順。唐復曰成德,晉又改曰順德,漢復曰成德。



 應州,故屬大同軍節度。唐明宗即位,以其應州人也,乃置彰國軍。



 新州,唐同光元年置威塞軍。



 府州,晉置永安軍,漢罷之,周復。



 并州,後唐建北都,其軍仍曰河東。



 潞州,唐故曰昭義。梁末帝時屬梁,改曰匡義,歲餘,唐滅梁,改曰安義。晉復曰昭義。



 廬州,周世宗克淮南,置保信軍。



 壽州,唐故曰忠正,南唐改曰清淮。周世宗平淮南,復曰忠正。



 五代之際,外屬之州,揚州曰淮南,宣州曰寧國,鄂州曰武昌,洪州曰鎮南,福州曰武威,杭州曰鎮海,越州曰鎮東,江陵府曰荊南,益州、梓州曰劍南東、西川,遂州曰武
 信,興元府曰山南西道,洋州曰武定,黔州曰黔南,潭州曰武安,桂州曰靜江,容州曰寧遠,邕州曰建武,廣州曰清海,皆唐故號,更五代無所易,而今因之者也。其餘僭偽改置之名,不可悉考,而不足道,其因著于今者,略注于譜。



 濟州,周廣順二年置,割鄆州之鉅野、鄆城,兗州之任城,單州之金鄉為屬縣而治鉅野。



 單州,唐末以宋州之碭山,梁太祖鄉里也,為置輝州,已而徙治單父。後唐滅梁,改輝州為單州。其屬縣置徙,傳記不同,今領單父、碭山、成武、魚臺四縣。



 耀州,李茂貞置,治華原縣。梁初改曰崇州,唐同光元年復為耀州。



 解州,漢乾祐元年九月置,割河中之聞喜、安邑、解縣為屬而治解。



 威州,晉天福四年置,割靈州之方渠,寧州之木波、馬嶺三鎮為屬而治方渠。



 周廣順二年改曰環州,顯德四年廢為通遠軍。



 五代置軍六,皆寄治於縣,隸於州,故不別出。監者,物務之名爾,故不載於地理。皇朝軍監始自置屬縣,與州府並列矣。



 乾州,李茂貞置,治奉天縣。



 磁州,梁改曰惠州,唐復曰磁州。



 景州,唐故治弓高。周顯德三年廢為定遠軍,割其屬安陵縣屬德州,廢弓高縣入東光縣,為定遠軍治所。



 濱州,周顯德三年置,以其濱海為名。初,五代之際,置榷鹽務於海傍,後為贍國軍,周因置州,割棣州之渤海、蒲臺為屬縣而治渤海。



 雄州,周顯德六年克瓦橋關置,治歸義;割易州之容城為屬,尋廢。



 霸州,周顯德六年克益津關置,治永清,割莫州之文安,瀛州之大城為屬。



 通州,本海陵之東境,南唐置靜海制置院,周世宗克淮
 南,升為靜海軍,後置通州,分其地置靜海、海門二縣為屬而治靜海。



 筠州,南唐李景置,割洪州之高安、上高、萬載、清江四縣為屬而治高安。



 劍州,南唐李景置,割建州之延平、劍浦、富沙三縣為屬而治延平。



 全州,楚王馬希範置,以潭州之湘川縣為清湘縣,又割灌陽縣為屬而治清湘。



 秀州,吳越王錢元瓘置,割杭州之嘉興縣為屬而治之。



 雄州,南漢劉割韶州之保昌置,治保昌。



 英州,南漢劉割廣州之湞陽置,治湞陽。



 開封府故統六縣。梁開平元年,割滑州之酸棗、長垣,鄭州之中牟、陽武,宋州之襄邑,曹州之考城更曰戴邑,許州之扶溝、焉陵,陳州之太康隸焉。唐分酸棗、中牟、襄邑、焉陵、太康五縣還其故,晉升汴州為東京,復割五縣隸焉。



 雍丘,晉改曰杞,漢復其故。



 長垣,唐改曰匡城。



 黎陽,故屬滑州,晉割隸衛州。



 葉、襄城,故屬許州,唐割隸汝州。



 楚丘,故屬單州,梁割隸宋州。



 密州膠西,故曰輔唐,梁改曰安丘,唐復其故,晉改曰膠西。



 渭南,故屬京兆,周改隸華州。



 同官,故屬京兆府,梁割隸同州,唐割隸耀州。



 美原,故屬同州,李茂貞置鼎州而治之。梁改為裕州,屬順義軍節度。後不見其廢時,唐同光三年,割隸耀州。



 平涼,故屬涇州。唐末渭州陷吐蕃,權於平涼置渭州而縣廢。後唐清泰三年,以故平涼之安國、耀武兩鎮置平涼縣,屬涇州。



 臨涇,故屬涇州。唐末原州陷吐蕃,權於臨涇置原州而
 涇州兼治其民。後唐清泰三年割隸原州。



 鄜州咸寧,周廢。



 稷山,故屬河中,唐割隸絳州。



 慈州仵城、呂香,周廢。



 大名府大名,故曰貴鄉。後唐改曰廣晉,漢改曰大名。



 滄州長蘆、乾符,周廢入清池;無棣,周置保順軍。



 安陵,故屬景州,周割隸德州。



 澶州頓丘,晉置德清軍。



 博州武水,周廢入聊城。



 博野,故屬深州,周割隸定州。



 武康,故屬湖州,梁割隸杭州。



 福州閩清,梁乾化元年,王審知於梅溪場置。



 蘇州吳江,梁開平三年,錢鏐置。



 明州望海,梁開平三年,錢鏐置。



 處州長松,故曰松陽,梁改曰長松。



 潭州龍喜,漢乾祐三年,馬希範置。



 天長、六合,故屬揚州。南唐以天長為軍,六合為雄州,周復故。



 漢陽,故屬鄂州,周置漢陽軍。



 水義川,故屬沔州,周割隸安州。



 襄州樂鄉,周廢入宜城。



 鄧州臨湍,漢改曰臨瀨;菊潭、向城,周廢。



 復州竟陵,晉改曰景陵。



 監利,故屬復州,梁割隸江陵。



 唐州慈丘,周廢。



 商州乾元,漢改曰乾祐,割隸京兆。



 洛南,故屬華州,周割隸商州。



 隨州唐城,梁改曰漢東,後唐復舊,晉又改漢東,漢復舊。



 雄勝軍,本鳳州固鎮,周置軍。



 秦州天水、隴城,唐末廢,後唐復置。



 成州慄亭,後唐置。



 自唐有方鎮,而史官不錄於地理之書,以謂方鎮兵戎之事,非職方所掌故也。



 然而後世因習,以軍目地,而沒其州名。又今置軍者,徒以虛名升建為州府之重,此不可以不書也。州、縣,凡唐故而廢於五代,若五代所置而見於今者,及縣之割隸今因之者,皆宜列以備職方之考。其餘嘗置而復廢,嘗改割而復舊者,皆不足書。



 山川物俗,職方之掌也,五代短世,無所遷變,故亦不復錄,而錄其方鎮軍名,以與前史互見之云。



\end{pinyinscope}