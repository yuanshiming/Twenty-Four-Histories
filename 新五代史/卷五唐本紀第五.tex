\article{卷五唐本紀第五}

\begin{pinyinscope}

 存勖,克用長子也。初,克用破孟方立于邢州,還軍上黨,置酒三垂崗,伶人奏《百年歌》,至于衰老之際,聲甚悲,坐上皆悽愴。時存勖在側,方五歲,克用慨然捋鬚,指而笑曰:「吾行老矣,此奇兒也,後二十年,其能代我戰于此乎!」



 存勖年十一,從克用破王行瑜,遣獻捷于京師,昭宗異其狀貌,賜以鸂鶒卮、翡翠盤,而撫其背曰:「兒有奇表,後當富貴,無忘予家。」及長,善騎射,膽勇過人,稍習《春秋》,
 通大義,尤喜音聲歌舞俳優之戲。



 天祐五年正月,即王位于太原。叔父克寧殺都虞候李存質,倖臣史敬鎔告克寧謀叛。二月,執而戕之,且以先王之喪、叔父之難告周德威,德威自亂柳還軍太原。



 梁夾城兵聞晉有大喪,德威軍且去,因頗懈。王謂諸將曰:「梁人幸我大喪,謂我少而新立,無能為也,宜乘其怠擊之。」乃出兵趨上黨,行至三重崗,歎曰:「此先王置酒處也!」會天大霧晝暝,兵行霧中,攻其夾城,破之,梁軍大敗,凱旋告廟。九月,蜀王王建、岐王李茂貞及楊崇本攻梁大安,晉亦遣周德威攻其晉州,敗梁軍於神山。



 六年,劉知俊叛梁,來乞師,王自
 將至陰地關,遣周德威攻晉州,敗梁軍于蒙阮。七年冬,梁遣王景仁攻趙,趙王王鎔來乞師,諸將皆疑鎔詐,未可出兵,王不聽,乃救趙。八年正月,敗梁軍於柏鄉,斬首二萬級,獲其將校三百人,馬三千匹。



 進攻邢州,不下,留兵圍之,去,攻魏。別遣周德威徇梁夏津、高唐,攻博州,破東武、朝城,遂擊黎陽、臨河、淇門,掠新鄉、共城。



 燕王劉守光聞晉攻梁深入,乃大治兵,聲言助晉,王患之,乃旋師。七月,會趙王王鎔於承天軍。劉守光稱帝于燕。九年正月,遣周德威會鎮、定以攻燕,守光求救於梁,梁軍攻趙,屠棗彊,李存審擊走之。八月,朱友謙以河中叛于梁來
 降,梁遣康懷英討友謙,友謙復臣于梁,而亦陰附于晉。十年十月,劉守光請降,王如幽州,守光背約不降,攻破之。十一年,殺燕王劉守光於太原,用其父仁恭於鴈門。



 於是趙王王鎔、北平王王處直奉冊推王為尚書令,始建行臺。七月,攻梁邢州,戰於張公橋,晉軍大敗。



 十二年,魏州軍亂,賀德倫以魏、博二州叛于梁來附。王入魏州,行至永濟,誅其亂首張彥,以其兵五百自衛,號帳前銀槍軍。六月,王兼領魏博節度使。取德州。七月,取澶州。劉鄩軍于洹水,王率百騎覘其營,遇鄩伏兵圍之數重,決圍而出,亡七八騎。八月,梁復取澶州,晉軍與鄩
 對壘于莘,晉軍數挑戰,鄩閉壁不出。



 十三年正月,王留李存審于莘,聲言西歸。鄩聞晉王且去,即引兵擊魏,攻城東。



 王行至貝州,返擊鄩,大敗之,追至于故元城,又敗之,鄩走黎陽。三月,攻梁衛州,降其刺史米昭;克磁州,殺其刺史靳昭。四月,克洺州。八月,圍邢州,降其節度使閻寶。梁張筠棄相州、戴思遠棄滄州而逃,遂取二州,而貝州人殺梁守將張源德,以城降。



 契丹寇蔚州,執振武節度使李嗣本。十四年,契丹寇新州,遂寇幽州,李嗣源擊走之。冬,梁謝彥章軍于楊劉。十二月,攻楊劉,王自負芻以堙塹,遂破之。十五年正月,梁、晉相距于楊劉,彥章決
 河水以隔晉軍。六月,渡水擊彥章,破其四寨。八月,大閱於魏,合盧龍、橫海、昭義、安國及鎮、定之兵十萬、馬萬匹,軍于麻家渡。謝彥章軍于行臺。十二月,進軍臨濮,梁軍追之,戰于胡柳,晉軍大敗,周德威死之。梁軍暮休于土山,晉軍復擊,大敗之,遂軍德勝,為夾寨。十六年正月,王兼領盧龍軍節度使。梁王瓚攻德勝南城,不克。十月,廣德勝北城。十二月,敗梁軍於河南。十七年,朱友謙襲同州,梁遣劉鄩擊友謙,李存審敗梁軍于同州。



 十八年正月,魏州僧傳真獻唐受命寶一。趙將張文禮弒其君鎔,文禮來請命。



 二月,以文禮為鎮州兵馬留後。三月,河中
 節度使朱友謙、昭義軍節度使李嗣昭、橫海軍節度使李存審、義武軍節度使王處直、安國軍節度使李嗣源、鎮州兵馬留後張文禮、領天平軍節度使閻寶,大同軍節度使李存璋、振武軍節度使李存進、匡國軍節度使朱令德,請王即皇帝位,王三辭,友謙等三請,王曰:「予當思之。」



 八月,遣趙王王鎔故將符習及閻寶、史建瑭等攻張文禮於鎮州。建瑭取趙州。



 張文禮卒,其子處瑾閉城拒守。九月,建瑭戰死。十月,梁戴思遠攻德勝北城,李嗣源敗之於戚城。王處直叛附於契丹,其子都幽處直以來附。十二月,契丹寇涿州,遂寇定州。十九年正月,敗契
 丹于新城、望都,追奔至于涿州。三月,閻寶敗于鎮州,以李嗣昭代之。四月,嗣昭戰死,以李存進代之。八月,梁取衛州。九月,存進敗鎮人于東垣,存進戰死。十月,李存審克鎮州。王兼領成德軍節度使。



 同光元年春三月,李繼韜以潞州叛附于梁。夏四月己巳,皇帝即位,大赦,改元,國號唐。行臺左丞相豆盧革為門下侍郎,右丞相盧程為中書侍郎:同中書門下平章事;中門使郭崇韜、昭義監軍張居翰為樞密使。以魏州為東京,太原為西京,鎮州為北都。閏月,追尊祖考為皇帝,妣為
 皇后:曾祖執宜、祖妣崔氏皆謚曰昭烈,廟號懿祖;祖國昌、祖妣秦氏皆謚曰文景,廟號獻祖;考謚曰武,廟號太祖。立廟于太原,自唐高祖、太宗、懿宗、昭宗為七廟。壬寅,李嗣源取鄆州。五月辛酉梁人取德勝南城。六月,及王彥章戰于新壘,敗之。是月,盧程罷。秋八月,梁人克澤州,守將裴約死之。九月戊辰,李嗣源及王彥章戰于遞坊,敗之。冬十月壬申,如鄆州以襲梁。
 甲戌,取中都。丁丑,取曹州。己卯,滅梁。敬翔自殺。丙戌,貶鄭玨為萊州司戶參軍,蕭頃登州司戶參軍;殺李振、趙巖、張漢傑、朱珪,滅其族。



 己丑,德音降死罪囚,流已下原之。十一月乙巳,復北都為鎮州,太原為北都。丙辰,復汴州為宣武軍。丁巳,尚書左丞趙光胤為中書侍郎,禮部侍郎韋說:同中書門下平章事。戊午,新羅國王金朴英遣使者來。辛酉,復永平軍為西都。甲子,如洛京。十二月庚午朔,至自汴州。辛巳,李繼韜伏誅。繼韜之弟繼達殺其兄繼儔于潞州。
 壬辰,畋于伊闕。



 二年春正月,河南尹張全義及諸鎮進暖殿物。己酉,求唐宦者。庚戌,新羅國王金樸英及其泉州範節度使王逢規皆遣使者來。乙卯,渤海國王大諲譔使大禹謨來。



 庚申,如河陽。辛酉,至自河陽。丁卯,七廟神主至自太原,祔于太廟。朝獻于太微宮。戊辰,享于太廟。二月己巳朔,有事于南郊,大赦。癸酉,各臣上尊號曰昭文睿武光孝皇帝。戊寅,幸李嗣源第。癸未,立劉氏
 為皇后。三月己酉,黨項來。



 庚戌,賜從平汴州及入洛南郊立仗軍士等功臣。庚申,工部郎中李塗為檢視諸陵使。



 潞州將楊立反。夏五月壬寅,教坊使陳俊為景州刺史,內園栽接使儲德源為憲州刺史。丙辰,渤海國王大湮撰遣使者來。丙寅,李嗣源克潞州。六月丙子,楊立伏誅。



 己丑,封回紇王仁美為英義可汗。秋七月己酉,如雷山賽天神。八月,大雨霖,河溢。九月壬子,置水于城門,以禳
 熒惑。甲寅,幸郭崇韜第。丙辰,黑水遣使者來。



 冬十月癸未,左熊威軍將趙暉妻一產三男子。十一月癸卯,畋于伊闕。丙午,至自伊闕。丁巳,回鶻使都督安千想來。十二月庚午,及皇后幸張全義第。



 三年春正月庚子,如東京,毀即位壇為鞠場。二月己巳,聚鞠于新場。乙亥,射鴈于王莽河。辛巳,突厥渾解樓、渤海國王大諲撰皆遣使者來。射鴈于北郊。乙酉,射鴨于郭泊。庚寅,射鴈于北郊。三月乙未,寒食望祭于西郊。庚申,至自東京。辛酉,改東京為鄴都,以洛京
 為東都。夏四月乙亥,及皇后幸郭崇韜、朱漢賓第。旱。庚寅,趙光胤薨。五月丁酉,後太妃薨,廢朝五日。己酉,黑水、女真皆遣使者來。六月辛未,宗正卿李紓為昭宗、少帝改卜園陵使。括馬。秋七月壬寅,皇太后崩。八月癸未,殺河南縣令羅貫。九月庚子,魏王繼岌為西川四面行營都統,郭崇韜為招討使以伐蜀。自六月雨至于是月。丁巳,射鴈于尖山。冬十月壬午,奚、吐渾、突厥皆遣使者來。戊子,葬貞簡太后于坤陵。十一月丁未,高麗遣使者來。



 己酉,蜀王衍降。
 郭崇韜殺王宗弼及其弟宗渥、宗訓,滅其族。十二月己卯,畋于白沙。癸未,至自白沙。閏月辛亥,封弟存美為邕王,存霸永王,存禮薛王,存渥申王,存乂睦王,存確通王,存紀雅王。



 四年春正月壬戌,降死罪以下囚。甲子,魏王繼岌殺郭崇韜及其三子于蜀。戊寅,契丹使梅老鞋里來。庚辰,殺其弟睦王存乂及河中讓國軍節度使李繼麟,滅其族。乙酉,沙州曹義金遣使者來。丙戌,回鶻阿咄欲遣使者來。丁亥,殺李繼麟之將史武、薛敬容、周唐殷、楊師太、王景、來
 仁、白奉國,皆滅其族。二月己丑,宣徽南院使李紹宏為樞密使。癸巳,鄴都軍將趙在禮反于貝州。甲午,畋于冷泉。



 趙在禮陷鄴都,武寧軍節度使李紹榮討之。邢州軍將趙太反,東北面招討使李紹真討之。甲辰,成德軍節度使李嗣源討趙在禮。三月,趙太伏誅。李嗣源反。博州守將翟建自稱刺史。甲子,殺王衍,滅其族。乙丑,如汴州。壬申,次滎澤。龍驤指揮軍使姚彥溫以前鋒軍叛降於李嗣源。嗣源入於汴州。甲戌,至自萬
 勝。從馬直指揮使郭從謙反。夏四月丁亥朔,皇帝崩。



\end{pinyinscope}