\article{卷六十一吳世家第一}

\begin{pinyinscope}

 嗚呼!自唐失其政,天下乘時,黥髡盜販,袞冕峨巍。吳暨南唐,姦豪竊攘。



 蜀險而富,漢險而貧,貧能自彊,富者先亡。閩陋荊蹙,楚開蠻服。剝剽弗堪,吳越其尤。牢牲視人,嶺皞遭劉。百年之間,並起爭雄,山川亦絕,風氣不通。語曰:清風興,群陰伏;日月出,爝火息。故真人作而天下同。作《十國世家》。



 楊行密,字化源,廬州合淝人也。為人長大有力,能手舉
 百斤。唐乾符中,江、淮群盜起,行密以為盜見獲,刺史鄭棨奇其狀貌,釋縛縱之。後應募為州兵,戍朔方,遷隊長。歲滿戍還,而軍吏惡之,復使出戍。行密將行,過軍吏舍,軍吏陽為好言,問行密行何所欲。行密奮然曰:「惟少公頭爾!」即斬其首,攜之而出,因起兵為亂,自號八營都知兵馬使。刺史郎幼復棄城走,行密遂據廬州。



 中和三年,唐即拜行密廬州刺史。淮南節度使高駢為畢師鐸所攻,駢表行密行軍司馬,行密率兵數千赴之。行至天長,師鐸已囚駢,召宣州秦彥入揚州,行密不得入,屯于蜀岡。師鐸兵眾數萬擊行密,行密陽敗,棄營走,師鐸兵
 飢,乘勝爭入營收軍實,行密反兵擊之,師鐸大敗,單騎走入城,遂殺高駢。行密聞駢死,縞軍向城哭三日,攻其西門,彥及師鐸奔於東塘,行密遂入揚州。



 是時,城中倉廩空虛,飢民相殺而食,其夫婦、父子自相牽,就屠賣之,屠者刲剔如羊豕。行密不能守,欲走。而蔡州秦宗權遣其弟宗衡掠地淮南,彥及師鐸還自東塘,與宗衡合,行密閉城不敢出。已而宗衡為偏將孫儒所殺,儒攻高郵破之,行密益懼。其客袁襲曰:「吾以新集之眾守空城,而諸將多駢舊人,非有厚恩素信力制而心服之也。今儒兵方盛,所攻必克,此諸將持兩端、因彊弱、擇嚮背之時
 也。海陵鎮使高霸,駢之舊將,必不為吾用。」行密乃以軍令召霸,霸率其兵入廣陵,行密欲使霸守天長,襲曰:「吾以疑霸而召之,其可復用乎?且吾能勝儒,無所用霸,不幸不勝,天長豈吾有哉!不如殺之,以并其眾。」行密因犒軍擒霸族之,得其兵數千。已而孫儒殺秦彥、畢師鐸,并其兵以攻行密。行密欲走海陵,襲曰:「海陵難守,而廬州吾舊治也,城廩完實,可為後圖。」行密乃走廬州。久之,未知所嚮,問襲曰:「吾欲卷甲倍道,西取洪州可乎?」襲曰:「鐘傳新得江西,勢未可圖,而秦彥之入廣陵也,召池州刺史趙鍠委以宣州。今彥且死,鍠失所恃,而守宣州非其
 本志,且其為人非公敵,此可取也。」行密乃引兵攻鍠,戰於曷山,大敗之。進圍宣州,鍠棄城走,追及殺之,行密遂入宣州。



 龍紀元年,唐拜行密宣州觀察使。行密遣田頵、安仁義、李神福等攻浙西,取蘇、常、潤州。二年,取滁、和州。景福元年,取楚州。孫儒自逐行密,入廣陵,久之,亦不能守,乃焚其城。殺民老疾以餉軍,驅其眾渡江,號五十萬,以攻行密。



 諸將田頵、劉威等遇之輒敗,行密欲走銅官。其客戴友規曰:「儒來氣銳而兵多,蓋其鋒不可當而可以挫,其眾不可敵而可久以敝之。若避而走,是就擒也。」劉威亦曰:「背城堅柵,可以不戰疲之。」行密以為然。久之,
 儒兵飢,又大疫,行密悉兵擊之,儒敗,被擒,將死,仰顧見威曰:「聞公為此策以敗我,使我有將如公者,其可敗邪!」行密收儒餘兵數千,以皂衣蒙甲,號「黑雲都」,常以為親軍。



 是歲,復入揚州,唐拜行密淮南節度使。乾寧二年,加檢校太傅、同中書門下平章事。行密以田頵守宣州,安仁義守潤州。昇州刺史馮弘鐸來附。分遣頵等攻掠,自淮以南、江以東諸州皆下之。進攻蘇州,擒其刺史成及。四年,兗州朱瑾奔於行密。初,瑾為梁所攻,求救於晉,晉遣李承嗣將勁騎數千助瑾,瑾敗,因與俱奔行密。行密兵皆江、淮人,淮人輕弱,得瑾勁騎,而兵益振。是歲,梁太
 祖遣葛從周、龐師古攻行密壽州,行密擊敗梁兵清口,殺師古,而從周收兵走,追至渒河,又大敗之。五年,錢鏐攻蘇州,及周本戰于白方湖,本敗,蘇州復入于越。天復元年,遣李神福攻越,戰臨安,大敗之,擒其將顧全武以歸。二年,馮弘鐸叛,襲宣州,及田頵戰于曷山,弘鐸敗,將入于海。行密自至東塘邀之,使人謂弘鐸曰:「勝敗,用兵常事也,一戰之衄,何苦自棄於海島?吾府雖小,猶足容君。」弘鐸感泣,行密從十餘騎,馳入其軍,以弘鐸為節度副使,以李神福代弘鐸為升州刺史。



 是歲,唐昭宗在岐,遣江淮宣諭使李儼拜行密東面諸道行營都統、檢
 校太師、中書令,封吳王。三年,以李神福為鄂岳招討使以攻杜洪,荊南成汭救洪,神福敗之于君山。梁兵攻青州,王師範來求救,遣王茂章救之,大敗梁兵,殺朱友寧。友寧,梁太祖子也,太祖大怒,自將以擊茂章,兵號二十萬,復為茂章所敗。田頵叛,襲昇州,執李神福妻子歸于宣州。行密召神福以討頵,頵遣其將王壇逆之,又遣神福書,以其妻子招之。神福曰:「吾以一卒從吳王起事,今為大將,忍背德而顧妻子乎?」立斬其使以自絕,軍士聞之,皆感奮。行至吉陽磯,頵執神福子承鼎以招之,神福叱左右射之,遂敗壇兵于吉陽。行密別遣臺濛擊頵,頵
 敗死。



 初,頵及安仁義、朱延壽等皆從行密起微賤,及江、淮甫定,思漸休息,而三人者皆猛悍難制,頗欲除之,未有以發。天復二年,錢鏐為其將許再思等叛而圍之,再思召頵攻鏐杭州,垂克,而行密納鏐賂,命頵解兵,頵恨之。頵嘗計事廣陵,行密諸將多就頵求賂,而獄吏亦有所求。頵怒曰:「吏欲我下獄也!」歸而遂謀反。



 仁義聞之亦反,焚東塘以襲常州。常州刺史李遇出戰,望見仁義,大罵之。仁義止其軍曰:「李遇乃敢辱我如此,其必有伏兵。」遂引軍卻,而伏兵果發,追至夾岡,仁義植幟解甲而食,遇兵不敢追,仁義復入潤州。行密遣王茂章、李德誠、米
 志誠等圍之。吳之軍中推朱瑾善槊,志誠善射,皆為第一。而仁義嘗以射自負,曰:「志城之弓十,不當瑾槊之一;瑾槊之十,不當仁義弓之一。」每與茂章等戰,必命中而後發,以此吳軍畏之,不敢進。行密亦欲招降之,仁義猶豫未決。茂章乘其怠,穴地道而入,執仁義,斬于廣陵。



 延壽者,行密夫人朱氏之弟也。頵及仁義之將叛也,行密疑之,乃陽為目疾,每接延壽使者,必錯亂其所見以示之。嘗行,故觸柱而仆,朱夫人扶之,良久乃蘇。



 泣曰:「吾業成而喪其目,是天廢我也!吾兒子皆不足以任事,得延壽付之,吾無恨矣。」夫人喜,急召延壽。延壽至,行密迎之
 寢門,刺殺之,出朱夫人以嫁之。



 天祐二年,遣劉存攻鄂州,焚其城,城中兵突圍而出,諸將請急擊之,存曰:「擊之復入,則城愈固,聽其去,城可取也。」是日城破,執杜洪,斬於廣陵。九月,梁兵攻破襄州,趙匡凝奔於行密。十一月,行密卒,年五十四,謚曰武忠。子渥立。溥僭號,追尊行密為太祖武皇帝,陵曰興陵。



 渥字承天,行密長子也。行密病,出渥為宣州觀察使。右衙指揮使徐溫私謂渥曰:「今王有疾而出嫡嗣,必有姦臣之謀,若它日召子,非溫使者慎無應命。」渥涕泣謝溫而去。行密病甚,命判官周隱作符召渥,隱慮渥幼弱不
 任事,勸行密用舊將有威望者代主軍政,乃薦大將劉威,行密未許。溫與嚴可求入問疾,行密以隱議告之,溫等大驚,遽詣隱所計事。隱未出,而溫見隱作召符猶在案上,急取遣之。



 渥見溫使,乃行。行密卒,渥嗣立,召周隱罵曰:「汝欲賣吾國者,復何面目見楊氏乎?」遂殺之。以王茂章為宣州觀察使。渥之入也,多輦宣州庫物以歸廣陵,茂章惜而不與,渥怒,命李簡以兵五千圍之,茂章奔於錢塘。



 天祐三年二月,劉存取岳州。四月,江西鐘傳卒,其子匡時代立,傳養子延規怨不得立,以兵攻匡時。渥遣秦裴率兵攻之。九月,克洪州,執匡時及司馬陳
 象以歸,斬象於市,赦匡時。以秦裴為江西制置使。



 梁太祖代唐,改元開平,渥仍稱天祐。鄂州劉存、岳州陳知新以舟師伐楚,敗于瀏陽,楚人執存及知新以歸。楚王馬殷素聞其名,皆欲活之,存等大罵殷曰:「昔歲宣城脫吾刃下,今日之敗,乃天亡我,我肯事汝以求活耶?我豈負楊氏者!」



 殷知不可屈,乃殺之,岳州復入于楚。



 初,渥之入廣陵也,留帳下兵三千於宣州,以其腹心陳璠、范遇將之。既入立,惡徐溫典牙兵,召璠等為東院馬軍以自衛。而溫與左衙都指揮使張顥皆行密時舊將,又有立渥之功,共惡璠等侵其權。四年正月,渥視事,璠等侍側,溫、
 顥擁牙兵入,拽璠等下,斬之,渥不能止,由是失政,而心憤未能發,溫等益不自安。五年五月,溫、顥共遣盜入寢中殺渥,渥說群盜能反殺溫等者皆為刺史。群盜皆諾,惟紀祥不從,執渥縊殺之,時年二十三,謚曰景。弟隆演立。溥僭號,追尊渥為烈宗景皇帝,陵曰紹陵。



 隆演字鴻源,行密第二子也。初名瀛,又名渭。初,溫、顥之弒渥也,約分其地以臣於梁,及渥死,顥欲背約自立。溫患之,問其客嚴可求,可求曰:「顥雖剛愎,而闇於成事,此易為也。」明日,顥列劍戟府中,召諸將議事,自大將硃瑾而下,皆去衛從然後入。顥問諸將,誰當立者,諸將莫敢
 對。顥三問,可求前密啟曰:「方今四境多虞,非公主之不可,然恐為之太速。且今外有劉威、陶雅、李簡、李遇,皆先王一等人也,公雖自立,未知此輩能降心以事公否。不若輔立幼主,漸以歲時,待其歸心,然後可也。」顥不能對。可求因趨出,書一教內袖中,率諸將入賀,諸將莫知所為。及出教宣之,乃渥母史氏教,言楊氏創業艱難,而嗣王不幸,隆演以次當立,告諸將以無負楊氏而善事之。辭旨激切,聞者感動。顥氣色皆沮,卒無能為,隆演乃得立。



 顥由此與溫有隙,諷隆演出溫潤州。可求謂溫曰:「今舍衙兵而出外郡,禍行至矣。」溫患之,可求因說顥曰:「公
 與徐溫同受顧託,議者謂公奪其衙兵,是將殺之於外,信乎?」顥曰:「事已行矣,安可止乎?」可求曰:「甚易也。」明日,從乂頁與諸將造溫,可求陽責溫曰:「古人不忘一飯之恩,況公楊氏三世之將,今幼嗣新立,多事之時,乃求居外以茍安乎?」溫亦陽謝曰:「公等見留,不願去也。」



 由是不行。行軍副使李承嗣與張顥善,覺可求有附溫意,諷顥使客夜刺殺之,客刺可求不能中。明日,可求詣溫,謀先殺顥,陰遣鐘章選壯士三十人,就衙堂斬顥,因以弒渥之罪歸之。溫由是專政,隆演備位而已。



 六月,撫州危全諷叛,攻洪州,袁州彭彥章、吉州彭釬、信州危仔倡皆起兵叛。



 隆演召嚴可求問誰可用者。可求薦周本,時本方攻蘇州敗歸,慚不肯出,可求彊起之。本曰:「蘇州之敗,非怯也,乃上將權輕,而下多專命爾。若必見任,願無用偏裨。」乃請兵七千。戰于象牙潭,敗之,執全諷、彥章,而玕奔于楚,仔倡奔於錢塘。全諷至廣陵,諸將議曰:「昔先王攻趙鍠,全諷屢餉給吳軍。」乃釋不殺。



 初,全諷欲舉兵也,錢鏐送王茂章于梁,道過全諷,謂曰:「聞公欲大舉,願見公兵,以知濟否。」全諷陣兵,與茂章登城望之,茂章曰:「我素事吳,吳兵三等,如公此眾,可當其下將爾,非得益兵十萬不可。」而全諷卒以此敗。



 八年,徐溫領昇州刺史,治舟師於
 金陵。宣州李遇自行密時為大將,勳位已高,憤溫用事,嘗曰:「徐溫何人?吾猶未識,而驟至於此。」溫聞之,怒,遣柴再用以兵送王壇代遇,且召之。遇疑不受命,再用圍之,隆演使客將何蕘諭遇使自歸。



 蕘因說曰:「公若欲反,可殺蕘以示眾,若本無心,何不隨蕘以出?」遇自以無反心,乃隨蕘出,溫諷再用伺其出,殺之,並族其家。



 九年,溫率將吏進隆演位太師、中書令、吳王。溫為行軍司馬、鎮海軍節度使、同中書門下平章事。陳章攻楚取岳州,執其刺史苑玫。十年,越人攻常州,徐溫敗之于無錫。梁遣王茂章攻壽春,溫敗之霍丘。十二年,封徐溫齊國公、兩
 浙都招討使,始鎮潤州。留其子知訓為行軍副使,秉政,而大事溫遙決之。冬,濬楊林江,水中出火,可以燃。



 十三年,宿衛將李球、馬謙挾隆演登樓,取庫兵以誅知訓,陣於門橋。知訓與戰,頻卻,朱瑾適自外來,以一騎前視其陣,曰:「此不足為也。」因反顧一麾,外兵爭進,遂斬球、謙,而亂兵皆潰。十四年,徐溫徙治金陵。十五年,遣王祺會洪、袁、信三州兵攻虔、韶,久之不克。祺病,以劉信代之。四月,副都統朱瑾殺徐知訓,瑾自殺。潤州徐知誥聞亂,率兵入,殺唐宣諭使李儼以止亂,遂秉政。



 徐氏之專政也,隆演幼懦,不能自持,而知訓尤凌侮之。嘗飲酒樓上,命
 優人高貴卿侍酒,知訓為參軍,隆演鶉衣髽髻為蒼鶻。知訓嘗使酒罵坐,語侵隆演,隆演愧恥涕泣,而知訓愈辱之。左右扶隆演起去,知訓殺吏一人,乃止。吳人皆仄目。



 知訓又與朱瑾有隙,瑾已殺知訓,攜其首馳府中示隆演曰:「今日為吳除患矣。」



 隆演曰:「此事非吾敢知。」遽起入內。瑾忿然,以首擊柱,提劍而出,府門已闔,踰垣,折其足,遂自刎死。米志誠聞瑾殺知訓,被甲率其家兵至天興門問瑾所在,聞瑾死,乃還。徐溫疑志誠助瑾,遣使殺之。嚴可求懼事不克,使人偽從湖南境上來告軍捷,召諸將入賀,擒志誠斬之。劉信克虔州,執譚全播以歸。



 十
 六年,春二月,溫率將吏請隆演即天子位,不許。夏四月,溫奉玉冊、寶綬尊隆演即吳王位。建宗廟、社稷,設百官如天子之制,改天祐十六年為武義元年,大赦境內,追尊行密孝武王,廟號太祖,渥景王,廟號烈祖。拜溫大丞相、都督中外諸軍事,封東海郡王,以徐知誥為左僕射、參知政事,嚴可求為門下侍郎,駱知祥為中書侍郎,殷文圭、沈顏為翰林學士,盧擇為吏部尚書,李宗、陳章為左、右雄武統軍,柴再用、錢鏢為左、右龍武統軍,王令謀為內樞密使,江西劉信征南大將軍,鄂州李簡鎮西大將軍,撫州李德誠平南大將軍,廬州張崇安西大將軍,
 海州王綰鎮東大將軍,文武以次進位。封宗室皆郡公。



 溫之徙鎮金陵也,以其養子知誥守潤州。嚴可求嘗謂溫曰:「二郎君非徐氏子,而推賢下士,人望頗歸,若不去之,恐為後患。」溫不能用其言。及知誥秉政,其語泄,知誥出可求於楚州,可求懼,詣金陵見溫謀曰:「唐亡於今十二年,而吳猶不敢改天祐,可謂不負唐矣。然吳所以征伐四方,而建基業者,常以興復為辭。今聞河上之戰,梁兵屢絀,若李氏復興,其能屈節乎?宜於此時先建國以自立。」溫深然之,因留可求不遣,方謀迫隆演僭號。



 二年五月,隆演卒。隆演少年嗣位,權在徐氏,及建國稱制,非
 其意,常怏怏,酣飲,稀復進食,遂至疾卒,年二十四,謚曰宣。弟溥立,僭號,追尊為高祖宣皇帝,陵曰肅陵。



 溥,行密第四子也,隆演建國,封丹陽郡公。隆演卒,弟廬江公蒙次當立,而徐氏秉政,不欲長君,乃立溥。七月,改升州大都督府為金陵府,拜徐溫金陵尹。



 明年二月,改元順義,赦境內。冬十一月,祀天於南郊。御天興樓,大赦。拜徐溫太師,嚴可求右僕射。



 三年,唐莊宗滅梁。遣司農卿盧蘋使于唐,嚴可求密條數事授蘋以行。蘋見洛陽,莊宗問之,蘋次第以對,皆如所授。



 四年,溥至白沙閱舟師,徐溫來見,以白沙為迎鑾鎮。



 五年,唐遣諫議大夫薛
 昭文使福州,假道江西,劉信出勞之,謂曰:「亞次聞有信否?」昭文曰:「天子新有河南,未熟公名也。」信曰:「漢有韓信,吳有劉信,君還,其語亞次,當來較射於淮上也。」乃酌大卮,望牙旗鎞首百步,謂昭文曰:「一發而中,願以此卮為壽,否則亦以自罰。」言訖,而箭已穿矣。



 六年,追爵大丞相徐溫四代祖考,立廟於金陵。左僕射徐知誥為侍中,右僕射嚴可求同平章事。是歲,莊宗崩,五月丁卯,詔為同光主輟朝七日。



 七年,大丞相徐溫率吳文、武上表勸溥即皇帝位,溥未許而溫病卒。十一月庚戌,溥御文明殿即皇帝位,改元曰乾貞,大赦境內,追尊行密武
 皇帝,渥景皇帝,隆演宣皇帝。以徐知誥為太尉兼侍中,拜溫子知詢輔國大將軍、金陵尹,治溫舊鎮。



 諸子皆封王。



 二年正月,封東海為廣德王,江瀆廣源王,淮瀆長源王,馬當上水府寧江王,采石中水府定江王,金山下水府鎮江王。六月,荊南高季興來附,封季興秦王。九月,季興敗楚師於白田,獲其將吏三十四人來獻。



 三年十一月,金陵尹徐知詢來朝,知誥誣其有反狀,留之不遣,以為左統軍,斬其客將周廷望。以徐知諤為金陵尹。溥加尊號睿聖文明孝皇帝,大赦境內,改元大和,以徐知誥為中書令。



 二年,冊其子江都王璉為太子。三年,以徐知
 誥為金陵尹,以其子景通為司徒,及左僕射王令謀、右僕射宋齊丘皆平章事。四年,封知誥東海王。五年,建都於金陵。六年閏正月,金陵火,罷建都,廢臨川王濛為歷陽公,知誥遣親信王宏以兵守之。拜令謀司徒,宋齊丘司空。知誥召景通還金陵,為鎮海軍節度副使,以其子景遷為太保、平章事,與令謀等執政。



 七年九月,溥加尊號曰睿聖文明光孝應天弘道廣德皇帝,大赦,改元天祚。知誥進位太師、天下兵馬大元帥,封齊王。二年,景遷病,以次子景遂為門下侍郎、參政事。三年,知誥建齊國,立宗廟、社稷,置左、右丞相已下,以金陵為西都,廣陵
 為東都。冬十月,溥遣江夏王璘奉冊禪位於齊王。十二月,溥卒於丹陽,年三十八,謚曰睿。



 昇元六年,李昪遷其子孫於海陵,號永寧宮,嚴兵守之,絕不通人。久而男女自為匹偶,吳人多哀憐之。顯德三年,世宗征淮南,下詔撫安楊氏子孫,而李景聞之,遣人盡殺其族。周先鋒都部署劉重進得其玉硯、馬腦碗、翡翠瓶以獻,楊氏遂絕。



 徐溫,字敦美,海州朐山人也,少以販鹽為盜,行密起合淝,隸帳下。行密所與起事劉威、陶雅之徒,號三十六英雄,獨溫未嘗有戰功。及行密欲殺硃延壽等,溫用其客嚴可求謀,教行密陽為目疾,事成,以功遷右衙指揮
 使,始預謀議。



 及行密病,平生舊將,皆以戰守在外,而溫居帳下,遂預立渥之功。及弒渥,又與張顥有隙,使鐘章殺之。章許諾,選壯士三十人,椎牛享之,刺血為盟。溫猶疑章不果,夜半使人探其意,陽謂曰:「溫有老母,懼事不成,不如且止。」章曰:「言已出口,寧可已乎?」溫乃安。明日,鐘章殺顥,溫因盡殺紀祥等,歸弒渥之罪於顥,以其事入白渥母史氏。史悸而泣曰:「吾兒年幼,禍亂若此,得保百口以歸合淝,公之惠也。」



 隆演立,溫遂專政,遷昇州刺史,治舟師於金陵。大將李遇怒溫用事,出嫚言,溫使柴再用族遇於宣州。行密舊將,人人皆自疑,溫因偽下之,
 恭謹如見行密,諸將乃安。八年,溫遷行軍司馬、潤州刺史、鎮海軍節度使、同平章事。十年,遣招討使李濤攻越,戰于臨安,裨將曹筠奔于越,濤敗被執。溫間遣人語筠曰:「吾用汝為將,汝軍有求,吾不能給,是吾過也。」赦筠妻子不誅,厚遇之。秋,越人攻毘陵,溫戰于無錫,筠感溫前言,臨戰奔歸,遂敗越兵。十二年,封溫齊國公,兼兩浙招討使,始就鎮潤州,以昇、潤、宣、常、池、黃六州為齊國。溫城昇州,建大都督府。十四年,徙治之,以其子知訓輔隆演於廣陵,而大事溫遙決之。知訓為朱瑾所殺,溫養子知誥自潤州先入,遂得政。



 溫雖姦詐多疑,而善用將吏。江
 西劉信圍虔州,久不克,使人說譚全播出降,遣使報溫,溫怒曰:「信以十倍之眾,攻一城不下,而反用說客降之,何以威敵國?」



 笞其使者而遣之,曰:「吾以笞信也。」因命濟師,遂破全播。人有誣信逗留陰縱全播,言信將反者,信聞之,因自獻捷至金陵見溫,溫與信博,信斂骰子厲聲祝曰:「劉信欲背吳,願為惡彩,茍無二心,當成渾花。」溫遽止之,一擲,六子皆赤,溫慚,自以卮酒飲信,然終疑之。及唐師伐王衍,溫急召信至廣陵,以為左統軍,託以內備,遂奪其地。



 溫客尤見信者,惟駱知祥、嚴可求,可求善籌畫,知祥長於財利,溫嘗以軍旅問可求,國用問知祥,吳
 人謂之「嚴、駱。」溫亦自喜為智詐,尤得吳人之心。初隨行密破趙鍠,諸將皆爭取金帛,溫獨據餘囷,作粥以食餓者。十六年,溫請隆演即皇帝位,不許,又請即吳王位,乃許,遂建國改元,拜溫大丞相、都督中外諸軍事,封東海郡王。隆演卒,溫越次立其弟溥。順義七年,溫又請溥即皇帝位,溥未許而溫病卒,年六十六,追封齊王,謚曰武。李昪僭號,號溫為義祖。



 嗚呼,盜亦有道,信哉!行密之書,稱行密為人,寬仁雅信,能得士心。其將蔡儔叛於廬州,悉毀行密墳墓,及儔敗,而諸將皆請毀其墓以報之。行密嘆曰:「儔以此為惡,吾
 豈復為邪?」嘗使從者張洪負劍而侍,洪拔劍擊行密,不中,洪死,復用洪所善陳紹負劍,不疑。又嘗罵其將劉信,信忿,奔孫儒,行密戒左右勿追,曰:「信負我者邪?其醉而去,醒必復來。」明日,果來。行密起於盜賊,其下皆驍武雄暴,而樂為之用者,以此也。故二世四主垂五十年。及渥已下,政在徐溫。於此之時,天下大亂,中國之禍,篡弒相尋,而徐氏父子,區區詐力,裴回三主,不敢輕取之,何也?豈其恩威亦有在人者歟!



 據《吳錄》、《運歷圖》、《九國志》,皆云行密以唐景福元年再入揚州,至晉天福二年,為李昪所篡,實四十六年。而《舊唐書》、《舊五代史》皆云大順二年入揚州,至被篡,四十七年。《吳錄》徐鉉等撰,《運歷圖》龔穎撰,二人皆江南故臣,所記宜得實。而唐末喪亂,中朝文字多差失,故今以鉉、穎
 所記為定。






\end{pinyinscope}