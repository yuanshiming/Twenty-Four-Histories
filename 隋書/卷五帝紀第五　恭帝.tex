\article{卷五帝紀第五 恭帝}

\begin{pinyinscope}

 恭皇帝,諱侑,元德太子之子也。母曰韋妃。性聰敏,有氣度。大業三年,立為陳王。後數載,徙為代王,邑萬戶。及煬帝親征遼東,令於京師總留事。十一年,從幸晉陽,拜太原太守。尋鎮京師。義兵入長安,尊煬帝為太上皇,奉帝纂業。



 義寧元年十一月壬戌,上即皇帝位於大興殿。詔曰:「王
 道喪亂,天步不康,古往今來,代有其事,屬之於朕,逢此百罹,彼蒼者天,胡寧斯忍!襁褓之歲,夙遭憫兇,孺子之辰,太上播越,興言感動,實疚於懷。太尉唐公,膺期作宰,時稱舟楫,大拯橫流,糾合義兵,翼戴皇室,與國休戚,再匡區夏,爰奉明詔,弼予幼沖,顯命光臨,天威咫尺,對揚尊號,悼心失圖。一人在遠,三讓不遂,黽勉南面,厝身無所,茍利社稷,莫敢或違,俯從群議,奉遵聖旨。可大赦天下,改大業十三年為義寧元年。十一月十六日昧爽以前,大闢罪以下,皆赦除之;常赦所不免者,不在赦限。」甲子,以光祿大夫、大將軍、太尉唐公為假黃鉞、使持節、大
 都督內外諸軍事、尚書令、大丞相,進封唐王。丙寅,詔曰:「朕惟孺子,未出深宮,太上遠巡,追蹤穆滿。時逢多難,委當尊極,辭不獲免,恭己臨朝,若涉大川,罔知所濟,撫躬永嘆,憂心孔棘。民之情偽,曾未之聞,王業艱難,載云其易。賴股肱戮力,上宰賢良,匡佐沖人,輔其不逮。軍國機務,事無大小,文武設官,位無貴賤,憲章賞罰,咸歸相府,庶績其凝,責成斯屬,逖聽前史,茲為典故。因循仍舊,非曰徒言,所存至公,無為讓德。」己巳,以唐王子隴西公建成為唐國世子,敦煌公為京兆尹,改封秦公,元吉為齊公,食邑各萬戶。太原置鎮北府。乙亥,張掖康老和舉兵
 反。十二月癸未,薛舉自稱天子,寇扶風。秦公為元帥,擊破之。丁亥,桂陽人曹武徹舉兵反,建元通聖。丁酉,義師擒驍衛大將軍屈突通於閿鄉,虜其眾數萬。乙巳,賊帥張善安陷廬江郡。



 二年春正月丁未,詔唐王劍履上殿,入朝不趨,贊拜不名,加前後羽葆鼓吹。



 壬戌,將軍王世充為李密所敗,河內通守孟善誼、武賁郎將王辯、楊威、劉長恭、梁德、董智通皆死之。庚戌,河陽郡尉獨孤武都降於李密。三月丙辰,右屯衛將軍宇文化及殺太上皇於江都宮,右御衛將軍獨孤盛死之。齊王暕,趙王杲,燕王倓,光祿大夫、開
 府儀同三司、行右翊衛大將軍宇文協,金紫光祿大夫、內史侍郎虞世基,銀青光祿大夫、御史大夫裴蘊,通議大夫、行給事郎許善心皆遇害。化及立秦王浩為帝,自稱大丞相,朝士文武皆受其官爵。光祿大夫、宿公麥才,折沖郎將、朝請大夫沈光,同謀討賊,夜襲化及營,反為所害。戊辰,詔唐王備九錫之禮,加璽紱、遠游冠、綠綟綬,位在諸侯王上。唐國置丞相已下,一依舊式。



 五月乙巳朔,詔唐王冕十有二旒,建天子旌旗,出警入蹕,金根車駕,備五時副車,置旄頭雲蒨車,儛八佾,設鐘虡宮懸。王后、王子、王女爵命之號,一遵舊典。戊午,詔曰:天禍隋國,
 大行太上皇遇盜江都,酷甚望夷,釁深驪北。憫予小子,奄逮丕愆,哀號承感,心情糜潰,仰惟荼毒,仇復靡申,形影相吊,罔知啟處。相國唐王,膺期命世,扶危拯溺,自北徂南,東征西怨,總九合於一匡,決百勝於千里,糾率夷夏,大庇氓黎,保乂朕躬,繄王是賴。德侔造化,功格蒼旻,兆庶歸心,歷數斯在,屈為人臣,載違天命。在昔虞夏,揖讓相推,茍非重華,誰堪命禹!當今九服崩離,三靈改卜,大運去矣,請避賢路,兆謀布德,顧己莫能,私僮命駕,須歸籓國。予本代王,及予而代,天之所廢,豈期如是!庶憑稽古之聖,以誅四兇,幸值惟新之恩,預充三恪。雪冤恥
 於皇祖,守禋祀為孝孫,朝聞夕殞,及泉無恨,今遵故事,遜於舊邸。庶官群闢,改事唐朝,宜依前典,趣上尊號。若釋重負,感泰兼懷,假手真人,俾除醜逆。濟濟多士,明知朕意。



 仍敕有司,凡有表奏,皆不得以聞。是日,上遜位於大唐,以為酅國公。武德二年夏五月崩,時年十五。



 史臣曰:恭帝年在幼沖,遭家多難,一人失德,四海土崩。群盜蜂起,豺狼塞路,南巢遂往,流彘不歸。既鐘百六之期,躬踐數終之運,謳歌有屬,笙鐘變響,雖欲不遵堯舜之跡,其庸可得乎!



\end{pinyinscope}