\article{卷四十三列傳第八 河間王弘(子慶)}

\begin{pinyinscope}

 河間王弘,字闢惡,高祖從祖弟也。祖愛敬,早卒。父元孫,少孤,隨母郭氏養於舅族。及武元皇帝與周太祖建義關中,元孫時在鄴下,懼為齊人所誅,因假外家姓為郭氏。元孫死,齊為周所並,弘始入關,與高祖相得。高祖哀之,為買田宅。



 弘性明悟,有文武幹略。數從征伐,累遷開府儀同三司。高祖為丞相,常置左右,委以心腹。高祖詣
 周趙王宅,將及於難,弘時立於戶外,以衛高祖。尋加上開府,賜爵永康縣公。及上受禪,拜大將軍,進爵郡公。尋贈其父為柱國、尚書令、河間郡公。其年立弘為河間王,拜右衛大將軍。歲餘,進授柱國。時突厥屢為邊患,以行軍元帥率眾數萬,出靈州道,與虜相遇,戰,大破之,斬數千級。賜物二千段,出拜寧州總管,進位上柱國。弘在州,治尚清靜,甚有恩惠。後數載,徵還京師。



 未幾,拜蒲州刺史,得以便宜從事。時河東多盜賊,民不得安。弘奏為盜者百餘人,投之邊裔,州境帖然,號為良吏。每晉王廣入朝,弘輒領揚州總管,及晉王歸籓,弘復還蒲州。在官十
 餘年,風教大洽。煬帝嗣位,徵還,拜太子太保。歲餘,薨。



 大業六年,追封郇王。子慶嗣。



 慶傾曲,善候時變。帝時猜忌骨肉,滕王綸等皆被廢放,唯慶獲全。累遷滎陽郡太守,頗有治績。及李密據洛口倉,榮陽諸縣多應密,慶勒兵拒守。密頻遣攻之,不能克。歲餘,城中糧盡,兵勢日蹙。密因遺慶書曰:自昏狂嗣位,多歷歲年,剝削生民,塗炭天下。璇室瑤臺之麗,未極驕奢;糟丘酒池之荒,非為淫亂。今者共舉義旗,勘剪兇虐,八方同德,萬里俱來,莫不期入關以亡秦,爭渡河而滅紂。東窮海、岱,南洎江、淮,凡厥遺人,承風慕義,唯滎陽一
 郡,王獨守迷。夫微子紂之元兄,族實為重,項伯籍之季父,戚乃非疏,然猶去朝歌而入周,背西楚而歸漢。豈不眷戀宗祊,留連骨肉,但識寶鼎之將移,知神器之先改。而王之先代,家住山東,本姓郭氏,乃非楊族。止為宿與隋朝先有勛舊,遂得預沾盤石,名在葭莩。婁敬之與漢高,殊非血胤,呂布之於董卓,良異天親,芝焚蕙嘆,事不同此。又王之昏主,心若豺狼,仇忿同胞,有逾沉、閼,惟勇及諒,咸磬甸師,況及族類為非,何能自保!為王計者,莫若舉城從義,開門送款,安若太山,高枕而臥,長守富貴,足為美談,乃至子孫,必有餘慶。今王世充屢被摧蹙,自
 救無聊,偷存晷漏,詎能支久?段達、韋津,東都自固,何暇圖人?世充朝亡,達便夕滅。又江都荒湎,流宕忘歸,內外崩離,人神怨憤。上江米船,皆被抄截,士卒饑餒,半菽不充,事切析骸,義均煮弩。舉烽火於驪山,諸侯莫至;浮膠船於漢水,還日未期。王獨守孤城,絕援千里,餱糧之計,僅有月餘,敝卒之多,才盈數百,有何恃賴,欲相拒抗!求枯魚於市肆,即事非虛;因歸雁以運糧,竟知何日。然城中豪傑,王之腹心,思殺長吏,將為內啟。正恐禍生匕首,釁發蕭墻,空以七尺之軀,懸賞千金之購,可為寒心,可為酸鼻者也。幸能三思,自求多福。



 於時江都敗問亦至,
 慶得書,遂降於密,改姓為郭氏。密為王世充所破,復歸東都,更為楊氏,越王侗不之責也。及侗稱制,拜宗正卿。世充將篡,慶首為勸進。



 世充既僭偽號,降爵郇國公,慶復為郭氏。世充以兄女妻之,署滎州刺史。及世充將敗,慶欲將其妻同歸長安,其妻謂之曰:「國家以妾奉箕帚於公者,欲以申厚意,結公心耳。今叔父窮迫,家國阽危,而公不顧婚姻,孤負付屬,為全身之計,非妾所能責公也。妾若至長安,則公家一婢耳,何用妾為!願得送還東都,君之惠也。」



 慶不許。其妻遂沐浴靚妝,仰藥而死。慶歸大唐,為宜州刺史、郇國公,復姓楊氏。



 其嫡母元太妃,年老,
 兩目失明,王世充以慶叛己而斬之。



 楊處綱楊處綱,高祖族父也。生長北邊,少習騎射。在周嘗以軍功拜上儀同。高祖受禪,贈其父鐘葵為柱國、尚書令、義城縣公,以處綱襲焉。授開府,督武候事。尋為太子宗衛率,轉左監門郎將。後數載,起授右領軍將軍。處綱雖無才藝,而性質直,在官強濟,亦為當時所稱。尋拜蒲州刺史,吏民悅之。進位大將軍。後遷秦州總管,卒官。謚曰恭。



 弟處樂,官至洛州刺史。漢王諒之反也,朝廷以為有二心,廢錮不齒。



 楊子崇楊子崇,高祖族弟也。父盆生,贈荊州刺史。子崇少好學,涉獵書記,有風儀,愛賢好士。開皇初,拜儀同,以車騎將軍恆典宿衛。後為司門侍郎。煬帝嗣位,累遷候衛將軍,坐事免。未幾,復令檢校將軍事。從帝幸汾陽宮,子崇知突厥必為寇患,屢請早還京師,帝不納。尋有雁門之圍。及賊退,帝怒之曰:「子崇怯軟,妄有陳請,驚動我眾心,不可居爪牙之寄。」出為離石郡太守,治有能名。自是突厥屢寇邊塞,胡賊劉六兒復擁眾劫掠郡境,子崇上表請兵鎮遏。帝復大怒,下書令子崇巡行長城。子崇出百餘
 里,四面路絕,不得進而歸。時百姓饑饉,相聚為盜,子崇前後捕斬數千人。歲餘,朔方梁師都、馬邑劉武周等各稱兵作亂,郡中諸胡復相嘯聚。子崇患之,言欲朝集,遂與心腹數百人自孟門關將還京師。輜重半濟,遇河西諸縣各殺長吏,叛歸師都,道路隔絕,子崇退歸離石。所將左右,既聞太原有兵起,不復入城,遂各叛去。子崇悉收叛者父兄斬之。後數日,義兵夜至城下,城中豪傑復出應之。城陷,子崇為仇家所殺。



 觀德王雄弟達觀德王雄,初名惠,高祖族子也。父紹,仕周,歷八州刺史、
 儻城縣公,賜姓叱呂引氏。雄美姿儀,有器度,雍容閑雅,進止可觀。周武帝時,為太子司旅下大夫。帝幸雲陽宮,衛王直作亂,以其徒襲肅章門,雄逆拒破之。進位上儀同,封武陽縣公,邑千戶。累遷右司衛上大夫。大象中,進爵邗國公,邑五千戶。高祖為丞相,雍州牧畢王賢謀作難,雄時為別駕,知其謀,以告高祖。賢伏誅,以功授柱國、雍州牧,仍領相府虞候。周宣帝葬,備諸王有變,令雄率六千騎送至陵所。進位上柱國。



 高祖受禪,除左衛將軍,兼宗正卿。俄遷右衛大將軍,參預朝政。進封廣平王,食邑五千戶,以邗公別封一子。雄請封弟士貴,朝廷許之。
 或奏高熲朋黨者,上詰雄於朝。雄對曰:「臣忝衛宮闈,朝夕左右,若有朋附,豈容不知!至尊欽明睿哲,萬機親覽,熲用心平允,奉法而行。此乃愛憎之理,惟陛下察之。」高祖深然其言。



 雄時貴寵,冠絕一時,與高熲、虞慶則、蘇威稱為「四貴」。



 雄寬容下士,朝野傾矚。高祖惡其得眾,陰忌之,不欲其典兵馬。乃下冊書,拜雄為司空,曰:「維開皇九年八月朔壬戌,皇帝若曰:於戲!惟爾上柱國、左衛大將軍、宗正卿、廣平王,風度寬弘,位望隆顯,爰司禁旅,綿歷十載。入當心腹,外任爪牙,驅馳軒陛,勤勞著績。念舊庸勛,禮秩加等。公輔之寄,民具爾瞻,宜竭乃誠,副茲名實,
 是用命爾為司空。往欽哉!光應寵命,得不慎歟!」外示優崇,實奪其權也。雄無職務,乃閉門不通賓客。尋改封清漳王。仁壽初,高祖曰:「清漳之名,未允聲望。」命職方進地圖,上指安德郡以示群臣曰:「此號足為名德相稱。」於是改封安德王。



 大業初,授太子太傅。及元德太子薨,檢校鄭州刺史事。歲餘,授懷州刺史。



 尋拜京兆尹。帝親征吐谷渾,詔雄總管澆河道諸軍。及還,改封觀王。上表讓曰:「臣早逢興運,預班末屬,有命有時,藉風雲之會,無才無德,濫公卿之首。蒙先皇不次之賞,荷陛下非分之恩,久紊臺槐,常慮盈滿,豈可仍叨匪服,重竊鴻名!



 臣實面墻,
 敢緣往例,臣誠昧寵,交懼身責。昔劉賈封王,豈備三階之任,曹洪上將,寧超五等之爵?況臣袞章逾於帝子,京尹亞於皇枝,錫士作籓,鈕金開國,於臣何以自處,在物謂其乖分。是以露款執愚,祈恩固守。伏願陛下曲留慈照,特鑒丹誠。頻觸宸嚴,伏增流汗。」優詔不許。



 遼東之役,檢校左翊衛大將軍,出遼東道。次瀘河鎮,遘疾而薨,時年七十一。



 帝為之廢朝,鴻臚監護喪事。有司考行,請謚曰懿。帝曰:「王道高雅俗,德冠生人。」乃賜謚曰德。贈司徒、襄國武安渤海清河上黨河間濟北高密濟陰長平等十郡太守。



 子恭仁,位至吏部侍郎。恭仁弟綝,性和厚,頗
 有文學。歷義州刺史、淮南太守。及父薨,起為司隸大夫。遼東之役,帝令綝於臨海頓別有所督。楊玄感之反也,玄感弟玄縱自帝所逃赴其兄,路逢綝。綝避人偶語久之,既別而復相就者數矣。司隸刺史劉休文奏之。時綝兄吏部侍郎恭仁將兵於外,帝以是寢之,未發其事。綝尤懼,發病而卒。綝弟續,仕至散騎侍郎。



 雄弟達,字士達。少聰敏,有學行。仕周,官至儀同、內史下大夫,遂寧縣男。



 高祖受禪,拜給事黃門侍郎,進爵為子。時吐谷渾寇邊,詔上柱國元諧為元帥,達為司馬。軍還、兼吏部侍郎,加開府。歲餘,轉內史侍郎,出為鄯、鄭、趙三
 州刺史,俱有能名。平陳之後,四海大同,上差品天下牧宰,達為第一,賜雜彩五百段,加以金帶,擢拜工部尚書,加位上開府。達為人弘厚,有局度。楊素每言曰:「有君子之貌,兼君子之心者,唯楊達耳。」獻皇后及高祖山陵制度,達並參豫焉。



 煬帝嗣位,轉納言,仍領營東都副監,帝甚信重之。遼東之役,領右武衛將軍,進位左光祿大夫,卒於師,時年六十二。帝嘆惜者久之,贈吏部尚書、始安侯。謚曰恭。贈物三百五十段。



 史臣曰:高祖始遷周鼎,眾心未附,利建同姓,維城宗社,是以河間、觀德,咸啟山河。屬乃葭莩,地非寵逼,故高位
 厚秩,與時終始。楊慶二三其德,志在茍生,變本宗如反掌,棄慈母如遺跡,及身而絕,宜其然矣。觀王位登臺袞,慶流後嗣,保茲寵祿,實仁厚之所致乎!



\end{pinyinscope}