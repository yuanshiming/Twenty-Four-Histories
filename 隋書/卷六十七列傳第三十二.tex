\article{卷六十七列傳第三十二}

\begin{pinyinscope}

 虞世基虞世基,字茂世,會稽餘姚人也。父荔,陳太子中庶子。世基幼沉靜,喜慍不形於色,博學有高才,兼善草隸。陳中書令孔奐見而曰:「南金之貴,屬在斯人。」



 少傅徐陵聞其名,召之,世基不往。後因公會,陵一見而奇之,顧謂朝士曰:「當今潘、陸也。」因以弟女妻焉。仕陳,釋褐建安王法曹參軍事,歷祠部殿中二曹郎、太子中舍人。遷中庶子、
 散騎常侍、尚書左丞。陳主嘗於莫府山校獵,令世基作《講武賦》,於坐奏之曰:夫玩居常者,未可論匡濟之功;應變通者,然後見帝王之略。何則?化有文質,進讓殊風,世或澆淳,解張累務。雖復順紀合符之後,望雲就日之君,且修戰於版泉,亦治兵於丹浦。是知文德武功,蓋因時而並用,經邦創制,固與俗而推移。所以樹鴻名,垂大訓,拱揖百靈,包舉六合,其唯聖人乎!



 鶉火之歲,皇上御宇之四年也。萬物交泰,九有乂安,俗躋仁壽,民資日用。



 然而足食足兵,猶載懷於履薄;可久可大,尚懍乎於御朽。至如昆吾遠贐,肅慎奇賝,史不絕書,府無虛月。貝胄雍
 弧之用,犀渠闕鞏之殷,鑄名劍於尚方,積雕戈於武庫。熊羆百萬,貔豹千群,利盡五材,威加四海。爰於農隙,有事春蒐,舍爵策動,觀使臣之以禮,沮勸賞罰,乃示民以知禁。盛矣哉,信百王之不易,千載之一時也!昔上林從幸,相如於是頌德,長楊校獵,子雲退而為賦。雖則體物緣情,不同年而語矣,英聲茂實,蓋可得而言焉。其辭曰:惟則天以稽古,統資始於群分。膺錄圖而出震,樹司牧以為君。既濟寬而濟猛,亦乃武而乃文。北怨勞乎殷履,南伐盛於唐勛。彼周乾與夏戚,粵可得而前聞。我大陳之創業,乃撥亂而為武。戡定艱難,平壹區宇。從喋喋之
 樂推,爰蒼蒼而再補。



 故累仁以積德,諒重規而襲矩。惟皇帝之休烈,體徇齊之睿哲。敷九疇而咸敘,奄四海而有截。既搜揚於帝難,又文思之安安。幽明請吏,俊乂在官。御璇璣而七政辨,朝玉帛而萬國歡。昧旦丕顯,未明思治。道藏往而知來,功參天而兩地。運聖人之上德,盡生民之能事。於是禮暢樂和,刑清政肅。西暨析支,東漸蟠木。罄圖諜而效祉,漏川泉而禔福。在靈貺而必臻,亦何思而不服。雖至治之隆平,猶戒國而強兵。選羽林於六郡,詔蹶張於五營。兼折沖而餘勇,咸重義而輕生。遂乃因農隙以教民,在春蒐而習戰。命司馬以示法,帥掌
 固而清甸。導旬始以前驅,伏鉤陳而後殿。抗鳥旌於析羽,飾魚文於被練。爾乃革軒按轡,玉虯齊鞅。屯左矩以啟行,擊右鐘而傳響。交雲罕之掩映,紛劍騎而來往。指攝提於斗極,洞閶闔之弘敞。跨玄武而東臨,款黃山而北上。隱圓闕之迢遞,屆方澤之塏爽。於斯時也,青春晚候,朝陽明岫。日月光華,煙雲吐秀。澄波瀾於江海,靜氛埃於宇宙。乘輿乃御太一之玉堂,授軍令於紫房。蘊龍韜之妙算,誓武旅於戎場。銳金顏於庸蜀,躪鐵騎於漁陽。彀神弩而持滿,彏天弧而並張。曳虹旗之正正,振夔鼓之鏜鏜。八陳肅而成列,六軍儼以相望。拒飛梯於縈
 帶,聳樓車於武岡。或掉鞅而直指,乍交綏而弗傷。裁應變而蛇擊,俄蹈厲以鷹揚。中小枝於戟刃,徹蹲札於甲裳。聊七縱於孟獲,乃兩擒於卡莊。始軒軒而鶴舉,遂離離以雁行。振川谷而橫八表,蕩海岳而耀三光。諒窈冥之不測,羌進退而難常。亦有投石扛鼎,超乘挾輈。沖冠聳劍,鐵楯銅頭。熊渠殆兇,武勇操牛。雖任鄙與賁、育,故無得而為仇。九攻既決,三略已周。鳴鐲振響,風卷電收。於是勇爵班,金奏設,登元、凱而陪位,命方、邵而就列。三獻式序,八音未闋。舞干戚而有豫,聽鼓鞞而載悅。俾挾纊與投醪,咸忘軀而殉節。



 方席卷而橫行,見王師之有
 征。登燕山而戮封豕,臨瀚海而斬長鯨。望雲亭而載蹕,禮升中而告成。實皇王之神武,信蕩蕩而難名者也。



 陳主嘉之,賜馬一匹。及陳滅歸國,為通直郎,直內史省。貧無產業,每傭書養親,怏怏不平。嘗為五言詩以見意,情理淒切,世以為工,作者莫不吟詠。未幾,拜內史舍人。



 煬帝即位,顧遇彌隆。禮書監河東柳顧言博學有才,罕所推謝,至是與世基相見,嘆曰:「海內當共推此一人,非吾儕所及也。」俄遷內史侍郎,以母憂去職,哀毀骨立。有詔起令視事,拜見之日,殆不能起,帝令左右扶之。哀其羸瘠,詔令進肉,世基食輒悲哽,不能下。帝使謂之曰:「方相
 委任,當為國惜身。」前後敦勸者數矣。帝重其才,親禮逾厚,專典機密,與納言蘇威、左翊衛大將軍宇文述、黃門侍郎裴矩、御史大夫裴蘊等參掌朝政。於時天下多事,四方表奏日有百數。帝方凝重,事不庭決,入閤之後,始召世基口授節度。世基至省,方為敕書,日且百紙,無所遺謬。其精審如是。遼東之役,進位金紫光祿大夫。後從幸雁門,帝為突厥所圍,戰士多敗。世基勸帝重為賞格,親自撫循,又下詔停遼東之事。帝從之,師乃復振。及圍解,勛格不行,又下伐遼之詔。由是言其詐眾,朝野離心。



 帝幸江都,次鞏縣,世基以盜賊日盛,請發兵屯洛口倉,
 以備不虞。帝不從,但答云:「卿是書生,定猶恇怯。」於時天下大亂,世基知帝不可諫止,又以高熲、張衡等相繼誅戮,懼禍及己,雖居近侍,唯諾取容,不敢忤意。盜賊日甚,郡縣多沒。世基知帝惡數聞之,後有告敗者,乃抑損表狀,不以實聞。是後外間有變,帝弗之知也。嘗遣太僕楊義臣捕盜於河北,降賊數十萬,列狀上聞。帝嘆曰:「我初不聞賊頓如此,義臣降賊何多也!」世基對曰:「鼠竊雖多,未足為慮。義臣克之,擁兵不少,久在閫外,此最非宜。」帝曰:「卿言是也。」遽追義臣,放其兵散。



 又越王侗遣太常丞元善達間行賊中,詣江都奏事,稱李密有眾百萬,圍逼
 京都,賊據洛口倉,城內無食,若陛下速還,烏合必散,不然者,東都決沒。因歔欷嗚咽,帝為之改容。世基見帝色憂,進曰:「越王年小,此輩誑之。若如所言,善達何緣來至?」帝乃勃然怒曰:「善達小人,敢廷辱我!」因使經賊中,向東陽催運,善達遂為群盜所殺。此後外人杜口,莫敢以賊聞奏。



 世基貌沉審,言多合意,是以特見親愛,朝臣無與為比。其繼室孫氏,性驕淫,世基惑之,恣其奢靡。雕飾器服,無復素士之風。孫復攜前夫子夏侯儼入世基舍,而頑鄙無賴,為其聚斂。鬻官賣獄,賄賂公行,其門如市,金寶盈積。其弟世南,素國士,而清貧不立,未曾有所贍。由
 是為論者所譏,朝野咸共疾怨。宇文化及殺逆也,世基乃見害焉。



 長子肅,好學多才藝,時人稱有家風。弱冠早沒。肅弟熙,大業末為符璽郎。



 次子柔、晦,並宣義郎。化及將亂之夕,宗人虞人及知而告熙曰:「事勢以然,吾將濟卿南渡,且得免禍,同死何益!」熙謂人及曰:「棄父背君,求生何地?感尊之懷,自此訣矣。」及難作,兄弟競請先死,行刑人於是先世基殺之。



 裴蘊裴蘊,河東聞喜人也。祖之平,梁衛將軍。父忌,陳都官尚書,與吳明徹同沒於周,賜爵江夏郡公,在隋十餘年而
 卒。蘊性明辯,有吏乾。在陳仕歷直閣將軍、興寧令。蘊以其父在北,陰奉表於高祖,請為內應。及陳平,上悉閱江南衣冠之士,次至蘊,上以為夙有向化之心,超授儀同。左僕射高熲不悟上旨,進諫曰:「裴蘊無功於國,寵逾倫輩,臣未見其可。」上又加蘊上儀同,熲復進諫,上曰:「可加開府。」熲乃不敢復言,即日拜開府儀同三司,禮賜優洽。歷洋、直、隸三州刺史,俱有能名。大業初,考績連最。煬帝聞其善政,徵為太常少卿。初,高祖不好聲技,遣牛弘定樂,非正聲清商及九部四儛之色,皆罷遣從民。至是,蘊揣知帝意,奏括天下周、齊、梁、陳樂家子弟,皆為樂戶。其
 六品已下,至於民庶,有善音樂及倡優百戲者,皆直太常。是後異技淫聲咸萃樂府,皆置博士弟子,遞相教傳,增益樂人至三萬餘。帝大悅,遷民部侍郎。



 於時猶承高祖和平之後,禁網疏闊,戶口多漏。或年及成丁,猶詐為小,未至於老,已免租賦。蘊歷為刺史,素知其情,因是條奏,皆令貌閱。若一人不實,則官司解職,鄉正裏長皆遠流配。又許民相告,若糾得一丁者,令被糾之家代輸賦役。



 是歲大業五年也,諸郡計帳,進丁二十四萬三千,新附口六十四萬一千五百。帝臨朝覽狀,謂百官曰:「前代無好人,致此罔冒。今進民戶口皆從實者,全由裴蘊一
 人用心。古語云,得賢而治,驗之信矣。」由是漸見親委,拜京兆贊治,發擿纖毫,吏民懾憚。



 未幾,擢授御史大夫,與裴矩、虞世基參掌機密。蘊善候伺人主微意,若欲罪者,則曲法順情,鍛成其罪。所欲宥者,則附從輕典,因而釋之。是後大小之獄皆以付蘊,憲部大理莫敢與奪,必稟承進止,然後決斷。蘊亦機辯,所論法理,言若懸河,或重或輕,皆由其口,剖析明敏,時人不能致詰。楊玄感之反也,帝遣蘊推其黨與,謂蘊曰:「玄感一呼而從者十萬,益知天下人不欲多,多即相聚為盜耳。



 不盡加誅,則後無以勸。」蘊由是乃峻法治之,所戮者數萬人,皆籍沒其家。
 帝大稱善,賜奴婢十五口。司隸大夫薛道衡以忤意獲譴,蘊知帝惡之,乃奏曰:「道衡負才恃舊,有無君之心。見詔書每下,便腹非私議,推惡於國,妄造禍端。論其罪名,似如隱昧,源其情意,深為悖逆。」帝曰:「然。我少時與此人相隨行役,輕我童稚,共高熲、賀若弼等外擅威權,自知罪當誣惣。及我即位,懷不自安,賴天下無事,未得反耳。公論其逆,妙體本心。」於是誅道衡。又帝問蘇威以討遼之策,威不願帝復行,且欲令帝知天下多賊,乃詭答曰:「今者之役,不願發兵,但詔赦群盜,自可得數十萬。遣關內奴賊及山東歷山飛、張金稱等頭別為一軍,出遼西
 道,諸河南賊王薄、孟讓等十餘頭並給舟楫,浮滄海道,必喜於免罪,競務立功,一歲之間,可滅高麗矣。」帝不懌曰:「我去尚猶未克,鼠竊安能濟乎?」威出後,蘊奏曰:「此大不遜,天下何處有許多賊!」帝悟曰:「老革多奸,將賊脅我。欲搭其口,但隱忍之,誠極難耐。」蘊知上意,遣張行本奏威罪惡,帝付蘊推鞫之,乃處其死。帝曰:「未忍便殺。」遂父子及孫三世並除名。蘊又欲重己權勢,令虞世基奏罷司隸刺史以下官屬,增置御史百餘人。於是引致奸黠,共為朋黨,郡縣有不附者,陰中之。於時軍國多務,凡是興師動眾,京都留守,及與諸蕃互市,皆令御史監之。賓
 客附隸,遍於郡國,侵擾百姓,帝弗之知也。以渡遼之役,進位銀青光祿大夫。及司馬德戡將為亂,江陽長張惠紹夜馳告之。蘊共惠紹謀,欲矯詔發郭下兵民,盡取榮公來護兒節度,收在外逆黨宇文化及等,仍發羽林殿腳,遣範富婁等入自西苑,取梁公蕭鉅及燕王處分,扣門援帝。謀議已定,遣報虞世基。世基疑反者不實,抑其計。須臾,難作,蘊嘆曰:「謀及播郎,竟誤人事。」遂見害。子愔為尚輦直長,亦同日死。



 裴矩裴矩,字弘大,河東聞喜人也。祖他,魏都官尚書。父訥之,
 齊太子舍人。矩襁褓而孤,及長好學,頗愛文藻,有智數。世父讓之謂矩曰:「觀汝神識,足成才士,欲求宦達,當資干世之務。」矩始留情世事。齊北平王貞為司州牧,闢為兵曹從事,轉高平王文學。及齊亡,不得調。高祖為定州總管,召補記室,甚親敬之。



 以母憂去職。高祖作相,遣使者馳召之,參相府記室事。及受禪,遷給事郎,奏舍人事。伐陳之役,領元帥記室。既破丹陽,晉王廣令矩與高熲收陳圖籍。明年,奏詔巡撫嶺南,未行而高智慧、汪文進等相聚作亂,吳、越道閉,上難遣矩行。矩請速進,上許之。行至南康,得兵數千人。時俚帥王仲宣逼廣州,遣其所
 部將周師舉圍東衡州。矩與大將軍鹿願赴之,賊立九柵,屯大庾嶺,共為聲援。矩進擊破之,賊懼,釋東衡州,據原長嶺。又擊破之,遂斬師舉,進軍自南海援廣州。仲宣懼而潰散。矩所綏集者二十餘州,又承制署其渠帥為刺史、縣令。及還報,上大悅,命升殿勞苦之,顧謂高熲,楊素曰:「韋洸將二萬兵,不能早度嶺,朕每患其兵少。



 裴矩以三千敝卒,徑至南康。有臣若此,朕亦何憂!」以功拜開府,賜爵聞喜縣公,賚物二千段。除民部侍郎,尋遷內史侍郎。



 時突厥強盛,都藍可汗妻大義公主,即宇文氏之女也,由是數為邊患。後因公主與從胡私通,長孫晟先
 發其事,矩請出使說都藍,顯戮宇文氏。上從之。竟如其言,公主見殺。後都藍與突利可汗構難,屢犯亭鄣,詔太平公史萬歲為行軍總管,出定襄道,以矩為行軍長史,破達頭可汗於塞外。萬歲被誅,功竟不錄。上以啟民可汗初附,令矩撫慰之,還為尚書左丞。其年,文獻皇后崩,太常舊無儀注,矩與牛弘據齊禮參定之。轉吏部侍郎,名為稱職。煬帝即位,營建東都,矩職修府省,九旬而就。時西域諸蕃,多至張掖,與中國交市。帝令矩掌其事。矩知帝方勤遠略,諸商胡至者,矩誘令言其國俗山川險易,撰《西域圖記》三卷,入朝奏之。其序曰:臣聞禹定九州,
 導河不逾積石;秦兼六國,設防止及臨洮。故知西胡雜種,僻居遐裔,禮教之所不及,書典之所罕傳。自漢氏興基,開拓河右,始稱名號者,有三十六國,其後分立,乃五十五王。仍置校尉、都護,以存招撫。然叛服不恆,屢經征戰,後漢之世,頻廢此官。雖大宛以來,略知戶數,而諸國山川,未有名目。



 至如姓氏風土,服章物產,全無纂錄,世所弗聞。復以春秋遞謝,年代久遠,兼並誅討,互有興亡。或地是故邦,改從今號,或人非舊類,因襲昔名。兼復部民交錯,封疆移改,戎狄音殊,事難窮驗。於闐之北,蔥嶺以東,考於前史,三十餘國。其後更相屠滅,僅有十存。自
 餘淪沒,掃地俱盡,空有丘墟,不可記識。皇上膺天育物,無隔華夷,率土黔黎,莫不慕化。風行所及,日入以來,職貢皆通,無遠不至。



 臣既因撫納,監知關市,尋討書傳,訪採胡人,或有所疑,即詳眾口。依其本國服飾儀形,王及庶人,各顯容止,即丹青模寫,為《西域圖記》,共成三卷,合四十四國。仍別造地圖,窮其要害。從西頃以去,北海之南,縱橫所亙,將二萬里。諒由富商大賈,周游經涉,故諸國之事,罔不遍知。復有幽荒遠地,卒訪難曉,不可憑虛,是以致闕。而二漢相踵,西域為傳,戶民數十,即稱國王,徒有名號,乃乖其實。今者所編,皆餘千戶,利盡西海,多
 產珍異。其山居之屬,非有國名,及部落小者,多亦不載。發自敦煌,至於西海,凡為三道,各有襟帶。北道從伊吾,經蒲類海鐵勒部突厥可汗庭,度北流河水,至拂菻國,達於西海。其中道從高昌、焉耆、龜茲、疏勒、度蔥嶺,又經鈸汗、蘇對沙那國、康國、曹國、何國、大小安國、穆國,至波斯,達於西海。其南道從鄯善,於闐,硃俱波、喝槃陀,度蔥嶺,又經護密、吐火羅、挹怛、忛延,漕國,至北婆羅門,達於西海。其三道諸國,亦各自有路,南北交通。其東女國、南婆羅門國等,並隨其所往,諸處得達。故知伊吾、高昌、鄯善,並西域之門戶也。總湊敦煌,是其咽喉之地。以國家
 威德,將士驍雄,泛濛汜而揚旌,越昆侖而躍馬,易如反掌,何往不至!但突厥、吐渾分領羌胡之國,為其擁遏,故朝貢不通。今並因商人密送誠款,引領翹首,願為臣妾。聖情含養,澤及普天,服而撫之,務存安輯。故皇華遣使,弗動兵車,諸蕃即從,渾、厥可滅。



 混一戎夏,其在茲乎!不有所記,無以表威化之遠也。



 帝大悅,賜物五百段,每日引矩至御坐,親問西方之事。矩盛言胡中多諸寶物,吐谷渾易可並吞。帝由是甘心,將通西域,四夷經略,咸以委之。轉民部侍郎,未視事,遷黃門侍郎。帝復令矩往張掖,引致西蕃,至者十餘國。大業三年,帝有事於恆岳,咸
 來助祭。帝將巡河右,復令矩往敦煌。矩遣使說高昌王麴伯雅及伊吾吐屯設等,啖以厚利,導使入朝。及帝西巡,次燕支山,高昌王、伊吾設等及西蕃胡二十七國,謁於道左。皆令佩金玉,被錦罽,焚香奏樂,歌儛喧噪。復令武威、張掖士女盛飾縱觀,騎乘填咽,周亙數十里,以示中國之盛。帝見而大悅。竟破吐谷渾,拓地數千里,並遣兵戍之。每歲委輸巨億萬計,諸蕃懾懼,朝貢相續。帝謂矩有綏懷之略,進位銀青光祿大夫。其冬,帝至東都,矩以蠻夷朝貢者多,諷帝令都下大戲。征四方奇技異藝,陳於端門街,衣錦綺、珥金翠者以十數萬。又勒百官及
 民士女列坐棚閣而縱觀焉。皆被服鮮麗,終月乃罷。又令三市店肆皆設帷帳,盛列酒食,遣掌蕃率蠻夷與民貿易,所至之處,悉令邀延就坐,醉飽而散。蠻夷嗟嘆,謂中國為神仙。帝稱其至誠,顧謂宇文述、牛弘曰:「裴矩大識朕意,凡所陳奏,皆朕之成算。未發之頃,矩輒以聞。自非奉國用心,孰能若是!」帝遣將軍薛世雄城伊吾,令矩共往經略。矩諷諭西域諸國曰:「天子為蕃人交易懸遠,所以城伊吾耳。」咸以為然,不復來競。及還,賜錢四十萬。矩又白狀,令反間射匱,潛攻處羅,語在《突厥傳》。後處羅為射匱所迫,竟隨使者入朝。帝大悅,賜矩以貂裘及西
 域珍器。



 從帝巡於塞北,幸啟民帳。時高麗遣使先通於突厥,啟民不敢隱,引之見帝。



 矩因奏狀曰:「高麗之地,本孤竹國也。周代以之封於箕子,漢世分為三郡,晉氏亦統遼東。今乃不臣,別為外域,故先帝疾焉,欲征之久矣。但以楊諒不肖,師出無功。當陛下之時,安得不事,使此冠帶之境,仍為蠻貊之鄉乎?今其使者朝於突厥,親見啟民,合國從化,必懼皇靈之遠暢,慮後伏之先亡。脅令入朝,當可致也。」



 帝曰:「如何?」矩曰:「請面詔其使,放還本國,遣語其王,令速朝觀。不然者,當率突厥,即日誅之。」帝納焉。高元不用命,始建征遼之策。王師臨遼,以本官領武
 賁郎將。明年,復從至遼東。兵部侍郎斛斯政亡入高麗,帝令矩兼掌兵事。以前後渡遼之役,進位右光祿大夫。於時皇綱不振,人皆變節,左翊衛大將軍宇文述、內史侍郎虞世基等用事,文武多以賄聞。唯矩守常,無贓穢之響,以是為世所稱。



 還至涿郡,帝以楊玄感初平,令矩安集隴右。因之會寧,存問曷薩那部落,遣闕達度設寇吐谷渾,頻有虜獲,部落致富。還而奏狀,帝大賞之。後從師至懷遠鎮,詔護北蕃軍事。矩以始畢可汗部眾漸盛,獻策分其勢,將以宗女嫁其弟叱吉設,拜為南面可汗。叱吉不敢受,始畢聞而漸怨。矩又言於帝曰:「突厥本淳,
 易可離間,但由其內多有群胡,盡皆桀黠,教導之耳。臣聞史蜀胡悉尤多奸計,幸於始畢,請誘殺之。」帝曰:「善。」矩因遣人告胡悉曰:「天子大出珍物,今在馬邑,欲共蕃內多作交關。若前來者,即得好物。」胡悉貪而信之,不告始畢,率其部落,盡驅六畜,星馳爭進,冀先互市。矩伏兵馬邑下,誘而斬之。詔報始畢曰:「史蜀胡悉忽領部落走來至此,雲背可汗,請我容納。突厥既是我臣,彼有背叛,我當共殺。



 今已斬之,故令往報。」始畢亦知其狀,由是不朝。十一年,帝北巡狩,始畢率騎數十萬,圍帝於雁門。詔令矩與虞世基每宿朝堂,以待顧問。及圍解,從至東都。



 屬
 射匱可汗遣其猶子,率西蕃諸胡朝貢,詔矩宴接之。



 尋從幸江都宮。時四方盜賊蜂起,郡縣上奏者不可勝計。矩言之,帝怒,遣矩詣京師接候蕃客,以疾不行。及義兵入關,帝令虞世基就宅問矩方略。矩曰:「太原有變,京畿不靜,遙為處分,恐失事機。唯願鑾輿早還,方可平定。」矩復起視事。俄而驍衛大將軍屈突通敗問至,矩以聞,帝失色。矩素勤謹,未嘗忤物,又見天下方亂,恐為身禍,其待遇人,多過其所望,故雖至廝役,皆得其歡心。時從駕驍果數有逃散,帝憂之,以問矩。矩答曰:「方今車駕留此,已經二年。驍果之徒,盡無家口,人無匹合,則不能久安。
 臣請聽兵士於此納室。」帝大喜曰:「公定多智,此奇計也。」因令矩檢校,為將士等娶妻。矩召江都境內寡婦及未嫁女,皆集宮監,又召將帥及兵等恣其所取。因聽自首,先有奸通婦女及尼、女冠等,並即配之。由是驍果等悅,咸相謂曰:「裴公之惠也。」



 宇文化及之亂,矩晨起將朝,至坊門,遇逆黨數人,控矩馬詣孟景所。賊皆曰:「不關裴黃門。」既而化及從百餘騎至,矩迎拜,化及慰諭之。令矩參定儀注,推秦王子浩為帝,以矩為侍內,隨化及至河北。及僭帝位,以矩為尚書右僕射,加光祿大夫,封蔡國公,為河北道安撫大使。及宇文氏敗,為竇建德所獲,以矩
 隋代舊臣,遇之甚厚。復以為吏部尚書,尋轉尚書右僕射,專掌選事。建德起自群盜,未有節文,矩為制定朝儀。旬月之間,憲章頗備,擬於王者。建德大悅,每諮訪焉。



 及建德渡河討孟海公,矩與曹旦等於洺州留守。建德敗於武牢。群帥未知所屬,曹旦長史李公淹、大唐使人魏徵等說旦及齊善行令歸順。旦等從之,乃令矩與徵、公淹領旦及八璽,舉山東之地歸於大唐。授左庶子,轉詹事、民部尚書。



 史臣曰:世基初以雅淡著名,兼以文華見重,亡國羈旅,特蒙任遇。參機衡之職,預帷幄之謀,國危未嘗思安,君
 昏不能納諫。方更鬻官賣獄,黷貨無厭,顛隕厥身,亦其所也。裴蘊素懷奸險,巧於附會,作威作福,唯利是視,滅亡之禍,其可免乎?裴矩學涉經史,頗有幹局,至於恪勤匪懈,夙夜在公,求諸古人,殆未之有。與聞政事,多歷歲年,雖處危亂之中,未虧廉謹之節,美矣。然承望風旨,與時消息,使高昌入朝,伊吾獻地,聚糧且末,師出玉門,關右騷然,頗亦矩之由也。



\end{pinyinscope}