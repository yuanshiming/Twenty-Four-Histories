\article{卷六十列傳第二十五}

\begin{pinyinscope}

 崔仲方崔仲方,字不齊,博陵安平人也。祖孝芬,魏荊州刺史。父宣猷,周小司徒。



 仲方少好讀書,有文武才幹。年十五,周太祖見而異之,令與諸子同就學。時高祖亦在其中,由是與高祖少相款密。後以明經為晉公宇文護參軍事,尋轉記室,遷司玉大夫,與斛斯徵、柳敏等同修禮律。後以軍功,授平東將軍、銀青光祿大夫,賜爵石城縣男,邑
 三百戶。時武帝陰有滅齊之志,仲方獻二十策,帝大奇之。後與少內史趙芬刪定格式。尋從帝攻晉州,齊之亞將崔景嵩請為內應,仲方與段文振等登城應接,遂下晉州,語在《文振傳》。又令仲方說翼城等四城,下之。授儀同,進爵範陽縣侯。後以行軍長史從郯公王軌擒陳將吳明徹於呂梁,仲方計策居多。宣帝嗣位,為少內史,奉使淮南而還。



 會帝崩,高祖為丞相,與仲方相見,握手極歡,仲方亦歸心焉。其夜上便宜十八事,高祖並嘉納之。又見眾望有歸,陰勸高祖應天受命,高祖從之。及受禪,上召仲方與高熲議正朔服色事。仲方曰:「晉為金行,後
 魏為水,周為木。皇家以火承木,得天之統。又聖躬載誕之初,有赤光之瑞,車服旗牲,並宜用赤。」又勸上除六官,請依漢、魏之舊。上皆從之。進位上開府,尋轉司農少卿,進爵安固縣公。



 令發丁三萬,於朔方、靈武築長城,東至黃河,西拒綏州,南至勃出嶺,綿亙七百里。明年,上復令仲方發丁十五萬,於朔方已東緣邊險要築數十城,以遏胡寇。



 丁父艱去職。未期,起為虢州刺史。上書論取陳之策曰:臣謹案晉太康元年歲在庚子,晉武平吳,至今開皇六年,歲次丙午,合三百七載。《春秋寶乾圖》云:「王者三百年一蠲法。」今年三百之期,可謂備矣。陳氏草竊,起於
 丙子,至今丙午,又子午為沖,陰陽之忌。昔史趙有言曰:「陳,顓頊之族,為水,故歲在鶉火以滅。」又云:「周武王克商,封胡公滿於陳。」至魯昭公九年,陳災,裨灶曰:「歲五及鶉火而後陳亡,楚克之。」楚,祝融之後也,為火正,故復滅陳。陳承舜後,舜承顓頊,雖太歲左行,歲星右轉,鶉火之歲,陳族再亡,戊午之年,媯虞運盡。語跡雖殊,考事無別。皇朝五運相承,感火德而王,國號為隋,與楚同分。楚是火正,午為鶉火,未為鶉首,申為實沉,酉為大梁。既當周、秦、晉、趙之分,若當此分發兵,將得歲之助,以今量古,陳滅不疑。臣謂午未申酉,並是數極。蓋聞天時不如地利,地
 利不如人和,況主聖臣良,兵強國富,動植回心,人神葉契。陳既主昏於上,民讟於下,險無百二之固,眾非九國之師。



 夏癸、殷辛尚不能立,獨此島夷而稽天討!伏度朝廷自有宏謨,但芻蕘所見,冀申螢爝。今唯須武昌已下,蘄、和、滁、方、吳、海等州更帖精兵,密營渡計。益、信、襄、荊、基、郢等州速造舟楫,多張形勢,為水戰之具。蜀、漢二江,是其上流,水路沖要,必爭之所。賊雖於流頭、荊門、延州、公安、巴陵、隱磯、夏首、蘄口、盆城置船,然終聚漢口、峽口,以水戰大決。若賊必以上流有軍,令精兵赴援者,下流諸將即須擇便橫渡。如擁眾自衛,上江水軍鼓行以前。雖
 恃九江五湖之險,非德無以為固,徒有三吳、百越之兵,無恩不能自立。



 上覽而大悅,轉基州刺史,徵入朝。仲方因面陳經略,上善之,賜以禦袍褲,並雜彩五百段,進位開府而遣之。及大舉伐陳,以仲方為行軍總管,率兵與秦王會。



 及陳平,坐事免。未幾,復位。後數載,轉會州總管。時諸羌猶未賓附,詔令仲方擊之,與賊三十餘戰,紫祖、四鄰、望方、涉題、乾碉、小鐵圍山、白男王、弱水等諸部悉平。賜奴婢一百三十口,黃金三十斤,雜物稱是。



 仁壽初,授代州總管,在職數年,被徵入朝。會上崩,漢王諒餘黨據呂州不下,煬帝令周羅攻之,中流矢卒,乃令仲方
 代總其眾,月餘拔之。進位大將軍,拜民部尚書,尋轉禮部尚書。後三載,坐事免。尋為國子祭酒,轉太常卿。朝廷以其衰老,出拜上郡太守。未幾,以母憂去職。歲餘,起為信都太守,上表乞骸骨,優詔許之。尋卒於家,時年七十六。子民壽,官至定陶令。



 於仲文兄凱從父弟璽於仲文,字次武,建平公義之兄子。父實,周大左輔、燕國公。仲文少聰敏,髫齔就學,耽閱不倦。其父異之曰:「此兒必興吾宗矣。」九歲,嘗於雲陽宮見周太祖,太祖問曰:「聞兒好讀書,書有何事?」仲文對曰:「資父事君,忠孝而已。」



 太
 祖甚嗟嘆之。其後就博士李祥受《周易》、《三禮》。略通大義。及長,倜儻有大志,氣調英拔,當時號為名公子。起家為趙王屬,尋遷安固太守。有任、杜兩家各失牛,後得一牛,兩家俱認,州郡久不能決。益州長史韓伯俊曰:「於安固少聰察,可令決之。」仲文對曰:「此易解耳。」於是令二家各驅牛群至,乃放所認者,遂向任氏群中。又陰使人微傷其牛,任氏嗟惋,杜家自若。仲文於是訶詰杜氏,杜氏服罪而去。始州刺史屈突尚,宇文護之黨也,先坐事下獄,無敢繩者。仲文至郡窮治,遂竟其獄。蜀中為之語曰:「明斷無雙有於公,不避強御有次武。」未幾,徵為御正下大夫,
 封延壽郡公,邑三千五百戶。數從征伐,累勛授儀同三司。宣帝時,為東郡太守。



 高祖為丞相,尉迥作亂,遣將檀讓收河南之地。復使人誘致仲文,仲文拒之。



 迥怒其不同己,遣儀同宇文威攻之。仲文迎擊,大破威眾,斬首五百餘級。以功授開府。迥又遣其將宇文胄渡石濟,宇文威、鄒紹自白馬,二道俱進,復攻仲文。賊勢逾盛,人情大駭,郡人赫連僧伽、敬子哲率眾應迥。仲文自度不能支,棄妻子,將六十餘騎,開城西門,潰圍而遁。為賊所追,且戰且行,所從騎戰死者十七八。



 仲文僅而獲免,達於京師。迥於是屠其三子一女。高祖見之,引入臥內,為之下
 泣。



 賜彩五百段,黃金二百兩,進位大將軍,領河南道行軍總管。給以鼓吹,馳傳詣洛陽發兵,以討檀讓。時韋孝寬拒迥於永橋,仲文詣孝寬有所計議。時總管宇文忻頗有自疑之心,因謂仲文曰:「公新從京師來,觀執政意何如也?尉迥誠不足平,正恐事寧之後,更有藏弓之慮。」仲文懼忻生變,因謂之曰:「丞相寬仁大度,明識有餘,茍能竭誠,必心無貳。仲文在京三日,頻見三善,以此為觀,非尋常人也。」



 忻曰:「三善如何?」仲文曰:「有陳萬敵者,新從賊中來,即令其弟難敵召募鄉曲,從軍討賊。此其有大度一也。上士宋謙,奉使勾檢,謙緣此別求他罪。丞相責
 之曰:『入網者自可推求,何須別訪,以虧大體。』此其不求人私二也。言及仲文妻子,未嘗不潸泫。此其有仁心三也。」忻自此遂安。



 仲文軍至汴州之東倪塢,與迥將劉子昂、劉浴德等相遇,進擊破之。軍次蓼堤,去梁郡七里,讓擁眾數萬,仲文以羸師挑戰。讓悉眾來拒,仲文偽北,讓軍頗驕。



 於是遣精兵左右翼擊之,大敗讓軍,生獲五千餘人,斬首七百級。進攻梁郡,迥守將劉子寬棄城遁走。仲文追擊,擒斬數千人,子寬僅以身免。初,仲文在蓼堤,諸將皆曰:「軍自遠來,士馬疲敝,不可決勝。」仲文令三軍趣食,列陣大戰。既而破賊,諸將皆請曰:「前兵疲不可交
 戰,竟而克勝,其計安在?」仲文笑曰:「吾所部將士皆山東人,果於速進,不宜持久。乘勢擊之,所以制勝。」諸將皆以為非所及也。進擊曹州,獲迥所署刺史李仲康及上儀同房勁。檀讓以餘眾屯城武,別將高士儒以萬人屯永昌。仲文詐移書州縣曰:「大將軍至,可多積粟。」讓謂仲文未能卒至,方槌牛享士。仲文知其怠,選精騎襲之,一日便至,遂拔城武。迥將席毗羅,眾十萬,屯於沛縣,將攻徐州。其妻子在金鄉。仲文遣人詐為毗羅使者,謂金鄉城主徐善凈曰:「檀讓明日午時到金鄉,將宣蜀公令,賞賜將士。」金鄉人謂為信然,皆喜。仲文簡精兵,偽建迥旗幟,
 倍道而進。善凈望見仲文軍且至,以為檀讓,乃出迎謁。仲文執之,遂取金鄉。諸將多勸屠之,仲文曰:「此城是毗羅起兵之所,當寬其妻子,其兵可自歸。如即屠之,彼望絕矣。」眾皆稱善。於是毗羅恃眾來薄官軍,仲文背城結陣,去軍數里,設伏於麻田中。兩陣才合,伏兵發,俱曳柴鼓噪,塵埃張天。毗羅軍大潰,仲文乘之,賊皆投洙水而死,為之不流。獲檀讓,檻送京師,河南悉平。毗羅匿滎陽人家,執斬之,傳首闕下。勒石紀功,樹於泗上。



 入朝京師,高祖引入臥內,宴享極歡。賜雜彩千餘段,妓女十人,拜柱國、河南道大行臺。屬高祖受禪,不行。未幾,其叔父太
 尉翼坐事下獄,仲文亦為吏所簿,於獄中上書曰:臣聞春生夏長,天地平分之功,子孝臣誠,人倫不易之道。曩者尉迥逆亂,所在影從。臣任處關河,地居沖要,嘗膽枕戈,誓以必死。迥時購臣位大將軍、邑萬戶。臣不顧妻子,不愛身命,冒白刃,潰重圍,三男一女,相繼淪沒,披露肝膽,馳赴闕庭。蒙陛下授臣以高官,委臣以兵革。於時河南兇寇,狼顧鴟張,臣以羸兵八千,掃除氛昆。摧劉寬於梁郡,破檀讓於蓼堤,平曹州,復東郡、安城、武定、永昌,解亳州圍,殄徐州賊。席毗十萬之眾,一戰土崩,河南蟻聚之徒,應時戡定。



 當群兇問鼎之際,黎元乏主之辰,臣第
 二叔翼先在幽州,總馭燕、趙,南鄰群寇,北捍旄頭,內外安撫,得免罪戾。臣第五叔智建斿黑水,與王謙為鄰,式遏蠻陬,鎮綏蜀道。臣兄顗作牧淮南,坐制勍敵,乘機剿定,傳首京師。王謙竊據二江,叛換三蜀。臣第三叔義受脤廟庭,龔行天討。自外父叔兄弟,皆當文武重寄,或銜命危難之間,或侍衛鉤陳之側,合門誠款,冀有可明。伏願下泣辜之恩,降雲雨之施,追草昧之始,錄涓滴之功,則寒灰更然,枯骨生肉,不勝區區之至,謹冒死以聞。



 上覽表,並翼俱釋之。



 未幾,詔仲文率兵屯白狼塞以備胡。明年,拜行軍元帥,統十二總管以擊胡。



 出服遠鎮,遇虜,
 破之,斬首千餘級,六畜巨萬計。於是從金河出白道,遣總管辛明瑾、元滂、賀蘭志、呂楚、段諧等二萬人出盛樂道,趨那頡山。至護軍川北,與虜相遇,可汗見仲文軍容齊肅,不戰而退。仲文率精騎五千,逾山追之,不及而還。



 上以尚書文簿繁雜,吏多奸計,令仲文勘錄省中事。其所發擿甚多,上嘉其明斷,厚加勞賞焉。上每憂轉運不給,仲文請決渭水,開漕渠。上然之,使仲文總其事。



 及伐陳之役,拜行軍總管,以舟師自章山出漢口。陳郢州刺史荀法尚、魯山城主誕法澄、鄧沙彌等請降,秦王俊皆令仲文以兵納之。高智慧等作亂江南,復以行軍總管
 討之。時三軍乏食,米粟踴貴,仲文私糶軍糧,坐除名。明年,復官爵,率兵屯馬邑以備胡。數旬而罷。



 晉王廣以仲文有將領之才,每常屬意,至是奏之,乃令督晉王軍府事。後突厥犯塞,晉王為元帥,以仲文將前軍,大破賊而還。仁壽初,拜太子右衛率。煬帝即位,遷右翊衛大將軍,參掌文武選事。從帝討吐谷渾,進位光祿大夫,甚見親幸。



 遼東之役,仲文率軍指樂浪道。軍次烏骨城,仲文簡羸馬驢數千,置於軍後。既而率眾東過,高麗出兵掩襲輜重,仲文回擊,大破之。至鴨綠水,高麗將乙支文德詐降,來入其營。仲文先奉密旨,若遇高元及文德者,必擒
 之。至是,文德來,仲文將執之。時尚書右丞劉士龍為慰撫使,固止之。仲文遂舍文德。尋悔,遣人紿文德曰:「更有言議,可復來也。」文德不從,遂濟。仲文選騎渡水追之,每戰破賊。



 文德遺仲文詩曰:「神策究天文,妙算窮地理。戰勝功既高,知足願云止。」仲文答書諭之,文德燒柵而遁。時宇文述以糧盡欲還,仲文議以精銳追文德,可以有功。



 述固止之,仲文怒曰:「將軍仗十萬之眾,不能破小賊,何顏以見帝!且仲文此行也,固無功矣。」述因厲聲曰:「何以知無功?」仲文曰:「昔周亞夫之為將也,見天子軍容不變。此決在一人,所以功成名遂。今者人各其心,何以赴
 敵!」初,帝以仲文有計畫,令諸軍咨稟節度,故有此言。由是述等不得已而從之,遂行。東至薩水,宇文述以兵餒退歸,師遂敗績。帝以屬吏,諸將皆委罪於仲文。帝大怒,釋諸將,獨系仲文。仲文憂恚發病,困篤方出之,卒於家,時年六十八。撰《漢書刊繁》三十卷、《略覽》三十卷。有子九人,欽明最知名。



 顗字元武,身長八尺,美須眉。周大塚宰宇文護見而器之,妻以季女。尋以父勛賜爵新野郡公,邑三千戶。授大都督,遷車騎大將軍、儀同三司。其後累以軍功,授上開府,歷左、右宮伯,郢州刺史。大象中,以水軍總管從韋孝
 寬經略淮南。顗率開府元紹貴、上儀同毛猛等,以舟師自潁口入淮。陳防主潘深棄柵而走,進與孝寬攻拔壽陽。復引師圍硤石,守將許約懼而降,顗乃拜東廣州刺史。



 尉迥之反也,時總管趙文表與顗素不協,顗將圖之,因臥閣內,詐得心疾,謂左右曰:「我見兩三人至我前者,輒大驚,即欲斫之,不能自制也。」其有賓客候問者,皆令去左右。顗漸稱危篤,文表往候之,令從者至大門而止,文表獨至顗所。



 顗醿然而起,抽刀斫殺之,」因唱言曰:「文表與尉迥通謀,所以斬之。」其麾下無敢動者。時高祖以尉迥未平,慮顗復生邊患,因而勞勉之,即拜吳州總管。
 陳將錢茂和率數千人襲江陽,顗逆擊走之。陳復遣將陳紀、周羅、燕合兒等襲顗,顗拒之而退,賜彩數百段。



 高祖受禪,文表弟詣闕稱兄無罪。上令案其事,太傅竇熾等議顗當死。上以門著勛績,特原之,貶為開府。後襲爵燕國公,邑萬六千戶。尋以疾免。開皇七年,拜澤州刺史。數年,免職,卒於家。子世虔嗣。



 璽字伯符。父翼,仕周為上柱國、幽州總管、任國公。高祖為丞相,尉迥作亂,遣人誘翼。翼鎖其使,送之長安,高祖甚悅。及高祖受禪,翼入朝,上為之降榻,握手極歡。數日,拜為太尉。歲餘,卒,謚曰穆。



 璽少有器幹,仕周,起家右侍
 上士。尋授儀同,領右羽林,遷少胥附。武帝時,從齊王憲破齊師於洛陽,以功賜爵豐寧縣子,邑五百戶。尋從帝平齊,加開府,改封黎陽縣公,邑千二百戶,授職方中大夫。及宣帝嗣位,轉右勛曹中大夫。尋領右忠義。高祖為丞相,加上開府。及受禪,進位大將軍,拜汴州刺史,甚有能名。上聞而善之,優詔褒揚,賜帛百匹。尋加上大將軍,進爵郡公。轉邵州刺史,在州數年,甚有恩惠。後檢校江陵總管,州人張願等數十人,詣闕上表,請留璽。上嘉嘆良久,令還邵州,父老相賀。尋遷洛州刺史,復為熊州刺史,並有惠政。以疾徵還京師。仁壽末,卒於家,謚曰靜。有
 子志本。



 段文振段文振,北海期原人也。祖壽,魏滄州刺史。父威,周洮、河、甘、渭四州刺史。文振少有膂力,膽氣過人,性剛直,明達時務。初為宇文護親信,護知其有乾用。擢授中外府兵曹。後武帝攻齊海昌王尉相貴於晉州,其亞將侯子欽、崔景嵩為內應。文振杖槊登城,與崔仲方等數十人先登。文振隨景嵩至相貴所,拔佩刀劫之,相貴不敢動,城遂下。帝大喜,賜物千段。進拔文侯、華谷、高壁三城,皆有力焉。



 及攻並州,陷東門而入,齊安德王延宗懼而出降。
 錄前後勛,將拜高秩,以讒毀獲譴,因授上儀同,賜爵襄國縣公,邑千戶。進平鄴都,又賜綺羅二千匹。後從滕王逌擊稽胡,破之。歷相州別駕、揚州總管長史。入為天官都上士,從韋孝寬經略淮南。



 俄而尉迥作亂,時文振老母妻子俱在鄴城,迥遣人誘之,文振不顧,歸於高祖。



 高祖引為丞相掾,領宿衛驃騎。司馬消難之奔陳也,高祖令文振安集淮南,還除衛尉少卿,兼內史侍郎。尋以行軍長史從達奚震討叛蠻,平之,加上開府。歲餘,遷鴻臚卿。衛王爽北征突厥,以文振為長史,坐勛簿不實免官。後為石、河二州刺史,甚有威惠,遷蘭州總管,改封龍崗
 縣公。突厥犯塞,以行軍總管擊破之,逐北至居延塞而還。九年,大舉伐陳,以文振為元帥秦王司馬,別領行軍總管。及平江南,授揚州總管司馬。尋轉並州總管司馬,以母憂去職。未幾,起令視事,固辭不許。



 後數年,拜雲州總管,尋為太僕卿。十九年,突厥犯塞,文振以行軍總管拒之,遇達頭可汗於沃野,擊破之。文振先與王世積有舊,初,文振北征,世積遺以駝馬。



 比還,世積以罪被誅,文振坐與交關,功遂不錄。明年,率眾出靈州道以備胡,無虜而還。越巂蠻叛,文振擊平之,賜奴婢二百口。仁壽初,嘉州獠作亂,文振以行軍總管討之。引軍出谷間,為賊
 所襲,前後阻險,不得相救,軍遂大敗。文振復收散兵,擊其不意,竟破之。文振性素剛直,無所降下,初,軍次益州,謁蜀王秀,貌頗不恭,秀甚銜之,及此,奏文振師徒喪敗。右僕射蘇威與文振有隙,因而譖之,坐是除名。及秀廢黜,文振上表自申理,高祖慰諭之,授大將軍。尋拜靈州總管。



 煬帝即位,徵為兵部尚書,待遇甚重。從征吐谷渾,文振督兵屯雪山,連營三百餘里,東接楊義臣,西連張壽,合圍渾主於覆袁川。以功進位右光祿大夫。帝幸江都,以文振行江都郡事。文振見高祖時容納突厥啟民居於塞內,妻以公主,賞賜重疊;及大業初,恩澤彌厚。文
 振以狼子野心,恐為國患,乃上表曰:「臣聞古者遠不間近,夷不亂華,周宣外攘戎狄,秦帝築城萬里,蓋遠圖良算,弗可忘也。竊見國家容受啟民,資其兵食,假以地利。如臣愚計,竊又未安。何則?夷狄之性,無親而貪,弱則歸投,強則反噬,蓋其本心也。臣學非博覽,不能遠見,且聞晉朝劉曜,梁代侯景,近事之驗,眾所共知。以臣量之,必為國患。如臣之計,以時喻遣,令出塞外。然後明設烽候,緣邊鎮防,務令嚴重,此乃萬歲之長策也。」時兵曹郎斛斯政專掌兵事,文振知政險薄,不可委以機要,屢言於帝,帝並弗納。



 及遼東之役,授左候衛大將軍,出南蘇道。
 在道疾篤,上表曰:「臣以庸微,幸逢聖世,濫蒙獎擢,榮冠儕伍。而智能無取,叨竊已多,言念國恩,用忘寢食。



 常思效其鳴吠,以報萬分,而攝養乖方,疾患遂篤。抱此深愧,永歸泉壤,不勝餘恨,輕陳管穴。竊見遼東小醜,未服嚴刑,遠降六師,親勞萬乘。但夷狄多詐,深須防擬,口陳降款,心懷背叛,詭伏多端,勿得便受。水潦方降,不可淹遲,唯願嚴勒諸軍,星馳速發,水陸俱前,出其不意,則平壤孤城,勢可拔也。若傾其本根,餘城自克。如不時定,脫遇秋霖,深為艱阻,兵糧又竭,強敵在前,靺軻出後,遲疑不決,非上策也。」後數日,卒於師。帝省表,悲嘆久之,贈光祿
 大夫、尚書右僕射、北平侯,謚曰襄。賜物一千段,粟麥二千石,威儀鼓吹,送至墓所。有子十人。



 長子詮,官至武牙郎將。次綸,少以俠氣聞。文振弟文操,大業中,為武賁郎將,性甚剛嚴。帝令督秘書省學士。時學士頗存儒雅,文操輒鞭撻之,前後或至千數,時議者鄙之。



 史臣曰:仲方兼資文武,雅有籌算,伐陳之策,信為深遠矣。聲績克舉,夫豈徒言哉!仲文博涉書記,以英略自許,尉迥之亂,遂立功名。自茲厥後,屢當推轂。



 遼東之役,實喪師徒。斯乃大樹將顛,蓋亦非戰人之罪也。文振少以膽略見重,終懷壯夫之志,時進讜言,頻稱諒直。其取高
 位厚秩,良有以也。



\end{pinyinscope}