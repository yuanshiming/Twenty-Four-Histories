\article{卷二十四志第十九 食貨}

\begin{pinyinscope}

 王
 者量地以制邑,度地以居人,總土地所生,料山澤之利,式遵行令,敬授人時,農商趣向,各本事業。《書》稱懋遷有無,言穀貨流通,咸得其所者也。《周官》太府掌九貢九賦之法,王之經用,各有等差。所謂取之以道,用之有節,故能養百官之拯,勖戰士之功,救天災,服方外,活國安人之大經也。爰自軒、頊,至於堯、舜,皆因其所利而勸之,
 因其所欲而化之。不奪其時,不窮其力,輕其征,薄其賦,此五帝三皇不易之教也。古語曰:「善為人者,愛其力而成其財。」若使之不以道,斂之如不及,財盡則怨,力盡則叛。昔禹制九等而康歌興,周人十一而頌聲作。於是東周遷洛,諸侯不軌,魯宣初稅畝,鄭產為丘賦,先王之制,靡有孑遺。秦氏起自西戎,力正天下,驅之以刑罰,棄之以仁恩,以太半之收,長城絕於地脈,以頭會之斂,屯戍窮於嶺外。漢高祖承秦凋敝,十五稅一,中元繼武,府稟彌殷。世宗得之,用成雄侈,開邊擊胡,蕭然咸罄。宮宇捫於天漢,巡游跨於海表,旱歲除道,兇年嘗秣,戶口以之
 減半,盜賊以之公行。於是譎詭賦稅,異端俱起,賦及童齔,算至船車。光武中興,聿遵前事,成賦單薄,足稱經遠。靈帝開鴻都之榜,通賣官之路,公卿州郡,各有等差。漢之常科,土貢方物,帝又遣先輸中署,名為導行,天下賄成,人受其敝。自魏、晉二十一帝,宋、齊十有五主,雖用度有眾寡,租賦有重輕,大抵不能傾人產業,道闕政亂。



 隋文帝既平江表,天下大同,躬先儉約,以事府帑。開皇十七年,戶口滋盛,中外倉庫,無不盈積。所有賚給,不逾經費,京司帑屋既充,積於廓廡之下,高祖遂停此年正賦,以賜黎元。煬皇嗣守鴻基,國家殷富,雅愛宏玩,肆情方
 騁,初造東都,窮諸巨麗。帝昔居籓翰,親平江左,兼以梁、陳曲折,以就規摹。曾雉逾芒,浮橋跨洛,金門象闕,咸竦飛觀,頹巖塞川,構成雲綺,移嶺樹以為林藪,包芒山以為苑囿。長城御河,不計於人力,運驢武馬,指期於百姓,天下死於役而家傷於財。既而一討渾庭,三駕遼澤,天子親伐,師兵大舉,飛糧輓秣,水陸交至。疆埸之所傾敗,勞敝之所殂殞,雖復太半不歸,而每年興發,比屋良家之子,多赴於邊陲,分離哭泣之聲,連響於州縣。老弱耕稼,不足以救饑餒,婦工紡織,不足以贍資裝。九區之內,鸞和歲動,從行宮掖,常十萬人,所有供須,皆仰州縣。租
 賦之外,一切徵斂,趣以周備,不顧元元,吏因割剝,盜其太半。遐方珍膳,必登庖廚,翔禽毛羽,用為玩飾,買以供官,千倍其價。人愁不堪,離棄室宇,長吏叩扉而達曙,猛犬迎吠而終夕。自燕趙跨於齊韓,江淮入於襄鄧,東周洛邑之地,西秦隴山之右,僭偽交侵,盜賊充斥。宮觀鞠為茂草,鄉亭絕其煙火,人相啖食,十而四五。



 關中癘疫,炎旱傷稼,代王開永豐之粟,以振饑人,去倉數百里,老幼雲集。吏在貪殘,官無攸次,咸資鏹貨,動移旬月,頓臥墟野,欲返不能,死人如積,不可勝計。雖復皇王撫運,天祿有終,而隋氏之亡,亦由於此。



 馬遷為《平準書》,班固述《
 食貨志》,上下數千載,損益粗舉。自此史官,曾無概見。夫厥初生人,食貨為本。聖王割廬井以業之,通貨財以富之。富而教之,仁義以之興,貧而為盜,刑罰不能止。故為《食貨志》,用編前書之末云。



 晉自中原喪亂,元帝寓居江左,百姓之自拔南奔者,並謂之僑人。皆取舊壤之名,僑立郡縣,往往散居,無有土著,而江南之俗,火耕水耨,土地卑濕,無有蓄積之資。諸蠻陬俚洞,沾沐王化者,各隨輕重,收其賧物,以裨國用。又嶺外酋帥,因生口翡翠明珠犀象之饒,雄於鄉曲者,朝廷多因而署之,以收其利。歷宋、齊、梁、陳,皆因而不改。
 其軍國所須雜物,隨土所出,臨時折課市取,乃無恆法定令。



 列州郡縣,制其任土所出,以為征賦。其無貫之人,不樂州縣編戶者,謂之浮浪人,樂輸亦無定數,任量,準所輸,終優於正課焉。都下人多為諸王公貴人左右、佃客、典計、衣食客之類,皆無課役。官品第一第二,佃客無過四十戶。第三品三十五戶。



 第四品三十戶。第五品二十五戶。第六品二十戶。第七品十五戶。第八品十戶。第九品五戶。其佃穀皆與大家量分。其典計,官品第一第二,置三人。第三第四,置二人。第五第六及公府參軍、殿中監、監軍、長史、司馬、部曲督、關外侯、材官、議郎已上,一
 人。皆通在佃客數中。官品第六已上,並得衣食客三人。第七第八二人。第九品及舉輦、跡禽、前驅、由基強弩司馬、羽林郎、殿中冗從武賁、殿中武賁、持椎斧武騎武賁、持鈒冗從武賁、命中武賁武騎,一人。客皆注家籍。其課,丁男調布絹各二丈,絲三兩,綿八兩,祿絹八尺,祿綿三兩二分,租米五石,祿米二石。丁女並半之。男女年十六歲已上至六十,為丁。男年十六,亦半課,年十八正課,六十六免課。女以嫁者為丁,若在室者,年二十乃為丁。其男丁,每歲役不過二十日。又率十八人出一運丁役之。其田,畝稅米二斗。蓋大率如此。其度量,鬥則三斗當今一
 斗,稱則三兩當今一兩,尺則一尺二寸當今一尺。



 其倉,京都有龍首倉,即石頭津倉也,臺城內倉,南塘倉,常平倉,東、西太倉,東宮倉,所貯總不過五十餘萬。在外有豫章倉、釣磯倉、錢塘倉,並是大貯備之處。自餘諸州郡臺傳,亦各有倉。大抵自侯景之亂,國用常褊。京官文武,月別唯得廩食,多遙帶一郡縣官而取其祿秩焉。揚、徐等大州,比令、僕班。寧、桂等小州,比參軍班。丹陽、吳郡、會稽等郡,同太子詹事、尚書班。高涼、晉康等小郡,三班而已。大縣六班,小縣兩轉方至一班。品第既殊,不可委載。州郡縣祿米絹布絲綿,當處輸臺傳倉庫。若給刺史守令
 等,先準其所部文武人物多少,由敕而裁。凡如此祿秩,既通所部兵士給之,其家所得蓋少。諸王諸主,出閤就第婚冠所須,及衣裳服飾,並酒米魚鮭香油紙燭等,並官給之。王及主婿外祿者,不給,解任還京,仍亦公給云。



 魏自永安之後,政道陵夷,寇亂實繁,農商失業。官有征伐,皆權調於人,猶不足以相資奉,乃令所在迭相糾發,百姓愁怨,無復聊生。尋而六鎮擾亂,相率內徙,寓食於齊、晉之郊。齊神武因之,以成大業。魏武西遷,連年戰爭,河、洛之間,又並空竭。天平元年,遷都於鄴,出粟一百三十萬石,以振貧人。是時六坊之眾,從武帝而西者,不能
 萬人,餘皆北徙,並給常廩,春秋二時賜帛,以供衣服之費。常調之外,逐豐稔之處,折絹糴粟,以充國儲。於諸州緣河津濟,皆官倉貯積,以擬漕運。於滄、瀛、幽、青四州之境,傍海置鹽官,以煮鹽,每歲收錢,軍國之資,得以周贍。自是之後,倉廩充實,雖有水旱兇饑之處,皆仰開倉以振之。元象、興和之中,頻歲大穰,穀斛至九錢。是時法網寬弛,百姓多離舊居,闕於徭賦。神武乃命孫騰、高隆之分括無籍之戶,得六十餘萬。於是僑居者各勒還本屬,是後租調之入有加焉。及文襄嗣業,侯景北叛,河南之地。困於兵革。尋而侯景亂梁,乃命行臺辛術,略有淮南
 之地。其新附州郡,羈縻輕稅而已。



 及文宣受禪,多所創革。六坊之內徙者,更加簡練,每一人必當百人,任其臨陣必死,然後取之,謂之百保鮮卑。又簡華人之勇力絕倫者,謂之勇士,以備邊要。



 始立九等之戶,富者稅其錢,貧者役其力。北興長城之役,南有金陵之戰,其後南征諸將,頻歲陷沒,士馬死者以數十萬計。重以修創臺殿,所役甚廣,而帝刑罰酷濫,吏道因而成奸,豪黨兼並,戶口益多隱漏。舊制,未娶者輸半床租調,陽翟一郡,戶至數萬,籍多無妻。有司劾之,帝以為生事,由是奸欺尤甚。戶口租調,十亡六七。是時用度轉廣,賜與無節,府藏之
 積,不足以供。乃減百官之祿,撤軍人常廩,並省州郡縣鎮戍之職。又制刺史守宰行兼者,並不給幹,以節國之費用焉。



 天保八年,議徙冀、定、瀛無田之人,謂之樂遷,於幽州範陽寬鄉以處之。百姓驚擾。屬以頻歲不熟,米糴踴貴矣。廢帝乾明中,尚書左丞蘇珍芝議修石鱉等屯,歲收數萬石。自是淮南軍防,糧廩充足。孝昭皇建中,平州刺史嵇曄建議,開幽州督亢舊陂,長城左右營屯,歲收稻粟數十萬石,北境得以周贍。又於河內置懷義等屯,以給河南之費。自是稍止轉輸之勞。



 至河清三年定令,乃命人居十家為比鄰,五十家為閭里,百家為族黨。
 男子十八以上六十五已下為丁,十六已上十七已下為中,六十六已上為老,十五已下為小。



 率以十八受田,輸租調,二十充兵,六十免力役,六十六退田,免租調。



 京城四面,諸坊之外三十里內為公田。受公田者,三縣代遷、內執事官一品已下,逮於羽林武賁,各有差。其外畿郡,華人官第一品已下,羽林武賁已上,各有差。職事及百姓請墾田者,名為永業田。奴婢受田者,親王止三百人;嗣王止二百人;第二品嗣王已下及庶姓王,止一百五十人;正三品已上及皇宗,止一百人;七品已上,限止八十人;八品已下至庶人,限止六十人。奴婢限外不給田
 者,皆不輸。



 其方百里外及州人,一夫受露田八十畝,婦四十畝。奴婢依良人,限數與在京百官同。丁牛一頭,受田六十畝,限止四牛。又每丁給永業二十畝,為桑田。其中種桑五十根,榆三根,棗五根,不在還受之限。非此田者,悉入還受之分。土不宜桑者,給麻田,如桑田法。



 率人一床,調絹一疋,綿八兩,凡十斤綿中,折一斤作絲,墾租二石,義租五斗。奴婢各準良人之半。牛調二尺,墾租一斗,義租五升。墾租送臺,義租納郡,以備水旱。墾租皆依貧富為三梟。其賦稅常調,則少者直出上戶,中者及中戶,多者及下戶。上梟輸遠處,中梟輸次遠,下梟輸當州
 倉。三年一校焉。租入臺者,五百里內輸粟,五百里外輸米。入州鎮者,輸粟。人欲輸錢者,準上絹收錢。諸州郡皆別置富人倉。初立之日,準所領中下戶口數,得支一年之糧,逐當州穀價賤時,斟量割當年義租充入。穀貴,下價糶之;賤則還用所糶之物,依價糶貯。



 每歲春月,各依鄉土早晚,課人農桑。自春及秋,男十五已上,皆布田畝。桑蠶之月,婦女十五已上,皆營蠶桑。孟冬,刺史聽審邦教之優劣,定殿最之科品。



 人有人力無牛,或有牛無力者,須令相便,皆得納種。使地無遺利,人無游手焉。



 緣邊城守之地,堪墾食者,皆營屯田,署都使子使以統之。
 一子使當田五十頃,歲終考其所入,以論褒貶。



 是時頻歲大水,州郡多遇沉溺,穀價騰踴。朝廷遣使開倉,從貴價以糶之,而百姓無益,饑饉尤甚。重以疾疫相乘,死者十四五焉。至天統中,又毀東宮,造修文、偃武、隆基嬪嬙諸院,起玳瑁樓。又於游豫園穿池,周以列館,中起三山,構臺,以象滄海,並大修佛寺,勞役鉅萬計。財用不給,乃減朝士之祿,斷諸曹糧膳及九州軍人常賜以供之。武平之後,權幸並進,賜與無限,加之旱蝗,國用轉屈,乃料境內六等富人,調令出錢。而給事黃門侍郎顏之推奏請立關市邸店之稅,開府鄧長顒贊成之,後主大悅。於
 是以其所入,以供御府聲色之費,軍國之用不豫焉。



 未幾而亡。



 後周太祖作相,創制六官。載師掌任土之法,辨夫家田里之數,會六畜車乘之稽,審賦役斂弛之節,制畿疆修廣之域,頒施惠之要,審牧產之政。司均掌田里之政令。凡人口十已上,宅五畝;口九已上,宅四畝,口五已下,宅三畝。有室者,田百四十畝,丁者田百畝。司賦掌功賦之政令。凡人自十八以至六十有四,與輕癃者,皆賦之。其賦之法,有室者,歲不過絹一匹,綿八兩,粟五斛;丁者半之。其非桑土,有室者,布一匹,麻十斤;丁者又半之。豐年
 則全賦,中年半之,下年一之,皆以時征焉。若艱兇札,則不徵其賦。司役掌力役之政令。凡人自十八以至五十有九,皆任於役。豐年不過三旬,中年則二旬,下年則一旬。凡起徒役,無過家一人。其人有年八十者,一子不從役,百年者,家不從役。廢疾非人不養者,一人不從役。若兇札,又無力征。掌鹽掌四鹽之政令。一曰散鹽,煮海以成之;二曰監鹽,引池以化之;三曰形鹽,物地以出之;四曰飴鹽,於戎以取之。凡監鹽形鹽,每地為之禁,百姓取之,皆稅焉。司倉掌辨九穀之物,以量國用。國用足,即蓄其餘,以待兇荒;不足則止。餘用足,則以粟貸人。春頒之,
 秋斂之。



 閔帝元年,初除市門稅。及宣帝即位,復興人市之稅。武帝保定元年,改八丁兵為十二丁兵,率歲一月役。建德二年,改軍士為侍官,募百姓充之,除其縣籍。



 是後夏人半為兵矣。宣帝時,發山東諸州,增一月功為四十五日役,以起洛陽宮。



 並移相州六府於洛陽,稱東京六府。



 武帝保定二年正月,初於蒲州開河渠,同州開龍首渠,以廣溉灌。高祖登庸,罷東京之役,除人市之稅。是時尉迥、王謙、司馬消難相次叛逆,興師誅討,賞費鉅萬。及受禪,又遷都,發山東丁,毀造宮室。仍依周制,役丁為十二
 番,匠則六番。及頒新令。制人五家為保,保有長。保五為閭,閭四為族,皆有正。畿外置里正,比閭正,黨長比族正,以相檢察焉。男女三歲已下為黃,十歲已下為小,十七已下為中,十八已上為丁。丁從課役。六十為老,乃免。自諸王已下,至於都督,皆給永業田,各有差。多者至一百頃,少者至四十畝。其丁男、中男永業露田,皆遵後齊之制。並課樹以桑榆及棗。其園宅,率三口給一畝,奴婢則五口給一畝。丁男一床,租粟三石,桑土調以絹絁,麻土以布,絹絁以匹,加綿三兩。布以端,加麻三斤。單丁及僕隸各半之。未受地者皆不課。有品爵及孝子順孫義夫
 節婦,並免課役。京官又給職分田。一品者給田五頃。每品以五十畝為差,至五品,則為田三頃,六品二頃五十畝。其下每品以五十畝為差,至九品為一頃。外官亦各有職分田,又給公廨田,以供公用。



 開皇三年正月,帝入新宮。初令軍人以二十一成丁。減十二番每歲為二十日役。



 減調絹一疋為二丈。先是尚依周末之弊,官置酒坊收利,鹽池鹽井,皆禁百姓採用。



 至是罷酒坊,通鹽池鹽井與百姓共之,遠近大悅。是時突厥犯塞,吐谷渾寇邊,軍旅數起,轉輸勞敝。帝乃令朔州總管趙仲卿,於長城以北大興屯田,以實塞下。又於
 河西勒百姓立堡,營田積穀。京師置常平監。是時山東尚承齊俗,機巧奸偽,避役惰游者十六七。四方疲人,或詐老詐小,規免租賦。高祖令州縣大索貌閱,戶口不實者,正長遠配,而又開相糾之科。大功已下,兼令析籍,各為戶頭,以防容隱。



 於是計帳進四十四萬三千丁,新附一百六十四萬一千五百口。高熲又以人間課輸,雖有定分,年常徵納,除注恆多,長吏肆情,文帳出沒,復無定簿,難以推校,乃為輸籍定樣,請遍下諸州。每年正月五日,縣令巡人,各隨便近,五黨三黨,共為一團,依樣定戶上下。帝從之。自是奸無所容矣。



 時百姓承平日久,雖數
 遭水旱,而戶口歲增。諸州調物,每歲河南自潼關,河北自蒲阪,達於京師,相屬於路,晝夜不絕者數月。帝既躬履儉約,六宮咸服浣濯之衣。乘輿供御有故敝者,隨令補用,皆不改作。非享燕之事,所食不過一肉而已。



 有司嘗進乾姜,以布袋貯之,帝用為傷費,大加譴責。後進香,復以氈袋,因笞所司,以為後誡焉。由是內外率職,府帑充實,百官祿賜及賞功臣,皆出於豐厚焉。



 九年,陳平,帝親御硃雀門勞凱旋師,因行慶賞。自門外夾道列布帛之積,達於南郭,以次頒給。所費三百餘萬段。帝以江表初定,給復十年。自餘諸州,並免當年租賦,十年五月,又
 以宇內無事,益寬徭賦。百姓年五十者,輸庸停防。十一年,江南又反,越國公楊素討平之,師還,賜物甚廣。其餘出師命賞,亦莫不優隆。十二年,有司上言,庫藏皆滿。帝曰:「朕既薄賦於人,又大經賜用,何得爾也?」



 對曰:「用處常出,納處常入。略計每年賜用至數百萬段,曾無減損。」於是乃更闢左藏之院,構屋以受之。下詔曰:「既富而教,方知廉恥,寧積於人,無藏府庫。



 河北、河東今年田租,三分減一,兵減半,功調全免。」



 時天下戶口歲增,京輔及三河,地少而人眾,衣食不給。議者咸欲徙就寬鄉。



 其年冬,帝命諸州考使議之。又令尚書以其事策問四方貢士,竟
 無長算。帝乃發使四出,均天下之田。其狹鄉,每丁才至二十畝,老小又少焉。



 十三年,帝命楊素出,於岐州北造仁壽宮。素遂夷山堙谷,營構觀宇,崇臺累榭,宛轉相屬。役使嚴急,丁夫多死,疲敝顛僕者,推填坑坎,覆以土石,因而築為平地。死者以萬數。宮成,帝行幸焉。時方暑月,而死人相次於道,素乃一切焚除之。帝頗知其事,甚不悅。及入新宮游觀,乃喜,又謂素為忠。後帝以歲暮晚日登仁壽殿,周望原隰,見宮外磷火彌漫,又聞哭聲。令左右觀之,報曰:「鬼火。」



 帝曰:「此等工役而死,既屬年暮,魂魄思歸耶?」乃令灑酒宣敕,以咒遣之,自是乃息。



 開皇三年,朝廷以京師倉廩尚虛,議為水旱之備,於是詔於蒲、陜、虢、熊、伊、洛、鄭、懷、邵、衛、汴、許、汝等水次十三州,置募運米丁。又於衛州置黎陽倉,洛州置河陽倉,陜州置常平倉,華州置廣通倉,轉相灌注。漕關東及汾、晉之粟,以給京師。又遣倉部侍郎韋瓚,向蒲、陜以東募人能於洛陽運米四十石,經砥柱之險,達於常平者,免其征戍。其後以渭水多沙,流有深淺,漕者苦之。四年,詔曰:京邑所居,五方輻湊,重關四塞,水陸艱難,大河之流,波瀾東注,百川海瀆,萬里交通。雖三門之下,或有危慮,但發自小平,陸運至陜,還從河水,入於渭川,兼及上流,控引
 汾、晉,舟車來去,為益殊廣。而渭川水力,大小無常,流淺沙深,即成阻閡。計其途路,數百而已,動移氣序,不能往復,泛舟之役,人亦勞止。朕君臨區宇,興利除害,公私之弊,情實愍之。故東發潼關,西引渭水,因藉人力,開通漕渠,量事計功,易可成就。已令工匠,巡歷渠道,觀地理之宜,審終久之義,一得開鑿,萬代無毀。可使官及私家,方舟巨舫,晨昏漕運,沿溯不停,旬日之功,堪省億萬。誠知時當炎暑,動致疲勤,然不有暫勞,安能永逸。宣告人庶,知朕意焉。



 於是命宇文愷率水工鑿渠,引渭水,自大興城東至潼關三百餘里,名曰廣通渠。



 轉運通利,關內賴
 之。諸州水旱兇饑之處,亦便開倉賑給。



 五年五月,工部尚書、襄陽縣公長孫平奏曰:「古者三年耕而餘一年之積,九年作而有三年之儲,雖水旱為災,而人無菜色,皆由勸導有方,蓄積先備故也。去年亢陽,關內不熟,陛下哀愍黎元,甚於赤子。運山東之粟,置常平之官,開發倉廩,普加賑賜。少食之人,莫不豐足。鴻恩大德,前古未比。其強宗富室,家道有餘者,皆競出私財,遞相賙贍。此乃風行草偃,從化而然。但經國之理,須存定式。」



 於是奏令諸州百姓及軍人,勸課當社,共立養倉。收獲之日,隨其所得,觀課出粟及麥,於當社造倉窖貯之。即委社司,執
 帳檢校,每年收積,勿使捐敗。若時或不熟,當社有饑謹者,即以此穀賑給。自是諸州儲峙委積。其後關中連年大旱,而青、兗、汴、許、曹、亳、陳、仁、譙、豫、鄭、洛、伊、潁、邳等州大水,百姓饑饉。



 高祖乃命蘇威等,分道開倉賑給。又命司農丞王稟,發廣通之粟三百餘萬石,以拯關中,又發故城中周代舊粟,賤糶與人。買牛驢六千餘頭,分給尤貧者,令往關東就食。其遭水旱之州,皆免其年租賦。



 十四年,關中大旱,人饑。上幸洛陽,因令百姓就食。從官並準見口賑給,不以官位為限。明年,東巡狩,因祠泰山。是時義倉貯在人間,多有費捐。十五年二月,詔曰:「本置義倉,
 止防水旱,百姓之徒,不思久計,輕爾費捐,於後乏絕。



 又北境諸州,異於餘處,雲、夏、長、靈、鹽、蘭、豐、鄯、涼、甘、瓜等州,所有義倉雜種,並納本州。若人有旱儉少糧,先給雜種及遠年粟。」十六年正月,又詔秦、疊、成、康、武、文、芳、宕、旭、洮、岷、渭、紀、河、廓、豳、隴、涇、寧、原、敷、丹、延、綏、銀、扶等州社倉,並於當縣安置。二月,又詔社倉,準上中下三等稅,上戶不過一石,中戶不過七斗,下戶不過四斗。其後山東頻年霖雨,杞、宋、陳、亳、曹、戴、譙、潁等諸州,達於滄海,皆困水災,所在沉溺。十八年,天子遣使,將水工,巡行川源,相視高下,發隨近丁以疏導之。困乏者,開倉賑給,前後用穀五
 百餘石。遭水之處,租調皆免。自是頻有年矣。



 開皇八年五月,高熲奏諸州無課調處,及課州管戶數少者,官人祿力,乘前已來,恆出隨近之州。但判官本為牧人,役力理出所部。請於所管戶內,計戶徵稅。



 帝從之。先是京官及諸州,並給公廨錢,回易生利,以給公用。至十四年六月,工部尚書、安平郡公蘇孝慈等,以為所在官司,因循往昔,以公廨錢物,出舉興生,唯利是求,煩擾百姓,敗損風俗,莫斯之甚。於是奏皆給地以營農,回易取利,一皆禁止。十七年十一月,詔在京及在外諸司公廨,在市回易及諸處興生,並聽之。



 唯禁出舉收利云。



 煬帝即位,是時戶口益多,府庫盈溢,乃除婦人及奴婢部曲之課。男子以二十二成丁。始建東都,以尚書令楊素為營作大監,每月役丁二百萬人。徙洛州郭內人及天下諸州富商大賈數萬家以實之。新置興洛及回洛倉。又於皁澗營顯仁宮,苑囿連接,北至新安,南及飛山,西至澠池,周圍數百里。課天下諸州,各貢草木花果、奇禽異獸於其中,開渠,引穀、洛水,自苑西入,而東注于洛。又自板渚引河,達於淮海,謂之御河。河畔築御道,樹以柳。又命黃門侍郎王弘,上儀同於士澄,往江南諸州採大木,引至東都。所經州縣,遞送往返,首尾相屬,不絕者
 千里。而東都役使促迫,殭僕而斃者,十四五焉。每月載死丁,東至城皋,北至河陽,車相望於道。時帝將事遼、碣,增置軍府,掃地為兵。自是租賦之人益減矣。又造龍舟鳳甗,黃龍赤艦,樓船篾舫。募諸水工,謂之殿腳,衣錦行褲,執青絲纜挽船,以幸江都,帝御龍舟,文武官五品已上給樓船,九品已上給黃篾舫,舳艫相接,二百餘里。所經州縣,並令供頓,獻食豐辦者加官爵,闕乏者譴至死。又盛修車輿輦輅,旌旗羽儀之飾。課天下州縣,凡骨角齒牙,皮革毛羽,可飾器用,堪為氅毦者,皆責焉。徵發倉卒,朝命夕辦,百姓求捕,網罟遍野,水陸禽獸殆盡,猶不
 能給,而買於豪富蓄積之家,其價騰踴。是歲,翟雉尾一直十縑,白鷺鮮半之。



 乃使屯田主事常駿使赤土國,致羅剎。又使朝請大夫張鎮州擊流求,俘虜數萬。



 士卒深入,蒙犯瘴癘,餒疾而死者十八九。又以西域多諸寶物,令裴矩往張掖,監諸商胡互市。啖之以利,勸令入朝。自是西域諸蕃,往來相繼,所經州郡,疲於送迎,糜費以萬萬計。



 明年,帝北巡狩。又興眾百萬,北築長城,西距榆林,東至紫河,綿亙千餘里,死者太半。四年,發河北諸郡百餘萬眾,引沁水,南達於河,北通涿郡。自是以丁男不供,始以婦人從役。五年,西巡河右。西域諸胡,佩金玉,被錦
 罽,焚香奏樂,迎候道左。帝乃令武威、張掖士女,盛飾縱觀。衣服車馬不鮮者,州縣督課,以誇示之。其年,帝親征吐谷渾,破之於赤水。慕容佛允委其家屬,西奔青海。帝駐兵不出,遇天霖雨,經大斗拔谷,士卒死者十二三焉,馬驢十八九。於是置河源郡、積石鎮。又於西域之地置西海、鄯善、且末等郡。謫天下罪人,配為戍卒,大開屯田,發西方諸郡運糧以給之。道里懸遠,兼遇寇抄,死亡相續。



 六年,將征高麗,有司奏兵馬已多損耗。詔又課天下富人,量其貲產,出錢市武馬,填元數。限令取足。復點兵具器仗,皆令精新,濫惡則使人便斬。於是馬匹至十萬。
 七年冬,大會涿郡。分江淮南兵,配驍衛大將軍來護兒,別以舟師濟滄海,舳艫數百里。並載軍糧,期與大兵會平壤。是歲山東、河南大水,漂沒四十餘郡,重以遼東覆敗,死者數十萬,因屬疫疾,山東尤甚。所在皆以徵斂供帳軍旅所資為務,百姓雖困,而弗之恤也。每急徭卒賦,有所徵求,長吏必先賤買之,然後宣下,乃貴賣與人,旦暮之間,價盈數倍,裒刻徵斂,取辦一時。強者聚而為盜,弱者自賣為奴婢。九年,詔又課關中富人,計其貲產出驢,往伊吾、河源、且末運糧。多者至數百頭,每頭價至萬餘。又發諸州丁,分為四番,於遼西柳城營屯,往來艱苦,
 生業盡罄。盜賊四起,道路隔絕,隴右牧馬,盡為奴賊所掠,楊玄感乘虛為亂。時帝在遼東,聞之,遽歸於高陽郡。及玄感平,帝謂侍臣曰:「玄感一呼而從者如市,益知天下人不欲多,多則為賊。不盡誅,後無以示勸。」乃令裴蘊窮其黨與,詔郡縣坑殺之,死者不可勝數。所在驚駭。舉天下之人十分,九為盜賊,皆盜武馬,始作長槍,攻陷城邑。帝又命郡縣置督捕以討賊。益遣募人征遼,馬少不充八馱,而許為六馱。又不足,聽半以驢充。在路逃者相繼,執獲皆斬之,而莫能止。帝不懌。



 遇高麗執送叛臣斛斯政,遣使求降,發詔赦之。囚政至於京師,於開遠門外,
 磔而射殺之。遂幸太原,為突厥圍於雁門。突厥尋散,遽還洛陽,募益驍果,以充舊數。



 是時百姓廢業,屯集城堡,無以自給。然所在倉庫,猶大充爨,吏皆懼法,莫肯賑救,由是益困。初皆剝樹皮以食之,漸及於葉,皮葉皆盡,乃煮土或搗稿為末而食之。其後人乃相食。十二年,帝幸江都。是時李密據洛口倉,聚眾百萬。越王侗與段達等守東都。東都城內糧盡,布帛山積,乃以絹為汲綆,然布以爨。代王侑與衛玄守京師,百姓饑饉,亦不能救。義師入長安,發永豐倉以賑之,百姓方蘇息矣。



 晉自過江,凡貨賣奴婢馬牛田宅,有文券,率錢一萬,輸
 估四百入官,賣者三百,買者一百。無文券者,隨物所堪,亦百分收四,名為散估。歷宋齊梁陳,如此以為常。以此人競商販,不為田業,故使均輸,欲為懲勵。雖以此為辭,其實利在侵削。又都西有石頭津,東有方山津,各置津主一人,賊曹一人,直水五人,以檢察禁物及亡叛者。其荻炭魚薪之類過津者,並十分稅一以入官。其東路無禁貨,故方山津檢察甚簡。淮水北有大市百餘,小市十餘所。大市備置官司,稅斂既重,時甚苦之。



 梁初,唯京師及三吳、荊、郢、江、湘、梁、益用錢。其餘州郡,則雜以穀帛交易。交、廣之域,全以金銀為貨。武帝乃鑄錢,
 肉好周郭,文曰「五銖」,重如其文。而又別鑄,除其肉郭,謂之女錢。二品並行。百姓或私以古錢交易,有直百五銖、五銖、女錢、太平百錢、定平一百、五銖雉錢、五銖對文等號。輕重不一。



 天子頻下詔書,非新鑄二種之錢,並不許用。而趣利之徒,私用轉甚。至普通中,乃議盡罷銅錢,更鑄鐵錢。人以鐵賤易得,並皆私鑄。及大同已後,所在鐵錢,遂如丘山,物價騰貴。交易者以車載錢,不復計數,而唯論貫。商旅奸詐,因之以求利,自破嶺以東,八十為百,名曰東錢。江、郢已上,七十為百,名曰西錢。京師以九十為百,名曰長錢。中大同元年,天子乃詔通用足陌。詔下
 而人不從,錢陌益少。至於末年,遂以三十五為百云。



 陳初,承梁喪亂之後,鐵錢不行。始梁末又有兩柱錢及鵝眼錢,於時人雜用,其價同,但兩柱重而鵝眼輕。私家多熔錢,又間以錫鐵,兼以粟帛為貨。至文帝天嘉五年,改鑄五銖。初出,一當鵝眼之十。宣帝太建十一年,又鑄大貨六銖,以一當五銖之十,與五銖並行。後還當一,人皆不便。乃相與訛言曰:「六銖錢有不利縣官之象。」未幾而帝崩,遂廢六銖而行五銖。竟至陳亡。其嶺南諸州,多以鹽米布交易,俱不用錢云。



 齊神武霸政之初,承魏猶用永安五銖。遷鄴已後,百姓
 私鑄,體制漸別,遂各以為名。有雍州青赤,梁州生厚、緊錢、吉錢,河陽生澀、天柱、赤牽之稱。冀州之北,錢皆不行,交貿者皆以絹布。神武帝乃收境內之銅及錢,仍依舊文更鑄,流之四境。未幾之間,漸復細薄,奸偽競起。文宣受禪,除永安之錢,改鑄常平五銖,重如其文。其錢甚貴,且制造甚精。至乾明、皇建之間,往往私鑄。鄴中用錢,有赤熟、青熟、細眉、赤生之異。河南所用,有青薄鉛錫之別。青、齊、徐、兗、梁、豫州,輩類各殊。武平已後,私鑄轉甚,或以生鐵和銅。至於齊亡,卒不能禁。



 後周之初,尚用魏錢。及武帝保定元年七月,及更鑄布泉之錢,以一當五,與五銖
 並行。時梁、益之境,又雜用古錢交易。河西諸郡,或用西域金銀之錢,而官不禁。建德三年六月,更鑄五行大布錢,以一當十,大收商估之利,與布泉錢並行。



 四年七月,又以邊境之上,人多盜鑄,乃禁五行大布,不得出入四關,布泉之錢,聽入而不聽出。五年正月,以布泉漸賤而人不用,遂廢之。初令私鑄者絞,從者遠配為戶。齊平已後,山東之人,猶雜用齊氏舊錢。至宣帝大象元年十一月,又鑄永通萬國錢。以一當十,與五行大布及五銖,凡三品並用。



 高祖既受周禪,以天下錢貨輕重不等,乃更鑄新錢。背面肉好,皆有周郭,文曰「五銖」,而重如其文。每
 錢一千重四斤二兩。是時錢既新出,百姓或私有熔鑄。



 三年四月,詔四面諸關,各付百錢為樣。從關外來,勘樣相似,然後得過。樣不同者,即壞以為銅,入官。詔行新錢已後,前代舊錢,有五行大布、永通萬國及齊常平,所在用以貿易不止。四年,詔仍依舊不禁者,縣令奪半年祿。然百姓習用既久,尚猶不絕。五年正月,詔又嚴其制。自是錢貨始一,所在流布,百姓便之。是時見用之錢,皆須和以錫鑞。錫鑞既賤,求利者多,私鑄之錢,不可禁約。其年,詔乃禁出錫鑞之處,並不得私有採取。十年,詔晉王廣聽於揚州立五爐鑄錢。其後奸狡稍漸磨鑢錢郭,取
 銅私鑄,又雜以錫錢。遞相放效,錢遂輕薄。乃下惡錢之禁。京師及諸州邸肆之上,皆令立榜,置樣為準。不中樣者,不入於市。十八年,詔漢王諒聽於並州立五爐鑄錢。是時江南人間錢少,晉王廣又聽於鄂州白紵山有銅筼處,錮銅鑄錢。於是詔聽置十爐鑄錢。又詔蜀王秀聽於益州立五爐鑄錢。是時錢益濫惡,乃令有司,括天下邸肆見錢,非官鑄者皆毀之,其銅入官。而京師以惡錢貿易,為吏所執,有死者。數年之間,私鑄頗息。大業已後,王綱弛紊,巨奸大猾,遂多私鑄,錢轉薄惡。初每千猶重二斤,後漸輕至一斤。或翦鐵鍱,裁皮糊紙以為錢,相雜
 用之。貨賤物貴,以至於亡。



\end{pinyinscope}