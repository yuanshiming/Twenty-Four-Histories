\article{卷十九志第十四 天文上}

\begin{pinyinscope}

 若夫法紫微以居中,擬明堂而布政,依分野而命國,體眾星而效官,動必順時,教不違物,故能成變化之道,合陰陽之妙。爰在庖犧,仰觀俯察,謂以天之七曜、二十八星,周於穹圓之度,以麗十二位也。在天成象,示見吉兇。五緯入房,啟姬王之肇跡,長星孛斗,鑒宋人之首亂,天意人事,同乎影響。自夷王下堂而見諸侯,赧王登臺而
 避責,《記》曰:「天子微,諸侯僭。」於是師兵吞滅,殭僕原野。秦氏以戰國之餘,怙茲兇暴,小星交鬥,長彗橫天。漢高祖驅駕英雄,墾除災害,五精從歲,七重暈畢,含樞曾緬,道不虛行。自西京創制,多歷年載。世祖中興,當塗馭物,金行水德,祗奉靈命,玄兆著明,天人不遠。昔者滎河獻籙,溫洛呈圖,六爻摛範,三光宛備,則星官之書,自黃帝始。高陽氏使南正重司天,北正黎司地,帝堯乃命羲、和,欽若昊天。夏有昆吾,殷有巫咸,周之史佚,宋之子韋,魯之梓慎,鄭之裨灶,魏有石氏,齊有甘公,皆能言天文、察微變者也。漢之傳天數者,則有唐都、李尋之倫。光武時,則
 有蘇伯況、郎雅光,並能參伍天文,發揚善道,補益當時,監垂來世。而河、洛圖緯,雖有星占星官之名,未能盡列。



 後漢張衡為太史令,鑄渾天儀,總序經星,謂之《靈憲》。其大略曰:「星也者,體生於地,精發於天。紫宮為帝皇之居,太微為五帝之坐,在野象物,在朝象官。居其中央,謂之北斗,動系於占,實司王命。四布於方,為二十八星,日月運行,歷示休咎。五緯經次,用彰禍福,則上天之心,於是見矣。中外之官,常明者百有二十,可名者三百二十,為星二千五百;微星之數萬一千五百二十,庶物蠢動,咸得系命。」而衡所鑄之圖,遇亂堙滅,星官名數,今亦不存。
 三國時,吳太史令陳卓,始列甘氏、石氏、巫咸三家星官,著於圖錄。並注占贊,總有二百五十四官,一千二百八十三星,並二十八宿及輔官附坐一百八十二星,總二百八十三官,一千五百六十五星。宋元嘉中,太史令錢樂之所鑄渾天銅儀,以硃黑白三色,用殊三家,而合陳卓之數。高祖平陳,得善天官者周墳,並得宋氏渾儀之器。乃命庾季才等,參校周、齊、梁、陳及祖恆、孫僧化官私舊圖,刊其大小,正彼疏密,依準三家星位,以為蓋圖。旁摛始分,甄表常度,並具赤黃二道,內外兩規。懸象著明,纏離攸次,星之隱顯,天漢昭回,宛若穹蒼,將為正範。以
 墳為太史令。墳博考經書,勤於教習,自此太史觀生,始能識天官。煬帝又遣宮人四十人,就太史局,別詔袁充,教以星氣,業成者進內,以參占驗云。史臣於觀臺訪渾儀,見元魏太史令晁崇所造者,以鐵為之,其規有六。其外四規常定,一象地形,二象赤道,其餘象二極。



 其內二規,可以運轉,用合八尺之管,以窺星度。周武帝平齊所得。隋開皇三年,新都初成,以置諸觀臺之上。大唐因而用焉。馬遷《天官書》及班氏所載,妖星暈珥,雲氣虹霓,存其大綱,未能備舉。自後史官,更無紀錄。《春秋傳》曰:「公既視朔,遂登觀臺,凡分至啟閉,必書云物。」神道司存,安可
 誣也!今略舉其形名占驗,次之經星之末云。



 天體古之言天者有三家,一曰蓋天,二曰宣夜,三曰渾天。



 蓋天之說,即《周髀》是也。其本庖犧氏立周天歷度,其所傳則周公受於殷商,周人志之,故曰《周髀》。髀,股也;股者,表也。其言天似蓋笠,地法覆槃,天地各中高外下。北極之下,為天地之中,其地最高,而滂沲四蕆,三光隱映,以為晝夜。天中高於外衡冬至日之所在六萬里,北極下地高於外衡下地亦六萬里,外衡高於北極下地二萬里。天地隆高相從,日去地恆八萬里。日麗天而平轉,分冬
 夏之間日所行道為七衡六間。每衡周徑里數,各依算術,用句股重差,推晷影極游,以為遠近之數,皆得於表股也,故曰《周髀》。



 又《周髀》家云:「天圓如張蓋,地方如棋局。天旁轉如推磨而左行,日月右行,天左轉,故日月實東行,而天牽之以西沒。譬之於蟻行磨石之上,磨左旋而蟻右去,磨疾而蟻遲,故不得不隨磨以左回焉。天形南高而北下,日出高故見,日入下故不見。天之居如倚蓋,故極在人北,是其證也。極在天之中,而今在人北,所以知天之形如倚蓋也。日朝出陰中,暮入陰中,陰氣暗冥,故從沒不見也。夏時陽氣多,陰氣少,陽氣光明,與日同
 暉,故日出即見,無蔽之者,故夏日長也。冬時陰氣多,陽氣少,陰氣暗冥,掩日之光,雖出猶隱不見,故冬日短也。」



 漢末,揚子雲難蓋天八事,以通渾天。其一云:「日之東行,循黃道。晝夜中規,牽牛距北極南百一十度,東井距北極南七十度,並百八十度。周三徑一,二十八宿周天當五百四十度,今三百六十度,何也?」其二曰:「春秋分之日正出在卯,入在酉,而晝漏五十刻。即天蓋轉,夜當倍晝。今夜亦五十刻,何也?」其三曰:「日入而星見,日出而不見,即斗下見日六月,不見日六月。北斗亦當見六月,不見六月。今夜常見,何也?」其四曰:「以蓋圖視天河,起斗而東入
 狼弧間,曲如輪。今視天河直如繩,何也?」其五曰:「周天二十八宿,以蓋圖視天,星見者當少,不見者當多。今見與不見等,何出入無冬夏,而兩宿十四星當見,不以日長短故見有多少,何也?」其六曰:「天至高也,地至卑也。日托天而旋,可謂至高矣。



 縱人目可奪,水與影不可奪也。今從高山上,以水望日,日出水下,影上行,何也?」



 其七曰:「視物近則大,遠則小。今日與北斗,近我而小,遠我而大,何也?」其八曰:「視蓋尞與車輻間,近杠轂即密,益遠益疏。今北極為天杠轂,二十八宿為天尞輻。以星度度天,南方次地星間當數倍。今交密,何也?」其後桓譚、鄭玄、蔡邕、陸
 績,各陳《周髀》考驗天狀,多有所違。逮梁武帝於長春殿講義,另擬天體,全同《周髀》之文,蓋立新意,以排渾天之論而已。



 宣夜之書,絕無師法。唯漢秘書郎郗萌記先師相傳云:「天了無質,仰而瞻之,高遠無極,眼瞀精絕,故蒼蒼然也。譬之旁望遠道之黃山而皆青,俯察千仞之深谷而窈黑,夫青非真色,而黑非有體也。日月眾星,自然浮生虛空之中,其行其止,皆須氣焉。是以七曜或逝或住,或順或逆,伏見無常,進退不同,由乎無所根系,故各異也。故辰極常居其所,而北斗不與眾星西沒也。」



 晉成帝咸康中,會稽虞喜因宣夜之說,作《安天論》,以為「天高
 窮於無窮,地深測於不測。天確乎在上,有常安之形,地魄焉在下,有居靜之體,當相覆冒,方則俱方,圓則俱圓,無方圓不同之義也。其光曜布列,各自運行,猶江海之有潮汐,萬品之有行藏也。」葛洪聞而譏之曰:「茍辰宿不麗於天,天為無用,便可言無。何必夏雲有之而不動乎?」由此而談,葛洪可謂知言之選也。喜族祖河間相聳,又立《穹天論》云:「天形穹隆如雞子幕,其際周接四海之表,浮乎元氣之上。譬如覆奩以抑水而不沒者,氣充其中故也。日繞辰極,沒西還東,而不出入地中。天之有極,猶蓋之有斗也。天北下於地三十度,極之傾在地卯酉之
 北亦三十度。人在卯酉之南十餘萬里,故斗極之下,不為地中,當對天地卯酉之位耳。日行黃道繞極。



 極北去黃道百一十五度,南去黃道六十七度,二至之所舍,以為長短也。」吳太常姚信,造《昕天論》云:「人為靈蟲,形最似天。今人頤前侈臨胸,而項不能覆背。



 近取諸身,故知天之體,南低入地,北則偏高也。又冬至極低,而天運近南,故日去人遠,而斗去人近,北天氣至,故水寒也。夏至極起,而天運近北,而斗去人遠,日去人近,南天氣至,故蒸熱也。極之高時,日行地中淺,故夜短;天去地高,故晝長也。極之低時,日行地中深,故夜長;天去地下,故晝短也。」
 自虞喜、虞聳、姚信,皆好奇徇異之說,非極數談天者也。



 前儒舊說,天地之體,狀如鳥卵,天包地外,猶殼之裹黃,周旋無端,其形渾渾然,故曰渾天。又曰:「天表裏有水,兩儀轉運,各乘氣而浮,載水而行。」漢王仲任,據蓋天之說以駁渾儀,云:「舊說,天轉從地下過。今掘地一丈輒有水,天何得從水中行乎?甚不然也。日隨天而轉,非入地。夫人目所望,不過十里,天地合矣。實非合也,遠使然耳。今視日入,非入也,亦遠耳。當日入西方之時,其下之人亦將謂之為中也。四方之人,各以其近者為出,遠者為入矣。何以明之?今試使一人把大炬火,夜行於平地,去人
 十里,火光滅矣。非火滅也,遠使然耳。今日西轉不復見,是火滅之類也。日月不圓也,望視之所以圓者,去人遠也。夫日,火之精也;月,水之精也。水火在地不圓,在天何故圓?」丹陽葛洪釋之曰:《渾天儀注》云:「天如雞子,地如中黃,孤居於天內,天大而地小。天表裏有水,天地各乘氣而立,載水而行。周天三百六十五度四分度之一,又中分之,則半覆地上,半繞地下,故二十八宿半見半隱。天轉如車轂之運也。」諸論天者雖多,然精於陰陽者少。張平子、陸公紀之徒,咸以為推步七曜之道,以度歷象昏明之證候,校以四八之氣,考以漏刻之分,占咎影之往
 來,求形驗於事情,莫密於渾象也。



 張平子既作銅渾天儀,於密室中,以漏水轉之,與天皆合如符契也。崔子玉為其碑銘曰:「數術窮天地,制作侔造化。高才偉藝,與神合契。」蓋由於平子渾儀及地動儀之有驗故也。若天果如渾者,則天之出入,行於水中,為必然矣。故《黃帝書》曰:「天在地外,水在天外。水浮天而載地者也。」又《易》曰:「時乘六龍。」



 夫陽爻稱龍,龍者居水之物,以喻天。天陽物也,又出入水中,與龍相似,故比以龍也。聖人仰觀俯察,審其如此。故《晉》卦坤上離下,以證日出於地也。又《明夷》之卦離下坤上,以證日入於地也。又《需》卦乾下坎上,此亦天
 入水中之象也。



 天為金,金水相生之物也。天出入水中,當有何損,而謂為不可乎?然則天之出入水中,無復疑矣。



 又今視諸星出於東者,初但去地小許耳。漸而西行,先經人上,後遂轉西而下焉,不旁旋也。其先在西之星,亦稍下而沒,無北轉者。日之出入亦然。若謂天磨石轉者,眾星日月,宜隨天而回,初在於東,次經於南,次到於西,次及於北,而復還於東,不應橫過去也。今日出於東,冉冉轉上,及其入西,亦復漸漸稍下,都不繞邊北去。了了如此,王生必固謂為不然者,疏矣。今日徑千里,其中足以當小星之數十也。若日以轉遠之故,但當光曜不
 能復來照及人耳,宜猶望見其體,不應都失其所在也。日光既盛,其體又大於星。今見極北之星,而不見日之在北者,明其不北行也。若日以轉遠之故,不復可見,其比入之間,應當稍小。而日方入之時,反乃更大,此非轉遠之徵也。王生以火炬喻日,吾亦將借子之矛,以刺子之盾焉。



 把火之人,去人轉遠,其光轉微,而日月自出至入,不漸小也。王生以火喻之,謬矣。又日之入西方,視之稍稍去,初尚有半,如橫破鏡之狀,須臾淪沒矣。若如王生之言,日轉北去者,其北都沒之頃,宜先如豎破鏡之狀,不應如橫破鏡也。如此言之,日入北方,不亦孤孑
 乎?又月之光微,不及日遠矣。月盛之時,雖有重雲蔽之,不見月體,而夕猶朗然,是月光猶從雲中而照外也。日若繞西及北者,其光故應如月在雲中之狀,不得夜便大暗也。又日入則星月出焉。明知天以日月分主晝夜,相代而照也。若日常出者,不應日亦入而星月出也。



 又案河、洛之文,皆云水火者,陰陽之餘氣也。夫言餘氣,則不能生日月可知也,顧當言日精生火者可耳。若水火是日月所生,則亦何得盡如日月之圓乎?今火出於陽燧,陽燧圓而火不圓也。水出於方諸,方諸方而水不方也。又陽燧可以取火於日,而無取日於火之理,此則日
 精之生火明矣。方諸可以取水於月,無取月於水之道,此則月精之生水了矣。王生又云:「遠故視之圓。」若審然者,月初生之時及既虧之後,何以視之不圓乎?而日食,或上或下,從側而起,或如鉤至盡。若遠視見圓,不宜見其殘缺左右所起也。此則渾天之體,信而有徵矣。



 宋何承天論渾天象體曰:「詳尋前說,因觀渾儀,研求其意,有悟天形正圓,而水居其半,地中高外卑,水周其下。言四方者,東曰昜谷,日之所出,西曰濛汜,日之所入。《莊子》又云:『北溟有魚,化而為鳥,將徙於南溟。』斯亦古之遺記,四方皆水證也。四方皆水,謂之四海。凡五行相生,水生於
 金。是故百川發源,皆自山出,由高趣下,歸注于海。日為陽精,光曜炎熾,一夜入水,所經焦竭。



 百川歸注,足以相補,故旱不為減,浸不為益。」又云:「周天三百六十五度、三百四分之七十五。天常西轉,一日一夜,過周一度。南北二極,相去一百一十六度、三百四分度之六十五強,即天經也。黃道袤帶赤道,春分交於奎七度,秋分交於軫十五度,冬至斗十四度半強,夏至井十六度半。從北極扶天而南五十五度強,則居天四維之中,最高處也,即天頂也。其下則地中也。」自外與王蕃大同。王蕃《渾天說》,具於《晉史》。



 舊說渾天者,以日月星辰,不問春秋冬夏,晝
 夜晨昏,上下去地中皆同,無遠近。《列子》曰:「孔子東游,見兩小兒鬥。問其故,一小兒曰:『我以日始出去人近,而日中時遠也。』一小兒曰:『我以為日初出遠,而日中時近也。』言初出近者曰:『日初出,大如車蓋,及其日中,裁如盤蓋。此不為遠者小,近者大乎?』言日初出遠者曰:『日初出時,滄滄涼涼,及其中時,熱如探湯。此不為近者熱,遠者涼乎?』」



 桓譚《新論》云:「漢長水校尉平陵關子陽,以為日之去人,上方遠而四傍近。



 何以知之?星宿昏時出東方,其間甚疏,相離丈餘。及夜半在上方,視之甚數,相離一二尺。以準度望之,逾益明白,故知天上之遠於傍也。日為天
 陽,火為地陽。



 地陽上升,天陽下降。今置火於地,從傍與上,診其熱,遠近殊不同焉。日中正在上,覆蓋人,人當天陽之沖,故熱於始出時。又新從太陰中來,故復涼於其西在桑榆間也。桓君山曰:子陽之言,豈其然乎?」



 張衡《靈臺》曰:「日之薄地,暗其明也。由暗視明,明無所屈,是以望之若大。方其中,天地同明,明還自奪,故望之若小。火當夜而揚光,在晝則不明也。



 月之於夜,與日同而差微。」



 晉著作郎陽平束皙,字廣微,以為傍方與上方等。傍視則天體存於側,故日出時視日大也。日無小大,而所存者有伸厭。厭而形小,伸而體大,蓋其理也。又日始出時色
 白者,雖大不甚,始出時色赤者,其大則甚,此終以人目之惑,無遠近也。



 且夫置器廣庭,則函牛之鼎如釜,堂崇十仞,則八尺之人猶短,物有陵之,非形異也。夫物有惑心,形有亂目,誠非斷疑定理之主。故仰游雲以觀月,月常動而云不移;乘船以涉水,水去而船不徙矣。



 姜岌云:「余以為子陽言天陽下降,日下熱,束皙言天體存於目,則日大,頗近之矣。渾天之體,圓周之徑,詳之於天度,驗之於晷影,而紛然之說,由人目也。



 參伐初出,在旁則其間疏,在上則其間數。以渾檢之,度則均也。旁之與上,理無有殊也。夫日者純陽之精也,光明外曜,以眩人目,故
 人視日如小。及其初出,地有游氣,以厭日光,不眩人目,即日赤而大也。無游氣則色白,大不甚矣。地氣不及天,故一日之中,晨夕日色赤,而中時日色白。地氣上升,蒙蒙四合,與天連者,雖中時亦赤矣。日與火相類,火則體赤而炎黃,日赤宜矣。然日色赤者,猶火無炎也。光衰失常,則為異矣。」



 梁奉朝請祖恆曰:自古論天者多矣,而群氏糾紛,至相非毀。竊覽同異,稽之典經,仰觀辰極,傍矚四維,睹日月之升降,察五星之見伏,校之以儀象,覆之以晷漏,則渾天之理,信而有徵。輒遺眾說,附渾儀云。《考靈曜》先儒求得天地相去十七萬八千五百里,以晷影
 驗之,失於過多。既不顯求之術,而虛設其數,蓋誇誕之辭,宜非聖人之旨也。學者多固其說而未之革,豈不知尋其理歟,抑未能求其數故也?王蕃所考,校之前說,不啻減半。雖非揆格所知,而求之以理,誠未能遙趣其實,蓋近密乎?輒因王蕃天高數,以求冬至、春分日高及南戴日下去地中數。法,令表高八尺與冬至影長一丈三尺,各自乘,並而開方除之為法。天高乘表高為實,實如法,得四萬二千六百五十八里有奇,即冬至日高也。以天高乘冬至影長為實,實如法,得六萬九千三百二十里有奇,即冬至南戴日下去地中數也。求春秋分數法,
 令表高及春秋分影長五尺三寸九分,各自乘,並而開方除之為法。因冬至日高實,而以法除之,得六萬七千五百二里有奇,即春秋分日高也。以天高乘春秋分影長實,實如法而一,得四萬五千四百七十九里有奇,即春秋分南戴日下去地中數也。南戴日下,所謂丹穴也。推北極裏數法,夜於地中表南,傅地遙望北辰紐星之末,令與表端參合。以人目去表數及表高各自乘,並而開方除之為法。天高乘表高數為實,實如法而一,即北辰紐星高地數也。天高乘人目去表為實,實如法,即去北戴極下之數也。北戴斗極為空桐。



 日去赤道表裏二十四度,遠寒近暑而中和。二分之日,去天頂三十六度。日去地中,四時同度,而有寒暑者,地氣上騰,天氣下降,故遠日下而寒,近日下而暑,非有遠近也。猶火居上,雖遠而炎,在傍,雖近而微。視日在傍而大,居上而小者,仰矚為難,平觀為易也。由視有夷險,非遠近之效也。今懸珠於百仞之上,或置之於百仞之前,從而觀之,則大小殊矣。先儒弗斯取驗,虛繁翰墨,夷途頓轡,雄辭析辯,不亦迂哉!今大寒在冬至後二氣者,寒積而未消也。大暑在夏至後二氣者,暑積而未歇也。寒暑均和,乃在春秋分後二氣者,寒暑積而未平也。譬之
 火始入室,而未甚溫,弗事加薪,久而逾熾。既已遷之,猶有餘熱也。



 渾天儀案《虞書》:「舜在〔璣玉衡,以齊七政,」則《考靈曜》所謂觀玉儀之游,昏明主時,乃命中星者也。〔璣中而星未中為急,急則日過其度,月不及其宿。



 〔璣未中而星中為舒,舒則日不及其度,月過其宿。〔璣中而星中為調,調則風雨時,庶草蕃蕪,而五穀登,萬事康也。所言〔璣者,謂渾天儀也。故《春秋文耀鉤》云:「唐堯即位,羲、和立渾儀。」而先儒或因星官書,北斗第二星名旋,第三星名璣,第五
 星名玉衡,仍七政之言,即以為北斗七星。載筆之官,莫之或辨。史遷、班固,猶且致疑。馬季長創謂璣衡為渾天儀。鄭玄亦云;「其轉運者為璣,其持正者為衡,皆以玉為之。七政者,日月五星也。以璣衡視其行度,以觀天意也。」故王蕃云:「渾天儀者,羲、和之舊器,積代相傳,謂之璣衡。其為用也,以察三光,以分宿度者也。又有渾天象者,以著天體,以布星辰。而渾象之法,地當在天中,其勢不便,故反觀其形,地為外匡,於已解者,無異在內。詭狀殊體,而合於理,可謂奇巧。然斯二者,以考於天,蓋密矣。」又云:「古舊渾象,以二分為一度,周七尺三寸半分。而莫知何代
 所造。」今案虞喜云:「落下閎為漢孝武帝於地中轉渾天,定時節,作《泰初歷》。」或其所制也。



 漢孝和帝時,太史揆候,皆以赤道儀,與天度頗有進退。以問典星待詔姚崇等,皆曰《星圖》有規法,日月實從黃道。官無其器。至永元十五年,詔左中郎將賈逵乃始造太史黃道銅儀。至桓帝延熹七年,太史令張衡更以銅制,以四分為一度,周天一丈四尺六寸一分。亦於密室中以漏水轉之,令司之者,閉戶而唱之,以告靈臺之觀天者。〔璣所加,某星始見,某星已中,某星今沒,皆如合符。蕃以古制局小,以布星辰,相去稠概,不得了察。張衡所作,又復傷大,難可轉
 移。蕃今所作,以三分為一度,周一丈九寸五分、四分分之三。長古法三尺六寸五分、四分分之一,減衡法亦三尺六寸五分、四分分之一。渾天儀法,黃赤道各廣一度有半。故今所作渾象,黃赤道各廣四分半,相去七寸二分。又云「黃赤二道,相共交錯,其間相去二十四度。以兩儀準之,二道俱三百六十五度有奇。又赤道見者,常一百八十二度半強。又南北考之,天見者亦一百八十二度半強。是以知天之體圓如彈丸,南北極相去一百八十二度半強也。而陸績所作渾象,形如鳥卵,以施二道,不得如法。若使二道同規,則其間相去不得滿二十四度。
 若令相去二十四度,則黃道當長於赤道。



 又兩極相去,不翅八十二度半強。案績說云:『天東西徑三十五萬七千里,直徑亦然。』則績意亦以天為正圓也。器與言謬,頗為乖僻。」然則渾天儀者,其制有機有衡。既動靜兼狀,以效二儀之情,又周旋衡管,用考三光之分。所以揆正宿度,準步盈虛,求古之遺法也。則先儒所言圓規徑八尺,漢候臺銅儀,蔡邕所欲寢伏其下者是也。



 梁華林重雲殿前所置銅儀,其制則有雙環規相並,間相去三寸許,正豎當子午。



 其子午之間,應南北極之衡,各合而為孔,以象南北樞。植楗於前後以屬焉。又有單橫規,高下正
 當渾之半。皆周市分為度數;署以維辰之位,以象地。又有單規,斜帶南北之中,與春秋二分之日道相應。亦周匝分為度數,而署以維辰,並相連者。



 屬楗植而不動。其里又有雙規相並,如外雙規。內徑八尺,周二丈四尺,而屬雙軸。



 軸兩頭出規外各二寸許,合兩為一。內有孔,圓徑二寸許,南頭入地下,注於外雙規南樞孔中,以象南極。北頭出地上,入於外雙規北樞孔中,以象北極。其運動得東西轉,以象天行。其雙軸之間,則置衡,長八尺,通中有孔,圓徑一寸。當衡之半,兩邊有關,各注著雙軸。衡即隨天象東西轉運,又自於雙軸間得南北低仰。所
 以準驗辰歷,分考次度,其於揆測,唯所欲為之者也。檢其鐫題,是偽劉曜光初六年,史官丞南陽孔挺所造,則古之渾儀之法者也。而宋御史中丞何承天及太中大夫徐爰,各著《宋史》,咸以為即張衡所造。其儀略舉天狀,而不綴經星七曜。魏、晉喪亂,沉沒西戎。義熙十四年,宋高祖定咸陽得之。梁尚書沈約著《宋史》,亦云然,皆失之遠矣。



 後魏道武天興初,命太史令晁崇修渾儀,以觀星象。十有餘載,至明元永興四年壬子,詔造太史候部鐵儀,以為渾天法,考〔璣之正。其銘曰;「於皇大代,配天比祚。赫赫明明,聲烈遐布。爰造茲器,考正宿度。貽法後葉,
 永垂典故。」其制並以銅鐵,唯志星度以銀錯之。南北柱曲抱雙規,東西柱直立,下有十字水平,以植四柱。十字之上,以龜負雙規。其餘皆與劉曜儀大同。即今太史候臺所用也。



 渾天象渾天象者,其制有機而無衡,梁末秘府有,以木為之。其圓如丸,其大數圍。



 南北兩頭有軸。遍體布二十八宿、三家星、黃赤二道及天漢等。別為橫規環,以匡其外。高下管之,以象地。南軸頭入地,注于南植,以象南極。北軸頭出於地上,注于北植,以象北極。正東西運轉。昏明中星,
 既其應度,分至氣節,亦驗,在不差而已。不如渾儀,別有衡管,測揆日月,分步星度者也。吳太史令陳苗云:「先賢制木為儀,名曰渾天。」即此之謂耶?由斯而言,儀象二器,遠不相涉。則張衡所造,蓋亦止在渾象七曜,而何承天莫辨儀象之異,亦為乖失。



 宋文帝以元嘉十三年詔太史更造渾儀。太史令錢樂之依案舊說,採效儀象,鑄銅為之。五分為一度,徑六尺八分少,周一丈八尺二寸六分少。地在天內,不動。



 立黃赤二道之規,南北二極之規,布列二十八宿、北斗極星。置日月五星於黃道上。



 為之杠軸,以象天運。昏明中星,與天相符。梁末,置於文德殿
 前。至如斯制,以為渾儀,儀則內闕衡管。以為渾象,而地不在外。是參兩法,別為一體。就器用而求,猶渾象之流,外內天地之狀,不失其位也。吳時又有葛衡,明達天官,能為機巧。改作渾天,使地居於天中。以機動之,天動而地止,以上應晷度,則樂之之所放述也。到元嘉十七年,又作小渾天,二分為一度,徑二尺二寸,周六尺六寸。安二十八宿中外官星備足。以白青黃等三色珠為三家星。其日月五星,悉居黃道。亦象天運,而地在其中。宋元嘉所造儀象器,開皇九年平陳後,並入長安。大業初,移於東都觀象殿。



 蓋圖晉侍中劉智云:「顓頊造渾儀,黃帝為蓋天。」然此二器,皆古之所制,但傳說義者,失其用耳。昔者聖王正歷明時,作圓蓋以圓列宿。極在其中,回之以觀天象。分三百六十五度、四分度之一,以定日數。日行於星紀,轉回右行,故圓規之,以為日行道。欲明其四時所在,故於春也,則以青為道;於夏也,則以赤為道;於秋也,則以白為道;於冬也,則以黑為道。四季之末,各十八日,則以黃為道。蓋圖已定,仰觀雖明,而未可正昏明,分晝夜,故作渾儀,以象天體。今案自開皇已後,天下一統,靈臺以後魏鐵渾
 天儀,測七曜盈縮,以蓋圖列星坐,分黃赤二道距二十八宿分度,而莫有更為渾象者矣。



 仁壽四年,河間劉焯造《皇極歷》,上啟於東宮。論渾天云:璇璣玉衡,正天之器,帝王欽若,世傳其象。漢之孝武,詳考律歷,糾落下閎、鮮於妄人等,共所營定。逮於張衡,又尋述作,亦其體制,不異閎等。雖閎制莫存,而衡造有器。至吳時,陸績、王蕃,並要修鑄。績小有異,蕃乃事同。宋有錢樂之,魏初晁崇等,總用銅鐵,小大有殊,規域經模,不異蕃造。觀蔡邕《月令章句》,鄭玄注《考靈曜》,勢同衡法,迄今不改。焯以愚管,留情推測,見其數制,莫不違爽。失之千里,差若毫厘,大象
 一乖,餘何可驗。況赤黃均度,月無出入,至所恆定,氣不別衡。分刻本差,輪回守故。其為疏謬,不可復言。亦既由理不明,致使異家間出。蓋及宣夜,三說並驅,平、昕、安、穹,四天騰沸。至當不二,理唯一揆,豈容天體,七種殊說?又影漏去極,就渾可推,百骸共體,本非異物。此真已驗,彼偽自彰,豈朗日未暉,爝火不息,理有而闕,詎不可悲者也?昔蔡邕自朔方上書曰:「以八尺之儀,度知天地之象,古有其器,而無其書。常欲寢伏儀下,案度成數,而為立說。」邕以負罪朔裔,書奏不許。邕若蒙許,亦必不能。邕才不逾張衡,衡本豈有遺思也?則有器無書,觀不能悟。焯
 今立術,改正舊渾。又以二至之影,定去極晷漏,並天地高遠,星辰運周,所宗有本,皆有其率。祛今賢之巨惑,稽往哲之群疑,豁若雲披,朗如霧散。為之錯綜,數卷已成,待得影差,謹更啟送。



 又云:「《周官》夏至日影,尺有五寸。張衡、鄭玄、王番、陸績先儒等,皆以為影千里差一寸。言南戴日下萬五千里,表影正同,天高乃異。考之算法,必為不可。寸差千里,亦無典說,明為意斷,事不可依。今交、愛之州,表北無影,計無萬里,南過戴日。是千里一寸,非其實差。焯今說渾,以道為率,道里不定,得差乃審。既大聖之年,升平之日,厘改群謬,斯正其時。請一水工並解算
 術士,取河南、北平地之所,可量數百里,南北使正。審時以漏,平地以繩,隨氣至分,同日度影。得其差率,里即可知。則天地無所匿其形,辰象無所逃其數,超前顯聖,效象除疑。請勿以人廢言。」不用。至大業三年,敕諸郡測影,而焯尋卒,事遂寢廢。



 地中《周禮·大司徒職》:「以土圭之法,測土深,正日景,以求地中。」此則渾天之正說,立儀象之大本。故云:「日南則景短多暑,日北則景長多寒,日東則景夕多風,日西則景朝多陰。日至之景,尺有五寸,謂之地中。天地之所合也。四時
 之所交也,風雨之所會也,陰陽之所和也。然則百物阜安,乃建王國焉。」又《考工記·匠人》:「建國,水地以縣。置S以縣,眡以景。為規,識日出之景與日入之景。晝參諸日中之景,夜考之極星,以正朝夕。」案土圭正影,經文闕略,先儒解說,又非明審。祖恆錯綜經注,以推地中。其法曰:「先驗昏旦,定刻漏,分辰次。乃立儀表於準平之地,名曰南表。漏刻上水,居日之中,更立一表於南表影末,名曰中表。夜依中表,以望北極樞,而立北表,令參相直。三表皆以懸準定,乃觀。三表直者,其立表之地,即當子午之正。三表曲者,地偏僻。每觀中表,以知所偏。中表在西,則立
 表處在地中之西,當更向東求地中。若中表在東,則立表處在地中之東也,當更向西求地中。取三表直者,為地中之正。又以春秋二分之日,旦始出東方半體,乃立表於中表之東,名曰東表。令東表與日及中表參相直。視日之夕,日入西方半體,又立表於中表之西,名曰西表。亦從中表西望西表及日,參相直。乃觀三表直者,即地南北之中也。若中表差近南,則所測之地在卯酉之南。



 中表差在北,則所測之地在卯酉之北。進退南北,求三表直正東西者,則其地處中,居卯酉之正也。」



 晷影
 昔者周公測晷影於陽城,以參考歷紀。其於《周禮》,在《大司徒之職》:「以土圭之法,測土深,正日景,以求地中。日至之景,尺有五寸,則天地之所合,四時之所交。百物阜安,乃建王國。」然則日為陽精,玄象之著然者也。生靈因之動息,寒暑由其遞代。觀陰陽之升降,揆天地之高遠,正位辨方,定時考閏,莫近於茲也。古法簡略,旨趣難究,術家考測,互有異同。先儒皆云:「夏至立八尺表於陽城,其影與土圭等。」案《尚書考靈曜》稱:「日永,景尺五寸;日短,景尺三寸。」《易通卦驗》曰:「冬至之日,樹八尺之表,日中視其晷景長短,以占和否。夏至景一尺四寸八分,冬至一丈
 三尺。」《周髀》云:「成周土中,夏至景一尺六寸,冬至景一丈三尺五寸。」劉向《鴻範傳》曰:「夏至景長一尺五寸八分,冬至一丈三尺一寸四分,春秋二分,景七尺三寸六分。」後漢《四分歷》、魏《景初歷》、宋《元嘉歷》、大明祖沖之歷,皆與《考靈曜》同。漢、魏及宋,所都皆別,四家歷法,候影則齊。且緯候所陳,恐難依據。劉向二分之影,直以率推,非因表候定其長短。然尋晷影尺丈,雖有大較,或地域不改,而分寸參差,或南北殊方,而長短維一。蓋術士未能精驗,馮占所以致乖。今刪其繁雜,附於此云。



 梁天監中,祖恆造八尺銅表,其下與圭相連。圭上為溝,置水,以取平正。



 揆
 測日晷,求其盈縮。至大同十年,太史令虞廣刂又用九尺表格江左之影。夏至一尺三寸二分,冬至一丈三尺七分,立夏、立秋二尺四寸五分,春分、秋分五尺三寸九分。陳氏一代,唯用梁法。齊神武以洛陽舊器並徙鄴中,以暨文宣受終,竟未考驗。至武平七年,訖干景禮始薦劉孝孫、張孟賓等於後主。劉、張建表測影,以考分至之氣。草創未就,仍遇朝亡。周自天和以來,言歷者紛紛復出。亦驗二至之影,以考歷之精粗。及高祖踐極之後,大議造歷。張胄玄兼明揆測,言日長之瑞。有詔司存,而莫能考決。至開皇十九年,袁充為太史令,欲成胄玄舊事,復
 表曰:「隋興已後,日景漸長。開皇元年冬至之影,長一丈二尺七寸二分,自爾漸短。至十七年冬至影,一丈二尺六寸三分。四年冬至,在洛陽測影,長一丈二尺八寸八分。二年夏至影,一尺四寸八分,自爾漸短。至十六年夏至影,一尺四寸五分。其十八年冬至,陰雲不測。元年、十七年、十八年夏至,亦陰雲不測。《周官》以土圭之法正日影,日至之影,尺有五寸。鄭玄云:『冬至之景,一丈三尺。』今十六年夏至之影,短於舊五分,十七年冬至之影,短於舊三寸七分。日去極近,則影短而日長;去極遠,則影長而日短。行內道則去極近,行外道則去極遠。《堯典》云:『日
 短星昴,以正仲冬。』據昴星昏中,則知堯時仲冬,日在須女十度。以歷數推之,開皇以來冬至,日在斗十一度,與唐堯之代,去極俱近。謹案《元命包》云:『日月出內道,〔璣得其常,天帝崇靈,聖王初功。』京房《別對》曰:『太平日行上道,升平日行次道,霸代日行下道。』伏惟大隋啟運,上感乾元,影短日長,振古希有。」



 是時廢庶人勇,晉王廣初為太子,充奏此事,深合時宜。上臨朝謂百官曰:「景長之慶,天之祐也。今太子新立,當須改元,宜取日長之意,以為年號。」由是改開皇二十一年為仁壽元年。此後百工作役,並加程課,以日長故也。皇太子率百官詣闕陳賀。案
 日徐疾盈縮無常,充等以為祥瑞,大為議者所貶。



 又《考靈曜》、《周髀》張衡《靈憲》及鄭玄注《周官》,並云:「日影於地,千里而差一寸。」案宋元嘉十九年壬午,使使往交州測影。夏至之日,影出表南三寸二分。何承天遙取陽城,云夏至一尺五寸。計陽城去交州,路當萬里,而影實差一尺八寸二分。是六百里而差一寸也。又當梁大同中,二至所測,以八尺表率取之,夏至當一尺一寸七分強。後魏信都芳注《周髀四術》,稱永平元年戊子,當梁天監之七年,見洛陽測影,又見公孫崇集諸朝士,共觀秘書影。同是夏至日,其中影皆長一尺五寸八分。以此推之,金陵去
 洛,南北略當千里,而影差四寸。則二百五十里而影差一寸也。況人路迂回,山川登降,方於鳥道,所校彌多,則千里之言,未足依也。其揆測參差如此,故備論之。



 漏刻昔黃帝創觀漏水,制器取則,以分晝夜。其後因以命官,《周禮》挈壺氏則其職也。其法,總以百刻,分於晝夜。冬至晝漏四十刻,夜漏六十刻。夏至晝漏六十刻,夜漏四十刻。春秋二分,晝夜各五十刻。日未出前二刻半而明,既沒後二刻半乃昏。減夜五刻,以益晝漏,謂之昏旦。漏刻皆隨氣增損。冬夏二至之間,晝夜長短,凡差二十刻。每
 差一刻為一箭。冬至互起其首,凡有四十一箭。晝有朝,有禺,有中,有晡,有夕。夜有甲、乙、丙、丁、戊。昏旦有星中。每箭各有其數,皆所以分時代守,更其作役。



 漢興,張蒼因循古制,猶多疏闊。及孝武考定星歷,下漏以追天度,亦未能盡其理。劉向《鴻範傳》記武帝時所用法云:「冬夏二至之間,一百八十餘日,晝夜差二十刻。」大率二至之後,九日而增損一刻焉。至哀帝時,又改用晝夜一百二十刻,尋亦寢廢。至王莽竊位,又遵行之。光武之初,亦以百刻九日加減法,編於《甲令》,為《常符漏品》。至和帝永元十四年,霍融上言:「官歷率九日增減一刻,不與天相應。或
 時差至二刻半,不如夏歷漏刻,隨日南北為長短。」乃詔用夏歷漏刻。依日行黃道去極,每差二度四分,為增減一刻。凡用四十八箭,終於魏、晉,相傳不改。



 宋何承天以月蝕所在,當日之衡,考驗日宿,知移舊六度。冬至之日,其影極長,測量晷度,知冬至移舊四日。前代諸漏,春分晝長,秋分晝短,差過半刻。皆由氣日不正,所以而然。遂議造漏法。春秋二分,昏旦晝夜漏各五十五刻。齊及梁初,因循不改。至天監六年,武帝以晝夜百刻,分配十二辰,辰得八刻,仍有餘分。



 乃以晝夜為九十六刻,一辰有全刻八焉。至大同十年,又改用一百八刻。依《尚書考靈
 曜》晝夜三十六頃之數,因而三之。冬至晝漏四十八刻,夜漏六十刻。夏至晝漏七十刻,夜漏三十八刻。春秋二分,晝漏六十刻,夜漏四十八刻。昏旦之數各三刻。先令祖恆為《漏經》,皆依渾天黃道日行去極遠近,為用箭日率。陳文帝天嘉中,亦命舍人硃史造漏,依古百刻為法。周、齊因循魏漏。晉、宋、梁大同,並以百刻分於晝夜。



 隋初,用周朝尹公正、馬顯所造《漏經》。至開皇十四年,鄜州司馬袁充上晷影漏刻。充以短影平儀,均布十二辰,立表,隨日影所指辰刻,以驗漏水之節。十二辰刻,互有多少,時正前後,刻亦不同。其二至二分用箭辰刻之法,今列之
 云。



 冬至:日出辰正,入申正,晝四十刻,夜六十刻。



 子、丑、亥各二刻,寅、戌各六刻,卯、酉各十三刻,辰、申各十四刻,巳、未各十刻,午八刻。



 右十四日改箭。



 春秋二分:日出卯正,入酉正,晝五十刻,夜五十刻。



 子四刻,醜、亥七刻,寅、戌九刻,卯、酉十四刻,辰、申九刻,巳、未七刻,午四刻。



 右五日改箭。



 夏至:日出寅正,入戌正,晝六十刻,夜四十刻。



 子八刻,醜、亥十刻,寅、戌十四刻,卯、酉十三刻,辰、申六刻,巳、未二刻,午二刻。



 右一十九日,加減一刻,改箭。



 袁充素不曉渾天黃道去極之數,茍役私智,變改舊章,其於施用,未為精密。



 開皇十七年,張胄玄用後魏渾天鐵儀,測知春秋二分,日出卯酉之北,不正當中。與何承天所測頗同,皆日出卯三刻五十五分,入酉四刻二十五分。晝漏五十刻十一分,夜漏四十九刻四十分,晝夜差六十分刻之四十。仁壽四年,劉焯上《皇極歷》,有日行
 遲疾,推二十四氣,皆有盈縮定日。春秋分定日,去冬至各八十八日有奇,去夏至各九十三日有奇。二分定日,晝夜各五十刻。又依渾天黃道,驗知冬至夜漏五十九刻、一百分刻之八十六,晝漏四十刻一十四分,夏至晝漏五十九刻八十六分,夜漏四十刻一十四分。冬夏二至之間,晝夜差一十九刻、一百分刻之七十二。胄玄及焯漏刻,並不施用。然其法制,皆著在歷術,推驗加時,最為詳審。



 大業初,耿詢作古欹器,以漏水注之,獻於煬帝。帝善之,因令與宇文愷依後魏道士李蘭所修道家上法稱漏制,造稱水漏器,以充行從。又作候影分箭上水
 方器,置於東都乾陽殿前鼓下司辰。又作馬上漏刻,以從行辨時刻。揆日晷,下漏刻,此二者,測天地正儀象之本也。晷漏沿革,今古大殊,故列其差,以補前闕。



 經星中宮北極五星,鉤陳六星,皆在紫宮中。北極,辰也。其紐星,天之樞也。天運無窮,三光迭耀,而極星不移。故曰:「居其所而眾星共之。」賈逵、張衡、蔡邕、王蕃、陸績,皆以北極紐星為樞,是不動處也。祖恆以儀準候不動處,在紐星之末,猶一度有餘。北極大星,太一之座也。第一星主月,太子也。第二星主日,帝王也。第三星主五星,庶子也。所謂第
 二星者,最赤明者也。北極五星,最為尊也。



 中星不明,主不用事。右星不明,太子憂。鉤陳,後宮也,太帝之正妃也,太帝之坐也。北四星曰女御宮,八十一御妻之象也。鉤陳口中一星,曰天皇太帝。其神曰耀魄寶,主御群靈,秉萬神圖。抱極樞四星曰四輔,所以輔佐北極,而出度授政也。



 太帝上九星曰華蓋,蓋所以覆蔽太帝之坐也。又九星直曰杠。蓋下五星曰五帝內坐,設敘順,帝所居也。客犯紫宮中坐,大臣犯主。華蓋杠旁六星曰六甲,可以分陰陽而紀節候,故在帝旁,所以布政教而授人時也。極東一星曰柱下史,主記過。古者有左右史,此之象也。
 柱史北一星曰女史,婦人之微者,主傳漏。故漢有侍史。傳舍九星在華蓋上,近河,賓客之館,主胡人入中國。客星守之,備奸使,亦曰胡兵起。傳舍南河中五星曰造父,御官也,一曰司馬,或曰伯樂。星亡,馬大貴。西河中九星如鉤狀,曰鉤星,伸則地動。天一一星,在紫宮門右星南,天帝之神也,主戰鬥,知人吉兇者也。太一一星,在天一南,相近,亦天帝神也,主使十六神,知風雨水旱,兵革饑饉,疾疫災害所生之國也。



 紫宮垣十五星,其西蕃七,東蕃八,在北斗北。一曰紫微,太帝之坐也,天子之常居也,主命,主度也。一曰長垣,一曰天營,一曰旗星,為蕃衛,備
 蕃臣也。



 宮闕兵起,旗星直,天子出,自將宮中兵。東垣下五星曰天柱,建政教,懸圖法之所也。常以朔望日懸禁令於天柱,以示百司。《周禮》以正歲之月,懸法象魏,此之類也。門內東南維五星曰尚書,主納言,夙夜諮謀,龍作納言,此之象也。尚書西二星曰陰德、陽德,主周急振無。宮門左星內二星曰大理,主平刑斷獄也。門外六星曰天床,主寢舍,解息燕休。西南角外二星曰內廚,主六宮之飲食,主後夫人與太子宴飲。東北維外六星曰天廚,主盛饌。



 北斗七星,輔一星在太微北,七政之樞機,陰陽之元本也,故運乎天中,而臨制四方,以建四時而均五
 行也。魁四星為〔璣,杓三星為玉衡。又象號令之主,又為帝車,取乎運動之義也。又魁第一星曰天樞,二曰〔,三曰璣,四曰權,五曰玉衡,六曰開陽,七曰搖光。一至四為魁,五至七為杓。樞為天,〔為地,璣為人,權為時,玉衡為音,開陽為律,搖光為星。石氏云:「第一曰正星,主陽德,天子之象也。二曰法星,主陰刑,女主之位也。三曰令星,主禍害也。四曰伐星,主天理,伐無道。五曰殺星,主中央,助四旁,殺有罪。六曰危星,主天倉五穀。七曰部星,亦曰應星,主兵。」又云:「一主天,二主地,三主火,四主水,五主土,六主木,七主金。」又曰:「一主秦,二主楚,三主梁,四主吳,五
 主趙,六主燕,七主齊。」



 魁中四星,為貴人之牢,曰天理也。輔星傅乎開陽,所以佐鬥成功也。又曰:「主危正,矯不平。」又曰:「丞相之象也。」七政星明,其國昌。不明,國殃。



 斗旁欲多星則安,斗中少星則人恐上,天下多訟法者。無星二十日。有輔星明而鬥不明,臣強主弱。半明輔不明,主強臣弱也。杓南三星及魁第一星,皆曰三公,宣德化,調七政,和陰陽之官也。



 文昌六星,在北斗魁前,天之六府也,主集計天道。一曰上將,大將建威武。



 二曰次將,尚書正左右。三曰貴相,太常理文緒。四曰司祿、司中,司隸賞功進。



 五曰司命、司怪,
 太史主滅咎。六曰司寇,大理佐理寶。所謂一者,起北斗魁前,近內階者也。明潤,大小齊,天瑞臻。



 文昌北六星曰內階,天皇之陛也。相一星在北斗南。相者總領百司而掌邦教,以佐帝王,安邦國,集眾事也。其明吉。太陽守一星,在相西,大將大臣之象也,主戒不虞,設武備也。非其常,兵起。西北四星曰勢。勢,腐刑人也。天牢六星在北斗魁下,貴人之牢也,主愆過,禁暴淫。



 太微,天子庭也,五帝之坐也,亦十二諸侯府也。其外蕃,九卿也。一曰太微為衡。衡,主平也。又為天庭,理法平辭,監升授德,列宿受符,諸神考節,舒情稽疑也。南蕃中二星間曰端門。東曰左
 執法,廷尉之象也。西曰右執法,御史大夫之象也。執法,所以舉刺兇奸者也。左執法之東,左掖門也。右執法之西,右掖門也。東蕃四星,南第一曰上相,其北東太陽門也。第二星曰次相,其北中華東門也。



 第三星曰次將,其北東太陰門也。第四星曰上將。所謂四輔也。西蕃四星:南第一星曰上將,其北西太陽門也。第二星曰次將,其北中華西門也。第三曰次相,其北西太陰門也。第四星曰上相。亦四輔也。東西蕃有芒及搖動者,諸侯謀天子也。執法移則刑罰尤急。月、五星所犯中坐,成刑。月、五星入太微軌道,吉。



 西南角外三星曰明堂,天子布政之宮
 也。明堂西三星曰靈臺,觀臺也。主觀雲物,察符瑞,候災變也。左執法東北一星曰謁者,主贊賓客也。謁者東北三星曰三公內坐,朝會之所居也。三公北三星曰九卿內坐,主治萬事。九卿西五星曰內五諸侯,內侍天子,不之國者也。闢雍之禮得,則太微諸侯明。



 黃帝坐一星,在太微中,含樞紐之神也。天子動得天度,止得地意,從容中道,則太微五帝坐明,坐以光。黃帝坐不明,人主求賢士以輔法,不然則奪勢。又曰太微五坐小弱青黑,天子國亡。四帝坐四星,四星夾黃帝坐。東方星,蒼帝靈威仰之神也。南方星,赤帝熛怒之神也。西方星,白帝招距之
 神也。北方星,黑帝葉光紀之神也。



 五帝坐北一星曰太子,帝儲也。太子北一星曰從官,侍臣也。帝坐東北一星曰幸臣。屏四星在端門之內,近右執法。屏,所以壅蔽帝庭也。執法主刺舉,臣尊敬君上,則星光明潤澤。郎位十五星,在帝坐東北,一曰依烏,郎位也。周官之元士,漢官之光祿、中散、諫議、議郎、三署郎中,是其職也。或曰今之尚書也。郎位主衛守也。其星明,大臣有劫主。又曰客犯上。其星不具,後死,幸臣誅。客星入之,大臣為亂。郎將一星在郎位北,主閱具,所以為武備也。武賁一星,在太微西蕃北,下臺南,靜室旄頭之騎官也。常陳七星,如畢狀,
 在帝坐北,天子宿衛武賁之土,以設強毅也。星搖動,天子自出,明則武兵用,微則武兵弱。



 三臺六星,兩兩而居,起文昌,列招搖、太微。一曰天柱,三公之位也。在天曰三臺,主開德宣符也。西近文昌二星曰上臺,為司命,主壽。次二星曰中臺,為司中,主宗。東二星曰下臺,為司祿,主兵,所以昭德塞違也。又曰三臺為天階,太一躡以上下。一曰泰階,上星為天子,下星為女主;中階,上星為諸侯三公,下星為卿大夫;下階,上星為士,下星為庶人。所以和陰陽而理萬物也。其星有變,各以所主占人。君臣和集,如其常度。



 南四星曰內平,近職執法平罪之官也。中臺之
 北一星曰大尊,貴戚也。下臺南一星曰武賁,衛官也。



 攝提六星,直斗杓之南,主建時節,伺禨祥。攝提為楯,以夾擁帝席也,主九卿。明大,三公恣,客星入之,聖人受制。西三星曰周鼎,主流亡。大角一星,在攝提間。大角者,天王座也。又為天棟,正經紀。北三星曰帝席,主宴獻酬酢。梗河三星,在大角北。梗河者,天矛也。一曰天鋒,主胡兵。又為喪,故其變動應以兵喪也。星亡,其國有兵謀。招搖一星在其北,一曰矛楯,主胡兵。占與梗河略相類也。招搖與北斗杓間曰天庫。星去其所,則有庫開之祥也。招搖欲與棟星、梗河、北斗相應,則胡常來受命於中國。招搖
 明而不正,胡不受命。玄戈二星,在招搖北。



 玄戈所主,與招搖同。或云主北夷。客星守之,胡大敗。天槍三星,在北斗杓東。



 一曰天鉞,天之武備也。故在紫宮之左,所以禦難也。女床三星,在其北,後宮御也,主女事。天棓五星,在女床北,天子先驅也,主忿爭與刑罰,藏兵,亦所以禦難也。槍棓皆以備非常也。一星不具,國兵起。



 東七星曰扶筐,盛桑之器,主勸蠶也。七公七星,在招搖東,天之相也,三公之象,主七政。貫索九星在其前,賤人之牢也。一曰連索,一曰運營,一曰天牢,主法律,禁暴強也。牢口一星為門,欲其開也。九星皆明,天下獄煩。七星見,小赦;五星,
 大赦。動則斧金質用,中空則更元。《漢志》云十五星。天紀九星,在貫索東,九卿也。九河主萬事之紀,理怨訟也。明則天下多辭訟,亡則政理壞,國紀亂,散絕則地震山崩。織女三星,在天紀東端,天女也,主果瓜絲帛珍寶也。王者至孝,神祗咸喜,則織女星俱明,天下和平。大星怒角,布帛貴。東足四星曰漸臺,臨水之臺也。主晷漏律呂之事。西之五星曰輦道,王者嬉游之道也,漢輦道通南、北宮象也。



 左右角間二星曰平道之官。平道西一星曰進賢,主卿相舉逸才。角北二星曰天田。亢北六星曰亢池。亢,舟航也;池,水也。主送往迎來。氐北一星曰天乳,主甘露。
 房中道一星曰歲,守之,陰陽平。房西二星南北列,曰天福,主乘輿之官,若《禮》巾車、公車之政。主祠事。東咸、西咸各四星,在房、心北,日月五星之道也。房之戶,所以防淫佚也。星明則吉,暗則兇。月、五星犯守之,有陰謀。東咸西三星,南北列,曰罰星,主受金贖。鍵閉一星,在房東北,近鉤鈐,主關鑰。



 天市垣二十二星,在房心東北,主權衡,主聚眾。一曰天旗庭,主斬戮之事也。



 市中星眾潤澤則歲實,星稀則歲虛。熒惑守之,戮不忠之臣。又曰,若怒角守之,戮者臣殺主。慧星除之,為徙市易都。客星入之,兵大起,出之有貴喪。市中六星臨箕,曰市樓市府也,主市價
 律度。其陽為金錢,其陰為珠玉。變見,各以所主占之。北四星曰天斛,主量者也。斛西北二星曰列肆,主寶玉之貨。市門左星內二星曰車肆,主眾賈之區。



 帝坐一星,在天市中,候星西,天庭也。光而潤則天子吉,威令行。微小兇,大人當之。侯一星,在帝坐東北,主伺陰陽也。明大輔臣強,四夷開。候細微則國安,亡則主失位,移則主不安。宦者四星,在帝坐西南,侍主刑餘之人也。星微則吉,明則兇,非其常,宦者有憂。斗五星,在宦者南,主平量。仰則天下斗斛不平,覆則歲穰。宗正二星,在帝坐東南,宗大夫也。慧星守之,若失色,宗正有事。客星守動,則天子親
 屬有變。客星守之,貴人死。宗星二,在候星東,宗室之象,帝輔血脈之臣也。客星守之,宗人不和。東北二星曰帛度,東北二星曰屠肆,各主其事。



 天江四星在尾北,主太陰。江星不具,天下津河關道不通。明若動搖,大水出,大兵起。參差則馬貴。熒惑守之,有立王。客星入之。河津絕。



 天籥八星,在南斗杓西,主關閉。建星六星,在南斗北,亦曰天旗,天之都關也。為謀事,為天鼓,為天馬。南二星,天庫也。中央二星,市也,鈇金質也。上二星,旗跗也。斗建之間,三光道也。星動則人勞。月暈之,蛟龍見,牛馬疫。月、五星犯之,大臣相譖,臣謀主;亦為關梁不通,有大水。東南四星曰
 狗國,主鮮卑、烏丸、沃且。熒惑守之,外夷為變。太白逆守之,其國亂。客星犯守之,有大盜,其王且來。狗國北二星曰天雞,主候時。天弁九星在建星北,市官之長也。主列肆圜闠,若市籍之事,以知市珍也。星欲明,吉。慧星犯守之,糴貴,囚徒起兵。



 河鼓三星,旗九星,在牽牛北,天鼓也,主軍鼓,主鈇金質。一曰三武,主天子三將軍。中央大星為大將軍,左星為左將軍,右星為右將軍。左星,南星也,所以備關梁而距難也,設守阻險,知謀徵也。旗即天鼓之旗,所以為旌表也。左旗九星,在鼓左旁。鼓欲正直而明,色黃光澤,將吉;不正,為兵憂也。星怒馬貴,動則兵起,曲
 則將失計奪勢。旗星戾,亂相陵。旗端四星南北列,曰天桴。桴,鼓桴也。星不明,漏刻失時。前近河鼓,若桴鼓相直,皆為桴鼓用。



 離珠五星,在須女北,須女之藏府也,女子之星也。星非故,後宮亂。客星犯之,後宮兇。虛北二星曰司命,北二星曰司祿,又北二星曰司危,又北二星曰司非。



 司命主舉過行罰,滅不祥。司祿增年延德,故在六宗北。犯司危,主驕佚亡下。司非以法多就私。瓠瓜五星,在離珠北,主陰謀,主後宮,主果食。明則歲熟,微則歲惡,後失勢。非其故,則山搖,穀多水。旁五星曰敗瓜,主種。天津九星,梁,所以度神通四方也。一星不備,津關道不通。星
 明動則兵起如流沙,死人亂麻。微而參差,則馬貴若死。星亡,若從河水為害,或曰水賊稱王也。東近河邊七星曰車府,主車之官也。車府東南五星曰人星,主靜眾庶,柔遠能邇。一曰臥星,主防淫。



 其南三星內析,東南四星曰杵臼,主給軍糧。客星入之,兵起,天下聚米。天津北四星如衡狀,曰奚仲,古車正也。



 騰蛇二十二星,在營室北,天蛇星主水蟲。星明則不安,客星守之,水雨為災,水物不收。王良五星,在奎北,居河中,天子奉車御官也。其四星曰天駟,旁一星曰王良,亦曰天馬。其星動,為策馬,車騎滿野。亦曰王良梁,為天橋,主御風雨水道,故或占津
 梁。其星移,有兵,亦曰馬病。客星守之,橋不通。前一星曰策,王良之御策也,主天子僕,在王良旁。若移在馬後,是謂策馬,則車騎滿野。閣道六星,在王良前,飛道也。從紫宮至河,神所乘也。一曰閣道,主道里,天子游別宮之道也。亦曰閣道,所以捍難滅咎也。一曰王良旗,一曰紫宮旗,亦所以為旌表,而不欲其動搖。旗星者,兵所用也。傅路一星,在閣道南,旁別道也。備閣道之敗,復而乘之也。一曰太僕,主御風雨,亦游從之義也。東壁北十星曰天廄,主馬之官,若今驛亭也,主傳令置驛,逐漏馳騖,謂其行急疾,與晷漏競馳。



 天將軍十二星,在婁北,主武兵。中
 央大星,天之大將也。外小星,吏士也。



 大將星搖,兵起,大將出。小星不具,兵發。南一星曰軍南門,主誰何出入。太陵八星,在胃北。陵者,墓也。太陵卷舌之口曰積京,主大喪也。積京中星絕,則諸侯有喪,民多疾,兵起,粟聚。少則粟散。星守之,有土功。太陵中一星曰積尸,明則死人如山。天船九星,在太陵北,居河中。一曰舟星,主度,所以濟不通也,亦主水旱。不在漢中,津河不通。中四星欲其均明,即天下大安。不則兵若喪。客彗星出入之,為大水,有兵。中一星曰積水,候水災。昴西二星曰天街,三光之道,主伺候關梁中外之境。天街西一星曰月。卷舌六星在
 北,主口語,以知佞讒也。曲者吉,直而動,天下有口舌之害。中一星曰天讒,主巫醫。



 五車五星,三柱九星,在畢北。五車者,五帝車舍也,五帝坐也,主天子五兵,一曰主五穀豐耗。西北大星曰天庫,主太白,主秦。次東北星曰獄,主辰星,主燕、趙。次東星曰天倉,主歲星,主魯、衛。次東南星曰司空,主填星,主楚。次西南星曰卿星,主熒惑,主魏。五星有變,皆以其所主而占之。三柱,一曰三泉,一曰休,一曰旗。五車星欲均明,闊狹有常也。天子得靈臺之禮,則五車、三柱均明。



 中有五星曰天潢。天潢南三星曰咸池,魚囿也。月、五星入天潢,兵起,道不通,天下亂,易政。咸
 池明,有龍墮死,猛獸及狼害人,若兵起。



 五車南六星曰諸王,察諸侯存亡。西五星曰厲石,金若客星守之,兵動。北八星曰八穀,主候歲。八穀一星亡,一穀不登。天關一星,在五車南,亦曰天門,日月所行也,主邊事,主開閉。芒角,有兵。五星守之,貴人多死。



 東井鉞前四星曰司怪,主候天地日月星辰變異,及鳥獸草木之妖,明主聞災,修德保福也。司怪西北九星曰坐旗,君臣設位之表也。坐旗西四星曰天高,臺榭之高,主遠望氣象。天高西一星曰天河,主察山林妖變。南河、北河各三星,夾東井。



 一曰天高天之闕門,主關梁。南河曰南戍,一曰南宮,一曰陽
 門,一曰越門,一曰權星,主火。北河一曰北戍,一曰北宮,一曰陰門,一曰胡門,一曰衡星,主水。



 兩河戍間,日月五星之常道也。河戍動搖,中國兵起。南河三星曰闕丘,主宮門外象魏也。五諸侯五星,在東井北,主刺舉,戒不虞。又曰理陰陽,察得失。亦曰主帝心。一曰帝師,二曰帝友,三曰三公,四曰博士,五曰太史。此五者常為帝定疑議。星明大潤澤,則天下大治,角則禍在中。五諸侯南三星曰天樽,主盛饘粥,以給酒食之正也。積薪一星,在積水東,供給庖廚之正也。水位四星,在東井東,主水衡。客星若水火守犯之,百川流溢。



 軒轅十七星,在七星北。軒轅,
 黃帝之神,黃龍之體也。後妃之主,士職也。



 一曰東陵,一曰權星,主雷雨之神。南大星,女主也。次北一星,妃也。次,將軍也。其次諸星,皆次妃之屬也。女主南小星,女御也。左一星少民,少後宗也。右一星大民,太后宗也。欲其色黃小而明也。軒轅右角南三星曰酒旗,酒官之旗也,主饗宴飲食。五星守酒旗,天下大甫,有酒肉財物,賜若爵宗室。酒旗南二星曰天相,丞相之象也。軒轅西四星曰爟,爟者烽火之爟也,邊亭之警候。



 爟北四星曰內平。少微四星,在太微西,士大夫之位也。一名處士,亦天子副主,或曰博士官。一曰主衛掖門。南第一星處士,第二星
 議士,第三星博士,第四星大夫。明大而黃,則賢士舉也。月、五星犯守之,處士、女主憂,宰相易。南四星曰長垣,主界域及胡夷。熒惑入之,胡入中國。太白入之,九卿謀。



\end{pinyinscope}