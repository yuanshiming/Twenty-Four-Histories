\article{卷二十七志第二十二 百官中}

\begin{pinyinscope}

 後齊制官,多循後魏,置太師、太傅、太保,是為三師,擬古上公,非勛德崇者不居。次有大司馬、大將軍,是為二大,並典司武事。次置太尉、司徒、司空,是為三公。三師、二大、三公府,三門,當中開黃閤,設內屏。各置長史,司馬,諮議參軍,從事中郎,掾屬,主簿,錄事,功曹,記室、戶曹、金曹、中兵、外兵、騎兵、長流、城局、刑獄等參軍事,東西閤祭酒及
 參軍事,法、墨、田、水、鎧、集、士等曹行參軍,兼左戶右戶行參軍,長兼行參軍,參軍、督護等員。司徒則加有左右長史。三公下次有儀同三司。加開府者,亦置長史已下官屬,而減記室、倉、城局、田、水、鎧、士等七曹,各一人。其品亦每官下三府一階。三師、二大置佐史,則同太尉府。乾明中,又置丞相。河清中,分為左右,亦各置府僚云。



 特進,左右光祿,金紫、銀青等光祿大夫,用人俱以舊德就閑者居之。自一品已下,從九品已上,又有驃騎、車騎、衛、四征、四鎮、中軍、鎮軍、撫軍、翊軍、四安、冠軍、輔國、龍驤、鎮遠、安遠、建忠、建節、中堅、中壘、振威、奮威、廣德、弘義、折
 沖、制勝、伏波、陵江、輕車、樓船、勁武、昭勇、明威、顯信、度遼、橫海、逾岷、越嶂、戎昭、武毅、雄烈、恢猛、揚麾、曜鋒、蕩邊,開城、靜漠、綏戎、平越、殄夷、飛騎、隼擊、武牙、武奮、清野、橫野、偏、裨等將軍,以褒賞勛庸。



 尚書省,置令、僕射,吏部、殿中、祠部、五兵、都官、度支等六尚書。又有錄尚書一人,位在令上,掌與令同,但不糾察。令則彈糾見事,與御史中丞更相廉察。僕射職為執法,置二則為左、右僕射,皆與令同。左糾彈,而右不糾彈。錄、令、僕射,總理六尚書事,謂之都省。其屬官,左丞、掌吏部、考功、主爵、殿中、儀曹、三公、祠部、主客、左右中兵、左右外兵、都官、二千石、度支、左右戶十七曹,並彈糾見事。又主管轄臺
 中,有違失者,兼糾駁之。右丞各一人。掌駕部、虞曹、屯田、起部、都兵、比部、水部、膳部、倉部、金部、庫部十一曹。亦管轄臺中,又主凡諸用度雜物、脂、燈、筆、墨、幃帳。唯不彈糾,餘悉與左同。並都令史八人,共掌其事。其六尚書,分統列曹。吏部統吏部、掌褒崇、選補等事。考功、掌考第及秀孝貢士等事。主爵掌封爵等事。三曹。殿中統殿中、掌駕行百官留守名帳,宮殿禁衛,供御衣倉等事。儀曹、掌吉兇禮制事。三公、掌五時讀時令,諸曹囚帳,斷罪,赦日建金雞等事。駕部掌車輿、牛馬廄牧等事。四曹。祠部統祠部、掌祠部醫藥,死喪贈賜等事。主客、掌諸蕃雜客等事。虞曹、掌地圖,山川遠近,園囿田獵,肴膳雜味等事。屯田、掌藉田、諸州屯田等事。起部掌諸興造非等事。五曹。祠部,無尚書則右僕射攝。五兵統左中兵、掌諸郡督告身、諸宿衛官等事。右中兵、掌畿內丁帳、事力、蕃兵等事。左外兵、掌河南及潼關已東諸州丁帳,及發召徵兵等事。右外兵、
 掌河北及潼關已西諸州,所典與左外同。都兵掌鼓吹、太樂、雜戶等事。五曹。都官統都官、掌畿內非違得失事。二千石、掌畿外得失等事。比部、掌詔書律令勾檢等事。水部、掌舟船、津梁,公私水事。膳部掌侍官百司禮食肴饌等事。五曹。度支統度支、掌計會,凡軍國損益、事役糧廩等事。倉部、掌諸倉帳出入等事。左戶、掌天下計帳、戶籍等事。右戶、掌天下公私田宅租調等事。金部、掌權衡量度、外內諸庫藏文帳等事。庫部掌凡是戎仗器用所須事。六曹。



 凡二十八曹。吏部、三公,郎中各二人,餘並一人。凡三十郎中。吏部、儀曹、三公、虞曹、都官、二千石、比部、左戶、各量事置掌故主事員。



 門下省,掌獻納諫正,及司進御之職。侍中、給事黃門侍郎各六人,錄事四人,通事令史、主事令史八人。統局六。領左右局,領左右各二人,掌知
 硃華閣內諸事。



 宣傳已下,白衣齋子已上,皆主之。左右直長四人。尚食局,典御二人,總知御膳事。丞、監各四人。尚藥局,典御及丞各二人,總知御藥事。侍御師、尚藥監各四人。主衣局、都統、子統各二人。掌御衣服玩等事。齋帥局,齋帥四人。掌鋪設灑掃事。殿中局,殿中監四人。掌駕前奏引行事,制請修補。東耕則進耒耜。



 中書省,管司王言,及司進御之音樂。監、令各一人,侍郎四人。並司伶官西涼部直長、伶官西涼四部、伶官龜茲四部、伶官清商部直長、伶官清商四部。又領舍人省,掌署敕行下,宣旨勞問。中書舍從、主書各十人。



 秘書省,典司經籍。監、丞各一人,郎中四人,校書郎十二
 人,正字四人。又領著作省,郎二人,佐郎八人,校書郎二人。



 集書省,掌諷議左右,從容獻納。散騎常侍、通直散騎常侍各六人,諫議大夫七人,散騎侍郎六人,員外散騎常侍二十人,通直散騎侍郎六人,給事中六人,員外散騎侍郎一百二十人,奉朝請二百四十人。又領起居省,散騎常侍、通直散騎常侍、散騎侍郎、通直散騎侍郎各一人,校書郎二人。



 中侍中省,掌出入門閤。中侍中二人,中常侍中、給事中各四人。又有中尚藥典御及丞,並中謁者僕射,各二人。
 中尚食局,典御、丞各二人,監四人。內謁者局,統、丞各一人。



 御史臺,掌察糾彈劾。中丞一人,治書侍御史二人,侍御史八人,殿中侍御史、檢校御史各十二人,錄事四人。領符節署,令一人,符璽郎中四人。



 都水臺,管諸津橋。使者二人,參事十人。又領都尉、合昌、坊城等三局。尉皆分司諸津橋。



 謁者臺,掌凡諸吉兇公事,導相禮儀事。僕射二人,謁者三十人,錄事一人。



 太常、光祿、衛尉、宗正、太僕、大理、鴻臚、司農、太府,是為九
 寺。置卿、少卿、丞各一人。各有功曹、五官、主簿、錄事等員。



 太常,掌陵廟群祀、禮樂儀制,天文術數衣冠之屬。其屬官有博士、四人,掌禮制。協律郎、二人,掌監調律呂音樂。八書博士二人。等員。統諸陵、掌守衛山陵等事。太廟、掌郊廟社稷等事。太樂、掌諸樂及行禮節奏等事。衣冠、掌冠幘、舄履之屬等事。鼓吹、掌百戲、鼓吹樂人等事。太祝、掌郊廟贊祝,祭社衣服等事。



 太史、掌天文地動,風雲氣色,律歷卜筮等事。太醫、掌醫藥等事。廩犧、掌養犧牲,供祭群祀等事。太宰掌諸神祀烹宰行禮事。等署令、丞。而太廟兼領郊祠、掌五郊群神事。崇虛掌五岳四瀆神祀,在京及諸州道士簿帳等事。二局丞,太樂兼領清商部丞,掌清商音樂等事。鼓吹兼領黃戶局丞,掌供樂人衣服。太史兼領靈臺、掌天文觀候。太卜掌諸卜筮。二局丞。



 光祿寺,掌諸膳食,帳幕器物,宮殿門戶等事。統守宮、掌凡張設等事。太官、掌食膳事。宮門、主諸門籥事。供府、掌供御衣服玩弄之事。肴藏、掌器物鮭味等事。清漳、主酒,歲二萬石。春秋中半。華林掌禁御林木等事。等署。宮門署,置僕射六人,以司其事。餘各有令、丞。又領東園局丞員。掌諸兇具。



 衛尉寺,掌禁衛甲兵。統城門寺,置校尉二人,以司其職。掌宮殿城門,並諸倉庫管籥等事。又領公車、掌尚書所不理,有枉屈,經判奏聞。武庫、掌甲兵及吉兇儀仗。衛士掌京城及諸門士兵。等署令。武庫又有修故局丞。掌領匠修故甲等事。



 大宗正寺,掌宗室屬籍。統皇子王國、諸王國、諸長公主
 家。



 太僕寺,掌諸車輦、馬、牛、畜產之屬。統驊騮、掌御馬及諸鞍乘。左右龍、左右牝、掌駝馬。駝牛、掌飼駝騾驢牛。司羊、掌諸羊。乘黃、掌諸輦輅。車府掌諸雜車。等署令、丞。驊騮署,又有奉承直長二人。左龍署,有左龍局。右龍署,有右龍局。左牝署,有左牝局。右牝署,有右牝局。駝牛署,有典駝、特牛、牸牛三局。司羊署,有特羊、牸羊局。諸局並有都尉。寺又領司訟、典臘、出入等三局丞。



 大理寺,掌決正刑獄。正、監、評各一人,律博士四人,明法掾二十四人,檻車督二人,掾十人,獄丞、掾各二人,司直、
 明法各十人。



 鴻臚寺,掌蕃客朝會,吉兇吊祭。統典客、典寺、司儀等署令、丞。典客署,又有京邑薩甫二人,諸州薩甫一人。典寺署,有僧祗部丞一人。司儀署,又有奉禮郎三十人。



 司農寺,掌倉市薪菜,園池果實。統平準、太倉、鉤盾、典農、導官、梁州水次倉、石濟水次倉、藉田等署令、丞。而鉤盾又別領大囿、上林、游獵、柴草、池藪、苜蓿等六部丞。典農署,又別領山陽、平頭、督亢等三部丞。導官署,又有御細部、曲面部、典庫部等倉督員。



 太府寺,掌金帛府庫,營造器物。統左、中、右三尚方,左藏、
 司染、諸冶東西道署、黃藏、右藏、細作、左校、甄官等署令、丞。左尚方,又別領別局、樂器、器作三局丞。中尚方,又別領別局、涇州絲局、雍州絲局、定州紬綾局四局丞。右尚方,又別領別局丞。司染署,又別領京坊、河東、信都三局丞。諸冶東道,又別領滏口、武安、白間三局丞。諸冶西道,又別領晉陽冶、泉部、大虧阜、原仇四局丞。甄官署,又別領石窟丞。



 國子寺,掌訓教胄子。祭酒一人,亦置功曹、五官、主簿、錄事員。領博士五人,助教十人,學生七十二人。太學博士十人,助教二十人,太學生二百人。四門學博士二十人,
 助教二十人,學生三百人。



 長秋寺,掌諸宮閣。卿、中尹各一人,並用宦者。丞二人。亦有功曹、五官、主簿、錄事員。領中黃門、掖庭、晉陽宮、中山宮、園池、中宮僕、奚官等署令、丞。又有暴室局丞。其中黃門,又有冗從僕射及博士四人。掖庭、晉陽、中山,各有宮教博士二人。中山署,又別有面豆局丞。園池署,又別有桑園部丞。中宮僕署,又別有乘黃局教尉、細馬車都督、車府部丞。奚官署,又別有染局丞。



 將作寺,掌諸營建。大匠一人,丞四人。亦有功曹,主簿,錄事員。若有營作,則立將、副將、長史、司馬、主簿、錄事各
 一人。又領軍主、副,幢主、副等。



 昭玄寺,掌諸佛教。置大統一人,統一人,都維那三人。亦置功曹、主簿員,以管諸州郡縣沙門曹。



 領軍府,將軍一人,掌禁衛宮掖。硃華閣外,凡禁衛官,皆主之。輿駕出入,督攝仗衛。中領軍亦同。有長史、司馬、功曹、五官、主簿、錄直,厘其府事。又領左右衛、領左右等府。



 左右衛府,將軍各一人,掌左右廂。所主硃華閣以外,各武衛將軍二人貳之。



 皆有司馬、功曹、主簿、錄事,厘其府事。其御仗屬官,有御仗正副都督、御仗五職、御仗等員。其直蕩屬官,有直蕩正副都督、直入正副都督、勛武前
 鋒正副都督、勛武前鋒五藏等員。直衛屬官,有直衛正副都督、翊衛正副都督、前鋒正副都督等員。直突屬官,有直突都督、勛武前鋒散都督等員。直閣屬官,有硃衣直閣、直閣將軍、直寢、直齋、直後之屬。又有武騎、雲騎將軍各一人,驍騎、游擊、前後左右等四軍將軍,左右中郎將,各五人,步兵、越騎、射聲、屯騎、長水等校尉,奉車都尉等,各十人,武賁中郎將,羽林監各十五人,冗從僕射三十人,騎都尉六十人,積弩、積射、強弩等將軍及武騎常侍,各二十五人,殿中將軍五十人,員外將軍一百人,殿中司馬督五十人,員外司馬督一百人。



 領左右府,有領左右將軍、領千牛備身,又有左右備身正副都督、左右備身五職、左右備身員。又有刀劍備身正副都督、刀劍備身五職、刀劍備身員。又有備身正副督、備身五職員。



 護軍府,將軍一人,掌四中關津。輿駕出則護駕。中護軍亦同,有長史、司馬、功曹、五官、主簿、錄事,厘其府事。其屬官,東西南北四中府皆統之。四府各中郎將一人,長史、司馬、錄事參軍、統府錄事各一人。又有統府直兵及功曹、倉曹、中兵、外兵、騎兵、長流、城局等參軍各一人,法、田、鎧等曹行參軍各一人。又領諸關尉、津尉。



 行臺,在令無文。其官置令、僕射。其尚書丞郎,皆隨權制而置員焉。其文未詳。



 太子太師、太傅、太保,是為三師,掌師範訓導,輔翊皇太子。少師、少傅、少保,是為三少,各一人,掌奉皇太子,以觀三師之德。出則三師在前,三少在後。



 詹事,總東宮內外眾務,事無大小,皆統之。府置丞、功曹、五官、主簿、錄事員。領家令,率更令、僕等三寺,左右衛二坊。三寺各置丞,二坊各置司馬,俱有功曹、主簿,以承其事。



 家令,領食官、典倉、司藏等署令、丞。又領內坊令、丞。掌知閣內
 諸事。其食官,又別領器局、酒局二丞,典倉又別領園丞。司藏又別領仗庫、典作二局丞。



 率更領中盾署令、丞各一人。掌周衛禁防,漏刻鐘鼓。僕寺領廄牧署令、丞,署又別有車輿局丞。



 左右衛坊率,各領騎官備身正副都督、騎官備身五職、騎官備身員。又有內直備身正副都督、內直備身五職、內直備身員。又有備身正副都督、備身五職員。又有直閣、直前、直後員。又有旅騎、屯衛、典軍等校尉各二人,騎尉三十人。



 門下坊,中庶子、中舍人,通事守舍人、主事守舍人,各四
 人。又領殿內、典膳、藥藏、齋帥等局,殿內局有內直監二人,副直監四人。典膳、藥藏局,監、丞各二人。藥藏又有侍醫四人。齋帥局,齋帥、內閣帥各二人。



 典書坊,庶子四人,舍人二十八人。又領典經坊,洗馬八人,守舍人二人,門大夫、坊門大夫、主簿各一人。並統伶官西涼二部、伶官清商二部。



 自諸省臺府寺,各因其繁簡而置吏。有令史、書令史、書吏之屬。又各置曹兵,以共其役。其員因繁簡而立。其餘主司專其事者,各因事立名,條流甚眾,不可得而具也。



 王,位列大司馬上。非親王則位在三公下。置師一人,餘官大抵與
 梁制不異。



 其封內之調,盡以入臺,三分食一。公已下,四分食一。



 皇子王國,置郎中令,大農,中尉,常侍,各一人。侍郎,二人。上、中、下三將軍,各一人。上、中大夫,各二人。防閣、四人。典書、典祠、學宮、典衛等令,各一人。齋帥、四人。食官、廄牧長、各一人。典醫丞、二人。典府丞、一人。



 執書、二人。謁者、四人。舍人十人。等員。



 諸王國,則加有陵長、廟長、常侍各一人,而無中將軍員。上、中大夫各減一人。諸公又減諸王防閣、齋帥、典醫丞等員。諸侯伯子男國,又減諸公國將軍、大夫員。諸公主則置家令、丞、主簿、錄事等員。



 司州,置牧。屬官有別駕從事史,治中從事史,州都,主簿,西曹書佐、記室、戶曹、功曹、金曹、租曹、兵曹、騎曹、都官、法曹、部郡等從事員。主簿置史,西曹已下各置掾史。又領西、東市署令、丞,及統清都郡諸畿郡。



 清都郡,置尹,丞,中正,功曹、主簿、督郵,五官,門下督,錄事,主記,議生,及功曹、記室、戶、田、金、租、兵、騎、賊、法等曹掾,中部掾等員。



 鄴、臨漳、成安三縣令,各置丞、中正、功曹、主簿、門下督、錄事、主記,議及功曹、記室、戶、田、金、租、兵、騎、賊、法等曹掾員。鄴又領右部、南部、西部三尉,又領十二行經途尉。凡
 一百三十五里,里置正。臨漳又領左部、東部二尉,左部管九行經途尉。凡一百一十四里,里置正。成安又領後部、北部二尉,後部管十一行經途尉,七十四里,里置正。清都郡諸縣令已下官員,悉與上上縣同。



 諸畿郡太守已下,悉與上上郡同。



 上上州刺史,置府。屬官有長史,司馬,錄事,功曹、倉曹、中兵等參軍事及掾史,主簿及掾,記室掾史,外兵、騎兵、長流、城局、刑獄等參軍事及掾史,參軍事及法、墨、田、鎧、集、士等曹行參軍及掾史,右戶掾史,行參軍,長兼行參軍,督護,統府錄事,統府直兵,箱錄事等員。州屬官,有別駕
 從事史,治中從事史,州都光迎主簿,主簿,西曹書佐,市令及史,祭酒從事史,部郡從事,皁服從事,典簽及史,門下督,省事,都錄事及史,箱錄事及史,朝直、刺奸、記室掾,戶曹、舊曹、金曹、租曹、兵曹、左戶等掾史等員。



 上上州府,州屬官佐史,合三百九十三人。上中州減上上州十人,上下州減上中州十人,中上州減上下州五十一人,中中州減中上州十人,中下州減中中州十人,下上州減中下州五十人,下中州減下上州十人,下下州減下中州十人。



 上上郡太守,屬官有丞,中正,光迎功曹,光迎主簿,功曹,
 主簿,五官,省事,錄事,及西曹、戶曹、金曹、租曹、兵曹、集曹等掾佐,太學博士,助教,太學生,市長,倉督等員。合屬官佐史二百一十二人。上中郡減上上郡五人,上下郡減上中郡五人,中上郡減上下郡四十五人,中中郡減中上郡五人,中下郡減中中郡五人,下上郡減中下郡四十人,下中郡減下上郡二人,下下郡減下中郡二人。



 上上縣令,屬官有丞,中正,光迎功曹,光迎主簿,功曹,主簿,錄事,及西曹、戶曹、金曹、租曹、兵曹等掾,市長等員。合屬官佐史五十四人。上中縣減上上縣五人,上下縣減上中縣五人,中上縣減上下縣六人,中中縣減中上縣
 五人,中下縣減中中縣一人,下上縣減中下縣一人,下中縣減下上縣一人,下下縣減下中縣一人。



 自州、郡、縣,各因其大小置白直,以供其役。



 三等諸鎮,置鎮將、副將,長史,錄事參軍,倉曹、中兵、長流、城局等參軍事,鎧曹行參軍,市長,倉督等員。



 三等戍,置戍主、副,掾,隊主、副等員。



 官一品,每歲祿八百匹,二百匹為一秩。從一品,七百匹,一百七十五匹為一秩。



 二品,六百匹,一百五十匹為一秩。從二品,五百匹,一百二十五匹為一秩。



 三品,四百匹,一百匹為一秩。從三品,三百匹,七十五匹為一秩。



 四品,二百四十匹,六十匹為一秩。從四品,二百匹,五十匹為一秩。



 五品,一百六十匹,四十匹為一秩。從五品,一百二十匹,三十匹為一秩。



 六品,一百匹,二十五匹為一秩。從六品,八十匹,二十匹為一秩。



 七品,六十匹,十五匹為一秩。從七品,四十匹,十匹為一秩。



 八品,三十六匹,九匹為一秩。從八品,三十二匹,八匹為一秩。



 九品,二十八匹,七匹為一秩。從九品,二十四匹,六匹為一秩。



 祿率一分以帛,一分以粟,一分以錢。事繁者優一秩,平者守本秩,閑者降一秩。長兼、試守者,亦降一秩。官非執事、不朝拜者,皆不給祿。又自一品已下,至於流外勛品,各給事力。一品至三十人,下至於流外勛品,或以五人為等,或以四人、三人、二人、一人為等。繁者加一等,平者守本力,閑者降一等焉。



 州、郡、縣制祿之法,刺史、守、令下車,各前取一時之秩。



 上上州刺史,歲秩八百匹,與司州牧同。上中、上下各以五十匹為差。中上降上下一百匹,中中及中下,亦以五十匹為差。下上降中下一百匹,下中、下下,亦各以五十匹為差。



 上郡太守,歲秩五百匹,降清都尹五十匹。上中、上下各以五十匹為差。中上降上下四十匹,中中及中下,各以三十匹為差。下上降中下四十匹,下中、下下各以二十匹為差。



 上上縣,歲秩一百五十匹,與鄴、臨漳、成安三縣同。上中、
 上下各以十匹為差。中上降上下三十匹,中中及中下,各以五匹為差。下上降中下二十匹,下中、下下各以十匹為差。



 州自長史已下,逮於史吏,郡縣自丞已下,逮於掾佐,亦皆以帛為秩。郡有尉者,尉減丞之半。皆以其所出常調課之。其鎮將,戍主,軍主、副,幢主、副,逮於掾史,亦各有差矣。



 諸州刺史、守、令已下,干及力,皆聽敕乃給。其乾出所部之人。一干輸絹十八匹,乾身放之。力則以其州、郡、縣白直充。



 三師、王、二大、大司馬、大將軍。三公,為第一品。



 開府儀同三司、開國郡公,為從一品。



 儀同三司,太子三師,特進,尚書令,驃騎、車騎將軍,二將軍加大者,在開國郡公下。衛將軍,加大者,在太子太師上。四征將軍,加大者,次衛大將軍。左右光祿大夫,散郡公,開國縣公,為第二品。



 尚書僕射,置二,左居右上。中書監,四鎮,加大者,次四征。中、鎮、撫軍將軍,三將軍,武職罷任者為之。領軍、加大者,在尚書令下。護軍、翊軍將軍,金紫光祿大夫,散縣公,開國縣侯,為從二品。



 吏部尚書,四安將軍,中領、護,太常、光祿、衛尉卿,太子三少,中書令,太子詹事,侍中,列曹尚書,四平將軍,大宗正、
 太僕、大理、鴻臚、司農、太府卿,清都尹,三等上州刺史,左右衛將軍,秘書監,銀青光祿大夫,散縣侯,開國縣伯,為第三品。



 散騎常侍、三等中州刺史、司徒左長史、四方中郎將、四護匈奴、羌戎、夷、蠻越。中郎將、國子祭酒、御史中丞、中侍中、長秋卿、將作大匠、冠軍將軍、太尉長史、領左右將軍、武衛將軍、太子左右衛率、輔國將軍、四護校尉、太中大夫、龍驤將軍、三等上郡太守、散縣伯,為從第三品。



 鎮遠、安遠將軍,太常、光祿、衛尉少卿,尚書,吏部郎中,給事黃門侍郎,太子中庶子,司徒右長史,司空長史,大宗
 正、太僕、大理、鴻臚、司農、太府少卿,三公府司馬,中常侍,中尹,城門校尉,武騎、雲騎、驍騎、游擊將軍,已前上階。建忠、建節將軍,通直散騎常侍,諸開府長史、中大夫,三等下州刺史,三等鎮將,諸開府司馬,開國縣子,為第四品。



 中堅、中壘將軍,尚書左丞,三公府諮議參軍事,司州別駕從事史,三等上州長史,太子三卿,前、左、右、後軍將軍,中書侍郎,太子庶子,三等中郡太守,左右備身、刀劍備身、備身、衛仗、直蕩等正都督,三等上州司馬,已前上階。振威、奮武將軍,諫議大夫,尚書右丞,諸開府諮議參軍,司州治中從事史,左右中郎將,步兵、越騎、射聲、屯騎、長水校
 尉,硃衣直閣,直閣將軍,太子騎官備身、內直備身等正都督,三等鎮副將,散縣子,為從第四品。



 廣德、弘義將軍,太子備身、直入、直衛等正都督,領左右、三等中州長史,三公府從事中郎,秘書丞,皇子友,國子博士,散騎侍郎,太子中舍人,員外散騎常侍,三等中州司馬,已前上階。折沖、制勝將軍,主衣都統,尚食、尚樂二典御,太子旅騎、屯衛、典軍校尉,領護府長史司馬,諸開府從事中郎,開國縣男,為第五品。



 伏波、陵江將軍,三等下州長史,三公府掾屬,著作郎,通直散騎侍郎,太子洗馬,左右備身、刀劍備身、御仗、直蕩
 等副都督,左右直長,中尚食、中尚藥典御,三等下州司馬,已前上階。輕車、樓船將軍,駙馬都尉,翊衛正都督,直寢,直齋,奉車都尉,都水使者,諸開府掾屬,崇聖、歸義、歸正、歸命、歸德侯,清都郡丞,治書侍御史,鄴、臨漳、成安三縣令,中給事中,三等下郡太守,大理司直,太子直閣、二衛隊主,太子騎官、內直備身副都督,開國鄉男,散縣男,為從第五品。



 勁武、昭勇將軍,尚書諸曹郎中,中書舍人,三公府主簿,三等上州別駕從事史,四中府三等鎮守長史,三公府錄事參軍事,皇子郎中令,三公府功曹、記室、戶、倉、中兵
 參軍事,皇子文學,謁者僕射,已前上階。明威、顯信將軍,太子備身副都督,四中府司馬,武賁中郎將,羽林監,冗從僕射,直入副都督,千牛備身,大理正、監、評,侍御師諸開府錄事、功曹、記室、倉、中兵等曹參軍事,三等上州錄事參軍事,治中從事史,三等上郡丞,三等上縣令,太子內直監,平準署令,為第六品。



 度遼、橫海將軍,直突都督,三等中州別駕從事史,三公府列曹參軍事,給事中,太子門大夫,三等上州功、倉、中兵等參軍事,皇子大農,騎都尉,直後,符璽郎中,三等中州錄事參軍事,已前上階。逾岷、越嶂將軍,直衛副都督,三等
 中州從事史,諸開府主簿、列曹參軍事,三等中州功、倉、中兵等參軍事,太子舍人,三寺丞,太子直前,太子副直監,太子諸隊主,為從第六品。



 戎昭、武毅將軍,勛武前鋒正都督,三公府東西閤祭酒,三等下州別駕從事史,三等上州府主簿、列曹參軍事,三等下州錄事參軍事,四中府錄事參軍事,王公國郎中令,積弩、積射將軍,員外散騎侍郎,皇子中尉,三公府參軍事,列曹行參軍,已前上階。雄烈、恢猛將軍,翊衛副都督,諸開府東西閤祭酒參軍事、列曹行參軍,三等下州功、倉、中兵參軍事,四中府功、倉、中兵參軍事,三等中州府
 主簿、列曹參軍事,二衛府司馬,詹事府丞,左右備身五職,三等鎮錄事參軍事,六寺丞,秘書郎中,著作佐郎,太子侍醫,太子騎尉,太子騎官備身五職,三等中郡丞,三等中縣令,為第七品。



 揚麾、曜鋒將軍,勛武前鋒副都督,強弩將軍,三公府行參軍,三等上州參軍事、列曹行參軍,三等下州府主簿、列曹參軍事,四中府列曹參軍事,王公國大農,長秋、將作寺丞,太子二率坊司馬,三等鎮倉、中兵參軍事,已前上階。蕩邊、開域將軍,勛武前鋒散都督,太學博士,皇子常侍,太常博士,武騎常侍,左右備身,刀劍備身五職,都將、別、
 統、軍主、幢主。三等中州參軍事、列曹行參軍,諸開府行參軍,奉朝請,國子助教,公車、京邑二市署令,三等鎮列曹參軍事,三縣丞,侍御史,尚食、尚藥丞,齋帥,中尚食、中尚藥丞,太子直後、二衛隊副,前鋒正都督,太子騎官備身,太子內直備身五職,已見前。諸戍主、軍主,為從第七品。



 靜漠、綏戎將軍,協律郎,三等上州行參軍,三等下州參軍事、列曹參軍事,四中府列曹行參軍,侯、伯國郎中令,殿中將軍,皇子侍郎,已前上階。平越、殄夷將軍,刀劍備身五職,已見前。前鋒副都督,太子內直備身,主書,殿中侍御史,太子典膳、藥藏丞,太子齋帥,三等中州行參軍,王、公國
 中尉,三等鎮鎧曹行參軍,三等下郡丞,三等下縣令。為第八品。



 飛騎、隼擊將軍,三公府長兼左右戶行參軍、長兼行參軍,門下錄事,尚書都令史,檢校御史,諸署令,諸開府典簽,中謁者僕射,中黃門冗從僕射,已前上階。



 武牙、武奮將軍。備身御仗五職,宮門署僕射,太子備身五職,侯、伯國大農,皇子上、中、下將軍,皇子上、中大夫,王、公國常侍,諸開府長兼左右戶行參軍,諸開府長兼行參軍,員外將軍,勛武前鋒五職,司州及三等上州典簽,太子諸隊副,諸戍諸軍副,清都郡丞,為從第八品。



 清野將軍,子、男國郎中令,諸署內謁者局統,三等上州長兼行參軍,中黃門、太子內坊令,公主家令,皇子防閣、典書令,四門博士,大理律博士,校書郎,三公府參軍督護,都水參軍事,七部尉,諸郡尉,已前上階。橫野將軍,王、公國侍郎,侯、伯國中尉、謁者,太子三寺丞,諸開府參軍督護,殿中司馬督,御仗,太子食官、中省、典倉等令,太子備身,平準、公車丞,三等中州典簽,為第九品。



 偏將軍,諸宮教博士,太子司藏、廄牧令,太子校書,諸署別局都尉,諸尉,諸關津尉,三等上州參軍督護,三等中州長兼行參軍,秘書省正字,皇太子三令,王、公國上中
 下將軍及上中大夫,諸署令,諸縣丞,已前上階。裨將軍,領軍護軍府、太常光祿衛尉寺、詹事府等功曹、五官、奉禮郎,子、男國大農,小黃門,員外司馬督,太學助教,諸幢主、遙途尉,中侍中,省錄事,三等下州典簽,尚書、門下、中書等省醫師,為從第九品。



 流內比視官十三等。第一領人酋長,視從第三品。第一不領人酋長,視第四品。



 第二領人酋長,第一領人庶長,視從第四品。諸州大中正,第二不領人酋長,第一不領人庶長,視第五品。諸州中正,畿郡邑中正,第三領人酋長,第二領人庶長,視從第五品。第三不領人酋長,第二
 不領人庶長,視第六品。第三領人庶長,視從第六品。第三不領人庶長,視第七品。司州州都主簿,國子學生,視從第七品。諸州州都督簿,司州西曹書佐,清都郡中正、功曹,視第八品。司州列曹從事,諸州西曹書佐,諸郡中正、功曹,清都郡主簿,視從第八品。司州部郡從事,諸州祭酒從事史,視第九品。諸州部郡從事,同州守從事,諸郡主簿,司州武猛從事,視從第九品。



 周太祖初據關內,官名未改魏號。及方隅粗定,改創章程,命尚書令盧辯,遠師周之建職,置三公三孤,以為論道之官。次置六卿,以分司庶務。其所制班序:
 內命,謂王朝之臣。三公九命,三孤八命,六卿七命,上大夫六命,中大夫五命,下大夫四命,上士三命,中士再命,下士一命。



 外命,謂諸侯及其臣。諸公九命,諸侯八命,諸伯七命,諸子六命,諸男五命,諸公之孤卿四命,侯之孤卿、公之大夫三命,子男之孤卿、侯伯之大夫、公之上士再命,子男之大夫、公之中士、侯伯之上士一命,公之下士、侯伯之中士下士、子男之士不命。



 其制祿秩,下士一百二十五石,中士已上,至於上大夫,各倍之。上大夫是為四千石。卿二分,孤三分,公四分,各益其一。公因盈數為一萬石。其九秩
 一百二十石,八秩至於七秩,每二秩六分而下各去其一,二秩一秩俱為四十石。凡頒祿,視年之上下。畝至四釜為上年,上年頒其正。三釜為中年,中年頒其半。二釜為下年,下年頒其一。無年為兇荒,不頒祿。六官所制如此。



 制度既畢,太祖以魏恭帝三年始命行之。所設官名,訖於周末,多有改更。並具《盧傳》,不復重序云。



\end{pinyinscope}