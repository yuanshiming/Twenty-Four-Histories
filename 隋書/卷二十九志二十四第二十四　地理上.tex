\article{卷二十九志二十四第二十四 地理上}

\begin{pinyinscope}

 自
 古聖王之受命也,莫不體國經野,以為人極。上應躔次,下裂山河,分疆畫界,建都錫社。是以放勳禦曆,修職貢者九州;文命會同,執玉帛者萬國。洎乎殷遷夏鼎,周黜殷命,雖質文之用不同,損益之途或革,而封建之制,率由舊章。於是分土惟三,列爵惟五,千里以制畿甸,九服以別要荒。十國為連,連有帥,倍連為卒,卒有正。皆所
 以式固鴻基,蕃屏王室,興邦致化,康俗庇人者歟!周德既衰,諸侯力政,干戈日用,戎馬生郊。強陵弱,眾暴寡,魯滅于楚,鄭滅于韓,田氏篡齊,六卿分晉。其餘弑君亡國,不得守其社稷者,不可勝數。逮於七雄競逐,二帝爭強,疆埸之事,一彼一此。秦始皇據百二之岩險,奮六世之餘烈,力爭天下,蠶食諸侯,在位二十餘年,遂乃削平宇內。懲周氏之微弱,恃狙詐以為強,蔑棄經典,罷侯置守。子弟無立錐之地,功臣無尺土之賞,身沒而區宇幅裂,及子而社稷淪胥。漢高祖挺神武之宏圖,掃清禍亂,矯秦皇之失策,封建王侯,並跨州連邑,有逾古典,而郡縣
 之制,無改于秦。逮于孝武,務勤遠略,南兼百越,東定三韓。通邛、笮之險途,斷匈奴之右臂,雖聲教遠洎,而人亦勞止。昭、宣之後,罷戰務農,戶口既其滋多,郡縣亦有增置。至於平帝,郡國一百有三,戶一千二百二十三萬。光武中興,承王莽之餘弊,兵戈不戢,饑疫薦臻,率土遺黎,十才一二,乃並省郡縣,四百餘所。明、章之後,漸至滋繁,郡縣之數,有加曩日。逮炎靈數盡,三國爭強,兵革屢興,戶口減半。有晉太康之後,文軌方同,大抵編戶二百六十余萬。尋而五胡逆亂,二帝播遷,東晉洎于宋、齊,僻陋江左,苻、姚之與劉、石,竊據中原,事蹟糾紛,難可具紀。



 梁武帝
 除暴寧亂,奄有舊吳,天監十年,有州二十三,郡三百五十,縣千二十二。其後務恢境宇,頻事經略,開拓閩、越,克復淮浦,平俚洞,破柯,又以舊州遐闊,多有析置。大同年中,州一百七,郡縣亦稱於此。既而侯景構禍,台城淪陷,墳籍散逸,注記無遺,郡縣戶口,不能詳究。逮于陳氏,土宇彌蹙,西亡蜀、漢,北喪淮、肥,威力所加,不出荊、揚之域。州有四十二,郡唯一百九,縣四百三十八,戶六十萬。後齊承魏末喪亂,與周人抗衡,雖開拓淮南,而郡縣僻小。天保之末,總加並省,洎乎國滅,州九十有七,郡一百六十,縣三百六十五,戶三百三萬,周氏初有關中,百度
 草創,遂乃訓兵教戰,務谷勸農,南清江漢,西兼巴蜀,卒能以寡擊眾,戡定強鄰。及于東夏削平,多有省廢。大象二年,通計州二百一十一,郡五百八,縣一千一百二十四。



 高祖受終,惟新朝政,開皇三年,遂廢諸郡。洎於九載,廓定江表,尋以戶口滋多,析置州縣。煬帝嗣位,又平林邑,更置三州。既而並省諸州,尋即改州為郡,乃置司隸刺史,分部巡察。五年,平定吐谷渾,更置四郡。大凡郡一百九十,縣一千二百五十五,戶八百九十萬七千五百四十六,口四千六百一萬九千九百五十六。墾田五千五百八十五萬四千四十一頃。其邑居道路,山河溝洫,
 沙磧鹹鹵,丘陵阡陌,皆不預焉。東西九千三百里,南北萬四千八百一十五裡,東南皆至於海,西至且末,北至五原,隋氏之盛,極於此也。



 京兆郡開皇三年,置雍州。城東西十八裡一百一十五步,南北十五裡一百七十五步。東面通化、春明、延興三門,南面啟夏、明德、安化三門,西面延平、金光、開遠三門,北面光化一門。裡一百六,市二。大業三年,改州為郡,故名焉。置尹。統縣二十二,戶三十萬八千四百九十九。



 大興開皇三年置。後周於舊郡置縣曰萬年。高祖龍潛,封號大興,故至是改焉。有長樂宮。有後魏杜城縣、西霸城縣、西魏山北縣,並後周廢。長安帶郡。有仙都、福陽、太平等宮。有關官。有舊長安城。始平故置扶風郡,開皇三年郡廢。武功後周置武功郡,建德中郡廢。有永豐渠、普濟渠。盩厔後周置周南郡及恆州,又有倉城、溫湯二縣,尋並廢。有司竹園,有宜壽、仙游、文山、鳳皇等宮。有關官。有太一山。有溫湯。醴
 泉後魏曰寧夷,西魏置寧夷郡。後周改為秦郡,後廢,又以新畤,甘泉二縣入焉。開皇十八年改縣名醴泉。有甘泉水、波水、浪水。有九抃山、溫秀嶺。上宜開皇十七年置,有舊莫西縣,十八年改名好畤,大業三年廢入焉。鄠有甘泉宮。有終南山。有澇水。藍田後周置藍田郡,尋廢郡,及白鹿、玉山二縣入焉。有關官。有滋水。新豐有溫湯。華原後魏置北雍州,西魏改為宜州,又置北地郡,尋改為通川郡。開皇初郡廢,大業初州廢,及土門縣入焉。有沮水、頻山。宜君舊置宜郡,開皇初郡廢,有清水。同官,鄭後魏置東雍州,並華山郡。西魏改曰華州。開皇初郡廢。大業初州廢。有少華山。渭南後魏置渭南郡,西魏分置靈源、中源二縣,後周郡及二縣並廢入焉。有步壽宮。萬年,高陵後魏曰高陸,大業初改焉。三原後周置建忠郡,建德初郡廢。涇陽舊置咸陽縣,開皇初廢。有茂農渠。雲陽舊置,後周置雲陽郡,開皇初郡廢。有涇水、五龍水、甘水、走馬水。富平舊置北地郡,後周改曰中華郡,尋罷。有荊山。華陰有興德宮。有關宮。有京輔都尉。有白渠。有華山。



 馮翊郡後魏置華州,西魏改曰同州。統縣八,戶九萬一千五百七十二。



 馮翊後魏曰華陰。西魏改為武鄉,置武鄉郡。開皇初郡廢,大業初改名馮翊,置馮翊郡。有沙苑。韓城開皇十八年置。有關官。有梁山,有鬼穀。郃陽,朝邑後魏曰南五泉,西魏改焉。有長春宮。有關官,有朝阪。澄城後魏置澄城郡,後周並五泉縣入焉。開皇初郡廢。蒲城舊置南、北二白水。西魏改為蒲城,置白水郡,開皇初郡廢。下邽舊置延壽郡。開皇初郡廢,大業初並蓮勺縣入焉。有金氏陂。白水有五龍山、馬蘭山。



 扶風郡舊置岐州,統縣九,戶九萬二千二百二十三。



 雍後魏置秦平郡,西魏改為歧山郡,開皇三年郡廢。大業初置扶風郡。有岐陽宮。岐山後周曰三龍縣,開皇十六年改名焉。又有後魏周城縣,後周廢。有岐山。陳倉後魏曰宛川,西魏改曰陳倉。後周置顯州,尋州縣俱廢。開皇十八年置,曰陳倉。有陳倉山,有關官。虢後魏置武都郡,西魏改縣曰洛邑。後周置朔州,州尋廢郡
 開皇初廢,大業初改縣為虢。郿舊曰平陽縣,西魏改曰郿城,後周廢入周城縣,開皇十八年改周城為渭濱,大業二年改為郿。又後周置雲州,建德中廢。有安仁宮、鳳泉宮。有太白山、五丈原。普閏大業初置。有仁壽宮。有漆水、岐水、杜水。汧源西魏置隴東郡及汧陰縣,後改縣曰杜陽。後周又曰汧陰。開皇三年郡廢,五年縣改曰汧源。又有西魏東秦州,後改為隴州,大業三年州廢。有關官。有隴山、汧山、汧水。汧陽舊置汧陽郡,後周罷。



 南由後魏置,西魏改為鎮,後周複置縣。又有舊長蛇縣,開皇末廢。有關官。有盤龍山。



 安定郡舊置涇州。統縣七,戶七萬六千二百八十一。



 安定帶郡。鶉觚舊置趙平郡。後周廢郡。並以宜祿縣入焉。大業初分置靈台縣,二年廢。陰盤後魏置平涼郡,開皇初郡廢。有盧水。朝那西魏置安武郡,及析置安武縣。開皇三年郡縣並廢入焉。良原大業初置。臨涇大業初置,初曰湫穀,尋改焉。華亭大業初置。有隴水、芮水。



 北地郡後魏置豳州,西魏改為寧州。大業初複曰豳州。統縣六,戶七萬六百九
 十。



 定安舊置趙興郡。開皇初郡廢,大業初置北地郡。羅川舊曰陽周,開皇中改焉。又西魏置顯州,後周廢。有橋山。彭原舊曰彭陽。後魏置西北地郡,有洛蟠城。西魏置蔚州,有豐城。西魏置雲州。後周二州並廢。開皇初郡廢,十八年改縣曰彭原。有珊瑚水。襄樂後魏置襄樂郡,後周廢。又西魏置燕州,後周廢。又有子午山。新平舊曰白土,西魏置豳州。開皇四年改縣為新平,大業初州廢。三水西魏置恆州,尋廢。



 上郡後魏置東秦州,後改為北華州。西魏改敷州。大業二年改為鄜城郡,後改為上郡。統縣五,戶五萬三千四百八十九。



 洛交開皇三年置。大業三年置上郡。內部舊置敷州及內部郡。開皇三年郡廢,大業初州廢。三川舊名長城,西魏改焉。又有利仁縣,尋廢入焉。鄜城後魏曰敷城,大業初改焉。洛川有鄜水。



 雕陰郡西魏置綏州,大業初改為上州。統縣十一,戶三萬六千一十八。



 上縣西魏置安寧郡,與安寧、綏德、安人三縣同置。開皇初郡廢,改安人為吉萬。大業初置雕陰郡,廢安寧、吉萬二縣入。又後周置義良縣,亦廢入焉。大斌西魏置,仍立安政郡。開皇初廢。有平水。延福西魏置,曰延陵。開皇中改焉。儒林後周置銀州,開皇三年改名焉。大業初州廢。真鄉西魏置。後周置真鄉郡,開皇初郡廢。開光舊置開光郡,開皇三年郡廢。有掞水。銀城後周置,曰石城,後改名焉。城平西魏置。開疆西魏置,有後魏撫寧郡,開皇三年郡廢。撫寧西魏置。綏德西魏置。



 延安郡後魏置東夏州。西魏改為延州,置總管府,開皇中府廢。統縣十一,戶五萬三千九百三十九。



 膚施大業三年置,及置延安郡,有豐林山。豐林後魏置,曰廣武,及遍城郡。開皇初郡廢,十八年改為豐林,大業初又並沃野縣入焉。魏平後魏置,並立朔方郡。後周廢郡,並朔方、政和二縣入焉。金明有冶官。有清水。臨真有西魏神水郡、真川縣,後周郡廢,大業初
 廢真川入焉。延川西魏置,曰文安,及置文安郡。開皇初郡廢,改縣為延川。延安西魏置,又置義鄉縣。大業中廢義鄉入焉。因城後魏置。後周廢,尋又置。義川西魏置汾州、義川郡,後改州為丹州。後周改縣為丹陽。開皇初郡廢,改縣曰義川,又廢樂川郡入。大業初州廢,又廢雲岩縣入焉。汾川舊曰安平,後周改曰汾川。大業初廢門山縣入焉。咸甯舊曰永寧,西魏改為太平。開皇中改為咸寧。



 弘化郡西魏置朔州,後周廢,開皇十六年,置慶州。統縣七,戶五萬二千四百七十三。



 合水開皇十六年置,大業初置弘化郡。馬嶺大業初置。華池仁壽初置。又西魏置蔚州,後周廢。歸德西魏置恆州,後周廢。有雕水。洛源大業初置。有博水、洱水。弘化開皇十八年置弘州,大業初州廢。弘德大業初置。



 平涼郡舊置原州,後周置總管府,大業初府廢。統縣五,戶二萬七千九百九十五。



 平高後魏置太平郡,後改為平高。開皇初郡廢。大業初置平涼郡。有關官。有笄頭山。
 百泉後魏置長城郡及黃石縣,西魏改黃石為長城。開皇初郡廢。大業初縣改為百泉。平涼後周置。有可藍山。會寧西魏置會州,後周廢,開皇十六年置縣。默亭



 朔方郡後魏置夏州,後周置總管府,大業初府廢。統縣三,戶一萬一千六百七十三。



 岩綠西魏置弘化郡。開皇初廢,大業初置朔方郡。寧朔後周置。長澤西魏置闡熙郡。又有後魏大安郡,及置長州。開皇三年郡廢,又廢山鹿、新蒨二縣入焉。大業三年州廢。



 鹽川郡西魏置西安州,後改為鹽州。統縣一,戶三千七百六十三。



 五原後魏置郡,曰大興。西魏改為五原,後又為大興。開皇初郡廢,大業初置鹽川郡。



 靈武郡後魏置靈州,後周置總管府,大業元年府廢。統縣六,戶一萬二千三百三十。



 回樂後周置,帶普樂郡。又西魏置臨河郡。開皇元年改臨河郡曰新昌,三年郡並廢。大業初置靈武郡。弘靜開皇十一年置。有賀蘭山。懷遠後周置,仍立懷遠郡。開皇三年郡廢。靈
 武后周置,曰建安,後又置曆城郡。開皇三年郡廢,十八年改建安為廣閏,仁壽元年改名焉。鳴沙後周置會州,尋廢,開皇十九年置環州及鳴沙縣。大業三年州廢。有關官。豐安開皇十年置。



 榆林郡開皇二十年,置勝州。統縣三,戶二千三百三十。



 榆林開皇七年置。大業初置郡。富昌開皇十年置。金河開皇三年置,曰陽壽,及置油雲縣,又置榆關總管。五年改置雲州總管。十八年改陽壽曰金河,二十年雲州移,二縣俱廢。仁壽二年又置金河縣,帶關。



 五原郡開皇五年置豐州,仁壽元年置總管府,大業元年府廢。統縣三,戶二千三百三十。



 九原開皇五年置。大業初置郡。永豐開皇五年置。安化開皇十一年置。



 天水郡舊秦州。後周置總管府,大業初府廢。統縣六,戶五萬二千一百三十。



 上邽故曰上邽,帶天水郡。開皇初郡廢,大業初複置郡,縣改名焉。有濛水。冀城後周曰冀城縣,廢入黃瓜縣。大業初改曰冀城。有石鼓崖。清水後魏置,及置清水郡。開皇初郡廢。有關官。有分水嶺。
 秦嶺後魏置,曰伯陽縣。開皇中改焉。隴城舊曰略陽,置略陽郡。開皇二年郡廢,縣改曰河陽,六年改曰隴城。成紀舊廢,後周置。有龍馬城、仙人硤。



 隴西郡舊渭州。統縣五,戶一萬九千二百四十七。



 襄武帶郡。隴西舊城內陶,置南安郡。開皇初郡廢,改為武陽,十年改名焉。渭源有鳥鼠山。有渭水。障後魏置。西魏置廣安郡,後周郡廢。長川後魏置安陽郡,領安陽、鳥水二縣。西魏改曰北秦州,後又改曰交州。開皇三年郡廢。十八年改州曰紀州,安陽曰長川。大業初州廢,又廢烏水入焉。



 金城郡開皇初,置蘭州總管府,大業初府廢。統縣二,戶六千八百一十八。



 金城舊縣曰子城,帶金城郡。開皇初郡廢。大業初改縣為金城。置金城郡。有關官。狄道後魏置臨洮郡、龍城縣,後周皆廢。又後魏置武始郡,開皇初廢。有白石山。



 枹罕郡舊置河州。統縣四,戶一萬三千一百五十七。



 枹罕
 舊置枹罕郡,開皇初郡廢。大業初置郡。有關官。有鳳林山。龍支後魏曰北金城,西魏改焉。有唐述山。大夏有金紐山。水池後魏曰蕈川,後周改焉。



 澆河郡後周武帝逐吐谷渾,以置廓州總管府。開皇初府廢。統縣二,戶二千二百四十。



 河津後周置洮河郡,領洮河、廣威、安戎三縣。開皇初郡廢,並三縣入焉。大業初置澆河郡。有濫水。達化後周置達化郡。開皇初郡廢,並綏遠縣入焉。有連雲山。



 西平郡舊置鄭州。統縣二,戶三千一百一十八。



 湟水舊曰西都,後周置樂都郡。開皇初郡廢,十八年改縣曰湟水。又有舊浩亹縣,又西魏置龍居、路倉二縣,並後周廢。大業初置西平郡。有土樓山。化隆舊魏曰廣威,西魏置澆河郡,後周廢郡,仁壽初改為化隆,有拔延山、湟水、盧水。



 武威郡舊涼州,後周置總管府,大業初府廢。統縣四,戶一萬一千七百五。



 姑臧舊置武威郡,開皇初郡廢。大業初複置武威郡。又後魏置武安郡、襄武縣,並西魏廢。又舊
 有顯美縣,後周廢。有茅五山。昌松後魏置昌松郡,後周廢郡,以榆次縣入。開皇初改縣為永世,後改曰昌松。又有後魏魏安郡,後周改置白山縣,尋廢。有白山。番和後魏置番和郡。後周郡廢,置鎮。開皇中為縣,又並力乾、安寧、廣城、障、燕支五縣之地入焉。有燕支山。允吾後魏置,曰廣武,及置廣武郡。開皇初郡廢,改縣曰邑次,尋改為廣武,後又改為邑次。大業初改為允吾。有青岩山。



 張掖郡西魏置西涼州,尋改曰甘州。統縣三,戶六千一百二十六。



 張掖舊曰永平縣,後周置張掖郡。開皇初郡廢,十七年縣改為酒泉。大業初改為張掖,置張掖郡。又有臨松縣,後周廢。有甘峻山、臨松山、合黎山、有玉石澗、大柳穀。刪丹後魏曰山丹,又有西郡、永寧縣。西魏郡廢,縣改為弱水。後周省入山丹。大業改為刪丹。又後周置金山縣,尋廢入焉。有祀山。有鹽池。有溺水。福祿舊置酒泉郡,開皇初郡廢。仁壽中以置肅州,大業初州尋廢,又後周置樂涫縣,尋廢。有祁連山,崆峒山、昆侖山,有石渠。



 敦煌郡舊置瓜州。統縣三,戶七千七百七十九。



 敦煌舊置敦煌郡,後周並效谷、壽皇二郡入焉。又並敦煌、鳴沙、平康、效穀、東鄉、龍勒六縣為鳴沙縣。開皇初郡廢。大業置敦煌郡,改鳴沙為敦煌。有神沙山、三危山,有流沙。常樂後魏置常樂郡。後周並涼興、大至、冥安、閏泉,合為涼興縣。開皇初郡廢,改縣為常樂。有關官。玉門後魏置會稽郡。後周廢郡,並會稽、新鄉、延興為會稽縣。開皇中改為玉門,並得後魏玉門郡地。



 鄯善郡大業五年平吐谷渾置,置在鄯善城,即古樓蘭城也。並置且末、西海、河源,總四郡。有蒲昌海、鄯善水。統縣二。



 顯武濟遠



 且末郡置在古且末城。有且末水、薩毗澤。統縣二。



 肅寧伏戎



 西海郡置在古伏俟城,即吐谷渾國都。有西王母石窟、青海、鹽池。統縣二。



 宣德威定



 河源郡置在古赤水城。有曼頭城、積石山,河所出。有七烏海。統縣二。



 遠化赤水



 《周禮·職方氏》:「正西曰雍州。」上當天文,自東井十度至柳八度,為鶉首。于辰在未,得秦之分野。考其舊俗,前史言之詳矣。化于姬德,則閒田而興讓,習於嬴敝,則相稽而反脣。斯豈土壤之殊乎?亦政教之移人也。京兆王都所在,俗具五方,人物混淆,華戎雜錯。去農從商,爭朝夕之利,遊手為事,競錐刀之末。貴者崇侈靡,賤者薄仁義,豪強者縱橫,貧窶者窘蹙。桴鼓屢驚,盜賊不禁,此乃古今之所同焉。自京城至於外郡,得馮翊、扶風,是漢之三輔。
 其風大抵與京師不異。安定、北地、上郡、隴西、天水、金城,于古為六郡之地,其人性猶質直。然尚儉約,習仁義,勤於稼穡,多畜牧,無複寇盜矣。雕陰、延安、弘化,連接山胡,性多木強,皆女淫而婦貞,蓋俗然也。平涼、朔方、鹽川、靈武、榆林、五原,地接邊荒,多尚武節,亦習俗然焉。河西諸郡,其風頗同,並有金方之氣矣。



 漢川郡舊置梁州。統縣八,戶一萬一千九百一十。



 南鄭舊置漢川郡。開皇初郡廢,大業初置郡。又西魏置白雲縣,至是併入焉。有黃牛山、龍岡山。西舊曰勣塚,大業初改焉。有關官,有定軍山、百牢山、街亭山、嶓塚山。有漢水。褒城開皇初曰褒內。仁壽九年因失印更給,改名焉。有關官。有女郎山。城固興勢舊置儻城郡,開皇初郡廢。西鄉舊曰豐寧,置洋州及洋川
 郡。開皇初廢郡,大業初廢州,改縣曰西鄉。又舊有懷昌郡,後周廢為懷昌縣,至是入焉。有洋水。黃金難江後周置集州及平桑郡。開皇初郡廢,大業初州廢。



 西城郡梁置梁州,尋改曰南梁州。西魏改置東梁州,尋改為金州,置總管府。開皇初府廢。統縣六,戶一萬四千三百四十一。



 金川,梁初曰上廉,後曰吉陽。西魏改曰吉安,後周以西城入焉。舊有金城、吉安二郡,開皇初並廢。十八年改縣為吉安。大業三年改曰金川,置西城郡。又後周置洵州,尋廢。有焦陵山。石泉舊曰永樂,置晉昌郡。西魏改郡曰魏昌,尋改永樂曰石泉,析置魏寧縣。後周省魏昌郡入中城郡,又省魏甯縣入石泉縣。洵陽舊置洵陽郡,開皇初郡廢。有洵水。安康舊曰甯都,齊置安康郡,後魏置東梁州,後蕭詧改直州。開皇初郡廢,大業初州廢,縣改曰安康。黃土西魏置涓陽郡。後周改郡,置縣曰長岡。後郡省入甲郡,置縣曰黃土,並赤石、甲、臨江三縣入焉。開皇初郡廢。豐利梁置南上洛郡,西魏改郡曰豐利。後周省郡入上津郡,以熊川、陽川二縣入豐利。後又廢上津郡入甲郡。
 有天心水。



 房陵郡西魏置光遷國。後周國廢,置遷州。大業初改名房州。統縣四,戶七千一百六。



 光遷舊曰房陵,置新城郡。梁末置岐州,後周郡縣並改為光遷,又有舊綏州,開皇初,與郡並廢。大業初置房陵郡。有房山、霍水。永清舊曰大洪,後周改焉。有照珠山、百武山、沮水、泛水。竹山梁曰安城,西魏改焉。置羅州。開皇十八年改曰房州,大業初州廢。有花林山、懸鼓山。上庸梁曰新豐,西魏改焉。後周改曰孔陽。開皇十八年複曰上庸。



 清化郡置巴州。統縣十四,戶一萬六千五百三十九。



 化成梁曰梁廣,仍置歸化郡。後周改縣曰化成。開皇初郡廢。大業初置清化郡。曾口梁置。清化梁置,曰伏強,有木門郡。開皇三年郡廢,七年縣改曰清化。有伏強山。清水。盤道梁置,曰難江。西魏改焉。有龍腹山。永穆梁置,曰永康,又有萬榮郡。開皇初郡廢,十八年縣改名焉。歸仁梁置,曰平州縣。後周改曰
 同昌,開皇中改名焉。始甯梁置,並置遂寧郡。開皇初郡廢。有始寧山。其章梁置。恩陽梁置,曰義陽。開皇末改。長池後周置,曰曲細。開皇末改焉。符陽舊置其章郡,開皇初廢。白石有文山。安固梁置。後周置蓬州,大業初州廢。有大蓬山。伏虞梁置,曰宣漢,及置伏虞郡。開皇初郡廢,十八年改焉。



 通川郡梁置萬州,西魏曰通州。統縣七,戶一萬二千六百二十四。



 通川梁曰石城,置東關郡。開皇初郡廢。大業初置通川郡。三岡梁置,屬新安郡。西魏改郡曰新寧。開皇初郡廢。石鼓西魏置遷州。後周廢州,置臨清郡。開皇初廢郡。東鄉西魏置石州,後周廢州,置三巴郡。開皇初郡廢。宣漢西魏置並州及永昌郡。開皇三年郡廢,五年州廢。西流後魏曰漢興。西魏改焉,又置開州,及周安、萬安、江會三郡。後周省江會入周安。開皇初郡並廢,大業初州廢。萬世後周置,及置萬世郡。開皇初郡廢。



 宕渠郡梁置渠州。統縣六,戶一萬四千三十五。



 流江後魏置縣,及置流江郡,開皇初郡廢,大業初置宕渠郡。賨城舊曰始安。開皇十八年改焉。鄰水梁置縣,並置鄰州,後魏改鄰山郡,開皇初郡廢。宕渠梁置,並置境陽郡。開皇初郡廢。咸安梁置,曰綏安。開皇末改名焉。墊江西魏置縣及容川、容山郡。後周改為魏安縣。開皇初郡廢,十八年縣改名焉。



 漢陽郡後魏曰南秦州,西魏曰成州。統縣三,戶一萬九百八十五。



 上祿舊置仇池郡,後魏置倉泉縣,後周廢階陵、豐川、建平、城階四縣入焉。開皇初郡廢,大業初置漢陽郡,改縣曰上祿。有百頃堆。潭水西魏置潭水郡。後周郡廢,並廢甘若,相山、武定三縣入焉。長道後魏置漢陽郡。後周郡廢,又省水南縣入焉。開皇初郡廢,十八年改曰長道。



 臨洮郡後周武帝逐吐谷渾,以置洮陽郡,尋立洮州。開皇初郡廢。統縣十一,戶二萬八千九百七十一。



 美相後周置縣,及置洮陽郡。開皇初郡廢,並洮陽縣入焉。大業
 初置監洮郡。疊川後周置疊州、疊川縣。開皇四年置總管府,大業元年府廢。有洮水、流水。合川後周置,仍立西疆郡。開皇初郡廢。有白嶺山。樂川後周置。歸政開皇二年置,仍立疆澤郡,三年廢,又後周立弘州及開遠、河濱二郡。開皇初州郡並廢。洮源後周置,曰金城,並立旭州,又置通義郡。開皇初郡廢,十八年縣改為美俗。大業初州廢,縣改名焉。洮陽後周置,曰廣恩,並置廣恩郡。開皇初郡廢,仁壽元年,改縣為洮河,大業初改曰洮陽。臨潭後周曰泛潭,開皇十一年改名焉。臨洮西魏置,曰溢樂,並置岷州及同和郡。開皇初郡廢,大業初州廢,更名縣曰臨洮。又後周置祐川郡、基城縣,尋郡縣俱廢。有岷山、崆峒山。當夷後周置。又立洪和郡,郡尋廢。又置博陵郡及博陵、寧人二縣。開皇初併入。和政後周置洮城郡,尋廢。



 宕昌郡後周置宕昌國,天和元年置宕州總管府。開皇四年府廢。統縣三,戶六千九百九十六。



 良恭後周置,初曰陽宕,置宕昌郡。開皇初郡廢,十八年改名焉。大業初置宕昌
 郡。和戎後周置。有良恭山。懷道後周置甘松郡,開皇初郡廢。



 武都郡西魏置武州。統縣七,戶一萬七百八十。



 將利舊曰石門,西魏改曰安育。後周改曰將利,置武都郡,後改曰永都郡。開皇初郡廢,大業初置武都郡。又有東平縣,後周併入焉。有河池水。建威後魏置白水郡,郡廢,改為白水縣。西魏複立郡,改為綏戎。後周郡廢,改為建威縣,並廢洪化縣入焉。又西魏有孔堤郡及縣,後周並廢。覆津後魏初曰玩當,置武階郡。西魏又置覆津縣,及置萬郡,統赤萬、接難、五部三縣。後周一郡三縣並玩當,並廢入焉。開皇初武階郡又廢。盤堤西魏置,曰南五部縣,後改名焉;並立武陽郡及茄蘆縣。後周郡廢,縣併入焉。長松西魏置,初曰建昌,置文州及盧北郡。開皇初郡廢,十八年縣改曰長松,大業初州廢。曲水西魏置。正西西魏置。



 同昌郡西魏逐吐谷渾,置鄧州。開皇七年改曰扶州。統縣八,戶一萬二千二百四十八。



 尚安西魏置縣及鄧寧郡。開皇初郡廢,大業初置同昌郡。有黑水。鉗川
 西魏置。有鉗川山。有白水。貼夷西魏置,又置昌寧郡。開皇三年郡廢。同昌西魏置。有鄧至山,雲鄧艾所至,故名焉。嘉誠後周置縣並龍涸郡及扶州總管府。開皇初府廢,三年郡廢,七年州廢。有雪山。封德後魏置,又立芳州,有深泉郡,開皇初郡廢,又省理定縣入焉。大業初州廢。常芬後周置,及立恆香郡。開皇初郡廢。有弱水。金崖後周置。



 河池郡後魏置南岐州,後周改曰鳳州。統縣四,戶一萬一千二百二。



 梁泉舊曰故道,後魏置郡,曰固道,縣曰涼泉,尋改曰梁泉。西魏改郡曰歸真。後周廢郡,又廢龍安、商樂二縣入。大業初置郡。兩當後魏置,及立兩當郡。開皇初郡廢。河池後魏曰廣化,並置廣化郡。開皇初郡廢,仁壽初縣改名焉。又後魏置思安縣,大業初省入。有河池水。同谷舊曰白石,置廣業郡。西魏改曰同穀,後周置康州。開皇初郡廢,大業初州廢。又有泥陽縣,西魏廢。



 順政郡後魏置東益州,梁為武興蕃王國,西魏改為興州。統縣四,戶四千二百
 六十一。



 順政舊曰略陽。西魏置郡,曰順政,縣曰漢曲;又置仇池縣,後改曰靈道。開皇初郡廢。十八年,縣改名焉。大業初置郡,又省靈道縣併入。鳴水西魏置,曰落叢,並置落叢郡。開皇初郡廢。六年,縣改為廚北。八年,改曰鳴水。長舉西魏置,又立盤頭郡。後周廢郡,有鳳溪水。修城舊置修城郡,縣曰廣長。後周郡廢;又廢下阪縣入。仁壽初,縣改名焉。又西魏置柏樹縣,後周廢。



 義城郡後魏立益州,世號小益州。梁曰黎州。西魏複曰益州,又改曰利州,置總管府。大業初府廢。統縣七,戶一萬五千九百五十。



 綿谷舊曰興安,置晉壽郡。開皇初郡廢。十八年,縣改名焉。大業初置郡。又有華陽郡,梁置華州,西魏並廢。有龍門山。益昌義城西魏置。葭萌後魏曰晉安,置新巴郡。開皇初郡廢。十八年,縣改名焉。大業初又並恩金縣入焉。岐坪景穀舊曰白水,置平興郡。後周省東洛郡入。開皇初郡廢,縣改名平興。十八年,改曰景穀。大業初又省魚般縣入焉。有關官。有木馬山、良珠山。有凍水。嘉川舊置宋熙郡,開
 皇初廢。



 平武郡西魏置龍州。統縣四,戶五千四百二十。



 江油後魏置江油郡,開皇三年郡廢,大業初置郡。有關官。馬盤後魏置馬盤郡,開皇三年郡廢。平武梁末,李文智自立為籓王,西魏廢為縣,有涪水、潺水。方維舊曰秦興,置建陽郡。開皇初郡廢,縣改名焉。



 汶山郡後周置汶州。開皇初改曰蜀州,尋為會州,置總管府。大業初府廢。統縣十一,戶二萬四千一百五十九。



 汶山舊曰廣陽。梁改為北部都尉,置繩州、北部郡。後周改曰汶州。開皇初郡廢,仁壽元年改名焉。北川後周置。有龍泉水、鷹門山,襄陽山。汶川後周置汝山郡,開皇初郡廢。交川開皇初置。有關官。通化開皇初置,曰金川,仁壽初改名焉。左封後周置,曰廣年,及置廣年郡、左封郡。開皇初郡並廢。仁壽初縣改名焉。又周置翼州,大業初廢,有汶山。平康後周置。有羊腸山。翼水後周置,曰龍求,及置清江郡。開皇初郡廢,縣改曰清江。十八年,又改名焉。翼針後周置,及翼針郡。開皇初郡廢。有石鏡山。江源後周置。通軌後周置縣及覃州,並
 覃川、榮鄉二郡。開皇初郡廢,四年州廢。有甘松山。



 普安郡梁置南梁州,後改為安州,西魏改為始州。統縣七,戶三萬一千三百五十一。



 普安舊曰南安。西魏改曰普安,置普安郡。開皇初郡廢,大業初置郡焉。永歸舊曰白水,西魏改焉。黃安舊曰華陽,西魏改焉,又置黃原郡。開皇初郡廢。陰平宋置北陰平郡,魏置龍州。西魏改郡為陰平,又名縣焉。後周從江油郡,改曰靜龍,縣曰陰平。開皇初郡廢。梓潼舊曰安壽,西魏置潼川郡。開皇初郡廢。大業初縣改名焉。有五婦山。武連舊曰武功,置輔劍郡。西魏改郡曰安都,縣曰武連。開皇初郡廢。臨津舊曰胡原。開皇七年改焉。



 金山郡西魏置潼州。開皇五年,改曰綿州。統縣七,戶三萬六千九百六十三。



 巴西舊曰涪,置巴西郡。西魏改縣曰巴西。開皇初郡廢。大業初置金山郡。有鹽井。昌隆有雲臺山。涪城舊置始平郡,西魏改郡為涪城,後周又改曰安城。開皇初郡廢,改縣曰安城。十六年,
 改為涪城。魏城西魏置。萬安舊曰孱亭,西魏改名焉,置萬安郡。開皇初郡廢。神泉舊曰西充國,開皇六年改名焉。金山舊置益昌、晉興二縣,西魏省晉興入益昌,後周別置金山。開皇四年,省益昌入金山。



 新城郡梁末置新州。開皇末改曰梓。統縣五,戶三萬七百二十七。



 郪舊曰伍城。西魏改曰昌城,仍置昌城郡。開皇初郡廢。大業初置新城郡,改縣名焉。射洪西魏置,曰射江,後周改名焉。鹽亭西魏置鹽亭郡。開皇初郡廢。有高渠縣。大業初併入焉。通泉舊曰通泉,置西宕渠郡。西魏改郡、縣俱曰勇泉。開皇初郡廢,縣改名,又並光漢縣入焉。飛烏開皇中置。



 巴西郡梁置南梁、北巴州,西魏置隆州。統縣十,戶四萬一千六十四。



 閬內梁置北巴郡,後魏平蜀,置盤龍郡,開皇初郡廢。大業初置巴西郡。有盤龍山、天柱山、靈山。南部舊曰南充國,梁曰南部,西魏置新安郡,後周郡廢。蒼溪舊曰漢昌,開皇末改名焉。南充舊曰安漢,
 置宕渠郡。開皇初郡廢。十八年,縣改名焉。相如梁置梓潼郡,後魏郡廢。西水梁置掌天郡,西魏改曰金遷,開皇初郡廢。晉城舊曰西充國,梁置木蘭郡。西魏廢郡,改縣名焉。有閬水。奉國梁置白馬、義陽二郡,開皇初郡廢,並廢義陽縣入焉。儀隴梁置,並置隆城郡。開皇初郡廢。大寅梁置。



 遂寧郡後周置遂州。仁壽二年,置總管府。大業初府廢。統縣三,戶一萬二千六百二十二。



 方義梁曰小溪,置東遂寧郡。西魏改縣名焉。後周改郡曰石山。開皇初郡廢。大業初置遂寧郡。青石舊曰晉興,西魏改名焉,又置懷化郡。開皇初郡廢。長江舊曰巴興,西魏改名焉,又置懷化郡。開皇初郡廢。



 涪陵郡西魏置合州。開皇末改曰涪州。統縣三,戶九千九百二十一。



 石鏡舊曰墊江,置宕渠郡。西魏改郡為墊江,縣為石鏡。開皇初郡廢。大業初置涪陵郡。漢初梁置新興郡。西魏改郡曰清居,名縣曰漢初。開皇初郡廢。赤水開皇八年置。



 巴郡梁置楚州。開皇初改曰渝州。統縣三,戶一萬四千四百二十三。



 巴舊置巴郡,後周廢枳、墊江二縣入焉。開皇初郡廢。大業初置巴郡。江津舊曰江州縣。西魏改為江陽,置七門郡。開皇初郡廢。十八年,縣改名焉。涪陵舊曰漢平,置涪陵郡。開皇初郡廢。十三年,縣改名焉。



 巴東郡梁置信州,後周置總管府,大業元年府廢。統縣十四,戶二萬一千三百七十。



 人復舊置巴東郡,縣曰魚複,西魏改曰人複。開皇初郡廢。大業初,置巴東郡。有鹽井、白鹽山。雲安舊曰朐,後周改焉。南浦後周置安鄉郡,後改縣曰安鄉,改郡曰萬川。開皇初郡廢。十八年,縣改名焉。梁山西魏置。有高梁山。有溪。大昌後周置永昌郡,尋廢,又廢北井縣入焉。巫山舊置建平郡,開皇初郡廢。有巫山。秭歸後周曰長寧,置秭歸郡。開皇初郡廢,改縣曰秭歸。巴東舊曰歸鄉,梁置信陵郡。後周郡廢,縣改曰樂鄉。開皇末,又改名焉。有巫峽。新浦後周置周安郡,
 開皇初郡廢。盛山梁曰漢豐,西魏改為永寧,開皇末,曰盛山。臨江梁置臨江郡,後周置臨州。開皇初郡廢,大業初州廢。有平都山。有彭溪。武甯後周置南州、南都郡、源陽縣,後改郡曰懷德,縣曰武寧。開皇初州郡並廢入焉。石城開皇初置庸州,大業初州廢。務川開皇末置。



 蜀郡舊置益州,開皇初廢。後周置總管府。開皇二年,置西南道行台省,三年,複置總管府,大業元年府廢。統縣十三,戶十萬五千五百八十六。



 成都舊置蜀郡,又有新都縣。梁置始康郡,西魏廢始康郡。開皇初廢蜀郡,並廢新繁入焉。十八年,改新都曰興樂。大業初置蜀郡,省興樂入焉。舊置懷甯、晉熙、宋興、宋寧四郡,至後周並廢。有武簷山。雙流舊曰廣都,置寧蜀郡,後同郡廢。仁壽元年改縣曰雙流。有女伎山。新津後周置,並置犍為郡。開皇初郡廢。大業初又廢僰道縣入焉。晉原舊曰江原,及置江原郡。後周廢郡,縣改名焉。清城舊置齊基郡,後周廢為清城縣。有鳴鵠山,清城山。九隴舊曰晉壽,梁置東益州。後周州廢,置九隴郡,並改縣曰九隴。仁壽初置濛州。開皇初郡廢,並隴
 泉、興固,青陽三縣入焉。大業初州廢。有太山、道場山。綿竹舊置晉熙郡及長楊、南武都二縣。後周並二縣為晉熙,後又廢晉熙入陽泉。開皇初郡廢,十八年改為孝水,大業二年改曰綿竹。有冶官。有綿水。有鹿堂山。郫西魏分置溫江縣,開皇初省入。仁壽初複置萬春縣,大業初又廢入焉。有金山、平樂山、天彭門。玄武舊曰伍城,後周置玄武郡。開皇初郡廢,改縣名焉。仁壽初置凱州,大業初廢。有三堆山、郪江。雒舊曰廣漢,又置廣漢郡。開皇初郡廢。十八年,改曰綿竹。大業初改名雒焉。又有西遂甯郡、南陰平郡。後周廢西遂寧,改為懷中,南陰平郡曰南陰平縣,尋並廢。陽安舊曰牛鞞,西魏改名焉,並置武康郡。開皇初郡廢。仁壽初置簡州,大業初州廢。有鹽井。平泉西魏置,曰婆閏。開皇十八年,改名焉,金泉西魏置縣及金泉郡。後周廢郡,並廢白牟縣入焉。有昌利山、銅官山、石城山。



 臨邛郡舊置雅州。統縣九,戶二萬三千三百四十八。



 嚴道西魏置,曰始陽縣,置蒙山郡。開皇初郡廢。十三年,改曰蒙山,尋置雅州。大業置臨邛郡,縣改名焉。有邛來山。
 名山舊曰蒙山。開皇十三年,改始陽曰蒙山,改蒙山曰名山。盧山仁壽末置。依政西魏置,及置邛州,大業初廢。臨邛舊置臨邛郡,開皇初廢。有火井。蒱江西魏置。曰廣定,及置蒱原郡,開皇初郡廢。仁壽初縣改名焉。蒱溪西魏置。沈黎後周置黎州,尋並縣廢。開皇中置縣。仁壽末置登州,大業初州廢。漢源大業初置。



 眉山郡西魏曰眉州。後周曰青州,後又曰嘉州。大業二年又改曰眉州。統縣八,戶二萬三千七百九十九。



 龍遊後周置,曰峨眉,及置平羌郡。開皇初郡廢。九年改縣為青衣。平陳日,龍見水,隨軍而進,十年改名焉。大業初置眉山郡。平羌後周置,仍置平羌郡。開皇初郡廢。夾江開皇三年置。峨眉開皇十三年置。有峨眉山、綏山。通義舊置齊通郡及青州。西魏改州曰眉州。開皇初郡廢,改齊通曰廣通。仁壽元年改為通義。大業初州廢。青神後周置,並置青神郡。開皇初郡廢。丹棱後周置,曰齊樂。開皇中改名焉。洪雅開皇十三年置。



 隆山郡西魏置陵州。統縣五,戶一萬一千四十二。



 仁壽梁置懷仁郡,西魏改縣曰普寧,開皇初郡廢,十八年縣改名焉。又西魏置蒲亭。大業初置隆山郡,蒲亭併入焉。有鹽井。貴平西魏置,又立和仁郡。後周又廢可曇、平井二縣入焉。開皇初郡廢。大業初,又廢籍縣入焉。井研始建開皇十一年置。有鐵山。隆山舊曰犍為,置江州。西魏改縣曰隆山。後周省州,置隆山郡。開皇初郡廢,又並江陽縣入焉。有冶官。有鼎鼻山。



 資陽郡西魏置資州。統縣九,戶二萬五千七百二十二。



 磐石後周置縣及資中郡,開皇初郡廢。大業初置資陽郡。內江後周置。威遠開皇初置。大牢開皇十三年置。安嶽後周置,並置普州。大業初州廢。普慈後周置郡曰普慈,縣曰多業。開皇初郡廢。十三年,縣改名焉。安居後周置,曰柔剛,及置安居郡。開皇初郡廢。十三年,縣改名焉。隆康後周置,曰永康。開皇十八年改焉。資陽後周置。



 瀘川郡梁置滬州。仁壽中置總管府,大業初府廢。統縣五,戶一千八百二。



 瀘川舊曰江陽,並置江陽郡。開皇初郡廢。大業初置瀘川郡,縣改名焉。富世後周置,及置洛源郡。開皇初郡廢。江安舊曰漢安,開皇十八年改名焉。合江後周置。綿水梁置。有綿溪。



 犍為郡梁置戎州。統縣四,戶四千八百五十九。



 僰道後周置,曰外江。大業初改曰僰道,置犍為郡。犍為後周置,曰武陽。開皇初改焉。南溪梁置,曰南廣,及置六同郡。開皇初郡廢。仁壽初縣改名焉。開邊開皇六年置,七年廢訓州入焉。大業初廢恭州、協州入焉。



 越巂郡後周置嚴州。開皇六年改曰西寧州,十八年又改曰巂州。統縣六,戶七千四百四十八。



 越巂帶郡。邛都蘇祗舊置亮善郡,開皇初郡廢。有孫水。可泉舊宣化郡,開皇初廢。台登舊置白沙郡。開皇初郡廢。邛部舊置邛部郡,又有平樂郡。開皇初並廢。有巂山。



 牂柯郡開皇初,置牂。統縣二。



 牂柯帶郡。賓化



 黔安郡後周置黔州,不帶郡。統縣二,戶一千四百六十。



 鼓水開皇十三年置。有伏牛山。出鹽井。涪川開皇五年置。



 梁州于天官上應參之宿。周時梁州,以並雍部。及漢,又析置益州。在《禹貢》,自漢川以下諸郡,皆其封域。漢中之人,質樸無文,不甚趨利。性嗜口腹,多事田漁,雖蓬室柴門,食必兼肉。好祀鬼神,尤多忌諱,家人有死,輒離其故宅。崇重道教,猶有張魯之風焉。每至五月十五日,必以酒食相饋,賓旅聚會,有甚於三元。傍南山雜有獠戶,富室者頗參夏人為婚,衣服居處言語,殆與華不別。西城、
 房陵、清化、通川、宕渠,地皆連接,風俗頗同。漢陽、臨洮、宕昌、武都、同昌、河池,順政、義城、平武、汶山、皆連雜氐羌。人尤勁悍,性多質直。皆務于農事,工習獵射,于書計非其長矣。蜀郡、臨邛、眉山、隆山、資陽、瀘川、巴東、遂甯、巴西、新城、金山、普安、犍為、越巂、柯、黔安,得蜀之舊域。其地四塞,山川重阻,水陸所湊,貨殖所萃,蓋一都之會也。昔劉備資之,以成三分之業。自金行喪亂,四海沸騰,李氏據之於前,譙氏依之於後。當梁氏將亡,武陵憑險而取敗,後周之末,王謙負固而速禍。故孟門不祀,古人所以誡焉。其風俗大抵與漢中不別。其人敏慧輕急,貌多蕞陋,
 頗慕文學,時有斐然,多溺于逸樂,少從宦之士,或至耆年白首,不離鄉邑。人多工巧,綾錦雕鏤之妙,殆侔于上國。貧家不務儲蓄,富室專於趨利。其處家室,則女勤作業,而士多自閑,聚會宴飲,尤足意錢之戲。小人薄于情禮,父子率多異居。其邊野富人,多規固山澤,以財物雄役夷、獠,故輕為奸藏,權傾州縣。此亦其舊俗乎?又有獽狿蠻賨,其居處風俗,衣服飲食,頗同於獠,而亦與蜀人相類。



\end{pinyinscope}