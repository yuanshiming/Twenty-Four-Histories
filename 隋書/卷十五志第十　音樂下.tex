\article{卷十五志第十 音樂下}

\begin{pinyinscope}

 開
 皇九年,平陳,獲宋、齊舊樂,詔於太常置清商署以管之。求陳太樂令蔡子元、于普明等,復居其職。由是牛弘奏曰:臣聞周有六代之樂,至《韶》、《武》而已。秦始皇改周舞曰《五行》,漢高帝改《韶舞》曰《文始》,以示不相襲也。又造《武德》,自表其功,故高帝廟奏《武德》、《文始》、《五行》之舞。又作《昭容》、《禮容》,增演其意。《昭容》生於《武德》,蓋猶古之《韶》也。《禮容》
 生於《文始》,矯秦之《五行》也。文帝又作《四時》之舞,故孝景帝立,追述先功,採《武德舞》作《昭德舞》,被之管弦,薦於太宗之廟。孝宣採《昭德舞》為《盛德舞》,更造新歌,薦於武帝之廟。據此而言,遞相因襲,縱有改作,並宗於《韶》。至明帝時,東平獻王採《文德舞》為《大武》之舞,薦於光武之廟。



 漢末大亂,樂章淪缺,魏武平荊州,獲杜夔,以為軍謀祭酒,使創雅樂。時散騎侍郎鄧靜善詠雅歌,樂師尹胡能習宗祀之曲,舞師馮肅曉知先代諸舞。總練研精,復於古樂,自夔始也。文帝黃初,改《昭容》之樂為《昭業樂》,《武德》之舞為《武頌舞》,《文始》之舞為《大韶舞》,《五行》之舞為《大武舞》。
 明帝初,公卿奏上太祖武皇帝樂曰《武始》之舞,高祖文皇帝樂曰《咸熙》之舞。又制樂舞,名曰《章斌》之舞,有事於天地宗廟及臨朝大饗,並用之。



 晉武帝泰始二年,遣傅玄等造行禮及上壽食舉歌詩。張華表曰:「按漢、魏所用,雖詩章辭異,興廢隨時,至其韻逗曲折,並系於舊,一皆因襲,不敢有所改也。」



 九年,荀勖典樂,使郭夏、宋識造《正德》、《大豫》之舞。改魏《昭武舞》曰《宣武舞》,羽籥舞曰《宜文舞》。江左之初,典章堙紊,賀循為太常卿,始有登歌之樂。大寧末,阮孚等又增益之。咸和間,鳩集遺逸,鄴沒胡後,樂人頗復南度,東晉因之,以具鐘律。太元間,破苻永固,又
 獲樂工楊蜀等,閑練舊樂,於是金石始備。尋其設懸音調,並與江左是同。



 慕容垂破慕容永於長子,盡獲苻氏舊樂。垂息為魏所敗,其鐘律令李佛等,將太樂細伎,奔慕容德於鄴。德遷都廣固,子超嗣立,其母先沒姚興,超以太樂伎一百二十人詣興贖母。及宋武帝入關,悉收南渡。永初元年,改《正德舞》曰《前舞》,《大武舞》曰《後舞》。文帝元嘉九年,太樂令鐘宗之,更調金石。至十四年,典書令奚縱,復改定之。又有《凱容》、《宣業》之舞,齊代因而用之。蕭子顯《齊書·志》曰:「宋孝建初,朝議以《凱容舞》為《韶舞》,《宣業舞》為《武德舞》。



 據《韶》為言,《宣業》即是古之《大武》,非《武德》也。」
 故《志》有《前舞凱容》歌辭,《後舞凱容》歌辭者矣。至於梁初,猶用《凱容》、《宣業》之舞,後改為《大壯》、《大觀》焉。今人猶喚《大觀》為《前舞》,故知樂名雖隨代而改,聲韻曲折,理應常同。



 前克荊州,得梁家雅曲,今平蔣州,又得陳氏正樂。史傳相承,以為合古。且觀其曲體,用聲有次,請修緝之,以備雅樂。其後魏洛陽之曲,據《魏史》云「太武平赫連昌所得」,更無明證。後周所用者,皆是新造,雜有邊裔之聲。戎音亂華,皆不可用。請悉停之。



 制曰:「制禮作樂,聖人之事也,功成化洽,方可議之。今宇內初平,正化未洽。遽有變革,我則未暇。」晉王廣又表請,帝乃許之。



 牛弘遂因鄭譯之
 舊,又請依古五聲六律,旋相為宮。雅樂每宮但一調,唯迎氣奏五調,謂之五音。縵樂用七調,祭祀施用。各依聲律尊卑為次。高祖猶憶妥言,注弘奏下,不許作旋宮之樂,但作黃鐘一宮而已。於是牛弘及秘書丞姚察、通直散騎常侍許善心、儀同三司劉臻、通直郎虞世基等,更共詳議曰:後周之時,以四聲降神,雖採《周禮》,而年代深遠,其法久絕,不可依用。



 謹案《司樂》:「凡樂,圜鐘為宮,黃鐘為角,太簇為徵,姑洗為羽,舞《雲門》以祭天。函鐘為宮,太簇為角,姑洗為徵,南呂為羽,舞《咸池》以祭地。黃鐘為宮,大呂為角,太簇為徵,圜鐘為羽,舞《韻》以祀宗廟。」馬融曰:「
 圜鐘,應鐘也。」賈逵、鄭玄曰:「圜鐘,夾鐘也。」鄭玄又云:「此樂無商聲,祭尚柔剛,故不用也。」干寶云:「不言商,商為臣。王者自謂,故置其實而去其名,若曰有天地人物,無德以主之,謙以自牧也。」先儒解釋,既莫知適從。然此四聲,非直無商,又律管乖次,以其為樂,無克諧之理。今古事異,不可得而行也。



 按《東觀書·馬防傳》,太子丞鮑鄴等上作樂事,下防。防奏言:「建初二年七月鄴上言,天子食飲,必順於四時五味,而有食舉之樂。所以順天地,養神明,求福應也。今官雅樂獨有黃鐘,而食舉樂但有太簇,皆不應月律,恐傷氣類。可作十二月均,各應其月氣。公卿朝
 會,得聞月律,乃能感天,和氣宜應。詔下太常評焉。太常上言,作樂器直錢百四十六萬,奏寢。今明詔復下,臣防以為可須上天之明時,因歲首之嘉月,發太簇之律,奏雅頌之音,以迎和氣。」其條貫甚具,遂獨施行。起於十月,為迎氣之樂矣。又《順帝紀》云:「陽嘉二年冬十月庚午,以春秋為闢雍,隸太學,隨月律。十月作應鐘,三月作姑洗。元和以來,音戾不調,修復黃鐘,作樂器,如舊典。」據此而言,漢樂宮懸有黃鐘均,食舉太簇均,止有二均,不旋相為宮,亦以明矣。計從元和至陽嘉二年,才五十歲,用而復止。驗黃帝聽鳳以制律呂,《尚書》曰「予欲聞六律五聲」,《
 周禮》有「分樂而祭」。此聖人制作,以合天地陰陽之和,自然之理,乃云音戾不調,斯言誣之甚也。



 今梁、陳雅曲,並用宮聲。按《禮》:「五聲十二律,還相為宮。」盧植云:「十二月三管流轉用事,當用事者為宮。宮,君也。」鄭玄曰:「五聲宮、商、角、徵、羽。其陽管為律,陰管為呂。布十二辰,更相為宮,始自黃鐘,終於南呂,凡六十也。」皇侃疏:「還相為宮者,十一月以黃鐘為宮,十二月以大呂為宮,正月以太簇為宮。餘月放此。凡十二管,各備五聲,合六十聲。五聲成一調,故十二調。」



 此即釋鄭義之明文,無用商、角、徵、羽為別調之法矣。《樂稽耀嘉》曰:「東方春,其聲角,樂當宮於夾鐘。餘
 方各以其中律為宮。」若有商、角之理,不得雲宮於夾鐘也。又云:「五音非宮不調,五味非甘不和。」又《動聲儀》:「宮唱而商和,是謂善本,太平之樂也。」《周禮》:「奏黃鐘,歌大呂,以祀天神。」鄭玄「以黃鐘之鐘,大呂之聲為均。」均,調也。故崔靈恩云:「六樂十二調,亦不獨論商、角、徵、羽也。」又云:「凡六樂者,皆文之以五聲,播之以八音。」故知每曲皆須五聲八音錯綜而能成也。《御寇子》云:「師文鼓琴,命宮而總四聲,則慶雲浮,景風翔。」唯《韓詩》云:「聞其宮聲,使人溫厚而寬大。聞其商聲,使人方廉而好義。」及古有清角、清徵之流。此則當聲為曲。今以五引為五聲,迎氣所用者是也。
 餘曲悉用宮聲,不勞商、角、徵、羽。何以得知?荀勖論三調為均首者,得正聲之名,明知雅樂悉在宮調。已外徵、羽、角,自為謠俗之音耳。且西涼、龜茲雜伎等,曲數既多,故得隸於眾調,調各別曲,至如雅樂少,須以宮為本,歷十二均而作,不可分配餘調,更成雜亂也。



 其奏大抵如此。帝並從之。故隋代雅樂,唯奏黃鐘一宮,郊廟饗用一調,迎氣用五調。舊工更盡,其餘聲律,皆不復通。或有能為蕤賓之宮者,享祀之際肆之,竟無覺者。



 弘又修皇后房內之樂,據毛萇、侯苞、孫毓故事,皆有鐘聲,而王肅之意,乃言不可。又陳統云:「婦人無外事,而陰教尚柔,柔以靜
 為體,不宜用於鐘。」弘等採肅、統以取正焉。高祖龍潛時,頗好音樂,常倚琵琶,作歌二首,名曰《地厚》、《天高》,托言夫妻之義。因即取之為房內曲。命婦人並登歌上壽並用之。職在宮內,女人教習之。



 初,後周故事,懸鐘磬法,七正七倍,合為十四。蓋準變宮、變徵,凡為七聲,有正有倍,而為十四也。長孫紹遠引《國語》冷州鳩云:「武王伐殷,歲在鶉火。」



 自鶉及駟,七位故也。既以七同其數,而以律和其聲,於是有七律。又引《尚書大傳》「謂之七始」,其注云:「謂黃鐘、林鐘、太簇、南呂、姑洗、應鐘、蕤賓也。」



 歌聲不應此者,皆去之。然據一均言也。宮、商、角、徵、羽為正,變宮、變徵為和,
 加倍而有十四焉。又梁武帝加以濁倍,三七二十一而同為架,雖取繁會,聲不合古。又後魏時,公孫崇設鐘磬正倍,參懸之。弘等並以為非,而據《周官·小胥職》「懸鐘磬,半之為堵,全之為肆」。鄭玄曰:「鐘磬編懸之,二八十六而在一虡。鐘一堵,磬一堵,謂之肆。」又引《樂緯》「宮為君,商為臣,君臣皆尊,各置一副,故加十四而懸十六」。又據漢成帝時,犍為水濱得石磬十六枚,此皆懸八之義也。懸鐘磬法,每虡準之,懸八用七,不取近周之法懸七也。



 又參用《儀禮》及《尚書大傳》,為宮懸陳布之法。北方北向,應鐘起西,磬次之,黃鐘次之,鐘次之,大呂次之,皆東陳。一建
 鼓在其東,東鼓。東方西向,太簇起北,磬次之,夾鐘次之,鐘次之,姑洗次之,皆南陳。一建鼓在其南,東鼓。



 南方北向,中呂起東,鐘次之,蕤賓次之,磬次之,林鐘次之,皆西陳。一建鼓在其西,西鼓。西方東向,夷則起南,鐘次之,南呂次之,磬次之,無射次之,皆北陳。一建鼓在其北,西鼓。其大射,則撤北面而加鉦鼓。祭天用雷鼓、雷鞀,祭地用靈鼓、靈鞀,宗廟用路鼓、路鞀。各兩設在懸內。



 又準《儀禮》,宮懸四面設鎛鐘十二虡,各依辰位。又甲、丙、庚、壬位,各設鐘一虡,乙、丁、辛、癸位,各陳磬一虡。共為二十虡。其宗廟殿庭郊丘社並同。



 樹建鼓於四隅,以象二十四氣。依
 月為均,四箱同作,蓋取毛傳《詩》云「四懸皆同」之義。古者鎛鐘據《儀禮》擊為節檢,而無合曲之義。又大射有二鎛,皆亂擊焉,乃無成曲之理。依後周以十二鎛相生擊之,聲韻克諧。每鎛鐘、建鼓各一人。



 每鐘、磬簨虡各一人,歌二人,執節一人,琴、瑟、箏、築各一人。每鐘虡,竽、笙、簫、笛、塤、篪各一人。懸內柷、敔各一人,柷在東,吾在西。二舞各八佾。



 樂人皆平巾幘、絳褠衣。樂器並採《周官》,參之梁代,擇用其尤善者。其簨虡皆金五博山,飾以崇牙,樹羽旒蘇。其樂器應漆者,天地之神皆硃漆,宗廟及殿庭則五色漆畫。晉、宋故事,箱別各有柷、敔,既同時戛之,今則不用。



 又《周官·大司樂》:「奏黃鐘,歌大呂,舞《雲門》,以祀天神。奏太簇,歌應鐘,舞《咸池》,以祭地祇。奏姑洗,歌南呂,舞《大韶》,以祀四望。奏蕤賓,歌函鐘,舞《大夏》,以祭山川。奏夷則,歌小呂,舞《大護》,以享先妣。



 奏無射,歌夾鐘,舞《大武》,以享先祖。」此乃周制,立二王三恪,通己為六代之樂。至四時祭祀,則分而用之。以六樂配十二調,一代之樂,則用二調矣。隋去六代之樂,又無四望、先妣之祭,今既與古祭法有別,乃以神祗位次分樂配焉。奏黃鐘,歌大呂,以祀圓丘。黃鐘所以宣六氣也,耀魄天神,最為尊極,故奏黃鐘以祀之。奏太簇,歌應鐘,以祭方澤。太簇所以贊陽出滯,昆
 侖厚載之重,故奏太簇以祀之。奏姑洗,歌南呂,以祀五郊、神州。姑洗所以滌潔百物,五郊神州,天地之次,故奏姑洗以祀之。奏蕤賓,歌函鐘,以祭宗廟。蕤賓所以安靜神人,祖宗有國之本,故奏蕤賓以祀之。奏夷則,歌小呂,以祭社稷、先農。夷則所以詠歌九穀,貴在秋成,故奏夷則以祀之。奏無射,歌夾鐘,以祭巡狩方岳。無射所以示人軌物,觀風望秩,故奏無射以祀之。同用文武二舞。其圓丘降神六變,方澤降神八變,宗廟禘祫降神九變,皆用《昭夏》。其餘祭享皆一變。又《周禮》,王出,奏《王夏》,尸出,奏《肆夏》。叔孫通法,迎神奏《嘉至》。今亦隨事立名。皇帝入出,
 皆奏《皇夏》。群官入出,皆奏《肆夏》。食舉上壽,奏《需夏》。迎、送神,奏《昭夏》。薦獻郊廟,奏《誠夏》。宴饗殿上,奏登歌。並文舞武舞,合為八曲。古有宮、商、角、徵、羽五引,梁以三朝元會奏之。今改為五音,其聲悉依宮商,不使差越。唯迎氣於五郊,降神奏之,《月令》所謂「孟春其音角」是也。通前為十三曲。並內宮所奏《天高》、《地厚》二曲,於房中奏之,合十五曲。



 其登歌法,準《禮·效特牲》「歌者在上,匏竹在下。」《大戴》云:「清廟之歌,懸一磬而尚拊搏。」又在漢代,獨登歌者,不以絲竹亂人聲。近代以來,有登歌五人,別升於上,絲竹一部,進處階前。此蓋《尚書》「戛擊鳴球,搏拊琴瑟以詠,祖考
 來格」之義也。梁武《樂論》以為登歌者頌祖宗功業,檢《禮記》乃非元日所奏。若三朝大慶,百闢俱陳,升工籍殿,以詠祖考,君臣相對,便須涕洟。



 以此說非通,還以嘉慶用之。後周登歌,備鐘、磬、琴、瑟,階上設笙、管。今遂因之。合於《儀禮》荷瑟升歌,及笙入,立於階下,間歌合樂,是燕飲之事矣。登歌法,十有四人,鐘東磬西,工各一人,琴、瑟、箏、築各一人,並歌者三人,執節七人,並坐階上。笙、竽、簫、笛、塤、篪各一人,並立階下。悉進賢冠,絳公服。斟酌古今,參而用之。祀神宴會通行之。若有大祀臨軒,陳於階壇之上。若冊拜王公,設宮懸,不用登歌。釋奠則唯用登歌,而不
 設懸。



 古者人君食,皆用當月之調,以取時律之聲。使不失五常之性,調暢四體,令得時氣之和。故鮑鄴上言,天子食飲,必順四時,有食舉樂,所以順天地,養神明,可作十二月均,感天和氣。此則殿庭月調之義也。祭祀既已分樂,臨軒朝會,並用當月之律。正月懸太簇之均,乃至十二月懸大呂之均,欲感君人情性,允協陰陽之序也。



 又文舞六十四人,並黑介幘,冠進賢冠,絳紗連裳,內單,皁褾、領、衣巽、裾、革帶,烏皮履。十六人執鸑。十六人執O。十六人執旄。十六人執羽,左手皆執籥。二人執纛,引前,在舞人數外,衣冠同舞人。武舞六十四人,並服武弁,硃褠
 衣,革帶,烏皮履。左執硃干,右執大戚,依硃干玉戚之文。二人執旌,居前,二人執鞀,二人執鐸。金錞二,四人輿,二人作。二人執鐃次之。二人執相,在左,二人執雅,在右,各工一人作。自旌以下來引,並在舞人數外,衣冠同舞人。《周官》所謂「以金錞和鼓,金鐲節鼓,金鐃止鼓,金鐸通鼓」也。又依《樂記》象德擬功,初來就位,總干而山立,思君道之難也。發揚蹈厲,威而不殘也。舞亂皆坐,四海咸安也。武,始而受命,再成而定山東,三成而平蜀道,四成而北狄是通,五成而江南是拓,六成復綴,以闡太平。高祖曰:「不須象功德,直象事可也。」然竟用之。近代舞出入皆作
 樂,謂之階步,咸用《肆夏》。今亦依定,即《周官》所謂樂出入奏鐘鼓也。又魏、晉故事,有《矛俞》、《弩俞》及硃儒導引。今據《尚書》直雲幹羽,《禮》文稱羽籥干戚。今文舞執羽籥,武舞執干戚,其《矛俞》、《弩俞》等,蓋漢高祖自漢中歸,巴、俞之兵,執仗而舞也。既非正典,悉罷不用。



 十四年三月,樂定。秘書監、奇章縣公牛弘,秘書丞、北絳郡公姚察,通直散騎常侍、虞部侍郎許善心,兼內史舍人虞世基,儀同三司、東宮學士饒陽伯劉臻等奏曰:「臣聞蕢桴土鼓,由來斯尚,雷出地奮,著自《易經》。邃古帝王,經邦馭物,揖讓而臨天下者,祀樂之謂也。秦焚經典,樂書亡缺,爰至漢興,始
 加鳩採,祖述增廣,緝成朝憲。魏、晉相承,更加論討,沿革之宜,備於故實。永嘉之後,九服崩離,燕、石、苻、姚,遁據華土。此其戎乎,何必伊川之上,吾其左衽,無復微管之功。前言往式,於斯而盡。金陵建社,朝士南奔,帝則皇規,粲然更備,與內原隔絕,三百年於茲矣。伏惟明聖膺期,會昌在運。今南征所獲梁、陳樂人,及晉、宋旗章,宛然俱至。曩代所不服者,今悉服之,前朝所未得者,今悉得之。



 化洽功成,於是乎在。臣等伏奉明詔,詳定雅樂,博訪知音,旁求儒彥,研校是非,定其去就,取為一代正樂,具在本司。」於是並撰歌辭三十首,詔並令施用,見行者皆停之。
 其人間音樂,流僻日久,棄其舊體者,並加禁約,務存其本。



 先是高祖遣內史侍郎李元操、直內史省盧思道等,列清廟歌辭十二曲。令齊樂人曹妙達於太樂教習,以代周歌。其初迎神七言,象《元基曲》,獻奠登歌六言,象《傾杯曲》,送神禮畢五言,象《行天曲》。至是弘等但改其聲,合於鐘律,而辭經敕定,不敢易之。至仁壽元年,煬帝初為皇太子,從饗於太廟,聞而非之。乃上言曰:「清廟歌辭,文多浮麗,不足以述宣功德,請更議定。」於是制詔吏部尚書、奇章公弘,開府儀同三司、領太子洗馬柳顧言,秘書丞、攝太常少卿許善心,內史舍人虞世基,禮部侍郎蔡
 徵等,更詳故實,創制雅樂歌辭。其祠圓丘,皇帝入,至版位定,奏《昭夏》之樂,以降天神。升壇,奏《皇夏》之樂。受玉帛,登歌,奏《昭夏》之樂。皇帝降南陛,詣罍洗,洗爵訖,升壇,並奏《皇夏》。初升壇,俎入,奏《昭夏》之樂。皇帝初獻,奏《諴夏》之樂。皇帝既獻,作文舞之舞。皇帝飲福酒,作《需夏》之樂。皇帝反爵於坫,還本位,奏《皇夏》之樂。武舞出,作《肆夏》之樂。送神作《昭夏》之樂。就燎位,還大次,並奏《皇夏》。



 圜丘:降神,奏《昭夏》辭:肅祭典,協良辰。具嘉薦,俟皇臻。禮方成,樂已變。感靈心,回天眷。闢華闕,下乾宮。
 乘精氣,御祥風。望爟火,通田燭。膺介圭,受瑄玉。神之臨,慶陰陽。煙衢洞,宸路深。善既福,德斯輔。流鴻祚,遍區宇。



 皇帝升壇,奏《皇夏》辭:於穆我君,昭明有融。道濟區域,功格玄穹。百神警衛,萬國承風,仁深德厚,信洽義豐。明發思政,勤憂在躬。鴻基惟永,福祚長隆。



 登歌辭:德深禮大,道高饗穆。就陽斯恭,陟配惟肅。血紘升氣,冕裘標服。誠感清玄,信陳史祝。
 祗承靈貺,載膺多福。



 皇帝初獻,奏《諴夏》辭:肇禋崇祀,大報尊靈。因高盡敬,掃地推誠。六宗隨兆,五緯陪營。雲和發韻,孤竹揚清。我粢既潔,我酌惟明。元神是鑒,百祿來成。



 皇帝既獻,奏文舞辭:皇矣上帝,受命自天。睿圖作極,文教遐宣。四方監觀,萬品陶甄。有苗斯格,無得稱焉。天地之經,和樂具舉。體徵咸萃,要荒式序。正位履端,秋霜春雨。



 皇帝飲福酒,奏《需夏》辭:禮以恭事,薦以饗時。載清玄酒,備潔薌萁。回旒分爵,思媚軒墀。惠均撤俎,祥降受釐。十倫以具,百福斯滋。克昌厥德,永祚鴻基。



 武舞辭:御歷膺期,乘乾表則。成功戡亂,順時經國。兵暢五材,武弘七德。憬彼遐裔,化行充塞。三道備舉,二儀交泰。情發自中,義均莫大。祀敬恭肅,鐘鼓繁會。萬國斯歡,兆人斯賴。享茲介福,康哉元首。惠我無疆,天長地久。



 送神奏《昭夏》辭:享序洽,祀禮施。神之駕,嚴將馳。奔精驅,長離耀。牲煙達,潔誠照。騰日馭,鼓電鞭。辭下土,升上玄。瞻寥廓,杳無際。澹群心,留餘惠。



 皇帝就燎,還大次,並奏《皇夏》,辭同上。



 五郊歌辭五首:迎送神、登歌,與圜丘同。



 青帝歌辭,奏角音:震宮初動,木德惟仁。龍精戒旦,鳥歷司春。陽光煦物,溫風先導。嚴處載驚,膏田已冒。
 犧牲豐潔,金石和聲。懷柔備禮,明德惟馨。



 赤帝歌辭,奏徵音:長贏開序,炎上為德。執禮司萌,持衡御國。重離得位,芒種在時。含櫻薦實,木權垂蕤。慶賞既行,高明可處。順時立祭,事昭福舉。



 黃帝歌辭,奏宮音:爰稼作土,順位稱坤。孕金成德,履艮為尊。黃本內色,宮實聲始。萬物資生,四時咸紀。靈壇汛掃,盛樂高張。威儀孔備,福履無疆。



 白帝歌辭,奏商音:
 西成肇節,盛德在秋。三農稍已,九穀行收。金氣肅殺,商威P戾。嚴風鼓莖,繁霜殞帶。厲兵詰暴,敕法慎刑。神明降嘏,國步惟寧。



 黑帝歌辭,奏羽音:玄英啟候,冥陵初起。虹藏於天,雉化於水。嚴關重閉,星回日窮。黃鐘動律,廣莫生風。玄樽示本,天產惟質。恩覃外區,福流景室。



 感帝奏《諴夏》辭:迎送神、登歌,與圜丘同。



 禘祖垂典,郊天有章。以春之孟,於國之陽。繭慄惟誠,陶匏斯尚。人神接禮,明幽交暢。
 火靈降祚,火歷載隆。蒸哉帝道,赫矣皇風。



 雩祭奏《諴夏》辭:迎送神、登歌,與圜丘同。



 硃明啟候時載陽,肅若舊典延五方。嘉薦以陳盛樂奏,氣序和平資靈祐。公田既雨私亦濡,人殷俗富政化敷。



 蠟祭奏《諴夏》辭:迎送神、登歌,與圜丘同。



 四方有祀,八蠟酬功。收藏既畢,榛葛送終。使之必報,祭之斯索。三時告勞,一日為澤。神祗必來,鱗羽咸致。惟義之盡,惟仁之至。年成物阜,罷役息人。皇恩已洽,靈慶無垠。



 朝日、夕月歌詩二首:迎送神、登歌,與圜丘同。



 朝日奏《諴夏》辭:扶木上朝暾,嵫山沉暮景。寒來游晷促,暑至馳輝永。時和合璧耀,俗泰重輪明。執圭盡昭事,服冕罄虔誠。



 夕月奏《諴夏》辭:澄輝燭地域,流耀鏡天儀。歷草隨弦長,珠胎逐望虧。成形表蟾兔,竊藥資王母。西郊禮既成,幽壇福惟厚。



 方丘歌辭四首:唯此四者異,餘並同圜丘。



 迎神奏《昭夏》辭:柔功暢,陰德昭。陳瘞典,盛玄郊。篚幕清,膋鬯馥。皇情虔,具僚肅。笙頌合,鼓鞀會。出桂旗,屯孔蓋。敬如在,肅有承。神胥樂,慶福膺。



 奠玉帛登歌:道惟生育,器乃包藏。報功稱範,殷薦有常,六瑚已饋,五齊流香。貴誠尚質,敬洽義彰。神祚惟永,帝業增昌。



 皇地祇歌辭,奏《諴夏》辭:
 厚載垂德,昆丘主神。陰壇吉禮,北至良辰。鑒水呈潔,牲慄表純。樽壺夕視,幣玉朝陳。群望咸秩,精靈畢臻。祚流於國,祉被於人。



 送神歌辭,奏《昭夏》辭:奠既徹,獻已周。竦靈駕,逝遠游。洞四極,匝九縣。慶方流,祉恆遍。埋玉氣,掩牲芬。晰神理,顯國文。



 神州奏《諴夏》辭:迎送神、登歌,與方丘同。



 四海之內,一和之壤。地曰神州,物賴生長。咸池既降,泰折斯饗。牲牷尚黑,珪玉實兩。
 九宇載寧,神功克廣。



 社稷歌辭四首:迎送神、登歌,與方丘同。



 春祈社,奏《諴夏》辭:厚地開靈,方壇崇祀。達以風露,樹之松梓。勾萌既申,芟柞伊始。恭祈粢盛,載膺休祉。



 春祈稷,奏《諴夏》辭:粒食興教,播厥有先。尊神致潔,報本惟虔。瞻榆束耒,望杏開田。方憑戩福,佇詠豐年。



 秋報社,奏《諴夏》辭:北墉申禮,單出表誠。豐犧入薦,華樂在庭。
 原顯既平,泉流又清。如雲已望,高廩斯盈。



 秋報稷,奏《諴夏》辭:人天務急,農亦勤止。或颭或藨,惟璟惟芑。涼風戒時,歲雲秋矣。物成則報,功施必祀。



 先農,奏《諴夏》辭:迎送神,與方丘同。



 農祥晨晰,土膏初起。春原俶載,青壇致祀。斂蹕長阡,回旌外壝。房俎飾薦,山罍沈滓。親事硃弦,躬持黛耜。恭神務穡,受釐降祉。



 先聖先師,奏《諴夏》辭:經國立訓,學重教先。《三墳》肇冊,《五典》留篇。
 開鑿理著,陶鑄功宣。



 東膠西序,春誦夏弦。芳塵載仰,祀典無騫。



 太廟歌辭:迎神歌辭:務本興教,尊神體國。霜露感心,享祀陳則。官聯式序,奔走在庭。幾筵結慕,裸獻惟誠。嘉樂載合,神其降止。永言保之,錫以繁祉。



 登歌辭:孝熙嚴祖,師象敬宗。惟皇肅事,有來邕邕。雕梁霞復,繡尞雲重。觀德自感,奉璋伊恭。
 彞斝盡飾,羽綴有容。升歌發藻,景福來從。



 俎入歌辭:郊丘、社、廟同。



 祭本用初,祀由功舉。駿奔咸會,供神有序。明酌盈樽,豐犧實俎。幽金既薦,繢錯維旅。享由明德,香非稷黍。載流嘉慶,克固鴻緒。



 皇高祖太原府君神室歌辭:締基發祥,肇源興慶。乃仁乃哲,克明克令。庸宣國圖,善流人詠。開我皇業,七百同盛。



 皇曾祖康王神室歌辭:皇條俊茂,帝系靈長。豐功疊軌,厚利重光。
 福由善積,代以德彰。嚴恭盡禮,永錫無疆。



 皇祖獻王神室歌辭:盛才必達,丕基增舊。涉渭同符,遷邠等構。弘風邁德,義高道富。神鑒孔昭,王猷克懋。



 皇考太祖武元皇帝神室歌辭:深仁冥著,至道潛敷。皇矣太祖,耀名天衢。翦商隆祚,奄宅隋區。有命既集,誕開靈符。



 飲福酒歌辭:郊丘、社、廟同。



 神道正直,祀事有融。肅邕備禮,莊敬在躬。羞燔已具,奠酹將終。降祥惟永,受福無窮。



 送神歌辭:饗禮具,利事成。佇旒冕,肅簪纓。金奏終,玉俎撤。盡孝敬,窮嚴潔。人祗分,哀樂半。降景福,憑幽贊。



 元會:皇帝出入殿庭,奏《皇夏》辭:郊丘、社、廟同。



 深哉皇度,粹矣天儀。司陛整蹕,式道先馳。八屯霧擁,七萃雲披。退揚進揖,步矩行規。勾陳乍轉,華蓋徐移。羽旗照耀,珪組陸離。居高念下,處安思危。照臨有度,紀律無虧。



 皇太子出入,奏《肆夏》辭:
 惟熙帝載,式固王猷。體乾建本,是曰孟侯。馳道美漢,寢門稱周。德心既廣,道業惟優。傅保斯導,賢才與游。瑜玉發響,畫輪停輈。皇基方峻,七鬯恆休。



 食舉歌辭八首:燔黍設教禮之始,五味相資火為紀。平心和德在甘旨,牢羞既陳鐘石俟,以斯而御揚盛軌。



 養身必敬禮食昭,時和歲阜庶物饒。鹽梅既濟鼎鉉調,特以膚臘加臐膮,
 威儀濟濟懋皇朝。



 饔人進羞樂侑作,川潛之膾雲飛勣。甘酸有宜芬勺藥,金敦玉豆盛交錯,御鼓既聲安以樂。



 玉食惟後膳必珍,芳菰既潔重秬新。是能安體又調神,荊包畢至海貢陳,用之有節德無垠。



 嘉羞入饋猶化謐,沃土名滋帝臺實。陽華之榮雕陵慄,鼎俎芬芳豆籩溢,通幽致遠車書一。



 道高物備食多方,山膚既善水豢良。桓蒲在位簨業張,加籩折俎爛成行,恩風下濟道化光。



 禮以安國仁為政,具物必陳饔牢盛。罝罘斤斧順時令,懷生熙熙皆得性,於茲宴喜流嘉慶。



 皇道四達禮樂成,臨朝日舉表時平。甘芳既飫醑以清,揚休玉卮正性情,隆我帝載永明明。



 上壽歌辭:
 俗已乂,時又良。朝玉帛,會衣裳。基同北辰久,壽共南山長。黎元鼓腹樂未央。



 宴群臣登歌辭:皇明馭歷,仁深海縣。載擇良辰,式陳高宴。隅隅卿士,昂昂侯甸。車旗煜龠,衣纓蔥蒨。樂正展懸,司宮飾殿。三揖稱禮,九賓為傳。圓鼎臨碑,方壺在面。



 《鹿鳴》成曲,嘉魚入薦。筐篚相輝,獻酬交遍。飲和飽德,恩風長扇。



 文舞歌辭:天眷有屬,後德惟明。君臨萬宇,昭事百靈。
 濯以江漢,樹之風聲。罄地必歸,窮天皆至。六戎仰朔,八蠻請吏。煙雲獻彩,龜龍表異。緝和禮樂,變理陰陽。功由舞見,德以歌彰。兩儀同大,日月齊光。



 武舞歌辭:惟皇御宇,惟帝乘乾。五材並用,七德兼宣。平暴夷險,拯溺救燔。九域載安,兆庶斯賴。績地之厚,補天之大。聲隆有截,化覃無外。鼓鐘既奮,干戚攸陳。功高德重,政謐化淳。鴻休永播,久而彌新。



 大射登歌辭:道謐金科照,時乂玉條明。優賢饗禮洽,選德射儀成。鑾旗鬱雲動,寶軑儼天行。巾車整三乏,司裘飾五正。鳴球響高殿,華鐘震廣庭。烏號傳昔美,淇衛著前名。揖讓皆時傑,升降盡朝英。附枝觀體定,杯水睹心平。豐觚既來去,燔炙復從橫。欣看禮樂盛,喜遇黃河清。



 《凱樂》歌辭三首:述帝德:
 於穆我後,睿哲欽明。膺天之命,載育群生。開元創歷,邁德垂聲。朝宗萬宇,祗事百靈。煥乎皇道,昭哉帝則。惠政滂流,仁風四塞。淮海未賓,江湖背德。運籌必勝,濯征斯克。八荒務卷,四表雲褰。雄圖盛略,邁後光前。寰區已泰,福祚方延。長歌凱樂,天子萬年。



 述諸軍用命:帝德遠覃,天維宏布。功高雲天,聲隆《韶《護》。惟彼海隅,未從王度。皇赫斯怒,元戎啟路。桓桓猛將,赳赳英謨。攻如燎發,戰似摧枯。
 救茲塗炭,克彼妖逋。塵清兩越,氣靜三吳。鯨鯢已夷,封疆載闢。班馬蕭蕭,歸旌弈弈。雲臺表效,司勛紀績。業並山河,道固金石。



 述天下太平:阪泉軒德,丹浦堯勛。始實以武,終乃以文。嘉樂聖主,大哉為君。出師命將,廓定重氛。書軌既並,干戈是戢。弘風設教,政成人立。禮樂聿興,衣裳載緝。風雲自美,嘉祥爰集。皇皇聖政,穆穆神猷。牢籠虞夏,度越姬劉。日月比曜,天地同休。永清四海,長帝九州。



 皇后房內歌辭:至順垂典,正內弘風。母儀萬國,訓範六宮。求賢啟化,進善宣功。家邦載序,道業斯融。



 大業元年,煬帝又詔修高廟樂,曰:「古先哲王,經國成務,莫不因人心而制禮,則天明而作樂。昔漢氏諸廟別所,樂亦不同,至於光武之後,始立共堂之制。



 魏文承運,初營廟寢,太祖一室,獨為別宮。自茲之後,兵車交爭,制作規模,日不暇給。伏惟高祖文皇帝,功侔造物,道濟生靈,享薦宜殊,樂舞須別。今若月祭時饗,既與諸祖共庭,至於舞功,獨於一室,交違禮意,未合人情。其詳議以聞。」



 有
 司未及陳奏,帝又以禮樂之事,總付秘書監柳顧言、少府副監何稠、著作郎諸葛潁、秘書郎袁慶隆等,增多開皇樂器,大益樂員,郊廟樂懸,並令新制。顧言等後親,帝復難於改作,其議竟寢。諸郊廟歌辭,亦並依舊制,唯新造《高祖廟歌》九首。今亡。又遣秘書省學士定殿前樂工歌十四首,終大業世,每舉用焉。帝又詔博訪知鐘律歌管者,皆追之。時有曹士立、裴文通、唐羅漢、常寶金等,雖知操弄,雅鄭莫分,然總付太常,詳令刪定。議修一百四曲,其五曲在宮調,黃鐘也;一曲應調,大呂也;二十五曲商調,太簇也;一十四曲角調,姑洗也;一十三曲變徵調,
 蕤賓也;八曲徵調,林鐘也;二十五曲羽調,南呂也;一十三曲變宮調,應鐘也。



 其曲大抵以詩為本,參以古調,漸欲播之弦歌,被之金石。仍屬戎車,不遑刊正,禮樂之事,竟無成功焉。



 自漢至梁、陳樂工,其大數不相逾越。及周並齊,隋並陳,各得其樂工,多為編戶。至六年,帝乃大括魏、齊、周、陳樂人子弟,悉配太常,並於關中為坊置之,其數益多前代。顧言等又奏,仙都宮內,四時祭享,還用太廟之樂,歌功論德,別制其辭。七廟同院,樂依舊式。又造饗宴殿庭宮懸樂器,布陳簨虡,大抵同前,而於四隅各加二建鼓、三案。又設十二鎛,鎛別鐘磬二架,各依辰位
 為調,合三十六架。至於音律節奏,皆依雅曲,意在演令繁會,自梁武帝之始也,開皇時,廢不用,至是又復焉。高祖時,宮懸樂器,唯有一部,殿庭饗宴用之。平陳所獲,又有二部,宗廟郊丘分用之。至是並於樂府藏而不用。更造三部:五郊二十架,工一百四十三人。廟庭二十架,工一百五十人。饗宴二十架,工一百七人。舞郎各二等,並一百三十二人。



 顧言又增房內樂,益其鐘磬,奏議曰:「房內樂者,主為王後弦歌諷誦而事君子,故以房室為名。燕禮饗飲酒禮,亦取而用也。故云:『用之鄉人焉,用之邦國焉。』文王之風,由近及遠,鄉樂以感人,須存雅正。既不
 設鐘鼓,義無四懸,何以取正於婦道也。《磬師職》云:『燕樂之鐘磬。」鄭玄曰:『燕樂,房內樂也,所謂陰聲,金石備矣。』以此而論,房內之樂,非獨弦歌,必有鐘磬也。《內宰職》云:『正後服位,詔其禮樂之儀。』鄭玄云:『薦撤之禮,當與樂相應。』薦撤之言,雖施祭祀,其入出賓客,理亦宜同。請以歌鐘歌磬,各設二虡,土革絲竹並副之,並升歌下管,總名房內之樂。女奴肄習,朝燕用之。」制曰:「可。」於是內宮懸二十虡。其鎛鐘十二,皆以大磬充。去建鼓,餘飾並與殿庭同。



 皇太子軒懸,去南面,設三鎛鐘於辰丑申,三建鼓亦如之。編鐘三虡,編磬三虡,共三鎛鐘為九虡。其登歌減者
 二人。簨虡金三博山。樂器應漆者硃漆之。其二舞用六佾。



 其雅樂鼓吹,多依開皇之故。雅樂合二十器,今列之如左:金之屬二:一曰鎛鐘,每鐘懸一簨虡,各應律呂之音,即黃帝所命伶倫鑄十二鐘,和五音者也。二曰編鐘,小鐘也,各應律呂,大小以次,編而懸之。上下皆八,合十六鐘,懸於一簨虡。



 石之屬一:曰磬,用玉若石為之,懸如編鐘之法。



 絲之屬四:一曰琴,神農制為五弦,周文王加二弦為七者也。二曰瑟,二十七弦,伏犧所作者也。三曰築,十二弦。
 四曰箏,十三弦,所謂秦聲,蒙恬所作者也。



 竹之屬三:一曰簫,十六管,長二尺,舜所造者也。二曰篪,長尺四寸,八孔,蘇公所作者也。三曰笛,凡十二孔,漢武帝時丘仲所作者也。京房備五音,有七孔,以應七聲。黃鐘之笛,長二尺八寸四分四厘有奇,其餘亦上下相次,以為長短。



 匏之屬二:一曰笙,二曰竽,並女媧之所作也。笙列管十九,於匏內施簧而吹之。竽大,三十六管。



 土之屬一:曰塤,六孔,暴辛公之所作者也。



 革之屬五:一曰建鼓,夏后氏加四足,謂之足鼓。殷人柱
 貫之,謂之楹鼓。周人懸之,謂之懸鼓。近代相承,植而貫之,謂之建鼓。蓋殷所作也。又棲翔鷺于其上,不知何代所加。或曰,鵠也,取其聲揚而遠聞。或曰,鷺,鼓精也。越王勾踐擊大鼓於雷門以壓吳。晉時移於建康,有雙鷺哾鼓而飛入雲。或曰,皆非也。《詩》云:「振振鷺,鷺于飛。鼓咽咽,醉言歸。」古之君子,悲周道之衰,頌聲之輟,飾鼓以鷺,存其風流。未知孰是。靈鼓、靈鞀,並入面。雷鼓、雷鞀,六面。路鼓、路鞀,四面。鼓以桴擊,鞀貫其中而手搖之。又有節鼓,不知誰所造也。



 木之屬二:一曰柷,如桶,方二尺八寸,中有椎柄,連底動
 之,令左右擊,以節樂。二曰敔,如伏獸,背有二十七鉏鋙,以竹長尺,橫櫟之,以止樂焉。



 簨虡,所以懸鐘磬,橫曰簨,飾以鱗屬,植曰虡,飾以臝及羽屬。簨加木板於上,謂之業。殷人刻其上為崇牙,以掛懸。周人畫繒為紵,戴之以璧,垂五採羽於其下,樹於簨虡之角。近代又加金博山於簨上,垂流蘇,以合採羽。五代相因,同用之。



 始開皇初定令,置《七部樂》:一曰《國伎》,二曰《清商伎》,三曰《高麗伎》,四曰《天竺伎》,五曰《安國伎》,六曰《龜茲伎》,七曰《文康伎》。又雜有疏勒、扶南、康國、百濟、突厥、新羅、倭國等伎。
 其後牛弘請存《鞞》、《鐸》、《巾》、《拂》等四舞,與新伎並陳。因稱:「四舞,按漢、魏以來,並施於宴饗。《鞞舞》,漢巴、渝舞也。至章帝造《鞞舞辭》云『關東有賢女』,魏明代漢曲云『明明魏皇帝』。《鐸舞》,傅玄代魏辭云『振鐸鳴金』,成公綏賦云『《鞞鐸》舞庭,八音並陳』是也。《拂舞》者,沈約《宋志》云:『吳舞,吳人思晉化。』其辭本云『白符鳩』是也。《巾舞》者,《公莫舞》也。伏滔云:『項莊因舞,欲劍高祖,項伯紆長袖以捍其鋒,魏、晉傳為舞焉。』檢此雖非正樂,亦前代舊聲。故梁武報沈約云:『《鞞》、《鐸》、《巾》、《拂》,古之遺風。』楊泓云:『此舞本二八人,桓玄即真,為八佾。後因而不改。』齊人王僧虔已論其事。平陳所得者,猶充
 八佾,於懸內繼二舞後作之,為失斯大。檢四舞由來,其實已久。請並在宴會,與雜伎同設,於西涼前奏之。」帝曰:「其聲音節奏及舞,悉宜依舊。惟舞人不須捉鞞拂等。」及大業中,煬帝乃定《清樂》、《西涼》、《龜茲》、《天竺》、《康國》、《疏勒》、《安國》、《高麗》、《禮畢》,以為《九部》。樂器工衣創造既成,大備於茲矣。



 《清樂》其始即《清商三調》是也,並漢來舊曲。樂器形制,並歌章古辭,與魏三祖所作者,皆被於史籍。屬晉朝遷播,夷羯竊據,其音分散。苻永固平張氏,始於涼州得之。宋武平關中,因而入南,不復存於內地。及平陳後獲之。高祖聽之,善其節奏,曰:「此華夏正聲也。昔因永嘉,流於江
 外,我受天明命,今復會同。



 雖賞逐時遷,而古致猶在。可以此為本,微更損益,去其哀怨,考而補之。以新定律呂,更造樂器。」其歌曲有《陽伴》,舞曲有《明君》、《並契》。其樂器有鐘、磬、琴、瑟、擊琴、琵琶、箜篌、築、箏、節鼓、笙、笛、簫、篪、塤等十五種,為一部。工二十五人。



 《西涼》者,起苻氏之末,呂光、沮渠蒙遜等,據有涼州,變龜茲聲為之,號為秦漢伎。魏太武既平河西得之,謂之《西涼樂》。至魏、周之際,遂謂之《國伎》。



 今曲項琵琶、豎頭箜篌之徒,並出自西域,非華夏舊器。《楊澤新聲》、《神白馬》之類,生於胡戎。胡戎歌非漢魏遺曲,故其樂器聲調,悉與書
 史不同。其歌曲有《永世樂》,解曲有《萬世豐》舞,曲有《于闐佛曲》。其樂器有鐘、磬、彈箏、搊箏、臥箜篌、豎箜篌、琵琶、五弦、笙、簫、大篳篥、長笛、小篳篥、橫笛、腰鼓、齊鼓、擔鼓、銅拔、貝等十九種,為一部。工二十七人。



 《龜茲》者,起自呂光滅龜茲,因得其聲。呂氏亡,其樂分散,後魏平中原,復獲之。其聲後多變易。至隋有《西國龜茲》、《齊朝龜茲》、《土龜茲》等,凡三部。開皇中,其器大盛於閭幹。時有曹妙達、王長通、李士衡、郭金樂、安進貴等,皆妙絕弦管,新聲奇變,朝改暮易,持其音技,估衒公王之間,舉時爭相慕尚。



 高祖病之,謂群臣曰:「聞公等皆好新變,所
 奏無復正聲,此不祥之大也。自家形國,化成人風,勿謂天下方然,公家家自有風俗矣。存亡善惡,莫不系之。樂感人深,事資和雅,公等對親賓宴飲,宜奏正聲;聲不正,何可使兒女聞也!」帝雖有此敕,而竟不能救焉。煬帝不解音律,略不關懷。後大制艷篇,辭極淫綺。令樂正白明達造新聲,創《萬歲樂》、《藏鉤樂》、《七夕相逢樂》、《投壺樂》、《舞席同心髻》、《玉女行觴》、《神仙留客》、《擲磚續命》、《鬥雞子》、《鬥百草》、《泛龍舟》、《還舊宮》、《長樂花》及《十二時》等曲,掩抑摧藏,哀音斷絕。帝悅之無已,謂幸臣曰:「多彈曲者,如人多讀書。讀書多則能撰書,彈曲多即能造曲。此理之然也。」因語明
 達云:「齊氏偏隅,曹妙達猶自封王。我今天下大同,欲貴汝,宜自修謹。」六年,高昌獻《聖明樂》曲,帝令知音者於館所聽之,歸而肄習。及客方獻,先於前奏之,胡夷皆驚焉。其歌曲有《善善摩尼》,解曲有《婆伽兒》,舞曲有《小天》,又有《疏勒鹽》。其樂器有豎箜篌、琵琶、五弦、笙、笛、簫、篳篥、毛員鼓、都曇鼓、答臘鼓、腰鼓、羯鼓、雞婁鼓、銅拔、貝等十五種,為一部。工二十人。



 《天竺》者,起自張重華據有涼州,重四譯來貢男伎,《天竺》即其樂焉。歌曲有《沙石疆》,舞曲有《天曲》。樂器有鳳首箜篌、琵琶、五弦、笛、銅鼓、毛員鼓、都曇鼓、銅拔、貝等九種,為
 一部。工十二人。



 《康國》,起自周武帝娉北狄為後,得其所獲西戎伎,因其聲。歌曲有《戢殿農和正》,舞曲有《賀蘭缽鼻始》、《末奚波地》、《農惠缽鼻始》、《前拔地惠地》等四曲。樂器有笛、正鼓、加鼓、銅拔等四種,為一部。工七人。



 《疏勒》、《安國》、《高麗》,並起自後魏平馮氏及通西域,因得其伎。後漸繁會其聲,以別於太樂。



 《疏勒》,歌曲有《亢利死讓樂》,舞曲有《遠服》,解曲有《鹽曲》。樂器有豎箜篌、琵琶、五弦、笛、簫、篳篥、答臘鼓、腰鼓、羯鼓、雞婁鼓等十種,為一部,工十二人。



 《
 安國》,歌曲有《附薩單時》,舞曲有《末奚》,解曲有《居和祗》。樂器有箜篌、琵琶、五弦、笛、簫、篳篥、雙篳篥、正鼓、和鼓、銅拔等十種,為一部。



 工十二人。



 《高麗》,歌曲有《芝棲》,舞曲有《歌芝棲》。樂器有彈箏、臥箜篌、豎箜篌、琵琶、五弦、笛、笙、簫、小篳篥、桃皮篳篥、腰鼓、齊鼓、擔鼓、貝等十四種,為一部。工十八人。



 《禮畢》者,本出自晉太尉庾亮家。亮卒,其伎追思亮,因假為其面,執翳以舞,象其容,取其謚以號之,謂之為《文康樂》。每奏九部樂終則陳之,故以禮畢為名。其行曲有《單交路》,舞曲有《散花》。樂器有笛、笙、簫、篪、鈴槃、鞞、腰鼓等七
 種,三懸為一部。工二十二人。



 始齊武平中,有魚龍爛漫、俳優、硃儒、山車、巨象、拔井、種瓜、殺馬、剝驢等,奇怪異端,百有餘物,名為百戲。周時,鄭譯有寵於宣帝,奏徵齊散樂人,並會京師為之。蓋秦角抵之流者也。開皇初,並放遣之。及大業二年,突厥染干來朝,煬帝欲誇之,總追四方散樂,大集東都。初於芳華苑積翠池側,帝帷宮女觀之。



 有舍利先來,戲於場內,須臾跳躍,激水滿衢,黿鼉龜鰲,水人蟲魚,遍覆於地。



 又有大鯨魚,噴霧翳日,倏忽化成黃龍,長七八丈,聳踴而出,名曰《黃龍變》。



 又以繩系兩柱,相去十丈,遣二倡女對舞
 繩上,相逢切肩而過,歌舞不輟。又為夏育扛鼎,取車輪石臼大甕器等,各於掌上而跳弄之。並二人戴竿,其上有舞,忽然騰透而換易之。又有神鰲負山,幻人吐火,千變萬化,曠古莫儔。染干大駭之。自是皆於太常教習。每歲正月,萬國來朝,留至十五日,於端門外,建國門內,綿亙八里,列為戲場。百官起棚夾路,從昏達旦,以縱觀之。至晦而罷。伎人皆衣錦繡繒彩。其歌舞者,多為婦人服,鳴環佩,飾以花毦者,殆三萬人。初課京兆、河南制此衣服,而兩京繒錦,為之中虛。三年,駕幸榆林,突厥啟民朝於行宮,帝又設以示之。六年,諸夷大獻方物。突厥啟民
 以下,皆國主親來朝賀。乃於天津街盛陳百戲,自海內凡有奇伎,無不總萃。崇侈器玩,盛飾衣服,皆用珠翠金銀,錦罽絺繡。其營費鉅億萬。關西以安德王雄總之,東都以齊王暕總之,金石匏革之聲,聞數十里外。彈弦手厭管以上,一萬八千人。大列炬火,光燭天地,百戲之盛,振古無比。自是每年以為常焉。



 故事,天子有事於太廟,備法駕,陳羽葆,以入於次。禮畢升車,而鼓吹並作。



 開皇十七年詔曰:「昔五帝異樂,三王殊禮,皆隨事而有損益,因情而立節文。仰惟祭享宗廟,瞻敬如在,罔極之感,情深茲日。而禮畢升路,鼓吹發音,
 還入宮門,金石振響。斯則哀樂同日,心事相違,情所不安,理實未允。宜改茲往式,用弘禮教。自今以後,享廟日不須設鼓吹,殿庭勿設樂懸。在廟內及諸祭,並依舊。其王公已下,祭私廟日,不得作音樂。」



 至大業中,煬帝制宴饗設鼓吹,依梁為十二案。案別有錞于、鉦、鐸、軍樂鼓吹等一部。案下皆熊罷貙豹,騰倚承之,以象百獸之舞。其大駕鼓吹,並硃漆畫。



 大駕鼓吹、小鼓加金鐲、羽葆鼓、鐃鼓、節鼓,皆五採重蓋,其羽葆鼓,仍飾以羽葆。長鳴、中鳴、大小橫吹,五採衣幡,緋掌,畫交龍,五採腳。大角幡亦如之。



 大鼓、長鳴、大橫吹、節鼓及橫吹後笛、簫、篳篥、笳、桃皮
 篳篥等工人服,皆緋地苣文為袍褲及帽。金鉦、㭎鼓,其鉦鼓皆加八角紫傘。小鼓、中鳴、小橫吹及橫吹後笛、簫、篳篥、笳、桃皮篳篥等工人服,並青地苣文袍褲及帽。羽葆鼓、鐃及歌、簫、笳工人服,並武弁,硃褠衣,革帶。大角工人,平巾幘,緋衫,白布大口褲。其鼓吹督帥服,與大角同。以下準督帥服,亦如之。



 㭎鼓一曲,十二變,與金鉦同。夜警用一曲俱盡。次奏大鼓。大鼓,一十五曲供大駕,一十二曲供皇太子,一十曲供王公等。小鼓,九曲供大駕,三曲供皇太子及王公等。



 長鳴色角,一百二十具供大駕,三十六具供皇太子,十
 八具供王公等。



 次鳴色角,一百二十具供大駕,十二具供皇太子,一十具供王公等。



 大角,第一曲起捉馬,第二曲被馬,第三曲騎馬,第四曲行,第五曲入陣,第六曲收軍,第七曲下營。皆以三通為一曲。其辭並本之鮮卑。



 鐃鼓,十二曲供大駕,六曲供皇太子,三曲供王公等。其樂器有鼓,並歌簫、笳。



 大橫吹,二十九曲供大駕,九曲供皇太子,七曲供王公。其樂器有角、節鼓、笛、簫、篳篥、笳、桃皮篳篥。



 小橫吹,十二
 曲供大駕,夜警則十二曲俱用。其樂器有角、笛、簫、篳篥、笳、桃皮篳篥。



\end{pinyinscope}