\article{卷八志第三 禮儀三}

\begin{pinyinscope}

 陳
 永定三年七月,武帝崩。新除尚書左丞庾持稱:「晉、宋以來,皇帝大行儀注,未祖一日,告南郊太廟,奏策奉謚。梓宮將登轀輬,侍中版奏,已稱某謚皇帝。



 遣奠,出於陛階下,方以此時,乃讀哀策。而前代策文,猶云大行皇帝,請明加詳正。」國子博士、領步兵校尉、知儀禮沈文阿等謂:「應劭《風俗通》,前帝謚未定,臣子稱大行,以別嗣主。近
 檢梁儀,自梓宮將登轀輬,版奏皆稱某謚皇帝登轀輬。伏尋今祖祭已奉策謚,哀策既在庭,遣祭不應猶稱大行。且哀策篆書,藏於玄宮。」謂「依梁儀稱謚,以傳無窮」。詔可之。



 天嘉元年八月癸亥,尚書儀曹請今月晦皇太后服安吉君禫除儀注。沈洙議:「謂至親期斷,加降故再期,而再周之喪,斷二十五月。但重服不可頓除,故變之以纖縞,創巨不可便愈,故稱之以祥禫。禫者,淡也,所以漸祛其情。至如父在為母出適後之子,則屈降之以期。期而除服,無復衰麻。緣情有本同之義,許以心制。



 心制既無杖
 絰可除,不容復改玄『,既是心憂,則無所更淡其心也。且禫杖期者,十五月已有禫制。今申其免懷之感,故斷以再周,止二十五月而已。所以宋元嘉立義,心喪以二十五月為限。大明中,王皇后父喪,又申明其制。齊建元中,太子穆妃喪,亦同用此禮。唯王儉《古今集記》云心制終二十七月,又為王逡所難。何佟之儀注用二十五月而除。案古循今,宜以再周二十五月為斷。今皇太后於安吉君心喪之期,宜除於再周,無復心禫之禮。」詔可之。



 隋制,諸嶽崩瀆竭,天子素服,避正寢,撤膳三日。遣使祭崩竭之山川,牲用太牢。



 皇帝本服大功已上親及外祖父母、皇后父母、諸官正一品喪,皇帝不視事三日。



 皇帝本服五服內親及嬪、百官正二品已上喪,並一舉哀。太陽虧、國忌日,皇帝本服小功緦麻親、百官三品已上喪,皇帝皆不視事一日。



 皇太后、皇后為本服五服內諸親及嬪,一舉哀。皇太子為本服五服之內親及東宮三師、三少、宮臣三品已上,一舉哀。



 梁天監元年,齊臨川獻王所生妾謝墓被發,不至埏門。蕭子晉傳重,諮禮官何佟之。佟之議,以為:「改葬服緦,見柩不可無服故也。此止侵墳土,不及於槨,可依新宮火
 處三日哭假而已。」帝以為得禮。二年,何佟之議:「追服三年無禫。」



 尚書議,並以佟之言為得。



 又二年,始興王嗣子喪。博士管晅議,使國長從服緦麻。



 四年,掌兇禮嚴植之定《儀注》,以亡月遇閏,後年中祥,疑所附月。帝曰:「閏蓋餘分,月節則各有所隸。若節屬前月,則宜以前月為忌,節屬後月,則宜以後月為忌。祥逢閏則宜取遠日。」



 又四年,安成國刺稱:「廟新建,欲克今日遷立所生吳太妃神主。國王既有妃喪,欲使臣下代祭。」明山賓議,以為:「不可。宜待王妃服竟,親奉盛禮。」



 五年,貴嬪母車喪,議者疑其儀。明山賓以為:「貴嬪既居母憂,皇太子出貴嬪別第,一舉哀,以申聖情,庶不乖禮。」帝從之。



 又五年,祠部郎司馬褧牒:「貴嬪母車亡,應有服制」,謂「宜準公子為母麻衣之制,既葬而除」。帝從之。



 六年,申明葬制,凡墓不得造石人獸碑,唯聽作石柱,記名位而已。



 七年,安成王慈太妃喪,周舍牒:「使安成、始興諸王以成服日一日為位受吊。」



 帝曰:「喪無二主。二王既在遠,嗣子宜祭攝事。」周舍牒:「嗣子著細布衣、絹領帶。單衣用十五
 升葛。凡有事及歲時節朔望,並於靈所朝夕哭。三年不聽樂。」



 十四年,舍人硃異議:「《禮》,年雖未及成人,已有爵命者,則不為殤。封陽侯年雖中殤,已有拜封,不應殤服。」帝可之。於是諸王服封陽侯依成人之服。



 大同六年,皇太子啟:「謹案下殤之小功,不行婚冠嫁三嘉之禮,則降服之大功,理不得有三嘉。今行三嘉之禮,竊有小疑。」帝曰:「《禮》云:『大功之末,可以冠子。父小功之末,可以冠子、嫁子、娶婦。己雖小功,既卒哭,可以冠、娶妻。下殤之小功則不可。』晉代蔡謨、謝沈、丁纂、馮懷等遂云:『降
 服大功,可以嫁女。』宋代裴松之、何承天又云:『女有大功之服,亦得出嫁。』範堅、荀伯子等,雖復率意致難,亦未能折。太始六年,虞和立議:『大功之末,乃可娶婦。』於時博詢,咸同和議。齊永明十一年,有大司馬長子之喪,武帝子女同服大功。左丞顧杲之議云:『大功之末,非直皇女嬪降無疑,皇子娉納,亦在非硋。』凡此諸議,皆是公背正文,務為通耳。徐爰、王文憲並云:『期服降為大功,皆不可以婚嫁。』於義乃為不乖,而又不釋其意。天監十年,信安公主當出適,而有臨川長子大功之慘,具論此義,粗已詳悉。太子今又啟審大功之末乃下殤之小功行婚冠嫁
 三吉之事。案《禮》所言下殤小功,本是期服,故不得有三吉之禮。況本服是期,降為大功,理當不可。人間行者,是用鄭玄逆降之義。《雜記》云:『大功之末,可以冠子嫁子。』此謂本服大功,子則小功,逾月以後,於情差輕,所以許有冠嫁。



 此則小功之末,通得取婦。前所云『大功之末,可以冠子嫁子』,此是簡出大功之身,不得取婦。後言『小功之末,可以冠子嫁子』,非直子得冠嫁,亦得取婦。故有出沒。婚禮國之大典,宜有畫一。今宗室及外戚,不得復輒有乾啟,禮官不得輒為曲議。可依此以為法。」



 後齊定令,親王、公主、太妃、妃及從三品已上喪者,借白
 鼓一面,喪畢進輸。



 王、郡公主、太妃、儀同三司已上及令僕,皆聽立兇門柏歷。三品已上及五等開國,通用方相。四品已下,達於庶人,以魌頭。旌則一品九旒,二品、三品七旒,四品、五品五旒,六品、七品三旒,八品已下,達於庶人,唯旐而已。其建旐,三品已上及開國子、男,其長至軫,四品、五品至輪,六品至於九品至較。勛品達於庶人,不過七尺。



 王元軌子欲改葬祖及祖母,列上未知所服。邢子才議曰:「《禮》『改葬緦麻』。



 鄭玄注:『臣為君,子為父,妻為夫。』唯三人而已。然嫡曾孫、孫承重者,曾祖父母、祖父母改葬,既並
 三年之服,皆應服緦。而止言三人,若非遺漏,便是舉其略耳。」



 開皇初,高祖思定典禮。太常卿牛弘奏曰:「聖教陵替,國章殘缺,漢、晉為法,隨俗因時,未足經國庇人,弘風施化。且制禮作樂,事歸元首,江南王儉,偏隅一臣,私撰儀注,多違古法。就廬非東階之位,兇門豈設重之禮?兩蕭累代,舉國遵行。後魏及齊,風牛本隔,殊不尋究,遙相師祖,故山東之人,浸以成俗。西魏已降,師旅弗遑,賓嘉之禮,盡未詳定。今休明啟運,憲章伊始,請據前經,革茲俗弊。」詔曰:「可。」弘因奏徵學者,撰儀禮百卷。悉用東齊《儀注》以
 為準,亦微採王儉禮。修畢,上之,詔遂班天下,咸使遵用焉。



 其喪紀,上自王公,下逮庶人,著令皆為定制,無相差越。正一品薨,則鴻臚卿監護喪事,司儀令示禮制。二品已上,則鴻臚丞監護,司儀丞示禮制。五品已上薨、卒,及三品已上有期親已上喪,並掌儀一人示禮制。官人在職喪,聽斂以朝服,有封者,斂以冕服,未有官者,白帢單衣。婦人有官品者,亦以其服斂。棺內不得置金銀珠寶。諸重,一品懸鬲六,五品已上四,六品已下二。轜車,三品已上油幰,硃絲絡網,施襈,兩箱畫龍,幰竿諸末垂六旒蘇。七品已上油幰,施襈,兩箱畫雲氣,垂四旒蘇。八品已
 下,達於庶人,鱉甲車,無幰襈旒蘇畫飾。執紼,一品五十人,三品已上四十人,四品三十人,並布幘深衣。三品已上四引、四披、六鐸、六翣。五品已上二引、二披、四鐸、四翣。九品已上二鐸、二翣。四品已上用方相,七品已上用魌頭。在京師葬者,去城七里外。三品已上立碑,螭首龜趺。趺上高不得過九尺。七品已上立碣,高四尺。圭首方趺。若隱淪道素,孝義著聞者,雖無爵,奏,聽立碣。



 三年及期喪,不數閏。大功已下數之。以閏月亡者,祥及忌日,皆以閏所附之月為正。



 兇服不入公門。期喪已下不解官者,在外曹示聶緣紗帽。
 若重喪被起者,皁絹下裙帽。若入宮殿及須朝見者,冠服依百官例。



 齊衰心喪已上,雖有奪情,並終喪不吊不賀不預宴。期喪未練,大功未葬,不吊不賀,並終喪不預宴。小功已下,假滿依例。居五服之喪,受冊及之職,儀衛依常式,唯鼓樂從而不作。若以戎事,不用此制。



 自秦兼天下,朝覲之禮遂廢。及周封蕭詧為梁王,訖於隋,恆稱籓國,始有朝見之儀。梁王之朝周,入畿,大塚宰命有司致積。其餼五牢,米九十筥,皞醢各三十五甕,酒十八壺,米禾各五十車,薪芻各百車。既至,大司空設九儐以致館。梁王束帛
 乘馬,設九介以待之。禮成而出。明日,王朝,受享於廟。既致享,大塚宰又命公一人,玄冕乘車,陳九儐,以束帛乘馬,致食於賓及賓之從各有差。致食訖,又命公一人,弁服乘車,執贄,設九儐以勞賓。王設九介,迎於門外。明日,朝服乘車,還贄於公。公皮弁迎於大門,授贄受贄,並於堂之中楹。又明日,王朝服,設九介,乘車,備儀衛,以見於公。事畢,公致享。明日,三孤一人,又執贄勞於梁王。明日,王還贄。又明日,王見三孤,如見三公。明日,卿一人,又執贄勞王。



 王見卿,又如三孤。於是三公、三孤、六卿,又各餼賓,並屬官之長為使。牢米束帛同三公。



 開皇四年正月,梁主蕭巋朝於京師,次於郊外。詔廣平王楊雄、吏部尚書韋世康持節以迎。衛尉設次於驛館。雄等降就便幕。巋服通天冠、絳紗袍、端珽,立於東階下,西面。文武陪侍,如其國。雄等立於門右,東面。巋攝內史令柳顧言出門請事。世康曰:「奉詔勞於梁帝。」顧言入告。巋出,迎於館門之外,西面再拜。



 持節者導雄與巋俱入,至於庭下。巋北面再拜受詔訖。雄等乃出,立於館門外道右東向。巋送於門外,西面再拜。及奉見,高祖冠通天冠,服絳紗袍,御大興殿,如朝儀。巋服遠游冠,朝服以入,君臣並拜,禮畢而出。



 古者天子征伐,則宜於社,造於祖,類於上帝。還亦以牲遍告。梁天監初,陸璉議定軍禮,遵其制。帝曰:「宜者請征討之宜,造者稟謀於廟,類者奉天時以明伐,並明不敢自專。陳幣承命可也。」璉不能對。嚴植之又爭之,於是告用牲幣,反亦如之。



 後齊天子親征纂嚴,則服通天冠,文物充庭。有司奏更衣,乃入,冠武弁,弁左貂附蟬以出。誓訖,擇日備法駕,乘木輅,以選於廟。載遷廟主於齋車,以俟行。



 次宜於社,有司以毛血釁軍鼓,載帝社石主於車,以俟行。次擇日陳六軍,備大駕,類於上帝。次擇日祈后土、神州、岳鎮、海瀆、
 源川等。乃為坎盟,督將列牲於坎南,北首。有司坎前讀盟文,割牲耳,承血。皇帝受牲耳,遍授大將,乃置於坎。



 又歃血,歃遍,又以置坎。禮畢,埋牲及盟書。又卜日,建牙旗於單,祭以太牢,及所過名山大川,使有司致祭。將屆戰所,卜剛日,備玄牲,列軍容,設柴於辰地,為墠而祃祭。大司馬奠矢,有司奠毛血,樂奏《大護》之音。禮畢,徹牲,柴燎。



 戰前一日,皇帝禱祖,司空禱社。戰勝則各報以太牢。又以太牢賞用命戰士於祖,引功臣入旌門,即神庭而授版焉。又罰不用命於社,即神庭行戮訖,振旅而還。格廟詣社訖,擇日行飲至禮,文物充庭。有司執簡,紀年號月
 朔,陳六師凱入格廟之事,飲至策勛之美,因述其功,不替賞典焉。



 隋制,行幸所過名山大川,則有司致祭。岳瀆以太牢,山川以少牢。親征及巡狩,則類上帝、宜社、造廟,還禮亦如之,將發軔,則「祭。其禮,有司於國門外委土為山象,設埋坎。有司刳羊,陳俎豆。駕將至,委奠幣,薦脯醢,加羊於「西首。又奠酒解羊,並饌埋於坎。駕至,太僕祭兩軹及軌前,乃飲,授爵,遂轢「上而行。



 大業七年,征遼東,煬帝遣諸將於薊城南桑乾河上築社稷二壇,設方壝,行宜社禮。帝齋於臨朔宮懷荒殿,預
 告官及侍從各齋於其所。十二衛士並齋。帝袞冕玉輅,備法駕。禮畢,御金輅,服通天冠,還宮。又於宮南類上帝,積柴於燎壇,設高祖位於東方。帝服大裘以冕,乘玉輅,祭奠玉帛,並如宜社。諸軍受胙畢,帝就位,觀燎,乃出。又於薊城北設壇,祭馬祖於其上,亦有燎。又於其日,使有司並祭先牧及馬步,無鐘鼓之樂。眾軍將發,帝御臨朔宮,親授節度。每軍大將、亞將各一人。騎兵四十隊。隊百人置一纛。十隊為團,團有偏將一人。第一團,皆青絲連明光甲、鐵具裝、青纓拂,建狻猊旗。第二團,絳絲連硃犀甲、獸文具裝、赤纓拂,建貔貅旗。第三團,白絲連明光甲、
 鐵具裝、素纓拂,建闢邪旗。第四團,烏絲連玄犀甲、獸文具裝、建纓拂,建六駁旗。前部鼓吹一部,大鼓、小鼓及鼙、長鳴、中鳴等各十八具,㧏鼓、金鉦各二具。後部鐃吹一部,鐃二面,歌簫及笳各四具,節鼓一面,吳吹篳篥、橫笛各四具,大角十八具。又步卒八十隊,分為四團。



 團有偏將一人。第一團,每隊給青隼蕩幡一。第二團,每隊黃隼蕩幡一。第三團,每隊白隼蕩幡一。第四團,每隊蒼隼蕩幡一。長槊楯弩及甲毦等,各稱兵數。受降使者一人,給二馬軺車一乘,白獸幡及節各一,騎吏三人,車輻白從十二人。承詔慰撫,不受大將制。戰陣則為監軍。軍將發,
 候大角一通,步卒第一團出營東門,東向陣。第二團出營南門,南向陣。第三團出營西門,西向陣。第四團出營北門,北向陣。陣四面團營,然後諸團嚴駕立。大角三通,則鐃鼓俱振,騎第一團引行。



 隊間相去各十五步。次第二團,次前部鼓吹,次弓矢一隊,合二百騎。建蹲獸旗,瓟槊二張,大將在其下。次誕馬二十匹,次大角,次後部鐃,次第三團,次第四團,次受降使者。次及輜重戎車散兵等,亦有四團。第一輜重出,收東面陣,分為兩道,夾以行。第二輜重出,收南面陣,夾以行。第三輜重出,收西面陣,夾以行。第四輜重出,收北面陣,夾以行。亞將領五百騎,
 建騰豹旗,殿軍後。至營,則第一團騎陣於東面,第二團騎陣於南面,鼓吹翊大將居中,駐馬南向。第三團騎陣於西面,第四團騎陣於北面,合為方陣。四團外向,步卒翊輜重入於陣內,以次安營。營定,四面陣者,引騎入營。亞將率驍騎游弈督察。其安營之制,以車外布,間設馬槍,次施兵幕,內安雜畜。事畢,大將、亞將等,各就牙帳。其馬步隊與軍中散兵,交為兩番,五日而代。於是每日遣一軍發,相去四十里,連營漸進。二十四日續發而盡。首尾相繼,鼓角相聞,旌旗亙九百六十里。天子六軍次發,兩部前後先置,又亙八十里。通諸道合三十軍,亙一千
 四十里。諸軍各以帛為帶,長尺五寸,闊二寸,題其軍號為記。御營內者,合十二衛、三臺、五省、九寺,並分隸內外前後左右六軍,亦各題其軍號,不得自言臺省。王公已下,至於兵丁廝隸,悉以帛為帶,綴於衣領,名「軍記帶」。諸軍並給幡數百,有事,使人交相去來者,執以行。不執幡而離本軍者,他軍驗軍記帶,知非部兵,則所在斬之。是歲也,行幸望海鎮,於禿黎山為壇,祀黃帝,行祃祭。詔太常少卿韋霽、博士褚亮奏定其禮。皇帝及諸預祭臣近侍官諸軍將,皆齋一宿。有司供帳設位,為埋坎神坐西北,內壝之外。建二旗於南門外。以熊席設帝軒轅神坐
 於壝內,置甲胄弓矢於坐側,建槊於坐後。皇帝出次入門,群官定位,皆再拜奠。禮畢,還宮。



 隋制,常以仲春,用少牢祭馬祖於大澤,諸預祭官,皆於祭所致齋一日,積柴於燎壇,禮畢,就燎。仲夏祭先牧,仲秋祭馬社,仲冬祭馬步,並於大澤,皆以剛日。牲用少牢,如祭馬祖,埋而不燎。



 開皇二十年,太慰晉王廣北伐突厥,四月己未,次於河上,祃祭軒轅黃帝,以太牢制幣,陳甲兵,行三獻之禮。



 後齊命將出征,則太卜詣太廟,灼靈龜,授鼓旗於廟。皇帝陳法駕,服袞冕,至廟,拜於太祖。遍告訖,降就中階,引
 上將,操鉞授柯,曰:「從此上至天,將軍制之。」又操斧授柯,曰:「從此下至泉,將軍制之。」將軍既受斧鉞,對曰:「國不可從外理,軍不可從中制。臣既受命,有鼓旗斧鉞之威,願假一言之命於臣。」



 帝曰:「茍利社稷,將軍裁之。」將軍就車,載斧鉞而出。皇帝推轂度閫,曰:「從此以外,將軍制之。」



 周大將出征,遣太祝,以羊一,祭所過名山大川。明帝武成元年,吐谷渾寇邊。



 帝常服乘馬,遣大司馬賀蘭祥於太祖之廟,司憲奉鉞,進授大將。大將拜受,以授從者。禮畢,出受甲兵。



 隋制,皇太子親戎,及大將出師,則以豭肫一釁鼓,皆告
 社廟。受斧鉞訖,不得反宿於家。開皇八年,晉王廣將伐陳,內史令李德林攝太尉,告於太祖廟。禮畢,又命有司宜於太社。



 古者三年練兵,入而振旅,至於春秋蒐浯,亦以講其事焉。梁、陳時,依宋元嘉二十五年蒐宣武場。其法,置行軍殿於幕府山南岡,並設王公百官幕。先獵一日,遣馬騎布圍。右領軍將軍督右,左領軍將軍督左,大司馬董正諸軍。獵日,侍中三奏,一奏搥一鼓為嚴,三嚴訖,引仗為小駕鹵簿。皇帝乘馬戎服,從者悉絳衫幘,黃麾警蹕,鼓吹如常儀。獵訖,宴會享勞,比校多少。戮一人以懲亂法。
 會畢,還宮。



 後齊常以季秋,皇帝講武於都外。有司先萊野為場,為二軍進止之節。又別墠於北場,輿駕停觀。遂命將簡士,教眾為戰陣之法。凡為陣,少者在前,長者在後。



 其還,則長者在前,少者在後。長者持弓矢,短者持旌旗。勇者持鉦鼓刀楯,為前行,戰士次之,槊者次之,弓箭為後行。將帥先教士目,使習見旌旗指麾之蹤,發起之意,旗臥則跪。教士耳,使習金鼓動止之節,聲鼓則進,鳴金則止。教士心,使知刑罰之苦,賞賜之利。教士手,使習持五兵之便,戰鬥之備。教士足,使習跪及行列嶮泥之塗。前五日,
 皆請兵嚴於場所,依方色建旗為和門。都墠之中及四角,皆建五採牙旗。應講武者,各集於其軍。戒鼓一通,軍士皆嚴備。二通,將士貫甲。



 三通,步軍各為直陣以相俟。大將各處軍中,立旗鼓下。有司陳小駕鹵簿,皇帝武弁,乘革輅,大司馬介胄乘,奉引入行殿。百司陪列。位定,二軍迭為客主。先舉為客,後舉為主。從五行相勝法,為陣以應之。



 後齊春蒐禮,有司規大防,建獲旗,以表獲車。蒐前一日,命布圍。領軍將軍一人,督左甄,獲軍將軍一人,督右甄。大司馬一人,居中,節制諸軍。天子陳小駕,服通天冠,乘
 木輅,詣行宮。將親禽,服戎服,鈒戟者皆嚴。武衛張甄圍,旗鼓相望,銜枚而進。甄常開一方,以令三驅。圍合,吏奔騎令曰:「鳥獸之肉,不登於俎者不射。皮革齒牙,骨角毛羽,不登於器者不射。」甄合,大司馬鳴鼓促圍,眾軍鼓噪鳴角,至期處而止。大司馬屯北旌門,二甄帥屯左右旌門。天子乘馬,從南旌門入,親射禽。謁者以護車收禽,載還,陳於護旗之北。王公已下以次射禽,皆送旗下。事畢,大司馬鳴鼓解圍,復屯。殿中郎中率其屬收禽,以實護車。天子還行宮。命有司每禽擇取三十,一曰乾豆,二曰賓客,三曰充君之皰。其餘即於圍下量食高將士。禮畢,改
 服,鈒者韜刃而還。夏苗、秋獮、冬狩,禮皆同。河清中定令,每歲十二月半後講武,至晦逐除。二軍兵馬,右入千秋門,左入萬歲門,並至永巷南下,至昭陽殿北,二軍交。一軍從西上閣,一軍從東上閣,並從端門南,出閶闔門前橋南,戲射並訖,送至城南郭外罷。



 後齊三月三日,皇帝常服乘輿,詣射所,升堂即坐,皇太子及群官坐定,登歌,進酒行爵。皇帝入便殿,更衣以出,驊騮令進御馬,有司進弓矢。帝射訖,還御坐,射懸侯,又畢,群官乃射五埒。一品二品三十發,一發調馬,十發射下,十發射上,三發射麞,三發射帖,三發射獸頭。三品二十五發,一發調馬,五發射下,十發射上,三發射麞,三發
 射帖,三發射獸頭。四品二十發一發調馬,五發射下,八發射上,二發射麞,二發射帖,二發射獸頭。五品十五發一發調馬,四發射下,五發射上,二發射麞,二發射帖,一發射獸頭。侍官御仗已上十發一發調馬,四發射下,五發射上。季秋大射,皇帝備大駕,常服,御七寶輦,射七埒。正三品已上,第一埒,一品五十發,一發調馬,十五發射下,二十五發射上,三發射麞,三發射帖,三發射獸頭。二品四十六發一發調馬,十五發射下,二十二發射上,二發射麞,三發射帖,三發射獸頭。從三品四品第二埒,三品四十二發一發調馬,十二發射下,二十二發射上,二發射麞,二發射帖,三發射獸頭。四品三十七發一發調馬,十一發射下,十九發射上,一發射麞,二發射帖,三發射獸頭。五品第三埒,三十二發一發調馬,九發射下,十七發射上,一發射麞,二發射帖,二發射獸頭。六品第四埒,二十七發。一發調馬,八發射下,十六發射上,一發射麞,一發射帖。七
 品第五埒,二十一發一發調馬,六發射下,十二發射上,一發射麞,一發射帖。八品第六埒,十六發一發調馬,四發射下,九發射上,一發射麞,一發射帖。九品第七埒,十發。一發調馬,三發射下,四發射上,一發射麞,一發射帖。大射置大將太尉公為之。射司馬各一人,錄事二人。七埒各置埒將、射正參軍各一人,埒士四人,威儀一人,乘白馬以導,的別參軍一人,懸侯下府參軍一人。又各置令史埒士等員,以司其事。



 後周仲春教振旅,大司馬建大麾於萊田之所。鄉稍之官,以旂物鼓鐸鉦鐃,各帥其人而致。誅其後至者。建麾於後表之中,以集眾庶。質明,偃麾,誅其不及者。



 乃陳徒騎,如戰之陣。大司馬北面誓之。軍中皆聽鼓角,以為進
 止之節。田之日,於所萊之北,建旗為和門。諸將帥徒騎序入其門。有司居門,以平其人。既入而分其地,險野則待前而騎後,易野則騎前而徒後。既陣,皆坐,乃設驅逆騎,有司表狢於陣前。以太牢祭黃帝軒轅氏,於狩地為墠,建二旗,列五兵於坐側,行三獻禮。



 遂蒐田致禽以祭社。仲夏教茇舍,如振旅之陣,遂以苗田如蒐法,致禽以享礿。仲秋教練兵,如振旅之陣,遂以獮田如蒐法,致禽以祀方。仲冬教大閱,如振旅之陣,遂以狩田如蒐法,致禽以享烝。



 孟秋迎太白,候太白夕見於西方。先見三日,大司馬戒
 期,遂建旗於陽武門外。



 司空除壇兆,有司薦毛血,登歌奏《昭夏》。在位者拜,事畢出。其日中後十刻,六軍士馬,俱介胄集旗下。左右武伯督十二帥嚴街,侍臣文武,俱介胄奉迎。樂師撞黃鐘,右五鐘皆應。皇帝介胄,警蹕以出,如常儀而無鼓角,出國門而「Q祭。



 至則舍於次。太白未見五刻,中外皆嚴,皇帝就位,六軍鼓噪,行三獻之禮。每獻,鼓噪如初獻。事訖,燔燎賜胙,畢,鼓噪而還。



 隋制,大射祭射侯於射所,用少牢。軍人每年孟秋閱戎具,仲冬教戰法。及大業三年,煬帝在榆林,突厥啟民及西域、東胡君長,並來朝貢。帝欲誇以甲兵之盛,乃命有
 司陳冬狩之禮。詔虞部量拔延山南北周二百里,並立表記。前狩二日,兵部建旗於表所。五里一旗,分為四十軍,軍萬人,騎五千匹。前一日,諸將各帥其軍,集於旗下。鳴鼓,後至者斬。詔四十道使,並揚旗建節,分申佃令,即留軍所監獵。



 布圍,圍闕南面,方行而前。帝服紫褲褶、黑介幘,乘闟豬車,其飾如木輅,重輞漫輪,虯龍繞轂,漢東京鹵簿所謂獵車者也。駕六黑鳷。太常陳鼓笳鐃簫角於帝左右,各百二十。百官戎服騎從,鼓行入圍。諸將並鼓行赴圍。乃設驅逆騎千有二百。闟豬停軔,有司斂大綏,王公已下,皆整弓矢,陳於駕前。有司又斂小綏,乃驅
 獸出,過於帝前。初驅過,有司整御弓矢以前,待詔。再驅過,備身將軍奉進弓矢。三驅過,帝乃從禽,鼓吹皆振,坐而射之。每驅必三獸以上。帝發,抗大綏。



 次王公發,則抗小綏。次諸將發射之,無鼓,驅逆之騎乃止。然後三軍四夷百姓皆獵。凡射獸,自左膘而射之,達於右腢,為上等。達右耳本,為次等。自左髀達於右鋋為下等。群獸相從,不得盡殺。已傷之獸,不得重射。又逆向人者,不射其面。



 出表者不逐之。佃將止,虞部建旗於圍內。從駕之鼓及諸軍鼓俱振,卒徒皆噪。諸獲禽者,獻於旗所,致其左耳。大獸公之,以供宗廟,使歸,薦臘於京師。小獸私之。



 齊制,季冬晦,選樂人子弟十歲以上十二以下為侲子,合二百四十人。一百二十人,赤幘、皁褠衣,執鞀。一百二十人赤布褲褶,執鞞角。方相氏黃金四目,熊皮蒙首,玄衣硃裳,執戈揚楯。又作窮奇、祖明之類,凡十二獸,皆有毛角。鼓吹令率之,中黃門行之,冗從僕射將之,以逐惡鬼於禁中。其日戊夜三唱,開諸里門,儺者各集,被服器仗以待事。戊夜四唱,開諸城門,二衛皆嚴。上水一刻,皇帝常服,即御座。王公執事官第一品已下、從六品已上,陪列預觀。儺者鼓噪,入殿西門,遍於禁內。分出二上閣,作方相與十二獸儛戲,喧呼周遍,前後鼓噪。出殿南門,
 分為六道,出於郭外。



 隋制,季春晦,儺,磔牲於宮門及城四門,以禳陰氣。秋分前一日,禳陽氣。



 季冬傍磔、大儺亦如之。其牲,每門各用羝羊及雄雞一。選侲子如後齊。冬八隊,二時儺則四隊。問事十二人,赤幘褠衣,執皮鞭。工人二十二人。其一人方相氏,黃金四目,蒙熊皮,玄衣硃裳。其一人為唱師,著皮衣,執棒。鼓角各十。有司預備雄雞羝羊及酒,於宮門為坎。未明,鼓噪以入。方相氏執戈揚楯,周呼鼓噪而出,合趣顯陽門,分詣諸城門。將出,諸祝師執事,預副牲胸,磔之於門,酌酒禳祝。



 舉牲並酒埋之。



 後齊制,日蝕,則太極殿西廂東向,東堂東廂西向,各設御座。群官公服。晝漏上水一刻,內外皆嚴。三門者閉中門,單門者掩之。蝕前三刻,皇帝服通天冠,即御座,直衛如常,不省事。有變,聞鼓音,則避正殿,就東堂,服白袷單衣。侍臣皆赤幘,帶劍,升殿侍。諸司各於其所,赤幘,持劍,出戶向日立。有司各率官屬,並行宮內諸門、掖門,屯衛太社。鄴令以官屬圍社,守四門,以硃絲繩繞系社壇三匝。太祝令陳辭責社。太史令二人,走馬露版上尚書,門司疾上之。又告清都尹鳴鼓,如嚴鼓法。日光復,乃止,奏解嚴。



 後魏每攻戰克捷,欲天下知聞,乃書帛,建於竿上,名為露布。其後相因施行。



 開皇中,乃詔太常卿牛弘、太子庶子裴政撰宣露布禮。及九年平陳,元帥晉王以驛上露布。兵部奏,請依新禮宣行。承詔集百官、四方客使等,並赴廣陽門外,服朝衣,各依其列。內史令稱有詔,在位者皆拜。宣訖,拜,蹈舞者三,又拜。郡縣亦同。



\end{pinyinscope}