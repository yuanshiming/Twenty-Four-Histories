\article{卷七十一列傳第三十六}

\begin{pinyinscope}

 誠節《易》稱:「聖人大寶曰位,何以守位曰仁。」又云:「立人之道曰仁與義。」



 然而士之立身成名,在乎仁義而已。故仁道不遠,則殺身以成仁,義重於生,則捐生而取義。是以龍逢投軀於夏癸,比干竭節於商辛,申蒯斷臂於齊莊,弘演納肝於衛懿。爰逮漢之紀信、欒布,晉之向雄、嵇紹,凡在立名之士,莫不庶幾焉。至於臨難忘身,見危授命,雖斯
 文不墜,而行之蓋寡,固知士之所重,信在茲乎!非夫內懷鐵石之心,外負凌霜之節,孰能安之若命,赴蹈如歸者也。皇甫誕等,當擾攘之際,踐必死之機,白刃臨頸,確乎不拔,可謂歲寒貞柏,疾風勁草,千載之後,懍懍如生。豈獨聞彼伯夷,懦夫立志,亦冀將來君子,有所庶幾。故掇採所聞,為《誠節傳》。



 劉弘劉弘,字仲遠,彭城叢亭里人,魏太常卿芳之孫也。少好學,有行檢,重節概。



 仕齊行臺郎中、襄城、沛郡、穀陽三郡太守、西楚州刺史。及齊亡,周武帝以為本郡太守。尉迥
 之亂也,遣其將席毗掠徐、兗。弘勒兵拒之,以功授儀同、永昌太守、齊州長史。志在立功,不安佐職。平陳之役,表請從軍,以行軍長史從總管吐萬緒度江。以功加上儀同,封濩澤縣公,拜泉州刺史。會高智慧作亂,以兵攻州,弘城守百餘日,救兵不至。前後出戰,死亡太半,糧盡無所食,與士卒數百人煮犀甲腰帶,及剝樹皮而食之,一無離叛。賊知其饑餓,欲降之,弘抗節彌厲。賊悉眾來攻,城陷,為賊所害。上聞而嘉嘆者久之,賜物二千段。子長信,襲其官爵。



 皇甫誕陶模敬釗
 皇甫誕,字玄慮,安定烏氏人也。祖和,魏膠州刺史。父璠,周隋州刺史。誕少剛毅,有器局。周畢王引為倉曹參軍。高祖受禪,為兵部侍郎。數年,出為魯州長史。開皇中,復入為比部、刑部二曹侍郎,俱有能名。遷治書侍御史,朝臣無不肅憚。上以百姓多流亡,令誕為河南道大使以檢括之。及還,奏事稱旨,上甚悅,令判大理少卿。明年,遷尚書右丞,俄以母憂去職。未期,起令視事。尋轉尚書左丞。時漢王諒為並州總管,朝廷盛選僚佐,前後長史、司馬,皆一時名士。上以誕公方著稱,拜並州總管司馬,總府政事,一以諮之,諒甚敬焉。及煬帝即位,徵諒入朝,諒
 用諮議王頍之謀,發兵作亂。誕數諫止,諒不納。誕因流涕曰:「竊料大王兵資,無敵京師者,加以君臣位定,逆順勢殊,士馬雖精,難以取勝。願王奉詔入朝,守臣子之節,必有松、喬之壽,累代之榮。如更遷延,陷身叛逆,一掛刑書,為布衣黔首不可得也。願察區區之心,思萬全之計,敢以死請。」諒怒而囚之。及楊素將至,諒屯清源以拒之。諒主簿豆盧毓出誕於獄,相與協謀,閉城拒諒。諒襲擊破之,並抗節而遇害。帝以誕亡身徇國,嘉悼者久之,下詔曰:「褒顯名節,有國通規,加等飾終,抑惟令典。並州總管司馬皇甫誕,性理淹通,志懷審正,效官贊務,聲績克
 宣。值狂悖構禍,兇威孔熾,確殉單誠,不從妖逆。雖幽縶寇手,而雅志彌厲,遂潛與義徒據城抗拒。眾寡不敵,奄致非命。可贈柱國,封弘義公,謚曰明。」子無逸嗣。



 無逸尋為淯陽太守,政甚有聲。《大業令》行,舊爵例除,以無逸誠義之後,賜爵平輿侯。入為刑部侍郎,守右武衛將軍。



 初,漢王諒之反也,州縣莫不響應,有嵐州司馬陶模、繁畤令敬釗,並抗節不從。



 陶模,京兆人也。性明敏,有器幹。仁壽初,為嵐州司馬。諒既作亂,刺史喬鐘葵發兵將赴逆,模拒之曰:「漢王所圖不軌,公荷國厚恩,致位方伯,謂當竭誠效命以答慈造,
 豈有大行皇帝梓宮未掩,翻為厲階!」鐘葵失色曰:「司馬反邪?」



 臨之以兵,辭氣不撓,葵義而釋之。軍吏進曰:「若不斬模,何以壓眾心?」於是囚之於獄,悉掠取資財,分賜黨與。及諒平,煬帝嘉之,拜開府,授大興令。楊玄感之反也,率兵從衛玄擊之,以攻進位銀青光祿大夫,卒官。



 敬釗字積善,河東蒲阪人也。父元約,周布憲中大夫。釗仁壽中為繁畤令,甚有能名。及賊至,力戰城陷。賊帥墨弼掠其資產而臨之以兵,釗辭氣不撓。弼義而止之,執送於偽將喬鐘葵所。鐘葵釋之,署為代州總管司馬,釗正色拒之,至於再三。鐘葵忿然曰:「受官則可,不然當斬!」
 釗答曰:「忝為縣宰,遭逢逆亂,進不能保境,退不能死節,為辱已多,何乃復以偽官相迫也?死生唯命,餘非所聞。」



 鐘葵怒甚,熟視釗曰:「卿不畏死邪?」復將殺之。會楊義臣軍至,鐘葵遽出戰,因而大敗,釗遂得免。大業三年,煬帝避暑汾陽宮,代州長史柳銓、司馬崔寶山上其狀,付有司將加褒賞,會虞世基奏格而止。後遷朝邑令,未幾,終。



 游元游元,字楚客,廣平任人,魏五更明根之玄孫也。父寶藏,位至太守。元少聰敏,年十六,齊司徒徐顯秀引為參軍事。周武帝平齊之後,歷壽春令、譙州司馬,俱有能名。
 開皇中,為殿內侍御史。晉王廣為揚州總管,以元為法曹參軍,父憂去職。後為內直監。煬帝嗣位,遷尚書度支郎。遼東之役,領左驍衛長史,為蓋牟道監軍,拜朝請大夫,兼治書侍御史。宇文述等九軍敗績,帝令元按其獄。述時貴幸,其子士及又尚南陽公主,勢傾朝廷。遣家僮造元,有所請屬。元不之見。他日,數述曰:「公地屬親賢,腹心是寄,當咎身責己,以勸事君,乃遣人相造,欲何所道?」



 按之愈急,仍以狀劾之。帝嘉其公正,賜朝服一襲。九年,奉使於黎陽督運,楊玄感作逆,乃謂元曰:「獨夫肆虐,天下士大夫肝腦塗地,加以陷身絕域之所,軍糧斷絕,此
 亦天亡之時也。我今親率義兵,以誅無道,卿意如何?」元正色答曰:「尊公荷國寵靈,功參佐命,高官重祿,近古莫儔,公之弟兄,青紫交映,當謂竭誠盡節,上答鴻恩。豈意墳土未幹,親圖反噬,深為明公不取,願思禍福之端。僕有死而已,不敢聞命。」玄感怒而囚之,屢脅以兵,竟不屈節,於是害之。帝甚嘉嘆,贈銀青光祿大夫,賜縑五百匹。拜其子仁宗為正議大夫、弋陽郡通守。



 馮慈明馮慈明,字無佚,信都長樂人也。父子琮,仕齊官至尚書右僕射。慈明在齊,以戚屬之故,年十四,為淮陽王開府
 參軍事。尋補司州主簿,進除中書舍人。周武平齊,授帥都督。高祖受禪,開三府官,除司空司倉參軍事。累遷行臺禮部侍郎。



 晉王廣為並州總管,盛選僚屬,以慈明為司士。後歷吏部員外郎,兼內史舍人。煬帝即位,以母憂去職。帝以慈明始事籓邸,後更在臺,意甚銜之,至是謫為伊吾鎮副。未之官,轉交止郡丞。大業九年,被徵入朝。時兵部侍郎斛斯政亡奔高麗,帝見慈明,深慰勉之。俄拜尚書兵曹郎,加位朝請大夫。十三年,攝江都郡丞事。



 李密之逼東都也,詔令慈明安集水廛、洛,追兵擊密。至鄢陵,為密黨崔樞所執。



 密延慈明於坐,勞苦之,因而謂曰:「
 隋祚已盡,區宇沸騰,吾躬率義兵,所向無敵,東都危急,計日將下。今欲率四方之眾,問罪於江都,卿以為何如?」慈明答曰:「慈明直道事人,有死而已,不義之言,非所敢對。」密不悅,冀其後改,厚加禮焉。慈明潛使人奉表江都,及致書東都留守,論賊形勢。密知其狀,又義而釋之。出至營門,賊帥翟讓怒曰:「爾為使人,為我所執,魏公相待至厚,曾無感戴,寧有畏乎?」慈明勃然曰:「天子使我來,正欲除爾輩,不圖為賊黨所獲。我豈從汝求活耶?欲殺但殺,何須罵詈!」因謂群賊曰:「汝等本無惡心,因饑饉逐食至此。官軍至,早為身計。」讓益怒,於是亂刀斬之。時年
 六十八。梁郡通守楊汪上狀,帝嘆惜之,贈銀青光祿大夫。拜其二子惇、怦俱為尚書承務郎。王充推越王侗為主,重贈柱國、戶部尚書、昌黎郡公,謚曰壯武。



 長子忱,先在東都,王充破李密,忱亦在軍中,遂遣奴負父尸柩詣東都,身不自送。未幾,又盛花燭納室。時論丑之。



 張須陀張須陀,弘農閿鄉人也。性剛烈,有勇略。弱冠從史萬歲討西爨,以功授儀同,賜物三百段。煬帝嗣位,漢王諒作亂並州,從楊素擊平之,加開府。大業中,為齊郡丞。會興遼東之役,百姓失業,又屬歲饑,穀米踴貴,須陀將開倉
 賑給,官屬咸曰:「須待詔敕,不可擅與。」須陀曰:「今帝在遠,遣使往來,必淹歲序。百姓有倒懸之急,如待報至,當委溝壑矣。吾若以此獲罪,死無所恨。」先開倉而後上狀,帝知之而不責也。明年,賊帥王薄聚結亡命數萬人,寇掠郡境。官軍擊之,多不利。須陀發兵拒之,薄遂引軍南,轉掠魯郡。須陀躡之,及於岱山之下。薄恃驟勝,不設備。須陀選精銳,出其不意擊之,薄眾大潰,因乘勝斬首數千級。薄收合亡散,得萬餘人,將北度河。須陀追之,至臨邑,復破之,斬五千餘級,獲六畜萬計。時天下承平日久,多不習兵,須陀獨勇決善戰。又長於撫馭,得士卒心,論者
 號為名將。薄復北戰,連豆子賊孫宣雅、石秪闍、郝孝德等眾十餘萬攻章丘。須陀遣舟師斷其津濟,親率馬步二萬襲擊,大破之,賊徒散走。既至津梁,復為舟師所拒,前後狼狽,獲其家累輜重不可勝計,露布以聞。帝大悅,優詔褒揚,令使者圖畫其形容而奏之。其年,賊裴長才、石子河等眾二萬,奄至城下,縱兵大掠。須陀未暇集兵,親率五騎與戰。賊競赴之,圍百餘重,身中數創,勇氣彌厲。會城中兵至,賊稍卻,須陀督軍復戰,長才敗走。後數旬,賊帥秦君弘、郭方預等合軍圍北海,兵鋒甚銳,須陀謂官屬曰:「賊自恃強,謂我不能救,吾今速去,破之必
 矣。」



 於是簡精兵,倍道而進,賊果無備,擊大破之,斬數萬級,獲輜重三千兩。司隸刺史裴操之上狀,帝遣使勞問之。十年,賊左孝友眾將十萬,屯於蹲狗山。須陀列八風營以逼之,復分兵扼其要害。孝友窘迫,面縛來降。其黨解象、王良、鄭大彪、李宛等眾各萬計,須陀悉討平之,威振東夏。以功遷齊郡通守,領河南道十二郡黜陟討捕大使。俄而賊廬明月眾十餘萬,將寇河北,次祝阿,須陀邀擊,殺數千人。



 賊呂明星、帥仁泰、霍小漢等眾各萬餘,擾濟北,須陀進軍擊走之。尋將兵拒東郡賊翟讓,前後三十餘戰,每破走之。轉滎陽通守。時李密說讓取洛口
 倉,讓憚須陀,不敢進。密勸之,讓遂與密率兵逼滎陽,須陀拒之。讓懼而退,須陀乘之,逐北十餘里。時李密先伏數千人於林木間,邀擊須陀軍,遂敗績。密與讓合軍圍之,須陀潰圍輒出,左右不能盡出,須陀躍馬入救之。來往數四,眾皆敗散,乃仰天曰:「兵敗如此,何面見天子乎?」乃下馬戰死。時年五十二。其所部兵,盡夜號哭,數日不止。越王侗遣左光祿大夫裴仁基招撫其眾,移鎮武牢。帝令其子元備總父兵,元備時在齊郡,遇賊,竟不果行。



 楊善會楊善會,字敬仁,弘農華陰人也。父初,官至毗陵太守。善
 會大業中為鄃令,以清正聞。俄而山東饑饉,百姓相聚為盜,善會以左右數百人逐捕之,往皆克捷。



 其後賊帥張金稱眾數萬,屯於縣界,屠城剽邑,郡縣莫能御。善會率勵所領,與賊搏戰,或日有數合,每挫其鋒。煬帝遣將軍段達來討金稱,善會進計於達,達不能用,軍竟敗焉。達深謝善會。後復與賊戰,進止一以謀之,於是大克。金稱復引渤海賊孫宣雅、高士達等眾數十萬,破黎陽而還,軍鋒甚盛。善會以勁兵千人邀擊,破之,擢拜朝請大夫、清河郡丞。金稱稍更屯聚,以輕兵掠冠氏。善會與平原通守楊元弘步騎數萬眾,襲其本營。武賁郎將王辯
 軍亦至,金稱釋冠氏來援,因與辯戰,不利,善會選精銳五百赴之,所當皆靡,辯軍復振。賊退守本營,諸軍各還。於時山東思亂,從盜如市,郡縣微弱,陷沒相繼。能抗賊者,唯善會而已。前後七百餘陣,未嘗負敗,每恨眾寡懸殊,未能滅賊。會太僕楊義臣討金稱,復為賊所敗,退保臨清。取善會之策,頻與決戰,賊乃退走。乘勝遂破其營,盡俘其眾。金稱將數百人遁逃,後歸漳南,招集餘黨。善會追捕斬之,傳首行在所。帝賜以尚方甲槊弓劍,進拜清河通守。其年,從楊義臣斬漳南賊帥高士達,傳首江都宮,帝下詔褒揚之。士達所部將竇建德,自號長樂王,
 來攻信都。臨清賊王安阻兵數千,與建德相影響。善會襲安斬之。建德既下信都,復擾清河,善會逆拒之,反為所敗,嬰城固守。賊圍之四旬,城陷,為賊所執。建德釋而禮之,用為貝州刺史。善會罵之曰:「老賊何敢擬議國士!恨吾力劣,不能擒汝等。我豈是汝屠酤兒輩,敢欲更相吏邪?」



 臨之以兵,辭氣不撓。建德猶欲活之,為其部下所請,又知終不為己用,於是害之。



 清河士庶莫不傷痛焉。



 獨孤盛獨孤盛,上柱國楷之弟也。性剛烈,有膽氣。煬帝在籓,盛以左右從,累遷為車騎將軍。及帝嗣位,以籓邸之舊,漸
 見親待,累轉為右屯衛將軍。宇文化及之作亂也,裴虔通引兵至成象殿,宿衛者皆釋仗而走。盛謂虔通曰:「何物兵?形勢太異也!」虔通曰:「事勢已然,不預將軍事。將軍慎無動。」盛大罵曰:「老賊是何物語!」不及被甲,與左右十餘人逆拒之,為亂兵所殺。越王侗稱制,贈光祿大夫、紀國公,謚曰武節。



 元文都元文都,洵陽公孝矩之兄子也。父孝則,周小塚宰、江陵總管。文都性鯁直,明辯有器幹。仕周為右侍上士。開皇初,授內史舍人,歷庫部、考功二曹郎,俱有能名。擢為尚
 書左丞,轉太府少卿。煬帝嗣位,轉司農少卿、司隸大夫,尋拜御史大夫,坐事免。未幾,授太府卿,帝漸任之,甚有當時之譽。大業十三年,帝幸江都宮,詔文都與段達、皇甫無逸、韋津等同為東都留守。及帝崩,文都與達、津等共推越王侗為帝。侗署文都為內史令、開府儀同三司、光祿大夫、左驍衛大將軍、攝右翊衛將軍、魯國公。既而宇文化及立秦王浩為帝,擁兵至彭城,所在響震。文都諷侗遣使通於李密。密於是請降,因授官爵,禮其使甚厚。王充不悅,因與文都有隙。文都知之,陰有誅充之計。侗復以文都領御史大夫,充固執而止。盧楚說文都曰:「
 王充外軍一將耳,本非留守之徒,何得預吾事!且洛口之敗,罪不容誅,今者敢懷跋扈,宰制時政,此而不除,方為國患。」文都然之,遂懷奏入殿。事臨發,有人以告充。充時在朝堂,懼而馳還含嘉城,謀作亂。文都頻遣呼之,充稱疾不赴。至夜作亂,攻東太陽門而入,拜於紫微觀下。侗遣人謂之曰:「何為者?」



 充曰:「元文都、盧楚謀相殺害,請斬文都,歸罪司寇。」侗見兵勢漸盛,度終不免,謂文都曰:「公自見王將軍也。」文都遷延而泣,侗遣其署將軍黃桃樹執文都以出。文都顧謂侗曰:「臣今朝亡,陛下亦當夕及。」侗慟哭而遣之,左右莫不憫默。出至興教門,充令左
 右亂斬之,諸子並見害。



 盧楚盧楚,涿郡範陽人也。祖景祚,魏司空掾。楚少有才學,鯁急口吃,言語澀難。



 大業中,為尚書右司郎,當朝正色,甚為公卿所憚。及帝幸江都,東都官僚多不奉法,楚每存糾舉,無所回避。越王侗稱尊號,以楚為內史令、左備身將軍、攝尚書左丞、右光祿大夫,封涿郡公,與元文都等同心戮力以輔幼主。及王充作亂,兵攻太陽門,武衛將軍皇甫無逸斬關逃難,呼楚同去。楚謂之曰:「僕與元公有約,若社稷有難,誓以俱死,今舍去不義。」及兵入,楚匿
 於太官署,賊黨執之,送於充所。充奮袂令斬之,於是鋒刃交下,肢體糜碎。



 劉子翊劉子翊,彭城叢亭里人也。父徧,齊徐州司馬。子翊少好學,頗解屬文,性剛謇,有吏乾。仕齊殿中將軍。開皇初,為南和丞,累轉秦州司法參軍事。十八年,入考功,尚書右僕射楊素見而異之,奏為侍御史。時永寧令李公孝四歲喪母,九歲外繼,其後父更別娶後妻,至是而亡。河間劉炫以無撫育之恩,議不解任。子翊駁之曰:《傳》云:「繼母如母,與母同也。」當以配父之尊,居母之位,齊杖之制,皆
 如親母。又「為人後者,為其父母期」。報期者,自以本生,非殊親之與繼也。



 父雖自處傍尊之地,於子之情,猶須隆其本重。是以令云:「為人後者,為其父母並解官,申其心喪。父卒母嫁,為父後者雖不服,亦申心喪。其繼母嫁不解官。」



 此專據嫁者生文耳。將知繼母在父之室,則制同親母。若謂非有撫育之恩,同之行路,何服之有乎?服既有之,心喪焉可獨異?三省令旨,其義甚明。今言令許不解,何其甚謬!且後人者為其父母期,未有變隔以親繼,親繼既等,故知心喪不殊。



 《服問》云:「母出則為繼母之黨服。」豈不以出母族絕,推而遠之,繼母配父,引而親之乎?
 子思曰:「為伋也妻,是為白也母。有為人及也妻,是不為白也母。」



 定知服以名重,情因父親,所以聖人敦之以孝慈,弘之以名義。是使子以名服,同之親母,繼以義報,等之己生。如謂繼母之來,在子出之後,制有淺深者,考之經傳,未見其文。譬出後之人,所後者初亡,後之者始至,此復可以無撫育之恩而不服重乎?昔長沙人王毖,漢末為上計詣京師,既而吳、魏隔絕,毖於內國更娶,生子昌。毖死後為東平相,始知吳之母亡,便情系居重,不攝職事。於時議者,不以為非。然則繼母之與前母,於情無別。若要以撫育始生服制,王昌復何足云乎?又晉鎮南將
 軍羊祜無子,取弟子伊為子。祜薨,伊不服重,祜妻表聞,伊辭曰:「伯生存養己,伊不敢違。然無父命,故還本生。」尚書彭權議:「子之出養,必由父命,無命而出,是為叛子。」於是下詔從之。然則心服之制,不得緣恩而生也。



 論云:「禮者稱情而立文,仗義而設教。」還以此義,諭彼之情。稱情者,稱如母之情,仗義者,仗為子之義。名義分定,然後能尊父順名,崇禮篤敬。茍以母養之恩始成母子,則恩由彼至,服自己來,則慈母如母,何得待父命?又云:「繼母慈母,本實路人,臨己養己,同之骨血。」若如斯言,子不由父,縱有恩育,得如母乎?其慈繼雖在三年之下,而居齊期
 之上,禮有倫例,服以稱情。繼母本以名服,豈藉恩之厚薄也。至於兄弟之子猶子也,私暱之心實殊,禮服之制無二。彼言「以」輕「如」重,自以不同。此謂如重之辭,即同重法,若使輕重不等,何得為「如」?律云「準枉法」者,但準其罪,「以枉法論」者,即同真法。律以弊刑,禮以設教,準者準擬之名,以者即真之稱。「如」「以」二字,義用不殊,禮律兩文,所防是一。將此明彼,足見其義,取譬伐柯,何遠之有。



 又論云:「取子為後者,將以供承祧廟,奉養己身,不得使宗子歸其故宅,以子道事本父之後妻也。」然本父後妻,因父而得母稱,若如來旨,本父亦可無心喪乎?何直父之後
 妻。論又云:「禮言舊君,其尊豈復君乎?已去其位,非復純臣,須言『舊』以殊之。別有所重,非復純孝,故言『其』已見之。目以其父之文,是名異也。」此又非通論。何以言之?「其「舊」訓殊,所用亦別,舊者易新之稱,其者因彼之辭,安得以相類哉?至如《禮》云:「其父析薪,其子不克負荷。」



 《傳》云:「衛雖小,其君在焉。」若其父而有異,其君復有異乎?斯不然矣,斯不然矣。今炫敢違禮乖令,侮聖干法,使出後之子,無情於本生,名義之分,有虧於風俗。徇飾非於明世,強媒蘗於禮經,雖欲揚己露才,不覺言之傷理。



 事奏,竟從子翊之議。仁壽中,為新豐令,有能名。大業三年,除大理正,
 甚有當時之譽。擢授治書侍御史,每朝廷疑議,子翊為之辯析,多出眾人意表。從幸江都。值天下大亂,帝猶不悟,子翊因侍切諫,由是忤旨,令子翊為丹陽留守。尋遣於上江督運,為賊吳棋子所虜。子翊說之,因以眾首。復遣領首賊清江。遇煬帝被殺,賊知而告之。子翊弗信,斬所言者。賊又欲請以為主,子翊不從。群賊執子翊至臨川城下,使告城中,云「帝已崩」。子翊反其言,於是見害,時年七十。



 堯君素陳孝意張季珣松贇堯君素,魏郡湯陰人也。煬帝為晉王時,君素以左右從。
 及嗣位,累遷鷹擊郎將。大業之末,盜賊蜂起,人多流亡,君素所部獨全。後從驍衛大將軍屈突通拒義兵於河東。俄而通引兵南遁,以君素有膽略,署領河東通守。義師遣將呂紹宗、韋義節等攻之,不克。及通軍敗,至城下呼之。君素見通,歔欷流涕,悲不自勝,左右皆哽咽,通亦泣下沾衿,因謂君素曰:「吾軍已敗,義旗所指,莫不響應。事勢如此,卿當早降,以取富貴。」君素答曰:「公當爪牙之寄,為國大臣,主上委公以關中,代王付公以社稷,國祚隆替,懸之於公。奈何不思報效,以至於此。縱不能遠慚主上,公所乘馬,即代王所賜也,公何面目乘之哉!」通曰:「
 籲!君素,我力屈而來。」君素曰:「方今力猶未屈,何用多言。」通慚而退。時圍甚急,行李斷絕,君素乃為木鵝,置表於頸,具論事勢,浮之黃河,沿流而下。河陽守者得之,達於東都。越王侗見而嘆息,於是承制拜君素為金紫光祿大夫,密遣行人勞苦之。監門直閣龐玉、武衛將軍皇甫無逸前後自東都歸義,俱造城下,為陳利害。大唐又賜金券,待以不死。君素卒無降心。其妻又至城下謂之曰:「隋室已亡,天命有屬,君何自苦,身取禍敗。」君素曰:「天下事非婦人所知。」引弓射之,應弦而倒。君素亦知事必不濟,然要在守死不易,每言及國家,未嘗不歔欷。嘗謂將
 士曰:「吾是籓邸舊臣,累蒙獎擢,至於大義,不得不死。今穀支數年,食盡此谷,足知天下之事。必若隋室傾敗,天命有歸,吾當斷頭以付諸君也。」時百姓苦隋日久,及逢義舉,人有息肩之望。然君素善於統領,下不能叛。歲餘,頗得外生口,城中微知江都傾覆。又糧食乏絕,人不聊生,男女相食,眾心離駭。白虹降於府門,兵器之端,夜皆光見。月餘,君素為左右所害。



 河東陳孝意,少有志尚,弱冠以貞介知名。大業初,為魯郡司法書佐,郡內號為廉平。太守蘇威嘗欲殺一囚,孝意固諫,至於再三,威不許。孝意因解衣,請先受死。良久,
 威意乃解,謝而遣之,漸加禮敬。及威為納言,奏孝意為侍御史。後以父憂去職,居喪過禮,有白鹿馴擾其廬,時人以為孝感之應。未期,起授雁門郡丞。在郡菜食齋居,朝夕哀臨,每一發聲,未嘗不絕倒,柴毀骨立,見者哀之。於時政刑日紊,長吏多賊污,孝意清節彌厲,發奸擿伏,動若有神,吏民稱之。煬帝幸江都,馬邑劉武周殺太守王仁恭,舉兵作亂。孝意率兵與武賁郎將王智辯討之,戰於下館城,反為所敗。武周遂轉攻傍郡,百姓兇兇,將懷叛逆。前郡丞楊長仁、雁門令王確等,並桀黠,為無賴所歸,謀應武周。孝意陰知之,族滅其家,郡中戰慄,莫敢
 異志。俄而武周引兵來攻,孝意拒之,每致克捷。但孤城獨守,外無聲援,孝意執志,誓以必死。每遣使江都,道路隔絕,竟無報命。孝意亦知帝必不反,每旦暮向詔敕庫俯伏流涕,悲動左右。圍城百餘日,糧盡,為校尉張倫所殺,以城歸武周。



 京兆張季珣,父祥,少為高祖所知,其後引為丞相參軍事。開皇中,累遷並州司馬。仁壽末,漢王諒舉兵反,遣其將劉建略地燕、趙。至井陘,祥勒兵拒守,建攻之,復縱火燒其郭下。祥見百姓驚駭,其城側有西王母廟,祥登城望之再拜,號泣而言曰:「百姓何罪,致此焚燒!神其有靈,
 可降雨相救。」言訖,廟上雲起,須臾驟雨,其火遂滅。士卒感其至誠,莫不用命。城圍月餘,李雄援軍至,賊遂退走。以功授開府,歷汝州刺史、靈武太守,入為都水監,卒官。季珣少慷慨有志節。



 大業末,為鷹擊郎將,其府據箕山為固,與洛口連接。及李密、翟讓攻陷倉城,遣人呼之。季珣罵密極口,密怒,遣兵攻之,連年不能克。時密眾數十萬在其城下,季珣四面阻絕,所領不過數百人,而執志彌固,誓以必死。經三年,資用盡,樵蘇無所得,撤屋而爨,人皆穴處,季珣撫巡之,一無離叛。糧盡,士卒羸病不能拒戰,遂為所陷。季珣坐聽事,顏色自若,密遣兵擒送之。
 群賊曳季珣令拜密,季珣曰:「吾雖為敗軍之將,猶是天子爪牙之臣,何容拜賊也!」密壯而釋之。翟讓從之求金不得,遂殺之,時年二十八。



 其弟仲琰,大業末為上洛令。及義兵起,率吏人城守,部下殺之以歸義。仲琰弟琮,為千牛左右,宇文化及之亂遇害。季珣家素忠烈,兄弟俱死國難,論者賢之。



 北海松贇,性剛烈,重名義,為石門府隊正。大業末,有賊楊厚擁徒作亂,來攻北海縣,贇從郡兵討之。贇輕騎覘賊,為厚所獲,厚令贇謂城中,雲郡兵已破,宜早歸降。贇偽許之。既至城下,大呼曰:「我是松贇,為官軍覘賊,邂逅
 被執,非力屈也。今官軍大來,並已至矣,賊徒寡弱,旦暮擒剪,不足為憂。」賊以刀築贇口,引之而去,毆擊交下。贇罵厚曰:「老賊何敢致辱賢良,禍自及也!」言未卒,賊已斬斷其腰。城中望之,莫不流涕扼腕,銳氣益倍。北海卒完。煬帝遣戶曹郎郭子賤討厚,破之,以贇亡身殉節,嗟悼不已,上表奏之。優詔褒揚,贈朝散大夫、本郡通守。



 史臣曰:古人以天下至大,方身則小,生為重矣,比義則輕。然則死有重於太山,生以理全者也,生有輕於鴻毛,死與義合者也。然死不可追,生無再得,故處不失節,所以為難矣。楊諒、玄感、李密反形已成,兇威方熾,皇甫誕、
 游元、馮慈明臨危不顧,視死如歸,可謂勇於蹈義矣。獨孤盛、元文都、盧楚、堯君素豈不知天之所廢,人不能興,甘就菹醢之誅,以徇忠貞之節。雖功未存於社稷,力無救於顛危,然視彼茍免之徒,貫三光而洞九泉矣。須陀、善會有溫序之風,子翊、松贇蹈解揚之烈。國家昏亂
 有
 忠臣,誠哉斯言也。



\end{pinyinscope}