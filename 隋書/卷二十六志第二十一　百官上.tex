\article{卷二十六志第二十一 百官上}

\begin{pinyinscope}

 《易》曰:「天尊地卑,乾坤定矣,卑高既陳,貴賤位矣。」是以聖人法乾坤以作則,因卑高以垂教,設官分職,錫珪胙土。由近以制遠,自中以統外。內則公卿大夫士,外則公侯伯子男。咸所以協和萬邦,平章百姓,允厘庶績,式敘彞倫。



 其由來尚矣。然古今異制,文質殊途。或以龍表官,或以雲紀職。放勛即分命四子,重華乃爰置九官,夏倍於
 虞,殷倍於夏,周監二代,沿革不同。其道既文,置官彌廣。逮於戰國,戎馬交馳,雖時有變革,然猶承周制。秦始皇廢先王之典,焚百家之言,創立朝儀,事不師古,始罷封侯之制,立郡縣之官。太尉主五兵,丞相總百揆,又置御史大夫,以貳於相。自餘眾職,各有司存。漢高祖除暴寧亂,輕刑約法,而職官之制,因於嬴氏,其間同異,抑亦可知。光武中興,聿遵前緒,唯廢丞相與御史大夫,而以三司綜理眾務。洎於叔世,事歸臺閣,論道之官,備員而已。魏、晉繼及,大抵略同,爰及宋、齊,亦無改作。梁武受終,多循齊舊,然而定諸卿之位,各配四時,置戎秩之官,百有
 餘號。陳氏繼梁,不失舊物。高齊創業,亦遵後魏,臺省位號,與江左稍殊,所有節文,備詳於志。有周創據關右,日不暇給,洎乎克清江、漢,爰議憲章。酌禜鎬之遺文,置六官以綜務,詳其典制,有可稱焉。



 高祖踐極,百度伊始,復廢周官,還依漢、魏。唯以中書為內史,侍中為納言,自餘庶僚,頗有損益。煬帝嗣位,意存稽古,建官分職,率由舊章。大業三年,始行新令。於時三川定鼎,萬國朝宗,衣冠文物,足為壯觀。即而以人從欲,待下若仇,號令日改,官名月易。尋而南征不復,朝廷播遷,圖籍注記,多從散逸。今之存錄者,不能詳備焉。



 梁武受命之初,官班多同宋、齊之舊,有丞相、太宰、太傅、太保、大將軍、大司馬、太尉、司徒、司空、開府儀同三司等官。諸公及位從公開府者,置官屬。



 有長史、司馬、諮議參軍,掾屬從事中郎、記室、主簿、列曹參軍、行參軍、舍人等官。其司徒則有左、右二長史,又增置左西掾一人,自餘僚佐,同於二府。有公則置,無則省。而司徒無公,唯省舍人,餘官常置。開府儀同三司,位次三公,諸將軍、左右光祿大夫,優者則加之,同三公,置官屬。



 特進,舊位從公。武帝以鄧禹列侯就第,特進奉朝請,是特引見之稱,無官定體。於是革之。



 尚書省,置令,左、右僕射各一人。又置吏部、祠部、度支、左戶、都官、五兵等六尚書。左右丞各一人。吏部、刪定、三公、比部、祠部、儀曹、虞曹、主客、度支、殿中、金部、倉部、左戶、駕部、起部、屯田、都官、水部、庫部、功論、中兵、外兵、騎兵等郎二十三人。令史百二十人,書令史百三十人。



 尚書掌出納王命,敷奏萬機。令總統之。僕射副令,又與尚書分領諸曹。令闕,則左僕射為主。其祠部尚書多不置,以右僕射主之。若左、右僕射並闕,則置尚書僕射以掌左事,置祠部尚書以掌右事。然則尚書僕射、祠部尚書不恆置矣。又有起部尚書,營宗廟宮室則權置之。事
 畢則省,以其事分屬都官、左戶二尚書。左、右丞各一人,佐令、僕射知省事。左掌臺內分職儀、禁令、報人章,督錄近道文書章表奏事,糾諸不法。右掌臺內藏及廬舍、凡諸器用之物,督錄遠道文書章表奏事。



 凡諸尚書文書,詣中書省者,密事皆以挈囊盛之,封以左丞印。自晉以後,八座及郎中多不奏事。天監元年詔曰:「自禮闈陵替,歷茲永久,郎署備員,無取職事。



 糠粃文案,貴尚虛閑,空有趨墀之名,了無握蘭之實。曹郎可依昔奏事。」自是始奏事矣。三年,置侍郎,視通直郎。其郎中在職勤能,滿二歲者,轉之。又有五都令史,與左、右丞共知所司。舊用人
 常輕,九年詔曰:「尚書五都,職參政要,非但總領眾局,亦乃方軌二丞。頃雖求才,未臻妙簡,可革用士流,每盡時彥,庶同持領,秉此群目。」於是以都令史視奉朝請。其年,以太學博士劉納兼殿中都,司空法曹參軍劉顯兼吏部都,太學博士孔虔孫兼金部都,司空法曹參軍蕭軌兼左戶都,宣毅墨曹參軍王顒兼中兵都。五人並以才地兼美,首膺茲選矣。駕部又別領車府署,庫部領南、北武庫二署令丞。



 門下省置侍中、給事黃門侍郎各四人,掌侍從左右,擯相威儀,盡規獻納,糾正違闕。監令嘗御藥,封璽書。侍
 中高功者,在職一年,詔加侍中祭酒,與侍郎高功者一人,對掌禁令,公車、太官、太醫等令,驊騮廄丞。



 集書省置散騎常侍、通直散騎常侍各四人。員外散騎常侍無員。散騎侍郎、通直郎各四人。又有員外散騎侍郎、給事中、奉朝請、常侍侍郎,掌侍從左右,獻納得失,省諸奏聞文書。意異者,隨事為駁。集錄比詔比璽,為諸優文策文,平處諸文章詩頌。常侍高功者一人為祭酒,與侍郎高功者一人,對掌禁令,糾諸逋違。



 駙馬、奉車、車騎三都尉,並無員。駙馬以加尚公主者,無班秩。



 散騎常侍、通直散騎常侍、員外散騎常侍,舊並為顯職,與侍中通官。宋代以來,或輕或雜,其官漸替。天監六年革選,詔曰:「在昔晉初,仰惟盛化,常侍、侍中,並奏帷幄,員外常侍,特為清顯。陸始名公之胤,位居納言,曲蒙優禮,方有斯授。可分門下二局,委散騎常侍尚書案奏,分曹入集書。通直常侍,本為顯爵,員外之選,宜參舊準人數,依正員格。」自是散騎視侍中,通直視中丞,員外視黃門郎。



 中書省置監、令各一人,掌出內帝命。侍郎四人,功高者一人,主省內事。又有通事舍人、主事令史等員,及置令
 史,以承其事。通事舍人,舊入直閤內。梁用人殊重,簡以才能,不限資地,多以他官兼領。其後除通事,直曰中書舍人。



 秘書省置監、丞各一人,郎四人,掌國之典籍圖書。著作郎一人,佐郎八人,掌國史,集注起居。著作郎謂之大著作,梁初周舍、裴子野,皆以他官領之。又有撰史學士,亦知史書。佐郎為起家之選。



 御史臺,梁國初建,置大夫,天監元年,復曰中丞。置一人,掌督司百僚。皇太子已下,其在宮門行馬內違法者,皆糾彈之。雖在行馬外,而監司不糾,亦得奏之。專道而行,
 逢尚書丞郎,亦得停駐。其尚書令、僕、御史中丞,各給威儀十人。



 其八人武冠絳韝,執青儀囊在前。囊題云「宜官吉」以受辭訴。一人緗衣,執鞭杖,依列行,七人唱呼入殿,引喤至階。一人執儀囊,不喤。屬官治書侍御史二人,掌舉劾官品第六已下,分統侍御史。侍御史九人,居曹,掌知其事,糾察不法。殿中御史四人,掌殿中禁衛內。又有符節令史員。



 謁者臺,僕射一人,掌朝覲賓饗之事。屬官謁者十人,掌奉詔出使拜假,朝會擯贊。高功者一人為假史,掌差次謁者。



 諸卿,梁初猶依宋、齊,皆無卿名。天監七年,以太常為太常卿,加置宗正卿,以大司農為司農卿,三卿是為春卿。加置太府卿,以少府為少府卿,加置太僕卿,三卿是為夏卿。以衛尉為衛尉卿,廷尉為廷尉卿,將作大匠為大匠卿。三卿是為秋卿。以光祿勛為光祿卿,大鴻臚為鴻臚卿,都水使者為太舟卿,三卿是為冬卿。凡十二卿,皆置丞及功曹、主簿。而太常視金紫光祿大夫,統明堂、二廟、太史、太祝、廩犧、太樂、鼓吹、乘黃、北館、典客館等令丞,及陵監、國學等。又置協律校尉、總章校尉監、掌故、樂正之屬,以掌樂事。太樂又有清商署丞,太史別有靈臺丞。
 詔以為陵監之名,不出前誥,且宗廟憲章,既備典禮,園寢職司,理不容異,諸正陵先立監者改為令。於是陵置令矣。



 國學,有祭酒一人,博士二人,助教十人,太學博士八人。又有限外博士員。



 天監四年,置五經博士各一人。舊國子學生,限以貴賤,帝欲招來後進,五館生皆引寒門俊才,不限人數。大同七年,國子祭酒到溉等,又表立正言博士一人,位視國子博士。置助教二人。



 宗正卿,位視列曹尚書,主皇室外戚之籍。以宗室為之。



 司農卿,位視散騎常侍,主農功倉廩。統太倉、導官、籍田、
 上林令,又管樂游、北苑丞,左右中部三倉丞,莢庫、荻庫、箬庫丞,湖西諸屯主。天監九年,又置勸農謁者,視殿中御史。



 太府卿,位視宗正,掌金帛府帑。統左右藏令、上庫丞,掌太倉、南北市令。



 關津亦皆屬焉。



 少府卿,位視尚書左丞,置材官將軍、左中右尚方、甄官、平水署、南塘邸稅庫、東西冶、中黃、細作、炭庫、紙官、柴署等令丞。



 衛僕卿,位視黃門侍郎,統南馬牧、左右牧、龍廄、內外廄丞。又有弘訓太僕,亦置屬官。



 衛尉卿,位視侍中,掌宮門屯兵。卿每月、丞每旬行宮徼,糾察不法。統武庫令、公車司馬令。又有弘訓衛尉,亦置屬官。



 廷尉卿,梁國初建,曰大理,天監元年,復改為廷尉。有正、監、平三人。元會,廷尉三官與建康三官,皆法冠玄衣朝服,以監東、西、中華門。手執方木,長三尺,方一寸,謂之執方。四年,置胄子律博士,位視員外郎。



 大匠卿,位視太僕,掌土木之工。統左、右校諸署。



 光祿卿,位視太子中庶子,掌宮殿門戶。統守宮、黃門、華林園、暴室等令。



 又有左右光祿、金紫光祿、太中、中散等
 大夫,並無員,以養老疾。



 鴻臚卿,位視尚書左丞,掌導護贊拜。



 太舟卿,梁初為都水臺,使者一人,參軍事二人,河堤謁者八人。七年,改焉。



 位視中書郎,列卿之最末者也。主舟航堤渠。



 大長秋,主諸宦者,以司宮闈之職。統黃門、中署、奚官、暴室、華林等署。



 領軍,護軍,左、右衛、驍騎、游騎等六將軍,是為六軍,又有中領、中護,資輕於領、護。又左右前後四將軍,左右中郎將,屯騎、步騎、越騎、長水、射聲等五營校尉,武賁、冗從、羽林三將軍,積射、強弩二軍,殿中將軍、武騎
 之職,皆以分司丹禁,侍衛左右。天監六年,置左右驍騎、左右游擊將軍,位視二率。改舊驍騎曰雲騎,游擊曰游騎,降左右驍、游一階。又置硃衣直閣將軍,以經為方牧者為之。其以左右驍、游帶領者,量給儀從。



 太子太傅一人,位視尚書令。少傅一人,位視左僕射。天監初,又置東宮常侍,皆散騎常侍為之。



 詹事,位視中護軍,任總宮朝。二傅及詹事,各置丞、功曹、主簿。五官、家令、率更令僕各一人。家令,自宋、齊已來,清流者不為之。天監六年,帝以三卿陵替,乃詔革選。家令視通直常侍,率更、僕視黃門三等,皆置丞。中大通三年,
 以昭明太子妃居金華宮,又置金華家令。



 左、右衛率各一人,位視御史中丞。各有丞。左率領果毅、統遠、立忠、建寧、陵鋒、夷寇、祚德等七營,右率領崇榮、永吉、崇和、細射等四營。二率各置殿中將軍十人,員外將軍十人,正員司馬四人。又有員外司馬督官。共屯騎、步兵、翊軍三校尉各一人,謂之三校。旅賁中郎將、冗從僕射各一人,謂之二將。左、右積弩將軍各一人。門大夫一人,視謁者僕射。



 中庶子四人,功高者一人為祭酒。行則負璽,前後部護駕。



 中舍人四人,功高者一人,與中庶子祭酒共掌其坊之禁令。又有通事守舍人、典事守舍人、典法守舍人員。



 庶子四人,掌侍從左右,獻納得失。高功者一人,與高功舍人共掌其坊之禁令。



 舍人十六人,掌文記。通事舍人二人,視南臺御史,多以餘官兼職。典經局洗馬八人,位視通直郎。置典經守舍人、典事守舍人員。又有外監殿局,內監殿局,導客局,齋內局,主璽、主衣、扶侍等局,門局,錫庫局,內廄局,中藥藏局,食官局,外廄局,車廄局等,各置有司,以承其事。



 皇弟、皇子府,置師,長史,司馬,從事中郎,諮議參軍,及掾
 屬中錄事、中記室、中直兵等參軍,功曹史,錄事、記室、中兵等參軍,文學,主簿,正參軍、行參軍、長兼行參軍等員。嗣王府則減皇弟皇子府師、友、文學、長兼行參軍。蕃王府則又減嗣王從事中郎,諮議參軍,掾屬錄事、記室、中兵參軍等員。自此以下,則並不登二品。



 王國置郎中令、將軍、常侍官。又置典祠令、廟長、陵長、典醫丞、典府丞、典書令、學官令、食官長、中尉、侍郎、執事中尉、司馬、謁者、典衛令、舍人、中大夫、大農等官。嗣王國則唯置郎中令、中尉、常侍、大農等員。蕃王則無常侍。



 自此以下,並不登二品。



 諸王皆假金獸符第一至第五左,竹使符第一至第十左。諸公侯皆假銅獸符,竹使符第一至第五。名山大澤不以封。鹽鐵金銀銅錫,及竹園別都,宮室園圃,皆不以屬國。



 諸王言曰令,境內稱之曰殿下。公侯封郡縣者,言曰教,境內稱之曰第下。自稱皆曰寡人。相以下,公文上事,皆詣典書。世子主國,其文書表疏,儀式如臣而不稱臣。文書下群官,皆言告。諸王公侯國官,皆稱臣。上於天朝,皆稱陪臣。有所陳,皆曰上疏。其公文曰言事。五等諸公,位視三公,班次之。開皇諸侯,位視孤卿、重號
 將軍、光祿大夫,班次之。開國諸伯,位視九卿,班次之。開國諸子,位視二千石,班次之。開國諸男,位視比二千石,班次之。公已下,各置相、典祠、典書令、典衛長一人。而伯子典書謂之長,典衛謂之丞。男典祠謂之長,典書謂之丞,無典衛。諸公已下,臺為選置相,掌知百姓事。典祠已下,自選補上。諸列侯食邑千戶已上,置家丞、庶子員。不滿千戶,則但置庶子員。



 州刺史二千石,受拜之明日,辭宮廟而行。州置別駕、治中從事各一人,主簿,西曹、議曹從事,祭酒從事,部傳從事,文學從事,各因其州之大小而置員。
 郡置太守,置丞。國曰內史。郡丞,三萬戶以上,置佐一人。



 縣為國曰相,大縣為令,小縣為長,皆置丞、尉。郡縣置吏,亦各準州法,以大小而制員。郡縣吏有書僮,有武吏,有醫,有迎新、送故等員。亦各因其大小而置焉。



 建康舊置獄丞一人。天監元年,詔依廷尉之官,置正、平、監,革選士流,務使任職。又令三官更直一日,分受罪系,事無小大,悉與令籌。若有大事,共詳,三人具辨。脫有同異,各立議以聞。尚書水部郎袁孝然、議曹郎孔休源並為之。位視給事中。



 天監初,武帝命尚書刪定郎濟陽蔡法度,定令為九品。
 秩定,帝於品下注一品秩為萬石,第二第三為中二千石,第四第五為二千石。至七年,革選,徐勉為吏部尚書,定為十八班。以班多者為貴,同班者,則以居下者為劣。



 丞相、太宰、太傅、太保、大司馬、大將軍、太尉、司徒、司空,為十八班。



 諸將軍開府儀同三司、左右光祿開府儀同三司,為十七班。



 尚書令、太子太傅、左右光祿大夫,為十六班。



 尚書左僕射,太子少傅,尚書僕射、右僕射,中書監,特進,領、護軍將軍,為十五班。



 中領、護軍,吏部尚書,太子詹事,金紫光祿大夫,太常卿,為十四班。



 中書令,列曹尚書,國子祭酒,宗正、太府卿,光祿大夫,為十三班。



 侍中,散騎常侍,左、右衛將軍,司徒左長史,衛尉卿,為十二班。



 御史中丞,尚書吏部郎,秘書監,通直散騎常侍,太子左、右二衛率,左、右驍騎,左、右游擊,太中大夫,皇弟皇子師,司農,少府、廷尉卿,太子中庶子,光祿卿,為十一班。



 給事黃門侍郎,員外散騎常侍,皇弟皇子府長史,太僕、
 大匠卿,太子家令、率更令、僕,揚州別駕,中散大夫,司徒右長史,雲騎,游騎,皇弟皇子府司馬,硃衣直閣將軍,為十班。



 尚書左丞,鴻臚卿,中書侍郎,國子博士,太子庶子,揚州中從事,皇弟皇子公府從事中郎,太舟卿,大長秋,皇弟皇子府諮議,嗣王府長史,前左右後四軍、嗣王府司馬,庶姓公府長史,司馬,為九班。



 秘書丞,太子中舍人,司徒左西掾,司徒屬,皇弟皇子友,散騎侍郎,尚書右丞,南徐州別駕,皇弟皇子公府掾屬,皇弟皇子單為二衛司馬,嗣王庶姓公府從事中郎,左、
 右中郎將,嗣王庶姓公府諮議,皇弟皇子之庶子府長史、司馬,蕃王府長史、司馬,庶姓持節府長史、司馬,為八班。



 五校,東宮三校,皇弟皇子之庶子府中錄事、中記室、中直兵參軍,南徐州中從事,皇弟皇子之庶子府、蕃王府諮議,為七班。



 太子洗馬,通直散騎侍郎,司徒主簿,尚書侍郎,著作郎,皇弟皇子府功曹史,五經博士,皇弟皇子府錄事、記室、中兵參軍,皇弟皇子荊江雍郢南兗五州別駕,領、護軍長史、司馬,嗣王庶姓公府掾屬,南臺治書侍御史,廷尉
 三官,謁者僕射,太子門大夫,嗣王庶姓公府中錄事、中記室、中直兵參軍,庶姓府諮議,為六班。



 尚書郎中,皇弟皇子文學及府主簿,太子太傅、少傅丞,皇弟皇子湘豫司益廣青衡七州別駕,皇弟皇子荊江雍郢南兗五州中從事,嗣王庶姓荊江雍郢南兗五州別駕,太常丞,皇弟皇子國郎中令、三將,東宮二將,嗣王府功曹史,庶姓公府錄事、記室、中兵參軍,皇弟皇子之庶子府、蕃王府中錄事、中記室、中直兵參軍,為五班。



 給事中,皇弟皇子府正參軍,中書舍人,建康三官,皇弟皇子北徐北兗梁交南梁五州別駕,皇弟皇子湘豫司
 益廣青衡七州別駕、中從事,嗣王庶姓湘豫司益廣青衡七州別駕,嗣王庶姓荊江雍郢南兗五州中從事,宗正、太府、衛尉、司農、少府、廷尉、太子詹事等丞,積射、強弩將軍,太子左右積弩將軍,皇弟皇子國大農,嗣王國郎中令,嗣王庶姓公府主簿,皇弟皇子之庶子府蕃王府功曹史,皇弟皇子之庶子府蕃王府錄事、記室、中兵參軍,為四班。



 太子舍人,司徒祭酒,皇弟皇子公府祭酒,員外散騎侍郎,皇弟皇子府行參軍,太子太傅少傅五官功曹主簿,二衛司馬,公車令,胄子律博士,皇弟皇子越桂寧霍四
 州別駕,皇弟皇子北徐北兗梁交南梁五州中從事,嗣王庶姓北徐北兗梁交南梁五州別駕,湘豫司益廣青衡七州中從事,嗣王庶姓公府正參軍,皇弟皇子之庶子府蕃王府曹主簿,武衛將軍,光祿丞,皇弟皇子國中尉,太僕大匠丞,嗣王國大農,蕃王國郎中令,庶姓持節府中錄事、中記室、中直兵參軍,北館令,為三班。



 秘書郎,著作佐郎,揚、南徐州主簿,嗣王庶姓公府祭酒,皇弟皇子單為領護詹事二衛等五官、功曹、主簿,太學博士,皇弟皇子國常侍,奉朝請,國子助教,皇弟皇子越桂寧霍四州中從事,皇弟皇子荊江雍郢南兗五州主
 簿,嗣王庶姓越桂寧霍四州別駕,嗣王庶姓北徐北兗梁交南梁五州中從事,鴻臚丞,尚書五都令史,武騎常侍,材官將軍,明堂二廟帝陵令,嗣王府庶姓公府行參軍,皇弟皇子之庶子府正參軍,蕃王國大農,庶姓持節府錄事、記室、中兵參軍,庶姓持節府功曹史,為二班。



 揚南徐州西曹祭酒從事,皇弟皇子國侍郎,嗣王國常侍,揚南徐州議曹從事,東宮通事舍人,南臺侍御史,太舟丞,二衛殿中將軍,太子二率殿中將軍,皇弟皇子之庶子府蕃王府行參軍,蕃王國中尉,皇弟皇子湘豫司益廣青衡七州主簿,皇弟皇子荊雍郢南兗四州西曹
 祭酒議曹從事,皇弟皇子江州西曹從事、祭酒議曹祭酒部傳從事,嗣王庶姓越桂寧霍四州中從事,嗣王庶姓荊江雍郢南兗五州主簿,庶姓持節府主簿,汝陰巴陵二國郎中令,太官、太樂、太市、太史、太醫、太祝、東西冶、左右尚方、南北武庫、車府等令,為一班。



 位不登二品者,又為七班。皇弟皇子府長兼參軍,皇弟皇子國三軍、嗣王國侍郎、蕃王國常侍、揚南徐州文學從事,殿中御史、庶姓持節府除正參軍、太子家令丞、二衛殿中員外將軍、太子二率殿中員外將軍、鎮蠻安遠護軍度支校尉等司馬,皇弟皇子北徐北兗梁交南梁
 五州主簿、皇弟皇子湘豫司益廣青衡七州西曹祭酒議曹從事,皇弟皇子荊雍郢三州從事史,江州議曹從事,南兗州文學從事,嗣王庶姓湘豫司益廣青衡七州主簿、嗣王庶姓荊雍郢南兗四州西曹祭酒議曹從事,嗣王庶姓江州西曹從事、祭酒部傳從事、勸農謁者,汝陰巴陵二王國大農,郡公國郎中令,為七班。



 皇弟皇子國典書令,嗣王國三軍,蕃王國侍郎,領護詹事五官功曹,皇弟皇子府參軍督護,嗣王府長兼參軍,庶姓公府長兼參軍,庶姓持節府板正參軍,皇弟皇子越桂寧霍四州主簿,皇弟皇子北徐北兗梁交南梁五
 州西曹祭酒議曹從事,嗣王庶姓北徐北兗梁交南梁五州主簿,嗣王庶姓湘豫司益廣青衡七州西曹祭酒議曹從事,皇弟皇子豫司益廣青五州文學從事,湘衡二州從事,嗣王庶姓荊霍郢三州從事史,江州議曹從事,南兗州文學從事,汝陰巴陵二王國中尉,皇弟皇子之庶子縣侯國郎中令,郡公國大農,縣公國郎中令,為六班。



 皇弟皇子國三令,嗣王國典書令,蕃王國三軍,皇弟皇子公府東曹督護,嗣王府庶姓公府參軍督護,皇弟皇子之庶子長兼參軍,蕃王府長兼參軍,二衛正員司馬
 督,太子二率正員司馬督,領護主簿,詹事主簿,二衛功曹,太常五官功曹,石頭戍軍功曹,庶姓持節府行參軍,皇弟皇子越桂寧霍四州西曹祭酒議曹從事,皇弟皇子北徐北兗梁交南梁五州文學從事,嗣王庶姓越桂寧霍四州主簿,嗣王庶姓北徐北兗梁交南梁五州西曹祭酒議曹從事,嗣王庶姓豫司益廣青五州文學從事,湘衡二州從事,汝陰巴陵二王國常侍,郡公國中尉,縣侯國郎中令,皇弟皇子府功曹督護,為五班。



 嗣王國三令,蕃王國典書令,嗣王府功曹督護,庶姓公府東曹督護,皇弟皇子之庶子府參軍督護,蕃王府參
 軍督護,二衛員外司馬督,太子二率員外司馬督,二衛主簿,太常主簿,宗正等十一卿五官功曹,石頭戍軍主簿,庶姓持節府板行參軍,皇弟皇子越桂寧霍四州文學從事,嗣王庶姓越桂寧霍四州西曹祭酒議曹從事,嗣王庶姓北徐北兗梁交南梁五州文學從事,汝陰巴陵二王國侍郎,縣公國中尉,為四班。



 蕃王國三令,皇弟皇子之庶子府蕃王府功曹督護,宗正等十一卿主簿,庶姓持節府長兼參軍,嗣王庶姓越桂寧霍四州文學從事,郡公國侍郎,為三班。



 庶姓持節府參軍督護,汝陰巴陵二王國典書令,縣公
 國侍郎,為二班。



 庶姓持節府功曹督護,汝陰巴陵二王國三令,郡公國典書令,為一班。



 又著作正令史,集書正令史,尚書度支三公正令史,函典書、殿中外監、齋監、東堂監、尚書都官左降正令史,諸州鎮監、石頭城監、瑯邪城監、東宮外監、殿中守舍人,齋臨、東宮典經守舍人,上庫令,太社令,細作令,導官令,平水令,太官市署丞,正廚丞,酒庫丞,柴署丞,太樂庫丞,別局校丞,清商丞,太史丞,太醫二丞,中藥藏丞,東冶小庫等三丞,作堂金銀局丞,木局丞,北武庫二丞,南武庫二丞,
 東宮食官丞,上林丞,湖西磚屯丞,祇箬庫丞,紋絹簟席丞,國子典學,材官司馬,宣陽等諸門候,東宮導客守舍人,運署謁者,都水左右二裝五城謁者,石城宣城陽新屯謁者,南康建安晉安伐船謁者,晉安練葛屯主,為三品蘊位。



 又門下集書主事通正令史,中書正令史,尚書正令史,尚書監籍正令史,都正令史,殿中內監,題閤監,婚局監,東宮門下通事守舍人,東宮典書守舍人,東宮內監,殿中守舍人,題閤監,乘黃令,右藏令,籍田令,廩犧令,梅根諸冶令,典客館令,太官四丞,庫存丞,太樂丞,東冶太庫丞,左尚方五丞,右尚方四丞,東宮衛庫丞,司農左
 右中部倉丞,廷尉律博士,公府舍人,諸州別署監,山陰獄丞,為三品勛位。



 其州二十三,並列其高下,選擬略視內職。郡守及丞,各為十班。縣制七班。



 用人各擬內職云。



 又詔以將軍之名,高卑舛雜,命更加厘定。於是有司奏置一百二十五號將軍。



 以鎮、衛、驃騎、車騎,為二十四班。內外通用。四征、東南西北,止施外。四中,軍、衛、撫、護,止施內。為二十三班。八鎮東南西北,止施在外。左右前後,止施在內。為二十二班。八安東西南北,止施在外。左右前後,止施在內。為二十一班。四平、東南西北。四翊,左右前後。為二十班。凡三十五號,為一品。是為重號將軍。忠武、軍師,為十九班。武臣、爪牙、龍騎、雲麾,為十八班。代舊前後左右四將軍。鎮兵、翊師、宣
 惠、宜毅,為十七班。代舊四中郎。十號為一品。智威、仁威、勇威、信威、嚴威,為十六班。代舊征虜。智武、仁武、勇武、信武、嚴武,為十五班。代舊冠軍。十號為一品,所謂五德將軍者也。輕車、征遠、鎮朔、武旅、貞毅,為十四班。代舊輔國。凡將軍加大者,唯至貞毅而已。通進一階。優者方得比加位從公。凡督府,置長史司馬諮議諸曹,有錄事記室等十八曹。天監七年,更置中錄事、中記室、中直兵參軍各一人。寧遠、明威、振遠、電耀、威耀,為十三班。代舊寧朔。十號為一品。武威、武騎、武猛、武壯、飆武,為十二班。



 電威、馳銳、追鋒、羽騎、突騎,為十一班。十號為一品。折沖、冠武、和戎、安壘、猛烈,為十班。掃狄、雄信、掃虜、武銳、摧鋒,為九班。十號為一品。略遠、貞威、決勝、開遠、光野,為八班。厲鋒、輕
 銳、討狄、蕩虜、蕩夷,為七班。十號為一品。武毅、鐵騎、樓船、宣猛、樹功,為六班。克狄、平虜、討夷、平狄、威戎,為五班。十號為一品。伏波、雄戟、長劍、沖冠、雕騎,為四班、佽飛、安夷、克戎、綏狄、威虜,為三班。十號為一品。前鋒、武毅、開邊、招遠、金威,為二班。綏虜、蕩寇、殄虜、橫野、馳射,為一班。十號為一品。凡十品,二十四班。亦以班多為貴。其制品十,取其盈數。班二十四,以法氣序。制簿悉以大號居後,以為選法自小遷大也。前史所記,以位得從公,故將軍之名,次於臺槐之下。



 至是備其班品,敘於百司之外。其不登二品,應須軍號者,有牙門、代舊建威。期門,代舊建武。為八班。候騎、
 代舊振威。熊渠,代舊振武。為七班。中堅、代舊奮威。典戎,代舊奮武。為六班。戈船、代舊揚威。繡衣,代舊揚武。為五班。執訊、代舊廣威。行陣,代舊廣武。為四班。鷹揚為三班。陵江為二班。偏將軍、裨將軍,為一班。凡十四號,別為八班,以象八風。所施甚輕。又有武安、鎮遠、雄義,擬車騎。為二十四班。四撫東南西北,擬四征。為二十三班。四寧東南西北,擬四鎮。為二十二班。四威東南西北,擬四安。為二十一班。四綏東南西北,擬四平。為二十班。凡十九號,為一品。安遠、安邊,擬忠武、軍師。為十九班。輔義、安沙、衛海、撫河,擬武臣等四號。為十八班。平遠、撫朔、寧沙、航海、擬鎮兵等四號。為十七班。凡十號,為一品。翊海、朔野、拓遠、威河、龍幕,擬智威等五號。為十六班。威隴、安漠、
 綏邊、寧寇、梯山,擬智武等五號。為十五班。凡十號,為一品。寧境、綏河、明信、明義、威漠,擬輕車等五號。為十四班。安隴、向義、宣節、振朔、候律,擬寧遠等五號。為十三班。凡十號,為一品。平寇、定遠、陵海、寧隴、振漠,擬武威等五號。為十二班。馳義、橫朔、明節、執信、懷德,擬電威等五號。為十一班。凡十號,為一品。撫邊、定隴、綏關、立信、奉義,擬折沖等五號。為十班。綏隴、寧邊、定朔、立節、懷威,擬掃狄等五號。為九班。



 凡十號,為一品。懷關、靜朔、掃寇、寧河、安朔,擬略遠等五號。為八班。揚化、超隴、執義、來化、度嶂,擬厲鋒等五號。為七班。凡十號,為一品。平河、振隴、雄邊、橫沙、寧關,擬武毅等五號。為六班。懷信、宣義、弘節、浮遼、鑿空,擬克狄等五號。為五班。凡十號,
 為一品。捍海、款塞、歸義、陵河、明信,擬伏波等五號。為四班。奉忠、守義、弘信、仰化、立義,擬人次飛等五號。為三班。凡十號,為一品。綏方、奉正、承化、浮海、度河,擬先鋒等五號。為二班。懷義、奉信、歸誠、懷澤、伏義,擬綏虜等五號。為一班。凡十號,為一品。大凡一百九號將軍,亦為十品,二十四班。正施於外國。



 及大通三年,有司奏曰:「天監七年,改定將軍之名,有因有革。普通六年,又置百號將軍,更加刊正,雜號之中,微有移異。大通三年,奏移寧遠班中明威將軍進輕車班中,以輕車班中徵遠度入寧遠班中。又置安遠將軍代貞武,宣遠代明烈。



 其戎夷之號,亦加附擬,選序則依此承用。」遂以定
 制。轉則進一班。黜則退一班。



 班即階也。同班以優劣為前後。有鎮、衛、驃騎、車騎同班。四中、四征同班。八鎮同班。八安同班。四平、四翊同班。忠武、軍師同班。武臣、爪牙、龍騎、雲麾、冠軍同班。鎮兵、翊師、宣惠、宣毅四將軍,東南西北四中郎將同班。智威、仁威、勇威、信威、嚴威同班。智武、仁武、勇武、信武、嚴武同班。謂為五德將軍。輕車、鎮朔、武旅、貞毅、明威同班。寧遠、安遠、征遠、振遠、宣遠同班。威雄、威猛、威烈、威振、威信、威勝、威略、威風、威力、威光同班。武猛、武略、武勝、武力、武毅、武健、武烈、武威、武銳、武勇同班。猛毅、猛烈、猛威、猛銳、猛震、猛進、猛智、猛武、猛勝、猛駿同
 班。壯武、壯勇、壯烈、壯猛、壯銳、壯盛、壯毅、壯志、壯意、壯力同班。驍雄、驍桀、驍猛、驍烈、驍武、驍勇、驍銳、驍名、驍勝、驍迅同班。雄猛、雄威、雄明、雄烈、雄信、雄武、雄勇、雄毅、雄壯、雄健同班。忠勇、忠烈、忠猛、忠銳、忠壯、忠毅、忠捍、忠信、忠義、忠勝同班。明智、明略、明遠、明勇、明烈、明威、明勝、明進、明銳、明毅、同班。



 光烈、光明、光英、光遠、光勝、光銳、光命、光勇、光戎、光野同班。飆勇、飆猛、飆烈、飆銳、飆奇、飆決、飆起、飆略、飆勝、飆出同班。龍驤、武視、雲旗、風烈、電威、雷音、馳銳、追銳、羽騎、突騎同班。折沖、冠武、和戎、安壘、超猛、英果、掃虜、掃狄、武銳、摧鋒同班。開遠、略遠、貞威、決勝、清野、堅
 銳、輕銳、拔山、雲勇、振旅同班。超武、鐵騎、樓船、宣猛、樹功、克狄、平虜、棱威、昭威、威戎同班。伏波、雄戟、長劍、沖冠、雕騎、佽飛、勇騎、破敵、克敵、威虜同班。前鋒、武毅、開邊、招遠、金威、破陣、蕩寇、殄虜、橫野、馳射同班。



 牙門、期門同班。候騎、熊渠同班。中堅、典戎同班。執訊、行陣同班。伏武、懷奇同班。偏、裨將軍同班。凡二百四十號,為四十四班。



 又雍州置寧蠻校尉,廣州置平越中郎將,北涼、南秦置西戎校尉,南秦、梁州置平戎校尉,寧州置鎮蠻校尉,西陽、南新蔡、晉熙、廬江等郡,置鎮蠻護軍,武陵郡置安遠護軍,巴陵郡置度支校尉。皆立府,隨府主號輕重而不為定。其
 將軍施於外國者,雄義、鎮遠、武安同班,擬鎮、衛等三號。四撫同班,擬四征,四威同班。擬四安。四綏同班,擬四平。安遠、安邊同班,擬忠武等號。撫河、衛海、安沙、輔義同班,擬武臣等號。航海、寧沙、撫朔、平遠同班,擬鎮兵等號。龍幕、威河、和戎、拓遠、朔野、翊海同班,擬智威等號。梯山、寧寇、綏邊、安漠、威隴五號同班,擬智武等號。威漠、明義、昭信、綏河、寧境同班,擬輕車等號。候律、振朔、宣節、向義、安隴同班,擬寧遠等號。振漠、寧隴、陵海、安遠、平寇同班,擬威雄等號。懷德、執信、明節、橫朔、馳義同班,擬武猛等號。安朔、寧河、掃寇、靜朔、懷關同班,擬驍雄等號。度嶂、奉
 化、康義、超隴、揚化同班,擬猛烈等號。寧關、橫沙、雄邊、振隴、平河同班,擬忠勇等號。鑿空、浮遼、弘節、宣義、懷信同班,擬明智等號。明信、陵河、歸義、款塞、捍海同班,擬光烈等號。立義、仰化、弘信、守義、奉忠同班,擬飆勇等號。奉誠、立誠、建誠、顯誠、義誠同班,擬龍驤等號。尉遼、寧渤、綏嶺、威塞、通侯同班,擬折沖等號。



 掃荒、威荒、定荒、開荒、理荒同班,擬開遠等號。奉節、掃節、建節、效節、伏節同班,擬超武等號。渡河、陵海、承化、奉正、綏方同班,擬伏波等號。伏義、懷澤、歸誠、奉信、懷義同班,擬前鋒等號。凡一百二十五將軍,二十八班,並施外國戎號,準於中夏焉。大同四
 年,魏彭城王爾硃仲遠來降,以為定洛大將軍,仍使其北討,故名云。



 陳承梁,皆循其制官,而又置相國,位列丞相上。並丞相、太宰、太傅、太保、大司馬、大將軍,並以為贈官。定令,尚書置五員,郎二十一員。其餘並遵梁制,為十八班,而官有清濁。自十二班以上並詔授,表啟不稱姓。從十一班至九班,禮數復為一等。又流外有七班,此是寒微士人為之。從此班者,方得進登第一班。其親王起家則為侍中。若加將軍,方得有佐史,無將軍則無府,止有國官。皇太子塚嫡者,起家封王,依諸王起家。餘子並封公,起家中
 書郎。諸王子並諸侯世子,起家給事。三公子起家員外散騎侍郎,令僕子起家秘書郎。若員滿,亦為板法曹,雖高半階,望終秘書郎下。次令僕子起家著作佐郎,亦為板行參軍。此外有揚州主簿、太學博士、王國侍郎、奉朝請、嗣王行參軍,並起家官,未合發詔。諸王公參佐等官,仍為清濁。或有選司補用,亦有府牒即授者,不拘年限,去留隨意。在府之日,唯賓游宴賞,時復修參,更無餘事。若隨府王在州,其僚佐等,或亦得預催督。若其驅使,便有職務。其衣冠子弟,多自修立,非氣類者,唯利是求,暴物亂政,皆此之類。國之政事,並由中書省。有中書舍人
 五人,領主事十人,書吏二百人。書吏不足,並取助書。分掌二十一局事,各當尚書諸曹,並為上司,總國內機要,而尚書唯聽受而已。被委此官,多擅威勢。其庶姓為州,若無將軍者,謂之單車。郡縣官之任代下,有迎新送故之法,餉饋皆百姓出,並以定令。其所制品秩,今列之云。



 相國,丞相,太宰,太傅,太保,大司馬,大將軍,太尉,司徒,司空,開府儀同三司,已上秩萬石。巴陵王、汝陰王後,尚書令,已上秩中二千石。品並第一。



 中書監,尚書左右僕射,特進,太子二傅,左右光祿大夫,已上中二千石。品並第二。



 中書令,侍中,散騎常侍,領、護軍,中領、護軍,吏部尚書,列曹尚書,金紫光祿大夫,光祿
 大夫,已上並中二千石。左右衛將軍,御史中丞,已上二千石。



 太后衛尉、太僕、少府三卿,太常、宗正、太府、衛尉、司農、少府、廷尉、光祿、大匠、太僕、鴻臚、太舟等卿,太子詹事,國子祭酒,已上中二千石。揚州刺史,凡單車刺史,加督進一品,都督進二品。不論持節假節,揚州、徐州加督,進二品右光祿已下。加都督,第一品尚書令下。南徐、東揚州刺史,皇弟皇子封國王世子,品並第三。



 通直散騎常侍,員外散騎常侍,黃門侍郎,已上二千石。秘書監,中二千石。



 左右驍騎、左右游擊等將軍,太子中庶子,已上二千石。太子左右衛率,二千石硃衣直閣,雲騎、游騎將軍,中書侍郎,已上千石。尚書左右丞,尚書、吏部侍郎、郎中,已上六百石。尚書郎中與吏部郎同列,今品同。太子三卿,太中、中散大
 夫,司徒左右長史,已上千石。諸王師,依秩減之例。國子博士,千石。荊江南兗郢湘雍等州刺史,六州加督,進在第三品東揚州下。加都督,進在第二品右光祿下。嗣王、蕃王、郡公、縣公等世子,品並第四。



 秘書丞,明堂、太廟、帝陵等令,已上六百石。散騎侍郎,前左右後軍將軍,左右中郎將,已上千石。大長秋,二千石。太子中舍人、庶子,六百石。豫益廣衡等州,青州領冀州,北兗北徐等州,梁州領南秦州,司南梁交越桂霍寧等十五州,加督,進在第四品雍州下。加都督,進在第三品南徐州下。不言秩。丹陽尹,中二千石。會稽太守,二千石。加督,進在第四品雍州下。加都督進在第三品南徐州下。



 諸郡若督及都督,皆以此差次為例。吳郡、吳興二太守,二千石。侯世子,不言秩。



 皇弟皇子府諮議參軍,八百石。皇弟皇
 子府板諮議參軍,不言秩。皇弟皇子府長史,千石。皇弟皇子府板長史,不言秩。皇弟皇子府司馬,千石。皇弟皇子府板司馬,不言秩。皇弟皇子公府從事中郎,六百石。品並第五。



 通直散騎侍郎,千石。著作郎,六百石。步兵、射聲、長水、越騎、屯騎五校尉,並千石。太子洗馬,六百石。太子步兵、翊軍、屯騎三校尉,並秩同臺校。司徒左西掾屬,並本秩四百石。依減秩例。皇弟皇子友,依減秩例。皇弟皇子公府屬,本秩四百石。依減秩例。五經博士,六百石。子男世子,不言秩。萬戶以上郡太守、內史、相,嗣王府、皇弟皇子之庶子府諮議參軍,六百石。板者不言秩。嗣王府、皇弟皇子之庶子府長史、司馬,並八百石。嗣王府官減正王府一階。其板長史、司馬,並不言秩。庶姓公府
 諮議參軍,六百石。與嗣王府同。其板者並不言秩。庶姓公府長史、司馬,並八百石。其板者並不言秩。嗣王庶姓公府從事中郎,六百石。



 皇帝皇子府中錄事參軍、板府中錄事參軍,中記室參軍、板中記室參軍,中直兵參軍、板中直兵參軍,揚州別駕中從事,皇弟皇子南徐荊江南兗郢湘雍州別駕中從事,並不言秩,品並第六。



 給事中,六百石,員外散騎侍郎,秘書著作佐郎,並四百石。依減秩例。奉車、駙馬都尉,武賁中郎將,羽林監,冗從僕射,已上並六百石。謁者僕射,千石。南臺治書侍御史,六百石。太子舍人,二百石。依減秩例。太子門大夫,六百石。太子旅賁中郎將、冗從僕射,並秩同臺將。司徒主簿,依減秩例。司徒祭酒,不言秩。



 領護軍長
 史、司馬,廷尉正、監、平,並六百石。皇弟皇子府錄事記室中兵等參軍、板錄事記室中兵等參軍、功曹史、主簿,公府祭酒,並不言秩。皇弟皇子文學,依減秩例。嗣王庶姓公府掾屬,並本秩四百石。依減秩例。太子二傅丞,並六百石。



 蕃王府諮議參軍,四百石。蕃王府板諮議參軍,不言秩。蕃王府長史、司馬,六百石。板者並不言秩。庶姓持節府諮議參軍,四百石。庶姓非公不持節將軍置長史,六百石。庶姓持節府板諮議參軍,不言秩。庶姓持節府長史、司馬,並六百石。板者皆不言秩。嗣王府、皇弟皇子之庶子、及庶姓公府中錄事中記室中直兵參軍、及板中錄事中記室中直兵參軍,並不言秩。不滿萬戶太守、內史、相,二千石。丹陽、會
 稽、吳郡、吳興及萬戶郡丞,並六百石。建康令,千石。建康正、監、平,秩同廷尉。品並第七。



 中書通事舍人,依減秩例。積射、強弩、武衛等將軍,公車令,太子左右積弩將軍,並六百石。奉朝請武騎常侍,依減秩例。太后三卿、十二卿、大長秋等丞,並六百石。左右衛司馬,不言秩。太子詹事丞,胄子律博士,並六百石。皇弟皇子府正參軍、板正參軍、行參軍、板行參軍,嗣王府、皇弟子皇子之庶子府錄事記室中兵參軍、板錄事記室中兵參軍、功曹史、主簿,庶姓非公不持節諸將軍置主簿,庶姓公府錄事記室中兵參軍、板錄事記室中兵參軍、主簿,嗣王庶姓公府祭酒,蕃王府中錄事記室直兵參軍、板中錄
 事記室直兵參軍、,庶姓持節府中錄事記室直兵參軍、及板中錄事記室直兵參軍,太子太傅、五官功曹史、主簿,少傅、五官功曹史、主簿,已上並不言秩。太學博士,六百石。國子助教,司樽郎,安蠻戎越校尉中郎將府等長史,六百石。蠻戎越等府佐無定品。自隨主軍號輕重。小府減大府一階。蠻戎越校尉中郎將等府板長史,不言秩。蠻戎越校尉中郎將等司馬,六百石。



 板者不言秩。庶姓南徐荊江南兗郢湘雍等州別駕中從事,不言秩。不滿萬戶已下郡丞,六百石。五千戶已上縣令、相,一千石。皇弟皇子國郎中令、大農、中尉,並六百石。品並第八。



 左右二衛殿中將軍,不言秩。南臺侍御史。依秩減例。東宮通事舍人,不言秩。



 材官將
 軍,六百石。太子左右二衛率、殿中將軍及丞、嗣王府、皇弟皇子之庶子府正參軍、板正參軍、行參軍、板行參軍,庶姓公府正參軍、板正參軍、蕃王府錄事記室中兵等參軍、板錄事記室中兵等參軍、功曹史、主簿,正參軍、板正參軍、行參軍、板行參軍,庶姓持節府錄事記室中兵等參軍、板錄事記室中兵等參軍、功曹史、主簿庶姓豫益廣衡青冀北兗北徐梁秦司南徐等州別駕中從事史,揚州主簿、西曹及祭酒、議曹二從事,南徐州主簿、西曹、祭酒議曹二從事,皇弟皇子諸州主簿、西曹,已上並不言秩。不滿五千戶已下縣令、相,六百石。皇弟皇子國常侍、侍郎,不
 言秩。嗣王國郎中令、大農、中尉,並四百石。嗣王國常侍,不言秩。蕃王國郎中令、大農、殿中,並二百石。品並第九。



 又有戎號擬官,自一品至於九品,凡二百三十七。鎮衛、驃騎、車騎等三號將軍,擬官品第一。比秩中二千石。四中、軍、撫、衛、權。四征、東南西北。八鎮東南西北,左右前後。等十六號將軍,擬官品第二。秩中二千石。八安、左前右後,東南西北。四翊、左前右後。四平東南西北。等十六號將軍,擬官品第三。秩中二千石。忠武、軍師、武臣、爪牙、龍騎、雲麾、冠軍、鎮兵、翊師、宣惠、宣毅等將軍,四中郎將,智、仁、勇、信、嚴等五威、五武將軍,合二十五號,擬官品第四。秩中二千石。輕車、鎮朔、武旅、貞毅、明威等將軍,將軍加大者至此。凡加大,通進一階。寧、安、征、振、宣等
 五遠將軍,寧蠻校尉,雍州小府、蠻越校尉中郎將,隨府主軍號輕重。若單作,則減刺史一階。若有將軍,減將軍一階。合十八號,擬官品第五。威雄、猛、烈、震、信、略、勝、風、力、光等十威,武猛、略、勝、力、毅、健、烈、威、銳、勇等十武,猛毅、烈、威、震、銳、進、智、勝、駿等十猛,壯武、勇、烈、猛、銳、威、力、毅、志、意等十壯,驍雄、桀、猛、烈、武、勇、銳、名、勝、迅等十驍,雄猛、威、明、烈、信、武、勇、毅、壯、健等十雄,忠勇、烈、猛、銳、壯、毅、捍、信、義、勝等十忠,明智、略、遠、勇、烈、威、銳、毅、勝、進等十明,光烈、明、英、遠、勝、銳、命、勇、戎、野等十光,飆勇、烈、猛、銳、奇、決、起、勝、略、出等十飆將軍,平越中郎,廣、梁、南秦、南梁、寧等州小府。西戎、平戎、鎮蠻三校尉等,擬官一百四號,品第六。並千石。龍驤、武
 視、雲旗、風烈、電威、雷音、馳銳、追銳、羽騎、突騎、折沖、冠武、和戎、安壘、超猛、英果、掃虜、掃狄、武銳、摧鋒、開遠、略遠、貞威、決勝、清野、堅銳、輕車、拔山、雲勇、振旅等將軍,擬官三十號,品第七。並六百石。超武、鐵騎、樓船、宣猛、樹功、克狄、平虜、棱威、戎昭、威戎、伏波、雄戟、長劍、沖冠、雕騎、佽飛、勇騎、破敵、克敵、威虜等將軍,鎮蠻護軍,西陽、南新蔡、晉熙、廬江郡小府、鎮蠻安遠護軍、度支校尉,隨府主號輕重。若單作,則減太守內史相一階。若有將軍,減一階。安遠護軍,度支校尉巴陵郡丞等,擬官二十三號,品第八。並六百石。前鋒、武毅、開邊、招遠、金威、破陣、蕩寇、殄虜、橫野、馳射等將軍,擬官十號,品第九。並四百石。諸將起自第六品已下,板則無秩。其雖
 除不領兵,領兵不滿百人,並除此官而為州郡縣者,皆依本條減秩石。二千石減為千石,千石降為六百石。自四百石降而無秩。其州郡縣,自各以本秩論。凡板將軍,皆降除一品。諸依此減降品秩。其應假給章印,各依舊差,不貶奪。



 其封爵亦為九等之差。郡王第一品。秩萬石。嗣王、蕃王、開國郡縣公,第二品。開國郡、縣侯,第三品。開國縣伯,第四品,並視中二千石。開國子,第五品。



 開國男,第六品。並視二千石。湯沐食侯,第七品。鄉、亭侯,第八品。並視千石。



 關中、關外侯,第九品。視六百石。



 陳依梁制,年未滿三十者,不得入仕。唯經學生策試得第,諸州光迎主簿,西曹左奏及經為挽郎得仕。其諸郡,
 唯正王任丹陽尹經迎得出身,庶姓尹則不得。必有奇才異行殊勛,別降恩旨敘用者,不在常例。其相知表啟通舉者,每常有之,亦無年常考校黜陟之法。既不為此式,所以勤惰無辨。凡選官無定期,隨闕即補,多更互遷官,未必即進班秩。其官唯論清濁,從濁官得微清,則勝於轉。若有遷授,或由別敕,但移轉一人為官,則諸官多須改動。其用官式,吏部先為白牒,錄數十人名,吏部尚書與參掌人共署奏。敕或可或不可。其不用者,更銓量奏請。若敕可,則付選,更色別,量貴賤,內外分之,隨才補用。以黃紙錄名,八座通署,奏可,即出付典名。而典以名
 貼鶴頭板,整威儀,送往得官之家。其有特發詔授官者,即宣付詔誥局,作詔章草奏聞。敕可,黃紙寫出門下。門下答詔,請付外施行。又畫可,付選司行召。得詔官者,不必皆須待召。但聞詔出,明日,即與其親入謝後,詣尚書,上省拜受。若拜王公則臨軒。



\end{pinyinscope}