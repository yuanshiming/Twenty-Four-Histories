\article{卷三十三志第二十八 經籍二 史}

\begin{pinyinscope}

 《史記》一百三十卷目錄一卷,漢中書令司馬遷撰。



 《史記》八十卷宋南中郎外兵參軍裴駰注。



 《史記音義》十二卷宋中散大夫徐野民撰。



 《史記音》三卷梁輕車錄事參軍鄒誕生撰。



 《古史考》二十五卷晉義陽亭侯譙周撰。



 《漢書》一百一十五卷漢護軍班固撰,太山太守應劭集解。



 《漢書集解音義》二十四卷應劭撰。



 《漢書音訓》一卷服虔撰。



 《漢書音義》七卷韋昭撰。



 《漢書音》二卷梁尋陽太守劉顯撰。



 《漢書音》二卷夏侯詠撰。



 《漢書音義》十二卷國子
 博士蕭該撰。



 《漢書音》十二卷廢太子勇命包愷等撰。



 《漢書集注》十三卷晉灼撰。



 《漢書注》一卷齊金紫光祿大夫陸澄撰。



 《漢書續訓》三卷梁平北諮議參軍韋棱撰。



 《漢書訓纂》三十卷陳吏部尚書姚察撰。



 《漢書集解》一卷姚察撰。



 《論前漢事》一卷蜀丞相諸葛亮撰。



 《漢書駁議》二卷晉安北將軍劉寶撰。



 《定漢書疑》二卷姚察撰。



 《漢書敘傳》五卷項岱撰。



 《漢疏》四卷梁有漢書孟康音九卷,劉孝標注《漢書》一百四十卷,陸澄注《漢書》一百二卷,梁元帝注《漢書》一百一十五卷,並亡。



 《東觀漢記》一百四十三卷起光武記注至靈帝,長水校尉劉珍等撰。



 《後漢書》一百三十卷無帝紀,吳武陵太守謝承撰。



 《後漢記》六十五卷本一百卷,梁有,今殘缺。晉散騎常侍薛瑩撰。



 《續漢書》八十三卷晉秘書監司馬彪撰。



 《後漢書》十七卷本九十七卷,今殘缺。晉少府卿華嶠撰。



 《後漢書》八十五卷本一百二十二卷,晉祠部郎謝
 沈撰。



 《後漢南記》四十五卷本五十五卷,今殘缺,晉江州從事張瑩撰。



 《後漢書》九十五卷本一百卷,晉秘書監袁山松撰。



 《後漢書》九十七卷宋太子詹事範曄撰。



 《後漢書》一百二十五卷範曄本,梁剡令劉昭注。



 《後漢書音》一卷後魏太常劉芳撰。



 《範漢音訓》三卷陳宗道先生臧競撰。



 《範漢音》三卷蕭該撰。



 《後漢書贊論》四卷範曄撰。



 《漢書纘》十八卷範曄撰。梁有蕭子顯《後漢書》一百卷,王韶《後漢林》二百卷,韋闡《後漢音》二卷,亡。



 《魏書》四十八卷晉司空王沈撰。



 《吳書》二十五卷韋昭撰。本五十五卷,梁有,今殘缺。



 《吳紀》九卷晉太學博士環濟撰。晉有張勃《吳錄》三十卷,亡。



 《三國志》六十五卷敘錄一卷,晉太子中庶子陳壽撰,宋太中大夫裴松之注。



 《魏志音義》一卷盧宗道撰。



 《論三國志》九卷何常侍撰。



 《三國志評》三卷徐眾撰。梁有《三國志序評》三卷,晉著作佐郎王濤撰,亡。



 《晉書》八十六卷本九十三卷,今殘缺。晉著作郎
 王隱撰。



 《晉書》二十六卷本四十四卷,訖明帝,今殘缺。晉散騎常侍虞預撰。



 《晉書》十卷未成,本十四卷,今殘缺。晉中書郎硃鳳撰,訖元帝。



 《晉中興書》七十八卷起東晉。宋湘東太守何法盛撰。



 《晉書》三十六卷宋臨川內史謝靈運撰。



 《晉書》一百一十卷齊徐州主簿臧榮緒撰。



 《晉書》十一卷本一百二卷,梁有,今殘缺。蕭子云撰。



 《晉史草》三十卷梁蕭子顯撰。梁有鄭忠《晉書》七卷,沈約《晉書》一百一十一卷,庾銑《東晉新書》七卷,亡。



 《宋書》六十五卷宋中散大夫徐爰撰。



 《宋書》六十五卷齊冠軍錄事參軍孫嚴撰。



 《宋書》一百卷梁尚書僕射沈約撰。梁有宋大明中所撰《宋書》六十一卷,亡。



 《齊書》六十卷梁吏部尚書蕭子顯撰。



 《齊紀》十卷劉陟撰。



 《齊紀》二十卷沈約撰。梁有江淹《齊史》十三卷,亡。



 《梁書》四十九卷梁中書郎謝吳撰,本一百卷。



 《梁史》五十三卷陳領軍、大著作郎許亭撰。



 《梁書帝紀》七卷姚察撰。



 《通史》四百八十卷梁武帝撰。起三
 皇;訖梁。



 《後魏書》一百三十卷後齊僕射魏收撰。



 《後魏書》一百卷著作郎魏彥深撰。



 《陳書》四十二卷訖宣帝,陳吏部尚書陸瓊撰。



 《周史》十八卷未成。吏部尚書牛弘撰。



 右六十七部,三千八十三卷。通計亡書,合八十部,四千三十卷。



 古者天子諸侯,必有國史,以紀言行,後世多務,其道彌繁。夏殷已上,左史記言,右史記事,周則太史、小史、內史、外史、御史,分掌其事,而諸侯之國,亦置史官。又《春秋國語》引周志、鄭書之說,推尋事蹟,似當時記事,各有職司,後又合而撰之,總成書記。其後陵夷衰亂,史官放絕,秦滅先王之典,遺制莫存。至漢武帝時,始置太史公,命司
 馬談為之,以掌其職。時天下計書,皆先上太史,副上丞相,遺文古事,靡不畢臻。談乃據《左氏》、《國語》、《世本》、《戰國策》、《楚漢春秋》,接其後事,成一家之言。談卒,其子遷又為太史令,嗣成其志。上自黃帝,訖於炎漢,合十二本紀、十表、八書、三十世家、七十列傳,謂之《史記》。遷卒以後,好事者亦頗著述,然多鄙淺,不足相繼。至後漢扶風班彪,綴後傳數十篇,並譏正前失。彪卒,明帝命其子固續成其志。以為唐、虞、三代,世有典籍,史遷所記,乃以漢氏繼于百王之末,非其義也。故斷自高祖,終於孝平、王莽之誅,為十二紀、八表、十志、六十九傳。潛心積思,二十餘年。建初
 中,始奏表及紀傳,其十志竟不能就。固卒後,始命曹大家續成之。先是明帝召固為蘭台令史,與諸先輩陳宗、尹敏、孟冀等,共成《光武本紀》。擢固為郎,典校秘書。固撰後漢事,作《列傳載記》二十八篇。其後劉珍、劉毅、劉陶、伏無忌等,相次著述東觀,謂之《漢記》。及三國鼎峙,魏氏及吳,並有史官。晉時,巴西陳壽刪集三國之事,唯魏帝為紀,其功臣及吳、蜀之主,並皆為傳,仍各依其國,部類相從,謂之《三國志》。壽卒後,梁州大中正範穎表奏其事,帝詔河南尹、洛陽令,就壽家寫之。自是世有著述,皆擬班、馬,以為正史,作者尤廣。一代之史,至數十家。唯《史記》、《漢
 書》,師法相傳,並有解釋。《三國志》及范曄《後漢》,雖有音注,既近世之作,並讀之可知。梁時,明《漢書》有劉顯、韋棱,陳時有姚察,隋代有包愷、蕭該,並為名家。《史記》傳者甚微。今依其世代,聚而編之,以備正史。



 古史



 《紀年》十二卷《汲塚書》,並《竹書同異》一卷。



 《漢紀》三十卷漢秘書監荀悅撰。



 《後漢紀》三十卷袁彥伯撰。



 《後漢紀》三十卷張璠撰。



 《獻帝春秋》十卷袁曄撰。



 《魏氏春秋》二十卷孫盛撰。



 《魏紀》十二卷左將軍陰澹撰。



 《漢魏春秋》九卷孔舒元撰。



 《晉紀》四卷陸機撰。



 《晉紀》二十三卷幹寶撰。訖湣帝。



 《晉紀》十卷晉前軍諮議曹嘉之撰。



 《漢晉陽秋》四十七卷訖湣帝。晉滎陽太守習鑿齒撰。



 《晉紀》十一卷訖明帝。晉荊州別駕鄧粲撰。



 《晉陽秋》三十二卷訖
 哀帝。孫盛撰。



 《晉紀》二十三卷宋中散大夫劉謙之撰。



 《晉紀》十卷宋吳興太守王韶之撰。



 《晉紀》四十五卷宋中散大夫徐廣撰。



 《續晉陽秋》二十卷宋永嘉太守檀道鸞撰。



 《續晉紀》五卷宋新興太守郭季產撰。



 《宋略》二十卷梁通直郎裴子野撰。



 《宋春秋》二十卷梁吳興令王琰撰。



 《齊春秋》三十卷梁奉朝請吳均撰。



 《齊典》五卷王逸撰。



 《齊典》十卷



 《三十國春秋》三十一卷梁湘東世子蕭方等撰。



 《戰國春秋》二十卷李概撰。



 《梁典》三十卷劉璠撰。



 《梁典》三十卷陳始興王諮議何之元撰。



 《梁撮要》三十卷陳征南諮議陰僧仁撰。



 《梁後略》十卷姚勖撰。



 《梁太清紀》十卷梁長沙蕃王蕭韶撰。



 《淮海亂離志》四卷蕭世怡撰。敘梁末侯景之亂。



 《齊紀》三十卷紀後齊事。崔子發撰。



 《齊志》十卷後齊事。王劭撰。



 右三十四部,六百六十六卷。



 自史官放絕,作者相承,皆以班、馬為准。起漢獻帝,雅好典籍,以班固《漢書》文繁難省,命潁川荀悅作《春秋左傳》之體,為《漢紀》三十篇。言約而事詳,辯論多美,大行於世。至晉太康元年,汲郡人發魏襄王塚,得古竹簡書,字皆科斗。發塚者不以為意,往往散亂。帝命中書監荀勖、令和嶠,撰次為十五部,八十七卷。多雜碎怪妄,不可訓知,唯《周易》、《紀年》,最為分了。其《周易》上下篇,與今正同。《紀年》皆用夏正建寅之月為歲首,起自夏、殷、周三代王事,無諸侯國別。唯特記晉國,起自稱殤叔,次文侯、昭侯,以至曲沃莊伯,盡晉國滅。獨記魏事,下至魏哀王,謂之「今王」。蓋
 魏國之史記也。其著書皆編年相次,文意大似《春秋經》。諸所記事,多與《春秋》、《左氏》扶同。學者因之,以為《春秋》則古史記之正法,有所著述,多依《春秋》之體。今依其世代,編而敘之,以見作者之別,謂之古史。



 雜史



 《周書》十卷《汲塚書》,似仲尼刪書之餘。



 《古文瑣語》四卷《汲塚書》。



 《春秋前傳》十卷何承天撰。



 《春秋前雜傳》九卷何承天撰。



 《春秋後傳》三十一卷晉著作郎樂資撰。



 《戰國策》三十二卷劉向錄。



 《戰國策》二十一卷高誘撰注。



 《戰國策論》一卷漢京兆尹延篤撰。



 《楚漢春秋》九卷陸賈撰。



 《古今注》八卷伏無忌撰。



 《越絕記》十六卷子貢撰。



 《吳越春秋》十二卷趙曄撰。



 《吳越春秋削繁》五卷楊方撰。



 《吳越春秋》十卷皇甫遵撰。



 《吳越
 記》六卷



 《南越志》八卷沈氏撰。



 《小史》八卷



 《漢靈獻二帝紀》三卷漢侍中劉芳撰,殘缺。梁有六卷。



 《山陽公載記》十卷樂資撰。



 《漢末英雄記》八卷王粲撰,殘缺。梁有十卷。



 《九州春秋》十卷司馬彪撰,記漢末事。



 《魏武本紀》四卷梁並曆五卷。



 《魏尚書》八卷孔衍撰。梁十卷,成。



 《魏晉世語》十卷晉襄陽令郭頒撰。



 《魏末傳》二卷梁又有《魏末傳》並《魏氏大事》三卷;亡。



 《呂布本事》一卷毛範撰。



 《晉諸公贊》二十一卷晉秘書監傅暢撰。



 《晉後略記》五卷晉下邳太守荀綽撰。



 《晉書鈔》三十卷梁豫章內史張緬撰。



 《晉書鴻烈》六卷張氏撰。



 《宋中興伐逆事》二卷



 《宋拾遺》十卷梁少府卿謝綽撰。



 《左史》六卷李概撰。



 《魏國統》二十卷梁祚撰。



 《梁帝紀》七卷



 《梁太清錄》八卷



 《梁承聖中興略》十卷劉仲威撰。



 《梁末代紀》一卷



 《
 梁皇帝實錄》三卷周興嗣撰。記武帝事。



 《梁皇帝實錄》五卷梁中書郎謝吳撰。記元帝事。



 《棲鳳春秋》五卷臧嚴撰。



 《陳王業曆》一卷陳中書郎趙齊旦撰。



 《史要》十卷漢桂陽太守衛颯撰。約《史記》要言,以類相從。



 《典略》八十九卷魏郎中魚豢撰。



 《史漢要集》二卷晉祠部郎王蔑撰。抄《史記》,入《春秋》者不錄。



 《三史略》二十九卷吳太子太傅張溫撰。



 《史記正傳》九卷張瑩撰。



 《後漢略》二十五卷張緬撰。



 《漢皇德紀》三十卷漢有道徵士侯瑾撰。起光武,至沖帝。



 《洞紀》四卷韋昭撰。記庖犧已來,至漢建安二十七年。



 《續洞紀》一卷臧榮緒撰。



 《帝王世紀》十卷皇甫謐撰。起三皇,盡漢、魏。



 《帝王世紀音》四卷虞綽撰。



 《帝王本紀》十卷來奧撰。



 《續帝王世紀》十卷何茂材撰。



 《十五代略》十卷吉文甫撰。起庖犧,至晉。



 《帝王要略》十二卷環濟撰。紀帝王及天官、地理、喪服。



 《周載》八卷東晉臨賀太守孟儀
 撰。略記前代,下至秦。本三十卷,今亡。



 《漢書鈔》三十卷晉散騎常侍葛洪撰。



 《拾遺錄》二卷偽秦姚萇方士王子年撰。



 《王子年拾遺記》十卷蕭綺撰。



 《華夷帝王世紀》三十卷楊曄撰。



 《正史削繁》九十四卷阮孝緒撰。



 《童悟》十二卷



 《帝王世錄》一卷甄鸞撰。



 《先聖本紀》十卷劉縚撰。



 《年曆帝紀》三十卷姚恭撰。



 《帝王諸侯世略》十一卷



 《王霸記》三卷潘傑撰。



 《歷代記》三十二卷



 《隋書》六十卷未成。秘書監王劭撰。



 右七十二部,九百一十七卷。通計亡書,七十三部,九百三十九卷。



 自秦撥去古文,篇籍遺散。漢初得《戰國策》,蓋戰國遊士記其策謀。其後陸賈作《楚漢春秋》,以述誅鋤秦、項之事。又有《越絕》,相承以為子貢所作。後漢趙曄又為《吳越春
 秋》。其屬辭比事,皆不與《春秋》、《史記》、《漢書》相似,蓋率爾而作,非史策之正也。靈、獻之世,天下大亂,史官失其常守。博達之士,湣其廢絕,各記聞見,以備遺亡。是後群才景慕,作者甚眾。又自後漢已來,學者多鈔撮舊史,自為一書,或起自人皇,或斷之近代,亦各其志,而體制不經。又有委巷之說,迂怪妄誕,真虛莫測。然其大抵皆帝王之事,通人君子,必博采廣覽,以酌其要,故備而存之,謂之雜史。



 霸史



 《趙書》十卷一曰《二石集》,記石勒事。偽燕太傅長史田融撰。



 《二石傳》二卷晉北中郎參軍王度撰。



 《二石偽治時事》二卷王度撰。



 《漢之書》十卷常璩撰。



 《華陽
 國志》十二卷常璩撰。梁有《蜀平記》十卷,《蜀漢偽官故事》一卷,亡。



 《燕書》二十卷記慕容雋事。偽燕尚書范亨撰。



 《南燕錄》五卷記慕容德事。偽燕尚書郎張詮撰。



 《南燕錄》六卷記慕容德事。偽燕中書郎王景暉撰。



 《南燕書》七卷遊覽先生撰。



 《燕志》十卷記馮跋事。魏侍中高閭撰。



 《秦書》八卷何仲熙撰。記苻健事。



 《秦記》十一卷宋殿中將軍裴景仁撰,梁雍州主簿席惠明注。



 《秦紀》十卷記姚萇事。魏左民尚書姚和都撰。



 《涼記》八卷記張軌事。偽燕右僕射張諮撰。



 《涼書》十卷記張軌事。偽涼大將軍從事中郎劉景撰。



 《西河記》二卷記張重華事。晉侍御史喻歸撰。



 《涼記》十卷記呂光事。偽涼著作佐郎段龜龍撰。



 《涼書》十卷高道讓撰。



 《涼書》十卷沮渠國史。



 《托跋涼錄》十卷



 《敦煌實錄》十卷劉景撰。



 《十六國春秋》一百卷魏崔鴻撰。



 《纂錄》一十卷



 《戰國春秋》二十卷李概撰。



 《漢趙記》十卷和苞撰。



 《吐谷渾記》二卷
 宋新亭侯段國撰。梁有《翟遼書》二卷,《諸國略記》二卷,《永嘉後纂年記》二卷,《段業傳》一卷,亡。



 《天啟紀》十卷記梁元帝子諝據湘州事。



 右二十七部,三百三十五卷。通計亡書,合三十三部,三百四十六卷。



 《傳》曰:「不有君子,其能國乎?」自晉永嘉之亂,皇綱失馭,九州君長,據有中原者甚眾。或推奉正朔,或假名竊號,然其君臣忠義之節,經國字民之務,蓋亦勤矣。而當時臣子,亦各記錄。後魏克平諸國,據有嵩、華,始命司徒崔浩,博采舊聞,綴述國史。諸國記注,盡集秘閣。爾硃之亂,並皆散亡。今舉其見在,謂之霸史。



 起居注



 《穆天子傳》六卷《汲塚書》。郭璞注。



 《漢獻帝起居注》五卷



 《晉泰始
 起居注》二十卷李軌撰。



 《晉咸寧起居注》十卷李軌撰。



 《晉泰康起居注》二十一卷李軌撰。



 《晉元康起居注》一卷梁有《永平、元康、永寧、起居注》六卷,又有《惠帝起居注》二卷,《永嘉、建興起居注》十三卷,亡。



 《晉建武、大興、永昌起居注》九卷梁有二十卷。



 《晉元康起居注》一卷



 《晉鹹和起居注》十六卷李軌撰。



 《晉咸康起居注》二十二卷



 《晉建元起居注》四卷



 《晉永和起居注》十七卷梁有二十四卷。



 《晉升平起居注》十卷



 《晉隆和、興寧起居注》五卷



 《晉咸安起居注》三卷



 《晉泰和起居注》六卷梁十卷。



 《晉甯康起居注》六卷



 《晉泰元起居注》二十五卷梁五十四卷。



 《晉隆安起居注》十卷



 《晉元興起居注》九卷



 《晉義熙起居注》十七卷
 梁三十四卷。



 《晉元熙起居注》二卷



 《晉起居注》三百一十七卷宋北徐州主簿劉道會撰。梁有三百二十二卷。



 《流別起居注》三十七卷梁有《晉宋起居注鈔》五十一卷,《晉宋先朝起居注》二十卷,亡。



 《宋永初起居注》十卷



 《宋景平起居注》三卷



 《宋元嘉起居注》五十五卷梁六十卷。



 《宋孝建起居注》十二卷



 《宋大明起居注》十五卷梁三十四卷,又有《景和起居注》四卷,《明帝在蕃注》三卷,亡。



 《宋泰始起居注》十九卷梁二十三卷。



 《宋泰豫起居注》四卷梁有《宋元徽起居注》二十卷,《升明起居注》六卷,亡。



 《齊永明起居注》二十五卷梁有三十四卷,又有《建元起居注》十二卷,《隆昌、延興、建武起居注》四卷,《中興起居注》四卷,亡。



 《梁大同起居注》十卷



 《後魏起居注》三百三十六卷



 《陳永定起居注》八卷



 《陳天嘉起居注》二十三卷



 《
 陳天康、光大起居注》十卷



 《陳太建起居注》五十六卷



 《陳至德起居注》四卷



 《後周太祖號令》三卷



 《隋開皇起居注》六十卷



 《南燕起居注》一卷



 右四十四部,一千一百八十九卷。



 起居注者,錄紀人君言行動止之事。《春秋傳》曰:「君舉必書。書而不法,後嗣何觀?」《周官》:內史掌王之命,遂書其副而藏之,是其職也。漢武帝有《禁中起居注》,後漢明德馬後撰《明帝起居注》,然則漢時起居,似在宮中,為女史之職。然皆零落,不可複知。今之存者,有漢獻帝及晉代已來《起居注》,皆近侍之臣所錄。晉時,又得《汲塚書》,有《穆天
 子傳》,體制與今起居正同,蓋周時內史所記王命之副也。近代已來,別有其職,事在《百官志》。今依其先後,編而次之。其偽國起居,唯《南燕》一卷,不可別出,附之於此。



 舊事



 《漢武帝故事》二卷



 《西京雜記》二卷



 《漢魏吳蜀舊事》八卷



 《晉朝雜事》二卷



 《晉宋舊事》一百三十五卷



 《晉要事》三卷



 《晉故事》四十三卷



 《晉建武故事》一卷



 《晉咸和、咸康故事》四卷晉孔愉撰。



 《晉修復山陵故事》五卷車灌撰。



 《交州雜事》九卷記士燮及陶璜事。



 《晉八王故事》十卷



 《晉四王起事》四卷晉廷尉盧綝撰。



 《大司馬陶公故事》三卷



 《郤太尉為尚書令故事》三卷



 《桓玄偽事》三卷



 《晉東宮舊
 事》十卷



 《秦漢已來舊事》十卷



 《尚書大事》二十卷范汪撰。



 《沔南故事》三卷應思遠撰。



 《天正舊事》三卷釋撰,亡名。



 《皇儲故事》二卷



 《梁舊事》三十卷內史侍郎蕭大圜撰。



 《東宮典記》七十卷左庶子宇文愷撰。



 《開業平陳記》二十卷



 右二十五部,四百四卷。



 古者朝廷之政,發號施令,百司奉之,藏於官府,各修其職,守而弗忘。《春秋傳》曰「吾視諸故府」,則其事也。《周官》:御史掌治朝之法,太史掌萬民之約契與質劑,以逆邦國之治。然則百司庶府,各藏其事,太史之職,又總而掌之。漢時,蕭何定律令,張蒼制章程,叔孫通定儀法,條流派
 別,制度漸廣。晉初,甲令已下,至九百余卷,晉武帝命車騎將軍賈充,博引群儒,刪采其要,增律十篇。其餘不足經遠者為法令,施行制度者為令,品式章程者為故事,各還其官府。搢紳之士,撰而錄之,遂成篇卷,然亦隨代遺失。今據其見存,謂之舊事篇。



 職官



 《漢官解詁》三篇漢新汲令王隆撰,胡廣注。



 《漢官》五卷應劭注。



 《漢官儀》十卷應劭撰。



 《漢官典職儀式選用》二卷漢衛尉蔡質撰。梁有《荀攸魏官儀》一卷,《韋昭官儀職訓》一卷,亡。



 《晉公卿禮秩故事》九卷傅暢撰。



 《晉新定儀注》十四卷梁有徐宣瑜《晉官品》一卷,荀綽《百官表注》十六卷,幹寶《司徒儀》一卷,宋《職官記》九卷,晉《百官儀服錄》五卷,大興二年《定官品事》五卷,《百官品》九卷,亡。



 《百官階次》一卷



 《齊職儀》五十
 卷齊長水校尉王珪之撰。梁有王珪之《齊儀》四十九卷,亡。



 《齊職儀》五卷



 《梁選簿》三卷徐勉撰。



 《梁勳選格》一卷



 《職官要錄》三十卷陶撰。



 《梁官品格》一卷



 《百官階次》三卷



 《新定將軍名》一卷



 《吏部用人格》一卷



 《官族傳》十四卷何晏撰。



 《百官春秋》五十卷王秀道撰。



 《百官春秋》二十卷



 《魏晉百官名》五卷



 《晉百官名》三十卷



 《晉官屬名》四卷



 《陳百官簿狀》二卷



 《陳將軍簿》一卷



 《新定官品》二十卷梁沈約撰。



 《梁尚書職制儀注》四十一卷



 《職令古今百官注》十卷郭演撰。



 右二十七部,三百三十六卷。通計亡書,合三十六部,四百三十三卷。



 古之仕者,名書於所臣之策,各有分職,以相統治。《周官》:
 塚宰掌建邦之六典,而御史數凡從正者。然則塚宰總六卿之屬,以治其政,御史掌其在位名數,先後之次焉。今《漢書百官表》列眾職之事,記在位之次,蓋亦古之制也。漢末,王隆、應劭等,以《百官表》不具,乃作《漢官解詁》、《漢官儀》等書。是後相因,正史表志,無複百僚在官之名矣。搢紳之徒,或取官曹名品之書,撰而錄之,別行於世。宋、齊已後,其書益繁,而篇卷零疊,易為亡散;又多瑣細,不足可紀,故刪。其見存可觀者,編為職官篇。



 儀注



 《漢舊儀》四卷衛敬仲撰。梁有衛敬仲《漢中興儀》一卷,亡。



 《晉新定儀注》四十卷晉安成太守傅瑗撰。



 《晉雜儀注》十一卷



 《晉尚書儀》十卷



 《甲辰
 儀》五卷江左撰。



 《封禪儀》六卷



 《宋儀注》十卷



 《宋儀注》二十卷



 《宋尚書雜注》十八卷本二十卷。



 《宋東宮儀記》二十三卷宋新安太守張鏡撰。



 《徐爰家儀》一卷



 《東宮新記》二十卷蕭子雲撰。



 《梁吉禮儀注》十卷明山賓撰。



 《梁賓禮儀注》九卷賀緌撰。案:梁明山賓撰《吉儀注》二百六卷,錄六卷;嚴植之撰《凶儀注》四百七十九卷,錄四十五卷;陸璉撰《軍儀注》一百九十卷,錄二卷;司馬襲撰《嘉儀注》一百一十二卷,錄三卷。並亡。存者唯《士》、《吉》及《賓》,合十九卷。



 《皇典》二十卷梁豫章太守丘仲孚撰。



 《雜凶禮》四十二卷



 《政禮儀注》十卷何胤撰。梁有何胤《士喪儀注》九卷,亡。



 《雜儀注》一百八十卷



 《陳尚書雜儀注》五百五十卷



 《陳吉禮》一百七十一卷



 《陳賓禮》六十五卷



 《陳軍禮》六卷



 《陳嘉禮》一百二卷



 《後魏儀注》五十卷



 《後
 齊儀注》二百九十卷



 《雜嘉禮》三十八卷



 《國親皇太子序親簿》一卷



 《隋朝儀禮》一百卷牛弘撰。



 《大漢輿服志》一卷魏博士董巴撰。



 《魏晉諡議》十三卷何晏撰。



 《汝南君諱議》二卷



 《決疑要注》一卷摯虞撰。



 《車服雜注》一卷徐廣撰。



 《禮儀制度》十三卷王逡之撰。



 《古今輿服雜事》二十卷梁周遷撰。



 《晉鹵簿圖》一卷



 《鹵簿儀》二卷



 《陳鹵簿圖》一卷



 《齊鹵簿儀》一卷



 《諸衛左右廂旗圖樣》十五卷



 《內外書儀》四卷謝元撰。



 《書儀》二卷蔡超撰。



 《書筆儀》二十一卷謝朏撰。



 《宋長沙檀太妃薨吊答書》十二卷



 《吊答儀》十卷王儉撰。



 《書儀》十卷王弘撰。



 《皇室儀》十三卷鮑行卿撰。



 《吉書儀》二卷王儉撰。



 《書儀疏》一卷周
 舍撰。



 《新儀》三十卷鮑泉撰。



 《文儀》二卷梁修端撰。



 《趙李家儀》十卷錄一卷,李穆叔撰。



 《書儀》十卷唐瑾撰。



 《言語儀》十卷



 《嚴植之儀》二卷



 《邇儀》四卷馬樞撰。



 《婦人書儀》八卷



 《僧家書儀》五卷釋曇瑗撰。



 《要典雜事》五十卷



 右五十九部,二千二十九卷。通計亡書,合六十九部,三千九十四卷。



 儀注之興,其所由來久矣。自君臣父子,六親九族,各有上下親疏之別。養生送死,吊恤賀慶,則有進止威儀之數。唐、虞已上,分之為三,在周因而為五。《周官》:宗伯所掌吉、凶、賓、軍、嘉,以佐王安邦國,親萬民,而太史執書以協事之類是也。是時典章皆具,可履而行。周衰,諸侯削除
 其籍。至秦,又焚而去之。漢興,叔孫通定朝儀,武帝時始祀汾陰後土,成帝時初定南北之郊,節文漸具。後漢又使曹褒定漢儀,是後相承,世有製作。然猶以舊章殘缺,各遵所見,彼此紛爭,盈篇滿牘。而後世多故,事在通變,或一時之制,非長久之道,載筆之士,刪其大綱,編于史志。而或傷於淺近,或失于未達,不能盡其旨要。遺文餘事,亦多散亡。今聚其見存,以為儀注篇。



 刑法



 《律本》二十一卷杜預撰。



 《漢晉律序注》一卷晉僮長張斐撰。



 《雜律解》二十一卷張斐撰。案:梁有《杜預雜律》七卷,亡。



 《晉、宋、齊、梁律》二十卷蔡法度撰。



 《梁律》二十卷梁義興太守蔡法度撰。



 《後魏律》二十卷



 《北齊律》十
 二卷目一卷。



 《陳律》九卷範泉撰。



 《周律》二十五卷



 《周大統式》三卷



 《隋律》十二卷



 《隋大業律》十一卷



 《晉令》四十卷



 《梁令》三十卷錄一卷。



 《梁科》三十卷



 《北齊令》五十卷



 《北齊權令》二卷



 《陳令》三十卷範泉撰。



 《陳科》三十卷範泉撰。



 《隋開皇令》三十卷目一卷。



 《隋大業令》三十卷



 《漢朝議駁》三十卷應劭撰。案:梁有《建武律令故事》二卷,劉邵《律略論》五卷亡。



 《晉雜議》十卷



 《晉彈事》十卷



 《南台奏事》二十二卷



 《漢名臣奏事》三十卷



 《魏王奏事》十卷



 《魏名臣奏事》四十卷目一卷,陳壽撰。



 《魏台雜訪議》三卷高堂隆撰。



 《魏廷尉決事》十卷



 《晉駁事》四卷



 《晉雜制》六十卷



 《晉刺史六條制》一卷



 《齊五
 服制》一卷



 《陳新制》六十卷



 右三十五部,七百一十二卷。通計亡書,合三十八部,七百二十六卷。



 刑法者,先王所以懲罪惡,齊不軌者也。《書》述唐、虞之世,五刑有服,而夏後氏正刑有五,科條三千。《周官》:司寇掌三典以刑邦國;司刑掌五刑之法,麗萬民之罪;太史又以典法逆于邦國;內史執國法以考政事。《春秋傳》曰:「在九刑不忘。」然而刑書之作久矣。蓋藏於官府,懼人之知爭端,而輕於犯。及其末也,肆情越法,刑罰僭濫。至秦,重之以苛虐,先王之正刑滅矣。漢初,蕭何定律九章,其後漸更增益,令甲已下,盈溢架藏。晉初,賈充、杜預刪而定
 之,有律,有令,有故事。梁時,又取故事之宜於時者為《梁科》。後齊武成帝時,又於麟趾殿刪正刑典,謂之《麟趾格》。後周太祖,又命蘇綽撰《大統式》。隋則律令格式並行。自律已下,世有改作,事在《刑法志》。《漢律》久亡,故事駁議,又多零失。今錄其見存可觀者,編為刑法篇。



 雜傳



 《三輔決錄》七卷漢太僕趙岐撰,摯虞注。



 《海內先賢傳》四卷魏明帝時撰。



 《四海耆舊傳》一卷



 《海內士品》一卷



 《先賢集》三卷



 《兗州先賢傳》一卷



 《徐州先賢傳》一卷



 《徐州先賢傳贊》九卷劉義慶撰。



 《海岱志》二十卷齊前將軍記室崔慰祖撰。



 《交州先賢傳》三卷晉範瑗傳。



 《益部耆舊傳》十四卷陳長壽撰。



 《續益部耆舊傳》二
 卷



 《諸國清賢傳》一卷



 《魯國先賢傳》二卷晉大司農白褒撰。



 《楚國先賢傳贊》十二卷晉張方撰。



 《汝南先賢傳》五卷魏周斐撰。



 《陳留耆舊傳》二卷漢議郎圈稱撰。



 《陳留耆舊傳》一卷魏散騎侍郎蘇林撰。



 《陳留先賢像贊》一卷陳英宗撰。



 《陳留志》十五卷東晉剡令江敞撰。



 《濟北先賢傳》一卷



 《廬江七賢傳》二卷



 《東萊耆舊傳》一卷王基撰。



 《襄陽耆舊記》五卷習鑿齒撰。



 《會稽先賢傳》七卷謝承撰。



 《會稽後賢傳記》二卷鐘離岫撰。



 《會稽典錄》二十四卷虞豫撰。



 《會稽先賢像贊》五卷



 《漢世要記》一卷



 《吳先賢傳》四卷吳左丞相陸凱撰。



 《東陽朝堂像贊》一卷晉南平太守留叔先撰。



 《豫章烈士傳》三卷徐整撰。



 《豫章舊志》三卷晉會稽太守熊默撰。



 《豫章舊志後撰》一
 卷熊欣撰。



 《零陵先賢傳》一卷



 《長沙耆舊傳贊》三卷晉臨川王郎中劉彧撰。



 《桂陽先賢畫贊》一卷吳左中郎張勝撰。



 《武昌先賢志》二卷宋天門太守郭緣生撰。



 《蜀文翁學堂像題記》二卷



 《呈賢高士傳贊》三卷嵇康撰,周續之注。



 《高士傳》六卷皇甫謐撰。



 《逸士傳》一卷皇甫謐撰。



 《逸民傳》七卷張顯撰。



 《高士傳》二卷虞槃佐撰。



 《至人高士傳贊》二卷晉廷尉卿孫綽撰。



 《高隱傳》十卷阮孝緒撰。



 《高隱傳》十卷



 《高僧傳》六卷虞孝敬撰。



 《止足傳》十卷



 《續高士傳》七卷周弘讓撰。



 《孝子傳贊》三卷王韶之撰。



 《孝子傳》十五卷晉輔國將軍蕭廣濟撰。



 《孝子傳》十卷宋員外郎鄭緝之撰。



 《孝子傳》八卷師覺授撰。



 《孝子傳》二十卷宋躬撰。



 《孝子傳略》二卷



 《孝德傳》三十卷梁元帝撰。



 《孝友傳》八卷



 《曾
 參傳》一卷



 《忠臣傳》三十卷梁元帝撰。



 《顯忠錄》二十卷梁元懌撰。



 《丹陽尹傳》十卷梁元帝撰。



 《英蕃可錄》二卷張萬賢撰,邵武侯新注。



 《高才不遇傳》四卷後齊劉晝撰。



 《良吏傳》十卷鐘岏撰。



 《海內名士傳》一卷



 《正始名士傳》三卷袁敬仲撰。



 《江左名士傳》一卷劉義慶撰。



 《竹林七賢論》二卷晉太子中庶子戴逵撰。



 《七賢傳》五卷孟氏撰。



 《文士傳》五十卷張騭撰。



 《列士傳》二卷劉向撰。



 《陰德傳》二卷宋光祿大夫范晏撰。



 《悼善傳》十一卷



 《雜傳》三十六卷任昉撰。本一百四十七卷,亡。



 《東方朔傳》八卷



 《毌丘儉記》三卷



 《管輅傳》三卷管辰撰。



 《雜傳》四十卷賀蹤撰。本七十卷,亡。



 《雜傳》十九卷陸澄撰。



 《雜傳》十一卷



 《玄晏春秋》三卷皇甫謐撰。



 《孔子弟子先儒傳》十卷。



 《李氏家傳》
 一卷



 《桓氏家傳》一卷



 《王朗、王肅家傳》一卷



 《太原王氏家傳》二十三卷



 《褚氏家傳》一卷褚凱等撰。



 《薛常侍家傳》一卷



 《江氏家傳》七卷江祚等撰。



 《庾氏家傳》一卷庾斐撰。



 《裴氏家傳》四卷裴松之撰。



 《虞氏家記》五卷虞覽撰。



 《曹氏家傳》一卷曹毗撰。



 《範氏家傳》一卷范汪撰。



 《紀氏家紀》一卷紀友撰。



 《韋氏家傳》一卷



 《何顒使君家傳》一卷



 《明氏家訓》一卷偽燕衛尉明岌撰。



 《明氏世錄》六卷梁信武記室明粲撰。



 《陸史》十五卷



 《王氏江左世家傳》二十卷王褒撰。



 《孔氏家傳》五卷



 《崔氏五門家傳》二卷崔氏撰。



 《暨氏家傳》一卷



 《周、齊王家傳》一卷姚氏撰。



 《爾硃家傳》一卷王氏撰。



 《周氏家傳》一卷



 《令狐氏家傳》一
 卷



 《新舊傳》四卷



 《漢南庾氏家傳》三卷



 《何氏家傳》三卷



 《童子傳》二卷王鎮之撰。



 《幼童傳》十卷劉昭撰。



 《訪來傳》十卷來奧撰。



 《懷舊志》九卷梁元帝撰。



 《知己傳》一卷盧思道撰。



 《全德志》一卷梁元帝撰。



 《同姓名錄》一卷梁元帝撰。



 《列女傳》十五卷劉向撰,曹大家注。



 《列女傳》七卷趙母注。



 《列女傳》八卷高氏撰。



 《列女傳頌》一卷劉歆撰。



 《列女傳頌》一卷曹植撰。



 《列女傳贊》一卷繆襲撰。



 《列女後傳》十卷項原撰。



 《列女傳》六卷皇甫謐撰。



 《列女傳》七卷綦毋邃撰。



 《列女傳要錄》三卷



 《女記》十卷杜預撰。



 《美婦人傳》六卷



 《姑記》二卷虞通之撰。



 《道人善道開傳》一卷康泓撰。



 《名僧傳》三十卷釋寶唱撰。



 《高僧傳》十四卷釋慧皎撰。



 《江東名德傳》三卷釋法進撰。



 《法師傳》十卷王
 巾撰。



 《眾僧傳》二十卷裴子野撰。



 《薩婆多部傳》五卷釋僧祐撰。



 《梁故草堂法師傳》一卷



 《尼傳》二卷釋寶唱撰。



 《法顯傳》二卷



 《法顯行傳》一卷



 《梁武皇帝大舍》三卷嚴灊撰。



 《列仙傳贊》三卷劉向撰,鬷續,孫綽贊。



 《列仙傳贊》二卷劉向撰,晉郭元祖贊。



 《神仙傳》十卷葛洪撰。



 《說仙傳》一卷硃思祖撰。



 《養性傳》二卷



 《漢武內傳》三卷



 《太元真人東鄉司命茅君內傳》一卷弟子李遵撰。



 《清虛真人王君內傳》一卷弟子華存撰。



 《清虛真人裴君內傳》一卷



 《正一真人三天法師張君內傳》一卷



 《太極左仙公葛君內傳》一卷



 《仙人馬君陰君內傳》一卷



 《仙人許遠遊傳》一卷



 《靈人辛玄子自序》一卷



 《劉君內記》一卷王珍撰。



 《
 陸先生傳》一卷孔稚珪撰。



 《列仙贊序》一卷郭元祖撰。



 《集仙傳》十卷



 《洞仙傳》十卷



 《王喬傳》一卷



 《關令內傳》一卷鬼谷先生撰。



 《南嶽夫人內傳》一卷



 《蘇君記》一卷周季通撰。



 《嵩高寇天師傳》一卷



 《華陽子自序》一卷



 《太上真人內記》一卷李氏撰。



 《道學傳》二十卷



 《宣驗記》十三卷劉義慶撰。



 《應驗記》一卷宋光祿大夫傅亮撰。



 《冥祥記》十卷王琰撰。



 《列異傳》三卷魏文帝撰。



 《感應傳》八卷王延秀撰。



 《古異傳》三卷宋永嘉太守袁王壽撰。



 《甄異傳》三卷晉西戎主簿戴祚撰。



 《述異記》十卷祖沖之撰。



 《異苑》十卷宋給事劉敬叔撰。



 《續異苑》十卷



 《搜神記》三十卷幹寶撰。



 《搜神後記》十卷陶潛撰。



 《靈鬼志》三卷荀氏撰。



 《志怪》二卷祖台之撰。



 《志怪》四卷孔氏撰。



 《神錄》五卷
 劉之遴撰。



 《齊諧記》七卷宋散騎侍郎東陽無疑撰。



 《續齊諧記》一卷吳均撰。



 《幽明錄》二十卷劉義慶撰。



 《補續冥祥記》一卷王曼穎撰。



 《漢武洞冥記》一卷郭氏撰。



 《嘉瑞記》三卷陸瓊撰。



 《祥瑞記》三卷



 《符瑞記》十卷許善心撰。



 《靈異錄》十卷



 《靈異記》十卷



 《研神記》十卷蕭繹撰。



 《旌異記》十五卷侯君素撰。



 《近異錄》二卷劉質撰。



 《鬼神列傳》一卷謝氏撰。



 《志怪記》三卷殖氏撰。



 《舍利感應記》三卷王劭撰。



 《真應記》十卷



 《周氏冥通記》一卷



 《集靈記》二十卷顏之推撰。



 《冤魂志》三卷顏之推撰。



 右二百一十七部,一千二百八十六卷。通計亡書,合二百一十九部,一千五百三卷。



 古之史官,必廣其所記,非獨人君之舉。《周官》:外史掌四方之志,則諸侯史記,兼而有之。《春秋傳》曰:「虢仲、虢叔,王季之穆,勳在王室,藏於盟府。」臧紇之叛,季孫命太史召掌惡臣而盟之。《周官》:司寇凡大盟約,蒞其盟書,登於天府。太史、內史、司會,六官皆受其貳而藏之。是則王者誅賞,具錄其事,昭告神明,百官史臣,皆藏其書。故自公卿諸侯,至於群士,善惡之跡,畢集史職。而又閭胥之政,凡聚眾庶,書其敬敏任恤者,族師每月書其孝悌睦涘有學者,党正歲書其德行道藝者,而入之于鄉大夫。鄉大夫三年大比,考其德行道藝,舉其賢者能者,而獻其書。
 王再拜受之,登於天府,內史貳之。是以窮居側陋之士,言行必達,皆有史傳。自史官曠絕,其道廢壞,漢初,始有丹書之約,白馬之盟。武帝從董仲舒之言,始舉賢良文學。天下計書,先上太史,善惡之事,靡不畢集。司馬遷、班固,撰而成之,股肱輔弼之臣,扶義俶儻之士,皆有記錄。而操行高潔,不涉於世者,《史記》獨傳夷齊,《漢書》但述楊王孫之儔,其餘皆略而不說。又漢時,阮倉作《列仙圖》,劉向典校經籍,始作《列仙》、《列士》、《列女》之傳,皆因其志尚,率爾而作,不在正史。後漢光武,始詔南陽,撰作風俗,故沛、三輔有耆舊節士之序,魯、廬江有名德先賢之贊。郡國
 之書,由是而作。魏文帝又作《列異》,以序鬼物奇怪之事,嵇康作《高士傳》,以敘聖賢之風。因其事類,相繼而作者甚眾,名目轉廣,而又雜以虛誕怪妄之說。推其本源,蓋亦史官之末事也。載筆之士,刪采其要焉。魯、沛、三輔,序贊並亡,後之作者,亦多零失。今取其見存,部而類之,謂之雜傳。



 地理



 《山海經》二十三卷郭璞注。



 《水經》三卷郭璞注。



 《黃圖》一卷記三輔宮觀陵廟明堂辟雍郊畤等事。



 《洛陽記》四卷



 《洛陽記》一卷陸機撰。



 《洛陽宮殿簿》一卷



 《洛陽圖》一卷晉懷州刺史楊牷期撰。



 《述征記》二卷郭緣生撰。



 《西征記》二卷戴延之撰。



 《婁地記》一卷吳顧啟期撰。



 《風土記》三
 卷晉平西將軍周處撰。



 《吳興記》三卷山謙之撰。



 《吳郡記》一卷顧夷撰。



 《京口記》二卷宋太常卿劉損撰。



 《南徐州記》二卷山謙之撰。



 《會稽土地記》一卷硃育撰。



 《會稽記》一卷賀循撰。



 《隨王入沔記》六卷宋侍中沈懷文撰。



 《荊州記》三卷宋臨川王侍郎盛弘之撰。



 《神壤記》一卷記榮陽山水。黃閔撰。



 《豫章記》一卷雷次宗撰。



 《蜀王本記》一卷揚雄撰。



 《三巴記》一卷譙周撰。



 《珠崖傳》一卷偽燕聘晉使蓋泓撰。



 《陳留風俗傳》三卷圈稱撰。



 《鄴中記》二卷晉國子助教陸翽撰。



 《春秋土地名》三卷晉裴秀客京相璠撰。



 《衡山記》一卷宗居士撰。



 《遊名山志》一卷謝靈運撰。



 《聖賢塚墓記》一卷李彤撰。



 《佛國記》一卷沙門釋法顯撰。



 《遊行外國傳》一卷沙門釋智猛撰。



 《交州以南外國傳》一卷



 《十洲記》一卷東方朔撰。



 《神異經》一卷東方朔撰,
 張華注。



 《異物志》一卷後漢議郎楊孚撰。



 《南州異物志》一卷吳丹陽太守萬震撰。



 《蜀志》一卷東京武平太守常寬撰。



 《發蒙記》一卷束皙撰。載物產之異。



 《地理書》一百四十九卷錄一卷。陸澄合《山海經》已來一百六十家,以為此書。澄本之外,其舊事並多零失。見存別部自行者,唯四十二家,今列之於上。



 《三輔故事》二卷晉世撰。



 《湘州記》二卷庾仲雍撰。



 《吳郡記》二卷晉本州主簿顧夷撰。



 《日南傳》一卷



 《江記》五卷庾仲雍撰。



 《漢水記》五卷庾仲雍撰。



 《居名山志》一卷謝靈運撰。



 《西征記》一卷戴祚撰。



 《廬山南陵雲精舍記》一卷



 《永初山川古今記》二十卷齊都官尚書劉澄之撰。



 《元康三年地記》六卷



 《司州記》二卷



 《並帖省置諸郡舊事》一卷



 《地記》二百五十二卷梁任昉增陸澄之書八十四家,以為此記。其所增舊書,亦多零失。見存別部行者,唯十二家,今列之於上。



 《山
 海經圖贊》二卷郭璞注。



 《山海經音》二卷



 《水經》四十卷酈善長注。



 《廟記》一卷



 《地理書抄》二十卷陸澄撰。



 《地理書抄》九卷任昉撰。



 《地理書抄》十卷劉黃門撰。



 《洛陽伽藍記》五卷後魏楊衒之撰。



 《荊南地志》二卷蕭世誠撰。



 《巴蜀記》一卷



 《交州異物志》一卷楊孚撰。



 《元康六年戶口名簿記》三卷



 《元嘉六年地記》三卷



 《九州郡縣名》九卷



 《扶南異物志》一卷硃應撰。



 《臨海水土異物志》一卷沈瑩撰。



 《益州記》三卷李氏撰。



 《湘州記》一卷郭仲產撰。



 《湘州圖副記》一卷



 《四海百川水源記》一卷釋道安撰。



 《京師寺塔記》十卷錄一卷。劉璆撰。



 《華山精舍記》一卷張光祿撰。



 《南雍州記》六卷鮑至撰。



 《京師寺塔記》二卷釋曇宗撰。



 《張騫出關志》一卷



 《外
 國傳》五卷釋曇景撰。



 《曆國傳》二卷釋法盛撰。



 《西京記》三卷



 《京師錄》七卷



 《尋江源記》一卷



 《後園記》一卷



 《江表行記》一卷



 《淮南記》一卷



 《古來國名》二卷



 《十三州志》十卷闞駰撰。



 《慧生行傳》一卷



 《宋武北征記》一卷戴氏撰。



 《林邑國記》一卷



 《涼州異物志》一卷



 《閟象傳》二卷閭先生撰。



 《司州山川古今記》三卷劉澄之撰。



 《江圖》一卷張氏撰。



 《江圖》二卷劉氏撰。



 《廣梁南徐州記》九卷虞孝敬撰。



 《水飾圖》二十卷



 《甌閩傳》一卷



 《北荒風俗記》二卷



 《諸蕃風俗記》二卷



 《男女二國傳》一卷



 《突厥所出風俗事》一卷



 《古今地譜》二卷



 《異地志》三十卷陳顧野王撰。



 《序行記》十卷姚最撰。



 《魏永安
 記》三卷溫子升撰。



 《國都城記》二卷



 《周地圖記》一百九卷



 《冀州圖經》一卷



 《齊州圖經》一卷



 《齊州記》四卷李叔布撰。



 《幽州圖經》一卷



 《魏聘使行記》六卷



 《聘北道裡記》三卷江德藻撰。



 《李諧行記》一卷



 《聘遊記》三卷劉師知撰。



 《朝覲記》六卷



 《封君義行記》一卷李繪撰。



 《輿駕東行記》一卷薛泰撰。



 《北伐記》七卷諸葛穎撰。



 《巡撫揚州記》七卷諸葛穎撰。



 《大魏諸州記》二十一卷



 《并州入朝道裡記》一卷蔡允恭撰。



 《趙記》十卷



 《代都略記》三卷



 《世界記》五卷釋僧祐撰。



 《州郡縣簿》七卷



 《大隋翻經婆羅門法師外國傳》五卷



 《隋區宇圖志》一百二十九卷



 《隋西域圖》三卷裴矩撰。



 《隋諸州圖經集》一
 百卷郎蔚之撰。



 《隋諸郡土俗物產》一百五十一卷



 《西域道裡記》三卷



 《諸蕃國記》十七卷



 《方物志》二十卷許善心撰。



 《并州總管內諸州圖》一卷



 右一百三十九部,一千四百三十二卷。通計亡書,合一百四十部,一千四百三十四卷。



 昔者先王之化民也,以五方土地,風氣所生,剛柔輕重,飲食衣服,各有其性,不可遷變。是故疆理天下,物其土宜,知其利害,達其志而通其欲,齊其政而修其教。故曰廣穀大川異制,人居其間異俗。《書》錄禹別九州,定其山川,分其圻界,條其物產,辨其貢賦,斯之謂也。周則夏官
 司險,掌建九州之圖,周知山林川澤之阻,達其道路。地官誦訓,掌方志以詔觀事,以知地俗。春官保章,以星土辨九州之地,所封之域,以觀祅祥。夏官職方,掌天下之圖地,辨四夷八蠻九貉五戎六狄之人,與其財用九穀六畜之數,周知利害,辨九州之國,使同其貫。司徒掌邦之土地之圖與其人民之教,以佐王擾邦國,周知九州之域,廣輪之數,辨其山林川澤丘陵墳衍原隰之名物,及土會之法。然則其事分在眾職,而塚宰掌建邦之六典,實總其事。太史以典逆塚宰之治,其書蓋亦總為史官之職。漢初,蕭何得秦圖書,故知天下要害。後又得《山
 海經》,相傳以為夏禹所記。武帝時,計書既上太史,郡國地志,固亦在焉。而史遷所記,但述河渠而已。其後劉向略言地域,丞相張禹使屬硃貢條記風俗,班固因之作《地理志》。其州國郡縣山川夷險時俗之異,經星之分,風氣所生,區域之廣,戶口之數,各有攸敘,與古《禹貢》、《周官》所記相埒。是後載筆之士,管窺末學,不能及遠,但記州郡之名而已。晉世,摯虞依《禹貢》、《周官》,作《畿服經》,其州郡及縣分野封略事業,國邑山陵水泉,鄉亭城道裡土田,民物風俗,先賢舊好,靡不具悉,凡一百七十卷,今亡。而學者因其經歷,並有記載,然不能成一家之體。齊時,陸
 澄聚一百六十家之說,依其前後遠近,編而為部,謂之《地理書》。任昉又增陸澄之書八十四家,謂之《地記》。陳時,顧野王抄撰眾家之言,作《輿地志》。隋大業中,普詔天下諸郡,條其風俗物產地圖,上于尚書。故隋代有《諸郡物產土俗記》一百五十一卷,《區宇圖志》一百二十九卷,《諸州圖經集》一百卷。其餘記注甚眾。今任、陸二家所記之內而又別行者,各錄在其書之上,自余次之於下,以備地理之記焉。



 譜系



 《世本王侯大夫譜》二卷



 《世本》二卷劉向撰。



 《世本》四卷宋衷撰。



 《漢氏帝王譜》三卷梁有《宋譜》四卷,劉湛《百家譜》二卷,亡。



 《齊帝譜屬》十卷



 《
 百家集譜》十卷王儉撰。梁有王逡之《續儉百家譜》四卷,《南族譜》二卷,《百家譜拾遺》一卷,又有《齊、梁帝譜》四卷,《梁帝譜》十三卷,亡。



 《百家譜》三十卷王僧孺撰。



 《百家譜集鈔》十五卷王僧孺撰。



 《百家譜》二十卷賈執撰。



 《百家譜》十五卷傅昭撰。



 《百家譜世統》十卷



 《百家譜鈔》五卷



 《姓氏英賢譜》一百卷賈執撰。案:梁有《王司空新集諸州譜》十一卷,又別有《諸姓譜》一百一十六卷,《益州譜》四十卷,《關東、關北譜》三十三卷,《梁武帝總集境內十八州譜》六百九十卷,亡。



 《後魏辯宗錄》二卷元暉業撰。



 《後魏皇帝宗族譜》四卷



 《魏孝文列姓族牒》一卷



 《後齊宗譜》一卷



 《益州譜》三十卷



 《冀州姓族譜》二卷



 《洪州諸姓譜》九卷



 《吉州諸姓譜》八卷



 《江州諸姓譜》十一卷



 《諸州雜譜》八卷



 《袁州諸姓譜》八卷



 《揚州譜
 鈔》五卷



 《京兆韋氏譜》二卷



 《謝氏譜》一十卷



 《楊氏血脈譜》二卷



 《楊氏家譜狀並墓記》一卷



 《楊氏枝分譜》一卷



 《楊氏譜》一卷



 《北地傅氏譜》一卷



 《蘇氏譜》一卷



 《述系傳》一卷姚最撰。



 《氏族要狀》十五卷



 《姓苑》一卷何氏撰。



 《複姓苑》一卷



 《齊永元中表簿》五卷



 《竹譜》一卷



 《錢譜》一卷顧烜撰。



 《錢圖》一卷



 右四十一部,三百六十卷。通計亡書,合五十三部,一千二百八十卷。



 氏姓之書,其所由來遠矣。《書》稱「別生分類」。《傳》曰:「天子建德,因生以賜姓。」周家小史定系世,辨昭穆,則亦史之職也。秦兼天下,剗除舊跡,公侯子孫,失其本系。漢初,得《世
 本》,敘黃帝已來祖世所出。而漢又有《帝王年譜》,後漢有《鄧氏官譜》。晉世,摯虞作《族姓昭穆記》十卷,齊、梁之間,其書轉廣。後魏遷洛,有八氏十姓,鹹出帝族。又有三十六族,則諸國之從魏者;九十二姓,世為部落大人者,並為河南洛陽人。其中國士人,則第其門閥,有四海大姓、郡姓、州姓、縣姓。及周太祖入關,諸姓子孫有功者,並令為其宗長,仍撰譜錄,紀其所承。又以關內諸州,為其本望。其《鄧氏官譜》及《族姓昭穆記》,晉亂已亡。自餘亦多遺失。今錄其見存者,以為譜系篇。



 簿錄



 《七略別錄》二十卷劉向撰。



 《七略》七卷劉歆撰。



 《晉中經》十四卷
 荀勖撰。



 《晉義熙已來新集目錄》三卷



 《宋元徽元年四部書目錄》四卷王儉撰。



 《今書七志》七十卷王儉撰。



 《梁天監六年四部書目錄》四卷殷鈞撰。



 《梁東宮四部目錄》四卷劉遵撰。



 《梁文德殿四部目錄》四卷劉孝標撰。



 《七錄》十二卷阮孝緒撰。



 《魏闕書目錄》一卷



 《陳秘閣圖書法書目錄》一卷



 《陳天嘉六年壽安殿四部目錄》四卷



 《陳德教殿四部目錄》四卷



 《陳承香殿五經史記目錄》二卷



 《開皇四年四部目錄》四卷



 《開皇八年四部書目錄》四卷



 《香廚四部目錄》四卷



 《隋大業正禦書目錄》九卷



 《法書目錄》六卷



 《雜儀注目錄》四卷



 《雜撰文章家集敘》十卷荀勖撰。



 《文
 章志》四卷摯虞撰。



 《續文章志》二卷傅亮撰。



 《晉江左文章志》三卷宋明帝撰。



 《宋世文章志》二卷沈約撰。



 《書品》二卷



 《名手畫錄》一卷



 《正流論》一卷



 右三十部,二百一十四卷。



 古者史官既司典籍,蓋有目錄,以為綱紀,體制堙滅,不可複知。孔子刪書,別為之序,各陳作者所由。韓、毛二《詩》,亦皆相類。漢時劉向《別錄》、劉歆《七略》,剖析條流,各有其部,推尋事蹟,疑則古之制也。自是之後,不能辨其流別,但記書名而已。博覽之士,疾其渾漫,故王儉作《七志》,阮孝緒作《七錄》,並皆別行。大體雖准向、歆,而遠不逮矣。其
 先代目錄,亦多散亡。今總其見存,編為簿錄篇。



 凡史之所記,八百一十七部,一萬三千二百六十四卷。通計亡書,合八百七十四部,一萬六千五百五十八卷。



 夫史官者,必求博聞強識,疏通知遠之士,使居其位,百官眾職,鹹所貳焉。是故前言往行,無不識也;天文地理,無不察也;人事之紀,無不達也。內掌八柄,以詔王治,外執六典,以逆官政。書美以彰善,記惡以垂戒,範圍神化,昭明令德,窮聖人之至賾,詳一代之亹亹。自史官廢絕久矣,漢氏頗循其舊,班、馬因之。魏、晉已來,其道逾替。南、董之位,以祿貴遊,政、駿之司,罕因才授。故梁世諺曰:「上
 車不落則著作,體中何如則秘書。」於是屍素之儔,盱衡延閣之上,立言之士,揮翰蓬茨之下。一代之記,至數十家,傳說不同,聞見舛駁,理失中庸,辭乖體要。致令允恭之德,有闕於典墳,忠肅之才,不傳于簡策。斯所以為蔽也。班固以《史記》附《春秋》,今開其事類,凡十三種,別為
 史部。



\end{pinyinscope}