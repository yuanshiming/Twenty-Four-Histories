\article{卷七十五列傳第四十}

\begin{pinyinscope}

 儒林儒之為教大矣,其利物博矣。篤父子,正君臣,尚忠節,重仁義,貴廉讓,賤貪鄙,開政化之本源,鑿生民之耳目,百王損益,一以貫之。雖世或污隆,而斯文不墜,經邦致治,非一時也。涉其流者,無祿而富,懷其道者,無位而尊。故仲尼頓挫於魯君,孟軻抑揚於齊後,荀卿見珍於強楚,叔孫取貴於隆漢。其餘處環堵以驕富貴,安陋巷而輕
 王公者,可勝數哉!自晉室分崩,中原喪亂,五胡交爭,經籍道盡。魏氏發跡代陰,經營河朔,得之馬上,茲道未弘。暨夫太和之後,盛修文教,搢紳碩學,濟濟盈朝,縫掖巨儒,往往傑出,其雅誥奧義,宋及齊、梁不能尚也。



 南北所治,章句好尚,互有不同。江左《周易》則王輔嗣,《尚書》則孔安國,《左傳》則杜元凱。河、洛《左傳》則服子慎,《尚書》、《周易》則鄭康成。



 《詩》則並主於毛公,《禮》則同遵於鄭氏。大抵南人約簡,得其英華,北學深蕪,窮其枝葉。考其終始,要其會歸,其立身成名,殊方同致矣。爰自漢、魏,碩學多清通,逮乎近古,巨儒必鄙俗。文、武不墜,弘之在人,豈獨愚蔽於
 當今,而皆明哲於往昔?在乎用與不用,知與不知耳。然曩之弼諧庶績,必舉德於鴻儒,近代左右邦家,咸取士於刀筆。縱有學優入室,勤逾刺股,名高海內,擢第甲科,若命偶時來,未有望於青紫,或數將運舛,必委棄於草澤。然則古之學者,祿在其中,今之學者,困於貧賤。明達之人,志識之士,安肯滯於所習,以求貧賤者哉?此所以儒罕通人,學多鄙俗者也。昔齊列康莊之第,多士如林,燕起碣石之宮,群英自遠。



 是知俗易風移,必由上之所好,非夫聖明御世,亦無以振斯頹俗矣。自正朔不一,將三百年,師說紛綸,無所取正。高祖膺期纂歷,平一寰宇,
 頓天網以掩之,賁旌帛以禮之,設好爵以縻之,於是四海九州強學待問之士,靡不畢集焉。天子乃整萬乘,率百僚,遵問道之儀,觀釋奠之禮。博士罄懸河之辯,侍中竭重席之奧,考正亡逸,研核異同,積滯群疑,渙然冰釋。於是超擢奇秀,厚賞諸儒,京邑達乎四方,皆啟黌校。齊、魯、趙、魏,學者尤多,負笈追師,不遠千里,講誦之聲,道路不絕。中州儒雅之盛,自漢、魏以來,一時而已。及高祖暮年,精華稍竭,不悅儒術,專尚刑名,執政之徒,咸非篤好。既仁壽間,遂廢天下之學,唯存國子一所,弟子七十二人。煬帝即位,復開庠序,國子郡縣之學,盛於開皇之初。
 征闢儒生,遠近畢至,使相與講論得失於東都之下,納言定其差次,一以聞奏焉。於時舊儒多已凋亡,二劉拔萃出類,學通南北,博極今古,後生鉆仰,莫之能測。所制諸經義疏,搢紳咸師宗之。既而外事四夷,戎馬不息,師徒怠散,盜賊群起,禮義不足以防君子,刑罰不足以威小人,空有建學之名,而無弘道之實。其風漸墜,以至滅亡,方領矩步之徒,亦多轉死溝壑。凡有經籍,自此皆湮沒於煨塵矣。遂使後進之士不復聞《詩》、《書》之言,皆懷攘奪之心,相與陷於不義。《傳》曰:「學者將植,不學者將落。」然則盛衰是系,興亡攸在,有國有家者可不慎歟!諸儒有
 身沒道存,遺風可想,皆採其餘論,綴之於此篇云。



 元善元善,河南洛陽人也。祖叉,魏侍中。父羅,初為梁州刺史,及叉被誅,奔於梁,官至征北大將軍、青冀二州刺史。善少隨父至江南,性好學,遂通涉五經,尤明《左氏傳》。及侯景之亂,善歸於周。武帝甚禮之,以為太子宮尹,賜爵江陽縣公。每執經以授太子。開皇初,拜內史侍郎,上每望之曰:「人倫儀表也。」凡有敷奏,詞氣抑揚,觀者屬目。陳使袁雅來聘,上令善就館受書,雅出門不拜。善論舊事有拜之儀,雅不能對,遂拜,成禮而去。後遷國子祭酒。上嘗
 親臨釋奠,命善講《孝經》。於是敷陳義理,兼之以諷諫。上大悅曰:「聞江陽之說,更起朕心。」



 賚絹百匹,衣一襲。善之通博,在何妥之下,然以風流醖藉,俯仰可觀,音韻清朗,聽者忘倦,由是為後進所歸。妥每懷不平,心欲屈善。因善講《春秋》,初發題,諸儒畢集。善私謂妥曰:「名望已定,幸無相苦。」妥然之。及就講肆,妥遂引古今滯義以難,善多不能對。善深銜之,二人由是有隙。善以高熲有宰相之具,嘗言於上曰:「楊素粗疏,蘇威怯芃,元胄、元旻,正似鴨耳。可以付社稷者,唯獨高熲。」上初然之,及熲得罪,上以善之言為熲游說,深責望之。善憂懼,先患消渴,於是疾
 動而卒,時年六十。



 辛彥之辛彥之,隴西狄道人也。祖世敘,魏涼州刺史。父靈輔,周滑州刺史。彥之九歲而孤,不交非類,博涉經史,與天水牛弘同志好學。後入關,遂家京兆。周太祖見而器之,引為中外府禮曹,賜以衣馬珠玉。時國家草創,百度伊始,朝貴多出武人,修定儀注,唯彥之而已。尋拜中書侍郎。及周閔帝受禪,彥之與少宗伯盧辯專掌儀制。明、武時,歷職典祀,太祝、樂部、御正四曹大夫,開府儀同三司。奉使迎突厥皇后還,賚馬二百匹,賜爵龍門縣公,邑千戶。
 尋進爵五原郡公,加邑千戶。



 宣帝即位,拜少宗伯。高祖受禪,除太常少卿,改封任城郡公,進位上開府。尋轉國子祭酒。歲餘,拜禮部尚書,與秘書監牛弘撰《新禮》。吳興沈重名為碩學,高祖嘗令彥之與重論議,重不能抗,於是避席而謝曰:「辛君所謂金城湯池,無可攻之勢。」高祖大悅。後拜隨州刺史。於時州牧多貢珍玩,唯彥之所貢,並供祭之物。



 高祖善之,顧謂朝臣曰:「人安得無學!彥之所貢,稽古之力也。」遷潞州刺史,前後俱有惠政。彥之又崇信佛道,於城內立浮圖二所,並十五層。開皇十一年,州人張元暴死,數日乃蘇,雲游天上,見新構一堂,制極
 崇麗。元問其故,人云潞州刺史辛彥之有功德,造此堂以待之。彥之聞而不悅。其年卒官。謚曰宣。彥之撰《墳典》一部,《六官》一部,《祝文》一部,《新要》一部,《新禮》一部,《五經異義》一部,並行於世。有子仲龕,官至猗氏令。



 何妥蕭該包凱何妥,字棲鳳,西城人也。父細胡,通商入蜀,遂家郫縣,事梁武陵王妃,主知金帛,因致巨富,號為西州大賈。妥少機警,八歲游國子學,助教顧良戲之曰:「汝既姓何,是荷葉之荷,為是河水之河?」應聲答曰:「先生姓顧,是眷顧之顧,是新故之故?」眾咸異之。十七,以技巧事湘東王,後知
 其聰明,召為誦書左右。



 時蘭陵蕭亦有俊才,住青楊巷,妥住白楊頭,時人為之語曰:「世有兩俊,白楊何妥,青楊蕭。」其見美如此。江陵陷,周武帝尤重之,授太學博士。宣帝初欲立五後,以問儒者辛彥之,對曰:「後與天子匹體齊尊,不宜有五。」妥駁曰:「帝嚳四妃,舜又二妃,亦何常數?」由是封襄城縣伯。高祖受禪,除國子博士,加通直散騎常侍,進爵為公。妥性勁急,有口才,好是非人物。時納言蘇威嘗言於上曰:「臣先人每誡臣云,唯讀《孝經》一卷,足可立身治國,何用多為!」上亦然之。妥進曰:「蘇威所學,非止《孝經》。厥父若信有此言,威不從訓,是其不孝。若
 無此言,面欺陛下,是其不誠。不誠不孝,何以事君!且夫子有云:『不讀《詩》無以言,不讀《禮》無以立。』豈容蘇綽教子獨反聖人之訓乎?」威時兼領五職,上甚親重之,妥因奏威不可信任。又以掌天文律度,皆不稱職,妥又上八事以諫:其一事曰:臣聞知人則哲,惟帝難之。孔子曰:「舉直錯諸枉則民服,舉枉錯諸直則民不服。」由此言之,政之治亂,必慎所舉,故進賢受上賞,蔽賢蒙顯戮。



 察今之舉人,良異於此,無論諂直,莫擇賢愚。心欲崇高,則起家喉舌之任;意須抑屈,必白首郎署之官。人之不服,實由於此。臣聞爵人於朝,與士共之,刑人於市,與眾棄之。伏見
 留心獄訟,愛人如子,每應決獄,無不詢訪群公,刑之不濫,君之明也。刑既如此,爵亦宜然。若有懋功簡在帝心者,便可擢用。自斯以降,若選重官,必須參以眾議,勿信一人之舉;則上不偏私,下無怨望。



 其二事曰:孔子云:「是察阿黨,則罪無掩蔽。」又曰:「君子周而不比,小人比而不周。」所謂比者,即阿黨也。謂心之所愛,既已光華榮顯,猶加提挈;心之所惡,既已沈滯屈辱,薄言必怒。提挈既成,必相掩蔽,則欺上之心生矣;屈辱既加,則有怨恨,謗讟之言出矣。伏願廣加逖訪,勿使朋黨路開,威恩自任。有國之患,莫大於此。



 其三事曰:臣聞舜舉十六族,所謂八
 元、八愷也。計其賢明,理優今日,猶復擇才授任,不相侵濫,故得四門雍穆,庶績咸熙。今官員極多,用人甚少,有一人身上乃兼數職,為是國無人也?為是人不善也?今萬乘大國,髦彥不少,縱有明哲,無由自達。東方朔言曰:「尊之則為將,卑之則為虜。」斯言信矣。今當官之人,不度德量力,既無呂望、傅說之能,自負傅巖、滋水之氣,不慮憂深責重,唯畏總領不多,安斯寵任,輕彼權軸,好致顛蹶,實此之由。《易》曰:「鼎折足,覆公餗,其形渥,兇。」言不勝其任也。臣聞窮力舉重,不能為用。伏願更任賢良,分才參掌,使各行有餘力,則庶事康哉。



 其四事曰:臣聞《禮》云:「析
 言破律,亂名改作,執左道以亂政者殺。」孔子曰:「仍舊貫,何必改作。」伏見比年以來,改作者多矣。至如範威漏刻,十載不成;趙翊尺稱,七年方決。公孫濟迂誕醫方,費逾巨萬;徐道慶回互子午,糜耗飲食。常明破律,多歷歲時;王渥亂名,曾無紀極。張山居未知星位,前已蹂藉太常;曹魏祖不識北辰,今復轔轢太史。莫不用其短見,便自誇毗,邀射名譽,厚相誣罔。請今日已後,有如此者,若其言不驗,必加重罰,庶令有所畏忌,不敢輕奏狂簡。



 其餘文多不載。時蘇威權兼數司,先嘗隱武功,故妥言自負傅巖、滋水之氣,以此激上。書奏,威大銜之。十二年,威定
 考文學,又與妥更相訶詆。威勃然曰:「無何妥,不慮無博士!」妥應聲曰:「無蘇威,亦何憂無執事!」由是與威有隙。



 其後上令妥考定鐘律,妥又上表曰:臣聞明則有禮樂,幽則有鬼神,然則動天地,感鬼神,莫近於禮樂。又云樂至則無怨,禮至則不爭,揖讓而治天下者,禮樂之謂也。臣聞樂有二,一曰奸聲,二曰正聲。夫奸聲感人而逆氣應之,逆氣成象而淫樂興焉。正聲感人而順氣應之,順氣成象而和樂興焉。故樂行而倫清,耳目聰明,血氣和平,移風易俗,天下皆寧。



 孔子曰:「放鄭聲,遠佞人。」故鄭、衛、宋、趙之聲出,內則發疾,外則傷人。



 是以宮亂則荒,其君驕;商亂則陂,其官壞;角亂則憂,其人怨;徵亂則哀,其事勤;羽亂則
 危,其財匱。五者皆亂,則國亡無日矣。魏文侯問子夏曰:「吾端冕而聽古樂則欲寐,聽鄭、衛之音而不知倦,何也?」子夏對曰:「夫古樂者,始奏以文,復亂以武,修身及家,平均天下。鄭、衛之音者,奸聲以亂,溺而不止,颻雜子女,不知父子。今君所問者樂也,所愛者音也。夫樂與音,相近而不同,為人君者,謹審其好惡。」案聖人之作樂也,非止茍悅耳目而已矣。欲使在宗廟之內,君臣同聽之則莫不和敬;在鄉里之內,長幼同聽之則莫不和順;在閨門之內,父子同聽之則莫不和親。此先王立樂之方也。故知聲而不知音者,禽獸是也,知音而不知樂者,眾庶
 是也。故黃鐘大呂,弦歌干戚,僮子皆能儛之。能知樂者,其唯君子!



 不知聲者,不可與言音,不知音者,不可與言樂,知樂則幾於道矣。紂為無道,太師抱樂器以奔周。晉君德薄,師曠固惜清徵。



 上古之時,未有音樂,鼓腹擊壤,樂在其間。《易》曰:「先王作樂崇德,殷薦之上帝,以配祖考。」至於黃帝作《咸池》,顓頊作《六莖》,帝嚳作《五英》,堯作《大章》,舜作《大韶》,禹作《大夏》,湯作《大濩》,武王作《大武》,從夏以來,年代久遠,唯有名字,其聲不可得聞。自殷至周,備於《詩》《頌》。



 故自聖賢已下,多習樂者,至如伏羲減瑟,文王足琴,仲尼擊磬,子路鼓瑟,漢高擊築,元帝吹簫。漢高祖之初,
 叔孫通因秦樂人制宗廟之樂。迎神於廟門,奏《嘉至》之樂,猶古降神之樂也。皇帝入廟門,奏《永至》之樂,以為行步之節,猶古《採薺》、《肆夏》也。乾豆上薦,奏《登歌》之樂,猶古清廟之歌也。《登歌》再終,奏《休成》之樂,美神饗也。皇帝就東廂坐定,奏《永安》之樂,美禮成也。



 其《休成》、《永至》二曲,叔孫通所制也。漢高祖廟奏《武德》、《文始》、《五行》之儛,當春秋時,陳公子完奔齊,陳是舜後,故齊有《韶》樂,孔子在齊聞《韶》,三月不知肉味是也。秦始皇滅齊,得齊《韶》樂。漢高祖滅秦,《韶》傳於漢,高祖改名《文始》,以示不相襲也。《五行儛》者,本周《大武》樂也,始皇改曰《五
 行》。及於孝文,復作四時之人舞,以示天下安和,四時順也。孝景採《武德儛》以為《昭德》,孝宣又採《昭德》以為《盛德》,雖變其名,大抵皆因秦舊事。至於魏、晉,皆用古樂。魏之三祖,並制樂辭。自永嘉播越,五都傾蕩,樂聲南度,是以大備江東。宋、齊已來,至於梁代,所行樂事,猶皆傳古,三雍四始,實稱大盛。及侯景篡逆,樂師分散,其四儛、三調,悉度偽齊。齊氏雖知傳受,得曲而不用之於宗廟朝廷也。臣少好音律,留意管弦,年雖耆老,頗皆記憶。及東土克定,樂人悉返,訪其逗遛,果云是梁人所教。今三調、四儛並皆有手,雖不能精熟,亦頗具雅聲。若令教習傳授,庶
 得流傳古樂。然後取其會歸,撮其指要,因循損益,更制嘉名。歌盛德於當今,傳雅正於來葉,豈不美與!謹具錄三調、四舞曲名,又制歌辭如別。其有聲曲流宕,不可以陳於殿庭者,亦悉附之於後。



 書奏,別敕太常取妥節度。於是作清、平、瑟三調聲,又作八佾、《鞞》、《鐸》、《巾》、《拂》四舞。先是,太常所傳宗廟雅樂,數十年唯作大呂,廢黃鐘。妥又以深乖古意,乃奏請用黃鐘。詔下公卿議,從之。俄而妥子蔚為秘書郎,有罪當刑,上哀之,減死論。是後恩禮漸薄。六年,出為龍州刺史。時有負笈游學者,妥皆為講說教授之。為《刺史箴》,勒於州門外。在職三年,以疾請還,詔許
 之。復知學事。時上方使蘇夔在太常,參議鐘律。夔有所建議,朝士多從之,妥獨不同,每言夔之短。高祖下其議,朝臣多排妥。妥復上封事,指陳得失,大抵論時政損益,並指斥當世朋黨。於是蘇威及吏部尚書盧愷、侍郎薛道衡等皆坐得罪。除伊州刺史,不行,尋為國子祭酒。卒官。謚曰肅。撰《周易講疏》十三卷,《孝經義疏》三卷,《莊子義疏》四卷,及與沈重等撰《三十六科鬼神感應等大義》九卷,《封禪書》一卷,《樂要》一卷,文集十卷,並行於世。



 蘭陵蕭該者,梁鄱陽王恢之孫也。少封攸侯。梁荊州陷,與何妥同至長安。性篤學,《詩》、《書》、《春秋》、《禮記》並通大義,尤
 精《漢書》,甚為貴游所禮。開皇初,賜爵山陰縣公,拜國子博士。奉詔書與妥正定經史,然各執所見,遞相是非,久而不能就,上譴而罷之。該后撰《漢書》及《文選》音義,咸為當時所貴。



 東海包愷,字和樂。其兄愉,明《五經》,愷悉傳其業。又從王仲通受《史記》、《漢書》,尤稱精究。大業中,為國子助教。於時《漢書》學者,以蕭、包二人為宗匠。聚徒教授,著錄者數千人,卒,門人為起墳立碣焉。



 房暉遠房暉遠,字崇儒,恆山真定人也。世傳儒學。暉遠幼有志
 行,治《三禮》、《春秋三傳》、《詩》、《書》、《周易》,兼善圖緯,恆以教授為務。遠方負笈而從者,動以千計。齊南陽王綽為定州刺史,聞其名,召為博士。周武帝平齊,搜訪儒俊,暉遠首應闢命,授小學下士。及高祖受禪,遷太常博士。太常卿牛弘每稱為五經庫。吏部尚書韋世康薦之,為太學博士。尋與沛公鄭譯修正樂章。丁母憂,解任。後數歲,授殄寇將軍,復為太常博士。未幾,擢為國子博士。會上令國子生通一經者,並悉薦舉,將擢用之。既策問訖,博士不能時定臧否。祭酒元善怪問之,暉遠曰:「江南、河北,義例不同,博士不能遍涉。學生皆持其所短,稱己所長,博士各
 各自疑,所以久而不決也。」祭酒因令暉遠考定之,暉遠覽筆便下,初無疑滯。或有不服者,暉遠問其所傳義疏,輒為始末誦之,然後出其所短,自是無敢飾非者。所試四五百人,數日便決,諸儒莫不推其通博,皆自以為不能測也。尋奉詔預修令式。高祖嘗謂群臣曰:「自古天子有女樂乎?」楊素以下莫知所出,遂言無女樂。暉遠進曰:「臣聞『窈窕淑女,鐘鼓樂之』,此即王者房中之樂,著於《雅頌》,不得言無。」高祖大悅。仁壽中卒官,時年七十二,朝廷嗟惜焉,賵賻甚厚,贈員外散騎常侍。



 馬光
 馬光,字榮伯,武安人也。少好學,從師數十年,晝夜不息,圖書讖緯,莫不畢覽,尤明《三禮》,為儒者所宗。開皇初,高祖征山東義學之士,光與張仲讓、孔籠、竇士榮、張黑奴、劉祖仁等俱至,並授太學博士,時人號為六儒。然皆鄙野無儀範,朝廷不之貴也。士榮尋病死。仲讓未幾告歸鄉里,著書十卷,自云此書若奏,我必為宰相。又數言玄象事。州縣列上其狀,竟坐誅。孔籠、張黑奴、劉祖仁未幾亦被譴去。唯光獨存。嘗因釋奠,高祖親幸國子學,王公以下畢集。光升座講禮,啟發章門。已而諸儒生以次論難者十餘人,皆當時碩學,光剖析疑滯,雖辭非俊辨,而
 理義弘贍,論者莫測其淺深,咸共推服,上嘉而勞焉。山東《三禮》學者,自熊安生後,唯宗光一人。初,教授瀛、博間,門徒千數,至是多負笈從入長安。



 後數年,丁母憂歸鄉里,遂有終焉之志。以疾卒於家,時年七十三。



 劉焯劉焯,字士元,信都昌亭人也。父洽,郡功曹。焯犀額龜背,望高視遠,聰敏沈深,弱不好弄。少與河間劉炫結盟為友,同受《詩》於同郡劉軌思,受《左傳》於廣平郭懋常,問《禮》於阜城熊安生,皆不卒業而去。武強交津橋劉智海家素多墳籍,焯與炫就之讀書,向經十載,雖衣食不繼,晏
 如也。遂以儒學知名,為州博士。刺史趙煚引為從事,舉秀才,射策甲科。與著作郎王劭同修國史,兼參議律歷,仍直門下省,以待顧問。俄除員外將軍。後與諸儒於秘書省考定群言。因假還鄉里,縣令韋之業引為功曹。尋復入京,與左僕射楊素、吏部尚書牛弘、國子祭酒蘇威、國子祭酒元善、博士蕭該、何妥、太學博士房暉遠、崔宗德、晉王文學崔賾等於國子共論古今滯義前賢所不通者。每升座,論難鋒起,皆不能屈,楊素等莫不服其精博。六年,運洛陽《石經》至京師,文字磨滅,莫能知者,奉敕與劉炫等考定。後因國子釋奠,與炫二人論義,深挫諸
 儒,咸懷妒恨,遂為飛章所謗,除名為民。於是優游鄉里,專以教授著述為務,孜孜不倦。賈、馬、王、鄭所傳章句,多所是非。



 《九章算術》、《周髀》、《七曜歷書》十餘部,推步日月之經,量度山海之術,莫不核其根本,窮其秘奧。著《稽極》十卷,《歷書》十卷,《五經述議》,並行於世。劉炫聰明博學,名亞於焯,故時人稱二劉焉。天下名儒後進,質疑受業,不遠千里而至者,不可勝數。論者以為數百年已來,博學通儒,無能出其右者。然懷抱不曠,又嗇於財,不行束修者,未嘗有所教誨,時人以此少之。廢太子勇聞而召之,未及進謁,詔令事蜀王,非其好也,久之不至。王聞而大怒,
 遣人枷送於蜀,配之軍防。其後典校書籍。王以罪廢,焯又與諸儒修定禮律,除雲騎尉。煬帝即位,遷太學博士,俄以疾去職。數年,復被徵以待顧問,因上所著《歷書》,與太史令張胄玄多不同,被駁不用。大業六年卒,時年六十七。劉炫為之請謚,朝廷不許。



 劉炫劉炫,字光伯,河間景城人也。少以聰敏見稱,與信都劉焯閉戶讀書,十年不出。炫眸子精明,視日不眩,強記默識,莫與為儔。左畫方,右畫圓,口誦,目數,耳聽,五事同舉,無有遺失。周武帝平齊,瀛州刺史宇文亢引為戶曹從
 事。後刺史李繪署禮曹從事,以吏乾知名。歲餘,奉敕與著作郎王劭同修國史。俄直門下省,以待顧問。又與諸術者修天文律歷,兼於內史省考定群言,內史令博陵李德林甚禮之。炫雖遍直三省,竟不得官,為縣司責其賦役。炫自陳於內史,內史送詣吏部,吏部尚書韋世康問其所能。炫自為狀曰:「《周禮》、《禮記》、《毛詩》、《尚書》、《公羊》、《左傳》、《孝經》、《論語》孔、鄭、王、何、服、杜等注,凡十三家,雖義有精粗,並堪講授。《周易》、《儀禮》、《穀梁》,用功差少。



 史子文集,嘉言美事,咸誦於心。天文律歷,窮核微妙。至於公私文翰,未嘗假手。」



 吏部竟不詳試,然在朝知名之士十餘人,保明炫
 所陳不謬,於是除殿內將軍。



 時牛弘奏請購求天下遺逸之書,炫遂偽造書百餘卷,題為《連山易》、《魯史記》等,錄上送官,取賞而去。後有人訟之,經赦免死,坐除名,歸於家,以教授為務。太子勇聞而召之,既至京師,敕令事蜀王秀,遷延不往。蜀王大怒,枷送益州。既而配為帳內,每使執杖為門衛。俄而釋之,典校書史。炫因擬屈原《卜居》,為《筮途》以自寄。



 及蜀王廢,與諸儒修定《五禮》,授旅騎尉。吏部尚書牛弘建議,以為禮諸侯絕傍期,大夫降一等。今之上柱國,雖不同古諸侯,比大夫可也,官在第二品,宜降傍親一等。議者多以為然。炫駁之曰:「古之仕者,宗
 一人而已,庶子不得進。



 由是先王重適,其宗子有分祿之義。族人與宗子雖疏遠,猶服縗三月,良由受其恩也。今之仕者,位以才升,不限適庶,與古既異,何降之有。今之貴者,多忽近親,若或降之,民德之疏,自此始矣。」遂寢其事。開皇二十年,廢國子四門及州縣學,唯置太學博士二人,學生七十二人。炫上表言學校不宜廢,情理甚切,高祖不納。



 開皇之末,國家殷盛,朝野皆以遼東為意。炫以為遼東不可伐,作《撫夷論》以諷焉,當時莫有悟者。及大業之季,三征不克,炫言方驗。



 煬帝即位,牛弘引炫修律令。高祖之世,以刀筆吏類多小人,年久長奸,勢使
 然也。又以風俗陵遲,婦人無節。於是立格,州縣佐史,三年而代之,九品妻無得再醮。炫著論以為不可,弘竟從之。諸郡置學官,及流外給廩,皆發自於炫。弘嘗從容問炫曰:「案《周禮》士多而府史少,今令史百倍於前,判官減則不濟,其故何也?」炫對曰:「古人委任責成,歲終考其殿最,案不重校,文不繁悉,府史之任,掌要目而已。古今不同,若此之相懸也,事繁政弊,職此之由。」弘又問:「魏、齊之時,令史從容而已,今則不遑寧舍,其事何由?」炫對曰:「齊氏立州不過數十,三府行臺,遞
 相統領,文書行下,不過十條。今州三百,其繁一也。往者州唯置綱紀,郡置守丞,縣唯令而已。其所具僚,則長官自闢,受詔赴任,每州不過數十。今則不然,大小之官,悉由吏部,纖介之跡,皆屬考功,其繁二也。省官不如省事,省事不如清心。官事不省而望從容,其可得乎?」弘甚善其言而不能用。納言楊達舉炫博學有文章,射策高第,除太學博士。歲餘,以品卑去任,還至長平,奉敕追詣行在所。或言其無行,帝遂罷之,歸於河間。於時群盜蜂起,穀食踴貴,經籍道息,教授不行。炫與妻子相去百里,聲問斷絕,鬱鬱不得志,乃自為贊曰:通人司馬相如、揚子
 雲、馬季長、鄭康成等,皆自敘風徽,傳芳來葉。餘豈敢仰均先達,貽笑眾昆。待以日迫桑榆,大命將近,故友飄零,門徒雨散,溘死朝露,埋魂朔野,親故莫照其心,後人不見其跡,殆及餘喘,薄言胸臆,貽及行邁,傳示州里,使夫將來俊哲知余鄙志耳。餘從綰發以來,迄於白首,嬰孩為慈親所恕,棰楚未嘗加,從學為明師所矜,榎楚弗之及。暨乎敦敘邦族,交結等夷,重物輕身,先人後己。昔在幼弱,樂參長者,爰及耆艾,數接後生。學則服而不厭,誨則勞而不倦,幽情寡適,心事方違。內省生平,顧循終始,其大幸有四,其深恨有一。性本愚蔽,家業貧窶,為父兄
 所饒,廁縉紳之末,遂得博覽典誥,窺涉今古,小善著於丘園,虛名聞於邦國,其幸一也。隱顯人間,沈浮世俗,數忝徒勞之職,久執城旦之書,名不掛於白簡,事不染於丹筆,立身立行,慚恧實多,啟手啟足,庶幾可免,其幸二也。以此庸虛,屢動神眷,以此卑賤,每升天府,齊鑣驥騄,比翼鵷鴻,整緗素於鳳池,記言動於麟閣,參謁宰輔,造請群公,厚禮殊恩,增榮改價,其幸三也。晝漏方盡,大耋已嗟,退反初服,歸骸故里,玩文史以怡神,閱魚鳥以散慮,觀省野物,登臨園沼,緩步代車,無罪為貴,其幸四也。仰休明之盛世,慨道教之陵遲,蹈先儒之逸軌,傷群言
 之蕪穢,馳騖墳典,厘改僻謬,修撰始畢,圖事適成,天違人願,途不我與。世路未夷,學校盡廢,道不備於當時,業不傳於身後。銜恨泉壤,實在茲乎?其深恨一也。



 時在郡城,糧餉斷絕,其門人多隨盜賊,哀炫窮乏,詣郡城下索炫,郡官乃出炫與之。炫為賊所將,過城下堡。未幾,賊為官軍所破,炫饑餓無所依,復投縣城。



 長吏意炫與賊相知,恐為後變,遂閉門不納。是時夜冰寒,因此凍餒而死,時年六十八。其後門人謚曰宣德先生。



 炫性躁競,頗俳諧,多自矜伐,好輕侮當世,為執政所醜,由是官途不遂。著《論語述議》十卷,《春秋攻昧》十卷,《五經正名》十二卷,《孝
 經述議》五卷,《春秋述議》四十卷,《尚書述議》二十卷,《毛詩述議》四十卷,《注詩序》一卷,《算術》一卷,並行於世。



 褚輝吳郡褚輝,字高明,以《三禮》學稱於江南。煬帝時,徵天下儒術之士,悉集內史省,相次講論,輝博辯,無能屈者,由是擢為太學博士。撰《禮疏》一百卷。



 顧彪餘杭顧彪,字仲文,明《尚書》、《春秋》。煬帝時為秘書學士,撰《古文尚書疏》二十卷。



 魯世達餘杭魯世達,煬帝時為國子助教,撰《毛詩章句義疏》四十二卷,行於世。



 張沖吳郡張沖,字叔玄。仕陳為左中郎將,非其好也,乃覃思經典,撰《春秋義略》,異於杜氏七十餘事,《喪服義》三卷,《孝經義》三卷,《論語義》十卷,《前漢音義》十二卷。官至漢王侍讀。



 王孝籍平原王孝籍,少好學,博覽群言,遍治五經,頗有文乾。與河間劉炫同志友善。



 開皇中,召入秘書,助王劭修國史。
 劭不之禮,在省多年,而不免輸稅。孝籍鬱鬱不得志,奏記於吏部尚書牛弘曰:竊以毒螫寔膚,則申旦不寐,饑寒切體,亦卒歲無聊。何則?痛苦難以安,貧窮易為蹙。況懷抱之內,冰火鑠脂膏,腠理之間,風霜侵骨髓,安可齰舌緘脣,吞聲飲氣,惡呻吟之響,忍酸辛之酷哉!伏惟明尚書公動哀矜之色,開寬裕之懷,咳唾足以活枯鱗,吹噓可用飛窮羽。芬椒蘭之氣,暖布帛之詞,許小人之請,聞大君之聽。雖復山川不遠,鬼神在茲,信而有徵,言無不履,猶恐拯溺遲於援手,救經緩於扶足。待越人之舟楫,求魯匠之雲梯,則必懸於槁樹之枝,沒於深淵之底
 矣。



 夫以一介貧人,七年直省,課役不免,慶賞不沾,賣貢禹之田,供釋之之費,有弱子之累,乏強兄之產。加以老母在堂,光陰遲暮,寒暑違闕,關山超遠,嚙臂為期,前途逾邈,倚閭之望,朝夕已勤。謝相如之病,無官可以免,發梅福之狂,非仙所能避。愁疾甚乎厲鬼,人生異夫金石。營魂且散,恐筮予無徵,齎恨入冥,則虛緣恩顧,此乃王稽所以致言,應侯為之不樂也。潛鬢發之內,居眉睫之間,子野未曾聞,離硃所不見,沈淪東觀,留滯南史,終無薦引,永同埋殯。三世不移,雖由寂寞,十年不調,實乏知己。夫不世出者,聖明之君也,不萬一者,誠賢之臣也。以
 夫不世出而逢不萬一,此小人所以為明尚書幸也。坐人物之源,運銓衡之柄,反披狐白,不好緇衣,此小人為明尚書不取也。昔荊玉未剖,刖卞和之足,百里未用,碎禽息之首。居得言之地,有能用之資,增耳目之明,無手足之蹙,憚而弗為,孰知其解!夫官或不稱其能,士或未申其屈,一夫竊議,語流天下。勞不見圖,安能無望!儻病未及死,狂還克念,汗窮愁之簡,屬離憂之詞,記志於前修,通心於來哲,使千載之下,哀其不遇,追咎執事,有點清塵,則不肖之軀,死生為累,小人之罪,方且未刊。願少加憐愍,留心無忽!



 弘亦知其有學業,而竟不得調。後歸
 鄉里,以教授為業,終於家。注《尚書》及《詩》,遭亂零落。



 史臣曰:古語云:「容體不足觀,勇力不足恃,族姓不足道,先祖不足稱,然而顯聞四方,流聲後胤者,其唯學乎?」信哉斯言也。暉遠、榮伯之徒,篤志不倦,自求諸己,遂能聞道下風,稱珍席上。或聚徒千百,或服冕乘軒,見重明時,實惟稽古之力也。江陽從容雅望,風韻閑遠,清談高論,籍甚當年;彥之敦經悅史,砥身礪行,志存典制,動蹈規矩;何妥通涉俊爽,神情警悟,雅有口才,兼擅詞筆,然訐以為直,失儒者之風焉;劉焯道冠縉紳,數窮天象,既精且博,洞幽究微,銘深致遠,源流不測,數百年來,斯人而
 已;劉炫學實通儒,才堪成務,九流七略,無不該覽,雖探賾索隱,不逮於焯,裁成義說,文雅過之。並道亞生知,時不我與,或才登於下士,或餒棄於溝壑,惜矣。子夏有言:「死生有命,富貴在天。」天之所與者聰明,所不與者貴仕,上聖且猶不免,焯、炫其如命何!



\end{pinyinscope}