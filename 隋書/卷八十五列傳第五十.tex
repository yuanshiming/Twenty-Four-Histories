\article{卷八十五列傳第五十}

\begin{pinyinscope}

 夫肖形天地,人稱最靈,以其知父子之道,識君臣之義,異夫禽獸者也。傳曰:「人生在三,事之如一。」然則君臣父子,其道不殊,父不可以不父,子不可以不子,君不可以不君,臣不可以不臣。故曰君猶天也,天可仇乎!是以有罪歸刑,見危授命,竭忠貞以立節,不臨難而茍免。故聞其風者,懷夫慷慨,千載之後,莫不願以為臣。此其所以生榮死哀,取貴前哲者矣。至於委質策名,代卿世祿,出
 受心膂之寄,人參帷幄之謀,身處機衡,肆趙高之奸宄,世荷權寵,行王莽之桀逆,生靈之所仇疾,犬豕不食其餘。雖薦社污宮,彰必誅之釁,斫棺焚骨,明篡殺之咎,可以懲夫既往,未足深誡將來。昔孔子修《春秋》,而亂臣賊子知懼,抑使之求名不得,欲蓋而彰者也。今故正其罪名,以冠於篇首,庶後之君子,見作者之意焉。



 宇文化及弟智及司馬德戡裴虔通宇文化及,左翊衛大將軍述之子也。性兇險,不循法度,好乘肥挾彈,馳騖道中,由是長安謂之輕薄公子。煬帝為太子時,常領千牛,出入臥內。累遷至太子僕。



 數以受
 納貨賄,再三免官。太子嬖暱之,俄而復職。又以其弟士及尚南陽公主。化及由此益驕,處公卿間,言辭不遜,多所陵轢。見人子女狗馬珍玩,必請托求之。



 常與屠販者游,以規其利。煬帝即位,拜太僕少卿,蓋恃舊恩,貪冒尤甚。大業初,煬帝幸榆林,化及與弟智及違禁與突厥交市。帝大怒,囚之數月,還至青門外,欲斬之而後入城,解衣辮發,以公主故,久之乃釋,並智及並賜述為奴。述薨後,煬帝追憶之,遂起化及為右屯衛將軍,智及為將作少監。



 是時李密據洛口,煬帝懼,留淮左,不敢還都。從駕驍果多關中人,久客羈旅,見帝無西意,謀欲叛歸。時武
 賁郎將司馬德戡總領驍果,屯於東城,風聞兵士欲叛,未之審,遣校尉元武達陰問驍果,知其情,因謀構逆。共所善武賁郎將元禮、直閣裴虔通互相扇惑曰:「今聞陛下欲築宮丹陽,勢不還矣。所部驍果莫不思歸,人人耦語,並謀逃去。我欲言之,陛下性忌,惡聞兵走,即恐先事見誅。今知而不言,其後事發,又當族滅我矣。進退為戮,將如之何?」虔通曰:「上實爾,誠為公憂之。」德戡謂兩人曰:「我聞關中陷沒,李孝常以華陰叛,陛下收其二弟,將盡殺之。吾等家屬在西,安得無此慮也!」虔通曰:「我子弟已壯,誠不自保,正恐旦暮及誅,計無所出。」德戡曰:「同相憂,
 當共為計取。驍果若走,可與俱去。」



 虔通等曰:「誠如公言,求生之計,無以易此。」因遞相招誘。又轉告內史舍人元敏、鷹揚郎將孟秉,符璽郎李覆、牛方裕、直長許弘仁、薛良,城門郎唐奉義,醫正張愷等,日夜聚博,約為刎頸之交,情相款暱,言無回避,於座中輒論叛計,並相然許。時李孝質在禁,令驍果守之,中外交通,所謀益急。趙行樞者,樂人之子,家產巨萬,先交智及,勛侍楊士覽者,宇文甥,二人同告智及。智及素狂悖,聞之喜,即共見德戡,期以三月十五日舉兵同叛,劫十二衛武馬,虜掠居人財物,結黨西歸。智及曰:「不然。當今天實喪隋,英雄並起,同
 心叛者已數萬人,因行大事,此帝王業也。」德戡然之。行樞、薛良請以化及為主,相約既定,方告化及。化及性本駑怯,初聞大懼,色動流汗,久之乃定。



 義寧二年三月一日,德戡欲宣言告眾,恐以人心未一,更思譎詐以協驍果,謂許弘仁、張愷曰:「君是良醫,國家任使,出言惑眾,眾必信。君可入備身府,告識者,言陛下聞說驍果欲叛,多釀毒酒,因享會盡鴆殺之,獨與南人留此。」弘仁等宣布此言,驍果聞之,遞相告語,謀叛逾急。德戡知計既行,遂以十日總召故人,諭以所為。眾皆伏曰:「唯將軍命!」其夜,奉義主閉城門,乃與虔通相知,諸門皆不下鑰。至夜三
 更,德戡於東城內集兵,得數萬人,舉火與城外相應。帝聞有聲,問是何事。虔通偽曰:「草坊被燒,外人救火,故喧囂耳。」中外隔絕,帝以為然。



 孟秉、智及於城外得千餘人,劫候衛武賁馮普樂,共布兵分捉郭下街巷。至五更中,德戡授虔通兵,以換諸門衛士。虔通因自開門,領數百騎,至成象殿,殺將軍獨孤盛。武賁郎將元禮遂引兵進,突衛者皆走。虔通進兵,排左閣,馳入永巷,問:「陛下安在?」有美人出,方指云:「在西閣。」從往執帝。帝謂虔通曰:「卿非我故人乎!何恨而反?」虔通曰:「臣不敢反,但將士思歸,奉陛下還京師耳。」



 帝曰:「與汝歸。」虔通因勒兵守之。至旦,孟
 秉以甲騎迎化及。化及未知事果,戰慄不能言,人有來謁之者,但低頭據鞍,答云「罪過」。時士及在公主第,弗之知也。智及遣家僮莊桃樹就第殺之,桃樹不忍,執詣智及,久之乃見釋。化及至城門,德戡迎謁,引入朝堂,號為丞相。令將帝出江都門以示群賊,因復將入。遣令狐行達弒帝於宮中,又執朝臣不同己者數十人及諸外戚,無少長害之,唯留秦孝王子浩,立以為帝。十餘日,奪江都人舟楫,從水路西歸。至顯福宮,宿公麥孟才、折沖郎將沈光等謀擊化及,反為所害。化及於是入據六宮,其自奉養,一如煬帝故事。每於帳中南面端坐,人有白事
 者,默然不對。下牙時,方收取啟狀,共奉義、方裕、良、愷等參決之。行至徐州,水路不通,復奪人車牛,得二千兩,並載宮人珍寶。其戈甲戎器,悉令軍士負之。道遠疲極,三軍始怨。德戡失望,竊謂行樞曰:「君大謬誤我。當今撥亂,必藉英賢,化及庸暗,君小在側,事將必敗,當若之何?」



 行樞曰:「在我等爾,廢之何難!」因共李本、宇文導師、尹正卿等謀,以後軍萬餘兵襲殺化及,更立德戡為主。弘仁知之,密告化及,盡收捕德戡及其支黨十餘人,皆殺之。引兵向東郡,通守王軌以城降之。



 元文都推越王侗為主,拜李密為太尉,令擊化及。密遣徐勣據黎陽倉。化及渡
 河,保黎陽縣,分兵圍勣。密壁清淇,與勣以烽火相應。化及每攻倉,密輒引兵救之。化及數戰不利,其將軍於弘達為密所擒,送於侗所,鑊烹之。化及糧盡,渡永濟渠,與密決戰於童山,遂入汲郡求軍糧,又遣使拷掠東郡吏民以責米粟。王軌怨之,以城歸於李密。化及大懼,自汲郡將率眾圖以北諸州。其將陳智略率嶺南驍果萬餘人,張童兒率江東驍果數千人,皆叛歸李密。化及尚有眾二萬,北走魏縣。張愷等與其將陳伯謀去之,事覺,為化及所殺。腹心稍盡,兵勢日蹙,兄弟更無他計,但相聚酣宴,奏女樂。醉後,因尤智及曰:「我初不知,由汝為計,強
 來立我。今所向無成,士馬日散,負殺主之名,天下所不納。今者滅族,豈不由汝乎?」持其兩子而泣。智及怒曰:「事捷之日,都不賜尤,及其將敗,乃欲歸罪。何不殺我以降建德?」兄弟數相鬥鬩,言無長幼,醒而復飲,以此為恆。其眾多亡,自知必敗,化及嘆曰:「人生故當死,豈不一日為帝乎?」於是鴆殺浩,僭皇帝位於魏縣,國號許,建元為天壽,署置百官。攻元寶藏於魏州,四旬不克,反為所敗,亡失千餘人。乃東北趣聊城,將招攜海曲諸賊。時遣士及徇濟北,求饋餉。大唐遣淮安王神通安撫山東,並招化及。化及不從,神通進兵圍之,十餘日不克而退。竇建德
 悉眾攻之。先是,齊州賊帥王薄聞其多寶物,詐來投附。化及信之,與共居守。至是,薄引建德入城,生擒化及,悉虜其眾。先執智及、元武達、孟秉、楊士覽、許弘仁,皆斬之。乃以轞車載化及之河間,數以殺君之罪,並二子承基、承趾皆斬之,傳首於突厥義成公主,梟於虜庭。士及自濟北西歸長安。



 智及幼頑兇,好與人群鬥,所共游處,皆不逞之徒,相聚鬥雞,習放鷹狗。初以父功賜爵濮陽郡公。蒸淫醜穢,無所不為。其妻長孫,妒而告述,述雖為隱,而大忿之,纖芥之愆,必加鞭箠。弟士及恃尚主,又輕忽之。唯化及每事營護,父再三欲殺,輒救免之,由是頗相
 親暱。遂勸化及遣人入蕃,私為交易。事發,當誅,述獨證智及罪惡,而為化及請命。帝因兩釋。述將死,抗表言其兇勃,必且破家。



 帝後思述,授智及將作少監。其江都殺逆之事,智及之謀也,化及為丞相,以為左僕射,領十二衛大將軍。化及僭號,封齊王。竇建德破聊城,獲而斬之,並其黨十餘人,皆暴尸梟首。



 司馬德戡,扶風雍人也。父元謙,仕周為都督。德戡幼孤,以屠豕自給。有桑門釋粲,通德戡母和氏,遂撫教之,因解書計。開皇中,為侍官,漸遷至大都督。



 從楊素出討漢王諒,充內營左右,進止便僻,俊辯多奸計,素大善之。以
 勛授儀同三司。大業三年,為鷹揚郎將。從討遼左,進位正議大夫,遷武賁郎將。煬帝甚暱之。從至江都,領左右備身驍果萬人,營於城內。因隋末大亂,乃率驍果謀反,語在化及事中。既獲煬帝,與其黨孟秉等推化及為丞相。化及首封德戡為溫國公,邑三千戶,加光祿大夫,仍統本兵,化及意甚忌之。後數日,化及署諸將,分配士卒,乃以德戡為禮部尚書,外示美遷,實奪其兵也。由是憤怨,所獲賞物皆賂於智及,智及為之言。行至徐州,舍舟登陸,令德戡將後軍,乃與趙行樞、李本、尹正卿、宇文導師等謀襲化及,遣人使於孟海公,結為外助。遷延未發,
 以待使報。許弘仁、張愷知之,以告化及,因遣其弟士及陽為游獵,至於後軍。德戡不知事露,出營參謁,因命執之,並其黨與。化及責之曰:「與公戮力共定海內,出於萬死。今始事成,願得同守富貴,公又何為反也?」德戡曰:「本殺昏主,苦其毒害。推立足下,而又甚之。逼於物情,不獲已也。」化及不對,命送至幕下,縊而殺之,時年三十九。



 裴虔通,河東人也。初,煬帝為晉王,以親信從,稍遷至監門校尉。煬帝即位,擢舊左右,授宣惠尉,遷監門直閣。累從征役,至通議大夫。與司馬德戡同謀作亂,先開宮門,騎至成象殿,殺將軍獨孤盛,擒帝於西閣。化及以虔通
 為光祿大夫、莒國公。化及引兵之北也,令鎮徐州。化及敗後,歸於大唐,即授徐州總管,轉辰州刺史,封長蛇男。尋以隋朝殺逆之罪,除名,徙於嶺表而死。



 王世充段達王充,字行滿,本西域人也。祖支頹辱,徙居新豐。頹辱死,其妻少寡,與儀同王粲野合,生子曰瓊,粲遂納之以為小妻。其父收幼孤,隨母嫁粲,粲愛而養之,因姓王氏,官至懷、汴二州長史。充卷發豺聲,沉猜多詭詐,頗窺書傳,尤好兵法,曉龜策推步盈虛,然未嘗為人言也。開皇中,為左翊衛,後以軍功拜儀同,授兵部員外。善敷奏,明習
 法律,而舞弄文墨,高下其心。或有駁難之者,充利口飾非,辭義鋒起,眾雖知其不可而莫能屈,稱為明辯。煬帝時,累遷至江都郡丞。



 時帝數幸江都,充善候人主顏色,阿諛順旨,每入言事,帝善之。又以郡丞領江都宮監,乃雕飾池臺,陰奏遠方珍物以媚於帝,由是益暱之。



 大業八年,隋始亂,充內懷徼幸,卑身禮士,陰結豪俊,多收眾心。江淮間人素輕悍,又屬盜賊群起,人多犯法,有系獄抵罪者,充皆枉法出之,以樹私恩。及楊玄感反,吳人硃燮、晉陵人管崇起兵江南以應之,自稱將軍,擁眾十餘萬。帝遣將軍吐萬緒、魚俱羅討之,不能克。充募江都萬
 餘人,擊頻破之。每有克捷,必歸功於下,所獲軍實,皆推與士卒,身無所受。由此人爭為用,功最居多。十年,齊郡賊帥孟讓自長白山寇掠諸郡,至盱眙,有眾十餘萬。充以兵拒之,而羸師示弱,保都梁山為五柵,相持不戰。後因其懈弛,出兵奮擊,大破之,乘勝盡滅賊,讓以數十騎遁去,斬首萬人,六畜、軍資莫不盡獲。帝以充有將帥才略,始遣領兵,討諸小盜,所向皆破之。然性矯偽,詐為善,能自勤苦,以求聲譽。十一年,突厥圍帝於雁門,充盡發江都人,將往赴難。在軍中,反首垢面,悲泣無度,曉夜不解甲,藉草而臥。帝聞之,以為愛己,益信任之。



 十二年,遷為江
 都通守。時厭次人格謙為盜數年,兵十餘萬,在豆子中。充帥師破斬之,威振群賊。又擊盧明月,破之於南陽,斬首數萬,虜獲極多。後還江都,帝大悅,自執杯酒以賜之。時充又知帝好內,乃言江淮良家有美女,並願備後庭,無由自進。帝逾喜,因密令閱視諸女,姿質端麗合法相者,取正庫及應入京物以娉納之。所用不可勝計,帳上云敕別用,不顯其實。有合意者,則厚賞充;或不中者,又以賚之。後令以船送東京,而道路賊起,使者苦役,於淮泗中沉船溺之者,前後十數。或有發露,充為秘之,又遽簡閱以供進。是後益見親暱。



 遇李密攻陷興洛倉,進逼
 東都,官軍數卻,光祿大夫裴仁基以武牢降於密,帝惡之,大發兵,將討焉。發中詔遣充為將軍,於洛口以拒密,前後百餘戰,互有勝負。充乃引軍渡洛水,逼倉城。李密與戰,充敗績,赴水溺死者萬餘人。時天寒大雪,兵士既渡水,衣皆沾濕,在道凍死者又數萬人,比至河陽,才以千數。充自系獄請罪,越王侗遣使赦之,召令還都。收合亡散,復得萬餘人,屯於含嘉城中,不敢復出。



 宇文化及殺帝於江都,充與太府卿元文都、將軍皇甫無逸、右司郎盧楚奉侗為主。侗以充為吏部尚書,封鄭國公。及侗取元文都、盧楚之謀,拜李密為太尉、尚書令,密遂稱臣,
 復以兵拒化及於黎陽,遣使告捷。眾皆悅,充獨謂其麾下諸將曰:「文都之輩,刀筆吏耳。吾觀其勢,必為李密所擒。且吾軍人每與密戰,殺其父兄子弟,前後已多,一旦為之下,吾屬無類矣。」出此言以激怒其眾。文都知而大懼,與楚等謀,將因充入內,伏甲而殺之。期有日矣,將軍段達遣其女婿張志以楚謀告之。充夜勒兵圍宮城,將軍費曜、田世闍等與戰於東太陽門外。曜軍敗,充遂攻門而入,無逸以單騎遁走。獲楚,殺之。時宮門尚閉,充令扣門言於侗曰:「元文都等欲執皇帝降於李密,段達知而以告臣。臣非敢謀反,誅反者耳。」文都聞變入,奉侗於
 乾陽殿,陳兵衛之。令將帥乘城以拒難,兵敗,又獲文都殺之。侗命開門以納充,充悉遣人代宿衛者,乃入謁,頓首流涕而言曰:「文都等無狀,謀相屠害,事急為此,不敢背國。」侗與之盟。充尋遣韋節等諷侗,令拜為尚書左僕射、總督內外諸軍事。又授其兄惲為內史令,入居禁中。



 未幾,李密破化及還,其勁兵良馬多戰死,士卒皆倦。充欲乘其敝而擊之,恐人不一,乃假托鬼神,言夢見周公,乃立祠於洛水之上,遣巫宣言周公欲令僕射急討李密,當有大功,不則兵皆疫死。充兵多楚人,俗信妖妄,故出此言以惑之。眾皆請戰。充簡練精勇,得二萬餘人,馬
 千餘,遷營於洛水南。密軍偃師北山上。時密新得志於化及,有輕充之心,不設壁壘。充夜遣二百餘騎潛入北山,伏溪谷中,令軍秣馬蓐食。既而宵濟,人奔馬馳,遲明而薄密。密出兵應之,陣未成列而兩軍合戰,其伏兵蔽山而上,潛登北原,乘高下馳,壓密營。營中亂,無能拒者,即入縱火。密軍大驚而潰,降其將張童兒、陳智略,進下偃師。初,充兄偉及子玄應隨化及至東郡,密得而囚之於城中,至是,盡獲之。又執密長史邴元真妻子、司馬鄭虔象之母及諸將子弟,皆撫慰之,各令潛呼其父兄。兵次洛口,邴元真、鄭虔象等舉倉城以應之。密以數十騎
 遁逸,充悉收其眾。而東盡於海,南至於江,悉來歸附。



 充又令韋節諷侗,拜為太尉,署置官屬,以尚書省為其府。尋自稱鄭王。遣其將高略帥師攻壽安,不利而旋。又帥師攻圍穀州,三日而退。明年,自稱相國,受九錫備物,是後不朝侗矣。



 有道士桓法嗣者,自言解圖讖,充暱之。法嗣乃以《孔子閉房記》,畫作丈夫持一干以驅羊。法嗣云:「楊,隋姓也。乾一者,王字也。居羊後,明相國代隋為帝也。」又取莊子《人間世》、《德充符》二篇上之,法嗣釋曰:「上篇言世,下篇言充,此即相國名矣。明當德被人間,而應符命為天子也。」充大悅曰:「此天命也。」再拜受之。即以法嗣為
 諫議大夫。充又羅取雜鳥,書帛系其頸,自言符命而散放之。或有彈射得鳥而來獻者,亦拜官爵。既而廢侗於別宮,僭即皇帝位,建元曰開明,國號鄭。大唐遣秦王率眾圍之,充頻出兵,戰輒不利,都外諸城相繼降款。充窘迫,遣使請救於竇建德,建德率精兵援之。師至武牢,為秦王所破,擒建德以詣城下。充將潰圍而出,諸將莫有應之者,自知潛竄無所,於是出降。至長安,為仇人獨孤修德所殺。



 段達,武威姑臧人也。父嚴,周朔州刺史。達在周,年始三歲,襲爵襄垣縣公。



 及長,身長八尺,美須髯,便弓馬。高祖
 為丞相,以大都督領親信兵,常置左右。



 及踐阼,為左直齋,累遷車騎將軍,兼晉王參軍。高智惠、李積等之作亂也,達率眾一萬,擊定方、滁二州,賜縑千段,遷進儀同。又破汪文進等於宣州,加開府,賜奴婢五十口,綿絹四千段。仁壽初,太子左衛副率。大業初,以蕃邸之舊,拜左翊衛將軍。征吐谷渾,進位金紫光祿大夫。帝征遼東,百姓苦役,平原祁孝德、清河張金稱等並聚眾為群盜,攻陷城邑,郡縣不能御。帝令達擊之,數為金稱等所挫,亡失甚多。諸賊輕之,號為段姥。後用鄃令楊善會之計,更與賊戰,方致克捷。還京師,以公事坐免。明年,帝征遼東,以
 達留守涿郡。俄復拜左翊衛將軍。高陽魏刀兒聚眾十餘萬,自號歷山飛,寇掠燕趙。達率涿郡通守郭絢擊敗之。於時盜賊既多,官軍惡戰,達不能因機決勝,唯持重自守,頓兵饋糧,多無克獲,時皆謂之為怯芃。十二年,帝幸江都宮,詔達與太府卿元文都留守東都。李密據洛口,縱兵侵掠城下,達與監門郎將龐玉、武牙郎將霍舉率內兵出御之。頗有功,遷左驍衛大將軍。王充之敗也,密復進據北芒,來至上春門,達與判左丞郭文懿、尚書韋津出兵拒之。達見賊盜,不陣而走,為密所乘,軍大潰,津沒於陣。由是賊勢日盛。及帝崩於江都,達與元文都
 等推越王侗為主,署開府儀同三司,兼納言,封陳國公。元文都等謀誅王充也,達陰告充,為之內應。及事發,越王侗執文都於充,充甚德於達,特見崇重。既破李密,達等勸越王加充九錫備物,尋諷令禪讓。充僭尊號,以達為司徒。及東都平,坐誅,妻子籍沒。



 史臣曰:化及庸芃下才,負恩累葉,王充斗筲小器,遭逢時幸,俱蒙獎擢,禮越舊臣。既屬崩剝之期,不能致身竭命,乃因利乘便,先圖乾紀,率群不逞,職為亂階,拔本塞源,裂冠毀冕。或躬為戎首,或親行鴆毒,釁深指鹿,事切食蹯,天地所不容,人神所同憤。故梟獍兇魁,相尋菹戮,
 蛇豕醜類,繼踵誅夷,快忠義於當年,垂炯戒於來葉。嗚呼,為人臣者可不殷鑒哉!可不殷鑒哉!



\end{pinyinscope}