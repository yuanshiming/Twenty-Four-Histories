\article{卷六十一列傳第二十六}

\begin{pinyinscope}

 宇文述云定興宇文述,字伯通,代郡武川人也。本姓破野頭,役屬鮮卑俟豆歸,後從其主為宇文氏。父盛,周上柱國。述少驍銳,便弓馬。年十一時,有相者謂述曰:「公子善自愛,後當位極人臣。」周武帝時,以父軍功,起家拜開府。述性恭謹沈密,周大塚宰宇文護甚愛之,以本官領護親信。及帝親總萬機,召為左宮伯,累遷英果中大夫,賜爵博陵郡公,
 尋改封濮陽郡公。



 高祖為丞相,尉迥作亂相州,述以行軍總管率步騎三千,從韋孝寬擊之。軍至河陽,迥遣將李俊攻懷州,述別擊俊軍,破之。又與諸將擊尉惇於永橋,述先鋒陷陣,俘馘甚眾。平尉迥,每戰有功,超拜上柱國,進爵褒國公,賜縑三千匹。開皇初,拜右衛大將軍。平陳之役,復以行軍總管率眾三萬,自六合而濟。時韓擒、賀若弼兩軍趣丹陽,述進據石頭,以為聲援。陳主既擒,而蕭瓛、蕭巖據東吳之地,擁兵拒守。述領行軍總管元契、張默言等討之,水陸兼進。落叢公燕榮以舟師自海至,亦受述節度。上下詔曰:「公鴻勛大業,名高望重,奉國
 之誠,久所知悉。金陵之寇,既已清蕩,而吳會之地,東路為遙,蕭巖、蕭瓛,並在其處。公率將戎旅,撫慰彼方,振揚國威,宣布朝化。以公明略,乘勝而往,風行電掃,自當稽服。若使干戈不用,黎庶獲安,方副朕懷,公之力也。」陳永新侯陳君範自晉陵奔瓛,並軍合勢。見述軍且至,瓛懼,立柵於晉陵城東,又絕塘道,留兵拒述。瓛自義興入太湖,圖掩述後。述進破其柵,回兵擊瓛,大敗之,斬瓛司馬曹勒叉。前軍復陷吳州,瓛以餘眾保包山,燕榮擊破之。述進至奉公埭,蕭巖、陳君範等以會稽請降。



 述許之,二人面縛路左,吳會悉平。以功拜一子開府,賜物三千段,
 拜安州總管。



 時晉王廣鎮揚州,甚善於述,欲述近己,因奏為壽州刺史總管。王時陰有奪宗之志,請計於述,述曰:「皇太子失愛已久,令德不聞於天下。大王仁孝著稱,才能蓋世,數經將領,深有大功。主上之與內宮,咸所鐘愛,四海之望,實歸於大王。



 然廢立者,國家之大事,處人父子骨肉之間,誠非易謀也。然能移主上者,唯楊素耳。素之謀者,唯其弟約。述雅知約,請朝京師,與約相見,共圖廢立。」晉王大悅,多齎金寶,資述入關。述數請約,盛陳器玩,與之酣暢,因而共博,每佯不勝,所齎金寶盡輸之。約所得既多,稍以謝述。述因曰:「此晉王之賜,令述與公
 為歡樂耳。」約大驚曰:「何為者?」述因為王申意。約然其說,退言於素,素亦從之。



 於是素每與述謀事。晉王與述情好益密,命述子士及尚南陽公主,前後賞賜不可勝計。及晉王為皇太子,以述為左衛率。舊令,率官第四品,上以述素貴,遂進率品為第三,其見重如此。



 煬帝嗣位,拜左衛大將軍,改封許國公。大業三年,加開府儀同三司,每冬正朝會,輒給鼓吹一部。從幸榆林,時鐵勒契弊歌棱攻敗吐谷渾,其部攜散,遂遣使請降求救。帝令述以兵屯西平之臨羌城,撫納降附。吐谷渾見述擁強兵,懼不敢降,遂西遁。述領鷹揚郎將梁元禮、張峻、崔師等追
 之,至曼頭城,攻拔之,斬三千餘級。乘勝至赤水城,復拔之。其餘黨走屯丘尼川,述進擊,大破之,獲其王公、尚書、將軍二百人,前後虜男女四千口而還。渾主南走雪山,其故地皆空。帝大悅。



 明年,從帝西幸,巡至金山,登燕支,述每為斥候。時渾賊復寇張掖,進擊走之。



 還至江都宮,敕述與蘇威常典選舉,參預朝政。述時貴重,委任與蘇威等,其親愛則過之。帝所得遠方貢獻及四時口味,輒見班賜,中使相望於道。述善於供奉,俯仰折旋,容止便闢,宿衛者咸取則焉。又有巧思,凡有所裝飾,皆出人意表。數以奇服異物進獻宮掖,由是帝彌悅焉。時述貴幸,
 言無不從,勢傾朝廷。左衛將軍張瑾與述連官,嘗有評議,偶不中意,述張目叱之,瑾惶懼而走,文武百僚莫敢違忤。



 然性貪鄙,知人有珍異之物,必求取之。富商大賈及隴右諸胡子弟,述皆接以恩意,呼之為兒。由是競加饋遺,金寶累積。後庭曳羅綺者數百,家僮千餘人,皆控良馬,被服金玉。述之寵遇,當時莫與為比。



 及征高麗,述為扶餘道軍將。臨發,帝謂述曰:「禮,七十者行役以婦人從,公宜以家累自隨。古稱婦人不入軍,謂臨戰時耳。至於營壘之間,無所傷也。項籍虞姬,即其故事。」述與九軍至鴨綠水,糧盡,議欲班師。諸將多異同,述又不測帝意。
 會乙支文德來詣其營,述先與於仲文俱奉密旨,令誘執文德。既而緩縱,文德逃歸,語在《仲文傳》。述內不自安,遂與諸將渡水追之。時文德見述軍中多饑色,欲疲述眾,每鬥便北。述一日之中七戰皆捷,既恃驟勝,又內逼群議,於是遂進,東濟薩水,去平壤城三十里,因山為營。文德復遣使偽降,請述曰:「若旋師者,當奉高元朝行在所。」述見士卒疲敝,不可復戰,又平壤險固,卒難致力,遂因其詐而還。眾半濟,賊擊後軍,於是大潰,不可禁止,九軍敗績,一日一夜,還至鴨綠水,行四百五十里。初,渡遼九軍三十萬五千人,及還至遼東城,唯二千七百人。帝
 大怒,以述等屬吏。至東都,除名為民。明年,帝有事遼東,復述官爵,待之如初。從至遼東,與將軍楊義臣率兵復臨鴨綠水。會楊玄感作亂,帝召述班師,令馳驛赴河陽,發諸郡兵以討玄感。時玄感逼東都,聞述軍將至,懼而西遁,將圖關中。述與刑部尚書衛玄、左御衛將軍來護兒、武衛將軍屈突通等躡之。至閿鄉皇天原,與玄感相及。述與來護兒列陣當其前,遣屈突通以奇兵擊其後,大破之,遂斬玄感,傳首行在所。賜物數千段。復從東征,至懷遠而還。



 突厥之圍雁門,帝懼,述請潰圍而出。樊子蓋固諫不可,帝乃止。及圍解,車駕次太原,議者多勸帝
 還京師,帝有難色。述因奏曰:「從官妻子多在東都,便道向洛陽,自潼關而入可也。」帝從之。是歲,至東都,述又觀望帝意,勸幸江都,帝大悅。述於江都遇疾,中使相望,帝將親臨視之,群臣苦諫乃止。遂遣司宮魏氏問述曰:「必有不諱,欲何所言?」述二子化及、智及,時並得罪於家,述因奏曰:「化及臣之長子,早預籓邸,願陛下哀憐之。」帝聞,泫然曰:「吾不忘也。」及薨,帝為之廢朝,贈司徒、尚書令、十郡太守,班劍四十人,轀京車,前後部鼓吹,謚曰恭,帝令黃門侍郎裴矩祭以太牢,鴻臚監護喪事。子化及,別有傳。



 雲定興者,附會於述。初,定興女為皇太子勇昭訓,及勇廢,除名配少府。定興先得昭訓明珠絡帳,私賂於述,自是數共交游。定興每時節必有賂遺,並以音樂乾述。述素好著奇服,炫耀時人。定興為制馬韉,於後角上缺方三寸,以露白色。



 世輕薄者爭放學之,謂為許公缺勢。又遇天寒,定興曰:「入內宿衛,必當耳冷。」



 述曰:「然。」乃制裌頭巾,令深袙耳。又學之,名為許公袙勢。述大悅曰:「雲兄所作,必能變俗。我聞作事可法,故不虛也。」後帝將事四夷,大造兵器,述薦之,因敕少府工匠並取其節度。述欲為之求官,謂定興曰:「兄所制器仗並合上心,而不得官者,
 為長寧兄弟猶未死耳。」定興曰:「此無用物,何不勸上殺之。」述因奏曰:「房陵諸子,年並成立。今欲動兵征討,若將從駕,則守掌為難;若留一處,又恐不可。進退無用,請早處分。」帝從之,因鴆殺長寧,又遣以下七弟分配嶺表,仍遣間使於路盡殺之。五年,大閱軍實,帝稱甲仗為佳。述奏曰:」並云定興之功也。」擢授少府丞。尋代何稠為少監,轉衛尉少卿,遷左御衛將軍,仍知少府事。十一年,授左屯衛大將軍。



 凡述所薦達,皆至大官。趙行樞以太常樂戶,家財億計,述謂為兄,多受其賄。



 稱其驍勇,起家為折沖郎將。



 郭衍郭衍,字彥文,自云太原介休人也。父以舍人從魏武帝入關,其後官至侍中。



 衍少驍武,善騎射。周陳王純引為左右,累遷大都督。時齊氏未平,衍奉詔於天水募人,以鎮東境,得樂徙千餘家,屯於陜城。拜使持節、車騎大將軍、儀同三司。



 每有寇至,輒率所領御之,一歲數告捷,頗為齊人所憚。王益親任之。建德中,周武帝出幸雲陽,衍朝於行所,時議欲伐齊,衍請為前鋒。攻河陰城,授儀同大將軍。



 武帝圍晉州,慮齊兵來援,令衍從陳王守千里徑。又從武帝與齊主大戰於晉州,追齊師至高壁,敗之。
 仍從平並州,以功加授開府,封武強縣公,邑一千二百戶,賜姓叱羅氏。宣政元年,為右中軍熊渠中大夫。



 尉迥之起逆,從韋孝寬戰於武陟,進戰於相州。先是,迥遣弟子勤為青州總管,率青、齊之眾來助迥。迥敗,勤與迥子惇、祐等欲東奔青州。衍將精騎一千追破之,執祐於陣,勤遂遁走,而惇亦逃逸。衍至濟州,入據其城,又擊其餘黨於濟北,累戰破之,執送京師。超授上柱國,封武山郡公。賞物七千段。密勸高祖殺周室諸王,早行禪代。由是大被親暱。開皇元年,敕復舊姓為郭氏。突厥犯塞,以衍為行軍總管,領兵屯於平涼。數歲,虜不入。徵為開漕渠
 大監。部率水工,鑿渠引渭水,經大興城北,東至於潼關,漕運四百餘里。關內賴之,名之曰富民渠。五年,授瀛州刺史。遇秋霖大水,其屬縣多漂沒,民皆上高樹,依大家。衍親備船伐,並齎糧拯救之,民多獲濟。衍先開倉賑恤,後始聞奏。上大善之,選授朔州總管。所部有恆安鎮,北接蕃境,常勞轉運。衍乃選沃饒地,置屯田,歲剩粟萬餘石,民免轉輸之勞。又築桑乾鎮,皆稱旨。十年,從晉王廣出鎮揚州。遇江表構逆,命衍為總管,領精銳萬人先屯京口。於貴洲南與賊戰,敗之,生擒魁帥,大獲舟楫糧儲,以充軍實,乃討東陽、永嘉、宣城、黟、歙諸洞,盡平之。授蔣
 州刺史。



 衍臨下甚踞,事上奸諂。晉王愛暱之,宴賜隆厚。遷洪州總管。王有奪宗之謀,托衍心腹,遣宇文述以情告之。衍大喜曰:「若所謀事果,自可為皇太子。如其不諧,亦須據淮海,復梁、陳之舊。副君酒客,其如我何?」王因召衍,陰共計議。



 又恐人疑無故來往,托以衍妻患癭,王妃蕭氏有術能療之。以狀奏高祖,高祖聽衍共妻向江都,往來無度。衍又詐稱桂州俚反,王乃奏衍行兵討之。由是大修甲仗,陰養士卒。及王入為太子,徵授左監門率,轉左宗衛率。高祖於仁壽宮將大漸,太子與楊素矯詔,令衍、宇文述領東宮兵,帖上臺宿衛,門禁並由之。及上崩,
 漢王起逆,而京師空虛,使衍馳還,總兵居守。大業元年,拜左武衛大將軍。帝幸江都,令衍統左軍,改授光祿大夫。又從討吐谷渾,出金山道,納絳二萬餘戶。衍能揣上意,阿諛順旨。帝每謂人曰:「唯有郭衍,心與朕同。」又嘗勸帝取樂,五日一視事,無得效高祖空自劬勞。帝從之,益稱其孝順。初,新令行,衍封爵從例除。六年,以恩幸封真定侯。七年,從往江都,卒。贈左衛大將軍,賵賜甚厚,謚曰襄。



 長子臻,武牙郎將。次子嗣本,孝昌縣令。



 史臣曰:謇謇匪躬,為臣之高節,和而不同,事君之常道。宇文述、郭衍以水濟水,如脂如韋,便闢足恭,柔顏取悅。
 君所謂可,亦曰可焉,君所謂不,亦曰不焉。無所是非,不能輕重,默默茍容,偷安高位,甘素餐之責,受彼己之譏。此固君子所不為,亦丘明之深恥也。



\end{pinyinscope}