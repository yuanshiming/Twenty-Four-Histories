\article{卷三十八列傳第三 劉昉}

\begin{pinyinscope}

 劉昉,博陵望都人也。父孟良,大司農。從魏武入關,周太祖以為東梁州刺史。



 昉性輕狡,有奸數。周武帝時,以功臣子入侍皇太子。及宣帝嗣位,以技佞見狎,出入宮掖,寵冠一時。授大都督,遷小御正,與御正中大夫顏之儀並見親信。及帝不悆,召方及之儀俱入臥內,屬以後事。帝喑不復能言。昉見靜帝幼沖,不堪負荷。



 然昉素知高
 祖,又以後父之故,有重名於天下,遂與鄭譯謀,引高祖輔政。高祖固讓,不敢當。昉曰:「公若為,當速為之;如不為,昉自為也。」高祖乃從之。



 及高祖為承相,以昉為司馬。時宣帝弟漢王贊居禁中,每與高祖同帳而坐。昉飾美妓進於贊,贊甚悅之。昉因說贊曰:「大王先帝之弟,時望所歸。孺子幼沖,豈堪大事!今先帝初崩,群情尚擾,王且歸第。待事寧之後,入為天子,此萬全之計也。」贊時年未弱冠,性識庸下,聞昉之說,以為信然,遂從之。高祖以昉有定策之功,拜下大將軍,封黃國公,與沛國公鄭譯皆為心膂。前後賞賜巨萬,出入以甲士自衛,朝野傾矚,稱為
 黃、沛。時人為之語曰:「劉昉牽前,鄭譯推後。」昉自恃其功,頗有驕色。然性粗疏,溺於財利,富商大賈,朝夕盈門。



 於時尉迥起兵,高祖令韋孝寬討之。至武陟,諸將不一。高祖欲遣昉、譯一人往監軍,因謂之曰:「須得心膂以統大軍,公等兩人,誰當行者?」昉自言未嘗為將,譯又以母老為請,高祖不怪。而高熲請行,遂遣之。由是恩禮漸薄。又王謙、司馬消難相繼而反,高祖憂之,忘寢與食。昉逸游縱酒,不以職司為意,相府事物,多所遺落。高祖深銜之,以高熲代為司馬。是後益見疏忌。及受禪,進位柱國,改封舒國公,閑居無事,不復任使。昉自以佐命元功,中被
 疏遠,甚不自安。後遇京師饑,上令禁酒,昉使妾賃屋,當壚沽酒。治書侍御史梁毗劾奏昉曰:「臣聞處貴則戒之以奢,持滿則守之以約。昉既位列群公,秩高庶尹,縻爵稍久,厚祿已淹,正當戒滿歸盈,鑒斯止足,何乃規曲蘗之潤,競錐刀之末,身暱酒徒,家為逋藪?



 若不糾繩,何以肅厲!」有詔不治。昉鬱鬱不得志。時柱國梁士彥、宇文忻俱失職忿望,昉並與之交,數相來往。士彥妻有美色,昉因與私通,士彥不之知也,情好彌協,遂相與謀反,許推士彥為帝。後事洩,上窮治之。昉自知不免,默無所對。



 下詔誅之,曰:朕君臨四海,慈愛為心。加以起自布衣,入升
 皇極,公卿之內,非親則友,位雖差等,情皆舊人。護短全長,恆思覆育,每殷勤戒約,言無不盡。天之歷數,定於杳冥,豈慮苞藏之心,能為國家之害?欲使其長守富貴,不觸刑書故也。上柱國、郕國公梁士彥,上柱國、巳國公宇文忻,柱國、舒國公劉昉等,朕受命之初,並展勤力,酬勛報效,榮高祿重。待之既厚,愛之實隆,朝夕宴言,備知朕意。但心如溪壑,志等豺狼,不荷朝恩,忽謀逆亂。士彥爰始幼來,恆自誣罔,稱有相者,云其應籙,年過六十,必據九五。初平尉迥,暫臨相州,已有反心,彰於行路。朕即遣人代之,不聲其罪。入京之後,逆意轉深。忻、昉之徒,言相
 扶助。士彥許率僮僕,克期不遠,欲於蒲州起事,即斷河橋,捉黎陽之關,塞河陽之路,劫調布以為牟甲,募盜賊而為戰士,就食之人,亦云易集。輕忽朝廷,嗤笑官人,自謂一朝奮發,無人當者。其第二子剛,每常苦諫,第三子叔諧,固深勸獎。朕既聞知,猶恐枉濫,乃授晉部之任,欲驗蒲州之情。士彥得以欣然,云是天贊,忻及昉等,皆賀時來。忻往定鄴城,自矜不已,位極人臣,猶恨賞薄。云我欲反,何慮不成。怒色忿言,所在流布。朕深念其功,不計其禮,任以武候,授以領軍,寄之爪牙,委之心腹。忻密為異計,樹黨宮闈,多奏親友,入參宿衛。朕推心待物,言刻
 依許。



 為而弗止,心跡漸彰,仍解禁兵,令其改悔。而志規不逞,愈結於懷,乃與士彥情意偏厚,要請神明,誓不負約。俱營賊逆,逢則交謀,委彥河東,自許關右,蒲津之事,即望從征,兩軍結東西之旅,一舉合連橫之勢,然後北破晉陽,還圖宗社。



 昉入佐相府,便為非法,三度事發,二度其婦自論。常云姓是「卯金刀」,名是「一萬日」,劉氏應王,為萬日天子。朕訓之導之,示其利害,每加寬宥,望其修改。口請自新,志存如舊,亦與士彥情好深重,逆節奸心,盡探肝鬲。嘗共士彥論太白所犯,問東井之間,思秦地之亂,訪軒轅之里,願宮掖之災。唯待蒲阪事興,欲在關
 內應接。殘賊之策,千端萬緒。惟忻及昉,名位並高,寧肯北面曲躬,臣於士彥,乃是各懷不遜,圖成亂階,一得擾攘之基,方逞吞並之事。人之奸詐,一至於此!雖國有常刑,罪在不赦,朕載思草創,咸著厥誠,情用愍然,未忍極法。士彥、忻、昉,身為謀首,叔諧贊成父意,義實難容,並已處盡。士彥、忻、昉兄弟叔侄,特恕其命,有官者除名。士彥小男女、忻母妻女及小男並放。士彥、叔諧妻妾及資財田宅,忻、昉妻妾及資財田宅,悉沒官。士彥、昉兒年十五以上遠配。上儀同薛摩兒,是士彥交舊,上柱國府戶曹參軍事裴石達,是士彥府僚,反狀逆心,巨細皆委。薛摩
 兒聞語,仍相應和,俱不申陳,宜從大闢。問即承引,頗是恕心,可除名免死。朕握圖當籙,六載於斯,政事徒勤,淳化未洽,興言軫念,良深嘆憤!



 臨刑,至朝堂,宇文忻見高熲,向之叩頭求哀。昉勃然謂忻曰:「事形如此,何叩頭之有!」於是伏誅,籍沒其家。後數日,上素服臨射殿,盡取昉敢、忻、士彥三家資物置於前,令百僚射取之,以為鑒誡云。



 鄭譯鄭譯,字正義,滎陽開封人也。祖瓊,魏太常。父道邕,魏司空。譯頗有學識,兼知鐘律,善騎射。譯從祖開府文寬,尚魏平陽公主,則周太祖元後之妹也。主無子,太祖令譯
 後之。由是譯少為太祖所親,恆令與諸子游集。年十餘歲,嘗詣相府司錄李長宗,長宗於眾中戲之。譯斂容謂長宗曰:「明公位望不輕,瞻仰斯屬,輒相玩狎,無乃喪德也。」長宗甚異之。文寬後誕二子,譯復歸本生。



 周武帝時,起家給事中士,拜銀青光祿大夫,轉左侍上士。與儀同劉昉恆侍帝側。譯時喪妻,帝命譯尚梁安固公主。及帝親總萬機,以為御正下大夫,俄轉太子宮尹。時太子多失德,內史中大夫烏丸軌每勸帝廢太子而立秦王,由是太子恆不自安。其後詔太子西征吐谷渾,太子乃陰謂譯曰:「秦王,上愛子也。烏丸軌,上信臣也。今吾此行,得
 無扶蘇之事乎?」譯曰:「願殿下勉著仁孝,無失子道而已。



 勿為他慮。」太子然之。既破賊,譯以功最,賜爵開國子,邑三百戶。後坐褻狎皇太子,帝大怒,除名為民。太子復召之,譯戲狎如初。因言於太子曰:「殿下何時可得據天下?」太子悅而益暱之。及帝崩,太子嗣位,是為宣帝。超拜開府、內史下大夫、封歸昌縣公,邑一千戶,委以朝政。俄遷內史上大夫,進封沛國公,邑五千戶,以其子善願為歸昌公,元琮為永安縣男,又監國史。譯頗專權,時帝幸東京,譯擅取官材,自營私第,坐是復除名為民。劉昉數言於帝,帝復召之,顧待如初。



 詔領內史事。



 初,高祖與譯有
 同學之舊,譯又素知高祖相表有奇,傾心相結。至是,高祖為宣帝所忌,情不自安,嘗在永巷私於譯曰:「久願出籓,公所悉也。敢布心腹,少留意焉。」譯曰:「以公德望,天下歸心,欲求多福,豈敢忘也。謹即言之。」時將遣譯南征,譯請元帥。帝曰:「卿意如何?」譯對曰:「若定江東,自非懿戚重臣無以鎮撫。可令隋公行,且為壽陽總管以督軍事。」帝從之。乃下詔以高祖為揚州總管,譯發兵俱會壽陽以伐陳。行有日矣,帝不悆,遂與御正下大夫劉昉謀,引高祖入受顧托。既而譯宣詔,文武百官皆受高祖節度。時御正中大夫顏之儀與宦者謀,引大將軍宇文仲輔政。
 仲已至御坐,譯知之,遽率開府楊惠及劉昉、皇甫績、柳裘俱入。仲與之儀見譯等,愕然,逡巡欲出,高祖因執之。於是矯詔復以譯為內史上大夫。明日,高祖為丞相,拜譯柱國、相府長史、治內史上大夫事。及高祖為大塚宰,總百揆,以譯兼領天官都府司會,總六府事。出入臥內,言無不從,賞賜玉帛不可勝計。每出入,以甲士從。拜其子元璹為儀同。時尉迥、王謙、司馬消難等作亂,高祖逾加親禮。俄而進位上柱國,恕以十死。



 譯性輕險,不親職務,而臟貨狼籍。高祖陰疏之,然以其有定策功,不忍廢放,陰敕官屬不得白事於譯。譯猶坐事,無所關預。譯
 懼,頓首求解職,高祖寬諭之,接以恩禮。及上受禪,以上柱國公歸第,賞賜豐厚。進子元璹爵城皋郡公,邑二千戶,元洵永安男。追贈其父及亡兄二人並為刺史。譯自以被疏,陰呼道士章醮以祈福助,其婢奏譯厭蠱左道。上謂譯曰:「我不負公,此何意也?」譯無以對。譯又與母別居,為憲司所劾,由是除名。下詔曰:「譯嘉謀良策,寂爾無聞,鬻獄賣官,沸騰盈耳。若留之於世,在人為不道之臣,戮之於朝,入地為不孝之鬼。有累幽顯,無以置之,宜賜以《孝經》,令其熟讀。」仍遣與母共居。



 未幾,詔譯參撰律令,復授開府、隆州刺史。請還治疾,有詔征之,見於醴泉宮。
 上賜宴甚歡,因謂譯曰:「貶退已久,情相矜愍。」於是復爵沛國公,位上柱國。上顧謂侍臣曰:「鄭譯與朕同生共死,間關危難,興言念此,何日忘之!」譯因奉觴上壽。上令內史令李德林立作詔書,高熲戲謂譯曰:「筆幹。」譯答曰:「出為方岳,杖策言歸,不得一錢,何以潤筆。」上大笑。未幾,詔譯參議樂事。



 譯以周代七聲廢缺,自大隋受命,禮樂宜新,更修七始之義,名曰《樂府聲調》,凡八篇。奏之,上嘉美焉。俄遷岐州刺史。在職歲餘,復奉詔定樂於太常,前後所論樂事,語在《音律志》。上勞譯曰:「律令則公定之,音樂則公正之。禮樂律令,公居其三,良足美也。」於是還岐州。
 開皇十一年,以疾卒官,時年五十二,上遣使吊祭焉。謚曰達。子元璹嗣。煬帝初立,五等悉除,以譯佐命元功,詔追改封譯莘公,以元璹襲。



 元璹初為驃騎將軍,後轉武賁郎將,數以軍功進位右光祿大夫,遷右候衛將軍。



 大業末,出為文城太守。及義兵起,義將張倫略地至文城,元璹以城歸之。



 柳裘柳裘,字茂和,河東解人,齊司空世隆之曾孫也。祖惔,梁尚書左僕射。父明,太子舍人、義興太守。裘少聰慧,弱冠有令名,在梁仕歷尚書郎、駙馬都尉。梁元帝為魏軍所
 逼,遣裘請和於魏。俄而江陵陷,遂入關中。周明、武間,自麟趾學士累遷太子侍讀,封昌樂縣侯。後除天官府都上士。宣帝即位,拜儀同三司,進爵為公,轉御飾大夫。及帝不悆,留侍禁中,與劉昉、韋鷿、皇甫績同謀,引高祖入總萬機。高祖固讓不許。裘進曰:「時不可再,機不可失,今事已然,宜早定大計。



 天與不取,反受其咎,如更遷延,恐貽後悔。」高祖從之。進位上開府,拜內史大夫,委以機密。及尉迥作亂,天下騷動,並州總管李穆頗懷猶豫,高祖令裘往喻之。



 裘見穆,盛陳利害,穆甚悅,遂歸心於高祖。後以奉使功,賜彩三百匹,金九環帶一腰。時司馬消難
 阻兵安陸,又令喻之,未到而消難奔陳。高祖即令裘隨便安集淮南,賜馬及雜物。開皇元年,進位大將軍,拜許州刺史。在官清簡,吏民懷之。復轉曹州刺史。其後上思裘定策功,欲加榮秩,將征之,顧問朝臣曰:「曹州刺史何當入朝?」或對曰:「即今冬也。」帝乃止。裘尋卒,高祖傷惜者久之,謚曰安。



 子惠童嗣。



 皇甫績韋紘皇甫績,字功明,安定朝那人也。祖穆,魏隴東太守。父道,周湖州刺史、雍州都督。績三歲而孤,為外祖韋孝寬所鞠養。嘗與諸外兄博奕,孝寬以其惰業,督以嚴訓,愍績
 孤幼,特舍之。績嘆曰:「我無庭訓,養於外氏,不能克躬勵己,何以成立?」深自感激,命左右自杖三十。孝寬聞而對之流涕。於是精心好學,略涉經史。周武帝為魯公時,引為侍讀。建德初,轉宮尹中士。武帝嘗避暑雲陽宮,時宣帝為太子監國。衛剌王作亂,城門已閉,百僚多有遁者。績聞難赴之,於玄武門遇皇太子,太子下樓執績手,悲喜交集。帝聞而嘉之,遷小宮尹。宣政初,錄前後功,封義陽縣男,拜畿伯下大夫,累轉御正下大夫。宣帝崩,高祖總己,績有力焉,語在《鄭譯傳》。加位上開府,轉內史中大夫,進封郡公,邑千戶。尋拜大將軍。



 開皇元年,出為豫州
 刺史,增邑通前二千五百戶。尋拜都官尚書。後數載,轉晉州刺史,將之官,稽首而言曰:「臣實庸鄙,無益於國,每思犯難以報國恩。今偽陳尚存,以臣度之,有三可滅。」上問其故,『績答曰:「大吞小,一也;以有道伐無道,二也;納叛臣蕭巖,於我有詞,三也。陛下若命鷹揚之將,臣請預戎行,展絲發之效。」上嘉其壯志,勞而遣之。及陳平,拜蘇州刺史。



 高智慧等作亂江南,州民顧子元發兵應之,因以攻績,相持八旬。子元素感績恩,於冬至日遣使奉牛酒。績遺子元書曰:「皇帝握符受籙,合極通靈,受揖讓於唐、虞,棄干戈於湯、武。東逾蟠木,方朔所未窮西盡流沙,張
 騫所不至。玄漠黃龍之外,交臂來王;蔥嶺、榆關之表,屈膝請吏。曩者偽陳獨阻聲教,江東士民困於荼毒。皇天輔仁,假手朝廷,聊申薄伐,應時瓦解。金陵百姓,死而復生,吳、會臣民,白骨還肉。唯當懷音感德,行歌擊壤,豈宜自同吠主,翻成反噬。卿非吾民,何須酒禮?吾是隋將,何容外交?易子析骸,未能相告,況是足食足兵,高城深塹,坐待強援,綽有餘力。何勞踵輕敝之俗,作虛偽之辭,欲阻誠臣之心,徒惑驍雄之志。以此見期,必不可得。卿宜善思活路,曉諭黎元,能早改迷,失道非遠。」



 子元得書,於城下頓首陳謝。楊素援兵至,合擊破之。拜信州總管、十
 二州諸軍事。



 俄以病乞骸骨,詔徵還京,賜以御藥,中使相望,顧問不絕。卒於家,時年五十二。



 謚曰安。子人思嗣。大業之世,官至尚書主爵郎。



 韋鷿者,京兆人也。仕周內史大夫。高祖以鷿有定策之功,累遷上柱國,封普安郡公。開皇初,卒於蒲州刺史。



 盧賁盧賁,字子徵,涿郡範陽人也。父光,周開府、燕郡公。賁略涉書記,頗解鐘律。周武帝時,襲爵燕郡公,邑一千九百戶。後歷魯陽太守、太子小宮尹、儀同三司。平齊有功,增邑四百戶,轉司武上士。時高祖為大司武,賁知高祖為
 非常人,深自推結。宣帝嗣位,加開府。



 及高祖初被顧托,群情未一,乃引賁置於左右。高祖將之東第,百官皆不知所去。高祖潛令賁部伍仗衛,因召公卿而謂曰:「欲求富貴者,當相隨來。」往往偶語,欲有去就。賁嚴兵而至,眾莫敢動。出崇陽門,至東宮,門者拒不內。賁諭之,不去,瞋目叱之,門者遂卻。既而高祖得入。賁恆典宿衛,後承問,進說曰:「周歷已盡,天人之望,實歸明公,願早應天順民也。天與不取,反受其咎。」高祖甚然之。及受禪,命賁清宮,因典宿衛。賁於是奏改周代旗幟,更為嘉名。其青龍、騶虞、硃雀、玄武、千秋、萬歲之旗,皆賁所創也。尋拜散騎常
 侍,兼太子左庶子、左領軍、右將軍。



 時高熲、蘇威共掌朝政,賁甚不平之。柱國劉昉時被疏忌,賁因諷昉及上柱國元諧、李詢、華州刺史張賓等,謀黜熲、威,五人相與輔政。又以晉王上之愛子,謀行廢立。復私謂皇太子曰:「賁將數謁殿下,恐為上所譴,願察區區之心。」謀洩,上窮治其事。昉等委罪於賓、賁,公卿奏二人坐當死。上以龍潛之舊,不忍加誅,並除名為民。賓未幾卒。



 歲餘,賁復爵位,檢校太常卿。賁以古樂宮懸七八,損益不同,歷代通儒,議無定準,於是上表曰:「殷人以上,通用五音,周武克殷,得鶉火、天駟之應,其音用七。漢興,加應鐘,故十六枚而
 在一虡。鄭玄注《周禮》,二八十六為虡。此則七八之義,其來遠矣。然世有沿革,用舍不同,至周武帝,復改懸七,以林鐘為宮。夫樂者,治之本也,故移風易俗,莫善於樂,是以吳札觀而辯興亡。然則樂也者,所以動天地,感鬼神,情發於聲,治亂斯應。周武以林鐘為宮,蓋將亡之徵也。



 且林鐘之管,即黃鐘下生之義。黃鐘,君也,而生於臣,明為皇家九五之應。又陰者臣也,而居君位,更顯國家登極之祥。斯實冥數相符,非關人事。伏惟陛下握圖禦宇,道邁前王,功成作樂,煥乎曩策。臣聞五帝不相沿樂,三王不相襲禮,此蓋隨時改制,而不失雅正者也。」上竟從之,
 即改七懸八,以黃鐘為宮。詔賁與儀同楊慶和刪定周、齊音律。



 未幾,拜郢州刺史,尋轉虢州刺史。後遷懷州刺史,決沁水東注,名曰利民渠,又派入溫縣,名曰溫潤渠,以溉舄鹵,民賴其利。後數年,轉齊州刺史。民饑,穀米踴貴,閉人糶而自糶之。坐是除名為民。



 後從幸洛陽,上從容謂賁曰:「我始為大司馬時,卿以布腹心於我。及總百揆,頻繁左右,與卿足為恩舊。卿若無過者,位與高熲齊。坐與兇人交構,由是廢黜。



 言念疇昔之恩,復當牧伯之位,何乃不思報效,以至於此!吾不忍殺卿,是屈法申私耳。」賁俯伏陳謝,詔復本官。後數日,對詔失旨,又自敘功
 績,有怨言。上大怒,顧謂群臣曰:「吾將與賁一州,觀此不可復用。」後皇太子為其言曰:「此輩並有佐命之功,雖性行輕險,誠不可棄。」上曰:「我抑屈之,全其命也。微劉昉、鄭譯及賁、柳裘、皇甫績等,則我不至此。然此等皆反覆子也。當周宣帝時,以無賴得幸,及帝大漸,顏之儀等請以宗王輔政,此輩行詐,顧命於我。我將為治,又欲亂之。故昉謀大逆於前,譯為巫蠱於後。如賁之徒,皆不滿志。任之則不遜,致之則怨,自難信也,非我棄之。眾人見此,或有竊議,謂我薄於功臣,斯不然矣。」



 蘇威進曰:「漢光武欲全功臣,皆以列侯奉朝請。至尊仁育,復用此道以安之。」



 上曰:「然。」遂廢於家,是歲卒,年五十四。



 史臣曰:高祖肇基王業,昉、譯實啟其謀,當軸執鈞,物無異論。不能忘身急病,以義斷恩,方乃慮難求全,偷安懷祿。暨夫帝遷明德,義非簡在,鹽梅之寄,自有攸歸。言追昔款,內懷觖望,恥居吳、耿之末,羞與絳、灌為伍。事君盡禮,既闕於宿心,不愛其親,遽彰於物議。其在周也,靡忠貞之節,其奉隋也,愧竭命之誠。非義掩其前功,畜怨興其後釁,而望不陷刑闢,保貴全生,難矣。柳裘、皇甫績、盧賁,因人成事,協規不二,大運光啟,莫參樞要。斯固在人欲其悅己,在我欲其罵人,理自然也。晏嬰有言:「一心可
 以事百君,百心不可以事一君。」於昉、譯見之矣。



\end{pinyinscope}