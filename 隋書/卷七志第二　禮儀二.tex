\article{卷七志第二 禮儀二}

\begin{pinyinscope}

 《春秋》「龍見而雩」,梁制不為恆祀。四月後旱,則祈雨,行七事:一,理冤獄及失職者;二,振鰥寡孤獨者;三,省繇輕賦;四,舉進賢良;五,黜退貪邪;六,命會男女,恤怨曠;七,撤膳羞,弛樂懸而不作。天子又降法服。七日,乃祈社稷;七日,乃祈山林川澤常興雲雨者;七日,乃祈群廟之主於太廟;七日,乃祈古來百闢卿士有益於人者;七日,乃大雩,
 祈上帝,遍祈所有事者。大雩禮,立圓壇於南郊之左,高及輪廣四丈,周十二丈,四陛。牲用黃牯牛一。祈五天帝及五人帝於其上,各依其方,以太祖配,位於青帝之南,五官配食於下。七日乃去樂。又遍祈社稷山林川澤,就故地處大雩。國南除地為墠,舞童六十四人。祈百闢卿士於雩壇之左,除地為墠,舞童六十四人,皆袨服,為八列,各執羽翳。每列歌《雲漢》詩一章而畢。旱而祈澍,則報以太牢,皆有司行事。唯雩則不報。若郡國縣旱請雨,則五事同時並行:一,理冤獄失職;二,存鰥寡孤獨;三,省徭役;四,進賢良;五,退貪邪。守令皆潔齋三日,乃祈社稷。七
 日不雨,更齋祈如初。三變仍不雨,復齋祈其界內山林川澤常興雲雨者。祈而澍,亦各有報。陳氏亦因梁制,祈而澍則報以少牢。武帝時,以德皇帝配,文帝時,以武帝配。廢帝即位,以文帝配青帝。



 牲用黃牯牛,而以清酒四升洗其首。其壇墠配饗歌舞,皆如梁禮。天子不親奉,則太宰、太常、光祿行三獻禮。其法皆採齊建武二年事也。梁、陳制,諸祠官皆給除穢氣藥,先齋一日報之,以取清潔。天監九年,有事雩壇。武帝以為雨既類陰,而求之正陽,其謬已甚。東方既非盛陽,而為生養之始,則雩壇應在東方,祈晴亦宜此地。於是遂移於東郊。十年,帝又以
 雩祭燔柴,以火祈水,於理為乖。儀曹郎硃異議曰:「案周宣《雲漢》之詩,毛注有瘞埋之文,不見有燔柴之說。若以五帝必柴,今明堂又無其事。於是停用柴,從坎瘞典。十一年,帝曰:「四望之祀,頃來遂絕。宜更議復。」硃異議:「鄭眾云:『四望謂日月星海。』鄭玄云:『謂五岳四鎮四瀆。』尋二鄭之說,互有不同。竊以望是不即之名,凡厥遙祭,皆有斯目。



 豈容局於星漢,拘於海瀆?請命司天,有關水旱之義,爰有四海名山大川,能興雲致雨,一皆備祭。」帝從之。又揚州主簿顧協又云:「《禮》『仲夏大雩』,《春秋》『龍見而雩』,則雩常祭也,水旱且又禱之,謂宜式備斯典。」太常博士亦從
 協議。祠部郎明巖卿以為:「祈報之祀,已備郊禋,沿革有時,不必同揆。」帝從其議,依舊不改。大同五年,又築雩壇於藉田兆內。有祈珝,則齋官寄藉田省雲。



 後齊以孟夏龍見而雩,祭太微五精帝於夏郊之東。為圓壇,廣四十五尺,高九尺,四面各一陛。為三壝外營,相去深淺,並燎壇,一如南郊。於其上祈穀實,以顯宗文宣帝配。青帝在甲寅之地,赤帝在丙巳之地,黃帝在己未之地,白帝在庚申之地,黑帝在壬亥之地。面皆內向,藉以槁秸。配帝在青帝之南,小退,藉以莞席,牲以騂。其儀同南郊。又祈禱者有九焉:一曰雩,二曰南郊,三曰堯廟,
 四曰孔、顏廟,五曰社稷,六曰五岳,七曰四瀆,八曰滏口,九曰豹祠。水旱癘疫,皆有事焉。無牲,皆以酒脯棗慄之饌。若建午、建未、建申之月不雨,則使三公祈五帝於雩壇。禮用玉幣,有燎,不設金石之樂,選伎工端潔善謳詠者,使歌《雲漢》詩於壇南。自餘同正雩。南郊則使三公祈五天帝於郊壇,有燎,座位如雩。五人帝各在天帝之左。其儀如郊禮。堯廟,則遣使祈於平陽。孔、顏廟,則遣使祈於國學,如堯廟。社稷如正祭。五岳,遣使祈於岳所。四瀆如祈五岳,滏口如祈堯廟,豹祠如祈滏口。



 隋雩壇,國南十三里啟夏門外道左。高一丈,周百二十
 尺。孟夏之月,龍星見,則雩五方上帝,配以五人帝於上,以太祖武元帝配饗,五官從配於下。牲用犢十,各依方色。京師孟夏後旱,則祈雨,理冤獄失職,存鰥寡孤獨,振困乏,掩骼埋胔,省徭役,進賢良,舉直言,退佞諂,黜貪殘,命有司會男女,恤怨曠。七日,乃祈嶽鎮海瀆及諸山川能興雲雨者;又七日,乃祈社稷及古來百闢卿士有益於人者;又七日,乃祈宗廟及古帝王有神祠者;又七日,乃修雩,祈神州;又七日,仍不雨,復從岳瀆已下祈如初典。秋分已後不雩,但禱而已。皆用酒脯。初請後二旬不雨者,即徙市禁屠。皇帝御素服,避正殿,減膳撤樂,或露
 坐聽政。百官斷傘扇。令人家造土龍。雨澍,則命有司報。州郡尉祈雨,則理冤獄,存鰥寡孤獨,掩骼埋胔,潔齋祈於社。七日,乃祈界內山川能興雨者,徙市斷屠如京師。祈而澍,亦各有報。



 霖雨則珝京城諸門,三珝不止,則祈山川岳鎮海瀆社稷。又不止,則祈宗廟神州。



 報以太牢。州郡縣苦雨,亦各珝其城門,不止則祈界內山川。及祈報,用羊豕。



 《禮》,天子每以四立之日及季夏,乘玉輅,建大旂,服大裘,各於其方之近郊為兆,迎其帝而祭之。所謂燔柴於泰壇,掃地而祭者也。春迎靈威仰者,三春之始,萬物稟之
 而生,莫不仰其靈德,服而畏之也。夏迎赤熛怒者,火色票怒,其靈炎至明盛也。秋迎白招拒者,招集,拒大也,言秋時集成萬物,其功大也。冬迎葉光紀者,葉拾,光華,紀法也,言冬時收拾光華之色,伏而藏之,皆有法也。中迎含樞紐者,含容也,樞機有開闔之義,紐者結也。言土德之帝,能含容萬物,開闔有時,紐結有法也。然此五帝之號,皆以其德而名焉。梁、陳、後齊、後周及隋,制度相循,皆以其時之日,各於其郊迎,而以太皞之屬五人帝配祭。並以五官、三辰、七宿於其方從祀焉。



 梁制,迎氣以始祖配,牲用特牛一,其儀同南郊。天監七
 年,尚書左丞司馬筠等議:「以昆蟲未蟄,不以火田,鳩化為鷹,罻羅方設。仲春之月,祀不用牲,止珪璧皮幣。斯又事神之道,可以不殺明矣。況今祀天,豈容尚此?請夏初迎氣,祭不用牲。」帝從之。八年,明山賓議曰:「《周官》祀昊天以大裘,祀五帝亦如之。



 頃代郊祀之服,皆用袞冕,是以前奏迎氣、祀五帝,亦服袞冕。愚謂迎氣、祀五帝亦宜用大裘,禮俱一獻。」帝從之。陳迎氣之法,皆因梁制。



 後齊五郊迎氣,為壇各於四郊,又為黃壇於未地。所祀天帝及配帝五官之神同梁。其玉帛牲各以其方色。其儀與南郊同。帝及後各以夕牲日之旦,太尉陳幣,告請
 其廟,以就配焉。其從祀之官,位皆南陛之東,西向。壇上設饌畢,太宰丞設饌於其座。亞獻畢,太常少卿乃於其所獻。事畢,皆撤。又云,立春前五日,於州大門外之東,造青土牛兩頭,耕夫犁具。立春,有司迎春於東郊,豎青幡於青牛之傍焉。



 後周五郊壇其崇及去國,如其行之數。其廣皆四丈,其方俱百二十步。內壝皆半之。祭配皆同後齊。星辰、七宿、岳鎮、海瀆、山林、川澤、丘陵、墳衍,亦各於其方配郊而祀之。其星辰為壇,崇五尺,方二丈。岳鎮為坎,方二丈,深二尺。



 山林已下,亦為坎。壇,崇三尺,坎深一尺,俱方一丈。其
 儀頗同南郊。塚宰亞獻,宗伯終獻,禮畢。



 隋五時迎氣。青郊為壇,國東春明門外道北,去宮八里。高八尺。赤郊為壇,國南明德門外道西,去宮十三里,高七尺。黃郊為壇,國南安化門外道西,去宮十二里,高七尺。白郊為壇,國西開遠門外道南,去宮八里,高九尺。黑郊為壇,宮北十一里丑地,高六尺。並廣四丈。各以四方立日,黃郊以季夏土王日。祀其方之帝,各配以人帝,以太祖武元帝配。五官及星三辰七宿,亦各依其方從祀。其牲依方色,各用犢二,星辰加羊豕各一。其儀同南郊。其岳瀆鎮海,各依五時迎氣日,遣使就其所,祭之以太
 牢。



 晉江左以後,乃至宋、齊相承,始受命之主,皆立六廟,虛太祖之位。宋武初為宋王,立廟於彭城,但祭高祖已下四世。中興二年,梁武初為梁公。曹文思議:「天子受命之日,便祭七廟。諸侯始封,即祭五廟。」祠部郎謝廣等並駁之,遂不施用。乃建臺,於東城立四親廟,並妃郗氏而為五廟。告祠之禮,並用太牢。其年四月,即皇帝位。謝廣又議,以為初祭是四時常祭,首月既不可移易,宜依前克日於東廟致齋。帝從之。遂於東城時祭訖,遷神主於太廟。始自皇祖太中府君、皇祖淮陰府君、皇高祖濟陰府
 君、皇曾祖中從事史府君、皇祖特進府君,並皇考,以為三昭三穆,凡六廟。追尊皇考為文皇帝,皇妣為德皇后,廟號太祖。皇祖特進以上,皆不追尊。擬祖遷於上,而太祖之廟不毀,與六親廟為七,皆同一堂,共庭而別室。



 春祀、夏礿、秋嘗、冬烝並臘,一歲凡五,謂之時祭。三年一禘,五年一袷,謂之殷祭。禘以夏,祫以冬,皆以功臣配。其儀頗同南郊。又有小廟,太祖太夫人廟也。



 非嫡,故別立廟。皇帝每祭太廟訖,乃詣小廟,亦以一太牢,如太廟禮。天監三年,尚書左丞何佟之議曰:「禘於首夏,物皆未成,故為小。祫於秋冬,萬物皆成,其禮尤大。司勛列功臣有六,
 皆祭於大烝,知祫尤大,乃及之也。近代禘祫,並不及功臣,有乖典制。宜改。」詔從之。自是祫祭乃及功臣。是歲,都令史王景之,列自江左以來,郊廟祭祀,帝已入齋,百姓尚哭,以為乖禮。佟之等奏:「案《禮》國門在皋門外,今之籬門是也。今古殊制,若禁兇服不得入籬門為太遠,宜以六門為斷。」詔曰:「六門之內,士庶甚多,四時烝嘗,俱斷其哭。若有死者,棺器須來,既許其大,而不許其細也。到齋日,宜去廟二百步斷哭。」四年,何佟之議:「案《禮》未祭一日,大宗伯省牲鑊,祭日之晨,君親牽牲麗碑。後代有冒暗之防,而人主猶必親奉,故有夕牲之禮。頃代人君,不復
 躬牽,相承丹陽尹牽牲,於古無取。宜依以未祭一日之暮,太常省牲視鑊,祭日之晨,使太尉牽牲出入也。少牢饋食,殺牲於廟門外,今《儀注》詣廚烹牲,謂宜儀舊。」帝可其奏。佟之又曰:「鄭玄云:『天子諸侯之祭禮,先有裸尸之事,乃迎牲。』今《儀注》乃至薦熟畢,太祝方執珪瓚裸地,違謬若斯。又近代人君,不復躬行裸禮。太尉既攝位,實宜親執其事,而越使卑賤太祝,甚乖舊典。愚謂祭日之晨,宜使太尉先行裸獻,乃後迎牲。」帝曰:「裸尸本使神有所附。今既無尸,裸將安設?」佟之曰「如馬、鄭之意,裸雖獻尸,而義在求神。今雖無尸,求神之義,恐不可闕。」帝曰:「此本
 因尸以祀神。今若無尸,則宜立寄求之所。」裸義乃定。佟之曰:「《祭統》云:『獻之屬,莫重於裸。』今既存尸卒食之獻,則裸鬯之求,實不可闕。又送神更裸,經記無文,宜依禮革。」奏未報而佟之卒。後明山賓復申其理。帝曰:「佟之既不復存,宜從其議也。」自是始使太尉代太祝行裸而又牽牲。太常任昉又以未明九刻呈牲,又加太尉裸酒,三刻施饌,間中五刻,行儀不辦。近者臨祭從事,實以二更,至未明三刻方辦。明山賓議:「謂九刻已疑太早,況二更非復祭旦。」帝曰:「夜半子時,即是晨始。宜取三更省牲,餘依《儀注》。」又有司以為三牲或離杙,依制埋瘞,豬羊死則不
 埋。請議其制。司馬褧等議,以為「牲死則埋,必在滌矣。謂三牲在滌,死悉宜埋。」帝從之。五年,明山賓議:「樽彞之制,《祭圖》唯有三樽:一曰象樽,周樽也;二曰山罍,夏樽也;三曰著樽,殷樽也。徒有彞名,竟無其器,直酌象樽之酒,以為珪瓚之實。竊尋裸重於獻,不容共樽,宜循彞器,以備大典。案禮器有六彞,春祠夏礿,裸用雞彞鳥彞。王以珪瓚初裸,後以璋瓚亞裸,故春夏兩祭,俱用二彞。今古禮殊,無復亞裸,止循其二。春夏雞彞,秋冬牛彞,庶禮物備也。」帝曰:「雞是金禽,亦主巽位。但金火相伏,用之通夏,於義為疑。」



 山賓曰:「臣愚管,不奉明詔,則終年乖舛。案鳥彞
 是南方之物,則主火位,木生於火,宜以鳥彞春夏兼用。」帝從之。七年,舍人周舍以為:「《禮》「玉輅以祀,金輅以賓』,則祭日應乘玉輅。」詔下其議。左丞孔休源議:「玉輅既有明文,而《儀注》金輅,當由宋、齊乖謬,宜依舍議。」帝從之。又禮官司馬筠議:「自今大事,遍告七廟,小事止告一室。」於是議以封禪,南、北郊,祀明堂,巡省四方,御臨戎出征,皇太子加元服,寇賊平蕩,築宮立闕,纂戎戒嚴、解嚴,合十一條,則遍告七廟。講武,修宗廟明堂,臨軒封拜公王,四夷款化貢方物,諸公王以愆削封,及詔封王紹襲,合六條,則告一室。帝從之。九年,詔簠簋之實,以藉田黑黍。



 十二
 年,詔曰:「祭祀用洗中水盥,仍又滌爵。爵以禮神,宜窮精潔,而一器之內,雜用洗手,外可詳議。」於是御及三公應盥及洗爵,各用一。十六年四月,詔曰:「夫神無常饗,饗於克誠,所以西鄰礿祭,實受其福。宗廟祭祀,猶有牲牢,無益至誠,有累冥道。自今四時烝嘗外,可量代。」八座議:「以大脯代一元大武。」



 八座又奏:「既停宰殺,無復省牲之事,請立省饌儀。其眾官陪列,並同省牲。」



 帝從之。十月,詔曰:「今雖無復牲腥,猶有脯修之類,即之幽明,義為未盡。可更詳定,悉薦時蔬。」左丞司馬筠等參議:「大餅代大脯,餘悉用蔬菜。」帝從之。



 又舍人硃異議:「二廟祀,相承止
 有一金幵羹,蓋祭祀之禮,應有兩羹,相承止於一金幵,即禮為乖。請加熬油羹一金幵。」帝從之。於是起至敬殿、景陽臺,立七廟座。月中再設凈饌。自是訖於臺城破,諸廟遂不血食。普通七年,祔皇太子所生丁貴嬪神主於小廟。其儀,未祔前,先修坎室,改塗。其日,有司行掃除,開坎室,奉皇考太夫人神主於坐。奠制幣訖,眾官入自東門,位定,祝告訖,撤幣,埋於兩楹間。有司遷太夫人神主於上,又奉穆貴嬪神主於下,陳祭器,如時祭儀。禮畢,納神主,閉於坎室。陳制,立七廟,一歲五祠,謂春夏秋冬臘也。每祭共以一太牢,始祖以三牲首,餘唯骨體而已。五歲再
 殷,殷大祫而合祭也。初,文帝入嗣,而皇考始興昭烈王廟在始興國,謂之東廟。天嘉四年,徙東廟神主,祔於梁之小廟,改曰國廟。祭用天子儀。



 後齊文襄嗣位,猶為魏臣,置王高祖秦州使君、王曾祖太尉武貞公、王祖太師文穆公、王考相國獻武王,凡四廟。文宣帝受禪,置六廟:曰皇祖司空公廟、皇祖吏部尚書廟、皇祖秦州使君廟、皇祖文穆皇帝廟、太祖獻武皇帝廟、世宗文襄皇帝廟,為六廟。獻武已下不毀,已上則遞毀。並同廟而別室。既而遷神主於太廟。文襄文宣,並太祖之子,文宣初疑其昭穆之次,欲別立廟。眾議不同。
 至二年秋,始祔太廟。春祠、夏礿、秋嘗、冬烝,皆以孟月,並臘,凡五祭。禘祫如梁之制。每祭,室一太牢,始以皇后預祭。河清定令,四時祭廟禘祭及元日廟庭,並設庭燎二所。



 王及五等開國,執事官、散官從三品已上,皆祀五世。五等散品及執事官、散官正三品已下從五品已上,祭三世。三品已上,牲用一太牢,五品已下,少牢。執事官正六品已下,從七品已上,祭二世,用特牲。正八品已下,達於庶人,祭於寢,牲用特肫,或亦祭祖禰。諸廟悉依其宅堂之制,其間數各依廟多少為限。其牲皆子孫見官之牲。



 後周之制,思復古之道,乃右宗廟而左社稷。置太祖之廟,並高祖已下二昭二穆,凡五。親盡則遷。其有德者謂之祧,廟亦不毀。閔帝受禪,追尊皇祖為德皇帝,文王為文皇帝,廟號太祖。擬已上三廟遞遷,至太祖不毀。其下相承置二昭二穆為五焉。明帝崩,廟號世宗,武帝崩,廟號高祖,並為祧廟而不毀。其時祭,各於其廟,祫禘則於太祖廟,亦以皇后預祭。其儀與後齊同。所異者,皇后亞獻訖,後又薦加豆之籩,其實菱芡芹菹兔醢。塚宰終獻訖,皇后親撤豆,降還板位。然後太祝撤焉。



 高祖既受命,遣兼太保宇文善、兼太尉李詢,奉策詣同
 州,告皇考桓王廟,兼用女巫,同家人之禮。上皇考桓王尊號為武元皇帝,皇妣尊號為元明皇后,奉迎神主,歸於京師。犧牲尚赤,祭用日出。是時帝崇建社廟,改周制,左宗廟而右社稷。



 宗廟未言始祖,又無受命之祧,自高祖已下,置四親廟,同殿異室而已。一曰皇高祖太原府君廟,二曰皇曾祖康王廟,三曰皇祖獻王廟,四曰皇考太祖武元皇帝廟。



 擬祖遷於上,而太祖之廟不毀。各以孟月,饗以太牢。四時薦新於太廟,有司行事,而不出神主。祔祭之禮,並準時饗。其司命,戶以春,灶以夏,門以秋,行以冬,各於享廟日,中霤則以季夏祀黃郊日,各命有
 司,祭於廟西門道南。牲以少牢。三年一祫,以孟冬,遷主、未遷主合食於太祖之廟。五年一禘,以孟夏,其遷主各食於所遷之廟,未遷之主各於其廟。禘祫之月,則停時饗,而陳諸瑞物及伐國所獲珍奇於廟庭,及以功臣配饗。並以其日,使祀先代王公:帝堯於平陽,以契配;帝舜於河東,咎繇配;夏禹於安邑,伯益配;殷湯於汾陰,伊尹配;文王、武王於斁渭之郊,周公、召公配;漢高帝於長陵,蕭何配。各以一太牢而無樂。配者饗於廟庭。



 大業元年,煬帝欲遵周法,營立七廟,詔有司詳定其禮。禮部侍郎、攝太常少卿許善心與博士褚亮等議曰:謹案《禮記》:「天
 子七廟,三昭三穆,與太祖之廟而七。」鄭玄注曰:「此周制也。七者,太祖及文王、武王之祧,與親廟四也。殷則六廟,契及湯與二昭二穆也。夏則五廟,無太祖,禹與二昭二穆而已。」玄又據王者禘其祖之所自出,而立四廟。案鄭玄義,天子唯立四親廟,並始祖而為五。周以文、武為受命之祖,特立二祧,是為七廟。王肅注《禮記》:「尊者尊統上,卑者尊統下。故天子七廟,諸侯五廟。其有殊功異德,非太祖而不毀,不在七廟之數。」案王肅以為天子七廟,是通百代之言,又據《王制》之文「天子七廟,諸侯五廟,大夫三廟」,降二為差。



 是則天子立四親廟,又立高祖之父,高
 祖之祖,並太祖而為七。周有文、武、姜嫄,合為十廟。漢諸帝之廟各立,無迭毀之義。至元帝時,貢禹、匡衡之徒,始建其禮,以高帝為太祖,而立四親廟,是為五廟。唯劉歆以為天子七廟,諸侯五廟,降殺以兩之義。七者,其正法,可常數也,宗不在數內,有功德則宗之,不可預設為數也。



 是以班固稱,考論諸儒之議,劉歆博而篤矣。光武即位,建高廟於洛陽,乃立南頓君以上四廟,就祖宗而為七。至魏初,高堂隆為鄭學,議立親廟四,太祖武帝,猶在四親之內,乃虛置太祖及二祧,以待後代。至景初間,乃依王肅,更立五世、六世祖,就四親而為六廟。晉武受禪,
 博議宗祀,自文帝以上六世祖征西府君,而宣帝亦序於昭穆,未升太祖,故祭止六也。江左中興,賀循知禮,至於寢廟之儀,皆依魏、晉舊事。宋武帝初受晉命為王,依諸侯立親廟四。即位之後,增祠五世祖相國掾府君、六世祖右北平府君,止於六廟。逮身歿,主升從昭穆,猶太祖之位也。



 降及齊、梁,守而弗革,加崇迭毀,禮無違舊。



 臣等又案姬周自太祖已下,皆別立廟,至於禘祫,俱合食於太祖。是以炎漢之初,諸廟各立,歲時嘗享,亦隨處而祭,所用廟樂,皆象功德而歌儛焉。至光武乃總立一堂,而群主異室,斯則新承寇亂,欲從約省。自此以來,因循
 不變。伏惟高祖文皇帝,睿哲玄覽,神武應期,受命開基,垂統聖嗣,當文明之運,定祖宗之禮。



 且損益不同,沿襲異趣,時王所制,可以垂法。自歷代以來,雜用王、鄭二義,若尋其指歸,校以優劣,康成止論周代,非謂經通,子雍總貫皇王,事兼長遠。今請依據古典,崇建七廟。受命之祖,宜別立廟祧,百代之後,為不毀之法。至於鑾駕親奉,申孝享於高廟,有司行事,竭誠敬於群主,俾夫規模可則,嚴祀易遵,表有功而彰明德,大復古而貴能變。臣又案周人立廟,亦無處置之文。據塚人處職而言之,先王居中,以昭穆為左右。阮忱撰《禮圖》,亦從此義。漢京諸廟
 既遠,又不序禘祫。今若依周制,理有未安,雜用漢儀,事難全採。謹詳立別圖,附之議末。



 其圖,太祖、高祖各一殿,準周文武二祧,與始祖而三。餘並分室而祭。始祖及二祧之外,從迭毀之法。詔可,未及創制。既營建洛邑,帝無心京師,乃於東都固本里北,起天經宮,以游高祖衣冠,四時致祭。於三年,有司奏,請準前議,於東京建立宗廟。帝謂秘書監柳抃曰:「今始祖及二祧已具,今後子孫,處朕何所?」



 又下詔,唯議別立高祖之廟,屬有行役,遂復停寢。



 自古帝王之興,皆稟五精之氣。每易姓而起,以致太平,
 必封乎太山,所以告成功也。封訖而禪乎梁甫。梁甫者,太山之支山卑下者也,能以其道配成高德。故禪乎梁甫,亦以告太平也。封禪者,高厚之謂也。天以高為尊,地以厚為德,增太山之高,以報天也,厚梁甫之基,以報地也。明天之所命,功成事就,有益於天地,若天地之更高厚云。《記》曰:「王者因天事天,因地事地。因名山升中於天,而鳳凰降,龜龍格。」齊桓公既霸而欲封禪,管仲言之詳矣。秦始皇既黜儒生,而封太山,禪梁甫,其封事皆秘之,不可得而傳也。漢武帝頗採方士之言,造為玉牒,而編以金繩,封廣九尺,高一丈二尺。光武中興,聿遵其故。晉、
 宋、齊、梁及陳,皆未遑其議。後齊有巡狩之禮,並登封之儀,竟不之行也。開皇十四年,群臣請封禪。高祖不納。晉王廣又率百官抗表固請,帝命有司草儀注。於是牛弘、辛彥之、許善心、姚察、虞世基等創定其禮,奏之。帝逡巡其事,曰:「此事體大,朕何德以堪之。但當東狩,因拜岱山耳。」十五年春,行幸兗州,遂次岱嶽。為壇,如南郊,又壝外為柴壇,飾神廟,展宮縣於庭。為埋坎二,於南門外。又陳樂設位於青帝壇,如南郊。帝服袞冕,乘金輅,備法駕而行。禮畢,遂詣青帝壇而祭焉。



 開皇十四年閏十月,詔東鎮沂山,南鎮會稽山,北鎮醫
 無閭山,冀州鎮霍山,並就山立祠;東海於會稽縣界,南海於南海鎮南,並近海立祠。及四瀆、吳山,並取側近巫一人,主知灑掃,並命多蒔松柏。其霍山,雩祀日遣使就焉。十六年正月,又詔北鎮於營州龍山立祠。東鎮晉州霍山鎮,若修造,並準西鎮吳山造神廟。大業中,昜帝因幸晉陽,遂祭恆岳。其禮頗採高祖拜岱宗儀,增置二壇,命道士女官數十人,於遺中設醮。十年,幸東都,過祀華嶽,築場於廟側。事乃不經,蓋非有司之定禮也。



 《禮》:天子以春分朝日於東郊,秋分夕月於西郊。漢法,不俟二分於東西郊,常以郊泰畤。旦出竹宮東向揖日,其
 夕西向揖月。魏文譏其煩褻,似家人之事,而以正月朝日於東門之外。前史又以為非時。及明帝太和元年二月丁亥,朝日於東郊。



 八月己丑,夕月於西郊。始合於古。後周以春分朝日於國東門外,為壇,如其郊。



 用特牲青幣,青圭有邸。皇帝乘青輅,及祀官俱青冕,執事者青弁。司徒亞獻,宗伯終獻。燔燎如圓丘。秋分夕月於國西門外,為壇於坎中,方四丈,深四尺,燔燎禮如朝日。開皇初,於國東春明門外為壇,如其郊。每以春分朝日。又於國西開遠門外為坎,深三尺,廣四丈。為壇於坎中,高一尺,廣四尺。每以秋分夕月。牲幣與周同。



 凡人非土不生,非穀不食,土穀不可偏祭,故立社稷以主祀。古先聖王,法施於人則祀之,故以勾龍主社,周棄主稷而配焉。歲凡再祭,蓋春求而秋報,列於中門之外,外門之內,尊而親之,與先祖同也。然而古今既殊,禮亦異制。故左社稷而右宗廟者,得質之道也;右社稷而左宗廟者,文之道也。



 梁社稷在太廟西,其初蓋晉元帝建武元年所創,有太社、帝社、太稷,凡三壇。



 門墻並隨其方色。每以仲春仲秋,並令郡國縣祠社稷、先農,縣又兼祀靈星、風伯、雨師之屬。及臘,又各祠社稷於壇。百姓則二十五家為一社,其舊社及人稀者,不限其家。春秋祠,水
 旱禱祈,祠具隨其豐約。其郡國有五嶽者,置宰祝三人,及有四瀆若海應祠者,皆以孟春仲冬祠之。舊太社,廩犧吏牽牲、司農省牲,太祝吏贊牲。天監四年,明山賓議,以為:「案郊廟省牲日,則廩犧令牽牲,太祝令贊牲。



 祭之日,則太尉牽牲。《郊特牲》云『社者神地之道』,國主社稷,義實為重。今公卿貴臣,親執盛禮,而令微吏牽牲,頗為輕末。且司農省牲,又非其義,太常禮官,實當斯職。《禮》,祭社稷無親事牽之文。謂宜以太常省牲,廩犧令牽牲,太祝令贊牲。」帝唯以太祝贊牲為疑,又以司農省牲,於理似傷,犧吏執紖,即事成卑。議以太常丞牽牲,餘依明議。於
 是遂定。至大同初,又加官社、官稷,並前為五壇焉。



 陳制皆依梁舊。而帝社以三牲首,餘以骨體。薦粢盛為六飯:粳以敦,稻以牟,黃粱以簠,白粱以簋,黍以瑚,粢以璉。又令太史署,常以二月八日,於署庭中以太牢祠老人星,兼祠天皇大帝、太一、日月、五星、鉤陳、北極、北斗、三臺、二十八宿、大人星、子孫星,都四十六坐。凡應預祠享之官,亦太醫給除穢氣散藥,先齋一日服之以自潔。其儀本之齊制。



 後齊立太社、帝社、太稷三壇於國右。每仲春仲秋月之元辰及臘,各以一太牢祭焉。皇帝親祭,則司農卿省牲
 進熟,司空亞獻,司農終獻。後周社稷,皇帝親祀,則塚宰亞獻,宗伯終獻。



 開皇初,社稷並列於含光門內之右,仲春仲秋吉戊,各以一太牢祭焉。牲色用黑。孟冬下亥,又臘祭之。州郡縣二仲月,並以少牢祭,百姓亦各為社。又於國城東南七里延興門外,為靈星壇,立秋後辰,令有司祠以一少牢。



 古典有天子東耕儀。江左未暇,至宋始有其典。梁初藉田,依宋、齊,以正月用事,不齋不祭。天監十二年,武帝以為:「啟蟄而耕,則在二月節內。《書》云:『以殷仲春。』藉田理在建卯。」於是改用二月。「又《國語》云:『王即齋宮,與百官御事
 並齋三日。』乃有沐浴裸饗之事。前代當以耕而不祭,故闕此禮。《國語》又云:『稷臨之,太史贊之。』則知耕藉應有先農神座,兼有贊述耕旨。今藉田應散齋七日,致齋三日,兼於耕所設先農神座,陳薦羞之禮。贊辭如社稷法。」



 又曰:「齊代舊事,藉田使御史乘馬車,載耒耜於五輅後。《禮》云:『親載耒耜,措於參保介之御間。』則置所乘輅上。若以今輅與古不同,則宜升之次輅,以明慎重。而遠在餘處,於義為乖。且御史掌視,尤為輕賤。自今宜以侍中奉耒耜,載於象輅,以隨木輅之後。」普通二年,又移藉田於建康北岸,築兆域大小,列種梨柏,便殿及齋官省,如南北郊。
 別有望耕臺,在壇東。帝親耕畢,登此臺,以觀公卿之推伐。又有祈年殿雲。



 北齊藉於帝城東南千畝內,種赤粱、白穀、大豆、赤黍、小豆、黑穄、麻子、小麥,色別一頃。自餘一頃,地中通阡陌,作祠壇於陌南阡西,廣輪三十六尺,高九尺,四陛三壝四門。又為大營於外,又設御耕壇於阡東陌北。每歲正月上辛後吉亥,使公卿以一太牢祠先農神農氏於壇上,無配饗。祭訖,親耕。先祠,司農進穜懸之種,六宮主之。行事之官並齋,設齋省。於壇所列宮懸。又置先農坐於壇上。



 眾官朝服,司空一獻,不燎。祠訖,皇帝乃服通天冠、青
 紗袍、黑介幘,佩蒼玉,黃綬,青帶、襪、舄,備法駕,乘木輅。耕官具朝服從。殿中監進御耒於壇南,百官定列。帝出便殿,升耕,壇南陛,即御座。應耕者各進於列。帝降自南陛,至耕位,釋劍執耒,三推三反,升壇即坐。耕官一品五推五反,二品七推七反,三品九推九反。藉田令帥其屬以牛耕,終千畝。以青箱奉穜「L種,跪呈司農,詣耕所灑之。櫌訖,司農省功,奏事畢。皇帝降之便殿,更衣饗宴。禮畢,班賚而還。



 隋制,於國南十四里啟夏門外,置地千畝,為壇,孟春吉亥,祭先農於其上,以後稷配。牲用一太牢。皇帝服袞冕,
 備法駕,乘金根車。禮三獻訖,因耕。司農授耒,皇帝三推訖,執事者以授應耕者,各以班五推九推。而司徒帥其屬終千畝。



 播殖九穀,納於神倉,以擬粢盛。穰稿以餉犧牲云。



 《周禮》王后蠶於北郊,而漢法皇后蠶於東郊。魏遵《周禮》,蠶於北郊。吳韋昭制《西蠶頌》,則孫氏亦有其禮矣。晉太康六年,武帝楊皇后蠶於西郊,依漢故事。江左至宋孝武大明四年,始於臺城西白石里為西蠶,設兆域。置大殿七間,又立蠶觀。自是有其禮。



 後齊為蠶坊於京城北之西,去皇宮十八里之外,方千
 步。蠶宮,方九十步,墻高一丈五尺,被以棘。其中起蠶室二十七口,別殿一區。置蠶宮,令丞佐史,皆宦者為之。路西置皇后蠶壇,高四尺,方二丈,四出,階廣八尺。置先蠶壇於桑壇東南,大路東,橫路之南。壇高五尺,方二丈,四出,階廣五尺。外兆方四十步,面開一門。有綠衣詹襦、褠衣、黃履,以供蠶母。每歲季春,穀雨後吉日,使公卿以一太牢祀先蠶黃帝軒轅氏於壇上,無配,如祀先農。禮訖,皇后因親桑於桑壇。備法駕,服鞠衣,乘重翟,帥六宮升桑壇東陛,即御座。女尚書執筐,女主衣執鉤,立壇下。皇后降自東陛,執筐者處右,執鉤者居左,蠶母在後。乃躬桑
 三條訖,升壇,即御座。內命婦以次就桑,鞠衣五條,展衣七條,褖衣九條,以授蠶母。還蠶室,切之授世婦,灑一簿。預桑者並復本位。後乃降壇,還便殿,改服,設勞酒,班賚而還。



 後周制,皇后乘翠輅,率三妃、三弋、御媛、御婉、三公夫人、三孤內子至蠶所,以一太牢親祭,進奠先蠶西陵氏神。禮畢,降壇,昭化嬪亞獻,淑嬪終獻,因以公桑焉。



 隋制,於宮北三里為壇,高四尺。季春上巳,皇后服鞠衣,乘重翟,率三夫人、九嬪、內外命婦,以一太牢,制幣,祭先蠶於壇上,用一獻禮。祭訖,就桑位於壇南,東面。尚功進
 金鉤,典制奉筐。皇后採三條,反鉤。命婦各依班採,五條九條而止。世婦亦有蠶母受切桑,灑訖,還依位。皇后乃還宮。自後齊、後周及隋,其典大抵多依晉儀。然亦時有損益矣。



 《禮》:仲春以玄鳥至之日,用太牢祀於高禖。漢武帝年二十九,乃得太子,甚喜,為立禖祠於城南,祀以特牲,因有其祀。晉惠帝元康六年,禖壇石中破為二。



 詔問石毀今應復不,博士議:「《禮》無高禖置石之文,未知造設所由;既已毀破,可無改造。」更下西府博議。而賊曹屬束皙議:「以石在壇上,蓋主道也。祭器弊則埋而置新,今宜埋而更
 造,不宜遂廢。」時此議不用。後得高堂隆故事,魏青龍中,造立此石,詔更鐫石,令如舊,置高禖壇上。埋破石入地一丈。案梁太廟北門內道西有石,文如竹葉,小屋覆之,宋元嘉中修廟所得。陸澄以為孝武時郊禖之石。



 然則江左亦有此禮矣。



 後齊高禖,為壇於南郊傍,廣輪二十六尺,高九尺,四陛三壝。每歲春分玄鳥至之日,皇帝親帥六宮,祀青帝於壇,以太昊配,而祀高禖之神以祈子。其儀,青帝北方南向,配帝東方西向,禖神壇下東陛之南,西向。禮用青珪束帛,牲共以一太牢。祀日,皇帝服袞冕,乘玉輅。皇后服
 褘衣,乘重翟。皇帝初獻,降自東陛,皇后亞獻,降自西陛,並詣便坐。夫人終獻,上嬪獻於禖神訖。帝及後並詣欑位,乃送神。皇帝皇后及群官皆拜。乃撤就燎,禮畢而還。隋制亦以玄鳥至之日,祀高禖於南郊壇。牲用太牢一。



 舊禮祀司中、司命、風師、雨師之法,皆隨其類而祭之。兆風師於西方者,就秋風之勁,而不從箕星之位。兆司中、司命於南郊,以天神是陽,故兆於南郊也。



 兆雨師於北郊者,就水位,在北也。



 隋制,於國城西北十里亥地,為司中、司命、司祿三壇,同壝。祀以立冬後亥。



 國城東北七里通化門外為風師壇,
 祀以立春後丑。國城西南八里金光門外為雨師壇,祀以立夏後申。壇皆三尺,牲以一少牢。



 昔伊耆氏始為蠟。蠟者,索也。古之君子,使人必報之。故周法,以歲十二月,合聚萬物而索饗之。仁之至,義之盡也。其祭法,四方各自祭之。若不成之方,則闕而不祭。後周亦存其典,常以十一月,祭神農氏、伊耆氏、后稷氏、田畯、鱗、羽、臝、毛、介、水、墉、坊、郵、表、畷、獸、貓之神於五郊。五方上帝、地祇、五星、列宿、蒼龍、硃雀、白獸、玄武、五人帝、五官之神、岳鎮海瀆、山林川澤、丘陵墳衍原隰,各分其方,合祭之。日月,五方皆祭之。上帝、地祇、神農、伊耆、人帝於壇
 上,南郊則以神農,既蠟,無其祀。三辰七宿則為小壇於其側,岳鎮海瀆、山林川澤、丘陵墳衍原隰,則各為坎,餘則於平地。皇帝初獻上帝、地祗、神農、伊耆及人帝,塚宰亞獻,宗伯終獻。上大夫獻三辰、五官、后稷、田畯、岳鎮海瀆,中大夫獻七宿、山林川澤已下。自天帝、人帝、田畯、羽毛之類,牲幣玉帛皆從燎;地祇、郵、表、畷之類,皆從埋。祭畢,皇帝如南郊便殿致齋,明日乃蠟祭於南郊,如東郊儀。祭訖,又如黃郊便殿致齋,明日乃祭。祭訖,又如西郊便殿,明日乃祭。



 祭訖,又如北郊便殿,明日蠟祭訖,還宮。隋初因周制,定令亦以孟冬下亥蠟百神,臘宗廟,祭社
 稷。其方不熟,則闕其方之蠟焉。



 又以仲冬祭名源川澤於北郊,用一太牢。祭井於社宮,用一少牢。季冬藏冰,仲春開冰,並用黑牡秬黍,於冰室祭司寒神。開冰,加以桃弧棘矢。



 開皇四年十一月,詔曰:「古稱臘者,接也。取新故交接。前周歲首,今之仲冬,建冬之月,稱蠟可也。後周用夏后之時,行姬氏之蠟。考諸先代,於義有違。



 其十月行蠟者停,可以十二月為臘。」於是始革前制。



 後齊,正月晦日,中書舍人奏祓除。年暮上臺,東宮奏擇吉日詣殿堂,貴臣與師行事所須,皆移尚書省備設雲。後主末年,祭非其鬼,至於躬自鼓儛,以事胡天。



 鄴中遂
 多淫祀,茲風至今不絕。後周欲招來西域,又有拜胡天制,皇帝親焉。其儀並從夷俗,淫僻不可紀也。



\end{pinyinscope}