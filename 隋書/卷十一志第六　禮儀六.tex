\article{卷十一志第六 禮儀六}

\begin{pinyinscope}

 梁制,乘輿郊天、祀地、禮明堂、祠宗廟、元會臨軒,則黑介幘,通天冠平冕,俗所謂平天冠者也。其制,玄表,硃綠裏,廣七寸,長尺二寸,加於通天冠上。前垂四寸,後垂三寸,前圓而後方。垂白玉珠,十有二旒,其長齊肩。以組為纓,各如其綬色,傍垂黈纊,珫珠以玉瑱。其衣,皁上絳下,前三幅,後四幅。衣畫而裳繡。衣則日、月、星辰、山、龍、華蟲、火、
 宗彞,畫以為繢。裳則藻、粉、米、黼黻,以為繡。凡十二章。素帶,廣四寸,硃里,以硃繡裨飾其側。中衣以絳緣領袖。赤皮為韠,蓋古之「X也。絳褲襪,赤舄。佩白玉,垂硃黃大綬,黃赤縹紺四採,革帶,帶劍,緄帶以組為之,如綬色。黃金闢邪首為帶鐍,而飾以白玉珠。又有通天冠,高九寸,前加金博山、述,黑介幘,絳紗袍,皁緣中衣,黑舄,是為朝服。元正賀畢,還儲更衣,出所服也。其釋奠先聖,則皁紗袍,絳緣中衣,絳褲襪,黑舄。臨軒亦服袞冕,未加元服,則空頂介幘。拜陵則箋布單衣,介幘。又有五梁進賢冠、遠游、平上幘武冠。單衣,黑介幘,宴會則服之。



 單衣、白帢,以代古之疑衰、皮弁為吊服,為群臣舉哀臨喪則服之。



 天監三年,何佟之議:「公卿以下祭服,里有中衣,即今之中單也。案後漢《輿服志》明帝永平二年,初詔有司採《周官》、《禮記》、《尚書》,乘輿服,從歐陽說;公卿以下服,從大小夏侯說。祭服,絳緣領袖為中衣,絳褲襪,示其赤心奉神。今中衣絳緣,足有所明,無俟於褲。既非聖法,謂不可施。」遂依議除之。



 四年,有司言:平天冠等一百五條,自齊以來,隨故而毀,未詳所送。何佟之議:「《禮》『祭服敝則焚之』。」於是並燒除之,
 其珠玉以付中署。



 七年,周舍議:「詔旨以王者袞服,宜畫鳳皇,以示差降。按《禮》:『有虞氏皇而祭,深衣而養老。』鄭玄所言皇,則是畫鳳皇羽也。又按《禮》所稱雜服,皆以衣定名,猶如袞冕,則是袞衣而冕。明有虞言皇者,是衣名,非冕,明矣。畫鳳之旨,事實灼然。」制:「可。」又王僧崇云:「今祭服,三公衣身畫獸,其腰及袖,又有青獸,形與獸同,義應是蜼,即宗彞也。兩袖各有禽鳥,形類鸞鳳,似是華蟲。今畫宗彞,即是周禮。但鄭玄云:『蜼,禺屬,昂鼻長尾。』是獸之輕小者,謂宜不得同獸。尋冕服無鳳,應改為雉。又裳有圓花,於禮無礙,疑是
 畫師加葩L耳。藻米黼黻,並乖古制,今請改正,並去圓花。」帝曰:「古文日月星辰,此以一辰攝三物也。山龍華蟲,又以一山攝三物也。藻火粉米,又以一藻攝三物也。



 是為九章。今袞服畫龍,則宜應畫鳳明矣。孔安國云:『華者,花也。』則為花非疑。若一向畫雉,差降之文,復將安寄?鄭義是所未允。」又帝曰:「《禮》:『王者祀昊天上帝,則大裘而冕,祀五帝亦如之。』又云:『莞席之安,而蒲越稿秸之用。』斯皆至敬無文,貴誠重質。今郊用陶匏,與古不異,而大裘蒲秸,獨不復存,其於質敬,恐有未盡。且一獻為質,其劍佩之飾及公卿所著冕服,可共詳定。」



 五經博士陸瑋等並云:「
 祭天猶存掃地之質,而服章獨取黼黻為文,於義不可。今南郊神座,皆用嵒席,此獨莞類,未盡質素之理。宜以稿秸為下藉,蒲越為上席。



 又《司服》云:『王祀昊天,服大裘』,明諸臣禮不得同。自魏以來,皆用袞服,今請依古,更制大裘。」制:「可。」瑋等又尋大裘之制,唯鄭玄注《司服》云「大裘,羔裘也」,既無所出,未可為據。案六冕之服,皆玄上纁下。今宜以玄繒為之。其制式如裘,其裳以纁,皆無文繡。冕則無旒。詔:「可」。



 又乘輿宴會,服單衣,黑介幘。舊三日九日小會,初出乘金輅服之。八年,帝改去還皆乘輦,服白紗帽。



 九年,司馬筠等參議:「《禮記·玉藻》云:『諸侯玄冕以祭,裨冕以朝。』《雜記》又云:『大夫冕而祭於公,弁而祭於己。』今之尚書,上異公侯,下非卿士,止有朝衣,本無冕服。但既預齋祭,不容同在於朝,宜依太常及博士諸齋官例,著皁衣,絳襈,中單,竹葉冠。若不親奉,則不須入廟。」帝從之。



 十一年,尚書參議:「按《禮》,跣襪,事由燕坐,履不宜陳尊者之側。今則極敬之所,莫不皆跣。清廟崇嚴,既絕恆禮,凡有履行者,應皆跣襪。」詔:「可。」



 陳永定元年,武帝即位,徐陵白:「所定乘輿御服,皆採梁之舊制。」又以為「冕旒,後漢用白玉珠,晉過江,服章多闕,遂
 用珊瑚雜珠,飾以翡翠」。侍中顧和奏:「今不能備玉珠,可用白〔。」從之。蕭驕子云:「白〔,蚌珠是也。」帝曰:「形制依此。今天下初定,務從節儉。應用繡、織成者,並可彩畫,金色宜塗,珠玉之飾,任用蚌也。」至天嘉初,悉改易之,定令具依天監舊事,然亦往往改革。



 今不同者,皆隨事於注言之;不言者,蓋無所改制云。



 皇太子,金璽龜鈕,硃綬,三百二十首朝服,遠游冠,金博山,佩瑜玉翠緌,垂組,硃衣,絳紗袍,皁緣白紗中衣,白曲領,帶鹿盧劍,火珠首,素革帶,玉鉤燮,獸頭鞶囊。其大小會、祠廟、朔望、五日還朝,皆朝服,常還上宮則硃服。若釋奠,則
 遠游冠,玄朝服,絳緣中單,絳褲襪,玄舄。講,則著介幘。又有三梁進賢冠。其侍祀則平冕九旒,袞衣九章,白紗絳緣中單,絳繒韠,赤舄,絳靺。若加元服,則中舍執冕從。皇太子舊有五時朝服,自天監之後則硃服。在上省則烏帽,永福省則白帽云。



 諸王,金璽龜鈕,纁硃綬,一百六十首朝服,遠游冠,介幘,硃衣,絳紗袍,皁緣中衣,素帶,黑舄。佩山玄玉,垂組,大帶,獸頭鞶,腰劍。若加餘官,則服其加官之服。



 開國公,金章龜鈕,玄硃綬,一百四十首朝服,紗硃衣,進賢三梁冠,佩水蒼玉,獸頭鞶,腰劍。



 開國公、伯,金章龜鈕,青硃綬,一百二十首朝服,紗硃衣,進賢三梁冠,佩水蒼玉,善頭鞶,腰劍。



 開國子、男,金章龜鈕,青綬,一百首朝服,紗硃衣,進賢三梁冠,佩水蒼玉,獸頭鞶,腰劍。



 縣、鄉、亭、關內、關中及名號侯,金印龜鈕,紫綬,朝服,進賢二梁冠,獸頭鞶,腰劍。關內、關中及名號侯則珪鈕。



 關外侯,銀印珪鈕,青綬,朝服,進賢二梁冠,獸頭鞶,腰劍。



 諸王嗣子,金印珪鈕,紫綬,八十首朝服,進賢二梁冠,佩山玄玉,獸頭鞶,腰劍。



 開國公、侯嗣子,銀印珪鈕,青綬,八十首朝服,進賢二梁冠,
 佩水蒼玉,獸頭鞶,腰劍。



 太宰、太傅、太保、司徒、司空,金章龜鈕,紫綬八十首朝服,進賢三梁冠,佩山玄玉,獸頭鞶,腰劍。《陳令》加有相國丞相,服制同。



 大司馬、大將軍、太尉、諸位從公者,金章龜鈕,紫綬,八十首朝服,武冠,佩山玄玉,獸頭鞶,腰劍。直將軍則不帶劍。



 凡公及位從公、言以將軍及以左右光祿、開府儀同者,各隨本位號。其文則曰「某位號儀同之章」。五等諸侯,助祭郊廟,皆平冕九旒,青玉為珠,有前無後。



 各以其綬色為組纓,旁垂黈纊。衣,玄上纁下,畫山龍已下九章,備五採,大佩,赤舄,絇履。錄尚書無章綬品秩,悉以餘官總司其任,服則餘官之服,猶執笏紫荷。



 其在
 都坐,則東面最上。



 尚書令、僕射、尚書,銅印墨綬,朝服,納言幘、進賢冠,佩水蒼玉,尚書則無印綬腰劍,紫荷,執笏。陳尚書令、僕射,金章龜鈕,紫綬,八十首,獸頭鞶。



 尚書無印綬及鞶。餘並同梁。



 侍中散騎常侍、通直常侍、員外常侍,朝服,武冠貂蟬,侍中左插,常侍右插。



 皆腰劍,佩水蒼玉。其員外常侍不給佩舊至尊朝會登殿,侍中常侍夾御,御下輿,則扶左右。侍中驂乘,則不帶劍。



 中書監、令、秘書監,銅印墨綬,朝服,進賢兩梁冠,佩水蒼玉,腰劍,獸頭鞶。陳制,銀章龜鈕,青綬,八十首,獸頭鞶,腰劍。餘同梁。



 左、右光祿大夫,皆與加金章紫綬同。其但加金紫者,謂
 之金紫光祿,但加銀青者,謂之光祿大夫。《陳令》有特進,進賢二梁冠,朝服,佩水蒼玉,腰劍《梁令》不載。



 光祿、太中、中散大夫,太常、光祿、弘訓太僕、太僕、廷尉、宗正、大鴻臚、大司農、少府、大匠諸卿,丹陽尹,太子保、傅,大長秋,太子詹事,銀章龜鈕,青綬,獸頭鞶,朝服,進賢冠二梁,佩水蒼玉。卿大夫助祭,則冠平,冕五旒,黑玉為珠,有前無後。各以其綬採為組纓,旁垂黈纊。衣,玄上纁下,畫華蟲七章,皆佩五採大佩,赤舄,絇履。陳宮卿改云慈訓,餘皆同梁。又有太舟卿,服章同。



 驃騎、車騎、衛將軍、中軍、冠軍、輔國將軍、四方中郎將,金
 章紫綬,中郎將則青綬。朝服,武冠,佩水蒼玉。《陳令》:鎮、衛、驃騎、車騎、中軍、中衛、中撫軍、中權、四征、四鎮、四安、四翊、四平將軍,金章獸鈕。其冠軍、四方中郎將,金章豹鈕,並紫綬,八十首,獸頭鞶,朝服,武冠,佩水蒼玉。自中軍已下諸將軍及冠軍、四方中郎將,並官不給佩。



 領、護軍,中領、護軍,五營校尉,銀印青綬,朝服,武冠,佩水蒼玉,獸頭鞶。其屯騎,夾御日,假給佩,餘校不給。《陳令》:領、護,金章龜鈕,紫綬,八十首。中領、護,銀章龜鈕,青綬,八十首。其五營校尉,銀印珪鈕,青綬,八十首。官不給佩。餘並同梁。



 弘訓衛尉,衛尉,陳宮卿云慈訓,服同諸卿,但武冠。司隸校尉,陳無官服左右衛、驍騎、游擊、前、左、右、後軍將軍,龍驤、寧朔、建威、振威、奮威、揚威、廣威、武威、建武、振武、奮武、揚武、廣武等將軍,積弩、積射、強弩將軍,監軍,銀章青綬,朝服,武冠,佩水蒼玉,
 獸頭珪。驍、游已下,並不給佩。驍、游夾侍日,假給。《陳令》:左、右衛,銀章龜鈕,不給劍。左右驍騎、游擊、雲騎、游騎、前、左、右、後軍將軍,左右中郎將,銀印珪鈕。餘服飾同梁,亦官不給佩。其驍、游、雲騎,夾御日,假給。其積弩、積射、強弩,銅印環鈕,墨綬,帶劍。餘服同梁。又有忠武、軍師、武臣、爪牙、龍騎、雲麾、鎮兵、翊帥、宣惠、宣毅、智威、仁威、勇威、信威、嚴威、智武、仁武、勇武、信武、嚴武,金章豹鈕,紫綬,八十首。官不給。輕車、鎮朔、武旅、貞毅、明威、寧遠、安遠、征遠、振遠、宣遠等將軍,金章貔鈕,紫綬,並獸頭鞶,朝服,武冠,佩水蒼玉。



 國子祭酒,皁朝服,進賢二梁冠,佩水蒼玉。



 御史中丞、都水使者,銀印,墨綬,朝服,進賢二梁冠,獸頭鞶,腰劍,佩水蒼玉。陳中丞,銀章龜鈕,青綬,八十首,二梁冠。餘同梁。其都水,陳、梁改為太舟卿,服在諸卿中見。



 謁者僕射,銅印環鈕,墨綬,八十首。朝服,高山冠,獸頭鞶,佩
 水蒼玉,腰劍。



 諸軍司,銀章龜鈕,青綬,朝服,武冠,獸頭鞶。



 給事中、黃門侍郎、散騎通直員外、散騎侍郎、奉朝請、太子中庶子、庶子、武衛將軍、武騎常侍,朝服,武冠,腰劍。《陳令》:庶子已上簪筆。其武衛不劍,正直夾御,白布褲褶。



 中書侍郎,朝服,進賢一梁冠,腰劍。冗從僕射、太子衛率,銅印,墨綬,獸頭鞶,朝服,武冠。陳衛率,銀章龜鈕,青綬,不劍。冗從,銅印環鈕,墨綬,腰劍。餘並同梁。



 武賁中郎將、羽林監,銅印環鈕,墨綬,朝服,武冠,獸頭鞶,腰劍。其在陛牙及備鹵簿,著毼尾,絳紗縠單衣。



 護匈奴中郎將,護羌、戎、夷、蠻、越、烏丸、西域校尉,銀印珪鈕,青綬,朝服,武冠,獸頭鞶。《陳令》無此官。其庶子,鎮蠻、寧蠻、平戎、西戎校尉,平越中郎將,服章同。



 安夷、撫夷護軍,州郡國都尉,奉車、駙馬、騎都尉,諸護軍,銀印珪鈕,青綬,獸頭鞶,朝服,武冠。陳安遠、鎮蠻護軍,州、郡、國都尉,奉車、駙馬、騎都尉,諸護軍,服章同。無餘文。



 州刺史,銅印,墨綬,獸頭鞶,腰劍,絳朝服,進賢二梁冠。陳銅章龜鈕,青綬。餘同梁。



 郡國太守、相、內史,銀章龜鈕,青綬,獸頭鞶,單衣,介幘。加中二千石,依卿尹冠服劍佩。



 尚書左、右丞,秘書丞,銅印環鈕,黃綬,獸爪鞶,朝服,進賢一梁冠。



 尚書,秘書著作郎,太子中舍人、洗馬、舍人,朝服,進賢一梁冠,腰劍。



 諸王友、文學,硃服,進賢一梁冠。《陳令》,諸王師服同。



 治書侍御史、侍御史,朝服,腰劍,法冠。治書侍御史,則有銅印環鈕,墨綬。



 陳又有殿中、蘭臺侍御史,朝服,法冠,腰劍,簪筆。



 諸博士,給皁朝服,進賢兩梁冠,佩水蒼玉。



 太學博士,正限八人,著佩,限外六人不給。



 廷尉律博士,無佩。並簪筆。



 國子助教,皁朝服,進賢一梁冠,簪筆。



 公府長史,獸頭鞶。諸卿尹丞,黃綬,獸爪鞶,簪筆。



 諸縣署令、秩千石者,獸爪鞶,銅印環鈕,墨綬,朝服,進賢兩梁冠。長史硃服,諸卿尹丞、建康令,玄服。



 公府掾屬、主簿、祭酒,硃服,進賢一梁冠。公府令史亦同。



 領、護軍長史,硃服,獸頭鞶。諸軍長史,單衣,介幘,獸頭鞶。



 諸卿部丞、獄丞,並皁朝服,一梁冠,黃綬,獸爪鞶,簪筆。



 太子保、傅、詹事丞,早朝服,一梁冠,簪筆,獸爪鞶,黃綬。



 郡國相、內史丞、長史,單衣,介幘。長史,獸頭鞶,其丞,黃綬,獸爪鞶。



 諸縣署令、長、相,單衣,介幘,獸頭鞶,銅印環鈕,墨綬,朝服,進賢一梁冠。諸署令,硃衣,武冠。州都大中正、郡中正,單衣,介幘。



 太子門大夫,獸頭鞶,陵令、長,獸爪鞶,銅印環鈕,墨綬,朝服,進賢一梁冠。令、長硃服,率更、家令、僕,朝服,兩梁冠,獸頭鞶,腰劍。



 黃門諸署令、僕、長丞,硃服,進賢一梁冠,銅印環鈕,墨綬。丞,黃綬。黃門冗從僕射監、太子寺人監,銅印環鈕,墨綬,朝服,武冠,獸頭般。



 公府司馬,領、護軍司馬,諸軍司馬,護匈奴中郎將,護羌、
 戎、夷、蠻、越、烏丸、戊己校尉長史、司馬,銅印環鈕,墨綬,獸頭鞶,朝服,武冠。諸軍司馬,單衣,平巾幘。長史,介幘。《陳令》:公府司馬,領、護軍司馬,諸軍司馬,鎮安蠻安遠護軍,蠻、戎、越校尉中郎將長史、司馬,其服章與梁官同。



 公府從事中郎,硃服,進賢一梁冠。諸將軍開府功曹、主簿,單衣,介幘,革帶。廷尉,建康正、監平,銅印環鈕,墨綬,皁零闢,朝服,法冠,獸爪鞶。



 左、右衛司馬,銅印環鈕,墨綬,單衣,帶,平巾幘,獸頭鞶。



 諸府參軍,單衣,平巾幘。



 諸州別駕、治中、從事、主簿、西曹從事,玄朝服,進賢一梁冠,簪筆。常公事,單衣,介幘,硃衣。



 直閣將軍,硃服,武冠,銅印珪鈕,青綬,獸頭鞶。



 直閣將軍、諸殿主帥,硃服,武冠。正直絳衫,從則衣雨襠衫。



 諸開國郎中令、大農、公、傅中尉,銅印環鈕,青綬,朝服,進賢兩梁冠,中尉武冠,皆獸頭鞶。



 諸開國三將軍,銅印環鈕,青綬,朝服,武冠。限外者不給印。陳制:墨綬,餘並同梁。



 開國掌書中尉、司馬,陵廟食官,廄牧長,典醫典府丞,銅印。



 常侍、侍郎、世子、庶子、謁者、中大夫、舍人,不給印。典書、典祠、學官令,典膳丞、長,銅印。限外者不給印。



 左右常侍、侍郎,典衛中尉司馬,朝服,武冠。典書、典祠、學
 官令,朝服,進賢一梁冠。餘悉硃服,一梁冠。常侍、侍郎、典書、典祠、學官令,簪筆,腰劍。



 太子衛率、率更、家令丞,銅印環鈕,黃綬,皁朝服,進賢一梁冠,獸爪鞶。



 太子常從武賁督,銅印環鈕,墨綬,朝服,武冠,獸爪鞶。



 殿中將軍、員外將軍,硃服,武冠。



 州郡國都尉司馬,銅印環鈕,墨綬,硃服,武冠,獸頭鞶。



 諸謁者,朝服,高山冠。



 中書通事舍人門下令史、主書典書令史、門下朝廷局書令史、太子門下通事守舍人、主書典守舍人、二宮齋
 內職左右職局齋乾已上,硃服,武冠。



 殿中內外局監、太子內外監、殿中守舍人,銅印環鈕,硃服,武冠。



 內外監典事書吏,硃服,進賢一梁冠。內監朝廷人領局典事、外監統軍隊諮詳發遣局典事,武冠。外監及典事書吏,悉著硃衣,唯正直及齋監並受使,不在例。



 其東宮內外監、殿典事書吏,依臺格。五校、三將將軍主事,內監主事,外監主事,三校主事,硃服,武冠。



 尚書都令史,都水參事,門下書令史,集書、中書、尚書、秘書著作掌書主書主圖主譜典客令史書令史,監、令、僕
 射省事,蘭臺、殿中蘭臺、謁、都水令史,公府令史書令史,太子導客、次客守舍人及諸省典事,硃衣,進賢一梁冠。



 尚書都算、度支算、左右校吏,硃服,進賢一梁冠。



 諸縣署丞、太子諸署丞、王公侯諸署及公主家令丞、僕,銅印環鈕,黃綬,硃服,進賢一梁冠。太官、太醫丞,武冠。



 諸縣尉,銅印環鈕,單衣,介幘,黃綬,獸爪鞶。節騎郎,硃服,武冠。其在陛列及備鹵簿者,毼尾,絳紗縠單衣。御節郎、黃鉞郎,朝服,赤介幘,簪筆。典儀、唱警、唱奏事、持兵、主麾等諸職,公事及備鹵簿,硃服,武冠。



 殿中中郎將、校尉、都尉,銀印珪鈕,青綬,硃服,武冠,獸頭鞶。



 城門侯,銅印環鈕,墨綬,硃服,武冠,獸頭鞶。



 部曲督、司馬吏、部曲將,銅印環鈕,硃服,武冠。司馬吏,假墨綬,獸爪鞶。



 太中、中散、諫議大夫,議郎、中郎、郎中、舍人,硃服,進賢一梁冠。



 諸門郎、僕射、佐吏,東宮門吏,其郎硃服,僕射皁零闢,朝服,進賢冠,吏卻非冠,佐吏著進賢冠。



 總章協律,銅印環鈕,艾綬,獸爪鞶,硃服,武冠。



 黃門後閣舍人、主書、齋帥、監食、主食、主客、扶侍、鼓吹,硃服,武冠。



 鼓吹進賢冠,齋帥墨綬,獸頭鞶。



 殿中司馬,銅印環鈕,墨綬,硃服,武冠,獸頭鞶。



 總章監、鼓吹監,銅印環鈕,艾綬,硃服,武冠。



 諸四品將兵都尉、牙門將、崇毅、材官、折難、輕騎、揚烈、威遠、寧遠、宣威、光威、驤威、威烈、威虜、平戎、綏遠、綏狄、綏邊、綏戎、獸威、威武、烈武、毅武、奮武、討寇、討虜、殄難、討難、討夷、厲武、橫野、陵江、鷹揚、執訊、蕩寇、蕩虜、蕩難、蕩逆、殄虜、掃虜、掃難、掃逆、掃寇、厲鋒、武奮、武牙、廣野,領兵滿五十人,給銀章,不滿五十,除板而已,不給章,硃服,武冠。



 以此官為刺史、太守,皆青綬。此條已下,皆陳制,與梁不同。



 典儀但帥、典儀正帥,硃衣,武冠。其本資有殿但、正帥,得
 帶艾綬,獸頭鞶。



 殿但帥、正帥,艾綬,獸頭鞶,硃服,武冠。殿帥、羽儀帥、員外帥,硃衣,武冠。



 威雄、猛、烈、振、信、勝、略、風、力、光等十威將軍,武猛、略、勝、力、毅、健、烈、威、銳、勇等十武將軍,並銀章熊鈕,青綬,獸頭鞶,武冠,朝服。



 猛毅、烈、威、銳、震、進、智、武、勝、駿等十猛將軍,銀章羆鈕,青綬,獸頭鞶,武冠,朝服。



 壯武、勇、烈、猛、銳、威、毅、志、意、力等十壯將軍,驍雄、桀、猛、烈、武、勇、銳、名、勝、迅等十驍將軍,雄猛、威、明、烈、信、武、勇、毅、壯、健等十雄將軍,並銀章羔鈕,青綬,獸頭鞶,武冠,朝服。



 忠勇、烈、猛、銳、壯、毅、捍、信、義、勝等十忠將軍,明智、略、遠、勇、烈、威、勝、進、銳、毅等十明將軍,光烈、明、英、遠、勝、銳、命、勇、武、野等十光將軍,飆勇、猛、烈、銳、奇、決、起、略、勝、出等十飆將軍,並銀章鹿鈕,青綬,獸頭鞶,武冠,朝服。



 龍驤、武視、雲旗、風烈、電威、雷音、馳、銳、進銳、羽騎、突騎、折沖、冠武、和戎、安壘、起猛、英果、掃虜、掃狄、武銳、摧鋒、開遠、略遠、貞威、決勝、清野、堅銳、輕銳、拔山、雲勇、振旅等三十號將軍,銀印菟鈕,青綬,獸頭鞶,朝服,武冠。



 超武、鐵騎、樓船、宣猛、樹功、克狄、平虜、棱威、戎昭、威戎、伏波、雄戟、長劍、沖冠、雕騎、佽飛、勇騎、破敵、克敵、威虜、前鋒、
 武毅、開邊、招遠、全威、破陣、蕩寇、殄虜、橫野、馳射等三十號將軍,銅印環鈕,墨綬,獸頭鞶,朝服,武冠。並左十二件將軍,除並假給章印綬,板則止硃服、武冠而已。其勛選除,亦給章印。



 建威、牙門、期門已下諸將軍,並銅印環鈕,墨綬,獸頭鞶,硃服,武冠。板則無印綬,止冠服而已。其在將官,以功次轉進,應署建威已下諸號,不限板除,悉給印綬。若武官署位轉進,登上條九品馳射已上諸戎號,亦不限板除,悉給印綬。



 千人督、校督司馬,武賁督、牙門將、騎督督、守將兵都尉、太子常從督別部司馬、假司馬,假銅印環鈕,硃服,武冠,
 墨綬,獸頭鞶。



 武猛中郎將、校尉、都尉,銅印環鈕,硃服,武冠。其以此官為千人司馬、道賁督已上及司馬,皆假墨綬,獸頭鞶。已上陳制,梁所無及不同者。



 陛長、甲僕射、主事吏將騎、廷上五牛旗假吏武賁,在陛列及備鹵簿,服錦文衣,武冠,毼尾。陛長者,假銅印環鈕,墨綬,獸頭鞶。



 假旄頭羽林,在陛列及備鹵簿,服絳單衣,上著韋畫腰襦,假旄頭。輿輦、跡禽、前驅、由基強弩司馬,給絳科單衣,武冠。其本位佩武猛都尉已上印者,假墨綬,別部司馬
 已下假墨綬,並獸頭鞶。



 殿中冗從武賁、殿中武賁、持鈒戟冗從武賁,假青綬,絳科單衣,武冠。《陳令》:絳科單衣,其本位職佩武猛、都尉等印,假鞶綬,依前條。



 持椎斧武騎武賁、五騎傳詔武賁、殿中羽林、太官尚食武賁、稱飯宰人、諸宮尚食武賁,假墨綬,給絳褠,武冠。其佩武猛、都尉等位印,皆依上條假鞶綬之例。



 其在陛列及備鹵簿,五騎武賁,服錦文衣,毼尾。宰人服離支衣。領軍捉刃人,烏總帽,褲褶,皮帶。



 絓是羽葆毦鼓吹,悉改著進賢冠,外給系毦。鼓吹著武
 冠。諸官鼓吹,尚書廊下都坐門下使守藏守閣、殿中威儀騶,武賁常直殿門雲龍門者、門下左右部武賁羽林騶,給傳事者諸導騶門下中書守閣、尚書門下武賁羽林騶,蘭臺五曹節藏僕射廊下守閣、威儀發符騶,都水使者廊下守給騶,謁者威儀騶,諸宮謁者騶,絳曈,武冠,衣服如舊。大誰、天門士,皁科單衣,樊噲冠。衛士,涅布曈,卻敵冠。



 諸將軍、使持節、都督執節史,硃衣,進賢一梁冠。自此條已下皆陳制,梁所無。



 持節節史,單衣,介幘。其纂戎戒嚴時,同使持節。制假節
 節史,單衣,介幘。



 凡節趺,以石為之。持節皆刻為鞶螭形,假節及給蠻夷節,皆刻為狗頭趺。



 諸王典簽帥,單衣,平巾幘。典簽書吏,褲褶,平巾幘。



 諸王書佐,單衣,介幘。



 公府書佐,硃衣,進賢冠。



 諸王國舍人、司理、謁者、閤下令史、中衛都尉,硃衣,進賢一梁冠。司理假銅印,謁者高山冠,令史已下武冠。



 太子太傅五官功曹、主簿,皁朝服,進賢一梁冠。



 太子二傅門下主記、錄事、功曹書佐,門下書佐,記室帳下督、都督省事,法曹書佐,太傅外都督,皁衣,進賢一梁
 冠。



 太子妃家令,絳朝服,進賢一梁冠。



 太子三校、二將,積弩、殿中將軍,衣服皆與上宮官同。



 太子正員司馬督、題閣監,銅印墨綬三校內主事、主章、扶侍,守舍人,衣帶仗局、服飾衣局、珍寶朝廷主衣統,奏事干,內局內幹,硃衣,武冠。



 諸公府御屬及省事,錄尚書省事,太子門下及內外監丞、典事、導客、算書吏,次功、典書函、典書、典經、五經典書諸守宮舍人,市買清慎食官督,內直兵吏,宣華、崇賢二門舍人,諸門吏,硃衣,進賢一梁冠。



 太子妃傳令,硃衣,武冠,執刀,烏信幡。



 太子二傅騎吏,玄衣,赤幘,武冠,常行則褲褶。執儀、齋帥、殿帥、典儀帥、傳令、執刀戟、主蓋扇麾傘、殿上持兵、車郎、扶車、注疏、萌床、齋閣食司馬、唱導飯、主食、殿前帥、殿前威儀、武賁威儀、散給使、閣將、鼓吹士帥副,武冠,〓。案軛、小輿、持車、軺車給使,平巾幘,黃布褲褶,赤罽帶。



 太子諸門將,涅布曈,樊噲冠。



 太子鹵簿戟吏,赤幘,武冠,絳褠。廉帥、整陣、禁防,平巾幘,白布褲褶。



 銚角五音帥、長麾,青布褲褶,岑帽,絳絞帶。都伯,平巾幘,黃布褲褶。



 文官曹干,白紗單衣、介幘。尚書二臺曹干亦同。



 武官問訊、將士給使,平巾幘,白布褲褶。



 通天冠,高九寸,正豎頂,少斜卻,乃直下,鐵為卷梁,前有展筒,冠前加金博山、述。乘輿所常服。



 遠游冠,制似通天,而前無山、述,有展筒,橫於冠前。皇太子及王者後、諸王服之。諸王加官者,自服其官之冠服,唯太子及王者後常冠焉。太子則以翠羽為緌,綴以白珠。其餘但青絲而已。



 進賢冠,古緇布冠遺象也,斯蓋文儒者之服。前高七寸,後高三寸,長八寸。



 有五梁、三梁、二梁、一梁之別。五梁唯天子所服,其三梁已下,為臣高卑之別云。



 武冠,一名武弁,一名大冠,一名繁冠,一名建冠,今人名曰籠冠,即古惠文冠也。天子元服,亦先加大冠。今左右侍臣及諸將軍武官通服之。侍中常侍,則加金璫附蟬焉,插以貂尾,黃金為飾云。



 高山冠,一名側注,高九寸,鐵為卷梁。制似通天,頂直豎,不斜,無山述展筒。高山者,取其矜莊賓遠,中外謁者僕射服之。



 法冠,一名柱後,或謂之獬豸冠,高五寸,以縰為展筒,鐵為柱卷,取其不曲撓也。侍御史、廷尉正監平,凡執法官皆服之。



 鶡冠,猶大冠也,加雙鶡尾,豎插兩邊,故以名焉。武賁中郎將、羽林監、節騎郎,在陛列及鹵簿者服之。



 長冠,一名齋冠。高七寸,廣三寸,漆「O為之。制如版,以竹為里。漢高祖微時,以竹皮為此冠,所謂劉氏冠。後除竹,用漆「O焉。司馬彪曰:「長冠,楚制也。人間或謂之鵲尾冠,非也。」後代以為祭服,尊敬之也。至天監三年,祠部郎沈宏議:「案竹葉冠,是高祖為亭長時所服,安可綿代為祭服哉?《禮》:『士弁祭於公。』請令太常丞、博士奉齋之服,宜改用爵弁。」明山賓同宏議。司馬褧云:「若必遵三王,則懼所改非一。長冠謂宜仍舊。案今之宗丞博士之服,未有可
 非。」帝竟不改。



 建華冠,以鐵為柱卷,貫大銅珠九枚。祀天地、五郊、明堂,舞人服之。



 樊噲冠,廣九寸,高七寸,前後出各四寸,制似平冕。凡殿門司馬衛士服之。



 卻敵冠,高四寸,通長四寸,後高三寸,制似進賢冠。凡宮殿門衛士服之。



 卻非冠,高五寸,制似長冠。宮殿門吏僕射冠之。



 幘,尊卑貴賤皆服之。文者長耳,謂之介幘;武者短耳,謂之平上幘。各稱其冠而制之。尚書令、僕射、尚書幘,收方
 三寸,名曰納言。未冠童子幘,無屋,施假髻者,示未成人也。



 幍,《傅子》云:「先未有歧,荀文若巾觸樹成歧,時人慕之,因而弗改。」



 今通為慶吊之服。白紗為之,或單或裌。初婚冠送餞亦服之。



 巾,國子生服,白紗為之。晉太元中,國子生見祭酒博士,單衣,角巾,執經一卷,以代手版。宋末,闕其制。齊立學,太尉王儉更造。今形如之。



 帽,自天子下及士人,通冠之。以白紗者,名高頂帽。皇太子在上省則烏紗,在永福省則白紗。又有繒皁雜紗為
 之,高屋下裙,蓋無定準。



 褲褶,近代服以從戎。今纂嚴,則文武百官咸服之。車駕親戎,則縛褲,不舒散也。中官紫褶,外官絳褶,腰皮帶,以代鞶革。



 笏,中世以來,唯八座尚書執笏。笏者白筆綴其頭,以紫囊裹之。其餘公卿,但執手版。荷紫者,以紫生為裌囊,綴之服外,加於左肩。周遷云:「昔周公負成王,制此衣,至今以為朝服。」蕭驕子云:「名契囊。」案《趙充國傳》云:「張子孺持囊簪筆,事孝武帝。」張晏云:「囊,契囊也。近臣負囊簪筆,從備顧問,有所記也。」



 入殿門,有籠冠者著之,有纓則下之。緣廂行,得提衣。省閣內得著履、烏紗帽。入齋閣及橫度殿庭,不得人提衣及捉服飾。入閣則執手板,自摳衣。幾席不得入齋正閣。介幘不得上正殿及東西堂。儀仗傘扇,有幰牽車,不得入臺門。臺官問訊皇太子,亦皆硃服,著襪;謁諸王,單衣,幘;庶姓,單衣,帢。詣三公,必衣帢。至黃閣,下履,過閣還,著履。



 古者君臣佩玉,尊卑有序,綬者,所以貫佩相承受也。又上下施「X,如蔽膝,貴賤亦各有殊。五霸之後,戰兵不息,佩非兵器,「X非戰儀,於是解去佩「X,留其系禭而已。「X
 佩既廢,秦乃以採組連結於禭,轉相結受,又謂之綬。漢承用之。至明帝始復制佩,而漢末又亡絕。魏侍中王粲識其形,乃復造焉。今之佩,粲所制也。



 皇后謁廟,服袿衣屬大衣,蓋嫁服也,謂之褘衣,皁上皁下。親蠶則青上縹下。



 皆深衣制,隱領袖,緣以條。首飾則假髻、步搖,俗謂之珠松是也。簪珥步搖,以黃金為山題,貫白珠,為桂枝相繆。八爵九華,熊、獸、赤羆、天鹿、闢邪、南山豐大特六獸。諸爵獸皆以翡翠為華。綬佩同乘輿。



 貴妃、貴嬪、貴姬,是為三夫人,金章龜紐,紫綬,八十首佩于闐玉,獸頭鞶。



 淑嬡、淑儀、淑容、昭華、昭儀、昭容、修華、修儀,修容,是為九嬪,金章龜鈕,青綬,八十首獸頭鞶,佩採瓄玉。



 婕妤、容華、充華、承徽、列榮五職,亞九嬪,銀印珪鈕,艾綬,獸頭鞶。



 美人、才人、良人三職,散位,銅印環鈕,墨綬,獸頭鞶。



 皇太子妃,金璽龜鈕,纁硃綬,一百六十首佩瑜玉,獸頭鞶。



 良娣,銀印珪鈕,佩採瓄玉,青綬,八十首獸爪鞶。



 保林,銀印珪鈕,佩水蒼玉,青綬,八十首獸爪鞶。



 諸王太妃、妃、諸長公主、公主、封君,金印龜鈕,紫綬,八十首佩山玄玉,獸頭鞶。



 開國公、侯太夫人,銀印珪鈕,青綬,八十首佩水蒼玉,獸頭鞶。



 公主、三夫人,大手髻,七鈿蔽髻。九嬪及公夫人,五鈿;世婦,三鈿。其長公主得有步搖。公主、封君已上,皆帶綬。以彩組為緄帶,各以其綬色。金闢邪,首為帶玦。



 公、特進、列侯、卿、校、中二千石夫人,紺繒幗,黃金龍首銜白珠,魚須擿,長一尺,為簪珥。入廟佐祭者,皁絹上下,助蠶者,縹絹上下,皆深衣制,緣自二千石夫人已上至皇后,皆以蠶衣為朝服。



 自晉左遷,中原禮儀多缺。後魏天興六年,詔有司始制
 冠冕,各依品秩,以示等差,然未能皆得舊制。至太和中,方考故實,正定前謬,更造衣冠,尚不能周洽。



 及至熙平二年,太傅、清河王懌、黃門侍郎韋廷祥等,奏定五時朝服,準漢故事,五郊衣幘,各如方色焉。及後齊因之。河清中,改易舊物,著令定制云。



 乘輿,平冕,黑介幘,垂白珠十二旒,飾以五採玉,以組為纓,色如其綬,黈纊,玉笄。白玉璽,黃赤綬,五採,黃赤縹綠紺,純黃質,長二丈九尺,五百首,廣一尺二寸。小綬長三尺二寸,與綬同採,而首半之。袞服,皁衣,絳裳,裳前三幅,後四幅,織成為之,十二章,緣絳中單,織成緄帶,硃紱,佩
 白玉,帶鹿盧劍,絳褲襪,赤舄。未加元服,則空頂介幘。又有通天金博山冠,則絳紗袍,皁緣中單。



 其五時服,則五色介幘,進賢五梁冠,五色紗袍。又有遠游五梁冠,並不通於下。



 四時祭廟、圓丘、方澤、明堂、五郊、封禪、大雩、出宮行事、正旦受朝及臨軒拜王公,皆服袞冕之服。還宮及齋,則服通天冠。籍田則冠冕,璪十二旒,佩蒼玉,黃綬,青帶,青襪,青舄。拜陵則黑介幘,白紗單衣。釋奠則服通天金博山冠,玄紗袍。春分朝日,則青紗朝服,青舄,秋分夕月,則白紗朝服,緗舄,俱冠五梁進賢冠。合朔,服通天金博山冠,絳紗袍。季秋講武、出征告廟,冠武弁,黃金附蟬,
 左貂。祃類宜社,武弁,硃衣。纂嚴升殿,服通天金博山冠,絳紗袍。入溫、涼室,冠武弁,右貂附蟬,絳紗服。徵還飲至,服通天冠。廟中遣上將,則袞冕,還宮則通天金博山冠。賞祖罰社,則武弁,左貂附蟬。元日、冬至大小會,皆通天金博山冠。四時畋、出宮,服通天冠,並赤舄。明堂則五時俱通天冠,各以其色服。東、西堂舉哀,服白帢。



 天子六璽:文曰「皇帝行璽」,封常行詔敕則用之。「皇帝之璽」,賜諸王書則用之。「皇帝信璽」,下銅獸符,發諸州征鎮兵,下竹使符,拜代徵召諸州刺史,則用之。並白玉為之,方一寸二分,螭獸鈕。「天子行璽」,封拜外國則用之。



 「天子
 之璽」,賜諸外國書則用之。「天子信璽」,發兵外國,若徵召外國,及有事鬼神,則用之。並黃金為之,方一寸二分,螭獸鈕。又有傳國璽,白玉為之,方四寸,螭獸鈕,上交五蟠螭,隱起鳥篆書。文曰「受天之命,皇帝壽昌」,凡八字。



 在六璽外,唯封禪以封石函。又有督攝萬機印一鈕,以木為之,長一尺二寸,廣二寸五分。背上為鼻鈕,鈕長九寸,厚一寸,廣七分。腹下隱起篆書為「督攝萬機」,凡四字。此印常在內,唯以印籍縫。用則左戶郎中、度支尚書奏取,印訖輸內。



 皇太子平冕,黑介幘,垂白珠九旒,飾以三採玉,以組為
 纓,色如其綬,金璽,硃綬,四採,赤黃縹紺。綬硃質,長二丈一尺,三百二十首,廣九寸。小綬長三尺二寸,與綬同色,而首半之。袞服,同乘輿而九章,絳紱,佩瑜玉,玉具劍、火珠標首,絳褲襪,赤舄。非謁廟則不服。未加元服,則空頂黑介幘,雙童髻,雙玉導。



 中舍人執遠游冠以從。其遠游三梁冠,黑介幘,翠緌纓,絳紗袍,皁緣中單,黑舄。



 大朝所服,亦服進賢三梁冠,黑介幘,皁朝服,絳緣中單,玄舄。為宮臣舉哀,白帢,單衣,烏皮履。未加元服,則素服。



 皇太子璽,黃金為之,方一寸,龜鈕,文曰「皇太子璽」。宮中大事用璽,小事用門下典書坊印。



 諸公卿平冕,黑介幘,青珠為旒,上公九,三公八,諸卿六,以組為纓,色如其綬。衣皆玄上纁下。三公山龍八章,降皇太子一等,九卿藻火六章,唯郊祀天地宗廟服之。



 遠游三梁,諸王所服。其未冠,則空頂黑介幘。開國公、侯、伯、子、男及五等散爵未冠者,通如之。



 進賢冠,文官二品已上,並三梁,四品已上,並兩梁,五品已下,流外九品已上,皆一梁。致事者,通著委貌冠。主兵官及侍臣,通著武弁。侍臣加貂璫。御史、大理著法冠。諸謁者、太子中導客舍人,著高山冠。宮門僕射、殿門吏、亭長、太子率更寺、宮門督、太子內坊察非吏、諸門吏等,皆
 著卻非冠。羽林、武賁,著鶡冠。錄令已下,尚書以上,著納言幘。又有赤幘,卑賤者所服。救日蝕,文武官皆免冠,著赤介幘,對朝服。賤者平巾,赤幘,示威武,以助於陽也。止雨亦服之。



 請雨則服緗幘,東耕則服青幘,庖人則服綠幘。



 印綬,二品已上,並金章,紫綬;三品銀章,青綬;三品已上,凡是五省官及中侍中省,皆為印,不為章。四品得印者,銀印,青綬;五品、六品得印者,銅印,墨綬,四品已下,凡是開國子、男及五等散品名號侯,皆為銀章,不為印。七品、八品、九品得印者,銅印,黃綬。金銀章印及銅印,並方一寸,皆龜鈕。東西南北四籓諸國王章,上籓用中金,中籓用
 下金,下籓用銀,並方寸,龜鈕。佐官唯公府長史、尚書二丞,給印綬。六品已下,九品已上,唯當曹為官長者給印。餘自非長官,雖位尊,並不給。



 諸王纁硃綬,四採,赤黃縹紺,純硃質,纁文織,長二丈一尺,二百四十首,廣九寸。開國郡縣公、散郡縣公,玄硃綬,四採,玄赤縹紺,硃質,玄文織,長一丈八尺,百八十首,廣八寸。開國縣侯伯、散縣侯伯,青硃綬,四採,青赤白縹,硃質,青文織,長一丈六尺,百四十首,廣七寸。開國縣子男、散縣子男、名號侯、開國鄉男,素硃綬,三採,青赤白,硃質,白文織,長一丈四尺,百二十首,廣六寸。一品、二品,紫綬,
 三採,紫黃赤,純紫質,長一丈八尺,百八十首,廣八寸。



 三品、四品,青綬,三採,青白紅,純青質,長一丈六尺,百四十首,廣七寸。五品、六品,墨綬,二採,青紺,純紺質,長一丈四尺,百首,廣六寸。七品、八品、九品,黃綬,二採,黃白,純黃質,長一丈二尺,六十首,廣五寸。官品從第二已上,小綬間得施玉環。凡綬,先合單紡為一絲,絲四為一扶,扶五為一首,首五成一文。採純為質。首多者絲細,首少者絲粗。官有綬者,則有紛,皆長八尺,廣三寸,各隨綬色。若服朝服則佩綬,服公服則佩紛。官無綬者,不合佩紛。



 鞶囊,二品已上金縷,三品金銀縷,四品銀縷,五品、六品
 彩縷,七、八、九品彩縷,獸爪鞶。官無印綬者,並不合佩鞶囊及爪。



 一品,玉具劍,佩山玄玉。二品,金裝劍,佩水蒼玉。三品及開國子男、五等散品名號侯雖四、五品,並銀裝劍,佩水蒼玉。侍中已下,通直郎已上,陪位則像劍。帶真劍者,入宗廟及升殿,若在仗內,皆解劍。一品及散郡公、開國公侯伯,皆雙佩。二品、三品及開國子男、五等散品名號侯,皆雙佩。綬亦如之。



 百官朝服公服,皆執手板。尚書錄令、僕射、吏部尚書,手板頭復有白筆,以紫皮裹之,名曰笏。朝服綴紫荷,錄令、
 左僕射左荷,右僕射、吏部尚書右荷。七品已上文官朝服,皆簪白筆。正王公侯伯子男、卿尹及武職,並不簪。朝服,冠、幘各一,絳紗單衣,白紗中單,皁領袖,皁襈,革帶,曲領,方心,蔽膝,白筆、舄、襪,兩綬,劍佩,簪導,鉤灊,為具服。七品已上服也。公服,冠、幘,紗單衣,深衣,革帶,假帶,履襪,鉤灊,謂之從省服。八品已下,流外四品已上服也。



 流外五品已下,九品已上,皆著褠衣為公服。



 皇后璽、綬、佩同乘輿,假髻,步搖,十二鈿,八雀九華。助祭朝會以褘衣,祠郊禖以褕狄,小宴以闕狄,親蠶以鞠衣,禮見皇帝以展衣,宴居以褖衣。六服俱有蔽膝、織成緄
 帶。皇太后、皇后璽,並以白玉為之,方一寸二分,螭獸鈕,文各如其號。璽不行用,有令,則太后以宮名衛尉印,皇后則以長秋印。



 內外命婦從五品已上,蔽髻,唯以鈿數花釵多少為品秩。二品已上金玉飾,三品已下金飾。內命婦、左右昭儀、三夫人視一品,假髻,九鈿,金章,紫綬,服褕翟,雙佩山玄玉。九嬪視三品,五鈿蔽髻,銀章,青綬,服鞠衣,佩水蒼玉。世婦視四品,三鈿,銀印,青綬,服展衣,無佩。八十一御女視五品,一鈿,銅印,墨綬,服褖衣。又有宮人女官服制,第二品七鈿蔽髻,服闕翟;三品五鈿,鞠衣;四品三鈿,展衣;
 五品一鈿,褖衣;六品褖衣;七品青紗公服。俱大首髻。八品、九品,俱青紗公服,偏髾髻。



 皇太子妃璽、綬、佩同皇太子,假髻,步搖,九鈿,服褕翟。從蠶則青紗公服。



 皇太子妃璽,以黃金,方一寸,龜鈕,文曰「皇太子妃之璽」。若有封書,則用內坊印。



 郡長公主、公主、王國太妃、妃,纁硃綬,髻章服佩同內命婦一品。郡長君七鈿蔽髻,玄硃綬,闕翟,章佩與公主同。郡君、縣主,佩水蒼玉,餘與郡長君同。



 太子良娣視九嬪服。縣主青硃綬,餘與良娣同。女侍中五鈿,假金印、紫綬,
 服鞠衣,佩水蒼玉。縣君銀章,青硃綬,餘與女侍中同。太子孺人同世婦。太子家人子同御女。鄉主、鄉君,素硃綬,佩水蒼玉,餘與御女同。外命婦章印綬佩,皆如其夫。若夫假章印綬佩,妻則不假。一品、二品,七鈿蔽髻,服闕翟。三品五鈿,服鞠衣。四品三鈿,服展衣。五品一鈿,服褖衣。內外命婦、宮人女官從蠶,則各依品次,還著蔽髻,皆服青紗公服。如外命婦,綬帶鞶囊,皆準其夫公服之例。百官之母詔加太夫人者,朝服公服,各與其命婦服同。



 後周設司服之官,掌皇帝十二服。祀昊天上帝,則蒼衣蒼冕;祀東方上帝及朝日,則青衣青冕;祀南方上帝,則
 硃衣硃冕;祭皇地祇、祀中央上帝,則黃衣黃冕;祀西方上帝及夕月,則素衣素冕;祀北方上帝,祭神州、社稷,則玄衣玄冕;享先皇、加元服、納后、朝諸侯,則象衣象冕。十有二章,日月星辰山龍華蟲六章在衣,火宗彞藻粉米黼黻六章在裳,凡十二等。享諸先帝、大貞於龜、食三老五更、享諸侯、耕籍,則服袞冕,自龍已下,凡九章十二等。宗彞已下五章在衣,藻、火已下四章在裳,衣重宗彞。祀星辰、祭四望、視朔、大射、饗群臣、巡犧牲、養國老,則服山冕,八章十二等。衣裳各四章,衣重火與宗彞。群祀、視朝、臨太學、入道法門、宴諸侯與群臣及燕射、養庶老、適諸
 侯家,則服鷩冕,七章十二等。衣三章,裳四章,衣重三章。袞、山、鷩三冕,皆裳重黼黻,俱十有二等。通以升龍為領褾。



 冕通十有二旒。巡兵即戎,則服韋弁,謂以韎韋為弁,又以為裳衣也。田獵行鄉畿,則服皮弁,謂以鹿子皮為弁,白布衣而素裳也。皇帝兇服斬衰。父母之喪上下達其吊服,錫衰以哭三公,緦衰以哭諸侯,皆十五升抽其半。錫者,浣其布,不浣其縷,哀在內,緦者皆素弁,如爵弁之數環絰。一服纏絰。凡大疫、大荒、大災則素服縞冠。凡疫病、荒饑、年災水旱也。



 諸公之服九:一曰方冕。二曰袞冕,九章,宗彞已上五章在衣,藻已下四章在裳。三曰山冕,八章,衣裳各四章,衣
 重宗彞,為九等。四曰鷩冕,七章,衣三章,裳四章,衣重火與宗彞。五曰火冕,六章,衣裳各三章,衣重宗彞及藻,裳重黻。



 六曰毳冕,五章,衣三章,裳二章,衣重藻粉米,裳重黼黻。山冕已下俱九等,皆以山為領褾,冕俱九旒。七曰韋弁。八曰皮弁。九曰玄冠。



 諸侯服,自方冕而下八,無袞冕。山冕八章,衣裳各四章。鷩冕七章,衣三章,裳四章,衣重宗彞。火冕六章,衣裳各三章,衣重藻,裳重黻。毳冕五章,衣三章,裳二章,衣重粉米,裳重黼黻。鷩冕已下俱八等,皆以華蟲為領褾。冕俱八旒。



 諸伯服,自方冕而下七,又無山冕。鷩冕七章,衣三章,裳四章。火冕六章,衣裳各三章,裳重黻。毳冕五章,衣三章,裳二章,裳重黼黻。火冕已下俱七等,皆以火為領褾。冕俱七旒。



 諸子服,自方冕而下六,又無鷩冕。火冕六章,衣裳各三章。毳冕五章,衣三章,裳二章,裳重黻。毳冕已下俱六等,皆以宗彞為領褾。冕俱六旒。



 諸男服,自方冕而下五,又無火冕。毳冕五章,衣三章,裳二章。以藻為領褾。



 冕五旒。



 三公之服九:一曰祀冕。二曰火冕,六章,衣裳各三章,衣
 重宗彞與藻,裳重黻。三曰毳冕,五章,衣三章,裳二章,衣重藻與粉米,裳重黼黻。四曰藻冕,四章,衣裳俱二章,衣重藻與粉米,裳重黼黻。五曰繡冕,三章,衣一章,裳二章,衣重粉米,裳重黼黻。俱九等,皆以宗彞為領褾。六曰爵弁。七曰韋弁。八曰皮弁。



 九曰玄冠。



 三孤之服,自祀冕而下八,無火冕。毳冕五章,衣三章,裳二章,衣重粉米,裳重黼黻。藻冕四章,衣裳各二章,衣重藻與粉米,裳重黼黻,俱八等,皆以藻為領褾。繡冕三章,衣一章,裳二章,衣重粉米,裳重黼黻,為八等。



 公卿之服,自祀冕而下七,又無毳冕。藻冕四章,衣裳各
 二章,衣重粉米,裳重黼黻,為七等,皆以粉米為領褾,各七。繡冕三章,衣一章,裳二章,衣重粉米,裳重黼黻,為七等。



 上大夫之服,自祀冕而下六,又無藻冕。繡冕三章,衣一章,裳二章,衣重粉米,裳重黼,為六等。



 中大夫之服,自祀冕而下五,又無皮弁。繡冕三章,衣一章,裳二章,衣重粉米,為五等。



 下大夫之服,自祀冕而下四,又無爵弁。繡冕三章,衣一章,裳二章,衣重粉米,為四等。



 士之服三:一曰祀弁,二曰爵弁,三曰玄冠。玄冠皆玄衣。其裳,上士以
 玄,中士以黃,下士雜裳,謂前玄後黃也。庶士之服一:玄冠。庶士,庶人在官,府史之屬。其服緇衣裳。



 後令文武俱著常服,冠形如魏帢,無簪有纓。其兇服皆與庶人同。其吊服,諸侯於其卿大夫,錫衰;同姓,緦衰;於士,疑衰。其當事則弁絰,否則皮弁。公孤卿大夫之吊服,錫衰弁絰,皮弁亦如之。士之吊服,疑衰素裳,當事弁絰,否則徒弁。



 皇后衣十二等。其翟衣六,從皇帝祀郊禖,享先皇,朝皇太后,則服翬衣。素質,五色。祭陰社,朝命婦,則服騑衣。青質,五色。祭群小祀,受獻繭,則服鷩衣。赤色採桑則服鳪衣。黃色從皇帝見賓客,聽女教,則服鵫衣。白色食命婦,歸寧,則服衣。
 玄色俱十有二等,以翬雉為領褾,各有二。臨婦學及法道門,燕命婦,有時見命婦,則蒼衣。春齋及祭還,則青衣。夏齋及祭還,則硃衣。採桑齋及採桑還,則黃衣。秋齋及祭還,則素衣。冬齋及祭還,則玄衣。自青衣而下,其領褾以相生之色。



 諸公夫人九服,其翟衣雉皆九等,俱以騑雉為領褾,各九。自騑衣已下五,曰騑衣、鷩衣、鳪衣、鵫衣、衣,並硃衣、黃衣、素衣、玄衣而九。自硃衣而下,其領褾亦同用相生之色。



 諸侯夫人,自鷩而下八。其翟衣雉皆八等,俱以鷩雉為
 領褾。無騑衣。



 諸伯夫人,自鳪而下七。其翟衣雉皆七等,俱以鳪雉為領襟,又無鵫衣。



 諸子夫人,自鵫而下六。其翟衣俱以鵫雉為領褾。又無鳪衣。



 諸男夫人,自而下五。其翟衣雉皆五等,俱以雉為領褾。又無鳪衣。



 三妃,三公夫人之服九:一曰鳪衣,二曰鵫衣、三曰衣,四曰青衣,五曰硃衣,六曰黃衣,七曰素衣,八曰玄衣,九曰鷩衣。似發華皆九樹。其雉衣亦皆九等,以褾雉為領褾,
 各九。



 三弋,三孤之內子,自鵫衣而下八。雉衣皆八等,以鵫雉為領褾,各八。



 六嬪,六卿之內子,自衣而下七。雉衣皆七等,以雉為領褾,各七。



 上媛,上大夫之孺人,自青衣而下六。



 中媛,中大夫之孺人,自硃衣而下五。



 下媛,下大夫之孺人,自黃衣而下四。



 御婉士之婦人,自素衣而下三。



 中宮六尚,緅衣。其色赤而微玄諸命秩之服,曰公服,其餘
 常服,曰私衣。皇後華皆有十二樹。諸侯之夫人,亦皆以命數為之節。三妃,三公夫人已下,又各依其命。一命再命者,又俱以三為節。



 皇后及諸侯夫人之服,皆舄履。三妃,三公夫人已下,翟衣則舄,其餘皆屨。



 舄、履各如其裳之色。



 皇后之兇服,斬衰、齊衰,降旁期已下吊服。為妃、嬪、三公之夫人、孤卿內子之喪,錫衰。錫者,十五升去其半。無事其縷,有事其布,哀在內也。為諸侯夫人之喪,緦衰。緦亦十五升去其半。有事其縷,無事其布,哀在外也。為媛、御婉及大夫孺人、士之婦人之喪,疑衰。十四升,疑於吉。皆吉笄,無首。象笄,去首飾。太陰虧則素服。蕩天下之陰事諸侯之夫人及三
 妃與三公之夫人已下兇事,則五衰:自緦已上皆服之其吊,諸侯夫人於卿之內子、大夫孺人,錫衰。於己之同姓之臣,緦衰。於士之婦人,疑衰。皆吉笄,無首。其三妃已下及媛,三公夫人已下及孺人,其吊服錫衰。御婉及士之婦人,吊服疑衰,疑衰同笄。九族已下皆骨笄韠,皇帝三章,龍、火、山;諸侯二章,去龍;卿大夫一章,以山。皆織彩以成之。



 皇帝八璽,有神璽,有傳國璽,皆寶而不用。神璽明受之於天,傳國璽明受之於運。皇帝負扆,則置神璽於筵前之右,置傳國璽於筵前之左。又有六璽。其一「皇帝行璽」,封命諸侯及三公
 用之。其二「皇帝之璽」,與諸侯及三公書用之。



 其三「皇帝信璽」,發諸夏之兵用之。其四「天子行璽」,封命蕃國之君用之。其五「天子之璽」,與蕃國之君書用之。其六「天子信璽」,徵蕃國之兵用之。六璽皆白玉為之,方一寸五分,高寸,螭獸鈕。



 皇后璽,文曰「皇后之璽」,白玉為之,方寸五分,高寸,麟鈕。



 三公諸侯皆金印,方寸二分,高八分,龜鈕。七命已上銀,四命已上銅,皆龜鈕。三命已上,銅印銅鼻。其方皆寸,其高六分,文曰「某公官之印」。



 皇帝之組綬以蒼,以青,以硃,以黃,以白,以玄,以纁,以紅,以紫,以緅,以碧,以綠,十有二色。諸公九色,自黃以下。諸
 侯八色,自白以下。諸伯七色,自玄以下。諸子六色,自纁已下。諸男五色,自紅已下。三公之綬,如諸公。三孤之綬,如諸侯。六卿之綬,如諸伯。上大夫之綬,如諸子。中大夫之綬,如諸男。



 下大夫綬,自紫已下。士之綬,自緅已下。其璽印之綬,亦如之。



 保定四年,百官始執笏,常服上焉。宇文護始命袍加下襴。



 宣帝即位,受朝於路門,初服通天冠,絳紗袍。群臣皆服漢魏衣冠。大象元年,制冕二十四旒,衣服以二十四章為準。二年下詔,天臺近侍及宿衛之官,皆著五色衣,以錦綺繢繡為緣,名曰品色衣。有大禮則服冕。內外命
 婦皆執笏,其拜俯伏方興。



\end{pinyinscope}