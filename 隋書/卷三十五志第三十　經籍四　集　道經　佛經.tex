\article{卷三十五志第三十 經籍四 集 道經 佛經}

\begin{pinyinscope}

 《楚辭》十二卷並目錄。後漢校書郎王逸注。



 《楚辭》三卷郭璞注。梁有《楚辭》十一卷,宋何偃刪王逸注,亡。



 《楚辭九悼》一卷楊穆撰。



 《參解楚辭》七卷皇甫遵訓撰。



 《楚辭音》一卷徐邈撰。



 《楚辭音》一卷宋處士諸葛氏撰。



 《楚辭音》一卷孟奧撰。



 《楚辭音》一卷《楚辭音》一卷釋道騫撰。



 《離騷草木疏》二卷劉杳撰。



 右十部,二十九卷。通計亡書,十一部,四十卷。



 《
 楚辭》者,屈原之所作也。自周室衰亂,詩人寢息,謅佞之道興,諷刺之辭廢。楚有賢臣屈原,被讒放逐,乃著《離騷》八篇,言己離別愁思,申杼其心,自明無罪,因以諷諫,冀君覺悟,卒不省察,遂赴汨羅死焉。弟子宋玉,痛惜其師,傷而和之。其後,賈誼、東方朔、劉向、揚雄,嘉其文彩,擬之而作。蓋以原楚人也,謂之「楚辭」。然其氣質高麗,雅致清遠,後之文人,咸不能逮。始漢武帝命淮南王為之章句,旦受詔,食時而奏之,其書今亡。後漢校書郎王逸,集屈原已下,迄於劉向,逸又自為一篇,並敘而注之,今行於世。隋時有釋道騫,善讀之,能為楚聲,音韻清切,至今傳《
 楚辭》者,皆祖騫公之音。



 楚蘭陵令《荀況集》一卷殘缺。梁二卷。



 楚大夫《宋玉集》三卷《漢武帝集》一卷梁二卷。



 《漢淮南王集》一卷梁二卷。又有《賈誼集》四卷,《晁錯集》三卷,漢弘農都尉《枚乘集》二卷,錄各一卷,亡。



 漢中書令《司馬遷集》一卷漢太中大夫《東方朔集》二卷梁有漢光祿大夫《吾丘壽王集》二卷,亡。



 漢孝文園令《司馬相如集》一卷漢膠西相《董仲舒集》一卷梁二卷。又有漢太常《孔臧集》二卷,亡。



 漢騎都尉《李陵集》二卷梁有漢丞相《魏相集》二卷,錄一卷;左馮翊《張敞集》一卷,錄一卷。亡。



 漢諫議大夫《王褒集》五卷漢諫議大夫《劉向集》六卷梁有漢射聲校尉《陳湯集》二卷,丞相《韋玄成集》二卷,亡。



 漢諫議大夫《谷永集》二卷梁有涼州刺史《杜鄴集》二卷,騎都尉《李尋集》二卷,亡。



 漢司空《師丹
 集》一卷梁三卷,錄一卷。



 漢光祿大夫《息夫躬集》一卷漢太中大夫《揚雄集》五卷漢太中大夫《劉歆集》五卷漢成帝《班婕妤集》一卷梁有《班昭集》三卷,王莽建新大尹《崔篆集》一卷,保成師友《唐林集》一卷,中謁者《史岑集》二卷,後漢《東平王蒼集》五卷,《桓譚集》五卷,亡。



 後漢司隸從事《馮衍集》五卷後漢徐令《班彪集》二卷梁五卷。又有司徒掾《陳元集》一卷,《王隆集》二卷,雲陽令《硃勃集》二卷,後漢處士《梁鴻集》二卷,亡。



 後漢車騎從事《杜篤集》一卷後漢車騎司馬《傅毅集》二卷梁五卷。



 後漢大將軍護軍司馬《班固集》十七卷梁有魏郡太守《黃香集》二卷,亡。



 後漢長岑長《崔駰集》十卷後漢侍中《賈逵集》一卷梁二卷。



 後漢校書郎《劉騊駼集》一卷梁二卷,錄一卷。又有樂安相《李尤集》五卷,大鴻臚《竇章集》二卷,亡。



 後漢濟北相《
 崔瑗集》六卷梁五卷。



 後漢《劉珍集》二卷錄一卷。



 後漢河間相《張衡集》十一卷梁十二卷,又一本十四卷。又有郎中《蘇順集》二卷,錄二卷;後漢太傅《胡廣集》二卷,錄一卷。亡。



 後漢黃門郎《葛龔集》六卷梁五卷,一本七卷。



 後漢司空《李固集》十二卷梁十卷。



 後漢南郡太守《馬融集》九卷梁有外黃令《高彪集》二卷,錄一卷;《王逸集》二卷,錄一卷;司徒掾《桓麟集》二卷,錄一卷。亡。



 後漢徵士《崔琦集》一卷梁二卷。又有《酈炎集》二卷,錄二卷;陳相《邊韶集》一卷,錄一卷;益州刺史《硃穆集》二卷,錄一卷。亡。



 後漢京兆尹《延篤集》一卷梁二卷,錄一卷。又有司農卿《皇甫規集》五卷;太常卿《張奐集》二卷,錄一卷;《王延壽集》三卷;五原太守《崔寔集》二卷,錄一卷;上計《趙臺集》二卷,錄一卷。亡。



 後漢諫議大夫《劉陶集》三卷梁二卷,錄一卷。又有外黃令《張升集》二卷,錄一卷;《侯瑾集》二卷,《盧植集》二卷,議郎《廉品集》二卷。亡。



 後漢司空《荀爽集》一卷梁三卷,錄一卷。



 後漢野王
 令《劉梁集》三卷梁三卷,錄一卷。又有《鄭玄集》二卷,錄一卷,亡。



 後漢左中郎將《蔡邕集》十二卷梁有二十卷,錄一卷。又有尚書令《士孫瑞集》二卷,亡。



 後漢太山太守《應劭集》二卷梁四卷。又有別部司馬《張超集》五卷,亡。



 後漢少府《孔融集》九卷梁十卷,錄一卷。



 後漢侍御史《虞翻集》二卷梁三卷,錄一卷。



 後漢討虜長史《張紘集》一卷梁二卷,錄一卷。梁有後漢處士《禰衡集》二卷,錄一卷,亡。



 後漢尚書右丞《潘勖集》二卷梁有錄一卷,亡。



 後漢丞相倉曹屬《阮瑀集》五卷梁有錄一卷,亡。



 魏太子文學《徐幹集》五卷梁有錄一卷,亡。



 魏太子文學《應瑒集》一卷梁有五卷,錄一卷,亡。



 後漢丞相軍謀掾《陳琳集》三卷梁十卷,錄一卷。



 魏太子文學《劉楨集》四卷錄一卷。



 後漢丞相主簿《繁欽集》十卷梁錄一卷,亡。



 後漢丞相主簿《楊修集》一
 卷梁二卷,錄一卷。



 後漢侍中《王粲集》十一卷梁有魏國郎中令《路粹集》二卷,錄一卷,行御史大夫《袁渙集》五卷,錄一卷;魏國奉常《王修集》二卷。亡。



 後漢尚書《丁儀集》一卷梁二卷,錄一卷。



 後漢黃門郎《丁暠集》一卷梁二卷,錄一卷。梁又有婦人後漢黃門郎秦嘉妻《徐淑集》一卷,後漢董祀妻《蔡文姬集》一卷,傅石甫妻《孔氏集》一卷,亡。



 《魏武帝集》二十六卷梁三十卷,錄一卷。梁又有《武皇帝逸集》十卷。亡。



 《魏武帝集新撰》十卷《魏文帝集》十卷梁二十三卷。



 《魏明帝集》七卷梁五卷,或九卷,錄一卷。梁又有《高貴鄉公集》四卷,亡。



 魏《陳思王曹植集》三十卷梁又有司徒《華歆集》二卷,亡。



 魏司徒《王朗集》三十四卷梁三十卷。又司空《陳群集》五卷,亡。



 魏給事中《邯鄲淳集》二卷梁有錄一卷。又有《劉暠集》二卷,侍中《吳質集》五卷,新城太守《孟達集》三卷,魏徵士《管寧集》三卷,錄一卷,亡。



 魏光祿勛《高堂隆集》六卷梁十卷,錄一卷。又有光祿勛《劉邵集》二卷,錄一卷,亡。



 魏
 散騎常侍《繆襲集》五卷梁有錄一卷。又有散騎常侍《王象集》一卷;光祿大夫《韋誕集》三卷,錄一卷;散騎常侍《麋元集》五卷;游擊將軍《卞蘭集》二卷,錄一卷;顯陽侯《李康集》二卷,錄一卷;陳郡太守《孫該集》二卷,錄一卷;尚書《傅巽集》二卷,錄一卷。亡。



 魏章武太守《殷褒集》一卷梁二卷。



 魏司空《王昶集》五卷梁有錄一卷。



 魏衛將軍《王肅集》五卷梁有錄一卷。又有《桓範集》二卷,中領軍《曹羲集》五卷,錄一卷,亡。



 魏尚書《何晏集》十一卷梁十卷,錄一卷。



 魏衛尉卿《應璩集》十卷梁有錄一卷。又有《王弼集》五卷,錄一卷;中書令《劉階集》二卷;太常卿《傅嘏集》二卷,錄一卷,樂安太守《夏侯惠集》二卷,錄一卷。亡。



 魏校書郎《杜摯集》二卷梁有《毌丘儉集》二卷,錄一卷;征東軍司馬《江奉集》二卷。亡。



 魏太常《夏侯玄集》三卷梁有車騎將軍《鐘毓集》五卷,錄一卷,亡。



 魏步兵校尉《阮籍集》十卷梁十三卷,錄一卷。



 魏中散大夫《嵇康集》十三卷梁十五卷,錄一卷。又有魏徵士《
 呂安集》二卷,錄一卷,亡。



 魏司徒《鐘會集》九卷梁十卷,錄一卷。



 魏汝南太守《程曉集》二卷梁錄一卷。



 蜀丞相《諸葛亮集》二十五卷梁二十四卷。又有蜀司徒《許靖集》二卷,錄一卷;征北將軍《夏侯霸集》二卷。亡。



 吳輔義中郎將《張溫集》六卷梁有《士燮集》五卷,亡。



 吳偏將軍《駱統集》十卷梁有錄一卷。又有太子少傅《薛綜集》三卷,錄一卷,亡。



 吳選曹尚書《暨艷集》二卷梁三卷,錄一卷。又有《姚信集》二卷,錄一卷;《謝承集》四卷。



 今亡。



 吳人《楊厚集》二卷梁又有錄一卷。



 吳丞相《陸凱集》五卷梁有錄一卷。



 吳侍中《胡綜集》二卷梁有錄一卷。又有東觀令《華核集》五卷,錄一卷,亡。



 吳侍中《張儼集》一卷梁二卷,錄一卷。又有《韋昭集》二卷,錄一卷,亡。



 吳中書令《紀騭集》三卷梁有錄一卷。又有《陸景集》一卷。亡。



 《晉宣帝集》五卷梁有錄一卷。



 《晉文帝集》三卷《齊王攸集》二卷梁三卷。



 晉《王沈集》
 五卷梁有《鄭褒集》二卷,亡。



 晉宗正《嵇喜集》一卷殘缺。梁二卷,錄一卷。



 晉散騎常侍《應貞集》一卷梁五卷。



 晉司隸校尉《傅玄集》十五卷梁五十卷,錄一卷,亡。



 晉著作郎《成公綏集》九卷殘缺。梁十卷。又有《裴秀集》三卷,錄一卷,亡。



 晉金紫光祿大夫《何楨集》一卷梁五卷。又有《袁準集》二卷,錄一卷,亡。



 晉少傅《山濤集》九卷梁五卷,錄一卷,又一本十卷。齊奉朝請裴津注。又梁有《向秀集》二卷,錄一卷;平原太守《阮種集》二卷,錄一卷;《阮侃集》五卷,錄一卷。亡。



 晉太傅《羊祜集》一卷殘缺。梁二卷,錄一卷。又有《蔡玄通集》五卷,太宰《賈充集》五卷,錄一卷;《荀勖集》三卷,錄一卷。亡。



 晉征南將軍《杜預集》十八卷晉輔國將軍《王濬集》一卷殘缺。梁二卷,錄一卷。



 晉徵士《皇甫謐集》二卷錄一卷。



 晉侍中《程咸集》三卷梁有光祿大夫《劉毅集》二卷,錄一卷;晉侍中《庾峻集》二卷,錄一卷。亡。



 晉巴西太守《郤正集》一卷
 晉散騎常侍《薛瑩集》三卷梁又有散騎常侍《陶濬集》二卷,錄一卷,亡。



 晉通事郎《江偉集》六卷梁有《宣舒集》五卷;散騎常侍《曹志集》二卷,錄一卷;《鄒湛集》三卷,錄一卷。亡。



 晉汝南太守《孫毓集》六卷晉處士《楊泉集》二卷錄一卷。梁有司徒《王渾集》五卷,黃州刺史《王深集》五卷,亡。



 晉徵士《閔鴻集》三卷梁有光祿大夫《裴楷集》二卷,錄一卷。亡。



 晉司空《張華集》十卷錄一卷。



 晉尚書僕射《裴頠集》九卷梁有太子中庶子《許孟集》三卷,錄一卷;太宰《何劭集》二卷,錄一卷;光祿大夫《劉頌集》三卷,錄一卷;《劉寔集》二卷,錄一卷。亡。



 晉散騎常侍《王佑集》三卷錄一卷。梁有晉驃騎將軍《王濟集》二卷,亡。



 《華嶠集》八卷梁二卷。



 晉秘書丞《司馬彪集》四卷梁三卷,錄一卷。又有尚書《庾儵集》二卷,錄一卷,國子祭酒《謝衡集》二卷。亡。



 晉漢中太守《李虔集》一卷梁二卷,錄一卷。



 晉司隸校尉《傅咸集》十七卷梁三十卷,錄一卷。又有太子中庶子《
 棗據集》二卷,錄一卷;《劉寶集》三卷。亡。



 晉馮翊太守《孫楚集》六卷梁十二卷,錄一卷。



 晉散騎常侍《夏侯湛集》十卷梁有錄一卷。又有弋陽太守《夏侯淳集》二卷,散騎侍郎《王贊集》五卷,亡。



 晉衛尉卿《石崇集》六卷梁有錄一卷。



 晉尚書郎《張敏集》二卷梁五卷。又有黃門郎《伏偉集》一卷,亡。



 晉黃門郎《潘岳集》十卷晉太常卿《潘尼集》十卷晉頓丘太守《歐陽建集》二卷梁有宗正《劉許集》二卷,錄一卷;散騎常侍《李重集》二卷;光祿大夫《樂廣集》二卷,錄一卷;《阮渾集》三卷,錄一卷。亡。



 晉侍中《嵇紹集》二卷錄一卷。梁有錢唐令《楊建集》九卷,長沙相《盛彥集》五卷,左長史《楊乂集》三卷,錄一卷。



 晉尚書《盧播集》一卷梁二卷,錄一卷。又有《欒肇集》五卷,錄一卷;南中郎長史《應亨集》二卷。亡。



 晉國子祭酒《杜育集》二卷晉太常卿《摯虞集》九卷梁十卷,錄一卷。又秘書監《繆徵集》二卷,錄一卷,亡。



 晉齊王府記室《左思
 集》二卷梁有五卷,錄一卷。又有晉豫章太守《夏靖集》二卷,錄一卷;吳王文學《鄭豐集》二卷,錄一卷;大司馬東曹掾《張翰集》二卷,錄一卷;清河王文學《陳略集》二卷,錄一卷;揚州從事《陸沖集》二卷,錄一卷。亡。



 晉平原內史《陸機集》十四卷梁四十七卷,錄一卷,亡。



 晉清河太守《陸雲集》十二卷梁十卷,錄一卷。又有少府丞《孫極集》二卷,錄一卷,亡。



 晉中書郎《張載集》七卷梁一本二卷,錄一卷。



 晉黃門郎《張協集》三卷梁四卷,錄一卷。



 晉著作郎《束皙集》七卷梁五卷,錄一卷。又有征南司馬《曹攄集》三卷,錄一卷;散騎常侍《江統集》十卷,錄一卷,著作郎《胡濟集》五卷,錄一卷。亡。



 晉中書令《卞粹集》一卷梁五卷。又有光祿勛《閭丘沖集》二卷,錄一卷,亡。



 晉太傅從事中郎《庾斂集》一卷梁五卷,錄一卷。又有太子中舍人《阮瞻集》二卷,錄一卷;太子洗馬《阮修集》二卷,錄一卷;廣威將軍《裴邈集》二卷,錄一卷。亡。



 晉太傅主簿《郭象集》二卷梁五卷,錄一卷。又有廣州刺史《嵇含集》十卷,錄一卷,亡。



 晉安豐太守《
 孫惠集》八卷梁十一卷,錄一卷。又有松滋令《蔡洪集》二卷,錄一卷,亡。



 晉平北將軍《牽秀集》四卷梁三卷,錄一卷。又有車騎從事中郎《蔡克集》二卷,錄一卷;游擊將軍《索靖集》三卷;隴西太守《閻纂集》二卷,錄一卷;秦州刺史《張輔集》二卷,錄一卷;交趾太守《殷巨集》二卷,錄一卷;太子洗馬《陶佐集》五卷,錄一卷;東晉鄱陽太守《虞溥集》二卷,錄一卷;益陽令《吳商集》五卷;《仲長敖集》二卷;晉太常卿《劉弘集》三卷,錄一卷;開府《山簡集》二卷,錄一卷;兗州刺史《宗岱集》二卷;侍中《王峻集》二卷,錄一卷;濟陽內史《王曠集》五卷,錄一卷。亡。



 晉散騎常侍《棗嵩集》一卷梁二卷,錄一卷。又有襄陽太守《棗腆集》二卷,錄一卷,亡。



 晉太尉《劉琨集》九卷梁十卷。



 《劉琨別集》十二卷晉司空從事中郎《盧諶集》十卷梁有錄一卷。



 晉秘書丞《傅暢集》五卷梁有錄一卷。又有《晉明帝集》五卷,錄一卷;《簡文帝集》五卷,錄一卷;《孝武帝集》二卷,錄一卷;《彭城王褷集》二卷,《譙烈王集》九卷,錄一卷。亡。



 晉會稽王《司馬道子集》八卷梁九
 卷。又有鎮東從事中郎《傅毅集》五卷,亡。



 晉衡陽內史《曾瑰集》三卷梁四卷,錄一卷。又有驃騎將軍《顧榮集》五卷,錄一卷,亡。



 晉司空《賀循集》十八卷梁二十卷,錄一卷。又有散騎常侍《張亢集》二卷,錄一卷;車騎長史《賈彬集》三卷,錄一卷。亡。



 晉光祿大夫《衛展集》十二卷梁十五卷。又有東晉太尉《荀組集》三卷,錄一卷,亡。



 晉秘書郎《張委集》九卷梁五卷。又有關內侯《傅氏集》一卷;光祿大夫《周顗集》二卷,錄一卷。亡。



 晉太常《謝鯤集》六卷梁二卷。



 晉驃騎將軍《王暠集》十卷梁三十四卷,錄一卷。又有《華譚集》二卷;亡。



 晉御史中丞《熊遠集》十二卷梁五卷,錄一卷。又有湘州秀才《穀儉集》一卷;大鴻臚《周嵩集》三卷,錄一卷,亡。



 晉弘農太守《郭璞集》十七卷梁十卷,錄一卷。



 晉《張駿集》八卷殘缺。



 晉大將軍《王敦集》十卷梁有吳興太守《沈充集》三卷;散騎常侍《傅純集》二卷,錄一卷。



 亡。



 晉光祿大夫《梅陶集》九卷梁二十卷,錄一卷。又有金紫光祿大夫《荀
 邃集》二卷,錄一卷,亡。



 晉散騎常侍《王鑒集》九卷梁五卷。又有晉著作佐郎《王濤集》五卷;廷尉卿《阮放集》十卷,錄一卷;宗正卿《張悛集》五卷;錄一卷;汝南太守《應碩集》二卷,金紫光祿大夫《張闓集》二卷,錄一卷;揚州從事《陸沈集》二卷,錄一卷;驃騎將軍《卞珣集》二卷,錄一卷;光祿勛《鐘雅集》一卷,衛尉卿《劉超集》二卷;衛將軍《戴邈集》五卷,錄一卷;光祿大夫《荀崧集》一卷,亡。



 晉大將軍《溫嶠集》十卷梁錄一卷。



 晉侍中《孔坦集》十七卷梁五卷,錄一卷。又有《臧沖集》一卷,晉鎮南大將軍《應詹集》五卷,亡。



 晉太僕卿《王嶠集》八卷梁有衛尉《荀愷集》一卷,鎮北將軍《劉隗集》二卷;大司馬《陶侃集》二卷,錄一卷。亡。



 晉丞相《王導集》十一卷梁十卷。錄一卷。



 晉太尉《郗鑒集》十卷錄一卷。



 晉太尉《庾亮集》二十一卷梁二十卷,錄一卷。又有《虞預集》十卷,錄一卷;平越司馬《黃整集》十卷,錄一卷。亡。



 晉護軍長史《庾堅集》十三卷梁十卷,錄一卷。



 晉司空《庾冰集》七卷梁二十卷,錄一卷。



 晉給事中《庾
 闡集》九卷梁十卷,錄一卷。



 晉著作郎《王隱集》十卷梁二十卷,錄一卷。



 晉散騎常侍《干寶集》四卷梁五卷。



 晉太常卿《殷融集》十卷梁有衛尉《張虞集》十卷,光祿大夫《諸葛恢集》五卷,錄一卷。



 亡。



 晉車騎將軍《庾翼集》二十二卷梁二十卷,錄一卷。



 晉司空《何充集》四卷梁五卷。又有御史中丞《郝默集》五卷,征西諮議《甄述集》十二卷,武昌太守《徐彥則集》十卷,亡。



 晉散騎常侍《王愆期集》七卷梁十卷,錄一卷。又有司徒左長史《王濛集》五卷;丹陽尹《劉惔集》二卷,錄一卷;益州刺史《袁喬集》七卷。亡。



 晉尚書令《顧和集》五卷梁有錄一卷。又有尚書僕射《劉遐集》五卷,徵士《江惇集》三卷,錄一卷;魏興太守《荀述集》一卷;平南將軍《賀翹集》五卷,《李軌集》八卷。亡。



 晉《李充集》二十二卷梁十五卷,錄一卷。



 晉司徒《蔡謨集》十七卷梁四十三卷。



 晉揚州刺史《殷浩集》四卷梁五卷,錄一卷。又有吳興孝廉《鈕滔集》五卷,錄一卷;宣城內史《劉系之集》五卷,錄
 一卷。亡。



 《庾赤玉集》四卷晉尋陽太守《庾統集》八卷梁有驃騎司馬《王修集》二卷,錄一卷;衛將軍《謝尚集》十卷,錄一卷;青州刺史《王浹集》二卷。亡。



 晉西中郎將《王胡之集》十卷梁五卷,錄一卷。



 晉中書令《王洽集》五卷錄一卷。梁有宜春令《範保集》七卷;徵士《範宣集》十卷,錄一卷;建安太守《丁纂集》四卷,錄一卷。亡。



 晉金紫光祿大夫《王羲之集》九卷梁十卷,錄一卷。



 晉散騎常侍《謝萬集》十六卷梁十卷。



 晉司徒長史《張憑集》五卷梁有錄一卷。梁有高涼太守《楊方集》二卷,亡。



 晉徵士《許詢集》三卷梁八卷,錄一卷。



 晉征西將軍《張望集》十卷梁十二卷,錄一卷。



 晉餘姚令《孫統集》二卷梁九卷,錄一卷。又有晉陵令《戴元集》三卷,錄一卷,亡。



 晉衛尉卿《孫綽集》十五卷梁二十五卷。



 晉太常《江逌集》九卷梁有《謝沈集》十卷,亡。



 晉《李顒集》十卷錄一卷。



 晉光祿勛《曹毗集》十
 卷梁十五卷,錄一卷。又有郡主簿《王篾集》五卷,亡。



 晉沙門《支遁集》八卷梁十三卷。又有《劉彧集》十六卷,亡。



 張重華酒泉太守《謝艾集》七卷梁八卷。又有撫軍長史《蔡系集》二卷;護軍將軍《江「〕集》五卷,錄一卷。亡。



 晉《範汪集》一卷梁十卷。



 晉尚書僕射《王述集》八卷梁又有《王度集》五卷,錄一卷;中領軍《庾龢集》二卷,錄一卷;將作大匠《喻希集》一卷;吳興太守《孔嚴集》十一卷,錄一卷。亡。



 晉大司馬《桓溫集》十一卷梁有四十三卷。又有《桓溫要集》二十卷,錄一卷;豫章太守《車灌集》五卷,錄一卷。亡。



 晉尚書僕射《王坦之集》七卷梁五卷,錄一卷,亡。



 晉左光祿《王彪之集》二十卷梁有錄一卷。



 晉中書郎《卻超集》九卷梁十卷。又有南中郎《桓嗣集》五卷;平固令《邵毅集》五卷,錄一卷;太學博士《滕輔集》五卷,錄一卷。亡。



 晉苻堅丞相《王猛集》九卷錄一卷。梁有《顧夷集》五卷,散騎常侍《鄭襲集》四卷,撫軍掾《劉暢集》一卷,亡。



 晉太常卿《韓康伯集》十六卷
 梁有黃門郎《範啟集》四卷;豫章太守《王恪集》十卷;零陵太守《陶混集》七卷,海鹽令《祖撫集》三卷;吳興太守《殷康集》五卷,錄一卷。亡。



 晉太傅《謝安集》十卷梁十卷,錄一卷。又有中軍參軍《孫嗣集》三卷,錄一卷;司徒左長史《劉袞集》三卷。亡。



 晉御史中丞《孔欣時集》八卷梁七卷。



 晉《伏滔集》十一卷並目錄。梁五卷,錄一卷。



 晉滎陽太守《習鑿齒集》五卷晉秘書監《孫盛集》五卷殘缺。梁十卷,錄一卷。



 晉東陽太守《袁宏集》十五卷梁二十卷,錄一卷。又有晉黃門郎《顧淳集》一卷,尋陽太守《熊鳴鵠集》十卷,車騎司馬《謝韶集》三卷,金紫光祿大夫《王獻之集》十卷,錄一卷;瑯邪內史《袁質集》二卷,錄一卷;太宰從事中郎《袁邵集》五卷,錄一卷;車騎長史《謝朗集》六卷,錄一卷;車騎將軍《謝頠集》十卷,錄一卷。亡。



 晉新安太守《卻愔集》四卷殘缺。梁五卷。又有吳郡功曹《陸法之集》十九卷,亡。



 晉太常卿《王氏集》十卷梁錄一卷。



 晉中散大夫《羅含集》三卷梁有太宰長史《庾蒨集》二卷,大司馬
 參軍《庾悠之集》三卷,司徒右長史《庾凱集》二卷,亡。



 晉國子博士《孫放集》一卷殘缺。梁十卷。



 晉聘士《殷叔獻集》四卷並目錄。梁三卷,錄一卷。



 晉湘東太守《庾肅之集》十卷錄一卷。梁有晉北中郎參軍《蘇彥集》十卷;太子左率《王肅之集》三卷,錄一卷;黃門郎《王徽之集》八卷;徵士《謝敷集》五卷,錄一卷,太常卿《孔汪集》十卷,《陳統集》七卷,太常《王愷集》十五卷;右將軍《王忱集》五卷,錄一卷;太常《殷允集》十卷。亡。晉徵士《戴逵集》九卷殘缺。梁十卷,錄一卷。又有晉光祿大夫《孫褵集》十卷,尚書左丞《徐禪集》六卷,亡。



 晉太子前率《徐邈集》九卷並目錄。梁二十卷,錄一卷。



 晉給事中《徐乾集》二十一卷並目錄。梁二十卷,錄一卷。又有晉冠軍將軍《張玄之集》五卷,錄一卷;員外常侍《荀世之集》八卷,《袁山松集》十卷,黃門郎《魏逖之集》五卷,驃騎參軍《卞湛集》五卷,金紫光祿大夫《褚爽集》十六卷,錄一卷。亡。



 晉豫章太守《範寧集》十六卷梁有晉餘杭令《範弘之集》六卷,亡。



 晉司徒《王絢集》十一卷並目錄。梁十卷,錄一
 卷,亡。



 晉處士《薄蕭之集》九卷梁十卷。又有晉安北參軍《薄要集》九卷,《薄邕集》七卷;延陵令《唐邁之集》十一卷,錄一卷。亡。



 晉《孫恩集》五卷梁有晉殿中將軍《傅綽集》十五卷,驍騎將軍《弘戎集》十六卷,御史中丞《魏叔齊集》十五卷,司徒右長史《劉寧之集》五卷,亡。



 晉臨海太守《辛德遠集》五卷梁四卷。又有晉車騎參軍《何瑾之集》十一卷,太保《王恭集》五卷,錄一卷;《殷覬集》十卷,錄一卷。亡。



 晉荊州刺史《殷仲堪集》十二卷並目錄。梁十卷,錄一卷,亡。



 晉驃騎長史《謝景重集》一卷晉《桓玄集》二十卷梁有晉丹陽尹《卞範之集》五卷,錄一卷;光祿勛《卞承之集》十卷,錄一卷。亡。



 晉東陽太守《殷仲文集》七卷梁五卷。



 晉司徒《王謐集》十卷錄一卷。梁有晉光祿大夫《伏系之集》十卷,錄一卷,亡。



 晉右軍參軍《孔璠集》二卷晉衛軍諮議《湛方生集》十卷錄一卷。



 晉光祿大夫《祖臺之集》十六卷梁二十卷。



 晉通直常侍《顧
 愷之集》七卷梁二十卷。



 晉太常卿《劉瑾集》九卷梁五卷。



 晉左僕射《謝混集》三卷梁五卷。



 晉秘書監《滕演集》十卷錄一卷。



 晉司徒長史《王誕集》二卷梁有晉太尉咨議《劉簡之集》十卷,亡。



 晉丹陽太守《袁豹集》八卷梁十卷,錄一卷。又有晉廬江太守《殷遵集》五卷,錄一卷;興平令《荀軌集》五卷。亡。



 晉西中郎長史《羊徽集》九卷梁十卷,錄一卷。



 晉國子博士《周祗集》十一卷梁二十卷,錄一卷。又有晉相國主簿《殷闡集》十卷,錄一卷;常《傅迪集》十卷。亡。



 晉始安太守《卞裕集》十三卷梁十五卷。又有晉《韋公藝集》六卷,亡。



 晉《毛伯成集》一卷晉沙門《支曇諦集》六卷晉沙門《釋惠遠集》十二卷晉姚萇沙門《釋僧肇集》一卷晉《王茂略集》四卷晉《曹毗集》四卷晉《宗欽集》二卷梁有晉中軍功曹《殷曠之集》五卷,太學博士《魏
 說集》十三卷;征西主簿《丘道護集》五卷,錄一卷;柴桑令《劉遺民集》五卷,錄一卷;《郭澄之集》十卷,徵士《周續之集》一卷,《孔瞻集》九卷。亡。



 晉江州刺史王凝之妻《謝道韞集》二卷梁有婦人晉司徒王渾妻《鐘夫人集》五卷,《晉武帝左九嬪集》四卷,晉太宰賈充妻《李扶集》一卷,晉武平都尉陶融妻《陳窈集》一卷,晉都水使者妻《陳玢集》五卷,晉海西令劉臻妻《陳參集》七卷,晉劉柔妻《王邵之集》十卷,晉散騎常侍傅伉妻《辛蕭集》一卷,晉松陽令鈕滔母《孫瓊集》二卷,晉成公道賢妻《龐馥集》一卷,晉宣城太守何殷妻《徐氏集》一卷,亡。



 《宋武帝集》十二卷梁二十卷,錄一卷。



 《宋文帝集》七卷梁十卷,亡。



 《宋孝武帝集》二十五卷梁三十一卷,錄一卷。又有《宋廢帝景和集》十卷,錄一卷;《明帝集》三十三卷。亡。



 宋《長沙王道憐集》十卷錄一卷。梁有《宋臨川王道規集》四卷,錄一卷,亡。



 《宋臨川王義慶集》八卷《宋江夏王義恭集》十一卷梁十五卷,錄一卷。又有《江夏王集別本》十五卷;宋《衡陽王義季集》十卷,錄一卷。亡。



 宋《南平王鑠集》五
 卷梁有宋《竟陵王誕集》二十卷,《建平王休度集》十卷,《新渝惠侯義宗集》十二卷,散騎常侍祖柔之集》二十卷,亡。



 宋豫章太守《謝瞻集》三卷梁有宋征虜將軍《沈林子集》七卷,亡。



 宋太常卿《孔琳之集》九卷並目錄,梁十卷,錄一卷。



 宋《王叔之集》七卷梁十卷,錄一卷。



 宋太中大夫《徐廣集》十五卷錄一卷。



 宋秘書監《盧繁集》一卷殘缺。梁十卷,錄一卷。



 宋侍中《孔寧子集》十一卷並目錄。梁十五卷,錄一卷。



 宋建安太守《卞瑾集》十卷梁十卷。



 宋太常卿《蔡廓集》九卷並目錄。梁十卷,錄一卷。又有宋《王韶之集》二十四卷,亡。



 宋尚書令《傅亮集》三十一卷梁二十卷,錄一卷。又有宋征南長史《孫康集十卷,左軍長史《範述集》三卷,亡。



 宋太常卿《鄭鮮之集》十三卷梁二十卷,錄一卷。



 宋徵士《陶潛集》九卷梁五卷,錄一卷。又有《張野集》十卷,宋零陵令《陶階集》八卷,東莞太守《張元瑾集》八卷;光祿大夫《王曇首集》二卷,錄一卷,亡。



 宋
 太常卿《範泰集》十九卷梁二十卷,錄一卷。



 宋中書郎《荀昶集》十四卷梁十五卷,錄一卷。又有《卞伯玉集》五卷,錄一卷;中散大夫《羊欣集》七卷。亡。



 宋司徒《王弘集》一卷梁二十卷,錄一卷。又有宋金紫光祿大夫《沈演集》十卷,廣平太守《範凱集》八卷,亡。



 宋沙門《釋惠琳集》五卷梁九卷,錄一卷。又有宋《範晏集》十四卷,亡。



 宋司徒府參軍《謝惠連集》六卷梁五卷,錄一卷。又有宋太常《謝弘微集》二卷,亡。



 宋臨川內史《謝靈運集》十九卷梁二十卷,錄一卷。



 宋給事中《丘深之集》七卷梁十五卷。又有義成太守《祖屳之集》五卷,荊州西曹《孫韶集》十卷,《殷淳集》二卷,揚州刺史《殷景仁集》九卷,國子博士《姚濤之集》二十卷,錄一卷,《周示殳集》十一卷。亡。



 《殷闡之集》一卷宋徵士《宗景集》十六卷梁十五卷。



 宋徵士《雷次宗集》十六卷梁二十九卷,錄一卷。



 宋奉朝請《伍緝之集》十二卷梁有宋南蠻主簿《衛令元集》八卷;《範曄集》十五卷,錄一卷;
 撫軍諮議《範廣集》一卷;右光祿大夫《王敬弘集》五卷,錄一卷;《任豫集》六卷。



 宋御史中丞《何承天集》二十卷梁三十二卷,亡。



 宋太中大夫《裴松之集》十三卷梁二十一卷。又有《王韶之集》十九卷,宋光祿大夫《江湛集》四卷,錄一卷。亡。



 宋太尉《袁淑集》十一卷並目錄。梁十卷,錄一卷。



 宋秘書監《王微集》十卷梁有錄一卷。又有宋太子舍人《王僧謙集》二卷,金紫光祿大夫《王僧綽集》一卷,征北行參軍《顧邁集》二十卷,魚復令《陳超之集》十卷,平南將軍《何長瑜集》八卷,亡。



 宋員外郎《荀雍集》二卷梁四卷。又有宋國子博士《範演集》八卷,錢唐令《顧昱集》六卷,臨成令《韓浚之集》八卷,南陽太守《沈亮之集》七卷,國子博士《孔欣集》九卷,臨海太守《江玄叔集》四卷,尚書郎《劉馥集》十一卷,太子中舍人《張演集》八卷,南昌令《蔡眇之集》三卷,太學博士《顧雅集》十三卷,巴東太守《孫仲之集》十一卷,太尉諮議參軍《謝元集》一卷,南海太守《陸展集》九卷,棘陽令《山謙之集》十二卷,廣州刺史《羊希集》九卷,員外常侍《周始之集》十一卷,主客郎《羊崇集》六卷,太子舍人《孔景亮集》三卷,亡。



 宋中
 書郎《袁伯文集》十一卷並目錄。梁有宋丞相諮議《蔡超集》七卷,亡。



 宋東中郎長史《孫緬集》八卷並目錄。梁十一卷。又有宋《賀道養集》十卷,太子洗馬《謝登集》六卷,新安太守《張鏡集》十卷;兼中書舍人《褚詮之集》八卷,錄一卷。亡。



 宋特進《顏延之集》二十五卷梁三十卷。又有《顏延之逸集》一卷,亡。



 宋東揚州刺史《顏竣集》十四卷並目錄。



 宋大司馬錄事《顏測集》十一卷並目錄。



 宋護軍將軍《王僧達集》十卷梁有錄一卷。又有國子博士《羊戎集》十卷,江寧令《蘇寶生集》四卷,兗州別駕《範義集》十二卷,吳興太守《劉瑀集》七卷,本郡孝廉《劉氏集》九卷,亡。



 宋會稽太守《張暢集》十二卷殘缺。梁十四卷,錄一卷。又有宋司空《何尚之集》十卷,亡。



 宋吏部尚書《何偃集》十九卷梁十六卷。又有廬江太守《周朗集》八卷,亡。



 宋侍中《沈懷文集》十二卷殘缺。梁十六卷。



 宋北中郎長史《江智深集》九卷並目一卷。



 宋太子中庶
 子《殷琰集》七卷梁又有宋武陵太守《袁凱集》八卷,《荀欽明集》六卷,安北參軍《王詢之集》五卷,越騎校尉《戴法興集》四卷,亡。



 宋黃門郎《虞通之集》十五卷梁二十卷。



 宋司徒左長史《沈勃集》十五卷梁二十卷。



 宋金紫光祿大夫《謝莊集》十九卷梁十五卷。又有宋金紫光祿大夫《謝協集》三卷,三巴校尉《張悅集》十一卷,揚州從事《賀頠集》十一卷,領軍長史《孔邁之集》八卷,撫軍參軍《賀弼集》十六卷,本州秀才《劉遂集》二卷,亡。



 宋《建平王景素集》十卷宋征虜記室參軍《鮑照集》十卷梁六卷。又有宋武康令《沈懷遠集》十九卷,《裴駰集》六卷,刪定郎《劉鯤集》五卷,宜都太守《費修集》十卷,亡。



 宋太中大夫《徐爰集》六卷梁十卷。又有宋護軍司馬《孫勃集》六卷,右光祿大夫《張永集》十卷,陽羨令《趙繹集》十六卷,亡。



 宋《庾蔚之集》十六卷梁二十卷。又有太子中舍人徵不就《王素集》十六卷,亡。



 宋豫章太守《劉愔集》八卷梁十卷。又有宋起部《費鏡運集》二十卷,光祿大夫《孫夐集》
 十一卷,太尉從事中郎《蔡頤集》三卷;司空《劉勔集》二十卷,錄一卷;青州刺史《明僧暠集》十卷,吳興太守《蕭惠開集》七卷,《沈宗之集》十卷,大司農《張辯集》十六卷,金紫光祿大夫《王瓚集》十五卷,錄一卷,《郭坦之集》五卷,會稽主簿《辛湛之集》八卷,太子舍人《硃百年集》二卷,東海王常侍《鮑德遠集》六卷,會稽郡丞《張緩集》六卷。亡。



 宋寧國令《劉薈集》七卷宋江州從事《吳邁遠集》一卷殘缺。梁八卷,亡。



 宋宛朐令《湯惠休集》三卷梁四卷。又有南海太守《孫奉伯集》十卷,右將軍《成元範集》十卷,奉朝請《虞喜集》十一卷,延陵令《唐思賢集》十五卷,《戴凱之集》六卷,亡。



 宋司徒《袁粲集》十一卷並目錄。梁九卷。又有婦人《牽氏集》一卷,宋後宮司儀《韓蘭英集》四卷,亡。



 《齊文帝集》一卷殘缺。梁十一卷。又有齊《晉安王子懋集》四卷,錄一卷;《隨王子隆集》七卷,亡。



 齊《竟陵王子良集》四十卷梁又有齊聞喜公《蕭遙欣集》十一卷,領軍諮議《劉祥集》十卷,亡。



 齊太宰《褚彥回集》十五卷梁又有齊黃門侍郎《崔祖思集》二十卷,中軍佐《鐘蹈集》十二卷;餘杭令《丘巨源
 集》十卷,錄一卷。亡。



 齊太尉《王儉集》五十一卷梁六十卷。又有齊東海太守《謝顥集》十六卷,《謝瀹集》十卷,豫州刺史《劉善明集》十卷,侍中《褚賁集》十二卷,徵士《劉虯集》二十四卷,司徒主簿徵不就《庾易集》十卷,《顧歡集》三十卷,《劉瓛集》三十卷,射聲校尉《劉璡集》三卷,亡。



 齊中書郎《周顒集》八卷梁十六卷。又有齊左侍郎《鮑鴻集》二十卷,錄一卷;雍州秀才《韋瞻集》十卷;正員郎《劉懷慰集》十卷,錄一卷;永嘉太守《江山圖集》十卷,驃騎記室參軍《荀憲集》十一卷。亡。



 齊前軍參軍《虞羲集》九卷殘缺。梁十一卷。又有平陽令《韋沈集》十卷,車騎參軍《任文集》十一卷,《卞鑠集》十六卷,《婁幼瑜集》六十六卷,長水校尉《祖沖之集》五十一卷,亡。



 齊中書郎《王融集》十卷齊吏部郎《謝朓集》十二卷《謝朓逸集》一卷梁又有《王巾集》十一卷,亡。



 齊司徒左長史《張融集》二十七卷梁十卷。又有張融《玉海集》十卷、《大澤集》十卷、《金波集》六十卷,又有齊羽林監《庾韶集》十卷,黃門郎《王僧佑集》十卷;太常卿《劉悛集》二十卷,錄一卷;秘書《王寂集》五卷。
 亡。



 齊金紫光祿大夫《孔稚珪集》十卷齊後軍法曹參軍《陸厥集》八卷梁十卷。



 齊太尉《徐孝嗣集》十卷梁七卷。又有侍中《劉暄集》一十一卷,通直常侍《裴昭明集》九卷,《虞炎集》七卷,吏部郎《劉瑱集》十卷,梁國從事中郎《劉繪集》十卷,亡。



 齊侍中《袁彖集》五卷並錄。



 齊中書郎《江奐集》九卷並錄。



 齊平西諮議《宗躬集》十三卷齊太子舍人《沈驎士集》六卷《梁武帝集》二十六卷梁三十二卷。



 《梁武帝詩賦集》二十卷《梁武帝雜文集》九卷《梁武帝別集目錄》二卷《梁武帝凈業賦》三卷《梁簡文帝集》八十五卷陸罩撰,並錄。



 《梁元帝集》五十二卷《梁元帝小集》十卷梁《昭明太子集》二十卷梁有《梁安成王集》三十卷,亡。



 梁《岳陽王詧集》十卷《梁王
 蕭巋集》十卷梁《邵陵王綸集》六卷梁《武陵王紀集》八卷梁《蕭琮集》七卷梁又有《安成煬王集》五卷,亡。



 梁司徒諮議《宗夬集》九卷並錄。



 梁國子博士《丘遲集》十卷並錄。梁十五卷,又有《謝朏集》十五卷,亡。



 梁金紫光祿大夫《江淹集》九卷梁二十卷。



 《江淹後集》十卷梁尚書僕射《範雲集》十一卷並錄。



 梁太常卿《任昉集》三十四卷梁有晉安太守《謝纂集》十卷,撫軍將軍《柳惔集》二十卷,中護軍《柳惲集》十二卷,豫州刺史《柳憕集》六卷,尚書令《柳忱集》十三卷,義興郡丞《何僴集》三卷,撫軍中兵參軍《韋溫集》十卷,鎮西錄事參軍《到洽集》十一卷,太子洗馬《劉苞集》十卷,南徐州秀才《諸葛璩集》十卷,亡。



 梁特進《沈約集》一百一卷並錄。梁又有《謝綽集》十一卷,亡。



 梁中軍府諮議《王僧孺集》三十卷梁尚書左丞《範縝集》十一卷梁護軍將軍《周舍集》
 二十卷梁有秘書張熾《金河集》六十卷,《劉敲集》八卷,玄貞處士《劉訏集》一卷,亡。



 《梁蕭洽集》二卷梁隱居先生《陶弘景集》三十卷《陶弘景內集》十五卷梁徵士《魏道微集》三卷梁黃門郎《張率集》三十八卷梁南徐州治中《王冏集》三卷梁都官尚書《江革集》六卷梁奉朝請《吳均集》二十卷梁光祿大夫《庾曇隆集》十卷並錄。



 梁儀同三司《徐勉前集》三十五卷《徐勉後集》十六卷並序錄。



 梁吏部郎《王錫集》七卷並錄。



 梁尚書左僕射《王暕集》二十一卷梁平西刑獄參軍《劉孝標集》六卷梁鴻臚卿《裴子野集》十四卷梁仁威府長史《司馬褧集》九卷梁《蕭子暉集》九卷梁始
 興內史《蕭子範集》十三卷梁建陽令《江洪集》二卷梁鎮西府記室《鮑畿集》八卷梁尚書祠部郎《虞爵集》十卷梁新田令《費昶集》三卷梁《蕭幾集》二卷梁東陽郡丞《謝瑱集》八卷梁通直郎《謝琛集》五卷梁仁威記室《何遜集》七卷梁有安西記室《劉緩集》四卷,沙門《釋智藏集》五卷,亡。



 梁太常卿《陸倕集》十四卷梁廷尉卿《劉孝綽集》十四卷梁都官尚書《劉孝儀集》二十卷梁太子庶子《劉孝威集》十卷梁東陽太守《王揖集》五卷梁黃門郎《陸雲公集》十卷梁國子祭酒《蕭子雲集》十九卷梁征西府長史《楊眺集》十一卷並錄。



 梁太子洗馬《王筠集》十一卷
 並錄。



 王筠《中書集》十一卷並錄。



 王筠《臨海集》十一卷並錄。



 王筠《左佐集》十一卷並錄。



 王筠《尚書集》九卷並錄。



 梁西昌侯《蕭深藻集》四卷並錄。



 梁中書郎《任孝恭集》十卷梁平北府長史《鮑泉集》一卷梁雍州刺史《張纘集》十一卷並錄。



 梁尚書僕射《張綰集》十一卷並錄。



 梁度支尚書《庚肩吾集》十卷梁太常卿《劉之遴前集》十一卷《劉之遴後集》二十一卷梁豫章世子侍讀《謝鬱集》五卷梁安成蕃王《蕭欣集》十卷梁中書舍人《硃超集》一卷梁護軍將軍《甄玄成集》十卷並錄。



 梁散騎常侍《沈君游集》十三卷。



 梁《臨安恭公主集》三卷武帝女。



 梁征西記室範靖妻《沈滿
 願集》三卷梁太子洗馬徐悱妻《劉令嫻集》三卷《後魏孝文帝集》三十九卷後魏司空《高允集》二十一卷後魏司農卿《李諧集》十卷後魏太常卿《盧元明集》十七卷後魏司空祭酒《袁躍集》十三卷後魏著作佐郎《韓顯宗集》十卷後魏散騎常侍《溫子升集》三十九卷後魏太常卿《陽固集》三卷北齊特進《邢子才集》三十一卷北齊尚書僕射《魏收集》六十八卷北齊儀同《劉逖集》二十六卷後周《明帝集》九卷後周《趙王集》八卷後周《滕簡王集》八卷後周儀同《宗懍集》十二卷並錄。



 後周沙門《釋亡名集》十卷後周小司空《
 王褒集》二十一卷並錄。



 後周少傅《蕭捴集》十卷後周開府儀同《庾信集》二十一卷並錄。



 《陳後主集》三十九卷《陳後主沈後集》十卷陳大匠卿《杜之偉集》十二卷陳金紫光祿大夫《周弘讓集》九卷陳《周弘讓後集》十二卷陳侍中《沈炯前集》七卷陳《沈炯後集》十三卷陳沙門《釋標集》二卷陳沙門《釋洪偃集》八卷陳沙門《釋瑗集》六卷陳沙門《釋靈裕集》四卷陳尚書僕射《周弘正集》二十卷陳鎮南府司馬《陰鏗集》一卷陳左衛將軍《顧野王集》十九卷陳沙門《策上人集》五卷陳尚書左僕射《徐陵集》三十卷陳右衛將軍《張
 式集》十四卷陳尚書度支郎《張正見集》十四卷陳司農卿《陸琰集》二卷陳少府卿《陸玠集》十卷陳光祿卿《陸瑜集》十一卷並錄。



 陳護軍將軍《蔡景歷集》五卷陳沙門《釋暠集》六卷陳御史中丞《褚玠集》十卷陳安右府諮議《司馬君卿集》二卷陳著作佐郎《張仲簡集》一卷。



 《煬帝集》五十五卷《王祐集》一卷武陽太守《盧思道集》三十卷金州刺史《李元操集》十卷蜀王府記室《辛德源集》三十卷太尉《楊素集》十卷懷州刺史《李德林集》十卷吏部尚書《牛弘集》十二卷司隸大夫《薛道衡集》三卷國子祭酒《何妥集》十卷
 秘書監《柳抃集》五卷開府《江總集》三十卷《江總後集》二卷記室參軍《蕭愨集》九卷著作郎《魏彥深集》三卷著作郎《諸葛穎集》十四卷劉子政母《祖氏集》九卷著作郎《王胄集》十卷右四百三十七部,四千三百八十一卷。通計亡書,合八百八十六部,八千一百二十六卷。



 別集之名,蓋漢東京之所創也。自靈均已降,屬文之士眾矣,然其志尚不同,風流殊別。後之君子,欲觀其體勢,而見其心靈,故別聚焉,名之為集。辭人景慕,並自記載,以成書部。年代遷徙,亦頗遺散。其高唱絕俗者,略皆具
 存,今依其先後,次之於此。



 《文章流別集》四十一卷梁六十卷,志二卷,論二卷,摯虞撰。



 《文章流別志》、《論》二卷摯虞撰。



 《文章流別本》十二卷謝混撰。



 《續文章流別》三卷孔寧撰。



 《集苑》四十五卷梁六十卷。



 《集林》一百八十一卷宋臨川王劉義慶撰。梁二百卷。



 《集林鈔》十一卷《集鈔》十卷沈約撰。梁有《集鈔》四十卷,丘遲撰,亡。



 《集略》二十卷《撰遺》六卷梁又有《零集》三十六卷,亡。



 《翰林論》三卷李充撰。梁五十四卷。



 《文苑》一百卷孔道撰。



 《文苑鈔》三十卷《文選》三十卷梁昭明太子撰。



 《詞林》五十八卷《文海》五十卷《吳朝士文集》十卷梁十三卷。又有《漢書文府》三卷,亡。



 《巾箱集》七卷梁有《文章志錄雜文》八卷,謝沈撰,又《名士雜文》八卷,亡。



 《婦人集》二十卷梁有《婦人集》三十卷,殷淳撰。又有《婦人集》十一
 卷,亡。



 《婦人集鈔》二卷《雜文》十六卷為婦人作。



 《文選音》三卷蕭該撰。



 《文心雕龍》十卷梁兼東宮通事舍人劉勰撰。



 《文章始》一卷姚察撰。梁有《文章始》一卷,任昉撰;《四代文章記》一卷,吳郡功曹張防撰。



 亡。



 《賦集》九十二卷謝靈運撰。梁又有《賦集》五十卷,宋新渝惠侯撰;《賦集》四十卷,宋明帝撰;《樂器賦》十卷;《伎藝賦》六卷。亡。



 《賦集鈔》一卷《賦集》八十六卷後魏秘書丞崔浩撰。



 《續賦集》十九卷殘缺。



 《歷代賦》十卷梁武帝撰。



 《皇德瑞應賦頌》一卷梁十六卷。



 《五都賦》六卷並錄。張衡及左思撰。



 《雜都賦》十一卷梁《雜賦》十六卷。又《東都賦》一卷,孔逭作;《二京賦音》二卷,李軌、綦毋邃撰;《齊都賦》二卷並音,左思撰;《相風賦》七卷,傅玄等撰;《迦維國賦》二卷,晉右軍行參軍虞干紀撰;《遂志賦》十卷,《乘輿赭白馬》二卷。亡。



 《述征賦》一卷《神雀賦》一卷後漢傅毅撰。



 《雜賦注本》三卷梁有郭璞注《子虛上林賦》一卷,薛綜注張衡《二京賦》二卷,晁矯注《二京賦》一卷,傅巽注《二京賦》二卷,張載
 及晉侍中劉逵、晉懷令衛權注左思《三都賦》三卷,綦毋邃注《三都賦》三卷,項氏注《幽通賦》,蕭廣濟注木玄虛《海賦》一卷,徐爰注《射雉賦》一卷,亡。



 《獻賦》十八卷《圍棋賦》一卷梁武帝撰。



 《觀象賦》一卷《洛神賦》一卷孫壑注。



 《枕賦》一卷張君祖撰。



 《二都賦音》一卷李軌撰。



 《百賦音》十卷宋御史褚詮之撰。梁有《賦音》二卷,郭徵之撰;《雜賦圖》十七卷。亡。



 《大隋封禪書》一卷《上封禪書》二卷梁有《雜封禪文》八卷,《秦帝刻石文》一卷,宋會稽太守褚淡撰,亡。



 《集雅篇》五卷《靖恭堂頌》一卷晉涼王李灊撰。梁有《頌集》二十卷,王僧綽撰;《木連理頌》二卷,太元十九年群臣上。亡。



 《詩集》五十卷謝靈運撰。梁五十一卷。又有宋侍中張敷、袁淑補謝靈運《詩集》一百卷;又《詩集》百卷,並例、錄二卷,顏峻撰;《詩集》四十卷,宋明帝撰;《雜詩》七十九卷,江邃撰;《雜詩》二十卷,宋太子洗馬劉和注;《二晉雜詩》二十卷;《古今五言詩美文》五卷,荀綽撰;《詩鈔》十卷。亡。



 《詩集鈔》十卷謝靈運撰。梁有《雜詩鈔》十卷,錄一卷,謝靈運撰,亡。



 《古詩集》九
 卷《六代詩集鈔》四卷梁有《雜言詩鈔》五卷,謝朏撰,亡。



 《詩英》九卷謝靈運集。梁十卷。又有《文章英華》三十卷,梁昭明太子撰,亡。



 《今詩英》八卷《古今詩苑英華》十九卷梁昭明太子撰。



 《詩纘》十三卷《眾詩英華》一卷《詩類》六卷《玉臺新詠》十卷徐陵撰。



 《百志詩》九卷干寶撰。梁五卷。又有《古游仙詩》一卷;應貞注應璩《百一詩》八卷;《百一詩》二卷,晉蜀郡太守李彪撰。亡。



 齊《釋奠會詩》一十卷《齊宴會詩》十七卷《青溪詩》三十卷齊宴會作。梁有魏、晉、宋《雜祖餞宴會詩集》二十一部,一百四十三卷,亡,今略其數。



 《西府新文》十一卷並錄。梁蕭淑撰。



 《百國詩》四十三卷《文林館詩府》八卷後齊文林館作。



 《詩評》三卷鐘嶸撰,或曰《詩品》。



 《古樂府》八卷《文會詩》三卷陳仁威記室徐伯陽撰。



 《五岳七星回文詩》一卷梁有《雜詩圖》一卷,亡。



 《毛伯成詩》一卷伯
 成,東晉征西參軍。



 《春秋寶藏詩》四卷張朏撰。



 《江淹擬古》一卷羅潛注。



 《樂府歌辭鈔》一卷《歌錄》十卷《古歌錄鈔》二卷《晉歌章》八卷梁十卷。



 《吳聲歌辭曲》一卷梁二卷。又有《樂府歌詩》二十卷,秦伯文撰;《樂府歌詩》十二卷,《樂府三校歌詩》十卷,《樂府歌辭》九卷;《太樂歌詩》八卷,《歌辭》四卷,張永記;《魏宴樂歌辭》七卷,《晉歌章》十卷;又《晉歌詩》十八卷,《晉宴樂歌辭》十卷,荀勖撰;《宋太始祭高禖歌辭》十一卷,《齊三調雅辭》五卷;《古今九代歌詩》七卷,張湛撰;《三調相和歌辭》五卷,《三調詩吟錄》六卷,《奏鞞鐸舞曲》二卷,《管弦錄》一卷,《伎錄》一卷;《太樂備問鍾鐸律奏舞歌》四卷,郝生撰;《回文集》十卷,謝靈運撰;又《回文詩》八卷,《織錦回文詩》一卷,苻堅秦州刺史竇氏妻蘇氏作;《頌集》二十卷,王僧綽撰;《木連理頌》二卷,晉太元十九年群臣上;又有鼓吹、清商、樂府、宴樂、高禖、鞞、鐸等《歌辭舞錄》,凡十部。



 《陳郊廟歌辭》三卷並錄。徐陵撰。



 《樂府新歌》十卷秦王記室崔子發撰。



 《樂府新歌》二卷秦王司馬殷僧首撰《古今箴銘集》十四卷張湛
 撰。錄一卷。梁有《箴集》十六卷,《雜誡箴》二十四卷,《女箴》一卷,《女史箴圖》一卷,又有《銘集》十一卷,又陸少玄撰《佛像雜銘》十三卷,釋僧祐撰《箴器雜銘》五卷,亡。



 《眾賢誡集》十卷殘缺。梁有《誡林》三卷,綦毋邃撰;《四帝誡》三卷,王誕撰;《雜家誡》七卷,《諸家雜誡》九卷,《集誡》二十二卷。亡。



 《諸葛武侯誡》一卷、《女誡》一卷《女誡》一卷曹大家撰。



 《女鑒》一卷梁有《女訓》十六卷。



 《婦人訓誡集》十一卷並錄。梁十卷。宋司空徐湛之撰。



 《娣姒訓》一卷馮少胄撰。



 《貞順志》一卷《贊集》五卷謝莊撰。



 《畫贊》五卷漢明帝殿閣畫,魏陳思王贊。梁五十卷。又有《誄集》十五卷,謝莊撰,亡。



 《七集》十卷謝靈運集。



 《七林》十卷梁十二卷,錄二卷。卞景撰。梁有又有《七林》三十卷,音一卷,亡。



 《七悟》一卷顏之推撰。梁有《吊文集》六卷,錄一卷;《吊文》二卷,亡。



 《碑集》二十九卷《雜碑集》二十九卷《雜碑集》二十二卷梁有《碑集》十卷,謝莊撰;《釋氏碑文》三十卷,梁元帝撰;《雜碑》二
 十二卷,《碑文》十五卷,晉將作大匠陳勰撰;《碑文》十卷,車灌撰;又有《羊祜墮淚碑》一卷,《桓宣武碑》十卷,《長沙景王碑文》三卷,《荊州雜碑》三卷,《雍州雜碑》四卷,《廣州刺史碑》十二卷,《義興周處碑》一卷,《太原王氏家碑誄頌贊銘集》二十六卷;《諸寺碑文》四十六卷,釋僧祐撰;《雜祭文》六卷,《眾僧行狀》四十卷,釋僧祐撰。亡。



 《設論集》二卷劉楷撰。梁有《設論集》三卷,東晉人撰;《客難集》二十卷,亡。



 《論集》七十三卷《雜論》十卷《明真論》一卷晉兗州刺史宗岱撰。



 《東西晉興亡論》一卷《陶神論》五卷《正流論》一卷《黃芳引連珠》一卷《梁武連珠》一卷沈約注。



 《梁武帝制旨連珠》十卷梁邵陵王綸注。



 《梁武帝制旨連珠》十卷陸緬注。梁有《設論連珠》十卷,謝靈運撰《連珠集》五卷,陳證撰《連珠》十五卷;又《連珠》一卷,陸機撰,何承天注;又班固《典引》一卷,蔡邕注。亡。



 《梁代雜文》三卷《詔集區分》四十一卷後周獸門學士宗乾撰。



 《魏朝雜詔》二卷梁有《漢高祖手詔》一卷,亡。



 《錄魏吳二志詔》二卷梁有《三國
 詔誥》十卷,亡。



 《晉咸康詔》四卷《晉朝雜詔》九卷梁有《晉雜詔》百卷,錄一卷。又有《晉雜詔》二十八卷,錄一卷;又《晉詔》六十卷,《晉文王》、《武帝雜詔》十二卷。亡。



 《錄晉詔》十四卷梁有《晉武帝詔》十二卷,《成帝詔草》十七卷,《康帝詔草》十卷,《建元直詔》三卷,《永和副詔》九卷,《升平、隆和、興寧副詔》十卷,《泰元、咸寧、寧康副詔》二十二卷,《隆安直詔》五卷,《元興大亨副詔》三卷,亡。



 《晉義熙詔》十卷梁有《義熙副詔》十卷,《義熙以來至於大明詔》三十卷,《晉宋雜詔》四卷;又《晉宋雜詔》八卷,王韶之撰;又《雜詔》十四卷,《班五條詔》十卷。亡。



 《宋永初雜詔》十三卷梁有《詔集》百卷,起漢訖宋;《武帝詔》四卷,宋《元熙詔令》五卷,《永初二年五年詔》三卷,《永初已來中書雜詔》二十卷。亡。



 《宋孝建詔》一卷梁有《宋景平詔》三卷,亡。



 《宋元嘉副詔》十五卷梁有《宋元嘉詔》六十二卷,又《宋孝武詔》五卷,《宋大明詔》七十卷,《宋永光、景和詔》五卷,《宋泰始、泰豫詔》二十二卷,《宋義嘉偽詔》一卷,《宋元徽詔》十三卷,《宋升明詔》四卷,亡。



 《齊雜詔》十卷《齊中興二年詔》三卷梁有《齊建元詔》五卷,《永明詔》三
 卷,《武帝中詔》十卷,《齊隆昌、延興、建武詔》九卷,《齊建武二年副詔》九卷,《梁天監元年至七年詔》十二卷,《天監九年、十年詔》二卷,亡。



 《後魏詔集》十六卷《後周雜詔》八卷《雜詔》八卷《雜赦書》六卷《陳天嘉詔草》三卷《霸朝集》三卷李德林撰。



 《皇朝詔集》九卷《皇朝陳事詔》十三卷梁有《雜九錫文》四卷,亡。



 《上法書表》一卷虞和撰。



 《梁中表》十一卷梁邵陵王撰。梁有《漢名臣奏》三十卷;《魏名臣奏》三十卷,陳長壽撰;《魏雜事》七卷,《晉諸公奏》十一卷,《雜表奏駁》三十五卷,《漢丞相匡衡、大司馬王鳳奏》五卷,《劉隗奏》五卷,《孔群奏》二十二卷,《晉金紫光祿大夫周閔奏事》四卷,《晉中丞劉邵奏事》六卷,《中丞司馬無忌奏事》十三卷,《中丞虞穀奏事》六卷,《中丞高崧奏事》五卷,又《諸彈事》等十四部。亡。



 《雜露布》十二卷梁有《雜檄文》十七卷,《魏武帝露布文》九卷,亡。



 《山公啟事》三卷《範寧啟事》三卷梁十卷。梁有《雜薦文》十二卷,《薦文集》七卷,亡。



 《善文》五十卷杜預撰。



 《雜集》一
 卷殷仲堪撰。



 《梁、魏、周、齊、陳皇朝聘使雜啟》九卷《政道集》十卷《書集》八十八卷晉散騎常侍王履撰。梁八十卷,亡。



 《書林》十卷《雜逸書》六卷梁二十二卷。徐爰撰。《應璩書林》八卷,夏赤松撰;《抱樸君書》一卷,葛洪撰;《蔡司徒書》三卷,蔡謨撰;《前漢雜筆》十卷,《吳晉雜筆》九卷,《吳朝文》二十四卷,《李氏家書》八卷,晉左將軍《王鎮惡與劉丹陽書》一卷,亡。



 《後周與齊軍國書》二卷《高澄與侯景書》一卷《策集》一卷殷仲堪撰。



 《策集》六卷梁有《孝秀對策》十二卷,亡。



 《宋元嘉策孝秀文》十卷《誹諧文》三卷《誹諧文》十卷袁淑撰。梁有《續誹諧文集》十卷;又有《誹諧文》一卷,沈宗之撰;《任子春秋》一卷,杜嵩撰;《博陽秋》一卷,宋零陵令辛邕之撰。亡。



 《法集》百七卷梁沙門釋寶唱撰。



 右一百七部,二千二百一十三卷。通計亡書,合二百四十九部,五
 千二百二十四卷。



 總集者,以建安之後,辭賦轉繁,眾家之集,日以滋廣,晉代摯虞苦覽者之勞倦,於是採摘孔翠,芟剪繁蕪,自詩賦下,各為條貫,合而編之,謂為《流別》。是後文集總鈔,作者繼軌,屬辭之士,以為覃奧,而取則焉。今次其前後,並解釋評論,總於此篇。



 凡集五百五十四部,六千六百二十二卷。通計亡書,合一千一百四十六部,一萬三千三百九十卷。



 文者,所以明言也。古者登高能賦,山川能祭,師旅能誓,喪紀能誄,作器能銘,則可以為大夫。



 言其因物騁辭,情
 靈無擁者也。唐歌虞詠,商頌周雅,敘事緣情,紛綸相襲,自斯已降,其道彌繁。世有澆淳,時移治亂,文體遷變,邪正或殊。宋玉、屈原,激清風於南楚,嚴、鄒、枚、馬,陳盛藻於西京,平子艷發於東都,王粲獨步於漳滏。愛逮晉氏,見稱潘、陸,並黼藻相輝,宮商間起,清辭潤乎金石,精義薄乎雲天。永嘉已後,玄風既扇,辭多平淡,文寡風力。降及江東,不勝其弊。宋、齊之世,下逮梁初,靈運高致之奇,延年錯綜之美,謝玄暉之藻麗,沈休文之富溢,輝煥斌蔚,辭義可觀。梁簡文之在東宮,亦好篇什,清辭巧制,止乎衽席之間,雕琢蔓藻,思極閨闈之內。後生好事,遞相放
 習,朝野紛紛,號為宮體。流宕不已,訖於喪亡。陳氏因之,未能全變。其中原則兵亂積年,文章道盡。後魏文帝,頗效屬辭,未能變俗,例皆淳古。齊宅漳濱,辭人間起,高言累句,紛紜絡繹,清辭雅致,是所未聞。後周草創,干戈不戢,君臣戮力,專事經營,風流文雅,我則未暇。其後南平漢沔,東定河朔,訖於有隋,四海一統,採荊南之巳梓,收會稽之箭竹,辭人才士,總萃京師。屬以高祖少文,煬帝多忌,當路執權,逮相擯壓。於是握靈蛇之珠,韞荊山之玉,轉死溝壑之內者,不可勝數,草澤怨刺,於是興焉。古者陳詩觀風,斯亦所以關乎盛衰者也。班固有《詩賦略》,
 凡五種,今引而伸之,合為三種,謂之集部。



 凡四部經傳三千一百二十七部,三萬六千七百八卷。通計亡書,合四千一百九十一部,四萬九千四百六十七卷。



 經戒三百一部,九百八卷。餌服四十六部,一百六十七卷。房中十三部,三十八卷。符錄十七部,一百三卷。



 右三百七十七部,一千二百一十六卷。



 道經者,云有元始天尊,生於太元之先,稟自然之氣,沖虛凝遠,莫知其極。所以說天地淪壞,劫數終盡,略與佛經同。以為天尊之體,常存不滅。每至天地初開,或在玉
 京之上,或在窮桑之野,授以秘道,謂之開劫度人。然其開劫非一度矣,故有延康、赤明、龍漢、開皇,是其年號。其間相去經四十一億萬載。所度皆諸天仙上品,有太上老君、太上丈人、天真皇人五方天帝及諸仙官,轉共承受,世人莫之豫也。所說之經,亦稟元一之氣,自然而有,非所造為,亦與天尊常在不滅。天地不壞,則蘊而莫傳,劫運若開,其文自見。凡八字,盡道體之奧,謂之天書。字方一丈,八角垂芒,光輝照耀,驚心眩目,雖諸天仙,不能省視。天尊之開劫也,乃命天真皇人,改囀天音而辯析之。自天真以下,至於諸仙,展轉節級,以次相授。諸仙得
 之,始授世人。然以天尊經歷年載,始一開劫,受法之人,得而寶秘,亦有年限,方始傳授。上品則年久,下品則年近。故今授道者,經四十九年,始得授人。推其大旨,蓋亦歸於仁愛清靜,積而修習,漸致長生,自然神化,或白日登仙,與道合體。其受道之法,初受《五千文籙》,次受《三洞籙》,次受《洞玄籙》,次受《上清籙》。籙皆素書,紀諸天曹官屬佐吏之名有多少,又有諸符,錯在其間,文章詭怪,世所不識。受者必先潔齋,然後齎金環一,並諸贄幣,以見於師。師受其贄,以籙授之,仍剖金環,各持其半,云以為約。弟子得籙,緘而佩之。



 其潔齋之法,有黃籙、玉籙、金籙、塗
 炭等齋。為壇三成,每成皆置綿蕝嶠,以為限域。傍各開門,皆有法象。齋者亦有人數之限,以次入於綿蕝之中,魚貫面縛,陳說愆咎,告白神祇,晝夜不息,或一二七日而止。其齋數之外有人者,並在綿蕝之外,謂之齋客,但拜謝而已,不面縛焉。而又有諸消災度厄之法,依陰陽五行數術,推人年命書之,如章表之儀,並具贄幣,燒香陳讀。雲奏上天曹,請為除厄,謂之上章。夜中於星辰之下,陳設酒脯餅餌幣物,歷祀天皇太一,祀五星列宿,為書如上章之儀以奏之,名之為醮。又以木為印,刻星辰日月於其上,吸氣執之,以印疾病,多有愈者。又能登刀入
 火而焚敕之,使刃不能割,火不能熱。而又有諸服餌、闢谷、金丹、玉漿、雲英,蠲除滓穢之法,不可殫記。雲自上古黃帝、帝嚳、夏禹之儔,並遇神人,咸受道籙,年代既遠,經史無聞焉。



 推尋事跡,漢時諸子,道書之流有三十七家,大旨皆去健羨,處沖虛而已,無上天官符籙之事。其《黃帝》四篇,《老子》二篇,最得深旨。故言陶弘景者,隱於句容,好陰陽五行,風角星算,修闢穀導引之法,受道經符籙,武帝素與之游。及禪代之際,弘景取圖讖之文,合成「景梁」字以獻之,由是恩遇甚厚。又撰《登真隱訣》,以證古有神仙之事;又言神丹可成,服之則能長生,與天地永畢。
 帝令弘景試合神丹,竟不能就,乃言中原隔絕,藥物不精故也。帝以為然,敬之尤甚。然武帝弱年好事,先受道法,及即位,猶自上章,朝士受道者眾。三吳及邊海之際,信之逾甚。陳武世居吳興,故亦奉焉。



 後魏之世,嵩山道士寇謙之,自云嘗遇真人成公興,後遇太上老君,授謙之為天師,而又賜之《雲中音誦科誡》二十卷。又使玉女授其服氣導引之法,遂得闢谷,氣盛體輕,顏色鮮麗。弟子十餘人,皆得其術。其後又遇神人李譜,云是老君玄孫,授其圖籙真經,劾召百神,六十餘卷,及銷煉金丹雲英八石玉漿之法。太武始光之初,奉其書而獻之。帝使
 謁者,奉玉帛牲牢,祀嵩岳,迎致其餘弟子,於代都東南起壇宇,給道士百二十餘人,顯揚其法,宣布天下。太武親備法駕而受符籙焉。自是道業大行,每帝即位,必受符籙,以為故事,刻天尊及諸仙之象而供養焉。遷洛已後,置道場於南郊之傍,方二百步。正月、十月之十五日,並有道士哥人百六人,拜而祠焉。後齊神武帝遷鄴,遂罷之。文襄之世,更置館宇,選其精至者使居焉。後周承魏,崇奉道法,每帝受籙,如魏之舊,尋與佛法俱滅,開皇初又興,高祖雅信佛法,於道士蔑如也。大業中,道士以術進者甚眾。其所以講經,由以《老子》為本,次講《莊子》及《靈
 寶》、《升玄》之屬。其餘眾經,或言傳之神人,篇卷非一。自云天尊姓樂名靜信,例皆淺俗,故世甚疑之。其術業優者,行諸符禁,往往神驗。而金丹玉液長生之事,歷代糜費,不可勝紀,竟無效焉。今考其經目之數,附之於此。大乘經六百一十七部,二千七十六卷。五百五十八部,一千六百九十七卷,經。五十九部,三百七十九卷,疏。小乘經四百八十七部,八百五十二卷。雜經三百八十部,七百一十六卷。雜經目殘缺,其見數如此。雜疑經一百七十二部,三百三十六卷。大乘律五十二部,九十一卷。小乘律八十部,四百七十二卷。七十七部,四百九十卷,律。二部,二十三卷,講疏。雜律二十七部,四十六卷。大乘論三
 十五部,一百四十一卷。三十部,九十四卷,論。十五部,四十七卷,疏。



 小乘論四十一部,五百六十七卷。二十一部,四百九十一卷,論。十部,七十六卷,講疏。雜論五十一部,四百三十七卷。三十二部,二百九十九卷,論。九部,一百三十八卷,講疏。記二十部,四百六十四卷。



 右一千九百五十部,六千一百九十八卷。



 佛經者,西域天竺之迦維衛國凈飯王太子釋迦牟尼所說。釋迦當周莊王之九年四月八日,自母右脅而生,姿貌奇異,有三十二相,八十二好。舍太子位,出家學道,勤行精進,覺悟一切種智,而謂之佛,亦曰佛陀,亦曰浮屠,皆胡言也。華言譯之為凈覺。其所說云,人身雖有生
 死之異,至於精神則恆不滅。



 此身之前,則經無量身矣。積而修習,精神清凈,則成佛道。天地之外,四維上下,更有天地,亦無終極,然皆有成有敗。一成一敗,謂之一劫。自此天地已前,則有無量劫矣。每劫必有諸佛得道,出世教化,其數不同。今此劫中,當有千佛。自初至於釋迦,已七佛矣。其次當有彌勒出世,必經三會,演說法藏,開度眾生。由其道者,有四等之果。一曰須陀洹,二曰斯陀含,三曰阿那含,四曰阿羅漢。至羅漢者,則出入生死,去來隱顯,而不為累。阿羅漢已上,至菩薩者,深見佛性,以至成道。每佛滅度,遺法相傳,有正、象、末三等淳樗之異。年
 歲遠近,亦各不同。末法已後,眾生愚鈍,無復佛教,而業行轉惡,年壽漸短,經數百千載間,乃至朝生夕死。然後有大水、大火、大風之災,一切除去之,而更立生人,又歸淳樸,謂之小劫。每一小劫,則一佛出世。



 初,天竺中多諸外道,並事水火毒龍,而善諸變幻。釋迦之苦行也,是諸邪道,並來嬲惱,以亂其心,而不能得。及佛道成,盡皆摧伏,並為弟子。弟子,男曰桑門,譯言息心,而總曰僧,譯言行乞。



 女曰比丘尼。皆剃落須發,釋累辭家,相與和居,治心修凈,行乞以自資,而防心攝行。僧至二百五十戒,尼五百戒。俗人信憑佛法者,男曰優婆塞,女曰優婆夷,皆
 去殺、盜、淫、妄言、飲酒,是為五誡。



 釋迦在世教化四十九年,乃至天龍人鬼並來聽法,弟子得道,以百千萬億數。然後於拘尸那城娑羅雙樹間,以二月十五日,入般涅槃。涅槃亦曰泥洹,譯言滅度,亦言常樂我凈。初釋迦說法,以人之性識根業各差,故有大乘小乘之說。至是謝世,弟子大迦葉與阿難等五百人,追共撰述,綴以文字,集載為十二部。後數百年,有羅漢菩薩,相繼著論,贊明其義。然佛所說,我滅度後,正法五百年,像法一千年,末法三千年,其義如此。



 推尋典籍,自漢已上,中國未傳。或云久以流布,遭秦之世,所以堙滅。其後張騫使西域,蓋
 聞有浮屠之教。哀帝時,博士弟子秦景使伊存口授浮屠經,中土聞之,未之信也。後漢明帝夜夢金人飛行殿庭,以問於朝,而傅毅以佛對。帝遣郎中蔡愔及秦景使天竺求之,得佛經四十二章及釋迦立像。並與沙門攝摩騰、竺法蘭東還。愔之來也,以白馬負經,因立白馬寺於洛城雍門西以處之。其經緘於蘭臺石室,而又畫像於清涼臺及顯節陵上。章帝時,楚王英以崇敬佛法聞,西域沙門,齎佛經而至者甚眾。永平中,法蘭又譯《十住經》。其餘傳譯,多未能通。至桓帝時,有安息國沙門安靜,齎經至洛,翻譯最為通解。



 靈帝時,有月支沙門支讖、天
 竺沙門竺佛朔等,並翻佛經。而支讖所譯《泥洹經》二卷,學者以為大得本旨。漢末,太守竺融,亦崇佛法。三國時,有西域沙門康僧會,齎佛經至吳譯之,吳主孫權,甚大敬信。魏黃初中,中國人始依佛戒,剃發為僧。先是西域沙門來此,譯《小品經》,首尾乖舛,未能通解。



 甘露中,有硃仕行者,往西域,至於闐國,得經九十章,晉元康中,至鄴譯之,題曰《放光般若經》。



 太始中,有月支沙門竺法護,西游諸國,大得佛經,至洛翻譯,部數甚多。佛教東流,自此而盛。



 石勒時,常山沙門衛道安,性聰敏,誦經日至萬餘言。以胡僧所譯《維摩》《法華》,未盡深旨,精思十年,心了神
 悟,乃正其乖舛,宣揚解釋。時中國紛擾,四方隔絕,道安乃率門徒,南游新野,欲令玄宗所在流布,分遣弟子,各趨諸方。法性詣揚州,法和入蜀,道安與慧遠之襄陽。後至長安,苻堅甚敬之。道安素聞天竺沙門鳩摩羅什,思通法門,勸堅致之。什亦聞安令問,遙拜致敬。姚萇弘始二年,羅什至長安,時道安卒後已二十載矣,什深慨恨。什之來也,大譯經論,道安所正,與什所譯,義如一,初無乖舛。



 初,晉元熙中,新豐沙門智猛,策杖西行,到華氏城,得《泥洹經》及《僧祗律》,東至高昌,譯《泥洹》為二十卷。後有天竺沙門曇摩羅讖復齎胡本,來至河西。沮渠蒙遜
 遣使至高昌取猛本,欲相參驗,未還而蒙遜破滅。姚萇弘始十年,猛本始至長安,譯為三十卷。曇摩羅讖又譯《金光明》等經。時胡僧至長安者數十輩,惟鳩摩羅什才德最優。其所譯則《維摩》、《法華》、《成實論》等諸經,及曇無懺所譯《金光明》,曇摩羅懺譯《泥洹》等經,並為大乘之學。而什又譯《十誦律》,天竺沙門佛陀耶舍譯《長阿含經》及《四方律》,兜佉勒沙門曇摩難提譯《增一阿含經》,曇摩耶舍譯《阿毗曇論》,並為小乘之學。其餘經論,不可勝記。自是佛法流通,極於四海矣。東晉隆安中,又有罽賓沙門僧伽提婆譯《增一阿含經》及《中阿含經》。義熙中,沙門支
 法領從於闐國得《華嚴經》三萬六千偈,至金陵宣譯。又有沙門法顯,自長安游天竺,經三十餘國,隨有經律之處,學其書語,譯而寫之。還至金陵,與天竺禪師跋羅參共辯定,謂《僧祗律》,學者傳之。



 齊梁及陳,並有外國沙門。然所宣譯,無大名部可為法門者。梁武大崇佛法,於華林園中,總集釋氏經典,凡五千四百卷。沙門寶唱撰《經目錄》。又後魏時,太武帝西征長安,以沙門多違佛律,群聚穢亂,乃詔有司,盡坑殺之,焚破佛像。長安僧徒,一時殲滅。自餘征鎮,豫聞詔書,亡匿得免者十一二。文成之世,又使修復。熙平中,遣沙門慧生使西域,採諸經律,得
 一百七十部。永平中,又有天竺沙門菩提留支,大譯佛經,與羅什相埒。其《地持》、《十地論》,並為大乘學者所重。後齊遷鄴,佛法不改。至周武帝時,蜀郡沙門衛元嵩上書,稱僧徒猥濫,武帝出詔,一切廢毀。



 開皇元年,高祖普詔天下;任聽出家,仍令計口出錢,營造經像。而京師及並州、相州、洛州等諸大都邑之處,並官寫一切經,置於寺內;而又別寫,藏於秘閣。天下之人,從風而靡,競相景慕,民間佛經,多於六經數十百倍。大業時,又令沙門智果,於東都內道場撰諸經目,分別條貫,以佛所說經為三部;一曰大乘,二曰小乘,三曰雜經。其餘似後人假托為
 之者,別為一部,謂之疑經。又有菩薩及諸深解奧義、贊明佛理者,名之為論,及戒律並有大、小及中三部之別。又所學者,錄其當時行事,名之為記。凡十一種。今舉其大數,列於此篇。



 右道、佛經二千三百二十九部,七千四百一十四卷。



 道、佛者,方外之教,聖人之遠致也。俗士為之,不通其指,多離以迂怪,假托變幻亂於世,斯所以為弊也。故中庸之教,是所罕言,然亦不可誣也。故錄其大綱,附於四部之末。



 大凡經傳存亡及道、佛,六千五百二十部,五萬六千八百八十一卷。



\end{pinyinscope}