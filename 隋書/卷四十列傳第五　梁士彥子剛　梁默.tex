\article{卷四十列傳第五 梁士彥子剛 梁默}

\begin{pinyinscope}

 梁士彥,字相如,安定烏氏人也。少任俠,不仕州郡。性剛果,喜正人之是非。



 好讀兵書,頗涉經史。周世以軍功拜儀同三司。武帝將有事東夏,聞其勇決,自扶風郡守除九曲鎮將,進位上開府,封建威縣公,齊人甚憚焉。尋遷熊州刺史。後從武帝拔晉州,進位柱國,除使持節、晉絳二州諸軍事、晉州刺史。及帝還後,齊後主親總六軍而
 圍之。獨守孤城,外無聲援,眾皆震懼,士彥慷慨自若。賊盡銳攻之,樓堞皆盡,城雉所存,尋仞而已。或短兵相接,或交馬出入。士彥謂將士曰:「死在今日,吾為爾先!」於是勇烈齊奮,呼聲動地,無不一當百。齊師少卻。乃令妻妾軍民子女,晝夜修城,三日而就。帝率六軍亦至,齊師解圍,營於城東十餘里。



 士彥見帝,持帝須而泣曰:「臣幾不見陛下!」帝亦為之流涕。時帝以將士疲倦,意欲班師。士彥叩馬諫曰:「今齊師遁,眾心皆動,因其懼也而攻之,其勢必舉。」



 帝從之,大軍遂進。帝執其手曰:「余之有晉州,為平齊之基。若不固守,則事不諧矣。朕無前慮,惟恐後變,
 善為我守之。」及齊平,封郕國公,進位上柱國、雍州主簿。宣帝即位,除東南道行臺、使持節、徐州總管、三十二州諸軍事、徐州刺史。與烏丸軌擒陳將吳明徹、裴忌於呂梁,別破黃陵,略定淮南地。



 高祖作相,轉亳州總管、二十四州諸軍事。尉迥之反也,以為行軍總管,從韋孝寬擊之。至河陽,與迥軍相對。令家僮梁默等數人為前鋒,士彥以其徒繼之,所當皆破。乘勝至草橋,迥眾復合,進戰,大破之。及圍鄴城,攻北門而入,馳啟西門,納宇文忻之兵。



 及迥平,除相州刺史。高祖忌之,未幾,徵還京師,閑居無事。自恃元功,甚懷怨望,遂與宇文忻、劉昉等謀作亂。
 將率僮僕,於享廟之際,因車駕出,圖以發機。復欲於蒲州起事,略取河北,捉黎陽關,塞河陽路,劫調布以為牟甲,募盜賊以為戰士。其甥裴通豫知其謀而奏之。高祖未發其事,授晉州刺史,欲觀其意。士彥欣然謂昉等曰:「天也!」又請儀同薛摩兒為長史,高祖從之。後與公卿朝謁,高祖令左右執士彥、忻、昉等於行間,詰之曰:「爾等欲反,何敢發此意?」初猶不伏,捕薛摩兒適至,於是庭對之。摩兒具論始末,云:「第二子剛垂泣苦諫,第三子叔諧曰:作猛獸要須成斑。」士彥失色,顧謂摩兒曰:「汝殺我!」於是伏誅,時年七十二。



 有子五人。操字孟德,出繼伯父,官至
 上開府、義鄉縣公、長寧王府驃騎,早卒。剛字永固,弱冠授儀同,以平尉迥勛,加開府。擊突厥有功,進位上大將軍、通政縣公、涇州刺史。士彥之誅也,以諫獲免,徙瓜州。叔諧官至上儀同、廣平縣公、車騎將軍。志遠為安定伯,務為建威伯,皆坐士彥誅。



 梁默者,士彥之蒼頭,驍武絕人。士彥每從征伐,常與默陷陣。仕周,致位開府。開皇末,以行軍總管從楊素北征突厥,進位大將軍。漢王諒之反也,復以行軍總管從楊素討平之,加授柱國。大業五年,從煬帝征吐谷渾,遇賊力戰而死,贈光祿大夫。



 宇文忻宇文忻,字仲樂,本朔方人,徙京兆。祖莫豆于,魏安平公。父貴,周大司馬、許國公。忻幼而敏慧,為兒童時,與群輩游戲,輒為部伍,進止行列,無不用命,有識者見而異之。年十二,能左右馳射,驍捷若飛。恆謂所親曰:「自古名將,唯以韓、白、衛、霍為美談,吾察其行事,未足多尚。若使與僕並時,不令豎子獨擅高名也。」其少小慷慨如此。年十八,從周齊王憲討突厥有功,拜儀同三司,賜爵興固縣公。韋孝寬之鎮玉壁也,以忻驍勇,請與同行。屢有戰功,加位開府、驃騎將軍,進爵化政郡公,邑二千戶。



 從武帝伐齊,攻拔晉州。齊後主親馭六軍,兵勢甚盛,帝憚之,欲
 旋師。忻諫曰:「以陛下之聖武,乘敵人之荒縱,何往不克!若使齊人更得令主,君臣協力,雖湯、武之勢,未易平也。今主暗臣愚,兵無鬥志,雖有百萬之眾,實為陛下奉耳。」



 帝從之,戰遂大克。及帝攻陷並州,先勝後敗,帝為賊所窘,左右皆殲,帝挺身而遁,諸將多勸帝還。忻勃然而進曰:「自陛下克晉州,破高緯,乘勝逐北,以至於此。致令偽主奔波,關東響振,自古行兵用師,未有若斯之盛也。昨日破城,將士輕敵,微有不利,何足為懷。丈夫當死中求生,敗中取勝。今者破竹,其勢已成,奈何棄之而去?」帝納其言,明日復戰,遂拔晉陽。及齊平,進位大將軍,賜物千
 段。尋與烏丸軌破陳將吳明徹於呂梁,進位柱國,賜奴婢二百口,除豫州總管。



 高祖龍潛時,與忻情好甚協,及為丞相,恩顧彌隆。尉迥作亂,以忻為行軍總管,從韋孝寬擊之。時兵屯河陽,諸軍莫敢先進。帝令高熲馳驛監軍,與熲密謀進取者,唯忻而已。迥遣子惇,盛兵武陟,忻先鋒擊走之。進臨相州,迥遣精甲三千伏於野馬岡,欲邀官軍。忻以五百騎襲之,斬獲略盡。進至草橋,迥又拒守,忻率奇兵擊破之,直趨鄴下。迥背城結陣,與官軍大戰,官軍不利。時鄴城士女觀戰者數萬人,忻與高熲、李詢等謀曰:「事急矣,當以權道破之。」於是擊所觀者,大囂
 而走,轉相騰藉,聲如雷霆。忻乃傳呼曰:「賊敗矣!」眾軍復振,齊力急擊之,迥軍大敗。及平鄴城,以功加上柱國,賜奴婢二百口,牛馬羊萬計。高祖顧謂忻曰:「尉迥傾山東之眾,運百萬之師,公舉無遺策,戰無全陣,誠天下之英傑也。」進封英國公,增邑三千戶。自是以後,每參帷幄,出入臥內,禪代之際,忻有力焉。



 後拜右領軍大將軍,恩顧彌重。



 忻妙解兵法,馭戎齊整,當時六軍有一善事,雖非忻所建,在下輒相謂曰:「此必英公法也。」其見推服如此。後改封巳國公。上嘗欲令忻率兵擊突厥,高熲言於上曰:「忻有異志,不可委以大兵。」乃止。忻既佐命功臣,頻經
 將領,有威名於當世。上由是微忌焉,以譴去官。忻與梁士彥暱狎,數相往來,士彥時亦怨望,陰圖不軌。忻謂士彥曰:「帝王豈有常乎?相扶即是。公於蒲州起事,我必從征。兩陣相當,然後連結,天下可圖也。」謀洩伏誅,年六十四,家口籍沒。



 忻兄善,弘厚有武藝。仕周,官至上柱國、許國公。高祖受禪,遇之甚厚,拜其子穎為上儀同。及忻誅,並廢於家。善未幾卒。穎至大業中為司農少卿。及李密逼東都,叛歸於密。忻弟愷,別有傳。



 王誼王誼,字宜君,河南洛陽人也。父顯,周鳳州刺史。誼少慷
 慨,有大志,便弓馬,博覽群言。周閔帝時,為左中侍上士。時大塚宰宇文護執政,勢傾王室,帝拱默無所關預。有朝士於帝側微為不恭,誼勃然而進,將擊之。其人惶懼請罪,乃止。



 自是朝士無敢不肅。歲餘,遷御正大夫。丁父艱,毀瘁過禮,廬於墓側,負士成墳。



 歲餘,起拜雍州別駕,固讓,不許。武帝即位,授儀同,累遷內史大夫,封楊國公。



 從帝伐齊,至並州,帝既入城,反為齊人所敗,左右多死。誼率麾下驍雄赴之,帝賴以全濟。時帝以六軍挫衄,將班師。誼固諫,帝從之。及齊平,授相州刺史。未幾,復徵為大內史。汾州稽胡為亂,誼率兵擊之。帝弟越王盛、譙王
 儉雖為總管,並受誼節度。其見重如此。及平賊而還,賜物五千段,封一子開國公。帝臨崩,謂皇太子曰:「王誼社稷臣,宜處以機密,不須遠任也。」



 皇太子即位,是為宣帝。憚誼剛正,出為襄州總管。及高祖為丞相,轉為鄭州總管。司馬消難舉兵反,高祖以誼為行軍元帥,率四總管討之。軍次近郊,消難懼而奔陳。於時北至商洛,南拒江淮,東西二千餘里,巴蠻多叛,共推渠帥蘭雒州為主。雒州自號河南王,以附消難,北連尉迥。誼率行軍總管李威、馮暉、李遠等分討之,旬月皆平。高祖以誼前代舊臣,甚加禮敬,遣使勞問,冠蓋不絕。以第五女妻其子奉孝,
 尋拜大司徒。誼自以與高祖有舊,亦歸心焉。



 及上受禪,顧遇彌厚,上親幸其第,與之極歡。太常卿蘇威立議,以為戶口滋多,民田不贍,欲減功臣之地以給民。誼奏曰:「百官者,歷世勛賢,方蒙爵土,一旦削之,未見其可。如臣所慮,正恐朝臣功德不建,何患人田有不足?」上然之,竟寢威議。開皇初,上將幸岐州。誼諫曰:「陛下初臨萬國,人情未洽,何用此行?」



 上戲之曰:「吾昔與公位望齊等,一朝屈節為臣,或當恥愧。是行也,震揚威武,欲以服公心耳。」誼笑而退。尋奉使突厥,上嘉其稱旨,進封郢國公。



 未幾,其子奉孝卒。逾年,誼上表,言公主少,請除服。御史大夫
 楊素劾誼曰:「臣聞喪服有五,親疏異節,喪制有四,降殺殊文。王者之所常行,故曰不易之道也。是以賢者不得逾,不肖者不得不及。而儀同王奉孝既尚蘭陵公主,奉孝以去年五月身喪,始經一周,而誼便請除釋。竊以雖曰王姬,終成下嫁之禮,公則主之,猶在移天之義。況復三年之喪,自上達下,及期釋服,在禮未詳。然夫婦則人倫攸始,喪紀則人道至大,茍不重之,取笑君子。故鉆燧改火,責以居喪之速;朝祥暮歌,譏以忘哀之早。然誼雖不自強,爵位已重,欲為無禮,其可得乎?乃薄俗傷教,為父則不慈;輕禮易喪,致婦於無義。若縱而不正,恐傷風
 俗,請付法推科。」有詔勿治,然恩禮稍薄。誼頗怨望。或告誼謀反,上令案其事。主者奏誼有不遜之言,實無反狀。上賜酒而釋之。於時上柱國元諧亦頗失意,誼數與相往來,言論醜惡。



 胡僧告之,公卿奏誼大逆不道,罪當死。上見誼,愴然曰:「朕與公舊為同學,甚相憐愍,將奈國法何?」於是下詔曰:「誼,有周之世,早豫人倫,朕共游庠序,遂相親好。然性懷險薄,巫覡盈門,鬼言怪語,稱神道聖。朕受命之初,深存誡約,口雲改悔,心實不悛。乃說四天正神道,誼應受命,書有誼讖,天有誼星,桃、鹿二川,岐州之下,歲在辰巳,興帝王之業。密令卜問,伺殿省之災。又說
 其身是明王,信用左道,所在詿誤,自言相表,當王不疑。此而赦之,將或為亂,禁暴除惡,宜伏國刑。」上復令大理正趙綽謂誼曰:「時命如此,將若之何!」於是賜死於家,時年四十六。



 元諧元諧,河南洛陽人也,家代貴盛。諧性豪俠,有氣調。少與高祖同受業於國子,甚相友愛。後以軍功,累遷大將軍。及高祖為丞相,引致左右。諧白高祖曰:「公無黨援,譬如水間一堵墻,大危矣。公其勉之。」尉迥作亂,遣兵寇小鄉,令諧擊破之。及高祖受禪,上顧諧笑曰:「水間墻竟何如
 也?」於是賜宴極歡。進位上大將軍,封樂安郡公,邑千戶。奉詔參修律令。



 時吐谷渾寇涼州,詔諧為行軍元帥,率行軍總管賀婁子干、郭竣、元浩等步騎數萬擊之。上敕諧曰:「公受朝寄,總兵西下,本欲自寧疆境,保全黎庶,非是貪無用之地,害荒服之民。王者之師,意在仁義。渾賊若至界首者,公宜曉示以德,臨之以教,誰敢不服也!」時賊將定城王鐘利房率騎三千渡河,連結黨項。諧率兵出鄯州,趣青海,邀其歸路。吐谷渾引兵拒諧,相遇於豐利山。賊鐵騎二萬,與諧大戰,諧擊走之。賊駐兵青海,遣其太子可博汗以勁騎五萬來掩官軍。諧逆擊,敗之,追
 奔三十餘里,俘斬萬計,虜大震駭。於是移書諭以禍福,其名王十七人、公侯十三人各率其所部來降。上大悅,下詔曰:「褒善疇庸,有聞前載,諧識用明達,神情警悟,文規武略,譽流朝野。申威拓土,功成疆埸,深謀大節,實簡朕心。加禮延代,宜隆賞典。可柱國,別封一子縣公。」諧拜寧州刺史,頗有威惠。然剛愎,好排詆,不能取媚於左右。嘗言於上曰:「臣一心事主,不曲取人意。」上曰:「宜終此言。」後以公事免。



 時上柱國王誼有功於國,與諧俱無任用,每相往來。胡僧告諧、誼謀反,上按其事,無逆狀,上慰諭而釋之。未幾,誼伏誅,諧漸被疏忌。然以龍潛之舊,每預
 朝請,恩禮無虧。及上大宴百僚,諧進曰:「陛下威德遠被,臣請突厥可汗為候正,陳叔寶為令史。」上曰:「朕平陳國,以伐罪吊人,非欲誇誕取威天下。公之所奏,殊非朕心。突厥不知山川,何能警候!叔寶昏醉,寧堪驅使!」諧默然而退。後數歲,有人告諧與從父弟上開府滂、臨澤侯田鸞、上儀同祁緒等謀反。上令案其事。



 有司奏:「諧謀令祁緒勒黨項兵,即斷巴蜀。時廣平王雄、左僕射高熲二人用事,諧欲譖去之,云:『左執法星動已四年矣,狀一奏,高熲必死。』又言:『太白犯月,光芒相照,主殺大臣,楊雄必當之。』諧嘗與滂同謁上,諧私謂滂曰:『我是主人,殿上者賊
 也。』因令滂望氣,滂曰:『彼云似蹲狗走鹿,不如我輩有福德云。』」



 上大怒,諧、滂、鸞、緒並伏誅,籍沒其家。



 王世積王世積,闡熙新渼人也。父雅,周使持節、開府儀同三司。世積容貌魁岸,腰帶十圍,風神爽拔,有傑人之表。在周有軍功,拜上儀同,封長子縣公。高祖為丞相,尉迥作亂,從韋孝寬擊之,每戰有功,拜上大將軍。高祖受禪,進封宜陽郡公。



 高熲美其才能,甚善之。嘗密謂熲曰:「吾輩俱周之臣子,社稷淪滅,其若之何?」



 熲深拒其言。未幾,授蘄州總管。平陳之役,以舟師自蘄水趣九江,與陳將紀瑱
 戰於蘄口,大破之。既而晉王廣已平丹陽,世積於是移書告諭,遣千金公權始璋略取新蔡。陳江州司馬黃人思棄城而遁,始璋入據其城。世積繼至,陳豫章太守徐璒、廬陵太守蕭廉、潯陽太守陸仲容、巴山太守王誦、太原太守馬頲、齊昌太守黃正始、安成太守任瓘等,及鄱陽、臨川守將,並詣世積降。以功進位柱國、荊州總管,賜絹五千段,加之寶帶,邑三千戶。後數歲,桂州人李光仕作亂,世積以行軍總管討平之。上遣都官員外郎辛凱卿馳勞之。及還,進位上柱國,賜物二千段。上甚重之。



 世積見上性忌刻,功臣多獲罪,由是縱酒,不與執政言及時
 事。上以為有酒疾,舍之宮內,令醫者療之。世積詭稱疾愈,始得就第。及起遼東之役,世積與漢王並為行軍元帥,至柳城,遇疾疫而還。拜涼州總管,令騎士七百人送之官。未幾,其親信安定皇甫孝諧有罪,吏捕之,亡抵世積。世積不納,由是有憾。孝諧竟配防桂州,事總管令狐熙。熙又不之禮,甚困窮,因徼幸上變,稱:「世積嘗令道人相其貴不,道人答曰:『公當為國主。』謂其妻曰:『夫人當為皇后。』又將之涼州,其所親謂世積曰:『河西天下精兵處,可以圖大事也。』世積曰:『涼州土曠人稀,非用武之國。』」由是被徵入朝,按其事。有司奏:「左衛大將軍元旻、右衛大
 將軍元胄、左僕射高熲,並與世積交通,受其名馬之贈。」世積竟坐誅,旻、胄等免官,拜孝諧為上大將軍。



 虞慶則虞慶則,京兆櫟陽人也。本姓魚。其先仕於赫連氏,遂家靈武,代為北邊豪傑。



 父祥,周靈武太守。慶則幼雄毅,性倜儻,身長八尺,有膽氣,善鮮卑語,身被重鎧,帶兩鞬,左右馳射,本州豪俠皆敬憚之。初以弋獵為事,中便折節讀書,常慕傅介子、班仲升為人。仕周,釋褐中外府行參軍,稍遷外兵參軍事,襲爵沁源縣公。



 宣政元年,授儀同大將軍,除並州總管長史。二年,授開府。時稽胡數為反
 叛,越王盛、內史下大夫高熲討平之。將班師,熲與盛謀,須文武幹略者鎮遏之。表請慶則,於是即拜石州總管。甚有威惠,境內清肅,稽胡慕義而歸者八千餘戶。



 開皇元年,進位大將軍,遷內史監、吏部尚書、京兆尹,封彭城郡公,營新都總監。二年冬,空厥入寇,慶則為元帥討之。部分失所,士卒多寒凍,墮指者千餘人。偏將達奚長儒率騎兵二千人別道邀賊,為虜所圍,甚急,慶則案營不救。由是長儒孤軍獨戰,死者十八九。上不之責也。尋遷尚書右僕射。



 後突厥主攝圖將內附,請一重臣充使,於是上遣慶則詣突厥所。攝圖恃強,初欲亢禮,慶則責以
 往事,攝圖不服。其介長孫晟又說諭之,攝圖及弟葉護皆拜受詔,因即稱臣朝貢,請永為籓附。初,慶則出使,高祖敕之曰:「我欲存立突厥,彼送公馬,但取五三匹。」攝圖見慶則,贈馬千匹,又以女妻之。上以慶則勛高,皆無所問。授上柱國,封魯國公,食任城縣千戶。詔以彭城公回授第二子義。



 高祖平陳之後,幸晉王第,置酒會群臣。高熲等奉觴上壽,上因曰:「高熲平江南,虞慶則降突厥,可謂茂功矣。」楊素曰:「皆由至尊威德所被。」慶則曰:「楊素前出兵武牢、硤石,若非至尊威德,亦無克理。」遂與互相長短。御史欲彈之,上曰:「今日計功為樂,宜不須劾。」上觀群
 臣宴射,慶則進曰:「臣蒙賚酒食,令盡樂,御史在側,恐醉而被彈。」上賜御史酒,因遣之出。慶則奉觴上壽,極歡。上謂諸公曰:「飲此酒,願我與諸公等子孫常如今日,世守富貴。」九年,轉為右衛大將軍,尋改為右武候大將軍。



 開皇十七年,嶺南人李賢據州反,高祖議欲討之。諸將二三請行,皆不許。高祖顧謂慶則曰:「位居宰相,爵乃上公,國家有賊,遂無行意,何也?」慶則拜謝恐懼,上乃遣焉。為桂州道行軍總管,以婦弟趙什柱為隨府長史。什柱先與慶則愛妾通,恐事彰,乃宣言曰:「慶則不欲此行。」遂聞於上。先是,朝臣出征,上皆宴別,禮賜遣之。及慶則南討
 辭上,上色不悅,慶則由是怏怏不得志。暨平賢,至潭州臨桂鎮,慶則觀眺山川形勢,曰:「此誠險固,加以足糧,若守得其人,攻不可拔。」遂使什柱馳詣京奏事,觀上顏色。什柱至京,因告慶則謀反。上案驗之,慶則於是伏誅。拜什柱為柱國。



 慶則子孝仁,幼豪俠任氣,起家拜儀同,領晉王親信。坐父事除名。煬帝嗣位,以籓邸之舊,授候衛長史,兼領金谷監,監禁苑。有巧思,頗稱旨。九年,伐遼,授都水丞,充使監運,頗有功。然性奢華,以駱駝負函盛水養魚而自給。十一年,或告孝仁謀圖不軌,遂誅之。其弟澄道,東宮通事舍人,坐除名。



 元胄元胄,河南洛陽人也,魏昭成帝之六代孫。祖順,魏濮陽王。父雄,武陵王。



 胄少英果,多武藝,美須眉,有不可犯之色。周齊王憲見而壯之,引致左右,數從征伐。官至大將軍。高祖初被召入,將受顧托,先呼胄,次命陶澄,並委以腹心,恆宿臥內。及為丞相,每典軍在禁中,又引弟威俱入侍衛。周趙王招知高祖將遷周鼎,乃要高祖就第。趙王引高祖入寢室,左右不得從,唯楊弘與胄兄弟坐於戶側。



 趙王謂其二子員、貫曰:「汝當進瓜,我因刺殺之。」及酒酣,趙王欲生變,以佩刀子刺瓜,連啖高祖,將為不利。
 胄進曰:「相府有事,不可久留。」趙王訶之曰:「我與丞相言,汝何為者!」叱之使卻。胄瞋目憤氣,扣刀入衛。趙王問其姓名,胄以實對。趙王曰:「汝非昔事齊王者乎?誠壯士也!」因賜之酒,曰:「吾豈有不善之意邪?卿何猜警如是!」趙王偽吐,將入後閤,胄恐其為變,扶令上坐,如此者再三。趙王稱喉乾,命胄就廚取飲,胄不動。會滕王逌後至,高祖降階迎之,胄與高祖耳語曰:「事勢大異,可速去。」高祖猶不悟,謂曰:「彼無兵馬,復何能為?」胄曰:「兵馬悉他家物,一先下手,大事便去。胄不辭死,死何益耶?」



 高祖復入坐。胄聞屋後有被甲聲,遽請曰:「相府事殷,公何得如此?」因扶
 高祖下床,趣而去。趙王將追之,胄以身蔽戶,王不得出。高祖及門,胄自後而至。趙王恨不時發,彈指出血。及誅趙王,賞賜不可勝計。



 高祖受禪,進位上柱國,封武陵郡公,邑三千戶。拜左衛將軍,尋遷右衛大將軍。高祖從容曰:「保護朕躬,成此基業,元胄功也。」後數載,出為豫州刺史,歷亳、淅二州刺史。時突厥屢為邊患,朝廷以胄素有威名,拜靈州總管,北夷甚憚焉。後復徵為右衛大將軍,親顧益密。嘗正月十五日,上與近臣登高,時胄下直,上令馳召之。及胄見,上謂曰:「公與外人登高,未若就朕勝也。」賜宴極歡。晉王廣每致禮焉。房陵王之廢也,胄豫其
 謀。上正窮治東宮事,左衛大將軍元旻苦諫,楊素乃譖之。上大怒,執旻於仗。胄時當下直,不去,因奏曰:「臣不下直者,為防元旻耳。」復以此言激怒上,上遂誅旻,賜胄帛千匹。蜀王秀之得罪,胄坐與交通,除名。



 煬帝即位,不得調。時慈州刺史上官政坐事徙嶺南,將軍丘和亦以罪廢。胄與和有舊,因數從之游。胄嘗酒酣謂和曰:「上官政壯士也,今徙嶺表,得無大事乎?」



 因自拊腹曰:「若是公者,不徒然矣。」和明日奏之,胄竟坐死。於是徵政為驍衛將軍,拜和代州刺史。



 史臣曰:昔韓信愆垓下之期,則項王不滅;英布無淮南
 之舉,則漢道未隆。以二子之勛庸,咸憤怨而菹戮,況乃無古人之殊績,而懷悖逆之心者乎!梁士彥、宇文忻皆一時之壯士也,遭雲雷之會,並以勇略成名,遂貪天之功以為己力。報者倦矣,施者未厭,將生厲階,求逞其欲,及茲顛墜,自取之也。王誼、元諧、王世積、虞慶則、元胄,或契闊艱厄,或綢繆恩舊,將安將樂,漸見遺忘,內懷怏怏,矜伐不已。雖時主之刻薄,亦言語以速禍乎?然高祖佐命元功,鮮有終其天命,配享清廟,寂寞無聞。斯蓋草創帝圖,事出權道,本異同心,故久而逾薄。其牽牛蹊田,雖則有罪,奪之非道,能無怨乎?皆深文巧詆,致之刑闢,高
 祖沉猜之心,固已甚矣。求其餘慶,不亦難哉!



\end{pinyinscope}