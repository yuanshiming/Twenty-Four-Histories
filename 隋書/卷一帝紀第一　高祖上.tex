\article{卷一帝紀第一 高祖上}

\begin{pinyinscope}

 高祖文皇帝,姓楊氏,諱堅,弘農郡華陰人也。漢太尉震八代孫鉉,仕燕為北平太守。鉉生元壽,後魏代為武川鎮司馬,子孫因家焉。元壽生太原太守惠嘏,嘏生平原太守烈,烈生寧遠將軍禎,禎生忠,忠即皇考也。皇考從周太祖起義關西,賜姓普六茹氏,位至柱國、大司空、隋國公。薨,贈太保,謚曰桓。



 皇妣呂氏,以大統七年六月癸丑夜生高祖於馮翊般若寺,紫氣充庭。有尼來自河東,謂皇妣曰:「此兒所從來甚異,不可於俗間處之。」尼將高祖舍於別館,躬自撫養。皇妣嘗抱高祖,忽見頭上角出,遍體鱗起。皇妣大駭,墜高祖於地。尼自外入見曰:「已驚我兒,致令晚得天下。」為人龍頷,額上有五柱入頂,目光外射,有文在手曰「王」。長上短下,沈深嚴重。初入太學,雖至親暱不敢狎也。



 年十四,京兆尹薛善闢為功曹。十五,以太祖勛授散騎常侍、車騎大將軍、儀同三司,封成紀縣公。十六,遷驃騎大將軍,加開府。周太祖見而嘆曰:「此兒風骨,不似代間人。」明帝即位,授右小宮伯,進封大興
 郡公。帝嘗遣善相者趙昭視之,昭詭對曰:「不過作柱國耳。」既而陰謂高祖曰:「公當為天下君,必大誅殺而後定。善記鄙言。」武帝即位,遷左小宮伯。出為隋州刺史,進位大將軍。後徵還,遇皇妣寢疾三年,晝夜不離左右,代稱純孝。宇文護執政,尤忌高祖,屢將害焉,大將軍侯伏、侯壽等匡護得免。其後襲爵隋國公。武帝聘高祖長女為皇太子妃,益加禮重。齊王憲言於帝曰:「普六茹堅相貌非常,臣每見之,不覺自失。恐非人下,請早除之。」帝曰:「此止可為將耳。」內史王軌驟言於帝曰:「皇太子非社稷主,普六茹堅貌有反相。」帝不悅,曰:「必天命有在,將若之何!」
 高祖甚懼,深自晦匿。



 建德中,率水軍三萬,破齊師於河橋。明年,從帝平齊,進位柱國。與宇文憲破齊任城王高湝於冀州,除定州總管。先是,定州城西門久閉不行,齊文宣帝時,或請開之,以便行路。帝不許,曰:「當有聖人來啟之。」及高祖至而開焉,莫不驚異。尋轉亳州總管。宣帝即位,以後父徵拜上柱國、大司馬。大象初,遷大後丞、右司武,俄轉大前疑。每巡幸,恆委居守。時帝為《刑經聖制》,其法深刻。高祖以法令滋章,非興化之道,切諫,不納。高祖位望益隆,帝頗以為忌。帝有四幸姬,並為皇后,諸家爭寵,數相毀譖。帝每忿怒,謂後曰:「必族滅爾家!」因召高祖,
 命左右曰:「若色動,即殺之。」高祖既至,容色自若,乃止。



 大象二年五月,以高祖為揚州總管,將發,暴有足疾,不果行。乙未,帝崩。



 時靜帝幼沖,未能親理政事。內史上大夫鄭譯、御正大夫劉昉以高祖皇后之父,眾望所歸,遂矯詔引高祖入總朝政,都督內外諸軍事。周氏諸王在籓者,高祖悉恐其生變,稱趙王招將嫁女於突厥為詞以征之。丁未,發喪。庚戌,周帝拜高祖假黃鉞、左大丞相,百官總己而聽焉。以正陽宮為丞相府,以鄭譯為長史,劉昉為司馬,具置僚佐。宣帝時,刑政苛酷,群心崩駭,莫有固志。至是,高祖大崇惠政,法令清簡,躬履節儉,天下悅
 之。



 六月,趙王招、陳王純、越王盛、代王達、滕王逌並至於長安。相州總管尉遲迥自以重臣宿將,志不能平,遂舉兵東夏。趙、魏之士,從者若流,旬日之間,眾至十餘萬。又宇文胄以滎州,石愻以建州,席毗以沛郡,毗弟叉羅以兗州,皆應於迥。迥遣子質於陳請援。高祖命上柱國、鄖國公韋孝寬討之。雍州牧畢王賢及趙、陳等五王,以天下之望歸於高祖,因謀作亂。高祖執賢斬之,寢趙王等之罪,因詔五王劍履上殿,入朝不趨,用安其心。



 七月,陳將陳紀、蕭摩訶等寇廣陵,吳州總管於顗轉擊破之。廣陵人杜喬生聚眾反,刺史元義討平之。韋孝寬破尉遲
 迥於相州,傳首闕下,餘黨悉平。初,迥之亂也,鄖州總管司馬消難據州響應,淮南州縣多同之。命襄州總管王誼討之,消難奔陳。荊、郢群蠻乘釁作亂,命亳州總管賀若誼討平之。先是,上柱國王謙為益州總管,既見幼主在位,政由高祖,遂起巴蜀之眾,以匡復為辭。高祖方以東夏、山南為事,未遑致討。謙進兵屯劍閣,陷始州。至是,乃命行軍元帥、上柱國梁睿討平之,傳首闕下。巴蜀阻險,人好為亂,於是更開平道,毀劍閣之路,立銘垂誡焉。



 五王陰謀滋甚,高祖齎酒肴以造趙王第,欲觀所為。趙王伏甲以宴高祖,高祖幾危,賴元胄以濟,語在胄傳。於
 是誅趙王招、越王盛。



 九月,以世子勇為洛州總管、東京小塚宰。壬子,周帝詔曰:「假黃鉞、使持節、左大丞相、都督內外諸軍事、上柱國、大塚宰、隋國公堅,感山河之靈,應星辰之氣,道高雅俗,德協幽顯。釋巾登仕,晉紳傾屬,開物成務,朝野承風。受詔先皇,弼諧寡薄,合天地而生萬物,順陰陽而撫四夷。近者內有艱虞,外聞妖寇,以鷹鸇之志,運帷帳之謀,行兩觀之誅,掃萬里之外。遐邇清肅,實所賴焉。四海之廣,百官之富,俱稟大訓,咸餐至道。治定功成,棟梁斯托,神猷盛德,莫二於時。可授大丞相,罷左、右丞相之官,餘如故。」冬十月壬申,詔贈高祖曾祖烈
 為柱國、太保、都督徐兗等十州諸軍事、徐州刺史、隋國公,謚曰康;祖禎為柱國、太傅、都督陜蒲等十三州諸軍事、同州刺史、隋國公,謚曰獻;考忠為上柱國、太師、大塚宰、都督冀定等十三州諸軍事、雍州牧。誅陳王純。癸酉,上柱國、鄖國公韋孝寬卒。十一月辛未,誅代王達、膝王逌。



 十二月甲子,周帝詔曰:天大地大,合其德者聖人;一陰一陽,調其氣者上宰。所以降神載挺,陶鑄群生,代蒼蒼之工,成巍巍之業。假黃鉞、使持節、大丞相、都督內外諸軍事、上柱國、大塚宰、隋國公,應百代之期,當千齡之運,家隆臺鼎之盛,門有翊贊之勤。



 心同伊尹,必致堯舜,
 情類孔丘,憲章文武。爰初入仕,風流映世,公卿仰其軌物,搢紳謂為師表。入處禁闈,出居籓政,芳猷茂績,問望彌遠。往平東夏,人情未安,燕南趙北,實為天府,擁節杖旄,任當連率,柔之以德,導之以禮,畏之若神,仰之若日,芳風美跡,歌頌獨存。淮海榛蕪,多歷年代,作鎮南鄙,選眾惟賢,威震殊俗,化行黔首。任掌金句陳,職司邦政,國之大事,朝寄更深,鑾駕巡游,留臺務廣。周公陜西之任,僅可為倫,漢臣關內之重,未足相況。



 及天崩地坼,先帝升遐,朕以眇年,奄經荼毒,親受顧命,保乂皇家。奸人乘隙,潛圖宗社,無君之意已成,竊發之期有日。英規潛運,大
 略川回,匡國庇人,罪人斯得。兩河遘亂,三魏稱兵,半天之下,洶洶鼎沸。祖宗之基已危,生人之命將殆。安陸作釁,南通吳越,蜂飛蠆聚,江漢騷然,巴蜀鴟張,翻將問鼎,秦途更阻,漢門重閉。畫籌帷帳,建出師車,諸將稟其謀,壯士感其義,不違時日,咸得清蕩。九功遠被,七德允諧,百僚師師,四門穆穆。光景照臨之地,風雲去來之所,允武允文,幽明同德,驟山驟水,遐邇歸心。使朕繼踵上皇,無為以治,聲高宇宙,道格天壤。伊尹輔殷,霍光佐漢,方之蔑如也。



 昔營丘、曲阜,地多諸國,重耳、小白,錫用殊禮。蕭何優贊拜之儀,番君越公侯之爵。姬、劉以降,代有令
 謨,宜崇典禮,憲章自昔。可授相國,總百揆,去都督內外諸軍事、大塚宰之號,進公爵為王,以隋州之崇業,鄖州之安陸、城陽,溫州之宜人,應州之平靖、上明,順州之淮南,士州之永川,昌州之廣昌、安昌,申州之義陽、淮安,息州之新蔡、建安,豫州之汝南、臨潁、廣寧、初安,蔡州之蔡陽,郢州之漢東二十郡為隋國。劍履上殿,入朝不趨,贊拜不名,備九錫之禮,加璽紱、遠游冠、相國印、綠綟綬,位在諸侯王上。隋國置丞相已下,一依舊式。



 高祖再讓,不許。乃受王爵、十郡而已。詔進皇祖、考爵並為王,夫人為王妃。



 辛巳,司馬消難以陳師寇江州,刺史成休寧擊卻
 之。



 大定元年春二月壬子,令曰:「已前賜姓,皆復其舊。」是日,周帝詔曰:「伊、周作輔,不辭殊禮之錫,桓、文為霸,允應異物之典,所以表格天之勛,彰不代之業。相國隋王,前加典策,式昭大禮,固守謙光,絲言未糸孛。宜申顯命,一如往旨。王功必先人,賞存後己,退讓為本,誠乖朕意。宜命百闢,盡詣王宮,眾心克感,必令允納。如有表奏,勿復通聞。」癸丑,文武百官詣閤敦勸,高祖乃受。甲寅,策曰:咨爾假黃鉞、使持節、大丞相、都督內外諸軍事、上柱國、大塚宰隋王:天覆地載,藉人事以財成;日往月來,由王道而盈昃。五氣陶鑄,萬物流形。誰代上玄之工,斯則大聖而
 已。曰惟先正,翊亮皇朝。種德積善,載誕上相。精採不代,風骨異人。匡國濟時,除兇撥亂。百神奉職,萬國宅心。殷相以先知悟人,周輔乃弘道於代,方斯蔑如也。今將授王典禮,其敬聽朕命:朕以不德,早承丕緒,上靈降禍,夙遭愍兇。妖丑覬覦,密圖社稷,宮省之內,疑慮驚心。公受命先皇,志在匡弼,輯諧內外,潛運機衡,奸人懾憚,謀用丕顯,俾贅旒之危,為太山之固。是公重造皇室,作霸之基也。伊我祖考之代,任寄已深,入掌禁兵,外司籓政,文經武略,久播朝野。戎軒大舉,長驅晉魏,平陽震熊羆之勢,冀部耀貔豹之威。初平東夏,人情未一,叢臺之北,易
 水之南,西距井陘,東至滄海,比數千里,舉袂如帷。委以連城,建旌杖節,教因其俗,刑用輕典,如泥從印,猶草隨風。此又公之功也。吳越不賓,多歷年代,淮海之外,時非國有。爰整其旅,出鎮於亳,武以威物,文以懷遠。群盜自奔,外戶不閉,人黎慕義,襁負而歸。自北之風,化行南國。此又公之功也。宣帝御宇,任重宗臣,入典八屯,外司九伐。禁衛勤巡警之務,治兵得搜狩之禮。此又公之功也。鑾駕游幸,頻委留臺,文武注意,軍國諮稟。萬事咸理,反顧無憂。此又公之功也。朕在諒暗,公實總己。



 磐石之宗,奸回者眾,招引無賴,連結群小。往者國衰甫爾,已創陰
 謀,積惡數旬,昆吾方稔。泣誅磬甸,宗廟以寧。此又公之功也。尉迥猖狂,稱兵鄴邑,欲長戟而指北闕,強弩而圍南斗,憑陵三魏之間,震驚九州之半,聚徒百萬,悉成蛇豕,淇水洹水,一飲而竭。人之死生,翻系兇豎,壽之長短,不由司命。公乃戒彼鷹揚,出車練卒,誓蒼兕於河朔,建瓴水於山東。口授兵書,手畫行陣,量敵制勝,指日克期。諸將遵其成旨,壯士感其大義,輕死忘生,轉鬥千里,旗鼓奮發,如火燎毛。



 玄黃變漳河之水,京觀比爵臺之峻。百城氛昆,一旦廓清。此又公之功也。青土連率,跨據東秦,藉負海之饒,倚連山之險,望三輔而將逐鹿,指六國
 而願連雞,風雨之兵,助鬼為虐。本根既拔,枝葉自殞,屈法申恩,示以大信。此又公之功也。



 申部殘賊,充斥一隅,蠅飛蟻聚,攻州略地。播以玄澤,迷更知反,服而舍之,無費遺鏃。此又公之功也。宇文胄親則宗枝,外籓巖邑,影響鄴賊,有同就燥,迫脅吏人,叛換城戍。偏師討蹙,遂入網羅,束之武牢,有同囹圄,事窮將軍,如伏國刑。此又公之功也。檀讓、席毗,擁眾河外。陳韓梁鄭,宋衛鄒魯,村落成梟獍之墟,人庶為豺狼之餌。強以陵弱,大則吞小,城有晝閉,巷無行人。授律出師,隨機掃定,讓既授首,毗亦梟懸。此又公之功也。司馬消難與國親姻,作鎮安陸,性
 多嗜欲,意好貪聚。屬城子女,劫掠靡餘,部人貨財,多少具罄。擅誅刺舉之使,專殺儀臺之臣。懼罪畏威,動而內奰。蠶食郡縣,鴆毒華夷,聞有王師,自投南裔。



 帝唐崇山之罰,僅可方此,大漢流禦之刑,是亦相匹。逋逃入藪,荊郢用安。此又公之功也。王謙在蜀,翻為厲階,閉劍閣之門,塞靈關之宇,自謂五丁復起,萬夫莫向。分閫推轂,嘗不逾時,風馳席卷,一舉大定,擒斬兇惡,掃地無遺。此又公之功也。陳頊因循偽業,自擅金陵,屢遣醜徒,趑趄江北。公指麾籓鎮,無不摧殄。



 方置文深之柱,非止慰佗之拜。此又公之功也。



 公有濟天下之勤,重之以明德,始於
 闢命,屈己登庸。素業清徽,聲掩廊廟,雄規神略,氣蓋朝野。序百揆而穆四門,恥一匡之舉九合。尊賢崇德,尚齒貴功,錄舊旌善,興亡繼絕。寬猛相濟,彞倫攸敘,敦睦帝親,崇獎王室。星象不拆,陰陽自調,玄冥祝融,如奉太公之召;雨師風伯,似應成王之宰。祥風嘉氣,觸石搖林,瑞獸異禽,游園鳴閣。至功至德,可大可久,盡品物之和,究杳冥之極。



 朕又聞之,昔者明王設官胙土,營丘四履,得征五侯,參墟寵章,異其禮物。



 故籓屏作固,垂拱責成,沈默巖廊,不下堂席。公道高往烈,賞薄前王。朕以眇身,托於兆人之上,求諸故實,甚用懼焉。往加大典,憲章在昔,
 謙以自牧,未應朝禮,日月不居,便已隔歲,時談物議,其謂朕何!今進授相國總百揆,以申州之義陽等二十郡為隋國。今命使持節、太傅、上柱國、杞國公椿,大宗伯、大將軍、金城公趙煚,授相國印綬。相國禮絕百闢,任總群官,舊職常典,宜與事革。昔堯臣太尉,舜佐司空,姬旦相周,霍光輔漢,不居籓國,唯在天朝。其以相國總百揆,去眾號焉。上所假節、大丞相、大塚宰印綬。



 又加九錫,其敬聽朕后命:以公執律修德,慎獄恤刑,為其訓範,人無異志,是用錫公大輅、戎輅各一,玄牡二駟。公勤心地利,所寶人天,崇本務農,公私殷阜,是用錫公袞冕之服,赤舄
 副焉。公樂以移風,雅以變俗,遐邇胥悅,天地咸和,是用錫公軒懸之樂,六佾之舞。公仁風德教,覃及海隅,荒忽幽遐,回首內向,是用錫公硃戶以居。公水鏡人倫,銓衡庶職,能官流詠,遺賢必舉,是用錫公納陛以登。公執鈞於內,正性率下,犯義無禮,罔不屏黜,是用錫公武賁之士三百人。公元本闕。是用錫公鈇鉞各一。公威嚴夏日,精厲秋霜,猾夏必誅,顧眄天壤,掃清奸宄,折沖無外,是用錫公彤弓一、彤矢百,盧弓十、盧矢千。惟公孝通神明,肅恭祀典,尊嚴如在,情切幽明,是用錫公秬鬯一卣,珪瓚副焉。隋國置丞相以下,一遵舊式。往欽哉!其敬循往策,
 祗服大典,簡恤爾庶功,對揚我太祖之休命。



 於是建臺置官。



 丙辰,詔王冕十有二旒,建天子旌旗,出警入蹕,乘金根車,駕六馬,備五時副車,置旄頭雲蒨,樂舞八佾,設鐘虡宮懸。王妃為王後,長子為太子。前後三讓,乃受。



 俄而周帝以眾望有歸,乃下詔曰:「元氣肇闢,樹之以君,有命不恆,所輔惟德。天心人事,選賢與能,盡四海而樂推,非一人而獨有。周德將盡,妖孽遞生,骨肉多虞,籓維構釁,影響同惡,過半區宇,或小或大,圖帝圖王,則我祖宗之業,不絕如線。相國隋王,睿聖自天,英華獨秀,刑法與禮儀同運,文德共武功俱遠。



 愛萬物其如己,任兆庶以
 為憂。手運璣衡,躬命將士,芟夷奸宄,刷蕩氛昆,化通冠帶,威震幽遐。虞舜之大功二十,未足相比,姬發之合位三五,豈可足論。況木行已謝,火運既興,河洛出革命之符,星辰表代終之象。煙雲改色,笙簧變音,獄訟咸歸,謳歌盡至。且天地合德,日月貞明,故以稱大為王,照臨下土。朕雖寡昧,未達變通,幽顯之情,皎然易識。今便祗順天命,出遜別宮,禪位於隋,一依唐虞、漢魏故事。」高祖三讓,不許。遣兼太傅、上柱國、杞國公椿奉冊曰:咨爾相國隋王:粵若上古之初,爰啟清濁,降符授聖,為天下君。事上帝而理兆人,和百靈而利萬物,非以區宇之富,未以
 宸極為尊。大庭、軒轅以前,驪連、赫胥之日,咸以無為無欲,不將不迎。遐哉其詳不可聞已,厥有載籍,遺文可觀。



 聖莫逾於堯,美未過於舜。堯得太尉,已作運衡之篇,舜遇司空,便敘精華之竭。



 彼褰裳脫屣,貳宮設饗,百闢歸禹,若帝之初。斯蓋上則天時,不敢不授,下祗天命,不可不受。湯代於夏,武革於殷,干戈揖讓,雖復異揆,應天順人,其道靡異。



 自漢迄晉,有魏至周,天歷逐獄訟之歸,神鼎隨謳歌之去。道高者稱帝,錄盡者不王,與夫文祖、神宗,無以別也。



 周德將盡,禍難頻興,宗戚奸回,咸將竊發。顧瞻宮闕,將圖宗社,籓維連率,逆亂相尋。搖蕩三方,不
 合如礪,蛇行鳥攫,投足無所。王受天明命,叡德在躬,救頹運之艱,匡墜地之業,拯大川之溺,撲燎原之火,除群兇於城社,廓妖氛於遠服,至德合於造化,神用洽於天壤。八極九野,萬方四裔,圓首方足,罔不樂推。



 往歲長星夜掃,經天晝見,八風比夏后之作,五緯同漢帝之聚,除舊之徵,昭然在上。近者赤雀降祉,玄龜效靈,鐘石變音,蛟魚出穴,布新之貺,煥焉在下。九區歸往,百靈協贊,人神屬望,我不獨知。仰祗皇靈,俯順人願,今敬以帝位禪於爾躬。天祚告窮,天祿永終。於戲!王宜允執厥和,儀刑典訓,升圓丘而敬蒼昊,御皇極而撫黔黎,副率土之心,
 恢無疆之祚,可不盛歟!



 遣大宗伯、大將軍、金城公趙煚奉皇帝璽紱,百官勸進。高祖乃受焉。



 開皇元年二月甲子,上自相府常服入宮,備禮即皇帝位於臨光殿。設壇於南郊,遣使柴燎告天。是日,告廟,大赦,改元。京師慶雲見。易周氏官儀,依漢、魏之舊。以柱國、相國司馬、渤海郡公高熲為尚書左僕射兼納言,相國司錄、沁源縣公虞慶則為內史監兼吏部尚書,相國內郎、咸安縣男李德林為內史令,上開府、漢安縣公韋世康為禮部尚書,上開府、義寧縣公元暉為都官尚書,開府、民部尚書、昌國縣公元巖為兵部尚書,上儀同、司宗
 長孫毗為工部尚書,上儀同、司會楊尚希為度支尚書,上柱國、雍州牧、邗國公楊惠為左衛大將軍。乙丑,追尊皇考為武元皇帝,廟號太祖,皇妣為元明皇后。遣八使巡省風俗。丙寅,修廟社。立王後獨孤氏為皇后,王太子勇為皇太子。丁卯,以大將軍、金城郡公趙煚為尚書右僕射,上開府、濟陽侯伊婁彥恭為左武候大將軍。己巳,以周帝為介國公,邑五千戶,為隋室賓。旌旗車服禮樂,一如其舊。上書不為表,答表不稱詔。周氏諸王,盡降為公。



 辛未,以皇弟同安郡公爽為雍州牧。乙亥,封皇弟邵國公慧為滕王,同安公爽為衛王;皇子雁門公廣為晉
 王,俊為秦王,秀為越王,諒為漢王。以上柱國、並州總管、申國公李穆為太師,上柱國、鄧國公竇熾為太傅,上柱國、幽州總管、任國公於翼為太尉,觀國公田仁恭為太子太師,武德郡公柳敏為太子太保,濟南郡公孫恕為太子少傅,開府蘇威為太子少保。丁丑,以晉王廣為並州總管,以陳留郡公楊智積為蔡王,興城郡公楊靜為道王。戊寅,以官牛五千頭分賜貧人。三月辛巳,高平獲赤雀,太原獲蒼烏,長安獲白雀,各一。宣仁門槐樹連理,眾枝內附。壬午,白狼國獻方物。甲申,太白晝見。乙酉,又晝見。以上柱國元景山為安州總管。丁亥,詔犬馬器玩
 口味不得獻上。戊子,弛山澤之禁。以上開府、當亭縣公賀若弼為楚州總管,和州刺史、新義縣公韓擒虎為廬州總管。己丑,盩厔縣獻連理樹,植之宮庭。



 辛卯,以上柱國、神武郡公竇毅為定州總管。戊戌,以太子少保蘇威兼納言、吏部尚書,餘官如故。庚子,詔曰:「自古帝王受終革代,建侯錫爵,多與運遷。朕應籙受圖,君臨海內,載懷沿革,事有不同。然則前帝後王,俱在兼濟,立功立事,爵賞仍行。茍利於時,其致一揆,何謂物我之異,無計今古之殊。其前代品爵,悉可依舊。」丁未,梁主蕭巋使其太宰蕭巖、司空劉義來賀。四月辛巳,大赦。壬午,太白、歲星晝
 見。戊戌,太常散樂並放為百姓。禁雜樂百戲。辛丑,陳散騎常侍韋鼎、兼通直散騎常侍王瑳來聘於周,至而上已受禪,致之介國。是月,發稽胡修築長城,二旬而罷。五月戊子,封邗國公楊雄為廣平王,永康郡公楊弘為河間王。辛未,介國公薨,上舉哀於朝堂,以其族人洛嗣焉。六月癸未,詔以初受天命,赤雀降祥,五德相生,赤為火色,其郊及社廟,依服冕之儀,而朝會之服,旗幟犧牲,盡令尚赤。戎服以黃。秋七月乙卯,上始服黃,百僚畢賀。庚午,靺鞨酋長貢方物。



 八月壬午,廢東京官。突厥阿波可汗遣使貢方物。甲午,遣行軍元帥樂安公元諧擊吐谷
 渾於青海,破而降之。九月戊申,戰亡之家,遣使賑給。庚午,陳將周羅攻陷胡墅,蕭摩訶寇江北。辛未,以越王秀為益州總管,改封為蜀王。壬申,以上柱國、薛國公長孫覽,上柱國、宋安公元景山並為行軍元帥以伐陳,仍命尚書左僕射高熲節度諸軍。突厥沙缽略可汗遣使貢方物。是月,行五銖錢。冬十月乙酉,百濟王扶餘昌遣使來賀,授昌上開府、儀同三司、帶方郡公。戊子,行新律。壬辰,行幸岐州。十一月乙卯,以永昌郡公竇榮定為右武候大將軍。丁卯,遣兼散騎侍郎鄭捴使於陳。己巳,有流星,聲如隤墻,光燭於地。十二月戊寅,以申州刺史爾
 硃敞為金州總管。甲申,以禮部尚書韋世康為吏部尚書。己丑,以柱國元袞為廓州總管,興勢郡公衛玄為淮州總管。庚子,至自岐州。壬寅,高麗王高陽遣使朝貢,授陽大將軍、遼東郡公。太子太保柳敏卒。



 二年春正月癸丑,幸上柱國王誼第。庚申,幸安成長公主第。陳宣帝殂,子叔寶立。辛酉,置河北道行臺尚書省於並州,以晉王廣為尚書令。置河南道行臺尚書省於洛州,以秦王俊為尚書令。置西南道行臺尚書省於益州,以蜀王秀為尚書令。



 戊辰,陳遣使請和,歸我胡墅。辛未,高麗、百濟並遣使貢方物。甲戌,詔舉賢良。



 二月己丑,
 詔高熲等班師。庚寅,以晉王廣為左武衛大將軍,秦王俊為右武衛大將軍,餘官並如故。辛卯,幸趙國公獨孤陀第。庚子,京師雨土。三月戊申,開渠,引杜陽水於三疇原。四月丁丑,以寧州刺史竇榮定為左武候大將軍。庚寅,大將軍韓僧壽破突厥於雞頭山;上柱國李充破突厥於河北山。五月戊申,以上柱國、開府長孫平為度支尚書。己酉,旱,上親省囚徒。其日大雨。己未,高寶寧寇平州,突厥入長城。庚申,以豫州刺史皇甫績為都官尚書。壬戌,太尉、任國公於翼薨。甲子,改傳國璽曰受命璽。六月壬午,以太府卿蘇孝慈為兵部尚書,雍州牧、衛王爽
 為原州總管。甲申,使使吊於陳國。乙酉,上柱國李充破突厥於馬邑。戊子,以上柱國叱李長叉為蘭州總管。辛卯,以上開府爾硃敞為徐州總管。丙申,詔曰:「朕祗奉上玄,君臨萬國,屬生人之敝,處前代之宮。常以為作之者勞,居之者逸,改創之事,心未遑也。而王公大臣陳謀獻策,咸云羲、農以降,至於姬、劉,有當代而屢遷,無革命而不徙。曹、馬之後,時見因循,乃末代之晏安,非往聖之宏義。



 此城從漢,凋殘日久,屢為戰場,舊經喪亂。今之宮室,事近權宜,又非謀筮從龜,瞻星揆日,不足建皇王之邑,合大眾所聚,論變通之數,具幽顯之情同心固請,詞情
 深切。然則京師百官之府,四海歸向,非朕一人之所獨有。茍利於物,其可違乎!



 且殷之五遷,恐人盡死,是則以吉兇之土,制長短之命。謀新去故,如農望秋,雖暫劬勞,其究安宅。今區宇寧一,陰陽順序,安安以遷,勿懷胥怨。龍首山川原秀麗,卉物滋阜,卜食相土,宜建都邑,定鼎之基永固,無窮之業在斯。公私府宅,規模遠近,營構資費,隨事條奏。」仍詔左僕射高熲、將作大匠劉龍、巨鹿郡公賀婁子干、太府少卿高龍叉等創造新都。秋八月癸巳,以左武候大將軍竇榮定為秦州總管。十月癸酉,皇太子勇屯兵咸陽以備胡。庚寅,上疾愈,享百僚於觀德
 殿。賜錢帛,皆任其自取,盡力而出。辛卯,以營新都副監賀婁子乾為工部尚書。十一月丙午,高麗遣使獻方物。十二月辛未,上講武於後園。甲戌,上柱國竇毅卒。丙子,名新都曰大興城。乙酉,遣沁源公虞慶則屯弘化備胡。突厥寇周槃,行軍總管達奚長儒擊之,為虜所敗。丙戌,賜國子生經明者束帛。丁亥,親錄囚徒。



 三年春正月庚子,將入新都,大赦天下。禁大刀長槊。癸亥,高麗遣使來朝。



 二月己巳朔,日有蝕之。壬申,宴北道勛人。癸酉,陳遣兼散騎常侍賀徹、兼通直散騎常侍蕭褒來聘。突厥寇邊。甲戌,涇陽獲毛龜。癸未,以左衛大將
 軍李禮成為右武衛大將軍。三月丁未,上柱國、鮮虞縣公謝慶恩卒。己酉,以上柱國達奚長儒為蘭州總管。丙辰,雨,常服入新都。京師醴泉出。丁巳,詔購求遺書於天下。庚申,宴百僚,班賜各有差。癸亥,城榆關。夏四月己巳,上柱國、建平郡公於義卒。



 庚午,吐谷渾寇臨洮,洮州刺史皮子信死之。辛未,高麗遣使來朝。壬申,以尚書右僕射趙煚兼內史令。丁丑,以滕王瓚為雍州牧。己卯,衛王爽破突厥於白道。庚辰,行軍總管陰壽破高寶寧於黃龍。甲申,旱,上親祈雨於國城之西南。丙戌,詔天下勸學行禮。以濟北郡公梁遠為汶州總管。己丑,陳郢州城主
 張子譏遣使請降,上以和好,不納。辛卯,遣兼散騎常侍薛舒、兼通直散騎常侍王劭使於陳。癸巳,上親雩。甲午,突厥遣使來朝。五月癸卯,行軍總管李晃破突厥於摩那渡口。甲辰,高麗遣使來朝。乙巳,梁太子蕭琮來賀遷都。丁未,靺鞨貢方物。戊申,幽州總管陰壽卒。辛酉,有事於方澤。壬戌,行軍元帥竇榮定破突厥及吐谷渾於涼州。丙寅,赦黃龍死罪已下。六月庚午,以衛王爽子集為遂安郡王。戊寅,突厥遣使請和。庚辰,行軍總管梁遠破吐谷渾於爾汗山,斬其名王。壬申,以晉州刺史燕榮為青州總管。己丑,以河間王弘為寧州總管。乙未,幸安成
 長公主第。秋七月辛丑,以豫州刺史周搖為幽州總管。壬戌,詔曰:「行仁蹈義,名教所先,厲俗敦風,宜見褒獎。



 往者山東河表,經此妖亂,孤城遠守,多不自全。濟陰太守杜猷身陷賊徒,命懸寇手,郡省事範臺玫傾產營護,免其戮辱。眷言誠節,實有可嘉,宜超恆賞,用明沮勸。臺玫可大都督、假湘州刺史。」丁卯,日有蝕之。八月丁丑,靺鞨貢方物。己卯,以右武衛大將軍李禮成為襄州總管。壬午,遣尚書左僕射高熲出寧州道,內史監虞慶則出原州道,並為行軍元帥以擊胡。戊子,上有事於太社。九月壬子,幸城東,觀稼穀。癸丑,大赦天下。冬十月甲戌,廢河
 南道行臺省,以秦王俊為秦州總管。十一月己酉,發使巡省風俗,因下詔曰:「朕君臨區宇,深思治術,欲使生人從化,以德代刑,求草萊之善,旌閭里之行。民間情偽,咸欲備聞。已詔使人,所在賑恤,揚鑣分路,將遍四海,必令為朕耳目。如有文武才用,未為時知,宜以禮發遣,朕將銓擢。其有志節高妙,越等超倫,亦仰使人就加旌異,令一行一善,獎勸於人。遠近官司,遐邇風俗,巨細必紀,還日奏聞。庶使不出戶庭,坐知萬里。」



 庚辰,陳遣散騎常侍周墳、通直散騎常侍袁彥來聘。陳主知上之貌異世人,使彥畫像持去。甲午,罷天下諸郡。閏十二月乙卯,遣兼
 散騎常侍曹令則、通直散騎常侍魏澹使於陳。戊午,以上柱國竇榮定為右武衛大將軍,刑部尚書蘇威為民部尚書。



 四年春正月甲子,日有蝕之。己巳,有事於太廟。辛未,有事於南郊。壬申,梁主蕭巋來朝。甲戌,大射於北苑,十日而罷。壬午,齊州水。辛卯,渝州獲獸似麋,一角同蹄。壬辰,班新歷。二月乙巳,上餞梁主於霸上。丁未,靺鞨貢方物。



 突厥蘇尼部男女萬餘人來降。庚戌,幸隴州。突厥可汗阿史那玷率其屬來降。夏四月己亥,敕總管、刺史父母及子年十五已上,不得將之官。庚子,以吏部尚書虞慶
 則為尚書右僕射,瀛州刺史楊尚希為兵部尚書,毛州刺史劉仁恩為刑部尚書。甲辰,以上柱國叱李長叉為信州總管。丁未,宴突厥、高麗、吐谷渾使者於大興殿。丁巳,以上大將軍賀婁子乾為榆關總管。五月癸酉,契丹主莫賀弗遣使請降,拜大將軍。



 丙子,以柱國馮昱為汾州總管。乙酉,以汴州刺史呂仲泉為延州總管。六月庚子,降囚徒。乙巳,以鴻臚卿乙弗實為冀州總管,上柱國豆盧勣為夏州總管。壬子,開渠,自渭達河,以通運漕。戊午,秦王俊來朝。秋七月丙寅,陳遣兼散騎常侍謝泉、兼通直散騎常侍賀德基來聘。八月甲午,遣十使巡省天
 下。戊戌,衛王爽來朝。是日,以秦王俊納妃,宴百僚,頒賜各有差。壬寅,上柱國、太傅、鄧國公竇熾薨。



 丁未,宴秦王官屬,賜物各有差。壬子,享陳使。乙卯,陳將夏侯苗請降,上以通和,不納。九月甲子,幸襄國公主第。乙丑,幸霸水,觀漕渠,賜督役者帛各有差。



 己巳,上親錄囚徒。庚午,契丹內附。甲戌,駕幸洛陽,關內饑也。癸未,太白晝見。冬十一月壬戌,遣兼散騎常侍薛道衡、通直散騎常侍豆盧實使於陳。癸亥,以榆關總管賀婁子乾為雲州總管。



 五年春正月戊辰,詔行新禮。三月戊午,以尚書左僕射高熲為左領軍大將軍,上柱國宇文忻為右領軍大將
 軍。夏四月甲午,契丹主多彌遣使貢方物。壬寅,上柱國王誼謀反,伏誅。乙巳,詔征山東馬榮伯等六儒。戊申,車駕至自洛陽。五月甲申,詔置義倉。梁主蕭巋殂,其太子琮嗣立。遣上大將軍元契使於突厥阿波可汗。



 秋七月庚申,陳遣兼散騎常侍王話、兼通直散騎常侍阮卓來聘。丁丑,以上柱國宇文慶為涼州總管。壬午,突厥沙缽略上表稱臣。八月丙戌,沙缽略可汗遣子庫合真特勤來朝。甲辰,河南諸州水,遣民部尚書邳國公蘇威賑給之。戊申,有流星數百,四散而下。己酉,幸慄園。九月丁巳,至自慄園。乙丑,改鮑陂曰杜陂,霸水為滋水。陳將湛文
 徹寇和州,儀同三司費寶首獲之。丙子,遣兼散騎常侍李若、兼通直散騎常侍崔君贍使於陳。冬十月壬辰,以上柱國楊素為信州總管,朔州總管吐萬緒為徐州總管。十一月甲子,以上大將軍源雄為朔州總管。丁卯,晉王廣來朝。十二月丁未,降囚徒。戊申,以上柱國達奚長儒為夏州總管。



 六年春正月甲子,黨項羌內附。庚午,班歷於突厥。辛未,以柱國韋洸為安州總管。壬申,遣民部尚書蘇威巡省山東。二月乙酉,山南荊、淅七州水,遣前工部尚書長孫毗賑恤之。丙戌,制刺史上佐每歲暮更入朝,上考課。丁
 亥,發丁男十一萬修築長城,二旬而罷。乙未,以上柱國崔弘度為襄州總管。庚子,大赦天下。三月己未,洛陽男子高德上書,請上為太上皇,傳位皇太子。上曰:「朕承天命,撫育蒼生,日旰孜孜,猶恐不逮。豈學近代帝王,事不師古,傳位於子,自求逸樂者哉!」癸亥,突厥沙缽略遣使貢方物。夏四月己亥,陳遣兼散騎常侍周磻、兼通直散騎常侍江椿來聘。秋七月辛亥,河南諸州水。乙丑,京師雨毛,如馬鬃尾,長者二尺餘,短者六七寸。八月辛卯,關內七州旱,免其賦稅。遣散騎常侍裴豪、兼通直散騎常侍劉顗聘於陳。戊申,上柱國、太師、申國公李穆薨。閏月
 己酉,以河州刺史段文振為蘭州總管。丁卯,皇太子鎮洛陽。辛未,晉王廣、秦王俊並來朝。丙子,上柱國、郕國公梁士彥,上柱國、杞國公宇文忻,柱國、舒國公劉昉,以謀反伏誅。上柱國、許國公宇文善坐事除名。九月辛巳,上素服御射殿,詔百僚射,賜梁士彥三家資物。丙戌,上柱國、宋安郡公元景山卒。庚子,以上柱國李詢為隰州總管。辛丑,詔大象已來死事之家,咸命賑恤。冬十月己酉,以河北道行臺尚書令、並州總管、晉王廣為雍州牧,餘官如故;兵部尚書楊尚希為禮部尚書。癸丑,置山南道行臺尚書省於襄州,以秦王俊為尚書令。丙辰,以芳州
 刺史駱平難為疊州刺史,衡州總管周法尚為黃州總管。甲子,甘露降於華林園。



 七年春正月癸巳,有事於太廟。乙未,制諸州歲貢三人。二月丁巳,祀朝日於東郊。己巳,陳遣兼散騎常侍王亨、兼通直散騎常侍王來聘。壬申,車駕幸醴泉宮。是月,發丁男十萬餘修築長城,二旬而罷。夏四月己酉,幸晉王第。庚戌,於揚州開山陽瀆,以通運漕。突厥沙缽略可汗卒,其子雍虞閭嗣立,是為都藍可汗。



 癸亥,頒青龍符於東方總管、刺史,西方以騶虞,南方以硃雀,北方以玄武。甲戌,遣兼散騎常侍楊同、兼通直散騎常侍崔儦使
 於陳。以民部尚書蘇威為吏部尚書。五月乙亥朔,日有蝕之。己卯,雨石於武安、滏陽間十餘里。秋七月己丑,衛王爽薨,上發喪於門下外省。八月丙午,以懷州刺史源雄為朔州總管。庚申,梁主蕭琮來朝。



 九月乙酉,梁安平王蕭巖掠於其國以奔陳。辛卯,廢梁國,曲赦江陵。以梁主蕭琮為柱國,封莒國公。冬十月庚申,行幸同州,以先帝所居,降囚徒。癸亥,幸蒲州。



 丙寅,宴父老,上極歡,曰:「此間人物,衣服鮮麗,容止閑雅,良由仕宦之鄉,陶染成俗也。」十一月甲午,幸馮翊,親祠故社。父老對詔失旨,上大怒,免其縣官而去。戊戌,至自馮翊。



\end{pinyinscope}