\article{卷五十九列傳第二十四}

\begin{pinyinscope}

 煬三子煬帝三男,蕭皇后生元德太子昭、齊王暕,蕭嬪生趙王杲。



 元德太子昭,煬帝長子也,生而高祖命養宮中。三歲時,於玄武門弄石師子,高祖與文獻後至其所。高祖適患腰痛,舉手憑後,昭因避去,如此者再三。高祖嘆曰:「天生長者,誰復教乎!」由是大奇之。高祖嘗謂曰:「當為爾娶婦。」
 昭應聲而泣。高祖問其故,對曰:「漢王未婚時,恆在至尊所,一朝娶婦,便則出外。



 懼將違離,是以啼耳。」上嘆其有至性,特鐘愛焉。



 年十二,立為河南王。仁壽初,徙為晉王,拜內史令,兼左衛大將軍。後三年,轉雍州牧。煬帝即位,便幸洛陽宮,昭留守京師。大業元年,帝遣使者立為皇太子。



 昭有武力,能引強弩。性謙沖,言色恂恂,未嘗忿怒。有深嫌可責者,但云「大不是」。所膳不許多品,帷席極於儉素。臣吏有老父母者,必親問其安否,歲時皆有惠賜。其仁愛如此。明年,朝於洛陽。後數月,將還京師,願得少留,帝不許,拜請無數。體素肥,因致勞疾。帝令巫者視之,
 云:「房陵王為祟。」未幾而薨。詔內史侍郎虞世基為哀冊文曰:維大業二年七月癸丑朔二十三日,皇太子薨於行宮。粵三年五月庚辰朔六日,將遷座於莊陵,禮也。蜃綍宵載,鶴關曉闢,肅文物以具陳,儼賓從其如昔。皇帝悼離方之雲晦,嗟震宮之虧象,顧守器以長懷,臨登餕而興想。先遠戒日,占謀允從,庭彞徹祖,階所收重,抗銘旌以啟路,動徐輪於振容。揆行度名,累德彰謚,爰詔史冊,式遵典志,俾浚哲之徽猷,播長久乎天地。其辭曰:宸基峻極,帝緒會昌。體元襲聖,儀耀重光。氣秀春陸,神華少陽。居周軼誦,處漢韜莊。有縱生知,誕膺惟睿。性道觿
 日,幾深綺歲。降跡大成,俯情多藝。樹親建國,命懿作籓。威蕤先路,舄奕渠門。庸服有紀,分器惟尊。風高楚殿,雅盛梁園。睿后膺儲,天人協順。本茂條遠,基崇體峻。改王參墟,奄有唐、晉。在貴能謙,居沖益慎。封畿千里,閶闔九重。神州王化,禁旅軍容。瞻言偃草,高視折沖。帷扆清秘,親賢允屬。泛景風瀾,飛華螭玉。揮翰泉湧,敷言藻縟。式是便煩,思謀啟沃。洪惟積德,豐衍繁祉。粵自天孫,光升元子。綠車逮事,翠纓奉祀。肅穆滿容,儀形讓齒。禮樂交暢,愛敬兼資。優游養德,恭己承儀。南山聘隱,東序尊師。有粹神儀,深穆其度。顯顯觀德,溫溫審諭。炯戒齊箴,留
 連王賦。入監出撫,日就月將。沖情玉裕,令問金相。宜綏景福,永作元良。神理冥漠,天道難究。



 仁不必壽,善或愆祐。遽瑤山之頹壞,忽桂宮之毀構。痛結幽明,悲纏宇宙。慟皇情之深憫,摧具僚其如疚。嗚呼哀哉!回環氣朔,荏苒居諸。沾零露於瑤圍,下申霜於玉除。夜漏盡兮空階曙,曉月懸兮帷殿虛。嗚呼哀哉!將寧甫,長違望苑。



 渡渭涘於造舟,遵長平之修阪。望鶴駕而不追,顧龍樓而日遠。嗚呼哀哉!永隔存沒,長分古今。去榮華於人世,即潛遂之幽深。霏夕煙而稍起,慘落景而將沉。



 聽哀挽之淒楚,雜灌木之悲吟。紛徒御而流袂,欷纓弁以沾衿。嗚
 呼哀哉!九地黃泉,千年白日。雖金石之能久,終天壤乎長畢。敢圖芳於篆素,永飛聲而騰實。



 帝深追悼。



 有子三人,韋妃生恭皇帝,大劉良娣生燕王倓,小劉良娣生越王侗。



 燕王倓字仁安。敏慧美姿儀,煬帝於諸孫中特所鐘愛,常置左右。性好讀書,尤重儒素,非造次所及,有若成人。良娣早終,每至忌日,末嘗不流涕嗚咽。帝由是益以奇之。宇文化及弒逆之際,倓覺變,欲入奏,恐露其事,因與梁公蕭鉅、千牛宇文皛等穿芳林門側水竇而入。至玄武門,詭奏曰:「臣卒中惡,命縣俄頃,請得面辭,死無所恨。」
 冀以見帝,為司宮者所遏,竟不得聞。俄而難作,為賊所害,時年十六。



 越王侗字仁謹,美姿儀,性寬厚。大業二年,立為越王。帝每巡幸,侗常留守東都。楊玄感作亂之際,與民部尚書樊子蓋拒之。及玄感平,朝於高陽,拜高陽太守。俄以本官復留守東都。十三年,帝幸江都,復令侗與金紫光祿大夫段達、太府卿元文都、攝民部尚書韋津、右武衛將軍皇甫無逸等總留臺事。宇文化及之弒逆也,文都等議,以侗元德太子之子,屬最為近,於是乃共尊立,大赦,改元曰皇泰。謚帝曰明,廟號世祖。追尊元德太子為孝
 成皇帝,廟號世宗。尊其母劉良娣為皇太后。



 以段達為納言、右翊衛大將軍、攝禮部尚書,王世充亦納言、左翊衛大將軍、攝吏部尚書,元文都內史令、左驍衛大將軍,盧楚亦內史令,皇甫無逸兵部尚書、右武衛大將軍,郭文懿內史侍郎,趙長文黃門侍郎,委以機務,為金書鐵券,藏之宮掖。



 於時洛陽稱段達等為「七貴」。



 未幾,宇文化及立秦王子浩為天子,來次彭城,所經城邑多從逆黨。侗懼,遣使者蓋琮、馬公政招懷李密。密遂遣使請降,侗大悅,禮其使甚厚。即拜密為太尉、尚書令、魏國公,令拒化及。下書曰:我大隋之有天下,於茲三十八載。高祖文
 皇帝聖略神功,載造區夏。世祖明皇帝則天法地,混一華戎。東暨蟠木,西通細柳,前逾丹徼,後越幽都。日月之所臨,風雨之所至,圓首方足,稟氣食芼,莫不盡入提封,皆為臣妾。加以寶貺畢集,靈瑞咸臻,作樂制禮,移風易俗。智周寰海,萬物咸受其賜,道濟天下,百姓用而不知。世祖往因歷試,統臨南服,自居皇極,順茲望幸。所以往歲省方,展禮肆覲,停鑾駐蹕,按駕清道,八屯如昔,七萃不移。豈意釁起非常,逮於軒陛,災生不意,延及冕旒。奉諱之日,五情崩隕,攀號荼毒,不能自勝。



 且聞之,自古代有屯剝,賊臣逆子,無世無之。至如宇文化及,世傳庸品。
 其父述,往屬時來,早沾厚遇,賜以婚媾,置之公輔,位尊九命,祿重萬鐘,禮極人臣,榮冠世表。徒承海岳之恩,未有涓塵之益。化及以此下材,夙蒙顧盼,出入外內,奉望階墀。昔陪籓國,統領禁衛,及從升皇祚,陪列九卿。但本性兇狠,恣其貪穢,或交結惡黨,或侵掠貨財,事重刑篇,狀盈獄簡。在上不遺簪履,恩加草芥,應至死辜,每蒙恕免。三經除解,尋復本職,再徙邊裔,仍即追還。生成之恩,昊天罔極,獎擢之義,人事罕聞。化及梟獍為心,禽獸不若,縱毒興禍,傾覆行宮。



 諸王兄弟,一時殘酷,痛暴行路,世不忍言。有窮之在夏時,犬戎之於周代,釁辱之極,亦
 未是過。朕所以刻骨崩心,飲膽嘗血,瞻天視地,無處容身。



 今王公卿士,庶僚百闢,咸以大寶鴻名,不可顛墜,元兇巨猾,須早夷殄,翼戴朕躬,嗣守寶位。顧惟寡薄,志不逮此。今者出黼扆而杖旄鉞,釋衰麻而擐甲胄,銜冤誓眾,忍淚治兵,指日遄征,以平大盜。且化及偽立秦王之子,幽遏比於囚拘,其身自稱霸相,專擅擬於九五。履踐禁禦,據有宮闈,昂首揚眉,初無慚色。衣冠朝望,外懼兇威,志士誠臣,內皆憤怨。以我義師,順彼天道,梟夷醜族,匪夕伊朝。



 太尉、尚書令、魏公丹誠內發,宏略外舉,率勤王之師,討違天之逆。果毅爭先,熊羆競逐,金鼓振讋,若
 火焚毛,鋒刃縱橫,如湯沃雪。魏公志在匡濟,投袂前驅,朕親御六軍,星言繼進。以此眾戰,以斯順舉,擘山可以動,射石可以入。



 況擁此人徒,皆有離德,京都侍衛,西憶鄉家,江左淳民,南思邦邑,比來表書駱驛,人信相尋。若王師一臨,舊章暫睹,自應解甲倒戈,冰銷葉散。且聞化及自恣,天奪其心,殺戮不辜,挫辱人士,莫不道路仄目,號天踞地。朕今復仇雪恥,梟轘者一人,拯溺救焚,所哀者士庶。唯天鑒孔殷,祐我宗社,億兆感義,俱會朕心。



 梟戮元兇,策勛飲至,四海交泰,稱朕意焉。兵術軍機,並受魏公節度。



 密見使者,大悅,北面拜伏,臣禮甚恭。密遂東
 拒化及。「七貴」頗不協,陰有相圖之計。未幾,元文都、盧楚、郭文懿、趙長文等為世充所殺,皇甫無逸遁歸長安。世充詣侗所陳謝,辭情哀苦。侗以為至誠,命之上殿,被發為盟,誓無貳志。



 自是侗無所關預。侗心不能平,遂與記室陸士季謀圖世充,事不果而止。及世充破李密,眾望益歸之,遂自為鄭王,總百揆,加九錫,備法物,侗不能禁也。段達、雲定興等十人入見於侗曰:「天命不常,鄭王功德甚盛,願陛下揖讓告禪,遵唐、虞之跡。」侗聞之怒曰:「天下者,高祖之天下,東都者,世祖之東都。若隋德未衰,此言不可發;必天命有改,亦何論於禪讓!公等或先朝舊
 臣,績宣上代,或勤王立節,身服軒冕,忽有斯言,朕復當何所望!」神色懍然,侍衛者莫不流汗。既而退朝,對良娣而泣。世充更使人謂侗曰:「今海內未定,須得長君。待四方乂安,復子明闢,必若前盟,義不違負。」侗不得已,遜位於世充,遂被幽於含涼殿。世充僭偽號,封為潞國公,邑五千戶。



 月餘,宇文儒童、裴仁基等謀誅世充,復尊立侗,事洩,並見害。世充兄世惲因勸世充害侗,以絕民望。世充遣其侄行本齎鴆詣侗所曰:「願皇帝飲此酒。」侗知不免,請與母相見,不許。遂布席焚香禮佛,咒曰:「從今以去,願不生帝王尊貴之家。」於是仰藥。不能時絕,更以帛縊
 之。世充偽謚為恭皇帝。



 齊王暕,字世朏,小字阿孩。美容儀,疏眉目,少為高祖所愛。開皇中,立為豫章王,邑千戶。及長,頗涉經史,尤工騎射。初為內史令。仁壽中,拜揚州總管沿淮以南諸軍事。煬帝即位,進封齊王,增邑四千戶。大業二年,帝初入東都,盛陳鹵簿,暕為軍導。尋轉豫州牧。俄而元德太子薨,朝野注望,咸以暕當嗣。帝又敕吏部尚書牛弘妙選官屬,公卿由是多進子弟。明年,轉雍州牧,尋徙河南尹、開府儀同三司。元德太子左右二萬餘人悉隸於暕,寵遇益隆,自樂平公主及諸戚屬競來致禮,百官稱謁,填咽
 道路。



 暕頗驕恣,暱近小人,所行多不法,遣喬令則、劉虔安、裴該、皇甫諶、庫狄仲錡、陳智偉等求聲色狗馬。令則等因此放縱,訪人家有女者,輒矯暕命呼之,載入暕宅,因緣藏隱,恣行淫穢,而後遣之。仲錡、智偉二人詣隴西,撾炙諸胡,責其名馬,得數匹以進於。暕令還主,仲錡等詐言王賜,將歸於家,暕不之知也。又樂平公主嘗奏帝,言柳氏女美者,帝未有所答。久之,主復以柳氏進於暕,暕習納之。其後帝問主柳氏女所在,主曰:「在齊王所。」帝不悅。暕於東都營第,大門無故而崩,聽事栿中折,識者以為不祥。其後從帝幸榆林,暕督後軍步騎五萬,恆與
 帝相去數十里而舍。會帝於汾陽宮大獵,詔暕以千騎入圍。暕大獲麋鹿以獻,而帝未有得也,乃怒從官,皆言為暕左右所遏,獸不得前。帝於是發怒,求暕罪失。



 時制縣令無故不得出境,有伊闕令皇甫詡幸於暕,違禁將之汾陽宮。又京兆人達奚通有妾王氏善歌,貴游宴聚,多或要致,於是展轉亦出入王家。御史韋德裕希旨劾暕,帝令甲士千餘大索暕第,因窮治其事。暕妃韋氏者,民部尚書沖之女也,早卒,暕遂與妃姊元氏婦通,遂產一女,外人皆不得知。陰引喬令則於第內酣宴,令則稱慶,脫暕帽以為歡樂。召相工令遍視後庭,相工指妃姊
 曰:「此產子者當為皇后。王貴不可言。」時國無儲副,暕自謂次當得立。又以元德太子有三子,內常不安,陰挾左道,為厭勝之事。至是,事皆發,帝大怒,斬令則等數人,妃姊賜死,暕府僚皆斥之邊遠。時趙王杲猶在孩孺,帝謂侍臣曰:「朕唯有暕一子,不然者,當肆諸市朝,以明國憲也。」暕自是恩寵日衰,雖為京尹,不復關預時政。帝恆令武賁郎將一人監其府事,暕有微失,武賁輒奏之。帝亦常慮暕生變,所給左右,皆以老弱,備員而已。暕每懷危懼,心不自安。又帝在江都宮,元會,暕具法服將朝,無故有血從裳中而下。又坐齋中,見群鼠數十,至前而死,視
 皆無頭。暕意甚惡之。



 俄而化及作亂,兵將犯蹕,帝聞,顧謂蕭后曰:「得非阿孩邪?」其見疏忌如此。



 化及復令人捕暕,暕時尚臥未起,賊既進,暕驚曰:「是何人?」莫有報者,暕猶謂帝令捕之,因曰:「詔使且緩。兒不負國家。」賊於是曳至街而斬之,及其二子亦遇害。暕竟不知殺者為誰。時年三十四。



 有遺腹子政道,與蕭後同入突厥,處羅可汗號為隋王,中國人沒入北蕃者,悉配之以為部落,以定襄城處之。及突厥滅,歸於大唐,授員外散騎侍郎。



 趙王杲,小字季子。年七歲,以大業九年封趙王。尋授光祿大夫,拜河南尹。



 從幸淮南,詔行江都太守事。杲聰令,
 美容儀,帝有所制詞賦,杲多能誦之。性至孝,常見帝風動不進膳,杲亦終日不食。又蕭後當灸,杲先請試炷,後不許之,杲泣請曰:「後所服藥,皆蒙嘗之。今灸,願聽嘗炷。」悲咽不已。後竟為其停灸,由是尤愛之。後遇化及反,杲在帝側,號慟不已。裴虔通使賊斬之於帝前,血湔御服。時年十二。



 史臣曰:元德太子雅性謹重,有君人之量,降年不永,哀哉!齊王敏慧可稱,志不及遠,頗懷驕僭,故煬帝疏而忌之。心無父子之親,貌展君臣之敬,身非積善,國有餘殃。至令趙及燕、越皆不得其死,悲夫!



\end{pinyinscope}