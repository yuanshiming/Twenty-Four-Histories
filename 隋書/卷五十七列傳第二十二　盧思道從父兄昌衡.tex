\article{卷五十七列傳第二十二 盧思道從父兄昌衡}

\begin{pinyinscope}

 盧思道,字子行,範陽人也。祖陽烏,魏秘書監。父道亮,隱居不仕。思道聰爽俊辯,通侻不羈。年十六,遇中山劉松,松為人作碑銘,以示思道。思道讀之,多所不解,於是感激,閉戶讀書,師事河間邢子才。後思道復為文,以示劉松,松又不能甚解。思道乃喟然嘆曰:「學之有益,豈徒然哉!」因就魏收借異書,數年之間,才學兼著。然不持操行,
 好輕侮人。齊天保中,《魏史》未出,思道先已誦之,由是大被笞辱。前後屢犯,因而不調。其後左僕射楊遵彥薦之於朝,解褐司空行參軍,長兼員外散騎侍郎,直中書省。文宣帝崩,當朝文士各作挽歌十首,擇其善者而用之。魏收、陽休之、祖孝徵等不過得一二首,唯思道獨得八首。故時人稱為「八米盧郎」。後漏洩省中語,出為丞相西閤祭酒,歷太子舍人、司徒錄事參軍。



 每居官,多被譴辱。後以擅用庫錢,免歸於家。嘗於薊北悵然感慨,為五言詩為見意,人以為工。數年,復為京畿主簿,歷主客郎、給事黃門侍郎,待詔文林館。周武帝平齊,授儀同三司,追赴
 長安,與同輩陽休之等數人作《聽蟬鳴篇》,思道所為,詞意清切,為時人所重。新野庾信遍覽諸同作者,而深嘆美之。未幾,以母疾還鄉,遇同郡祖英伯及從兄昌期、宋護等舉兵作亂,思道預焉。周遣柱國宇文神舉討平之,罪當法,已在死中。神舉素聞其名,引出之,令作露布。思道援筆立成,文無加點,神舉嘉而宥之。後除掌教上士。高祖為丞相,遷武陽太守,非其好也。



 為《孤鴻賦》以寄其情曰:余志學之歲,自鄉里游京師,便見識知音,歷受群公之眷。年登弱冠,甫就朝列,談者過誤,遂竊虛名。通人楊令君、邢特進已下,皆分庭致禮,倒屣相接,翦拂吹噓,
 長其光價。而才本駑拙,性實疏懶,勢利貨殖,淡然不營。雖籠絆朝市且三十載,而獨往之心未始去懷抱也。攝生舛和,有少氣疾。分符坐嘯,作守東原。



 洪河之湄,沃野彌望,囂務既屏,魚鳥為鄰。有離群之鴻,為羅者所獲,野人馴養,貢之於餘。置諸池庭,朝夕賞玩,既用銷憂,兼以輕疾。《大易》稱「鴻漸於陸」,羽儀盛也。《揚子》曰「鴻飛冥冥」,騫翥高也。《淮南》云「東歸碣石」,違溽暑也。平子賦曰「南寓衡陽」,避祁寒也。若其雅步清音,遠心高致,鵷鸞以降,罕見其儔,而鎩翮墻陰,偶影獨立,唼喋粃粺,雞鶩為伍,不亦傷乎!餘五十之年,忽焉已至,永言身事,慨然多緒,乃為
 之賦,聊以自慰云。其詞曰:惟此孤鴻,擅奇羽蟲,實稟清高之氣,遠生遼碣之東。氄毛將落,和鳴順風,壯冰云厚,矯翅排空。出島嶼之綿邈,犯霜露之溟濛,驚絓魚之密網,畏落雁之虛弓。若其斗柄東指,女夷司月,乃遙集於寒門,遂輕舉於玄闕。至如天高氣肅,搖落在時,既嘯儔於淮浦,亦弄吭於江湄。摩赤霄以凌厲,乘丹氣之威夷,逆商飆之裊裊,玩陽景之遲遲。彭蠡方春,洞庭初綠,理翮整翰,群浮侶浴。振雪羽而臨風,掩霜毛而候旭,饜江湖之菁藻,飫原野之菽粟。行離離而高逝,響噰噰而相續,潔齊國之冰紈,皓密山之華玉。若乃晨沐清露,安趾
 徐步;夕息芳洲,延頸乘流;違寒競逐,浮沅水宿;避暑言歸,絕漠雲飛。望玄鵠而為侶,比硃鷺而相依,倦天衢之冥漠,降河渚之芳菲。忽值羅人設網,虞者懸機,永辭寥廓,蹈跡重圍。始則窘束籠樊,憂憚刀俎,靡軀絕命,恨失其所。終乃馴狎園庭,棲托池禦,稻粱為惠,恣其容與。於是翕羽宛頸,屏氣銷聲,滅煙霞之高想,悶江海之幽情。何時驤首奮翼,上凌太清,騫翥鼓舞,遠薄層城。惡禽視而不貴,小鳥顧而相輕,安控地而無恥,豈沖天之復榮!若夫圖南之羽,偉而去羨,棲睫之蟲,微而不賤,各遂性於天壤,弗企懷以交戰。不聽咸池之樂,不饗太牢之薦,
 匹晨雞而共飲,偶野鳧以同膳。



 匪揚聲以顯聞,寧校體而求見,聊寓形乎沼沚,且夷心於溏澱。齊榮辱以晏如,承君子之餘眄。



 開皇初,以母老,表請解職,優詔許之。思道自恃才地,多所陵轢,由是官途淪滯。既而又著《勞生論》,指切當時,其詞曰:《莊子》曰:「大塊勞我以生。」誠哉斯言也!餘年五十,羸老云至,追惟疇昔,勤矣厥生。乃著茲論,因言時云爾。



 罷郡屏居,有客造餘者,少選之頃,盱衡而言曰:「生者天地之大德,人者有生之最靈,所以作配兩儀,稱貴群品,妍蚩愚智之辯,天懸壤隔,行己立身之異,入海登山。今吾子生於右地,九葉卿族,天授俊才,萬夫
 所仰,學綜流略,慕孔門之游、夏,辭窮麗則,擬漢日之卿、雲。行藏有節,進退以禮,不諂不驕,無慍無懌,偃仰貴賤之間,從容語默之際,何其裕也!下走所欣羨焉。」



 餘莞爾而笑曰:「未之思乎?何所言之過也!子其清耳,請為左右陳之。夫人之生也,皆未若無生。在餘之生,勞亦勤止,紈綺之年,伏膺教義,規行矩步,從善而登。巾冠之後,濯纓受署,韁鎖仁義,籠絆朝市。失翹陸之本性,喪江湖之遠情,淪此風波,溺於倒躓,憂勞總至,事非一緒。何則?地胄高華,既致嫌於管庫,才識美茂,亦受嫉於愚庸。篤學強記,聾瞽於焉側目,清言河瀉,木訥所以疚心。



 豈徒蟲惜
 春漿,鴟吝腐鼠,相江都而永嘆,傅長沙而不歸,固亦魯值臧倉,楚逢靳尚,趙壹為之哀歌,張升於是慟哭。有齊之季,不遇休明,申脰就鞅,屏跡無地。



 段珪、張讓,金貝是視,賈謐、郭淮,腥臊可饜。淫刑以逞,禍近池魚,耳聽惡來之讒,足踐龍逢之血。周氏末葉,仍值僻王,斂笏升階,汗流浹背,莒客之踵躋焦原,匹茲非險,齊人之手執馬尾,方此未危。若乃羊腸、句注之道,據鞍振策,武落、雞田之外,櫛風沐雨,三旬九食,不敢稱弊,此之為役,蓋其小小者耳。今泰運肇開,四門以穆,冕旒司契於上,夔、龍佐命於下,岐伯、善卷,恥徇幽憂,卞隨、務光,悔從木石。餘年在
 秋方,已迫知命,情禮宜退,不獲晏安。一葉從風,無損鄧林之攢植,雙鳧退飛,不虧渤澥之游泳。耕田鑿井,晚息晨興,候南山之朝雲,攬北堂之明月。氾勝九穀之書,觀其節制,崔實四人之令,奉以周旋。晨荷蓑笠,白屋黃冠之伍,夕談穀稼,沾體塗足之倫。濁酒盈樽,高歌滿席,恍兮惚兮,天地一指。此野人之樂也,子或以是羨餘乎?」



 客曰:「吾子之事,既聞之矣。他人有心,又請論其梗概。」余答曰:「雲飛泥沉,卑高異等,圓行方止,動息殊致。是以摩霄運海,輕罻羅於藪澤,五衢四照,忽斤斧於山林。余晚值昌辰,遂其弱尚,觀人事之隕獲,睹時路之邅危。玄冬修
 夜,靜言長想,可以累嘆悼心,流涕酸鼻。人之百年,脆促已甚,奔駒流電,不可為辭。



 顧慕周章,數紀之內,窮通榮辱,事無足道。而有識者鮮,無識者多,褊隘凡近,輕險躁薄。居家則人面獸心,不孝不義,出門則諂諛讒佞,無愧無恥。退身知足,忘伯陽之炯戒,陳力就列,棄周任之格言。悠悠遠古,斯患已積,迄於近代,此蠹尤深。範卿捴讓之風,搢紳不嗣,《夏書》昏墊之罪,執政所安。朝露未晞,小車盈董、石之巷,夕陽且落,皁蓋填閻、竇之里。皆如脂如韋,俯僂匍匐,啖惡求媚,舐痔自親。美言諂笑,助其愉樂,詐泣佞哀,恤其喪紀。近通旨酒,遠貢文蛇,艷姬美女,委
 如脫屣,金銑玉華,棄同遺跡。及鄧通失路,一簪之賄無餘,梁冀就誅,五侯之貴將起。向之求官買職,晚謁晨趨,刺促望塵之舊游,伊優上堂之夜客,始則亡魂褫魄,若牛兄之遇獸,心戰色沮,似葉公之見龍;俄而抵掌揚眉,高視闊步,結侶棄廉公之第,攜手哭聖卿之門。華轂生塵,來如激矢,雀羅暫設,去等絕弦。



 飴蜜非甘,山川未阻,千變萬化,鬼出神入。為此者皆衣冠士族,或有藝能,不恥不仁,不畏不義,靡愧友朋,莫慚妻子。外呈厚貌,內蘊百心,繇是則紆青佩紫,牧州典郡,冠幘劫人,厚自封殖。妍歌妙舞,列鼎撞鐘,耳倦絲桐,口飫珍旨。雖素論以為
 非,而時宰之不責,末俗蚩蚩,如此之敝。餘則違時薄宦,屏息窮居,甚恥驅馳,深畏乾沒。心若死灰,不營勢利,家無儋石,不費囊錢。偶影聯官,將數十載,駑拙致笑,輕生所以告勞也。真人御宇,斫雕為樸,人知榮辱,時反邕熙。



 風力上宰,內敷文教,方、邵重臣,外揚武節。被之大道,洽以淳風,舉必以才,爵無濫授。稟斯首鼠,不預衣簪,阿黨比周,掃地俱盡,輕薄之儔,滅影竄跡。礫石變成瑜瑾,莨莠化為芝蘭。曩之扇俗攪時,駭耳穢目,今悉不聞不見,莫余敢侮。



 《易》曰:『聖人作而萬物睹』,斯之謂乎!」



 歲餘,被徵,奉詔郊勞陳使。頃之,遭母憂,未幾,起為散騎侍郎,奏內
 史侍郎事。於時議置六卿,將除大理。思道上奏曰:「省有駕部,寺留太僕,省有刑部,寺除大理,斯則重畜產而賤刑名,誠為未可。」又陳殿庭非杖罰之所,朝臣犯笞罪,請以贖論,上悉嘉納之。是歲,卒於京師,時年五十二。上甚惜之,遣使吊祭焉。



 有集三十卷,行於時。子赤松,大業中,官至河東長史。



 昌衡字子均。父道虔,魏尚書僕射。昌衡小字龍子,風神淡雅,容止可法,博涉經史,工草行書。從弟思道,小字釋奴,宗中俱稱英妙。故幽州為之語曰:「盧家千里,釋奴、龍子。」年十七,魏濟陰王元暉業召補太尉參軍事,兼外兵
 參軍。



 齊氏受禪,歷平恩令、太子舍人。尋為僕射祖孝徵所薦,遷尚書金部郎。孝徵每曰:「吾用盧子均為尚書郎,自謂無愧幽州矣。」其後兼散騎侍郎,迎勞周使。武帝平齊,授司玉中士,與大宗伯斛斯徵修禮令。開皇初,拜尚書祠部侍郎。高祖嘗大集群下,令自陳功績,人皆競進,昌衡獨無所言。左僕射高熲目而異之。陳使賀徹、周濆相繼來聘,朝廷每令昌衡接對之。未幾,出為徐州總管長史,甚有能名。吏部尚書蘇威考之曰:「德為人表,行為士則。」論者以為美談。嘗行至浚儀,所乘馬為他牛所觸,因致死。牛主陳謝,求還價直,昌衡謂之曰:「六畜相觸,自
 關常理,此豈人情也,君何謝?」拒而不受。性寬厚不校,皆此類也。轉壽州總管長史。總管宇文述甚敬之,委以州務。歲餘,遷金州刺史。仁壽中,奉詔持節為河南道巡省大使,及還,以奉使稱旨,授儀同三司,賜物三百段。昌衡自以年在懸車,表乞骸骨,優詔不許。大業初,徵為太子左庶子,行詣洛陽,道卒,時年七十二。子寶素、寶胤。



 李孝貞李孝貞,字元操,趙郡柏人人也。父希禮,齊信州刺史,世為著姓。孝貞少好學,能屬文。在齊釋褐司徒府參軍事。簡靜不妄通賓客,與從兄儀曹郎中騷、太子舍人季節、
 博陵崔子武、範陽盧詢祖為斷金之契。後以射策甲科拜給事中。於時黃門侍郎高乾和親要用事,求婚於孝貞。孝貞拒之,由是有隙,陰譖之,出為太尉府外兵參軍。後歷中書舍人、博陵太守、司州別駕,復兼散騎常侍、聘周使副,還除給事黃門侍郎。周武帝平齊,授儀同三司、少典祀下大夫。宣帝即位,轉吏部下大夫。高祖為丞相,尉迥作亂相州,孝貞從韋孝寬擊之,以功授上儀同三司。開皇初,拜馮翊太守,為犯廟諱,於是稱字。後數歲,遷蒙州刺史,吏民安之。自此不復留意於文筆,人問其故,慨然嘆曰:「五十之年,倏焉而過,鬢垂素發,筋力已衰,宦
 意文情,一時盡矣,悲夫!」然每暇日,輒引賓客弦歌對酒,終日為歡。徵拜內史侍郎,與內史李德林參典文翰。然孝貞無干劇之用,頗稱不理,上譴怒之,敕御史劾其事,由是出為金州刺史。卒官。所著文集二十卷,行於世。有子允玉。



 孝貞弟孝威,亦有雅望,大業中,官至大理少卿。



 薛道衡從弟孺薛道衡,字玄卿,河東汾陰人也。祖聰,魏濟州刺史。父孝通,常山太守。道衡六歲而孤,專精好學。年十三,講《左氏傳》,見子產相鄭之功,作《國僑贊》,頗有詞致,見者奇之。其後才名益著,齊司州牧、彭城王浟引為兵曹從事。尚書
 左僕射弘農楊遵彥,一代偉人,見而嗟賞。授奉朝請。吏部尚書隴西辛術與語,嘆曰:「鄭公業不亡矣。」河東裴讞目之曰:「自鼎遷河朔,吾謂關西孔子罕值其人,今復遇薛君矣。」武成作相,召為記室,及即位,累遷太尉府主簿。歲餘,兼散騎常侍,接對周、陳二使。武平初,詔與諸儒修定《五禮》,除尚書左外兵郎。陳使傅縡聘齊,以道衡兼主客郎接對之。縡贈詩五十韻,道衡和之,南北稱美。魏收曰:「傅縡所謂以蚓投魚耳。」待詔文林館,與範陽盧思道、安平李德林齊名友善。復以本官直中書省,尋拜中書侍郎,仍參太子侍讀。後主之時,漸見親用,於時頗有附
 會之譏。後與侍中斛律孝卿參預政事,道衡具陳備周之策,孝卿不能用。及齊亡,周武引為御史二命士。後歸鄉里,自州主簿入為司祿上士。



 高祖作相,從元帥梁睿擊王謙,攝陵州刺史。大定中,授儀同,攝邛州刺史。



 高祖受禪,坐事除名。河間王弘北征突厥,召典軍書,還除內史舍人。其年,兼散騎常侍,聘陳主使。道衡因奏曰:「江東蕞爾一隅,僭擅遂久,實由永嘉已後,華夏分崩。劉、石、符、姚、慕容、赫連之輩,妄竊名號,尋亦滅亡。魏氏自北徂南,未遑遠略。周、齊兩立,務在兼並,所以江表逋誅,積有年祀。陛下聖德天挺,光膺寶祚,比隆三代,平一九州,豈容
 使區區之陳,久在天網之外?臣今奉使,請責以稱籓。」高祖曰:「朕且含養,置之度外,勿以言辭相折,識朕意焉。」江東雅好篇什,陳主尤愛雕蟲,道衡每有所作,南人無不吟誦焉。及八年伐陳,授淮南道行臺尚書吏部郎,兼掌文翰。王師臨江,高熲夜坐幕下,謂之曰:「今段之舉,克定江東已不?君試言之。」道衡答曰:「凡論大事成敗,先須以至理斷之。《禹貢》所載九州,本是王者封域。後漢之季,群雄競起,孫權兄弟遂有吳、楚之地。晉武受命,尋即吞並,永嘉南遷,重此分割。自爾已來,戰爭不息,否終斯泰,天道之恆。郭璞有云:『江東偏王三百年,還與中國合。』今數
 將滿矣。以運數而言,其必克一也。有德者昌,無德者亡,自古興滅,皆由此道。主上躬履恭儉,憂勞庶政,叔寶峻宇雕墻,酣酒荒色。上下離心,人神同憤,其必克二也。為國之體,在於任寄,彼之公卿,備員而已。拔小人施文慶委以政事,尚書令江總唯事詩酒,本非經略之才,蕭摩訶、任蠻奴是其大將,一夫之用耳。其必克三也。我有道而大,彼無德而小,量其甲士,不過十萬。西自巫峽,東至滄海,分之則勢懸而力弱,聚之則守此而失彼。其必克四也。席卷之勢,其在不疑。」熲忻然曰:「君言成敗,事理分明,吾今豁然矣。本以才學相期,不意籌略乃爾。」還除吏
 部侍郎。後坐抽擢人物,有言其黨蘇威,任人有意故者,除名,配防嶺表。晉王廣時在揚州,陰令人諷道衡從揚州路,將奏留之。道衡不樂王府,用漢王諒之計,遂出江陵道而去。尋有詔徵還,直內史省。晉王由是銜之,然愛其才,猶頗見禮。後數歲,授內史侍郎,加上儀同三司。



 道衡每至構文,必隱坐空齋,蹋壁而臥,聞戶外有人便怒,其沉思如此。高祖每曰:「薛道衡作文書稱我意。」然誡之以迂誕。後高祖善其稱職,謂楊素、牛弘曰:「道衡老矣,驅使勤勞,宜使其硃門陳戟。」於是進位上開府,賜物百段。道衡辭以無功,高祖曰:「爾久勞階陛,國家大事,皆爾宣
 行,豈非爾功也?」道衡久當樞要,才名益顯,太子諸王爭相與交,高熲、楊素雅相推重,聲名籍甚,無競一時。



 仁壽中,楊素專掌朝政,道衡既與素善,上不欲道衡久知機密,因出檢校襄州總管。道衡久蒙驅策,一旦違離,不勝悲戀,言之哽咽。高祖愴然改容曰:「爾光陰晚暮,侍奉誠勞。朕欲令爾將攝,兼撫萌俗。今爾之去,朕如斷一臂。」於是賚物三百段,九環金帶,並時服一襲,馬十匹,慰勉遣之。在任清簡,吏民懷其惠。



 煬帝嗣位,轉番州刺史。歲餘,上表求致仕。帝謂內史侍郎虞世基曰:「道衡將至,當以秘書監待之。」道衡既至,上《高祖文皇帝頌》,其詞曰:太始
 太素,荒茫造化之初;天皇地皇,杳冥書契之外。其道絕,其跡遠,言談所不詣,耳目所不追。至於入穴登巢,鶉居鷇飲,不殊於羽族,取類於毛群,亦何貴於人靈,何用於心識?羲、軒已降,爰暨唐、虞,則乾象而施法度,觀人文而化天下,然後帝王之位可重,聖哲之道為尊。夏后、殷、周之國,禹、湯、文、武之主,功濟生民,聲流《雅頌》,然陵替於三五,慚德於干戈。秦居閏位,任刑名為政本,漢執靈圖,雜霸道而為業。當塗興而三方峙,典午末而四海亂。九州封域,窟穴鯨鯢之群;五都遺黎,蹴踏戎馬之足。雖玄行定嵩、洛,木運據崤、函,未正滄海之流,詎息昆山之燎!協
 千齡之旦暮,當萬葉之一朝者,其在大隋乎?



 粵若高祖文皇帝,誕聖降靈,則赤光照室,韜神晦跡,則紫氣騰天。龍顏日角之奇,玉理珠衡之異,著在圖籙,彰乎儀表。而帝系靈長,神基崇峻,類邠、岐之累德,異豐、沛之勃起。俯膺歷試,納揆賓門,位長六卿,望高百闢,猶重華之為太尉,若文命之任司空。蒼歷將盡,率土糜沸,玉弩驚天,金芒照野。奸雄挺禍,據河朔而連海岱;猾長縱惡,杜白馬而塞成皋。庸、蜀逆命,憑銅梁之險;鄖、黃背誕,引金陵之寇。三川已震,九鼎將飛。高祖龍躍鳳翔,濡足授手,應赤伏之符,受玄狐之籙,命百下百勝之將,動九天九地之
 師,平共工而殄蚩尤,翦犬契窳而戮鑿齒。不煩二十八將,無假五十二徵,曾未逾時,妖逆咸殄,廓氛霧於區宇,出黎元於塗炭。天柱傾而還正,地維絕而更紐。殊方稽顙,識牛馬之內向;樂師伏地,懼鐘石之變聲。萬姓所以樂推,三靈於是改卜。壇場已備,猶弘五讓之心;億兆難違,方從四海之請。光臨寶祚,展禮郊丘,舞六代而降天神,陳四圭而饗上帝,乾坤交泰,品物咸亨。酌前王之令典,改易徽號;因庶萌之子來,移創都邑。天文上當硃鳥,地理下據黑龍,正位辨方,揆影於日月,內宮外座,取法於辰象。懸政教於魏闕,朝群后於明堂,除舊布新,移風易
 俗。天街之表,地脈之外,獯獫孔熾,其來自久,橫行十萬,樊噲於是失辭,提步五千,李陵所以陷沒。周、齊兩盛,競結旄頭,娉狄后於漠北,未足息其侵擾,傾珍藏於山東,不能止其貪暴。炎靈啟祚,聖皇馭宇,運天策於帷扆,播神威於沙朔,柳室、氈裘之長,皆為臣隸,瀚海、蹛林之地,盡充池苑。三吳、百越,九江五湖,地分南北,天隔內外,談黃旗紫蓋之氣,恃龍蟠獸據之險,恆有僭偽之君,妄竊帝王之號。時經五代,年移三百,爰降皇情,永懷大道,愍彼黎獻,獨為匪人。今上利建在唐,則哲居代,地憑宸極,天縱神武,受脤出車,一舉平定。於是八荒無外,九服大
 同,四海為家,萬里為宅。



 乃休牛散馬,偃武修文。



 自華夏亂離,綿積年代,人造戰爭之具,家習澆偽之風,聖人之遺訓莫存,先王之舊典咸墜。爰命秩宗,刊定《五禮》,申敕太子,改正六樂。玉帛樽俎之儀,節文乃備;金石匏革之奏,雅俗始分。而留心政術,垂神聽覽,早朝晏罷,廢寢忘食,憂百姓之未安,懼一物之失所。行先王之道,夜思待旦;革百王之弊,朝不及夕。見一善事,喜彰於容旨;聞一愆犯,嘆深於在予。薄賦輕徭,務農重谷,倉廩有紅腐之積,黎萌無阻饑之慮。天性弘慈,聖心惻隱,恩加禽獸,胎卵於是獲全,仁沾草木,牛羊所以勿踐。至於憲章重典,
 刑名大闢,申法而屈情,決斷於俄頃,故能彞倫攸敘,上下齊肅。左右絕諂諛之路,縉紳無勢力之門。小心翼翼,敬事於天地;終日乾乾,誡慎於亢極。陶黎萌於德化,致風俗於太康,公卿庶尹,遐邇岳牧,僉以天平地成,千載之嘉會,登封降禪,百王之盛典,宜其金泥玉檢,展禮介丘,飛聲騰實,常為稱首。天子為而不恃,成而不居,沖旨凝邈,固辭弗許。而雖休勿休,上德不德,更乃潔誠岱嶽,遜謝愆咎。方知六十四卦,謙捴之道為尊,七十二君,告成之義為小,巍巍蕩蕩,無得以稱焉。而深誠至德,感達於穹壤,和氣薰風,充溢於宇宙。二儀降福,百靈薦祉,日
 月星象,風雲草樹之祥,山川玉石,鱗介羽毛之瑞,歲見月彰,不可勝紀。至於振古所未有,圖籍所不載,目所不見,耳所未聞。古語稱聖人作,萬物睹,神靈滋,百寶用,此其效矣。



 既而游心姑射,脫屣之志已深;鑄鼎荊山,升天之駕遂遠。凡在黎獻,具惟帝臣,慕深考妣,哀纏弓劍,塗山幽峻,無復玉帛之禮,長陵寂寞,空見衣冠之游。



 若乃降精熛怒,飛名帝籙,開運握圖,創業垂統,聖德也;撥亂反正,濟國寧人,六合八紘,同文共軌,神功也;玄酒陶匏,雲和孤竹,禋祀上帝,尊極配天,大孝也;偃伯戢戈,正禮裁樂,納民壽域,驅俗福林,至政也。張四維而臨萬宇,侔
 三皇而並五帝,豈直錙銖周、漢,麼麼魏、晉而已。雖五行之舞,每陳於清廟,九德之歌,無絕於樂府,而玄功暢洽,不局於形器,懿業遠大,豈盡於揄揚。



 臣輕生多幸,命偶興運,趨事紫宸,驅馳丹陛,一辭天闕,奄隔鼎湖,空有攀龍之心,徒懷蓐蟻之意。庶憑毫翰,敢希贊述!昔堙海之禽不增於大地,泣河之士非益於洪流,盡其心之所存,望其力之所及,輒緣斯義,不覺斐然。乃作頌曰:悠哉邃古,邈矣季世,四海九州,萬王千帝。三代之後,其道逾替,爰逮金行,不勝其弊。戎狄猾夏,群兇縱慝,竊號淫名,十有餘國。怙威逞暴,悖禮亂德,五嶽塵飛,三象霧塞。玄精
 啟歷,發跡幽方,並吞寇偽,獨擅雄強。載祀二百,比祚前王,江湖尚阻,區域未康。句吳閩越,河朔渭涘,九縣瓜分,三方鼎跱。狙詐不息,干戈競起,東夏雖平,亂離瘼矣。五運葉期,千年肇旦,赫矣高祖,人靈攸贊。



 聖德迥生,神謀獨斷,癉惡彰善,夷兇靜難。宗伯撰儀,太史練日,孤竹之管,雲和之瑟。展禮上玄,飛煙太一,珪璧朝會,山川望秩。占揆星景,移建邦畿,下憑赤壤,上葉紫微。布政衢室,懸法象魏,帝宅天府,固本崇威。匈河瀚海,龍荒狼望,種落陸梁,時犯亭障。皇威遠懾,帝德遐暢,稽顙歸誠,稱臣內向。吳越提封,鬥牛星象,積有年代,自稱君長。大風未繳,
 長鯨漏網,授鉞天人,豁然清蕩。戴日戴斗,太平太蒙,禮教周被,書軌大同。復禹之跡,成舜之功,禮以安上,樂以移風。憂勞庶績,矜育黔首,三面解羅,萬方引咎。納民軌物,驅時仁壽,神化隆平,生靈熙阜。虔心恭己,奉天事地,協氣橫流,休徵紹至。壇場望幸,雲亭虛位,推而不居,聖道彌粹。齊跡姬文,登發嗣聖,道類漢光,傳莊寶命。知來藏往,玄覽幽鏡,鼎業靈長,洪基隆盛。崆峒問道,汾射窅然,御辯遐逝,乘雲上仙。哀纏率土,痛感穹玄,流澤萬葉,用教百年。尚想睿圖,永惟聖則,道洽幽顯,仁沾動植。爻象不陳,乾坤將息,微臣作頌,用申罔極。



 帝覽之不悅,顧
 謂蘇威曰:「道衡致美先朝,此《魚藻》之義也。」於是拜司隸大夫,將置之罪。道衡不悟。司隸刺史房彥謙素相善,知必及禍,勸之杜絕賓客,卑辭下氣,而道衡不能用。會議新令,久不能決,道衡謂朝士曰:「向使高熲不死,令決當久行。」有人奏之,帝怒曰:「汝憶高熲邪?」付執法者勘之。道衡自以非大過,促憲司早斷。暨於奏日,冀帝赦之,敕家人具饌,以備賓客來候者。及奏,帝令自盡。道衡殊不意,未能引訣。憲司重奏,縊而殺之,妻子徙且末。時年七十。



 天下冤之。有集七十卷,行於世。



 有子五人,收最知名,出繼族父孺。孺清貞孤介,不交流俗,涉歷經史,有才思,雖
 不為大文,所有詩詠,詞致清遠。開皇中,為侍御史、揚州總管司功參軍。



 每以方直自處,府僚多不便之。及滿,轉清陽令、襄城郡掾,卒官。所經並有惠政。



 與道衡偏相友愛,收初生,即與孺為後,養於孺宅。至於成長,殆不識本生。太常丞胡仲操曾在朝堂,就孺借刀子割爪甲。孺以仲操非雅士,竟不與之。其不肯妄交,清介獨行,皆此類也。



 道衡兄子邁,官至選部郎,從父弟道實,官至禮部侍郎、離石太守,並知名於世。從子德音,有雋才,起家為游騎尉。佐魏澹修《魏史》,史成,遷著作佐郎。



 及越王侗稱制東都,王世充之僭號也,軍書羽檄,皆出其手。世充平,以
 罪伏誅。



 所有文筆,多行於時。



 史臣曰:二三子有齊之季,皆以辭藻著聞,爰歷周、隋,咸見推重。李稱一代俊偉,薛則時之令望,握靈蛇以俱照,騁逸足以並驅,文雅縱橫,金聲玉振。靜言揚榷,盧居二子之右。李、薛紆青拖紫,思道官途寥落,雖窮通有命,抑亦不護細行之所致也。



\end{pinyinscope}