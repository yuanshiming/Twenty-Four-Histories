\article{卷六十二列傳第二十七}

\begin{pinyinscope}

 王韶王韶,字子相,自云太原晉陽人也,世居京兆。祖諧,原州刺史。父諒,早卒。



 韶幼而方雅,頗好奇節,有識者異之。在周累以軍功官至車騎大將軍、議同三司。



 復轉軍正。武帝既拔晉州,意欲班師,韶諫曰:「齊失紀綱,於茲累世,天獎王室,一戰而扼其喉。加以主昏於上,民懼於下,取亂侮亡,正在今日。方欲釋之而去,以臣愚固,深所未解,願
 陛下圖之。」帝大悅,賜縑一百匹。及平齊氏,以功進位開府,封晉陽縣公,邑五百戶,賜口馬雜畜以萬計。遷內史中大夫。宣帝即位,拜豐州刺史,改封昌樂縣公。高祖受禪,進爵項城郡公,邑二千戶。轉靈州刺史,加位大將軍。



 晉王廣之鎮並州也,除行臺右僕射,賜彩五百匹。韶性剛直,王甚憚之,每事諮詢,不致違於法度。韶嘗奉使檢行長城,其後王穿池起三山,韶既還,自鎖而諫,王謝而罷之。高祖聞而嘉嘆,賜金百兩,並後宮四人。平陳之役,以本官為元帥府司馬,帥師趣河陽,與大軍會。既至壽陽,與高熲支度軍機,無所壅滯。及克金陵,韶即鎮焉。晉
 王廣班師,留韶於石頭防遏,委以後事,幾歲餘。徵還,高祖謂公卿曰:「晉王以幼稚出籓,遂能克平吳、越,綏靜江湖,子相之力也。」於是進位柱國,賜奴婢三百口,綿絹五千段。



 開皇十一年,上幸並州,以其稱職,特加勞勉。其後,上謂韶曰:「自朕至此,公須鬢漸白,無乃憂勞所致?柱石之望,唯在於公,努力勉之!」韶辭謝曰:「臣比衰暮,殊不解作官人。」高祖曰:「是何意也?不解者,是未用心耳。」韶對曰:「臣昔在昏季,猶且用心,況逢明聖,敢不罄竭!但神化精微,非駑蹇所逮。加以今年六十有六,桑榆雲晚,比於疇昔,昏忘又多。豈敢自寬,以速身累,恐以衰暮,虧紊朝綱耳。」
 上勞而遣之。秦王俊為並州總管,仍為長史。歲餘,馳驛入京,勞敝而卒,時年六十八。高祖甚傷惜之,謂秦王使者曰:「語爾王,我前令子相緩來,如何乃遣馳驛?殺我子相,豈不由汝邪?」言甚淒愴。使有司為之立宅,曰:「往者何用宅為,但以表我深心耳。」又曰:「子相受我委寄,十有餘年,終始不易,寵章未極,舍我而死乎!」發言流涕。因命取子相封事數十紙,傳示群臣。上曰:「其直言匡正,裨益甚多,吾每披尋,未嘗釋手。」煬帝即位,追贈司徒、尚書令、靈豳等十州刺史、魏國公。子士隆嗣。



 士隆略知書計,尤便弓馬,慷慨有父風。大業之世,頗見親重,官至備身將軍,
 改封耿公。數令討擊山賊,往往有捷。越王侗稱帝,士隆率數千兵自江、淮而至。



 會王世充僭號,甚禮重之,署尚書右僕射。士隆憂憤,疽發背卒。



 元巖元巖,字君山,河南洛陽人也。父禎,魏敷州刺史。巖好讀書,不治章句,剛鯁有器局,以名節自許,少與渤海高熲、太原王韶同志友善。仕周,釋褐宣威將軍、武賁給事。大塚宰宇文護見而器之,以為中外記室。累遷內史中大夫,昌國縣伯。



 宣帝嗣位,為政昏暴,京兆郡丞樂運乃輿櫬詣朝堂,陳帝八失,言甚切至。帝大怒,將戮之。朝臣皆
 恐懼,莫有救者。巖謂人曰:「臧洪同日,尚可俱死,其況比干乎!



 若樂運不免,吾將與之俱斃。」詣閣請見,言於帝曰:「樂運知書奏必死,所以不顧身命者,欲取後世之名。陛下若殺之,乃成其名,落其術內耳。不如勞而遣之,以廣聖度。」運因獲免。後帝將誅烏丸軌,巖不肯署詔。御正顏之儀切諫不入,巖進繼之,脫巾頓顙,三拜三進。帝曰:「汝欲黨烏丸軌邪?」巖曰:「臣非黨軌,正恐濫誅失天下之望。」帝怒,使閹豎搏其面,遂廢於家。



 高祖為丞相,加位開府、民部中大夫。及受禪,拜兵部尚書,進爵平昌郡公,邑二千戶。巖性嚴重,明達世務,每有奏議,侃然正色,庭諍面
 折,無所回避。上及公卿,皆敬憚之。時高祖初即位,每懲周代諸侯微弱,以致滅亡,由是分王諸子,權侔王室,以為磐石之固,遣晉王廣鎮並州,蜀王秀鎮益州。二王年並幼稚,於是盛選貞良有重望者為之僚佐。於時巖與王韶俱以骨鯁知名,物議稱二人才具侔於高熲,由是拜巖為益州總管長史,韶為河北道行臺右僕射。高祖謂之曰:「公宰相大器,今屈輔我兒,如曹參相齊之意也。」及巖到官,法令明肅,吏民稱焉。蜀王性好奢侈,嘗欲取獠口以為閹人,又欲生剖死囚,取膽為藥。巖皆不奉教,排閣切諫,王輒謝而止,憚巖為人,每循法度。蜀中獄訟,
 巖所裁斷,莫不悅服。其有得罪者,相謂曰:「平昌公與吾罪,吾何怨焉。」上甚嘉之,賞賜優洽。十三年,卒官,上悼惜久之。益州父老,莫不殞涕,於今思之。巖卒之後,蜀王竟行其志,漸致非法,造渾天儀、司南車、記里鼓,凡所被服,擬於天子。又共妃出獵,以彈彈人,多捕山獠,以充宦者。僚佐無能諫止。及秀得罪,上曰:「元巖若在,吾兒豈有是乎!」



 子弘嗣。仕歷給事郎、司朝謁者、北平通守。



 劉行本劉行本,沛人也。父瑰,仕梁,歷職清顯。行本起家武陵國常侍。遇蕭修以梁州北附,遂與叔父璠同歸於周,寓居
 京兆之新豐。每以諷讀為事,精力忘疲,雖衣食乏絕,晏如也。性剛烈,有不可奪之志。周大塚宰宇文護引為中外府記室。武帝親總萬機,轉御正中士,兼領起居注。累遷掌朝下大夫。周代故事,天子臨軒,掌朝典筆硯,持至御坐,則承御大夫取以進之。及行本為掌朝,將進筆於帝,承御復欲取之。行本抗聲謂承御曰:「筆不可得。」帝驚視問之,行本言於帝曰:「臣聞設官分職,各有司存。臣既不得佩承御刀,承御亦焉得取臣筆。」帝曰:「然。」



 因令二司各行所職。及宣帝嗣位,多失德,行本切諫忤旨,出為河內太守。



 高祖為丞相,尉迥作亂,進攻懷州。行本率吏民
 拒之,拜儀同,賜爵文安縣子。



 及踐阼,徵拜諫議大夫,檢校治書侍御史。未幾,遷黃門侍郎。上嘗怒一郎,於殿前笞之。行本進曰:「此人素清,其過又小,願陛下少寬假之。」上不顧。行本於是正當上前曰:「陛下不以臣不肖,置臣左右。臣言若是,陛下安得不聽?臣言若非,當致之於理,以明國法,豈得輕臣而不顧也!臣所言非私。」因置笏於地而退,上斂容謝之,遂原所笞者。於時天下大同,四夷內附,行本以黨項羌密邇封域,最為後服,上表劾其使者曰:「臣聞南蠻遵校尉之統,西域仰都護之威。比見西羌鼠竊狗盜,不父不子,無君無臣,異類殊方,於斯為下。
 不悟羈縻之惠,詎知含養之恩,狼戾為心,獨乖正朔。使人近至,請付推科。」上奇其志焉。雍州別駕元肇言於上曰:「有一州吏,受人饋錢三百文,依律合杖一百。然臣下車之始,與其為約。



 此吏故違,請加徒一年。」行本駁之曰:「律令之行,並發明詔,與民約束。今肇乃敢重其教命,輕忽憲章。欲申己言之必行,忘朝廷之大信,虧法取威,非人臣之禮。」上嘉之,賜絹百匹。



 在職數年,拜太子左庶子,領治書如故。皇太子虛襟敬憚。時唐令則亦為左庶子,太子暱狎之,每令以弦歌教內人。行本責之曰:「庶子當匡太子以正道,何有嬖暱房帷之間哉!」令則甚慚而不
 能改。時沛國劉臻、平原明克讓、魏郡陸爽並以文學為太子所親。行本怒其不能調護,每謂三人曰:「卿等正解讀書耳。」時左衛率長史夏侯福為太子所暱,嘗於閣內與太子戲。福大笑,聲聞於外。行本時在閣下聞之,待其出,行本數之曰:「殿下寬容,賜汝顏色。汝何物小人,敢為褻慢!」



 因付執法者治之。數日,太子為福致請,乃釋之。太子嘗得良馬,令福乘而觀之。



 太子甚悅,因欲令行本復乘之。行本不從,正色而進曰:「至尊置臣於庶子之位者,欲令輔導殿下以正道,非為殿下作弄臣也。」太子慚而止。復以本官領大興令,權貴憚其方直,無敢至門者。由
 是請托路絕,法令清簡,吏民懷之。未幾,卒官,上甚傷惜之。及太子廢,上曰:「嗟乎!若使劉行本在,勇當不及於此。」無子。



 梁毗梁毗,字景和,安定烏氏人也。祖越,魏涇、豫、洛三州刺史,郃陽縣公。父茂,周滄、兗二州刺史。毗性剛謇,頗有學涉。周武帝時,舉明經,累遷布憲下大夫。平齊之役,以毗為行軍總管長史,克並州,毗有力焉。除為別駕,加儀同三司。



 宣政中,封易陽縣子,邑四百戶。遷武藏大夫。高祖受禪,進爵為侯。開皇初,置御史官,朝廷以毗鯁正,拜治書
 侍御史,名為稱職。尋轉大興令,遷雍州贊治。毗既出憲司,復典京邑,直道而行,無所回避,頗失權貴心,由是出為西寧州刺史,改封邯鄲縣侯。在州十一年。先是,蠻夷酋長皆服金冠,以金多者為豪俊,由此遞相陵奪,每尋干戈,邊境略無寧歲。毗患之。後因諸酋長相率以金遺毗,於是置金坐側,對之慟哭而謂之曰:「此物饑不可食,寒不可衣。汝等以此相滅,不可勝數。



 今將此來,欲殺我邪?」一無所納,悉以還之。於是蠻夷感悟,遂不相攻擊。高祖聞而善之,徵為散騎常侍、大理卿。處法平允,時入稱之。歲餘,進位上開府。



 毗見左僕射楊素貴寵擅權,百僚
 震懾,恐為國患,因上封事曰:「臣聞臣無有作威福。臣之作威福,其害乎而家,兇乎而國。竊見左僕射、越國公素,幸遇愈重,權勢日隆,搢紳之徒,屬其視聽。忤意者嚴霜夏零,阿旨者膏雨冬澍,榮枯由其脣吻,廢興候其指麾。所私皆非忠讜,所進咸是親戚,子弟布列,兼州連縣。天下無事,容息異圖,四海稍虞,必為禍始。夫奸臣擅命,有漸而來。王莽資之於積年,桓玄基之於易世,而卒殄漢祀,終傾晉祚。季孫專魯,田氏篡齊,皆載典誥,非臣臆說。陛下若以素為阿衡,臣恐其心未必伊尹也。伏願揆鑒古今,量為外置,俾洪基永固,率土幸甚。輕犯天顏,伏聽
 斧金質。」高祖大怒,命有司禁止,親自詰之。



 毗極言曰:「素既擅權寵,作威作福,將領之處,殺戮無道。又太子及蜀王罪廢之日,百僚無不震悚,惟素揚眉奮肘,喜見容色,利國家有事以為身幸。」毗發言謇謇,有誠亮之節,高祖無以屈也,乃釋之。素自此恩寵漸衰。但素任寄隆重,多所折挫,當時朝士無不懾伏,莫有敢與相是非。辭氣不撓者,獨毗與柳彧及尚書右丞李綱而已。後上不復專委於素,蓋由察毗之言也。



 煬帝即位,遷刑部尚書,並攝御史大夫事。奏劾宇文述私役部兵,帝議免述罪,毗固諍,因忤旨,遂令張衡代為大夫。毗憂憤,數月而卒。帝令吏
 部尚書牛弘吊之,贈縑五百匹。



 子敬真,大業之世,為大理司直。時帝欲成光祿大夫魚俱羅之罪,令敬直治其獄,遂希旨陷之極刑。未幾,敬真有疾,見俱羅為之厲,數日而死。



 柳彧柳彧,字幼文,河東解人也。七世祖卓,隨晉南遷,寓居襄陽。父仲禮,為梁將,敗歸周,復家本土。彧少好學,頗涉經史。周大塚宰宇文護引為中外府記室,久而出為寧州總管掾。武帝親總萬機,彧詣闕求試。帝異之,以為司武中士。轉鄭令。平齊之後,帝大賞從官,留京者不預。彧上
 表曰:「今太平告始,信賞宜明,酬勛報勞,務先有本。屠城破邑,出自聖規,斬將搴旗,必由神略。若負戈擐甲,徵捍劬勞,至於鎮撫國家,宿衛為重。俱稟成算,非專己能,留從事同,功勞須等。



 皇太子以下,實有守宗廟之功。昔蕭何留守,茅土先於平陽,穆之居中,沒後猶蒙優策。不勝管見,奉表以聞。」於是留守並加泛級。



 高祖受禪,累遷尚書虞部侍郎,以母憂去職。未幾,起為屯田侍郎,固讓弗許。



 時制三品已上,門皆列戟。左僕射高熲子弘德封應國公,申牒請戟。彧判曰:「僕射之子更不異居,父之戟槊已列門外。尊有壓卑之義,子有避父之禮,豈容外門既
 設,內閤又施!」事竟不行,熲聞而嘆伏。後遷治書侍御史,當朝正色,甚為百僚之所敬憚。上嘉其婞直,謂彧曰:「大丈夫當立名於世,無容容而已。」賜錢十萬,米百石。



 於時刺史多任武將,類不稱職。彧上表曰:「方今天下太平,四海清謐,共治百姓,須任其才。昔漢光武一代明哲,起自布衣,備知情偽,與二十八將披荊棘,定天下,及功成之後,無所職任。伏見詔書,以上柱國和乾子為杞州刺史,其人年垂八十,鐘鳴漏盡。前任趙州,暗於職務,政由群小,賄賂公行,百姓籲嗟,歌謠滿道。乃云:『老禾不早殺,餘種穢良田。』古人有云:『耕當問奴,織當問婢。』此言各有所
 能也。乾子弓馬武用,是其所長,治民蒞職,非其所解。至尊思治,無忘寢興,如謂優老尚年,自可厚賜金帛,若令刺舉,所損殊大。臣死而後已,敢不竭誠。」上善之,乾子竟免。有應州刺史唐君明,居母喪,娶雍州長史庫狄士文之從父妹。彧劾之曰:「臣聞天地之位既分,夫婦之禮斯著,君親之義生焉,尊卑之教攸設。是以孝惟行本,禮實身基,自國刑家,率由斯道。竊以愛敬之情,因心至切,喪紀之重,人倫所先。君明鉆燧雖改,在文無變,忽劬勞之痛,成宴爾之親,冒此苴縗,命彼褕翟。不義不暱,《春秋》載其將亡,無禮無儀,詩人欲其遄死。



 士文贊務神州,名位
 通顯,整齊風教,四方是則,棄二姓之重匹,違六禮之軌儀。



 請禁錮終身,以懲風俗。」二人竟坐得罪。隋承喪亂之後,風俗頹壞,彧多所矯正,上甚嘉之。又見上勤於聽受,百僚奏請,多有煩碎,因上疏諫曰:「臣聞自古聖帝,莫過唐、虞,象地則天,布政施化,不為叢脞,是謂欽明。語曰:『天何言哉,四時行焉。』故知人君出令,誡在煩數。是以舜任五臣,堯咨四岳,設官分職,各有司存,垂拱無為,天下以治。所謂勞於求賢,逸於任使。又云:『天子穆穆,諸侯皇皇。』此言君臣上下,體裁有別。比見四海一家,萬機務廣,事無大小,咸關聖聽。陛下留心治道,無憚疲勞,亦由群官
 懼罪,不能自決,取判天旨。聞奏過多,乃至營造細小之事,出給輕微之物,一日之內,酬答百司,至乃日旰忘食,夜分未寢,動以文簿,憂勞聖躬。伏願思臣至言,少減煩務,以怡神為意,以養性為懷,思武王安樂之義,念文王勤憂之理。若其經國大事,非臣下裁斷者,伏願詳決,自餘細務,責成所司,則聖體盡無疆之壽,臣下蒙覆育之賜也。」上覽而嘉之。後以忤旨免。未幾,復令視事,因謂彧曰:「無改爾心。」以其家貧,敕有司為之築宅。



 因曰:「柳彧正直士,國之寶也。」其見重如此。



 右僕射楊素當途顯貴,百僚懾憚,無敢忤者。嘗以少譴,敕送南臺。素恃貴,坐彧床。
 彧從外來,見素如此,於階下端笏整容謂素曰:「奉敕治公之罪。」素遽下。彧據案而坐,立素於庭,辨詰事狀。素由是銜之。彧時方為上所信任,故素未有以中之。



 彧見近代以來,都邑百姓每至正月十五日,作角抵之戲,遞相誇競,至於糜費財力,上奏請禁絕之,曰:「臣聞昔者明王訓民治國,率履法度,動由禮典。非法不服,非道不行。道路不同,男女有別,防其邪僻,納諸軌度。竊見京邑,爰及外州,每以正月望夜,充街塞陌,聚戲朋游。鳴鼓聒天,燎炬照地,人戴獸面,男為女服,倡優雜技,詭狀異形。以穢嫚為歡娛,用鄙褻為笑樂,內外共觀,曾不相避。



 高棚跨
 路,廣幕陵雲,袨服靚妝,車馬填噎。肴醑肆陳,絲竹繁會,竭貲破產,竟此一時。盡室並孥,無問貴賤,男女混雜,緇素不分。穢行因此而生,盜賊由斯而起。浸以成俗,實有由來,因循敝風,曾無先覺。非益於化,實損於民。請頒行天下,並即禁斷。康哉《雅》、《頌》,足美盛德之形容,鼓腹行歌,自表無為之至樂。敢有犯者,請以故違敕論。」詔可其奏。是歲,持節巡省河北五十二州,奏免長吏贓污不稱職者二百餘人,州縣肅然,莫不震懼。上嘉之,賜絹布二百匹、氈三十領,拜儀同三司。歲餘,加員外散騎常侍,治書如故。仁壽初,復持節巡省太原道十九州。及還,賜絹百
 五十匹。



 彧嘗得博陵李文博所撰《治道集》十卷,蜀王秀遣人求之。彧送之於秀,秀復賜彧奴婢十口。及秀得罪,楊素奏彧以內臣交通諸侯,除名為民,配戍懷遠鎮。行達高陽,有詔徵還。至晉陽,值漢王諒作亂,遣使馳召彧,將與計事。彧為使所逼,初不知諒反,將入城而諒反形已露。彧度不得免,遂詐中惡不食,自稱危篤。諒怒,囚之。及諒敗,楊素奏彧心懷兩端,以候事變,跡雖不反,心實同逆,坐徙敦煌。



 楊素卒後,乃自申理,有詔徵還京師,卒於道。有子紹,為介休令。



 趙綽
 趙綽,河東人也,性質直剛毅。在周初為天官府史,以恭謹恪勤,擢授夏官府下士。稍以明乾見知,累轉內史中士。父艱去職,哀毀骨立,世稱其孝。既免喪,又為掌教中士。高祖為丞相,知其清正,引為錄事參軍。尋遷掌朝大夫,從行軍總管是雲暉擊叛蠻,以功拜儀同,賜物千段。高祖受禪,授大理丞。處法平允,考績連最,轉大理正。尋遷尚書都官侍郎,未幾轉刑部侍郎。治梁士彥等獄,賜物三百段,奴婢十口,馬二十匹。每有奏讞,正色侃然,上嘉之,漸見親重。上以盜賊不禁,將重其法。綽進諫曰:「陛下行堯、舜之道,多存寬宥。況律者天下之大信,其可失
 乎!」上忻然納之,因謂綽曰:「若更有聞見,宜數陳之也。」遷大理少卿。



 故陳將蕭摩訶,其子世略在江南作亂,摩訶當從坐。上曰:「世略年未二十,亦何能為!以其名將之子,為人所逼耳。」因赦摩訶。綽固諫不可,上不能奪,欲綽去而赦之,固命綽退食。綽曰:「臣奏獄未決,不敢退朝。」上曰:「大理其為朕特赦摩訶也。」因命左右釋之。刑部侍郎辛亶,嘗衣緋衣軍,俗云利於官,上以為厭蠱,將斬之。綽曰:「據法不當死,臣不敢奉詔。」上怒甚,謂綽曰:「卿惜辛亶而不自惜也?」命左僕射高熲將綽斬之,綽曰:「陛下寧可殺臣,不得殺辛亶。」



 至朝堂,解衣當斬,上使人謂綽曰:「竟何如?」
 對曰:「執法一心,不敢惜死。」



 上拂衣而入,良久乃釋之。明日,謝綽,勞勉之,賜物三百段。時上禁行惡錢,有二人在市,以惡錢易好者,武候執以聞,上令悉斬之。綽進諫曰:「此人坐當杖,殺之非法。」上曰:「不關卿事。」綽曰:「陛下不以臣愚暗,置在法司,欲妄殺人,豈得不關臣事?」上曰:「撼大木不動者,當退。」對曰:「臣望感天心,何論動木!」上復曰:「啜羹者,熱則置之。天子之威,欲相挫耶?」綽拜而益前,訶之不肯退。上遂入。治書侍御史柳彧復上奏切諫,上乃止。上以綽有誠直之心,每引入閤中,或遇上與皇后同榻,即呼綽坐,評論得失。前後賞賜萬計。其後進位開府,贈
 其父為蔡州刺史。時河東薛胄為大理卿,俱名平恕。然胄斷獄以情,而綽守法,俱為稱職。上每謂綽曰:「朕於卿無所愛惜,但卿骨相不當貴耳。」仁壽中卒官,時年六十三。上為之流涕,中使吊祭,鴻臚監護喪事。有二子,元方、元襲。



 裴肅裴肅,字神封,河東聞喜人也。父俠,周民部大夫。肅少剛正有局度,少與安定梁毗同志友善。仕周,釋褐給事中士,累遷御正下大夫。以行軍長史從韋孝寬征淮南。屬高祖為丞相,肅聞而嘆曰:「武帝以雄才定六合,墳土未
 幹,而一朝遷革,豈天道歟!」高祖聞之,甚不悅,由是廢於家。開皇五年,授膳部侍郎。後二歲,遷朔州總管長史,轉貝州長史,俱有能名。仁壽中,肅見皇太子勇、蜀王秀、左僕射高熲俱廢黜,遣使上書曰:「臣聞事君之道,有犯無隱,愚情所懷,敢不聞奏。



 竊見高熲以天挺良才,元勛佐命,陛下光寵,亦已優隆。但鬼瞰高明,世疵俊異,側目求其長短者,豈可勝道哉!願陛下錄其大功,忘其小過。臣又聞之,古先聖帝,教而不誅,陛下至慈,度越前聖。二庶人得罪已久,寧無革心?願陛下弘君父之慈,顧天性之義,各封小國,觀其所為。若能遷善,漸更增益,如或不悛,
 貶削非晚。



 今者自新之路永絕,愧悔之心莫見,豈不哀哉!」書奏,上謂楊素曰:「裴肅憂我家事,此亦至誠也。」於是徵肅入朝。皇太子聞之,謂左庶子張衡曰:「使勇自新,欲何為也?」衡曰:「觀肅之意,欲令如吳太伯、漢東海王耳。」皇太子甚不悅。



 頃之,肅至京師,見上於含章殿,上謂肅曰:「吾貴為天子,富有四海,後宮寵幸,不過數人,自勇以下,並皆同母,非為憎愛,輕事廢立。」因言勇不可復收之意。



 既而罷遣之。未幾,上崩。煬帝嗣位,不得調者久之,肅亦杜門不出。後執政者以嶺表荒遐,遂希旨授肅永平郡丞,甚得民夷心。歲餘,卒,時年六十二。夷、獠思之,為立廟
 於鄣江之浦。有子尚賢。



 史臣曰:猛獸之處山林,藜藿為之不採;正臣之立朝廷,奸邪為之折謀。皆志在匪躬,義形於色,豈惟綱紀由其隆替,抑亦社稷系以存亡者也。晉、蜀二王,帝之愛子,擅以權寵,莫拘憲令,求其恭肅,不亦難乎!元巖、王韶,任當彼相,並見嚴憚,莫敢為非,謇諤之風,有足稱矣。行本正色於房陵,梁毗抗言於楊素,直辭鯁氣,懍焉可想。趙綽之居大理,囹圄無冤,柳彧之處憲臺,奸邪自肅。然不畏強御,梁毗其有焉,邦之司直,行本、柳彧近之矣。裴肅朝不坐,宴不預,忠誠慷慨,犯忤龍鱗,固知嫠婦憂宗周之
 亡,處女悲太子之少,非徒語也。方諸前載,有閻纂之風焉。



\end{pinyinscope}