\article{卷九志第四 禮儀四}

\begin{pinyinscope}

 周大定元年,靜帝遣兼太傅、上柱國、杞國公椿,大宗伯、大將軍、金城公煚,奉皇帝璽紱策書,禪位於隋。司錄虞慶則白,請設壇於東第。博士何妥議,以為受禪登壇,以告天也,故魏受漢禪,設壇於繁昌,為在行旅,郊壇乃闕。至如漢高在汜,光武在鄗,盡非京邑所築壇。自晉、宋揖讓,皆在都下,莫不並就南郊,更無別築之義。又後魏即
 位,登硃雀觀,周帝初立,受朝於路門,雖自我作古,皆非禮也。今即府為壇,恐招後誚。議者從之。二月甲子,椿等乘象輅,備鹵簿,持節,率百官至門下,奉策入次。百官文武,朝服立於門南,北面。高祖冠遠游冠,府僚陪列。記室入白,禮曹導高祖,府僚從,出大門東廂西向。椿奉策書,煚奉璽紱,出次,節導而進。高祖揖之,入門而左,椿等入門而右。百官隨入庭中。椿南向,讀冊書畢,進授高祖。高祖北面再拜,辭不奉詔。上柱國李穆進喻朝旨,又與百官勸進,高祖不納。椿等又奉策書進而敦勸,高祖再拜,俯受策,以授高熲;受璽,以授虞慶則。退就東階位。使者
 與百官皆北面再拜,搢笏,三稱萬歲。有司請備法駕,高祖不許,改服紗帽、黃袍,入幸臨光殿。就閣內服袞冕,乘小輿,出自西序,如元會儀。禮部尚書以案承符命及祥瑞牒,進東階下。納言跪御前以聞。內史令奉宣詔大赦,改元曰開皇。是日,命有司奉冊祀於南郊。



 後齊將崇皇太后,則太尉以玉帛告圓丘方澤,以幣告廟。皇帝乃臨軒,命太保持節,太尉副之。設九儐,命使者受璽綬冊及節,詣西上閤。其日,昭陽殿文物具陳,臨軒訖,使者就位,持節及璽綬稱詔。二侍中拜進,受節及冊璽綬,以付小黃門。黃門以詣閤。皇太后服褘衣,處昭陽
 殿,公主及命婦陪列於殿,皆拜。小黃門以節綬入,女侍中受,以進皇太后。皇太后興,受,以授左右。復坐,反節於使者。



 使者受節出。冊皇后,如太后之禮。



 後齊冊皇太子,則皇帝臨軒,司待為使,司空副之。太子服遠游冠,入至位。



 使者入,奉冊讀訖,皇太子跪受冊於使,以授中庶子。又受璽綬於尚書,以授庶子。



 稽首以出。就冊,則使者持節至東宮,宮臣內外官定列。皇太子階東,西面。若幼,則太師抱之,主衣二人奉空頂幘服從,以受冊。明日,拜章表於東宮殿庭,中庶子、中舍人乘軺車,奉章詣朝堂謝。擇日齋於崇正殿,服冕,乘石山安車謁
 廟。擇日群臣上禮,又擇日會。明日,三品以上箋賀。



 冊諸王,以臨軒日上水一刻,吏部令史乘馬,齎召版,詣王第。王乘高車,鹵簿至東掖門止,乘軺車。既入,至席。尚書讀冊訖,以授王,又授章綬。事畢,乘軺車,入鹵簿,乘高車,詣閶闔門,伏闕表謝。報訖,拜廟還第。就第,則鴻臚卿持節,吏部尚書授冊,侍御史授節。使者受而出,乘軺車,持節,詣王第。入就西階,東面。王入,立於東階,西面。使者讀冊,博士讀版,王俯伏。興,進受冊章綬茅土,俯伏三稽首,還本位,謝如上儀。在州鎮,則使者受節冊,乘軺車至州,如王第。



 諸王、三公、儀同、尚書令、五等開國、太妃、妃、公主恭拜冊,軸一枚,長二尺,以白練衣之。用竹簡十二枚,六枚與軸等,六枚長尺二寸。文出集書,書皆篆字。哀冊、贈冊亦同。諸王、五等開國及鄉男恭拜,以其封國所在方,取社壇方面土,包以白茅,內青箱中。函方五寸,以青塗飾,封授之,以為社。



 隋臨軒冊命三師、諸王、三公,並陳車輅。餘則否。百司定列,內史令讀冊訖,受冊者拜受出。又引次受冊者,如上儀。若冊開國,郊社令奉茅土,立於仗南,西面。每受冊訖,授茅土焉。



 後齊皇帝加元服,以玉帛告圓丘方澤,以幣告廟,擇日臨軒。中嚴,群官位定,皇帝著空頂介幘以出。太尉盥訖,升,脫空頂幘,以黑介幘奉加訖,太尉進太保之右,北面讀祝訖,太保加冕,侍中系玄紱,脫絳紗袍,加袞服,事畢,太保上壽,群官三稱萬歲。皇帝入溫室,移御坐,會而不上壽。後日,文武群官朝服,上禮酒十二鐘,米十二囊,牛十二頭。又擇日親拜圓丘方澤,謁廟。



 皇太子冠,則太尉以制幣告七廟,擇日臨軒。有司供帳於崇正殿。中嚴,皇太子空頂幘公服出,立東階之南,西面,使者入,立西階之南,東面。皇太子受詔訖,入室盥櫛,
 出,南面。使者進揖,詣冠席,西面坐。光祿卿盥訖,詣太子前疏櫛。



 使者又盥,奉進賢三梁冠,至太子前,東面祝,脫空頂幘,加冠。太子興,入室更衣,出,又南面就席。光祿卿盥櫛。使者又盥祝,脫三梁冠,加遠游冠。太子又入室更衣。設席中楹之西,使者揖就席,南面。光祿卿洗爵酌醴,使者詣席前,北面祝。太子拜受醴,即席坐,祭之,啐之,奠爵,降階,復本位,西面。三師、三少及在位群官拜事訖。又擇日會宮臣,又擇日謁廟。



 隋皇太子將冠,前一日,皇帝齋於大興殿。皇太子與賓贊及預從官齋於正寢。



 其日質明,有司告廟,各設筵於
 阼階。皇帝袞冕入拜,即御座。賓揖皇太子進,升筵,西向坐。贊冠者坐櫛,設纚。賓盥訖,進加緇布冠。贊冠進設頍纓。賓揖皇太子適東序,衣玄衣素裳以出。贊冠者又坐櫛,賓進加遠游冠。改服訖,賓又受冕。



 太子適東序,改服以出。賓揖皇太子南面立,賓進受醴,進筵前,北面立祝。皇太子拜受觶。賓復位,東面答拜。贊冠者奉饌於筵前,皇太子祭奠。禮畢,降筵,進當御東面拜。納言承詔,詣太子戒訖,太子拜。贊冠者引太子降自西階。賓少進,字之。贊冠者引皇太子進,立於庭,東面。諸親拜訖,贊冠者拜,太子皆答拜。與賓贊俱復位。納言承詔降,令有司致禮。
 賓贊又拜。皇帝降復阼階,拜,皇太子已下皆拜。皇帝出,更衣還宮。皇太子從至闕,因入見皇后,拜而還。



 後齊皇帝納后之禮,納採、問名、納徵訖,告圓丘方澤及廟,如加元服,是日,皇帝臨軒,命太尉為使,司徒副之。持節詣皇后行宮,東向,奉璽綬冊,以授中常侍。皇后受冊於行殿。使者出,與公卿以下皆拜。有司備迎禮。太保太尉,受詔而行。主人公服,迎拜於門。使者入,升自賓階,東面。主人升自阼階,西面。禮物陳於庭。設席於兩楹間,童子以璽書版升,主人跪受。送使者,拜於大門之外。有司先於昭陽殿兩楹間供帳,為同牢之具。皇后服大嚴繡
 衣,帶綬珮,加幜。女長御引出,升畫輪四望車。女侍中負璽陪乘。鹵簿如大駕。皇帝服袞冕出,升御坐。皇后入門,大鹵簿住門外,小鹵簿入。到東上閤,施步鄣,降車,席道以入昭陽殿。



 前至席位,姆去幜,皇后先拜後起,皇帝後拜先起。帝升自西階,詣同牢坐,與皇后俱坐。各三飯訖,又各酳二爵一巹。奏禮畢,皇后興,南面立。皇帝御太極殿,王公已下拜,皇帝興,入。明日,後展衣,於昭陽殿拜表謝。又明日,以榛慄棗修,見皇太后於昭陽殿。擇日,群官上禮。又擇日謁廟。皇帝使太尉先以太牢告,而後遍見群廟。
 皇太子納妃禮,皇帝遣使納採,有司備禮物。會畢,使者受詔而行。主人迎於大門外。禮畢,會於聽事。其次問名、納吉,並如納採。納徵,則使司徒及尚書令為使,備禮物而行。請期,則以太常宗正卿為使,如納採。親迎,則太尉為使。三日,妃朝皇帝於昭陽殿,又朝皇后於宣光殿。擇日,群官上禮。他日,妃還。



 又他日,皇太子拜閤。



 隋皇太子納妃禮,皇帝臨軒,使者受詔而行。主人俟於廟。使者執雁,主人迎拜於大門之東。使者入,升自西階,立於楹間,南面。納採訖,乃行問名儀。事畢,主人請致禮於從者。禮有幣馬。其次擇日納吉,如納採。又擇日,以玉
 帛乘馬納徵。



 又擇日告期。又擇日,命有司以特牲告廟,冊妃。皇太子將親迎,皇帝臨軒,醮而誡曰:「往迎爾相,承我宗事,勖帥以敬。」對曰:「謹奉詔。」既受命,羽儀而行。主人幾筵於廟,妃服褕翟,立於東房。主人迎於門外,西面拜。皇太子答拜。



 主人揖皇太子先入,主人升,立於阼階,西面。皇太子升進,當房戶前,北面,跪奠雁,俯伏,興拜,降出。妃父少進,西面戒之。母於西階上,施衿結帨,及門內,施鞶申之。出門,妃升輅,乘以幾。姆加幜。皇太子乃御,輪三周,御者代之。皇太子出大門,乘輅,羽儀還宮。妃三日,雞鳴夙興以朝。奠於皇帝,皇帝撫之。



 又奠於皇后,皇
 后撫之。席於戶牖間,妃立於席西,祭奠而出。



 後齊娉禮,一曰納採,二曰問名,三曰納吉,四曰納徵,五曰請期,六曰親迎。



 皆用羔羊一口,雁一雙,酒黍稷稻米面各一斛。自皇子王已下至於九品皆同,流外及庶人則減其半。納徵,皇子王用玄三匹,纁二匹,束帛十匹,大璋一第一品已下至從三品,用璧玉,四品已下皆無。獸皮二第一品已下至從五品,用豹皮二,六品已下至從九品,用鹿皮。錦彩六十匹一品錦彩四十匹,二品三十匹,三品二十匹,四品雜彩十六匹,五品十匹,六品、七品五匹。絹二百匹,一品一百四十匹,二品一百二十匹,三品一百匹,四品八十匹,五品六十匹,六品、七品五十匹,八品、九品三十匹。羔羊一口,羊四口,犢二頭,酒黍稷稻米面各十斛。一品至三品,減羊二口,酒黍稷稻米面各減六斛,四品、五品減
 一犢,酒黍稷稻米面又減二斛,六品以下無犢,酒黍稷稻米面各一斛。諸王之子,已封未封,禮皆同第一品。新婚從車,皇子百乘,一品五十乘,第二、第三品三十乘,第四、第五品二十乘,第六、第七品十乘,八品達於庶人五乘。各依其秩之飾。



 梁大同五年,臨城公婚,公夫人於皇太子妃為姑侄,進見之制,議者互有不同。



 令曰:「纁雁之儀,既稱合於二姓,酒食之會,亦有姻不失親。若使榛慄段修,贄饋必舉,副笄編珈,盛飾斯備,不應婦見之禮,獨以親闕。頃者敬進酏醴,已傳婦事之則,而奉盤沃盥,不行侯服之家。是知繁省不同,質文異世,臨城公夫人於妃既是姑侄,宜停
 省。」



 後齊將講於天子,先定經於孔父廟,置執經一人,侍講二人,執讀一人,擿句二人,錄義六人,奉經二人。講之旦,皇帝服通天冠、玄紗袍,乘象輅,至學,坐廟堂上。講訖,還便殿,改服絳紗袍,乘象輅,還宮。講畢,以一太牢釋奠孔父,配以顏回,列軒懸樂,六佾舞。行三獻禮畢,皇帝服通天冠、絳紗袍,升阼,即坐。



 宴畢,還宮。皇太子每通一經,亦釋奠,乘石山安車,三師乘車在前,三少從後而至學焉。



 梁天監八年,皇太子釋奠。周舍議,以為:「釋奠仍會,既惟大禮,請依東宮元會,太子著絳紗襮,樂用軒懸。預升殿
 坐者,皆服硃衣。」帝從之。又有司以為:「《禮》云:『凡為人子者,升降不由阼階。』案今學堂凡有三階,愚謂客若降等,則從主人之階。今先師在堂,義所尊敬,太子宜登阼階,以明從師之義。若釋奠事訖,宴會之時,無復先師之敬,太子升堂,則宜從西階,以明不由阼義。」吏部郎徐勉議:「鄭玄云:『由命士以上,父子異宮。』宮室既異,無不由阼階之禮。請釋奠及宴會,太子升堂,並宜由東階。若輿駕幸學,自然中陛。又檢《東宮元會儀注》,太子升崇正殿,不欲東西階。責東宮典儀,列云:『太子元會,升自西階』,此則相承為謬。請自今東宮大公事,太子升崇正殿,並由阼階。其
 預會賓客,依舊西階。」



 大同七年,皇太子表其子寧國、臨城公入學,時議者以與太子有齒胄之義,疑之。侍中、尚書令臣敬容、尚書僕射臣纘、尚書臣僧旻、臣之遴、臣筠等,以為:「參、點並事宣尼,回、路同諮泗水,鄒魯稱盛,洙汶無譏。師道既光,得一資敬,無虧亞貳,況於兩公,而云不可?」制曰:「可。」



 後齊制,新立學,必釋奠禮先聖先師,每歲春秋二仲,常行其禮。每月旦,祭酒領博士已下及國子諸學生已上,太學、四門博士升堂,助教已下、太學諸生階下,拜孔揖顏。日出行事而不至者,記之為一負。雨沾服則止。學生
 每十日給假,皆以丙日放之。郡學則於坊內立孔、顏廟,博士已下,亦每月朝雲。



 隋制,國子寺,每歲以四仲月上丁,釋奠於先聖先師。年別一行鄉飲酒禮。州郡學則以春秋仲月釋奠。州郡縣亦每年於學一行鄉飲酒禮。學生皆乙日試書,丙日給假焉。



 梁元會之禮,未明,庭燎設,文物充庭。臺門闢,禁衛皆嚴,有司各從其事。



 太階東置白獸樽。群臣及諸蕃客並集,各從其班而拜。侍中奏中嚴,王公卿尹各執珪璧入拜。侍中乃奏外辦,皇帝服袞冕,乘輿以出。侍中扶左,常侍
 扶右,黃門侍郎一人執曲直華蓋從。至階,降輿,納舄升坐。有司御前施奉珪藉。王公以下,至阼階,脫舄劍,升殿,席南奉贄珪璧畢,下殿,納舄佩劍,詣本位。主客即徙珪璧於東廂。帝興,入,徙御坐於西壁下,東向。設皇太子王公已下位。又奏中嚴,皇帝服通天冠,升御坐。王公上壽禮畢,食。食畢,樂伎奏。太宮進御酒,主書賦黃甘,逮二品已上。尚書騶騎引計吏,郡國各一人,皆跪受詔。侍中讀五條詔,計吏每應諾訖,令陳便宜者,聽詣白獸樽,以次還坐。宴樂罷,皇帝乘輿以入。皇太子朝,則遠游冠服,乘金輅,鹵簿以行。預會則劍履升坐。會訖,先興。天監六年
 詔曰:「頃代以來,元日朝畢,次會群臣,則移就西壁下,東向坐。求之古義,王者宴萬國,唯應南面,何更居東面?」於是御坐南向,以西方為上。皇太子以下,在北壁坐者,悉西邊東向。尚書令以下在南方坐者,悉東邊西向。舊元日御坐東向,酒壺在東壁下。御坐既南向,乃詔壺於南蘭下。又詔:「元日受五等贄,珪璧並量付所司。」周舍案:「《周禮》塚宰,大朝覲,贊玉幣。尚書,古之塚宰。頃王者不親撫玉,則不復須塚宰贊助。尋尚書主客曹郎,既塚宰隸職,今元日五等奠玉既竟,請以主客郎受。鄭玄注《覲禮》云:『既受之後,出付玉人於外。』漢時少府,職掌珪璧,請主客
 受玉,付少府掌。」帝從之。又尚書僕射沈約議:「《正會儀注》,御出,乘輿至太極殿前,納舄升階。尋路寢之設,本是人君居處,不容自敬宮室。



 案漢氏則乘小車升殿。請自今元正及大公事,御宜乘小輿至太極階,仍乘版輿升殿。」



 制:「可。」



 陳制,先元會十日,百官並習儀注,令僕已下,悉公服監之。設庭燎,街闕、城上、殿前皆嚴兵,百官各設部位而朝。宮人皆於東堂,隔綺疏而觀。宮門既無籍,外人但絳衣者,亦得入觀。是日,上事人發白獸樽。自餘亦多依梁禮
 云。



 後齊正日,侍中宣詔慰勞州郡國使。詔牘長一尺三寸,廣一尺,雌黃塗飾,上寫詔書三。計會日,侍中依儀勞郡國計吏,問刺史太守安不,及穀價麥苗善惡,人間疾苦。又班五條詔書於諸州郡國使人,寫以詔牘一枚,長二尺五寸,廣一尺三寸,亦以雌黃塗飾,上寫詔書。正會日,依儀宣示使人,歸以告刺史二千石。一曰,政在正身,在愛人,去殘賊,擇良吏,正決獄,平徭賦。二曰,人生在勤,勤則不匱,其勸率田桑,無或煩擾。三曰,六極之人,務加寬養,必使生有以自救,沒有以自給。四曰,長吏華浮,奉客以求小譽,逐末舍本,政之所疾,宜謹察之。五曰,人事意
 氣,干亂奉公,外內溷淆,綱紀不設,所宜糾劾。正會日,侍中黃門宣詔勞諸郡上計。勞訖付紙,遣陳土宜。字有脫誤者,呼起席後立。書跡濫劣者,飲墨水一升。文理孟浪無可取者,奪容刀及席。即而本曹郎中考其文跡才辭可聚者,錄牒吏部,簡同流外三品敘。元正大饗,百官一品已下,流外九品已上預會。一品已下、正三品已上、開國公侯伯、散品公侯及特命之官、下代刺史,並升殿。從三品已下、從九品以上及奉正使人比流官者,在階下。勛品已下端門外。



 隋制,正旦及冬至,文物充庭,皇帝出西房,即御座。皇太
 子鹵簿至顯陽門外,入賀。復詣皇后御殿,拜賀訖,還宮。皇太子朝訖,群官客使入就位,再拜。上公一人,詣西階,解劍,升賀;降階,帶劍,復位而拜。有司奏諸州表。群官在位者又拜而出。皇帝入東房,有司奏行事訖,乃出西房。坐定,群官入就位,上壽訖,上下俱拜。皇帝舉酒,上下舞蹈,三稱萬歲。皇太子預會,則設坐於御東南,西向。



 群臣上壽畢,入,解劍以升。會訖,先興。



 後齊元日,中宮朝會,陳樂,皇后褘衣乘輿,以出於昭陽殿。坐定,內外命婦拜,皇后興,妃主皆跪。皇后坐,妃主皆起,長公主一人,前跪拜賀。禮畢,皇后入室,乃移幄坐於
 西廂。皇后改服褕狄以出。坐定,公主一人上壽訖,就坐。御酒食,賜爵,並如外朝會。



 隋儀如後齊制,而又有皇后受群臣賀禮。則皇后御坐,而內侍受群臣拜以入,承令而出,群臣拜而罷。



 後齊皇太子月五朝。未明二刻,乘小輿出,為三師降。至承華門,升石山安車,三師軺車在前,三少在後,自雲龍門入。皇帝御殿前,設拜席位,至柏閣,齋帥引,洗馬、中庶子從。至殿前席南,北面再拜。



 天保元年,皇太子監國,在西林園冬會。群議皆東面。二年,於北城第內冬會,又議東面。吏部郎陸卬疑非禮,魏
 收改為西面。邢子才議欲依前,曰:凡禮有同者,不可令異。《詩》說,天子至於大夫,皆乘四馬,況以方面之少,何可皆不同乎?若太子定西面者,王公卿大夫士,復何面邪?南面人君正位,今一官之長,無不南面,太子聽政,亦南面坐。議者言皆晉舊事,太子在東宮西面,為避尊位,非為向臺殿也。子才以為東晉博議,依漢、魏之舊,太子普臣四海,不以為嫌,又何疑於東面?《禮》「世子絕旁親」,「世子冠於阼」,「塚子生,接以太牢」。漢元著令,太子絕馳道。此皆禮同於君。又晉王公世子,攝命臨國,乘七旒安車,駕用三馬,禮同三公。近宋太子乘象輅,皆有同處,不以為嫌。
 況東面者,君臣通禮,獨何為避?明為向臺,所以然也。近皇太子在西林園,在於殿猶且東面,於北城非宮殿之處,更不得邪?諸人以東面為尊,宴會須避。案《燕禮》、《燕義》,君位在東,賓位則在西,君位在阼階,故有《武王踐阼篇》,不在西也。《禮》「乘君之車,不敢曠左」。君在,惡空其位,左亦在東,不在西也。「君在阼,夫人在房」。鄭注「人君尊東也」。前代及今,皇帝宴會接客,亦東堂西面。若以東面為貴,皇太子以儲后之禮,監國之重,別第宴臣賓,自得申其正位。禮者皆東宮臣屬,公卿接宴,觀禮而已。若以西面為卑,實是君之正位。太公不肯北面說《丹書》,西面則道之,
 西面乃尊也。君位南面,有東有西,何可皆避?且事雖少異,有可相比者。周公,臣也,太子,子也。周公為塚宰,太子為儲貳。明堂尊於別第,朝諸侯重於宴臣賓,南面貴於東面。臣疏於子,塚宰輕於儲貳。周公攝政,得在明堂南面朝諸侯。今太子監國,不得於別第異宮東面宴客,情所未安。且君行以太子監國,君宴不以公卿為賓,明父子無嫌,君臣有嫌。案《儀注》,親王受詔冠婚,皇子皇女皆東面。今不約王公南面,而獨約太子,何所取邪?議者南尊改就西面,轉君位,更非合禮。方面既少,難為節文。東西二面,君臣通用,太子宜然,於禮為允。



 魏收議云:去天
 保初,皇太子監國。冬會群官於西園都亭,坐從東面,義取於向中宮臺殿故也。二年於宮冬會,坐乃東面,收竊以為疑。前者遂有別議,議者同之。邢尚書以前定東面之議,復申本懷,此乃國之大禮,無容不盡所見。收以為太子東宮,位在於震,長子之義也。案《易》八卦,正位向中。皇太子今居北城,於宮殿為東北,南面而坐,於義為背也。前者立議,據東宮為本。又案《東宮舊事》,太子宴會,多以西面為禮,此又成證,非徒言也。不言太子常無東南二面之坐,但用之有所。



 至如西園東面,所不疑也。未知君臣車服有同異之議,何為而發?就如所云,但知禮有
 同者,不可令異。不知禮有異者,不可令同。茍別君臣同異之禮,恐重紙累札,書不盡也。



 子才竟執東面,收執西面,授引經據,大相往復。其後竟從西面為定。時議又疑宮吏之姓與太子名同。子才又謂曰:「案《曲禮》『大夫士之子,不與世子同名。』《鄭注》云:『若先之生,亦不改。』漢法,天子登位,布名於天下,四海之內,無不咸避。案《春秋經》『衛石惡出奔晉』,在衛侯衎卒之前。衎卒,其子惡始立。



 明石惡於長子同名。諸侯長子,在一國之內,與皇太子於天子,禮亦不異。鄭言先生不改,蓋以此義。衛石惡、宋向戌皆與君同名,《春秋》不譏。皇太子雖有儲貳之重,未為海內
 所避,何容便改人姓。然事有消息,不得皆同於古。宮吏至微,而有所犯,朝夕從事,亦是難安,宜聽出宮,尚書更補他職。」制曰:「可。」



 後周制,正之二日,皇太子南面,列軒懸,宮官朝賀。及開皇初,皇太子勇準故事,張樂受朝,宮臣及京官北面稱慶。高祖誚之。是後定儀注,西面而坐,唯宮臣稱慶,臺官不復總集。煬帝之為太子,奏降章服,宮官請不稱臣。詔許之。後齊立春日,皇帝服通天冠、青介幘、青紗袍,佩蒼玉,青帶、青褲、青襪舄,而受朝於太極殿。尚書令等坐定,三公
 郎中詣席,跪讀時令訖,典御酌酒卮,置郎中前,郎中耑,還席伏飲,禮成而出。立夏、季夏、立秋讀令,則施御座於中楹,南向。



 立冬如立春,於西廂東向。各以其時之色服,儀並如春禮。



 後齊每策秀孝,中書策秀才,集書策考貢士,考功郎中策廉良,皇帝常服,乘輿出,坐於朝堂中楹。秀孝各以班草對。其有脫誤、書濫、孟浪者,起立席後,飲墨水,脫容刀。



 後齊宴宗室禮,皇帝常服,別殿西廂東向。七廟子孫皆公服,無官者,單衣介幘,集神武門。宗室尊卑,次於殿庭。七十者二人扶拜,八十者扶而不拜。升殿就位,皇帝興,
 宗室伏。皇帝坐,乃興拜而坐。尊者南面,卑者北面,皆以西為上。



 八十者一坐。再至,進絲竹之樂。三爵畢,宗室避席,待詔而後復位。乃行無算爵。



 正晦泛舟,則皇帝乘輿,鼓吹至行殿。升御坐,乘版輿,以與王公登舟,置酒。



 非預泛者,坐於便幕。



 仲春令辰,陳養老禮。先一日,三老五更齋於國學。皇帝進賢冠、玄紗袍,至璧雍,入總章堂。列宮懸。王公已下及國老庶老各定位。司徒以羽儀武賁安車,迎三老五更於國學。並進賢冠、玄服、黑舄、素帶。國子生黑介幘、青衿、單衣,乘馬從以至。皇帝釋劍,執珽,迎於門內。三老至門,
 五更去門十步,則降車以入。



 皇帝拜,三老五更攝齊答拜。皇帝揖進,三老在前,五更在後,升自右階,就筵。



 三老坐,五更立。皇帝升堂,北面。公卿升自左階,北面。三公授幾杖,卿正履,國老庶老各就位。皇帝拜三老,群臣皆拜。不拜五更。乃坐,皇帝西向,肅拜五更。



 進珍羞酒食,親袒割,執醬以饋,執爵以酳。以次進五更。又設酒酏於國老庶老。



 皇帝升御坐,三老乃論五孝六順,典訓大綱。皇帝虛躬請受,禮畢而還。又都下及外州人年七十已上,賜鳩杖黃帽。有敕即給,不為常也。



 後周保定三年,陳養老之禮。以太傅、燕國公於謹為三
 老。有司具禮擇日,高祖幸太學以食之。事見謹傳。



\end{pinyinscope}