\article{卷五十三列傳第十八 達奚長儒}

\begin{pinyinscope}

 達奚長儒,字富仁,代人也。祖俟,魏定州刺史。父慶,驃騎大將軍、儀同三司。長儒少懷節操,膽烈過人。十五襲爵樂安公。魏大統中,起家奉車都尉。周太祖引為親信,以質直恭勤,授大都督。數有戰功,假輔國將軍,累遷使持節、撫軍將軍、通直散騎常侍。平蜀之役,恆為先鋒,攻城野戰,所當必破之。除車騎大將軍、儀同三司,增邑三百
 戶。天和中,除渭南郡守,遷驃騎大將軍、開府儀同三司。



 從武帝平齊,遷上開府,進爵成安郡公,邑千二百戶,別封一子縣公。宣政元年,除左前軍勇猛中大夫。後與烏丸軌圍陳將吳明徹於呂梁,陳遣驍將劉景率勁勇七千來為聲援,軌令長儒逆拒之。長儒於是取車輪數百,系以大石,沉之清水,連轂相次,以待景軍。景至,船艦礙輪不得進,長儒乃縱奇兵,水陸俱發,大破之,俘數千人。及獲吳明徹,以功進位大將軍。尋授行軍總管,北巡沙塞,卒與虜遇,接戰,大破之。高祖作相,王謙舉兵於蜀,沙氐上柱國楊永安扇動利、興、武、文、沙、龍等六州以應謙,詔
 長儒擊破之。謙二子自京師亡歸其父,長儒並捕斬之。高祖受禪,進位上大將軍,封蘄春郡公,邑二千五百戶。



 開皇二年,突厥沙缽略可汗並弟葉護及潘那可汗眾十餘萬,寇掠而南,詔以長儒為行軍總管,率眾二千擊之。遇於周,眾寡不敵,軍中大懼,長儒慷慨,神色愈烈。為虜所沖突,散而復聚,且戰且行,轉鬥三日,五兵咸盡,士卒以拳毆之,手皆見骨,殺傷萬計,虜氣稍奪,於是解去。長儒身被五創,通中者二;其戰士死傷者十八九。突厥本欲大掠秦、隴,既逢長儒,兵皆力戰,虜意大沮,明日,於戰處焚尸慟哭而去。高祖下詔曰:「突厥猖狂,輒犯邊
 塞,犬羊之眾,彌亙山原。而長儒受任北鄙,式遏寇賊,所部之內,少將百倍,以晝通宵,四面抗敵,凡十有四戰,所向必摧。兇徒就戮,過半不反,鋒刃之餘,亡魂竄跡。自非英威奮發,奉國情深,撫御有方,士卒用命,豈能以少破眾,若斯之偉?言念勛庸,宜隆名器,可上柱國,餘勛回授一子。其戰亡將士,皆贈官三轉,子孫襲之。」



 其年,授寧州刺史,尋轉鄜州刺史,母憂去職。長儒性至孝,水漿不入口五日,毀悴過禮,殆將滅性,天子嘉嘆。起為夏州總管三州六鎮都將事,匈奴憚之,不敢窺塞。以病免。又除襄州總管,在職二年,轉蘭州總管。高祖遣涼州總管獨孤
 羅、原州總管元褒、靈州總管賀若誼等發卒備胡,皆受長儒節度。長儒率眾出祁連山北,西至蒲類海,無虜而還。復轉荊州總管三十六州諸軍事,高祖謂之曰:「江陵要害,國之南門,今以委公,聯無慮也。」歲餘,卒官。謚曰威。子暠,大業時,官至太僕少卿。



 賀婁子乾賀婁子干,字萬壽,本代人也。隨魏氏南遷,世居關右。祖道成,魏侍中、太子太傅。父景賢,右衛大將軍。子乾少以驍武知名。周武帝時,釋褐司水上士,稱為強濟。累遷小司水,以勤勞,封思安縣子,俄授使持節、儀同大將軍。大
 象初,領軍器監,尋除秦州刺史,進爵為伯。



 及尉迥作亂,子干與宇文司錄從韋孝寬討之。遇賊圍懷州,子干與宇文述等擊破之。高祖大悅,手書曰:「逆賊尉迥,敢遣蟻眾,作寇懷州。公受命誅討,應機蕩滌,聞以嗟贊,不易可言。丈夫富貴之秋,正在今日,善建功名,以副朝望也。」



 其後每戰先登,及破鄴城,與崔弘度逐迥至樓上。進位上開府,封武川縣公,邑三千戶,以思安縣伯別封子皎。



 開皇元年,進爵巨鹿郡公。其年,吐谷渾寇涼州,子乾以行軍總管從上柱國元諧擊之,功最優,詔褒美。高祖慮邊塞未安,即令子乾鎮涼川。明年,突厥寇蘭川,子幹率眾
 拒之,至可洛峐山,與賊相遇。賊眾甚盛,子幹阻川為營,賊軍不得水數日,人馬甚敝,縱擊,大破之。於是冊授子乾為上大將軍曰:「於戲!敬聽朕命。



 唯爾器量閑明,志情強果,任經武將,勤績有聞。往歲兇醜未寧,屢驚疆埸,拓土靜亂,殊有厥勞。是用崇茲賞典,加此車服,往欽哉!祗承榮冊,可不慎歟!」徵授營新都副監,尋拜工部尚書。其年,突厥復犯塞,以行軍總管從竇榮定擊之。子干別路破賊,斬首千餘級,高祖嘉之,遣通事舍人曹威齎優詔勞勉之。子干請入朝,詔令馳驛奉見。吐谷渾復寇邊,西方多被其害,命子幹討之。馳驛至河西,發五州兵,入掠
 其國,殺男女萬餘口,二旬而還。高祖以隴西頻被寇掠,甚患之。彼俗不設村塢,敕子干勒民為堡,營田積穀,以備不虞。子干上書曰:「比者兇寇侵擾,蕩滅之期,匪朝伊夕。伏願聖慮,勿以為懷。今臣在此,觀機而作,不得準詔行事。



 且隴西、河右,土曠民稀,邊境未寧,不可廣為田種。比見屯田之所,獲少費多,虛役人功,卒逢踐暴。屯田疏遠者,請皆廢省。但隴右之民以畜牧為事,若更屯聚,彌不獲安。只可嚴謹斥候,豈容集人聚畜。請要路之所,加其防守。但使鎮戍連接,烽候相望,民雖散居,必謂無慮。」高祖從之。俄而虜寇岷、洮二州,子干勒兵赴之,賊聞而
 遁去。



 高祖以子乾曉習邊事,授榆關總管十鎮諸軍事。歲餘,拜雲州刺史,甚為虜所憚。後數年,突厥雍虞閭遣使請降,並獻羊馬。詔以子乾為行軍總管,出西北道應接之。還拜雲州總管,以突厥所獻馬百匹、羊千口以賜之,乃下書曰:「自公守北門,風塵不警。突厥所獻,還以賜公。」母憂去職。朝廷以榆關重鎮,非子乾不可,尋起視事。十四年,以病卒官,時年六十。高祖傷惜者久之,賻縑千匹,米麥千斛,贈懷、魏等四州刺史,謚曰懷。子善柱嗣,官至黔安太守。



 子乾兄詮,亦有才器,位至銀青光祿大夫、鄯純深三州刺史、北地太守、東安郡公。



 史萬歲史萬歲,京兆杜陵人也。父靜,周滄州刺史。萬歲少英武,善騎射,驍捷若飛。



 好讀兵書,兼精占候。年十五,值周、齊戰於芒山,萬歲時從父入軍,旗鼓正相望,萬歲令左右趣治裝急去。俄而周師大敗,其父由是奇之。武帝時,釋褐侍伯上士。



 及平齊之役,其父戰沒,萬歲以忠臣子拜開府儀同三司,襲爵太平縣公。



 尉迥之亂也,萬歲從梁士彥擊之。軍次馮翊,見群雁飛來,萬歲謂士彥曰:「請射行中第三者。」既射之,應弦而落,三軍莫不悅服。及與迥軍相遇,每戰先登。鄴城之陣,官軍稍卻,萬歲謂左右曰:「
 事急矣,吾當破之。」於是馳馬奮擊,殺數十人,眾亦齊力,官軍乃振。及迥平,以功拜上大將軍。



 爾硃勛以謀反伏誅,萬歲頗相關涉,坐除名,配敦煌為戍卒。其戍主甚驍武,每單騎深入突厥中,掠取羊馬,輒大克獲。突厥無眾寡莫之敢當。其人深自矜負,數罵辱萬歲。萬歲患之,自言亦有武用。戍主試令馳射而工,戍主笑曰:「小人定可。」萬歲請弓馬,復掠突厥中,大得六畜而歸。戍主始善之,每與同行,輒入突厥數百里,名讋北夷。竇榮定之擊突厥也,萬歲詣轅門請自效。榮定數聞其名,見而大悅。因遣人謂突厥曰:「士卒何罪過,令殺之,但當各遣一壯士決
 勝負耳。」



 突厥許諾,因遣一騎挑戰。榮定遣萬歲出應之,萬歲馳斬其首而還。突厥大驚,不敢復戰,遂引軍而去。由是拜上儀同,領車騎將軍。平陳之役,又以功加上開府。



 及高智慧等作亂江南,以行軍總管從楊素擊之。萬歲率眾二千,自東陽別道而進,逾嶺越海,攻陷溪洞不可勝數。前後七百餘戰,轉鬥千餘里,寂無聲問者十旬,遠近皆以萬歲為沒。萬歲以水陸阻絕,信使不通,乃置書竹筒中,浮之於水。汲者得之,以言於素。素大悅,上其事。高祖嗟嘆,賜其家錢十萬,還拜左領軍將軍。



 先是,南寧夷爨來降,拜昆州刺史,既而復叛。遂以萬歲為
 行軍總管,率眾擊之。



 入自蜻蛉川,經弄凍,次小勃弄、大勃弄,至於南中。賊前後屯據要害,萬歲皆擊破之。行數百里,見諸葛亮紀功碑,銘其背曰:「萬歲之後,勝我者過此。」萬歲令左右倒其碑而進。渡西二河,入渠濫川,行千餘里,破其三十餘部,虜獲男女二萬餘口。諸夷大懼,遣使請降,獻明珠徑寸。於是勒石頌美隋德。萬歲遣使馳奏,請將入朝,詔許之。爨玩陰有二心,不欲詣闕,因賂萬歲以金寶,萬歲於是舍玩而還。蜀王時在益州,知其受賂,遣使將索之。萬歲聞而悉以所得金寶沉之於江,索無所獲。以功進位柱國。晉王廣虛衿敬之,待以交友
 之禮。上知為所善,令萬歲督晉府軍事。明年,爨玩復反,蜀王秀奏萬歲受賂縱賊,致生邊患,無大臣節。上令窮治其事,事皆驗,罪當死。上數之曰:「受金放賊,重勞士馬。朕念將士暴露,寢不安席,食不甘味,卿豈社稷臣也?」萬歲曰:「臣留爨玩者,恐其州有變,留以鎮撫。臣還至瀘水,詔書方到,由是不將入朝,實不受賂。」上以萬歲心有欺隱,大怒曰:「朕以卿為好人,何乃官高祿重,翻為國賊也?」顧有司曰:「明日將斬之。」萬歲懼而服罪,頓首請命。左僕射高熲、左衛大將軍元旻等進曰:「史萬歲雄略過人,每行兵用師之處,未嘗不身先士卒,尤善撫御,將士樂為
 致力,雖古名將未能過也。」上意少解,於是除名為民。歲餘,復官爵。尋拜河州刺史,復領行軍總管以備胡。



 開皇末,突厥達頭可汗犯塞,上令晉王廣及楊素出靈武道,漢王諒與萬歲出馬邑道。萬歲率柱國張定和、大將軍李藥王、楊義臣等出塞,至大斤山,與虜相遇。



 達頭遣使問曰:「隋將為誰?」候騎報:「史萬歲也。」突厥復問曰:「得非敦煌戍卒乎?」候騎曰:「是也。」達頭聞之,懼而引去。萬歲馳追百餘里乃及,擊大破之,斬數千級,逐北入磧數百里,虜遁逃而還。楊素害其功,因譖萬歲云:「突厥本降,初不為寇,來於塞上畜牧耳。」遂寢其功。萬歲數抗表陳狀,上未
 之悟。



 會上從仁壽宮初還京師,廢皇太子,窮東宮黨與。上問萬歲所在,萬歲實在朝堂,楊素見上方怒,因曰:「萬歲謁東宮矣。」以激怒上。上謂為信然,令召萬歲,時所將士卒在朝稱冤者數百人,萬歲謂之曰:「吾今日為汝極言於上,事當決矣。」



 既見上,言將士有功,為朝廷所抑,詞氣憤厲,忤於上。上大怒,令左右暴殺之。



 既而悔,追之不及,因下詔罪萬歲曰:「柱國、太平公萬歲,拔擢委任,每總戎機。



 往以南寧逆亂,令其出討。而昆州刺史爨玩包藏逆心,為民興患。朕備有成敕,令將入朝。萬歲乃多受金銀,違敕令住,致爨玩尋為反逆,更勞師旅,方始平定。所
 司檢校,罪合極刑,舍過念功,恕其性命,年月未久,即復本官。近復總戎,進討蕃裔。突厥達頭可汗領其兇眾,欲相拒抗,既見軍威,便即奔退,兵不血刃,賊徒瓦解。如此稱捷,國家盛事,朕欲成其勛庸,復加褒賞。而萬歲、定和通簿之日,乃懷奸詐,妄稱逆面交兵,不以實陳,懷反覆之方,弄國家之法。若竭誠立節,心無虛罔者,乃為良將,至如萬歲,懷詐要功,便是國賊,朝憲難虧,不可再舍。」



 死之日,天下士庶聞者,識與不識,莫不冤惜。



 萬歲為將,不治營伍,令士卒各隨所安,無警夜之備,虜亦不敢犯。臨陣對敵,應變無方,號為良將。有子懷義。



 劉方馮昱王璟李充楊武通陳永貴房兆劉方,京兆長安人也。性剛決,有膽氣。仕周承御上士,尋以戰功拜上儀同。



 高祖為丞相,方從韋孝寬破尉迥於相州,以功加開府,賜爵河陰縣侯,邑八百戶。



 高祖受禪,進爵為公。開皇三年,從衛王爽破突厥於白道,進位大將軍。其後歷甘、瓜二州刺史,尚未知名。仁壽中,會交州俚人李佛子作亂,據越王故城,遣其兄子大權據龍編城,其別帥李普鼎據烏延城。左僕射楊素言方有將帥之略,上於是詔方為交州道行軍總管,以度支侍郎敬德亮為長史,統二十七營而進。方法令嚴肅,軍容齊整,
 有犯禁者,造次斬之。然仁而愛士,有疾病者,親自撫養。長史敬德亮從軍至尹州,疾甚,不能進,留之州館。分別之際,方哀其危篤,流涕嗚咽,感動行路。其有威惠如此,論者稱為良將。至都隆嶺,遇賊二千餘人來犯官軍,方遣營主宋纂、何貴、嚴願等擊破之。進兵臨佛子,先令人諭以禍福,佛子懼而降,送於京師。其有桀黠者,恐於後為亂,皆斬之。



 尋授歡州道行軍總管,以尚書右丞李綱為司馬,經略林邑。方遣欽州刺史寧長真、歡州刺史李暈、上開府秦雄以步騎出越常,方親率大將軍張愻、司馬李綱舟師趣比景。高祖崩,煬帝即位,大業元年正月,
 軍至海口。林邑王梵志遣兵守險,方擊走之。師次闍黎江,賊據南岸立柵,方盛陳旗幟,擊金鼓,賊懼而潰。既渡江,行三十里,賊乘巨象,四面而至。方以弩射象,象中創,卻蹂其陣,王師力戰,賊奔於柵,因攻破之,俘馘萬計。於是濟區粟,度六里,前後逢賊,每戰必擒。進至大緣江,賊據險為柵,又擊破之。逕馬援銅柱,南行八日,至其國都。林邑王梵志棄城奔海,獲其廟主金人,污其宮室,刻石紀功而還。士卒腳腫,死者十四五。方在道遇患而卒,帝甚傷惜之,乃下詔曰:「方肅承廟略,恭行天討,飲冰湍邁,視險若夷。摧鋒直指,出其不意,鯨鯢盡殪,巢穴咸傾,役
 不再勞,肅清海外。致身王事,誠績可嘉,可贈上柱國、盧國公。」子通仁嗣。



 開皇時,有馮昱、王厓、李充、楊武通、陳永貴、房兆,俱為邊將,名顯當時。



 昱、厓,並不知何許人也。昱多權略,有武藝。高祖初為丞相,以行軍總管與王誼、李威等討叛蠻,平之,拜柱國。開皇初,又以行軍總管屯乙弗泊以備胡。突厥數萬騎來掩之,昱力戰累日,眾寡不敵,竟為虜所敗,亡失數千人,殺虜亦過當。其後備邊數年,每戰常大克捷。勇善射,高祖以其有將帥才,每以行軍總管屯兵江北,御陳寇。數有戰功,為陳人所憚。伐陳之役,及高智慧反,攻討皆有殊績。官至柱國、白水
 郡公。充,隴西成紀人也。少慷慨,有英略。開皇中,頻以行軍總管擊突厥有功,官至上柱國、武陽郡公、拜朔州總管,甚有威名,為虜所憚。後有人譖其謀反,徵還京師,上譴怒之。充性素剛,遂憂憤而卒。武通,弘農華陰人,性果烈,善馳射。數以行軍總管討西南夷,每有功,封白水郡公,拜左武衛大將軍。時黨項羌屢為邊患,朝廷以其有威名,歷岷、蘭二州總管以鎮之。後與周法尚討嘉州叛獠,法尚軍初不利,武通率數千人,為賊斷其歸路。武通於是束馬懸車,出賊不意,頻戰破之。賊知其孤軍無援,傾部落而至。武通轉鬥數百里,為賊所拒,四面路絕。



 武
 通輕騎接戰,墜馬,為賊所執,殺而啖之。永貴,隴右胡人也,本姓白氏,以勇烈知名。高祖甚親愛之,數以行軍總管鎮邊,每戰必單騎陷陣。官至柱國、蘭利二州總管,封北陳郡公。兆,代人也,本姓屋引氏,剛毅有武略。頻為行軍總管擊胡,以功官至柱國、徐州總管。並史失其事。



 史臣曰:長儒等結發從戎,俱有驍雄之略;總統師旅,各擅禦侮之功。長儒以步卒二千抗十萬之虜,師殲矢盡,勇氣彌歷,壯哉!子乾西涉青海,北臨玄塞,胡夷懾憚,烽候無警,亦有可稱。萬歲實懷智勇,善撫士卒,人皆樂死,師不疲勞。



 北卻匈奴,南平夷、獠,兵鋒所指,威驚絕域。論
 功杖氣,犯忤貴臣,偏聽生奸,死非其罪,人皆痛惜,有李廣之風焉。劉方號令無私,治軍嚴肅,克剪林邑,遂清南海,徼外百蠻,無思不服。凡此諸將,志烈過人,出當推轂之重,入受爪牙之寄,雖馬伏波之威行南裔,趙充國之聲動西羌,語事論功,各一時也。



\end{pinyinscope}