\article{卷十八志第十三 律歷下}

\begin{pinyinscope}

 開皇二十年,
 袁充奏日長影短,高祖因以歷事付皇太子,遣更研詳著日長之候。太子徵天下歷算之士,咸集於東宮。劉焯以太子新立,復增修其書,名曰《皇極歷》,駁正胄玄之短。太子頗嘉之,未獲考驗。焯為太學博士,負其精博,志解胄玄之印,官不滿意,又稱疾罷歸。至仁壽四年,焯言胄玄之誤於皇太子:其一曰,張胄玄所上見
 行歷,日月交食,星度見留,雖未盡善,得其大較,官至五品,誠無所愧。



 但因人成事,非其實錄,就而討論,違舛甚眾。



 其二曰,胄玄弦望晦朔,違古且疏,氣節閏候,乖天爽命。時不從子半,晨前別為後日。日躔莫悟緩急,月逡妄為兩種,月度之轉,輒遺盈縮,交會之際,意造氣差。七曜之行,不循其道,月星之度,行無出入,應黃反赤,當近更遠,虧食乖準,陰陽無法。星端不協,珠璧不同,盈縮失倫,行度愆序。



 去極晷漏,應有而無,食分先後,彌為煩碎。測今不審,考古莫通,立術之疏,不可紀極。今隨事糾駁,凡五百三十六條。



 其三曰,胄玄以開皇五年,與李文琮於
 張賓歷行之後,本州貢舉,即齎所造歷擬以上應。其歷在鄉陽流布,散寫甚多,今所見行,與焯前歷不異。玄前擬獻,年將六十,非是忽迫倉卒始為,何故至京未幾,即變同焯歷,與舊懸殊?焯作於前,玄獻於後,舍己從人,異同暗會。且孝孫因焯,胄玄後附孝孫,歷術之文,又皆是孝孫所作,則元本偷竊,事甚分明。恐胄玄推諱,故依前歷為駁,凡七十五條,並前歷本俱上。



 其四曰,玄為史官,自奏虧食,前後所上,多與歷違,今算其乖舛有一十三事。又前與太史令劉暉等校其疏密五十四事,云五十三條新。計後為歷應密於舊,見用算推,更疏於本。今糾
 發並前,凡四十四條。



 其五曰,胄玄於歷,未為精通。然孝孫初造,皆有意,徵天推步,事必出生,不是空文,徒為臆斷。



 其六曰,焯以開皇三年,奉敕修造,顧循記注,自許精微,秦漢以來,無所與讓。尋聖人之跡,悟曩哲之心,測七曜之行,得三光之度,正諸氣朔,成一歷象,會通今古,符允經傳,稽於庶類,信而有徵。胄玄所違,焯法皆合,胄玄所闕,今則盡有,隱括始終,謂為總備。



 仍上啟曰:「自木鐸寢聲,緒言成燼,群生蕩析,諸夏沸騰,曲技雲浮,疇官雨絕,歷紀廢壞,千百年矣。焯以庸鄙,謬荷甄擢,專精藝業,耽玩數象,自力群儒之下,冀睹聖人之意。開皇之初,奉
 敕修撰,性不諧物,功不克終,猶被胄玄竊為己法,未能盡妙,協時多爽,尸官亂日,實玷皇猷。請徵胄玄答,驗其長短。」



 焯又造歷家同異,名曰《稽極》。大業元年,著作郎王邵、諸葛潁二人,因入侍宴,言劉焯善歷,推步精審,證引陽明。帝曰:「知之久矣。」仍下其書與胄玄參校。胄玄駁難云:「焯歷有歲率、月率,而立定朔,月有三大、三小。案歲率、月率者,平朔之章歲、章月也。以平朔之率而求定朔,值三小者,猶以減三五為十四;值三大者,增三五為十六也。校其理實,並非十五之正。故張衡及何承天創有此意,為難者執數以校其率,率皆自敗,故不克成。今焯為
 定朔,則須除其平率,然後為可。」互相駁難,是非不決,焯又罷歸。



 四年,駕幸汾陽宮,太史奏曰:「日食無效。」帝召焯,欲行其歷。袁允方幸於帝,左右胄玄,共排焯歷,又會焯死,歷竟不行。術士咸稱其妙,故錄其術云。甲子元,距大隋仁壽四年甲子積一百萬八千八百四十算。



 歲率,六百七十六。



 月率,八千三百六十一。



 朔日法,千二百四十二。



 朔實,三萬六千六百七十七。



 旬周,六十。



 朔辰,百三半。



 日乾元,五十二。



 日限,十一。



 盈泛,十六。



 虧總,十七。



 推經朔術:置入元距所求年,月率乘之,如歲率而一,為積月,不滿為閏衰。朔實乘積月,滿朔日法得一,為積日,不滿為朔餘。旬周去積日,不盡為日,即所求年天正經朔日及餘。



 求上下弦、望:加經朔日七、餘四百七十五小,即上弦經日及餘。又加得望、下弦及後月朔。就徑求望者,加日十四、餘九百五十半;下弦加日二十二、餘百八十三大;後月朔加日二十九,餘六百五十九。每月加閏衰二十大,即各其月閏衰也。



 凡月建子為天正,建丑為地正,建寅為人正。即以人正為正月,統求所起,本於天正。若建歲歷從正月始,氣、候、月、星,所值節度,雖有前卻,並亦隨之。其前地正為十二月,天正為十一月,並諸氣度皆屬往年。其日之初,亦從星起,晨前多少,俱歸昨日。若氣在夜半之後,量影以後日為正。諸因加者,各以其餘減法,殘者為全
 餘。若所因之餘滿全餘以上,皆增全一而加之,減其全餘;即因餘少於全餘者,不增全加,皆得所求。分度亦爾。凡日不全為餘,積以成餘者曰秒;度不全為分,積以成分者曰篾;其有不成秒曰麼,不成篾曰么。其分、餘、秒、篾,皆一為小,二為半,三為大,四為全,加滿全者從一。其三分者,一為少,二為太。若加者,秒篾成法,從分餘。分餘滿法從日度一,日度有所滿,則從去之。而日命以日辰者,滿旬周則亦除;命有連分、餘、秒、篾者,亦隨全而從去。其日度雖滿,而分秒不滿者,未可從去,仍依本數。若減者,秒篾不足,減分餘一,加法而減之;分餘不足減者,加所從去或
 前日度乃減之。即其名有總,而日度全及分餘共者,須相加除,當皆連全及分餘共加除之。若須相乘,有分餘者,母必通全內子,乘訖報除。或分餘相並,母不同者,子乘而並之。母相乘為法,其並,滿法從一為全,此即齊同之也。既除為分餘而有不成,若例有秒篾,法乘而又法除,得秒篾數。已為秒篾及正有分餘,而所不成不復須者,須過半從一,無半棄之。若分餘其母不等,須變相通,以彼所法之母乘此分餘,而此母除之,得彼所須之子。所有秒篾者,亦法乘,不滿此母,又除而得其數。麼么亦然。其所除去而有不盡全,則謂之不盡,亦曰不如。其
 不成全,全乃為不滿分、餘、秒、篾,更曰不成。凡以數相減,而有小及半、太須相加減,同於分餘法者,皆以其母三四除其氣度日法,以半及太、大本率二三乘之,少、小即須因所除之數隨其分餘而加減焉。秋分後春分前為盈泛,春分後秋分前為虧總,須取其數。泛總為名,指用其時,春分為主,虧日分後,盈日分前。凡所不見,皆放於此。



 氣日法,四萬六千六百四十四。



 歲數,千七百三萬六千四百六十六半。



 度準,三百三十八。



 約率,九。



 氣辰,三千八百八十七。



 餘通,八百九十七。



 秒法,四十八。



 麼法,五。



 推氣術:半閏衰乘朔實,又度準乘朔餘,加之,如約率而一,所得滿氣日法為去經朔日,不滿為氣餘。以去經朔日,即天正月冬至恆日定餘,乃加夜數之半者,減日一,滿者因前,皆為定日。命日甲子算外,即定冬至日。其餘如半氣
 辰千九百四十三半以下者,為氣加子半後也;過以上,先加此數,乃氣辰而一,命以辰算外,即氣所在辰。十二辰外,為子初以後餘也。又十二乘辰餘:四為小太,亦曰少;五為半步;六為半;七為半太;八為大少,亦曰太;九為太;十為大太;十一為窮辰少。



 其又不成法者,半以上為進,以下為退。退以配前為強,進以配後為弱。即初不成一而有退者,謂之沾辰;初成
 十一而有進者,謂之窮辰。未旦其名有重者,則於間可以加之,命辰通用其餘,辨日分辰而判諸日。因別亦皆準此。因冬至有減日者,還加之。每加日十五、餘萬一百九十二、秒三十七,即各次氣恆日及餘。諸月齊其閏衰,如求冬至法,亦即其月中氣恆日去經朔數。其求後月節氣恆日,如次之求前節者減之。



 推每日遲速數術:見求所在氣陟降率,並後氣率半之,以日限乘而泛總除,得氣末率。又日限乘二率相減之殘,泛總除,為總差。其總差亦日限乘而泛總除,為別差。率前少者,以總差減末率,為初率,乃別差加之;前多者,即以總差加末率,皆為氣初日陟
 降
 數。以別差前多者日減,前少者日加初數,得每日數。所歷推定氣日隨算其數,陟加、降減其遲速,為各遲速數。其後氣無同率及有數同者,皆因前末,以末數為初率,加總差為末率,及差漸加初率,為每日數,通計其秒,調而御之。



 求月朔弦望應平會日所入
 遲速:各置其經餘為辰,以入氣辰減之,乃日限乘日,日內辰為入限,以乘其氣前多之末率,前少之初率,日限而一,為總率。其前多者,入限減泛總之殘,乘總差,泛總而一,為入差,並於總差,入限乘,倍日限除,加以總率;前少者,入限自乘再乘別差,日限自乘,倍而除,亦加總率,皆為總數。乃以陟加、降減其氣遲速數為定,即速加、遲減其經餘,各其月平會日所入遲速定日及餘。



 求每日所入先後:各置其氣躔衰與衰總,皆以餘通乘之,所乃躔衰如陟降率;衰總如遲速數,亦如求遲速法,即得每所入先後及定數。



 求定氣:其每日所入先後數即為氣餘,其所歷日皆以先加之,以後減之,隨算其日,通準其餘,滿一恆氣,即為二至後一氣之數。以加二氣,如法用別其日而命之。又算其次,每相加命,各得其定氣日及餘也。亦以其先後已通者,先減後加其恆氣,即次氣定日及餘。亦因別其日,命以甲子,各得所求。



 求土王:距四立各四氣外所入先後加減,滿二十二日、餘八千一百五十四、秒十、麼二。除所滿日外,即土始王日。



 求侯日:定氣即初候日也。三除恆氣,各為平候日。餘亦以所入先後數為氣餘,所歷之日皆以先加、後減,隨計
 其日,通準其餘,每滿其平,以加氣日而命之,即得次候日。亦算其次,每相加命,又得末候及次氣日。



 倍夜半之漏,得夜刻也。以減百刻,不盡為晝刻。每減晝刻五,以加夜刻,即其晝為日見、夜為不見刻數。刻分以百為母。



 求日出入辰刻:十二除百刻,得辰刻數,為法。半
 不見刻以
 半辰加之,為日出實,又加日出見刻,為日入實。如法而一,命子算外,即所在辰,不滿法,為刻及分。



 求辰前餘數:氣、朔日法乘夜半刻,百而一,即其餘也。



 求每日刻差:每氣準為十五日,全刻二百二十五為法。其二至各前後於二分,而數因相加減,間皆六氣;各盡於四立,為三氣。至與前日為一,乃每日增太;又各二氣,每日增少;其末之氣,每日增少之小,而末六日,不加而裁焉。二望至前後一氣之末日,終於十少;二氣初日,稍
 增為十二半,終於二十太,三氣初日,二十一,終於三十少;四立初日,三十一,終於三十五太;五氣亦少增,初日三十六太,終四十一少;末氣初日,四十一少,終於四十二。每氣前後累算其數,又百八十乘為實,各泛總乘法而除,得其刻差。隨而加減夜刻而半之,各得入氣夜定刻。其分後十五日外,累算盡日,乃副置之,百八十乘,虧總除,為其所因數。以減上位,不盡為所加也。不全日者,隨辰率之。



 求晨去中星:加周度一,各昏去中星減之,不盡為晨去度。



 求每日度差:準日因增加裁,累算所得,百四十三之,四百而一,亦百八十乘,泛總除,為度差數。



 滿轉法為度,隨日加減,各得所求。分後氣間,亦求準外與前求刻,至前加減,皆因日數逆算求之。亦可因至向背其刻,冬減夏加,而度冬加夏減。若至前,以入氣減氣間,不盡者,因後氣而反之,以不盡日累算乘除所定,從後氣而逆以加減,皆得其數。此但略校其總,若精存於《稽極》云。



 轉終日,二十七;餘,千二百五十五。



 終法,二千二百六十三。



 終實,六萬二千三百五十六。



 終全餘,千八。



 轉法,五十二。



 篾法,八百九十七。



 閏限,六百七十六。



 推入轉術:終實去積日,不盡,以終法乘而又去,不如終實者,滿終法得一日,滿為餘,即其年天正經朔夜半入轉日及餘。



 求次日:加一日,每日滿轉終則去之,其二十八日者加全餘為夜半入初日餘。



 求弦望:皆因朔加其經日,各得夜半所入日餘。



 求次月:加大月二日,小月一日,皆及全餘,亦其夜半所入。



 求經辰所入朔弦望:經餘變從轉,不成為秒,加其夜半所入,皆其辰入日及餘。因朔辰所入,每加日七、餘八百六十五、秒千一百六十大,秒滿日法成餘,亦得上弦。望、下弦、次朔經辰所入徑求者,加望日十四、餘千七百三十一、秒千七十九半,下弦日二十二、餘三百三十四、秒九百九十八小,次朔日一、餘二千二百八、秒九百一十七。亦朔望各增日一,減其全餘,望五百三十一、秒百六十二半,朔五十四、秒三百二十五。



 求月平應會日所入:以月朔弦望會日所入遲速定數,亦變從轉餘,乃速加、遲減其經辰所入餘,即各平會所入日餘。



 推朔弦望定日術:各以月平會所入之日加減限,限並後限而半之,為通率;又二限相減,為限衰。前多者,以入餘減終法,殘乘限衰,終法而一,並於限衰而半之;前少者,半入餘乘限衰,亦終法而一,減限衰。皆加通率,入餘乘之,日法而一,所得為平會加減限數。其限數又別從轉餘為變餘,朓減、朒加本入餘。限前多者,朓以減與未減,朒以加與未加,皆減終法,並而半之,以乘限衰;前少者,亦朓朒各並二入餘,半之,以乘限衰;皆終法而一,加於通率,變餘乘之,
 日法而一。所得以朓減、朒加限數,加減朓朒積而定朓朒。



 乃朓減、朒加其平會日所入餘,滿若不足進退之,即朔弦望定日及餘。不滿晨前數者,借減日算,命甲子算外,各其日也。不減與減,朔日立算與後月同。若俱無立算者,月大,其定朔算後加所借減算。閏衰限滿閏限,定朔無中氣者為閏,滿之前後,在分前若近春分後、秋分前,而或月有二中者,皆量置其朔,不必依定。其後無同限者,亦因前多以通率數為半衰而減之,二前少,即為通率。其加減變餘進退日者,分為一日,隨餘初末如法求之,所得並以加減限數。凡分餘秒篾,事非因舊,文不著母者,皆十為法。若法當求數,用相加減,而更不過通遠,率少數微者,則不須算。其入七日餘二千一十一,十四日餘千七百五十九,
 二十一日餘千五百七,二十八日始終餘以下為初數,各減終法以上為末數。其初末數皆加減相返,其要各為九分,初則七日八分,十四日七分,二十一日六分,二十八日五分;末則七日一分,十四日二分,二十一日三分,二十八日四分。雖初稍弱而末微強,餘差止一,理勢兼舉,皆今有轉差,各隨其數。若恆算所求,七日與二十一日得初衰數,而末初加隱而不顯,且數與平行正等。亦初末有數而恆算所無,其十四日、二十八日既初末數存,而虛衰亦顯,其數當去,恆法不見。



 求朔弦望之辰所加:定餘半朔辰五十一大以下,為加子過;以上,加此數,乃朔辰而一,亦命以子,十二算外,又加子初。



 以後其求入辰強弱,如氣。



 求入辰法度:度法,四萬六千六百四十四。



 周數,千七百三萬七千七十六。



 周分,萬二千一十六。



 轉,十三。



 篾,三百五十五。



 周差,六百九半。



 在日謂之餘通,在度謂之篾法,亦氣為日法、為度法,隨事名異,其數本同。女末接虛,謂之周分。



 變周從轉,謂之轉。晨昏所距日在黃道中,準度赤道計之。



 斗二十六牛八女十二虛十危十七室十六壁九北方玄武七宿,九十八度。



 奎十六婁十二胃十四昴十一畢十六觜二參九西方白虎七宿,八十度。



 井三十三鬼四柳十五星七張十八翼十八軫十七南方硃雀七宿,百一十二度。



 角十二亢九氐十
 五房五心五尾十八箕十一東方蒼龍七宿,七十五度。



 前皆赤道度,其數常定,紘帶天中,儀極攸準。



 推黃道術:準冬至所在為赤道度,後於赤道四度為限。初數九十七,每限增一,以終百七。其三度少弱,平。乃初限百九,亦每限增一,終百一十九,春分所在。因百一十九每限損一,又終百九。亦三度少弱,平。乃初限百七,每限損一,終九十七,夏至所在。又加冬至後法,得秋分、冬至所在數。各
 以數乘其限度,百八而一,累而總之,即皆黃道度也。度有分者,前後輩之,宿有前卻,度亦依體,數逐差遷,道不常定,準令為度,見步天行,歲久差多,隨術而變。



 斗二
 十四牛七女十一
 半
 虛十危十七室十七壁十北方九十六度半。



 奎十七婁十三胃十五昴十
 一畢十五
 半觜
 二參九西方八十二度半。



 井三十鬼四柳十四半星七張十七翼十九軫十八南方一百九度半。



 角十三亢十氐十六房五心五尾十七箕十半東方七十六度半。



 前皆黃道度,步日所行。月與五星出入,循此。



 推月道所行度術:準交定前後所在度半之,亦於赤道四度為限,初十一,每限損一,以終於一。其三度強,平。乃初限數一,每限增一,亦終十一,為交所在。即因十一,每限損一,以終於一。亦三度強,平。又初限數一,每限增一,終於十一,復至交半,返前表裏。仍因十一增損,如道得後交及交半數。各積其數,百八十而一,即道所行每與黃道差數。其月在表,半後交前,損減增加;交後半前,損加增減於黃道。其月在裏,各返之,即得月道所行度。其限未盡四度,以所直
 行數乖入度,四而一。若月在黃道度,增損於黃道之表裏,不正當於其極,可每日準去黃道度,增損於黃道,而計去赤道之遠近,準上黃道之率以求之,遁伏相消,朓朒互補,則可知也。積交差多,隨交為正。其五星先候,在月表裏出入之漸,又格以黃儀,準求其限。若不可推明者,依黃道命度。



 推日度術:置入元距所求年歲數乘之,為積實,周數去之,不盡者,滿度法得積度,不滿為分。以冬至餘減分;命積度以黃道起於虛一宿次除之,不滿宿算外,即所求年天正冬
 至夜半日所在度及分。



 求年天正定朔度:以定朔日至冬至每日所入先後餘為分,日為度,加分以減冬至度,即天正定朔夜半日在所度分。亦去朔日乘衰總已通者,以至前定氣除之,又如上求差加以並去朔日乃減度,亦即天正定朔日所在度。皆日為度,餘為分。其所入先後及衰總用增損者,皆分前增、分後損其平日之度。



 求次日:每日所入先後分增損度,以加定朔度,得夜半。



 求弦望:去定朔每日所入分,累而增損去定朔日,乃加定朔度,亦得其夜半。



 求次月:歷算大月三十日,小月二十九日,每日所入先後分增損其月,以加前朔度,即各夜半所在至虛去周分。



 求朔弦望辰所加:各以度準乘定餘,約率而一,為平分。又定餘乘其日所入先後分,日法而一,乃增損其平分,以加其夜半,即各辰所加。其分皆篾法約之,為轉分,不成為篾。凡朔辰所
 加者,皆為合朔日月同度。



 推月而與日同度術:各以朔平會加減限數加減朓朒,為平會朓朒。以加減定朔,度準乘,約率除,以加減定朔辰所加日度,即平會辰日所在。又平會餘乘度準,約率除,減其辰所在,為平會夜半日所在。乃以四百六十四半乘平會餘,亦以周差乘,朔實除,從之,以減夜半日所在,即月平會夜半所在。三十七半乘平會餘,增其所減,以加減半,得月平會辰平行度。五百二乘朓棵,亦以周差乘,朔實除而從之,朓減、朒加其平行,即月定朔辰所在度,而與日同。若即
 以平會朓朒所得分加減平會辰所在,亦得同度。



 求月弦望定辰度:各置其弦望辰所加日度及分,加上弦度九十一,轉分十六,篾三百一十三;望度百八十二,轉分三十二,篾六百二十六;下弦度二百七十三,轉分四十九,篾四十二,皆至虛,去轉周求之。



 定朔夜半入轉:經朔夜半所入準於定朔日有增損者,亦以一日加減之,否者因經朔為定。



 其因定求朔次日、弦望、次月夜半者,如於經月法為之。



 推月轉日定分術:以夜半入轉餘乘逡差,終法而一,為見差。以息加、消減其日逡分,為月每日所行逡定分。



 求次日:各以逡定分加轉分,滿轉法從度,皆其夜半。因日轉若各加定日,皆得朔、弦望夜半月所在定度。其就辰加以求夜半,各以半逡差減逡分,消者,定餘乘差,終法除,並差而半之;息者,半定餘以乘差,終法而一。皆加所減,乃以定餘乘之,日法而一,各減辰所加度,亦得其夜半度。因夜半亦如此求逡分,以加之,亦得辰所加度。諸轉可初以逡分及
 差為篾,而求其次,皆訖,乃除為轉分。因經朔夜半求定辰度者,以定辰去經朔夜半減,而求其增損數,乃以數求逡定分,加減其夜半,亦各定辰度。



 求月晨昏度:如前氣與所求每日夜漏之半,以逡定分乘之,百而一,為晨分;減逡定分,為昏分。除為轉度,望前以昏,後以晨,加夜半定度,得所在。



 求晨昏中星:各以度數加夜半定度,即中星度。其朔、弦、望,以百刻乘定餘,滿日法得一刻,即各定辰近入刻數。



 皆減其夜半漏,不盡為晨,初刻不滿者屬昨日。



 復月,五千四百五十八。



 交月,二千七百二十九。



 交率,四百六十五。



 交數,五千九百二十三。



 交法,七百三十五萬六千三百六十六。



 會法,五十七萬七千五百三十。



 交復日,二十七。餘,二百六十三。秒,三千四百三十五。



 交日,十三。餘,七百五十二。秒,四千六百七十九。



 交限,日,十二。餘,五百五十五。秒,四百七十三半。



 望差,日,一。餘,百九十七。秒,四千二百五半。



 朔差,日,二。餘,三百九十五。秒,二千四百八十八。



 會限,百五十八。餘,六百七十六。秒,五十半。



 會日,百七十三。餘,三百八十四。秒,二百八十三。



 推月行入交表裏術:置入元積月,復月去之,不盡。交率乘而復去,不如復月者,滿交月去之,為在裏數;不滿為在表數,即所求年天正經入交表裏數。



 求次月:以交率加之,滿交月去之,前表者在裏,前里者在表。



 推月入交日術:以朔實乘表裏數,為交實;滿交法為日,不滿者交數而一,為餘,不成為秒,命日算外,即其經朔月平入交日餘。



 求望:以望差加之,滿交日去之,則月在表裏與朔同;不滿者與朔返。其月食者,先交與當月朔,後交與月朔表裏同。



 求次月:朔差加月朔所入,滿交日去之,表裏與前月返;不滿者,與前月同。



 求經朔望入交常日:
 以月入氣朔望平會日遲速定數,速加、遲減其平入交日餘,為經交常日及餘。



 求定朔望入交定日:以交率乘定朓朒,交數而一,所得以朓減、朒加常日餘,即定朔望所入定日及餘。其去交如望差以下、交限以上者月食,月在里者日食。



 推日入會日術:會法除交實為日,不滿者,如交率為餘,不成為秒,命日算外,即經朔日入平會日及餘。



 求望:加望日及餘,次月加經朔,其表裏皆準入交。



 求入
 會常日:以交數乘月入氣朔望所平會日遲速定數,交率而一,以速加、遲減其入平會日餘,即所入常日餘。亦以定朓朒,而朓減、朒加其常日餘,即日定朔望所入會日及餘。皆滿會日去之,其朔望去會,如望以下、會限以上者,亦月食;月日道表在日道里則日食。



 求月定朔望入交定日夜半:交率乘定餘,交數而一,以減定朔望所入定日餘,即其夜半所定入。



 求次日:以每日遲速數,分前增、分後損定朔所入定日餘,以加
 其日,各得所入定日及餘。



 求次月:加定朔,大月二日,小月一日,皆餘九百七十八,秒二千四百八十八。各以一月遲速數,分前增、分後損其所加,為定。其入七日,餘九百九十七,秒二千三百三十九半以下者,進;其入此以上,盡全餘二百四十四,秒三千五百八十三半者,退。其入十四日,如交餘及秒以下者,退;其入此以上,盡全餘四百八十九,秒千二百四十四者,進而復也。其要為五分,初則七日四分,十四日三分;末則七日後一分,十四日後二分,雖初強末弱,衰率有檢。



 求月入交去日道:皆同其數,以交餘為秒積,以後衰並去交衰,半之,為通數。進則秒積減衰法,以乘衰,交法除,而並衰以半之;退者,半秒積以乘衰,交法而一;皆加通數,秒積乘,交法除,所得以進退衰積,十而一為度,不滿者求其強弱,則月去日道數。月朔望入交,如限以上,減交日,殘為去後交數;如望差以下即為去先交數。有全日同為餘,各朔辰而一,得去交辰。其月在日道里,日應食而有不食者;月日道表在日不應食而亦有食者。



 推應食不食術:朔先後在夏至十日內,去交十二辰少;二十日內,十二
 辰半;一月內,十二辰大;閏四月、六月,十三辰以上,加南方三辰。若朔在夏至二十日內,去交十三辰,以加辰申半以南四辰;閏四月、六月,亦加四辰;穀雨後、處暑前,加三辰;清明後、白露前,加巳半以西、未半以東二辰;春分後秋分前,加午一辰。皆去交十三辰半以上者,並或不食。



 推不應食而食術:朔在夏至前後一月內,去交二辰;四十六日內,一辰半,以加二辰;又一月內,亦一辰半,加三辰及加四辰,與四十六日內加三辰;穀雨後、處暑前,加巳少後、未太前;清明後、白露前,加二辰;春分後、秋分前,加一辰。皆去交半
 辰以下者,並得食。



 推月食多少術:望在分後,以去夏至氣數三之;其分前,又以去分氣數倍而加分後者;皆又以十加去交辰倍而並之,減其去交餘,為不食定餘。乃以減望差,殘者九十六而一,不滿者求其強弱,亦如氣辰法,以十五為限,命之,即各月食多少。



 推日食多少術:月在內者,朔在夏至前後二氣,加南二辰,增去交餘一辰太;加三辰,增一辰少,加四辰,增太。三氣內,加二辰,增
 一辰;加三辰,增太;加四辰,增少。四氣內,加二辰,增太;加三辰及五氣內,加二辰,增少。自外所加辰,立夏后、立秋前,依本其氣內加四辰,五氣內加三辰,六氣內加二辰。六氣內加二辰者,亦依平。自外所加之北諸辰,各依其去立夏、立秋、清明、白露數,隨其依平辰,辰北每辰以其數三分減去交餘;雨水後、霜降前,又半其去分日數,以加二分去二立之日,乃減去交餘;其在冬至前後,更以去霜降、雨水日數三除之,以加霜降雨水當氣所得之數;而減去交餘,皆為定不食餘。以減望差,乃如月食法。月在外者,其去交辰數,若日氣所系之限,止一而無等次者,
 加所去辰一,即為食數。若限有等次,加別系同者,隨所去交辰數而返其衰,以少為多,以多為少,亦加其一,以為食數。皆以十五為限,乃以命之,即各日之所食多少。



 凡日食,月行黃道,體所映蔽,大較正交如累璧,漸減則有差,在內食分多,在外無損。雖外全而月下,內損而更高,交淺則閑遙;交深則相搏而不淹。因遙而蔽多,所觀之地又偏,所食之時亦別。月居外道,此不見虧,月外之人反以為食。交分正等,同在南方,冬損則多,夏虧乃少。假均冬夏,早晚又殊。處南辰體則高,居東西傍而下視有邪正。理不可一,由準率若實而違。古史所詳,事有
 紛互,今故推其梗概,求者知其指歸。茍地非於陽城,皆隨所而漸異。然月食以月行虛道,暗氣所沖,日有暗氣,天有虛道,正黃道常與日對,如鏡居下,魄耀見陰,名曰暗虛,奄月則食,故稱「當月月食,當星星亡。」雖夜半之辰,子午相對,正隔於地,虛道即虧。既月兆日光,當午更耀,時亦隔地,無廢稟明。諒以天光神妙,應感玄通,正當夜半,何害虧稟。月由虛道,表裏俱食。日之與月,體同勢等,校其食分,月盡為多,容或形差,微增虧數,疏而不漏,綱要克舉。



 推日食所在辰術:
 置定餘,倍日限,克減之,月在裏,三乘朔辰為法,除之,所得以艮巽坤乾為次。命艮算外,不滿法者半法減之,無可減者為前,所減之殘為後,前則因餘,後者減法,各為其率。乃以十加去交辰,三除之,以乘率,十四而一,為差。其朔所在氣二分前後一氣內,即為定差。近冬至,以去寒露、驚蟄,近夏至,以去清明、白露氣數,倍而三除去交辰,增之。近冬至,艮巽以加,坤乾以減;近夏至,艮巽以減,坤乾以加其差為定差。乃艮以坤加,巽以乾減定餘。月在外,直三除去交辰,以乘率,十四而一,亦為定差。艮坤以減,巽乾以加定餘,皆為食餘。如氣求入辰法,即日食所
 在辰及大小。其求辰刻,以辰克乘辰餘,朔辰而一,得刻及分。



 若食近朝夕者,以朔所入氣日之出入刻,校食所在,知食見否之少多所在辰,為正見。



 推月食所在辰術:三日阻減望定餘半。置望之所入氣日,不見刻,朔日法乘之,百而一,所得若食餘與之等、以下,又以此所得減朔日法,其殘食餘與之等、以上,為食正見數。其食餘亦朔辰而一,如求加辰所在。又如前求刻校之,月在沖辰食,日月食既有起訖晚早,亦或變常進退,皆於正見前後十二刻半候之。



 推日月食起訖辰術:準其食分十五分為率,全以下各為衰。十四分以上,以一為衰,以盡於五分。每因前衰每降一分,積衰增二,以加於前,以至三分。每積增四。二分每增四,二分增六,一分增十九,皆累算為各衰。三百為率,各衰減之,各以其殘乘朔日法,皆率而一,所得為食衰數。其率全,即以朔日法為衰數,以衰數加減食餘,其減者為起,加者為訖,數亦如氣。



 求入辰法及求刻:以加減食所刻等,得起訖晚早之辰,與校正見多少之數。史書虧復起訖不同,今以其全一
 辰為率。



 推日月食所起術:月在內者,其正南,則起右上,虧左上。若正東,月自日上邪北而下。其在東南維前,東向望之,初不正,橫月高日下;乃月稍西北,日漸東南,過於維后,南向望之,月更北,日差西南;以至於午之後,亦南望之,月欹西北,日復東南。西南維后,西向而望,月為東北,日則西南。正西,自日北下邪虧,而亦後不正,橫月高日下。若食十二分以上,起右虧左。其正東,起上近虧下而北,午前則漸自上邪下。維西,起西北,虧東南。



 維北,起西南,虧東北;午後則稍從
 下傍下。維東,起西南,虧東北。維南,起西北,虧東南。在東則以上為東,在西則以下為西。



 月在外者,其正南,起右下,虧左上。在正東,月自日南邪下而映。維北,則月微東南,日返西。維西南,日稍移東北,以至於午,月南日北,過午之後,月稍東南,日更西北。維北,月有西南,日復東北。正西,月自日下邪南而上。皆準此體以定起虧,隨其所處,每用不同。其月之所食,皆依日虧起,每隨類反之,皆與日食限同表裏,而與日返其逆順,上下過其分。



 五星:歲為木熒惑為火鎮為土
 太白為金辰為水木數,千八百六十萬五千四百六十八。



 伏半平,八十三萬六千八百四十八。



 復日,三百九十八;餘,四萬一千一百五十六。



 歲一,殘日,三十三;餘,二萬九千七百四十九半。



 見去日,十四度。



 平見,在春分前,以四乘去立春日;小滿前,又三乘去春分日,增春分所乘者;白露後,亦四乘去寒露日;小暑,加七日;小雪前,以八乘去寒露日;冬至後,以八乘去立春日,為減,小雪至冬至減七日。



 見,初日行萬一千八百一十八分,日益遲七十分,百一十日行十八度、分四萬七百三十八而留。二十八日乃逆,日退六千四百三十六分,八十七日退十二度、分二百四。又留二十八日。初日行四千一百八十八分,日益疾七十分,百一十日亦行十八度、分四萬七百三十八而伏。



 火數,三千六百三十七萬七千五百九十五。



 伏半平,三百三十七萬九千三百二十七半。



 復日,七百七十九;餘,四萬一千九百一十九。



 歲再,殘日,四十九;餘,萬九千一百六。



 見去日,十六度。



 平見,在雨水前,以十九乘去大寒日:清明前,又十八乘去雨水日,增雨水所乘者;夏至後,以十六乘去處暑日;小滿後,又十五日;寒露前,以十八乘去白露日;小雪前,又十七乘去寒露日,增寒露所乘者;大雪後,二十九乘去大寒日,為減,小雪至大雪減二十五日。



 見,初在冬至,則二百三十六日行百五十八度,以後日度隨其日數增損各一;盡三十日,一日半損一;又八十六日,二日損一;復三十八日,同;又十五日,三日損一;復十二日,同;又三十九日,三日增一;又二十四日,二日增一;又五十八日,一日增一;復三十三日,同;又三十日,二日損
 一,還終至冬至,二百三十六日行百五十八度。其立春盡春分,夏至盡立夏,八日減一日;春分至立夏,減六日;立秋至秋分,減五度,各其初行日及度數。白露至寒露,初日行半度,四十日行二十度。以其殘日及度,計充前數,皆差行,日益遲二十分,各盡其日度乃遲,初日行分二萬二千六百六十九,日益遲一百一十分,六十一日行二十五度、分萬五千四百九。初減度五者,於此初日加分三千八百二十三、篾十七;以遲日為母,盡其遲日行三十度,分同,而留十三日。



 前減日分於二留,乃逆,日退分萬二千五百二十六,六
 十三日退十六度、分四萬二千八百三十四。又留十三日而行,初日萬六千六十九,日益疾百一十分,六十一日行二十五度、分萬五千四百九。立秋盡秋分,增行度五,加初日分同前,更疾。在冬至則二百一十三日行百三十五度;盡三十六日,一日損一;又二十日,二日損一;復二十四日,同;又五十四日,三日日增一;又十二日,二日增一;又四十二日,一日增一;又十四日,一日增一半;又十二日,增一;復四十五日,同;又一百六日,二日損一,亦終冬至二百一十三日,行百三十五度。



 前增行度五者,於此亦減五度,為疾日及數。其立夏盡
 夏至初,日行半度,六十日行三十度。夏至盡立秋,亦初日行半度,四十日行二十度。其殘亦計充如前,皆差行,日益疾二十分,各盡其日度而伏。



 土數,千七百六十三萬五千五百九十四。



 伏半平,八十六萬四千九百九十五。



 復日,三百七十八;餘,四千一百六十二。



 歲一,殘日,十二;餘,三萬九千三百九十九半。



 見去日,十六度半。



 平見,在大暑前,以七乘去小滿日;寒露後,九乘去小雪日,為加,大暑至寒露加八日。小寒前,以九乘去小雪日;
 雨水後,以四乘去小滿日;立春後,又三乘去雨水日,增雨水所乘者,為減,小寒至立春減八日。



 見,日行分四千三百六十四,八十日行七度、分二萬二千六百一十二而留三十九日乃逆,日退分二千八百二十,百三日退六度、分萬五百九十六。又留三十九日,亦行分日四千三百六十四,八十日行七度、分二萬二千六百一十二而伏。



 金數,二千七百二十三萬六千二百八。



 晨伏半平,百九十五萬七千一百四。



 復日,五百八十三;餘,四萬二千七百五十六。



 歲一,殘日,二百一十八;餘,三萬一千三百四十九半。



 夕見伏,二百五十六日。



 晨見伏,三百二十七日;餘與復同。



 見去日,十一度。



 夕平見,在立秋前,以六乘去芒種日;秋分後,以五乘去小雪日;小雪後,又四乘去大雪日,增小雪所乘者,為加,立秋至秋分加七日。立春前,以五乘去大雪日;雨水前,又四乘去立春日,增立春所乘者;清明後,以六乘去芒種日,為減,雨水至清明減七日。



 晨平見,在小寒前,以六乘去冬至日;立春前,又五乘去
 小寒日,增小寒所乘者;芒種前,以六乘去夏至日;立夏前,又五乘去芒種日,增芒種所乘者,為加,立春至立夏加五日。小暑前,以六乘去夏至日;立秋前,又五乘去小暑日;增小暑所乘者;大雪後,以六乘去冬至日;立冬後,又五乘去大雪日,增大雪所乘者,為減,立秋至立冬減五日。



 夕見,百七十一日行二百六度。其穀雨至小滿、白露至寒露,皆十日加一度;小滿至白露,加三度。乃十二日行十二度。冬至後,十二日減日度各一,雨水盡夏至,日度七;夏至後六日增一。大暑至立秋,還日度十二;至寒露,日度
 二十二,後六日減一。自大雪盡冬至,又日度十二而遲。日益疾五百二十分,初日行分二萬三千七百九十一、篾三十五,行日為母,四十三日行三十二度。



 前加度者,此依減之。留九日乃逆,日退太半度,九日退六度,而夕伏晨見。日退太半度,九日退六度。



 復留,九日而行,日益遲五百二十分,初日行分四萬五千六百三十一、篾三十五,四十三日行三十二度。芒種至小暑,大雪至立冬,十五日減一度;小暑至立冬,減二度。又十二日行十二度。冬至後,十五日增日度各一。



 驚蟄至春分,日度十七,後十五日減一,盡夏至,還日度十二。後六日減一,至
 白露,日度皆盡。霜降後,五日增一,盡冬至,又日度十二。乃疾,百七十一日行二百六度。前減者,此亦加之,而晨伏。



 水數,五百四十萬五千六。



 晨伏半平,七十九萬九十九。



 復日,百一十五;餘,四萬九百四十六。



 夕見伏,五十一日。



 晨見伏,六十四日;餘與復同。



 見去日,十七度。



 夕應見,在立秋後小雪前者不見;其白露前立夏後,時有見者。



 晨應見,在立春後小滿前者不見;其驚蟄前立冬後,時有見者。



 夕見,日行一度太,十二日行二十度。小暑至白露,行度半,十二日行十八度,乃八日行八度。大暑後,二日去度一,訖十六日,而日度俱盡。而遲,日行半度,四日行二度。益遲,日行少半度,三日行一度。前行度半者,去此益遲。乃留四日而夕伏晨見,留四日,為日行少半度,三日行一度。大寒至驚蟄,無此行,更疾,日行半度;四日行二度;又八日行八度。亦大寒後,二日去度一;訖十六日,亦日度俱盡。益疾,日行一度太,十二日行二十度。初無遲者,此
 行度半,十二日行十八度而晨伏。



 推星平見術:各以伏半減積半實,乃以其數去之;殘返減數,滿氣日法為日,不滿為餘,即所求年天正冬至後平見日餘。



 金、水滿晨見伏日者,去之,晨平見。求平見月日:以冬至去定朔日、餘,加其後日及餘,滿復日又去,起天正月,依定大小朔除之,不盡算外日,即星見所在。求後平見,因前見去其歲一、再,皆以殘日加之,亦可。



 其復日,金水準以晨夕見伏日,加晨得夕,加夕得晨。



 求常見日:以轉法除所得加減者,為日;其不滿,以餘通
 乘之,為餘;並日,皆加減平見日、餘,即為常見日及餘。



 求定見日:以其先後已通者,先減後加常見日,即得定見日餘。



 求星見所在度:置星定見、其日夜半所在宿度及分,以其日先後餘,分前加、分後減氣日法,而乘定見餘,氣日法而一所得加夜半度分,乃以星初見去日度數,晨減夕加之,即星初見所在宿度及分。



 求次日:各加一日所行度及分。其有益疾、遲者副置一日行分,
 各以其分疾增、遲損,乃加之。有篾者,滿法從分,其母有不等,齊而進退之。留即因前,逆則依減入虛去分,逆出先加。皆以篾法除,為轉分;其不盡者,仍謂之篾,各得每日所在知去日度。增以日所入先後分,定之。諸行星度求水其外內,準月行增損黃道而步之;不明者,依黃道而求所去日度。先後分亦分明前加後減。其金、火諸日度,計數增損定之者。其日少度多,以日減度之殘者,與日多度少之度,皆度法乘之,日數而一,所得為分。不滿篾,以日數為母。日少者以分並減之一度,日多者直為度分,即皆一日平行分。其差行者,皆減所行日數一,乃半
 其益疾、益遲分而乘之,益疾以減、益遲以加一日平行分,皆初日所行分。有計日加減,而日數不滿未得成度者,以氣日法若度法乘,見已所行日即日數除之,所得以增損其氣日疾法,為日及度。其不成者,亦即為篾。其木、火、土,晨有見而夕有伏;金、水即夕見還夕伏,晨見即晨伏。然火之初行及後疾,距冬至日計日增損日度者,皆當先置從冬至日餘數,累加於位上,以知其去冬至遠近,乃以初見與後疾初日去冬至日數而增損定之,而後依其所直日度數行之也。



\end{pinyinscope}