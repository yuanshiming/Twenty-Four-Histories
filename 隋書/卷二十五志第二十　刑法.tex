\article{卷二十五志第二十 刑法}

\begin{pinyinscope}

 夫刑者,制死生之命,詳善惡之源,翦亂除暴,禁人為非者也。聖王仰視法星,旁觀習坎,彌縫五氣,取則四時,莫不先春風以播恩,後秋霜而動憲。是以宣慈惠愛,導其萌芽,刑罰威怒,隨其肅殺。仁恩以為情性,禮義以為綱紀,養化以為本,明刑以為助。上有道,刑之而無刑;上無道,殺之而不勝也。《記》曰:「教之以德,齊之以禮,則人有格
 心。教之以政,齊之以刑,則人有遁心。」而始乎勸善,終乎禁暴,以此字人,必兼刑罰。至於時逢交泰,政稱忠厚,美化與車軌攸同,至仁與嘉祥間出,歲布平典,年垂簡憲。昭然如日月,望之者不迷,曠乎如大路,行之者不惑。



 刑者甲兵焉,鈇鉞焉,刀鋸鉆鑿,鞭撲榎楚,陳乎原野而肆諸市朝,其所由來,亦已久矣。若夫龍官之歲,鳳紀之前,結繩而不違,不令而人畏。五帝畫象,殊其衣服,三王肉刑,刻其膚體。若重華之眚災肆赦,文命之刑罰三千,而都君恤刑,尚奉唐堯之德,高密泣罪,猶懷虞舜之心。殷因以降,去德滋遠。若紂能遵成湯,不造砲格,設刑兼禮,
 守位依仁,則西伯斂轡,化為田叟。周王立三刺以不濫,弘三宥以開物,成、康以四十二年之間,刑厝不用。薰風潛暢,頌聲遐舉,越裳重譯,萬里來歸。若乃魯接燕、齊,荊鄰鄭、晉,時之所尚,資乎辯舌,國之所恃,不在威刑,是以才鼓夷蒐,宣尼致誚,既鑄刑闢,叔向貽書,夫勃澥之浸,沾濡千里,列國之政,豈周之膏潤者歟!秦氏僻自西戎,初平區夏,於時投戈棄甲,仰恩祈惠,乃落嚴霜於政教,揮流電於邦國,棄灰偶語,生愁怨於前,毒網凝科,害肌膚於後。



 玄鉞肆於朝市,赭服飄於路衢,將閭有一劍之哀,茅焦請列星之數。漢高祖初以三章之約,以慰秦人,
 孝文躬親玄默,遂疏天網。孝宣樞機周密,法理詳備,選於定國為廷尉,黃霸以為廷平。每以季秋之後,諸所請讞,帝常幸宣室,齋而決事,明察平恕,號為寬簡。光武中興,不移其舊,是以二漢群後,罕聞殘酷。魏武造易釱之科,明皇施減死之令,中原凋敝,吳、蜀三分,哀矜折獄,亦所未暇。晉氏平吳,九州寧一,乃命賈充,大明刑憲。內以平章百姓,外以和協萬邦,實曰輕平,稱為簡易。是以宋、齊方駕,轥其餘軌。若乃刑隨喜怒,道暌正直,布憲擬於秋荼,設網逾於朝脛,恣興夷翦,取快情靈。若隋高祖之揮刃無辜,齊文宣之輕刀臠割,此所謂匹夫私仇,非關國
 典。孔子曰:「刑亂及諸政,政亂及諸身。」心之所詣,則善惡之本原也。彪、約所制,無刑法篇,臧、蕭之書,又多漏略,是以撮其遺事,以至隋氏,附於篇云。



 梁武帝承齊昏虐之餘,刑政多僻。既即位,乃制權典,依周、漢舊事,有罪者贖。其科,凡在官身犯,罰金。鞭杖杖督之罪,悉入贖停罰。其臺省令史士卒欲贖者,聽之。時欲議定律令,得齊時舊郎濟陽蔡法度,家傳律學,雲齊武時,刪定郎王植之,集注張、杜舊律,合為一書,凡一千五百三十條,事未施行,其文殆滅,法度能言之。於是以為兼尚書刪定郎,使損益植之舊本,以為《梁律》。天監元年
 八月,乃下詔曰:「律令不一,實難去弊。殺傷有法,昏墨有刑,此蓋常科,易為條例。至如三男一妻,懸首造獄,事非慮內,法出恆鈞。前王之律,後王之令,因循創附,良各有以。若游辭費句,無取於實祿者,宜悉除之。求文指歸,可適變者,載一家為本,用眾家以附。丙丁俱有,則去丁以存丙。若丙丁二事注釋不同,則二家兼載。咸使百司,議其可不,取其可安,以為標例。宜云:『某等如乾人同議,以此為長』,則定以為《梁律》。留尚書比部,悉使備文,若班下州郡,止撮機要。



 可無二門侮法之弊。」法度又請曰:「魏、晉撰律,止關數人,今若皆咨列位,恐緩而無決。」於是以尚
 書令王亮、侍中王瑩、尚書僕射沈約、吏部尚書範雲、長兼侍中柳惲、給事黃門侍郎傅昭、通直散騎常侍孔藹、御史中丞樂藹、太常丞許懋等,參議斷定,定為二十篇:一曰刑名,二曰法例,三曰盜劫,四曰賊叛,五曰詐偽,六曰受賕,七曰告劾,八曰討捕,九曰系訊,十曰斷獄,十一曰雜,十二曰戶,十三曰擅興,十四曰毀亡,十五曰衛宮,十六曰水火,十七曰倉庫,十八曰廄,十九曰關市,二十曰違制。其制刑為十五等之差:棄市已上為死罪,大罪梟其首,其次棄市。刑二歲已上為耐罪,言各隨伎能而任使之也。有髡鉗五歲刑,笞二百收贖絹,男子六十匹。
 又有四歲刑,男子四十八匹。又有三歲刑,男子三十六匹。又有二歲刑,男子二十四匹。罰金一兩已上為贖罪。贖死者金二斤,男子十六匹。贖髡鉗五歲刑笞二百者,金一斤十二兩,男子十四匹。贖四歲刑者,金一斤八兩,男子十二匹。贖三歲刑者,金一斤四兩,男子十匹。贖二歲刑者,金一斤,男子八匹。罰金十二兩者,男子六匹。罰金八兩者,男子四匹。罰金四兩者,男子二匹。罰金二兩者,男子一匹。罰金一兩者,男子二丈。女子各半之。五刑不簡,正於五罰,五罰不服,正於五過,以贖論,故為此十五等之差。又制九等之差:有一歲刑,半歲刑,百日刑,鞭
 杖二百,鞭杖一百,鞭杖五十,鞭杖三十,鞭杖二十,鞭杖一十。又有八等之差:一曰免官,加杖督一百;二曰免官;三曰奪勞百日,杖督一百;四曰杖督一百;五曰杖督五十;六曰杖督三十;七曰杖督二十;八曰杖督一十。論加者上就次,當減者下就次。凡系獄者,不即答款,應加測罰,不得以人士為隔。若人士犯罰,違捍不款,宜測罰者,先參議牒啟,然後科行。斷食三日,聽家人進粥二升。



 女及老小,一百五十刻乃與粥,滿千刻而止。囚有械、杻、斗械及鉗,並立輕重大小之差,而為定制。其鞭有制鞭、法鞭、常鞭,凡三等之差。制鞭,生革廉成;法鞭,生革去廉;常鞭,
 熟靼不去廉。皆作鶴頭紐,長一尺一寸。梢長二尺七寸,廣三分,靶長二尺五寸。杖皆用生荊,長六尺。有大杖、法杖、小杖三等之差。大杖,大頭圍一寸三分,小頭圍八分半。法杖,圍一寸三分,小頭五分。小杖,圍一寸一分,小頭極杪。諸督罰,大罪無過五十、三十,小者二十。當笞二百以上者,笞半,餘半後決,中分鞭杖。老小於律令當得鞭杖罰者,皆半之。其應得法鞭、杖者,以熟靼鞭、小杖。過五十者,稍行之。將吏已上及女應有罰者,以罰金代之。其以職員應罰,及律令指名制罰者,不用此令。其問事諸罰,皆用熟靼鞭、小杖。其制鞭制杖,法鞭法杖,自非特
 詔,皆不得用。詔鞭杖在京師者,皆於雲龍門行。女子懷孕者,勿得決罰。其謀反、大逆已上皆斬。父子同產田,無少長皆棄市。母妻姊妹及應從坐棄市者,妻子女妾同補奚官為奴婢。貲財沒官。劫身皆斬,妻子補兵。遇赦降死者,黵面為劫字,髡鉗,補冶鎖士終身。其下又謫運配材官冶士、尚方鎖士,皆以輕重差其年數。其重者或終身。



 士人有禁錮之科,亦有輕重為差。其犯清議,則終身不齒。耐罪囚八十已上,十歲已下,及孕者、盲者、侏儒當械系擊者,及郡國太守相、都尉、關中侯已上,亭侯已上之父母妻子,及所生坐非死罪除名之罪,二千石已
 上非檻徵者,並頌系之。



 丹陽尹月一詣建康縣,令三官參共錄獄,察斷枉直。其尚書當錄人之月者,與尚書參共錄之。大凡定罪二千五百二十九條。



 二年四月癸卯,法度表上新律,又上《令》三十卷,《科》三十卷。帝乃以法度守廷尉卿,詔班新律於天下。



 三年八月,建康女子任提女,坐誘口當死。其子景慈對鞫辭云,母實行此。是時法官虞僧虯啟稱:「案子之事親,有隱無犯,直躬證父,仲尼為非。景慈素無防閑之道,死有明目之據,陷親極刑,傷和損俗。凡乞鞫不審,降罪一等,豈得避五歲之刑,忽死母之命!景慈宜加罪闢。」詔流於交州。至是復有流徒之
 罪。其年十月甲子,詔以金作權典,宜在蠲息。於是除贖罪之科。



 武帝敦睦九族,優借朝士,有犯罪者,皆諷群下,屈法申之。百姓有罪,皆案之以法。其緣坐則老幼不免,一人亡逃,則舉家質作。人既窮急,奸宄益深。後帝親謁南郊,秣陵老人遮帝曰:「陛下為法,急於黎庶,緩於權貴,非長久之術。誠能反是,天下幸甚。」帝於是思有以寬之。舊獄法,夫有罪,逮妻子,子有罪,逮父母。十一年正月壬辰,乃下詔曰:「自今捕謫之家,及罪應質作,若年有老小者,可停將送。」十四年,又除黵面之刑。



 帝銳意儒雅,疏簡刑法,自公卿大臣,咸不以鞫獄留意。奸吏招權,巧文弄
 法,貨賄成市,多致枉濫。大率二歲刑已上,歲至五千人。是時徙居作者具五任,其無任者,著斗械。若疾病,權解之。是後囚徒或有優劇。大同中,皇太子在春宮視事,見而愍之,乃上疏曰:「臣以比時奉敕,權親京師雜事。切見南北郊壇、材官、車府、太官下省、左裝等處上啟,並請四五歲已下輕囚,助充使役。自有刑均罪等,愆目不異,而甲付錢署,乙配郊壇。錢署三所,於事為劇,郊壇六處,在役則優。



 今聽獄官詳其可否,舞文之路,自此而生。公平難遇其人,流泉易啟其齒,將恐玉科重輕,全關墨綬,金書去取,更由丹筆。愚謂宜詳立條制,以為永準。」帝手敕
 報曰:「頃年已來,處處之役,唯資徒謫,逐急充配。若科制繁細,義同簡絲,切須之處,終不可得。引例興訟,紛紜方始。防杜奸巧。自是為難。更當別思,取其便也。」竟弗之從。是時王侯子弟皆長,而驕蹇不法。武帝年老,厭於萬機,又專精佛戒,每斷重罪,則終日弗懌。嘗游南苑,臨川王宏伏人於橋下,將欲為逆。事覺,有司請誅之。帝但泣而讓曰:「我人才十倍於爾,處此恆懷戰懼。爾何為者?



 我豈不能行周公之事,念汝愚故也。」免所居官。頃之,還復本職。由是王侯驕橫轉甚,或白日殺人於都街,劫賊亡命,咸於王家自匿,薄暮塵起,則剝掠行路,謂之打稽。武帝
 深知其弊,而難於誅討。十一年十月,復開贖罪之科。中大同元年七月甲子,詔自今犯罪,非大逆,父母、祖父母勿坐。自是禁網漸疏,百姓安之,而貴戚之家,不法尤甚矣。尋而侯景逆亂。



 及元帝即位,懲前政之寬,且帝素苛刻,及周師至,獄中死囚且數千人,有司請皆釋之,以充戰士。帝不許,並令棒殺之。事未行而城陷。敬帝即位,刑政適陳矣。



 陳氏承梁季喪亂,刑典疏闊。及武帝即位,思革其弊,乃下詔曰:「朕聞唐、虞道盛,設畫象而不犯,夏、商德衰,雖孥戮其未備。洎乎末代,綱目滋繁,矧屬亂離,憲章遺紊。朕
 始膺寶歷,思廣政樞,外可搜舉良才,冊改科令,群僚博議,務存平簡。」於是稍求得梁時明法吏,令與尚書刪定郎範泉參定律令。又敕尚書僕射沈欽、吏部尚書徐陵、兼尚書左丞宗元饒、兼尚書左丞賀朗參知其事,制《律》三十卷,《令律》四十卷。採酌前代,條流冗雜,綱目雖多,博而非要。其制唯重清議禁錮之科。若縉紳之族,犯虧名教,不孝及內亂者,發詔棄之,終身不齒。先與士人為婚者,許妻家奪之。其獲賊帥及士人惡逆,免死付治,聽將妻入役,不為年數。又存贖罪之律,復父母緣坐之刑。自餘篇目條綱,輕重簡繁,一用梁法。其有贓驗顯然而不
 款,則上測立。立測者,以土為垛,高一尺,上圓劣,容囚兩足立。



 鞭二十,笞三十訖,著兩械及杻,上垛。一上測七刻,日再上。三七日上測,七日一行鞭。凡經杖,合一百五十,得度不承者,免死。其髡鞭五歲刑,降死一等,鎖二重。其五歲刑已下,並鎖一重。五歲四歲刑,若有官,準當二年,餘並居作。其三歲刑,若有官,準當二年,餘一年贖。若公坐過誤,罰金。其二歲刑,有官者,贖論。一歲刑,無官亦贖論。寒庶人,準決鞭杖。囚並著械,徒並著鎖,不計階品。



 死罪將決,乘露車,著三械。加壺手。至市,脫手械及壺手焉。當刑於市者,夜須明,雨須晴。晦朔、八節、六齊、月在張心
 日,並不得行刑。廷尉寺為北獄,建康縣為南獄,並置正監平。又制,常以三月,侍中、吏部尚書、尚書、三公郎、部都令史、三公錄冤局,令史、御史中丞、侍御史、蘭臺令史,親行京師諸獄及冶署,理察囚徒冤枉。



 文帝性明察,留心刑政,親覽獄訟,督責群下,政號嚴明。是時承寬政之後,功臣貴戚有非法,帝咸以法繩之,頗號峻刻。及宣帝即位,優借文武之士,崇簡易之政,上下便之。其後政令即寬,刑法不立,又以連年北伐,疲人聚為劫盜矣。後主即位,信任讒邪,群下縱恣,鬻獄成市,賞罰之命,不出於外。後主性猜忍疾忌,威令不行,左右有忤意者,動至夷戮。百
 姓怨叛,以至於滅。



 齊神武、文襄,並由魏相,尚用舊法。及文宣天保元年,始命群官刊定魏朝《麟趾格》。是時軍國多事,政刑不一,決獄定罪,罕依律文,相承謂之變法從事。



 清河房超為黎陽郡守,有趙道德者,使以書屬超。超不發書,棒殺其使。文宣於是令守宰各設棒,以誅屬請之使。後都官郎中宋軌奏曰:「昔曹操懸棒,威於亂時,今施之太平,未見其可。若受使請賕,猶致大戮,身為枉法,何以加罪?」於是罷之。即而司徒功曹張老上書,稱大齊受命已來,律令未改,非所以創制垂法,革人視聽。於是始命群官,議造《齊
 律》,積年不成。其決獄猶依魏舊。是時刑政尚新,吏皆奉法。自六年之後,帝遂以功業自矜,恣行酷暴,昏狂酗W,任情喜怒。為大鑊、長鋸、坐刂碓之屬,並陳於庭,意有不快,則手自屠裂,或命左右臠啖,以逞其意。時僕射楊遵彥乃令憲司先定死罪囚,置於仗衛之中,帝欲殺人,則執以應命,謂之供御囚。經三月不殺者,則免其死。帝嘗幸金鳳臺,受佛戒,多召死囚,編籧篨為翅,命之飛下,謂之放生。墜皆致死,帝視以為觀笑。時有司折獄,又皆酷法。



 訊囚則用車輻犬芻杖,夾指壓踝,又立之燒犁耳上,或使以臂貫燒車釭。既不勝其苦,皆致誣伏。七年,豫州檢使
 白手剽為左丞盧斐所劾,乃於獄中誣告斐受金。文宣知其奸罔,詔令按之,果無其事。乃敕八座議立案劾格,負罪不得告人事。於是挾奸者畏糾,乃先加誣訟,以擬當格,吏不能斷。又妄相引,大獄動至千人,多移歲月。然帝猶委政輔臣楊遵彥,彌縫其闕,故時議者竊云,主昏於上,政清於下。



 孝昭在籓,已知其失,即位之後,將加懲革,未幾而崩。武成即位,思存輕典,大寧元年,乃下詔曰:「王者所用,唯在賞罰,賞貴適理,罰在得情。然理容進退,事涉疑似,盟府司勛,或有開塞之路,三尺律令,未窮畫一之道。想文王之官人,念宣尼之止訟,刑賞之宜,思獲其
 所。自今諸應賞罰,皆賞疑從重,罰疑從輕。」



 又以律令不成,頻加催督。河清三年,尚書令、趙郡王睿等,奏上《齊律》十二篇:一曰名例,二曰禁衛,三曰婚戶,四曰擅興,五曰違制,六曰詐偽,七曰鬥訟,八曰賊盜,九曰捕斷,十曰毀損,十一曰廄牧,十二曰雜。其定罪九百四十九條。又上《新令》四十卷,大抵採魏、晉故事。其制,刑名五:一曰死,重者轘之,其次梟首,並陳尸三日;無市者,列於鄉亭顯處。其次斬刑,殊身首。其次絞刑,死而不殊。凡四等。二曰流刑,謂論犯可死,原情可降,鞭笞各一百,髡之,投於邊裔,以為兵卒,未有道里之差。其不合遠配者,男子長徒,女
 子配舂,並六年。三曰刑罪,即耐罪也。有五歲、四歲、三歲、二歲、一歲之差。凡五等。各加鞭一百。其五歲者,又加笞八十,四歲者六十,三歲者四十,二歲者二十,一歲者無笞。並鎖輸左校而不髡。無保者鉗之。婦人配舂及掖庭織。四曰鞭,有一百、八十、六十、五十、四十之差,凡五等。五曰杖,有三十、二十、十之差,凡三等。大凡為十五等。當加者上就次,當減者下就次。贖罪舊以金,皆代以中絹。死一百匹,流九十二匹,刑五歲七十八匹,四歲六十四匹,三歲五十匹,二歲三十六匹。各通鞭笞論。



 一歲無笞,則通鞭二十四匹。鞭杖每十,贖絹一匹。至鞭百,則絹十匹。
 無絹之鄉,皆準絹收錢。自贖笞十已上至死。又為十五等之差。當加減次,如正決法。合贖者,謂流內官及爵秩比視、老小閹癡並過失之屬。犯罰絹一匹及杖十已上,皆名為罪人。



 盜及殺人而亡者,即懸名注籍,甄其一房配驛戶。宗室則不注盜,及不入奚官,不加宮刑。自犯流罪已下合贖者,及婦人犯刑已下,侏儒、篤疾、癃殘非犯死罪,皆頌系之。罪刑年者鎖,無鎖以枷。流罪已上加杻械。死罪者桁之。決流刑鞭笞者,鞭其背。五十,一易執鞭人。鞭鞘皆用熟皮,削去廉棱。鞭瘡長一尺。笞者笞臂,而不中易人。杖長三尺五寸,大頭徑二分半,小頭徑一分
 半。決三十已下杖者,長四尺,大頭徑三分,小頭徑二分。在官犯罪,鞭杖十為一負。閑局六負為一殿,平局八負為一殿,繁局十負為一殿。加於殿者,復計為負焉。赦日,則武庫令設金雞及鼓於閶闔門外之右。勒集囚徒於闕前,撾鼓千聲,釋枷鎖焉。又列重罪十條:一曰反逆,二曰大逆,三曰叛,四曰降,五曰惡逆,六曰不道,七曰不敬,八曰不孝,九曰不義,十曰內亂。其犯此十者,不在八議論贖之限。是後法令明審,科條簡要,又敕仕門之子弟常講習之。齊人多曉法律,蓋由此也。其不可為定法者,別制《權令》二卷,與之並行。後平秦王高歸彥謀反,須有
 約罪,律無正條,於是遂有《別條權格》,與律並行。大理明法,上下比附,欲出則附依輕議,欲入則附從重法,奸吏因之,舞文出沒。至於後主,權幸用事,有不附之者,陰中以法。綱紀紊亂,卒至於亡。



 周文帝之有關中也,霸業初基,典章多闕。大統元年,命有司斟酌今古通變可以益時者,為二十四條之制,奏之。七年,又下十二條制。十年,魏帝命尚書蘇綽,總三十六條,更損益為五卷,班於天下。其後以河南趙肅為廷尉卿,撰定法律。肅積思累年,遂感心疾而死。乃命司憲大夫拓拔迪掌之。至保定三年三月庚子乃就,謂之《大
 律》,凡二十五篇:一曰刑名,二曰法例,三曰祀享,四曰朝會,五曰婚姻,六曰戶禁,七曰水火,八曰興繕,九曰衛宮,十曰市廛,十一曰鬥競,十二曰劫盜,十三曰賊叛,十四曰毀亡,十五曰違制,十六曰關津,十七曰諸侯,十八曰廄牧,十九曰雜犯,二十曰詐偽,二十一曰請求,二十二曰告言,二十三曰逃亡,二十四曰系訊,二十五曰斷獄。大凡定罪一千五百三十七條。其制罪,一曰權刑五,自十至五十。二曰鞭刑五,自六十至於百。三曰徒刑五,徒一年者,鞭六十,笞十。



 徒二年者,鞭七十,笞二十。徒三年者,鞭八十,笞三十。徒四年者,鞭九十,笞四十。徒五年者,
 鞭一百,笞五十。四曰流刑五,流衛服,去皇畿二千五百里者,鞭一百,笞六十。流要服,去皇畿三千里者,鞭一百,笞七十。流荒服,去皇畿三千五百里者,鞭一百,笞八十。流鎮服,去皇畿四千里者,鞭一百,笞九十。流蕃服,去皇畿四千五百里者,鞭一百,笞一百。五曰死刑五,一曰磬,二曰絞,三曰斬,四曰梟,五曰裂。五刑之屬各有五,合二十五等。不立十惡之目,而重惡逆、不道、大不敬、不孝、不義、內亂之罪。凡惡逆,肆之三日。盜賊群攻鄉邑及入人家者,殺之無罪。若報仇者,告於法而自殺之,不坐。經為盜者,注其籍。唯皇宗則否。凡死罪枷而拲,流罪枷而梏,
 徒罪枷,鞭罪桎,杖罪散以待斷。皇族及有爵者,死罪已下鎖之,徒已下散之。獄成將殺者,書其姓名及其罪於拲而殺之市。唯皇族與有爵者隱獄。



 其贖杖刑五,金一兩至五兩。贖鞭刑五,金六兩至十兩。贖徒刑五,一年金十二兩,二年十五兩,三年一斤二兩,四年一斤五兩,五年一斤八兩。贖流刑,一斤十二兩,俱役六年,不以遠近為差等。贖死罪,金二斤。鞭者以一百為限。加笞者,合二百止。應加鞭笞者,皆先笞後鞭。婦人當笞者,聽以贖論。徒輸作者,皆任其所能而役使之。杖十已上,當加者上就次,數滿乃坐。當減者,死罪流蕃服,蕃服已下俱至徒
 五年。五年以下,各以一等為差。盜賊及謀反大逆降叛惡逆罪當流者,皆甄一房配為雜戶。其為盜賊事發逃亡者,懸名注配。若再犯徒、三犯鞭者,一身永配下役。應贖金者,鞭杖十,收中絹一匹。流徒者,依限歲收絹十二匹。死罪者一百匹。其贖刑,死罪五旬,流刑四旬,徒刑三旬,鞭刑二旬,杖刑一旬。限外不輸者,歸於法。貧者請而免之。大凡定法一千五百三十七條,班之天下。其大略滋章,條流苛密,比於齊法,煩而不要。



 又初除復仇之法,犯者以殺論。時晉公護將有異志,欲寬政以取人心,然暗於知人,所委多不稱職。既用法寬弛,不足制奸,子弟
 僚屬,皆竊弄其權,百姓愁怨,控告無所。武帝性甚明察,自誅護後,躬覽萬機,雖骨肉無所縱舍,用法嚴正,中外肅然。自魏、晉相承,死罪其重者,妻子皆以補兵。魏虜西涼之人,沒入名為隸戶。魏武入關,隸戶皆在東魏,後齊因之,仍供廝役。建德六年,齊平後,帝欲施輕典於新國,乃詔凡諸雜戶,悉放為百姓。自是無復雜戶。其後又以齊之舊欲,未改昏政,賊盜奸宄,頗乖憲章。其年,又為《刑書要制》以督之。其大抵持仗群盜一匹以上,不持仗群盜五匹以上,監臨主掌自盜二十匹以上,盜及詐請官物三十匹以上,正長隱五戶及十丁以上及地三頃以上,皆
 死。自餘依《大律》。由是澆詐頗息焉。



 宣帝性殘忍暴戾,自在儲貳,惡其叔父齊王憲及王軌、宇文孝伯等。及即位,並先誅戮,由是內外不安,俱懷危懼。帝又恐失眾望,乃行寬法,以取眾心。宣政元年八月,詔制九條,宣下州郡。大象元年,又下詔曰:「高祖所立《刑書要制》,用法深重,其一切除之。」然帝荒淫日甚,惡聞其過,誅殺無度,疏斥大臣。又數行肆赦,為奸者皆輕犯刑法,政令不一,下無適從。於是又廣《刑書要制》,而更峻其法,謂之《刑經聖制》。宿衛之官,一日不直,罪至削除。逃亡者皆死,而家口籍沒。上書字誤者,科其罪。鞭杖皆百二十為度,名曰天杖。其
 後又加至二百四十。又作礔歷車,以威婦人。其決人罪,雲與杖者,即一百二十,多打者,即二百四十。帝既酣飲過度,嘗中飲,有下士楊文祐白宮伯長孫覽,求歌曰:「朝亦醉,暮亦醉。日日恆常醉,政事日無次。」鄭譯奏之,帝怒,命賜杖二百四十而致死。



 後更令中士皇甫猛歌,猛歌又諷諫。鄭譯又以奏之,又賜猛杖一百二十。是時下自公卿,內及妃後,咸加棰楚,上下愁怨。及帝不豫,而內外離心,各求茍免。隋高祖為相,又行寬大之典,刪略舊律,作《刑書要制》。既成奏之,靜帝下詔頒行。



 諸有犯罪未科決者,並依制處斷。



 高祖既受周禪,開皇元年,乃詔尚書左僕射、勃海公高熲,上柱國、沛公鄭譯,上柱國、清河郡公楊素,大理前少卿、平源縣公常明,刑部侍郎、保城縣公韓浚,比部侍郎李諤,兼考功侍郎柳雄亮等,更定新律,奏上之。其刑名有五:一曰死刑二,有絞,有斬。二曰流刑三,有一千里、千五百里、二千里。應配者,一千里居作二年,一千五百里居作二年半,二千里居作三年。應住居作者,三流俱役三年。



 近流加杖一百,一等加三十。三曰徒刑五,有一年、一年半、二年、二年半、三年。



 四曰杖刑五,自五十至於百。五曰笞刑五,自十至於五十。而蠲除前代鞭刑及梟首
 轘裂之法。其流徒之罪皆減縱輕。唯大逆謀反叛者,父子兄弟皆斬,家口沒官。又置十惡之條,多採後齊之制,而頗有損益。一曰謀反,二曰謀大逆,三曰謀叛,四曰惡逆,五曰不道,六曰大不敬,七曰不孝,八曰不睦,九曰不義,十曰內亂。犯十惡及故殺人獄成者,雖會赦,猶除名。其在八議之科及官品第七已上犯罪,皆例減一等。其品第九已上犯者,聽贖。應贖者,皆以銅代絹。贖銅一斤為一負,負十為殿。笞十者銅一斤,加至杖百則十斤。徒一年,贖銅二十斤,每等則加銅十斤,三年則六十斤矣。流一千里,贖銅八十斤,每等則加銅十斤,二千里則百
 斤矣。二死皆贖銅百二十斤。犯私罪以官當徒者,五品已上,一官當徒二年;九品已上,一官當徒一年;當流者,三流同比徒三年。若犯公罪者,徒各加一年,當流者各加一等。其累徒過九年者,流二千里。



 定訖,詔頒之曰:「帝王作法,沿革不同,取適於時,故有損益。夫絞以致斃,斬則殊刑,除惡之體,於斯已極。梟首轘身,義無所取,不益懲肅之理,徒表安忍之懷。鞭之為用,殘剝膚體,徹骨侵肌,酷均臠切。雖雲遠古之式,事乖仁者之刑,梟轘及鞭,並令去也。貴礪帶之書,不當徒罰,廣軒冕之廕,旁及諸親。流役六年,改為五載,刑徒五歲,變從三祀。其餘以輕
 代重,化死為生,條目甚多,備於簡策。



 宜班諸海內,為時軌範,雜格嚴科,並宜除削。先施法令,欲人無犯之心,國有常刑,誅而不怒之義。措而不用,庶或非遠,萬方百闢,知吾此懷。」自前代相承,有司訊考,皆以法外。或有用大棒束杖,車輻鞋底,壓踝杖桄之屬,楚毒備至,多所誣伏。雖文致於法,而每有枉濫,莫能自理。至是盡除苛慘之法,訊囚不得過二百,枷杖大小,咸為之程品,行杖者不得易人。帝又以律令初行,人未知禁,故犯法者眾。又下吏承苛政之後,務鍛煉以致人罪。乃詔申敕四方,敦理辭訟。有枉屈縣不理者,令以次經郡及州,至省仍不理,乃
 詣闕申訴。有所未愜,聽撾登聞鼓,有司錄狀奏之。



 帝又每季親錄囚徒。常以秋分之前,省閱諸州申奏罪狀。三年,因覽刑部奏,斷獄數猶至萬條。以為律尚嚴密,故人多陷罪。又敕蘇威、牛弘等,更定新律。除死罪八十一條,流罪一百五十四條,徒杖等千餘條,定留唯五百條。凡十二卷。一曰名例,二曰衛禁,三曰職制,四曰戶婚,五曰廄庫,六曰擅興,七曰賊盜,八曰鬥訟,九曰詐偽,十曰雜律,十一曰捕亡,十二曰斷獄。自是刑網簡要,疏而不失。



 於是置律博士弟子員。斷決大獄,皆先牒明法,定其罪名,然後依斷。五年,侍官慕容天遠糾都督田元冒請義
 倉,事實,而始平縣律生輔恩舞文陷天遠,遂更反坐。



 帝聞之,乃下詔曰:「人命之重,懸在律文,刊定科條,俾令易曉。分官命職,恆選循吏,小大之獄,理無疑舛。而因襲往代,別置律官,報判之人,推其為首。殺生之柄,常委小人,刑罰所以未清,威福所以妄作,為政之失,莫大於斯。其大理律博士、尚書刑部曹明法、州縣律生,並可停廢。」自是諸曹決事,皆令具寫律文斷之。六年,敕諸州長史已下,行參軍已上,並令習律,集京之日,試其通不。又詔免尉迥、王謙、司馬消難三道逆人家口之配沒者,悉官酬贖,使為編戶。因除孥戮相坐之法,又命諸州囚有處死,
 不得馳驛行決。



 高祖性猜忌,素不悅學,既任智而獲大位,因以文法自矜,明察臨下。恆令左右覘視內外,有小過失,則加以重罪。又患令史贓污,因私使人以錢帛遺之,得犯立斬。每於殿廷打人,一日之中,或至數四。嘗怒問事揮楚不甚,即命斬之。十年,尚書左僕射高熲、治書侍御史柳彧等諫,以為朝堂非殺人之所,殿庭非決罰之地。



 帝不納。熲等乃盡詣朝堂請罪,曰:「陛下子育群生,務在去弊,而百姓無知,犯者不息,致陛下決罰過嚴。皆臣等不能有所裨益,請自退屏,以避賢路。」帝於是顧謂領左右都督田元曰:「吾杖重乎?」元曰:「重。」帝問其狀,元舉
 手曰:「陛下杖大如指,棰楚人三十者,比常杖數百,故多致死。」帝不懌,乃令殿內去杖,欲有決罰,各付所由。後楚州行參軍李君才上言帝寵高熲過甚,上大怒,命杖之,而殿內無杖,遂以馬鞭笞殺之。自是殿內復置杖。未幾怒甚,又於殿庭殺人,兵部侍朗馮基固諫,帝不從,竟於殿庭行決。帝亦尋悔,宣慰馮基,而怒群僚之不諫者。十二年,帝以用律者多致踳駁,罪同論異。詔諸州死罪不得便決,悉移大理案覆,事盡然後上省奏裁。十三年,改徒及流並為配防。十五年制,死罪者三奏而後決。十六年,有司奏合川倉粟少七千石,命斛律孝卿鞫問其事,
 以為主典所竊。



 復令孝卿馳驛斬之,沒其家為奴婢,鬻粟以填之。是後盜邊糧者,一升已上皆死,家口沒官。上又以典吏久居其職,肆情為奸。諸州縣佐史,三年一代,經任者不得重居之。十七年,詔又以所在官人,不相敬憚,多自寬縱,事難克舉。諸有殿失,雖備科條,或據律乃輕,論情則重,不即決罪,無以懲肅。其諸司屬官,若有愆犯,聽於律外斟酌決杖。於是上下相驅,迭行棰楚,以殘暴為幹能,以守法為懦弱。



 是時帝意每尚慘急,而奸回不止,京市白日,公行掣盜,人間強盜,亦往往而有。帝患之,問群臣斷禁之法,楊素等未及言,帝曰:「朕知之矣。」詔
 有能糾告者,沒賊家產業,以賞糾人。時月之間,內外寧息。其後無賴之徒,候富人子弟出路者,而故遺物於其前,偶拾取則擒以送官,而取其賞。大抵被陷者甚眾。帝知之,乃命盜一錢已上皆棄市。行旅皆晏起早宿,天下懍懍焉。此後又定制,行署取一錢已上,聞見不告言者,坐至死。自此四人共盜一榱桷,三人同竊一瓜,事發即時行決。有數人劫執事而謂之曰:「吾豈求財者邪?但為枉人來耳。而為我奏至尊,自古以來,體國立法,未有盜一錢而死也。而不為我以聞,吾更來,而屬無類矣。」



 帝聞之,為停盜取一錢棄市之法。



 帝嘗發怒,六月棒殺人。大
 理少卿趙綽固爭曰:「季夏之月,天地成長庶類。



 不可以此時誅殺。」帝報曰:「六月雖曰生長,此時必有雷霆。天道既於炎陽之時震其威怒,我則天而行,有何不可!」遂殺之。大理掌固來曠上封事,言大理官司恩寬。帝以曠為忠直,遣每旦於五品行中參見。曠又告少卿趙綽濫免徒囚,帝使信臣推驗,初無阿曲。帝又怒曠,命斬之。綽因固爭,以為曠不合死。帝乃拂衣入閣,綽又矯言,臣更不理曠,自有他事未及奏聞。帝命引入閤,綽再拜請曰:「臣有死罪三。臣為大理少卿,不能制馭掌固,使曠觸掛天刑,死罪一也。囚不合死,而臣不能死爭,死罪二也。臣本
 無他事,而妄言求入,死罪三也。」帝解顏。會獻皇后在坐,帝賜綽二金杯酒,飲訖,並以杯賜之。曠因免死,配徒廣州。



 帝以年齡晚暮,尤崇尚佛道,又素信鬼神。二十年,詔沙門道士壞佛像天尊,百姓壞岳瀆神像,皆以惡逆論。帝猜忌,二朝臣僚,用法尤峻。御史監師,於元正日不劾武官衣劍之不齊者,或以白帝,帝謂之曰:「爾為御史,何縱舍自由。」命殺之。諫議大夫毛思祖諫,又殺之。左領軍府長史考校不平,將作寺丞以諫麥簹遲晚,武庫令以署庭荒蕪,獨孤師以受蕃客鸚鵡,帝察知,並親臨斬決。



 仁壽中,用法益峻,帝既喜怒不恆,不復依準科律。時楊
 素正被委任,素又稟性高下,公卿股慄,不敢措言。素於鴻臚少卿陳延不平,經蕃客館,庭中有馬屎,又庶僕氈上樗蒲。旋以白帝,帝大怒曰:「主客令不灑掃庭內,掌固以私戲污敗官氈,罪狀何以加此!」皆於西市棒殺,而榜棰陳延,殆至於斃。大理寺丞楊遠、劉子通等,性愛深文,每隨牙奏獄,能承順帝旨。帝大悅,並遣於殿庭三品行中供奉,每有詔獄,專使主之。候帝所不快,則案以重抵,無殊罪而死者,不可勝原。遠又能附楊素,每於途中接候,而以囚名白之,皆隨素所為輕重。其臨終赴市者,莫不途中呼枉,仰天而哭。越公素侮弄朝權,帝亦不之能
 悉。



 煬帝即位,以高祖禁網深刻,又敕修律令,除十惡之條。時鬥稱皆小舊二倍,其贖銅亦加二倍為差。杖百則三十斤矣。徒一年者六十斤,每等加三十斤為差,三年則一百八十斤矣。流無異等,贖二百四十斤。二死同贖三百六十斤。其實不異。



 開皇舊制,X門子弟,不得居宿衛近侍之官。先是蕭嚴以叛誅,崔君綽坐連庶人勇事,家口籍沒。嚴以中宮故,君綽緣女入宮愛幸,帝乃下詔革前制曰:「罪不及嗣,既弘至孝之道,恩由義斷,以勸事君之節。故羊鮒從戮,彌見叔向之誠,季布立勛,無預丁公之禍,用能樹聲往代,貽範將來。朕虛己為政,思遵舊
 典,推心待物,每從寬政。六位成象,美厥含弘,一眚掩德,甚非謂也。諸犯罪被戮之門,期已下親,仍令合仕,聽預宿衛近侍之官。」



 三年,新律成。凡五百條,為十八篇。詔施行之,謂之《大業律》。一曰名例,二曰衛宮,三曰違制,四曰請求,五曰戶,六曰婚,七曰擅興,八曰告劾,九曰賊,十曰盜,十一曰斗,十二曰捕亡,十三曰倉庫,十四曰廄牧,十五曰關市,十六曰雜,十七曰詐偽,十八曰斷獄。其五刑之內,降從輕典者,二百餘條。其枷杖決罰訊囚之制,並輕於舊。是時百姓久厭嚴刻,喜於刑寬。後帝乃外征四夷,內窮嗜欲,兵革歲動,賦斂滋繁。有司皆臨時迫脅,茍
 求濟事,憲章遐棄,賄賂公行,窮人無告,聚為盜賊。帝乃更立嚴刑,敕天下竊盜已上,罪無輕重,不待聞奏,皆斬。百姓轉相群聚,攻剽城邑,誅罰不能禁。帝以盜賊不息,乃益肆淫刑。九年,又詔為盜者籍沒其家。自是群賊大起,郡縣官人,又各專威福,生殺任情矣。及楊玄感反,帝誅之,罪及九族。其尤重者,行轘裂梟首之刑。或磔而射之。命公卿已下,臠啖其肉。百姓怨嗟,天下大潰,及恭帝即位,獄訟有歸焉。



\end{pinyinscope}