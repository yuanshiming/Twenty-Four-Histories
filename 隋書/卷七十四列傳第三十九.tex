\article{卷七十四列傳第三十九}

\begin{pinyinscope}

 酷吏夫為國之體有四焉:一曰仁義,二曰禮制,三曰法令,四曰刑罰。仁義禮制,政之本也,法令刑罰,政之末也。無本不立,無末不成。然教化遠而刑罰近,可以助化而不可以專行,可以立威而不可以繁用。《老子》曰:「其政察察,其人缺缺。」



 又曰:「法令滋章,盜賊多有。」然則令之煩苛,吏之嚴酷,不能致理,百代可知。



 考覽前載,有時而用之矣。昔
 秦任獄吏,赭衣滿道。漢革其風,矯枉過正。禁網疏闊,遂漏吞舟,大奸巨猾,犯義侵禮。故剛克之吏,摧拉兇邪,一切禁奸,以救時弊,雖垂教義,或有所取焉。高祖膺期,平一江左,四海九州,服教從義。至於威行郡國,力折公侯,乘傳賦人,探丸斫吏者,所在蔑聞焉。無曩時之弊,亦已明矣。



 士文等功不足紀,才行無聞,遭遇時來,叨竊非據,肆其褊性,多行無禮,君子小人,咸罹其毒。凡厥所蒞,莫不懍然。居其下者,視之如蛇虺,過其境者,逃之如寇仇。與人之恩,心非好善,加人之罪,事非疾惡。其所笞辱,多在無辜,察其所為,豺狼之不若也。無禁奸除猾之志,肆
 殘虐幼賤之心,君子惡之,故編為《酷吏傳》也。



 厙狄士文厙狄士文,代人也。祖乾,齊左丞相。父敬,武衛將軍、肆州刺史。士文性孤直,雖鄰里至親莫與通狎。少讀書。在齊襲封章武郡王,官至領軍將軍。周武帝平齊,山東衣冠多迎周師,唯士文閉門自守。帝奇之,授開府儀同三司、隨州刺史。



 高祖受禪,加上開府,封湖陂縣子,尋拜貝州刺史。性清苦,不受公料,家無餘財。



 其子常啖官廚餅,士文枷之於獄累日,杖之一百,步送還京。僮隸無敢出門,所買鹽菜,必於外境。凡有出入,皆封署其門,親舊絕跡,
 慶吊不通。法令嚴肅,吏人股戰,道不拾遺。有細過,必深文陷害。嘗入朝,遇上置酒高會,賜公卿入左藏,任取多少。人皆極重,士文獨口銜絹一匹,兩手各持一匹。上問其故,士文曰:「臣口手俱滿,餘無所須。」上異之,別加賞物,勞而遣之。士文至州,發擿奸隱,長吏尺布升粟之贓,無所寬貸。得千餘人而奏之,上悉配防嶺南,親戚相送,哭泣之聲遍於州境。至嶺南,遇瘴癘死者十八九,於是父母妻子唯哭士文。士文聞之,令人捕捉,撾捶盈前,而哭者彌甚。有京兆韋焜為貝州司馬,河東趙達為清河令,二人並苛刻,唯長史有惠政。時人為之語曰:「刺史羅剎
 政,司馬蝮蛇瞋,長史含笑判,清河生吃人。」上聞而嘆曰:「士文之暴,過於猛獸。」竟坐免。未幾,以為雍州長史,士文謂人曰:「我向法深,不能窺候要貴,必死此官矣。」及下車,執法嚴正,不避貴戚,賓客莫敢至門,人多怨望。士文從父妹為齊氏嬪,有色,齊滅之後,賜薛國公長孫覽為妾。覽妻鄭氏性妒,譖之於文獻後,後令覽離絕。士文恥之,不與相見。後應州刺史唐君明居母憂,娉以為妻,由是士文、君明並為御史所劾。士文性剛,在獄數日,憤恚而死。家無餘財,有子三人,朝夕不繼,親友無內之者。



 田式田式,字顯標,馮翊下邦人也。祖安興,父長樂,仕魏,俱為本郡太守。式性剛果,多武藝,拳勇絕人。周明帝時,年十八,授都督,領鄉兵。後數載,拜渭南太守,政尚嚴猛,吏人重足而立,無敢違法者。遷本郡太守,親故屏跡,請托不行。



 武帝聞而善之,進位儀同三司,賜爵信都縣公,擢拜延州刺史。從帝平齊,以功加上開府,徙為建州刺史,改封梁泉縣公。高祖總百揆,尉迥作亂鄴城,從韋孝寬擊之。以功拜大將軍,進爵武山郡公。及受禪,拜襄州總管,專以立威為務。每視事於外,必盛氣以待其下,官屬股慄,無敢仰視。有犯禁者,雖至親暱,無所容貸。



 其女婿京
 兆杜寧,自長安省之,式誡寧無出入。寧久之不得還,竊上北樓,以暢羈思。式知之,笞寧五十。其所愛奴,嘗詣式白事,有蟲上其衣衿,揮袖拂去之。式以為慢己,立棒殺之。或僚吏奸贓,部內劫盜者,無問輕重,悉禁地牢中,寢處糞穢,令其苦毒,自非身死,終不得出。每赦書到州,式未暇讀,先召獄卒,殺重囚,然後宣示百姓。其刻暴如此。由是為上所譴,除名為百姓。式慚恚不食,妻子至其所,輒怒,唯侍僮二人給使左右。從家中索椒,欲以自殺,家人不與。陰遣所侍僮詣市買毒藥,妻子又奪而棄之。式恚臥。其子信時為儀同,至式前流涕曰:「大人既是朝廷
 舊臣,又無大過。比見公卿放辱者多矣,旋復升用,大人何能久乎?乃至於此!」式欻然而起,抽刀斫信,信遽走避之,刃中於閾。上知之,以式為罪己之深,復其官爵。尋拜廣州總管,卒官。



 燕榮燕榮,字貴公,華陰弘農人也。父偘,周大將軍。榮性剛嚴,有武藝,仕周為內侍上士。從武帝伐齊,以功授開府儀同三司,封高邑縣公。高祖受禪,進位大將軍,封落叢郡公,拜晉州刺史。從河間王弘擊突厥,以功拜上柱國,遷青州總管。



 榮在州,選絕有力者為伍伯,吏人過之者,必
 加詰問,輒楚撻之,創多見骨。奸盜屏跡,境內肅然。他州縣人行經其界者,畏若寇仇,不敢休息。上甚善之。後因入朝覲,特加勞勉。榮以母老,請每歲入朝,上許之。及辭,上賜宴於內殿,詔王公作詩以餞之。伐陳之役,以為行軍總管,率水軍自東萊傍海,入太湖,取吳郡。既破丹陽,吳人共立蕭瓛為主,阻兵於晉陵,為宇文述所敗,退保包山。榮率精甲五千躡之,瓛敗走,為榮所執,晉陵、會稽悉平。檢校揚州總管。尋徵為右武候將軍。



 突厥寇邊,以為行軍總管,屯幽州。母憂去職。明年,起為幽州總管。榮性嚴酷,有威容,長史見者,莫不惶懼自失。範陽盧氏,代
 為著姓,榮皆署為吏卒以屈辱之。



 鞭笞左右,動至千數,流血盈前,飲啖自若。嘗按部,道次見叢荊,堪為笞棰,命取之,輒以試人。人或自陳無咎,榮曰:「後若有罪,當免爾。」及後犯細過,將撾之,人曰:「前日被杖,使君許有罪宥之。」榮曰:「無過尚爾,況有過邪!」



 榜棰如舊。榮每巡省管內,聞官人及百姓妻女有美色,輒舍其室而淫之。貪暴放縱日甚。是時元弘嗣被除為幽州長史,懼為榮所辱,固辭。上知之,敕榮曰:「弘嗣杖十已上罪,皆須奏聞。」榮忿曰:「豎子何敢弄我!」於是遣弘嗣監納倉粟,颺得一糠一秕,輒罰之。每笞雖不滿十,然一日之中,或至三數。如是歷年,
 怨隙日構,榮遂收付獄,禁絕其糧。弘嗣饑餒,抽衣絮,雜水咽之。其妻詣闕稱冤,上遣考功侍郎劉士龍馳驛鞫問。奏榮虐毒非虛,又賊穢狼籍,遂征還京師,賜死。先是,榮家寢室無故有蛆數斛,從地墳出。未幾,榮死於蛆出之處。有子詢。



 趙仲卿趙仲卿,天水隴西人也。父剛,周大將軍。仲卿性粗暴,有膂力,周齊王憲甚禮之。從擊齊,攻臨秦、統戎、威遠、伏龍、張壁等五城,盡平之。又擊齊將段孝先於姚襄城,苦戰連日,破之。以功授大都督,尋典宿衛。平齊之役,以功遷
 上儀同,兼趙郡太守。入為畿伯中大夫。王謙作亂,仲卿使在利州,即與總管豆盧勣發兵拒守。為謙所攻,仲卿督兵出戰,前後一十七陣。及謙平,進位大將軍,封長垣縣公,邑千戶。高祖受禪,進爵河北郡公。開皇三年,突厥犯塞,以行軍總管從河間王弘出賀蘭山。仲卿別道俱進,無虜而還。復鎮平涼,尋拜石州刺史。法令嚴猛,纖微之失,無所容舍,鞭笞長史,輒至二百。官人戰心慄,無敢違犯,盜賊屏息,皆稱其能。遷兗州刺史,未之官,拜朔州總管。於時塞北盛興屯田,仲卿總管統之。



 微有不理者,仲卿輒召主掌,撻其胸背,或解衣倒曳於荊棘中。時人謂之
 猛獸。事多克濟,由是收獲歲廣,邊戍無饋運之憂。會突厥啟民可汗求婚於國,上許之。仲卿因是間其骨肉,遂相攻擊。十七年,啟民窘迫,與隋使長孫晟投通漢鎮。仲卿率騎千餘馳援之,達頭不敢逼。潛遣人誘致啟民所部,至者二萬餘家。其年,從高熲指白道以擊達頭。仲卿率兵三千為前鋒,至族蠡山,與虜相遇,交戰七日,大破之。



 追奔至乞伏泊,復破之,虜千餘口,雜畜萬計。突厥悉眾而至,仲卿為方陣,四面拒戰。經五日,會高熲大兵至,合擊之,虜乃敗走。追度白道,逾秦山七百餘里。



 時突厥降者萬餘家,上命仲卿處之恆安。以功進位上柱國,賜
 物三千段。朝廷慮達頭掩襲啟民,令仲卿屯兵二萬以備之,代州總管韓洪、永康公李藥王、蔚州刺史劉隆等,將步騎一萬鎮恆安。達頭騎十萬來寇,韓洪軍大敗,仲卿自樂寧鎮邀擊,斬首虜千餘級。明年,督役築金河、定襄二城,以居啟民。時有表言仲卿酷暴者,上令御史王偉按之,並實,惜其功不罪也。因勞之曰:「知公清正,為下所惡。」賜物五百段。仲卿益恣,由是免官。仁壽中,檢校司農卿。蜀王秀之得罪,奉詔往益州窮按之。秀賓客經過之處,仲卿必深文致法,州縣長吏坐者太半。上以為能,賞婢奴五十口,黃金二百兩,米粟五千石,奇寶雜物稱
 是。煬帝嗣位,判兵部、工部二曹尚書事。其年,卒,時年六十四。謚曰肅。贈物五百段。子弘嗣。



 崔弘度弟弘升崔弘度,字摩訶衍,博陵安平人也。祖楷,魏司空。父說,周敷州刺史。弘度膂力絕人,儀貌魁岸,須面甚偉。性嚴酷。年十七,周大塚宰宇文護引為親信。尋授都督,累轉大都督。時護子中山公訓為蒲州刺史,令弘度從焉。嘗與訓登樓,至上層,去地四五丈,俯臨之,訓曰:「可畏也。」弘度曰:「此何足畏!」欻然擲下,至地無損傷。訓以其拳捷,大奇之。後以戰勛,授儀同。從武帝滅齊,進位上開府,鄴縣公,賜
 物三千段,粟麥三千石,奴婢百口,雜畜千計。尋從汝南公宇文神舉破盧昌期於範陽。宣帝嗣位,從鄖國公韋孝寬經略淮南。弘度與化政公宇文忻、司水賀婁子干至肥口,陳將潘琛率兵數千來拒戰,隔水而陣。忻遣弘度諭以禍福,琛至夕而遁。進攻壽陽,降陳守將吳文立,弘度功最。以前後勛,進位上大將軍,襲父爵安平縣公。及尉迥作亂,以弘度為行軍總管,從韋孝寬討之。弘度募長安驍雄數百人為別隊,所當無不披靡。弘度妹先適迥子為妻,及破鄴城,迥窘迫升樓,弘度直上龍尾追之。迥彎弓將射弘度,弘度脫兜鍪謂迥曰:「相識不?今日
 各圖國事,不得顧私。以親戚之情,謹遏亂兵,不許侵辱。事勢如此,早為身計,何所待也?」迥擲弓於地,罵大丞相極口而自殺。弘度顧其弟弘升曰:「汝可取迥頭。」



 弘升遂斬之,進位上柱國。時行軍總管例封國公,弘度不時殺迥,致縱惡言,由是降爵一等,為武鄉郡公。開皇初,突厥入寇,弘度以行軍總管出原州以拒之。虜退,弘度進屯靈武。月餘而還,拜華州刺史。納其妹為秦孝王妃。尋遷襄州總管。弘度素貴,御下嚴急,動行捶罰,吏人讋氣,聞其聲,莫不戰慄。所在之處,令行禁止,盜賊屏跡。梁王蕭琮來朝,上以弘度為江陵總管,鎮荊州。弘度未至,而琮
 叔父嚴擁居人以叛,弘度追之不及。陳人憚弘度,亦不敢窺荊州。平陳之役,以行軍總管從秦孝王出襄陽道。及陳平,賜物五千段。高智慧等作亂,復以行軍總管出泉門道,隸於楊素。弘度與素,品同而年長,素每屈下之,一旦隸素,意甚不平,素言多不用。素亦優容之。及還,檢校原州事,仍領行軍總管以備胡,無虜而還,上甚禮之,復以其弟弘升女為河南王妃。仁壽中,檢校太府卿。自以一門二妃,無所降下,每誡其僚吏曰:「人當誠恕,無得欺誑。」皆曰:「諾。」後嘗食鱉,侍者八九人,弘度一一問之曰:「鱉美乎?」人懼之,皆云:「鱉美。」弘度大罵曰:「傭奴何敢誑我?
 汝初未食鱉,安知其美?」俱杖八十。官屬百工見之者,莫不流汗,無敢欺隱。時有屈突蓋為武候驃騎,亦嚴刻,長安為之語曰:「寧飲三升酢,不見崔弘度。寧茹三升艾,不逢屈突蓋。」然弘度理家如官,子弟斑白,動行捶楚,閨門整肅,為當時所稱。未幾,秦王妃以罪誅,河南王妃復被廢黜。弘度憂恚,謝病於家,諸弟乃與之別居,彌不得志。煬帝即位,河南王為太子,帝將復立崔妃,遣中使就第宣旨。使者詣弘升家,弘度不之知也。使者返,帝曰:「弘度有何言?」使者曰:「弘度稱有疾不起。」帝默然,其事竟寢。弘度憂憤,未幾,卒。



 弘升字上客,在周為右侍上士。尉迥作亂相州,與兄弘度擊之,以功拜上儀同。



 尋加上開府,封黃臺縣侯,邑八百戶。高祖受禪,進爵為公,授驃騎將軍。宿衛十餘年,以勛舊遷慈州刺史。數歲,轉鄭州刺史。後以戚屬之故,待遇愈隆,遷襄州總管。及河南王妃罪廢,弘升亦免官。煬帝即位,歷冀州刺史、信都太守,進位金紫光祿大夫,轉涿郡太守。遼東之役,檢校左武衛大將軍事,指平壤。與宇文述等同敗績,奔還,發病而卒,時年六十。



 元弘嗣元弘嗣,河南洛陽人也。祖剛,魏漁陽王。父經,周漁陽郡
 公。弘嗣少襲爵,十八為左親衛。開皇九年,從晉王平陳,以功授上儀同。十四年,除觀州總管長史,在州專以嚴峻任事,吏人多怨之。二十年,轉幽州總管長史。於時燕榮為總管,肆虐於弘嗣,每被笞辱。弘嗣心不伏,榮遂禁弘嗣於獄,將殺之。及榮誅死,弘嗣為政,酷又甚之。每推鞫囚徒,多以酢灌鼻,或襜弋其下竅,無敢隱情,奸偽屏息。



 仁壽末,授木工監,修營東都。大業初,煬帝潛有取遼東之意,遣弘嗣往東萊海口監造船。諸州役丁苦其捶楚,官人督役,晝夜立於水中,略不敢息,自腰以下,無不生蛆,死者十三四。尋遷黃門侍郎,轉殿內少監。遼東之
 役,進位金紫光祿大夫。



 明年,帝復征遼東,會奴賊寇隴右,詔弘嗣擊之。及玄感作亂,逼東都,弘嗣屯兵安定。或告之謀應玄感者,代王侑遣使執之,送行在所。以無反形當釋,帝疑不解,除名,徙日南,道死,時年四十九。有子仁觀。



 王文同王文同,京兆潁陽人也。性明辯,有乾用。開皇中,以軍功拜儀同,尋授桂州司馬。煬帝嗣位,徵為光祿少卿,以忤旨,出為恆山郡丞。有一人豪猾,每持長吏長短,前後守令咸憚之。文同下車,聞其名,召而數之。因令左右剡木
 為大橛,埋之於庭,出尺餘,四角各埋小橛。令其人踣心於木橛上,縛四支於小橛,以棒毆其背,應時潰爛。郡中大駭,吏人相視懾氣。及帝征遼東,令文同巡察河北諸郡。文同見沙門齋戒菜食者,以為妖妄,皆收系獄。比至河間,召諸郡官人,小有遲違者,輒皆覆面於地而箠殺之。求沙門相聚講論,及長老共為佛會者數百人,文同以為聚結惑眾,盡斬之。又悉裸僧尼,驗有淫狀非童男女者數千人,復將殺之。郡中士女號哭於路,諸郡驚駭,各奏其事。帝聞而大怒,遣使者達奚善意馳鎖之,斬於河間,以謝百姓,仇人剖其棺,臠其肉而啖之,斯須咸盡。



 史臣曰:御之良者,不在於煩策,政之善者,無取於嚴刑。故雖寬猛相資,德刑互設,然不嚴而化,前哲所重。士文等運屬欽明,時無桀黠,未閑道德,實懷殘忍。賊人肌體,同諸木石,輕人性命,甚於芻狗。長惡不悛,鮮有不及,故或身嬰罪戮,或憂恚顛隕。凡百君子,以為有天道焉。嗚呼!後來之士,立身從政,縱不能為子高門以待封,其可令母掃墓而望喪乎?



\end{pinyinscope}