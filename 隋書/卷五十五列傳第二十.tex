\article{卷五十五列傳第二十}

\begin{pinyinscope}

 杜彥杜彥,雲中人也。父遷,屬葛榮之亂,徙家於幽。彥性勇果,善騎射。仕周,釋褐左侍上士,後從柱國陸通擊陳將吳明徹於土州,破之。又擊叛蠻,克倉塠、白楊二柵,並斬其渠帥。進平郢州賊帥樊志,以戰功拜大都督。尋遷儀同,治隆山郡事。明年,拜隴州刺史,賜爵永安縣伯。高祖為丞相,從韋孝寬擊尉迥於相州,每戰有功,賜物三千段,
 奴婢三十口。進位上開府,改封襄武縣侯,拜魏郡太守。開皇初,授丹州刺史,進爵為公。後六歲,徵為左武衛將軍。平陳之役,以行軍總管與新義公韓擒相繼而進。軍至南陵,賊屯據江岸,彥遣儀同樊子蓋率精兵擊破其柵,獲船六百餘艘。渡江,擊南陵城,拔之,擒其守將許翼。進至新林,與擒合軍。及陳平,賜物五千段,粟六千石,進位柱國,賜子寶安爵昌陽縣公。高智慧等之作亂也,復以行軍總管從楊素討之,別解江州圍。智慧餘黨往往屯聚,保投溪洞,彥水陸兼進,攻錦山、陽父、若、石壁四洞,悉平之,皆斬其渠帥。賊李陀擁眾數千,據彭山,彥襲擊
 破之,斬陀,傳其首。又擊徐州、宜豐二洞,悉平之。賜奴婢百餘口。拜洪州總管,甚有治名。



 歲餘,雲州總管賀婁子乾卒,上悼惜者久之,因謂侍臣曰:「榆林國之重鎮,安得子乾之輩乎?」後數日,上曰:「吾思可以鎮榆林者,莫過杜彥。」於是徵拜雲州總管。突厥來寇,彥輒擒斬之,北夷畏憚,胡馬不敢至塞。後數年,朝廷復追錄前功,賜子寶虔爵承縣公。十八年,遼東之役,以行軍總管從漢王至營州。上以彥曉習軍旅,令總統五十營事。及還,拜朔州總管。突厥復寇雲州,上令楊素擊走之,是後猶恐為邊患,以彥素為突厥所憚,復拜雲州總管。未幾,以疾徵還,卒,
 時年六十。子寶虔,大業末,文城郡丞。



 高勱高勱,字敬德,渤海蓚人也,齊太尉、清河王岳之子也。幼聰敏,美風儀,以仁孝聞,為齊顯祖所愛。年七歲,襲爵清河王。十四為青州刺史,歷右衛將軍、領軍大將軍、祠部尚書、開府儀同三司,改封樂安王。性剛直,有才幹,甚為時人所重。斛律明月雅敬之,每有征伐,則引之為副。遷侍中、尚書右僕射。及後主為周師所敗,勱奉太后歸鄴。時宦官放縱,儀同茍子溢尤稱寵幸,勱將斬之以徇。太后救之,乃釋。劉文殊竊謂勱曰:「子溢之徒,言成禍福,何得
 如此!」勱攘袂曰:「今者西寇日侵,朝貴多叛,正由此輩弄權,致使衣冠解體。若得今日殺之,明日受誅,無所恨也。」文殊甚愧。既至鄴,勱勸後主:「五品已上家累,悉置三臺之上,因脅之曰:『若戰不捷,則燒之。』此輩惜妻子,必當死戰,可敗也。」後主不從,遂棄鄴東遁。勱恆後殿,為周軍所得。武帝見之,與語,大悅,因問齊亡所由。勱發言流涕,悲不自勝,帝亦為之改容。授開府儀同三司。



 高祖為丞相,謂勱曰:「齊所以亡者,由任邪佞。公父子忠良,聞於鄰境,宜善自愛。」勱再拜謝曰:「勱亡齊末屬,世荷恩榮,不能扶危定傾,以致淪覆。既蒙獲宥,恩幸已多,況復濫叨名位,
 致速官謗。」高祖甚器之,以勱檢校揚州事。



 後拜楚州刺史,民安之。先是,城北有伍子胥廟,其俗敬鬼。祈禱者必以牛酒,至破產業。勱嘆曰:「子胥賢者,豈宜損百姓乎?」乃告諭所部,自此遂止,百姓賴之。



 七年,轉光州刺史,上取陳五策,又上表曰:「臣聞夷兇翦暴,王者之懋功;取亂侮亡,往賢之雅誥。是以苗民逆命,爰興兩階之舞;有扈不賓,終召六師之伐。



 皆所以寧一宇內,匡濟群生者也。自昔晉氏失馭,天網絕維,群兇於焉胃起,三方因而鼎立。陳氏乘其際運,拔起細微,茜頊縱其長蛇,竊據吳會;叔寶肆其昏虐,毒被金陵。數年已來,荒悖滋甚。牝雞司旦,
 暱近奸回,尚方役徒,積骸千數,疆埸防守,長戍三年。或微行暴露,沉湎王侯之宅;或奔馳駿騎,顛墜康衢之首。有功不賞,無辜獲戮,烽燧日警,未以為虞,耽淫靡嫚,不知紀極。天厭亂德,妖實人興,或空裏時有大聲,或行路共傳鬼怪,或刳人肝以祠天狗,或自舍身以厭妖訛。



 民神怨憤,災異薦發,天時人事,昭然可知。臣以庸才,猥蒙朝寄,頻歷籓任,與其鄰接,密邇仇讎,知其動靜,天討有罪,此即其時。若戎車雷動,戈船電邁,臣難驚怯,請效鷹犬。」高祖覽表嘉之,答以優詔。及大舉伐陳,以勱為行軍總管,從宜陽公王世積下陳江州。以功拜上開府,賜物
 三千段。



 隴右諸羌數為寇亂,朝廷以勱有威名,拜洮州刺史。下車大崇威惠,民夷悅附,其山谷間生羌相率詣府稱謁,前後至者,數千餘戶。豪猾屏跡,路不拾遺,在職數年,稱為治理。後遇吐谷渾來寇,勱遇疾不能拒戰,賊遂大掠而去。憲司奏勱亡失戶口,又言受羌饋遺,竟坐免官。後卒於家,時年五十六。子士廉,最知名。



 爾硃敞爾硃敞,字乾羅,秀容契胡人,爾硃榮之族子也。父彥伯,官至司徒、博陵王。



 齊神武帝韓陵之捷,盡誅爾硃氏,敞小,隨母養於宮中。及年十二,自竇而走,至於大街,見童
 兒群戲者,敞解所著綺羅金翠之服,易衣而遁。追騎尋至,初不識敞,便執綺衣兒。比究問知非,會日已暮,由是得免。遂入一村,見長孫氏媼踞胡床而坐。敞再拜求哀,長孫氏愍之,藏於復壁。三年,購之愈急,跡且至,長孫氏曰:「事急矣,不可久留。」資而遣之。遂詐為道士,變姓名,隱嵩山,略涉經史。數年之間,人頗異之。嘗獨坐巖石之下,泫然而嘆曰:「吾豈終於此乎?伍子胥獨何人也!」於是間行微服,西歸於周。太祖見而禮之,拜大都督、行臺郎中,封靈壽縣伯,邑千五百戶。遷通直散騎常侍,轉車騎大將軍、儀同三司,進爵為侯。保定中,遷使持節、驃騎大將
 軍、開府儀同三司。天和中,增邑五百戶,歷信、臨、熊、潼四州刺史,進爵為公。武帝東征,上表求從,許之。攻城陷陣,所當皆破,進位上開府。除南光州刺史,入為護軍大將軍。歲餘,轉膠州刺史。於是迎長孫氏及弟置於家,厚資給之。高祖受禪,改封邊城郡公。黔安蠻叛,命敞討平之。師旋,拜金州總管。尋轉徐州總管。在職數年,號為明肅,民吏懼之。後以年老,上表乞骸骨,賜二馬軺車,歸於河內,卒於家,時年七十二。子最嗣。



 周搖周搖,字世安,其先與後魏同源,初為普乃氏,及居洛陽,
 改為周氏。曾祖拔拔,祖右六肱,俱為北平王。父恕延,歷行臺僕射、南荊州總管。搖少剛果,有武藝,性謹厚,動遵法度。仕魏,官至開府儀同三司。周閔帝受禪,賜姓車非氏,封金水郡公。歷夙、楚二州刺史,吏民安之。從帝平齊,每戰有功,超授柱國,進封夔國公。未幾,拜晉州總管。時高祖為定州總管,文獻皇后自京師詣高祖,路經搖所,主禮甚薄。既而白後曰:「公廨甚富於財,限法不敢輒費。又王臣無得效私。」



 其質直如此。高祖以其奉法,每嘉之。及為丞相,徙封濟北郡公,尋拜豫州總管。



 高祖受禪,復姓周氏。開皇初,突厥寇邊,燕、薊多被其患,前總管李崇
 為虜所殺,上思所以鎮之,臨朝曰:「無以加周搖者。」拜為幽州總管六州五十鎮諸軍事。搖修鄣塞,謹斥候,邊民以安。後六載,徙為壽州。初,自以年老,乞骸骨,上召之。



 既引見,上勞之曰:「公積行累仁,歷仕三代,克終富貴,保茲遐壽,良足善也。」



 賜坐褥,歸於第。歲餘,終於家,謚曰恭,時年八十四。



 獨孤楷獨孤楷,字修則,不知何許人也,本姓李氏。父屯,從齊神武帝與周師戰於沙苑,齊師敗績,因為柱國獨孤信所擒,配為士伍,給使信家,漸得親近,因賜姓獨孤氏。楷少
 謹厚,便弄馬槊,為宇文護執刀,累轉車騎將軍。其後數從征伐,賜爵廣阿縣公,邑千戶,拜右侍下大夫。周末,從韋孝寬平淮南,以功賜子景雲爵西河縣公。高祖為丞相,進授開府,每督親信兵。及受禪,拜右監門將軍,進封汝陽郡公。數歲,遷右衛將軍。仁壽初,出為原州總管。時蜀王秀鎮益州,上征之,猶豫未發。朝廷恐秀生變,拜楷益州總管,馳傳代之。秀果有異志,楷諷諭久之,乃就路。楷察秀有悔色,因勒兵為備。秀至興樂,去益州四十餘里,將反襲楷,密令左右覘所為,知楷不可犯而止。楷在益州,甚有惠政,蜀中父老於今稱之。煬帝即位,轉並州
 總管。遇疾喪明,上表乞骸骨。帝曰:「公先朝舊臣,歷職二代,高風素望,臥以鎮之,無勞躬親簿領也。」遣其長子凌雲監省郡事。其見重如此。數載,轉長平太守,未視事而卒。謚曰恭。子凌雲、平雲,彥云,皆知名。楷弟盛,見《誠節傳》。



 乞伏慧乞伏慧,字令和,馬邑鮮卑人也。祖周,魏銀青光祿大夫,父纂,金紫光祿大夫,並為第一領民酋長。慧少慷慨有大節,便弓馬,好鷹犬。齊文襄帝時,為行臺左丞,加蕩寇將軍,累遷右衛將軍、太僕卿,自永寧縣公封宜民郡王。
 其兄貴和又以軍功為王,一門二王,稱為貴顯。周武平齊,授使持節、開府儀同大將軍,拜佽飛右旅下大夫,轉熊渠中大夫。高祖為丞相,從韋孝寬擊尉惇於武陟,所當皆破,授大將軍,賜物八百段。及平尉迥,進位柱國,賜爵西河郡公,邑三千戶,賚物二千三百段。請以官爵讓兄,朝廷不許,論者義之。高祖受禪,拜曹州刺史。曹土舊俗,民多奸隱,戶口簿帳,恆不以實。慧下車按察,得戶數萬。遷涼州總管。先是,突厥屢為寇抄,慧於是嚴警烽燧,遠為斥候,虜亦素憚其名,竟不入境。歲餘,轉齊州刺史,得隱戶數千。遷壽州總管。其年,左轉杞州刺史,在職數
 年,遷徐州總管。時年逾七十,上表求致仕,不許。俄轉荊州總管,又領潭、桂二州總管三十一州諸軍事。其俗輕剽,慧躬行樸素以矯之,風化大洽。曾見人以穀捕魚者,出絹買而放之,其仁心如此。百姓美之,號其處曰西河公穀。轉秦州總管。煬帝即位,為天水太守。大業五年,征吐谷渾,郡濱西境,民苦勞役,又遇帝西巡,坐為道不整,獻食疏薄,帝大怒,命左右斬之。見其無發,乃釋,除名為民。卒於家。



 張威張威,不知何許人也。父琛,魏弘農太守。威少倜儻有大
 志,善騎射,膂力過人。在周,數從征伐,位至柱國、京兆尹,封長壽縣公,邑千戶。王謙作亂,高祖以威為行軍總管,從元帥梁睿擊之。軍次通谷,謙守將李三王擁勁兵拒守,睿以威為先鋒。三王初閉壘不戰,威令人詈侮以激怒之,三王果出陣。威令壯士奮擊,三王軍潰,大兵繼至,於是擒斬四千餘人。進至開遠,謙將趙儼眾十萬,連營三十里。



 威鑿山通道,自西嶺攻其背,儼遂敗走。追至成都,與謙大戰,威將中軍。及謙平,進位上柱國,拜瀘州總管。高祖受禪,歷幽、洛二州總管,改封晉熙郡公。尋拜河北道行臺僕射,後督晉王軍府事。數年,拜青州總管,賜
 錢八十萬,米五百石,雜彩三百段。威在青州,頗治產業,遣家奴於民間鬻蘆菔根,其奴緣此侵擾百姓。上深加譴責,坐廢於家。後從上祠太山,至洛陽,上謂威曰:「自朕之有天下,每委公以重鎮,可謂推赤心矣。何乃不修名行,唯利是視?豈直孤負朕心,亦且累卿名德。」因問威曰:「公所執笏今安在?」威頓首曰:「臣負罪虧憲,無顏復執,謹藏於家。」上曰:「可持來。」威明日奉笏以見,上曰:「公雖不遵法度,功效實多,朕不忘之。今還公笏。」於是復拜洛州刺史,後封睆城郡公。尋轉相州刺史,卒官。有子植,大業中,至武賁郎將。



 和洪和洪,汝南人也。少有武力,勇烈過人。周武帝時,數從征伐,以戰功累遷車騎大將軍、儀同三司。時龍州蠻任公忻、李國立等聚眾為亂,刺史獨孤善不能御。



 朝議以洪有武略,代善為刺史。月餘,擒公忻、國立,皆斬首梟之,餘黨悉平。從帝攻河陰,洪力戰,陷其西門。帝壯之,賞物千段。復從帝平齊,進位上儀同,賜爵北平侯,邑八百戶,拜左勛曹下大夫。柱國王軌之擒吳明徹也,洪有功焉,加位開府,遷折沖中大夫。尉迥作亂相州,以洪為行軍總管,從韋孝寬擊之。軍至河陽,迥遣兵圍懷州,洪與總管
 宇文述等擊走。又破尉惇於武陟。及平相州,每戰有功,拜柱國,封廣武郡公,邑二千戶。前後賜物萬段,奴婢五十口,金銀各百挺,牛馬百匹。時東夏初平,物情尚梗,高祖以洪有威名,令領冀州事,甚得人和。數歲,徵入朝,為漕渠總管監,轉拜泗州刺史。屬突厥寇邊,詔洪為北道行軍總管,擊走虜,至磧而還。後遷徐州總管,卒,時年六十四。



 侯莫陳穎侯莫陳穎,字遵道,代人也。與魏南遷,世為列將。父崇,魏、周之際,歷職顯要,官至大司空。穎少有器量,風神警發,
 為時輩所推。魏大統末,以父軍功賜爵廣平侯,累遷開府儀同三司。周武帝時,從滕王逌擊龍泉、文城叛胡,與柱國豆盧勣各帥兵分路而進。穎懸軍五百餘里,破其三柵。先是,稽胡叛亂,輒略邊人為奴婢。至是詔胡敢有壓匿良人者誅,籍沒其妻子。有人言為胡村所隱匿者,勣將誅之,穎謂勣曰:「將在外,君命有所不行。諸胡固非悉反,但相迫脅為亂耳。大兵臨之,首亂者知懼,脅從者思降。今漸加撫慰,自可不戰而定。如即誅之,轉相驚恐,為難不細。未若召其渠帥,以隱匿者付之,令自歸首,則群胡可安。」勣從之。



 群胡感悅,爭來降附,北土以安。遷司
 武,加振威中大夫。高祖為丞相,拜昌州刺史。會受禪,竟不行,加上開府,進爵升平郡公。俄拜延州刺史。數年,轉陳州刺史。平陳之役,以行軍總管從秦王俊出魯山道。屬陳將荀法尚、陳紀降,穎與行軍總管段文振度江安集初附。尋拜饒州刺史,未之官,遷瀛州刺史,甚有惠政。在職數年,坐與秦王俊交通免官。百姓將送者,莫不流涕,因相與立碑,頌穎清德。未幾,檢校汾州事,俄拜邢州刺史。仁壽中,吏部尚書牛弘持節巡撫山東,以穎為第一。高祖嘉嘆,優詔褒揚。時朝廷以嶺南刺史、縣令多貪鄙,蠻夷怨叛,妙簡清吏以鎮撫之,於是征穎入朝。及進見,上與穎言及平生,以為歡笑。數日,進位大將軍,拜桂州總管十七州諸軍事,賜物而遣之。及到官,大崇恩信,民夷悅服,溪洞生越,多來歸附。煬帝即位,穎兄梁國公芮坐事徙邊,朝廷恐穎不自安,征歸京師。



 數年,拜恆山太守。其年,嶺南、閩越多不附,帝以穎前在桂州有惠政,為南土所信伏,復拜南海太守。後四歲,卒官。謚曰定。子虔會,最知名。



 史臣曰:杜彥東夏、南服,屢有戰功,作鎮朔垂,胡塵不起。高勱死亡之際,志氣懍然,疾彼奸邪,致茲餘慶。爾硃敞幼有權奇,終能止足,崇基墜而復構,不亦仁且智乎!周搖以質實見知,獨孤以恤人流譽,乞伏慧能以國讓,侯莫陳所居治理,或知牧人之道,或踐仁義之路,皆有可稱焉。慧以供帳不厚,至於放黜,並結發登朝,出入三代,終享祿位,不夭性齡,蓋其任心而行,不為矯飾之所致也。



\end{pinyinscope}