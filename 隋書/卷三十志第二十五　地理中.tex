\article{卷三十志第二十五 地理中}

\begin{pinyinscope}

 河南郡舊置洛州。大業元年移都,改曰豫州。東面三門,北曰上春,中曰建陽,南曰永通。南面二門,東曰長夏,正南曰建國。里一百三,市三。三年改為郡,置尹。統縣十八,戶二十萬二千二百三十。



 河南帶郡。有關官。有郟山。有瀍水。洛陽有漢已來舊都。後魏置司州,東魏改曰洛州。後周置東京六府、洛州總管。開皇元年改六府,置東京尚書省。其年廢東京尚書省。二年廢總管,置河南道行臺省。三年廢行臺,以洛州刺史領總監。十四年於金墉城別置總監。煬帝即位,廢省。置河南縣,東魏遷鄴,改為宜遷縣。後周復曰河南。大業元年徙入新都。又東魏置洛陽郡、河陰縣。開皇初郡並廢,又析置伊川縣。
 大業初河陰、伊川二縣並入焉。閿鄉舊曰湖城,開皇十六年改焉。有王澗、全鳩澗、秦山。桃林開皇十六年置。有上陽宮。有淄水。陜後魏置,及置陜州、恆農郡。後周又置崤郡。開皇初郡並廢。大業初州廢,置弘農宮。有常平倉、溫湯。



 有砥柱。熊耳後周置,有同軌郡。開皇初郡廢。又有後魏崤縣,大業初廢入。有二崤。有天柱山、大頭山、硤石山、穀水。澠池後周置河南郡,大象中廢。新安後周置中州及東垣縣,州尋廢。開皇十六年置穀州,仁壽四年州廢,又廢新安入東垣。



 大業初改名新安。有冶官。有騩山、強山、缺門山、孝水、澗水、金谷水。偃師舊廢,開皇十六年置。有關官。有河陽倉。有都尉府。有首陽山、酈山、乾脯山。鞏後齊廢,開皇十六年復。有興洛倉。有九山,有天陵山、緱山、東首陽山。宜陽後魏置宜陽郡,東魏置陽州,後周改曰熊州。又復後魏置南澠池縣。後周改曰昌洛。



 開皇初郡廢。十八年改昌洛曰洛水。大業初廢熊州,省洛水入宜陽。又東魏置金門郡,後周廢。有福昌宮、金門山、女幾山、太陰山、嶕嶢山。壽安後魏置縣曰甘棠,仁壽四年改焉。有顯仁宮。有慈澗。陸渾東魏置伊川郡,領南陸渾縣。開皇初廢郡,改縣曰伏流。大業初改曰陸渾。又有
 東魏北荊州,後周改曰和州。開皇初又改曰伊州。大業初州廢。又有東魏東亭縣,尋廢。有方山、三塗山、孤山、陽山、王母澗。



 伊闕舊曰新城,東魏置新城郡。開皇初郡廢。十八年縣改名焉。有伊闕山。興泰大業初置。有鹿歸山、石墨山、鐘山。緱氏舊廢,東魏置。開皇十六年廢,大業初又置。有緱氏山、轘轅山、景山。崇陽後魏置,曰潁陽。東魏分置堙陽,後周廢潁陽入。開皇六年改曰武林。十八年改曰輪氏,大業元年改曰嵩陽。又有東魏中川郡,後周廢。有嵩高山、少室山、潁水。陽城後魏置陽城郡,開皇初廢。十六年置嵩州,仁壽四年廢。又後魏置康城縣,仁壽四年廢入焉。有箕山、偃月山、荊山、禹山、崤山。



 滎陽郡舊鄭州。開皇十六年置管州。大業初復曰鄭州。統縣十一,戶十六萬九百六十四。



 管城舊曰中牟,東魏置廣武郡。開皇初郡廢,改中牟曰內牟。十六年析置管城。



 十八年改內牟曰圃田入焉。後魏置曲梁縣,後齊廢,有鄭水。汜水舊曰成皋,即武牢也。後魏置東中府,東魏置北豫州,後周置滎州。開皇初曰鄭州,十八年改成皋曰汜水。大業初置武牢都尉府。有周山、天陵山。
 滎澤開皇四年置,曰廣武。仁壽元年改名焉。原武開皇十六年置。陽武圃田開皇十六年置,曰郟城。大業初改焉。



 浚儀東魏置梁州、陳留郡,後齊廢開封郡入,後周改曰汴州。開皇初郡廢,大業初州廢。有關官。有通濟渠、蔡水。酸棗後齊廢,開皇六年復。有關官。新鄭後魏廢,開皇十六年復,大業初並宛陵縣入焉。有關官。有大騩山。滎陽舊置滎陽郡。後齊省卷、京二縣入,改曰成皋郡。開皇初郡廢。有京索水、梧桐澗。開封東魏置郡,後齊廢。



 梁郡開皇十六年置宋州。統縣十三,戶十五萬五千四百七十七。



 宋城舊曰睢陽,置梁郡。開皇初郡廢,十八年縣改名焉。大業初又置郡。又梁置北新安郡,尋廢。雍丘後魏置陽夏郡。開皇初郡廢,十六年置杞州。大業初州廢。



 襄邑後齊廢,開皇十六年復。寧陵後齊廢,開皇六年復。虞城後魏曰蕭,後齊廢。



 開皇十六年置,改名焉。又後魏置沛郡,後齊廢。穀熟後魏廢,開皇十六年復。陳留後魏廢,開皇六年復。十六年析置新里縣,大業初廢入焉。又有小黃縣,後
 齊廢入。有睢水、渙水。下邑後齊廢巳吾縣入焉。考城後魏曰考陽,置北梁郡。後齊郡縣並廢,為城安縣。開皇十八年以重名,改曰考城。楚丘後魏曰巳氏,置北譙郡。



 後齊郡縣並廢。開皇四年又置巳氏,六年改曰楚丘。碭山後魏置,曰安陽。開皇十八年改名焉。有碭山、魚山。圉城舊曰圉,後齊廢,開皇六年復置,曰圉城。有谷水。柘城舊曰柘,久廢。開皇十六年置,曰柘城。



 譙郡後魏置南充州。後周置總管府,後改曰亳州。開皇元年府廢。統縣六,戶七萬四千八百一十七。



 譙舊曰小黃,置陳留郡。開皇初郡廢,十六年分置梅城縣。大業三年,改小黃為譙縣,並梅城入焉。酂舊廢,開皇十六年復。舊有馬頭郡,後魏又置下邑縣,後齊並廢。城父宋置,曰浚儀。開皇十八年改焉。穀陽後齊省,開皇六年復。山桑後魏置渦州、渦陽縣,又置譙郡。梁改渦州曰西徐州。東魏改曰譙州。開皇初郡廢,十六年改渦陽為肥水。大業初州廢,改縣曰山桑。又梁置北新安郡,東魏改置蒙郡。



 後齊廢郡,置蒙縣,後又置郡。開皇初郡廢。又梁置陽夏郡,東魏廢。臨渙後魏置臨渙郡,又別
 置丹城縣。東魏析置白椫縣,後齊郡廢。開皇元年丹城省,大業初白椫又省,並入焉。有嵇山、龍岡。



 濟陰郡後魏置西兗州,後周改曰曹州。統縣九,戶十四萬九百四十八。



 濟陰後魏置沛郡,後齊廢。又開皇六年分置黃縣,十八年改為蒙澤,大業初廢入焉。外黃後齊廢成安縣入。又開皇十八年置首城縣,大業初廢入焉。濟陽成武後齊置永昌郡。開皇初郡廢,十六年置戴州。大業初州廢。冤句乘氏定陶單父後魏曰離狐,置北濟陰郡。後齊郡縣並廢。開皇六年更置,名單父。金鄉開皇十六年分置昌邑縣,大業初並入。



 襄城郡東魏置北荊州,後周改曰和州。開皇初改為伊州,大業初改曰汝州。統縣八,戶十萬五千九百一十七。



 承休舊曰汝原,置汝北郡,後改曰汝陰郡。後周郡廢。大業初改縣曰承休,置襄城郡。有黃水。梁舊置汝北郡,後齊廢。有濫泉。郟城舊曰龍山。東魏置順陽郡及南陽郡、南陽縣。開皇初改龍山曰汝南,三年二郡並廢。十八年改汝南曰輔城,南陽曰期城。大業初改輔城
 曰郟城,廢期城入焉。有關官,有大留山。陽翟東魏置陽翟郡,開皇初郡廢。有釣臺。有九山祠。汝源汝南有後魏汝南郡及符壘縣,並後齊廢。魯後魏置荊州,尋廢,立魯陽郡,後置魯州。開皇初郡廢,大業初州廢。



 有關官。有和山、大義山。犨城舊曰雉陽。開皇十八年改曰湛水,大業初改名焉。



 又有後周置武山郡,開皇初廢。後魏置南陽縣、河山縣,大業初並廢入焉。有應山。



 潁川郡舊置潁州,東魏改曰鄭州,後周改曰許州。統縣十四,戶十九萬五千六百四十。



 潁川舊曰長社,置穎川郡,後齊廢潁陰縣入,開皇初廢郡改縣焉。又東魏置黃臺縣,大業初廢入焉,置郡。襄城舊置襄城郡,後周置汝州。開皇初郡廢,大業初州廢。有溵水。汝墳後齊置漢廣郡,尋廢,有首山。葉後齊置襄州。後周廢襄州,置南襄城郡。開皇初郡廢。又東魏置定南郡,後周廢為定南縣。大業初省入。北舞舊置定陵郡,開皇初廢。有百尺溝。郾城開皇初置,十六年置道州,大業初州廢。



 又後魏置潁川郡,後齊改為臨潁郡,開皇初郡廢。又有邵陵縣,大業初廢。有溵水。



 繁昌
 臨潁尉氏後齊廢,開皇六年復。長葛開皇六年置。許昌水隱強開皇十六年置,曰陶城,大業初改焉。扶溝鄢陵東魏置許昌郡,後齊廢縣。開皇初郡廢,七年復鄢陵縣。十六年置洧州,大業初州廢。又開皇十六年置蔡陂縣,至是省入焉。



 汝南郡後魏置豫州,東魏置行臺。後周置總管府,後改曰舒州,尋復曰豫州,及改洛州為豫州,此為溱州,又改曰蔡州。統縣十一,戶十五萬二千七百八十五。



 汝陽舊曰上蔡,置汝南郡。開皇初郡廢。大業初置郡,改縣曰汝陽,並廢保城縣入焉。有鴻卻陂。城陽舊廢,梁置,又有義興縣。後魏置城陽郡,梁置楚州,東魏置西楚州,後齊曰永州。開皇九年,廢入純州。十八年改義興為純義。大業初州縣並廢入焉。又梁置伍城郡,後齊廢。有十丈山、大木山。真陽舊置郢州。東魏廢州,置義陽郡。後齊廢郡入保城縣。開皇十一年廢縣。十六年置縣,曰真丘。大業初改曰真陽。又有白狗縣,梁置淮州。後齊廢州,以置齊興郡,郡尋廢。開皇初,改縣曰淮川,至是亦省入焉。又有後魏安陽縣,後廢。有汶水。新息後
 魏置東豫州。



 梁改曰西豫州。又改曰淮州。東魏復曰東豫州,後周改曰息州,大業初州廢。又後魏置汝南郡,開皇初郡廢。又梁置滇州,尋廢。又梁置北光城郡,東魏廢,又有北新息縣,後齊廢。褒信宋改曰包信。大業初改復舊焉。又梁置梁安郡,開皇初廢。



 又有長陵郡,後齊廢為縣。大業初又省縣焉。上蔡後魏置,曰臨汝。後齊廢。開皇中置,曰武津。大業初改名焉。平輿舊廢,大業初改新蔡置焉。有葛陂。新蔡齊置北新蔡郡,魏曰新蔡郡,東魏置蔡州。後齊廢州置廣寧郡。開皇初郡廢。十六年置舒州及舒縣、廣寧縣。仁壽元年改廣寧曰汝北。大業初州廢,改汝北曰新蔡。又後齊置永康縣,後改名曰澺水,至是及舒縣並廢入焉。朗山舊曰安昌,置初安郡。廢,十八年縣改名焉。又梁置除州,後魏廢,又齊置荊州,尋廢。後周又置威州,後又廢。吳房故曰遂寧,後齊省綏義縣入焉。大業初改曰吳房。西平後魏置襄城郡,後齊改郡曰文城,開皇初郡廢。又有故武陽縣,十八年改曰吳房,大業初省。又有故洧州、灊州,並後齊置,開皇初皆廢。



 淮陽郡開皇十六年置陳州。統縣十,戶十二萬七千一百四。



 宛丘後魏曰項,置陳郡。開皇初縣改名宛
 丘,尋廢郡,後析置臨蔡縣。大業初置淮陽郡,並臨蔡縣入焉。又後魏置南陽郡,東魏廢。西華舊曰長平,開皇十八年改曰鴻溝。大業初改焉。有舊長平縣,後齊廢。溵水開皇十六年置,又有後魏汝陽郡及縣,後齊郡廢,大業初縣廢。撫樂開皇十六年置。有渦水。太康舊曰陽夏,並置淮陽郡。開皇初郡廢,七年更名太康。有窪水。鹿邑舊曰武平,開皇十八年改名焉。項城東魏置揚州及丹陽郡、秣陵縣,梁改曰殷州,東魏又改曰北揚州,後齊改曰信州,後周改曰陳州。開皇初改秣陵為項縣。十六年分置沈州,大業初州廢,又有項城郡,開皇初分立陳郡,三年並廢。南頓舊置南頓郡。後齊廢郡及平鄉縣入,改曰和城。大業初又改為南頓。開皇六年置。鮦陽後齊廢,開皇十一年復。又東魏置財州,後齊廢,以置包信縣。開皇初廢。



 汝陰郡舊置潁州。統縣五,戶六萬五千九百二十六。



 汝陰舊置汝陰郡,開皇初郡廢。大業初復置。潁陽梁曰陳留,並置陳留郡及陳州。東魏廢州。開皇初廢郡,十
 八年縣改名焉。有鄭縣,後齊廢。清丘梁曰許昌,及置潁川郡。開皇初廢郡,十八年縣改名焉。潁上梁置下蔡郡,後齊廢郡。大業初縣改名焉。下蔡梁置汴郡,後齊郡廢。大業初縣改名焉。又梁置淮陽郡,後齊改曰潁川郡。開皇初郡廢。



 上洛郡。舊置洛州,後周改為商州。統縣五,戶一萬五百一十六。



 上洛舊置上洛郡,開皇初郡廢,大業初復置。有秦嶺山、熊耳山、洛水、丹水。



 商洛有關官。洛南舊曰拒陽,置拒陽郡。開皇初郡廢,縣改名焉。有玄扈山、陽虛山。豐陽後周置,開皇初並南陽縣入。有洵水、甲水。上津舊置北上洛郡,梁改為南洛州,西魏又改為上州,後周並漫川、開化二縣入,大業初廢州。有天柱山、詔及山、女思山。



 弘農郡大業三年置。統縣四,戶二萬七千四百六十六。



 弘農舊置西恆農郡,後周廢。大業初置弘農郡。又有石城郡、玉城縣,西魏並廢。有石堤山。盧氏後
 魏置漢安郡,西魏置義川郡。開皇初郡廢,州改為虢州。大業初州廢。有關官。有石扇山。長泉後魏曰南陜,西魏改焉。有松楊山、檀山。硃陽舊置硃陽郡,後周郡廢。有邑陽縣,開皇末改為邑川,大業初並入。有肺山,有湖水。



 淅陽郡西魏置淅州。統縣七,戶三萬七千二百五十。



 南鄉舊置南鄉郡,後周並龍泉、湖里、白亭三縣入。又有左南鄉縣,並置左鄉郡。西魏改郡為秀山,改縣為安山。後周秀山郡廢。開皇初南鄉郡廢。大業初置淅陽郡,並安山縣入焉。有石墨山。內鄉舊曰西淅陽郡,西魏改為內鄉。後周廢,並淅川、石人二縣入焉。有淅水。丹水舊置丹川郡。後周郡廢,並茅城、倉陵、許昌三縣入。有胡保山。武當舊置武當郡。又僑置始平郡,後改為齊興郡。梁置興州,後周改為豐州。開皇初二郡並廢,改為均州。大業初州廢。有石階山、武當山。均陽梁置。安福梁置,曰廣福,並為郡。開皇初郡廢,仁壽初改焉。鄖鄉有防
 山。



 南陽郡舊置荊州。開皇初,改為鄧州。統縣八,戶七萬七千五百二十。



 穰帶郡。有白水。新野舊曰棘陽,置新野郡。又有漢廣郡,西魏改為黃岡郡。



 又有南棘陽縣,改為百寧縣。後周二郡並廢,並南棘縣入焉。開皇初更名新野。南陽舊曰上陌,置南陽郡。後周並宛縣入,更名上宛。開皇初郡廢,又改為南陽。課陽舊曰涅陽,開皇初改焉。有課水、涅水。順陽舊置順陽郡。西魏析置鄭縣,尋改為清鄉。後周又並順陽入清鄉。開皇初又改為順陽。冠軍菊潭舊曰酈,開皇初改焉。有東弘農郡,西魏改為武關,至是廢入。有梅溪、湍水。新城西魏改為臨湍,開皇初復名焉。有朝水。



 淯陽郡西魏置蒙州。仁壽中,改曰淯州。統縣三,戶一萬七千九百。



 武川帶郡。有雉衡山。有淯水、紵水、灃水。向城西魏置,又立雉陽郡。開皇初郡廢。方城西魏置,及置襄邑郡。開皇初廢。東魏又置建城郡及建城縣,後齊郡縣並廢。又有業縣,開皇末改為灃水,大業並入。有西唐
 山。



 淮安郡後魏置東荊州,西魏改為淮州。開皇五年又改為顯州。統縣七,戶四萬六千八百四十。



 比陽帶郡。後魏曰陽平,開皇七年改為饒良,大業初又改焉。又有後魏城陽縣,置殷州、城陽郡。開皇初郡並廢,其縣尋省。又有昭越縣,大業初改為同光,尋廢。



 又有東南陽郡,西魏改為南郭郡,後周廢。又有比陽故縣,置西郢州。西魏改為鴻州,後周廢為真昌郡。開皇初郡廢,大業初縣廢。平氏舊置漢廣郡,開皇初郡廢。



 有淮水。真昌舊曰北平,開皇九年改焉。顯岡舊置舞陰郡,開皇初郡廢。臨舞東魏置,及置期城郡。開皇初郡廢。又有東舞陽縣,開皇十八年改為昆水,大業初廢。



 慈丘後魏曰江夏,並置江夏郡。開皇初郡廢,更置慈丘於其北境。後魏有鄭州、潘州、溱州及襄城、周康二郡,上蔡、青山、震山三縣,並開皇初廢。有比水。桐柏梁置,曰淮安,並立華州,又立上川郡。西魏改州為淮州,後改為純州,尋廢。開皇初郡廢,更名縣曰桐柏。又梁置西義陽郡,西魏置淮陽郡及輔州,後周州郡並廢,又置淮南縣。開皇末改為油水,大業初廢。又有大義郡,後周置,開皇初廢。有桐柏山。



 豫州於《禹貢》為荊州之地。其在天官,自氐五度至尾九度,為大火,於辰在卯,宋之分野,屬豫州。自柳九度至張十六度,為鶉火,於辰在午,周之分野,屬三河,則河南。準之星次,亦豫州之域。豫之言舒也,言稟平和之氣,性理安舒也。



 洛陽得土之中,賦貢所均,故周公作洛,此焉攸在。其俗尚商賈,機巧成俗。故《漢志》云「周人之失,巧偽趨利,賤義貴財」,此亦自古然矣。滎陽古之鄭地,梁郡梁孝故都,邪僻傲蕩,舊傳其俗。今則好尚稼穡,重於禮文,其風皆變於古。



 譙郡、濟陰、襄城、潁川、汝南、淮陽、汝陰,其風頗同。南陽古帝鄉,搢紳所出,自三方鼎立,地處邊疆,戎
 馬所萃,失其舊俗。上洛、弘農,本與三輔同俗。自漢高發巴蜀之人,定三秦,遷巴之渠率七姓,居於商洛之地,由是風俗不改其壤。其人自巴來者,風俗猶同巴郡。淅陽、淯陽、亦頗同其俗云。



 東郡開皇九年置杞州,十六年改為滑州,大業二年為兗州。統縣九,戶十二萬一千九百五。



 白馬舊置東郡,後齊並涼城縣入焉。大業初復置郡。靈昌開皇十六年置。衛南開皇十六年置,大業初廢西濮陽入焉。又有後魏平昌、長樂二縣,後齊並廢。濮陽開皇十六年分置昆吾縣,大業初入焉。封丘後齊廢,開皇十六年復。匡城後齊曰長垣,開皇十六年改焉。胙城舊曰東燕,開皇十八年改焉。韋城開皇六年置,十六年分置長垣縣,大業初省入焉。離狐。



 東平郡後周置魯州,尋廢。開皇十年置鄆州。統縣六,戶八萬六千九十。



 鄆城後周置,曰清澤,又置高平郡。開皇初郡廢,改縣曰萬安。十八年改曰鄆城。大業初置郡,並廩丘入焉。鄄城舊置濮陽郡,開皇初郡廢,十六年置濮州,大業初州廢。有關官。須昌開皇十六年置。有梁山。宿城後齊曰須昌,開皇十六年改焉。舊置東平郡,後齊並廢。雷澤舊曰城陽,後齊廢。開皇十六年置,曰雷澤,又分置臨濮縣。大業初並入焉。有歷山、雷澤。鉅野舊廢,開皇十六年復,又置乘丘縣,大業初廢入焉。



 濟北郡舊置濟州。統縣九,戶十萬五千六百六十。



 盧舊置郡,開皇初廢。六年分置濟北縣,大業初省入焉,尋置郡。有關官。有成回倉,有魚山、游仙山。範後齊廢,開皇十六年置。陽穀開皇十六年置。東阿有浮山、監山、狼水。平陰開皇十四年置,曰榆山,大業初改焉。長清開皇十四年置。又有東太原郡,後齊廢。濟北開皇十四年置,曰時平,大業初改焉。壽張
 肥城宋置濟北郡,後齊廢。後周置肥城郡,尋廢,又復。開皇初又廢。



 武陽郡後周魏置魏州。統縣十四,戶二十一萬三千三十五。



 貴鄉東魏置。又有平邑縣,後齊廢,開皇十六年又置。大業初置武陽郡,並省平邑縣入焉。有愜山。元城後齊廢。開皇六年復,又置馬陵縣,大業初廢入焉。有沙麓山。繁水舊曰昌樂,置昌樂郡。東魏郡廢,後周又置。舊有魏城縣,後齊廢。



 開皇初廢郡,六年置縣,曰繁水。大業初廢昌樂縣入焉。魏後齊廢,開皇六年復。



 十六年析置漳陰縣,大業初省入焉。莘舊曰陽平,後齊改曰樂平。開皇六年復曰陽平,八年改曰清邑,十六年置莘州。大業初州廢,改縣名莘,又廢莘亭縣入焉。後周置武陽郡焉,開皇初廢。頓丘後齊省,開皇六年置。又有舊陰安縣,後齊廢。觀城舊曰衛國,開皇六年改。臨黃後魏置,後齊省,開皇六年復,十六年分置河上縣,大業初省入焉。武陽後齊省,後周置。武水開皇十六年置。館陶舊置毛州,大業初州廢。又有舊陽平郡,開皇初廢。堂邑開皇六年置。冠氏開皇六年置。聊城
 舊置南冀州及平原郡,未幾,州廢。開皇初郡廢。十六年置博州,大業初州廢。



 渤海郡開皇六年置棣州,大業二年為滄州。統縣十,戶十二萬二千九百九。



 陽信帶郡。樂陵舊置樂陵郡,開皇初郡廢。十六年分置鬲津縣,大業初廢入焉。



 滳河開皇十六置。又有後魏濕沃縣,後齊廢。有關官。厭次後齊廢,開皇十六年復。



 蒲臺開皇十六年置。饒安舊置滄州、浮陽郡,開皇初郡廢,大業初州廢。無棣開皇六年置。鹽山舊曰高成。開皇十六年又置浮水縣。十八年改高成曰鹽山。大業初省浮水入焉。有鹽山、峽山。南皮清池舊曰浮陽,開皇十八年改。



 平原郡開皇九年置德州。統縣九,戶十三萬五千八百二十二。



 安樂舊置平原郡,開皇初郡廢,大業初復,又開官皇十六年置繹幕縣,至是廢入焉。又有後魏鬲縣,後齊廢,有關官。平原後齊並鄃縣入焉。有關官。又後魏置東青州,置未久而為。將陵開皇十六年置。
 平昌後魏置東安郡,後齊廢,並以重平縣入焉。般後齊省,開皇十六年復。長河舊曰廣川。後齊省,開皇六年復置,仁壽初改名焉。弓高舊廢,開皇十六年置。東光舊置渤海郡,開皇初郡廢。九年置觀州,大業初州廢,又並安陵入焉。有天胎山。胡蘇舊廢,開皇十六年置。



 兗州於《禹貢》為濟、河之地。其於天官,自軫十二度至氐四度,為壽星,於辰在辰,鄭之分野。兗州蓋取沇水為名,亦曰兗,兗之為言端也,言陽精端端,故其氣纖殺也。東郡、東平、濟北、武陽、平原等郡,得其地焉。兼得鄒、魯、齊、衛之交。舊傳太公唐叔之教,亦有周孔遺風。今此數郡,其人尚多好儒學,性質直懷義,有古之風烈矣。



 信都郡舊置冀州。統縣十二,戶十六萬八千七百一十八。



 長樂舊曰信都,帶長樂郡,後齊廢扶柳縣入焉。開皇初郡廢,分信都置長樂縣。



 十六年又分長樂置澤城縣。大業初廢信都及澤城入焉,置信都郡。堂陽舊縣,後齊廢,開皇十六年復。衡水開皇十六年置。棗強舊縣,後齊廢索蘆、廣川二縣入焉。



 武邑舊縣,後齊廢。開皇六年置,並得後齊觀津縣地。十六年分武強置昌亭縣,大業初廢入焉。武強舊置武邑郡,後齊郡廢,又廢武遂縣入焉。南宮舊縣,後齊廢,開皇六年復。斌強鹿城舊曰梟阜,後齊改曰安國。開皇六年改為安定,十八年改。



 開皇十六年又置晏城,大業初廢入。下博,蓚舊曰脩,開皇五年改。十六年分置觀津縣,大業初廢。阜城清河郡後周置貝州。統縣十四,戶三十萬千六五百四十四。



 清河舊曰武城,置清河郡。開皇初郡廢,改名焉,仍別置武城縣。十六年置夏津縣,大業初廢入,置清河郡。清陽舊曰清河縣,後齊省貝丘入焉,改為貝丘,開皇六年改為清陽。又有後魏候城縣,後齊省以入武城,亦入焉。武城舊曰上城。開皇初改武城為清河縣,於此置武城。歷亭開皇十六年分武城
 置焉。漳南開皇六年置,曰東陽,十八年改為漳南。有後魏故索盧城,後齊以入棗強,至是入。鄃舊廢,開皇十六年置。臨清後齊廢,開皇六年年復。又十六年置沙丘縣,大業初廢入焉。清泉後齊廢千童縣入。開皇十六年置貝丘縣,大業二年廢入。清平開皇六年置,曰貝丘,十六年改曰清平。高唐後魏置南清河郡,後齊郡廢。經城後齊廢,開皇六年置,十六年分置府城縣,大業初省入焉。宗城舊曰廣宗,仁壽元年改。博平開皇六年置靈縣,大業初省入。茌平後齊廢,開皇初復。



 魏郡後魏置相州,東魏改曰司州牧。後周又改曰相州,置六府。宣政初府移洛,以置總管府,未幾,府廢。統縣十一,戶十二萬二百二十七。



 安陽周大象初,置相州及魏郡,因改名鄴,開皇初郡廢,十年復,名安陽,分置相縣,鄴還復舊。大業初廢相入焉,置魏郡有韓陵山。鄴東魏都。後周平齊,置相州。大象初縣隨州徙安陽,此改為靈芝縣。開皇十年又改焉。臨漳東魏置。成安後齊置。靈泉後周置。有龍山。堯城開皇十年置,名長樂,十八年改焉。洹水後周置。
 滏陽後周置。開皇十年置慈州,大業初州廢。臨水有慈石山、鼓山、滏山。林慮後魏置林慮郡,後齊郡廢,後又置。開皇初郡廢,又分置淇陽縣。十六年置巖州。



 大業初州廢,又廢淇陽入焉。有林慮谼、仙人臺、洹水。臨淇東魏置,尋廢,開皇十六年復。有淇水。



 汲郡東魏置義州,後周為衛州。統縣八,戶十一萬一千七百二十一。



 衛舊曰朝歌,置汲郡。後周又分置修武郡。開皇初郡並廢,十六年又置清淇縣。



 大業初置汲郡,改朝歌縣曰衛,廢清淇入焉。有朝陽山、同山。有紂朝歌城、比干墓。汲東魏僑置七郡十八縣。後齊省,以置伍城郡,後周廢為伍城縣,開皇六年改焉。隋興開皇六年置。後析置陽源縣,大業初並入焉。有倉巖山。黎陽後魏置黎陽郡,後置黎州。開皇初州郡並廢。十六年又置黎州,大業初,罷。有倉。有關官。



 有大伾山、枉人山。內黃舊廢,開皇六年置。十六年分置繁陽縣,大業初廢入。湯陰舊廢,開皇六年又置。有博望岡。臨河開皇六年置。澶水開皇十六年置。



 河內郡舊置懷州。統縣十,戶十三萬三千六百六。



 河內舊曰野王,置河內郡。開皇初郡廢,十六年縣改焉。有軹縣,大業初廢入,尋置郡。有大行,有丹水。有絺城。溫舊廢,開皇十六年置。古溫城。濟源開皇十六年置。舊有沁水縣,後齊廢入。有孔山、母山。有濟水、濝水、古原城。河陽,舊廢,開皇十六年置。有盟津。有古河陽城治。安昌舊曰州縣,置武德郡。開皇初郡廢,十八年縣改為邢丘。大業初改名安昌,又廢懷縣入焉。舊有平高縣,後齊廢。



 王屋舊曰長平,後周改焉,後又置懷州。及平齊,廢州置王屋郡。開皇初郡廢。有王屋山、齊子嶺。有軹關。獲嘉後周置修武郡,開皇初郡廢。十六年置殷州,大業初州廢。新鄉開皇初年置。有關官。舊有獲嘉縣,後齊廢。修武後魏置修武,後齊並入焉。開皇十六年析置武陟,大業初並入焉。又有東魏廣寧郡,後周廢。共城舊曰共,後齊廢。開皇六年復署,曰共城。有共山、白鹿山。



 長平郡舊曰建州。開皇初改為澤州。統縣六,戶五萬四千九百一十
 三。



 丹川舊曰高都。後齊置長平、高都二郡,後周並為高平郡。開皇初郡廢,十八年改為丹川,大業初置長平郡。有太行山。沁水舊置廣寧郡。後齊郡廢,縣改為永寧。開皇十八年改焉。有輔山。端氏後魏置安平郡,開皇初郡廢。有巨峻山、秦川水。濩澤有嶕嶢山,濩澤山。高平舊曰平高,齊末改焉,又並玄氏縣入焉。有關官。



 陵川開皇十六年置。



 上黨郡後周置潞州。統縣十,戶十二萬五千五十七。



 上黨舊置上黨郡,開皇初郡廢。有壺關縣。大業初復置郡,廢壺關入焉。有羊頭山、抱犢山。長子後齊廢。開皇九年置,曰寄氏縣。十八年改為長子。舊有屯留、樂陽二縣,後齊廢。有濁漳水、堯水。潞城開皇十六年置。有黃阜山。屯留後齊廢,開皇十六年復。襄垣舊置襄垣郡,後齊郡廢。後周置韓州,大業初州廢。有鹿臺山。



 黎城後魏以潞縣被誅遺人置,十八年改名黎城。有積布山、松門嶺。涉後魏廢,開皇十八年復。有崇山。鄉石勒置武鄉郡,後魏去武字。開皇初郡廢,十六年分置榆社縣,大業初廢。又有後魏南垣州,尋改豐州,
 後周廢。銅鞮有舊涅縣,後魏改為陽城。開皇十八年改為甲水,大業初省入。有銅鞮水。沁源後魏置縣及義寧郡,開皇初廢。十六年置沁州。又義寧縣十八年改為和川。大業初州廢,又廢和川縣入。



 河東郡後魏曰秦州,後周改曰蒲州。統縣十,戶十五萬七千七十八。



 河東舊曰蒲阪縣,置河東郡。開皇初郡廢,十六年析置河東縣。大業初置河東郡,並蒲阪入。有酒官。有首山。有媯、汭水。桑泉開皇十六年置。有三疑山。汾陰舊置汾陰郡,開皇初郡廢。有龍門山。龍門後魏置,並置龍門郡。開皇初郡廢。



 芮城舊置,曰安戎。後周改焉,又置永樂郡,後省入焉。有關官。安邑開皇十六年置虞州,大業初州廢。有鹽池、銀冶。夏舊置安邑郡,開皇初郡廢。有巫咸山、稷山、虞阪。河北舊置河北郡,開皇初郡廢。有關官。有砥柱山。有傅巖。猗氏西魏改曰桑泉,後周復焉。虞鄉後魏曰安定,西魏改曰南解,又改曰綏化,又曰虞鄉。



 有石錐山、百梯山、百徑山。



 絳郡後魏置東雍州,後周改曰絳州。統縣八,戶七萬一千八百七十六。



 正平舊曰臨汾,置正平郡。開皇初郡廢,十八年縣改名焉。大業初置絳郡。又有後魏南絳郡,後周廢郡,又並南絳縣入小鄉縣。開皇十八年改曰汾東,大業初省入焉。翼城後魏置,曰北絳縣,並置北絳郡。後齊廢新安縣,並南絳郡入焉。開皇初郡廢,十八年改為翼城。有烏嶺山、東涇山。有澮水。絳舊置絳郡,開皇初郡廢。



 後周置晉州,建德五年廢。曲沃後周置,建德六年廢。有絳山、橋山。稷山後魏曰高涼,開皇十八年改焉。有後魏龍門郡,開皇初廢。又有後周勛州,置總管,後改曰絳州,開皇初移。聞喜有景山。有董澤陂。垣後魏置邵郡及白水縣。後周置邵州,改白水為亳城。開皇初郡廢。大業初州廢,縣改為垣縣,又省後魏所置清廉縣及後周所置蒲原縣入焉。有黑山。太平後魏置,後齊省臨汾縣入焉。有關官。



 文城郡東魏置南汾州,後周改為汾州,後齊為西汾州。後周平齊,置總管府。



 開皇四年府廢,十六年改為耿州,後復為汾州。統縣四,戶二萬二千三百。



 吉昌後魏曰定陽縣,並置定陽郡。開皇初郡廢,十八年縣改名焉。大業初,置文城郡。有風山。文城後魏置。有石門山。伍城
 後魏置,曰刑軍縣,後改為伍城,後又置伍城郡。開皇初郡廢,又廢後魏平昌縣入焉。大業初又廢大寧縣入焉。昌寧後魏置,並內陽郡。開皇初郡廢。有壺口山,崿山。



 臨汾郡後魏置唐州,改曰晉州。後周置總管府,開皇初府廢。統縣七,戶七萬一千八百七十四。



 臨汾後魏曰平陽,並置平陽郡。開皇初改郡為平河,改縣為臨汾,尋郡廢。又有東魏西河、敷城、伍城、北伍城、定陽等五郡,後周廢為西河、定陽二郡。開皇初郡並廢。又有後魏永安縣,開皇初改為西河,大業初省。又有舊襄城縣,後齊省。



 有姑射山。襄陵後魏太武禽赫連昌,乃分置禽昌縣。齊並襄陵入禽昌縣。大業初又改為襄陵。冀氏後魏置冀氏郡,領冀氏、合陽二縣。後齊郡廢,又廢合陽入焉。楊霍邑後魏曰永安,並置永安郡。開皇初郡廢。十六年置汾州,十八年改為呂州,縣曰霍邑。大業初州廢。有霍山。有彘水。汾西後魏曰臨汾,並置汾西郡。開皇初郡廢,十八年縣改為汾西。又有後周新城縣,開皇十年省入。岳陽後魏置,曰安澤。



 大業初改焉。



 龍泉郡後周置汾州。開皇四年置西汾州總管,五年改為隰州總管。大業初府廢。



 統縣五,戶二萬五千八百三十。



 隰川後周置縣,初曰長壽,又置龍泉郡。開皇初郡廢,縣改曰隰川。大業初置郡。永和後周置,曰臨河縣及臨河郡。開皇初郡廢,十八年縣改名焉。有關官。樓山後周置,曰歸化。開皇十八年改名焉。有北石樓山,有孔山。石樓舊置吐京郡及吐京縣,開皇初郡廢,十八年縣改名。蒲後周置,有伍城郡及石城郡及石城縣,周末並廢。又有後魏平昌縣,開皇中改曰蒲川,大業初廢入焉。



 西河郡後魏置汾州,後齊置南朔州,後周改曰介州。統縣六,戶六萬七千三百五十一。



 隰城舊置西河郡,開皇初郡廢,大業初復。有隱泉山。介休後魏置定陽郡、平昌縣。後周改郡曰介休,以介休縣入焉。開皇初郡廢,十八年縣改曰介休。永安有雀鼠谷。平遙開皇十六年析置清世縣,大業初廢入焉。又後魏置蔚州,後周廢。有鹿臺山。靈石開皇十年置。有介山,
 有靖巖山。綿上開皇十六年置。有沁水。



 離石郡後齊置西汾州,後周改為石州。統縣五,戶二萬四千八十一。



 離石後齊曰昌化縣,置懷政郡。後周改曰離石郡及縣,又置寧鄉縣。開皇初郡廢。大業初置郡,並寧鄉入焉。修化後周置,曰窟胡,並置窟胡郡。開皇初郡廢,後縣改為修化。又後周置盧山縣,大業初並入焉。有伏盧山。定胡後周置,及置定胡郡。開皇初郡廢。有關官。平夷後周置。太和後周置,曰烏突,及置烏突郡。開皇初郡廢,縣尋改焉。有湫水。



 雁門郡後周置肆州。開皇五年改為代州,置總管府。大業初府廢。統縣五,戶四萬二千五百二。



 雁門舊曰廣武,置雁門郡。開皇初郡廢,十八年改曰雁門。大業初置雁門郡。



 有關官。有長城。有摐頭山,有夏屋山。繁畤後魏置,並置繁畤郡。後周郡縣並廢。



 開皇十八年復置縣。有東魏武州及吐京、齊、新安三郡,寄在城中。後齊改為北靈州,尋廢。有長城、滹沱水、泒水、唐山。崞後魏置,曰石城
 縣。東魏置廓州。有廣安、永定、建安三郡,寄山城。後齊廢郡。改為北顯州。後周廢。開皇十年改縣曰平寇。大業初改為崞縣。又有雲中城,東魏僑置恆州,尋廢。有無京山、崞山。



 有土城。五臺舊曰慮虎,久廢。後魏置,曰驢夷。大業初改焉。有五臺山。靈丘後魏置靈丘郡,後齊省莎泉縣入焉。後周置蔚州,又立大昌縣。開皇初郡廢,縣並入焉。大業初州廢。



 馬邑郡舊置朔州。開皇初置總管府,大業初府廢。統縣四,戶四千六百七十四。



 善陽後齊置,縣曰招遠,郡曰廣安。開皇初郡廢。大業初縣改曰善陽,置代郡,尋曰馬邑。又有後魏桑乾郡,後齊以置朔州及廣寧郡。後周郡廢,大業初州廢。神武後魏置神武郡,後齊改曰太平,後周罷郡。有桑乾水。雲內後魏立平齊郡,尋廢。



 後齊改曰太平縣,後周改曰雲中,開皇初改曰雲內。有後魏都,置司州,又有後齊安遠、臨塞、威遠、臨陽等郡屬北恆州,後周並廢。有純真山、白登山、武周山。



 有濕水。開陽舊曰長寧,後齊置齊德、長寧二郡。後周廢齊德郡。開皇初郡廢,十九年縣改曰開陽。



 定襄郡開皇五年置雲州總管府,大業元年府廢。統縣一,戶三百七十四。



 大利大業初置,帶郡。有長城。有陰山。有紫河。



 樓煩郡大業四年置。統縣三,戶二萬四千四百二十七。



 靜樂舊曰岢嵐。開皇十八年改為汾源,大業四年改焉。有長城。有汾陽宮。有關官。有管涔山、天池、汾水。臨泉後齊置,曰蔚汾。大業四年改焉。秀容舊置肆州,後齊又置平寇縣。後周州徙雁門。開皇初置新興郡、銅川縣。郡尋廢。十年廢平寇縣。十八年置忻州,大業初州廢,又廢銅川。有程侯山、系舟山。有嵐水。



 太原郡後齊並州,置省,立別宮。後周置並州六府,後置總管,廢六府。開皇二年置河北道行臺,九年改為總管府,大業初府廢。統縣十五,戶十七萬五千三。



 晉陽後齊置,曰龍山。帶太原郡。開皇初郡廢,十年改縣曰晉陽,十六年又置清源縣,大業初省入焉。有龍山、蒙山。太原舊曰晉陽,帶郡。開皇十年分置陽真縣,大業初省入焉。有晉陽宮。有晉水。交城開皇十六年置。汾陽
 舊曰陽曲。開皇六年改為陽直,十六年又改名焉,復分置孟縣,大業初廢。有摩笄山文水舊曰受陽,開皇十年改焉。有文水、泌水。祁後齊廢,開皇中復。壽陽開皇十年改州南受陽縣為文水,分州東故壽陽置壽陽。有甗巖。榆次後齊曰中都,開皇中改焉。太谷舊曰陽邑,開皇十八年改焉。樂平舊置樂平郡,開皇初廢郡。十六年分置遼州及東山縣,大業初廢州及東山縣。有皋洛山。有清漳水。和順舊曰梁榆,開皇十年改。有九京山。遼山後魏曰遼陽,後齊省。開皇十年置,改名焉。十六年屬遼州,並置交漳縣。



 大業初廢州,並罷交漳入焉。有萁尞水。平城開皇十六年置。有塗水。石艾有蒙山。孟開皇十六年置,曰原仇,大業初改焉。有白鹿山。



 襄國郡開皇十六年置邢州。統縣七,戶十萬五千八百七十三。



 龍岡舊曰襄國,開皇九年改名焉。十六年又置青山縣,大業初省入焉。有黑山。



 有幹水。南和舊置北廣平郡,後齊省入廣平郡,後周分置南和郡。開皇初郡廢,十六年置任縣,大業初廢入。平鄉沙
 河開皇十六置。有罄山。鉅鹿後齊廢,開皇六年置南欒縣,後廢入焉。內丘有乾言山。柏仁有鵲山。



 武安郡後周置洛州。統縣八,戶十一萬八千五百九十五。



 永年舊曰廣平,置廣平郡,後齊廢北廣平郡及曲梁、廣平二縣入。開皇初郡廢,復置廣平,後改曰雞澤。仁壽元年改廣平為永年。大業初置武安郡,又並雞澤縣入。



 肥鄉東魏省,開皇十年復。清漳開皇十六年置。平恩洺水舊曰斥漳,後齊省入平恩。開皇六年分置曲周,大業初廢入焉。武安開皇十年分置陽邑縣,大業初廢入焉。



 有榆溪,有閼與山,有浸水。邯鄲東魏廢。開皇十六年復置陟鄉,大業初省入焉。



 臨洺舊曰易陽。後齊廢入襄國縣,置襄國郡。後周改為易陽縣,別置襄國縣。開皇六年改易陽為邯鄲,十年改邯鄲為臨洺。開皇初郡廢。有紫山、狗山、塔山。



 趙郡開皇十六年置欒州,大業三年改為趙州。統縣十一,戶十四萬八千一
 百五十六。



 平棘舊置趙郡,開皇初省。有宋子縣,後齊廢。大業初置趙郡,廢宋子縣入焉。



 高邑贊皇開皇十六年置。有孔子嶺,有白溝。元氏舊縣,後齊廢,開皇六年置。



 大業初置趙郡,廢宋子縣入焉。高邑贊皇開皇十六年置。有孔子嶺,有白溝。元氏舊縣,後齊廢,開皇六年置。十六年分置靈山縣,大業初廢入焉。有靈山。Y陶舊曰Y遙,開皇六年改為「陶」。欒城舊縣,後齊廢,開皇十六年復。大陸舊曰廣阿,置殷州及南鉅鹿郡。後改為南趙郡,改州為趙州。開皇十六年分置欒州,仁壽元年改為象城。大業初州廢,縣改為大陸。又開皇十六年所置大陸縣,亦廢入焉。



 柏鄉開皇十六年置。有宣務山。房子舊縣,後齊省,開皇六年復。有贊皇山。



 有彭水。城後齊廢下曲陽入焉。改為高城縣,置鉅鹿郡。開皇初郡廢。十年置廉州,十八年改為城縣,大業初州廢。又開皇十六年置柏鄉縣,亦廢入焉。鼓城舊曰曲陽,後齊廢。開皇十六年分置昔陽縣,十八年改為鼓城。十六年又置廉平縣,大業初並入。



 恆山郡後周置恆州。統縣八,戶十七萬七千五百七十一。



 真定舊置常山郡,開皇初郡廢。十六年分置常山縣。大業初置恆山郡,省常山入焉。滋陽開皇六年
 置。十六年又置王亭縣,大業初省入焉。有大茂山、歲山。行唐石邑舊縣,後齊改曰井陘,開皇六年改焉。十六年析置鹿泉縣,大業初並入。



 有封龍山、抱犢山。九門後齊廢,開皇六年復。大業初,又並新市縣入焉。有許春壘。井陘後齊廢石邑,以置井陘。開皇六年復石邑縣,分置井陘。十六年於井陘置井州,及置葦澤縣。大業初廢州,並廢葦澤縣及蒲吾縣入焉。房山開皇十六年置。



 靈壽後周置蒲吾郡,開皇初郡廢。



 博陵郡舊置定州。後周置總管府,尋罷。統縣十,戶十萬二千八百一十七。



 鮮虞舊曰盧奴,置鮮虞郡。後齊廢盧奴入安喜。開皇初廢郡,以置鮮虞縣。大業初置博陵郡,又廢安喜入焉。有盧水。北平舊置北平郡。後齊郡廢,又並望都、蒲陰二縣來入。開皇六年又置望都,大業初又廢。有都山、伊祁山。有濡水。唐舊縣,後齊廢,開皇十六年復。有堯山、郎山、中山。恆陽舊曰上曲陽,後齊去上字。



 開皇六年改為石邑,七年改曰恆陽。有恆山,有恆陽溪,有範水。新樂開皇十六年置。有黃山。
 隋昌後魏曰魏昌,後齊廢。開皇十六年復,仍改焉。毋極義豐開皇六年置。舊有安國縣,後齊廢。深澤後齊廢,開皇六年復。安平後齊置博陵郡,開皇初廢。十六年置深州,大業初州廢。



 河間郡舊置瀛州。統縣十三,戶十七萬三千八百八十三。



 河間舊置河間郡,開皇初郡廢。大業初復置郡,並武垣縣入焉。文安有狐貍澱。



 樂壽舊曰樂城,開皇十八年改為廣城,仁壽初改焉。束城舊曰束州,後齊廢。開皇十六年置,後改名焉。景城舊曰成平,開皇十八年改焉。高陽舊置高陽郡,開皇初郡廢。十六年置蒲州,大業初州廢,並任丘縣入焉。鄚有易城縣,後齊廢。開皇中置永寧縣,大業初廢入焉。博野舊曰博陸,後魏改為博野,後齊廢蠡吾縣入焉。有君子澱。清苑舊曰樂鄉。後齊省樊興、北新城、清苑、樂鄉入永寧,改名焉。開皇十八年改為清苑。長蘆開皇初置,並立漳河郡,郡尋廢。十六年置景州,大業初州廢。平舒舊置章武郡,開皇初廢。魯城開皇十六年置。饒陽開皇十六年分置安平、蕪蔞二縣,大業初省入焉。



 涿郡舊置幽州,後齊置東北道行臺。後周平齊,改置總管府。大業初府廢。統縣九,戶八萬四千五十九。



 薊舊置燕郡,開皇初廢,大業初置涿郡。良鄉安次涿舊置範陽郡,開皇初郡廢。固安舊曰故安,開皇六年改焉。雍奴昌平舊置東燕州及平昌郡。後周州郡並廢,後又置平昌郡。開皇初郡廢,又省萬年縣入焉。有關官。有長城。懷戎後齊置北燕州,領長寧、永豐二郡。後周去北字。開皇初郡廢,大業初州廢。有喬山,歷陽山,大、小翮山。有漷水、鳷水、涿水、阪泉水。潞舊置漁陽郡,開皇初廢。



 上谷郡開皇元年置易州。統縣六,戶三萬八千七百。



 易開皇初置黎郡,尋廢。十六年置縣。大業初置上谷郡。舊有故安縣,後齊廢。



 有駁牛山、五回嶺。有易水、徐水。淶水舊曰Z縣,後周廢。開皇元年,以範陽為Z,更置範陽於此。六年改為固安,八年廢。十年又置,為永陽。十八年改為淶水。



 Z舊範陽居此,俗號小範陽。開皇初改為Z。遂城舊曰武遂。後魏置南營州,準營州置五郡十一縣:龍城、廣興、定荒屬昌黎郡;石城、廣都
 屬建德郡;襄平、新昌屬遼東郡;永樂屬樂浪郡;富平、帶方、永安屬營丘郡。後齊唯留昌黎一郡,領永樂、新昌二縣,餘並省。開皇元年州移,三年郡廢,十八年改為遂城。有龍山。永樂舊曰北平,後周改名焉。有郎山。飛狐後周置,曰廣昌。仁壽初改焉。有慄山。



 有巨馬河。



 漁陽郡開皇六年徙玄州於此,並立總管府。大業初府廢。統縣一,戶三千九百二十五。



 無終後齊置,後周又廢徐無縣入焉。大業初置漁陽郡。有長城。有燕山、無終山。有泃河、如河、庚水、水壘水、濫水。有海。



 北平郡舊置平州。統縣一,戶二千二百六十九。



 盧龍舊置北平郡,領新昌、朝鮮二縣。後齊省朝鮮入新昌,又省遼西郡並所領海陽縣入肥如。開皇六年又省肥如入新昌,十八年改名盧龍。大業初置北平郡。有長城。有關官。有臨渝宮。有覆舟山。有碣石。有玄水、盧水、溫水、閏水、龍鮮水、巨梁水。有海。



 安樂郡舊置安州,後周改為玄州。開皇十六年州徙,尋置檀州。統縣二,戶七千五百九十九。



 燕樂後魏置廣陽郡,領大興、方城、燕樂三縣。後齊廢郡,以大興、方城入焉。



 大業初置安樂郡。有長城。有沽河。密雲後魏置密雲郡,領白檀、要陽、密雲三縣。



 後齊廢郡及二縣入密雲。又有舊安樂郡,領安市、土垠二縣,後齊廢土垠入安市,後周廢安市入密雲縣。開皇初郡廢。有長城。有桃花山、螺山。有漁水。



 遼西郡舊置營州,開皇初置總管府,大業初府廢。統縣一,戶七百五十一。



 柳城後魏置營州於和龍城,領建德、冀陽、昌黎、遼東、樂浪、營丘等郡,龍城、大興、永樂、帶方、定荒、石城、廣都、陽武、襄平、新昌、平剛、柳城、富平等縣。後齊唯留建德、冀陽二郡,永樂、帶方、龍城、大興等縣,其餘並廢。開皇元年唯留建德一郡,龍城一縣,其餘並廢。尋又廢郡,改縣為龍山,十八年改為柳城。大業初,置遼西郡。有帶方山、禿黎山、雞鳴山、松山。有渝水、
 白狼水。



 冀州於古,堯之都也。舜分州為十二,冀州析置幽、並。其於天文,自胃七度至畢十一度,為大梁,屬冀州。自尾十度至南斗十一度,為析木,屬幽州。自危十六度至奎四度,為娵訾,屬並州。自柳九度至張十六度,為鶉火,屬三河,則河內、河東也。準之星次,本皆冀州之域,帝居所在,故其界尤大。至夏廢幽、並入焉,得唐之舊矣。信都、清河、河間、博陵、恆山、趙郡、武安、襄國,其俗頗同。人性多敦厚,務在農桑,好尚儒學,而傷於遲重。前代稱冀、幽之士鈍如椎,蓋取此焉。俗重氣俠,好結朋黨,其相赴死生,亦出於仁義。故《班志》述其土風,悲歌慷慨,椎剽掘塚,亦自古
 之所患焉。前諺云「仕官不偶遇冀部」,實弊此也。魏郡,鄴都所在,浮巧成俗,雕刻之工,特云精妙,士女被服,咸以奢麗相高,其性所尚習,得京、洛之風矣。語曰:「魏郡、清河,天公無奈何!」斯皆輕狡所致。汲郡、河內,得殷之故壤,考之舊說,有紂之餘教。汲又衛地,習仲由之勇,故漢之官人,得以便宜從事,其多行殺戮,本以此焉。今風俗頗移,皆向於禮矣。長平、上黨,人多重農桑,性尤樸直,蓋少輕詐。河東、絳郡、文城、臨汾、龍泉、西河,土地沃少瘠多,是以傷於儉嗇。其俗剛強,亦風氣然乎?太原山川重復,實一都之會,本雖後齊別都,人物殷阜,然不甚機巧。俗與上
 黨頗同,人性勁悍,習於戎馬。離石、雁門、馬邑、定襄、樓煩、涿郡、上谷、漁陽、北平、安樂、遼西,皆連接邊郡,習尚與太原同俗,故自古言勇俠者,皆推幽、並云。然涿郡、太原,自前代已來,皆多文雅之士,雖俱曰邊郡,然風教不為比也。



 北海郡舊置青州,後周置總管府,開皇十四年府廢。統縣十,戶十四萬七千八百四十五。



 益都舊置齊郡,開皇初廢,大業初置北海郡。有堯山、鋋山。臨淄及東安平、西安,並後齊廢。開皇十六年又置臨淄及時水縣。大業初廢高陽、時水二縣入焉。



 有社山、葵丘、牛山、稷山。千乘舊置樂安郡,開皇初郡廢。博昌舊曰樂安,開皇十六年改焉。又十八年析置新河縣,大業初廢入焉。壽光開皇十六年置閭丘縣,大業初廢入焉。臨朐舊曰昌國。開皇六年改為逢山,又置般陽
 縣。大業初改曰臨朐,並廢般陽入焉。有逢山、沂山、穆陵山、大峴山。有汶水、浯水。都昌有箕山、阜山、白狼山。北海舊曰下密,置北海郡。後齊改郡曰高陽,開皇初郡廢。十六年分置濰州,大業初州廢,縣改名焉。營丘後齊廢,開皇十六年復。有叢角山、女節山。



 下密後魏曰膠東,後齊廢。開皇六年復,改為濰水。大業初改名焉。有鐵山。有溉水。



 齊郡舊曰齊州。統縣十,戶十五萬二千三百二十三。



 歷城舊置濟南郡,開皇初廢。大業初置齊郡,廢山茌縣入焉。有舜山、雞山、盧山、鵲山、華山、鮑山。祝阿臨邑臨濟開皇六年置,曰朝陽。十六年改曰臨濟,別置朝陽。大業初廢入焉。鄒平舊曰平原,開皇十八年改名焉。章丘舊曰高唐,開皇十六年改焉,又置營城縣。大業初廢入焉。又宋置東魏郡,後齊廢。有東陵山、長白山、龍盤山。長山舊曰武強,置廣川郡,並東清河、平原二郡入,改曰東平原郡。開皇初郡廢。又十六年置濟南縣,十八年改武強曰長山。大業初省濟南縣入焉。



 高苑後齊曰長樂。開皇十八年
 改為會城。大業初改焉。亭山舊曰衛國,後齊並土鼓,肥鄉入焉。開皇六年改名亭山。有龍舟山、儒山。淄川舊曰貝丘,置東清河郡。後齊郡廢。開皇十六年置淄州,十八年縣改名焉。大業初州廢。



 東萊郡舊置光州,開皇五年改曰萊州。統縣九,戶九萬三百五十一。



 掖舊置東萊郡,後齊並曲城,當利二縣入焉。開皇初廢郡,大業初復置郡。有缶山。有掖水、光水。膠水舊曰長廣,仁壽元年改名焉。有明堂山。盧鄉後齊盧鄉及挺城並廢。開皇十六年復置盧鄉,並廢挺城入焉。即墨後齊及不其縣並廢。開皇十六年復,並廢不其入焉。有大勞山、馬山。有田橫島。觀陽後周廢。開皇十六年復,又分置牟州。大業初州廢。昌陽有巨神山。黃舊置東牟,長廣二郡,後齊廢東牟郡入長廣郡,開皇初郡廢。牟平有牟山、龍山、金山、九目山。文登後齊置。有石橋。有文登山、斥山、之罘山。



 高密郡舊置膠州,開皇五年改為密州。統縣七,戶七萬一千九百二十。



 諸城舊曰東武,置高密郡。開皇初郡廢,十八年縣改名焉。大業初復置郡。有烽火山。東莞後齊並姑幕縣入焉。有箕山、濰水。郚城舊置平昌郡。後齊廢郡,置瑯邪縣,廢硃虛入焉。大業初改名郚城。安丘開皇十六年置,曰牟山。大業初改名,並省安昌入焉。高密後齊廢淳于縣入焉。膠西舊曰黔陬,置平昌郡。開皇初郡廢。



 十六年置縣,曰膠西。大業初又以黔陬入焉。瑯邪開皇十六年置,曰豐泉。大業初改焉。有徐山、盧山、鄣日山、膠水。



 《周禮·職方氏》:「正東曰青州。」其在天官,自須女八度至危十五度,為玄枵,於辰在子,齊之分野。吳札觀樂,聞齊之歌曰:「泱泱乎大風也哉,國未可量也。」在漢之時,俗彌侈泰,織作冰紈綺繡純麗之物,號為冠帶衣履天下。始太公以尊賢尚智為教,故士庶傳習其風,莫不矜於功名,
 依於經術,闊達多智,志度舒緩。其為失也,誇奢朋黨,言與行謬。齊郡舊曰濟南,其俗好教飾子女淫哇之音,能使骨騰肉飛,傾詭人目。俗云「齊倡」,本出此也。祝阿縣俗,賓婚大會,肴饌雖豐,至於蒸膾,嘗之而已,多則謂之不敬,共相誚責,此其異也。大抵數郡風俗,與古不殊,男子多務農桑,崇尚學業,其歸於儉約,則頗變舊風。東萊人尤樸魯,故特少文義。



\end{pinyinscope}