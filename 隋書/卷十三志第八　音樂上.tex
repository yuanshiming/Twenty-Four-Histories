\article{卷十三志第八 音樂上}

\begin{pinyinscope}

 夫音,本乎太始而生於人心,隨物感動,播於形氣。形氣既著,協於律呂,宮商克諧,名之為樂。樂者,樂也。聖人因百姓樂己之德,正之以六律,文之以五聲,詠之以九歌,舞之以八佾。實升平之冠帶,王化之源本。《記》曰:「感於物而動,故形於聲。」夫人者,兩儀之播氣,而性情之所起也,恣其流湎,往而不歸,是以五帝作樂,三王制禮,標舉人
 倫,削平淫放。其用之也,動天地,感鬼神,格祖考,諧邦國。樹風成化,象德昭功,啟萬物之情,通天下之志。若夫升降有則,宮商垂範。禮逾其制則尊卑乖,樂失其序則親疏亂。禮定其象,樂平其心,外敬內和,合情飾貌,猶陰陽以成化,若日月以為明也。



 《記》曰:「大夫無故不撤懸,士無故不撤琴瑟。」聖人造樂,導迎和氣,惡情屏退,善心興起。伊耆有葦籥之音,伏犧有網罟之詠,葛天八闋,神農五弦,事與功偕,其來已尚。黃帝樂曰《咸池》,帝嚳曰《六英》,帝顓頊曰《五莖》,帝堯曰《大章》,帝舜曰《簫韶》,禹曰《大夏》,殷湯曰《護》,武王曰《武》,周公曰《勺》。教之以風賦,弘之以孝友,大
 禮與天地同節,大樂與天地同和,禮意風猷,樂情膏潤。《傳》曰:「如有王者,必世而後仁。」成、康化致升平,刑厝而不用也。古者天子聽政,公卿獻詩。秦人有作,罕聞斯道。漢高祖時,叔孫通爰定篇章,用祀宗廟。唐山夫人能楚聲,又造房中之樂。武帝裁音律之響,定郊丘之祭,頗雜謳謠,非全雅什。漢明帝時,樂有四品:一曰《大予樂》,郊廟上陵之所用焉。則《易》所謂「先王作樂崇德,殷薦之上帝,以配祖考」者也。二曰雅頌樂,闢雍饗射之所用焉。則《孝經》所謂「移風易俗,莫善於樂」者也。三曰黃門鼓吹樂,天子宴群臣之所用焉。則《詩》所謂「坎坎鼓我,蹲蹲儛我」者也。
 其四曰短簫鐃歌樂,軍中之所用焉。黃帝時,岐伯所造,以建武揚德,風敵勵兵,則《周官》所謂「王師大捷,則令凱歌」者也。又採百官詩頌,以為登歌,十月吉辰,始用烝祭。董卓之亂,正聲咸蕩。漢雅樂郎杜夔,能曉樂事,八音七始,靡不兼該。



 魏武平荊州,得夔,使其刊定雅律。魏有先代古樂,自夔始也。自此迄晉,用相因循,永嘉之寇,盡淪胡羯。於是樂人南奔,穆皇羅鐘磬,苻堅北敗,孝武獲登歌。



 晉氏不綱,魏圖將霸,道武克中山,太武平統萬,或得其宮懸,或收其古樂,於時經營是迫,雅器斯寢。孝文頗為詩歌,以勖在位,謠俗流傳,布諸音律。大臣馳騁漢、魏,
 旁羅宋、齊,功成奮豫,代有制作。莫不各揚廟舞,自造郊歌,宣暢功德,輝光當世,而移風易俗,浸以陵夷。



 梁武帝本自諸生,博通前載,未及下車,意先風雅,爰詔凡百,各陳所聞。帝又自糾擿前違,裁成一代。周太祖發跡關隴,躬安戎狄,群臣請功成之樂,式遵周舊,依三材而命管,承六典而揮文。而《下武》之聲,豈姬人之唱,登歌之奏,協鮮卑之音,情動於中,亦人心不能已也。昔仲尼返魯,風雅斯正,所謂有其藝而無其時。高祖受命惟新,八州同貫,制氏全出於胡人,迎神猶帶於邊曲。及顏、何驟請,頗涉雅音,而繼想聞《韶》,去之彌遠。若夫二南斯理,八風揚
 節,順序旁通,妖淫屏棄,宮徵流唱,翱翔率舞,弘仁義之道,安性命之真,君子益厚,小人無悔,非大樂之懿,其孰能與於此者哉!是以舜詠《南風》而虞帝昌,紂歌北鄙而殷王滅。



 大樂不紊,則王政在焉。故錄其不相因襲,以備於志。《周官》大司樂一千三百三十九人。漢郊廟及武樂,三百八十人。煬帝矜奢,頗玩淫曲,御史大夫裴蘊,揣知帝情,奏括周、齊、梁、陳樂工子弟,及人間善聲調者,凡三百餘人,並付太樂。



 倡優〕雜,咸來萃止。其哀管新聲,淫弦巧奏,皆出鄴城之下,高齊之舊曲云。



 梁氏之初,樂緣齊舊。武帝思弘古樂,天監元年,遂下詔
 訪百僚曰:「夫聲音之道,與政通矣,所以移風易俗,明貴辨賤。而《韶》、《護》之稱空傳,《咸》、《英》之實靡托,魏晉以來,陵替滋甚。遂使雅鄭混淆,鐘石斯謬,天人缺九變之節,朝宴失四懸之儀。朕昧旦坐朝,思求厥旨,而舊事匪存,未獲厘正,寤寐有懷,所為嘆息。卿等學術通明,可陳其所見。」於是散騎常侍、尚書僕射沈約奏答曰:「竊以秦代滅學,《樂經》殘亡。至於漢武帝時,河間獻王與毛生等,共採《周官》及諸子言樂事者,以作《樂記》。其內史丞王定,傳授常山王禹。劉向校書,得《樂記》二十三篇,與禹不同。向《別錄》,有《樂歌詩》四篇、《趙氏雅琴》七篇、《師氏雅琴》八篇、《龍氏雅
 琴》百六篇。唯此而已。《晉中經簿》無復樂書,《別錄》所載,已復亡逸。案漢初典章滅絕,諸儒捃拾溝渠墻壁之間,得片簡遺文,與禮事相關者,即編次以為禮,皆非聖人之言。《月令》取《呂氏春秋》,《中庸》、《表記》、《防記》、《緇衣》皆取《子思子》,《樂記》取《公孫尼子》,《檀弓》殘雜,又非方幅典誥之書也。禮既是行己經邦之切,故前儒不得不補綴以備事用。樂書事大而用緩,自非逢欽明之主,制作之君,不見詳議。漢氏以來,主非欽明,樂既非人臣急事,故言者寡。陛下以至聖之德,應樂推之符,實宜作樂崇德,殷薦上帝。而樂書淪亡,尋案無所。宜選諸生,分令尋討經史百家,凡
 樂事無小大,皆別纂錄。乃委一舊學,撰為樂書,以起千載絕文,以定大梁之樂。使《五英》懷慚,《六莖》興愧。」



 是時對樂者七十八家,咸多引流略,浩蕩其詞,皆言樂之宜改,不言改樂之法。



 帝既素善鐘律,詳悉舊事,遂自制定禮樂。又立為四器,名之為通。通受聲廣九寸,宣聲長九尺,臨岳高一寸二分。每通皆施三弦。一曰玄英通:應鐘弦,用一百四十二絲,長四尺七寸四分差強;黃鐘弦,用二百七十絲,長九尺;大呂弦,用二百五十二絲,長八尺四寸三分差弱。二曰青陽通:太簇弦,用二百四十絲,長八尺;夾鐘弦,用二百二十四絲,長七尺五寸弱;姑洗弦,用
 二百一十四絲,長七尺一寸一分強。三曰硃明通:中呂弦,用一百九十九絲,長六尺六寸六分弱;蕤賓弦,用一百八十九絲,長六尺三寸二分強;林鐘弦,用一百八十絲,長六尺。四曰白藏通:夷則弦,用一百六十八絲,長五尺六寸二分弱;南呂弦,用一百六十絲,長五尺三寸二分大強;無射弦,用一百四十九絲,長四尺九寸九分強。因以通聲,轉推月氣,悉無差違,而還相得中。又制為十二笛:黃鐘笛,長三尺八寸,大呂笛,長三尺六寸,太簇笛,長三尺四寸,夾鐘笛,長三尺二寸,姑洗笛,長三尺一寸,中呂笛,長二尺九寸,蕤賓笛,長二尺八寸,林鐘笛,
 長二尺七寸,夷則笛,長二尺六寸,南呂笛,長二尺五寸,無射笛,長二尺四寸,應鐘笛,長二尺三寸。用笛以寫通聲,飲古鐘玉律並周代古鐘,並皆不差。於是被以八音,施以七聲,莫不和韻。



 是時北中郎司馬何佟之上言:「案《周禮》『王出入則奏《王夏》,尸出入則奏《肆夏》,牲出入則奏《昭夏》。今樂府之《夏》,唯變《王夏》為《皇夏》,蓋緣秦、漢以來稱皇故也。而齊氏仍宋儀注,迎神奏《昭夏》,皇帝出入奏《永至》,牲出入更奏引牲之樂。其為舛謬,莫斯之甚。請下禮局改正。」周舍議,以為「《禮》『王入奏《王夏》』,大祭祀與朝會,其用樂一也。而漢制,皇帝在廟,奏《永至》樂,朝會之日,別有《
 皇夏》。二樂有異,於禮為乖,宜除《永至》,還用《皇夏》。又《禮》『尸出入奏《肆夏》,賓入大門奏《肆夏》』,則所設唯在人神,其與迎牲之樂,不可濫也。宋季失禮,頓虧舊則,神入廟門,遂奏《昭夏》,乃以牲牢之樂,用接祖考之靈。斯皆前代之深疵,當今所宜改也。」時議又以為《周禮》云:「若樂六變,天神皆降。」神居上玄,去還怳忽,降則自至,迎則無所。可改迎為降,而送依前式。又《周禮》云「若樂八變,則地祇皆出,可得而禮」,地宜依舊為迎神。並從之。又以明堂設樂,大略與南郊不殊,惟壇堂異名,而無就燎之位。明堂則遍歌五帝,其餘同於郊式焉。



 初宋、齊代,祀天地,祭宗廟,準漢
 祠太一后土,盡用宮懸。又太常任昉亦據王肅議云:「《周官》『以六律、五聲、八音、六舞大合樂,以致鬼神,以和邦國,以諧兆庶,以安賓客,以悅遠人。』是謂六同,一時皆作。今六代舞獨分用之,不厭人心。」遂依肅議,祀祭郊廟,備六代樂。至是帝曰:「《周官》分樂饗祀,《虞書》止鳴兩懸,求之於古,無宮懸之議。何?事人禮縟,事神禮簡也。天子襲袞,而至敬不文,觀天下之物,無可以稱其德者,則以少為貴矣。大合樂者,是使六律與五聲克諧,八音與萬舞合節耳。豈謂致鬼神只用六代樂也?其後即言『分樂序之,以祭以享。』此乃曉然可明,肅則失其旨矣。推檢載籍,初無
 郊禋宗廟遍舞六代之文。唯《明堂位》曰:『禘祀周公於太廟,硃干玉戚,冕而舞《大武》,皮弁素積,裼而舞《大夏》。納夷蠻之樂於太廟,言廣魯於天下也。』夫祭尚於敬,無使樂繁禮黷。是以季氏逮暗而祭,繼之以燭,有司跛倚。其為不敬大矣。他日祭,子路與焉,質明而始,晏朝而退。孔子聞之,曰:「誰謂由也不知禮乎?』若依肅議,郊既有迎送之樂,又有登歌,各頌功德;遍以六代,繼之出入,方待樂終。此則乖於仲尼韙晏朝之意矣。」於是不備宮懸,不遍舞六代,逐所應須。即設懸,則非宮非軒,非判非特,宜以至敬所應施用耳。宗廟省迎送之樂,以其閟宮靈宅也。



 齊
 永明中,舞人冠幘並簪筆,帝曰:「筆笏蓋以記事受言,舞不受言,何事簪筆?



 豈有身服朝衣,而足綦宴履?」於是去筆。



 又晉及宋、齊,懸鐘磬大準相似,皆十六架。黃鐘之宮:北方,北面,編磬起西,其東編鐘,其東衡大於鎛,不知何代所作,其東鎛鐘。太簇之宮:東方,西面,起北。蕤賓之宮:南方,北面,起東。姑洗之宮:西方,東面,起南。所次皆如北面。設建鼓於四隅,懸內四面,各有柷爆。帝曰:「著晉、宋史者,皆言太元、元嘉四年,四廂金石大備。今檢樂府,止有黃鐘、姑洗、蕤賓、太簇四格而已。六律不具,何謂四廂?備樂之文,其義焉在?」於是除去衡鐘,設十二鎛鐘,各依辰
 位,而應其律。每一鎛鐘,則設編鐘磬各一虡,合三十六架。植建鼓於四隅。元正大會備用之。



 乃定郊禋宗廟及三朝之樂,以武舞為《大壯舞》,取《易》云「大者壯也」,正大而天地之情可見也。以文舞為《大觀舞》,取《易》云「大觀在上」,觀天之神道而四時不忒也。國樂以「雅」為稱,取《詩序》云:「言天下之事,形四方之風,謂之雅。雅者,正也。」止乎十二,則天數也。乃去階步之樂,增撤食之雅焉。



 眾官出入,宋元徽三年《儀注》奏《肅咸樂》,齊及梁初亦同。至是改為《俊雅》,取《禮記》:「司徒論選士之秀者而升之學,曰俊士也。」二郊、太廟、明堂,三朝同用焉。皇帝出入,宋孝建二年秋《起
 居注》奏《永至》,齊及梁初亦同。至是改為《皇雅》,取《詩》「皇矣上帝,臨下有赫」也。二郊、太廟同用。皇太子出入,奏《胤雅》,取《詩》「君子萬年,永錫爾胤」也。王公出入,奏《寅雅》,取《尚書》、《周官》「貳公弘化,寅亮天地」也。上壽酒,奏《介雅》,取《詩》「君子萬年,介爾景福」也。食舉,奏《需雅》,取《易》「雲上於天,需,君子以飲食宴樂」也。撤饌,奏《雍雅》,取《禮記》「大饗客出以《雍》撤也。」並三朝用之。牲出入,宋元徽二年《儀注》奏《引牲》,齊及梁初亦同。至是改為《滌雅》,取《禮記》「帝牛必在滌三月」也。薦毛血,宋元徽三年《儀注》奏《嘉薦》,齊及梁初亦同。至是改為《牷雅》,取《春秋左氏傳》「牲牷肥腯」也。



 北郊明堂、太廟
 並同用。降神及迎送,宋元徽三年《儀注》奏《昭夏》,齊及梁初亦同。至是改為《誠雅》,取《尚書》「至誠感神」也。皇帝飲福酒,宋元徽三年《儀注》奏《嘉祚》,至齊不改,梁初,改為《永祚》。至是改為《獻雅》,取《禮記·祭統》「尸飲五,君洗玉爵獻卿」。今之福酒,亦古獻之義也。北郊、明堂、太廟同用。就燎位,宋元徽三年《儀注》奏《昭遠》,齊及梁不改。就埋位,齊永明六年《儀注》奏《隸幽》。至是燎埋俱奏《禋雅》,取《周禮·大宗伯》「以禋祀祀昊天上帝」也。其辭並沈約所制。今列其歌詩三十曲云。



 《俊雅》,歌詩三曲,四言:
 設官分職,髦俊攸俟。髦俊伊何?貴德尚齒。唐乂咸事,周寧多士。區區衛國,猶賴君子。漢之得人,帝猷乃理。



 開我八襲,闢我九重。珩佩流響,纓紱有容。袞衣前邁,列闢雲從。義兼東序,事美西雍。分階等肅,異列齊恭。



 重列北上,分庭異陛。百司揚職,九賓相禮。齊宋舅甥,魯衛兄弟。思皇藹藹,群龍濟濟。我有嘉賓,實惟愷悌。



 《皇雅》,三曲,五言:
 帝德實廣運,車書靡不賓。執瑁朝群後,垂旒御百神。八荒重譯至,萬國婉來親。



 華蓋拂紫微,勾陳繞太一。容裔被緹組,參差羅蒨畢。星回照以爛,天行徐且謐。



 清蹕朝萬宇,端冕臨正陽。青絇黃金繶,袞衣文繡裳。既散華蟲採,復流日月光。



 《胤雅》,一曲,四言:自昔殷代,哲王迭有。降及周成,惟器是守。上天乃眷,大梁既受。灼灼重明,仰承元首。體乾作貳,命服斯九。置保置師,居前居後。
 前星北耀,克隆萬壽。



 《寅雅》,一曲,三言:禮莫違,樂具舉。延籓闢,朝帝所。執桓蒲,列齊莒。垂袞毳,紛容與。升有儀,降有序。齊簪紱,忘笑語。始矜嚴,終酣醑。



 《介雅》,三曲,五言:百福四象初,萬壽三元始。拜獻惟袞職,同心協卿士。北極永無窮,南山何足擬。



 壽隨百禮洽,慶與三朝升。惟皇集繁祉,景福互相仍。申錫永無遺,穰簡必來應。



 百味既含馨,六飲莫能尚。玉罍信湛湛,金卮頗搖漾。敬舉發天和,祥祉流嘉貺。



 《需雅》,八曲,七言:實體平心待和味,庶羞百品多為貴。或鼎或鼒宣九沸,楚桂胡鹽芼芳卉。加籩列俎雕且蔚。



 五味九變兼六和,令芳甘旨庶且多。三危之露九期禾,圓案方丈粲星羅。皇舉斯樂同山河。



 九州上腴非一族,玄芝碧樹壽華木。
 終朝採之不盈掬,用拂腥膻和九穀。既甘且飫致遐福。



 人欲所大味為先,興和盡敬咸在旃。碧鱗硃尾獻嘉鮮,紅毛綠翼墜輕翾。臣拜稽首萬斯年。



 擊鐘以俟惟大國,況乃御天流至德。侑食斯舉揚盛則,其禮不愆儀不忒。風猷所被深且塞。



 膳夫奉職獻芳滋,不麝不夭咸以時。調甘適苦別澠淄,其德不爽受福厘。
 於焉逸豫永無期。



 備味斯饗惟至聖,咸降人神禮為盛。或風或雅流歌詠,負鼎言歸啟殷命。悠悠四海同茲慶。



 道我六穗羅八珍,洪鼎自爨匪勞薪。荊包海物必來陳,滑甘滌水隨味和神。以斯至德被無垠。



 《雍雅》,三曲,四言:明明在上,其儀有序。終事靡愆,收鉶撤俎。乃升乃降,和樂備舉。天德莫違,人謀是與。
 敬行禮達,茲焉宴語。



 我餕惟阜,我肴孔庶。嘉味既充,食旨斯飫。屬厭無爽,沖和在御。擊壤齊歡,懷生等豫。蒸庶乃粒,實由仁恕。



 百司警列,皇在在陛。既飫且醑,卒食成禮。其容穆穆,其儀濟濟。凡百庶僚,莫不愷悌。奄有萬國,抑由天啟。



 《滌雅》,一曲,四言:將修盛禮,其儀孔熾。有腯斯牲,國門是置。不黎不翽,靡愆靡忌。呈肌獻體,永言昭事。
 俯休皇德,仰綏靈志。百福具膺,嘉祥允洎。駿奔伊在,慶覃遐嗣。



 《牷雅》,一曲,四言:反本興敬,復古昭誠。禮容宿設,祀事孔明。華俎待獻,崇碑麗牲。充哉繭握,肅矣簪纓。其膋既啟,我豆既盈。庖丁游刃,葛盧驗聲。多祉攸集,景福來並。



 《誠雅》,一曲,三言:南郊降神用懷忽慌,瞻浩蕩。盡誠潔,致虔想。出杳冥,降無象。皇情肅,具僚仰。人禮盛,神途敞。
 人愛明靈,申敬饗。感蒼極,洞玄壤。



 《誠雅》,一曲,三言:北郊迎神用地德溥,昆丘峻。揚羽翟,鼓應。出尊祗,展誠信。招海瀆,羅岳鎮。惟福祉,咸昭晉。



 《誠雅》,一曲,四言:南北郊、明堂、太廟送神同用我有明德,馨非稷黍。牲玉孔備,嘉薦惟旅。金懸宿設,和樂具舉。禮達幽明,敬行樽俎。鼓鐘雲送,遐福是與。



 《獻雅》,一曲,四言:神宮肅肅,天儀穆穆。禮獻既同,膺此釐福。
 我有馨明,無愧史祝。



 《禋雅》,一曲,四言:就燎紫宮昭煥,太一微玄。降臨下土,尊高上天。載陳珪壁,式備牲牷。雲孤清引,栒虞高懸。俯昭象物,仰致高煙。肅彼靈祉,咸達皇虔。



 《禋雅》,一曲,四言:就理盛樂斯舉,協徵調宮。靈饗慶洽,祉積化融。八變有序,三獻已終。坎牲瘞玉,酬德報功。振垂成呂,投壤生風。道無虛致,事由感通。於皇盛烈,此祚華嵩。



 普通中,薦蔬之後,改諸雅歌,敕蕭子云制詞。既無牲牢,遂省《滌雅》、《牷雅》云。



 南郊,舞奏黃鐘,取陽始化也。北郊,舞奏林鐘,取陰始化也。明堂宗廟,所尚者敬,蕤賓是為敬之名,復有陰主之義,故同奏焉。其南北郊、明堂、宗廟之禮,加有登歌。今又列其歌詩一十八曲云。



 南郊皇帝初獻,奏登歌,二曲,三言:暾既明,禮告成。惟聖祖,主上靈。爵已獻,罍又盈。息羽籥,展歌聲。人愛如在,結皇情。



 禮容盛,樽俎列。玄酒陳,陶匏設。獻清旨,
 致虔潔。王既升,樂已闋。降蒼昊,垂芳烈。



 北郊皇帝初獻,奏登歌,二曲,四言:方壇既坎,地祇已出。盛典弗愆,群望咸秩。乃升乃獻,敬成禮卒。靈降無兆,神饗載謐。允矣嘉祚,其升如日。



 至哉坤元,實惟厚載。躬茲奠饗,誠交顯晦。或升或降,搖珠動佩。德表成物,慶流皇代。純嘏不愆。祺福是賚。



 宗廟皇帝初獻,奏登歌,七曲,四言:功高禮洽,道尊樂備。三獻具舉,百司在位。
 誠敬罔愆,幽明同致。茫茫億兆,無思不遂。蓋之如天,容之如地。



 殷兆玉筐,周始邠王。於赫文祖,基我大梁。肇土七十,奄有四方。帝軒百祀,人思未忘,永言聖烈,祚我無疆。



 有夏多罪,殷人塗炭。四海倒懸,十室思亂。自天命我,殲兇殄難。既躍乃飛,言登天漢。爰饗爰祀,福祿攸贊。



 犧象既飾,罍俎斯具。我鬱載馨,黃流乃注。峨峨卿士,駿奔是務。佩上鳴階,纓還拂樹。
 悠悠億兆,天臨日煦。



 猗與至德,光被黔首。鑄熔蒼昊,甄陶區有。肅恭三獻,對揚萬壽。比屋可封,含生無咎。匪徒七百,天長地久。



 有命自天,於皇后帝。悠悠四海,莫不來祭。繁祉具膺,八神聳衛,福至有兆,慶來無際。播此餘休,於彼荒裔。



 祀典昭潔,我禮莫違。八簋充室,六龍解驂。神宮肅肅,靈寢微微。嘉薦既饗,景福攸歸。至德光被,洪祚載輝。



 明堂遍歌五帝登歌,五曲,四言:歌青帝辭:帝居在震,龍德司春。開元布澤,含和尚仁。群居既散,歲雲陽止。飭農分地,人粒惟始。雕梁繡栱,丹楹玉墀。靈威以降,百福來綏。



 歌赤帝辭:炎光在離,火為威德。執禮昭訓,持衡受則。靡草既凋,溫風以至。嘉薦惟旅,時羞孔備。齊醍在堂,笙鏞在下。匪惟七百,無絕終始。



 歌黃帝辭:
 鬱彼中壇,含靈闡化。回環氣象,輪無輟駕。布德焉在,四序將收。音宮數五,飯稷驂鳷。宅屏居中,旁臨外宇。升為帝尊,降為神主。



 歌白帝辭:神在秋方,帝居四皓。允茲金德,裁成萬寶。鴻來雀化,參見火邪。幕無玄鳥,菊有黃華。載列笙磬,式陳彞俎。靈罔常懷,惟德是與。



 歌黑帝辭:德盛乎水,玄冥紀節。陰降陽騰,氣凝象閉。司智蒞坎,駕鐵衣玄。祁寒坼地,晷度回天。
 悠悠四海,駿奔奉職。祚我無疆,永隆人極。



 太祖太夫人廟舞歌:閟宮肅肅,清廟濟濟。於穆夫人,固天攸啟。祚我梁德,膺斯盛禮。文泬達向,重簷丹陛。飾我俎彞,潔我粢盛。躬事奠饗,推尊盡敬。悠悠萬國,具承茲慶。大孝追遠,兆庶攸詠。



 太祖太夫人廟登歌:光流者遠,禮貴彌申。嘉饗云備,盛典必陳。追養自本,立愛惟親。皇情乃慕,帝服來尊。駕齊六轡,旂耀三辰。感茲霜露,事彼冬春。
 以斯孝德,永被蒸民。



 《大壯舞》奏夷則,《大觀舞》奏姑洗,取其月王也。二郊、明堂、太廟,三朝並同用。今亦列其歌詩二曲云。



 《大壯舞》歌,一曲,四言:高高在上,實愛斯人。眷求聖德,大拯彞倫。率土方燎,如火在薪。心棄心棄黔首,暮不及晨。硃光啟耀,兆發穹旻。我皇鬱起,龍躍漢津。言屆牧野,電激雷震。闕鞏之甲,彭濮之人。或貔或武,漂杵浮輪。我邦雖舊,其命惟新。六伐乃止,七德必陳。君臨萬國,遂撫八夤。



 《
 大觀舞》歌,一曲,四言:皇矣帝烈,大哉興聖。奄有四方,受天明命。居上不怠,臨下唯敬。舉無愆則,動無失正。物從其本,人遂其性。昭播九功,肅齊八柄。寬以惠下,德以為政。三趾晨儀,重輪夕映。棧壑忘阻,梯山匪夐。如日有恆,與天無竟。載陳金石,式流舞詠。《咸》、《英》、《韶》、《夏》,於茲比盛。



 相和五引:角引:萌生觸發歲在春,《咸池》始奏德尚仁,
 惉滯以息和且均。



 徵引:執衡司事宅離方,滔滔夏日火德昌,八音備舉樂無疆。



 宮引:八音資始君五聲,興此和樂感百精,優游律呂被《咸》《英》。



 商引:司秋紀兌奏西音,激揚鐘石和瑟琴,風流福被樂愔愔。



 羽引:玄英紀運冬冰折,物為音本和且悅,窮高測深長無絕。



 普通中,薦蔬以後,敕蕭子云改諸歌辭為相和引,則依五音宮商角徵羽為第次,非隨月次也。



 舊三朝設樂有登歌,以其頌祖宗之功烈,非君臣之所獻也,於是去之。三朝,第一,奏《相和五引》:第二,眾官入,奏《俊雅》;第三,皇帝入閤,奏《皇雅》;第四,皇太子發西中華門,奏《胤雅》;第五,皇帝進,王公發足;第六,王公降殿,同奏《寅雅》;第七,皇帝入儲變服;第八,皇帝變服出儲,同奏《皇雅》;
 第九,公卿上壽酒,奏《介雅》;第十,太子入預會,奏《胤雅》;十一,皇帝食舉,奏《需雅》;十二,撤食,奏《雍雅》;十三,設《大壯》武舞;十四,設《大觀》文舞;十五,設《雅歌》五曲,十六,設俳伎;十七,設《鼙舞》;十八,設《鐸舞》;十九,設《拂舞》;二十,設《巾舞》並《白紵》;二十一,設舞盤伎;二十二,設舞輪伎;二十三,設刺長追花幢伎;二十四,設受猾伎;二十五,設車輪折脰伎;二十六,設長蹻伎;二十七,設須彌山、黃山、三峽等伎;二十八,設跳鈴伎;二十九,設跳劍伎;三十,設擲倒伎;三十一,設擲倒案伎;三十二,設青絲幢伎;三十三,設一傘花幢伎;三十四,設雷幢伎;三十五,設金輪幢伎;三十六,設白
 獸幢伎;三十七,設擲蹻伎;三十八,設獮猴幢伎;三十九,設啄木幢伎;四十,設五案幢咒願伎;四十一,設闢邪伎;四十二,設青紫鹿伎;四十三,設白武伎,作訖,將白鹿來迎下;四十四,設寺子導安息孔雀、鳳凰、文鹿胡舞登連《上雲樂》歌舞伎;四十五,設緣高糸亙伎;四十六,設變黃龍弄龜伎;四十七,皇太子起,奏《胤雅》;四十八,眾官出,奏《俊雅》;四十九,皇帝興,奏《皇雅》。



 自宋、齊已來,三朝有鳳凰銜書伎。至是乃下詔曰:「朕君臨南面,道風蓋闕,嘉祥時至,為愧已多。假令巢侔軒閣,集同昌戶,猶當顧循寡德,推而不居。況於名實頓爽,自欺耳目。一日元會,太樂奏鳳
 凰銜書伎,至乃舍人受書,升殿跪奏。



 誠復興乎前代,率由自遠,內省懷慚,彌與事篤。可罷之。」



 天監四年,掌賓禮賀瑒,請議皇太子元會出入所奏。帝命別制養德之樂。瑒謂宜名《元雅》,迎送二傅亦同用之。取《禮》「一有元良,萬國以貞」之義。明山賓、嚴植之及徐勉等,以為周有九《夏》,梁有十二《雅》。此並則天數,為一代之曲。今加一雅,便成十三。瑒又疑東宮所奏舞,帝下其議。瑒以為,天子為樂,以賞諸侯之有德者。觀其舞,知其德。況皇儲養德春宮,式瞻攸屬,謂宜備《大壯》、《大觀》二舞,以宣文武之德。帝從之。於是改皇太子樂為《元貞》,奏二舞。是時禮樂制度,粲
 然有序。其後臺城淪沒,簡文帝受制於侯景。景以簡文女溧陽公主為妃,請帝及主母範淑妃宴於西州,奏梁所常用樂。景儀同索超世亦在宴筵。帝潸然屑涕。景興曰:「陛下何不樂也?」帝強笑曰:「丞相言索超世聞此以為何聲?」



 景曰:「臣且不知,何獨超世?」自此樂府不修,風雅咸盡矣。及王僧辯破侯景,諸樂並送荊州。經亂,工器頗闕,元帝詔有司補綴才備。荊州陷沒,周人不知採用,工人有知音者,並入關中,隨例沒為奴婢。



 鼓吹,宋、齊並用漢曲,又充庭用十六曲。高祖乃去四曲,留其十二,合四時也。更制新歌,以述功德。其第一,漢曲《
 硃鷺》改為《木紀謝》,言齊謝梁升也。



 第二,漢曲《思悲翁》改為《賢首山》,言武帝破魏軍於司部,肇王跡也。第三,漢曲《艾如張》改為《桐柏山》,言武帝牧司,王業彌章也。第四,漢曲《上之回》改為《道亡》,言東昏喪道,義師起樊鄧也。第五,漢曲《擁離》改為《忱威》,言破加湖元勛也。第六,漢曲《戰城南》改為《漢東流》,言義師克魯山城也。第七,漢曲《巫山高》改為《鶴樓峻》,言平郢城,兵威無敵也。第八,漢曲《上陵》改為《昏主恣淫慝》,言東昏政亂,武帝起義,平九江、姑熟,大破硃雀,伐罪吊人也。第九,漢曲《將進酒》改為《石首局》,言義師平京城,仍廢昏,定大事也。



 第十,漢曲《有所思》改為《
 期運集》,言武帝應籙受禪,德盛化遠也。十一,漢曲《芳樹》改為《於穆》,言大梁闡運,君臣和樂,休祚方遠也。十二,漢曲《上邪》改為《惟大梁》,言梁德廣運,仁化洽也。



 天監七年,將有事太廟。詔曰「《禮》云『齋日不樂』,今親奉始出宮,振作鼓吹。外可詳議。」八座丞郎參議,請與駕始出,鼓吹從而不作,還宮如常儀。帝從之,遂以定制。



 初武帝之在雍鎮,有童謠云:「襄陽白銅蹄,反縛揚州兒。」識者言,白銅蹄謂馬也;白,金色也。及義師之興,實以鐵騎,揚州之士,皆面縛,果如謠言。故即位之後,更造新聲,帝自為之詞三曲,又令沈約為三曲,以被弦管。帝既篤敬佛法,又制《善哉》、《
 大樂》、《大歡》、《天道》、《仙道》、《神王》、《龍王》、《滅過惡》、《除愛水》、《斷苦輪》等十篇,名為正樂,皆述佛法。又有法樂童子伎、童子倚歌梵唄,設無遮大會則為之。



 陳初,武帝詔求宋、齊故事。太常卿周弘讓奏曰:「齊氏承宋,咸用元徽舊式,宗祀朝饗,奏樂俱同,唯北郊之禮,頗有增益。皇帝入壝門。奏《永至》;飲福酒,奏《嘉胙》;太尉亞獻,奏《凱容》;埋牲,奏《隸幽》;帝還便殿,奏《休成》;眾官並出,奏《肅成》。此乃元徽所闕,永明六年之所加也。唯送神之樂,宋孝建二年秋《起居注》云『奏《肆夏》』,永明中,改奏《昭夏》。」帝遂依之。是時並用梁樂,唯改七室舞辭,今列之云。



 皇祖步兵府君神室奏《凱容舞》辭:於赫皇祖,宮墻高嶷。邁彼厥初,成茲峻極。縵樂簡簡,閟寢翼翼。裸饗若存,惟靈靡測。



 皇祖正員府君神室奏《凱容舞》辭:昭哉上德,浚彼洪源。道光前訓,慶流後昆。神猷緬邈,清廟斯存。以享以祀,惟祖惟尊。



 皇祖懷安府君神室奏《凱容舞》辭:選辰崇饗,飾禮嚴敬。靡愛牲牢,兼馨粢盛。明明列祖,龍光遠映。肇我王風,形斯舞詠。



 皇高祖安成府君神室奏《凱容舞》辭:
 道遙積慶,德遠昌基。永言祖武,致享從思。九章停列,八舞回墀。靈其降止,百福來綏。



 皇曾祖太常府君神室奏《凱容舞》辭:肇跡帝基,義標鴻篆。恭惟載德,瓊源方闡。享薦三清,筵陳四璉。增我堂構,式敷帝典。



 皇祖景皇帝神室奏《景德凱容舞》辭:皇祖執德,長發其祥。顯仁藏用,懷道韜光。寧斯閟寢,合此蕭薌。永昭貽厥,還符翦商。



 皇考高祖武皇帝神室奏《武德舞》辭:烝哉聖祖,撫運升離。道周經緯,功格玄祗。
 方軒邁扈,比舜陵媯。緝熙是詠,欽明在斯。



 雲雷遘屯,圖南共舉。大定揚越,震威衡楚。四奧宅心,九疇還敘。景星出翼,非雲入呂。



 德暢容辭,慶昭羽綴。於穆清廟,載揚徽烈。嘉玉既陳,豐盛斯潔。是將是享,鴻猷無絕。



 天嘉元年,文帝始定圓丘、明堂及宗廟樂。都官尚書到仲舉權奏:「眾官入出,皆奏《肅成》。牲入出,奏《引犧》。上毛血,奏《嘉薦》。迎送神,奏《昭夏》。



 皇帝入壇,奏《永至》。皇帝升陛,奏登歌。皇帝初獻及太尉亞獻、光祿勛終獻,並奏《宣烈》。皇帝飲福酒,奏《嘉胙》;就燎位,奏《昭遠》;還便殿,奏《休成》。至太
 建元年,定三朝之樂,採梁故事:第一,奏《相和》五引,各隨王月,則先奏其鐘。唯眾官入,奏《俊雅》,林鐘作,太簇參應之,取其臣道也。鼓吹作。



 皇帝出閣,奏《皇雅》,黃鐘作,太簇、夾鐘、姑洗、大呂皆應之。鼓吹作。皇太子入至十字陛,奏《胤雅》,太簇作,南呂參應之,取其二月少陽也。皇帝延王公登,奏《寅雅》,夷則作,夾鐘應之,取其月法也。皇帝入寧變服,奏《皇雅》,黃鐘作,林鐘參應之。鼓吹作。皇帝出寧及升座,皆奏《皇雅》,並如變服之作。



 上壽酒,奏《介雅》,太簇作,南呂參應之,取其陽氣盛長,萬物輻湊也。食舉,奏《需雅》,蕤賓作,大呂參應之,取火主於禮,所謂「食我以禮」也。撤
 饌,奏《雍雅》,無射作,中呂參應之,取其津潤已竭也。武舞奏《大壯》,夷則作,夾鐘參應之,七月金始王,取其堅斷也。鼓吹引而去來。文舞奏《大觀》,姑洗作,應鐘參應之,三月萬物必榮,取其布惠者也。鼓吹引而去來。眾官出,奏《俊雅》,蕤賓作,林鐘、夷則、南呂、無射、應鐘、太簇參應之。鼓吹作。皇帝起,奏《皇雅》,黃鐘作,林鐘、夷則、南呂、無射參應之。鼓吹作。祠用宋曲,宴準梁樂,蓋取人神不雜也。制曰:「可。」



 五年,詔尚書左丞劉平、儀曹郎張崖定南北郊及明堂儀注。改天嘉中所用齊樂,盡以韶為名。工就位定,協律校尉舉麾,太樂令跪贊云:「奏《懋韶》之樂。」降神,奏《通韶》;牲
 入出,奏《潔韶》;帝入壇及還便殿,奏《穆韶》。帝初再拜,舞《七德》,工執干楯,曲終復綴。出就懸東,繼舞《九序》,工執羽籥。獻爵於天神及太祖之座,奏登歌。帝飲福酒,奏《嘉韶》;就望燎,奏《報韶》。



 至六年十一月,侍中尚書左僕射、建昌侯徐陵,儀曹郎中沈罕奏來年元會儀注,稱舍人蔡景歷奉敕,先會一日,太樂展宮懸、高糸亙、五案於殿庭。客入,奏《相和》五引。帝出,黃門侍郎舉麾於殿上,掌故應之,舉於階下,奏《康韶》之樂。



 詔延王公登,奏《變韶》。奉珪璧訖,初引下殿,奏亦如之。帝興,入便殿,奏《穆韶》。更衣又出,奏亦如之。帝舉酒,奏《綏韶》。進膳,奏《侑韶》。帝御茶果,太常丞跪請
 進舞《七德》,繼之《九序》。其鼓吹雜伎,取晉、宋之舊,微更附益。舊元會有黃龍變、文鹿、師子之類,太建初定制,皆除之。至是蔡景歷奏,悉復設焉。其制,鼓吹一部十六人,則簫十三人,笳二人,鼓一人。東宮一部,降三人,簫減二人,笳減一人。諸王一部,又降一人,減簫一。庶姓一部,又降一人,復減簫一。



 及後主嗣位,耽荒於酒,視朝之外,多在宴筵。尤重聲樂,遣宮女習北方簫鼓,謂之《代北》,酒酣則奏之。又於清樂中造《黃鸝留》及《玉樹後庭花》、《金釵兩臂垂》等曲,與幸臣等制其歌詞,綺艷相高,極於輕薄。男女唱和,其音甚哀。



\end{pinyinscope}