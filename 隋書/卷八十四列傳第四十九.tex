\article{卷八十四列傳第四十九}

\begin{pinyinscope}

 北狄突厥突厥之先,平涼雜胡也,姓阿史那氏。後魏太武滅沮渠氏,阿史那以五百家奔茹茹,世居金山,工於鐵作。金山狀如兜鍪,俗呼兜鍪為「突厥」,因以為號。或云,其先國於西海之上,為鄰國所滅,男女無少長盡殺之。至一兒,不忍殺,刖足斷臂,棄於大澤中。有一牝狼,每銜肉至其所,此兒因食之,得以不死。其後遂與狼交,狼有孕焉。彼鄰
 國者,復令人殺此兒,而狼在其側。使者將殺之,其狼若為神所憑,欻然至於海東,止於山上。其山在高昌西北,下有洞穴,狼入其中,遇得平壤茂草,地方二百餘里。其後狼生十男,其一姓阿史那氏,最賢,遂為君長,故牙門建狼頭纛,示不忘本也。有阿賢設者,率部落出於穴中,世臣茹茹。至大葉護,種類漸強。當後魏之末,有伊利可汗,以兵擊鐵勒,大敗之,降五萬餘家,遂求婚於茹茹。茹茹主阿那瑰大怒,遣使罵之。伊利斬其使,率眾襲茹茹,破之。卒,弟逸可汗立,又破茹茹。病且卒,舍其子攝圖,立其弟俟斗,稱為木桿可汗。木桿勇而多智,遂擊茹茹,滅
 之,西破挹怛,東走契丹,北方戎狄悉歸之,抗衡中夏。後與西魏師入侵東魏,至於太原。



 其俗畜牧為事,隨逐水草,不恆厥處。穹廬氈帳,被發左衽,食肉飲酪,身衣裘褐,賤老貴壯。官有葉護,次設特勤,次俟利發,次吐屯發,下至小官,凡二十八等,皆世為之。有角弓、鳴鏑、甲、槊、刀、劍。善騎射,性殘忍。無文字,刻木為契。候月將滿,輒為寇抄。謀反叛殺人者皆死,淫者割勢而腰斬之。鬥傷人目者償之以女,無女則輸婦財,折支體以輸馬,盜者則償贓十倍。有死者,停尸帳中,家人親屬多殺牛馬而祭之,繞帳號呼,以刀劃面,血淚交下,七度而止。於是擇日置尸
 馬上而焚之,取灰而葬。表木為塋,立屋其中,圖畫死者形儀及其生時所經戰陣之狀。嘗殺一人,則立一石,有至千百者。父兄死,子弟妻其群母及嫂。五月中,多殺羊馬以祭天,男子好樗蒲,女子踏鞠,飲馬酪取醉,歌呼相對。敬鬼神,信巫覡,重兵死而恥病終,大抵與匈奴同俗。



 木桿在位二十年,卒,復舍其子大邏便而立其弟,是為佗缽可汗。佗缽以攝圖為爾伏可汗,統其東面,又以其弟褥但可汗子為步離可汗,居西方。時佗缽控弦數十萬,中國憚之,周、齊爭結姻好,傾府藏以事之。佗缽益驕,每謂其下曰:「我在南兩兒常孝順,何患貧也!」齊有沙門
 惠琳,被掠入突厥,因謂佗缽曰:「齊國富強者,為有佛法耳。」遂說以因緣果報之事。佗缽聞而信之,建一伽藍,遣使聘於齊氏,求《凈名》、《涅槃》、《華嚴》等經,並《十誦律》。佗缽亦躬自齋戒,繞塔行道,恨不生內地。在位十年,病且卒,謂其子菴羅曰:「吾聞親莫過於父子。



 吾兄不親其子,委地於我。我死,汝當避大邏便也。」及佗缽卒,國中將立大邏便,以其母賤,眾不服。菴羅母貴,突厥素重之。攝圖最後至,謂國中曰:「若立菴羅者,我當率兄弟以事之;如立大邏便,我必守境,利刃長矛以相待矣。」攝圖長而且雄,國人皆憚,莫敢拒者,竟以菴羅為嗣。大邏便不得立,心
 不服菴羅,每遣人罵辱之。菴羅不能制,因以國讓攝圖。國中相與議曰:「四可汗之子,攝圖最賢。」



 因迎立之,號伊利俱盧設莫何始波羅可汗,一號沙缽略。治都斤山。菴羅降居獨洛水,稱第二可汗。大邏便乃請沙缽略曰:「我與爾俱可汗子,各承父後。爾今極尊,我獨無位,何也?」沙缽略患之,以為阿波可汗,還領所部。



 沙缽略勇而得眾,北夷皆歸附之。及高祖受禪,待之甚薄,北夷大怨。會營州刺史高寶寧作亂,沙缽略與之合軍,攻陷臨渝鎮。上敕緣邊修保鄣,峻長城,以備之,仍命重將出鎮幽、並。沙缽略妻,宇文氏之女,曰千金公主,自傷宗祀絕滅,每懷
 復隋之志,日夜言之於沙缽略。由是悉眾為寇,控弦之士四十萬。上令柱國馮昱屯乙弗泊,蘭州總管叱李長叉守臨洮,上柱國李崇屯幽州,達奚長儒據周盤,皆為虜所敗。於是縱兵自木硤、石門兩道來寇,武威、天水、安定、金城、上郡、弘化、延安六畜咸盡。天子震怒,下詔曰:往者魏道衰敝,禍難相尋,周、齊抗衡,分割諸夏。突厥之虜,俱通二國。周人東慮,恐齊好之深,齊氏西虞,懼周交之厚。謂虜意輕重,國逐安危,非徒並有大敵之憂,思減一邊之防。竭生民之力,供其來往,傾府庫之財,棄於沙漠,華夏之地,實為勞擾。猶復劫剝烽戍,殺害吏民,無歲月
 而不有也。惡積禍盈,非止今日。朕受天明命,子育萬方,愍臣下之勞,除既往之弊。以為厚斂兆庶,多惠豺狼,未嘗感恩,資而為賊,違天地之意,非帝王之道。節之以禮,不為虛費,省徭薄賦,國用有餘。因入賊之物,加賜將士,息道路之民,務於耕織。清邊制勝,成策在心。



 兇醜愚暗,未知深旨,將大定之日,比戰國之時,乘昔世之驕,結今時之恨。近者盡其巢窟,俱犯北邊,朕分置軍旅,所在邀截,望其深入,一舉滅之。而遠鎮偏師,逢而摧翦,未及南上,遽已奔北,應弦染鍔,過半不歸。且彼渠帥,其數凡五,昆季爭長,父叔相猜,外示彌縫,內乖心腹,世行暴虐,家
 法殘忍。東夷諸國,盡挾私仇,西戎群長,皆有宿怨。突厥之北,契丹之徒,切齒磨牙,常伺其便。達頭前攻酒泉,其後於闐、波斯、挹怛三國一時即叛。沙缽略近趣周盤,其部內薄孤、束紇羅尋亦翻動。往年利稽察大為高麗、靺鞨所破,娑毗設又為紇支可汗所殺。與其為鄰,皆願誅剿。部落之下,盡異純民,千種萬類,仇敵怨偶,泣血拊心,銜悲積恨。圓首方足,皆人類也,有一於此,更切朕懷。彼地咎徵妖作,年將一紀,乃獸為人語,人作神言,云其國亡,訖而不見。每冬雷震,觸地火生,種類資給,惟藉水草。去歲四時,竟無雨雪,川枯蝗暴,卉木燒盡,饑疫死亡,人
 畜相半。舊居之所,赤地無依,遷徙漠南,偷存晷刻。斯蓋上天所忿,驅就齊斧,幽明合契,今也其時。故選將治兵,贏糧聚甲,義士奮發,壯夫肆憤,願取名王之首,思撻單于之背,雲歸霧集,不可數也。東極滄海,西盡流沙,縱百勝之兵,橫萬里之眾,亙朔野之追躡,望天崖而一掃。此則王恢所說,其猶射癰,何敵能當,何遠不服!但皇王舊跡,北止幽都,荒遐之表,文軌所棄。得其地不可而居,得其民不忍皆殺,無勞兵革,遠規溟海。諸將今行,義兼含育,有降者納,有違者死。異域殊方,被其擁抑,放聽復舊。廣闢邊境,嚴治關塞,使其不敢南望,永服威刑。臥鼓息
 烽,暫勞終逸,制御夷狄,義在斯乎!何用侍子之朝,寧勞渭橋之拜。普告海內,知朕意焉。



 於是以河間王弘、上柱國豆盧勣、竇榮定、左僕射高熲、右僕射虞慶則並為元帥,出塞擊之。沙缽略率阿波、貪汗二可汗等來拒戰,皆敗走遁去。時虜饑甚,不能得食,於是粉骨為糧,又多災疫,死者極眾。既而沙缽略以阿波驍悍,忌之,因其先歸,襲擊其部,大破之,殺阿波之母。阿波還無所歸,西奔達頭可汗。達頭者,名玷厥,沙缽略之從父也,舊為西面可汗。既而大怒,遣阿波率兵而東,各落歸之者將十萬騎,遂與沙缽略相攻。又有貪汗可汗,素睦於阿波,沙缽
 略奪其眾而廢之,貪汗亡奔達頭。沙缽略從弟地勤察別統部落,與沙缽略有隙,復以眾叛歸阿波。連兵不已,各遣使詣闕,請和求援,上皆不許。會千金公主上書,請為一子之例,高祖遣開府徐平和使於沙缽略。晉王廣時鎮並州,請因其釁而乘之,上不許。沙缽略遣使致書曰:「辰年九月十日,從天生大突厥天下賢聖天子伊利俱盧設莫何始波羅可汗致書大隋皇帝:使人開府徐平和至,辱告言語,具聞也。皇帝是婦父,即是翁,此是女夫,即是兒例。兩境雖殊,情義是一。今重疊親舊,子子孫孫,乃至萬世不斷,上天為證,終不違負。此國所有羊馬,
 都是皇帝畜生,彼有繒彩,都是此物,彼此有何異也!」高祖報書曰:「大隋天子貽書大突厥伊利俱盧設莫何沙缽略可汗:得書,知大有好心向此也。既是沙缽略婦翁,今日看沙缽略共兒子不異。既以親舊厚意,常使之外,今特別遣大臣虞慶則往彼看女,復看沙缽略也。」沙缽略陳兵,列其寶物,坐見慶則,稱病不能起,且曰:「我父伯以來,不向人拜。」慶則責而喻之。千金公主私謂慶則曰:「可汗豺狼性,過與爭,將嚙人。」長孫晟說諭之,攝圖辭屈,乃頓顙跪受璽書,以戴於首。既而大慚,其群下因相聚慟哭。慶則又遣稱臣,沙缽略謂其屬曰:「何名為臣?」報曰:「
 隋國稱臣,猶此稱奴耳。」沙缽略曰:「得作大隋天子奴,虞僕射之力也。」贈慶則馬千匹,並以從妹妻之。



 時沙缽略既為達頭所困,又東畏契丹,遣使告急,請將部落度漠南,寄居白道川內,有詔許之。詔晉王廣以兵援之,給以衣食,賜以車服鼓吹。沙缽略因而擊阿波,破擒之。而阿拔國部落乘虛掠其妻子。官軍為擊阿拔,敗之,所獲悉與沙缽略。



 沙缽略大喜,乃立約,以磧為界,因上表曰:大突厥伊利俱盧設始波羅莫何可汗臣攝圖言:大使尚書右僕射虞慶則至,伏奉詔書,兼宣慈旨,仰惟恩信之著,逾久愈明,徒知負荷,不能答謝。伏惟大隋皇帝之有
 四海,上契天心,下順民望,二儀之所覆載,七曜之所照臨,莫不委質來賓,回首面內。實萬世之一聖,千年之一期,求之古昔,未始聞也。突厥自天置以來,五十餘載,保有沙漠,自王蕃隅。地過萬里,士馬億數,恆力兼戎夷,抗禮華夏,在於北狄,莫與為大。頃者氣候清和,風雲順序,意以華夏其有大聖興焉。況今被沾德義,仁化所及,禮讓之風,自朝滿野。竊以天無二日,土無二王,伏惟大隋皇帝,真皇帝也。豈敢阻兵恃險,偷竊名號,今便感慕淳風,歸心有道,屈膝稽顙,永為籓附。雖復南瞻魏闕,山川悠遠,北面之禮,不敢廢失。當今待子入朝,神馬歲貢,朝
 夕恭承,唯命是視。至於削衽解辮,革音從律,習俗已久,未能改變。闔國同心,無不銜荷,不任下情欣慕之至。謹遣第七兒臣窟含真等奉表以聞。



 高祖下詔曰:「沙缽略稱雄漠北,多歷世年,百蠻之大,莫過於此。往雖與和,猶是二國,今作君臣,便成一體。情深義厚,朕甚嘉之。荷天之休,海外有截,豈朕薄德所能致此!已敕有司肅告郊廟,宜普頒天下,咸使知聞。」自是詔答諸事並不稱其名以異之。其妻可賀敦,周千金公主,賜姓楊氏,編之屬籍,改封大義公主。



 策拜窟含真為柱國,封安國公,宴於內殿,引見皇后,賞勞甚厚。沙缽略大悅,於是歲時貢獻不
 絕。七年正月,沙缽略遣其子入貢方物,因請獵於恆、代之間,又許之,仍遣人賜其酒食。沙缽略率部落再拜受賜。沙缽略一日手殺鹿十八頭,齎尾舌以獻。還至紫河鎮,其牙帳為火所燒,沙缽略惡之,月餘而卒。上為廢朝三日,遣太常吊祭焉。贈物五千段。



 初,攝圖以其子雍虞閭性芃,遺令立其弟葉護處羅侯;雍虞閭遣使迎處羅侯,將立之。處羅侯曰:「我突厥自木桿可汗以來,多以弟代兄,以庶奪嫡,失先祖之法,不相敬畏。汝當嗣位,我不憚拜汝也。」雍虞閭又遣使謂處羅侯曰:「叔與我父,共根連體,我是枝葉。寧有我作主,令根本反同枝葉,令叔父
 之尊下我卑稚!



 又亡父之命,其可廢乎!願叔勿疑。」相讓者五六,處羅侯竟立,是為葉護可汗。



 以雍虞閭為葉護。遣使上表言狀,上賜之鼓吹幡旗。處羅侯長頤僂背,眉目疏朗,勇而有謀,以隋所賜旗鼓西征阿波。敵人以為得隋兵所助,多來降附,遂生擒阿波。



 既而上書請阿波死生之命,上下其議。左僕射高熲進曰:「骨肉相殘,教之蠹也。



 存養以示寬大。」上曰:「善。」熲因奉觴進曰:「自軒轅以來,獯粥多為邊患。



 今遠窮北海,皆為臣妾,此之盛事,振古未聞,臣敢再拜上壽。」其後處羅侯又西征,中流矢而卒,其眾奉雍虞閭為主,是為頡伽施多那都藍可汗。雍
 虞閭遣使詣闕,賜物三千段。每歲遣使朝貢。時有流人楊欽亡入突厥中,謬云彭國公劉昶與宇文氏謀反,令大義公主發兵擾邊。都藍執欽以聞,並貢葧布、魚膠。其弟欽羽設部落強盛,都藍忌而擊之,斬首於陣。其年,遣其母弟褥但特勤獻於闐玉杖,上拜褥但為柱國、康國公。明年,突厥部落大人相率遣使貢馬萬匹,羊二萬口,駝、牛各五百頭。尋遣使請緣邊置市,與中國貿易,詔許之。



 平陳之後,上以陳叔寶屏風賜大義公主,主心恆不平,因書屏風為詩,敘陳亡自寄。其辭曰:「盛衰等朝暮,世道若浮萍。榮華實難守,池臺終自平。富貴今何在?空事
 寫丹青。杯酒恆無樂,弦歌詎有聲!余本皇家子,飄流入虜庭。一朝睹成敗,懷抱忽縱橫。古來共如此,非我獨申名。唯有《明君曲》,偏傷遠嫁情。」上聞而惡之,禮賜益薄。公主復與西面突厥泥利可汗連結,上恐其為變,將圖之。會主與所從胡私通,因發其事,下詔廢黜之。恐都藍不從,遣奇章公牛弘將美妓四人以啖之。時沙缽略子曰染干,號突利可汗,居北方,遣使求婚。上令裴矩謂之曰:「當殺大義主者,方許婚。」突利以為然,復譖之,都藍因發怒,遂殺公主於帳。



 都藍與達頭可汗有隙,數相征伐,上和解之,各引兵而去。



 十七年,突利遣使來逆女,上舍之
 太常,教習六禮,妻以宗女安義公主。上欲離間北夷,故特厚其禮,遣牛弘、蘇威、斛律孝卿相繼為使,突厥前後遣使入朝三百七十輩。突利本居北方,以尚主之故,南徙度斤舊鎮,錫賚優厚。雍虞閭怒曰:「我大可汗也,反不如染干!」於是朝貢遂絕,數為邊患。十八年,詔蜀王秀出靈州道以擊之。明年,又遣漢王諒為元帥,左僕射高熲率將軍王詧、上柱國趙仲卿並出朔州道,右僕射楊素率柱國李徹、韓僧壽出靈州,上柱國燕榮出幽州,以擊之。



 雍虞閭與玷厥舉兵攻染干,盡殺其兄弟子侄,遂度河,入蔚州。染干夜以五騎與隋使長孫晟歸朝。上令染
 干與雍虞閭使者因頭特勤相辯詰,染干辭直,上乃厚待之。



 雍虞閭弟都速六棄其妻子,與突利歸朝,上嘉之。敕染干與都速六樗蒲,稍稍輸以寶物,用慰其心。夏六月,高煩、楊素擊玷厥,大破之。拜染干為意利珍豆啟民可汗,華言「意智健」也。啟民上表謝恩曰:「臣既蒙豎立,復改官名,昔日奸心,今悉除去,奉事至尊,不敢違法。」上於朔州築大利城以居之。是時安義主已卒,上以宗女義成公主妻之,部落歸者甚眾。雍虞閭又擊之,上復令入塞。雍虞閭侵掠不已,遷於河南,在夏、勝二州之間,發徒掘塹數百里,東西拒河,盡為啟民畜牧之地。於是遣越
 國公楊素出靈州,行軍總管韓僧壽出慶州,太平公史萬歲出燕州,大將軍姚辯出河州,以擊都藍。師未出塞,而都藍為其麾下所殺,達頭自立為步迦可汗,其國大亂。遣太平公史萬歲出朔州以擊之,遇達頭於大斤山,虜不戰而遁,追斬首虜二千餘人。晉王廣出靈州,達頭遁逃而去。尋遣其弟子俟利伐從磧東攻啟民。上又發兵助啟民守要路,俟利伐退走入磧。啟民上表陳謝曰:「大隋聖人莫緣可汗,憐養百姓,如天無不覆也,如地無不載也。諸姓蒙威恩,赤心歸服,並將部落歸投聖人可汗來也。或南入長城,或住白道,人民羊馬,遍滿山谷。染
 干譬如枯木重起枝葉,枯骨重生皮肉,千萬世長與大隋典羊馬也。」



 仁壽元年,代州總管韓洪為虜所敗於恆安,廢為庶人。詔楊素為雲州道行軍元帥,率啟民北征。斛薛等諸姓初附於啟民,至是而叛。素軍河北,值突厥阿勿思力俟斤等南度,掠啟民男女六千口、雜畜二十餘萬而去。素率上大將軍梁默輕騎追之,轉戰六十餘里,大破俟斤,悉得人畜以歸啟民。素又遣柱國張定和、領軍大將軍劉升別路邀擊,並多斬獲而還。兵既渡河,賊復掠啟民部落,素率驃騎範貴於窟結谷東南奮擊,復破之,追奔八十餘里。是歲,泥利可汗及葉護俱被鐵
 勒所敗。步迦尋亦大亂,奚、霫五部內徙,步迦奔吐谷渾。啟民遂有其眾,歲遣朝貢。



 大業三年四月,煬帝幸榆林,啟民及義成公主來朝行宮,前後獻馬三千匹。帝大悅,賜物萬二千段。啟民上表曰:「已前聖人先帝莫緣可汗存在之日,憐臣,賜臣安義公主,種種無少短。臣種末為聖人先帝憐養,臣兄弟妒惡,相共殺臣,臣當時無處去,向上看只見天,下看只見地,實憶聖人先帝言語,投命去來。聖人先帝見臣,大憐臣,死命養活,勝於往前,遣臣作大可汗坐著也。其突厥百姓,死者以外,還聚作百姓也。至尊今還如聖人先帝,捉天下四方坐也。還養活臣
 及突厥百姓,實無少短。臣今憶想聖人及至尊養活事,具奏不可盡,並至尊聖心裡在。臣今非是舊日邊地突厥可汗,臣即是至尊臣民,至尊憐臣時,乞依大國服飾法用,一同華夏。



 臣今率部落,敢以上聞,伏願天慈,不違所請。」表奏,帝下其議,公卿請依所奏。



 帝以為不可,乃下詔曰:「先王建國,夷夏殊風,君子教民,不求變俗。斷發文身,咸安其性,旃裘卉服,各尚所宜,因而利之,其道弘矣。何必化諸削衽,縻以長纓,豈遂性之至理,非包含之遠度。衣服不同,既辨要荒之敘,庶類區別,彌見天地之情。」仍璽書答啟民,以為磧北未靜,猶須征戰,但使好心孝
 順,何必改變衣服也。



 帝法駕御千人大帳,享啟民及其部落酋長三千五百人,賜物二十萬段,其下各有差。



 復下詔曰:「德合天地,覆載所以弗遣,功格區宇,聲教所以咸洎。至於梯山航海,請受正朔,襲冠解辮,同彼臣民。是故《王會》納貢,義彰前冊,呼韓入臣,待以殊禮。突厥意利珍豆啟民可汗志懷沈毅,世修籓職。往者挺身違難,拔足歸仁,先朝嘉此款誠,授以徽號。資其甲兵之眾,收其破滅之餘,復祀於既亡之國,繼絕於不存之地。斯固施均亭育,澤漸要荒者矣。朕以薄德,祗奉靈命,思播遠猷,光融今緒,是以親巡朔野,撫寧籓服。啟民深委誠心,入
 奉朝覲,率其種落,拜首軒墀,言念丹款,良以嘉尚。宜隆榮數,式優恆典。可賜路車、乘馬、鼓吹、幡旗,贊拜不名,位在諸侯王上。」帝親巡云內,水斥金河而東,北幸啟民所居。啟民奉觴上壽,跪伏甚恭。帝大悅,賦詩曰:「鹿塞鴻旗駐,龍庭翠輦回。氈帳望風舉,穹廬向日開。呼韓頓顙至,屠耆接踵來。索辮擎膻肉,韋韝獻酒杯。何如漢天子,空上單于臺。」帝賜啟民及主金甕各一,及衣服被褥錦彩,特勤以下各有差。先是,高麗私通使啟民所,啟民推誠奉國,不敢隱境外之交。是日,將高麗使人見,敕令牛弘宣旨謂之曰:「朕以啟民誠心奉國,故親至其所。明年當往
 涿郡。爾還日,語高麗王知,宜早來朝,勿自疑懼。存育之禮,當同於啟民。如或不朝,必將啟民巡行彼土。」使人甚懼。啟民仍扈從入塞,至定襄,詔令歸籓。



 明年,朝於東都,禮賜益厚。是歲,疾終,上為之廢朝三日,立其子咄吉世,是為始畢可汗。表請尚公主,詔從其俗。十一年,來朝於東都。其年,車駕避暑汾陽宮,八月,始畢率其種落入寇,圍帝於雁門。詔諸郡發兵赴行在所,援軍方至,始畢引去。由是朝貢遂絕。明年,復寇馬邑,唐公以兵擊走之。隋末亂離,中國人歸之者無數,遂大強盛,勢陵中夏。迎蕭皇后,置於定襄。薛舉、竇建德、王世充、劉武周、梁師都、李
 軌、高開道之徒,雖僭尊號,皆北面稱臣,受其可汗之號。使者往來,相望於道也。



 西突厥西突厥者,木桿可汗之子大邏便也。與沙缽略有隙,因分為二,漸以強盛。東拒都斤,西越金山,龜茲、鐵勒、伊吾及西域諸胡悉附之。大邏便為處羅侯所執,其國立鞅素特勤之子,是為泥利可汗。卒,子達漫立,號泥撅處羅可汗。其母向氏,本中國人,生達漫而泥利卒,向氏又嫁其弟婆實特勤。開皇末,婆實共向氏入朝,遇達頭亂,遂留京師,每舍之鴻臚寺。處羅可汗居無恆處,然多在烏
 孫故地。復立二小可汗,分統所部。一在石國北,以制諸胡國。一居龜茲北,其地名應娑。官有俟發、閻洪達,以評議國事,自餘與東國同。每五月八日,相聚祭神,歲遣重臣向其先世所居之窟致祭焉。



 當大業初,處羅可汗撫御無道,其國多叛,與鐵勒屢相攻,大為鐵勒所敗。時黃門侍郎裴矩在敦煌引致西域,聞國亂,復知處羅思其母氏,因奏之。煬帝遣司朝謁者崔君肅齎書慰諭之。處羅甚踞,受詔不肯起。君肅謂處羅曰:「突厥本一國也,中分為二,自相仇敵。每歲交兵,積數十年而莫能相滅者,明知啟民與處羅國其勢敵耳。今啟民舉其部落,兵且
 百萬,入臣天子,甚有丹誠者,何也?但以切恨可汗而不能獨制,故卑事天子以借漢兵,連二大國,欲滅可汗耳。百官兆庶咸請許之,天子弗違,師出有日矣。顧可汗母向氏,本中國人,歸在京師,處於賓館。聞天子之詔,懼可汗之滅,旦夕守闕,哭泣悲哀。是以天子憐焉,為其輟策。向夫人又匍匐謝罪,因請發使以召可汗,令入內屬,乞加恩禮,同於啟民。天子從之,故遣使到此。可汗若稱籓拜詔,國乃永安,而母得延壽;不然者,則向夫人為誑天子,必當取戮而傳首虜庭。發大隋之兵,資北蕃之眾,左提右挈,以擊可汗,死亡則無日矣。奈何惜兩拜之禮,剿
 慈母之命,吝一句稱臣,喪匈奴國也!」處羅聞之,矍然而起,流涕再拜,跪受詔書。君肅又說處羅曰:「啟民內附,先帝嘉之,賞賜極厚,故致兵強國富。今可汗後附,與之爭寵,須深結於天子,自表至誠。既以道遠,未得朝覲,宜立一功,以明臣節。」處羅曰:「如何?」君肅曰:「吐谷渾者,啟民少子莫賀咄設之母家也。今天子又以義成公主妻於啟民,啟民畏天子之威而與之絕。



 吐谷渾亦因憾漢故,職貢不修。可汗若請誅之,天子必許。漢擊其內,可汗攻其外,破之必矣。然後身自入朝,道路無阻,因見老母,不亦可乎?」處羅大喜,遂遣使朝貢。



 帝將西狩,六年,遣侍御史
 韋節召處羅,今與車駕會於大斗拔谷。其國人不從,處羅謝使者,辭以他故。帝大怒,無如之何。適會其酋長射匱遣使來求婚,裴矩因奏曰:「處羅不朝,恃強大耳。臣請以計弱之,分裂其國,即易制也。射匱者,都六之子,達頭之孫,世為可汗,君臨西面。今聞其失職,附隸於處羅,故遣使來,以結援耳。願厚禮其使,拜為大可汗,則突厥勢分,兩從我矣。」帝曰:「公言是也。」因遣裴矩朝夕至館,微諷諭之。帝於仁風殿召其使者,言處羅不順之意,稱射匱有好心,吾將立為大可汗,令發兵誅處羅,然後當為婚也。帝取桃竹白羽箭一枝以賜射匱,因謂之曰:「此事宜
 速,使疾如箭也。」使者返,路經處羅,處羅愛箭,將留之,使者譎而得免。射匱聞而大喜,興兵襲處羅,處羅大敗,棄妻子,將左右數千騎東走。在路又被劫掠,遁於高昌東,保時羅漫山。高昌王麴伯雅上狀,帝遣裴矩將向氏親要左右,馳至玉門關晉昌城。矩遣向氏使詣處羅所,論朝廷弘養之義,丁寧曉諭之,遂入朝,然每有怏怏之色。以七年冬,處羅朝於臨朔宮,帝享之。處羅稽首謝曰:「臣總西面諸蕃,不得早來朝拜,今參見遲晚,罪責極深,臣心裡悚懼,不能道盡。」帝曰:「往者與突厥相侵擾,不得安居。今四海既清,與一家無異,朕皆欲存養,使遂性靈。譬
 如天上止有一個日照臨,莫不寧帖;若有兩個三個日,萬物何以得安?比者亦知處羅總攝事繁,不得早來相見。今日見處羅,懷抱豁然歡喜,處羅亦當豁然,不煩在意。」明年元會,處羅上壽曰:「自天以下,地以上,日月所照,唯有聖人可汗。今是大日,願聖人可汗千歲萬歲常如今日也。」



 詔留其累弱萬餘口,令其弟達度關牧畜會寧郡。處羅從征高麗,賜號為曷薩那可汗,賞賜甚厚。十年正月,以信義公主嫁焉,賜錦彩袍千具,彩萬匹。帝將復其故地,以遼東之役,故未遑也。每從巡幸。江都之亂,隨化及至河北。化及將敗,奔歸京師,為北蕃突厥所害。



 鐵勒鐵勒之先,匈奴之苗裔也,種類最多。自西海之東,依據山谷,往往不絕。獨洛河北有僕骨、同羅、韋紇、拔也古、覆羅並號俟斤,蒙陳、吐如紇、斯結、渾、斛薛等諸姓,勝兵可二萬。伊吾以西,焉耆之北,傍白山,則有契弊、薄落職、乙咥、蘇婆、那曷、烏言雚、紇骨、也咥、於尼言雚等,勝兵可二萬。金山西南,有薛延陀、咥勒兒、十槃、達契等,一萬餘兵。康國北,傍阿得水,則有訶咥、曷昚、撥忽、比干、具海、曷比悉、何嵯蘇、拔也未渴達等,有三萬許兵。得嶷海東西,有蘇路羯、三索咽、蔑促、隆忽等諸姓,八千餘。拂菻東則有恩屈、
 阿蘭、北褥九離、伏嗢昏等,近二萬人。北海南則都波等。雖姓氏各別,總謂為鐵勒。並無君長,分屬東、西兩突厥。居無恆所,隨水草流移。人性兇忍,善於騎射,貪婪尤甚,以寇抄為生。近西邊者,頗為藝植,多牛羊而少馬。自突厥有國,東西征討,皆資其用,以制北荒。



 開皇末,晉王廣北征,納啟民,大破步迦可汗,鐵勒於是分散。大業元年,突厥處羅可汗擊鐵勒諸部,厚稅斂其物,又猜忌薛延陀等,恐為變,遂集其魁帥數百人盡誅之。由是一時反叛,拒處羅,遂立俟利發俟斤契弊歌楞為易勿真莫何可汗,居貪汗山。復立薛延陀內俟斤字也咥為小可汗。處
 羅可汗既敗,莫何可汗始大。莫何勇毅絕倫,甚得眾心,為鄰國所憚,伊吾、高昌、焉耆諸國悉附之。



 其俗大抵與突厥同,唯丈夫婚畢,便就妻家,待產乳男女,然後歸舍,死者埋殯之,此其異也。大業三年,遣使貢方物,自是不絕云。



 奚奚本曰庫莫奚,東部胡之種也。為慕容氏所破,遺落者竄匿松、漠之間。其俗甚為不潔,而善射獵,好為寇鈔。初臣於突厥,後稍強盛,分為五部:一曰辱紇王,二曰莫賀弗,三曰契個,四曰木昆,五曰室得。每部俟斤一人為其
 帥。隨逐水草,頗同突厥。有阿會氏,五部中為盛,諸部皆歸之。每與契丹相攻擊,虜獲財畜,因而得賞。死者以葦薄裹尸,懸之樹上。自突厥稱籓之後,亦遣使入朝,或通或絕,最為無信。大業時,歲遣使貢方物。



 契丹室韋契丹之先,與庫莫奚異種而同類,並為慕容氏所破,俱竄於松、漠之間。其後稍大,居黃龍之北數百里。其俗頗與靺鞨同。好為寇盜。父母死而悲哭者,以為不壯。但以其尸置於山樹之上,經三年之後,乃收其骨而焚之。因酹而祝曰:「冬月時,向陽食。若我射獵時,使我多得豬鹿。」
 其無禮頑嚚,於諸夷最甚。當後魏時,為高麗所侵,部落萬餘口求內附,止於白貔河。其後為突厥所逼,又以萬家寄於高麗。開皇四年,率諸莫賀弗來謁。五年,悉其眾款塞,高祖納之,聽居其故地。六年,其諸部相攻擊,久不止,又與突厥相侵,高祖使使責讓之。其國遣使詣闕,頓顙謝罪。其後契丹別部出伏等背高麗,率眾內附。高祖納之,安置於渴奚那頡之北。



 開皇末,其別部四千餘家背突厥來降。上方與突厥和好,重失遠人之心,悉令給糧還本,敕突厥撫納之。固辭不去。部落漸眾,遂北徙逐水草,當遼西正北二百里,依托紇臣水而居。東西亙五
 百里,南北三百里,分為十部。兵多者三千,少者千餘,逐寒暑,隨水草畜牧。有征伐,則酋帥相與議之,興兵動眾合符契。突厥沙缽略可汗遣吐屯潘垤統之。



 室韋,契丹之類也。其南者為契丹,在北者號室韋,分為五部,不相總一,所謂南室韋、北室韋、缽室韋、深末怛室韋、大室韋。並無君長,人民貧弱,突厥常以三吐屯總領之。



 南室韋在契丹北三千里,土地卑濕,至夏則移向西北貸勃、欠對二山,多草木,饒禽獸,又多蚊蚋,人皆巢居,以避其患。漸分為二十五部,每部有餘莫弗瞞咄,猶酋長也。死則子弟代立,嗣絕則擇賢豪而立之。其俗丈夫皆被發,婦人盤
 發,衣服與契丹同。乘牛車,籧篨為屋,如突厥氈車之狀。渡水則束薪為伐,或以皮為舟者。馬則織草為韉,結繩為轡。寢則屈為屋,以籧篨覆上,移則載行。以豬皮為席,編木為藉。婦女皆抱膝而坐。氣候多寒,田收甚薄,無羊,少馬,多豬牛。造酒食啖,與靺鞨同俗。婚嫁之法,二家相許,婿輒盜婦將去,然後送牛馬為娉,更將歸家。待有娠,乃相隨還舍。婦人不再嫁,以為死人之妻難以共居。部落共為大棚,人死則置尸其上。居喪三年,年唯四哭。其國無鐵,取給於高麗。多貂。



 南室韋北行十一日至北室韋,分為九部落,繞吐紇山而居。其部落渠帥號乞引莫
 賀咄,每部有莫何弗三人以貳之。氣候最寒,雪深沒馬。冬則入山,居土穴中,牛畜多凍死。饒麞鹿,射獵為務,食肉衣皮。鑿冰,沒水中而網射魚鱉。地多積雪,懼陷坑阱,騎木而行。俗皆捕貂為業,冠以狐狢,衣以魚皮。



 又北行千里,至缽室韋,依胡布山而住,人眾多北室韋,不知為幾部落。用樺皮蓋屋,其餘同北室韋。



 從缽室韋西南四日行,至深末怛室韋,因水為號也。冬月穴居,以避太陰之氣。



 又西北數千里,至大室韋,徑路險阻,語言不通。尤多貂及青鼠。



 北室韋時遣使貢獻,餘無至者。



 史臣曰:四夷之為中國患也久矣,北狄尤甚焉。種落實
 繁,迭雄邊塞,年代遐邈,非一時也。五帝之世,則有獯粥焉;其在三代,則獫狁焉;逮乎兩漢,則匈奴焉;當塗、典午,則烏丸、鮮卑焉;後魏及周,則蠕蠕、突厥焉。此其酋豪,相繼互為君長者也。皆以畜牧為業,侵鈔為資,倏來忽往,雲飛鳥集。智謀之士,議和親於廟堂之上,折沖之臣,論奮擊於塞垣之下。然事無恆規,權無定勢,親疏因其強弱,服叛在其盛衰。衰則款塞頓顙,盛則彎弓寇掠,屈申異態,強弱相反。正朔所不及,冠帶所不加,唯利是視,不顧盟誓。至於莫相救讓,驕黠憑陵,和親約結之謀,行師用兵之事,前史論之備矣,故不詳而究焉。及蠕蠕衰微,
 突厥始大,至於木桿,遂雄朔野。東極東胡舊境,西盡烏孫之地,彎弓數十萬,列處於代陰,南向以臨周、齊。二國莫之能抗,爭請盟好,求結和親。乃與周合從,終亡齊國。高祖遷鼎,厥徒孔熾,負其眾力,將蹈秦郊。內自相圖,遂以乖亂,達頭可汗遠遁,啟民願保塞下。於是推亡固存,返其舊地,助討餘燼,部眾遂強。卒於仁壽,不侵不叛,暨乎始畢,未虧臣禮。煬帝撫之非道,始有雁門之圍。俄屬群盜並興,於此浸以雄盛,豪傑雖建名號,莫不請好息民。於是分置官司,總統中國,子女玉帛,相繼於道,使者之車,往來結轍。自古蕃夷驕僭,未有若斯之甚也。及聖
 哲膺期,掃除氛昆,暗於時變,猶懷旅拒,率其群醜,屢隳亭鄣,殘毀我雲、代,搖蕩我太原,肆掠於涇陽,飲馬於渭汭。聖上奇謀潛運,神機密動,遂使百世不羈之虜一舉而滅,瀚海龍庭之地,畫為九州,幽都窮發之民,隸於編戶,實帝皇所不及,書契所未聞。由此言之,雖天道有盛衰,亦人事之工拙也。加以為而弗恃,有而弗居,類天地之含容,同陰陽之化育,斯乃大道之行也,固無得而稱
 焉。



\end{pinyinscope}