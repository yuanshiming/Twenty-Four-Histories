\article{卷四十二列傳第七 李德林子百藥}

\begin{pinyinscope}

 李德林,字公輔,博陵安平人也。祖壽,湖州戶曹從事。父敬族,歷太學博士、鎮遠將軍。魏孝靜帝時,命當世通人正定文籍,以為內校書,別在直閤省。德林幼聰敏,年數歲,誦左思《蜀都賦》,十餘日便度。高隆之見而嗟嘆,遍告朝士,云:「若假其年,必為天下偉器。」鄴京人士多就宅觀之,月餘,日中車馬不絕。年十五,誦五經及古今文集,日
 數千言。俄而該博墳典,陰陽緯候,無不通涉。善屬文,辭核而理暢。魏收嘗對高隆之謂其父曰:「賢子文筆終當繼溫子升。」隆之大笑曰:「魏常侍殊已嫉賢,何不近比老彭,乃遠求溫子!」年十六,遭父艱,自駕靈輿,反葬故里。時正嚴冬,單衰跣足,州里人物由是敬慕之。博陵豪族有崔諶者,僕射之兄,因休假還鄉,車服甚盛。將從其宅詣德林赴吊,相去十餘里,從者數十騎,稍稍減留。比至德林門,才餘五騎,云不得令李生怪人燻灼。德林居貧感軻,母氏多疾,方留心典籍,無復宦情。其後,母病稍愈,逼令仕進。



 任城王湝為定州刺史,重其才,召入州館。朝夕
 同游,殆均師友,不為君民禮數。嘗語德林云:「竊聞蔽賢蒙顯戮。久令君沈滯,吾獨得潤身,朝廷縱不見尤,亦懼明靈所譴。」於是舉秀才入鄴,於時天保八年也。王因遺尚書令楊遵彥書云:「燕趙固多奇士,此言誠不為謬。今歲所貢秀才李德林者,文章學識,固不待言,觀其風神器宇,終為棟梁之用。至如經國大體,是賈生、晁錯之儔;雕蟲小技,殆相如、子雲之輩。今雖唐、虞君世,俊乂盈朝,然修大廈者,豈厭夫良材之積也?



 吾嘗見孔文舉《薦禰衡表》云:『洪水橫流,帝思俾乂。』以正平比夫大禹,常謂擬諭非倫。今以德林言之,便覺前言非大。」遵彥即命德林
 制《讓尚書令表》,援筆立成,不加治點。因大相賞異,以示吏部郎中陸卬。卬云:「已大見其文筆,浩浩如長河東注。比來所見,後生制作,乃涓澮之流耳。」卬仍命其子乂與德林周旋,戒之曰:「汝每事宜師此人,以為模楷。」時遵彥銓衡,深慎選舉,秀才擢第,罕有甲科。德林射策五條,考皆為上,授殿中將軍。既是西省散員,非其所好,又以天保季世,乃謝病還鄉,闔門守道。乾明初,遵彥奏追德林入議曹。皇建初,下詔搜揚人物,復追赴晉陽。撰《春思賦》一篇,代稱典麗。是時長廣王作相,居守在鄴。敕德林還京,與散騎常侍高元海等參掌機密。王引授丞相府行
 參軍。未幾而王即帝位,授奉朝請,寓直舍人省。河清中,授員外散騎侍郎,帶齋帥,仍別直機密省。天統初,授給事中,直中書,參掌詔誥。尋遷中書舍人。武平初,加通直散騎侍郎。又敕與中書侍郎宋士素、副侍中趙彥深別典機密。尋丁母艱去職,勺飲不入口五日。因發熱病,遍體生瘡,而哀泣不絕。諸士友陸騫、宋士素,名醫張子彥等,為合湯藥。德林不肯進,遍體洪腫,數日間,一時頓差,身力平復。諸人皆云孝感所致。太常博士巴叔仁表上其事,朝廷嘉之。才滿百日,奪情起復,德林以羸病屬疾,請急罷歸。



 魏收與陽休之論《齊書》起元事,敕集百司會
 議。收與德林書曰:「前者議文,總諸事意,小如混漫,難可領解。今便隨事條列,幸為留懷,細加推逐。凡言或者,皆是敵人之議。既聞人說,因而探論耳。」德林復書曰:「即位之元,《春秋》常義。謹按魯君息姑不稱即位,亦有元年,非獨即位得稱元年也。議云受終之元,《尚書》之古典。謹按《大傳》,周公攝政,一年救亂,二年伐殷,三年踐奄,四年建侯衛,五年營成周,六年制禮作樂,七年致政成王。論者或以舜、禹受終,是為天子。然則周公以臣禮而死,此亦稱元,非獨受終為帝也。蒙示議文,扶病省覽,荒情迷識,暫得發蒙。當世君子,必無橫議,唯應閣筆贊成而已。輒
 謂前二條有益於議,仰見議中不錄,謹以寫呈。」收重遺書曰:「惠示二事,感佩殊深。以魯公諸侯之事,昨小為疑。息姑不書即位,舜、禹亦不言即位。息姑雖攝,尚得書元,舜、禹之攝稱元,理也。周公居攝,乃云一年救亂,似不稱元。自無《大傳》,不得尋討。一之與元,其事何別?更有所見,幸請論之。」德林答曰:攝之與相,其義一也。故周公攝政,孔子曰「周公相成王」;魏武相漢,曹植曰「如虞翼唐」。或云高祖身未居攝,灼然非理。攝者專賞罰之名,古今事殊,不可以體為斷。陸機見舜肆類上帝,班瑞群後,便云舜有天下,須格於文祖也,欲使晉之三主異於舜攝。竊以
 為舜若堯死,獄訟不歸,便是夏朝之益,何得不須格於文祖也?若使用王者之禮,便曰即真,則周公負扆朝諸侯,霍光行周公之事,皆真帝乎?斯不然矣。必知高祖與舜攝不殊,不得從士衡之謬。



 或以為書元年者,當時實錄,非追書也。大齊之興,實由武帝,謙匿受命,豈直史也?比觀論者聞追舉受命之元,多有河漢,但言追數受命之歲,情或安之。似所怖者元字耳,事類朝三,是許其一年,不許其元年也。案《易》「黃裳元吉」,鄭玄注云:「如舜試天子,周公攝政。」是以試攝不殊。《大傳》雖無元字,一之與元,無異義矣。《春秋》不言一年一月者,欲使人君體元以居
 正,蓋史之婉辭,非一與元別也。漢獻帝死,劉備自尊崇。陳壽蜀人,以魏為漢賊。寧肯蜀主未立,已云魏武受命乎?士衡自尊本國,誠如高議,欲使三方鼎峙,同為霸名。習氏《漢晉春秋》,意在是也。至司馬炎兼並,許其帝號。魏之君臣,吳人並以為戮賊,亦寧肯當塗之世,云晉有受命之徵?史者,編年也,故魯號《紀年》。墨子又云,吾見《百國春秋》。史又有無事而書年者,是重年驗也。若欲高祖事事謙沖,即須號令皆推魏氏。便是編魏年,紀魏事,此即魏末功臣之傳,豈復皇朝帝紀者也。



 陸機稱紀元立斷,或以正始,或以嘉平。束皙議云,赤雀白魚之事。恐晉朝
 之議,是並論受命之元,非止代終之斷也。公議云陸機不議元者,是所未喻,願更思之。陸機以刊木著於《虞書》,龕黎見於商典,以蔽晉朝正始、嘉平之議,斯又謬矣。唯可二代相涉,兩史並書,必不得以後朝創業之跡,斷入前史。若然,則世宗、高祖皆天保以前,唯入魏氏列傳,不作齊朝帝紀,可乎?此既不可,彼復何證!



 是時中書侍郎杜臺卿上《世祖武成皇帝頌》,齊主以為未盡善,令和士開以頌示德林。宣旨云:「臺卿此文,未當朕意。以卿有大才,須敘盛德,即宜速作,急進本也。」德林乃上頌十六章並序,文多不載。武成覽頌善之,賜名馬一匹。三年,祖孝
 徵入為侍中,尚書左僕射趙彥深出為兗州刺史。朝士有先為孝徵所待遇者,間德林,云是彥深黨與,不可仍掌機密。孝徵曰:「德林久滯絳衣,我常恨彥深待賢未足。內省文翰,方以委之。尋當有佳處分,不宜妄說。」尋除中書侍郎,仍詔修國史。齊主留情文雅,召入文林館。又令與黃門侍郎顏之推二人同判文林館事。五年,敕令與黃門侍郎李孝貞、中書侍郎李若別掌宣傳。尋除通直散騎常侍,兼中書侍郎。隆化中,假儀同三司。承光中,授儀同三司。



 及周武帝克齊,入鄴之日,敕小司馬唐道和就宅宣旨慰喻,云:「平齊之利,唯在於爾。朕本畏爾逐齊
 王東走,今聞猶在,大以慰懷,宜即入相見。」道和引之入內,遣內史字文昂訪問齊朝風俗政教、人物善惡,即留內省,三宿乃歸。仍遣從駕至長安,授內史上士。自此以後,詔誥格式,及用山東人物,一以委之。武帝嘗於雲陽宮作鮮卑語謂群臣云:「我常日唯聞李德林名,及見其與齊朝作詔書移檄,我正謂其是天上人。豈言今日得其驅使,復為我作文書,極為大異。」神武公紇豆陵毅答曰:「臣聞明王聖主,得騏驎鳳凰為瑞,是聖德所感,非力能致之。瑞物雖來,不堪使用。如李德林來受驅策,亦陛下聖德感致,有大才用,無所不堪,勝於騏驎鳳凰遠矣。」
 武帝大笑曰:「誠如公言。」宣政末,授御正下大夫。大象初,賜爵成安縣男。



 宣帝大漸,屬高祖初受顧命,邗國公楊惠謂德林曰:「朝廷賜令總文武事,經國任重,非群才輔佐,無以克成大業。今欲與公共事,必不得辭。」德林聞之甚喜,乃答云:「德林雖庸芃,微誠亦有所在。若曲相提獎,必望以死奉公。」高祖大悅,即召與語。劉昉、鄭譯初矯詔召高祖受顧命輔少主,總知內外兵馬事。諸衛既奉敕,並受高祖節度。鄭譯、劉昉議,欲授高祖塚宰,鄭譯自攝大司馬,劉昉又求小塚宰。



 高祖私問德林曰:「欲何以見處?」德林云:「即宜作大丞相,假黃鉞,都督內外諸軍事。不
 爾,無以壓眾心。」及發喪,便即依此。以譯為相府長史,帶內史上大夫,昉但為丞相府司馬。譯、昉由是不平。以德林為丞相府屬,加儀同大將軍。未幾而三方構亂,指授兵略,皆與之參詳。軍書羽檄,朝夕填委,一日之中,動逾百數。或機速競發,口授數人,文意百端,不加治點。鄖公韋孝寬為東道元帥,師次永橋,為沁水泛長,兵未得度。長史李詢上密啟云:「大將梁士彥、宇文忻、崔弘度並受尉遲迥餉金,軍中慅慅,人情大異。」高祖得詢啟,深以為憂,與鄭譯議,欲代此三人。德林獨進計云:「公與諸將,並是國家貴臣,未相伏馭,今以挾令之威,使得之耳。安知
 後所遣者,能盡腹心,前所遣人,獨致乖異?又取金之事,虛實難明,即令換易,彼將懼罪,恐其逃逸,便須禁錮。然則鄖公以下,必有驚疑之意。且臨敵代將,自古所難,樂毅所以辭燕,趙括以之敗趙。如愚所見,但遣公一腹心,明於智略,為諸將舊來所信服者,速至軍所,使觀其情偽。縱有異志,必不敢動。」丞相大悟曰:「若公不發此言,幾敗大事。」即令高熲馳驛往軍所,為諸將節度,竟成大功。凡厥謀謨,多此類也。進授丞相府從事內郎。禪代之際,其相國總百揆、九錫殊禮詔策箋表璽書,皆德林之辭也。高祖登阼之日,授內史令。



 初,將受禪,虞慶則勸高祖
 盡滅宇文氏,高熲、楊惠亦依違從之。唯德林固爭,以為不可。高祖作色怒云:「君讀書人,不足平章此事。」於是遂盡誅之。自是品位不加,出於高、虞之下,唯依班例授上儀同,進爵為子。開皇元年,敕令與太尉任國公於翼、高熲等同修律令。事訖奏聞,別賜九環金帶一腰,駿馬一匹,賞損益之多也。格令班後,蘇威每欲改易事條。德林以為格式已頒,義須畫一,縱令小有踳駁,非過蠹政害民者,不可數有改張。威又奏置五百家鄉正,即令理民間辭訟。



 德林以為本廢鄉官判事,為其里閭親戚,剖斷不平,今令鄉正專治五百家,恐為害更甚。且今時吏部,
 總選人物,天下不過數百縣,於六七百萬戶內,詮簡數百縣令,猶不能稱其才,乃欲於一鄉之內,選一人能治五百家者,必恐難得。又即時要荒小縣,有不至五百家者,復不可令兩縣共管一鄉。敕令內外群官,就東宮會議。自皇太子以下,多從德林議。蘇威又言廢郡,德林語之云:「修令時,公何不論廢郡為便?今令才出,其可改乎!」然高熲同威之議,稱德林狠戾,多所固勢。由是高祖盡依威議。



 五年,敕令撰錄作相時文翰,勒成五卷,謂之《霸朝雜集》。序其事曰:竊以陽烏垂曜,微藿傾心,神龍騰舉,飛雲觸石。聖人在上,幽顯冥符,故稱比屋可封,萬物斯
 睹。臣皇基草創,便豫驅馳,遂得參可封之民,為萬物之一,其為嘉慶,固以多也。若夫帝臣王佐,應運挺生,接踵於朝,諒有之矣。而班爾之妙,曲木變容,硃藍所染,素絲改色。二十二臣,功成盡美;二十八將,效力於時。種德積善,豈皆比於稷、契,計功稱伐,非悉類于耿、賈。書契已還,立言立事,質非殆庶,何世無之。蓋上稟睿後,旁資群傑,牧商鄙賤,屠釣幽微,化為侯王,皆由此也。有教無類,童子羞於霸功;見德思齊,狂夫成於聖業。治世多士,亦因此焉。煙霧可依,騰蛇與蛟龍俱遠;棲息有所,蒼蠅同騏驥之速。因人成事,其功不難。自此而談,雖非上智,事受
 命之主,委質為臣,遇高世之才,連官接席,皆可以翊亮天地,流名鐘鼎,何必蒼頡造書,伊尹制命,公旦操筆,老聃為史,方可敘帝王之事,談人鬼之謀乎?至若臣者,本慚賓實,非勛非德,廁軒冕之流,無學無才,處藝文之職。若不逢休運,非遇天恩,光大含弘,博約文禮,萬官百闢,才悉兼人,收拙裏閭,退仕鄉邑,不種東陵之瓜,豈過南陽之掾,安得出入閶闔之閫,趨走太微之庭,履天子之階,侍聖皇之側,樞機帷幄,沾及榮寵者也!昔歲木行將季,諒闇在辰,火運肇興,群官總己。有周典八柄之所,大隋納百揆之日,兩朝文翰,臣兼掌之。時溥天之下,三方
 構亂,軍國多務,朝夕填委。簿領紛紜,羽書交錯,或速均發弩,或事大滔天,或日有萬幾,或幾有萬事。皇帝內明外順,經營區宇,吐無窮之術,運不測之神,幽贊兩儀,財成萬類。咨謀臺閣,曉喻公卿,訓率土之濱,責反常之賊。三軍奉律,戰勝攻取之方;萬國承風,安上治民之道。讓受終之禮,報群臣之令,有憲章古昔者矣,有隨事作故者矣。千變萬化,譬彼懸河;寸陰尺日,不棄光景。大則天壤不遺,小則毫毛無失。遠尋三古,未聞者盡聞;逖聽百王,未見者皆見。發言吐論,即成文章,臣染翰操牘,書記而已。昔放勛之化,老人睹而未知;孔丘之言,弟子聞而
 不達。愚情稟聖,多必乖舛。加以奏閣趨墀,盈懷滿袖,手披目閱,堆案積幾。心無別慮,筆不暫停,或畢景忘餐,或連宵不寐,以勤補拙,不遑自處。其有詞理疏謬,遺漏闕疑,皆天旨訓誘,神筆改定。運籌建策,通幽達冥,從命者獲安,違命者悉禍。懸測萬里,指期來事,常如目見,固乃神知。變大亂而致大平,易可誅而為淳粹,化成道洽,其在人文,盡出聖懷,用成典誥,並非臣意所能至此。伯禹矢謨,成湯陳誓,漢光數行之札,魏武《接要》之書,濟時拯物,無以加也。屬神器大寶,將遷明德,天道人心,同謨歸往。周靜南面,每詔褒揚,在位諸公,各陳本志,璽書表奏,
 群情賜委。臣寰海之內,忝曰一民,樂推之心,切於黎獻,欣然從命,輒不敢辭。比夫潘勖之冊魏王,阮籍之勸晉後,道高前世,才謝往人,內手捫心,夙宵慚惕。檄書露板,及以諸文,有臣所作之,有臣潤色之。唯是愚思,非奏定者,雖詞乖黼藻,而理歸霸德,文有可忽,事不可遺。前奉敕旨,集納麓已還,至於受命文筆,當時制述,條目甚多,今日收撰,略為五卷云爾。



 高祖省讀訖,明旦謂德林曰:「自古帝王之興,必有異人輔佐。我昨讀《霸朝集》,方知感應之理。昨宵恨夜長,不能早見公面。必令公貴與國始終。」於是追贈其父恆州刺史。未幾,上曰:「我本意欲深榮
 之。」復贈定州刺史、安平縣公,謚曰孝,以德林襲焉。德林既少有才名,重以貴顯,凡制文章,動行於世。或有不知者,謂為古人焉。



 德林以梁士彥及元諧之徒頻有逆意,大江之南,抗衡上國。乃著《天命論》上之,其辭曰:粵若邃古,玄黃肇闢,帝王神器,歷數有歸。生其德者天,應其時者命,確乎不變,非人力所能為也。龍圖鳥篆,號謚遺跡,疑而難信,缺而未詳者,靡得而明焉。其在典文,煥乎緗素,欽明至德,莫盛於唐、虞,貽謀長世,莫過於文、武。



 大隋神功積於文王,天命顯於唐叔。昔邑姜方娠,夢帝謂己:「餘命而子曰虞,將與之唐,而蕃育其子孫。」及生,有文在
 其手曰「虞」,遂以命之。成王滅唐而封太叔。又唐叔之封也,箕子曰:「其後必大。」《易》曰:「崇高富貴,莫大於帝王。」《老子》謂:「域內四大,王居一焉。」此則名虞與唐,美兼二聖,將令其後必大,終致唐、虞之美,蕃育子孫,用享無窮之祚。



 逮皇家建國,初號大興,箕子必大之言,於茲乃驗。天之眷命,懸屬聖朝,重耳區區,豈足雲也!有娀玄鳥,商以興焉;姜嫄巨跡,周以興焉;邑姜夢帝,隋以興焉。古今三代,靈命如一,本枝種德,奕葉丕基。佐高帝而滅楚,立宣皇以定漢。



 東京太尉,關西孔子,生感遺鱣之集,歿降巨鳥之奇,累仁積善,大申休命。太祖挺生,庇民匡主,立殊勛於
 魏室,建盛業於周朝。啟翼軫之國,肇炎精之紀,爰受厥命,陟配彼天。皇帝載誕之初,神光滿室,具興王之表,韞大聖之能。或氣或云,廕映於廊廟;如天如日,臨照於軒冕。內明外順,自險獲安,豈非萬福扶持,百祿攸集。有周之末,朝野騷然,降志執均,鎮衛宗社。明神饗其德,上帝付其民,誅奸逆於九重,行神化於四海。於斯時也,尉迥據有齊累世之都,乘新國易亂之俗,驅馳蛇豕,連合縱橫,地乃九州陷三,民則十分擁六。王謙乘連率之威,憑全蜀之險,興兵舉眾,震蕩江山,鴆毒巴、庸,蠶食秦、楚。此二虜也,窮兇極逆,非欲割鴻溝之地,閉劍閣之門,皆將
 長戟強弩,睥睨宸極。從漳河而達負海,連岱嶽而距華陽,迫脅荊蠻,吐納江漢。佐鬥嫁禍,紛若蝟毛;曝骨履腸,間不容礪。爾乃奉殪戎之命,運先天之略,不出戶庭,推轂分閫,一麾以定三方,數旬而清萬國。



 蕩滌天壤之速,規摹指畫之神,造化以來,弗之聞也。光熙前緒,罔有不服,煙雲改色,鐘石變音,三靈顧望,萬物影響。木運告盡,褰裳克讓,天歷在躬,推而弗有。百闢庶尹,四方岳牧,稽圖讖之文,順億兆之請,披肝瀝膽,晝歌夜吟,方屈箕潁之高,式允幽明之願。基命宥密,如恆如升,推帝居歆,創業垂統。殊徽號,改服色,建都邑,敘彞倫,薄賦輕徭,慎刑
 恤獄,除繁苛之政,興清靜之風,去無用之官,省相監之職。奇才間出,盛德無隱,星精雲氣,共趨走於階墀,山神海靈,咸燮理於臺閣。東漸日穀,西被月川,教暨北溟之表,聲加南海之外。悠悠沙漠,區域萬里,蠢蠢百蠻,莫之與競。五帝所不化,三王所未賓,屈膝頓顙,盡為臣妾。



 殊方異類,書契不傳,梯山越海,貢琛奉贄,欣欣如也。巢居穴處,化以宮室;不火不粒,訓以庖廚。禮樂合天地之同,律呂節寒暑之候,制作詳垂衣之後,淳粹得神農之前。遨游文雅之場,出入杳冥之極,合神謨鬼,通幽洞微。群物歲成,含生日用,飲和氣以自得,沐玄澤而不知也。丹
 雀為使,玄龜載書,甘露自天,醴泉出地。神禽異獸,珍木奇草,望風觀海,應化歸風。備休祥於圖牒,罄幽遐而戾止。



 猶且父天子民,兢兢翼翼,至矣大矣,七十四帝,曷可同年而語哉!



 若夫天下之重,不可妄據,故唐之許由,夏之伯益,懷道立事,人授而弗可也。



 軒初四帝,周餘六王,藉世因基,自取而不得也。孟軻稱仲尼之德過於堯、舜,著述成帝者之事,弟子備王佐之才,黑不代蒼,泣麟嘆鳳,棲棲汲汲,雖聖達而莫許也。蚩尤則黃帝抗衡,共工則黑帝勍敵,項羽誅秦摧漢,宰割神州,角逐爭驅,盡威力而無就也。其餘欻起妖妄,曾何足數!賊子逆臣,所以
 為亂,皆由不識天道,不悟人謀,牽逐鹿之邪說,謂飛鳧而為鼎。若使四兇爭八元之誠,三監同九臣之志,韓信、彭越深明帝子之符,孫述、隗囂妙識真人之出,尉迥同謳歌之類,王謙比獄訟之民,福祿蟬聯,胡可窮也!而違天逆物,獲罪人神。嗚呼!此前事之大戒矣。



 誅夷烹醢,歷代共尤,僭逆兇邪,時煩獄吏,其可不戒慎哉!蓋積惡既成,心自絕於善道,物類相感,理必至於誅戮。天奪其魄,鬼惡其盈故也。大帝聰明,群臣正直,耳目濫於率土,賞罰參於國朝,輔助一人,覆育兆庶。豈有食人之祿,受人之榮,包藏禍心而不殲盡者也?必當執法未處其罪,司
 命已除其籍。自古明哲,慮遠防微,執一心,持一德,立功坐樹,上書削槁,位尊而心逾下,祿厚而志彌約,寵盛思之以懼,道高守之以恭,克念於此,則奸回不至。事乃畏天,豈惟愛禮,謙光滿覆,義在知幾,吉兇由人,妖不自作。



 眾星共極,在天成象。夙沙則主雖愚蔽,民盡知歸;有苗則始為跋扈,終而大服。漢南諸國,見一面以從殷;河西將軍,率五郡以歸漢。故能招信順之助,保太山之安。彼陳國者,盜竊江外,民少一郡,地減半州,遇受命之主,逢太平之日,自可獻土銜璧,乞同溥天。乃復養喪家之疹,遵顛覆之軌,趑趄吳越,仍為匪民。



 雖時屬大道,偃兵舞
 戚,然國家當混一之運,金陵是殄滅之期,有命不恆,斷可知矣。房風之戮,元龜匪遙;孫皓之侯,守株難得。迷而未覺,諒可愍焉。斯故未辯玄天之心,不聞君子之論也。



 德林自隋有天下,每贊平陳之計。八年,車駕幸同州,德林以疾不從。敕書追之,書後御筆注云:「伐陳事意,宜自隨也。」時高熲因使入京,上語熲曰:「德林若患未堪行,宜自至宅,取其方略。」高祖以之付晉王廣。後從駕還,在途中,高祖以馬鞭南指云:「待平陳訖,會以七寶裝嚴公,使自山東無及之者。」及陳平,授柱國、郡公,實封八百戶,賞物三千段。晉王廣已宣敕訖,有人說高熲曰:「天子畫策,
 晉王及諸將戮力之所致也。今乃歸功於李德林,諸將必當憤惋,且後世觀公有若虛行。」熲入言之,高祖乃止。



 初,大象末,高祖以逆人王謙宅賜之,文書已出,至地官府,忽復改賜崔謙。



 上語德林曰:「夫人欲得,將與其舅。於公無形跡,不須爭之,可自選一好宅。若不稱意,當為營造,並覓莊店作替。」德林乃奏取逆人高阿那肱衛國縣市店八十塸為王謙宅替。九年,車駕幸晉陽,店人上表訴稱:「地是民物,高氏強奪,於內造舍。」上命有司料還價直。遇追蘇威自長安至,奏云:「高阿那肱是亂世宰相,以諂媚得幸,枉取民地,造店賃之。德林誣誷,妄奏自入。」李
 圓通、馮世基等又進云:「此店收利如食千戶,請計日追贓。」上因責德林,德林請勘逆人文簿及本換宅之意,上不聽,乃悉追店給所住者。自是益嫌之。十年,虞慶則等於關東諸道巡省使還,並奏云:「五百家鄉正,專理辭訟,不便於民。黨與愛憎,公行貨賄。」



 上仍令廢之。德林復奏云:「此事臣本以為不可。然置來始爾,復即停廢,政令不一,朝成暮毀,深非帝王設法之義。臣望陛下若於律令輒欲改張,即以軍法從事。



 不然者,紛紜未已。」高祖遂發怒,大詬云:「爾欲將我作王莽邪?」初,德林稱父為太尉諮議,以取贈官,李元操與陳茂等陰奏之曰:「德林之父終
 於校書,妄稱諮議。」上甚銜之。至是,復庭議忤意,因數之曰:「公為內史,典朕機密,比不可豫計議者,以公不弘耳。寧自知乎?朕方以孝治天下,恐斯道廢闕,故立五教以弘之。公言孝由天性,何須設教。然則孔子不當說《孝經》也。又誷冒取店,妄加父官,朕實忿之而未能發。今當以一州相遣耳。」因出為湖州刺史。德林拜謝曰:「臣不敢復望內史令,請預散參。待陛下登封告成,一觀盛禮,然後收拙丘園,死且不恨。」上不許,轉懷州刺史。在州逢亢旱,課民掘井溉田,空致勞擾,竟無補益,為考司所貶。歲餘,卒官,時年六十一。贈大將軍、廉州刺史,謚曰文。及將葬,
 敕令羽林百人,並鼓吹一部,以給喪事。贈物三百段,粟千石,祭以太牢。



 德林美容儀,善談吐,齊天統中,兼中書侍郎,於賓館受國書。陳使江總目送之曰:「此即河朔之英靈也。」器量沉深,時人未能測,唯任城王湝、趙彥深、魏收、陸遝大相欽重,延譽之言,無所不及。德林少孤,未有字,魏收謂之曰:「識度天才,必至公輔,吾輒以此字卿。」從官以後,即典機密,性重慎,嘗云古人不言溫樹,何足稱也。少以才學見知,及位望稍高,頗傷自任,爭名之徒,更相譖毀,所以運屬興王,功參佐命,十餘年間竟不徙級。所撰文集,勒成八十卷,遭亂亡失,見五十卷行於世。敕
 撰《齊史》未成。



 有子曰百藥,博涉多才,詞藻清贍。釋巾太子通事舍人,後遷太子舍人、尚書禮部員外郎,襲爵安平縣公,桂州司馬。煬帝惡其初不附己,以為步兵校尉。大業末,轉建安郡丞。



 史臣曰:德林幼有操尚,學富才優,譽重鄴中,聲飛關右。王基締構,協贊謀猷,羽檄交馳,絲綸間發,文誥之美,時無與二。君臣體合,自致青雲,不患莫己知,豈徒言也!



\end{pinyinscope}