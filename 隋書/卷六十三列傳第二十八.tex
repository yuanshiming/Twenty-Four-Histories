\article{卷六十三列傳第二十八}

\begin{pinyinscope}

 樊子蓋樊子蓋,字華宗,廬江人也。祖道則,梁越州刺史。父儒,侯景之亂奔於齊,官至仁州刺史。子蓋解褐武興王行參軍,出為慎縣令,東汝、北陳二郡太守,員外散騎常侍,封富陽縣侯,邑五百戶。周武帝平齊,授儀同三司,治郢州刺史。高祖受禪,以儀同領鄉兵,後除樅陽太守。平陳之役,以功加上開府,改封上蔡縣伯,食邑七百戶,賜物三
 千段,粟九千斛。拜辰州刺史,俄轉嵩州刺史。母憂去職。未幾,起授齊州刺史,固讓,不許。其年,轉循州總管,許以便宜從事。十八年入朝,奏嶺南地圖,賜以良馬雜物,加統四州,令還任所,遣光祿少卿柳謇之餞於霸上。



 煬帝即位,徵還京師,轉涼州刺史。子蓋言於帝曰:「臣一居嶺表,十載於茲,犬馬之情,不勝戀戀。願趨走闕庭,萬死無恨。」帝賜物三百段,慰諭遣之,授銀青光祿大夫、武威太守,以善政聞。大業三年入朝,帝引之內殿,特蒙褒美。乃下詔曰:「設官之道,必在用賢,安人之術,莫如善政。龔、汲振德化於前,張、杜垂清風於後,共治天下,實資良守。子
 蓋乾局通敏,操履清潔,自剖符西服,愛惠為先,撫道有方,寬猛得所。處脂膏不潤其質,酌貪泉豈渝其性,故能治績克彰,課最之首。凡厥在位,莫匪王臣,若能人思奉職,各展其效,朕將冕旒垂拱,何憂不治哉!」於是進位金紫光祿大夫,賜物千段,太守如故。五年,車駕西巡,將入吐谷渾。子蓋以彼多鄣氣,獻青木香以御霧露。及帝還,謂之曰:「人道公清,定如此不?」子蓋謝曰:「臣安敢言清,止是小心不敢納賄耳。」由此賜之口味百餘斛,又下詔曰:「導德齊禮,實惟共治,懲惡勸善,用明黜陟。朕親巡河右,觀省人風,所歷郡縣,訪採治績,罕遵法度,多蹈刑網。而
 金紫光祿大夫、武威太守樊子蓋,執操清潔,處涅不渝,立身雅正,臨人以簡。威惠兼舉,寬猛相資,故能畏而愛之,不嚴斯治。實字人之盛績,有國之良臣,宜加褒顯,以弘獎勵。可右光祿大夫,太守如故。」賜縑千匹,粟麥二千斛。子蓋又自陳曰:「臣自南裔,即適西垂,常為外臣,未居內職。不得陪屬車,奉丹陛,溘死邊城,沒有遺恨。惟陛下察之。」帝曰:「公侍朕則一人而已,委以西方則萬人之敵,宜識此心。」六年,帝避暑隴川宮,又云欲幸河西。子蓋傾望鑾輿,願巡郡境,帝知之,下詔曰:「卿夙懷恭順,深執誠心,聞朕西巡,欣然望幸。丹款之至,甚有可嘉。宜保此純
 誠,克終其美。」是歲,朝於江都宮,帝謂之曰:「富貴不還故鄉,真衣繡夜行耳。」敕廬江郡設三千人會,賜米麥六千石,使謁墳墓,宴故老。當時榮之。還除民部尚書。



 時處羅可汗及高昌王款塞,復以子蓋檢校武威太守,應接二蕃。



 遼東之役,徵攝左武衛將軍,出長岑道。後以宿衛不行。進授左光祿大夫,尚書如故。其年帝還東都,以子蓋為涿郡留守。九年,車駕復幸遼東,命子蓋為東都留守。屬楊玄感作逆,來逼王城,子蓋遣河南贊治裴弘策逆擊之,返為所敗,遂斬弘策以徇。國子祭酒楊汪小有不恭,子蓋又將斬之。汪拜謝,頓首流血,久乃釋免。



 於是三
 軍莫不戰慄,將吏無敢仰視。玄感每盡銳攻城,子蓋徐設備御,至輒摧破,故久不能克。會來護兒等救至,玄感解去。子蓋凡所誅殺者數萬人。



 又檢校河南內史。車駕至高陽,追詣行在所。既而引見,帝逆勞之曰:「昔高祖留蕭何於關西,光武委寇恂以河內,公其人也。」子蓋謝曰:「臣任重器小,寧可竊譬兩賢!但以陛下威靈,小盜不足除耳。」進位光祿大夫,封建安侯,尚書如故。賜縑三千匹,女樂五十人。子蓋固讓,優詔不許。帝顧謂子蓋曰:「朕遣越王留守東都,示以皇枝盤石;社稷大事,終以委公。特宜持重,戈甲五百人而後出,此亦勇夫重閉之義也。無
 賴不軌者,便誅鋤之。凡可施行,無勞形跡。今為公別造玉麟符,以代銅獸。」又指越、代二王曰:「今以二孫委公與衛文升耳。宜選貞良宿德有方幅者教習之。動靜之節,宜思其可。」於是賜以良田、甲第。十年冬,車駕還東都,帝謂子蓋曰:「玄感之反,神明故以彰公赤心耳。析珪進爵,宜有令謨。」



 是日下詔,進爵為濟公,言其功濟天下,特為立名,無此郡國也。賜縑三千匹,奴婢二十口。後與蘇威、宇文述陪宴積翠亭,帝親以金杯屬子蓋酒,曰:「良算嘉謀,俟公後動,即以此杯賜公,用為永年之瑞。」並綺羅百匹。



 十一年,從駕汾陽宮。至於雁門,車駕為突厥所圍,頻
 戰不利。帝欲以精騎潰圍而出,子蓋諫曰:「陛下萬乘之主,豈宜輕脫,一朝狼狽,雖悔不追。未若守城以挫其銳,四面徵兵,可立而待。陛下亦何所慮,乃欲身自突圍!」因垂泣,「願暫停遼東之役,以慰眾望。聖躬親出慰撫,厚為勛格,人心自奮,不足為憂。」帝從之。其後援兵稍至,虜乃引去。納言蘇威追論勛格太重,宜在斟酌。子蓋執奏不宜失信。帝曰:「公欲收物情邪?」子蓋默然不敢對。從駕還東都。時絳郡賊敬槃陀、柴保昌等阻兵數萬,汾、晉苦之。詔令子蓋進討。於時人物殷阜,子蓋善惡無所分別,汾水之北,村塢盡焚之。百姓大駭,相率為盜。其有歸首者,
 無少長悉坑之。擁數萬之眾,經年不能破賊,有詔徵還。又將兵擊宜陽賊,以疾停,卒於京第,時年七十有二。上悲傷者久之,顧謂黃門侍郎裴矩曰:「子蓋臨終有何語?」矩對曰:「子蓋病篤,深恨雁門之恥。」帝聞而嘆息,令百官就吊,賜縑三百匹,米五百斛,贈開府儀同三司,謚曰景。會葬者萬餘人。武威民吏聞其死,莫不嗟痛,立碑頌德。



 子蓋無他權略,在軍持重,未嘗負敗,臨民明察,下莫敢欺。然嚴酷少恩,果於殺戮,臨終之日,見斷頭鬼前後重沓為之厲云。



 史祥
 史祥,字世休,朔方人也。父寧,周少司徒。祥少有文武才幹,仕周太子車右中士,襲爵武遂縣公。高祖踐阼,拜儀同,領交州事,進爵陽城郡公。祥在州頗有惠政。後數年,轉驃騎將軍。伐陳之役,從宜陽公王世積,以舟師出九江道,先鋒與陳人合戰,破之,進拔江州。上聞而大悅,下詔曰:「朕以陳叔寶世為僭逆,挻虐生民,故命諸軍,救彼塗炭。小寇狼狽,顧恃江湖之險,遂敢泛舟楫擬抗王師。



 公親率所部,應機奮擊,沉溺俘獲,厥功甚茂。又聞帥旅進取江州。行軍總管、襄邑公賀若弼既獲京口,新義公韓擒尋克姑熟。驃騎既渡江岸,所在橫行。晉王兵馬即
 入建業,清蕩吳、越,旦夕非遠。驃騎高才壯志,是朕所知,善為經略,以取大賞,使富貴功名永垂竹帛也。」進位上開府。尋拜蘄州總管,未幾,徵拜左領左右將軍。後以行軍總管從晉王廣擊突厥於靈武,破之。遷右衛將軍。



 仁壽中,率兵屯弘化以備胡。煬帝時在東宮,遺祥書曰:將軍總戎塞表,胡虜清塵,秣馬休兵,猶事校獵,足使李廣慚勇,魏尚愧能,冠彼二賢,獨在吾子。昔餘濫舉,推轂治兵,振皇靈於塞外,驅犬羊乎大漠。於時同行軍旅,契闊戎旃,望龍城而沖冠,眄狼居而發憤。將軍英圖不世,猛氣無前,但物不遂心,俛從事。每一思此,我勞如何。將
 軍宿心素志,早同膠漆,久而敬之,方成魚水。近者陪隨鑾駕,言旋上京,本即述職南蕃,宣條下國,不悟皇鑒曲發,備位少陽,戰戰兢兢,如臨冰谷。至如建節邊境,征伐四方,褰帷作牧,綏撫百姓,上稟成規,下盡臣節,是所願也,是所甘心。仰慕前修,庶得自效。謬其入守神器,元良萬國,身輕負重,何以克堪!所望故人,匡其不逮。比監國多暇,養疾閑宮,厭北閣之端居,罷南皮之馳射。博望之苑,既乏名賢,飛蓋之園,理乖終宴。親朋遠矣,琴書寂然,想望吾賢,疹如疾首。



 祥答書曰:行人戾止,奉所賜況,恩紀綢繆,形於文墨。不悟飛雪增冰之地,忽載三陽,毳幙
 韋韝之鄉,俄聞九奏。精駭思越,莫知啟處。祥少不學軍旅,長遇升平,幸以先人緒餘,備職宿衛。懼駑蹇無致遠之用,朽薄非折沖之材,豈欲追蹤古人,語其優劣?曩者王師薄伐,天人受脤,絕漠揚旌,威震海外。當此之時,猛將如云,謀夫如雨。至若祥者,列於卒伍,預聞指蹤之規,得免逗遛之責,循涯揣分,實為幸甚。爰以情喻雷、陳,事方劉、葛,信聖人之屈己,非庸人之擬議。何則?川澤之大,污潦攸歸,松柏之高,蔦蘿斯托。微心眷眷,孟侯所知也。抑惟體元良之德,煥重離之暉,三善克修,萬邦以正。斯固道高周誦,契葉商皓,豈在管蠡所能窺測!



 伏承監國
 多暇,養德怡神,咀嚼六經,逍遙百氏。追西園之愛客,眷南皮之出游,疇昔之恩,無忘造次。祥自忝式遏,載罹寒暑,身在邊隅,情馳魏闕。每至清風夕起,朗月孤照,想鳴葭之啟路,思托乘於後車。塞表京華,山川悠遠,瞻望浮雲,伏增潸結。



 太子甚親遇之。



 煬帝即位,漢王諒發兵作亂,遣其將綦良自滏口徇黎陽,塞白馬津,餘公理自太行下河內。帝以祥為行軍總管,軍於河陰,久不得濟。祥謂軍吏曰:「余公理輕而無謀,才用素不足稱,又新得志,謂其眾可恃。恃眾必驕。且河北人先不習兵,所謂擁市人而戰。以吾籌之,不足圖也。」乃令軍中修攻具,公理使
 諜知之,果屯兵於河陽內城以備祥。祥於是艤船南岸,公理聚甲以當之。祥乃簡精銳於下流潛渡,公理率眾拒之。祥至須水,兩軍相對,公理未成列,祥縱擊,大破之。東趣黎陽討綦良等。良列陣以待,兵未接,良棄軍而走。於是其眾大潰,祥縱兵乘之,殺萬餘人。進位上大將軍,賜縑彩七千段,女妓十人,良馬二十匹。轉太僕卿。帝嘗賜祥詩曰:「伯煚朝寄重,夏侯親遇深。貴耳唯聞古,賤目詎知今,早厓勁草質,久有背淮心。掃逆黎山外,振旅河之陰。功已書王府,留情《太僕箴》。」祥上表辭謝,帝降手詔曰:「昔歲勞公問罪河朔,賊爾日塞兩關之路,據倉阻河,百姓脅從,人亦眾矣。公竭誠奮勇,一舉克定。《詩》不
 云乎:『喪亂既平,既安且寧。』非英才大略,其孰能與於此邪!故聊示所懷,亦何謝也。」



 尋遷鴻臚卿。時突厥啟民可汗請朝,帝遣祥迎接之。從征吐谷渾,祥率眾出間道擊虜,破之,俘男女千餘口。賜奴婢六十人,馬三百匹。進位左光祿大夫,拜左驍衛將軍。及遼東之役,出蹋頓道,不利而還。由是除名為民。俄拜燕郡太守,被賊高開道所圍,祥稱疾不視事。及城陷,開道甚禮之。會開道與羅藝通和,送祥於涿郡,卒於途。



 有子義隆,永年令。祥兄云,字世高,弟威,字世儀,並有幹局。雲官至萊州刺史、武平縣公,威官至武賁郎將、武當縣公。



 元壽元壽,字長壽,河南洛陽人也。祖敦,魏侍中、邵陵王。父寶,周涼州刺史。



 壽少孤,性仁孝,九歲喪父,哀毀骨立,宗族鄉黨咸異之。事母以孝聞。及長,方直,頗涉經史。周武成初,封隆城縣侯,邑千戶,保定四年,改封儀隴縣侯,授儀同三司。開皇初,議伐陳,以壽有思理,奉使於淮浦監修船艦,以強濟見稱。四年,參督漕渠之役,授尚書主爵侍郎。八年,從晉王伐陳,除行臺左丞,兼領元帥府屬。



 及平陳,拜尚書左丞。高祖嘗出苑觀射,文武並從焉。開府蕭摩訶妻患且死,奏請遣子向江南收其家產,御史見而
 不言。壽奏劾之曰:臣聞天道不言,功成四序,聖皇垂拱,任在百司。御史之官,義存糾察,直繩莫舉,憲典誰寄?今月五日,鑾輿徙蹕,親臨射苑,開府儀同三司蕭摩訶幸廁朝行,預觀盛禮,奏稱請遣子世略暫往江南重收家產。妻安遇患,彌留有日,安若長逝,世略不合此行。竊以人倫之義,伉儷為重,資愛之道,烏鳥弗虧。摩訶遠念資財,近忘匹好,又命其子舍危惙之母,為聚斂之行。一言才發,名教頓盡。而兼殿內侍御史臣韓微之等,親所聞見,竟不彈糾。若知非不舉,事涉阿縱;如不以為非,豈關理識?謹按儀同三司、太子左庶子、檢校治書侍御史臣
 劉行本,出入宮省,備蒙任遇,攝職憲臺,時月稍久,庶能整肅纓冕,澄清風教。而在法司,虧失憲體,瓶罄罍恥,何所逃愆!臣謬膺朝寄,忝居左轄,無容寢默,謹以狀聞。其行本、微之等,請付大理。



 上嘉納之。尋授太常少卿。數年,拜基州刺史,在任有公廉之稱。入為太府少卿。進位開府。煬帝嗣位,漢王諒舉兵反,左僕射楊素為行軍元帥,壽為長史。壽每遇賊,為士卒先,以功授大將軍,遷太府卿。四年,拜內史令,從帝西討吐谷渾。



 壽率眾屯金山,東西連營三百餘里,以圍渾主。及還,拜右光祿大夫。七年,兼左翊衛將軍,從征遼東,行至涿郡,遇疾卒,時年六十
 三。帝悼惜焉,哭之甚慟。贈尚書右僕射、光祿大夫,謚曰景。



 子敏,頗有才辯,而輕險多詐。壽卒後,帝追思之,擢敏為守內史舍人,而交通博徒,數漏洩省中語。化及之反也,敏創其謀,偽授內史侍郎,為沈光所殺。



 楊義臣楊義臣,代人也,本姓尉遲氏。父崇,仕周為儀同大將軍,以兵鎮恆山。時高祖為定州總管,崇知高祖相貌非常,每自結納,高祖甚親待之。及為丞相,尉迥作亂,崇以宗族之故,自囚於獄,遣使請罪。高祖下書慰諭之,即令馳驛入朝,恆置左右。開皇初,封秦興縣公。歲餘,從行軍總管達奚長儒擊突厥於周盤,力戰而死。



 贈大將軍、
 豫州刺史,以義臣襲崇官爵。時義臣尚幼,養於宮中,年未弱冠,奉詔宿衛如千牛者數年,賞賜甚厚。上嘗從容言及恩舊,顧義臣嗟嘆久之,因下詔曰:「朕受命之初,群兇未定,明識之士,有足可懷。尉義臣與尉迥,本同骨肉,既狂悖作亂鄴城,其父崇時在常山,典司兵甲,與迥鄰接,又是至親,知逆順之理,識天人之意,即陳丹款,慮染惡徒,自執有司,請歸相府。及北夷內侵,橫戈制敵,輕生重義,馬革言旋。操表存亡,事貫幽顯,雖高官大賞,延及於世,未足表松筠之志,彰節義之門。義臣可賜姓楊氏,賜錢三萬貫,酒三十斛,米麥各百斛,編之屬籍,為皇從
 孫。」未幾,拜陜州刺史。義臣性謹厚,能馳射,有將領之才,由是上甚重之。其後突厥達頭可汗犯塞,以行軍總管率步騎三萬出白道,與賊遇,戰,大破之。明年,突厥又寇邊,雁門、馬邑多被其患。義臣擊之,虜遂出塞,因而追之,至大斤山,與虜相遇。時太平公史萬歲軍亦至,義臣與萬歲合軍擊虜,大破之,萬歲為楊素所陷而死,義臣功竟不錄。仁壽初,拜朔州總管,賜以御甲。



 煬帝嗣位,漢王諒作亂並州。時代州總管李景為漢王將喬鐘葵所圍,詔義臣救之。義臣率馬步二萬,夜出西陘,遲明行數十里。鐘葵覘見義臣兵少,悉眾拒之。



 鐘葵亞將王拔驍勇,
 善用矛,射之者不能中,每以數騎陷陣。義臣患之,募能當拔者。車騎將軍楊思恩請當之。義臣見思恩氣貌雄勇,顧之曰:「壯士也!」賜以卮酒。思恩望見拔立於陣後,投觴於地,策馬赴之。再往不克,義臣復選騎士十餘人從之。思恩遂突擊,殺數人,直至拔麾下。短兵方接,所從騎士退,思恩為拔所殺。



 拔遂乘之,義臣軍北者十餘里。於是購得思恩尸,義臣哭之甚慟,三軍莫不下泣。



 所從騎士皆腰斬。義臣自以兵少,悉取軍中牛驢,得數千頭,復令兵數百人,人持一鼓,潛驅之澗谷間,出其不意。義臣晡後復與鐘葵軍戰,兵初合,命驅牛驢者疾進。一時鳴
 鼓,塵埃張天,鐘葵軍不知,以為伏兵發,因而大潰,縱擊破之。以功進位上大將軍,賜物二千段,雜彩五百段,女妓十人,良馬二十匹。尋授相州刺史。



 後三歲,徵為宗正卿。未幾,轉太僕卿。從征吐谷渾,令義臣屯琵琶峽,連營八十里,南接元壽,北連段文振,合圍渾主於覆袁川。其後復征遼東,以軍將指肅慎道。



 至鴨綠水,與乙支文德戰,每為先鋒,一日七捷。後與諸軍俱敗,竟坐免。俄而復位。明年,以為軍副,與大將軍宇文述趣平壤。至鴨綠水,會楊玄感作亂,班師,檢校趙郡太守。妖賊向海公聚眾作亂,寇扶風、安定間,義臣奉詔擊平之。尋從帝復征遼
 東,進位左光祿大夫。時渤海高士達,清河張金稱並相聚為盜,眾已數萬,攻陷郡縣。帝遣將軍段達討之,不能克。詔義臣率遼東還兵數萬擊之,大破士達,斬金稱。又收合降賊,入豆子,討格謙,擒之,以狀聞奏。帝惡其威名,遽追入朝,賊由是復盛。義臣以功進位光祿大夫,尋拜禮部尚書。未幾,卒官。



 衛玄衛玄,字文升,河南洛陽人也。祖悅,魏司農卿,父手剽,侍中、左武衛大將軍,玄少有器識,周武帝在籓,引為記室。遷給事上士,襲爵興勢公,食邑四千戶。



 轉宣納下大夫。武
 帝親總萬機,拜益州總管長史,賜以萬釘寶帶。稍遷開府儀同三司、太府中大夫,治內史事,仍領京兆尹,稱為強濟。宣帝時,以忤旨免官。高祖作相,檢校熊州事。和州蠻反,玄以行軍總管擊平之。及高祖受禪,遷淮州總管,進封同軌郡公,坐事免。未幾,拜嵐州刺史。會起長城之役,詔玄監督之。俄檢校朔州總管事。後為衛尉少卿。仁壽初,山獠作亂,出為資州刺史以鎮撫之。玄既到官,時獠攻圍大牢鎮,玄單騎造其營,謂群獠曰:「我是刺史,銜天子詔安養汝等,勿驚懼也。」諸賊莫敢動。於是說以利害,渠帥感悅,解兵而去。前後歸附者十餘萬口。高祖大
 悅,賜縑二千匹,除遂州總管,仍令劍南安撫。煬帝即位,復徵為衛尉卿。夷、獠攀戀,數百里不絕。玄曉之曰:「天子詔征,不可久住。」因與之訣,夷、獠各揮涕而去。歲餘,遷工部尚書。其後拜魏郡太守,尚書如故。帝謂玄曰:「魏郡名都,沖要之所,民多奸宄,是用煩公。此郡去都,道里非遠,宜數往來,詢謀朝政。」賜物五百段而遣之。未幾,拜右候衛大將軍,檢校左候衛事。大業八年,轉刑部尚書。遼東之役,檢校右御衛大將軍,率師出增地道。時諸軍多不利,玄獨全眾而還。拜金紫光祿大夫。九年,車駕幸遼東,使玄與代王侑留守京師,拜為京兆內史,尚書如故。許
 以便宜從事,敕代王待以師傅之禮。



 會楊玄感圍逼東都,玄率步騎七萬援之。至華陰,掘楊素塚,焚其骸骨,夷其塋域,示士卒以必死。既出潼關,議者恐崤、函有伏兵,請於陜縣沿流東下,直趣河陽,以攻其背。玄曰:「以吾度之,此計非豎子所及。」於是鼓行而進。既度函谷,卒如所量。於是遣武賁郎將張峻為疑軍於南道,玄以大兵直趣城北。玄感逆拒之,且戰且行,屯軍金穀。於軍中掃地而祭高祖曰:「刑部尚書、京兆內史臣衛文升,敢昭告於高祖文皇帝之靈:自皇家啟運,三十餘年,武功文德,漸被海外。楊玄感孤負聖恩,躬為蛇豕,蜂飛蟻聚,犯我王
 略。臣二世受恩,一心事主,董率熊羆,志梟兇逆。若社稷靈長,宜令醜徒冰碎,如或大運去矣,幸使老臣先死。」詞氣抑揚,三軍莫不涕咽。時眾寡不敵,與賊頻戰不利,死傷大半。玄感盡銳來攻,玄苦戰,賊稍卻,進屯北芒。會宇文述、來護兒等援兵至,玄感懼而西遁。玄遣通議大夫斛斯萬善、監門直閣龐玉前鋒追之,及於閿鄉,與宇文述等合擊破之。車駕至高陽,徵詣行在所。帝勞之曰:「社稷之臣也。使朕無西顧之憂。」乃下詔曰:「近者妖氛充斥,擾動關、河,文升率勵義勇,應機響赴,表里奮擊,摧破兇醜,宜升榮命,式弘賞典。可右光祿大夫。」賜以良田、甲第,
 資物巨萬。還鎮京師,帝謂之曰:「關右之任,一委於公。公安,社稷乃安;公危,社稷亦危。出入須有兵衛,坐臥恆宜自牢,勇夫重閉,此其義也。今特給千兵,以充侍從。」賜以玉麟符。十一年,詔玄安撫關中。時盜賊蜂起,百姓饑饉,玄竟不能救恤,而官方壞亂,貨賄公行。玄自以年老,上表乞骸骨,帝使內史舍人封德彞馳諭之曰:「京師國本,王業所基,宗廟園陵所在,藉公耆舊,臥以鎮之。朕為國計,義無相許,故遣德彞口陳指意。」玄乃止。義師入關,自知不能守,憂懼稱疾,不知政事。城陷,歸於家。義寧中卒,時年七十七。



 子孝則,官至通事舍人、兵部承務郎,早卒。



 劉權劉權,字世略,彭城豐人也。祖軌,齊羅州刺史。權少有俠氣,重然諾,藏亡匿死,吏不敢過門。後更折節好學,動循法度。初為州主簿,仕齊,釋褐奉朝請、行臺郎中。及齊滅,周武帝以為假淮州刺史。高祖受禪,以車騎將軍領鄉兵。後從晉王廣平陳,以功進授開府儀同三司,賜物三千段。宋國公賀若弼甚禮之。開皇十二年,拜蘇州刺史,賜爵宗城縣公。於時江南初平,物情尚擾,權撫以恩信,甚得民和。煬帝嗣位,拜衛尉卿,進位銀青光祿大夫。大業五年,從征吐谷渾,權率眾出伊吾道,與賊相遇,擊走
 之。逐北至青海,虜獲千餘口,乘勝至伏俟城。帝復令權過曼頭、赤水,置河源郡、積石鎮,大開屯田,留鎮西境。在邊五載,諸羌懷附,貢賦歲入,吐谷渾餘燼遠遁,道路無壅。徵拜司農卿。加位金紫光祿大夫。尋為南海太守。行至鄱陽,會群盜起,不得進,詔令權召募討之。權率兵與賊相遇,不與戰,先乘單舸詣賊營,說以利害。群賊感悅,一時降附,帝聞而嘉之。既至南海,甚有異政。數歲,遇盜賊群起,數來攻郡,豪帥多願推權為首,權竟盡力固守以拒之。子世徹又密遣人齎書詣權,稱四方擾亂,英雄並起,時不可失,諷令舉兵。權召集佐僚,對斬其使,竟無
 異圖,守之以死。卒官,時年七十。



 世徹倜儻不羈,頗為時人所許。大業末,群雄並起,世徹所至之處,輒為所忌,多拘禁之,後竟為兗州賊帥徐圓朗所殺。



 權從父烈,字子將,美容儀,有器局,官至鷹揚郎將。有子德威,知名於世。



 史臣曰:子蓋雅有幹局,質性嚴敏,見義而勇,臨機能斷,保全都邑,勤亦懋哉!楊諒干紀,史祥著獨克之效,群盜侵擾,義臣致三捷之功,此皆名重當年,聲流後葉者也。元壽彈奏行本,有意存夫名教,然其計功稱伐,猶居義臣之後,端揆之贈,不已優乎?文升東都解圍,頗亦宣力,西京居守,政以賄成,鄙哉鄙哉,夫何足數!劉權淮楚舊
 族,早著雄名,屬擾攘之辰,居尉佗之地,遂能拒子邪計,無所覬覦,雖謝勤王之謀,足為守節之士矣。



\end{pinyinscope}