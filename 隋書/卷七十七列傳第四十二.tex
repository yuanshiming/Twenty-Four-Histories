\article{卷七十七列傳第四十二}

\begin{pinyinscope}

 隱逸自肇有書契,綿歷百王,雖時有盛衰,未嘗無隱逸之士。故《易》稱「遁世無悶」,又曰「不事王侯」;《詩》云「皎皎白駒,在彼空穀」;《禮》云「儒有上不臣天子,下不事王侯」;語曰「舉逸民,天下之人歸心焉」。雖出處殊途,語默異用,各言其志,皆君子之道也。洪崖兆其始,箕山扇其風,七人作乎周年,四皓光乎漢日,魏、晉以降,其流逾廣。其大者則輕天下,
 細萬物,其小者則安苦節,甘賤貧。或與世同塵,隨波瀾以俱逝,或違時矯俗,望江湖而獨往。狎玩魚鳥,左右琴書,拾遺粒而織落毛,飲石泉而廕松柏。放情宇宙之外,自足懷抱之中,然皆欣欣於獨善,鮮汲汲於兼濟。而受命哲王,守文令主,莫不束帛交馳,蒲輪結轍,奔走巖谷,唯恐不逮者,何哉?以其道雖未弘,志不可奪,縱無舟楫之功,終有賢貞之操。足以立懦夫之志,息貪競之風,與夫茍得之徒,不可同年共日。所謂無用以為用,無為而無不為者也。故敘其人,列其行,以備《隱逸篇》云。



 李士謙
 李士謙,字子約,趙郡平棘人也。髫齔喪父,事母以孝聞。母曾嘔吐,疑為中毒,因跪而嘗之。伯父魏岐州刺史瑒,深所嗟尚,每稱曰:「此兒吾家之顏子也。」



 年十二,魏廣平王贊闢開府參軍事。後丁母憂,居喪骨立。有姊適宋氏,不勝哀而死。士謙服闋,舍宅為伽藍,脫身而出。詣學請業,研精不倦,遂博覽群籍,兼善天文術數。齊吏部尚書辛術召署員外郎,趙郡王睿舉德行,皆稱疾不就。和士開亦重其名,將諷朝廷,擢為國子祭酒。士謙知而固辭,得免。隋有天下,畢志不仕。



 自以少孤,未嘗飲酒食肉,口無殺害之言。至於親賓來萃,輒陳樽俎,對之危坐,終日
 不倦。李氏宗黨豪盛,每至春秋二社,必高會極歡,無不沉醉喧亂。嘗集士謙所,盛饌盈前,而先為設黍,謂群從曰:「孔子稱黍為五穀之長,荀卿亦云食先黍稷,古人所尚,容可違乎?」少長肅然,不敢馳惰,退而相謂曰:「既見君子,方覺吾徒之不德也。」士謙聞而自責曰:「何乃為人所疏,頓至於此!」家富於財,躬處節儉,每以振施為務。州里有喪事不辦者,士謙輒奔走赴之,隨乏供濟。有兄弟分財不均,至相鬩訟,士謙聞而出財,補其少者,令與多者相埒。兄弟愧懼,更相推讓,卒為善士。有牛犯其田者,士謙牽置涼處飼之,過於本主。望見盜刈其禾黍者,默而
 避之。其家僮嘗執盜粟者,士謙慰諭之曰:「窮困所致,義無相責。」



 遽令放之。其奴嘗與鄉人董震因醉角力,震扼其喉,斃於手下。震惶懼請罪,士謙謂之曰:「卿本無殺心,何為相謝!然可遠去,無為吏之所拘。」性寬厚,皆此類也。其後出粟數千石,以貸鄉人,值年穀不登,債家無以償,皆來致謝。士謙曰:「吾家餘粟,本圖振贍,豈求利哉!」於是悉召債家,為設酒食,對之燔契,曰:「債了矣,幸勿為念也。」各令罷去。明年大熟,債家爭來償謙,謙拒之,一無所受。他年又大饑,多有死者,士謙罄竭家資,為之糜粥,賴以全活者將萬計。收埋骸骨,所見無遺。至春,又出糧種,分
 給貧乏。趙郡農民德之,撫其子孫曰:「此乃李參軍遺惠也。」或謂士謙曰:「子多陰德。」士謙曰:「所謂陰德者何?猶耳鳴,已獨聞之,人無知者。今吾所作,吾子皆知,何陰德之有!」



 士謙善談玄理,嘗有一客在坐,不信佛家應報之義,以為外典無聞焉。士謙喻之曰:「積善餘慶,積惡餘殃,高門待封,掃墓望喪,豈非休咎之應邪?佛經云輪轉五道,無復窮已,此則賈誼所言,千變萬化,未始有極,忽然為人之謂也。佛道未東,而賢者已知其然矣。至若鯀為黃熊,杜宇為鶗鴂,褒君為龍,牛哀為獸,君子為鵠,小人為猿,彭生為豕,如意為犬,黃母為黿,宣武為鱉,鄧艾為牛,
 徐伯為魚,鈴下為烏,書生為蛇,羊祜前身,李氏之子,此非佛家變受異形之謂邪?」



 客曰:「邢子才云,豈有松柏後身化為樗櫟,僕以為然。」士謙曰:「此不類之談也。變化皆由心而作,木豈有心乎?」客又問三教優劣,士謙曰:「佛,日也;道,月也,儒,五星也。」客亦不能難而止。



 士謙平生喜為詠懷詩,輒毀棄其本,不以示人。又嘗論刑罰,遺文不具,其略曰:「帝王制法,沿革不同,自可損益,無為頓改。今之贓重者死,是酷而不懲也。



 語曰:『人不畏死,不可以死恐之。』愚謂此罪宜從肉刑,刖其一趾,再犯者斷其右腕。流刑刖去右手三指,又犯者下其腕。小盜宜黥,又犯則落
 其所用三指,又不悛下其腕,無不止也。無賴之人,竄之邊裔,職為亂階,適所以召戎矣,非求治之道也。博弈淫游,盜之萌也,禁而不止,黥之則可。」有識者頗以為得治體。



 開皇八年,終於家,時年六十六。趙郡士女聞之,莫不流涕曰:「我曹不死,而令李參軍死乎!」會葬者萬餘人。鄉人李景伯等以士謙道著丘園,條其行狀,詣尚書省請先生之謚,事寢不行,遂相與樹碑於墓。



 其妻範陽盧氏,亦有婦德,及夫終後,所有賻贈,一無所受,謂州里父老曰:「參軍平生好施,今雖殞歿,安可奪其志哉!」於是散粟五百石以賑窮乏。



 崔廓子賾崔廓,字士玄,博陵安平人也。父子元,齊燕州司馬。廓少孤貧而母賤,由是不為邦族所齒。初為里佐,屢逢屈辱,於是感激,逃入山中。遂博覽書籍,多所通涉,山東學者皆宗之。既還鄉里,不應闢命。與郡李士謙為忘言之友,每相往來,時稱崔、李。及士謙死,廓哭之慟,為之作傳,輸之秘府。士謙妻盧氏寡居,每有家事,輒令人諮廓取定。郭嘗著論,言刑名之理,其義甚精,文多不載。大業中,終於家,時年八十。有子曰賾。



 賾字祖浚,七歲能屬文,容貌短小,有口才。開皇初,秦孝王薦之,射策高第,詔與諸
 儒定禮樂,授校書郎。尋轉協律郎,太常卿蘇威雅重之。母憂去職,性至孝,水漿不入口者五日。徵為河南、豫章二王侍讀,每更日來往二王之第。及河南為晉王,轉記室參軍,自此去豫章。王重之不已,遺賾書曰:昔漢氏西京,梁王建國,平臺、東苑,慕義如林。馬卿辭武騎之官,枚乘罷弘農之守。每覽史傳,嘗切怪之,何乃脫略官榮,棲遲籓邸?以今望古,方知雅志。



 彼二子者,豈徒然哉!足下博聞強記,鉤深致遠,視漢臣之三篋,似涉蒙山,對梁相之五車,若吞雲夢。吾兄欽賢重士,敬愛忘疲,先築郭隗之宮,常置穆生之醴。



 今者重開土宇,更誓山河,地方七
 百,牢籠曲阜,城兼七十,包舉臨淄,大啟南陽,方開東閤。想得奉飛蓋,曳長裾,藉玳筵,躡珠履,歌山桂之偃蹇,賦池竹之檀欒。



 其崇貴也如彼,其風流也如此,幸甚幸甚,何樂如之!高視上京,有懷德祖,才謝天人,多慚子建,書不盡意,寧俟繁辭。



 賾答曰:一昨伏奉教書,榮貺非恆,心靈自失。若乃理高《象》、《系》,管輅思而不解,事富《山海》,郭璞注而未詳。至於五色相宣,八音繁會,鳳鳴不足喻,龍章莫之比。吳札之論《周頌》,詎盡揄揚,郢客之奏《陽春》,誰堪赴節!伏惟令王殿下,稟潤天潢,承輝日觀,雅道貴於東平,文藝高於北海。漢則馬遷、蕭望,晉則裴楷、張華,雞樹
 騰聲,鵷池播美,望我清塵,悠然路絕。祖浚燕南贅客,河朔惰游,本無意於希顏,豈有心於慕藺!未嘗聚螢映雪,懸頭刺股,讀《論》唯取一篇,披《莊》不過盈尺。復況桑榆漸暮,藜藿屢空,舉燭無成,穿楊盡棄。但以燕求馬首,薛養雞鳴,謬齒鴻儀,虛班驥皁。挾太山而超北海,比報德而非難,堙昆侖以為池,匹酬恩而反易。忽屬周桐錫瑞,康水承家,門有將相,樹宜桃李。真龍將下,誰好有名,濫吹先逃,何須別聽!但慈旨抑揚,損上益下,江海所以稱王,丘陵為之不逮。曹植儻預聞高論,則不隕令名,楊修若切在下風,亦詎虧淳德。無任荷戴之至,謹奉啟以聞。



 豫
 章得書,賚米五十石,並衣服錢帛。時晉邸文翰,多成其手。王入東宮,除太子齋帥,俄遷舍人。及元德太子薨,以疾歸於家。後徵授起居舍人。大業四年,從駕汾陽宮,次河陽鎮。藍田令王曇於藍田山得一玉人,長三尺四寸,著大領衣,冠幘,奏之。詔問群臣,莫有識者,賾答曰:「謹按漢文已前,未有冠幘,即是文帝以來所制作也。臣見魏大司農盧元明撰《嵩高山廟記》云,有神人,以玉為形,像長數寸,或出或隱,出則令世延長。伏惟陛下應天順民,定鼎嵩洛,嶽神自見,臣敢稱慶。」因再拜,百官畢賀,天子大悅,賜縑二百匹。從駕登太行山,詔問賾曰:「何處有羊
 腸阪?」賾對曰:「臣按《漢書·地理志》,上黨壺關縣有羊腸阪。」



 帝曰:「不是。」又答曰:「臣按皇甫士安撰《地書》云,太原北九十里有羊腸阪。」



 帝曰:「是也。」因謂牛弘曰:「崔祖浚所謂問一知二。」五年,受詔與諸儒撰《區宇圖志》二百五十卷,奏之。帝不善之,更令虞世基、許善心衍為六百卷。以父憂去職,尋起令視事。遼東之役,授鷹揚長史,置遼東郡縣名,皆賾之議也。奉詔作《東征記》。九年,除越王長史。於時山東盜賊蜂起,帝令撫慰高陽、襄國,歸首者八百餘人。十二年,從駕江都。宇文化及之弒帝也,引為著作郎,稱疾不起。



 在路發疾,卒於彭城,時年六十九。



 賾與洛陽元
 善、河東柳抃、太原王劭、吳興姚察、瑯邪諸葛潁、信都劉焯、河間劉炫相善,每因休假,清談竟日。所著詞賦碑志十餘萬言,撰《洽聞志》七卷,《八代四科志》三十卷,未及施行,江都傾覆,咸為煨燼。



 徐則徐則,東海郯人也。幼沈靜,寡嗜欲。受業於周弘正,善三玄,精於議論,聲擅都邑,則嘆曰:「名者實之賓,吾其為賓乎!」遂懷棲隱之操,杖策入縉雲山。



 後學數百人,苦請教授,則謝而遣之。不娶妻,常服巾褐。陳太建時,應召來憩於至真觀。期月,又辭入天臺山,因絕穀養性,所資唯松
 水而已,雖隆冬洹寒,不服綿絮。太傅徐陵為之刊山立頌。初在縉雲山,太極真人徐君降之曰:「汝年出八十,當為王者師,然後得道也。」晉王廣鎮揚州,知其名,手書召之曰:「夫道得眾妙,法體自然,包涵二儀,混成萬物,人能弘道,道不虛行。先生履德養空,宗玄齊物,深明義味,曉達法門。悅性沖玄,怡神虛白,餐松餌術,棲息煙霞。望赤城而待風雲,游玉堂而駕龍鳳,雖復藏名臺岳,猶且騰實江淮,藉甚嘉猷,有勞寤寐。欽承素道,久積虛襟,側席幽人,夢想巖穴。霜風已冷,海氣將寒,偃息茂林,道體休悆。昔商山四皓,輕舉漢庭,淮南八公,來儀籓邸。古今雖
 異,山谷不殊,市朝之隱,前賢已說,導凡述聖,非先生而誰!故遣使人,往彼延請,想無勞束帶,賁然來思,不待蒲輪,去彼空谷。希能屈己,佇望披雲。」則謂門人曰:「吾今年八十一,王來召我,徐君之旨,信而有徵。」於是遂詣揚州。晉王將請受道法,則辭以時日不便。其後夕中,命侍者取香火,如平常朝禮之儀。至於五更而死,支體柔弱如生,停留數旬,顏色無變。晉王下書曰:「天臺真隱東海徐先生,虛確居宗,沖玄成德,齊物處外,檢行安身。草褐蒲衣,餐松餌術,棲隱靈岳,五十餘年。卓矣仙才,飄然勝氣,千尋萬頃,莫測其涯。寡人欽承道風,久餐德素,頻遣使
 乎,遠此延屈,冀得虔受上法,式建良緣。至此甫爾,未淹旬日,厭塵羽化,反真靈府。



 身體柔軟,顏色不變,經方所謂尸解地仙者哉!誠復師禮未申,而心許有在,雖忘怛化,猶愴於懷,喪事所資,隨須供給。霓裳羽蓋,既且騰雲,空槨餘衣,詎藉墳壟!但杖為猶存,示同俗法,宜遣使人,送還天臺定葬。」是時自江都至於天臺,在道多見則徒步,雲得放還。至其舊居,取經書道法,分遺弟子,仍令凈掃一房,曰:「若有客至,宜延之於此。」然後跨石梁而去,不知所之。須臾,尸柩至,方知其靈化。時年八十二。晉王聞而益異之,賵物千段,遣畫工圖其狀貌,令柳抃為之贊
 曰:「可道非道,常道無名。上德不德,至德無盈。玄風扇矣,而有先生。夙煉金液,怡神玉清。石髓方軟,云丹欲成。言追葛稚,將侶茅嬴。我王遙屬,爰感靈誠。柱下暫啟,河上沉精。留符告信,化杖飛聲。永思靈跡,曷用攄情?時披素繪,如臨赤城。」



 時有建安宋玉泉、會稽孔道茂、丹陽王遠知等,亦行闢谷,以松水自給,皆為煬帝所重。



 張文詡張文詡,河東人也。父琚,開皇中為洹水令,以清正聞。有書數千卷,教訓子侄,皆以明經自達。文詡博覽文籍,特精《三禮》,其《周易》、《詩》、《書》及《春秋三傳》,並皆通習。每好鄭玄
 注解,以為通博,其諸儒異說,亦皆詳究焉。



 高祖引致天下名儒碩學之士,其房暉遠、張仲讓、孔籠之徒,並延之於博士之位。



 文詡時游太學,暉遠等莫不推伏之,學內翕然,咸共宗仰。其門生多詣文詡,請質凝滯,文詡輒博引證據,辨說無窮,唯其所擇。治書侍御史皇甫誕一時朝彥,恆執弟子之禮。適至南臺,遽飾所乘馬,就學邀屈。文詡每牽馬步進,意在不因人以自致也。右僕射蘇威聞其名而召之,與語,大悅,勸令從官。文詡意不在仕,固辭焉。



 仁壽末,學廢,文詡策杖而歸,灌園為業。州郡頻舉,皆不應命。事母以孝聞。每以德化人,鄉黨頗移風俗。嘗
 有人夜中竊刈其麥者,見而避之,盜因感悟,棄麥而謝。文詡慰諭之,自誓不言,固令持去。經數年,盜者向鄉人說之,始為遠近所悉。



 鄰家築墻,心有不直,文詡因毀舊堵以應之。文詡嘗有腰疾,會醫者自言善禁,文詡令禁之,遂為刃所傷,至於頓伏床枕。醫者叩頭請罪,文詡遽遣之,因為其隱,謂妻子曰:「吾昨風眩,落坑所致。」其掩人之短,皆此類也。州縣以其貧素,將加振恤,輒辭不受。每閑居無事,從容長嘆曰:「老冉冉而將至,恐修名之不立!」



 以如意擊幾,皆有處所,時人方之閔子騫原憲焉。終於家,年四十。鄉人為立碑頌,號曰張先生。



 史臣曰:古之所謂隱逸者,非伏其身而不見也,非閉其言而不出也,非藏其智而不發也。蓋以恬淡為心,不曒不昧,安時處順,與物無私者也。士謙等忘懷纓冕,畢志丘園,隱不違親,貞不絕俗,不教而勸,虛往實歸,愛之如父母,懷之如親戚,非有自然之純德,其孰能至於斯乎?然士謙聞譽不喜,文詡見傷無慍,徐則志在沉冥,不可親疏,莫能貴賤,皆抱樸之士矣。崔廓感於屈辱,遂以肥遁見稱,祖浚文籍之美,足以克隆先構,父子雖動靜殊方,其於成名一也,美哉!



\end{pinyinscope}