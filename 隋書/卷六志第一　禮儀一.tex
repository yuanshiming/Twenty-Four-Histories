\article{卷六志第一 禮儀一}

\begin{pinyinscope}

 唐、
 虞之時,祭天之屬為天禮,祭地之屬為地禮,祭宗廟之屬為人禮。故《書》云命伯夷典朕三禮,所以彌綸天地,經緯陰陽,辨幽賾而洞幾深,通百神而節萬事。



 殷因於夏,有所損益,旁垂祗訓,以勸生靈。商辛無道,雅章湮滅。周公救亂,弘制斯文,以吉禮敬鬼神,以兇禮哀邦國,以賓禮親賓客,以軍禮誅不虔,以嘉禮合姻好,謂之五禮。
 故曰「禮經三百,威儀三千,未有入室而不由戶者」也。成、康由之,而刑厝不用。自犬戎弒後,遷周削弱,禮失樂微,風凋俗敝。仲尼預蠟賓而嘆曰:「丘有志焉,禹、湯、文、武、成王、周公未有不謹於禮者也。」於是緝禮興樂,欲救時弊。君棄不顧,道鬱不行。故敗國喪家亡人,必先廢其禮。昭公娶孟子而諱姓,楊侯竊女色而傷人,故曰婚姻之禮廢,則淫僻之罪多矣。群飲而逸,不知其郵,鄉飲酒之禮廢,則爭鬥之獄繁矣。魯侯逆五廟之祀,漢帝罷三年之制,喪祭之禮廢,則骨肉之恩薄矣。諸侯下堂於天子,五伯召君於河陽,朝聘之禮廢,則侵陵之漸起矣。秦氏以
 戰勝之威,並吞九國,盡收其儀禮,歸之咸陽。唯採其尊君抑臣,以為時用。至於退讓起於趨步,忠孝成於動止,華葉靡舉,鴻纖並擯。甚芻狗之棄路,若章甫之游越,儒林道盡,《詩》《禮》為煙。漢高祖既平秦亂,初誅項羽,放賞元勛,未遑朝制。群臣飲酒爭功,或拔劍擊柱,高祖患之。叔孫通言曰:「儒者難與進取,可與守成。」於是請起朝儀而許焉,猶曰:「度吾能行者為之。」



 微習禮容,皆知順軌。若祖述文武,憲章洙泗,則良由不暇,自畏之也。武帝興典制而愛方術,至於鬼神之祭,流宕不歸。世祖中興,明皇纂位,祀明堂,襲冠冕,登靈臺,望雲物,得其時制,百姓悅之。
 而朝廷憲章,其來已舊,或得之於升平之運,或失之於兇荒之年。而世載遐邈,風流訛舛,必有人情,將移禮意,殷周所以異軌,秦漢於焉改轍。至於增輝風俗,廣樹堤防,非禮威嚴,亦何以尚!譬山祗之有嵩岱,海若之有滄溟,飾以涓塵,不貽伊敗。而高堂生於所傳《士禮》亦謂之儀,弘暢人情,粉飾行事。洎西京以降,用相裁準,咸稱當世之美,自有周旋之節。黃初之詳定朝儀,太始之削除乖謬,則《宋書》言之備矣。



 梁武始命群儒,裁成大典。吉禮則明山賓,兇禮則嚴植之,軍禮則陸璉,賓禮則賀瑒,嘉禮則司馬褧。帝又命沈約、周舍、徐勉、何佟之等,咸在參
 詳。陳武克平建業,多準梁舊,仍詔尚書左丞江德藻、員外散騎常侍沈洙、博士沈文阿、中書舍人劉師知等,或因行事,隨時取舍。後齊則左僕射陽休之、度支尚書元修伯、鴻臚卿王晞、國子博士熊安生,在周則蘇綽、戶辯、宇文弼,並習於儀禮者也,平章國典,以為時用。高祖命牛弘、辛彥之等採梁及北齊《儀注》,以為五禮云。



 《禮》曰:「萬物本乎天,人本乎祖,所以配上帝也。」秦人蕩六籍以為煨燼,祭天之禮殘缺,儒者各守其所見物而為之義焉。一云:祭天之數,終歲有九,祭地之數,一歲有二,圓丘、方澤,三年一行。若圓丘、方澤之年,祭天有九,祭地
 有二。若天不通圓丘之祭,終歲有八;地不通方澤之祭,終歲有一。此則鄭學之所宗也。一云:「唯有昊天,無五精之帝。而一天歲二祭,壇位唯一。圓丘之祭,即是南郊,南郊之祭,即是圓丘。日南至,於其上以祭天,春又一祭,以祈農事,謂之二祭,無別天也。五時迎氣,皆是祭五行之人帝太皞之屬,非祭天也。天稱皇天,亦稱上帝,亦直稱帝。五行人帝亦得稱上帝,但不得稱天。故五時迎氣及文、武配祭明堂,皆祭人帝,非祭天也。此則王學之所宗也。梁、陳以降,以迄於隋,議者各宗所師,故郊丘互有變易。



 梁南郊,為圓壇,在國之南。高二丈七尺,上徑十一丈,
 下徑十八丈。其外再壝,四門。常與北郊間歲。正月上辛行事,用一特牛,祀天皇上帝之神於其上,以皇考太祖文帝配。禮以蒼璧制幣。五方上帝、五官之神、太一、天一、日、月、五星、二十八宿、太微、軒轅、文昌、北斗、三臺、老人、風伯、司空、雷電、雨師,皆從祀。其二十八宿及雨師等座有坎,五帝亦如之,餘皆平地。器以陶匏,席用稿秸。太史設柴壇於丙地。皇帝齋於萬壽殿,乘玉輅,備大駕以行禮。禮畢,變服通天冠而還。北郊,為方壇於北郊。上方十丈,下方十二丈,高一丈。四面各有陛。



 其外為壝再重。與南郊間歲。正月上辛,以一特牛,祀后地之神於其上,以德
 後配。



 禮以黃琮制幣。五官之神、先農、五岳、沂山、岳山、白石山、霍山、無閭山、蔣山、四海、四瀆、松江、會稽江、錢塘江、四望,皆從祀。太史設埋坎於壬地焉。



 天監三年,左丞吳操之啟稱:「《傳》云『啟蟄而郊』,郊應立春之後。」尚書左丞何佟之議:「今之郊祭,是報昔歲之功,而祈今年之福。故取歲首上辛,不拘立春之先後。周冬至於圓丘,大報天也。夏正又郊,以祈農事,故有啟蟄之說。



 自晉太始二年,並圓丘、方澤同於二郊。是知今之郊禋,禮兼祈報,不得限以一途也。」帝曰:「圓丘自是祭天,先農即是祈穀。但就陽之位,故在郊也。冬至之夜,陽氣起於甲子,既祭昊天,宜
 在冬至。祈穀時可依古,必須啟蟄。在一郊壇,分為二祭。」自是冬至謂之祀天,啟蟄名為祈穀。何佟之又啟:「案鬯者盛以六彞,覆以畫,備其文飾,施之宗廟。今南北二郊,《儀注》有課,既乖尚質,謂宜革變。」



 博士明山賓議,以為:「《表記》『天子親耕,粢盛秬鬯,以事上帝』,蓋明堂之裸耳。郊不應裸。」帝從之。又有司以為祀竟,器席相承還庫,請依典燒埋之。佟之等議:「案《禮》『祭器弊則埋之』。今一用便埋,費而乖典。」帝曰:「薦藉輕物,陶匏賤器,方還付庫,容復穢惡。但敝則埋之,蓋謂四時祭器耳。」自是從有司議,燒埋之。四年,佟之云:「《周禮》『天曰神,地曰祇』。今天不稱神,地不
 稱祇,天欑題宜曰皇天座,地欑宜曰後地座。又南郊明堂用沉香,取本天之質,陽所宜也。北郊用上和香,以地於人親,宜加雜馥。」帝並從之。五年,明山賓稱:「伏尋制旨,周以建子祀天,五月祭地。殷以建丑祀天,六月祭地。夏以建寅祀天,七月祭地。自頃代以來,南北二郊,同用夏正。」詔更詳議。山賓以為二儀並尊,三朝慶始,同以此日二郊為允。並請迎五帝於郊,皆以始祖配饗。及郊廟受福,唯皇帝再拜,明上靈降祚,臣下不敢同也。」詔並依議。六年,議者以為北郊有嶽鎮海瀆之座,而又有四望之座,疑為煩重。儀曹郎硃異議曰:「望是不即之名,豈容局
 於星海,拘於岳瀆?」明山賓曰:「《舜典》云『望於山川』。《春秋傳》曰『江、漢、沮、漳,楚之望也』。而今北郊設嶽鎮海瀆,又立四望,竊謂煩黷,宜省。」徐勉曰:「岳瀆是山川之宗。至於望祀之義,不止於岳瀆也。若省四望,於義為非。」議久不能決。至十六年,有事北郊,帝復下其議。於是八座奏省四望、松江、浙江、五湖等座。其鐘山、白石,既土地所在,並留如故。七年,帝以一獻為質,三獻則文,事天之道,理不應然,詔下詳議。博士陸瑋、明山賓、禮官司馬褧以為「宗祧三獻,義兼臣下,上天之禮,主在帝王,約理申義,一獻為允」。自是天地之祭皆一獻,始省太慰亞獻,光祿終獻。又太
 常丞王僧崇稱:「五祀位在北郊,圓丘不宜重設。」帝曰:「五行之氣,天地俱有,故宜兩從。」僧崇又曰:「風伯、雨師,即箕、畢星矣。而今南郊祀箕、畢二星,復祭風師、雨師,恐乖祀典。」帝曰:「箕、畢自是二十八宿之名,風師、雨師自是箕、畢星下隸。兩祭非嫌。」十一年,太祝牒,北郊止有一海,及二郊相承用柒俎盛牲,素案承玉。又制南北二郊壇下眾神之座,悉以白茅,詔下詳議。八座奏:「《禮》云『觀天下之物,無可以稱其德』,則知郊祭為俎,理不應柒。又藉用白茅,禮無所出。皇天大帝坐既用俎,則知郊有俎義。」於是改用素俎,並北郊置四海座。五帝以下,悉用蒲席槁薦,並
 以素俎。又帝曰:「《禮》『祭月於坎』,良由月是陰義。今五帝天神,而更居坎。又《禮》云『祭日於壇,祭月於坎』,並是別祭,不關在郊,故得各從陰陽而立壇坎。於南郊,就陽之義,居於北郊,就陰之義。既云就陽,義與陰異。



 星月與祭,理不為坎。」八座奏曰:「五帝之義,不應居坎。良由齊代圓丘小而且峻,邊無安神之所。今丘形既大,易可取安。請五帝座悉於壇上,外壝二十八宿及雨師等座,悉停為坎。」自是南北二郊,悉無坎位矣。十七年,帝以威仰、魄寶俱是天帝,於壇則尊,於下則卑。且南郊所祭天皇,其五帝別有明堂之祀,不煩重設。



 又郊祀二十八宿而無十二
 辰,於義闕然。於是南郊始除五帝祀,加十二辰座,與二十八宿各於其方而為壇。



 陳制,亦以間歲。正月上辛,用特牛一,祀天地於南北二郊。永定元年,武帝受禪,修南郊,圓壇高二丈二尺五寸,上廣十丈,柴燎告天。明年正月上辛,有事南郊,以皇考德皇帝配,除十二辰座,加五帝位,其餘準梁之舊。北郊為壇,高一丈五尺,廣八丈,以皇妣昭後配,從祀亦準梁舊。及文帝天嘉中,南郊改以高祖配,北郊以德皇帝配天。太中大夫、領大著作、攝太常卿許享奏曰:「昔梁武帝云:『天數五,地數五,五行之氣,天地俱有。』故南北郊內,並
 祭五祀。臣按《周禮》:『以血祭社稷五祀。』鄭玄云:『陰祀自血起,貴氣臭也。五祀,五官之神也。』五神主五行,隸於地,故與埋沈副辜同為陰祀。既非煙柴,無關陽祭。故何休云:『周爵五等者,法地有五行也。』五神位在北郊,圓丘不宜重設。」制曰:「可。」



 亨又奏曰:「梁武帝議,箕、畢自是二十八宿之名,風師、雨師自是箕、畢下隸,非即星也。故郊雩之所,皆兩祭之。臣案《周禮》大宗伯之職云:『燎祀司中、司令、風師、雨師。』鄭眾云:『風師,箕也;雨師,畢也。』《詩》云:『月離於畢,俾滂沱矣。』如此則風伯、雨師即箕、畢星矣。而今南郊祀箕、畢二星,復祭風伯、雨師,恐乖祀典。」制曰:「若郊設星位,
 任即除之。」享又奏曰:「《梁儀注》曰:『一獻為質,三獻為文。事天之事,故不三獻。』臣案《周禮》司樽所言,三獻施於宗祧,而鄭注『一獻施於群小祀』。今用小祀之禮施於天神大帝,梁武此義為不通矣。且樽俎之物,依於質文,拜獻之禮,主於虔敬。今請凡郊丘祀事,準於宗祧,三獻為允。」制曰:「依議。」廢帝光大中,又以昭後配北郊。及宣帝即位,以南北二郊卑下,更議增廣。久而不決。至太建十一年,尚書祠部郎王元規議曰:案前漢《黃圖》,上帝壇徑五丈,高九尺;後土壇方五丈,高六尺。梁南郊壇上徑十一丈,下徑十八丈,高二丈七尺,北郊壇上方十丈,下方十二丈,
 高一丈。



 即日南郊壇廣十丈,高二丈二尺五寸,北郊壇廣九丈三尺,高一丈五寸。今議增南郊壇上徑十二丈,則天大數,下徑十八丈,取於三分益一,高二丈七尺,取三倍九尺之堂。北郊壇上方十丈,以則地義,下至十五丈,亦取二分益一,高一丈二尺,亦取二倍漢家之數。《禮記》云:「為高必因丘陵,為下必因川澤。因名山升中於天,因吉土饗帝於郊。」《周官》云:「冬日至,祠天於地上之圓丘。夏日至,祭地於澤中之方丘。」《祭法》云:「燔柴於泰壇,祭天也。瘞埋於泰折,祭地也。」



 《記》云:「至敬不壇,掃地而祭。」於其質也,以報覆燾持載之功。《爾雅》亦云:「丘,言非人所造為。」
 古圓方兩丘,並因見有而祭。本無高廣之數。後世隨事遷都,而建立郊禮。或有地吉而未必有丘,或有見丘而不必廣潔。故有築建之法,而制丈尺之儀。愚謂郊祀事重,圓方二丘,高下廣狹,既無明文,但五帝不相沿,三王不相襲。今謹述漢、梁並即日三代壇不同,及更增修丈尺如前。聽旨。



 尚書僕射臣繕,左戶尚書臣元饒、左丞臣周確、舍人臣蕭淳、儀曹郎臣沈客卿同元規議。詔遂依用。後主嗣立,無意曲禮之事,加舊儒碩學,漸以凋喪,至於朝亡,竟無改作。



 後齊制,圓丘方澤,並三年一祭,謂之帝祀。圓丘在國南郊。丘下廣輪二百七十尺,上廣輪四
 十六尺,高四十五尺。三成,成高十五尺,上中二級,四面各一陛,下級方維八陛。周以三壝,去丘五十步。中壝去內壝,外壝去中壝,各二十五步。



 皆通八門。又為大營於外壝之外,輪廣三百七十步。其營塹廣一十二尺,深一丈,四面各通一門。又為燎壇,於中壝之外,當丘之丙地。廣輪三十六尺,高三尺,四面各有陛。方澤為壇在國北郊。廣輪四十尺,高四尺,面各一陛。其外為三壝,相去廣狹同圓丘。壝外大營,廣輪三百二十步。營塹廣一十二尺,深一丈,四面各通一門。又為瘞坎於壇之壬地,中壝之外,廣深一丈二尺。圓丘則以蒼璧束帛,正月上辛,祀
 昊天上帝於其上,以高祖神武皇帝配。五精之帝,從祀於其中丘。面皆內向。日月、五星、北斗、二十八宿、司中、司命、司人、司祿、風師、雨師、靈星於下丘,為眾星之位,遷於內壝之中。合用蒼牲九。夕牲之旦,太尉告廟,陳幣於神武廟訖,埋於兩楹間焉。皇帝初獻,太尉亞獻,光祿終獻。司徒獻五帝,司空獻日月、五星、二十八宿,太常丞已下薦眾星。方澤則以黃琮束帛,夏至之日,禘昆侖皇地祇於其上,以武明皇后配。其神州之神、社稷、岱嶽、沂鎮、會稽鎮、雲雲山、亭亭山、蒙山、羽山、嶧山、崧岳、霍嶽、衡鎮、荊山、內方山、大別山、敷淺原山、桐柏山、陪尾山、華嶽、太嶽
 鎮、積石山、龍門山、江山、岐山、荊山、嶓塚山、壺口山、雷首山、底柱山、析城山、王屋山、西傾硃圉山、鳥鼠同穴山、熊耳山、敦物山、蔡蒙山、梁山、岷山、武功山、太白山、恆岳,醫無閭山鎮、陰山、白登山、碣石山、太行山、狼山、封龍山、漳山、宣務山、閼山、方山、茍山、狹龍山、淮水、東海、泗水、沂水、淄水、濰水、江水、南海、漢水、穀水、洛水、伊水、漾水、沔水、河水、西海、黑水、澇水、渭水、涇水、酆水、濟水、北海、松水、京水、桑乾水、漳水、呼沲水、衛水、洹水、延水,並從祀。其神州位在青陛之北甲寅地,社位赤陛之西未地,稷位白陛之南庚地;自餘並內壝之內,內向,各如其方。合用牲十二,
 儀同圓丘。其後諸儒定禮,圓丘改以冬至云。其南北郊則歲一祀,皆以正月上辛。南郊為壇於國南,廣輪三十六尺,高九尺,四面各一陛。



 為三壝,內壝去壇二十五步,中壝、外壝相去如內壝。四面各通一門。又為大營於外壝之外,廣輪二百七十步。營塹廣一丈,深八尺,四面各一門。又為燎壇於中壝之外丙地,廣輪二十七尺,高一尺八寸,四面各一陛。祀所感帝靈威仰於壇,以高祖神武皇帝配。禮用四圭有邸,幣各如方色。其上帝及配帝,各用騂特牲一,儀燎同圓丘。其北郊則為壇如南郊壇,為瘞坎如方澤坎,祀神州神於其上,以武明皇后配。禮
 用兩圭有邸,各用黃牲一,儀瘞如北郊。



 後周憲章姬周,祭祀之式,多依《儀禮》。司量掌為壇之制,圓丘三成,成崇一丈二尺,深二丈。上徑六丈,十有二階,每等十有二節。在國陽七里之郊。圓壝徑三百步,內壝半之。方一成,下崇一丈,徑六丈八尺,上崇五尺,方四丈,八方,方一階,階十級,級一尺。方丘在國陰六里之郊。丘一成,八方,下崇一丈,方六丈八尺,上崇五尺,方四丈。方一階,尺一級。其壝八面,徑百二十步,內壝半之。



 南郊為方壇於國南五里。其崇一丈二尺,其廣四丈。其壝方百二十步,內壝半之。



 神州之壇,崇一丈,方四丈,在北郊方
 丘之右。其壝如方丘。其祭圓丘及南郊,並正月上辛。圓丘則以其先炎帝神農氏配昊天上帝於其上。五方上帝、日月、內官、中官、外官、眾星,並從祀。皇帝乘蒼輅,載玄冕,備大駕而行。預祭者皆蒼服。



 南郊,以始祖獻侯莫那配所感帝靈威仰於其上。北郊方丘,則以神農配後地之祇。



 神州則以獻侯莫那配焉。其用牲之制,祀昊天上帝,祭皇地祇及五帝、日月、五星、十二辰、四望、五官,各以其方色毛。宗廟以黃,社稷以黝,散祭祀用純,表貉磔禳用龐。



 高祖受命,欲新制度。乃命國子祭酒辛彥之議定祀典。
 為圓丘於國之南,太陽門外道東二里。其丘四成,各高八尺一寸。下成廣二十丈,再成廣十五丈,又三成廣十丈,四成廣五丈。再歲冬至之日,祀昊天上帝於其上,以太祖武元皇帝配。五方上帝、日月、五星、內官四十二座、次官一百三十六座、外官一百一十一座、眾星三百六十座,並皆從祀。上帝、日月在丘之第二等,北斗五星、十二辰、河漢、內官在丘第三等,二十八宿、中官在丘第四等,外官在內壝之內,眾星在內壝之外。



 其牲,上帝、配帝用蒼犢二,五帝、日月用方色犢各一,五星已下用羊豕各九。為方丘於宮城之北十四里。其丘再成,成高五尺,
 下成方十丈,上成方五丈。夏至之日,祭皇地祇於其上,以太祖配。神州、迎州、冀州、戎州、拾州、柱州、營州、咸州、陽州九州山、海、川、林、澤、丘陵、墳衍、原隰,並皆從祀。地祇及配帝在壇上,用黃犢二。神州九州神座於第二等八陛之間:神州東南方,迎州南方,冀州、戎州西南方,拾州西方,柱州西北方,營州北方,咸州東北方,陽州東方,各用方色犢一。九州山海已下,各依方面八陛之間。其冀州山林川澤,丘陵墳衍,於壇之南少西,加羊豕各九。南郊為壇於國之南,太陽門外道西一里,去宮十里。壇高七尺,廣四丈。孟春上辛,祠所感帝赤熛怒於其上,以太祖
 武元皇帝配。其禮四圭有邸,牲用騂犢二。北郊孟冬祭神州之神,以太祖武元皇帝配。牲用犢二。凡大祀,齋官皆於其晨集尚書省,受誓戒。散齋四日,致齋三日。祭前一日,晝漏上水五刻,到祀所,沐浴,著明衣,咸不得聞見衰絰哭泣。昊天上帝、五方上帝、日月、皇地祇、神州社稷、宗廟等為大祀,星辰、五祀、四望等為中祀,司中、司命、風師、雨師及諸星、諸山川等為小祀。大祀養性,在滌九旬,中祀三旬,小祀一旬。



 其牲方色難備者,聽以純色代。告祈之牲者不養。祭祀犧牲,不得捶撲。其死則埋之。



 初,帝既受周禪,恐黎元未愜,多說符瑞以耀之。其或造作而
 進者,不可勝計。



 仁壽元年冬至祠南郊,置昊天上帝及五方天帝位,並於壇上,如封禪禮。板曰:維仁壽元年,歲次作噩,嗣天子臣堅,敢昭告於昊天上帝:璇璣運行,大明南至。臣蒙上天恩造,群靈降福,撫臨率土,安養兆人。顧惟虛薄,德化未暢,夙夜憂懼,不敢荒怠。天地靈祇,降錫休瑞,鏡發區宇,昭彰耳目。爰始登極,蒙授龜圖,遷都定鼎,醴泉出地,平陳之歲,龍引舟師。省俗巡方,展禮東嶽,盲者得視,喑者得言,復有蹙人,忽然能步。自開皇已來,日近北極,行於上道,晷度延長。



 天啟太平,獸見一角,改元仁壽,楊樹生松。石魚彰合符之徵,玉兔顯永昌之
 慶,山圖石瑞,前後繼出,皆載臣姓名,褒紀國祚。經典諸緯,爰及玉龜,文字義理,遞相符會。宮城之內,及在山谷,石變為玉,不可勝數。桃區一嶺,盡是琉璃,黃銀出於神山,碧玉生於瑞獻。多楊山響,三稱國興,連雲山聲,萬年臨國。野鵝降天,仍住池沼,神鹿入苑,頻賜引導。騶虞見質,游驎在野,鹿角生於楊樹,龍湫出於荊谷。慶雲發彩,壽星垂耀。宮殿樓閣,咸出靈芝,山澤川原,多生寶物。



 威香散馥,零露凝甘。敦煌烏山,黑石變白,弘祿巖嶺,石華遠照。玄狐玄豹,白兔白狼,赤雀蒼烏,野蠶天豆,嘉禾合穗,珍木連理。神瑞休徵,洪恩景福,降賜無疆,不可具紀。
 此皆昊天上帝,爰降明靈,矜愍蒼生,寧靜海內,故錫茲嘉慶,咸使安樂,豈臣微誠所能上感。虔心奉謝,敬薦玉帛犧齊,粢盛庶品,燔祀於昊天上帝。皇考太祖武元皇帝,配神作主。



 大業元年,孟春祀感帝,孟冬祀神州,改以高祖文帝配。其餘並用舊禮。十年,冬至祀圓丘,帝不齋於次。詰朝,備法駕,至便行禮。是日大風,帝獨獻上帝,三公分獻五帝。禮畢,御馬疾驅而歸。



 明堂在國之陽。梁初,依宋、齊,其祀之法,猶依齊制。禮有不通者,武帝更與學者議之。舊齊儀,郊祀,帝皆以袞冕。至天監七年,始造大裘,而《明堂儀注》猶云袞服。十年,儀
 曹郎硃異以為:「《禮》大裘而冕,祭昊天上帝。五帝亦如之。



 良由天神高遠,義須誠質,今從泛祭五帝,理不容文。」於是改服大裘。異又以為:「齊儀初獻樽彞,明堂貴質,不應三獻。又不應象樽。《禮》云:『朝踐用太樽。』鄭云:『太樽,瓦也。』《記》又云:『有虞氏瓦樽。』此皆在廟所用,猶以質素,況在明堂,禮不容象。今請改用瓦樽,庶合文質之衷。」又曰:「宗廟貴文,故庶羞百品,天義尊遠,則須簡約。今《儀注》所薦,與廟不異,即理徵事,如為未允。



 請自今明堂肴膳準二郊。但帝之為名,本主生育,成歲之功,實為顯著。非如昊天,義絕言象,雖曰同郊,復應微異。若水土之品,蔬果之屬,猶
 宜以薦,止用梨棗橘慄四種之果,姜蒲葵韭四種之俎,粳稻黍粱四種之米。自此以外,郊所無者,請並從省除。」初,博士明山賓制《儀注》,明堂祀五帝,行禮先自赤帝始。異又以為:「明堂既泛祭五帝,不容的有先後,東階而升,宜先春帝。請改從青帝始。」又以為:「明堂籩豆等器,皆以雕飾。尋郊祀貴質,改用陶匏,宗廟貴文,誠宜雕俎。



 明堂之禮,既方郊為文,則不容陶匏,比廟為質,又不應雕俎。斟酌二途,須存厥衷,請改用純漆。」異又以「舊儀,明堂祀五帝,先酌鬱鬯,灌地求神,及初獻清酒,次酃,終醁。禮畢,太祝取俎上黍肉,當御前以授。請依郊儀,止一獻清酒。



 且五帝天神,不可求之於地,二郊之祭,並無黍肉之禮。並請停灌及授俎法。」又以為:「舊明堂皆用太牢。案《記》云:『郊用特牲』;又云『天地之牛,角繭慄』。



 五帝既曰天神,理無三牲之祭。而《毛詩·我將》篇,云祀文王於明堂,有『維羊維牛』之說。良由周監二代,其義貴文,明堂方郊,未為極質,故特用三牲,止為一代之制。今斟酌百王,義存通典,蔬果之薦,雖符周禮,而牲牢之用,宜遵夏殷。



 請自今明堂止用特牛,既合質文之中,又見貴誠之義。」帝並從之。先是,帝欲有改作,乃下制旨,而與群臣切磋其義。制曰:「明堂準《大戴禮》:『九室八牖,三十六戶。以茅蓋屋,上圓下方。』
 鄭玄據《援神契》,亦云『上圓下方』,又云『八窗四達』。明堂之義,本是祭五帝神,九室之數,未見其理。若五堂而言,雖當五帝之數,向南則背葉光紀,向北則背赤熛怒,東向西向,又亦如此,於事殊未可安。且明堂之祭五帝,則是總義,在郊之祭五帝,則是別義。宗祀所配,復應有室,若專配一室,則是義非配五,若皆配五,則便成五位。以理而言,明堂本無有室。」硃異以為:「《月令》『天子居明堂左個、右個』。聽朔之禮,既在明堂,今若無室,則於義成闕。」制曰:「若如鄭玄之義,聽朔必在明堂,於此則人神混淆,莊敬之道有廢。《春秋》云:『介居二大國之間。』此言明堂左右個
 者,謂所祀五帝堂之南,又有小室,亦號明堂,分為三處聽朔。既三處,則有左右之義。在營域之內,明堂之外,則有個名,故曰明堂左右個也。以此而言,聽朔之處,自在五帝堂之外,人神有別,差無相干。」其議是非莫定,初尚未改。十二年,太常丞虞爵復引《周禮》明堂九尺之筵,以為高下修廣之數,堂崇一筵,故階高九尺。



 漢家制度,猶遵此禮,故張衡云「度堂以筵」者也。鄭玄以廟寢三制既同,俱應以九尺為度。制曰:「可。」於是毀宋太極殿,以其材構明堂十二間,基準太廟。以中央六間安六座,悉南向。東來第一青帝,第二赤帝,第三黃帝,第四白帝,第五黑
 帝。配帝總配享五帝,在阼階東上,西向。大殿後為小殿五間,以為五佐室焉。



 陳制,明堂殿屋十二間。中央六間,依齊制,安六座。四方帝各依其方,黃帝居坤維,而配饗坐依梁法。武帝時,以德帝配。文帝時,以武帝配。廢帝已後,以文帝配。牲以太牢,粢盛六飯,金幵羹果蔬備薦焉。後齊採《周官·考工記》為五室,周採漢《三輔黃圖》為九室,各存其制,而竟不立。



 高祖平陳,收羅杞梓,郊丘宗社,典禮粗備,唯明堂未立。開皇十三年,詔命議之。禮部尚書牛弘、國子祭酒辛彥之等定議,事在弘傳。後檢校將作大匠事宇文愷依《月
 令》文,造明堂木樣,重簷復廟,五房四達,丈尺規矩,皆有準憑,以獻。



 高祖異之,命有司於郭內安業里為規兆。方欲崇建,又命詳定,諸儒爭論,莫之能決。弘等又條經史正文重奏。時非議既多,久而不定,又議罷之。及大業中,愷又造《明堂議》及樣奏之。煬帝下其議,但令於霍山採木,而建都興役,其制遂寢。



 終隋代,祀五方上帝,止於明堂,恆以季秋在雩壇上而祀。其用幣各於其方。人帝各在天帝之左。太祖武元皇帝在太昊南,西向。五官在庭,亦各依其方。牲用犢十二。皇帝、太尉、司農行三獻禮於青帝及太祖。自餘有司助奠。祀五官於堂下,行一獻禮。
 有燎。其省牲進熟,如南郊
 儀。



\end{pinyinscope}