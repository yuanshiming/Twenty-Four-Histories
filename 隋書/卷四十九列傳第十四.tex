\article{卷四十九列傳第十四}

\begin{pinyinscope}

 牛弘牛弘,字裏仁,安定鶉觚人也,本姓裛氏。祖熾,郡中正。父允,魏侍中、工部尚書、臨涇公,賜姓為牛氏。弘初在襁褓,有相者見之,謂其父曰:「此兒當貴,善愛養之。」及長,須貌甚偉,性寬裕,好學博聞。在周,起家中外府記室、內史上士。俄轉納言上士,專掌文翰,甚有美稱。加威烈將軍、員外散騎侍郎,修起居注。其後襲封臨涇公。宣政元年,轉
 內史下大夫,進位使持節、大將軍,儀同三司。



 開皇初,遷授散騎常侍、秘書監。弘以典籍遺逸,上表請開獻書之路,曰:經籍所興,由來尚矣。爻畫肇於庖羲,文字生於蒼頡。聖人所以弘宣教導,博通古今,揚於王庭,肆於時夏。故堯稱至聖,猶考古道而言;舜其大智,尚觀古人之象。《周官》外史掌三皇五帝之書,及四方之志。武王問黃帝、顓頊之道,太公曰:「在《丹書》。」是知握符御歷,有國有家者,曷嘗不以《詩》、《書》而為教,因禮樂而成功也。昔周德既衰,舊經紊棄。孔子以大聖之才,開素王之業,憲章祖述,制《禮》刊《詩》,正五始而修《春秋》,闡《十翼》而弘《易》道。治國立身,
 作範垂法。及秦皇馭宇,吞滅諸侯,任用威力,事不師古,始下焚書之令,行偶語之刑。先王墳籍,掃地皆盡。本既先亡,從而顛覆。臣以圖讖言之,經典盛衰,信有徵數。此則書之一厄也。漢興,改秦之弊,敦尚儒術,建藏書之策,置校書之官,屋壁山巖,往往間出。外有太常、太史之藏,內有延閣、秘書之府。至孝成之世,亡逸尚多,遣謁者陳農求遺書於天下,詔劉向父子讎校篇籍。漢之典文,於斯為盛。及王莽之末,長安兵起,宮室圖書,並從焚燼。此則書之二厄也。光武嗣興,尤重經誥,未及下車,先求文雅。於是鴻生巨儒,繼踵而集,懷經負帙,不遠斯至。肅宗
 親臨講肄,和帝數幸書林,其蘭臺、石室,鴻都、東觀,秘牒填委,更倍於前。及孝獻移都,吏民擾亂,圖書縑帛,皆取為帷囊。所收而西,裁七十餘乘。屬西京大亂,一時燔蕩。此則書之三厄也。魏文代漢,更集經典,皆藏在秘書、內外三閣,遣秘書郎鄭默刪定舊文。時之論者,美其硃紫有別。晉氏承之,文籍尤廣。晉秘書監荀勖定魏《內經》,更著《新簿》。雖古文舊簡,猶云有缺,新章後錄,鳩集已多,足得恢弘正道,訓範當世。屬劉、石憑陵,京華覆滅,朝章國典,從而失墜。此則書之四厄也。永嘉之後,寇竊競興。因河據洛,跨秦帶趙。論其建國立家,雖傳名號,憲章禮樂,
 寂滅無聞。劉裕平姚,收其圖籍,五經子史,才四千卷,皆赤軸青紙,文字古拙。僭偽之盛,莫過二秦,以此而論,足可明矣。故知衣冠軌物,圖畫記注,播遷之餘,皆歸江左。晉、宋之際,學藝為多,齊、梁之間,經史彌盛。宋秘書丞王儉,依劉氏《七略》,撰為《七志》。梁人阮孝緒,亦為《七錄》。總其書數,三萬餘卷。及侯景渡江,破滅梁室,秘省經籍,雖從兵火,其文德殿內書史,宛然猶存。蕭繹據有江陵,遣將破平侯景,收文德之書,及公私典籍,重本七萬餘卷,悉送荊州。故江表圖書,因斯盡萃於繹矣。及周師入郢,繹悉焚之於外城,所收十才一二。此則書之五厄也。後魏
 爰自幽方,遷宅伊、洛,日不暇給,經籍闕如。周氏創基關右,戎車未息。保定之始,書止八千,後加收集,方盈萬卷。高氏據有山東,初亦採訪,驗其本目,殘缺猶多。及東夏初平,獲其經史,四部重雜,三萬餘卷。所益舊書,五千而已。今御書單本,合一萬五千餘卷,部帙之間,仍有殘缺。比梁之舊目,止有其半。至於陰陽河洛之篇,醫方圖譜之說,彌復為少。臣以經書自仲尼已後,迄於當今,年逾千載,數遭五厄,興集之期,屬膺聖世。伏惟陛下受天明命,君臨區宇,功無與二,德冠往初。自華夏分離,彞倫攸斁,其間雖霸王遞起,而世難未夷,欲崇儒業,時或未可。
 今土宇邁於三王,民黎盛於兩漢,有人有時,正在今日。方當大弘文教,納俗升平,而天下圖書,尚有遺逸,非所以仰協聖情,流訓無窮者也。臣史籍是司,寢興懷懼。昔陸賈奏漢祖云「天下不可馬上治之」,故知經邦立政,在於典謨矣。為國之本,莫此攸先。今秘藏見書,亦足披覽,但一時載籍,須令大備。不可王府所無,私家乃有。然士民殷雜,求訪難知,縱有知者,多懷吝惜,必須勒之以天威,引之以微利。若猥發明詔,兼開購賞,則異典必臻,觀閣斯積,重道之風,超於前世,不亦善乎!伏願天鑒,少垂照察。



 上納之,於是下詔:獻書一卷,賚縑一匹。一二年間,
 篇籍稍備。進爵奇章郡公,邑千五百戶。



 三年,拜禮部尚書,奉敕修撰《五禮》,勒成百卷,行於當世。弘請依古制修立明堂,上議曰:竊謂明堂者,所以通神靈,感天地,出教化,崇有德。《孝經》曰:「宗祀文王於明堂,以配上帝。」《祭義》云:「祀於明堂,教諸侯孝也。」黃帝曰合宮,堯曰五府,舜曰總章,布政興治,由來尚矣。《周官·考工記》曰:「夏后氏世室,堂修二七,廣四修一。」鄭玄注云:「修十四步,其廣益以四分修之一,則堂廣十七步半也。」「殷人重屋,堂修七尋,四阿重屋。」鄭云:「其修七尋,廣九尋也。」



 「周人明堂,度九尺之筵,南北七筵,五室,凡室二筵。」鄭云:「此三者,或舉宗廟,或舉
 王寢,或舉明堂,互言之,明其同制也。」馬融、王肅、干寶所注,與鄭亦異,今不具出。漢司徒馬宮議云:「夏后氏世室,室顯於堂,故命以室。殷人重屋,屋顯於堂,故命以屋。周人明堂,堂大於夏室,故命以堂。夏后氏益其堂之廣百四十四尺,周人明堂,以為兩序間大夏后氏七十二尺。」若據鄭玄之說,則夏室大於周堂,如依馬宮之言,則周堂大於夏室。後王轉文,周大為是。但宮之所言,未詳其義。此皆去聖久遠,禮文殘缺,先儒解說,家異人殊。鄭注《玉藻》亦云:「宗廟路寢,與明堂同制。」《王制》曰:「寢不逾廟。」明大小是同。今依鄭玄注,每室及堂,止有一丈八尺,四壁
 之外,四尺有餘。若以宗廟論之,祫享之時,周人旅酬六尸,並后稷為七,先公昭穆二尸,先王昭穆二尸,合十一尸,三十六主,及君北面行事於二丈之堂,愚不及此。若以正寢論之,例須朝宴。據《燕禮》:「諸侯宴,則賓及卿大夫脫屨升坐。」是知天子宴,則三公九卿並須升堂。《燕義》又云:「席,小卿次上卿。」言皆侍席。止於二筵之間,豈得行禮?若以明堂論之,總享之時,五帝各於其室。設青帝之位,須於木室之內,少北西面。太昊從食,坐於其西,近南北面。祖宗配享者,又於青帝之南,稍退西面。丈八之室,神位有三,加以簠簋籩豆,牛羊之俎,四海九州美物咸設,
 復須席上升歌,出樽反坫,揖讓升降,亦以隘矣。據茲而說,近是不然。



 案劉向《別錄》及馬宮、蔡邕等所見,當時有《古文明堂禮》、《王居明堂禮》、《明堂圖》、《明堂大圖》、《明堂陰陽》、《太山通義》、《魏文侯孝經傳》等,並說古明堂之事。其書皆亡,莫得而正。今《明堂月令》者,鄭玄云:「是呂不韋著,《春秋十二紀》之首章,禮家鈔合為記。」蔡邕、王肅云:「周公所作《周書》內有《月令》第五十三,即此也。各有證明,文多不載。束皙以為夏時之書。」劉獻云:「不韋鳩集儒者,尋於聖王月令之事而記之。不韋安能獨為此記?」今案不得全稱《周書》,亦未可即為秦典,其內雜有虞、夏、殷、周之法,皆聖
 王仁恕之政也。蔡邕具為章句,又論之曰:「明堂者,所以宗祀其祖以配上帝也。夏后氏曰世室,殷人曰重屋,周人曰明堂。東曰青陽,南曰明堂,西曰總章,北曰玄堂,內曰太室。聖人南面而聽,向明而治,人君之位莫不正焉。故雖有五名,而主以明堂也。制度之數,各有所依。堂方一百四十四尺,坤之策也,屋圓楣徑二百一十六尺,乾之策也。太廟明堂方六丈,通天屋徑九丈,陰陽九六之變,且圓蓋方覆,九六之道也。八闥以象卦,九室以象州,十二宮以應日辰。三十六戶,七十二牖,以四戶八牖乘九宮之數也。戶皆外設而不閉,示天下以不藏也。通天
 屋高八十一尺,黃鐘九九之實也。二十八柱布四方,四方七宿之象也。堂高三尺,以應三統,四向五色,各象其行。水闊二十四丈,象二十四氣,於外以象四海。王者之大禮也。」觀其模範天地,則象陰陽,必據古文,義不虛出。今若直取《考工》,不參《月令》,青陽總章之號不得而稱,九月享帝之禮不得而用。漢代二京所建,與此說悉同。



 建安之後,海內大亂,京邑焚燒,憲章泯絕。魏氏三方未平,無聞興造。晉則侍中裴頠議曰:「尊祖配天,其義明著,而廟宇之制,理據未分。宜可直為一殿,以崇嚴父之祀,其餘雜碎,一皆除之。」宋、齊已還,咸率茲禮。此乃世之通儒,
 時無思術,前王盛事,於是不行。後魏代都所造,出自李沖,三三相重,合為九室。



 簷不覆基,房間通街,穿鑿處多,迄無可取。及遷宅洛陽,更加營構,五九紛競,遂至不成,宗配之事,於焉靡托。



 今皇猷遐闡,化覃海外,方建大禮,垂之無窮。弘等不以庸虛,謬當議限。今檢明堂必須五室者何?《尚書帝命驗》曰:「帝者承天立五府,赤曰文祖,黃曰神鬥,白曰顯紀,黑曰玄矩,蒼曰靈府。」鄭玄注曰:「五府與周之明堂同矣。」且三代相沿,多有損益,至於五室,確然不變。夫室以祭天,天實有五,若立九室,四無所用。布政視朔,自依其辰。鄭司農云:「十二月分在青陽等左右
 之位。」不云居室。鄭玄亦言:「每月於其時之堂而聽政焉。」《禮圖》畫個,皆在堂偏,是以須為五室。明堂必須上圓下方者何?《孝經援神契》曰:「明堂者,上圓下方,八窗四達,布政之宮。」《禮記·盛德篇》曰:「明堂四戶八牖,上圓下方。」



 《五經異義》稱講學大夫淳于登亦云:「上圓下方。」鄭玄同之。是以須為圓方。



 明堂必須重屋者何?案《考工記》,夏言「九階,四旁兩夾窗,門堂三之二,室三之一。」殷、周不言者,明一同夏制。殷言「四阿重屋」,周承其後不言屋,制亦盡同可知也。」其「殷人重屋」之下,本無五室之文,鄭注云:「五室者,亦據夏以知之。」明周不云重屋,因殷則有,灼然可見。《禮
 記·明堂位》曰:「太廟天子明堂。」言魯為周公之故,得用天子禮樂,魯之太廟與周之明堂同。又曰:「復廟重簷,刮楹達向,天子之廟飾。」鄭注:「復廟,重屋也。」據廟既重屋,明堂亦不疑矣。《春秋》文公十三年:「太室屋壞。」《五行志》曰:「前堂曰太廟,中央曰太室,屋其上重者也。」服虔亦云:「太室,太廟太室之上屋也。」《周書·作洛篇》曰:「乃立太廟宗宮路寢明堂,咸有四阿反坫,重亢重廊。」孔晁注曰:「重亢累棟,重廊累屋也。」依《黃圖》所載,漢之宗廟皆為重屋。此去古猶近,遺法尚在,是以須為重屋。明堂必須為闢雍者何?《禮記·盛德篇》云:「明堂者,明諸侯尊卑也。外水曰闢雍。」《明堂
 陰陽錄》曰:「明堂之制,周圜行水,左旋以象天,內有太室以象紫宮。」此明堂有水之明文也。然馬宮、王肅以為明堂、闢雍、太學同處,蔡邕、盧植亦以為明堂、靈臺、闢雍、太學同實異名。邕云:「明堂者,取其宗祀之清貌,則謂之清廟,取其正室,則曰太室,取其堂,則曰明堂,取其四門之學,則曰太學,取其周水圜如璧,則曰璧雍。其實一也。」其言別者,《五經通義》曰:「靈臺以望氣,明堂以布政,闢雍以養老教學。」三者不同。袁準、鄭玄亦以為別。歷代所疑,豈能輒定?今據《郊祀志》云:「欲治明堂,未曉其制。濟南人公玉帶上黃帝時《明堂圖》,一殿無壁,蓋之以茅,水圜宮垣,
 天子從之。」以此而言,其來則久。漢中元二年,起明堂、闢雍、靈臺於洛陽,並別處。



 然明堂亦有壁水,李尤《明堂銘》云「流水洋洋」是也。以此須有闢雍。



 夫帝王作事,必師古昔,今造明堂,須以《禮經》為本。形制依於周法,度數取於《月令》,遺闕之處,參以餘書,庶使該詳沿革之理。其五室九階,上圓下方,四阿重屋,四旁兩門,依《考工記》、《孝經》說。堂方一百四十四尺,屋圓楣徑二百一十六尺,太室方六丈,通天屋徑九丈,八達二十八柱,堂高三尺,四向五色,依《周書·月令》論。殿垣方在內,水周如外,水內徑三百步,依《太山盛德記》、《覲禮經》。仰觀俯察,皆有則象,足以盡
 誠上帝,祗配祖宗,弘風布教,作範於後矣。弘等學不稽古,輒申所見,可否之宜,伏聽裁擇。



 上以時事草創,未遑制作,竟寢不行。



 六年,除太常卿。九年,詔改定雅樂,又作樂府歌詞,撰定圓丘五帝凱樂,並議樂事。弘上議云:謹案《禮》,五聲、六律、十二管還相為宮。《周禮》奏黃鐘,歌大呂,奏太簇,歌應鐘,皆是旋相為宮之義。蔡邕《明堂月令章句》曰:「孟春月則太簇為宮,姑洗為商,蕤賓為角,南呂為徵,應鐘為羽,大呂為變宮,夷則為變徵。他月放此。」



 故先王之作律呂也,所以辯天地四方陰陽之聲。揚子雲曰:「聲生於律,律生於辰。」



 故律呂配五行,通八風,歷十二辰,
 行十二月,循環轉運,義無停止。譬如立春木王火相,立夏火王土相,季夏餘分,土王金相,立秋金王水相,立冬水王木相。還相為宮者,謂當其王月,名之為宮。今若十一月不以黃鐘為宮,十三月不以太簇為宮,便是春木不王,夏王不相,豈不陰陽失度,天地不通哉?劉歆《鐘律書》云:「春宮秋律,百卉必凋;秋宮春律,萬物必榮;夏宮冬律,雨雹必降;冬宮夏律,雷必發聲。」以斯而論,誠為不易。且律十二,今直為黃鐘一均,唯用七律,以外五律,竟復何施?恐失聖人制作本意。故須依禮作還相為宮之法。



 上曰:「不須作旋相為宮,且作黃鐘一均也。」弘又論六十
 律不可行:謹案《續漢書·律歷志》,元帝遣韋玄成問京房於樂府,房對:「受學故小黃令焦延壽。六十律相生之法,以上生下,皆三生二,以下生上,皆三生四。陽下生陰,陰上生陽,終於中呂,而十二律畢矣。中呂上生執始,執始下生去滅,上下相生,終於南事,六十律畢矣。十二律之變至於六十,猶八卦之變至於六十四也,冬至之聲,以黃鐘為宮,太簇為商,姑洗為角,林鐘為徵,南呂為羽,應鐘為變宮,蕤賓為變徵。此聲氣之元,五音之正也。故各統一日。其餘以次運行,當日者各自為宮,而商徵以類從焉。」房又曰:「竹聲不可以度調,故作準以定數。準之狀
 如瑟,長一丈而十三弦,隱間九尺,以應黃鐘之律九寸。中央一弦,下畫分寸,以為六十律清濁之節。」執始之類,皆房自造。房云受法於焦延壽,未知延壽所承也。



 至元和年,待詔候鐘律殷肜上言:「官無曉六十律以準調音者。故待詔嚴崇具以準法教其子宣,願召宣補學官,主調樂器。」大史丞弘試宣十二律,其二中,其四不中,其六不知何律,宣遂罷。自此律家莫能為準施弦。熹平年,東觀召典律者太子舍人張光問準意。光等不知,歸閱舊藏,乃得其器,形制如房書,猶不能定其弦緩急,故史官能辨清濁者遂絕。其可以相傳者,唯大榷常數及候氣而
 已。據此而論,京房之法,漢世已不能行。沈約《宋志》曰:「詳案古典及今音家,六十律無施於樂。」《禮》云「十二管還相為宮」,不言六十。《封禪書》云:「大帝使素女鼓五十弦瑟而悲,破為二十五弦。」假令六十律為樂,得成亦所不用。取「大樂必易,大禮必簡」之意也。



 又議曰:案《周官》云:「大司樂掌成均之法。」鄭眾注云:「均,調也。樂師主調其音。」《三禮義宗》稱:「《周官》奏黃鐘者,用黃鐘為調,歌大呂者,用大呂為調。奏者謂堂下四懸,歌者謂堂上所歌。但一祭之間,皆用二調。」是知據宮稱調,其義一也。明六律六呂迭相為宮,各自為調。今見行之樂,用黃鐘之宮,乃以林鐘為調,
 與古典有違。晉內書監荀勖依典記,以五聲十二律還相為宮之法,制十二笛。



 黃鐘之笛,正聲應黃鐘,下徵應林鐘,以姑洗為清角。大呂之笛,正聲應大呂,下徵應夷則。以外諸均,例皆如是。然今所用林鐘,是勖下徵之調。不取其正,先用其下,於理未通,故須改之。



 上甚善其義,詔弘與姚察、許善心、何妥、虞世基等正定新樂,事在《音律志》。



 是後議置明堂,詔弘條上故事,議其得失,事在《禮志》。上甚敬重之。



 時楊素恃才矜貴,輕侮朝臣,唯見弘未當不改容自肅。素將擊突厥,詣太常與弘言別。弘送素至中門而止,素謂弘曰:「大將出征,故來敘別,何相送之
 近也?」



 弘遂揖而退。素笑曰:「奇章公可謂其智可及,其愚不可及也。」亦不以屑懷。



 尋授大將軍,拜吏部尚書。時高祖又令弘與楊素、蘇威、薛道衡、許善心、虞世基、崔子發等並召諸儒,論新禮降殺輕重。弘所立議,眾咸推服之。仁壽二年,獻皇后崩,三公已下不能定其儀注。楊素謂弘曰:「公舊學,時賢所仰,今日之事,決在於公。」弘了不辭讓,斯須之間,儀注悉備,皆有故實。素嘆曰:「衣冠禮樂,盡在此矣,非吾所及也!」弘以三年之喪,祥禫具有降殺,期服十一月而練者,無所象法,以聞於高祖,高祖納焉。下詔除期練之禮,自弘始也。弘在吏部,其選舉先德行而
 後文才,務在審慎。雖致停緩,所有進用,並多稱職。吏部侍郎高孝基,鑒賞機晤,清慎絕倫,然爽俊有餘,跡似輕薄,時宰多以此疑之。唯弘深識其真,推心委任。隋之選舉,於斯為最。時論彌服弘識度之遠。



 煬帝之在東宮也,數有詩書遺弘,弘亦有答。及嗣位之後,嘗賜弘詩曰:「晉家山吏部,魏世盧尚書,莫言先哲異,奇才並佐餘。學行敦時俗,道素乃沖虛,納言雲閣上,禮儀皇運初。彞倫欣有敘,垂拱事端居。」其同被賜詩者,至於文詞贊揚,無如弘美。大業二年,進位上大將軍。三年,改為右光祿大夫。從拜恆岳,壇場珪幣,墠畤牲牢,並弘所定。還下太行,煬
 帝嘗引入內帳,對皇后賜以同席飲食。



 其禮遇親重如此。弘謂其諸子曰:「吾受非常之遇,荷恩深重。汝等子孫,宜以誠敬自立,以答恩遇之隆也。」六年,從幸江都。其年十一月,卒於江都郡,時年六十六。帝傷惜之,贈甚厚。歸葬安定,贈開府儀同三司、光祿大夫、文安侯,謚曰憲。



 弘榮寵當世,而車服卑儉,事上盡禮,待下以仁,訥於言而敏於行。上嘗令其宣敕,弘至階下,不能言,退還拜謝,云:「並忘之。」上曰:「傳語小辯,故非宰臣任也。」愈稱其質直。大業之世,委遇彌隆。性寬厚,篤志於學,雖職務繁雜,書不釋手。隋室舊臣,始終信任,悔吝不及,唯弘一人而已。
 有弟曰弼,好酒而酗,嘗因醉,射殺弘駕車牛。弘來還宅,其妻迎謂之曰:「叔射殺牛矣。」弘聞之,無所怪問,直答云:「作脯。」坐定,其妻又曰:「叔忽射殺牛,大是異事!」弘曰:「已知之矣。」顏色自若,讀書不輟。其寬和如此。有文集十三卷行於世。



 長子方大,亦有學業,官至內史舍人。次子方裕,性兇險無人心,從幸江都,與裴虔通等同謀弒逆,事見《司馬德勘傳》。



 史臣曰:牛弘篤好墳籍,學優而仕,有淡雅之風,懷曠遠之度,採百王之損益,成一代之典章,漢之叔孫,不能尚也。綢繆省闥,三十餘年,夷險不渝,始終無際。



 雖開物成
 務,非其所長,然澄之不清,混之不濁,可謂大雅君子矣。子實不才,崇基不構,干紀犯義,以墜家風,惜哉!



\end{pinyinscope}