\article{卷四十六列傳第十一 趙煚}

\begin{pinyinscope}

 趙煚,字賢通,天水西人也。祖超宗,魏河東太守。父仲懿,尚書左丞。煚少孤,養母至孝。年十四,有人盜伐其父墓中樹者,煚對之號慟,因執送官。見魏右僕射周惠達,長揖不拜,自述孤苦,涕泗交集,惠達為之隕涕,嘆息者久之。及長,深沉有器局,略涉書史。周太祖引為相府參軍事。尋從破洛陽。及太祖班師,煚請留撫納亡叛,太祖從之。
 煚於是帥所領與齊人前後五戰,斬郡守、鎮將、縣令五人,虜獲甚眾,以功封平定縣男,邑三百戶。累轉中書侍郎。



 閔帝受禪,遷陜州刺史。蠻酋向天王聚眾作亂,以兵攻信陵、秭歸。煚勒所部五百人,出其不意,襲擊破之,二郡獲全。時周人於江南岸置安蜀城以御陳,屬霖雨數旬,城頹者百餘步。蠻酋鄭南鄉叛,引陳將吳明徹欲掩安蜀。議者皆觀煚益修守御,煚曰:「不然,吾自有以安之。」乃遣使說誘江外生蠻向武陽,令乘虛掩襲所居,獲其南鄉父母妻子。南鄉聞之,其黨各散,陳兵遂退。明年,吳明徹屢為寇患,煚勒兵御之,前後十六戰,每挫其鋒。獲
 陳裨將覃冏、王足子、吳朗等三人,斬首百六十級。以功授開府儀同三司,遷荊州總管長史。入為民部中大夫。



 武帝出兵鞏、洛,欲收齊河南之地。煚諫曰:「河南洛陽,四面受敵,縱得之,不可以守。請從河北,直指太原,傾其巢穴,可一舉以定。」帝不納,師竟無功。



 尋從上柱國於翼率眾數萬,自三鴉道以伐陳,克陳十九城而還。以讒毀,功不見錄,除益州總管長史。未幾,入為天官司會,累遷御正上大夫。煚與宗伯斛斯徵素不協,徵後出為齊州刺史,坐事下獄,自知罪重,遂逾獄而走。帝大怒,購之甚急。煚上密奏曰:「徵自以負罪深重,懼死遁逃,若不北竄匈
 奴,則南投吳越。徵雖愚陋,久歷清顯,奔彼敵國,無益聖朝。今者炎旱為災,可因茲大赦。」帝從之。徵賴而獲免,煚卒不言。



 高祖為丞相,加上開府,復拜天官都司會。俄遷大宗伯。及踐阼,煚授璽紱,進位大將軍,賜爵金城郡公,邑二千五百戶,拜相州刺史。朝廷以煚曉習故事,徵拜尚書右僕射。視事未幾,以忤旨,尋出為陜州刺史,俄轉冀州刺史,甚有威德。



 煚嘗有疾,百姓奔馳,爭為祈禱,其得民情如此。冀州谷薄,市井多奸詐,煚為銅斗鐵尺,置之於肆,百姓便之。上聞而嘉焉,頒告天下,以為常法。嘗有人盜煚田中蒿者,為吏所執。煚曰:「此乃刺史不能宣
 風化,彼何罪也。」慰諭而遣之,令人載蒿一車以賜盜者。盜者愧恧,過於重刑。其以德化民,皆此類也。上幸洛陽,煚來朝,上勞之曰:「冀州大籓,民用殷實,卿之為政,深副朕懷。」開皇十九年卒,時年六十八。子義臣嗣,官至太子洗馬。後同楊諒反,誅。



 趙芬趙芬,字士茂,天水西人也。父演,周秦州刺史。芬少有辯智,頗涉經史。周太祖引為相府鎧曹參軍,歷記室,累遷熊州刺史。撫納降附,得二千戶,加開府儀同三司。大塚宰宇文護召為中外府掾,俄遷吏部下大夫。芬性強濟,
 所居之職,皆有聲績。武帝親總萬機,拜內史下大夫,轉少御正。芬明習故事,每朝廷有所疑議,眾不能決者,芬輒為評斷,莫不稱善。後為司會,申國公李穆之討齊也,引為行軍長史,封淮安縣男,邑五百戶。復出為淅州刺史,轉東京小宗伯,鎮洛陽。



 高祖為丞相,尉迥與司馬消難陰謀往來,芬察知之,密白高祖。由是深見親委,遷東京左僕射,進爵郡公。開皇初,罷東京官,拜尚書左僕射,與郢國公王誼修律令。俄兼內史令,上甚信任之。未幾,以老病出拜蒲州刺史,加金紫光祿大夫,仍領關東運漕,賜錢百萬、粟五千石而遣之。後數年,上表乞骸骨,徵
 還京師,賜以二馬軺車,幾杖被褥,歸於家,皇太子又致巾帔。後數年,卒。上遣使致祭,鴻臚監護喪事。



 子元恪嗣,官至揚州總管司馬,左遷候衛長史。少子元楷,與元恪皆明幹世事。



 元楷大業中為歷陽郡丞,與廬江郡丞徐仲宗皆竭百姓之產以貢於帝。仲宗遷南郡丞,元楷超拜江都郡丞,兼領江都宮使。



 楊尚希楊尚希,弘農人也。祖真,魏天水太守。父承賓,商、直、淅三州刺史。尚希齠齔而孤。年十一,辭母請受業長安。涿郡盧辯見而異之,令入太學,專精不倦,同輩皆共推伏。周
 太祖嘗親臨釋奠,尚希時年十八,令講《孝經》,詞旨可觀。太祖奇之,賜姓普六茹氏,擢為國子博士。累轉舍人。仕明、武世,歷太學博士、太子宮尹、計部中大夫,賜爵高都縣侯,東京司憲中大夫。宣帝時,令尚希撫慰山東、河北,至相州而帝崩,與相州總管尉迥發喪於館。尚希出謂左右曰:「蜀公哭不哀而視不安,將有他計。吾不去,將及於難。」遂夜中從捷徑而遁。遲明,迥方覺,分數十騎自驛路追之,不及,遂歸京師。高祖以尚希宗室之望,又背迥而至,待之甚厚。及迥屯兵武陟,遣尚希督宗室兵三千人鎮潼關。尋授司會中大夫。



 高祖受禪,拜度支尚書,進
 爵為公。歲餘,出為河南道行臺兵部尚書,加銀青光祿大夫。尚希時見天下州郡過多,上表曰:「自秦並天下,罷侯置守,漢、魏及晉,邦邑屢改。竊見當今郡縣,倍多於古,或地無百里,數縣並置,或戶不滿千,二郡分領。具僚以眾,資費日多;吏卒人倍,租調歲減。清幹良才,百分無一,動須數萬,如何可覓?所謂民少官多,十羊九牧。琴有更張之義,瑟無膠柱之理。今存要去閑,並小為大,國家則不虧粟帛,選舉則易得賢才,敢陳管見,伏聽裁處。」



 帝覽而嘉之,於是遂罷天下諸郡。尋拜瀛州刺史,未之官,奉詔巡省淮南。還除兵部尚書。俄轉禮部尚書,授上儀同。



 尚希性弘厚,兼以學業自通,甚有雅望,為朝廷所重。上時每旦臨朝,日側不倦,尚希諫曰:「周文王以憂勤損壽,武王以安樂延年。願陛下舉大綱,責成宰輔,繁碎之務,非人主所宜親也。」上歡然曰:「公愛我者。」尚希素有足疾,上謂之曰:「蒲州出美酒,足堪養病,屈公臥治之。」於是出拜蒲州刺史,仍領本州宗團驃騎。尚希在州,甚有惠政,復引瀵水,立堤防,開稻田數千頃,民賴其利。開皇十年卒官,時年五十七。謚曰平。子旻嗣,後改封丹水縣公,官至安定郡丞。



 長孫平
 長孫平,字處均,河南洛陽人也。父儉,周柱國。平美容儀,有器幹,頗覽書記。仕周,釋褐衛王侍讀。時武帝逼於宇文護,謀與衛王誅之,王前後常使平往來通意於帝。及護伏誅,拜開府、樂部大夫。宣帝即位,置東京官屬,以平為小司寇,與小宗伯趙芬分掌六府。高祖龍潛時,與平情好款洽,及為丞相,恩禮彌厚。尉迥、王謙、司馬消難並稱兵內侮,高祖深以淮南為意。時賀若弼鎮壽陽,恐其懷二心,遣平馳驛往代之。弼果不從,平麾壯士執弼,送於京師。



 開皇三年,徵拜度支尚書。平見天下州縣多罹水旱,百姓不給,奏令民間每秋家出粟麥一石已下,貧
 富差等,儲之閭巷,以備兇年,各曰義倉。因上書曰:「臣聞國以民為本,民以食為命,勸農重谷,先王令軌。古者三年耕而餘一年之積,九年作而有三年之儲,雖水旱為災,而民無菜色,皆由勸導有方,蓄積先備者也。去年亢陽,關右饑餒,陛下運山東之粟,置常平之官,開發倉廩,普加賑賜,大德鴻恩,可謂至矣。然經國之道,義資遠算,請勒諸州刺史、縣令,以勸農積穀為務。



 「上深嘉納。自是州里豐衍,民多賴焉。



 後數載,轉工部尚書,名為稱職。時有人告大都督邴紹非毀朝廷為憒憒者,上怒,將斬之。平進諫曰:「川澤納污,所以成其深;山嶽藏疾,所以就其
 大。臣不勝至願,願陛下弘山海之量,茂寬裕之德。鄙諺曰:『不癡不聾,未堪作大家翁。』此言雖小,可以喻大。邴紹之言,不應聞奏,陛下又復誅之,臣恐百代之後,有虧聖德。」上於是赦紹。因敕群臣,誹謗之罪,勿復以聞。



 其後突厥達頭可汗與都藍可汗相攻,各遣使請授。上使平持節宣諭,令其和解,賜縑三百匹,良馬一匹而遣之。平至突厥所,為陳利害,遂各解兵。可汗贈平馬二百匹。及還,平進所得馬,上盡以賜之。未幾,遇譴,以尚書檢校汴州事。歲餘,除汴州刺史。其後歷許、貝二州,俱有善政。鄴都俗薄,舊號難治,前後刺史多不稱職。朝廷以平所在善
 稱,轉相州刺史。甚有能名。在州數年,會正月十五日,百姓大戲,畫衣裳為鍪甲之象,上怒而免之。俄而念平鎮淮南時事,進位大將軍,拜太常卿,判吏部尚書事。仁壽中卒官。謚曰康。



 子師孝,性輕狡好利,數犯法。上以其不克負荷,遣使吊平國官。師孝後為渤海郡主簿,屬大業之季,政教陵遲,師孝恣行貪濁,一郡苦之。後為王世充所害。



 元軍元暉,字叔平,河南洛陽人也。祖琛,魏恆、朔二州刺史。父翌,尚書左僕射。



 暉須眉如畫,進止可觀,頗好學,涉獵書
 記。少得美名於京下,周太祖見而禮之,命與諸子游處,每同席共硯,情契甚厚。弱冠,召補相府中兵參軍,尋遷武伯下大夫。於時突厥屢為寇患,朝廷將結和親,令暉齎錦彩十萬,使於突厥。暉說以利害,申國厚禮,可汗大悅,遣其名王隨獻方物。俄拜儀同三司、賓部下大夫。保定初,大塚宰宇文護引為長史,會齊人來結盟好,以暉多才辯,與千乘公崔睦俱使於齊。



 遷振威中大夫。武帝之娉突厥後也,令暉致禮焉。加開府,轉司憲大夫。及平關東,使暉安集河北,封義寧子,邑四百戶。



 高祖總百揆,加上開府,進爵為公。開皇初,拜都官尚書,兼領太僕。奏
 請決杜陽水灌三畤原,溉舄鹵之地數千頃,民賴其利。明年,轉左武候將軍,太僕卿如故。尋轉兵部尚書,監漕渠之役。未幾,坐事免。頃之,拜魏州刺史,頗有惠政。



 在任數年,以疾去職。歲餘,卒於京師,時年六十。上嗟悼久之,敕鴻臚監護喪事。



 謚曰元。子肅嗣,官至光祿少卿。肅弟仁器,性明敏,官至日南郡丞。



 韋師韋師,字公穎,京兆杜陵人也。父瑱,周驃騎大將軍。師少沉謹,有至性。初就學,始讀《孝經》,舍書而嘆曰:「名教之極,其在茲乎!」少丁父母憂,居喪盡禮,州里稱其孝行。及長,
 略涉經史,尤工騎射。周大塚宰宇文護引為中外府記室,轉賓曹參軍。師雅知諸蕃風俗及山川險易,其有夷狄朝貢,師必接對,論其國俗,如視諸掌。夷人驚服,無敢隱情。齊王憲為雍州牧,引為主簿,本官如故。及武帝親總萬機,轉少府大夫。及平高氏,詔師安撫山東,徙為賓部大夫。



 高祖受禪,拜使部侍郎,賜爵井陘侯,邑五百戶。數年,遷河北道行臺兵部尚書,詔為山東河南十八州安撫大使。奏事稱旨,賜錢三百萬,兼領晉王廣司馬。其族人世康,為吏部尚書,與師素懷勝負。於時晉王為雍州牧,盛存望第,以司空楊雄、尚書左僕射高熲並為州
 都督,引師為主簿。而世康弟世約為法曹從事。世康恚恨不能食,又恥世約在師之下,召世約數之曰:「汝何故為從事?」遂杖之。



 後從上幸醴泉宮,上召師與左僕射高熲、上柱國韓擒等,於臥內賜宴,令各敘舊事,以為笑樂。平陳之役,以本官領元帥掾,陳國府藏,悉委於師,秋毫無所犯,稱為清白。後上為長寧王儼納其女為妃。除汴州刺史,甚有治名,卒官。謚曰定。



 子德政嗣,大業中,仕至給事郎。



 楊異楊異,字文殊,弘農華陰人也。祖鈞,魏司空。父儉,侍中。異
 美風儀,沉深有器局。髫齔就學,日誦千言,見者奇之。九歲丁父憂,哀毀過禮,殆將滅性。及免喪之後,絕慶吊,閉戶讀書。數年之間,博涉書記。周閔帝時,為寧都太守,甚有能名。賜爵昌樂縣子。後數以軍功,進為侯。高祖作相,行濟州事。及踐阼,拜宗正少卿,加上開府。蜀王秀之鎮益州也,朝廷盛選綱紀,以異方直,拜益州總管長史,賜錢二十萬、縑三百匹、馬五十匹而遣之。尋遷西南道行臺兵部尚書。數載,復為宗正少卿。未幾,擢拜刑部尚書。歲餘,出除吳州總管,甚有能名。時晉王廣鎮揚州,詔令異每歲一與王相見,評論得失,規諷疑闕。數載,卒官,時
 年六十二。



 子虔遜。



 蘇孝慈兄子沙羅蘇孝慈,扶風人也。父武周,周兗州刺史。孝慈少沉謹,有器幹,美容儀。周初為中侍上士。後拜都督,聘於齊,以奉使稱旨,遷大都督。其年又聘於齊,還授宣納上士。後從武帝伐齊,以功進位開府,賜爵文安縣公,邑千五百戶。尋改封臨水縣公,增邑千二百戶,累遷工部上大夫。



 高祖受禪,進爵安平郡公,拜太府卿。於時王業初基,百度伊始,徵天下工匠,纖微之巧,無不畢集。孝慈總其事,世以為能。俄遷大司農,歲餘,拜兵部尚書,待遇逾密。時皇
 太子勇頗知時政,上欲重宮官之望,多令大臣領其職。於是拜孝慈為太子右衛率,尚書如故。明年,上於陜州置常平倉,轉輸京下。以渭水多沙,流乍深乍淺,漕運者苦之,於是決謂水為渠以屬河,令孝慈督其役。渠成,上善之。



 又領太子右庶子,轉授左衛率,仍判工部、民部二尚書,稱為幹理。數載,進位大將軍,轉工部尚書,率如故。先是,以百僚供費不足,臺省府寺咸置廨錢,收息取給。孝慈以為官民爭利,非興化之道,上表請罷之,請公卿以下給職田各有差,上並嘉納焉。開皇十八年,將廢太子,憚其在東宮,出為淅州刺史。太子以孝慈去,甚不平,
 形於言色。其見重如此。仁壽初,遷洪州總管,俱有惠政。共後桂林山越相聚為亂,詔孝慈為行軍總管,擊平之。其年卒官。有子會昌。



 孝慈兄子沙羅,字子粹。父順,周眉州刺史。沙羅仕周,釋褐都督。後從韋孝寬破尉迥,以功授開府儀同三司,封通秦縣公。開皇初,蜀王秀鎮益州,沙羅以本官從,拜資州刺史。八年,冉尨羌作亂,攻汶山、金川二鎮,沙羅率兵擊破之,授邛州刺史。後數載,檢校利州總管事。從史萬歲擊西爨,累戰有功,進位大將軍,賜物千段。尋檢校益州總管長史。會越歸人王奉舉兵作亂,沙羅從段文振討平之,賜奴婢百口。會蜀王秀廢,
 吏案奏沙羅云:「王奉為奴所殺,秀乃詐稱左右斬之。



 又調熟獠,令出奴婢,沙羅隱而不奏。」由是除名,卒於家。有子康。



 李雄李雄,字毗盧,趙郡高邑人也。祖榼,魏太中大夫。父徽伯,齊陜州刺史,陷於周,雄因隨軍入長安。雄少慷慨,有大志。家世並以學業自通,雄獨習騎射。其兄子旦讓之曰:「棄文尚武,非士大夫之素業。」雄答曰:「竊覽自古誠臣貴仕,文武不備而能濟其功業者鮮矣。雄雖不敏,頗觀前志,但不守章句耳。既文且武,兄何病焉!」子旦無以應之。



 周太祖時,釋褐輔國將軍。從達奚武平漢中,定興州,又討汾州叛胡,錄前後功,拜驃騎大將軍、儀同三司。閔帝受禪,進爵為公,遷小賓部。其後復從達奚武與齊人戰於芒山,諸軍大敗,雄所領獨全。武帝時,從陳王純迎後於突厥,進爵奚伯,拜硤州刺史。數歲,徵為本府中大夫。尋出為涼州總管長史。從滕王逌破吐谷渾於青海,以功加上儀同。宣帝嗣位,從行軍總管韋孝寬略定淮南。雄以輕騎數百至硤石,說下十餘城,拜豪州刺史。



 高祖總百揆,徵為司會中大夫。以淮南之功,加位上開府。及受禪,拜鴻臚卿,進爵高都郡公,食邑二千戶。後數年,晉王
 廣出鎮並州,以雄為河北行臺兵部尚書。



 上謂雄曰:「吾兒既少,更事未多,以卿兼文武才,今推誠相委,吾無北顧之憂矣。」



 雄頓首而言曰:「陛下不以臣之不肖,寄臣以重任。臣雖愚固,心非木石,謹當竭誠效命,以答鴻恩。」歔欷流涕,上慰諭而遣之。雄當官正直,侃然有不可犯之色,王甚敬憚,吏民稱焉。歲餘,卒官。子公挺嗣。



 張煚劉仁恩郭均馮世基厙狄颭張煚,字士鴻,河間鄚人也。父羨,少好學,多所通涉,仕魏為蕩難將軍。從武帝入關,累遷銀青光祿大夫。周太祖引為從事中郎,賜姓叱羅氏。歷司職大夫,雍州治中、雍
 州刺史、儀同三司,賜爵虞鄉縣公。復入為司成中大夫,典國史。周代公卿,類多武將,唯羨以素業自通,甚為當時所重。後以年老,致仕於家。及高祖受禪,欽其德望,以書征之曰:「朕初臨四海,思存政術,舊齒名賢,實懷勤佇。



 儀同昔在周室,德業有聞,雖云致仕,猶克壯年。即宜入朝,用副虛想。」及謁見,敕令勿拜,扶升殿,上降榻執手,與之同坐,宴語久之,賜以幾杖。會遷都龍首,羨上表勸以儉約,上優詔答之。俄而卒,時年八十四。贈滄州刺史,謚曰定。撰《老子》、《莊子》義,名曰《道言》,五十二篇。



 煚好學,有父風。在魏釋褐奉朝請,遷員外侍郎。周太祖引為外兵曹。
 閔帝受禪,加前將軍。明、武世,歷膳部大夫、塚宰司錄,賜爵北平縣子,邑四百戶。宣帝時,加儀同,進爵為伯。高祖為丞相,煚深自推結,高祖以其有乾用,甚親遇之。



 及受禪,拜尚書右丞,進爵為侯。俄遷太府少卿,領營新都監丞。丁父憂去職,柴毀骨立。未期,起令視事,固讓不許,授儀同三司,襲爵虞鄉縣公,增邑通前千五百戶。尋遷太府卿,拜民部尚書。晉王廣為揚州總管,授煚司馬,加銀青光祿大夫。



 煚性和厚,有識度,甚有當時之譽。後拜冀州刺史,晉王廣頻表請之,復為晉王長史,檢校蔣州事。及晉王為皇太子,復為冀州刺史,進位上開府,吏民悅
 服,稱為良二千石。仁壽四年卒官,時年七十四。子慧寶,官至絳郡丞。



 開皇時有劉仁恩者,不知何許人也,倜儻有文武干用。初為毛州刺史,治績號天下第一,擢拜刑部尚書。又以行軍總管從楊素伐陳,與素破陳將呂仲肅於荊門,仁恩之計居多,授上大將軍,甚有當時之譽。馮翊郭均、上黨馮世基,並明悟有幹略,相繼為兵部尚書。代人厙狄嶔,性弘厚,有局度,官至民部尚書。此四人俱顯名於當世,然事行闕落,史莫能詳。



 史臣曰:二趙明習故事,當世所推,及居端右,無聞殊績。固知人之才器,各有分限,大小異宜,不可逾量。長孫平
 諫赦誹謗之罪,可謂仁人之言,高祖悅而從之,其利亦已博矣。元暉以明敏顯達,韋師以清白成名,楊尚希、楊異,宗室之英,譽望隆重,蘇孝慈、李雄、張煚,內外所履,咸稱貞幹,並任開皇之初,蓋當時之選也。



\end{pinyinscope}