\article{卷四十七列傳第十二 韋世康(弟洸 藝 沖 從父弟壽)}

\begin{pinyinscope}

 韋
 世康,京兆杜陵人也,世為關右著姓。祖旭,魏南幽州刺史。父夐,隱居不仕,魏、周二代,十徵不出,號為逍遙公。世康幼而沉敏,有器度。年十歲,州闢主簿。在魏,弱冠為直寢,封漢安縣公,尚周文帝女襄樂公主,授儀同三司。後仕周,自典祠下大夫歷沔、硤二州刺史。從武帝平齊,授司州總管長史。於時東夏初定,百姓未安,世康綏撫
 之,士民胥悅。歲餘,入為民部中大夫,進位上開府,轉司會中大夫。



 尉迥之作亂也,高祖憂之,謂世康曰:「汾、絳舊是周、齊分界,因此亂階,恐生搖動。今以委公,善為吾守。」因授絳州刺史,以雅望鎮之,闔境清肅。世康性恬素好古,不以得喪干懷。在州嘗慨然有止足之志,與子弟書曰:「吾生因緒餘,夙沾纓弁,驅馳不已,四紀於茲。亟登袞命,頻蒞方岳,志除三惑,心慎四知,以不貪而為寶,處膏脂而莫潤。如斯之事,頗為時悉。今耄雖未及,壯年已謝,霜早梧楸,風先蒲柳。眼暗更劇,不見細書,足疾彌增,非可趨走。祿豈須多,防滿則退,年不待暮,有疾便辭。況娘
 春秋已高,溫清宜奉,晨昏有闕,罪在我躬。今世穆、世文並從戎役,吾與世沖復嬰遠任,陟岵瞻望,此情彌切,桓山之悲,倍深常戀。意欲上聞,乞遵養禮,未訪汝等,故遣此及。興言遠慕,感咽難勝。」諸弟報以事恐難遂,於是乃止。



 在任數年,有惠政,奏課連最,擢為禮部尚書。世康寡嗜欲,不慕貴勢,未嘗以位望矜物。聞人之善,若己有之,亦不顯人過咎,以求名譽。尋進爵上庸郡公,加邑至二千五百戶。其年轉吏部尚書,餘官如故。四年,丁母憂去職。未期,起令視事。世康固請,乞終私制,上不許。世康之在吏部,選用平允,請托不行。開皇七年,將事江南,議重
 方鎮,拜襄州刺史。坐事免。未幾,授安州總管,尋遷為信州總管。十三年,入朝,復拜吏部尚書。前後十餘年間,多所進拔,朝廷稱為廉平。



 嘗因休暇,謂子弟曰:「吾聞功遂身退,古人常道。今年將耳順,志在懸車,汝輩以為雲何?」子福嗣答曰:「大人澡身浴德,名立官成,盈滿之誠,先哲所重。欲追蹤二疏,伏奉尊命。」後因侍宴,世康再拜陳讓曰:「臣無尺寸之功,位亞臺鉉。



 今犬馬齒濆,不益明時,恐先朝露,無以塞責。願乞骸骨,退避賢能。」上曰:「朕夙夜庶幾,求賢若渴,冀與公共治天下,以致太平。今之所請,深乖本望,縱令筋骨衰謝,猶屈公臥治一隅。」於是出拜荊
 州總管。時天下唯置四大總管,並、揚、益三州,並親王臨統,唯荊州委於世康,時論以為美。世康為政簡靜,百姓愛悅,合境無訟。十七年,卒於州,時年六十七。上聞而痛惜之,贈賻甚厚。贈大將軍,謚曰文。



 世康性孝友,初以諸弟位並隆貴,獨季弟世約宦途不達,共推父時田宅盡以與之,世多其義。



 長子福子,官至司隸別駕。次子福嗣,仕至內史舍人,後以罪黜。楊玄感之作亂也,以兵逼東都,福嗣從衛玄戰於城北,軍敗,為玄感所擒,令作文檄,辭甚不遜。尋背玄感還東都,帝銜之不已,車裂於高陽。少子福獎,通事舍人,在東都與玄感戰沒。



 洸字世穆,性剛毅,有器幹,少便弓馬。仕周,釋褐直寢上士。數從征伐,累遷開府,賜爵衛國縣公,邑千二百戶。高祖為丞相,從季父孝寬擊尉迥於相州,以功拜柱國,進封襄陽郡公,邑二千戶。時突厥寇邊,皇太子屯咸陽,令洸統兵出原州道,與虜相遇,擊破之。尋拜江陵總管。未幾,以母疾徵還。俄拜安州總管。伐陳之役,領行軍總管。及陳平,拜江州總管,率步騎二萬,略定九江。陳豫章太守徐璒據郡持兩端,洸遣開府呂昂、長史馮世基以兵相繼而進。既至城下,璒偽降,其夜率所部二千人襲擊昂。昂與世基合擊,大破之,擒璒於陣。高梁女子洗氏率
 眾迎洸,遂進圖嶺南。上遣洸書曰:「公鴻勛大業,名高望重,率將戎旅,撫慰彼方,風行電掃,咸應稽服。若使干戈不用,兆庶獲安,方副朕懷,是公之力。」至廣州,說陳渝州都督王猛下之,嶺表皆定。上聞而大悅,許以便宜從事。洸所綏集二十四州,拜廣州總管。歲餘,番禺夷王仲宣聚眾為亂,以兵圍洸,洸勒兵拒之,中流矢而卒。贈上柱國,賜綿絹萬段,謚曰敬。子協嗣。



 協字欽仁,好學,有雅量。起家著作佐郎,後轉秘書郎。開皇中,其父在廣州有功,上令協齎詔書勞問,未至而父卒。上以其父身死王事,拜協柱國。後歷定、息、秦三州刺史,皆有能名,卒官。



 藝字世文,少受業國子。周武帝時,數以軍功致位上儀同,賜爵修武縣侯,邑八百戶。授左旅下大夫。出為魏郡太守。及高祖為丞相,尉迥險圖不軌,朝廷微知之,遣藝季父孝寬馳往代迥。孝寬將至鄴,因詐病,止傳舍,從迥求藥,以察其變。



 迥遣藝迎孝寬。孝寬問迥所為,藝黨於迥,不以實答。孝寬怒,將斬之,藝懼,乃言迥反狀。孝寬於是將藝西遁,每至亭驛,輒盡驅傳馬而去。復謂驛司曰:「蜀公將至,宜速具酒食。」迥尋遣騎追孝寬,追人至驛,輒逢盛饌,又無馬,遂遲留不進,孝寬與藝由是得免。高祖以孝寬故,弗問藝之罪,加授上開府,即從孝寬擊迥。



 及
 破尉惇,平相州,皆有力焉。以功進位上大將軍,改封武威縣公,邑千戶。以修武縣侯別封一子。高祖受禪,進封魏興郡公。歲餘,拜齊州刺史。為政清簡,士庶懷惠。在職數年,遷營州總管。藝容貌瑰偉,每夷狄參謁,必整儀衛,盛服以見之,獨坐滿一榻。番人畏懼,莫敢仰視。而大治產業,與北夷貿易,家資巨萬,頗為清論所譏。開皇十五年卒官,時年五十八。謚曰懷。



 沖字世沖,少以名家子,在周釋褐衛公府禮曹參軍。後從大將軍元定渡江伐陳,為陳人所虜,周武帝以幣贖而還之。帝復令沖以馬千匹使於陳,以贖開府賀拔華
 等五十人及元定之柩而還。沖有辭辯,奉使稱旨,累遷少御伯下大夫,加上儀同。於時稽胡屢為寇亂,沖自請安集之,因拜汾州刺史。高祖踐阼,徵為兼散騎常侍,進位開府,賜爵安固縣侯。歲餘,發南汾州胡千餘人北築長城,在途皆亡。上呼沖問計,沖曰:「夷狄之性,易為反覆,皆由牧宰不稱之所致也。臣請以理綏靜,可不勞兵而定。」上然之,因命沖綏懷叛者。月餘皆至,並赴長城,上下書勞勉之。尋拜石州刺史,甚得諸胡歡心。以母憂去職。俄而起為南寧州總管,持節撫慰。復遣柱國王長述以兵繼進。沖上表固讓。詔曰:「西南夷裔,屢有生梗,每相殘
 賊,朕甚愍之,已命戎徒,清撫邊服。以開府器幹堪濟,識略英遠,軍旅事重,故以相任。



 知在艱疚,日月未多,金革奪情,蓋有通式。宜自抑割,即膺往旨。」沖既至南寧,渠帥爨震及西爨首領皆詣府參謁。上大悅,下詔褒揚之。其兄子伯仁,隨沖在府,掠人之妻,士卒縱暴,邊人失望。上聞而大怒,令蜀王秀治其事。益州長史元巖,性方正,案沖無所寬貸,沖竟坐免。其弟太子洗馬世約,譖巖於皇太子。上謂太子曰:「古人有沽酒酸而不售者,為噬犬耳。今何用世約乎?適累汝也。」世約遂除名。後數載,令沖檢校括州事。時東陽賊帥陶子定、吳州賊帥羅慧方並聚
 眾為亂,攻圍婺州永康、烏程諸縣,沖率兵擊破之。改封義豐縣候,檢校泉州事。尋拜營州總管。沖容貌都雅,寬厚得眾心。懷撫靺鞨、契丹,皆能致其死力。奚、霫畏懼,朝貢相續。高麗嘗入寇,沖率兵擊走之。仁壽中,高祖為豫章王暕納沖女為妃,徵拜民部尚書。未幾,卒,時年六十六。少子挺,最知名。



 壽字世齡。父孝寬,周上柱國、鄖國公。壽在周,以貴公子,早有令譽,為右侍上士,遷千牛備身。趙王為雍州牧,引為主簿。尋遷少御伯。武帝親征高氏,拜京兆尹,委以後事。以父軍功,賜爵永安縣侯,邑八百戶。高祖為丞相,以
 其父平尉迥,拜壽儀同三司,進封滑國公,邑五千戶。俄以父喪去職。高祖愛禪,起令視事,尋遷恆、毛二州刺史,頗有治名。開皇十年,以疾徵還,卒於家,時年四十二。



 謚曰定。仁壽中,高祖為晉王昭納其女為妃。以其子保巒嗣。



 壽弟霽,位至太常少卿,安邑縣伯。津,位至內史侍郎,判民部尚書事。



 世康從父弟操,字元節,剛簡有風概。仕周,致位上開府、光州刺史。高祖為丞相,以平尉迥功,進位柱國,封平桑郡公,歷青、荊二州總管,卒官。謚曰靜。



 柳機子述機弟旦肅從弟雄亮從子謇之族弟昂昂子調柳機,字匡時,河東解人也。父慶,魏尚書左僕射。機偉儀
 容,有器局,頗涉經史。年十九,周武帝時為魯公,引為記室。及帝嗣位,自宣納上士累遷少納言、太子宮尹,封平齊縣公。從帝平齊,拜開府,轉司宗中大夫。宣帝時,遷御正上大夫。機見帝失德,屢諫不聽,恐禍及己,托於鄭譯,陰求出外,於是拜華州刺史。



 及高祖作相,徵還京師。時周代舊臣皆勸禪讓,機獨義形於色,無所陳請。俄拜衛州刺史。及踐阼,進爵建安郡公,邑二千四百戶,徵為納言。機性寬簡,有雅望,然當近侍,無所損益,又好飲酒,不親細務,在職數年,復出為華州刺史。奉詔每月朝見。尋轉冀州刺史。後徵入朝,以其子述尚蘭陵公主,禮遇益
 隆。



 初,機在周,與族人文城公昂俱歷顯要。及此,機、昂並為外職,楊素時為納言,方用事,因上賜宴,素戲機曰:「二柳俱摧,孤楊獨聳。」坐者歡笑,機竟無言。未幾,還州。前後作牧,俱稱寬惠。後數年,以疾徵還京師,卒於家,時年五十六。贈大將軍、青州刺史,謚曰簡。子述嗣。



 柳述,字業隆,性明敏,有幹略,頗涉文藝。少以父廕,為太子親衛。後以尚主之故,拜開府儀同三司、內史侍郎。上於諸婿中,特所寵敬。歲餘,判兵部尚書事。丁父艱去職。未幾,起攝給事黃門侍郎事,襲爵建安郡公。仁壽中,判吏部尚書事。述雖職務修理,為當時所稱,然不達大體,
 暴於馭下,又怙寵驕豪,無所降屈。楊素時稱貴幸,朝臣莫不讋憚,述每陵侮之,數於上前面折素短。判事有不合素意,素或令述改之,輒謂將命者曰:「語僕射,道尚書不肯。」素由是銜之。俄而楊素亦被疏忌,不知省務。述任寄逾重,拜兵部尚書,參掌機密。述自以無功可紀,過叨匪服,抗表陳讓。上許之,令攝兵部尚書事。上於仁壽宮寢疾,述與楊素、黃門侍郎元巖等侍疾宮中。時皇太子無禮於陳貴人,上知而大怒,因令述召房陵王。



 述與元巖出外作敕書,楊素聞之,與皇太子協謀,便矯詔執述、巖二人,持以屬吏。



 及煬帝嗣位,述竟坐除名,與公主離絕。
 徙述於龍川郡。公主請與述同徙,帝不聽,事見《列女傳》。述在龍川數年,復徙寧越,遇瘴癘而死,時年三十九。



 旦字匡德,工騎射,頗涉書籍。起家周左侍上士,累遷兵部下大夫。頃之,益州總管王謙起逆,拜為行軍長史,從梁睿討平之,以功授儀同三司。開皇元年,加授開府,封新城縣男,遷授掌設驃騎。歷羅、淅、魯三州刺史,並有能名。大業初,拜龍川太守。民居山洞,好相攻擊,旦為開設學校,大變其風。帝聞而善之,下詔褒美。四年,徵為太常少卿,攝判黃門侍郎事。卒官,年六十一。子燮,官至河內掾。



 肅字匡仁,少聰敏,閑於占對。起家周齊王文學。武帝見而異之,召拜宣納上士。高祖作相,引為賓曹參軍。開皇初,授太子洗馬。陳使謝泉來聘,以才學見稱,詔肅宴接,時論稱其華辯。轉太子內舍人,遷太子僕。太子廢,坐除名為民。大業中,帝與段達語及庶人罪惡之狀,達云:「柳肅在宮,大見疏斥。」帝問其故,答曰:「學士劉臻,嘗進章仇太翼於宮中,為巫蠱事。肅知而諫曰:『殿下帝之塚子,位當儲貳,誡在不孝,無患見疑。劉臻書生,鼓搖脣舌,適足以相誑誤,願殿下勿納之。』庶人不懌,他日謂臻曰:『汝何故漏洩,使柳肅知之,令面折我?』自是後言皆不用。」帝曰:「
 肅橫除名,非其罪也。」召守禮部侍郎,轉工部侍郎,大見親任。每行幸遼東,常委之於涿郡留守。十一年卒,時年六十二。



 雄亮字信誠。父檜,仕周華陽太守。遇黃眾寶作亂,攻陷華陽,檜為賊所害。



 雄亮時年十四,哀毀過禮,陰有復仇之志。武帝時,眾寶率其所部歸於長安,帝待之甚厚。雄亮手斬眾寶於城中,請罪闕下,帝特原之。尋治梁州總管記室,遷湖城令,累遷內史中大夫,賜爵汝陽縣子。司馬消難作亂江北,高祖令雄亮聘於陳,以結鄰好。及還,會高祖受禪,拜尚書考功侍郎,尋遷給事黃門侍郎。尚
 書省凡有奏事,雄亮多所駁正,深為公卿所憚。俄以本官檢校太子左庶子,進爵為伯。秦王俊之鎮隴右也,出為秦州總管府司馬,領山南道行臺左丞,卒官,時年五十一。有子贊。



 謇之字公正。父蔡年,周順州刺史。謇之身長七尺五寸,儀容甚偉,風神爽亮,進止可觀。為童兒時,周齊王憲嘗遇謇之於途,異而與語,大奇之。因奏入國子,以明經擢第,拜宗師中士,尋轉守廟下士。武帝嘗有事太廟,謇之讀祝文,音韻清雅,觀者屬目。帝善之,擢為宣納上士。及高祖作相,引為田曹參軍,仍諮典簽事。



 開皇初,拜通事
 舍人,尋遷內史舍人,歷兵部、司勛二曹侍郎。朝廷以謇之有雅望,善談謔,又飲酒至石不亂,由是每梁、陳使至,輒令謇之接對。後遷光祿少卿。出入十餘年,每參掌敷奏。會吐谷渾來降,朝廷以宗女光化公主妻之,以謇之兼散騎常侍,送公主於西域。俄而突厥啟民可汗求結和親,復令謇之送義成公主於突厥。



 謇之前後奉使,得二國所贈馬千餘匹,雜物稱是,皆散之宗族,家無餘財。仁壽中,出為肅州刺史,尋轉息州刺史,俱有惠政。後二歲,以母憂去職。煬帝踐阼,復拜光祿少卿。大業初,啟民可汗自以內附,遂畜牧於定襄、馬邑間,帝使謇之諭令
 出塞。及還,奏事稱旨,拜黃門侍郎。時元德太子初薨,朝野注望,皆以齊王當立。



 帝方重王府之選,大業三年,車駕還京師,拜為齊王長史。帝法服臨軒,備儀衛,命齊王立於西朝堂之前,北面。遣吏部尚書牛弘、內史令楊約、左衛大將軍宇文述等,從殿廷引謇之詣齊王所,西面立。牛弘宣敕謂齊王曰:「我昔階緣恩寵,啟封晉陽,出籓之初,時年十二。先帝立我於西朝堂,乃令高熲、虞慶則、元旻等,從內送王子相於我。於時誡我曰:『以汝幼沖,未更世事,今令子相作輔於汝,事無大小,皆可委之。無得暱近小人,疏遠子相。若從我言者,有益於社稷,成立汝
 名行。如不用此言,唯國及身,敗無日矣。』吾受敕之後,奉以周旋,不敢失墜。微子相之力,吾無今日矣。若與謇之從事,一如子相也。」又敕謇之曰:「今以卿作輔於齊,善思匡救之理,副朕所望。若齊王德業修備,富貴自當鐘卿一門。若有不善,罪亦相及。」時齊王正擅寵,左右放縱,喬令則之徒,深見暱狎。謇之雖知其罪失,不能匡正。及王得罪,謇之竟坐除名。帝幸遼東,召謇之檢校燕郡事。及帝班師,至燕郡,坐供頓不給,配戍嶺南。卒于洭口,時年六十。子威明。



 昂字千里。父敏,有高名,好禮篤學,治家如官。仕周,歷職
 清顯。開皇初,為太子太保。昂有器識,幹局過人。周武帝時,為大內史,賜爵文城郡公,致位開府,當途用事,百僚皆出其下。宣帝嗣位,稍被疏遠,然不離本職。及高祖為丞相,深自結納。高祖大悅之,以為大宗伯。昂受拜之日,遂得偏風,不能視事。高祖受禪,昂疾愈,加上開府,拜潞州刺史。昂見天下無事,可以勸學行禮,因上表曰:臣聞帝王受命,建學制禮,故能移既往之風,成惟新之俗。自魏道將謝,分割九區,關右、山東,久為戰國,各逞權詐,俱殉干戈,賦役繁重,刑政嚴急。蓋救焚拯溺,無暇從容,非朝野之願,以至於此。晚世因循,遂成希慕,俗化澆敝,流
 宕忘反,自非天然上哲,挺生於時,則儒雅之道,經禮之制,衣冠民庶,莫肯用心。



 世事所以未清,軌物由茲而壞。伏惟陛下稟靈上帝,受命昊天,合三陽之期,膺千祀之運。往者周室頹毀,區宇沸騰,聖策風行,神謀電發,端坐廊廟,蕩滌萬方,俯順幽明,君臨四海。擇萬古之典,無善不為;改百王之弊,無惡不盡。至若因情緣義,為其節文,故以三百三千,事高前代。然下土黎獻,尚未盡行。臣謬蒙獎策,從政籓部,人庶軌儀,實見多闕,儒風以墜,禮教猶微,是知百姓之心,未能頓變。



 仰惟深思遠慮,情念下民,漸被以儉,使至於道。臣恐業淹事緩,動延年世。若行
 禮勸學,道教相催,必當靡然向風,不遠而就。家知禮節,人識義方,比屋可封,輒謂非遠。



 上覽而善之,因下詔曰:建國重道,莫先於學,尊主庇民,莫先於禮。自魏氏不競,周、齊抗衡,分四海之民,斗二邦之力,遞為強弱,多歷年所。務權詐而薄儒雅,重干戈而輕俎豆,民不見德,唯爭是聞。朝野以機巧為師,文吏用深刻為法,風澆俗弊,化之然也。



 雖復建立庠序,兼啟黌塾,業非時貴,道亦不行。其間服膺儒術,蓋有之矣,彼眾我寡,未能移俗。然其維持名教,獎飾彞倫,微相弘益,賴斯而已。王者承天,休咎隨化,有禮則祥瑞必降,無禮則妖孽興起。人稟五常,性
 靈不一,有禮則陰陽合德,無禮則禽獸其心。治國立身,非禮不可。朕受命於天,財成萬物,去華夷之亂,求風化之宜。戒奢崇儉,率先百闢,輕徭薄賦,冀以寬弘。而積習生常,未能懲革,閭閻士庶,吉兇之禮,動悉乖方,不依制度。執憲之職,似塞耳而無聞,蒞民之官,猶蔽目而不察。宣揚朝化,其若是乎?古人之學,且耕且養。今者民丁非役之日,農畝時候之餘,若敦以學業,勸以經禮,自可家慕大道,人希至德。豈止知禮節,識廉恥,父慈子孝,兄恭弟順者乎?始自京師,爰及州郡,宜祗朕意,勸學行禮。



 自是天下州縣皆置博士習禮焉。



 昂在州,甚有惠政,數年,
 卒官。



 子調,起家秘書郎,尋轉侍御史。左僕射楊素嘗於朝堂見調,因獨言曰:「柳條通體弱,獨搖不須風。」調斂板正色曰:「調信無取者,公不當以為侍御史;調信有可取,不應發此言。公當具瞻之秋,樞機何可輕發!」素甚奇之。煬帝嗣位,累遷尚書左司郎。時王綱不振,朝士多贓貨,唯調清素守常,為時所美。然於乾用,非其所長。



 史臣曰:韋氏自居京兆,代有人物。世康昆季,餘慶所鐘,或入處禮闈,或出總方岳,硃輸接軫,P「P」成陰,在周暨隋,勛庸並茂,盛矣!建安風韻閑雅,望重當時。述恃寵驕人,終致傾敗。旦屢有惠政,肅每存誠讜。雄亮名節自立,
 忠正見稱,謇之神情開爽,頗為疏放。文城歷仕二朝,咸見推重,獻書高祖,遂興學校,言能弘道,其利博哉!



\end{pinyinscope}