\article{卷八十二列傳第四十七}

\begin{pinyinscope}

 南蠻南蠻雜類,與華人錯居,曰蜒,曰狼,曰俚,曰獠,曰頠,俱無君長,隨山洞而居,古先所謂百越是也。其俗斷發文身,好相攻討,浸以微弱,稍屬於中國,皆列為郡縣,同之齊人,不復詳載。大業中,南荒朝貢者十餘國,其事跡多湮滅而無聞。今所存錄,四國而已。



 林邑林邑之先,因漢末交阯女子徵側之亂,內縣功曹子區
 連殺縣令,自號為王。無子,其甥範熊代立,死,子逸立。日南人範文因亂為逸僕隸,遂教之築宮室,造器械。逸甚信任,使文將兵,極得眾心。文因問其子弟,或奔或徙。及逸死,國無嗣,文自立為王。其後範佛為晉揚威將軍戴桓所破。宋交州刺史檀和之將兵擊之,深入其境。至梁、陳,亦通使往來。



 其國延袤數千里,土多香木金寶,物產大抵與交阯同。以磚為城,蜃灰塗之,東向戶。尊官有二:其一曰西那婆帝,其二曰薩婆地歌。其屬官三等:其一曰倫多姓,次歌倫致帝,次乙他伽蘭。外官分為二百餘部。其長官曰弗羅,次曰可輪,如牧宰之差也。王戴金花
 冠,形如章甫,衣朝霞布,珠璣瓔珞,足躡革履,時復錦袍。



 良家子侍衛者二百許人,皆執金裝刀。有弓、箭、刀、槊,以竹為弩,傅毒於矢。



 樂有琴、笛、琵琶、五弦,頗與中國同。每擊鼓以警眾,吹蠡以即戎。



 其人深目高鼻,發拳色黑。俗皆徒跣,以幅布纏身。冬月衣袍。婦人椎髻。施椰葉席。每有婚媾,令媒者齎金銀釧、酒二壺、魚數頭至女家。於是擇日,夫家會親賓,歌舞相對。女家請一婆羅門,送女至男家,婿盥手,因牽女授之。王死七日而葬,有官者三日,庶人一日。皆以函盛尸,鼓舞導從,輿至水次,積薪焚之。收其餘骨,王則內金甕中,沉之於海,有官者以銅甕,沉
 之於海口;庶人以瓦,送之於江。男女皆截發,隨喪至水次,盡哀而止,歸則不哭。每七日,然香散花,復哭,盡哀而止。盡七七而罷,至百日、三年,亦如之。人皆奉佛,文字同於天竺。



 高祖既平陳,乃遣使獻方物,其後朝貢遂絕。時天下無事,群臣言林邑多奇寶者。仁壽末,上遣大將軍劉方為驩州道行軍總管,率欽州刺史寧長真、驩州刺史李暈、開府秦雄步騎萬餘及犯罪者數千人擊之。其王梵志率其徒乘巨象而戰,方軍不利。方於是多掘小坑,草覆其上,因以兵挑之。梵志悉眾而陣,方與戰,偽北,梵志逐之,至坑所,其眾多陷,轉相驚駭,軍遂亂。方縱兵
 擊之,大破之。頻戰輒敗,遂棄城而走。方入其都,獲其廟主十八枚,皆鑄金為之,蓋其有國十八葉矣。方班師,梵志復其故地,遣使謝罪,於是朝貢不絕。



 赤土赤土國,扶南之別種也。在南海中,水行百餘日而達所都。土色多赤,因以為號。東波羅刺國,西婆羅娑國,南訶羅旦國,北拒大海,地方數千里。其王姓瞿曇氏,名利富多塞,不知有國近遠。稱其父釋王位出家為道,傳位於利富多塞,在位十六年矣。有三妻,並鄰國王之女也。居僧祗城,有門三重,相去各百許步。每門圖畫飛仙、仙人、
 菩薩之像,縣金花鈴毦,婦女數十人,或奏樂,或捧金花。又飾四婦人,容飾如佛塔邊金剛力士之狀,夾門而立。門外者持兵仗,門內者執白拂。



 夾道垂素網,綴花。王宮諸屋悉是重閣,北戶,北面而坐。坐三重之榻。衣朝霞布,冠金花冠,垂雜寶瓔珞。四女子立侍,左右兵衛百餘人。王榻後作一木龕,以金銀五香木雜鈿之。龕後懸一金光焰,夾榻又樹二金鏡,鏡前並陳金甕,甕前各有金香爐。當前置一金伏牛,牛前樹壹寶蓋,蓋左右皆有寶扇。婆羅門等數百人,東西重行,相向而坐。其官有薩陀迦羅一人,陀拏達義二人,迦利蜜迦三人,共掌政事;俱羅
 末帝一人,掌刑法。每城置那邪迦一人,缽帝十人。



 其俗等皆穿耳剪發,無跪拜之禮。以香油塗身。其俗敬佛,尤重婆羅門。婦人作髻於項後。男女通以朝霞、朝雲雜色布為衣。豪富之室,恣意華靡,唯金鎖非王賜不得服用。每婚嫁,擇吉日,女家先期五日,作樂飲酒,父執女手以授婿,七日乃配焉。既娶則分財別居,唯幼子與父同居。父母兄弟死則剔發素服,就水上構竹木為棚,棚內積薪,以尸置上。燒香建幡,吹蠡擊鼓以送之,縱火焚薪,遂落於水。



 貴賤皆同。唯國王燒訖,收灰貯以金瓶,藏於廟屋。冬夏常溫,雨多霽少,種植無時,特宜稻、穄、白豆、黑麻,
 自餘物產,多同於交阯。以甘蔗作酒,雜以紫瓜根。



 酒色黃赤,味亦香美。亦名椰漿為酒。



 煬帝即位,募能通絕域者。大業三年,屯田主事常駿、虞部主事王君政等請使赤土。帝大悅,賜駿等帛各百匹,時服一襲而遣。齎物五千段,以賜赤土王。其年十月,駿等自南海郡乘舟,晝夜二旬,每值便風。至焦石山而過,東南泊陵伽缽拔多洲,西與林邑相對,上有神祠焉。又南行,至師子石,自是島嶼連接。又行二三日,西望見狼牙須國之山,於是南達雞籠島,至於赤土之界。其王遣婆羅門鳩摩羅以舶三十艘來迎,吹蠡擊鼓,以樂隋使,進金鎖以纜駿船。月餘,
 至其都,王遣其子那邪迦請與駿等禮見。先遣人送金盤,貯香花並鏡鑷,金合二枚,貯香油,金瓶八枚,貯香水,白疊布四條,以擬供使者盥洗。其日未時,那邪迦又將象二頭,持孔雀蓋以迎使人,並致金花、金盤以藉詔函。男女百人奏蠡鼓,婆羅門二人導路,至王宮。駿等奉詔書上閣,王以下皆坐。宣詔訖,引駿等坐,奏天竺樂。事畢,駿等還館,又遣婆羅門就館送食,以草葉為盤,其大方丈。因謂駿曰:「今是大國中人,非復赤土國矣。飲食疏薄,願為大國意而食之。」後數日,請駿等入宴,儀衛導從如初見之禮。王前設兩床,床上並設草葉盤,方一丈五尺,
 上有黃白紫赤四色之餅,牛、羊、魚、鱉、豬、蝳蝐之肉百餘品。延駿升床,從者坐於地席,各以金鐘置酒,女樂迭奏,禮遺甚厚。尋遣那邪迦隨駿貢方物,並獻金芙蓉冠、龍腦香。



 以鑄金為多羅葉,隱起成文以為表,金函封之,令婆羅門以香花奏蠡鼓而送之。既入海,見綠魚群飛水上。浮海十餘日,至林邑東南,並山而行。其海水闊千餘步,色黃氣腥,舟行一日不絕,云是大魚糞也。循海北岸,達於交止。駿以六年春與那邪迦於弘農謁帝,大悅,賜駿等物二百段,俱授秉義尉,那邪迦等官賞各有差。



 真臘
 真臘國,在林邑西南,本扶南之屬國也。去日南郡舟行六十日,而南接車渠國,西有硃江國。其王姓剎利氏,名質多斯那。自其祖漸已強盛,至質多斯那,遂兼扶南而有之。死,子伊奢那先代立。居伊奢那城,郭下二萬餘家。城中有一大堂,是王聽政之所。總大城三十,城有數千家,各有部帥,官名與林邑同。其王三日一聽朝,坐五香七寶床,上施寶帳。其帳以文木為竿,象牙、金鈿為壁,狀如小屋,懸金光焰,有同於赤土。前有金香爐,二人侍側。王著朝霞古貝,瞞絡腰腹,下垂至脛,頭戴金寶花冠,被真珠瓔珞,足履革屣,耳懸金璫。常服白疊,以象牙為屩。



 若露發,則不加瓔珞。臣人服制,大抵相類。有五大臣,一曰孤落支,二曰高相憑,三曰婆何多陵,四曰舍摩陵,五曰髯多婁,及諸小臣。朝於王者,輒以階下三稽首。



 王喚上階,則跪,以兩手抱膊,繞王環坐。議政事訖,跪伏而去。階庭門閣,侍衛有千餘人,被甲持仗。其國與參半、硃江二國和親,數與林邑、陀桓二國戰爭。其人行止皆持甲仗,若有征伐,因而用之。其俗非王正妻子,不得為嗣。王初立之日,所有兄弟並刑殘之,或去一指,或劓其鼻,別處供給,不得仕進。



 人形小而色黑。婦人亦有白者。悉拳發垂耳,性氣捷勁。居處器物,頗類赤土。



 以右手為凈,左
 手為穢。每旦澡洗,以楊枝凈齒,讀誦經咒。又澡灑乃食,食罷還用楊枝凈齒,又讀經咒。飲食多蘇酪、沙糖、秔粟、米餅。欲食之時,先取雜肉羹與餅相和,手擩而食。娶妻者,唯送衣一具,擇日遣媒人迎婦。男女二家各八日不出,晝夜燃燈不息。男婚禮畢,即與父母分財別居。父母死,小兒未婚者,以餘財與之。若婚畢,財物入官。其喪葬,兒女皆七日不食,剔發而哭,僧尼、道士、親故皆來聚會,音樂送之。以五香木燒尸,收灰以金銀瓶盛,送於大水之內。貧者或用瓦,而以彩色畫之。亦有不焚,送尸山中,任野獸食者。



 其國北多山阜,南有水澤,地氣尤熱,無霜雪,
 饒瘴癘毒蟲。土宜粱稻,少黍粟,果菜與日南、九真相類。異者有婆那娑樹,無花,葉似柿,實似冬瓜;奄羅樹,花葉似棗,實似李;毗野樹,花似木瓜,葉似杏,實似楮;婆田羅樹,花葉實並似棗而小異;歌畢佗樹,花似林檎,葉似榆而厚大,實似李,其大如升。自餘多同九真。海中有魚名建同,四足,無鱗,其鼻如象,吸水上噴,高五六十尺。有浮胡魚,其形似且,嘴如鸚鵡,有八足。多大魚,半身出水,望之如山。



 每五六月中,毒氣流行,即以白豬、白牛、白羊於城西門外祠之。不然者,五穀不登,六畜多死,人眾疾疫。近都有陵伽缽婆山,上有神祠,每以兵五千人守衛
 之。城東有神名婆多利,祭用人肉。其王年別殺人,以夜祀禱,亦有守衛者千人。



 其敬鬼如此。多奉佛法,尤信道士,佛及道士並立像於館。



 大業十二年,遣使貢獻,帝禮之甚厚,其後亦絕。



 婆利婆利國,自交阯浮海,南過赤土、丹丹,乃至其國。國界東西四月行,南北四十五日行。王姓剎利邪伽,名護濫那婆。官曰獨訶邪挐,次曰獨訶氏挐。國人善投輪刀,其大如鏡,中有竅,外鋒如鋸,遠以投人,無不中。其餘兵器,與中國略同。



 俗類真臘,物產同於林邑。其殺人及盜,截其
 手,奸者鎖其足,期年而止。祭祀必以月晦,盤貯酒肴,浮之流水。每十一月,必設大祭。海出珊瑚。有鳥名舍利,解人語。大業十二年,遣使朝貢,後遂絕。於時南荒有丹丹、盤盤二國,亦來貢方物,其風俗物產,大抵相類云。



 史臣曰:《禮》云:「南方曰蠻,有不火食者矣。」《書》稱:「蠻夷猾夏。」



 《詩》曰:「蠢爾蠻荊。」種類實繁,代為紛梗。自秦並二楚,漢平百越,地窮丹徼,景極日南,水陸可居,咸為郡縣。暨乎境分吳、蜀,時經晉、宋,道有污隆,服叛不一。高祖受命,克平九宇。煬帝纂業,威加八荒。甘心遠夷,志求珍異,故師出於流求,兵加於林邑,威振殊俗,過於秦、漢遠矣。雖有荒
 外之功,無救域中之敗,《傳》曰:「非聖人,外寧必內憂。」誠哉斯言也!



\end{pinyinscope}