\article{卷八十一列傳第四十六}

\begin{pinyinscope}

 東夷高麗高麗之先,出自夫餘。夫餘王嘗得河伯女,因閉於室內,為日光隨而照之,感而遂孕,生一大卵,有一男子破殼而出,名曰硃蒙。夫餘之臣以硃蒙非人所生,咸請殺之,王不聽。及壯,因從獵,所獲居多,又請殺之。其母以告硃蒙,硃蒙棄夫餘東南走。遇一大水,深不可越。硃蒙曰:「我是河伯外孫,日之子也。今有難,而追兵且及,如何得渡?」
 於是魚鱉積而成橋,硃蒙遂渡,追騎不得濟而還。硃蒙建國,自號高句麗,以高為氏。硃蒙死,子閭達嗣。至其孫莫來興兵,遂並夫餘。



 至裔孫位宮,以魏正始中入寇西安平,毌丘儉拒破之。位宮玄孫之子曰昭列帝,為慕容氏所破,遂入丸都,焚其宮室,大掠而還。昭列帝後為百濟所殺。其曾孫璉,遣使後魏。璉六世孫湯,在周遣使朝貢,武帝拜湯上開府、遼東郡公、遼東王。高祖受禪,湯復遣使詣闕,進授大將軍,改封高麗王。歲遣使朝貢不絕。



 其國東西二千里,南北千餘里。都於平壤城,亦曰長安城,東西六里,隨山屈曲,南臨浿水。復有國內城、漢城,並
 其都會之所,其國中呼為「三京」。與新羅每相侵奪,戰爭不息。官有太大兄,次大兄,次小兄,次對盧,次意侯奢,次烏拙,次太大使者,次大使者,次小使者,次褥奢,次翳屬,次仙人,凡十二等。復有內評、外評、五部褥薩。人皆皮冠,使人加插鳥羽。貴者冠用紫羅,飾以金銀。服大袖衫,大口褲,素皮帶,黃革屨。婦人裙襦加襈。兵器與中國略同。每春秋校獵,王親臨之。人稅布五匹,穀五石。游人則三年一稅,十人共細布一匹,租戶一石,次七斗,下五斗。反逆者縛之於柱,爇而斬之,籍沒其家。盜則償十倍。用刑既峻,罕有犯者。樂有五弦、琴、箏、篳篥、橫吹、簫、鼓之屬,吹
 蘆以和曲。每年初,聚戲於浿水之上,王乘腰輿,列羽儀以觀之。事畢,王以衣服入水,分左右為二部,以水石相濺擲,喧呼馳逐,再三而止。俗好蹲踞。潔凈自喜,以趨走為敬,拜則曳一腳,立各反拱,行必搖手。性多詭伏。父子同川而浴,共室而寢。婦人淫奔,俗多游女。有婚嫁者,取男女相悅,然即為之,男家送豬酒而已,無財聘之禮。或有受財者,人共恥之。死者殯於屋內,經三年,擇吉日而葬。居父母及夫之喪,服皆三年,兄弟三月。初終哭泣,葬則鼓舞作樂以送之。埋訖,悉取死者生時服玩車馬置於墓側,會葬者爭取而去。敬鬼神,多淫祠。



 開皇初,頻有
 使入朝。及平陳之後,湯大懼,治兵積穀,為守拒之策。十七年,上賜湯璽書曰:朕受天命,愛育率土,委王海隅,宣揚朝化,欲使圓首方足,各遂其心。王每遣使人,歲常朝貢,雖稱籓附,誠節未盡。王既人臣,須同朕德,而乃驅逼靺鞨,固禁契丹。諸籓頓顙,為我臣妾,忿善人之慕義,何毒害之情深乎?太府工人,其數不少,王必須之,自可聞奏。昔年潛行財貨,利動小人,私將弩手,逃竄下國。



 豈非修理兵器,意欲不臧,恐有外聞,故為盜竊?時命使者,撫尉王籓,本欲問彼人情,教彼政術。王乃坐之空館,嚴加防守,使其閉目塞耳,永無聞見。有何陰惡,弗欲人知,禁
 制官司,畏其訪察?又數遣馬騎,殺害邊人,屢馳奸謀,動作邪說,心在不賓。朕於蒼生,悉如赤子,賜王土宇,授王官爵,深恩殊澤,彰著遐邇。王專懷不信,恆自猜疑,常遣使人,密覘消息,純臣之義,豈若是也?蓋當由朕訓導不明,王之愆違,一已寬恕,今日以後,必須改革。守籓臣之節,奉朝正之典,自化爾籓,勿忤他國,則長享富貴,實稱朕心。彼之一方,雖地狹人少,然普天之下,皆為朕臣。今若黜王,不可虛置,終須更選官屬,就彼安撫。王若灑心易行,率由憲章,即是朕之良臣,何勞別遣才彥也?昔帝王作法,仁信為先,有善必賞,有惡必罰,四海之內,具聞
 朕旨。王若無罪,朕忽加兵,自餘籓國,謂朕何也!王必虛心,納朕此意,慎勿疑惑,更懷異圖。往者陳叔寶代在江陰,殘害人庶,驚動我烽候,抄掠我邊境。朕前後誡敕,經歷十年,彼則恃長江之外,聚一隅之眾,昏狂驕傲,不從朕言。故命將出師,除彼兇逆,來往不盈旬月,兵騎不過數千,歷代逋寇,一朝清蕩,遐邇乂安,人神胥悅。聞王嘆恨,獨致悲傷,黜陟幽明,有司是職,罪王不為陳滅,賞王不為陳存,樂禍好亂,何為爾也?王謂遼水之廣,何如長江?高麗之人,多少陳國?朕若不存含育,責王前愆,命一將軍,何待多力!殷勤曉示,許王自新耳。宜得朕懷,自求
 多福。



 湯得書惶恐,將奉表陳謝,會病卒。子元嗣立。高祖使使拜元為上開府、儀同三司,襲爵遼東郡公,賜衣一襲。元奉表謝恩,並賀祥瑞,因請封王。高祖優冊元為王。



 明年,元率靺鞨之眾萬餘騎寇遼西,營州總管韋沖擊走之。高祖聞而大怒,命漢王諒為元帥,總水陸討之,下詔黜其爵位。時饋運不繼,六軍乏食,師出臨渝關,復遇疾疫,王師不振。及次遼水,元亦惶懼,遣使謝罪,上表稱「遼東糞土臣元」



 雲云。上於是罷兵,待之如初,元亦歲遣朝貢。煬帝嗣位,天下全盛,高昌王、突厥啟人可汗並親詣闕貢獻,於是徵元入朝。元懼籓禮頗闕。大業七年,帝
 將討元之罪,車駕渡遼水,上營於遼東城,分道出師,各頓兵於其城下。高麗率兵出拒,戰多不利,於是皆嬰城固守。帝令諸軍攻之,又敕諸將:「高麗若降者,即宜撫納,不得縱兵。」城將陷,賊輒言請降,諸將奉旨不敢赴機,先令馳奏。比報至,賊守禦亦備,隨出拒戰。如此者再三,帝不悟。由是食盡師老,轉輸不繼,諸軍多敗績,於是班師。是行也,唯於遼水西拔賊武厲邏,置遼東郡及通定鎮而還。九年,帝復親征之,乃敕諸軍以便宜從事。諸將分道攻城,賊勢日蹙。會楊玄感作亂,反書至,帝大懼,即日六軍並還。兵部侍郎斛斯政亡入高麗,高麗具知事實,
 悉銳來追,殿軍多敗。十年,又發天下兵,會盜賊蜂起,人多流亡,所在阻絕,軍多失期。至遼水,高麗亦困弊,遣使乞降,囚送斛斯政以贖罪。帝許之,頓於懷遠鎮,受其降款。



 仍以俘囚軍實歸。至京師,以高麗使者親告於太廟,因拘留之。仍徵元入朝,元竟不至。帝敕諸軍嚴裝,更圖後舉,會天下大亂,遂不克復行。



 百濟百濟之先,出自高麗國。其國王有一侍婢,忽懷孕,王欲殺之,婢云:「有物狀如雞子,來感於我,故有娠也。」王舍之。後遂生一男,棄之廁溷,久而不死,以為神,命養之,名曰
 東明。及長,高麗王忌之,東明懼,逃至淹水,夫餘人共奉之。東明之後,有仇臺者,篤於仁信,始立其國於帶方故地。漢遼東太守公孫度以女妻之,漸以昌盛,為東夷強國。初以百家濟海,因號百濟。歷十餘代,代臣中國,前史載之詳矣。開皇初,其王餘昌遣使貢方物,拜昌為上開府、帶方郡公、百濟王。



 其國東西四百五十里,南北九百餘里,南接新羅,北拒高麗。其都曰居拔城。


官有十六品:長曰左平,次大率,次恩率,次德率,次桿率,次奈率,次將德,服紫帶;次施德,皁帶;次固德,赤帶;次李德,青帶;次對德以下,皆黃帶;次文督,次武督,次佐軍,次振武,次克虞,皆
 用白帶。其冠制並同,唯奈率以上飾以銀花。長史三年一交代。畿內為五部,部有五巷,士人倨焉。五方各有方領一人,方佐貳之。方有十郡,郡有將。其人雜有新羅、高麗、倭等,亦有中國人。其衣服與高麗略同。婦人不加粉黛,女辮發垂後,已出嫁則分為兩道,盤於頭上。俗尚騎射,讀書史,能吏事,亦知醫藥、蓍龜、占相之術。以兩手據地為敬。有僧尼,多寺塔。有鼓角、箜篌、箏、竽、
 \gezhu{
  箎}
 、笛之樂,投壺、圍棋、樗蒲、握槊、弄珠之戲。行宋《元嘉歷》,以建寅月為歲首。國中大姓有八族,沙氏、燕氏、刀氏、解氏、貞氏、國氏、木氏、苗氏。婚娶之禮,略同於華。喪制如高麗。有五穀、牛、豬、
 雞,多不火食。厥田下濕,人皆山居。有巨慄。每以四仲之月,王祭天及五帝之神。立其始祖仇臺廟於國城,歲四祠之。國西南人島居者十五所,皆有城邑。



 平陳之歲,有一戰船漂至海東牟羅國,其船得還,經於百濟,昌資送之甚厚,並遣使奉表賀平陳。高祖善之,下詔曰:「百濟王既聞平陳,遠令奉表,往復至難,若逢風浪,便致傷損。百濟王心跡淳至,朕已委知。相去雖遠,事同言面,何必數遣使來相體悉。自今以後,不須年別入貢,朕亦不遣使往,王宜知之。」使者舞蹈而去。開皇十八年,昌使其長史王辯那來獻方物,屬興遼東之役,遣使奉表,請為軍導。
 帝下詔曰:「往歲為高麗不供職貢,無人臣禮,故命將討之。高元君臣恐懼,畏服歸罪,朕已赦之,不可致伐。」厚其使而遣之。高麗頗知其事,以兵侵掠其境。



 昌死,子餘宣立,死,子餘璋立。大業三年,璋遣使者燕文進朝貢。其年,又遣使者王孝鄰入獻,請討高麗。煬帝許之,令覘高麗動靜。然璋內與高麗通和,挾詐以窺中國。七年,帝親征高麗,璋使其臣國智牟來請軍期。帝大悅,厚加賞錫,遣尚書起部郎席律詣百濟,與相知。明年,六軍渡遼,璋亦嚴兵於境,聲言助軍,實持兩端。尋與新羅有隙,每相戰爭。十年,復遣使朝貢。後天下亂,使命遂絕。



 其南海行三
 月,有牟羅國,南北千餘里,東西數百里,土多麞鹿,附庸於百濟。



 百濟自西行三日,至貊國云。



 新羅新羅國,在高麗東南,居漢時樂浪之地,或稱斯羅。魏將毌丘儉討高麗,破之,奔沃沮。其後復歸故國,留者遂為新羅焉。故其人雜有華夏、高麗、百濟之屬,兼有沃沮、不耐、韓獩之地。其王本百濟人,自海逃入新羅,遂王其國。傳祚至金真平,開皇十四年,遣使貢方物。高祖拜真平為上開府、樂浪郡公、新羅王。其先附庸於百濟,後因百濟征高麗,高麗人不堪戎役,相率歸之,遂致強盛,因襲
 百濟,附庸於迦羅國。



 其官有十七等:其一曰伊罰干,貴如相國;次伊尺干,次迎幹,次破彌干,次大阿尺干,次阿尺干,次乙吉干,次沙咄干,次及伏乾,次大奈摩干,次奈摩,次大舍,次小舍,次吉土,次大烏,次小烏,次造位。外有郡縣。其文字、甲兵同於中國。選人壯健者悉入軍,烽、戍、邏俱有屯管部伍。風俗、刑政、衣服,略與高麗、百濟同。每正月旦相賀,王設宴會,班賚群官。其日拜日月神。至八月十五日,設樂,令官人射,賞以馬布。其有大事,則聚群官詳議而定之。服色尚素。婦人辮發繞頭,以雜彩及珠為飾。婚嫁之禮,唯酒食而已,輕重隨貧富。新婚之夕,女
 先拜舅姑,次即拜夫。死有棺斂,葬起墳陵。王及父母妻子喪,持服一年。田甚良沃,水陸兼種。其五穀、果菜、鳥獸物產,略與華同。大業以來,歲遣朝貢。新羅地多山險,雖與百濟構隙,百濟亦不能圖之。



 靺鞨靺鞨,在高麗之北,邑落俱有酋長,不相總一。凡有七種:其一號粟末部,與高麗相接,勝兵數千,多驍武,每寇高麗中。其二曰伯咄部,在粟末之北,勝兵七千。其三曰安車骨部,在伯咄東北。其四曰拂涅部,在伯咄東。其五曰號室部,在拂涅東。其六曰黑水部,在安車骨西北。其七
 曰白山部,在粟末東南。勝兵並不過三千,而黑水部尤為勁健。自拂涅以東,矢皆石鏃,即古之肅慎氏也。所居多依山水,渠帥曰大莫弗瞞咄,東夷中為強國。有徒太山者,俗甚敬畏,上有熊羆豹狼,皆不害人,人亦不敢殺。地卑濕,築土如堤,鑿穴以居,開口向上,以梯出入。相與偶耕,土多粟麥穄。水氣咸,生鹽於木皮之上。其畜多豬。嚼米為酒,飲之亦醉。



 婦人服布,男子衣豬狗皮。俗以溺洗手面,於諸夷最為不潔。其俗淫而妒,其妻外淫,人有告其夫者,夫輒殺妻,殺而後悔,必殺告者,由是奸淫之事終不發揚。人皆射獵為業,角弓長三尺,箭長尺有二
 寸。常以七八月造毒藥,傅矢以射禽獸,中者立死。



 開皇初,相率遣使貢獻。高祖詔其使曰:「朕聞彼土人庶多能勇捷,今來相見,實副朕懷。朕視爾等如子,爾等宜敬朕如父。」對曰:「臣等僻處一方,道路悠遠,聞內國有聖人,故來朝拜。既蒙勞賜,親奉聖顏,下情不勝歡喜,願得長為奴僕也。」



 其國西北與契丹相接,每相劫掠。後因其使來,高祖誡之曰:「我憐念契丹與爾無異,宜各守土境,豈不安樂?何為輒相攻擊,甚乖我意!」使者謝罪。高祖因厚勞之,令宴飲於前。使者與其徒皆起舞,其曲折多戰鬥之容。上顧謂侍臣曰:「天地間乃有此物,常作用兵意,何其
 甚也!」然其國與隋懸隔,唯粟末、白山為近。



 煬帝初與高麗戰,頻敗其眾,渠帥度地稽率其部來降。拜為右光祿大夫,居之柳城,與邊人來往。悅中國風俗,請被冠帶,帝嘉之,賜以錦綺而褒寵之。及遼東之役,度地稽率其徒以從,每有戰功,賞賜優厚。十三年,從帝幸江都,尋放歸柳城。在途遇李密之亂,密遣兵邀之,前後十餘戰,僅而得免。至高陽,復沒於王須拔。未幾,遁歸羅藝。



 流求國流求國,居海島之中,當建安郡東,水行五日而至。土多山洞。其王姓歡斯氏,名渴剌兜,不知其由來有國代數
 也。彼土人呼之為可老羊,妻曰多拔荼。所居曰波羅檀洞,塹柵三重,環以流水,樹棘為籓。王所居舍,其大一十六間,雕刻禽獸。



 多鬥鏤樹,似橘而葉密,條纖如發然下垂。國有四五帥,統諸洞,洞有小王。往往有村,村有鳥了帥,並以善戰者為之,自相樹立,理一村之事。男女皆以白糸寧繩纏發,從項後般繞至額。其男子用鳥羽為冠,裝以珠貝,飾以赤毛,形制不同。婦人以羅紋白布為帽,其形正方。織鬥鏤皮並雜色糸寧及雜毛以為衣,制裁不一。綴毛垂螺為飾,雜色相間,下垂小貝,其聲如佩,綴璫施釧,懸珠於頸。織藤為笠,飾以毛羽。有刀、槊、弓、箭、劍、鈹之
 屬。其處少鐵,刃皆薄小,多以骨角輔助之。編糸寧為甲,或用熊豹皮。王乘木獸,令左右輿之而行,導從不過數十人。小王乘機,鏤為獸形。國人好相攻擊,人皆驍健善走,難死而耐創。諸洞各為部隊,不相救助。兩陣相當,勇者三五人出前跳噪,交言相罵,因相擊射。如其不勝,一軍皆走,遣人致謝,即共和解。收取鬥死者,共聚而食之,仍以髑髏將向王所。王則賜之以冠,使為隊帥。無賦斂,有事則均稅。用刑亦無常準,皆臨事科決。犯罪皆斷於鳥了帥;不伏,則上請於王,王令臣下共議定之。獄無枷鎖,唯用繩縛。決死刑以鐵錐,大如箸,長尺餘,鉆頂而殺之。
 輕罪用杖。俗無文字,望月虧盈以紀時節,候草藥枯以為年歲。



 人深目長鼻,頗類於胡,亦有小慧。無君臣上下之節、拜伏之禮。父子同床而寢。男子拔去髭鬢,身上有毛之處皆亦除去。婦人以墨黥手,為蟲蛇之文。嫁娶以酒肴珠貝為娉,或男女相悅,便相匹偶。婦人產乳,必食子衣,產後以火自炙,令汗出,五日便平復。以木槽中暴海水為鹽,木汁為酢,釀米麥為酒,其味甚薄。食皆用手。偶得異味,先進尊者。凡有宴會,執酒者必待呼名而後飲。上王酒者,亦呼王名。銜杯共飲,頗同突厥。歌呼蹋蹄,一人唱,從皆和,音頗哀怨。扶女子上膊,搖手而舞。其死
 者氣將絕,舉至庭,親賓哭泣相吊。浴其尸,以布帛纏之,裹以葦草,親土而殯,上不起墳。子為父者,數月不食肉。南境風俗少異,人有死者,邑里共食之。



 有熊羆豺狼,尤多豬雞,無牛羊驢馬。厥田良沃,先以火燒而引水灌之。持一插,以石為刃,長尺餘,闊數寸,而墾之。土宜稻、梁、沄、黍、麻、豆、赤豆、胡豆、黑豆等,木有楓、栝、樟、松、楩、楠、杉、梓、竹、藤、果、藥,同於江表,風土氣候與嶺南相類。



 俗事山海之神,祭以酒肴,鬥戰殺人,便將所殺人祭其神。或依茂樹起小屋,或懸髑髏於樹上,以箭射之,或累石系幡以為神主。王之所居,壁下多聚髑髏以為佳。人間門戶上必
 安獸頭骨角。



 大業元年,海師何蠻等,每春秋二時,天清風靜,東望依希,似有煙霧之氣,亦不知幾千里。三年,煬帝令羽騎尉硃寬入海求訪異俗,何蠻言之,遂與蠻俱往,因到流求國。言不相通,掠一人而返。明年,帝復令寬慰撫之,流求不從,寬取其布甲而還。時倭國使來朝,見之曰:「此夷邪久國人所用也。」帝遣武賁郎將陳棱、朝請大夫張鎮州率兵自義安浮海擊之。至高華嶼,又東行二日至郤鼊嶼,又一日便至流求。初,棱將南方諸國人從軍,有昆侖人頗解其語,遣人慰諭之,流求不從,拒逆官軍。棱擊走之,進至其都,頻戰皆敗,焚其宮室,虜其男
 女數千人,載軍實而還。自爾遂絕。



 倭國倭國,在百濟、新羅東南,水陸三千里,於大海之中依山島而居。魏時譯通中國。三十餘國,皆自稱王。夷人不知里數,但計以日。其國境東西五月行,南北三月行,各至於海。其地勢東高西下。都於邪靡堆,則《魏志》所謂邪馬臺者也。古雲去樂浪郡境及帶方郡並一萬二千里,在會稽之東,與儋耳相近。漢光武時,遣使入朝,自稱大夫。安帝時,又遣使朝貢,謂之倭奴國。桓、靈之間,其國大亂,遞相攻伐,歷年無主。有女子名卑彌呼,能以鬼道惑眾,
 於是國人共立為王。有男弟,佐卑彌理國。其王有侍婢千人,罕有見其面者,唯有男子二人給王飲食,通傳言語。



 其王有宮室樓觀,城柵皆持兵守衛,為法甚嚴。自魏至於齊、梁,代與中國相通。



 開皇二十年,倭王姓阿每,字多利思北孤,號阿輩雞彌,遣使詣闕。上令所司訪其風俗。使者言倭王以天為兄,以日為弟,天未明時出聽政,跏趺坐,日出便停理務,雲委我弟。高祖曰:「此太無義理。」於是訓令改之。王妻號雞彌,後宮有女六七百人。名太子為利歌彌多弗利。無城郭。內官有十二等:一曰大德,次小德,次大仁,次小仁,次大義,次小義,次大禮,次小禮,
 次大智,次小智,次大信,次小信,員無定數。有軍尼一百二十人,猶中國牧宰。八十戶置一伊尼翼,如今裡長也。十伊尼翼屬一軍尼。其服飾,男子衣裙襦,其袖微小,履如屨形,漆其上,系之於腳。人庶多跣足。不得用金銀為飾。故時衣橫幅,結束相連而無縫。頭亦無冠,但垂發於兩耳上。至隋,其王始制冠,以錦彩為之,以金銀鏤花為飾。婦人束發於後,亦衣裙襦,裳皆有襈。躭竹為梳,編草為薦,雜皮為表,緣以文皮。有弓、矢、刀、槊、弩、、斧,漆皮為甲,骨為矢鏑。雖有兵,無征戰。其王朝會,必陳設儀仗,奏其國樂。戶可十萬。



 其俗殺人強盜及奸皆死,盜者計贓
 酬物,無財者沒身為奴。自餘輕重,或流或杖。每訊究獄訟,不承引者,以木壓膝,或張強弓,以弦鋸其項。或置小石於沸湯中,令所競者探之,雲理曲者即手爛。或置蛇甕中,令取之,雲曲者即螫手矣。人頗恬靜,罕爭訟,少盜賊。樂有五弦、琴、笛。男女多黥臂點面文身,沒水捕魚。



 無文字,唯刻木結繩。敬佛法,於百濟求得佛經,始有文字。知卜筮,尤信巫覡。



 每至正月一日,必射戲飲酒,其餘節略與華同。好棋博、握槊、樗蒲之戲。氣候溫暖,草木冬青,土地膏腴,水多陸少。以小環掛鷺鶿項,令入水捕魚,日得百餘頭。



 俗無盤俎,藉以解葉,食用手哺之。性質直,有
 雅風。女多男少,婚嫁不取同姓,男女相悅者即為婚。婦入夫家,必先跨犬,乃與夫相見。婦人不淫妒。死者斂以棺郭,親賓就尸歌舞,妻子兄弟以白布制服。貴人三年殯於外,庶人卜日而瘞。及葬,置尸船上,陸地牽之,或以小輿。有阿蘇山,其石無故火起接天者,俗以為異,因行禱祭。有如意寶珠,其色青,大如雞卵,夜則有光,雲魚眼精也。新羅、百濟皆以倭為大國,多珍物,並敬仰之,恆通使往來。



 大業三年,其王多利思北孤遣使朝貢。使者曰:「聞海西菩薩天子重興佛法,故遣朝拜,兼沙門數十人來學佛法。」其國書曰「日出處天子至書日沒處天子無
 恙」



 雲云。帝覽之不悅,謂鴻臚卿曰:「蠻夷書有無禮者,勿復以聞。」明年,上遣文林郎裴清使於倭國。度百濟,行至竹島,南望羅國,經都斯麻國,乃在大海中。又東至一支國,又至竹斯國,又東至秦王國,其人同於華夏,以為夷洲,疑不能明也。



 又經十餘國,達於海岸。自竹斯國以東,皆附庸於倭。倭王遣小德阿輩臺,從數百人,設儀仗,鳴鼓角來迎。後十日,又遣大禮,哥多毗,從二百餘騎郊勞。既至彼都,其王與清相見,大悅,曰:「我聞海西有大隋,禮義之國,故遣朝貢。我夷人僻在海隅,不聞禮義,是以稽留境內,不即相見。今故清道飾館,以待大使,冀聞大
 國惟新之化。」清答曰:「皇帝德並二儀,澤流四海,以王慕化,故遣行人來此宣諭。」既而引清就館。其後清遣人謂其王曰:「朝命既達,請即戒途。」於是設宴享以遣清,復令使者隨清來貢方物。此後遂絕。



 史臣曰:廣穀大川異制,人生其間異俗,嗜欲不同,言語不通,聖人因時設教,所以達其志而通其俗也。九夷所居,與中夏懸隔,然天性柔順,無獷暴之風,雖綿邈山海,而易以道御。夏、殷之代,時或來王。暨箕子避地朝鮮,始有八條之禁,疏而不漏,簡而可久,化之所感,千載不絕。今遼東諸國,或衣服參冠冕之容,或飲食有俎豆之器,
 好尚經術,愛樂文史,游學於京都者,往來繼路,或亡沒不歸。



 非先哲之遺風,其孰能致於斯也?故孔子曰:「言忠信,行篤敬,雖蠻貊之邦行矣。」



 誠哉斯言。其俗之可採者,豈徒楛矢之貢而已乎?自高祖撫有周餘,惠此中國,開皇之末,方事遼左,天時不利,師遂無功。二代承基,志包宇宙,頻踐三韓之域,屢發千鈞之弩。小國懼亡,敢同困獸,兵連不戢,四海騷然,遂以土崩,喪身滅國。



 兵志有之曰:「務廣德者昌,務廣地者亡。」然遼東之地,不列於郡縣久矣。諸國朝正奉貢,無闕於歲時,二代震而矜之,以為人莫若己,不能懷以文德,遽動干戈。



 內恃富強,外思廣
 地,以驕取怨,以怒興師。若此而不亡,自古未之聞也。然則四夷之戒,安可不深念哉!



\end{pinyinscope}