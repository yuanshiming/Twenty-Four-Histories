\article{宋本原跋}

\begin{pinyinscope}

 《隋書》
 自開皇、仁壽時,王劭纂書八十卷,以類相從,定為篇目。至於編年紀傳,並闕其體。唐武德五年,起居舍人令狐德棻奏請修《五代史》。五代謂梁、陳、齊、周、隋也。十二月,詔中書令封德彞、舍人顏師古修《隋史》,綿歷數載,不就而罷。貞觀三年,續詔秘書監魏徵修《隋史》,左僕射房喬總監。徵又奏於中書省置秘書內省,令前中書侍郎顏師古、給事中孔穎達、著作郎許敬宗撰《隋史》。徵總知其務,多所損益,務存簡正。序、論皆徵所作。凡成帝紀五,列傳五十。十年正月壬子,徵等詣闕上之。十五年,又詔左僕射於志寧、太史令李淳風、著作郎韋安仁、符璽郎李延壽同
 修《五代史志》。凡勒成十志三十卷。顯慶元年五月己卯,太尉長孫無忌等詣朝堂上進,詔藏秘閣。後又編第入《隋書》,其實別行,亦呼為《五代史志》。案魏徵本傳,貞觀七年為侍中,十年,《五代史》成,加光祿大夫,進封鄭國公。俄請遜位,拜特進。今諸本並云特進。又《經籍志》四卷,獨雲侍中、鄭國公魏徵撰。《無忌傳》又云,永徽三年,始受詔監修,疑當時先已刊修,無忌因成書而進。今紀傳題以徵,志以無忌,從眾本所載也。紀傳亦有題太子少師許敬宗撰。案《敬宗傳》,貞觀八年,除著作郎,修國史,遷中書舍人。十年,左授洪州司馬。龍朔三年,始拜太子少師。與今錄年月官位不同,疑後人所益。房喬、志寧初並受詔。又《李延壽傳》云,被詔與著作佐郎敬播同修《五代史志》。按延壽貞觀三年與顏師古同被敕修《隋史》,其年以內憂去職。今諸本並不載喬等名位。《天文》《律歷》《五行》三志,皆淳風獨作。《五行志序》,諸本云褚遂良作。案本傳未嘗受詔撰述,疑只為一序,今故略其名氏。天聖二年五月十一日上。御藥供奉藍元用奉
 傳聖旨,齎禁中《隋書》一部,付崇文院。至六月五日,敕差官校勘,時命臣綬、臣燁提點,右正言、直史館張觀等校勘。觀尋為度支判官,續命黃鑒代之。仍內出版式雕造。



\end{pinyinscope}