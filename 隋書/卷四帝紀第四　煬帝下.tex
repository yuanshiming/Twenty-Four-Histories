\article{卷四帝紀第四 煬帝下}

\begin{pinyinscope}

 八年春正月辛巳,大軍集於涿郡。以兵部尚書段文振為左候衛大將軍。壬午,下詔曰:天地大德,降繁霜於秋令;聖哲至仁,著甲兵於刑典。故知造化之有肅殺,義在無私;帝王之用干戈,蓋非獲已。版泉、丹浦,莫匪龔行,取亂覆昏,咸由順動。



 況乎甘野誓師,夏開承大禹之業;商郊問罪,周發成文王之志。永監前載,屬當朕躬。粵我有
 隋,誕膺靈命,兼三才而建極,一六合而為家。提封所漸,細柳、盤桃之外,聲教爰暨,紫舌、黃枝之域,遠至邇安,罔不和會。功成治定,於是乎在。



 而高麗小醜,迷昏不恭,崇聚勃碣之間,薦食遼「之境。雖復漢魏誅戳,巢窟暫傾,亂離多阻,種落還集。萃川藪於往代,播實繁以迄今,眷彼華壤,剪為夷類。歷年永久,惡稔既盈,天道禍淫,亡徵已兆。亂常敗德,非可勝圖,掩慝懷奸,唯日不足。移告之嚴,未嘗面受,朝覲之禮,莫肯躬親。誘納亡叛,不知紀極,充斥邊垂,亟勞烽候,關柝以之不靜,生人為之廢業。在昔薄伐,已漏天網,既緩前擒之戮,未即後服之誅。曾不
 懷恩,翻為長惡,乃兼契丹之黨,虔劉海戍,習靺鞨之服,侵軼遼西。又青丘之表,咸修職貢,碧海之濱,同稟正朔,遂復奪攘琛」,遏絕往來,虐及弗辜,誠而遇禍。輶軒奉使,爰暨海東,旌節所次,途經籓境,而擁塞道路,拒絕王人,無事君之心,豈為臣之禮!此而可忍,孰不可容!且法令苛酷,賦斂煩重,強臣豪族,咸執國鈞,朋黨比周,以之成俗,賄貨如市,冤枉莫申。重以仍歲災兇,比屋饑饉,兵戈不息,徭役無期,力竭轉輸,身填溝壑。百姓愁苦,爰誰適從?境內哀惶,不勝其弊。回首面內,各懷性命之圖,黃發稚齒,咸興酷毒之嘆。



 省俗觀風,爰屈幽朔,吊人問罪,
 無俟再駕。於是親總六師,用申九伐,拯厥阽危,協從天意,殄茲逋穢,克嗣先謨。今宜援律啟行,分麾屈路,掩勃澥而雷震,歷夫餘以電掃。比戈按甲,誓旅而後行,三令五申,必勝而後戰。左第一軍可鏤方道。



 第二軍可長岑道,第三軍可海冥道,第四軍可蓋馬道,第五軍可建安道,第六軍可南蘇道,第七軍可遼東道,第八軍可玄菟道,第九軍可扶餘道,第十軍可朝鮮道,第十一軍可沃沮道,第十二軍可樂浪道,右第一軍可黏蟬道,第二軍可含資道,第三軍可渾彌道,第四軍可臨屯道,第五軍可候城道,第六軍可提奚道,第七軍可踏頓道,第八軍
 可肅慎道,第九軍可碣石道,第十軍可東暆道,第十一軍可帶方道,第十二軍可襄平道。凡此眾軍,先奉廟略,駱驛引途,總集平壤。莫非如豺如貔之勇,百戰百勝之雄,顧眄則山嶽傾頹,叱吒則風雲騰鬱,心德攸同,爪牙斯在。朕躬馭元戎,為其節度,涉遼而東,循海之右,解倒懸於遐裔,問疾苦於遺黎。其外輕齎游闕,隨機赴響,卷甲銜枚,出其不意。又滄海道軍舟艫千里,高帆電逝,巨艦雲飛,橫斷浿江,逕造平壤,島嶼之望斯絕,坎井之路已窮。其餘被發左衽之人,控弦待發,微盧彭濮之旅,不謀同辭。杖順臨逆,人百其勇,以此眾戰,勢等摧枯。



 然則
 王者之師,義存止殺,聖人之教,必也勝殘。天罰有罪,本在元惡,人之多僻,脅從罔治。若高元泥首轅門,自歸司寇,即宜解縛焚櫬,弘之以恩。其餘臣人歸朝奉順,咸加慰撫,各安生業,隨才任用,無隔夷夏。營壘所次,務在整肅,芻堯有禁,秋毫勿犯,布以恩宥,喻以禍福。若其同惡相濟,抗拒官軍,國有常刑,俾無遺類。明加曉示,稱朕意焉。



 總一百一十三萬三千八百,號二百萬,其餽運者倍之。癸未,第一軍發,終四十日,引師乃盡,旌旗亙千里。近古出師之盛,未之有也。乙未,以右候衛大將軍衛玄為刑部尚書。甲辰,內史令元壽卒。二月甲寅,詔曰:「朕觀風
 燕裔,問罪遼濱。文武協力,爪牙思奮,莫不執銳勤王,舍家從役,罕蓄倉廩之資,兼損播殖之務。朕所以夕惕愀然,慮其匱乏。雖復素飽之眾,情在忘私,悅使之人,宜從其厚。



 諸行從一品以下,佽飛募人以上家口,郡縣宜數存問。若有糧食少,皆宜賑給;或雖有田疇,貧弱不能自耕種,可於多丁富室勸課相助。使夫居者有斂積之豐,行役無顧後之慮。」壬戌,司空、京兆尹、光祿大夫觀王雄薨。三月辛卯,兵部尚書、左候衛大將軍段文振卒。癸巳,上御師。甲午,臨戎於遼水橋。戊戌,大軍為賊所拒,不果濟。右屯衛大將軍、左光祿大夫麥鐵杖,武賁郎將錢
 士雄、孟金叉等,皆死之。甲午,車駕渡遼。大戰於東岸,擊賊破之,進圍遼東。乙未,大頓,見二大鳥,高丈餘,皜身硃足,游泳自若。上異之,命工圖寫,並立銘頌。五月壬午,納言楊達卒。於時諸將各奉旨,不敢越機。既而高麗各城守,攻之不下。六月己未,幸遼東,責怒諸將。止城西數里,御六合城。七月壬寅,宇文述等敗績於薩水,右屯衛將軍辛世雄死之。九軍並陷,將帥奔還亡者二千餘騎。癸卯,班師。九月庚辰,上至東都。己丑,詔曰:「軍國異容,文武殊用,匡危拯難,則霸德攸興,化人成俗,則王道斯貴。時方撥亂,屠販可以登朝,世屬隆平,經術然後升仕。豐都
 爰肇,儒服無預於周行,建武之朝,功臣不參於吏職。自三方未一,四海交爭,不遑文教,唯尚武功。設官分職,罕以才授,班朝治人,乃由勛敘,莫非拔足行陣,出自勇夫,斅學之道,既所不習,政事之方,故亦無取。是非暗於在己,威福專於下吏,貪冒貨賄,不知紀極,蠢政害民,實由於此。自今已後,諸授勛官者,並不得回授文武職事,庶遵彼更張,取類於調瑟,求諸名制,不傷於美錦。若吏部輒擬用者,御史即宜糾彈。」冬十月甲寅,工部尚書宇文愷卒。十一月己卯,以宗女華容公主嫁於高昌王。辛巳,光祿大夫韓壽卒。甲申,敗將宇文述、於仲文等並除名
 為民,斬尚書右丞劉士龍以謝天下。是歲,大旱,疫,人多死,山東尤甚。密詔江、淮南諸郡閱視民間童女,姿質端麗者,每歲貢之。



 九年春正月丁丑,徵天下兵,募民為驍果,集於涿郡。壬午,賊帥杜彥冰、王潤等陷平原郡,大掠而去。辛卯,置折沖、果毅、武勇、雄武等郎將官,以領驍果。



 乙未,平原李德逸聚眾數萬,稱「阿舅賊」,劫掠山東。靈武白榆妄稱「奴賊」,劫掠牧馬,北連突厥,隴右多被其患。遣將軍範貴討之,連年不能克。戊戌,大赦。



 己亥,遣代王侑、刑部尚書衛玄鎮京師。辛丑,以右驍騎將軍李渾為右驍衛大將軍。



 二
 月己未,濟北人韓進洛聚眾數萬為群盜。壬午,復宇文述等官爵。又徵兵討高麗。



 三月丙子,濟陰人孟海公起兵為盜,眾至數萬。丁丑,發丁男十萬城大興。戊寅,幸遼東。以越王侗、民部尚書樊子蓋留守東都。庚子,北海人郭方預聚徒為盜,自號盧公,眾至三萬,攻陷郡城,大掠而去。夏四月庚午,車駕渡遼。壬申,遣宇文述、楊義臣趣平壤。五月丁丑,熒惑入南斗。己卯,濟北人甄寶車聚眾萬餘,寇掠城邑。六月乙巳,禮部尚書楊玄感反於黎陽。丙辰,玄感逼東都。河南贊務裴弘策拒之,反為賊所敗。戊辰,兵部侍郎斛斯政奔於高麗。庚午,上班師。高麗犯
 後軍,敕右武衛大將軍李景為後拒。遣左翊衛大將軍宇文述、左候衛將軍屈突通等馳傳發兵,以討玄感。秋七月己卯,令所在發人城縣府驛。癸未,餘杭人劉元進舉兵反,眾至數萬。八月壬寅,左翊衛大將軍宇文述等破楊玄感於閿鄉,斬之,餘黨悉平。



 癸卯,吳人硃燮、晉陵人管崇擁眾十萬餘,自稱將軍,寇江左。甲辰,制驍果之家蠲免賦役。丁未,詔郡縣城去道過五里已上者,徙就之。戊申,制盜賊籍沒其家。



 乙卯,賊帥陳瑱等眾三萬攻陷信安郡。辛酉,司農卿、光祿大夫、葛國公趙元淑以罪伏誅。九月己卯,濟陰人吳海流、東海人彭孝才並舉兵
 為盜,眾數萬。庚辰,賊帥梁慧尚率眾四萬陷蒼梧郡。甲午,車駕次上谷,以供費不給,上大怒,免太守虞荷等官。丁酉,東陽人李三兒、向但子舉兵作亂,眾至萬餘。閏月己巳,幸博陵。



 庚午,上謂侍臣曰:「朕昔從先朝周旋於此,年甫八歲,日月不居,倏經三紀,追惟平昔,不可復希!」言未卒,流涕嗚咽,侍衛者皆泣下沾襟。冬十月丁丑,賊帥呂明星率眾數千圍東郡,武賁郎將費青奴擊斬之。乙酉,詔曰:「博陵昔為定州,地居沖要,先皇歷試所基,王化斯遠,故以道冠《豳風》,義高姚邑。朕巡撫氓庶,爰屈茲邦,瞻望郊廛,緬懷敬止,思所以宣播德澤,覃被下人,崇紀
 顯號,式光令緒。可改博陵為高陽郡。赦境內死罪已下。給復一年。」於是召高祖時故吏,皆量材授職。壬辰,以納言蘇威為開府儀同三司。硃燮、管崇推劉元進為天子。遣將軍吐萬緒、魚俱羅討之,連年不能克。齊人孟讓、王薄等眾十餘萬,據長白山,攻剽諸郡,清河賊張金稱眾數萬,渤海賊帥格謙自號燕王,孫宣雅自號齊王,眾各十萬,山東苦之。丁亥,以右候衛將軍郭榮為右候衛大將軍。十一月己酉,右候衛將軍馮孝慈討張金稱於清河,反為所敗,孝慈死之。十二月甲申,車裂玄感弟朝請大夫積善及黨與十餘人,仍焚而揚之。丁亥,扶風人向
 海明舉兵作亂,稱皇帝,建元白烏。



 遣太僕卿楊義臣擊破之。



 十年春正月甲寅,以宗女為信義公主,嫁於突厥曷娑那可汗。二月辛未,詔百僚議伐高麗,數日無敢言者。戊子,詔曰:「竭力王役,致身戎事,咸由徇義,莫匪勤誠。委命草澤,棄骸原野,興言念之,每懷愍惻。往年出車問罪,將屆遼濱,廟算勝略,具有進止。而諒惛兇,罔識成敗,高熲愎很,本無智謀,臨三軍猶兒戲,視人命如草芥,不遵成規,坐貽撓退,遂令死亡者眾,不及埋藏。今宜遣使人分道收葬,設祭於遼西郡,立道場一所。恩加泉壤,庶弭窮
 魂之冤,澤及枯骨,用弘仁者之惠。」辛卯,詔曰:黃帝五十二戰,成湯二十七征,方乃德施諸侯,令行天下。盧芳小盜,漢祖尚且親戎,隗囂餘燼,光武猶自登隴,豈不欲除暴止戈,勞而後逸者哉!朕纂成寶業,君臨天下,日月所照,風雨所沾,孰非我臣,獨隔聲教。蕞爾高麗,僻居荒表,鴟張狼噬,侮慢不恭,抄竊我邊陲,侵軼我城鎮。是以去歲出軍,問罪遼碣,殪長蛇於玄菟,戮封豕於襄平。扶餘眾軍,風馳電逝,追奔逐北,徑逾浿水,滄海舟楫,沖賊腹心,焚其城郭,污其宮室。高元伏金質泥首,送款軍門,尋請入朝,歸罪司寇。朕以許其改過,乃詔班師。而長惡靡悛,
 宴安鴆毒,此而可忍,孰不可容!便可分命六師,百道俱進。朕當親執武節,臨御諸軍,秣馬丸都,觀兵遼水,順天誅於海外,救窮民於倒懸。征伐以正之,明德以誅之,止除元惡,餘無所問。若有識存亡之分,悟安危之機,翻然北首,自求多福;必其同惡相濟,擠拒王師,若火燎原,刑茲無赦。有司便宜宣布,咸使知聞。



 丁酉,扶風人唐弼舉兵反,眾十萬,推李弘為天子,自稱唐王。三月壬子,行幸涿郡。癸亥,次臨渝宮,親御戎服,祃祭黃帝,斬叛軍者以釁鼓。夏四月辛未,彭城賊張大彪聚眾數萬,保懸薄山為盜。遣榆林太守董純擊破,斬之。甲午,車駕次北平。五
 月庚子,詔舉郡孝悌廉潔各十人。壬寅,賊帥宋世謨陷瑯邪郡。庚申,延安人劉迦論舉兵反,自稱皇王,建元大世。六月辛未,賊帥鄭文雅、林寶護等眾三萬,陷建安郡,太守楊景祥死之。秋七月癸丑,車駕次懷遠鎮。乙卯,曹國遣使貢方物。甲子,高麗遣使請降,囚送斛斯政。上大悅。八月己巳,班師。庚午,右衛大將軍、左光祿大夫鄭榮卒。冬十月丁卯,上至東都。己丑,還京師。十一月丙申,支解斛斯政於金光門外。乙巳,有事於南郊。己酉,賊帥司馬長安破長平郡。



 乙卯,離石胡劉苗王舉兵反,自稱天子,以其弟六兒為永安王,眾至數萬。將軍潘長文討之,
 不能克。是月,賊帥王德仁擁眾數萬,保林慮山為盜。十二月壬申,上如東都。其日,大赦天下。戊子,入東都。庚寅,賊帥孟讓眾十餘萬,據都梁宮。



 遣江都郡丞王世充擊破之,盡虜其眾。



 十一年春正月甲午朔,大宴百僚。突厥、新羅、靺鞨、畢大辭、訶咄、傳越、烏那曷、波臘、吐火羅、俱慮建、忽論、靺鞨、訶多、沛汗、龜茲、疏勒、于闐、安國、曹國、何國、穆國、畢、衣密、失範延、伽折、契丹等國並遣使朝貢。戊戌,武賁郎將高建毗破賊帥顏宣政於齊郡,虜男女數千口。乙卯,大會蠻夷,設魚龍曼延之樂,頒賜各有差。二月戊辰,賊帥楊仲
 緒率眾萬餘,攻北平,滑公李景破斬之。



 庚午,詔曰:「設險守國,著自前經,重門御暴,事彰往策,所以宅土寧邦,禁邪固本。而近代戰爭,居人散逸,田疇無伍,郛郭不修,遂使游惰實繁,寇襄未息。



 今天下平一,海內晏如,宜令人悉城居,田隨近給,使強弱相容,力役兼濟,穿窬無所厝其奸宄,萑蒲不得聚其逋逃。有司具為事條,務令得所。」丙子,上谷人王須拔反,自稱漫天王,國號燕,賊帥魏刁兒自稱歷山飛,眾各十餘萬,北連突厥,南寇趙。五月丁酉,殺右驍衛大將軍、光祿大夫、郕公李渾,將作監、光祿大夫李敏,並族滅其家。癸卯,賊帥司馬長安破西河郡。
 己酉,幸太原,避暑汾陽宮。秋七月己亥,淮南人張起緒舉兵為盜,眾至三萬。辛丑,光祿大夫、右御衛大將軍張壽卒。八月乙丑,巡北塞。戊辰,突厥始畢可汗率騎數十萬,謀襲乘輿,義成公主遣使告變。壬申,車駕馳幸雁門。癸酉,突厥圍城,官軍頻戰不利。上大懼,欲率精騎潰圍而出,民部尚書樊子蓋固諫乃止。齊王暕以後軍保於崞縣。甲申,詔天下諸郡募兵,於是守令各來赴難。九月甲辰,突厥解圍而去。丁未,曲赦太原、雁門郡死罪已下。冬十月壬戌,上至於東都。丁卯,彭城人魏騏麟聚眾萬餘為盜,寇魯郡。壬申,賊帥盧明月聚眾十餘萬,寇陳、汝
 間。東海賊帥李子通擁眾度淮,自號楚王,建元明政,寇江都。十一月乙卯,賊帥王須拔破高陽郡。十二月戊寅,有大流星如斛,墜明月營,破其沖車。庚辰,詔民部尚書樊子蓋發關中兵,討絳郡賊敬盤陀、柴保昌等,經年不能克。譙郡人硃粲擁眾數十萬,寇荊襄,僭稱楚帝,建元昌達,漢南諸郡多為所陷焉。



 十二年春正月甲午,雁門人翟松柏起兵於靈丘,眾至數萬,轉攻傍縣。二月己未,真臘國遣使貢方物。甲子夜,有二大鳥似雕,飛入大業殿,止於御幄,至明而去。癸亥,東海賊盧公暹率眾萬餘,保於蒼山。夏四月丁巳,顯陽
 門災。癸亥,魏刁兒所部將甄翟兒復號歷山飛,眾十萬,轉寇太原。將軍潘長文討之,反為所敗,長文死之。五月丙戌朔,日有蝕之,既。癸巳,大流星隕於吳郡,為石。壬午,上於景華宮徵求螢火,得數斛,夜出游山,放之,光遍巖谷。秋七月壬戌,民部尚書、光祿大夫濟北公樊子蓋卒。甲子,幸江都宮,以越王侗、光祿大夫段達、太府卿元文都、檢校民部尚書韋津、右武衛將軍皇甫無逸、右司郎盧楚等總留後事。奉信郎崔民象以盜賊充斥,於建國門上表,諫不宜巡幸。上大怒,先解其頤,乃斬之。戊辰,馮翊人孫華自號總管,舉兵為盜。高涼通守洗珤徹舉兵
 作亂,嶺南溪洞多應之。



 己巳,熒惑守羽林,月餘乃退。車駕次汜水,奉信郎王愛仁以盜賊日盛,諫上請還西京。上怒,斬之而行。八月乙巳,賊帥趙萬海眾數十萬,自恆山寇高陽。壬子,有大流星如斗,出王良閣道,聲如隤墻。癸丑,大流星如彳育又,出羽林。九月丁酉,東海人杜揚州、沈覓敵等作亂,眾至數萬。右御衛將軍陳棱擊破之。戊午,有二枉矢出北斗魁,委曲蛇形,注于南斗。壬戌,安定人荔非世雄殺臨涇令,舉兵作亂,自號將軍。冬十月己丑,開府儀同三司、左翊衛大將軍、光祿大夫、許公宇文述薨。十二月癸未,鄱陽賊操天成舉兵反,自號元興
 王,建元始興,攻陷豫章郡。



 乙酉,以右翊衛大將軍來護兒為開府儀同三司、行左翊衛大將軍。壬辰,鄱陽人林士弘自稱皇帝,國號楚,建元太平,攻陷九江、廬陵郡。唐公破甄翟兒於西河,擄男女數千口。



 十三年春正月壬子,齊郡賊杜伏威率眾渡淮,攻陷歷陽郡。丙辰,勃海賊竇建德設壇於河間之樂壽,自稱長樂王,建元丁丑。辛巳,賊帥徐圓朗率眾數千,破東平郡。弘化人劉企成聚眾萬餘人為盜,傍郡苦之。二月壬午,朔方人梁師都殺郡丞唐世宗,據郡反,自稱大丞相。遣銀青光祿大夫張世隆擊之,反為所敗。戊子,賊帥王子
 英破上谷郡。己丑,馬邑校尉劉武周殺太守王仁恭,舉兵作亂,北連突厥,自稱定楊可汗。庚寅,賊帥李密、翟讓等陷興洛倉。越王侗遣武賁郎將劉長恭、光祿少卿房崱擊之,反為所敗,死者十五六。庚子,李密自號魏公,稱元年,開倉以振群盜,眾至數十萬,河南諸郡相繼皆陷焉。壬寅,劉武周破武賁郎將王智辯於桑乾鎮,智辯死之。三月戊午,廬江人張子路舉兵反。遣右御衛將軍陳棱討平之。丁丑,賊帥李通德眾十萬寇廬江,左屯衛將軍張鎮州擊破之。夏四月癸未,金城校尉薛舉率眾反,自稱西秦霸王,建元秦興,攻陷隴右諸郡。己丑,賊帥孟
 讓夜入東都外郭,燒豐都市而去。癸巳,李密陷回洛東倉。丁酉,賊帥房憲伯陷汝陰郡。是月,光祿大夫裴仁基、淮陽太守趙佗等並以眾叛歸李密。五月辛酉,夜有流星如彳育又墜於江都。甲子,唐公起義師於太原。丙寅,突厥數千寇太原,唐公擊破之。秋七月壬子,熒惑守積尸。丙辰,武威人李軌舉兵反,攻陷河西諸郡,自稱涼王,建元安樂。八月辛巳,唐公破武牙郎將宋老生於霍邑,斬之。九月己丑,帝括江都人女寡婦以配從兵。是月,武陽郡丞元寶藏以郡叛歸李密,與賊帥李文相攻陷黎陽倉。



 彗星見於營室。冬十月丁亥,太原楊世洛聚眾萬餘人,
 寇掠城邑。丙申,羅令蕭銑以縣反,鄱陽人董景珍以反,迎銑於羅縣,號為梁王,攻陷傍郡。戊戌,武賁郎將高毗敗濟北郡賊甄寶車於監山。十一月丙辰,唐公入京師。辛酉,遙尊帝為太上皇,立代王侑為帝,改元義寧。上起宮丹陽,將遜於江左。有烏鵲來巢幄帳,驅不能止。熒惑犯太微。有石自江浮入於揚子。日光四散如流血。上甚惡之。



 二年三月,右屯衛將軍宇文化及,武賁郎將司馬德戡、元禮,監門直閣裴虔通,將作少監宇文智及,武勇郎將趙行樞,鷹揚郎將孟景,內史舍人元敏,符璽郎李覆、牛
 方裕,千牛左右李孝本、弟孝質,直長許弘仁、薛世良,城門郎唐奉義,醫正張愷等,以驍果作亂,入犯宮闈。上崩於溫室,時年五十。蕭後令宮人撤床簀為棺以埋之。化及發後,右御衛將軍陳棱奉梓宮於成象殿,葬吳公臺下。發斂之始,容貌若生,眾咸異之。大唐平江南之後,改葬雷塘。



 初,上自以籓王,次不當立,每矯情飾行,以釣虛名,陰有奪宗之計。時高祖雅信文獻皇后,而性忌妾媵。皇太子勇內多嬖幸,以此失愛。帝後庭有子,皆不育之,示無私寵,取媚於後。大臣用事者,傾心與交。中使至第,無貴賤,皆曲承顏色,申以厚禮。婢僕往來者,無不稱其
 仁孝。又常私入宮掖,密謀於獻後,楊素等因機構扇,遂成廢立。自高祖大漸,暨諒闇之中,烝淫無度,山陵始就,即事巡游。



 以天下承平日久,士馬全盛,慨然慕秦皇、漢武之事,乃盛治宮室,窮極侈靡,召募行人,分使絕域。諸蕃至者,厚加禮賜,有不恭命,以兵擊之。盛興屯田於玉門、柳城之外。課天下富室,益市武馬,匹直十餘萬,富強坐是凍餒者十家而九。帝性多詭譎,所幸之處,不欲人知。每之一所,輒數道置頓,四海珍羞殊味,水陸必備焉,求市者無遠不至。郡縣官人,競為獻食,豐厚者進擢,疏儉者獲罪。奸吏侵漁,內外虛竭,頭會箕斂,人不聊生。於
 時軍國多務,日不暇給,帝方驕怠,惡聞政事,冤屈不治,奏請罕決。又猜忌臣下,無所專任,朝臣有不合意者,必構其罪而族滅之。故高熲、賀若弼先皇心膂,參謀帷幄,張衡、李金才籓邸惟舊,績著經綸,或惡其直道,或忿其正議,求其無形之罪,加以刎頸之誅。其餘事君盡禮,謇謇匪躬,無辜無罪,橫受夷戮者,不可勝紀。政刑弛紊,賄貨公行,莫敢正言。道路以目。



 六軍不息,百役繁興,行者不歸,居者失業。人饑相食,邑落為墟,上不之恤也。



 東西游幸,靡有定居,每以供費不給,逆收數年之賦。所至唯與後宮流連耽湎,惟日不足,招迎姥媼,朝夕共肆醜言,
 又引少年,令與宮人穢亂,不軌不遜,以為娛樂。區宇之內,盜賊蜂起,劫掠從官,屠陷城邑,近臣互相掩蔽,隱賊數不以實對。



 或有言賊多者,輒大被詰責。各求茍免,上下相蒙,每出師徒,敗亡相繼。戰士盡力,必不加賞,百姓無辜,咸受屠戮。黎庶憤怨,天下土崩,至於就擒,而猶未之寤也。



 史臣曰:煬帝爰在弱齡,早有令聞,南平吳會,北卻匈奴,昆弟之中,獨著聲績。於是矯情飾貌,肆厥奸回,故得獻後鐘心,文皇革慮,天方肇亂,遂登儲兩,踐峻極之崇基,承丕顯之休命。地廣三代,威振八紘,單于頓顙,越裳重
 譯。赤仄之泉,流溢於都內,紅腐之粟,委積於塞下。負其富強之資,思逞無厭之欲,狹殷周之制度,尚秦漢之規摹。恃才矜己,傲狠明德,內懷險躁,外示凝簡,盛冠服以飾其奸,除諫官以掩其過。淫荒無度,法令滋章,教絕四維,刑參五虐,鋤誅骨肉,屠剿忠良,受賞者莫見其功,為戮者不知其罪。驕怒之兵屢動,土木之功不息。頻出朔方,三駕遼左,旌旗萬里,征稅百端,猾吏侵漁,人不堪命。乃急令暴條以擾之,嚴刑峻法以臨之,甲兵威武以董之,自是海內騷然,無聊生矣。俄而玄感肇黎陽之亂,匈奴有雁門之圍,天子方棄中土,遠之揚越。奸宄乘釁,強
 弱相陵,關梁閉而不通,皇輿往而不反。加之以師旅,因之以饑饉,流離道路,轉死溝壑,十八九焉。於是相聚萑蒲,胃毛而起,大則跨州連郡,稱帝稱王,小則千百為群,攻城剽邑,流血成川澤,死人如亂麻,炊者不及析骸,食者不遑易子。茫茫九土,並為麋鹿之場,心棄心棄黔黎,俱充蛇豕之餌。四方萬里,簡書相續,猶謂鼠竊狗盜,不足為虞,上下相蒙,莫肯念亂,振蜉蝣之羽,窮長夜之樂。土崩魚爛,貫盈惡稔,普天之下,莫匪仇讎,左右之人,皆為敵國。終然不悟,同彼望夷,遂以萬乘之尊,死於一夫之手。億兆靡感恩之士,九牧無勤王之師。子弟同就誅夷,骸
 骨棄而莫掩,社稷顛隕,本枝殄絕,自肇有書契以迄於茲,宇宙崩離,生靈塗炭,喪身滅國,未有若斯之甚也。《書》曰:「天作孽,猶可違,自作孽,不可逭。」《傳》曰:「吉兇由人,祆不妄作。」又曰:「兵猶火也,不戢將自焚。」觀隋室之存亡,斯言信而有徵矣!



\end{pinyinscope}