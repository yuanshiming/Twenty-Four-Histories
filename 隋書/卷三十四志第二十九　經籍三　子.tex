\article{卷三十四志第二十九 經籍三 子}

\begin{pinyinscope}

 《晏子春秋》七卷齊大夫晏嬰撰。



 《曾子》二卷目一卷。魯國曾參撰。



 《子思子》七卷魯穆公師孔人及撰。



 《公孫尼子》一卷尼,似孔子弟子。



 《孟子》十四卷齊卿孟軻撰,趙岐注。



 《孟子》七卷鄭玄注。



 《孟子》七卷劉熙注。梁有《孟子》九卷,綦毋邃撰,亡。



 《孫卿子》十二卷楚蘭陵令荀況撰。梁有王孫子一卷,亡。



 《董子》一卷戰國時董無心撰。



 《魯連子》五卷、錄一卷魯連,齊人,不仕,稱為先生。



 《新語》二卷陸賈撰。



 《賈子》十卷錄一卷。漢梁太傅賈誼撰。



 《鹽鐵論》十卷漢廬江府丞桓寬撰。



 《新序》三十卷錄一
 卷。劉向撰。



 《說苑》二十卷劉向撰。



 《揚子法言》十五卷、解一卷揚雄撰,李軌注。梁有《揚子法言》六卷,侯苞注。亡。



 《揚子法言》十三卷宋衷注。



 《揚子太玄經》九卷宋衷注。梁有《揚子太玄經》九卷,揚雄自作章句,亡。



 《揚子太玄經》十卷陸績、宋衷注。



 《揚子太玄經》十卷蔡文邵注。梁有《揚子太玄經》十四卷,虞翻注;《揚子太玄經》十三卷,陸凱注;《揚子太玄經》七卷,王肅注。亡。



 《桓子新論》十七卷後漢六安丞桓譚撰。



 《潛夫論》十卷後漢處士王符撰。梁有王逸《正部論》八卷,後漢侍中王逸撰;《後序》十二卷,後漢司隸校尉應奉撰,《周生子要論》一卷,錄一卷,魏侍中周生烈撰。亡。



 《申鑒》五卷荀悅撰。



 《魏子》三卷後漢會稽人魏朗撰。梁有《文檢》六卷,似後漢末人作,亡。



 《牟子》二卷後漢太尉牟融撰。



 《典論》五卷魏文帝撰。



 《徐氏中論》六卷魏太子文學徐幹撰,梁目一卷。



 《王子正論》十卷王肅撰。梁有《去伐論集》三卷,王粲撰。亡。



 《杜氏體論》四卷魏幽州刺史杜恕撰。梁有《新書》五卷,王基
 撰;《周子》九卷,吳中書郎周昭撰。亡。



 《顧子新語》十二卷吳太常顧譚撰。《通語》十卷,晉尚書左丞殷興撰;《典語》十卷、《典語別》二卷,並吳中夏督陸景撰。亡。



 《譙子法訓》八卷譙周撰。梁有《譙子五教志》五卷,亡。



 《袁子正論》十九卷袁準撰。梁又有《袁子正書》二十五卷,袁準撰;《孫氏成敗志》三卷,孫毓撰;《古今通論》二卷,王嬰撰;《蔡氏化清經》十卷,松滋令蔡洪撰;《通經》二卷,晉丞相從事中郎王長文撰。



 《新論》十卷晉散騎常侍夏侯湛撰。梁有《楊子物理論》十六卷,《楊子大元經》十四卷,並晉徵士楊泉撰;《新論》十卷,晉金紫光祿大夫華譚撰;《梅子新論》一卷。亡。



 《志林新書》三十卷虞喜撰。梁有《廣林》二十四卷,又《後林》十卷,虞喜撰;《乾子》十八卷,干寶撰;《閎論》二卷,晉江州從事蔡韶撰;《顧子》十卷,晉揚州主簿顧夷撰。亡。



 《要覽》十卷晉郡儒林祭酒呂竦撰。



 《正覽》六卷梁太子詹事周舍撰。梁有《三統五德論》二卷,曹思文撰,亡。



 《諸葛武侯集誡》二卷《眾賢誡》十三卷《女篇》一卷《女鑒》一卷《婦人訓誡集》十一卷《娣
 姒訓》一卷《曹大家女誡》一卷《貞順志》一卷右六十二部,五百三十卷。通計亡書,合六十七部,六百九卷。



 儒者,所以助人君明教化者也。聖人之教,非家至而戶說,故有儒者宣而明之。其大抵本於仁義及五常之道,黃帝、堯、舜、禹、湯、文、武,咸由此則。《周官》:太宰以九兩系邦國之人,其四曰儒是也。其後陵夷衰亂,儒道廢闕。仲尼祖述前代,修正六經,三千之徒,並受其義。至於戰國,孟軻、子思、荀卿之流,宗而師之,各有著述,發明其指。所謂中庸之教,百王不易者也。俗儒為之,不顧其本,茍欲嘩眾,多設問難,便辭巧說,亂其大體,致令學者難曉,故曰「
 博而寡要」。



 《鬻子》一卷周文王師鬻熊撰。



 《老子道德經》二卷周柱下史李耳撰。漢文帝時河上公注。梁有戰國時河上丈人注《老子經》二卷,漢長陵三老丘望之注《老子》二卷,《漢》徵士嚴遵注《老子》二卷,虞翻注《老子》二卷,亡。



 《老子道德經》二卷王弼注。梁有《老子道德經》二卷,張嗣注;《老子道德經》二卷,蜀才注。



 亡。



 《老子道德經》二卷鐘會注。梁有《老子道德經》二卷,晉太傅羊祜解釋;《老子經》二卷,東晉江州刺史王尚述注;《老子》二卷,晉郎中程韶集解;《老子》二卷,邯鄲氏注;《老子》二卷,常氏傳;《老子》二卷,孟氏注;《老子》二卷,盈氏注。亡。



 《老子道德經》二卷、音一卷晉尚書郎孫登注。



 《老子道德經》二卷劉仲融注。梁有《老子道德經》二卷,巨生解;《老子道德經》二卷,晉西中郎將袁真注;《老子道德經》二卷,張憑注;《老子道德經》二卷,釋惠琳注;《老子道德經》二卷,釋惠嚴注;《老子道德經》二卷,王玄載注。亡。



 《老子道德經》二卷盧景裕撰。



 《老子音》一卷李軌撰。梁有《老子音》一卷,
 晉散騎常侍戴逵撰,亡。



 《老子》四卷梁曠撰。



 《老子指歸》十一卷嚴遵注。



 《老子指趣》三卷蠙丘望之撰。



 《老子義綱》一卷顧歡撰。梁有《老子道德論》二卷,何晏撰;《老子序決》一卷,葛仙公撰;《老子雜論》一卷,何、王等注;《老子私記》十卷,梁簡文帝撰;《老子玄示》一卷,韓壯撰;《老子玄譜》一卷,晉柴桑令劉遺民撰;《老子玄機》三卷,宗塞撰;《老子幽易》五卷,又《老子志》一卷,山琮撰。亡。



 《老子義疏》一卷顧歡撰。梁有《老子義疏》一卷,釋慧觀撰,亡。



 《老子義疏》五卷孟智周私記。



 《老子義疏》四卷韋處玄撰。



 《老子講疏》六卷梁武帝撰。



 《老子義疏》九卷戴詵撰。



 《老子節解》二卷《老子章門》一卷《文子》十二卷文子,老子弟子。《七略》有九篇,梁《七錄》十卷,亡。



 《鶡冠子》三卷楚之隱人。



 《列子》八卷鄭之隱人列禦寇撰,東晉光祿勛張湛注。



 《莊子》二十卷梁漆園吏莊周撰,晉散騎常侍向秀注。本十二卷,今闕。梁有《莊子》十卷,東晉議郎崔撰注,亡。



 《莊子》十六卷司馬彪注。本二十一卷,今闕。



 《莊
 子》三十卷、目一卷晉太傅主簿郭象注。梁《七錄》三十三卷。



 《集注莊子》六卷梁有《莊子》三十卷,晉丞相參軍李賾注;《莊子》十八卷,孟氏注,錄一卷。



 亡。



 《莊子》音一卷李軌撰。



 《莊子音》三卷徐邈撰。



 《莊子集音》三卷徐邈撰。



 《莊子注音》一卷司馬彪等撰。



 《莊子音》三卷郭象撰。梁有向秀《莊子音》一卷。



 《莊子外篇雜音》一卷《莊子內篇音義》一卷《莊子講疏》十卷梁簡文帝撰。本二十卷,今闕。



 《莊子講疏》二卷張譏撰,亡。



 《莊子講疏》八卷《莊子文句義》二十八卷本三十卷,今闕。梁有《莊子義疏》十卷,又《莊子義疏》三卷,宋處士王叔之撰,亡。



 《莊子內篇講疏》八卷周弘正撰。



 《莊子義疏》八卷戴詵撰。



 《南華論》二十五卷梁曠撰,本三十卷。



 《南華論音》三卷《莊成子》十二卷梁有《蹇子》一卷,今亡。



 《玄言新記明莊部》二卷梁澡撰。



 《守白論》一卷《
 任子道論》十卷魏河東太守任嘏撰。梁有《渾輿經》一卷,魏安成令桓威撰,亡。



 《唐子》十卷吳唐滂撰。梁有《蘇子》七卷,晉北中郎參軍蘇彥撰;《宣子》二卷,晉宜城令宣舒撰;《陸子》十卷,陸云撰。亡。



 《杜氏幽求新書》二十卷杜夷撰。



 《抱樸子內篇》二十一卷、音一卷葛洪撰。梁有《顧道士新書論經》三卷,晉方士顧穀撰,亡。



 《孫子》十二卷孫綽撰。



 《符子》二十卷東晉員外郎符朗撰。梁有《賀子述言》十卷,宋太學博士賀道養撰;《少子》五卷,齊司徒左長史張融撰;梁有《養生論》三卷,嵇康撰;《攝生論》二卷,晉河內太守阮侃撰;《無宗論》四卷,《聖人無情論》六卷。亡。《夷夏論》一卷顧歡撰。梁二卷。梁又有《談眾》三卷,亡。



 《簡文談疏》六卷晉簡文帝撰。



 《無名子》一卷張太衡撰。



 《玄子》五卷《游玄桂林》二十一卷、目一卷張譏撰。



 《廣成子》十三卷商洛公撰。張太衡注,疑近人作。



 右七十八部,合五百二十五卷。



 道者,蓋為萬物之奧,聖人之至賾也。《易》曰:「一陰一陽之謂道。」又曰:「仁者見之謂之仁,智者見之謂之智,百姓日用而不知。」夫陰陽者,天地之謂也。天地變化,萬物蠢生,則有經營之跡。至於道者,精微淳粹,而莫知其體。處陰與陰為一,在陽與陽不二。仁者資道以成仁,道非仁之謂也;智者資道以為智,道非智之謂也;百姓資道而日用,而不知其用也。聖人體道成性,清虛自守,為而不恃,長而不宰,故能不勞聰明而人自化,不假修營而功自成。其玄德深遠,言象不測。先王懼人之惑,置於方外,六經之義,是所罕言。《周官》九兩,其三曰師,蓋近之矣。然自
 黃帝以下,聖哲之士,所言道者,傳之其人,世無師說。漢時,曹參始薦蓋公能言黃老,文帝宗之。自是相傳,道學眾矣。下士為之,不推其本,茍以異俗為高,狂狷為尚,迂誕譎怪而失其真。



 《管子》十九卷齊相管夷吾撰。



 《商君書》五卷秦相衛鞅撰。梁有《申子》三卷,韓相申不害撰,亡。



 《慎子》十卷戰國時處士慎到撰。



 《韓子》二十卷、目一卷韓非撰。梁有《晁氏新書》三卷,漢御史大夫晁錯撰,亡。



 《正論》六卷漢大尚書崔寔撰。梁有《法論》十卷,劉邵撰;《政論》五卷,魏侍中劉暠撰;《阮子正論》五卷,魏清河太守阮武撰。亡。



 《世要論》十二卷魏大司農桓範撰。梁有二十卷。又有《陳子要言》十四卷,吳豫章太守陳融撰;《蔡司徒難論》五卷,晉三公令史黃命撰。亡。



 右六部,合七十二卷。



 法者,人君所以禁淫慝,齊不軌,而輔於治者也。《易》著「先生明罰飭法」,《書》美「明於五刑,以弼五教」。《周官》,司寇「掌建國之三典,以佐王刑邦國,詰四方」;司刑「以五刑之法,麗萬民之罪」是也。刻者為之,則杜哀矜,絕仁愛,欲以威劫為化,殘忍為治,乃至傷恩害親。



 《鄧析子》一卷析;鄭大夫。



 《尹文子》二卷尹文,周之處士,游齊稷下。



 《士品》一卷魏文帝撰。梁有《刑聲論》一卷,亡。



 《人物志》三卷劉邵撰。梁有《士緯新書》十卷,姚信撰,又《姚氏新書》二卷,與《士緯》相似;《九州人士論》一卷,魏司空盧毓撰;《通古人論》一卷。亡。



 右四部,合七卷。



 名者,所以正百物,敘尊卑,列貴賤,各控名而責實,無相
 僭濫者也。《春秋傳》曰:「古者名位不同,節文異數。」《孔子》曰:「名不正則言不順,言不順則事不成。」《周官》,宗伯「以九儀之命,正邦國之位,辯其名物之類」,是也。



 拘者為之,則苛察繳繞,滯於析辭而失大體。



 《墨子》十五卷、目一卷宋大夫墨翟撰。



 《隋巢子》一卷巢,似墨翟弟子。



 《胡非子》一卷非,似墨翟弟子。梁有《田俅子》一卷,亡。



 右三部,合一十七卷。



 墨者,強本節用之術也。上述堯、舜、夏禹之行,茅茨不翦,糲粱之食,桐棺三寸,貴儉兼愛,嚴父上德,以孝示天下,右鬼神而非命。《漢書》以為本出清廟之守。然則《周官》宗
 伯「掌建邦之天神地禋人鬼」,肆師「掌立國祀及兆中廟中之禁令」,是其職也。愚者為之,則守於節儉,不達時變,推心兼愛,而混於親疏也。



 《鬼谷子》三卷皇甫謐注。鬼谷子,周世隱於鬼谷。梁有《補闕子》十卷,《湘東鴻烈》十卷,並元帝撰。亡。



 《鬼谷子》三卷樂一注。



 右二部,合六卷。



 從橫者,所以明辯說,善辭令,以通上下之志者也。《漢書》以為本出行人之官,受命出疆,臨事而制。故曰:「誦《詩》三百,使於四方,不能專對,雖多亦奚以為?」《周官》,掌交「以節與幣,巡邦國之諸侯及萬姓之聚,導王之德意志慮,使
 闢行之,而和諸侯之好,達萬民之說,諭以九稅之利,九儀之親,九牧之維,九禁之難,九戎之威」是也。佞人為之,則便辭利口,傾危變詐,至於賊害忠信,覆邦亂家。



 《尉繚子》五卷梁並錄六卷。尉繚,梁惠王時人。



 《尸子》二十卷、目一卷梁十九卷。秦相衛鞅上客尸佼撰。其九篇亡,魏黃初中續。



 《呂氏春秋》二十六卷秦相呂不韋撰,高誘注。



 《淮南子》二十一卷漢淮南王劉安撰,許慎注。



 《淮南子》二十一卷高誘注。



 《論衡》二十九卷後漢徵士王充撰。梁有《洞序》九卷、錄一卷,應奉撰,亡。



 《風俗通義》三十一卷錄一卷。應劭撰。梁三十卷。



 《仲長子昌言》十二卷錄一卷。漢尚書郎仲長統撰。



 《蔣子萬機論》八卷蔣濟撰。梁有《篤論》四卷,杜恕撰;《芻蕘論》五卷,釧會撰;梁有《諸葛子》五卷,吳太傅諸葛恪撰。亡。



 《傅子》百二十卷晉司隸校尉傅玄撰。《默記》三卷,吳大鴻臚張儼
 撰。《裴氏新言》五卷,吳大鴻臚裴玄撰。梁有《新義》十八卷,吳太子中庶子劉褵撰;《析言論》二十卷,晉議郎張顯撰;《桑丘先生書》二卷,晉征南軍師楊偉撰。亡。



 《時務論》十二卷楊偉撰。梁有《古世論》十七卷,《桓子》一卷;《秦子》三卷,吳秦菁撰;《劉子》十卷,《何子》五卷。亡。



 《立言》六卷蘇道撰。梁有《孔氏說林》二卷,孔衍撰,亡。



 《抱樸子外篇》三十卷葛洪撰。梁有五十一卷。



 《金樓子》十卷梁元帝撰。



 《博物志》十卷張華撰。



 《張公雜記》一卷張華撰。梁有五卷,與《博物志》相似,小小不同。又有《雜記》十卷,何氏撰,亡。



 《雜記》十一卷張華撰。梁有《子林》二十卷,孟儀撰。亡。



 《廣志》二卷郭義恭撰。



 《部略》十五卷《博覽》十三卷《諫林》五卷齊晉陵令何翌之撰。



 《述政論》十三卷陸澄撰。



 《古今注》三卷崔豹撰。



 《古今訓》十一卷張顯撰。



 《古今善言》三十卷宋車騎將軍範泰撰。



 《善諫》二卷宋領軍長史虞通之撰。



 《缺文》十三卷陸澄撰。



 《政論》十三卷陸澄撰。



 《記聞》二卷
 宋後軍參軍徐益壽撰。



 《新舊傳》四卷《釋欲語》八卷劉霽撰。



 《稱謂》五卷後周大將軍盧辯撰。



 《備遺記》三卷《纂要》一卷戴安道撰,亦云顏延之撰。



 《方類》六卷《俗說》三卷沈約撰。梁五卷。



 《雜說》二卷沈約撰《袖中記》二卷沈約撰。



 《袖中略集》一卷沈約撰。



 《珠叢》一卷沈約撰。



 《採璧》三卷梁中書舍人庾肩吾撰。



 《物始》十卷謝吳撰。



 《宜覽》二十二卷《玉府集》八卷《鴻寶》十卷《顯用》九卷《墳典》三十卷盧辯撰。



 《玉燭寶典》十二卷著作郎杜臺卿撰。



 《典言》四卷後魏人李穆叔撰。



 《典言》四卷後齊中書郎荀士遜等撰。



 《補文》六卷《四時錄》十二卷《正訓》二十卷《內訓》二十卷《雜略》十三卷《清神》三卷《前言》八卷《會林》五卷《對林》十卷《道言》六卷叱羅羨撰。



 《道
 術志》三卷《述伎藝》一卷《諸書要略》一卷魏彥深撰。



 《文府》五卷梁有《文章義府》三十卷。



 《語對》十卷硃澹遠撰。



 《語麗》十卷硃澹遠撰。



 《對要》三卷《雜語》三卷《眾書事對》三卷《廊廟五格》二卷王彬撰。



 《名數》八卷《新言》四卷裴立撰。



 《善說》五卷《君臣相起發事》三卷《物重名》五卷《真注要錄》一卷《天地體》二卷《雜事鈔》二十四卷《雜書鈔》四十四卷《子抄》三十卷梁黟令庾仲容撰。



 《子鈔》二十卷梁有《子鈔》十五卷,沈約撰,亡。



 《論集》八十六卷殷仲堪撰。梁九十六卷。梁又有《雜論》五十八卷,《雜論》十三卷,亡。



 《皇覽》一百二十卷繆襲等撰。梁六百八十卷。梁又有《皇覽》一百二十三卷,何承天合;《皇覽》五十卷,徐爰合,《皇覽目》四卷;又有《皇覽抄》二十卷,梁特進蕭琛抄。亡。



 《帝王集要》三十卷崔安撰。



 《類苑》一百
 二十卷梁征虜刑獄參軍劉孝標撰。梁《七錄》八十二卷。



 《華林遍略》六百二十卷梁綏安令徐僧權等撰。



 《要錄》六十卷《壽光書苑》二百卷梁尚書左丞劉杳撰。



 《科錄》二百七十卷元暉撰。



 《書圖泉海》二十卷陳張式撰。



 《呈壽堂御覽》三百六十卷《長洲玉鏡》二百三十八卷《書鈔》一百七十四卷《釋氏譜》十五卷《內典博要》三十卷《凈住子》二十卷齊竟陵王蕭子良撰。



 《因果記》十卷《歷代三寶記》三卷費長房撰。



 《真言要集》十卷《義記》二十卷蕭子良撰。



 《感應傳》八卷宋尚書郎王延秀撰。



 《眾僧傳》二十卷裴子野撰。



 《高僧傳》六卷虞孝敬撰。



 《皇帝菩薩清凈大舍記》三卷謝吳撰,亡。



 《寶臺四法藏目錄》一百卷大業中撰。



 《玄門寶海》一百二十卷大業中撰。



 右九十七部,合二千七百二十卷。



 雜者,兼儒、墨之道,通眾家之意,以見王者之化,無所不冠者也。古者司史歷記前言往行,禍福存亡之道。然則雜者,蓋出史官之職也。放者為之,不求其本,材少而多學,言非而博,是以雜錯漫羨,而無所指歸。



 《氾勝之書》二卷漢議郎氾勝之撰。



 《四人月令》一卷後漢大尚書崔寔撰。



 《禁苑實錄》一卷《齊民要術》十卷賈思勰撰。



 《春秋濟世六常擬議》五卷楊瑾撰。梁有《陶硃公養魚法》,《卜式養羊法》、《養豬法》、《月政畜牧栽種法》,各一卷,亡。



 右五部,一十九卷。



 農者,所以播五穀,藝桑麻,以供衣食者也。《書》敘八政,其
 一曰食,二曰貨。孔子曰:「所重民食。」《周官》:塚宰「以九職任萬民」,其一曰「三農生九穀」,地官司稼「掌巡邦野之稼,而辨穜懸之種,周知其名與其所宜地,以為法而懸於邑閭」,是也。鄙者為之,則棄君臣之義,徇耕稼之利,而亂上下之序。《燕丹子》一卷丹,燕王喜太子。梁有《青史子》一卷;又《宋玉子》一卷、錄一卷,楚大夫宋玉撰;《群英論》一卷,郭頒撰;《語林》十卷,東晉處士裴啟撰。亡。



 《雜語》五卷《郭子》三卷東晉中郎郭澄之撰。



 《雜對語》三卷《要用語對》四卷《文對》三卷《瑣語》一卷梁金紫光祿大夫顧協撰。



 《笑林》三卷後漢給事中邯鄲淳撰。



 《笑苑》四卷《解頤》二卷陽玠松撰。



 《世說》八卷宋臨川王劉義慶撰。



 《世說》十卷劉孝標注。梁有《俗說》
 一卷,亡。



 《小說》十卷梁武帝敕安右長史殷蕓撰。梁目,三十卷。



 《小說》五卷《邇說》一卷梁南臺治書伏挺撰。



 《辯林》二十卷蕭賁撰。



 《辯林》二十卷希秀撰。



 《瓊林》七卷周獸門學士陰顥撰。



 《古今藝術》二十卷《雜書鈔》十三卷《座右方》八卷庾元威撰。



 《座右法》一卷《魯史欹器圖》一卷儀同劉微注。



 《器準圖》三卷後魏丞相士曹行參軍信都芳撰。



 《水飾》一卷右二十五部,合一百五十五卷。



 小說者,街說巷語之說也。《傳》載輿人之誦,《詩》美詢於芻蕘。古者聖人在上,史為書,瞽為詩,工誦箴諫,大夫規誨,士傳言而庶人謗。孟春,徇木鐸以求歌謠,巡省觀人詩,以知風俗。過則正之,失則改之,道聽途說,靡不畢紀。《周
 官》:誦訓「掌道方志以詔觀事,道方慝以詔闢忌,以知地俗」;而訓方氏「掌道四方之政事,與其上下之志,誦四方之傳道而觀衣物」是也。孔子曰:「雖小道,必有可觀者焉,致遠恐泥。」



 《司馬兵法》三卷齊將司馬穰苴撰。



 《孫子兵法》二卷吳將孫武撰,魏武帝注。梁三卷。



 《孫子兵法》一卷魏武、王凌集解。



 《孫武兵經》二卷張子尚注。



 《鈔孫子兵法》一卷魏太尉賈詡鈔。梁有《孫子兵法》二卷,孟氏解詁;《孫子兵法》二卷,吳處士沈友撰;又《孫子八陣圖》一卷。亡。



 《吳起兵法》一卷賈詡注。



 《吳孫子牝牡八變陣圖》二卷《續孫子兵法》二卷魏武帝撰。



 《孫子兵法雜占》四卷梁有《諸葛亮兵法》五卷,又《慕容氏兵法》一卷,亡。



 《皇帝兵法》一卷宋武帝所傳神人書。梁有《雜兵注》二十四卷,《兵
 法序》二卷,亡。



 《太公六韜》五卷梁六卷。周文王師姜望撰。



 《太公陰謀》一卷梁六卷。梁又有《太公陰謀》三卷,魏武帝解。



 《太公陰符鈐錄》一卷《太公金匱》二卷《太公兵法》二卷梁三卷《太公兵法》六卷梁有《太公雜兵書》六卷。



 《太公伏符陰陽謀》一卷《黃帝兵法孤虛雜記》一卷《太公三宮兵法》一卷梁有《太一三宮兵法立成圖》二卷。



 《太公書禁忌立成集》二卷《太公枕中記》一卷《周書陰符》九卷《周呂書》一卷《黃石公內記敵法》一卷《黃石公三略》三卷下邳神人撰,成氏注。梁又有《黃石公記》三卷,《黃石公略注》三卷。



 《黃石公三奇法》一卷梁有《兵書》一卷,《張良經》與《三略》往往同,亡。



 《黃石公五壘圖》一卷《黃石公陰謀行軍秘法》一卷梁有《黃石公秘經》二卷。



 《大將軍兵法》一卷《黃
 石公兵書》三卷《兵書接要》十卷魏武帝撰。梁有《兵書接要別本》五卷,又有《兵書要論》七卷,亡。



 《兵法接要》三卷魏武帝撰。



 《三宮用兵法》一卷《兵書略要》九卷魏武帝撰。梁有《兵要》二卷。



 《魏武帝兵法》一卷梁有《魏時群臣表伐吳策》一卷,《諸州策》四卷,《軍令》八卷,《尉繚子兵書》一卷。



 《兵林》六卷東晉江都相孔衍撰。



 《兵林》一卷《玄女戰經》一卷《武林》一卷王略撰。



 《黃帝問玄女兵法》四卷梁三卷。



 《秦戰鬥》一卷《梁主兵法》一卷《梁武帝兵書鈔》一卷《梁武帝兵書要鈔》一卷《玉韜》十卷梁元帝撰。



 《金韜》十卷《金策》十九卷《兵書要略》五卷後周齊王宇文憲撰。



 《兵書》七卷《兵書要術》四卷伍景志撰。



 《兵記》八卷司馬彪撰。一本二十卷。



 《兵書要序》十卷趙氏撰。



 《兵法》五卷《雜兵書》十卷
 梁有《雜兵書》八卷,《三家兵法要集》三卷,《戎略機品》二卷,亡。



 《大將軍》一卷《雜兵圖》二卷《兵略》五卷《軍勝見》十卷許昉撰。



 《戎決》十三卷許昉撰。



 《陣圖》一卷《陰策》二十二卷大都督劉祐撰。



 《陰策林》一卷《承神兵書》二十卷《真人水鏡》十卷《戰略》二十六卷金城公趙煚撰。



 《金海》三十卷蕭吉撰。



 《兵書》二十五卷《雜撰陰陽兵書》五卷莫珍寶撰。



 《黃帝兵法雜要決》一卷《黃帝軍出大師年命立成》一卷《黃帝復姓符》二卷許昉撰。梁有《闢兵法》一卷。



 《黃帝太一兵歷》一卷《黃帝蚩尤風後行軍秘術》二卷梁有《黃帝蚩尤兵法》一卷,亡。



 《老子兵書》一卷《吳有道占出軍決勝負事》一卷梁二卷。又《黃帝出軍雜用決》十二卷,《風氣占軍決勝戰》二卷,太史令吳範撰。



 《對敵權變》一
 卷吳氏撰。



 《對敵占風》一卷梁有《黃帝夏氏占氣》六卷,《兵法風氣等占》三卷,亡。



 《對敵權變逆順》一卷《兵法權儀》一卷《六甲孤虛雜決》一卷梁有《孫子戰鬥六甲兵法》一卷。



 《六甲孤虛兵法》一卷《孤虛法》十卷梁有《兵法遁甲孤虛斗中域法》九卷。



 《兵書雜占》十卷梁有《兵法日月風雲背向雜占》十二卷,《兵法》三卷,《虛占》三卷,《京氏征伐軍候》八卷。



 《兵書雜歷》八卷《太一兵書》一十一卷梁二十卷。



 《兵書內術》二卷《兵法書決》九卷闕一卷。



 《軍國要略》一卷《兵法要錄》二卷《用兵撮要》二卷《用兵要術》一卷《用兵秘法雲氣占》一卷《五家兵法》一卷《兵法三家軍占秘要》一卷李行撰。



 《氣經上部占》一卷《天大芒霧氣占》一卷《鬼谷先生占氣》一卷《五行候
 氣占災》一卷《乾坤氣法》一卷《雜匈奴占》一卷漢武帝王朔注。



 《對敵占》一卷《雜占》八卷梁有《推元嘉十二年日時兵法》二卷,《逆推元嘉五十年太歲計用兵法》一卷。



 《兵殺歷》一卷《馬槊譜》一卷梁二卷。梁有《騎馬都格》一卷,《騎馬變圖》一卷,《馬射譜》一卷,亡。



 《棋勢》四卷梁有《術藝略序》五卷,孫暢之撰;《圍棋勢》七卷,湘東太守徐泓撰;《齊高棋圖》二卷;《圍棋九品序錄》五卷,範汪等撰;《圍棋勢》二十九卷,晉趙王倫舍人馬朗等撰;《棋品敘略》三卷;建元永明《棋品》二卷,宋員外殿中將軍褚思莊撰;天監《棋品》一卷,梁尚書僕射柳惲撰。亡。



 《雜博戲》五卷《投壺經》一卷梁東宮撰《太一博法》一卷《雙博法》一卷《皇博法》一卷梁有《大小博法》一卷;《投壺經》四卷,《投壺變》一卷,晉左光祿大夫虞潭撰;《投壺道》一卷,郝沖撰;《擊壤經》一卷。亡。



 《象經》一卷周武帝撰。



 《博塞經》一卷邵綱撰。



 《棋勢》十卷沈敞撰。



 《棋勢》十卷二卷,成。



 《棋勢》十卷王子沖撰。



 《
 棋勢》八卷《棋圖勢》十卷《棋九品序錄》一卷範汪等注。



 《棋後九品序》一卷袁遵撰。



 《圍棋品》一卷梁武帝撰。



 《棋品序》一卷陸雲公撰。



 《棋法》一卷梁武帝撰。



 《彈棋譜》一卷徐廣撰。



 《二儀十博經》一卷《象經》一卷王褒注。



 《象經》三卷王裕注。



 《象經》一卷何妥注。



 《象經發題義》一卷右一百三十三部,五百一十二卷。



 兵者,所以禁暴靜亂者也。《易》曰:「古者弦木為弧,剡木為矢,弧矢之利,以威天下。」孔子曰:「不教人戰,是謂棄之。」《周官》:大司馬「掌九法九伐,以正邦國」是也。然皆動之以仁,行之以義,故能誅暴靜亂,以濟百姓。下至三季,恣情逞
 欲,爭伐尋常,不撫其人,設變詐而滅仁義,至乃百姓離叛,以致於亂。



 《周髀》一卷趙嬰注。



 《周髀》一卷甄鸞重述。



 《周髀圖》一卷《靈憲》一卷張衡撰。



 《渾天象注》一卷吳散騎常侍王蕃撰。



 《渾天義》二卷《渾天圖》一卷石氏《渾天圖》一卷《渾天圖記》一卷梁有《昕天論》一卷,姚信撰;《安天論》六卷,虞喜撰;《圖天圖》一卷,《原天論》一卷,《神光內抄》一卷。



 《定天論》三卷《天儀說要》一卷陶弘景撰。



 《玄圖》一卷《石氏星簿經贊》一卷《星經》二卷《廿氏四七法》一卷《巫咸五星占》一卷《天儀說要》一卷陶弘景撰。



 《錄軌象以頌其章》一卷內有圖。



 《天文集占》十卷晉太史令陳卓定。



 《天文要集》四十卷晉太史令韓楊撰。



 《天文要集》
 四卷《天文要集》三卷《天文集占》十卷梁百卷。梁有《石氏》、《甘氏天文占》各八卷。



 《天文占》六卷李暹撰。



 《天文占》一卷《天文占氣書》一卷《天文集要鈔》二卷《天文書》二卷梁有《雜天文書》二十五卷。



 《雜天文橫占》一卷《天文橫圖》一卷高文洪撰。



 《天文集占圖》十一卷梁有《天文五行圖》十二卷,《天文雜占》十六卷,亡。



 《天文錄》三十卷梁奉朝請祖恆撰。



 《天文志》十二卷吳雲撰。



 《天文志雜占》一卷吳雲撰。梁有《天文雜占》十五卷,亡。



 《天文》十二卷史崇注。



 《天文十二次圖》一卷梁有《天宮宿野圖》一卷,亡。



 《婆羅門天文經》二十一卷婆羅門舍仙人所說。



 《婆羅門竭伽仙人天文說》三十卷《婆羅門天文》一卷《陳卓四方宿占》一卷梁四卷《黃帝五星占》一卷《五星占》一卷丁巡撰。



 《五
 星占》一卷梁有《五星集占》六卷,《日月五星集占》十卷。



 《五星占》一卷陳卓撰。



 《五星犯列宿占》六卷《雜星書》一卷《星占》二十八卷孫僧化等撰。



 《星占》一卷梁有《石氏星經》七卷,陳卓記;又《石氏星官》十九卷,又《星經》七卷,郭歷撰。亡。



 《天官星占》十卷陳卓撰。梁《天官星占》二十卷,吳襲撰。



 《星占》八卷梁又有《星占》十八卷。



 《中星經簿》十五卷梁有《星官簿贊》十三卷,又有《星書》三十四卷,《雜家星占》六卷,《論星》一卷,亡。



 《著明集》十卷《雜星圖》五卷《天文外官占》八卷《雜星占》七卷《雜星占》十卷《海中星占》一卷梁有《論星》一卷。



 《星圖海中占》一卷《解天命星宿要決》一卷《摩登伽經說星圖》一卷《星圖》二卷梁有《星書圖》七卷。



 《雪星占》一卷《妖星流星形名占》一卷《太白占》一卷《流星占》一卷《石氏
 星占》一卷吳襲撰。



 《候雲氣》一卷《星官次占》一卷《彗孛占》一卷《二十八宿二百八十三官圖》一卷《荊州占》二十卷宋通直郎劉嚴撰。梁二十二卷。



 《翼氏占風》一卷《日月暈》三卷梁《日月暈圖》二卷。



 《孝經內記》二卷《京氏釋五星災異傳》一卷《京氏日占圖》三卷《夏氏日旁氣》一卷許氏撰。梁四卷。



 《日食弗候占》一卷《魏氏日旁氣圖》一卷《日旁雲氣圖》五卷《天文占雲氣圖》一卷梁有《雜望氣經》八卷,《候氣占》一卷,《章賢十二時雲氣圖》二卷。



 《天文洪範日月變》一卷《洪範占》二卷梁有《洪範五行星歷》四卷。



 《黃道晷景占》一卷梁有《晷景記》二卷。



 《月行黃道圖》一卷梁有《日月交會圖鄭玄注》一卷,又《日月本次位圖》二卷。



 《月暈占》一卷《日月食暈占》四卷《
 日食占》一卷《日月薄蝕圖》一卷《日變異食占》一卷《日月暈珥雲氣圖占》一卷梁有《君失政大雲雨日月占》二卷。



 《二十八宿十二次》一卷《二十八宿分野圖》一卷《五緯合雜》一卷《五星合雜說》一卷《垂象志》一百四十八卷《太史注記》六卷《靈臺秘苑》一百一十五卷太史令庾季才撰。



 右九十七部,合六百七十五卷。



 天文者,所以察星辰之變,而參於政者也。《易》曰:「天垂象,見吉兇。」《書》稱:「天視自我人視,天聽自我人聽。」故曰:「王政不修,謫見於天,日為之蝕。後德不修,謫見於天,月為之
 蝕。」其餘孛彗飛流,見伏陵犯,各有其應。《周官》:馮相「掌十有二歲、十有二月、十有二辰、十日、二十有八星之位,辨其敘事,以會天位」是也。小人為之,則指兇為吉,謂惡為善,是以數術錯亂而難明。



 《四分歷》三卷梁《四分歷》三卷,漢修歷人李梵撰。梁又有《三統歷法》三卷,劉歆撰,亡。



 《趙隱居四分歷》一卷《魏甲子元三統歷》一卷《姜氏三紀歷》一卷《歷序》一卷姜氏撰。



 《乾象歷》三卷吳太子太傅闞澤撰。梁有《乾象歷》五卷,漢會稽都尉劉洪等注;又有闞澤注五卷,又《乾象五星幻術》一卷,亡。



 《歷術》一卷吳太史令吳範撰。



 《景初歷》三卷晉楊偉撰。梁有《景初歷術》二卷,《景初歷》法三卷,又一本五卷,並楊偉撰;並《景初歷略要》二卷。亡。



 《景初壬辰元歷》一卷楊沖撰。



 《正歷》四卷晉太常劉智撰。



 《河
 西甲寅元歷》一卷涼太史趙匪又撰。



 《甲寅元歷序》一卷趙匪又撰。



 《宋元嘉歷》二卷何承天撰。梁又有《元嘉歷統》二卷,《元嘉中論歷事》六卷,《元嘉歷疏》一卷,《元嘉二十六年度日景數》一卷,亡。



 《歷術》一卷何承天撰。梁有《驗日食法》三卷,何承天撰;又有《論頻月合朔法》五卷,《雜歷》七卷,《歷法集》十卷,又《歷術》十卷;《京氏要集歷術》四卷,姜岌撰。亡。



 《歷術》一卷崔浩撰。



 《神龜壬子元歷》一卷後魏護軍將軍祖瑩撰。



 《魏後元年甲子歷》一卷《壬子元歷》一卷後魏校書郎李業興撰。



 《甲寅元歷序》一卷趙匪又撰。



 《魏武定歷》一卷《齊甲子元歷》一卷宋氏撰。



 《宋景業歷》一卷景業,後齊散騎常侍。



 《周天和年歷》一卷甄鸞撰。



 《甲子元歷》一卷李業興撰。



 《周大象年歷》一卷王琛撰。



 《歷術》一卷王琛撰。



 《壬辰元歷》一卷《甲午紀歷術》一卷《新造歷法》一卷《開皇甲子元歷》
 一卷《歷術》一卷華州刺史張寶撰。



 《七曜本起》三卷後魏甄叔遵撰。



 《七曜小甲子元歷》一卷《七曜歷術》一卷梁《七曜歷法》四卷。



 《七曜要術》一卷《七曜歷法》一卷《推七曜歷》一卷《五星歷術》一卷《天圖歷術》一卷《陳永定七曜》四卷《陳天嘉曜歷》七卷《陳天康二年七曜歷》一卷《陳光大元年七曜歷》二卷《陳光大二年七曜歷》一卷《陳太建年七曜歷》十三卷《陳至德年七曜歷》二卷《陳禎明年七曜歷》二卷《開皇七曜年歷》一卷《仁壽二年七曜歷》一卷《七曜歷經四卷張賓撰。



 《春秋去交分歷》一卷《歷日義說》一卷《律歷注解》一卷《龍歷草》
 一卷《推漢書律歷志術》一卷《推歷法》一卷崔隱居撰。



 《歷疑質讞序》二卷《興和歷疏》二卷《七曜歷數算經》一卷趙匪又撰。



 《算元嘉歷術》一卷《七曜歷疏》一卷李業興撰。



 《七曜義疏》一卷李業興撰。



 《七曜術算》二卷甄鸞撰。



 《七曜歷疏》五卷太史令張胄玄撰。



 《陰陽歷術》一卷趙匪又撰。梁有《朔氣長歷》二卷,皇甫謐撰;,《歷章句》二卷,《月令七十二候》一卷,《三五歷說圖》一卷。亡。



 《雜注》一卷《歷注》一卷《歷記》一卷《雜歷》二卷《雜歷術》一卷梁《三棋推法》一卷。



 《太史注記》六卷《太史記注》六卷《見行歷》一卷《八家歷》一卷《漏刻經》一卷何承天撰。梁有後漢待詔太史霍融、何承天、楊偉等撰三卷,亡。



 《漏刻經》一卷祖恆撰。



 《漏刻經》一卷梁中書舍人硃史撰。



 《漏刻經》一卷梁代撰。梁有《天監五年修漏刻
 事》一卷,亡。



 《漏刻經》一卷陳太史令宋景撰。



 《雜漏刻法》十一卷皇甫洪澤撰。



 《晷漏經》一卷《九章術義序》一卷《九章算術》十卷劉徽撰。



 《九章算術》二卷徐岳、甄鸞重述。



 《九章算術》一卷李遵義疏。



 《九九算術》二卷楊淑撰。



 《九章別術》二卷《九章算經》二十九卷徐岳、甄鸞等撰。



 《九章算經》二卷徐岳注。



 《九章六曹算經》一卷《九章重差圖》一卷劉徽撰。



 《九章推圖經法》一卷張崚撰。



 《綴術》六卷《孫子算經》二卷《趙匪又算經》一卷《夏侯陽算經》二卷《張丘建算經》二卷《五經算術錄遺》一卷《五經算術》一卷《算經異義》一卷張纘撰。



 《張去斤算疏》一卷《算法》一卷《黃鐘算法》三十八卷《算律呂法》一卷《
 眾家算陰陽法》一卷《婆羅門算法》三卷《婆羅門陰陽算歷》一卷《婆羅門算經》三卷右一百部,二百六十三卷。



 歷數者,所以揆天道,察昏明,以定時日,以處百事,以辨三統,以知厄會,吉隆終始,窮理盡性,而至於命者也。《易》曰:「先王以治曆明時。」《書》敘:「期,三百有六旬有六日,以閏月定四時,成歲。」《春秋傳》曰:「先王之正時也,履端於始,舉正於中,歸余於終。」又曰:「閏以正時,時以序事,事以厚生,生民之道。」其在《周官》,則亦太史之職。小人為之,則壞大為小,削遠為近,是以道術破碎而難知。



 五行



 《黃帝飛鳥曆》一卷張衡撰。



 《黃帝四神曆》一卷吳範撰。



 《黃帝地曆》一卷



 《黃帝鬥曆》一卷



 《黃石公北斗三奇法》一卷



 《風角集要占》十二卷



 《風角要占》三卷梁八卷,京房撰。



 《風角占》三卷梁有《侯公領中風角占》四卷。亡。



 《風角總占要決》十一卷梁有《風角總集》一卷,《風角雜占要決》十二卷,亡。



 《風角雜占》四卷梁有《風角雜占》十卷,亡。



 《風角要集》十卷



 《風角要集》六卷梁十一卷。



 《風角要集》一卷



 《風角要候》十一卷翼奉撰。



 《風角書》十二卷梁十卷。



 《風角》七卷章仇太翼撰。



 《風角占候》四卷梁有《風角雜兵候》十三卷,亡。



 《風角鐶曆占》二卷呂氏撰。



 《風角要候》一卷章仇太翼撰。



 《兵法風角式》一卷



 《戰鬥風角鳥情》三卷梁有《風角五音六情經》十三卷,《風角兵候》十二卷。亡。



 《風角鳥情》一卷翼氏
 撰。



 《風角鳥情》二卷儀同臨孝恭撰。



 《陰陽風角相動法》一卷梁有《風角回風卒起占》五卷,《風角地辰》一卷,《風角望氣》八卷,《風雷集占》一卷。



 《五音相動法》二卷



 《五音相動法》一卷梁有《風角五音占》五卷,京房撰,亡。



 《風角五音圖》二卷



 《風角雜占五音圖》五卷翼氏撰。梁十三卷,京房撰,翼奉撰。亡。



 《黃帝九宮經》一卷



 《九宮經》三卷鄭玄注。梁有《黃帝四部九宮》五卷,亡。



 《九宮行棋經》三卷鄭玄注。



 《九宮行棋經》三卷



 《九宮行棋法》一卷房氏撰。



 《九州行棋立成法》一卷王琛撰。



 《九宮行棋雜法》一卷



 《九宮行棋法》一卷



 《行棋新術》一卷



 《九宮行棋鈔》一卷



 《九宮推法》一卷



 《三元九宮立成》二卷



 《九宮要集》一卷豆盧晃撰。



 《九宮經解》二卷李氏注。



 《九宮圖》一卷



 《九宮變
 圖》一卷



 《九宮八卦式蟠龍圖》一卷



 《九宮郡縣錄》一卷



 《九宮雜書》十卷梁有《太一九宮雜占》十二卷,亡。



 《射候》二卷



 《太一飛鳥曆》一卷王琛撰。



 《太一飛鳥曆》一卷



 《太一飛鳥曆》二卷



 《太一十精飛鳥曆》一卷



 《太一飛鳥立成》一卷



 《太一飛鳥雜決捕盜賊法》一卷



 《太一三合五元要決》一卷梁有《黃帝太一雜書》十六卷,《黃帝太一度厄秘術》八卷,《太一帝記法》八卷,《太一雜用》十四卷,《太一雜要》七卷,《雜太一經》八卷,亡。



 《太一龍首式經》一卷董氏注。梁三卷。梁又有《式經》三十三卷,亡。



 《太一經》二卷宋琨撰。



 《太一式雜占》十卷梁二十卷。



 《太一九宮雜占》十卷



 《黃帝飛鳥曆》一卷



 《黃帝集靈》三卷



 《黃帝絳圖》一卷



 《黃帝龍首經》二卷



 《黃帝式經三十六用》
 一卷曹氏撰。



 《黃帝式用當陽經》二卷



 《黃帝奄心圖》一卷



 《玄女式經要法》一卷



 《黃帝陰陽遁甲》六卷



 《遁甲決》一卷吳相伍子胥撰。



 《遁甲文》一卷伍子胥撰。



 《遁甲經要鈔》一卷



 《遁甲萬一決》二卷



 《遁甲九元九局立成法》一卷



 《遁甲肘後立成囊中秘》一卷葛洪撰。



 《遁甲囊中經》一卷



 《遁甲囊中經疏》一卷



 《遁甲立成》六卷



 《遁甲敘三元玉曆立成》一卷郭弘遠撰。



 《遁甲立成》一卷



 《遁甲立成法》一卷臨孝恭撰。



 《遁甲穴隱秘處經》一卷



 《黃帝九元遁甲》一卷王琛撰。



 《黃帝出軍遁甲式法》一卷



 《遁甲法》一卷



 《遁甲術》一卷



 《陽遁甲用局法》一卷臨孝恭撰。



 《雜遁甲鈔》四卷



 《三元
 遁甲上圖》一卷



 《三元遁甲圖》三卷



 《遁甲九宮八門圖》一卷



 《遁甲開山圖》三卷榮氏撰。



 《遁甲返覆圖》一卷葛洪撰。



 《遁甲年錄》一卷



 《遁甲支手決》一卷



 《遁甲肘後立成》一卷



 《遁甲行日時》一卷



 《遁甲孤虛記》一卷伍子胥撰。



 《遁甲孤虛注》一卷



 《東方朔歲占》一卷



 《鬥中孤虛圖》一卷



 《孤虛占》一卷



 《遁甲九宮亭亭白奸書》一卷



 《戰鬥博戲等法》一卷



 《玉女反閉局法》三卷



 《逆刺》一卷京房撰。



 《逆刺占》一卷



 《逆刺總決》一卷



 《壬子決》一卷



 《鳥情占》一卷王喬撰。



 《鳥情逆占》一卷



 《鳥情書》二卷



 《鳥情雜占禽獸語》一卷



 《占鳥情》二卷



 《六情決》一
 卷王琛撰。



 《六情鳥音內秘》一卷焦氏撰。



 《孝經元辰決》九卷



 《孝經元辰》二卷



 《元辰本屬經》一卷



 《推元辰厄會》一卷



 《元辰事》一卷



 《元辰救生削死法》一卷



 《推元辰要秘次序》一卷



 《元辰章用》二卷



 《雜推元辰要秘立成》六卷



 《元辰立成譜》一卷



 《方正百對》一卷京房撰。



 《晉災祥》一卷京房撰。



 《災祥集》七十六卷



 《地形志》八十七卷庾季才撰。



 《海中仙人占災祥書》三卷



 《周易占事》十二卷漢魏郡太守京房撰。



 《遁甲》三卷梁有《遁甲經》十卷,《遁甲正經》五卷,《太一遁甲》一卷,亡。



 《遁甲要用》四卷葛洪撰。



 《遁甲秘要》一卷葛洪撰。



 《遁甲要》一卷葛洪撰。



 《遁甲》三十三卷後魏信都芳撰。



 《三元遁甲》六卷許昉撰。



 《三元遁甲》六卷
 陳員外散騎常侍劉毗撰。



 《三元遁甲》二卷梁《太一遁甲》一卷,《遁甲三元》三卷。



 《三元九宮遁甲》二卷梁有《遁甲三元》三卷,亡。



 《三正遁甲》一卷杜仲撰。



 《遁甲》三十五卷



 《遁甲時下決》三十三卷



 《陰陽遁甲》十四卷



 《遁甲正經》三卷梁五卷



 《遁甲經》十卷



 《遁甲開山圖》一卷梁《遁甲開山經圖》一卷。



 《遁甲九星曆》一卷



 《遁甲三奇》三卷



 《遁甲推時要》一卷



 《遁甲三元九甲立成》一卷



 《雜遁甲》五卷梁九卷。《遁甲經外篇》一百卷,《六甲隱圖》並《遁甲圖》二卷,亡。



 《陽遁甲》九卷釋智海撰。



 《陰遁甲》九卷



 《武王須臾》二卷



 《六壬式經雜占》九卷梁有《六壬式經》三卷,亡。



 《六壬釋兆》六卷



 《破字要決》一卷



 《桓安吳式經》一卷梁有《雜式占》五卷,《式經雜要》、《決式立成》各九卷,《式王曆》、《伍子胥式經章句》、《起射覆式》、《
 越相范蠡玉笥式》,各二卷,亡。



 《光明符》十二卷錄一卷,梁簡文帝撰。



 《龜經》二卷晉掌卜大夫史蘇撰。梁有《史蘇龜經》十卷;《龜決》二卷,葛洪撰;《管郭近要決》、《龜音色》、《九宮著龜序》各一卷;《龜卜要決》、《龜圖五行九親》各四卷;又《龜親經》三十卷,周子曜撰。亡。



 《史蘇沉思經》一卷



 《龜卜五兆動搖決》一卷



 《周易占》十二卷京房撰。梁《周易妖占》十三卷,京房撰。



 《周易守林》三卷京房撰。



 《周易集林》十二卷京房撰。《七錄》雲伏萬壽撰。



 《周易飛候》九卷京房撰。梁有《周易飛候六日七分》八卷,亡。



 《周易飛候》六卷京房撰。



 《周易四時候》四卷京房撰。



 《周易錯卦》七卷京房撰。



 《周易混沌》四卷京房撰。



 《周易委化》四卷京房撰。



 《周易逆刺占災異》十二卷京房撰。



 《周易占》一卷張浩撰。



 《周易雜占》十三卷



 《周易雜占》十一卷



 《周易雜占》九卷尚廣撰。梁有《周易雜占》八卷,武靖撰。亡。



 《易林》十
 六卷焦贛撰。梁又本三十二卷。



 《易林變占》十六卷焦贛撰。



 《易林》二卷費直撰。梁五卷。



 《易內神筮》二卷費直撰。梁有《周易筮占林》五卷,費直撰,亡。



 《易新林》一卷後漢方士許峻等撰。梁十卷。



 《易災條》二卷許峻撰。



 《易決》一卷許峻撰。梁有《易雜占》七卷,許峻撰,又《易要決》三卷,亡。



 《周易通靈決》二卷魏少府丞管輅撰。



 《周易通靈要決》一卷管輅撰。



 《周易集林律曆》一卷虞翻撰。梁有《周易筮占》二十四卷,晉徵士徐苗撰,亡。



 《周易新林》四卷郭璞撰。梁有《周易雜占》十卷,葛洪撰。亡。



 《周易新林》九卷郭璞撰。梁有《周易林》五卷,郭璞撰,亡。



 《易洞林》三卷郭璞撰。



 《周易新林》一卷



 《周易新林》二卷



 《易林》三卷魯洪度撰。



 《周易林》十卷梁《周易林》三十三卷,錄一卷。



 《易贊林》二卷



 《易立成林》二卷郭氏撰。



 《易立成》四卷



 《易玄成》一卷



 《周易立成占》三卷顏氏撰。



 《神農
 重卦經》二卷



 《文王幡音》一卷



 《易三備》三卷



 《易三備》一卷



 《易占》三卷



 《易射覆》二卷



 《易射覆》一卷



 《周易孔子通覆決》三卷顏氏撰。



 《易林要決》一卷



 《易要決》二卷梁有《周易曆》、《周易初學筮要法》各一卷。



 《周易髓腦》二卷



 《易腦經》一卷鄭氏撰。



 《周易玄品》二卷



 《易律曆》一卷虞翻撰。



 《易曆》七卷



 《易曆決疑》二卷



 《周易卦林》一卷



 《洞林》三卷梁元帝撰。



 《連山》三十卷梁元帝撰。



 《雜筮占》四卷



 《五兆算經》一卷



 《十二靈棋卜經》一卷梁有《管公明算占書》一卷,《五行雜卜經》十卷,亡。



 《京君明推偷盜書》一卷



 《天皇大神氣君注曆》一卷



 《太史公萬歲曆》一卷



 《千歲曆祠》一卷任氏撰。



 《萬歲曆祠》二卷



 《萬年
 曆二十八宿人神》一卷



 《六甲周天曆》一卷孫僧化撰。



 《六十甲子曆》八卷



 《曆祀》一卷



 《田家曆》十二卷



 《三合紀饑穰》一卷



 《師曠書》三卷



 《海中仙人占災祥書》三卷



 《東方朔占》二卷



 《東方朔書》二卷



 《東方朔書鈔》二卷



 《東方朔曆》一卷



 《東方朔占候水旱下人善惡》一卷梁有《擇日書》十卷,《太歲所在占善惡書》一卷,亡。



 《雜忌曆》二卷魏光祿勳高堂隆撰。



 《百忌大曆要鈔》一卷



 《百忌曆術》一卷



 《百忌通曆法》一卷梁有《雜百忌》五卷。亡。



 《曆忌新書》十二卷



 《太史百忌曆圖》一卷梁有《太史百忌》一卷,亡。



 《雜殺曆》九卷梁有《秦災異》一卷,後漢中郎郗萌撰;《後漢災異》十五卷,《晉災異簿》二卷,《宋災異簿》四卷,《雜凶妖》一卷,《破書》、《玄武書契》各一卷。亡。



 《二儀曆頭堪餘》一卷



 《
 堪餘曆》二卷



 《注曆堪餘》一卷



 《地節堪餘》二卷



 《堪餘曆注》一卷



 《堪餘》四卷



 《大小堪餘曆術》一卷梁《大小堪餘》三卷。



 《四序堪餘》二卷殷紹撰。梁有《堪餘天赦書》七卷,《雜堪餘》四卷,亡。



 《八曾堪餘》一卷



 《雜要堪餘》一卷



 《元辰五羅算》一卷



 《孝經元辰》四卷梁有《五行元辰厄會》十三卷,《孝經元辰會》九卷,《孝經元辰決》一卷,亡。



 《元辰曆》一卷



 《雜元辰祿命》二卷



 《河祿命》三卷梁有《五行祿命厄會》十卷,亡。



 《乾坤氣法》一卷許辯撰。



 《易通統卦驗玄圖》一卷



 《易通統圖》二卷



 《易新圖序》一卷



 《易通統圖》一卷



 《易八卦命錄鬥內圖》一卷郭璞撰。



 《易鬥圖》一卷郭璞撰。



 《易八卦鬥內圖》二卷



 《八卦鬥內圖》二卷梁有《周易八卦五行圖》、《周易鬥中八卦絕命圖》、《周易
 鬥中八卦推遊年圖》各一卷,亡。



 《周易分野星圖》一卷



 《舉百事略》一卷



 《五姓歲月禁忌》一卷



 《舉百事要》一卷



 《嫁娶經》四卷



 《陰陽婚嫁書》四卷



 《雜陰陽婚嫁書》三卷



 《婚嫁書》二卷



 《婚嫁黃籍科》一卷



 《六合婚嫁曆》一卷梁《六合婚嫁書》及圖,各一卷。



 《嫁娶迎書》四卷



 《雜婚嫁書》六卷



 《嫁娶陰陽圖》二卷



 《陰陽嫁娶圖》二卷



 《雜嫁娶房內圖術》四卷



 《九天嫁娶圖》一卷



 《六甲貫胎書》一卷



 《產乳書》二卷



 《產經》一卷



 《推產婦何時產法》一卷王琛撰。



 《推產法》一卷



 《雜產書》六卷



 《生產符儀》一卷



 《產圖》二卷



 《雜產圖》四卷



 《拜官書》三卷



 《臨官冠帶書》一卷



 《
 仙人務子傳神通黃帝登壇經》一卷



 《壇經》一卷四等撰。



 《登壇經》三卷



 《五姓登壇圖》一卷



 《登壇文》一卷梁有《二公地基》一卷,《雜地基立成》五卷,《八神圖》二卷,《十二屬神圖》一卷,亡。



 《沐浴書》一卷梁有《裁衣書》一卷,亡。



 《占夢書》三卷京房撰。



 《占夢書》一卷崔元撰。



 《竭伽仙人占夢書》一卷



 《占夢書》一卷周宣等撰。



 《新撰占夢書》十七卷並目錄。



 《夢書》十卷



 《解夢書》二卷



 《海中仙人占體及雜吉凶書》三卷



 《海中仙人占吉凶要略二卷》



 《雜占夢書》一卷梁有《師曠占》五卷,《東方朔占》七卷,《黃帝太一雜占》十卷,《和菟鳥鳴書》、《王喬解鳥語經》、《騑書》、《耳鳴書》、《目書》各一卷,《董仲舒請禱圖》三卷,亡。



 《灶經》十四卷梁簡文帝撰。梁又有《祠灶書》一卷,《六甲祀書》二卷,又有《太玄禁經》、《白獸七變經》、《墨子枕中五行要記》、《淮南萬畢經》、《淮南變化術》、《陶硃變化術》各一卷,《三五步剛》三十
 卷,《五行變化墨子》五卷,《淮南中經》四卷,《六甲隱形圖》五卷,太史公《素王妙論》二卷,亡。



 《瑞應圖》三卷



 《瑞圖贊》二卷梁有孫柔之《瑞應圖記》、《孫氏瑞應圖贊》各三卷,亡。



 《祥瑞圖》十一卷



 《祥瑞圖》八卷侯亶撰。



 《芝英圖》一卷



 《祥異圖》十一卷



 《災異圖》一卷



 《地動圖》一卷



 《張掖郡玄石圖》一卷高堂隆撰



 《張掖郡玄石圖》一卷孟眾撰。梁有《晉玄石圖》一卷,《晉德易天圖》二卷,亡。



 《天鏡》二卷



 《乾坤鏡》二卷梁《天鏡》、《地鏡》、《日月鏡》、《四規鏡經》各一卷,《地鏡圖》六卷,亡。



 《望氣書》七卷



 《雲氣占》一卷梁《望氣相山川寶藏秘記》一卷,《仙寶劍經》二卷,亡。



 《地形志》八十卷庾季才撰。



 《宅吉凶論》三卷



 《相宅圖》八卷



 《五姓墓圖》一卷梁有《塚書》、《黃帝葬山圖》各四卷,《五音相墓書》五卷,《五音圖墓書》九十一卷,《五姓圖山龍》及《科墓葬不傳》各一卷,《雜相墓書》四十五卷,亡。



 《相書》四十六卷



 《相經要錄》
 二卷蕭吉撰。《相經》三十卷,鐘武隸撰;《相書》十一卷,樊、許、唐氏《武王相書》一卷,《雜相書》九卷,《相書圖》七卷。亡。



 《相手板經》六卷梁《相手板經》、《受版圖》、韋氏《相板印法指略抄》、魏征東將軍程申伯《相印法》各一卷,亡。



 《大智海》四卷



 《白澤圖》一卷



 《相馬經》一卷梁有《伯樂相馬經》、《闕中銅馬法》、《周穆王八馬圖》、《齊侯大夫甯戚相牛經》、《王良相牛經》、《高堂隆相牛經》、《淮南八公相鵠經》、《浮丘公相鶴書》、《相鴨經》、《相雞經》、《相鵝經》、《相貝經》、《祖恆權衡記》、《稱物重率術》各二卷,《劉潛泉圖記》三卷,亡。



 右二百七十二部,合一千二十二卷。



 五行者,金、木、水、火、土,五常之形氣者也。在天為五星,在人為五藏,在目為五色,在耳為五音,在口為五味,在鼻為五臭。在上則出氣施變,在下則養人不倦。故《傳》曰:「天生五材,廢一不可。」是以聖人推其終始,以通神明之變,
 為卜筮以考其吉凶,占百事以觀於來物,觀形法以辨其貴賤。《周官》則分在保章、馮相、卜師、筮人、占夢、綍,而太史之職,實司總之。小數者才得其十觕,便以細事相亂,以惑於世。



 醫方



 《黃帝素問》九卷梁八卷。



 《黃帝甲乙經》十卷音一卷。梁十二卷。



 《黃帝八十一難》二卷梁有《黃帝眾難經》一卷,呂博望注,亡。



 《黃帝針經》九卷梁有《黃帝針炙經》十二卷,徐悅、龍銜素《針經並孔穴暇蟆圖》三卷,《雜針經》四卷,程天祚《針經》六卷,《灸經》五卷,《曹氏灸方》七卷,秦承祖《偃側雜針灸經》三卷,亡。



 《徐叔響針灸要鈔》一卷



 《玉匱針經》一卷



 《赤烏神針經》一卷



 《岐伯經》十卷



 《脈經》十卷王叔和撰。



 《脈經》二卷梁《脈經》十四卷,又《脈生死要訣》二卷;又《脈經》六卷,黃公興撰;《脈經》六卷,秦承祖撰;《脈
 經》十卷,康普思撰。亡。



 《黃帝流注脈經》一卷梁有《明堂流注》六卷,亡。



 《明堂孔穴》五卷梁《明堂孔穴》二卷,《新撰針灸穴》一卷,亡。



 《明堂孔穴圖》三卷



 《明堂孔穴圖》三卷梁有《偃側圖》八卷,又《偃側圖》二卷。



 《神農本草》八卷梁有《神農本草》五卷,《神農本草屬物》二卷,《神農明堂圖》一卷,《蔡邕本草》七卷,《華佗弟子吳普本草》六卷,《陶隱居本草》十卷,《隨費本草》九卷,《秦承祖本草》六卷,《王季璞本草經》三卷,《李譡之本草經》、《談道術本草經鈔》各一卷,《宋大將軍參軍徐叔響本草病源合藥要鈔》五卷,《徐叔響等四家體療雜病本草要鈔》十卷,《王末鈔小兒用藥本草》二卷,《甘浚之癰疽耳眼本草要鈔》九卷,《陶弘景本草經集注》七卷,《趙贊本草經》一卷,《本草經輕行》、《本草經利用》各一卷,亡。



 《神農本草》四卷雷公集注。



 《甄氏本草》三卷



 《桐君藥錄》三卷梁有《雲麾將軍徐滔新集藥錄》四卷,《李譡之藥錄》六卷,《藥法》四十二卷,《藥律》三卷,《藥性》《藥對》各二卷,《藥目》三卷,《神農采藥經》二卷,《藥忌》一卷,亡。



 《太清草木集要》二卷陶隱居撰。



 《張仲景方》十五
 卷仲景,後漢人。梁有《黃素藥方》二十五卷,亡。



 《華佗方》十卷吳普撰。佗,後漢人。梁有《華佗內事》五卷,又《耿奉方》六卷,亡。



 《集略雜方》十卷



 《雜藥方》一卷梁有《雜藥方》四十六卷。



 《雜藥方》十卷



 《寒食散論》二卷梁有《寒食散湯方》二十卷,《寒食散方》一十卷,《皇甫謐、曹翕論寒食散方》二卷,亡。



 《寒食散對療》一卷釋道洪撰。



 《解寒食散方》二卷釋智斌撰。梁《解散論》二卷。



 《解寒食散論》二卷梁有《徐叔響解寒食散方》六卷,《釋慧義寒食解雜論》七卷,亡。



 《雜散方》八卷梁有《解散方》、《解散論》各十三卷,《徐叔響解散消息節度》八卷,《范氏解散方》七卷,《解釋慧義解散方》一卷,亡。



 《湯丸方》十卷



 《雜丸方》十卷梁有《百病膏方》十卷,《雜湯丸散酒煎薄帖膏湯婦人少小方》九卷,《羊中散雜湯丸散酒方》一卷,《療下湯丸散方》十卷。



 《石論》一卷



 《醫方論》七卷梁有《張仲景辨傷寒》十卷,《療傷寒身驗方》、《徐文伯辨傷寒》各一卷,《傷寒總要》二卷,《支法存申蘇方》五卷,《王叔和論病》六卷,《張仲景評病要方》一卷,《徐叔響、談道述、徐悅體療雜病
 疾源》三卷,《甘浚之癰疽部黨雜病疾源》三卷,《府藏要》三卷,亡。



 《肘後方》六卷葛洪撰。梁二卷。《陶弘景補闕肘後百一方》九卷,亡。



 《姚大夫集驗方》十二卷



 《范東陽方》一百五卷錄一卷。范汪撰。梁一百七十六卷。梁又有《阮河南藥方》十六卷,阮文叔撰;《釋僧深藥方》三十卷,《孔中郎雜藥方》二十九卷,《宋建平王典術》一百二十卷;《羊中散藥方》三十卷,羊欣撰;《褚澄雜藥方》二十卷,齊吳郡太守褚澄撰。亡。



 《秦承祖藥方》四十卷見三卷。梁有《陽眄藥方》二十八卷,《夏侯氏藥方》七卷,《王季琰藥方》一卷,《徐叔響雜療方》二十二卷,《徐叔響雜病方》六卷,《李譡之藥方》一卷,《徐文伯藥方》二卷,亡。



 《胡洽百病方》二卷梁有《治卒病方》一卷;《徐奘要方》一卷,無錫令徐奘撰;《遼東備急方》三卷,都尉臣廣上;《殷荊州要方》一卷,殷仲堪撰。亡。



 《俞氏療小兒方》四卷梁有《範氏療婦人藥方》十一卷,《徐叔響療少小百病雜方》三十七卷,《療少小雜方》二十卷,《療少小雜方》二十九卷,《範氏療小兒藥方》一卷,《王末療小兒雜方》十七卷,亡。



 《徐嗣伯落年方》三卷梁有《徐叔響療腳弱雜方》八卷,《徐文
 伯辨腳弱方》一卷,《甘浚之療癰疽金創要方》十四卷,《甘浚之療癰疽毒惋雜病方》三卷,《甘伯齊療癰疽金創方》十五卷。亡。



 《陶氏效驗方》六卷梁五卷。梁又有《療目方》五卷,《甘浚之療耳眼方》十四卷,《神枕方》一卷,《雜戎狄方》一卷,宋武帝撰;《摩訶出胡國方》十卷,摩訶胡沙門撰;又《范曄上香方》一卷,《雜香膏方》一卷。亡。



 《彭祖養性經》一卷



 《養生要集》十卷張湛撰。



 《玉房秘決》十卷



 《墨子枕內五行紀要》一卷梁有《神枕方》一卷,疑此即是。



 《如意方》十卷



 《練化術》一卷



 《神仙服食經》十卷



 《雜仙餌方》八卷



 《服食諸雜方》二卷梁有《仙人水玉酒經》一卷。



 《老子禁食經》一卷



 《崔氏食經》四卷



 《食經》十四卷梁有《食經》二卷,又《食經》十九卷;《劉休食方》一卷,齊冠軍將軍劉休撰。亡。



 《食饌次第法》一卷梁有《黃帝雜飲食忌》二卷。



 《四時禦食經》一卷梁有《太官食經》五卷,又《太官食法》二十卷,《食法雜酒食要方白酒》並《作物法》十二卷,《
 家政方》十二卷,《食圖》、《四時酒要方》、《白酒方》、《七日面酒法》、《雜酒食要法》、《雜藏釀法》、《雜酒食要法》、《酒》並《飲食方》、《絺及鐺蟹方》、《羹翽法》、《且膢朐法》、《北方生醬法》各一卷,亡。



 《療馬方》一卷梁有《伯樂療馬經》一卷,疑與此同。



 《黃帝素問》八卷全元起注。



 《脈經》二卷徐氏撰。



 《華佗觀形察色並三部脈經》一卷



 《脈經決》二卷徐氏新撰。



 《脈經鈔》二卷許建吳撰。



 《黃帝素問女胎》一卷



 《三部四時五藏辨診色決事脈》一卷



 《脈經略》一卷



 《辨病形證》七卷



 《五藏決》一卷



 《論病源候論》五卷目一卷,吳景賢撰。



 《服石論》一卷



 《癰疽論方》一卷



 《五藏論》五卷



 《虐論並方》一卷



 《神農本草經》三卷



 《本草經》四卷蔡英撰。



 《藥目要用》二卷



 《本草經略》一卷



 《本草》二卷徐太山撰。



 《本草經類用》三卷



 《本草
 音義》三卷姚最撰。



 《本草音義》七卷甄立言撰。



 《本草集錄》二卷



 《本草鈔四卷》四卷



 《本草雜要決》一卷



 《本草要方》三卷甘浚之撰。



 《依本草錄藥性》三卷錄一卷。



 《靈秀本草圖》六卷原平仲撰。



 《芝草圖》一卷



 《入林采藥法》二卷



 《太常采藥時月》一卷



 《四時采藥及合目錄》四卷



 《藥錄》二卷李密撰。



 《諸藥異名》八卷沙門行矩撰。本十卷,今闕。



 《諸藥要性》二卷



 《種植藥法》一卷



 《種神芝》一卷



 《藥方》二卷徐文伯撰。



 《解散經論並增損寒食節度》一卷



 《張仲景療婦人方》二卷



 《徐氏雜方》一卷



 《少小方》一卷



 《療小兒丹法》一卷



 《徐太山試驗方》二卷



 《徐文伯療婦人瘕》一卷



 《徐太山巾箱中方》
 三卷



 《藥方》五卷徐嗣伯撰。



 《墮年方》二卷徐太山撰。



 《效驗方》三卷徐氏撰。



 《雜要方》一卷



 《玉函煎方》五卷葛洪撰。



 《小品方》十二卷陳延之撰。



 《千金方》三卷范世英撰。



 《徐王方》五卷



 《徐王八世家傳效驗方》十卷



 《徐氏家傳秘方》二卷



 《藥方》五十七卷後魏李思祖撰。本百一十卷。



 《稟丘公論》一卷



 《太一護命石寒食散》二卷宋尚撰。



 《皇甫士安依諸方撰》一卷



 《序服石方》一卷



 《服玉方法》一卷



 《劉涓子鬼遺方》十卷龔慶宣撰。



 《療癰經》一卷



 《療三十六瘺方》一卷



 《王世榮單方》一卷



 《集驗方》十卷姚僧垣撰。



 《集驗方》十二卷



 《備急單要方》三卷許澄撰。



 《藥方》二十一卷徐辨卿撰。



 《名醫集驗方》六卷



 《名醫別
 錄》三卷陶氏撰。



 《刪繁方》十三卷謝士秦撰。



 《吳山居方》三卷



 《新撰藥方》五卷



 《療癰疽諸瘡方》二卷秦政應撰。



 《單複要驗方》二卷釋莫滿撰。



 《釋道洪方》一卷



 《小兒經》一卷



 《散方》二卷



 《雜散方》八卷



 《療百病雜丸方》三卷釋曇鸞撰。



 《療百病散》三卷



 《雜湯方》十卷成毅撰。



 《雜療方》十三卷



 《雜藥酒方》十五卷



 《趙婆療漯方》一卷



 《議論備豫方》一卷於法開撰。



 《扁鵲陷水丸方》一卷



 《扁鵲肘後方》三卷



 《療消渴眾方》一卷謝南郡撰。



 《論氣治療方》一卷釋曇鸞撰。



 《梁武帝所服雜藥方》一卷



 《大略丸》五卷



 《靈壽雜方》二卷



 《經心錄方》八卷宋俠撰。



 《黃帝養胎經》一卷



 《療婦人產後雜方》三卷



 《
 黃帝明堂偃人圖》十二卷



 《黃帝針灸蝦蟆忌》一卷



 《明堂蝦蟆圖》一卷



 《針灸圖要決》一卷



 《針灸圖經》十一卷本十八卷。



 《十二人圖》一卷



 《針灸經》一卷



 《扁鵲偃側針灸圖》三卷



 《流注針灸》一卷



 《曹氏灸經》一卷



 《偃側人經》二卷秦承祖撰。



 《華佗枕中炙刺經》一卷



 《謝氏針經》一卷



 《殷元針經》一卷



 《要用孔穴》一卷



 《九部針經》一卷



 《釋僧匡針灸經》一卷



 《三奇六儀針要經》一卷



 《黃帝十二經脈明堂五藏人圖》一卷



 《老子石室蘭台中治癩符》一卷



 《龍樹菩薩藥方》四卷



 《西域諸仙所說藥方》二十三卷目一卷。本二十五卷。



 《香山仙人藥方》十卷



 《
 西域波羅仙人方》三卷



 《西域名醫所集要方》四卷本十二卷。



 《婆羅門諸仙藥方》二十卷



 《婆羅門藥方》五卷



 《耆婆所述仙人命論方》二卷目一卷。本三卷。



 《乾陀利治鬼方》十卷



 《新錄乾陀利治鬼方》四卷本五卷,闕。



 《伯樂治馬雜病經》一卷



 《治馬經》三卷俞極撰,亡。



 《治馬經》四卷



 《治馬經目》一卷



 《治馬經圖》二卷



 《馬經孔穴圖》一卷



 《雜撰馬經》一卷



 《治馬牛駝騾等經》三卷目一卷。



 《香方》一卷宋明帝撰。



 《雜香方》五卷



 《龍樹菩薩和香法》二卷



 《食經》三卷馬琬撰。



 《會稽郡造海味法》一卷



 《論服餌》一卷



 《淮南王食經》並目百六十五卷大業中撰。



 《膳羞養療》二十卷



 《金匱錄》二
 十三卷目一卷。京裡先生撰。



 《練化雜術》一卷陶隱居撰。



 《玉衡隱書》七十卷目一卷。周弘讓撰。



 《太清諸丹集要》四卷陶隱居撰。



 《雜神丹方》九卷



 《合丹大師口訣》一卷



 《合丹節度》四卷陶隱居撰。



 《合丹要略序》一卷孫文韜撰。



 《仙人金銀經並長生方》一卷



 《狐剛子萬金決》二卷葛仙公撰。



 《雜仙方》一卷



 《神仙服食經》十卷



 《神仙服食神秘方》二卷



 《神仙服食藥方》十卷抱樸子撰。



 《神仙餌金丹沙秘方》一卷



 《衛叔卿服食雜方》一卷



 《金丹藥方》四卷



 《雜神仙丹經》十卷



 《雜神仙黃白法》十二卷



 《神仙雜方》十五卷



 《神仙服食雜方》十卷



 《神仙服食方》五卷



 《服食諸雜方》二卷



 《服餌方》三卷陶
 隱居撰。



 《真人九丹經》一卷



 《太極真人九轉還丹經》一卷



 《練寶法》二十五卷目三卷。本四十卷,闕。



 《太清璿璣文》七卷沖和子撰。



 《陵陽子說黃金秘法》一卷



 《神方》二卷



 《狐子雜決》三卷



 《太山八景神丹經》一卷



 《太清神丹中經》一卷



 《養生注》十一卷目一卷。



 《養生術》一卷翟平撰。



 《龍樹菩薩養性方》一卷



 《引氣圖》一卷



 《道引圖》三卷立一,坐一,臥一。



 《養身經》一卷



 《養生要術》一卷



 《養生服食禁忌》一卷



 《養生傳》二卷



 《帝王養生要方》二卷蕭吉撰。



 《素女秘道經》一卷並《玄女經》。



 《素女方》一卷



 《彭祖養性》一卷



 《郯子說陰陽經》一卷



 《序房內秘術》一卷葛氏撰。



 《玉房秘決》八卷



 《徐太山房
 內秘要》一卷



 《新撰玉房秘決》九卷



 《四海類聚方》二千六百卷



 《四海類聚單要方》三百卷



 右二百五十六部,合四千五百一十卷。



 醫方者,所以除疾病,保性命之術者也。天有陰陽風雨晦明之氣,人有喜怒哀樂好惡之情。節而行之,則和平調理,專壹其情,則溺而生火。是以聖人原血脈之本,因針石之用,假藥物之滋,調中養氣,通滯解結,而反之於素。其善者,則原脈以知政,推疾以及國。《周官》:醫師之職「掌聚諸藥物,凡有疾者治之」,是其事也。鄙者為之,則反本傷性。故曰:「有疾不治,恆得中醫。



 凡諸子,合八百五十二部,六千四百三十七部。



 《易》曰:「天下同歸而殊途,一致而百慮。」儒、道、小說,聖人之教也,而有所偏。兵及醫方,聖人之政也,所施各異。世之治也,列在眾職,下至衰亂,官失其守。或以其業遊說諸侯,各崇所習,分鑣並騖。若使總而不遺,折之中道,亦可以興化致治者矣。《漢書》有《諸子》、《兵書》、《數術》、《方伎》之略,今合而敘之,為十四種,謂之子部。



\end{pinyinscope}