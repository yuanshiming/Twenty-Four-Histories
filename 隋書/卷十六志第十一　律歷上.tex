\article{卷十六志第十一 律歷上}

\begin{pinyinscope}

 自夫有天
 地焉,有人物焉,樹司牧以君臨,懸政教而成務,莫不擬乾坤之大象,稟中和以建極,揆影響之幽賾,成律呂之精微。是用範圍百度,財成萬品。昔者淳古葦籥,創睹人籟之源,女媧笙簧,仍昭鳳律之首。後聖廣業,稽古彌崇,伶倫含少,乃擅比竹之工,虞舜昭華,方傳刻玉之美。是以《書》稱:「葉時月正日,同律度量衡。」又曰:「予欲
 聞六律、五聲、八音、七始詠,以出納五言。」此皆候金常而列管,憑璇璣以運鈞,統三極之元,紀七衡之響,可以作樂崇德,殷薦上帝。



 故能動天地,感鬼神,和人心,移風俗,考得失,徵成敗者也。粵在夏、商,無聞改作。其於《周禮》,曲同則「掌六律六同之和,以辨天地四方陰陽之聲,以為樂器。」景王鑄鐘,問律於泠州鳩,對曰:「夫律者,所以立鈞出度。」鈞有五,則權衡規矩準繩咸備。故《詩》曰:「尹氏太師,執國之鈞,天子是裨,俾眾不迷」



 是也。太史公《律書》云:「王者制事立物,法度軌則,一稟於六律,為萬事之本。



 其於兵械,尤所重焉。故云:『望敵知吉兇,聞聲效勝負。」百王不
 易之道也。」



 及秦氏滅學,其道浸微。漢室初興,丞相張蒼,首言音律,未能審備。孝武帝創置協律之官,司馬遷言律呂相生之次詳矣。及王莽之際,考論音律,劉歆條奏,班固因志之。蔡邕又記建武以後言律呂者,司馬紹統採而續之。炎歷將終,而天下大亂,樂工散亡,器法湮滅。魏武始獲杜夔,使定音律,夔依當時尺度,權備典章。



 及晉武受命,遵而不革。至泰始十年,光祿大夫荀勖,奏造新度,更鑄律呂。元康中,勖子籓復嗣其事。未及成功,屬永嘉之亂,中朝典章,咸沒於石勒。及帝南遷,皇度草昧,禮容樂器,掃地皆盡。雖稍加採掇,而多所淪胥,終於恭、
 安,竟不能備。宋錢樂之衍京房六十律,更增為三百六十,梁博士沈重,述其名數。後魏、周、齊,時有論者。今依班志,編錄五代聲律度量,以志於篇云。



 《漢志》言律,一曰備數,二曰和聲,三曰審度,四曰嘉量,五曰衡權。自魏、晉已降,代有沿革。今列其增損之要云。



 備數五數者,一、十、百、千、萬也。《傳》曰:「物生而後有象,滋而後有數。」



 是以言律者,雲數起於建子,黃鐘之律,始一,而每辰三之,歷九辰至酉,得一萬九千六百八十三,而五數備成,以為律法。又參之,終亥,凡歷十二辰,得十有七萬七
 千一百四十七,而辰數該矣,以為律積。以成法除該積,得九寸,即黃鐘宮律之長也。此則數因律起,律以數成,故可歷管萬事,綜核氣象。其算用竹,廣二分,長三寸,正策三廉,積二百一十六枚,成六觚,乾之策也。負策四廉,積一百四十四枚,成方,坤之策也。觚方皆經十二,天地之大數也。是故探賾索隱,鉤深致遠,莫不用焉。一、十、百、千、萬,所同由也。律、度、量、衡、歷、率,其別用也。



 故體有長短,檢之以度,則不失毫厘;物有多少,受之以器,則不失圭撮;量有輕重,平之以權衡,則不失黍絲;聲有清濁,協之以律呂,則不失宮商;三光運行,紀以歷數,則不差晷刻;
 事物糅見,御之以率,則不乖其本。故幽隱之情,精微之變,可得而綜也。



 夫所謂率者,有九流焉:一曰方田,以御田疇界域。二曰慄米,以御交質變易。



 三曰衰分,以御貴賤廩稅。四曰少廣,以御積冪方圓。五曰商功,以御功程積實。



 六曰均輸,以御遠近勞費。七曰盈肭,以御隱雜互見。八曰方程,以御錯糅正負。



 九曰句股,以御高深廣遠。皆乘以散之,除以聚之,齊同以通之,今有以貫之。則算數之方,盡於斯矣。



 古之九數,圓周率三,圓徑率一,其術疏舛。自劉歆、張衡、劉徽、王蕃、皮延宗之徒,各設新率,未臻折衷。宋末,南徐州從事史祖沖之,更開密法,以圓徑
 一億為一丈,圓周盈數三丈一尺四寸一分五厘九毫二秒七忽,朒數三丈一尺四寸一分五厘九毫二秒六忽,正數在盈朒二限之間。密率,圓徑一百一十三,圓周三百五十五。約率,圓徑七,周二十二。又設開差冪,開差立,兼以正圓參之。指要精密,算氏之最者也。所著之書,名為《綴術》,學官莫能究其深奧,是故廢而不理。



 和聲傳稱黃帝命伶倫斷竹,長三寸九分,而吹以為黃鐘之宮,曰含少。次制十二管,以聽鳳鳴,以別十二律,此雌雄之聲,以分律呂。上下相生,因黃鐘為始。《虞書》云:「葉時月
 正日,同律度量衡。」夏禹受命,以聲為律,以身為度。《周禮》,樂器以十二律為之度數。司馬遷《律書》云:「黃鐘長八寸七分之一,太簇長七寸七分二,林鐘長五寸七分三,應鐘長四寸三分二。」此樂之三始,十二律之本末也。



 班固、司馬彪《律志》:「黃鐘長九寸,聲最濁;太簇長八寸;林鐘長六寸;應鐘長四寸七分四厘強,聲最清。」鄭玄《禮·月令注》、蔡邕《月令章句》及杜夔、荀勖等所論,雖尺有增損,而十二律之寸數並同。《漢志》京房又以隔八相生,一始自黃鐘,終於中呂,十二律畢矣。中呂上生黃鐘,不滿九寸,謂之執始,下生去滅。上下相生,終於南事,更增四十八律,
 以為六十。其依行在辰,上生包育,隔九編於冬至之後。分焉、遲內,其數遂減應鐘之清。宋元嘉中,太史錢樂之因京房南事之餘,引而伸之,更為三百律,終於安運,長四寸四分有奇。總合舊為三百六十律。日當一管,宮徵旋韻,各以次從。何承天《立法制議》云:「上下相生,三分損益其一,蓋是古人簡易之法。猶如古歷周天三百六十五度四分之一,後人改制,皆不同焉。而京房不悟,謬為六十。」承天更設新率,則從中呂還得黃鐘,十二旋宮,聲韻無失。黃鐘長九寸,太簇長八寸二厘,林鐘長六寸一厘,應鐘長四寸七分九厘強。其中呂上生所益之分,還
 得十七萬七千一百四十七,復十二辰參之數。



 梁初,因晉、宋及齊,無所改制。其後武帝作《鐘律緯》,論前代得失。其略云:案律呂,京、馬、鄭、蔡,至蕤賓,並上生大呂;而班固《律歷志》,至蕤賓,仍以次下生。若從班義,夾鐘唯長三寸七分有奇。律若過促,則夾鐘之聲成一調,中呂復去調半,是過於無調。仲春孟夏,正相長養,其氣舒緩,不容短促。求聲索實,班義為乖。鄭玄又以陰陽六位,次第相生。若如玄義,陰陽相逐生者,止是升陽,其降陽復將何寄?就筮數而論,乾主甲壬而左行,坤主乙癸而右行,故陰陽得有升降之義。陰陽從行者,真性也,六位升降者,象
 數也。今鄭乃執象數以配真性,故言比而理窮。云九六相生,了不釋十二氣所以相通,鄭之不思,亦已明矣。



 案京房六十,準依法推,乃自無差。但律呂所得,或五或六,此一不例也。而分焉上生,乃復遲內上生盛變,盛變仍復上生分居,此二不例也。房妙盡陰陽,其當有以,若非深理難求,便是傳者不習。



 比敕詳求,莫能辨正。聊以餘日,試推其旨,參校舊器,及古夾鐘玉律,更制新尺,以證分毫,制為四器,名之為通。四器弦間九尺,臨岳高一寸二分。黃鐘之弦二百七十絲,長九尺,以次三分損益其一,以生十二律之弦絲數及弦長。各以律本所建之月,
 五行生王,終始之音,相次之理,為其名義,名之為通。通施三弦,傳推月氣,悉無差舛。即以夾鐘玉律命之,則還相中。



 又制為十二笛,以寫通聲。其夾鐘笛十二調,以飲玉律,又不差異。山謙之《記》云:「殿前三鐘,悉是周景王所鑄無射也。」遣樂官以今無射新笛飲,不相中。以夷則笛飲,則聲韻合和。端門外鐘,亦案其銘題,定皆夷則。其西廂一鐘,天監中移度東。以今笛飲,乃中南呂。驗其鐫刻,乃是太簇,則下今笛二調。重敕太樂丞斯宣達,令更推校,鐘定有鑿處,表裏皆然。借訪舊識,乃是宋泰始中,使張永鑿之,去銅既多,故其調嘽下。以推求鐘律,便可得
 而見也。宋武平中原,使將軍陳傾致三鐘,小大中各一。則今之太極殿前二鐘,端門外一鐘是也。案西鐘銘則云「清廟撞鐘」,秦無清廟,此周制明矣。又一銘云「太簇鐘徵」,則林鐘宮所施也。京房推用,似有由也。檢題既無秦、漢年代,直云夷則、太簇,則非秦、漢明矣。古人性質,故作僮僕字,則題而言,彌驗非近。且夫驗聲改政,則五音六律,非可差舛。工守其音,儒執其文,歷年永久,隔而不通。無論樂奏,求之多缺,假使具存,亦不可用。周頌漢歌,各敘功德,豈容復施後王,以濫名實?今率詳論,以言所見,並詔百司,以求厥中。



 未及改制,遇侯景亂。陳氏制度,亦
 無改作。



 西魏廢帝元年,周文攝政。又詔尚書蘇綽詳正音律。綽時得宋尺,以定諸管,草創未就會閔帝受禪,政由塚宰,方有齊寇,事竟不行。後掘太倉,得古玉斗,按以造律及衡,其事又多湮沒。



 至開皇初,詔太常牛弘議定律呂。於是博徵學者,序論其法,又未能決。遇平江右,得陳氏律管十有二枚,並以付弘。遣曉音律者陳山陽太守毛爽及太樂令蔡子元、于普明等,以候節氣,作《律譜》。時爽年老,以白衣見高祖,授淮州刺史,辭不赴官。因遣協律郎祖孝孫就其受法。弘又取此管,吹而定聲。既天下一統,異代器物,皆集樂府,曉音律者,頗議考核,以定
 鐘律。更造樂器,以被《皇夏》十四曲,高祖與朝賢聽之,曰:「此聲滔滔和雅,令人舒緩。」



 然萬物人事,非五行不生,非五行不成,非五行不滅。故五音用火尺,其事火重。用金尺則兵,用木尺則喪,用土尺則亂,用水尺則律呂合調,天下和平。魏及周、齊,貪布帛長度,故用土尺。今此樂聲,是用水尺。江東尺短於土,長於水。



 俗間不知者,見玉作,名為玉尺,見鐵作,名為鐵尺。詔施用水尺律樂,其前代金石,並鑄毀之,以息物議。



 至仁壽四年,劉焯上啟於東宮,論張胄玄歷,兼論律呂。其大旨曰:「樂主於音,音定於律,音不以律,不可克諧,度律均鐘,於是乎在。但律終小
 呂,數復黃鐘,舊計未精,終不復始。故漢代京房,妄為六十,而宋代錢樂之更為三百六十。



 考禮詮次,豈有得然,化未移風,將恐由此。匪直長短失於其差,亦自管圍乖於其數。又尺寸意定,莫能詳考,既亂管弦,亦舛度量。焯皆校定,庶有明發。」其黃鐘管六十三為實,以次每律減三分,以七為寸法。約之,得黃鐘長九寸,太簇長八寸一分四厘,林鐘長六寸,應鐘長四寸二分八厘七分之四。其年,高祖崩,煬帝初登,未遑改作,事遂寢廢。其書亦亡。大業二年,乃詔改用梁表律調鐘磬八音之器,比之前代,最為合古。其制度文議,並毛爽舊律,並在江都淪喪。



 律管圍容黍《漢志》云:「黃鐘圍九分,林鐘圍六分,太簇圍八分。」《續志》及鄭玄並云:「十二律空,皆徑三分,圍九分。」後魏安豐王依班固《志》,林鐘空圍六分,及太簇空圍八分,作律吹之,不合黃鐘商徵之聲。皆空圍九分,乃與均鐘器合。開皇九年平陳後,牛弘、辛彥之、鄭譯、何妥等,參考古律度,各依時代,制其黃鐘之管,俱徑三分,長九寸。度有損益,故聲有高下;圓徑長短,與度而差,故容黍不同。今列其數云。



 晉前尺黃鐘容黍八百八粒。



 梁法尺黃鐘容八百二十八。



 梁表尺黃鐘三:其一容九百二十五,其一容九百一十,其一容一千一百二十。



 漢官尺黃鐘容九百三十九。



 古銀錯題黃鐘籥容一千二百。



 宋氏尺,即鐵尺,黃鐘凡二:其一容一千二百,其一容一千四十七。



 後魏前尺黃鐘容一千一百一十五。



 後周玉尺黃鐘容一千二百六十七。



 後魏中尺黃鐘容一千五百五十五。



 後魏后尺黃鐘容一千八百一十九。



 東魏尺黃鐘容二千八百六十九。



 萬寶常水尺律母黃鐘容黍一千三百二十。



 梁表、鐵尺律黃鐘副別者,其長短及口空之圍徑並同,而容黍或多或少,皆是作者旁庣其腹,使有盈虛。



 侯氣後齊神武霸府田曹參軍信都芳,深有巧思,能以管候氣,仰觀雲色。嘗與人對語,即指天曰:「孟春之氣至矣。」人往驗管,而飛灰已應。每月所候,言皆無爽。



 又為輪扇二十四,埋地中,以測二十四氣。每一氣感,則一扇自動,他扇並住,與管灰相應,若符契焉。



 開皇九年平陳後,高祖
 遣毛爽及蔡子元、于普明等,以候節氣。依古,於三重密屋之內,以木為案,十有二具。每取律呂之管,隨十二辰位,置於案上,而以土埋之,上平於地,中實葭莩之灰,以輕緹素覆律口。每其月氣至,與律冥符,則灰飛沖素,蔂出於外。而氣應有早晚,灰飛有多少,或初入月其氣即應;或至中下旬間,氣始應者;或灰飛出,三五夜而盡;或終一月,才飛少許者。高祖異之,以問牛弘。弘對曰:「灰飛半出為和氣,吹灰全出為猛氣,吹灰不能出為衰氣。和氣應者其政平,猛氣應者其臣縱,衰氣應者其君暴。」高祖駁之曰:「臣縱君暴,其政不平,非月別而有異也。今十
 二月律,於一歲內應並不同。安得暴君縱臣,若斯之甚也?」弘不能對。令爽等草定其法。爽因稽諸故實,以著於篇,名曰《律譜》。



 其略云:臣爽按,黃帝遣伶倫氏取竹於解谷,聽鳳阿閣之下,始造十二律焉。乃致天地氣應,是則數之始也。陽管為律,陰管為呂,其氣以候四時,其數以紀萬物。云隸首作數,蓋律之本也。夫一、十、百、千、萬、億、兆者,引而申焉,歷度量衡,出其中矣。故有虞氏用律和聲,鄒衍改之,以定五始。正朔服色,亦由斯而別也。



 夏正則人,殷正則地,周正則天。孔子曰:「吾得夏時焉。」謂得氣數之要矣。



 漢初興也,而張蒼定律,乃推五勝之法,以為水
 德。實因戰國官失其守,後秦滅學,其道浸微,蒼補綴之,未獲詳究。及孝武創制,乃置協律之官,用李延年以為都尉,頗解新聲變曲,未達音律之源,故其服色不得而定也。至於元帝,自曉音律,郎官京房,亦達其妙,因使韋玄成等雜試問房。房自敘云:「學焦延壽,用六十律相生之法。以上生下,皆三生二,以下生上,皆三生四。陽下生陰,陰上生陽,乃還相為宮之正法也。」於後劉歆典領條奏,著其始末,理漸研精。班氏《漢志》,盡歆所出也,司馬彪《志》,並房所出也。



 至於後漢,尺度稍長。魏代杜夔,亦制律呂,以之候氣,灰悉不飛。晉光祿大夫荀勖,得古銅管,校
 夔所制,長古四分,方知不調,事由其誤。乃依《周禮》,更造古尺,用之定管,聲韻始調。



 左晉之後,漸又訛謬。至梁武帝時,猶有汲塚玉律,宋蒼梧時,鉆為橫吹,然其長短厚薄,大體具存。臣先入棲誠,學算於祖恆,問律於何承天,沈研三紀,頗達其妙。後為太常丞,典司樂職,乃取玉管及宋太史尺,並以聞奏。詔付大匠,依樣制管。自斯以後,律又飛灰。侯景之亂,臣兄喜於太樂得之。後陳宣帝詣荊州為質,俄遇梁元帝敗,喜沒於周。適欲上聞,陳武帝立,遂又以十二管衍為六十律,私侯氣序,並有徵應。至太建時,喜為吏部尚書,欲以聞奏。會宣帝崩,後主嗣立,
 出喜為永嘉內史,遂留家內,貽諸子孫。陳亡之際,竟並遺失。



 今正十二管在太樂者,陽下生陰,始於黃鐘,陰上生陽,終於中呂,而一歲之氣,畢於此矣。中呂上生執始,執始下生去滅,終於南事。六十律候,畢於此矣。



 仲冬之月,律中黃鐘。黃鐘者,首於冬至,陽之始也。應天之數而長九寸,十一月氣至,則黃鐘之律應,所以宣養六氣,緝和九德也。自此之後,並用京房律準,長短宮徵,次日而用。凡十二律,各有所攝,引而申之,至於六十。亦由八卦衍而重之,以為六十四也。相生者相變。始黃鐘之管,下生林鐘,以陽生陰,故變也。相攝者相通。如中呂之管,攝
 於物應,以母權子。故相變者,異時而各應,相通者,同月而繼應。應有早晚者,非正律氣,乃子律相感,寄母中應也。



 其律,大業末於江都淪喪。



 律直日宋錢樂之因京房南事之餘,更生三百律。至梁博士沈重鐘《律議》曰:「《易》以三百六十策當期之日,此律歷之數也。《淮南子》云:『一律而生五音,十二律而為六十音,因而六之,故三百六十音,以當一歲之日。律歷之數,天地之道也。』此則自古而然矣。」重乃依《淮南》本數,用京房之術求之,得三百六十律。各因月之本律,以為一部。以一部
 律數為母,以一中氣所有日為子,以母命子,隨所多少,各一律所建日辰分數也。以之分配七音,則建日冬至之聲,黃鐘為宮,太簇為商,林鐘為徵,南呂為羽,姑洗為角,應鐘為變宮,蕤賓為變徵。五音七聲,於斯和備。其次日建律,皆依次類運行。當日者各自為宮,而商徵亦以次從。以考聲徵氣,辨識時序,萬類所宜,各順其節。自黃鐘終於壯進,一百五十律,皆三分損一以下生。自依行終於億兆,二百九律,皆三分益一以上生。唯安運一律為終,不生。



 其數皆取黃鐘之實十七萬七千一百四十七為本,以九三為法,各除其實,得寸分及小分,餘皆委
 之。即各其律之長也。修其律部,則上生下生宮徵之次也。今略其名次云。



 黃鐘:包育含微帝德廣運下濟克終執始握鑒持樞黃中通聖潛升殷普景盛滋萌光被咸亨乃文乃聖微陽分動生氣雲繁鬱湮升引屯結開元質未人愛昧逋建玄中玉燭調風右黃鐘一部,三十四律。每律直三十四分日之三十
 一大呂:荄動始贊大有坤元輔時匡弼分否又繁唯微棄望庶幾執義秉強陵陰侶陽識沈緝熙知道適時權變少出阿衡同雲承明善述休光右大呂一部,二十七律。每律直一日及二十七分日之三太簇:未知其己義建亭毒條風湊始時息達生匏奏初角少陽柔橈
 商音屈齊扶弱承齊動植咸擢兼山止速隨期龍躍勾芒調序青要結萼延敷刑晉辨秩東作贊揚顯滯俶落右太簇一部,三十四律。



 夾鐘明庶協侶陰贊風從布政萬化開時震德乘條芬芳散朗淑氣風馳佚喜幹黨四隙種生恣性逍遙仁威爭南旭旦晨朝生遂
 群分潔新右夾鐘一部,二十七律。



 姑洗:南授懷來考神方顯攜角洗陳變虞擢穎嘉氣始升卿云媚嶺疏道路時日旅實沈炎風首節柔條方結刑始方齊物華革荑茂實登明壯進下生安運依行上生包育少選道從硃黻揚庭含貞右姑洗一部,三十四律。



 中呂:硃明啟運景風初緩羽物斯奮南中離春率農有程南訛敬致相趣內貞硃草含輝屈軼曜疇巳氣清和物應戒荒落貞軫天庭祚周右中呂一部,二十七律。



 蕤賓:南事京房終律謐靜則選布萼滿羸潛動盛變賓安懷遠聲暨軌同
 海水息沴離躬安壯崇明遠眺升中鳳翥朝陽制時瑞通鶉火乂次高焰其煌。



 右蕤賓一部,二十七律。



 林鐘:謙侍崇德循道方壯陰升靡慝去滅華銷朋慶雲布均任仰成寬中安度德均無蹇禮溢智深任肅純恪歸嘉美音溫風候節蓂華繡嶺物無否與景口曜井
 日煥重輪財華右林鐘一部,三十四律。



 夷則:升商清爽氣精陰德白藏御敘鮮刑貞克金天劉獮會道歸仁陰侶去南陽消柔辛延乙和庚靡卉荑晉分積孔修九德咸藎僉惟俾乂右夷則一部,二十七律。



 南呂:
 白呂捐秀敦實素風勁物酋稔結躬肥遁羸中晟陰抗節威遠有截歸期中德王猷允塞蓐收撙轡搖落未印質隨分滿道心貞堅蓄止歸藏夷汗均義悅使亡勞九有光賁右南呂一部,三十四律。



 無射:思沖懷謙恭儉休老恤農銷祥閉奄降婁藏邃日在旋春閹藏
 明奎鄰齊軌眾大蓄嗇斂下濟息肩無邊期保延年秋深野色玄月澄天右無射一部,二十七律。



 應鐘:分焉祖微據始功成乂定靜謐遲內無為而乂姑射凝晦動寂應徵未育萬機萬壽無疆地久天長修復遲時方制無休九野八荒億兆安運
 右應鐘一部,二十八律。



 審度《史記》曰:「夏禹以身為度,以聲為律。」《禮記》曰:「丈夫布手為尺。」



 《周官》云:「璧羨起度。」鄭司農云:「羨,長也。此璧徑尺,以起度量。」



 《易緯通卦驗》:「十馬尾為一分。」《淮南子》云:「秋分而禾緌定,緌定而禾熟。律數十二而當一粟,十二粟而當一寸。」緌者,禾穗芒也。《說苑》云:「度量權衡以粟生,一粟為一分。」《孫子算術》云:「蠶所生吐絲為忽,十忽為秒,十秒為毫,十毫為厘,十厘為分。」此皆起度之源,其文舛互。唯《漢志》:「度者,所以度長短也,本起黃鐘之長。以子穀秬黍中
 者,一黍之廣度之,九十黍為黃鐘之長。一黍為一分,十分為一寸,十寸為一尺,十尺為一丈,十丈為一引,而五度審矣。」後之作者,又憑此說,以律度量衡,並因秬黍散為諸法,其率可通故也。



 黍有大小之差,年有豐耗之異,前代量校,每有不同,又俗傳訛替,漸致增損。今略諸代尺度一十五等,並異同之說如左。



 一、周尺《漢志》王莽時劉歆銅斛尺。



 後漢建武銅尺。



 晉泰始十年荀勖律尺,為晉前尺。



 祖沖之所傳銅尺。



 徐廣、徐爰、王隱等《晉書》云:「武帝泰始九年,中書監荀勖校太樂八音,不和,始知為後漢至魏,尺長於古四分有餘。勖乃部著作郎劉恭,依《周禮》制尺,所謂古尺也。依古尺更鑄銅律呂,以調聲韻。以尺量古器,與本銘尺寸無差。又汲郡盜發魏襄王塚,得古周時玉律及鐘磬,與新律聲韻暗同。於時郡國或得漢時故鐘,吹新律命之,皆應。」梁武《鐘律緯》云:「祖沖之所傳銅尺,其銘曰:『晉泰始十年,中書考古器,揆校今尺,長四分半。所校古法有七品:一曰姑洗玉律,二曰小呂玉律,三曰西京銅望臬,四曰
 金錯望臬,五曰銅斛,六曰古錢,七曰建武銅尺。



 姑洗微強,西京望臬微弱,其餘與此尺同。』銘八十二字。此尺者,勖新尺也。今尺者,杜夔尺也。雷次宗、何胤之二人作《鐘律圖》,所載荀勖校量古尺文,與此銘同。而蕭吉樂譜,謂為梁朝所考七品,謬也。今以此尺為本,以校諸代尺」云。



 二、晉田父玉尺梁法尺,實比晉前尺一尺七厘。



 《世說》稱,有田父於野地中得周時玉尺,便是天下正尺。荀勖試以校尺,所造金石絲竹,皆短校一米。梁武帝《鐘律緯稱》,主衣從上相承,有周時銅尺一枚,古玉律八枚。
 檢主衣周尺,東昏用為章信,尺不復存。玉律一囗蕭,餘定七枚夾鐘,有昔題刻。乃制為尺,以相參驗。取細毫中黍,積次詶定,今之最為詳密,長祖沖之尺校半分。以新尺制為四器,名為通。又依新尺為笛,以命古鐘,按刻夷則,以笛命飲和韻,夷則定合。案此兩尺長短近同。



 三、梁表尺實比晉前尺一尺二分二厘一毫有奇。



 蕭吉云:「出於《司馬法》。梁朝刻其度於影表,以測影。」案此即奉朝請祖恆所算造銅主影表者也。經陳滅入朝。大業中,議以合古,乃用之調律,以制鐘磬等八音樂器。



 四、漢官尺實比晉前尺一尺三分七毫。



 晉時始平掘地得古銅尺。



 蕭吉《樂譜》云:「漢章帝時,零陵文學史奚景於泠道縣舜廟下得玉律,度為此尺。」傅暢《晉諸公贊》云:「葛勖造鐘律,時人並稱其精密,唯陳留阮咸,譏其聲高。後始平掘地,得古銅尺,歲久欲腐,以校荀勖今尺,短校四分。時人以咸為解。」此兩尺長短近同。



 五、魏尺杜夔所用調律,比晉前尺一尺四分七厘。



 魏陳留王景元四年,劉徽注《九章》云,王莽時劉歆斛尺,弱於今尺四分五厘,比魏尺,其斛深九寸五分五厘。即晉荀勖所云「杜夔尺長於今尺四分半」是也。



 六、晉后尺實比晉前尺一尺六分二厘。



 蕭吉雲,晉氏江東所用。



 七、後魏前尺實比晉前尺一尺二寸七厘。



 八、中尺實比晉前尺一尺二寸一分一厘。



 九、后尺實比晉前尺一尺二寸八分一厘。即開皇官尺及後周市尺後周市尺,比玉尺一尺九分三厘。



 開皇官尺,即鐵尺,一尺二寸。



 此後魏初及東西分國,後周未用玉尺之前,雜用此等尺。



 甄鸞《算術》云:「周朝市尺,得玉尺九分二厘。」或傳梁時有志公道人作此尺,寄入周朝;雲與多須老翁。周太祖
 及隋高祖,各自以為謂己。周朝人間行用。



 及開皇初,著令以為官尺,百司用之,終於仁壽。大業中,人間或私用之。



 十、東後魏尺實比晉前尺一尺五寸八毫。



 此是魏中尉元延明累黍用半周之廣為尺,齊朝因而用之。魏收《魏史·律歷志》云:「公孫崇永平中更造新尺,以一黍之長,累為寸法。尋太常卿劉芳受詔修樂,以秬黍中者一黍之廣,即為一分。而中尉元匡,以一黍之廣度黍二縫,以取一分。



 三家紛競,久不能決。大和十九年高祖詔,以一黍之廣,用成分體,九十之黍,黃鐘之長,以定
 銅尺。有司奏從前詔,而芳尺同高祖所制,故遂典修金石。迄武定未有論律者。」



 十一、蔡邕銅籥尺後周玉尺,實比晉前尺一尺一寸五分八厘。



 從上相承,有銅籥一,以銀錯題,其銘曰:「籥,黃鐘之宮,長九寸,空圍九分,容秬黍一千二百粒,稱重十二銖,兩之為一合。三分損益,轉生十二律。」祖孝孫云:「相承傳是蔡邕銅籥。」



 後周武帝保定中,詔遣大宗伯盧景宣、上黨公長孫紹遠、岐國公斛斯徵等,累黍造尺,從橫不定。後因修倉掘地,得古玉斗,以為正器,據斗造律度量衡。因用
 此尺,大赦,改元天和,百司行用,終於大象之末。其律黃鐘,與蔡邕古籥同。



 十二、宋氏尺實比晉前尺一尺六分四厘。



 錢樂之渾天儀尺。



 後周鐵尺。



 開皇初調鐘律尺及平陳後調鐘律水尺。



 此宋代人間所用尺,傳入齊、梁、陳,以制樂律。與晉后尺及梁時俗尺、劉曜渾天儀尺,略相依近。當由人間恆用,增損訛替之所致也。周建德六年平齊後,即以此同律度量,頒於天下。其後宣帝時,達奚震及牛弘等議曰:竊惟權衡度量,
 經邦懋軌,誠須詳求故實,考校得衷。謹尋今之鐵尺,是太祖遣尚書故蘇綽所造,當時檢勘,用為前周之尺。驗其長短,與宋尺符同,即以調鐘律,並用均田度地。今以上黨羊頭山黍,依《漢書·律歷志》度之。若以大者稠累,依數滿尺,實於黃鐘之律,須撼乃容。若以中者累尺,雖復小稀,實於黃鐘之律,不動而滿。計此二事之殊,良由消息未善,其於鐵尺,終有一會。且上黨之黍,有異他鄉,其色至烏,其形圓重,用之為量,定不徒然。正以時有水旱之差,地有肥瘠之異,取黍大小,未必得中。案許慎解,秬黍體大,本異於常。疑今之大者,正是其中,累百滿尺,即
 是會古。賓籥之外,才剩十餘,此恐圍徑或差,造律未妙。



 就如撼動取滿,論理亦通。今勘周漢古錢,大小有合,宋氏渾儀,尺度無舛。又依《淮南》,累粟十二成寸。明先王制法,索隱鉤深,以律計分,義無差異。《漢書·食貨志》云:「黃金方寸,其重一斤。」今鑄金校驗,鐵尺為近。依文據理,符會處多。且平齊之始,已用宣布,今因而為定,彌合時宜。至於玉尺累黍,以廣為長,累既有剩,實復不滿。尋訪古今,恐不可用。其晉、梁尺量,過為短小,以黍實管,彌復不容,據律調聲,必致高急。且八音克諧,明王盛範,同律度量,哲後通規。臣等詳校前經,斟量時事,謂用鐵尺,於理為
 便。



 未及詳定,高祖受終,牛弘、辛彥之、鄭譯、何妥等,久議不決。既平陳,上以江東樂為善,曰:「此華夏舊聲,雖隨俗改變,大體猶是古法。」祖孝孫云:「平陳後,廢周玉尺律,便用此鐵尺律,以一尺二寸即為市尺。」



 十三、開皇十年萬寶常所造律呂水尺實比晉前尺一尺一寸八分六厘。



 今太樂庫及內出銅律一部,是萬寶常所造,名水尺律。說稱其黃鐘律當鐵尺南呂倍聲。南呂,黃鐘羽也,故謂之水尺律。



 十四、雜尺趙劉曜渾天儀土圭尺,長於梁法尺四分三厘,實比晉前尺一尺五分。



 十五、梁朝俗間尺長於梁法尺六分三厘、於劉曜渾儀尺二分,實比晉前尺一尺七分一厘。梁武《鐘律緯》云:「宋武平中原,送渾天儀土圭,云是張衡所作。



 驗渾儀銘題,是光初四年鑄,土圭是光初八年作。並是劉曜所制,非張衡也。制以為尺,長今新尺四分三厘,短俗間尺二分。」新尺謂梁法尺也。



 嘉量《周禮》,蠙氏「為量,鬴深尺,內方尺而圓其外,其實一鬴;其臀一寸,其實一豆;其耳三寸,其實一升。重一鈞。其聲中黃鐘。概而不稅。其銘曰:時文思索,允臻其極。嘉量既成,
 以觀四國。永啟厥後,茲器維則。」《春秋左氏傳》曰:「齊舊四量,豆、區、鬴、鐘。四升曰豆,各自其四,以登於鬴。」六斗四升也。



 「鬴十則鐘」,六十四斗也。鄭玄以為方尺積千寸,比九章粟米法少二升八十一分升之二十二。祖沖之以算術考之,積凡一千五百六十二寸半。方尺而圓其外,減傍一厘八毫,共徑一尺四寸一分四毫七秒二忽有奇而深尺,即古斛之制也。《九章商功法》程粟一斛,積二千七百寸。米一斛,積一千六百二十寸。菽荅麻麥一斛,積二千四百三十寸。此據精粗為率,使價齊而不等。其器之積寸也,以米斛為正,則同於《漢志》。《孫子算術》曰:六粟
 為圭,十圭為秒,十秒為撮,十撮為勺,十勺為合。」應勛曰:「圭者自然之形,陰陽之始。四圭為撮。」孟康曰:「六十四黍為圭。」《漢志》曰:「量者,籥、合、升、斗、斛也,所以量多少也。本起於黃鐘之龠。用度數審其容,以子穀秬黍中者千有二百實其龠,以井水準其概。十龠為合,十合為升,十升為斗,十斗為斛,而五量嘉矣。其法用銅方尺而圓其外,旁有庣焉。其上為斛,其下為斗,左耳為升,右耳為合、龠。其狀似爵,以縻爵祿。



 上三下二,參天兩地。圓而函方,左一右二,陰陽之象也。圓象規,其重二鈞,備氣物之數,各萬有一千五百二十也。聲中黃鐘,始於黃鐘而反覆焉。」
 其斛銘曰:「律嘉量斛,方尺而圓其外,庣旁九厘五毫,冪百六十二寸,深尺,積一千六百二十寸,容十斗。」祖沖之以圓率考之,此斛當徑一尺四寸三分六厘一毫九秒二忽,庣旁一分九毫有奇。劉歆庣旁少一厘四毫有奇,歆數術不精之所致也。



 魏陳留王景元四年,劉徽注《九章商功》曰:「當今大司農斛圓徑一尺三寸五分五厘,深一尺,積一千四百四十一寸十分之三。王莽銅斛於今尺為深九寸五分五厘,徑一尺三寸六分八厘七毫。以徽術計之,於今斛為容九斗七升四合有奇。」此魏斛大而尺長,王莽斛小而尺
 短也。



 梁、陳依古。齊以古升五升為一斗。



 後周武帝「保定元年辛巳五月,晉國造倉,獲古玉斗。暨五年乙酉冬十月,詔改制銅律度,遂致中和。累黍積龠,同茲玉量,與衡度無差。準為銅升,用頒天下。



 內徑七寸一分,深二寸八分,重七斤八兩。天和二年丁亥,正月癸酉朔,十五日戊子校定,移地官府為式。」此銅升之銘也。其玉升銘曰:「維大周保定元年,歲在重光,月旅蕤賓,晉國之有司,修繕倉廩,獲古玉升,形制典正,若古之嘉量。太師晉國公以聞,敕納於天府。暨五年歲在協洽,皇帝乃詔稽準繩,考灰律,
 不失圭撮,不差累黍。遂熔金寫之,用頒天下,以合太平權衡度量。」今若以數計之,玉升積玉尺一百一十寸八分有奇,斛積一千一百八五分七厘三毫九秒。又甄鸞《算術》云:「玉升一升,得官斗一升三合四勺。「此玉升大而官斗小也。以數計之,甄鸞所據後周官斗,積玉尺九十七寸有奇,斛積九百七十七寸有奇。後周玉斗並副金錯銅斗及建德六年金錯題銅斗,實同以秬黍定量。以玉稱權之,一升之實,皆重六斤十三兩。



 開皇以古斗三升為一升。大業初,依復古斗。



 衡權
 衡者,平也;權者,重也。衡所以任權而鈞物平輕重也。其道如底,以見準之正,繩之直。左旋見規,右折見矩。其在天也,佐助璇璣,斟酌建指,以齊七政,故曰玉衡。權者,銖、兩、斤、鈞、石也,以秤物平施,知輕重也。古有黍、R、錘、錙、鐶、鉤、鋝、鎰之目,歷代差變,其詳未聞。《前志》曰:權本起於黃鐘之重。一龠容千二百黍,重十二銖。兩之為兩,二十四銖為兩。十六兩為斤。三十斤為鈞。四鈞為石。五權謹矣。其制以義立之,以物鈞之。其餘大小之差,以輕重為宜。圜而環之,令之肉倍好者,周旋亡端,終而復始,亡窮已也。權與物鈞而生衡,衡運生規,規圓生矩,矩方生繩,繩
 直生準。準正則衡平而鈞權矣。是為五則,備於鈞器,以為大範。案《趙書》,石勒十八年七月,造建德殿,得圓石,狀如水碓。其銘曰:「律權石,重四鈞,同律度量衡。有辛氏造。」續咸議是王莽時物。



 後魏景明中,並州人王顯達獻古銅權一枚,上銘八十一字。其銘云:律權石,重四鈞。」又云:「黃帝初祖,德匝於虞。虞帝始祖,德匝於新。歲在大梁,龍集戊辰。



 戊辰直定,天命有人。據土德受,正號即真。改正建丑,長壽隆崇。同律度量衡,稽當前人。龍在己巳,歲次實沈,初班天下,萬國永遵。子子孫孫,享傳億年。」



 此亦王莽所制也。其時太樂令公孫崇依《漢志》先修稱尺,及見此權,
 以新稱稱之,重一百二十斤。新稱與權,合若符契。於是付崇調樂。孝文時,一依《漢志》作斗尺。



 梁、陳依古稱。齊以古稱一斤八兩為一斤。周玉稱四兩,當古稱四兩半。開皇以古稱三斤為一斤,大業中,依復古秤。



\end{pinyinscope}