\article{卷六十六列傳第三十一}

\begin{pinyinscope}

 李諤
 李諤,字士恢,趙郡人也。好學,解屬文。仕齊為中書舍人,有口辯,每接對陳使。周武帝平齊,拜天官都上士,諤見高祖有奇表,深自結納。及高祖為丞相,甚見親待,訪以得失。於時兵革屢動,國用虛耗,諤上《重穀論》以諷焉。高祖深納之。及受禪,歷比部、考功二曹侍郎,賜爵南和伯。諤性公方,明達世務,為時論所推。遷治書侍御史,上謂
 群臣曰:「朕昔為大司馬,每求外職,李諤陳十二策,苦勸不許,朕遂決意在內。今此事業,諤之力也。」賜物二千段。



 諤見禮教凋敝,公卿薨亡,其愛妾侍婢,子孫輒嫁賣之,遂成風俗。諤上書曰:「臣聞追遠慎終,民德歸厚,三年無改,方稱為孝。如聞朝臣之內,有父祖亡沒,日月未久,子孫無賴,便分其妓妾,嫁賣取財。有一於茲,實損風化。妾雖微賤,親承衣履,服斬三年,古今通式。豈容遽褫縗弊,強傅鉛華,泣辭靈幾之前,送付他人之室。凡在見者,猶致傷心,況乎人子,能堪斯忍?復有朝廷重臣,位望通貴,平生交舊,情若弟兄,及其亡沒,杳同行路,朝聞其死,夕
 規其妾,方便求娉,以得為限,無廉恥之心,棄友朋之義。且居家理治,可移於官,既不正私,何能贊務?」



 上覽而嘉之。五品以上妻妾不得改醮,始於此也。



 諤又以屬文之家,體尚輕薄,遞相師效,流宕忘反,於是上書曰:臣聞古先哲王之化民也,必變其視聽,防其嗜欲,塞其邪放之心,示以淳和之路。五教六行為訓民之本,《詩》《書》《禮》《易》為道義之門。故能家復孝慈,人知禮讓,正俗調風,莫大於此。其有上書獻賦,制誄鐫銘,皆以褒德序賢,明勛證理。茍非懲勸,義不徒然。降及後代,風教漸落。魏之三祖,更尚文詞,忽君人之大道,好雕蟲之小藝。下之從上,有同
 影響,競騁文華,遂成風俗。江左齊、梁,其弊彌甚,貴賤賢愚,唯務吟詠。遂復遺理存異,尋虛逐微,競一韻之奇,爭一字之巧。連篇累牘,不出月露之形,積案盈箱,唯是風雲之狀。世俗以此相高,朝廷據茲擢士。祿利之路既開,愛尚之情愈篤。於是閭里童昏,貴游總丱,未窺六甲,先制五言。至如羲皇、舜、禹之典,伊、傅、周、孔之說,不復關心,何嘗入耳。



 以傲誕為清虛,以緣情為勛績,指儒素為古拙,用詞賦為君子。故文筆日繁,其政日亂,良由棄大聖之軌模,構無用以為用也。損本逐末,流遍華壤,遞相師祖,久而愈扇。及大隋受命,聖道聿興,屏黜輕浮,遏止華
 偽,自非懷經抱質,志道依仁,不得引預搢紳,參廁纓冕。開皇四年,普詔天下,公私之翰,並宜實錄。其年九月,泗州刺史司馬幼之文表華艷,付所司治罪。自是公卿大臣,咸知正路,莫不鉆仰墳集,棄絕華綺,擇先王之令典,行大道於茲世。如聞外州遠縣,仍鐘敝風,選吏舉人,未遵典則,至有宗黨稱孝,鄉曲歸仁,學必典謨,交不茍合,則擯落私門,不加收齒;其學不稽古,逐俗隨時,作輕薄之篇章,結朋黨而求譽,則選充吏職,舉送天朝。蓋由縣令、刺史未行風教,猶挾私情,不存公道。臣既忝憲司,職當糾察。



 若聞風即劾,恐掛網者多,請勒諸司,普加搜訪,
 有如此者,具狀送臺。



 諤又以當官者好自矜伐,復上奏曰:臣聞舜戒禹云:「汝惟不矜,天下莫與汝爭能;汝惟不伐,天下莫與汝爭功。」



 言偃又云:「事君數,斯辱矣,朋友數,斯疏矣。」此皆先哲之格言,後王之軌轍。



 然則人臣之道,陳力濟時,雖勤比大禹,功如師望,亦不得厚自矜伐,上要君父。



 況復功無足紀,勤不補過,而敢自陳勛績,輕干聽覽!世之喪道,極於周代,下無廉恥,上使之然。用人唯信其口,取士不觀其行。矜誇自大,便以幹濟蒙擢;謙恭靜退,多以恬默見遺。是以通表陳誠,先論己之功狀;承顏敷奏,亦道臣最用心。



 自衒自媒,都無慚恥之色;強幹
 橫請,唯以乾沒為能。自隋受命,此風頓改,耕夫販婦,無不革心,況乃大臣,仍遵敝俗!如聞刺史入京朝覲,乃有自陳勾檢之功,喧訴階墀之側,言辭不遜,高自稱譽,上黷冕旒,特為難恕,凡如此輩,具狀送臺,明加罪黜,以懲風軌。



 上以諤前後所奏頒示天下,四海靡然向風,深革其弊。諤在職數年,務存大體,不尚嚴猛,由是無剛謇之譽,而潛有匡正多矣。邳公蘇威以臨道店舍,乃求利之徒,事業污雜,非敦本之義,遂奏高祖,約遣歸農,有願依舊者,所在州縣錄附市籍,仍撤毀舊店,並令遠道,限以時日。正值冬寒,莫敢陳訴。諤因別使,見其如此,以為四
 民有業,各附所安,逆旅之與旗亭,自古非同一概,即附市籍,於理不可,且行旅之所托,豈容一朝而廢,徒為勞擾,於事非宜,遂專決之,並令依舊,使還詣闕,然後奏聞。高祖善之曰:「體國之臣,當如此矣。」以年老,出拜通州刺史,甚有惠政,民夷悅服。後三歲,卒官,有子四人。大體、大鈞,並官至尚書郎。世子大方襲爵,最有材品,大業初,判內史舍人。帝方欲任之,遇卒。



 鮑宏鮑宏,字潤身,東海郯人也。父機,以才學知名。事梁,官至治書侍御史。宏七歲而孤,為兄泉之所愛育。年十二,能
 屬文,嘗和湘東王繹詩,繹嗟賞不已,引為中記室,遷鎮南府諮議、尚書水部郎,轉通直散騎侍郎。江陵既平,歸於周。明帝甚禮之,引為麟趾殿學士。累遷遂伯下大夫,與杜子暉聘於陳,謀伐齊也。陳遂出兵江北以侵齊。帝嘗問宏取齊之策,宏對云:「我強齊弱,勢不相侔。齊主暱近小人,政刑日紊,至尊仁惠慈恕,法令嚴明。事等建瓴,何憂不克。但先皇往日出師洛陽,彼有其備,每不克捷。如臣計者,進兵汾、潞,直掩晉陽,出其不虞,以為上策。」帝從之。及定山東,除少御正,賜爵平遙縣伯,邑六百戶,加上儀同。



 高祖作相,奉使山南。會王謙舉兵於蜀,路次潼
 州,為謙將達奚期所執,逼送成都,竟不屈節。謙敗之後,馳傳入京,高祖嘉之,賜以金帶。及受禪,加開府,除利州刺史,進爵為公。轉邛州刺史,秩滿還京。時有尉義臣者,其父崇不從尉迥,後復與突厥戰死,上嘉之,將賜姓為金氏。訪及群下,宏對曰:「昔項伯不同項羽,漢高賜姓劉氏,秦真父能死難,魏武賜姓曹氏。如臣愚見,請賜以皇族。」高祖曰:「善。」因賜義臣姓為楊氏。後授均州刺史,以目疾免,卒於家,時年九十六。初,周武帝敕宏修《皇室譜》一部,分為《帝緒》、《疏屬》、《賜姓》三篇。有集十卷,行於世。



 裴政
 裴政,字德表,河東聞喜人也。高祖壽孫,從宋武帝徙家於壽陽,歷前軍長史、廬江太守。祖邃,梁侍中、左衛將軍、豫州大都督。父之禮,廷尉卿。政幼明敏,博聞強記,達於時政,為當時所稱。年十五,闢邵陵王府法曹參軍事,轉起部郎、枝江令。湘東王之臨荊州也,召為宣惠府記室,尋除通直散騎侍郎。侯景作亂,加壯武將軍,帥師隨建寧侯王琳進討之。擒賊率宋子仙,獻於荊州。及平侯景,先鋒入建鄴,以軍功連最封夷陵侯。征授給事黃門侍郎,復帥師副王琳拒蕭紀,破之於硤口。加平越中郎將、鎮南府長史。及周師圍荊州,琳自桂州來赴難,次於長
 沙。



 政請從間道先報元帝。至百里洲,為周人所獲,蕭詧謂政曰:「我武皇帝之孫也,不可為爾君乎?爾亦何煩殉身於七父?若從我計,則貴及子孫;如或不然,分腰領矣。」政詭曰:「唯命。」詧鎖之,送至城下,使謂元帝曰:「王僧辯聞臺城被圍,已自為帝。王琳孤弱,不復能來。」政許之。既而告城中曰:「援兵大至,各思自勉。吾以間使被擒,當以碎身報國。」監者擊其口,終不易辭。詧怒,命趣行戮。



 蔡大業諫曰:「此民望也。若殺之,則荊州不可下矣。」因得釋。會江陵陷,與城中朝士俱送於京師。周文帝聞其忠,授員外散騎侍郎,引事相府。命與盧辯依《周禮》建六卿,設公卿
 大夫士,並撰次朝儀,車服器用,多遵古禮,革漢、魏之法,事並施行。尋授刑部下大夫,轉少司憲。政明習故事,又參定《周律》。能飲酒,至數斗不亂。簿案盈幾,剖決如流,用法寬平,無有冤濫。囚徒犯極刑者,乃許其妻子入獄就之,至冬,將行決,皆曰:「裴大夫致我於死,死無所恨。」其處法詳平如此。又善鐘律,嘗與長孫紹遠論樂,語在《音律志》。宣帝時,以忤旨免職。



 高祖攝政,召復本官。開皇元年,轉率更令,加位上儀司三司。詔與蘇威等修定律令。政採魏、晉刑典,下至齊、梁,沿革輕重,取其折衷。同撰著者十有餘人,凡疑滯不通,皆取決於政。進位散騎常侍,轉
 左庶子,多所匡正,見稱純愨。東宮凡有大事,皆以委之。右庶子劉榮,性甚專固。時武職交番,通事舍人趙元愷作辭見帳,未及成。太子有旨,再三催促,榮語元愷云:「但爾口奏,不須造帳。」及奏,太子問曰:「名帳安在?」元愷曰:「稟承劉榮,不聽造帳。」太子即以詰榮,榮便拒諱,云「無此語」。太子付政推問。未及奏狀,有附榮者先言於太子曰:「政欲陷榮,推事不實。」太子召責之,政奏曰:「凡推事有兩,一察情,一據證,審其曲直,以定是非。臣察劉榮,位高任重,縱令實語元愷,蓋是纖介之愆。計理而論,不須隱諱。又察元愷受制於榮,豈敢以無端之言妄相點累。二人之
 情,理正相似。元愷引左衛率崔茜等為證,茜等款狀悉與元愷符同。察情既敵,須以證定。



 臣謂榮語元愷,事必非虛。」太子亦不罪榮,而稱政平直。



 政好面折人短,而退無後言。時雲定興數入侍太子,為奇服異器,進奉後宮,又緣女寵,來往無節。政數切諫,太子不納。政因謂定興曰:「公所為者,不合禮度。又元妃暴薨,道路籍籍,此於太子非令名也。願公自引退,不然將及禍。」定興怒,以告太子,太子益疏政,由是出為襄州總管。妻子不之官,所受秩奉,散給僚吏。民有犯罪者,陰悉知之,或竟歲不發,至再三犯,乃因都會時,於眾中召出,親案其罪,五人處死,
 流徙者甚眾,合境惶懾,令行禁止,小民蘇息,稱為神明。



 爾後不修囹圄,殆無爭訟。卒官,年八十九。著《承聖降錄》十卷。及太子廢,高祖追憶之曰:「向遣裴政、劉行本在,共匡弼之,猶應不令至此。」子南金,仕至膳部郎。



 柳莊柳莊,字思敬,河東解人也。祖季遠,梁司徒從事中郎。父遐,霍州刺史。莊少有遠量,博覽墳籍,兼善辭令。濟陽蔡大寶有重名於江左,時為岳陽王蕭詧咨議,見莊便嘆曰:「襄陽水鏡,復在於茲矣。」大寶遂以女妻之,俄而詧闢為參軍,轉法曹。及詧稱帝,還署中書舍人,歷給事黃門
 侍郎、吏部郎中、鴻臚卿。及高祖輔政,蕭巋令莊奉書入關。時三方構難,高祖懼巋有異志,及莊還,謂莊曰:「孤昔以開府從役江陵,深蒙梁主殊眷。今主幼時艱,猥蒙顧托,中夜自省,實懷慚懼。



 梁主奕葉重光,委誠朝廷,而今已後,方見松筠之節。君還本國,幸申孤此意於梁主也。」遂執莊手而別。時梁之將帥咸潛請興師,與尉迥等為連衡之勢,進可以盡節於周氏,退可以席卷山南。唯巋疑為不可。會莊至自長安,具申高祖結托之意,遂言於巋曰:「昔袁紹、劉表、王凌、諸葛誕之徒,並一時之雄傑也。及據要害之地,擁哮闞之群,功業莫建,而禍不旋踵者,
 良由魏武、晉氏挾天子,保京都,仗大義以為名,故能取威定霸。今尉迥雖曰舊將,昏耄已甚,消難、王謙,常人之下者,非有匡合之才。況山東、庸蜀從化日近,周室之恩未洽,在朝將相,多為身計,競效節於楊氏。以臣料之,迥等終當覆滅,隋公必移周國。未若保境息民,以觀其變。」巋深以為然,眾議遂止。未幾,消難奔陳,迥及謙相次就戮,巋謂莊曰:「近者若從眾人之言,社稷已不守矣。」



 高祖踐阼,莊又入朝,高祖深慰勉之。及為晉王廣納妃於梁,莊因是往來四五反,前後賜物數千段。蕭琮嗣位,遷太府卿。及梁國廢,授開府儀同三司,尋除給事黃門侍郎,
 並賜以田宅。莊明習舊章,雅達政事,凡所駁正,帝莫不稱善。蘇威為納言,重莊器識,常奏帝云:「江南人有學業者,多不習世務,習世務者,又無學業。能兼之者,不過於柳莊。」高熲亦與莊甚厚。莊與陳茂同官,不能降意,茂見上及朝臣多屬意於莊,心每不平,常謂莊為輕己。帝與茂有舊,曲被引召,數陳莊短。經歷數載,譖醖頗行。尚書省嘗奏犯罪人依法合流,而上處以大闢。莊奏曰:「臣聞張釋之有言,法者天子所與天下共也。今法如是,更重之,是法不信於民心。



 方今海內無事,正是示信之時,伏願陛下思釋之之言,則天下幸甚。」帝不從,由是忤旨。俄
 屬尚藥進丸藥不稱旨,茂因密奏莊不親監臨,帝遂怒。十一年,徐璒等反於江南,以行軍總管長史隨軍討之。璒平,即授饒州刺史,甚有治名。後數載卒官,年六十二。



 源師源師,字踐言,河南洛陽人也。父文宗,有重名於齊,開皇初,終於莒州刺史。



 師早有聲望,起家司空府參軍事,稍遷尚書左外兵郎中,又攝祠部。後屬孟夏,以龍見請雩。時高阿那肱為相,謂真龍出見,大驚喜,問龍所在,師整容報曰:「此是龍星初見,依禮當雩祭郊壇,非謂真龍別有所降。」阿那肱忿然作色曰:「何乃乾知星宿!」祭竟不行。
 師出而竊嘆曰:「國家大事,在祀與戎。禮既廢也,何能久乎?齊亡無日矣。」七年,周武帝平齊,授司賦上士。高祖受禪,除魏州長史,入為尚書考功侍郎,仍攝吏部。朝章國憲,多所參定。十七年,歷尚書左右丞,以明乾著稱。時蜀王秀頗違法度,乃以師為益州總管司馬。俄而秀被徵,秀恐京師有變,將謝病不行。師數勸之不可違命,秀作色曰:「此自我家事,何預卿也!」師垂涕對曰:「師荷國厚恩,忝參府幕,僚吏之節,敢不盡心。但比年以來,國家多故,秦孝王寢疾,奄至薨殂,庶人二十年太子,相次淪廢。聖上之情,何以堪處!



 而有敕追王,已淹時月,今乃遷延未
 去,百姓不識王心,儻生異議,內外疑駭,發雷霆之詔,降一介之使,王何以自明?願王自計之。」秀乃從征。秀廢之後,益州官屬多相連坐,師以此獲免。後加儀同三司。煬帝即位,拜大理少卿。帝在顯仁宮,敕宮外衛士不得輒離所守。有一主帥,私令衛士出外,帝付大理繩之。師據律奏徒,帝令斬之,師奏曰:「此人罪誠難恕,若陛下初便殺之,自可不關文墨。既付有司,義歸恆典,脫宿衛近侍者更有此犯,將何以加之?」帝乃止。轉刑部侍郎。師居職強明,有口辯,而無廉平之稱。未幾,卒官。有子昆玉。



 郎茂
 郎茂,字蔚之,恆山新市人也。父基,齊潁川太守。茂少敏慧,七歲誦《騷》、《雅》,日千餘言。十五師事國子博士河間權會,受《詩》、《易》、《三禮》及玄象、刑名之學。又就國子助教長樂張率禮受《三傳》群言,至忘寢食。家人恐茂成病,恆節其燈燭。及長,稱為學者,頗解屬文。年十九,丁父憂,居喪過禮。仕齊,解褐司空府行參軍。會陳使傅縡來聘,令茂接對之。後奉詔於秘書省刊定載籍。



 遷保城令,有能名,百姓為立《清德頌》。及周武平齊,上柱國王誼薦之,授陳州戶曹。屬高祖為亳州總管,見而悅之,命掌書記。時周武帝為《象經》,高祖從容謂茂曰:「人主之所為也,感天地,動
 鬼神,而《象經》多糾法,將何以致治?」



 茂竊嘆曰:「此言豈常人所及也!」乃陰自結納,高祖亦親禮之。後還家為州主薄。



 高祖為丞相,以書召之,言及疇昔,甚歡。授衛州司錄,有能名。尋除衛國令。時有系囚二百,茂親自究審數日,釋免者百餘人。歷年辭訟,不詣州省。魏州刺史元暉謂茂曰:「長史言衛國民不敢申訴者,畏明府耳。」茂進曰:「民猶水也,法令為堤防。堤防不固,必致奔突,茍無決溢,使君何患哉?」暉無以應之。有民張元預,與從父弟思蘭不睦。丞尉請加嚴法,茂曰:「元預兄弟,本相憎疾,又坐得罪,彌益其忿,非化民之意也。」於是遣縣中耆舊更往敦諭,
 道路不絕。元預等各生感悔,詣縣頓首請罪。茂曉之以義,遂相親睦,稱為友悌。



 茂自延州長史轉太常丞,遷民部侍郎。時尚書右僕射蘇威立條章,每歲責民間五品不遜。或答者乃云:「管內無五品之家。」不相應領,類多如此。又為餘糧簿,擬有無相贍。茂以為繁紆不急,皆奏罷之。數歲,以母憂去職。未期,起令視事。



 又奏身死王事者,子不退田,品官年老不減地,皆發於茂。茂性明敏,剖決無滯,當時以吏乾見稱。仁壽初,以本官領大興令。煬帝即位,遷雍州司馬,尋轉太常少卿。後二歲,拜尚書左丞,參掌選事。茂工法理,為世所稱。時工部尚書宇文愷、右
 翊衛大將軍於仲文競河東銀窟。茂奏劾之曰:「臣聞貴賤殊禮,士農異業,所以人知局分,家識廉恥。宇文愷位望已隆,祿賜優厚,拔葵去織,寂爾無聞,求利下交,曾無愧色。於仲文大將,宿衛近臣,趨侍階庭,朝夕聞道,虞、芮之風,抑而不慕,分銖之利,知而必爭。何以貽範庶僚,示民軌物!若不糾繩,將虧政教。」



 愷與仲文竟坐得罪。茂撰《州郡圖經》一百卷奏之,賜帛三百段,以書付秘府。



 於時帝每巡幸,王綱已紊,法令多失。茂既先朝舊臣,明習世事,然善自謀身,無謇諤之節。見帝忌刻,不敢措言,唯竊嘆而已。以年老,上表乞骸骨,不許。會帝親征遼東,以茂
 為晉陽宮留守。其年,恆山贊治王文同與茂有隙,奏茂朋黨,附下罔上。詔遣納言蘇威、御史大夫裴蘊雜治之。茂素與二人不平,因深文巧詆,成其罪狀。帝大怒,及其弟司隸別駕楚之皆除名為民,徙且末郡。茂怡然受命,不以為憂。在途作《登壟賦》以自慰,詞義可觀。復附表自陳,帝頗悟。十年,追還京兆,歲餘而卒,時年七十五。有子知年。



 高構高構,字孝基,北海人也。性滑稽,多智,辯給過人,好讀書,工吏事,弱冠,州補主簿。仕齊河南王參軍事,歷徐州司
 馬、蘭陵、平原二郡太守。劉滅後,周武帝以為許州司馬。高祖受禪,轉冀州司馬,甚有能名。徵拜比部侍郎,尋轉民部。



 時內史侍郎晉平東與兄子長茂爭嫡,尚書省不能斷,朝臣三議不決。構斷而合理,上以為能,召入內殿,勞之曰:「我聞尚書郎上應列宿,觀卿才識,方知古人之言信矣。嫡庶者,禮教之所重,我讀卿判數遍,詞理愜當,意所不能及。」賜米百石。



 由是知名。尋遷雍州司馬,以明斷見稱。歲餘,轉吏部侍郎,號為稱職。復徙雍州司馬,坐事左轉盩啡令,甚有治名。上善之,復拜雍州司馬,又為吏部侍郎,以公事免。煬帝立,召令復位。時為吏部者,多
 以不稱職去官,唯構最有能名,前後典選之官,皆出其下。時人以構好劇談,頗謂輕薄,然其內懷方雅,特為吏部尚書牛弘所重。後以老病解職,弘時典選,凡將有所擢用,輒遣人就第問其可不。河東薛道衡才高當世,每稱構有清鑒,所為文筆,必先以草呈構,而後出之。構有所詆訶,道衡未嘗不嗟伏。大業七年,終於家,時年七十二。所舉杜如晦、房玄齡等,後皆自致公輔,論者稱構有知人之鑒。



 開皇中,昌黎豆盧實為黃門侍郎,稱為慎密。河東裴術為右丞,多所糾正。河東士燮、平原東方舉、安定皇甫聿道,俱為刑部,並執法平允。弘農劉士龍、清河
 房山基為考功,河東裴鏡民為兵部,並稱明幹。京兆韋焜為民曹,屢進讜言。南陽韓則為延州長史,甚有惠政。此等事行遺闕,皆有吏乾,為當時所稱。



 張虔威張虔威,字元敬,清河東武城人也。父晏之,齊北徐州刺史。虔威性聰敏,涉獵群書。其世父嵩之謂人曰:「虔威,吾家千里駒也。」年十二,州補主簿。十八為太尉中兵參軍,後累遷太常丞。及齊亡,仕周為宣納中士。高祖得政,引為相府典簽。開皇初,晉王廣出鎮並州,盛選僚佐,以虔威為刑獄參軍,累遷為屬。王甚美其才,與河內張衡俱
 見禮重,晉邸稱為「二張」焉。及王為太子,遷員外散騎侍郎、太子內舍人。煬帝即位,授內史舍人、儀同三司。尋以籓邸之舊,加開府。尋拜謁者大夫,從幸江都,以本官攝江都贊治,稱為幹理。虔威嘗在途見一遺囊,恐其主求失,因令左右負之而行。後數日,物主來認,悉以付之。淮南太守楊綝嘗與十餘人同來謁見,帝問虔威曰:「其首立者為誰?」虔威下殿就視而答曰:「淮南太守楊綝。」帝謂虔威曰:「卿為謁者大夫,而乃不識參見人,何也?」虔威對曰:「臣非不識楊綝,但慮不審,所以不敢輕對。石建數馬足,蓋慎之至也。」帝甚嘉之。其廉慎皆此類也。於時帝數
 巡幸,百姓疲敝,虔威因上封事以諫。帝不悅,自此見疏。未幾,卒官。有子爽,仕至蘭陵令。



 虔威弟虔雄,亦有才器。秦孝王俊為秦州總管,選為法曹參軍。王嘗親案囚徒,虔雄誤不持狀,口對百餘人,皆盡事情,同輩莫不嘆服。後歷壽春、陽城二縣令,俱有治績。



 榮毗兄建緒榮毗,字子諶,北平無終人也。父權,魏兵部尚書。毗少剛鯁有局量,涉獵群言,仕周,釋褐漢王記室,轉內史下士。開皇中,累遷殿內監。時以華陰多盜賊,妙選長吏,楊素薦毗為華州長史,世號為能。素之田宅,多在華陰,左右
 放縱,毗以法繩之,無所寬貸。毗因朝集,素謂之曰:「素之舉卿,適以自罰也。」毗答曰:「奉法一心者,但恐累公所舉。」素笑曰:「前者戲耳。卿之奉法,素之望也。」



 時晉王在揚州,每令人密覘京師消息。遣張衡於路次往往置馬坊,以畜牧為辭,實給私人也。州縣莫敢違,毗獨遏絕其事。上聞而嘉之,賚絹百匹,轉蒲州司馬。漢王諒之反也,河東豪傑以城應諒。刺史丘和覺,遁歸關中。長史渤海高義明謂毗曰:「河東要害,國之東門,若失之,則為難不細。城中雖復恟渙,非悉反也。但收桀黠者十餘人斬之,自當立定耳,」毗然之。義明馳馬追和,將與協計。至城西門,為
 反者所殺,毗亦被執。及諒平,拜治書侍御史,帝謂之曰:「今日之舉,馬坊之事也。無改汝心。」帝亦敬之。毗在朝侃然正色,為百僚所憚。後以母憂去職,歲餘,起令視事,尋卒官。贈鴻臚少卿。



 毗兄建緒,性甚亮直,兼有學業。仕周為載師下大夫、儀同三司。及平齊之始,留鎮鄴城,因著《齊紀》三十卷。建緒與高祖有舊,及為丞相,加位開府,拜息州剌史。將之官,時高祖陰有禪代之計,因謂建緒曰:「且躊躇,當共取富貴。」建緒自以周之大夫,因義形於色曰:「明公此旨,非僕所聞。」高祖不悅,建緒遂行。



 開皇初來朝,上謂之曰:「卿亦悔不?」建緒稽首曰:「臣位非徐廣,情類
 楊彪。」



 上笑曰:「朕雖不解書語,亦知卿此言不遜也。」歷始、洪二州刺史,俱有能名。



 陸知命陸知命,字仲通,吳郡富春人也。父敖,陳散騎常侍。知命性好學,通識大體,以貞介自持,釋褐陳始興王行參軍,後歷太學博士、南獄正。及陳滅,歸於家,會高智慧等作亂於江左,晉王廣鎮江都,以其三吳之望,召令諷諭反者。知命說下賊十七城,得其渠帥陳正緒、蕭思行等三百餘人,以功拜儀同三司,賜以田宅,復用其弟恪為汧陽令。知命以恪非百里才,上表陳讓,朝廷許之。時見天
 下一統,知命勸高祖都洛陽,因上《太平頌》以諷焉。文多不載。數年不得調,詣朝堂上表,請使高麗,曰:「臣聞聖人當扆,物色芻蕘,匹夫奔踶,或陳狂瞽。伏願暫輟旒纊,覽臣所謁。昔軒轅馭歷,既緩夙沙之誅,虞舜握圖,猶稽有苗之伐,陛下當百代之末,膺千載之期,四海廓清,三邊底定,唯高麗小豎,狼顧燕垂。王度含弘,每懷遵養者,良由惡殺好生,欲諭之以德也。臣請以一節,宣示皇風,使彼君臣面縛闕下。」書奏,天子異之。歲餘,授普寧鎮將。人或言其正直者,由是待詔於御史臺。



 煬帝嗣位,拜治書侍御史,侃然正色,為百僚所憚,帝甚敬之,後坐事免。歲
 餘,復職。時齊王暕頗驕縱,暱小人,知命奏劾之。暕竟得罪,百僚震慄。遼東之役,為東暆道受降使者,卒於師,時年六十七。贈御史大夫。



 房彥謙房彥謙,字孝沖,本清河人也,七世祖諶,仕燕太尉掾,隨慕容氏遷於齊,子孫因家焉。世為著姓。高祖法壽,魏青、冀二州刺史,壯武侯。曾祖伯祖,齊郡、平原二郡太守。祖翼,宋安太守,並世襲爵壯武侯。父熊,釋褐州主簿,行清河、廣川二郡守。彥謙早孤,不識父,為母兄之所鞠養。長兄彥詢,雅有清鑒,以彥謙天性穎悟,每奇之,親教讀書。
 年七歲,誦數萬言,為宗黨所異。十五,出後叔父子貞,事所繼母,有逾本生,子貞哀之,撫養甚厚。後丁所繼母憂,勺飲不入口者五日。事伯父樂陵太守豹,竭盡心力,每四時珍果,口弗先嘗。遇期功之戚,必蔬食終禮,宗從取則焉。其後受學於博士尹琳,手不釋卷,遂通涉五經。解屬文,工草隸,雅有詞辯,風概高人。年十八,屬廣寧王孝珩為齊州刺史,闢為主簿。時禁網疏闊,州郡之職,尤多縱弛,及彥謙在職,清簡守法,州境肅然,莫不敬憚。及周師入鄴,齊主東奔,以彥謙為齊州治中。彥謙痛本朝傾覆,將糾率忠義,潛謀匡輔。事不果而止。齊亡,歸於家。周
 帝遣柱國辛遵為齊州刺史,為賊帥輔帶劍所執。



 彥謙以書諭之,帶劍慚懼。送遵還州,諸賊並各歸首。及高祖受禪之後,遂優游鄉曲,誓無仕心。



 開皇七年,刺史韋藝固薦之,不得已而應命。吏部尚書盧愷一見重之,擢授承奉郎,俄遷監察御史。後屬陳平,奉詔安撫泉、括等十州,以銜命稱旨,賜物百段,米百石,衣一襲,奴婢七口。遷秦州總管錄事參軍。嘗因朝集,時左僕射高熲定考課,彥謙謂熲曰:「書稱三載考績,黜陟幽明,唐、虞以降,代有其法。黜陟合理,褒貶無虧,便是進必得賢,退皆不肖,如或舛謬,法乃虛設。比見諸州考校,執見不同,進退多少,
 參差不類。況復愛憎肆意,致乖平坦,清介孤直,未必高名,卑諂巧官,翻居上等,直為真偽混淆,是非瞀亂。宰貴既不精練,斟酌取舍,曾經驅使者,多以蒙識獲成,未歷臺省者,皆為不知被退。又四方懸遠,難可詳悉,唯量準人數,半破半成。徒計官員之少多,莫顧善惡之眾寡,欲求允當,其道無由。明公鑒達幽微,平心遇物,今所考校,必無阿枉,脫有前件數事,未審何以裁之?唯願遠布耳目,精加採訪,褒秋毫之善,貶纖介之惡,非直有光至治,亦足標獎賢能。」



 詞氣侃然,觀者屬目。熲為之動容,深見嗟賞。因歷問河西、隴右官人景行,彥謙對之如響,熲顧
 謂諸州總管、刺史曰:「與公言,不如獨與秦州考使語。」後數日,熲言於上,上弗能用。以秩滿,遷長葛令,甚有惠化,百姓號為慈父。仁壽中,上令持節使者巡行州縣,察長吏能不,以彥謙為天下第一,超授鄀州司馬。吏民號哭相謂曰:「房明府今去,吾屬何用生為!」其後百姓思之,立碑頌德。鄀州久無刺史,州務皆歸彥謙,名有異政。



 內史侍郎薛道衡,一代文宗,位望清顯,所與交結,皆海內名賢。重彥謙為人,深加友敬,及兼襄州總管,辭翰往來,交錯道路。煬帝嗣位,道衡轉牧番州,路經彥謙所,留連數日,屑涕而別。黃門侍郎張衡,亦與彥謙相善。於時帝營
 東都,窮極侈麗,天下失望。又漢王構逆,罹罪者多,彥謙見衡當途而不能匡救,以書諭之曰:竊聞賞者所以勸善,刑者所以懲惡,故疏賤之人,有善必賞,尊貴之戚,犯惡必刑,未有罰則避親,賞則遺賤者也。今諸州刺史,受委宰牧,善惡之間,上達本朝,懾憚憲章,不敢怠慢。國家祗承靈命,作民父母,刑賞曲直,升聞於天,夤畏照臨,亦宜謹肅。故文王云:「我其夙夜,畏天之威。」以此而論,雖州國有殊,高下懸邈,然憂民慎法,其理一也。至如並州畔逆,須有甄明。若楊諒實以詔命不通,慮宗社危逼,徵兵聚眾,非為干紀,則當原其本情,議其刑罰,上副聖主友
 于之意,下曉愚民疑惑之心;若審知內外無虞,嗣後纂統,而好亂樂禍,妄有覬覦,則管、蔡之誅,當在於諒,同惡相濟,無所逃罪,梟懸孥戮,國有常刑。其間乃有情非協同,力不自固,或被擁逼,淪陷兇威,遂使籍沒流移,恐為冤濫。恢恢天網,豈其然乎?罪疑從輕,斯義安在?昔叔向置鬻獄之死,晉國所嘉,釋之斷犯蹕之刑,漢文稱善。羊舌寧不愛弟,廷尉非茍違君,但以執法無私,不容輕重。且聖人大寶,是曰神器,茍非天命,不可妄得。故蚩尤、項籍之驍勇,伊尹、霍光之權勢,李老、孔丘之才智,呂望、孫武之兵術,吳、楚連磐石之據,產、祿承母後之基,不應歷
 運之兆,終無帝王之位。況乎蕞爾一隅,蜂扇蟻聚,楊諒之愚鄙,群小之兇慝,而欲憑陵畿甸,覬幸非望者哉!開闢以降,書契云及,帝皇之跡,可得而詳。自非積德累仁,豐功厚利,孰能道洽幽顯,義感靈祇!是以古之哲王,昧旦丕顯,履冰在念,御朽競懷。逮叔世驕荒,曾無戒懼,肆於民上,聘嗜奔欲,不可具載,請略陳之。



 襄者齊、陳二國,並居大位,自謂與天地合德,日月齊明,罔念憂虞,不恤刑政。近臣懷寵,稱善而隱惡,史官曲筆,掩瑕而錄美。是以民庶呼嗟,終閉塞於視聽,公卿虛譽,日敷陳於左右。法網嚴密,刑闢日多,徭役煩興,老幼疲苦。昔鄭有子產,
 齊有晏嬰,楚有叔敖,晉有士會。凡此小國,尚足名臣,齊、陳之疆,豈無良佐?但以執政壅蔽,懷私徇軀,忘國憂家,外同內忌。設有正直之士,才堪幹持,於己非宜,即加擯壓;倘遇諂佞之輩,行多穢匿,於我有益,遂蒙薦舉。以此求賢,何從而至!夫賢材者,非尚膂力,豈系文華,唯須正身負載,確乎不動。譬棟之處屋,如骨之在身,所謂棟梁骨鯁之材也。齊、陳不任骨鯁,信近讒諛,天高聽卑,監其淫僻,故總收神器,歸我大隋。向使二國祗敬上玄,惠恤鰥寡,委任方直,斥遠浮華,卑菲為心,惻隱為務,河朔強富,江湖險隔,各保其業,民不思亂,泰山之固,弗可動也。
 然而寢臥積薪,宴安鴆毒,遂使禾黍生廟,霧露沾衣,吊影撫心,何嗟及矣!故詩云:「殷之未喪師,克配上帝。宜鑒於殷,駿命不易。」萬機之事,何者不須熟慮哉!



 伏惟皇帝望雲就日,仁孝夙彰,錫社分珪,大成規矩。及總統淮海,盛德日新,當璧之符,遐邇僉屬。贊歷甫爾,寬仁已布,率土蒼生,翹足而喜。並州之亂,變起倉卒,職由楊諒詭惑,詿誤吏民,非有構怨本朝,棄德從賊者也。而有司將帥,稱其願反,非止誣陷良善,亦恐大點皇猷。足下宿當重寄,早預心膂,粵自籓邸,柱石見知。方當書名竹帛,傳芳萬古,稷、契、伊、呂,彼獨何人?既屬明時,須存謇諤,立當世
 之大誡,作將來之憲範。豈容曲順人主,以愛虧刑,又使脅從之徒,橫貽罪譴?忝蒙眷遇,輒寫微誠,野人愚瞽,不知忌諱。



 衡得書嘆息,而不敢奏聞。



 彥謙知王綱不振,遂去官隱居不仕,將結構蒙山之下,以求其志。會置司隸官,盛選天下知名之士。朝廷以彥謙公方宿著,時望所歸,徵授司隸刺史。彥謙亦慨然有澄清天下之志,凡所薦舉,皆人倫表式。其有彈射,當之者曾無怨言。司隸別駕劉灹,陵上侮下,訐以為直,刺史憚之,皆為之拜。唯彥謙執志不撓,亢禮長揖,有識嘉之。



 灹亦不敢為恨。大業九年,從駕渡遼,監扶餘道軍。其後隋政漸亂,朝廷靡然,
 莫不變節。彥謙直道守常,介然孤立,頗為執政者之所嫉,出為涇陽令。未幾,終於官,時年六十九。



 彥謙居家,每子侄定省,常為講說督勉之,亹癖不倦。家有舊業,資產素殷,又前後居官,所得俸祿,皆以周恤親友,家無餘財,車服器用,務存素儉。自少及長,一言一行,未嘗涉私,雖致屢空,怡然自得。嘗從容獨笑,顧謂其子玄齡曰:「人皆因祿富,我獨以官貧。所遺子孫,在於清白耳。」所有文筆,恢廓閑雅,有古人之深致。又善草隸,人有得其尺牘者,皆寶玩之。太原王邵,北海高構,蓚縣李綱,河東柳彧、薛孺,皆一時知名雅澹之士,彥謙並與為友。雖冠蓋成列,
 而門無雜賓。體資文雅,深達政務,有識者咸以遠大許之。初,開皇中,平陳之後,天下一統,論者咸云將致太平。彥謙私謂所親趙郡李少通曰:「主上性多忌克,不納諫爭。太子卑弱,諸王擅威,在朝唯行苛酷之政,未施弘大之體。天下雖安,方憂危亂。」少通初謂不然,及仁壽、大業之際,其言皆驗。大唐馭宇,追贈徐州都督、臨淄縣公。謚曰定。



 史臣曰:大廈雲構,非一木之枝;帝王之功,非一士之略。長短殊用,大小異宜,咨咨棁棟梁,莫可棄也。李諤等或文能遵義,或才足乾時,識用顯於當年,故事留於臺閣。參
 之有隋多士,取其開物成務,皆廊廟之榱桷,亦北辰之眾星也。



\end{pinyinscope}