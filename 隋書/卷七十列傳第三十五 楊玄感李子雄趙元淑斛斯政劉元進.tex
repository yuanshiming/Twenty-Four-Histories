\article{卷七十列傳第三十五 楊玄感李子雄趙元淑斛斯政劉元進}

\begin{pinyinscope}

 楊玄感李子雄趙元淑斛斯政劉元進



 楊玄感,司徒素之子也。體貌雄偉,美須髯。少時晚成,人多謂之癡,其父每謂所親曰:「此兒不癡也。」及長,好讀書,便騎射。以父軍功,位至柱國,與其父俱為第二品,朝會則齊列。其後高祖命玄感降一等,玄感拜謝曰:「不意陛下寵臣之甚,許以公廷獲展私敬。」初拜郢州刺史,到官,潛布耳目,察長吏能不。其有善政及臟污者,纖介必知
 之,往往發其事,莫敢欺隱。吏民敬服,皆稱其能。後轉宋州刺史,父憂去職。歲餘,起拜鴻臚卿,襲爵楚國公,遷禮部尚書。性雖驕倨,而愛重文學,四海知名之士多趨其門。自以累世尊顯,有盛名於天下,在朝文武多是父之將吏,復見朝綱漸紊,帝又猜忌日甚,內不自安,遂與諸弟潛謀廢帝,立秦王浩。及從征吐谷渾,還至大斗拔谷,時從官狼狽,玄感欲襲擊行宮。其叔慎謂玄感曰:「士心尚一,國未有釁,不可圖也。」玄感乃止。時帝好征伐,玄感欲立威名,陰求將領。謂兵部尚書段文振曰:「玄感世荷國恩,寵逾涯分,自非立效邊裔,何以塞責!若方隅有風
 塵之警,庶得執鞭行陣,少展絲發之功。明公兵革是司,敢布心腹。」文振因言於帝,帝嘉之,顧謂群臣曰:「將門必有將,相門必有相,故不虛也。」於是賚物千段,禮遇益隆,頗預朝政。



 帝征遼東,命玄感於黎陽督運。於時百姓苦役,天下思亂,玄感遂與武賁郎將王仲伯、汲郡贊治趙懷義等謀議,欲令帝所軍眾饑餒,每為逗留,不時進發。帝遲之,遣使者逼促,玄感揚言曰:「水路多盜賊,不可前後而發。」其弟武賁郎將玄縱、鷹揚郎將萬碩並從幸遼東,玄感潛遣人召之。時將軍來護兒以舟師自東萊將入海,趣平壤城,軍未發。玄感無以動眾,乃遣家奴偽為
 使者,從東方來,謬稱護兒失軍期而反。玄感遂入黎陽縣,閉城大索男夫。於是取帆布為牟甲,署官屬,皆準開皇之舊。移書傍郡,以討護兒為名,各令發兵,會於倉所。以東光縣尉元務本為黎州刺史,趙懷義為衛州刺史,河內郡主簿唐禕為懷州刺史。有眾且一萬,將襲洛陽。唐幃至河內,馳往東都告之。越王侗、民部尚書樊子蓋等大懼,勒兵備御。修武縣民相率守臨清關,玄感不得濟,遂於汲郡南渡河,從亂者如市。數日,屯兵上春門,眾至十餘萬。子蓋令河南贊治裴弘策拒之,弘策戰敗。瀍、洛父老競致牛酒。



 玄感屯兵尚書省,每誓眾曰:「我身為
 上柱國,家累巨萬金,至於富貴,無所求也。



 今者不顧破家滅族者,但為天下解倒懸之急,救黎元之命耳。」眾皆悅,詣轅門請自效者,日有數千。與樊子蓋書曰:夫建忠立義,事有多途,見機而作,蓋非一揆。昔伊尹放太甲於桐宮,霍光廢劉賀於昌邑,此並公度內,不能一二披陳。高祖文皇帝誕膺天命,造茲區宇,在〔璣以齊七政,握金鏡以馭六龍,無為而至化流,垂拱而天下治。今上纂承寶歷,宜固洪基,乃自絕於天,殄民敗德。頻年肆書,盜賊於是滋多,所在修治,民力為之凋盡。荒淫酒色,子女必被其侵,耽玩鷹犬,禽獸皆離其毒。朋黨相扇,貨賄公
 行,納邪佞之言,杜正直之口。加以轉輸不息,遙役無期,士卒填溝壑,骸骨蔽原野。黃河之北,則千里無煙,江淮之間,則鞠為茂草。玄感世荷國恩,位居上將,先公奉遺詔曰:「好子孫為我輔弼之,惡子孫為我屏黜之。」所以上稟先旨,下順民心,廢此淫昏,更立明哲。四海同心,九州響應,士卒用命,如赴私讎,民庶相趨,義形公道。天意人事,較然可知。公獨守孤城,勢何支久!願以黔黎在念,社稷為心,勿拘小禮,自貽伊戚。誰謂國家,一旦至此,執筆潸泫,言無所具。



 遂進逼都城。



 刑部尚書衛玄,率眾數萬,自關中來援東都。以步騎二萬渡瀍、澗挑戰,玄感偽北。
 玄逐之,伏兵發,前軍盡沒。後數日,玄復與玄感戰,兵始合,玄感詐令人大呼曰:「官軍已得玄感矣。」玄軍稍怠,玄感與數千騎乘之,於是大潰,擁八千人而去。玄感驍勇多力,每戰親運長矛,身先士卒,喑嗚叱吒,所當者莫不震懾。



 論者方之項羽。又善撫馭,士樂致死,由是戰無不捷。玄軍日蹙,糧又盡,乃悉眾決戰,陣於北邙,一日之間,戰十餘合。玄感弟玄挺中流矢而斃,玄感稍卻。樊子蓋復遣兵攻尚書省,又殺數百人。帝遣武賁郎將陳棱攻元務本於黎陽,武衛將軍屈突通屯河陽,左翊大將軍宇文述發兵繼進,右驍衛大將軍來護兒復來赴援。
 玄感請計於前民部尚書李子雄,子雄曰:「屈突通曉習兵事,若一渡河,則勝負難決,不如分兵拒之。通不能濟,則樊、衛失援。」玄感然之,將拒通。子蓋知其謀,數擊其營,玄感不果進。通遂濟河,軍於破陵。玄感為兩軍。西抗衛玄,東拒屈突通。



 子蓋復出兵,於是大戰,玄感軍頻北。復請計於子雄,子雄曰:「東都援軍益至,我師屢敗,不可久留。不如直入關中,開永豐倉以賑貧乏,三輔可指麾而定。據有府庫,東面而爭天下,此亦霸王之業。」會華陰諸楊請為鄉導,玄感遂釋洛陽,西圖關中,宣言曰:「我已破東都,取關西矣。」宇文述等諸軍躡之。至弘農宮,父老遮
 說玄感曰:「宮城空虛,又多積粟,攻之易下。進可絕敵人之食,退可割宜陽之地。」玄感以為然,留攻之。三日城不下,追兵遂至。玄感西至閿鄉,上盤豆,布陣亙五十里,與官軍且戰且行,一日三敗。復陣於董杜原,諸軍擊之,玄感大敗,獨與十餘騎竄林木間,將奔上洛。追騎至,玄感叱之,皆懼而返走。至葭蘆戍,玄感窘迫,獨與弟積善步行。自知不免,謂積善曰:「事敗矣。我不能受人戮辱,汝可殺我。」積善抽刀斫殺之,因自刺,不死,為追兵所執,與玄感首俱送行在所,磔其尸於東都市三日,復臠而焚之。餘黨悉平。其弟玄獎為義陽太守,將歸玄感,為郡丞周
 〔玉所殺。玄縱弟萬碩,自帝所逃歸,至高陽,止傳舍,監事許華與郡兵執之,斬於涿郡。萬碩弟民行,官至朝請大夫,斬於長安。並具梟磔。公卿請改玄感姓為梟氏,詔可之。



 初,玄感圍東都也,梁郡人韓相國舉兵應之,玄感以為河南道元帥。旬月間,眾十餘萬,攻剽郡縣。至於襄城,遇玄感敗,兵漸潰散,為吏所執,傳首東都。



 李子雄,渤海蓚人也。祖伯賁,魏諫議大夫。父桃枝,東平太守。與鄉人高仲密同歸於周,官至冀州刺史。子雄少慷慨有壯志,弱冠從周武帝平齊,以功授帥都督。高祖作相,從韋孝寬破尉迥於相州,拜上開府,賜爵建昌縣
 公。高祖受禪,為驃騎將軍。伐陳之役,以功進位大將軍,歷郴、江二州刺史,並有能名。仁壽中,坐事免。漢王諒之作亂也,煬帝將發幽州兵以討之。時竇抗為幽州總管,帝恐其有二心,問可任者於楊素。素進子雄,授大將軍,拜廉州刺史,馳至幽州,止傳舍,召募得千餘人。抗恃素貴,不時相見。子雄遣人諭之。後二日,抗從鐵騎二千,來詣子雄所。子雄伏甲,請與相見,因擒抗。遂發幽州兵步騎三萬,自井陘以討諒。



 時諒遣大將軍劉建略地燕、趙,正攻井陘,相遇於抱犢山下,力戰,破之。遷幽州總管,尋徵拜民部尚書。



 子雄明辯有器幹,帝甚任之。新羅嘗遣
 使朝貢,子雄至朝堂與語,因問其冠制所由。其使者曰:「皮弁遺象。安有大國君子而不識皮弁也!」子雄因曰:「中國無禮,求諸四夷。」使者曰:「自至已來,此言之外,未見無禮。」憲司以子雄失詞,奏劾其事,竟坐免。俄而復職,從幸江都。帝以仗衛不整,顧子雄部伍之。子雄立指麾,六軍肅然。帝大悅曰:「公真武侯才也。」尋轉右武侯大將軍,後坐事除名。遼東之役,帝令從軍自效,因從來護兒自東平將指滄海。會楊玄感反於黎陽,帝疑之,詔鎖子雄送行在所。子雄殺使者,亡歸玄感。玄感每請計於子雄,語在《玄感傳》。及玄感敗,伏誅,籍沒其家。



 博陵趙元淑,父世模,初事高寶寧,後以眾歸周,授上開府,寓居京兆之雲陽。



 高祖踐阼,恆典宿衛。後從晉王伐陳,先鋒遇賊,力戰而死。朝廷以其身死王事,以元淑襲父本官,賜物二千段。元淑性疏誕,不治產業,家徒壁立。後數歲,授驃騎將軍。將之官,無以自給。時長安富人宗連,家累千金,仕周為三原令。有季女,慧而有色,連獨奇之,每求賢夫,聞元淑如是,請與相見。連有風儀,美談笑,元淑亦異之。及至其家,服玩居處擬於將相。酒酣,奏女樂,元淑所未見也。元淑辭出,連曰:「公子有暇,可復來也。」後數日,復造之,宴樂更侈。如此者再三,因謂元淑曰:「知公
 子素貧,老夫當相濟。」因問元淑所須,盡買與之。臨別,元淑再拜致謝,連復拜曰:「鄙人竊不自量,敬慕公子。今有一女,願為箕帚妾,公子意何如?」元淑感愧,遂娉為妻。連復送奴婢二十口、良馬十餘匹,加以縑帛錦綺及金寶珍玩。元淑遂為富人。及煬帝嗣位,漢王諒作亂,元淑從楊素擊平之。以功進位柱國,拜德州刺史,尋轉潁川太守,並有威惠。因入朝,會司農不時納諸郡租穀,元淑奏之。帝謂元淑曰:「如卿意者,幾日當了?」元淑曰:「如臣意不過十日。」帝即日拜元淑為司農卿,納天下租,如言而了。帝悅焉。禮部尚書楊玄感潛有異志,以元淑可與共亂,
 遂與結交,多遺金寶。遼東之役,領將軍典宿衛,加授光祿大夫,封葛公。明年,帝復征高麗,以元淑鎮臨渝。及玄感作亂,其弟玄縱自帝所逃歸,路經臨渝。元淑出其小妻魏氏見玄縱,對宴極歡,因與通謀,並授玄縱賂遺。及玄感敗,人有告其事者,帝以屬吏。元淑言與玄感結婚,所得金寶則為財娉,實無他故。魏氏復言初不受金。帝親臨問,卒無異辭。帝大怒,謂侍臣曰:「此則反狀,何勞重問!」元淑及魏氏俱斬於涿郡,籍沒其家。



 河南斛斯政,祖椿,魏太保、尚書令、常山文宣王。父恢,散騎常侍、新蔡郡公。政明悟有器幹,初為親衛,後以軍功
 授儀同,甚為楊素所禮。大業中,為尚書兵曹郎。政有風神,每奏事,未嘗不稱旨。煬帝悅之,漸見委信。楊玄感兄弟俱與之交。遼東之役,兵部尚書段文振卒,侍郎明雅復以罪廢,帝彌屬意。尋遷兵部侍郎。於時外事四夷,軍國多務,政處斷辯速,稱為幹理。玄感之反也,政與通謀。



 及玄縱等亡歸,亦政之計也。帝在遼東,將班師,窮治玄縱黨與。內不自安,遂亡奔高麗。明年,帝復東征,高麗請降,求執送政。帝許之,遂鎖政而還。至京師,以政告廟,左翊衛大將軍字文述奏曰:「斛斯政之罪,天地所不容,人神所同忿。



 若同常刑,賊臣逆子何以懲肅?請變常法。」帝
 許之。於是將政出金光門,縛政於柱,公卿百僚並親擊射,臠割其肉,多有啖者。啖後烹煮,收其餘骨,焚而揚之。



 餘杭劉元進,少好任俠,為州里所宗。兩手各長尺餘,臂垂過膝。煬帝與遼東之役,百姓騷動,元進自以相表非常,陰有異志,遂聚眾,合亡命。會帝復征遼東,徵兵吳會,士卒皆相謂曰:「去年吾輩父兄從帝征者,當全盛之時,猶死亡太半,骸骨不歸;今天下已罷敝,是行也,吾屬其無遺類矣。」於是多有亡散,郡縣捕之急。既而楊玄感起於黎陽,元進知天下思亂,於是舉兵應之。三吳苦役者莫不響至,旬月眾至數萬。將渡江,而玄感敗。吳郡硃燮、
 晉陵管崇亦舉兵,有眾七萬,共迎元進,奉以為主。據吳郡,稱天子,燮、崇俱為僕射,署置百官。毗陵、東陽、會稽、建安豪傑多執長吏以應之。帝令將軍吐萬緒、光祿大夫魚俱羅率兵討焉。元進西屯茅浦,以抗官軍,頻戰互有勝負。元進退保曲阿,與硃燮、管崇合軍,眾至十萬。緒進軍逼之。相持百餘日,為緒所敗,保於黃山。緒復破之,燮戰死,元進引趣建安,休兵養士。二將亦以師老,頓軍自守。俄而二將俱得罪,帝令江都郡丞王世充發淮南兵擊之,有大流星墜於江都,未及地而南逝,磨拂竹木皆有聲,至吳郡而落於地。元進惡之,令掘地,入二丈,得一石,
 徑丈餘。後數日,失石所在。世充既渡江,元進將兵拒戰,殺千餘人,世充窘急,退保延陵柵。元進遣兵,人各持茅,因風縱火。世充大懼,將棄營而遁。遇反風,火轉,元進之眾懼燒而退。世充簡銳卒掩擊,大破之,殺傷太半,自是頻戰輒敗。元進謂管崇曰:「事急矣,當以死決之。」於是出挑戰,俱為世充所殺。其眾悉降,世充坑之於黃亭澗,死者三萬人。其餘黨往往保險為盜。其後董道沖、沈法興、李子通等乘此而起,戰爭不息,逮於隋亡。



 李密裴仁基李密,字法主,真鄉公衍之從孫也。祖耀,周邢國公。父寬,
 驍勇善戰,幹略過人,自周及隋,數經將領,至柱國、蒲山郡公,號為名將。密多籌算,才兼文武,志氣雄遠,常以濟物為己任。開皇中,襲父爵蒲山公,乃散家產,周贍親故,養客禮賢,無所愛吝。與楊玄感為刎頸之交。後更折節,下帷耽學,尤好兵書,誦皆在口。師事國子助教包愷,受《史記》、《漢書》,勵精忘倦,愷門徒皆出其下。大業初,授親衛大都督,非其所好,稱疾而歸。



 及楊玄感在黎陽,有逆謀,陰遣家僮至京師召密,令與弟玄挺等同赴黎陽。玄感舉兵而密至,玄感大喜,以為謀主。玄感謀計於密,密曰:「愚有三計,惟公所擇。今天子出征,遠在遼外,地去幽州,
 懸隔千里。南有巨海之限,北有胡戎之患,中間一道,理極艱危。今公擁兵,出其不意,長驅入薊,直扼其喉。前有高麗,退無歸路,不過旬月,齎糧必盡。舉麾一召,其眾自降,不戰而擒,此計之上也。又關中四塞,天府之國,有衛文升,不足為意。今宜率眾,經城勿攻,輕齎鼓行,務早西入。天子雖還,失其襟帶,據險臨之,故當必克,萬全之勢,此計之中也。若隨近逐便,先向東都,唐禕告之,理當固守。引兵攻戰,必延歲月,勝負殊未可知,此計之下也。」玄感曰:「不然。公之下計,乃上策矣。今百官家口並在東都,若不取之,安能動物?且經城不拔,何以示威?」密計遂不
 行。玄感既至東都,皆捷,自謂天下響應,功在朝夕。及獲韋福嗣,又委以腹心,是以軍旅之事,不專歸密。



 福嗣既非同謀,因戰被執,每設籌畫,皆持兩端。後使作檄文,福嗣固辭不肯。密揣知其情,因謂玄感曰:「福嗣元非同盟,實懷觀望。明公初起大事,而奸人在側,聽其是非,必為所誤矣。請斬謝眾,方可安輯。」玄感曰:「何至於此!」密知言之不用,退謂所親曰:「楚公好反而不欲勝,如何?吾屬今為虜矣!」後玄感將西入,福嗣竟亡歸東都。



 時李子雄勸玄感速稱尊號,玄感以問於密。密曰:「昔陳勝自欲稱王,張耳諫而被外,魏武將求九錫,荀彧止而見疏。今者密
 欲正言,還恐追蹤二子,阿諛順意,又非密之本圖。何者?兵起已來,雖復頻捷,至於郡縣,未有從者。東都守禦尚強,天下救兵益至,公當身先士眾,早定關中。乃欲急自尊崇,何示不廣也!」玄感笑而止。及宇文述、來護兒等軍且至,玄感謂密曰:「計將安出?」密曰:「元弘嗣統強兵於隴右,今可揚言其反,遣使迎公,因此入關,可得紿眾。」玄感遂以密謀號令其眾,因引西入。至陜縣,欲圍弘農宮,密諫之曰:「公今詐眾入西,軍事在速,況乃追兵將至,安可稽留!若前不得據關,退無所守,大眾一散,何以自全?」



 玄感不從,遂圍之,三日攻不能拔,方引而西。至於閿鄉,追
 兵遂及。



 玄感敗,密間行入關,與玄感從叔詢相隨,匿於馮翊詢妻之舍。尋為鄰人所告,遂捕獲,囚於京兆獄。是時煬帝在高陽,與其黨俱送帝所。在途謂其徒曰:「吾等之命,同於朝露,若至高陽,必為俎醢。今道中猶可為計,安得行就鼎鑊,不規逃避也?」眾咸然之。其徒多有金,密令出示使者曰:「吾等死日,此金並留付公,幸用相瘞。其餘即皆報德。」使者利其金,遂相然許。及出關外,防禁漸馳,密請通市酒食,每宴飲喧嘩竟夕,使者不以為意。行次邯鄲,夜宿村中,密等七人皆穿墻而遁,與王仲伯亡抵平原賊帥郝孝德。孝德不甚禮之,備遭饑饉,至削樹
 皮而食。



 仲伯潛歸天水,密詣淮陽,舍於村中,變姓名稱劉智遠,聚徒教授。經數月,密鬱鬱不得志,為五言詩曰:「金鳳蕩初節,玉露凋晚林。此夕窮途士,空軫鬱陶心。



 眺聽良多感,慷慨獨沾襟。沾襟何所為?悵然懷古意。秦俗猶未平,漢道將何冀!



 樊噲市井徒,蕭何刀筆吏。一朝時運合,萬古傳名器。寄言世上雄,虛生真可愧。」



 詩成而泣下數行。時人有怪之者,以告太守趙他。縣捕之,密乃亡去,抵其妹夫雍丘令丘君明。後君明從子懷義以告,帝令捕密,密得遁去,君明竟坐死。



 會東郡賊帥翟讓聚黨萬餘人,密歸之,其中有知密是玄感亡將,潛勸讓害之。



 密大懼,乃因王伯當以策干讓。讓遣說諸小賊,所至輒降下,讓始敬焉,召與計事。



 密謂讓曰:「今兵眾既多,糧無所出,若曠日持久,則人馬困敝,大敵一臨,死亡無日。未若直趣滎陽,休兵館穀,待士馬肥充,然可與人爭利。」讓從之,於是破金堤關,掠滎陽諸縣,城堡多下之。滎陽太守郇王慶及通守張須陀以兵討讓。讓數為須陀所敗,聞其來,大懼,將遠避之。密曰:「須陀勇而無謀,兵又驟勝,既驕且狠,可一戰而擒。公但列陣以待,保為公破之。」讓不得已,勒兵將戰,密分兵千餘人於林木間設伏。讓與戰不利,軍稍卻,密發伏自後掩之,須陀眾潰。與讓合擊,
 大破之,遂斬須陀於陣。讓於是令密建牙,別統所部。密復說讓曰:「昏主蒙塵,播蕩吳越,蝟毛競起,海內饑荒。明公以英桀之才,而統驍雄之旅,宜當廓清天下,誅剪群兇,豈可求食草間,常為小盜而已!今東都士庶,中外離心,留守諸官,政令不一,明公親率大眾,直掩興洛倉,發粟以賑窮乏,遠近孰不歸附!百萬之眾,一朝可集,先發制人,此機不可失也。」讓曰:「僕起隴畝之間,望不至此。



 必如所圖,請君先發,僕領諸軍,便為後殿。得倉之日,當別議之。」密與讓領精兵七千人,以大業十三年春,出陽城,北逾方山,自羅口襲興洛倉,破之。開倉恣民所取,老弱
 負繦,道路不絕。



 越王侗武賁郎將劉長恭率步騎二萬五千討密,密一戰破之,長恭僅以身免。讓於是推密為主。密城洛口周回四十里以居之。房彥藻說下豫州,東都大懼。讓上密號為魏公。密初辭不受,諸將等固請,乃從之。設壇場,即位,稱元年,置官屬,以房彥藻為左長史,邴元真右長史,楊德方左司馬,鄭德韜右司馬。拜讓司徒,封東郡公。其將帥封拜各有差。長白山賊孟讓掠東都,燒豐都市而歸。密攻下鞏縣,獲縣長柴孝和,拜為護軍。武賁郎將裴仁基以武牢歸密,因遣仁基與孟讓率兵二萬餘人襲回洛倉,破之,燒天津橋,遂縱兵大掠。東
 都出兵乘之,仁基等大敗,僅以身免。密復親率兵三萬逼東都,將軍段達、武賁郎將高毗、劉長恭等出兵七萬拒之,戰於故都,官軍敗走,密復下回洛倉而據之。俄而德韜、德方俱死,復以鄭頲為左司馬,鄭虔象為右司馬。柴孝和說密曰:「秦地阻山帶河,西楚背之而亡,漢高都之而霸。如愚意者,令仁基守回洛,翟讓守洛口,明公親簡精銳,西襲長安,百姓孰不郊迎,必當有征無戰。既克京邑,業固兵強,方更長驅崤、函,掃蕩京、洛,傳檄指捴,天下可定。但今英雄競起,實恐他人我先,一朝失之,噬臍何及!」密曰:「君之所圖,僕亦思之久矣,誠為上策。但昏主
 尚在,從兵猶眾,我之所部,並山東人,既見未下洛陽,何肯相隨西入!諸將出於群盜,留之各競雌雄。若然者,殆將敗矣。」孝和曰:「誠如公言,非所及也。大軍既未可西出,請間行觀隙。」



 密從之。孝和與數十騎至陜縣,山賊歸之者萬餘人。密時兵鋒甚銳,每入苑,與官軍連戰。會密為流矢所中,臥於營內,後數日,東都出兵擊之,密眾大潰,棄回洛倉,歸洛口。孝和之眾聞密退,各分散而去。孝和輕騎歸密。帝遣王世充率江淮勁卒五萬來討密,密逆拒之,戰不利。柴孝和溺死於洛水,密甚傷之。世充營於洛西,與密相拒百餘日。武陽郡丞元寶藏、黎陽賊帥李
 文相、洹水賊帥張升、清河賊帥趙君德、平原賊帥郝孝德並歸於密,共襲破黎陽倉,據之。周法明舉江、黃之地以附密,齊郡賊帥徐圓朗、任城大俠徐師仁、淮陽太守趙他等前後款附,以千百數。



 翟讓所部王儒信勸讓為大塚宰,總統眾務,以奪密權。讓兄寬復謂讓曰:「天子止可自作,安得與人?汝若不能作,我當為之。」密聞其言,有圖讓之計。會世充列陣而至,讓出拒之,為世充所擊退者數百步。密與單雄信等率精銳赴之,世充敗走。讓欲乘勝進破其營,會日暮,密固止之。明日,讓與數百人至密所,欲為宴樂。密具饌以待之,其所將左右,各分令就
 食。諸門並設備,讓不之覺也。密引讓入坐,有好弓,出示讓,遂令讓射。讓引滿將發,密遣壯士蔡建自後斬之,殞於床下。遂殺其兄寬及王儒信,並其從者亦有死焉。讓所部將徐世勣,為亂兵所斫中,重創,密遽止之,僅而得免。單雄信等皆叩頭求哀,密並釋而慰諭之。於是率左右數百人詣讓本營。王伯當、邴元真、單雄信等入營,告以殺讓之意,眾無敢動者。



 乃令徐世勣、單雄信、王伯當分統其眾。



 未幾,世充夜襲倉城,密逆拒破之,斬武賁郎將費青奴。世充復移營洛北,南對鞏縣,其後遂於洛水造浮橋,悉眾以擊密。密與千騎拒之,不利而退。世充因
 薄其城下,密簡銳卒數百人,分為三隊出擊之。官軍稍卻,自相陷溺,死者數萬人,武賁郎將楊威、王辯、霍世舉、劉長恭、梁德重、董智通等諸將率皆沒於陣。世充僅而獲免,不敢還東都,遂走河陽。其夜雨雪尺餘,眾隨之者,死亡殆盡。密於是修金墉故城居之,眾三十餘萬。復來攻上春門,留守韋津出拒戰,密擊敗之,執津於陣。其黨勸密即尊號,密不許。及義師圍東都,密出軍爭之,交綏而退。



 俄而宇文化及殺逆,率眾自江都北指黎陽,兵十餘萬。密乃自率步騎二萬拒之。



 會越王侗稱尊號,遣使者授密太尉、尚書令、東南道大行臺、行軍元帥、魏國公,
 令先平化及,然後入朝輔政。密遣使報謝焉。化及與密相遇,密知其軍少食,利在急戰,故不與交鋒,又遏其歸路,使不得西。密遣徐世勣守倉城,化及攻之,不能下。密與化及隔水而語,密數之曰:「卿本匈奴皁隸破野頭耳,父兄子弟並受隋室厚恩,富貴累世,至妻公主,光榮隆顯,舉朝莫二。荷國士之遇者,當須國士報之,豈容主上失德,不能死諫,反因眾叛,躬行殺虐,誅及子孫,傍立支庶,擅自尊崇,欲規篡奪,污辱妃後,枉害無辜?不追諸葛瞻之忠誠,乃為霍禹之惡逆。天地所不容,人神所莫祐。擁逼良善,將欲何之!今若速來歸我,尚可得全後嗣。」化
 及默然,俯視良久,乃嗔目大言曰:「共你論相殺事,何須作書語邪?」密謂從者曰:「化及庸懦如此,忽欲圖為帝王,斯乃趙高、聖公之流,吾當折杖驅之耳。」化及盛修攻具,以逼黎陽倉城,密領輕騎五百馳赴之。倉城兵又出相應,焚其攻具,經夜火不滅。密知化及糧且盡,因偽與和,以敝其眾。化及不之悟,大喜,恣其兵食,冀密饋之。會密下有人獲罪,亡投化及,具言密情,化及大怒。其食又盡,乃渡永濟渠,與密戰於童山之下,自辰達酉。密為流矢所中,頓於汲縣。化及掠汲郡,北趣魏縣,其將陳智略、張童仁等所部兵歸於密者,前後相繼。初,化及以輜重留
 於東郡,遣其所署刑部尚書王軌守之。至是,軌舉郡降密,以軌為滑州總管。密引兵而西,遣記室參軍李儉朝於東都,執殺煬帝人於弘達以獻越王侗。侗以儉為司農少卿,使之反命,召密入朝。密至溫縣,聞世充已殺元文都、盧楚等,乃歸金墉。



 世充既得擅權,乃厚賜將士,繕治器械,人心漸銳。然密兵少衣,世充乏食,乃請交易。密初難之,邴元真等各求私利,遞來勸密,密遂許焉。初,東都絕糧,人歸密者,日有數百。至此,得食,而降人益少,密方悔而止。密雖據倉,無府庫,兵數戰不獲賞,又厚撫初附之兵,於是眾心漸怨。時遣邴元真守興洛倉。元真起
 自微賤,性又貪鄙,宇文溫疾之,每謂密曰:「不殺元真,公難未已。」密不答,而元真知之,陰謀叛密。揚慶聞而告密,密固疑焉。會世充悉眾來決戰,密留王伯當守金墉,自引精兵就偃師,北阻邙山以待之。世充軍至,令數百騎渡御河,密遣裴行儼率眾逆之。會日暮,暫交而退,行儼、孫長樂、程金等驍將十數人皆遇重創,密甚惡之。世充夜潛濟師,詰朝而陣,密方覺之,狼狽出戰,於是敗績,與萬餘人馳向洛口。世充夜圍偃師,守將鄭頲為其部下所翻,以城降世充。密將入洛口倉城,元真已遣人潛引世充矣。密陰知之而不發其事,因與眾謀,待世充之
 兵半濟洛水,然後擊之。及世充軍至,密候騎不時覺,比將出戰,世充軍悉已濟矣。密自度不能支,引騎而遁。元真竟以城降於世充。



 密眾漸離,將如黎陽。人或謂密曰:「殺翟讓之際,徐世勣幾至於死。今創猶未復,其心安可保乎?」密乃止。時王伯當棄金墉,保河陽,密以輕騎自武牢渡河以歸之,謂伯當曰:「兵敗矣!久苦諸君,我今自刎,請以謝眾。」眾皆泣,莫能仰視。密復曰:「諸君幸不相棄,當共歸關中。密身雖愧無功,諸君必保富貴。」



 其府掾柳燮對曰:「昔盆子歸漢,尚食均輸,明公與長安宗族有疇昔之遇,雖不陪起義,然而阻東都,斷隋歸路,使唐國不戰
 而據京師,此亦公之功也。」眾咸曰:「然。」密遂歸大唐,封邢國公,拜光祿卿。



 河東裴仁基,字德本。祖伯鳳,周汾州刺史。父定,上儀同。仁基少驍武,便弓馬。開皇初,為親衛。平陳之役,先登陷陣,拜儀同,賜物千段。以本官領漢王諒府親信。煬帝嗣位,諒舉兵作亂,仁基苦諫。諒大怒,囚之於獄。及諒敗,帝嘉之,超拜護軍。數歲,改授武賁郎將,從將軍李景討叛蠻向思多於黔安,以功進位銀青光祿大夫,賜奴婢百口,絹五百匹。擊吐谷渾於張掖,破之,加授金紫光祿大夫。斬獲寇掠靺鞨,拜左光祿大夫。從征高麗,進位光祿
 大夫。



 帝幸江都,李密據洛口,令仁基為河南道討捕大使,據武牢以拒密。及滎陽通守張須陀為密所殺,仁基悉收其眾,每與密戰,多所斬獲。時隋大亂,有功者不錄。



 仁基見強寇在前,士卒勞敝,所得軍資,即用分賞。監軍御史蕭懷靜每抑止之,眾咸怨怒。懷靜又陰持仁基長短,欲有所奏劾。仁基懼,遂殺懷靜,以其眾歸密。密以為河東郡公。其子行儼,驍勇善戰,密復以為絳郡公,甚相委暱。王世充以東都食盡,悉眾詣偃師,與密決戰。密問計於諸將,仁基對曰:「世充盡銳而至,洛下必虛,可分兵守其要路,令不得東。簡精兵三萬,傍河西出,以逼東都。
 世充卻還,我且按甲,世充重出,我又逼之。如此則此有餘力,彼勞奔命,兵法所謂『彼出我歸,彼歸我出,數戰以疲之,多方以誤之』者也。」密曰:「公知其一,不知其二。



 東都兵馬有三不可當:器械精,一也;決計而來,二也;食盡求斷,三也。我按甲蓄力,以觀其敝,彼求斷不得,欲走無路,不過十日,世充之首可懸於麾下。」單雄信等諸將輕世充,皆請戰,仁基苦爭不得。密難違諸將之言,戰遂大敗,仁基為世充所虜。世充以其父子並驍銳,深禮之,以兄女妻行儼。及僭尊號,署仁基為禮部尚書,行儼為左輔大將軍。行儼每有攻戰,所當皆披靡,號為「萬人敵」。世充
 憚其威名,頗加猜防。仁基知其意,不自安,遂與世充所署尚書左丞宇文儒童、尚食直長陳謙、秘書丞崔德本等謀反,令陳謙於上食之際,持匕首以劫世充,行儼以兵應於階下,指麾事定,然後出越王侗以輔之。事臨發,將軍張童仁知其謀而告之,俱為世充所殺。



 史臣曰:古先帝王之興也,非夫至德深仁格於天地,有豐功博利,弘濟艱難,不然,則其道無由矣。自周邦不競,隋運將隆,武元、高祖並著大功於王室,平南國,摧東夏,總百揆,定三方,然後變謳歌,遷寶鼎。於時匈奴驕倨,勾吳不朝,既爭長於黃池,亦飲馬於清渭。高祖內綏外御,
 日不暇給,委心膂于俊傑,寄折沖於爪牙,文武爭馳,群策畢舉。服猾夏之虜,掃黃旗之寇,峻五岳以作鎮,環四海以為池,厚澤被於域中,餘威震於殊俗。煬帝蒙故業,踐丕基,阻伊、洛而固崤、函,跨兩都而總萬國。矜歷數之在己,忽王業之艱難,不務以道恤人,將以申威海外。運拒諫之智,騁飾非之辯,恥轍跡之未遠,忘德義之不修。於是鑿通渠,開馳道,樹以柳杞,隱以金槌。西出玉門,東逾碣石,塹山堙谷,浮河達海。民力凋盡,徭戍無期,率土之心,鳥驚魚潰。方西規奄蔡,南討流求,視總八狄之師,屢踐三韓之域。自以威行萬物,顧指無違,又躬為長君,
 功高曩列,寵不假於外戚,權不逮於群下,足以轥轢軒、唐,奄吞周、漢,子孫萬代,人莫能窺,振古以來,一君則已。遂乃外疏猛士,內忌忠良,恥有盜竊之聲,惡聞喪亂之事。出師命將,不料眾寡,兵少力屈者,以畏懦受顯誅,謁誠克勝者,以功高蒙隱戮。或斃鋒刃之下,或殞鴆毒之中。賞不可以有功求,刑不可以無罪免,畏首畏尾,進退維谷。彼山東之群盜,多出廝役之中,無尺土之資,十家之產,豈有陳涉亡秦之志,張角亂漢之謀哉!皆苦於上欲無厭,下不堪命,饑寒交切,救死萑蒲。莫識旌旗什伍之容,安知行師用兵之勢!但人自為戰,眾怒難犯,故攻
 無完城,野無橫陣,星離棋布,以千百數。豪傑因其機以動之,乘其勢而用之,雖有勇敢之士,明智之將,連踵復沒,莫之能御。煬帝魂氣懾,望絕兩京,謀竄身於江湖,襲永嘉之舊跡。既而禍生轂下,釁起舟中,思早告而莫追,唯請死而獲可。身棄南巢之野,首懸白旗之上,子孫剿絕,宗廟為墟。



 夫以開皇之初,比於大業之盛,度土地之廣狹,料戶口之眾寡,算甲兵之多少,校倉廩之虛實,九鼎之譬鴻毛,未喻輕重,培塿之方嵩岱,曾何等級!論地險則遼隧未擬於長江,語人謀則勾麗不侔於陳國。高祖掃江南以清六合,煬帝事遼東而喪天下。其故何
 哉?所為之跡同,所用之心異也。高祖北卻強胡,南並百越,十有餘載,戎車屢動,民亦勞止,不為無事。然其動也,思以安之,其勞也,思以逸之。



 是以民致時雍,師無怨讟,誠在於愛利,故其興也勃焉。煬帝嗣承平之基,守已安之業,肆其淫放,虐用其民,視億兆如草芥,顧群臣如寇讎,勞近以事遠,求名而喪實。兵纏魏闕,阽危弗圖,圍解雁門,慢游不息。天奪之魄,人益其災,群盜並興,百殃俱起,自絕民神之望,故其亡也忽焉。訊之古老,考其行事,此高祖之所由興,而煬帝之所以滅者也,可不謂然乎!其隋之得失存亡,大較與秦相類。始皇並吞六國,高祖
 統一九州,二世虐用威刑,煬帝肆行猜毒,皆禍起於群盜,而身殞於匹夫。原始要終,若合符契矣。



 玄感宰相之子,荷國重恩,君之失德,當竭股肱。未議致身,先圖問鼎,遂假伊、霍之事,將肆莽、卓之心。人神同疾,敗不旋踵,兄弟就菹醢之誅,先人受焚如之酷,不亦甚乎!李密遭會風雲,奪其鱗翼,思封函谷,將割鴻溝。期月之間,眾數十萬,破化及,摧世充,聲動四方,威行萬里。雖運乖天眷,事屈興王,而義協人謀,雄名克振,壯矣!然志性輕狡,終致顛覆,其度長挈大,抑陳、項之季孟歟?



\end{pinyinscope}