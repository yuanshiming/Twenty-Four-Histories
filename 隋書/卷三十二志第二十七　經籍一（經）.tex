\article{卷三十二志第二十七 經籍一(經)}

\begin{pinyinscope}

 夫
 經籍也者,機神之妙旨,聖哲之能事,所以經天地,緯陰陽,正紀綱,弘道德,顯仁足以利物,藏用足以獨善。學之者將殖焉,不學者將落焉。大業崇之,則成欽明之德;匹夫克念,則有王公之重。其王者之所以樹風聲,流顯號,美教化,移風俗,何莫由乎斯道。故曰:其為人也,溫柔敦厚,《詩》教也;疏通知遠,《書》教也;廣博易良,《樂》教也;潔靜
 精微,《易》教也;恭儉莊敬,《禮》教也;屬辭比事,《春秋》教也。遭時制宜,質文迭用,應之以通變,通變之以中庸。中庸則可久,通變則可大。其教有適,其用無窮。實仁義之陶鈞,誠道德之橐籥也。其為用大矣,隨時之義深矣,言無得而稱焉。故曰:不疾而速,不行而至。



 今之所以知古,後之所以知今,其斯之謂也。是以大道方行,俯龜象而設卦;後聖有作,仰鳥跡以成文。書契已傳,繩木棄而不用;史官既立,經籍於是與焉。



 夫經籍也者,先聖據龍圖,握鳳紀,南面以君天下者,咸有史官,以紀言行。



 言則左史書之,動則右史書之。故曰「君舉必書」,懲勸斯在。考之前載,
 則《三墳》、《五典》、《八索》、《九丘》之類是也。下逮殷、周,史官尤備,紀言書事,靡有闕遺,則《周禮》所稱,太史掌建邦之六典、八法、八則,以詔王治;小史掌邦國之志,定世系,辨昭穆;內史掌王之八柄,策命而貳之;外史掌王之外令及四方之志,三皇、五帝之書;御史掌邦國都鄙萬民之治令,以贊塚宰。此則天子之史,凡有五焉。諸侯亦各有國史,分掌其職。則《春秋傳》,晉趙穿弒靈公,太史董狐書曰「趙盾殺其君」,以示於朝。宣子曰「不然。」對曰:「子為正卿,亡不越境,反不討賊,非子而誰?」齊崔杼弒莊公,太史書曰「崔杼弒其君」,崔子殺之。其弟嗣書,死者二人。其弟又書,乃
 舍之。南史聞太史盡死,執簡以往,聞既書矣,乃還。楚靈王與右尹子革語,左史倚相趨而過。王曰:「此良史也,能讀《三墳》、《五典》、《八索》、《九丘》。」然則諸侯史官,亦非一人而已,皆以記言書事,太史總而裁之,以成國家之典。不虛美,不隱惡,故得有所懲勸,遺文可觀,則《左傳》稱《周志》,《國語》有《鄭書》之類是也。



 暨夫周室道衰,紀綱散亂,國異政,家殊俗,褒貶失實,隳紊舊章。孔丘以大聖之才,當傾頹之運,嘆鳳鳥之不至,惜將墜於斯文,乃述《易》道而刪《詩》、《書》,修《春秋》而正《雅》、《頌》。壞禮崩樂,咸得其所。自哲人萎而微言絕,七十子散而大義乖,戰國縱橫,真偽莫辨,諸子之
 言,紛然淆亂。聖人之至德喪矣,先王之要道亡矣。陵夷踳駁,以至於秦。秦政奮豺狼之心,刬先代之跡,焚《詩》、《書》,坑儒士,以刀筆吏為師,制挾書之令。學者逃難,竄伏山林,或失本經,口以傳說。



 漢氏誅除秦、項,未及下車,先命叔孫通草綿蕝之儀,救擊柱之弊。其後張蒼治律歷,陸賈撰《新語》,曹參薦蓋公言黃老,惠帝除挾書之律,儒者始以其業行於民間。猶以去聖既遠,經籍散逸,簡札錯亂,傳說紕繆,遂使《書》分為二,《詩》分為三,《論語》有齊、魯之殊,《春秋》有數家之傳。其餘互有踳駁,不可勝言。此其所以博而寡要,勞而少功者也。武帝置太史公,命天下計
 書,先上太史,副上丞相,開獻書之路,置寫書之官,外有太常、太史、博士之藏,內有延閣、廣內、秘室之府。司馬談父子世居太史,探採前代,斷自軒皇,逮於孝武,作《史記》一百三十篇。詳其禮制,蓋史官之舊也。至於孝成,秘藏之書,頗有亡散,乃使謁者陳農,求遺書於天下。命光祿大夫劉向校經傳諸子詩賦,步兵校尉任宏校兵書,太史令尹咸校數術,太醫監李柱國校方技。每一書就,向輒撰為一錄,論其指歸,辨其訛謬,敘而奏之。向卒後,哀帝使其子歆嗣父之業。乃徙溫室中書於天祿閣上。歆遂總括群篇,撮其指要,著為《七略》:一曰《集略》,二曰《六藝
 略》,三曰《諸子略》,四曰《詩賦略》,五曰《兵書略》,六曰《術數略》,七曰《方技略》。大凡三萬三千九十卷。王莽之末,又被焚燒。光武中興,篤好文雅,明、章繼軌,尤重經術。四方鴻生巨儒,負袠自遠而至者,不可勝算。石室、蘭臺,彌以充積。又於東觀及仁壽閣集新書,校書郎班固、傅毅等典掌焉。並依《七略》而為書部,固又編之,以為《漢書·藝文志》。董卓之亂,獻帝西遷,圖書縑帛,軍人皆取為帷囊。所收而西,猶七十餘載。兩京大亂,掃地皆盡。



 魏氏代漢,採掇遺亡,藏在秘書中、外三閣。魏秘書郎鄭默,始制《中經》,秘書監荀勖,又因《中經》,更著《新簿》,分為四部,總括群書。一曰
 甲部,紀六藝及小學等書;二曰乙部,有古諸子家、近世子家、兵書、兵家、術數;三曰丙部,有史記、舊事、皇覽簿、雜事;四曰丁部,有詩賦、圖贊、汲塚書。大凡四部合二萬九千九百四十五卷。但錄題及言,盛以縹囊,書用緗素。至於作者之意,無所論辯。惠、懷之亂,京華蕩覆,渠閣文籍,靡有孑遺。



 東晉之初,漸更鳩聚。著作郎李充以勖舊簿校之,其見存者,但有三千一十四卷。充遂總沒眾篇之名,但以甲乙為次。自爾因循,無所變革。其後中朝遺書,稍流江左。宋元嘉八年,秘書監謝靈運造《四部目錄》,大凡六萬四千五百八十二卷。



 元徽元年,秘書丞王儉又
 造《目錄》,大凡一萬五千七百四卷。儉又別撰《七志》:一曰《經典志》,紀六藝、小學、史記、雜傳;二曰《諸子志》,紀今古諸子;三曰《文翰志》,紀詩賦;四曰《軍書志》,紀兵書;五曰《陰陽志》,紀陰陽圖緯;六曰《術藝志》,紀方技;七曰《圖譜志》,紀地域及圖書。其道、佛附見,合九條。然亦不述作者之意,但於書名之下,每立一傳,而又作九篇條例,編乎首卷之中。文義淺近,未為典則。齊永明中,秘書丞王亮、監謝朏,又造《四部書目》,大凡一萬八千一十卷。齊末兵火,延燒秘閣,經籍遺散。梁初,秘書監任昉躬加部集,又於文德殿內列藏眾書,華林園中總集釋典,大凡二萬三千一
 百六卷,而釋氏不豫焉。梁有秘書監任昉、殷鈞《四部目錄》,又《文德殿目錄》。其術數之書,更為一部,使奉朝請祖恆撰其名。故梁有《五部目錄》。普通中,有處士阮孝緒,沉靜寡欲,篤好墳史,博採宋、齊已來王公之家凡有書記,參校官簿,更為《七錄》:一曰《經典錄》,紀六藝;二曰《記傳錄》,紀史傳;三曰《子兵錄》,紀子書、兵書;四曰《文集錄》,紀詩賦;五曰《技術錄》,紀數術;六曰《佛錄》;七曰《道錄》。其分部題目,頗有次序,割析辭義,淺薄不經。梁武敦悅詩書,下化其上,四境之內,家有文史。元帝克平侯景,收文德之書及公私經籍,歸於江陵,大凡七萬餘卷。周師入郢,咸自焚
 之。陳天嘉中,又更鳩集,考其篇目,遺闕尚多。



 其中原則戰爭相尋,干戈是務,文教之盛,苻、姚而已。宋武入關,收其圖籍,府藏所有,才四千卷。赤軸青紙,文字古拙。後魏始都燕代,南略中原,粗收經史,未能全具。孝文徙都洛邑,借書於齊,秘府之中,稍以充實。暨於爾硃之亂,散落人間。後齊遷鄴,頗更搜聚,迄於天統、武平,校寫不輟。後周始基關右,外逼強鄰,戎馬生郊,日不暇給。保定之始,書止八千,後稍加增,方盈萬卷。周武平齊,先封書府,所加舊本,才至五千。



 隋開皇三年,秘書監牛弘表請分遣使人,搜訪異本。每書一卷,賞絹一匹,校寫既定,本即歸主。
 於是民間異書,往往間出。及平陳已後,經籍漸備。檢其所得,多太建時書,紙墨不精,書亦拙惡。於是總集編次,存為古本。召天下工書之士,京兆韋霈、南陽杜頵等,於秘書內補續殘缺,為正副二本,藏於宮中,其餘以實秘書內、外之閣,凡三萬餘卷。煬帝即位,秘閣之書,限寫五十副本,分為三品:上品紅琉璃軸,中品紺琉璃軸,下品漆軸。於東都觀文殿東西廂構屋以貯之,東屋藏甲乙,西屋藏丙丁。又聚魏已來古跡名畫,於殿後起二臺,東曰妙楷臺,藏古跡;西曰寶跡臺,藏古畫。又於內道場集道、佛經,別撰目錄。



 大唐武德五年,克平偽鄭,盡收其圖
 書及古跡焉。命司農少卿宋遵貴載之以船,溯河西上,將致京師。行經底柱,多被漂沒,其所存者,十不一二。其《目錄》亦為所漸濡,時有殘缺。今考見存,分為四部,合條為一萬四千四百六十六部,有八萬九千六百六十六卷。其舊錄所取,文義淺俗、無益教理者,並刪去之。其舊錄所遺,辭義可採,有所弘益者,咸附入之。遠覽馬史、班書,近觀王、阮志、錄,挹其風流體制,削其浮雜鄙俚,離其疏遠,合其近密,約文緒義,凡五十五篇,各列本條之下,以備《經籍志》。雖未能研幾探賾,窮極幽隱,庶乎弘道設教,可以無遺闕焉。夫仁義禮智,所以治國也,方技數術,
 所以治身也;諸子為經籍之鼓吹,文章乃政化之黼黻,皆為治之具也。故列之於此志云。



 《歸藏》十三卷晉太尉參軍薛貞注。



 《周易》二卷魏文侯師卜子夏傳,殘缺。梁六卷。



 《周易》十卷漢魏郡太守京房章句。



 《周易》八卷漢曲臺長孟喜章句,殘缺。梁十卷。又有漢單父長費直注《周易》四卷,亡。



 《周易》九卷後漢大司農鄭玄注。梁又有漢南郡太守馬融注《周易》一卷,亡。



 《周易》五卷漢荊州牧劉表章句。梁有漢荊州五業從事宋忠注《周易》十卷,亡。



 《周易》十一卷漢司空荀爽注。



 《周易》十卷魏衛將軍王肅注。



 《周易》十卷魏尚書郎王弼注《六十四卦》六卷,韓康伯注《系辭》以下三卷,王弼又撰《易略例》一卷,梁有魏大司農卿董遇注《周易》十卷,魏散騎常侍荀煇注《周易》十卷,亡。



 《周易》十卷吳太常姚信注。



 《周易》四卷晉儒林從事黃穎注。梁有十卷,今殘缺。



 《周易》九卷吳侍御史虞翻注。



 《周易》十五卷吳鬱林太守陸績注。



 《周易》
 十卷晉散騎常侍干寶注。



 《周易》三卷晉驃騎將軍王暠注,殘缺。梁有十卷。



 《周易》八卷晉著作郎張璠注,殘缺。梁有十卷。



 《周易馬、鄭、二王四家集解》十卷《周易荀爽九家注》十卷《周易楊氏集二王注》五卷,梁有《集馬、鄭、二王解》十卷,亡。



 《周易》十卷蜀才注。梁有齊安參軍費元珪注《周易》九卷,謝氏注《周易》八卷,尹濤注《周易》六卷,亡。



 《周易》十卷後魏司徒崔浩注。



 《周易》十卷梁處士何胤注。梁有臨海令伏曼容注《周易》八卷,侍中硃異集注《周易》一百卷,又《周易集注》三十卷,亡。



 《周易》七卷姚規注。



 《周易》十三卷崔覲注。



 《周易》十三卷傅氏注。



 《周易》一帙十卷盧氏注。



 《周易系辭》二卷晉桓玄注。



 《周易系辭》二卷晉西中郎將謝萬等注。



 《周易系辭》二卷晉太常韓康伯注。



 《周易系辭》二卷梁太中大夫宋褰注。又有宋東陽太守卞伯玉注《系辭》二卷,亡。



 《周易系辭》二卷荀柔之注。



 《周易集注系辭》二卷梁
 有宋太中大夫徐爰注《系辭》二卷,亡。



 《周易音》一卷東晉太子前率徐邈撰。



 《周易音》一卷東晉尚書郎李軌弘範撰。



 《周易音》一卷範氏撰。



 《周易並注音》七卷秘書學士陸德明撰。



 《周易盡神論》一卷魏司空鐘會撰。梁有《周易無互體論》三卷,鐘會撰,亡。



 《周易象論》三卷晉尚書郎欒肇撰。



 《周易卦序論》一卷晉司徒右長史楊乂撰。



 《周易統略》五卷晉少府卿鄒湛撰。



 《周易論》二卷晉馮翊太守阮常撰。



 《周易論》一卷晉荊州刺史宋岱撰。梁有《擬周易說》八卷,範氏撰;《周易宗塗》四卷,干寶撰;《周易問難》二卷,王氏撰;《周易問答》一卷,揚州從事徐伯珍撰;《周易難王輔嗣義》一卷,晉揚州刺史顧夷等撰;《周易雜論》十四卷。



 亡。



 《周易義》一卷宋陳令範歆撰。



 《周易玄品》二卷《周易論》十卷齊中書郎周顒撰。梁有三十卷,亡。



 《周易論》四卷範氏撰。



 《周易統例》十卷崔覲撰。



 《周易爻義》一卷干寶撰。



 《周易乾坤義》一卷齊步兵
 校尉劉瓛撰。梁又有齊臨沂令李玉之、梁釋法通等《乾坤義》各一卷,亡。



 《周易大義》二十一卷梁武帝撰。



 《周易幾義》一卷梁南平王撰。梁有《周易疑通》五卷,宋中散大夫何諲之撰;《周易四德例》一卷,劉瓛撰。亡。



 《周易大義》一卷梁有《周易錯》八卷,京房撰;《周易日月變例》六卷,虞翻、陸績撰;《周易卦象數旨》六卷,東晉樂安亭侯李顒撰;《周易爻》一卷,馬揩撰。亡。



 《周易太義》二卷陸德明撰。



 《周易釋序義》三卷《周易開題義》十卷梁蕃撰。



 《周易問》二十卷《周易義疏》十九卷宋明帝集群臣講。梁又有《國子講易》議六卷;《宋明帝集群臣講易義疏》二十卷;《齊永明國學講周易講疏》二十六卷;又《周易義》三卷,沈林撰。亡。



 《周易講疏》三十五卷梁武帝撰。



 《周易講疏》十六卷梁五經博士褚仲都撰。



 《周易義疏》十四卷梁都官尚書蕭子政撰。



 《周易系辭義疏》三卷蕭子政撰。



 《周易講疏》三十卷陳諮議參軍張譏撰。



 《周易文句義》二十卷梁有《擬周
 易義疏》十三卷。



 《周易義疏》十六卷陳尚書左僕射周弘正撰。



 《周易私記》二十卷《周易講疏》十三卷國子祭酒何妥撰。



 《周易系辭義疏》二卷劉瓛撰。



 《周易系辭義疏》一卷梁武帝撰。



 《周易系辭義疏》二卷蕭子政撰。梁有《周易乾坤三象》、《周易新圖》各一卷;又《周易普玄圖》八卷,薛景和撰;《周易大演通統》一卷,顏氏撰。



 《周易譜》一卷。



 右六十九部,五百五十一卷。通計亡書,合九十四部,八百二十九卷。



 昔宓羲氏始畫八卦,以通神明之德,以類萬物之情,蓋因而重之,為六十四卦。



 及乎三代,實為三《易》,夏曰《連山》;殷曰《歸藏》;周文王作卦辭,謂之《周易》。周公又作《爻辭》,孔子為《彖》、《象》、《系辭》、《文言》、《序卦》、《說卦》、《雜卦》,而子夏為之傳。及秦焚書,《周易》獨以卜筮得存,唯失《說卦》三篇。後河內女
 子得之。漢初,傳《易》者有田何,何授丁寬,寬授田王孫,王孫授沛人施仇、東海孟喜、瑯邪梁丘賀。由是有施、孟、梁丘之學。



 又有東郡京房,自云受《易》於梁國焦延壽,別為京氏學。嘗立,後罷。後漢施、孟、梁丘、京氏,凡四家並立,而傳者甚眾。漢初又有東萊費直傳《易》,其本皆古字,號曰《古文易》。以授瑯邪王璜,璜授沛人高相,相以授子康及蘭陵母將永。



 故有費氏之學,行於人間,而未得立。後漢陳元、鄭眾,皆傳費氏之學。馬融又為其傳,以授鄭玄。玄作《易注》,荀爽又作《易傳》。魏代王肅、王弼,並為之注。



 自是費氏大興,高氏遂衰。梁丘、施氏、高氏,亡於西晉。孟氏、京
 氏,有書無師。



 梁、陳鄭玄、王弼二注,列於國學。齊代唯傳鄭義。至隋,王注盛行,鄭學浸微,今殆絕矣。《歸藏》,漢初已亡,案晉《中經》有之,唯載卜筮,不似聖人之旨。



 以本卦尚存,故取貫於《周易》之首,以備《殷易》之缺。



 《古文尚書》十三卷漢臨淮太守孔安國傳。



 《今字尚書》十四卷孔安國傳。



 《尚書》十一卷馬融注。



 《尚書》九卷鄭玄注。



 《尚書》十一卷王肅注。



 《尚書》十五卷晉祠部郎謝沈撰。



 《集解尚書》十一卷李顒注。



 《集釋尚書》十一卷宋給事中姜道盛注。



 《古文尚書舜典》一卷晉豫章太守範寧注。梁有《尚書》十卷,範寧注,亡。



 《尚書亡篇序》一卷梁五經博士劉叔嗣注。梁有《尚書》二十一卷,劉叔嗣注;又有《尚書新集序》一卷。亡。



 《尚書逸篇》二卷《古文尚書音》一卷徐
 邈撰。梁有《尚書音》五卷,孔安國、鄭玄、李軌、徐邈等撰。



 《今文尚書音》一卷秘書學士顧彪撰。



 《尚書大傳》三卷鄭玄注。



 《大傳音》二卷顧彪撰。



 《尚書洪範五行傳論》十一卷漢光祿大夫劉向注。



 《尚書駁議》五卷王肅撰。梁有《尚書義問》三卷,鄭玄、王肅及晉五經博士孔晁撰;《尚書釋問》四卷,魏侍中王粲撰;《尚書王氏傳問》二卷;《尚書義》二卷,吳太尉範順問,劉毅答。亡。



 《尚書新釋》二卷李顒撰。



 《尚書百問》一卷齊太學博士顧歡撰。



 《尚書大義》二十卷梁武帝撰。



 《尚書百釋》三卷梁國子助教巢猗撰。



 《尚書義》三卷巢猗撰。



 《尚書義疏》十卷梁國子助教費甝撰。梁有《尚書義疏》四卷,晉樂安王友伊說撰,亡。



 《尚書義疏》三十卷蕭詧司徒蔡大寶撰。



 《尚書義注》三卷呂文優撰。



 《尚書義疏》七卷《尚書述義》二十卷國子助教劉炫撰。



 《尚書疏》二十卷顧彪撰。



 《尚書閏義》一卷《尚書義》三卷劉先生撰。



 《尚書
 釋問》一卷虞氏撰。



 《尚書文外義》一卷顧彪撰。



 右三十二部,二百四十七卷。通計亡書,合四十一部,共二百九十六卷。



 《書》之所興,蓋與文字俱起。孔子觀《書》周室,得虞、夏、商、周四代之典,刪其善者,上自虞,下至周,為百篇,編而序之。遭秦滅學,至漢,唯濟南伏生口傳二十八篇。又河內女子得《泰誓》一篇,獻之。伏生作《尚書傳》四十一篇,以授同郡張生,張生授千乘歐陽生,歐陽生授同郡倪寬,寬授歐陽生之子,世世傳之,至曾孫歐陽高,謂之《尚書》歐陽之學。又有夏侯都尉,受業於張生,以授族子始昌,始昌傳族子勝,為大夏侯之學。勝傳從子建,別為小夏侯之
 學。故有歐陽,大、小夏侯,三家並立。訖漢東京,相傳不絕,而歐陽最盛。初漢武帝時,魯恭王壞孔子舊宅,得其末孫惠所藏之書,字皆古文。孔安國以今文校之,得二十五篇。



 其《泰誓》與河內女子所獻不同。又濟南伏生所誦,有五篇相合。安國並依古文,開其篇第,以隸古字寫之,合成五十八篇。其餘篇簡錯亂,不可復讀,並送之官府。



 安國又為五十八篇作傳,會巫蠱事起,不得奏上,私傳其業於都尉朝,朝授膠東庸生,謂之《尚書古文》之學,而未得立。後漢扶風杜林,傳《古文尚書》,同郡賈逵為之作訓,馬融作傳,鄭玄亦為之注。然其所傳,唯二十九篇,又
 雜以今文,非孔舊本。自餘絕無師說。



 晉世秘府所存,有《古文尚書》經文,今無有傳者。及永嘉之亂,歐陽,大、小夏侯《尚書》並亡。濟南伏生之傳,唯劉向父子所著《五行傳》是其本法,而又多乖戾。至東晉,豫章內史梅賾,始得安國之傳,奏之,時又闕《舜典》一篇。齊建武中,吳姚方興於大桁市得其書,奏上,比馬、鄭所注多二十八字,於是始列國學。梁、陳所講,有孔、鄭二家,齊代唯傳鄭義。至隋,孔、鄭並行,而鄭氏甚微。



 自餘所存,無復師說。又有《尚書逸篇》,出於齊、梁之間,考其篇目,似孔壁中書之殘缺者,故附《尚書》之末。



 《
 韓詩》二十二卷漢常山太傅韓嬰,薛氏章句。



 《韓詩翼要》十卷漢侯苞傳。



 《韓詩外傳》十卷梁有《韓詩譜》二卷,《詩神泉》一卷,漢有道徵士趙曄撰,亡。



 《毛詩》二十卷漢河間太傅毛萇傳,鄭氏箋。梁有《毛詩》十卷,馬融注,亡。



 《毛詩》二十卷王肅注。梁有《毛詩》二十卷,鄭玄,王肅合注;《毛詩》二十卷,謝沈注;《毛詩》二十卷,晉兗州別駕江熙注。亡。



 《集注毛詩》二十四卷梁桂州刺史崔靈恩注。梁有《毛詩序》一卷,梁隱居先生陶弘景注,亡。



 《毛詩箋音證》十卷後魏太常卿劉芳撰。梁有《毛詩音》十六卷,徐邈等撰;《毛詩音》二卷,徐邈撰;《毛詩音隱》一卷,干氏撰。亡。



 《毛詩並注音》八卷秘書學士魯世達撰。



 《毛詩譜》三卷吳太常卿徐整撰。



 《毛詩譜》二卷太叔求及劉炫注。



 《謝氏毛詩譜鈔》一卷梁有《毛詩雜議難》十卷,漢侍中賈逵撰,亡。



 《毛詩義問》十卷魏太子文學劉楨撰。



 《毛詩義駁》八卷王肅撰。



 《毛詩奏事》一卷王肅撰。有《毛詩問難》二卷,王肅撰,亡。



 《毛詩駁》一卷魏司空王基撰,殘缺。梁五
 卷。又有《毛詩答問》、《駁譜》,合八卷;又《毛詩釋義》十卷,謝沈撰;《毛詩義》四卷,《毛詩箋傳是非》二卷,並魏秘書郎劉潘撰;《毛詩答雜問》七卷,吳侍中韋昭、侍中硃育等撰;《毛詩義注》四卷。亡。



 《毛詩異同評》十卷晉長沙太守孫毓撰。



 《難孫氏毛詩評》四卷晉徐州從事陳統撰。梁有《毛詩表隱》二卷,陳統撰,亡。



 《毛詩拾遺》一卷郭璞撰。梁又有《毛詩略》四卷,亡。



 《毛詩辨異》三卷晉給事郎楊乂撰。梁有《毛詩背隱義》二卷,宋中散大夫徐廣撰;《毛詩引辨》一卷,宋奉朝請孫暢之撰;《毛詩釋》一卷,宋金紫光祿大夫何偃撰;《毛詩檢漏義》二卷,梁給事郎謝曇濟撰;《毛詩總集》六卷,《毛詩隱義》十卷,並梁處士何胤撰。亡。



 《毛詩異義》二卷楊乂撰。梁有《毛詩雜義》五卷,楊乂撰;《毛詩義疏》十卷,謝沈撰;《毛詩雜義》四卷,晉江州刺史殷仲堪撰;《毛詩義疏》五卷,張氏撰。亡。



 《毛詩集解敘義》一卷顧歡等撰。



 《毛詩序義》二卷宋通直郎雷次宗撰。梁有《毛詩義》一卷,雷次宗撰;《毛詩序注》一卷,宋交州刺史阮珍之撰;《毛詩序義》七卷,孫暢之撰。亡。



 《毛詩集小序》一卷劉炫注。



 《毛詩序義疏》一卷
 劉瓛等撰,殘缺。梁三卷。梁有《毛詩篇次義》一卷,劉瓛撰;《毛詩雜義注》三卷。亡。



 《毛詩發題序義》一卷梁武帝撰。



 《毛詩大義》十一卷梁武帝撰。梁有《毛詩十五國風義》二十卷,梁簡文撰。



 《毛詩大義》十三卷《毛詩草木蟲魚疏》二卷烏程令吳郡陸機撰。



 《毛詩義疏》二十卷舒援撰。



 《毛詩誼府》三卷後魏安豐王元延明撰。



 《毛詩義疏》二十八卷蕭巋散騎常侍沈重撰。



 《毛詩義疏》二十卷《毛詩義疏》二十九卷《毛詩義疏》十卷《毛詩義疏》十一卷《毛詩義疏》二十八卷《毛詩述義》四十卷國子助教劉炫撰。



 《毛詩章句義疏》四十卷魯世達撰。



 《毛詩釋疑》一卷梁有《毛詩圖》三卷,《毛詩孔子經圖》十二卷,《毛詩古聖賢圖》二卷,亡。



 《業詩》二十卷宋奉朝請業遵注。



 右三十九部,四百四十二卷。通計亡書,合七十六部,六百八十三卷。



 《
 詩》者,所以導達心靈,歌詠情志者也。故曰:「在心為志,發言為詩。」



 上古人淳俗樸,情志未惑。其後君尊於上,臣卑於下,面稱為諂,目諫為謗,故誦美譏惡,以諷刺之。初但歌詠而已,後之君子,因被管弦,以存勸戒。夏、殷已上,詩多不存。周氏始自後稷,而公劉克篤前烈,太王肇基王跡,文王光昭前緒,武王克平殷亂,成王、周公化至太平,誦美盛德,踵武相繼。幽、厲板蕩,怨刺並興。



 其後王澤竭而詩亡,魯太師摯次而錄之。孔子刪詩,上採商,下取魯,凡三百篇。



 至秦,獨以為諷誦,不滅。漢初,有魯人申公,受《詩》於浮丘伯,作詁訓,是為《魯詩》。齊人轅固生亦傳《詩》,是
 為《齊詩》。燕人韓嬰亦傳《詩》,是為《韓詩》。終於後漢,三家並立。漢初,又有趙人毛萇善《詩》,自云子夏所傳,作《詁訓傳》,是為《毛詩》古學,而未得立。後漢有九江謝曼卿,善《毛詩》,又為之訓。東海衛敬仲,受學於曼卿。先儒相承,謂之《毛詩》。序,子夏所創,毛公及敬仲又加潤益。鄭眾、賈逵、馬融,並作《毛詩傳》,鄭玄作《毛詩箋》。



 《齊詩》,魏代已亡;《魯詩》亡於西晉;《韓詩》雖存,無傳之者。唯《毛詩鄭箋》,至今獨立。又有《業詩》,奉朝請業遵所注,立義多異,世所不行。



 《周官禮》十二卷馬融注。



 《周官禮》十二卷鄭玄注。



 《周官禮》十二卷王肅注。



 《周官禮》十二卷伊說注。



 《周官禮》十二卷干寶注。梁又有《周官
 寧朔新書》八卷,晉燕王師王懋約撰,亡。



 《集注周官禮》二十卷崔靈恩注。



 《禮音》三卷劉昌宗撰。



 《周官禮異同評》十二卷晉司空長史陳劭撰。



 《周官禮駁難》四卷孫略撰。梁有《周官駁難》三卷,孫琦問,干寶駁,晉散騎常侍虞喜撰。



 《周官禮義疏》四十卷沈重撰。



 《周官禮義疏》十九卷《周官禮義疏》十卷《周官禮義疏》九卷《周官分職》四卷《周官禮圖》十四卷梁有《郊祀圖》二卷,亡。



 《儀禮》十七卷鄭玄注。



 《儀禮》十七卷王肅注。梁有李軌、劉昌宗音各一卷,鄭玄音二卷,亡。



 《儀禮義疏見》二卷《儀禮義疏》六卷《喪服經傳》一卷馬融注。



 《喪服經傳》一卷鄭玄注。



 《喪服經傳》一卷王肅注。



 《喪服經傳》一卷晉給事中袁準注。



 《集注喪服經傳》一卷晉廬陵太守孔倫撰。



 《喪服經傳》一卷陳銓注。



 《集注喪服經傳》一卷宋太
 中大夫裴松之撰。



 《略注喪服經傳》一卷雷次宗注。



 《集注喪服經傳》二卷宋丞相諮議參軍蔡超注。梁又有《喪服經傳》一卷,宋徵士劉道拔注,亡。



 《集解喪服經傳》二卷齊東平太守田僧紹解。



 《喪服義疏》二卷梁步兵校尉、五經博士賀瑒撰。梁又有《喪服經傳義疏》五卷,齊散騎郎司馬憲撰;《喪服經傳義疏》二卷,齊給事中樓幼瑜撰;《喪服經傳義疏》一卷,劉瓛撰;《喪服經傳義疏》一卷,齊徵士沈麟士撰。



 《喪服經傳義疏》一卷梁尚書左丞何佟之撰,亡。



 《喪服傳》一卷梁通直郎裴子野撰。



 《喪服文句義疏》十卷梁國子助教皇侃撰。



 《喪服義》十卷陳國子祭酒謝嶠撰。



 《喪服義鈔》三卷梁有《喪服經傳隱義》一卷,亡。



 《喪服要記》一卷王肅注。



 《喪服要記》一卷蜀丞相蔣琬撰。梁有《喪服變除圖》五卷,吳齊王傅射慈撰,亡。



 《喪服要集》二卷晉征南將軍杜預撰。又有《喪服要記》二卷,晉侍中劉逵撰,亡。



 《喪服儀》一卷晉太保衛瓘撰。梁有《喪服要記》六卷,晉司空賀循撰;《喪服要問》六卷,劉德明撰;《喪
 服》三十一卷,宋員外郎散騎庾蔚之撰;《喪服要問》二卷,張耀撰;《喪服難問》六卷,崔凱撰;《喪服雜記》二十卷,伊氏撰;《喪服釋疑》二十卷,劉智撰。亡。



 《漢荊州刺史劉表新定禮》一卷《喪服要略》一卷晉太學博士環濟撰。



 《喪服要略》二卷《喪服制要》一卷徐氏撰。



 《喪服譜》一卷鄭玄注。



 《喪服譜》一卷晉開府儀同三司蔡謨撰。



 《喪服譜》一卷賀循撰。



 《喪服變除》一卷晉散騎常侍葛洪撰。



 《兇禮》一卷晉廣陵相孔衍撰。



 《喪服要記》十卷賀循撰。梁有《喪服要記》,宋員外常侍庾蔚之注;又《喪服世要》一卷,庾蔚之撰;《喪服集議》十卷,宋撫軍司馬費沈撰。



 《喪服古今集記》三卷齊太尉王儉撰。



 《喪服世行要記》十卷齊光祿大夫王逡撰。



 《喪服答要難》一卷袁祈撰。



 《喪服記》十卷王氏撰。



 《喪服五要》一卷嚴氏撰。



 《駁喪服經傳》一卷卜氏傳。



 《喪服疑問》一卷樊氏撰。



 《喪服圖》一卷王儉撰。



 《喪服圖》一
 卷賀游撰。



 《喪服圖》一卷崔逸撰。梁有《喪服祥禫雜議二十九卷,《喪服雜議故事》二十一卷,又《戴氏喪服五家要記圖譜》五卷,《喪服君臣圖儀》一卷,亡。



 《五服圖》一卷《五服圖儀》一卷《喪服禮圖》一卷《五服略例》一卷《喪服要問》一卷《喪服問答目》十三卷皇侃撰。



 《喪服假寧制》三卷《喪禮五服》七卷大將軍袁憲撰。



 《論喪服決》一卷《喪禮鈔》三卷王隆伯撰。



 《大戴禮記》十三卷漢信都王太傅戴德撰。梁有《謚法》三卷,後漢安南太守劉熙注,亡。



 《夏小正》一卷戴德撰。



 《禮記》十卷漢北中郎將盧植注。



 《禮記》二十卷漢九江太守戴聖撰,鄭玄注。



 《禮記》三十卷王肅注。梁有《禮記》十二卷,業遵注,亡。



 《禮記寧朔新書》八卷王懋約注。梁有二十卷。



 《月令章句》十二卷漢左中郎將蔡邕撰。



 《禮記音義隱》一卷謝氏撰。



 《禮記音》二卷宋中散大夫徐爰撰。梁有鄭玄、王肅、
 射慈、射貞、孫毓、繆炳音各一卷,蔡謨、東晉安北諮議參軍曹耽、國子助教尹毅、李軌、員外郎範宣音各二卷,徐邈音三卷,劉昌宗音五卷,亡。



 《禮記音義隱》七卷《禮記》三十卷魏秘書監孫炎注。



 《禮略》二卷《禮記要鈔》十卷緱氏撰。梁有禮義四卷,魏侍中鄭小同撰;《摭遺別記》一卷,樓幼瑜撰,亡。



 《禮記新義疏》二十卷賀瑒撰。梁有《義疏》三卷,宋豫章郡丞雷肅之撰,亡。



 《禮記講疏》九十九卷皇侃撰。



 《禮記義疏》四十八卷皇侃撰。



 《禮記義疏》四十卷沈重撰。



 《禮記義》十卷何氏撰。



 《禮記義疏》三十八卷。



 《禮記疏》十一卷《禮記大義》十卷梁武帝撰。



 《禮記文外大義》二卷秘書學士褚暉撰。



 《禮大義》十卷《禮記義證》十卷劉芳撰。



 《禮大義章》七卷《喪禮雜義》三卷《禮記中庸傳》二卷宋散騎常侍戴顒撰。



 《中庸講疏》一卷梁武帝撰。



 《私記制旨
 中庸義》五卷《禮記略解》十卷庚氏撰。



 《禮記評》十一卷劉雋撰。



 《石渠禮論》四卷戴聖撰。梁有《群儒疑義》十二卷,戴聖撰。



 《禮論》三百卷宋御史中丞何承天撰。



 《禮論條牒》十卷宋太尉參軍任預撰。



 《禮論帖》三卷任預撰。梁四卷。



 《禮論鈔》二十卷庾蔚之撰。



 《禮論要鈔》十卷王儉撰。梁三卷。



 《禮論要鈔》一百卷賀瑒撰。



 《禮論鈔》六十九卷《禮論要鈔》十卷梁有齊御史中丞荀萬秋《鈔略》二卷;尚書儀曹郎丘季彬論五十八卷,議一百三十卷,統六卷。亡。



 《禮論答問》八卷宋中散大夫徐廣撰。



 《禮論答問》十三卷徐廣撰。



 《禮答問》二卷徐廣撰,殘缺。梁十一卷。



 《禮答問》六卷庾蔚之撰。



 《禮答問》三卷王儉撰。梁有晉益陽令吳商《禮難》十二卷,《雜議》十二卷,又《禮議雜記故事》十三卷,《喪雜事》二十卷;宋光祿大夫傅隆議二卷,《祭法》五卷。亡。



 《禮答問》十二卷《禮雜問》十卷範寧撰。



 《禮答問》十卷何
 佟之撰。梁二十卷。



 《禮雜問》十卷《禮雜答問》八卷《禮雜答問》六卷《禮雜問答鈔》一卷何佟之撰。



 《問禮俗》十卷董勛撰。



 《問禮俗》九卷董子弘撰。



 《答問雜儀》二卷任預撰。



 《禮義答問》八卷王儉撰。



 《禮疑義》五十二卷梁護軍周舍撰。



 《制旨革牲大義》三卷梁武帝撰。



 《禮樂義》十卷《禮秘義》三卷《三禮目錄》一卷鄭玄撰。梁有陶弘景注一卷,亡。



 《三禮義宗》三十卷崔靈恩撰。



 《三禮宗略》二十卷元延明撰。



 《三禮大義》十三卷《三禮大義》四卷《三禮雜大義》三卷梁有《司馬法》三卷,《李氏訓記》三卷;又《郊丘議》三卷,魏太尉蔣濟撰;《祭法》五卷,又《明堂議》三卷,王肅撰;《雜祭法》六卷,晉司空中郎盧諶撰;《祭典》三卷,晉安北將軍範汪撰;《七廟議》一卷,又《後養議》五卷,干寶撰;《雜鄉射等議》三卷,晉太尉庾亮撰;《逆降義》三卷,宋特進延顏之撰;《逆降義》一卷,田僧紹撰;《分明士制》三卷,何承天撰;《釋疑》二卷,郭鴻撰;《答
 問》四卷,徐廣撰;《答問》五十卷,何胤撰;又《答問》十卷。亡。



 《三禮圖》九卷鄭玄及後漢侍中阮諶等撰。



 《周室王城明堂宗廟圖》一卷祁諶撰。梁又有《冠服圖》一卷,《五宗圖》一卷,《月令圖》一卷,亡。右一百三十六部,一千六百二十二卷。通計亡書,二百一十一郎,二千一百八十六卷。



 自大道既隱,天下為家,先王制其夫婦、父子、君臣、上下、親疏之節。至於三代,損益不同。周衰,諸侯僭忒,惡其害己,多被焚削。自孔子時,已不能具,至秦而頓滅。漢初,有高堂生傳十七篇,又有古經,出於淹中,而河間獻王好古愛學,收集餘燼,得而獻之,合五十六篇,並威儀之事。
 而又得《司馬穰苴兵法》一百五十五篇,及《明堂陰陽》之記,並無敢傳之者。唯古經十七篇與高堂生所傳不殊,而字多異。自高堂生至宣帝時後蒼,最明其業,乃為《曲臺記》。蒼授梁人戴德,及德從兄子聖、沛人慶普,於是有大戴、小戴、慶氏,三家並立。後漢唯曹元傳慶氏,以授其子褒。然三家雖存並微,相傳不絕。漢末,鄭玄傳小戴之學,後以古經校之,取其於義長者作注,為鄭氏學。其《喪服》一篇,子夏先傳之,諸儒多為注解,今又別行。而漢時有李氏得《周官》。《周官》蓋周公所制官政之法,上於河間獻王,獨闕《冬官》一篇。獻王購以千金不得,遂取《考工記》
 以補其處,合成六篇奏之。至王莽時,劉歆始置博士,以行於世。河南緱氏及杜子春受業於歆,因以教授。是後馬融作《周官傳》,以授鄭玄,玄作《周官注》。漢初,河間獻王又得仲尼弟子及後學者所記一百三十一篇獻之,時亦無傳之者。至劉向考校經籍,檢得一百三十篇,向因第而敘之。而又得《明堂陰陽記》三十三篇、《孔子三朝記》七篇、《王史氏記》二十一篇、《樂記》二十三篇,凡五種,合二百十四篇。戴德刪其煩重,合而記之,為八十五篇,謂之《大戴記》。而戴聖又刪大戴之書,為四十六篇,謂之《小戴記》。漢末馬融,遂傳小戴之學。融又定《月令》一篇、《明堂
 位》一篇、《樂記》一篇,合四十九篇;而鄭玄受業於融,又為之注。今《周官》六篇、古經十七篇、《小戴記》四十九篇,凡三種。唯《鄭注》立於國學,其餘並多散亡,又無師說。



 《樂社大義》十卷梁武帝撰。



 《樂論》三卷梁武帝撰。梁有《樂義》十一卷,武帝集朝臣撰,亡。



 《樂論》一卷衛尉少卿蕭吉撰。



 《古今樂錄》十二卷陳沙門智匠撰。



 《樂書》七卷後魏丞相士曹行參軍信都芳撰。



 《樂雜書》三卷《樂元》一卷魏僧撰。



 《管弦記》十卷凌秀撰。



 《樂要》一卷何妥撰。



 《樂部》一卷《春官樂部》五卷梁有《宋元嘉正聲伎錄》一卷,張解撰,亡。



 《樂府聲調》六卷岐州刺史、沛國公鄭譯撰。



 《樂府聲調》三卷鄭譯撰。



 《樂經》四卷《琴操》三卷晉廣陵相孔衍撰。



 《琴操鈔》二卷《琴操鈔》一卷《琴譜》四卷戴氏撰。



 《琴經》一
 卷《琴說》一卷《琴歷頭簿》一卷《新雜漆調弦譜》一卷《樂譜》四卷《樂譜集》二十卷蕭吉撰。



 《樂略》四卷《樂律義》四卷沈重撰。



 《鐘律義》一卷《樂簿》十卷《齊朝曲簿》一卷《大隋總曲簿》一卷《推七音》二卷並尺法。



 《樂論事》一卷《樂事》一卷《正聲伎雜等曲簿》一卷《太常寺曲名》一卷《太常寺曲簿》十一卷《歌曲名》五卷《歷代樂名》一卷《鐘磬志》二卷公孫崇撰。



 《樂懸》一卷何晏等撰議。



 《樂懸圖》一卷《鐘律緯辯宗見》一卷《當管七聲》二卷魏僧撰。



 《黃鐘律》一卷梁有《鐘律緯》六卷,梁武帝撰,亡。



 右四十二部,一百四十二卷。通計亡書,合四十六部,二百六十三卷。



 樂者,先王所以致神祇,和邦國,諧萬姓,安賓客,悅遠人,所從來久矣。周人存六代之樂,曰《雲門》、《咸池》、《大韶》、《大夏》、《大護》、《大武》。



 其後衰微崩壞,及秦而頓滅。漢初,制氏雖紀其鏗鏘彭儛,而不能通其義。其後竇公、河間獻王、常山王、張禹,咸獻《樂書》。魏、晉已後,雖加損益,去正轉遠,事在《聲樂志》。今錄其見書,以補樂章之闕。



 《春秋經》十一卷吳衛將軍士燮注。



 《春秋左氏長經》二十卷漢侍中賈逵章句。



 《春秋左氏解詁》三十卷賈逵撰。



 《春秋左氏傳解誼》三十一卷漢九江太守服虔注。



 《春秋左氏傳》三十卷王肅注。



 《春秋左氏傳》三十卷董遇章句。



 《春秋左氏傳義注》十八卷孫毓注。



 《春秋左
 氏傳》十二卷魏司徒王朗撰。



 《春秋左氏經傳集解》三十卷杜預撰。



 《春秋杜氏》、《服氏注春秋左傳》十卷殘缺。



 《春秋左氏傳音》三卷魏中散大夫嵇康撰。梁有服虔、杜預音三卷,魏高貴鄉公《春秋左氏傳音》三卷,曹軀音、尚書左人郎荀訥等音四卷,亡。



 《春秋左氏傳音》三卷李軌撰。



 《春秋左氏傳音》三卷徐邈撰。



 《春秋釋訓》一卷賈逵撰。



 《春秋左氏經傳硃墨列》一卷賈逵撰。



 《春秋釋例》十卷漢公車徵士穎容撰。梁有《春秋左氏傳條例》九卷,漢大司農鄭眾撰。



 《春秋左氏膏肓釋痾》十卷服虔撰。梁有《春秋漢議駁》二卷,服虔撰,亡。



 《駁何氏漢議》二卷鄭玄撰。



 《春秋成長說》九卷服虔撰。梁有《春秋左氏達義》一卷,漢司徒掾王玢撰,亡。



 《春秋塞難》三卷服虔撰。梁有《春秋雜議難》五卷,漢少府孔融撰;《春秋左氏釋駁》一卷,王朗撰。亡。



 《春秋說要》十卷魏樂平太守糜信撰。



 《春秋釋例》十五卷杜
 預撰。梁有《春秋釋例引序》一卷,齊正員郎杜乾光撰,亡。



 《春秋左氏傳評》二卷杜預撰。



 《春秋條例》十一卷晉太尉劉寔撰。梁有《春秋公羊達義》三卷,劉寔撰,亡。



 《春秋經例》十二卷晉方範撰。梁有《春秋釋滯》十卷,晉尚書左丞殷興撰,《春秋釋難》三卷,晉護軍範堅撰。亡。



 《春秋左氏傳條例》二十五卷《春秋義例》十卷《春秋左傳例苑》十九卷梁有《春秋經傳說例疑隱》一卷,吳略撰;《春秋左氏分野》一卷;《春秋十二公名》一卷,鄭玄撰。亡。



 《春秋左氏經傳通解》四卷王述之撰。



 《春秋左氏傳賈、服異同略》五卷孫毓撰。



 《春秋左氏函傳義》十五卷干寶撰。



 《春秋左氏區別》三十卷尚書功論郎何始真撰。



 《春秋文苑》六卷《春秋叢林》十二卷《春秋義林》一卷《春秋大夫辭》三卷《春秋嘉語》六卷《春秋左氏諸大夫世譜》十三卷《春
 秋五辯》二卷梁五經博士沈宏撰。



 《春秋辯證》六卷《春秋旨通》十卷王述之撰。



 《春秋經傳解》六卷崔靈恩撰。



 《春秋申先儒傳論》十卷崔靈恩撰。



 《春秋左氏傳立義》十卷崔靈恩撰。



 劉寔等《集解春秋序》一卷《春秋序論》二卷干寶撰。



 《春秋序》一卷賀道養注。



 《春秋序》一卷崔靈恩撰。



 《春秋序一卷田元休注。



 《春秋左傳杜預序集解》一卷劉炫注。



 《春秋左氏經傳義略》二十五卷陳國子博士沈文阿撰。



 《王元規續沈文阿春秋左氏傳義略》十卷《春秋義略》三十卷陳右軍將軍張沖撰。



 《春秋左氏義略》八卷《春秋五十凡義疏》二卷《春秋左氏傳述義》四十卷東京太學博士劉炫撰。



 《春秋序義疏》一卷梁有《春秋發題》一卷,梁簡文帝撰;《春秋左氏圖》十卷,漢太子太傅嚴彭祖撰;《古
 今春秋盟會地圖》一卷。亡。



 《春秋公羊傳》十二卷嚴彭祖撰。



 《春秋公羊解詁》十一卷漢諫議大夫何休注。



 《春秋公羊經傳》十三卷晉散騎常侍王愆期注。梁有《春秋公羊傳》十二卷,晉河南太守高龍注;《春秋公羊傳》十四卷,孔衍集解;《春秋公羊音》,李軌、晉徵士江淳撰,各一卷。



 《春秋繁露》十七卷漢膠西相董仲舒撰。



 《春秋決事》十卷董仲舒撰。



 《春秋決疑論》一卷《春秋左氏膏盲》十卷何休撰。



 《春秋穀梁廢疾》三卷何休撰。



 《春秋漢議》十三卷何休撰。



 《駁何氏漢議》二卷鄭玄撰。梁有《漢議駁》二卷,服虔撰,亡。



 《駁何氏漢議敘》一卷《春秋公羊墨守》十四卷何休撰。



 《春秋公羊例序》五卷刁氏撰。



 《春秋公羊謚例》一卷何休撰。梁有《春秋公羊傳條例》一卷,何休撰;《春秋公羊傳問答》五卷,荀爽問,魏安平太守徐欽答;《春秋公羊論》二卷,晉車騎將軍庾翼問,王愆期答。亡。



 《春秋公羊解序》一卷
 鮮於公撰。



 《春秋公羊疏》十二卷《春秋穀梁傳》十三卷吳僕射唐固注。梁有《春秋穀梁傳》十五卷,漢諫議大夫尹更始撰,亡。



 《春秋穀梁傳》十二卷魏樂平太守糜信注。



 《穀梁傳》十卷晉堂邑太守張靖注。梁有《春秋穀梁傳》十三卷,晉給事郎徐乾注;《春秋穀梁傳》十卷,胡訥集解。亡。



 《春秋穀梁傳》十六卷程闡撰。



 《春秋穀梁傳》十四卷孔衍撰。



 《春秋穀梁傳》十二卷徐邈撰。



 《春秋穀梁傳》十四卷段肅注,疑漢人。



 《春秋穀梁傳》五卷孔君揩訓,殘缺。梁十四卷。



 《春秋穀梁傳》十二卷範甯集解。梁有《穀梁音》一卷,亡。



 《春秋穀梁傳》四卷殘缺,張、程、孫、劉四家集解。



 糜信《理何氏漢議》二卷魏人撰。



 《春秋穀梁傳義》十卷徐邈撰。



 《春秋議》十卷何休撰。



 徐邈《答春秋穀梁義》三卷薄叔玄《問穀梁義》二卷梁四卷。



 《春秋穀梁傳例》一卷範寧撰。



 《春秋公
 羊、穀梁傳》十二卷晉博士劉兆撰。



 《春秋穀梁廢疾》三卷何休撰,鄭玄釋,張靖箋。



 《春秋公羊、穀梁二傳評》三卷《春秋三家經本訓詁》十二卷賈逵撰。宋有《三家經》二卷,亡。



 《春秋三傳論》十卷魏大長秋韓益撰。



 《春秋經合三傳》十卷潘叔度撰。



 《春秋成奪》十卷潘叔度撰。



 《春秋三傳評》十卷胡訥撰。梁有《春秋集三師難》三卷,《春秋集三傳經解》十卷,胡訥撰。今亡。



 《春秋土地名》三卷晉裴秀客京相璠等撰。



 《春秋外傳國語》二十卷賈逵注。



 《春秋外傳國語》二十一卷虞翻注。



 《春秋外傳章句》一卷王肅撰。梁二十二卷。



 《春秋外傳國語》二十二卷韋昭注。



 《春秋外傳國語》二十卷晉五經博士孔晁注。



 《春秋外傳國語》二十一卷唐固注。梁有《春秋古今盟會地圖》一
 卷,亡。



 右九十七部,九百八十三卷。通計亡書,合一百三十部,一千一百九十二卷。



 《春秋》者,魯史策書之名。昔成周微弱,典章淪廢,魯以周公之故,遺制尚存。仲尼因其舊史,裁而正之,或婉而成章,以存大順,或直書其事,以示首惡。



 故有求名而亡,欲蓋而彰,亂臣賊子,於是大懼。其所褒貶,不可具書,皆口授弟子。弟子退而異說,左丘明恐失其真,乃為之傳。遭秦滅學,口說尚存。漢初,有公羊、穀梁、鄒氏、夾氏,四家並行。王莽之亂,鄒氏無師,夾氏亡。初,齊人胡母子都傳《公羊春秋》,授東海嬴公。嬴公授東海孟卿,孟卿授魯人眭
 孟,眭孟授東海嚴彭祖、魯人顏安樂。故後漢《公羊》有嚴氏、顏氏之學,與穀梁三家並立。



 漢末,何休又作《公羊解說》。而《左氏》漢初出於張蒼之家,本無傳者。至文帝時,梁太傅賈誼為訓詁,授趙人貫公。其後劉歆典校經籍,考而正之,欲立於學,諸儒莫應。至建武中,尚書令韓歆請立而未行。時陳元最明《左傳》,又上書訟之。



 於是乃以魏郡李封為《左氏》博士。後群儒蔽固者,數廷爭之。及封卒,遂罷。然諸儒傳《左氏》者甚眾。永平中,能為《左氏》者,擢高第為講郎。其後賈逵、服虔並為訓解。至魏,遂行於世。晉時,杜預又為《經傳集解》。《穀梁》範甯注、《公羊》何休注、《左氏》
 服虔、杜預注,俱立國學。然《公羊》、《穀梁》,但試讀文,而不能通其義。後學三傳通講,而《左氏》唯傳服義。至隋,杜氏盛行,服義及《公羊》、《穀梁》浸微,今殆無師說。



 《古文孝經》一卷孔安國傳。梁末亡逸,今疑非古本。



 《孝經》一卷鄭氏注。梁有馬融、鄭眾注孝經》二卷,亡。



 《孝經》一卷王肅解。梁有魏散騎常侍蘇林,吏部尚書何晏,光祿大夫劉邵、孫氏等注《孝經》各一卷,亡。



 《孝經解贊》一卷韋昭解。



 《孝經默注》一卷徐整注。



 《集解孝經》一卷謝萬集。



 《集議孝經》一卷晉中書郎荀昶撰,亡。



 《集議孝經》一卷晉東陽太守袁敬仲集。梁有《孝經皇義》一卷,宋均撰;又有晉給事中楊泓,處士虞槃佐、孫氏,東陽太守殷仲文,晉陵太守殷叔道,丹陽尹車胤,孔光各注《孝經》一卷;荀昶注《孝經》二卷;宋何承天、費沈,齊光祿大夫王玄載,國子博士明僧紹,梁五經博士嚴植之,尚書功論郎曹思文,羽林監江系之,江遜等注《孝經》各一卷;釋
 慧始注《孝經》一卷;陶弘景《集注孝經》一卷;諸葛循《孝經序》一卷。亡。



 《孝經》一卷釋慧琳注。梁有晉穆帝時《晉孝經》一卷,武帝時《送總明館孝經講》、《議》各一卷,宋大明中《東宮講》,齊永明三年《東宮講》,齊永明中《諸王講》及賀緌講、議《孝經義疏》各一卷,齊臨沂令李玉之為始興王講《孝經義疏》二卷,亡。



 《孝經義疏》十八卷梁武帝撰。梁有皇太子講《孝經義》三卷,天監八年皇太子講《孝經義》一卷,梁簡文《孝經義疏》五卷,蕭子顯《孝經義疏》一卷,亡。



 《孝經敬愛義》一卷梁吏部尚書蕭子顯撰。



 《孝經私記》四卷無名先生撰。



 《孝經義》一卷《孝經義疏》一卷趙景韶撰。



 《孝經義疏》三卷皇侃撰。



 《孝經私記》二卷周弘正撰。



 《古文孝經述義》五卷劉炫撰。



 《孝經講疏》六卷徐孝克撰。



 《孝經義》一卷梁揚州文學從事太史叔明撰。梁有《孝經玄》、《孝經圖》各一卷,《孝經孔子圖》二卷,亡。



 《國語孝經》一卷右十八部,合六十三卷。通計亡書,合五十九部,一百一十四卷。



 夫孝者,天之經,地之義,人之行。自天子達於庶人,雖尊卑有差,及乎行孝,其義一也。先王因之以治國家,化天下,故能不嚴而順,不肅而成。斯實生靈之至德,王者之要道。孔子既敘六經,題目不同,指意差別,恐斯道離散,故作《孝經》,以總會之,明其枝流雖分,本萌於孝者也。遭秦焚書,為河間人顏芝所藏。漢初,芝子貞出之,凡十八章,而長孫氏、博士江翁、少府後蒼、諫議大夫翼奉、安昌侯張禹,皆名其學。又有《古文孝經》,與《古文尚書》同出,而長孫有《閨門》一章,其餘經文,大較相似,篇簡缺解,又有衍出三章,並前合為二十二章,孔安國為之傳。至劉向
 典校經籍,以顏本比古文,除其繁惑,以十八章為定。鄭眾、馬融,並為之注。又有鄭氏注,相傳或云鄭玄,其立義與玄所注餘書不同,故疑之。梁代,安國及鄭氏二家,並立國學,而安國之本,亡於梁亂。陳及周、齊,唯傳鄭氏。至隋,秘書監王劭於京師訪得《孔傳》,送至河間劉炫。炫因序其得喪,述其議疏,講於人間,漸聞朝廷,後遂著令,與鄭氏並立。儒者喧喧,皆云炫自作之,非孔舊本,而秘府又先無其書。又云魏氏遷洛,未達華語,孝文帝命侯伏侯可悉陵,以夷言譯《孝經》之旨,教於國人,謂之《國語孝經》。今取以附此篇之末。



 《
 論語》十卷鄭玄注。梁有《古文論語》十卷,鄭玄注;又王肅、虞翻、譙周等注《論語》各十卷。亡。



 《論語》九卷鄭玄注,晉散騎常侍虞喜贊。



 《集解論語》十卷何晏集。



 《集注論語》六卷晉八卷,晉太保衛瓘注。梁有《論語補闕》二卷,宋明帝補衛瓘闕,亡。



 《論語集義》八卷晉尚書左中兵郎崔豹集。梁十卷。



 《論語》十卷晉著作郎李充注。



 《集解論語》十卷晉廷尉孫綽解。梁有盈氏及孟整注《論語》各十卷,亡。



 《集解論語》十卷晉兗州別駕江熙解。



 《論語》七卷盧氏注。梁有晉國子博士梁覬、益州刺史袁喬、尹毅、司徒左長史張憑及陽惠明、宋新安太守孔澄之、齊員外郎虞遐及許容、曹思文注,釋僧智略解,梁太史叔明集解,陶弘景集注《論語》各十卷;又《論語音》二卷,徐邈等撰。



 亡。



 《論語難鄭》一卷梁有《古論語義注譜》一卷,徐氏撰;《論語隱義注》三卷,《論語義注》三卷。亡。



 《論語難鄭》一卷《論語標指》一卷司馬氏撰。



 《論語雜問》一卷《論語孔子弟子目錄》一卷鄭玄撰。



 《論語體略》二卷晉太
 傅主簿郭象撰。



 《論語旨序》三卷晉衛尉繆播撰。



 《論語釋疑》三卷王弼撰。



 《論語釋》一卷張憑撰。



 《論語釋疑》十卷晉尚書郎欒肇撰。梁有《論語釋駁》三卷,王肅撰;《論語駁序》二卷,欒肇撰;《論語隱》一卷,郭象撰;《論語藏集解》一卷,應琛撰;《論語釋》一卷,曹毗撰;《論語君子無所爭》一卷,庾亮撰;《論語釋》一卷,李充撰;《論語釋》一卷,庾翼撰;《論語義》一卷,王濛撰;又蔡系《論語釋》一卷,張隱《論語釋》一卷,卻原《通鄭》一卷,王氏《修鄭錯》一卷,姜處道《論釋》一卷。亡。



 《論語別義》十卷範暠撰。梁有《論語疏》八卷,宋司空法曹張略等撰;《新書對張論》十卷,虞喜撰。



 《論語義疏》十卷褚仲都撰。



 《論語義疏》十卷皇侃撰。



 《論語述義》十卷劉炫撰。



 《論語義疏》八卷《論語講疏文句義》五卷徐孝克撰,殘缺。



 《論語義疏》二卷張沖撰。梁有《論語義注圖》十二卷,亡。



 《孔叢》七卷陳勝博士孔鮒撰。梁有《孔志》十卷,梁太尉參軍劉被撰,亡。



 《孔子家語》二十一卷王肅解。梁有《當家語》二卷,魏博士張融撰,亡。



 《孔子正言》
 二十卷梁武帝撰。



 《爾雅》三卷漢中散大夫樊光注。梁有漢劉歆,犍為文學、中黃門李巡《爾雅》各三卷,亡。



 《爾雅》七卷孫炎注。



 《爾雅》五卷郭璞注。



 《集注爾雅》十卷梁黃門郎沈〔注。



 《爾雅音》八卷秘書學士江水崔撰。梁有《爾雅音》二卷,孫炎、郭璞撰。



 《爾雅圖》十卷郭璞撰。梁有《爾雅圖贊》二卷,郭璞撰,亡。



 《廣雅》三卷魏博士張揖撰。梁有四卷。



 《廣雅音》四卷秘書學士曹憲撰。



 《小爾雅》一卷李軌略解。



 《方言》十三卷漢揚雄撰,郭璞注。



 《釋名》八卷劉熙撰。



 《辯釋名》一卷韋昭撰。



 《五經音》十卷徐邈撰。



 《五經正名》十二卷劉炫撰。



 《白虎通》六卷《五經異義》十卷後漢太尉祭酒許慎撰。



 《五經然否論》五卷晉散騎常侍譙周撰。



 《五經拘沈》十卷晉高涼太守楊方撰。



 《五經大義》三卷戴逵撰。梁有《通五經》五卷,王氏撰;《五經咨疑》八卷,周楊撰;《五經異同評》一卷,賀緌撰;《五經秘表要》三卷。亡。



 《五經大義》十卷後周縣伯中大夫樊
 文深撰。



 《經典大義》十二卷沈文阿撰。



 《五經大義》五卷何妥撰。



 《五經通義》八卷梁九卷。



 《五經義》六卷梁七卷。梁又有《五經義略》一卷,亡。



 《五經要義》五卷梁十七卷,雷氏撰。



 《五經析疑》二十八卷邯鄲綽撰。



 《五經宗略》二十三卷元延明撰。



 《五經雜義》六卷孫暢之撰。



 《長春義記》一百卷梁簡文帝撰。



 《大義》九卷《游玄桂林》九卷張譏撰。



 《六經通數》十卷梁舍人鮑泉撰。



 《七經義綱》二十九卷樊文深撰。



 《七經論》三卷樊文深撰。



 《質疑》五卷樊文深撰。



 《經典玄儒大義序錄》二卷沈文阿撰。



 《玄義問答》二卷《六藝論》一卷鄭玄撰。



 《聖證論》十二卷王肅撰。



 《鄭志》十一卷魏侍中鄭小同撰。



 《鄭記》六卷鄭玄弟子撰。



 《謚法》三卷劉熙撰。



 《謚法》十卷特進、中軍將軍沈約撰。



 《謚法》五卷梁太府卿賀緌撰。



 《江都集禮》一百
 二十六卷右七十三部,七百八十一卷。通計亡書,合一百一十六部,一千二十七卷。



 《論語》者,孔子弟子所錄。孔子既敘六經,講於洙、泗之上,門徒三千,達者七十。其與夫子應答,及私相講肄,言合於道,或書之於紳,或事之無厭。仲尼既沒,遂緝而論之,謂之《論語》。漢初,有齊、魯之說。其齊人傳者二十二篇,魯人傳者二十篇。齊則昌邑中尉王吉、少府宗畸、御史大夫貢禹、尚書令五鹿充宗、膠東庸生。魯則常山都尉龔奮、長信少府夏侯勝、韋丞相節侯父子、魯扶卿、前將軍
 蕭望之、安昌侯張禹,並名其學。張禹本授《魯論》,晚講《齊論》,後遂合而考之,刪其煩惑。除去《齊論·問王》、《知道》二篇,從《魯論》二十篇為定,號《張侯論》,當世重之。周氏、包氏為之章句,馬融又為之訓。又有古《論語》,與《古文尚書》同出,章句煩省,與《魯論》不異,唯分《子張》為二篇,故有二十一篇。孔安國為之傳。漢末,鄭玄以《張侯論》為本,參考《齊論》、古《論》而為之注。魏司空陳群、太常王肅、博士周生烈,皆為義說。吏部尚書何晏又為集解。是後諸儒多為之注,《齊論》遂亡。古《論》先無師說,梁、陳之時,唯鄭玄、何晏立於國學,而鄭氏甚微。周、齊,鄭學獨立。至隋,何、鄭並行,鄭氏
 盛於人間。其《孔叢》、《家語》,並孔氏所傳仲尼之旨。《爾雅》諸書,解古今之意,並五經總義,附於此篇。



 《河圖》二十卷梁《河圖洛書》二十四卷,目錄一卷,亡。



 《河圖龍文》一卷《易緯》八卷鄭玄注。梁有九卷。



 《尚書緯》三卷鄭玄注,梁六卷。



 《尚書中候》五卷鄭玄注。梁有八卷,今殘缺。



 《詩緯》十八卷魏博士宋均注。梁十卷。



 《禮緯》三卷鄭玄注,亡。



 《禮記默房》二卷宋均注。梁有三卷,鄭玄注,亡。



 《樂緯》三卷宋均注。梁有《樂五鳥圖》一卷。亡。



 《春秋災異》十五卷郗萌撰。梁有《春秋緯》三十卷,宋均注;《春秋內事》四卷,《春秋包命》二卷,《春秋秘事》十一卷,《書、易、詩、孝經、春秋、河洛緯秘要》一卷,《五帝鉤命訣圖》一卷。亡。



 《孝經勾命訣》六卷宋均注。



 《孝經援神契》七卷宋均注。



 《孝經內事》一卷梁有《孝經雜緯》十卷,宋均注;《孝經元命包》一卷,《孝經古秘援神》二卷,《孝經古秘圖》一卷,《孝經左右握》二卷,《孝經左右契圖》一卷,《孝經
 雌雄圖》三卷,《孝經異本雌雄圖》二卷,《孝經分野圖》一卷,《孝經內事圖》二卷,《孝經內事星宿講堂七十二弟子圖》一卷,又《口授圖》一卷;又《論語讖》八卷,宋均注;《孔老讖》十二卷,《老子河洛讖》一卷,《尹公讖》四卷,《劉向讖》一卷,《雜讖書》二十九卷,《堯戒舜、禹》一卷,《孔子王明鏡》一卷,《郭文金雄記》一卷,《王子年歌》一卷,《嵩高道士歌》一卷。亡。



 右十三部,合九十二卷。通計亡書,合三十二部,共二百三十二卷。



 《易》曰:「河出圖,洛出書。」然則聖人之受命也,必因積德累業,豐功厚利,誠著天地,澤被生人,萬物之所歸往,神明之所福饗,則有天命之應。蓋龜龍銜負,出於河、洛,以紀易代之徵,其理幽昧,究極神道。先王恐其惑人,秘而不傳。說者又云,孔子既敘六經,以明天人之道,知後世不能稽同其意,故別立緯及讖,以遺來世。其書出於前漢,
 有《河圖》九篇,《洛書》六篇,雲自黃帝至周文王所受本文。又別有三十篇,雲自初起至於孔子,九聖之所增演,以廣其意。又有《七經緯》三十六篇,並云孔子所作,並前合為八十一篇。而又有《尚書中候》、《洛罪級》、《五行傳》、《詩推度災》、《氾歷樞》、《含神務》、《孝經勾命訣》、《援神契》、《雜讖》等書。漢代有郗氏、袁氏說。漢末,郎中郗萌集圖緯讖雜占為五十篇,謂之《春秋災異》。宋均、鄭玄並為讖律之注。然其文辭淺俗,顛倒舛謬,不類聖人之旨。相傳疑世人造為之後,或者又加點竄,非其實錄。起王莽好符命,光武以圖讖興,遂盛行於世。漢時,又詔東平王蒼正五經章句,皆命
 從讖。俗儒趨時,益為其學,篇卷第目,轉加增廣。言五經者,皆憑讖為說。唯孔安國、毛公、王璜、賈逵之徒獨非之,相承以為妖妄,亂中庸之典。故因漢魯恭王、河間獻王所得古文,參而考之,以成其義,謂之「古學」。當世之儒,又非毀之,竟不得行。魏代王肅,推引古學,以難其義。王弼、杜預,從而明之,自是古學稍立。至宋大明中,始禁圖讖,梁天監已後,又重其制。及高祖受禪,禁之逾切。煬帝即位,乃發使四出,搜天下書籍與讖緯相涉者,皆焚之,為吏所糾者至死。自是無復其學,秘府之內,亦多散亡。今錄其見存,列於六經之下,以備異說。



 《
 三蒼》三卷郭璞注。秦相李斯作《蒼頡篇》,漢揚雄作《訓纂篇》,後漢郎中賈魴作《滂喜篇》,故曰《三蒼》。梁有《蒼頡》二卷,後漢司空杜林注,亡。



 《埤蒼》三卷張揖撰。梁有《廣蒼》一卷,樊恭撰,亡。



 《急就章》一卷漢黃門令史游撰。



 《急就章》二卷崔浩撰。



 《急就章》三卷豆盧氏撰。



 《吳章》二卷陸機撰。



 《小學篇》一卷晉下邳內史王義撰。



 《少學》九卷楊方撰。



 《始學》一卷《勸學》一卷蔡邕撰。有司馬相如《凡將篇》,班固《太甲篇》、《在昔篇》,崔瑗《飛龍篇》,蔡邕《聖皇篇》、《黃初篇》、《吳章篇》,蔡邕《女史篇》,合八卷,又《幼學》二卷,硃育撰;《始學》十二卷,吳郎中項峻撰;又《月儀》十二卷。亡。



 《發蒙記》一卷晉著作郎束皙撰。



 《啟蒙記》三卷晉散騎常侍顧愷之撰。



 《啟疑記》三卷顧愷之撰。



 《千字文》一卷梁給事郎周興嗣撰。



 《千字文》一卷梁國子祭酒蕭子云注。



 《千字文》一卷胡肅注。



 《篆書千字文》一卷《演千字文》五卷《草書千字文》一卷《古今字詁》三卷張揖撰。
 梁有《難字》一卷,《錯誤字》一卷,並張揖撰;《異字》二卷,硃育撰;《字屬》一卷,賈魴撰。亡。



 《雜字解詁》四卷魏掖庭右丞周氏撰。梁有《解文字》七卷,周成撰;《字義訓音》六卷,《古今字苑》十卷,曹侯彥撰。亡。



 《雜字指》一卷後漢太子中庶子郭顯卿撰。



 《字指》二卷晉朝議大夫李彤撰。梁有《單行字》四卷,李彤撰;又《字偶》五卷。亡。



 《說文》十五卷許慎撰。梁有《演說文》一卷,庾儼默注,亡。



 《說文音隱》四卷《字林》七卷晉弦令呂忱撰。



 《字林音義》五卷宋揚州督護吳恭撰。



 《古今字書》十卷《字書》三卷《字書》十卷《字統》二十一卷陽承慶撰。



 《玉篇》三十一卷陳左衛將軍顧野王撰。



 《字類敘評》三卷侯洪伯撰。



 《要字苑》一卷宋豫章太守謝康樂撰。梁有《常用字訓》一卷,殷仲堪撰;《要用字對誤》四卷,梁輕車參軍鄒誕生撰,亡。



 《要用雜字》三卷鄒里撰。梁有《文字要記》三卷,王義撰,亡。



 《俗語難字》一卷秘書少監王劭撰。



 《雜字要》三卷密州行參軍李少通撰。



 《文字整疑》一卷《
 正名》一卷《文字集略》六卷梁文貞處士阮孝緒撰。



 《今字辯疑》三卷李少通撰。



 《異字同音》一卷梁有《釋字同音》三卷,宋散騎常侍吉文甫撰。



 《字宗》三卷薛立撰。



 《文字譜》一卷梁有《古今文字序》一卷,劉歊撰;《文字統略》一卷,焦子明撰。亡。



 《文字辯嫌》一卷彭立撰。



 《辯字》一卷戴規撰。



 《雜字音》一卷《借音字》一卷《音書考源》一卷《聲韻》四十一卷周研撰。



 《聲類》十卷魏左校令李登撰。



 《韻集》十卷《韻集》六卷晉安復令呂靜撰。



 《四聲韻林》二十八卷張諒撰。



 《韻集》八卷段弘撰。



 《群玉典韻》五卷梁有《文章音韻》二卷,王該撰;又《五音韻》五卷。亡。



 《韻略》一卷陽休之撰。



 《修續音韻訣疑》十四卷李概撰。



 《纂韻鈔》十卷《四聲指歸》一卷劉善經撰。



 《四聲》一卷梁太子少傅沈約撰。



 《四聲韻略》十三卷夏侯詠撰。



 《音譜》四卷李概撰。



 《韻
 英》三卷釋靜洪撰。



 《通俗文》一卷服虔撰。



 《訓俗文字略》一卷後齊黃門郎顏之推撰。



 《證俗音字略》六卷梁有《詁幼》二卷,顏延之撰;《廣詁幼》一卷,宋給事中荀楷撰。亡。



 《文字音》七卷晉蕩昌長王延撰。梁有《纂文》三卷,亡。



 《翻真語》一卷王延撰。



 《真言鑒誡》一卷《字書音同異》一卷《敘同音義》三卷《河洛語音》一卷王長孫撰。



 《國語》十五卷《國語》十卷《鮮卑語》五卷《國語物名》四卷後魏侯伏侯可悉陵撰。



 《國語真歌》十卷《國語雜物名》三卷侯伏侯可悉陵撰。



 《國語十八傳》一卷《國語御歌》十一卷《鮮卑語》十卷《國語號令》四卷《國語雜文》十五卷《鮮卑號令》一卷周武帝撰。



 《雜號令》一卷《古文官書》一卷後漢議郎衛敬仲撰。



 《古今奇字》一卷郭顯卿撰。



 《六文書》
 一卷《四體書勢》一卷晉長水校尉衛恆撰。



 《雜體書》九卷釋正度撰。



 《古今八體六文書法》一卷《古今篆隸雜字體》一卷蕭子政撰。



 《古今文等書》一卷《篆隸雜體書》二卷《文字圖》二卷《古今字圖雜錄》一卷秘書學士曹憲撰。



 《婆羅門書》一卷梁有《扶南胡書》一卷。



 《外國書》四卷《秦皇東巡會稽刻石文》一卷《一字石經周易》一卷梁有三卷。



 《一字石經尚書》六卷梁有《今字石經鄭氏尚書》八卷,亡《一字石經魯詩》六卷梁有《毛詩》二卷,亡。



 《一字石經儀禮》九卷《一字石經春秋》一卷梁有一卷。



 《一字石經公羊傳》九卷《一字石經論語》一卷梁有二卷。



 《一字石經典論》一卷《三字石經尚書》九卷梁有十三卷。



 《三字石經尚書》五卷《三
 字石經春秋》三卷梁有十二卷。



 右一百八部,四百四十七卷。通計亡書,合一百三十五部,五百六十九卷。



 孔子曰:「必也正名乎!」名謂書字。「名不正則言不順,言不順則事不成。」



 說者以為書之所起,起自黃帝蒼頡。比類象形謂之文,形聲相益謂之字,著於竹帛謂之書。故有象形、諧聲、會意、轉注、假借、處事六義之別。古者童子示而不誑,六年教之數與方名。十歲入小學,學書計。二十而冠,始習先王之道,故能成其德而任事。然自蒼頡訖於漢初,書經五變:一曰古文,即蒼頡所作。二曰大篆,周
 宣王時史籀所作。三曰小篆,秦時李斯所作。四曰隸書,程邈所作。五曰草書,漢初作。秦世既廢古文,始用八體,有大篆、小篆、刻符、摹印、蟲書、署書、殳書、隸書。漢時以六體教學童,有古文、奇字、篆書、隸書、繆篆、蟲鳥,並槁書、楷書、懸針、垂露、飛白等二十餘種之勢,皆出於上六書,因事生變也。魏世又有八分書,其字義訓讀,有《史籀篇》、《蒼頡篇》、《三蒼》、《埤蒼》、《廣蒼》等諸篇章,訓詁、《說文》《字林》、音義、聲韻、體勢等諸書。自後漢佛法行於中國,又得西域胡書,能以十四字貫一切音,文省而義廣,謂之婆羅門書,與八體六文之義殊別,今取以附體勢之下。又後魏初定
 中原,軍容號令,皆經夷語,後染華俗,多不能通,故錄其本言,相傳教習,謂之「國語」,今取以附音韻之末。又後漢鐫刻七經,著於石碑,皆蔡邕所書。魏正始中,又立三字石經,相承以為七經正字。後魏之末,齊神武執政,自洛陽徙於鄴都,行至河陽,值岸崩,遂沒於水。



 其得至鄴者,不盈太半。至隋開皇六年,又自鄴京載入長安,置於秘書內省,議欲補緝,立於國學。尋屬隋亂,事遂寢廢,營造之司,因用為柱礎。貞觀初,秘書監臣魏徵,始收聚之,十不存一。其相承傳拓之本,猶在秘府,並秦帝刻石,附於此篇,以備小學。



 凡六藝經緯六百二十七部,五千三百七十一卷。通計亡書,合九百五十部,七千二百九十卷。



 《傳》曰:「玉不琢,不成器;人不學,不知道。」古之君子,多識而不窮,畜疑以待問:學不逾等,教不陵節;言約而易曉,師逸而功倍;且耕且養,三年而成一藝。自孔子沒而微言絕,七十子喪而大義乖,學者離群索居,各為異說。至於戰國,典文遺棄,六經之儒,不能究其宗旨,多立小數,一經至數百萬言。致令學者難曉,虛誦問答,脣腐齒落而不知益。且先王設教,以防人欲,必本於人事,折之中道。上天之命,略而罕言,方外之理,固所未說。至後漢好圖
 讖,晉世重玄言,穿鑿妄作,日以滋生。先王正典,雜之以妖妄,大雅之論,汨之以放誕。陵夷至於近代,去正轉疏,無復師資之法。學不心解,專以浮華相尚,豫造雜難,擬為仇對,遂有芟角、反對、互從等諸翻競之說。馳騁煩言,以紊彞敘,嘵嘵成俗,而不知變,此學者之蔽也。班固列六藝為九種,或以緯書解經,合為十種。



\end{pinyinscope}