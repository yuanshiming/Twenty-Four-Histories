\article{卷二十三志第十八 五行下}

\begin{pinyinscope}

 《洪範五行傳》曰:「視之不明,是謂不知。厥咎舒,厥罰常燠,厥極疾。時則有草妖,時則有羽蟲之孽。故有羊禍,故有目疾,有赤眚赤祥。惟水沴火。」



 常燠後齊天保八年三月,大熱,人或曷死。劉向《五行傳》曰:「視不明,用近習,賢者不進,不肖不退,百職廢壞,庶事不從,
 其過在政教舒緩。」時帝狂躁、荒淫無度之應。



 草妖高祖時,上黨有人,宅後每夜有人呼聲,求之不得。去宅一里所,但見人參一本,枝葉峻茂。因掘去之,其根五尺餘,具體人狀,呼聲遂絕。蓋草妖也。視不明之咎。時晉王陰有奪宗之計,諂事親要,以求聲譽譖皇太子,高祖惑之。人參不當言,有物憑之。上黨,黨與也。親要之人,乃黨晉王而譖太子。高祖不悟,聽邪言,廢無辜,有罪用,因此而亂也。



 羽蟲之孽
 梁中大同元年,邵陵王綸在南徐州,坐事。有野鳥如鳶數百,飛屋梁上,彈射不中。俄頃失所在。京房《易飛候》曰:「野鳥入君室,其邑虛,君亡之他方。」



 後綸為湘東王所襲,竟致奔亡,為西魏所殺。



 侯景在梁,將受錫命,陳備物於庭。有野鳥如山鵲,赤嘴,集於冊書之上,鵂鶹鳥鳴於殿。與中大同元年同占。景尋敗,將亡入海中,為羊鵾所殺。



 陳後主時,蔣山有眾鳥,鼓翼而鳴曰:「奈何帝。」京房《易飛候》曰:「鳥鳴門闕,如人音,邑且亡。」蔣山,吳之望也。鳥於上鳴,吳空虛之象。及陳亡,建康為墟。又陳未亡時,有一足
 鳥集於殿庭,以嘴畫地成文,曰:「獨足上高臺,盛草變成灰。」獨足者,叔寶獨行無眾之應。成草成灰者,陳政無穢,被隋火德所焚除也。叔寶至長安,館於都水臺上,高臺之義也。



 後齊孝昭帝即位之後,有雉飛上御座。占同中大同元年。又有鳥止於後園,其色赤,形似鴨而有九頭。其年帝崩。



 天統三年九月,萬春鳥集仙都苑。京房《易飛候》曰:「非常之鳥,來宿於邑中,邑有兵。」周師入鄴之應也。



 武成胡後生後主初,有梟升後帳而鳴。梟不孝之鳥,不
 祥之應也。後主嗣位,胡後淫亂事彰,遂幽后於北宮焉。



 武平七年,有鸛巢太極殿,又巢並州嘉陽殿。雉集晉陽宮御座,獲之。京房《易飛候》曰:「鳥無故巢居君門及殿屋上,邑且虛。」其年國滅。



 周大象二年二月,有禿鶖集洛陽宮太極殿。其年帝崩,後宮常虛。



 開皇初,梁主蕭琮新起後,有鵂鳥集其帳隅。未幾,琮入朝,被留於長安。梁國遂廢。



 大業末,京師宮室中,恆有鴻雁之類無數,翔集其間。俄而長安不守。



 十三年十一月,烏鵲巢帝帳幄,驅不能止。帝尋逢弒。



 羊禍開皇十二年六月,繁昌楊悅見雲中二物,如羝羊,黃色,大如新生犬,鬥而墜。



 悅獲其一,數旬失所在。近羊禍也。《洪範五行傳》曰:「君不明,逆火政之所致也。」狀如新生犬者,羔類也。云體掩蔽,邪佞之象。羊,國姓。羔,羊子也。皇太子勇既升儲貳,晉王陰毀而被廢黜。二羔斗,一羔墜之應也。



 恭帝義寧二年,麟游太守司馬武,獻羊羔,生而無尾。時議者以為楊氏子孫無後之象。是歲,煬帝被殺於江都,
 恭帝遜位。



 赤眚赤祥梁天監十五年七月,荊州市殺人而身不殭,首墮於地,動口張目,血如竹箭,直上丈餘,然後如雨細下。是歲荊州大旱。近赤祥,冤氣之應。



 陳太建十四年三月,御座幄上見一物,如車輪,色正赤。尋而帝患,無故大叫數聲而崩。



 至德三年十二月,有赤物隕於太極殿前,初下時,鐘皆鳴。又嘗進白飲,忽變為血。又有血沾殿階,瀝瀝然至御榻。尋而國滅。



 後齊河清二年,太原雨血。劉向曰:「血者陰之精,傷害之象,殭尸之類也。」



 明年,周師與突厥入並州,大戰城西,伏尸百餘里。京房《易飛候》曰:「天雨血染衣,國亡君戮。」亦後主亡國之應。



 四年三月,有物隕於殿庭,色赤,形如數斗器,眾星隨者如小鈴。四月,婁太后崩。



 武平中,有血點地,自咸陽王斛律明月宅而至於太廟。大將,社稷之臣也,後主以讒言殺之。天戒若曰,殺明月,則宗廟隨而覆矣。後主不悟,國祚竟絕。



 《洪範五行傳》曰:「聽之不聰,是謂不謀。厥咎急,厥罰寒,厥
 極貧。時則有鼓妖,有魚孽,有彘禍,有黑眚黑祥,惟火沴水。」



 寒東魏武定四年二月,大寒。人畜凍死者,相望於道。京房《易飛候》曰:「誅過深,當燠而寒。」是時後齊神武作相。先是爾硃文暢等謀害神武,事洩伏誅,諸與交通者,多有濫死。



 河清元年,歲大寒。京房《易傳》曰:「有德遭險,茲謂逆命。厥異寒。」曰:「殺無罪,其寒必異。」是時帝淫於文宣李後,因生子,後愧恨,不舉之。帝大怒,於後前殺其子太原王紹
 德。後大哭,帝惈後而撻殺之,投於水中,良久乃蘇。



 冤酷之應。



 梁天監三年三月,六年三月,並隕霜殺草。京房《易傳》曰:「興兵妄誅,謂亡法,厥罰霜。」是時,大發卒,拒魏軍於鐘離,連兵數歲。



 大同三年六月,朐山隕霜。



 陳太建十年八月,隕霜,殺稻菽。是時,大興師選眾,遣將吳明徹,與周師相拒於呂梁。



 鼓妖梁天監四年十一月,天清朗,西南有電光,有雷聲二。《易》
 曰:「鼓之以雷霆。」霆近鼓妖。



 《洪範五行傳》曰:「雷霆托於雲,猶君之托於人也。君不恤於天下,故兆人有怨叛之心也。」是歲,交州刺史李凱舉兵反。



 十九年九月,西北隱隱有聲如雷,赤氣下至地。是歲,盜殺東莞、瑯邪二郡守,以朐山引魏軍。



 中大通六年十二月,西南有聲如雷。其年北梁州刺史蘭欽舉兵反。



 陳太建二年十二月,西北有聲如雷。其年湘州刺史華皎舉兵反。



 齊天保四年四月,西南有聲如雷。是時,帝不恤天下,興
 師旅。



 後周建德六年正月,西方有聲如雷。未幾,吐谷渾寇邊。



 開皇十四年正月旦,廓州連雲山有聲如雷。是時五羌反叛,侵擾邊鎮。二十年,無雲而雷。京房《易飛候》曰:「國將易君,下人不靜,小人先命。國兇,有兵甲。」



 後數歲,帝崩,漢王諒舉兵反,徙其黨數十萬家。



 大業中,滏陽石鼓頻歲鳴。其後,天下大亂,兵戎並起。



 魚孽梁大同十年三月,帝幸硃方,至四塹中,及玄武湖,魚皆
 驤首見於上,若望乘輿者。帝入宮而沒。《洪範五行傳》曰:「魚陰類也,下人象。又有鱗甲,兵之應也。」下人將舉兵圍宮,而闢睨乘輿之象也。後果有侯景之亂。



 齊後主武平七年,相州鸕鶿泊,魚盡飛去而水涸。《洪範五行傳》曰:「急之所致。魚陰類,下人象也。」晏子曰:「河伯以水為國,以魚為百姓。」水涸魚飛,國亡人散之象。明年而國亡。



 後周大象元年六月,陽武有鯉魚乘空而鬥。猶臣下興起,小人縱之而鬥也。明年帝崩,國失政。尉迥起兵相州,高祖遣兵擊敗之。



 開皇十七年,大興城西南四里,有袁村,設佛會。有老翁,皓首,白裙襦衣,來食而去。眾莫識,追而觀之,行二里許,不復見。但有一陂,中有白魚,長丈餘,小魚從者無數。人爭射之,或弓折弦斷。後竟中之,剖其腹,得粳飯,始知此魚向老翁也。後數日,漕渠暴溢,射人皆溺死。



 大業十二年,淮陽郡驅人入子城,鑿斷羅郎郭。至女垣之下,有穴,其中得鯉魚,長七尺餘。昔魏嘉平四年,魚集武庫屋上。王肅以為魚生於水,而亢於屋,水之物失其所也,邊將殆棄甲之變。後果有東闕之敗。是時,長白山賊寇掠河南,月餘,賊至城下。郡兵拒之,反為所敗,男女
 死者萬餘人。



 蟲妖梁大同初,大蝗,籬門松柏葉皆盡。《洪範五行傳》曰:「介蟲之孽也。」與魚同占。京房《易飛候》曰:「食祿不益聖化,天視以蟲。蟲無益於人而食萬物也。」



 是時公卿皆以虛淡為美,不親職事,無益食物之應也。



 後齊天保八年,河北六州、河南十二州蝗。畿人皆祭之。帝問魏尹丞崔叔瓚曰:「何故蟲?」叔瓚對曰:「《五行志》云:『土功不時則蝗蟲為災。』今外築長城,內修三臺,故致災也。」帝大怒,毆其頰,擢其發,溷中物塗其頭。役者不止。九年,
 山東又蝗,十年,幽州大蝗。《洪範五行傳》曰:「刑罰暴虐,食貪不厭,興師動眾,取城修邑,而失眾心,則蟲為災。」是時帝用刑暴虐,勞役不止之應也。



 後周建德二年,關中大蝗。



 開皇十六年,並州蝗。時秦孝王俊裒刻百姓,盛修邸第。後竟獲譴而死。



 彘禍開皇末,渭南有沙門三人,行頭陀法於人場圃之上。夜見大豕來詣其所,小豕從者十餘,謂沙門曰:「阿練,我欲得賢聖道,然猶負他一命。」言罷而去。賢聖道者,君上之
 所行也。皇太子勇當嗣業,行君上之道,而被囚廢之象也。一命者,言為煬帝所殺。



 開皇末,渭南有人寄宿他舍,夜中聞二豕對語。其一曰:「歲將盡,阿耶明日殺我供歲,何處避之?」一答曰:「可向水北姊家。」因相隨而去。天將曉,主人覓豕不得,意是宿客而詰之。宿客言狀,主人如其言而得豕。其後蜀王秀得罪,帝將殺之,樂平公主每匡救,得全。後數年而帝崩,歲盡之應。



 黑眚黑祥梁承聖三年六月,有黑氣如龍,見於殿內。近黑祥也。黑,
 周所尚之色。今見於殿內,周師入梁之象。其年,為周所滅,帝亦遇害。



 陳太建五年六月,西北有黑雲屬地,散如豬者十餘。《洪範五行傳》曰:「當有兵起西北。」時後周將王軌軍於呂梁。明年,擒吳明徹,軍皆覆沒。



 火沴水後齊河清元年四月,河、濟清。襄楷曰:「河,諸侯之象。應濁反清,諸侯將為天子之象。」是後十餘歲,隋有天下。



 大業三年,武陽郡河清,數里鏡澈。十二年,龍門又河清。後二歲,大唐受禪。



 陳太建十四年七月,江水赤如血,自建康西至荊州。禎明中,江水赤,自方州東至海。《洪範五行傳》曰:「火沴水也。法嚴刑酷,傷水性也。五行變節,陰陽相干,氣色繆亂,皆敗亂之象也。」京房《易占》曰:「水化為血,兵且起。」是時後主初即位,用刑酷暴之應。其後為隋師所滅。



 禎明二年四月,郢州南浦水,黑如墨。黑水在關中,而今淮南水黑,荊、揚州之地,陷於關中之應。



 後周大象元年六月,咸陽池水變為血。與陳太建十四年同占。是時,刑罰嚴急,未幾國亡。



 《洪範五行傳》曰:「思心不容,是謂不聖。厥咎瞀,厥罰常風,
 厥極兇短折。



 有脂夜之妖,有華孽,有牛禍,有心腹之痾,有黃眚黃祥,木金水火沴土。」



 常風梁天監六年八月戊戌,大風折木。京房《易飛候》曰:「角日疾風,天下昏。



 不出三月中,兵必起。」是歲魏軍入鐘離。



 承聖三年十一月癸未,帝閱武於南城,北風大急,普天昏暗。《洪範五行傳》曰:「人君瞀亂之應。」時帝既平侯景,公卿咸勸帝反丹陽,帝不從,又多猜忌,有瞀亂之行,故天變應之以風。是歲為西魏滅。



 陳天嘉六年七月癸未,大風起西南,吹倒靈臺候樓。《洪
 範五行傳》,以為大臣專恣之咎。時太子沖幼,安成王頊專政,帝不時抑損。明年崩,皇太子嗣位,頊遂廢之。



 太建十二年六月壬戌,大風吹壞皋門中闥。十二年九月,夜又風,發屋拔樹。



 始興王叔陵專恣之應。



 至德中,大風吹倒硃雀門。



 禎明三年六月丁巳,大風,自西北,激濤水入石頭、淮。是時,後主任司馬申,誅戮忠諫。沈客卿、施文慶專行邪僻。江總、孔範等崇長淫縱,杜塞聰明,瞀亂之咎。



 後齊河清二年,大風,三旬乃止。時帝初委政佞臣和士開,專恣日甚。
 天統三年五月,大風,晝晦,發屋拔樹。天變再見,而帝不悟。明年帝崩。後主詔內外表奏,皆先詣士開,然後聞徹。趙郡王睿、馮翊王潤按士開驕恣,不宜仍居內職,反為士開所譖,睿竟坐死。士開出入宮掖,生殺在口,尋為瑯邪王儼所誅。



 七年三月,大風起西北,發屋拔樹。五日乃止。時高阿那瑰、駱提婆等專恣之應。



 開皇二十年十一月,京都大風,發屋拔樹,秦、隴壓死者千餘人。地大震,鼓皆應。凈剎寺鐘三鳴,佛殿門鎖自開,銅像自出戶外。鐘鼓自鳴者,近鼓妖也。揚雄以為人君
 不聰,為眾所惑,空名得進,則鼓妖見。時獨孤皇后干預政事,左僕射楊素權傾人主。帝聽二人之讒,而黜僕射高熲,廢太子勇為庶人,晉王釣虛名而見立。思心瞀亂,陰氣盛之象也。鎖及銅像,並金也。金動木震之,水沴金之應。



 《洪範五行傳》曰:「失眾心甚之所致也。」高熲、楊勇無罪而咸廢黜,失眾心也。



 仁壽二年,西河有胡人,乘騾在道,忽為回風所飄,並一車上千餘尺,乃墜,皆碎焉。京房《易傳》曰:「眾逆同志,至德乃潛,厥異風。」後二載,漢王諒在並州,潛謀逆亂,車及騾騎之象也。升空而墜,顛隕之應也。天戒若曰,無妄動車
 騎,終當覆敗,而諒不悟。及高祖崩,諒發兵反,州縣響應,眾至數十萬。月餘而敗。



 夜妖梁承聖二年十月丁卯,大風,晝晦,天地昏暗。近夜妖也。京房《易飛候》曰:「羽日風,天下昏,人大疾。不然,多寇盜。」三年為西魏所滅。



 陳禎明三年正月朔旦,雲霧晦冥,入鼻辛酸。後主昏昧,近夜妖也。《洪範五行傳》曰:「王失中,臣下強盛,以蔽君明,則雲陰。」是時北軍臨江,柳莊、任蠻奴並進中款,後主惑佞臣孔範之言,而昏暗不能用,以至覆敗。



 東魏武定四年冬,大霧六日,晝夜不解。《洪範五行傳》曰:「晝而晦冥若夜者,陰侵陽,臣將侵君之象也。」明年,元瑾、劉思逸謀殺大將軍之應。



 周大象二年,尉迥敗於相州。坑其黨與數萬人於游豫園。其處每聞鬼夜哭聲。



 範洪《五行傳》曰:「哭者死亡之表,近夜妖也。鬼而夜哭者,將有死亡之應。」



 京房《易飛候》曰:「鬼夜哭,國將亡。」明年,周氏王公皆見殺,周室亦亡。



 仁壽中,仁壽宮及長城之下,數聞鬼哭。尋而獻後及帝,相次而崩於仁壽宮。



 大業八年,楊玄感作亂於東都。尚書樊子蓋坑其黨與
 於長夏門外,前後數萬。



 洎於末年,數聞其處鬼哭,有呻吟之聲。與前同占。其後王世充害越王侗於洛陽。



 華孽後齊武平元年,槐華而不結實。槐,三公之位也,華而不實,萎落之象。至明年,錄尚書事和士開伏誅。隴東王胡長仁,太保、瑯邪王儼皆遇害。左丞相段韶薨。



 陳後主時,有張貴妃、孔貴嬪,並有國色,稱為妖艷。後主惑之,寵冠宮掖,每充侍從,詩酒為娛。一入後庭,數旬不出,荒淫侈靡,莫知紀極。府庫空竭,頭會箕斂,天下怨叛,將士離心。敵人鼓行而進,莫有死戰之士。女德之咎也。
 及敗亡之際,後主與此姬俱投於井,隋師執張貴妃而戮之,以謝江東。《洪範五行傳》曰:「華者,猶榮華容色之象也。以色亂國,故謂華孽。」



 齊後主有寵姬馮小憐,慧而有色,能彈琵琶,尤工歌儛。後主惑之,拜為淑妃。



 選彩女數千,為之羽從,一女之飾,動費千金。帝從禽於三堆,而周師大至,邊吏告急,相望於道。帝欲班師,小憐意不已,更請合圍。帝從之。由是遲留,而晉州遂陷。後與周師相遇於晉州之下,坐小憐而失機者數矣,因而國滅。齊之士庶,至今咎之。



 牛禍
 梁武陵王紀祭城隍神,將烹牛,忽有赤蛇繞牛口,牛禍也。象類言之,又為龍蛇之孽。魯宣公三年,郊牛之口傷,時以為天不享。棄宣公也。《五行傳》曰:「逆君道傷,故有龍蛇之孽。」是時紀雖以赴援為名,而實妄自尊亢。思心之咎,神不享,君道傷之應。果為元帝所敗。



 後齊武平二年,並州獻五足牛,牛禍也。《洪範五行傳》曰:「牛事應,宮室之象也。」帝尋大發卒,於仙都苑穿池築山,樓殿間起,窮華極麗。功始就而亡國。



 後周建德六年,陽武有獸三,狀如水牛,一黃,一赤,一黑。與黑者鬥久之,黃者自傍觸之,黑者死,黃亦俱入
 於河。近牛禍也。黑者,周之所尚色。死者,滅亡之象。後數載,周果滅而隋有天下,旗牲尚赤,戎服以黃。



 大業初,恆山有牛,四腳膝上各生一蹄。其後建東都,築長城,開溝洫。



 心腹之痾陳禎明三年,隋師臨江,後主從容而言曰:「齊兵三來,周師再來,無弗摧敗。



 彼何為者?」都官尚書孔範曰:「長江天塹,古以為限隔南北。今日北軍豈能飛渡耶?臣每患官卑,彼若渡來,臣為太尉矣。」後主大悅,因奏妓縱酒,賦詩不輟。



 心腹之痾也。存亡之機,定之俄頃,君臣旰食不暇,
 後主已不知懼,孔範從而蕩之,天奪其心,曷能不敗?陳國遂亡,範亦遠徙。



 齊文宣帝嘗宴於東山,投杯赫怒,下詔西伐,極陳甲兵之盛。既而泣謂群臣曰:「黑衣非我所制。」卒不行。有識者以帝精魄已亂,知帝祚之不永。帝後竟得心疾,耽荒酒色,性忽狂暴,數年而崩。



 武成帝丁太后憂,緋袍如故。未幾,登三臺,置酒作樂,侍者進白袍,帝大怒,投之臺下。未幾而崩。



 黃眚黃祥梁大同元年,天雨土。二年,天雨灰,其色黃。近黃祥也。京
 房《易飛候》曰:「聞善不及,茲謂有知。厥異黃,厥咎龍,厥災不嗣。蔽賢絕道之咎也。」時帝自以為聰明博達,惡人勝己。又篤信佛法,舍身為奴,絕道蔽賢之罰也。



 大寶元年正月,天雨黃沙。二年,簡文帝夢丸土而吞之。尋為侯景所廢,以土囊壓之而斃,諸子遇害,不嗣之應也。



 陳後主時,夢黃衣人圍城。後主惡之,繞城橘樹,盡伐去之。隋高祖受禪之後,上下通服黃衣。未幾隋師攻圍之應也。



 後周大象二年正月,天雨黃土,移時乃息。與大同元年
 同占。時帝昏狂滋甚,期年而崩,至於靜帝,用遜厥位。絕道不嗣之應也。



 開皇二年,京師雨土。是時帝懲周室諸侯微弱,以亡天下,故分封諸子,並為行臺,專制方面。失土之故,有土氣之祥,其後諸王各謀為逆亂。京房《易飛候》曰:「天雨土,百姓勞苦而無功。」其時營都邑。後起仁壽宮,頹山堙谷,丁匠死者太半。



 裸蟲之孽梁太清元年,丹陽有莫氏妻,生男,眼在頂上,大如兩歲兒。墜地而言曰:「兒是旱疫鬼,不得住。」母曰:「汝當令我得
 過。」疫鬼曰:「有上官,何得自由。母可急作絳帽,故當無憂。」母不暇作帽,以絳系發。自是旱疫者二年,揚、徐、兗、豫尤甚。莫氏鄉鄰,多以絳免,他土效之無驗。



 大寶二年,京口人於藏兒,年五歲,登城西南角大樓,打鼓作《長江櫑》。鼓,兵象也。是時侯景亂江南。



 陳永定三年,有人長三丈,見羅浮山,通身潔白,衣服楚麗。京房占曰:「長人見,亡。」後二歲,帝崩。



 後主為太子時,有婦人突入東宮而大言曰:「畢國主。」後主立而祚終之應也。



 至德三年八月,建康人家婢死,埋之九日而更生。有牧
 牛人聞而出之。



 禎明二年,有船下,忽聞人言曰:「明年亂。」視之,得死嬰兒,長二尺而無頭。明年陳滅。



 齊天保中,臨漳有婦人產子,二頭共體。是後政由奸佞,上下無別,兩頭之應也。



 後主時,有桑門,貌若狂人,見烏則向之作禮,見沙門則毆辱之。烏,周色也。



 未幾,齊為周所吞,滅除佛法。



 後周保定三年,有人產子男,陰在背上如尾,兩足指如獸爪。陰不當生於背而生於背者,陰陽反覆,君臣顛倒之象。人足不當有爪而有爪者,將致攫人之變也。



 是時,
 晉蕩公宇文護專擅朝政,征伐自己,陰懷篡逆。天戒若曰,君臣之分已倒矣,將行攫噬之禍。帝見變而悟,遂誅晉公,親萬機,躬節儉,克平齊國,號為高祖。



 轉禍為福之效也。



 武帝時,有強練者,佯狂,持一瓠,至晉蕩公護門而擊破之,曰:「身尚可,子苦矣。」時護專政,因朝太后,帝擊殺之。發兵捕其諸子,皆備楚毒而死。強練又行乞於市,人或遺之粟麥,輒以無底袋受之。因大笑曰:「盛空。」未幾,周滅,高祖移都,長安城為墟矣。



 開皇六年,霍州有老翁,化為猛獸。



 七年,相州有桑門,變為蛇,尾繞樹而自抽,長二丈許。



 仁壽四年,有人長數丈,見於應門,其跡長四尺五寸。其年帝崩。



 大業元年,雁門人房回安,母年百歲,額上生角,長二寸。《洪範五行傳》曰:「婦人,陰象也。角,兵象也。下反上之應。」是後天下果大亂,陰戎圍帝於雁門。



 四年,雁門宋穀村有婦人生一肉卵,大如斗,埋之。後數日,所埋處雲霧盡合,從地雷震而上,視之洞穴,失卵所在。



 六年,趙郡李來王家婢產一物,大如卵。



 六年正月朔旦,有盜衣白練裙襦,手持香花,自稱彌勒佛出世。入建國門,奪衛士仗,將為亂。齊王暕遇而斬之。後三年,楊玄感作亂,引兵圍洛陽,戰敗伏誅。



 八年,有澄公者,若狂人,於東都大叫唱賊。帝聞而惡之。明年,玄感舉兵,圍洛陽。



 十二年,澄公又叫賊。李密逼東都,孟讓燒豐都市而去。



 九年,帝在高陽。唐縣人宋子賢,善為幻術。每夜,樓上有光明,能變作佛形,自稱彌勒出世。又懸大鏡於堂上,紙素上畫為蛇為獸及人形。有人來禮謁者,轉側其鏡,遣
 觀來生形像。或映見紙上蛇形,子賢輒告云:「此罪業也,當更禮念。」



 又令禮謁,乃轉人形示之。遠近惑信,日數百千人。遂潛謀作亂,將為無遮佛會,因舉兵,欲襲擊乘輿。事洩,鷹揚郎將以兵捕之。夜至其所,繞其所居,但見火坑,兵不敢進。郎將曰:「此地素無坑,止妖妄耳。」及進,無復火矣。遂擒斬之,並坐其黨與千餘家。其後復有桑門向海明,於扶風自稱彌勒佛出世,潛謀逆亂。人有歸心者,輒獲吉夢。由是人皆惑之,三輔之士,翕然稱為大聖。因舉兵反,眾至數萬。官軍擊破之。京房《易飛候》曰:「妖言動眾者,茲謂不信。路無人行。不出三年,起兵。」自是天下大
 亂,路無人行。



 木金水火沴土梁天監五年十一月,京師地震,木金水火沴土也。《洪範五行傳》曰:「臣下盛,將動而為害。」京房《易飛候》曰:「地動以冬十一月者,其邑饑亡。」時交州刺史李凱舉兵反。明年,霜,歲儉人饑。



 普通三年正月,建康地震。是時,義州刺史文僧朗以州叛。



 六年十二月,地震。京房《易飛候》曰:「地冬動有音,以十二月者,其邑有行兵。」是時,帝令豫章王琮將兵北伐。



 中大通五年正月,建康地震。京房《易飛候》曰:「地以春動,歲不昌。」是歲,大水,百姓饑饉。



 大同三年十一月,建康地震。京房《易飛候》曰:「地震以十一月,邑有大喪及饑亡。」明年,霜為災,百姓饑。



 三年十月,建康地震。是歲,會稽山賊起。



 七年二月,建康地震。是歲,交州人李賁舉兵,逐刺史蕭諮。



 九年閏正月,地震。李賁自稱皇帝,署置百官。



 太清三年四月,建康地再震。時侯景自為大丞相、錄尚書事,帝所須不給。是月,以憂憤崩。



 陳永定二年五月,建康地震。時王琳立蕭莊於郢州。



 太建四年十一月,地震。陳寶應反閩中。



 禎明元年正月,地震。施文慶、沈客卿專恣之應也。



 東魏武定二年十一月,西河地陷而且燃。京房《易妖占》曰:「地自陷,其君亡。」祖釭曰:「火,陽精也;地者,陰主也。地燃,越陰之道,行陽之政,臣下擅恣,終以自害。」時後齊神武作宰,而侯景專擅河南。後二歲,神武果崩,景遂作亂,而自取敗亡之應。



 後齊河清二年,並州地震。和士開專恣之應。



 後周建德二年,涼州地頻震。城郭多壞,地裂出泉。京房《
 易妖占》曰:「地分裂,羌夷叛。」時吐谷渾頻寇河西。



 開皇十四年五月,京師地震。京房《易飛候》曰:「地動以夏五月,人流亡。」



 是歲關中饑,帝令百姓就糧於關東。



 仁壽二年四月,岐、雍地震。京房《易飛候》曰:「地動以夏四月,五穀不熟,人大饑。」



 三年,梁州就谷山崩。《洪範五行傳》曰:「崩散落,背叛不事上之類也。」



 梁州為漢地。明年,漢王諒舉兵反。



 大業七年,砥柱山崩,壅河,逆流數十里。劉向《洪範五行傳》曰:「山者,君之象;水者,陰之表,人之類也。天戒若曰,君人擁威重,將崩壞,百姓不得其所。」時帝興遼東之師,百
 姓不堪其役,四海怨叛。帝不能悟,卒以滅亡。



 《洪範五行傳》曰:「皇之不極,是謂不建。厥咎瞀,厥罰常陰,厥極弱。時則有射妖,則有龍蛇之孽,則有馬禍。」



 雲陰開皇二十年十月,久陰不雨。劉向曰:「王者失中,臣下強盛而蔽君明,則雲陰。」是時,獨孤後遂與楊素陰譖太子勇,廢為庶人。



 射妖東魏武定四年,後齊神武作宰,親率諸軍,攻西魏於玉壁。其年十一月,帝不豫,班師。將士震懼,皆曰:「韋孝寬以
 定功弩射殺丞相。」西魏下令國中曰:「勁弩一發,兇身自殞。」神武聞而惡之,其疾暴增,近射妖也。《洪範五行傳》曰:「射者,兵戎禍亂之象,氣逆天則禍亂將起。」神武行,殿中將軍曹魏祖諫曰:「王以死氣逆生氣,為客不利,主人則可。」帝不從,頓軍五旬,頻戰沮衄。又聽孤虛之言,於城北斷汾水,起土山。其處天險千餘尺,功竟不就,死者七萬。氣逆天之咎也。其年帝崩。明年,王思政擾河南。



 武平,後主自並州還鄴,至八公嶺,夜與左右歌而行。有一人忽發狂,意後主以為狐媚,伏草中彎弓而射之。傷數人,幾中後主。後主執而斬之。其人不自覺也。



 狐而能
 媚,獸之妖妄也。時帝不恤國政,專與內人閹豎酣歌為樂。或衣襤縷衣,行乞為娛。此妖妄之象。人又射之,兵戎禍亂之應也。未幾而國滅。



 龍蛇之孽梁天監二年,北梁州潭中有龍鬥,濆霧數里。龍蛇之孽。《洪範五行傳》曰:「龍,獸之難害者也。天之類,君之象。天氣害,君道傷,則龍亦害。鬥者兵革之象也。」京房《易飛候》曰:「眾心不安,厥妖龍鬥。」是時帝初即位,而有陳伯之、劉季連之亂,國內危懼。



 普通五年六月,龍鬥於曲阿王陂,因西行,至建陵城,所
 經之處,樹木皆折開數十丈。與天監二年同占。經建陵而樹木折者,國有兵革之禍,園陵殘毀之象。時帝專以講論為務,不崇耕戰,將輕而卒惰。君道既傷,故有龍孽之應。帝殊不悟。



 至太清元年,黎州水中又有龍鬥。波浪湧起,雲霧四合,而見白龍南走,黑龍隨之。



 其年,侯景以兵來降,帝納之而無備,國人皆懼。俄而難作,帝以憂崩。



 大同十年夏,有龍夜因雷而墮延陵人家井中,明旦視之,大如驢。將以戟刺之,俄見庭中及室中各有大蛇,如數百斛船,家人奔走。《洪範五行傳》曰:「龍,陽類,貴象也。上則在天,下則在地,不當見庶人邑里室家。井中,幽深之
 象也,諸侯且有幽執之禍,皇不建之咎也。」後侯景反,果幽殺簡文於酒庫,宗室王侯皆幽死。



 陳太建十一年正月,龍見南兗州池中,與梁大同十年同占。未幾,後主嗣位,驕淫荒怠,動不得中。其後竟以國亡,身被幽執。



 東魏武定元年,有大蛇見武牢城。是時,北豫州刺史高仲密妻李氏,慧而艷。



 世子澄悅之,仲密內不自安,遂以武牢叛,陰引西魏,大戰於河陽。神武為西兵所窘,僅而獲免,死者數千。



 後齊天保九年,有龍長七八丈,見齊州大堂。占同大同
 十年。時常山、長廣二王權重,帝不思抑損。明年帝崩,太子殷嗣立。常山王演果廢帝為濟南王,幽而害之。



 河清元年,龍見濟州浴堂中。占同天保九年。先是平秦王歸彥受昭帝遺詔,立太子百年為嗣。而歸彥遂立長廣王湛,是為武成帝。而廢百年為樂陵王,竟以幽死。



 天統四年,貴鄉人伐枯木,得一黃龍,折腳,死於孔中,齊稱木德。龍,君象。



 木枯龍死,不祥之甚。其年武成崩。



 武平三年,龍見邯鄲井中,其氣五色屬天。又見汲郡佛寺涸井中。占同河清元年。後主竟降周,後被誅。



 武平七年,並州招遠樓下,有赤蛇與黑蛇鬥,數日,赤蛇
 死。赤,齊尚色;黑,周尚色。斗而死,滅亡之象也。後主任用邪佞,與周師連兵於晉州之下。委軍於孽臣高阿那肱,竟啟敵人,皇不建之咎也。後主遂為周師所虜。



 瑯邪王儼壞北宮中白馬浮圖,石趙時澄公所建。見白蛇長數丈,回旋失所在。



 時儼專誅,失中之咎也。見變不知戒,以及於難。



 後周建德五年,黑龍墜於亳州而死。龍,君之象。黑,周所尚色。墜而死,不祥之甚。時皇太子不才,帝每以為慮,直臣王軌、宇文孝伯等驟請廢立,帝不能用。



 後二歲,帝崩,太子立,虐殺齊王及孝伯等,因而國亡。



 仁壽四年,龍見代州總管府井中。其龍或變為鐵馬甲士彎弓上射之象。變為鐵馬,近馬禍也。彎弓上射,又近射妖,諸侯將有兵革之變,以致幽囚也。是時漢王諒潛謀逆亂,故變兵戒之。諒不悟,遂興兵反,事敗,廢為庶人,幽囚數年而死。



 馬禍侯景僭尊號於江南,每將戰,其所乘白馬,長鳴蹀足者輒勝,垂頭者輒不利。



 西州之役,馬臥不起,景拜請,且棰之,竟不動。近馬禍也。《洪範五行傳》曰:「馬者兵象。將有寇戎之事,故馬為怪。」景因此大敗。



 陳太建五年,衡州馬生角。《洪範五行傳》曰:「馬生角,兵之象,敗亡之表也。」是時宣帝遣吳明徹出師呂梁,與周師拒。連兵數歲,眾軍覆沒,明徹竟為周師所虜。



 天保中,廣宗有馬,兩耳間生角,如羊尾。京房《易傳》曰:「天子親伐,則馬生角。」四年,契丹犯塞,文宣帝親御六軍以擊之。



 大業四年,太原廄馬死者太半,帝怒,遣使案問。主者曰:「每夜廄中馬無故自驚,因而致死。」帝令巫者視之。巫者知帝將有遼東之役,因希旨言曰:「先帝令楊素、史萬歲取之,將鬼兵以伐遼東也。」帝大悅,因釋主者。《洪範五行
 傳》曰:「逆天氣,故馬多死。」是時,帝每歲巡幸,北事長城,西通且末,國內虛耗,天戒若曰,除廄馬,無事巡幸。帝不悟,遂至亂。



 十一年,河南、扶風三郡,並有馬生角,長數寸。與天保初同占。是時,帝頻歲親征高麗。



 義寧元年,帝在江都宮,龍廄馬無故而死,旬日,死至數百匹。與大業四年同占。



\end{pinyinscope}