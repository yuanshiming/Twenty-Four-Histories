\article{卷十四志第九 音樂中}

\begin{pinyinscope}

 齊
 神武霸跡肇創,遷都於鄴,猶曰人臣,故咸遵魏典。及文宣初禪,尚未改舊章。宮懸各設十二鎛鐘於其辰位,四面並設編鐘磐各一簨虡,合二十架。設建鼓於四隅。郊廟朝會同用之。其後將有創革,尚藥典御祖珽自言,舊在落下,曉知舊樂,上書曰:「魏氏來自雲、朔,肇有諸華,樂操土風,未移其俗。至道武帝皇始元年,破慕容寶於
 中山,獲晉樂器,不知採用,皆委棄之。天興初,吏部郎鄧彥海奏上廟樂,創制宮懸,而鐘管不備。樂章既闕,雜以《簸邏回歌》。初用八佾,作《皇始》之舞。至太武帝平河西,得沮渠蒙遜之伎,賓嘉大禮,皆雜用焉。此聲所興,蓋苻堅之末,呂光出平西域,得胡戎之樂,因又改變,雜以秦聲,所謂秦漢樂也。至永熙中,錄尚書長孫承業,共臣先人太堂卿瑩等,斟酌繕修,戎華兼採,至於鐘律,煥然大備。自古相襲,損益可知,今之創制,請以為準。」珽因採魏安豐王延明及信都芳等所著《樂說》而定正聲。始具宮懸之器,仍雜西涼之曲,樂名《廣成》,而舞不立號,所謂「洛陽
 舊樂」者也。



 武成之時,始定四郊、宗廟、三朝之樂。群臣入出,奏《肆夏》。牲入出,薦毛血,並奏《昭夏》。迎送神及皇帝初獻、禮五方上帝,並奏《高明》之樂,為《覆壽》之舞。皇帝入壇門及升壇飲福酒,就燎位,還便殿,並奏《皇夏》。以高祖配饗,奏《武德》之樂。為《昭烈》之舞。裸地,奏登歌。其四時祭廟及禘祫皇六世祖司空、五世祖吏部尚書、高祖秦州刺史、曾祖太慰武貞公、祖文穆皇帝諸神室,並奏《始基》之樂,為《恢祚》之舞。高祖神武皇帝神室,奏《武德》之樂,為《昭烈》之舞。文襄皇帝神室,奏《文德》之樂,為《宣政》之舞。顯祖文宣皇帝神室,奏《文正》之樂,為《光大》之舞。肅宗孝昭皇
 帝神室,奏《文明》之樂,為《休德》之舞。其入出之儀,同四郊之禮。今列其辭云。



 大禘圜丘及北郊歌辭:夕牲群臣入門,奏《肆夏》樂辭:肇應靈序,奄字黎人。乃朝萬國,爰徵百神。祗展方望,幽顯咸臻。禮崇聲協,贄列珪陳。翼差鱗次,端笏垂紳。來趨動色,式贊天人。



 迎神奏《高明樂》辭登歌辭同惟神監矣,北郊云:惟祗監矣。皇靈肅止。圓璧展事,北郊云:方琮展事成文即始。北郊云:即陰成理士備八能,樂合六變。北郊云:樂
 合八變。風湊伊雅,光華襲薦。宸衛騰景,靈駕霏煙。嚴壇生白,綺席凝玄。



 牲出入,奏《昭夏》辭:剛柔設位,惟皇配之。言肅其禮,念暢在茲。飾牲舉獸,載歌且舞。既舍伊腯,致精靈府。物色惟典,齋沐加恭。宗族咸暨,罔不率從。



 薦毛血,奏《昭夏》辭:群臣出,奏《肆夏》,進熟,群臣入,奏《肆夏》,辭同初入。



 展禮上月,肅事應時。繭慄為用,交暢有期。弓矢斯發,盆簝將事。圓神致祀,北郊云:方祗致祀。率由先志。和以鑾刀,臭以血膋。致哉敬矣,厥義
 孔高。



 進熟,皇帝入門,奏《皇夏》辭:帝敬昭宣,皇誠肅致。玉帛齊軌,屏攝咸次。三垓上列,北郊云:重垓上列四陛旁升。北郊云:分陛旁升龍陳萬騎,鳳動千乘。神儀天藹,睟容離曜。金根停軫,奉光先導。



 皇帝升丘,奏《皇夏》辭:壇上登歌辭同紫壇雲暖,北郊云:層壇雲暖紺幄霞褰。北郊云:嚴幄霞褰我其陟止,載致其虔。百靈竦聽,萬國咸仰。人神咫尺,玄應肸蚃。



 皇帝初獻,奏《高明樂》辭:上下眷,旁午從。爵以質,獻以恭。咸斯暢,樂惟雍。孝敬闡,臨萬邦。



 皇帝奠爵訖,奏《高明樂》、《覆燾》之舞辭:自天子之,會昌神道。丘陵肅事,北郊云:方澤祗事克光天保。九關洞開,百靈環列。八樽呈備,五聲投節。



 皇帝獻太祖配饗神座,奏《武德》之樂、《昭烈》之舞辭:皇帝小退,當昊天上帝神座前,奏《皇夏》,辭同上《皇夏》。



 配神登聖,主極尊靈。敬宣昭燭,咸達窅冥。
 禮弘化定,樂贊功成。穰穰介福,下被群生。



 皇帝飲福酒,奏《皇夏》之樂:皇帝詣東陛,還便坐,又奏《皇夏》,辭同初入門。



 皇心緬且感,吉蠲奉至誠。赫哉光盛德,乾巛詔百靈。報福歸昌運,承祐播休明。風雲馳九域,龍蛟躍四溟。浮幕呈光氣,儷象燭華精。《護》《武》方知恥,《韶》《夏》僅同聲。



 送神,降丘南陛,奏《高明樂》辭:皇帝之望燎位,又奏《皇夏》,辭同上《皇夏》。



 獻享畢,懸佾周。神之駕,將上游。北郊云:將下游。超斗極,北郊云:超荒極。絕河流。北郊云:憩昆丘。懷萬國,寧九州。欣帝道,心顧留。幣上下,荷皇休。



 紫壇既燎,奏《昭夏》樂辭:皇帝自望燎還本位,奏《皇夏》,辭同上《皇夏》。



 玄黃覆載,元首照臨。合德致禮,有契其心。敬申事闋,潔誠云報。玉帛載升,北郊云:牲玉載陳。棫樸斯燎。寥廓幽曖,播以馨香。皇靈惟監,降福無疆。



 皇帝還便殿,奏《皇夏》辭:群臣出,奏《肆夏》,辭同上《肆夏》。祠感帝用圜丘辭。



 天大親嚴,匪敬伊孝。永言肆饗,宸明增耀。陽丘既暢,北郊云:陰澤雲暢。



 大典逾光。乃安斯息,欽若舊章。天回地旋,鳴鑾引警。且萬且億,皇歷惟永。



 五郊迎氣樂辭:青帝降神,奏《高明樂》辭:歲雲獻,谷風歸。斗東指,雁北飛。電鞭激,雷車遽。虹旌靡,青龍馭。和氣洽,具物滋。翻降止,應帝期。



 赤帝降神,奏《高明樂》辭:婺女司旦中呂宣,硃精禦節離景延。根荄俊茂溫風發,柘火風水應炎月。執衡長物德孔昭,赤旂霞曳會今朝。



 黃帝降神,奏《高明樂》辭:
 居中市五運,乘衡畢四時。含養資群物,協德固皇基。嘽緩契王風,持載符君德。良辰動靈駕,承祀昌邦國。



 白帝降神,奏《高明樂》辭:風涼露降,馳颺易寒精。山川搖落,平秩在西成。蓋藏成積,蒸人被嘉祉。從享來儀,鴻休溢千祀。



 黑帝降神,奏《高明樂》辭:虹藏雉化告寒,冰壯地坼年殫。日次月紀方極,九州萬邦獻力。協光是紀歲窮,微陽潛光方融。
 天子赫赫明聖,享神降福惟敬。



 祠五帝於明堂樂歌辭:先祀一日,夕牲,群官入自門,奏《肆夏》:國陽崇祀,嚴恭有聞。荒華胥暨,樂我大君。冕瑞有列,禽帛恭敘。群後師師,威儀容與。執禮辨物,司樂考章。率由靡墜,休有烈光。



 太祝令迎神,奏《高明樂》、《覆燾舞》辭:祖德光,國圖昌。祗上帝,禮四方。闢紫宮,洞華闕。龍獸奮,風雲發。飛硃雀,從玄武。攜日月,帶雷雨。耀宇內,溢區中。眷帝道,
 感皇風。帝道康,皇風扇。粢盛列,椒糈薦。神且寧,會五精。歸福祿,幸閭亭。



 太祖配饗,奏《武德樂》、《昭烈舞》辭:五方天帝奏《高明》之樂、《覆燾》之舞,辭同迎氣。



 我惟我祖,自天之命。道被歸仁,時屯啟聖。運鐘千祀,授手萬姓。夷兇掩虐,匡頹翼正。載經載營,庶士咸寧。九功以洽,七德兼盈。丹書入告,玄玉來呈。露甘泉白,雲鬱河清。聲教咸往,舟車畢會。仁加有形,化洽無外。嚴親惟重,陟配惟大。既祐斯歌,率土攸賴。



 牲出入,奏《昭夏樂》辭:
 孝饗不匱,精潔臨年。滌牢委溢,形色博牷。於以用之,言承歆祀。肅肅威儀,敢不敬止。載飾載省,維牛維羊。明神有察,保茲萬方。



 薦血毛,奏《昭夏》辭:群臣出,奏《肆夏》,進熟,群臣入,奏《肆夏》,同上《肆夏》辭。



 我將宗祀,夤獻厥誠。鞠躬如在,側聽無聲。薦色斯純,呈氣斯臭。有滌有濯,惟神其祐。五方來格,一人多祉。明德惟馨,於穆不已。



 進熟,皇帝入門,奏《皇夏》辭:皇帝升壇,奏《皇夏》,辭同。



 象乾上構,儀巛下基。集靈崇祖,永言孝思。室陳簋豆,庭羅懸佾。夙夜畏威,保茲貞吉。
 舞貴其夜,歌重其升。降斯百錄,惟響惟應。



 皇帝初獻,奏《高明樂》、《覆燾舞辭》:度幾筵,闢牖戶。禮上帝,感皇祖。酌惟潔,滌以清。薦心款,達神明。



 皇帝裸獻,奏《高明樂》、《覆燾舞》辭:帝精求降,應我明德。禮殫義展,流祉邦國。既受多祉,實資孝敬。祀竭其誠,荷天休命。



 皇帝飲福酒,奏《皇夏》辭:恭祀洽,盛禮宣。英猷爛層景,廣澤同深泉。上靈鐘百福,群神歸萬年。
 月軌咸梯岫,日域盡浮川。瑞鳥飛玄扈,潛鱗躍翠漣。皇家膺寶歷,兩地復參天。



 太祝送神,奏《高明樂》、《覆燾舞》辭:青陽奏,發硃明。歌西皓,唱玄冥。大禮罄,廣樂成。神心懌,將遠征。飾龍駕,矯鳳憩。指閶闔,憩層城。出溫谷,邁炎庭。跨西汜,過北溟。忽萬億,耀光精。比電騖,與雷行。嗟皇道,懷萬靈。固王業,震天聲。



 皇帝還便殿,奏《皇夏》辭:交物備矣,聲明有章。登薦唯肅,禮邈前王。
 鬯齊雲終,折旋告罄。穆穆旒冕,蘊誠畢敬。屯衛按部,鑾蹕回途。暫留紫殿,將及清都。



 享廟樂辭:先祀一日,夕牲,群臣入,奏《肆夏》辭:霜淒雨暢,烝哉帝心。有敬其祀,肅事惟歆。昭昭車服,濟濟衣簪。鞠躬貢酎,磬折奉琛。差以五列,和以八音。式祗王度,如玉如金。



 迎神奏《高明》登歌樂辭:日卜惟吉,辰擇其良,奕奕清廟,黼黻周張。大呂為角,應鐘為羽。路鞀陰竹,德歌昭舞。
 祀事孔明,百神允穆。神心乃顧,保茲介福。



 牲出入,奏《昭夏樂》辭:大祀云事,獻奠有儀。既歌既展,贊顧迎牲。執從伊竦,芻飾惟慄。俟用於庭,將升於室。且握且驛,以致其誠。惠我貽頌,降祉千齡。



 薦血毛,奏《昭夏》辭:三公出,奏《肆夏》,進熟,群臣入,奏《肆夏》,辭同。



 愐彼遐慨,悠然永思。留連七享,纏綿四時。神升魄沈,靡聞靡見。陰陽載俟,臭聲兼薦。祖考其鑒,言萃王休。降神敷錫,百福是由。



 進熟,皇帝入北門,奏《皇夏樂》辭:
 齊居嚴殿,夙駕層闈。車略垂彩,旒袞騰輝。聳誠載仰,翹心有慕。洞洞自形,斤斤表步。閟宮有邃,神道依稀。孝心緬邈,爰屬爰依。



 太祝裸地,奏登歌樂辭:皇帝詣東陛,奏《皇夏》,升殿,又奏《皇夏》,辭同。



 太室窅窅,神居宿設。鬱鬯惟芬,珪璋惟潔。彞斝應時,龍蒲代用。藉茅無咎,福祿攸降。端感會事,儼思修禮。齊齊勿勿,俄俄濟濟。



 皇帝升殿,殿上作登歌樂辭:我祠我祖,永惟厥先。炎農肇聖,靈祉蟬聯。霸圖中造,帝業方宣。道昌基構,撫運承天。
 奄家六合,爰光八埏。尊神致禮,孝思惟纏。寒來暑反,惕薦在年。匪敬伊慕,備物不愆。設虡設業,鞉鼓填填。闢公在位,有容伊虔。登歌啟佾,下管應懸。厥容無爽,幽明肅然。誠幣厚地,和達穹玄。既調風雨,載協山川。周庭有列,湯孫永延。教聲惟被,邁後光前。



 皇帝初獻皇祖司空公神室,奏《始基樂》、《恢祚舞》辭:克明克俊,祖武惟昌。業弘營土,聲被海方。有流厥德,終耀其光。明神幽贊,景祚攸長。



 皇帝初獻皇祖吏部尚書神室,奏《始基樂》、《恢祚舞》辭:
 顯允盛德,隆我前構。瑤源彌瀉,瓊根愈秀。誕惟有族,丕緒克茂。大業崇新,洪基增舊。



 皇帝初獻皇祖秦州使君神室,奏《始基樂》、《恢祚舞》辭:祖德丕顯,明哲知機。豹變東國,鵲起西歸。禮申官次,命改朝衣。敬思孝享,多福無違。



 皇帝獻太祖太尉武貞公神室,奏《始基樂》、《恢祚舞》辭:兆靈有業,潛德無聲。韜光戢耀,貫幽洞冥。道弘舒卷,施博藏行。緬追歲事,夜遽不寧。



 皇帝獻皇祖文穆皇帝神室,奏《始其樂》、《恢祚舞》辭:皇皇祖德,穆穆其風。語默自己,明睿在躬。
 荷天之錫,聖表克隆。高山作矣,寶祚其崇。離光旦旦,載煥載融。感薦惟永,神保無穹。



 皇帝獻高祖神武皇帝神室,奏《武德樂》、《昭烈舞》辭:天造草昧,時難糾紛。敦拯斯溺,靡救其焚。大人利見,緯武經文。顧指惟極,吐吸風雲。開天闢地,峻岳夷海。冥工掩跡,上德不宰。神心有應,龍化無待。義征九服,仁兵告凱。上平下成,靡或不寧。匪王伊帝,偶極崇靈。享親則孝,潔祀惟誠。禮備樂序,肅贊神明。



 皇帝獻文襄皇帝神室,奏《文德樂》、《宣政舞》辭:
 聖武丕基,睿文顯統。眇哉神啟,鬱矣天縱。道則人弘,德云邁種。昭冥咸敘,崇深畢綜。自中徂外,經朝庇野。政反淪風,威還缺雅。旁作穆穆,格於上下。維享維宗,來鑒來假。



 皇帝獻顯祖文宣皇帝,奏《文正樂》、《光大舞》辭:玄歷已謝,蒼靈告期。圖璽有屬,揖讓惟時。龍升獸變,弘我帝基。對揚穹昊,實啟雍熙。欽若皇猷,永懷王度。欣賞斯穆,威刑允措。軌物俱宣,憲章咸布。俗無邪指,下歸正路。茫茫九域,振以乾綱。混通華裔,配括天壤。
 作禮視德,列樂傳響。薦祀惟虔,衣冠載仰。



 皇帝還東壁,飲福酒,奏《皇夏》樂辭:孝心翼翼,率禮兢兢。時洗時薦,或降或升。在堂在戶,載湛載凝。多品斯奠,備物攸膺。蘭芬敬挹,玉俎恭承。受祭之祜,如彼岡陵。



 送神,奏《高明樂》辭:仰榱桷,慕衣冠。禮云罄,祀將闌。神之駕,紛奕奕。乘白雲,無不適。窮昭域,極幽途。歸帝祉,眷皇都。



 皇帝詣便殿,奏《皇夏》樂辭:群官出,奏《肆夏》,辭同。



 禮行斯畢,樂奏以終。受嘏先退,載暢其衷。鑾軒循轍,麾旌復路。光景徘徊,弦歌顧慕,靈之相矣,有錫無疆。國圖日競,家歷天長。



 元會大饗,協律不得升陛,黃門舉麾於殿上。今列其歌辭云。



 賓入門,四箱奏《肆夏》辭:昊蒼眷命,興王統天。業高帝始,道邈皇先。禮成化穆,樂合風宣。賓朝荒夏,揚對穹玄。



 皇帝出閣,奏《皇夏樂》辭:夏正肇旦,周物充庭。具僚在位,俯伏無聲。
 大君穆穆,宸儀動睟。日煦天回,萬靈胥萃。



 皇帝當扆,群臣奉賀,奏《皇夏》辭:天子南面,乾覆離明。三千咸列,萬國填並。猶從禹會,如次湯庭。奉茲一德,上下和平。



 皇帝入寧變服,黃鐘、太簇二箱奏《皇夏》辭:我應天歷,四海為家。協同內外,混一戎華。鶴蓋龍馬,風乘雲車。夏章夷服,其會如麻。九賓有儀,八音有節。肅肅於位,飲和在列。四序氤氳,三光昭晣。君哉大矣,軒唐比轍。



 皇帝變服,移幄坐於西箱,帝出升御坐,姑洗奏《皇夏》辭:
 皇運應籙,廓定區宇。受終以文,構業以武。堯昔命舜,舜亦命禹。大人馭歷,重規沓矩。欽明在上,昭納入夤。從靈體極,誕聖窮神。化生群品,陶育蒸人。展禮肆樂,協此元春。



 王公奠璧,奏《肆夏》辭:萬方咸暨,三揖以申。垂旒馮玉,五瑞交陳。拜稽有章,升降有節。聖皇負扆,虞唐比烈。



 上壽,黃鐘箱奏上壽曲辭:仰三光,奏萬壽。人皇御六氣,天地同長久。



 皇太子入,至坐位,酒至御,殿上奏登歌辭:
 大齊統歷,道化光明。馬圖呈寶,龜籙告靈。百蠻非眾,八荒非逖。同作堯人,俱包禹跡。其一天覆地載,成以四時。惟皇是則,比大於茲。群星拱極,眾川赴海。萬宇駿奔,一朝咸在。其二齊之以禮,相趨帝庭。應規蹈矩,玉色金聲。動之以樂,和風四布。龍申鳳舞,鸞歌麟步。其三食至御前,奉食舉樂辭:三端正啟,萬方觀禮。具物充庭,二儀合體。百華照曉,千門洞晨。或華或裔,奉贄惟新。悠悠亙六合,員首莫不臣。仰施如雨,晞和猶
 春。風化表笙鏞,歌謳被琴瑟。誰言文軌異,今朝混為一。其一彤庭爛景,丹陛流光。懷黃綰白,鵷鷺成行。文贊百揆,武鎮四方。折沖鼓雷電,獻替協陰陽。大矣哉,道邁上皇。陋五帝,狹三王。窮禮物,該樂章。序冠帶,垂衣裳。其二天壤和,家國穆。悠悠萬類咸孕育。契冥化,侔大造。靈效珍,神歸寶。興雲氣,飛龍蒼。麟一角,鳳五光。硃雀降,黃玉表。九尾馴,三足擾。化之定,至矣哉。瑞感德,四方來。其
 三囹圄空,水火菽粟。求賢振滯棄珠玉。衣不靡,宮以卑。當陽端默,垂拱無為。



 雲雲萬有,其樂不訾。其四嗟此舉時,逢至道。肖形咸自持,賦命無傷夭。行氣進皇輿,游龍服帝早。聖主寧區宇,乾坤永相保。其五牧野征,鳴條戰。大齊家萬國,拱揖應終禪。奧主廓清都,大君臨赤縣。高居深視,當扆正殿。旦暮之期今一見。其六兩儀分,牧以君。陶有象,化無垠。大齊德,
 邈誰群。超鳳火,冠龍雲。露以潔,風以薰。榮光至,氣氤氳。其七神化遠,人靈協。寒暑調,風雨變。披泥檢,受圖諜。圖諜啟,期運昌。分四序,綴三光。延寶祚,眇無疆。其八惟皇道,升平日。河水清,海不溢。雲幹呂,風入律。驅黔首,入仁壽。與天高,並地厚。其九刑以厝,頌聲揚。皇情邈,眷汾襄。岱山高,配林壯。亭亭聳,雲雲望。旗葳蕤,駕騤騤。刊金闕,奠玉龜。其十
 文舞將作,先設階步辭:我後降德,肇峻皇基。搖鈴大號,振鐸命期。雲行雨洽,天臨地持。茫茫區宇,萬代一時。文來武肅,成定於茲。象容則舞,歌德言詩。鏘鏘金石,列列匏絲。鳳儀龍至,樂我雍熙。



 文舞辭:皇天有命,歸我大齊。受茲華玉,爰錫玄珪。奄家環海,實子蒸黎。圖開寶匣,檢封芝泥。無思不順,自東徂西。教南暨朔,罔敢或攜。比日之明,如天之大。神化斯洽,率土無外。
 眇眇舟車,華戎畢會。祠我春秋,服我冠帶。儀協震象,樂均天籟。蹈武在庭,其容藹藹。



 武舞將作,先設階步辭:大齊統歷,天鑒孔昭。金人降泛,火鳳來巢。眇均虞德,干戚降苗。夙沙攻主,歸我軒朝。禮符揖讓,樂契《咸》《韶》。蹈揚惟序,律度時調。



 武舞辭:天眷橫流,宅心玄聖。祖功宗德,重光襲映。我皇恭己,誕膺靈命。宇外斯燭,域中咸鏡。悠悠率土,時惟保定。微微動植,莫違其性。
 仁豐庶物,施洽群生。海寧洛變,契此休明。雅宣茂烈,頌紀英聲。鏗皇鐘鼓,掩抑簫笙。歌之不足,舞以禮成。鑠矣王度,緬邁千齡。



 皇帝入,鐘鼓奏《皇夏》辭:禮終三爵,樂奏九成。允也天子,穹壤和平。載色載笑,反寢宴息。一人有祉,百神奉職。



 鼓吹二十曲,皆改古名,以敘功德。第一,漢《硃鷺》改名《水德謝》,言魏謝齊興也。第二,漢《思悲翁》改名《出山東》,言神武帝戰廣阿,創大業,破爾硃兆也。第三,漢《艾如張》改名《戰韓陵》,言神武滅四胡,定京洛,遠近賓服也。第四,漢《上
 之回》改名《殄關隴》,言神武遣侯莫陳悅誅賀拔岳,定關、隴,平河外,漠北款,秦中附也。第五,漢《擁離》改名《滅山胡》,言神武屠劉蠡升,高車懷殊俗,蠕蠕來向化也。第六,漢《戰城南》改名《立武定》,言神武立魏主,天下既安,而能遷於鄴也。第七,漢《巫山高》改名《戰芒山》,言神武斬周十萬之眾,其軍將脫身走免也。第八,漢《上陵》改名《擒蕭明》,言梁遣兄子貞陽侯來寇彭、宋,文襄帝遣太尉、清河王岳,一戰擒殄,俘馘萬計也。第九,漢《將進酒》改名《破侯景》,言文襄遣清河王岳摧殄侯景,克復河南也。第十,漢《君馬黃》改名《定汝潁》,言文襄遣清河王岳,擒周大將軍王思
 政於長葛,汝、潁悉平也。第十一,漢《芳樹》改名《克淮南》。言文襄遣清河王岳,南翦梁國,獲其司徒陸法和,克壽春、合肥、鐘離、淮陰,盡取江北之地也。第十二,漢《有所思》改名《嗣丕基》,言文宣帝統纘大業也。第十三,漢《稚子班》改名《聖道洽》,言文宣克隆堂構,無思不服也。第十四,漢《聖人出》改名《受魏禪》,言文宣應天順人也。第十五,漢《上邪》改名《平瀚海》,言蠕蠕盡部落入寇武州之塞,而文宣命將出征,平殄北荒,滅其國也。第十六,漢《臨高臺》改名《服江南》,言文宣道洽無外,梁主蕭繹來附化也。第十七,漢《遠如期》改名《刑罰中》,言孝昭帝舉直措枉,獄訟無怨也。
 第十八,漢《石留行》改名《遠夷至》,言時主化沾海外,西夷諸國,遣使朝貢也。第十九,漢《務成》改名《嘉瑞臻》,言時主應期,河清龍見,符瑞總至也。第二十,漢《玄云》改名《成禮樂》,言時主功成化洽,制禮作樂也。古又有《黃雀》、《釣竿》二曲,略而不用,並議定其名,被於鼓吹。



 諸州鎮戍,各給鼓吹樂,多少各以大小等級為差。諸王為州,皆給赤鼓、赤角,皇子則增給吳鼓、長鳴角,上州刺史皆給青鼓、青角,中州已下及諸鎮戍,皆給黑鼓、黑角。樂器皆有衣,並同鼓色。



 雜樂有西涼鼙舞、清樂、龜茲等。然吹笛、彈琵琶、五弦及
 歌舞之伎,自文襄以來,皆所愛好。至河清以後,傳習尤盛。後主唯賞胡戎樂,耽愛無已。於是繁手淫聲,爭新哀怨。故曹妙達、安未弱、安馬駒之徒,至有封王開府者,遂服簪纓而為伶人之事。後主亦自能度曲,親執樂器,悅玩無倦,倚弦而歌。別採新聲,為《無愁曲》,音韻窈窕,極於哀思,使胡兒閹官之輩,齊唱和之,曲終樂闋,莫不殞涕。雖行幸道路,或時馬上奏之,樂往哀來,竟以亡國。



 周太祖迎魏武入關,樂聲皆闕。恭帝元年,平荊州,大獲梁氏樂器,以屬有司。



 及建六官,乃詔曰:「六樂尚矣,其聲歌之節,舞蹈之容,寂寥已絕,不可得而詳也。但方行古
 人之事,可不本於茲乎?自宜依準,制其歌舞,祀五帝日月星辰。」



 於是有司詳定:郊廟祀五帝日月星辰,用黃帝樂,歌大呂,舞《雲門》。祭九州、社稷、水旱雩珝,用唐堯樂,歌應鐘,舞《大咸》。祀四望,饗諸侯,用虞舜樂,歌南呂,舞《大韶》。祀四類,幸闢雍,用夏禹樂,歌函鐘,舞《大夏》。祭山川,用殷湯樂,歌小呂,舞《大護》。享宗廟,用周武王樂,歌夾鐘,舞《大武》。皇帝出入,奏《皇夏》。賓出入,奏《肆夏》。牲出入,奏《昭夏》。蕃國客出入,奏《納夏》。有功臣出入,奏《章夏》。皇后進羞,奏《深夏》。宗室會聚,奏《族夏》。上酒宴樂,奏《陔夏》。諸侯相見,奏《驁夏》。皇帝大射,歌《騶虞》,諸侯歌《貍首》,大夫歌《採蘋》,士歌《採蘩》。
 雖著其文,竟未之行也。



 及閔帝受禪,居位日淺。明帝踐阼,雖革魏氏之樂,而未臻雅正。天和元年,武帝初造《山雲舞》,以備六代。南北郊、雩壇、太廟、禘祫、俱用六舞。南郊則《大夏》降神,《大護》獻熟,次作《大武》、《正德》、《武德》、《山雲之舞》。



 北郊則《大護》降神,《大夏》獻熟,次作《大武》、《正德》、《武德》、《山雲之舞》。雩壇以《大武》降神,《正德》獻熟,次作《大夏》、《大護》、《武德》、《山雲之舞》。太廟祫帝,則《大武》降神,《山云》獻熟,次作《正德》、《大夏》、《大護》、《武德之舞》。時享太廟,以《山云》降神,《大夏》獻熟,次作《武德之舞》。拜社,以《大護》降神,《大武》獻熟,次作《正德之舞》。



 五郊朝日,以《大夏》降神,《大護》獻熟。神州、夕月、籍
 田,以《正德》降神,《大護》獻熟。



 建德二年十月甲辰,六代樂成,奏於崇信殿。群臣咸觀。其宮懸,依梁三十六架。朝會則皇帝出入,奏《皇夏》。皇太子出入,奏《肆夏》。王公出入,奏《驁夏》。五等諸侯正日獻玉帛,奏《納夏》。宴族人,奏《族夏》。大會至尊執爵,奏登歌十八曲。食舉,奏《深夏》,舞六代《大廈》、《大護》、《大武》、《正德》、《武德》、《山雲之舞》。於是正定雅音,為郊廟樂。創造鐘律,頗得其宜。



 宣帝嗣位,郊廟皆循用之,無所改作。今採其辭云。



 員丘歌辭:降神,奏《昭夏》:
 重陽禋祀大報天,丙午封壇肅且圜。孤竹之管雲和弦,神光未下風肅然。王城七里通天臺,紫微斜照影徘徊。連珠合璧重光來,天策暫轉鉤陳開。



 皇帝將入門,奏《皇夏》:旌回外壝,蹕靜郊門。千乘按轡,萬騎雲屯。藉茅無咎,掃地惟尊。揖讓展禮,衡璜節步。星漢就列,風雲相顧。取法於天,降其永祚。



 俎入,奏《昭夏》:日至大禮,豐犧上辰。牲牢修牧,繭慄毛純。
 俎豆斯立,陶匏以陳。大報反命,居陽兆日。六變鼓鐘,三和琴瑟。俎奇豆偶,惟誠惟質。



 奠玉帛,奏《昭夏》:員玉已奠,蒼幣斯陳。瑞形成象,璧氣含春。禮從天數,智總員神。為祈為祀,至敬咸遵。



 皇帝升壇,奏《皇夏》:七星是仰,八陛有憑。就陽之位,如日之升。思虔肅肅,施敬繩繩。祝史陳信,玄象斯格。惟類之典,惟靈之澤。幽顯對揚,人神咫尺。



 皇帝初獻,作《雲門》之舞:
 獻以誠,鬱以清。山罍舉,沈齊傾。惟尚饗,洽皇情。降景福,通神明。



 皇帝初獻配帝,作《雲門》之舞:長丘遠歷,大電遙源。弓藏高隴,鼎沒寒門。人生於祖,物本於天。尊神配德,迄用康年。



 皇帝初獻及獻配帝畢,奏登歌:歲之祥,國之陽。蒼靈敬,翠雲長。象為飾,龍為章。乘長日,坯蟄戶。列雲漢,迎風雨,大呂歌,雲門舞。省滌濯,奠牲牷。鬱金酒,鳳凰樽。回天眷,顧中原。



 皇帝飲福酒,奏《皇夏》:國命在禮,君命在天。陳誠惟肅,欽福惟虔。洽斯百禮,福以千年。鉤陳掩映,天駟徘徊。雕禾飾斝,翠羽承罍。受斯茂祉,從天之來。



 撤奠奏《雍樂》:禮將畢,樂將闌。回日轡,動天關。翠鳳搖,和鑾響。五雲飛,三步上。風為馭,雷為車。無轍跡,有煙霞。暢皇情,休靈命。雨留甘,雲餘慶。



 帝就望燎位,奏《皇夏》:
 六曲聯事,九司咸則。率由舊章,於焉允塞。掌禮移次,燔柴在焉。煙升玉帛,氣斂牲牛全。休氣馨香,膋芳昭晰。翼翼虔心,明明上徹。



 帝還便座,奏《皇夏》:玉帛禮畢,人神事分。嚴承乃眷,瞻仰回雲。輦路千門,王城九軌。式道移候,司方回指。得一惟清,於萬斯寧。受茲景命,於天告成。



 方澤歌辭:降神,奏《昭夏》:報功陰澤,展禮玄郊。平琮鎮瑞,方鼎升庖。
 調歌絲竹,縮酒江茅,聲舒鐘鼓,器質陶匏。列耀秀華,凝芳都荔。川澤茂祉,丘陵容衛。雲飾山罍,蘭浮泛齊。日至之禮,歆茲大祭。



 奠玉,奏《昭夏》:曰若厚載,欽明方澤。敢以敬恭,陳之玉帛。德包含養,功藏靈跡。斯箱既千,子孫則百。



 初獻,奏登歌辭:舞詞同員丘。



 質明孝敬,求陰順陽。壇有四陛,琮為八方。牲牷蕩滌,蕭合馨香。和鑾戾止,振鷺來翔。威儀簡簡,鐘鼓喤娶。聲和孤竹,韻入空桑。
 封中雲氣,坎上神光。下元之主,功深蓋藏。



 望坎位,奏《皇夏》:司筵撤席,掌禮移次,回顧封壇,恭臨坎位。瘞玉埋俎,藏芬斂氣。是曰就幽,成斯地意。



 祀五帝歌辭:奠玉帛,奏《皇夏》辭:嘉玉惟芳,嘉幣惟量。成形依禮,稟色隨方。神班有次,歲禮惟常。威儀抑抑,率由舊章。



 初獻,奏《皇夏》:惟令之月,惟嘉之辰。司壇宿設,掌史誠陳。
 敢用明禮,言功上神。鉤陳旦闢,閶闔朝分。旒垂象冕,樂奏《山云》。將回霆策,暫轉天文。五運周環,四時代序。



 鱗次玉帛,循回樽俎。神其降之,介福斯許。



 皇帝初獻青帝,奏《雲門舞》:甲在日,鳥中星。禮東後,奠蒼靈。樹春旗,命青史。候雁還,東風起。歌木德,舞震宮。泗濱石,龍門桐。孟之月,陽之天。億斯慶,兆斯年。



 皇帝初獻配帝,奏舞:
 帝出於震,蒼德於神。其明在日,其位居春。勞以定國,功以施人。言從配祀,近取諸身。



 皇帝初獻赤帝,奏《雲門舞》:招搖指午對南宮,日月相會實沈中。離光布政動溫風,純陽之月樂炎精。赤雀丹書飛送迎,硃弦絳鼓罄虔誠,萬物含養各長生。



 皇帝獻配帝,奏舞:以炎為政,以火為官,位司南陸,享配離壇。三和實俎,百味浮蘭。神其茂豫,天步艱難。



 皇帝初獻黃帝,奏《雲門舞》:三光儀表正,四氣風雲同。戊己行初歷,黃鐘始變宮。平琮禮內鎮,陰管奏司中。齋壇芝曄曄,清野桂馮馮。夕牢芬六鼎,安歌韻八風。神光乃超忽,佳氣恆蔥蔥。



 皇帝初獻配帝,奏舞:四時咸一德,五氣或同論。猶吹鳳凰管,尚對梧桐園。器圜居土厚,位總配神尊。始知今奏樂,還用我《雲門》。



 皇帝初獻白帝,奏《雲門舞》:
 肅靈兌景,承配秋壇。雲高火落,露白蟬寒。帝律登年,金精行令。瑞獸霜輝,祥禽雪映。司藏肅殺,萬保咸宜。厥田上上,收功在斯。



 皇帝初獻配帝,奏舞:金行秋令,白帝硃宣。司正五雉,歌庸九川。執文之德,對越彼天。介以福祉,君子萬年。



 皇帝初獻黑帝,奏《雲門舞》:北辰為政玄壇,北陸之祀員官。宿設玄圭浴蘭,坎德陰風禦寒。次律將回窮紀,微陽欲動細泉。管猶調於陰竹,聲未入於春弦。待歸餘於送歷,
 方履慶於斯年。



 皇帝初獻配帝,奏舞:地始坼,虹始藏。服玄玉,居玄堂。沐蕙氣,浴蘭湯。匏器潔,水泉香。陟配彼,福無疆。君欣欣,此樂康。



 宗廟歌辭:皇帝入廟門,奏《皇夏》:肅肅清廟,嚴嚴寢門。欹器防滿,金人戒言。應懸鼓,崇牙樹羽。階變升歌,庭紛象舞。閑安象設,緝熙清奠。春鮪初登,新萍先薦。
 人愛然入室,儼乎其位。



 淒愴履之,非寒之謂。



 降神奏《昭夏》:永惟祖武,潛慶靈長。龍圖革命,鳳歷歸昌。功移上墋,德耀中陽。清廟肅肅,猛虡煌煌。曲高大夏,聲和盛唐。牲牷蕩滌,蕭合聲香。和鑾戾止,振鷺來翔。永敷萬國,是則四方。



 俎入,皇帝升階,奏《皇夏》:年祥辯日,上協龜言。奉酎承列,來庭駿奔。雕禾飾斝,翠羽承樽。敬殫如此,恭惟執燔。



 皇帝獻皇高祖,奏《皇夏》:
 慶緒千重秀,鴻源萬里長。無時猶戢翼,有道故韜光。盛德必有後,仁義終克昌。明星初肇慶,大電久呈祥。



 皇帝獻皇曾祖德皇帝,奏《皇夏》:克昌光上烈,基聖穆西籓。崇仁高涉渭,積德被居原。帝圖張往跡,王業茂前尊。重芬德陽廟,疊慶壽陵園。百靈光祖武,千年福孝孫。



 皇帝獻皇祖太祖文皇帝,奏《皇夏》:雄圖屬天造,宏略遇群飛。風雲猶聽命,
 龍躍遂乘機。百二當天險,三分拒樂推。函谷風塵散,河陽氛霧晞。濟弱淪風起,扶危頹運歸。地紐崩還正,天樞落更追。原祠乍超忽,畢隴或綿微。終封三尺劍,長卷一戎衣。



 皇帝獻文宣皇太后,奏《皇夏》:月靈興慶,沙祥發源。功參禹跡,德贊堯門。言容典禮,示俞狄徽章。儀形溫德,令問昭陽。日月不居,歲時宛晚。瑞雲纏心,閟宮惟遠。



 皇帝獻閔皇帝,奏《皇夏》:
 龍圖基代德,天步屬艱難。謳歌還受瑞,揖讓乃登壇。升與芒刺重,入位據關寒。卷舒雲泛濫,游揚日浸微。出鄭終無反,居桐竟不歸。祀夏今惟舊,尊靈謚更追。



 皇帝獻明皇帝,奏《皇夏》:若水逢降君,窮桑屬惟政。丕哉馭帝籙,鬱矣當天命。方定五雲官,先齊八風令。文昌氣似珠,太史河如鏡。南宮學已開,東觀書還聚。文辭金石韻,毫翰風飆豎。清室桂馮馮,齋房芝詡詡。寧思玉管笛,
 空見靈衣舞。



 皇帝獻高祖武皇帝,奏《皇夏》:南河吐雲氣,北斗降星神。百靈咸仰德,千年一聖人。書成紫微動,律定鳳凰馴。六軍命西土,甲子陳東鄰。戎衣此一定,萬里更無塵。煙雲同五色,日月並重輪。流沙既西靜,盤木又東臣。凱樂聞硃雁,鐃歌見白麟。今為六代祀,還得九疑賓。



 皇帝還東壁,飲福酒,奏《皇夏》:禮殫裸獻,樂極休成。長離前掞,宗祀文明。
 縮酌浮蘭,澄罍合鬯。磬折禮容,旋回靈貺。受厘徹俎,飲福移樽。惟光惟烈,文子文孫。



 皇帝還便坐,奏《皇夏》:庭闋四始,筵終三薦。顧步階墀,徘徊餘奠。六龍矯首,七萃警途。鼓移行漏,風轉相烏。翼翼從事,綿綿四時。惟神降嘏,永言保之。



 太祖輔魏之時,高昌款附,乃得其伎,教習以備饗宴之禮。及天和六年,武帝罷掖庭四夷樂。其後帝娉皇后於北狄,得其所獲康國、龜茲等樂,更雜以高昌之舊,並於大司樂習焉。採用其聲,被於鐘石,取《周官》制以陳之。



 明帝武成二年,正月朔旦,會群臣於紫極殿,始用百戲。武帝保定元年,詔罷之。及宣帝即位,而廣召雜伎,增修百戲。魚龍漫衍之伎,常陳殿前,累日繼夜,不知休息。好令城市少年有容貌者,婦人服而歌舞相隨,引入後庭,與宮人觀聽。



 戲樂過度,游幸無節焉。



 武帝以梁鼓吹熊羆十二案,每元正大會,列於懸間,與正樂合奏。宣帝時,革前代鼓吹,制為十五曲。第一,改漢《硃鷺》為《玄精季》,言魏道陵遲,太祖肇開王業也。第二,改漢《思悲翁》為《征隴西》,言太祖起兵,誅侯莫陳悅,掃清隴右也。第三,改漢《艾如張》為《迎魏帝》,言武帝西幸,太祖奉迎,宅關中也。



 第四,
 改漢《上之回》為《平竇泰》,言太祖擁兵討泰,悉擒斬也。第五,改漢《擁離》為《復恆農》,言太祖攻復陜城,關東震肅也。第六,改漢《戰城南》為《克沙苑》,言太祖俘斬齊十萬眾於沙苑,神武脫身至河,單舟走免也。第七,改漢《巫山高》為《戰河陰》,言太祖破神武於河上,斬其將高敖曹、莫多婁貸文也。



 第八,改漢《上陵》為《平漢東》,言太祖命將平隨郡安陸,俘馘萬計也。第九,改漢《將進酒》為《取巴蜀》,言太祖遣軍平定蜀地也。第十,改漢《有所思》為《拔江陵》,言太祖命將擒蕭繹,平南土也。第十一,改漢《芳樹》為《受魏禪》,言閔帝受終於魏,君臨萬國也。第十二,改漢《上邪》為《宣重
 光》,言明帝入承大統,載隆皇道也。第十三,改漢《君馬黃》為哲皇出,言高祖以聖德繼天,天下向風也。第十四,改漢《稚子班》為《平東夏》,言高祖親率六師破齊,擒齊主於青州,一舉而定山東也。第十五,改古《聖人出》為《擒明徹》,言陳將吳明徹侵軼徐部,高祖遣將盡俘其眾也。宣帝晨出夜還,恆陳鼓吹。嘗幸同州,自應門至赤岸,數十里間,鼓樂俱作。祈雨仲山還,令京城士女於衢巷奏樂以迎之。公私頓敝,以至於亡。



 高祖既受命,定令,宮懸四面各二虡,通十二鎛鐘,為二十虡。虡各一人。建鼓四人,祝敔各一人。歌、琴、瑟、簫、築、箏、
 掐箏、臥箜篌、小琵琶,四面各十人,在編磬下。笙、竽、長笛、橫笛、簫、篳篥、篪、熏,四面各八人,在編鐘下,舞各八佾。宮懸簨虡,金五博山,飾以旒蘇樹羽。其樂器應漆者,天地之神皆硃,宗廟加五色漆畫。天神懸內加雷鼓,地祇加靈鼓,宗廟加路鼓,登歌,鐘一虡,磬一虡,各一人;歌四人,兼琴瑟;簫、笙、竽、橫笛、篪、熏各一人。其漆畫及博山旒蘇樹羽,與宮懸同。登歌人介幘、硃連裳、烏皮履。宮懸及下管人,平巾幘,硃連裳。凱樂人,武弁,硃褠衣,履襪。文舞,進賢冠,絳紗連裳,帛內單,皁領袖褠,烏皮,左執籥,右執翟。二人執纛,引前,在舞人數外,衣冠同舞人。武弁,硃褠
 衣,烏皮履。三十二人,執戈,龍楯。三十二人執戚,龜。二人執旍,居前。二人執鞀,二人執鐸,二人執鐃,二人執金享。四人執弓矢,四人執殳,四人執戟,四人執矛。自旍已下夾引,並在舞人數外,衣冠同舞人。



 皇帝宮懸及登歌,與前同。應漆者皆五色漆畫。懸內不設鼓。



 皇太子軒懸,去南面,設三鎛鐘於辰丑申。三建鼓亦如之。其登歌,去兼歌者,減二人。其簨虡金三博山。樂器漆者,皆硃漆之。其餘與宮懸同。



 大鼓、小鼓、大駕鼓吹,並硃漆畫。大鼓加金鐲,凱樂及節
 鼓,飾以羽葆。其長鳴、中鳴、橫吹,皆五採衣幡,緋掌,畫交龍,五採腳。大角幡亦如之。大鼓、長鳴工人,皁地苣文;金鉦、㭎鼓、小鼓、中鳴、吳橫吹工人,青地苣文;凱樂工人,武弁,硃褠衣,橫吹,緋地苣文。並為帽、褲褶。大角工人,平巾幘、緋衫,白布大口褲。內宮鼓樂服色,皆準此。



 皇太子鐃及節鼓,硃漆畫,飾以羽葆。餘鼓吹並硃漆。大鼓、小鼓無金鐲。長鳴、中鳴、橫吹,五採衣幡,緋掌,畫蹲獸,五採腳。大角幡亦如之。大鼓、長鳴、橫吹工人,紫帽,緋褲褶。金鉦、㭎鼓、小鼓、中鳴工人,青帽,青褲褶。鐃吹工人,武弁,硃褠衣。大角工人,平巾幘,緋衫,白布大口褲。



 正一品,鐃及節鼓,硃漆畫,飾以羽葆。餘鼓吹並硃漆。長鳴、中鳴、橫吹,五採衣幡,緋掌,畫蹲獸,五採腳。大角幡亦如之。大鼓、長鳴、橫吹工人,紫帽,赤布褲褶。金鉦、㭎鼓、小鼓、中鳴工人,青帽,青布褲褶。饒吹工人,武弁,硃褠衣。大角工人,平巾幘,緋衫,白布大口褲。三品以上,硃漆饒,飾以五採。騶、哄工人,武弁,硃褠衣。餘同正一品。四品,鐃及工人衣服同三品。餘鼓皆綠沈。



 金鉦、㭎鼓、大鼓工人,青帽,青布褲褶。



 開皇二年,齊黃門侍郎顏之推上言:「禮崩樂壞,其來自久。今太常雅樂,並用胡聲,請馮梁國舊事,考尋古典。」高
 祖不從,曰:「梁樂亡國之音,奈何遣我用邪?」是時尚因周樂,命工人齊樹提檢校樂府,改換聲律,益不能通。俄而柱國、沛公鄭譯奏上,請更修正。於是詔太常卿牛弘、國子祭酒辛彥之、國子博士何妥等議正樂。然淪謬既久,音律多乖,積年議不定。高祖大怒曰:「我受天命七年,樂府猶歌前代功德邪?」命治書侍御史李諤引弘等下,將罪之。諤奏:「武王克殷,至周公相成王,始制禮樂。斯事體大,不可速成。」高祖意稍解。又詔求知音之士,集尚書,參定音樂。譯云:「考尋樂府鐘石律呂,皆有宮、商、角、徵、羽、變宮、變徵之名。七聲之內,三聲乖應,每恆求訪,終莫能通。
 先是周武帝時,有龜茲人曰蘇祗婆,從突厥皇后入國,善胡琵琶。聽其所奏,一均之中間有七聲。因而問之,答云:『父在西域,稱為知音。代相傳習,調有七種。』以其七調,勘校七聲,冥若合符。一曰『娑陀力』,華言平聲,即宮聲也。二曰『雞識』,華言長聲,即商聲也。三曰『沙識』,華言質直聲,即角聲也。四曰『沙侯加濫』,華言應聲,即變徵聲也。五曰『沙臘』,華言應和聲,即徵聲也。六曰『般贍』,華言五聲,即羽聲也。七曰『俟利』,華言斛牛聲,即變宮聲也。」譯因習而彈之,始得七聲之正。然其就此七調,又有五旦之名,旦作七調。以華言譯之,旦者則謂均也。其聲亦應黃鐘、
 太簇、林鐘、南呂、姑洗五均,已外七律,更無調聲。譯遂因其所捻琵琶弦柱相飲為均,推演其聲,更立七均。合成十二,以應十二律。律有七音,音立一調,故成七調十二律,合八十四調,旋轉相交,盡皆和合。仍以其聲考校太樂所奏,林鐘之宮,應用林鐘為宮,乃用黃鐘為宮;應用南呂為商,乃用太簇為商;應用應鐘為角,乃取姑洗為角。故林鐘一宮七聲,三聲並戾。其十一宮七十七音,例皆乖越,莫有通者,又以編懸有八,因作八音之樂。七音之外,更立一聲,謂之應聲。譯因作書二十餘篇,以明其指。至是譯以其書宣示朝廷,並立議正之。時邳國公世
 子蘇夔,亦稱明樂,駁譯曰:「《韓詩外傳》所載樂聲感人,及《月令》所載五音所中,並皆有五,不言變宮、變徵。又《春秋左氏》所云:『七音六律,以奉五聲。』準此而言,每宮應立五調,不聞更加變宮、變徵二調為七調。七調之作,所出未詳。」譯答之曰:「周有七音之律,《漢書·律歷志》,天地人及四時,謂之七始。黃鐘為天始,林鐘為地始,太簇為人始,是為三始。姑洗為春,蕤賓為夏,南呂為秋,應鐘為冬,是為四時。四時三始,是以為七。今若不以二變為調曲,則是冬夏聲闕,四時不備。是故每宮須立七調。」眾從譯議。譯又與夔俱云:「案今樂府黃鐘,乃以林鐘為調首,失君臣
 之義,清樂黃鐘宮,以小呂為變徵,乖相生之道。今請雅樂黃鐘宮以黃鐘為調首,清樂去小呂,還用蕤賓為變徵。」眾皆從之。



 夔又與譯議,欲累黍立分,正定律呂。時以音律久不通,譯、夔等一朝能為之,以為樂聲可定。而何妥舊以學聞,雅為高祖所信。高祖素不悅學,不知樂,妥又恥己宿儒,不逮譯等,欲沮壞其事。乃立議非十二律旋相為宮,曰:「經文雖道旋相為宮,恐是直言其理,亦不通隨月用調,是以古來不取。若依鄭玄及司馬彪,須用六十律方得和韻。今譯唯取黃鐘之正宮,兼得七始之妙義。非止金石諧韻,亦乃簨虡不繁,可以享百神,可以
 合萬舞矣。」而又非其七調之義,曰:「近代書記所載,縵樂鼓琴吹笛之人,多云三調。三調之聲,其來久矣。請存三調而已。」時牛弘總知樂事,弘不能精知音律。又有識音人萬寶常,修洛陽舊曲,言幼學音律,師於祖孝徵,知其上代修調古樂。周之璧翣,殷之崇牙,懸八用七,盡依《周禮》備矣。



 所謂正聲,又近前漢之樂,不可廢也。是時競為異議,各立朋黨,是非之理,紛然淆亂。或欲令各修造,待成,擇其善者而從之。妥恐樂成,善惡易見,乃請高祖張樂試之。遂先說曰:「黃鐘者,以象人君之德。」及奏黃鐘之調,高祖曰:「滔滔和雅,甚與我心會。」妥因陳用黃鐘一宮,
 不假餘律,高祖大悅,班賜妥等修樂者。



 自是譯等議寢。



\end{pinyinscope}