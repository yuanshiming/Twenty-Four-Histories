\article{卷四十一列傳第六 高熲}

\begin{pinyinscope}

 高熲,字昭玄,一名敏,自云渤海蓚人也。父賓,背齊歸周,大司馬獨孤信引為僚佐,賜姓獨孤氏。及信被誅,妻子徙蜀。文獻皇后以賓父之故吏,每往來其家。



 賓後官至鄀州刺史,及熲貴,贈禮部尚書、渤海公。



 熲少明敏,有器局,略涉書史,尤善詞令。初,孩孺時,家有柳樹,高百許尺,亭亭如蓋。里中父老曰:「此家當出貴人。」年十七,周齊王
 憲引為記室。武帝時,襲爵武陽縣伯,除內史上士,尋遷下大夫。以平齊功,拜開府。尋從越王盛擊隰州叛胡,平之。高祖得政,素知熲強明,又習兵事,多計略,意欲引之入府,遣邗國公楊惠諭意。熲承旨欣然曰:「願受驅馳。縱令公事不成,熲亦不辭滅族。」於是為相府司錄。時長史鄭譯、司馬劉昉並以奢縱被疏,高祖彌屬意於熲,委以心膂。



 尉迥之起兵也,遣子惇率步騎八萬,進屯武陟。高祖令韋孝寬擊之,軍至河陽,莫敢先進。高祖以諸將不一,令崔仲方監之,仲方辭父在山東。時熲又見劉昉、鄭譯並無去意,遂自請行,深合上旨,遂遣熲。熲受命便發,
 遣人辭母,云忠孝不可兩兼,歔欷就路。至軍,為橋于沁水,賊於上流縱大伐,熲預為土狗以御之。既渡,焚橋而戰,大破之。遂至鄴下,與迥交戰,仍共宇文忻、李詢等設策,因平尉迥。



 軍還,侍宴於臥內,上撤禦帷以賜之。進位柱國,改封義寧縣公,遷相府司馬,任寄益隆。



 高祖受禪,拜尚書左僕射,兼納言,進封渤海郡公,朝臣莫與為比,上每呼為獨孤而不名也。熲深避權勢,上表遜位,讓於蘇威。上欲成其美,聽解僕射。數日,上曰:「蘇威高蹈前朝,熲能推舉。吾聞進賢受上賞,寧可令去官!」於是命熲復位。俄拜左衛大將軍,本官如故。時突厥屢為寇患,詔熲
 鎮遏緣邊。及還,賜馬百餘匹,牛羊千計。領新都大監,制度多出於熲。熲每坐朝堂北槐樹下以聽事,其樹不依行列,有司將伐之。上特命勿去,以示後人。其見重如此。又拜左領軍大將軍,餘官如故。母憂去職,二旬起令視事。熲流涕辭讓,優詔不許。



 開皇二年,長孫覽、元景山等伐陳,令熲節度諸軍。會陳宣帝薨,熲以禮不伐喪,奏請班師。蕭巖之叛也,詔熲綏集江漢,甚得人和。上嘗問熲取陳之策,熲曰:「江北地寒,田收差晚,江南土熱,水田早熟。量彼收積之際,微徵士馬,聲言掩襲。彼必屯兵禦守,足得廢其農時。彼既聚兵,我便解甲,再三若此,賊以為
 常。



 後更集兵,彼必不信,猶豫之頃,我乃濟師,登陸而戰,兵氣益倍。又江南土薄,舍多竹茅,所有儲積,皆非地窖。密遣行人,因風縱火,待彼修立,復更燒之。不出數年,自可財力俱盡。」上行其策,由是陳人益敝。九年,晉王廣大舉伐陳,以熲為元帥長史,三軍諮稟,皆取斷於熲。及陳平,晉王欲納陳主寵姬張麗華。熲曰:「武王滅殷,戮妲己。今平陳國,不宜取麗華。」乃命斬之,王甚不悅。及軍還,以功加授上柱國,進爵齊國公,賜物九千段,定食千乘縣千五百戶。上因勞之曰:「公伐陳後,人言公反,朕已斬之。君臣道合,非青蠅所間也。」熲又遜位,詔曰:「公識鑒通遠,
 器略優深,出參戎律,廓清淮海,入司禁旅,實委心腹。自朕受命,常典機衡,竭誠陳力,心跡俱盡。此則天降良輔,翊贊朕躬,幸無詞費也。」其優獎如此。



 是後右衛將軍龐晃及將軍盧賁等,前後短熲於上。上怒之,皆被疏黜。因謂熲曰:「獨孤公猶鏡也,每被磨瑩,皎然益明。」未幾,尚書都事姜曄、楚州行參軍李君才並奏稱水旱不調,罪由高熲,請廢黜之。二人俱得罪而去,親禮逾密。上幸並州,留熲居守。及上還京,賜縑五千匹,復賜行宮一所,以為莊舍。其夫人賀拔氏寢疾,中使顧問,絡繹不絕。上親幸其第,賜錢百萬,絹萬匹,復賜以千里馬。



 上嘗從容命熲
 與賀若弼言及平陳事,熲曰:「賀若弼先獻十策,後於蔣山苦戰破賊。



 臣文吏耳,焉敢與大將軍論功!」帝大笑,時論嘉其有讓。尋以其子表仁取太子勇女,前後賞賜不可勝計。時熒惑入太微,犯左執法。術者劉暉私言於熲曰:「天文不利宰相,可修德以禳之。」熲不自安,以暉言奏之。上厚加賞慰。突厥犯塞,以熲為元帥,擊賊破之。又出白道,進圖入磧,遣使請兵。近臣緣此言熲欲反,上未有所答,熲亦破賊而還。



 時太子勇失愛於上,潛有廢立之意。謂熲曰:「晉王妃有神憑之,言王必有天下,若之何?」熲長跪曰:「長幼有序,其可廢乎!」上默然而止,獨孤皇后知
 熲不可奪,陰欲去之,夫人卒,後言於上曰:「高僕射老矣,而喪夫人,陛下何能不為之娶!」上以後言謂熲,熲流涕謝曰:「臣今已老,退朝之後,唯齋居讀佛經而已。雖陛下垂哀之深,至於納室,非臣所願。」上乃止。至是,熲愛妾產男,上聞之極歡,後甚不悅。上問其故,後曰:「陛下當復信高熲邪?始陛下欲為熲娶,熲心存愛妾,面欺陛下。今其詐已見,陛下安得信之!」上由是疏熲。會議伐遼東,熲固諫不可。上不從,以熲為元帥長史,從漢王征遼東,遇霖潦疾疫,不利而還。



 後言於上曰:「熲初不欲行,陛下強遣之,妾固知其無功矣。」又上以漢王年少,專委軍於熲。
 熲以任寄隆重,每懷至公,無自疑之意。諒所言多不用,甚銜之。及還,諒泣言於後曰:「兒幸免高熲所殺。」上聞之,彌不平。俄而上柱國王世積以罪誅,當推核之際,乃有宮禁中事,云於熲處得之。上欲成熲之罪,聞此大驚。時上柱國賀若弼、吳州總管宇文彌、刑部尚書薛胄、民部尚書斛律孝卿、兵部尚書柳述等明熲無罪,上逾怒,皆以之屬吏。自是朝臣莫敢言者。熲竟坐免,以公就第。



 未幾,上幸秦王俊第,召熲侍宴。熲歔欷悲不自勝,獨狐皇后亦對之泣,左右皆流涕。上謂熲曰:「朕不負公,公自負也。」因謂侍臣曰:「我於高熲勝兒子,雖或不見,常似目前。
 自其解落,瞑然忘之,如本無高熲。不可以身要君,自云第一也。」



 頃之,熲國令上熲陰事,稱:「其子表仁謂熲曰:『司馬仲達初托疾不朝,遂有天下。公今遇此,焉知非福!』」於是上大怒,囚熲於內史省而鞫之。憲司復奏熲他事,云:「沙門真覺嘗謂熲云:『明年國有大喪。』尼令暉復云:『十七、十八年,皇帝有大厄。十九年不可過。』上聞而益怒,顧謂群臣曰:「帝王豈可力求!



 孔子以大聖之才,作法垂世,寧不欲大位邪?天命不可耳。熲與子言,自比晉帝,此何心乎?」有司請斬熲。上曰:「去年殺虞慶則,今茲斬王世積,如更誅熲,天下其謂我何?」於是除名為民。熲初為僕射,其
 母誡之曰:「汝富貴已極,但有一斫頭耳,爾宜慎之!」熲由是常恐禍變。及此,熲歡然無恨色,以為得免於禍。



 煬帝即位,拜為太常。時詔收周、齊故樂人及天下散樂。熲奏曰:「此樂久廢。



 今或征之,恐無識之徒棄本逐末,遞相教習。」帝不悅。帝時侈靡,聲色滋甚,又起長城之役。熲甚病之,謂太常丞李懿曰:「周天元以好樂而亡,殷鑒不遙,安可復爾!」時帝遇啟民可汗恩禮過厚,熲謂太府卿何稠曰:「此虜頗知中國虛實、山川險易,恐為後患。」復謂觀王雄曰:「近來朝廷殊無綱紀。」有人奏之,帝以為謗訕朝政,於是下詔誅之,諸子徙邊。



 熲有文武大略,明達世務。及
 蒙任寄之後,竭誠盡節,進引貞良,以天下為己任。蘇威、楊素、賀若弼、韓擒等,皆熲所推薦,各盡其用,為一代名臣。自餘立功立事者,不可勝數。當朝執政將二十年,朝野推服,物無異議。治致升平,熲之力也,論者以為真宰相。及其被誅,天下莫不傷惜,至今稱冤不已。所有奇策密謀及損益時政,熲皆削稿,世無知者。



 其子盛道,官至莒州刺史,徙柳城而卒。次弘德,封應國公,晉王府記室。次表仁,封渤海郡公,徙蜀郡。



 蘇威子夔蘇威,字無畏,京兆武功人也。父綽,魏度支尚書。威少有
 至性,五歲喪父,哀毀有若成人。周太祖時,襲爵美陽縣公,仕郡功曹。大塚宰宇文護見而禮之,以其女新興主妻焉。見護專權,恐禍及己,逃入山中,為叔父所逼,卒不獲免。然威每屏居山寺,以諷讀為娛。未幾,授使持節、車騎大將軍、儀同三司,改封懷道縣公。武帝親總萬機,拜稍伯下大夫。前後所授,並辭疾不拜。有從父妹者,適河南元雄。雄先與突厥有隙,突厥入朝,請雄及其妻子,將甘心焉。周遂遣之。威曰:「夷人昧利,可以賂動。」遂標賣田宅,罄家所有以贖雄,論者義之。宣帝嗣位,就拜開府。



 高祖為丞相,高熲屢言其賢,高祖亦素重其名,召之。及至,
 引入臥內,與語大悅。居月餘,威聞禪代之議,遁歸田里。高熲請追之,高祖曰:「此不欲預吾事,且置之。」及受禪,徵拜太子少保。追贈其父為邳國公,邑三千戶,以威襲焉。俄兼納言、民部尚書。威上表陳讓,詔曰:「舟大者任重,馬駿者遠馳。以公有兼人之才,無辭多務也。」威乃止。



 初,威父在西魏,以國用不足,為征稅之法,頗稱為重。既而嘆曰:「今所為者,正如張弓,非平世法也。後之君子,誰能弛乎?」威聞其言,每以為己任。至是,奏減賦役,務從輕典,上悉從之。漸見親重,與高熲參掌朝政。威見宮中以銀為幔鉤,因盛陳節儉之美以諭上。上為之改容,雕飾舊物,
 悉命除毀。上嘗怒一人,將殺之,威入閤進諫,不納。上怒甚,將自出斬之,威當上前不去。上避之而出,威又遮止。上拂衣而入。良久,乃召威謝曰:「公能若是,吾無憂矣。」於是賜馬二匹,錢十餘萬。尋復兼大理卿、京兆尹、御史大夫,本官悉如故。



 治書侍御史梁毗以威領五職,安繁戀劇,無舉賢自代之心,抗表劾威。上曰:「蘇威朝夕孜孜,志存遠大,舉賢有闕,何遽迫之!」顧謂威曰:「用之則行,舍之則藏,唯我與爾有是夫!」因謂朝臣曰:「蘇威不值我,無以措其言;我不得蘇威,何以行其道?楊素才辯無雙,至若斟酌古今,助我宣化,非威之匹也。蘇威若逢亂世,南山
 四皓,豈易屈哉!」其見重如此。



 未幾,拜刑部尚書,解少保、御史大夫之官。後京兆尹廢,檢校雍州別駕。時高熲與威同心協贊,政刑大小,無不籌之,故革運數年,天下稱治。俄轉民部尚書,納言如故。屬山東諸州民饑,上令威賑恤之。後二載,遷吏部尚書。歲餘,兼領國子祭酒。隋承戰爭之後,憲章踳駁,上令朝臣厘改舊法,為一代通典。律令格式,多威所定,世以為能。九年,拜尚書右僕射。其年,以母憂去職,柴毀骨立。上敕威曰:「公德行高人,情寄殊重,大孝之道,蓋同俯就。必須抑割,為國惜身。朕之於公,為君為父,宜依朕旨,以禮自存。」未幾,起令視事,固辭,
 優詔不許。



 明年,上幸並州,命與高熲同總留事。俄追詣行在所,使決民訟。



 威子夔,少有盛名於天下,引致賓客,四海士大夫多歸之。後議樂事,夔與國子博士何妥各有所持。於是夔、妥俱為一議,使百僚署其所同。朝廷多附威,同夔者十八九。妥恚曰:「吾席間函丈四十餘年,反為昨暮兒之所屈也!」遂奏威與禮部尚書盧愷、吏部侍郎薛道衡、尚書右丞王弘、考功侍郎李同和等共為朋黨,省中呼王弘為世子,李同和為叔,言二人如威之子弟也。復言威以曲道任其從父弟徹、肅等罔冒為官。又國子學請蕩陰人王孝逸為書學博士,威屬盧愷,以為
 其府參軍。



 上令蜀王秀、上柱國虞慶則等雜治之,事皆驗。上以《宋書·謝晦傳》中朋黨事令威讀之。威惶懼,免冠頓首。上曰:「謝已晚矣。」於是免威官爵,以開府就第。



 知名之士坐威得罪者百餘人。未幾,上曰:「蘇威德行者,但為人所誤耳。」命之通籍。歲餘,復爵邳公,拜納言。從祠太山,坐不敬免。俄而復位。上謂群臣曰:「世人言蘇威詐清,家累金玉,此妄言也。然其性狠戾,不切世要,求名太甚,從己則悅,違之必怒,此其大病耳。」尋令持節巡撫江南,得以便宜從事。過會稽,逾五嶺而還。時突厥都藍可汗屢為邊患,復使威至可汗所,與結和親。可汗即遣使獻
 方物。以勤勞,進位大將軍。仁壽初,復拜尚書右僕射。上幸仁壽宮,以威總留後事。及上還,御史奏威職事多不理,請推之。上怒,詰責威。威拜謝,上亦止。



 後上幸仁壽宮,不豫,皇太子自京師來侍疾,詔威留守京師。



 煬帝嗣位,加上大將軍。及長城之役,威諫止之。高熲、賀若弼等之誅也,威坐與相連,免官。歲餘,拜魯郡太守。俄召還,參預朝政。未幾,拜太常卿。其年從征吐谷渾,進位左光祿大夫。帝以威先朝舊臣,漸加委任。後歲餘,復為納言。



 與左翊衛大將軍宇文述、黃門侍郎裴矩、御史大夫裴蘊、內史侍郎虞世基參掌朝政,時人稱為「五貴」。及遼東之役,
 以本官領左武衛大將軍,進位光祿大夫,賜爵寧陵侯。其年,進封房公。威以年老,上表乞骸骨。上不許,復以本官參掌選事。明年,從征遼東,領右御衛大將軍。



 楊玄感之反也,帝引威帳中,懼見於色,謂威曰:「此小兒聰明,得不為患乎?」



 威曰:「夫識是非,審成敗者,乃所謂聰明。玄感粗疏,非聰明者,必無所慮。但恐浸成亂階耳。」威見勞役不息,百姓思亂,微以此諷帝,帝竟不寤。從還至涿郡,詔威安撫關中。以威孫尚輦直長儇為副。其子鴻臚少卿夔,先為關中簡黜大使,一家三人,俱奉使關右,三輔榮之。歲餘,帝下手詔曰:「玉以潔潤,丹紫莫能渝其質;松表
 歲寒,霜雪莫能凋其採。可謂溫仁勁直,性之然乎!房公威器懷溫裕,識量弘雅,早居端揆,備悉國章,先皇舊臣,朝之宿齒。棟梁社稷,弼諧朕躬,守文奉法,卑身率禮。昔漢之三傑,輔惠帝者蕭何;周之十亂,佐成王者邵奭。國之寶器,其在得賢,參燮臺階,具瞻斯允。雖復事藉論道,終期獻替,銓衡時務,朝寄為重,可開府儀同三司,餘並如故。」威當時見尊重,朝臣莫與為比。



 後從幸雁門,為突厥所圍,朝廷危憚。帝欲輕騎潰圍而出,威諫曰:「城守則我有餘力,輕騎則彼之所長。陛下萬乘之主,何宜輕脫!」帝乃止。突厥俄亦解圍而去。車駕至太原,威言於帝曰:「
 今者盜賊不止,士馬疲敝。願陛下還京師,深根固本,為社稷之計。」帝初然之,竟用宇文述等議,遂往東都。時天下大亂,威知帝不可改,意甚患之。屬帝問侍臣盜賊事,宇文述曰:「盜賊信少,不足為虞。」



 威不能詭對,以身隱於殿柱。帝呼威而問之。威對曰:「臣非職司,不知多少,但患其漸近。」帝曰:「何謂也?」威曰:「他日賊據長白山,今者近在滎陽、汜水。」



 帝不悅而罷。尋屬五月五日,百僚上饋,多以珍玩。威獻《尚書》一部,微以諷帝,帝彌不平。後復問伐遼東事,威對願赦群盜,遣討高麗,帝益怒。御史大夫裴蘊希旨,令白衣張行本奏威昔在高陽典選,濫授人官,畏
 怯突厥,請還京師。帝令案其事。及獄成,下詔曰:「威立性朋黨,好為異端,懷挾詭道,徼幸名利,詆訶律令,謗訕臺省。昔歲薄伐,奉述先志,凡預切問,各盡胸臆,而威不以開懷,遂無對命。



 啟沃之道,其若是乎!資敬之義,何其甚薄!」於是除名為民。後月餘,有人奏威與突厥陰圖不軌者,大理簿責威。威自陳奉事二朝三十餘載,精誠微淺不能上感,咎釁屢彰,罪當萬死。帝憫而釋之。其年從幸江都宮,帝將復用威。裴蘊、虞世基奏言昏耄贏疾。帝乃止。



 宇文化及之弒逆也,以威為光祿大夫、開府儀同三司。化及敗,歸於李密。未幾密敗,歸東都,越王侗以為上
 柱國、邳公。王充僭號,署太師。威自以隋室舊臣,遭逢喪亂,所經之處,皆與時消息,以求容免。及大唐秦王平王充,坐於東都閶闔門內,威請謁見,稱老病不能拜起。王遣人數之曰:「公隋朝宰輔,政亂不能匡救,遂令品物塗炭,君弒國亡。見李密、王充,皆拜伏舞蹈。今既老病,無勞相見也。」



 尋歸長安,至朝堂請見,又不許。卒於家。時年八十二。



 威治身清儉,以廉慎見稱。每至公議,惡人異己,雖或小事,必固爭之。時人以為無大臣之體。所修格令章程,並行於當世,然頗傷苛碎,論者以為非簡允之法。



 及大業末年,尤多征役,至於論功行賞,威每承望風旨,輒
 寢其事。時群盜蜂起,郡縣有表奏詣闕者,又訶詰使人,令減賊數。故出師攻討,多不克捷。由是為物議所譏。子夔。



 夔字伯尼,少聰敏,有口辯。八歲誦詩書,兼解騎射。年十三,從父至尚書省,與安德王雄馳射,賭得雄駿馬而歸。十四詣學,與諸儒論議,詞致可觀,見者莫不稱善。及長,博覽群言,尤以鐘律自命。初不名夔,其父改之,頗為有識所哂。起家太子通事舍人。楊素甚奇之,素每戲威曰:「楊素無兒,蘇夔無父。」後與沛國公鄭譯、國子博士何妥議樂,因而得罪,議寢不行。著《樂志》十五篇,以見其志。



 數
 載,遷太子舍人。後加武騎尉。仁壽末,詔天下舉達禮樂之源者,晉王昭時為雍州牧,舉夔應之。與諸州所舉五十餘人謁見,高祖望夔謂侍臣:「唯此一人,稱吾所舉。」於是拜晉王友。煬帝嗣位,遷太子洗馬,轉司朝謁者。以父免職,夔亦去官。後歷尚書職方郎、燕王司馬。遼東之役,夔領宿衛,以功拜朝散大夫。時帝方勤遠略,蠻夷朝貢,前後相屬。帝嘗從容謂宇文述、虞世基等曰:「四夷率服,觀禮華夏,鴻臚之職,須歸令望。寧有多才藝,美容儀,可以接對賓客者為之乎?」



 咸以夔對。帝然之,即日拜鴻臚少卿。其年,高昌王曲伯雅來朝,朝廷妻以公主。



 夔有雅
 望,令主婚焉。其後弘化、延安等數郡盜賊蜂起,所在屯結,夔奉詔巡撫關中。突厥之圍雁門也,夔領城東面事。夔為弩樓車箱獸圈,一夕而就。帝見而善之,以功進位通議大夫。坐父事,除名為民。復丁母憂,不勝哀而卒,時年四十九。



 史臣曰:齊公霸圖伊始,早預經綸,魚水冥符,風雲玄感。正身直道,弼諧與運,心同契合,言聽計從。東夏克平,南國底定,參謀帷幄,決勝千里。高祖既復禹跡,思布堯心,舟楫是寄,鹽梅斯在。兆庶賴以康寧,百僚資而輯睦,年將二紀,人無間言。屬高祖將廢儲宮,由忠信而得罪;逮
 煬帝方逞浮侈,以忤時而受戮。若使遂無猜釁,克終厥美,雖未可參縱稷、契,足以方駕蕭、曹。繼之實難,惜矣!



 邳公周道云季,方事幽貞;隋室龍興,首應旌命。綢繆任遇,窮極榮寵;久處機衡,多所損益;罄竭心力,知無不為。然志尚清儉,體非弘曠,好同惡異,有乖直道,不存易簡,未為通德。歷事二帝,三十餘年,雖廢黜當時,終稱遺老。君邪而不能正言,國亡而情均眾庶。予違汝弼,徒聞其語;疾風勁草,未見其人。禮命闕於興王,抑亦此之由也。夔志識沉敏,方雅可稱,若天假之年,足以不虧堂構矣。



\end{pinyinscope}