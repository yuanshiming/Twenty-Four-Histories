\article{卷二帝紀第二 高祖下}

\begin{pinyinscope}

 八年春正月乙亥,陳遣散騎常侍袁雅、兼通直散騎常侍周止水來聘。二月庚子,鎮星入東井。辛酉,陳人寇硤州。三月辛未,上柱國、隴西郡公李詢卒。壬申,以成州刺史姜須達為會州總管。甲戌,遣兼散騎常侍程尚賢、兼通直散騎常侍韋惲使於陳。戊寅,詔曰:昔有苗不賓,唐堯薄伐,孫皓僭虐,晉武行誅。有陳竊據江表,逆天暴物。
 朕初受命,陳頊尚存,思欲教之以道,不以龔行為令,往來修睦,望其遷善。時日無幾,釁惡已聞。厚納叛亡,侵犯城戍,勾吳閩越,肆厥殘忍。於時王師大舉,將一車書,陳頊反地收兵,深懷震懼,責躬請約,俄而致殞。矜其喪禍,仍詔班師。叔寶承風,因求繼好,載佇克念,共敦行李。每見珪璪入朝,輶軒出使,何嘗不殷勤曉喻,戒以惟新。而狼子之心,出而彌野。威侮五行,怠棄三正,誅翦骨肉,夷滅才良。據手掌之地,恣溪壑之險,劫奪閭閻,資產俱竭,驅蹙內外,勞役弗已。征責女子,擅造宮室,日增月益,止足無期,帷薄嬪嬙,有逾萬數。寶衣玉食,窮奢極侈,淫聲
 樂飲,俾晝作夜。斬直言之客,滅無罪之家,剖人之肝,分人之血。欺天造惡,祭鬼求恩,歌儛衢路,酣醉宮閫。盛粉黛而執干戈,曳羅綺而呼警蹕,躍馬振策,從旦至昏,無所經營,馳走不息。負甲持仗,隨逐徒行,追而不及,即加罪譴。自古昏亂,罕或能比。介士武夫,饑寒力役,筋髓罄於土木,性命俟於溝渠。



 君子潛逃,小人得志,家家隱殺戳,各各任聚斂。天災地孽,物怪人妖,衣冠鉗口,道路以目。傾心翹足,誓告於我,日月以冀,文奏相尋。重以背德違言,搖蕩疆埸,巴峽之下,海筮已西,江北江南,為鬼為蜮。死隴窮發掘之酷,生居極攘奪之苦。



 抄掠人畜,斷截
 樵蘇,市井不立,農事廢寢。歷陽廣陵,窺覦相繼,或謀圖城邑,或劫剝吏人,晝伏夜游,鼠竄狗盜。彼則羸兵敝卒,來必就擒,此則重門設險,有勞籓捍。天之所覆,無非朕臣,每關聽覽,有懷傷惻。有梁之國,我南籓也,其君入朝,潛相招誘,不顧朕恩。士女深迫脅之悲,城府致空虛之嘆。非直朕居人上,懷此無忘,既而百闢屢以為言,兆庶不堪其請,豈容對而不誅,忍而不救!近日秋始,謀欲吊人。益部樓船,盡令東騖,便有神龍數十,騰躍江流,引伐罪之師,向金陵之路,船住則龍止,船行則龍去,四日之內,三軍皆睹,豈非蒼旻愛人,幽明展事,降神先路,協贊
 軍威!以上天之靈,助戡定之力,便可出師授律,應機誅殄,在斯舉也,永清吳越。其將士糧仗,水陸資須,期會進止,一準別敕。



 秋八月丁未,河北諸州饑,遣吏部尚書蘇威賑恤之。九月丁丑,宴南征諸將,頒賜各有差。癸巳,嘉州言龍見。冬十月己亥,太白出西方。己未,置淮南行臺省於壽春,以晉王廣為尚書令。辛酉,陳遣兼散騎常侍王琬、兼通直散騎常侍許善心來聘,拘留不遣。甲子,將伐陳,有事於太廟。命晉王廣、秦王俊、清河公楊素並為行軍元帥以伐陳。於是晉王廣出六合,秦王俊出襄陽,清河公楊素出信州,荊州刺史劉仁恩出江陵,宜陽公
 王世積出蘄春,新義公韓擒虎出廬江,襄邑公賀若弼出吳州,落叢公燕榮出東海,合總管九十,兵五十一萬八千,皆受晉王節度。東接滄海,西拒巴蜀,旌旗舟楫,橫亙數千里。曲赦陳國。有星孛於牽牛。十一月丁卯,車駕餞師。詔購陳叔寶位上柱國、萬戶公。乙亥,行幸定城,陳師誓眾。丙子,幸河東。十二月庚子,至自河東。



 九年春正月己巳,白虹夾日。辛未,賀若弼拔陳京口,韓擒虎拔陳南豫州。癸酉,以尚書右僕射虞慶則為右衛大將軍。丙子,賀若弼敗陳師於蔣山,獲其將蕭摩訶。韓擒虎進師入建鄴,獲其將任蠻奴,獲陳主叔寶。陳國平,
 合州三十,郡一百,縣四百。癸巳,遣使持節巡撫之。二月乙未,廢淮南行臺省。丙申,制五百家為鄉,正一人;百家為里,長一人。丁酉,以襄州總管韋世康為安州總管。夏四月己亥,幸驪山,親勞旋師。乙巳,三軍凱入,獻俘於太廟。拜晉王廣為太尉。庚戌,上御廣陽門宴將士,頒賜各有差。辛亥,大赦天下。己未,以陳都官尚書孔範,散騎常侍王瑳、王儀,御史中丞沈觀等,邪佞於其主,以致亡滅,皆投之邊裔。辛酉,以信州總管楊素為荊州總管,吏部侍郎宇文弼為刑部尚書,宗正少卿楊異為工部尚書。



 壬戌,詔曰:往以吳越之野,群黎塗炭,干戈方用,積習未
 寧。今率土大同,含生遂性,太平之法,方可流行。凡我臣僚,澡身浴德,開通耳目,宜從茲始。喪亂已來,緬將十載,君無君德,臣失臣道,父有不慈,子有不孝,兄弟之情或薄,夫婦之義或違,長幼失序,尊卑錯亂。朕為帝王,志存愛養,時有臻道,不敢寧息。內外職位,遐邇黎人,家家自修,人人克念,使不軌不法,蕩然俱盡。兵可立威,不可不戢,刑可助化,不可專行。禁衛九重之餘,鎮守四方之外,戎旅軍器,皆宜停罷。代路既夷,群方無事,武力之子,俱可學文,人間甲仗,悉皆除毀。有功之臣,降情文藝,家門子侄,各守一經,令海內翕然,高山仰止。京邑庠序,爰及
 州縣,生徒受業,升進於朝,未有灼然明經高第,此則教訓不篤,考課未精,明勒所由,隆茲儒訓。官府從宦,丘園素士,心跡相表,寬弘為念,勿為跼促,乖我皇猷。朕君臨區宇,於茲九載,開直言之路,披不諱之心,形於顏色,勞於興寢。自頃逞藝論功,昌言乃眾,推誠切諫,其事甚疏。公卿士庶,非所望也,各啟至誠,匡茲不逮。見善必進,有才必舉,無或噤默,退有後言。頒告天下,咸悉此意。



 閏月甲子,以安州總管韋世康為信州總管。丁丑,頒木魚符於總管、刺史,雌一雄一。己卯,以吏部尚書蘇威為尚書右僕射。六月乙丑,以荊州總管楊素為納言。



 丁丑,以吏
 部侍郎盧愷為禮部尚書。時朝野物議,咸願登封。秋七月丙午,詔曰:「豈可命一將軍,除一小國,遐邇注意,便謂太平。以薄德而封名山,用虛言而干上帝,非朕攸聞。而今以後,言及封禪,宜即禁絕。」八月壬戌,以廣平王雄為司空。冬十一月壬辰,考使定州刺史豆盧通等上表,請封禪,上不許。庚子,以右衛大將軍虞慶則為右武候大將軍,右領軍將軍李安為右領軍大將軍。甲寅,降囚徒。



 十二月甲子,詔曰:「朕祗承天命,清蕩萬方。百王衰敝之後,兆庶澆浮之日,聖人遺訓,掃地俱盡,制禮作樂,今也其時。朕情存古樂,深思雅道。鄭衛淫聲,魚龍雜戲,樂府
 之內,盡以除之。今欲更調律呂,改張琴瑟。且妙術精微,非因教習,工人代掌,止傳糟粕,不足達神明之德,論天地之和。區域之間,奇才異藝,天知神授,何代無哉!蓋晦跡於非時,俟昌言於所好,宜可搜訪,速以奏聞,庶睹一藝之能,共就九成之業。」仍詔太常牛弘、通直散騎常侍許善心、秘書丞姚察、通直郎虞世基等議定作樂。己巳,以黃州總管周法尚為永州總管。



 十年春正月乙未,以皇孫昭為河南王,楷為華陽王。二月庚申,幸並州。夏四月辛酉,至自並州。五月乙未,詔曰:「魏末喪亂,宇縣瓜分,役車歲動,未遑休息。兵士軍人,權
 置坊府,南征北伐,居處無定。家無完堵,地罕包桑,恆為流寓之人,竟無鄉里之號。朕甚愍之。凡是軍人,可悉屬州縣,墾田籍帳,一與民同。



 軍府統領,宜依舊式。罷山東河南及北方緣邊之地新置軍府。」六月辛酉,制人年五十,免役收庸。癸亥,以靈州總管王世積為荊州總管,淅州刺史元胄為靈州總管。



 秋七月癸卯,以納言楊素為內史令。庚戌,上親錄囚徒。辛亥,高麗遼東郡公高陽卒。壬子,吐谷渾遣使來朝。八月壬申,遣柱國、襄陽郡公韋洸,上開府、東萊郡公王景,並持節巡撫嶺南,百越皆服。冬十月甲子,頒木魚符於京師官五品已上。



 戊辰,以永
 州總管周法尚為桂州總管。十一月辛卯,幸國學,頒賜各有差。丙午,契丹遣使朝貢。辛丑,有事於南郊。是月,婺州人汪文進、會稽人高智慧、蘇州人沈玄懀皆舉兵反,自稱天子,署置百官。樂安蔡道人、蔣山李棱、饒州吳代華、永嘉沈孝澈、泉州王國慶、餘杭楊寶英、交趾李春等皆自稱大都督,攻陷州縣。詔上柱國、內史令、越國公楊素討平之。



 十一年春正月丁酉,以平陳所得古器多為妖變,悉命毀之。辛丑,高麗遣使朝貢。丙午,皇太子妃元氏薨,上舉哀於文思殿。二月戊午,吐谷渾遣使貢方物。以大將軍
 蘇孝慈為工部尚書。丙子,以臨潁令劉曠治術尤異,擢為莒州刺史。己卯,突厥遣使獻七寶碗。辛巳晦,日有蝕之。三月壬午,遣通事舍人若干洽使於吐谷渾。



 癸未,以幽州總管周搖為壽州總管,朔州總管吐萬緒為夏州總管。夏四月戊午,突厥雍虞閭可汗遣其特勤來朝。五月甲子,高麗遣使貢方物。癸卯,詔百官悉詣朝堂上封事。乙巳,以右衛將軍元旻為左衛大將軍。秋七月己丑,以柱國杜彥為洪州總管。八月壬申,幸慄園。滕王瓚薨。乙亥,至自慄園。上柱國、沛國公鄭譯卒。十二月丙辰,靺鞨遣使貢方物。



 十二年春正月壬子,以蘇州刺史皇甫績為信州總管,宣州刺史席代雅為廣州總管。二月己巳,以蜀王秀為內史令,兼右領軍大將軍,漢王諒為雍州牧、右衛大將軍。夏四月辛卯,以壽州總管周搖為襄州總管。五月辛亥,廣州總管席代雅卒。秋七月乙巳,尚書右僕射、邳國公蘇威,禮部尚書、容城縣侯盧愷並坐事除名。壬戌,幸昆明池,其日還宮。己巳,有事於太廟。壬申晦,日有蝕之。八月甲戌,制天下死罪,諸州不得便決,皆令大理覆治。乙亥,幸龍首池。癸巳,制宿衛者不得輒離所守。丁酉,上柱國、夏州總管、楚國公豆盧勣卒。戊戌,上親錄囚徒。九
 月丁未,以工部尚書楊異為吳州總管。冬十月丁丑,以遂安王集為衛王。壬午,有事於太廟。



 至太祖神主前,上流涕嗚咽,悲不自勝。十一月辛亥,有事於南郊。壬子,宴百僚,頒賜各有差。己未,上柱國、新義郡公韓擒虎卒。庚申,以豫州刺史權武為潭州總管。甲子,百僚大射於武德殿。十二月癸酉,突厥遣使來朝。乙酉,以上柱國、內史令楊素為尚書右僕射。己酉,吐谷渾、靺鞨並遣使貢方物。



 十三年春正月乙巳,上柱國、郇國公韓建業卒。丙午,契丹、奚、霫、室韋並遣使貢方物。壬子,親祀感帝。己未,以信
 州總管韋世康為吏部尚書。壬戌,行幸岐州。二月丙子,詔營仁壽宮。丁亥,至自岐州。戊子,宴考使於嘉則殿。己卯,立皇孫暕為豫章王。戊子,晉州刺史、南陽郡公賈悉達,顯州總管、撫寧郡公韓延等以賄伏誅。己丑,制坐事去官者,配流一年。丁酉,制私家不得隱藏緯候圖讖。



 夏四月癸未,制戰亡之家,給復一年。五月癸亥,詔人間有撰集國史、臧否人物者,皆令禁絕。秋七月戊申,靺鞨遣使貢方物。壬子,左衛大將軍、雲州總管、鉅鹿郡公賀婁子乾卒。丁巳,幸昆明池。戊辰晦,日有蝕之。九月丙辰,降囚徒。庚申,以邵國公楊綸為滕王。乙丑,以柱國杜彥為
 雲州總管。冬十月乙卯,上柱國、華陽郡公梁彥光卒。



 十四年夏四月乙丑,詔曰:「在昔聖人,作樂崇德,移風易俗,於斯為大。自晉氏播遷,兵戈不息,雅樂流散,年代已多,四方未一,無由辨正。賴上天鑒臨,明神降福,拯茲塗炭,安息蒼生,天下大同,歸於治理,遺文舊物,皆為國有。比命所司,總令研究,正樂雅聲,詳考已訖,宜即施用,見行者停。人間音樂,流僻日久,棄其舊體,競造繁聲,浮宕不歸,遂以成俗。宜加禁約,務存其本。」五月辛酉,京師地震。關內諸州旱。六月丁卯,詔省府州縣,皆給公廨田,不得治生,與人爭利。秋七月乙未,以邳國公蘇威為納言。
 八月辛未,關中大旱,人饑。上率戶口就食於洛陽。九月己未,以齊州刺史樊子蓋為循州總管。丁巳,以基州刺史崔仲方為會州總管。冬閏十月甲寅,詔曰:「齊、梁、陳往皆創業一方,綿歷年代。



 既宗祀廢絕,祭奠無主,興言矜念,良以愴然。莒國公蕭琮及高仁英、陳叔寶等,宜令以時修其祭祀。所須器物,有司給之。」乙卯,制外官九品已上,父母及子年十五已上,不得將之官。十一月壬戌,制州縣佐吏,三年一代,不得重任。癸未,有星孛於角亢。十二月乙未,東巡狩。



 十五年春正月壬戌,車駕次齊州,親問疾苦。丙寅,旅王
 符山。庚午,上以歲旱,祠太山,以謝愆咎。大赦天下。二月丙辰,收天下兵器,敢有私造者,坐之。



 關中緣邊,不在其例。丁巳,上柱國、蔣國公梁睿卒。三月己未,至自東巡狩。望祭五岳海瀆。丁亥,幸仁壽宮。營州總管韋藝卒。夏四月己丑朔,大赦天下。甲辰,以趙州刺史楊達為工部尚書。丁未,以開府儀同三司韋沖為營州總管。五月癸酉,吐谷渾遣使朝貢。丁亥,制京官五品已上,佩銅魚符。六月戊子,詔鑿底柱。庚寅,相州刺史豆盧通貢綾文布,命焚之於朝堂。乙未,林邑遣使來貢方物。辛丑,詔名山大川未在祀典者,悉祠之。秋七月乙丑,晉王廣獻毛龜。甲
 戌,遣邳國公蘇威巡省江南。戊寅,至自仁壽宮。辛巳,制九品已上官以理去職者,聽並執笏。冬十月戊子,以吏部尚書韋世康為荊州總管。十一月辛酉,幸溫湯。乙丑,至自溫湯。十二月戊子,敕盜邊糧一升已上皆斬,並籍沒其家。己丑,詔文武官以四考交代。



 十六年春正月丁亥,以皇孫裕為平原王,筠為安成王,嶷為安平王,恪為襄城王,該為高陽王,韶為建安王,煚為潁川王。夏五月丁巳,以懷州刺史龐晃為夏州總管,蔡陽縣公姚辯為靈州總管。六月甲午,制工商不得進仕。並州大蝗。辛丑,詔九品已上妻、五品已上妾夫亡不
 得改嫁。秋八月丙戌,詔決死罪者,三奏而後行刑。冬十月己丑,幸長春宮。十一月壬子,至自長春宮。



 十七年春二月癸未,太平公史萬歲擊西寧羌,平之。庚寅,幸仁壽宮。庚子,上柱國王世積討桂州賊李光仕,平之。壬寅,河南王昭納妃,宴群臣,頒賜各有差。



 三月丙辰,詔曰:「分職設官,共理時務,班位高下,各有等差。



 若所在官人不相敬憚,多自寬縱,事難克舉。諸有殿失,雖備科條,或據律乃輕,論情則重,不即決罪,無以懲肅。其諸司論屬官,若有愆犯,聽於律外斟酌決杖。」



 辛酉,上親錄囚徒。癸亥,上柱國、彭國公劉昶以罪伏誅。庚午,遣治書侍
 御史柳彧、皇甫誕巡省河南、河北。夏四月戊寅,頒新歷。壬午,詔曰:「周歷告終,群兇作亂,釁起蕃服,毒被生人。朕受命上玄,廓清區宇,聖靈垂祐,文武同心。申明公穆、鄖襄公孝寬、廣平王雄、蔣國公睿、楚國公勣、齊國公熲、越國公素、魯國公慶則、新寧公長叉、宜陽公世積、趙國公羅雲、隴西公詢、廣業公景、真昌公振、沛國公譯、項城公子相、鉅鹿公子幹等,登庸納揆之時,草昧經綸之日,丹誠大節,心盡帝圖,茂績殊勛,力宣王府。宜弘其門緒,與國同休。其世子世孫未經州任者,宜量才升用,庶享榮位,世祿無窮。」五月,宴百僚於玉女泉,頒賜各有差。己巳,
 蜀王秀來朝。高麗遣使貢方物。甲戌,以左衛將軍獨孤羅雲為涼州總管。



 閏月己卯,群鹿入殿門,馴擾侍衛之內。秋七月丁丑,桂州人李代賢反,遣右武候大將軍虞慶則討平之。丁亥,上柱國、並州總管秦王俊坐事免,以王就第。戊戌,突厥遣使貢方物。八月丁卯,荊州總管、上庸郡公韋世康卒。九月甲申,至自仁壽宮。庚寅,上謂侍臣曰:「禮主於敬,皆當盡心。黍稷非馨,貴在祗肅。廟庭設樂,本以迎神,齋祭之日,觸目多感。當此之際,何可為心!在路奏樂,禮未為允。群公卿士,宜更詳之。」冬十月丁未,頒銅獸符於驃騎、車騎府。戊申,道王靜薨。



 庚午,詔曰:「五
 帝異樂,三王殊禮,皆隨事而有損益,因情而立節文。仰惟祭享宗廟,瞻敬如在,罔極之感,情深茲日。而禮畢升路,鼓吹發音,還入宮門,金石振響。斯則哀樂同日,心事相違,情所不安,理實未允。宜改茲往式,用弘禮教。



 自今已後,享廟日不須備鼓吹,殿庭勿設樂懸。」辛未,京師大索。十一月丁亥,突厥遣使來朝。十二月壬子,上柱國、右武候大將軍、魯國公虞慶則以罪伏誅。



 十八年春正月辛丑,詔曰:「吳越之人,往承弊俗,所在之處,私造大船,因相聚結,致有侵害。其江南諸州,人間有船長三丈已上,悉括入官。」二月甲辰,幸仁壽宮。乙巳,以
 漢王諒為行軍元帥,水陸三十萬伐高麗。三月乙亥,以柱國杜彥為朔州總管。夏四月癸卯,以蔣州刺史郭衍為洪州總管。五月辛亥,詔畜貓鬼、蠱毒、厭魅、野道之家,投於四裔。六月丙寅,下詔黜高麗王高元官爵。秋七月壬申,詔以河南八州水,免其課役。丙子,詔京官五品已上,總管、刺史,以志行修謹、清平幹濟二科舉人。九月己丑,漢王諒師遇疾疫而旋,死者十八九。庚寅,敕舍客無公驗者,坐及刺史、縣令。辛卯,至自仁壽宮。冬十一月甲戌,上親錄囚徒。



 癸未,有事於南郊。十二月庚子,上柱國、夏州總管、任城郡公王景以罪伏誅。是月,自京師至仁
 壽宮,置行宮十有二所。



 十九年春正月癸酉,大赦天下。戊寅,大射武德殿,宴賜百官。二月己亥,晉王廣來朝。辛丑,以並州總管長史宇文弼為朔州總管。甲寅,幸仁壽宮。夏四月丁酉,突厥利可汗內附。達頭可汗犯塞,遣行軍總管史萬歲擊破之。六月丁酉,以豫章王暕為內史令。秋八月癸卯,上柱國、尚書左僕射、齊國公高熲坐事免。辛亥,上柱國、皖城郡公張威卒。甲寅,上柱國、城陽郡公李徹卒。九月乙丑,以太常卿牛弘為吏部尚書。冬十月甲午,以突厥利可汗為啟人可汗,築大利城處其部落。庚子,以朔州總管宇
 文弼為代州總管。十二月乙未,突厥都藍可汗為部下所殺。丁丑,星隕於勃海。



 二十年春正月辛酉朔,上在仁壽宮。突厥、高麗、契丹並遣使貢方物。癸亥,以代州總管宇文弼為吳州總管。二月己巳,以上柱國崔弘度為原州總管。丁丑,無雲而雷。三月辛卯,熙州人李英林反,遣行軍總管張衡討平之。夏四月壬戌,突厥犯塞,以晉王廣為行軍元帥,擊破之。乙亥,天有聲如瀉水,自南而北。六月丁丑,秦王俊薨。秋八月,老人星見。九月丁未,至自仁壽宮。癸丑,吳州總管楊異卒。



 冬十月己未,太白晝見。乙丑,皇太子勇及諸子
 並廢為庶人。殺柱國、太平縣公史萬歲。己巳,殺左衛大將軍、五原郡公元旻。十一月戊子,天下地震,京師大風雪。



 以晉王廣為皇太子。十二月戊午,詔東宮官屬不得稱臣於皇太子。辛巳,詔曰:「佛法深妙,道教虛融,咸降大慈,濟度群品,凡在含識,皆蒙覆護。所以雕鑄靈相,圖寫真形,率土瞻仰,用申誠敬。其五岳四鎮,節宣雲雨,江河淮海,浸潤區域,並生養萬物,利益兆人,故建廟立祀,以時恭敬。敢有毀壞偷盜佛及天尊像、岳鎮海瀆神形者,以不道論。沙門壞佛像,道士壞天尊者,以惡逆論。



 仁壽元年春正月乙酉朔,大赦,改元。以尚書右僕射楊
 素為尚書左僕射,納言蘇威為尚書右僕射。丁酉,徙河南王昭為晉王。突厥寇恆安,遣柱國韓洪擊之,官軍敗績。以晉王昭為內史令。辛丑,詔曰:「君子立身,雖云百行,唯誠與孝最為其首。故投主殉節,自古稱難,殞身王事,禮加二等。而代俗之徒,不達大義,至於致命戎旅,不入兆域,虧孝子之意,傷人臣之心。興言念此,每深愍嘆。且入廟祭祀,並不廢闕,何止墳塋,獨在其外。自今已後,戰亡之徒,宜入墓域。」二月乙卯朔,日有蝕之。辛巳,以上柱國獨孤楷為原州總管。三月壬辰,以豫章王暕為揚州總管。夏四月,以淅州刺史蘇孝慈為洪州總管。五月己
 丑,突厥男女九萬口來降。壬辰,驟雨震雷,大風拔木,宜君湫水移於始平。六月癸丑,洪州總管蘇孝慈卒。乙卯,遣十六使巡省風俗。乙丑,詔曰:「儒學之道,訓教生人,識父子君臣之義,知尊卑長幼之序,升之於朝,任之以職,故能贊理時務,弘益風範。朕撫臨天下,思弘德教,延集學徒,崇建庠序,開進仕之路,佇賢雋之人。而國學胄子,垂將千數,州縣諸生,咸亦不少。徒有名錄,空度歲時,未有德為代範,才任國用。



 良由設學之理,多而未精。今宜簡省,明加獎勵。」於是國子學唯留學生七十人,太學、四門及州縣學並廢。其日,頒舍利於諸州。秋七月戊戌,改
 國子為太學。九月癸未,以柱國杜彥為雲州總管。十一月己丑,有事於南郊。壬辰,以資州刺史衛玄為遂州總管。



 二年春二月辛亥,以邢州刺史侯莫陳穎為桂州總管,宗正楊文紀為荊州總管。



 三月己亥,幸仁壽宮。壬寅,以齊州刺史張喬為潭州總管。夏四月庚戌,岐、雍二州地震。秋七月丙戌,詔內外官各舉所知。戊子,以原州總管獨孤楷為益州總管。



 八月己巳,皇后獨孤氏崩。九月丙戌,至自仁壽宮。壬辰,河南北諸州大水,遣工部尚書楊達賑恤之。乙未,上柱國、襄州總管、金水郡公周搖卒。隴西
 地震。冬十月壬子,曲赦益州管內。癸丑,以工部尚書楊達為納言。閏月甲申,詔尚書左僕射楊素與諸術者刊定陰陽舛謬。己丑,詔曰:「禮之為用,時義大矣。黃琮蒼璧,降天地之神,粢盛牲食,展宗廟之敬,正父子君臣之序,明婚姻喪紀之節。故道德仁義,非禮不成,安上治人,莫善於禮。自區宇亂離,綿歷年代,王道衰而變風作,微言絕而大義乖,與代推移,其弊日甚。至於四時郊祀之節文,五服麻葛之隆殺,是非異說,踳駁殊途,致使聖教凋訛,輕重無準。朕祗承天命,撫臨生人,當洗滌之時,屬干戈之代,克定禍亂,先運武功,刪正彞典,日不暇給。今四
 海乂安,五戎勿用,理宜弘風訓俗,導德齊禮,綴往聖之舊章,興先王之茂則。尚書左僕射、越國公楊素,尚書右僕射、邳國公蘇威,吏部尚書、奇章公牛弘,內史侍郎薛道衡,秘書丞許善心,內史舍人虞世基,著作郎王劭,或任居端揆,博達古今,或器推令望,學綜經史,委以裁緝,實允僉議。可並修定五禮。」壬寅,葬獻皇后於太陵。



 十二月癸巳,上柱國、益州總管蜀王秀廢為庶人。交州人李佛子舉兵反,遣行軍總管劉方討平之。



 三年春二月己卯,原州總管、比陽縣公龐晃卒。戊子,以大將軍、蔡陽郡公姚辯為左武候大將軍。夏五月癸卯,
 詔曰:「哀哀父母,生我劬勞,欲報之德,昊天罔極。但風樹不靜,嚴敬莫追,霜露既降,感思空切。六月十三日,是朕生日,宜令海內為武元皇帝、元明皇后斷屠。」六月甲午,詔曰:《禮》云:「至親以期斷。」蓋以四時之變易,萬物之更始,故聖人象之。其有三年,加隆爾也。但家無二尊,母為厭降,是以父存喪母,還服於期者,服之正也,豈容期內而更小祥!然三年之喪而有小祥者,《禮》云:「期祭,禮也。期而除喪,道也。」以是之故,雖未再期,而天地一變,不可不祭,不可不除。故有練焉,以存喪祭之本。然期喪有練,於理未安。雖云十一月而練,乃無所法象,非期非時,豈可除
 祭。而儒者徒擬三年之喪,立練禫之節,可謂茍存其變,而失其本,欲漸於奪,乃薄於喪。致使子則冠練去絰,黃裏縓緣,絰則布葛在躬,粗服未改。



 豈非絰哀尚存,子情已奪,親疏失倫,輕重顛倒!乃不順人情,豈聖人之意也!故知先聖之禮廢於人邪,三年之喪尚有不行之者,至於祥練之節,安能不墜者乎?



 《禮》云:「父母之喪,無貴賤一也。」而大夫士之喪父母,乃貴賤異服。然則禮壞樂崩,由來漸矣。所以晏平仲之斬粗縗,其老謂之非禮,滕文公之服三年,其臣咸所不欲。蓋由王道既衰,諸侯異政,將逾越於法度,惡禮制之害己,乃滅去篇籍,自制其宜。遂
 至骨肉之恩,輕重從俗,無易之道,隆殺任情。況孔子沒而微言隱,秦滅學而經籍焚者乎!有漢之興,雖求儒雅,人皆異說,義非一貫。又近代亂離,唯務兵革,其於典禮,時所未遑。夫禮不從天降,不從地出,乃人心而已者,謂情緣於恩也。故恩厚者其禮隆,情輕者其禮殺。聖人以是稱情立文,別親疏貴賤之節。



 自臣子道消,上下失序,莫大之恩,逐情而薄,莫重之禮,與時而殺。此乃服不稱喪,容不稱服,非所謂聖人緣恩表情,制禮之義也。



 然喪與易也,寧在於戚,則禮之本也。禮有其餘,未若於哀,則情之實也。今十一月而練者,非禮之本,非情之實。由是
 言之,父存喪母,不宜有練。但依禮十三月而祥,中月而禫。庶以合聖人之意,達孝子之心。



 秋七月丁卯,詔曰:日往月來,唯天所以運序;山鎮川流,唯地所以宣氣。運序則寒暑無差,宣氣則雲雨有作,故能成天地之大德,育萬物而為功。況一人君於四海,睹物欲運,獨見致治,不藉群才,未之有也。是以唐堯欽明,命羲、和以居嶽,虞舜叡德,升元、凱而作相。伊尹鼎俎之媵,為殷之阿衡,呂望漁釣之夫,為周之尚父。此則鳴鶴在陰,其子必和,風雲之從龍虎,賢哲之應聖明。君德不回,臣道以正,故能通天地之和,順陰陽之序,豈不由元首而有股肱乎?自王
 道衰,人風薄,居上莫能公道以御物,為下必踵私法以希時。上下相蒙,君臣義失,義失則政乖,政乖則人困。蓋同德之風難嗣,離德之軌易追,則任者不休,休者不任,則眾口鑠金,戮辱之禍不測。是以行歌避代,辭位灌園,卷而可懷,黜而無慍,放逐江湖之上,沈赴河海之流,所以自潔而不悔者也。至於閭閻秀異之士,鄉曲博雅之儒,言足以佐時,行足以勵俗,遺棄於草野,堙滅而無聞,豈勝道哉!所以覽古而嘆息者也。方今區宇一家,煙火萬里,百姓乂安,四夷賓服,豈是人功,實乃天意。朕惟夙夜祗懼,將所以上嗣明靈,是以小心勵己,日慎一日。以
 黎元在念,憂兆庶未康,以庶政為懷,慮一物失所。雖求傅巖,莫見幽人,徒想崆峒,未聞至道。唯恐商歌於長夜,抱關於夷門,遠跡犬羊之間,屈身僮僕之伍。其令州縣搜揚賢哲,皆取明知今古,通識治亂,究政教之本,達禮樂之源。不限多少,不得不舉。限以三旬,咸令進路。徵召將送,必須以禮。



 八月壬申,上柱國、檢校幽州總管、落叢郡公燕榮以罪伏誅。九月壬戌,置常平官。甲子,以營州總管韋沖為民部尚書。十二月癸酉,河南諸州水,遣納言楊達賑恤之。



 四年春正月丙辰,大赦。甲子,幸仁壽宮。乙丑,詔賞罰支
 度,事無巨細,並付皇太子。夏四月乙卯,上不豫。六月庚申,大赦天下。有星入月中,數日而退。



 長人見於雁門。秋七月乙未,日青無光,八日乃復。己亥,以大將軍段文振為雲州總管。甲辰,上以疾甚,臥於仁壽宮,與百僚辭訣,並握手歔欷。丁未,崩於大寶殿,時年六十四。遺詔曰:嗟乎!自昔晉室播遷,天下喪亂,四海不一,以至周、齊,戰爭相尋,年將三百。故割疆土者非一所,稱帝王者非一人,書軌不同,生人塗炭。上天降鑒,爰命於朕,用登大位,豈關人力!故得撥亂反正,偃武修文,天下大同,聲教遠被,此又是天意欲寧區夏。所以昧旦臨朝,不敢逸豫,一日
 萬機,留心親覽,晦明寒暑,不憚劬勞,匪曰朕躬,蓋為百姓故也。王公卿士,每日闕庭,刺史以下,三時朝集,何嘗不罄竭心府,誡敕殷勤。義乃君臣,情兼父子。庶藉百僚智力,萬國歡心,欲令率土之人,永得安樂,不謂遘疾彌留,至於大漸。此乃人生常分,何足言及!但四海百姓,衣食不豐,教化政刑,猶未盡善,興言念此,唯以留恨。朕今年逾六十,不復稱夭,但筋力精神,一時勞竭。如此之事,本非為身,止欲安養百姓,所以致此。人生子孫,誰不愛念,既為天下,事須割情。勇及秀等,並懷悖惡,既知無臣子之心,所以廢黜。古人有言:「知臣莫若於君,知子莫若
 於父。」若令勇、秀得志,共治家國,必當戮辱遍於公卿,酷毒流於人庶。今惡子孫已為百姓黜屏,好子孫足堪負荷大業。此雖朕家事,理不容隱,前對文武侍衛,具已論述。皇太子廣,地居上嗣,仁孝著聞,以其行業,堪成朕志。但令內外群官,同心戮力,以此共治天下,朕雖瞑目,何所復恨。但國家事大,不可限以常禮。既葬公除,行之自昔,今宜遵用,不勞改定。兇禮所須,才令周事。務從節儉,不得勞人。諸州總管、刺史已下,宜各率其職,不須奔赴。自古哲王,因人作法,前帝後帝,沿革隨時。律令格式,或有不便於事者,宜依前敕修改,務當政要。嗚呼,敬之哉!
 無墜朕命!



 乙卯,發喪。河間楊柳四株無故黃落,既而花葉復生。八月丁卯,梓宮至自仁壽宮。丙子,殯於大興前殿。冬十月己卯,合葬於太陵,同墳而異穴。



 上性嚴重,有威容,外質木而內明敏,有大略。初,得政之始,群情不附,諸子幼弱,內有六王之謀,外致三方之亂。握強兵、居重鎮者,皆周之舊臣。上推以赤心,各展其用,不逾期月,克定三邊,未及十年,平一四海。薄賦斂,輕刑罰,內修制度,外撫戎夷。每旦聽朝,日昃忘倦,居處服玩,務存節儉,令行禁止,上下化之。開皇、仁壽之間,丈夫不衣綾綺,而無金玉之飾,常服率多布帛,裝帶不過以銅鐵骨角而已。
 雖嗇於財,至於賞賜有功,亦無所愛吝。乘輿四出,路逢上表者,則駐馬親自臨問。或潛遣行人採聽風俗,吏治得失,人間疾苦,無不留意。嘗遇關中饑,遣左右視百姓所食。有得豆屑雜糠而奏之者,上流涕以示群臣,深自咎責,為之撤膳,不御酒肉者殆將一期。及東拜太山,關中戶口就食洛陽者,道路相屬。上敕斥候,不得輒有驅逼。男女參廁於仗衛之間,逢扶老攜幼者,輒引馬避之,慰勉而去。至艱險之處,見負擔者,遽令左右扶助之。其有將士戰沒,必加優賞,仍令使者就家勞問。自強不息,朝夕孜孜,人庶殷繁,帑藏充實。雖未能臻於至治,亦足
 稱近代之良主。然天性沉猜,素無學術,好為小數,不達大體,故忠臣義士,莫得盡心竭辭。其草創元勛及有功諸將,誅夷罪退,罕有存者。又不悅詩書,廢除學校,唯婦言是用,廢黜諸子。逮於暮年,持法尤峻,喜怒不常,過於殺戮。嘗令左右送西域朝貢使出玉門關,其人所經之處,或受牧宰小物,饋遺鸚鵡、麖皮、馬鞭之屬,上聞而大怒。又詣武庫,見署中蕪穢不治,於是執武庫令及諸受遺者,出開遠門外,親自臨決,死者數十人。又往往潛令人賂遺令史府史,有受者必死,無所寬貸。議者以此少之。



 史臣曰:高祖龍德在田,奇表見異,晦明藏用,故知我者希。始以外戚之尊,受托孤之任,與能之議,未為當時所許,是以周室舊臣,咸懷憤惋。既而王謙固三蜀之阻,不逾期月,尉迥舉全齊之眾,一戰而亡,斯乃非止人謀,抑亦天之所贊也。



 乖茲機運,遂遷周鼎。於時蠻夷猾夏,荊、揚未一,劬勞日昃,經營四方。樓船南邁,則金陵失險,驃騎北指,則單于款塞,《職方》所載,並入疆理,《禹貢》所圖,咸受正朔。雖晉武之克平吳會,漢宣之推亡固存,比義論功,不能尚也。七德既敷,九歌已洽,要荒咸暨,尉候無警。於是躬節儉,平徭賦,倉廩實,法令行,君子咸樂其生,小
 人各安其業,強無陵弱,眾不暴寡,人物殷阜,朝野歡娛。二十年間,天下無事,區宇之內晏如也。考之前王,足以參蹤盛烈。但素無術學,不能盡下,無寬仁之度,有刻薄之資,暨乎暮年,此風逾扇。又雅好符瑞,暗於大道,建彼維城,權侔京室,皆同帝制,靡所適從。聽哲婦之言,惑邪臣之說,溺寵廢嫡,托付失所。滅父子之道,開昆弟之隙,縱其尋斧,剪伐本枝。墳土未幹,子孫繼踵屠戮,松檟才列,天下已非隋有。惜哉!跡其衰怠之源,稽其亂亡之兆,起自高祖,成於煬帝,所由來遠矣,非一朝一夕。其不祀忽諸,未為不幸也。



\end{pinyinscope}