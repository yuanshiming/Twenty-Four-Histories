\article{卷八十列傳第四十五}

\begin{pinyinscope}

 列女自昔貞專淑媛,布在方策者多矣。婦人之德,雖在於溫柔,立節垂名,咸資於貞烈。溫柔,仁之本也;貞烈,義之資也。非溫柔無以成其仁,非貞烈無以顯其義。



 是以詩書所記,風俗所在,圖像丹青,流聲竹素,莫不守約以居正,殺身以成仁者也。若文伯、王陵之母,白公、杞植之妻,魯之義姑,梁之高行,衛君靈主之妾,夏侯文寧之女,或抱
 信以含貞,或蹈忠而踐義,不以存亡易心,不以盛衰改節,其修名彰於既往,徽音傳於不朽,不亦休乎!或有王公大人之妃偶,肆情於淫僻之俗,雖衣繡衣,食珍膳,坐金屋,乘玉輦,不入彤管之書,不沾良史之筆,將草木以俱落,與麋鹿而同死,可勝道哉!永言載思,實庶姬之恥也。觀夫今之靜女,各勵松筠之操,甘於玉折而蘭摧,足以無絕今古。故述其雅志,以纂前代之列女云。



 蘭陵公主蘭陵公主,字阿五,高祖第五女也。美姿儀,性婉順,好讀書,高祖於諸女中特所鐘愛。初嫁儀同王奉孝,卒,適河
 東柳述,時年十八。諸姊並驕貴,主獨折節遵於婦道,事舅姑甚謹,遇有疾病,必親奉湯藥。高祖聞之大悅。由是述漸見寵遇。



 初,晉王廣欲以主配其妃弟蕭瑒,高祖初許之,後遂適述,晉王因不悅。及述用事,彌惡之。高祖既崩,述徙嶺表。煬帝令主與述離絕,將改嫁之。公主以死自誓,不復朝謁,上表請免主號,與述同徙。帝大怒曰:「天下豈無男子,欲與述同徙耶?」



 主曰:「先帝以妾適於柳家,今其有罪,妾當從坐,不願陛下屈法申恩。」帝不從,主憂憤而卒,時年三十二。臨終上表曰:「昔共姜自誓,著美前詩,鄎媯不言,傳芳往誥。妾雖負罪,竊慕古人。生既不得
 從夫,死乞葬於柳氏。」帝覽之愈怒,竟不哭,乃葬主於洪瀆川,資送甚薄。朝野傷之。



 南陽公主者,煬帝之長女也。美風儀,有志節,造次必以禮。年十四,嫁於許國公宇文述子士及,以謹肅聞。及述病且卒,主視調飲食,手自奉上,世以此稱之。



 及宇文化及殺逆,主隨至聊城,而化及為竇建德所敗,士及自濟北西歸大唐。時隋代衣冠並在其所,建德引見之,莫不惶懼失常,唯主神色自若。建德與語,主自陳國破家亡,不能報怨雪恥,淚下盈襟,聲辭不輟,情理切至。建德及
 觀聽者莫不為之動容隕涕,咸肅然敬異焉。及建德誅化及,時主有一子,名禪師,年且十歲。建德遣武賁郎將於士澄謂主曰:「宇文化及躬行殺逆,人神所不容。今將族滅其家,公主之子,法當從坐,若不能割愛,亦聽留之。」主泣曰:「武賁既是隋室貴臣,此事何須見問!」建德竟殺之。主尋請建德削發為尼。及建德敗,將歸西京,復與士及遇於東都之下,主不與相見。士及就之,立於戶外,請復為夫妻。主拒之曰:「我與君仇家。今恨不能手刃君者,但謀逆之日,察君不預知耳。」因與告絕,訶令速去。士及固請之,主怒曰:「必欲就死,可相見也。」士及見其言切,知
 不可屈,乃拜辭而去。



 襄城王恪妃襄城王恪妃者,河東柳氏女也。父旦,循州刺史。妃姿儀端麗,年十餘,以良家子合法相,娉以為妃。未幾而恪被廢,妃修婦道,事之愈敬。煬帝嗣位,恪復徙邊,帝令使者殺之於道。恪與辭訣,妃曰:「若王死,妾誓不獨生。」於是相對慟哭。恪既死,棺斂訖,妃謂使者曰:「妾誓與楊氏同穴。若身死之後得不別埋,君之惠也。」遂撫棺號慟,自經而卒。見者莫不為之涕流。



 華陽王楷妃
 華陽王楷妃者,河南元氏之女也。父巖,性明敏,有氣幹。仁壽中,為黃門侍郎,封龍涸縣公。煬帝嗣位,坐與柳述連事,除名為民,徙南海。後會赦,還長安。



 有人譖巖逃歸,收而殺之。妃有姿色,性婉順,初以選為妃。未幾而楷被幽廢,妃事楷逾謹,每見楷有憂懼之色,輒陳義理以慰諭之,楷甚敬焉。及江都之亂,楷遇宇文化及之逆,以妃賜其黨元武達。武達初以宗族之禮,置之別舍,後因醉而逼之。



 妃自誓不屈,武達怒,撻之百餘,辭色彌厲。因取甓自毀其面,血淚交下,武達釋之。妃謂其徒曰:「我不能早死,致令將見侵辱,我之罪也。」因不食而卒。



 譙國夫人譙國夫人者,高涼洗氏之女也。世為南越首領,跨據山洞,部落十餘萬家。夫人幼賢明,多籌略,在父母家,撫循部眾,能行軍用師,壓服諸越。每勸親族為善,由是信義結於本鄉。越人之俗,好相攻擊,夫人兄南梁州刺史挺,恃其富強,侵掠傍郡,嶺表苦之。夫人多所規諫,由是怨隙止息,海南、儋耳歸附者千餘洞。梁大同初,羅州刺史馮融聞夫人有志行,為其子高涼太守寶娉以為妻。融本北燕苗裔,初,馮弘之投高麗也,遣融大父業以三百人浮海歸宋,因留於新會。自業及融,三世為守牧,他鄉
 羈旅,號令不行。至是,夫人誡約本宗,使從民禮。每共寶參決辭訟,首領有犯法者,雖是親族,無所舍縱。自此政令有序,人莫敢違。遇侯景反,廣州都督蕭勃徵兵援臺。高州刺史李遷仕據大皋口,遣召寶。寶欲往,夫人止之曰:「刺史無故不合召太守,必欲詐君共為反耳。」寶曰:「何以知之?」夫人曰:「刺史被召援臺,乃稱有疾,鑄兵聚眾,而後喚君。今者若往,必留質,追君兵眾。



 此意可見,願且無行,以觀其勢。」數日,遷仕果反,遣主帥杜平虜率兵入灨石。



 寶知之,遽告,夫人曰:「平虜,驍將也,領兵入灨石,即與官兵相拒,未得還。



 遷仕在州,無能為也。若君自往,必
 有戰鬥。宜遣使詐之,卑辭厚禮,雲身未敢出,欲遣婦往參。彼聞之喜,必無防慮。於是我將千餘人,步擔雜物,唱言輸賧,得至柵下,賊必可圖。」寶從之,遷仕果大喜,覘夫人眾皆擔物,不設備。夫人擊之,大捷。遷仕遂走,保於寧都。夫人總兵與長城侯陳霸先會於灨石。還謂寶曰:「陳都督大可畏,極得眾心。我觀此人必能平賊,君宜厚資之。」



 及寶卒,嶺表大亂,夫人懷集百越,數州晏然。至陳永定二年,其子僕年九歲,遺帥諸首領朝於丹陽,起家拜陽春郡守。後廣州刺史歐陽紇謀反,召僕至高安,誘與為亂。僕遣使歸告夫人,夫人曰:「我為忠貞,經今兩代,不
 能惜汝,輒負國家。」



 遂發兵拒境,帥百越酋長迎章昭達。內外逼之,紇徒潰散。僕以夫人之功,封信都侯,加平越中郎將,轉石龍太守。詔使持節冊夫人為中郎將、石龍太夫人,賚繡「W油絡駟馬安車一乘,給鼓吹一部,並麾幢旌節,其鹵簿一如刺史之儀。至德中,僕卒。後遇陳國亡,嶺南未有所附,數郡共奉夫人,號為聖母,保境安民。



 高祖遣總管韋洸安撫嶺外,陳將徐璒以南康拒守。洸至嶺下,逡巡不敢進。初,夫人以扶南犀杖獻於陳主,至此,晉王廣遣陳主遺夫人書,諭以國亡,令其歸化,並以犀杖及兵符為信,夫人見杖,驗知陳亡,集首領數千,盡
 日慟哭。遣其孫魂帥眾迎洸,入至廣州,嶺南悉定。表魂為儀同三司,冊夫人為宋康郡夫人。未幾,番禺人王仲宣反,首領皆應之,圍洸於州城,進兵屯衡嶺。夫人遣孫暄帥師救洸。暄與逆黨陳佛智素相友善,故遲留不進。夫人知之,大怒,遣使執暄,系於州獄。又遣孫盎出討佛智,戰克,斬之。進兵至南海,與鹿願軍會,共敗仲宣。夫人親被甲,乘介馬,張錦傘,領彀騎,衛詔使裴矩巡撫諸州,其蒼梧首領陳坦、岡州馮岑翁、梁化鄧馬頭、藤州李光略、羅州龐靖等皆來參謁。還令統其部落,嶺表遂定。高祖異之,拜盎為高州刺史,仍赦出暄,拜羅州刺史。追贈
 寶為廣州總管、譙國公,冊夫人為譙國夫人。以宋康邑回授僕妾洗氏。仍開譙國夫人幕府,置長史以下官屬,給印章,聽發部落六州兵馬,若有機急,便宜行事。降敕書曰:「朕撫育蒼生,情均父母,欲使率土清凈,兆庶安樂。而王仲宣等輒相聚結,擾亂彼民,所以遣往誅翦,為百姓除害。夫人情在奉國,深識正理,遂令孫盎斬獲佛智,竟破群賊,甚有大功。今賜夫人物五千段。暄不進愆,誠合罪責,以夫人立此誠效,故特原免。夫人宜訓導子孫,敦崇禮教,遵奉朝化,以副朕心。」皇后以首飾及宴服一襲賜之,夫人並盛於金篋,並梁、陳賜物各藏於一庫。每
 歲時大會,皆陳於庭,以示子孫,曰:「汝等宜盡赤心向天子。我事三代主,唯用一好心。今賜物具存,此忠孝之報也,願汝皆思念之。」



 時番州總管趙訥貪虐,諸俚獠多有亡叛。夫人遣長史張融上封事,論安撫之宜,並言訥罪狀,不可以招懷遠人。上遣推訥,得其贓賄,竟致於法。降敕委夫人招慰亡叛。夫人親載詔書,自稱使者,歷十餘州,宣述上意,諭諸俚獠,所至皆降。高祖嘉之,賜夫人臨振縣湯沐邑,一千五百戶。贈僕為巖州總管、平原郡公。仁壽初,卒,賻物一千段,謚為誠敬夫人。



 鄭善果母
 鄭善果母者,清河崔氏之女也。年十三,出適鄭誠,生善果。而誠討尉迥,力戰死於陣。母年二十而寡,父彥穆欲奪其志,母抱善果謂彥穆曰:「婦人無再見男子之義。且鄭君雖死,幸有此兒。棄兒為不慈,背死為無禮。寧當割耳截發以明素心。違禮滅慈,非敢聞命。」善果以父死王事,年數歲,拜使持節、大將軍,襲爵開封縣公,邑一千戶。開皇初,進封武德郡公。年十四,授沂州刺史,轉景州刺史,尋為魯郡太守。



 母性賢明,有節操,博涉書史,通曉治方。每善果出聽事,母恆坐胡床,於鄣後察之。聞其剖斷合理,歸則大悅,即賜之坐,相對談笑。若行事不允,或妄
 瞋怒,母乃還堂,蒙被而泣,終日不食。善果伏於床前,亦不敢起。母方起謂之曰:「吾非怒汝,乃愧汝家耳。吾為汝家婦,獲奉灑掃,如汝先君,忠勤之士也,在官清恪,未嘗問私,以身徇國,繼之以死,吾亦望汝副其此心。汝既年小而孤,吾寡婦耳,有慈無威,使汝不知禮訓,何可負荷忠臣之業乎?汝自童子承襲茅土,位至方伯,豈汝身致之邪?安可不思此事而妄加瞋怒,心緣驕樂,墮於公政!內則墜爾家風,或亡失官爵,外則虧天子之法,以取罪戾。吾死之日,亦何面目見汝先人於地下乎?」



 母恆自紡績,夜分而寐。善果曰:「兒封侯開國,位居三品,秩俸幸足,
 母何自勤如是邪?」答曰:「嗚呼!汝年已長,吾謂汝知天下之理,今聞此言,故猶未也。至於公事,何由濟乎?今此秩俸,乃是天子報爾先人之徇命也。當須散贍六姻,為先君之惠,妻子奈何獨擅其利,以為富貴哉!又絲枲紡織,婦人之務,上自王後,下至大夫士妻,各有所制。若墮業者,是為驕逸。吾雖不知禮,其可自敗名乎?」



 自初寡,便不御脂粉,常服大練。性又節儉,非祭禮賓客之事,酒肉不妄陳於前。



 靜室端居,未嘗輒出門閣。內外姻戚有吉兇事,但厚加贈遺,皆不詣其家。非自手作及莊園祿賜所得,雖親族禮遺,悉不許入門。



 善果歷任州郡,唯內自出
 饌,於衙中食之,公廨所供,皆不許受,悉用修治廨宇及分給僚佐。善果亦由此克己,號為清吏。煬帝遣御史大夫張衡勞之,考為天下最。征授光祿卿。其母卒後,善果為大理卿,漸驕恣,清公平允遂不如疇昔焉。



 孝女王舜孝女王舜者,趙郡王子春之女也。子春與從兄長忻不協,屬齊滅之際,長忻與其妻同謀殺子春。舜時年七歲,有二妹,粲年五歲,璠年二歲,並孤苦,寄食親戚。



 舜撫育二妹,恩義甚篤。而舜陰有復仇之心,長忻殊不為備。姊妹俱長,親戚欲嫁之,輒拒不從。乃密謂其二妹曰:「我無
 兄弟,致使父仇不復。吾輩雖是女子,何用生為?我欲共汝報復,汝意如何?」二妹皆垂泣曰:「唯姊所命。」是夜,姊妹各持刀逾墻而入,手殺長忻夫妻,以告父墓。因詣縣請罪,姊妹爭為謀首,州縣不能決。高祖聞而嘉嘆,特原其罪。



 韓覬妻韓覬妻者,洛陽于氏女也,字茂德,父實,周大左輔。於氏年十四,適於覬。



 雖生長膏腴,家門鼎盛,而動遵禮度,躬自儉約,宗黨敬之。年十八,覬從軍戰沒,於氏哀毀骨立,慟感行路。每至朝夕奠祭,皆手自捧持。及免喪,其父以
 其幼少無子,將嫁之。誓無異志。復令家人敦喻,於氏盡夜涕泣,截發自誓。其父喟然傷感,遂不奪其志焉。因養夫之孽子世隆為嗣,身自撫育,愛同己生,訓導有方,卒能成立。自孀居已後,唯時或歸寧,至於親族之家,絕不來往。有尊卑就省謁者,送迎皆不出戶庭。蔬食布衣,不聽聲樂,以此終身。高祖聞而嘉嘆,下詔褒美,表其門閭,長安中號為節婦闕。終於家,年七十二。



 陸讓母陸讓母者,上黨馮氏女也。性仁愛,有母儀,讓即其孽子也。仁壽中,為番州刺史,數有聚斂,贓貨狼籍,為司馬所
 奏。上遣使按之皆驗,於是囚詣長發,親臨問。讓稱冤,上復令治書侍御史撫按之,狀不易前。乃命公卿百僚議之,咸曰「讓罪當死」。詔可其奏。讓將就刑,馮氏蓬頭垢面詣朝堂數讓曰:「無汗馬之勞,致位刺史,不能盡誠奉國,以答鴻恩,而反違犯憲章,贓貨狼籍。若言司馬誣汝,百姓百官不應亦皆誣汝。若言至尊不憐愍汝,何故治書覆汝?豈誠臣?豈孝子?不誠不孝,何以為人!」於是流涕嗚咽,親持盂粥勸讓令食。既而上表求哀,詞情甚切,上愍然為之改容。獻皇后甚奇其意,致請於上。治書侍御史柳彧進曰:「馮氏母德之至,有感行路。如或殺之,何以為
 勸?」上於是集京城士庶於硃雀門,遣舍人宣詔曰:「馮氏以嫡母之德,足為世範,慈愛之道,義感人神,特宜矜免,用獎風俗。



 讓可減死,除名為民。」復下詔曰:「馮氏體備仁慈,夙閑禮度。孽讓非其所生,往犯憲章,宜從極法。躬自詣闕,為之請命,匍匐頓顙。朕哀其義,特免死辜。使天下婦人皆如馮者,豈不閨門雍睦,風俗和平!朕每嘉嘆不能已。宜標揚優賞,用章有德。可賜物五百段。」集諸命婦,與馮相識,以寵異之。



 劉昶女劉昶女者,河南長孫氏之婦也。昶在周,尚公主,官至柱
 國、彭國公,數為將帥,位望隆顯。與高祖有舊。及受禪,甚親任,歷左武衛大將軍、慶州總管。其子居士,為太子千牛備身,聚徒任俠,不遵法度,數得罪。上以昶故,每輒原之。居士轉恣,每大言曰:「男兒要當辮頭反縛,籧篨上作獠舞。」取公卿子弟膂力雄健者,輒將至家,以車輪括其頸而棒之。殆死能不屈者,稱為壯士,釋而與交。黨與三百人,其趫捷者號為餓鶻隊,武力者號為蓬轉隊。每韝鷹紲犬,連騎道中,毆擊路人,多所侵奪。長安市里無貴賤,見之者皆闢易,至於公卿妃主,莫敢與校者。



 其女則居士之姊也,每垂泣誨之,殷勤懇惻。居士不改,至破家
 產。昶年老,奉養甚薄。其女時寡居,哀昶如此,每歸寧於家,躬勤紡績,以致其甘脆。有人告居士與其徒游長安城,登故未央殿基,南向坐,前後列隊,意有不遜,每相約曰:「當為一死耳。」又時有人言居士遣使引突厥令南寇,當於京師應之。上謂昶曰:「今日之事,當復如何?」昶猶恃舊恩,不自引咎,直前曰:「黑白在於至尊。」上大怒,下昶獄,捕居士黨與,治之甚急。憲司又奏昶事母不孝。其女知昶必不免,不食者數日,每親調飲食,手自捧持,詣大理餉其父。見獄卒,長跪以進,歔欷嗚咽,見者傷之。居士坐斬,昶竟賜死於家。詔百僚臨視。時其女絕而復蘇者數
 矣,公卿慰諭之。其女言父無罪,坐子以及於禍。詞情哀切,人皆不忍聞見。遂布衣蔬食以終其身。上聞而嘆曰:「吾聞衰門之女,興門之男,固不虛也!」



 鐘士雄母鐘士雄母者,臨賀蔣氏女也。士雄仕陳為伏波將軍。陳主以士雄嶺南酋帥,慮其反覆,每質蔣氏於都下。及晉王廣平江南,以士雄在嶺表,欲以恩義致之,遣蔣氏歸臨賀。既而同郡虞子茂、鐘文華等作亂,舉兵攻城,遣人召士雄,士雄將應之。



 蔣氏謂士雄曰:「我前在揚都,備嘗辛苦。今逢聖化,母子聚集,沒身不能上報,焉得為逆哉!
 汝若禽獸其心,背德忘義者,我當自殺於汝前。」士雄於是遂止。蔣氏復為書與子茂等,諭以禍福。子茂不從,尋為官軍所敗。上聞蔣氏,甚異之,封為安樂縣君。



 時尹州寡婦胡氏者,不知何氏妻也,甚有志節,為邦族所重。當江南之亂,諷諭宗黨,皆守險不從叛逆,封為密陵郡君。



 孝婦覃氏孝婦覃氏者,上郡鐘氏婦也。與其夫相見未幾而夫死,時年十八。事後姑以孝聞。數年之間,姑及伯叔皆相繼而死,覃氏家貧,無以葬。於是躬自節儉,晝夜紡績,蓄財十年,而葬八喪,為州里所敬,上聞而賜米百石,表其門
 閭。



 元務光母元務光母者,範陽盧氏女也。少好讀書,造次以禮。盛年寡居,諸子幼弱,家貧不能就學,盧氏每親自教授,勖以義方,世以此稱之。仁壽末,漢王諒舉兵反,遣將綦良往山東略地。良以務光為記室。及良敗,慈州刺史上官政簿籍務光之家,見盧氏,悅而逼之,盧氏以死自誓。政為人兇悍,怒甚,以燭燒其身。盧氏執志彌固,竟不屈節。



 裴倫妻裴倫妻,河東柳氏女也,少有風訓。大業末,倫為渭源令。
 屬薛舉之亂,縣城為賊所陷,倫遇害。柳時年四十,有二女及兒婦三人,皆有美色。柳氏謂之曰:「我輩遭逢禍亂,汝父已死,我自念不能全汝。我門風有素,義不受辱於群賊,我將與汝等同死,如何?」其女等皆垂泣曰:「唯母所命。」柳氏遂自投於井,其女及婦相繼而下,皆重死於井中。



 趙元楷妻趙元楷妻者,清河崔氏之女也。父儦,在《文學傳》。家有素範,子女皆遵禮度。元楷父為僕射,家富於財,重其門望,厚禮以聘之。元楷甚敬崔氏,雖在宴私,不妄言笑,進止
 容服,動合禮儀。化及之反也,元楷隨至河北,將歸長安。至滏口,遇盜攻掠,元楷僅以身免。崔氏為賊所拘,賊請以為妻,崔氏謂賊曰:「我士大夫女,為僕射子妻,今日破亡,自可即死。遣為賊婦,終必不能。」群賊毀裂其衣,形體悉露,縛於床簀之上,將凌之。崔氏懼為所辱,詐之曰:「今力已屈,當聽處分,不敢相違,請解縛。」賊遽釋之。崔因著衣,取賊佩刀,倚樹而立曰:「欲殺我,任加刀鋸。若覓死,可來相逼!」賊大怒,亂射殺之。元楷後得殺妻者,支解之,以祭崔氏之柩。



 史臣曰:夫稱婦人之德,皆以柔順為先,斯乃舉其中庸,
 未臻其極者也。至於明識遠圖,貞心峻節,志不可奪,唯義所在,考之圖史,亦何世而無哉!蘭陵主質邁寒松,南陽主心逾匪石、洗媼孝女之忠壯,崔、馮二母之誠懇,足使義勇慚其志烈,蘭玉謝其貞芳。襄城、華陽之妃,裴倫、元楷之婦,時逢艱阻,事乖好合,甘心同穴,顛沛靡它,志勵冰霜,言逾皎日,雖《詩》詠共姜之自誓,《傳》述伯姬之守死,其將復何以加焉!



\end{pinyinscope}