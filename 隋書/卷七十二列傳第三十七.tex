\article{卷七十二列傳第三十七}

\begin{pinyinscope}

 孝義《孝經》云:「夫孝,天之經也,地之義也,人之行也。」《論語》云:「君子務本,本立而道生。孝悌也者,其為仁之本與!」《呂覽》云:「夫孝,三皇、五帝之本務,萬事之綱紀也。執一術而百善至,百邪去,天下順者,其唯孝乎!」



 然則孝之為德至矣,其為道遠矣,其化人深矣。故聖帝明王行之於四海,則與天地合其德,與日月齊其明。諸侯卿大夫行之於國家,
 則永保其宗社,長守其祿位。匹夫匹婦行之於閭閻,則播微烈於當年,揚休名於千載。此皆資純至以感物,故聖哲之所重。田翼、郎方貴等,闕稽古之學,無俊偉之才,並能任其自然,情無矯飾。



 篤於天性,勤其四體,竭股肱之力,盡愛敬之心,自足膝下之歡,忘懷軒冕之貴。



 不言之化,人神通感。雖或位登臺輔,爵列王侯,祿積萬鐘,馬逾千駟,死之日,曾不得與斯人之徒隸齒。孝之大也,不其然乎!故述其所行,為《孝義傳》。



 陸彥師陸彥師,字雲房,魏郡臨漳人。祖希道,魏定州刺史。父子
 彰,中書監。彥師少有行檢,為邦族所稱。長而好學,解屬文。魏襄城王元旭引為參軍事。以父艱去職,哀毀殆不勝喪。與兄卬廬於墓次,負土成墳。公卿重之,多就墓側存問,晦朔之際,車馬不絕。齊文宣聞而嘉嘆,旌表其閭,號其所住為孝終里。中書令河間邢邵表薦之,未報,彭城王浟為司州牧,召補主簿。後歷中外府東閣祭酒。兄卬當襲父始平侯,以彥師昆弟中最幼,表讓封焉。彥師固辭而止。時稱友悌孝義,總萃一門。遷中書舍人,尋轉通直散騎侍郎。每陳使至,必令高選主客,彥師所接對者,前後六輩。歷中書黃門侍郎,以不阿宦者,遇讒,出為
 中山太守,有惠政。數年,徵為吏部郎中。周武平齊,授載師下大夫。宣帝時,轉少納言,賜爵臨水縣男,奉使幽、薊。俄而高祖為丞相,彥師遇疾,請假還鄴。尉迥將為亂,彥師微知之,遂委妻子,潛歸長安。高祖嘉之,授內史下大夫,拜上儀同。高祖受禪,拜尚書左丞,進爵為子。彥師素多病,未幾,以務劇疾動,乞解所職,有詔聽以本官就第。歲餘,轉吏部侍郎。隋承周制,官無清濁,彥師在職,凡所任人,頗甄別於士庶,論者美之。後復以病出為汾州刺史,卒官。



 田德懋
 田德懋,觀國公仁恭之子也。少以孝友著名。開皇初,以父軍功賜爵平原郡公,授太子千牛備身。丁父艱,哀毀骨立,廬於墓側,負土成墳。上聞而嘉之,遣員外散騎侍郎元志就吊焉。復降璽書曰:「皇帝謝田德懋。知在窮疾,哀毀過禮,倚廬墓所,負土成墳。朕孝理天下,思弘名教,復與汝通家,情義素重,有聞孝感,嘉嘆兼深。春日暄和,氣力何似?宜自抑割,以禮自存也。」並賜縑二百匹,米百石。



 復下詔表其門閭。後歷太子舍人、義州司馬。大業中,為給事郎、尚書駕部郎,卒官。



 薛濬薛濬,字道賾,刑部尚書、內陽公胄之從祖弟也。父琰,周渭南太守。濬少喪父,早孤,養母以孝聞。幼好學,有志行,尋師於長安。時初平江陵,何妥歸國,見而異之,授以經業。周天和中,襲爵虞城侯,歷納言上士、新豐令。開皇初,擢拜尚書虞部侍郎,尋轉考功侍郎。帝聞濬事母至孝,以其母老,賜輿服機杖,四時珍味,當時榮之。後其母疾,濬貌甚憂瘁,親故弗之識也。暨丁母艱,詔鴻臚監護喪事,歸葬夏陽。於時隆冬極寒,濬衰絰徒跣,冒犯霜雪,自京及鄉,五百餘里,足凍墜指,瘡血流離,朝野為之傷痛。州里賵助,一無所受。尋起令視事,濬屢陳誠款,請終喪
 制,優詔不許。及至京,上見其毀瘠過甚,為之改容,顧謂群臣曰:「吾見薛濬哀毀,不覺悲感傷懷。」嗟異久之。濬竟不勝喪,病且卒。其弟謨時為晉王府兵曹參軍事,在揚州,濬遺書與謨曰:吾以不造,幼丁艱酷,窮游約處,屢絕簞瓢。晚生早孤,不聞《詩》《禮》,賴奉先人貽厥之訓,獲稟母氏聖善之規,負笈裹糧,不憚艱遠,從師就業,欲罷不能。砥行厲心,困而彌篤,服膺教義,爰至長成。自釋耒登朝,於茲二十三年矣。



 雖官非聞達,而祿喜逮親,庶保期頤,得終色養。何圖精誠無感,禍酷薦臻,兄弟俱被奪情,苫廬靡申哀訴。是用扣心泣血,隕氣摧魂者也。既而創巨
 釁深,不勝荼毒,啟手啟足,幸及全歸。使夫死而有知,得從先人於地下矣,豈非至願哉。但念爾伶俜孤宦,遠在邊服,顧此恨恨,如何可言。適已有書,冀得與汝面訣,忍死待汝,已歷一旬。汝既未來,便成今古,緬然永別,為恨何言。勉之哉,勉之哉!



 書成而絕,時年四十二。有司以聞,高祖為之屑涕,降使齎冊書吊祭曰:「皇帝咨故考功侍郎薛濬:於戲!惟爾操履貞和,器業詳敏,允膺列宿,勤謇克彰。及遘私艱,奄從毀滅。嘉爾誠孝,感於朕懷,奠酹有加,抑惟朝典。故遣使人,指申往命,魂而有靈,歆茲榮渥。嗚呼哀哉!」濬性清儉,死之日,家無遺資。濬初為童兒時,
 與宗中諸兒游戲於澗濱。見一黃蛇有角及足,召群兒共視,了無見者。濬以為不祥,歸大憂悴。母逼而問之,濬以實對。時有胡僧詣宅乞食,濬母怖而告之,僧曰:「此乃兒之吉應。且是兒也,早有名位,然壽不過六七耳。」言終而出,忽然不見,時咸異之。既而終於四十二,六七之言,於是驗矣。子乾福,武安郡司倉書佐。



 王頒王頒,字景彥,太原祁人也。祖神念,梁左衛將軍。父僧辯,太尉。頒少俶儻,有文武幹局。其父平侯景,留頒質於荊州,遇元帝為周師所陷,頒因入關。聞其父為陳武帝所
 殺,號慟而絕,食頃乃蘇,哭泣不絕聲,毀瘠骨立。至服闋,常布衣蔬食,藉槁而臥。周明帝嘉之,召授左侍上士,累遷漢中太守,尋拜儀同三司。開皇初,以平蠻功,加開府,封蛇丘縣公。獻取陳之策,上覽而異之,召與相見,言畢而歔欷,上為之改容。及大舉伐陳,頒自請行,率徒數百人,從韓擒先鋒夜濟。力戰被傷,恐不堪復鬥,悲感嗚咽。夜中因睡,夢有人授藥,比寤而創不痛,時人以為孝感。及陳滅,頒密召父時士卒,得千餘人,對之涕泣。其間壯士或問頒曰:「郎君來破陳國,滅其社稷,讎恥已雪,而悲哀不止者,將為霸先早死,不得手刃之邪?請發其丘壟,
 斷櫬焚骨,亦可申孝心矣。」頒頓顙陳謝,額盡流血,答之曰:「其為帝王,墳塋甚大,恐一宵發掘,不及其尸,更至明朝,事乃彰露,若之何?」



 諸人請具鍬鍤,一旦皆萃。於是夜發其陵,剖棺,見陳武帝須並不落,其本皆出自骨中。頒遂焚骨取灰,投水而飲之。既而自縛,歸罪於晉王。王表其狀,高祖曰:「朕以義平陳,王頒所為,亦孝義之道也,朕何忍罪之!」舍而不問。有司錄其戰功,將加柱國,賜物五千段,頒固辭曰:「臣緣國威靈,得雪怨恥,本心徇私,非是為國,所加官賞,終不敢當。」高祖從之。拜代州刺史,甚有惠政。母憂去職。



 後為齊州刺史,卒官,時年五十二。弟頍,
 見《文學傳》。



 楊慶楊慶,字伯悅,河間人也。祖玄,父剛,並以至孝知名。慶美姿儀,性辯慧。



 年十六,齊國子博士徐遵明見而異之。及長,頗涉書記。年二十五,郡察孝廉,以侍養不行。其母有疾,不解襟帶者七旬。及居母憂,哀毀骨立,負土成墳。齊文宣帝表其門閭,賜帛三十匹,綿十屯,粟五十石。高祖受禪,屢加褒賞,擢授儀同三司,版授平陽太守。年八十五,終於家。



 郭俊
 郭俊,字弘乂,太原文水人也。家門雍睦,七葉共居,犬豕同乳,烏鵲通巢,時人以為義感之應。州縣上其事,上遣平昌公宇文弼詣其家勞問之。治書御史柳彧巡省河北,表其門閭。漢王諒為並州總管,聞而嘉嘆,賜兄弟二十餘人衣各一襲。



 田翼田翼,不知何許人也。性至孝,養母以孝聞。其後母臥疾歲餘,翼親易燥濕,母食則食,母不食則不食。母患暴痢,翼謂中毒,遂親嘗惡。及母終,翼一慟而絕,其妻亦不勝哀而死,鄉人厚共葬之。



 紐回紐回,字孝政,河東安邑人也。性至孝,周武成中,父母喪,廬於墓側,負土成墳。廬前生麻一株,高丈許,圍之合拱,枝葉鬱茂,冬夏恆青。有烏棲其上,回舉聲哭,烏即悲鳴,時人異之。周武帝表其閭,擢授甘棠令。開皇初卒。



 子士雄,少質直孝友,喪父,復廬於墓側,負土成墳。其庭前有一槐樹,先甚鬱茂,及士雄居喪,樹遂枯死。服闋還宅,死樹復榮。高祖聞之,嘆其父子至孝,下詔褒揚,號其所居為累德里。



 劉士俊
 劉士俊,彭城人也。性至孝,丁母喪,絕而復蘇者數矣。勺飲不入口者七日,廬於墓側,負土成墳,列植松柏。狐狼馴擾,為之取食。高祖受禪,表其門閭。



 郎方貴郎方貴,淮南人也。少有志尚,與從父弟雙貴同居。開皇中,方貴嘗因出行遇雨,淮水泛長,於津所寄渡,船人怒之,撾方貴臂折。至家,其弟雙貴驚問所由,方貴具言之。雙貴恚恨,遂向津毆擊船人致死。守津者執送之縣官,案問其狀,以方貴為首,當死,雙貴從坐,當流。兄弟二人爭為首坐,縣司不能斷,送詣州。兄弟各引咎,州不能定,
 二人爭欲赴水而死。州狀以聞,上聞而異之,特原其罪,表其門閭,賜物百段,後為州主簿。



 翟普林翟普林,楚丘人也。性仁孝,事親以孝聞。州郡闢命,皆固辭不就,躬耕色養,鄉鄰謂為楚丘先生。後父母疾,親易燥濕,不解衣者七旬。大業初,父母俱終,哀毀殆將滅性。廬於墓側,負土為墳,盛冬不衣繒絮,唯著單縗而已。家有一烏犬,隨其在墓,若普林哀臨,犬亦悲號,見者嗟異焉。有二鵲巢其廬前柏樹,每入其廬,馴狎無所驚懼。大業中,司隸巡察,奏其孝感,擢授孝陽令。



 李德饒李德饒,趙郡柏人人也。祖徹,魏尚書右丞。父純,開皇中為介州長史。德饒少聰敏好學,有至性,宗黨咸敬之。弱冠為校書郎,仍直內史省,參掌文翰。轉監察御史,糾正不避貴戚。大業三年,遷司隸從事,每巡四方,理雪冤枉,褒揚孝悌。



 雖位秩未通,其德行為當時所重。凡與交結,皆海內髦彥。性至孝,父母寢疾,輒終日不食,十旬不解衣。及丁憂,水漿不入口五日,哀慟嘔血數升。及送葬之日,會仲冬積雪,行四十餘里,單縗徒跣,號踴幾絕。會葬者千餘人,莫不為之流涕。



 後甘露降於庭樹,有鳩巢其
 廬。納言楊達巡省河北,詣其廬吊慰之,因改所居村名孝敬村,里為和順里。後為金河長,未之官,值群盜蜂起,賊帥格謙、孫宣雅等十餘頭,聚眾於渤海。時有敕許其歸首,謙等懼不敢降,以德饒信行有聞,遣使奏曰:「若使德饒來者,即相率歸首。」帝於是遣德饒往渤海慰諭諸賊。行至冠氏,會他盜攻陷縣城,德饒見害。



 其弟德佋,性重然諾。大業末,為離石郡司法書佐,太守楊子崇特禮之。及義兵起,子崇遇害,棄尸城下,德佋赴哭盡哀,收瘞之。至介休,詣義師,請葬子崇。



 大將軍嘉之,因贈子崇官,令德佋為使者,往離石禮葬子崇焉。



 華秋華秋,汲郡臨河人也。幼喪父,事母以孝聞。家貧,傭賃為養。其母遇患,秋容貌毀悴,須鬢頓改,州里咸嗟異之。及母終之後,遂絕櫛沐,發盡禿落。廬於墓側,負土成墳,有人欲助之者,秋輒拜而止之。大業初,調狐皮,郡縣大獵。有一兔,人逐之,奔入秋廬中,匿秋膝下。獵人至廬所,異而免之。自爾此兔常宿廬中,馴其左右。郡縣嘉其孝感,俱以狀聞。煬帝降使勞問,表其門閭。後群盜起,常往來廬之左右,咸相誡曰:「勿犯孝子。」鄉人賴秋而全者甚眾。



 徐孝肅
 徐孝肅,汲郡人也。宗族數千家,多以豪侈相尚,唯孝肅性儉約,事親以孝聞。



 雖在幼齒,宗黨間每有爭訟,皆至孝肅所平論之,為孝肅所短者,無不引咎而退。



 孝肅早孤,不識父,及長,問其母父狀,因求畫工,圖其形像,構廟置之而定省焉,朔望享祭。養母至孝,數十年,家人未見其有忿恚之色。及母老疾,孝肅親易燥濕,憂悴數年,見者無不悲悼。母終,孝肅茹蔬飲水,盛冬單縗,毀瘠骨立。祖父母、父母墓皆負土成墳,廬於墓所四十餘載,被發徒跣,遂以身終。



 其弟德備,聰敏,通涉五經,河朔間稱為儒者。德備終,子處默又廬於墓側,奕葉稱孝焉。



 史臣曰:昔者弘愛敬之理,必籍王公大人,近古敦孝友之情,多茅屋之下。而彥師、道賾,或家傳纓冕,或身誓山河,遂乃負土成墳,致毀滅性。雖乖先王之制,亦觀過以知仁矣。郎貴昆弟,爭死而身全,田翼夫妻,俱喪而名立,德饒仁懷群盜,德佋義感興王,亦足稱也。紐回、劉俊之倫,翟林、華秋之輩,或茂草嘉樹榮枯於庭宇,或走獸翔禽馴狎於廬墓,非夫孝悌之至,通於神明者乎!



\end{pinyinscope}