\article{卷三十九列傳第四 於義子宣道 宣敏}

\begin{pinyinscope}

 於
 義,字慈恭,河南洛陽人也。父謹,從魏武帝入關,仕周,官至太師,因家京兆。義少矜嚴,有操尚,篤志好學。大統末,以父功,賜爵平昌縣伯,邑五百戶。



 起家直閤將軍。其後改封廣都縣公。周閔帝受禪,增邑六百戶。累遷安武太守,專崇德教,不尚威刑。有郡民張善安、王叔兒爭財相訟,義曰:「太守德薄不勝任之所致,非其罪也。」於是取
 家財,倍與二人,喻而遣去。善安等各懷恥愧,移貫他州。於是風教大洽。其以德化人,皆此類也。進封建平郡公。明、武世,歷西京、瓜、邵三州刺史。數從征伐,進位開府。宣帝嗣位,政刑日亂,義上疏諫。時鄭譯、劉昉以恩幸當權,謂義不利於己,先惡之於帝。帝覽表色動,謂侍臣曰:「於義謗訕朝廷也。」御正大夫顏之儀進曰:「古先哲王立誹謗之木,置敢諫之鼓,猶懼不聞過。於義之言,不可罪也。」帝乃解。及高祖作相,王謙構逆,高祖將擊之,問將於高熲。熲答曰:「於義素有經略,可為元帥。」高祖初然之。劉昉進曰:「梁睿位望素重,不可居義之下。」高祖乃止。於是以
 睿為元帥,以義為行軍總管。謙將達奚惎擁眾據開遠,義將左軍擊破之。尋拜潼州總管,賜奴婢五百口,雜彩三千段,超拜上柱國。時義兄翼為太尉,弟智、兄子仲文並上柱國,大將軍已上十餘人,稱為貴戚。歲餘,以疾免職,歸於京師。數月卒,時年五十。贈豫州刺史,謚曰剛。



 賻物千段,粟米五百石。子宣道、宣敏,並知名。



 宣道字元明,性謹密,不交非類。仕周,釋褐左侍上士。以父功,賜爵成安縣男,邑二百戶。後轉小承御上士。高祖為丞相,引為外兵曹,尋拜儀同。及踐阼,遷內史舍人,進爵為子。丁父憂,水漿不入口者累日。獻皇后命中使敦
 諭,歲餘,起令視事。免喪,拜車騎將軍,兼左衛長史,舍人如故。後六歲,遷太子左衛副率,進位上儀同。卒,年四十二。子志寧,早知名,出繼叔父宣敏。



 宣敏字仲達,少沉密,有才思。年十一,詣周趙王招,王命之賦詩。宣敏為詩,甚有幽貞之志。王大奇之,坐客莫不嗟賞。起家右侍上士,遷千牛備身。高祖踐阼,拜奉車都尉,奉使撫慰巴蜀。及還,上疏曰:臣聞開盤石之宗,漢室於是惟永;建維城之固,周祚所以靈長。昔秦皇置牧守而罷諸侯,魏後暱諂邪而疏骨肉,遂使宗社移於他族,神器傳於異姓。此事之明,甚於觀火。然山川設險,非親
 勿居。且蜀土沃饒,人物殷阜,西通邛僰,南屬荊巫。



 周德之衰,茲土遂成戎首;炎政失御,此地便為禍先。是以明者防於無形,治者制其未亂,方可慶隆萬世,年逾七百。伏惟陛下日角龍顏,膺樂推之運,參天貳地,居揖讓之期,億兆宅心。百神受職,理須樹建籓屏,封植子孫,繼周、漢之宏圖,改秦、魏之覆軌,抑近習之權勢,崇公族之本枝。但三蜀、三齊,古稱天險,分王戚屬,今正其時。若使利建合宜,封樹得所,巨猾息其非望,奸臣杜其邪謀。盛業洪基,同天地之長久;英聲茂實,齊日月之照臨。臣雖學謝多聞,然情深體國,輒申管見,戰灼惟深。



 帝省表嘉之,
 謂高熲曰:「於氏世有人焉。」竟納其言,遣蜀王秀鎮於蜀。宣敏常以盛滿之誡,昔賢所重,每懷靜退,著《述志賦》以見其志焉。未幾,卒官,時年二十九。



 陰壽子世師骨儀陰壽,字羅雲,武威人也。父嵩,周夏州刺史。壽少果烈,有武干,性謹厚,敦然諾。周世屢以軍功,拜儀同。從武帝平齊,進位開府,賜物千段,奴婢百口,女樂二十人。及高祖為丞相,引壽為掾。尉迥作亂,高祖以韋孝寬為元帥擊之,令壽監軍。時孝寬有疾,不能親總戎事,每臥帳中,遣婦人傳教命。三軍綱紀,皆取決於壽。以功進位上柱國。
 尋以行軍總管鎮幽州,即拜幽州總管,封趙國公。時有高寶寧者,齊氏之疏屬也,為人桀黠,有籌算,在齊久鎮黃龍。及齊滅,周武帝拜為營州刺史,甚得華夷之心。高祖為丞相,遂連結契丹、靺鞨舉兵反。高祖以中原多故,未遑進討,以書喻之而不得。開皇初,又引突厥攻圍北平。至是,令壽率步騎數萬,出盧龍塞以討之。寶寧求救於突厥。時衛王爽等諸將數道北征,突厥不能援。寶寧棄城奔於磧北,黃龍諸縣悉平。壽班師,留開府成道昂鎮之。寶寧遣其子僧伽率輕騎掠城下而去。尋引契丹、靺鞨之眾來攻,道昂苦戰連日乃退。壽患之,於是重購
 寶寧,又遣人陰間其所親任者趙世模、王威等。月餘,世模率其眾降,寶寧復走契丹,為其麾下趙修羅所殺,北邊遂安。賜物千段。未幾,卒官,贈司空。



 子世師嗣。



 世師少有節概,性忠厚,多武藝。弱冠,以功臣子拜儀同,累遷驃騎將軍。煬帝嗣位,領東都瓦工監。後三歲,拜張掖太守。先是,吐谷渾及黨項羌屢為侵掠,世師至郡,有來寇者,親自捕擊,輒擒斬之,深為戎狄所憚。入為武賁郎將。遼東之役,出襄平道。明年,帝復擊高麗,以本官為涿郡留守。於時盜賊蜂起,世師逐捕之,往往克捷。及帝還,大加賞勞,拜樓煩太守。時帝在汾陽宮,世師聞始畢
 可汗將為寇,勸帝幸太原。帝不從,遂有雁門之難。尋遷左翊衛將軍,與代王留守京師。及義軍至,世師自以世荷隋恩,又籓邸之舊,遂勒兵拒守。月餘,城陷,與京兆郡丞骨儀等見誅,時年五十三。



 骨儀,京兆長安人也。性剛鯁,有不可奪之志。開皇初,為侍御史,處法平當,不為勢利所回。煬帝嗣位,遷尚書右司郎。於時朝政漸亂濁,貨賂公行,凡當樞要之職,無問貴賤,並家累金寶。天下士大夫莫不變節,而儀勵志守常,介然獨立。



 帝嘉其清苦,超拜京兆郡丞,公方彌著。時刑部尚書衛玄兼領京兆內史,頗行詭道,輒為儀所執正。
 玄雖不便之,不能傷也。及義兵至,而玄恐禍及己,遂稱老病,無所干預。儀與世師同心協契,父子並誅,其後遂絕。世師有子弘智等,以年幼獲全。



 竇榮定竇榮定,扶風平陵人也。父善,周太僕。季父熾,開皇初,為太傅。榮定沈深有器局,容貌瑰偉,美須髯,便弓馬。魏文帝時,為千牛備身。周太祖見而奇之,授平東將軍,賜爵宜君縣子,邑三百戶。後從太祖與齊人戰於北芒,周師不利,榮定與汝南公宇文神慶帥精騎二千邀擊之,齊師乃卻。以功拜上儀同。後從武元皇帝引突厥木桿侵
 齊之並州,賜物三百段。襲爵永富縣公,邑千戶,進位開府,除忠州刺史。從武帝平齊,加上開府,拜前將軍、佽飛中大夫。其妻則高祖姊安成長公主也。高祖少小與之情契甚厚,榮定亦知高祖有人君之表,尤相推結。及高祖作相,領左右宮伯,使鎮守天臺,總統露門內兩箱仗衛,常宿禁中。遇尉迥初平,朝廷頗以山東為意,乃拜榮定為洛州總管以鎮之。前後賜縑四千匹,西涼女樂一部。



 高祖受禪,來朝京師。上顧謂群臣曰:「朕少惡輕薄,性相近者,唯竇榮定而已。」賜馬三百匹,部曲八十戶而遣之。坐事除名,高祖以長公主之故,尋拜右武候大將軍。
 上數幸其第,恩賜甚厚。每令尚食局日供羊一口,珍味稱是。以佐命功,拜上柱國、寧州刺史。未幾,復為右武候大將軍。尋除秦州總管,賜吳樂一部。突厥沙缽略寇邊,以為行軍元帥,率九總管,步騎三萬,出涼州。與虜戰於高越原,兩軍相持,其地無水,士卒渴甚,至刺馬血而飲,死者十有二三。榮定仰天太息。



 俄而澍雨,軍乃復振。於是進擊,數挫其鋒,突厥憚之,請盟而去。賜縑萬匹,進爵安豐郡公,增邑千六百戶。復封子憲為安康郡公,賜縑五千匹。歲餘,拜右武衛大將軍,俄轉左武衛大將軍。上欲以為三公,榮定上書曰:「臣每觀西朝衛、霍,東都梁、鄧,
 幸托葭莩,位極臺鉉,寵積驕盈,必致傾覆。向使前賢,少自貶損,遠避權勢,推而不居,則天命可保,何覆宗之有!臣每覽前修,實為畏懼。」上於是乃止。前後賞賜,不可勝計。開皇六年卒,時年五十七。上為之廢朝,令左衛大將軍元旻監護喪事,賻縑三千匹。上謂侍臣曰:「吾每欲致榮定於三事,其人固讓不可。今欲贈之,重違其志。」於是贈冀州刺史、陳國公,謚曰懿。子抗嗣。



 抗美容儀,性通率,長於巧思。父卒之後,恩遇彌隆,所賜錢帛金寶,亦以巨萬。抗官至定州刺史,復檢校幽州總管。煬帝即位,漢王諒構逆,以為抗與通謀,由是除名,以其弟慶襲封陳公
 焉。



 慶亦有姿儀,性和厚,頗工草隸。初封永富郡公,官至河東太守、衛尉卿。大業之末,出為南郡太守,為盜賊所害。



 慶弟璇,亦工草隸,頗解鐘律。官歷潁川、南郡、扶風太守。



 元景山元景山,字珤岳,河南洛陽人也。祖燮,魏安定王。父琰,宋安王。景山少有器局,幹略過人。周閔帝時,從大司馬賀蘭祥擊吐谷渾,以功拜撫軍將軍。其後數從征伐,累遷儀同三司,賜爵文昌縣公,授亹川防主。後與齊人戰於北邙,斬級居多,加開府,遷建州刺史,進封宋安郡公,邑
 三千戶。從武帝平齊,每戰有功,拜大將軍,改封平原郡公,邑二千戶,賜女樂一部,帛六千匹,奴婢二百五十口,牛羊數千。



 治亳州總管。先是,州民王回洛、張季真等聚結亡命,每為劫盜。前後牧守不能制。景山下車,逐捕之,回洛、季真挺身奔江南。禽其黨與數百人,皆斬之。法令明肅,盜賊屏跡,稱為大治。陳人張景遵以淮南內屬,為陳將任蠻奴所攻,破其數柵。景山發譙、潁兵援之,蠻奴引軍而退。徵為候正。宣帝嗣位,從上柱國韋孝寬經略淮南。鄖州總管宇文亮謀圖不軌,以輕兵襲孝寬。孝寬窘迫,未得整陣,為亮所薄。景山率鐵騎三百出擊,破之,
 斬亮傳首。以功拜亳州總管。



 高祖為丞相,尉迥稱兵作亂。滎州刺史宇文胄與迥通謀,陰以書諷動景山。景山執其使,封書詣相府。高祖甚嘉之,進位上大將軍。司馬消難之以鄖州入陳也,陳遣將樊毅、馬傑等來援。景山率輕騎五百馳赴之。毅等懼,掠居民而遁。景山追之,一日一夜行三百餘里,與毅戰於漳口,二合皆克。毅等退保甑山鎮。其城邑為消難所陷者,悉平之。拜安州總管,進位柱國,前後賜帛二千匹。時桐柏山蠻相聚為亂,景山復擊平之。



 高祖受禪,拜上柱國。明年,大舉伐陳,以景山為行軍元帥,率行軍總管韓延、呂哲出漢口。遣上開
 府鄧孝儒將勁卒四千,攻陳甑山鎮。陳人遣其將陸綸以舟師來援。孝儒逆擊,破之。陳將魯達、陳紀以兵守溳口,景山復遣兵擊走之。陳人大駭,甑山、沌陽二鎮守將皆棄城而遁。景山將濟江,會陳宣帝卒,有詔班師。景山大著威名,甚為敵人所憚。後數載,坐事免,卒於家。時年五十五。贈梁州總管,賜縑千匹,謚曰襄。子成壽嗣。



 成壽便弓馬,起家千牛備身。以上柱國世子,拜儀同。後為秦王庫真車騎。煬帝嗣位,徵為左親衛郎將。楊玄感之亂也,從刑部尚書衛玄擊之,以功進位正議大夫,拜西平通守。



 源雄源雄,字世略,西平樂都人也。祖懷、父纂,俱為魏隴西王。雄少寬厚,偉姿儀。在魏起家秘書郎,尋加征虜將軍。屬其父為高氏所誅,雄脫身而遁,變姓名,西歸長安。周太祖見而器之,賜爵隴西郡公。後從武帝伐齊,以功授開府,改封朔方郡公,拜冀州刺史。時以突厥寇邊,徙雄為平州刺史以鎮之。未幾,檢校徐州總管。



 及高祖為丞相,尉迥作亂,時雄家累在相州,迥潛以書誘之,雄卒不顧。高祖遺雄書曰:「公妻子在鄴城,雖言離隔,賊徒翦滅,聚會非難。今日已後,不過數旬之別,遲能開慰,無以累懷。
 徐部大蕃,東南襟帶,密邇吳寇,特須安撫。藉公英略,委以邊謀,善建功名,用副朝委也。」迥遣其將畢義緒據蘭陵,席毗陷昌慮、下邑。雄遣徐州刺史劉仁恩擊義緒,儀同劉弘、李琰討席毗,悉平之。



 陳人見中原多故,遣其將陳紀、蕭摩訶、任蠻奴、周羅、樊毅等侵江北,西自江陵,東距壽陽,民多應之,攻陷城鎮。雄與吳州總管於顗、揚州總管賀若弼、黃州總管元景山等擊走之,悉復故地。東潼州刺史曹孝達據州作亂,雄遣兵襲斬之。



 進位上大將軍,拜徐州總管。後數歲,轉懷州刺史,尋遷朔州總管。突厥有來寇掠,雄輒捕斬之,深為北夷所憚。



 伐陳之
 役,高祖下冊書曰:「於戲!唯爾上大將軍、朔方公雄,識悟明允,風神果毅。往牧徐方,時逢寇逆,建旟馬邑,安撫北蕃。嘉謀絕外境之虞,挺劍息韋韝之望。沙漠以北,俱荷威恩,呂梁之間,罔不懷惠。但江淮蕞爾,有陳僭逆,今將董率戎旅,清彼東南,是用命爾為行軍總管。往欽哉!」於是從秦王俊出信州道。



 及陳平,以功進位上柱國。賜子崇爵端氏縣伯,褒為安化縣伯,賜物五千段,復鎮朔州。二歲,上表乞骸骨,徵還京師,卒於家,時年七十。



 子崇嗣,官至儀同。大業中,自上黨贊治入為尚書虞部郎。及天下盜起,將兵討北海,與賊力戰而死,贈正議大夫。



 豆盧勣子毓勣兄通豆盧勣,字定東,昌黎徒河人也。本姓慕容,燕北地王精之後也。中山敗,歸魏,北人謂歸義為「豆盧」,因氏焉。祖萇,魏柔玄鎮大將。父寧,柱國、太保。



 勣初生時,周太祖親幸寧家稱慶,時遇新破齊師,太祖因字之曰定東。勣聰悟,有器局。少受業國子學,略涉文藝。魏大統十二年,太祖以勣勛臣子,封義安縣侯。



 周閔帝受禪,授稍伯下大夫、開府儀同三司,改封丹陽郡公,邑千五百戶。明帝時,為左武伯中大夫。勣自以經業未通,請解職游露門學。帝嘉之,敕以本官就學。未幾,齊王憲納勣妹為妃,恩禮逾
 厚。



 會武帝嗣位,拜邛州刺史。未之官,渭源燒當羌因饑饉作亂,以勣有才略,轉渭州刺史。甚有惠政,華夷悅服,德澤流行,大致祥瑞。鳥鼠山俗呼為高武隴,其下渭水所出,其山絕壁千尋,由來乏水,諸羌苦之。勣馬足所踐,忽飛泉湧出。有白鳥翔止前,乳子而後去,又白狼見於襄武。民為之謠曰:「我有丹陽,山出玉漿。濟我民夷,神鳥來翔。」百姓因號其泉為玉漿泉。後丁父艱,毀瘁過禮。天和二年,授邵州刺史,襲爵楚國公。復徵為天官府司會,歷信、夏二州總管、相州刺史。以母憂還京。宣帝大象二年,拜利州總管,進位上大將軍。月餘,拜柱國。



 高祖為
 丞相,益州總管王謙作亂。勣嬰城固守,謙遣其將達奚念、高阿那肱、乙弗虔等眾十萬攻之,起土山,鑿城為七十餘穴,堰江水以灌之。勣時戰士不過二千,晝夜相拒。經四旬,勢漸迫。勣於是出奇兵擊之,斬數千級,降二千人。梁睿軍且至,賊因而解去。高祖遣開府趙仲卿勞之,詔曰:「勣器識優長,氣調英遠,總馭籓部,風化已行。巴蜀稱兵,奄來圍逼,入守出戰,大摧兇醜。貞節雄規,厥功甚茂,可使持節、上柱國。賜一子爵中山縣公。」



 開皇二年,突厥犯塞,以勣為北道行軍元帥以備邊。歲餘,拜夏州總管。上以其家世貴盛,勛效克彰,甚重之。後為漢王諒納
 勣女為妃,恩遇彌厚。七年,詔曰:「上柱國、楚國公勣,蜀人寇亂之日,稱兵犯順,固守金湯,隱如敵國。嘉猷大節,其勞已多,可食始州臨津縣邑千戶。」十年,以疾徵還京師,詔諸王並至勣第,中使顧問,道路不絕。其年卒,時年五十五。上悼惜者久之,特加賵贈,鴻臚監護喪事,謚曰襄。子賢嗣,官至顯州刺史、大理少卿、武賁郎將。賢弟毓。



 毓字道生,少英果,有氣節。漢王諒出鎮並州,毓以妃兄為王府主簿。從趙仲卿北征突厥,以功授儀同三司。及高祖崩,煬帝即位,徵諒入朝。諒納諮議王頍之謀,發兵作亂。毓苦諫不從,因謂弟懿曰:「吾匹馬歸朝,自得免禍。
 此乃身計,非為國也。今且偽從,以思後計。」毓兄顯州刺史賢言於帝曰:「臣弟毓素懷志節,必不從亂,但逼兇威,不能克遂。臣請從軍,與毓為表裏,諒不足圖也。」帝以為然,許之。賢密遣家人齎敕書至毓所,與之計議。諒出城,將往介州,令毓與總管屬硃濤留守。毓謂濤曰:「漢王構逆,敗不旋踵,吾豈坐受夷滅,孤負家國邪!當與卿出兵拒之。」濤驚曰:「王以大事相付,何得有是語!」因拂衣而去。毓追斬之。時諒司馬皇甫誕前以諫諒被囚,毓於是出誕,與之協計,及開府、盤石侯宿勤武,開府宇文永昌,儀同成端、長孫愷,車騎、安成侯元世雅,原武令皇甫文顥
 等,閉城拒諒。部分未定,有人告諒,諒襲擊之。毓見諒至,紿其眾曰:「此賊軍也。」



 諒攻城南門,毓時遣稽胡守堞,稽胡不識諒,射之,箭下如雨。諒復至西門,守兵皆並州人,素識諒,即開門納之。毓遂見害,時年二十八。及諒平,煬帝下詔曰:「褒顯名節,有國通規,加等飾終,抑推令典。毓深識大義,不顧姻親,出於萬死,首建奇策。去逆歸順,殉義亡身,追加榮命,宜優恆禮。可贈大將軍,封正義縣公,賜帛二千匹,謚曰愍。



 子願師嗣,尋拜儀同三司。大業初,行新令,五等並除。未幾,帝復下詔曰:「故大將軍、正義愍公毓,臨節能固,捐生殉國,成為令典,沒世不忘。象賢無
 墜,德隆必祀,改封雍丘愍侯。」復以願師承襲。大業末,授千牛左右。



 通字平東,勣之兄也,一名會。弘厚有器局。在周,少以父功,賜爵臨貞縣侯,邑千戶。尋授大都督,俄遷儀同三司。大塚宰宇文護引之令督親信兵,改封沃野縣公,邑四千七百戶。後加開府,歷武賁中大夫、北徐州刺史。及高祖為丞相,尉迥作逆,遣其所署莒州刺史烏丸尼率眾來攻。通逆擊,破之。賜物八百段,進位大將軍。開皇初,進爵南陳郡公。尋徵入朝,以本官典宿衛。歲餘,出拜定州刺史。後轉相州刺史。尚高祖妹昌樂長公主,自是恩禮
 漸隆。遷夏州總管、洪州總管。所在之職,並稱寬惠。十七年,卒官,年五十九。謚曰安。有子寬。



 賀若誼賀若誼字道機,河南洛陽人也。祖伏連,魏雲州刺史。父統,右衛將軍。誼性剛果,有幹略。在魏以功臣子賜爵容城縣男。累遷直閤將軍、大都督、通直散騎常侍、尚食典御。周太祖據有關中,引之左右。嘗使詣杏城,屬茹茹種落摧貳,屯於河表。誼因譬以禍福,誘令歸附,降者萬餘口。太祖深奇之,賜金銀百兩。齊遣其舍人楊暢結好於茹茹,太祖恐其並力,為邊境之患,使誼聘茹茹。誼因啖
 以厚利,茹茹信之,遂與周連和,執暢付誼。太祖嘉之,拜車騎大將軍、儀同三司、略陽公府長史。周閔帝受禪,除司射大夫,改封霸城縣子,轉左宮伯,尋加開府。後歷靈邵二州刺史,原信二州總管,俱有能名。其兄敦,為金州總管,以讒毀伏誅。坐是免職。



 武帝親總萬機,召誼治熊州刺史。平齊之役,誼率兵出函谷,先據洛陽,即拜洛州刺史,進封建威縣侯。齊範陽王高紹義之奔突厥也,誼以兵追之,戰於馬邑,遂擒紹義。以功進位大將軍。高祖為丞相,拜亳州總管,馳驛之部。西遏司馬消難,東拒尉迥。申州刺史李慧反,誼討之,進爵範陽郡公,授上大將
 軍。



 開皇初,入為右武候將軍。河間王弘北征突厥,以誼為副元帥。軍還,轉左武候大將軍。坐事免。歲餘,拜華州刺史,俄轉敷州刺史,改封海陵郡公,復轉涇州刺史。時突厥屢為邊患,朝廷以誼素有威名,拜靈州刺史,進位柱國。誼時年老,而筋力不衰,猶能重鎧上馬,甚為北夷所憚。數載,上表乞骸骨,優詔許之。誼家富於財,於郊外構一別廬,多植果木。每邀賓客,列女樂,游集其間。卒於家,時年七十七。子舉襲爵。



 庶長子協,官至驃騎將軍。協弟祥,奉車都尉。祥弟與,車騎將軍。誼兄子弼,別有傳。



 史臣曰:於義、竇榮定等,或南陽姻亞,或豐邑舊游,運屬
 時來,俱宣力用。



 以勞定國,以功懋賞,保其祿位,貽厥子孫。析薪克荷,崇基弗墜,盛矣!豆盧毓遇屯剝之機,亡身殉義;陰世師遭天之所廢,舍命不渝。使夫死者有知,足以無愧君親矣。



\end{pinyinscope}