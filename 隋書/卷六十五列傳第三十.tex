\article{卷六十五列傳第三十}

\begin{pinyinscope}

 周羅周羅,字公布,九江尋陽人也。父法暠,仕梁冠軍將軍、始興太守、通直散騎常侍、南康內史,臨蒸縣侯。羅年十五,善騎射,好鷹狗,任俠放蕩,收聚亡命,陰習兵書。從祖景彥誡之曰:「吾世恭謹,汝獨放縱,難以保家。若不喪身,必將滅吾族。」羅終不改。陳宣帝時,以軍功授開遠將軍、句容令。後從大都督吳明徹與齊師戰於江陽,為
 流矢中其左目。齊師圍明徹於宿預也,諸軍相顧,莫有鬥心。羅躍馬突進,莫不披靡。太僕卿蕭摩訶因而副之,斬獲不可勝計。進師徐州,與周將梁士彥戰於彭城,摩訶臨陣墜馬,羅進救,拔摩訶於重圍之內,勇冠三軍。明徹之敗也,羅全眾而歸,拜光遠將軍、鐘離太守。十一年,授使持節、都督霍州諸軍事。平山賊十二洞,除右軍將軍、始安縣伯,邑四百戶,總管檢校揚州內外諸軍事。賜金銀三千兩,盡散之將士,分賞驍雄。陳宣帝深嘆美之。出為晉陵太守,進爵為侯,增封一千戶。除太僕卿,增封並前一千六百戶。尋除雄信將軍,使持節、都督
 豫章十郡諸軍事、豫章內史。獄訟庭決,不關吏手,民懷其惠,立碑頌德焉。至德中,除持節,都督南川諸軍事。江州司馬吳世興密奏羅甚得人心,擁眾嶺表,意在難測,陳主惑焉。蕭摩訶、魯廣達等保明之。外有知者,或勸其反,羅拒絕之。軍還,除太子左衛率,信任逾重,時參宴席。陳主曰:「周左率武將,詩每前成,文士何為後也?」都官尚書孔範對曰:「周羅執筆制詩,還如上馬入陣,不在人後。」自是益見親禮。出督湘州諸軍事,還拜散騎常侍。



 晉王廣之伐陳也,都督巴峽緣江諸軍事,以拒秦王俊,軍不得渡,相持逾月。



 遇丹陽陷,陳主被擒,上江猶不
 下,晉王廣遣陳主手書命之,羅與諸將大臨三日,放兵士散,然後乃降。高祖慰諭之,許以富貴。羅垂泣而對曰:「臣荷陳氏厚遇,本朝淪亡,無節可紀。陛下所賜,獲全為幸,富貴榮祿,非臣所望。」高祖甚器之。



 賀若弼謂之曰:「聞公郢、漢捉兵,即知揚州可得。王師利涉,果如所量。」羅答曰:「若得與公周旋,勝負未可知也。」其年秋,拜上儀同三司,鼓吹羽儀,送之於宅。先是,陳裨將羊翔歸降於我,使為鄉導,位至上開府,班在羅上。韓擒於朝堂戲之曰:「不知機變,立在羊翔之下,能無愧乎?」羅答曰:「昔在江南,久承令問,謂公天下節士。今日所言,殊匪誠
 臣之論。」擒有愧色。其年冬,除豳州刺史,俄轉涇州刺史,母憂去職。未期,復起,授豳州刺史,並有能名。



 十八年,起遼東之役,徵為水軍總管。自東萊泛海,趣平壤城,遭風,船多飄沒,無功而還。十九年,突厥達頭可汗犯塞,從楊素擊之。虜眾甚盛,羅白素曰:「賊陣未整,請擊之。」素許焉,與輕勇二十騎直沖虜陣,從申至酉,短兵屢接,大破之。進位大將軍。仁壽元年,為東宮右虞候率,賜爵義寧郡公,食邑一千五百戶。俄轉右衛率。煬帝即位,授右武候大將軍。漢王諒反,詔副楊素討平之,進授上大將軍。其年冬,帝幸洛陽。陳主卒,羅請一臨哭,帝許之。縗絰
 送至墓所,葬還,釋服而後入朝。帝甚嘉尚,世論稱其有禮。時諒餘黨據晉、絳等三州未下,詔羅行絳、晉、呂三州諸軍事,進兵圍之。為流矢所中,卒於師,時年六十四。



 送柩還京,行數里,無故輿馬自止,策之不動,有飄風旋繞焉。絳州長史郭雅稽顙咒曰:「公恨小寇未平邪?尋即除殄,無為戀恨。」於是風靜馬行,見者莫不悲嘆。



 其年秋七月,子仲隱夢見羅曰:「我明日當戰。」其靈坐所有弓箭刀劍,無故自動,若人帶持之狀。絳州城陷,是其日也。贈柱國、右翊衛大將軍,謚曰壯。贈物千段。子仲安,官至上開府。



 周法尚周法尚,字德邁,汝南安成人也。祖靈起,梁直閤將軍、義陽太守、廬桂二州刺史。父炅,定州刺史、平北將軍。法尚少果勁有風概,好讀兵書。年十八,為陳始興王中兵參軍,尋加伏波將軍。其父卒後,監定州事,督父本兵。數有戰功,遷使持節、貞毅將軍、散騎常侍,領齊昌郡事,封山陰縣侯,邑五千戶。以其兄武昌縣公法僧代為定州刺史。



 法尚與長沙王叔堅不相能,叔堅言其將反。陳宣帝執禁法僧,發兵欲取法尚。



 其下將吏皆勸之歸北,法尚猶豫未決。長史殷文則曰:「樂毅所以辭燕,良由不猶已。
 事勢如此,請早裁之。」法尚遂歸於周。宣帝甚優寵之,拜開府、順州刺史,封歸義縣公,邑千戶。賜良馬五匹,女妓五人,彩物五百段,加以金帶。陳將樊猛濟江討之,法尚遣部曲督韓明詐為背己,奔於陳,偽告猛曰:「法尚部兵不願降北,人皆竊議,盡欲叛還。若得軍來,必無鬥者,自當於陣倒戈耳。」猛以為然,引師急進。法尚乃陽為畏懼,自保於江曲。猛陳兵挑戰,法尚先伏輕舸於浦中,又伏精銳於古村之北,自張旗幟,迎流拒之。戰數合,偽退登岸,投古村,猛舍舟逐之,法尚又疾走。行數里,與村北軍合,復前擊猛。猛退走赴船,既而浦中伏舸取其舟楫,建
 周旗幟。猛於是大敗,僅以身免,虜八千人。



 高祖為丞相,司馬消難作亂,陰遣上開府段珣率兵陽為助守,因欲奪其城。法尚覺其詐,閉門不納,珣遂圍之。於時倉卒,兵散在外,因率吏士五百人守拒二十日。外無救援,自度力不能支,遂拔所領,棄城遁走。消難虜其母弟及家累三百人歸於陳。高祖受禪,拜巴州刺史,破三鵶叛蠻於鐵山,復從柱國王誼擊走陳寇。遷衡州總管四州諸軍事,改封譙郡公,邑二千戶。後上幸洛陽,召之,及引見,賜金鈿酒鐘一雙,彩五百段,良馬十五匹,奴婢三百口,給鼓吹一部。法尚固辭,上曰:「公有大功於國,特給鼓吹者,
 欲令公鄉人知朕之寵公也。」固與之。歲餘,轉黃州總管。上降密詔,使經略江南,伺候動靜。及伐陳之役,以行軍總管隸秦孝王,率舟師三萬出於樊口。陳城州刺史熊門超出師拒戰,擊破之,擒超於陣。轉鄂州刺史,尋遷永州總管,安集嶺南,賜縑五百段,良馬五匹,仍給黃州兵三千五百人為帳內。陳桂州刺史錢季卿、南康內史柳璇、西衡州刺史鄧暠、陽山太守毛爽等前後詣法尚降。陳定州刺史呂子廓據山洞反,法尚引兵逾嶺,子廓兵眾日散,與千餘人走保巖險,其左右斬之而降。賜彩五百段,奴婢五十口,並銀甕寶帶,良馬十匹。



 十年,尋轉桂
 州總管,仍為嶺南安撫大使。



 後數年入朝,以本官宿衛。賜彩三百段,米五百石,絹五百匹。未幾,桂州人李光仕舉兵作亂,令法尚與上柱國王世積討之。法尚馳往桂州,發嶺南兵,世積出岳州,徵嶺北軍,俱會於尹州。光仕來逆戰,擊走之。世積所部多遇瘴,不能進,頓於衡州,法尚獨討之。光仕帥勁兵保白石洞,法尚捕得其弟光略、光度,大獲家口。其黨有來降附,輒以妻子還之。居旬日,降者數千人。法尚捕兵列陣,以當光仕,親率奇兵,蔽林設伏。兩陣始交,法尚馳擊其柵,柵中人皆走散,光仕大潰,追斬之。賜奴婢百五十口,黃金百五十兩,銀百五十
 斤。仁壽中,遂州獠叛,復以行軍總管討平之。巂州烏蠻叛,攻陷州城,詔令法尚便道擊之。軍將至,賊棄州城,散走山谷間,法尚捕不能得。於是遣使慰諭,假以官號,偽班師,日行二十里。軍再舍,潛遣人覘之,知其首領盡歸柵,聚飲相賀。法尚選步騎數千人,襲擊破之,獲其渠帥數千人,虜男女萬餘口。賜奴婢百口,物三百段,蜀馬二十匹。軍還,檢校潞州事。



 煬帝嗣位,轉雲州刺史。後三歲,轉定襄太守,進位金紫光祿大夫。時帝幸榆林,法尚朝於行宮。內史令元壽言於帝曰:「漢武出塞,旍旗千里。今御營之外,請分為二十四軍,日別遣一軍發,相去三十
 里,旗幟相望,鉦鼓相聞,首尾連注,千里不絕。此亦出師之盛者也。」法尚曰:「不然,兵亙千里,動間山川,卒有不虞,四分五裂。腹心有事,首尾未知,道阻且長,難以相救。雖是故事,此乃取敗之道也。」帝不懌曰:「卿意以為如何?」法尚曰:「結為方陣,四面外距,六宮及百官家口並住其間。若有變起,當頭分抗,內引奇兵,出外奮擊,車為壁壘,重設鉤陳,此與據城理亦何異!若戰而捷,抽騎追奔,或戰不利,屯營自守。臣謂牢固萬全之策也。」帝曰:「善。」因拜左武衛將軍,賜良馬一匹,絹三百匹。



 明年,黔安夷向思多反,殺將軍鹿願,圍太守蕭造,法尚與將軍李景分路討
 之。



 法尚擊思多於清江,破之,斬首三千級。還,從討吐谷渾,法尚別出松州道,逐捕亡散,至於青海。賜奴婢一百口,物二百段,馬七十匹。出為敦煌太守,尋領會寧太守。遼東之役,以舟師指朝鮮道,會楊玄感反,與將軍宇文述,來護兒等破之。



 以功進右光祿大夫,賜物九百段。時有齊郡人王薄、孟讓等舉兵為盜,眾十餘萬,保長白山。頻戰,每挫其銳。賜奴婢百口。明年,復臨滄海,在軍疾甚,謂長史崔君肅曰:「吾再臨滄海,未能利涉,時不我與,將辭人世。立志不果,命也如何!」



 言畢而終,時年五十九。贈武衛大將軍,謚曰僖。有子六人。長子紹基,靈壽令,少子
 紹範,最知名。



 李景李景,字道興,天水休官人也。父超,周應、戎二州刺史。景容貌奇偉,膂力過人,美須髯,驍勇善射。平齊之役,頗有力焉,授儀同三司。以平尉迥,進位開府,賜爵平寇縣公,邑千五百戶。開皇九年,以行軍總管從王世積伐陳,陷陳有功,進位上開府,賜奴婢六十口,物千五百段。及高智慧等作亂江南,復以行軍總管從楊素擊之。別平倉嶺,還授鄜州刺史。十七年,遼東之役,為馬軍總管。及還,配事漢王。高祖奇其壯武,使袒而觀之,曰:「卿相表當位
 極人臣。」尋從史萬歲擊突厥於大斤山,別路邀賊,大破之。後與上明公楊紀送義成公主於突厥,至恆安,遇突厥來寇。時代州總管韓洪為虜所敗,景率所領數百人援之。力戰三日,殺虜甚眾,賜物三千段,授韓州刺史。以事王故,不之官。仁壽中,檢校代州總管。漢王諒作亂並州,景發兵拒之。諒遣劉暠襲景,戰於城東。升樓射之,無不應弦而倒。



 選壯士擊之,斬獲略盡。諒復遣嵐州刺史喬鐘葵率勁勇三萬攻之。景戰士不過數千,加以城池不固,為賊沖擊,崩毀相繼。景且戰且築,士卒皆殊死鬥,屢挫賊鋒。司馬馮孝慈、司法參軍呂玉並驍勇善戰,儀
 同三司侯莫陳乂多謀畫,工拒守之術。景知將士可用,其後推誠於此三人,無所關預,唯在閣持重,時出撫循而已。月餘,朔州總管楊義臣以兵來援,合擊,大破之。先是,景府內井中甃上生花如蓮,並有龍見,時變為鐵馬甲士。又有神人長數丈見於城下,其跡長四尺五寸。景問巫,對曰:「此是不祥之物,來食人血耳。」景大怒,推出之。旬日而兵至,死者數萬焉。



 景尋被徵入京,進位柱國,拜右武衛大將軍,賜縑九千匹,女樂一部,加以珍物。



 景智略非所長,而忠直為時所許,帝甚信之。擊叛蠻向思多,破之,賜奴婢八十口。明年,擊吐谷渾於青海,破之,進位
 光祿大夫。賜奴婢六十口,縑二千匹。



 五年,車駕西巡,至天水,景獻食於帝。帝曰:「公,主人也。」賜坐齊王暕之上。



 至隴川宮,帝將大獵,景與左武衛大將軍郭衍俱有難言,為人所奏。帝大怒,令左右Ξ之,竟以坐免。歲餘,復位,與宇文述等參掌選舉。明年,攻高麗武厲城,破之,賜爵苑丘侯,物一千段。八年,出渾彌道。九年,復出遼東。及旋師,以景為殿。高麗追兵大至,景擊走之。賚物三千段,進爵滑國公。楊玄感之反也,朝臣子弟多預焉,而景獨無關涉。帝曰:「公誠直天然,我之梁棟也。」賜以美女。帝每呼李大將軍而不名,其見重如此。十二年,帝令景營遼東戰
 具於北平,賜御馬一匹,名師子吉。會幽州賊楊仲緒率眾萬餘人來攻北平,景督兵擊破之,斬仲緒。於時盜賊蜂起,道路隔絕,景遂召募,以備不虞。武賁郎將羅藝與景有隙,遂誣景將反。



 帝遣其子慰諭之曰:「縱人言公窺天闕,據京師,吾無疑也。」後為高開道所圍,獨守孤城,外無聲援,歲餘,士卒患腳腫而死者十將六七,景撫循之,一無離叛。



 遼東軍資多在其所,粟帛山積,既逢離亂,景無所私焉。及帝崩於江都,遼西太守鄧暠率兵救之,遂歸柳城。後將還幽州,在道遇賊,見害。契丹、靺軻素感其恩,聞之莫不流涕,幽、燕人士於今傷惜之。有子世謨。



 慕容三藏慕容三藏,燕人也。父紹宗,齊尚書左僕射、東南道大行臺。三藏幼聰敏,多武略,頗有父風。仕齊,釋褐太尉府參軍事,尋遷備身都督。武平初,襲爵燕郡公,邑八百戶。其年,敗周師於孝水,又破陳師於壽陽,轉武衛將軍。又敗周師於河陽,授武衛大將軍。又轉右衛將軍,別封範陽縣公,食邑千戶。周師入鄴也,齊後主失守東遁,委三藏等留守鄴宮。齊之王公以下皆降,三藏猶率麾下抗拒周師。及齊平,武帝引見,恩禮甚厚,詔曰:「三藏父子誠節著聞,宜加榮秩。」授開府儀同大將軍。其年,稽胡叛,令三
 藏討平之。開皇元年,授吳州刺史。九年,奉詔持節涼州道黜陟大使。其年,嶺南酋長王仲宣反,圍廣州,詔令柱國、襄陽公韋洸為行軍總管,三藏為副。至廣州,與賊交戰,洸為流矢所中,卒,詔令三藏檢校廣州道行軍事。十年,賊眾四面攻圍,三藏固守月餘。城中糧少矢盡,三藏以為不可持久,遂自率驍銳,夜出突圍擊之。賊眾敗散,廣州獲全。以功授大將軍,賜奴婢百口,加以金銀雜物。十二年,授廓州刺史。州極西界,與吐谷渾鄰接,奸宄犯法者皆遷配彼州,流人多有逃逸。及三藏至,招納綏撫,百姓愛悅,繦負日至,吏民歌頌之。



 高祖聞其能,屢有勞
 問。其年,當州畜產繁孳,獲醍醐奉獻,賚物百段。十三年,州界連雲山響,稱萬年者三,詔頒郡國,仍遣使醮於山所。其日景雲浮於上,雉間兔馴壇側,使還具以聞,上大悅。十五年,授疊州總管。黨項羌時有翻叛,三藏隨便討平之,部內夷夏咸得安輯。仁壽元年,改封河內縣男。大業元年,授和州刺史。



 三年,轉任淮南郡太守,所在有惠政。其年,改授金紫光祿大夫。大業七年卒。



 三藏從子遐,為澶水丞,漢王反,抗節不從,以誠節聞。



 薛世雄薛世雄,字世英,本河東汾陰人也,其先寓居關中。父回,
 字道弘,仕周,官至涇州刺史。開皇初,封舞陰郡公,領漕渠監,以年老致事,終於家。世雄為兒童時,與群輩游戲,輒畫地為城郭,令諸兒為攻守之勢,有不從令者,世雄輒撻之,諸兒畏憚,莫不齊整。其父見而奇之,謂人曰:「此兒當興吾家矣。」年十七,從周武帝平齊,以功拜帥都督。開皇時,數有戰功,累遷儀同三司、右親衛車騎將軍。



 煬帝嗣位,番禺夷、獠相聚為亂,詔世雄討平之。遷右監門郎將。從帝征吐谷渾,進位通議大夫。



 世雄性廉謹,凡所行軍破敵之處,秋毫無犯,帝由是嘉之。帝嘗從容謂群臣曰:「我欲舉好人,未知諸君識不?」群臣咸曰:「臣等何能
 測聖心。」帝曰:「我欲舉者薛世雄。」群臣皆稱善。帝復曰:「世雄廉正節概,有古人之風。」於是超拜右翊衛將軍。歲餘,以世雄為玉門道行軍大將,與突厥啟民可汗連兵擊伊吾。師次玉門,啟民可汗背約,兵不至,世雄孤軍度磧。伊吾初謂隋軍不能至,皆不設備,及聞世雄兵已度磧,大懼,請降,詣軍門上牛酒。世雄遂於漢舊伊吾城東築城,號新伊吾,留銀青光祿大夫王威以甲卒千餘人戍之而還。天子大悅,進位正議大夫,賜物二千段。遼東之役,以世雄為沃沮道軍將,與宇文述同敗績於平壤。還次白石山,為賊所圍百餘重,四面矢下如雨。世雄以羸
 師為方陣,選勁騎二百先犯之,賊稍卻,因而縱擊,遂破之而還。所亡失多,竟坐免。明年,帝復征遼東,拜右候衛將軍,兵指蹋頓道。軍至烏骨城,會楊玄感作亂,班師。帝至柳城,以世雄為東北道大使,行燕郡太守,鎮懷遠。於時突厥頗為寇盜,緣邊諸郡多苦之,詔世雄發十二郡士馬,巡塞而還。十年,復從帝至遼東,遷左御衛大將軍,仍領涿郡留守。未幾,李密逼東都,中原騷動,詔世雄率幽、薊精兵將擊之。軍次河間,營於郡城南,河間諸縣並集兵,依世雄大軍為營,欲討竇建德。建德將家口遁,自選精銳數百,夜來襲之。先犯河間兵,潰奔世雄營。時遇
 雰霧晦冥,莫相辨識,軍不得成列,皆騰柵而走,於是大敗。世雄與左右數十騎遁入河間城,慚恚發病,歸於涿郡,未幾而卒,時年六十三。有子萬述、萬淑、萬鈞、萬徹,並以驍武知名。



 王仁恭王仁恭,字元實,天水上邽人也。祖建,周鳳州刺史。父猛,鄯州刺史。仁恭少剛毅修謹,工騎射。弱冠,州補主簿,秦孝王引為記室,轉長道令,遷車騎將軍。



 從楊素擊突厥於靈武,以功拜上開府,賜物三千段。以驃騎將軍典蜀王軍事。山獠作亂,蜀王命仁恭討破之,賜奴婢三百口。
 及蜀王以罪廢,官屬多罹其患。上以仁恭素質直,置而不問。煬帝嗣位,漢王諒舉兵反,從楊素擊平之。以功進位大將軍,拜呂州刺史,賜帛四千匹,女妓十人。歲餘,轉衛州刺史,尋改為汲郡太守,有能名。徵入朝,帝呼上殿,勞勉之,賜雜彩六百段,良馬二匹。遷信都太守,汲郡吏民扣馬號哭於道,數日不得出境,其得人情如此。遼東之役,以仁恭為軍將。及帝班師,仁恭為殿,遇賊,擊走之。進授左光祿大夫,賜絹六千段,馬四十匹。明年,復以軍將指扶餘道,帝謂之曰:「往者諸軍多不利,公獨以一軍破賊。古人云,敗軍之將不可以言勇,諸將其可任乎?今
 委公為前軍,當副所望也。」賜良馬十匹,黃金百兩。仁恭遂進軍,至新城,賊數萬背城結陣,仁恭率勁騎一千擊破之。賊嬰城拒守,仁恭四面攻圍。帝聞而大悅,遣舍人詣軍勞問,賜以珍物。進授光祿大夫,賜絹五千匹。會楊玄感作亂,其兄子武賁郎將仲伯預焉,仁恭由是坐免。尋而突厥屢為寇患,帝以仁恭宿將,頻有戰功,詔復本官,領馬邑太守。其年,始畢可汗率騎數萬來寇馬邑,復令二特勤將兵南過。時郡兵不滿三千,仁恭簡精銳逆擊,破之。



 其二特勤眾亦潰,仁恭縱兵乘之,獲數千級,並斬二特勤。帝大悅,賜縑三千匹。



 其後突厥復入定襄,仁
 恭率兵四千掩擊,斬千餘級,大獲六畜而歸。於時天下大亂,百姓饑餒,道路隔絕,仁恭頗改舊節,受納貨賄,又不敢輒開倉廩,賑恤百姓。其麾下校尉劉武周與仁恭侍婢奸通,恐事洩,將為亂,每宣言郡中曰:「父老妻子凍餒,填委溝壑,而王府君閉倉不救百姓,是何理也!」以此激怒眾,吏民頗怨之。



 其後仁恭正坐事,武周率其徒數十人大呼而入,因害之,時年六十。武周於是開倉賑給,郡內皆從之,自稱天子,署置百官,轉攻傍郡。



 權武權武,字武挵,天水人也。祖超,魏秦州刺史。父襲慶,周開
 府,從武元皇帝與齊師戰於並州,被圍百餘重。襲慶力戰矢盡,短兵接戰,殺傷甚眾,刀矛皆折,脫胄擲地,向賊大罵曰:「何不來斫頭也!」賊遂殺之。武以忠臣子,起家拜開府,襲爵齊郡公,邑千二百戶。武少果勁,勇力絕人,能重甲上馬。嘗倒投於井,未及泉,復躍而出,其拳捷如此。從王謙破齊服龍等五城,增邑八百戶。平齊之役,攻陷邵州,別下六城,以功增邑三百戶。宣帝時,拜勁捷左旅上大夫,進位上開府。



 高祖為丞相,引置左右。及受禪,增邑五百戶。後六歲,拜淅州刺史。伐陳之役,以行軍總管從晉王出六合,還拜豫州刺史。在職數年,以創業之舊,
 進位大將軍,檢校潭州總管。其年,桂州人李世賢作亂,武以行軍總管與武候大將軍虞慶則擊平之。慶則以罪誅,功竟不錄,復還於州。多造金帶,遺嶺南酋領,其人復答以寶物,武皆納之,由是致富。後武晚生一子,與親客宴集,酒酣,遂擅赦所部內獄囚。武常以南越邊遠,治從其俗,務適便宜,不依律令,而每言當今法急,官不可為。上令有司案其事,皆驗。上大怒,命斬之。武於獄中上書,言其父為武元皇帝戰死於馬前,以此求哀。由是除名為民。仁壽中,復拜大將軍,封邑如舊。未幾,授太子右衛率。煬帝即位,拜右武衛大將軍,坐事免,授桂州刺史。
 俄轉始安太守。久之,徵拜右屯衛大將軍,尋坐事除名。卒於家。有子弘。



 吐萬緒吐萬緒,字長緒,代郡鮮卑人也。父通,周郢州刺史。緒少有武略,在周,起家撫軍將軍,襲爵元壽縣公。數從征伐,累遷大將軍、少司武。高祖受禪,拜襄州總管,進封谷城郡公,邑二千五百戶。尋轉青州總管,頗有治名。歲餘,突厥寇邊,朝廷以緒有威略,徙為朔州總管,甚為北夷所憚。其後高祖潛有吞陳之志,轉徐州總管,令修戰具。及大舉濟江,以緒領行軍總管,與西河公紇豆陵、洪景屯
 兵江北。



 及陳平,拜夏州總管。晉王廣之在籓也,頗見親遇,及為太子,引為左虞候率。煬帝嗣位,漢王諒時鎮並州,帝恐其為變,拜緒晉、絳二州刺史,馳傳之官。緒未出關,諒已遣兵據蒲阪,斷河橋,緒不得進。詔緒率兵從楊素擊破之,拜左武候將軍。



 大業初,轉光祿卿。賀若弼之遇讒也,引緒為證,緒明其無罪,由是免官。歲餘,守東平太守。未幾,帝幸江都,路經其境,迎謁道傍。帝命升龍舟,緒因頓首陳謝往事。帝大悅,拜金紫光祿大夫,太守如故。遼東之役,請為先鋒,帝嘉之,拜左屯衛大將軍,率馬步數萬指蓋馬道。及班師,留鎮懷遠,進位左光祿大夫。
 時劉元進作亂江南,以兵攻潤州,帝征緒討之。緒率眾至楊子津,元進自茅浦將渡江,緒勒兵擊走。緒因濟江,背水為柵。明旦,元進來攻,又大挫之,賊解潤州圍而去。



 緒進屯曲阿,元進復結柵拒。緒挑之,元進出戰,陣未整,緒以騎突之,賊眾遂潰,赴江水而死者數萬。元進挺身夜遁,歸保其壘。偽署僕射硃燮、管崇等屯於毗陵,連營百餘里。緒乘勢進擊,復破之,賊退保黃山。緒進軍圍之,賊窮蹙請降,元進、硃燮僅以身免。於陣斬管崇及其將軍陸顗等五千餘人,收其子女三萬餘口,送江都宮。進解會稽圍。元進復據建安,帝令進討之,緒以士卒疲敝,
 請息甲待至來春。



 帝不悅,密令求緒罪失,有司奏緒怯懦違詔,於是除名為民,配防建安。尋有詔徵詣行在所,緒鬱鬱不得志,還至永嘉,發疾而卒。



 董純董純,字德厚,隴西成紀人也。祖和,魏太子左衛率。父升,周柱國。純少有膂力,便弓馬。在周仕歷司御上士、典馭下大夫,封固始縣男,邑二百戶。從武帝平齊,以功拜儀同,進爵大興縣侯,增邑通前八百戶。高祖受禪,進爵漢曲縣公,累遷驃騎將軍。後以軍功進位上開府。開皇末,以勞舊擢拜左衛將軍,尋改封順政縣公。漢王諒作亂
 並州,以純為行軍總管、河北道安撫副使,從楊素擊平之。以功拜柱國,進爵為郡公,增邑二千戶。轉左備身將軍,賜女妓十人,縑彩五千匹。數年,轉左驍衛將軍、彭城留守。齊王暕之得罪也,純坐與交通,帝庭譴之曰:「汝階緣宿衛,以至大官,何乃附傍吾兒,欲相離間也?」純曰:「臣本微賤下才,過蒙獎擢,先帝察臣小心,寵逾涯分,陛下重加收採,位至將軍。欲竭餘年,報國恩耳。比數詣齊王者,徒以先帝、先後往在仁壽宮,置元德太子及齊王於膝上,謂臣曰:『汝好看此二兒,勿忘吾言也。』臣奉詔之後,每於休暇出入,未嘗不詣王所。



 臣誠不敢忘先帝之言。
 於時陛下亦侍先帝之側。」帝改容曰:「誠有斯旨。」於是舍之。後數日,出為汶山太守。歲餘,突厥寇邊,朝廷以純宿將,轉為榆林太守。



 虜有至境,純輒擊卻之。會彭城賊帥張大彪、宗世模等眾至數萬,保懸薄山,寇掠徐、兗。帝令純討之。純初閉營不與戰,賊屢挑之不出,賊以純為怯,不設備,縱兵大掠。純選精銳擊之,合戰於昌慮,大破之,斬首萬餘級,築為京觀。賊魏騏驎眾萬餘人,據單父,純進擊,又破之。及帝重征遼東,復以純為彭城留守。東海賊彭孝才眾數千,掠懷仁縣,轉入沂水,保五不及山。純以精兵擊之,擒孝才於陣,車裂之,餘黨各散。時百姓思
 亂,盜賊日益,純雖頻戰克捷,所在蜂起。有人譖純怯懦,不能平賊,帝大怒,遣使鎖純詣東都。有司見帝怒甚,遂希旨致純死罪,竟伏誅。



 趙才趙才,字孝才,張掖酒泉人也。祖隗,魏銀青光祿大夫、樂浪太守。父壽,周順政太守。才少驍武,便弓馬,性粗悍,無威儀。周世為輿正上士。高祖受禪,屢以軍功遷上儀同三司。配事晉王,及王為太子,拜右虞候率。煬帝即位,轉左備身驃騎,後遷右驍衛將軍。帝以才籓邸舊臣,漸見親待。才亦恪勤匪懈,所在有聲。歲餘,轉右候衛將軍。從
 征吐谷渾,以為行軍總管,率衛尉卿劉權、兵部侍郎明雅等出合河道,與賊相遇,擊破之,以功進位金紫光祿大夫。及遼東之役,再出碣石道,還授左候衛將軍。俄遷右候衛大將軍。時帝每有巡幸,才恆為斥候,肅遏奸非,無所回避。在途遇公卿妻子有違禁者,才輒醜言大罵。多所援及,時人雖患其不遜,然才守正,無如之何。十年駕幸汾陽宮,以才留守東都。十二年,帝在洛陽,將幸江都。才見四海土崩,恐為社稷之患,自以荷恩深重,無容坐看亡敗,於是入諫曰:「今百姓疲勞,府藏空竭,盜賊蜂起,禁令不行。願陛下還京師,安兆庶,臣雖愚蔽,敢以死
 請。」帝大怒,以才屬吏。旬日,帝意頗解,乃令出之。



 帝遂幸江都,待遇逾暱。時江都糧盡,將士離心,內史侍郎虞世基、秘書監袁充等多勸帝幸丹陽。帝廷議其事,才極陳入京之策,世基盛言渡江之便。帝默然無言,才與世基相忿而出。宇文化及弒逆之際,才時在苑北,化及遣驍果席德方矯詔追之。



 才聞詔而出,德方命其徒執之,以詣化及。化及謂才曰:「今日之事,只得如此,幸勿為懷。」才默然不對。化及忿才無言,將殺之,三日乃釋。以本官從事,鬱鬱不得志。才嘗對化及宴飲,請勸其同謀逆者一十八人楊士覽等酒,化及許之。才執杯曰:「十八人止可
 一度作,勿復餘處更為。」諸人默然不對。行至聊城,遇疾。



 俄而化及為竇建德所破,才復見虜。心彌不平,數日而卒,時年七十三。



 仁壽、大業間,有蘭興浴、賀蘭蕃,俱為武候將軍,剛嚴正直,不避強御,咸以稱職知名。



 史臣曰:羅、法尚、李景、世雄、慕容三藏並以驍武之姿,當有事之日,致茲富貴,自取之也。仁恭初在汲郡,以清能顯達,後居馬邑,以貪吝敗亡,鮮克有終,惜矣!吐萬緒、董純各以立效當年,取斯高秩。緒請息兵見責,純遭譖毀被誅。



 大業之季,盜可盡乎!淫刑暴逞,能不及焉!趙才雖人而無儀,志在強直,固拒世基之議,可謂不茍同矣。
 權武素無行檢,不拘刑憲,終取黜辱,宜哉。



\end{pinyinscope}