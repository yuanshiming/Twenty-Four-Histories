\article{卷七十六列傳第四十一}

\begin{pinyinscope}

 文
 學《易》曰:「觀乎天文,以察時變,觀乎人文,以化成天下。」《傳》曰:「言,身之文也,言而不文,行之不遠。」故堯曰則天,表文明之稱,周云盛德,著煥乎之美。然則文之為用,其大矣哉!上所以敷德教於下,下所以達情志於上,大則經緯天地,作訓垂範,次則風謠歌頌,匡主和民。或離讒放逐之臣,途窮後門之士,道感軻而未遇,志鬱抑而不申,憤激
 委約之中,飛文魏闕之下,奮迅泥滓,自致青雲,振沈溺於一朝,流風聲於千載,往往而有。是以凡百君子,莫不用心焉。



 自漢、魏以來,迄乎晉、宋,其體屢變,前哲論之詳矣。暨永明、天監之際,太和、天保之間,洛陽、江左,文雅尤盛。於時作者,濟陽江淹、吳郡沈約、樂安任昉、濟陰溫子升、河間邢子才、巨鹿魏伯起等,並學窮書圃,思極人文,縟彩鬱於雲霞,逸響振於金石。英華秀發,波瀾浩蕩,筆有餘力,詞無竭源。方諸張、蔡、曹、王,亦各一時之選也。聞其風者,聲馳景慕,然彼此好尚,互有異同。江左宮商發越,貴於清綺,河朔詞義貞剛,重乎氣質。氣質則理勝其
 詞,清綺則文過其意,理深者便於時用,文華者宜於詠歌,此其南北詞人得失之大較也。若能掇彼清音,簡茲累句,各去所短,合其兩長,則文質斌斌,盡善盡美矣。梁自大同之後,雅道淪缺,漸乖典則,爭馳新巧。簡文、湘東,啟其淫放,徐陵、庾信,分路揚鑣。其意淺而繁,其文匿而彩,詞尚輕險,情多哀思。格以延陵之聽,蓋亦亡國之音乎!周氏吞並梁、荊,此風扇於關右,狂簡斐然成俗,流宕忘反,無所取裁。高祖初統萬機,每念鵶雕為樸,發號施令,咸去浮華。然時俗詞藻,猶多淫麗,故憲臺執法,屢飛霜簡。煬帝初習藝文,有非輕側之論,暨乎即位,一變其
 風。其《與越公書》、《建東都詔》、《冬至受朝詩》及《擬飲馬長城窟》,並存雅體,歸於典制。雖意在驕淫,而詞無浮蕩,故當時綴文之士,遂得依而取正焉。所謂能言者未必能行,蓋亦君子不以人廢言也。爰自東帝歸秦,逮乎青蓋入洛,四庾咸暨,九州攸同,江漢英靈,燕趙奇俊,並該天網之中,俱為大國之寶。言刈其楚,片善無遺,潤木圓流,不能十數,才之難也,不其然乎!時之文人,見稱當世,則範陽盧思道、安平李德林、河東薛道衡、趙郡李元操、巨鹿魏澹、會稽虞世基、河東柳抃、高陽許善心等,或鷹揚河朔,或獨步漢南,俱騁龍光,並驅雲路,各有本傳,論而敘
 之。其潘徽、萬壽之徒,或學優而不切,或才高而無貴仕,其位可得而卑,其名不可堙沒,今總之於此,為《文學傳》云。



 劉臻劉臻,字宣摯,沛國相人也。父顯,梁尋陽太守。臻年十八,舉秀才,為邵陵王東閣祭酒。元帝時,遷中書舍人。江陵陷沒,復歸蕭詧,以為中書侍郎。周塚宰宇文護闢為中外府記室,軍書羽檄,多成其手。後為露門學士,授大都督,封饒陽縣子,歷藍田令、畿伯下大夫。高祖受禪,進位儀同三司。左僕射高熲之伐陳也,以臻隨軍,典文翰,進
 爵為伯。皇太子勇引為學士,甚褻狎之。臻無吏乾,又性恍惚,耽悅經史,終日覃思,至於世事,多所遺忘。有劉訥者亦任儀同,俱為太子學士,情好甚密。臻住城南,訥住城東,臻嘗欲尋訥,謂從者曰:「汝知劉儀同家乎?」



 從者不知尋訥,謂臻還家,答曰:「知。」於是引之而去,既扣門,臻尚未悟,謂至訥家。乃據鞍大呼曰:「劉儀同可出矣。」其子迎門,臻驚曰:「此汝亦來耶?」



 其子答曰:「此是大人家。」於是顧盼,久之乃悟,叱從者曰:「汝大無意,吾欲造劉訥耳。」性好啖蜆,以音同父諱,呼為扁螺。其疏放多此類也。精於《兩漢書》,時人稱為漢聖。開皇十八年卒,年七十二。有集十
 卷行於世。



 王頍王頍,字景文,齊州刺史頒之弟也。年數歲,值江陵陷,隨諸兄入關。少好游俠,年二十,尚不知書。為其兄顒所責怒,於是感激,始讀《孝經》、《論語》,盡夜不倦。遂讀《左傳》、《禮》、《易》、《詩》、《書》,乃嘆曰:「書無不可讀者!」勤學累載,遂遍通五經,究其旨趣,大為儒者所稱。解綴文,善談論。



 年二十二,周武帝引為露門學士。每有疑決,多頍所為。而頍性識甄明,精力不倦,好讀諸子,偏記異書,當代稱為博物。又曉兵法,益有縱橫之志,每嘆不逢時,常以將相自許。開皇五
 年,授著作佐郎。尋令於國子講授。會高祖親臨釋奠,國子祭酒元善講《孝經》,頍與相論難,詞義鋒起,善往往見屈。高祖大奇之,超授國子博士。後坐事解職,配防嶺南。數載,授漢王諒府諮議參軍,王甚禮之。時諒見房陵及秦、蜀二王相次廢黜,潛有異志。頍遂陰勸諒繕治兵甲。及高祖崩,諒遂舉兵反,多頍之計也。頍後數進奇策,諒不能用。楊素至蒿澤,將戰,頍謂其子曰:「氣候殊不佳,兵必敗。汝可隨從我。」既而兵敗,頍將歸突厥,至山中,徑路斷絕,知必不免,謂其子曰:「吾之計數,不減楊素,但坐言不見從,遂至於此。不能坐受擒執,以成豎子名也。吾死
 之後,汝慎勿過親故。」於是自殺,瘞之石窟中。



 其子數日不得食,遂過其故人,竟為所擒。楊素求頍尸,得之,斬首,梟於太原。



 時年五十四。撰《五經大義》三十卷,有集十卷,並因兵亂,無復存者。



 崔儦崔儦,字岐叔,清河武城人也。祖休,魏青州刺史。父仲文,齊高陽太守。世為著姓。儦年十六,太守請為功曹,不就。少與範陽盧思道、隴西辛德源同志友善。



 每以讀書為務,負恃才地,忽略世人。大署其戶曰:「不讀五千卷書者,無得入此室。」數年之間,遂博覽群言,多所通涉。解屬文,
 在齊舉秀才,為員外散騎侍郎,遷殿中侍御史。尋與熊安生、馬敬德等議《五禮》,兼修律令。尋兼散騎侍郎,聘於陳。使還,待詔文林館。歷殿中、膳部、員外三曹郎中。儦與頓丘李若俱見稱重,時人為之語曰:「京師灼灼,崔儦、李若。」齊亡,歸鄉里,仕郡為功曹,州補主簿。開皇四年,徵授給事郎,尋兼內史舍人。後數年,兼通直散騎侍郎,聘於陳,還授員外散騎侍郎。越國公楊素時方貴幸,重儦門地,為子玄縱娶其女為妻。聘禮甚厚。親迎之始,公卿滿座,素令騎迎儦,人麃故敝其衣冠,騎驢而至。素推令上座,儦有輕素之色,禮甚倨,言又不遜。素忿然,拂衣而起,竟
 罷座。後數日,儦方來謝,素待之如初。仁壽中,卒於京師,時年七十二。子世濟。



 諸葛潁諸葛潁,字漢,丹陽建康人也。祖銓,梁零陵太守。父規,義陽太守。潁年八歲,能屬文,起家梁邵陵王參軍事,轉記室。侯景之亂,奔齊,待詔文林館。歷太學博士、太子舍人。周武平齊,不得調,杜門不出者十餘年。習《周易》、圖緯、《倉》、《雅》、《莊》、《老》,頗得其要。清辨有俊才,晉王廣素聞其名,引為參軍事,轉記室。及王為太子,除藥藏監。煬帝即位,遷著作郎,甚見親幸。出入臥內,帝每賜之曲宴,輒與皇后嬪
 御連席共榻。潁因間隙,多所譖毀,是以時人謂之「冶葛」。後錄恩舊,授朝散大夫。帝常賜潁詩,其卒章曰:「參翰長洲苑,侍講肅成門。名理窮研核,英華恣討論。實錄資平允,傳芳導後昆。」其見待遇如此。後征吐谷渾,加正議大夫。後從駕北巡,卒於道,年七十七。



 潁性褊急,與柳抃每相忿鬩,帝屢責怒之而猶不止,於後帝亦薄之。有集二十卷,撰《鑾駕北巡記》三卷,《幸江都道里記》一卷,《洛陽古今記》一卷,《馬名錄》二卷,並行於世。有子嘉會。



 孫萬壽孫萬壽,字仙期,信都武強人也。祖寶,魏散騎常侍。父靈
 暉,齊國子博士。



 萬壽年十四,就阜城熊安生受五經,略通大義,兼博涉子史。善屬文,美談笑,博陵李德林見而奇之。在齊,年十七,奉朝請。高祖受禪,滕穆王引為文學,坐衣冠不整,配防江南。行軍總管宇文述召典軍書。萬壽本自書生,從容文雅,一旦從軍,鬱鬱不得志,為五言詩贈京邑知友曰:賈誼長沙國,屈平湘水濱。江南瘴癘地,從來多逐臣。粵餘非巧宦,少小拙謀身。欲飛無假翼,思鳴不值晨。如何載筆士,翻作負戈人!飄飄如木偶,棄置同兇狗。失路乃西浮,非狂亦東走。晚歲出函關,方春度京口。石城臨獸據,天津望牛斗。牛鬥盛妖氛,梟獍已
 成群。郗超初入幕,王粲始從軍。裹糧楚山際,被甲吳江汶。吳江一浩蕩,楚山何糾紛。驚波上濺日,喬木下臨雲。擊越恆資辯,喻蜀幾飛文。魯連唯救患,吾彥不爭勛。羈游歲月久,歸思常搔首。非關不樹萱,豈為無杯酒!數載辭鄉縣,三秋別親友。壯志後風雲,衰鬢先蒲柳。心緒亂如絲,空懷疇昔時。昔時游帝裏,弱歲逢知己。旅食南館中,飛蓋西園裡。河間本好書,東平唯愛士。英辯接天人,清言洞名理。鳳池時寓直,麟閣常游止。勝地盛賓僚,麗景相攜招。舟泛昆明水,騎指渭津橋。祓除臨灞岸,供帳出東郊。宜城醖始熟,陽翟曲新調。繞樹烏啼夜,雊麥雉
 飛朝。細塵梁下落,長袖掌中嬌。歡娛三樂至,懷抱百憂銷。夢想猶如昨,尋思久寂寥。一朝牽世網,萬里逐波潮。回輪常自轉,懸旆不堪搖。登高視衿帶,鄉關白雲外。回首望孤城,愁人益不平。華亭宵鶴唳,幽谷早鶯鳴。斷絕心難續,惝恍魂屢驚。群紀通家好,鄒魯故鄉情。若值南飛雁,時能訪死生。



 此詩至京,盛為當時之所吟誦,天下好事者多書壁而玩之。後歸鄉里,十餘年不得調。仁壽初,徵拜豫章王長史,非其好也。王轉封於齊,即為齊王文學。當時諸王官屬多被夷滅,由是彌不自安,因謝病免。久之,授大理司直,卒於官,時年五十二。有集十卷行
 於世。



 王貞王貞,字孝逸,梁郡東留人也。少聰敏,七歲好學,善《毛詩》、《禮記》、《左氏傳》、《周易》,諸子百家,無不畢覽。善屬文詞,不治產業,每以諷讀為娛。開皇初,汴州刺史樊叔略引為主簿,後舉秀才,授縣尉,非其好也。謝病於家。



 煬帝即位,齊王暕鎮江都,聞其名,以書召之曰:夫山藏美玉,光照廊廡之間,地蘊神劍,氣浮星漢之表。是知毛遂穎脫,義感平原,孫慧文詞,來遷東海。顧循寡薄,有懷髦彥,籍甚清風,為日久矣,未獲披覿,良深佇遲。比高天流火,早應涼
 飆,陵雲仙掌,方承清露,想攝衛攸宜,與時休適。前園後圃,從容丘壑之情,左琴右書,蕭散煙霞之外。茂陵謝病,非無《封禪》之文,彭澤遺榮,先有《歸來》之作。優游儒雅,何樂如之!餘屬當籓屏,宣條揚、越,坐棠聽訟,事絕詠歌,攀桂摛詞,眷言高遁。至於揚旌北渚,飛蓋西園,托乘乏應、劉,置醴闕申、穆,背淮之賓,徒聞其語,趨燕之客,罕值其人。卿道冠鷹揚,聲高鳳舉,儒墨泉海,詞章苑囿,棲遲衡泌,懷寶迷邦,徇茲獨善,良以於邑。今遣行人,具宣往意,側望起予,甚於饑渴,想便輕舉,副此虛心。無信投石之談,空慕鑿壞之逸,書不盡言,更慚詞費。



 及貞至,王以客
 禮待之,朝夕遣問安不。又索文集,貞啟謝曰:屬賀德仁宣教,須少來所有拙文。昔公旦之才藝,能事鬼神,夫子之文章,性與天道,雅志傳於游、夏,餘波鼓於屈、宋,雕龍之跡,具在風騷,而前賢後聖,代相師祖。賞逐時移,出門分路,變清音於正始,體高致於元康,咸言坐握蛇珠,誰許獨為麟角。孝逸生於戰爭之季,長於風塵之世,學無半古,才不逮人。往屬休明,寸陰已昃,雖居可封之屋,每懷貧賤之恥。適鄢郢而迷途,入邯鄲而失步,歸來反覆,心灰遂寒。豈謂橫議過實,虛塵睿覽,枉高車以載鼷,費明珠以彈雀,遂得裹糧三月,重高門之餘地,背淮千里,
 望章臺之後塵。與懸黎而並肆,將駿驥而同阜,終朝擊缶,匪黃鐘之所諧,日暮卻行,何前人之能及!顧想平生,觸途多感,但以積年沈痼,遺忘日久,拙思所存,才成三十三卷。仰而不至,方見學仙之遠,窺而不睹,始知游聖之難。咫尺天人,周章不暇,怖甚真龍之降,慚過白豕之歸,伏紙陳情,形神悚越。



 齊王覽所上集,善之,賜良馬四匹。貞復上《江都賦》,王賜錢十萬貫,馬二匹。未幾,以疾甚還鄉里,終於家。



 虞綽辛大德虞綽,字士裕,會稽餘姚人也。父孝曾,陳始興王諮議。綽
 身長八尺,姿儀甚偉,博學有俊才,尤工草隸。陳左衛將軍傅縡有盛名於世,見綽詞賦,嘆謂人曰:「虞郎之文,無以尚也!」仕陳為太學博士,遷永陽王記室。及陳亡,晉王廣引為學士。大業初,轉為秘書學士,奉詔與秘書郎虞世南、著作佐郎庾自直等撰《長洲玉鏡》等書十餘部。綽所筆削,帝未嘗不稱善,而官竟不遷。初為校書郎,以籓邸左右,加宣惠尉。遷著作佐郎,與虞世南、庾自直、蔡允恭等四人常居禁中,以文翰待詔,恩盼隆洽。從征遼東,帝舍臨海頓,見大鳥,異之,詔綽為銘。其辭曰:維大業八年,歲在壬申,夏四月丙子,皇帝底定遼碣,班師振旅,龍
 駕南轅,鸞旗西邁,行宮次於柳城縣之臨海頓焉。山川明秀,實仙都也。旌門外設,款跨重阜,帳殿周施,降望大壑。息清蹕,下輕輿,警百靈,綏萬福,踐素砂,步碧沚。



 同軒皇之襄野,邁漢宗於河上,想汾射以開襟,望蓬瀛而載佇。窅然齊肅,藐屬殊庭,兼以聖德遐宣,息別風與淮雨,休符潛感,表重潤於夷波。璧日曬光,卿雲舒採,六合開朗,十洲澄鏡。少選之間,倏焉靈感,忽有祥禽,皎同鶴鷺,出自霄漢,翻然雙下。高逾一丈,長乃盈尋,靡霜暉於羽翮,激丹華於觜距。鸞翔鳳跱,鵲起鴻騫,或蹶或啄,載飛載止,徘徊馴擾,咫尺乘輿。不藉揮琴,非因拊石,樂我君
 德,是用來儀。斯固類仙人之騏驥,冠羽族之宗長,西王青鳥,東海赤雁,豈可同年而語哉!竊以銘基華嶽,事乖靈異,紀跡鄒山,義非盡美,猶方冊不泯,遺文可觀。況盛德成功,若斯懿鑠,懷真味道,加此感通,不鐫名山,安用銘異!臣拜稽首,敢勒銘云:來蘇興怨,帝自東征,言復禹績,乃御軒營。六師薄伐,三韓肅清,龔行天罰,赫赫明明。文德上暢,靈武外薄,車徒不擾,苛慝靡作。凱歌載路,成功允鑠,反旆還軒,遵林並壑。停輿海水筮,駐驛巖阯,窅想遐凝,藐屬千里。金臺銀闕,雲浮岳峙,有感斯應,靈禽效祉。飛來清漢,俱集華泉,好音玉響,皓質水鮮。狎仁馴德,
 習習翩翩,絕跡無泯,於萬斯年。



 帝覽而善之,命有司勒於海上。以渡遼功,授建節尉。綽恃才任氣,無所降下。



 著作郎諸葛潁以學業幸於帝,綽每輕侮之,由是有隙。帝嘗問綽于潁,潁曰:「虞綽粗人也。」帝頷之。時禮部尚書楊玄感稱為貴倨,虛襟禮之,與結布衣之友。綽數從之游。其族人虞世南誡之曰:「上性猜忌,而君過厚玄感。若與絕交者,帝知君改悔,可以無咎;不然,終當見禍。」綽不從。尋有告綽以禁內兵書借玄感,帝甚銜之。及玄感敗後,籍沒其家,妓妾並入宮。帝因問之,玄感平常時與何人交往,其妾以虞綽對。帝令大理卿鄭善果窮治其事,綽
 曰:「羈旅薄游,與玄感文酒談款,實無他謀。」帝怒不解,徙綽且末。綽至長安而亡,吏逮之急,於是潛渡江,變姓名,自稱吳卓。游東陽,抵信安令天水辛大德,大德舍之。歲餘,綽與人爭田相訟,因有識綽者而告之,竟為吏所執,坐斬江都,時年五十四。所有詞賦,並行於世。



 大德為令,誅翦群盜,甚得民和。與綽俱為使者所執,其妻泣曰:「每諫君無匿學士,今日之事,豈不哀哉!」大德笑曰:「我本圖脫長者,反為人告之,吾罪也。當死以謝綽。」會有詔,死罪得以擊賊自效。信安吏民詣使者叩頭曰:「辛君人命所懸,辛君若去,亦無信安矣。」使者留之以討賊。帝怒,斬使者,大
 德獲全。



 王胄王胄,字承基,瑯邪臨沂人也。祖筠,梁太子詹事。父祥,陳黃門侍郎。胄少有逸才,仕陳,起家鄱陽王法曹參軍,歷太子舍人、東陽王文學。及陳滅,晉王廣引為學士。仁壽末,從劉方擊林邑,以功授帥都督。大業初,為著作佐郎,以文詞為煬帝所重。帝常自東都還京師,賜天下大酺,因為五言詩,詔胄和之。其詞曰:「河洛稱朝市,崤函實奧區。周營曲阜作,漢建奉春謨。大君苞二代,皇居盛兩都。



 招搖正東指,天駟乃西驅。展軨齊玉軑,式道耀金吾。千
 門駐罕罼,四達儼車徒。



 是節春之暮,神皋華實敷。皇情感時物,睿思屬枌榆。詔問百年老,恩隆五日酺。



 小人荷熔鑄,何由答大爐。」帝覽而善之,因謂侍臣曰:「氣高致遠,歸之於胄;詞清體潤,其在世基;意密理新,推庾自直。過此者,未可以言詩也。」帝所有篇什,多令繼和。與虞綽齊名,同志友善,於時後進之士咸以二人為準的。從征遼東,進授朝散大夫。胄性疏率不倫,自恃才大,鬱鬱於薄宦,每負氣陵傲,忽略時人。



 為諸葛潁所嫉,屢譖之於帝,帝愛其才而不罪。禮部尚書楊玄感虛襟與交,數游其第。及玄感敗,與虞綽俱徙邊。胄遂亡匿,潛還江左,為吏
 所捕,坐誅,時年五十六。所著詞賦,多行於世。



 胄兄,字元恭,博學多通。少有盛名於江左。仕陳,歷太子洗馬、中舍人。



 陳亡,與胄俱為學士。煬帝即位,授秘書郎,卒官。



 庾自直庾自直,潁川人也。父持,陳羽林監。自直少好學,沉靜寡欲。仕陳,歷豫章王府外兵參軍、宣惠記室。陳亡,入關,不得調。晉王廣聞之,引為學士。大業初,授著作佐郎。自直解屬文,於五言詩尤善。性恭慎,不妄交游,特為帝所愛。帝有篇章,必先示自直,令其詆訶。自直所難,帝輒改之,或至於再三,俟其稱善,然後方出。其見親禮如此。後以
 本官知起居舍人事。化及作逆,以之北上,自載露車中,感激發病卒。有文集十卷行於世。



 潘徽潘徽,字伯彥,吳郡人也。性聰敏,少受《禮》於鄭灼,受《毛詩》於施公,受《書》於張沖,講《莊》、《老》於張譏,並通大義。尤精三史。善屬文,能持論。陳尚書令江總引致文儒之士,徽一詣總,總甚敬之。釋褐新蔡王國侍郎,選為客館令。隋遣魏澹聘於陳,陳人使徽接對之。澹將返命,為啟於陳主曰:「敬奉弘慈,曲垂餞送。」徽以為「伏奉」為重,「敬奉」為輕,卻其啟而不奏。澹立議曰:「《曲禮》注曰:『禮主於敬。』《詩》曰:『維桑
 與梓,必恭敬止。』《孝經》曰:『宗廟致高。』又云:『不敬其親,謂之悖禮。』孔子敬天之怒,成湯聖敬日躋。宗廟極重,上天極高,父極尊,君極貴,四者咸同一敬,五經未有異文,不知以敬為輕,竟何所據?」徽難之曰:「向所論敬字,本不全以為輕,但施用處殊,義成通別。《禮》主於敬,此是通言,猶如男子『冠而字之』,注云『成人敬其名也』。《春秋》有冀缺,夫妻亦云『相敬』。既於子則有敬名之義,在夫亦有敬妻之說,此可復並謂極重乎?至若『敬謝諸公』,固非尊地,『公子敬愛』,止施賓友,『敬問』『敬報』,彌見雷同,『敬聽』『敬酬』,何關貴隔!當知敬之為義,雖是不輕,但敬之於語,則有時混漫。今云『
 敬奉』,所以成疑。聊舉一隅,未為深據。」澹不能對,遂從而改焉。及陳滅,為州博士,秦孝王俊聞其名,召為學士。



 嘗從俊朝京師,在途,令徽於馬上為賦,行一驛而成,名曰《述恩賦》。俊覽而善之。復令為《萬字文》,並遣撰集字書,名為《韻篡》。徽為序曰:文字之來尚矣。初則羲皇出震,觀象緯以法天,次則史頡佐軒,察蹄跡而取地。



 於是八卦爰始,爻文斯作,繩用既息,墳籍生焉。至如龍策授河,龜威出洛,綠綈白檢,述勛、華之運,金繩玉字,表殷、夏之符,銜甲示於姬壇,吐卷徵於孔室,莫不理包遠邇,跡會幽明,仰協神功,俯照人事。其制作也如彼,其祥瑞也如此,故
 能宣流萬代,正名百物,為生民之耳目,作後王之模範,頌美形容,垂芬篆素。



 暨大隋之受命也,追從三五,並曜參辰,外振武功,內修文德。飛英聲而勒嵩岱,彰大定而銘鐘鼎。春干秋羽,盛禮樂於膠庠,省俗觀風,採歌謠於唐衛。我秦王殿下,降靈霄極,稟秀天機,質潤珪璋,文兼黼黻。楚詩早習,頗屬懷於言志,沛《易》先通,每留神於索隱。尊儒好古,三雍之對已遒,博物多能,百家之工彌洽。



 遨游必名教,漁獵唯圖史。加以降情引汲,擇善芻微,築館招賢,攀枝佇異。剖連城於井里,賁束帛於丘園,薄技無遺,片言便賞。所以人加脂粉,物競琢磨,俱報稻粱,各
 施鳴吠。於時歲次鶉火,月躔夷則,驂駕務隙,靈光意靜。前臨竹沼,卻倚桂巖,泉石瑩仁智之心,煙霞發文彩之致,賓僚霧集,教義風靡。乃討論群藝,商略眾書,以為小學之家,尤多舛雜,雖復周禮漢律,務在貫通,而巧說邪辭,遞生同異。且文訛篆隸,音謬楚夏,《三蒼》、《急就》之流,微存章句,《說文》、《字林》之屬,唯別體形。至於尋聲推韻,良為疑混,酌古會今,未臻功要。末有李登《聲類》、呂靜《韻集》,始判清濁,才分宮羽,而全無引據,過傷淺局,詩賦所須,卒難為用。遂躬紆睿旨,摽摘是非,撮舉宏綱,裁斷篇部,總會舊轍,創立新意,聲別相從,即隨注釋。詳之詁訓,證以
 經史,備包《騷》《雅》,博牽子集,汗簡云畢,題為《韻篡》,凡三十卷,勒成一家。方可藏彼名山,副諸石室,見群玉之為淺,鄙懸金之不定。爰命末學,制其都序。徽業術已寡,思理彌殫,心若死灰,文慚生氣。徒以犬馬識養,飛走懷仁,敢執顛沛之辭,遂操狂簡之筆。而齊魯富經學,楚鄭多良士,西河之彥,幸不誚於索居,東里之才,請能加於潤色。



 未幾,俊薨,晉王廣復引為揚州博士,令與諸儒撰《江都集禮》一部。復令徽作序曰:禮之為用至矣。大與天地同節,明與日月齊照,源開三本,體合四端。巢居穴處之前,即萌其理,龜文鳥跡以後,稍顯其事。雖情存簡易,意非
 玉帛,而夏造殷因,可得知也。至如秩宗三禮之職,司徒五禮之官,邦國以和,人神惟敬,道德仁義,非此莫成,進退俯仰,去茲安適!若璽印塗,猶防止水,豈直譬彼耕耨,均斯粉澤而已哉!自世屬坑焚,時移漢、魏,叔孫通之碩解,高堂隆之博識,專門者霧集,制作者風馳,節文頗備,枝條互起。皇帝負扆垂旒,辨方正位,纂勛華之歷象,綴文武之憲章。車書之所會通,觸境斯應,雲雨之所沾潤,無思不韙。東探石簣之符,西蠹羽陵之策,鳴鑾太室,偃伯靈臺,樂備五常,禮兼八代。上柱國、太尉、揚州總管、晉王握珪璋之寶,履神明之德,隆化贊傑,藏用顯仁。地居
 周邵,業冠河楚,允文允武,多才多藝。戎衣而籠關塞,朝服而掃江湖,收杞梓之才,闢康莊之館。加以佃漁六學,網羅百氏,繼稷下之絕軌,弘泗上之淪風,賾無隱而不探,事有難而必綜。至於採標綠錯,華垂丹篆,刑名長短,儒墨是非,書圃翰林之域,理窟談叢之內,謁者所求之餘,侍醫所校之逸,莫不澄涇辨渭,拾珠棄蚌。以為質文遞改,損益不同,《明堂》、《曲臺》之記,南宮、東觀之說,鄭、王、徐、賀之答,崔、譙、何、瘐之論,簡牒雖盈,菁華蓋鮮。乃以宣條暇日,聽訟餘晨,娛情窺寶之鄉,凝相觀濤之岸,總括油素,躬披緗縹,芟蕪刈楚,振領提綱,去其繁雜,撮其指要,
 勒成一家,名曰《江都集禮》。凡十二帙,一百二十卷,取方月數,用比星周,軍國之義存焉,人倫之紀備矣。昔者龜、蒙令後,睢、渙名籓,誠復出警入蹕,擬乘輿之制度,建韣載旂,用天子之禮樂。求諸述作,未聞茲典。方可韜之P水,副彼名山,見刻石之非工,嗤懸金之已陋。是知《沛王通論》,不獨擅於前修,《寧朔新書》,更追慚於往冊。徽幸樓仁岳,忝游聖海,謬承恩獎,敢敘該博之致雲。



 煬帝嗣位,詔徽與著作佐郎陸從典、太常博士褚亮、歐陽詢等助越公楊素撰《魏書》,會素薨而止。授京兆郡博士。楊玄感兄弟甚重之,數相來往。及玄感敗,凡交關多罹其患。徽
 以玄感故人,為帝所不悅,有司希旨,出徽為西海郡威定縣主簿。意甚不平,行至隴西,發病卒。



 杜正玄弟正藏杜正玄,字慎徽,其先本京兆人,八世祖曼,為石趙從事中郎,因家於鄴。自曼至正玄,世以文學相授。正玄尤聰敏,博涉多通。兄弟數人,俱未弱冠,並以文章才辨籍甚三河之間。開皇末,舉秀才,尚書試方略,正玄應對如響,下筆成章。



 僕射楊素負才傲物,正玄抗辭酬對,無所屈撓,素甚不悅。久之,會林邑獻白鸚鵡,素促召正玄,使者相望。及至,即令作賦。正玄倉卒之際,援筆立成。素見文
 不加點,始異之。因令更擬諸雜文筆十餘條,又皆立成,而辭理華贍,素乃嘆曰:「此真秀才,吾不及也!」授晉王行參軍,轉豫章王記室,卒官。弟正藏。



 正藏字為善,尤好學,善屬文。弱冠舉秀才,授純州行參軍,歷下邑正。大業中,學業該通,應詔舉秀才,兄弟三人俱以文章一時詣闕,論者榮之。著碑誄銘頌詩賦百餘篇。又著《文章體式》,大為後進所寶,時人號為文軌,乃至海外高麗、百濟,亦共傳習,稱為《杜家新書》。



 常得志京兆常得志,博學善屬文,官至秦王記室。及王薨,過故
 宮,為五言詩,辭理悲壯,甚為時人所重。復為《兄弟論》,義理可稱。



 尹式河間尹式,博學解屬文,少有令問。仁壽中,官至漢王記室,王甚重之,及漢王敗,式自殺。其族人正卿、彥卿俱有俊才,名顯於世劉善經河間劉善經,博物洽聞,尤善詞筆。歷仕著作佐郎、太子舍人。著《酬德傳》三十卷,《諸劉譜》三十卷,《四聲指歸》一卷,行於世。



 祖君彥範陽祖君彥,齊尚書僕射孝徵之子也。容貌短小,言辭訥澀,有才學。大業末,官至東平郡書佐。郡陷於翟讓,因為李密所得。密甚禮之,署為記室,軍書羽檄,皆成於其手。及密敗,為王世充所殺。



 孔德紹會稽孔德紹,有清才,官至景城縣丞。竇建德稱王,署為中書令,專典書檄。



 及建德敗,伏誅。



 劉斌南陽劉斌,頗有詞藻,官至信都郡司功書佐。竇建德署
 為中書舍人。建德敗,復為劉闥中書侍郎,與劉闥亡歸突厥,不知所終。



 史臣曰:魏文有言「古今文人,類不護細行,鮮能以名節自立」,信矣!王胄、虞綽之輩,崔儦、孝逸之倫,或矜氣負才,遺落世事,或學優命薄,調高位下,心鬱抑而孤憤,志盤桓而不定,嘯傲當世,脫略公卿。是知跅弛見遺,嫉邪忤物,不獨漢陽趙壹、平原禰衡而已。故多離咎悔,鮮克有終。然其學涉稽古,文詞辨麗,並鄧林之一枝,昆山之片玉矣。有隋總一寰宇,得人為盛,秀異之貢,不過十數。



 正玄昆季三人預焉,華萼相耀,亦為難兄弟矣。



\end{pinyinscope}