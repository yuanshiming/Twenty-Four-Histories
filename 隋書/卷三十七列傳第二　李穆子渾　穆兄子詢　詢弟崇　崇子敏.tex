\article{卷三十七列傳第二 李穆子渾 穆兄子詢 詢弟崇 崇子敏}

\begin{pinyinscope}

 李穆,字顯慶,自雲隴西成紀人,漢騎都尉陵之後也。陵沒匈奴,子孫代居北狄,其後隨魏南遷,復歸汧、隴。祖斌,以都督鎮高平,因家焉。父文保,早卒,及穆貴,贈司空。穆風神警俊,倜儻有奇節。周太祖首建義旗,穆便委質,釋褐統軍。永熙末,奉迎魏武帝,授都督,封永平縣子,邑三百戶。又領鄉兵,累以軍功進爵為伯。從太祖擊齊師於
 芒山,太祖臨陣墮馬,穆突圍而進,以馬策擊太祖而詈之,授以從騎,潰圍俱出。賊見其輕侮,謂太祖非貴人,遂緩之,以故得免。既而與穆相對泣,顧謂左右曰:「成我事者,其此人乎!」即令撫慰關中,所至克定,擢授武衛將軍、儀同三司,進封安武郡公,增邑一千七百戶,賜以鐵券,恕其十死。



 尋加開府,領侍中。初,芒山之敗,穆以驄馬授太祖。太祖於是廄內驄馬盡以賜之,封穆姊妹皆為郡縣君,宗從舅氏,頒賜各有差。轉太僕。從於謹破江陵,增邑千戶,進位大將軍。擊曲沔蠻,破之,授原州刺史,拜嫡子惇為儀同三司。穆以二兄賢、遠並為佐命功臣,而子
 弟布列清顯,穆深懼盈滿,辭不受拜。太祖不許。俄遷雍州刺史,兼小塚宰。周元年,增邑三千戶,通前三千七百戶。又別封一子為升遷伯。



 穆讓兄子孝軌,許之。



 宇文護執政,穆兄遠及其子植俱被誅,穆當從坐。先是,穆知植非保家之主,每勸遠除之,遠不能用。及遠臨刑,泣謂穆曰:「顯慶,吾不用汝言,以至於此,將復奈何!」穆以此獲免,除名為民,及其子弟亦免官。植弟淅州刺史基,當坐戮,穆請以二子代基之命,護義而兩釋焉。未幾,拜開府儀同三司、直州刺史,復爵安武郡公。武成中,子弟免官爵者悉復之。尋除少保,進位大將軍。歲餘,拜小司徒,進位
 柱國,轉大司空。奉詔築通洛城。天和中,進爵申國公,持節綏集東境,築武申、旦郛、慈澗、崇德、安民、交城、鹿盧等諸鎮。建德初,拜太保。歲餘,出為原州總管。數年,進位上柱國,轉並州總管。大象初,加邑至九千戶,拜大左輔,總管如故。



 高祖作相,尉迥之作亂也,遣使招穆。穆鎖其使,上其書。穆子士榮,以穆所居天下精兵處,陰勸穆反。穆深拒之,乃奉十三環金帶於高祖,蓋天子之服也。穆尋以天命有在,密表勸進。高祖既受禪,下詔曰:「公既舊德,且又父黨,敬惠來旨,義無有違。便以今月十三日恭膺天命。」俄而穆來朝,高祖降坐禮之,拜太師,贊拜不名,真
 食成安縣三千戶。於是穆子孫雖在襁褓,悉拜儀同,其一門執象笏者百餘人。穆之貴盛,當時無比。穆上表乞骸骨,詔曰:「朕初臨宇內,方藉嘉猷,養老乞言,實懷虛想。七十致仕,本為常人。至若呂尚以期頤佐周,張蒼以華皓相漢,高才命世,不拘恆禮,遲得此心,留情規訓。公年既耆舊,筋力難煩,今勒所司,敬蠲朝集。如有大事,須共謀謨,別遣侍臣,就第詢訪。」



 時太史奏云,當有移都之事。上以初受命,甚難之。穆上表曰:帝王所居,隨時興廢,天道人事,理有存焉。始自三皇,暨夫兩漢,有一世而屢徙,無革命而不遷。曹、馬同洛水之陽,魏、周共長安之內,此
 之四代,蓋聞之矣。曹則三家鼎立,馬則四海尋分,有魏及周,甫得平定,事乃不暇,非曰師古。



 往者周運將窮,禍生華裔,廟堂冠帶,屢睹奸回,士有苞藏,人稀柱石。四海萬國,皆縱豺狼,不叛不侵,百城罕一。伏惟陛下膺期誕聖,秉籙受圖,始晦君人之德,俯從將相之重。內翦群兇,崇朝大定,外誅巨猾,不日肅清。變大亂之民,成太平之俗,百靈符命,兆庶謳歌。幽顯樂推,日月填積,方屈箕、潁之志,始順內外之請。自受命神宗,弘道設教,陶冶與陰陽合德,覆育共天地齊旨。萬物開闢之初,八表光華之旦,視聽以革,風俗且移。至若帝室天居,未議經創,非所
 謂發明大造,光贊惟新。自漢已來,為喪亂之地,爰從近代,累葉所都。未嘗謀龜問筮,瞻星定鼎,何以副聖主之規,表大隨之德?竊以神州之廣,福地之多,將為皇家興廟建寢,上玄之意,當別有之。伏願遠順天人,取決卜筮,時改都邑,光宅區夏。任子來之民,垂無窮之業,應神宮於辰極,順和氣於天壤,理康物阜,永隆長世。臣日薄桑榆,位高軒冕,經邦論道,自顧缺然。丹赤所懷,無容噤默。



 上素嫌臺城制度迮小,又宮內多鬼妖,蘇威嘗勸遷,上不納。遇太史奏狀,意乃惑之。至是,省穆表,上曰:「天道聰明,已有徵應,太師民望,復抗此請,則可矣。」遂從之。歲餘,
 下詔曰:「禮制凡品,不拘上智,法備小人,不防君子。



 太師、上柱國、申國公,器宇弘深,風猷遐曠,社稷佐命,公為稱首,位極帥臣,才為人傑,萬頃不測,百煉彌精。乃無伯玉之非,豈有顏回之貳,故以自居寥廓,弗關憲網。然王者作教,惟旌善人,去法弘道,示崇年德。自今已後,雖有愆罪,但非謀逆,縱有百死,終不推問。」



 開皇六年薨於第,年七十七。遺令曰:「吾荷國恩,年宦已極,啟足歸泉,無所復恨。竟不得陪玉鑾於岱宗,預金泥於梁甫,眷眷光景,其在斯乎!」詔遣黃門侍郎監護喪事,賵馬四匹,粟麥二千斛,布絹一千匹。贈使持節、冀定趙相瀛毛魏衛洛懷十
 州諸軍事、冀州刺史。謚曰明。賜以石槨、前後部羽葆鼓吹、轀輬車。百僚送之郭外。詔遣太常卿牛弘齎哀冊,祭以太牢。孫筠嗣。



 筠父惇,字士獻,穆長子也。仕周,官至安樂郡公、鳳州刺史,先穆卒。筠幼以穆功,拜儀同。開皇八年,以嫡孫襲爵。仁壽初,叔父渾忿其吝嗇,陰遣兄子善衡賊殺之。求盜不獲,高祖大怒,盡禁其親族。初,筠與從父弟瞿曇有隙,時渾有力,遂證瞿曇殺之。瞿曇竟坐斬,而善衡獲免。四年,議立嗣。邳公蘇威奏筠不義,骨血相殺,請絕其封。上不許。惇弟怡,官至儀同,早卒,贈渭州刺史。



 怡弟雅,少有識量。周保定中,屢以軍功封西安縣男,
 拜大都督。天和中,從元定征江西,時諸軍失利,遂沒於陳。後得歸國,拜開府儀同三司,領左右軍。其年,從太子西征吐谷渾,雅率步騎二千,督軍糧於洮河,為賊所躡,相持數日。雅患之,遂與偽和,虜備稍解,縱奇兵擊破之。賜奴婢百口,封一子為侯。後拜齊州刺史,俄征還京。數載,授瀛州刺史。高祖作相,鎮靈州以備胡。還授大將軍,遷荊州總管,加邑八百戶。開皇初,進爵為公。



 雅弟恆,官至鹽州刺史,封陽曲侯。恆弟榮,官至合州刺史、長城縣公。榮弟直,官至車騎將軍、歸政縣侯。直弟雄,官至柱國、密國公、驃騎將軍。雄弟渾,最知名。



 渾字金才,穆第十子也。姿貌瑰偉,美須髯。起家周左侍上士。尉迥反於鄴,時穆在並州,高祖慮其為迥所誘,遣渾乘驛往布腹心。穆遽令渾入京,奉熨斗於高祖,曰:「願執威柄以熨安天下也。」高祖大悅。又遣渾詣韋孝寬所而述穆意焉。



 適遇平鄴,以功授上儀同三司,封安武郡公。開皇初,進授象城府驃騎將軍。晉王廣出籓,渾以驃騎領親信,從往揚州。仁壽元年,從左僕射楊素為行軍總管,出夏州北三百里,破突厥阿勿俟斤於納遠川,斬首五百級。進位大將軍,拜左武衛將軍,領太子宗衛率。



 初,穆孫筠卒,高祖議立嗣,渾規欲紹之,謂其妻兄太子
 左衛率宇文述曰:「若得襲封,當以國賦之半每歲奉公。」述利之,因入白皇太子曰:「立嗣以長,不則以賢。今申明公嗣絕,遍觀其子孫,皆無賴,不足以當榮寵。唯金才有勛於國,謂非此人無可以襲封者。」太子許之,竟奏高祖,封渾為申國公,以奉穆嗣。大業初,轉右驍衛將軍。六年,有詔追改穆封為郕國公,渾仍襲焉。累加光祿大夫。九年,遷右驍衛大將軍。



 渾既紹父業,日增豪侈,後房曳羅綺者以百數。二歲之後,不以俸物與述。述大恚之,因醉,乃謂其友人於象賢曰:「我竟為金才所賣,死且不忘!」渾亦知其言,由是結隙。後帝討遼東,有方士安伽陀,自言
 曉圖讖,謂帝曰:「當有李氏應為天子。」勸盡誅海內凡姓李者。述知之,因誣構渾於帝曰:「伽陀之言信有徵矣。



 臣與金才夙親,聞其情趣大異。常日數共李敏、善衡等,日夜屏語,或終夕不寐。



 渾大臣也,家代隆盛,身捉禁兵,不宜如此。願陛下察之。」帝曰:「公言是矣,可覓其事。」述乃遣武賁郎將裴仁基表告渾反,即日發宿衛千餘人付述,掩渾等家,遣左丞元文都、御史大夫裴蘊雜治之。案問數曰,不得其反狀,以實奏聞。帝不納,更遣述窮治之。述入獄中,召出敏妻宇文氏謂之曰:「夫人,帝甥也,何患無賢夫!



 李敏、金才,名當妖讖,國家殺之,無可救也。夫人當
 自求全,若相用語,身當不坐。」敏妻曰:「不知所出,惟尊長教之。」述曰:「可言李家謀反,金才嘗告敏云:『汝應圖籙,當為天子。今主上好兵,勞擾百姓,此亦天亡隋時也,正當共汝取之。若復渡遼,吾與汝必為大將,每軍二萬餘兵,固以五萬人矣。又發諸房子侄,內外親婭,並募從征。吾家子弟,決為主帥,分領兵馬,散在諸軍,伺候間隙,首尾相應。吾與汝前發,襲取御營,子弟響起,各殺軍將。一日之間,天下足定矣。」



 述口自傳授,令敏妻寫表,封雲上密。述持入奏之,曰:「已得金才反狀,並有敏妻密表。」帝覽之泣曰:「吾宗社幾傾,賴親家公而獲全耳。」於是誅渾、敏等
 宗族三十二人,自餘無少長,皆徙嶺外。



 渾從父兄威,開皇初,以平蠻功,官至上柱國、黎國公。



 詢字孝詢。父賢,周大將軍。詢沉深有大略,頗涉書記。仕周納言上士,俄轉內史上士,兼掌吏部,以幹濟聞。建德三年,武帝幸雲陽宮,拜司衛上士,委以留府事。周衛王直作亂,焚肅章門,詢於內益火,故賊不得入。帝聞而善之,拜儀同三司,遷長安令。累遷英果中大夫。屢以軍功,加位大將軍,賜爵平高郡公。



 高祖為丞相,尉迥作亂,遣韋孝寬擊之,以詢為元帥長史,委以心膂。軍至永橋,諸將不一,詢密啟高祖,請重臣監護。高祖遂令高熲監軍,
 與熲同心協力,唯詢而已。及平尉迥,進位上柱國,改封隴西郡公,賜帛千匹,加以口馬。



 開皇元年,引杜陽水灌三趾原,詢督其役,民賴其利。尋檢校襄州總管事。歲餘,拜顯州總管。數年,以疾徵還京師,中使顧問不絕。卒於家,時年四十九,上悼惜者久之。謚曰襄。有子元方嗣。



 崇字永隆,英果有籌算,膽力過人。周元年,以父賢勛,封乃樂縣侯。時年尚小,拜爵之日,親族相賀,崇獨泣下。賢怪而問之,對曰:「無勛於國,而幼少封侯,當報主恩,不得終於孝養,是以悲耳。」賢由此大奇之。起家州主簿,非其所好,辭不就官,求為將兵都督。隨宇文護伐齊,以功最,
 擢授儀同三司。尋除小司金大夫,治軍器監。建德初,遷少侍伯大夫,轉少承御大夫,攝太子宮正。周武帝平齊,引參謀議,以幼加授開府,封襄陽縣公,邑一千戶。尋改封廣宗縣公,轉太府中大夫,歷工部中大夫,遷右司馭。高祖為丞相,遷左司武上大夫,加授上開府儀同大將軍。尋為懷州刺史,進爵郡公,加邑至二千戶。尉迥反,遣使招之。崇初欲相應,後知叔父穆以並州附高祖,慨然太息曰:「合家富貴者數十人,值國有難,竟不能扶傾繼絕,復何面目處天地間乎!」韋孝寬亦疑之,與俱臥起。其兄詢時為元帥長史,每諷諭之,崇由是亦歸心焉。及破
 尉惇,拜大將軍。既平尉迥,授徐州總管,尋進位上柱國。



 開皇三年,除幽州總管。突闕犯塞,崇輒破之。奚、霫、契丹等懾其威略,爭來內附。其後突厥大為寇掠,崇率步騎三千拒之,轉戰十餘日,師人多死,遂保於砂城。突厥圍之。城本荒廢,不可守御,曉夕力戰,又無所食,每夜出掠賊營,復得六畜,以繼軍糧。突厥畏之,厚為其備,每夜中結陣以待之。崇軍苦饑,出輒遇敵,死亡略盡,遲明奔還城者,尚且百許人,然多傷重,不堪更戰。突厥意欲降之,遣使謂崇曰:「若來降者,封為特勤。」崇知必不免,令其士卒曰:「崇喪師徒,罪當死,今日效命以謝國家。待看吾死,
 且可降賊,方便散走,努力還鄉。若見至尊,道崇此意。」乃挺刃突賊,復殺二人。賊亂射之,卒於陣,年四十八。贈豫鄎申永澮亳六州諸軍事、豫州刺史,謚曰壯。子敏嗣。



 敏字樹生。高祖以其父死王事,養宮中者久之。及長,襲爵廣宗公,起家左千牛。美姿儀,善騎射,歌舞管弦,無不通解。開皇初,周宣帝後封樂平公主,有女娥英,妙擇婚對,敕貴公子弟集弘聖宮者,日以百數。公主親在帷中,並令自序,並試技藝。選不中者,輒引出之。至敏而合意,竟為姻媾。敏假一品羽儀,禮如尚帝之女。後將侍宴,公主謂敏曰:「我以四海與至尊,唯一女夫,當為汝求柱
 國。



 若授餘官,汝慎無謝。」及進見上,上親御琵琶,遣敏歌舞。既而大悅,謂公主曰:「李敏何官?」對曰:「一白丁耳。」上因謂敏曰:「今授汝儀同。」敏不答。上曰:「不滿爾意邪?今授汝開府。」敏又不謝。上曰:「公主有大功於我,我何得向其女婿而惜官乎!今授卿柱國。」敏乃拜而蹈舞。遂於坐發詔授柱國,以本官宿衛。後避諱,改封經城縣公,邑一千戶。歷蒲、豳、金、華、敷州刺史,多不蒞職,常留京師,往來宮內,侍從游宴,賞賜超於功臣。後幸仁壽宮,以為岐州刺史。



 大業初,轉衛尉卿。樂平公主之將薨也,遺言於煬帝曰:「妾無子息,唯有一女。不自憂死,但深憐之。今湯沐邑,乞
 回與敏。」帝從之。竟食五千戶,攝屯衛將軍。楊玄感反後城大興,敏之策也。轉將作監,從征高麗,領新城道軍將,加光祿大夫。十年,帝復征遼東,遣敏於黎陽督運。時或言敏一名洪兒,帝疑「洪」字當讖,嘗面告之,冀其引決。敏由是大懼,數與金才、善衡等屏人私語。宇文述知而奏之,竟與渾同誅,年三十九。其妻宇文氏,後數月亦賜鴆而終。



 梁睿梁睿,字恃德,安定烏氏人也。父御,西魏太尉。睿少沉敏,有行檢。周太祖時,以功臣子養宮中者數年。其後命諸
 子與睿游處,同師共業,情契甚歡。七歲,襲爵廣平郡公,累加儀同三司,邑五百戶。尋為本州大中正。魏恭帝時加開府,改封為五龍郡公,拜渭州刺史。周閔帝受禪,徵為御伯。未幾,出為中州刺史,鎮新安,以備齊。齊人來寇,睿輒挫之,帝甚嘉嘆。拜大將軍,進爵蔣國公,入為司會。



 後從齊王憲拒齊將斛律明月於洛陽,每戰有功,遷小塚宰。武帝時,歷敷州刺史、涼安二州總管,俱有惠政,進位柱國。



 高祖總百揆,代王謙為益州總管。行至漢川而謙反,遣兵攻始州,睿不得進。



 高祖命睿為行軍元帥,率行軍總管於義、張威、達奚長儒、梁升、石孝義步騎二十
 萬討之。時謙遣開府李三王等守通谷,睿使張威擊破之,擒數千人,進至龍門。謙將趙儼、秦會擁眾十萬,據嶮為營,周亙三十里。睿令將士銜枚出自間道,四面奮擊,力戰破之。蜀人大駭,睿鼓行而進。謙將敬豪守劍閣,梁巖拒平林,並懼而來降。謙又令高阿那肱、達奚惎等以盛兵攻利州。聞睿將至,惎分兵據開遠。睿顧謂將士曰:「此虜據要,欲遏吾兵勢,吾當出其不意,破之必矣。」遣上開府拓拔宗趣劍閣,大將軍宇文夐詣巴西,大將軍趙達水軍入嘉陵。睿遣張威、王倫、賀若震、於義、韓相貴、阿那惠等分道攻惎,自午及申,破之。惎奔歸於謙。睿進逼
 成都,謙令達奚惎、乙弗虔城守,親率精兵五萬,背城結陣。睿擊之,謙不利,將入城,惎、虔以城降,拒謙不內。謙將麾下三十騎遁走,新都令王寶執之。睿斬謙於市,劍南悉平。進位上柱國,總管如故。賜物五千段,奴婢一千口,金二千兩,銀三千兩,食邑千戶。



 睿時威振西川,夷、獠歸附,唯南寧酋帥爨震恃遠不賓。睿上疏曰:「竊以遠撫長駕,王者令圖,易俗移風,有國恆典。南寧州,漢世䍧柯之地,近代已來,分置興古、雲南、建寧、硃提四郡。戶口殷眾,金寶富饒,二河有駿馬、明珠,益寧出鹽井、犀角。晉太始七年,以益州曠遠,分置寧州。至偽梁南寧州刺史徐文
 盛,被湘東征赴荊州,屬東夏尚阻,未遑遠略。土民爨瓚遂竊據一方,國家遙授刺史。



 其子震,相承至今。而震臣禮多虧,貢賦不入,每年奉獻,不過數十匹馬。其處去益,路止一千,硃提北境,即興戎州接界。如聞彼人苦其苛政,思被皇風。伏惟大丞相匡贊聖朝,寧濟區宇,絕後光前,方垂萬代,闢土服遠,今正其時。幸因平蜀士眾,不煩重興師旅,押獠既訖,即請略定南寧。自盧、戎已來,軍糧須給,過此即於蠻夷徵稅,以供兵馬。其寧州、硃提、雲南、西爨,並置總管州鎮。計彼熟蠻租調,足供城防食儲。一則以肅蠻夷,二則裨益軍國。今謹件南寧州郡縣及事
 意如別。有大都督杜神敬,昔曾使彼,具所諳練,今並送往。」書未答,又請曰:「竊以柔遠能邇,著自前經,拓土開疆,王者所務。南寧州,漢代䍧柯之郡,其地沃壤,多是漢人,既饒寶物,又出名馬。今若往取,仍置州郡,一則遠振威名,二則有益軍國。其處與交、廣相接,路乃非遙。漢代開此,本為討越之計。伐陳之日,復是一機,以此商量,決謂須取。」高祖深納之,然以天下初定,恐民心不安,故未之許。後竟遣史萬歲討平之,並因睿之策也。



 睿威惠兼著,民夷悅服,聲望逾重,高祖陰憚之。薛道衡從軍在蜀,因入接宴,說睿曰:「天下之望,已歸於隋。」密令勸進,高祖大
 悅。及受禪,顧待彌隆。睿復上平陳之策,上善之,下詔曰:「公英風震動,妙算縱橫,清蕩江南,宛然可見。



 循環三復,但以欣然。公既上才,若管戎律,一舉大定,固在不疑。但朕初臨天下,政道未洽,恐先窮武事,未為盡善。昔公孫述、隗囂,漢之賊也,光武與其通和,稱為皇帝。尉佗之於高祖,初猶不臣。孫晧之答晉文,書尚雲白。或尋款服,或即滅亡。王者體大,義存遵養,雖陳國來朝,未盡籓節,如公大略,誠須責罪,尚欲且緩其誅,宜知此意。淮海未滅,必興師旅,若命永襲,終當相屈。想以身許國,無足致辭也。」睿乃止焉。



 睿時見突厥方強,恐為邊患,復陳鎮守之
 策十餘事,上書奏之曰:「竊以戎狄作患,其來久矣。防遏之道,自古為難。所以周無上算,漢收下策,以其倏來忽往,雲屯霧散,強則騁其犯塞,弱又不可盡除故也。今皇祚肇興,宇內寧一,唯有突厥種類,尚為邊梗。此臣所以廢寢與食,寤寐思之。昔匈奴未平,去病辭宅,先零尚在,充國自劾。臣才非古烈,而志追昔士。謹件安置北邊城鎮烽候,及人馬糧貯戰守事意如別,謹並圖上呈,伏惟裁覽。」上嘉嘆久之,答以厚意。



 睿時自以周代舊臣,久居重鎮,內不自安,屢請入朝,於是徵還京師。及引見,上為之興,命睿上殿,握手極歡。睿退謂所親曰:「功遂身退,今
 其時也。」遂謝病於家,闔門自守,不交當代。上賜以版輿,每有朝覲,必令三衛輿上殿。睿初平王謙之始,自以威名太盛,恐為時所忌,遂大受金賄以自穢。由是勛簿多不以實,詣朝堂稱屈者,前後百數。上令有司案驗其事,主者多獲罪。睿惶懼,上表陳謝,請歸大理。上慰諭遣之。



 十五年,從上至洛陽而卒,時年六十五。謚曰襄。子洋嗣,官歷嵩、徐二州刺史、武賁郎將。大業六年,詔追改封睿為戴公,命以洋襲焉。



 史臣曰:李穆、梁睿,皆周室功臣,高祖王業初基,俱受腹心之寄。故穆首登師傅,睿終膺殊寵,觀其見機而動,抑
 亦民之先覺。然方魏朝之貞烈,有愧王陵,比晉室之忠臣,終慚徐廣。穆之子孫,特為隆盛,硃輪華轂,凡數十人,見忌當時,禍難遄及,得之非道,可不戒歟!



\end{pinyinscope}