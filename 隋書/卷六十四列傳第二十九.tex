\article{卷六十四列傳第二十九}

\begin{pinyinscope}

 李圓通,京兆涇陽人也。父景,以軍士隸武元皇帝,因與家僮黑女私,生圓通。



 景不之認,由是孤賤,給使高祖家。及為隋國公,擢授參軍事。初,高祖少時,每宴賓客,恆令圓通監廚。圓通性嚴整,左右婢僕咸所敬憚。唯世子乳母恃寵輕之,賓客未供,每有干請,圓通不許,或輒持去。圓通大怒,叱廚人撾之數十,叫呼之聲徹於閤內,僚吏
 左右代其失色。賓去之後,高祖具知之,召圓通,命坐賜食,從此獨善之,以為堪當大任。高祖作相,賜封懷昌男。久之,授帥都督,進爵新安子,委以心膂。圓通多力勁捷,長於武用。周氏諸王素憚高祖,每伺高祖之隙,圖為不利,賴圓通保護,獲免者數矣。高祖深感之,由是參預政事。授相國外兵曹,仍領左親信。尋授上儀同。高祖受禪,拜內史侍郎,領左衛長史,進爵為伯。歷左右庶子、給事黃門侍郎、尚書左丞,攝刑部尚書,深被任信。後以左丞領左翊衛驃騎將軍。伐陳之役,圓通以行軍總管從楊素出信州道,以功進位大將軍,進封萬安縣侯,拜揚州
 總管長史。尋轉並州總管長史。秦孝王仁柔自善,少斷決,府中事多決於圓通。入為司農卿、治粟內史,遷刑部尚書。後數歲,復為並州長史。孝王以奢侈得罪,圓通亦坐免官。尋檢校刑部尚書事。仁壽中,以勛舊進爵郡公。煬帝嗣位,拜兵部尚書。帝幸揚州,以圓通留守京師。判宇文述田以還民,述訴其受賂。帝怒而征之,見帝於洛陽,坐是免官。圓通憂懼發疾而卒。贈柱國,封爵悉如故。子孝常,大業末,為華陰令。



 陳茂,河東猗氏人也。家世寒微,質直恭謹,為州里所敬。
 高祖為隋國公,引為僚佐,遇待與圓通等。每令典家事,未嘗不稱旨,高祖善之。後從高祖與齊師戰於晉州,賊甚盛,高祖將挑戰,茂固止不得,因捉馬鞚。高祖忿之,拔刀斫其額,流血被面,詞氣不撓。高祖感而謝之,厚加禮敬。其後官至上士。高祖為丞相,委以心膂。及受禪,拜給事黃門侍郎,封魏城縣男,每典機密。在官十餘年,轉益州總管司馬,遷太府卿,進爵為伯。後數載,卒官。子政嗣。



 政字弘道,倜儻有文武大略,善鐘律,便弓馬。少養宮中,年十七,為太子千牛備身。時京師大俠劉居士重政才氣,數從之游。圓通子孝常與政相善,並與居士交結。及
 居士下獄誅,政及孝常當從坐,上以功臣子,撻之二百而赦之。由是不得調。煬帝時,授協律郎,遷通事謁者,兵曹承務郎。帝美其才,甚重之。宇文化及之亂也,以為太常卿。後歸大唐,卒於梁州總管。



 張定和,字處謐,京兆萬年人也。少貧賤,有志節。初為侍官。會平陳之役,定和當從征,無以自給。其妻有嫁時衣服,定和將鬻之,妻靳固不與,定和於是遂行。以功拜儀同,賜帛千匹,遂棄其妻。是後數以軍功加上開府、驃騎將軍。從上柱國李充擊突厥,先登陷陣,虜刺之中頸,定
 和以草塞創而戰,神氣自若,虜遂敗走。上聞而壯之,遣使者齎藥,馳詣定和所勞問之。進位柱國,封武安縣侯,賞物二千段,良馬二匹,金百兩。煬帝嗣位,拜宜州刺史,尋轉河內太守,頗有惠政。



 歲餘,徵拜左屯衛大將軍。從帝征吐谷渾,至覆袁川。時吐谷渾主與數騎而遁,其名王詐為渾主,保車我真山,帝命定和率師擊之。既與賊相遇,輕其眾少,呼之令降,賊不肯下。定和不被甲,挺身登山,賊伏兵於巖谷之下,發矢中之而斃。其亞將柳武建擊賊,悉斬之。帝為流涕,贈光祿大夫。時舊爵例除,於是復封武安侯,謚曰壯武。贈絹千匹,米千石。子世立嗣,
 尋拜為光祿大夫。



 張奫,字文懿,自雲清河人也,家於淮陰。好讀兵書,尤便刀楯。周世,鄉人郭子翼密引陳寇,奫父雙欲率子弟擊之,猶豫未決。奫贊成其謀,竟以破賊,由是以勇決知名。起家州主簿。高祖作相,授大都督,領鄉兵。賀若弼之鎮壽春也,恆為間諜,平陳之役,頗有功焉。進位開府儀同三司,封文安縣子,邑八百戶,賜物二千五百段,粟二千五百石。歲餘,率水軍破逆賊笮子游於京口、薛子建於和州。



 徵入朝,拜大將軍。高祖命升御坐而宴之,謂奫曰:「
 卿可為朕兒,朕為卿父。今日聚集,示無外也。」其後賜綺羅千匹,綠沉甲、獸文具裝。尋從楊素征江表,別破高智慧於會稽、吳世華於臨海。進位上大將軍,賜奴婢六十口,縑彩三百匹。歷撫、顯、齊三州刺史,俱有能名。開皇十八年,為行軍總管,從漢王諒征遼東。諸軍多物故,奫眾獨全。高祖善之,賜物二百五十段。仁壽中,遷潭州總管,在職三年卒。有子孝廉。



 麥鐵杖,始興人也。驍勇有膂力,日行五百里,走及奔馬。性疏誕使酒,好交游,重信義,每以漁獵為事,不治產業。
 陳太建中,結聚為群盜,廣州刺史歐陽頠俘之以獻,沒為官戶,配執御傘。每罷朝後,行百餘里,夜至南徐州,俞城而入,行光火劫盜。旦還,及時仍又執傘。如此者十餘度,物主識之,州以狀奏。朝士見鐵杖每旦恆在,不之信也。後數告變,尚書蔡徵曰:「此可驗耳。」於仗下時,購以百金,求人送詔書與南徐州刺史。鐵杖出應募,齎敕而往,明旦及奏事。帝曰:「信然,為盜明矣。」惜其勇捷,誡而釋之。



 陳亡後,徙居清流縣。遇江東反,楊素遣鐵杖頭戴草束,夜浮渡江,覘賊中消息,具知還報。後復更往,為賊所擒。逆帥李棱遣兵仗三十人衛之,縛送高智慧。



 行至慶亭,
 衛者憩食,哀其餒,解手以給其餐。鐵杖取賊刀,亂斬衛者,殺之皆盡,悉割其鼻,懷之以歸。素大奇之。後敘戰勛,不及鐵杖,遇素馳驛歸於京師,鐵杖步追之,每夜則同宿。素見而悟,特奏授儀同三司。以不識書,放還鄉里。成陽公李徹稱其驍武,開皇十六年,徵至京師,除車騎將軍,仍從楊素北征突厥,加上開府。煬帝即位,漢王諒反於並州,又從楊素擊之,每戰先登。進位柱國。尋除萊州刺史,無治名。後轉汝南太守,稍習法令,群盜屏跡。後因朝集,考功郎竇威嘲之曰:「麥是何姓?」鐵杖應口對曰:「麥豆不殊,那忽相怪!」威赧然,無以應之,時人以為敏慧。尋
 除右屯衛大將軍,帝待之逾密。



 鐵杖自以荷恩深重,每懷竭命之志。及遼東之役,請為前鋒,顧謂醫者吳景賢曰:「大丈夫性命自有所在,豈能艾炷灸頞,瓜蒂噴鼻,治黃不差,而臥死兒女手中乎?」將渡遼,謂其三子曰:「阿奴當備淺色黃衫。吾荷國恩,今是死日。我既被殺,爾當富貴。唯誠與孝,爾其勉之。」及濟,橋未成,去東岸尚數丈,賊大至。



 鐵杖跳上岸,與賊戰,死。武賁郎將錢士雄、孟金叉亦死之,左右更無及者。帝為之流涕,購得其尸,下詔曰:「鐵杖志氣驍果,夙著勛庸,陪麾問罪,先登陷陣,節高義烈,身殞功存。興言至誠,追懷傷悼,宜賚殊榮。用彰飾德。
 可贈光祿大夫、宿國公。謚曰武烈。」子孟才嗣。尋授光祿大夫。孟才有二弟,仲才、季才,俱拜正議大夫。賵贈巨萬,賜轀輬車,給前後部羽葆鼓吹。平壤道敗將宇文述等百餘人皆為執紼,王公已下送至郊外,士雄贈左光祿大夫、右屯衛將軍、武強侯,謚曰剛。



 子傑嗣。金叉贈右光祿大夫,子善誼襲官。



 孟才字智棱,果烈有父風。帝以孟才死節將子,恩賜殊厚,拜武賁郎將。及江都之難,慨然有復仇之志。與武牙郎錢傑素交友,二人相謂曰:「吾等世荷國恩,門著誠節。今賊臣弒逆,社稷淪亡,無節可紀,何面目視息世間哉!」於是流涕扼腕,遂相與謀,糾合恩
 舊,欲於顯福宮邀擊宇文化及。事臨發,陳籓之子謙知其謀而告之,與其黨沈光俱為化及所害,忠義之士哀焉。



 沈光,字總持,吳興人也。父君道,仕陳吏部侍郎,陳滅,家於長安。皇太子勇引署學士。後為漢王諒府掾,諒敗,除名。光少驍捷,善戲馬,為天下之最。略綜書記,微有詞藻,常慕立功名,不拘小節。家甚貧窶,父兄並以傭書為事,光獨跅馳,交通輕俠,為京師惡少年之所朋附。人多贍遺,得以養親,每致甘食美服,未嘗困匱。初建禪定寺,其
 中幡竿高十餘丈,適遇繩絕,非人力所及,諸僧患之。



 光見而謂僧曰:「可持繩來,當相為上耳。」諸僧驚喜,因取而與之。光以口銜索,拍竿而上,直至龍頭。系繩畢,手足皆放,透空而下,以掌拒地,倒行數十步。觀者駭悅,莫不嗟異,時人號為「肉飛仙」。



 大業中,煬帝徵天下驍果之士以伐遼左,光預焉。同類數萬人,皆出其下。光將詣行在所,賓客送至灞上者百餘騎。光酹酒而誓曰:「是行也,若不能建立功名,當死於高麗,不復與諸君相見矣。」及從帝攻遼東,以沖梯擊城,竿長十五丈,光升其端,臨城與賊戰,短兵接,殺十數人。賊競擊之而墜,未及於地,適遇竿
 有垂絙,光接而復上。帝望見,壯異之,馳召與語,大悅,即日拜朝請大夫,賜寶刀良馬,恆致左右,親顧漸密。未幾,以為折沖郎將,賞遇優重。帝每推食解衣以賜之,同輩莫與為比。



 光自以荷恩深重,思懷竭節。及江都之難,潛構義勇,將為帝復仇。先是,帝寵暱官奴,名為給使,宇文化及以光驍勇,方任之,令其總統,營於禁內。時孟才、錢傑等陰圖化及,因謂光曰:「我等荷國厚恩,不能死難以衛社稷,斯則古人之所恥也。今又俯首事讎,受其驅率,有熏面目,何用生為?吾必欲殺之,死無所恨,公義士也,肯從我乎?」光泣下沾衿,曰:「是所望於將軍也。僕領給使
 數百人,並荷先帝恩遇,今在化及內營。以此復讎,如鷹鸇之逐鳥雀。萬世之功,在此一舉,願將軍勉之。」孟才為將軍,領江淮之眾數千人,期以營將發時,晨起襲化及。光語洩,陳謙告其事。化及大懼曰:「此麥鐵杖子也,及沈光者,並勇決不可當,須避其鋒。」是夜即與腹心走出營外,留人告司馬德戡等,遣領兵馬,逮捕孟才。光聞營內喧聲,知事發,不及被甲,即襲化及營,空無所獲。值舍人元敏,數而斬之。



 遇德戡兵入,四面圍合。光大呼潰圍,給使齊奮,斬首數十級,賊皆披靡。德戡輒復遣騎,持弓弩,翼而射之。光身無介胄,遂為所害。麾下數百人皆斗而
 死,一無降者。時年二十八。壯士聞之,莫不為之隕涕。



 來護兒,字崇善,江都人也。幼而卓詭,好立奇節。初讀《詩》,至「擊鼓其鏜,踴躍用兵」、「羔裘豹飾,孔武有力」,舍書而嘆曰:「大丈夫在世當如是。



 會為國滅賊以取功名,安能區區久事隴畝!」群輩驚其言而壯其志。護兒所住白土村,密邇江岸。於時江南尚阻,賀若弼之鎮壽州也,常令護兒為間諜,授大都督。



 平陳之役,護兒有功焉,進位上開府。從楊素擊高智慧於浙江,而賊據岸為營,周亙百餘里,船艦被江,鼓噪而進。素令護兒率數百輕艓徑登江
 岸,直掩其營,破之。



 時賊前與素戰不勝,歸無所據,因而潰散。智慧將逃於海,護兒追至泉州,智慧窮蹙,遁走閩、越。進位大將軍,除泉州刺史。時有盛道延擁兵作亂,侵擾州境,護兒進擊,破之。又從蒲山公李寬破汪文進於黟、歙,進位柱國。仁壽三年,除瀛州刺史,賜爵黃縣公,邑三千戶。尋加上柱國,除右御衛將軍。煬帝即位,遷右驍衛大將軍,帝甚親重之。大業六年,從駕江都,賜物千段,令上先人塚,宴父老,州里榮之。數歲,轉右翊衛大將軍。遼東之役,護兒率樓船,指滄海,入自壩水,去平壤六十里,與高麗相遇。進擊,大破之,乘勝直造城下,破其郛郭。
 於是縱軍大掠,稍失部伍,高元弟建武募敢死士五百人邀擊之。護兒因卻,屯營海浦,以待期會。後知宇文述等敗,遂班師。明年,又出滄海道,師次東萊,會楊玄感作逆黎陽,進逼鞏、洛,護兒勒兵與宇文述等擊破之。封榮國公,邑二千戶。十年,又帥師度海,至卑奢城,高麗舉國來戰,護兒大破之,斬首千餘級。將趣平壤,高元震懼,遣使執叛臣斛斯政,詣遼東城下,上表請降。帝許之,遣人持節詔護兒旋師。護兒集眾曰:「三度出兵,未能平賊,此還也,不可重來。今高麗困弊,野無青草,以我眾戰,不日克之。吾欲進兵,徑圍平壤,取其偽主,獻捷而歸。」答表請
 行,不肯奉詔。長史崔君肅固爭,不許。護兒曰:」賊勢破矣,專以相任,自足辦之。吾在閫外,事合專決,豈容千里稟聽成規!俄頃之間,動失機會,勞而無功,故其宜也。吾寧征得高元,還而獲譴,舍此成功,所不能矣。」君肅告眾曰:「若從元帥,違拒詔書,必當聞奏,皆獲罪也。」諸將懼,盡勸還,方始奉詔。十三年,轉為左翊衛大將軍,進位開府儀同三司,任委逾密,前後賞賜不可勝計。江都之難,宇文化及忌而害之。



 長子楷,以父軍功授散騎郎、朝散大夫。楷弟弘,仕至果毅郎將、金紫光祿大夫。弘第整,武賁郎將、右光祿大夫。整尤驍勇,善撫士眾,討擊群盜,所向皆
 捷。



 諸賊甚憚之,為作歌曰:「長白山頭百戰場,十十五五把長槍,不畏官軍十萬眾,只畏榮公第六郎。」化及反,皆遇害,唯少子恆、濟獲免。



 魚俱羅,馮翊下邦人也。身長八尺,膂力絕人,聲氣雄壯,言聞數百步。弱冠為親衛,累遷大都督。從晉王廣平陳,以功拜開府,賜物一千五百段。未幾,沈玄懀、高智慧等作亂江南,楊素以俱羅壯勇,請與同行。每戰有功,加上開府、高唐縣公,拜疊州總管。以母憂去職。還至扶風,會楊素率兵將出靈州道擊突厥,路逢俱羅,大悅,遂奏與
 同行。及遇賊,俱羅與數騎奔擊,瞋目大呼,所當皆披靡,出左入右,往返若飛。以功進位柱國,拜豐州總管。初,突厥數入境為寇,俱羅輒擒斬之,自是突厥畏懼屏跡,不敢畜牧於塞上。



 初,煬帝在籓,俱羅弟贊以左右從,累遷大都督。及帝嗣位,拜車騎將軍。贊性兇暴,虐其部下,令左右炙肉,遇不中意,以簽刺瞎其眼。有溫酒不適者,立斷其舌。帝以贊籓邸之舊,不忍加誅,謂近臣曰:「弟既如此,兄亦可知。」因召俱羅,譴責之,出贊於獄,令自為計。贊至家,飲藥而死。帝恐俱羅不自安,慮生邊患,轉為安州刺史。歲餘,遷趙郡太守。後因朝集,至東都,與將軍梁伯
 隱有舊,數相往來。又從郡多將雜物以貢獻,帝不受,因遺權貴。御史劾俱羅以郡將交通內臣,帝大怒,與伯隱俱坐除名。



 未幾,越巂飛山蠻作亂,侵掠郡境。詔俱羅白衣領將,並率蜀郡都尉段鐘葵討平之。大業九年,重征高麗,以俱羅為碣石道軍將。及還,江南劉元進作亂,詔俱羅將兵向會稽諸郡逐捕之。於時百姓思亂,從盜如市,俱羅擊賊帥硃燮、管崇等,戰無不捷。然賊勢浸盛,敗而復聚。俱羅度賊非歲月可平,諸子並在京、洛,又見天下漸亂,終恐道路隔絕。於時東都饑饉,穀食踴貴,俱羅遣家僕將船米至東都糶之,益市財貨,潛迎諸子。朝廷
 微知之,恐其有異志,發使案驗。使者至,前後察問,不得其罪。帝復令大理司直梁敬真就鎖將詣東都。俱羅相表異人,目有重瞳,陰為帝之所忌。敬真希旨,奏俱羅師徒敗衄,於是斬東都市,家口籍沒。



 陳棱,字長威,廬江襄安人也。祖碩,以漁釣自給。父峴,少驍勇,事章大寶為帳內部曲。告大寶反,授譙州刺史。陳滅,廢於家。高智慧、汪文進等作亂江南,廬江豪傑亦舉兵相應,以峴舊將,共推為主。峴欲拒之,棱謂峴曰:「眾亂既作,拒之禍且及己。不如偽從,別為後計。」峴然之。時柱
 國李徹軍至當塗,峴潛使棱至徹所,請為內應。徹上其事,拜上大將軍、宣州刺史,封譙郡公,邑一千戶,詔徹應接之。徹軍未至,謀洩,為其黨所殺,棱僅以獲免。上以其父之故,拜開府,尋領鄉兵。煬帝即位,授驃騎將軍。大業三年,拜武賁郎將。後三歲,與朝請大夫張鎮周發東陽兵萬餘人,自義安泛海,擊流求國,月餘而至。流求人初見船艦,以為商旅,往往詣軍中貿易。棱率眾登岸,遣鎮周為先鋒。其主歡斯渴剌兜遣兵拒戰,鎮周頻擊破之。棱進至低沒檀洞,其小王歡斯老模率兵拒戰,棱擊敗之,斬老模。



 其日霧雨晦冥,將士皆懼,棱刑白馬以祭海
 神。既而開霽,分為五軍,趣其都邑。



 渴剌兜率眾數千逆拒,棱遣鎮周又先鋒擊走之。棱乘勝逐北,至其柵,渴剌兜背柵而陣。棱盡銳擊之,從辰至未,苦鬥不息。渴剌兜自以軍疲,引入柵。棱遂填塹,攻破其柵,斬渴剌兜,獲其子島槌,虜男女數千而歸。帝大悅,進棱位右光祿大夫,武賁如故,鎮周金紫光祿大夫。遼東之役,以宿衛遷左光祿大夫。明年,帝復征遼東,棱為東萊留守。楊玄感之作亂也,棱率眾萬餘人擊平黎陽,斬玄感所署刺史元務本。棱尋奉詔於江南營戰艦。至彭城,賊帥孟讓眾將十萬,據都梁宮,阻淮為固。



 棱潛於下流而濟,至江都,率兵
 襲讓,破之。以功進位光祿大夫。賜爵信安侯。後帝幸江都宮,俄而李子通據海陵,左才相掠淮北,杜伏威屯六合,眾各數萬。帝遣棱率宿衛兵擊之,往往克捷。超拜右御衛將軍。復渡清江,擊宣城賊。俄而帝以弒崩,宇文化及引軍北上,召棱守江都。棱集眾縞素,為煬帝發喪,備儀衛,改葬於吳公臺下,衰杖送喪,慟感行路,論者深義之。棱後為李子通所陷,奔杜伏威,伏威忌之,尋而見害。



 王辯,字警略,馮翊蒲城人也。祖訓,以行商致富。魏世,出粟助給軍糧,為假清河太守。辯少習兵書,尤善騎射,慷
 慨有大志。在周以軍功授帥都督。開皇初,遷大都督。仁壽中,遷車騎將軍。漢王諒之作亂也,從楊素討平之,賜爵武寧縣男,邑三百戶。後三歲,遷尚舍奉御。從征吐谷渾,拜朝請大夫。數年,轉鷹揚郎將。



 遼東之役,以功加通議大夫,尋遷武賁郎將。及山東盜賊起,上谷魏刀兒自號歷山飛,眾十餘萬,劫掠燕、趙。帝引辯升御榻,問以方略。辯論取賊形勢,帝稱善,曰:「誠如此計,賊何足憂也。」於是發從行步騎三千,擊敗之,賜黃金二百兩。



 明年,渤海賊帥高士達自號東海公,眾以萬數。復令辯擊之,屢挫其銳。帝在江都宮,聞而馳召之。及引見,禮賜甚厚,復令
 往信都經略。士達於是復戰,破之,優詔褒顯。時賊帥郝孝德、孫宣雅、時季康、竇建德、魏刀兒等,往往屯聚,大至十萬,小至數千,寇掠河北。辯進兵擊之,所往皆捷,深為群賊所憚。及翟讓寇徐、豫,辯進,頻擊走之。讓尋與李密屯據洛口倉,辯與王世充討密,阻洛水相持經年。



 辯率諸將攻敗密,因薄其營,戰破外柵。密諸營已有潰者,乘勝將入城,世充不知,恐將士勞倦,於是鳴角收兵,翻為密徒所乘。官軍大潰,不可救止。辯至洛水,橋已壞,不得渡,遂涉水,至中流,為溺人所引墜馬。辯時身被重甲,敗兵前後相蹈藉,不能復上馬,竟溺死焉。時年五十六。三
 軍莫不痛惜之。



 河南斛斯萬善,驍勇果毅,與辯齊名。大業中,從衛玄討楊玄感,頻戰有功。



 及玄感敗走,萬善與數騎追及之,玄感窘迫自殺。由是知名,拜武賁郎將。突厥始畢之圍雁門也,萬善奮擊之,所向皆破。每賊至,輒出當其鋒,或下馬坐地,引強弓射賊,所中皆殪。由是突厥莫敢逼城,十許日竟退,萬善之力也。其後頻討群盜,累功至將軍。時有將軍鹿願、範貴、馮孝慈,俱為將帥,數從征討,並有名於世。



 然事皆亡失,故史官無所述焉。



 史臣曰:楚、漢未分,絳、灌所以宣力;曹、劉競逐,關、張所以立名。然則名立資草昧之初,力宣候經輪之會,攀附鱗
 翼,世有之矣。圓通、護兒之輩,定和、鐵杖之倫,皆一時之壯士,困於貧賤。當其鬱抑未遇,亦安知其有鴻鵠之志哉!終能振拔污泥之中,騰躍風雲之上,符馬革之願,快生平之心,非遇其時,焉能至於此也!俱羅欲加之罪,非其咎畔,王辯殞身勍敵,志實勤王。陳棱縞素發喪,哀感行路,義之所動,固已深乎!孟才、錢傑、沈光等,感恩懷舊,臨難忘生,雖功無所成,其志有可稱矣。



\end{pinyinscope}