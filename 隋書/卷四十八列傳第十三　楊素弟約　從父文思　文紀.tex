\article{卷四十八列傳第十三 楊素弟約 從父文思 文紀}

\begin{pinyinscope}

 楊素,字處道,弘農華陰人也。祖暄,魏輔國將軍、諫議大夫。父敷,周汾州刺史,沒於齊。素少落拓,有大志,不拘小節,世人多未之知,唯從叔祖魏尚書僕射寬深異之,每謂子孫曰:「處道當逸群絕倫,非常之器,非汝曹所逮也。」後與安定牛弘同志好學,研精不倦,多所通涉。善屬文,工草隸,頗留意於風角。美須髯,有英傑之表。周大塚宰
 宇文護引為中外記室,後轉禮曹,加大都督。武帝親總萬機,素以其父守節陷齊,未蒙朝命,上表申理,帝不許。至於再三,帝大怒,命左右斬之。素乃大言曰:「臣事無道天子,死其分也。」帝壯其言,由是贈敷為大將軍,謚曰忠壯。拜素為車騎大將軍、儀同三司,漸見禮遇。帝命素為詔書,下筆立成,詞義兼美。帝嘉之,顧謂素曰:「善自勉之,勿憂不富貴。」素應聲答曰:「臣但恐富貴來逼臣,臣無心圖富貴。」



 及平齊之役,素請率父麾下先驅。帝從之,賜以竹策,曰:「朕方欲大相驅策,故用此物賜卿。」從齊王憲與齊人戰於河陰,以功封清河縣子,邑五百戶。其年授司
 城大夫。明年,復從憲拔晉州。憲屯兵雞棲原,齊主以大軍至,憲懼而宵遁,為齊兵所躡,眾多敗散。素與驍將十餘人盡力苦戰,憲僅而獲免。其後每戰有功。及齊平,加上開府,改封成安縣公,邑千五百戶,賜以粟帛、奴婢、雜畜。從王軌破陳將吳明徹於呂梁,治東楚州事。封弟慎為義安侯。陳將樊毅築城於泗口,素擊走之,夷毅所築。宣帝即位,襲父爵臨貞縣公,以弟約為安成公。尋從韋孝寬徇淮南,素別下盱眙、鐘離。



 及高祖為丞相,素深自結納。高祖甚器之,以素為汴州刺史。行至洛陽,會尉迥作亂,滎州刺史宇文胄據武牢以應迥,素不得進。高祖
 拜素大將軍,發河內兵擊胄,破之。遷徐州總管,進位柱國,封清河郡公,邑二千戶。以弟岳為臨貞公。高祖受禪,加上柱國。開皇四年,拜御史大夫。其妻鄭氏性悍,素忿之曰:「我若作天子,卿定不堪為皇后。」鄭氏奏之,由是坐免。



 上方圖江表,先是,素數進取陳之計,未幾,拜信州總管,賜錢百萬、錦千段、馬二百匹而遣之。素居永安,造大艦,名曰五牙,上起樓五層,高百餘尺,左右前後置六拍竿,並高五十尺,容戰士八百人,旗幟加於上。次曰黃龍,置兵百人。自餘平乘、舴艋等各有差。及大舉伐陳,以素為行軍元帥,引舟師趣三硤。軍至流頭灘,陳將戚欣以
 青龍百餘艘、屯兵數千人守狼尾灘,以遏軍路。其地險峭,諸將患之。素曰:「勝負大計,在此一舉。若晝日下船,彼則見我,灘流迅激,制不由人,則吾失其便。」乃以夜掩之。素親率黃龍數千艘,銜枚而下,遣開府王長襲引步卒從南岸擊欣別柵,令大將軍劉仁恩率甲騎趣白沙北岸,遲明而至,擊之,欣敗走。



 悉虜其眾,勞而遣之,秋毫不犯,陳人大悅。素率水軍東下,舟艫被江,旌甲曜日。



 素坐平乘大船,容貌雄偉,陳人望之懼曰:「清河公即江神也。」陳南康內史呂仲肅屯岐亭,正據江峽,於北岸鑿巖,綴鐵鎖三條,橫截上流,以遏戰船。素與仁恩登陸俱發,先
 攻其柵。仲肅軍夜潰,素徐去其鎖。仲肅復據荊門之延洲。素遣巴蜒卒千人,乘五牙四艘,以柏檣碎賊十餘艦,遂大破之,俘甲士二千餘人,仲肅僅以身免。陳主遣其信州刺史顧覺鎮安蜀城,荊州刺史陳紀鎮公安,皆懾而退走。巴陵以東,無敢守者。湘州刺史、岳陽王陳叔慎遣使請降。素下至漢口,與秦孝王會。



 及還,拜荊州總管,進爵郢國公,邑三千戶,真食長壽縣千戶。以其子玄感為儀同,玄獎為清河郡公。賜物萬段,粟萬石,加以金寶,又賜陳主妹及女妓十四人。素言於上曰:「里名勝母,曾子不入。逆人王誼,前封於郢,臣不願與之同。」於是改封
 越國公。尋拜納言。歲餘,轉內史令。



 俄而江南人李稜等聚眾為亂,大者數萬,小者數千,共相影響,殺害長吏。以素為行軍總管,帥眾討之。賊硃莫問自稱南徐州刺史,以盛兵據京口。素率舟師入自楊子津,進擊破之。晉陵顧世興自稱太守,與其都督鮑遷等復來拒戰。素逆擊破之,執遷,虜三千餘人。進擊無錫賊帥葉略,又平之。吳郡沈玄懀、沈傑等以兵圍蘇州,刺史皇甫績頻戰不利。素率眾援之,玄懀勢迫,走投南沙賊帥陸孟孫。素擊孟孫於松江,大破之,生擒孟孫、玄懀。黟、歙賊帥沈雪、沈能據柵自固,又攻拔之。浙江賊帥高智慧自號東揚州刺
 史,船艦千艘,屯據要害,兵甚勁。素擊之,自旦至申,苦戰而破。智慧逃入海,素躡之,從餘姚泛海趣永嘉。智慧來拒戰,素擊走之,擒獲數千人。賊帥汪文進自稱天子,據東陽,署其徒蔡道人為司空,守樂安。



 進討,悉平之。又破永嘉賊帥沈孝徹。於是步道向天臺,指臨海郡,逐捕遺逸寇。



 前後百餘戰,智慧遁守閩越。



 上以素久勞於外,詔令馳傳入朝。加子玄感官為上開府,賜彩物三千段。素以餘賊未殄,恐為後患,又自請行。乃下詔曰:「朕憂勞百姓,日旰忘食,一物失所,情深納隍。江外狂狡,妄構妖逆,雖經殄除,民未安堵。猶有賊首兇魁,逃亡山洞,恐其聚
 結,重擾蒼生。內史令、上柱國、越國公素,識達古今,經謀長遠,比曾推轂,舊著威名,宜任以大兵,總為元帥,宣布朝風,振揚威武,擒剪叛亡,慰勞黎庶。軍民事務,一以委之。」素復乘傳至會稽。先是,泉州人王國慶,南安豪族也,殺刺史劉弘,據州為亂,諸亡賊皆歸之。自以海路艱阻,非北人所習,不設備伍。



 素泛海掩至,國慶遑遽,棄州而走,餘黨散入海島,或守溪洞。素分遣諸將,水陸追捕。乃密令人謂國慶曰:「爾之罪狀,計不容誅。唯有斬送智慧,可以塞責。」



 國慶於是執送智慧,斬於泉州。自餘支黨,悉來降附,江南大定。上遣左領軍將軍獨孤陀至浚儀迎
 勞。比到京師,問者日至。拜素子玄獎為儀同,賜黃金四十斤,加銀瓶,實以金錢,縑三千段,馬二百匹,羊二千口,公田百頃,宅一區。代蘇威為尚書右僕射,與高熲專掌朝政。



 素性疏而辯,高下在心,朝臣之內,頗推高熲,敬牛弘,厚接薛道衡,視蘇威蔑如也。自餘朝貴,多被陵轢。其才藝風調,優於高熲,至於推誠體國,處物平當,有宰相識度,不如熲遠矣。



 尋令素監營仁壽宮,素遂夷山堙谷,督役嚴急,作者多死,宮側時聞鬼哭之聲。



 及宮成,上令高熲前視,奏稱頗傷綺麗,大損人丁,高祖不悅。素尤懼,計無所出,即於北門啟獨孤皇后曰:「帝王法有離宮別
 館,今天下太平,造此一宮,何足損費!」



 後以此理諭上,上意乃解。於是賜錢百萬,錦絹三千段。



 十八年,突厥達頭可汗犯塞,以素為靈州道行軍總管,出塞討之,賜物二千段,黃金百斤。先是,諸將與虜戰,每慮胡騎奔突,皆以戎車步騎相參,輿鹿角為方陣,騎在其內。素謂人曰:「此乃自固之道,非取勝之方也。」於是悉除舊法,令諸軍為騎陣。達頭聞之大喜,曰:「此天賜我也。」因下馬仰天而拜,率精騎十餘萬而至。素奮擊,大破之,達頭被重創而遁,殺傷不可勝計,群虜號哭而去。優詔褒揚,賜縑二萬匹,及萬釘寶帶。加子玄感位大將軍,玄獎、玄縱、積善並上
 儀同。



 素多權略,乘機赴敵,應變無方,然大抵馭戎嚴整,有犯軍令者立斬之,無所寬貸。每將臨寇,輒求人過失而斬之,多者百餘人,少不下十數。流血盈前,言笑自若。及其對陣,先令一二百人赴敵,陷陣則已,如不能陷陣而還者,無問多少,悉斬之。又令三二百人復進,還如向法。將士股心慄,有必死之心,由是戰無不勝,稱為名將。素時貴幸,言無不從,其從素征伐者,微功必錄,至於他將,雖有大功,多為文吏所譴卻。故素雖嚴忍,士亦以此願從焉。



 二十年,晉王廣為靈朔道行軍元帥,素為長史。王卑躬以交素。及為太子,素之謀也。



 仁壽初,代高熲為尚
 書左僕射,賜良馬百匹,牝馬二百匹,奴婢百口。其年,以素為行軍元帥,出雲州擊突厥,連破之。突厥退走,率騎追躡,至夜而及之。將復戰,恐賊越逸,令其騎稍後。於是親將兩騎,並降突厥二人,與虜並行,不之覺也。候其頓舍未定,趣後騎掩擊,大破之。自是突厥遠遁,磧南無復虜庭。以功進子玄感位為柱國,玄縱為淮南郡公。賞物二萬段。



 及獻皇后崩,山陵制度,多出於素。上善之,下詔曰:君為元首,臣則股肱,共治萬姓,義同一體。上柱國、尚書左僕射、仁壽宮大監、越國公素,志度恢弘,機鑒明遠,懷佐時之略,包經國之才。王業初基,霸圖肇建,策名委
 質,受脤出師,擒剪兇魁,克平虢、鄭。頻承廟算,揚旍江表,每稟戎律,長驅塞陰,南指而吳越肅清,北臨而獯獫摧服。自居端揆,參贊機衡,當朝正色,直言無隱。論文則詞藻縱橫,語武則權奇間出。既文且武,唯朕所命,任使之處,夙夜無怠。獻皇后奄離六宮,遠日雲及,塋兆安厝,委素經營。然葬事依禮,唯卜泉石,至如吉兇,不由於此。素義存奉上,情深體國,欲使幽明俱泰,寶祚無窮。以為陰陽之書,聖人所作,禍福之理,特須審慎。乃遍歷川原,親自占擇,纖介不善,即更尋求,志圖元吉,孜孜不已。心力備盡,人靈協贊,遂得神皋福壤,營建山陵。論素此心,事
 極誠孝,豈與夫平戎定寇比其功業?非唯廊廟之器,實是社稷之臣,若不加褒賞,何以申茲勸勵?可別封一子義康郡公,邑萬戶,子子孫孫,承襲不絕。餘如故。



 並賜田三十頃,絹萬段,米萬石,金缽一,實以金,銀缽一,實以珠,並綾錦五百段。



 時素貴寵日隆,其弟約、從父文思、弟文紀,及族父異,並尚書列卿。諸子無汗馬之勞,位至柱國、刺史。家僮數千,後庭妓妾曳綺羅者以千數。第宅華侈,制擬宮禁。有鮑亨者,善屬文,殷胄者,工草隸,並江南士人,因高智慧沒為家奴。



 親戚故吏,布列清顯,素之貴盛,近古未聞。煬帝初為太子,忌蜀王秀,與素謀之,構成其
 罪,後竟廢黜。朝臣有違忤者,雖至誠體國,如賀若弼、史萬歲、李綱、柳彧等,素皆陰中之。若有附會及親戚,雖無才用,必加進擢。朝廷靡然,莫不畏附。



 唯兵部尚書柳述,以帝婿之重,數於上前面折素。大理卿梁毗,抗表上言素作威作福。上漸疏忌之,後因出敕曰:「僕射國之宰輔,不可躬親細務,但三五日一度向省,評論大事。」外示優崇,實奪之權也。終仁壽之末,不復通判省事。上賜王公以下射,素箭為第一,上手以外國所獻金精盤,價直巨萬,以賜之。四年,從幸仁壽宮,宴賜重疊。



 及上不豫,素與兵部尚書柳述、黃門侍郎元巖等入閤侍疾。時皇太子
 入居大寶殿,慮上有不諱,須豫防擬,乃手自為書,封出問素。素錄出事狀以報太子。宮人誤送上所,上覽而大恚。所寵陳貴人又言太子無禮。上遂發怒,欲召庶人勇。太子謀之於素,素矯詔追東宮兵士帖上臺宿衛,門禁出入,並取宇文述、郭衍節度,又令張衡侍疾。上以此日崩,由是頗有異論。



 漢王諒反,遣茹茹天保來據蒲州,燒斷河橋。又遣王聃子率數萬人並力拒守。



 素將輕騎五千襲之,潛於渭口宵濟,遲明擊之,天保敗走,聃子懾而以城降。有詔徵還。初,素將行也,計日破賊,皆如所量。帝於是以素為並州道行軍總管、河北安撫大使,率眾數
 萬討諒。時晉、絳、呂三州並為諒城守,素各以二千人縻之而去。



 諒遣趙子開擁眾十餘萬,策絕徑路,屯據高壁,布陣五十里。素令諸將以兵臨之,自引奇兵潛入霍山,緣崖谷而進,直指其營,一戰破之,殺傷數萬。諒所署介州刺史梁修羅屯介休,聞素至,懼,棄城而走。進至清源,去並州三十里,諒率其將王世宗、趙子開、蕭摩訶等,眾且十萬,來拒戰。又擊破之,擒蕭摩訶。諒退保並州,素進兵圍之,諒窮蹙而降,餘黨悉平。帝遣素弟修武公約齎手詔勞素曰:我有隋之御天下也,於今二十有四年,雖復外夷侵叛,而內難不作,修文偃武,四海晏然。朕以不
 天,銜恤在疚,號天叩地,無所逮及。朕本以籓王,謬膺儲兩,復以庸虛,纂承鴻業。天下者,先皇之天下也,所以戰戰兢兢,弗敢失墜,況復神器之重,生民之大哉!賊諒包藏禍心,自幼而長,羊質獸心,假托名譽,不奉國諱,先圖叛逆,違君父之命,成莫大之罪。誑惑良善,委任奸回,稱兵內侮,毒流百姓。



 私假署置,擅相謀戮,小加大,少凌長,民怨神怒,眾叛親離,為惡不同,同歸於亂。朕寡兄弟,猶未忍及言,是故開關門而待寇,揖干戈而不發。朕聞之,天生蒸民,為之置君,仰惟先旨,每以子民為念,朕豈得枕伏苫廬,顛而不救也!大義滅親,《春秋》高義,周旦以誅
 二叔,漢啟乃戮七籓,義在茲乎?事不獲已,是以授公戎律,問罪太原。且逆子賊臣,何代不有,豈意今者,近出家國。所嘆荼毒甫爾,便及此事。由朕不能和兄弟,不能安蒼生,德澤未弘,兵戈先動,賊亂者止一從,塗炭者乃眾庶。非唯寅畏天威,亦乃孤負付囑,薄德厚恥,愧乎天下。



 公乃先朝功臣,勛庸克茂。至如皇基草創,百物惟始,便匹馬歸朝,誠識兼至。



 汴部鄭州,風卷秋籜,荊南塞北,若火燎原,早建殊勛,夙著誠節。及獻替朝端,具瞻惟允,爰弼朕躬,以濟時難,昔周勃、霍光,何以加也!賊乃竊據蒲州,關梁斷絕,公以少擊眾,指期平殄。高壁據嶺,抗拒官
 軍,公以深謀,出其不意,霧廓雲除,冰消瓦解,長驅北邁,直趣巢窟。晉陽之南,蟻徒數萬,諒不量力,猶欲舉斧。公以棱威外討,發憤於內,忘身殉義,親當矢石。兵刃暫交,魚潰鳥散,殭尸蔽野,積甲若山。諒遂守窮城,以拒鈇鉞。公董率驍勇,四面攻圍,使其欲戰不敢,求走無路,智力俱盡,面縛軍門。斬將搴旗,伐叛柔服,元惡既除,東夏清晏,嘉庸茂績,於是乎在。昔武安平趙,淮陰定齊,豈若公遠而不勞,速而克捷者也!朕殷憂諒闇,不得親御六軍,未能問道於上庠,遂使劬勞於行陣。言念於此,無忘寢食。公乃建累世之元勛,執一心之確志。古人有言曰:「疾
 風知勁草,世亂有誠臣。」



 公得之矣。乃銘之常鼎,豈止書勛竹帛哉!功績克諧,哽嘆無已。稍冷,公如宜。



 軍旅務殷,殊當勞慮,故遣公弟,指宣往懷。迷塞不次。



 素上表陳謝曰:臣自惟虛薄,志不及遠,州郡之職,敢憚劬勞,卿相之榮,無階覬望。然時逢昌運,王業惟始,雖涓流赴海,誠心屢竭,輕塵集岳,功力蓋微。徒以南陽里閭,豐沛子弟,高位重爵,榮顯一時。遂復入處朝端,出總戎律,受文武之任,預帷幄之謀。豈臣才能,實由恩澤。欲報之德,義極昊天。伏惟陛下照重離之明,養繼天之德,牧臣於疏遠,照臣以光暉,南服降枉道之書,春宮奉肅成之旨。然草
 木無識,尚榮枯候時,況臣有心,實自效無路。盡夜回徨,寢食慚惕,常懼朝露奄至,虛負聖慈。賊諒包藏禍心,有自來矣,因幸國哀,便肆兇逆,興兵晉、代,搖蕩山東。



 陛下拔臣於凡流,授臣以戎律,蒙心膂之寄,稟平亂之規。蕭王赤心,人皆以死,漢皇大度,天下爭歸,妖寇廓清,豈臣之力!曲蒙使臣弟約齎詔書問勞,高旨峻筆,有若天臨,洪恩大澤,便同海運。悲欣慚懼,五情振越,雖百殞微軀,無以一報。



 其月還京師,因從駕幸洛陽,以素領營東京大監。以平諒之功,拜其子萬石、仁行、侄玄挺皆儀同三司,賚物五萬段,綺羅千匹,諒之妓妾二十人。大業元年,
 遷尚書令,賜東京甲第一區,物二千段。尋拜太子太師,餘官如故。前後賞錫,不可勝計。明年,拜司徒,改封楚公,真食二千五百戶。其年,卒官。謚曰景武,贈光祿大夫、太尉公、弘農河東絳郡臨汾文城河內汲郡長平上黨西河十郡太守。給轀車,班劍四十人,前後部羽葆鼓吹,粟麥五千石,物五千段。鴻臚監護喪事。帝又下詔曰:「夫銘功彞器,紀德豐碑,所以垂名跡於不朽,樹風聲於沒世。故楚景武公素,茂績元勛,劬勞王室,竭盡誠節,協贊朕躬。故以道邁三傑,功參十亂。未臻遐壽,遽揖清徽。春秋遞代,方綿歲祀,式播雕篆,用圖勛德,可立碑宰隧,以彰
 盛美。」素嘗以五言詩七百字贈番州刺史薛道衡,詞氣宏拔,風韻秀上,亦為一時盛作。未幾而卒,道衡嘆曰:「人之將死,其言也善,豈若是乎!」有集十卷。



 素雖有建立之策及平楊諒功,然特為帝所猜忌,外示殊禮,內情甚薄。太史言隋分野有大喪,因改封於楚。楚與隋同分,欲以此厭當之。素寢疾之日,帝每令名醫診候,賜以上藥。然密問醫人,恆恐不死。素又自知名位已極,不肯服藥,亦不將慎,每語弟約曰:「我豈須更活耶?」素貪冒財貨,營求產業。東、西二京,居宅侈麗,朝毀夕復,營繕無已。爰及諸方都會處,邸店、水磑並利田宅以千百數,時議以此鄙
 之。子玄感嗣,別有傳。諸子皆坐玄感誅死。



 約字惠伯,素異母弟也。在童兒時,嘗登樹墮地,為查所傷,由是竟為宦者。



 性好沉靜,內多譎詐,好學強記。素友愛之,凡有所為,必先籌於約而後行之。在周末,以素軍功,賜爵安成縣公,拜上儀同三司。高祖受禪,授長秋卿。久之,為邵州刺史,入為宗正少卿,轉大理少卿。



 時皇太子無寵,而晉王廣規欲奪宗,以素幸於上,而雅信約。於是用張衡計,遣宇文述大以金寶賂遺於約,因通王意,說之曰:「夫守正履道,固人臣之常致,反經合義,亦達者之令圖。自古賢人君子,莫不與時消息,以避禍患。公之
 兄弟,功名蓋世,當途用事,有年歲矣。朝臣為足下家所屈辱者,可勝數哉!又儲宮以所欲不行,每切齒於執政。公雖自結於人主,而欲危公者固亦多矣。主上一旦棄群臣,公亦何以取庇?今皇太子失愛於皇后,主上素有廢黜之心,此公所知也。今若請立晉王,在賢兄之口耳。誠能因此時建大功,王必鐫銘於骨髓,斯則去累卵之危,成太山之安也。」約然之,因以白素。素本兇險,聞之大喜,乃撫掌而對曰:「吾之智思,殊不及此,賴汝起予。」約知其計行,復謂素曰:「今皇后之言,上無不用,宜因機會,早自結托,則匪唯長保榮祿,傳祚子孫,又晉王傾身禮士,
 聲名日盛,躬履節儉,有主上之風,以約料之,必能安天下。兄若遲疑,一旦有變,令太子用事,恐禍至無日矣。」素遂行其策,太子果廢。



 及晉王入東宮,引約為左庶子,改封修武縣公,進位大將軍。及素被高祖所疏,出約為伊州刺史。入朝仁壽宮,遇高祖崩,遣約入朝,易留守者,縊殺庶人勇,然後陳兵集眾,發高祖兇問。煬帝聞之曰:「令兄之弟,果堪大任。」即位數日,拜內史令。約有學術,兼達時務,帝甚任之。後數載,加位右光祿大夫。



 後帝在東都,令約詣京師享廟,行至華陰,見其兄墓,遂枉道拜哭,為憲司所劾,坐是免官。未幾,拜淅陽太守。其兄子玄感,時
 為禮部尚書,與約恩義甚篤。



 既愴分離,形於顏色,帝謂之曰:「公比憂瘁,得非為叔邪?」玄感再拜流涕曰:「誠如聖旨。」帝亦思約廢立功,由是徵入朝。未幾,卒,以素子玄挺後之。



 文思字溫才,素從叔也。父寬,魏左僕射,周小塚宰。文思在周,年十一,拜車騎大將軍、儀同三司、散騎常侍。尋以父功,封新豐縣子,邑五百戶。天和初,治武都太守,十姓獠反,文思討平之,復治翼州事。黨項羌叛,文思率州兵討平之。



 進擊資中、武康、隆山生獠及東山獠,並破之。後從陳王攻齊河陰城,又從武帝攻拔晉州,以勛進授上
 儀同三司,改封永寧縣公,增邑至千戶。壽陽劉叔仁作亂,從清河公宇文神舉討之,戰於磚井,在陣生擒叔仁。又別從王誼破賊於鯉魚柵。其後累以軍功,遷果毅右旅下大夫。高祖為丞相,從韋孝寬拒尉迥於武陟。迥遣其將李圍俊懷州,與行軍總管宇文述擊走之。破尉惇,平鄴城,皆有功,進授上大將軍,改封洛川縣公。尋拜隆州刺史。開皇元年,進爵正平郡公,加邑二千戶。後為魏州刺史,甚有惠政,及去職,吏民思之,為立碑頌德。轉冀州刺史。煬帝嗣位,徵為民部尚書。轉納言,改授右光祿大夫。從幸江都宮,以足疾不堪趨奏,復授民部尚書,加
 位左光祿大夫。卒官,時年七十。謚曰定。初,文思當襲父爵,自以非嫡,遂讓封於弟文紀,當世多之。



 文紀字溫範,少剛正,有器局。在周襲爵華山郡公,邑二千七百戶。自右侍上士累遷車騎大將軍、儀同三司、安州總管長史。將兵迎陳降將李瑗於齊安,與陳將周法尚軍遇,擊走之。以功進授開府,入為虞部下大夫。高祖為丞相,改封汾陰縣公。從梁睿討王謙,以功進授上大將軍。前後增邑三千戶。拜資州刺史。入為宗正少卿,坐事除名。後數載,復其爵位,拜熊州刺史。改封上明郡公。除宗正卿。兼給事黃門侍郎,判禮部尚書事。仁壽二年,
 遷荊州總管。歲餘,卒官,時年五十八。



 謚曰恭。



 史臣曰:楊素少而輕俠,俶儻不羈,兼文武之資,包英奇之略,志懷遠大,以功名自許。高祖龍飛,將清六合,許以腹心之奇,每當推彀之重。掃妖氛於牛斗,江海無波;摧驍騎於龍庭,匈奴遠遁。考其夷兇靜亂,功臣莫居其右;覽其奇策高文,足為一時之傑。然專以智詐自立,不由仁義之道,阿諛時主,高下其心。營構離宮,陷君於奢侈;謀廢塚嫡,致國於傾危。終使宗廟丘墟,市朝霜露,究其禍敗之源,實乃素之由也。幸而得死,子為亂階,墳土未幹,闔門殂戮,丘隴發掘,宗族誅夷。則知積惡餘殃,信非
 徒語。多行無禮必自及,其斯之謂歟!約外示溫柔,內懷狡算,為蛇畫足,終傾國本,俾無遺育,宜哉!



\end{pinyinscope}