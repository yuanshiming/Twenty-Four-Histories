\article{卷七十三列傳第三十八}

\begin{pinyinscope}

 循吏古之善牧人者,養之以仁,使之以義,教之以禮,隨其所便而處之,因其所欲而與之,從其所好而勸之。如父母之愛子,如兄之愛弟,聞其饑寒為之哀,見其勞苦為之悲,故人敬而悅之,愛而親之。若子產之理鄭國,子賤之居單父,賈琮之牧冀州,文翁之為蜀郡,皆可以恤其災患,導以忠厚,因而利之,惠而不費。其暉映千祀,聲芳不
 絕,夫何為哉?用此道也。然則五帝、三王不易人而化,皆在所由化之而已。故有無能之吏,無不可化之人。高祖膺運撫圖,除兇靜亂,日旰忘食,思邁前王。然不敦詩書,不尚道德,專任法令,嚴察臨下。吏存茍免,罕聞寬惠,乘時射利者,多以一切求名。既煬帝嗣興,志存遠略,車轍馬跡,將遍天下,綱紀馳紊,四維不張。其或善於侵漁,強於剝割,絕億兆之命,遂一人之求者,謂之奉公,即時升擢。其或顧名節,存綱紀,抑奪攘之心,以從百姓之欲者,則謂之附下,旋及誅夷。夫吏之侵漁,得其所欲,雖重其禁,猶或為之。吏之清平,失其所欲,雖崇其賞,猶或不為。
 況於上賞其奸,下得其欲,求得廉潔,不亦難乎!彥光等立嚴察之朝,屬昏狂之主,執心平允,終行仁恕,餘風遺愛,沒而不忘,寬惠之音,足以傳於來葉。故列其行事,以系《循吏》之篇爾。



 梁彥光梁彥光,字修芝,安定烏氏人也。祖茂,魏秦、華二州刺史。父顯,周邢州刺史。彥光少岐嶷,有至性,其父每謂所親曰:「此兒有風骨,當興吾宗。」七歲時,父遇篤疾,醫云餌五石可愈。時求紫石英不得。彥光憂瘁不知所為,忽於園中見一物,彥光所不識,怪而持歸,即紫石英也。親屬咸
 異之,以為至孝所感。魏大統末,入太學,略涉經史,有規檢,造次必以禮。解褐秘書郎,時年十七。周受禪,遷舍人上士。武帝時,累遷小馭下大夫。母憂去職,毀瘁過禮。未幾,起令視事,帝見其毀甚,嗟嘆久之,頻蒙慰諭。後轉小內史下大夫。建德中,為御正下大夫。從帝平齊,以功授開府、陽城縣公,邑千戶。宣帝即位,拜華州刺史,進封華陽郡公,增邑五百戶,以陽城公轉封一子。尋進位上大將軍,遷御正上大夫。俄拜柱國、青州刺史,屬帝崩,不之官。及高祖受禪,以為岐州刺史,兼領岐州宮監,增邑五百戶,通前二千戶。甚有惠政,嘉禾連理,出於州境。開皇
 二年,上幸岐州,悅其能,乃下詔曰:「賞以勸善,義兼訓物。彥光操履平直,識用凝遠,布政岐下,威惠在人,廉慎之譽,聞於天下。三載之後,自當遷陟,恐其匱乏,且宜旌善。可賜粟五百斛,物三百段,御傘一枚,庶使有感朕心,日增其美。四海之內,凡曰官人,慕高山而仰止,聞清風而自勵。」未幾,又賜錢五萬。後數歲,轉相州刺史。彥光前在岐州,其俗頗質,以靜鎮之,合境大化,奏課連最,為天下第一。及居相部,如岐州法。鄴都雜俗,人多變詐,為之作歌,稱其不能理化。上聞而譴之,竟坐免。



 歲餘,拜趙州刺史,彥光言於上曰:「臣前待罪相州,百姓呼為戴帽餳。臣
 自分廢黜,無復衣冠之望,不謂天恩復垂收採。請復為相州,改弦易調,庶有以變其風俗,上答隆恩。」上從之,復為相州刺史。豪猾者聞彥光自請而來,莫不嗤笑。彥光下車,發摘奸隱,有若神明,於是狡猾之徒,莫不潛竄,合境大駭。初,齊亡後,衣冠士人多遷關內,唯技巧、商販及樂戶之家移實州郭。由是人情險詖,妄起風謠,訴訟官人,萬端千變。彥光欲革其弊,乃用秩俸之物,招致山東大儒,每鄉立學,非聖哲之書不得教授。常以季月召集之,親臨策試。有勤學異等、聰令有聞者,升堂設饌,其餘並坐廊下。有好諍訟、惰業無成者,坐之庭中,設以草具。
 及大成,當舉行賓貢之禮,又於郊外祖道,並以財物資之。於是人皆克勵,風俗大改。有滏陽人焦通,性酗酒,事親禮闕,為從弟所訟。彥光弗之罪,將至州學,令觀於孔子廟。於時廟中有韓伯瑜,母杖不痛,哀母力弱,對母悲泣之像,通遂感悟,既悲且愧,若無自容。彥光訓諭而遣之。後改過勵行,卒為善士。以德化人,皆此類也。



 吏人感悅,略無諍訟。後數歲,卒官,時年六十。贈冀、定、青、瀛四州刺史,謚曰襄。子文謙嗣。



 文謙弘雅有父風,以上柱國嫡子,例授儀同。開皇十五年,拜上州刺史。煬帝即位,轉饒州刺史。歲餘,為鄱陽太守,稱為天下之最。徵拜戶部侍
 郎。遼東之役,領武賁郎將,尋以本官兼檢校太府、衛尉二少卿。明年,又領武賁郎將,為盧龍道軍副。會楊玄感作亂,其弟武賁郎將玄縱先隸文謙,玄感反問未至而玄縱逃走,文謙不之覺,坐是配防桂林而卒,時年五十六。



 少子文讓,初封陽城縣公,後為鷹揚郎將。從衛玄擊楊玄感於東都,力戰而死,贈通議大夫。



 樊叔略樊叔略,陳留人也。父歡,仕魏為南兗州刺史、阿陽侯。屬高氏專權,將謀興復之計,為高氏所誅。叔略時在髫齔,遂被腐刑,給使殿省。身長九尺,志氣不凡,頗為高氏所
 忌。內不自安,遂奔關西。周太祖見而器之,引置左右。尋授都督,襲爵為侯。大塚宰宇文護執政,引為中尉。叔略多計數,曉習時事,護漸委信之,兼督內外。累遷驃騎大將軍、開府儀同三司。護誅後,齊王憲引為園苑監。時憲素有吞關東之志,叔略因事數進兵謀,憲甚奇之。建德五年,從武帝伐齊,叔略部率精銳,每戰身先士卒。以功加上開府,進封清鄉縣公,邑千四百戶。拜汴州刺史,號為明決。宣帝時,於洛陽營建東京,以叔略有巧思,拜營構監,宮室制度,皆叔略所定。功未就而帝崩。尉迥之亂,高祖令叔略鎮大梁。迥將宇文威來寇,叔略擊走之。以
 功拜大將軍,復為汴州刺史。高祖受禪,加位上大將軍,進爵安定郡公。在州數年,甚有聲譽。鄴都俗薄,號曰難化,朝廷以叔略所在著稱,遷相州刺史,政為當時第一。上降璽書褒美之,賜物三百段,粟五百石,班示天下。百姓為之語曰:「智無窮,清鄉公。上下正,樊安定。」徵拜司農卿,吏人莫不流涕,相與立碑頌其德政。自為司農,凡種植,叔略別為條制,皆出人意表。朝廷有疑滯,公卿所未能決者,叔略輒為評理。雖無學術,有所依據,然師心獨見,暗與理合。甚為上所親委,高熲、楊素亦禮遇之。叔略雖為司農,往往參督九卿事。性頗豪侈,每食必方丈,
 備水陸。十四年,從祠太山,行至洛陽,上令錄囚徒。具狀將奏,晨起,至獄門,於馬上暴卒,時年五十九。上悼惜久之,贈亳州刺史,謚曰襄。



 趙軌趙軌,河南洛陽人也。父肅,魏廷尉卿。軌少好學,有行檢。周蔡王引為記室,以清苦聞。遷衛州治中。高祖受禪,轉齊州別駕,有能名。其東鄰有桑,葚落其家,軌遣人悉拾還其主,誡其諸子曰:「吾非以此求名,意者非機杼之物,不願侵人。



 汝等宜以為誡。」在州四年,考績連最。持節使者郃陽公梁子恭狀上,高祖嘉之,賜物三百段,米三百
 石,徵軌入朝。父老相送者各揮涕曰:「別駕在官,水火不與百姓交,是以不敢以壺酒相送。公清若水,請酌一杯水奉餞。」軌受而飲之。既至京師,詔與奇章公牛弘撰定律令格式。時衛王爽為原州總管,上見爽年少,以軌所在有聲,授原州總管司馬。在道夜行,其左右馬逸入田中,暴人禾。軌駐馬待明,訪禾主酬直而去。原州人吏聞之,莫不改操。後數年,遷硤州刺史,撫緝萌夷,甚有恩惠。尋轉壽州總管長史。芍陂舊有五門堰,蕪穢不修。軌於是勸課人吏,更開三十六門,灌田五千餘頃,人賴其利。秩滿歸鄉里,卒於家,時年六十二。子弘安、弘智,並知名。



 房恭懿房恭懿,字慎言,河南洛陽人也。父謨,齊吏部尚書。恭懿性沉深,有局量,達於從政。仕齊,釋褐開府參軍事,歷平恩令、濟陰守,並有能名。會齊亡,不得調。尉迥之亂,恭懿預焉,迥敗,廢於家。開皇初,吏部尚書蘇威薦之,授新豐令,政為三輔之最。上聞而嘉之,賜物四百段,恭懿以所得賜分給窮乏。未幾,復賜米三百石,恭懿又以賑貧人。上聞而止之。時雍州諸縣令每朔朝謁,上見恭懿,必呼至榻前,訪以理人之術。蘇威重薦之,超授澤州司馬,有異績,賜物百段,良馬一匹。遷德州司馬,在職歲餘,盧愷
 復奏恭懿政為天下之最。上甚異之,復賜百段,因謂諸州朝集使曰:「如房恭懿志存體國,愛養我百姓,此乃上天宗廟之所佑助,豈朕寡薄能致之乎!朕即拜為刺史。豈止為一州而已,當今天下模範之,卿等宜師學也。」上又曰:「房恭懿所在之處,百姓視之如父母。朕若置之而不賞,上天宗廟其當責我。內外官人宜知我意。」於是下詔曰:「德州司馬房恭懿出宰百里,毗贊二籓,善政能官,標映倫伍。班條按部,實允僉屬,委以方岳,聲實俱美。可使持節海州諸軍事、海州刺史。」未幾,會國子博士何妥奏恭懿尉迥之黨,不當仕進,威、愷二人朋黨,曲相薦舉。
 上大怒,恭懿竟得罪,配防嶺南。未幾,徵還京師,行至洪州,遇患卒。論者於今冤之。



 公孫景茂公孫景茂,字元蔚,河間阜城人也。容貌魁梧,少好學,博涉經史。在魏,察孝廉,射策甲科,為襄城王長史,兼行參軍。遷太常博士,多所損益,時人稱為書庫。後歷高唐令、大理正,俱有能名。及齊滅,周武帝聞而召見,與語器之,授濟北太守。以母憂去職。



 開皇初,詔徵入朝,訪以政術,拜汝南太守。郡廢,轉曹州司馬。在職數年,以老病乞骸骨,優詔不許。俄遷息州刺史,法令清靜,德化大行。時屬
 平陳之役,征人在路,有疾病者,景茂撤減俸祿,為饘粥湯藥,分賑濟之,賴全活者以千數。



 上聞而嘉之,詔宣告天下。十五年,上幸洛陽,景茂謁見,時年七十七。上命升殿坐,問其年幾。景茂以實對。上哀其老,嗟嘆久之。景茂再拜曰:「呂望八十而遇文王,臣逾七十而逢陛下。」上甚悅,賜物三百段。詔曰:「景茂修身潔己,耆宿不虧,作牧化人,聲績顯著。年終考校,獨為稱首,宜升戎秩,兼進籓條。可上儀同三司,伊州刺史。」明年,以疾徵,吏人號泣於道。及疾愈,復乞骸骨,又不許,轉道州刺史。悉以秩俸買牛犢雞豬,散惠孤弱不自存者。好單騎巡人,家至戶入,閱
 視百姓產業。有修理者,於都會時乃褒揚稱述。如有過惡,隨即訓導,而不彰也。



 由是人行義讓,有無均通,男子相助耕耘,婦人相從紡績。大村或數百戶,皆如一家之務。其後請政事,上優詔聽之。仁壽中,上明公楊紀出使河北,見景茂神力不衰,還以狀奏。於是就拜淄州刺史,賜以馬轝,便道之官。前後歷職,皆有德政,論者稱為良牧。大業初卒官,年八十七。謚曰康。身死之日,諸州人吏赴喪者數千人,或不及葬,皆望墳慟哭,野祭而去。



 辛公義辛公義,隴西狄道人也。祖徽,魏徐州刺史。父季慶,青州
 刺史。公義早孤,為母氏所養,親授書傳。周天和中,選良家子任太學生,以勤苦著稱。武帝時,召入露門學,令受道義。每月集御前令與大儒講論,數被嗟異,時輩慕之。建德初,授宣納中士。從平齊,累遷掌治上士、掃寇將軍。高祖作相,授內史上士,參掌機要。開皇元年,除主客侍郎,攝內史舍人事,賜爵安陽縣男,邑二百戶。每陳使來朝,常奉詔接宴。轉駕部侍郎,使往江陵安輯邊境。七年,使勾檢諸馬牧,所獲十餘萬匹。高祖喜曰:「唯我公義,奉國罄心。」從軍平陳,以功除岷州刺史。土俗畏病,若一人有疾,即合家避之,父子夫妻不相看養,孝義道絕,由是
 病者多死。



 公義患之,欲變其俗。因分遣官人巡檢部內,凡有疾病,皆以床輿來,安置事。



 暑月疫時,病人或至數百,廊悉滿。公義親設一榻,獨坐其間,終日連夕,對之理事。所得秩俸,盡用市藥,為迎醫療之,躬勸其飲食,於是悉差,方召其親戚而諭之曰:「死生由命,不關相著。前汝棄之,所以死耳。今我聚病者,坐臥其間,若言相染,那得不死,病兒復差!汝等勿復信之。」諸病家子孫慚謝而去。後人有遇病者,爭就使君,其家無親屬,因留養之。始相慈愛,此風遂革,合境之內呼為慈母。後遷牟州刺史,下車,先至獄中,因露坐牢側,親自驗問。十餘日間,決
 斷咸盡,方還大。受領新訟,皆不立文案,遣當直佐僚一人,側坐訊問。事若不盡,應須禁者,公義即宿事,終不還閤。人或諫之曰:「此事有程,使君何自苦也!」



 答曰:「刺史無德可以導人,尚令百姓系於囹圄,豈有禁人在獄而心自安乎?」罪人聞之,咸自款服。後有欲諍訟者,其鄉閭父老遽相曉曰:「此蓋小事,何忍勤勞使君。」訟者多兩讓而止。時山東霖雨,自陳、汝至於滄海,皆苦水災。境內犬牙,獨無所損。山出黃銀,獲之以獻。詔水部郎婁崱就公義禱焉。乃聞空中有金石絲竹之響。仁壽元年,追充揚州道黜陟大使。豫章王暕恐其部內官僚犯法,未入
 州境,預令屬公義。公義答曰:「奉詔不敢有私。」及至揚州,皆無所縱舍,暕銜之。及煬帝即位,揚州長史王弘入為黃門侍郎,因言公義之短,竟去官。吏人守闕訴冤,相繼不絕。後數歲,帝悟,除內史侍郎。丁母憂。未幾,起為司隸大夫,檢校右御衛武賁郎將。從征至柳城郡卒,時年六十二。子融。



 柳儉郭絢敬肅柳儉,字道約,河東解人也。祖元璋,魏司州大中正、相華二州刺史。父裕,周聞喜令。儉有局量,立行清苦,為州里所敬,雖至親暱,無敢狎侮。周代歷宣納上士、畿伯大夫。
 及高祖受禪,擢拜水部侍郎,封率道縣伯。未幾,出為廣漢太守,甚有能名。俄而郡廢。時高祖初有天下,勵精思政,妙簡良能,出為牧宰,以儉仁明著稱,擢拜蓬州刺史。獄訟者庭遣,不為文書,約束佐史,從容而已。獄無系囚。



 蜀王秀時鎮益州,列上其事,遷邛州刺史。在職十餘年,萌夷悅服。蜀王秀之得罪也,儉坐與交通,免職。及還鄉里,乘敝車羸馬,妻子衣食不贍,見者咸嘆服焉。



 煬帝嗣位,征之。於時以功臣任職,牧州領郡者,並帶戎資,唯儉起自良吏。帝嘉其績,用特授朝散大夫,拜弘化太守,賜物一百段而遣之。儉清節逾勵。大業五年入朝,郡國畢集,
 帝謂納言蘇威、吏部尚書牛弘曰:「其中清名天下第一者為誰?」



 威等以儉對。帝又問其次,威以涿郡丞郭絢、潁川郡丞敬肅等二人對。帝賜儉帛二百匹,絢、肅各一百匹。令天下朝集使送至郡邸,以旌異焉。論者美之。及大業末,盜賊蜂起,數被攻逼。儉撫結人夷,卒無離叛,竟以保全。及義兵至長安,尊立恭帝,儉與留守李粲縞素於州,南向慟哭。既而歸京師,相國賜儉物三百段,就拜上大將軍。歲餘,卒於家,時年八十九。



 郭絢,河東安邑人也。家素寒微。初為尚書令史,後以軍功拜儀同,歷數州司馬長史,皆有能名。大業初,刑部尚
 書宇文弼巡省河北,引絢為副。煬帝將有事於遼東,以涿郡為沖要,訪可任者。聞絢有幹局,拜涿郡丞,吏人悅服。數載,遷為通守,兼領留守。及山東盜賊起,絢逐捕之,多所克獲。時諸郡無復完者,唯涿郡獨全。後將兵擊竇建德於河間,戰死,人吏哭之,數月不息。



 敬肅,字弘儉,河東蒲阪人也。少以貞介知名,釋褐州主簿。開皇初,為安陵令,有能名,擢拜秦州司馬,轉豳州長史。仁壽中,為衛州司馬,俱有異績。煬帝嗣位,遷潁川郡丞。大業五年,朝東都,帝令司隸大夫薛道衡為天下群官之狀。道衡狀稱肅曰:「心如鐵石,老而彌篤。」時左翊衛
 大將軍宇文述當途用事,其邑在潁川,每有書屬肅。肅未嘗開封,輒令使者持去。述賓客有放縱者,以法繩之,無所寬貸。由是述銜之。八年,朝於涿郡,帝以其年老有治名,將擢為太守者數矣,輒為述所毀,不行。大業末,乞骸骨,優詔許之。去官之日,家無餘財。歲餘,終於家,時年八十。



 劉曠劉曠,不知何許人也。性謹厚,每以誠恕應物。開皇初,為平鄉令,單騎之官。



 人有諍訟者,輒丁寧曉以義理,不加繩劾,各自引咎而去。所得俸祿,賑施窮乏。



 百姓感其德
 化,更相篤勵,曰:「有君如此,何得為非!」在職七年,風教大洽,獄中無系囚,爭訟絕息,囹圄盡皆生草,庭可張羅。及去官,吏人無少長,號泣於路,將送數百里不絕。遷為臨潁令,清名善政,為天下第一。尚書左僕射高熲言其狀,上召之,及引見,勞之曰:「天下縣令固多矣,卿能獨異於眾,良足美也!」



 顧謂侍臣曰:「若不殊獎,何以為勸!」於是下優詔,擢拜莒州刺史。



 王伽王伽,河間章武人也。開皇末,為齊州行參軍,初無足稱。後被州使送流囚李參等七十餘人詣京師。時制,流人
 並枷鎖傳送。伽行次滎陽,哀其辛苦,悉呼而謂之曰:「卿輩既犯國刑,虧損名教,身嬰縲紲,此其職也。今復重勞援卒,民獨不愧於心哉!」參等辭謝。伽曰:「汝等雖犯憲法,枷鎖亦大辛苦。吾欲與汝等脫去,行至京師總集,能不違期不?」皆拜謝曰:「必不敢違。」伽於是悉脫其枷,停援卒,與期曰:「某日當至京師,如致前卻,吾當為汝受死。」舍之而去。流人咸悅,依期而至,一無離叛。上聞而驚異之,召見與語,稱善久之。於是悉召流人,並令攜負妻子俱入,賜宴於殿庭而赦之。乃下詔曰:「凡在有生,含靈稟性,咸知好惡,並識是非。若臨以至誠,明加勸導,則俗必從化,
 人皆遷善。往以海內亂離,德教廢絕,官人無慈愛之心,兆庶懷奸詐之意,所以獄訟不息,澆薄難治。朕受命上天,安養萬姓,思遵聖法,以德化人,朝夕孜孜,意在於此。而伽深識朕意,誠心宣導。



 參等感悟,自赴憲司。明是率土之人非為難教,良是官人不加曉示,致令陷罪,無由自新。若使官盡王伽之儔,人皆李參之輩,刑厝不用,其何遠哉!」於是擢伽為雍令,政有能名。



 魏德深魏德深,本巨鹿人也。祖沖,仕周為刑部大夫、建州刺史,因家弘農。父毗,鬱林令。德深初為文帝挽郎,後歷馮翊
 書佐、武陽司戶書佐,以能遷貴鄉長。為政清凈,不嚴而治。會與遼東之役,征稅百端,使人往來,責成郡縣。於時王綱弛紊,吏多贓賄,所在征斂,下不堪命。唯德深一縣,有無相通,不竭其力,所求皆給,百姓不擾,稱為大治。於時盜賊群起,武陽諸城多被淪陷,唯貴鄉獨全。郡丞元寶藏受詔逐捕盜賊,每戰不利,則器械必盡,輒徵發於人,動以軍法從事,如此者數矣。其鄰城營造,皆聚於事,吏人遞相督責,晝夜喧囂,猶不能濟。德深各問其所欲任,隨便修營,官府寂然,恆若無事。唯約束長吏,所修不須過勝餘縣,使百姓勞苦。然在下各自竭心,常為諸
 縣之最。尋轉館陶長,貴鄉吏人聞之,相與言及其事,皆歔欷流涕,語不成聲。及將赴任,傾城送之,號泣之聲,道路不絕。既至館陶,闔境老幼皆如見其父母。有猾人員外郎趙君實,與郡丞元寶藏深相交結,前後令長未有不受其指麾者。自德深至縣,君實屏處於室,未嘗輒敢出門。逃竄之徒,歸來如市。貴鄉父老冒涉艱險,詣闕請留德深,有詔許之。館陶父老復詣郡相訟,以貴鄉文書為詐。郡不能決。會持節使者韋霽、杜整等至,兩縣詣使訟之,乃斷從貴鄉。貴鄉吏人歌呼滿道,互相稱慶。館陶眾庶合境悲哭,因而居住者數百家。寶藏深害其能。會
 越王侗徵兵於郡,寶藏遂令德深率兵千人赴東都。俄而寶藏以武陽歸李密。德深所領,皆武陽人也,以本土從賊,念其親戚,輒出都門東向慟哭而反。



 人或謂之曰:「李密兵馬近在金墉,去此二十餘里。汝必欲歸,誰能相禁,何為自苦如此!」其人皆垂泣曰:「我與魏明府同來,不忍棄去,豈以道路艱難乎!」其得人心如此。後與賊戰,沒於陣,貴鄉、館陶人庶至今懷之。



 時有櫟陽令渤海高世衡、蕭令彭城劉高、城皋令弘農劉熾,俱有恩惠。大業之末,長吏多贓污,衡、高及熾清節逾厲,風教大洽,獄無系囚,為吏人所稱。



 史臣曰:古語云,善為水者,引之使平,善化人者,撫之使靜。水平則無損於堤防,人靜則不犯於憲章。然則易俗移風,服教從義,不資於明察,必藉於循良者也。彥光等皆內懷直道,至誠待物,故得所居而化,所去見思。至於景茂之遏惡揚善,公義之撫視疾病,劉曠之化行所部,德深之愛結人心,雖信臣、杜詩、鄭渾、硃邑,不能繼也。《詩》云:「愷悌君子,人之父母。」豈徒言哉!恭懿所在尤異,屢簡帝心,追既往之一眚,遂流亡於道路,惜乎!柳儉去官,妻子不贍,趙軌秩滿,酌水餞離,清矣!



\end{pinyinscope}