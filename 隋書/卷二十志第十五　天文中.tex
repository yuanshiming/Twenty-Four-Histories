\article{卷二十志第十五 天文中}

\begin{pinyinscope}

 二十八
 舍東方:角二星,為天闕,其間天門也,其內天庭也。故黃道經其中,七曜之所行也。左角為天田,為理,主刑,其南為太陽道。右角為將,主兵,其北為太陰道。



 蓋天之三門,猶房之四表。其星明大,王道太平,賢者在朝。動搖移徙,王者行。



 亢四星,天子之內朝也。總攝天下奏事,聽訟理獄
 錄功者也。一曰疏廟,主疾疫。星明大,輔納忠,天下寧,人無疾疫。動則多疾。



 氐四星,王者之宿宮,後妃之府,休解之房。前二星適也,後二星妾也。將有徭役之事,氐先動。星明大則臣奉度,人無勞。



 房四星為明堂,天子布政之宮也,亦四輔也。下第一星,上將也;次,次將也;次,次相也;上星,上相也。南二星君位,北二星夫人位。又為四表,中間為天衢之大道,為天闕,黃道之所經也。南間曰陽環,其南曰太陽。北間曰陰間,其北曰太陰。七曜由乎天衢,則天下平和。由陽道則主旱喪,由陰道則主水兵。亦曰天駟,為天馬,主車駕。南星曰左驂,次左服,次右服,次右
 驂。亦曰天廄,又主開閉,為畜藏之所由也。房星明則王者明。驂星大則兵起,星離則人流。又北二小星曰鉤鈐,房之鈐鍵,天之管籥,主閉藏,鍵天心也。王者孝則鉤鈐明。近房,天下同心,遠則天下不和,王者絕後。房鉤鈐間有星及疏圻,則地動河清。



 心三星,天王正位也。中星曰明堂,天子位,為大辰,主天下之賞罰。天下變動,心星見祥。星明大,天下同,暗則主暗。前星為太子,其星不明,太子不得代。



 後星為庶子,後星明,庶子代。心星變黑,大人有憂。直則王失勢,動則國有憂急,角搖則有兵,離則人流。



 尾九星,後宮之場,妃后之府。上第一星,後也;次三星,
 夫人;次星,嬪妾。



 第三星傍一星,名曰神宮,解衣之內室。尾亦為九子。星色欲均明,大小相承,則後宮有敘,多子孫。星微細暗,後有憂疾。疏遠,後失勢。動搖則君臣不和,天下亂。就聚則大水。



 箕四星,亦後宮妃后之府。亦曰天津,一曰天雞。主八風,凡日月宿在箕、東壁、翼、軫者,風起。又主口舌,主客蠻夷胡貉,故蠻胡將動,先表箕焉。星大明直則穀熟,內外有差。就聚細微,天下憂。動則蠻夷有使來。離徙則人流動,不出三日,大風。



 北方:南斗六星,天廟也,丞相太宰之位,主褒賢進士,稟授爵祿,又主兵。



 一曰天機。南二星魁,天梁也。中央二星,
 天相也。北二星杓,天府庭也,亦為天子壽命之期也。將有天子之事,占於斗。斗星盛明,王道平和,爵祿行。芒角動搖,天子愁,兵起移徙,其臣逐。



 牽牛六星,天之關梁,主犧牲事。其北二星,一曰即路,一曰聚火。又曰,上一星主道路,次二星主關梁,次三星主南越。搖動變色則占之。星明大,王道昌,關梁通,牛貴。怒則馬貴。不明失常,穀不登。細則牛賤。中星移上下,牛多死。



 小星亡,牛多疫。又曰,牽牛星動為牛災。



 須女四星,天之少府也。須,賤妾之稱,婦職之卑者也,主布帛裁制嫁娶。星明,天下豐,女功昌,國充富。小暗則國藏虛。動則有嫁娶出納裁制之事。



 虛
 二星,塚宰之官也。主北方,主邑居廟常祭祀祝禱事,又主死喪哭泣。



 危三星,主天府天庫架屋,餘同虛占。星不明,客有誅。動則王者作宮殿,有土功。墳墓四星,屬危之下,主死喪哭泣,為填墓也。星不明,天下旱。動則有喪。



 營室二星,天子之宮也。一曰玄宮,一曰清廟,又為軍糧之府及土功事。星明國昌,小不明,祠祀鬼神不享,國家多疾。動則有土功,兵出野。離宮六星,天子之別宮,主隱藏休息之所。



 東壁二星,主文章,天下圖書之秘府也,主土功。星明,王者興,道術行,國多君子。星失色,大小不同,王者好武,經士不用,圖書隱。星動則有土功。離徙就聚,為
 田宅事。



 西方:奎十六星,天之武庫也。一曰天豕,亦曰封豕。主以兵禁暴,又主溝瀆。



 西南大星,所謂天豕目,亦曰大將,欲其明。若帝淫佚,政不平,則奎有角。角動則有兵,不出年中,或有溝瀆之事。又曰,奎中星明,水大出。



 婁三星,為天獄,主苑牧犧牲,供給郊祀,亦為興兵聚眾。星明,天下平和,郊祀大享,多子孫。動則有聚眾。星直則有執主之命者。就聚,國不安。



 胃三星,天之廚藏,主倉廩五穀府也。明則和平倉實,動則有輸運事,就聚則穀貴人流。



 昴七星,天之耳目也,主西方,主獄事。又為旄頭,胡星也。又主喪。昴
 畢間為天街,天子出,旄頭罕畢以前驅,此其義也。黃道之所經也。昴明則天下牢獄平。



 昴六星皆明,與大星等,大水。七星黃,兵大起。一星亡,為兵喪。搖動,有大臣下獄,及白衣之會。大而數盡動,若跳躍者,胡兵大起。一星獨跳躍,餘不動者,胡欲犯邊境也。



 畢八星,主邊兵,主弋獵。其大星曰天高,一曰邊將,主四夷之尉也。星明大則遠夷來貢,天下安。失色則邊亂。一星亡,為兵喪。動搖,邊城兵起,有讒臣。



 離徙,天下獄亂。就聚,法令酷。附耳一星在畢下,主聽得失,伺愆邪,察不祥。



 星盛則中國微,有盜賊,邊候驚,外國反,鬥兵連年。若移動,佞讒行,兵大起,邊
 尤甚。月入畢,多雨。



 觜觿三星,為三軍之候,行軍之藏府,主葆旅,收斂萬物。明則軍儲盈,將得勢。動而明,盜賊群行,葆旅起。動移,將有逐者。



 參十星,一曰參伐,一曰大辰,一曰天市,一曰鈇鉞,主斬刈。又為天獄,主殺伐。又主權衡,所以平理也。又主邊城,為九譯,故不欲其動也。參,白獸之體。



 其中三星橫列,三將也。東北曰左肩,主左將。西北曰右肩,主右將。東南曰左足,主後將軍。西南曰右足,主偏將軍。故《黃帝占》參應七將。中央三小星曰伐,天之都尉也,主胡、鮮卑、戎狄之國,故不欲明。七將皆明大,天下兵精也。王道缺則芒角張。伐星明與參等,大臣皆謀,
 兵起。參星失色,軍散。參芒角動搖,邊候有急,天下兵起。又曰,有斬伐之事。參星移,客伐主。參左足入玉井中,兵大起,秦大水,若有喪,山石為怪。參星差戾,王臣貳。



 南方:東井八星,天之南門,黃道所經,天之亭候。主水衡事,法令所取平也。



 王者用法平,則井星明而端列。鉞一星,附井之前,主伺淫奢而斬之,故不欲其明。



 明與井齊,則用鉞,大臣有斬者,以欲殺也。月宿井,有風雨。



 輿鬼五星,天目也,主視,明察奸謀。東北星主積馬,東南星主積兵,西南星主積布帛,西北星主積金玉,隨變占之。中央為積尸,主死喪祠祀。一曰鈇質,主誅斬。鬼星明大,穀成。
 不明,人散。動而光,上賦斂重,徭役多。星徙,人愁,政令急。鬼質欲其忽忽不明則安,明則兵起,大臣誅。



 柳八星,天之廚宰也,主尚食,和滋味,又主雷雨,若女主驕奢。一曰天相,一曰天庫,一曰注,又主木功。星明,大臣重慎,國安,廚食具。注舉首,王命興,輔佐出。星直,天下謀伐其主。星就聚,兵滿國門。



 七星七星,一名天都,主衣裳文繡,又主急兵,守盜賊,故欲明。星明,王道昌,暗則賢良不處,天下空,天子疾。動則兵起,離則易政。



 張六星,主珍寶,宗廟所用及衣服,又主天廚,飲食賞賚之事。星明則王者行五禮,得天之中。動則賞賚,離徙天下有逆人,就聚有兵。



 翼
 二十二星,天之樂府,主俳倡戲樂,又主夷狄遠客,負海之賓。星明大,禮樂興,四夷賓。動則蠻夷使來,離徙則天子舉兵。



 軫四星,主塚宰輔臣也,主車騎,主載任。有軍出入,皆占於軫。又主風,主死喪。軫星明,則車駕備。動則車騎用。離徙,天子憂。就聚,兵大起。轄星,傅軫兩傍,主王侯。左轄為王者同姓,右轄為異姓。星明,兵大起。遠軫兇。軫轄舉,南蠻侵。車無轄,國主憂。長沙一星,在軫之中,主壽命。明則主壽長,子孫昌。



 右四方二十八宿並輔官一百八十二星。



 星官在二十八宿之外者
 庫樓十星,其六大星為庫,南四星為樓,在角南。一曰天庫,兵車之府也。旁十五星,三三而聚者,柱也。中央四小星,衡也。主陳兵。又曰,天庫空則兵四合。



 東北二星曰陽門,主守隘塞也。南門二星在庫樓南,天之外門也。主守兵。平星二星,在庫樓北,平天下之法獄事,廷尉之象也。天門二星,在平星北。



 亢南七星曰折威,主斬殺。頓頑二星,在折威東南,主考囚情狀,察詐偽也。



 騎官二十七星,在氐南,若天子武賁,主宿衛。東端一星,騎陳將軍,騎將也。



 南三星車騎,車騎之將也。陣車三星,在騎官東北,革車也。



 積卒十二星,在房心南,主為衛也。他星守之,近臣
 誅。從官二星,在積卒西北。



 龜五星,在尾南,主卜,以占吉兇。傅說一星,在尾後。傅說主章祝巫官也。



 章,請號之聲也。主王後之內祭祀,以祈子孫,廣求胤嗣。《詩》云:「克禋克祀,以弗無子。」此之象也。星明大,王者多子孫。魚一星,在尾後河中,主陰事,知雲雨之期也。星不明,則魚多亡,若魚少。動搖則大水暴出。出漢中,則大魚多死。



 杵三星,在箕南,杵給庖舂。客星入杵臼,天下有急。糠一星,在箕舌前,杵西北。



 鱉十四星,在南斗南。鱉為水蟲,歸太陰。有星守之,白衣會,主有水令。農丈人一星,在南斗西南,老農主稼穡也。狗二星,在南斗魁前,主吠守。



 天田九星,在牛
 南。羅堰九星,在牽牛東,岠馬也,以壅畜水潦,灌溉溝渠也。



 九坎九星,在牽牛南。坎,溝渠也,所以導達泉源,疏瀉盈溢,通溝洫也。九坎間十星曰天池,一曰三池,一曰天海,主灌溉事。九坎東列星:北一星曰齊,齊北二星曰趙,趙北一星曰鄭,鄭北一星曰越,越東二星曰周,周東南北列二星曰秦,秦南二星曰代,代西一星曰晉,晉北一星曰韓,韓北一星曰魏,魏西一星曰楚,楚南一星曰燕。其星有變,各以其國。秦、代東三星南北列,曰離瑜。離圭衣也,瑜玉飾,皆婦人之服星也。



 虛南二星曰哭,哭東二星曰泣,泣哭皆近墳墓。泣南十三星,曰天壘城,如貫索
 狀,主北夷丁零、匈奴。敗臼四星,在虛危南,知兇災。他星守之,饑兵起。



 危南二星曰蓋屋,主治宮室之官也。虛梁四星,在蓋屋南,主園陵寢廟。非人所處,故曰虛梁。



 室南六星曰雷電。室西南二星曰土功吏,主司過度。



 壁南二星曰土公,土公西南五星曰礔礪,礔礪南四星曰雲雨,皆在壘壁北。



 羽林四十五星,在營室南。一曰天軍,主軍騎,又主翼王也。壘壁陣十二星,在羽林北,羽林之垣壘也,主軍位,為營壅也。五星有在天軍中者,皆為兵起,熒惑、太白、辰星尤甚。北落師門一星,在羽林南。北者,宿在北方也。落,天之蕃落也。師,眾也。師門猶軍門也。長安城
 北門曰北落門,以象北也。主非常,以候兵。有星守之,虜入塞中,兵起。北落西北有十星,曰天錢。北落西南一星,曰天綱,主武帳。北落東南九星,曰八魁,主張禽獸。客星入之,多盜賊。八魁西北三星曰鈇質,一曰鈇鉞。有星入之,皆為大臣誅。



 奎南七星曰外屏。外屏南七星曰天溷,廁也。屏所以障之也。天溷南一星曰土司空,主水土之事故,又知禍殃也。客星入之,多土功,天下大疾。



 婁東五星曰左更,山虞也,主澤藪竹木之屬,亦主仁智。婁西五星曰右更,牧師也,主養牛馬之屬,亦主禮義。二更,秦爵名也。天倉六星,在婁南,倉穀所藏也。星黃而大,歲熟。西
 南四星曰天庾,積廚粟之所也。



 天囷十三星在胃南。囷,倉廩之屬也,主給御糧也。星見則囷倉實,不見即虛。



 天廩四星在昴南,一曰天」W,主畜黍稷,以供饗祀,《春秋》所謂御廩,此之象也。天苑十六星,在昴畢南,天子之苑囿,養禽獸之所也,主馬牛羊。星明則牛馬盈,希則死。苑西六星曰芻槁,以供牛馬之食也。一曰天積,天子之藏府也。



 星盛則歲豐穰,希則貨財散。苑南十三星曰天園,植果菜之所也。



 畢附耳南八星,曰天節,主使臣之所持者也。天節下九星,曰九州殊口,曉方俗之官,通重譯者也。畢柄西五星曰天陰。



 參旗九星在參西,一曰天旗,一曰
 天弓,主司弓弩之張,候變禦難。玉井四星,在參左足下,主水漿,以給廚。西南九星曰九游,天子之旗也。玉井東南四星曰軍井,行軍之井也。軍井未達,將不言渴,名取此也。屏二星在玉井南,屏為屏風。



 客星入之,四足蟲大疾。天廁四星,在屏東,溷也,主觀天下疾病。天矢一星在廁南,色黃則吉,他色皆兇。軍市十三星,在參東南,天軍貿易之市,使有無通也。



 野雞一星,主變怪,在軍市中。軍市西南二星曰丈人,丈人東二星曰子,子東二星曰孫。



 東井西南四星曰水府,主水之官也。東井南垣之東四星,曰四瀆,江、河、淮、濟之精也。狼一星,在東井東南。狼為
 野將,主侵掠。色有常,不欲變動也。角而變色動搖,盜賊萌,胡兵起,人相食。躁則人主不靜,不居其宮,馳騁天下。北七星曰天狗,主守財。弧九星在狼東南,天弓也,主備盜賊,常向於狼。弧矢動移,不如常者,多盜賊,胡兵大起。狼弧張,害及胡,天下乖亂。又曰,天弓張,天下盡兵,主與臣相謀。弧南六星為天社。昔共工氏之子句龍,能平水土,故祀以配社,其精為星。老人一星在弧南,一曰南極。常以秋分之旦見於丙,春分之夕而沒於丁。



 見則化平,主壽昌,亡則君危代天。常以秋分候之南郊。



 柳南六星曰外廚。廚南一星曰天紀,主禽獸之齒。



 稷五星在七星
 南。稷,農正也。取乎百穀之長,以為號也。



 張南十四星曰天廟,天子之祖廟也。客星守之,祠官有憂。



 翼南五星曰東區,蠻夷星也。



 軫南三十二星曰器府,樂器之府也。青丘七星在軫東南,蠻夷之國號也。青丘西四星曰土司空,主界域,亦曰司徒。土司空北二星曰軍門,主營候豹尾威旗。



 自攝提至此,大凡二百五十四官,一千二百八十三星。並二十八宿輔官,名曰經星常宿。遠近有度,小大有差。茍或失常,實表災異。



 天漢,起東方,經尾箕之間,謂之漢津。乃分為二道,其南
 經傅說、魚、天籥、天弁、河鼓,其北經龜,貫箕下,次絡南斗魁、左旗,至天津下而合南道。乃西南行,又分夾匏瓜,絡人星、杵、造父、騰蛇、王良、傅路、閣道北端、太陵、天船、卷舌而南行,絡五車,經北河之南,入東井水位而東南行,絡南河、闕丘、天狗、天紀、天稷,在七星南而沒。



 天占《鴻範五行傳》曰:「清而明者,天之體也,天忽變色,是謂易常。天裂,陽不足,是謂臣強,下將害上,國後分裂,其下之主當之。天開見光,流血滂滂。天裂見人,兵起國亡。天鳴有聲,至尊憂且驚。皆亂國之所生也。」



 漢惠帝二年,天開東北,長三十餘丈,廣十餘丈。後有呂氏變亂。



 晉惠帝太安二年,天中裂。穆帝升平五年,又裂,廣數丈,並有聲如雷。其後皆有兵革之應。



 七曜日循黃道東行,一日一夜行一度,三百六十五日有奇而周天。行東陸謂之春,行南陸謂之夏,行西陸謂之秋,行北陸謂之冬。行以成陰陽寒暑之節。是故《傳》云:「日為太陽之精,主生養恩德,人君之象也。」又人君有瑕,必露其慝,以告示焉。故日月行有道之國則光明,人君吉昌,
 百姓安寧。日變色,有軍軍破,無軍喪侯王。其君無德,其臣亂國,則日赤無光。日失色,所臨之國不昌。日晝昏,行人無影,到暮不止者,上刑急,下人不聊生,不出一年,有大水。日晝昏,烏鳥群鳴,國失政。日中烏見,主不明,為政亂,國有白衣會。日中有黑子、黑氣、黑雲,乍三乍五,臣廢其主。日食,陰侵陽,臣掩君之象,有亡國,有死君,有大水。日食見星,有殺君,天下分裂。王者修德以禳之。



 月者,陰之精也。其形圓,其質清,日光照之,則見其明。日光所不照,則謂之魄。故月望之日,日月相望,人居其間,盡睹其明,故形圓也。二弦之日,日照其側,人觀其傍,故
 半明半魄也。晦朔之日,日照其表,人在其里,故不見也。其行有遲疾。其極遲則日行十二度強,極疾則日行十四度半強。遲則漸疾,疾極漸遲,二十七日半強而遲疾一終矣。又月行之道,斜帶黃道。十三日有奇在黃道表,又十三日有奇在黃道里。表裏極遠者,去黃道六度。二十七日有奇,陰陽一終。張衡云:「對日之沖,其大如日,日光不照,謂之暗虛。暗虛逢月則月食,值星則星亡。」



 今歷家月望行黃道,則值暗虛矣。值暗虛有表裏深淺,故食有南北多少。月為太陰之精,以之配日,女主之象也。以之比德,刑罰之義。列之朝廷,諸侯大臣之類。



 故君明則
 月行依度,臣執權則月行失道。大臣用事,兵刑失理,則月行乍南乍北。



 女主外戚擅權,則或進或退。月變色,將有殃。月晝明,奸邪並作,君臣爭明,女主失行,陰國兵強,中國饑,天下謀僭。數月重見,國以亂亡。



 歲星曰東方春木。於人五常,仁也;五事,貌也。仁虧貌失,逆春令,傷木氣,則罰見歲星。歲星盈縮,以其舍命國。其所居久,其國有德厚,五穀豐昌,不可伐。



 其對為沖,歲乃有殃。歲星安靜中度,吉。盈縮失次,其國有變,不可舉事用兵。



 又曰,人主出象也。色欲明光潤澤,德合同。又曰,進退如度,奸邪息;變色亂行,主無福。又主福,主大司農,主
 齊、吳,主司天下諸侯人君之過,主歲五穀。赤而角,其國昌;赤黃而沉,其野大穰。



 熒惑曰南方夏火。禮也,視也。禮虧視失,逆夏令,傷火氣,罰見熒惑。熒惑法使行無常,出則有兵,入則兵散。以舍命國,為亂,為賊,為疾,為喪,為饑,為兵,居國受殃。環繞勾已,芒角動搖變色,乍前乍後,乍左乍右,其殃愈甚。其南丈夫、北女子喪。周旋止息,乃為死喪,寇亂其野,亡地。其失行而速,兵聚其下,順之戰勝。又曰,熒惑主大鴻臚,主死喪,主司空,又為司馬,主楚、吳、越以南,又司天下群臣之過,司驕奢亡亂妖孽,主歲成敗。又曰,熒惑不動,兵不
 戰,有誅將。其出色赤怒,逆行成鉤已,戰兇,有圍軍。鉤已,有芒角如鋒刃,人主無出宮,下有伏兵。芒大則人民怒,君子遑遑,小人浪浪,不有亂臣,則有大喪,人欺吏,吏欺王。又為外則兵,內則理政,為天子之理也。故曰,雖有明天子,必視熒惑所在。其入守犯太微、軒轅、營室、房、心,主命惡之。



 填星曰中央季夏土。信也,思心也。仁義禮智,以信為主,貌言視聽,以心為政,故四星皆失,填乃為之動。動而盈,侯王不寧。縮,有軍不復。所居之宿,國吉,得地及女子,有福,不可伐。去之,失地,若有女憂。居宿久,國福厚,易則薄。
 失次而上二三宿曰盈,有主命不成,不乃大水。失次而下曰縮,後戚,其歲不復,不乃天裂,若地動。一曰,填為黃帝之德,女主之象,主德厚,安危存亡之機,司天下女主之過。又曰,天子之星也。天子失信,則填星大動。



 太白曰西方秋金。義也,言也。義虧言失,逆秋令,傷金氣,罰見太白。太白進退以候兵,高埤遲速,靜躁見伏,用兵皆象之,吉。其出西方,失行,夷狄敗;出東方,失行,中國敗。未盡期日,過參天,病其對國。若經天,天下革,人更王,是謂亂紀,人民流亡。晝與日爭明,強國弱,小國強,女主昌。又曰,太白大臣,其號上公也,大司馬位謹候此。



 辰星曰北方冬水。智也,聽也。智虧聽失,逆冬令,傷水氣,罰見辰星。辰星見,主刑,主廷尉,主燕、趙,又為燕、趙、代以北,宰相之象,亦為殺伐之氣,戰鬥之象。又曰,軍於野,辰星為偏將之象,無軍為刑事。和陰陽,應其時。不和,出失其時,寒暑失其節,邦當大饑。當出不出,是謂擊卒,兵大起。在於房心間,地動。亦曰,辰星出入躁疾,常主夷狄。又曰,蠻夷出星,亦主刑法之得失。色黃而小,地大動。



 凡五星有色,大小不同,各依其行而順時應節。色變有類。凡青皆比參左肩,赤比心大星,黃比參右肩,白比狼星,黑比奎大星。不失本色,而應其四時者,吉;色害其行,
 兇。



 凡五星所出所行所直之辰,其國為得位者,歲星以德,熒惑有禮,填星有福,太白兵強,辰星陰陽和。所行所直之辰,順其色而有角者勝,其色害者敗。居實,有德也。居虛,無德也。色勝位,行勝色,行得盡勝之。營室為清廟,歲星廟也。



 心為明堂,熒惑廟也。南斗為文太室,填星廟也。亢為疏廟,太白廟也。七星為員官,辰星廟也。五星行至其廟,謹候其命。



 凡五星盈縮失位,其精降於地為人。歲星降為貴臣;熒惑降為童兒,歌謠嬉戲;填星降為老人婦女;太白降為
 壯夫,處於林麓;辰星降為婦人。吉兇之應,隨其象告。



 凡五星,木與土合,為內亂、饑;與水合,為變謀而更事;與火合,為饑,為旱;與金合,為白衣之會,合鬥,國有內亂,野有破軍,為水。太白在南,歲星在北,名曰牡年,穀大熟。太白在北,歲星在南,年或有或無。火與金合,為爍為喪,不可舉事用兵。從軍為軍憂,離之軍卻。出太白陰,分宅,出其陽,偏將戰。與土合,為憂,主孽。與水合,為北軍,用兵舉事大敗。一曰,火與水合為焠,不可舉事用兵。土與水合,為壅沮,不可舉事用兵,有覆軍下師。一曰,為變謀更事,必為旱。與金合,為疾,為白衣會,為內兵,國亡地。與木合,
 國饑。水與金合,為變謀,為兵憂。入太白中而上出,破軍殺將,客勝。下出,客亡地,視旗所指,以命破軍。環繞太白,若與鬥,大戰,客勝。



 凡木、火、土、金與水斗,皆為戰,兵不在外,皆為內亂。



 凡同舍為合,相陵為斗。二星相近,其殃大,相遠無傷,七寸以內必之。



 凡月蝕五星,其國亡。歲以饑,熒惑以亂,填以殺,太白以強國戰,辰以女亂。



 凡五星入月,其野有逐相。太白,將僇。



 凡五星所聚,其國王,天下從。歲以義從,熒惑以禮從,填
 以重從,太白以兵從,辰以法,各以其事致天下也。三星若合,是謂驚立絕行,其國外內有兵,天喪人民,改立侯王。四星若合,是謂太陽,其國兵喪並起,君子憂,小人流。五星若合,是謂易行,有德受慶,改立王者,奄有四方,子孫蕃昌;亡德受殃,離其國家,滅其宗廟,百姓離去,被滿四方。五星皆大,其事亦大;皆小,事亦小。



 凡五星色,其圜白,為喪,為旱;赤中不平,為兵,為憂;青為水;黑為疾疫,為多死;黃為吉。皆角,赤,犯我城;黃,地之爭;白,哭泣聲;青,有兵憂;黑,有水。五星同色,天下偃兵,百姓安寧,歌儛以行,不見災疾,五穀蕃昌。



 凡五星歲政緩則不行,急則過分,逆則占。熒惑,緩則不入,急則不出,違道則占。填,緩則不還,急則過舍,逆則占。太白,緩則不出,急則不入,逆則占。



 辰星,緩則不出,急則不入,非時則占。五星不失行,則年穀豐昌。



 凡五星分天之中,積於東方,中國;積於西方,外國。用兵者利。辰星不出,太白為客;其出,太白為主。出而與太白不相從,及各出一方,為格,野有軍不戰。



 五星為五德之主,其行或入黃道里,或出黃道表,猶月行出有陰陽也。終出入五常,不可以算數求也。其東行曰順,西行曰逆,順則疾,逆則遲,通而率之,終為東行矣。不東不西曰留。
 與日相近而不見,曰伏。伏與日同度曰合。其留行逆順掩合犯法陵變色芒角,凡其所主,皆以時政五常、五官、五事之得失,而見其變。



 木、火、土三星行遲,夜半經天。其初皆與日合度,而後順行漸遲,追日不及,晨見東方。行去日稍遠,朝時近中則留。留經旦過中則逆行。逆行至夕時近中則又留。留而又順,先遲漸速,以至於夕伏西方,乃更與日合。金、水二星,行速而不經天。自始與日合之後,行速而先日,夕見西方。去日前稍遠,夕時欲近南方則漸遲,遲極則留。留而近日,則逆行而合日,在於日後。晨見東方。逆
 極則留,留而後遲。遲極去日稍遠,旦時欲近南方,則速行以追日,晨伏於東方,復與日合。此五星合見、遲速、逆順、留行之大經也。昏旦者,陰陽之大分也。南方者,太陽之位,而天地之經也。七曜行至陽位,當天之經,則虧昃留逆而不居焉。此天之常道也。三星經天,二星不經天,三天兩地之道也。



 凡五星見伏留行,逆順遲速,應歷度者,為得其行,政合於常。違歷錯度,而失路盈縮者,為亂行。亂行則為妖星彗孛,而有亡國革政,兵饑喪亂之禍云。



 古歷五星並順行,秦歷始有金火之逆。又甘、石並時,自
 有差異。漢初測候,乃知五星皆有逆行,其後相承罕能察。至後魏末,清河張子信,學藝博通,尤精歷數。因避葛榮亂,隱於海島中,積三十許年,專以渾儀測候日月五星差變之數,以算步之,始悟日月交道,有表裏遲速,五星見伏,有感召向背。言日行在春分後則遲,秋分後則速。合朔月在日道里則日食,若在日道外,雖交不虧。月望值交則虧,不問表裏。又月行遇木、火、土、金四星,向之則速,背之則遲。五星行四方列宿,各有所好惡。所居遇其好者,則留多行遲,見早。遇其惡者,則留少行速,見遲。



 與常數並差,少者差至五度,多者差至三十許度。其辰
 星之行,見伏尤異。晨應見在雨水後立夏前,夕應見在處暑後霜降前者,並不見。啟蟄、立夏、立秋、霜降四氣之內,晨夕去日前後三十六度內,十八度外,有木、火、土、金一星者見,無者不見。後張胄玄、劉孝孫、劉焯等,依此差度,為定入交食分及五星定見定行,與天密會,皆古人所未得也。



 梁奉朝請祖恆,天監中,受詔集古天官及圖緯舊說,撰《天文錄》三十卷。



 逮周氏克梁,獲庾季才,為太史令,撰《靈臺秘苑》一百二十卷,占驗益備。今略其雜星、瑞星、妖星、客星、流星及雲氣名狀,次之於此云。



 瑞星
 一曰景星,如半月,生於晦朔,助月為明。或曰,星大而中空。或曰,有三星,在赤方氣,與青方氣相連。黃星在赤方氣中,亦名德星。二曰周伯星,黃色煌煌然,所見之國大昌。三曰含譽,光耀似彗,喜則含譽射。



 星雜變一曰星晝見。若星與日並出,名曰嫁女。星與日爭光,武且弱,文且強,女子為王,在邑為喪,在野為兵。又曰,臣有奸心,上不明,臣下從橫,大水浩洋。又曰,星晝見,虹不滅,臣人生明,星奪日光,天下有立王。二曰恆星不見。恆星者,在位人君之類。不見者,象諸侯之背畔,不佐王者奉
 順法度,無君之象也。又曰,恆星不見,主不嚴,法度消。又曰,天子失政,諸侯橫暴。又曰,常星列宿不見,象中國諸侯微滅也。三曰星斗,星斗天下大亂。四曰星搖,星搖人眾將勞。五曰星隕。大星隕下,陽失其位,災害之萌也。又曰,眾星墜,人失其所也。凡星所墜,國易政。又曰,星墜,當其下有戰場,天下亂,期三年。又曰,奔星之所墜,其下有兵,列宿之所墜,滅家邦,眾星之所墜,眾庶亡。又曰,填星墜,海水泆,黃星騁,海水躍。又曰,黃星墜,海水傾。亦曰,賁星墜而勃海決。星隕如雨,天子微,諸侯力政,五伯代興,更為盟主,眾暴寡,大並小。又曰,星辰附離天,猶庶人附
 離王者也。王者失道,綱紀廢,下將畔去。故星畔天而隕,以見其象。國有兵兇,則星墜為鳥獸。天下將亡,則星墜為飛蟲。天下大兵,則星墜為金鐵。天下有水,則星墜為土。國主亡,有兵,則星墜為草木。兵起,國主亡,則星墜為沙。星墜,為人而言者,善惡如其言。又曰,國有大喪,則星墜為龍。



 妖星妖星者,五行之氣,五星之變名,見其方,以為殃災。各以其日五色占,知何國吉兇決矣。行見無道之國,失禮之邦,為兵為饑,水旱死亡之徵也。又曰,凡妖星所出,形狀
 不同,為殃如一。其出不過一年,若三年,必有破國屠城。其君死,天下大亂,兵士亂行,戰死於野,積尸從橫。餘殃不盡,為水旱兵饑疾疫之殃。又曰,凡妖星出見,長大,災深期遠;短小,災淺期近。三尺至五尺,期百日。五尺至一丈,期一年。一丈至三丈,期三年。三丈至五丈,期五年。五丈至十丈,期七年。十丈以上,期九年。審以察之,其災必應。



 彗星,世所謂掃星,本類星,末類彗,小者數寸,長或竟天。見則兵起,大水。



 主掃除,除舊布新。有五色,各依五行本精所主。史臣案,彗體無光,傅日而為光,故夕見則東指,晨見則西指,在日南北,皆隨日光而指。頓挫其芒,或
 長或短,光芒所及則為災。



 又曰,孛星,彗之屬也。偏指曰彗,芒氣四出曰孛。孛者,孛然非常,惡氣之所生也。內不有大亂,則外有大兵,天下合謀,暗蔽不明,有所傷害。晏子曰:「君若不改,孛星將出,彗星何懼乎?」由是言之,災甚於彗。



 歲星之精,流為天棓、天槍、天猾、天沖、國皇、反登。一曰天棓,一名覺星,或曰天格。本類星,末銳,長四丈。主滅兵,主奮爭。又曰,天棓出,其國兇,不可舉事用兵。又曰,期三月,必有破軍拔城。又曰,天棓見,女主用事。其本者為主人。二曰天槍,主捕制。或曰,攙雲如牛,槍雲如馬。或曰,如槍,
 左右銳,長數丈。天攙本類星,未銳,長丈。三曰天猾,主招亂。又曰,人主自恣,逆天暴物,則天猾起。四曰天沖,狀如人,蒼衣赤首,不動。主滅位。又曰,沖星出,臣謀主,武卒發。又曰,天沖抱極泣帝前,血濁霧下,天下冤。五曰國皇。或曰,機星散為國皇。國皇之星,大而赤,類南極老人星也。主滅奸,主內寇難。見則兵起,天下急。或云,去地一二丈,如炬火狀。後客星內亦有國皇,名同而占狀異。六曰反登,主夷分,皆少陽之精,司徒之類,青龍七宿之域。有謀反,若恣虐為害,主失春政者,以出時沖為期。皆主君征也。



 熒惑之精,流為析旦、蚩尤旗、昭明、司危、天攙。一曰析旦,
 或曰昭旦,主弱之符。又曰,析旦橫出,參翟百尺,為相誅滅。二曰蚩尤旗。或曰,旋星散為蚩尤旗。或曰,蚩尤旗,五星盈縮之所生也。狀類彗而後曲,象旗。或曰,四望無雲,獨見赤雲,蚩尤旗也。或曰,蚩尤旗如箕,可長二丈,末有星。又曰,亂國之王,眾邪並積,有云若植雚竹長,黃上白下,名曰蚩尤旗。主誅逆國。又曰,帝將怒,則蚩尤旗出。又曰,虐王反度,則蚩尤旗出。或曰,本類星,而後委曲,其像旗T,可長二三丈。見則王者旗鼓,大行征伐,四方兵大起。不然,國有大喪。三曰昭明者,五星變出於西方,名曰昭明,金之氣也。又曰,赤彗分為昭明。昭明滅光,象如太
 白,七芒,故以為起霸之徵。或曰,機星散為昭明。又曰,西方有星,望之去地可六丈而有光,其類太白,數動,察之中赤,是謂西方之野星,名曰昭明。



 出則兵大起。其出也,下有喪。出南方,則西方之邦失地。或曰,昭明如太白,不行,主起有德。又曰,西方有星,大而白,有角,目下視之,名曰昭明。金之精,出則兵大起。若守房心,國有喪,必有屠城。昭明下則為天狗,所下者大戰流血。



 四曰司危。或曰,機星散為司危。又曰,白彗之氣,分為司危。司危平,以為乖爭之徵。或曰,司危星大,有毛,兩角。又曰,司危星類太白,數動,察之而赤。司危出,強國盈,主擊強侯兵也。又曰,
 司危見則主失法,期八年,豪傑起,天子以不義失國。有聲之臣,行主德也。又曰,司危見,則其下國相殘賊。又曰,司危星出正西,西方之野星,去地可六丈,大而白,類太白。一曰,見,兵起強。又曰,司危出則非,其下有兵沖不利。五曰天攙,其狀白小,數動,是謂攙星,一名斬星。



 天攙主殺罰。又曰,天攙見,女主用事者,其本為主人。又曰,天攙出,其下相攙,為饑為兵,赤地千里,枯骨籍籍。亦曰,天攙出,其國內亂。又曰,太陽之精,赤鳥七宿之域,有謀反,恣虐為害,主失夏政。



 填星之精,流為五殘、六賊、獄漢、大賁、炤星、絀流、茀星、旬
 始、擊咎。



 一曰五殘。或曰,旋星散為五殘。亦曰,蒼彗散為五殘。故為毀敗之徵。或曰,五殘五分。亦曰,一本而五枝也。期九年,奸興。三九二十七,大亂不可禁。又曰,五殘者,五行之變,出於東方,五殘木之氣也。一曰,五縫又曰五殘,星出正東,東方之野星,狀類辰星,可去地六七丈,大而白,主乖亡。或曰,東方有星,望之去地可六丈,大而赤,察之中青。或曰,星表青氣如暈,有毛,其類歲星,是謂東方之野星,名曰五殘。出則兵大起。其出也,下有喪。出北則東方之邦失地。又曰,五殘出,四蕃虛,天子有急兵。或曰,五殘大而赤,數動,察之有青。又曰,五殘出則兵起。二
 曰六賊者,五行之氣,出於南方。或曰,六賊火之氣也。或曰,六賊星形如彗。又曰,南方有星,望之可去地六丈,赤而數動,察之有光,其類熒惑,是謂南方之野星,名曰六賊。出則兵起,其國亂。其出也,下有喪。出東方則南方之邦失地。又曰,六賊星見,出正南,南方之星,去地可六丈,大而赤,數動有光。



 三曰獄漢,一曰咸漢。或曰,權星散為獄漢。又曰,咸漢者,五行之氣,出於北方,水之氣也。獄漢青中赤表,下有三彗從橫,主逐王刺王。又曰,北方有星,望之可去地六丈,大而赤,數動,察之中青黑,其類辰星,是謂北方之野星,名曰咸漢。



 出則兵起,其下有喪。出西
 方則北方之邦失地。又曰,獄漢動,諸侯驚,出則陰橫。



 四曰大賁,主暴沖。五曰炤星,主滅邦。六曰絀流,動天下敖主伏逃。又曰,絀流,主自理,無所逃。七曰茀星,在東南,本有星,末類茀,所當之國,實受其殃。八曰旬始。或曰,樞星散為旬始。或曰,五星盈縮之所生也。亦曰,旬始妖氣。又曰,旬始蚩尤也。又曰,旬始出於北斗旁,狀如雄雞。其怒青黑,象伏鱉。又曰,黃彗分為旬始。旬始者,今起也。狀如雄雞,土含陽,以交白接,精象雞,故以為立主之題。期十年,聖人起代。又曰,旬始主爭兵,主亂,主招橫。又曰,旬始照,其下必有滅王。五奸爭作,暴骨積骸,以子續食。見則
 臣亂兵作,諸侯為虐。又曰,常以戊戌日,視五車及天軍天庫中有奇怪,曰旬始。狀如鳥有喙,而見者則兵大起,攻戰當其首者破死。又曰,出見北斗,聖人受命,天子壽,王者有福。九曰擊咎,出,臣下主。一曰,臣禁主,主大兵。又曰,土精,斗七星之域,以長四方,司空之位,有謀反恣虐者,占如上。



 太白之精,散為天杵、天柎、伏靈、大敗、司奸、天狗、天殘、卒起。一曰天杵,主䍧羊。二曰天柎,主擊殃。三曰伏靈,主領讒。伏靈出,天下亂復人。四曰大敗,主斗沖。或曰,大敗出,擊咎謀。五曰司奸,主見妖。六曰天狗。亦曰,五星氣合之
 變,出西南,金火氣合,名曰天狗。或曰,天狗星有毛,旁有短彗,下有如狗形者,主征兵,主討賊。亦曰,天狗流,五將鬥。又曰,西北方有星,長三丈,而出水金氣交,名曰天狗。亦曰,西北三星,大而白,名曰天狗。見則大兵起,天下饑,人相食。又曰,天狗所下之處,必有大戰,破軍殺將,伏尸流血,天狗食之。



 皆期一年,中二年,遠三年,各以其所下之國,以占吉兇。後流星內天狗,名同,占狀小異。七曰天殘,主貪殘。八曰卒起。卒起見,禍無時,諸變有萌,臣運柄。



 又曰,少陰之精,大司馬之類,白獸七宿之域,有謀反,若恣虐為害,主失秋政者,期如上占,禍亦應之。



 辰星之精,散為枉矢、破女、拂樞、滅寶、繞廷、驚理、大奮祀。一曰枉矢。



 或曰,填星之變為枉矢。又曰,機星散為枉矢。亦曰,枉矢,五星盈縮之所生也,弓弩之像也。類大流星,色蒼黑,蛇行,望之如有毛目,長數匹,著天。主反萌,主射愚。又曰,黑彗分為枉矢。枉矢者,射是也。枉矢見,謀反之兵合,射所誅,亦為以亂伐亂。又曰,人君暴專己,則有枉矢動。亦曰,枉矢類流星,望之有尾目,長可一匹布,皎皎著天。見則大兵起,大將出,弓弩用,期三年。曰,枉矢所觸,天下之所伐,射滅之象也。二曰破女。破女若見,君臣皆誅,主勝之符。三曰拂樞。



 拂樞動亂,駭擾無調時。又曰,拂
 樞主制時。四曰滅寶。滅寶起,相得之。又曰,滅寶主伐之。五曰繞廷。繞廷主亂孳。六曰驚理。驚理主相署。七曰大奮祀。大奮祀主招邪。或曰,大奮祀出,主安之。太陰之精,玄武七宿之域,有謀反,若恣虐為害,主失冬政者,期如上占,禍亦應之。又曰,五精潛潭,皆以類逆所犯,行失時指,下臣承類者,乘而害之,皆滅亡之徵也。入天子宿,主滅,諸侯五百謀。



 雜妖一曰天鋒。天鋒,彗象矛鋒者也,主從橫。天下從橫,則天鋒星見。



 二曰燭星,狀如太白,其出也不行,見則不久而
 滅。或曰,主星上有三彗上出。



 燭星所出邑反。又曰,燭星所燭者城邑亂。又曰,燭星所出,有大盜不成。



 三曰蓬星,一名王星,狀如夜火之光,多即至四五,少即一二。亦曰,蓬星在西南,修數丈,左右銳,出而易處。又曰,有星,其色黃白,方不過三尺,名曰蓬星。又曰,蓬星狀如粉絮,見則天下道術士當有出者,布衣之士貴,天下太平,五穀成。又曰,蓬星出北斗,諸侯有奪地,以地亡,有兵起。星所居者,期不出三年。



 又曰,蓬星出太微中,天子立王。



 四曰長庚,狀如一匹布著天。見則兵起。



 五曰四填,星出四隅,去地六丈餘。或曰,四填去地可四丈。或曰,四填星大而赤,
 去地二丈,當以夜半時出。四填星見,十月而兵起。又曰,四填星見四隅,皆為兵起其下。



 六曰地維臧光。地維臧光者,五行之氣,出於四季土之氣也。又曰,有星出,大而赤,去地二三丈,如月,始出謂之地維臧光。四隅有星,望之可去地四丈,而赤黃搖動,其類填星,是謂中央之野星,出於四隅,名曰地維臧光。出東北隅,天下大水。出東南隅,天下大旱。出西南隅,則有兵起。出西北隅,則天下亂,兵大起。又曰,地維臧光見,下有亂者亡,有德者昌。



 七曰女帛。女帛者,五星氣合變,出東北,水木氣合也。又曰,東北有星,長三丈而出,名曰女帛,見則天下兵起,若有
 大喪。又東北有大星出,名曰女帛,見則天下有大喪。



 八曰盜星。盜星者,五星氣合之變,出東南,火木氣合也。又曰,東南有星,長三丈而出,名曰盜星,見則天下有大盜,多寇賊。



 九曰積陵。積陵者,五星氣合之變,出西北,金水氣合也。又曰,西南有星,長三丈,名曰積陵,見則天下隕霜,兵大起,五穀不成,人饑。



 十曰端星。端星者,五星氣合之變,出與金木水火合於四隅。又四隅有星,大而赤,察之中黃,數動,長可四丈。此土之氣,效於四季,名曰四隅端星,所出,兵大起。



 十一曰昏昌。有星出西北,氣青赤以環之,中赤外青,名曰昏昌,見則天下兵起,國易政。先起
 者昌,後起者亡。高十丈,亂一年。高二十丈,亂二年。高三十丈,亂三年。



 十二曰莘星。有星出西北,狀如有環二,名山勤。一星見則諸侯有失地,西北國。



 十三曰白星。有如星非星,狀如削瓜,有勝兵,名曰白星。白星出,為男喪。



 十四曰菟昌。西北菟昌之星,有赤青環之,有殃,有青為水。此星見,則天下改易。



 十五曰格澤,狀如炎火。又曰,格澤星也,上黃下白,從地而上,下大上銳,見則不種而獲。又曰,不有土功,必有大客鄰國來者,期一年、二年。又曰,格澤氣赤如火,炎炎中天,上下同色,東西糸亙天,若於南北,長可四五里。此熒惑之變,見則兵起,其下伏尸流血,期
 三年。



 十六曰歸邪,狀如星非星,如云非雲。或曰,有兩赤彗上向,上有蓋狀如氣,下連星。或曰,見必有歸國者。



 十七曰濛星,夜有赤氣如牙旗,長短四面,西南最多。又曰刀星,亂之象。又曰,遍天薄雲,四方生赤黃氣,長三尺,乍見乍沒,尋皆消滅。又曰,刀星見,天下有兵,戰鬥流血。或曰,遍天薄雲,四方合有八氣,蒼白色,長三尺,乍見乍沒。



 漢京房著《風角書》,有《集星章》,所載妖星,皆見於月旁,互有五色方云,以五寅日見,各五星所生云。



 天槍星生箕宿中,天根星生尾宿中,天荊星生心宿中,真若星生房宿中,天手袁星生氐宿中,天樓
 星生亢宿中,天垣星生左角宿中,皆歲星所生也。見以甲寅日,其星咸有兩青方在其旁。



 天陰星生軫宿中,晉若星生翼宿中,官張星生張宿中,天惑星生七宿中,天雀星生柳宿中,赤若星生鬼宿中,蚩尤星生井宿中,皆熒惑之所生也。出在丙寅日,有兩赤方在其旁。



 天上、天伐、從星、天樞、天翟、天沸、荊彗,皆鎮星之所生也。出在戊寅日,有兩黃方在其旁。



 若星生參宿中,帚星生觜宿中,若彗星生畢宿中,竹彗星生昴宿中,墻星生胃宿中,榬星生婁宿中,
 白雚星生奎宿中,皆太白之所生也。出在庚寅日,有兩白方在其旁。



 天美星生壁宿中,天毚星生室宿中,天杜星生危宿中,天麻星生虛宿中,天林星生女宿中,天高星生牛宿中,端下星生鬥宿中,皆辰星之所生也。出以壬寅日,有兩黑方在其旁。



 已前三十五星,即五行氣所生,皆出月左右方氣之中,各在其所生星將出不出日數期候之。當其未出之前而見,見則有水旱兵喪饑亂,所指亡國失地,王死,破軍殺將。



 客星客星者,周伯、老子、王蓬絮、國皇、溫星,凡五星,皆客星也。行諸列舍,十二國分野,各在其所臨之邦,所守之宿,以占吉兇。周伯,大而色黃,煌煌然。



 見其國兵起,若有喪,天下饑,眾庶流亡去其鄉。瑞星中名狀與此同,而占異。老子,明大,色白,淳淳然。所出之國,為饑,為兇,為善,為惡,為喜,為怒。常出見則兵大起,人主有憂。王者以赦除咎則災消。王蓬絮,狀如粉絮,拂拂然。見則其國兵起,若有喪,白衣之會,其邦饑亡。又曰,王蓬絮,星色青而熒熒然。所見之國,風雨不如節,焦旱,物不生,五穀不成登,蝗蟲多。國皇星,出而
 大,其色黃白,望之有芒角。見則兵起,國多變,若有水饑,人主惡之,眾庶多疾。溫星,色白而大,狀如風動搖,常出四隅。出東南,天下有兵,將軍出於野。出東北,當有千里暴兵。出西北,亦如之。出西南,其國兵喪並起,若有大水,人饑。又曰,溫星出東南,為大將軍服屈不能發者。出於東北,暴骸三千里。出西亦然。



 凡客星見其分,若留止,即以其色占吉兇。星大事大,星小事小。星色黃得地,色白有喪,色青有憂,色黑有死,色赤有兵,各以五色占之,皆不出三年。又曰,客星入列宿中外官者,各以其所出部舍官名為其事。所之者為其
 謀,其下之國,皆受其禍。以所守之舍為其期,以五氣相賊者為其使。



 流星流星,天使也。自上而降曰流,自下而升曰飛。大者曰奔,奔亦流星也。星大者使大,星小者使小。聲隆隆者,怒之象也。行疾者期速,行遲者期遲。大而無光者,眾人之事。小而光者,貴人之事。大而光者,其人貴且眾也。乍明乍滅者,賊敗成也。前大後小者,恐憂也。前小後大者,喜事也。蛇行者,奸事也。往疾者,往而不返也。長者,其事長久也。短者,事疾也。奔星所墜,其下有兵。無風雲,有流星見,
 良久間乃入,為大風發屋折木。小流星百數,四面行者,庶人流移之象。



 流星異狀,名占不同。今略古書及《荊州占》所載雲。



 流星之尾,長二三丈,暉然有光竟天,其色白者,主使也,色赤者,將軍使也。



 流星有光,其色黃白者,從天墜有音,如炬熛火下地,野雉盡鳴,斯天保也。所墜國安有喜,若水。流星其色青赤,名曰地雁,其所墜者起兵。流星有光青赤,其長二三丈,名曰天雁,軍之精華也。其國起兵,將軍當從星所之。流星暉然有光,白,長竟天者,人主之星也,主將相軍從星所之。凡星如甕者,為發謀起事。大如
 桃者為使事。流星大如缶,其光赤黑,有喙者,名曰梁星,其所墜之鄉有兵,君失地。



 飛星大如缶若甕,後皎然白,前卑後高,此謂頓頑,其所從者多死亡,削邑而不戰。有飛星大如缶若甕,後皎然白,前卑後高,搖頭,乍上乍下,此謂降石,所下民食不足。飛星大如缶若甕,後皎然白,星滅後,白者曲環如車輪,此謂解銜。



 其國人相斬為爵祿,此謂自相嚙食。有飛星大如缶若甕,其後皎然白,長數丈,星滅後,後者化為雲流下,名曰大滑,所下有流血積骨。有飛星大如缶若甕,後皎白,縵縵然長可十餘丈而委曲,名曰天刑,一曰天
 飾,將軍均封疆。



 天狗,狀如大奔星,色黃有聲,其止地類狗,所墜,望之如火光,炎炎沖天,其上銳,其下圓,如數頃田處。或曰,星有毛,旁有短彗,下有狗形者。或曰,星出,其狀赤白有光,下即為天狗。一曰,流星有光,見人面,墜無音,若有足者,名曰天狗。其色白,其中黃,黃如遺火狀。主候兵討賊,見則四方相射,千里破軍殺將。或曰,五將鬥,人相食,所往之鄉有流血。其君失地,兵大起,國易政,戒守御。余占同前。營頭,有雲如壞山墮,所謂營頭之星,所墮,其下覆軍,流血千里。亦曰,流星晝隕名營頭。



 雲氣瑞氣一曰慶雲,若煙非煙,若云非雲,鬱鬱紛紛,蕭索輪囷,是謂慶雲,亦曰景雲。



 此喜氣也,太平之應。一曰昌光,赤如龍狀。聖人起,帝受終則見。



 妖氣一曰虹蜺,日旁氣也。斗之亂精,主惑心,主內淫,主臣謀君,天子詘後妃,顓妻不一。二曰䍧云,如狗,赤色長尾,為亂君,為兵喪。



\end{pinyinscope}