\article{卷二十一志第十六 天文下}

\begin{pinyinscope}

 十
 煇《周禮》,眡祲氏掌十煇之法,以觀妖祥,辨吉兇。一曰祲,謂陰陽五色之氣,昆淫相侵。或曰,抱珥背璚之屬,如虹而短是也。二曰象,謂雲如氣,成形象,雲如赤烏,夾日以飛之類是也。三曰鐫,日旁氣刺日,形如童子所佩之鐫也。四曰監,謂雲氣臨在日上也。五曰暗,謂日月蝕,或日光
 暗也。六曰曹,謂瞢瞢不光明也。



 七曰彌,謂白虹彌天而貫日也。八曰序,謂氣若山而在日上。或曰,冠珥背璚,重疊次序,在於日旁也。九曰U,謂暈氣也。或曰,虹也。《詩》所謂「朝U於西」



 者也。十曰想,謂氣五色,有形想也,青饑,赤兵,白喪,黑憂,黃熟。或曰,想,思也,赤氣為人獸之形,可思而知其吉兇。自周已降,術士間出,今採其著者而言之。



 日,君乘土而王,其政太平,則日五色。又曰,或黑或青或黃,師破。又曰,游氣蔽天,日月失色,皆是風雨之候也。若天氣清靜,無諸游氣,日月不明,乃為失色。或天氣下降,地氣未升,厚則日紫,薄則日赤,若於夜則月白,皆將雨
 也。



 或天氣未降,地氣上升,厚則日黃,薄則日白,若於夜則月赤,將旱且風。亦為日月暈之候,雨少而多陰。或天氣已降,地氣又升,上下未交則日青,若於夜則月綠色,將寒候也。或天地氣雖交而未密,則日黑,若於夜則月青,將雨不雨,變為雺霧,暈背虹蜺。又曰,沉陰,日月俱無光,晝不見日,夜不見星,皆有雲鄣之,兩敵相當,陰相圖議也。日曚曚光,士卒內亂。日薄赤,見日中烏,將軍出,旌旗舉,此不祥,必有敗亡。又曰,數日俱出若鬥,天下兵大戰。日斗下有拔城。



 日戴者,形如直狀,其上微起,在日上為戴。戴者德也,國有喜也。一云,立日上為戴。青赤氣抱
 在日上,小者為冠,國有喜事。青赤氣小,而交於日下,為纓。



 青赤氣小而圓,一二在日下左右者,為紐。青赤氣如小半暈狀,在日上為負。負者得地為喜。又曰,青赤氣長而斜倚日傍為戟。青赤氣圓而小,在日左右為珥。黃白者有喜。又曰有軍。日有一珥為喜,在日西,西軍戰勝,在日東,東軍戰勝。南北亦如之,無軍而珥,為拜將。又日旁如半環,向日為抱。青赤氣如月初生,背日者為背。又曰,背氣青赤而曲,外向為叛象,分為反城。璚者如帶,璚在日四方。青赤氣長,而立日旁,為直。日旁有一直,敵在一旁欲自立,從直所擊者勝。日旁有二直三抱,欲自立者
 不成。順抱擊者勝,殺將。氣形三抱,在日四方,為提。青赤氣橫在日上下為格。氣如半暈,在日下為承。承者,臣承君也。又曰,日下有黃氣三重若抱,名曰承福,人主有吉喜,且得地。青白氣如履,在日下者為履。日旁抱五重,戰順抱者勝。日一抱一背為破走。抱者,順氣也,背者,逆氣也。兩軍相當,順抱擊逆者勝,故曰破走。日抱且兩珥,一虹貫抱,至日,順虹擊者勝。日重抱,內有璚,順抱擊者勝;亦曰軍內有欲反者。日重抱,左右二珥,有白虹貫抱,順抱擊勝,得二將。有三虹,得三將。日抱黃白潤澤,內赤外青,天子有喜,有和親來降者。軍不戰,敵降,軍罷。色青,
 將喜;赤,將兵爭;白,將有喪;黑,將死。



 日重抱且背,順抱擊者勝,得地,若有罷師。日重抱,抱內外有璚,兩珥,順抱擊者勝,破軍,軍中不和,不相信。日旁有氣,圓而周市,內赤而外青,名為暈。日暈者,軍營之象。周環匝日無厚薄,敵與軍勢齊等。若無軍在外,天子失御,民多叛。日暈有五色,有喜;不得五色,有憂。



 凡占兩軍相當,必謹審日月暈氣,知其所起,留止遠近,應與不應,疾遲大小,厚薄長短,抱背為多少,有無實虛久亟,密疏澤枯。相應等者勢等。近勝遠,疾勝遲,大勝小,厚勝薄,長勝短,抱勝背,多勝少,有勝無,實勝虛,久勝亟,
 密勝疏,澤勝枯。重背大破,重抱為和親,抱多親者益多,背為不和。分離相去,背於內者離於內,背於外者離於外也。



 凡占分離相去,赤內青外,以和相去;青內赤外,以惡相去。日暈明久,內赤外青,外人勝;內青外赤,內人勝;內黃外青黑,內人勝;外黃內青黑,外人勝;外白內青,外人勝;內白外青,內人勝,內黃外青,外人勝;內青外黃,內人勝。



 日暈周匝,東北偏厚,厚為軍福,在東北戰勝,西南戰敗。日暈黃白,不鬥兵未解;青黑,和解分地;色黃,土功動,人不安;日色黑,有水,陰國盛。日暈七日無風雨,兵大作,不
 可起,眾大敗。不及日蝕,日暈而明,天下有兵,兵罷;無兵,兵起不戰。日暈始起,前滅而後成者,後成面勝。日暈有兵在外者,主人不勝。日暈,內赤外青,群臣親外;外赤內青,群臣親內其身,身外其心。日有朝夕暈,是謂失地,主人必敗。



 日暈而珥,主有謀,軍在外,外軍有悔。日暈抱珥上,將軍易。日暈而珥如井幹者,國亡,有大兵交。日暈上西,將軍易,兩敵相當。日暈兩珥,平等俱起而色同,軍勢等,色厚潤澤者賀喜。日暈有直珥為破軍,貫至日為殺將。日暈員且戴,國有喜,戰從戴所擊者勝,得地。日暈而珥背左右,如大軍輞者,兵起,其國亡城,兵滿野而城復
 歸。



 日暈,暈內有珥一抱,所謂圍城者在內,內人則勝。日暈有重抱,後有背,戰順抱者勝,得地有軍。日暈有一抱,抱為順,貫暈內,在日西,西軍勝,有軍。



 日暈有一背,背為逆,在日西,東軍勝。餘方放此。日暈而背,兵起,其分,失城。日暈有背,背為逆,有降叛者,有反城。在日東,東有叛。餘方放此。日暈背氣在暈內,此為不和,分離相去。其色青外赤內,節臣受王命有所之。日暈上下有兩背,無兵兵起,有兵兵入。日暈四背在暈內,名曰不和,有內亂。日暈而四背如大車輞者四提,設其國眾在外,有反臣。日暈四提,必有大將出亡者。日暈有四背璚,其背端盡出暈
 者,反從內起。



 日暈而兩珥在外,有聚雲在內與外,不出三日,城圍出戰。日暈有背珥直,而有虹貫之者,順虹擊之,大勝得地。日暈,有白虹貫暈至日,從虹所指戰勝,破軍殺將。日暈,有虹貫暈,不至日,戰從貫所擊之勝,得小將。日暈,有一虹貫暈內,順虹擊者勝,殺將。日暈,二白虹貫暈,有戰,客勝。日重暈,有四五白虹氣,從內出外,以此圍城,主人勝,城不拔。又日重暈,攻城圍邑不拔。日暈二重,其外清內濁不散,軍會聚。日暈三重,有拔城。日交暈無厚薄,交爭,力勢均,厚者勝。



 日交暈,人主左右有爭者,兵在外戰。日在暈上,軍罷。交暈貫日,天下有破軍死將。
 日交暈而爭者先衰,不勝即兩敵相向。交暈至日月,順以戰勝,殺將。一法日在上者勝。日有交者,赤青如暈狀,或如合背,或正直交者,偏交也,兩氣相交也,或相貫穿,或相向,或相背也。交主內亂,軍內不和。日交暈如連環,為兩軍兵起,君爭地。日有三暈,軍分為三。日方暈而上下聚二背,將敗人亡。日暈若井垣,若車輪,二國皆兵亡。又曰有軍。



 日暈不匝,半暈在東,東軍勝,在西,西軍勝。南北亦如之。日暈如車輪半,軍在外者罷。日半暈東向者,西夷羌胡來入國。半暈西向者,東夷人欲反入國。半暈北向者,南夷人欲反入國。半暈南向者,北夷人欲反入
 國。



 又曰,軍在外,月暈師上,其將戰必勝。月暈黃色,將軍益秩祿,得位。月暈有兩珥,白虹貫之,天下大戰。月暈而珥,兵從珥攻擊者利。月暈有蜺雲,乘之以戰,從蜺所往者大勝。月暈,虹蜺直指暈至月者,破軍殺將。



 雜氣天子氣,內赤外黃正四方,所發之處,當有王者。若天子欲有游往處,其地亦先發此氣。或如城門,隱隱在氣霧中,恆帶殺氣森森然,或如華蓋在氣霧中,或有五色,多在晨昏見。或如千石倉在霧中,恆帶殺氣,或如高樓在霧氣中,或如山鎮。



 蒼帝起,青雲扶日。赤帝起,赤雲扶日。
 黃帝起,黃雲扶日。白帝起,白雲扶日。



 黑帝起,黑雲扶日。或日氣象青衣人,無手,在日西,天子之氣也。敵上氣如龍馬,或雜色鬱鬱沖天者,此帝王之氣,不可擊。若在吾軍,戰必大勝。凡天子之氣,皆多上達於天,以王相日見。



 凡猛將之氣如龍。兩軍相當,若氣發其上,則其將猛銳。或如虎,在殺氣中。



 猛將欲行動,亦先發此氣;若無行動,亦有暴兵起。或如火煙之狀,或白如粉沸,或如火光之狀,夜照人,或白而赤氣繞之,或如山林竹木,或紫黑如門上樓,或上黑下赤,狀似黑旌,或如張弩,或如埃塵,頭銳而卑,本大而高。兩軍相當,敵軍上氣如囷倉,正白,見
 日逾明,或青白如膏,將勇。大戰氣發,漸漸如云,變作此形,將有深謀。



 凡氣上與天連,軍中有貞將,或云賢將。



 凡軍勝氣,如堤如阪,前後磨地,此軍士眾強盛,不可擊。軍上氣如火光,將軍勇,士卒猛,好擊戰,不可擊。軍上氣如山堤,山上若林木,將士驍勇。軍上氣如埃塵粉沸,其色黃白,旌旗無風而颺,揮揮指敵,此軍必勝。敵上有白氣粉沸如樓,繞以赤氣者,兵銳。營上氣黃白色,重厚潤澤者,勿與戰。兩敵相當,有氣如人持斧向敵,戰必大勝。兩敵相當,上有氣如蛇舉首向敵者戰勝。敵上氣如一
 匹帛者,此雍軍之氣,不可攻。望敵上氣如覆舟,雲如牽牛,有白氣出,似旌幟,在軍上,有雲如鬥雞,赤白相隨,在氣中,或發黃氣,皆將士精勇,不可擊。軍營上有赤黃氣,上達於天,亦不可攻。



 凡軍營上五色氣,上與天連,此天應之軍,不可擊。其氣上小下大,其軍日增益士卒。軍上氣如堤,以覆其軍上,前赤後白,此勝氣。若覆吾軍,急往擊之,大勝。天氣銳,黃白團團而潤澤者,敵將勇猛,且士卒能強戰,不可擊。雲如日月而赤氣繞之,如日月暈狀有光者,所見之地大勝,不可攻。



 凡雲氣,有獸居上者勝。軍上有氣如塵埃,前下後高者,將士精銳。敵上氣如乳武豹伏者,難攻。軍上恆有氣者,其軍難攻。軍上雲如華蓋者,勿往與戰。雲如旌旗,如蜂向人者,勿與戰。兩軍相當,敵上有雲如飛鳥,徘徊其上,或來而高者,兵精銳,不可擊。軍上雲如馬,頭低尾仰,勿與戰。軍上雲如狗形,勿與戰。望四方有氣如赤鳥,在烏氣中,如烏人在赤氣中,如赤杵在烏氣中,如人十十五五,或如旌旗,在烏氣中,有赤氣在前者,敵人精悍,不可當。敵上有雲如山,不可說。



 有雲如引素,如陣前銳,或一或四,黑色有陰謀,赤色饑,青色兵有反,黃色急去。



 凡氣,上黃下白,名曰善氣。所臨之軍,欲求和退。若氣出北方,求退向北,其眾死散。向東則不可信,終能為害。向南將死。敵上氣囚廢枯散。或如馬肝色,如死灰色,或類偃蓋,或類偃魚,皆為將敗。軍上氣乍見乍不見,如霧起,此衰氣,可擊。上大下小,士卒日減。



 凡軍營上十日無氣發,則軍必勝。而有赤白氣乍出即滅,外聲欲戰,其實欲退散。黑氣如壞山墮軍上者,名曰營頭之氣,其軍必敗。軍上氣昏發連夜,夜照人,則軍士散亂。軍上氣半而絕,一敗,再絕再敗,三絕三敗。在東發白氣者,災深。



 軍上氣中有黑雲如牛形,或如豬形者,此
 是瓦解之氣,軍必敗。敵上氣如粉如塵者,勃勃如煙,或五色雜亂,或東西南北不定者,其軍欲敗。軍上氣如群羊群豬在氣中,此衰氣,擊之必勝。軍上有赤氣,炎降於天,則將死,士眾亂。赤光從天流下入軍,軍亂將死。彼軍上有蒼氣,須臾散去,擊之必勝。在我軍上,須自堅守。軍有黑氣如牛形,或如馬形,從氣霧中下,漸漸入軍,名曰天狗下食血,則軍破。軍上氣或如群鳥亂飛,或如懸衣,如人相隨,或紛紛如轉蓬,或如揚灰,或云如卷席,如匹布亂穰者,皆為敗徵。氣乍見乍沒,乍聚乍散,如霧之始起,為敗氣。氣如系牛,如人臥,如敗車,如雙蛇,如飛鳥,如
 決堤垣,如壞屋,如人相指,如人無頭,如驚鹿相逐,如兩雞相向,皆為敗氣。



 凡降人氣,如人十十五五,皆叉手低頭。又云,如人叉手相向。白氣如群鳥,趣入屯營,連結百餘里不絕,而能徘徊,須臾不見者,當有他國來降。氣如黑山,以黃為緣者,欲降服。敵上氣青而高漸黑者,將欲死散。軍上氣如燔生草之煙,前雖銳,後必退。黑氣臨營,或聚或散,如鳥將宿,敵人畏我,心意不定,終必逃背,逼之大勝。



 凡白氣從城中南北出者,不可攻,城不可屠。城中有黑雲如星,名曰軍精,急解圍去,有突兵出,客敗。城上白氣
 如旌旗,或青雲臨城,有喜慶。黃雲臨城,有大喜慶,青色從中南北出者,城不可攻。或氣如青色,如牛頭觸人者,城不可屠。



 城中氣出東方,其色黃,此太一。城白氣從中出,青氣從城北入,反向還者,軍不得入。攻城圍邑,過旬雷雨者,為城有輔。疾去之,勿攻。城上氣如煙火,主人欲出戰。其氣無極者,不可攻。城上氣如雙蛇者,難攻。赤氣如杵形,從城中向外者,內兵突出,主人戰勝。城上有云,分為兩彗狀,攻不可得。赤氣在城上,黃氣四面繞之,城中大將死,城降。城上赤氣如飛鳥,如敗車,及無雲氣,士卒必散。城營中有赤黑氣,如貍皮斑及赤者,並亡。城上
 氣上赤而下白色,或城中氣聚如樓,出見於外,城皆可屠。城營上有雲如眾人頭,赤色,下多死喪流血。城上氣如灰,城可屠。氣出而北,城可克。其氣出復入,城中人欲逃亡。其氣出而覆其軍,軍必病。



 氣出而高,無所止,用日久長。有白氣如蛇來指城,可急攻。白氣從城指營,宜急固守。攻城若雨霧死風至,兵勝。日色無光為日死。雲氣如雄雉臨城,其下必有降者。濛氛圍城而入城者,外勝,得入。有雲如立人五枚,或如三牛,邊城圍。



 凡軍上有黑氣,渾渾圓長,赤氣在其中,其下必有伏兵。白氣粉沸起,如樓狀,其下必有藏兵萬人,皆不可輕擊。伏
 兵之氣,如幢節狀,在烏雲中,或如赤杵在烏雲中,或如烏人在赤雲中。



 凡暴兵氣,白如瓜蔓連結,部隊相逐,須臾罷而復出,至八九來而不斷,急賊卒至,宜防固之。白氣如仙人衣,千萬連結,部隊相逐,罷而復興,如是八九者,當有千里兵來,視所起備之。黑雲從敵上來,之我軍上,欲襲我。敵人告發,宜備不宜戰。壬子日,候四望無雲,獨見赤雲如旌旗,其下有兵起,若遍四方者,天下盡有兵。若四望無雲,獨見黑雲極天,天下兵大起。半天,半起。三日內有雨,災解。敵欲來者,其氣上有云,下有氛零,中天而下,敵必至。
 雲氣如旌旗,賊兵暴起。暴兵氣,如人持刀楯,雲如人,赤色,所臨城邑,有卒兵至,驚怖,須臾去。



 赤氣如人持節,兵來未息。雲如方虹,有暴兵。赤雲如火者,所向兵至。天有白氣,狀如匹布,經丑未者,天下多兵。



 凡戰氣,青白如膏,將勇。大戰氣,如人無頭,如死人臥。敵上氣如丹蛇,赤氣隨之,必大戰,殺將。四望無雲,見赤氣如狗入營,其下有流血。



 凡連陰十日,晝不見日,夜不見月,亂風四起,欲雨而無雨,名曰蒙,臣謀君。



 故曰,久陰不雨臣謀主。霧氣若晝若夜,其色青黃,更相掩冒,乍合乍散,臣謀君,逆者喪。山中
 冬霧十日不解者,欲崩之候。視四方常有大雲,五色具者,其下有賢人隱也。青雲潤澤蔽日,在西北為舉賢良。雲氣如亂穰,大風將至,視所從來避之。



 雲甚潤而厚,大雨必暴至。四始之日,有黑雲氣如陣,厚重大者,多雨。氣若霧非霧,衣冠不雨而濡,見則其城帶甲而趣。日出沒時,有雲橫截之,白者喪,烏者驚。



 三日內雨者各解。有黑氣入營者,兵相殘。有赤青氣入營者,兵弱。有雲如蛟龍,所見處將軍失魄。有雲如鵠尾,來廕國上,三日亡。有雲如日月暈,赤色,其國兇。



 青白色,有大水。有雲狀如龍行,國有大水,人流亡。有雲赤黃色,四塞終日,竟夜照地者,
 大臣縱恣。有雲如氣,昧而濁,賢人去,小人在位。



 凡白虹者,百殃之本,眾亂所基。霧者,眾邪之氣,陰來冒陽。



 凡遇四方盛氣,無向之戰。甲乙日青氣在東方,丙丁日赤氣在南方,庚辛日白氣在西方,壬癸日黑氣在北方,戊巳日黃氣在中央。四季戰當此日氣,背之吉。日中有黑氣,君有小過而臣不諫,又掩君惡而揚君善,故日中有黑氣不明也。



 凡白虹霧,奸臣謀君,擅權立威。晝霧夜明,臣志得申,夜霧晝明,臣志不申。



 霧終日終時,君有憂。色黃小雨。白言
 兵喪,青言疾,黑有暴水,赤有兵喪,黃言土功,或有大風。



 凡夜霧,白虹見,臣有憂。晝霧白虹見,君有憂。虹頭尾至地,流血之象。



 凡霧氣不順四時,逆相交錯,微風小雨,為陰陽氣亂之象。從寅至辰巳上,周而復始,為逆者不成。積日不解,晝夜昏暗,天下欲分離。



 凡霧四合,有虹各見其方,隨四時色吉,非時色兇。氣色青黃,更相掩覆,乍合乍散,臣欲謀君,為逆者不成,自亡。



 凡霧氣四方俱起,百步不見人,名曰晝昏,不有破國,必有滅門。



 凡天地四方昏濛若下塵,十日五日以上,或一日,或一時,雨不沾衣而有土,名曰霾。故曰,天地霾,君臣乖,大旱。



 凡海傍蜃氣象樓臺,廣野氣成宮闕,北夷之氣如牛羊群畜穹閭,南夷之氣類舟船幡旗。自華以南,氣下黑上赤。嵩高、三河之郊,氣正赤。恆山之北,氣青。勃、碣、海、岱之間,氣皆正黑。江湖之間,氣皆白。東海氣如圓簦。附漢、河水,氣如引布。江、漢氣勁如杼。濟水氣如黑。滑水氣如狼白尾。淮南氣如帛。少室氣如白兔青尾。恆山氣如黑牛青尾。東夷氣如樹,西夷氣如室屋,南夷氣如闍臺,或類舟船。陣雲如立垣,杼軸雲類軸搏,兩端銳。心勺雲如繩,
 居前亙天,其半半天,其V者類闕旗,故鉤云勾曲。諸此云見,以五色占而澤摶密。其見,動人及有兵,必起合斗。其直,雲氣如三匹帛,廣前銳後,大軍行氣也。韓雲如布,趙雲如牛,楚雲如日,宋雲如車,魯雲如馬,衛雲如犬,周云如車輪,秦雲如行人,魏雲如鼠,鄭、齊雲如絳衣,越雲如龍,蜀雲如囷。車氣乍高乍下,往往而聚。騎氣卑而布,卒氣摶。前卑後高者疾,前方而高,後銳而卑者卻。其氣平者,其行徐。前高後卑者,不止而返。校騎之氣正蒼黑,長數百丈,游兵之氣如彗掃,一雲長數百丈,無根本。喜氣上黃下白,怒氣上下赤,憂氣上下黑,土功氣黃白,徙
 氣白。



 凡候氣之法,氣初出時,若云非雲,若霧非霧,仿佛若可見。初出森森然,在桑榆上,高五六尺者,是千五百里外。平視則千里,舉目望則五百里。仰瞻中天,則百里內。平望桑榆間二千里,登高而望,下屬地者,三千里。



 凡欲知我軍氣,常以甲巳日及庚、子、辰、戌、午、未、亥日,及八月十八日,去軍十里許,登高望之可見,依別記占之。百人以上皆有氣。



 凡占災異,先推九宮分野,六壬日月,不應陰霧風雨而陰霧者,乃可占。對敵而坐,氣來甚卑下,其陰覆人,上掩
 溝蓋道者,是大賊必至。敵在東,日出候。在南,日中候。在西,日入候。在北,夜半候。王相色吉,囚死色兇。



 凡軍上氣,高勝下,厚勝薄,實勝虛,長勝短,澤勝枯。我軍在西,賊軍在東,氣西厚東薄,西長東短,西高東下,西澤東枯,則知我軍必勝。



 凡氣初出,似甑上氣,勃勃上升。氣積為霧,霧為陰,陰氣結為虹蜺姿暈珥之屬。



 凡氣不積不結,散漫一方,不能為災。必須和雜殺氣,森森然疾起,乃可論占。



 軍上氣安則軍安,氣不安則軍不安。氣南北則軍南北,氣東西則軍亦東西。氣散則為軍
 破敗。



 候氣,常以平旦、下晡、日出沒時處氣,以見知大。占期內有大風雨久陰,則災不成。故風以散之,陰以諫之,云以幡之,雨以厭之。



 五代災變應梁武帝天監元年八月壬寅,熒惑守南斗。占曰:「糴貴,五穀不成,大旱,多火災,吳、越有憂,宰相死。」是歲大旱,米斗五千,人多餓死。其二年五月,尚書範云卒。



 二年五月丙辰,月犯心。占曰:「有亂臣,不出三年,有亡國。」其四年,交州刺史李凱舉兵反。七月丙子,太白犯軒轅
 大星。



 四年六月壬戌,歲星晝見。占曰:「歲色黃潤,立竿影見,大熟。」是歲大穰,米斛三十。又曰:「星與日爭光,武且弱,文且強。」自此後,帝崇尚文儒,躬自講說,終於太清,不修武備。八月庚子,老人星見。占曰:「老人星見,人主壽昌。」



 自此後,每年恆以秋分後見於參南,至春分而伏。武帝壽考之象云。



 七年九月巳亥,月犯東井。占曰:「有水災。」其年京師大水。



 十年九月丙申,天西北隆隆有聲,赤氣上至地。占曰:「天狗也,所往之鄉有流血,其君失地。」其年十二月,馬仙琕
 大敗魏軍,斬馘十餘萬,克復朐山城。十二月壬戌朔,日食,在牛四度。



 十三年二月丙午,太白失行,在天關。占曰:「津梁不通,又兵起。」其年填星守天江。占曰:「有江河塞,有決溢,有土功。」其年,大發軍眾造浮山堰,以堨淮水。至十四年,填星移去天江而堰壞,奔流決溢。



 十四年十月辛未,太白犯南斗。



 十七年閏八月戊辰,月行掩昴。



 普通元年春正月丙子,日有食之。占曰:「日食,陰侵陽,陽不克陰也。為大水。」其年七月,江、淮、海溢。九月乙亥,有星
 晨見東方,光爛如火。占曰:「國皇見,有內難,有急兵反叛。」其三年,義州刺史文僧朗以州叛。



 四年十一月癸未朔,日有食之,太白晝見。



 六年三月丙午,歲星入南斗。庚申,月食。五月己酉,太白晝見。六月癸未,太白經天。九月壬子,太白犯右執法。



 七年正月癸卯,太白歲星在牛相犯。占曰:「其國君兇,易政。」明年三月,改元,大赦。大通元年八月甲申,月掩填星。閏月癸酉,又掩之。占曰:「有大喪,天下無主,國易政。」其後中大通元年九月癸巳,上又幸同泰寺舍身,王公以一億萬錢奉贖。十月己酉
 還宮,大赦,改元。中大通三年,太子薨,皆天下無主、易政及大喪之應。



 中大通元年閏月壬戌,熒惑犯鬼積尸。占曰:「有大喪,有大兵,破軍殺將。」



 其二年,蕭玩帥眾援巴州,為魏梁州軍所敗,玩被殺。



 四年七月甲辰,星隕如雨。占曰:「星隕,陽失其位,災害之象萌也。」又曰:「星隕如雨,人民叛,下有專討。」又曰:「大人憂。」其後侯景狡亂,帝以憂崩,人眾奔散,皆其應也。



 五年正月己酉,長星見。



 六年四月丁卯,熒惑在南斗。占曰:「熒惑出入留舍南斗
 中,有賊臣謀反,天下易政,更元。」其年十二月,北梁州刺史蘭欽舉兵反,後年改為大同元年。



 大同三年三月乙丑,歲星掩建星。占曰:「有反臣。」其年,會稽山賊起。其七年,交州刺史李賁舉兵反。



 五年十月辛丑,彗出南斗,長一尺餘,東南指,漸長一丈餘。十一月乙卯,至婁滅。占曰:「天下有謀王者。」其八年正月,安成民劉敬躬挾左道以反,黨與數萬。其九年,李賁僭稱皇帝於交州。



 太清二年五月,兩月見。占曰:「其國亂,必見於亡國。」



 三年正月壬午,熒惑守心。占曰:「王者惡之。」乙酉,太白晝
 見。占曰:「不出三年,有大喪,天下革政更王,強國弱,小國強。」三月丙子,熒惑又守心。



 占曰:「大人易政,主去其宮。」又曰:「人饑亡,海內哭,天下大潰。」是年,帝為侯景所幽,崩。七月,九江大饑,人相食十四五。九月戊午,月在斗,掩歲星。



 占曰:「天下亡君。」其後侯景篡殺。



 簡文帝大寶元年正月丙寅,月晝光見。占曰:「月晝光,有隱謀,國雄逃。」



 又云:「月晝明,奸邪並作,擅君之朝。」其後侯景篡殺,皆國亂亡君。大喪更政之應也。



 元帝承聖三年九月甲午,月犯心中星。占曰:「有反臣,王者惡之,有亡國。」



 其後三年,帝為周軍所俘執,陳氏取國,
 梁氏以亡。



 陳武帝永定三年九月辛卯朔,月入南斗。占曰:「月入南斗,大人憂。」一曰:「太子殃。」後二年,帝崩,太子昌在周為質,文帝立。後昌還國,為侯安都遣盜迎殺之。



 三年五月丙辰朔,日有食之。占曰:「日食君傷。」又曰:「日食帝德消。」



 六月庚子,填星鉞與太白並。占:「太白與填合,為疾為內兵。」



 文帝天嘉元年五月辛亥,熒惑犯右執法。占曰:「大臣有憂,執法者誅。」後四年,司空侯安都賜死。



 九月癸丑,彗星長四尺,見芒,指西南。占曰:「彗星見則敵
 國兵起,得本者勝。」其年,周將獨孤盛領眾趣巴湘,侯瑱襲破之。



 二年五月己酉,歲星守南斗。六月丙戌,熒惑犯東井。七月乙丑,熒惑入鬼中。



 戊辰,熒惑犯斧質。十月,熒惑行在太微右掖門內。



 三年閏二月己丑,熒惑逆行,犯上相。甲子,太白犯五車、填星。七月,太白犯輿鬼。八月癸卯,月犯南斗。丙午,月犯牽牛。庚甲,太白入太微。十一月丁丑,月犯畢左股。辛巳,熒惑犯歲星。戊子,月犯角。庚寅,月入氐。



 四年六月癸丑,太白犯右執法。七月戊子,熒惑犯填星。
 八月甲午,熒惑犯軒轅大星。丁未,太白犯房。九月戊寅,熒惑入太微,犯右執法。癸未,太白入南斗。



 占曰:「太白入斗,天下大亂,將相謀反,國易政。」又曰:「君死,不死則廢。」



 又曰:「天下受爵祿。」其後安成王為太傅,廢少帝而自立,改官受爵之應也。辛卯,熒惑犯左執法。十一月辛酉,熒惑犯右執法。甲戌,月犯畢左股。



 五年正月甲子,月犯畢大星奎。丁卯,月犯星。四月庚子,太白歲星合在奎,金在南,木在北,相去二尺許。壬寅,月入氐,又犯熒惑,太白歲星又合,在婁,相去一尺許。癸卯,月犯房上星。五月庚午,熒惑逆行二十一日,犯氐東南、
 西南星。占曰:「月有賊臣。」又曰:「人主無出,廊廟間有伏兵。」又曰:「君死,有赦。」後二年,少帝廢之應也。六月丙申,月犯亢。七月戊寅,月犯畢大星。閏十月庚申,月犯牽牛。丙子,又犯左執法。十一月乙未,月食畢大星。



 六年正月己亥,太白犯熒惑,相去二寸。占曰:「其野有兵喪,改立侯王。」



 三月丁卯,日入後,眾星未見,有流星白色,大如斗,從太微間南行,尾長尺餘。



 占曰:「有兵與喪。」四月丁巳,月犯軒轅。占曰:「女主有憂。」五月丁亥,太白犯軒轅。占曰:「女主失勢。」又曰:「四方禍起。」其後年,少帝廢,廢後慈訓太后崩。六月己未,月犯氐。辛酉,有彗長可丈餘。占曰:「
 陰謀奸宄起。」一曰:「宮中火起。」後安成王錄尚書、都督中外諸軍事,廢少帝而自立,陰謀之應。



 八月戊辰,月掩畢大星。丙子,月與太白並,光芒相著,在太微西蕃南三尺所。九月辛巳,熒惑犯左執法。癸未,太白犯右執法。辛卯,犯左執法。乙巳,月犯上相,太白犯熒惑。其夜,月又犯太白。占曰:「其國內外有兵喪,改立侯王。」明年,帝崩,又少帝廢之應也。



 七年二月庚午,日無光,烏見。占曰:「王者惡之。」其日庚午,吳、楚之分野。四月甲子,日有交暈,白虹貫之。是月癸酉,帝崩。



 廢帝天康元年五月庚辰,月犯軒轅女御大星。占曰:「女主憂。」後年,慈訓太后崩。癸未,月犯左執法。



 光大元年正月甲寅,月犯軒轅大星。占曰:「女主當之。」八月戊寅,月食哭星。占曰:「有喪泣事。」明年,太后崩,臨海王薨,哭泣之應也。壬午,鎮星辰星合於軫。九月戊午,辰星太白相犯。占曰:「改立侯王。」己未,月犯歲星。占曰:「國亡君。」十二月辛巳,月又犯歲星。辛卯,月犯建星。占曰:「大人惡之。」



 二年正月戊申,月掩歲星。占曰:「國亡君。」五月乙未,月犯太白。六月丙寅,太白犯右執法。壬子,客星見氐東。八月
 庚寅,月犯太微。九月庚戌,太白逆行,與鎮星合,在角。占曰:「為白衣之會。」又曰:「所合之國,為亡地,為疾兵。」戊午,太白晝見。占曰:「太白晝見,國更政易王。」十一月丙午,歲星守右執法,甲申,月犯太微東南星。戊子,太白入氐。十二月甲寅,慈訓太后廢帝為臨海王,太建二年四月薨,皆其應也。



 宣帝太建七年四月丙戌,有星孛於大角。占曰:「人主亡。」五月庚辰,熒惑犯右執法。壬子,又犯右執法。



 十年二月癸亥,日上有背。占曰:「其野失地,有叛兵。」甲子,吳明徹軍敗於呂梁,將卒並為周軍所虜。來年,淮南之
 地,盡沒於周。十月癸卯,月食熒惑。



 占曰:「國敗君亡,大兵起,破軍殺將。」來年三月,吳明徹敗於呂梁,十三年帝崩,敗國亡君之應也。



 十一年四月己丑,歲星太白辰星合於東井。



 十二年二月壬寅,白虹見西方。占曰:「有喪。」其後十三年帝崩。十月戊午,月犯牽牛吳越之野。占曰:「其國亡,君有憂。」後年帝崩。辛酉,歲星犯執法。



 十二月癸酉,辰星在太白上。甲戌,辰星太白交相掩。占曰:「大兵在野,大戰。」



 辛巳,彗星見西南。占曰:「有兵喪。」明年帝崩,始興王叔陵作亂。



 後主至德元年正月壬戌,蓬星見。占曰:「必有亡國亂臣。」
 後帝於太皇寺舍身作奴,以祈冥助,不恤國政,為施文慶等所惑,以至國亡。



 魏普泰元年十月,歲星熒惑填星太白聚於觜參,色甚明大。占曰:「當有王者興。」其月,齊高祖起於信都,至中興二年春而破爾硃兆,遂開霸業。



 魏武定四年九月丁未,高祖圍玉壁城,有星墜於營,眾驢皆鳴。占曰:「破軍殺將。」高祖不豫,五年正月丙午崩。



 齊文宣帝天保元年十二月甲申,熒惑犯房北頭第一星及鉤鈐。占曰:「大臣有反者。」其二年二月壬辰,太尉彭樂謀反,誅。



 八年二月己亥,歲星守少微,經六十三日。占曰:「五官亂。」五月癸卯,歲星犯太微上將。占曰:「大將憂,大臣死。」其十年五月,誅諸元宗室四十餘家,乾明元年,誅楊遵彥等,皆五官亂,大將憂,大臣死之應也。



 八年七月甲辰,月掩心星。占曰:「人主惡之。」十年十月,帝崩。



 九年二月,熒惑犯鬼質。占曰:「斧質用,有大喪。」三月甲午,熒惑犯軒轅。



 占曰:「女主惡之。」其十年五月,誅魏氏宗室,十月帝崩,斧質用,有大喪之應也。



 十年六月庚子,填星犯井鉞,與太白並。占曰:「子為玄枵,
 齊之分野,君有戮死者,大臣誅,斧鉞用。」其明年二月乙巳,太師常山王誅尚書令楊遵彥、右僕射燕子獻、領軍可硃渾天和、侍中宋欽道等。八月壬午,廢少帝為濟南王。



 廢帝乾明元年三月甲午,熒惑入軒轅。占曰:「女主兇。」後太寧二年四月,太后崩。



 肅宗皇建二年四月丙子,日有食之。子為玄枵,齊之分野。七月乙丑,熒惑入鬼中,戊辰,犯鬼質。占曰:「有大喪。」十一月,帝以暴疾崩。



 武成帝河清元年七月乙亥,太白犯輿鬼。占曰:「有兵謀,
 誅大臣,斧質用。」



 其年十月壬申,冀州刺史平秦王高歸彥反,段孝先討擒,斬之於都市,又其二年,殺太原王紹德,皆斧質用之應也。八月甲寅,月掩畢。占曰:「其國君死,大臣有誅者,有邊兵大戰,破軍殺將。」其十月,平秦王歸彥以反誅,其三年,周師與突厥入並州,大戰城西,伏尸流血百餘里,皆其應也。



 四年正月己亥,太白犯熒惑,相去二寸,在奎。甲辰,太白、熒惑、歲星合在婁。占曰:「甲為齊。三星若合,是謂驚立絕行,其分有兵喪,改立侯王,國易政。」



 三月戊子,慧星見。占曰:「除舊布新,有易王。」至四月,傳位於太子,改元。



 後主天統元年六月壬戌,彗星見於文昌,長數寸,入文昌,犯上將,然後經紫微宮西垣入危,漸長一丈餘,指室壁。後百餘日,在虛危滅。占曰:「有大喪,有亡國易政。」其四年十二月,太上皇崩。



 三年五月戊寅,甲夜,西北有赤氣竟天,夜中始滅。十月丙午,天西北頻有赤氣。占曰:「有大兵大戰。」後周武帝總眾來伐,大戰,有大兵之應也。



 四年六月,彗星見東井。占曰:「大亂,國易政。」七月,孛星見房心,白如粉絮,大如斗,東行。八月,入天市,漸長四丈,犯瓠瓜,歷虛危,入室,犯離宮。



 九月入奎,至婁而滅。孛者,孛
 亂之氣也。占曰:「兵喪並起,國大亂易政,大臣誅。」其後,太上皇崩。至武平二年七月,領軍庫狄伏連、治書侍御史王子宜,受瑯邪王儼旨,矯詔誅錄尚書、淮南王和士開於南臺,伏連等即日伏誅,右僕射馮子琮賜死。此國亂之應也。



 五年二月戊辰,歲星逆行,掩太微上將。占曰:「天下大驚,四輔有誅者。」



 五月甲午,熒惑犯鬼積尸。甲,齊也。占曰:「大臣誅,兵大起,斧質用,有大喪。」



 至武平二年九月,誅瑯邪王儼,三年五月,誅右丞相、咸陽王斛律明月,四年七月,誅蘭陵王長恭,皆懿親名將也。四年十月,又誅崔季舒
 等,此斧質用之應也。



 武平三年八月癸未,填星、歲星、太白合於氐,宋之分野。占曰:「其國內外有兵喪,改立侯王。」其四年十月,陳將吳明徹寇彭城,右僕射崔季舒,國子祭酒張雕,黃門裴澤、郭遵,尚書左丞封孝琰等,諫車駕不宜北幸並州。帝怒,並誅之,內外兵喪之應也。九月庚申,月在婁,食既,至旦不復。占曰:「女主兇。」其三年八月,廢斛律皇后,立穆後。四年,又廢胡後為庶人。十一月乙亥,天狗下西北。



 占曰:「其下有大戰流血。」後周武帝攻晉州,進兵平並州,大戰流血。



 三年十二月辛丑,日食歲星。占曰:「有亡國。」至七年,而齊亡。



 四年五月癸巳,熒惑犯右執法。占曰:「大將死,執法者誅,若有罪。」其年,誅右丞相斛律明月,明年,誅蘭陵王長恭,後年,誅右僕射崔季舒,皆大將死,執法誅之應也。



 周閔帝元年五月癸卯,太白犯軒轅。占曰:「太白行軒轅中,大臣出令。」又曰:「皇后失勢。」辛亥,熒惑犯東井北端第二星。占曰:「其國亂。」又曰:「大旱。」其年九月,塚宰護逼帝遜位,幽於舊邸,月餘殺崩,司會李植、軍司馬孫恆及宮伯乙弗鳳等被誅害。其冬大旱。皆大臣出令、大臣死、旱之應
 也。



 明帝二年三月甲午,熒惑入軒轅。占曰:「王者惡之,女主兇。」其月,王後獨孤氏崩。六月庚子,填星犯井鉞,與太白並。占曰:「傷成於鉞,君有戮死者。」



 其年,太師宇文護進食,帝遇毒崩。



 武帝保定元年九月乙巳,客星見於翼。十月甲戌,日有食之。戊寅,熒惑犯太微上將,合為一。



 二年閏正月癸巳,太白入昴。二月壬寅,熒惑犯太微上相。三月壬午,熒惑犯左執法。七月乙亥,太白犯輿鬼。九月戊辰,日有食之,既。十一月壬午,熒惑犯歲星於危南。



 三年三月乙丑朔,日有食之。九月甲子,熒惑犯太微上將。占曰:「上將誅死。」



 十月壬辰,熒惑犯左執法。



 四年二月庚寅朔,日有食之。甲午,熒惑犯房右驂。三月己未,熒惑又犯房右驂。占曰:「上相誅,車馳人走,天下兵起。」其年十月,塚宰晉公護率軍伐齊。



 十二月,柱國、庸公王雄力戰死之,遂班師。兵起將死之應也。八月丁亥,朔,日有蝕之。



 五年正月辛卯,白虹貫日。占曰:「為兵喪。」甲辰,太白、熒惑、歲星合於婁。六月庚申,慧星出三臺,入文昌,犯上將,後經紫宮西垣入危,漸長一丈餘,指室壁,後百餘日稍短,
 長二尺五寸,在虛危滅,齊之分野。七月辛巳,朔,日有食之。



 天和元年正月己卯,日有食之。十月乙卯,太白晝見,經天。



 二年,正月癸酉朔,日有食之。五月己丑,歲星與熒惑合在井宿,相去五尺。



 並為秦分。占曰:「其國有兵,為饑旱,大臣匿謀,下有反者,若亡地。」閏六月丁酉,歲星、太白合,在柳,相去一尺七寸。柳為周分。占曰:「為內兵。」又曰:「主人兇憂,失城。」是歲,陳湘州刺史華皎率眾來附,遣衛公直將兵援之,因而南伐。九月,衛公直與陳將淳於量戰於沌口,王師失利。元定、韋世沖以步騎數千先度,遂沒陳。七
 月庚戌,太白犯軒轅大星,相去七寸。占曰:「女主失勢,大臣當之。」又曰:「西方禍起。」其十一月癸丑,太保、許公宇文貴薨,大臣當之驗也。十月辛卯,有黑氣一,大如杯,在日中。甲午,又加一,經六日乃滅。占曰:「臣有蔽主之明者。」十一月戊戌朔,日有食之。庚子,熒惑犯鉤鈐,去之六寸。



 占曰:「王者有憂。」又曰:「車騎驚,三公謀。」



 三年三月己未,太白犯井北轅第一星。占曰:「將軍惡之。」其七月壬寅,隋公楊忠薨。四月辛巳,太白入輿鬼,犯積尸。占曰:「大臣誅。」又曰:「亂臣在內,有屠城。」六月甲戌,彗見東井,長一丈,上白下赤而銳,漸東行,至七月癸卯,在鬼
 北八寸所乃滅。占曰:「為兵,國政崩壞。」又曰:「將軍死,大臣誅。」



 七月己未,客星見房心,白如粉絮,大如斗,漸大,東行;八月,入天市,長如匹所,復東行,犯河鼓右將;癸未,犯瓠瓜,又入室,犯離宮;九月壬寅,入奎,稍小;壬戌,至婁北一尺所滅。凡六十九日。占曰:「兵起,若有喪,白衣會,為饑旱,國易政。」又曰:「兵犯外城,大臣誅。」



 四年二月戊辰,歲星逆行,掩太微上將。占曰:「天下大驚,國不安,四輔有誅,必有兵革,天下大赦。」庚午,有流星,大如斗,出左攝提,流至天津滅,有聲如雷。五月癸巳,熒惑犯輿鬼。甲午,犯積尸。占曰:「午,秦也。大臣有誅,兵大起。」後
 三年,太師、大塚宰、晉國公宇文護以不臣誅,皆其應也。



 五年正月乙巳,月在氐,暈,有白虹長丈所貫之,而有兩珥連接,規北斗第四星。占曰:「兵大起,大戰,將軍死於野。」是冬,齊將斛律明月寇邊,於汾北築城,自華谷至於龍門。其明年正月,詔齊公憲率師御之。三月己酉,憲自龍門度河,攻拔其新築五城,兵起大戰之應也。



 六年二月己丑夜,有蒼云,廣三丈,經天,自戌加辰。四月戊寅朔,日有蝕之。



 己卯,熒惑逆行,犯輿鬼。占曰:「有兵喪,大臣誅,兵大起。」其月,又率師取齊宜陽等九城。六月,齊將攻陷汾州。六月庚辰,熒惑太白合,在張宿,相去一尺。



 占曰:「主人兵不勝,所合國有殃。」



 建德元年三月丙辰,熒惑、太白合壁。占曰:「其分有兵喪,不可舉事,用兵必受其殃。」又曰:「改立侯王,有德者興,無德者亡。」其月,誅晉公護、護子譚公會、莒公至、崇業公靜等,大赦。癸亥,詔以齊公憲為大塚宰,是其驗也。七月丙午,辰與太白合於井,相去七寸。占曰;「其下之國,必有重德致天下。」後四年,上帥師平齊,致天下之應也。九月己酉,月犯心中星,相去一寸。占曰:「亂臣在傍,不出五年,下有亡國。」後周武伐齊,平之,有亡國之應也。



 二年二月辛亥,白虹貫日。占曰:「臣謀君,不出三年。」又曰:「
 近臣為亂。」



 後年七月,衛王直在京師舉兵反。癸亥,熒惑掩鬼西北星。占曰:「大賊在大人之側。」又曰:「大臣有誅。」四月己亥,太白掩西北星,壬寅,又掩東北星。占曰:「國有憂,大臣誅。」六月丙辰,月犯心中後二星。占曰:「亂臣在傍,不出三年,有亡國。」又曰:「人主惡之。」九月癸酉,太白犯左執法。占曰:「大臣有憂,執法者誅,若有罪。」十一月壬子,太白掩填星,在尾。占曰:「填星為女主,尾為後宮。」明年皇太后崩。



 三年二月戊午,客星大如桃,青白色,出五車東南三尺所,漸東行,稍長二尺所;至四月壬辰,入文昌;丁未,入北
 斗魁中,後出魁,漸小。凡見九十三日。占曰:「天下兵起,車騎滿野,人主有憂。」又曰:「天下有亂,兵大起,臣謀主。」



 其七月乙酉,衛王直在京師舉兵反,討擒之,廢為庶人。至十月,始州民王鞅擁眾反,討平之。四月乙卯,星孛於紫宮垣外,大如拳,赤白,指五帝座,漸東南行,稍長一丈五尺;五月甲子,至上臺北滅。占曰:「天下易政,無德者亡。」後二年,武帝率六軍滅齊。十一月丙子,歲星與太白相犯,光芒相及,在危。占曰:「其野兵,人主兇,失其城邑。危,齊之分野。」後二年,宇文神舉攻拔陸渾等五城。十二月庚寅,月犯歲星,在危,相去二寸。占曰:「其邦流亡,不出三年。」辛卯,
 月行在營室,食太白。占曰:「其國以兵亡,將軍戰死。營室,衛也,地在齊境。」



 後齊亡入周。



 四年三月甲子,月犯軒轅大星。占曰:「女主有憂,又五官有亂。」



 五年十月庚戌,熒惑犯太微西蕃上將星。占曰:「天下不安,上將誅,若有罪,其止。」



 六年二月,皇太子巡撫西土,仍討吐谷渾。八月,至伏俟城而旋。吐谷渾寇邊,天下不安之應也。六月庚午,熒惑入鬼。占曰:「有喪旱。」其七月,京師旱。十月戊午,歲星犯大陵。又己未、庚申,月連暈,規昴、畢、五車及參。占曰:「兵起爭
 地。」又曰:「王自將兵。」又曰:「天下大赦。」癸亥,帝率眾攻晉州。是日虹見晉州城上,首向南,尾入紫宮,長十餘丈。庚午,克之。丁卯夜,白虹見,長十餘丈,頭在南,尾入紫宮中。占曰:「其下兵戰流血。」又曰:「若無兵,必有大喪。」至六年正月,平齊,與齊軍大戰。十一月稽胡反,齊王討平之。



 六年四月,先此熒惑入太微宮二百日,犯東蕃上相,西蕃上將,句已往還。至此月甲子,出端門。占曰:「為大臣代主。」又曰:「臣不臣,有反者。」又曰:「必有大喪。」後宣、武繼崩,高祖以大運代起。十月癸卯,月食,熒惑在斗。占曰:「國敗,其君亡,兵大起,破軍殺將。斗為吳、越之星,陳之分野。」十一
 月,陳將吳明徹侵呂梁,徐州總管梁士彥出軍與戰,不利。明年三月,郯公王軌討擒陳將吳明徹,俘斬三萬餘人。十一月甲辰,晡時,日中有黑子,大如杯。占曰:「君有過而臣不諫,人主惡之。」十二月癸丑,流星大如月,西流有聲,蛇行屈曲,光照地。占曰:「兵大起,下有戰場。」戌辰平旦,有流星大如三斗器,色赤,出紫宮,凝著天,乃北下。占曰:「人主去其宮殿。」是月,營州刺史高寶寧據州反。



 其明年五月,帝總戎北伐。後年,武帝崩。



 宣政元年正月丙子,月食昴。占曰:「有白衣之會。」又曰:「匈奴侵邊。」



 其月,突厥寇幽州,殺略吏人。五月,帝總戎北伐。
 六月,帝疾甚,還京,次雲陽而崩。六月壬午,癸丑,木火金三星合,在井。占曰:「其國霸。」又曰:「其國外內有兵喪,改立侯王。」是月,幽州人盧昌期據範陽反,改立王侯,兵喪之驗也。



 七年辛丑,月犯心前星。占曰:「太子惡之,若失位。」後靜帝立為天子,不終之徵也。丙辰,熒惑、太白合,在七星,相去二尺八寸所。占曰:「君憂。」又曰:「其國有兵,改立王侯,有德興,無德亡。」後年,改署四輔官,傳位太子,改立王侯之應也。己未,太白犯軒轅大星。占曰:「女主兇。」後二年,宣帝崩,楊后令其父隋公為大丞相,總軍國事。隋氏受命,廢後為樂平公主,餘四後悉廢為比丘尼。八月庚辰,太
 白入太微。占曰:「為天下驚。」又曰:「近臣起兵,大臣相殺,國有憂。」其後,趙、陳等五王為執政所誅,大臣相殺之應也。九月丁酉,熒惑入太微西掖門,庚申,犯左執法,相去三寸。占曰:「天下不安,大臣有憂。」又曰:「執法者誅若有罪。」是月,汾州稽胡反,討平之。十一月,突厥寇邊,圍酒泉,殺略吏人。明年二月,殺柱國、郯公王軌。皆其應也。十二月癸未,熒惑入氐,守犯之三十日。占曰:「天子失其宮。」又曰:「賊臣在內,下有反者。」又曰:「國君有系饑死,若毒死者。」靜帝禪位,隋高祖幽殺之。



 宣帝大成元年正月丙午、癸丑,日皆有背。占曰:「臣為逆,
 有反叛,邊將去之。」又曰:「卿大夫欲為主。」其後,隋公作霸,尉迥、王謙、司馬消難各舉兵反。



 大象元年四月戊子,太白、歲星、辰星合,在井。占曰:「是謂驚立,是謂絕行,其國內外有兵喪,改立王公。」又曰:「其國可霸,修德者強,無德受殃。」



 其五月,趙、陳、越、代、滕五王並入國。後二年,隋王受命,宇文氏宗族相繼誅滅。六月丁卯,有流星一,大如雞子,出氐中,西北流,有尾跡,長一丈所,入月中,即滅。占曰:「不出三年,人主有憂。」又曰:「有亡國。」靜帝幽閉之應也。



 己丑,有流星一,大如斗,色青,有光明照地,出營室,抵壁入濁。七月壬辰,熒惑掩房北頭第一
 星。占曰:「亡君之誡。」又曰:「將軍為亂,王者惡之,大臣有反者,天子憂。」其十二月,帝親御驛馬,日行三百里。四皇后及文武侍衛數百人,並乘馹以從。房為天駟,熒惑主亂,此宣帝亂道德,馳騁車騎,將亡之誡。八月辛巳,熒惑犯南斗第五星。占曰:「且有反臣,道路不通,破軍殺將。」尉迥、王謙等起兵敗亡之徵也。九月己酉,太白入南斗魁中。占曰:「天下有大亂,將相謀反,國易政。」又曰:「君死,不死則疾。」又曰:「天下爵祿。」皆高祖受命、群臣分爵之徵也。十月壬戌,歲星犯軒轅大星。占曰:「女主憂,若失勢。」周自宣政元年,熒惑、太白從歲星聚東井。大象元年四月,太白、歲
 星、辰星又聚井。十月,歲星守軒轅。其年,又守翼。東井,秦分,翼,楚分,漢東為楚地,軒轅後族,隋以後族興於秦地之象,而周之後妃失勢之徵也。乙酉,熒惑在虛,與填星合。占曰:「兵大起,將軍為亂,大人惡之。」是月,相州段德舉謀反,伏誅。其明年三月,杞公宇文亮舉兵反,擒殺之。



 二年四月乙丑,有星大如斗,出天廚,流入紫宮,抵鉤陳乃滅。占曰:「有大喪,兵大起,將軍戮。」又曰:「臣犯上,主有憂。」其五月,帝崩,隋公執國政,大喪、臣犯主之應。趙王、越王以謀執政被誅。又荊、豫、襄三州諸蠻反,尉迥、王謙、司馬消難各舉兵畔,不從執政,終以敗亡。皆大兵起、將軍戮
 之應也。五月甲辰,有流星一,大如三斗器,出太微端門,流入翼,色青白,光明照地,聲若風吹幡旗。占曰:「有立王,若徙王。」又曰:「國失君。」其月己酉,帝崩,劉昉矯制,以隋公受遺詔輔政,終受天命,立王、徙王、失君之應也。七月壬子,歲星、太白合於張,有流星,大如斗,出五車東北流,光明燭地。九月甲申,熒惑、歲星合於翼。



 靜帝大定元年正月乙酉,歲星逆行,守右執法,熒惑掩房北第一星。占曰:「房為明堂,布政之宮,無德者失之。」二月甲子,隋王稱尊號。



 高祖文皇帝開皇元年三月甲申,太白晝見。占曰:「太白
 經天晝見,為臣強,為革政。」四月壬午,歲星晝見。占曰:「大臣強,有逆謀,王者不安。」其後,劉昉等謀反,伏誅。十一月己巳,有流星,聲如隤墻,光燭地。占曰:「流星有光有聲,名曰天保,所墜國安有喜。」其九年,平陳,天下一統。五年八月戊申,有流星數百,四散而下。占曰:「小星四面流行者,庶人流移之象也。」其九年,平陳,江南士人,悉播遷入京師。



 八年二月庚子,填星入東井。占曰:「填星所居有德,利以稱兵。」其年大舉伐陳,克之。十月甲子,有星孛於牽牛。占曰:「臣殺君,天下合謀。」又曰:「內不有大亂,則外有大兵。牛,
 吳、越之星,陳之分野。」後年,陳氏滅。



 九年正月己巳,白虹夾日。占曰:「白虹銜日,臣有背主。」又曰:「人主無德者亡。」是月,滅陳。



 十四年十一月癸未,有彗星孛於虛危及奎婁,齊、魯之分野。其後魯公虞慶則伏法,齊公高熲除名。



 十九年十二月乙未,星隕於渤海。占曰:「陽失其位,災害之萌也。」又曰:「大人憂。」二十年十月,太白晝見。占曰:「大臣強,為革政,為易王。」右僕射楊素,熒惑高祖及獻後,勸廢嫡立庶。其月乙丑,廢皇太子勇為庶人。明年改元。



 皆陽失位及革政易王之
 驗也。



 仁壽四年六月庚午,有星入於月中。占曰:「有大喪,有大兵,有亡國,有破軍殺將。」七月乙未,日青無光,八日乃復。占曰:「主勢奪。」又曰:「日無光,有死王。」甲辰,上疾甚,丁未,宮車晏駕。漢王諒反,楊素討平之。皆兵喪亡國死王之應。



 煬帝大業元年六月甲子,熒惑入太微。占曰:「熒惑為賊,為亂入宮,宮中不安。」



 三年三月辛亥,長星見西方,竟天,干歷奎婁、角亢而沒;至九月辛未,轉見南方,亦竟天,又幹角亢,頻掃太微帝座,干犯列宿,唯不及參、井。經歲乃滅。



 占曰:「去穢布新,天
 所以去無道,建有德,見久者災深,星大者事大,行遲者期遠。兵大起,國大亂而亡。餘殃為水旱饑饉,土功疾疫。」其後,築長城,討吐谷渾及高麗,兵戎歲駕,略無寧息。水旱饑饉疾疫,土功相仍,而有群盜並起,邑落空虛。九年五月,禮部尚書楊玄感於黎陽舉兵反。丁未,熒惑逆行入南斗,色赤如血,如三斗器,光芒震耀,長七八尺,於斗中句巳而行。占曰:「有反臣,道路不通,國大亂,兵大起。」斗,吳、越分野,玄感父封於越,後徙封楚地,又次之,天意若曰,使熒惑句巳之,除其分野。至七月,宇文述討平之。其兄弟悉梟首車裂,斬其黨與數萬人。其年,硃燮、管崇亦
 於吳郡擁眾反。此後群盜屯聚,剽略郡縣,尸橫草野,道路不通,齎詔敕使人,皆步涉夜行,不敢遵路。



 十一年六月,有星孛於文昌東南,長五六寸,色黑而銳,夜動搖,西北行,數日至文昌,去宮四五寸,不入,卻行而滅。占曰:「為急兵。」其八月,突厥圍帝於雁門,從兵悉馮城御寇,矢及帝前。七月,熒惑守羽林。占曰:「衛兵反。」十二月戊寅,大流星如斛,墜賊盧明月營,破其沖輣,壓殺十餘人。占曰:「奔星所墜,破軍殺將。」其年,王充擊盧明月城,破之。



 十二年五月丙戌朔,日有食之,既。占曰:「日食既,人主亡,
 陰侵陽,下伐上。」其後宇文化及等行殺逆。癸巳,大流星隕於吳郡,為石。占曰:「有亡國,有死王,有大戰,破軍殺將。」其後大軍破逆賊劉元進於吳郡,斬之。八月壬子,有大流星如斗,出王良閣道,聲如隕墻;癸丑,大流星如甕,出羽林。九月戊午,有枉矢二,出北斗魁,委曲蛇形,注于南斗。占曰:「主以兵去,天之所伐。」亦曰:「以亂伐亂,執矢者不正。」後二年,化及殺帝僭號,王充亦於東都殺恭帝,篡號鄭。皆殺逆無道,以亂代亂之應也。



 十三年五月辛亥,大流星如甕,墜於江都。占曰:「其下有大兵戰,流血破軍殺將。」六月,有星孛於太微五帝座,色
 黃赤,長三四尺所,數日而滅。占曰:「有亡國,有殺君。」明年三月,宇文化及等殺帝也。十一月辛酉,熒惑犯太微,日光四散如流血。占曰:「賊入宮,主以急兵見伐。」又曰:「臣逆君。」明年三月,化及等殺帝,諸王及幸臣並被戮。



\end{pinyinscope}