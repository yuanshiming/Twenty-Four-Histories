\article{卷七十八列傳第四十三}

\begin{pinyinscope}

 藝術夫陰陽所以正時日,順氣序者也;卜筮所以決嫌疑,定猶豫者也;醫巫所以御妖邪,養性命者也;音律所以和人神,節哀樂者也;相術所以辯貴賤,明分理者也;技巧所以利器用,濟艱難者也。此皆聖人無心,因民設教,救恤災患,禁止淫邪。



 自三五哲王,其所由來久矣。然昔之言陰陽者,則有箕子、裨灶、梓慎、子韋;曉音律者,則師曠、
 師摯、伯牙、杜夔;敘卜筮,則史扁、史蘇、嚴君平、司馬季主;論相術,則內史叔服、姑布子卿、唐舉、許負;語醫,則文摯、扁鵲、季咸、華佗;其巧思,則奚仲、墨翟、張平子、馬德衡。凡此諸君者,仰觀俯察,探賾索隱,咸詣幽微,思侔造化,通靈入妙,殊才絕技。或弘道以濟時,或隱身以利物,深不可測,固無得而稱焉。近古涉乎斯術者,鮮有存夫貞一,多肆其淫僻,厚誣天道。或變亂陰陽,曲成君欲,或假托神怪,熒惑民心。遂令時俗妖訛,不獲返其真性,身罹災毒,莫得壽終而死。藝成而下,意在茲乎?歷觀經史百家之言,無不存夫藝術,或敘其玄妙,或記其迂誕,非徒用
 廣異聞,將以明乎勸戒。是以後來作者,或相祖述,故今亦採其尤著者,列為《藝術篇》云。



 庾季才子質盧太翼耿詢庾季才,字叔奕,新野人也。八世祖滔,隨晉元帝過江,官至散騎常侍,封遂昌侯,因家於南郡江陵縣。祖詵,梁處士,與宗人易齊名。父曼倩,光祿卿。季才幼穎悟,八歲誦《尚書》,十二通《周易》,好占玄象。居喪以孝聞。梁廬陵王績闢荊州主簿,湘東王繹重其術藝,引授外兵參軍。西臺建,累遷中書郎,領太史,封宜昌縣伯。季才固辭太史,元帝曰:「漢司馬遷歷世尸掌,魏高堂隆猶領此職,不無前
 例,卿何憚焉。」帝亦頗明星歷,因共仰觀,從容謂季才曰:「朕猶慮禍起蕭墻,何方可息?」季才曰:「頃天象告變,秦將入郢,陛下宜留重臣,作鎮荊陜,整旆還都,以避其患。假令羯寇侵蹙,止失荊湘,在於社稷,可得無慮。必久停留,恐非天意也。」帝初然之,後與吏部尚書宗懍等議,乃止。俄而江陵陷滅,竟如其言。



 周太祖一見季才,深加優禮,令參掌太史。每有征討,恆預侍從。賜宅一區,水田十頃,並奴婢牛羊什物等,謂季才曰:「卿是南人,未安北土,故有此賜者,欲絕卿南望之心。宜盡誠事我,當以富貴相答。」初,郢都之陷也,衣冠士人多沒為賤。季才散所賜物,
 購求親故。文帝問:「何能若此?」季才曰:「僕聞魏克襄陽,先昭異度,晉平建業,喜得士衡。伐國求賢,古之道也。今郢都覆敗,君信有罪,晉紳何咎,皆為賤隸!鄙人羈旅,不敢獻言,誠切哀之,故贖購耳。」太祖乃悟曰:「吾之過也。微君遂失天下之望!」因出令免梁俘為奴婢者數千口。



 武成二年,與王褒、庾信同補麟趾學士。累遷稍伯大夫、車騎大將軍、儀同三司。其後大塚宰宇文護執政,謂季才曰:「比日天道,有何徵祥?」季才對曰:「荷恩深厚,若不盡言,便同木石。頃上臺有變,不利宰輔,公宜歸政天子,請老私門。此則自享期頤,而受旦、奭之美,子孫籓屏,終保維城
 之固。不然者,非復所知。」護沈吟久之,謂季才曰:「吾本意如此,但辭未獲免耳。公既王官,可依朝例,無煩別參寡人也。」自是漸疏,不復別見。及護滅之後,閱其書記,武帝親自臨檢,有假托符命,妄造異端者,皆致誅戮。唯得季才書兩紙,盛言緯候災祥,宜反政歸權。帝謂少宗伯斛斯徵曰:「庾季才至誠謹愨,甚得人臣之禮。」因賜粟三百石,帛二百段。遷太史中大夫,詔撰《靈臺秘苑》,加上儀同,封臨潁伯,邑六百戶。宣帝嗣位,加驃騎大將軍、開府儀同三司,增邑三百戶。



 及高祖為丞相,嘗夜召季才而問曰:「吾以庸虛,受茲顧命,天時人事,卿以為何如?」季才曰:「
 天道精微,難可意察,切以人事卜之,符兆已定。季才縱言不可,公豈復得為箕、潁之事乎?」高祖默然久之,因舉首曰:「吾今譬猶騎獸,誠不得下矣。」因賜雜彩五十匹,絹二百段,曰:「愧公此意,宜善為思之。」大定元年正月,季才言曰:「今月戊戌平旦,青氣如樓闕,見於國城之上,俄而變紫,逆風西行。《氣經》云:『天不能無雲而雨,皇王不能無氣而立。』今王氣已見,須即應之。二月日出卯入酉,居天之正位,謂之二八之門。日者,人君之象,人君正位,宜用二月。其月十三日甲子,甲為六甲之始,子為十二辰之初,甲數九,子數又九,九為天數。其日即是驚蟄,陽氣壯
 發之時。昔周武王以二月甲子定天下,享年八百,漢高帝以二月甲午即帝位,享年四百,故知甲子、甲午為得天數。今二月甲子,宜應天受命。」上從之。



 開皇元年,授通直散騎常侍。高祖將遷都,夜與高熲、蘇威二人定議,季才旦而奏曰:「臣仰觀玄象,俯察圖記,龜兆允襲,必有遷都。且堯都平陽,舜都冀土,是知帝王居止,世代不同。且漢營此城,經今將八百歲,水皆咸鹵,不甚宜人。願陛下協天人之心,為遷徙之計。」高祖愕然,謂熲等曰:「是何神也!」遂發詔施行,購絹三百段,馬兩匹,進爵為公。謂季才曰:「朕自今已後,信有天道矣。」



 於是令季才與其子質撰《
 垂象》、《地形》等志。上謂季才曰:「天地秘奧,推測多途,執見不同,或致差舛。朕不欲外人干預此事,故使公父子共為之也。」及書成奏之,賜米千石,絹六百段。九年,出為均州刺史。策書始降,將就籓,時議以季才術藝精通,有詔還委舊任。季才以年老,頻表去職,每降優旨不許。會張胄玄歷行,及袁充言日影長。上以問季才,季才因言充謬。上大怒,由是免職,給半祿歸第。所有祥異,常使人就家訪焉。仁壽三年卒,時年八十八。



 季才局量寬弘,術業優博,篤於信義,志好賓游。常吉日良辰,與瑯琊王褒、彭城劉、河東裴政及宗人信等,為文酒之會。次有劉臻、
 明克讓、柳抃之徒,雖為後進,亦申游款。撰《靈臺秘苑》一百二十卷,《垂象志》一百四十二卷,《地形志》八十七卷,並行於世。



 瘐質,字行修,少而明敏,早有志尚。八歲誦梁世祖《玄覽》、《言志》等十賦,拜童子郎。仕周齊煬王記室。開皇元年,除奉朝請,歷鄢陵令,遷隴州司馬。



 大業初,授太史令。操履貞愨,立言忠鯁,每有災異,必指事面陳。而煬帝性多忌刻,齊王暕亦被猜嫌。質子儉時為齊王屬,帝謂質曰:「汝不能一心事我,乃使兒事齊王,何向背如此邪?」質曰:「臣事陛下,子事齊王,實是一心,不敢有二。」



 帝怒不解,由是
 出為合水令。八年,帝親伐遼東,徵詣行在所。至臨渝謁見,帝謂質曰:「朕承先旨,親事高麗,度其土地人民,才當我一郡,卿以為克不?」質對曰:「以臣管窺,伐之可克,切有愚見,不願陛下親行。」帝作色曰:「朕今總兵至此,豈可未見賊而自退也?」質又曰:「陛下若行,慮損軍威。臣猶願安駕住此,命驍將勇士指授規模,倍道兼行,出其不意。事宜在速,緩必無功。」帝不悅曰:「汝既難行,可住此也。」及師還,授太史令。九年,復征高麗,又問質曰:「今段復何如?」對曰:「臣實愚迷,猶執前見。陛下若親動萬乘,糜費實多。」帝怒曰:「我自行尚不能克,直遣人去,豈有成功也!」帝遂行。
 既而禮部尚書楊玄感據黎陽反,兵部侍郎斛斯政奔高麗,帝大懼,遽而西還,謂質曰:「卿前不許我行,當為此耳。今者玄感其成事乎?」質曰:「玄感地勢雖隆,德望非素,因百姓之勞苦,冀僥幸而成功。今天下一家,未易可動。」帝曰:「熒惑入斗如何?」對曰:「斗,楚之分,玄感之所封也。今火色衰謝,終必無成。」十年,帝自西京將往東都,質諫曰:「比歲伐遼,民實勞敝,陛下宜鎮撫關內,使百姓畢力歸農。三五年間,令四海少得豐實,然後巡省,於事為宜。陛下思之。」帝不悅,質辭疾不從。



 帝聞之,怒,遣使馳傳,鎖質詣行在所。至東都,詔令下獄,竟死獄中。



 子儉,亦傳父業,
 兼有學識。仕歷襄武令、元德太子學士、齊王屬。義寧初,為太史令,時有盧太翼、耿詢,並以星歷知名。



 盧太翼,字協昭,河間人也,本姓章仇氏。七歲詣學,日誦數千言,州里號曰神童。及長,閑居味道,不求榮利。博綜群書,爰及佛道,皆得其精微。尤善占候算歷之術。隱於白鹿山,數年徙居林慮山茱萸澗。請業者自遠而至,初無所拒,後憚其煩,逃於五臺山。地多藥物,與弟子數人廬於巖下,蕭然絕世,以為神仙可致。



 皇太子勇聞而召之,太翼知太子必不為嗣,謂所親曰:「吾拘逼而來,不知所稅駕也!」及太子廢,坐法當死,高祖惜其才而不害,配
 為官奴。久之,乃釋。其後目盲,以手摸書而知其字。仁壽末,高祖將避暑仁壽宮,太翼固諫不納,至於再三。



 太翼曰:「臣愚豈敢飾詞,但恐是行鑾輿不反。」高祖大怒,系之長安獄,期還而斬之。高祖至宮寢疾,臨崩,謂皇太子曰:「章仇翼,非常人也,前後言事,未嘗不中。吾來日道當不反,今果至此,爾宜釋之。」及煬帝即位,漢王諒反,帝以問之。答曰:「上稽玄象,下參人事,何所能為?」未幾,諒果敗。帝常從容言及天下氏族,謂太翼曰:「卿姓章仇,四岳之胄,與盧同源。」於是賜姓為盧氏。大業九年,從駕至遼東,太翼言於帝曰:「黎陽有兵氣。」後數日而玄感反書聞,帝甚
 異之,數加賞賜。太翼所言天文之事,不可稱數,關諸秘密,世莫得聞。後數載,卒於洛陽。



 耿詢,字敦信,丹陽人也。滑稽辯給,伎巧絕人。陳後主之世,以客從東衡州刺史王勇於嶺南。勇卒,詢不歸,遂與諸越相結,皆得其歡心。會郡俚反叛,推詢為主。柱國王世積討擒之,罪當誅。自言有巧思,世積釋之,以為家奴。久之,見其故人高智寶以玄象直太史,詢從之受天文算術。詢創意造渾天儀,不假人力,以水轉之,施於暗室中,使智寶外候天時,合如符契。世積知而奏之,高祖配詢為官奴,給使太史局。後賜蜀王秀,從往益州,秀甚信
 之。及秀廢,復當誅,何稠言於高祖曰:「耿詢之巧,思若有神,臣誠為朝廷惜之。」上於是特原其罪。詢作馬上刻漏,世稱其妙。煬帝即位,進欹器,帝善之,放為良民。歲餘,授右尚方署監事。



 七年,車駕東征,詢上書曰:「遼東不可討,師必無功。」帝大怒,命左右斬之,何稠苦諫得免。及平壤之敗,帝以詢言為中,以詢守太史丞。宇文化及弒逆之後,從至黎陽,謂其妻曰:「近觀人事,遠察天文,宇文必敗,李氏當王,吾知所歸矣。」



 詢欲去之,為化及所殺。著《鳥情占》一卷,行於世。



 韋鼎
 韋鼎,字超盛,京兆杜陵人也。高祖玄,隱於商山,因而歸宋。祖睿,梁開府儀同三司。父正,黃門侍郎。鼎少通脫,博涉經史,明陰陽逆刺,尤善相術。仕梁,起家湘東王法曹參軍。遭父憂,水漿不入口者五日,哀毀過禮,殆將滅性。服闋,為邵陵王主簿。侯景之亂,鼎兄昂卒於京城,鼎負尸出,寄於中興寺。求棺無所得,鼎哀憤慟哭,忽見江中有物,流至鼎所,鼎切異之。往見,乃新棺也,因以充殮。



 元帝聞之,以為精誠所感。侯景平,司徒王僧辯以為戶曹屬,歷太尉掾、大司馬從事、中書侍郎。



 陳武帝在南徐州,鼎望氣知其當王,遂寄孥焉。因謂陳武帝曰:「明年有大
 臣誅死,後四歲,梁其代終,天之歷數當歸舜後。昔周滅殷氏,封媯滿於宛丘,其裔子孫因為陳氏。僕觀明公天縱神武,繼絕統者,無乃是乎!」武帝陰有圖僧辯意,聞其言,大喜,因而定策。及受禪,拜黃門侍郎,俄遷司農卿、司徒右長史、貞威將軍,領安右晉安王長史、行府國事,轉廷尉卿。太建中,為聘周主使,加散騎常侍。尋為秘書監、宣遠將軍,轉臨海王長史,行吳興郡事。入為太府卿。至德初,鼎盡質貨田宅,寓居僧寺。友人大匠卿毛彪問其故,答曰:「江東王氣盡於此矣。



 吾與爾當葬長安。期運將及,故破產耳。」



 初,鼎之聘周也,嘗與高祖相遇,鼎謂高祖
 曰:「觀公容貌,故非常人,而神監深遠,亦非群賢所逮也。不久必大貴,貴則天下一家,歲一周天,老夫當委質。



 公相不可言,願深自愛。」及陳平,上馳召之,授上儀同三司,待遇甚厚。上每與公王宴賞,鼎恆預焉。高祖嘗從容謂之曰:「韋世康與公相去遠近?」鼎對曰:「臣宗族分派,南北孤絕,自生以來,未嘗訪問。」帝曰:「公百世卿族,何得爾也。」乃命官給酒肴,遣世康與鼎還杜陵,樂飲十餘日。鼎乃考校昭穆,自楚太傅孟以下二十餘世,作《韋氏譜》七卷。時蘭陵公主寡,上為之求夫,選親衛柳述及蕭瑒等以示於鼎。鼎曰:「瑒當封侯,而無貴妻之相,述亦通顯,而守
 位不終。」



 上曰:「位由我耳。」遂以主降述。上又問鼎:「諸兒誰得嗣?」答曰:「至尊、皇后所最愛者,即當與之,非臣敢預知也。」上笑曰:「不肯顯言乎?」



 開皇十二年,除光州刺史,以仁義教導,務弘清靜。州中有土豪,外修邊幅,而內行不軌,常為劫盜。鼎於都會時謂之曰:「卿是好人,那忽作賊?」因條其徒黨謀議逗留,其人驚懼,即自首伏。又有人客游,通主家之妾,及其還去,妾盜珍物,於夜亡,尋於草中為人所殺。主家知客與妾通,因告客殺之。縣司鞫問,具得奸狀,因斷客死。獄成,上於鼎,鼎覽之曰:「此客實奸,而殺非也。乃某寺僧詃妾盜物,令奴殺之,贓在某處。」即放
 此客,遣掩僧,並獲贓物。自是部內肅然不言,咸稱其有神,道無拾遺。尋追入京,以年老多病,累加優賜。頃之,卒,年七十九。



 來和來和,字弘順,京兆長安人也。少好相術,所言多驗。大塚宰宇文護引之左右,由是出入公卿之門。初為夏官府下士,累遷少卜上士,購爵安定鄉男。遷畿伯下大夫,進封洹水縣男。高祖微時,來詣和相,和待人去,謂高祖曰:「公當王有四海。」



 及為丞相,拜儀同,既受禪,進爵為子。開皇末,和上表自陳曰:臣早奉龍顏,自周代天和三年已
 來,數蒙陛下顧問,當時具言至尊膺圖受命,光宅區宇。此乃天授,非由人事所及。臣無勞效,坐致五品,二十餘年。臣是何人,敢不慚懼!愚臣不任區區之至,謹錄陛下龍潛之時,臣有所言一得,書之秘府,死無所恨。昔陛下在周,嘗與永富公竇榮定語臣曰:「我聞有行聲,即識其人。」臣當時即言公眼如曙星,無所不照,當王有天下,願忍誅殺。建德四年五月,周武帝在雲陽宮,謂臣曰:「諸公皆汝所識,隋公相祿何如?」臣報武帝曰:「隋公止是守節人,可鎮一方。若為將領,陳無不破。」臣即於宮東南奏聞。陛下謂臣,此語不忘。明年,烏丸軌言於武帝曰:「隋公非
 人臣。」帝尋以問臣,臣知帝有疑,臣詭報曰:「是節臣,更無異相。」於時王誼、梁彥光等知臣此語。大象二年五月,至尊從永巷東門入,臣在永巷門東,北面立,陛下問臣曰:「我無災障不?」臣奏陛下曰:「公骨法氣色相應,天命已有付屬。」未幾,遂總百揆。



 上覽之大悅,進位開府,購物五百段,米三百石,地十頃。



 和同郡韓則,嘗詣和相,和謂之曰:「後四五當得大官。」人初不知所謂。則至開皇十五年五月而終,人問其故,和曰:「十五年為三五,加以五月為四五。大官,槨也。」和言多此類。著《相經》四十卷。



 道士張賓、焦子順、雁門人董子華,此三人,當高祖龍潛時,並私謂高
 祖曰:「公當為天子,善自愛。」及踐阼,以賓為華州刺史,子順為開府,子華為上儀同。



 蕭吉楊伯醜臨孝恭劉祐蕭吉,字文休,梁武帝兄長沙宣武王懿之孫也。博學多通,尤精陰陽算術。江陵陷,遂歸於周,為儀同。宣帝時,吉以朝政日亂,上書切諫。帝不納。及隋受禪,進上儀同,以本官太常考定古今陰陽書。吉性孤峭,不與公卿相沉浮,又與楊素不協,由是擯落於世,鬱鬱不得志。見上好徵祥之說,欲乾沒自進,遂矯其跡為悅媚焉。開皇十四年上書曰:「今年歲在甲寅,十一月朔旦,以辛酉為冬至。
 來年乙卯,正月朔旦,以庚申為元日,冬至之日,即在朔旦。《樂汁圖徵》云:『天元十一月朔旦冬至,聖王受享祚。』今聖主在位,居天元之首,而朔旦冬至,此慶一也。辛酉之日,即是至尊本命,辛德在丙,此十一月建丙子。酉德在寅,正月建寅為本命,與月德合,而居元朔之首,此慶二也。庚申之日,即是行年,乙德在庚,卯德在申,來年乙卯,是行年與歲合德,而在元旦之朝,此慶三也。《陰陽書》云:『年命與歲月合德者,必有福慶。』《洪範傳》云:『歲之朝,月之朝,日之朝,主王者。』經書並謂三長應之者,延年福吉。況乃甲寅部首,十一月陽之始,朔旦冬至,是聖王上元。正
 月是正陽之月,歲之首,月之先。朔旦是歲之元,月之朝,日之先,嘉辰之會。而本命為九元之先,行年為三長之首,並與歲月合德。所以《靈寶經》云:『角音龍精,其祚日強。』來歲年命納音俱角,歷之與經,如合符契。又甲寅、乙卯,天地合也,甲寅之年,以辛酉冬至,來年乙卯,以甲子夏至。冬至陽始,郊天之日,即是至尊本命,此慶四也。夏至陰始,祀地之辰,即是皇后本命,此慶五也。



 至尊德並乾之覆育,皇後仁同地之載養,所以二儀元氣,並會本辰。」上覽之大悅,賜物五百段。



 房陵王時為太子,言東宮多鬼巉,鼠妖數見。上令吉詣東宮,禳邪氣。於宣慈殿設神
 坐,有回風從艮地鬼門來,掃太子坐。吉以桃湯葦火驅逐之,風出宮門而止。



 又謝土,於未地設壇,為四門,置五帝坐。於時至寒,有楎蟆從西南來,入人門,升赤帝坐,還從人門而出。行數步,忽然不見。上大異之,賞賜優洽。又上言太子當不安位,時上陰欲廢立,得其言是之。由此每被顧問。



 及獻皇后崩,上令吉卜擇葬所,吉歷筮山原,至一處,云「卜年二千,卜世二百」,具圖而奏之。上曰:「吉兇由人,不在於地。高緯父葬,豈不卜乎?國尋滅亡。正如我家墓田,若云不吉,朕不當為天子;若云不兇,我弟不當戰沒。」然竟從吉言。吉表曰:「去月十六日,皇后山陵西北,
 雞未鳴前,有黑雲方圓五六百步,從地屬天。東南又有旌旗車馬帳幕,布滿七八里,並有人往來檢校,部伍甚整,日出乃滅,同見者十餘人。謹案《葬書》云:『氣王與姓相生,大吉。』今黑氣當冬王,與姓相生,是大吉利,子孫無疆之候也。」上大悅。其後上將親臨發殯,吉復奏上曰:「至尊本命辛酉,今歲斗魁及天岡,臨卯酉,謹按《陰陽書》,不得臨喪。」



 上不納。退而告族人蕭平仲曰:「皇太子遣宇文左率深謝餘云:『公前稱我當為太子,竟有其驗,終不忘也。今卜山陵,務令我早立。我立之後,當以富貴相報。』吾記之曰:『後四載,太子御天下。』今山陵氣應,上又臨喪,兆益
 見矣。且太子得政,隋其亡乎!當有真人出治之矣。吾前紿云卜年二千者,是三十字也;卜世二百者,取三十二運也。吾言信矣,汝其志之。」



 及煬帝嗣位,拜太府少卿,加位開府。嘗行經華陰,見楊素塚上有白氣屬天,密言於帝。帝問其故,吉曰:「其候素家當有兵禍,滅門之象。改葬者,庶可免乎!』帝後從容謂楊玄感曰:「公家宜早改葬。」玄感亦微知其故,以為吉祥,托以遼東未滅,不遑私門之事。未幾而玄感以反族滅,帝彌信之。後歲餘,卒官。著《金海》三十卷,《相經要錄》一卷,《宅經》八卷,《葬經》六卷,《樂譜》二十卷及《帝王養生方》二卷,《相手版要決》一卷,《太一立成》
 一卷,並行於世。



 時有楊伯醜,臨孝恭、劉祐,俱以陰陽術數知名。



 楊伯醜,馮翊武鄉人也。好讀《易》,隱於華山。開皇初,被徵入朝,見公卿不為禮,無貴賤皆汝之。人不能測也。高祖召與語,竟無所答。上賜之衣服,至朝堂,舍之而去。於是被發陽狂,游行市里,形體垢穢,未嘗櫛沐。嘗有張永樂者,賣卜京師,伯醜每從之游。永樂為卦有不能決者,伯丑輒為分析爻象,尋幽入微。



 永樂嗟服,自以為非所及也。伯醜亦開肆賣卜。有人嘗失子,就伯醜筮者。卦成,伯醜曰:「汝子在懷遠坊南門道東北壁上,有青裙女子抱
 之,可往取也。」如言果得。或者有金數兩,夫妻共藏之,於後失金,其夫意妻有異志,將逐之。其妻稱冤,以詣伯醜,為筮之曰:「金在矣。」悉呼其家人,指一人曰:「可取金來!」其人赧然,應聲而取之。道士韋知常詣伯丑問吉兇,伯醜曰:「汝勿東北行,必不得已,當早還。不然者,楊素斬汝頭。」未幾,上令知常事漢王諒。俄而上崩,諒舉兵反,知常逃歸京師。知常先與楊素有隙,及素平並州,先訪知常,將斬之,賴此獲免。



 又人有失馬,來詣伯醜卜者。時伯醜為皇太子所召,在途遇之,立為作卦,卦成,曰:「我不遑為卿占之,卿且向西市東壁門南第三店,為我買魚作膾,當
 得馬矣。」



 其人如此言,須臾,有一人牽所失馬而至,遂擒之。崖州嘗獻徑寸珠,其使者陰易之,上心疑焉,召伯醜令筮。伯醜曰:「有物出自水中,質圓而色光,是大珠也。



 今為人所隱。」具言隱者姓名容狀。上如言簿責之,果得本珠。上奇之,賜帛二十匹。國子祭酒何妥嘗詣之論《易》,聞妥之言,倏然而笑曰:「何用鄭玄、王弼之言乎!」久之,微有辨答,所說辭義,皆異先儒之旨,而思理玄妙,故論者以為天然獨得,非常人所及也。竟以壽終。



 臨孝恭,京兆人也。明天文算術,高祖甚親遇之。每言災祥之事,未嘗不中,上因令考定陰陽。官至上儀同。著《欹
 器圖》三卷,《地動銅儀經》一卷,《九宮五墓》一卷,《遁甲月令》十卷,《元辰經》十卷,《元辰厄》一百九卷,《百怪書》十八卷,《祿命書》二十卷,《九宮龜經》一百一十卷,《太一式經》三十卷,《孔子馬頭易卜書》一卷,並行於世。



 劉祐,滎陽人也。開皇初,為大都督,封索盧縣公。其所占候,合如符契,高祖甚親之。初與張賓、劉暉、馬顯定歷。後奉詔撰兵書十卷,名曰《金韜》,上善之。復著《陰策》二十卷,《觀臺飛候》六卷,《玄象要記》五卷,《律歷術文》一卷,《婚姻志》三卷,《產乳志》二卷,《式經》四卷,《四時立成法》一卷,《安歷志》十二卷,《歸正易》十卷,並行於世。



 張胄玄張胄玄,渤海蓚人也。博學多通,尤精術數。冀州刺史趙煚薦之,高祖征授雲騎尉,直太史,參議律歷事。時輩多出其下,由是太史令劉暉等甚忌之。然暉言多不中,胄玄所推步甚精密,上異之。令楊素與術數人立議六十一事,皆舊法久難通者,令暉與胄玄等辯析之。暉杜口一無所答,胄玄通者五十四焉。由是擢拜員外散騎侍郎,兼太史令,賜物千段,暉及黨與八人皆斥逐之。改定新歷,言前歷差一日,內史通事顏敏楚上言曰:「漢時落下閎改《顓頊歷》作《太初歷》,云後當差一日。



 八百年當有
 聖者定之。計今相去七百一十年,術者舉其成數,聖者之謂,其在今乎!」



 上大悅,漸見親用。



 胄玄所為歷法,與古不同者有三事:其一,宋祖沖之於歲周之末,創設差分,冬至漸移,不循舊軌。每四十六年,卻差一度。至梁虞廣刂歷法,嫌沖之所差太多,因以一百八十六年冬至移一度。胄玄以此二術,年限懸隔,追檢古注,所失極多,遂折中兩家,以為度法。冬至所宿,歲別漸移,八十三年卻行一度,則上合堯時日永星火,次符漢歷宿起牛初。明其前後,並皆密當。



 其二,周馬顯造《丙寅元歷》,有陰陽轉法,加減章分,進退蝕餘,乃推定日,創開此數。當時術者,多
 不能曉。張賓因而用之,莫能考正。胄玄以為加時先後,逐氣參差,就月為斷,於理未可。乃因二十四氣列其盈縮所出,實由日行遲則月逐日易及,令合朔加時早,日行速則月逐日少遲,令合朔加時晚。檢前代加時早晚,以為損益之率。日行自秋分已後至春分,其勢速,計一百八十二日而行一百八十度。



 自春分已後至秋分,日行遲,計一百八十二日而行一百七十六度。每氣之下,即其率也。



 其三,自古諸歷,朔望值交,不問內外,入限便食。張賓立法,創有外限,應食不食,猶未能明。胄玄以日行黃道,歲一周天,月行月道,二十七日有餘一周天。



 月
 道交絡黃道,每行黃道內十三日有奇而出,又行黃道外十三日有奇而入,終而復始,月經黃道,謂之交,朔望去交前後各十五度已下,即為當食。若月行內道,則在黃道之北,食多有驗。月行外道,在黃道之南也,雖遇正交,無由掩映,食多不驗。遂因前法,別立定限,隨交遠近,逐氣求差,損益食分,事皆明著。



 其超古獨異者有七事:其一,古歷五星行度皆守恆率,見伏盈縮,悉無格準。胄玄推之,各得其真率,合見之數,與古不同。其差多者,至加減三十許日。即如熒惑平見在雨水氣,即均加二十九日,見在小雪氣,則均減二十五日。雖減平見,以為定
 見。諸星各有盈縮之數,皆如此例,但差數不同。特其積候所知,時人不能原其意旨。



 其二,辰星舊率,一終再見,凡諸古歷,皆以為然,應見不見,人未能測。胄玄積候,知辰星一終之中,有時一見,及同類感召,相隨而出。即如辰星平晨見在雨水氣者,應見即不見,若平晨見在啟蟄氣者,去日十八度外,三十六度內,晨有木火土金一星者,亦相隨見。



 其三,古歷步術,行有定限,自見已後,依率而推。進退之期,莫知多少。胄玄積候,知五星遲速留退真數皆與古法不同,多者至差八十餘日,留回所在亦差八十餘度。即如熒惑前疾初見在立冬初,則二百五
 十日行一百七十七度,定見在夏至初,則一百七十日行九十二度。追步天驗,今古皆密。



 其四,古歷食分,依平即用,推驗多少,實數罕符。胄玄積候,知月從木、火、土、金四星行有向背。月向四星即速,背之則遲,皆十五度外,乃循本率。遂於交分,限其多少。



 其五,古歷加時,朔望同術。胄玄積候,知日食所在,隨方改變,傍正高下,每處不同。交有淺深,遲速亦異,約時立差,皆會天象。



 其六,古歷交分即為食數,去交十四度者食一分,去交十三度食二分,去交十度食三分。每近一度,食益一分,當交即食既。其應少反多,應多反少,自古諸歷,未悉其原。胄玄積
 候,知當交之中,月掩日不能畢盡,其食反少,去交五六時,月在日內,掩日便盡,故食乃既。自此已後,更遠者其食又少。交之前後在冬至皆爾。



 若近夏至,其率又差。所立食分,最為詳密。



 其七,古歷二分,晝夜皆等。胄玄積候,知其有差,春秋二分,晝多夜漏半刻,皆由日行遲疾盈縮使其然也。



 凡此胄玄獨得於心,論者服其精密。大業中卒官。



 許智藏許智藏,高陽人也。祖道幼,嘗以母疾,遂覽醫方,因而究極,世號名醫。誡其諸子曰:「為人子者,嘗膳視藥,不知方
 術,豈謂孝乎?」由是世相傳授。仕梁,官至員外散騎侍郎。父景,武陵王諮議參軍。智藏少以醫術自達,仕陳為散騎侍郎。



 及陳滅,高祖以為員外散騎侍郎,使詣揚州。會秦孝王俊有疾,上馳召之。俊夜中夢其亡妃崔氏泣曰:「本來相迎,比聞許智藏將至,其人若到,當必相苦,為之奈何?」明夜,俊又夢崔氏曰:「妾得計矣,當入靈府中以避之。」及智藏至,為俊診脈,曰:「疾已入心,郎當發巘,不可救也。」果如言,俊數日而薨。上奇其妙,賚物百段。煬帝即位,智藏時致仕於家,帝每有所苦,輒令中使就詢訪,或以蒐迎入殿,扶登御床。智藏為方奏之,用無不效。年八十,
 卒於家。



 宗人許澄,亦以醫術顯。父奭,仕梁太常丞、中軍長史。隨柳仲禮入長安,與姚僧垣齊名,拜上儀同三司。澄有學識,傳父業,尤盡其妙。歷尚藥典御、諫議大夫,封賀川縣伯。父子俱以藝術名重於周、隋二代。史失事,故附見雲。



 萬寶常王令言萬寶常,不知何許人也。父大通,從梁將王琳歸於齊。後復謀還江南,事洩,伏誅。由是寶常被配為樂戶,因而妙達鐘律,遍工八音。造玉磬以獻於齊。又嘗與人方食,論及聲調。時無樂器,寶常因取前食器及雜物,以箸扣之,
 品其高下,宮商畢備,諧於絲竹,大為時人所賞。然歷周洎隋,俱不得調。開皇初,沛國公鄭譯等定樂,初為黃鐘調。寶常雖為伶人,譯等每召與議,然言多不用。後譯樂成奏之,上召寶常,問其可不,寶常曰:「此亡國之音,豈陛下之所宜聞!」上不悅。寶常因極言樂聲哀怨淫放,非雅正之音,請以水尺為律,以調樂器。上從之。寶常奉詔,遂造諸樂器,其聲率下鄭譯調二律。並撰《樂譜》六十四卷,具論八音旋相為宮之法,改弦移柱之變。為八十四調,一百四十四律,變化終於一千八百聲。時人以《周禮》有旋宮之義,自漢、魏已來,知音者皆不能通,見寶常特創
 其事,皆哂之。



 至是,試令為之,應手成曲,無所凝滯,見者莫不嗟異。於是損益樂器,不可勝紀,其聲雅淡,不為時人所好,太常善聲者多排毀之。又太子洗馬蘇夔以鐘律自命,尤忌寶常。夔父威,方用事,凡言樂者,皆附之而短寶常。數詣公卿怨望,蘇威因詰寶常,所為何所傳受。有一沙門謂寶常曰:「上雅好符瑞,有言徵祥者,上皆悅之。



 先生當言就胡僧受學,云是佛家菩薩所傳音律,則上必悅。先生所為,可以行矣。」



 寶常然之,遂如其言以答威。威怒曰:「胡僧所傳,乃是四夷之樂,非中國所宜行也。」其事竟寢。寶常嘗聽太常所奏樂,泫然而泣。人問其故,
 寶常曰:「樂聲淫厲而哀,天下不久相殺將盡。」時四海全盛,聞其言者皆謂為不然。大業之末,其言卒驗。



 寶常貧無子,其妻因其臥疾,遂竊其資物而逃。寶常饑餒,無人贍遺,竟餓而死。將死也,取其所著書而焚之,曰:「何用此為?」見者於火中探得數卷,見行於世,時論哀之。



 開皇之世,有鄭譯、何妥、盧賁、蘇夔、蕭吉,並討論墳籍,撰著樂書,皆為當世所用。至於天然識樂,不及寶常遠矣。安馬駒、曹妙達、王長通、郭令樂等,能造曲,為一時之妙,又習鄭聲,而寶常所為,皆歸於雅。此輩雖公議不附寶常,然皆心服,謂以為神。



 時有樂人王令言,亦妙達音律。大業末,
 煬帝將幸江都,令言之子嘗從,於戶外彈胡琵琶,作翻調《安公子曲》。令言時臥室中,聞之大驚,蹶然而起曰:「變,變!」急呼其子曰:「此曲興自早晚?」其子對曰:「頃來有之。」令言遂歔欷流涕,謂其子曰:「汝慎無從行,帝必不返。」子問其故,令言曰:「此曲宮聲往而不反,宮者君也,吾所以知之。」帝竟被殺於江都。



 史臣曰:陰陽卜祝之事,聖人之教在焉,雖不可以專行,亦不可得而廢也。人能弘道,則博利時俗,行非其義,則咎悔及身,故昔之君子所以戒乎妄作。今韋、來之骨法氣色,庾、張之推步盈虛,雖落下、高堂、許負、硃建,不能尚
 也。伯醜龜策,近知鬼神之情,耿詢渾儀,不差辰象之度,寶常聲律,動應宮商之和,雖不足遠擬古人,皆一時之妙也。許氏之運針石,世載可稱,蕭吉之言陰陽,近於誣誕矣。



\end{pinyinscope}