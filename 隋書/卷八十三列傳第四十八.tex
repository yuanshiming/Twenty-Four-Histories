\article{卷八十三列傳第四十八}

\begin{pinyinscope}

 西域漢氏初開西域,有三十六國,其後分立五十五王,置校尉、都護以撫納之。王莽篡位,西域遂絕。至於後漢,班超所通者五十餘國,西至西海,東西四萬里,皆來朝貢,復置都護、校尉以相統攝。其後或絕或通,漢朝以為勞弊中國,其官時廢時置。暨魏、晉之後,互相吞滅,不可詳焉。煬帝時,遣侍御史韋節、司隸從事杜行滿使於西蕃諸國。至罽賓,得碼杯,王舍城,得佛經,史國得十舞女、師
 子皮、火鼠毛而還。帝復令聞喜公裴矩於武威、張掖間往來以引致之。其有君長者四十四國。矩因其使者入朝,啖以厚利,令其轉相諷諭。大業年中,相率而來朝者三十餘國,帝因置西域校尉以應接之。尋屬中國大亂,朝貢遂絕。然事多亡失,今所存錄者,二十國焉。



 吐谷渾吐谷渾,本遼西鮮卑徒河涉歸子也。初,涉歸有二子,庶長曰吐谷渾,少曰若洛廆。涉歸死,若洛廆代統部落,是為慕容氏。吐谷渾與若洛廆不協,遂西度隴,止於甘松之南,洮水之西,南極白蘭山,數千里之地,其後遂以吐
 谷渾為國氏焉。



 當魏、周之際,始稱可汗。都伏俟城,在青海西十五里。有城郭而不居,隨逐水草。



 官有王公、僕射、尚書、郎中、將軍。其主以皁為帽,妻戴金花。其器械衣服略與中國同。其王公貴人多戴,婦人裙襦辮發,綴以珠貝。國無常稅。殺人及盜馬者死,餘坐則徵物以贖罪。風俗頗同突厥。喪有服制,葬訖而除。性皆貪忍。有大麥、粟、豆。青海周回千餘里,中有小山,其俗至冬輒放牝馬於其上,言得龍種。



 吐谷渾嘗得波斯草馬,放入海,因生驄駒,能日行千里,故時稱青海驄焉。多犛牛,饒銅、鐵、硃砂。地兼鄯善、且末。西北有流沙數百里,夏有熱風,傷
 斃行旅。風之將至,老駝預知之,則引項而鳴,聚立,以口鼻埋沙中。人見則知之,以氈擁蔽口鼻而避其患。



 其主呂誇,在周數為邊寇,及開皇初,以兵侵弘州。高祖以弘州地曠人梗,因而廢之,遣上柱國元諧率步騎數萬擊之。賊悉發國中兵,自曼頭至於樹敦,甲騎不絕。其所署河西總管、定城王鐘利房及其太子可博汗,前後來拒戰。諧頻擊破之,俘斬甚眾。呂誇大懼,率其親兵遠遁。其名王十三人,各率部落而降。上以其高寧王移茲裒素得眾心,拜為大將軍,封河南王,以統降眾,自餘官賞各有差。未幾,復來寇邊,旭州刺史皮子信出兵拒戰,為賊
 所敗,子信死之。汶州總管梁遠以銳卒擊之,斬千餘級,奔退。俄而入寇廓州,州兵擊走之。



 呂誇在位百年,屢因喜怒廢其太子而殺之。其後太子懼見廢辱,遂謀執呂誇而降,請兵於邊吏。秦州總管、河間王弘請將兵應之,上不許。太子謀洩,為其父所殺,復立其少子嵬王訶為太子。疊州刺史杜粲請因其釁而討之,上又不許。六年,嵬王訶復懼其父誅之,謀率部落萬五千人戶將歸國,遣使詣闕,請兵迎接。上謂侍臣曰:「渾賊風俗,特異人倫,父既不慈,子復不孝。朕以德訓人,何有成其惡逆也!吾當教之以養方耳。」乃謂使者曰:「朕受命於天,撫育四海,
 望使一切生人皆以仁義相向。況父子天性,何得不相親愛也!吐谷渾主既是嵬王之父,嵬王是吐谷渾主太子,父有不是,子須陳諫。若諫而不從,當令近臣親戚內外諷諭。必不可,泣涕而道之。人皆有情,必當感悟。不可潛謀非法,受不孝之名。溥天之下,皆是朕臣妾,各為善事,即稱朕心。嵬王既有好意,欲來投朕,朕唯教嵬王為臣子之法,不可遠遣兵馬,助為惡事。」嵬王乃止。八年,其名王拓拔木彌請以千餘家歸化。



 上曰:「溥天之下,皆曰朕臣,雖復荒遐,未識風教,朕之撫育,俱以仁孝為本。



 渾賊昏狂,妻子懷怖,並思歸化,自救危亡。然叛夫背父,不
 可收納。又其本意,正自避死,若今遣拒,又復不仁。若更有意信,但宜慰撫,任其自拔,不須出兵馬應接之。其妹夫及甥欲來,亦任其意,不勞勸誘也。」是歲河南王移茲裒死,高祖令其弟樹歸襲統其眾。平陳之後,呂誇大懼,遁逃保險,不敢為寇。



 十一年,呂誇卒,子伏立。使其兄子無素奉表稱籓,並獻方物,請以女備後庭。



 上謂滕王曰:「此非至誠,但急計耳。」乃謂無素曰:「朕知渾主欲令女事朕,若依來請,他國聞之,便當相學。一許一塞,是謂不平。若並許之,又非好法。朕情存安養,欲令遂性,豈可聚斂子女以實後宮乎?」竟不許。十二年,遣刑部尚書宇文弼
 撫慰之。十六年,以光化公主妻伏,伏上表稱公主為天後,上不許。明年,其國大亂,國人殺伏,立其弟伏允為主。使使陳廢立之事,並謝專命之罪,且請依俗尚主,上從之。自是朝貢歲至,而常訪國家消息,上甚惡之。



 煬帝即位,伏允遣其子順來朝。時鐵勒犯塞,帝遣將軍馮孝慈出敦煌以御之,孝慈戰不利。鐵勒遣使謝罪,請降,帝遣黃門侍郎裴矩慰撫之,諷令擊吐谷渾以自效。鐵勒許諾,即勒兵襲吐谷渾,大敗之。伏允東走,保西平境。帝復令觀王雄出澆河、許公宇文述出西平以掩之,大破其眾。伏允遁逃,部落來降者十萬餘口。六畜三十餘萬。述
 追之急,伏允懼,南遁於山谷間。其故地皆空,自西平臨羌城以西,且末以東,祁連以南,雪山以北,東西四千里,南北二千里,皆為隋有。置郡縣鎮戍,發天下輕罪徙居之。於是留順不之遣。伏允無以自資,率其徒數千騎客於黨項。



 帝立順為主,送出玉門,令統餘眾,以其大寶王尼洛周為輔。至西平,其部下殺洛周,順不果入而還。大業末,天下大亂,伏允復其故地,屢寇河右,郡縣不能御焉。



 黨項黨項羌者,三苗之後也。其種有宕昌、白狼,皆自稱獼猴種。東接臨洮、西平,西拒葉護,南北數千里,處山谷間。每
 姓別為部落,大者五千餘騎,小者千餘騎。



 織犛牛尾及𦍩䍽毛以為屋。服裘褐,披氈,以為上飾。俗尚武力,無法令,各為生業,在戰陣則相屯聚。無徭賦,不相往來。牧養犛牛、羊、豬以供食,不知稼穡。其俗淫穢蒸報,於諸夷中最為甚。無文字,但候草木以記歲時。三年一聚會,殺牛羊以祭天。人年八十以上死者,以為令終,親戚不哭,少而死者,則云大枉,共悲哭之。有琵琶、橫吹、擊缶為節。魏、周之際,數來擾邊。高祖為丞相時,中原多故,因此大為寇掠。蔣公梁睿既平王謙,請因還師以討之,高祖不許。開皇四年,有千餘家歸化。五年,拓拔寧叢等各率眾詣
 旭州內附,授大將軍,其部下各有差。十六年,復寇會州,詔發隴西兵以討之,大破其眾。又相率請降,願為臣妾,遣子弟入朝謝罪。高祖謂之曰:「還語爾父兄,人生須有定居,養老長幼。而乃乍還乍走,不羞鄉里邪!」自是朝貢不絕。



 高昌高昌國者,則漢車師前王庭也,去敦煌十三日行。其境東西三百里,南北五百里,四面多大山。昔漢武帝遣兵西討,師旅頓敝,其中尤困者因住焉。其地有漢時高昌壘,故以為國號。初,蠕蠕立闞伯周為高昌王。伯周死,子
 義成立,為從兄首歸所殺,首歸自立為高昌王,又為高車阿伏至羅所殺,以敦煌人張孟明為主。孟明為國人所殺,更以馬儒為王,以鞏顧、麴嘉二人為左右長史。儒又通使後魏,請內屬。內屬人皆戀土,不顧東遷,相與殺儒,立嘉為王。嘉字靈鳳,金城榆中人,既立,又臣於茹茹。及茹茹主為高車所殺,嘉又臣於高車。屬焉耆為挹怛所破,眾不能自統,請主於嘉。嘉遣其第二子為焉耆王,由是始大,益為國人所服。嘉死,子堅立。



 其都城周回一千八百四十步,於坐室畫魯哀公問政於孔子之像。國內有城十八。



 官有令尹一人,次公二人,次左右衛,次八
 長史,次五將軍,次八司馬,次侍郎、校郎、主簿、從事、省事。大事決之於王,小事長子及公評斷,不立文記。男子胡服,婦人裙襦,頭上作髻。其風俗政令與華夏略同。地多石磧,氣候溫暖,穀麥再熟,宜蠶,多五果。有草名為羊刺,其上生蜜,而味甚佳。出赤鹽如硃,白鹽如玉。



 多蒲陶酒。俗事天神,兼信佛法。國中羊馬牧於隱僻之處,以避外寇,非貴人不知其所。北有赤石山,山北七十里有貪汗山,夏有積雪。此山之北,鐵勒界也。從武威西北,有捷路,度沙磧千餘里,四面茫然,無有蹊徑。欲往者,尋有人畜骸骨而去。路中或聞歌哭之聲,行人尋之,多致亡失,蓋
 魑魅魍魎也。故商客往來,多取伊吾路。開皇十年,突厥破其四城,有二千人來歸中國。堅死,子伯雅立。其大母本突厥可汗女,其父死,突厥令依其俗,伯雅不從者久之。突厥逼之,不得已而從。



 煬帝嗣位,引致諸蕃。大業四年,遣使貢獻,帝待其使甚厚。明年,伯雅來朝。



 因從擊高麗,還尚宗女華容公主。八年冬歸蕃,下令國中曰:「夫經國字人,以保存為貴,寧邦緝政,以全濟為大。先者以國處邊荒,境連猛狄,同人無咎,被發左衽。今大隋統御,宇宙平一,普天率土,莫不齊向。孤既沐浴和風,庶均大化,其庶人以上皆宜解辮削衽。」帝聞而甚善之,下詔曰:「彰
 德嘉善,聖哲所隆,顯誠遂良,典謨貽則。光祿大夫、弁國公、高昌王伯雅識量經遠,器懷溫裕,丹款夙著,亮節遐宣。本自諸華,歷祚西壤,昔因多難,淪迫獯戎,數窮毀冕,翦為胡服。自我皇隋平一宇宙,化偃九圍,德加四表,伯雅逾沙忘阻,奉」來庭,觀禮容於舊章,慕威儀之盛典。於是襲纓解辮,削衽曳裾,變夷從夏,義光前載。可賜衣冠之具,仍班制造之式。並遣使人部領將送。被以採章,復見車服之美,棄彼氈毛,還為冠帶之國。」然伯雅先臣鐵勒,而鐵勒恆遣重臣在高昌國,有商胡往來者,則稅之送於鐵勒。雖有此令取悅中華,然竟畏鐵勒而不敢
 改也。自是歲令使人貢其方物。



 康國康國者,康居之後也。遷徙無常,不恆故地,然自漢以來相承不絕。其王本姓溫,月氏人也。舊居祁連山北昭武城,因被匈奴所破,西逾蔥嶺,遂有其國。支庶各分王,故康國左右諸國並以昭武為姓,示不忘本也。王字代失畢,為人寬厚,甚得眾心。其妻突厥達度可汗女也。都於薩寶水上阿祿迪城。城多眾居。大臣三人共掌國事。其王索發,冠七寶金花,衣綾羅錦繡白疊。其妻有髻,幪以皁巾。丈夫翦發錦袍。名為強國,而西域諸國多歸之。米
 國、史國、曹國、何國、安國、小安國、那色波國、烏那曷國、穆國皆歸附之。有胡律,置於祆祠,決罰則取而斷之。重罪者族,次重者死,賊盜截其足。人皆深目高鼻,多須髯。善於商賈,諸夷交易,多湊其國。有大小鼓、琵琶、五弦、箜篌、笛。婚姻喪制與突厥同。國立祖廟,以六月祭之,諸國皆來助祭。俗奉佛,為胡書。氣候溫,宜五穀,勤修園蔬,樹木滋茂。



 出馬、駝、騾、驢、封牛、黃金、鐃沙、香、阿薩那香、瑟瑟、麖皮、氍、錦疊。



 多蒲陶酒,富家或致千石,連年不敗。大業中,始遣使貢方物,後遂絕焉。



 安國安國,漢時安息國也。王姓昭武氏,與康國王同族,字設力登。妻,康國王女也。都在那密水南,城有五重,環以流水。宮殿皆為平頭。王坐金駝座,高七八尺。



 每聽政,與妻相對,大臣三人評理國事。風俗同於康國。唯妻其姊妹,及母子遞相禽獸,此為異也。煬帝即位之後,遣司隸從事杜行滿使於西域,至其國,得五色鹽而返。



 國之西百餘里有畢國,可千餘家。其國無君長,安國統之。大業五年,遣使貢獻,後遂絕焉。



 石國石國,居於藥殺水,都城方十餘里。其王姓石,名涅。國城
 之東南立屋,置座於中,正月六日、七月十五日以王父母燒餘之骨,金甕盛之,置於床上,巡繞而行,散以花香雜果,王率臣下設祭焉。禮終,王與夫人出就別帳,臣下以次列坐,享宴而罷。有粟麥,多良馬。其俗善戰,曾貳於突厥,射匱可汗興兵滅之,令特勤甸職攝其國事。南去鏺汗六百里,東南去瓜州六千里。甸職以大業五年遣使朝貢,其後不復至。



 女國女國,在蔥嶺之南,其國代以女為王。王姓蘇毗,字末羯,在位二十年。女王之夫,號曰金聚,不知政事。國內丈夫
 唯以征伐為務。山上為城,方五六里,人有萬家。王居九層之樓,侍女數百人,五日一聽朝。復有小女王,共知國政。其俗貴婦人,輕丈夫,而性不妒忌。男女皆以彩色塗面,一日之中,或數度變改之。人皆被發,以皮為鞋,課稅無常。氣候多寒,以射獵為業。出金俞石、硃砂、麝香、犛牛、駿馬、蜀馬。尤多鹽,恆將鹽向天竺興販,其利數倍。亦數與天竺及黨項戰爭。



 其女王死,國中則厚斂金錢,求死者族中之賢女二人,一為女王,次為小王。貴人死,剝取皮,以金屑和骨肉置於瓶內而埋之。經一年,又以其皮內於鐵器埋之。俗事阿修羅神。又有樹神,歲初以人祭,或用
 獼猴。祭畢,入山祝之,有一鳥如雌雉,來集掌上,破其腹而視之,有粟則年豐,沙石則有災,謂之鳥卜。開皇六年,遣使朝貢,其後遂絕。



 焉耆焉耆國,都白山之南七十里,漢時舊國也。其王姓龍,字突騎。都城方二里。



 國內有九城,勝兵千餘人。國無綱維。其俗奉佛書,類婆羅門。婚姻之禮有同華夏。



 死者焚之,持服七日。男子剪發。有魚鹽蒲葦之利。東去高昌九百里,西去龜茲九百里,皆沙磧。東南去瓜州二千二百里。大業中,遣使貢方物。



 龜茲龜茲國,都白山之南百七十里,漢時舊國也。其王姓白,字蘇尼咥。都城方六里。勝兵者數千。俗殺人者死,劫賊斷其一臂,並刖一足。俗與焉耆同。王頭系彩帶,垂之於後,坐金師子座,土多稻、粟、菽、麥,饒銅、鐵、鉛、麖皮、氍、鐃沙、鹽綠、雌黃、胡粉、安息香、良馬、封牛。東去焉耆九百里,南去于闐千四百里,西去疏勒千五百里,西北去突厥牙六百餘里,東南去瓜州三千一百里。大業中,遣使貢方物。



 疏勒
 疏勒國,都白山南百餘里,漢時舊國也。其王字阿彌厥。手足皆六指。產子非六指者,即不育。都城方五里。國內有大城十二,小城數十,勝兵者二千人。王戴金師子冠。土多稻、粟、麻、麥、銅、鐵、錦、雌黃,每歲常供送於突厥。南有黃河,西帶蔥嶺,東去龜茲千五百里,西去鏺汗國千里,南去硃俱波八九百里,東北去突厥牙千餘里,東南去瓜州四千六百里。大業中,遣使貢方物。



 于闐於闐國,都蔥嶺之北二百餘里。其王姓王,字卑示閉練。都城方八九里。國中大城有五,小城數十,勝兵者數千
 人。俗奉佛,尤多僧尼,王每持齋戒。城南五十里有贊摩寺者,云是羅漢比丘比盧旃所造,石上有闢支佛徒跣之跡。於闐西五百里有比摩寺,云是老子化胡成佛之所。俗無禮義,多賊盜淫縱。王錦帽,金鼠冠,妻戴金花。其王發不令人見。俗云,若見王發,年必儉。土多麻、麥、粟、稻、五果,多園林,山多美玉。東去鄯善千五百里,南去女國三千里,西去硃俱波千里,北去龜茲千四百里,東北去瓜州二千八百里。大業中,頻遣使朝貢。



 鏺汗鏺汗國,都蔥嶺之西五百餘里,古渠搜國也。王姓昭武,
 字阿利柒。都城方四里。勝兵數千人。王坐金羊床,妻戴金花。俗多硃砂、金、鐵。東去疏勒千里,西去蘇封沙那國五百里,西北去石國五百里,東北去突厥牙二千餘里,東去瓜州五千五百里。大業中,遣使貢方物。



 吐火羅吐火羅國,都蔥嶺西五百里,與挹怛雜居。都城方二里。勝兵者十萬人,皆習戰。其俗奉佛。兄弟同一妻,迭寢焉,每一人入房,戶外掛其衣以為志。生子屬其長兄。其山穴中有神馬,每歲牧牝馬於穴所,必產名駒。南去漕國千七百里,東去瓜州五千八百里。大業中,遣使朝貢。



 挹怛挹怛國,都烏滸水南二百餘里,大月氏之種類也。勝兵者五六千人。俗善戰。



 先時國亂,突厥遣通設字詰強領其國。都城方十餘里。多寺塔,皆飾以金。兄弟同妻。婦人有一夫者,冠一角帽,夫兄弟多者,依其數為角。南去曹國千五百里,東去瓜州六千五百里。大業中,遣使貢方物。



 米國米國,都那密水西,舊康居之地也。無王。其城主姓昭武,康國王之支庶,字閉拙。都城方二里。勝兵數百人。西北
 去康國百里,東去蘇對沙那國五百里,西南去史國二百里,東去瓜州六千四百里。大業中,頻貢方物。



 史國史國,都獨莫水南十里,舊康居之地也。其王姓昭武,字逖遮,亦康國王之支庶也。都城方二里。勝兵千餘人。俗同康國。北去康國二百四十里,南去吐火羅五百里,西去那色波國二百里,東北去米國二百里,東去瓜州六千五百里。大業中,遣使貢方物。



 曹國曹國,都那密水南數里,舊是康居之地也。國無主,康國
 王令子烏建領之。都城方三里。勝兵千餘人。國中有得悉神,自西海以東諸國並敬事之。其神有金人焉,金破羅闊丈有五尺,高下相稱。每日以駝五頭、馬十匹、羊一百口祭之,常有千人食之不盡。東南去康國百里,西去何國百五十里,東去瓜州六千六百里。大業中,遣使貢方物。



 何國何國,都那密水南數里,舊是康居之地也。其王姓昭武,亦康國王之族類,字敦。都城方二里。勝兵千人。其王坐金羊座。東去曹國百五十里,西去小安國三百里,東去
 瓜州六千七百五十里。大業中,遣使貢方物。



 烏那曷烏那曷國,都烏滸水西,舊安息之地也。王姓昭武,亦康國種類,字佛食。都城方二里。勝兵數百人。王坐金羊座。東北去安國四百里,西北去穆國二百餘里,東去瓜州七千五百里。大業中,遣使貢方物。



 穆國穆國,都烏滸河之西,亦安息之故地,與烏那曷為鄰。其王姓昭武,亦康國王之種類也,字阿濫密。都城方三里,勝兵二千人。東北去安國五百里,東去烏那曷二百餘
 里,西去波斯國四千餘里,東去瓜州七千七百里。大業中,遣使貢方物。



 波斯波斯國,都達曷水之西蘇藺城,即條支之故地也。其王字庫薩和。都城方十餘里。勝兵二萬餘人,乘象而戰。國無死刑,或斷手刖足,沒家財,或剃去其須,或系排於項,以為標異。人年三歲已上,出口錢四文。妻其姊妹。人死者,棄尸於山,持服一月。王著金花冠,坐金師子座,傅金屑於須上以為飾。衣錦袍,加瓔珞於其上。土多良馬,大驢,師子,白象,大鳥卵,真珠,頗黎,獸魄,珊瑚,琉璃,瑪瑙,水
 精,瑟瑟,呼洛羯,呂騰,火齊,金剛,金,銀,金俞石,銅,鑌鐵,錫,錦疊,細布,氍,毾,護那,越諾布,檀,金縷織成,赤麖皮,硃沙,水銀,薰陸、鬱金、蘇合、青木等諸香,胡椒,畢撥,石蜜,半密,千年棗,附子,訶黎勒,無食子,鹽綠,雌黃。突厥不能至其國,亦羈縻之。波斯每遣使貢獻。西去海數百里,東去穆國四千餘里,西北去拂菻四千五百里,東去瓜州萬一千七百里。煬帝遣雲騎尉李昱使通波斯,尋遣使隨昱貢方物。



 漕國漕國,在蔥嶺之北,漢時罽賓國也。其王姓昭武,字順達,
 康國王之宗族。都城方四里。勝兵者萬餘人。國法嚴整,殺人及賊盜皆死。其俗淫祠。蔥嶺山有順天神者,儀制極華,金銀鍱為屋,以銀為地,祠者日有千餘人。祠前有一魚脊骨,其孔中通,馬騎出入。國王戴金魚頭冠,坐金馬座。土多稻、粟、豆、麥;饒象,馬,封牛,金,銀,鑌鐵,氍,硃砂,青黛,安息、青木等香,石蜜,半密,黑鹽,阿魏,沒藥,白附子。北去帆延七百里,東去劫國六百里,東北去瓜州六千六百里。



 大業中,遣使貢方物。



 附國附國者,蜀郡西北二千餘里,即漢之西南夷也。有嘉良
 夷,即其東部,所居種姓自相率領,土俗與附國同,言語少殊,不相統一。其人並無姓氏。附國王字宜繒。



 其國南北八百里,東南千五百里,無城柵,近川谷,傍山險。俗好復仇,故壘石為巢而居,以避其患。其巢高至十餘丈,下至五六丈,每級丈餘,以木隔之。基方三四步,巢上方二三步,狀似浮圖。於下級開小門,從內上通,夜必關閉,以防賊盜。國有二萬餘家,號令自王出。嘉良夷政令系之酋帥,重罪者死,輕罪罰牛。



 人皆輕捷,便於擊劍。漆皮為牟甲,弓長六尺,以竹為弦。妻其群母及嫂,兒弟死,父兄亦納其妻。好歌舞,鼓簧,吹長笛。有死者,無服制,置尸高
 床之上,沐浴衣服,被以牟甲,覆以獸皮。子孫不哭,帶甲舞劍而呼云:「我父為鬼所取,我欲報冤殺鬼。」自餘親戚哭三聲而止。婦人哭,必以兩手掩面。死家殺牛,親屬以豬酒相遺,共飲啖而瘞之。死後十年而大葬,其葬必集親賓,殺馬動至數十匹。立其祖父神而事之。其俗以皮為帽,形圓如缽,或帶。衣多毛毼皮裘,全剝牛腳皮為靴。項系鐵鎖,手貫鐵釧。王與酋帥,金為首飾,胸前懸一金花,徑三寸。其土高,氣候涼,多風少雨。土宜小麥、青梁。山出金、銀,多白雉。水有嘉魚,長四尺而鱗細。



 大業四年,其王遣使素福等八人入朝。明年,又遣其弟子宜林
 率嘉良夷六十人朝貢。欲獻良馬,以路險不通,請開山道以修職貢。煬帝以勞人不許。



 嘉良有水,闊六七十丈,附國有水,闊百餘丈,並南流,用皮為舟而濟。



 附國南有薄緣夷,風俗亦同。西有女國。其東北連山,綿亙數千里,接於黨項。



 往往有羌:大、小左封,昔衛,葛延,白狗,向人,望族,林臺,春桑,利豆,迷桑,婢藥,大硤,白蘭,叱利摸徒,那鄂,當迷,渠步,桑悟,千碉,並在深山窮谷,無大君長。其風俗略同於黨項,或役屬吐谷渾,或附附國。大業中,來朝貢。



 緣西南邊置諸道總管,以遙管之。



 史臣曰:自古開遠夷,通絕域,必因宏放之主,皆起好事
 之臣。張騫鑿空於前,班超投筆於後,或結之以重寶,或懾之以利劍,投軀萬死之地,以要一旦之功,皆由主尚來遠之名,臣殉輕生之節。是知上之所好,下必有甚者也。煬帝規摹宏侈,掩吞秦、漢,裴矩方進《西域圖記》以蕩其心,故萬乘親出玉門關,置伊吾、且末,而關右暨於流沙,騷然無聊生矣。若使北狄無虞,東夷告捷,必將修輪臺之戍,築烏壘之城,求大秦之明珠,致條支之鳥卵,往來轉輸,將何以堪其敝哉!古者哲王之制,方五千里,務安諸夏,不事要荒。豈威不能加,德不能被?蓋不以四夷勞中國,不以無用害有用也。是以秦戍五嶺,漢事三邊,
 或道堇相望,或戶口減半。



 隋室恃其強盛,亦狼狽於青海。此皆一人失其道,故億兆罹其毒。若深思即敘之義,固辭都護之請,返其千里之馬,不求白狼之貢,則七戎九夷,候風重譯,雖無遼東之捷,豈及江都之禍乎!



\end{pinyinscope}