\article{卷三十六列傳第一 後妃}

\begin{pinyinscope}

 夫陰陽肇分,乾坤定位,君臣之道斯著,夫婦之義存焉。陰陽和則裁成萬物,家道正則化行天下,由近及遠,自家刑國,配天作合,不亦大乎!興亡是系,不亦重乎!是以先王慎之,正其本而嚴其防。後之繼體,靡克聿修,甘心柔曼之容,罔念幽閑之操。成敗攸屬,安危斯在。故皇、英降而虞道隆,任、姒歸而姬宗盛,妹、妲致夏、殷之釁,褒、趙
 結周、漢之禍。爰歷晉、宋,實繁有徒。皆位以寵升,榮非德進,恣行淫僻,莫顧禮儀,為梟為鴟,敗不旋踵。後之伉儷宸極,正位居中,罕蹈平易之途,多遵覆車之轍。雎鳩之德,千載寂寥;牝雞之晨,殊邦接響。窈窕淑女,靡有求於寤寐;鏗鏘環佩,鮮克嗣於徽音。永念前修,嘆深彤管。覽載籍於既往,考行事於當時,存亡得失之機,蓋亦多矣。故述《皇后列傳》,所以垂戒將來。



 然後妃之制,夏、殷以前略矣。周公定禮,內職始備列焉。秦、漢以下,代有沿革,品秩差次,前史載之詳矣。齊、梁以降,歷魏暨周,廢置益損,參差不一。



 周宣嗣位,不率典章,衣禕翟、稱中宮者,凡有
 五。夫人以下,略無定數。高祖思革前弊,大矯其違,唯皇后正位,傍無私寵,婦官稱號,未詳備焉。開皇二年,著內官之式,略依《周禮》,省滅其數。嬪三員,掌教四德,視正三品。世婦九員,掌賓客祭祀,視正五品。女御三十八員,掌女工絲枲,視正七品。又採漢、晉舊儀,置六尚、六司、六典,遞相統攝,以掌宮掖之政。一曰尚宮,掌導引皇后及閨閤廩賜。管司令三人,掌圖籍法式,糾察宣奏;典綜三人,掌綜璽器玩。二曰尚儀,掌禮儀教學。管司樂三人,掌音律之事;典贊三人,掌導引內外命婦朝見。三曰尚服,掌服章寶藏。管司飾三人,掌簪珥花嚴;典櫛三人,掌巾櫛
 膏沐。四曰尚食,掌進膳先嘗。管司醫三人,掌方藥卜筮;典器三人,掌樽彞器皿。五曰尚寢,掌幃帳床褥。管司筵三人,掌鋪設灑掃;典執三人,掌扇傘燈燭。六曰尚工,掌營造百役。



 管司制三人,掌衣服裁縫;典會三人,掌財帛出入。六尚各三員,視從九品,六司視勛品,六典視流外二品。初,文獻皇后功參歷試,外預朝政,內擅宮闈,懷嫉妒之心,虛嬪妾之位,不設三妃,防其上逼。自嬪以下,置六十員。加又抑損服章,降其品秩。至文獻崩後,始置貴人三員,增嬪至九員,世婦二十七員,御女八十一員。貴人等關掌宮闈之務,六尚已下,皆分隸焉。



 煬帝時,後妃
 嬪御,無厘婦職,唯端容麗飾,陪從宴游而已。帝又參詳典故,自制嘉名,著之於令。貴妃、淑妃、德妃,是為三夫人,品正第一。順儀、順容、順華、修儀、修容、修華、充儀、充容、充華,是為九嬪,品正第二。婕妤一十二員,品正第三,美人、才人一十五員,品正第四,是為世婦。寶林二十四員,品正第五;御女二十四員,品正第六;採女三十七員,品正第七,是為女御。總一百二十,以敘於宴寢。又有承衣刀人,皆趨侍左右,並無員數,視六品已下。



 時又增置女官,準尚書省,以六局管二十四司。一曰尚宮局,管司言,掌宣傳奏啟;司簿,掌名錄計度;司正,掌格式推罰,司闈,掌
 門閣管鑰。二曰尚儀局,管司籍,掌經史教學,紙筆幾案;司樂,掌音律;司賓,掌賓客;司贊,掌禮儀贊相導引。三曰尚服局,管司璽,掌琮璽符節;司衣,掌衣服;司飾,掌湯沐巾櫛玩弄;司仗,掌仗衛戎器。四曰尚食局,管司膳,掌膳羞;司釀,掌酒醴醯醢;司藥,掌醫巫藥劑;司饎,掌廩餼柴炭。五曰尚寢局,管司設,掌床席帷帳,鋪設灑掃;司輿,掌輿輦傘扇,執持羽儀;司苑,掌園絪種植,蔬菜瓜果;司燈,掌火燭。六曰尚工局,管司制,掌營造裁縫;司寶,掌金玉珠璣錢貨;司彩,掌繒帛;司織,掌織染。六尚二十二司,員各二人,唯司樂、司膳員各四人。每司又置典及掌,以貳
 其職。六尚十人,品從第五;司二十八人,品從第六;典二十八人,品從第七;掌二十八人,品從第九。女使流外,量局閑劇,多者十人已下,無定員數。聯事分職,各有司存焉。



 文獻獨狐皇后,河南洛陽人,周大司馬、河內公信之女也。信見高祖有奇表,故以後妻焉,時年十四。高祖與後相得,誓無異生之子。後初亦柔順恭孝,不失婦道。后姊為周明帝後,長女為周宣帝後,貴戚之盛,莫與為比,而後每謙卑自守,世以為賢。及周宣帝崩,高祖居禁中,總百揆,後使人謂高祖曰:「大事已然,騎獸之勢,必不得下,
 勉之!」高祖受禪,立為皇后。



 突厥嘗與中國交市,有明珠一篋,價值八百萬,幽州總管陰壽白後市之。後曰:「非我所須也。當今戎狄屢寇,將士罷勞,未若以八百萬分賞有功者。」百僚聞而畢賀。高祖甚寵憚之。上每臨朝,後輒與上方輦而進,至閣乃止。使宦官伺上,政有所失,隨則匡諫,多所弘益。候上退朝而同反燕寢,相顧欣然。後早失二親,常懷感慕,見公卿有父母者,每為致禮焉。有司奏以《周禮》百官之妻,命於王後,憲章在昔,請依古制。後曰:「以婦人與政,或從此漸,不可開其源也。」不許。



 後每謂諸公主曰:「周家公主,類無婦德,失禮於舅姑,離薄人骨
 肉,此不順事,爾等當誡之。」大都督崔長仁,後之中外兄弟也,犯法當斬。高祖以後之故,欲免其罪。後曰:「國家之事,焉可顧私!」長仁竟坐死。後異母弟陀,以貓鬼巫蠱咒詛於後,坐當死。後三日不食,為之請命曰:「陀若蠢政害民者,妾不敢言。今坐為妾身,敢請其命。」陀於是減死一等。後每與上言及政事,往往意合,宮中稱為二聖。



 後頗仁愛,每聞大理決囚,未嘗不流涕。然性尤妒忌,後宮莫敢進御。尉遲迥女孫有美色,先在宮中。上於仁壽宮見而悅之,因此得幸。後伺上聽朝,陰殺之。



 上由是大怒,單騎從苑中而出,不由徑路,入山谷間二十餘里。高熲、楊
 素等追及上,扣馬苦諫。上太息曰:「吾貴為天子,而不得自由!」高熲曰:「陛下豈以一婦人而輕天下!」上意少解,駐馬良久,中夜方始還宮。後俟上於閣內,及上至,後流涕拜謝,熲、素等和解之。上置酒極歡,後自此意頗衰折。初,後以高熲是父之家客,甚見親禮。至是,聞熲謂己為一婦人,因此銜恨。又以熲夫人死,其妾生男,益不善之,漸加譖毀,上亦每事唯後言是用。後見諸王及朝士有妾孕者,必勸上斥之。時皇太子多內寵,妃元氏暴薨,後意太子愛妾雲氏害之。由是諷上黜高熲,竟廢太子,立晉王廣,皆後之謀也。



 仁壽二年八月甲子,月暈四重,己已,
 太白犯軒轅。其夜,後崩於永安宮,時年五十。葬於太陵。其後,宣華夫人陳氏、容華夫人蔡氏俱有寵,上頗惑之,由是發疾。及危篤,謂侍者曰:「使皇后在,吾不及此」云。



 宣華夫人陳氏,陳宣帝之女也。性聰慧,姿貌無雙。及陳滅,配掖庭,後選入宮為嬪。時獨孤皇后性妒,後宮罕得進御,唯陳氏有寵。晉王廣之在籓也,陰有奪宗之計,規為內助,每致禮焉。進金蛇、金駝等物,以取媚於陳氏。皇太子廢立之際,頗有力焉。及文獻皇后崩,進位為貴人,專房擅寵,主斷內事,六宮莫與為比。



 及上大漸,遺詔拜為宣華夫人。



 初,上寢疾於仁壽宮也,夫人與皇太子同
 侍疾。平旦出更衣,為太子所逼,夫人拒之得免,歸於上所。上怪其神色有異,問其故。夫人泫然曰:「太子無禮。」



 上恚曰:「畜生何足付大事,獨狐誠誤我!」意謂獻皇后也。因呼兵部尚書柳述、黃門侍郎元巖曰:「召我兒!」述等將呼太子,上曰:「勇也。」述、巖出閣為敕書訖,示左僕射楊素。素以其事白太子,太子遣張衡入寢殿,遂令夫人及後宮同侍疾者,並出就別室。俄聞上崩,而未發喪也。夫人與諸後宮相顧曰:「事變矣!」



 皆色動股慄。晡後,太子遣使者齎金合子,帖紙於際,親署封字,以賜夫人。夫人見之惶懼,以為鴆毒,不敢發。使者促之,於是乃發,見合中有同
 心結數枚。諸宮人咸悅,相謂曰:「得免死矣!」陳氏恚而卻坐,不肯致謝。諸宮人共逼之,乃拜使者。其夜,太子烝焉。及煬帝嗣位之後,出居仙都宮。尋召入,歲餘而終,時年二十九。帝深悼之,為制《神傷賦》。



 容華夫人蔡氏,丹陽人也。陳滅之後,以選入宮,為世婦。容儀婉,上甚悅之。



 以文獻皇后故,希得進幸。及后崩,漸見寵遇,拜為貴人,參斷宮掖之務,與陳氏相亞。上寢疾,加號容華夫人。上崩後,自請言事,亦為煬帝所烝。



 煬帝蕭皇后,梁明帝巋之女也。江南風俗,二月生子者不舉。後以二月生,由是季父岌收而養之。未幾,岌夫妻
 俱死,轉養舅氏張軻家。然軻甚貧窶,後躬親勞苦。煬帝為晉王時,高祖將為王選妃於梁,遍占諸女,諸女皆不吉。巋迎後於舅氏,令使者占之,曰:「吉。」於是遂策為王妃。



 後性婉順,有智識,好學解屬文,頗知占候。高祖大善之,帝甚寵敬焉。及帝嗣位,詔曰:「朕祗承丕緒,憲章在昔,爰建長秋,用承饗薦。妃蕭氏,夙稟成訓,婦道克修,宜正位軒闈,式弘柔教,可立為皇后。」帝每游幸,後未嘗不隨從。時後見帝失德,心知不可,不敢厝言,因為《述志賦》以自寄。其詞曰:承積善之餘慶,備箕帚於皇庭。恐修名之不立,將負累於先靈。乃夙夜而匪懈,實寅懼於玄冥。雖自
 強而不息,亮愚朦之所滯。思竭節於天衢,才追心而弗逮。實庸薄之多幸,荷隆寵之嘉惠。賴天高而地厚,屬王道之升平。均二儀之覆載,與日月而齊明。乃春生而夏長,等品物而同榮。願立志於恭儉,私自競於誡盈。孰有念於知足,茍無希於濫名。惟至德之弘深,情不邇於聲色。感懷舊之餘恩,求故劍於宸極。叨不世之殊盼,謬非才而奉職。何寵祿之逾分,撫胸襟而未識。雖沐浴於恩光,內慚惶而累息。顧微躬之寡昧,思令淑之良難。實不遑於啟處,將何情而自安!



 若臨深而履薄,心戰慄其如寒。夫居高而必危,慮處滿而防溢。知恣誇之非道,乃攝
 生於沖謐。嗟寵辱之易驚,尚無為而抱一。履謙光而守志,且願安乎容膝。珠簾玉箔之奇,金屋瑤臺之美,雖時俗之崇麗,蓋吾人之所鄙。愧絺綌之不工,豈絲竹之喧耳。知道德之可尊,明善惡之由己。蕩囂煩之俗慮,乃伏膺於經史。綜箴誡以訓心,觀女圖而作軌。遵古賢之令範,冀福祿之能綏。時循躬而三省,覺今是而昨非。嗤黃老之損思,信為善之可歸。慕周姒之遺風,美虞妃之聖則。仰先哲之高才,貴至人之休德。質菲薄而難蹤,心恬愉而去惑。乃平生之耿介,實禮義之所遵。雖生知之不敏,庶積行以成仁。懼達人之蓋寡,謂何求而自陳。誠素
 志之難寫,同絕筆於獲麟。



 及帝幸江都,臣下離貳,有宮人白後曰:「外聞人人欲反。」後曰:「任汝奏之。」宮人言於帝,帝大怒曰:「非所宜言!」遂斬之。後人復白後曰:「宿衛者往往偶語謀反。」後曰:「天下事一朝至此,勢已然,無可救也。何用言之,徒令帝憂煩耳。」自是無復言者。及宇文氏之亂,隨軍至聊城。化及敗,沒於竇建德。



 突厥處羅可汗遣使迎後於洺州,建德不敢留,遂入於虜庭。大唐貞觀四年,破滅突厥,乃以禮致之,歸於京師。



 史臣曰:二後,帝未登庸,早儷宸極,恩隆好合,始終不渝。文獻德異鳲鳩,心非均一,擅寵移嫡,傾覆宗社,惜哉!《書》
 曰:「牝雞之晨,惟家之索。」高祖之不能敦睦九族,抑有由矣。蕭后初歸籓邸,有輔佐君子之心。煬帝得不以道,便謂人無忠信。父子之間,尚懷猜阻,夫婦之際,其何有焉!暨乎國破家亡,竄身無地,飄流異域,良足悲矣!



\end{pinyinscope}