\article{卷四十四列傳第九 滕穆王瓚嗣王綸}

\begin{pinyinscope}

 膝穆王瓚,字恆生,一名慧,高祖母弟也。周世以太祖軍功封竟陵郡公,尚武帝妹順陽公主,自右中侍上士遷御伯中大夫。保定四年,改為納言,授儀同。瓚貴公子,又尚公主,美姿儀,好書愛士,甚有令名於當世,時人號曰楊三郎。武帝甚親愛之。平齊之役,諸王咸從,留瓚居守,帝謂之曰:「六府事殷,一以相付。朕將遂事東方,無西顧
 之憂矣。」其見親信如此。宣帝即位,遷吏部中大夫,加上儀同。未幾,帝崩,高祖入禁中,將總朝政,令廢太子勇召之,欲有計議。瓚素與高祖不協,聞召不從,曰:「作隋國公恐不能保,何乃更為族滅事邪?」高祖作相,遷大將軍。尋拜大宗伯,典修禮律。進位上柱國、邵國公。瓚見高祖執政,群情未一,恐為家禍,陰有圖高祖之計,高祖每優容之。及受禪,立為滕王。後拜雍州牧。



 上數與同坐,呼為阿三。後坐事去牧,以王就第。



 瓚妃宇文氏,先時與獨孤皇后不平,及此鬱鬱不得志,陰有咒詛。上命瓚出之,瓚不忍離絕,固請。上不得已,從之,宇文氏竟除屬籍。瓚由是
 忤旨,恩禮更薄。



 開皇十一年,從幸慄園,暴薨,時年四十二。人皆言其遇鴆以斃。子綸嗣。



 綸字斌籀,性弘厚,美姿容,頗解鐘律。高祖受禪,封邵國公,邑八千戶。明年,拜邵州刺史。晉王廣納妃於梁,詔綸致禮焉,甚為梁人所敬。綸以穆王之故,當高祖之世,每不自安。煬帝即位,尤被猜忌。綸憂懼不知所為,呼術者王琛而問之。琛答曰:「王相祿不凡。」乃因曰:「滕即騰也,此字足為善應。」有沙門惠恩、崛多等,頗解占候,綸每與交通,常令此三人為度星法。有人告綸怨望咒詛,帝命黃門侍郎王弘窮治之。弘見帝方怒,遂希旨奏綸厭蠱惡
 逆,坐當死。帝令公卿議其事,司徒楊素等曰:「綸希冀國災,以為身幸。原其懷惡之由,積自家世。惟皇運之始,四海同心,在於孔懷,彌須協力。其先乃離阻大謀,棄同即異,父悖於前,子逆於後,非直覬覦朝廷,便是圖危社稷。為惡有狀,其罪莫大,刑茲無赦,抑有舊章,請依前律。」帝以公族不忍,除名為民,徙始安。諸弟散徙邊郡。大業七年,親征遼東,綸欲上表,請從軍自效,為郡司所遏。未幾,復徙硃崖。及天下大亂,為賊林仕弘所逼,攜妻子竄於儋耳。後歸大唐,為懷化縣公。



 綸弟坦,字文籀,初封竟陵郡公,坐綸徙長沙。坦弟猛,字武籀,徙衡山。猛弟溫,字明
 籀,初徙零陵。溫好學,解屬文,既而作《零陵賦》以自寄,其辭哀思。



 帝見而怒之,轉徙南海。溫弟詵,字弘籀,前亦徙零陵。帝以其修謹,襲封滕王,以奉穆王嗣。大業末,薨於江都。



 道悼王靜道悼王靜,字賢籀,滕穆王瓚之子也。出繼叔父嵩。嵩在周代,以太祖軍功,賜爵興城公,早卒。高祖踐位,追封道王,謚曰宣。以靜襲焉。卒,無子,國除。



 衛昭王爽嗣王集衛昭王爽,字師仁,小字明達,高祖異母弟也。周世,在襁
 褓中,以太祖軍功,封同安郡公。六歲而太祖崩,為獻皇后之所鞠養,由是高祖於諸弟中特寵愛之。十七為內史上士。高祖執政,拜大將軍、秦州總管。未之官,轉授蒲州刺史,進位柱國。及受禪,立為衛王。尋遷雍州牧,領左右將軍。俄遷右領軍大將軍,權領並州總管。歲餘,進位上柱國,轉涼州總管。爽美風儀,有器局,治甚有聲。



 其年,以爽為行軍元帥,步騎七萬以備胡。出平涼,無虜而還。明年,大舉北伐,又為元帥。河間王弘、豆盧勣、竇榮定、高熲、虞慶則等分道而進,俱受爽節度。爽親率李充節等四將出朔州,遇沙缽略可汗於白道,接戰,大破之,虜獲
 千餘人,驅馬牛羊巨萬。沙缽略可汗中重創而遁。高祖大悅,賜爽真食梁安縣千戶。六年,復為元帥,步騎十五萬,出合川。突厥遁逃而返。明年,徵為納言。高祖甚重之。



 未幾,爽寢疾,上使巫者薛榮宗視之,雲眾鬼為厲。爽令左右驅逐之。居數日,有鬼物來擊榮宗,榮宗走下階而斃。其夜爽薨,時年二十五。贈太尉、冀州刺史。



 子集嗣。



 集字文會,初封遂安王,尋襲封衛王。煬帝時,諸侯王恩禮漸薄,猜防日甚。



 集憂懼不知所為,乃呼術者俞普明章醮以祈福助。有人告集咒詛,憲司希旨,鍛成其獄,奏集惡逆,坐當死。天子下公卿議其事,楊素等曰:「集密懷
 左道,厭蠱君親,公然咒詛,無慚幽顯。情滅人理,事悖先朝,是君父之罪人,非臣子之所赦,請論如律。」時滕王綸坐與相連,帝不忍加誅,乃下詔曰:「綸、集以附萼之華,猶子之重,縻之好爵,匪由德進。正應與國升降,休戚是同,乃包藏妖禍,誕縱邪僻。在三之義,愛敬俱淪;急難之情,孔懷頓滅。公卿議既如此,覽以潸然。雖復王法無私,恩從義斷,但法隱公族,禮有親親。致之極闢,情所未忍。」於是除名為民,遠徙邊郡。遇天下大亂,不知所終。



 蔡王智積蔡王智積,高祖弟整之子也。整周明帝時,以太祖軍功,
 賜爵陳留郡公。尋授開府、車騎大將軍。從武帝平齊,至並州,力戰而死。及高祖作相,贈柱國、大司徒、冀定瀛相懷衛趙貝八州刺史。高祖受禪,追封蔡王,謚曰景。以智積襲焉。又封其弟智明為高陽郡公,智才為開封縣公。尋拜智積為開府儀同三司,授同州刺史,儀衛資送甚盛。頃之,以修謹聞,高祖善之。在州未嘗嬉戲游獵,聽政之暇,端坐讀書,門無私謁。有侍讀公孫尚儀,山東儒士,府佐楊君英、蕭德言,並有文學,時延於座,所設唯餅果,酒才三酌。家有女妓,唯年節嘉慶,奏於太妃之前,其簡靜如此。昔高祖龍潛時,景王與高祖不睦,其太妃尉氏
 又與獨孤皇后不相諧,以是智積常懷危懼,每自貶損。高祖知其若是,亦哀憐之。人或勸智積治產業者,智積曰:「昔平原露朽財帛,苦其多也。吾幸無可露,何更營乎?」有五男,止教讀《論語》、《孝經》而已,亦不令交通賓客。或問其故,智積答曰:「卿非知我者。」



 其意恐兒子有才能,以致禍也。開皇二十年,徵還京第,無他職任,闔門自守,非朝覲不出。



 煬帝即位,滕王綸、衛王集並以讒構得罪,高陽公智明亦以交游奪爵,智積逾懼。大業七年,授弘農太守,委政僚佐,清靜自居。及楊玄感作亂,自東都引軍而西,智積謂官屬曰:「玄感聞大軍將至,欲西圖關中。若成
 其計,則根本固矣。當以計縻之,使不得進。不出一旬,自可擒耳。」及玄感軍至城下,智積登陴詈辱之,玄感怒甚,留攻之。城門為賊所燒,智積乃更益火,賊不得入。數日,宇文述等援軍至,合擊破之。



 十二年,從駕江都,寢疾。帝時疏薄骨肉,智積每不自安,及遇患,不呼醫。



 臨終,謂所親曰:「吾今日始知得保首領沒於地矣。」時人哀之。有子道玄。



 史臣曰:周建懿親,漢開盤石,內以敦睦九族,外以輯寧億兆,深根固本,崇獎王室。安則有以同其樂,衰則有以恤其危,所由來久矣。魏、晉以下,多失厥中,不遵王度,各
 徇所私。抑之則勢齊於匹夫,抗之則權侔於萬乘,矯枉過正,非一時也。得失詳乎前史,不復究而論焉。高祖昆弟之恩,素非篤睦,閨房之隙,又不相容。至於二世承基,其弊愈甚。是以滕穆暴薨,人皆竊議;蔡王將沒,自以為幸。



 唯衛王養於獻後,故任遇特隆,而諸子遷流,莫知死所,悲夫!其錫以茅土,稱為盤石,行無甲兵之衛,居與氓隸為伍。外內無虞,顛危不暇,時逢多難,將何望焉!



\end{pinyinscope}