\article{卷五十二列傳第十七}

\begin{pinyinscope}

 韓擒虎弟僧壽洪韓擒,字子通,河南東垣人也,後家新安。父雄,以武烈知名,仕周,官至大將軍、洛虞等八州刺史。擒少慷慨,以膽略見稱,容貌魁岸,有雄傑之表。性又好書,經史百家皆略知大旨。周太祖見而異之,令與諸子游集。後以軍功,拜都督、新安太守,稍遷儀同三司,襲爵新義郡公。武帝伐齊,齊將獨孤永業守金墉城,擒說下之。進平範陽,加
 上儀同,拜永州刺史。陳人逼光州,擒以行軍總管擊破之。



 又從宇文忻平合州。高祖作相,遷和州刺史。陳將甄慶、任蠻奴、蕭摩訶等共為聲援,頻寇江北,前後入界。擒屢挫其鋒,陳人奪氣。



 開皇初,高祖潛有吞並江南之志,以擒有文武才用,夙著聲名,於是拜為廬州總管,委以平陳之任,甚為敵人所憚。及大舉伐陳,以擒為先鋒。擒率五百人宵濟,襲採石,守者皆醉,擒遂取之。進攻姑熟,半日而拔,次於新林。江南父老素聞其威信,來謁軍門,晝夜不絕。陳人大駭,其將樊巡、魯世真、田瑞等相繼降之。晉王廣上狀,高祖聞而大悅,宴賜群臣。晉王遣行軍
 總管杜彥與擒合軍,步騎二萬,陳叔寶遣領軍蔡徵守硃雀航,聞擒將至,眾懼而潰。任蠻奴為賀若弼所敗,棄軍降於擒。擒以精騎五百,直入硃雀門。陳人欲戰,蠻奴捴之曰:「老夫尚降,諸君何事!」眾皆散走。遂平金陵,執陳主叔寶。時賀若弼亦有功。乃下詔於晉王曰:「此二公者,深謀大略,東南逋寇,朕本委之,靜地恤民,悉如朕意。九州不一,已數百年,以名臣之功,成太平之業,天下盛事,何用過此!聞以欣然,實深慶快。



 平定江表,二人之力也。」賜物萬段。又下優詔於擒、弼曰:「申國威於萬里,宣朝化於一隅,使東南之民俱出湯火,數百年寇旬日廓清,專
 是公之功也。高名塞於宇宙,盛業光於天壤,逖聽前古,罕聞其匹。班師凱入,誠知非遠,相思之甚,寸陰若歲。」及至京,弼與擒爭功於上前,弼曰:「臣在蔣山死戰,破其銳卒,擒其驍將,震揚威武,遂平陳國。韓擒略不交陣,豈臣之比!」擒曰:「本奉明旨,令臣與弼同時合勢,以取偽都。弼乃敢先期,逢賊遂戰,致令將士傷死甚多。臣以輕騎五百,兵不血刃,直取金陵,降任蠻奴,執陳叔寶,據其府庫,傾其巢穴。弼至夕,方扣北掖門,臣啟關而納之。斯乃救罪不暇,安得與臣相比!」上曰:「二將俱合上勛。」於是進位上柱國,賜物八千段。有司劾擒放縱士卒,淫污陳宮,坐
 此不加爵邑。



 先是,江東有謠歌曰:「黃斑青驄馬,發自壽陽涘。來時冬氣末,去日春風始。」



 皆不知所謂。擒本名豹,平陳之際,又乘青驄馬,往反時節與歌相應,至是方悟。



 其後突厥來朝,上謂之曰:「汝聞江南有陳國天子乎?」對曰:「聞之。」上命左右引突厥詣擒前,曰:「此是執得陳國天子者。」擒厲然顧之,突厥惶恐,不敢仰視,其有威容如此。別封壽光縣公,食邑千戶。以行軍總管屯金城,御備胡寇,即拜涼州總管。俄征還京,上宴之內殿,恩禮殊厚。無何,其鄰母見擒門下儀衛甚盛,有同王者,母異而問之。其中人曰:「我來迎王。」忽然不見。又有人疾篤,忽驚走至
 擒家曰:「我欲謁王。」左右問曰:「何王也?」答曰:「閻羅王。」擒子弟欲撻之,擒止之曰:「生為上柱國,死作閻羅王,斯亦足矣。」因寢疾,數日竟卒,時年五十五。子世諤嗣。



 世諤倜儻驍捷,有父風。楊玄感之作亂也,引世諤為將,每戰先登。及玄感敗,為吏所拘。時帝在高陽,送詣行所。世諤曰令守者市酒殽以酣暢,揚言曰:「吾死在朝夕,不醉何為!」漸以酒進守者,守者狎之,遂飲令致醉。世諤因得逃奔山賊,不知所終。



 僧壽字玄慶,擒母弟也,亦以勇烈知名。周武帝時,為侍伯中旅下大夫。高祖得政,從韋孝寬平尉迥,每戰有功,
 授大將軍,封昌樂公,邑千戶。開皇初,拜安州刺史。時擒為廬州總管,朝廷不欲同在淮南,轉為熊州刺史。後轉蔚州刺史,進爵廣陵郡公。尋以行軍總管擊突厥於雞頭山,破之。後坐事免。數歲,復拜蔚州刺史。突厥甚憚之。十七年,屯蘭州以備胡。明年,遼東之役,領行軍總管,還,檢校靈州總管事。從楊素擊突厥,破之,進位上柱國,改封江都郡公。煬帝即位,又改封新蔡郡公。自是之後,不復任用。大業五年,從幸太原。有京兆人達奚通妾王氏,能清歌,朝臣多相會觀之,僧壽亦豫焉,坐是除名。尋令復位。八年,卒於京師,時年六十五。有子孝基。



 洪字叔明,擒季弟也。少驍勇,善射,膂力過人。仕周侍伯上士,後以軍功拜大都督。高祖為丞相,從韋孝寬破尉迥於相州,加上開府,甘棠縣侯,邑八百戶。



 高祖受禪,進爵為公。尋授驃騎將軍。開皇九年,平陳之役,授行軍總管。及陳平,晉王廣大獵於蔣山,有猛獸在圍中,眾皆懼。洪馳馬射之,應弦而倒。陳氏諸將,列觀於側,莫不嘆伏焉。王大喜,賜縑百匹。尋以功加柱國,拜蔣州刺史。數歲,轉廉州刺史。時突厥屢為邊患,朝廷以洪驍勇,檢校朔州總管事。尋拜代州總管。



 仁壽元年,突厥達頭可汗犯塞,洪率蔚州刺史劉隆、大將軍李藥王拒之。遇虜於恆
 安,眾寡不敵,洪四面搏戰,身被重創,將士沮氣。虜悉眾圍之,矢下如雨。洪偽與虜和,圍少解。洪率所領潰圍而出,死者大半,殺虜亦倍。洪及藥王除名為民,隆竟坐死。煬帝北巡,至恆安,見白骨被野,以問侍臣。侍臣曰:「往者韓洪與虜戰處也。」帝憫然傷之,收葬骸骨,命五郡沙門為設佛供,拜洪隴西太守。未幾,硃崖民王萬昌作亂,詔洪擊平之。以功加位金紫光祿大夫,領郡如故。俄而萬昌弟仲通復叛,又詔洪討平之。師未旋,遇疾而卒,時年六十三。



 賀若弼
 賀若弼,字輔伯,河南洛陽人也。父敦,以武烈知名,仕周為金州總管,宇文護忌而害之。臨刑,呼弼謂之曰:「吾必欲平江南,然此心不果,汝當成吾志。且吾以舌死,汝不可不思。」因引錐刺弼舌出血,誡以慎口。弼少慷慨有大志,驍勇便弓馬,解屬文,博涉書記,有重名於當世。周齊王憲聞而敬之,引為記室。未幾,封當亭縣公,遷小內史。周武帝時,上柱國烏丸軌言於帝曰:「太子非帝王器,臣亦嘗與賀若弼論之。」帝呼弼問之,弼知太子不可動搖,恐禍及己,詭對曰:「皇太子德業日新,未睹其闕。」帝默然。弼既退,軌讓其背己,弼曰:「君不密則失臣,臣不密則失
 身,所以不敢輕議也。」及宣帝嗣位,軌竟見誅,弼乃獲免。尋與韋孝寬伐陳,攻拔數十城,弼計居多。拜壽州刺史,改封襄邑縣公。高祖為丞相,尉迥作亂鄴城,恐弼為變,遣長孫平馳驛代之。



 高祖受禪,陰有並江南之志,訪可任者。高熲曰:「朝臣之內,文武才幹,無若賀若弼者。」高祖曰:「公得之矣。」於是拜弼為吳州總管,委以平陳之事,弼忻然以為己任。與壽州總管源雄並為重鎮。弼遺雄詩曰:「交河驃騎幕,合浦伏波營,勿使騏驎上,無我二人名。」



 獻取陳十策,上稱善,賜以寶刀。開皇九年,大舉伐陳,以弼為行軍總管。將渡江,酹酒而咒曰:「弼親承廟略,遠振
 國威,伐罪吊民,除兇翦暴,上天長江,鑒其若此。如便福善禍淫,大軍利涉;如事有乖違,得葬江魚腹中,死且不恨。」



 先是,弼請緣江防人每交代之際,必集歷陽。於是大列旗幟,營幕被野。陳人以為大兵至,悉發國中士馬。既知防人交代,其眾復散。後以為常,不復設備。及此,弼以大軍濟江,陳人弗之覺也。襲陳南徐州,拔之,執其刺史黃恪。軍令嚴肅,秋毫不犯。有軍士於民間沽酒者,弼立斬之。進屯蔣山之白土岡,陳將魯達、周智安、任蠻奴、田瑞、樊毅、孔範、蕭摩訶等以勁兵拒戰。田瑞先犯弼軍,弼擊走之。魯達等相繼遞進,弼軍屢卻。弼揣知其驕,士卒
 且惰,於是督厲將士,殊死戰,遂大破之。麾下開府員明擒摩訶至,弼命左右牽斬之。摩訶顏色自若,弼釋而禮之。從北掖門而入。時韓擒已執陳叔寶,弼至,呼叔寶視之。叔寶惶懼流汗,股心慄再拜。



 弼謂之曰:「小國之君,當大國卿,拜,禮也。入朝不失作歸命侯,無勞恐懼。」



 既而弼恚恨不獲叔寶,功在韓擒之後,於是與擒相詢,挺刃而出。上聞弼有功,大悅,下詔褒揚,語在《韓擒傳》。晉王以弼先期決戰,違軍命,於是以弼屬吏。上驛召之,及見,迎勞曰:」克定三吳,公之功也。「命登御坐,賜物八千段,加位上柱國,進爵宋國公,真食襄邑三千戶,加以寶劍、寶帶、金甕、
 金盤各一,並雉尾扇、曲蓋,雜彩二千段,女樂二部,又賜陳叔寶妹為妾。拜右領軍大將軍,尋轉右武候大將軍。



 弼時貴盛,位望隆重,其兄隆為武都郡公,弟東為萬榮郡公,並刺史、列將。



 弼家珍玩不可勝計,婢妾曳綺羅者數百,時人榮之。弼自謂功名出朝臣之右,每以宰相自許。既而楊素為右僕射,弼仍為將軍,甚不平,形於言色,由是免官,弼怨望愈甚。後數年,下弼獄,上謂之曰:「我以高熲、楊素為宰相,汝每倡言,云此二人惟堪啖飯耳,是何意也?」弼曰:「熲,臣之故人,素,臣之舅子,臣並知其為人,誠有此語。」公卿奏弼怨望,罪當死。上惜其功,於是除名為
 民。歲餘,復其爵位。上亦忌之,不復任使,然每宴賜,遇之甚厚。開皇十九年,上幸仁壽宮,宴王公,詔弼為五言詩,詞意憤怨,帝覽而容之。嘗遇突厥入朝,上賜之射,突厥一發中的。上曰:「非賀若弼無能當此。」於是命弼。弼再拜祝曰:「臣若赤誠奉國者,當一發破的。如其不然,發不中也。」既射,一發而中。上大悅,顧謂突厥曰:「此人天賜我也!」



 煬帝之在東宮,嘗謂弼曰:「楊素、韓擒、史萬歲三人,俱稱良將,優劣如何?」



 弼曰:「楊素是猛將,非謀將;韓擒是鬥將,非領將;史萬歲是騎將,非大將。」



 太子曰:「然則大將誰也?」弼拜曰:「唯殿下所擇。」弼意自許為大將。及煬帝嗣位,尤
 被疏忌。大業三年,從駕北巡,至榆林。帝時為大帳,其下可坐數千人,召突厥啟民可汗饗之。弼以為大侈,與高熲、宇文弼等私議得失,為人所奏,竟坐誅,時年六十四。妻子為官奴婢,群從徙邊。



 子懷亮,慷慨有父風,以柱國世子拜儀同三司。坐弼為奴,俄亦誅死。



 史臣曰:夫天地未泰,聖哲啟其機;疆埸尚梗,爪牙宣其力。周之方、邵,漢室韓、彭,代有其人,非一時也。自晉衰微,中原幅裂,區宇分隔,將三百年。陳氏憑長江之地險,恃金陵之餘氣,以為天限南北,人莫能窺。高祖爰應千齡,將一函夏。賀若弼慷慨,申必取之長策,韓擒奮發,賈餘
 勇以爭先,勢甚疾雷,鋒逾駭電。隋氏自此一戎,威加四海。稽諸天道,或時有廢興,考之人謀,實二臣之力。



 其俶儻英略,賀若居多,武毅威雄,韓擒稱重。方於晉之王、杜,勛庸綽有餘地。



 然賀若功成名立,矜伐不已,竟顛殞於非命,亦不密以失身。若念父臨終之言,必不及於斯禍矣。韓擒累世將家,威聲動俗,敵國既破,名遂身全,幸也。廣陵、甘棠,咸有武藝,驍雄膽略,並為當時所推,赳赳干城,難兄難弟矣。



\end{pinyinscope}