\article{卷三帝紀第三 煬帝上}

\begin{pinyinscope}

 煬
 皇帝,諱廣,一名英,小字阿摐,高祖第二子也。母曰文獻獨孤皇后。上美姿儀,少敏慧,高祖及後於諸子中特所鐘愛。在周,以高祖勛,封雁門郡公。開皇元年,立為晉王,拜柱國、並州總管,時年十三。尋授武衛大將軍,進位上柱國、河北道行臺尚書令,大將軍如故。高祖令項城公韶、安道公李徹輔導之。上好學,善屬文,沉深嚴重,
 朝野屬望。高祖密令善相者來和遍視諸子,和曰:「晉王眉上雙骨隆起,貴不可言。」既而高祖幸上所居第,見樂器弦多斷絕,又有塵埃,若不用者,以為不好聲妓,善之。上尤自矯飾,當時稱為仁孝。嘗觀獵遇雨,左右進油衣,上曰:「士卒皆沾濕,我獨衣此乎!」乃令持去。六年,轉淮南道行臺尚書令。



 其年,徵拜雍州牧、內史令。八年冬,大舉伐陳,以上為行軍元帥。及陳平,執陳湘州刺史施文慶、散騎常侍沈客卿、市令陽慧朗、刑法監徐析、尚書都令史暨慧,以其邪佞,有害於民,斬之右闕下,以謝三吳。於是封府庫,資財無所取,天下稱賢。進位太尉,賜輅車、乘
 馬,袞冕之服,玄珪、白璧各一。復拜並州總管。俄而江南高智慧等相聚作亂,徙上為揚州總管,鎮江都,每歲一朝。高祖之祠太山也,領武候大將軍。明年歸籓。後數載,突厥寇邊,復為行軍元帥,出靈武,無虜而還。



 及太子勇廢,立上為皇太子。是月,當受冊。高祖曰:「吾以大興公成帝業。」令上出舍大興縣。其夜,烈風大雪,地震山崩,民舍多壞,壓死者百餘口。仁壽初,奉詔巡撫東南。是後高祖每避暑仁壽宮,恆令上監國。



 四年七月,高祖崩,上即皇帝位於仁壽宮。八月,奉梓宮還京師。並州總管漢王諒舉兵反,詔尚書左僕射楊素討平之。九月乙巳,以備身
 將軍崔彭為左領軍大將軍。十一月乙未,幸洛陽。丙申,發丁男數十萬掘塹,自龍門東接長平、汲郡,抵臨清關,度河,至浚儀、襄城,達於上洛,以置關防。癸丑,詔曰:乾道變化,陰陽所以消息,沿創不同,生靈所以順敘。若使天意不變,施化何以成四時,人事不易,為政何以厘萬姓!《易》不云乎:「通其變,使民不倦」;「變則通,通則久。」「有德則可久,有功則可大。」朕又聞之,安安而能遷,民用丕變。是故姬邑兩周,如武王之意,殷人五徙,成湯後之業。若不因人順天,功業見乎變,愛人治國者可不謂歟!然洛邑自古之都,王畿之內,天地之所合,陰陽之所和。控以三河,
 固以四塞,水陸通,貢賦等。故漢祖曰:「吾行天下多矣,唯見洛陽。」自古皇王,何嘗不留意,所不都者蓋有由焉。或以九州未一,或以困其府庫,作洛之制所以未暇也。我有隋之始,便欲創茲懷、洛,日復一日,越暨於今。



 念茲在茲,興言感哽!朕肅膺寶歷,纂臨萬邦,遵而不失,心奉先志。今者漢王諒悖逆,毒被山東,遂使州縣或淪非所。此由關河懸遠,兵不赴急,加以並州移戶,復在河南。周遷殷人,意在於此。況復南服遐遠,東夏殷大,因機順動,今也其時。



 群司百闢,僉諧厥議。但成周墟脊,弗堪葺宇。今可於伊、洛營建東京,便即設官分職,以為民極也。夫宮
 室之制本以便生,上棟下宇,足避風露,高臺廣廈,豈曰適形。故《傳》云:「儉德之共,侈惡之大。」宣尼有云:「與其不遜也,寧儉。」



 豈謂瑤臺瓊室方為宮殿者乎,土階採椽而非帝王者乎?是知非天下以奉一人,乃一人以主天下也。民惟國本,本固邦寧,百姓足,孰與不足!今所營構,務從節儉,無令雕墻峻宇復起於當今,欲使卑宮菲食將貽於後世。有司明為條格,稱朕意焉。



 十二月乙丑,以右武衛將軍來護兒為右驍衛大將軍。戊辰,以柱國李景為右武衛大將軍。以右衛率周羅為右武候大將軍。



 大業元年春正月壬辰朔,大赦,改元。立妃蕭氏為皇后。
 改豫州為溱州,洛州為豫州。廢諸州總管府。丙申,立晉王昭為皇太子。丁酉,以上柱國宇文述為左衛大將軍,上柱國郭衍為左武衛大將軍,延壽公於仲文為右衛大將軍。己亥,以豫章王暕為豫州牧。戊申,發八使巡省風俗。下詔曰:昔者哲王之治天下也,其在愛民乎。既富而教,家給人足,故能風淳俗厚,遠至邇安。治定功成,率由斯道。朕嗣膺寶歷,撫育黎獻,夙夜戰兢,若臨川谷。雖則聿遵先緒,弗敢失墜,永言政術,多有缺然。況以四海之遠,兆民之眾,未獲親臨,問其疾苦。每慮幽仄莫舉,冤屈不申,一物失所,乃傷和氣,萬方有罪,責在朕躬,所以
 寤寐增嘆,而夕惕載懷者也。今既布政惟始,宜存寬大。可分遣使人,巡省方俗,宣揚風化,薦拔淹滯,申達幽枉。孝悌力田,給以優復。鰥寡孤獨不能自存者,量加賑濟。義夫節婦,旌表門閭。高年之老,加其版授,並依別條,賜以粟帛。篤疾之徒,給侍丁者,雖有侍養之名,曾無賙贍之實,明加檢校,使得存養。



 若有名行顯著,操履修潔,及學業才能,一藝可取,咸宜訪採,將身入朝。所在州縣,以禮發遣。其有蠢政害人,不便於時者,使還之日,具錄奏聞。



 己酉,以吳州總管宇文弼為刑部尚書。二月己卯,以尚書左僕射楊素為尚書令。



 三月丁未,詔尚書令楊素、
 納言楊達、將作大匠宇文愷營建東京,徙豫州郭下居人以實之。戊申,詔曰:「聽採輿頌,謀及庶民,故能審政刑之得失。是知昧旦思治,欲使幽枉必達,彞倫有章。而牧宰任稱朝委,茍為徼幸,以求考課,虛立殿最,不存治實,綱紀於是弗理,冤屈所以莫申。關河重阻,無由自達。朕故建立東京,躬親存問。今將巡歷淮海,觀省風俗,眷求讜言,徒繁詞翰,而鄉校之內,闕爾無聞。



 恇然夕惕,用忘興寢。其民下有知州縣官人政治苛刻,侵害百姓,背公徇私,不便於民者,宜聽詣朝堂封奏,庶乎四聰以達,天下無冤。」又於皁澗營顯仁宮,採海內奇禽異獸草木之
 類,以實園苑。徙天下富商大賈數萬家於東京。辛亥,發河南諸郡男女百餘萬,開通濟渠,自西苑引穀、洛水達於河,自板渚引河通於淮。庚申,遣黃門侍郎王弘、上儀同於士澄往江南採木,造龍舟、鳳甗、黃龍、赤艦、樓船等數萬艘。夏四月癸亥,大將軍劉方擊林邑,破之。五月庚戌,民部尚書義豐侯韋沖卒。六月甲子,熒惑入太微。秋七月丁酉,制戰亡之家給復十年。丙午,滕王綸、衛王集並奪爵徙邊。閏七月甲子,以尚書令楊素為太子太師,安德王雄為太子太傅,河間王弘為太子太保。丙子,詔曰:君民建國,教學為先,移風易俗,必自茲始。而言絕義
 乖,多歷年代,進德修業,其道浸微。漢採坑焚之餘,不絕如線,晉承板蕩之運,掃地將盡。自時厥後,軍國多虞,雖復黌宇時建,示同愛禮,函丈或陳,殆為虛器。遂使紆青拖紫,非以學優,制錦操刀,類多墻面。上陵下替,綱維靡立,雅缺道消,實由於此。朕纂承洪緒,思弘大訓,將欲尊師重道,用闡厥繇,講信修睦,敦獎名教。方今宇宙平一,文軌攸同,十步之內,必有芳草,四海之中,豈無奇秀!諸在家及見入學者,若有篤志好古,耽悅典墳,學行優敏,堪膺時務,所在採訪,具以名聞,即當隨其器能,擢以不次。若研精經術,未願進仕者,可依其藝業深淺,門廕高
 卑,雖未升朝,並量準給祿。庶夫恂恂善誘,不日成器,濟濟盈朝,何遠之有!其國子等學,亦宜申明舊制,教習生徒,具為課試之法,以盡砥礪之道。



 八月壬寅,上御龍舟,幸江都。以左武衛大將軍郭衍為前軍,右武衛大將軍李景為後軍。文武官五品已上給樓船,九品已上給黃蔑。舳艫相接,二百餘里。冬十月己丑,赦江淮已南。揚州給復五年,舊總管內給復三年。十一月己未,以大將軍崔仲方為禮部尚書。



 二年春正月辛酉,東京成,賜監督者各有差。以大理卿梁毗為刑部尚書。丁卯,遣十使並省州縣。二月丙戌,詔
 尚書令楊素、吏部尚書牛弘、大將軍宇文愷、內史侍郎虞世基、禮部侍郎許善心制定輿服。始備輦路及五時副車。上常服,皮弁十有二琪,文官弁服,佩玉,五品已上給犢車、通幰,三公親王加油絡,武官平巾幘,褲褶,三品已上給瓟槊。下至胥吏,服色皆有差。非庶人不得戎服。戊戌,置都尉官。三月庚午,車駕發江都。先是,太府少卿何稠、太府丞云定興盛修儀仗,於是課州縣送羽毛。百姓求捕之,網羅被水陸,禽獸有堪氅毦之用者,殆無遺類。至是而成。夏四月庚戌,上自伊闕陳法駕,備千乘萬騎,入於東京。辛亥,上御端門,大赦,免天下今年租稅。癸
 丑,以冀州刺史楊文思為民部尚書。五月甲寅,金紫光祿大夫、兵部尚書李通坐事免。乙卯,詔曰:「旌表先哲,式存饗祀,所以優禮賢能,顯彰遺愛。朕永鑒前修,尚想名德,何嘗不興嘆九原,屬懷千載。其自古已來賢人君子,有能樹聲立德、佐世匡時、博利殊功、有益於人者,並宜營立祠宇,以時致祭。墳壟之處,不得侵踐。有司量為條式,稱朕意焉。」六月壬子,以尚書令、太子太師楊素為司徒。進封豫章王暕為齊王。秋七月癸丑,以衛尉卿衛玄為工部尚書。庚申,制百官不得計考增級,必有德行功能灼然顯著者擢之。壬戌,擢籓邸舊臣鮮於羅等二十
 七人官爵有差。甲戌,皇太子昭薨。乙亥,上柱國、司徒、楚國公楊素薨。八月辛卯,封皇孫倓為燕王,侗為越王,侑為代王。九月乙丑,立秦孝王俊子浩為秦王。冬十月戊子,以靈州刺史段文振為兵部尚書。十二月庚寅,詔曰:「前代帝王,因時創業,君民建國,禮尊南面。而歷運推移,年世永久,丘壟殘毀,樵牧相趨,塋兆堙蕪,封樹莫辨。興言淪滅,有愴於懷。自古已來帝王陵墓,可給隨近十戶,蠲其雜役,以供守視。



 三年春正月癸亥,敕並州逆黨已流配而逃亡者,所獲之處,即宜斬決。丙子,長星竟天,出於東壁,二旬而止。是
 月,武陽郡上言,河水清。二月己丑,彗星見於奎,掃文昌,歷大陵、五車、北河,入太微,掃帝坐,前後百餘日而止。三月辛亥,車駕還京師。壬子,以大將軍姚辯為左屯衛將軍。癸丑,遣羽騎尉硃寬使於流求國。乙卯,河間王弘薨。夏四月庚辰,詔曰:「古者帝王觀風問俗,皆所以憂勤兆庶,安集遐荒。自蕃夷內附,未遑親撫,山東經亂,須加存恤。今欲安輯河北,巡省趙、魏。所司依式。」甲申,頒律令,大赦天下,關內給復三年。壬辰,改州為郡。改度量權衡,並依古式。改上柱國已下官為大夫。甲午,詔曰:天下之重,非獨治所安,帝王之功,豈一士之略。自古明君哲後,立
 政經邦,何嘗不選賢與能,收採幽滯。周稱多士,漢號得人,常想前風,載懷欽佇。朕負扆夙興,冕旒待旦,引領巖谷,置以周行,冀與群才共康庶績。而匯茅寂寞,投竿罕至,豈美璞韜採,未值良工,將介石在懷,確乎難拔?永鑒前哲,憮然興嘆!凡厥在位,譬諸股肱,若濟巨川,義同舟楫。豈得保茲寵祿,晦爾所知,優游卒歲,甚非謂也。祁大夫之舉善,良史以為至公,臧文仲之蔽賢,尼父譏其竊位。求諸往古,非無褒貶,宜思進善,用匡寡薄。夫孝悌有聞,人倫之本,德行敦厚,立身之基。



 或節義可稱,或操履清潔,所以激貪厲俗,有益風化。強毅正直,執憲不撓,學
 業優敏,文才美秀,並為廊廟之用,實乃瑚璉之資。才堪將略,則拔之以禦侮,膂力驍壯,則任之以爪牙。爰及一藝可取,亦宜採錄,眾善畢舉,與時無棄。以此求治,庶幾非遠。文武有職事者,五品已上,宜依令十科舉人。有一於此,不必求備。朕當待以不次,隨才升擢。其見任九品已上官者,不在舉送之限。



 丙申,車駕北巡狩。丁酉,以刑部尚書宇文弼為禮部尚書。戊戌,敕百司不得踐暴禾稼,其有須開為路者,有司計地所收,即以近倉酬賜,務從優厚。己亥,次赤岸澤。以太牢祭故太師李穆墓。五月丁巳,突厥啟民可汗遣子拓特勤來朝。戊午,發河北十
 餘郡丁男鑿太行山,達於並州,以通馳道。丙寅,啟民可汗遣其兄子毗黎伽特勤來朝。辛未,啟民可汗遣使請自入塞,奉迎輿駕。上不許。癸酉,有星孛於文昌上將,星皆動搖。六月辛巳,獵於連穀。丁亥,詔曰:聿追孝饗,德莫至焉,崇建寢廟,禮之大者。然則質文異代,損益殊時,學滅坑焚,經典散逸,憲章湮墜,廟堂制度,師說不同。所以世數多少,莫能是正,連室異宮,亦無準定。朕獲奉祖宗,欽承景業,永惟嚴配,思隆大典。於是詢謀在位,博訪儒術。咸以為高祖文皇帝受天明命,奄有區夏,拯群飛於四海,革凋敝於百王,恤獄緩刑,生靈皆遂其性,輕徭薄
 賦,比屋各安其業。恢夷宇宙,混壹車書。東漸西被,無思不服,南征北怨,俱荷來蘇。駕毳乘風,歷代所弗至,辮發左衽,聲教所罕及,莫不厥角關塞,頓顙闕庭。譯靡絕時,書無虛月,韜戈偃武,天下晏如。



 嘉瑞休徵,表裏禔福,猗歟偉歟,無得而名者也。朕又聞之,德厚者流光,治辨者禮縟。是以周之文、武,漢之高、光,其典章特立,謚號斯重,豈非緣情稱述,即崇顯之義乎?高祖文皇帝宜別建廟宇,以彰巍巍之德,仍遵月祭,用表蒸蒸之懷。



 有司以時創選,務合典制。又名位既殊,禮亦異等。天子七廟,事著前經,諸侯二昭,義有差降,故其以多為貴。王者之禮,今
 可依用,貽厥後昆。



 戊子,次榆林郡。丁酉,啟民可汗來朝。己亥,吐谷渾、高昌並遣使貢方物。



 甲辰,上御北樓,觀漁於河,以宴百僚。秋七月辛亥,啟民可汗上表請變服,襲冠帶。詔啟民贊拜不名,位在諸侯王上。甲寅,上於郡城東禦大帳,其下備儀衛,建旌旗,宴啟民及其部落三千五百人,奏百戲之樂。賜啟民及其部落各有差。丙子,殺光祿大夫賀若弼、禮部尚書宇文弼、太常卿高熲。尚書左僕射蘇威坐事免。發丁男百餘萬築長城,西距榆林,東至紫河,一旬而罷,死者十五六。八月壬午,車駕發榆林。乙酉,啟民飾廬清道,以候乘輿。帝幸其帳,啟民奉觴
 上壽,宴賜極厚。



 上謂高麗使者曰:「歸語爾王,當早來朝見。不然者,吾與啟民巡彼土矣。」皇后亦幸義城公主帳。己丑,啟民可汗歸蕃。癸巳,入樓煩關。壬寅,次太原。詔營晉陽宮。九月己未,次濟源。幸御史大夫張衡宅,宴享極歡。己巳,至於東都。壬申,以齊王暕為河南尹、開府儀同三司。癸酉,以民部尚書楊文思為納言。



 四年春正月乙巳,詔發河北諸郡男女百餘萬開永濟渠,引沁水,南達於河,北通涿郡。庚戌,百僚大射於允武殿。丁卯,賜城內居民米各十石。壬申,以太府卿元壽為內史令,鴻臚卿楊玄感為禮部尚書。癸酉,以工部尚書
 衛玄為右候衛大將軍,大理卿長孫熾為民部尚書。二月己卯,遣司朝謁者崔毅使突厥處羅,致汗血馬。三月辛酉,以將作大匠宇文愷為工部尚書。壬戌,百濟、倭、赤土、迦羅舍國並遣使貢方物。乙丑,車駕幸五原,因出塞巡長城。丙寅,遣屯田主事常駿使赤土,致羅剎。夏四月丙午,以離石之汾源、臨泉、雁門之秀容為樓煩郡。起汾陽宮。癸丑,以河內太守張定和為左屯衛大將軍。乙卯,詔曰:「突厥意利珍豆啟民可汗率領部落,保附關塞,遵奉朝化,思改戎俗,頻入謁覲,屢有陳請。以氈墻毳幕,事窮荒陋,上棟下宇,願同比屋。誠心懇切,朕之所重。宜於
 萬壽戍置城造屋,其帷帳床褥已上,隨事量給,務從優厚,稱朕意焉。」五月壬申,蜀郡獲三足烏,張掖獲玄狐,各一。秋七月辛巳,發丁男二十餘萬築長城,自榆谷而東。乙未,左翊衛大將軍宇文述破吐谷渾於曼頭、赤水。八月辛酉,親祠恆岳,河北道郡守畢集。大赦天下。車駕所經郡縣,免一年租調。九月辛未,徵天下鷹師悉集東京,至者萬餘人。



 戊寅,彗星出於五車,掃文昌,至房而滅。辛巳,詔免長城役者一年租賦。冬十月丙午,詔曰:「先師尼父,聖德在躬,誕發天縱之姿,憲章文武之道。命世膺期,蘊茲素王,而頹山之嘆,忽逾於千祀,盛德之美,不存
 於百代。永惟懿範,宜有優崇。可立孔子後為紹聖侯。有司求其苗裔,錄以申上。」辛亥,詔曰:「昔周王下車,首封唐虞之胤,漢帝承歷,亦命殷周之後。皆所以褒立先代,憲章在昔。朕嗣膺景業,傍求雅訓,有一弘益,欽若令典。以為周兼夏殷,文質大備,漢有天下,車書混一,魏晉沿襲,風流未遠。並宜立後,以存繼絕之義。有司可求其胄緒列聞。」



 乙卯,頒新式於天下。



 五年春正月丙子,改東京為東都。癸未,詔天下均田。戊子,上自東都還京師。



 己丑,制民間鐵叉、搭鉤、槊刃之類,皆禁絕之。太守每歲密上屬官景跡。二月戊戌,次於閿
 鄉。詔祭古帝王陵及開皇功臣墓。庚子,制魏、周官不得為廕。辛丑,赤土國遣使貢方物。戊申,車駕至京師。丙辰,宴耆舊四百人於武德殿,頒賜各有差。己未,上御崇德殿之西院,愀然不怡,顧謂左右曰:「此先帝之所居,實用增感,情所未安,宜於此院之西別營一殿。」壬戌,制父母聽隨子之官。三月己巳,車駕西巡河右。庚午,有司言,武功男子史永遵與從父昆弟同居。上嘉之,賜物一百段,米二百石,表其門閭。乙亥,幸扶風舊宅。夏四月己亥,大獵於隴西。壬寅,高昌、吐谷渾、伊吾並遣使來朝。乙巳,次狄道,黨項羌來貢方物。癸亥,出臨津關,渡黃河,至西平,
 陳兵講武。五月乙亥,上大獵於拔延山,長圍周亙二千里。



 庚辰,入長寧穀。壬午,度星嶺。甲申,宴群臣於金山之上。丙戌,梁浩亹御馬度而橋壞,斬朝散大夫黃亙及督役者九人。吐谷渾王率眾保覆袁川,帝分命內史元壽南屯金山,兵部尚書段文振北屯雪山,太僕卿楊義臣東屯琵琶峽,將軍張壽西屯泥嶺,四面圍之。渾主伏允以數十騎遁出,遣其名王詐稱伏允,保車我真山。壬辰,詔右屯衛大將軍張定和往捕之。定和挺身挑戰,為賊所殺。亞將柳武建擊破之,斬首數百級。甲午,其仙頭王被圍窮蹙,率男女十餘萬口來降。六月丁酉,遣左光祿
 大夫梁默、右翊衛將軍李瓊等追渾主,皆遇賊死之。癸卯,經大斗拔谷,山路隘險,魚貫而出。風霰晦冥,與從官相失,士卒凍死者太半。丙午,次張掖。辛亥,詔諸郡學業該通才藝優洽、膂力驍壯超絕等倫、在官勤奮堪理政事、立性正直不避強御四科舉人。壬子,高昌王麴伯雅來朝,伊吾吐屯設等獻西域數千里之地。上大悅。



 癸丑,置西海、河源、鄯善、且末等四郡。丙辰,上御觀風行殿,盛陳文物,奏九部樂,設魚龍曼延,宴高昌王、吐屯設於殿上,以寵異之。其蠻夷陪列者三十餘國。



 戊午,大赦天下。開皇已來流配,悉放還鄉,晉陽逆黨,不在此例。隴右諸
 郡,給復一年,行經之所,給復二年。秋七月丁卯,置馬牧於青海渚中,以求龍種,無效而止。九月癸未,車駕入長安。冬十月癸亥,詔曰:「優德尚齒,載之典訓,尊事乞言,義彰膠序。鬻熊為師,取非筋力,方叔元老,克壯其猷。朕永言稽古,用求至治,是以龐眉黃發,更令收敘,務簡秩優,無虧藥膳,庶等臥治,佇其弘益。今歲耆老赴集者,可於近郡處置。年七十以上,疾患沉滯,不堪居職,即給賜帛,送還本郡。其官至七品已上者,量給廩,以終厥身。」十一月丙子,車駕幸東都。



 六年春正月癸亥朔,旦,有盜數十人,皆素冠練衣,焚香
 持華,自稱彌勒佛,入自建國門。監門者皆稽首。既而奪衛士仗,將為亂。齊王暕遇而斬之。於是都下大索,與相連坐者千餘家。丁丑,角抵大戲於端門街,天下奇伎異藝畢集,終月而罷。帝數微服往觀之。己丑,倭國遣使貢方物。二月乙巳,武賁郎將陳棱、朝請大夫張鎮州擊流求,破之,獻俘萬七千口,頒賜百官。乙卯,詔曰:「夫帝圖草創,王業艱難,咸仗股肱,葉同心德,用能拯厥頹運,克膺大寶,然後疇庸茂賞,開國承家,誓以山河,傳之不朽。近代喪亂,四海未一,茅土妄假,名實相乖,歷茲永久,莫能懲革。皇運之初,百度伊始,猶循舊貫,未暇改作,今天下
 交泰,文軌攸同,宜率遵先典,永垂大訓。自今已後,唯有功勛乃得賜封,仍令子孫承襲。」丙辰,改封安德王雄為觀王,河間王子慶為郇王。庚申,征魏、齊、周、陳樂人,悉配太常。三月癸亥,幸江都宮。甲子,以鴻臚卿史祥為左驍衛大將軍。夏四月丁未,宴江淮已南父老,頒賜各有差。六月辛卯,室韋、赤土並遣使貢方物。壬辰,雁門賊帥尉文通聚眾三千,保於莫壁穀。遣鷹揚楊伯泉擊破之。甲寅,制江都太守秩同京尹。冬十月壬申,刑部尚書梁毗卒。壬子,民部尚書、銀青光祿大夫長孫熾卒。



 十二月己未,左光祿大夫、吏部尚書牛弘卒。辛酉,硃崖人王萬昌
 舉兵作亂,遣隴西太守韓洪討平之。



 七年春正月壬寅,左武衛大將軍、光祿大夫、真定侯郭衍卒。二月己未,上升釣臺,臨揚子津,大宴百僚,頒賜各有差。庚申,百濟遣使朝貢。乙亥,上自江都御龍舟入通濟渠,遂幸於涿郡。壬午,詔曰:「武有七德,先之以安民;政有六本,興之以教義。高麗高元,虧失籓禮,將欲問罪遼左,恢宣勝略。雖懷伐國,仍事省方。今往涿郡,巡撫民俗。其河北諸郡及山西、山東年九十已上者,版授太守,八十者授縣令。」三月丁亥,右光祿大夫、左屯衛大將軍姚辯卒。夏四月庚午,至涿郡之臨朔宮。五月戊子,以武威
 太守樊子蓋為民部尚書。秋,大水,山東、河南漂沒三十餘郡,民相賣為奴婢。冬十月乙卯,底柱山崩,偃河逆流數十里。戊午,以東平太守吐萬緒為左屯衛大將軍。十二月己未,西面突厥處羅多利可汗來朝。上大悅,接以殊禮。於時遼東戰士及餽運者填咽於道,晝夜不絕,苦役者始為群盜。甲子,敕都尉、鷹揚與郡縣相知追捕,隨獲斬決之。



\end{pinyinscope}