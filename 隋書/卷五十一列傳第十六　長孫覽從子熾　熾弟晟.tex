\article{卷五十一列傳第十六 長孫覽從子熾 熾弟晟}

\begin{pinyinscope}

 長孫覽,字休因,河南洛陽人也。祖稚,魏太師、假黃鉞、上黨文宣王。父紹遠,周小宗伯、上黨郡公。覽性弘雅,有器量,略涉書記,尤曉鐘律。魏大統中,起家東宮親信。周明帝時,為大都督。武帝在籓,與覽親善,及即位,彌加禮焉,超拜車騎大將軍。每公卿上奏,必令省讀。覽有口辯,聲氣雄壯,凡所宣傳,百僚屬目,帝每嘉嘆之。覽初名善,帝
 謂之曰:「朕以萬機,委卿先覽。」遂賜名焉。



 及誅宇文護,以功進封薛國公。其後歷小司空。從平齊,進位柱國,封第二子寬管國公。宣帝時,進位上柱國、大司徒,俄歷同、涇二州刺史。高祖為丞相,轉宜州刺史。



 開皇二年,將有事於江南,徵為東南道行軍元帥,統八總管出壽陽,水陸俱進。



 師臨江,陳人大駭。會陳宣帝卒,覽欲乘釁遂滅之,監軍高熲以禮不伐喪而還。上常命覽與安德王雄、上柱國元諧、李充、左僕射高熲、右衛大將軍虞慶則、吳州總管賀若弼等同宴,上曰:「朕昔在周朝,備展誠節,但苦猜忌,每致寒心。為臣若此,竟何情賴?朕之於公,義則
 君臣,恩猶父子。朕當與公共享終吉,罪非謀逆,一無所問。朕亦知公至誠,特付太子,宜數參見之,庶得漸相親愛。柱臣素望,實屬於公,宜識朕意。」其恩禮如此。又為蜀王秀納覽女為妃。其後以母憂去職。歲餘,起令復位。俄轉涇州刺史,所在並有政績。卒官。子洪嗣。仕歷宋順臨三州刺史、司農少卿、北平太守。



 熾字仲光,上黨文宣王稚之曾孫也。祖裕,魏太常卿、冀州刺史。父兕,周開府儀同三司、熊絳二州刺史、平原侯。熾性敏慧,美姿儀,頗涉群書,兼長武藝。



 建德初,武帝尚道法,尤好玄言,求學兼經史、善於談論者,為通道館學
 士。熾應其選,與英俊並游,通涉彌博。建德二年,授雍州倉城令,尋轉盩啡令。頻宰二邑,考績連最,遷崤郡守。入為御正上士。高祖作相,擢為丞相府功曹參軍,加大都督,封陽平縣子,邑二百戶。遷稍伯下大夫。其年王謙反,熾從信州總管王長述溯江而上。以熾為前軍,破謙一鎮,定楚、合等五州,擒偽總管荊山公元振,以功拜儀同三司。及高祖受禪,熾率官屬先入清宮,即日授內史舍人、上儀同三司。尋以本官攝判東宮右庶子,出入兩宮,甚被委遇。加以處事周密,高祖每稱美之。授左領軍長史,持節,使於東南道三十六州,廢置州郡,巡省風俗。還
 授太子僕,加諫議大夫,攝長安令。與大興令梁毗俱為稱職。然毗以嚴正聞,熾以寬平顯,為政不同,部內各化。尋領右常平監,遷雍州贊治,改封饒良縣子。遷鴻臚少卿。後數歲,轉太常少卿,進位開府儀同三司。復持節為河南道二十八州巡省大使,於路授吏部侍郎。大業元年,遷大理卿,復為西南道大使,巡省風俗。擢拜戶部尚書。吐谷渾寇張掖,令熾率精騎五千擊走之,追至青海而還,以功授銀青光祿大夫。六年,幸江都宮,留熾於東都居守,仍攝左候衛將軍事。其年卒官,時年六十二。謚曰靜。子安世,通事謁者。



 晟字季晟,性通敏,略涉書記,善彈工射,趫捷過人。時周室尚武,貴游子弟咸以相矜,每共馳射,時輩皆出其下。年十八,為司衛上士。初未知名,人弗之識也,唯高祖一見,深嗟異焉,乃攜其手而謂人曰:「長孫郎武藝逸群,適與其言,又多奇略。後之名將,非此子邪?」



 宣帝時,突厥攝圖請婚於周,以趙王招女妻之。然周與攝圖各相誇競,妙選驍勇以充使者,因遣晟副汝南公宇文神慶送千金公主至其牙。前後使人數十輩,攝圖多不禮,見晟而獨愛焉,每共游獵,留之竟歲。嘗有二雕,飛而爭肉,因以兩箭與晟曰:「請射取之。」晟乃彎弓馳往,遇雕相攫,遂一
 發而雙貫焉。攝圖喜,命諸子弟貴人皆相親友,冀暱近之,以學彈射。其弟處羅侯號突利設,尤得眾心。而為攝圖所忌,密托心腹,陰與晟盟。晟與之游獵,因察山川形勢,部眾強弱,皆盡知之。時高祖作相,晟以狀白高祖。高祖大喜,遷奉車都尉。



 至開皇元年,攝圖曰:「我周家親也,今隋公自立而不能制,復何面目見可賀敦乎」?因與高寶寧攻陷臨渝鎮,約諸面部落謀共南侵。高祖新立,由是大懼,修築長城,發兵屯北境,命陰壽鎮幽州,虞慶則鎮並州,屯兵數萬人以為之備。晟先知攝圖、玷厥、阿波、突利等叔侄兄弟各統強兵,俱號可汗,分居四面,內懷
 猜忌,外示和同,難以力征,易可離間,因上書曰:「臣聞喪亂之極,必致升平,是故上天啟其機,聖人成其務。伏惟皇帝陛下當百王之末,膺千載之期,諸夏雖安,戎場尚梗,興師致討,未是其時,棄於度外,又復侵擾。故宜密運籌策,漸以攘之,計失則百姓不寧,計得則萬代之福。吉兇所系,伏願詳思。臣於周末,忝充外使,匈奴倚伏,實所具知。玷厥之於攝圖,兵強而位下,外名相屬,內隙已彰,鼓動其情,必將自戰。又處羅侯者,攝圖之弟,奸多而勢弱,曲取於眾心,國人愛之,因為攝圖所忌,其心殊不自安,跡示彌縫,實懷疑懼。又阿波首鼠,介在其間,頗畏攝
 圖,受其牽率,唯強是與,未有定心。今宜遠交而近攻,離強而合弱,通使玷厥,說合阿波,則攝圖回兵,自防右地。又引處羅,遣連奚、霫,則攝圖分眾,還備左方。



 首尾猜嫌,腹心離阻,十數年後,承釁討之,必可一舉而空其國矣。」上省表大悅,因召與語。晟復口陳形勢,手畫山川,寫其虛實,皆如指掌。上深嗟異,皆納用焉。



 因遣太僕元暉出伊吾道,使詣玷厥,賜以狼頭纛,謬為欽敬,禮數甚優。玷厥使來,引居攝圖使上。反間既行,果相猜貳。授晟車騎將軍,出黃龍道,齎幣賜奚、霫、契丹等,遣為向導,得至處羅侯所,深布心腹,誘令內附。



 二年,攝圖四十萬騎自蘭
 州入,至於周盤,破達奚長儒軍,更欲南入。玷厥不從,引兵而去。時晟又說染干詐告攝圖曰:「鐵勒等反,欲襲其牙。」攝圖乃懼,回兵出塞。



 後數月,突厥大入,發八道元帥分出拒之。阿波至涼州,與竇榮定戰,賊帥累北。時晟為偏將,使謂之曰:「攝圖每來,戰皆大勝。阿波才入,便即致敗,此乃突厥之恥,豈不內愧於心乎?且攝圖之與阿波,兵勢本敵。今攝圖日勝,為眾所崇,阿波不利,為國生辱。攝圖必當因以罪歸於阿波,成其夙計,滅北牙矣。願自量度,能御之乎?」阿波使至,晟又謂之曰:「今達頭與隋連和,而攝圖不能制。可汗何不依附天子,連結達頭,相合
 為強,此萬全之計。豈若喪兵負罪,歸就攝圖,受其戮辱邪?」阿波納之,因留塞上,使人隨晟入朝。時攝圖與衛王軍遇,戰於白道,敗走至磧。聞阿波懷貳,乃掩北牙,盡獲其眾而殺其母。阿波還無所歸,西奔玷厥,乞師十餘萬,東擊攝圖,復得故地,收散卒數萬,與攝圖相攻。阿波頻勝,其勢益張。攝圖又遣使朝貢,公主自請改姓,乞為帝女,上許之。



 四年,遣晟副虞慶則使於攝圖,賜公主姓為楊氏,改封大義公主。攝圖奉詔,不肯起拜,晟進曰:「突厥與隋俱是大國天子,可汗不起,安敢違意。但可賀敦為帝女,則可汗是大隋女婿,奈何無禮,不敬婦公乎?」攝圖
 乃笑謂其達官曰:「須拜婦公,我從之耳。」於是乃拜詔書。使還稱旨,授儀同三司、左勛衛車騎將軍。



 七年,攝圖死,遣晟持節拜其弟處羅侯為莫何可汗,以其子雍閭為葉護可汗。



 處羅侯因晟奏曰:「阿波為天所滅,與五六千騎在山谷間,伏聽詔旨,當取之以獻。」乃召文武議焉。樂安公元諧曰:「請就彼梟首,以懲其惡。」武陽公李充曰:「請生將入朝,顯戮以示百姓。」上謂晟曰:「於卿何如?」晟對曰:「若突厥背誕,須齊之以刑。今其昆弟自相夷滅,阿波之惡,非負國家,因其困窮,取而為戮,恐非招遠之道,不如兩存之。」上曰:「善。」八年,處羅侯死,遣晟往吊,仍齎陳國所
 獻寶器以賜雍閭。



 十三年,流人楊欽亡入突厥,詐言彭公劉昶共宇文氏女謀欲反隋,稱遣其來,密告主。雍閭信之,乃不修職貢。又遣晟出使,微觀察焉。公主見晟,乃言辭不遜,又遣所私胡人安遂迦共欽計議,扇惑雍閭。晟至京師,具以狀奏。又遣晟往索欽,雍閭欲勿與,謬答曰:「檢校客內,無此色人。」晟乃貨其達官,知欽所在,夜掩獲之,以示雍閭,因發公主私事,國人大恥。雍閭執遂迦等,並以付晟。上大喜,加授開府,仍遣入籓,蒞殺大義公主。雍閭又表請婚,僉議將許之。晟又奏曰:「臣觀雍閭,反覆無信,特共玷厥有隙,所以依倚國家。縱與為婚,
 終當必叛。今若得尚公主,承藉威靈,玷厥、染干必又受其徵發。強而更反,後恐難圖。且染干者,處羅侯之子也,素有誠款,於今兩代。臣前與相見,亦乞通婚,不如許之,招令南徙,兵少力弱,易可撫馴,使敵雍閭,以為邊捍。」上曰:「善。」又遣慰喻染干,許尚公主。



 十七年,染干遣五百騎隨晟來逆女,以宗女封安義公主以妻之。晟說染干率眾南徙,居度斤舊鎮。雍閭疾之,亟來抄略。染干伺知動靜,輒遣奏聞,是以賊來每先有備。



 十九年,染干因晟奏,雍閭作攻具,欲打大同城。詔發六總管,並取漢王節度,分道出塞討之。雍閭大懼,復共達頭同盟,合力掩襲染
 干,大戰於長城下。染干敗績,殺其兄弟子侄,而部落亡散。染干與晟獨以五騎逼夜南走,至旦,行百餘里,收得數百騎,乃相與謀曰:「今兵敗入朝,一降人耳,大隋天子豈禮我乎?玷厥雖來,本無冤隙,若往投之,必相存濟。」晟知其懷貳,乃密遣從者入伏遠鎮,令速舉烽。染干見四烽俱發,問晟曰:「城上然烽何也?」晟紿之曰:「城高地迥,必遙見賊來。我國家法,若賊少舉二烽,來多舉三烽,大逼舉四烽,使見賊多而又近耳。」染干大懼,謂其眾曰:「追兵已逼,且可投城。」既入鎮,晟留其達官執室以領其眾,自將染干馳驛入朝。帝大喜,進授左勛衛驃騎將軍,持節
 護突厥。晟遣降虜覘候雍閭,知其牙內屢有災變,夜見赤虹,光照數百里,天狗隕,雨血三日,流星墜其營內,有聲如雷。每夜自驚,言隋師且至。並遣奏知,仍請出討突厥。都速等歸染干,前後至者男女萬餘口,晟安置之。由是突厥悅附。尋以染干為意利珍豆啟人可汗,賜射於武安殿。選善射者十二人,分為兩朋。啟人曰:「臣由長孫大使得見天子,今日賜射,願入其朋。」許之。給晟箭六侯,發皆入鹿,啟人之朋竟勝。時有群飛,上曰:「公善彈,為我取之。」十發俱中,並應丸而落。是日百官獲賚,晟獨居多。尋遣領五萬人,於朔州築大利城以處染干。安義公
 主死,持節送義城公主,復以妻之。晟又奏:「染干部落歸者既眾,雖在長城之內,猶被雍閭抄略,往來辛苦,不得寧居。請徙五原,以河為固,於夏、勝兩州之間,東西至河,南北四百里,掘為橫塹,令處其內,任情放牧,免於抄略,人必自安。」上並從之。



 二十年,都藍大亂,為其部下所殺。晟因奏請曰:「今王師臨境,戰數有功,賊內攜離,其主被殺,乘此招誘,必並來降,請遣染干部下分頭招慰。」上許之,果盡來附。達頭恐怖,又大集兵。詔晟部領降人,為秦川行軍總管,取晉王廣節度出討。達頭與王相抗,晟進策曰:「突厥飲泉,易可行毒。」因取諸藥毒水上流,達頭人
 畜飲之多死,於是大驚曰:「天雨惡水,其亡我乎?」因夜遁。晟追之,斬首千餘級,俘百餘口,六畜數千頭。王大喜,引晟入內,同宴極歡。有突厥達官來降,時亦預坐,說言突厥之內,大畏長孫總管,聞其弓聲,謂為霹靂,見其走馬,稱為閃電。王笑曰:「將軍震怒,威行域外,遂與雷霆為比,一何壯哉!」師旋,授上開府儀同三司,復遣還大利城,安撫新附。



 仁壽元年,晟表奏曰:「臣夜登城樓,望見磧北有赤氣,長百餘里,皆如雨足,下垂被地。謹驗兵書,此名灑血,其下之國必且破亡。欲滅匈奴,宜在今日。」詔楊素為行軍元帥,晟為受降使者,送染干北伐。二年,軍次北河,
 值賊帥思力俟斤等領兵拒戰,晟與大將軍梁默擊走之,轉戰六十餘里,賊眾多降。晟又教染干分遣使者,往北方鐵勒等部招攜取之。三年,有鐵勒、思結、伏利具、渾、斛薩、阿拔、僕骨等十餘部,盡背達頭,請來降附。達頭眾大潰,西奔吐谷渾。晟送染干安置於磧口。



 事畢,入朝,遇高祖崩,匿喪未發。煬帝引晟於大行前委以內衙宿衛,知門禁事,即日拜左領軍將軍。遇楊諒作逆,敕以本官為相州刺史,發山東兵馬,與李雄等共經略之。晟辭曰:「有男行布,今在逆地,忽蒙此任,情所不安。」帝曰:「公著勤誠,朕之所悉。今相州之地,本是齊都,人俗澆浮,易可搔
 擾。儻生變動,賊勢即張,思所以鎮之,非公莫可。公體國之深,終不可以兒害義,故用相委,公其勿辭。」於是遣捉相州。諒破,追還,轉武衛將軍。



 大業三年,煬帝幸榆林,欲出塞外,陳兵耀武,經突厥中,指於涿郡。仍恐染干驚懼,先遣晟往喻旨,稱述帝意。染干聽之,因召所部諸國,奚、霫、室韋等種落數十酋長咸萃。晟以牙中草穢,欲令染干親自除之,示諸部落,以明威重,乃指帳前草曰:「此根大香。」染干遽嗅之曰:「殊不香也。」晟曰:「天子行幸所在,諸侯躬親灑掃,耘除御路,以表至敬之心。今牙中蕪穢,謂是留香草耳。」染干乃悟曰:「奴罪過。奴之骨肉,皆天子賜
 也,得效筋力,豈敢有辭?特以邊人不知法耳,賴將軍恩澤而教導之。將軍之惠,奴之幸也。」遂拔所佩刀,親自芟草,其貴人及諸部爭放效之。乃發榆林北境,至於其牙,又東達於薊,長三千里,廣百步,舉國就役而開御道。帝聞晟策,乃益嘉焉。後除淮陽太守,未赴任,復為右驍衛將軍。



 五年,卒,時年五十八。帝深悼惜之,賵贈甚厚。後突厥圍雁門,帝嘆曰:「向使長孫晟在,不令匈奴至此!」晟好奇計,務功名。性至孝,居憂毀瘠,為朝士所稱。貞觀中,追贈司空、上柱國、齊國公,謚曰獻。少子無忌嗣。



 其長子行布,亦多謀略,有父風。起家漢王諒庫真,甚見親狎。後遇
 諒於並州起逆,率眾南拒官軍,乃留行布城守,遂與豆盧毓等閉門拒諒,城陷,遇害。次子恆安,以兄功授鷹揚郎將。



 史臣曰:長孫氏爰自代陰,來儀京洛,門傳鍾鼎,家誓山河。漢代八王,無以方其茂績;張氏七葉,不能譬此重光。覽獨擅雄辨,熾早稱爽俊,俱司禮閣,並統師旅,且公且侯,文武不墜。晟體資英武,兼包奇略,因機制變,懷彼戎夷。傾巢盡落,屈膝稽顙,塞垣絕鳴鏑之旅,渭橋有單于之拜。惠流邊朔,功光王府,保茲爵祿,不亦宜乎!



\end{pinyinscope}