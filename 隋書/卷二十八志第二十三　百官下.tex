\article{卷二十八志第二十三 百官下}

\begin{pinyinscope}

 高
 祖既受命,改周之六官,其所制名,多依前代之法。置三師、三公及尚書、門下、內史、秘書、內侍等省,御史、都水等臺,太常、光祿、衛尉、宗正、太僕、大理、鴻臚、司農、太府、國子、將作等寺,左右衛、左右武衛、左右武候、左右領、左右監門、左右領軍等府,分司統職焉。



 三師,不主事,不置府僚,蓋與天子坐而論道者也。



 三公,參議國之大事,依後齊置府僚。無其人則闕。祭祀則太尉亞獻,司徒奉俎,司空行掃除。其位多曠,皆攝行事。尋省府及僚佐,置公則坐於尚書都省。朝之眾務,總歸於臺閣。



 尚書省,事無不總。置令、左右僕射各一人,總吏部、禮部、兵部、都官、度支、工部等六曹事,是為八座。屬官左、右丞各一人,都事八人,分司管轄,吏部尚書統吏部侍郎二人,主爵侍郎一人,司勛侍郎二人,考功侍郎一人。禮部尚書統禮部、祠部侍郎各一人,主客、膳部侍郎各二人。兵部尚書統兵部、職方侍郎各二人,駕部、庫部侍郎各
 一人。都官尚書統都官侍郎二人,刑部、比部侍郎各一人,司門侍郎二人。度支尚書統度支、戶部侍郎各二人,金部、倉部侍郎各一人。工部尚書統工部、屯田侍郎各二人,虞部、水部侍郎各一人。凡三十六侍郎,分司曹務,直宿禁省,如漢之制。



 門下省,納言二人,給事黃門侍郎四人,錄事、通事令史各六人。又有散騎常侍、通直散騎常侍各四人,諫議大夫七人,散騎侍郎四人,員外散騎常侍六人,通直散騎侍郎四人,並掌部從朝直。又有給事二十人,員外散騎侍郎二十人,奉朝請四十人,並掌同散騎常侍等,兼出
 使勞問。統城門、尚食、尚藥、符璽、御府、殿內等六局。城門局,校尉二人,直長四人。尚食局,典御二人,直長四人,食醫四人。尚藥局,典御二人,侍御醫、直長各四人,醫師四十人。符璽、御府,殿內局,監各二人,直長各四人。



 內史省,置監、令各一人。尋廢監。置令二人,侍郎四人,舍人八人,通事舍人十六人,主書十人,錄事四人。



 秘書省,監、丞各一人,郎四人,校書郎十二人,正字四人,錄事二人。領著作、太史二曹。著作曹,置郎二人,佐郎八人,校書郎、正字各二人。太史曹,置令、丞各二人,司歷二人,監候四人。其歷、天文、漏刻、視昆,各有博士及生員。



 內侍省,內侍、內常侍各二人,內給事四人,內謁者監六人,內寺伯二人,內謁者十二人,寺人六人,伺非八人。並用宦者。領內尚食、掖庭、宮闈、奚官、內僕、內府等局。尚食,置典御及丞各二人。餘各置令、丞,皆二人。其宮闈、內僕,則加置丞各一人。掖庭又有宮教博士二人。



 御史臺,大夫一人,治書侍御史二人,侍御史八人,殿內侍御史、監察御史,各十二人,錄事二人。後魏延昌中,王顯有寵於宣武,為御史中尉,請革選御史。



 此後踵其事,每一中尉,則更置御史。自開皇后,始自吏部選用,仍依舊入直禁中。



 都水臺,使者及丞各二人,參軍三十人,河堤謁者六十
 人,錄事二人。領掌船局、都水尉二人,又領諸津。上津每尉一人,丞二人。中津每尉、丞各一人。下津每典作一人,津長四人。



 太常、光祿、衛尉、宗正、太僕、大理、鴻臚、司農、太府等九寺,並置卿少卿各一人。太僕尋加少卿一人。各置丞,太常、衛尉、宗正、大理、鴻臚、將作二人,光祿、太僕各三人,司農五人,太府六人。主簿、太府四人。餘寺各二人。錄事各二人。光祿則加至三人,司農、太府則各四人。等員。



 太常寺又有博士四人,協律郎二人,奉禮郎十六人。統郊社、太廟、諸陵、太祝、衣冠、太樂、清商、鼓吹、太醫、太卜、廩犧等署。各置令、並一人。太樂、太醫則各加至二人。丞。各一人。郊社、太樂、鼓吹則各加至二人。
 郊社署又有典瑞。四人。太祝署有太祝。二人。太樂署、清商署,各有樂師員。太樂八人,清商二人。鼓吹署有哄師。二人。太醫署有主藥、二人。醫師、二百人。藥園師、二人。



 醫博士、二人。助教、二人。按摩博士、二人。祝禁博士二人。等員。太卜署有卜師、二十人。相師、十人。男覡、十六人。女巫、八人。太卜博士、助教、各二人。



 相博士、助教各一人。等員。



 光祿寺統太官、肴藏、良醖、掌醢等署。各置令、太官三人,肴藏、良醖各二人,掌醢一人。丞。太官八人,肴藏、掌醢各二人,良醖四人。太官又有監膳,十二人。良醖有掌醖,五十人。掌醢有掌醢十人。等員。



 衛尉寺統公車、武庫、守宮等署。各置令、公車一人,武庫、守宮各二人。丞
 公車一人,武庫二人。等員。



 宗正寺不統署。



 太僕寺又有獸醫博士員。一百二十人。統驊騮、乘黃、龍廄、車府、典牧、牛羊等署。各置令、二人。乘黃、車府則各減一人。丞二人。乘黃則一人,典牧牛羊則各三人。等員。



 大理寺,不統署。又有正、監、評、各一人。司直、十人。律博士、八人。明法、二十人。獄掾。八人。



 鴻臚寺統典客、司儀、崇玄三署。各置令。二人。崇玄則惟置一人。典客署又有掌客,十人。司儀有掌儀二十人。等員。



 司農寺統太倉、典農、平準、廩市、鉤盾、華林、上林、導官等
 署。各置令。



 二人。鉤盾、上林則加至三人,華林惟置一人。太倉又有米稟督、二人。穀倉督、四人。鹽倉督,二人。京市有肆長,四十人。導官有御細倉督、二人。曲面倉督二人。等員。



 太府寺統左藏、左尚方、內尚方、右尚方、司染、右藏、黃藏、掌冶、甄官等署。各置令、二人。左、右尚方則加至二人,黃藏則惟置一人。丞四人。左尚則八人,右尚則六人,黃藏則一人。等員。



 國子寺元隸太常。祭酒,一人。屬官有主簿、錄事。各一人。統國子、太學、四門、書算學,各置博士、國子、太學、四門各五人,書、算各二人。助教、國子、太學、四門各五人,書、算各二人。學生國子一百四十人,太學、四門各三百六十人,書四十人,算八十人。等員。



 將
 作寺大匠、一人。丞、主簿、錄事。各二人。統左右校署令、各二人。丞、左校四人,右校三人。各有監作左校十二人,右校八人。等員。



 左右衛、左右武衛、左右武候,各大將軍,一人。將軍,二人。並有長史,司馬,錄事,功、倉、兵、騎等曹參軍,法曹、鎧曹行參軍,各一人。行參軍左右衛、左右武候各六人,左右武衛各八人。等員。



 左右衛,掌宮掖禁禦,督攝仗衛。又各有直閣將軍、六人。直寢、十二人。直齋、直後,各十五人。並掌宿衛侍從。奉車都尉,六人。掌馭副車。武騎常侍、十人。殿內將軍、十五人。員外將軍、三十人。殿內司馬督、二十人。員外司馬督、四十人。並以參軍府朝,出使勞問。左右衛又各統親衛。置開府。左勛衛開府,左翊一開府、二開府、三開府、
 四開府,及武衛、武候、領事、東宮領兵開府準此。



 府置開府,一人。有長史,司馬,錄事,及倉、兵等曹參軍,法曹行參軍,各一人。



 行參軍。三人。又有儀同府。武衛、武候、領軍、東宮領兵儀同皆準此。儀同已下,置員同開府,但無行參軍員。諸府皆領軍坊。每坊東宮軍坊準此。置坊主、一人。



 佐。二人。每鄉團東宮鄉團準此。置團主、一人。佐。二人。



 左右武衛府,無直閣已下員,但領外軍宿衛。



 左右武候,掌車駕出,先驅後殿,晝夜巡察,執捕奸非,烽候道路,水草所置。



 巡狩師田,則掌其營禁。右加置司辰師、四人。漏刻生。一百一十人。



 左右領左右府,各大將軍、一人。將軍,二人。掌侍衛左右,供御
 兵仗。領千牛備身,十二人。掌執千牛刀;備身左右,十二人。掌供御弓箭;備身,六十人。



 掌宿衛侍從。各置長史,司馬、錄事,及倉、兵二曹參軍事,鎧曹行參軍各一人。



 等員。



 左右監門府各將軍,一人。掌宮殿門禁及守衛事。各置郎將,二人。校尉,直長,各三十人。長史,司馬,錄事,及倉、兵曹參軍,鎧曹行參軍,各一人。行參軍四人。等員。



 左右領軍府,各掌十二軍籍帳、差科、辭訟之事。不置將軍。唯有長史,司馬,掾屬及錄事,功、倉、戶、騎、兵等曹參軍,法、鎧等曹行參軍,各一人。行參軍十六人。等員。又置明法,四人。隸於法司,掌律令輕重。



 行臺省,則有尚書令,僕射,左、右任置。兵部、兼吏部、禮部。度支兼都官、工部。尚書及丞左、右任置。各一人,都事四人。有考功、兼吏部、爵部、司勛。



 禮部、兼祠部、主客。膳部、兵部、兼職方。駕部、庫部、刑部、兼都官、司門。



 度支、兼倉部。戶部、兼比部。金部、工部、屯田兼水部、虞部。侍郎,各一人。



 每行臺置食貨,農圃,武器,百工監、副監,各一人。各置丞、食貨四人,農圃六人,武器二人,百工四人。錄事食貨、農圃、百工各二人,武器一人。等員。



 太子置太師、太傅、太保、少師、少傅、少保。開皇初,置詹事。二年定令,罷之。



 門下坊,置左庶子二人,內舍人四人,錄事二人,主事令史四人。統司經、宮門、內直、典膳、藥藏、齋帥等六局。司經
 置洗馬四人,校書六人,正字二人。宮門置大夫二人。內直置監、副監各二人,監殿舍人四人。典膳、藥藏,並置監、丞各二人。藥藏又有侍醫四人。齋帥置四人。



 典書坊,右庶子二人,舍人、通事舍人各八人,錄事二人,主事令史四人,內坊典內及丞各二人,丞直四人,錄事一人。內廄置尉二人,掌內車輿之事。



 家令、掌刑法、食膳、倉庫、什物、奴婢等事。率更令、掌伎樂漏刻。僕、掌宗族親疏,車輿騎乘。各一人。三寺各置丞、家令二人,寺各一人。錄事。家令二人,寺各一人。家令領食官、典倉、司藏三署令、各一人。丞。食官二人,典倉一人,司藏三人。僕寺領廄牧令一人。員。



 左右衛,各置率一人,副率二人,掌宮中禁衛。各置長史,司馬及錄事,功、倉、兵、騎兵等曹參軍事,法曹、鎧曹行參軍,各一人,行參軍四人。員。又各有直閣四人,直齋八人,直齊、直後各十人。



 左右宗衛,制官如左右衛,各掌以宗人侍衛。加置行參軍二人,而無直閣、直寢、直齋、直後等員。



 左右虞候,各置開府一人,掌斥候伺非。長史已下如左右衛,而無錄事參軍員,減行參軍一人。



 左右內率、副率,各一人,掌領備身已上禁內侍衛,供奉兵仗。又無功、騎兵、法等曹及行參軍員,餘與虞候同。有
 千牛備身八人,掌執千牛刀;備身左右八人,掌供奉弓箭;備身二十人,掌宿衛侍從。



 左右監門,各率一人,副率二人,掌諸門禁。長史已下,同內率府,而各有直長十人。



 高祖又採後周之制,置上柱國、柱國、上大將軍、大將軍、上開府儀同三司、開府儀同三司、上儀同三司、儀同三司、大都督、帥都督、都督,總十一等,以酬勤勞。又有特進、左右光祿大夫、金紫光祿大夫、銀青光祿大夫、朝議大夫、朝散大夫,並為散官,以加文武官之德聲者,並不理事。六品已下,又有翊軍等四十三號將軍,品凡十六等,
 為散號將軍,以加泛授。居曹有職務者為執事官,無職務者為散官。戎上柱國已下為散實官,軍為散號官。諸省及左右衛、武候、領左右監門府為內官,自餘為外官。



 國王、郡王、國公、郡公、縣公、侯、伯、子、男,凡九等。皇伯叔昆弟、皇子為親王。置師、友各二人,文學二人,嗣王則無師友。長史、司馬、諮議參軍事,掾屬,各一人,主簿二人,錄事,功曹,記室,戶、倉、兵等曹,騎兵、城局等參軍事,東西閤祭酒,各一人,參軍事四人,法、田、水、鎧、士等曹行參軍各一人,行參軍六人,長兼行參軍八人,典簽二人。



 上柱國、嗣王、郡王,無主簿、錄事參軍、東西閤祭酒、長兼
 行參軍等員,而加參軍事為五人,行參軍為十二人。柱國又無騎兵參軍事、水曹行參軍等員,而減參軍事、行參軍各一人。上大將軍又無諮議參軍事,田曹、鎧曹行參軍員,又減行參軍一人。大將軍又無掾屬員,又減參軍事二人。上開府又無法曹、士曹行參軍,參軍事員。開府又無典簽員,減行參軍二人。上儀同又無功曹、城局參軍事員,又減行參軍二人。儀同又無倉曹員,減行參軍三人。



 三師、三公,置府佐,與柱國同。若上柱國任三師、三公,唯從上柱國置。王公已下,三品已上,又並有親信、帳內,各
 隨品高卑而制員。



 諸王置國官。有令、大農各一人,尉各二人,典衛各八人,常侍各二人,侍郎各四人,廟長、學官長各一人,食官,廄牧長、丞各一人,典府長、丞各一人,舍人各四人等員。上柱國、柱國公,減典衛二人,無侍郎員。侯、伯又減典衛二人,食官、廄牧長各一人。子、男又減尉、典衛、常侍、舍人各一人。上大將軍、大將軍公,同柱國、子、男。其侯、伯減公典衛、侍郎、廄牧丞各一人。子、男無令,無典衛,又減舍人一人。上開府、開府公,同大將軍、子、男。其侯、伯又無常侍,無食官、廄牧丞。子、男又無侍郎、廄牧長。上儀同、儀同公,同
 開府子、男。其侯、伯又無尉,無學官長。子、男又無廄長、食官長。二王後,置國官,與諸王同。



 郡王與上柱國公同。國公無上開府已上官者,與開府公同。散郡公與儀同侯、伯同。



 散縣公與儀同子、男同。大長公主、長公主、公主,並置家令、丞各一人,主簿謁者、舍人各二人等員。郡主唯減主簿員。



 雍州,置牧。屬官有別駕,贊務,州都,郡正,主簿,錄事,西曹書佐,金、戶、兵、法、士等曹從事,部郡從事,武猛從事等員。並佐史,合五百二十四人。



 京兆郡,置尹,丞,正,功曹,主簿,金、戶、兵、法、士等曹佐等員。
 並佐史,合二百四十四人。



 大興、長安縣,置令,丞,正,功曹,主簿,西曹,金、戶、兵、法、士曹等員。並佐史,合一百四十七人。



 上上州,置刺史,長史,司馬,錄事參軍事,功曹,戶、兵等曹參軍事,法、士曹等行參軍,行參軍,典簽,州都光初主簿,郡正,主簿,西曹書佐,祭酒從事,部郡從事,倉督,市令、丞等員。並佐史,合三百二十三人。上中州,減上州吏屬十二人。上下州,減上中州十六人。中上州,減上下州二十九人。
 中中州,減中上州二十人。中下州,減中中州二十人。下上州,減中下州三十二人。下中州,減下上州十五人。下下州,減下中州十二人。



 郡,置太守,丞,尉,正,光初功曹,光初主簿,縣正,功曹,主簿,西曹,金、戶、兵、法、士等曹,市令等員。並佐史,合一百四十六人。上中郡,減上上郡吏屬五人。上下郡,減上中郡四人。
 中上郡,減上下郡十九人。中中郡,減中上郡六人。中下郡,減中中郡五人。下上郡,減中下郡十九人。下中郡,減下上郡五人。下下郡,減下中郡六人。



 縣,置令,丞,尉,正,光初功曹,光初主簿,功曹,主簿,西曹,金、戶、兵、法、士等曹佐,及市令等員。合九十九人。上中縣,減上上縣吏屬四人。上下縣,減上中縣五人。
 中上縣,減上下縣十人。中中縣,減中上縣五人。中下縣,減中中縣五人。下上縣,減中下縣十二人。下中縣,減下上縣六人。下下縣,減下中縣五人。



 州,置總管者,列為上中下三等。總管刺史加使持節。



 鎮,置將、副。戍,置主、副。關,置令、丞。其制,官屬各立三等之差。



 同州,總監、副監各一人,置二丞。統食貨農圃二監、副監。
 岐州亦置監、副監。諸冶亦置三等監。各有丞員。



 鹽池,置總監、副監、丞等員。管東西南北面等四監,亦各置副監及丞。隴右牧,置總監、副監、丞,以統諸牧。其驊騮牧及二十四軍馬牧,每牧置儀同及尉、大都督、帥都督等員。驢騾牧,置帥都督及尉。原州羊牧,置大都督並尉。原州駝牛牧,置尉。又有皮毛監、副監及丞、錄事。又鹽州牧監,置監及副監,置丞,統諸羊牧,牧置尉。苑川十二馬牧,每牧置大都督及尉各一人,帥都督二人。沙苑羊牧,置尉二人。緣邊交市監及諸屯監,每監置監、副監各一人。畿內者隸司農,自外隸諸州焉。



 五嶽各置令,又有吳山令,以供其灑掃。



 三師、王、三公,為正一品。



 上柱國、郡王、國公、開國郡縣公,為從一品。



 柱國、太子三師、特進、尚書令、左右光祿大夫、開國侯,為正二品。



 上大將軍、尚書左右僕射、雍州牧、金紫光祿大夫,為從二品。



 大將軍,吏部尚書,太常、光祿、衛尉等三卿,太子三少,納言,內史令,左右衛、左右武衛、左右武候、領左右等大將軍,禮部、兵部、都官、度支、工部尚書,宗正、太僕、大理、鴻臚、
 司農、太府等六卿,上州刺史,京兆尹,秘書監,銀青光祿大夫,開國伯,為正三品。



 上開府儀同三司,散騎常侍,左右衛、武衛、武候、領左右、監門等將軍,國子祭酒,御史大夫,將作大匠,中州刺史,親王師,朝議大夫,為從三品。



 驃騎將軍,開府儀同三司,太常、光祿、衛尉等三少卿,太子左右衛、宗衛、內等率,尚書吏部侍郎,給事黃門侍郎,太子左庶子,宗正、太僕、大理、鴻臚、司農、太府等少卿,下州刺史,已前上階。內史侍郎,太子右庶子,通直散騎常侍,左右監門郎將,朝散大夫,開國子,為正四品。



 上儀同三司,尚書左丞,太子左右衛、宗衛、內等副率,左右監門率,上郡太守,雍州別駕,親王府長史,太子家令,率更令、僕,內侍,城門校尉,已前上階。



 尚書右丞,上鎮將軍,雍州贊務,直闔將軍,親王府司馬,諫議大夫,為從四品。



 車騎將軍,儀同三司,內常侍,秘書丞,國子博士,散騎侍郎,太子內舍人,太子左右監門副率,員外散騎常侍,上州長史,親王府諮議參軍事,開國男,已前上階。尚食、尚藥典御,上州司馬,為正五品。



 著作郎,通直散騎侍郎,中郡太守,直寢,太子洗馬,中州長史,奉車都尉,已前上階。都水使者,治書侍御史,大興、長安
 令,大理司直,直齋,太子直閤,京兆郡丞,中州司馬,中鎮將,上鎮副,內給事,駙馬都尉,親王友,員外散騎侍郎,為從五品。



 翊軍、翊師將軍,尚書諸曹侍郎,內史舍人,下郡太守,大都督,親王府掾屬,下州長史,已前上階。四征將軍,征東、征南、征西、征北。三將軍,內軍、鎮軍、撫軍。大理正、監、評、千牛備身左右,左右監門校尉,內尚食典御,符璽監,御府監,殿內監,太子內直監,下州司馬,下鎮將,中鎮副,為正六品。



 四平將軍,平東、平南、平西、平北。四將軍,前軍、後軍、左軍、右軍。通事舍人,親王文學,帥都督,左右領軍府長史,太子直寢,親王府主簿,
 親王府錄事參軍事,太子門大夫,給事,上縣令,已前上階。冠軍、輔國二將軍,太子舍人,直後,三寺丞,親王府功曹、記室、倉戶曹參軍事,城門直長,太子直齋,太子副直監,太子典內,左右領軍府司馬,下鎮副,為從六品。



 鎮遠、安遠二將軍,員外散騎侍郎,御醫,左右衛、武衛、武候、領左右等府長史,親衛,親王府諸曹參軍事,已前上階。建威、寧朔二將軍,六寺丞,秘書郎,著作佐郎,太子千牛備身,太子備身左右,尚食、尚藥、左右監門等直長,太子通事舍人,左右衛、武衛、武候、領左右等府司馬,都督,太子典膳、藥藏等監,太子齋帥,上戍主,為正七品。



 寧遠、振威二將軍,左右監門府長史,太子左右衛、宗衛等率,左右虞侯、左右內率等府長史,符璽、御府、殿內等直長,上州錄事參軍事,左右領軍府掾屬,親王府東西閣祭酒,中縣令,上郡丞,太子親衛,將作丞,勛衛,親王府參軍事,上鎮長史,已前上階。伏波、輕車二將軍,太學、太常二博士,武騎常侍,奉朝請,國子助教,親王府諸曹行參軍,太子直後,太子左右監門直長,大興、長安縣丞,太子侍醫,侍御史,太史令,上州諸曹參軍事,左右監門府、太子左右衛、左右宗衛、左右虞候、左右內率等司馬,上鎮司馬,為從七品。



 宣威、明威二將軍,協律郎,都水丞,殿內將軍,太子左右監門率府長史,別將,下縣令,中郡丞,中州錄事參軍事,上上州諸曹行參軍事,親王府行參軍,左右領軍府錄事參軍事,中鎮長史,太子內坊丞,太子勛衛,已前上階。襄威、厲威二將軍,殿內御史,掖庭、宮闈二令,上署令,公車、郊社、太廟、太祝、平準、太樂、驊騮、武庫、典客、鉤盾、左藏、太倉、左尚方、右尚方、司染、典農、京市、太官、鼓吹。太子左右監門率府司馬,中州諸曹參軍事,左右衛、武衛、武候等府錄事參軍事,左右領軍府諸曹參軍事,內尚食丞,中戍主,上戍副,為正八品。



 威戎、討寇二將軍,四門博士,主書,門下錄事,尚書都事,
 監察御史,內謁者監,上關令,中署令,太醫、右藏、黃廟、乘黃、龍廟、衣冠、守宮、華林、上林、掌冶、導官、左校、右校、牛羊、典牧。下郡丞,下州錄事參軍事,中州諸曹行參軍,備身,左右衛、武衛、武候、領左右等府諸曹參軍事,左右領軍府諸曹行參軍,太子左右衛、宗衛、率等府錄事參軍事,下鎮長史,太子翊衛,已前上階。



 蕩寇、蕩難二將軍,親王府長兼行參軍及典簽,員外將軍,統軍,太子三寺丞,中關令,奚官、內僕二令,下署令,諸陵、崇玄、太卜、車府、清商、司儀、肴藏、良醖、掌醢、甄官、廩犧。上津尉,下州諸曹參軍事,左右衛、武衛、武候等府諸曹行參軍,領左右府鎧曹行參軍,左右監門、太子左右衛、宗衛等率,左右虞候,左右內率等府諸曹
 參軍事,掌船局都尉,上鎮諸曹參軍事,上縣丞,上郡尉,為從八品。



 殄寇、殄難二將軍,太學助教,太子備身,大理寺律博士,諸校書郎,都水參軍事,內史錄事,內謁者令,內寺伯,中縣丞,下關令,中津尉,下州諸曹行參軍,上州行參軍,左右監門府鎧曹行參軍,太子左右衛、宗衛、虞候府等諸曹行參軍,太子左右內率府鎧曹行參軍,左右領軍府行參軍,中鎮諸曹參軍事,上鎮士曹行參軍,中郡尉,已前上階。掃寇、掃難二將軍,殿內司馬督,太子食官、典倉、司藏等令,尚食、尚醫、軍主、太史、掖庭、宮闈局等丞,上署丞,太
 子左右監門率府諸曹參軍事,中州行參軍,左右衛、武衛、武候等府行參軍,上州典簽,下戍主,上關丞,太子典膳、藥藏等局丞,下郡尉,典客署掌客,司辰師,為正九品。



 曠野、橫野二將軍,掖庭局宮教博士,太祝,太子廄牧令,太子校書,下縣丞,中署丞,左右監門率府鎧曹行參軍,下州行參軍,中州典簽,左右監門府、太子左右衛、宗衛、虞候、率府等行參軍,正字,太子內坊丞直,中關、上津丞,下鎮諸曹參軍事,中鎮士曹行參軍,上縣尉,已前上階。偏、裨二將軍,四門助教,書算學博士,奉禮郎,員外司馬督,幢主、奚官、內僕等局丞,下署丞,下州典簽,內謁者局丞,中
 津丞,中縣尉,太子正字,太史監候,太官監膳,御府局監事,左右校及掖庭監作,太史司歷,諸樂師,為從九品。



 又有流內視品十四等:行臺尚書令,為視正二品。



 上總管、行臺尚書僕射,為視從二品。



 中總管、行臺諸曹尚書,為視正三品。



 下總管,為視從三品。



 行臺尚書左右丞,為視從四品。



 同州總監、隴右牧總監,為視從五品。



 行臺諸曹侍郎,為視正六品。



 上柱國、嗣王、郡王、柱國府長史、司馬、諮議參軍事,鹽池總監,同州、隴右牧總副監,王、二王後國令,為視從六品。



 上大將軍、大將軍府長史、司馬,上柱國、嗣王、郡王、柱國府掾屬,嗣王文學,公國令,王、二王後大農尉、典衛,為視正七品。



 上開府、開府府長史、司馬,上大將軍、大將軍府掾屬,上柱國、嗣王、郡王、柱國府諸曹參軍事,鹽池總副監,鹽州牧監,諸屯監,國子學生,侯、伯國令,公國大農尉、典衛、雍州薩保,為視從七品。



 上儀同、儀同府長史、司馬,上大將軍、大將軍府諸曹參
 軍事,上柱國、嗣王、郡王、柱國府參軍事,諸曹行參軍,行臺諸監,同州諸監,鹽池四面監,皮毛監,岐州監,同州總監、隴右牧監等丞,諸大冶監,雍州州都主簿,子、男國令,侯、伯國大農尉、典衛,王、二王後國常侍,為視正八品。



 行臺尚書都事,上開府、開府府諸曹參軍事,上大將軍、大將軍府參軍事、諸曹行參軍,上柱國、嗣王、郡王、柱國府行參軍,五岳、四瀆、吳山等令,鹽池四面副監,諸皮毛副監,行臺諸副監,諸屯副監,諸中冶監,諸緣邊交市監,鹽池總監丞,諸州州都主簿,雍州西曹書佐、諸曹從事,京兆郡正功曹,太學生,子、男國大農、典衛,為視從八品。



 開府府法曹行參軍,上儀同、儀同府諸曹參軍事,上大將軍、大將軍府行參軍,上柱國、嗣王、郡王、柱國府典簽,同州諸副監,岐州副監,諸小冶監,鹽州牧監丞,諸大冶監丞,諸緣邊交市副監,諸郡正、功曹,京兆郡主簿,諸州西曹書佐、祭酒從事,雍州部郡從事,公國常侍,王、二王後國侍郎,公主家令,諸州胡二百戶已上薩保,為視正九品。



 儀同府法曹行參軍,上開府、開府府行參軍,上大將軍、大將軍府典簽,上儀同、儀同府行參軍,上開府府典簽,行臺諸監丞,鹽池四面監丞,皮毛監丞,諸中冶監丞,四
 門學生,諸郡主簿,諸州部郡從事,雍州武猛從事,大興、長安縣正、功曹、主簿,侯、伯、子、男國常侍,公國侍郎,為視從九品。



 又有流外勛品、二品、三品、四品、五品、六品、七品、八品、九品之差。又視流外,亦有視勛品、視二品、視三品、視四品、視五品、視六品、視七品、視八品、視九品之差。極於胥吏矣,皆無上下階雲。



 京官正一品,祿九百石,其下每以百石為差,至正四品,是為三百石。從四品,二百五十石,其下每以五十石為差,至正六品,是為百石。從六品,九十石,以下每以十石
 為差,至從八品,是為五十石。食封及官不判事者,並九品,皆不給祿。



 其給皆以春秋二季。刺史、太守、縣令,則計戶而給祿,各以戶數為九等之差。大州六百二十石,其下每以四十石為差,至於下下,則三百石。大郡三百四十石,其下每以三十石為差,至於下下,則百石。大縣百四十石,其下每以十石為差,至於下下,則六十石。其祿唯及刺史二佐及郡守、縣令。



 三年四月,詔尚書左僕射,掌判吏部、禮部、兵部三尚書事,御史糾不當者,兼糾彈之。尚書右僕射,掌判都官、度支、工部三尚書事,又知用度。餘皆依舊。



 尋改度支尚書
 為戶部尚書,都官尚書為刑部尚書。諸曹侍郎及內史舍人,並加為從五品。增置通事舍人十二員,通舊為二十四員。廢光祿寺及都水臺入司農,廢衛尉入太常尚書省,廢鴻臚亦入太常。罷大理寺監、評及律博士員,加置正為四人。罷郡,以州統縣,改別駕、贊務,以為長史、司馬。舊周、齊州郡縣職,自州都、郡縣正已下,皆州郡將縣令至而調用,理時事。至是不知時事,直謂之鄉官。別置品官,皆吏部除授,每歲考殿最。刺史、縣令,三年一遷,佐官四年一遷。佐官以曹為名者,並改為司。六年,尚書省二十四司,各置員外郎一人,以司其曹之籍帳。



 侍郎闕,
 則厘其曹事。吏部又別置朝議、通議、朝請、朝散、給事、承奉、儒林、文林等八郎,武騎、屯騎、驍騎、游騎、飛騎、旅騎、雲騎、羽騎八尉。其品則正六品以下,從九品以上。上階為郎,下階為尉。散官番直,常出使監檢。罷門下省員外散騎常侍、奉朝請、通事令史員,及左右衛、殿內將軍,司馬督,武騎常侍等員。



 十二年,復置光祿、衛尉、鴻臚等寺。諸州司以從事為名者,改為參軍。



 十三年,復置都水臺。國子寺罷隸太常,又改寺為學。



 十四年,諸省各置主事令史員。改九等州縣為上、中、中
 下、下,凡四等。



 十五年,罷州縣、鄉官。



 十六年,內侍省加置內主事員二十人,以承門閣。



 十八年,置備身府。



 二十年,改將作寺為監,以大匠為大監。初加置副監。



 仁壽元年,改都水臺為監,更名使者為監。罷國子學,唯立太學一所,置博士五人,從五品,學生七十二人。



 三年,監門府又置門候一百二十人。



 煬帝即位,多所改革。三年定令,品自第一至於第九,唯置正從,而除上下階。



 罷諸總管,廢三師、特進官。分門下、
 太僕二司,取殿內監名,以為殿內省,並尚書、門下、內史、秘書,以為五省。增置謁者、司隸二臺,並御史為三臺。分太府寺為少府監。改內侍省為長秋監,國子學為國子監,將作寺為將作監,並都水監,總為五監,改左右衛為左右翊衛,左右備身為左右騎衛。左右武衛依舊名。改領軍為左右屯衛,加置左右御。改左右武候為左右候衛。是為十二衛。又改領左右府為左右備身府,左右監門依舊名,凡十六府。其朝之班序,以品之高卑為列。品同則以省府為前後,省府同則以局署為前後焉。



 尚書省六曹,各侍郎一人,以貳尚書之職。又增左、右丞
 階,與六侍郎並正四品。諸曹侍郎並改為郎。又改吏部為選部郎,戶部為人部郎,禮部為儀曹郎,兵部為兵曹郎,刑部為憲部郎,工部為起部郎,以異六侍郎之名,廢諸司員外郎,而每增置一曹郎,各為二員。都司郎各一人,品同曹郎,掌都事之職,以都事為正八品,分隸六尚書。諸司主事,並去令史之名。其令史隨曹閑劇而置。每十令史置一主事,不滿十者亦置一人。其餘四省三臺,亦皆曰令史,九寺五監諸衛府,則皆曰府史。



 後又改主客郎為司蕃郎。尋又每減一郎,置承務郎一人,同員外之職。



 舊都督已上,至上柱國,凡十一等,及八郎、八尉、四十三號將軍官,皆罷之。



 並省朝議大夫。自一品至九品,置光祿、從一品。左右光祿、左正二品,右從二品。



 金紫、正三品。銀青光祿、從三品。正議、正四品。通議、從四品。朝請、正五品。



 朝散從五品。等九大夫,建節、正六品。奮武、從六品。宣惠、正七品。綏德、從七品。懷仁、正八品。守義、從八品。奉誠、正九品。立信從九品。等八尉,以為散職。開皇中,以開府儀同三司為四品散實官,至是改為從一品,同漢、魏之制,位次王公。門下省減給事黃門侍郎員,置二人,去給事之名,移吏部給事郎名為門下之職,位次黃門下。置員四人,從五品,省讀奉案。廢散騎常侍、通直散騎
 常侍、諫議大夫、散騎侍郎等常員。改符璽監為郎,置員二人,為從六品。加錄事階為正八品。以城門、殿內、尚食、尚藥、御府等五局隸殿內省。十二年,又改納言為侍內。



 內史省減侍郎員為二人,減內史舍人員為四人。加置起居舍人員二人,從六品。



 次舍人下。改通事舍人員為謁者臺職。減主書員,置四人,加為正八品。十二年,改內史為內書。



 殿內省置監、正四品。少監、從四品。丞,從五品。各一人,掌諸供奉。又有奉車都尉十二人,掌進御輿馬。統尚食、尚藥、尚衣、尚舍、尚乘、尚輦等六局,各置奉御二人,正五品。皆置直長,
 以貳之。正七品。尚食直長六人,又有食醫員。



 尚藥直長四人,又有侍御醫、司醫、醫佐員。尚衣即舊御府也,改名之,有直長四人。尚舍即舊殿中局也,改名之,有直長八人。尚乘局置左右六閑:一左右飛黃閑,二左右吉良閑,三左右龍媒閑,四左右騊駼閑,五左右駃騠閑,六左右天苑閑。有直長十四人,又有奉乘十人。尚輦有直長四人,又有掌輦六人。城門置校尉一人,降為正五品。後又改校尉為城門郎,置員四人,從六品。自殿內省隸為門下省官。



 秘書省降監為從三品,增置少監一人。從四品。增著作郎
 階為正五品,減校書郎為十人。改太史局為監,進令階為從五品,又減丞為一人。置司辰師八人,增置監候為十人。其後又改監、少監為令、少令。增秘書郎為從五品,加置佐郎四人,從六品。以貳郎之職。降著作郎階為從五品。又置儒林郎十人,正七品。掌明經待問,唯詔所使。文林郎二十人,從八品。掌撰錄文史,檢討舊事。此二郎皆上在籓已來直司學士。增校書郎員四十人,加置楷書郎員二十人,從九品。掌抄寫御書。



 御史臺增治書侍御史為正五品。省殿內御史員,增監察御史員十六人,加階為從七品。開皇中,御史直宿禁
 中,至是罷其制。又置主簿、錄事員各二人。五年,又降大夫階為正四品,減治書侍御史為從五品;增侍御史為正七品,唯掌侍從糾察,其臺中簿領,皆治書侍御史主之。後又增置御史,從九品,尋又省。



 謁者臺大夫一人,從四品。五年,改為正四品。掌受詔勞問,出使慰撫,持節察授,及受冤枉而申奏之。駕出,對御史引駕。置司朝謁者二人以貳之。從五品。



 屬官有丞一人,主簿、錄事各一人等員。又有通事謁者二十人,從六品。即內史通事舍人之職也。次有議郎二十四人,通直三十六人,將事謁者三十人,謁者七十人,皆掌出使。其後廢議郎,通直、將
 事謁者,謁者等員,而置員外郎八十員。尋詔門下、內史、御史、司隸、謁者五司,監受表,以為恆式,不復專謁者矣。尋又置散騎郎,從五品,二十人,承議郎、正六品。通直郎,從六品。各三十人,宣德郎、正七品。宣義郎,從七品,各四十人,從事郎、正八品。將仕郎、從八品。常從郎、正九品。奉信郎,從九品。各五十人,是為正員。並得祿當品。又各有散員郎,無員無祿。尋改常從為登仕,奉信為散從。自散騎已下,皆主出使,量事大小,據品以發之。



 司隸臺大夫一人,正四品。掌諸巡察。別駕二人,從五品。分察畿內,一人案東都,一人案京師。刺史十四人,正六品。巡察
 畿外。諸郡從事四十人,副刺史巡察。其所掌六條:一察品官以上理政能不。二察官人貪殘害政。三察豪強奸猾,侵害下人,及田宅逾制,官司不能禁止者。四察水旱蟲災,不以實言,枉征賦役,及無災妄蠲免者。五察部內賊盜,不能窮逐,隱而不申者。六察德行孝悌,茂才異行,隱不貢者。每年二月,乘軺巡郡縣,十月入奏。置丞、從六品。主簿、從八品。錄事從九品。各一人,後又罷司隸臺,而留司隸從事之名,不為常員。臨時選京官清明者,權攝以行。



 光祿已下八寺卿,皆降為從三品。少卿各加置二人,為從四品。諸寺上署令,並增為正六品,中署令為從六品,
 下置令為正七品。始開皇中,署司唯典掌受納,至是署令為判首,取二卿判。丞唯知勾檢。令闕,丞判。五年,寺丞並增為從五品。



 太常寺罷太祝署,而留太祝員八人,屬寺。後又增為十人。奉禮減置六人。太廟署又置陰室丞,守視陰室。改樂師為樂正,置十人。太卜又省博士員,置太卜卜正二十人,以掌其事。太醫又置醫監五人,正十人。罷衣冠、清商二署。



 太僕減驊騮署入殿內尚乘局,改龍廄曰典廄署,有左、右駁皁二廄。加置主乘、司庫、司廩官。罷牛羊署。



 大理寺丞改為勾檢官,增正員為六人,分判獄事。置司直十六人,降為從六品,後加至二十人。又置評事四十八人,掌頗同司直,正九品。



 鴻臚寺改典客署為典蕃署。初煬帝置四方館於建國門外,以待四方使者,後罷之,有事則置,名隸鴻臚寺,量事繁簡,臨時損益。東方曰東夷使者,南方曰南蠻使者,西方曰西戎使者,北方曰北狄使者,各一人,掌其方國及互市事。每使者署,典護錄事、敘職、敘儀、監府、監置、互市監及副、參軍各一人。錄事主綱紀。敘職掌其貴賤立功合敘者。敘儀掌小大次序。監府掌其貢獻財貨。監置
 掌安置其駝馬船車,並糾察非違。互市監及副掌互市。參軍事出入交易。



 司農但統上林、太倉、鉤盾、導官四署,罷典農、華林二署,而以平準、京市隸太府。



 太府寺既分為少府監,而但管京都市五署及平準、左右藏等,凡八署。京師東市曰都會,西市曰利人。東都東市曰豐都,南市曰大同,北市曰通遠。及改諸令為監,唯市署曰令。



 國子監依舊置祭酒,加置司業一人,從四品,丞三人,加為從六品。並置主簿、錄事各一人。國子學置博士,正五
 品,助教,從七品,員各一人。學生無常員,太學博士、助教各二人,學生五百人。先是仁壽元年,省國子祭酒、博士,置太學博士員五人,為從五品,總知學事。至是太學博士降為從六品。



 將作監改大監、少監為大匠、少匠,丞加為從六品。統左右校及甄官署。五年,又改大匠為大監,正四品,少匠為少監,正五品。十三年,又改監、少監為令、少令。丞加品至從五品。



 少府監置監,從三品,少監,從四品,各一人。丞從五品,二人。統左尚、右尚、內尚、司織、司染、鎧甲、弓弩、掌冶等署。復
 改監、少監為令、少令。並司織、司染為織染署,廢鎧甲、弓弩二署。



 都水監改為使者,增為正五品,丞為從七品。統舟楫、河渠二署。舟楫署每津置尉一人。五年,又改使者為監,四品,加置少監,為五品。後又改監、少監為令,從三品,少令,從四品。



 長秋監置令一人,正四品,少令一人,從五品,丞二人,正七品。並用士人。



 改內常侍為內承奉,置二人,正五品;給事為內承直,置四人,從五品。並用宦者。



 罷內謁者官,領掖庭、宮闈、奚官等三署,並參用士人。後又置內謁者員。



 十二衛,各置大將軍一人,將軍二人,總府事,並統諸鷹揚府。改驃騎為鷹揚郎將,正五品;車騎為鷹揚副郎將,從五品;大都督為校尉;帥都督為旅帥;都督為隊正,增置隊副以貳之。改三衛為三侍。其直閤將軍、直寢、奉車都尉、駙馬都尉、直齋、別將、統軍、軍主、幢主之屬,並廢。以武候府司辰師員,隸為太史局官。其軍士,左右衛所領名為驍騎,左右驍衛所領名豹騎,左右武衛所領名熊渠,左右屯衛所領名羽林,左右御所領名射聲,左右候衛所領名佽飛,而總號衛士,每衛置護軍四人,掌副貳將軍。將軍無則一人攝。尋改護軍為武賁郎將,正四
 品,而置武牙郎將六人,副焉,從四品。諸衛皆置長史,從五品。又有錄事參軍,司倉、兵、騎、鎧等員。翊衛又加有親侍。鷹揚府每府置鷹揚郎將一人,正五品,副鷹揚郎將一人,從五品,各有司馬及兵、倉兩司。其府領親、勛、武三侍,非翊衛府,皆無三侍。鷹揚每府置越騎校尉二人,掌騎士,步兵校尉二人,領步兵,並正六品。



 外軍鷹揚官並同。左右候衛增置察非掾二人,專糾彈之事。五年,又改副郎將並為鷹擊郎將。



 左右領左右府,改為左右備身府,各置備身郎將一人。又各置直齋二人以貳之,並正四品,掌侍衛左右。統千
 牛左右、司射左右各十六人,並正六品。千牛掌執千牛刀宿衛,司射掌供御弓箭。置長史,正六品,錄事,司兵、倉、騎,參軍等員,並正八品。有折沖郎將,各三人,正四品,掌領驍果。又各置果毅郎將三人以貳之,從四品。其驍果,置左、右雄武府雄武郎將以領之。以武勇郎將為副員,同鷹揚、鷹擊。有司兵、司騎二局,並置參軍事。



 左右監門府,改將軍為郎將,各置一人,正四品,直閣各六人,正五品。置官屬,並同備身府。又增左右門尉員一百二十人,正六品;置門候員二百四十人,正七品。並分掌門禁守衛。



 門下坊減內舍人、洗馬員,各置二人,減侍醫,置二人。改
 門大夫為宮門監,正字為正書。



 典書坊改太子舍人為管記舍人,減置四人,改通事舍人為宣令舍人,為八員。



 家令改為司府令,內坊承直改為典直。



 左右衛率改為左右侍率,正四品。改親衛為功曹,勛衛為義曹,翊衛為良曹。



 罷直齋、直閣員。



 左右宗衛率改為左右武侍率,正四品。



 左右虞候開府改為左右虞候率,正四品,並置副率。



 左右內率降為正五品。千牛備身改為司仗左右,備身左右改為主射左右。各員八人。



 左右監門率改為宮門將,降為正五品。監門直長改為直事,置六十人。



 開皇中,置國王,郡王,國公,郡公,縣公、侯、伯、子、男為九等者,至是唯留王、公、侯三等。餘並廢之。



 王府諸司參軍,更名諸司書佐,屬參軍則直以屬為名。改國令為家令。自餘以國為名者,皆去之。



 行宮所在,皆立總監以司之。上宮正五品,中宮從五品,下宮正七品。隴右諸牧,置左、右牧監各一人,以司統之。



 罷州置郡,郡置太守。上郡從三品,中郡正四品,下郡從四品。京兆、河南則俱為尹,並正三品。罷長史、司馬,置贊
 務一人以貳之。京兆、河南從四品,上郡正五品,中郡從五品,下郡正六品。次置東西曹掾,京兆、河南從五品,上郡正六品,中郡從六品,下郡正七品。主簿,司功、倉、戶、兵、法、士曹等書佐,各因郡之大小而為增減。改行參軍為行書佐。舊有兵處,則刺史帶諸軍事以統之,至是別置都尉,副都尉。都尉正四品,領兵,與郡不相知。副都尉正五品。又置京輔都尉,從三品,立府於潼關,主兵領遏。並置副都尉,從四品。又置諸防主、副官,掌同諸鎮。大興、長安、河南、洛陽四縣令,並增為正五品。諸縣皆以所管閑劇及沖要以為等級。丞、主簿如故。其後諸郡各加置通守一人,位次太守,京兆、河南,則謂之內史。又改郡贊務為
 丞,位在通守下,縣尉為縣正,尋改正為戶曹、法曹,分司以承郡之六司。河南、洛陽、長安、大興,則加置功曹,而為三司,司各二人。



 郡縣佛寺,改為道場,道觀改為玄壇,各置監、丞。京都諸坊改為里,皆省除裏司,官以主其事。



 帝自三年定令之後,驟有制置,制置未久,隨復改易。其餘不可備知者,蓋史之闕文云。



\end{pinyinscope}