\article{卷六十九列傳第三十四}

\begin{pinyinscope}

 王劭王劭,字君懋,太原晉陽人也。父松年,齊通直散騎侍郎。劭少沈默,好讀書。



 弱冠,齊尚書僕射魏收闢參開府軍事,累遷太子舍人,待詔文林館。時祖孝徵、魏收、陽休之等嘗論古事,有所遺忘,討閱不能得,因呼劭問之。劭具論所出,取書驗之,一無舛誤。自是大為時人所許,稱其博物。後遷中書舍人。齊滅,入周,不得調。高祖受禪,授著
 作佐郎。以母憂去職,在家著《齊書》。時制禁私撰史,為內史侍郎李元操所奏。上怒,遣使收其書,覽而悅之。於是起為員外散騎侍郎,修起居注。劭以古有鉆燧改火之義,近代廢絕,於是上表請變火,曰:「臣謹案《周官》,四時變火,以救時疾。明火不數變,時疾必興。聖人作法,豈徒然也!在晉時,有以洛陽火渡江者,代代事之,相續不滅,火色變青。昔師曠食飯,云是勞薪所爨。晉平公使視之,果然車輞。今溫酒及炙肉,用石炭、柴火、竹火、草火、麻荄火,氣味各不同。以此推之,新火舊火,理應有異。伏願遠遵先聖,於五時取五木以變火,用功甚少,救益方大。縱使
 百姓習久,未能頓同,尚食內廚及東宮諸主食廚,不可不依古法。」上從之。劭又言上有龍顏戴乾之表,指示群臣。上大悅,賜物數百段。拜著作郎。劭上表言符命曰:昔周保定二年,歲在壬午,五月五日,青州黃河變清,十里鏡澈,齊氏以為己瑞,改元曰河清。是月,至尊以大興公始作隋州刺史,歷年二十,隋果大興。臣謹案《易坤靈圖》曰:「聖人受命,瑞先見於河。河者最濁,未能清也。」竊以靈貺休祥,理無虛發,河清啟聖,實屬大隋。午為鶉火,以明火德,仲夏火王,亦明火德。月五日五,合天數地數,既得受命之辰,允當先見之兆。開皇初,邵州人楊令悊近河,
 得青石圖一,紫石圖一,皆隱起成文,有至尊名,下云:「八方天心。」



 永州又得石圖,剖為兩段,有楊樹之形,黃根紫葉。汝水得神龜,腹下有文曰:「天卜楊興。」安邑掘地,得古鐵版,文曰:「皇始天年,賚楊鐵券,王興。」同州得石龜,文曰:「天子延千年,大吉。」臣以前之三石,不異龍圖。何以用石?



 石體久固,義與上名符合。龜腹七字,何以著龜?龜亦久固,兼是神靈之物。孔子嘆河不出圖,洛不出書,今於大隋聖世,圖書屢出。建德六年,亳州大周村有龍鬥,白者勝,黑者死。大象元年夏,焚陽汴水北有龍鬥,初見白氣屬天,自東方歷陽武而來。及至,白龍也,長十許丈。有黑
 龍乘雲而至,兩相薄,乍合乍離,自午至申,白龍升天,黑龍墜地。謹案:龍,君象也。前鬥於亳州周村者,蓋象至尊以龍鬥之歲為亳州總管,遂代周有天下。後鬥於熒陽者,「熒」字三火,明火德之盛也。白龍從東方來,歷陽武者,蓋象至尊將登帝位,從東第入自崇陽門也。西北升天者,當乾位天門。《坤靈圖》曰:「聖人殺龍。」龍不可得而殺,皆盛氣也。又曰:「泰姓商名宮,黃色,長八尺,六十世,河龍以正月辰見,白龍與五黑龍鬥,白龍陵,故泰人有命。」謹案:此言皆為大隋而發也。聖人殺龍者,前後龍死是也。姓商者,皇家於五姓為商也。名宮者,武元皇帝諱於五聲
 為宮。黃色者,隋色尚黃。



 長八尺者,武元皇帝身長八尺。河龍以正月辰見者,泰正月卦,龍見之所,於京師為辰地。白龍與黑龍鬥者,亳州熒陽龍鬥是也。勝龍所以白者,楊姓納音為商,至尊又辛酉歲生,位皆在西方,西方色白也。死龍所以黑者,周色黑。所以稱五者,周閔、明、武、宣、靖凡五帝。趙、陳、代、越、滕五王,一時伏法,亦當五數。



 白龍陵者,陵猶勝也。鄭玄說:「陵當為除。」凡鬥能去敵曰除。臣以泰人有命者,泰之為言通也,大也,明其人道通德大,有天命也。《乾鑿度》曰:「泰表戴干。」



 鄭玄注云:「表者,人形體之彰識也。乾,盾也。泰人之表戴干。」臣伏見至尊有戴
 乾之表,益知泰人之表不爽毫厘。《坤靈圖》所云,字字皆驗。《緯書》又稱「漢四百年」,終如其言,則知六十世亦必然矣。昔宗周卜世三十,今則倍之。



 《稽覽圖》云:「太平時,陰陽和合,風雨咸同,海內不偏,地有阻險,故風有遲疾。雖太平之政,猶有不能均同,唯平均乃不鳴條,故欲風於亳。亳者,陳留也。」謹案:此言蓋明至尊者為陳留公世子,亳州總管,遂受天命,海內均同,不偏不黨,以成太平之風化也。在大統十六年,武元皇帝改封陳留公。是時齊國有《秘記》云:「天王陳留入並州。」齊王高洋為是誅陳留王彭樂。其後武元皇帝果將兵入並州。周武帝時,望氣者
 云亳州有天子氣,於是殺亳州刺史紇豆陵恭,至尊代為之。又陳留老子祠有枯柏,世傳云老子將度世,云待枯柏生東南枝回指,當有聖人出,吾道復行。至齊,枯柏從下生枝,東南上指。夜有三童子相與歌曰:「老子廟前古枯樹,東南狀如傘,聖主從此去。」及至尊牧亳州,親至祠樹之下。自是柏枝回抱,其枯枝,漸指西北,道教果行。校考眾事,太平主出於亳州陳留之地,皆如所言。《稽覽圖》又云:「治道得,則陰物變為陽物。」鄭玄注云:「蔥變為韭亦是。」謹案:自六年以來,遠近山石,多變為玉,石為陰,玉為陽。又左衛園中蔥皆變為韭。



 上覽之大悅,賜物五百
 段。未幾,劭復上書曰:《易乾鑿度》曰:「隨上六,拘系之,乃從維之,王用享於西山。隨者二月卦,陽德施行,籓決難解,萬物隨陽而出。故上六欲九五拘系之,維持之,明被陽化而陰隨從之也。」《易稽覽圖》:「坤六月,有子女,任政,一年,傳為復。五月貧之從東北來立,大起土邑,西北地動星墜,陽衛。屯十一月神人從中山出,趙地動。



 北方三十日,千里馬數至。」謹案:「凡此《易》緯所言,皆是大隋符命。隨者二月之卦,明大隋以二月即皇帝位也。陽德施行者,明楊氏之德教施行於天下也。籓決難解者,明當時籓鄣皆是通決,險難皆解散也。萬物隨陽而出者,明天地間
 萬物盡隨楊氏而出見也。上六欲九五拘系之者,五為王,六為宗廟,明宗廟神靈欲令登九五之位,帝王拘民以禮,系民以義也。「拘民以禮」,系民以義」,此二句亦是《乾鑿度》之言。維持之者,明能以綱維持正天下也。被陽化而欲陰隨之者,明陰類被服楊氏之風化,莫不隨從。陰謂臣下也。王用享於西山者,蓋明至尊常以歲二月幸西山仁壽宮也。凡四稱隨,三稱陽,欲美隋楊,丁寧之至也。坤六月者,坤位在未,六月建未,言至尊以六月生也。有子女任政者,言樂平公主是皇帝子女,而為周後,任理內政也。一年傳為復者,復是坤之一世卦,陽氣初生,
 言周宣帝崩後一年,傳位與楊氏也。五月貧之從東北來立者,「貧之」當為「真人」,字之誤也。



 言周宣帝以五月崩,真人革命,當在此時,至尊謙讓而逆天意,故逾年乃立。昔為定州總管,在京師東北,本而言之,故曰真人從東北來立。大起土邑者,大起即大興,言營大興城邑也。西北地動星墜者,蓋天意去周授隋,故變動也。陽衛者,言楊氏得天衛助。屯十一月神人從中山出者,此卦動而大亨作,故至尊以十一月被授亳州總管,將從中山而出也。趙之時,停留三十
 日也。千里馬者,蓋至尊舊所乘狖騮馬也。屯卦震下坎上,震於馬作足,坎於馬為美脊,是故狖騮馬脊有肉鞍,行則先作弄四足也。數至者,言歷數至也。



 《河圖帝通紀》曰:「形瑞出,變矩衡。赤應隨,葉靈皇。」《河圖皇參持》曰:「皇闢出,承元訖。道無為,治率。被遂矩,戲作術。開皇色,握神日。投輔提,象不絕。立皇后,翼不格。道終始,德優劣。帝任政,河曲出。葉輔嬉,爛可述。」謹案:凡此《河圖》所言,亦是大隋符命。形瑞出、變矩衡者,矩,法也,衡,北斗星名,所謂璇璣玉衡者也。大隋受命,形兆之瑞始出,天象則為之變動。



 北斗主天之法度,故曰矩衡。《易》緯「伏戲矩衡神」,鄭玄注
 亦以為法玉衡之神。



 與此《河圖》矩衡義同。赤應隋者,言赤帝降精,感應而生隋也。故隋以火德為赤帝天子。葉靈皇者,葉,合也,言大隋德合上靈天皇大帝也。又年號開皇,與《靈寶經》之開皇年相合,故曰葉靈皇。皇闢出者,皇,大也,闢,君也,大君出蓋謂至尊受命出為天子也。承元訖者,言承周天元終訖之運也。道無為、治率者,治下脫一字,言大道無為,治定天下率從。被遂矩、戲作術者,矩,法也。昔遂皇握機矩,伏戲作八卦之術,言大隋被服三皇之法術也。遂皇機矩,語見《易》緯。開皇色者,言開皇年易服色也。握神日者,握持群神,明照如日也。又開皇
 以來日漸長,亦其義。投輔提者,言投授政事於輔佐,使之提挈也。象不絕者,法象不廢絕也。



 立皇后、翼不格者,格,至也,言本立太子以為皇家後嗣,而其輔翼之人不能至於善也。道終始、德優劣者,言前東宮道終而德劣,今皇太子道始而德優也。帝任政、河曲出者,言皇帝親任政事,而邵州河濱得石圖也。葉輔嬉、爛可述者,葉,合也,嬉,興也,言群臣合心輔佐,以興政治,爛然可紀述也。所以於《皇參持》、《帝通紀》二篇陳大隋符命者,明皇道帝德,盡在隋也。



 上大悅,以劭為至誠,寵錫日隆。時有人於黃鳳泉浴,得二白石,頗有文理,遂附致其文以為字,復
 言有諸物象而上奏曰:「其大玉有日月星辰,八卦五岳,及二麟雙鳳,青龍硃雀,騶虞玄武,各當其方位。又有五行、十日、十二辰之名,凡二十七字,又有『天門地戶人門鬼門閉』九字。又有卻非及二鳥,其鳥皆人面,則《抱樸子》所謂『千秋萬歲』也。其小玉亦有五岳、卻非、虯犀之象。二玉俱有仙人玉女乘雲控鶴之象。別有異狀諸神,不可盡識,蓋是風伯、雨師、山精、海若之類。又有天皇大帝,皇帝及四帝坐,鉤陳、北斗、三公、天將軍、土司空、老人、天倉、南河、北河、五星、二十八宿,凡四十五宮。諸字本無行伍,然往往偶對。



 於大玉則有皇帝姓名,並臨南面,與日字
 正鼎足。復有老人星,蓋明南面象日而長壽也。皇后二字在西,上有月形,蓋明象月也。於次玉則皇帝名與九千字次比,兩楊字與萬年字次比,隋與吉字正並,蓋明長久吉慶也。」劭復回互其字,作詩二百八十篇奏之。上以為誠,賜帛千匹。劭於是採民間歌謠,引圖書讖緯,依約符命,捃摭佛經,撰為《皇隋靈感志》,合三十卷,奏之。上令宣示天下。劭集諸州朝集使,洗手焚香,閉目而讀之,曲折其聲,有如歌詠。經涉旬朔,遍而後罷。上益喜,賞賜優洽。



 仁壽中,文獻皇后崩,劭復上言曰:「佛說人應生天上,及上品上生無量壽國之時,天佛放大光明,以香花
 妓樂來迎之。如來以明星出時入涅盤。伏惟大行皇后聖德仁慈,福善禎符,備諸秘記,皆云是妙善菩薩。臣謹案:八月二十二日,仁壽宮內再雨金銀之花。二十三日,大寶殿後夜有神光。二十四日卯時,永安宮北有自然種種音樂,震滿虛空。至夜五更中,奄然如寐,便即升遐,與經文所說,事皆符驗。臣又以愚意思之,皇后遷化,不在仁壽、大興宮者,蓋避至尊常居正處也。在永安宮者,象京師之永安門,平生所出入也。後升遐後二日,苑內夜有鐘聲三百餘處,此則生天之應顯然也。」上覽而且悲且喜。



 時蜀王秀以罪廢,上顧謂劭曰:「嗟乎!吾有五子,
 三子不才。」劭進曰:「自古聖帝明王,皆不能移不肖之子。黃帝有二十五子,同姓者二,餘各異德。堯十子,舜九子,皆不肖。夏有五觀,周有三監。」上然其言。其後上夢欲上高山而不能得,崔彭捧腳,李盛扶肘得上,因謂彭曰:「死生當與爾俱。」劭曰:「此夢大吉。上高山者,明高崇大安,永如山也。彭猶彭祖,李猶李老,二人扶侍,實為長壽之徵。」上聞之,喜見容色。其年,上崩。未幾,崔彭亦卒。



 煬帝嗣位,漢王諒作亂,帝不忍加誅。劭上書曰:「臣聞黃帝滅炎,蓋雲母弟,周公誅管,信亦天倫。叔向戮叔魚,仲尼謂之遺直,石碏殺石厚,丘明以為大義。



 此皆經籍明文,帝王常
 法。今陛下置此逆賊,度越前聖,含弘寬大,未有以謝天下。



 謹案賊諒毒被生民者也。是知古者同德則同姓,異德則異姓,故黃帝有二十五子,其得姓者十有四人,唯青陽、夷鼓,與黃帝同為姬姓。諒既自絕,請改其氏。」劭以此求媚,帝依違不從。遷秘書少監,數載,卒官。



 劭在著作將二十年,專典國史,撰《隋書》八十卷。多錄口敕,又採迂怪不經之語及委巷之言,以類相從,為其題目,辭義繁雜,無足稱者,遂使隋代文武名臣列將善惡之跡,堙沒無聞。初撰《齊志》為編年體,二十卷,復為《齊書》紀傳一百卷,及《平賊記》三卷。或文詞鄙野,或不軌不物,駭人視聽,
 大為有識所嗤鄙。



 然其採擿經史謬誤,為《讀書記》三十卷,時人服其精博。爰自志學,暨乎暮齒,篤好經史,遺落世事。用思既專,性頗恍忽,每至對食,閉目凝思,盤中之肉,輒為僕從所啖。劭弗之覺,唯責肉少,數罰廚人。廚人以情白劭,劭依前閉目,伺而獲之,廚人方免笞辱。其專固如此。



 袁充袁充,字德符,本陳郡陽夏人也。其後寓居丹陽。祖昂,父君正,俱為梁侍中。



 充少敬悟,年十餘歲,其父黨至門,時冬初,充尚衣葛衫。客戲充曰:「袁郎子絺兮霡兮,淒其以
 風。」充應聲答曰:「唯絺與霡,服之無數。」以是大見嗟賞。仕陳,年十七,為秘書郎。歷太子舍人、晉安王文學、吏部侍郎、散騎常侍。及陳滅,歸國,歷蒙、鄜二州司馬。充性好道術,頗解占候,由是領太史令。時上將廢皇太子,正窮治東宮官屬,充見上雅信符應,因希旨進曰:「比觀玄象,皇太子當廢。」



 上然之。充復表奏,隋興已後,日影漸長,曰:「開皇元年,冬至日影一丈二尺七寸二分,自爾漸短。至十七年,冬至影一丈二尺六寸三分。四年冬至,在洛陽測影,一丈二尺八寸八分。二年,夏至影一尺四寸八分,自爾漸短。至十六年,夏至影一尺四寸五分。《周官》以土圭
 之法正日影,日至之影尺有五寸。鄭玄云:『冬至之影一丈三尺。』今十六年夏至之影,短於舊影五分,十七年冬至之影,短於舊影三寸七分。日去極近則影短而日長,去極遠則影長而日短,行內道則去極近,外道則去極遠。《堯典》云:『日短星昴,以正仲冬。』據昴星昏中,則知堯時仲冬,日在須女十度。以歷數推之,開皇已來冬至,日在斗十一度,與唐堯之代去極並近。



 謹案《春秋元命包》云:『日月出內道,璇璣得常,天帝崇靈,聖王祖功。』京房《別對》曰:『太平日行上道,升平行次道,霸世行下道。』伏惟大隋啟運,上感乾元,影短日長,振古未之有也。」上大悅,告天
 下。將作役功,因加程課,丁匠苦之。仁壽初,充言上本命與陰陽律呂合者六十餘條而奏之,因上表曰:「皇帝載誕之初,非止神光瑞氣,嘉祥應感,至於本命行年,生月生日,並與天地日月、陰陽律呂運轉相符,表裏合會。此誕聖之異,寶歷之元。今與物更新,改年仁壽。歲月日子,還共誕聖之時並同,明合天地之心,得仁壽之理。故知洪基長算,永永無窮。」上大悅,賞賜優崇,儕輩莫之比。



 仁壽四年甲子歲,煬帝初即位,充及太史丞高智寶奏言:「去歲冬至,日影逾長,今歲皇帝即位,與堯受命年合。昔唐堯受命四十九年,到上元第一紀甲子,天正十一月
 庚戌冬至,陛下即位,其年即當上元第一紀甲子,天正十一月庚戌冬至,正與唐堯同。自放勛以來,凡經八上元,其間綿代,未有仁壽甲子之合。謹案:第一紀甲子,太一在一宮,天目居武德,陰陽歷數並得符同。唐堯丙辰生,丙子年受命,止合三五,未若己丑甲子,支乾並當六合。允一元三統之期,合五紀九章之會,共帝堯同其數,與皇唐比其蹤。信所謂皇哉唐哉,唐哉皇哉者矣。」仍諷齊王暕率百官拜表奉賀。其後熒惑守太微者數旬,於時繕治宮室,征役繁重,充上表稱「陛下修德,熒惑退舍」。百僚畢賀。帝大喜,前後賞賜將萬計。時軍國多務,充候
 帝意欲有所為,便奏稱天文見象,須有改作,以是取媚於上。大業六年,遷內史舍人。



 從征遼東,拜朝請大夫、秘書少監。其後天下亂,帝初罹雁門之厄,又盜賊益起,帝心不自安。充復假托天文,上表陳嘉瑞,以媚於上曰:臣聞皇天輔德,皇天福謙,七政斯齊,三辰告應。伏惟陛下握錄圖而馭黔首,提萬善而化八紘,以百姓為心,匪以一人受慶,先天罔違所欲,後天必奉其時。是以初膺寶歷,正當上元之紀,乾之初九,又與天命符會。斯則聖人冥契,故能動合天經。謹按去年已來,玄象星瑞,毫厘無爽,謹錄尤異,上天降祥、破突厥等狀七事:其一,去八月
 二十八日夜,大流星如斗,出王良北,正落突厥營,聲如崩墻。



 其二,八月二十九日夜,復有大流星如斗,出羽林,向北流,正當北方。依占,頻二夜流星墜賊所,賊必敗散。其三,九月四日夜,頻有兩星大如斗,出北斗魁,向東北流。依占,北斗主殺伐,賊必敗。其四,歲星主福德,頻行京、都二處分野。



 依占,國家之福。其五,七月內,熒惑守羽林,九月七日已退舍。依占,不出三日,賊必敗散。其六,去年十一月二十日夜,有流星赤如火,從東北向西南,落賊帥盧明月營,破其童車。其七,十二月十五日夜,通漢鎮北有赤氣亙北方,突厥將亡之應也。依勘《城錄》,河南洛
 陽並當甲子,與乾元初九爻及上元甲子符合。此是福地,永無所慮。旋觀往政,側聞前古,彼則異時間出,今則一朝總萃。豈非天贊有道,助殲兇孽,方清九夷於東獩,沉五狄於北溟,告成岱嶽,無為汾水。



 書奏,帝大悅,超拜秘書令,親待逾暱。帝每欲征討,充皆預知之,乃假托星象,獎成帝意,在位者皆切患之。宇文化及殺逆之際,並誅充,時年七十五。



 史臣曰:王劭爰自幼童,迄乎白首,好學不倦,究極群書。搢紳洽聞之士,無不推其博物。雅好著述,久在史官,既撰《齊書》,兼修隋典。好詭怪之說,尚委巷之談,文詞鄙穢,
 體統繁雜。直愧南、董,才無遷、固,徒煩翰墨,不足觀採。



 袁充少在江左,初以警悟見稱,委質隋朝,更以玄象自命。並要求時幸,干進務入。



 劭經營符瑞,雜以妖訛,充變動星占,謬增晷影。厚誣天道,亂常侮眾,刑茲勿舍,其在斯乎!且劭為河朔清流,充乃江南望族,乾沒榮利,得不以道,頹其家聲,良可嘆息。



\end{pinyinscope}