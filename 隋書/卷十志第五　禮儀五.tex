\article{卷十志第五 禮儀五}

\begin{pinyinscope}

 輿輦之別,蓋先王之所以列等威也。然隨時而變,代有不同。梁初尚遵齊制,其後武帝既議定禮儀,乃漸有變革。始永明中,步兵校尉伏曼容奏,宋大明中,尚書左丞荀萬秋議,金玉二輅,並建碧旂,象革木輅,並建赤旂,非時運所上,又非五方之色。今五輅五牛及五色幡旗,並請準齊所尚青色。時議所駁,不行。及天監三年,乃改五
 輅旗同用赤而旒不異,以從行運所尚也。七年,帝曰:「據《禮》『玉輅以祀,金輅以賓』,而今大祀,並乘金輅。」詔下詳議。周舍以為:「金輅以之齋車,本不關於祭祀。」於是改陵廟皆乘玉輅,大駕則太僕卿御,法駕則奉車郎馭。其餘四輅,則使人執轡,以硃絲為之。執者武冠、硃衣。又齊永明制,玉輅上施重屋,棲寶鳳皇,綴金鈴,鑷珠璫、玉蚌佩。四角金龍,銜五彩。又畫麒麟頭加於馬首者。十二年,帝皆省之。初,齊武帝造大小輦,並如軺車,但無輪轂,下橫轅軛。梁初,漆畫代之。後帝令上可加笨輦,形如犢車,自茲始也。中方八尺,左右開四望。金為龍首。飾其五末,謂
 轅轂頭及衡端也。金鸞棲軛。其下施重層,以空青雕鏤為龍鳳象。漆木橫前,名為望板。其下交施三十六橫。小輿形似軺車,金裝漆畫,但施八橫。元正大會,乘出上殿。西堂舉哀亦乘之。行則從後。一名輿車。



 羊車一名輦,其上如軺,小兒衣青布褲褶,五辮髻,數人引之。時名羊車小史。



 漢氏或以人牽,或駕果下馬。梁貴賤通得乘之,名曰牽子。



 畫輪車,一乘,駕牛。乘用如齊制,舊史言之詳矣。



 衣書車,十二乘,駕牛。漢皁蓋硃里,過江加綠油幢。硃絲絡,青交路,黃金塗五末。一曰副車。梁朝謂之衣書車。



 皇太子鸞輅,駕三馬,左右騑。硃斑輪,倚獸較,伏鹿軾,九旒,畫降龍,青蓋畫幡,文輈,黃金塗五末。近代亦謂之鸞輅,即象蓋也。梁東宮初建及太子釋奠、元正朝會則乘之。以畫輪為副。若常乘畫輪,以軺衣書車為副。畫輪車,上開四望,綠油幢,硃繩絡,兩箱裏飾以錦,黃金塗五末。



 二千石四品已上及列侯,皆給軺車,駕牛。伏兔箱,青油幢,硃絲絡,轂輞皆黑漆。天監二年令,三公、開府、尚書令,則給鹿幡軺,施耳,後戶,皁輞。尚書僕射、左右光祿大夫、侍中、中書監令、秘書監,則給鳳轄軺,後戶,皁輞。領、護、國子祭酒、太子詹事、尚書、侍中、列卿、散騎常侍,給聊泥軺,
 無後戶,漆輪。車騎、驃騎及諸王除刺史、帶將軍,給龍雀軺,以金銀飾。御史中丞給方蓋軺,形如小傘。



 諸王三公有勛德者,皆特加皁輪車,駕牛,形如犢車。但烏漆輪轂,黃金雕裝,上加青油幢,硃絲絡,通幰或四望。上臺,三夫人亦乘之,以拓幢涅幰為副。王公加禮者,給油幢絡車,駕牛。硃輪華轂。天監二年令,上臺,六宮、長公主、公主、諸王太妃、妃,皆乘青油輿幢車通幰車,拓幢涅幰為副。採女、皇女、諸王嗣子、侯夫人,皆乘赤油拓幢車,以涅幰為副。侍女、直乘涅幰之乘。諸王三公並乘通幰平乘車,竹箕子壁、仰,資榆為輞。如今犢車,但舉幰通覆上。
 方州刺史,並乘通幰平肩輿,從橫施八橫,亦得金渡裝較。天子至於下賤,通乘步輿,方四尺,上施隱膝以及襻,舉之。無禁限。載輿亦如之,但不施腳,以其就席便也。優禮者,人輿以升殿。司徒謝朏,以腳疾優之。



 五牛旗,左青赤,右白黑,黃居其中,蓋古之五時副車也。舊有五色立車,五色安車,合十乘,名為五時車。建旗十二,各如車色。立車則正豎其旗,安車則斜注。馬亦隨五時之色,白馬則硃其鬣尾。左右騑驂,金鍰鏤錫,黃屋左纛,如金根之制。行則從後。名五時副車。晉過江,不恆有事,則權以馬車代之,建旗其上。



 後但以五色木牛象車,
 豎旗於牛背,使人輿之。旗常纏不舒,唯天子親戎,乃舒其旆。周遷以為晉武帝平吳後造五牛之旗,非過江始為也。



 指南車,大駕出,為先啟之乘。漢初,置俞兒騎,並為先驅。左太沖曰:「俞騎騁路,指南司方。」後廢其騎而存其車。



 記里車,駕牛。其中有木人執槌,車行一里,則打一槌。



 鼓吹車,上施層樓,四角金龍,銜旒蘇羽葆。凡鼓吹,陸則樓車,水則樓船,在殿庭則畫筍虡為樓。樓上有翔鷺棲烏,或為鵠形。



 陳承梁末,王琳縱火,延燒車府。至天嘉元年,敕守都官
 尚書、寶安侯到仲舉,議造玉金象革木等五輅及五色副車。皆金薄交龍,為輿倚較,文貔伏軾,虯首銜軛,左右吉陽筒,鸞雀立衡,𣝛文畫轓,綠油蓋,黃絞里,相思尞,金華末。斜注旂旗於車之左,各依方色。加棨戟於車之右,韜以黻繡之衣。獸頭幡,長丈四尺,懸於戟杪。玉輅,正副同駕六馬,餘輅皆駕四馬。馬並黃金為文髦,插以翟尾,玉為鏤錫。又以彩畫赤油,長三尺,廣八寸,系兩軸頭,古曰飛軨,改以彩畫蛙蟆幡,綴兩軸頭,即古飛軨遺象也。五輅兩箱後,皆用玳瑁為鵾翅,加以金銀雕飾,故俗人謂之金鵾車。兩箱之里,衣以紅錦,金花帖釘,上用紅紫
 錦為後簷,青絞純帶,夏用簟,冬用綺繡褥。此後漸修,具依梁制。



 後魏天興初,詔儀曹郎董謐撰朝饗儀,始制軒冕,未知古式,多違舊章。孝文帝時,儀曹令李韶更奏詳定,討論經籍,議改正之。唯備五輅,各依方色,其餘車輦,猶未能具。至熙平九年,明帝又詔侍中崔光與安豐王延明、博士崔瓚採其議,大造車服。定制,五輅並駕五馬。皇太子乘金輅,硃蓋赤質,四馬。三公及王,硃屋青表,制同於輅,名曰高車,駕三馬。庶姓王、侯及尚書令、僕已下,列卿已上,並給軺車,駕用一馬。或乘四望通幰車,駕一牛。自斯
 以後,條章粗備,北齊咸取用焉。其後因而著令,並無增損。



 王、庶姓王、儀同三司已上、親公主,雉尾扇,紫傘。皇宗及三品已上官,青傘硃里。其青傘碧裏,達於士人,不禁。



 正從第一品執事官、散官及儀同三司、諸公主,得乘油色硃絡網車,車牛飾得用金塗及純銀。二品、三品得乘卷通幰車,車牛飾用金塗。四品已下,七品已上,得乘偏幰車,車牛飾用銅。



 尚書令給哄士十五人,左右僕射、御史中丞,各十二人。周氏設六官,置司輅之職,以掌公車之政,辨其名品,與
 其物色。



 皇帝之輅,十有二等:一曰蒼輅,以祀昊天上帝。二曰青輅,以祀東方上帝。



 三曰硃輅,以祀南方上帝及朝日。四曰黃輅,以祭地祇中央上帝。五曰白輅,以祀西方上帝及夕月。六曰玄輅,以祀北方上帝及感帝,祭神州。此六輅,通漆之而已,不用他物為飾。皆疏面,旒就以方色,俱十有二。疏面,刻皮當顱。七曰玉輅,以享先皇,加元服,納後。八曰碧輅,以祭社稷,享諸先帝,大貞於龜,食三老五更,享食諸侯及耕籍。九曰金輅,以祀星辰,祭四望,視朔,大射,賓射,饗群臣,巡犧牲,養國老。十曰象輅,以望秩群祀,視朝,
 燕諸侯及群臣,燕射,養庶老,適諸侯家,巡省,臨太學,幸道法門。十一曰革輅,以巡兵即戎。十二曰木輅,以田獵,行鄉畿。此六輅,又以六色漆而畫之,用玉碧金象革物以飾諸末。皆錫面、金鉤,就以五採,俱十有二。錫面,鏤金當顱。鉤以屬勒鞶纓。



 皇后之車,亦十二等:一曰重翟,以從皇帝,重翟羽為車蕃祀郊禖,享先皇,朝皇太后。二曰厭翟,以祭陰社。次其羽也三曰翟輅,以採桑。翟羽飾之四曰翠輅,以從皇帝,見賓客。翠羽飾之五曰雕輅,以歸寧。刻諸末也六曰篆輅,以臨諸道法門。篆諸飾也六輅皆錫面,硃總總以硃絲為之,置馬勒,直兩耳與兩鑣也。金鉤。



 七曰蒼輅,以適
 命婦家。八曰青輅,九曰硃輅,十曰黃輅,十一曰白輅,十二曰玄輅。五時常出入則供之。六輅皆疏面,繢總。以畫繒為之諸公之輅九:方輅各象方之色碧輅、金輅,皆錫面,鞶纓九就,金鉤。象輅、犀輅、貝輅、革輅、篆輅、木輅,皆疏面,鞶纓九就。凡就,皆以硃白蒼三採。諸侯自方輅而下八,又無碧輅。諸伯自方輅而下七,又無金輅。諸子自方輅而下六,又無象輅。諸男自方輅而下五,又無犀輅。凡就,各如其命。



 諸公夫人之輅車九:厭翟、翟輅、翠輅,皆錫面,硃總,金鉤。雕輅、篆輅,皆勒面,刻白黑韋為當顱繢總。硃輅、黃輅、白輅、玄輅,
 皆雕面,刻漆韋為當顱鷖總。總青黑色繒,其著如硃總。諸侯夫人自翟輅而下八,諸伯夫人自翠輅而下七,諸子夫人自雕輅而下六,諸男夫人自篆輅而下五。鞶纓就數,各視其君。



 公孤卿大夫,皆以中之色乘祀輅。士乘祀車。



 三公之輅車九:祀輅、犀輅、貝輅、篆輅、木輅、夏篆、夏縵、墨車、戔車。



 自篆已上,金塗諸末,疏錫,鞶纓,金鉤。木輅已下,銅飾諸末,疏,鞶纓皆九就。



 三孤自祀輅而下八,無犀輅。六卿自祀輅而下七,又無貝輅。上大夫自祀輅而下六,又無篆輅。中大夫自祀輅而下五,又無木輅。下大夫自祀輅而下四,又無夏篆。士車三:祀車、墨車、戔車。凡就,各
 如其命之數。自孤下,就以硃綠二採。



 三妃、三公夫人之輅九:篆輅、硃輅、黃輅、白輅、玄輅,皆勒面,繢總。夏篆、夏縵、墨車、戔車,皆雕面,鷖總。三弋、三孤內子,自硃輅已下八。六嬪、六卿內子,自黃輅而下七。上媛婦、中大夫孺人,自玄輅而下五。下媛婦、大夫孺人,自夏篆而下四。御婉、士婦人,自夏縵而下三。其鞶纓就,各以其等。皆簟每笰,漆之。君以赤,卿大夫士以玄。



 君駕四,三輈六轡。卿大夫駕三,二輈五轡。士駕二,一輈四轡。



 輅之制,重輪重較而加耳焉。皇帝、皇后之輅,輿廣六尺有六寸,輪高七尺。



 畫輪轂、輈衡以雲牙,箱軾以虡文,虡
 內畫以雜獸。獸伏軾,倚較。諸侯及夫人、命夫、命婦之輅車,廣六尺有二寸,輪崇六尺有六寸。畫轂以雲牙,軾以虡文,虡內畫以雲華。倚較。士不畫。後、夫人、內子已下,同去獸與鹿。



 凡旗,太常畫三辰,日、月、五星。旃畫青龍皇帝升龍,諸侯交龍。旟畫硃雀,旌畫黃麟,旗畫白獸,旐畫玄武,皆加雲。其旃物在軍,亦書其事號,加之以雲氣。



 徽幟亦如之。通帛為旃,雜帛為物。在軍亦書其人官與姓名之事號。徽幟亦書之,但畫其所書之例。旌節又畫白獸,而析羽於其上。



 司常,掌旗物之藏。通帛之旗六,以供郊丘之祀。一曰蒼
 旗,二曰青旗,三曰硃旗,四曰黃旗,五曰白旗,六曰玄旗。畫繢之旗六,以充玉輅之等。一曰三辰之常,二曰青龍之旗,三曰硃鳥之旟,四曰黃麟之旌,五曰白獸之旗,六曰玄武之旐。



 皆左建旗而右建闟戟。又有繼旗四,以施軍旅。一曰麾,以供軍將。二曰K,以供師帥。三曰枿,以供旅帥。四曰旆,以供倅長。諸公方輅、碧輅建旂,金輅建旟,象輅建物,木輅建旐。諸侯自金輅而下,如諸公之旗。諸伯自象輅而下,如諸侯之旗。諸子自犀輅而下,如諸伯之旗。諸男自象輅而下,如諸子之旗。三公犀輅、貝輅、篆輅建旃,木輅建旐,夏篆、夏縵及戔車建物。孤卿已下,各
 以其等建其旗。



 旌杠,皇帝六刃,諸侯五刃,大夫四刃,士三刃。



 旒,皇帝曳地,諸侯及軹,大夫及轂,士及軫。凡注毛於杠首曰綏,析羽曰旌,全羽曰K。其幓,皇帝諸侯加以弧韣。闟戟,方六尺而被之以黻,唯皇帝諸侯輅建焉。闟戟、杠綢與旗同。



 車之蓋圓以象天,輿方以象地。輪輻三十,以象日月。蓋尞二十有八,以象列宿。設和鑾以節趨行,被旗旒以表貴賤。其取象也大,其彰德也明,是以王者尚之。



 皇帝、皇后在喪之車五:一曰木車,初喪乘之。二曰素車,卒哭乘之。三曰藻車,既練乘之。四曰駹車,祥而乘之。五
 曰漆車,禫而乘之。及平齊,得其輿輅,藏於中府,盡不施用。至大象初,遣鄭譯閱視武庫,得魏舊物,取尤異者,並加雕飾,分給六宮。有乾象輦,羽葆圓蓋,畫日月五星、二十八宿、天街雲罕、山林奇怪及游麟飛鳳、硃雀玄武、騶虞青龍,駕二十四馬,以給天中皇后,助祭則乘。又有大樓輦車,龍輈十二,加以玉飾,四轂六衡,方輿圓蓋,金雞樹羽,寶鐸旒蘇,鸞雀立衡,六螭龍銜軛,建太常,畫升龍日月,駕二十牛。又有象輦,左右金鳳,白鹿仙人,羽葆旒蘇,金鈴玉佩,初駕二象,後以六駝代之。並有游觀小樓等輦,駕十五馬車等,合十餘乘,皆魏天興中之所制也。
 宣帝至是,咸復御之。復令天下車,皆以渾成木為輪。



 開皇元年,內史令李德林奏,周、魏輿輦乖制,請皆廢毀。高祖從之。唯留魏太和時儀曹令李韶所制五輅,齊天保所遵用者。又留魏熙平中,太常卿穆紹議皇后之輅,其從祭則御金根車,親桑則御雲母車,並駕四馬。歸寧則御紫罽車,游行則御安車,吊問則御紺罽軿車,並駕三馬。於後著令,制五輅。



 玉輅,青質,以玉飾諸末。重箱盤輿,左青龍,右白虎,金鳳翅,畫虡文鳥獸。



 黃屋左纛,金鳳在軾前,八鸞在衡,二鈴在軾。龍輈,前設鄣塵。青蓋黃裏,繡飾。



 博山鏡子,樹羽。輪
 皆硃斑重牙。左建旗,十有二旒,幓旒皆畫升龍,其長曳地。



 右載闟戟,長四尺,廣三尺,黻文。旂首金龍頭,銜結綬及鈴緌。駕蒼龍,金方,插翟尾五隼,鏤錫,鞶纓十有二就。錫馬當顱,鏤金為之。鞶馬大帶,纓馬鞅,皆以五彩飾之。就成也,一幣為一就。祭祀、納後則供之。



 金輅,赤質,以金飾諸末。左建旟,右建闟戟。旟畫鳥隼餘與玉輅同。駕赤鳷。



 朝覲會同,饗射飲至則供之。



 象輅,黃質,以象飾諸末。左建旌,右建闟戟。旌畫黃麟駕黃鳷。行道則供之。



 革輅,白質,挽之以革。左建旗,右建闟戟。旗畫白獸駕白駱。巡
 守臨兵事則供之。



 木輅,漆之。左建旐,右建闟戟。旐畫龜蛇駕黑鳷。田獵則供之。



 五輅之蓋,旌旗之質,及鞶纓,皆從輅之色。蓋之裏俱用黃。其鏤錫五輅同。



 安車,飾重輿,曲壁,紫油纁硃里,通幰,硃絲絡網,硃鞶贐纓,硃覆發,具絡。駕赤鳷。臨幸則供之。



 四望車,制同犢車金飾,青油纁硃里,通幰。拜陵臨吊則供之。



 皇后、皇太后重翟,青質,金飾諸末。硃輪,金根硃牙。其箱飾以重翟羽,青油纁硃里,通幰,繡紫帷,硃絲絡網,繡紫絡帶。八鑾在衡,錫,鞶纓十二就,金棨方釳,插翟尾,硃總。
 總以硃為之,如馬纓而小,著馬勒,在兩耳兩鑣也。駕蒼龍。受冊、從郊禖、享廟則供之。



 厭翟,赤質,金飾諸末。輪畫硃牙。其箱飾以次翟羽,紫油纁硃里,通幰,紅錦帷,硃絲絡網,紅錦絡帶。其餘如重翟。駕黃騮。親桑則供之。



 翟車,黃質,金飾諸末。輪畫硃牙。其車側飾以翟羽,黃油纁黃裏,通幰,白紅錦帷,硃絲絡網,白紅錦絡帶。其餘如重翟。駕黃騮。歸寧則供之。諸鞶纓之色,皆從車質。



 安車,赤質,金飾。紫通幰硃里。駕四馬。臨幸及吊則供之。



 皇太子金輅,赤質,金飾諸末。重較,箱畫虡文鳥獸,黃屋,
 伏鹿軾,龍輈。



 金鳳一,在軾前。設鄣塵。硃蓋黃里。輪畫硃牙。左建旂,九旒,右載闟戟。旂首金龍頭。銜結綬及鈴緌。駕赤鳷四。八鑾在衡,二鈴在軾。金棨方釳,插翟尾五隼、鏤錫,鞶纓九就。從祀享、正冬大朝、納妃則乘之。



 軺車,金飾諸末。紫通幰硃里。駕一馬。五日常朝及朝饗宮臣,出入行道乘之。



 四望車,金飾諸末。紫油纁通幰硃里,硃絲絡網。駕一馬。吊臨則乘之。



 公及一品象輅,黃質,以象飾諸末。建旟,畫以鳥隼。受冊告廟,升壇上任,親迎及葬則乘之。



 侯伯及二品三品革輅,白質,以革飾諸末。建旟,畫熊獸。受冊告廟,親迎及葬則乘之。



 子男及四品木輅,黑質,以漆飾之。建旟,畫以龜蛇。受冊告廟,親迎及葬則乘之。



 象輅已下,旒及就數,各依爵品,雖依禮制名,未及創造。開皇三年閏十二月,並詔停造,而盡用舊物。至九年平陳,又得輿輦。舊著令者,以付有司,所不載者,並皆毀棄。雖從儉省,而於禮多闕。十四年,詔又以見所乘車輅,因循近代,事非經典,令更議定。於是命有司詳考故實,改造五輅及副。玉輅青質,祭祀乘之。金輅赤質,朝會禮還
 乘之。象略黃質,臨幸乘之。革輅白質,戎事乘之。木輅玄質,耕藉乘之。五輅皆硃斑輪、龍輈、重輿,建十二旒,並畫升龍。左建闟戟。旂旒與輅同色。樊纓十有二就。王、五等開國、第一第二品及刺史輅,硃質,硃蓋,斑輪。



 左建旂,旂畫龍,一升一降。左建闟戟。第三第四品輅,硃質,硃蓋,左建旃,通帛為之,旂旃皆赤。其旒及樊纓就數,各依其品。大業元年,更制車輦,五輅之外,設副車。詔尚書令楚公楊素、吏部尚書奇章公牛弘、工部尚書安平公宇文愷、內史侍郎虞世基、禮部侍郎許善心、太府少卿何稠、朝請郎閻毗等,詳議奏決。於是審擇前朝故事,定其取舍
 云。



 玉輅,禋祀所用,飾以玉。《白虎通》云:「玉輅,大輅也。」《周禮》巾車氏所掌,「鏤錫,樊纓十有再就,建太常,十有二旒」。虞氏謂之鸞車,夏后氏謂之鉤車,殷謂之大輅,周謂之乘輅。《大戴禮》著其形式,上蓋如規象天,二十八象列星,下方輿象地,三十輻象一月。前視則睹鑾和之聲,側觀則睹四時之運。



 昔成湯用而郊祀,因有山車之瑞,亦謂桑根車。蔡邕《獨斷》論漢制度,凡乘輿車,皆有六馬,羽蓋金爪,黃屋左纛,鏤棨方釳,重轂繁纓,黃繒為蓋里也。左纛,以旄牛尾建於竿上,其大如斗,立於左騑也。鏤棨高闊
 各五寸,上如傘形,施於發上,而插翟尾也。方釳當顱,蓋馬冠也。繁纓,膺前索也。重轂,重施轂也。應劭《漢官》,大輅龍旂,畫龍於旂上也。董巴《志》謂為瑞山車,秦謂金根,即殷輅矣。



 司馬彪《志》亦云:「漢備五輅,或謂德車,其所駕馬,皆如方色。」唯晉太常卿摯虞獨疑大輅,謂非玉輅。摯虞之說,理實可疑,而歷代通儒,混為玉輅,詳其施用,義亦不殊。左建太常。案《釋名》:「日月為常,畫日月於旗端,言常明也。」



 又云:「自夏始也。」奚仲為夏車正,加以旂常,於是旒就有差,用明尊卑之別也。



 董巴所述,全明漢制。天子建太常,十二斿,曳地,日月升龍,象天明也。今之玉輅,參用
 舊典,消息取舍,裁其折中。以青為質,玉飾其末。重箱盤輿,左龍右獸,金鳳翅,畫虡文,軛左立纛。金鳳一,在軾前。八鸞在衡,二鈴在軾。龍輈之上,前設鄣塵。青蓋黃裏,繡游帶。金博山,綴以鏡子,下垂八佩。樹四十葆羽。輪皆硃斑重牙,復轄。左建太常,十有二旒,皆畫升龍日月,其長曳地。右載闟戟,長四尺,闊三尺,黻文。旗首金龍頭,銜鈴及緌,垂以結綬。駕蒼龍,金棨方釳,插翟尾五隼,鏤錫,鞶纓十有二就,皆五繒罽,以為文飾。天子祭祀、納後則乘之。



 馭士二十八人,餘輅準此。



 副車,案蔡邕《獨斷》,五輅之外,乃復設五色安車、立車各
 一乘,皆駕四馬,是為五時副車。俗人名曰五帝車者,蓋副車也。故張良狙擊秦皇帝,誤中副車。漢家制度,亦備副車。司馬彪云:「德車駕六,後駕四,是為副車。」《魏志》亦云:「天子命太祖駕金根六馬,設五時副車。」江左乃闕,至梁始備。開皇中,不置副車,平陳得之,毀而弗用。至是復並設之。副玉輅,色及旗章,一同正輅,唯降二等。駕用四馬,馭士二十四人。餘四副準此。



 金輅,案《尚書》,即綴輅也。《周官》:「金輅,鏤錫,繁纓九就,建大旂,以賓,同姓以封。」夫禮窮則通,下得通於上也,故天子乘之,接賓宴,同姓諸侯,受而出封。是以漢太子、諸王皆
 乘金輅及安車,並硃斑輪,倚獸較,伏鹿軾,黑𣝛文,畫籓,青蓋,金華施尞,硃畫轅,金塗飾。非皇子為王,不錫此乘,皆左右騑,駕三馬。旂九旒,畫降龍。皇孫乘綠車,亦駕之。魏、晉制,太子及諸王皆駕四馬。依摯虞議,天子金輅,次在第二。又云,金輅以朝,象輅以賓。則是晉用輅與周異矣。《宋起居注》,泰始四年,尚書令建安王休仁議:「天子之元子,士也,故齒胄於闢雍,欲使知教而後尊,不得生而貴矣。既命之後,禮同上公,故天子賜之金輅,但減旂章為等級。象及革木,賜異姓諸侯。在朝卿士,亦準斯例。」



 此則皇太子及帝子王者,通得乘之。自晉過江,王公以下,
 車服卑雜,唯有太子禮秩崇異。又乘山石安車,義不經見,事無所出。賜金輅者,此為古制,降乘輿二等,駕用四馬。唯天子五輅,通駕六馬。旟旌旗旐,並十二旒。左建旟。案《爾雅》:「錯革鳥曰旟。」郭璞云:「此謂全剝鳥皮毛,置之竿上也。」舊說,刻為革鳥。



 孫叔敖云:「革,急也。言畫急疾鳥於旒上也。」《周官》所謂鳥隼為旟,亦是急義。今之金輅,赤質,黃金飾諸末。左建旂,畫飛隼,右建闟戟,鞶輿鳳翅等,並同玉輅。駕赤鳷。臨朝會同,饗射飲至則用之。



 皇太子輅,古者金飾。宋、齊以來,並乘象輅。宇文愷、閻毗奏:「案宋大明六年,初備五輅,有司奏云:『秦改周輅,創制金根,漢、魏
 因循,其形莫改。而金玉二輅,雕飾略同,造次瞻睹,殆無差別。若錫於東儲,在禮嫌重,非所以崇峻陛級,表示等威。今皇太子宜乘象輅,碧旂九葉,進不斥尊,退不逼下,酌時沿古,於禮為中。』觀宋此義,乃無副車。新置五輅,金玉同體,至象已下,即為差降。



 所以太子不得乘金輅,欲示等威,故令給象。今取《周禮》之名,依漢家之制,天子五輅,形飾並同。旒及繁纓,例皆十二,黃屋左纛,金根重轂,無不悉同,唯應五方色以為殊耳。若用此輅,給於太子,革木盡皆不可,何況金象者乎?既制副車,駕用四馬,至於金輅,自有等差。《春秋》之義,降下以兩。今天子金輅,駕
 用六馬,十二旒,太子金輅,駕用四馬,降龍九旒,制頗同於副車,又有旌旗之別。並嫡皇孫及親王等輅,並給金輅,而減其雕飾,合於古典。臣謂非嫌。」制曰:「可。」



 於是太子金輅,赤質,制同副車,具體而小,亦駕四馬,馭士二十人。皇嫡孫金輅,綠質,降太子一等。去盤輿重轂,轅上起箱,末以金飾,旌長七刃,七旒。駕用四馬,馭士一十八人。親王金輅,以赤為質,餘同於皇嫡孫。唯在其國及納妃親迎則給之,常朝則乘象輅。



 象輅,案《尚書》,即先輅也。《周禮》:「象輅,硃繁纓五就,建大赤,以朝,異姓以封。」左建旌。案《爾雅注》「旄首曰旌」,許慎所說「
 游車載旌」。



 《廣雅》云:「天子旌高九刃,諸侯七刃,大夫五刃。」《周書·王會》:「張羽鳧旌。」《禮記》云:「龍旂九旒,天子之旌也。」今象輅,以黃為質,象飾諸末。



 左建旌,畫綠麟,右建闟戟。駕黃鳷。祀后土則用之。



 革輅,案《釋名》:「天子車也」。《周禮》:「革輅,龍勒,絳纓五就,建大白,用之即戎,以封四衛。」古者革挽而漆之,更無他飾。又有「戎輅之萃,廣車之萃,闕車之萃,輕車之萃」。此皆兵車,所謂五戎。然革輅亦名戎輅,天子在軍所乘。廣車,橫陣車也。闕車,補闕車也。飾並以革,故「師供革車,各以其萃」。



 摯虞議云,革輅第四。左建旌。案《釋名》「熊獸為旗」,《周官》「龍
 旂九旒,以象大火」。今革輅白質,鞔之以革。左建旗,畫騶虞,右建闟戟,駕白駱。巡守臨兵則用之。三品已下,並乘革輅,硃色為質。馭士十六人。



 木輅,案《尚書》,即次輅也。《周官》:「木輅,緇樊鵠纓建麾,以畋,以封籓國。」晉摯虞云,畋輅第五。唯宋泰始詔,乘木輅以耕稼。徐爰《釋疑略》曰:「天子五輅,晉遷江左,闕其三,唯有金輅以郊,木輅即戎。宋大明時,始備其數。」



 凡五輅之蓋,旌旗之質及鞶纓皆從方色。蓋里並黃,雕飾如一。沈約曰:「金象革木,《禮圖》不載其形。」今旒數羽葆,並同玉輅。左建旐。案《周官》:「龜蛇為旐。」《釋名》云:「龜知氣兆
 之吉兇也。」許慎云:「旐有四斿,以象營室。」



 今木輅黑質,漆之。左建旐,畫玄武,右建闟戟。駕黑鳷。畋獵用之。四品方伯乘木輅,赤質,駕士十四人。



 安車,案《禮》,卿大夫致事則乘之。其制如輜軿。蔡邕《獨斷》有五色安車,皆畫輪重轂。今畫輪,重輿,曲壁,紫油幢絳里,通幰,硃絲絡網,赤鞶纓。駕四馬。省問臨幸則乘之。皇太子安車,斑輪,赤質,制略同乘輿,亦駕四馬。



 四望車,案晉《中朝大駕鹵簿》,四望車,駕牛中道。《東宮舊儀》,皇太子及妃,皆有畫輪四望車。今四望車制同犢車,黃金飾,青油幢硃里,紫通幰,紫絲網。駕一牛。拜陵臨吊
 則用之。皇太子四望車,綠油幢,青通幰,硃絲絡網。



 耕根車,案沈約云:「親幸耕籍御之。三蓋車,一名芝車,又名耕根車。置耒耜於軾上。」即潘岳所謂「紺轅屬於黛耜」者也。開皇無之,駕出親耕,則乘木輅,蓋依宋泰始之故事也。今耕根車,以青為質,三重施蓋,羽葆雕裝,並同玉輅。駕六馬。其軾平,以青囊盛耒而加於上。籍千畝,行三推禮,則親乘焉。



 羊車,案晉司隸校尉劉毅奏護軍羊琇私乘者也。開皇無之,至是始置焉。其制如軺車,金寶飾,紫錦幰,硃絲網。馭童二十人,皆兩鬟髻,服青衣,取年十四五者為,謂之
 羊車小史。駕以果下馬,其大如羊。



 屬車,案古者諸侯貳車九乘,秦滅九國,兼其車服,故為八十一乘。漢遵不改。



 武帝祠太一甘泉,則盡用之。明帝上原陵,又用之。法駕三十六乘,小駕十二乘。



 開皇中,大駕十二乘,法駕減半。大業初,屬車備八十一乘,並如犢車,紫通幰,硃絲絡網,黃金飾。駕一牛。在鹵簿中,單行正道。至三年二月,帝嫌其多,問起部郎閻毗。毗曰:「臣共宇文愷參詳故實,此起於秦,遂為後式,故張衡賦云『屬車九九』是也。次及法駕,三分減一,此漢制也。故《文帝紀》『奉天子法駕迎代邸』,如淳曰『屬車三十六乘』是也。又據宋
 孝建時,有司奏議,晉遷江左,唯設五乘,尚書令建平王宏曰:『八十一乘,無所準憑,江左五乘,儉不中禮。但帝王旂旒之數,皆用十二,今宜準此,設十二乘。開皇平陳,因以為法令。憲章往古,大駕依秦,法駕依漢,小駕依宋,以為差等。帝曰:「大駕宜用三十六,法駕宜用十二,小駕除之可也。」



 輦,案《釋名》「人所輦也。」漢成帝游後庭則乘之。徐爰《釋問》云:「天子御輦,侍中陪乘。」今輦制象軺車,而不施輪,通幰硃絡,飾以金玉,用人荷之。



 副輦,加笨,制如犢車,亦通幰硃絡,謂之蓬輦。自梁武帝
 始也。



 輿,案《說文》云:「箯,竹輿也。」《周官》曰:「周人上輿。」漢室制度,以雕為之,方徑六尺。今輿制如輦而但小耳,宮苑宴私則御之。



 小輿,幰方,形同幄帳。自閤出升正殿則御之。



 軺車,案《六韜》,一名遙車,蓋言遙遠四顧之車也。漢武帝迎申公,弟子二人乘軺傳從。此又是馳傳車也。《晉氏鹵簿》,御史軺車行中道。《晉公卿禮秩》云:「尚書令軺,黑耳後戶。」今軺車,青通幰,駕二馬。王侯入學,五品朝婚,通給之。司隸刺史及縣令、詔使品第六七,則並駕一馬。



 犢車,案《魏武書》,贈楊彪七香車二乘,用牛駕之。蓋犢車也。《長沙耆舊傳》曰:「劉壽常乘通幰車。」今犢車通幰,自王公已下,至五品已上,並給乘之。



 三品已上,青幰硃里,五品已上,紺幰碧裏,皆白銅裝。唯有慘及吊喪者,則不張幰而乘鐵裝車。六品已下不給,任自乘犢車,弗許施幰。初,五品已上,乘偏幰車,其後嫌其不美,停不行用,以亙幰代之。三品已上通幰車則青壁,一品軺車,油幰硃網,唯車輅一等,聽敕始得乘之。



 馬珂,三品已上九子,四品七子,五品五子。



 皇后重翟車,案《周禮》,正後亦有五輅:一曰重翟,二曰厭
 翟,三曰安車,四曰翟車,五曰輦車。漢制,後法駕,乘重翟車。今重翟,青質,金飾諸末。畫輪,金根硃牙,重轂。其箱飾以重翟羽。青油幢硃里,通幰,紫繡帷,硃絲絡,紫繡帶。



 八鑾在衡,鏤錫,鞶纓十有二就,金棨方釳,插翟尾,硃總,綴於馬勒及兩金鑣之上。駕蒼龍。受冊從祀郊禖享廟則供之。



 厭翟,赤質,金飾諸末。硃輪,畫硃牙。其箱飾以次翟羽,紫油幢硃里,通幰,紅錦帷,硃絲絡網,紅錦帶。其餘如重翟。駕赤鳷。採桑則供之。



 翟車,黃質,金飾諸末。輪畫硃牙。其箱飾以翟羽,黃油幢
 黃裏,通幰,白紅錦帷,硃絲絡網,白紅錦帶。其餘如重翟。駕黃鳷。歸寧則供之。諸鞶纓之色,皆從車質。



 安車,金飾,紫通幰,硃里。駕四馬。臨幸及吊則供之。



 輦車,金飾,同於蓬輦,通幰,斑輪,駕用四馬。宮苑近行則乘之。



 皇後屬車三十六乘,初宇文愷、閻毗奏定,請減乘輿之半。禮部侍郎許善心奏駁曰:「謹案《周禮》,後備六服,並設五輅,採章之數,並與王同,屬車之制,不應獨異。又宋孝建時,議定輿輦,天子屬車,十有二乘。至大明元年九月,有司奏皇后副車,未有定式,詔下禮官,議正其數。博士王燮之議:『鄭玄云:後象王立六宮,亦正寢一而燕寢
 五。推其所立,每與王同,謂十二乘通關為允。』宋帝從之,遂為後式。今請依乘輿,不須差降。」制曰:「可。」



 三妃乘翟車,以赤為質,駕二馬。九嬪已下,並乘犢車,青幰,硃絡網。



 皇太子妃乘翟車,以赤為質,駕三馬,畫轅金飾。犢車為副,紫幰,硃絡網。



 良娣已下,並乘犢車,青幰硃里。



 三公夫人、公主、王妃,並犢車,紫幰,硃絡網。五品已上命婦,並乘青幰,與其夫同。



\end{pinyinscope}