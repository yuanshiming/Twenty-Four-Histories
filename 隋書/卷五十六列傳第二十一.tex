\article{卷五十六列傳第二十一}

\begin{pinyinscope}

 盧愷盧愷,字長仁,涿郡範陽人也。父柔,終於魏中書監。愷性孝友,神情爽悟,略涉書記,頗解屬文。周齊王憲引為記室。其後襲爵容城伯,邑千一百戶。從憲伐齊,愷說柏杜鎮下之。遷小吏部大夫,增邑七百戶。染工上士王神歡者,嘗以賂自進,塚宰宇文護擢為計部下大夫。愷諫曰:「古者登高能賦,可為大夫,求賢審官,理須詳慎。今神歡
 出自染工,更無殊異,徒以家富自通,遂與搢紳並列,實恐惟鵜之刺,聞之外境。」護竟寢其事。建德中,增邑二百戶。歲餘,轉內史下大夫。武帝在雲陽宮,敕諸屯簡老牛,欲以享士。愷進諫曰:「昔田子方贖老馬,君子以為美談。向奉明敕,欲以老牛享士,有虧仁政。」帝美其言而止。轉禮部大夫,為聘陳使副。先是,行人多從其國禮,及愷為使,一依本朝,陳人莫能屈。四年秋,李穆攻拔軹關、柏崖二鎮,命愷作露布,帝讀之大悅,曰:「盧愷文章大進,荀景倩故是令君之子。」尋授襄州總管司錄,轉治中。大象元年,徵拜東京吏部大夫。開皇初,加上儀同三司,除尚書
 吏部侍郎,進爵為侯,仍攝尚書左丞。每有敷奏,侃然正色,雖逢喜怒,不改其常。帝嘉愷有吏乾,賜錢二十萬,並賚雜彩三百匹,加散騎常侍。八年,上親考百僚,以愷為上。愷固讓,不敢受,高祖曰:「吏部勤幹,舊所聞悉。今者上考,僉議攸同,當仁不讓,何愧之有!皆在朕心,無勞飾讓。」



 歲餘,拜禮部尚書,攝吏部尚書事。會國子博士何妥與右僕射蘇威不平,奏威陰事。



 愷坐與相連,上以愷屬吏。憲司奏愷曰:「房恭懿者,尉遲迥之黨,不當仕進。威、愷二人曲相薦達,累轉為海州刺史。又吏部預選者甚多,愷不即授官,皆注色而遣。



 威之從父弟徹、肅二人,並以鄉
 正徵詣吏部。徹文狀後至而先任用,肅左足攣蹇,才用無算,愷以威故,授朝請郎。愷之朋黨,事甚明白。」上大怒曰:「愷敢將天官以為私惠!」愷免冠頓首曰:「皇太子將以通事舍人蘇夔為舍人,夔即蘇威之子,臣以夔未當遷,固啟而止。臣若與威有私,豈當如此!」上曰:「蘇威之子,朝廷共知,卿乃固執,以徼身幸。至所不知者,便行朋附,奸臣之行也。」於是除名為百姓。未幾,卒於家。自周氏以降,選無清濁,及愷攝吏部,與薛道衡、陸彥師等甄別士流,故涉黨固之譖,遂及於此。子義恭嗣。



 令狐熙
 令狐熙,字長熙,燉煌人也,代為西州豪右。父整,仕周,官至大將軍、始、豐二州刺史。熙性嚴重,有雅量,雖在私室,終日儼然。不妄通賓客,凡所交給,必一時名士。博覽群書,尤明《三禮》,善騎射,頗知音律。起家以通經為吏部上士,尋授都督、輔國將軍,轉夏官府都上士,俱有能名。以母憂去職,殆不勝喪。



 其父戒之曰:「大孝在於安親,義不絕嗣。吾今見存,汝又只立,何得過爾毀頓,貽吾憂也!」熙自是稍加饘粥。服闋,除小駕部,復丁父憂,非杖不起,人有聞其哭聲,莫不為之下泣。河陰之役,詔令墨縗從事,還授職方下大夫,襲爵彭陽縣公,邑二千一百戶。及武
 帝平齊,以留守功,增邑六百戶。進位儀同,歷司勛、吏部二曹中大夫,甚有當時之譽。高祖受禪之際,熙以本官行納言事。尋除司徒左長史,加上儀同,進爵河南郡公。時吐谷渾寇邊,以行軍長史從元帥元諧討之,以功進位上開府。會蜀王秀出鎮於蜀,綱紀之選,咸屬正人,以熙為益州總管長史。未之官,拜滄州刺史。時山東承齊之弊,戶口簿籍類不以實。熙曉諭之,令自歸首,至者一萬戶。在職數年,風教大洽,稱為良二千石。開皇四年,上幸洛陽,熙來朝,吏民恐其遷易,悲泣於道。及熙復還,百姓出境迎謁,歡叫盈路。在州獲白烏、白麞、嘉麥,甘露降
 於庭前柳樹。八年,徙為河北道行臺度支尚書,吏民追思,相與立碑頌德。及行臺廢,授並州總管司馬。後徵為雍州別駕。尋為長史,遷鴻臚卿。後以本官兼吏部尚書,往判五曹尚書事,號為明乾,上甚任之。及上祠太山還,次汴州,惡其殷盛,多有奸俠,於是以熙為汴州刺史。下車禁游食,抑工商,民有向街開門者杜之,船客停於郭外星居者,勒為聚落,僑人逐令歸本,其有滯獄,並決遣之,令行禁止,稱為良政。上聞而嘉之,顧謂侍臣曰:「鄴都天下難理處也。」敕相州刺史豆盧通,令習熙之法。其年來朝,考績為天下之最,賜帛三百匹,頒告天下。



 上以嶺
 南夷、越數為反亂,徵拜桂州總管十七州諸軍事,許以便宜從事,刺史以下官得承制補授。給帳內五百人,賜帛五百匹,發傳送其家累,改封武康郡公。熙至部,大弘恩信,其溪洞渠帥更相謂曰:「前時總管皆以兵威相脅,今者乃以手教相諭,我輩其可違乎?」於是相率歸附。先是,州縣生梗,長吏多不得之官,寄政於總管府。熙悉遣之,為建城邑,開設學校,華夷感敬,稱為大化。時有寧猛力者,與陳後主同日生,自言貌有貴相,在陳日,已據南海,平陳後,高祖因而撫之,即拜安州刺史。然驕倨,恃其阻險,未嘗參謁。熙手書諭之,申以交友之分。其母有疢,
 熙復遺以藥物。猛力感之,詣府請謁,不敢為非。熙以州縣多有同名者,於是奏改安州為欽州,黃州為峰州,利州為智州,德州為歡州,東寧為融州,上皆從之。



 在職數年,上表曰:「臣忝寄嶺表,四載於茲,犬馬之年,六十有一。才輕任重,愧懼兼深,常願收拙避賢,稍免官謗。然所管遐曠,綏撫尤難,雖未能頓革夷風,頗亦漸識皇化。但臣夙患消渴,比更增甚,筋力精神,轉就衰邁。昔在壯齒,猶不如人,況今年疾俱侵,豈可猶當重寄!請解所任。」優詔不許,賜以醫藥。熙奉詔,令交州渠帥李佛子入朝。佛子欲為亂,請至仲冬上道,熙意在羈縻,遂從之。有人詣闕
 訟熙受佛子賂而舍之,上聞而固疑之。既而佛子反問至,上大怒,以為信然,遣使者鎖熙詣闕。熙性素剛,鬱鬱不得志,行至永州,憂憤發病而卒,時年六十三。



 上怒不解,於是沒其家財。及行軍總管劉方擒佛子送於京師,言熙實無贓貨,上乃悟,於是召其四子,聽預仕焉。少子德棻,最知名。



 薛胄薛胄,字紹玄,河東汾陰人也。父端,周蔡州刺史。胄少聰明,每覽異書,便曉其義。常嘆訓注者不會聖人深旨,輒以意辯之,諸儒莫不稱善。性慷慨,志立功名。周明帝時,
 襲爵文城郡公。累遷上儀同,尋拜司金大夫,後加開府。高祖受禪,擢拜魯州刺史,未之官,檢校廬州總管事。尋除兗州刺史。及到官,系囚數百,胄剖斷旬日便了,囹圄空虛。有陳州人向道力者,偽作高平郡守,將之官,胄遇諸途,察其有異,將留詰之。司馬王君馥固諫,乃聽詣郡。既而悔之,即遣主簿追禁道力。



 有部人徐俱羅者,嘗任海陵郡守,先是已為道力偽代之。比至秩滿,公私不悟。俱羅遂語君馥曰:「向道力以經代俱羅為郡,使君豈容疑之?」君馥以俱羅所陳,又固請胄。胄呵君馥曰:「吾已察知此人詐也。司馬容奸,當連其坐!」君馥乃止。



 遂往收之,
 道力懼而引偽。其發奸摘伏,皆此類也,時人謂為神明。先是,兗州城東沂、泗二水合而南流,泛濫大澤中,胄遂積石堰之,使決令西注,陂澤盡為良田。



 又通轉運,利盡淮海,百姓賴之,號為薛公豐兗渠。胄以天下太平,登封告禪,帝王盛烈,遂遣博士登太山,觀古跡,撰《封禪圖》及儀上之。高祖謙讓不許。後轉郢州刺史,前後俱有惠政。徵拜衛尉卿,尋轉大理卿,持法寬平,名為稱職。後遷刑部尚書。時左僕射高熲稍被疏忌,及王世積之誅也,熲事與相連,上因此欲成熲罪。胄明雪之,正議其獄。由是忤旨,械系之,久而得免。檢校相州事,甚有能名。



 會漢王
 諒作亂並州,遣偽將綦良東略地,攻逼慈州。刺史上官政請援於胄,胄畏諒兵鋒,不敢拒,良又引兵攻胄,胄欲以計卻之,遣親人魯世範說良曰:「天下事未可知,胄為人臣,去就須得其所,何遽相攻也?」良於是釋去,進圖黎陽。及良為史祥所攻,棄軍歸胄。朝廷以胄懷貳心,鎖詣大理。相州吏人素懷其恩,詣闕理胄者百餘人,胄竟坐除名,配防嶺南,道病卒。有子、獻,並知名。



 宇文弼宇文弼,字公輔,河南洛陽人也,其先與周同出。祖直力覲,魏巨鹿太守。父珍,周宕州刺史。弼慷慨有大節,博學
 多通,仕周為禮部上士。嘗奉使鄧至國及黑水、龍涸諸羌,前後降附三十餘部。及還,奉詔修定《五禮》,書成奏之,賜公田十二頃,粟百石。累遷少吏部,擢八人為縣令,皆有異績,時以為知人。轉內史都上士。武帝將出兵河陽以伐齊,謀及臣下,弼進策曰:「齊氏建國,於今累葉,雖曰無道,籓屏之寄,尚有其人。今之用兵,須擇其地。河陽沖要,精兵所聚,盡力攻圍,恐難得志。如臣所見,彼汾之曲,戍小山平,攻之易拔。用武之地,莫過於此,願陛下詳之。」帝不納,師竟無功。建德五年,大舉伐齊,卒用弼計。弼於是募三輔豪俠少年數百人以為別隊,從帝攻拔晉州。
 身被三創,苦戰不息,帝奇而壯之。後從帝平齊,以功拜上儀同,封武威縣公,邑千五百戶,賜物千五百段,奴婢百五十口,馬牛羊千餘頭,拜司州總管司錄。宣帝嗣位,遷左守廟大夫。時突厥寇甘州,帝令侯莫陳昶率兵擊之,弼為監軍。謂昶曰:「黠虜之勢,來如激矢,去若絕弦,若欲追躡,良為難及。且宜選精騎,直趨祁連之西。賊若收軍,必自蓼泉之北,此地險隘,兼復下濕,度其人馬,三日方度,緩轡追討,何慮不及?彼勞我逸,破之必矣。若邀此路,真上策也。」昶不能用之,西取合黎,大軍行遲,虜已出塞。



 其年,弼又率兵從梁士彥攻拔壽陽,尋改封安樂
 縣公,增邑六百戶,賜物六百段,加以口馬。除澮州刺史,俄轉南司州刺史。後司馬消難之奔陳也,弼追之不及。遇陳將樊毅,戰於漳口,自旦及午,三戰三捷,虜獲三千人。除黃州刺史,尋轉南定州刺史。開皇初,以前功封平昌縣公,加邑一千二百戶,入為尚書右丞。時西羌內附,詔弼持節安集之,置鹽澤、蒲昌二郡而還。遷尚書左丞,當官正色,為百僚所憚,三年,突厥寇甘州,以行軍司馬從元帥竇榮定擊破之。還除太僕少卿,轉吏部侍郎。平陳之役,楊素出信州道,令弼持節為諸軍節度,仍領行軍總管。劉仁恩之破陳將呂仲肅也,弼有謀焉。加開府,
 擢拜刑部尚書,領太子虞候率。上嘗親臨釋奠,弼與博士論議,詞致清遠,觀者屬目。上大悅,顧謂侍臣曰:「朕今睹周公之制禮,見宣尼之論孝,實慰朕心。」於是頒賜各有差。時朝廷以晉陽為重鎮,並州總管必屬親王,其長史、司馬亦一時高選。前長史王韶卒,以弼有文武干用,出為並州長史。俄以父艱去職,尋詔起之。十八年,遼東之役,授元帥漢王府司馬,仍尋領行軍總管。軍還之後,歷朔、代、吳三州總管,皆有能名。煬帝即位,徵拜刑部尚書,仍持節巡省河北。還除泉州刺史。歲餘,復拜刑部尚書,尋轉禮部尚書。



 弼既以才能著稱,歷職顯要,聲望甚
 重,物議時談,多見推許,帝頗忌之。時帝漸好聲色,尤勤遠略,弼謂高熲曰:「昔周天元好聲色而國亡,以今方之,不亦甚乎?」



 又言:「長城之役,幸非急務。」有人奏之,竟坐誅死,時年六十二,天下冤之。



 所著辭賦二十餘萬言,為《尚書》、《孝經注》行於時。有子儉、瑗。



 張衡張衡,字建平,河內人也。祖嶷,魏河陽太守。父光,周萬州刺史。衡幼懷志尚,有骨鯁之風。年十五,詣太學受業,研精覃思,為同輩所推。周武帝居太后憂,與左右出獵,衡露發輿櫬,扣馬切諫。帝嘉焉,賜衣一襲,馬一匹,擢拜漢
 王侍讀。



 衡又就沈重受《三禮》,略究大旨。累遷掌朝大夫。高祖受禪,拜司門侍郎。及晉王廣為河北行臺,衡歷刑部、度支二曹郎。後以臺廢,拜並州總管掾。及王轉牧揚州,衡復為掾,王甚親任之。衡亦竭慮盡誠事之,奪宗之計,多衡所建也。以母憂去職,歲餘,起授揚州總管司馬,賜物三百段。開皇中,熙州李英林聚眾反,署置百官,以衡為行軍總管,率步騎五萬人討平之。拜開府,賜奴婢一百三十口,物五百段,金銀雜畜稱是。及王為皇太子,拜衡右庶子,仍領給事黃門侍郎。煬帝嗣位,除給事黃門侍郎,進位銀青光祿大夫,俄遷御史大夫,甚見親重。
 大業三年,帝幸榆林郡,還至太原,謂衡曰:「朕欲過公宅,可為朕作主人。」衡於是馳至河內,與宗族具牛酒。帝上太行,開直道九十里,以抵其宅。帝悅其山泉,留宴三日,因謂衡曰:「往從先皇拜太山之始,途經洛陽,瞻望於此,深恨不得相過,不謂今日得諧宿願。」衡俯伏辭謝,奉斛上壽。帝益歡,賜其宅傍田三十頃,良馬一匹,金帶,縑彩六百段,衣一襲,御食器一具。衡固讓,帝曰:「天子所至稱幸者,蓋為此也,不足為辭。」衡復獻食於帝,帝令頒賜公卿,下至衛士,無不沾洽。衡以籓邸之舊,恩寵莫與為比,頗自驕貴。明年,帝幸汾陽宮,宴從官,特賜絹五百匹。



 時帝
 欲大汾陽宮,令衡與紀弘整具圖奏之。衡承間進諫曰:「比年勞役繁多,百姓疲敝,伏願留神,稍加折損。」帝意甚不平。後嘗目衡謂侍臣曰:「張衡自謂由其計畫,令我有天下也。」時齊王暕失愛於上,帝密令人求暕罪失。有人譖暕違制,將伊闕令皇甫詡從之汾陽宮。又錄前幸涿郡及祠恆岳時,父老謁見者衣冠多不整。



 帝譴衡以憲司皆不能舉正,出為榆林太守。明年,帝復幸汾陽宮,衡督役築樓煩城,因而謁帝。帝惡衡不損瘦,以為不念咎,因謂衡曰:「公甚肥澤,宜且還郡。」衡復之榆林。俄而敕衡督役江都宮。有人詣衡訟宮監者,衡不為理,還以訟書
 付監,其人大為監所困。禮部尚書楊玄感使至江都,其人詣玄感稱冤。玄感固以衡為不可。



 及與衡相見,未有所言,又先謂玄感曰:「薛道衡真為枉死。」玄感具上其事,江都丞王世充又奏衡頻減頓具。帝於是發怒,鎖衡詣江都市,將斬之,久而乃釋,除名為民,放還田里。帝每令親人覘衡所為。八年,帝自遼東還都,衡妾言衡怨望,謗訕朝政,竟賜盡於家。臨死大言曰:「我為人作何物事,而望久活!」監刑者塞耳,促令殺之。義寧中,以死非其罪,贈大將軍、南陽郡公,謚曰忠。有子希玄。



 楊汪
 楊汪,字元度,本弘農華陰人也,曾祖順,徙居河東。父琛,儀同三司,及汪貴,追贈平鄉縣公。汪少兇疏,好與人群鬥,拳所毆擊,無不顛踣。長更折節勤學,專精《左氏傳》,通《三禮》。解褐周冀王侍讀,王甚重之,每曰:「楊侍讀德業優深,孤之穆生也。」其後問《禮》於沈重,受《漢書》於劉臻,二人推許之曰:「吾弗如也。」由是知名,累遷夏官府都上士。及高祖居相,引知兵事,遷掌朝下大夫。高祖受禪,賜爵平鄉縣伯,邑二百戶。歷尚書司勛兵部二曹侍郎、秦州總管長史,名為明幹。遷尚書左丞,坐事免。後歷荊、洛二州長史,每聽政之暇,必延生徒講授,時人稱之。數年,高祖
 謂諫議大夫王達曰:「卿為我覓一好左丞。」達遂私於汪曰:「我當薦君為左丞,若事果,當以良田相報也。」汪以達所言奏之,達竟以獲罪,卒拜汪為尚書左丞。汪明習法令,果於剖斷,當時號為稱職。煬帝即位,守大理卿。汪視事二日,帝將親省囚徒。其時系囚二百餘人,汪通宵究審,詰朝而奏,曲盡事情,一無遺誤,帝甚嘉之。歲餘,拜國子祭酒。帝令百僚就學,與汪講論,天下通儒碩學多萃焉,論難鋒起,皆不能屈。帝令御史書其問答奏之,省而大悅,賜良馬一匹。大業中,為銀青光祿大夫。及楊玄感反河南,贊治裴弘策出師御之,戰不利,弘策出還,遇汪
 而屏人交語。既而留守樊子蓋斬弘策,以狀奏汪,帝疑之,出為梁郡通守。後李密已逼東都,其徒頻寇梁郡,汪勒兵拒之,頻挫其銳。



 煬帝崩,王世充推越王侗為主,徵拜吏部尚書,頗見親委。及世充僭號,汪復用事,世充平,以兇黨誅死。



 史臣曰:盧愷諫說可稱,令狐熙所居而治,薛胄執憲平允,宇文弼聲望攸歸,張衡以鯁正立名,楊汪以學業自許。然皆有善始,鮮克令終,九仞之基,俱傾於一匱,惜哉!夫忠為令德,施非其人尚或不可,況托足邪徑,而又不得其人者歟!語曰:「無為權首,將受其咎。」又曰:「無始禍,無
 召亂。」張衡既召亂源,實為權首,動不以順,其能不及於此乎?



\end{pinyinscope}