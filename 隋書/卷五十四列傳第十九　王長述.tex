\article{卷五十四列傳第十九 王長述}

\begin{pinyinscope}

 王長述,京兆霸城人也。祖羆,魏太尉。父慶遠,周淮州刺史。長述幼有儀範,年八歲,周太祖見而異之,曰:「王公有此孫,足為不朽。」解褐員外散騎侍郎,封長安縣伯。累遷撫軍將軍、銀青光祿大夫、太子舍人。長述早孤,少為祖羆所養,及羆薨,居喪過禮,有詔褒異之。免喪,襲封扶風郡公,邑三千戶。除中書舍人,修起居注,改封龍門郡公。
 從於謹平江陵有功,增邑五百戶。周受禪,又增邑通前四千七百戶。拜賓部大夫。出為晉州刺史,轉玉壁總管長史。尋授司憲大夫,出拜廣州刺史。甚有威惠,吏人懷之,在任數年,蠻夷歸之者三萬餘戶。朝議嘉之,就拜大將軍。後歷襄、仁二州總管,並有能名。及高祖為丞相,授信州總管,部內夷、獠猶有未賓,長述討平之,進位上大將軍。王謙作亂益州,遣使致書於長述,因執其使,上其書,又陳取謙之策。上大悅,前後賜黃金五百兩,授行軍總管,率眾討謙。以功進位柱國。開皇初,復獻平陳之計,修營戰艦,為上流之師。上善其能,頻加賞勞,下書曰:「每
 覽高策,深相嘉嘆,命將之日,當以公為元帥也。」後數歲,以行軍總管擊南寧,未至,道病卒。上甚傷惜之,令使者吊祭,贈上柱國、冀州刺史,謚曰莊。子謨嗣。謨弟軌,大業末,東郡通守。少子文楷,起部郎。



 李衍李衍,字拔豆,遼東襄平人也。父弼,周太師。衍少專武藝,慷慨有志略。周太祖時,釋褐千牛備身,封懷仁縣公。加開府,改封普寧縣公,遷義州刺史。尋從韋孝寬鎮玉壁城,數與賊戰,敵人憚之。及平齊,以軍功進授大將軍,改封真鄉郡公,拜左宮伯,賜雜彩三百匹,奴婢二十口,賜
 子仲威爵浮陽郡公。後歷定、鄜二州刺史。及王謙作亂,高祖以衍為行軍總管,從梁睿擊平之。進位上大將軍,賜縑二千匹。開皇元年,又以行軍總管討叛蠻,平之。進位柱國,賜帛二千匹。尋檢校利州總管事。明年,突厥犯塞,以行軍總管率眾討之,不見虜而還。轉介州刺史。



 後數年,朝廷將有事江南,詔衍於襄州道營戰船。及大舉伐陳,授行軍總管,從秦王俊出襄陽道,以功賜帛三千匹,米六百石。拜安州總管,頗有惠政,歲餘,以疾還京師,卒於家,時年五十七。子仲威嗣。



 衍弟子長雅,尚高祖女襄國公主,襲父綸爵,為河陽郡公。開皇初,拜將軍、散騎
 常侍,歷內史侍郎、河州刺史、檢校秦州總管。



 衍從孫密,別有傳。



 伊婁謙伊婁謙,字彥恭,本鮮卑人也。其先代為酋長,隨魏南遷。祖信,中部太守。



 父靈,相、隆二州刺史。謙性忠直,善辭令。仕魏為直閣將軍。周受禪,累遷宣納上士,使持節、車騎大將軍。武帝將伐齊,引入內殿,從容謂曰:「朕將有事戎馬,何者為先?」謙對曰:「愚臣誠不足以知大事,但偽齊僭擅,跋扈不恭,沈溺倡優,耽昏曲蘗。其折沖之將斛律明月已斃,讒人之口,上下離心,道路仄目。若命六師,臣之
 願也。」帝大笑,因使謙與小司寇拓拔偉聘齊觀釁。帝尋發兵。齊主知之,令其僕射陽休之責謙曰:「貴朝盛夏徵兵,馬首何向?」謙答曰:「僕憑式之始,未聞興師。設復西增白帝之城,東益巴丘之戍,人情恆理,豈足怪哉!」謙參軍高遵以情輸於齊,遂拘留謙不遣。帝克並州,召謙勞之曰:「朕之舉兵,本俟卿還;不圖高遵中為叛逆,乖朕宿心,遵之罪也。」乃執遵付謙,任令報復。謙頓首請赦之,帝曰:「卿可聚眾唾面,令知愧也。」謙跪曰:「以遵之罪,又非唾面之責。」帝善其言而止。謙竟待遵如初。其寬厚仁恕,皆此類也。尋賜爵濟陽縣伯,累遷前驅中大夫。大象中,進爵
 為侯,加位開府。高祖作相,授亳州總管,俄征還京。既平王謙,謙恥與逆人同名,因爾稱字。高祖受禪,以彥恭為左武候將軍,俄拜大將軍,進爵為公。數年,出為澤州刺史,清約自處,甚得人和。以疾去職,吏人攀戀,行數百里不絕。數歲,卒於家,時年七十。子傑嗣。



 田仁恭田仁恭,字長貴,平涼長城人也。父弘,周大司空。仁恭性寬仁,有局度。在周以明經為掌式中士。後以父軍功賜爵鶉陰子。大塚宰宇文護引為中外兵曹。後數載,復以父功拜開府儀同三司,遷中外府掾。從護征伐,數有戰
 功,改封襄武縣公,邑五百戶。從武帝平齊,加授上開府,進封淅陽郡公,增邑二千戶,拜幽州總管。



 宣帝時,進爵雁門郡公。高祖為丞相,徵拜小司馬,進位大將軍。從韋孝寬破尉遲迥於相州,拜柱國。高祖受禪,進上柱國,拜太子太師,甚見親重,嘗幸其第,宴飲極歡,禮賜殊厚。奉詔營廟社,進爵觀國公,增邑通前五千戶。未幾,拜右武衛大將軍。歲餘,卒官,時年四十七。贈司空,謚曰敬。子世師嗣。次子德懋,在《孝義傳》。



 時有任城郡公王景、鮮虞縣公謝慶恩,並官至上柱國。大義公辛遵及其弟韶,並官至柱國。高祖以其俱佐命功臣,特加崇貴,親禮與仁恭
 等。事皆亡失雲。



 元亨元亨,字德良,一名孝才,河南洛陽人也。父季海,魏司徒、馮翊王,遇周、齊分隔,季海遂仕長安。亨時年數歲,與母李氏在洛陽。齊神武帝以亨父在關西,禁錮之。其母則魏司空李沖之女也,素有智謀,遂詐稱凍餒,請就食於滎陽。齊人以其去關西尚遠,老婦弱子,不以為疑,遂許之。李氏陰托大豪李長壽,攜亨及孤侄八人,潛行草間,得至長安。周太祖見而大悅,以亨功臣子,甚優禮之。亨年十二,魏恭帝在儲宮,引為交友。釋褐千牛備身。大統
 末,襲爵馮翊王,邑千戶。授拜之日,悲慟不能自勝。俄遷通直散騎常侍,歷武衛將軍、勛州刺史,改封平涼王。



 周閔帝受禪,例降為公。明、武時,歷隴州刺史、御正大夫、小司馬。宣帝時,為洛州刺史。高祖為丞相,遇尉遲迥作亂,洛陽人梁康、邢流水等舉兵應迥。旬日之間,眾至萬餘。州治中王文舒潛與梁康相結,將圖亨。亨陰知其謀,乃選關中兵,得二千人為左右,執文舒斬之,以兵襲擊梁康、邢流水,皆破之。高祖受禪,徵拜太常卿,增邑七百戶。尋出為衛州刺史,加大將軍。衛土俗薄,亨以威嚴鎮之,在職八年,風化大洽。後以老病,表乞骸骨,吏人詣闕上
 表,請留臥治,上嗟嘆者久之。其年,亨以篤疾,重請還京,上令使者致醫藥,問動靜,相望於道。歲餘,卒於家,時年六十九。謚曰宣。



 杜整杜整,字皇育,京兆杜陵人也。祖盛,魏直閣將軍、潁川太守。父闢,渭州刺史。整少有風概,九歲丁父憂,哀毀骨立,事母以孝聞。及長,驍勇有膂力,好讀孫、吳《兵法》。魏大統末,襲爵武鄉侯。周太祖引為親信。後事宇文護子中山公訓,甚被親遇。俄授都督。明帝時,為內侍上士,累遷儀同三司,拜武州刺史。從武帝平齊,加上儀同,進爵平原
 縣公,邑千戶,入為勛曹中大夫。高祖為丞相,進位開府。及受禪,加上開府,進封長廣郡公,俄拜左武衛將軍。在職數年,以母憂去職,起令視事。開皇六年,突厥犯塞,詔遣衛王爽總戎北伐,以整為行軍總管兼元帥長史。至合川,無虜而還。整密進取陳之策,上善之,於是以行軍總管鎮襄陽。



 尋病卒,時年五十五。高祖聞而傷之,贈帛四百匹,米四百石,謚曰襄。子楷嗣。



 官至開府。



 整弟肅,亦少有志行。開皇初,為通直散騎常侍、北地太守。



 李徹李徹,字廣達,朔方巖綠人也。父和,開皇初為柱國。徹性
 剛毅,有器幹,偉容儀,多武藝。大塚宰宇文護引為親信,尋拜殿中司馬,累遷奉車都尉。護以徹謹厚有才具,甚禮之。護子中山公訓為蒲州刺史,護令徹以本官從焉。未幾,拜車騎大將軍、儀同三司。武帝時,從皇太子西征吐谷渾,以功賜爵同昌縣男,邑三百戶。



 後從帝拔晉州。及帝班師,徹與齊王憲屯雞棲原。齊主高緯以大軍至,憲引兵西上,以避其鋒。緯遣其驍將賀蘭豹子率勁騎躡憲,戰於晉州城北。憲師敗,徹與楊素、宇文慶等力戰,憲軍賴以獲全。復從帝破齊師於汾北,乘勝下高壁,拔晉陽,擒高湝於冀州,俱有力焉。錄前後功,加開府,別封蔡
 陽縣公,邑千戶。宣帝即位,從韋孝寬略定淮南,每為先鋒。及淮南平,即授淮州刺史,安集初附,甚得其歡心。



 高祖受禪,加上開府,轉雲州刺史。歲餘,徵為左武衛將軍。及晉王廣之鎮並州也,朝廷妙選正人有文武才幹者,為之僚佐。上以徹前代舊臣,數持軍旅,詔徹總晉王府軍事,進爵齊安郡公。時蜀王秀亦鎮益州,上謂侍臣曰:「安得文同王子相,武如李廣達者乎?」其見重如此。



 明年,突厥沙缽略可汗犯塞,上令衛王爽為元帥,率眾擊之,以徹為長史。遇虜於白道,行軍總管李充言於爽曰:「周、齊之世,有同戰國,中夏力分,其來久矣。突厥每侵邊,諸
 將輒以全軍為計,莫能死戰。由是突厥勝多敗少,所以每輕中國之師。今者沙缽略悉國內之眾,屯據要險,必輕我而無備,精兵襲之,可破也。」



 爽從之。諸將多以為疑,唯徹獎成其計,請與同行。遂與充率精騎五千,出其不意,掩擊大破之。沙缽略棄所服金甲,潛草中而遁。以功加上大將軍。沙缽略因此屈膝稱籓。未幾,沙缽略為阿拔所侵,上疏請援。以徹為行軍總管,率精騎一萬赴之。



 阿拔聞而遁去。及軍還,復領行軍總管,屯平涼以備胡寇,封安道郡公。開皇十年,進位柱國。及晉王廣轉牧淮海,以徹為揚州總管司馬,改封德廣郡公。尋徙封城陽
 郡公。其後突厥犯塞,徹復領行軍總管擊破之。



 左僕射高熲之得罪也,以徹素與熲相善,因被疏忌,不復任使。後出怨言,上聞而召之,入臥內賜宴,言及平生,因遇鴆而卒。大業中,其妻宇文氏為孽子安遠誣以咒詛,伏誅。



 崔彭崔彭,字子彭,博陵安平人也。祖楷,魏殷州刺史。父謙,周荊州總管。彭少孤,事母以孝聞。性剛毅,有武略,工騎射。善《周官》、《尚書》,略通大義。



 周武帝時,為侍伯上士,累轉門正上士。及高祖為丞相,周陳王純鎮齊州,高祖恐純為變,遣彭以兩騎征純入朝。彭未至齊州三十里,因詐病,
 止傳舍,遣人謂純曰:「天子有詔書至王所,彭苦疾,不能強步,願王降臨之。」純疑有變,多將從騎至彭所。彭出傳舍迎之,察純有疑色,恐不就徵,因詐純曰:「王可避人,將密有所道。」純麾從騎,彭又曰:「將宣詔,王可下馬。」純遽下,彭顧其騎士曰:「陳王不從詔征,可執也。」騎士因執而鎖之。彭乃大言曰:「陳王有罪,詔徵入朝,左右不得輒動。」其從者愕然而去。高祖見而大悅,拜上儀同。及踐阼,遷監門郎將,兼領右衛長史,賜爵安陽縣男。數歲,轉車騎將軍,俄轉驃騎,恆典宿衛。性謹密,在省闥二十餘年,每當上在仗,危坐終日,未嘗有怠惰之容,上甚嘉之。上每謂
 彭曰:「卿當上日,我寢處自安。」又嘗曰:「卿弓馬固以絕人,頗知學不?」



 彭曰:「臣少愛《周禮》、《尚書》,每於休沐之暇,不敢廢也。」上曰:「試為我言之。」彭因說君臣戒慎之義,上稱善。觀者以為知言。後加上開府,遷備身將軍。



 上嘗宴達頭可汗使者於武德殿,有鴿鳴於梁上。上命彭射之,既發而中。上大悅,賜錢一萬。及使者反,可汗復遣使於上曰:「請得崔將軍一與相見。」上曰:「此必善射聞於虜庭,所以來請耳。」遂遣之。及至匈奴中,可汗召善射者數十人,因擲肉於野,以集飛鳶,遣其善射者射之,多不中。復請彭射之,彭連發數矢,皆應弦而落,突厥相顧,莫不嘆服。可
 汗留彭不遣百餘日,上賂以繒彩,然後得歸。



 仁壽末,進爵安陽縣公,邑二千戶。



 煬帝即位,遷左領軍大將軍。從幸洛陽,彭督後軍。時漢王諒初平,餘黨往往屯聚,令彭率眾數萬鎮遏山東,復領慈州事。帝以其清,賜絹五百匹。未幾而卒,時年六十三。帝遣使吊祭,贈大將軍,謚曰肅。子寶德嗣。



 史臣曰:王長述等,或出總方岳,或入司禁旅,咸著聲績,以功名終,有以取之也。伊婁謙志量弘遠,不念舊惡,請赦高遵之罪,有國士之風焉。崔彭巡警巖廊,毅然難犯,禦侮之寄,有足稱乎!



\end{pinyinscope}