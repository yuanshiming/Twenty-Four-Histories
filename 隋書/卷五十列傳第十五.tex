\article{卷五十列傳第十五}

\begin{pinyinscope}

 宇文慶宇文慶,字神慶,河南洛陽人也。祖金殿,魏征南大將軍,仕歷五州刺史、安吉侯。父顯和,夏州刺史。慶沉深有器局,少以聰敏見知。周初,受業東觀,頗涉經史。既而謂人曰:「書足記姓名而已,安能久事筆硯,為腐儒之業!」於時文州民夷相聚為亂,慶應募從征。賊據保巖谷,徑路懸絕,慶束馬而進,襲破之,以功授都督。衛王直之鎮山南
 也,引為左右。慶善射,有膽氣,好格猛獸,直甚壯之。



 稍遷車騎大將軍、儀同三司、柱國府掾。及誅宇文護,慶有謀焉,進授驃騎大將軍,加開府。後從武帝攻河陰,先登攀堞,與賊短兵接戰,良久,中石乃墜,絕而後蘇。



 帝勞之曰:「卿之餘勇,可以賈人也。」復從武帝拔晉州。其後齊師大至,慶與宇文憲輕騎而覘,卒與賊相遇,為賊所窘。憲挺身而遁,慶退據汾橋,眾賊爭進,慶引弓射之,所中人馬必倒,賊乃稍卻。及破高緯,拔高壁,克並州,下信都,禽高湝,功並居最。周武帝詔曰:「慶勛庸早著,英望華遠,出內之績,簡在朕心。戎車自西,俱總行陣,東夏蕩定,實有茂
 功。高位縟禮,宜崇榮冊。」於是進位大將軍,封汝南郡公,邑千六百戶。尋以行軍總管擊延安反胡,平之,拜延州總管。俄轉寧州總管。高祖為丞相,復以行軍總管南征江表。師次白帝,徵還,以勞進位上大將軍。高祖與慶有舊,甚見親待,令督丞相軍事,委以心腹。尋加柱國。開皇初,拜左武衛將軍,進位上柱國。數年,出除涼州總管。歲餘,徵還,不任以職。



 初,上潛龍時,嘗從容與慶言及天下事,上謂慶曰:「天元實無積德,視其相貌,壽亦不長。加以法令繁苛,耽恣聲色,以吾觀之,殆將不久。又復諸侯微弱,各令就國,曾無深根固本之計。羽翮既剪,何能及遠
 哉!尉迥貴戚,早著聲望,國家有釁,必為亂階。然智量庸淺,子弟輕佻,貪而少惠,終致亡滅。司馬消難反覆之虜,亦非池內之物,變成俄頃,但輕薄無謀,未能為害,不過自竄江南耳。庸、蜀險隘,易生艱阻,王謙愚蠢,素無籌略,但恐為人所誤,不足為虞。」未幾,上言皆驗。及此,慶恐上遺忘,不復收用,欲見舊蒙恩顧,具錄前言為表而奏之曰:「臣聞智侔造化,二儀無以隱其靈;明同日月,萬象不能藏其狀。先天弗違,實聖人之體道;未萌見兆,諒達節之神機。伏惟陛下特挺生知,徇齊誕御,懷五岳其猶輕,吞八荒而不梗,蘊妙見於胸襟,運奇謨於掌握。臣以微
 賤,早逢天眷,不以庸下,親蒙推赤。所奉成規,纖毫弗舛,尋惟聖慮,妙出蓍龜,驗一人之慶有徵,實天子之言無戲。臣親聞親見,實榮實喜。」上省表大悅,下詔曰:「朕之與公,本來親密,懷抱委曲,無所不盡。話言歲久,尚能記憶,今覽表奏,方悟昔談。何謂此言,遂成實錄。古人之先知禍福,明可信也,朕言之驗,自是偶然。公乃不忘,彌表誠節,深感至意,嘉尚無已。」自是上每加優禮。卒於家。



 子靜禮,初為太子千牛備身,尋尚高祖女廣平公主,授儀同,安德縣公,邑千五百戶,後為熊州刺史。先慶卒。



 子協,歷武賁郎將、右翊衛將軍,宇文化及之亂遇害。



 協弟皛,字
 婆羅門,大業之世,少養宮中。後為千牛左右,煬帝甚親暱之。每有游宴,皛必侍從,至於出入臥內,伺察六宮,往來不限門禁,其恩幸如此。時人號曰宇文三郎。皛與宮人淫亂,至於妃嬪公主,亦有醜聲。蕭後言於帝,皛聞而懼,數日不敢見。其兄協因奏曰:「皛今已壯,不可在宮掖。」帝曰:「皛安在?」協曰:「在朝堂。」帝不之罪,因召入,待之如初。宇文化及弒逆之際,皛時在玄覽門,覺變,將入奏,為門司所遏,不得時進。會日瞑,宮門閉,退還所守。俄而難作,皛與五十人赴之,為亂兵所害。



 李禮成
 李禮成,字孝諧,隴西狄道人也。涼王暠之六世孫。祖延實,魏司徒。父彧,侍中。禮成年七歲,與姑之子蘭陵太守滎陽鄭顥隨魏武帝入關。顥母每謂所親曰:「此兒平生未嘗回顧,當為重器耳。」及長,沉深有行檢,不妄通賓客。魏大統中,釋褐著作郎,遷太子洗馬、員外散騎常侍。周受禪,拜平東將軍、散騎常侍。於時貴公子皆競習弓馬,被服多為軍容。禮成雖善騎射,而從容儒服,不失素望。後以軍功拜車騎大將軍、儀同三司,賜爵修陽縣侯,拜遷州刺史。時朝廷有所徵發,禮成度以蠻夷不可擾,擾必為亂,上表固諫。周武帝從之。伐齊之役,從帝圍晉陽,
 禮成以兵擊南門,齊將席毗羅率精甲數千拒帝,禮成力戰,擊退之。加開府,進封冠軍縣公,拜北徐州刺史。未幾,徵為民部中大夫。



 禮成妻竇氏早沒,知高祖有非常之表,遂聘高祖妹為繼室,情契甚歡。及高祖為丞相,進位上大將軍,遷司武上大夫,委以心膂。及受禪,拜陜州刺史,進封絳郡公,賞賜優洽。尋徵為左衛將軍,遷右武衛大將軍。歲餘,出拜襄州總管,稱有惠政。後數載,復為左衛大將軍。時突厥屢為寇患,緣邊要害,多委重臣,由是拜寧州刺史。歲餘,以疾徵還京師,終於家。其子世師,官至度支侍郎。



 元孝矩弟褒元孝矩,河南洛陽人也。祖修義,父子均,並為魏尚書僕射。孝矩西魏時襲爵始平縣公,拜南豐州刺史。時見周太祖專政,將危元氏,孝矩每慨然有興復社稷之志,陰謂昆季曰:「昔漢氏有諸呂之變,硃虛、東牟,卒安劉氏。今宇文之心,路人所見,顛而不扶,焉用宗子?盍將圖之?」為兄則所遏,孝矩乃止。其後周太祖為兄子晉公護娶孝矩妹為妻,情好甚密。及閔帝受禪,護總百揆,孝矩之寵益隆。



 及護誅,坐徙蜀。數戰,徵還京師,拜益州總管司馬,轉司憲大夫。



 高祖重其門地,娶其女為房陵王妃。及高
 祖為丞相,拜少塚宰,進位柱國,賜爵洵陽郡公。時房陵王鎮洛陽,及上受禪,立為皇太子,令孝矩代鎮。既而立其女為皇太子妃,親禮彌厚。俄拜壽州總管,賜孝矩璽書曰:「揚、越氛昆,侵軼邊鄙,爭桑興役,不識大猷。以公志存遠略,今故鎮邊服,懷柔以禮,稱朕意焉。」時陳將任蠻奴等屢寇江北,復以孝矩領行軍總管,屯兵於江上。後數載,自以年老,筋力漸衰,不堪軍旅,上表乞骸骨。轉涇州刺史,高祖下書曰:「知執謙捴,請歸初服。恭膺寶命,實賴元功,方欲委裘,寄以分陜,何容便請高蹈,獨為君子者乎!



 若以邊境務煩,即宜徙節涇郡,養德臥治也。」在州
 歲餘,卒官,年五十九。謚曰簡。子無竭嗣。



 孝矩兄子文鬱,見《誠節傳》。孝矩次弟雅,字孝方,有文武干用。開皇中,歷左領左右將軍、集沁二州刺史,封順陽郡公。季弟褒,最知名。



 褒字孝整,便弓馬,少有成人之量。年十歲而孤,為諸兄所鞠養。性友悌,善事諸兄。諸兄議欲別居,褒泣諫不得,家素富,多金寶,褒無所受,脫身而出,為州里所稱。及長,寬仁大度,涉獵書史。仕周,官至開府、北平縣公、趙州刺史。



 及高祖為丞相,從韋孝寬擊尉迥,以功超拜柱國,進封河間郡公,邑二千戶。開皇二年,拜安州總管。歲餘,徙
 原州總管。有商人為賊所劫,其人疑同宿者而執之,褒察其色冤而辭正,遂舍之。商人詣闕訟褒受金縱賊,上遣使窮治之。使者簿責褒曰:「何故利金而舍盜也?」褒便即引咎,初無異詞。使者與褒俱詣京師,遂坐免官。其盜尋發於他所,上謂褒曰:「公朝廷舊人,位望隆重,受金舍盜非善事,何至自誣也?」對曰:「臣受委一州,不能息盜賊,臣之罪一也。州民為人所謗,不付法司,懸即放免,臣之罪二也。牽率愚誠,無顧形跡,不恃文書約束,至令為物所疑,臣之罪三也。臣有三罪,何所逃責?臣又不言受賂,使者復將有所窮究,然則縲紲橫及良善,重臣之罪,是
 以自誣。」上嘆異之,稱為長者。十四年,以行軍總管屯兵備邊。遼東之役,復以行軍總管從漢王至柳城而還。仁壽初,嘉州夷、獠為寇,褒率步騎二萬擊平之。煬帝即位,拜齊州刺史,尋改為齊郡太守,吏民安之。



 及興遼東之役,郡官督事者前後相屬,有西曹掾當行,詐疾,褒詰之,掾理屈,褒杖之,掾遂大言曰:「我將詣行在所,欲有所告。」褒大怒,因杖百餘,數日而死,坐是免官。卒於家,時年七十三。



 郭榮郭榮,字長榮,自云太原人也。父徽,魏大統末為同州司
 馬。時武元皇帝為刺史,由是與高祖有舊。徽後官至洵州刺史、安城縣公。及高祖受禪,拜太僕卿,數年,卒官。榮容貌魁岸,外疏內密,與其交者多愛之。周大塚宰宇文護引為親信。



 護察榮謹厚,擢為中外府水曹參軍。時齊寇屢侵,護令榮於汾州觀賊形勢。時汾州與姚襄鎮相去懸遠,榮以為二城孤迥,勢不相救,請於州鎮之間更築一城,以相控攝,護從之。俄而齊將段孝先攻陷姚襄、汾州二城,唯榮所立者獨能自守。護作浮橋,出兵渡河,與孝先戰。孝先於上流縱大筏以擊浮橋,護令榮督便水者引取其筏。



 以功授大都督。護又以稽胡數為寇亂,
 使榮綏集之。榮於上郡、延安築周昌、弘信、廣安、招遠、咸寧等五城,以遏其要路,稽胡由是不能為寇。武帝親總萬機,拜宣納中士。後從帝平齊,以戰功,賜馬二十匹,綿絹六百段,封平陽縣男,遷司水大夫。



 榮少與高祖親狎,情契極歡,嘗與高祖夜坐月下,因從容謂榮曰:「吾仰觀玄象,俯察人事,周歷已盡,我其代之。」榮深自結納。宣帝崩,高祖總百揆,召榮,撫其背而笑曰:「吾言驗未?」即拜相府樂曹參軍。俄以本官復領蕃部大夫。高祖受禪,引為內史舍人,以龍潛之舊,進爵蒲城郡公,加位上儀同。累遷通州刺史。



 仁壽初,西南夷、獠多叛,詔榮領八州諸軍
 事行軍總管,率兵討之。歲餘悉平,賜奴婢三百餘口。



 煬帝即位,入為武候驃騎將軍,以嚴正聞。後數歲,黔安首領田羅駒阻清江作亂,夷陵諸郡,民夷多應者,詔榮擊平之。遷左候衛將軍。從帝西征吐谷渾,拜銀青光祿大夫。遼東之役,以功進位左光祿大夫。明年,帝復事遼東,榮以為中國疲敝,萬乘不宜屢動,乃言於帝曰:「戎狄失禮,臣下之事。臣聞千鈞之弩不為鼷鼠發機,豈有親辱大駕以臨小寇?」帝不納。復從軍攻遼東城,榮親蒙矢石,晝夜不釋甲胄百餘日。帝每令人窺諸將所為,知榮如是,帝大悅,每勞勉之。九年,帝至東都,謂榮曰:「公年德漸
 高,不宜久涉行陣,當與公一郡,任所選也。」榮不願違離,頓首陳讓,辭情哀苦,有感帝心,於是拜為右候衛大將軍。後數日,帝謂百僚曰:「誠心純至如郭榮者,固無比矣。」其見信如此。楊玄感之亂,帝令馳守太原。明年,復從帝至柳城,遇疾,帝令存問動靜,中使相望。卒於懷遠鎮,時年六十八。帝為之廢朝,贈兵部尚書,謚曰恭,贈物千段。有子福善。



 龐晃龐晃,字元顯,榆林人也。父虯,周驃騎大將軍。晃少以良家子,刺史杜達召補州都督。周太祖既有關中,署晃大
 都督,領親信兵,常置左右。晃因徙居關中。



 後遷驃騎將軍,襲爵比陽侯。衛王直出鎮襄州,晃以本官從。尋與長湖公元定擊江南,孤軍深入,遂沒於陣。數年,衛王直遣晃弟車騎將軍元俊齎絹八百匹贖焉,乃得歸朝。拜上儀同,賜彩二百段,復事衛王。



 時高祖出為隨州刺史,路經襄陽,衛王令晃詣高祖。晃知高祖非常人,深自結納。及高祖去官歸京師,晃迎見高祖於襄邑。高祖甚歡,晃因白高祖曰:「公相貌非常,名在圖籙。九五之日,幸願不忘。」高祖笑曰:「何妄言也!」頃之,有一雄雉鳴於庭,高祖命晃射之,曰:「中則有賞。然富貴之日,持以為驗。」晃既射而
 中,高祖撫掌大笑曰:「此是天意,公能感之而中也。」因以二婢賜之,情契甚密。武帝時,晃為常山太守,高祖為定州總管,屢相往來。俄而高祖轉亳州總管,將行,意甚不悅。晃因白高祖曰:「燕、代精兵之處,今若動眾,天下不足圖也。」



 高祖握晃手曰:「時未可也。」晃亦轉為車騎將軍。及高祖為揚州總管,奏晃同行。



 既而高祖為丞相,進晃位開府,命督左右,甚見親待。及踐阼,謂晃曰:「射雉之符,今日驗不?」晃再拜曰:「陛下應天順民,君臨宇內,猶憶曩時之言,不勝慶躍。」上笑曰:「公之此言,何得忘也!」尋加上開府,拜右衛將軍,進爵為公,邑千五百戶。河間王弘之擊
 突厥也,晃以行軍總管從至馬邑。別路出賀蘭山,擊賊破之,斬首千餘級。



 晃性剛悍,時廣平王雄當途用事,勢傾朝廷,晃每陵侮之。嘗於軍中臥,見雄不起,雄甚銜之。復與高熲有隙,二人屢譖晃。由是宿衛十餘年,官不得進。出為懷州刺史,數歲,遷原州總管。仁壽中卒官,年七十二。高祖為之廢朝,贈物三百段,米三百石,謚曰敬。子長壽,頗知名,官至驃騎將軍。



 李安李安,字玄德,隴西狄道人也。父蔚,仕周為朔燕恆三州刺史、襄武縣公。安美姿儀,善騎射。周天和中,釋褐右侍
 上士,襲爵襄武公。俄授儀同、少師右上士。



 高祖作相,引之左右,遷職方中大夫。復拜安弟悊為儀同。安叔父梁州刺史璋,時在京師,與周趙王謀害高祖,誘悊為內應。悊謂安曰:「寢之則不忠,言之則不義,失忠與義,何以立身?」安曰:「丞相父也,其可背乎?」遂陰白之。及趙王等伏誅,將加官賞,安頓首而言曰:「兄弟無汗馬之勞,過蒙獎擢,合門竭節,無以酬謝。不意叔父無狀,為兇黨之所蠱惑,覆宗絕嗣,其甘若薺。蒙全首領,為幸實多,豈可將叔父之命以求官賞?」於是俯伏流涕,悲不自勝。高祖為之改容曰:「我為汝特存璋子。」乃命有司罪止璋身,高祖亦為
 安隱其事而不言。尋授安開府,進封趙郡公,悊上儀同、黃臺縣男。



 高祖即位,授安內史侍郎,轉尚書左丞、黃門侍郎。平陳之役,以為楊素司馬,仍領行軍總管,率蜀兵順流東下。時陳人屯白沙,安謂諸將曰:「水戰非北人所長。



 今陳人依險泊船,必輕我而無備。以夜襲之,賊可破也。」諸將以為然。安率眾先鋒,大破陳師。高祖嘉之,詔書勞曰:「陳賊之意,自言水戰為長,險隘之間,彌謂官軍所憚。開府親將所部,夜動舟師,摧破賊徒,生擒虜眾,益官軍之氣,破賊人之膽,副朕所委,聞以欣然。」進位上大將軍,除郢州刺史。數日,轉鄧州刺史。



 安請為內職,高祖重
 違其意,除左領左右將軍。俄遷右領軍大將軍,復拜悊開府儀同三司、備身將軍。兄弟俱典禁衛,恩信甚重。八年,突厥犯塞,以安為行軍總管,從楊素擊之。安別出長川,會虜渡河,與戰破之。仁壽元年,出安為寧州刺史,悊為衛州刺史。安子瓊,悊子瑋,始自襁褓,乳養宮中,至是年八九歲,始命歸家。



 其見親顧如是。



 高祖嘗言及作相時事,因愍安兄弟滅親奉國,乃下詔曰:「先王立教,以義斷恩,割親愛之情,盡事君之道,用能弘獎大節,體此至公。往者周歷既窮,天命將及,朕登庸惟始,王業初基,承此澆季,實繁奸宄。上大將軍、寧州刺史、趙郡公李安,其
 叔璋潛結籓枝,扇惑猶子,包藏不逞,禍機將發。安與弟開府儀同三司、衛州刺史、黃臺縣男悊,深知逆順,披露丹心,兇謀既彰,罪人斯得。朕每念誠節,嘉之無已,懋庸冊賞,宜不逾時。但以事涉其親,猶有疑惑,欲使安等名教之方,自處有地,朕常為思審,遂致淹年。今更詳按聖典,求諸往事,父子天性,誠孝猶不並立,況復叔侄恩輕,情禮本有差降,忘私奉國,深得正理,宜錄舊勛,重弘賞命。」於是拜安、悊俱為柱國,賜縑各五千匹,馬百匹,羊千口。復以悊為備身將軍,進封順陽郡公。安謂親族曰:「雖家門獲全,而叔父遭禍,今奉此詔,悲愧交懷。」因歔欷悲
 感,不能自勝。先患水病,於是疾甚而卒,時年五十三。謚曰懷。



 子瓊嗣。少子孝恭,最有名。悊後坐事除名,配防嶺南,道病卒。



 史臣曰:宇文慶等,龍潛惟舊,疇昔親姻,或素盡平生之言,或早有腹心之托。



 沾雲雨之餘潤,照日月之末光,騁步天衢,與時升降。高位厚秩,貽厥後昆,優矣!



 皛幼養宮中,未聞教義,煬帝愛之不以禮,其能不及於此乎?安、悊之於高祖,未有君臣之分,陷其骨肉,使就誅夷,大義滅親,所聞異於此矣。雖有悲悼,何損於侃。



\end{pinyinscope}