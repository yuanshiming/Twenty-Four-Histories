\article{卷四十五列傳第十 文四子}

\begin{pinyinscope}

 高祖五男,皆文獻皇后之所生也。長曰房陵王勇,次煬帝,次秦孝王俊,次庶人秀,次庶人諒。



 房陵王勇,字睍地伐,高祖長子也。周世,以太祖軍功封博平侯。及高祖輔政,立為世子,拜大將軍、左司衛,封長寧郡公。出為洛州總管、東京小塚宰,總統舊齊之地。後徵還京師,進位上柱國、大司馬,領內史御正,諸禁衛皆
 屬焉。高祖受禪,立為皇太子,軍國政事及尚書奏死罪已下,皆令勇參決之。上以山東民多流冗,遣使按檢,又欲徙民北實邊塞。勇上書諫曰:「竊以導俗當漸,非可頓革,戀土懷舊,民之本情,波迸流離,蓋不獲已。有齊之末,主暗時昏,周平東夏,繼以威虐,民不堪命,致有逃亡,非厭家鄉,願為羈旅。加以去年三方逆亂,賴陛下仁聖,區宇肅清,鋒刃雖屏,瘡痍未復。若假以數歲,沐浴皇風,逃竄之徒,自然歸本。雖北夷猖獗,嘗犯邊烽,今城鎮峻峙,所在嚴固,何待遷配,以致勞擾。臣以庸虛,謬當儲貳,寸誠管見,輒以塵聞。」上覽而嘉之,遂寢其事。是後時政不
 便,多所損益,上每納之。上嘗從容謂群臣曰:「前世皇王,溺於嬖幸,廢立之所由生。朕傍無姬侍,五子同母,可謂真兄弟也。豈若前代多諸內寵,孽子忿諍,為亡國之道邪!」



 勇頗好學,解屬詞賦,性寬仁和厚,率意任情,無矯飾之行。引明克讓、姚察、陸開明等為之賓友。勇嘗文飾蜀鎧,上見而不悅,恐致奢侈之漸,因而誡之曰:「我聞天道無親,唯德是與,歷觀前代帝王,未有奢華而得長久者。汝當儲後,若不上稱天心,下合人意,何以承宗廟之重,居兆民之上?吾昔日衣服,各留一物,時復看之,以自警戒。今以刀子賜汝,宜識我心。」



 其後經冬至,百官朝勇,勇
 張樂受賀。高祖知之,問朝臣曰:「近聞至節,內外百官相率朝東宮,是何禮也?」太常少卿辛亶對曰:「於東宮是賀,不得言朝。」



 高祖曰:「改節稱賀,正可三數十人,逐情各去。何因有司徵召,一時普集,太子法服設樂以待之?東宮如此,殊乖禮制。」於是下詔曰:「禮有等差,君臣不雜,爰自近代,聖教漸虧,俯仰逐情,因循成俗。皇太子雖居上嗣,義兼臣子,而諸方岳牧,正冬朝賀,任土作貢,別上東宮,事非典則,宜悉停斷。」自此恩寵始衰,漸生疑阻。時高祖令選宗衛侍官,以入上臺宿衛。高熲奏稱,若盡取強者,恐東宮宿衛太劣。高祖作色曰:「我有時行動,宿衛須得
 雄毅。太子毓德東宮,左右何須強武?此極敝法,甚非我意。如我商量,恆於交番之日,分向東宮上下,團伍不別,豈非好事?我熟見前代,公不須仍踵舊風。」蓋疑高熲男尚勇女,形於此言,以防之也。



 勇多內寵,昭訓云氏,尤稱嬖幸,禮匹於嫡。勇妃元氏無寵,嘗遇心疾,二日而薨。獻皇后意有他故,甚責望勇。自是云昭訓專擅內政,後彌不平,頗遣人伺察,求勇罪過。晉王知之,彌自矯飾,姬妾但備員數,唯共蕭妃居處。皇后由是薄勇,愈稱晉王德行。其後晉王來朝,車馬侍從,皆為儉素,敬接朝臣,禮極卑屈,聲名籍甚,冠於諸王。臨還揚州,入內辭皇后,因進
 言曰:「臣鎮守有限,方違顏色,臣子之戀,實結於心。一辭階闥,無由侍奉,拜見之期,杳然未日。」因哽咽流涕,伏不能興。皇后亦曰:「汝在方鎮,我又年老,今者之別,有切常離。」又泫然泣下,相對歔欷。王曰:「臣性識愚下,常守平生昆弟之意,不知何罪,失愛東宮,恆蓄盛怒,欲加屠陷。每恐讒譖生於投杼,鴆毒遇於杯勺,是用勤憂積念,懼履危亡。」皇后忿然曰:「睍地伐漸不可耐,我為伊索得元家女,望隆基業,竟不聞作夫妻,專寵阿云,使有如許豚犬。前新婦本無病痛,忽爾暴亡,遣人投藥,致此夭逝。事已如是,我亦不能窮治,何因復於汝處發如此意?我在尚
 爾,我死後,當魚肉汝乎?每思東宮竟無正嫡,至尊千秋萬歲之後,遣汝等兄弟向阿雲兒前再拜問訊,此是幾許大苦痛邪!」晉王又拜,嗚咽不能止,皇后亦悲不自勝。此別之後,知皇后意移,始構奪宗之計。因引張衡定策,遣褒公宇文述深交楊約,令喻旨於越國公素,具言皇后此語。素瞿然曰:「但不知皇后如何?必如所言,吾又何為者!」後數日,素入侍宴,微稱晉王孝悌恭儉,有類至尊,用此揣皇后意。皇后泣曰:「公言是也。我兒大孝順,每聞至尊及我遣內使到,必迎於境首。言及違離,未嘗不泣。



 又其新婦亦大可憐,我使婢去,常與之同寢共食。豈若
 睍地伐共阿云相對而坐,終日酣宴,暱近小人,疑阻骨肉。我所以益憐阿摐者,常恐暗地殺之。」素既知意,因盛言太子不才。皇后遂遺素金,始有廢立之意。



 勇頗知其謀,憂懼,計無所出。聞新豐人王輔賢能占候,召而問之。輔賢曰:「白虹貫東宮門,太白襲月,皇太子廢退之象也。」以銅鐵五兵造諸厭勝。又於後園之內作庶人村,屋宇卑陋,太子時於中寢息,布衣草褥,冀以當之。高祖知其不安,在仁壽宮,使楊素觀勇。素至東宮,偃息未入,勇束帶待之,故久不進,以激怒勇。勇銜之,形於言色。素還,言勇怨望,恐有他變,願深防察。高祖聞素譖毀,甚疑之。皇
 后又遣人伺覘東宮,纖介事皆聞奏,因加媒蘗,構成其罪。高祖惑於邪議,遂疏忌勇。乃於玄武門達至德門量置候人,以伺動靜,皆隨事奏聞。又東宮宿衛之人,侍官已上,名藉悉令屬諸衛府,有健兒者,咸屏去之。晉王又令段達私於東宮幸臣姬威,遺以財貨,令取太子消息,密告楊素。於是內外喧謗,過失日聞。



 段達脅姬威曰:「東宮罪過,主上皆知之矣,已奉密詔,定當廢立。君能靠之,則大富貴。」威遂許諾。



 九月壬子,車駕至自仁壽宮,翌日,御大興殿,謂侍臣曰:「我新還京師,應開懷歡樂,不知何意,翻邑然愁苦?」吏部尚書牛弘對曰:「由臣等不稱職,故
 至尊憂勞。」高祖既數聞讒譖,疑朝臣皆具委,故有斯問,冀聞太子之愆。弘為此對,大乖本旨。高祖因作色謂東宮官屬曰:「仁壽宮去此不遠,而令我每還京師,嚴備仗衛,如入敵國。我為患利,不脫衣臥。昨夜欲得近廁,故在後房,恐有警急,還移就前殿。豈非爾輩欲壞我國家邪?」於是執唐令則等數人,付所司訊鞫。令楊素陳東宮事狀,以告近臣。素顯言之曰:「臣奉敕向京,令皇太子檢校劉居士餘黨。



 太子奉詔,乃作色奮厲,骨肉飛騰,語臣云:『居士黨盡伏法,遣我何處窮討?爾作右僕射,委寄不輕,自檢校之,何關我事?』又云:『若大事不遂,我先被誅。



 今作
 天子,竟乃令我不如諸弟。一事以上,不得自由。』因長嘆回視云:『我大覺身妨。』」高祖曰:「此兒不堪承嗣久矣。皇后恆勸我廢之,我以布素時生,復是長子,望其漸改,隱忍至今。勇昔從南兗州來,語衛王云:「阿娘不與我一好婦女,亦是可恨。」因指皇后侍兒曰:「是皆我物。」此言幾許異事。其婦初亡,即以斗帳安餘老嫗。新婦初亡,我深疑使馬嗣明藥殺。我曾責之,便懟曰:「會殺元孝矩。」



 此欲害我而遷怒耳。初,長寧誕育,朕與皇后共抱養之,自懷彼此,連遣來索。且云定興女,在外私合而生,想此由來,何必是其體胤!昔晉太子取屠家女,其兒即好屠割。今儻非
 類,便亂宗社。又劉金驎諂佞人也,呼定興作親家翁,定興愚人,受其此語。我前解金驎者,為其此事。勇嘗引曹妙達共定興女同燕,妙達在外說云:『我今得勸妃酒。」直以其諸子偏庶,畏人不服,故逆縱之,欲收天下之望耳。我雖德慚堯、舜,終不以萬姓付不肖子也。我恆畏其加害,如防大敵,今欲廢之,以安天下。」



 左衛大將軍、五原公元旻諫曰:「廢立大事,天子無二言,詔旨若行,後悔無及。讒言罔極,惟陛下察之。」旻辭直爭強,聲色俱厲,上不答。



 是時姬威又抗表告太子非法。高祖謂威曰:「太子事跡,宜皆盡言。」威對曰:「皇太子由來共臣語,唯意在驕奢,欲
 得從樊川以至於散關,總規為苑。兼云:『昔漢武帝將起上林苑,東方朔諫之,賜朔黃金百斤,幾許可笑。我實無金輒賜此等。若有諫者,正當斬之,不過殺百許人,自然永息。』前蘇孝慈解左衛率,皇太子奮髯揚肘曰:『大丈夫會當有一日,終不忘之,決當快意。』又宮內所須,尚書多執法不與,便怒曰:『僕射以下,吾會戮一二人,使知慢我之禍。』又於苑內築一小城,春夏秋冬,作役不輟,營起亭殿,朝造夕改。每云:『至尊嗔我多側庶,高緯、陳叔寶豈是孽子乎?』嘗令師姥卜吉兇,語臣曰:『至尊忌在十八年,此期促矣。』」高祖泫然曰:「誰非父母生,乃至於此!我有舊使
 婦女,令看東宮,奏我云:『勿令廣平王至皇太子處。東宮憎婦,亦廣平教之。』元贊亦知其陰惡,勸我於左藏之東,加置兩隊。初平陳後,宮人好者悉配春坊,如聞不知厭足,於外更有求訪。朕近覽《齊書》,見高歡縱其兒子,不勝忿憤,安可效尤邪!」於是勇及諸子皆被禁錮,部分收其黨與。楊素舞文巧詆,鍛煉以成其獄。勇由是遂敗。



 居數日,有司承素意,奏言左衛元旻身備宿衛,常曲事於勇,情存附托,在仁壽宮,裴弘將勇書於朝堂與旻,題封云勿令人見。高祖曰:「朕在仁壽宮,有纖小事,東宮必知,疾於驛馬。怪之甚久,豈非此徒耶?」遣武士執旻及弘付法
 治其罪。



 先是,勇嘗從仁壽宮參起居還,途中見一枯槐,根幹蟠錯,大且五六圍,顧左右曰:「此堪作何器用?」或對曰:「古槐尤堪取火。」於時衛士皆佩火燧,勇因令匠者造數千枚,欲以分賜左右。至是,獲於庫。又藥藏局貯艾數斛,亦搜得之。



 大將為怪,以問姬威。威曰:「太子此意別有所在。比令長寧王已下,詣仁壽宮還,每嘗急行,一宿便至。恆飼馬千匹,雲徑往捉城門,自然餓死。」素以威言詰勇,勇不服曰:「竊聞公家馬數萬匹,勇忝備位太子,有馬千匹,乃是反乎?」素又發洩東宮服玩,似加周飾者,悉陳之於庭,以示文武群官,為太子之罪。高祖遣將諸物示
 勇,以誚詰之。皇后又責之罪。高祖使使責問勇,勇不服。太史令袁充進曰:「臣觀天文,皇太子當廢。」上曰:「玄象久見矣,群臣無敢言者。」於是使人召勇。勇見使者,驚曰:「得無殺我耶?」高祖戎服陳兵,御武德殿,集百官,立於東面,諸親立於西面,引勇及諸子列於殿庭。命薛道衡宣廢勇之詔曰:「太子之位,實為國本,茍非其人,不可虛立。自古儲副,或有不才,長惡不悛,仍令守器,皆由情溺寵愛,失於至理,致使宗社傾亡,蒼生塗地。由此言之,天下安危,系乎上嗣,大業傳世,豈不重哉!皇太子勇,地則居長,情所鐘愛,初登大位,即建春宮,冀德業日新,隆茲負荷。
 而性識庸暗,仁孝無聞,暱近小人,委任奸佞,前後愆釁,難以具紀。但百姓者,天之百姓,朕恭天命,屬當安育,雖欲愛子,實畏上靈,豈敢以不肖之子而亂天下。勇及其男女為王、公主者,並可廢為庶人。顧惟兆庶,事不獲已,嘆言及此,良深愧嘆!」令薛道衡謂勇曰:「爾之罪惡,人神所棄,欲求不廢,其可得耶?」勇再拜而言曰:「臣合尸之都市,為將來鑒誡,幸蒙哀憐,得全性命。」言畢,泣下流襟,既而舞蹈而去。左右莫不憫默。又下詔曰:自古以來,朝危國亂,皆邪臣佞媚,兇黨扇惑,致使禍及宗社,毒流兆庶。若不標明典憲,何以肅清天下!左衛大將軍、五原郡公
 元旻,任掌兵衛,委以心膂,陪侍左右,恩寵隆渥,乃包藏奸伏,離間君親,崇長厲階,最為魁首。太子左庶子唐令則,策名儲貳,位長宮僚,諂曲取容,音技自進,躬執樂器,親教內人,贊成驕侈,導引非法。太子家令鄒文騰,專行左道,偏被親暱,心腹委付,巨細關知,占問國家,希覬災禍。左衛率司馬夏侯福,內事諂諛,外作威勢,凌侮上下,褻濁宮闈。典膳監元淹,謬陳愛憎,開示怨隙,妄起訕謗,潛行離阻,進引妖巫,營事厭禱。前吏部侍郎蕭子寶,往居省閣,舊非宮臣,稟性浮躁,用懷輕險,進畫奸謀,要射榮利,經營間構,開造禍端。前主璽下士何竦,假托玄象,
 妄說妖怪,志圖禍亂,心在速發,兼制奇器異服,皆竦規摹,增長驕奢,糜費百姓。凡此七人,為害乃甚,並處斬,妻妾子孫皆悉沒官。車騎將軍閻毗、東郡公崔君綽、游騎尉沈福寶、瀛州民章仇太翼等四人,所為之事,皆是悖惡,論其狀跡,罪合極刑。但朕情存好生,未能盡戮,可並特免死,各決杖一百,身及妻子資財田宅,悉可沒官。副將作大匠高龍義,豫追番丁,輒配東宮使役,營造亭舍,進入春坊。率更令晉文建,通直散騎侍郎、判司農少卿事元衡,料度之外,私自出給,虛破丁功,擅割園地。並處盡。



 於是集群官於廣陽門外,宣詔以戮之。廣平王雄答
 詔曰:「至尊為百姓割骨肉之恩,廢黜無德,實為大慶,天下幸甚!」乃移勇於內史省,立晉王廣為皇太子,仍以勇付之,復囚於東宮。賜楊素物三千段,元胄、楊約並千段,楊難敵五百段,皆鞫勇之功賞也。



 時文林郎楊孝政上書諫曰:「皇太子為小人所誤,宜加訓誨,不宜廢黜。」上怒,撻其胸。尋而貝州長史裴肅表稱:「庶人罪黜已久,當克己自新,請封一小國。」



 高祖知勇之黜也,不允天下之情,乃徵肅入朝,具陳廢立之意。



 時勇自以廢非其罪,頻請見上,面申冤屈。而皇太子遏之,不得聞奏。勇於是升樹大叫,聲聞於上,冀得引見。素因奏言:「勇情志昏亂,為癲
 鬼所著,不可復收。」上以為然,卒不得見。素誣陷經營,構成其罪,類皆如此。



 高祖寢疾於仁壽宮,徵皇太子入侍醫藥,而奸亂宮闈,事聞於高祖。高祖抵床曰:「枉廢我兒!」因遣追勇。未及發使,高祖暴崩,秘不發喪。遽收柳述、元巖,系於大理獄,偽為高祖敕書,賜庶人死。追封房陵王,不為立嗣。



 勇有十男:云昭訓生長寧王儼、平原王裕、安城王筠,高良娣生安平王嶷、襄城王恪,王良媛生高陽王該、建安王韶,成姬生潁川王煚,後宮生孝實、孝範。



 長寧王儼,勇長子也。誕乳之初,以報高祖,高祖曰:「此即皇太孫,何乃生不得地?」云定興奏曰:「天生龍種,所以因
 云而出。」時人以為敏對。六歲,封長寧郡王。勇敗,亦坐廢黜。上表乞宿衛,辭情哀切,高祖覽而憫焉。楊素進曰:「伏願聖心同於螫手,不宜復留意。」煬帝踐極,儼常從行,卒於道,實鴆之也。



 諸弟分徙嶺外,仍敕在所皆殺焉。



 秦孝王俊,字阿祗,高祖第三子也。開皇元年立為秦王。二年春,拜上柱國、河南道行臺尚書令、洛州刺史,時年十二。加右武衛大將軍,領關東兵。三年,遷秦州總管。隴右諸州盡隸焉。俊仁恕慈愛,崇敬佛道,請為沙門,上不許。六年,遷山南道行臺尚書令。伐陳之役,以為山南道行軍元帥,督三十總管,水陸十餘萬,屯漢口,為上流節
 度。陳將周羅、荀法尚等,以勁兵數萬屯鸚鵡洲,總管崔弘度請擊之。俊慮殺傷,不許。羅亦相率而降。於是遣使奉章詣闕,垂泣謂使者曰:「謬當推轂,愧無尺寸之功,以此多慚耳。」上聞而善之。授揚州總管四十四州諸軍事,鎮廣陵。歲餘,轉並州總管二十四州諸軍事。初頗有令問,高祖聞而大悅,下書獎勵焉。其後俊漸奢侈,違犯制度,出錢求息,民吏苦之。上遣使按其事,與相連坐者百餘人。俊猶不悛,於是盛治宮室,窮極侈麗。俊有巧思,每親運斤斧,工巧之器,飾以珠玉。為妃作七寶」,又為水殿,香塗粉壁,玉砌金階。梁柱楣棟之間,周以明
 鏡,間以寶珠,極榮飾之美。每與賓客妓女弦歌於其上。俊頗好內,妃崔氏性妒,甚不平之,遂於瓜中進毒。俊由是遇疾,徵還京師。上以其奢縱,免官,以王就第。左武衛將軍劉升諫曰:「秦王非有他過,但費官物營舍而已。臣謂可容。」上曰:「法不可違。」升固諫,上忿然作色,升乃止。其後楊素復進諫曰:「秦王之過,不應至此,願陛下詳之。」上曰:「我是五兒之父,若如公意,何不別制天子兒律?以周公之為人,尚誅管、蔡,我誠不及周公遠矣,安能虧法乎?」



 卒不許。



 俊疾篤,未能起,遣使奉表陳謝。上謂其使曰:「我戮力關塞,創茲大業,作訓垂範,庶臣下守之而不失。
 汝為吾子,而欲敗之,不知何以責汝!」俊慚怖,疾甚。大都督皇甫統上表,請復王官,不許。歲餘,以疾篤,復拜上柱國。二十年六月,薨於秦邸。上哭之數聲而已。俊所為侈麗之物,悉命焚之。敕送終之具,務從儉約,以為後法也。王府僚佐請立碑,上曰:「欲求名,一卷史書足矣,何用碑為?



 若子孫不能保家,徒與人作鎮石耳。」



 妃崔氏以毒王之故,下詔廢絕,賜死於其家。子浩,崔氏所生也。庶子曰湛。



 群臣議曰:「《春秋》之義,母以子貴,子以母貴。貴既如此,罪則可知。故漢時慄姬有罪,其子便廢,郭后被廢,其子斯黜。大既然矣,小亦宜同。今秦王二子,母皆罪廢,不合
 承嗣。」於是以秦國官為喪主。俊長女永豐公主,年十二,遭父憂,哀慕盡禮,免喪,遂絕魚肉。每至忌日,輒流涕不食。有開府王延者,性忠厚,領親信兵十餘年,俊甚禮之。及俊有疾,延恆在閤下,衣不解帶。俊薨,勺飲不入口者數日,羸頓骨立。上聞而憫之,賜以御藥,授驃騎將軍,典宿衛。俊葬之日,延號慟而絕。上嗟異之,令通事舍人吊祭焉。詔葬延于俊墓側。



 煬帝即位,立浩為秦王,以奉孝王嗣。封湛為濟北侯。後以浩為河陽都尉。楊玄感作逆之際,左翊衛大將軍宇文述勒兵討之。至河陽,修啟於浩,浩復詣述營,兵相往復。有司劾浩,以諸侯交通內臣,
 竟坐廢免。宇文化及殺逆之始,立浩為帝。



 化及敗於黎陽,北走魏縣,自僭偽號,因而害之。湛驍果,有膽烈。大業初,為滎陽太守,坐浩免,亦為化及所害。



 庶人秀,高祖第四子也。開皇元年,立為越王。未幾,徙封於蜀,拜柱國、益州刺史、總管,二十四州諸軍事。二年,進位上柱國、西南道行臺尚書令,本官如故。歲餘而罷。十二年,又為內史令、右領軍大將軍。尋復出鎮於蜀。



 秀有膽氣,容貌瑰偉,美須髯,多武藝,甚為朝臣所憚。上每謂獻皇后曰:「秀必以惡終。我在當無慮,至兄弟必反。」兵部侍郎元衡使於蜀,秀深結於衡,以左右為請。既還京師,
 請益左右,上不許。大將軍劉噲之討西爨也,高祖令上開府楊武通將兵繼進。秀使嬖人萬智光為武通行軍司馬,上以秀任非其人,譴責之。



 因謂群臣曰:「壞我法者,必在子孫乎?譬如猛獸,物不能害,反為毛間蟲所損食耳。」於是遂分秀所統。



 秀漸奢侈,違犯制度,車馬被服,擬於天子。及太子勇以讒毀廢,晉王廣為皇太子,秀意甚不平。皇太子恐秀終為後變,陰令楊素求其罪而譖之。仁壽二年,徵還京師,上見,不與語。明日,使使切讓之。秀謝曰:「忝荷國恩,出臨籓岳,不能奉法,罪當萬死。」皇太子及諸王流涕庭謝。上曰:「頃者秦王糜費財物,我以父道
 訓之。今秀蠹害生民,當以君道繩之。」於是付執法者。開府慶整諫曰:「庶人勇既廢,秦王已薨,陛下兒子無多,何至如是?然蜀王性甚耿介,今被重責,恐不自全。」上大怒,欲斷其舌。因謂群臣曰:「當斬秀於市,以謝百姓。」乃令楊素、蘇威、牛弘、柳述、趙綽等推治之。太子陰作偶人,書上及漢王姓字,縛手釘心,令人埋之華山下,令楊素發之。又作檄文曰:「逆臣賊子,專弄威柄,陛下唯守虛器,一無所知。」陳甲兵之盛,云「指期問罪」。置秀集中,因以聞奏。上曰:「天下寧有是耶!」於是廢為庶人,幽內侍省,不得與妻子相見,令給獠婢二人驅使。與相連坐者百餘人。



 秀既
 幽逼,憤懣不知所為,乃上表曰:「臣以多幸,聯慶皇枝,蒙天慈鞠養,九歲榮貴,唯知富樂,未嘗憂懼。輕恣愚心,陷茲刑網,負深山嶽,甘心九泉。不謂天恩尚假餘漏,至如今者,方知愚心不可縱,國法不可犯,撫膺念咎,自新莫及。



 猶望分身竭命,少答慈造,但以靈祗不祜,福祿消盡,夫婦抱思,不相勝致。只恐長辭明世,永歸泉壤,伏願慈恩,賜垂矜愍,殘息未盡之間,希與爪子相見。請賜一穴,令骸骨有所。」爪子即其愛子也。上因下詔數其罪曰:汝地居臣子,情兼家國,庸、蜀要重,委以鎮之。汝乃干紀亂常,懷惡樂禍,闢睨二宮,佇遲災釁,容納不逞,結構異端。
 我有不和,汝便覘候,望我不起,便有異心。皇太子汝兄也,次當建立,汝假托妖言,乃云不終其位。妄稱鬼怪,又道不得入宮,自言骨相非人臣,德業堪承重器,妄道清城出聖,欲以己當之,詐稱益州龍見,托言吉兆。重述木易之姓,更治成都之宮;妄說禾乃之名,以當八千之運。橫生京師妖異,以證父兄之災;妄造蜀地徵祥,以符己身之籙。汝豈不欲得國家惡也,天下亂也,輒造白玉之廷,又為白羽之箭,文物服飾,豈似有君,鳩集左道,符書厭鎮。漢王於汝,親則弟也,乃畫其形像,書其姓名,縛手釘心,枷鎖杻械。仍雲請西嶽華山慈父聖母神兵九億
 萬騎,收楊諒魂神,閉在華山下,勿令散蕩。我之於汝,親則父也,復雲請西嶽華山慈父呈母,賜為開化楊堅夫妻,回心歡喜。又畫我形像,縛手撮頭,仍雲請西嶽神兵收楊堅魂神。如此形狀,我今不知楊諒、楊堅是汝何親也?苞藏兇慝,圖謀不軌,逆臣之跡也;希父之災,以為身幸,賊子之心也;懷非分之望,肆毒心於兄,悖弟之行也;嫉妒於弟,無惡不為,無孔懷之情也;違犯制度,壞亂之極也;多殺不幸,豺狼之暴也;剝削民庶,酷虐之甚也;唯求財貨,市井之業也;專事妖邪,頑囂之性也;弗克負荷,不材之器也。凡此十者,滅天理,逆人倫,汝皆為之,不祥
 之甚也,欲免禍患,長守富貴,其可得乎!



 後復聽與其子同處。



 煬帝即位,禁錮如初。宇文化及之弒逆也,欲立秀為帝,群議不許。於是害之,並其諸子。



 庶人諒,字德章,一名傑,開皇元年,立為漢王。十二年,為雍州牧,加上柱國、右衛大將軍。歲餘,轉左衛大將軍。十七年,出為並州總管,上幸溫湯而送之。



 自山以東,至於滄海,南拒黃河,五十二州盡隸焉。特許以便宜,不拘律令。十八年,起遼東之役,以諒為行軍元帥,率眾至遼水,遇疾疫,不利而還。十九年,突厥犯塞,以諒為行軍元帥,竟不臨戎。高祖甚寵愛之。諒自以所居天下精兵處,以
 太子讒廢,居常怏怏,陰有異圖。遂諷高祖云:「突厥方強,太原即為重鎮,宜修武備。」高祖從之。於是大發工役,繕治器械,貯納於並州。招傭亡命,左右私人,殆將數萬。王頍者,梁將王僧辯之子也,少倜儻,有奇略,為諒咨議參軍。蕭摩訶者,陳氏舊將。二人俱不得志,每鬱鬱思亂,並為諒所親善。



 及蜀王以罪廢,諒愈不自安。會高祖崩,征之不赴,遂發兵反。總管司馬皇甫誕切諫,諒怒,收擊之。王頍說諒曰:「王所部將吏家屬,盡在關西,若用此等,即宜長驅深入,直據京都,所謂疾雷不及掩耳。若但欲割據舊齊之地,宜任東人。」



 諒不能專定,乃兼用二策,唱言
 曰:「楊素反,將誅之。」聞喜人總管府兵曹裴文安說諒曰:「井陘以西,是王掌握之內,山東士馬,亦為我有,宜悉發之。分遣羸兵,屯守要路,仍令隨方略地。率其精銳,直入蒲津。文安請為前鋒,王以大軍繼後,風行電擊,頓於霸上,咸陽以東可指麾而定。京師震擾,兵不暇集,上下相疑,群情離駭,我即陳兵號令,誰敢不從,旬日之間,事可定矣。」諒大悅。於是遣所署大將軍餘公理出太谷,以趣河陽。大將軍綦良出滏口,以趣黎陽。大將軍劉建出井陘,以略燕趙。柱國喬鐘葵出雁門。署文安為柱國,紇單貴、王聃、大將軍茹茹天保、侯莫陳惠直指京師。未至蒲
 津百餘里,諒忽改圖,令紇單貴斷河橋,守蒲州,而召文安。文安至曰:「兵機詭速,本欲出其不意。王既不行,文安又退,使彼計成,大事去矣。」諒不對。以王聃為蒲州刺史,裴文安為晉州,薛粹為絳州,梁菩薩為潞州,韋道正為韓州,張伯英為澤州。煬帝遣楊素率騎五千,襲王聃、紇單貴於蒲州,破之。於是率步騎四萬趣太原。諒使趙子開守高壁,楊素擊走之。諒大懼,拒素於蒿澤。屬天大雨,諒欲旋師,王頍諫曰:「楊素懸軍,士馬疲弊,王以銳卒親戎擊之,其勢必舉。今見敵而還,示人以怯,阻戰士之心,益西軍之氣,願王必勿還也。」諒不從,退守清源。素進擊
 之,諒勒兵與官軍大戰,死者萬八千人。諒退保並州,楊素進兵圍之。諒窮蹙,降於素。百僚奏諒罪當死,帝曰:「終鮮兄弟,情不忍言,欲屈法恕諒一死。」於是除名為民,絕其屬籍,竟以幽死。子顥,因而禁錮,宇文化及弒逆之際,遇害。



 史臣曰:高祖之子五人,莫有終其天命,異哉!房陵資於骨肉之親,篤以君臣之義,經綸締構,契闊夷險,撫軍監國,凡二十年,雖三善未稱,而視膳無闕。恩寵既變,讒言間之,顧復之慈,頓隔於人理,父子之道,遂滅於天性。隋室將亡之效,眾庶皆知之矣。《慎子》有言曰:「一兔走街,百
 人逐之,積兔於市,過者不顧。」豈有無欲哉?分定故也。房陵分定久矣,高祖一朝易之,開逆亂之源,長覬覦之望。又維城肇建,崇其威重,恃寵而驕,厚自封植,進之既逾制,退之不以道。



 俊以憂卒,實此之由。俄屬天步方艱,讒人已勝,尺布斗粟,莫肯相容。秀窺岷蜀之阻,諒起晉陽之甲,成茲亂常之釁,蓋亦有以動之也。《棠棣》之詩徒賦,有鼻之封無期,或幽囚於囹圄,或顛殞於鴆毒。本根既絕,枝葉畢剪,十有餘年,宗社淪陷。自古廢嫡立庶,覆族傾宗者多矣,考其亂亡之禍,未若有隋之酷。《詩》曰:「殷鑒不遠,在夏后之世。」後之有國有家者,可不深戒哉!



\end{pinyinscope}