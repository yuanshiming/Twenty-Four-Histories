\article{卷六十八列傳第三十三}

\begin{pinyinscope}

 宇文愷宇文愷,字安樂,杞國公忻之弟也。在周,以功臣子,年三歲,賜爵雙泉伯,七歲,進封安平郡公,邑二千戶。愷少有器局。家世武將,諸兄並以弓馬自達,愷獨好學,博覽書記,解屬文,多伎藝,號為名父公子。初為千牛,累遷御正中大夫、儀同三司。高祖為丞相,加上開府中大夫。及踐阼,誅宇文氏,愷初亦在殺中,以其與周本別,兄忻有功
 於國,使人馳赦之,僅而得免。後拜營宗廟副監、太子左庶子。廟成,別封甑山縣公,邑千戶。及遷都,上以愷有巧思,詔領營新都副監。高熲雖總大綱,凡所規畫,皆出於愷。後決渭水達河,以通運漕,詔愷總督其事。後拜萊州刺史,甚有能名。兄忻被誅,除名於家,久不得調。會朝廷以魯班故道久絕不行,令愷修復之。既而上建仁壽宮,訪可任者,右僕射楊素言愷有巧思,上然之,於是檢校將作大匠。歲餘,拜仁壽宮監,授儀同三司,尋為將作少監。文獻皇后崩,愷與楊素營山陵事,上善之,復爵安平郡公,邑千戶。煬帝即位,遷都洛陽,以愷為營東都副監,
 尋遷將作大匠。愷揣帝心在宏侈,於是東京制度窮極壯麗。帝大悅之,進位開府,拜工部尚書。及長城之役,詔愷規度之。時帝北巡,欲誇戎狄,令愷為大帳,其下坐數千人。帝大悅,賜物千段。又造觀風行殿,上容侍衛者數百人,離合為之,下施輪軸,推移倏忽,有若神功。戎狄見之,莫不驚駭。帝彌悅焉,前後賞賚,不可勝紀。



 自永嘉之亂,明堂廢絕,隋有天下,將復古制,議者紛然,皆不能決。愷博考群籍,奏《明堂議表》曰:臣聞在天成象,房心為布政之宮,在地成形,丙午居正陽之位。觀雲告月,順生殺之序;五室九宮,統人神之際。金口木舌,發令兆民;玉瓚
 黃琮,式嚴宗祀。



 何嘗不矜莊扆寧,盡妙思於規摹,凝睟冕旒,致子來於矩矱。



 伏惟皇帝陛下,提衡握契,御辯乘乾,減五登三,復上皇之化,流兇去暴,丕下武之緒。用百姓之異心,驅一代以同域,康哉康哉,民無能而名矣。故使天符地寶,吐醴飛甘,造物資生,澄源反樸。九圍清謐,四表削平,襲我衣冠,齊其文軌。



 茫茫上玄,陳珪璧之敬;肅肅清廟,感霜露之誠。正金奏《九韶》、《六莖》之樂,定石渠五官、三雍之禮。乃卜瀍西,爰謀洛食,辨方面勢,仰稟神謀,敷土浚川,為民立極。兼聿遵先言,表置明堂,爰詔下臣,占星揆日。於是採崧山之秘簡,披汶水之靈圖,訪通
 議於殘亡,購《冬官》於散逸。總集眾論,勒成一家。昔張衡渾象,以三分為一度,裴秀輿地,以二寸為千里。臣之此圖,用一分為一尺,推而演之,冀輪奐有序。而經構之旨,議者殊途,或以綺井為重屋,或以圓楣為隆棟,各以臆說,事不經見。今錄其疑難,為之通釋,皆出證據,以相發明。議曰:臣愷謹案《淮南子》曰:「昔者神農之治天下也,甘雨以時,五穀蕃植,春生夏長,秋收冬藏,月省時考,終歲獻貢,以時嘗谷,祀於明堂。明堂之制,有蓋而無四方,風雨不能襲,燥濕不能傷,遷延而入之。」臣愷以為上古樸略,創立典刑。



 《尚書帝命驗》曰:「帝者承天立五府,以尊天
 重象。赤曰文祖,黃曰神鬥,白曰顯紀,黑曰玄矩,蒼曰靈府。」注云:「唐、虞之天府,夏之世室,殷之重屋,周之明堂,皆同矣。」《尸子》曰:「有虞氏曰總章。」《周官·考工記》曰:「夏后氏世室,堂修二七,博四修一。」注云:「修南北之深也。夏度以步,今修十四步,其博益以四分修之一,則明堂博十七步半也。」臣愷按,三王之世,夏最為古,從質尚文,理應漸就寬大,何因夏室乃大殷堂?相形為論,理恐不爾。《記》云「堂修七,博四修一」,若夏度以步,則應修七步。注云「今堂修十四步」,乃是增益《記》文。殷、周二堂獨無加字,便是其義,類例不同。山東《禮》本輒加二七之字,何得殷無加尋之
 文,周闕增筵之義?研核其趣,或是不然。讎校古書,並無二字,此乃桑間俗儒信情加減。《黃圖議》云:「夏后氏益其堂之大一百四十四尺,周人明堂以為兩杼間。」馬宮之言,止論堂之一面,據此為準,則三代堂基並方,得為上圓之制。諸書所說,並云下方,鄭注《周官》,獨為此義,非直與古違異,亦乃乖背禮文。尋文求理,深恐未愜。《尸子》曰:「殷人陽館。」《考工記》曰:「殷人重屋,堂修七尋,堂崇三尺,四阿重屋。」注云:其修七尋,五丈六尺,放夏周則其博九尋,七丈二尺。」又曰:「周人明堂,度九尺之筵,東西九筵。南北七筵。堂崇一筵。五室,凡二筵。」《禮記·明堂位》曰:「天子之廟,
 復廟重簷。」鄭注云:「復廟,重屋也。」注《玉藻》云:「天子廟及露寢,皆如明堂制。」



 《禮圖》云:「於內室之上,起通天之觀,觀八十一尺,得宮之數,其聲濁,君之象也。」《大戴禮》曰:「明堂者,古有之。凡九室,一室有四戶八牖。以茅蓋,上圓下方,外水曰璧雝。赤綴戶,白綴牖。堂高三尺,東西九仞,南北七筵。其宮方三百步。凡人民疾,六畜疫,五穀災,生於天道不順。天道不順,生於明堂不飾。



 故有天災,則飾明堂。」《周書·明堂》曰:「堂方百一十二尺,高四尺,階博六尺三寸。室居內,方百尺,室內方六十尺。戶高八尺,博四尺。」《作洛》曰:「明堂太廟露寢,咸有四阿,重亢重廊。」孔氏注云:「重亢累
 棟,重廊累屋也。」



 《禮圖》曰:「秦明堂九室十二階,各有所居。」《呂氏春秋》曰:「有十二堂。」



 與《月令》同,並不論尺丈。臣愷案,十二階雖不與《禮》合,一月一階,非無理思。《黃圖》曰:「堂方百四十四尺,法坤之策也,方象地。屋圓楣徑二百一十六尺,法乾之策也,圓象天。太室九宮,法九州。太室方六丈,法陰之變數。十二堂法十二月,三十六戶法極陰之變數,七十二牖法五行所行日數。八達象八風,法八卦。通天臺徑九尺,法乾以九覆六。高八十一尺,法黃鐘九九之數。二十八柱象二十八宿。堂高三尺,上階三等,法三統。堂四向五色,法四時五行。殿門去殿七十二步,法五
 行所行。門堂長四丈,取太室三之二。垣高無蔽目之照,牖六尺,其外倍之。殿垣方,在水內,法地陰也。水四周於外,象四海,圓法陽也。水闊二十四丈,象二十四氣。水內徑三丈,應《覲禮經》。」武帝元封二年,立明堂汶上,無室。其外略依此制。《泰山通議》今亡,不可得而辨也。



 元始四年八月,起明堂、闢雍長安城南門,制度如儀。一殿,垣四面,門八觀,水外周,堤壤高四尺,和會築作三旬。五年正月六日辛未,始郊太祖高皇帝以配天。



 二十二日丁亥,宗祀孝文皇帝於明堂以配上帝,及先賢、百闢、卿士有益者,於是秩而祭之。親扶三老五更,袒而割牲,跪而進之。
 因班時令,宣恩澤。諸侯王、宗室、四夷君長、匈奴、西國侍子,悉奉貢助祭。



 《禮圖》曰:「建武三十年作明堂,明堂上圓下方,上圓法天,下方法地,十二堂法日辰,九室法九州。室八牖,八九七十二,法一時之王。室有二戶,二九十八戶,法土王十八日。內堂正壇高三尺,土階三等。」胡伯始注《漢官》云:「古清廟蓋以茅,今蓋以瓦,瓦下藉茅,以存古制。」《東京賦》曰:「乃營三宮,布政頒常。復廟重屋,八達九房。造舟清池,惟水泱泱。」薛綜注云:「復重廇覆,謂屋平覆重棟也。」《續漢書·祭祀志》云:「明帝永平二年,祀五帝於明堂,五帝坐各處其方,黃帝在未,皆如南郊之位。光武位在
 青帝之南,少退西面,各一犢,奏樂如南郊。」臣愷按《詩》云,《我將》祀文王於明堂,「我將我享,維牛維羊」。



 據此則備太牢之祭。今雲一犢,恐與古殊。自晉以前,未有鶉尾,其圓墻璧水,一依本圖。《晉起居注》裴頠議曰:「尊祖配天,其義明著,廟宇之制,理據未分。



 直可為一殿,以崇嚴祀,其餘雜碎,一皆除之。」臣愷案,天垂象,聖人則之。闢雍之星,既有圖狀,晉堂方構,不合天文。既闕重樓,又無璧水,空堂乖五室之義,直殿違九階之文。非古欺天,一何過甚!後魏於北臺城南造圓墻,在璧水外,門在水內迥立,不與墻相連。其堂上九室,三三相重,不依古制,室間通巷,違
 舛處多。



 其室皆用墼累,極成褊陋。後魏《樂志》曰:「孝昌二年立明堂,議者或言九室,或言五室,詔斷從五室。後元叉執政,復改為九室,遭亂不成。」《宋起居注》曰:「孝武帝大明五年立明堂,其墻宇規範,擬則太廟,唯十二間,以應期數。依漢《汶上圖儀》,設五帝位。太祖文皇帝對饗,鼎俎簠簋,一依廟禮。」梁武即位之後,移宋時太極殿以為明堂。無室,十二間。《禮疑議》云:「祭用純漆俎瓦樽,文於郊,質於廟。止一獻,用清酒。」平陳之後,臣得目觀,遂量步數,記其尺丈。



 猶見基內有焚燒殘柱,毀斫之餘,入地一丈,儼然如舊。柱下以樟木為跗,長丈餘,闊四尺許,兩兩相並。
 瓦安數重。宮城處所,乃在郭內。雖湫隘卑陋,未合規摹,祖宗之靈,得崇嚴祀。周、齊二代,闕而不修,大饗之典,於焉靡托。



 自古明堂圖惟有二本,一是宗周,劉熙、阮諶、劉昌宗等作,三圖略同。一是後漢建武三十年作,《禮圖》有本,不詳撰人。臣遠尋經傳,傍求子史,研究眾說,總撰今圖。其樣以木為之,下為方堂,堂有五室,上為圓觀,觀有四門。



 帝可其奏。會遼東之役,事不果行。以渡遼之功,進位金紫光祿大夫。其年卒官,時年五十八。帝甚惜之。謚曰康。撰《東都圖記》二十卷、《明堂圖議》二卷、《釋疑》一卷,見行於世。子儒童,游騎尉。少子溫,起部承務郎。



 閻毗閻毗,榆林盛樂人也。祖進,魏本郡太守。父慶,周上柱國、寧州總管。毗七歲,襲爵石保縣公,邑千戶。及長,儀貌矜嚴,頗好經史。受《漢書》於蕭該,略通大旨。能篆書,工草隸,尤善畫,為當時之妙。周武帝見而悅之,命尚清都公主。



 宣帝即位,拜儀同三司,授千牛左右。高祖受禪,以技藝侍東宮,數以雕麗之物取悅於皇太子,由是甚見親待,每稱之於上。尋拜車騎,宿衛東宮。上嘗遣高熲大閱於龍臺澤,諸軍部伍多不齊整,唯毗一軍法制肅然。熲言之於上,特蒙賜帛。俄兼太子宗衛率長史,尋加上儀同。
 太子服玩之物,多毗所為。及太子廢,毗坐杖一百,與妻子俱配為官奴婢。後二歲,放免為民。煬帝嗣位,盛修軍器,以毗性巧,諳練舊事,詔典其職。尋授朝請郎。毗立議,輦輅車輿,多所增損,語在《輿服志》。



 擢拜起部郎。



 帝嘗大備法駕,嫌屬車太多,顧謂毗曰:「開皇之日,屬車十有二乘,於事亦得。今八十一乘,以牛駕車,不足以益文物。朕欲減之,從何為可?」毗對曰:「臣初定數,共宇文愷參詳故實,據漢胡伯始、蔡邕等議,屬車八十一乘,此起於秦,遂為後式。故張衡賦云『屬車九九』是也。次及法駕,三分減一,為三十六乘。



 此漢制也。又據宋孝建時,有司奏議,晉
 遷江左,惟設五乘,尚書令、建平王宏曰:『八十一乘,議兼九國,三十六乘,無所準憑。江左五乘,儉不中禮。但帝王文物,旂旒之數,爰及冕玉,皆同十二。今宜準此,設十二乘。』開皇平陳,因以為法。



 今憲章往古,大駕依秦,法駕依漢,小駕依宋,以為差等。」帝曰:「何用秦法乎?



 大駕宜三十六,法駕宜用十二,小駕除之。」毗研精故事,皆此類也。



 長城之役,毗總其事。及帝有事恆岳,詔毗營立壇場。尋轉殿內丞,從幸張掖郡。高昌王朝於行所,詔毗持節迎勞,遂將護入東都。尋以母憂去職。未期,起令視事。將興遼東之役,自洛口開渠,達於涿郡,以通運漕。毗督其役。明
 年,兼領右翊衛長史,營建臨朔宮。及征遼東,以本官領武賁郎將,典宿衛。時眾軍圍遼東城,帝令毗詣城下宣諭,賊弓弩亂發,所乘馬中流矢,毗顏色不變,辭氣抑揚,卒事而去。尋拜朝請大夫,遷殿內少監,又領將作少監事。後復從帝征遼東,會楊玄感作逆,帝班師,兵部侍郎斛斯政奔遼東,帝令毗率騎二千追之,不及。政據高麗柏崖城,毗攻之二日,有詔徵還。從至高陽,暴卒,時年五十。帝甚悼惜之,贈殿內監。



 何稠劉龍黃亙亙弟袞何稠,字桂林,國子祭酒妥之兄子也。父通,善斫玉。稠性
 絕巧,有智思,用意精微。年十餘歲,遇江陵陷,隨妥入長安。仕周御飾下士。及高祖為丞相,召補參軍,兼掌細作署。開皇初,授都督,累遷御府監,歷太府丞。稠博覽古圖,多識舊物。波斯嘗獻金綿錦袍,組織殊麗。上命稠為之。稠錦既成,逾所獻者,上甚悅。



 時中國久絕琉璃之作,匠人無敢厝意,稠以綠瓷為之,與真不異。尋加員外散騎侍郎。



 開皇末,桂州俚李光仕聚眾為亂,詔稠召募討之。師次衡嶺,遣使者諭其渠帥洞主莫崇解兵降款。桂州長史王文同鎖崇以詣稠所。稠詐宣言曰:「州縣不能綏養,致邊民擾叛,非崇之罪也。」乃命釋之,引崇共坐,並從
 者四人,為設酒食而遣之。



 崇大悅,歸洞不設備。稠至五更,掩入其洞,悉發俚兵,以臨餘賊。象州逆帥杜條遼、羅州逆帥龐靖等相繼降款。分遣建州開府梁暱討叛夷羅壽,羅州刺史馮暄討賊帥李大檀,並平之,傳首軍門。承制署首領為州縣官而還,眾皆悅服。有欽州刺史寧猛力,帥眾迎軍。初,猛力倔強山洞,欲圖為逆,至是惶懼,請身入朝。稠以其疾篤,因示無猜貳,遂放還州,與之約曰:「八九月間,可詣京師相見。」稠還奏狀,上意不懌。其年十月,猛力卒,上謂稠曰:「汝不前將猛力來,今竟死矣。」



 稠曰:「猛力共臣為約,假令身死,當遣子入侍。越人性直,其
 子必來。」初,猛力臨終,誡其子長真曰:「我與大使為約,不可失信於國士。汝葬我訖,即宜上路。」



 長真如言入朝,上大悅曰:「何稠著信蠻夷,乃至於此。」以勛授開府。



 仁壽初,文獻皇后崩,與宇文愷參典山陵制度。稠性少言,善候上旨,由是漸見親暱。及上疾篤,謂稠曰:「汝既曾葬皇后,今我方死,宜好安置。屬此何益,但不能忘懷耳。魂其有知,當相見於地下。」上因攬太子頸謂曰:「何稠用心,我付以後事,動靜當共平章。」



 大業初,煬帝將幸揚州,謂稠曰:「今天下大定,朕承洪業,服章文物,闕略猶多。卿可討閱圖籍,營造輿服羽儀,送至江都也。」其日,拜太府少卿。稠
 於是營黃麾三萬六千人仗,及車輿輦輅、皇后鹵簿、百官儀服,依期而就,送於江都。



 所役工十萬餘人,用金銀錢物巨億計。帝使兵部侍郎明雅、選部郎薛邁等勾核之,數年方竟,毫厘無舛。稠參會今古,多所改創。魏、晉以來,皮弁有纓而無笄導。



 稠曰:「此古田獵之服也。今服以入朝,宜變其制。」故弁施象牙簪導,自稠始也。



 又從省之服,初無佩綬,稠曰:「此乃晦朔小朝之服。安有人臣謁帝而去印綬,兼無佩玉之節乎?」乃加獸頭小綬及佩一只。舊制,五輅於轅上起箱,天子與參乘同在箱內。稠曰:「君臣同所,過為相逼。」乃廣為盤輿,別構欄盾,侍臣立於其
 中。於內復起須彌平坐,天子獨居其上。自餘麾幢文物,增損極多,事見《威儀志》。



 帝復令稠造戎車萬乘,鉤陳八百連,帝善之,以稠守太府卿。後三歲,兼領少府監。



 遼東之役,攝右屯衛將軍,領御營弩手三萬人。時工部尚書宇文愷造遼水橋不成,師不得濟,右屯衛大將軍麥鐵杖因而遇害。帝遣稠造橋,二日而就。初,稠制行殿及六合城,至是,帝於遼左與賊相對,夜中施之。其城周回八里,城及女垣合高十仞,上布甲士,立仗建旗。四隅置闕,面別一觀,觀下三門,遲明而畢。高麗望見,謂若神功。是歲,加金紫光祿大夫。明年,攝左屯衛將軍,從至遼左。



 十
 二年,加右光祿大夫,從幸江都。遇宇文化及作亂,以為工部尚書。化及敗,陷於竇建德,建德復以為工部尚書、舒國公。建德敗,歸於大唐,授將作少匠,卒。



 開皇時,有劉龍者,河間人也。性強明,有巧思。齊後主知之,令修三爵臺,甚稱旨,因而歷職通顯。及高祖踐阼,大見親委,拜右衛將軍,兼將作大匠。遷都之始,與高熲參掌制度,代號為能。



 大業時,有黃亙者,不知何許人也,及其弟袞,俱巧思絕人。煬帝每令其兄弟直少府將作。於時改創多務,亙、袞每參典其事。凡有所為,何稠先令亙、袞立樣,當時工人皆稱其善,莫能有所損益。亙官至朝散大夫,袞官
 至散騎侍郎。



 史臣曰:宇文愷學藝兼該,思理通贍,規矩之妙,參縱班、爾,當時制度,咸取則焉。其起仁壽宮,營建洛邑,要求時幸,窮侈極麗,使文皇失德,煬帝亡身,危亂之源,抑亦此之由。至於考覽書傳,定《明堂圖》,雖意過其通,有足觀者。



 毗、稠巧思過人,頗習舊事,稽前王之採章,成一代之文物。雖失之於華盛,亦有可傳於後焉。



\end{pinyinscope}