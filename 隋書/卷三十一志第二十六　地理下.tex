\article{卷三十一志第二十六 地理下}

\begin{pinyinscope}

 彭城郡舊置徐州,後齊置東南道行臺,後周立總管府。開皇七年行臺廢,大業四年府廢。統縣十一,戶一十三萬二百三十二。



 彭城舊置郡,後周並沛及南陽平二郡入。開皇初郡廢,大業初復置郡。有呂梁山、徐山。蘄梁置蘄郡。後齊置仁州,又析置龍亢郡。開皇初郡廢,大業初州廢。



 穀陽後齊置穀陽郡,開皇初郡廢。又有巳吾、義城二縣,後齊並以為臨淮縣,大業初並入焉。沛留後齊廢,開皇十六年復。有微山、黃山。豐蕭舊置沛郡,後齊廢為承高縣。開皇六年改為龍城,十八年改為臨沛,大業初改曰蕭。有相山。滕舊曰蕃,置蕃郡。後齊廢。開皇十六年改曰滕縣。蘭
 陵舊曰承,置蘭陵郡。開皇初郡廢,十六年分承置鄶州及蘭陵縣。大業初州廢,又並蘭陵、鄶城二縣入焉,尋改承為蘭陵。



 有抱犢山。符離後齊置睢南郡,開皇初郡廢,有竹邑縣,梁置睢州,開皇三年州廢,又廢竹邑入焉。有女山、定陶山。方與後齊廢,開皇十六年復。



 魯郡舊兗州,大業二年改為魯郡。統縣十,戶十二萬四千一十九。



 瑕丘舊廢,開皇十三年復,帶郡。任城舊置高平郡,開皇初廢。鄒有鄒山、承匡山。曲阜舊曰魯郡,後齊改郡為任城。開皇三年郡廢,四年改縣曰汶陽,十六年改名曲阜。泗水開皇十六年置。有陪尾山、尼丘山、防山。有洙、泗水。平陸後齊曰樂平,開皇十六年改焉。龔丘後齊曰平原縣,開皇十六年改焉。梁父有龜山。博城舊曰博,置泰山郡。後齊改郡曰東平,又並博平、牟入焉。開皇初郡廢,十六年改縣曰汶陽,尋改曰博城。有奉高縣,開皇六年改曰岱山,大業初州廢,又廢岱山縣入焉。有岱山、玉符山。嬴開皇十六年分置牟城縣,大業初並入焉。有艾山。有淄水。



 瑯邪郡舊置北徐州,後周改曰沂州。統縣七,戶六萬三千四百二十三。



 臨沂舊曰即丘,帶郡。開皇初郡廢,十六年分置臨沂,大業初並即丘入焉。有大祠山。費顓臾舊曰南城武陽,開皇十八年改名焉。又有南城縣,後齊廢。有開明山。新泰後齊廢蒙陰縣入焉。沂水舊置南青州及東安郡,後周改州為莒州。開皇初郡廢,改縣曰東安。十六年又改曰沂水。大業初州廢。東安後齊廢,開皇十六年復。



 有松山。莒舊置東莞郡。後齊廢,後置義唐郡。開皇初廢。



 東海郡梁置南、北二青州,東魏改為海州。統縣五,戶二萬七千八百五十八。



 朐山舊曰朐,置瑯邪郡。後周改縣曰朐山,郡曰朐山。開皇初郡廢,大業初復,帶郡。有朐山、羽山。東海舊置廣饒縣及東海郡,後齊分廣饒置東海縣。開皇初廢郡及東海縣,仁壽元年,改廣饒曰東海。有謝祿山、鬱林山。漣水舊曰襄賁。置東海郡。東魏改曰海安。開皇初郡廢,縣又改焉。沭陽梁置潼陽郡。東魏改曰沭陽郡,置縣曰懷文。後周改縣曰沭陽。開皇初郡廢。懷仁梁置南、北二青州。東魏廢
 州,立義唐郡及懷仁縣。開皇初郡廢。



 下邳郡後魏置南徐州,梁改為東徐州,東魏又改曰東楚州,陳改為安州,後周改為泗州。統縣七,戶五萬二千七十。



 宿豫舊置宿豫郡,開皇初郡廢。大業初置下邳郡。又梁置朝陽、臨沭二郡,後齊置晉寧郡,尋並廢。夏丘後齊置,並置夏丘郡,尋立潼州。後周改州為宋州,縣曰晉陵。開皇初郡廢,十八年州廢,縣復曰夏丘。又東魏置臨潼郡,睢陵縣,後齊改郡為潼郡。又梁置潼州,後齊改曰睢州,尋廢,亦入潼郡。開皇初郡縣並廢。徐城梁置高平郡。東魏又並梁東平、陽平、清河、歸義四郡為高平縣,又並梁硃沛、循儀、安豐三郡置硃沛縣。又有安遠郡,後齊廢,後周又並硃沛入高平。開皇初郡廢,十八年更名徐城。淮陽梁置淮陽郡。東魏並綏化、呂梁二郡置綏化縣。後周改縣為淮陽。開皇初郡廢。又有梁臨清、天水、浮陽三郡,東魏並為甬城縣,後齊改曰文城縣,後周又改為臨清,開皇三年省入焉。下邳梁曰歸政,置武州,下邳郡。



 魏改縣為下邳,置郡不改,改州曰東徐。後周改州為邳州。開皇初郡廢,大業初州廢。有嶧
 山、磬石山。良城梁置武安郡,開皇初郡廢,十一年縣更名曰良城。有徐山。郯舊置郡,開皇初廢。



 《禹貢》:「海、岱及淮惟徐州。」彭城、魯郡、瑯邪、東海、下邳,得其地焉。在於天文,自奎五度至胃六度,為降婁,於辰在戌。其在列國,則楚、宋及魯之交。考其舊俗,人頗勁悍輕剽,其士子則挾任節氣,好尚賓游,此蓋楚之風焉。



 大抵徐、兗同俗,故其餘諸郡,皆得齊、魯之所尚。莫不賤商賈,務稼穡,尊儒慕學,得洙泗之俗焉。



 江都郡梁置南兗州,後齊改為東廣州,陳復曰南兗,後周改為吳州。開皇九年改為揚州,置總管府,大業初府廢。統縣十六,戶十一萬五千五百二十四。



 江陽舊曰廣陵,後齊置廣陵、江陽二郡。開皇初郡廢,十八年改縣為邗江,大業初。更名江陽。有江都宮、揚子宮。有陵湖。
 江都自梁及隋,或廢或置。海陵梁置海陵郡。開皇初郡廢,又並建陵縣入,尋析置江浦縣,大業初省入。寧海開皇初並如皋縣入。高郵梁析置竹塘、三歸二縣,及置廣業郡,尋以有嘉禾,為神農郡。



 開皇初郡廢,又並竹塘、三歸、臨澤三縣入焉。安宜梁置陽平郡及東莞郡。開皇初郡廢,又廢石鱉縣入焉。有白馬湖。山陽舊置山陽郡,開皇初郡廢。十二年置楚州,大業初州廢。有後魏淮陰郡,東魏改為淮州,後齊並魯、富陵立懷恩縣,後周改曰壽張,又僑立東平郡。開皇元年改郡為淮陰,並立楚州,尋廢郡,更改縣曰淮陰。



 大業初州廢,縣並入焉。盱眙舊魏置盱眙郡。陳置北譙州,尋省。開皇初郡廢,又並考城、直瀆、陽城三縣入。有都梁山。鹽城後齊置射陽郡,陳改曰鹽城,開皇初郡廢。清流舊曰頓丘,置新昌郡及南譙州。開皇初改為滁州,郡廢。又廢樂鉅、高塘二縣入頓丘,改曰新昌。十八年又改為清流。大業初州廢。有白禪山、曲亭山。



 全椒梁曰北譙,置北譙郡。後齊改郡為臨滁,後周又曰北譙。開皇初郡廢,改縣為滁水。大業初改名焉。有銅官山、九斗山。六合舊曰尉氏,置秦郡。後齊置秦州。



 後周改州曰方州,改郡曰六合。開皇初郡廢,四
 年改尉氏曰六合,省堂邑、方山二縣入焉。大業初州廢。又後齊置瓦梁郡,陳廢。有瓜步山、六合山。永福舊曰沛,梁置涇城、東陽二郡,陳廢州,並二郡為梁郡。後周改梁郡為石梁郡,改沛縣曰石梁縣,省橫山縣入焉。開皇初郡廢。大業初改縣曰永福。有香山、永福山。句容有茅山、浮山、四平山。延陵舊置南徐州、南東海郡,梁改曰蘭陵郡,陳又改為東海。



 開皇九年州郡並廢,又廢丹徒縣入焉。十五年置潤州,大業初州廢。有句驪山、黃鵠山、蒜山、長塘湖。曲阿有武進縣,梁改為蘭陵,開皇九年並入。



 鐘離郡後齊曰西楚州,開皇二年改曰濠州。統縣四,戶三萬五千一十五。



 鐘離舊置郡,開皇初郡廢。大業中復置郡。定遠舊曰東城。梁改曰定遠,置臨濠郡。後齊改曰廣安。開皇初郡廢。又有舊九江郡,後齊廢為曲陽縣,縣尋廢。又有梁置安州,侯景亂廢。化明故曰睢陵,置濟陰郡。後齊改縣曰池南,陳復曰睢陵,後周改為昭義。開皇初郡廢,大業初縣改名焉。塗山舊曰當塗。後齊改曰馬頭,置郡曰荊山。開皇初改
 縣曰塗山,廢郡。有當塗山。



 淮南郡舊曰豫州,後魏曰揚州,梁曰南豫州,東魏曰揚州,陳又曰豫州,後周曰揚州。開皇九年曰壽州,置總管府,大業元年府廢。統縣四,戶三萬四千二百七十八。



 壽春舊有淮南、梁郡、北譙、汝陰等郡,開皇初並廢,並廢蒙縣入焉。大業初置淮南郡。有八公山、門溪。安豐梁置陳留、安豐二郡,開皇初並廢。有芍陂。霍丘梁置安豐郡,東魏廢。開皇十九年置縣,名焉。長平梁置北陳郡,開皇初廢,又並西華縣人。



 弋陽郡梁置光州。統縣六,戶四萬一千四百三十三。



 光山舊置光城郡。開皇初郡廢,十八年置縣焉。大業初置光陽郡。又有舊黃川郡,梁廢。樂安梁置宋安郡,及宋安、光城二縣,又有豐安郡,開皇三年並廢入焉。



 有弋陽山、浮光山、金山、錫山。定城後齊置南郢州,後廢入南、北二弋陽縣,後又省北弋陽入南弋陽,改為定遠焉。又後魏置弋陽郡,及有梁東新蔡縣。後周改為淮南郡。又後齊置齊安、新蔡二郡,及廢舊義州,立東光城郡。至開皇初,五郡及郢州並廢。殷城舊曰包信,開皇
 初改名焉。梁置義城郡及建州,並所領平高、新蔡、新城三郡,開皇初並廢。有大蘇山、南松山。固始梁曰蓼縣。後齊改名焉,置北建州,尋廢州,置新蔡郡。後周改置澮州。開皇初州郡並廢入,又改縣為固始。有安陽山。期思陳置邊城郡。開皇初郡廢,改縣名焉。有後齊光化郡,亦廢入焉。有大別山。



 蘄春郡後齊置雍州,後周改曰蘄州。開皇初置總管府,九年府廢。統縣五,戶三萬四千六百九十。



 蘄春舊曰蘄陽,梁改曰蘄水。後齊改曰齊昌,置齊昌郡。開皇十八年改為蘄春。



 開皇初郡廢。有安山。浠水舊置永安郡,開皇初郡廢。有石鼓山。蘄水舊曰蘄春,梁改名焉。有鼓吹山。有蘄水。黃梅舊曰永興,開皇初改曰新蔡,十八年改名焉。



 有黃梅山。羅田梁置義州、義城郡,開皇初並廢。



 廬江郡梁置南豫州,又改為合州。開皇初改為廬州。統縣七,戶四萬一千六百三十二。



 合肥梁曰汝陰,置汝陰郡。後齊分置北陳郡。開皇初郡廢,縣改名焉。
 廬江齊置廬江郡,梁置湘州,後齊州廢,開皇初郡廢。有冶甫山、上薄山、三公山、聖山、藍家山。襄安梁曰蘄,開皇初改焉。有龜山、紫微山、亞父山、半陽山、白石山、四鼎山。慎東魏置平梁郡,陳曰梁郡,開皇初郡廢。有浮闍山。霍山梁置霍州及岳安郡、岳安縣。後齊州廢。開皇初郡廢,縣改名焉。渒水梁置北沛郡及新蔡縣。開皇初郡廢,又廢新蔡入焉。有墜星山。開化梁置。有衡山、九公山、蹋鼓山、天山、多智山。



 同安郡梁置豫州,後改曰晉州,後齊改曰江州,陳又曰晉州,開皇初曰熙州。



 統縣五,戶二萬一千七百六十六。



 懷寧舊置晉熙郡,開皇初郡廢。大業三年置同安郡。宿松梁置高塘郡。開皇初郡廢,改縣曰高塘,十八年又改名焉。有雷水。太湖開皇初改為晉熙,十八年復改名焉。望江陳置大雷郡。開皇十一年改曰義鄉,十八年改名焉。同安舊曰樅陽,並置樅陽郡。開皇初郡廢,十八年縣改名焉。有浮度山。



 歷陽郡後齊立和州。統縣二,戶八千二百五十四。



 歷陽
 舊置歷陽郡,開皇初郡廢。大業初復置郡。烏江梁置江都郡,後齊改為齊江郡,陳又改為臨江郡,周改為同江郡。開皇初郡廢。大業初置歷陽郡。有六合山。



 彤陽郡自東晉已後置郡曰揚州。平陳,詔並平蕩耕墾,更於石頭城置蔣州,統縣三,戶二萬四千一百二十五。



 江寧梁置丹陽郡及南丹陽郡,陳省南丹陽郡。平陳,又廢丹陽郡,並以秣陵、建康、同夏三縣入焉。大業初置丹陽郡。有蔣山。當塗舊置淮南郡。平陳,廢郡,並襄垣、於湖、繁昌、西鄉入焉。,有天門山、楚山。溧水舊曰溧陽。開皇九年廢丹陽郡入,十八年改焉。有赭山、廬山、楚山。



 宣城郡舊置南豫州。平陳,改為宣州。統縣六,戶一萬九千九百七十九。



 宣城舊曰宛陵,置宣城郡。平陳,郡廢,仍並懷安、寧國、當塗、浚遒四縣入焉。大業初置郡。有敬亭山。涇平陳,省安吳、南陽二縣入焉。有蓋山、陵陽山。



 南陵梁置,並置南陵郡,陳置北江州。平陳,州
 郡並廢,並所管石城、臨城、定陵、故治、南陵五縣入焉。秋浦舊曰石城。平陳廢,開皇十九年置,改名焉。永世平陳廢,開皇十二年又置。有靈光山。綏安舊曰石封,平陳,改名焉。梁末立大梁郡,又改為陳留。平陳,郡廢,省大德、故鄣、安吉、原鄉四縣入焉。



 毗陵郡平陳,置常州。統縣四,戶一萬七千五百九十九。



 晉陵舊置晉陵郡。平陳,郡廢。大業初置郡。有橫山。江陰梁置,及置江陰郡。



 平陳,廢郡,及利城梁豐縣入焉。有毗陵山。無錫有九龍山。義興舊曰陽羨,置義興郡。平陳,郡廢,改縣名焉。又廢義鄉、國山、臨津三縣入焉。有計山、洞庭山。



 吳郡陳置吳州。平陳,改曰蘇州,大業初復曰吳州。統縣五,戶一萬八千三百七十七。



 吳舊置吳郡。平陳,郡廢,大業初復置。有胥山、橫山、華山、黃山、姑蘇山、太湖。昆山梁置。平陳廢,開皇十八年復。常熟舊曰南沙,梁置信義郡。平陳廢,
 並所領海陽、前京、信義、海虞、興國、南沙入焉。有虞山。烏程舊置吳興郡。平陳,郡廢,並東遷縣入焉。仁壽中置湖州,大業初州廢。有雉山。長城平陳廢,仁壽二年復。有卞山。



 會稽郡梁置東揚州。陳初省,尋復。平陳,改曰吳州,置總管府。大業初府廢,置越州。統縣四,戶二萬二百七十一。



 會稽舊置會稽郡。平陳,郡廢,及廢山陰、永興、上虞、始寧四縣入,大業初置郡。有稷山、種山、會稽山。句章平陳,並餘姚、鄞、鄮三縣入。有太白山、方山。剡有桐柏山。諸暨有洩溪、大農湖。



 餘杭郡平陳,置杭州。仁壽中置總管府,大業初府廢。統縣六,戶一萬五千三百八十。



 錢唐舊置錢唐郡。平陳,廢郡,並所領新城縣入。大業三年置餘杭郡。有粟山、石甑山、臨平湖。富陽有石頭山、雞籠山。餘杭有由拳山、金鵝山。於灊有天目山、石鏡山。有桐溪。鹽官有蜀山。武康平陳廢,仁壽二年復。有
 封嵎山、青山、白鵠山。



 新安郡平陳,置歙州。統縣三,戶六千一百六十四。



 休寧舊曰海寧,開皇十八年改名焉。大業初置郡。歙平陳廢,十一年復。黟平陳廢,十一年復。



 東陽郡平陳,置婺州。統縣四,戶一萬九千八百五。



 金華舊曰長山,置金華郡。平陳,郡廢,又廢建德、太末、豐安三縣入,改為吳寧縣。十二年改曰東陽,十八年改名焉。大業初置東陽郡。有長山、龍山、樓山、丘山。有赤松澗。永康烏傷有香山、歌山。信安有江山、悲思嶺。有定陽溪。



 永嘉郡開皇九年置處州,十二年改曰括州。統縣四,戶一萬五百四十二。



 括倉平陳,置縣,大業初置永嘉郡。有縉雲山、括倉山。永嘉舊曰永寧,置永嘉郡。平陳,郡廢,縣改名焉。有芙蓉山。松陽臨海舊曰章安,置臨海郡。平陳,郡廢,縣改名焉。有赤山、天臺山。



 建安郡陳置閩州,仍廢,後又置豐州。平陳,改曰泉州。大業初改曰閩州。統縣四,戶
 一萬二千四百二十。



 閩舊曰東侯官,置晉安郡。平陳,郡廢,縣改曰原豐。十二年改曰閩,大業初置建安郡。有岱山、飛山。建安舊置建安郡。平陳廢。南安舊曰晉安,置南安郡。



 平陳,郡廢,縣改名焉;又置莆田縣,尋廢入焉。龍溪梁置,開皇十二年並蘭水、綏安二縣入焉。



 遂安郡仁壽三年置睦州。統縣三,戶七千三百四十三。



 雉山舊置新安郡。平陳,廢為新安縣。大業初縣改名焉,置遂安郡。有仙壇山。



 遂安平陳廢,仁壽中復。桐廬平陳廢,仨壽中復。有白石山。



 鄱陽郡梁置吳州,陳廢。平陳,置饒州。統縣三,戶一萬一百二。



 鄱陽舊置鄱陽郡。平陳廢,又有陳銀城縣廢入焉。大業初復置郡。餘干弋陽舊曰葛陽,開皇十二年改。有弋水。



 臨川郡平陳,置撫州。統縣四,戶一萬九百。



 臨川舊置臨川郡。平陳,郡廢,
 大業三年復置郡。有銅山、黃山。有夢水。南城有五章山。崇仁梁置巴山郡,領大豐、新安、巴山、新建、興平、豐城、西寧七縣。平陳,郡縣並廢,以置縣焉。邵武開皇十二年置。



 廬陵郡平陳,置吉州。統縣四,戶二萬三千七百一十四。



 廬陵舊置廬陵郡。平陳廢,大業初復置。泰和平陳置,曰西昌。十一年省東昌入,更名焉。安復舊置安成郡。平陳,郡廢,縣改曰安成。十八年又曰安復。有更生山、長嶺。新淦有玉笥山。



 南康郡開皇九年置虔州。統縣四,戶一萬一千一百六十八。



 贛舊曰南康,置南康郡。平陳,郡廢。大業初縣改名焉,尋置郡。有儲山。有贛水。虔化舊曰寧都,開皇十八年改名焉。有石鼓山。雩都舊廢,平陳置。有金雞山、君山。南康舊曰贛,大業初改名焉。有廩山、上洛山、贛山。



 宜春郡平陳,置袁州。統縣三,戶一萬一百一十六。



 宜春舊曰宜
 陽。開皇十一年廢吳平縣入,十八年改名焉。大業初置郡。有廬溪、渝水。萍鄉有宜春江。新喻豫章郡平陳,置洪州總管府。大業初府廢。統縣四,戶一萬二千二十一。



 豫章舊置豫章郡。平陳,郡廢。大業初復置郡。豐城平陳廢。十二年置,曰廣豐。仁壽初改名焉。建昌開皇九年省並、永修、豫章、新吳四縣入焉。建城有然石。



 南海郡舊置廣州,梁、陳並置都督府。平陳,置總管府。仁壽元年置番州,大業初府廢。統縣十五,戶三萬七千四百八十二。



 南海舊置南海郡。平陳,郡廢;又分置番禺縣,尋廢入焉。大業初置郡。曲江舊置始興郡。平陳廢,十六年又廢湞陽縣入焉。有玉山、銀山。始興齊曰正階,梁改名焉,又置安遠郡,置東衡州。平陳,改郡置大庾縣,又於此置廣州總管。開皇末移向南海,又十六年廢大庾入焉。翁源梁置,陳又置清遠郡。平陳郡廢。增城舊置東官郡。平陳廢。有羅浮山。寶安樂昌梁置,曰梁化,又分置平石縣。開皇十二年省平石入,十八年改焉。四會舊置綏建郡,又
 有樂昌郡。平陳,二郡並廢。大業初又並始昌縣入焉。化蒙大業初廢威城縣入焉。清遠舊置清遠郡,又分置威正、廉平、恩洽、浮護等四縣。平陳並廢,以置清遠縣。又齊置齊康郡,至是亦廢入焉。



 含洭梁置衡州、陽山郡。平陳,州改曰洭州,廢郡。二十年州廢。有堯山。政賓舊置東官郡。平陳,郡廢。懷集新會舊置新會郡。平陳,郡廢,又並盆允,永昌、新建、熙潭、化召、懷集六縣入,為封州。十一年改為允州,後又改為岡州。大業初州廢,並廢封樂縣入。有社山。義寧開皇十年廢新夷、初賓二縣入;又有始康縣,廢入封平。大業初又廢封平入焉。有茂山。



 龍川郡平陳,置循州總管府。大業初府廢。統縣五,戶六千四百二十。



 歸善帶郡。有歸化山、懷安山。河源開皇十一年省龍川縣入焉。又有新豐縣,十八年改曰休吉,大業初省入焉。有龍山、亢山。有修江。博羅興寧海豐有黑龍山。有漲海。



 義安郡梁置東揚州,後改曰瀛州,及陳州廢。平陳,置潮州。統縣五,戶二千六十
 六。



 海陽舊置義安郡。平陳,郡廢。大業初置郡。有風皇山。程鄉潮陽海寧有龍溪山。萬川舊曰義招,大業初改名焉。



 高涼郡梁置高州。統縣九,戶九千九百一十七。



 高涼舊置高涼郡。平陳廢,大業初復置。連江梁置連江郡。平陳,郡廢。梁又置梁封縣,開皇十八年改為義封。梁又置南巴郡。平陳,郡廢為南巴縣。大業初二縣並廢入。電白梁置電白郡。平陳,郡廢。又有海昌郡廢入焉。杜原舊曰杜陵。梁置杜陵郡,又有永寧、宋康二郡。平陳,並廢為縣。十八年改杜陵曰杜原,宋康曰義康。大業二年二縣並廢入杜原。海安舊曰齊安,置齊安郡。平陳,郡廢。開皇十八年改縣名焉。陽春梁置陽春郡。平陳,郡廢。石龍舊置羅州、高興郡。平陳,郡廢。大業初州廢。吳川茂名信安郡平陳,置端州。統縣七,戶一萬七千七百八十七高要舊置高要郡。平陳,郡廢。大業初置信安郡。有定山。端溪舊置晉康郡。



 平陳,郡廢。有端水。樂城
 開皇十二年廢文招、悅成二縣入。平興舊置宋隆郡,領初寧、建寧、熙穆、崇德、召興、崇化、南安等縣。平陳,郡廢,並所領縣入焉。



 又梁置梁泰郡及縣。平陳,郡廢,縣改曰清泰。大業初廢入焉。新興梁置新州、新寧郡。平陳,郡廢。大業初州廢,又廢索盧縣入焉。博林大業初廢撫納縣入。銅陵有流南縣,開皇十八年改曰南流。又有西城縣,大業初廢入。



 永熙郡梁置瀧州。統縣六,戶一萬四千三百一十九。



 瀧水舊置開陽縣,置開陽、平原、羅陽等郡。平陳,郡並廢,以名縣。開皇十八年改平原曰瀧水,羅陽縣為正義。大業初置永熙郡,開陽、正義俱廢入焉。懷德舊曰梁德,置梁德郡。平陳,廢郡。十八年改名懷德。良德陳置,曰務德,後改名焉。安遂梁置建州、廣熙郡,尋廢。州大業初廢。永業梁置永業郡,尋改為縣,後省。開皇十六年又置。永熙大業初並安南縣入。



 蒼梧郡梁置成州,開皇初改為封州。統縣四,戶四千五百七十八。



 封川梁曰梁信,置梁信郡。平陳,郡廢。十八年改為封川。大業初又廢封興縣入焉。都城開皇十二年省威城、晉化二縣入焉。蒼梧舊置蒼梧郡。平陳,郡廢。封陽。



 始安郡梁置桂州。平陳,置總管府。大業元年府廢。統縣十五,戶五萬四千五百一十七。



 始安舊置始安、梁化二郡。平陳,郡並廢。大業初廢興安縣入焉。平樂有目山。



 荔浦建陵陽朔象隋化義熙舊曰齊熙,置齊熙、黃水二郡及東寧州。平陳,郡並廢。十八年改州曰融州,縣曰義熙。大業初州廢,並廢臨䍧、黃水二縣入焉。



 龍城梁置。馬平開皇十二年置象州,大業初州廢。桂林大業初並西寧縣入。陽壽有馬平、桂林、象、韶陽等四郡。平陳,並廢。又有淮陽縣,開皇十八年改曰陽寧。



 大業初省入焉。富川舊置臨賀、樂梁二郡。平陳,並廢,置賀州。大業初州廢,又置臨賀、綏越、蕩山三縣入焉。龍平梁置靜州,梁壽、靜慰二郡。平陳,並廢,又置歸化縣。大業初州廢,又廢歸化、安樂、博勞三縣入焉。豪靜梁置開江、武城二郡,
 陳置逍遙郡。平陳,郡並廢。又有猛陵、開江二縣,大業初並廢入焉。



 永平郡平陳,置藤州。統縣十一,戶三萬四千四十九。



 永平舊置永平郡。平陳,郡廢。大業置郡。武林有燕石山。隋建開皇十九年置。



 安基梁置建陵郡。平陳,郡廢。隋安開皇十九年置。普寧舊曰陰石,梁置陰石郡。



 平陳,郡廢,改縣為奉化。開皇十九年又改名焉。戎成梁置,曰遂成。開皇十一年改名焉。有農山。寧人開皇十五年置,曰安人。十八年改名焉。有壽原山。淳人開皇十九年置。大賓開皇十五年置。賀川開皇十九年置,又陳置建陵、綏越、蒼梧、永建等四郡。平陳,並廢。



 鬱林郡梁置定州,後改為南定州。平陳,改為尹州。大業初改為鬱州。統縣十二,戶五萬九千二百。



 鬱林舊置鬱林郡。平陳,郡廢。大業初又置郡,又廢武平、龍山、懷澤、布山四縣入。鬱平領方梁置領方郡。平陳,郡廢。阿林石南陳置石南郡。平陳,廢郡。



 桂平梁置桂平郡。平陳,郡廢。大業初又廢皇化縣入。馬度安成梁置安成郡。平陳,郡廢。寧浦舊置寧浦郡,梁分立簡陽郡。平陳,郡廢,置簡州。十八年改為緣州。大業二年州廢。樂山梁置樂陽郡。平陳,改為樂陽縣。十八年改名焉。嶺山梁置嶺山郡。平陳,改為嶺縣。十八年改為嶺山。大業初並武緣縣入。有武緣山。宣化舊置晉興郡。平陳,廢為縣。開皇十八年改名焉。



 合浦郡舊置越州。大業初改為祿州,尋改為合州。統縣十一,戶二萬八千六百九十。



 合浦舊置合浦郡。平陳,郡廢。大業初置郡。南昌北流大業初廢陸川縣入。



 封山大業初廢廉昌縣入。定川舊立定川郡。平陳,郡廢。龍蘇舊置龍蘇郡。平陳,郡廢。大業初又並大廉縣入。海康梁大通中,割番州合浦立高州,尋又分立合州。



 大同末,以合肥為合州,此置南合州。平陳,以此為合州,置海康縣。大業初州廢,又廢摸落、羅阿、雷川三縣入。抱成舊曰抱,並置郡。平陳,郡廢。十八年改曰抱成。隋康舊曰齊康,置齊康郡。平陳,郡廢,縣改名焉。扇沙舊有椹縣,開皇十八
 年改為椹川,大業初廢入。鐵杷開皇十年置。



 珠崖郡梁置崖州。統縣十,戶一萬九千五百。



 義倫帶郡。感恩顏盧毗善昌化有藤山。吉安延德寧遠澄邁武德有扶山。



 寧越郡梁置安州,開皇十八年改曰欽州。統縣六,戶一萬二千六百七十。



 欽江舊置宋壽郡。平陳,郡廢。開皇十八年改曰欽江,大業初置寧越郡。安京舊置安京郡。平陳,郡廢。有羅浮山。有武郎江。內亭舊置宋廣郡。平陳,郡廢。



 十七年改曰新化縣,十八年改名焉。南賓開皇十八年置。遵化開皇二十年置。海安梁置,曰安平,置黃州及寧海郡。平陳,郡廢。十八年改州曰玉州。大業初州廢,其年又省海平、玉山二縣入。



 交趾郡舊曰交州。統縣九,戶三萬五十六。



 宋平舊置宋平郡。平陳,郡
 廢。大業初置交趾郡。龍編舊置交趾郡。平陳,郡廢。硃枿舊置武平郡。平陳,郡廢。隆平舊曰武定,置武平郡。平陳,郡廢。開皇十八年縣改名焉。平道舊曰國昌,開皇十二年改名焉。交趾嘉寧舊置興州、新昌郡。平陳,郡廢。十八年改曰峰州,大業初州廢。新昌安人舊曰臨西,開皇十八年改名焉。



 九真郡梁置愛州。統縣七,戶一萬六千一百三十五。



 九真帶郡。有陽山、堯山。移風舊置九真郡。平陳,郡廢。胥浦隆安舊曰高安,開皇十八年改名焉,軍安安順舊曰常樂,開皇十六年改名焉。日南日南郡梁置德州,開皇十八年改曰驩州。統縣八,戶九千九百一十五。



 九德帶郡。咸驩浦陽越常金寧梁置利州。開皇十八年改為智州,大業初州廢。交穀梁置明州,大業初州廢。安遠光安舊曰西安,開皇十八年改名焉。



 比景郡大業元年平林邑,置蕩州,尋改為郡。統縣四,戶一千八百一十五。



 比景硃吾壽泠西手卷海陰郡大業元年平林邑,置農州,尋改為郡。統縣四,戶一千一百。



 新容真龍多農安樂林邑郡大業元年平林邑,置沖州,尋改為郡。統縣四,戶一千二百二十。



 象浦金山交江南極揚州於《禹貢》為淮海之地。在天官,自斗十二度至須女七度,為星紀,於辰在丑,吳、越得其分野。江南之俗,火耕水耨,食魚與稻,以漁獵為業,雖無蓄積之資,然而亦無饑餒。其俗信鬼神,好淫祀,父子或異居,此大抵然也。江
 都、弋陽、淮南、鐘離、蘄春、同安、廬江、歷陽,人性並躁勁,風氣果決,包藏禍害,視死如歸,戰而貴詐,此則其舊風也。自平陳之後,其俗頗變,尚淳質,好儉約,喪紀婚姻,率漸於禮。其俗之敝者,稍愈於古焉。丹陽舊京所在,人物本盛,小人率多商販,君子資於官祿,市厘列肆,埒於二京,人雜五方,故俗頗相類。京口東通吳會,南接江湖,西連都邑,亦一都會也。其人本並習戰,號為天下精兵。俗以五月五日為鬥力之戲,各料強弱相敵,事類講武。宣城、毗陵、吳郡、會稽、餘杭、東陽,其俗亦同。然數郡川澤沃衍,有海陸之饒,珍異所聚,故商賈並湊。其人君子尚禮,庸
 庶敦厖,故風俗澄清,而道教隆洽,亦其風氣所尚也。豫章之俗,頗同吳中,其君子善居室,小人勤耕稼。衣冠之人,多有數婦,暴面市廛,競分銖以給其夫。及舉孝廉,更要富者,前妻雖有積年之勤,子女盈室,猶見放逐,以避後人。



 俗少爭訟,而尚歌舞。一年蠶四五熟,勤於紡績,亦有夜浣紗而旦成布者,俗呼為雞鳴布。新安、永嘉、建安、遂安、鄱陽、九江、臨川、廬陵、南康、宜春,其俗又頗同豫章,而廬陵人厖淳,率多壽考。然此數郡,往往畜蠱,而宜春偏甚。其法以五月五日聚百種蟲,大者至蛇,小者至虱,合置器中,令自相啖,餘一種存者留之,蛇則曰蛇蠱,虱
 則曰虱蠱,行以殺人。因食入人腹內,食其五藏,死則其產移入蠱主之家。三年不殺他人,則畜者自鐘其弊。累世子孫相傳不絕,亦有隨女子嫁焉。干寶謂之為鬼,其實非也。自侯景亂後,蠱家多絕,既無主人,故飛游道路之中則殞焉。



 自嶺已南二十餘郡,大率土地下濕,皆多瘴厲,人尤夭折。南海、交趾,各一都會也,並所處近海,多犀象玳瑁珠璣,奇異珍瑋,故商賈至者,多取富焉。其人性並輕悍,易興逆節,椎結踑踞,乃其舊風。其俚人則質直尚信,諸蠻則勇敢自立,皆重賄輕死,唯富為雄。巢居崖處,盡力農事。刻木以為符契,言誓則至死不改。



 父子
 別業,父貧,乃有質身於子。諸獠皆然。並鑄銅為大鼓,初成,懸於庭中,置酒以招同類。來者有豪富子女,則以金銀為大釵,執以叩鼓,竟乃留遺主人,名為銅鼓釵。俗好相殺,多構仇怨,欲相攻則鳴此鼓,到者如云。有鼓者號為「都老」,群情推服。本之舊事,尉陀於漢,自稱「蠻夷大酋長、老夫臣」,故俚人猶呼其所尊為「倒老」也。言訛,故又稱「都老」云。



 南郡舊置荊州。西魏以封梁為蕃國,又置江陵總管府。開皇初府廢。七年並梁,又置江陵總管,二十年改為荊州總管。大業初廢。統縣一十,戶五萬八千八百三十六。



 江陵帶南郡。開皇初郡廢,大業初復置郡。長楊開皇八年置,並立睦州,十七年州廢。有宜陽山。宜昌
 開皇九年置松州,又省歸化、受陵二縣入。十一年州廢,又省宜都縣入。有丹山,、黃牛山。枝江當陽後周置平州,領漳川、安遠二郡,屬梁蕃。開皇七年改為玉州,九年州郡並廢。梁又置安居縣,開皇十八年改曰昭丘,大業初改曰荊臺,尋廢入。有清溪山。松滋江左舊置河東郡。平陳,郡廢。有涔水。



 長林舊曰長寧縣。開皇十一年省長林縣入,十八年改曰長林。公安陳置荊州。開皇九年省孱陵、永安二縣入。有黃山。有靈溪水。安興舊置廣牧縣,開皇十一年省安興縣入,仁壽初改曰安興。又有定襄縣,大業初廢入。紫陵西魏置華陵縣,後周改名焉。其城南面,梁置都州,又置雲澤縣。大業初州縣俱廢入焉。有硤石山。



 夷陵郡梁置宜州,西魏改曰拓州,後周改曰硤州。統縣三,戶五千一百七十九。



 夷陵帶郡。有馬穴。夷道舊置宜都郡,開皇七年廢。有女觀山。遠安舊曰高安,置汶陽郡。又周改縣曰安遠。開皇七年郡廢。



 竟陵郡舊置郢州。統縣八,戶五萬三千三百八十五。



 長壽
 後周置石城郡,開皇初郡廢,大業初置竟陵郡。又梁置北新州及梁寧等八郡,後周保定中,州及八郡總管廢入焉。有敖山。藍水宋僑立馮翊郡,蓮勺縣,西魏改郡為漢東,縣為藍水。又宋置高陸縣,西魏改曰滶水。開皇初郡廢,大業初省滶水入焉。有唐水。棨川後周置,及置滶川郡。又置清縣,西魏改曰滶陂。開皇初郡廢,大業初省滶陂入焉。漢東齊置,曰上蔡,及置齊興郡。後周郡廢。開皇十八年縣改名焉。有東溫山。清騰梁置,曰梁安,又立崇義郡。後周廢郡。後周又有遂安郡。開皇初廢,七年改名焉。有清騰山。樂鄉舊置武寧郡,西魏置鄀州。又梁置旌陽縣,後改名惠懷,西魏又改曰武山。開皇七年郡廢,大業初州廢,又廢武山入焉。有武陵山。豐鄉西魏置,又置基州及章山郡。開皇七年郡廢,大業初州廢。章山西魏置,曰祿麻,及立上黃郡。開皇七年郡廢,大業初縣改名焉。



 沔陽郡後周置復州,大業初改曰沔州。統縣五,戶四萬一千七百一十四。



 沔陽梁置沔陽、營陽、州城三郡。西魏省州陵、惠懷二縣,置縣曰建興。後周置復州,後又省營陽、
 州城二郡入建興。開皇初州移郡廢,仁壽三年復置州。大業初改建興曰沔陽,州廢,復置沔陽郡焉。監利竟陵舊曰霄城,置竟陵郡。後周改縣曰竟陵。開皇初置復州,仁壽三年州復徙建興。又有京山縣,齊置建安郡,西魏改曰光川,後周郡廢。大業初京山縣又廢入焉。甑山梁置梁安郡。西魏改曰魏安郡,置江州,尋改郡曰汶川。後周置甑山縣,建德二年州廢。開皇初郡廢。有陽臺山。



 漢陽開皇十七年置,曰漢津,大業初改焉。有沌水。



 沅陵郡開皇九年置辰州。統縣五,戶四千一百四十。



 沅陵舊置沅陵郡。平陳,郡廢,大業初復。大鄉梁置。鹽泉梁置。龍檦梁置。



 有武山。辰溪舊曰辰陽。平陳,改名;並廢故夜郎郡,置靜人縣,尋廢。又梁置南陽郡,建昌縣,陳廢縣。開皇初廢郡,置壽州,十八年改為充州,大業初州廢。有郎溪。



 武陵郡梁置武州,後改曰沅州。平陳,為朗州。統縣二,戶三千四百一十六。



 武陵舊置武陵郡。平陳,郡廢,並臨沅、沅南、漢壽三縣置武陵縣。大業初復置武陵郡。有望夷山、龍山。龍陽有白查湖。



 清江郡後周置亭州,大業初改為庸州。統縣五,戶二千六百五十八。



 鹽水後周置縣,並置資田郡,開皇初郡廢,大業初置清江郡。巴山梁置宜都郡、宜昌縣,後周置江州。開皇初置清江縣,十八年改江州為津州,大業初廢州,省清江入焉。清江後周置施州及清江郡。開皇初郡廢,五年置清江縣,大業初州廢。在陽瞿水。開夷後周置,曰烏飛,開皇初改焉。建始後周置業州及軍屯郡。開皇初郡廢,五年置縣,大業初州廢。



 襄陽郡江左並僑置雍州。西魏改曰襄州,置總管府。大業初府廢。統縣十一,戶九萬九千五百七十七。



 襄陽帶襄陽郡。開皇初郡廢,大業初復置。有鐘山、峴山、鳳林山。安養西魏置河南郡,後周廢樊城。山都二縣入,開皇初郡廢焉。穀城舊曰義城,置義城郡。



 後周廢郡,
 開皇十八年改縣名焉。又梁有築陽,開皇初廢,又梁有興國,義城二郡,並西魏廢。有穀城山、闕林山。上洪宋僑立略陽縣,梁又立德廣郡。西魏改縣曰上洪。開皇初郡廢。又梁置新野郡,西魏改曰威寧,後周廢。有亞山。率道梁置。漢南宋曰華山,置華山郡。西魏改縣為漢南,屬宜城郡。後周廢武建郡及惠懷、石梁、歸仁、鄢等四縣入,後省宜城郡入武泉。又梁置秦南郡,後周並武泉縣俱廢。有石梁山。陰城西魏置酂城郡,後周廢。又梁置南陽郡,西魏改為山都郡,後周省。義清梁置,曰穰縣。西魏改為義清,屬歸義郡。後周廢郡及左安、開南、歸仁三縣入焉。又有武泉郡,開皇初廢。有柤山、靈山。有檀溪水、襄水。南漳西魏並新安、武昌、武平、安武、建平五縣置,初曰重陽,又立南襄陽郡。後周置沮州,尋廢,復改重陽縣曰思安。開皇初郡廢,十八年改縣曰南漳。有荊山。常平西魏置,曰義安,置長湖郡,後改縣曰常平。開皇初郡廢。又後魏置旱停縣,大業初廢。鄀舂陵郡後魏置南荊州,西魏改曰昌州。統縣六,戶四萬二千八百四十
 七。



 棗陽舊曰廣昌,並置廣昌郡。開皇初郡廢,仁壽元年縣改名焉。大業初置舂陵郡。又西魏置東荊州,尋廢。有霸山。有溲水。舂陵舊置安昌郡,開皇初郡廢。又後魏置豐良縣,大業初廢。有石鼓山。有四望水。清潭有大洪山。有淯水。湖陽後魏置西淮安郡及南襄州,後郡廢,州改為南平州。西魏改曰升州,後又改曰湖州。



 後周改置升平郡。開皇初郡廢。仁壽初改曰升州,大業初州廢。又後魏置順陽郡,西魏改為柘林郡。後周省郡,改縣曰柘林。大業初縣廢入焉。有蓼山。上馬後魏置,曰石馬,後訛為上馬,因改焉。有鐘離縣,置洞州、洞川郡。後周州廢,開皇初郡廢。十八年改鐘離曰洞川縣,大業初廢入焉。蔡陽梁置蔡陽郡,後魏置南雍州。西魏改曰蔡州,分置南陽縣,後改曰雙泉;又置千金郡、瀴源縣。開皇初郡並廢,大業初州廢,雙泉、瀴源二縣並廢入焉。有唐子山、大鼓山。有瀴水。



 漢東郡西魏置並州,後改曰隋州。統縣八,戶四萬七千一百九十三。



 隋舊置隨郡,西魏又析置水厥西郡及水厥西縣。梁又置曲陵郡。開皇初郡並廢。



 大業初廢水厥西縣,尋置漢東郡。土山梁曰
 龍巢,置土州、東西二永寧、真陽三郡,及置石武縣。後周廢三郡為齊郡,改龍巢曰左陽;又有阜陵縣,改為漳川縣。開皇初郡廢。十八年改左陽為真陽,石武為宜人。大業初又改真陽為土山,州及宜人、漳川並廢入焉。唐城後魏曰水厥西,置義陽郡。西魏改水厥西為下溠,又立肆州,尋曰唐州。後周省均、款、溳、歸四州入,改曰唐州。又有東魏南豫州,至是改為水厥川郡,又置清嘉縣。開皇初郡並廢。十六年改下溠曰唐城,大業初州及諸縣並廢入焉。有清臺山。有水厥水。安貴梁置,曰定陽,又置北郢州。西魏改定陽曰安貴,改北郢州為款州,又尋廢為溳水郡,別置戟城郡及戟城縣。後廢戟城郡,改戟城縣曰橫山。開皇初溳水郡廢,大業初又廢橫山縣入焉。順義梁置北隨郡。西魏改為南陽,析置淮南郡;以厲城、順義二縣立冀州,尋改為順州;又置安化縣。開皇初郡並廢,十八年改安化曰寧化。大業初州廢,改厲城為順義,其舊順義及寧化,並廢入焉。有浮山。平林梁置上明郡,開皇初廢。有漂水。上明西魏置,曰洛平縣,開皇十八年改名焉。有鸚鵡山。光化舊曰安化,西魏改為新化,後周又改焉。



 安陸郡梁置南司州,尋罷。西魏置安州總管府,開皇十四年府廢。統縣八,戶六萬八千四十二。



 安陸舊置安陸郡。開皇初郡廢,大業初復置郡。有舊永陽縣,西魏改曰吉陽,至是廢入。孝昌西魏置岳州及岳山郡,後周州郡並廢。又有嵒岳郡,開皇初廢。有鳳皇岡。吉陽梁置,曰平陽,及立汝南郡。西魏改郡為董城,改縣曰京池。後周置嵒州,尋州郡並廢。大業初改縣曰吉陽。又梁置義陽郡,西魏改為南司州,尋廢。



 應陽西魏置,曰應城,又置城陽郡。開皇初郡廢,大業初縣改名焉。有潼水、溫水。



 雲夢西魏置。京山舊曰新陽,梁置新州、梁寧郡。西魏改州為溫州,改縣為角陵,又置盤陂縣。開皇初郡廢,大業初州廢;改角陵曰京山,廢盤陂入焉。有角陵山、京山。富水舊曰南新市。西魏改為富水,又置富水郡。開皇初郡廢。應山梁置,曰永陽,仍置應州,又有平靖郡。西魏又置平靖縣。開皇初郡廢,大業初州廢,又省平靖縣入焉。有大龜山、安居山。



 永安郡後齊置衡州,陳廢,後周又置,開皇五年改曰黃州。統縣四,戶二萬八千
 三百九十八。



 黃岡齊曰南安,又置齊安郡。開皇初郡廢,十八年改縣曰黃岡。又後齊置巴州,陳廢。後周置,曰弋州,統西陽、弋陽、邊城三郡。開皇初州郡並廢,大業初置永安郡。黃陂後齊置南司州。後周改曰黃州,置總管府,又有安昌郡。開皇初府廢。



 又後齊置滻州,陳廢之。木蘭梁曰梁安,置梁安郡,又有永安、義陽二郡。後齊置湘州,後改為北江州。開皇初別置廉城縣,尋及州、二郡相次並廢。十八年改縣曰木蘭。麻城梁置信安,又有北西陽縣。陳廢北西陽,置定州。後周改州曰亭州,又有建寧、陰平、定城三郡。開皇初州郡並廢,十八年縣改名焉。有陰山。



 義陽郡齊置司州。梁曰北司州,後復曰司州。後魏改曰郢州,後周改曰申州,大業二年為義州。統縣五,戶四萬五千九百三十。



 義陽舊曰平陽,置宋安郡。開皇初郡廢,縣改名焉。大業初置義陽郡。有大龜山、金山。鐘山舊曰盟阜。後齊改曰齊安,仍置郡。開皇初郡廢,縣改曰鐘山。有鐘山。羅山後齊置,曰高安。開皇初廢,十六年置,曰羅山。禮山舊曰東隨,開
 皇九年改焉。有關官。有禮山。淮源後齊置,曰慕化,置淮安郡。開皇初郡廢,大業初縣改名焉。有油水。



 九江郡舊置江州。統縣二,戶七千六百一十七。



 湓城舊曰柴桑,置尋陽郡。梁又立汝南縣。平陳,郡廢,又廢汝南、柴桑二縣,立尋陽縣,十八年改曰彭蠡。大業初置郡,縣改名焉。有巢湖、彭蠡湖。有廬山、望夫山。彭澤梁置太原郡,領彭澤、晉陽、和城、天水。平陳,郡縣並廢,置龍城縣。開皇十八年改名焉。有釣磯。



 江夏郡舊置郢州。梁分置北新州,尋又分北新立土、富、洄、泉、豪五州。平陳,改置鄂州。統縣四,戶一萬三千七百七十一。



 江夏舊置江夏郡。平陳,郡廢,大業初復置。有烽火山、塗水。武昌舊置武昌郡。平陳,郡廢,又廢西陵、鄂二縣入焉。有樊山、白山。永興陳曰陽新。平陳,改曰富川。開皇十一年廢永興縣入,十八年改名焉。有五龍山。蒲圻梁置上雋郡,又有沙陽縣,置沙州,州尋廢。平陳,郡廢。有石頭山、魚嶽山、鮑山。



 澧陽郡平陳,置松州,尋改為澧州。統縣六,戶八千九百六。



 澧陽平陳,置縣,大業初置郡。有藥山。有油水。石門舊置天門郡。平陳,郡廢。孱陵舊曰作唐,置南平郡。平陳,郡廢,縣改名焉。安鄉舊置義陽郡。平陳,郡廢。有皇山。崇義後周置衡州。開皇中置縣,名焉。十八年改州曰崇州,大業初州廢。有澧水。慈利開皇中置,曰零陵,十八年改名焉。有始零山。



 巴陵郡梁置巴州。平陳,改曰岳州,大業初改曰羅州。統縣五,戶六千九百三十四。



 巴陵舊置巴陵郡。平陳,郡廢,大業初復置郡。華容舊曰安南,梁置南安湘郡,尋廢。開皇十八年縣改名焉。沅江梁置,曰藥山,仍為郡。平陳,郡廢,縣改曰安樂,十八年改曰沅江。湘陰梁置岳陽郡及羅州,陳廢州。平陳,廢郡及湘陰入岳陽縣,置玉州。尋改岳陽為湘陰,廢玉山縣入焉。十二年廢玉州。羅開皇九年廢吳昌、湘濱二縣入。有水貿水、汩水。



 長沙郡舊置湘州,平陳置潭州總管府,大業初府廢。統縣四,戶一萬四千二
 百七十五。



 長沙舊曰臨湘,置長沙郡。平陳,郡廢,縣改名焉。有銅山、錫山。衡山舊置衡陽郡。平陳,郡廢,並衡山、湘鄉、湘西三縣入焉。益陽平陳,並新康縣入焉。



 有浮梁山。邵陽舊置邵陵郡。平陳,郡廢,並扶夷、都梁二縣入焉。



 衡山郡平陳,置衡州。統縣四,戶五千六十八。



 衡陽舊置湘東郡。平陳,郡廢,並省臨烝、新城、重安三縣入焉。有衡山、武水、連水。洡陰舊曰洡陽。平陳,改名焉。有肥水、酃水。湘潭平陳,廢茶陵、攸水、陰山、建寧四縣入焉。有武陽山。有歷水。新寧有宜溪水、舂江。



 桂陽郡平陳,置郴州。統縣三,戶四千六百六十六。



 郴舊置桂陽郡。平陳,郡廢,大業初復置。有萬歲山。有溱水。臨武有華陰山。



 盧陽陳置盧陽郡。平陳,郡廢。有淥水。



 零陵郡平陳初,置永州總管府,尋廢府。統縣五,戶六千八百四十五。



 零陵舊曰泉陵,置零陵郡。平陳,郡廢,又廢應陽、永昌、祁陽三縣入焉。大業初復置郡。湘源平
 陳,廢洮陽、灌陽、零陵三縣置縣。有黃華山。有觀水、湘水、洮水。永陽舊曰營陽,梁置永陽郡。平陳,郡廢,並營浦、謝沐二縣入焉。營道平陳,並冷道、舂陵二縣入。有九疑山、營山。馮乘有馮水。



 熙平郡平陳,置連州。統縣九,戶一萬二百六十五。



 桂陽梁置陽山郡。平陳,郡廢。大業初置熙平郡。有貞女山、方山。有盧水、洭水。陽山有斟水。連山梁置,曰廣德,隋改曰廣澤,仁壽元年改名焉。有黃連山。宣樂梁置,曰梁樂,並置梁樂郡,平陳,郡廢,十八年改為宣樂。游安熙平舊置齊樂郡,平陳,郡廢。武化梁置。桂嶺舊曰興安,開皇十八年改名焉。開建梁置南靜郡,平陳,郡廢。



 《尚書》:「荊及衡陽惟荊州。」上當天文,自張十七度至軫十一度,為鶉首,於辰在巳,楚之分野。其風俗物產,頗同揚州。其人率多勁悍決烈,蓋亦天性然也。



 南郡、夷陵、竟陵、
 沔陽、沅陵、清江、襄陽、舂陵、漢東、安陸、永安、義陽、九江、江夏諸郡,多雜蠻左,其與夏人雜居者,則與諸華不別。其僻處山谷者,則言語不通,嗜好居處全異,頗與巴、渝同俗。諸蠻本其所出,承盤瓠之後,故服章多以班布為飾。其相呼以蠻,則為深忌。自晉氏南遷之後,南郡、襄陽,皆為重鎮,四方湊會,故益多衣冠之緒,稍尚禮義經籍焉。九江襟帶所在,江夏、竟陵、安陸,各置名州,為籓鎮重寄,人物乃與諸郡不同。大抵荊州率敬鬼,尤重祠祀之事,昔屈原為制《九歌》,蓋由此也。屈原以五月望日赴汨羅,土人追到洞庭不見,湖大船小,莫得濟者,乃歌曰:「何由
 得渡湖!」因爾鼓棹爭歸,競會亭上,習以相傳,為競渡之戲。其迅楫齊馳,棹歌亂響,喧振水陸,觀者如云,諸郡率然,而南郡、襄陽尤甚。二郡又有牽鉤之戲,雲從講武所出,楚將伐吳,以為教戰,流遷不改,習以相傳。鉤初發動,皆有鼓節,群噪歌謠,振驚遠近,俗云以此厭勝,用致豐穰。



 其事亦傳於他郡。梁簡文之臨雍部,發教禁之,由是頗息,其死喪之紀,雖無被發袒踴,亦知號叫哭泣。始死,即出尸於中庭,不留室內。斂畢,送到山中,以十三年為限。先擇吉日,改入小棺,謂之拾骨。拾骨必須女婿,蠻重女婿,故以委之。



 拾骨者,除肉取骨,棄小取大。當葬之夕,
 女婿或三數十人,集會於宗長之宅,著芒心接籬,名曰茅綏。各執竹竿,長一丈許,上三四尺許,猶帶枝葉。其行伍前卻,皆有節奏,歌吟叫呼,亦有章典。傳雲盤瓠初死,置之於樹,乃以竹木刺而下之,故相承至今,以為風俗。隱諱其事,謂之刺北斗。既葬設祭,則親疏咸哭,哭畢,家人既至,但歡飲而歸,無復祭哭也。其左人則又不同,無衰服,不復魄。始死,置尸館舍,鄰里少年,各持弓箭,繞尸而歌,以箭扣弓為節。其歌詞說平生樂事,以到終卒,大抵亦猶今之挽歌。歌數十闋,乃衣衾棺斂,送往山林,別為廬舍,安置棺柩。亦有於村側瘞之,待二三十喪,總葬
 石窟。長沙郡又雜有夷蜒,名曰莫徭,自云其先祖有功,常免徭役,故以為名。其男子但著白布褌衫,更無巾褲;其女子青布衫、班布裙,通無鞋屩。婚嫁用鐵鈷莽為聘財。武陵、巴陵、零陵、桂陽、澧陽、衡山、熙平皆同焉。其喪葬之節,頗同於諸左云。



\end{pinyinscope}