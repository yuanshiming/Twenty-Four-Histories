\article{卷十二志第七 禮儀七}

\begin{pinyinscope}

 高祖初即位,將改周制,乃下詔曰:「宣尼制法,雲行夏之時,乘殷之輅。弈葉共遵,理無可革。然三代所尚,眾論多端,或以為所建之時,或以為所感之瑞,或當其行色,因以從之。今雖夏數得天,歷代通用,漢尚於赤,魏尚於黃,驪馬玄牲,已弗相踵,明不可改,建寅歲首,常服於黑。朕初受天命,赤雀來儀,兼姬周已還,於茲六代,三正回復,
 五德相生,總以言之,並宜火色。垂衣已降,損益可知,尚色雖殊,常兼前代。其郊丘廟社,可依袞冕之儀,朝會衣裳,宜盡用赤。昔丹烏木運,姬有大白之旂,黃星土德,曹乘黑首之馬,在祀與戎,其尚恆異。今之戎服,皆可尚黃,在外常所著者,通用雜色。祭祀之服,須合禮經,宜集通儒,更可詳議。」太子庶子、攝太常少卿裴政奏曰:「竊見後周制冕,加為十二,即與前禮數乃不同,而色應五行,又非典故。謹案三代之冠,其名各別。六等之冕,承用區分,璪玉五採,隨班異飾,都無迎氣變色之文。唯《月令》者,起於秦代,乃有青旂赤玉,白駱黑衣,與四時而色變,全不
 言於弁冕。五時冕色,《禮》既無文,稽於正典,難以經證。且後魏已來,制度咸闕。天興之歲,草創繕修,所造車服,多參胡制。故魏收論之,稱為違古,是也。周氏因襲,將為故事,大象承統,咸取用之,輿輦衣冠,甚多迂怪。今皇隋革命,憲章前代,其魏、周輦輅不合制者,已敕有司盡令除廢,然衣冠禮器,尚且兼行。乃有立夏袞衣,以赤為質,迎秋平冕,用白成形,既越典章,須革其謬。謹案《續漢書·禮儀志》云『立春之日,京都皆著青衣』,秋夏悉如其色。逮於魏、晉,迎氣五郊,行禮之人,皆同此制。考尋故事,唯幘從衣色。今請冠及冕,色並用玄,唯應著幘者,任依漢、晉。」制
 曰:「可。」



 於是定令,採用東齊之法。乘輿袞冕,垂白珠十有二旒,以組為纓,色如其綬,黈纊充耳,玉笄。玄衣,纁裳。衣,山、龍華蟲、火、宗彞五章;裳,藻、粉米、黼、黻四章。衣重宗彞,裳重黼黻,為十二等。衣褾、領織成升龍,白紗內單,黼領,青褾、襈、裾。革帶,玉鉤灊,大帶,素帶硃里,紕其外,上以硃,下以綠。



 「X隨裳色,龍、火、山三章。鹿盧玉具劍,火珠鏢首。白玉雙佩,玄組。雙大綬,六採,玄黃赤白縹綠,純玄質,長二丈四尺,五百首,廣一尺;小雙綬,長二尺六寸,色同大綬,而首半之,間施三玉環。硃襪,赤舄,舄加金飾。祀圓丘、方澤、感帝、明堂、五郊、雩、蠟、封禪、朝日、夕月、宗廟、社稷、籍
 田、廟遣上將、徵還飲至、元服、納后、正月受朝及臨軒拜王公,則服之。通天冠,加金博山,附蟬,十二首,施珠翠,黑介幘,玉簪導。絳紗袍,深衣制,白紗內單,皁領、褾、襈、裾,絳紗蔽膝,白假帶,方心曲領。其革帶、劍、佩、綬、舄,與上同。若未加元服,則雙童髻,空頂黑介幘,雙玉導,加寶飾。朔日受朝、元會及冬會、諸祭還,則服之。武弁,金附蟬,平巾幘,餘服具服。講武、出征、四時蒐狩、大射、祃、類、宜社、賞祖、罰社、纂嚴,則服之。黑介幘,白紗單衣,烏皮履,拜陵則服之。白紗帽,白練裙襦,烏皮履,視朝、聽訟及宴見賓客,皆服之。白帢,白紗單衣,烏皮履,舉哀則服之。



 神璽,寶而不用。受命璽,封禪則用之。「皇帝行璽」,封命諸侯及三師、三公則用之。「皇帝之璽」,賜諸侯及三師、三公書則用之。「皇帝信璽」,徵諸夏兵則用之。「天子行璽」,封命蕃國之君則用之。「天子之璽」,賜蕃國之君書則用之。「天子信璽」,徵蕃國兵則用之。常行詔敕,則用內史門下印。



 皇帝臨臣之喪,三品已上,服錫衰;五等諸侯,緦衰;四品已下,疑衰。



 皇太子袞冕,垂白珠九旒,青纊充耳,犀笄。玄衣,纁裳。衣,山、龍、華蟲、火、宗彞五章;裳,藻、粉米、黼、黻四章。織成為之。白紗內單,黼領,青褾、襈、裾。革帶,金鉤灊,大帶,素帶不硃
 里,亦紕以硃綠。黻隨裳色,火、山二章。



 玉具劍,火珠鏢首。瑜玉雙佩,硃組。雙,大綬,四採,赤白縹紺,純硃質,長一丈八尺,三百二十首,廣九寸;小雙綬,長二尺六寸,色同大綬,而首半之,間施二玉環。硃襪,赤舄,以金飾。侍從皇帝祭祀及謁廟、元服、納妃,則服之。



 遠游三梁冠,加金附蟬,九首,施珠翠,黑介幘,纓翠緌,犀簪導。絳紗袍,白紗內單,皁領、褾、襈、裾,白假帶,方心曲領,絳紗蔽膝,襪,舄。其革帶、劍、佩、綬與上同。未冠則雙童髻,空頂黑介幘,雙玉導,加寶飾。謁廟、還宮、元日朔日入朝、釋奠,則服之。



 遠游冠,公服,絳紗單衣,革帶,金鉤灊,假帶,方心。紛長六尺四寸,廣二寸四分,色同其綬。金縷鞶囊,襪履。五日常朝,則服之。



 白帢,單衣,烏皮履,為宮臣舉哀,則服之。



 皇太子璽,宮內大事用之。小事用左、右庶子印。



 皇太子臨吊三師、三少,則錫衰;宮臣四品已上,緦衰;五品已下,疑衰。



 袞冕,青珠九旒,以組為纓,色如其綬。自此已下,纓皆如之。服九章,同皇太子。王、國公、開國公初受冊,執贄,入朝,祭,親迎,則服之。三公助祭者亦服之。



 冕,侯八旒,伯七旒。服七章。衣,華蟲、火、宗彞三章;裳,藻、粉、米、黼黻四章。八旒者,重宗彞。侯、伯初受冊,執贄,入朝,祭,親迎,則服之。



 毳冕,子六旒,男五旒。服五章。衣,宗彞、藻粉米三章,裳、黼、黻二章。



 六旒者裳重黻子、男初受冊,執贄,入朝,祭,親迎,則服之。



 衣黹冕,三品七旒,四品六旒,五品五旒。服三章。七旒者,衣粉米一章為三重,裳黼、黻二章各二重。六旒者,減黼一重。五旒,又減黻一重。正三品已下,從五品已上,助祭則服之。



 自王公已下服章,皆繡為之。祭服冕,皆簪導、青纊充耳。玄衣,纁裳,白紗內單,黼領,衣黹冕已下,內單青領。青褾、襈、裾。革帶,鉤灊,大帶,王、三公及公、侯、伯、子、男,素帶,不硃里,皆紕其外,上以硃,下以綠。正三品已下,從五品已上,素
 帶,紕其垂,外以玄,內以黃。紐約皆用青組。硃「X凡「X皆隨裳色,袞、鷩、毳,火、山二章。衣黹,山一章。劍,佩,綬,襪,赤舄。



 爵弁,玄纓無旒,從九品已上,助祭,則服之。其制服簪導,玄衣、薰裳無章,白絹內單,青領、褾、襈、裾,革帶,大帶,練帶紕其垂,內外以緇。紐約用青組。



 爵韠,襪,赤履。



 武弁,平巾幘,諸武職及侍臣通服之。侍臣加金璫附蟬,以貂為飾,侍左者左珥,右者右珥。



 遠游三梁冠,黑介幘,諸王服之。



 進賢冠,黑介幘,文官服之。從三品已上三梁,從五品已上兩梁,流內九品已上一梁。



 法冠,一名獬豸冠,鐵為柱,其上施珠兩枚,為獬豸角形。法官服之。



 高山冠,謁者服之。



 卻非冠,門者及禁防伺非服之。



 黑介幘,平巾黑幘,應服者,並上下通服之。庖人則綠幘。



 白帢,白紗單衣,烏皮履,上下通服之。



 委貌冠,未冠則雙童髻,空頂黑介幘,皆深衣,青領,烏皮履。國子太學四門生服之。



 朝服,亦名具服冠,幘簪導,白筆,絳紗單衣,白紗內單,皁領、袖,皁襈,革帶,鉤灊,假帶,曲領方心,絳紗蔽膝,襈,舄,綬,劍,佩。從
 五品已上,陪祭、朝饗、拜表,凡大事則服之。六品已下,從七品已上,去劍、佩、綬,餘並同。



 自餘公事,皆從公服。亦名從省服冠,幘,簪導,絳紗單衣,革帶,鉤灊,假帶,方心,襪,履,紛,鞶囊。從五品已上服之。



 絳褠衣公服,鷿衣即單衣之不垂胡也。袖狹,形直如鷿內。餘同從省。流外五品已下、九品已上服之。



 綬,王,纁硃綬,四採,赤黃縹紺,純硃質,纁文織,長一丈八尺,二百四十首,廣九寸。公,玄硃綬,四採,玄赤縹紺,純硃質,玄文織,長一丈八尺,二百四十首,廣九寸。侯、伯,青硃
 綬,四採,青赤白縹,純硃質,青文織,長一丈六尺,百八十首,廣八寸。子、男,素硃綬,三採,青赤白,純硃質,白文織成,一丈四尺,百四十首,廣七寸。正、從一品,綠綟綬,四採,綠紫黃赤,純綠質,長一丈八尺,二百四十首,廣九寸。從三品已上,紫綬,三採,紫黃赤,純紫質,長一丈六尺,百八十首,廣八寸。銀青光祿大夫,朝議大夫及正、從四品,青綬,三採,青白紅,純青質,長一丈四尺,百四十首,廣七寸。正、從五品,墨綬,二採,青紺,純紺質,長一丈二尺,百首,廣六寸。自王公已下,皆有小雙綬,長二尺六寸,色同大綬,而首半之。正、從一品,施二玉環,已下不合。其有綬者則有
 紛,皆長六尺四寸,廣二寸四分,各隨其綬色。



 鞶囊,二品已上金縷,三品金銀縷,四品及開國男銀縷,五品彩縷。官無綬者,則不合劍佩。一品及五等諸侯,並佩山玄玉。五品已上,佩水蒼玉。



 年高致仕及以理去官,被召謁見,皆服前官從省服。州郡秀孝,試見之日,皆假進賢一梁冠,絳公服。



 隱居道素之士,被召入謁見者,黑介幘,白單衣,革帶,烏皮履。



 左右衛、左右武衛、左右武候大將軍、領左右大將軍,並武弁,絳朝服,劍,佩,綬。侍從則平巾幘,紫衫,大口褲褶,金
 玳瑁裝兩襠甲。唯左右武衛大將軍執赤檉杖。左右衛、左右武衛、左右武候將軍、領左右將軍、左右監門衛將軍、太子左右衛、左右宗衛、左右內等率、左右監門郎將及諸副率,並武弁,絳朝服,劍,佩,綬。侍從則平巾幘,紫衫,大口褲,金裝兩襠甲。唯左右武衛將軍、太子左右宗衛率,執白檀杖。



 直閣將軍、直寢、直齋、太子直閣,武弁,絳朝服,劍,佩,綬。侍從則平巾幘,絳衫,大口褲褶,銀裝兩襠甲。



 皇后首飾,花十二樹。皇太子妃,公主,王妃,三師、三公及公夫人,一品命婦,並九樹。侯夫人,二品命婦,並八樹。伯
 夫人,三品命婦,並七樹。子夫人,世婦及皇太子昭訓,四品已上官命婦,並六樹。男夫人,五品命婦,五樹。女御及皇太子良娣,三樹。自皇后已下,小花並如大花之數,並兩博鬢也。



 皇后褘衣,深青織成為之。為翬翟之形,素質,五色,十二等。青紗內單,黼領,羅縠褾、襈,蔽膝,隨裳色,用翟為章,三等。大帶,隨衣色,硃里,紕其外,上以硃錦,下以綠錦。紐約用青組。以青衣,革帶,青襪、舄舄加金飾白玉佩,玄組、綬。章採尺寸,與乘輿同。祭及朝會,凡大事則服之。



 鞠衣,黃羅為之。應服者皆同其蔽膝、大帶及衣、革帶、舄,隨衣色。餘與褘衣同,唯無雉。親蠶則服之。應服者皆以助祭青衣,青羅為之,制與鞠衣同。去花、大帶及佩綬。以禮見
 皇帝,則服之。



 硃衣,緋羅為之,制如青衣。宴見賓客則服之。



 皇太后服與皇后同。皇太后璽,不行用,若封令書,則用宮官之印。



 皇后璽,不行用,若封令書,則用內侍之印。



 皇太子妃褕翟,青織成為之。為搖翟之形,青質,五色,九等。青紗內單,黼領,羅縠褾、襈,蔽膝,隨衣色,以搖翟為章,三等。大帶,隨衣色,下硃里,紕其外,上以硃錦,下以綠錦。紐約用青組。以青衣,革帶,青襪、舄,舄加金飾瑜玉佩,純硃綬。章採尺寸,與皇太子同。助祭朝會,凡大事則服之。亦有鞠衣。



 皇太子妃璽,不行用,若封書,則用典內之印。



 公主,王妃,三師、三公及公侯伯夫人,服褕翟。繡為之。公主,王妃,三師三公及公夫人為九等,侯夫人八等,伯夫人七等。助祭朝會,凡大事則服之。亦有鞠衣。



 子、男夫人,服闕翟。緋羅為之。刻赤繒為翟形,不繡,綴於服上。子夫人六等,男夫人五等。助祭朝會,凡大事則服之。亦有鞠衣。



 諸王、公、侯、伯、子、男之母,與妃、夫人同。其郡縣君,各視其夫及子。



 若郡縣君品高及無夫、子者,準品。



 嬪及從三品已上官命婦,青服。制與褕翟同,青羅為之,唯無雉。助祭朝會,凡大事則服之。亦有鞠衣。



 世婦及皇太子昭訓,從五品已上官命婦,服青服。助祭
 從蠶朝會,凡大事則服之。



 女御及皇太子良媛,硃服。制與青服同,去佩綬。助祭從蠶朝會,凡大事則服之。



 六尚,硃絲布公服。助祭從蠶朝會,凡大事則服之。



 六司、六典及皇太子三司、三典、三掌,青紗公服。助祭從蠶朝會,凡大事則服之。



 佩綬,嬪同九卿,世婦及皇太子昭訓同五品,公主、王妃同諸王,三師、三公、五等國夫人及從五品已上官命婦,皆準其夫。無夫者準品。



 定令訖。



 高祖元正朝會,方禦通天服,郊丘宗廟,盡用龍袞衣,大
 裘毳衣黹,皆未能備。



 至平陳,得其器物,衣冠法服,始依禮具。然皆藏御府,弗服用焉。百官常服,同於匹庶,皆著黃袍,出入殿省。高祖朝服亦如之,唯帶加十三環,以為差異。蓋取於便事。及大業元年,煬帝始詔吏部尚書牛弘、工部尚書宇文愷、兼內史侍郎虞世基、給事郎許善心、儀曹郎袁朗等,憲章古制,創造衣冠,自天子逮於胥皁,服章皆有等差。若先所有者,則因循取用,弘等議定乘輿服,合八等焉。



 大裘冕之制,案《周禮》,大裘之冕,無旒。《三禮衣服圖》:「大裘而冕,王祀昊天上帝及五帝之服。」至秦,除六冕,唯留玄
 冕。漢明帝永平中,方始創制。



 董巴《志》云:「漢六冕同制皆闊七寸,長尺二寸,前圓後方。」於是遂依此為大裘冕制,青表,硃里,不施旒纊,不通於下。其大裘之服,案《周官》注「羔裘也」。



 其制,準《禮圖》,以羔正黑者為之,取同色繒以為領袖。其裳用纁,而無章飾,絳襪,赤舄。祀圓丘、感帝、封禪、五郊、明堂、雩、蠟,皆服之。



 袞冕之制,案《禮·玉藻》「十有二旒」。《大戴禮》云:「冕而加旒,以蔽明也,琇纊塞耳,以蔽聰也。」又《禮含文嘉》:「前後邃延,不視邪也,加以黈纊,不聽讒也。」三王之冕,既不通制,故夫子云:「行夏之時,服周之冕。」今以採綖貫珠,為旒十二。邃
 延者,出冕前後而下垂之,旒齊於髆,纊齊於耳,組為纓,玉笄導。其為服之制,案《釋名》云:「袞,卷也」,謂畫龍於上也。是時虞世基奏曰:後周故事,升日月於旌旗,乃闕三辰,而章無十二。但有山、龍、華蟲作繪,宗彞、藻、火、粉米、黼、黻,乃與三公不異。開皇中,就裏欲生分別,故衣重宗彞,裳重黼黻,合重二物,以就九章,為十二等。但每一物,上下重行。袞服用九,鷩服用七,今重此三物,乃非典故。且周氏執謙,不敢負於日月,所以綴此三象,唯施太常,天王袞衣,章乃從九。但天子譬日,德在照臨,辰為帝位,月主正後,負此三物,合德齊明,自古有之,理應無惑。周執謙
 道,殊未可依,重用宗彞,又乖法服。今準《尚書》:「予欲觀古人之服,日、月、星辰、山、龍、華蟲作會,宗彞、藻、火、粉、米、黼黻絺繡。」具依此,於左右髆上為日月各一,當後領下而為星辰,又山、龍九物,各重行十二。又近代故實,依《尚書大傳》:「山龍純青,華蟲純黃,作會;宗彞純黑,藻純白,火純赤。」以此相間,而為五採。鄭玄議已自非之,云:「五採相錯,非一色也。」今並用織成於繡,五色錯文。準孔安國,衣質以玄,加山、龍、華蟲、火、宗彞等,並織成為五物,裳質以纁,加藻、粉米、黼、黻之四。衣裳通數,此為九章,兼上三辰,而備十二也。衣褾、領上各帖升龍,漢、晉以來,率皆如此。既是
 先王法服,不可乘於夏制,征而用之,理將為允。



 墨敕曰:「可。」承以單衣。又案董巴《輿服志》宗廟冕服云:「絳領、袖為內單衣。」又《車服雜記》曰:「天子釋奠、郊祭而單衣,以絳緣。」今用白紗為內單,黼領,絳褾,青裾及襈。革帶,玉鉤灊,大帶硃里,紕其外。紐約用組,上加硃「X。又案《說文》:「韠,「X也。所以蔽前。」《禮記》曰:「有虞氏「X,夏后氏山,殷火,周龍章。」鄭玄曰:「冕之「X也,舜始作之,以尊祭服。



 禹、湯至周,增以文飾。」《禮記》曰:「君硃韠。」鄭曰:「韠象裳色。」今依《白武通注》,以蔽裳前,上闊一尺,象天數也;下闊二尺,象地數也;長三尺,象三才也;加龍章山火,以備三代之法也。於是制袞冕
 之服,玄衣,纁裳,合九章為十二等。白紗內單,黼領,青褾、襈。革帶,玉鉤灊,大帶,「X,鹿盧玉具劍,火珠鏢首,白玉雙佩,玄組,大、小綬。硃襪,赤舄,舄飾以金。宗廟、社稷、籍田、方澤、朝日、夕月、遣將授律、徵還飲至、加元服、納后、正冬受朝、臨軒拜爵,皆服之。



 通天冠之制,案董巴《志》:「冠高九寸,形正豎,頂少邪卻,後乃直下為鐵卷梁,前有高山。」故《禮圖》或謂之高山冠也。《晉起居注》,成帝咸和五年,制詔殿內曰:「平天、通天冠,並不能佳,可更修理之。」雖在《禮》無文,故知天子所冠,其來久矣。又徐氏《輿服注》曰:「通天冠,高九寸,黑介幘,金博山。」



 徐爰亦曰:「博山附蟬,謂之金顏。」今制依此,不通於下,獨天子元會臨軒服之。



 其服絳紗袍,深衣制,白紗內單,皁領、褾、裾、襈,絳紗蔽膝,白假帶,方心曲領。其劍、佩、綬、舄、革帶,皆與上同。元冬饗會、諸祭還,則服之。四時視朔,則內單、領、襈,各隨其方色。唯秋方色白,以綠代之。



 遠游冠之制,案《漢雜事》曰:「太子諸王服之。」故《淮南子》曰:「楚莊王冠通梁,組纓。」注云:「通梁,遠游也。」晉令:「皇太子諸王,給遠游冠。」



 徐氏《雜注》曰:「天子雜服,遠游五梁。太子諸王三梁。」董巴《志》曰:「制如通天,有展筒,橫之幘上。」今制依此,天子加金博山,九首,施珠翠,黑介幘,金緣,以承之。翠
 緌纓,犀簪導。太子親王加金附蟬,宗室王去附蟬,並不通於庶姓。其乘輿遠游冠服,白紗單衣,承以裙襦,烏皮履。拜山陵則服之。



 武弁之制,案徐爰《宋志》,謂籠冠是也。《禮圖》曰:「武士服之。」董巴《輿服志》云:「諸常侍、內常侍,加黃金附蟬、毦尾,謂之惠文冠。」今制,天子金博山,三公已上玉冠枝,四品已上金枝。侍臣加附蟬,毦豐貂,文官七品已上毦白筆,八品已下及武官,皆不毦筆。其乘輿武弁之服,衣、裳、綬如通天之服。



 講武、出征、四時蒐狩、大射、祃、類、宜社、賞祖、罰社、纂嚴,皆服之。



 弁之制,案《五經通義》「高五寸,前後玉飾。」《詩》云:「璯弁如星。」



 董巴曰:「以鹿皮為之。」《尚書顧命》:「四人綦弁,執戈。」故知自天子至於執戈,通貴賤矣。《魏臺訪議》曰:「天子以五採玉珠十二飾之。」今參準此,通用烏漆紗而為之。天子十二琪,皇太子及一品九琪,二品八琪,三品七琪,四品六琪,五品五琪,六品已下無琪。唯文官服之,不通武職。案《禮圖》,有結纓而無笄導。少府少監何稠,請施象牙簪導。詔許之。弁加簪導,自茲始也。乘輿鹿皮弁服,緋大襦,白羅裙,金烏皮履,革帶,小綬長二尺六寸,色同大綬,而首半之,間施三玉環,白玉佩一雙。視朝聽訟則服之。凡弁服,
 自天子已下,內外九品已上,弁皆以烏為質,並衣褲褶。五品已上以紫,六品已下以絳。宿衛及在仗內,加兩襠,螣蛇絳褠衣,連裳。典謁贊引,流外冗吏,通服之,以縵。後制鹿皮弁,以賜近臣。



 帽,古野人之服也。董巴云:「上古穴居野處,衣毛帽皮。」以此而言,不施衣冠明矣。案宋、齊之間,天子宴私,著白高帽,士庶以烏,其制不定。或有卷荷,或有下裙,或有紗高屋,或有烏紗長耳。後周之時,咸著突騎帽,如今胡帽,垂裙覆帶,蓋索發之遺象也。又文帝項有瘤疾,不欲人見,每常著焉。相魏之時,著而謁帝,故後周一代,將為雅服,
 小朝公宴,咸許戴之。開皇初,高祖常著烏紗帽,自朝貴已下,至於冗吏,通著入朝。今復制白紗高屋帽,其服,練裙襦,烏皮履。



 宴接賓客則服之。



 白帢,案《傅子》:「魏太祖以天下兇荒,資財乏匱,擬古皮弁,裁縑帛以為之。」蓋自魏始也。《梁令》,天子為朝臣等舉哀則服之。今亦準此。其服,白紗單衣,承以裙襦,烏皮履。舉哀臨喪則服之。



 幘,案董巴云:「起於秦人,施於武將,初為絳袙,以表貴賤焉。至漢孝文時,乃加以高顏。」孝元帝額有壯發,不欲人見,乃始進幘。又董偃召見,綠幘傅韝。



 《東觀記》云:「詔賜段
 熲赤幘大冠一具。」故知自上已下,至於皁隸,及將帥等,皆通服之。今天子畋獵御戎,文官出游田里,武官自一品已下,至於九品,並流外吏色,皆同烏。廚人以綠,卒及馭人以赤,舉輦人以黃。駕五輅人,逐其車色。承遠游、進賢者,施以掌導,謂之介幘。承武弁者,施以笄導,謂之平巾。其乘輿黑介幘之服,紫羅褶,南布褲,玉梁帶,紫絲鞋,長靿靴。畋獵豫游則服之。



 皇太子服六等,袞冕九旒,硃組纓,青纊珫耳,犀簪導。紺衣,纁裳,去日月星辰為九章。白紗內單,黼黻領,青褾、襈、裾。革帶,金鉤灊,大帶,「X二章,玉具劍。侍從祭祀,及謁廟、
 加元服、納妃,則服之。據晉咸寧四年故事,衣色用玄,改用紺。舊章用織成,降以繡。玉具劍,故事以火珠鏢首,改以白珠。開皇中,皇太子冕同天子,貫白珠。及仁壽元年,煬帝為太子,以白珠太逼,表請從青珠。



 於是太子袞冕與三公王等,皆青珠九旒。旒短不及髆,降天子二寸。



 遠游冠,金附蟬,加寶飾珠翠,九首,珠纓翠緌,犀簪導。絳紗袍,白紗內單,皁領、褾、襈、裾。白假帶,方心曲領,絳紗蔽膝。襪,舄,革帶,劍,佩,綬同袞冕。未冠則雙童髻,空頂黑介幘,雙玉導,加寶飾珠翠,二首。謁廟還,元日、朔旦入朝,釋奠,則服之。
 始後周採用《周禮》,皇太子朝駕,皆袞冕九章服。開皇初,自非助祭,皆冠遠游冠。至此,牛弘奏云:「皇太子冬正大朝,請服袞冕。」



 帝問給事郎許善心曰:「太子朝謁,著遠游冠,有何典故?」對曰:「晉令皇太子給五時朝服、遠游冠。至宋泰始六年,更議儀注,儀曹郎丘仲起議:『案《周禮》,公自袞冕已下,至卿大夫之玄冕,皆其朝聘之服也。伏尋古之公侯,尚得服袞,以入朝見,況皇太子儲副之尊,謂宜式遵盛典,服袞朝賀。』兼左丞陸澄議:『服冕以朝,實著經典,自秦除六冕之制,後漢始備古章。魏、晉以來,非祀宗廟,不欲令臣下服於袞冕,位為公者,必加侍官,故太子
 入朝,因亦不著。但承天作副,禮絕群後,宜遵前王之令典,革近代之陋制,皇太子朝,請服冕。』自宋以下,始定此儀。至梁簡文之為太子,嫌於上逼,還冠遠游,下及於陳,皆依此法。後周之時,亦言服袞入朝。至於開皇,復遵魏、晉故事。臣謂袞冕之服,章玉雖差,一日而觀,頗欲相類。臣子之道,義無上逼。故晉武帝太始三年,詔太宰安平王孚著侍內之服,四年,又賜趙、燕、樂安王等散騎常侍之服。自斯以後,臺鼎貴臣,並加貂璫武弁,故皇太子遂著遠游,謙不逼尊,於理為允。」帝曰:「善。」竟用開皇舊式。



 遠游三梁冠,從省服,絳紗單衣,革帶,金鉤灊,假帶,方心,
 佩一隻,紛長六尺四寸,闊二寸四分,色同於綬。金縷鞶囊,白襪,烏皮履,金飾。五日常朝則服之。



 鹿皮弁,九琪,服絳羅襦,白羅裙,革帶,履,襪,佩,紛,如從省服。在宮聽政則服之。平巾,黑幘,玉冠枝,金花飾,犀簪導,紫羅褶,南布褲,玉梁帶,長靿靴。侍從田狩則服之。



 白帢,素單衣,烏皮履。為宮臣舉哀吊喪則服之。



 諸王三公已下,為服之制,袞冕九章服。三公攝祭及諸王初受冊、執贄、入朝、助祭、親迎,則服之。綬各依其色。



 冕,案《禮圖》:「王祭先公及卿之服。」天子九旒,用玉二百
 一十六。侯伯服以助祭,七旒,用玉八十。新制依此。服七章。三品及公侯助祭則服之。



 毳冕,案《禮圖》:「王祀四望山川之服。」天子七旒,用玉百六十八。子男服以助祭,五旒,用玉五十。新制依此。服五章。四品及伯助祭則服之。



 衣黹冕,案《禮圖》:「王者祭社稷五祀之服。」天子五旒,用玉百二十。孤卿服以助祭,四旒,用玉三十二。新制依此。服三章。五品及子男助祭則服之。



 玄冕,案《禮圖》:「王祭群小祀及視朝服。」天子四旒,用玉三十二。諸侯服以祭其宗廟,三旒,用玉十八。新制依此。服
 三章。通給庶姓。一品已下,五品已上,自制於家,祭其私廟。三品省衣粉米,加三重;裳黼黻,加二重。四品減黼一重,五品減黻一重。禮自玄冕以上,加旒一等,天子祭祀,節級服之。



 開皇以來,天子唯用袞冕,自鷩之下,不施於尊,具依前式。而六等之冕,皆有黈纊,黃綿為之,其大如橘。自皇太子以下,三犀導,青纓爵弁。案董巴《志》:「同於爵形,一名冕,有收持笄,所謂夏收、殷冔者也。」祠天地、五郊、明堂,《雲翹》舞人服之。《禮》云:「硃干玉戚,冕而舞《大夏》。」此之謂也。《禮圖》云:「士助君祭服之,色如爵頭,無旒有纊。」新制依此。角為
 簪導,衣青,裳縓,並縵,無章。六品已下,皆通服之。



 遠游冠服,王所服也。衣裳內單。如皇太子,佩山玄玉,金章龜鈕。宋孝建故事亦謂之璽,今文曰印。又並歸於官府,身不自佩,例以銅易之。大綬四採,小綬同色,施二玉環,玉具劍,烏皮舄,舄加金飾。唯帝子宗室封國王者服之。



 進賢冠,案《漢官》云:「平帝元始五年,令公卿列侯冠三梁,二千石兩梁,千石以下一梁。」梁別貴賤,自漢始也。董巴釋曰:「如緇布冠,文儒之服也。」



 前高七寸而卻,後高三寸而立。王莽之時,以幘承之。新制依此。內外文官通服之。



 三品已上三梁,五品已上兩梁,九品已上一梁,用明尊卑之等也。其朝服,亦名具服。絳紗單衣,白紗內單,玄領、裾、襈、袖,革帶,金鉤灊,假帶,曲領方心,絳紗蔽膝,白襪,烏皮舄。雙佩、綬,如遠游之色。自一品已下,五品已上,衣服盡同,而綬依其品。陪祭朝饗拜表,凡大事皆服之。六品、七品,去劍、佩、綬。



 八品、九品,去白筆、內單,而用履代舄。其五品已上,一品已下,又有公服,亦名從省服。並烏皮履,去曲領、內單、白筆、蔽膝。開皇故事,亦去鞶囊、佩、綬。



 何稠請去大綬,而偏垂一小綬,綴於獸頭鞶囊,獨一只佩,正當於後。詔從之。一品已下,五品已上,同。



 高山冠,案董巴《志》云:「一曰側注,謁者僕射之所服也。」胡伯始以為齊王冠,秦滅齊,以賜謁者。《傅子》曰:「魏明帝以高山冠似通天,乃毀變其形,除去卷筒,令如介幘。幘上加物,以象山峰,行人使者,通皆服之。」新制參用其事,形如進賢,於冠前加三峰,以象魏制。謁者大夫已下服之。梁依其品。



 獬豸冠,案《禮圖》曰:「法冠也,一曰柱後惠文。」如淳注《漢官》曰:「惠,蟬也,細如蟬翼。」今御史服之。《禮圖》又曰:獬豸冠,高五寸,秦制也。



 法官服之。」董巴《志》曰:「獬豸,神羊也。」蔡邕云:「如麟,一角。」應劭曰:「古有此獸,主觸不直,故執憲者,為冠
 以象之。秦滅楚,以其冠賜御史。」



 此即是也。開皇中,御史戴卻非冠,而無此色。新制又以此而代卻非。御史大夫以金,治書侍御史以犀,侍御史已下,用羚羊角,獨御史、司隸服之。



 巾,案《方言》云:「巾,趙、魏間通謂之承露。」《郭林宗傳》曰:「林宗嘗行遇雨,巾沾角折。」又袁紹戰敗,幅巾渡河。此則野人及軍旅服也。制有二等。



 今高人道士所著,是林宗折角;庶人農夫常服,是袁紹幅巾。故事,用全幅皁而向後襆發,俗人謂之襆頭。自周武帝裁為四腳,今通於貴賤矣。



 簪導,案《釋名》云:「簪,建也,所以建冠於發也。一曰笄。笄,系
 也,所以拘冠使不墜也。導,所以導擽鬢發,使入巾幘之里也。」今依《周禮》,天子以玉笄,而導亦如之。又《史記》曰:「平原君誇楚,為玳瑁簪。」班固《與弟書》云:「今遺仲升以黑犀簪。」《士燮集》云:「遣功曹史貢皇太子通天犀導。」故知天子獨得用玉,降此通用玳瑁及犀。今並準是,唯弁用白牙笄導焉。



 貂蟬,案《漢官》:「侍內金蟬左貂,金取剛固,蟬取高潔也。」董巴《志》曰:「內常侍,右貂金璫,銀附蟬,內書令亦同此。」今宦者去貂,內史令金蟬右貂,納言金蟬左貂。開皇時,加散騎常侍在門下者,皆有貂蟬,至是罷之。唯加常侍聘外
 國者,特給貂蟬,還則輸納於內省。



 白筆,案徐氏《雜注》云:「古者貴賤皆執笏,有事則書之,故常簪筆。今之白筆,是遺象也。」《魏略》曰:「明帝時大會而史簪筆。」今文官七品已上,通毦之。武職雖貴,皆不毦也。



 纓,案《儀禮》曰:「天子硃纓,諸侯丹組纓。」今冕,天子已下皆硃纓。又《尉繚子》曰:「天子玄纓,諸侯素纓。」別尊卑也。今不用素,並從冠色焉。



 佩,案《禮》,天子佩白玉。董巴、司馬彪云:「君臣佩玉,尊卑有序,所以章德也。」今參用杜夔之法,天子白玉,太子瑜玉,王山玄玉。自公已下,皆水蒼玉。



 綬,案《禮》:「天子玄組綬,侯伯硃組綬,大夫純組綬,世子綦組綬。」



 《漢官》云:「蕭何為相國,佩綠綬,公侯紫,卿二千石青,令長千石黑。」今大抵準此。天子以雙綬,六採,玄黃赤白縹綠,純玄質,長二丈四尺,五百首,闊一尺;雙小綬,長二尺六寸,色同大綬,而首半之,間施四玉環。開皇用三,今加一。



 皇太子,硃雙綬,四採,赤白縹紺,純硃質,長一丈八尺,三百二十首,闊九寸;雙小綬,長一尺六寸,色同大綬,而首半之,間施三玉環。開皇用二,今加一。三公,綠綟綬,四採,綠黃縹紫,純綠質,黃文織之,長一丈八尺,二百四十首,闊九寸,與親王綬俱施二玉環。諸王,纁硃綬,四採,
 赤黃縹紺,純硃質,纁文織之,長一丈八尺,二百四十首,闊九寸。公,玄硃綬,四採,赤縹玄紺,純硃質,玄文織之,長一丈八尺,二百四十首,闊九寸。侯、伯,青硃綬,四採,青赤白縹,純硃質,青文織,長一丈六尺,百八十首,闊八寸。子、男,素硃綬,三採,青赤白,純硃質,素文織之,長一丈四尺,百四十首,闊七寸。二品已上,纁紫綬,四採,纁紫赤黃,純紫質,纁文織之,長一丈四尺,百四十首,闊八寸。三品,紺紫綬,四採,紫紺黃縹,純紫質,紺文織之,長一丈六尺,百八十首,闊八寸。四品,青綬,三採,青白紅,純青質,長一丈四尺,百四十首,闊七寸。五品,墨綬,二採,青紺,純紺質,
 長一丈二尺,百二十首,闊六寸。自王公已下,皆有小綬二枚,色同大綬,而首半之。正、從一品,施二玉環。凡有綬者,皆有紛,並長六尺四寸,闊二寸四分,隨於綬色。



 鞶囊,案《禮》:「男鞶革,女鞶絲。」《東觀書》:「詔賜鄧遵獸頭鞶囊一枚。」班固《與弟書》:「遺仲升獸頭旁囊,金錯鉤也。」古佩印皆貯懸之,故有囊稱。或帶於旁,故班氏謂為旁囊,綬印鈕也。今雖不佩印,猶存古制,有佩綬者,通得佩之。無佩則不。今採梁、陳、東齊制,品極尊者,以金織成,二品以上服之。



 次以銀織成,三品已上服之。下以綖織成,五品已上服之。分為三等。



 革帶,案《禮》「博二寸」。《禮圖》曰:「璫綴於革帶。」阮諶以為有章印則於革帶佩之。《東觀記》:「楊賜拜太常,詔賜自所著革帶。」故知形制尊卑不別。今博三寸半,加金縷灊,螳良鉤,以相拘帶。自大裘至於小朝服,皆用之。



 劍,案漢自天子至於百官,無不佩刀。蔡謨議云:「大臣優禮,皆劍履上殿。



 非侍臣,解之。」蓋防刃也。近代以木,未詳所起。東齊著令,謂為象劍,言象於劍。周武帝時,百官燕會,並帶刀升座。至開皇初,因襲舊式,朝服登殿,亦不解焉。十二年,因蔡徵上事,始制凡朝會應登殿坐者,劍履俱脫。其不坐者,敕召奏事及須升殿,亦就席解劍乃登。
 納言、黃門、內史令、侍郎、舍人,既夾侍之官,則不脫。其劍皆真刃,非假。既合舊典,弘制依定。又準晉咸康元年定令故事,自天子已下,皆衣冠帶劍。今天子則玉具火珠鏢首,餘皆玉鏢首。唯侍臣帶劍上殿,自王公已下,非殊禮引升殿,皆就席解而後升。六品以下,無佩綬者,皆不帶。



 曲領,案《釋名》,在單衣內襟領上,橫以雍頸。七品已上有內單者則服之,從省服及八品已下皆無。



 珽,案《禮》:「天子搢珽,方正於天下也。」又《五經異義》:「天子笏曰珽,珽直無所屈也。」今制準此,長尺二寸,方而不折。以
 球玉為之。



 笏,案《禮》:「諸侯以象,大夫魚須文竹,士以竹,本象可也。」凡有指畫於君前,受命書於笏,笏畢用也。《五經要義》曰:「所以記事,防忽忘。」《禮圖》云:「度二尺有六寸,中博二寸,其殺六分去一。」晉、宋以來,謂之手板,此乃不經,今還謂之笏,以法古名。自西魏以降,五品已上,通用象牙,六品已下,兼用竹木。



 履、舄,案《圖》云:「復下曰舄,單下曰履。夏葛冬皮。」近代或以重皮,而不加木,失於乾臘之義。今取乾臘之理,以木重底。冕服者色赤,冕衣者色烏,履同烏色。諸非侍臣,皆脫而
 升殿。凡舄,唯冕服及具服著之,履則諸服皆用。唯褶服以靴。靴,胡履也,取便於事,施於戎服。



 諸建華、絺鸃、鶡冠、委貌、長冠、樊噲、卻敵、巧士、術氏、卻非等,前代所有,皆不採用。



 皇后服四等,有褘衣、鞠衣、青服、硃服。



 褘衣,深青質,織成領袖,文以翬翟,五採重行,十二等。首飾花十二鈿,小花毦十二樹,並兩博鬢。素紗內單,黼領,羅縠褾、襈,色皆以硃。蔽膝隨裳色,以緅為緣,用翟三章。大帶隨衣裳,飾以硃綠之錦,青緣。革帶,青襪、舄,舄以金飾。白玉佩,玄組,綬,章採尺寸同於乘輿。祭及朝會,凡大
 事皆服之。



 鞠衣,黃羅為質,織成領袖,小花十二樹。蔽膝,革帶及舄,隨衣色。餘準褘衣,親蠶服也。



 青服,去花、大帶及佩綬,金飾履。禮見天子則服之。



 硃服,制如青服。宴見賓客則服之。



 有金璽,盤螭鈕,文曰「皇后之璽」。冬正大朝,則並黃琮,各以笥貯,進於座隅。



 皇太后服,同於後服。而貴妃以下,並亦給印。



 貴妃、德妃、淑妃,是為三妃。服褕翟之衣,首飾花九鈿,並二博鬢。金章龜鈕,文從其職。紫綬,一百二十首,長一丈
 七尺,金縷織成獸頭鞶囊,佩于闐玉。



 順儀、順容、順華、修儀、修容、修華、充儀、充容、充華,是為九嬪。服闕翟之衣,首飾花八鈿,並二博鬢。金章龜鈕,文從其職。紫綬,一百首,長一丈七尺,金縷織成獸頭鞶囊,佩採瓄玉。



 婕妤,銀縷織成獸頭鞶囊,首飾花七鈿。他如嬪服。



 美人、才人,服鞠衣,首飾花六鈿,並二博鬢。銀印珪鈕,文從其職。青綬,八十首,長一丈六尺。彩縷織成獸爪鞶囊,佩水蒼玉。



 寶林,服展衣,首飾花五鈿,並二博鬢。銀印環鈕,文如其
 職。艾綬,八十首,長一丈六尺。鞶囊,佩玉,同於婕妤。



 承衣刀人、採女,皆服褖衣,無印綬。參準宋泰始四年及梁、陳故事,增損用之。



 皇太子妃,服褕翟之衣,青質,五採織成為搖翟,以備九章。首飾花九鈿,並二博鬢。金璽龜鈕,文如其職。素紗內單,黼領,羅褾、襈,色皆用硃,蔽膝二章。



 大帶,同褘衣,青綠革帶,硃襪,青舄,舄加金飾。佩瑜玉,纁硃綬,一百六十首,長二丈,獸頭鞶囊,凡大禮見皆服之。唯侍親桑,則用鞠衣之服,花鈿佩綬,與褕衣同。準宋孝建二年故事而增損之。



 良娣,鞠衣之服,銀印珪鈕,文如其職。佩採瓄玉,青綬,八十首,長一丈六尺,獸爪鞶囊。餘同世婦。



 保林、八子,展衣之服,銅印環鈕,文如其職。佩水蒼玉,艾綬,八十首,長一丈六尺,獸爪鞶囊。自良娣等,準宋大明六年故事而損益之。



 諸王太妃、妃、長公主、公主、三公夫人、一品命婦,褕翟之服,繡為九章。



 首飾花九鈿,佩山玄玉,獸頭鞶囊。綬同夫色。



 公夫人,縣主、二品命婦,亦服褕翟,繡為八章。首飾八鈿。侍從親桑,同用鞠衣。自此之下,佩皆水蒼玉。



 侯、伯夫人、三品命婦,亦服祃翟,繡為七章。首飾七鈿。



 子夫人、四品命婦,服闕翟之衣,刻赤繒為翟,綴於服上,以為六章。首飾六鈿。



 男夫人、五品命婦,亦服闕翟之衣,刻繒為翟,綴於服上,以為五章。首飾五鈿。若當從侍親桑,皆同鞠衣。



 議既定,帝幸修文殿覽之,乃令何稠、起部郎閻毗等造樣上呈。二年總了,始班行焉,軒冕之盛,貫古今矣。



 三年正月朔旦,大陳文物。時突厥染干朝見,慕之,請襲冠冕。帝不許。明日,率左光祿大夫、褥但特勤阿史那職御,左光祿大夫、特勤阿史那伊順,右光祿大夫、意利發史蜀胡悉等,
 並拜表,固請衣冠。帝大悅,謂弘等曰:「昔漢制初成,方知天子之貴。今衣冠大備,足致單于解辮,此乃卿等功也。」弘、愷、善心、世基、何稠、閻毗等賜帛各有差,並事出優厚。



 是後師旅務殷,車駕多行幸。百官行從,唯服褲褶,而軍旅間不便。至六年後,詔從駕涉遠者,文武官等皆戎衣。貴賤異等,雜用五色。五品已上,通著紫袍,六品已下,兼用緋綠,胥吏以青,庶人以白,屠商以皁,士卒以黃。



 卓彼上天,宮室混成,玄戈居其左,上將居其右,弧矢揚威,羽林置陳。《易》曰:「天垂象,聖人則之。」昔軒轅氏之有天下也,以師兵為營衛,降至三代,其儀大備。西漢武帝,每
 上甘泉,則列鹵簿,車千乘,騎萬匹。其居前殿,則植戟懸楯,以戒不虞。其所由來者尚矣。



 梁武受禪於齊,侍衛多循其制。正殿便殿閣及諸門上下,各以直閤將軍等直領。



 又置刀釤、御刀、御楯之屬,直御左右。兼有御仗、鋌槊、赤氅、角抵、勇士、青氅、衛仗、長刀、刀劍、細仗、羽林等左右二百七十六人,以分直諸門。行則儀衛左右。又有左右夾轂、蜀客、楯劍、格獸羽林、八從游蕩、十二不從游蕩、直從細射、廉察、刀戟、腰弩、大弩等隊,凡四十九隊,亦分直諸門上下。行則量為儀衛。



 東西掖、端、大司馬、東西華、承明、大通等門,又各二隊,及防殿
 三隊,雖行幸不從。又有八馬游蕩、馬左右夾轂、左右馬百騎等各二隊,及騎官、閱武馬容、雜伎馬容及左右馬騎直隊,行則侍衛左右,分為警衛。車駕晨夜出入及涉險,皆作函。



 鹵簿應宿衛軍騎,皆執兵持滿,各當其所保護方面。天明及度險,乃奏解函,撾鼓而依常列。乘輿行則有大駕、法駕、小駕。大賀以郊饗上天,臨馭九伐。法駕以祭方澤,祀明堂,奉宗廟,藉千畝。小駕以敬園陵,親蒐狩。大駕則公卿奉引,大將軍驂乘,太搢馭。法駕小駕,皆侍中驂乘,奉車郎馭,公卿不引。其餘行幸,送往勞旋,則槊仗。近宴則隊仗。三駕法天,二仗法地,其動也參天而
 兩地也。陳氏承梁,亦無改革。



 齊文宣受禪之後,警衛多循後魏之儀。及河清中定令,宮衛之制,左右各有羽林郎十二隊。又有持鈒隊、鋌槊隊、長刀隊、細仗隊,楯鎩隊、雄戟隊、格獸隊、赤氅隊、角抵隊、羽林隊、步游蕩隊、馬游蕩隊。又左右各武賁十隊,左右翊各四隊,又步游蕩、馬游蕩左右各三隊,是為武賁。又有直從武賁,左右各六隊,在左者為前驅隊,在右者為後拒隊。又有募員武賁隊、強弩隊,左右各一隊,在左者皆左衛將軍總之,在右者皆右衛將軍總之,以備警衛。其領軍、中領將軍,侍從出入,則著兩襠甲,手執檉杖。
 左右衛將軍、將軍則兩襠甲,手執檀杖。侍從左右,則有千牛備身、左右備身刀劍備身之屬。兼有武威、熊渠、鷹揚等備身三隊,皆領左右將軍主之,宿衛左右,而戎服執仗。兵有斧鉞弓箭刀槊,旌旗皆囊首,五色節文,旆悉赭黃。天子御正殿,唯大臣夾侍,兵仗悉在殿下。郊祭鹵簿,則督將平巾幘,緋衫甲,大口褲。



 後周警衛之制,置左右宮伯,掌侍衛之禁,各更直於內。小宮伯貳之。臨朝則分在前侍之首,並金甲,各執龍環金飾長刀。行則夾路車左右。中侍,掌御寢之禁,皆金甲,左執龍環,右執獸環長刀,並飾以金。次左右侍,陪中侍
 之後,並銀甲,左執鳳環,右執麟環長刀。次左右前侍,掌御寢南門之左右,並銀甲,左執師子環,右執象環長刀。次左右後侍,掌御寢北門之左右,並銀甲,左執犀環,右執兕環長刀。左右騎侍,立於寢之東西階,並銀甲,左執羆環,右執熊環長刀,十二人,兼執師子彤楯,列左右侍之外。自左右侍以下,刀並以銀飾。左右宗侍,陪左右前侍之後,夜則衛於寢庭之中,皆服金塗甲,左執豹環,右執貔環長刀,並金塗飾,十二人,兼執師子彤楯,列於左右騎侍之外。自左右中侍已下,皆行則兼帶黃弓矢,巡田則常服,帶短刀,如其長刀之飾。左右庶侍,掌非皇帝
 所御門閣之禁,並服金塗甲,左執獬豸環,右執獜環長劍,並金飾,十二人,兼執師子彤楯,列於左右宗侍之外。行則兼帶皓弓矢。左右勛侍,掌陪左右庶侍而守出入,則服金塗甲,左執吉良環,右執猙環長劍,十二人,兼執師子彤楯,列於左右庶侍之外。行則兼帶盧弓矢,巡田則與左右庶侍俱常服,佩短劍,如其長劍之飾。諸侍官,大駕則俱侍,中駕及露寢半之,小駕三分之一。左右武伯,掌內外衛之禁令,兼六率之士。皇帝臨軒,則備三仗於庭,服金甲,執金金口杖,立於殿上東西階之側。行則列兵於帝之左右,從則服金甲,被繡袍。左右小武伯各二
 人,貳之,服執同於武伯,分立於大武伯下及露門之左右塾。行幸則加錦袍。左右武賁,率掌武賁之士,其隊器服皆玄,以四色飾之,各總左右持金及之隊。皇帝臨露寢,則立於左右三仗第一行之南北。出則分在隊之先後。其副率貳之。左右旅賁,率掌旅賁士,其隊器服皆青,以硃為飾,立於三仗第二行之南北。其副率貳之。左右射聲,率掌射聲之士,其器服皆硃,以黃為飾,立於三仗第三行之南北。其副率貳之。左右驍騎,率掌驍騎之士,器服皆黃,以皓為飾,立於三仗第四行之南北。其副率貳之。左右羽林,率掌羽林之士,其隊器服皆皓,以玄為飾,
 立於三仗第五行之南北。其副率貳之。左右游擊,率掌游擊之士,其器服皆玄,以青為飾。其副率貳之。武賁已下六率,通服金甲師子文袍,執銀金口檀杖。副率通服金甲獸文袍。各有人卒長、帥長,相次陪列。行則引前。人卒長通服銀甲豹文袍,帥長通服銀甲鶡文袍。自副率已下,通執獸環銀飾長刀。凡大駕則盡行,中駕及露寢則半之,小駕半中駕。常行軍旅,則衣色尚烏。



 高祖受命,因周、齊宮衛,微有變革。戎服臨朝大仗,則領左右大將軍二人,分在左右廂。左右直寢、左右直齋、左右直後、千牛備身、左右備身等,夾侍供奉於左右及坐
 後。左右衛大將軍、左右直閤將軍、以次左右衛將軍,各領儀刀,為十二行。內四行親衛,行別以大都督領。次外四行勛衛,以帥都督領。次外四行翊衛,以都督領。行各二人執金花師子盾、猨刀。一百四十人,分左右,帶橫刀。後監門直長十二人,左青龍旗,右白獸旗。左右武衛開府,各領三仗六行,在大仗內,行別六十人,大都督一人領之,帥都督一人後之。大駕則執黃麾仗。其次戟二十四,左青龍幢,右白獸幢,蒨、畢各一,鈒金二十四,金節十二道,蓋獸,又絳引幡,硃幢,為持鈒前隊,應蹕,大都督二人領之,在御前橫街南。左右武衛大將軍,領大仗左右
 廂,各六行,行別三百六十人,大都督一人領之。



 及大業四年,煬帝北巡出塞,行宮設六合城。方一百二十步,高四丈二尺。六合,以木為之,方六尺,外面一方有板,離合為之,塗以青色。壘六板為城,高三丈六尺,上加女墻板,高六尺。開南北門。又於城四角起樓敵二,門觀、門樓檻皆丹青綺畫。又造六合殿、千人帳,載以槍車,車載六合三板。其車軨解合交叉,即為馬槍。每車上張幕,幕下張平一弩,傅矢,五人更守。兩車之間,施車軨馬槍,皆外其轅,以為外圍。次內布鐵菱,次內施蟄鞬。每一蟄鞬,中施弩床,長六尺,闊三尺。床桄陛插鋼錐,皆長五寸,謂之蝦
 須。皆施機關,張則錐皆外向。其床上施旋機弩,以繩連弩機,人從外來,觸繩則弩機旋轉,向觸所而發。其外又以矰周圍行宮,二丈一鈴一柱,柱舉矰,去地二尺五寸。當行宮南北門,施槌磬,連矰,以機發之。有人觸矰,則眾鈴發響,槌擊兩磬,以知所警,名為擊警。八年征遼,又造鉤陳,以木板連如帳子。張之則綺文,卷之則直焉。帝御營與賊城相對,夜中設六合城,周回八里。城及女垣合高十仞,上布甲士,立仗建旗。又四隅有闕,面別一觀,觀下開三門。其中施行殿,殿上容侍臣及三衛仗,合六百人。一宿而畢,望之若真,高麗旦忽見,謂之為神焉。



\end{pinyinscope}