\article{卷五十八列傳第二十三}

\begin{pinyinscope}

 明克讓明克讓,字弘道,平原鬲人也。父山賓,梁侍中。克讓少好儒雅,善談論,博涉書史,所覽將萬卷。《三禮》禮論,尤所研精,龜策歷象,咸得其妙。年十四,釋褐湘東王法曹參軍。時舍人硃異在儀賢堂講《老子》,克讓預焉。堂邊有修竹,異令克讓詠之。克讓攬筆輒成,其卒章曰:「非君多愛賞,誰貴此貞心。」異甚奇之。仕歷司徒祭酒、尚書都官郎中、
 散騎侍郎,兼國子博士、中書侍郎。梁滅,歸於長安,周明帝引為麟趾殿學士,俄授著作上士,轉外史下大夫,出為衛王友,歷漢東、南陳二郡守。武帝即位,復徵為露門學士,令與太史官屬正定新歷。拜儀同三司,累遷司調大夫,賜爵歷城縣伯,邑五百戶。高祖受禪,拜太子內舍人,轉率更令,進爵為侯。太子以師道處之,恩禮甚厚。每有四方珍味,輒以賜之。於時東宮盛徵天下才學之士,至於博物洽聞,皆出其下。詔與太常牛弘等修禮議樂,當朝典故多所裁正。開皇十四年,以疾去官,加通直散騎常侍。卒,年七十。上甚傷惜焉,賻物五百段,米三百石。
 太子又贈絹布二千匹,錢十萬,朝服一具,給棺槨。



 著《孝經義疏》一部,《古今帝代記》一卷,《文類》四卷,《續名僧記》一卷,集二十卷。



 子餘慶,官至司門郎。越王侗稱制,為國子祭酒。



 魏澹魏澹,字彥深,巨鹿下曲陽人也。祖鸞,魏光州刺史。父季景,齊大司農卿,稱為著姓,世以文學自業。澹年十五而孤,專精好學,博涉經史,善屬文,詞採贍逸。齊博陵王濟聞其名,引為記室。及瑯邪王儼為京畿大都督,以澹為鎧曹參軍,轉殿中侍御史。尋與尚書左僕射魏收、吏部
 尚書陽休之、國子博士熊安生同修《五禮》。又與諸學士撰《御覽》,書成,除殿中郎中、中書舍人。復與李德林俱修國史。周武帝平齊,授納言中士。及高祖受禪,出為行臺禮部侍郎。尋為散騎常侍、聘陳主使。還除太子舍人。廢太子勇深禮遇之,屢加優錫,令注《庾信集》,復撰《笑苑》、《詞林集》,世稱其博物。數年,遷著作郎,仍為太子學士。



 高祖以魏收所撰書褒貶失實,平繪為《中興書》事不倫序,詔澹別成《魏史》。



 澹自道武下及恭帝,為十二紀,七十八傳,別為史論及例一卷,並《目錄》合九十二卷。澹之義例與魏收多所不同:其一曰,臣聞天子者,繼天立極,終始絕
 名。故《穀梁傳》曰:「太上不名。」



 《曲禮》曰:「天子不言出,諸侯不生名。」諸侯尚不生名,況天子乎!若為太子,必須書名。良由子者對父生稱,父前子名,禮之意也。是以桓公六年九月丁卯,子同生,《傳》曰:「舉以太子之禮。」杜預注云:「桓公子莊公也。」十二公唯子同是嫡夫人之長子,備用太子之禮,故史書之於策。即位之日,尊成君而不名,《春秋》之義,聖人之微旨也。至如馬遷,周之太子並皆言名,漢之儲兩俱沒其諱,以尊漢卑周,臣子之意也。竊謂雖立此理,恐非其義。何者?《春秋》、《禮記》,太子必書名,天王不言出。此仲尼之褒貶,皇王之稱謂,非當時與異代遂為優劣
 也。



 班固、範曄、陳壽、王隱、沈約參差不同,尊卑失序。至於魏收,諱儲君之名,書天子之字,過又甚焉。今所撰史,諱皇帝名,書太子字,欲以尊君卑臣,依《春秋》之義也。



 其二曰,五帝之聖,三代之英,積德累功,乃文乃武,賢聖相承,莫過周室,名器不及後稷,追謚止於三王,此即前代之茂實,後人之龜鏡也。魏氏平文以前,部落之君長耳。太祖遠追二十八帝,並極崇高,違堯舜憲章,越周公典禮。但道武出自結繩,未師典誥,當須南、董直筆,裁而正之。反更飾非,言是觀過,所謂決渤澥之水,復去堤防,襄陵之災,未可免也。但力微天女所誕,靈異絕世,尊為始祖,
 得禮之宜。平文、昭成雄據塞表,英風漸盛,圖南之業,基自此始。長孫斤之亂也,兵交御坐,太子授命,昭成獲免。道武此時,后緡方娠,宗廟復存,社稷有主,大功大孝,實在獻明。此之三世,稱謚可也。自茲以外,未之敢聞。



 其三曰,臣以為南巢桀亡,牧野紂滅,斬以黃鉞,懸首白旗,幽王死於驪山,厲王出奔於彘,未嘗隱諱,直筆書之,欲以勸善懲惡,貽誡將來者也。而太武、獻文並皆非命,前史立紀,不異天年,言論之間,頗露首尾。殺主害君,莫知名姓,逆臣賊子,何所懼哉!君子之過,如日月之食,圓首方足,孰不瞻仰?況復兵交御坐,矢及王屋,而可隱沒者乎!
 今所撰史,分明直書,不敢回避。且隱、桓之死,閔、昭殺逐,丘明據實敘於經下,況復懸隔異代而致依違哉!



 其四曰,周道陵遲,不勝其敝,楚子親問九鼎,吳人來征百牢,無君之心,實彰行路,夫子刊經,皆書曰卒。自晉德不競,宇宙分崩,或帝或王,各自署置。當其生日,聘使往來,略如敵國,及其終也,書之日死,便同庶人。存沒頓殊,能無懷愧!今所撰史,諸國凡處華夏之地者,皆書曰卒,同之吳、楚。



 其五曰,壺遂發問,馬遷答之,義已盡矣。後之述者,仍未領悟。董仲舒、司馬遷之意,本云《尚書》者,隆平之典,《春秋》者,撥亂之法,興衰理異,制作亦殊。治定則直敘欽
 明,世亂則辭兼顯晦,分路命家,不相依放。故云「周道廢,《春秋》作焉,堯、舜盛,《尚書》載之」是也。「漢興以來,改正朔,易服色,臣力誦聖德,仍不能盡,餘所謂述故事,而君比之《春秋》,謬哉」。然則紀傳之體出自《尚書》,不學《春秋》,明矣。而範曄云:「《春秋》者,文既總略,好失事形,今之擬作,所以為短。紀傳者,史、班之所變也,網羅一代,事義周悉,適之後學,此焉為優,故繼而述之。」觀曄此言,豈直非聖人之無法,又失馬遷之意旨。孫盛自謂鉆仰具體而放之。魏收云:「魯史既修,達者貽則,子長自拘紀傳,不存師表,蓋泉源所由,地非企及。」雖復遜辭畏聖,亦未思紀傳所由來
 也。



 澹又以為司馬遷創立紀傳以來,述者非一,人無善惡,皆為立論。計在身行跡,具在正書,事既無奇,不足懲勸。再述乍同銘頌,重敘唯覺繁文。案丘明亞聖之才,發揚聖旨,言「君子曰」者,無非甚泰,其間尋常,直書而已。今所撰史,竊有慕焉,可為勸戒者,論其得失,其無損益者,所不論也。



 澹所著《魏書》,甚簡要,大矯收、繪之失。上覽而善之。未幾,卒,時年六十五。有《文集》三十卷行於世。子信言,頗知名。



 澹弟彥玄,有文學,歷揚州總管府記室、洧州司馬。有子滿行。



 陸爽侯白
 陸爽,字開明,魏郡臨漳人也。祖順宗,魏南青州刺史。父概之,齊霍州刺史。



 爽少聰敏,年九歲就學,日誦二千餘言。齊尚書僕射楊遵彥見而異之,曰:「陸氏代有人焉。」年十七,齊司州牧、清河王岳召為主簿。擢殿中侍御史,俄兼治書,累轉中書侍郎。及齊滅,周武帝聞其名,與陽休之、袁叔德等十餘人俱徵入關。諸人多將輜重,爽獨載書數千卷。至長安,授宣納上士。高祖受禪,轉太子內直監,尋遷太子洗馬。與左庶子宇文愷等撰《東宮典記》七十卷。朝廷以其博學有口辯,陳人至境,常令迎勞。開皇十一年,卒官,時年五十三,贈上儀同、宣州刺史,賜帛百
 匹。



 子法言,敏學有家風,釋褐承奉郎。初,爽之為洗馬,嘗奏高祖云:「皇太子諸子未有嘉名,請依《春秋》之義,更立名字。」上從之。及太子廢,上追怒爽云:「我孫制名,寧不自解?陸爽乃爾多事!扇惑於勇,亦由此人。其身雖故,子孫並宜屏黜,終身不齒。」法言竟坐除名。



 爽同郡侯白,字君素,好學有捷才,性滑稽,尤辯俊。舉秀才,為儒林郎。通侻不恃威儀,好為誹諧雜說,人多愛狎之,所在之處,觀者如市。楊素甚狎之。素嘗與牛弘退朝,白謂素曰:「日之夕矣。」素大笑曰:「以我為牛羊下來邪?」高祖聞其名,召與語,甚悅之,令於秘書修國史。每將擢之,高祖輒曰:「侯白不
 勝官」而止。後給五品食,月餘而死,時人傷其薄命。著《旌異記》十五卷,行於世。



 杜臺卿杜臺卿,字少山,博陵曲陽人也。父弼,齊衛尉卿。臺卿少好學,博覽書記,解屬文。仕齊奉朝請,歷司空西閤祭酒、司徒戶曹、著作郎、中書黃門侍郎。性儒素,每以雅道自居。及周武帝平齊,歸於鄉里,以《禮記》、《春秋》講授子弟。



 開皇初,被徵入朝。臺卿嘗採《月令》,觸類而廣之,為書名《玉燭寶典》十二卷。



 至是奏之,賜絹二百匹。臺卿患聾,不堪吏職,請修國史。上許之,拜著作郎。十四年,上表請致仕,
 敕以本官還第。數載,終於家。有集十五卷,撰《齊記》二十卷,並行於世。無子。



 有兄蕤,學業不如臺卿,而幹局過之。仕至開州刺史。子公贍,少好學,有家風,卒於安陽令。公贍子之松,大業中,為起居舍人。



 辛德源辛德源,字孝基,隴西狄道人也。祖穆,魏平原太守。父子馥,尚書右丞。德源沉靜好學,年十四,解屬文。及長,博覽書記,少有重名。齊尚書僕射楊遵彥、殿中尚書辛術皆一時名士,見德源,並虛襟禮敬,因同薦之於文宣帝。起家奉朝請,後為兼員外散騎侍郎,聘梁使副。後歷馮翊、
 華山二王記室。中書侍郎劉逖上表薦德源曰:「弱齡好古,晚節逾厲,枕藉六經,漁獵百氏。文章綺艷,體調清華,恭慎表於閨門,謙捴著於朋執。實後進之辭人,當今之雅器。必能效節一官,騁足千里。」由是除員外散騎侍郎,累遷比部郎中,復兼通直散騎常侍。聘於陳,及還,待詔文林館,除尚書考功郎中,轉中書舍人。及齊滅,仕周為宣納上士。因取急詣相州,會尉迥作亂,以為中郎。德源辭不獲免,遂亡去。高祖受禪,不得調者久之,隱於林慮山,鬱鬱不得志,著《幽居賦》以自寄,文多不載。德源素與武陽太守盧思道友善,時相往來。魏州刺史崔彥武奏
 德源潛為交結,恐其有奸計。由是謫令從軍討南寧,歲餘而還。秘書監牛弘以德源才學顯著,奏與著作郎王劭同修國史。德源每於務隙撰《集注春秋三傳》三十卷,注揚子《法言》二十三卷。蜀王秀聞其名而引之,居數歲,奏以為掾。後轉諮議參軍,卒官。有集二十卷,又撰《政訓》、《內訓》各二十卷。有子素臣、正臣,並學涉有文義。


柳
 \gezhu{
  巧言}
 柳
 \gezhu{
  巧言}
 ,字顧言,本河東人也,永嘉之亂,徙家襄陽。祖惔,梁侍中。父暉,都官尚書。抃少聰敏,解屬文,好讀書,所覽將萬卷。仕梁,釋褐著作佐郎。後蕭詧據荊州,以為侍中,領
 國子祭酒、吏部尚書。及梁國廢,拜開府、通直散騎常侍,尋遷內史侍郎。以無吏乾去職,轉晉王諮議參軍。王好文雅,招引才學之士諸葛潁、虞世南、王胄、硃瑒等百餘人以充學士,而抃為之冠。王以師友處之,每有文什,必令其潤色,然後示人。嘗朝京師還,作《歸籓賦》,命抃為序,詞甚典麗。初,王屬文,為庾信體,及見抃已後,文體遂變。仁壽初,引抃為東宮學士,加通直散騎常侍,檢校洗馬,甚見親待。每召入臥內,與之宴謔。抃尤俊辯,多在侍從,有所顧問,應答如響。性又嗜酒,言雜誹諧,由是彌為太子之所親狎。以其好內典,令撰《法華玄宗》,為二十卷,奏
 之。太子覽而大悅,賞賜優洽,儕輩莫與為比。



 煬帝嗣位,拜秘書監,封漢南縣公。帝退朝之後,便命入閣,言宴諷讀,終日而罷。



 帝每與嬪後對酒,時逢興會,輒遣命之至,與同榻共席,恩若友朋。帝猶恨不能夜召,於是命匠刻木偶人,施機關,能坐起拜伏,以像於抃。帝每在月下對酒,輒令宮人置之於座,與相酬酢,而為歡笑。從幸揚州,遇疾卒,年六十九。帝傷惜者久之,贈大將軍,謚曰康。撰《晉王北伐記》十五卷,有集十卷,行於世。



 許善心許善心,字務本,高陽北新城人也。祖懋,梁太子中庶子,
 始平、天門二郡守、散騎常侍。父亨,仕梁至給事黃門侍郎,在陳歷羽林監、太中大夫、衛尉卿,領大著作。善心九歲而孤,為母範氏所鞠養。幼聰明有思理,所聞輒能誦記,多聞默識,為當世所稱。家有舊書萬餘卷,皆遍通涉。十五解屬文,箋上父友徐陵,陵大奇之,謂人曰:「才調極高,此神童也。」起家除新安王法曹。太子詹事江總舉秀才,對策高第,授度支郎中,轉侍郎,補撰史學士。禎明二年,加通直散騎常侍,聘於隋。



 遇高祖伐陳,禮成而不獲反命,累表請辭。上不許,留縶賓館。及陳亡,高祖遣使告之。善心衰服號哭於西階之下,藉草東向,經三日。敕書
 唁焉。明日,有詔就館,拜通直散騎常侍。賜衣一襲。善心哭盡哀,入房改服,復出北面立,垂涕再拜受詔。



 明日乃朝,伏泣於殿下,悲不能興。上顧左右曰:「我平陳國,唯獲此人。既能懷其舊君,即是我誠臣也。」敕以本官直門下省,賜物千段,皁馬二十匹。從幸太山,還授虞部侍郎。



 十六年,有神雀降於含章闥,高祖召百官賜宴,告以此瑞。善心於座請紙筆,制《神雀頌》,其詞曰:臣聞觀象則天,乾元合其德,觀法審地,域大表其尊。雨施雲行,四時所以生殺,川流岳立,萬物於是裁成。出震乘離之君,紀雁司鳳之後,玉錘玉斗而降,金版金縢以傳。並陶冶性靈,含
 煦動植,眇玄珠於赤水,寂明鏡乎虛堂。莫不景福氤氳,嘉貺』集,馳聲南董,越響《云》《韶》。粵我皇帝之君臨,闡大方,抗太極,負鳳邸,據龍圖。不言行焉,攝提建指,不肅清焉,喉鈴啟閉。括地復夏,截海翦商,就望體其尊,登咸昌其會。綿區浹宇,遐至邇安,騰實飛聲,直暢傍施。無體之禮,威儀布政之宮,無聲之樂,綴兆總章之觀。上庠養老,躬問百年,下土字民,心為百姓。月棲日浴,熱阪寒門,吹鱗沒羽之荒,赤蛇青馬之裔,解辮請吏,削衽承風。豈止呼韓北場,頫勒狼居之岫,熄慎南境,近表不耐之城。故使天弗愛道,地寧吝寶,川岳展異,幽明效靈。狎素游赬,
 團膏漱醴,半景青赤,孳歷虧盈。足足懷仁,般般擾義,祥祐之來若此,升隆之化如彼。而登封盛典,雲亭佇白檢之儀,致治成功,柴燎靡玄珪之告。雖奉常定禮,武騎草文,天子抑而未行,推而不有。



 允恭克讓,其在斯乎?七十二君,信蔑如也!故神禽顯賁,玄應特昭,白爵主鐵豸之奇,赤爵銜丹書之貴。班固《神爵》之頌,履武戴文,曹植《嘉爵》之篇,棲庭集牖。未若於飛武帳,來賀文棕,刷採青蒲,將翱赤罽。玉幾朝御,取玩軒楯之間,金門旦開,兼留暈翟之鑒。終古曠世,未或前聞,福召冥徵,得之茲日。歲次上章,律諧大呂,玄枵會節,玄英統時。至尊未明求衣,晨
 興於含章之殿。爰有瑞爵,翱翔而下。載行載止,當扆寧而徐前,來集來儀,承軒墀而顧步。夫瑞者符也,明主之休徵;雀者爵也,聖人之大寶。謹案《考異郵》云:「軒轅有黃爵赤頭,立日傍。」



 占云:「土精之應。」又《禮稽命徵》云:「祭祀合其宜,則黃爵集。」昔漢集泰畤之殿,魏下文昌之宮,一見雍丘之祠,三入平東之府,並旁觀回矚,事陋人微,奚足稱矣。抑又聞之,不刳胎剖卵,則鸞鳳馴鳴;不漉浸焚原,則螭龍盤蜿。是知陛下止殺,故飛走宅心,皇慈好生,而浮潛育德。臣面奉綸綍,垂示休祥,預承嘉宴,不勝藻躍。李虔僻處西土,陸機少長東隅,微臣慚於往賢,逢時盛
 乎曩代,輒竭庸瑣,敢獻頌云:太素式肇,大德資生,功玄不器,道要無名。質文鼎革,沿習因成,祥圖瑞史,赫赫明明。天保大定,於鑠我君,武義乃武,文教惟文。橫塞宇宙,旁凝射、汾,軒物重造,姚風再薰。煥發王策,昭彰帝道,御地七神,飛天五老。山祗吐秘,河靈孕寶,黑羽升壇,青鱗伏皁。丹烏流火,白雉從風,棲阿德劭,鳴岐祚隆。未如神爵,近賀王宮,五靈何有,百福攸同。孔圖獻赤,荀文表白,節節奇音,行行瑞跡。化玉黼扆,銜環陛戟,上天之命,明神所格。綏應在旃,伊臣預焉,永緝韋素,方流管弦。頌歌不足,蹈儛無宣,臣拜稽首,億萬斯年。



 頌成,奏之,高祖甚
 悅,曰:「我見神雀,共皇後觀之。今旦召公等入,適述此事,善心於座始知,即能成頌,文不加點,筆不停毫,常聞此言,今見其事。」



 因賜物二百段。十七年,除秘書丞。於時秘藏圖籍尚多淆亂,善心放阮孝緒《七錄》,更制《七林》,各為總敘,冠於篇首。又於部錄之下,明作者之意,區分其類例焉。



 又奏追李文博、陸從典等學者十許人,正定經史錯謬。仁壽元年,攝黃門侍郎。二年,加攝太常少卿,與牛弘等議定禮樂,秘書丞、黃門,並如故。四年,留守京師。



 高祖崩於仁壽宮,煬帝秘喪不發,先易留守官人,出除巖州刺史。逢漢王諒反,不之官。



 大業元年,轉禮部侍郎,奏
 薦儒者徐文遠為國子博士,包愷、陸德明、褚徽、魯世達之輩並加品秩,授為學官。其年,副納言楊達為冀州道大使,以稱旨,賜物五百段。左衛大將軍宇文述每旦借本部兵數十人以供私役,常半日而罷。攝御史大夫梁毗奏劾之。上方以腹心委述,初付法推,千餘人皆稱被役。經二十餘日,法官候伺上意,乃言役不滿日,其數雖多,不合通計,縱令有實,亦當無罪。諸兵士聞之,更雲初不被役。上欲釋之,付議虛實,百僚咸議為虛。善心以為述於仗衛之所抽兵私役,雖不滿日,闕於宿衛,與常役所部,情狀乃殊。又兵多下番,散還本府,分道追至,不謀
 同辭。今殆一月,方始翻覆,奸狀分明,此何可舍。蘇威、楊汪等二十餘人,同善心之議。其餘皆議免罪。煬帝可免罪之奏。後數月,述譖善心曰:「陳叔寶卒,善心與周羅、虞世基、袁充、蔡徵等同往送葬。善心為祭文,謂為陛下,敢於今日加叔寶尊號。」召問有實,自援古例,事得釋,而帝甚惡之。又太史奏帝即位之年,與堯時符合,善心議,以國哀甫爾,不宜稱賀。述諷御史劾之,左遷給事郎,降品二等。四年,撰《方物志》奏之。七年,從至涿郡,帝方自御戎以東討,善心上封事忤旨,免官。其年復徵為守給事郎。九年,攝左翊衛長史,從渡遼,授建節尉。帝嘗言及高
 祖受命之符,因問鬼神之事,敕善心與崔祖濬撰《靈異記》十卷。



 初,善心父撰著《梁史》,未就而歿。善心述成父志,修續家書,其《序傳》末,述制作之意曰:謹案太素將萌,洪荒初判,乾儀資始,辰象所以正時,巛載厚生,品物於焉播氣。參三才而育德,肖二統而降靈。有人民焉,樹之君長,有貴賤矣,為其宗極。



 保上天之眷命,膺下土之樂推,莫不執大方,振長策,感召風雲,驅馳英俊。干戈揖讓,取之也殊功,鼎玉龜符,成之也一致。革命創制,竹素之道稍彰,紀事記言,筆墨之官漸著。炎農以往,存其名而漏其跡,黃軒以來,晦其文而顯其用。登丘納麓,具訓誥及
 典謨,貫昴入房,傳夏正與殷祀。洎辨方正位,論時訓功,南北左右,兼四名之別,檮杌乘車,擅一家之稱。國惡雖諱,君舉必書,故賊子亂臣,天下大懼,元龜明鏡,昭然可察。及三郊遞襲,五勝相沿,俱稱百穀之主,並以四海自任,重光累德,何世無哉!



 逮有梁之君臨天下,江左建國,莫斯為盛。受命在於一君,繼統傳乎四主,克昌四十八載,餘祚五十六年。武皇帝出自諸生,爰升寶歷,拯百王之弊,救萬姓之危,反澆季之末流,登上皇之獨道。朝多君子,野無遺賢,禮樂必備,憲章咸舉。



 弘深慈於不殺,濟大忍於無刑,蕩蕩巍巍,可為稱首。屬陰戎入潁,羯胡侵
 洛,沸騰磣黷,三季所未聞,掃地滔天,一元之巨厄。廊廟有序,翦成狐兔之場,珪帛有儀,碎夫犬羊之手。福善積而身禍,仁義在而國亡。豈天道歟?豈人事歟?嘗別論之,在《序論》之卷。



 先君昔在前代,早懷述作,凡撰《齊書》為五十卷;《梁書》紀傳,隨事勒成,及闕而未就者,《目錄》注為一百八卷。梁室交喪,墳籍銷盡。塚壁皆殘,不準無所盜,帷囊同毀,陳農何以求!秦儒既坑,先王之道將墜,漢臣徒請,口授之文亦絕。所撰之書,一時亡散。有陳初建,詔為史官,補闕拾遺,心識口誦。依舊目錄,更加修撰,且成百卷,已有六帙五十八卷,上秘閣訖。



 善心早嬰荼蓼,弗荷
 薪構,太建之末,頻抗表聞,至德之初,蒙授史任。方願油素採訪,門庭記錄,俯勵弱才,仰成先志;而單宗少強近,虛室類原、顏,退屏無所交游,棲遲不求進益。假班嗣之書,徒聞其語,給王隱之筆,未見其人。加以庸瑣涼能,孤陋末學,忝職郎署,兼撰《陳史》,致此書延時,未即成續。禎明二年,以臺郎入聘,值本邑淪覆,他鄉播遷,行人失時,將命不復。望都亭而長慟,遷別館而懸壺。家史舊書,在後焚蕩。今止有六十八卷在,又並缺落失次。自入京已來,隨見補葺,略成七十卷。《四帝紀》八卷,《後妃》一卷,《三太子錄》一卷,為一帙十卷。《宗室王侯列傳》一帙十卷。《具臣
 列傳》二帙二十卷。《外戚傳》一卷,《孝德傳》一卷,《誠臣傳》一卷,《文苑傳》二卷,《儒林傳》二卷,《逸民傳》一卷,《數術傳》一卷,《籓臣傳》一卷,合一帙十卷。《止足傳》一卷,《列女傳》一卷,《權幸傳》一卷,《羯賊傳》二卷,《逆臣傳》二卷,《叛臣傳》二卷,《敘傳論述》一卷,合一帙十卷。凡稱史臣者,皆先君所言,下稱名案者,並善心補闕。別為《敘論》一篇,托於《敘傳》之末。



 十年,又從至懷遠鎮,加授朝散大夫。突厥圍雁門,攝左親衛武賁郎將,領江南兵宿衛殿省。駕幸江都郡,追敘前勛,授通議大夫。詔還本品,行給事郎。十四年,化及殺逆之日,隋官盡詣朝堂謁賀,善心獨不至。許弘仁馳告之
 曰:「天子已崩,宇文將軍攝政,合朝文武莫不咸集。天道人事,自有代終,何預於叔而低徊若此!」善心怒之,不肯隨去。弘仁反走上馬,泣而言曰:「將軍於叔全無惡意,忽自求死,豈不痛哉!」還告唐奉義,以狀白化及,遣人就宅執至朝堂。化及令釋之,善心不舞蹈而出。化及目送之曰:「此人大負氣。」命捉將來,罵云:「我好欲放你,敢如此不遜!」其黨輒牽曳,因遂害之,時年六十一。及越王稱制,贈左光祿大夫、高陽縣公,謚曰文節。



 善心母範氏,梁太子中舍人孝才之女,少寡養孤,博學有高節。高祖知之,敕尚食每獻時新,常遣分賜。嘗詔範入內,侍皇后講讀,封
 永樂郡君。及善心遇禍,範年九十有二,臨喪不哭,撫柩曰:「能死國難,我有兒矣。」因臥不食,後十餘日亦終。



 李文博博陵李文博,性貞介鯁直,好學不倦,至於教義名理,特所留心。每讀書至治亂得失,忠臣列士,未嘗不反覆吟玩。開皇中,為羽騎尉,特為吏部侍郎薛道衡所知,恆令在聽事帷中披檢書史,並察己行事。若遇治政善事,即抄撰記錄,如選用疏謬,即委之臧否。道衡每得其語,莫不欣然從之。後直秘書內省,典校墳籍,守道居貧,晏如也。雖衣食乏絕,而清操逾厲,不妄通賓客,恆以禮法自
 處,儕輩莫不敬憚焉。道衡知其貧,每延於家,給以資費。文博商略古今,治政得失,如指諸掌,然無吏乾。稍遷校書郎。後出為縣丞,遂得下考,數歲不調。道衡為司隸大夫,遇之於東都尚書省,甚嗟愍之,遂奏為從事。因為齊王司馬李綱曰:「今日遂遇文博,得奏用之。」以為歡笑。其見賞知音如此。在洛下,曾詣房玄齡,相送於衢路。



 玄齡謂之曰:「公生平志尚,唯在正直,今既得為從事,故應有會素心。比來激濁揚清,所為多少?」文博遂奮臂厲聲曰:「夫清其流者必潔其源,正其末者須端其本。今治源混亂,雖日免十貪郡守,亦何所益!」其瞽直疾惡,不知忌諱,
 皆此類也。於時朝政浸壞,人多贓賄,唯文博不改其操,論者以此貴之。遭離亂播遷,不知所終。



 初,文博在內校書,虞世基子亦在其內,盛飾容服,而未有所卻。文博因從容問之年紀,答云:「十八。」文博乃謂之曰:「昔賈誼當此之年,議論何事?君今徒事儀容,故何為者!」又秦孝王妃生男,高祖大喜,頒賜群官各有差。文博家道屢空,人謂其悅,乃云:「賞罰之設,功過所歸,今王妃生男,於群官何事,乃妄受賞也!」其循名責實,錄過計功,必使賞罰不濫,功過無隱者皆爾。文博本為經學,後讀史書,於諸子及論尤所該洽。性長議論,亦善屬文,著《治道集》十卷,大行
 於世。



 史臣曰:明克讓、魏澹等,或博學洽聞,詞藻贍逸,既稱燕趙之俊,實曰東南之美。所在見寶,咸取祿位,雖無往非命,蓋亦道有存焉。澹之《魏書》,時稱簡正,條例詳密,足傳於後。此外諸子,各有記述,雖道或小大,皆志在立言,美矣。



\end{pinyinscope}