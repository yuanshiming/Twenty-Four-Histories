\article{卷二十二志第十七 五行上}

\begin{pinyinscope}

 《易》以八卦定吉兇,則庖犧所以稱聖也。《書》以九疇論休咎,則大禹所以為明也。《春秋》以災祥驗行事,則仲尼所以垂法也。天道以星象示廢興,則甘、石所以先知也。是以祥符之兆可得而言,妖訛之占所以徵驗。夫神則陰陽不測,天則欲人遷善,均乎影響,殊致同歸。漢時有伏生、董仲舒、京房、劉向之倫,能言災異,顧盼六經,有足觀
 者。劉向曰:「君道得則和氣應,休徵生。君道違則乖氣應,咎徵發。」夫天有七曜,地有五行。五事愆違則天地見異,況於日月星辰乎?



 況於水火金木土乎?若梁武之降號伽藍,齊文宣之盤游市里,陳則蔣山之鳥呼曰「奈何」,周則陽武之魚集空而鬥,隋則鵲巢黼帳,火炎門闕,豈唯天道,亦曰人妖,則祥眚呈形,於何不至?亦有脫略政教,張羅樽糈,崇信巫史,重增愆罰。昔懷王事神而秦兵逾進,萇弘尚鬼而諸侯不來。性者,生之靜也。欲者,心之使也。



 置情攸往,引類同歸。雀乳於空城之側,鷮飛於鼎耳之上。短長之制,既曰由人;黔隧崇山,同車共軫。必有神
 道,裁成倚伏。一則以為殃釁,一則以為休徵。故曰德勝不祥而義厭不惠。是以聖王常由德義,消伏災咎也。



 《洪範五行傳》曰:「木者東方,威儀容貌也。古者聖王垂則,天子穆穆,諸侯皇皇。登輿則有鸞和之節,降車則有佩玉之度,田狩則有三驅之制,飲食則有享獻之禮。無事不出境。此容貌動作之得節,所以順木氣也。如人君違時令,失威儀,田獵馳騁,不反宮室,飲食沉湎,不顧禮制,縱欲恣睢,出入無度,多徭役以奪人時,增賦稅以奪人財,則木不曲直。」



 齊後主武平五年,鄴城東青桐樹,有如人狀。京房《易傳》
 曰:「王德衰,下人將起,則有木生為人狀。」是時後主怠於國政,耽荒酒色,威儀不肅,馳騁無度,大發徭役,盛修宮室,後二歲而亡。木不曲直之效也。



 七年,宮中有樹,大數圍,夜半無故自拔。齊以木德王,無故自拔,亡國之應也。其年,齊亡。



 開皇八年四月,幽州人家以白楊木懸灶上,積十餘年,忽生三條,皆長三尺餘,甚鮮茂。仁壽二年春,盩厔人以楊木為屋梁,生三條,長二尺。京房《易傳》曰:「妃後有顓,木僕反立,斷枯復生。」獨孤後專恣之應也。



 仁壽元年十月,蘭州楊樹上松生,高三尺,六節十二枝,《
 宋志》曰:「松不改柯易葉,楊者危脆之木,此永久之業,將集危亡之地也。」是時帝惑讒言,幽廢塚嫡,初立晉王為皇太子。天戒若曰,皇太子不勝任,永久之業,將致危亡。帝不悟。及帝崩,太子立,是為煬帝,竟以亡國。



 仁壽四年八月,河間柳樹無故枯落,既而花葉復生。京房《易飛候》曰:「木再榮,國有大喪。」是歲,宮車晏駕。



 《洪範五行傳》曰:「金者西方,萬物既成,殺氣之始也。古之王者,興師動眾,建立旗鼓,以誅殘賊,禁暴虐,安天下,殺伐必應義,以順金氣。如人君樂侵陵,好攻戰,貪城邑之賂,以輕百姓之命,人皆不安,外內騷動,則金不從革。」



 陳禎明二年五月,東冶鐵鑄,有物赤色,大如斗,自天墜熔所。隆隆有聲,鐵飛破屋而四散,燒人家。時後主與隋雖結和好,遣兵度江,掩襲城鎮,將士勞敝,府藏空竭。東冶者,陳人鑄兵之所。鐵飛為變者,金不從革之應。天戒若曰,陳國小而兵弱,當以和好為固,無鑄兵而黷武,以害百姓。後主不悟,又遣偽將陳紀、任蠻奴、蕭摩訶數寇江北,百姓不堪其役。及隋師渡江,而二將降款,卒以滅亡。



 《洪範五行傳》曰:「火者南方,陽光為明也。人君向南,蓋取象也。昔者聖帝明王,負扆攝袂,南面而聽斷天下。攬海
 內之雄俊,積之於朝,以續聰明,推邪佞之偽臣,投之於野,以通壅塞,以順火氣。夫不明之君,惑於讒口,白黑雜揉,代相是非,眾邪並進,人君疑惑。棄法律,間骨肉,殺太子,逐功臣,以孽代宗,則火失其性。」



 梁天監元年五月,有盜入南、北掖,燒神武門總章觀。時帝初即位,而火燒觀闕,不祥之甚也。既而太子薨,皇孫不得立。及帝暮年,惑於硃異之口,果有侯景之亂,宮室多被焚燒。天誡所以先見也。



 普通二年五月,琬琰殿火,延燒後宮三千餘間。中大通元年,硃雀航華表災。



 明年,同泰寺災。大同三年,硃雀門
 災。水沴火也。是時帝崇尚佛道,宗廟牲牷,皆以面代之,又委萬乘之重,數詣同泰寺,舍身為奴,令王公已下贖之。初陽為不許,後為默許,方始還宮。天誡若曰,梁武為國主,不遵先王之法,而淫於佛道,橫多糜費,將使其社稷不得血食也。天數見變而帝不悟,後竟以亡。及江陵之敗,闔城為賤隸焉,即舍身為奴之應也。



 陳永定三年,重雲殿災。



 東魏天平二年十一月,閶闔門災。是時齊神武作宰,而大野拔斬樊子鵠,以州來降,神武聽讒而殺之。司空元暉免。逐功臣大臣之罰也。



 武定五年八月,廣宗郡火,燒數千家。



 後齊後主天統三年,九龍殿災,延燒西廊。四年,昭陽、宣光、瑤華三殿災,延燒龍舟。是時讒言任用,正士道消,祖孝徵作歌謠,斛律明月以誅死。讒夫昌,邪勝正之應也。京房《易傳》曰:「君不思道,厥妖火燒宮。」



 開皇十四年,將祠泰山,令使者致石像神祠之所。未至數里,野火欻起,燒像碎如小塊。時帝頗信讒言,猜阻骨肉,滕王瓚失志而死,創業功臣,多被夷滅,故天見變,而帝不悟,其後太子勇竟被廢戮。



 大業十二年,顯陽門災,舊名廣陽,則帝之姓名也。國門
 之崇顯,號令之所由出也。時帝不遵法度,驕奢荒怠,裴蘊、虞世基之徒,阿諛順旨,掩塞聰明,宇文述以讒邪顯進,忠諫者咸被誅戮。天戒若曰,信讒害忠,則除「廣陽」也。



 《洪範五行傳》曰:「水者,北方之藏,氣至陰也。宗廟者,祭祀之象也。故天子親耕以供粢盛,王後親蠶以供祭服。敬之至也。發號施令,十二月咸得其氣,則水氣順。如人君簡宗廟,不禱祀,逆天時,則水不潤下。」



 梁天監二年六月,太末、信安、豐安三縣大水。《春秋考異郵》曰:「陰盛臣逆人悲,則水出河決。」是時江州刺史陳伯之、益州刺史劉季連舉兵反叛,師旅數興,百姓愁怨,臣
 逆人悲之應也。



 六年八月,建康大水,濤上御道七尺。七年五月,建康又大水。是時數興師旅,以拒魏軍。十二年四月,建康大水。是時大發卒築浮山堰,以遏淮水,勞役連年,百姓悲怨之應也。



 中大通五年五月,建康大水,御道通船。京房《易飛候》曰:「大水至國,賤人將貴。」蕭棟、侯景僭稱尊號之應也。



 後齊河清二年十二月,兗、趙、魏三州大水。天統三年,並州汾水溢。曰:「水者純陰之精,陰氣洋溢者,小人專制。」是時和士開、元文遙、趙彥深專任之應也。



 武平六年八月,山東諸州大水。京房《易飛候》曰:「小人踴躍,無所畏忌,陰不制於陽,則湧水出。」是時群小用事,邪佞滿朝。閹豎嬖幸,伶人封王。此其所以應也。



 開皇十八年,河南八州大水。是時獨孤皇后干預政事,濫殺宮人,放黜宰相。



 楊素頗專。水陰氣,臣妾盛強之應也。



 仁壽二年,河南、河北諸州大水。京房《易傳》曰:「顓事有智,誅罰絕理,則厥災水。」亦由帝用刑嚴急,臣下有小過,帝或親臨斬決,又先是柱國史萬歲以忤旨被戮,誅罰絕理之應也。



 大業三年,河南大水,漂沒三十餘郡。帝嗣位已來,未親郊廟之禮,簡宗廟,廢祭祀之應也。



 《洪範五行傳》曰:「土者中央,為內事。宮室臺榭,夫婦親屬也。古者自天子至於士,宮室寢居,大小有差,高卑異等,骨肉有恩。故明王賢君,修宮室之制,謹夫婦之別,加親戚之思,敬父兄之禮,則中氣和。人君肆心縱意,大為宮室,高為臺榭,雕文刻鏤,以疲人力,淫泆無別,妻妾過度,犯親戚,侮父兄,中氣亂,則稼穡不成。」



 齊後主武平四年,山東饑。是時,大興土木之功於仙都苑。又起宮於邯鄲,窮侈極麗。後宮侍御千餘人,皆寶衣
 玉食。逆中氣之咎也。



 煬帝大業五年,燕、代、齊、魯諸郡饑。先是建立東都,制度崇侈。又宗室諸王,多遠徙邊郡。



 《洪範五行傳》曰:「貌之不恭,是謂不肅,則下不敬。陰氣勝,故厥咎狂,厥罰常雨,厥極惡。時則有服妖,時則有龜孽,有雞禍,有下體生上體之痾,有青眚青祥。惟金沴木。」



 貌不恭侯景僭即尊號,升圓丘,行不能正履,有識者知其不免。景尋敗。



 梁元帝既平侯景,破蕭紀,而有驕矜之色。性又沉猜,由
 是臣下離貳。既位三年而為西魏所陷,帝竟不得其死。



 陳後主每祀郊廟,必稱疾不行。建寧令章華上奏諫曰:「拜三妃以臨軒,祀宗廟而稱疾,非祗肅之道。」後主怒而斬之。又引江總、孔範等內宴,無復尊卑之序,號為狎客,專以詩酒為娛,不恤國政。秘書監傅縡上書諫曰:「人君者,恭事上帝,子愛下人,省嗜欲,遠邪佞,未明求衣,日旰忘食,是以澤被區宇,慶流子孫。陛下頃來,酒色過度,不虔郊廟大神,專媚淫昏之鬼。小人在側,宦豎擅權,惡誠直如仇讎,視時人如草芥。後宮曳羅綺,廄馬餘菽粟,百姓流離,轉尸蔽野。神怒人怨,眾叛親離。臣恐東南王氣,自
 斯而盡。」後主不聽,驕恣日甚。未幾而國滅。



 陳司空侯安都,自以有安社稷之功,驕矜日甚,每侍宴酒酣,輒箕踞而坐。嘗謂文帝曰:「何如作臨川王時?」又借華林園水殿,與妻妾賓客置酒於其上,帝甚惡之。後竟誅死。



 東魏武定五年,後齊文襄帝時為世子,屬神武帝崩,秘不發喪,朝魏帝於鄴。



 魏帝宴之,文襄起儛。及嗣位,又朝魏帝於鄴,侍宴而惰。有識者知文襄之不免。



 後果為盜所害。



 神武時,司徒高昂嘗詣相府,將直入門,門者止之。昂怒,
 引弓射門者,神武不之罪。尋為西魏所殺。



 後齊後主為周師所迫,至鄴集兵。斛律孝卿勸後主親勞將士,宜流涕慷慨,以感激之,人當自奮。孝卿授之以辭,後主然之。及對眾,默無所言,因赧然大笑,左右皆哂。將士怒曰:「身尚如此,吾輩何急!」由是皆無戰心,俄為周師所虜。



 煬帝自負才學,每驕天下之士。嘗謂侍臣曰:「天下當謂朕承藉餘緒而有四海耶?設令朕與士大夫高選,亦當為天子矣。」謂當世之賢,皆所不逮。《書》云:「謂人莫己若者亡。」帝自矜己以輕天下,能不亡乎?帝又言習吳音,其後
 竟終於江都,此亦魯襄公終於楚宮之類也。



 常雨水梁天監七年七月,雨,至十月乃霽。《洪範五行傳》曰:「陰氣強積,然後生水雨之災。」時武帝頻年興師,是歲又大舉北伐,諸軍頗捷,而士卒罷敝,百姓怨望,陰氣畜積之應也。



 陳太建十二年八月,大雨霪霖。時始興王叔陵驕恣,陰氣盛強之應也。明年,宣帝崩,後主立。叔陵刺後主於喪次。宮人救之,人堇而獲免。叔陵出閤,就東府作亂。後主令蕭摩訶破之,死者千數。



 東魏武定五年秋,大雨七十餘日,元瑾、劉思逸謀殺後齊文襄之應也。



 後齊河清三年六月庚子,大雨,晝夜不息,至甲辰。山東大水,人多餓死。是歲,突厥寇並州,陰戎作梗,此其應也。



 天統三年十月,積陰大雨。胡太后淫亂之所感也。



 武平七年七月,大霖雨,水澇,人戶流亡。是時駱提婆、韓長鸞等用事,小人專政之罰也。



 後周建德三年七月,霖雨三旬。時衛刺王直潛謀逆亂。屬帝幸雲陽宮,以其徒襲肅章門,尉遲運逆拒破之。其日雨霽。



 大雨雪梁普通二年三月,大雪,平地三尺。《洪範五行傳》曰:「庶徵之常,雨也,然尤甚焉。雨,陰也;雪,又陰畜積甚盛也。皆妾不妾、臣不臣之應。」時義州刺史文僧朗以州叛於魏,臣不臣之應也。



 大同三年七月,青州雪,害苗稼。是時交州刺史李賁舉兵反,僭尊號,置百官,擊之不能克。



 十年十二月,大雪,平地三尺。是時邵陵王綸、湘東王繹、武陵王紀並權侔人主,頗為驕恣,皇太子甚惡之,帝不能抑損。上天見變,帝又不悟。及侯景之亂,諸王各擁強
 兵,外有赴援之名,內無勤王之實,委棄君父,自相屠滅,國竟以亡。



 東魏興和二年五月,大雪。時後齊神武作宰,發卒十餘萬築鄴城,百姓怨思之徵也。



 武定四年二月,大雪,人畜凍死,道路相望。時後齊霸政,而步落稽舉兵反,寇亂數州,人多死亡。



 後齊河清二年二月,大雪連雨,南北千餘里,平地數尺,繁霜晝下。是時突厥木桿可汗與周師入並州,殺掠吏人,不可勝紀。



 天統二年十一月,大雪;三年正月,又大雪,平地二尺;武
 平三年正月,又大雪。是時馮淑妃、陸令萱內制朝政,陰氣盛積,故天變屢見,雷雨不時。



 陳太建元年七月,大雨;震萬安陵華表,又震慧日寺剎,瓦官寺重閤門下一女子震死。京房《易飛候》曰:「雷雨霹靂丘陵者,逆先人令,為火殺人者,人君用讒言殺正人。」時蔡景歷以奸邪任用,右僕射陸繕以讒毀獲譴,發病而死。



 十年三月,震武庫。時帝好兵,頻年北伐,內外虛竭,將士勞敝。既克淮南,又進圖彭、汴,毛喜切諫,不納。由是吳明徹諸軍皆沒,遂失淮南之地。武庫者,兵器之所聚也,而
 震之,天戒若曰,宜戢兵以安百姓。帝不悟,又大興軍旅,其年六月,又震太皇寺剎、莊嚴寺露槃、重陽閣東樓、鴻臚府門。太皇、莊嚴二寺,陳國奉佛之所,重陽閣每所游宴,鴻臚賓客禮儀之所在,而同歲震者,天戒若曰,國威已喪,不務修德,後必有恃佛道,耽宴樂,棄禮儀而亡國者。陳之君臣竟不悟。



 至後主之代,災異屢起,懼而於太皇寺舍身為奴,以祈冥助,不恤國政,耽酒色,棄禮法,不修鄰好,以取敗亡。



 齊武平元年夏,震丞相段孝先南門柱。京房《易傳》曰:「震擊貴臣門及屋者,不出三年,佞臣被誅。」後歲,和士開被
 戮。



 木冰東魏武定四年冬,天雨木冰。《洪範五行傳》曰:「陰之盛而凝滯也。木者少陽,貴臣象也。將有害,則陰氣脅木,木先寒,故得雨而冰襲之。木冰一名介,介者兵之象也。」時司徒侯景制河南,及神武不豫,文襄懼其為亂而征之,景因舉兵反。豫州刺史高元成、襄州刺史李密、廣州刺史暴顯並為景所執辱,貴臣有害之應也。其後左僕射慕容紹宗與景戰於渦陽,俘斬五萬。



 後齊天保二年,雨木冰三日。初,清河王岳為高歸彥所
 譖,是歲以憂死。



 武平元年冬,雨木冰;明年二月,又木冰。時錄尚書事和士開專政。其年七月,太保、瑯邪王儼矯詔殺之。領軍大將軍庫狄伏連、尚書右僕射馮子琮並坐儼賜死。



 九月,儼亦遇害。



 六年、七年,頻歲春冬木冰。其年周師入晉陽,因平鄴都。後主走青州,貴臣死散,州郡被兵者不可勝數。



 大雨雹梁中大通元年四月,大雨雹。《洪範五行傳》曰:「雹,陰脅陽之象也。」時帝數舍身為奴,拘信佛法,為沙門所制。



 陳太建二年六月,大雨雹;十年四月,又大雨雹;十三年九月,又雨雹。時始興王叔陵驕恣,陰結死士,圖為不逞,帝又寵遇之,故天三見變。帝不悟。及帝崩,叔陵果為亂逆。



 服妖後齊婁後臥疾,寢衣無故自舉。俄而後崩。



 文宣帝末年,衣錦綺,傅粉黛,數為胡服,微行市里。粉黛者,婦人之飾,陽為陰事,君變為臣之象也。及帝崩,太子嗣位,被廢為濟南王。又齊氏出自陰山,胡服者,將反初服也。錦彩非帝王之法服,微服者布衣之事,齊亡之效
 也。



 後主好令宮人以白越布折額,狀如髽幗;又為白蓋。此二者,喪禍之服也。後主果為周武帝所滅,父子同時被害。



 武平時,後主於苑內作貧兒村,親衣襤褸之服而行乞其間,以為笑樂。多令人服烏衣,以相執縛。後主果為周所敗,被虜於長安而死;妃後窮困,至以賣燭為業。



 後周大象元年,服冕二十有四旒,車服旗鼓,皆以二十四為節。侍衛之官,服五色,雜以紅紫。令天下車以大木為輪,不施輻。朝士不得佩綬,婦人墨妝黃眉。



 又造下帳,
 如送終之具,令五皇后各居其一,實宗廟祭器於前,帝親讀版而祭之。



 又將五輅載婦人,身率左右步從。又倒懸雞及碎瓦於車上,觀其作聲,以為笑樂。



 皆服妖也。帝尋暴崩,而政由於隋,周之法度,皆悉改易。



 開皇中,房陵王勇之在東宮,及宜陽公王世積家,婦人所有領巾制同槊幡軍幟。



 婦人為陰,臣象也,而服兵幟,臣有兵禍之應矣。勇竟而遇害,世積坐伏誅。



 雞禍開皇中,有人上書,言頻歲已來,雞鳴不鼓翅,類腋下有物而妨之,翮不得舉,肘腑之臣,當為變矣。書奏不省。京
 房《易飛候》曰:「雞鳴不鼓翅,國有大害。」



 其後大臣多被夷滅,諸王廢黜,太子幽廢。



 大業初,天下雞多夜鳴,京房《易飛候》曰:「雞夜鳴,急令。」又云:「昏而鳴,百姓有事;人定鳴,多戰;夜半鳴,流血漫漫。」及中年已後,軍國多務,用度不足,於是急令暴賦,責成守宰,百姓不聊生矣,各起而為盜,戰爭不息,尸骸被野。



 龜孽開皇中,掖庭宮每夜有人來挑宮人。宮司以聞。帝曰:「門衛甚嚴,人何從而入?當是妖精耳。」因戒宮人曰:「若逢,但斫之。」其後有物如人,夜來登床,宮人抽刀斫之,若中枯
 骨。其物落床而走,宮人逐之,因入池而沒。明日,帝令涸池,得一龜,徑尺餘,其上有刀跡。殺之,遂絕。龜者水居而靈,陰謀之象,晉王諂媚宮掖求嗣之應云。



 青眚青祥陳禎明二年四月,群鼠無數,自蔡洲岸入石頭淮,至青塘兩岸。數日死,隨流出江。近青祥也。京房《易飛候》曰:「鼠無故群居,不穴眾聚者,其君死。」未幾而國亡。



 金沴木陳天嘉六年秋七月,儀賢堂無故自壓,近金沴木也。時帝盛修宮室,起顯德等五殿,稱為壯麗,百姓失業,故木
 失其性也。儀賢堂者,禮賢尚齒之謂,無故自壓,天戒若曰,帝好奢侈,不能用賢使能,何用虛名也。帝不悟,明年竟崩。



 禎明元年六月,宮內水殿若有刀鋸斫伐之聲,其殿因無故而倒。七月,硃雀航又無故自沉。時後主盛修園囿,不虔宗廟。水殿者,游宴之所,硃雀航者,國門之大路,而無故自壞,天戒若曰,宮室毀,津路絕。後主不悟,竟為隋所滅,宮廟為墟。



 後齊孝昭帝將誅楊愔,乘車向省,入東門,「W竿無故自折。帝甚惡之,歲餘而崩。



 河清三年,長廣郡聽事梁忽剝若人狀,太守惡而削去之,明日復然。長廣,帝本封也,木為變,不祥之兆。其年帝崩。



 武平七年秋,穆後將如晉陽,向北宮辭胡太后。至宮內門,所乘七寶車無故陷入於地,牛沒四足。是歲齊滅,後被虜於長安。



 後周建德六年,青城門無故自崩。青者東方色,春宮之象也。時皇太子無威儀禮節,青城門無故自崩者,皇太子不勝任之應。帝不悟。明年太子嗣位,果為無道。



 周室危亡,實自此始。



 大業中,齊王暕於東都起第,新構寢堂,其栿無故而折。時上無太子,天下皆以暕次當立,公卿屬望,暕遂驕恣,呼術者令相,又為厭勝之事。堂栿無故自折,木失其性,奸謀之應也。天見變以戒之,暕不悟,後竟得罪於帝。



 《洪範五行傳》曰:「言之不從,是謂不乂。厥咎僭,厥罰常暘,厥極憂。時則有詩妖,時則有毛蟲之孽,時則有犬禍。故有口舌之痾,有白眚白祥。惟木沴金。



 言不從梁武陵王紀僭即帝位,建元曰天正。永豐侯蕭捴曰:「王不克矣。昔桓玄年號大亨,有識者以為『二月了』,而玄之
 敗,實在仲春。今日天正,正之為文『一止』,其能久乎!」果一年而敗。



 後齊文宣帝時,太子殷當冠,詔令邢子才為制字。子才字之曰正道。帝曰:「正,一止也。吾兒其替乎?」子才請改,帝不許,曰:「天也。」因顧謂常山王演曰:「奪時任汝,慎無殺也。」及帝崩,太子嗣位,常山果廢之而自立。殷尋見害。



 武成帝時,左僕射和士開言於帝曰:「自古帝王,盡為灰土,堯舜、桀紂,竟亦何異。陛下宜及少壯,恣意歡樂,一日可以當千年,無為自勤約也。」帝悅其言,彌加淫侈。士開既導帝以非道,身又擅權,竟為御史中丞所殺。



 武平中,陳人寇彭城,後主發言憂懼,侍中韓長鸞進曰:「縱失河南,猶得為龜茲國子。淮南今沒,何足多慮。人生幾何時,但為樂,不須憂也。」帝甚悅,遂耽荒酒色,不以天下為虞。未幾,為周所滅。



 武平七年,後主為周師所敗,走至鄴,自稱太上皇,傳位於太子恆,改元隆化。



 時人離合其字曰「降死」。竟降周而死。



 周武帝改元為宣政,梁主蕭巋離合其字為「宇文亡日」。其年六月,帝崩。



 宣帝在東宮時,不修法度,武帝數撻之。及嗣位,摸其痕
 而大罵曰:「死晚也。」



 年又改元為大象,蕭巋又離合其字曰「天子塚」。明年而帝崩。



 開皇初,梁王蕭琮改元為廣運。江陵父老相謂曰:「運之為字,軍走也。吾君當為軍所走乎?」其後琮朝京師而被拘留不反,其叔父巖掠居人以叛,梁國遂廢。



 文帝名皇太子曰勇,晉王曰英,秦王曰俊,蜀王曰秀。開皇初,有人上書曰:「勇者一夫之用。又千人之秀為英,萬人之秀為俊。斯乃布衣之美稱,非帝王之嘉名也。」帝不省。時人呼楊姓多為嬴者。或言於上曰:「楊英反為嬴殃。」帝聞而不懌,遽改之。其後勇、俊、秀皆被廢黜,煬帝嗣位,
 終失天下,卒為楊氏之殃。



 煬帝即位,號年曰大業。識者惡之,曰:「於字離合為『大苦來』也。」尋而天下喪亂,率土遭荼炭之酷焉。



 煬帝常從容謂秘書郎虞世南曰:「我性不欲人諫。若位望通顯而來諫我,以求當世之名者,彌所不耐。至於卑賤之士,雖少寬假,然卒不置之於地。汝其知之!」



 時議者以為古先哲王之馭天下也,明四目,達四聰,懸敢諫之鼓,立書謗之木,以聞言者之路,猶恐忠言之不至。由是澤敷四海,慶流子孫。而帝惡直言,讎諫士,其能久乎!竟逢殺逆。



 旱梁天監元年,大旱,米斗五千,人多餓死。《洪範五行傳》曰:「君持亢陽之節,興師動眾,勞人過度,以起城邑,不顧百姓,臣下悲怨。然而心不能縱,故陽氣盛而失度,陰氣沉而不附。陽氣盛,旱災應也。」初,帝起兵襄陽,破張沖,敗陳伯之,及平建康,前後連戰,百姓勞敝,及即位後,復與魏交兵不止之應也。



 陳太建十二年春,不雨至四月。先是周師掠淮北,始興王叔陵等諸軍敗績,淮北之地皆沒於周,蓋其應也。



 東魏天平四年,並、肆、汾、建、晉、絳、秦、陜等諸州大旱,人多
 流散。是歲,齊神武與西魏戰於沙苑,敗績,死者數萬。



 東魏武定二年冬春旱。先是西魏師入洛陽,神武親帥軍大戰於邙山,死者數萬。



 後齊天保九年夏,大旱。先是大發卒築長城四百餘里,勞役之應也。



 乾明元年春,旱。先是發卒數十萬築金鳳、聖應、崇光三臺,窮極侈麗,不恤百姓,亢陽之應也。



 河清二年四月,並、晉已西五州旱。是歲,發卒築軹關。突厥二十萬眾毀長城,寇恆州。



 後主天統二年春,旱。是時大發卒,起大明宮。



 開皇四年已後,京師頻旱。時遷都龍首,建立宮室,百姓勞敝,亢陽之應也。



 大業四年,燕、代緣邊諸郡旱。時發卒百餘萬築長城,帝親巡塞表,百姓失業,道殣相望。



 八年,天下旱,百姓流亡。時發四海兵,帝親征高麗,六軍凍餒,死者十八九。



 十三年,天下大旱。時郡縣鄉邑,悉遣築城,發男女,無少長,皆就役。



 詩妖梁天監三年六月八日,武帝講於重雲殿,沙門志公忽
 然起儛歌樂,須臾悲泣,因賦五言詩曰:「樂哉三十餘,悲哉五十里!但看八十三,子地妖災起。佞臣作欺妄,賊臣滅君子。若不信吾語,龍時侯賊起。且至馬中間,銜悲不見喜。」梁自天監至於大同,三十餘年,江表無事。至太清二年,臺城陷,帝享國四十八年,所言五十里也。太清元年八月十三,而侯景自懸瓠來降。在丹陽之北,子地。帝惑硃異之言以納景。景之作亂,始自戊辰之歲。至午年,帝憂崩。十年四月八日,志公於大會中又作詩曰:「兀尾狗子始著狂,欲死不死嚙人傷,須臾之間自滅亡。患在汝陰死三湘,橫尸一旦無人藏。」侯景小字狗子,初自懸
 瓠來降,懸瓠則古之汝南也。



 巴陵南有地名三湘,即景奔敗之所。



 天監中,茅山隱士陶弘景為五言詩曰:「夷甫任散誕,平叔坐談空,不意昭陽殿,忽作單于宮。」及大同之季,公卿唯以談玄為務。夷甫、平叔,朝賢也。侯景作亂,遂居昭陽殿。



 大同中,童謠曰:「青絲白馬壽陽來。」其後侯景破丹陽,乘白馬,以青絲為羈勒。



 陳初,有童謠曰:「黃班青驄馬,發自壽陽涘。來時冬氣末,去日春風始。」



 其後陳主果為韓擒所敗。擒本名擒虎,黃
 班之謂也。破建康之始,復乘青驄馬,往反時節皆相應。



 陳時,江南盛歌王獻之《桃葉》之詞曰:「桃葉復桃葉,渡江不用楫。但度無所苦,我自迎接汝。」晉王伐陳之始,置營桃葉山下,及韓擒渡江,大將任蠻奴至新林以導北軍之應。



 陳後主造齊雲觀,國人歌之曰:「齊雲觀,寇來無際畔。」功未畢,而為隋師所虜。



 禎明初,後主作新歌,詞甚哀怨,令後宮美人習而歌之。其辭曰:「玉樹後庭花,花開不復久。」時人以歌讖,此其不久兆也。



 齊神武始移都於鄴,時有童謠云:「可憐青雀子,飛入鄴城里。作窠猶未成,舉頭失鄉里。寄書與婦母,好看新婦子。」魏孝靜帝者,清河王之子也。後則神武之女。鄴都宮室未備,即逢禪代,作窠未成之效也。孝靜尋崩,文宣以後為太原長公主,降於楊愔。時婁後尚在,故言寄書於婦母。新婦子,斥後也。



 武定中,有童謠云:「百尺高竿摧折,水底燃燈澄滅。」高者,齊姓也。澄,文襄名。五年,神武崩,摧折之應。七年,文襄遇盜所害,澄滅之徵也。



 天保中,陸法和入國,書其屋壁曰:「十年天子為尚可,百
 日天子急如火,周年天子迭代坐。」時文宣帝享國十年而崩。廢帝嗣立百餘日,用替厥位,孝昭即位一年而崩。此其效也。



 武平元年,童謠曰:「狐截尾,你欲除我我除你。」其年四月,隴東王胡長仁謀遣刺客殺和士開,事露,返為士開所譖死。



 二年,童謠曰:「和士開,七月三十日,將你向南臺。」小兒唱訖,一時拍手云:「殺卻。」至七月二十五日,御史中丞、瑯邪王儼執士開,送於南臺而斬之。



 是歲,又有童謠曰:「七月刈禾傷早,九月吃糕正好。十月洗蕩飯甕,十一月出卻
 趙老。」七月士開被誅,九月瑯邪王遇害,十一月趙彥深出為西兗州刺史。



 武平末,童謠曰:「黃花勢欲落,清樽但滿酌。」時穆后母子淫僻,干預朝政,時人患之。穆後小字黃花,尋逢齊亡,欲落之應也。



 鄴中又有童謠曰:「金作掃帚玉作把,凈掃殿屋迎西家。」未幾,周師入鄴。



 周初有童謠曰:「白楊樹頭金雞鳴,只有阿舅無外甥。」靜帝隋氏之甥,既遜位而崩,諸舅強盛。



 周宣帝與宮人夜中連臂蹋蹀而歌曰:「自知身命促,把
 燭夜行游。」帝即位三年而崩。



 開皇十年,高祖幸並州,宴秦孝王及王子相。帝為四言詩曰:「紅顏詎幾,玉貌須臾。一朝花落,白發難除。明年後歲,誰有誰無。」明年而子相卒,十八年而秦孝王薨。



 大業十一年,煬帝自京師如東都,至長樂宮,飲酒大醉,因賦五言詩。其卒章曰:「徒有歸飛心,無復因風力」。令美人再三吟詠,帝泣下沾襟,侍御者莫不欷歔。帝因幸江都,復作五言詩曰:「求歸不得去,真成遭個春。鳥聲爭勸酒,梅花笑殺人。」帝以三月被弒,即遭春之應也。是年盜賊蜂起,道路隔絕,帝懼,遂無還心。帝復夢二豎子歌曰:「
 住亦死,去亦死。未若乘船渡江水。」由是築宮丹陽,將居焉。功未就而帝被殺。



 大業中,童謠曰:「桃李子,鴻鵠繞陽山,宛轉花林裏。莫浪語,誰道許。」



 其後李密坐楊玄感之逆,為吏所拘,在路逃叛。潛結群盜,自陽城山而來,襲破洛口倉,後復屯兵苑內。莫浪語,密也。宇文化及自號許國,尋亦破滅。誰道許者,蓋驚疑之辭也。



 毛蟲之孽梁武帝中大同元年,邵陵王綸在南徐州臥內,方晝,有貍鬥於櫩上,墮而獲之。



 太清中,遇侯景之亂,將兵援臺
 城。至鐘山,有蟄熊無何至,嚙綸所乘馬。毛蟲之孽也。綸尋為王僧辯所敗,亡至南陽,為西魏所殺。



 中大同中,每夜狐鳴闕下,數年乃止。京房《易飛候》曰:「野獸群鳴,邑中且空虛。」俄而國亂,丹陽死喪略盡。



 陳禎明初,狐入床下,捕之不獲。京房《易飛候》曰:「狐入君室,室不居。」



 未幾而國滅。



 東魏武定三年九月,豹入鄴城南門,格殺之。五年八月,豹又上銅爵臺。京房《易飛候》曰:「野獸入邑,及至朝廷若道,上官府門,有大害,君亡。」是歲,東魏師敗於玉壁,神武遇疾崩。



 後齊武平二年,有兔出廟社之中。京房《易飛候》曰:「兔入王室,其君亡。」



 案廟者,祖宗之神室也。後五歲,周師入鄴,後主東奔。



 武平末,並、肆諸州多狼而食人。《洪範五行傳》曰:「狼,貪暴之獸,大體以白色為主,兵之表也。又似犬,近犬禍也。」京房《易傳》曰:「君將無道,害將及人,去之深山以全身。厥妖狼食人。」時帝任用小人,竟為貪暴,殘賊人物,食人之應。尋為周軍所滅,兵之象也。



 武平中,朔州府門外,無何有小兒腳跡,又擁土為城雉之狀。時人怪而察之,乃狐媚所為,漸流至並、鄴。與武定
 三年同占。是歲,南安王思好起兵於北朔,直指並州,為官軍所敗。鄭子饒、羊法皓等復亂山東。



 犬禍後齊天保四年,鄴中及頓丘並有犬與女子交。《洪範五行傳》曰:「異類不當交而交,悖亂之氣。犬交人為犬禍。」犬禍者,亢陽失眾之應也。時帝不恤國政,恩澤不流於其國。



 後主時,犬為開府儀同,雌者有夫人郡君之號,給兵以奉養,食以粱肉,藉以茵蓐。天奪其心,爵加於犬,近犬禍也。天意若曰,卿士皆類犬。後主不悟,遂以取滅。



 後周保定三年,有犬生子,腰已後分為兩身,二尾六足。犬猛畜而有爪牙,將士之象也。時宇文護與侯伏、侯龍恩等,有謀懷貳。犬體後分,此其應也。



 大業元年,雁門百姓間犬多去其主,群聚於野,形頓變如狼而啖噬行人,數年而止。《五行傳》曰:「犬,守御者也,而今去其主,臣下不附之象。形變如狼,狼色白,為主兵之應也。」其後帝窮兵黷武,勞役不息。天戒若曰,無為勞役,守禦之臣將叛而為害。帝不悟,遂起長城之役。續有西域、遼東之舉,天下怨叛。及江都之變,並宿衛之臣也。



 白眚白祥
 梁大同二年,地生白毛,長二尺,近白祥也。孫盛以為勞人之異。先是大發卒築浮山堰,功費鉅億,功垂就而復潰者,數矣。百姓厭役,籲嗟滿道。



 齊河清元年九月,滄州及長城之下,地多生毛,或白或黑,長四五寸,近白祥也。時北築長城,內興三臺,人苦勞役。



 開皇六年七月,京師雨毛,如發尾,長者三尺餘,短者六七寸。京房《易飛候》曰:「天雨毛,其國大饑。」是時關中旱,米粟湧貴。



 後齊天統初,岱山封禪壇玉璧自出,近白祥也。岱山,王
 者易姓告代之所,玉璧所用幣而自出,將有易姓者用幣之象。其後齊亡,地入於周,及高祖受周禪,天下一統,焚柴太山告祠之應也。



 武平三年,白水巖下青石壁傍,有文曰:「齊亡走。」人改之為「上延」,後主以為嘉瑞,百僚畢賀。後周師入國,後主果棄鄴而走。



 開皇十七年,石隕於武安、滏陽間十餘。《洪範五行傳》曰:「石自高隕者,君將有危殆也。」後七載,帝崩。



 開皇末,高祖於宮中埋二小石於地,以志置床之所。未幾,變為玉。劉向曰:「玉者至貴也。賤將為貴之象。」及大業
 末,盜皆僭名號。



 大業十三年,西平郡有石,文曰:「天子立千年。」百僚稱賀。有識者尤之曰:「千年萬歲者,身後之意也。今稱立千年者,禍在非遠。」明年而帝被殺。



 木沴金梁大同十二年,曲阿建陵隧口石麒麟動。木沴金也。動者,遷移之象。天戒若曰,園陵無主,石麟將為人所徙也。後竟國亡。



 後齊河清四年,殿上石自起,兩兩相擊。眭孟以為石陰類,下人象,殿上石自起者,左右親人離叛之應。及周師
 東伐,寵臣尉相願、乞扶貴和兄弟、韓建業之徒,皆叛入周。



 梁大同十二年正月,送闢邪二於建陵。左雙角者至陵所。右獨角者,將引,於車上振躍者三。車兩轅俱折。因換車。未至陵二里,又躍者三,每一振則車側人莫不聳奮,去地三四尺,車輪陷入土三寸。木暕金也。劉向曰:「失眾心,令不行,言不從,以亂金氣也。石為陰,臣象也。臣將為變之應。」梁武暮年,不以政事為意,君臣唯講佛經、談玄而已。朝綱紊亂,令不行,言不從之咎也。其後果致侯景之亂。



 周建德元年,濮陽郡有石像,郡官令載向府,將刮取金。在道自躍投地,如此者再。乃以大繩縛著車壁,又絕繩而下。時帝既滅齊,又事淮南,征伐不息,百姓疲敝,失眾心之應也。



\end{pinyinscope}