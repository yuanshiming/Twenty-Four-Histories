\article{卷七十九列傳第四十四}

\begin{pinyinscope}

 外戚歷觀前代外戚之家,乘母後之權以取高位厚秩者多矣,然而鮮有克終之美,必罹顛覆之患,何哉?皆由乎無德而尊,不知紀極,忽於滿盈之戒,罔念高危之咎,故鬼瞰其室,憂必及之。夫其誠著艱難,功宣社稷,不以謙沖自牧,未免顛蹶之禍,而況道不足以濟時,仁不足以利物,自矜於己,以富貴驕人者乎?此呂、霍、上官、閻、梁、竇、鄧
 所以繼踵而亡滅者也。昔文皇潛躍之際,獻後便相推轂,煬帝大橫方兆,蕭妃密勿經綸,是以恩禮綢繆,始終不易。然內外親戚,莫預朝權,昆弟在位,亦無殊寵。至於居擅玉堂,家稱金穴,暉光戚里,重灼四方,將三司以比儀,命五侯而同拜者,終始一代,寂無聞焉。考之前王,可謂矯其弊矣。故雖時經擾攘,無有陷於不義,市朝遷貿,而皆得以保全。比夫憑藉寵私,階緣恩澤,乘其非據,旋就顛隕者,豈可同日而言哉!此所謂愛之以禮,能改覆車。輒敘其事,為《外戚傳》云。



 高祖外家呂氏
 高祖外家呂氏,其族蓋微,平齊之後,求訪不知所在。至開皇初,濟南郡上言,有男子呂永吉,自稱有姑字苦桃,為楊忠妻。勘驗知是舅子,始追贈外祖雙周為上柱國、太尉、八州諸軍事、青州刺史,封齊郡公,謚曰敬,外祖母姚氏為齊敬公夫人。詔並改葬,於齊州立廟,置守塚十家。以永吉襲爵,留在京師。大業中,授上黨郡太守,性識庸劣,職務不理。後去官,不知所終。



 永吉從父道貴,性尤頑騃,言詞鄙陋。初自鄉里征入長安,上見之悲泣。道貴略無戚容,但連呼高祖名,云:「種末定不可偷,大似苦桃姊。」是後數犯忌諱,動致違忤,上甚耽之。乃命高熲厚加
 供給,不許接對朝士。拜上儀同三司,出為濟南太守,令即之任,斷其入朝。道貴還至本郡,高自崇重,每與人言,自稱皇舅。



 數將儀衛出入閭里,從故人游宴,官民咸苦之。後郡廢,終於家,子孫無聞焉。



 獨孤羅弟陀獨孤羅,字羅仁,雲中人也。父信,初仕魏為荊州刺史。武帝之入關也,信棄父母妻子西歸長安,歷職顯貴,羅由是遂為高氏所囚。信後仕周為大司馬。及信為宇文護所誅,羅始見釋,寓居中山,孤貧無以自給。齊將獨孤永業以宗族之故,見而哀之,為買田宅,遺以資畜。初,信入
 關之後,復娶二妻,郭氏生子六人,善、穆、藏、順、陀、整,崔氏生獻皇后。及齊亡,高祖為定州總管,獻皇后遣人尋羅,得之,相見悲不自勝,侍御者皆泣。於是厚遺車馬財物。未幾,周武帝以羅功臣子,久淪異域,徵拜楚安郡太守。以疾去官,歸於京師。諸弟見羅少長貧賤,每輕侮之,不以兄禮事也。然性長者,亦不與諸弟校競長短,後由是重之。及高祖為丞相,拜儀同,常置左右。既受禪,下詔追贈羅父信官爵曰:「褒德累行,往代通規,追遠慎終,前王盛典。故柱國信,風宇高曠,獨秀生民,睿哲居宗,清猷映世。宏謀長策,道著於弼諧,緯義經仁,事深於拯濟。方當
 宣風廊廟,亮採臺階,而運屬艱危,功高弗賞,眷言令範,事切於心。今景運初開,椒闈肅建,載懷塗山之義,無忘褒、紀之典。可贈太師、上柱國、冀定等十州刺史、趙國公,邑萬戶。」其諸弟以羅母沒齊,先無夫人之號,不當承襲。上以問後,後曰:「羅誠嫡長,不可誣也。」於是襲爵趙國公。以其弟善為河內郡公,穆為金泉縣公,藏為武平縣公,陀為武喜縣公,整為千牛備身。擢拜羅為左領左右將軍,尋遷左衛將軍,前後賞賜不可勝計。



 久而出為涼州總管,進位上柱國。仁壽中,徵拜左武衛大將軍。煬帝嗣位,改封蜀國公。未幾,卒官,謚曰恭。



 子纂嗣,仕至河陽郡
 尉。纂弟武都,大業末,亦為河陽郡尉。庶長子開遠,宇文化及之弒逆也,裴虔通率賊入成象殿,宿衛兵士皆從逆,開遠時為千牛,與獨孤盛力戰於閤下,為賊所執,賊義而舍之。善後官至柱國。卒,子覽嗣,仕至左候衛將軍,大業末卒。



 獨孤陀,字黎邪。仕周胥附上士,坐父徙蜀郡十餘年。宇文護被誅,始歸長安。



 高祖受禪,拜上開府、右領左右將軍。久之,出為郢州刺史,進位上大將軍,累轉延州刺史。好左道。其妻母先事貓鬼,因轉入其家。上微聞而不之信也。會獻皇后及楊素妻鄭氏俱有疾,召醫者視之,皆
 曰:「此貓鬼疾也。」上以陀后之異母弟,陀妻楊素之異母妹,由是意陀所為,陰令其兄穆以情喻之。上又避左右諷陀,陀言無有。上不悅,左轉遷州刺史。出怨言。上令左僕射高熲、納言蘇威、大理正皇甫孝緒、大理丞楊遠等雜治之。陀婢徐阿尼言,本從陀母家來,常事貓鬼。每以子日夜祀之。言子者鼠也。其貓鬼每殺人者,所死家財物潛移於畜貓鬼家。陀嘗從家中素酒,其妻曰:「無錢可酤。」陀因謂阿尼曰:「可令貓鬼向越公家,使我足錢也。」



 阿尼便咒之歸。數日,貓鬼向素家。十一年,上初從並州還,陀於園中謂阿尼曰:「可令貓鬼向皇后所,使多賜吾物。」
 阿尼復咒之,遂入宮中。楊遠乃於門下外省遣阿尼呼貓鬼。阿尼於是夜中置香粥一盆,以匙扣而呼之曰:「貓女可來,無住宮中。」久之,阿尼色正青,若被牽曳者,云貓鬼已至。上以其事下公卿,奇章公牛弘曰:「妖由人興,殺其人可以絕矣。」上令以犢車載陀夫妻,將賜死於其家。陀弟司勛侍中整詣闕求哀,於是免陀死,除名為民,以其妻楊氏為尼。先是,有人訟其母為人貓鬼所殺者,上以為妖妄,怒而遣之。及此,詔誅被訟行貓鬼家。經未幾而卒。煬帝即位,追念舅氏,聽以禮葬,乃下詔曰:「外氏衰禍,獨孤陀不幸早世,遷卜有期。言念渭陽之情,追懷傷
 切,宜加禮命,允備哀榮。可贈正議大夫。」帝意猶不已,復下詔曰:「舅氏之尊,戚屬斯重,而降年弗永,凋落相繼。緬惟先往,宜崇徽秩。復贈銀青光祿大夫。」有二子:延福、延壽。



 陀弟整,官至幽州刺史,大業初卒,贈金紫光祿大夫,平鄉侯。



 蕭巋子琮琮弟瓛蕭巋,字仁遠,梁昭明太子統之孫也。父詧,初封岳陽王,鎮襄陽。侯景之亂,其兄河東王譽與其叔父湘東王繹不協,為繹所害。及繹嗣位,詧稱籓於西魏,乞師請討繹。周太祖以詧為梁主,遣柱國於謹等率騎五萬襲繹,滅
 之。詧遂都江陵,有荊郡、其西平州延袤三百里之地,稱皇帝於其國,車服節文一同王者。仍置江陵總管,以兵戍之。詧薨,巋嗣位,年號天保。巋俊辯有才學,兼好內典。周武帝平齊之後,巋來賀,帝享之甚歡。親彈琵琶,令巋起舞,巋曰:「陛下親御五糸玄,臣敢不同百獸!」高祖受禪,恩禮彌厚,遣使賜金五百兩,銀千兩,布帛萬匹,馬五百匹。巋來朝,上甚敬焉,詔巋位在王公之上。巋被服端麗,進退閑雅,天子矚目,百僚傾慕。賞賜以億計。月餘歸籓,帝親餞於滻水之上。後備禮納其女為晉王妃,又欲以其子瑒尚蘭陵公主。由是漸見親待。獻皇后言於上曰:「梁
 主通家,腹心所寄,何勞猜防也。」上然之,於是罷江陵總管,巋專制其國。歲餘,巋又來朝,賜縑萬匹,珍玩稱是。及還,上親執手曰:「梁主久滯荊楚,未復舊都,故鄉之念,良軫懷抱。朕當振旅長江,相送旋反耳。」巋拜謝而去。其年五月,寢疾,臨終上表曰:「臣以庸暗,曲荷天慈,寵冠外籓,恩逾連山,爰及子女,尚主婚王。每願躬擐甲胄,身先士卒,掃蕩逋寇,上報明時。而攝生乖舛,遽罹痾疾,屬纊在辰,顧陰待謝。長違聖世,感戀嗚咽,遺嗣孤藐,特乞降慈。伏願聖躬與山嶽同固,皇基等天日俱永,臣雖九泉,實無遣恨。」並獻所服金裝劍,上覽而嗟悼焉。巋在位二十
 三年,年四十四薨,梁之臣子謚曰孝明皇帝,廟號世宗。子琮嗣。巋著《孝經》、《周易義記》及《大小乘幽微》十四卷,行於世。



 琮字溫文,性寬仁,有大度,倜儻不羈,博學有文義。兼善弓馬,遣人伏地著帖,琮馳馬射之,十發十中,持帖者亦不懼。初封東陽王,尋立為梁太子。及嗣位,上賜璽書曰:「負荷堂構,其事甚重,雖窮憂勞,常須自力。輯諧內外,親任才良,聿遵世業,是所望也。彼之疆守,咫尺陳人,水潦之時,特宜警備。陳氏比日雖復朝聘相尋,疆埸之間猶未清肅,唯當恃我必不可干,勿得輕人而不設備。朕與
 梁國,積世相知,重以親姻,情義彌厚。江陵之地,朝寄非輕,為國為民,深宜抑割,恆加饘粥,以禮自存。」又賜梁之大臣璽書,誠勉之。時琮年號廣運,有識者曰:「運之為字,軍走也,吾君將奔走乎?」其年,琮遣大將軍戚昕以舟師襲陳公安,不克而還。征琮叔父岑入朝,拜為大將軍,封懷義公,因留不遣。復置江陵總管以監之。琮所署大將軍許世武密以城召陳將宜黃侯陳紀,謀洩,琮誅之。後二歲,上征琮入朝,率其臣下二百餘人朝於京師,江陵父老莫不隕涕相謂曰:「吾君其不反矣!」上以琮來朝,遣武鄉公崔弘度將兵戍之。軍至鄀州,琮叔父巖及弟瓛
 等懼弘度掩襲之,遂引陳人至城下,虜居民而叛,於是廢梁國。上遣左僕射高熲安集之,曲赦江陵死罪,給民復十年。梁二主各給守墓十戶。拜琮為柱國,賜爵莒國公。煬帝嗣位,以皇后之故,甚見親重。拜內史令,改封梁公。琮之宗族,緦麻以上,並隨才擢用,於是諸蕭昆弟布列朝廷。琮性淡雅,不以職務自嬰,退朝縱酒而已。內史令楊約與琮同列,帝令約宣旨誡勵,約復以私情喻之。琮答曰:「琮若復事事,則何異於公哉!」約笑而退。約兄素,時為尚書令,見琮嫁從父妹於鉗耳氏,因謂琮曰:「公,帝王之族,望高戚美,何乃適妹鉗耳氏乎?」琮曰:「前已嫁妹
 於侯莫陳氏,此復何疑!」素曰:「鉗耳,羌也,侯莫陳,虜也,何得相比!」素意以虜優羌劣。琮曰:「以羌異虜,未之前聞。」素慚而止。琮雖羈旅,見北間豪貴,無所降下。嘗與賀若弼深相友善,弼既被誅,復有童謠曰:「蕭蕭亦復起。」帝由是忌之,遂廢於家,未幾而卒。贈左光祿大夫。子鉉,襄城通守。復以琮弟子鉅為梁公。鉅小名藏,煬帝甚暱之,以為千牛,與宇文皛出入宮掖,伺察內外。帝每有游宴,鉅未嘗不從焉,遂於宮中多行淫穢。江都之變,為宇文化及所殺。



 瓛字欽文,少聰敏,解屬文。在梁為荊州刺史,頗有能名。
 崔弘度以兵至若阜州,瓛懼,與其叔父巖奔於陳。陳主以為侍中、安東將軍、吳州刺史,甚得物情,三吳父老皆曰:「瓛吾君子也。」及陳亡,吳人推瓛為主。吳人見梁武、簡文及詧、巋等兄弟並第三而踐尊位,瓛自以巋之第三子也,深自矜負。有謝異者,頗知廢興,梁、陳之際,言無不驗,江南人甚敬信之。及陳主被擒,異奔於瓛,由是益為眾所歸。褒國公宇文述以兵討之,瓛遣王哀守吳州,自將拒述。述遣兵別道襲吳州,哀懼,衣道士服,棄城而遁。瓛眾聞之,悉無鬥志,與述一戰而敗。瓛將左右數人逃於太湖,匿於民家,為人所執,送於述所,斬之長安,時年二十
 一。



 弟璟,為朝請大夫、尚衣奉御。瑒,歷衛尉卿、秘書監、陶丘侯。瑀,歷內史侍郎、河池太守。



 史臣曰:三五哲王,防深慮遠,舅甥之國,罕執鈞衡,母後之家,無聞傾敗。



 爰及漢、晉,顛覆繼軌,皆由乎進不以禮,故其斃亦速。若使獨孤權侔呂、霍,必敗於仁壽之前,蕭氏勢均梁、竇,豈全於大業之後!今或不隕舊基,或更隆先構,豈非處之以道,不預權寵之所致乎!



\end{pinyinscope}