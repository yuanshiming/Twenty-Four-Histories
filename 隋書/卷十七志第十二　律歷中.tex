\article{卷十七志第十二 律歷中}

\begin{pinyinscope}

 夫
 歷者,
 紀陰陽之通變,極往數以知來,可以迎日授時,先天成務者也。然則懸象著明,莫大於二曜,氣序環復,無信於四時。日月相推而明生矣,寒暑迭進而歲成焉,遂能成天地之文,極乾坤之變。天數五,地數五,五位相乘而各有合。天數二十有五,地數三十,凡天地之數五十有五,所以成變化而行鬼神也。乾之策二百一十有
 六,坤之策一百四十有四,凡三百六十,以當期之日也。至乃陰陽迭用。



 剛柔相摩,四象既陳,八卦成列,此乃造文之元始,創歷之厥初者歟?洎乎炎帝分八節,軒轅建五部,少昊以鳳鳥司歷,顓頊以南正司天,陶唐則分命和仲,夏後乃備陳《鴻範》,湯武革命,咸率舊章。然文質既殊,正朔斯革,故天子置日官,諸侯有日御,以和萬國,以協三辰。至於寒暑晦明之徵,陰陽生殺之數,啟閉升降之紀,消息盈虛之節,皆應躔次而不淫,遂得該浹生靈,堪輿天地,開物成務,致遠鉤深。周德既衰,史官廢職,疇人分散,禨祥莫理。秦兼天下,頗推五勝,自以獲水德之
 瑞,以十月為正。漢氏初興,多所未暇,百有餘載,猶行秦歷。至於孝武,改用夏正。時有古歷六家,學者疑其紕繆,劉向父子,咸加討論,班固因之,採以為志。光武中興,未能詳考。逮於永平之末,乃復改行四分,七十餘年,儀式方備。



 其後復命劉洪、蔡邕,共修律歷,司馬彪用之以續《班史》。當塗受命,亦有史官,韓翊創之於前,楊偉繼之於後,咸遵劉洪之術,未及洪之深妙。中、左兩晉,迭有增損。至於西涼,亦為蔀法,事跡糾紛,未能詳記。宋氏元嘉,何承天造歷,迄於齊末,相仍用之。梁武初興,因循齊舊,天監中年,方改行宋祖沖之《甲子元歷》。



 陳武受禪,亦無創
 改。後齊文宣,用宋景業歷。西魏入關,行李業興歷。逮於周武帝,乃有甄鸞造《甲寅元歷》,遂參用推步焉。大象之初,太史上士馬顯,又上《丙寅元歷》,便即行用。迄於開皇四年,乃改用張賓歷,十七年,復行張胄玄歷,至於義寧。今採梁天監以來五代損益之要,以著於篇云。



 梁初因齊,用宋《元嘉歷》。天監三年下詔定歷,員外散騎侍郎祖恆奏曰:「臣先在晉已來,世居此職。仰尋黃帝至今十二代,歷元不同,周天、斗分,疏密亦異,當代用之,各垂一法。宋大明中,臣先人考古法,以為正歷,垂之於後,事皆符驗,不可改張。」八年,恆又上疏論之。詔使太史令
 將匠道秀等,候新舊二歷氣朔、交會及七曜行度,起八年十一月,訖九年七月,新歷密,舊歷疏。恆乃奏稱:「史官今所用何承天歷,稍與天乖,緯緒參差,不可承案。被詔付靈臺,與新歷對課疏密,前期百日,並又再申。始自去冬,終於今朔,得失之效,並已月別啟聞。夫七曜運行,理數深妙,一失其源,則歲積彌爽。所上脫可施用,宜在來正。」



 至九年正月,用祖沖之所造《甲子元歷》頒朔。至大同十年,制詔更造新歷,以甲子為元,六百一十九為章歲,一千五百三十六為日法,一百八十三年冬至差一度,月朔以遲疾定其小餘,有三大二小。未及施用而遭侯
 景亂,遂寢。



 陳氏因梁,亦用祖沖之歷,更無所創改。後齊文宣受禪,命散騎侍郎宋景業葉圖讖,造《天保歷》。景業奏:依《握誠圖》及《元命包》,言齊受錄之期,當魏終之紀,得乘三十五以為蔀,應六百七十六以為章。」文宣大悅,乃施用之。期歷統曰:「上元甲子,至天保元年庚午,積十一萬五百六算外,章歲六百七十六,度法二萬三千六百六十,斗分五千七百八十七,歷餘十六萬二千二百六十一。」至後主武平七年,董峻、鄭元偉立議非之曰:「宋景業移閏於天正,退命於冬至交會之際,承二大之後,三月之交,妄減平分。臣案,景業學非探賾,識殊深解,
 有心改作,多依舊章,唯寫子換母,頗有變革,妄誕穿鑿,不會真理。乃使日之所在,差至八度,節氣後天,閏先一月。朔望虧食,既未能知其表裏,遲疾之歷步,又不可以傍通。妄設平分,虛退冬至,虛退則日數減於周年,平分妄設,故加時差於異日。



 五星見伏,有違二旬,遲疾逆留,或乖兩宿。軌褵之術,妄刻水旱。今上《甲寅元歷》,並以六百五十七為章,二萬二千三百三十八為蔀,五千四百六十一為斗分,甲寅歲甲子日為元紀。」又有廣平人劉孝孫、張孟賓二人,同知歷事。孟賓受業於張子信,並棄舊事,更制新法。又有趙道嚴,準晷影之長短,定日行之
 進退,更造盈縮,以求虧食之期。劉孝孫以百一十九為章,八千四十七為紀,九百六十六為歲餘,甲子為上元,命日度起虛中。張孟賓以六百一十九為章,四萬八1千九百為紀,九百四十八為日法,萬四千九百四十五為斗分。元紀共命,法略旨遠。日月五星,並從斗十一起。盈縮轉度,陰陽分至,與漏刻相符,共日影俱合,循轉無窮。上拒春秋,下盡天統,日月虧食及五星所在,以二人新法考之,無有不合。其年,訖乾敬禮及歷家豫刻日食疏密。六月戊申朔,太陽虧,劉孝孫言食於卯時,張孟賓言食於甲時,鄭元偉、董峻言食於辰時,宋景業言食於巳
 時。至日食,乃於卯甲之間,其言皆不能中。爭論未定,遂屬國亡。



 西魏入關,尚行李業興《正光歷》法。至周明帝武成元年,始詔有司造周歷。



 於是露門學士明克讓、麟趾學士庾季才及諸日者,採祖恆舊議,通簡南北之術。



 自斯已後,頗觀其謬,故周、齊並時,而歷差一日。克讓儒者,不處日官,以其書下於太史。及武帝時,甄鸞造《天和歷》。上元甲寅至天和元年丙戌,積八十七萬五千七百九十二算外,章歲三百九十一,蔀法二萬三千四百六十,日法二十九萬一百六十,朔餘十五萬三千九百九十一,斗分
 五千七百三十一,會餘九萬三千五百一十六,歷餘一十六萬八百三十,冬至斗十五度,參用推步。終於宣政元年。大象元年,太史上士馬顯等,又上《丙寅元歷》,抗表奏曰:臣案九章五紀之旨,三統四分之說,咸以節宣發斂,考詳晷緯,布政授時,以為皇極者也。而乾維難測,鬥憲易差,盈縮之期致舛,咎徵之道斯應。寧止蛇或乘龍,水能沴火,因亦玉羊掩曜,金雞喪精。王化關以盛衰,有國由其隆替,歷之時義,於斯為重。自炎漢已還,迄於有魏,運經四代,事涉千年,日御天官,不乏於世,命元班朔,互有沿改。驗近則疊璧應辰,經遠則連珠失次,義難循
 舊,其在茲乎?大周受圖膺錄,牢籠萬古,時夏乘殷,斟酌前代,歷變壬子,元用甲寅。高祖武皇帝索隱探賾,盡性窮理,以為此歷雖行,未臻其妙,爰降詔旨,博訪時賢,並敕太史上士馬顯等,更事刊定,務得其宜。然術藝之士,各封異見,凡所上歷,合有八家,精粗踳駁,未能盡善。去年冬,孝宣皇帝乃詔臣等,監考疏密,更令同造。



 謹案史曹舊簿及諸家法數,棄短取長,共定今術。開元發統,肇自丙寅,至於兩曜虧食,五星伏見,參校積時,最為精密。庶鐵炭輕重,無失寒燠之宜,灰箭飛浮,不爽陰陽之度。上元丙寅至大象元年己亥,積四萬一千五百五十四
 算上。日法五萬三千五百六十三,亦名蔀會法。章歲四百四十八,斗分三千一百六十七,蔀法一萬二千九百九十二。章中為章會法。日法五萬三千五百六十三,歷餘二萬九千六百九十三,會日百七十三,會餘一萬六千六百一十九,冬至日在斗十二度。小周餘、盈縮積,其歷術別推入蔀會,分用陽率四百九十九,陰率九。每十二月下各有日月蝕轉分,推步加減之,乃為定蝕大小餘,而求加時之正。



 其術施行。時高祖作輔,方行禪代之事,欲以符命曜於天下。道士張賓,揣知上意,自云玄相,洞曉星歷,因盛言有代謝之徵,又稱上儀表非人臣相。
 由是大被知遇,恆在幕府。及受禪之初,擢賓為華州刺史,使與儀同劉暉、驃騎將軍董琳、索盧縣公祐、前太史上士馬顯、太學博士鄭元偉、前保章上士任悅、開府掾張撤、前蕩邊將軍張膺之、校書郎衡洪建、太史監候粟相、太史司歷郭翟、劉宜、兼算學博士張乾敘、門下參人王君瑞、荀隆伯等,議造新歷,仍令太常卿盧賁監之。賓等依何承天法,微加增損,四年二月撰成奏上。高祖下詔曰:「張賓等存心算數,通洽古今,每有陳聞,多所啟沃。畢功表奏,具已披覽。使後月復育,不出前晦之宵,前月之餘,罕留後朔之旦。減朓就朒,懸殊舊準。月行表裏,
 厥途乃異,日交弗食,由循陽道。驗時轉算,不越纖毫,逖德前修,斯秘未啟。有一於此,實為精密,宜頒天下,依法施用。」



 張賓所造歷法,其要:以上元甲子已來,至開皇四年歲在甲辰,積四百一十二萬九千一,算上。



 蔀法,一十萬二千九百六十。



 章歲,四百二十九。



 章月,五千三百六。



 通月,五百三十七萬二千二百九。



 日法,一十八萬一千九百二十。



 斗分,二萬五千六十三。



 會月,一千二百九十七。



 會率,二百二十一。



 會數,一百一十半。



 會分,一十一億八千七百二十五萬八千一百八十九。



 會日法,四千二十萬四千三百二十。



 會日,百七十三。



 餘,五萬六千一百四十三。



 小分,一百一十。



 交法,五億一千二百一十萬四千八百。



 交分法,二千八百一十五。



 陰陽歷,一十三。



 餘,十一萬二百六十三。



 小分,二千三百二十八。



 朔差,二。



 餘,五萬七千九百二十一。



 小分,九百七十四。



 蝕限,一十二。



 餘,八萬一千三百三。



 小分,四百三十三半。



 定差,四萬四千五百四十八。



 周日,二十七。



 餘,一十萬八百五十九。亦名少大法木精曰歲星,合率四千一百六萬三千八百八十九。



 火精曰熒惑,合率八千二十九萬七千九百二十六。



 土精曰鎮星,合率三千八百九十二萬五千四百一十三。



 金精曰太白,合率六千一十一萬九千六百五十五。



 水精曰辰星,合率一千一百九十三萬一千一百二十五。



 張賓所創之歷既行,劉孝孫與冀州秀才劉焯,並稱其
 失,言學無師法,刻食不中,所駁凡有六條:其一云,何承天不知分閏之有失,而用十九年之七閏。其二云,賓等不解宿度之差改,而冬至之日守常度。其三云,連珠合璧,七曜須同,乃以五星別元。其四云,賓等唯知日氣餘分恰盡而為立元之法,不知日月不合,不成朔旦冬至。其五雲,賓等但守立元定法,不須明有進退。其六云,賓等唯識轉加大餘二十九以為朔,不解取日月合會準以為定。此六事微妙,歷數大綱,聖賢之通術,而暉未曉此,實管窺之謂也。若乃驗影定氣,何氏所優,賓等推測,去之彌遠。合朔順天,何氏所劣,賓等依據,循彼迷蹤。蓋
 是失其菁華,得其糠粃者也。又云,魏明帝時,有尚書郎楊偉,修《景初歷》,乃上表立義,駁難前非,云:「加時後天,食不在朔。」然觀楊偉之意,故以食朔為真,未能詳之而制其法。至宋元嘉中,何承天著歷,其上表云:「月行不定,或有遲疾,合朔月食,不在朔望,亦非歷之意也。」然承天本意,欲立合朔之術,遭皮延宗飾非致難,故事不得行。至後魏獻帝時,有龍宜弟復修延興之歷,又上表云:「日食不在朔,而習之不廢,據《春秋》書食,乃天之驗朔也。」此三人者,前代善歷,皆有其意,未正其書。但歷數所重,唯在朔氣。朔為朝會之首,氣為生長之端,朔有告餼之文,氣
 有郊迎之典,故孔子命歷而定朔旦冬至,以為將來之範。今孝孫歷法,並按明文,以月行遲疾定其合朔,欲今食必在朔,不在晦、二之日也。縱使頻月一小、三大,得天之統。大抵其法有三,今列之云。



 第一,勘日食證恆在朔。



 引《詩》云:「十月之交,朔日辛卯,日有食之。」今以甲子元歷術推算,符合不差。《春秋經》書日食三十五。二十七日食,經書有朔,推與甲子元歷不差。



 八食,經書並無朔字。《左氏傳》云:「不書朔,官失之也。」《公羊傳》云:「不言朔者,食二日也。「《穀梁傳》云:「不言朔者,食晦也。」今以甲子元歷推算,俱是朔日。丘明受經夫子,於理尤
 詳,《公羊》、《穀梁》皆臆說也。



 《春秋左氏》隱公三年二月己巳,日有食之。推合己巳朔莊公十八年春三月,日有食之。推合壬子朔僖公十二年三月庚午,日有食之。推合庚午朔十五年夏五月,日有食之。推合癸未朔襄公十五年秋八月丁巳,日有食之。推合丁巳朔前、後漢及魏、晉四代所記日食,朔、晦及先晦,都合一百八十一,今以甲子元歷術推之,並合朔日而食。



 前漢合有四十五食。三食並先晦一日,三十二食並皆晦日,十食並是朔日後漢合有七十四食。三十七食並皆晦日,三十七食並皆朔日
 魏合有十四食。四食並皆晦日,十食並皆朔日晉合有四十八食。二十五食並皆晦日,二十三食並皆朔日第二,勘度差變驗。



 《尚書》云:「日短星昴,以正仲冬。」即是唐堯之時,冬至之日,日在危宿,合昏之時,昴正午。案《竹書紀年》,堯元年丙子。今以甲子元歷術推算得合堯時冬至之日,合昏之時,昴星正午。《漢書》武帝太初元年丁丑歲,落下閎等考定太初歷冬至之日,日在牽牛初。今以甲子元歷術算,即得鬥末牛初矣。晉時有姜岌,又以月食驗於日度,知冬至之日日在斗十七度。宋文帝元嘉十年癸酉歲,何承
 天考驗乾度,亦知冬至之日日在斗十七度。雖言冬至後上三日,前後通融,只合在斗十七度。但堯年漢日,所在既殊,唯晉及宋,所在未改,故知其度,理有變差。至今大隋甲辰之歲,考定歷數象,以稽天道,知冬至之日日在斗十三度。



 第三,勘氣影長驗。



 《春秋緯命歷序》云:「魯僖公五年正月壬子朔旦冬至。」今以甲子元歷術推算,得合不差。《宋書》元嘉十年,何承天以土圭測影,知冬至已差三日。詔使付外考驗,起元嘉十三年為始,畢元嘉二十年,八年之中,冬至之日恆與
 影長之日差校三日。今以甲子元歷術推算,但是冬至之日恆與影長之符合不差。詳之如左:十三年丙子,天正十八日歷注冬至,十五日影長,即是今歷冬至日。



 十四年丁丑,天正二十九日歷注冬至,二十六日影長,即是今歷冬至日。



 十五年戊寅,天正十一日歷注冬至,陰,無影可驗,今歷八日冬至。



 十六年己卯,天正二十一日歷注冬至,十八日影長,即是今歷冬至日。



 十七年庚辰,天正二日歷注冬至,
 十月二十九日影長,即是今歷冬至日。



 十八年辛巳,天正十三日歷注冬至,十日影長,即是今歷冬至日。



 十九年壬午,天正二十九日歷注冬至,陰,無影可驗,今歷二十二日冬至。



 二十年癸未,天正六日歷注冬至,三日影長,即是今歷冬至日。



 於時新歷初頒,賓有寵於高祖,劉暉附會之,被升為太史令。二人協議,共短孝孫,言其非毀天歷,率意迂怪,焯又妄相扶證,惑亂時人。孝孫、焯等,竟以他事斥罷。後賓死,孝孫為掖縣丞,委官入京,又上,前後為劉暉所詰,事寢不行。



 仍留孝孫直太史,累年不調,寓宿觀臺。乃抱其書,弟子輿櫬,來詣闕下,伏而慟哭。執法拘以奏之,高祖
 異焉,以問國子祭酒何妥。妥言其善,即日擢授大都督,遣與賓歷比校短長。先是信都人張胄玄,以算術直太史,久未知名。至是與孝孫共短賓歷,異論鋒起,久之不定。至十四年七月,上令參問日食事。楊素等奏:「太史凡奏日食二十有五,唯一晦三朔,依克而食,尚不得其時,又不知所起,他皆無驗。胄玄所克,前後妙衷,時起分數,合如符契。孝孫所克,驗亦過半。」於是高祖引孝孫、胄玄等,親自勞徠。孝孫因請先斬劉暉,乃可定歷。高祖不懌,又罷之。



 俄而孝孫卒,楊素、牛弘等傷惜之,又薦胄玄。上召見之,胄玄因言日長影短之事,高祖大悅,賞賜甚厚,
 令與參定新術。劉焯聞胄玄進用,又增損孝孫歷法,更名《七曜新術》,以奏之。與胄玄之法,頗相乖爽,袁充與胄玄害之。焯又罷。至十七年,胄玄歷成,奏之。上付楊素等校其短長。劉暉與國子助教王頍等執舊歷術,迭相駁難,與司歷劉宜援據古史影等,駁胄玄云:《命歷序》僖公五年天正壬子朔旦日至,《左氏傳》僖公五年正月辛亥朔日南至。張賓歷,天正壬子朔冬至,合《命歷序》,差《傳》一日。張胄玄歷,天正壬子朔,合《命歷序》,差《傳》一日;三日甲寅冬至,差《命歷序》二日,差《傳》三日。成公十二年,《命歷序》天正辛卯朔旦日至。張賓歷,天正辛卯朔冬至,合《命歷
 序》。張胄玄歷,天正辛卯朔,合《命歷序》;二日壬辰冬至,差《命歷序》一日。昭公二十年,《春秋左氏傳》二月己丑朔日南至,準《命歷序》庚寅朔旦日至。張賓歷,天正庚寅朔冬至,並合《命歷序》,差《傳》一日。張胄玄歷,天正庚寅朔,合《命歷序》,差《傳》一日;二日辛卯冬至,差《命歷序》一日,差《傳》二日。宜案《命歷序》及《春秋左氏傳》,並閏餘盡之歲,皆須朔旦冬至。



 若依《命歷序》勘《春秋》三十七食,合處至多;若依《左傳》,合者至少,是以知《傳》為錯。今張胄玄信情置閏,《命歷序》及《傳》氣朔並差。又宋元嘉冬至影有七,張賓歷合者五,差者二,亦在前一日。張胄玄歷合者三,差者四,在
 後一日。元嘉十二年十一月甲寅朔,十五日戊辰冬至,日影長。張賓歷合戊辰冬至,張胄玄歷己巳冬至,差後一日。十三年十一月己酉朔,二十六日甲戌冬至,日影長。



 張賓歷癸酉冬至,差前一日,張胄玄歷合甲戌冬至。十五年十一月丁卯朔,十八日甲申冬至,日影長。二歷並合甲申冬至。十六年十一月辛酉朔,二十九日己丑冬至,日影長。張賓歷合己丑冬至,張胄玄歷庚寅冬至,差後一日。十七年十一月乙酉朔,十日甲午冬至,日影長。張賓歷合甲午冬至,張胄玄歷乙未冬至,差後一日。十八年十一月己卯朔,二十一日己亥冬至,日影長。張
 賓歷合己亥冬至,張胄玄歷庚子冬至,差後一日。十九年十一月癸卯朔,三日乙巳冬至,影長。張賓歷甲辰冬至,差前一日,張胄玄歷合乙巳冬至。



 又周從天和元年丙戌至開皇十五年乙卯,合得冬夏至日影一十四。張賓歷合得者十,差者四,三差前一日,一差後一日。張胄玄歷合者五,差者九,八差後一日,一差前一日。天和二年十一月戊戌朔,三日庚子冬至,日影長。張賓歷合庚子冬至,張胄玄歷辛丑冬至,差後一日。三年十一月壬辰朔,十四日乙巳冬至,日影長。張賓歷合乙巳冬至,張胄玄歷丙午冬至,差後一日。建德元年十一月己亥
 朔,二十九日丁卯冬至,日影長。張賓歷丙寅冬至,差前一日,張胄玄歷合丁卯冬至。二年五月丙寅朔,三日戊辰夏至,日影短。張賓歷己巳夏至,差後一日,張胄玄歷庚午夏至,差後二日。三年十一月戊午朔,二十日丁丑冬至,日影長。張賓歷合丁丑冬至,張胄玄歷戊寅冬至,差後一日。六年十一月庚午朔,二十三日壬辰冬至,日影長。



 張賓歷合壬辰冬至,張胄玄歷癸巳冬至,差後一日。宣政元年十一月甲午朔,五日戊戌冬至,日影長。兩歷並合戊戌冬至。開皇四年十一月己未朔,十一日己巳冬至,日影長。張賓歷合己巳冬至,張胄玄歷庚午冬
 至,差後一日。五年十一月甲寅朔,二十二日乙亥冬至,日影長。張賓歷甲戌冬至,差前一日,張胄玄歷合庚辰冬至。



 七年五月乙亥朔,九日癸未夏至,日影短。張賓歷壬午夏至,差前一日,張胄玄歷合癸未夏至。十一月壬申朔,十四日乙酉冬至,日影長。張賓歷合乙酉冬至,張胄玄歷丙戌冬至,差後一日。十一年十一月己卯朔,二十八日丙午冬至,日影長。張賓歷合丙午冬至,張胄玄歷丁未冬至,差後一日。十四年十一月辛酉朔旦冬至。張賓歷合十一月辛酉朔旦冬至,張胄玄歷十一月辛酉朔,二日壬戌冬至,差後一日。



 建德四年四月大、乙酉
 朔,三十日甲寅,月晨見東方。張賓歷四月大、乙酉朔,三十日甲寅,月晨見東方,張胄玄歷四月小、乙酉朔,五月大,甲寅朔,月晨見東方。



 宜案影極長為冬至,影極短為夏至,二至自古史分可勘者二十四,其二十一有影,三有至日無影。見行歷合一十八,差者六。旅騎尉張胄玄歷合者八,差者一十六,二差後二日,一十四差後一日。又開皇四年,在洛州測冬至影,與京師二處,進退絲毫不差。周天和已來案驗並在後。更檢得建德四年,晦朔東見;張胄玄歷,五月朔日,月晨見東方。今十七年,張賓歷閏七月,張胄玄歷閏五月。又審至以定閏,胄玄歷至
 既不當,故知置閏必乖。見行歷四月、五月頻大,張胄玄歷九月、十月頻大,為胄玄朔弱,頻大在後晨,故朔日殘月晨見東方。



 宜又案開皇四年十二月十五日癸卯,依歷月行在鬼三度,時加酉,月在卯上,食十五分之九,虧起西北。今伺候,一更一籌起食東北角,十五分之十,至四籌還生,至二更一籌復滿。五年六月三十日,依歷太陽虧,日在七星六度,加時在午少強上,食十五分之一半強,虧起西南角。今伺候,日乃在午後六刻上始食,虧起西北角,十五分之六,至未後一刻還生,至五刻復滿。六年六月十五日,依歷太陰虧,加時酉,在卯上,食十五
 分之九半弱,虧起西南,當其時陰雲不見月。至辰巳,雲裏見月,已食三分之二,虧從東北,既還雲合。至巳午間稍生,至午後,雲裏暫見,已復滿。十月三十日丁丑,依歷太陽虧,日在斗九度,時加在辰少弱上,食十五分之九強,虧起東北角。今候所見,日出山一丈,辰二刻始食,虧起正西,食三分之二,辰後二刻始生,入巳時三刻上復滿。十年三月十六日癸卯,依歷月行在氐七度,時加戌,月在辰太半上,食十五分之七半強,虧起東北。今候,月初出卯南,帶半食,出至辰初三分,可食二分許,漸生,辰未已復滿。見行歷九月十六日庚子,月行在胃四度,時
 加丑,月在未半強上,食十分之三半強,虧起正東。今伺候,月以午後二刻,食起正東,須臾如南,至未正上,食南畔五分之四,漸生,入申一刻半復滿。十二年七月十五日己未,依歷月行在室七度,時加戌,月在辰太強上,食十五分之十二半弱,虧起西北。今伺候,一更三籌起西北上,食準三分之二強,與歷注同。十三年七月十六日,依歷月在申半強上,食十五分之半弱,虧起西南。十五日夜,從四更候月,五更一籌起東北上,食半強,入雲不見。十四年七月一日,依歷時加巳弱上,食十五分之十二半強。至未後三刻,日乃食,虧起西北,食半許,入雲不
 見,食頃暫見,猶未復生,因即云鄣。十五年十一月十六日庚午,依歷月行在井十七度,時加亥,月在巳半上,食十五分之九半強,虧西北。其夜一更四籌後,月在辰上起食,虧東南,至二更三籌,月在巳上,食三分之二許,漸生,至三更一籌,月在丙上,復滿。十六年十一月十六日乙丑,依歷月行在井十七度,時加丑,月在未太弱上,食十五分之十二半弱,虧起東南。十五日夜伺候,至三更一籌,月在丙上,雲裏見,已食十五分之三許,虧起正東,至丁上,食既,後從東南生,至四更三籌,月在未末,復滿。而胄玄不能盡中。



 迭相駁難,高祖惑焉,逾時不決。會通
 事舍人顏慜楚上書云:「漢落下閎改《顓頊歷》作《太初歷》,云後八百歲,此歷差一日。」語在胄玄傳。高祖欲神其事,遂下詔曰:「朕應運受圖,君臨萬宇,思欲興復聖教,恢弘令典,上順天道,下授人時,搜揚海內,廣延術士。旅騎尉張胄玄,理思沉敏,術藝宏深,懷道白首,來上歷法。令與太史舊歷,並加勘審。仰觀玄象,參驗璇璣,胄玄歷數與七曜符合,太史所行,乃多疏舛,群官博議,咸以胄玄為密。太史令劉暉,司歷郭翟、劉宜,驍騎尉任悅,往經修造,致此乖謬。通直散騎常侍、領太史令庾季才,太史丞邢俊,司歷郭遠,歷博士蘇粲,歷助教傅俊、成珍等,既是職
 司,須審疏密。遂虛行此歷,無所發明。論暉等情狀,已合科罪,方共飾非護短,不從正法。季才等附下罔上,義實難容。」於是暉等四人,元造詐者,並除名;季才等六人,容隱奸慝,俱解見任。胄玄所造歷法,付有司施行。擢拜胄玄為員外散騎侍郎,領太史令。胄玄進袁充,互相引重,各擅一能,更為延譽。胄玄言充歷妙極前賢,充言胄玄歷術冠於今古。胄玄學祖沖之,兼傳其師法。自茲厥後,克食頗中。其開皇十七年所行歷術,命冬至起虛五度。後稍覺其疏,至大業四年劉焯卒後,乃敢改法,命起虛七度,諸法率更有增損,朔終義寧。今錄戊辰年所定歷
 術著之於此云。



 自甲子元至大業四年戊辰,百四十二萬七千六百四十四年,算外。



 章歲,四百一十。



 章閏,百五十一。



 章月,五千七十一。



 日法,千一百四十四。



 月法,三萬三千七百八十三。



 辰法,二百八十六。



 歲分,一千五百五十七萬二千九百六十三。



 度法,四萬二千六百四十。



 沒分,五百一十九萬一千三百一十一沒法,七萬四千五百二十一。



 周天分,一千五百五十七萬四千四百六十六。



 斗分,一萬八百六十六。



 氣法,四十六萬九千四十。



 氣時法,一萬六百六十。



 周日,二十七。



 日餘,一千四百一十三。



 周通,七萬二百九。



 周法,二千五百四十八。



 推積月術:置入元已來至所求年,以章月乘之,如章歲得一,為積月,餘為閏餘。閏餘三百九十七巳上,若冬至不在其月,加積月一推月朔弦望術:以月法乘積月,如法得一,為積日,餘為小餘。以六十去
 積日,餘為大餘,命以甲子算外,為所求年天正月朔日。天正月者,建子月也,今為去年十一月。凡朔小餘五百四十七巳上,其月大。



 加大餘七,小餘四百三十七太;凡四分一為少,二為半,三為太。小餘滿日法去之,從大餘;滿六十去之,命如前,為上弦日。又加,得望、下弦、後月朔。朔餘滿五百三十七,其月大,減者小餘。



 推二十四氣術:以月法乘閏餘,又以章歲乘朔小餘,加之,如氣法得一,為日,命朔算外,為冬至日。不盡者,以十一約之,為日分。



 求次氣:加日十五,日分九千三百一十五,小分一;小分滿八從日分一,日分滿度法從日一;如月大小去之,日
 不滿月,算外,為次氣日。其月無中氣者,為閏。



 求
 朔望入氣盈縮術:以入氣日算乘損益率,如十五得一,餘八已上,從一;以損益盈縮數為定盈縮。其入氣日十五算者,如十六得一,餘半法已上亦從一,以下皆準此。



 推土王術:加分至日二十七,日分一萬六千七百六十七,小分九;小分滿四十從日分一,滿去如前,即分至後土始王日。



 推沒日術:其氣有小分者,以八乘日分,內小分,又以十五乘之,以減沒分;無小分者,以百二十乘日分,以減之;滿沒法為日,不盡為日分,以其氣去朔日加之,去、命如前。



 求次沒:加日六十九,日分四萬九千三百七十二;日分滿沒法,從日,去、命如前。



 推入遲疾歷術:
 以周通去朔積日,餘以周法乘之,滿周通又去之,餘滿周法得一日,餘為日餘,即所求年天正朔算外夜半入歷日及餘。



 求次月:大月加二日,小月加一日,日餘皆千一百三十五,滿周日及日餘去之。



 求次日:加一,滿、去如前。



 求朔望加時入歷術:以四十九乘朔小餘,滿二十二得一為日餘,不盡為小分,以加夜半入歷日及餘分。



 求次月:加日一,餘二千四百八十六,小分二十一,滿、去如前,即次月入歷日及餘。



 求望:加日十四日,餘千九百四十九,小分二十一半,滿、去如前,為望入歷日及餘。



 推朔望加時定日及小餘術:以入歷日餘乘所入歷日損益率,以損益盈縮積分,如差法而一,為定積分。如差法乃與入氣定盈縮,皆以盈減、縮加本朔望小餘;不足減者,加日法乃減之,加時在往日;加之,滿日法者去之,則在來日;餘為定小餘。無食者不須氣盈縮。



 角十二度亢九度氐十五度房五度心
 五度尾十八度箕十一度東方七宿七十五度斗二十六度牛八度女十二度虛十度危十七度室十六度壁九度北方七宿九十八度奎十六度婁十二度胃十四度昴十一度畢十六度觜二度參九度西方七宿八十度井三十三度鬼四度柳十五度星七度張十
 八度翼十八度軫十七度南方七宿百一十
 二度
 推日度術:置入元至所求年,以歲分乘之,為通實,滿周天分去之,餘如度法而一,為積度,不盡為度分。命度以虛七度宿次去之,經斗去其分,度不滿宿,算外,即所求年天正冬至日所在度及分。以冬至去朔日以減分度數,分不足減者,減度一,加度法,乃減之,命如前,即天正朔前夜半日所在度及分。須求朔共度者,用去定用日數減之,俟後所須。



 求次月:大月加度三十,小月加度二十九,宿次去之,經斗去其
 分。



 求次日:加度一,去、命如前。



 求朔望加時日所在度術:各以定小餘乘章歲,滿十一為度分,以加其前夜半度分,滿之去如前。凡朔加時日月同度求轉分:以千四十約度分,不盡為小分。



 求望加時月所在度術:置望加時日所在度及分,加度一百八十二,轉分二十五,小分七百五十三;小分滿千四十從轉分一,轉分滿四十一從度;去、命如前,經斗去轉分十,小分四百六十
 六。



 求月行遲疾日轉定分術:以夜半入歷日餘乘轉差,滿周法得一為變差,以進加、退減日轉分為定分。



 推朔望夜半月定度術:以定小餘乘所入歷日轉定分,滿日法得一為分,分滿四十一為度,各以減加時月所在度,即各其前夜半定度。



 求次日:以日轉定分加轉分,滿四十一從度,去、命如前;朔日不用前加。



 推五星術:木數,千七百萬八千三百三十二四分火數,三千三百二十五萬六千二十六。



 土數,千六百一十二萬一千七百六十七。



 金數,二千四百八十九萬八千四百一十七。



 水數,四百九十四萬一千九十八。



 木終日,三百九十八,日分,三萬七千六百一十二四分。



 火終日,七百七十九,日分,三萬九千四百六十六。



 土終日,三百七十八,日分,三千八百四十七。



 金終日,五百八十三,日分,三萬九千二百九十七。晨見伏,三百二十七日,分同;
 夕見伏,二百五十六日。



 水終日,百一十五,日分,三萬七千四百九十八。晨見伏,六十三日,分同;夕見伏,五十二日。



 求星見術:置通實,各以數去之,餘以減數,其餘如度法得一為日,不盡為日分,即所求年天正冬至後晨平見日及分。其金、水,以夕見伏日去之,得者餘為夕平見日及分。



 求平見見月日:置冬至去朔日數及分,各以冬至後日數及分加之,分滿度法從日,起天正月,依大小去之,不滿月者為去朔日,命日算外,即星見所在月日及分。



 求後見:各以終日及分加之,滿去如前。其金、水各以晨夕加之,滿去如
 前,加晨得夕,加夕得晨。



 木:平見在春分前者,以三千三百四十乘去大寒後十日數,以加平見分,滿法去之,以為定見日及分。立秋後者,以四千二百乘去寒露日,加之,滿同前。春分至清明均加四日,後至立夏五日,以後至芒種加六日,均至立秋。小雪前者,以七千四百乘去寒露日數,以減平見日分;冬至後者,以八千三百乘去大寒後十日數,以減之;小雪至冬至均減八日,為定日數。初見伏去日各十四度。



 火:平見在雨水前,以二萬六千八百八十乘去大寒日
 數;在立夏後,以萬三千四百四十乘去立秋日數,以加見日分,滿去如前;雨水至立夏,均加二十九日。小雪前,以萬一千五百八十乘去處暑日數;冬至後,以三萬四千三百八十乘去大寒日數,滿去如前,以減之;小雪至冬至,均減二十五日。初見伏去日各十七度。



 土:平見在處暑前,以萬二千三百七十乘去大暑日數;白露後,以八千三百四十乘去霜降日數,以加見日分,滿去如前;處暑至白露均加九日。小寒前,以四千九百八十乘去霜降日數,小寒至立春均減九日,立春後減八日,啟蟄後去七,氣別去一,至穀雨去三,夏至後十日去
 一,至大暑去盡。初見伏去日各十七度。



 金:晨平見,在立春前者,以四千一百二十乘去小寒日數小滿後,以四千一百二十乘去夏至日數,以加見日分,滿去如前立春至小滿均加三日。立秋前,以四千一百二十乘去小暑日數,小雪後以四千一百二十乘去冬至日數,滿去如前,以減之,立秋至小雪均減三日。夕平見,在啟蟄前,以六千三百九十乘去小雪日數。清明後,以六千二百九十乘去芒種日數,滿去如前,以減之,啟蟄至清明均減九日。處暑前,以六千二百九十乘去夏至日數;寒露後,以六千二百九十乘去大雪日數;以加之,處暑至寒露均加九日。初見伏去日各十一度。



 水:晨平見,在雨水後、立夏前者,應見不見。啟蟄至雨水,
 去日十八度外、四十六度內,晨有木、火、土、金一星已上者,見;無者不見。立夏至小滿,去日度如前,晨有木、火、土、金一星已上者,見;無者亦不見。從霜降至小雪加一日,冬至至小寒減四日,立春至雨水減三日。冬至前,一去三,二去二,三去一。夕平見,在處暑後、霜降前者,應見不見。立秋至處暑,夕有星,去日如前者,見;無者亦不見。霜降至立冬,夕有星,去日如前者,見;無者亦不見。從穀雨至夏至,減二日。初見伏去日各十七度。



 行五星法:置星定見之前夜半日所在宿度算及分,各以定見日
 分加其分,滿度法從度。又以星初見去日度數,晨減、夕加之,滿去如前,即星初見所在度及分。



 求次日:各加一日所行度及分,有小分者,各日數為母,小分滿其母去從分,分滿度法從度。其行有益疾遲者,副置一日行分,各以其分疾益遲損之。留者因前,退則減之,伏不注度,順行出斗去其分,退行入斗先加分。訖,皆以千四十約分,為大分,以四十一為母。



 木:初見,順,日行萬六百一十八分,日益遲六十分,一百一十四日行十九度、萬三千八百三十二分而留。二十六日乃退,日六千一百一分,八十四日退十二度、八百四分。又留二十五日、三萬七千六百一十二分、小分四,
 乃順。初日行三千八百三十七分,日益疾六十分,百一十四日行十九度、萬三千七百一十八分而伏。



 土:初見,順,日行三千八百一十四分,八十三日行七度、萬八千八十二分而留。三十八日乃退,日二千五百六十三分,百日退六度、四百六十分。又留三十七日、三千八百四十七分乃順,日三千八百一十三分,八十三日行七度萬七千九百九十九分,如初乃伏。



 火:初見已後各如其法:
 見
 在雨水前,以見去小寒日數,小滿後,以去大暑日數;三約之,所得減日為定日;雨水至小滿,均去二十日為定日。已前皆前疾日數及度數。各計冬至後日數,依損益之,為定日數及度數。以度法乘定度,如定日得一,即平行一日分,不盡為小分。大寒至立秋差行,餘平行。處暑至白露,皆去定日,定度六。白露至寒露,初日行半度,四十日行二十度,餘日及餘度續同前。置日數減一,以三十乘之,加平行一日分,為初日分。差行者,日益遲六十分,各盡其日度而遲。初日行二萬六百分,日益遲百分,六十日行二十四度、
 三萬五千六百四十分其前疾去度六者,此遲初日加四千二百六十四分,六十日行三十度,分同。而留。十三日前去日者,分、日於二留,奇縱後留。乃退,日萬二千八十二分,六十日退十七度、四十分。又留,十二日三萬九千四百六十六分。又順,遲,初日行萬四千七百分,日益疾百分,六十日行二十四度,分同前,此遲在立秋至秋分加一日,行分四千二百六十四,六十日行四十度,分同前。而後疾。



 後遲加六度者,此後疾去度為定度,已前皆後疾日數及度數。其在立夏至,小暑,日行半度,盡六十日,行三十度。小暑至立秋,盡四十日,行二十度。計餘日及度,從前法。前法皆平行。求行分亦如前。各盡其日度而伏。



 金:晨初見,乃退,日半度,十日退五度而留。九日乃順,遲,差行,先遲日益五百分,四十日行三十度。小暑前以去芒種日數,十日減一度;立冬後以去大雪日數,十日減一度;小暑至立冬,均減三度為定度。大雪至芒種不加減。求初日,以三十乘度法,四十得一為平分。又以三十九乘二百五十,以減平分為初日行分。



 平行,日一度,
 十五日行十五度。小寒後十日,益日度各一,至雨水二十一日,行二十一度。均至春分後十日減一,至小滿,復十五日行十五度。其後六日減一,至處暑,日及度皆盡。至霜降後,四日益一,至冬至復十五日行十五度疾,百七十日行二百四度。前順遲減度者,計減數益此度為定度。求一日行度分者,以百七十日日一度以減定度,餘乘度法,如百七十得一,為一日平行度分。晨伏東方。夕初見,順,疾,百七十日行二百四度。夏至前,以見去小滿日數,六日加一度;小暑後,以去立秋日數,六日加一度,夏至至小暑均加五度,為定度。白露至清明,差行,先疾日益遲百分。清明至白露,平行,求一日平行同,晨疾求差行,以五十乘百六十九,加之,為初日行度分。平行,日一度,十五日行十五度。冬至後十日減日度各一,至啟蟄九日行九度。均至夏至後五日益一,至大暑復十五日行十五度。均至立秋後六日益一,至寒露二十五日行二十五度。後六日減一,至大雪復十五日行十五度,均至冬至。順,遲,差行,先疾,日益五百分,四十日行三
 十度。前加度者,此依數減之,求初日行分。如晨遲,唯減者為加之。又留,九日乃退,日半度,十日退五度,而夕伏西方。



 水:晨初見,留六日。順,遲,日行萬六百六十分,四日行一度。大寒至雨水不須此遲行。平行,日一度,十日行十度。大寒後二日,去日度各一,盡二十日,日及度俱盡。疾,日行一度三萬八千三百七十六分,十日行十九度,前無遲行者,減此分萬二千七百九十二分,十日行十六度。晨伏東方。夕初見,順,疾,日行一度三萬八千三百七十六分,十日行十九度。小暑至白露減萬二千七百九十二分,十日行十六度。平行,日一度,十日行十度。大暑後二日,去日度各一,盡二十日,日及度俱盡。遲,日行萬六百六十分,四日行一度。疾減萬二千七百九十二分者,不須此遲。行又留六日,夕伏
 西方。



 推交會術:會通,千六十四萬六千七百二十九。



 朔差,九十萬七千五十七。



 望差,四十五萬三千五百二十八半。



 單數,五百三十二萬三千三百六十四半。



 時法,三萬二千六百四。



 望數,五百七十七萬六千八百九十三。



 外限,四百八十六萬九千八百三十六。



 內限,千一十九萬三千二百半。



 中限,五百六十四萬九千四百四半。



 次限,千三十二萬六百八十九。



 推入交法:以會通去積月,餘以朔望差乘之,滿會通又去之,餘為所求年天正朔入交餘。



 求望,望數加之,滿、去如前。



 求次月,以朔差加之,滿、去如前。



 推交道內外及先後去交術:其朔望在啟蟄前,以一千三百八十乘去小寒日數;在穀雨後,以乘去芒種日數,為氣差以加之,啟蟄至穀
 雨均加六萬三千六百;滿會通去之,餘為定餘。其小寒至春分,立夏至芒種,朔值盈二時已下,皆半氣差而加之;二時已上,皆不加。朔入交餘如望差、望數已下,中限已上,有星伏,木、土去見十日外,火去見四十日外,金、晨伏去見二十二日外。有一星者不加氣差。朔望在白露前者,以九百乘去小暑日數;在立冬後者,以千七百七十乘去大雪日數,以減之;白露至立冬均減五萬五千,不足減者,加會通乃減之,餘為定餘。朔入交餘如外限、內限已上,單數次限已下有星伏,如前者,不減氣差。定餘不滿單數者,為在外;滿去之,餘在內。其餘如望差已下、外限已上,望則月食;在內者,朔則日食。其餘如望差已下者,即為去先交餘;如外限已上者,以減單數,餘為去後交餘。如時法得一,然為去交時數。



 推月食加時術:置食定日小餘,三之,如辰法得一辰,命以子算外,即所在辰。不盡為時餘,四之,如法,無所得為辰初,一為少,二為半,三為太。又不盡者,三之,如法,得一為強,以並少為少強,並半為半強,並太為太強;得二強者為少弱,並少為半弱,並半為太弱,並太為辰末。此加時謂食時月在沖也。



 推日食加時術:置食定日小餘,秋三月,內道,去交八時已上,加二十四,十二時以加四十八;春三月,內道,去交七時已上,加二十四。乃以三乘之,如辰法得一辰,以命子算外,即所在
 辰。不盡為時餘。副置時餘,仲辰不滿半辰,減半辰,已上去半辰;季辰者直加半辰;孟辰者減辰法,餘加半辰為差率。



 又,置去交時數,三已下加三,六已下加二,九已下加一,九已上依數,十二已上從十二;以乘差率,如十四得一為時差。子半至卯半、午半至酉半,以加時餘;卯半至午半、酉半至子半,以減時餘。加之,滿辰法去之,進一辰,減之若不足,退一辰,餘為定時餘。乃如月食法,子午卯酉為仲,辰戌丑未為季,寅申巳亥為孟。



 日出前入後各二時外,不注日食。三乘氣時法得一,命子算外為時。



 求外道日食法:去交一時內者,食。夏去交二時內,加時在南方三辰者,食。若去分至十二時內,去交六時內者,亦食。若去春分三日內,後交二時內,秋分三日內,先交二時內者,亦食。先交二時內,值盈二時外,及後交二時內,值縮二時外,亦食。諸去交三時內,星伏如前者,食。



 求內道日不食法:加時南方三辰,五月朔先交十三時外,六月朔後交十三時外,不食。啟蟄至穀雨,先交十三時外,值縮加時在未以西者,不食。處暑至霜降,後交十三時外,值盈加時在
 巳以東者,不食。



 求月食分:春後交、秋先交、冬後交,皆去不食餘一時,不足去者,食既。餘以三萬二百三十五為法,得一為不食分。不盡者,半法已上為半強,已下為半弱,以減十五,餘為食分。



 推日食分術:在秋分前者,以去夏至日數乘二千,以減去交餘,餘為不食餘;不足減者,反減十八萬四千,餘為不食餘。亦減望差為定法。其後交值縮,並不減望差,直以望差為定法。在啟蟄後者,以去夏至日數乘千五百以減之;秋分至啟蟄,均減十八萬四千,不足
 減者,如前;大寒至小滿,去後交五時外,皆去不食餘一時。時差減者,先交減之,後交加之,不足減者食既;值加,先交加之,後交減之。不足減者食。



 求所起:內道西北,虧東北;外道西南,虧東南。十三分以上,正左起。虧皆據甚時,月則行上起。



 求日出入
 所在術:
 以所入氣辰刻及分,與後氣辰刻及分相減,餘乘入氣日算,如十五得一,以損益所入氣,依刻及分為定刻。



\end{pinyinscope}