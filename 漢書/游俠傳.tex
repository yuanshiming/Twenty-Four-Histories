\article{游俠傳}

\begin{pinyinscope}
古者天子建國,諸侯立家,自卿大夫以至于庶人各有等差,是以民服事其上,而下無覬覦。孔子曰:「天下有道,政不在大夫。」百官有司奉法承令,以脩所職,失職有誅,侵官有罰。夫然,故上下相順,而庶事理焉。

周室既微,禮樂征伐自諸侯出。桓文之後,大夫世權,陪臣執命。陵夷至於戰國,合從連衡,力政爭彊。繇是列國公子,魏有信陵,趙有平原,齊有孟嘗,楚有春申,皆藉王公之勢,競為游俠,雞鳴狗盜,無不賓禮。而趙相虞卿棄國捐君,以周窮交魏齊之厄;信陵無忌竊符矯命,戮將專師,以赴平原之急:皆以取重諸侯,顯名天下。搤腕而游談者,以四豪為稱首。於是背公死黨之議成,守戰奉上之義廢矣。

及至漢興,禁網疏闊,未之匡改也。是故代相陳豨從車千乘,而吳濞、淮南皆招賓客以千數。外戚大臣魏其、武安之屬競逐於京師,布衣游俠劇孟、郭解之徒馳騖於閭閻,權行州域,力折公侯。眾庶榮其名跡,覬而慕之。雖其陷於刑辟,自與殺身成名,若季路、仇牧,死而不悔也。故曾子曰:「上失其道,民散久矣。」非明王在上,視之以好惡,齊之以禮法,民曷繇知禁而反正乎!

古之正法:五伯,三王之罪人也;而六國,五伯之罪人也。夫四豪者,又六國之罪人也。況於郭解之倫,以匹夫之細,竊殺生之權,其罪已不容於誅矣。觀其溫良泛愛,振窮周急,謙退不伐,亦皆有絕異之姿。惜乎不入於道德,苟放縱於末流,殺身亡宗,非不幸也!

自魏其、武安、淮南之後,天子切齒,衛、霍改節。然郡國豪桀處處各有,京師親戚冠蓋相望,亦古今常道,莫足言者。唯成帝時,外家王氏賓客為盛,而樓護為帥。及王莽時,諸公之間陳遵為雄,閭里之俠原涉為魁。

朱家,魯人,高祖同時也。魯人皆以儒教,而朱家用俠聞。所臧活豪士以百數,其餘庸人不可勝言。然終不伐其能,飲其德,諸所嘗施,唯恐見之。振人不贍,先從貧賤始。家亡餘財,衣不兼采,食不重味,乘不過軥牛。專趨人之急,甚於己私。既陰脫季布之厄,及布尊貴,終身不見。自關以東,莫不延頸願交。楚田仲以俠聞,父事朱家,自以為行弗及也。田仲死後,有劇孟。

劇孟者,洛陽人也。周人以商賈為資,劇孟以俠顯。吳楚反時,條侯為太尉,乘傳東,將至河南,得劇孟,喜曰:「吳楚舉大事而不求劇孟,吾知其無能為已。」天下騷動,大將軍得之若一敵國云。劇孟行大類朱家,而好博,多少年之戲。然孟母死,自遠方送喪蓋千乘。及孟死,家無十金之財。而符離王孟,亦以俠稱江淮之間。是時,濟南瞷氏、陳周膚亦以豪聞。景帝聞之,使使盡誅此屬。其後,代諸白、梁韓毋辟、陽翟薛況、陝寒孺,紛紛復出焉。

郭解,河內軹人也,溫善相人許負外孫也。解父任俠,孝文時誅死。解為人靜悍,不飲酒。少時陰賊感概,不快意,所殺甚眾。以軀耤友報仇,臧命作姦剽攻,休乃鑄錢掘冢,不可勝數。適有天幸,窘急常得脫,若遇赦。

及解年長,更折節為儉,以德報怨,厚施而薄望。然其自喜為俠益甚。既已振人之命,不矜其功,其陰賊著於心本發於睚眥如故云。而少年慕其行,亦輒為報讎,不使知也。

解姊子負解之勢,與人飲,使之釂,非其任,彊灌之。人怒,刺殺解姊子,去亡。解姊怒曰:「以翁伯時人殺吾子,賊不得!」棄其尸道旁,弗葬,欲以辱解。解使人微知賊處。賊窘自歸,具以實告解。解曰:「公殺之當,吾兒不直。」遂去其賊,罪其姊子,收而葬之。諸公聞之,皆多解之義,益附焉。

解出,人皆避,有一人獨箕踞視之。解問其姓名,客欲殺之。解曰:「居邑屋不見敬,是吾德不脩也,彼何罪!」乃陰請尉史曰:「是人吾所重,至踐更時脫之。」每至直更,數過,吏弗求。怪之,問其故,解使脫之。箕踞者乃肉袒謝罪。少年聞之,愈益慕解之行。

洛陽人有相仇者,邑中賢豪居間以十數,終不聽。客乃見解。解夜見仇家,仇家曲聽。解謂仇家:「吾聞洛陽諸公在間,多不聽。今子幸而聽解,解奈何從它縣奪人邑賢大夫權乎!」乃夜去,不使人知,曰:「且毋庸,待我去,令洛陽豪居間乃聽。」

解為人短小,恭儉,出未嘗有騎,不敢乘車入其縣庭。之旁郡國,為人請求事,事可出,出之;不可者,各令厭其意,然後乃敢嘗酒食。諸公以此嚴重之,爭為用。邑中少年及旁近縣豪夜半過門,常十餘車,請得解客舍養之。

及徙豪茂陵也,解貧,不中訾。吏恐,不敢不徙。衛將軍為言「郭解家貧,不中徙」。上曰:「解布衣,權至使將軍,此其家不貧!」解徙,諸公送者出千餘萬。軹人楊季主子為縣掾,鬲之,解兄子斷楊掾頭。解入關,關中賢豪知與不知,聞聲爭交驩。邑人又殺楊季主,季主家上書人又殺闕下。上聞,乃下吏捕解。解亡,置其母家室夏陽,身至臨晉。臨晉籍少翁素不知解,因出關。籍少翁已出解。解傳太原,所過輒告主人處。吏逐跡至籍少翁,少翁自殺,口絕。久之得解,窮治所犯為,而解所殺,皆在赦前。

軹有儒生侍使者坐,客譽郭解,生曰:「解專以姦犯公法,何謂賢?」解客聞之,殺此生,斷舌。吏以責解,解實不知殺者,殺者亦竟莫知為誰。吏奏解無罪。御史大夫公孫弘議曰:「解布衣為任俠行權,以睚眥殺人,解不知,此罪甚於解知殺之。當大逆無道。」遂族解。

自是之後,俠者極眾,而無足數者。然關中長安樊中子,槐里趙王孫,長陵高公子,西河郭翁中,太原魯翁孺,臨淮兒長卿,東陽陳君孺,雖為俠而恂恂有退讓君子之風。至若北道姚氏,西道諸杜,南道仇景,東道佗羽公子,南陽趙調之徒,盜跖而居民間者耳,曷足道哉!此乃鄉者朱家所羞也。

萭章字子夏,長安人也。長安熾盛,街閭各有豪俠,章在城西柳市,號曰「城西萭子夏」。為京兆尹門下督,從至殿中,侍中諸侯貴人爭欲揖章,莫與京兆尹言者。章逡循甚懼。其後京兆不復從也。

與中書令石顯相善,亦得顯權力,門車常接轂。至成帝初,石顯坐專權擅勢免官,徙歸故郡。顯貲巨萬,當去,留床席器物數百萬直,欲以與章,章不受。賓客或問其故,章歎曰:「吾以布衣見哀於石君,石君家破,不能有以安也,而受其財物,此為石氏之禍,萭氏反當以為福邪!」諸公以是服而稱之。

河平中,王尊為京兆尹,捕擊豪俠,殺章及箭張回、酒市趙君都、賈子光,皆長安名豪,報仇怨養刺客者也。

樓護字君卿,齊人。父世醫也,護少隨父為醫長安,出入貴戚家。護誦醫經、本草、方術數十萬言,長者咸愛重之,共謂曰:「以君卿之材,何不宦學乎?」繇是辭其父,學經傳,為京兆吏數年,甚得名譽。

是時王氏方盛,賓客滿門,五侯兄弟爭名,其客各有所厚,不得左右,唯護盡入其門,咸得其驩心。結士大夫,無所不傾,其交長者,尤見親而敬,眾以是服。為人短小精辯,論議常依名節,聽之者皆竦。與谷永俱為五侯上客,長安號曰「谷子雲筆札,樓君卿脣舌」,言其見信用也。母死,送葬者致車二三千兩,閭里歌之曰:「

五侯治喪樓君卿。」

久之,平阿侯舉護方正,為諫大夫,使郡國。護假貸,多持幣帛,過齊,上書求上先人冢,因會宗族故人,各以親疏與束帛,一日散百金之費。使還,奏事稱意,擢為天水太守。數歲免,家長安中。時成都侯商為大司馬衛將軍,罷朝,欲候護,其主簿諫:「將軍至尊,不宜入閭巷。」商不聽,遂往至護家。家狹小,官屬立車下,久住移時,天欲雨,主簿謂西曹諸掾曰:「不肯彊諫,反雨立閭巷!」商還,或白主簿語,商恨,以他職事去主簿,終身廢錮。

後護復以薦為廣漢太守。元始中,王莽為安漢公,專政,莽長子宇與妻兄呂寬謀以血塗莽第門,欲懼莽令歸政。發覺,莽大怒,殺宇,而呂寬亡。寬父素與護相知,寬至廣漢過護,不以事實語也。到數日,名捕寬詔書至,護執寬。莽大喜,徵護入為前煇光,封息鄉侯,列於九卿。

莽居攝,槐里大賊趙朋、霍鴻等群起,延入前煇光界,護坐免為庶人。其居位,爵祿賂遺所得亦緣手盡。既退居里巷,時五侯皆已死,年老失勢,賓客益衰。至王莽篡位,以舊恩召見護,封為樓舊里附城。而成都侯商子邑為大司空,貴重,商故人皆敬事邑,唯護自安如舊節,邑亦父事之,不敢有闕。時請召賓客,邑居樽下,稱「

賤子上壽」。坐者百數,皆離席伏,護獨東鄉正坐,字謂邑曰:「公子貴如何!」

初,護有故人呂公,無子,歸護。護身與呂公、妻與呂嫗同食。及護家居,妻子頗厭呂公。護聞之,流涕責其妻子曰:「呂公以故舊窮老託身於我,義所當奉。」遂養呂公終身。護卒,子嗣其爵。

陳遵字孟公,杜陵人也。祖父遂,字長子,宣帝微時與有故,相隨博弈,數負進。及宣帝即位,用遂,稍遷至太原太守,乃賜遂璽書曰:「制詔太原太守:官尊祿厚,可以償博進矣。妻君寧時在旁,知狀。」遂於是辭謝,因曰:「事在元平元年赦令前。」其見厚如此。元帝時,徵遂為京兆尹,至廷尉。

遵少孤,與張竦伯松俱為京兆史。竦博學通達,以廉儉自守,而遵放縱不拘,操行雖異,然相親友,哀帝之末俱著名字,為後進冠。並入公府,公府掾史率皆羸車小馬,不上鮮明,而遵獨極輿馬衣服之好,門外車騎交錯。又日出醉歸,曹事數廢。西曹以故事適之,侍曹輒詣寺舍白遵曰:「陳卿今日以某事適。」遵曰:「滿百乃相聞。」故事,有百適者斥,滿百,西曹白請斥。大司徒馬宮大儒優士,又重遵,謂西曹:「此人大度士,奈何以小文責之?」乃舉遵能治三輔劇縣,補郁夷令。久之,與扶風相失,自免去。

槐里大賊趙朋、霍鴻等起,遵為校尉,擊朋、鴻有功,封嘉威侯。居長安中,列侯近臣貴戚皆重貴之。牧守當之官,及郡國豪桀至京師者,莫不相因到遵門。

遵耆酒,每大飲,賓客滿堂,輒關門,取客車轄投井中,雖有急,終不得去。嘗有部刺史奏事,過遵,值其方飲,刺史大窮,候遵霑醉時,突入見遵母,叩頭自白當對尚書有期會狀,母乃令從從閤出去。遵大率常醉,然事亦不廢。

長八尺餘,長頭大鼻,容貌甚偉。略涉傳記,贍於文辭。性善書,與人尺牘,主皆藏去以為榮。請求不敢逆,所到,衣冠懷之,唯恐在後。時列侯有與遵同姓字者,每至人門,曰陳孟公,坐中莫不震動,既至而非,因號其人曰陳驚坐云。

王莽素奇遵材,在位多稱譽者,繇是起為河南太守。既至官,當遣從史西,召善書吏十人於前,治私書謝京師故人。遵馮几,口占書吏,且省官事,書數百封,親疏各有意,河南大驚。數月免。

初,遵為河南太守,而弟級為荊州牧,當之官,俱過長安富人故淮陽王外家左氏飲食作樂。後司直陳崇聞之,劾奏「遵兄弟幸得蒙恩超等歷位,遵爵列侯,備郡守,級州牧奉使,皆以舉直察枉宣揚聖化為職,不正身自慎。始遵初除,乘藩車入閭巷,過寡婦左阿君置酒歌謳,遵起舞跳梁,頓仆坐上,暮因留宿,為侍婢扶臥。遵知飲酒飫宴有節,禮不入寡婦之門,而湛酒溷肴,亂男女之別,輕辱爵位,羞汙印韍,惡不可忍聞。臣請皆免。」遵既免,歸長安,賓客愈盛,飲食自若。

久之,復為九江及河內都尉,凡三為二千石。而張竦亦至丹陽太守,封淑德侯。後俱免官,以列侯歸長安。竦居貧,無賓客,時時好事者從之質疑問事,論道經書而已。而遵晝夜呼號,車騎滿門,酒肉相屬。

先是黃門郎揚雄作酒箴以諷諫成帝,其文為酒客難法度士,譬之於物,曰:「子猶瓶矣。觀瓶之居,居井之眉,處高臨深,動常近危。酒醪不入口,臧水滿懷,不得左右,牽於纆徽。一旦赙礙,為瓽所轠,身提黃泉,骨肉為泥。自用如此,不如鴟夷。鴟夷滑稽,腹如大壺,盡日盛酒,人復借酤。常為國器,託於屬車,出入兩宮,經營公家。繇是言之,酒何過乎!」遵大喜之,常謂張竦:「吾與爾猶是矣。足下諷誦經書,苦身自約,不敢差跌,而我放意自恣,浮湛俗間,官爵功名,不減於子,而差獨樂,顧不優邪!」竦曰:「人各有性,長短自裁。子欲為我亦不能,吾而效子亦敗矣。雖然,學我者易持,效子者雖將,吾常道也。」

及王莽敗,二人俱客於池陽,竦為賊兵所殺。更始至長安,大臣薦遵為大司馬護軍,與歸德侯劉颯俱使匈奴。單于欲脅詘遵,遵陳利害,為言曲直,單于大奇之,遣還。會更始敗,遵留朔方,為賊所敗,時醉見殺。

原涉字巨先。祖父武帝時以豪桀自陽翟徙茂陵。涉父哀帝時為南陽太守。天下殷富,大郡二千石死官,賦斂送葬皆千萬以上,妻子通共受之,以定產業。時又少行三年喪者。及涉父死,讓還南陽賻送,行喪冢廬三年,繇是顯名京師。禮畢,扶風謁請為議曹,衣冠慕之輻輳。為大司徒史丹舉能治劇,為谷口令,時年二十餘。谷口聞其名,不言而治。

先是涉季父為茂陵秦氏所殺,涉居谷口半歲所,自劾去官,欲報仇。谷口豪桀為殺秦氏,亡命歲餘,逢赦出。郡國諸豪及長安、五陵諸為氣節者皆歸慕之。涉遂傾身與相待,人無賢不肖闐門,在所閭里盡滿客。或譏涉曰:「子本吏二千石之世,結髮自修,以行喪推財禮讓為名,正復讎取仇,猶不失仁義,何故遂自放縱,為輕俠之徒乎?」涉應曰:「子獨不見家人寡婦邪?始自約敕之時,意乃慕宋伯姬及陳孝婦,不幸壹為盜賊所汙,遂行淫失,知其非禮,然不能自還。吾猶此矣!」

涉自以為前讓南陽賻送,身得其名,而令先人墳墓儉約,非孝也。乃大治起冢舍,周閣重門。初,武帝時,京兆尹曹氏葬茂陵,民謂其道為京兆仟。涉慕之,乃買地開道,立表署曰南陽仟,人不肯從,謂之原氏仟。費用皆卬富人長者,然身衣服車馬纔具,妻子內困。專以振施貧窮赴人之急為務。人嘗置酒請涉,涉入里門,客有道涉所知母病避疾在里宅者。涉即往候,叩門。家哭,涉因入弔,問以喪事。家無所有,涉曰:「但絜埽除沐浴,待涉。」還至主人,對賓客歎息曰:「人親臥地不收,涉何心鄉此!願徹去酒食。」賓客爭問所當得,涉乃側席而坐,削牘為疏,具記衣被棺木,下至飯含之物,分付諸客。諸客奔走市買,至日昳皆會。涉親閱視已,謂主人:「願受賜矣。」既共飲食,涉獨不飽,乃載棺物,從賓客往至喪家,為棺斂勞來畢葬。其周急待人如此。後人有毀涉者曰「姦人之雄也」,喪家子即時刺殺言者。

賓客多犯法,罪過數上聞。王莽數收繫欲殺,輒復赦出之。涉懼,求為卿府掾史,欲以避客。文母太后喪時,守復土校尉。已為中郎,后免官。涉欲上冢,不欲會賓客,密獨與故人期會。涉單車敺上茂陵,投暮,入其里宅,因自匿不見人。遣奴至市買肉,奴乘涉氣與屠爭言,斫傷屠者,亡。是時,茂陵守令尹公新視事,涉未謁也,聞之大怒。知涉名豪,欲以示眾厲俗,遣兩吏脅守涉。至日中,奴不出,吏欲便殺涉去。涉迫窘不知所為。會涉所與期上冢者車數十乘到,皆諸豪也,共說尹公。尹公不聽,諸豪則曰:「原巨先奴犯法不得,使肉袒自縛,箭貫耳,詣廷門謝罪,於君威亦足矣。」尹公許之。涉如言謝,復服遣去。

初,涉與新豐富人祁太伯為友,太伯同母弟王游公素嫉涉,時為縣門下掾,說尹公曰:「君以守令辱原涉如是,一旦真令至,君復單車歸為府吏,涉刺客如雲,殺人皆不知主名,可為寒心。涉治冢舍,奢僭踰制,罪惡暴著,主上知之。今為君計,莫若墮壞涉冢舍,條奏其舊惡,君必得真令。如此,涉亦不敢怨矣。」尹公如其計,莽果以為真令。涉繇此怨王游公,選賓客,遣長子初從車二十乘劫王游公家。游公母即祁太伯母也,諸客見之皆拜,傳曰「無驚祁夫人」。遂殺游公父及子,斷兩頭去。

涉性略似郭解,外溫仁謙遜,而內隱好殺。睚眥於塵中,獨死者甚多。王莽末,東方兵起,諸王子弟多薦涉能得士死,可用。莽乃召見,責以罪惡,赦貰,拜鎮戎大尹天水太守。涉至官無幾,長安敗,郡縣諸假號起兵攻殺二千石長吏以應漢。諸假號素聞涉名,爭問原尹何在,拜謁之。時莽州牧使者依附涉者皆得活。傳送致涉長安,更始西屏將軍申屠建請涉與相見,大重之。故茂陵令尹公壞涉冢舍者為建主簿,涉本不怨也。涉從建所出,尹公故遮拜涉,謂曰:「易世矣,宜勿復相怨!」涉曰:「尹君,何壹魚肉涉也!」涉用是怒,使客刺殺主簿。

涉欲亡去,申屠建內恨恥之,陽言「吾欲與原巨先共鎮三輔,豈以一吏易之哉!」賓客通言,令涉自繫獄謝,建許之。賓客車數十乘共送涉至獄。建遣兵道徼取涉於車上,送車分散馳,遂斬涉,縣之長安市。

自哀、平間,郡國處處有豪桀,然莫足數。其名聞州郡者,霸陵杜君敖,池陽韓幼孺,馬領繡君賓,西河漕中叔,皆有謙退之風。王莽居攝,誅鉏豪俠,名捕漕中叔,不能得。素善強弩將軍孫建,莽疑建藏匿,泛以問建。建曰:「臣名善之,誅臣足以塞責。」莽性果賊,無所容忍,然重建,不竟問,遂不得也。中叔子少游,復以俠聞於世云。


\end{pinyinscope}