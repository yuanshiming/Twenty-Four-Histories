\article{西域傳}

\begin{pinyinscope}
西域以孝武時始通,本三十六國,其後稍分至五十餘,皆在匈奴之西,烏孫之南。南北有大山,中央有河,東西六千餘里,南北千餘里。東則接漢,阨以玉門、陽關,西則限以蔥嶺。其南山,東出金城,與漢南山屬焉。其河有兩原:一出蔥嶺山,一出于闐。于闐在南山下,其河北流,與蔥嶺河合,東注蒲昌海。蒲昌海,一名鹽澤者也,去玉門、陽關三百餘里,廣袤三百里。其水亭居,冬夏不增減,皆以為潛行地下,南出於積石,為中國河云。

自玉門、陽關出西域有兩道。從鄯善傍南山北,波河西行至莎車,為南道;南道西踰蔥嶺則出大月氏、安息。自車師前王廷隨北山,波河西行至疏勒,為北道;北道西踰蔥嶺則出大宛、康居、奄蔡焉耆。

西域諸國大率土著,有城郭田畜,與匈奴、烏孫異俗,故皆役屬匈奴。匈奴西邊日逐王置僮僕都尉,使領西域,常居焉耆、危須、尉黎間,賦稅諸國,取富給焉。

自周衰,戎狄錯居涇渭之北。及秦始皇攘卻戎狄,築長城,界中國,然西不過臨洮。

漢興至于孝武,事征四夷,廣威德,而張騫始開西域之跡。其後驃騎將軍擊破匈奴右地,降渾邪、休屠王,遂空其地,始築令居以西,初置酒泉郡,後稍發徙民充實之,分置武威、張掖、敦煌,列四郡,據兩關焉。自貳師將軍伐大宛之後,西域震懼,多遣使來貢獻,漢使西域者益得職。於是自敦煌西至鹽澤,往往起亭,而輪臺、渠犁皆有田卒數百人,置使者校尉領護,以給使外國者。

至宣帝時,遣衛司馬使護鄯善以西數國。及破姑師,未盡殄,分以為車師前後王及山北六國。時漢獨護南道,未能盡并北道也,然匈奴不自安矣。其後日逐王畔單于,將眾來降,護鄯善以西使者鄭吉迎之。既至漢,封日逐王為歸德侯,吉為安遠侯。是歲,神爵三年也。乃因使吉并護北道,故號曰都護。都護之起,自吉置矣。僮僕都尉由此罷,匈奴益弱,不得近西域。於是徙屯田,田於北胥鞬,披莎車之地,屯田校尉始屬都護。都護督察烏孫、康居諸外國動靜,有變以聞。可安輯,安輯之;可擊,擊之。都護治烏壘城,去陽關二千七百三十八里,與渠犁田官相近,土地肥饒,於西域為中,故都護治焉。

至元帝時,復置戊己校尉,屯田車師前王庭。是時匈奴東蒲類王茲力支將人眾千七百餘人降都護,都護分車師後王之西為烏貪訾離地以處之。

自宣、元後,單于稱藩臣,西域服從,其土地山川王侯戶數道里遠近翔實矣。

出陽關,自近者始,曰婼羌。婼羌國王號去胡來王。去陽關千八百里,去長安六千三百里,辟在西南,不當孔道。戶四百五十,口千七百五十,勝兵者五百人。西與且末接。隨畜逐水草,不田作,仰鄯善、且末穀。山有鐵,自作兵,兵有弓、矛、服刀、劍、甲。西北至鄯善,乃當道云。

鄯善國,本名樓蘭,王治扜泥城,去陽關千六百里,去長安六千一百里。戶千五百七十,口萬四千一百,勝兵二千九百十二人。輔國侯、卻胡侯、鄯善都尉、擊車師都尉、左右且渠、擊車師君各一人,譯長二人。西北去都護治所千七百八十五里,至山國千三百六十五里,西北至車師千八百九十里。地沙鹵,少田,寄田仰穀旁國。國出玉,多葭葦、檉柳、胡桐、白草。民隨畜牧逐水草,有驢馬,多橐它。能作兵,與婼羌同。

初,武帝感張騫之言,甘心欲通大宛諸國,使者相望於道,一歲中多至十餘輩。樓蘭、姑師當道,苦之,攻劫漢使王恢等,又數為匈奴耳目,令其兵遮漢使。漢使多言其國有城邑,兵弱易擊。於是武帝遣從票侯趙破奴將屬國騎及郡兵數萬擊姑師。王恢數為樓蘭所苦,上令恢佐破奴將兵。破奴與輕騎七百人先至,虜樓蘭王,遂破姑師,因暴兵威以動烏孫、大宛之屬。還,封破奴為浞野侯,恢為浩侯。於是漢列亭障至玉門矣。

樓蘭既降服貢獻,匈奴聞,發兵擊之。於是樓蘭遣一子質匈奴,一子質漢。後貳師軍擊大宛,匈奴欲遮之,貳師兵盛不敢當,即遣騎因樓蘭候漢使後過者,欲絕勿通。時漢軍正任文將兵屯玉門關,為貳師後距,捕得生口,知狀以聞。上詔文便道引兵捕樓蘭王。將詣闕,簿責王,對曰:「小國在大國間,不兩屬無以自安。願徙國入居漢地。」上直其言,遣歸國,亦因使候司匈奴。匈奴自是不甚親信樓蘭。

征和元年,樓蘭王死,國人來請質子在漢者,欲立之。質子常坐漢法,下蠶室宮刑,故不遣。報曰:「侍子,天子愛之,不能遣。其更立其次當立者。」樓蘭更立王,漢復責其質子,亦遣一子質匈奴。後王又死,匈奴先聞之,遣質子歸,得立為王。漢遣使詔新王,令入朝,天子將加厚賞。樓蘭王後妻,故繼母也,謂王曰:「先王遣兩子質漢皆不還,奈何欲往朝乎?」王用其計,謝使曰:「新立,國未定,願待後年入見天子。」然樓蘭國最在東垂,近漢,當白龍堆,乏水草,常主發導,負水儋糧,送迎漢使,又數為吏卒所寇,懲艾不便與漢通。後復為匈奴反間,數遮殺漢使。其弟尉屠耆降漢,具言狀。

元鳳四年,大將軍霍光白遣平樂監傅介子往刺其王。介子輕將勇敢士,齎金幣,揚言以賜外國為名。既至樓蘭,詐其王欲賜之,王喜,與介子飲,醉,將其王屏語,壯士二人從後刺殺之,貴人左右皆散走。介子告諭以「王負漢罪,天子遣我誅王,當更立王弟尉屠耆在漢者。漢兵方至,毋敢動,自令滅國矣!」介子遂斬王嘗歸首,馳傳詣闕,縣首北闕下。封介子為義陽侯。乃立尉屠耆為王,更名其國為鄯善,為刻印章,賜宮女為夫人,備車騎輜重,丞相率百官送至橫門外,祖而遣之。王自請天子曰:「身在漢久,今歸,單弱,而前王有子在,恐為所殺。國中有伊循城,其地肥美,願漢遣二將屯田積穀,令臣得依其威重。」於是漢遣司馬一人、吏士四十人,田伊循以填撫之。其後更置都尉。伊循官置始此矣。

鄯善當漢道衝,西通且末七百二十里。自且末以往皆種五穀,土地草木,畜產作兵,略與漢同,有異乃記云。

且末國,王治且末城,去長安六千八百二十里。戶二百三十,口千六百一十,勝兵三百二十人。輔國侯、左右將、譯長各一人。西北至都護治所二千二百五十八里,北接尉犁,南至小宛可三日行。有蒲陶諸果。西通精絕二千里。

小宛國,王治扜零城,去長安七千二百一十里。戶百五十,口千五十,勝兵二百人。輔國侯、左右都尉各一人。西北至都護治所二千五百五十八里,東與婼羌接,辟南不當道。

精絕國,王治精絕城,去長安八千八百二十里。戶四百八十,口三千三百六十,勝兵五百人。精絕都尉、左右將、譯長各一人。北至都護治所二千七百二十三里,南至戎盧國四日行,地阨骥,西通扜彌四百六十里。

戎盧國,王治卑品城,去長安八千三百里。戶二百四十,口千六百一十,勝兵三百人。東北至都護治所二千八百五十八里,東與小宛、南與婼羌、西與渠勒接,辟南不當道。

扜彌國,王治扜彌城,去長安九千二百八十里。戶三千三百四十,口二萬四十,勝兵三千五百四十人。輔國侯、左右將、左右都尉、左右騎君各一人,譯長二人。東北至都護治所三千五百五十三里,南與渠勒、東北與龜茲、西北與姑墨接,西通于闐三百九十里。今名寧彌。

渠勒國,王治鞬都城,去長安九千九百五十里。戶三百一十,口二千一百七十,勝兵三百人。東北至都護治所三千八百五十二里,東與戎盧、西與婼羌、北與扜彌接。

于闐國,王治西城,去長安九千六百七十里。戶三千三百,口萬九千三百,勝兵二千四百人。輔國侯、左右將、左右騎君、東西城長、譯長各一人。東北至都護治所三千九百四十七里,南與婼羌接,北與姑墨接。于闐之西,水皆西流,注西海;其東,水東流,注鹽澤,河原出焉。多玉石。西通皮山三百八十里。

皮山國,王治皮山城,去長安萬五十里。戶五百,口三千五百,勝兵五百人。左右將、左右都尉、騎君、譯長各一人。東北至都護治所四千二百九十二里,西南至烏秅國千三百四十里,南與天篤接,北至姑墨千四百五十里,西南當罽賓、烏弋山離道,西北通莎車三百八十里。

烏秅國,王治烏秅城,去長安九千九百五十里。戶四百九十,口二千七百三十三,勝兵七百四十人。東北至都護治所四千八百九十二里,北與子合、蒲犁,西與難兜接。山居,田石間。有白草。累石為室。民接手飲。出小步馬,有驢無牛。其西則有縣度,去陽關五千八百八十八里,去都護治所五千二百里。縣度者,石山也,谿谷不通,以繩索相引而度云。

西夜國,王號子合王,治呼犍谷,去長安萬二百五十里。戶三百五十,口四千,勝兵千人。東北到都護治所五千四十六里,東與皮山、西南與烏秅、北與莎車、西與蒲犁接。蒲犁反依耐、無雷國皆西夜類也。西夜與胡異,其種類羌氐行國,隨畜逐水草往來。而子合土地出玉石。

蒲犁國,王治蒲犁谷,去長安九千五百五十里。戶六百五十,口五千,勝兵二千人。東北至都護治所五千三百九十六里,東至莎車五百四十里,北至疏勒五百五十里,南與西夜子合接,西至無雷五百四十里。侯、都尉各一人。寄田莎車。種俗與子合同。

依耐國,王治去長安萬一百五十里。戶一百二十五,口六百七十,勝兵三百五十人。東北至都護治所二千七百三十里,至莎車五百四十里,至無雷五百四十里,北至疏勒六百五十里,南與子合接,俗相與同。少穀,寄田疏勒、莎車。

無雷國,王治盧城,去長安九千九百五十里。戶千,口七千,勝兵三千人。東北至都護治所二千四百六十五里,南至蒲犁五百四十里,南與烏秅、北與捐毒、西與大月氏接。衣服類烏孫,俗與子合同。

難兜國,王治去長安萬一百五十里。戶五千,口三萬一千,勝兵八千人。東北至都護治所二千八百五十里,西至無雷三百四十里,西南至罽賓三百三十里,南與婼羌、北與休循、西與大月氏接。種五穀、蒲陶諸果。有銀銅鐵,作兵與諸國同,屬罽賓。

罽賓國,王治循鮮城,去長安萬二千二百里。不屬都護。戶口勝兵多,大國也。東北至都護治所六千八百四十里,東至烏秅國二千二百五十里,東北至難兜國九日行,西北與大月氏、西南與烏弋山離接。

昔匈奴破大月氏,大月氏西君大夏,而塞王南君罽賓。塞種分散,往往為數國。自疏勒以西北,休循、捐毒之屬,皆故塞種也。

罽賓地平,溫和,有目宿,雜草奇木,檀、櫰、梓、竹、漆。種五穀、蒲陶諸果,糞治園田。地下溼,生稻,冬食生菜。其民巧,雕文刻鏤,治宮室,織罽,剌文繡,好治食。有金銀銅錫,以為器。市列。以金銀為錢,文為騎馬,幕為人面。出封牛、水牛、象、大狗、沐猴、孔爵、珠璣、珊瑚、虎魄、璧流離。它畜與諸國同。

自武帝始通罽賓,自以絕遠,漢兵不能至,其王烏頭勞數剽殺漢使。烏頭勞死,子代立,遣使奉獻。漢使關都尉文忠送其使。王復欲害忠,忠覺之,乃與容屈王子陰末赴共合謀,攻罽賓,殺其王,立陰末赴為罽賓王,授印綬。後軍候趙德使罽賓。與陰末赴相失,陰末赴鎖琅當德,殺副已下七十餘人,遣使者上書謝。孝元帝以絕域不錄,放其使者於縣度,絕而不通。

成帝時,復遣使獻謝罪,漢欲遣使者報送其使,杜欽說大將軍王鳳曰:「前罽賓王陰末赴本漢所立,後卒畔逆。夫德莫大於有國子民,罪莫大於執殺使者,所以不報恩,不懼誅者,自知絕遠,兵不至也。有求則卑辭,無欲則嬌嫚,終不可懷服。凡中國所以為通厚蠻夷,轺快其求者,為壤比而為寇也。今縣度之阨,非罽賓所能越也。其鄉慕,不足以安西域;雖不附,不能危城郭。前親逆節,惡暴西域,故絕而不通;今悔過來,而無親屬貴人,奉獻者皆行賈賤人,欲通貨市買,以獻為名,故煩使者送至縣度,恐失實見欺。凡遣使送客者,欲為防護寇害也。起皮山南,更不屬漢之國四五,斥候士百餘人,五分夜擊刀斗自守,尚時為所侵盜。驢畜負糧,須諸國稟食,得以自贍。國或貧小不能食,或桀黠不肯給,擁彊漢之節,餒山谷之間,乞饨無所得,離一二旬則人畜棄捐曠野而不反。又歷大頭痛、小頭痛之山,赤土、身熱之阪,令人身熱無色,頭痛嘔吐,驢畜盡然。又有三池、盤石阪,道骥者尺六七寸,長者徑三十里。臨崢嶸不測之深,行者騎步相持,繩索相引,二千餘里乃到縣度。畜隊,未半阬谷盡靡碎;人墮,勢不得相收視。險阻危害,不可勝言。聖王分九州,制五服,務盛內,不求外。今遣使者承至尊之命,送蠻夷之賈,勞吏士之眾,涉危難之路,罷弊所恃以事無用,非久長計也。使者業已受節,可至皮山而還。」於是鳳白從欽言。罽賓實利賞賜賈市,其使數年而壹至云。

烏弋山離國,王去長安萬二千二百里。不屬都護。戶口勝兵,大國也。東北至都護治所六十日行,東與罽賓、北與撲挑、西與犁靬、條支接。

行可百餘日,乃至條支。國臨西海,暑溼,田稻。有大鳥,卵如甕。人眾甚多,往往有小君長,安息役屬之,以為外國。善眩。安息長老傳聞條支有弱水、西王母。亦未嘗見也。自條支乘水西行,可百餘日,近日所入云。

烏弋地暑熱莽平,其草木、畜產、五穀、果菜、食飲、宮室、市列、錢貨、兵器、金珠之屬皆與罽賓同,而有桃拔、師子、犀牛。俗重妄殺。其錢獨文為人頭。幕為騎馬。以金銀飾杖。絕遠,漢使希至。自玉門、陽關出南道,歷鄯善而南行,至烏弋山離,南道極矣。

安息國,王治番兜城,去長安萬一千六百里。不屬都護。北與康居、東與烏弋山離、西與條支接。土地風氣,物類所有,民俗與烏弋、罽賓同。亦以銀為錢,文獨為王面,幕為夫人面。王死輒更鑄錢。有大馬爵。其屬小大數百城,地方數千里,最大國也。臨媯水,商賈車船行旁國。書革,旁行為書記。

武帝始遣使至安息,王令將將二萬騎迎於東界。東界去王都數千里,行比至,過數十城,人民相屬。因發使隨漢使者來觀漢地,以大鳥卵及犁靬眩人獻於漢,天子大說。安息東則大月氏。

大月氏國,治監氏城,去長安萬一千六百里。不屬都護。戶十萬,口四十萬,勝兵十萬人。東至都護治所四千七百四十里,西至安息四十九日行,南與罽賓接。土地風氣,物類所有,民俗錢貨,與安息同。出一封橐駝。

大月氏本行國也,隨畜移徙,與匈奴同俗。控弦十餘萬,故彊輕匈奴。本居敦煌、祁連間,至冒頓單于攻破月氏,而老上單于殺月氏,以其頭為飲器,月氏乃遠去,過大宛,西擊大夏而臣之,都媯水北為王庭。其餘小眾不能去者,保南山羌,號小月氏。

大夏本無大君長,城邑往往置小長,民弱畏戰,故月氏徙來,皆臣畜之,共稟漢使者。有五翕侯:一曰休密翕侯,治和墨城,去都護二千八百四十一里,去陽關七千八百二里;二曰雙靡翕侯,治雙靡城,去都護三千七百四十一里,去陽關七千七百八十二里;三曰貴霜翕侯,治護澡城,去都護五千九百四十里,去陽關七千九百八十二里;四曰肸頓翕侯,治薄茅城,去都護五千九百六十二里,去陽關八千二百二里;五曰高附臓侯,治高附城,去都護六千四十一里,去陽關九千二百八十三里。凡五翕侯,皆屬大月氏。

康居國,王冬治樂越匿地。到卑闐城。去長安萬二千三百里。不屬都護。至越匿地馬行七日,至王夏所居蕃內九千一百四里。戶十二萬,口六十萬,勝兵十二萬人。東至都護治所五千五百五十里。與大月氏同俗。東羈事匈奴。

宣帝時,匈奴乖亂,五單于並爭,漢擁立呼韓邪單于,而郅支單于怨望,殺漢使者,西阻康居。其後都護甘延壽、副校尉陳湯發戊已校尉西域諸國兵至康居,誅滅郅支單于,語在甘延壽、陳湯傳。是歲,元帝建昭三年也。

至成帝時,康居遣子侍漢,貢獻,然自以絕遠,獨驕嫚,不肯與諸國相望。都護郭舜數上言:「本匈奴盛時,非以兼有烏孫、康居故也;及其稱臣妾,非以失二國也。漢雖皆受其質子,然三國內相輸遺,交通如故,亦相候司,見便則發;合不能相親信,離不能相臣役。以今言之,結配烏孫竟未有益,反為中國生事。然烏孫既結在前,今與匈奴俱稱臣,義不可距。而康居驕黠,訖不肯拜使者。都護吏至其國,坐之烏孫諸使下,王及貴人先飲食已,乃飲啗都護吏,故為無所省以夸旁國。以此度之,何故遣子入侍?其欲賈市為好,辭之詐也。匈奴百蠻大國,今事漢甚備,聞康居不拜,且使單于有自下之意,宜歸其侍子,絕勿復使,以章漢家不通無禮之國。敦煌、酒泉小郡及南道八國,給使者往來人馬驢橐駝食,皆苦之。空罷耗所過,送迎驕黠絕遠之國。非至計也。」漢為其新通,重致遠人,終羈縻而未絕。

其康居西北可二千里,有奄蔡國。控弦者十餘萬大。與康居同俗。臨大澤,無崖,蓋北海云。

康居有小王五:一曰蘇筹王,治蘇筹城,去都護五千七百七十六里,去陽關八千二十五里;二曰附墨王,治附墨城,去都護五千七百六十七里,去陽關八千二十五里;三曰窳匿王,治窳匿城,去都護五千二百六十六里,去陽關七千五百二十五里;四曰罽王,治罽城,去都護六千二百九十六里,去陽關八千五百五十五里;五曰奧鞬王,治奧鞬城,去都護六千九百六里,去陽關八千三百五十五里。凡五王,屬康居。

大宛國,王治貴山城,去長安萬二千二百五十里。戶六萬,口三十萬,勝兵六萬人。副王,輔國王各一人。東至都護治所四千三十一里,北至康居卑闐城千五百一十里,西南至大月氏六百九十里。北與康居、南與大月氏接,土地風氣物類民俗與大月氏、安息同。大宛左右以蒲陶為酒,富人藏酒至萬餘石,久者至數十歲不敗。俗耆酒,馬耆目宿。

宛別邑七十餘城,多善馬。馬汗血,言其先天馬子也。

張騫始為武帝言之,上遣使者持千金及金馬,以請宛善馬。宛王以漢絕遠,大兵不能至,愛其寶馬不肯與。漢使妄言,宛遂攻殺漢使,取其財物。於是天子遣貳師將軍李廣利將兵前後十餘萬人伐宛,連四年。宛人斬其王毋寡首,獻馬三千匹,漢軍乃還,語在張騫傳。貳師既斬宛王,更立貴人素遇漢善者名昧蔡為宛王。後歲餘,宛貴人以為昧蔡諂,使我國遇屠,相與兵殺昧蔡,立毋寡弟蟬封為王,遣子入侍,質於漢,漢因使使賂賜鎮撫之。又發數十餘輩,抵宛西諸國求其物,因風諭以代宛之威。宛王蟬封與漢約,歲獻天馬二匹。漢使采蒲陶、目宿種歸。天子以天馬多,又外國使來眾,益種蒲陶、目宿離宮館旁,極望焉。

自宛以西至安息國,雖頗異言,然大同,自相曉知也。其人皆深目,多須敘。善賈市,爭分銖。貴女子;女子所言,丈夫乃決正。其地皆絲漆,不知鑄鐵器。及漢使亡卒降,教鑄作它兵器。得漢黃白金,輒以為器,不用為幣。

自烏孫以西至安息,近匈奴。匈奴嘗困月氏,故匈奴使持單于一信到國,國傳送食,不敢留苦。及至漢使,非出幣物不得食,不市畜不得騎,所以然者,以遠漢,而漢多財物,故必市乃得所欲。及呼韓邪單于朝漢,後咸尊漢矣。

桃槐國,王去長安萬一千八十里。戶七百,口五千,勝兵千人。

休循國,王治鳥飛谷,在蔥嶺西,去長安萬二百一十里。戶三百五十八,口千三十,勝兵四百八十人。東至都護治所三千一百二十一里,至捐毒衍敦谷二百六十里,西北至大宛國九百二十里,西至大月氏千六百一十里。民俗衣服類烏孫,因畜隨水草,本故塞種也。

捐毒國,王治衍敦谷,去長安九千八百六十里。戶三百八十,口千一百,勝兵五百人。東至都護治所二千八百六十一里。至疏勒。南與蔥領屬,無人民。西上蔥領,則休循也。西北至大宛千三十里,北與烏孫接。衣服類烏孫,隨水草,依蔥領,本塞種也。

莎車國,王治莎車城,去長安九千九百五十里。戶二千三百三十九,口萬六千三百七十三,勝兵三千四十九人。輔國侯、左右將、左右騎君、備西夜君各一人,都尉二人,譯長四人。東北至都護治所四千七百四十六里,西至疏勒五百六十里,西南至蒲犁七百四十里。有鐵山,出青玉。

宣帝時,烏孫公主小子萬年,莎車王愛之。莎車王無子死,死時萬年在漢。莎車國人計欲自託於漢,又欲得烏孫心,即上書請萬年為莎車王。漢許之,遣使者奚充國送萬年。萬年初立,暴惡,國人不說。莎車王弟呼屠徵殺萬年,并殺漢使者,自立為王,約諸國背漢。會衛候馮奉世使送大宛客,即以便宜發諸國兵擊殺之,更立它昆弟子為莎車王。還,拜奉世為光祿大夫。是歲,元康元年也。

疏勒國,王治疏勒城,去長安九千三百五十里。戶千五百一十,口萬八千六百四十七,勝兵二千人。疏勒侯、擊胡侯、輔國侯、都尉、左右將、左右騎君、左右譯長各一人。東至都護治所二千二百一十里,南至莎車五百六十里。有市列,西當大月氏、大宛、康居道也。

尉頭國,王治尉頭谷,去長安八千六百五十里。戶三百,口二千三百,勝兵八百人。左右都尉各一人,左右騎君各一人。東至都護治所千四百一十一里,南與疏勒接,山道不通,西至捐毒千三百一十四里,徑道馬行二日。田畜隨水草,衣服類烏孫。

烏孫國,大昆彌治赤谷城,去長安八千九百里。戶十二萬,口六十三萬,勝兵十八萬八千八百人。相,大祿,左右大將二人,侯三人,大將、都尉各一人,大監二人,大吏一人,舍中大吏二人,騎君一人。東至都護治所千七百二十一里,西至康居蕃內地五千里。地莽平。多雨,寒。山多松樠。不田作種樹,隨畜逐水草,與匈奴同俗。國多馬,富人至四五千匹。民剛惡,貪狠無信,多寇盜,最為彊國。故服匈奴,後盛大,取羈屬,不肯往朝會。東與匈奴、西北與康居、西與大宛、南與城郭諸國相接。本塞地也,大月氏西破走塞王,塞王南越縣度,大月氏居其地。後烏孫昆莫擊破大月氏,大月氏徙西臣大夏,而烏孫昆莫居之,故烏孫民有塞種、大月氏種云。

始張騫言烏孫本與大月氏共在敦煌間,今烏孫雖彊大,可厚賂招,令東居故地,妻以公主,與為昆弟,以制匈奴。語在張騫傳。武帝即位,令騫齎金幣往。昆莫見騫如單于禮,騫大慚,謂曰:「

天子致賜,王不拜,則還賜。」昆莫起拜,其它如故。

初,昆莫有十餘子,中子大祿彊,善將,將眾萬餘騎別居。大祿兄太子,太子有子曰岑陬。太子蚤死,謂昆莫曰:「必以岑陬為太子。」昆莫哀許之。大祿怒,乃收其昆弟,將眾畔,謀攻岑陬。昆莫與岑陬萬餘騎,令別居,昆莫亦自有萬餘騎以自備。國分為三,大總羈屬昆莫。騫既致賜,諭指曰:「烏孫能東居故地,則漢遣公主為夫人,結為昆弟,共距匈奴,不足破也。」烏孫遠漢,未知其大小,又近匈奴,服屬日久,其大臣皆不欲徙。昆莫年老國分,不能專制,乃發使送騫,因獻馬數十匹報謝。其使見漢人眾富厚,歸其國,其國後乃益重漢。

匈奴聞其與漢通,怒欲擊之。又漢使烏孫,乃出其南,抵大宛、月氏,相屬不絕。烏孫於是恐,使使獻馬,願得尚漢公主,為昆弟。天子問群臣,議許,曰:「必先內聘,然後遣女。」烏孫以馬千匹聘。漢元封中,遣江都王建女細君為公主,以妻焉。賜乘輿服御物,為備官屬宦官侍御數百人,贈送甚盛。烏孫昆莫以為右夫人。匈奴亦遣女妻昆莫,昆莫以為左夫人。

公主至其國,自治宮室居,歲時一再與昆莫會,置酒飲食,以幣帛賜王左右貴人。昆莫年老,語言不通,公主悲愁,自為作歌曰:「

吾家嫁我兮天一方,遠託異國兮烏孫王。穹廬為室兮旃為牆,以肉為食兮酪為漿。居常土思兮心內傷,願為黃鵠兮歸故鄉。」天子聞而憐之,間歲遣使者持帷帳錦繡給遺焉。

昆莫年老,欲使其孫岑陬尚公主。公主不聽,上書言狀,天子報曰:「從其國俗,欲與烏孫共滅胡。」岑陬遂妻公主。昆莫死,岑陬代立。岑陬者,官號也,名軍須靡。昆莫,王號也,名獵驕靡。後書「昆彌」云。岑陬尚江都公主,生一女少夫。公主死,漢復以楚王戊之孫解憂為公主,妻岑陬。岑陬胡婦子泥靡尚小,岑陬且死,以國與季父大祿子翁歸靡,曰:「泥靡大,以國歸之。」

翁歸靡既立,號肥王,復尚楚主解憂,生三男兩女:長男曰元貴靡;次曰萬年,為莎車王;次曰大樂,為左大將;長女弟史為龜茲王絳賓妻;小女素光為若呼翕侯妻。

昭帝時,公主上書,言「匈奴發騎田車師,車師與匈奴為一,共侵烏孫,唯天子幸救之!」漢養士馬,議欲擊匈奴。會昭帝崩,宣帝初即位,公主及昆彌皆遣使上書,言「匈奴復連發大兵侵擊烏孫,取車延、惡師地,收人民去,使使謂烏孫趣持公主來,欲隔絕漢。昆彌願發國半精兵,自給人馬五萬騎,盡力擊匈奴。唯天子出兵以救公主、昆彌。」漢兵大發十五萬騎,五將軍分道並出。語在匈奴傳。遣校尉常惠使持節護烏孫兵,昆彌自將翕侯以下五萬騎從西方入,至右谷蠡王庭,獲單于父行及嫂、居次、名王、犁汙都尉、千長、騎將以下四萬級,馬牛羊驢橐駝七十餘萬頭,烏孫皆自取所虜獲。還,封惠為長羅侯。是歲,本始三年也。漢遣惠持金幣賜烏孫貴人有功者。

元康二年,烏孫昆彌因惠上書:「願以漢外孫元貴靡為嗣,得令復尚漢公主,結婚重親,畔絕匈奴,願聘馬执各千匹。」詔下公卿議,大鴻臚蕭望之以為「烏孫絕域,變故難保,不可許。」上美烏孫新立大功,又重絕故業,遣使者至烏孫,先迎取聘。昆彌及太子、左右大將、都尉皆遣使,凡三百餘人,入漢迎取少主。上乃以烏孫主解憂弟子相夫為公主,置官屬侍御百餘人,舍上林中,學烏孫言。天子自臨平樂觀,會匈奴使者、外國君長大角抵,設樂而遣之。使長盧侯光祿大夫惠為副,凡持節者四人,送少主至敦煌。未出塞,聞烏孫昆彌翁歸靡死,烏孫貴人共從本約,立岑陬子泥靡代為昆彌,號狂王。惠上書:「願留少主敦煌,惠馳至烏孫責讓不立元貴靡為昆彌,還迎少主。」事下公卿,望之復以為「烏孫持兩端,難約結。前公主在烏孫四十餘年,恩愛不親密,邊竟未得安,此已事之驗也。今少主以元貴靡不立而還,信無負於夷狄,中國之福也。少主不止,繇役將興,其原起此。」天子從之,徵還少主。

狂王復尚楚主解憂,生一男鴟靡,不與主和,又暴惡失眾。漢使衛司馬魏和意、副候任昌送侍子,公主言狂王為烏孫所患苦,易誅也。遂謀置酒會,罷,使士拔劍擊之。劍旁下,狂王傷,上馬馳去。其子細沈瘦會兵圍和意、昌及公主於赤谷城。數月,都護鄭吉發諸國兵救之,乃解去。漢遣中郎將張遵持醫藥治狂王,賜金二十斤,采繒。因收和意、昌係瑣,從尉犁檻車至長安,斬之。車騎將軍長史張翁留驗公主與使者謀殺狂王狀,主不服,叩頭謝,張翁捽主頭罵詈。主上書,翁還,坐死。副使季都別將醫養視狂王,狂王從十餘騎送之。都還,坐知狂王當誅,見便不發,下蠶室。

初,肥王翁歸靡胡婦子烏就屠,狂王傷時驚,與諸翕侯俱去,居北山中,揚言母家匈奴兵來,故眾歸之。後逐襲殺狂王,自立為昆彌。漢遣破羌將軍辛武賢將兵萬五千人至敦煌,遣使者案行表,穿卑鞮侯井以西,欲通渠轉穀,積居廬倉以討之。

初,楚主侍者馮嫽能史書,習事,嘗持漢節為公主使,行賞賜於城郭諸國,敬信之,號曰馮夫人。為烏孫右大將妻,右大將與烏就屠相愛,都護鄭吉使馮夫人說烏就屠,以漢兵方出,必見滅,不如降。烏就屠恐,曰:「願得小號。」宣帝徵馮夫人,自問狀。遣謁者竺次、期門甘廷壽為副,送馮夫人。馮夫人錦車持節,詔

焉烏就屠詣長羅侯赤谷城,立元貴靡為大昆彌,烏就屠為小昆彌,皆賜印綬。破羌將軍不出塞還。後烏就屠不盡歸諸翕侯民眾,漢復遣長羅侯惠將三校屯赤谷,因為分別其人民地界,大昆彌戶六萬餘,小昆彌戶四萬餘,然眾心皆附小昆彌。

元貴靡、鴟靡皆病死,公主上書言年老土思,願得歸骸骨,葬漢地。天子閔而迎之,公主與烏孫男女三人俱來至京師。是歲,甘露三年也。時年且七十,賜以公主田宅奴婢,奉養甚厚,朝見儀比公主。後二歲卒,三孫因留守墳墓云。

元貴靡子星靡代為大昆彌,弱,馮夫人上書,願使烏孫鎮撫星彌。漢遣之,卒百人送烏孫焉。都護韓宣奏,烏孫大吏、大祿、大監皆可以賜金印紫綬,以尊輔大昆彌,漢許之。後都護韓宣復奏,星靡怯弱,可免,更以季父左大將樂代為昆彌,漢不許。後段會宗為都護,招還亡畔,安定之。

星靡死,子雌栗靡代。小昆彌烏就屠死,子拊離代立,為弟日貳所殺。漢遣使者立拊離子安日為小昆彌。日貳亡,阻康居。漢徙己校屯姑墨,欲候便討焉。安日使貴人姑莫匿等三人詐亡從日貳,刺殺之。都護廉褒賜姑莫匿等金人二十斤,繒三百匹。

後安日為降民所殺,漢立其弟末振將代。時大昆彌雌栗靡健,翕侯皆畏服之,告民牧馬畜無使入牧,國中大安和翁歸靡時。小昆靡末振將恐為所并,使貴人烏日領詐降刺殺雌栗靡。漢欲以兵討之而未能,遣中郎將段會宗持金幣與都護圖方略,立雌栗靡季父公主孫伊秩靡為大昆彌。漢沒入小昆彌侍子在京師者。久之,大昆彌翕侯難栖殺末振將,末振將兄安日子安犁莠代為小昆彌。漢恨不自責誅末振將,復使段會宗即斬其太子番丘。還,賜爵關內侯。是歲,元延二年也。

會宗以翕侯難栖殺末振將,雖不指為漢,合於討賊,奏以為堅守都尉。責大祿、大吏、大監以雌栗靡見殺狀,奪金印紫綬,更與銅墨云。末振將弟卑爰疐本共謀殺大昆彌,將眾八萬餘口北附康居,謀欲藉兵兼并兩昆彌。兩昆彌畏之,親倚都護。

哀帝元壽二年,大昆彌伊秩靡與單于並入朝,漢以為榮。至元始中,卑爰疐殺烏日領以自效,漢封為歸義侯。兩昆彌皆弱,卑爰疐侵陵,都護孫建襲殺之。自烏孫分立兩昆彌後,漢用憂勞,且無寧歲。

姑墨國,王治南城,去長安八千一百五十里。戶三千五百,口二萬四千五百,勝兵四千五百人。姑墨侯、輔國侯、都尉、左右將、左右騎君各一人,譯長二人。東至都護治所一千二十一里,南至於闐馬行十五日,北與烏孫接。出銅、鐵、雌黃。東通龜茲六百七十里。王莽時,姑墨王丞殺溫宿王,并其國。

溫宿國,王治溫宿城,去長安八千三百五十里。戶二千二百,口八千四百,勝兵千五百人。輔國侯、左右將、左右都尉、左右騎君、譯長各二人。東至都護治所二千三百八十里,西至尉頭三百里,北至烏孫赤谷六百一十里。土地物類所有與鄯善諸國同。東通姑墨二百七十里。

龜茲國,王治延城,去長安七千四百八十里。戶六千九百七十,口八萬一千三百一十七,勝兵二萬一千七十六人。大都尉丞、輔國侯、安國侯、擊胡侯、卻胡都尉、擊車師都尉、左右將、左右都尉、左右騎君、左右力輔君各一人,東西南北部千長各二人,卻胡君三人,譯長四人。南與精絕、東南與且末、西南與杅彌、北與烏孫、西與姑墨接。能鑄冶,有鉛。東至都護治所烏壘城三百五十里。

烏壘,戶百一十,口千二百,勝兵三百人。城都尉、譯長各一人。與都護同治。其南三百三十里至渠犁。

渠犁,城都尉一人,戶百三十,口千四百八十,勝兵百五十人。東北與尉犁、東南與且末、南與精絕接。西有河,至龜茲五百八十里。

自武帝初通西域,置校尉,屯田渠犁。是時軍旅連出,師行三十二年,海內虛耗。征和中,貳師將軍李廣利以軍降匈奴。上既悔遠征伐,而搜粟都尉桑弘羊與丞相御史奏言:「故輪臺以東捷枝、渠犁皆故國,地廣,饒水草,有溉田五千頃以上,處溫和,田美,可益通溝渠,種五穀,與中國同時孰。其旁國少錐刀,貴黃金采繒,可以易穀食,宜給足不可乏。臣愚以為可遣屯田卒詣故輪臺以東,置校尉三人分護,各舉圖地形,通利溝渠,務使以時益種五穀。張掖、酒泉遣騎假司馬為斥候,屬校尉,事有便宜,因騎置以聞。田一歲,有積穀,募民壯健有累重敢徙者詣田所,就畜積為本業,益墾溉田,稍築列亭,連城而西,以威西國,輔烏孫,為便。臣謹遣徵事臣昌分部行邊,嚴敕太守都尉明餍火,選士馬,謹斥候,蓄茭草。願陛下遣使使西國,以安其意。臣昧死請。」

上乃下詔,深陳既往之悔,曰:「前有司奏,欲益民賦三十助邊用,是重困老弱孤獨也。而今又請遣卒田輪臺。輪臺西於車師千餘里,前開陵侯擊車師時,危須、尉犁、樓蘭六國子弟在京師者皆先歸,發畜食迎漢軍,又自發兵,凡數萬人,王各自將,共圍車師,降其王。諸國兵便罷,力不能復至道上食漢軍。漢軍破城,食至多,然士自載不足以竟師,彊者盡食畜產,羸者道死數千人。朕發酒泉驢橐駝負食,出玉門迎軍。吏卒起張掖,不甚遠,然尚廝留甚眾。曩者,朕之不明,以軍候弘上書言『匈奴縛馬前後足,置城下,馳言「秦人,我饨若馬」』,又漢使者久留不還,故興師遣貳師將軍,欲以為使者威重也。古者卿大夫與謀,參以蓍龜,不吉不行。乃者以縛馬書遍視丞相御史二千石諸大夫郎為文學者,乃至郡屬國都尉成忠、趙破奴等,皆以『虜自縛其馬,不祥甚哉!』或以為『

欲以見彊,夫不足者視人有餘。』易之,卦得大過,爻在九五,匈奴困敗。公車方士、太史治星望氣,及太卜龜蓍,皆以為吉,匈奴必破,時不可再得也。又曰『北伐行將,於釜山必克。』卦諸將,貳師最吉。故朕親發貳師下釜山,詔之必毋深入。今計謀卦兆皆反繆。重合侯

毋虜侯者,言『聞漢軍當來,匈奴使巫埋羊牛所出諸道及水上以詛軍。單于遺天子馬裘,常使巫祝之。縛馬者,詛軍事也。』又卜『漢軍一將不吉』。匈奴常言『漢極大,然不能飢渴,失一狼,走千羊。』乃者貳師敗,軍士死略離散,悲痛常在朕心。今請遠田輪臺,欲起亭隧,是擾勞天下,非所以優民也。今朕不忍聞。大鴻臚等又議,欲募囚徒送匈奴使者,明封侯之賞以報忿,五伯所弗能為也。且匈奴得漢降者,常提掖搜索,問以所聞。今邊塞未正,闌出不禁,障候長吏使卒獵獸,以皮肉為利,卒苦而烽火乏,失亦上集不得,後降者來,若捕生口虜,乃知之。當今務在禁苛暴,止擅賦,力本農,脩馬復令,以補缺,毋乏武備而已。郡國二千石各上進畜馬方略補邊狀,與計對。」由是不復出軍。而封丞相車千秋為富民侯,以明休息,思富養民也。

初,貳師將軍李廣利擊大宛,還過杅彌,杅彌遣太子賴丹為質於龜茲。廣利責龜茲曰:「外國皆臣屬於漢,龜茲何以得受杅彌質?」即將賴丹入至京師。昭帝乃用桑弘羊前議,以杅彌太子賴丹為校尉將軍,田輪臺,輪臺與渠犁地皆相連也。龜茲貴人姑翼謂其王曰:「賴丹本臣屬吾國,今佩漢印綬來,迫吾國而田,必為害。」王即殺賴丹,而上書謝漢,漢未能征。

宣帝時,長羅侯常惠使烏孫還,便宜發諸國兵,合五萬人攻龜茲,責以前殺校尉賴丹。龜茲王謝曰:「乃我先王時為貴人姑翼所誤,我無罪。」執姑翼詣惠,惠斬之。時烏孫公主遣女來至京師學鼓琴,漢遣侍郎樂奉送主女,過龜茲。龜茲前遣人至烏孫求公主女,未還。會女過龜茲,龜茲王留不遣,復使使報公主,主許之。後公主上書,願令女比宗室入朝,而龜茲王絳賓亦愛其夫人,上書言得尚漢外孫為昆弟,願與公主女俱入朝。元康元年,逐來朝賀。王及夫人皆賜印綬。夫人號稱公主,賜以車騎旗鼓,歌吹數十人,綺繡雜繒琦珍凡數千萬。留且一年,厚贈送之。後數來朝賀,樂漢衣服制度,歸其國,治宮室,作徼道周衛,出入傳呼,撞鐘鼓,如漢家儀。外國胡人皆曰:「驢非驢,馬非馬,若龜茲王,所謂执也。」絳賓死,其子丞德自謂漢外孫,成、哀帝時往來尤數,漢遇之亦甚親密。

東通尉犁六百五十里。

尉犁國,王治尉犁城,去長安六千七百五十里。戶千二百,口九千六百,勝兵二千人。尉犁侯、安世侯、左右將、左右都尉、擊胡君各一人,譯長二人。西至都護治所三百里,南與鄯善、且末接。

危須國,王治危須城,去長安七千二百九十里。戶七百,口四千九百,勝兵二千人。擊胡侯、擊胡都尉、左右將、左右都尉、左右騎君、擊胡君、譯長各一人。西至都護治所五百里,至焉耆百里。

焉耆國,王治員渠城,去長安七千三百里。戶四千,口三萬二千一百,勝兵六千人。擊胡侯、卻胡侯、輔國侯、左右將、左右都尉、擊胡左右君、擊車師君、歸義車師君各一人,擊胡都尉、擊胡君各二人,譯長三人。西南至都護治所四百里,南至尉犁百里,北與烏孫接。近海水多魚。

烏貪訾離國,王治于婁谷,去長安萬三百三十里。戶四十一,口二百三十一,勝兵五十七人。輔國侯、左右都尉各一人。東與單桓、南與且彌、西與烏孫接。

卑陸國,王治天山東乾當國,去長安八千六百八十里。戶二百二十七,口千三百八十七,勝兵四百二十二人。輔國侯、左右將、左右都尉、左右譯長各一人。西南至都護治所千二百八十七里。

卑陸後國,王治番渠類谷,去長安八千七百一十里。戶四百六十二,口千一百三十七,勝兵三百五十人。輔國侯、都尉、譯長各一人,將二人。東與郁立師、北與匈奴、西與劫國、南與車師接。

郁立師國,王治內咄谷,去長安八千八百三十里。戶百九十,口千四百四十五,勝兵三百三十一人。輔國侯、左右都尉、譯長各一人。東與車師後城長、西與卑陸、北與匈奴接。

單桓國,王治單桓城,去長安八千八百七十里。戶二十七,口百九十四,勝兵四十五人。輔國侯、將、左右都尉、譯長各一人。

蒲類國,王治天山西疏榆谷,去長安八千三百六十里。戶三百二十五,口二千三十二,勝兵七百九十九人。輔國侯、左右將、左右都尉各一人。西南至都護治所千三百八十七里。

蒲類後國,王去長安八千六百三十里。戶百,口千七十,勝兵三百三十四人。輔國侯、將、左右都尉、譯長各一人。

西且彌國,王治天山東于大谷,去長安八千六百七十里。戶三百三十二,口千九百二十六,勝兵七百三十八人。西且彌侯、左右將、左右騎君各一人。西南至都護治所千四百八十七里。

東且彌國,王治天山東兌虛谷,去長安八千二百五十里。戶百九十一,口千九百四十八,勝兵五百七十二人。東且彌侯、左右都尉各一人。西南至都護治所千五百八十七里。

劫國,王治天山東丹渠谷,去長安八千五百七十里。戶九十九,口五百,勝兵百一十五人。輔國侯、都尉、譯長各一人。西南至都護治所千四百八十七里。

狐胡國,王治車師柳谷,去長安八千二百里。戶五十五,口二百六十四,勝兵四十五人。輔國侯、左右都尉各一人。西至都護治所千一百四十七里,至焉耆七百七十里。

山國,王去長安七千一百七十里。戶四百五十,口五千,勝兵千人。輔國侯、左右將、左右都尉、譯長各一人。西至尉犁二百四十里,西北至焉耆百六十里,西至危須二百六十里,東南與鄯善、且末接。山出鐵,民山居,寄田糴穀於焉耆、危須。

車師前國,王治交河城。河水分流繞城下,故號交河。去長安八千一百五十里。戶七百,口六千五十,勝兵千八百六十五人。輔國侯、安國侯、左右將、都尉、歸漢都尉、車師君、通善君、鄉善君各一人,譯長二人。西南至都護治所千八百七里,至焉耆八百三十五里。

車師後王國,治務塗谷,去長安八千九百五十里。戶五百九十五,口四千七百七十四,勝兵千八百九十人。擊胡侯、左右將、左右都尉、道民君、譯長各一人。西南至都護治所千二百三十七里。

車師都尉國,戶四十,口三百三十三,勝兵八十四人。

車師後城長國,戶百五十四,口九百六十,勝兵二百六十人。

武帝天漢二年,以匈奴降者介和王為開陵侯,將樓蘭國兵始擊車師,匈奴遣右賢王將數萬騎救之,漢兵不利,引去。征和四年,遣重合侯馬通將四萬騎擊匈奴,道過車師北,復遣開陵侯將樓蘭、尉犁、危須凡六國兵別擊車師,勿令得遮重合侯。諸國兵共圍車師,車師王降服,臣屬漢。

昭帝時,匈奴復使四千騎田車師。宣帝即位,遣五將將兵擊匈奴,車師田者驚去,車師復通於漢。匈奴怒,召其太子軍宿,欲以為質。軍宿,焉耆外孫,不欲質匈奴,亡走焉耆。車師王更立子烏貴為太子。及烏貴立為王,與匈奴結婚姻,教匈奴遮漢道通烏孫者。

地節二年,漢遣侍郎鄭吉、校尉司馬憙將免刑罪人田渠犁,積穀,欲以攻車師。至秋收穀,吉、憙發城郭諸國兵萬餘人,自與所將田士千五百人共擊車師,攻交河城,破之。王尚在其北石城中,未得,會軍食盡,吉等且罷兵,歸渠犁田。秋收畢,復發兵攻車師王於石城。王聞漢兵且至,北走匈奴求救,匈奴未為發兵。王來還,與貴人蘇猶議欲降漢,恐不見信。蘇猶教王擊匈奴邊國小蒲類,斬首,略其人民,以降吉。車師旁小金附國隨漢軍後盜車師,車師王復自請擊破金附。

匈奴聞車師降漢,發兵攻車師,吉、憙引兵北逢之,匈奴不敢前。吉、憙即留一候與卒二十人留守王,吉等引兵歸渠犁。車師王恐匈奴兵復至而見殺也,乃輕騎奔烏孫,吉即迎其妻子置渠犁。東奏事,至酒泉,有詔還田渠犁及車師,益積穀以安西國,侵匈奴。吉還,傳送車師王妻子詣長安,賞賜甚厚,每朝會四夷,常尊顯以示之。於是吉始使吏卒三百人別田車師。得降者言,單于大臣皆曰「車師地肥美,近匈奴,使漢得之,多田積穀,必害人國,不可不爭也。」果遣騎來擊田者,吉乃與校尉盡將渠犁田士千五百人往田,匈奴復益遣騎來,漢田卒少不能當,保車師城中。匈奴將即其城下謂吉曰:「

單于必爭此地,不可田也。」圍城數日乃解。後常數千騎往來守車師,吉上書言:「車師去渠犁千餘里,間以河山,北近匈奴,漢兵在渠犁者勢不能相救,願益田卒。」公卿議以為道遠煩費,可且罷車師田者。詔遣長羅侯將張掖、酒泉騎出車師北千餘里,揚威武車師旁。胡騎引去,吉乃得出,歸渠犁,凡三校尉屯田。

車師王之走烏孫也,烏孫留不遣,遣使上書,願留車師王,備國有急,可從西道以擊匈奴。漢許之。於是漢召故車師太子軍宿在焉耆者,立以為王,盡徙車師國民令居渠犁,遂以車師故地與匈奴。車師王得近漢田官,與匈奴絕,亦安樂親漢。後漢使侍郎殷廣德責烏孫,求車師王烏孫貴,將詣闕,賜第與其妻子居。是歲,元康四年也。其後置戊己校尉屯田,居車師故地。

元始中,車師後王國有新道,出五船北,通玉門關,往來差近,戊己校尉徐普欲開以省道里半,避白龍堆之阨。車師後王姑句以道當為拄置,心不便也。地又頗與匈奴南將軍地接,普欲分明其界然後奏之,召姑句使證之,不肯,繫之。姑句數以牛羊賕吏,求出不得。姑句家矛端生火,其妻股紫陬謂姑句曰:「矛端生火,此兵氣也,利以用兵。前車師前王為都護司馬所殺,今久繫必死,不如降匈奴。」即馳突出高昌壁,入匈奴。

又去胡來王唐兜,國比大種赤水羌,數相寇,不勝,告急都護。都護但欽不以時救助,唐兜困急,怨欽,東守玉門關。玉門關不內,即將妻子人民千餘人亡降匈奴。匈奴受之,而遣使上書言狀。是時,新都侯王莽秉政,遣中郎將王昌等使匈奴,告單于西域內屬,不當得受。單于謝罪,執二王以付使者。莽使中郎王萌待西域惡都奴界上逢受。單于遣使送,因請其罪。使者以聞,莽不聽,詔下會西域諸國王,陳軍斬姑句、唐兜以示之。

至莽篡位,建國二年,以廣新公甄豐為右伯,當出西域。車師後王須置離聞之,與其右將股鞮、左將尸泥支謀曰:「聞甄公為西域太伯,當出,故事給使者牛羊穀芻茭,導譯,前五威將過,所給使尚未能備。今太伯復出,國益貧,恐不能稱。」欲亡入匈奴。戊己校尉刀護聞之,召置離驗問,辭服,乃械致都護但欽在所埒婁城。置離人民知其不還,皆哭而送之。至,欽則斬置離。置離兄輔國侯狐蘭支將置離眾二千餘人,驅畜產,舉國亡降匈奴。

是時,莽易單于璽,單于恨怒,遂受狐蘭支降,遣兵與共寇擊車師,殺後城長,傷都護司馬,及狐蘭兵復還入匈奴。時戊己校尉刀護病,遣史陳良屯桓且谷備匈奴寇,史終帶取糧食,司馬丞韓玄領諸壁,右曲候任商領諸壘,相與謀曰:「西域諸國頗背叛,匈奴欲大侵,要死。可殺校尉,將人眾降匈奴。」即將數千騎至校尉府,脅諸亭令燔積薪,分告諸壁曰:「匈奴十萬騎來入,吏士皆持兵,後者斬!」得三百四人,去校尉府數里止,晨火然。校尉開門擊鼓收吏士,良等隨入,遂殺校尉刀護及子男四人、諸昆弟子男,獨遺婦女小兒。止留戊己校尉城,遣人與匈奴南將軍相聞,南將軍以二千騎迎良等。良等盡脅略戊己校尉吏士男女二千餘人入匈奴。單于以良、帶為烏賁都尉。

後三歲,單于死,弟烏絫單于咸立,復與莽和親。莽遣使者多齎金幣賂單于,購求陳良、終帶等。單于盡收四人及手殺刀護者芝音妻子以下二十七人,皆械檻車付使者。到長安,莽皆燒殺之。其後莽復欺詐單于,和親遂絕。匈奴大擊北邊,而西域亦瓦解。焉耆國近匈奴,先叛,殺都護但欽,莽不能討。

天鳳二年,乃遣五威將王駿、西域都護李崇將戊己校尉出西域,諸國皆郊迎,送兵穀。焉耆詐降而聚兵自備。駿等將莎車、龜茲兵七千餘人,分為數部入焉耆,焉耆伏兵要遮駿。及姑墨、尉犁、危須國兵為反間,還共襲擊駿等,皆殺之。唯戊己校尉郭欽別將兵,後至焉耆。焉耆兵未還,欽擊殺其老弱,引兵還。莽封欽為剼胡子。李崇收餘士,還保龜茲。數年莽死,崇遂沒,西域因絕。

最凡國五十。自譯長、城長、君、監、吏、大祿、百長、千長、都尉、且渠、當戶、將、相至侯、王,皆佩漢印綬,凡三百七十六人。而康居、大月氏、安息、罽賓、烏弋之屬,皆以絕遠不在數中,其來貢獻則相與報,不督錄總領也。

贊曰:孝武之世,圖制匈奴,患其兼從西國,結黨南羌,乃表河曲,列西郡,開玉門,通西域,以斷匈奴右臂,隔絕南羌、月氏。單于失援,由是遠遁,而幕南無王庭。

遭值文、景玄默,養民五世,天下殷富,財力有餘,士馬彊盛。故能睹犀布、玳瑁則建珠崖七部,感枸醬、竹杖則開牂柯、越嶲,聞天馬、蒲陶則通大宛、安息。自是之後,明珠、文甲、通犀、翠羽之珍盈於後宮,蒲梢、龍文、魚目、汗血之馬充於黃門,鉅象、師子、猛犬、大雀之群食於外囿。殊方異物,四面而至。於是廣開上林,穿昆明池,營千門萬戶之宮,立神明通天之臺,興造甲乙之帳,落以隨珠和璧,天子負黼依,襲翠被,馮玉几,而處其中。設酒池肉林以饗四夷之客,作巴俞都盧、海中碭極、漫衍魚龍、角抵之戲以觀視之。及賂遺贈送,萬里相奉,師旅之費,不可勝計。至於用度不足,乃榷酒酤,筦鹽鐵,鑄白金,造皮幣,算至車船,租及六畜。民力屈,財用竭,因之以凶年,寇盜並起,道路不通,直指之使始出,衣繡杖斧,斷斬於郡國,然後勝之。是以末年遂棄輪臺之地,而下哀痛之詔,豈非仁聖之所悔哉!且通西域,近有龍堆,遠則蔥嶺,身熱、頭痛、縣度之阨。淮南、杜欽、揚雄之論,皆以為此天地所以界別區域,絕外內也。書曰「西戎即序」,禹既就而序之,非上威服致其貢物也。

西域諸國,各有君長,兵眾分弱,無所統一,雖屬匈奴,不相親附。匈奴能得其馬畜旃罽,而不能統率與之進退。與漢隔絕,道里又遠,得之不為益,棄之不為損。盛德在我,無取於彼。故自建武以來,西域思漢威德,咸樂內屬。唯其小邑鄯善、車師,界迫匈奴,尚為所拘。而其大國莎車、于闐之屬,數遣使置質于漢,願請屬都護。聖上遠覽古今,因時之宜,羇縻不絕,辭而未許。雖大禹之序西戎,周公之讓白雉,太宗之卻走馬,義兼之矣,亦何以尚茲!


\end{pinyinscope}