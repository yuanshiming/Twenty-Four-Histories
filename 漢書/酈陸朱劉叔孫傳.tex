\article{酈陸朱劉叔孫傳}

\begin{pinyinscope}
酈食其,陳留高陽人也。好讀書,家貧落魄,無衣食業。為里監門,然吏縣中賢豪不敢役,皆謂之狂生。

及陳勝、項梁等起,諸將徇地過高陽者數十人,食其聞其將皆握齱好荷禮自用,不能聽大度之言,食其乃自匿。後聞沛公略地陳留郊,沛公麾下騎士適食其里中子,沛公時時問邑中賢豪。騎士歸,食其見,謂曰:「吾聞沛公嫚易人,有大略,此真吾所願從游,莫為我先。若見沛公,謂曰『臣里中有酈生,年六十餘,長八尺,人皆謂之狂生』,自謂我非狂。」騎士曰:「

沛公不喜儒,諸客冠儒冠來者,沛公輒解其冠,溺其中。與人言,常大罵。未可以儒生說也。」食其曰:「第言之。」騎士從容言食其所戒者。

沛公至高陽傳舍,使人召食其。食其至,入謁,沛公方踞床令兩女子洗,而見食其。食其入,即長揖不拜,曰:「足下欲助秦攻諸侯乎?欲率諸侯攻秦乎?」沛公罵曰:「豎儒!夫天下同苦秦久矣,故諸侯相率攻秦,何謂助秦?」食其曰:「必欲聚徒合義兵誅無道秦,不宜踞見長者。」於是沛公輟洗,起衣,延食其上坐,謝之。食其因言六國從衡時。沛公喜,賜食其食,問曰:「計安出?」食其曰:「足下起瓦合之卒,收散亂之兵,不滿萬人,欲以徑入彊秦,此所謂探虎口者也。夫陳留,天下之衝,四通五達之郊也,今其城中又多積粟。臣知其令,今請使,令下足下。即不聽,足下舉兵攻之,臣為內應。」於是遣食其往,沛公引隨之,遂下陳留。號食其為廣野君。

食其言弟商,使將數千人從沛公西南略地。食其嘗為說客,馳使諸侯。

漢三年秋,項羽擊漢,拔滎陽,漢兵遁保鞏。楚人聞韓信破趙,彭越數反梁地,則分兵救之。韓信方東擊齊,漢王數困滎陽、成皋,計欲捐成皋以東,屯鞏、雒以距楚。食其因曰:「臣聞之,知天之天者,王事可成;不知天之天者,王事不可成。王者以民為天,而民以食為天。夫敖倉,天下轉輸久矣,臣聞其下乃有臧粟甚多。楚人拔滎陽,不堅守敖倉,乃引而東,令適卒分守成皋,此乃天所以資漢。方今楚易取而漢反卻,自奪便,臣竊以為過矣。且兩雄不俱立,楚漢久相持不決,百姓騷動,海內搖蕩,農夫釋耒,紅女下機,天下之心未有所定也。願足下急復進兵,收取滎陽,據敖庾之粟,塞成皋之險,杜太行之道,距飛狐之口,守白馬之津,以示諸侯形制之勢,則天下知所歸矣。方今燕、趙已定,唯齊未下。今田廣據千里之齊,田間將二十萬之眾軍於歷城,諸田宗彊,負海岱,阻河濟,南近楚,齊人多變詐,足下雖遣數十萬師,未可以歲月破也。臣請得奉明詔說齊王使為漢而稱東藩。」上曰:「善。」

乃從其畫,復守敖倉,而使食其說齊王,曰:「王知天下之所歸乎?」曰:「不知也。」曰:「知天下之所歸,則齊國可得而有也;若不知天下之所歸,即齊國未可保也。」齊王曰:「天下何歸?」食其曰:「天下歸漢。」齊王曰:「先生何以言之?」曰:「漢王與項王戮力西面擊秦,約先入咸陽者王之,項王背約不與,而王之漢中。項王遷殺義帝,漢王起蜀漢之兵擊三秦,出關而責義帝之負處,收天下之兵,立諸侯之後。降城即以侯其將,得賂則以分其士,與天下同其利,豪英賢材皆樂為之用。諸侯之兵四面而至,蜀漢之粟方船而下。項王有背約之名,殺義帝之負;於人之功無所記,於人之罪無所忘;戰勝而不得其賞,拔城而不得其封;非項氏莫得用事;為人刻印,玩而不能授;攻城得賂,積財而不能賞。天下畔之,賢材怨之,而莫為之用。故天下之士歸於漢王,可坐而策也。夫漢王發蜀漢,定三秦;涉西河之外,援上黨之兵;下井陘,誅成安君;破北魏,舉三十二城:此黃帝之兵,非人之力,天之福也。今已據敖庾之粟,塞成皋之險,守白馬之津,杜太行之阨,距飛狐之口,天下後服者先亡矣。王疾下漢王,齊國杜稷可得而保也;不下漢王,危亡可立而待也。」田廣以為然,乃聽食其,罷歷下兵守戰備,與食其日縱酒。

韓信聞食其馮軾下齊七十餘城,乃夜度兵平原襲齊。齊王田廣聞漢兵至,以為食其賣己,乃亨食其,引兵走。

漢十二年,曲周侯酈商以丞相將兵擊黥布,有功。高祖舉功臣,思食其。食其子疥數將兵,上以其父故,封疥為高梁侯。後更食武陽,卒,子遂嗣。三世,侯平有罪,國除。

陸賈,楚人也。以客從高祖定天下,名有口辯,居左右,常使諸侯。

時中國初定,尉佗平南越,因王之。高祖使賈賜佗印為南越王。賈至,尉佗魋結箕踞見賈。賈因說佗曰:「足下中國人,親戚昆弟墳墓在真定。今足下反天性,棄冠帶,欲以區區之越與天子伉衡為敵國,禍且及身矣。夫秦失其正,諸侯豪桀並起,唯漢王先入關,據咸陽。項籍背約,自立為西楚霸王,諸侯皆屬,可謂至彊矣。然漢王起巴蜀,鞭笞天下,劫諸侯,遂誅項羽。五年之間,海內平定,此非人力,天之所建也。天子聞君王王南越,而不助天下誅暴逆,將相欲移兵而誅王,天子憐百姓新勞苦,且休之,遣臣授君王印,剖符通使。君王宜郊迎,北面稱臣,乃欲以新造未集之越屈強於此。漢誠聞之,掘燒君王先人家墓,夷種宗族,使一偏將將十萬眾臨越,即越殺王降漢,如反覆手耳。」

於是佗乃蹶然起坐,謝賈曰:「居蠻夷中久,殊失禮義。」因問賈曰:「我孰與蕭何、曹參、韓信賢?」賈曰:「王似賢也。」復問曰:「我孰與皇帝賢?」賈曰:「皇帝起豐沛,討暴秦,誅彊楚,為天下興利除害,繼五帝三王之業,統天下,理中國。中國之人以億計,地方萬里,居天下之膏腴,人眾車輿,萬物殷富,政由一家,自天地剖判未始有也。今王眾不過數萬,皆蠻夷,崎嶇山海間,譬如漢一郡,王何乃比於漢!」佗大笑曰:「吾不起中國,故王此。使我居中國,何遽不若漢?」乃大說賈,留與飲數月。曰:「越中無足與語,至生來,令我日聞所不聞。」賜賈橐中裝直千金,它送亦千金。賈卒拜佗為南越王,令稱臣奉漢約。歸報,高帝大說,拜賈為太中大夫。

賈時時前說稱詩書。高帝罵之曰:「乃公居馬上得之,安事詩書!」賈曰:「馬上得之,寧可以馬上治乎?且湯武逆取而以順守之,文武並用,長久之術也。昔者吳王夫差、智伯極武而亡;秦任刑法不變,卒滅趙氏。鄉使秦以并天下,行仁義,法先聖,陛下安得而有之?」高帝不懌,有慚色,謂賈曰:「試為我著秦所以失天下,吾所以得之者,及古成敗之國。」賈凡著十二篇。每奏一篇,高帝未嘗不稱善,左右呼萬歲,稱其書曰新語。

孝惠時,呂太后用事,欲王諸呂,畏大臣及有口者。賈自度不能爭之,乃病免。以好畤田地善,往家焉。有五男,乃出所使越橐中裝,賣千金,分其子,子二百金,令為生產。賈常乘安車駟馬,從歌鼓瑟侍者十人,寶劍直百金,謂其子曰:「與女約:過女,女給人馬酒食極飲,十日而更。所死家,得寶劍車騎侍從者。一歲中以往來過它客,率不過再過,數擊鮮,毋久溷女為也。」

呂太后時,王諸呂,諸呂擅權,欲劫少主,危劉氏。右丞相陳平患之,力不能爭,恐禍及己。平嘗燕居深念。賈往,不請,直入坐,陳平方念,不見賈。賈曰:「何念深也?」平曰:「生揣我何念?」賈曰:「足下位為上相,食三萬戶侯,可謂極富貴無欲矣。然有憂念,不過患諸呂、少主耳。」陳平曰:「然。為之奈何?」賈曰:「天下安,注意相;天下危,注意將。將相和,則士豫附;士豫附,天下雖有變,則權不分。權不分,為社稷計,在兩君掌握耳。臣常欲謂太尉絳侯,絳侯與我戲,易吾言。君何不交驩太尉,深相結?」為陳平畫呂氏數事。平用其計,乃以五百金為絳侯壽,厚具樂飲太尉,太尉亦報如之。兩人深相結,呂氏謀益壞。陳平乃以奴婢百人,車馬五十乘,錢五百萬,遺賈為食飲費。賈以此游漢廷公卿間,名聲籍甚。及誅呂氏,立孝文,賈頗有力。

孝文即位,欲使人之南越,丞相平乃言賈為太中大夫,往使尉佗,去黃屋稱制,令比諸侯,皆如意指。語在南越傳。陸生竟以壽終。

朱建,楚人也。故嘗為淮南王黥布相,有罪去,後復事布。布欲反時,問建,建諫止之。布不聽,聽梁父侯,遂反。漢既誅布,聞建諫之,高祖賜建號平原君,家徙長安。

為人辯有口,刻廉剛直,行不苟合,義不取容。辟陽侯行不正,得幸呂太后,欲知建,建不肯見。及建母死,貧未有以發喪,方假貣服具。陸賈素與建善,乃見辟陽侯,賀曰:「平原君母死。」,何乃賀我?」陸生曰:「前日君侯欲知平原君,平原君義不知君,以其母故。今其母死,君誠厚送喪,則彼為君死矣。」辟陽侯乃奉百金裞,列侯貴人以辟陽侯故,往賻凡五百金。

久之,人或毀辟陽侯,惠帝大怒,下吏,欲誅之。太后慚,不可言。大臣多害辟陽侯行,欲遂誅之。辟陽侯困急,使人欲見建。建辭曰:「獄急,不敢見君。」建乃求見孝惠幸臣閎籍孺,說曰:「君所以得幸帝,天下莫不聞。今辟陽侯幸太后而下吏,道路皆言君讒,欲殺之。今日辟陽侯誅,旦日太后含怒,亦誅君。君何不肉袒為辟陽侯言帝?帝聽君出辟陽侯,太后大驩。兩主俱幸君,君富貴益倍矣。」於是閎籍孺大恐,從其計,言帝,帝果出辟陽侯。辟陽侯之囚,欲見建,建不見,辟陽侯以為背之,大怒。及其成功出之,大驚。

呂太后崩,大臣誅諸呂,辟陽侯與諸呂至深,卒不誅。計畫所以全者,皆陸生、平原君之力也。

孝文時,淮南厲王殺辟陽侯,以黨諸呂故。孝文聞其客朱建為其策,使吏捕欲治。聞吏至門,建欲自殺。諸子及吏皆曰:「事未可知,何自殺為?」建曰:「我死禍絕,不及乃身矣。」遂自剄。文帝聞而惜之,曰:「吾無殺建意也。」乃召其子,拜為中大夫。使匈奴,單于無禮,罵單于,遂死匈奴中。

婁敬,齊人也。漢五年,戍隴西,過雒陽,高帝在焉。敬脫輓輅,見齊人虞將軍曰:「臣願見上言便宜。」虞將軍欲與鮮衣,敬曰:「臣衣帛,衣帛見,衣褐,衣褐見,不敢易衣。」虞將軍入言上,上召見,賜食。

已而問敬,敬說曰:「陛下都雒陽,豈欲與周室比隆哉?」上曰:「然。」敬曰:「陛下取天下與周異。周之先自后稷,堯封之邰,積德絫善十餘世。公劉避桀居豳。大王以狄伐故,去豳,杖馬箠去居岐,國人爭歸之。及文王為西伯,斷虞芮訟,始受命,呂望、伯夷自海濱來歸之。武王伐紂,不期而會孟津上八百諸侯,遂滅殷。成王即位,周公之屬傅相焉,乃營成周都雒,以為此天下中,諸侯四方納貢職,道里鈞矣,有德則易以王,無德則易以亡。凡居此者,欲令務以德致人,不欲阻險,令後世驕奢以虐民也。及周之衰,分而為二,天下莫朝周,周不能制。非德薄,形勢弱也。今陛下起豐沛,收卒三千人,以之徑往,卷蜀漢,定三秦,與項籍戰滎陽,大戰七十,小戰四十,使天下之民肝腦塗地,父子暴骸中野,不可勝數,哭泣之聲不絕,傷夷者未起,而欲比隆成康之時,臣竊以為不侔矣。且夫秦地被山帶河,四塞以為固,卒然有急,百萬之眾可具。因秦之故,資甚美膏腴之地,此所謂天府。陛下入關而都之,山東雖亂,秦故地可全而有也。夫與人鬥,不搤其亢,拊其背,未能全勝。今陛下入關而都,按秦之故,此亦搤天下之亢而拊其背也。」

高帝問群臣,群臣皆山東人,爭言周王數百年,秦二世則亡,不如都周。上疑未能決。及留侯明言入關便,即日駕西都關中。

於是上曰:「本言都秦地者婁敬,婁者劉也。」賜姓劉氏,拜為郎中,號曰奉春君。

漢七年,韓王信反,高帝自往擊。至晉陽,聞信與匈欲擊漢,上大怒,使人使匈奴。匈奴匿其壯士肥牛馬,徒見其老弱及羸畜。使者十輩來,皆言匈奴易擊。上使劉敬復往使匈奴,還報曰:「兩國相擊,此宜夸矜見所長。今臣往,徒見羸胔老弱,此必欲見短,伏奇兵以爭利。愚以為匈奴不可擊也。」是時漢兵以踰句注,三十餘萬眾,兵已業行。上怒,罵敬曰:「齊虜!以舌得官,乃今妄言沮吾軍。」械繫敬廣武。遂往,至平城,匈奴果出奇兵圍高帝白登,七日然後得解。高帝至廣武,赦敬,曰:「吾不用公言,以困平城。吾已斬先使十輩言可擊者矣。」乃封敬二千戶,為關內侯,號建信侯。

高帝罷平城歸,韓王信亡入胡。當是時,冒頓單于兵彊,控弦四十萬騎,數苦北邊。上患之,問敬。敬曰:「天下初定,士卒罷於兵革,未可以武服也。冒頓殺父代立,妻群母,以力為威,未可以仁義說也。獨可以計久遠子孫為臣耳,然陛下恐不能為。」上曰:「誠可,何為不能!顧為奈何?」敬曰:「陛下誠能以適長公主妻單于,厚奉遺之,彼知漢女送厚,蠻夷必慕,以為閼氏,生子必為太子,代單于。何者?貪漢重幣。陛下以歲時漢所餘彼所鮮數問遺,使辯士風諭以禮節。冒頓在,固為子婿;死,外孫為單于。豈曾聞外孫敢與大父亢禮哉?可毋戰以漸臣也。若陛下不能遣長公主,而今宗室及後宮詐稱公主,彼亦知不肯貴近,無益也。」高帝曰:「善。」欲遣長公主。呂后泣曰:「妾唯以一太子、一女,奈何棄之匈奴!」上竟不能遣長公主,而取家人子為公主,妻單于。使敬往結和親約。

敬從匈奴來,因言「匈奴河南白羊、樓煩王,去長安近者七百里,輕騎一日一夕可以至。秦中新破,少民,地肥饒,可益實。夫諸侯初起時,非齊諸田,楚昭、屈、景莫與。今陛下雖都關中,實少人。北近胡寇,東有六國彊族,一日有變,陛下亦未得安枕而臥也。臣願陛下徙齊諸田,楚昭、屈、景,燕、趙、韓、魏後,及豪傑名家,且實關中。無事,可以備胡;諸侯有變,亦足率以東伐。此彊本弱末之術也。」上曰:「善。」乃使劉敬徙所言關中十餘萬口。

叔孫通,薛人也。秦時以文學徵,待詔博士。數歲,陳勝起,二世召博士諸儒生問曰:「楚戍卒攻蘄入陳,於公何如?」博士諸生三十餘人前曰:「人臣無將,將則反,罪死無赦。願陛下急發兵擊之。」二世怒,作色。通前曰:「諸生言皆非。夫天下為一家,毀郡縣城,鑠其兵,視天下弗復用。且明主在上,法令具於下,吏人人奉職,四方輻輳,安有反者!此特群盜鼠竊狗盜,何足置齒牙間哉?郡守尉令捕誅,何足憂?」二世喜,盡問諸生,諸生或言反,或言盜。於是二世令御史按諸生言反者下吏,非所宜言。諸生言盜者皆罷之。乃賜通帛二十疋,衣一襲,拜為博士。通已出,反舍,諸生曰:「生何言之諛也?」通曰:「公不知,我幾不免虎口!」乃亡去之薛,薛已降楚矣。

及項梁之薛,通從之。敗定陶,從懷王。懷王為義帝,徙長沙,通留事項王。漢二年,漢王從五諸侯入彭城,通降漢王。

通儒服,漢王憎之,乃變其服,服短衣,楚製。漢王喜。

通之降漢,從弟子百餘人,然無所進,剸言諸故群盜壯士進之。弟子皆曰:「事先生數年,幸得從降漢,今不進臣等,剸言大猾,何也?」通乃謂曰:「漢王方蒙矢石爭天下,諸生寧能鬥乎?故先言斬將搴旗之士。諸生且待我,我不忘矣。」漢王拜通為博士,號稷嗣君。

漢王已并天下,諸侯共尊為皇帝於定陶,通就其儀號。高帝悉去秦儀法,為簡易。群臣飲爭功,醉或妄呼,拔劍擊柱,上患之。通知上益饜之,說上曰:「夫儒者難與進取,可與守成。臣願徵魯諸生,與臣弟子共起朝儀。」高帝曰:「得無難乎?」通曰:「五帝異樂,三王不同禮。禮者,因時世人情為之節文者也。故夏、殷、周禮所因損益可知者,謂不相復也。臣願頗采古禮與秦儀雜就之。」上曰:「可試為之,令易知,度吾所能行為之。」

於是通使徵魯諸生三十餘人。魯有兩生不肯行,曰:「公所事者且十主,皆面諛親貴。今天下初定,死者未葬,傷者未起,又欲起禮樂。禮樂所由起,百年積德而後可興也。吾不忍為公所為。公所為不合古,吾不行。公往矣,毋污我!」通笑曰:「若真鄙儒,不知時變。」

遂與所徵三十人西,及上左右為學者與其弟子百餘人為綿蕞野外。習之月餘,通曰:「上可試觀。」上使行禮,曰:「吾能為此。」乃令群臣習肄,會十月。

漢七年,長樂宮成,諸侯群臣朝十月。儀:先平明,謁者治禮,引以次入殿門,廷中陳車騎戍卒衛官,設兵,張旗志。傳曰「趨」。殿下郎中俠陛,陛數百人。功臣列侯諸將軍軍吏以次陳西方,東鄉;文官丞相以下陳東方,西鄉。大行設九賓,臚句傳。於是皇帝輦出房,百官執戟傳警,引諸侯王以下至吏六百石以次奉賀。自諸侯王以下莫不震恐肅敬。至禮畢,盡伏,置法酒。諸侍坐殿上皆伏抑首,以尊卑次起上壽。觴九行,謁者言「罷酒」。御史執法舉不如儀者輒引去。竟朝置酒,無敢讙譁失禮者。於是高帝曰:「吾乃今日知為皇帝之貴也。」拜通為奉常,賜金五百斤。

通因進曰:「諸弟子儒生隨臣久矣,與共為儀,願陛下官之。」高帝悉以為郎。通出,皆以五百金賜諸生。諸生乃喜曰:「叔孫生聖人,知當世務。」

九年,高帝徙通為太子太傅。十二年,高帝欲以趙王如意易太子,通諫曰:「昔者晉獻公以驪姬故,廢太子,立奚齊,晉國亂者數十年,為天下笑。秦以不早定扶蘇,胡亥詐立,自使滅祀,此陛下所親見。今太子仁孝,天下皆聞之;呂后與陛下共苦食啖,其可背哉!陛下必欲廢適而立少,臣願先伏誅,以頸血汙地。」高帝曰:「公罷矣,吾特戲耳。」通曰:「太子天下本,本壹搖天下震動,奈何以天下戲!」高帝曰:「吾聽公。」及上置酒,見留侯所招客從太子入見,上遂無易太子志矣。

高帝崩,孝惠即位,乃謂通曰:「先帝園陵寢廟,群臣莫習。」徙通為奉常,定宗廟儀法。及稍定漢諸儀法,皆通所論著也。惠帝為東朝長樂宮,及間往,數蹕煩民,作復道,方築武庫南,通奏事,因請間,曰:「陛下何自築復道高帝寢,衣冠月出游高廟?子孫奈何乘宗廟道以行哉!」惠帝懼,曰:「急壞之。」通曰:「人主無過舉。今已作,百姓皆知之矣。願陛下為原廟渭北,衣冠月出游之,益廣宗廟,大孝之本。」上乃詔有司立原廟。

惠帝常出游離宮,通曰:「古者有春嘗协,方今櫻桃孰,可獻,願陛下出,因取櫻桃獻宗廟。」上許之。諸协獻由此興。

贊曰:高祖以征伐定天下,而縉紳之徒騁其知辯,並成大業。語曰「廊廟之材非一木之枝,帝王之功非一士之略」,信哉!劉敬脫輓輅而建金城之安,叔孫通舍枹鼓而立一王之儀,遇其時也。酈生自匿監門,待主然後出,猶不免鼎鑊。朱建始名廉直,既距辟陽,不終其節,亦以喪身。陸賈位止大夫,致仕諸呂,不受憂責,從容平、勃之間,附會將相以彊社稷,身名俱榮,其最優乎!


\end{pinyinscope}