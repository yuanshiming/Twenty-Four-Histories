\article{樊酈滕灌傅靳周傳}

\begin{pinyinscope}
樊噲,沛人也,以屠狗為事。後與高祖俱隱於芒碭山澤間。

陳勝初起,蕭何、曹參使噲求迎高祖,立為沛公。噲以舍人從攻胡陵、方與,還守豐,擊泗水監豐下,破之。復東定沛,破泗水守薛西。與司馬镛戰碭東,卻敵,斬首十五級,賜爵國大夫。常從,沛公擊章邯軍濮陽,攻城先登,斬首二十三級,賜爵列大夫。從攻陽城,先登。下戶牖,破李由軍,斬首十六級,賜上聞爵。後攻圉都尉、東郡守尉於成武,卻敵,斬首十四級,捕虜十六人,賜爵五大夫。從攻秦軍,出亳南。河間守軍於杠里,破之。擊破趙賁軍開封北,以卻敵先登,斬候一人,首六十八級,捕虜二十六人,賜爵卿。從攻破揚熊於曲遇。攻宛陵,先登,斬首八級,捕虜四十四人,賜爵封號賢成君。從攻長社、轘轅,絕河津,東攻秦軍尸鄉,南攻秦軍於犨。破南陽守齮於陽城。東攻宛城,先登。西至酈,以卻敵,斬首十四級,捕虜四十四人,賜重封。攻武關,至霸上,斬都尉一人,首十級,捕虜百四十六人,降卒二千九百人。

項羽在戲下,欲攻沛公。沛公從百餘騎因項伯面見項羽,謝無有閉關事。項羽既饗軍土,中酒,亞父謀欲殺沛公,令項莊拔劍舞坐中,欲擊沛公,項伯常屏蔽之。時獨沛公與張良得入坐,樊噲居營外,聞事急,乃持盾入。初入營,營衛止噲,噲直撞入,立帳下。項羽目之,問為誰。張良曰:「沛公參乘樊噲也。」項羽曰:「壯士。」賜之卮酒彘肩。噲既飲酒,拔劍切肉食之。項羽曰:「能復飲乎?」噲曰:「臣死且不辭,豈特卮酒乎!且沛公先入定咸陽,暴師霸上,以待大王。大王今日至,聽小人之言,與沛公有隙,臣恐天下解心疑大王也。」項羽默然。沛公如廁,麾噲去。既出,沛公留車騎,獨騎馬,噲等四人步從,從山下走歸霸上軍,而使張良謝項羽。羽亦因遂已,無誅沛公之心。是日微樊噲奔入營譙讓項羽,沛公幾殆。

後數日,項羽入屠咸陽,立沛公為漢王。漢王賜噲爵為列侯,號臨武侯。遷為郎中,從入漢中。

還定三秦,別擊西丞白水北,擁輕車騎雍南,破之。從攻雍、焘城,先登。擊章平軍好畤,攻城,先登陷陣,斬縣令丞各一人,首十一級,虜二十人,遷為郎中騎將。從擊秦車騎壤東,卻敵,遷為將軍。攻趙賁,下郿、槐里、柳中、咸陽;灌廢丘,最。至櫟陽,賜食邑杜之樊鄉。從攻項籍,屠煮棗,擊破王武、程處軍於外黃。攻鄒、魯、瑕丘、薛。項羽敗漢王於彭城,盡復取魯、梁地。噲還至滎陽,益食平陰二千戶,以將軍守廣武一歲。項羽引東,從高祖擊項籍,下陽夏,虜楚周將軍卒四千人。圍項籍陳,大破之。屠胡陵。

項籍死,漢王即皇帝位,以噲有功,益食邑八百戶。其秋,燕王臧荼反,噲從攻虜荼,定燕地。楚王韓信反,噲從至陳,取信,定楚。更賜爵列侯,與剖符,世世勿絕,食舞陽,號為舞陽侯,除前所食。以將軍從攻反者韓王信於代。自霍人以往至雲中,與絳侯等共定之,益食千五百戶。因擊陳狶與曼丘臣軍,戰襄國,破柏人,先登,降之定清河、常山凡二十七縣,殘東垣,遷為左丞相。破得綦母印、尹潘軍於無終、廣昌。破豨別將胡人王黃軍代南,因擊韓信軍參合。軍所將卒斬韓信,擊豨胡騎橫谷,斬將軍趙既,虜代丞相馮梁、守孫奮、大將王黃、將軍大將一人、太僕解福等十人。與諸將共定代鄉邑七十三。後燕王盧綰反,噲以相國擊綰,破其丞相抵薊南,定燕縣十八,鄉邑五十一。益食千三百戶,定食舞陽五千四百戶。從,斬首百七十六級,虜二百八十七人。別,破軍七,下城五,定郡六,縣五十二,得丞相一人,將軍十三人,二千石以下至三百石十二人。

噲以呂后弟呂須為婦,生子伉,故其比諸將最親。先黥布反時,高帝嘗病,惡見人,臥禁中,詔戶者無得入群臣。群臣絳、灌等莫敢入。十餘日,噲乃排闥直入,大臣隨之。上獨枕一宦者臥。噲等見上流涕曰:「始陛下與臣等起豐沛,定天下,何其壯也!今天下已定,又何憊也!且陛下病甚,大臣震恐,不見臣等計事,顧獨與一宦者絕乎?且陛下獨不見趙高之事乎?」高帝笑而起。

其後盧綰反,高帝使噲以相國擊燕。是時高帝病甚,人有惡噲黨於呂氏,即上一日宮車晏駕,則噲欲以兵盡誅戚氏、趙王如意之屬。高帝大怒,乃使陳平載絳侯代將,而即軍中斬噲。陳平畏呂后,執噲詣長安。至則高帝已崩,呂后釋噲,得復爵邑。

孝惠六年,噲薨,諡曰武侯,子伉嗣。而伉母呂須亦為臨光侯,噲高后時用事顓權,大臣盡畏之。高后崩,大臣誅呂須等,因誅伉,舞陽侯中絕數月。孝文帝立,乃復封噲庶子市人為侯,復故邑。薨,諡曰荒侯。子佗廣嗣。六歲,其舍人上書言:「荒侯巿人病不能為人,令其夫人與其弟亂而生佗廣,佗廣實非荒侯子。」下吏,免。平帝元始二年,繼絕世,封噲玄孫之子章為舞陽侯,邑千戶。

酈商,高陽人也。陳勝起,商聚少年得數千人。沛公略地六月餘,商以所將四千人屬沛公於岐。從攻長社,先登,賜爵封信成君。從攻緱氏,絕河津,破秦軍雒陽東。從下宛、穰,定十七縣。別將攻旬關,西定漢中。

沛公為漢王,賜商爵信成君,以將軍為隴西都尉。別定北地郡,破章邯別將於烏氏、栒邑、泥陽,賜食邑武城六千戶。從擊項籍軍,與鍾離眛戰,受梁相國印,益食四千戶。從擊項羽二歲,攻胡陵。

漢王即帝位,燕王臧荼反,商以將軍從擊荼,戰龍脫,先登陷陣,破荼軍易下,卻敵,遷為右丞相,賜爵列侯,與剖符,世世勿絕,食邑涿郡五千戶。別定上谷,因攻代,受趙相國印。與絳侯等定代郡、鴈門,得代丞相程縱、守相郭同、將軍以下至六百石十九人。還,以將軍將太上皇衛一歲。十月,以右丞相擊陳豨,殘東垣。又從擊黥布,攻其前垣,陷兩陳,得以破布軍,更封為曲周侯,食邑五千一百戶,除前所食。凡別破軍三,降定郡六,縣七十三,得丞相、守相、大將軍各一人,小將軍二人,二千石以下至六百石十九人。

商事孝惠帝、呂后。呂后崩,商疾不治事。其子寄,子況,與呂祿善。及高后崩,大臣欲誅諸呂,呂祿為將軍,軍於北軍,太尉勃不得入北軍,於是乃使人劫商,令其子寄紿呂祿。呂祿信之,與出游,而太尉勃乃得入據北軍,遂以誅諸呂。商是歲薨,諡曰景侯。子寄嗣。天下稱酈況賣友。

孝景時,吳、楚、齊、趙反,上以寄為將軍,圍趙城,七月不能下。欒布自平齊來,乃滅趙。孝景中二年,寄欲取平原君姊為夫人,景帝怒,下寄吏,免。上乃封商它子堅為繆侯,奉商後。傳至玄孫終根,武帝時為太常,坐巫蠱誅,國除。元始中,賜高祖時功臣自酈商以下子孫爵乎關內侯,食邑凡百餘人。

夏侯嬰,沛人也。為沛廄司御,每送使客,還過泗上亭,與高祖語,未嘗不移日也。嬰已而試補縣吏,與高祖相愛。高祖戲而傷嬰,人有告高祖。高祖時為亭長,重坐傷人,告故不傷嬰,嬰證之。移獄覆,嬰坐高祖繫歲餘,掠笞數百,終脫高祖。

高祖之初與徒屬欲攻沛也,嬰時以縣令史為高祖使。上降沛一日,高祖為沛公,賜爵七大夫,以嬰為太僕,常奉車。從攻胡陵,嬰與蕭何降泗水監平,平以胡陵降,賜嬰爵五大夫。從擊秦軍碭東,攻濟陽,下戶牖,破李由軍雍丘,以兵車趣攻戰疾,破之,賜爵執帛。從擊章邯軍東阿、濮陽下,以兵車趣攻戰疾,破之,賜爵執圭。從擊趙賁軍開封,楊熊軍曲遇。嬰從捕虜六十八人,降卒八百五十人,得印一匱。又擊秦軍雒陽東,以兵車趣攻戰疾,賜爵封,轉為滕令。因奉車從攻定南陽,戰於藍田、芷陽,至霸上。沛公為漢王,賜嬰爵列侯,號昭平侯,復為太僕,從入蜀漢。

還定三秦,從擊項籍。至彭城,項羽大破漢軍。漢王不利,馳去。見孝惠、魯元,載之。漢王急,馬罷,虜在後,常蹳兩兒棄之,嬰常收載行,面雍樹馳。漢王怒,欲斬嬰者十餘,卒得脫,而致孝惠、魯元於豐。

漢王既至滎陽,收散兵,復振,賜嬰食邑沂陽。擊項籍下邑,追至陳,卒定楚。至魯,益食茲氏。

漢王即帝位,燕王臧荼反,嬰從擊荼。明年,從至陳,取楚王信。更食汝陰,剖符,世世勿絕。從擊代,至武泉、雲中,益食千戶。因從擊韓信軍胡騎晉陽旁,大破之。追北至平城,為胡所圍,七日不得通。高帝使使厚遺閼氏,冒頓乃開其圍一角。高帝出欲馳,嬰固徐行,弩皆持滿外鄉,卒以得脫。益食嬰細陽千戶。從擊胡騎句注北,大破之。擊胡騎平城南,三陷陳,功為多,閼所奪邑五百戶。從擊陳豨、黥布軍,陷陳卻敵,益千戶,定食汝陰六千九百戶,除前所食。

嬰自上初起沛,常為太僕從,竟高祖崩。以太僕事惠帝。惠帝及高后德嬰之脫孝惠、魯元於下邑間也,乃賜嬰北第第一,曰「

近我」,以尊異之。惠帝崩,以太僕事高后。高后崩,代王之來,嬰以太僕與東牟侯入清宮,廢少帝,以天子法駕迎代王代邸,與大臣共立文帝,復為太僕。八歲薨,諡曰文侯。傳至曾孫頗,尚平陽公主,坐與父御婢奸,自殺,國除。

初嬰為滕令奉車,故號滕公。及曾孫頗尚主,主隨外家姓,號孫公主,故滕公子孫更為孫氏。

灌嬰,睢陽販繒者也。高祖為沛公,略地至雍丘,章邯殺項梁,而沛公還軍於碭,嬰以中涓從,擊破東郡尉於成武及秦軍於杠里,疾鬥,賜爵七大夫。又從攻秦軍亳南、開封、曲遇,戰疾力,賜爵執帛,號宣陵君。從攻陽武以西至雒陽,破秦軍尸北。北絕河津,南破南陽守齮陽城東,遂定南陽郡。西入武關,戰於藍田,疾力,至霸上,賜爵執圭,號昌文君。

沛公為漢王,拜嬰為郎中,從入漢中,十月,拜為中謁者。從還定三秦,下櫟陽,降塞王。還圍章邯廢丘,未拔。從東出臨晉關,擊降殷王,定其地。擊項羽將龍且、魏相項佗軍定陶南,疾戰,破之。賜嬰爵列侯,號昌文侯,食杜平鄉。

復以中謁者從降下碭,以北至彭城。項羽擊破漢王,漢王遁而西,嬰從還,軍於雍丘。王武、魏公申徒反,從擊破之。攻下外黃,西收軍於滎陽。楚騎來眾,漢王乃擇軍中可為騎將者,皆推故秦騎士重泉人李必、駱甲習騎兵,今為校尉,可為騎將。漢王欲拜之,必、甲曰:「臣故秦民,恐軍不信臣,臣願得大王左右善騎者傅之。」嬰雖少,然數力戰,乃拜嬰為中大夫,令李必、駱甲為左右校尉,將郎中騎兵擊楚騎於滎陽東,大破之。受詔別擊楚軍後,絕其饟道,起陽武至襄邑。擊項羽之將項冠於魯下,破之,所將卒斬右司馬、騎將各一人。擊破柘公王武軍燕西,所將卒斬樓煩將五人,連尹一人。擊王武別將桓嬰白馬下,破之,所將卒斬都尉一人。以騎度河南,送漢王到雒陽,從北迎相國韓信軍於邯鄲。還至敖倉,嬰遷為御史大夫。

三年,以列侯食邑杜平鄉。受詔將郎中騎兵東屬相國韓信,擊破齊軍於歷下,所將卒虜單騎將軍華毋傷及將吏四十六人。降下臨淄,得相田光。追齊相田橫至嬴、博,擊破其騎,所將卒斬騎將一人,生得騎將四人。攻下嬴、博,破齊將軍田吸於千乘,斬之。東從韓信攻龍且、留公於假密,卒斬龍且,生得右司馬、連尹各一人,樓煩將十人,身生得亞將周蘭。

齊地已定,韓信自立為齊王,使嬰別將擊楚將公杲於魯北,破之。轉南,破薛郡長,身虜騎將入。攻博陽,前至下相以東南僮、取慮、徐。度淮,盡降其城邑,至廣陵。項羽使項聲、薛公、郯公復定淮北,嬰度淮擊破項聲、郯公下邳,斬薛公,下下邳、壽春。擊破楚騎平陽,遂降彭城。虜柱國項佗,降留、薛、沛、酇、蕭、相。攻苦、譙,復得亞將。與漢王會頤鄉。從擊項籍軍陳下,破之。所將卒斬樓煩將二人,虜將八人。賜益食邑二千五百戶。

項籍敗垓下去也,嬰以御史大夫將車騎別追項籍至東城,破之。所將卒五人共斬項籍,皆賜爵列侯。降左右司馬各一人,卒萬二千人,盡得其軍將吏。下東城、歷陽。度江,破吳郡長吳下,得吳守,遂定吳、豫章、會稽郡。還定淮北,凡五十二縣。

漢王即帝位,賜益嬰邑三千戶。以車騎將軍從擊燕王荼。明年,從至陳,取楚王信。還,剖符,世世勿絕,食潁陰二千五百戶。

從擊漢王信於代,至馬邑,別降樓煩以北六縣,斬代左將,破胡騎將於武泉北。復從擊信胡騎晉陽下,所將卒斬胡白題將一人。又受詔并將燕、趙、齊、梁、楚車騎,擊破胡騎於硰石。至平城,為胡所困。

從擊陳豨,別攻豨丞相侯敞軍曲逆下,破之,卒斬敞及特將五人。降曲逆、盧奴、上曲陽、安國、安平。攻下東垣。

黥布反,以車騎將軍先出,攻布別將於相,破之,斬亞將樓煩將三人。又進擊破布上柱國及大司馬軍。又進破布別將肥銖。嬰身生得左司馬一人,所將卒斬其小將十人,追北至淮上。益食邑二千五百戶。布已破,高帝歸,定令嬰食潁陰五千戶,除前所食邑。凡從所得二千石二人,別破軍十六,降城四十六,定國一,郡二,縣五十二,得將軍二人,柱國、相各一人,二千石十人。

嬰自破布歸,高帝崩,以列侯事惠帝及呂后。呂后崩,呂祿等欲為亂。齊哀王聞之,舉兵西,呂祿等以嬰為大將軍往擊之。嬰至滎陽,乃與絳侯等謀,因屯兵滎陽,風齊王以誅呂氏事,齊兵止不前。絳侯等既誅諸呂,齊王罷兵歸。嬰自滎陽還,與絳侯、陳平共立文帝。於是益封嬰三千戶,賜金千斤,為太尉。

三歲,絳侯勃免相,嬰為丞相,罷太尉官。是歲,匈奴大入北地,上令丞相嬰將騎八萬五千擊匈奴。匈奴去,濟北王反,詔罷嬰兵。後歲餘,以丞相薨,諡曰懿侯。傳至孫疆,有罪,絕。武帝復封嬰孫賢為臨汝侯,奉嬰後,後有罪,國除。

傅寬,以魏五大夫騎將從,為舍人,起橫陽。從攻安陽、杠里,趙賁軍於開封,及擊楊熊曲遇、陽武,斬首十二級,賜爵卿。從至霸上。沛公為漢王,賜寬封號共德君。從入漢中,為右騎將。定三秦,賜食邑雕陰。從擊項籍,待懷,賜爵通德侯。從擊項冠、周蘭、龍且,所將卒斬騎將一人敖下,益食邑。

屬淮陰,擊破齊歷下軍,擊田解。屬相國參,殘博,益食邑。因定齊地,剖符世世勿絕,封陽陵侯,二千六百戶,除前所食。為齊右丞相,備齊。五歲為齊相國。

四月,擊陳豨,屬太尉勃,以相國代丞相噲擊豨。一月,徙為代相國,將屯。二歲,為丞相,將屯。孝惠五年薨,諡曰景侯。傳至曾孫偃,謀反,誅,國除。

靳歙,以中涓從,起宛朐。攻濟陽。破李由軍。擊秦軍開封東,斬騎千人將一人,首五十七級,捕虜七十三人,賜爵封臨平君。又戰藍田北,斬車司馬二人,騎長一人,首二十八級,捕虜五十七人。至霸上。沛公為漢王,賜歙爵建武侯,遷騎都尉。

從定三秦。別西擊章平軍於隴西,破之,定隴西六縣,所將卒斬車司馬、候各四人,騎長十二人。從東擊楚,至彭城。漢軍敗還,保雍丘,擊反者王武等。略梁地,別西擊邢說軍菑南,破之,身得說都尉二人,司馬、候十二人,降吏卒四千六百八十人。破楚軍滎陽東。食邑四千二百戶。

別之河內,擊趙賁軍朝歌,破之,所將卒得騎將二人,車馬二百五十匹。從攻安陽以東,至棘蒲,下十縣。別攻破趙軍,得其將司馬二人,候四人,降吏卒二千四百人。從降下邯鄲。別下平陽,身斬守相,所將卒斬兵守郡一人,降鄴。從攻朝歌、邯鄲,又別擊破趙郡,降邯鄲郡六縣。還軍敖倉,破項籍軍成皋南,擊絕楚饟道,起滎陽至襄邑。破項冠魯下。略地東至觞、郯、下邳,南至蘄、竹邑。擊項悍濟陽下。還擊項籍軍陳下,破之。別定江陵,降柱國、大司馬以下八人,身得江陵王,致雒陽,因定南郡。從至陳,取楚王信,剖符世世勿絕,定食四千六百戶,為信武侯。

以騎都尉擊代,攻韓信平城下,還軍東垣。有功,遷為車騎將軍,并將梁、趙、齊、燕、楚車騎,別擊陳豨丞相敞,破之,因降曲逆。從擊黥布有功,益封,定食邑五千三百戶。凡斬首九十級,虜百四十二人,別破軍十四,降城五十九,定郡、國各一,縣二十三,得王、柱國各一人,二千石以下至五石三十九人。

高后五年,薨,諡曰肅侯。子亭嗣,有罪,國除。

周惞,沛人也。以舍人從高祖起沛。至霸上,西入蜀漢,還定三秦,常為參乘,賜食邑池陽。從東擊項羽滎陽,絕甬道,從出度平陰,遇韓信軍襄國,戰有利不利,終亡離上心。上以惞為信武侯,食邑三千三百戶。

上欲自擊陳豨,惞泣曰:「始秦攻破天下,未曾自行,今上常自行,是亡人可使者乎?」上以為「愛我」,賜入殿門不趨。

十二年,更封惞為糨城侯,孝文五年薨,諡曰貞侯。子昌嗣,有罪,國除。景帝復封惞子應為鄲侯,薨,諡曰康侯。子仲居嗣,坐為太常有罪,國除。

贊曰:仲尼稱「犁牛之子騂且角,雖欲勿用,山川其舍諸?」言士不繫於世類也。語曰「雖有茲基,不如逢時」,信矣!樊噲、夏侯嬰、灌嬰之徒,方其鼓刀僕御販繒之時,豈自知附驥之尾,勤功帝籍,慶流子孫哉?當孝文時,天下以酈寄為賣友。夫賣友者,謂見利而忘義也。若寄父為功臣而又執劫,雖摧呂祿,以安社稷,誼存君親,可也。


\end{pinyinscope}