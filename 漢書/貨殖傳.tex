\article{貨殖傳}

\begin{pinyinscope}
昔先王之制,自天子公侯卿大夫士至于皁隸抱關擊还者,其爵祿奉養宮室車服棺槨祭祀死生之制各有差品,小不得僭大,賤不得踰貴。夫然,故上下序而民志定。於是辯其土地川澤丘陵衍沃原隰之宜,教民種樹畜養;五穀六畜及至魚鱉鳥獸雚蒲材幹器械之資,所以養生送終之具,靡不皆育。育之以時,而用之有節。屮木未落,斧斤不入於山林;豺獺未祭,罝網不布於野澤;鷹隼未擊,矰弋不施於徯隧。既順時而取物,然猶山不茬櫱,澤不伐夭,蝝魚麛卵,咸有常禁。所以順時宣氣,蕃阜庶物,蓄足功用,如此之備也。然後四民因其土宜,各任智力,夙興夜寐,以治其業,相與通功易事,交利而俱贍,非有徵發期會,而遠近咸足。故《易》曰「后以財成輔相天地之宜,以左右民」,「備物致用,立成器以為天下利,莫大乎聖人」,此之謂也。管子云古之四民不得雜處。士相與言仁誼於閒宴,工相與議技巧於官府,商相與語財利於市井,農相與謀稼穡於田野,朝夕從事,不見異物而遷焉。故其父兄之教不肅而成,子弟之學不勞而能,各安其居而樂其業,甘其食而美其服,雖見奇麗紛華,非其所習,辟猶戎翟之與于越,不相入矣。是以欲寡而事節,財足而不爭。於是在民上者,道之以德,齊之以禮,故民有恥而且敬,貴誼而賤利。此三代之所以直道而行,不嚴而治之大略也。

及周室衰,禮法墮,諸侯刻桷丹楹,大夫山節藻梲,八佾舞於庭,雍徹於堂。其流至乎士庶人,莫不離制而棄本,稼穡之民少,商旅之民多,穀不足而貨有餘。

陵夷至乎桓、文之後,禮誼大壞,上下相冒,國異政,家殊俗,耆欲不制,僭差亡極。於是商通難得之貨,工作亡用之器,士設反道之行,以追時好而取世資。偽民背實而要名,姦夫犯害而求利,篡弒取國者為王公,圉奪成家者為雄桀。禮誼不足以拘君子,刑戮不足以威小人。富者木土被文錦,犬馬餘肉粟,而貧者裋褐不完,唅菽飲水。其為編戶齊民,同列而以財力相君,雖為僕虜,猶亡慍色。故夫飾變詐為姦軌者,自足乎一世之間;守道循理者,不免於飢寒之患。其教自上興,繇法度之無限也。故列其行事,以傳世變云。

昔粵王句踐困於會稽之上,乃用范蠡、計然。計然曰:「

知鬥則修備,時用則知物,二者形則萬貨之情可得見矣。故旱則資舟,水則資車,物之理也。」推此類而脩之,十年國富,厚賂戰士,遂報彊吳,刷會稽之恥。范蠡歎曰:「計然之策,十用其五而得意。既以施國,吾欲施之家。」乃乘扁舟,浮江湖,變姓名,適齊為鴟夷子皮,之陶為朱公。以為陶天下之中,諸侯四通,貨物所交易也,乃治產積居,與時逐而不責於人。故善治產者,能擇人而任時。十九年之間三致千金,而再散分與貧友昆弟。後年衰老,聽子孫脩業而息之,遂至鉅萬。故言富者稱陶朱。

子贛既學於仲尼,退而仕衛,發貯鬻財曹、魯之間。七十子之徒,賜最為饒,而顏淵簞食瓢飲,在于陋巷。子贛結駟連騎,束帛之幣聘享儲侯,所至,國君無不分庭與之抗禮。然孔子賢顏淵而譏子贛,曰:「回也其庶乎,屢空。賜不受命,而貨殖焉,意則屢中。」

白圭,周人也。當魏文侯時,李克務盡地力,而白圭樂觀時變,故人棄我取,人取我予。能薄飲食,忍嗜欲,節衣服,與用事僮僕同苦樂,趨時若猛獸摯鳥之發。故曰:「吾治生猶伊尹、呂尚之謀,孫吳用兵,商鞅行法是也。故智不足與權變,勇不足以決斷,仁不能以取予,彊不能以有守,雖欲學吾術,終不告也。」蓋天下言治生者祖白圭。

猗頓用盬鹽起,邯鄲郭縱以鑄冶成業,與王者埒富。

烏氏蠃畜牧,及眾,斥賣,求奇繒物,間獻戎王。戎王十倍其償,予畜,畜至用谷量牛馬。秦始皇令蠃比封君,以時與列臣朝請。

巴寡婦清,其先得丹穴,而擅其利數世,家亦不訾。清寡婦能守其業,用財自衛,人不敢犯。始皇以為貞婦而客之,為築女懷清臺。

秦漢之制,列侯封君食租稅,歲率戶二百。千戶之君則二十萬,朝覲聘享出其中。庶民農工商賈,率亦歲萬息二千,百萬之家即二十萬,而更繇租賦出其中,衣食好美矣。故曰陸地牧馬二百蹄,牛千蹄角,千足羊,澤中千足彘,水居千石魚波,山居千章之萩。安邑千樹棗;燕、秦千樹栗;蜀、漢、江陵千樹橘;淮北滎南河濟之間千樹萩;陳、夏千畝桼;齊、魯千畝桑麻;渭川千畝竹;及名國萬家之城,帶郭千畝畝鐘之田,若千畝卮茜,千畦薑韭:此其人皆與千戶侯等。

諺曰:「以貧求富,農不如工,工不如商,刺繡文不如倚市門。」此言末業,貧者之資也。通邑大都酤一歲千釀,醯醬千瓨,漿千儋,屠牛羊彘千皮,穀糴千鐘,薪槁千車,船長千丈,木千章,竹竿萬丢,軺車百乘,牛車千兩;木器桼者千枚,銅器千鈞,素木鐵器若卮茜千石,馬蹄噭千,牛千足,羊彘千雙,童手指千,筋角丹沙千斤,其帛絮細布千鈞,文采千匹,荅布皮革千石,桼千大斗,糱麴鹽豉千合,鮐鮆千斤,鮿鮑千鈞,棗栗千石者三之,狐貂裘千皮,羔羊裘千石,旃席千具,它果采千種,子貸金錢千貫,節駔儈,貪賈三之,廉賈五之,亦比千乘之家,此其大率也。

蜀卓氏之先,趙人也,用鐵冶富。秦破趙,遷卓氏之蜀,夫妻推輦行。諸遷虜少有餘財,爭與吏,求近處,處葭萌。唯卓氏曰:「此地骥薄。吾聞崏山之下沃野,下有踆鴟,至死不飢。民工作市,易賈。」乃求遠遷。致之臨邛,大憙,即鐵山鼓鑄,運籌算,賈滇、蜀民,富至童八百人,田池射獵之樂擬於人君。

程鄭,山東遷虜也,亦冶鑄,賈魋結民,富埒卓氏。

程、卓既衰,至成、哀間,成都羅裒訾至鉅萬。初,裒賈京師,隨身數十百萬,為平陵石氏持錢。其人彊力。石氏訾次如、苴,親信,厚資遣之,令往來巴蜀,數年間致千餘萬。裒舉其半賂遺曲陽、定陵侯,依其權力,賒貸郡國,人莫敢負。擅鹽井之利,期年所得自倍,遂殖其貨。

宛孔氏之先,梁人也,用鐵冶為業。秦滅魏,遷孔氏南陽,大鼓鑄,規陂田,連騎游諸侯,因通商賈之利,有游閒公子之名。然其贏得過當,瘉於孅嗇,家致數千金,故南陽行賈盡法孔氏之雍容。

魯人俗儉嗇,而丙氏尤甚,以鐵冶起,富至鉅萬。然家自父兄子弟約,頫有拾,卬有取,貰貸行賈遍郡國。鄒、魯以其故,多去文學而趨利。

齊俗賤奴虜,而刀閒獨愛貴之。桀黠奴,人之所患,唯刀閒收取,使之逐魚鹽商賈之利,或連車騎交守相,然愈益任之,終得其力,起數千萬。故曰「寧爵無刀」,言能使豪奴自饒,而盡其力也。刀閒既衰,至成、哀間,臨淄姓偉訾五千萬。

周人既孅,而師史尤甚,轉轂百數,賈郡國,無所不至。雒陽街居在齊秦楚趙之中,富家相矜以久賈,過邑不入門。設用此等,故師史能致十千萬。

師史既衰,至成、哀、王莽時,雒陽張長叔、薛子仲訾亦十千萬。莽皆以為納言士,欲法武帝,然不能得其利。

宣曲任氏,其先為督道倉吏。秦之敗也,豪桀爭取金玉,任氏獨窖倉粟。楚漢相距滎陽,民不得耕種,米石至萬,而豪桀金玉盡歸任氏,任氏以此起富。富人奢侈,而任氏折節為力田畜。人爭取賤賈,任氏獨取貴善,富者數世。然任公家約,非田畜所生不衣食,公事不畢則不得飲酒食肉。以此為閭里率,故富而主上重之。

塞之斥也,唯橋桃以致馬千匹,牛倍之,羊萬,粟以萬鍾計。

吳楚兵之起,長安中列侯封君行從軍旅,齎貣子錢家,子錢家以為關東成敗未決,莫肯予。唯母鹽氏出捐千金貸,其息十之。三月,吳楚平。一歲之中,則母鹽氏息十倍,用此富關中。

關中富商大賈,大氐盡諸田,田牆、田蘭。韋家栗氏、安陵杜氏亦鉅萬。前富者既衰,自元、成訖王莽,京師富人杜陵樊嘉,茂陵摯網,平陵如氏、苴氏,長安丹王君房,豉樊少翁、王孫大卿,為天下高訾。樊嘉五千萬,其餘皆鉅萬矣。王孫卿以財養士,與雄桀交,王莽以為京司市師,漢司東市令也。

此其章章尤著者也。其餘郡國富民兼業顓利,以貨賂自行,取重於鄉里者,不可勝數。故秦楊以田農而甲一州,翁伯以販脂而傾縣邑,張氏以賣醬而隃侈,質氏以洒削而鼎食,濁氏以冒脯而連騎,張里以馬醫而擊鍾,皆越法矣。然常循守事業,積累贏利,漸有所起。至於蜀卓,宛孔,齊之刀閒,公擅山川銅鐵魚鹽市井之入,運其籌策,上爭王者之利,下錮齊民之業,皆陷不軌奢僭之惡。又況掘冢搏掩,犯姦成富,曲叔、稽發、雍樂成之徒,猶復齒列,傷化敗俗,大亂之道也。


\end{pinyinscope}