\article{古今人表}

\begin{pinyinscope}
自書契之作,先民可得而聞者,經傳所稱,唐虞以上,帝王有號諡。輔佐不可得而稱矣,而諸子頗言之,雖不考虖孔氏,然猶著在篇籍,歸乎顯善昭惡,勸戒後人,故博采焉。孔子曰:「若聖與仁,則吾豈敢?」又曰:「何事於仁,必也聖乎!」「未知,焉得仁?」「生而知之者,上也;學而知之者,次也;困而學之,又其次也;困而不學,民斯為下矣。」又曰:「中人以上,可以語上也。」「唯上智與下愚不移。」

傳曰:譬如堯舜,禹、稷、镨與之為善則行,鯀、讙兜欲與為惡則誅。可與為善,不可與為惡,是謂上智。桀紂,龍逢、比干欲與之為善則誅,于莘、崇侯與之為惡則行。可與為惡,不可與為善,是謂下愚。齊桓公,管仲相之則霸,豎貂輔之則亂。可與為善,可與為惡,是謂中人。因茲以列九等之序,究極經傳,繼世相次,總備古今之略要云。

上上聖人上中仁人上下智人中上中中中下下上下中下下愚人


\end{pinyinscope}