\article{揚雄傳}

\begin{pinyinscope}
揚雄字子雲,蜀郡成都人也。其先出自有周伯僑者,以支庶初食采於晉之楊,因氏焉,不知伯僑周何別也。揚在河、汾之間,周衰而揚氏或稱侯,號曰揚侯。會晉六卿爭權,韓、魏、趙興而范中行、知伯弊。當是時,偪揚侯,揚侯逃於楚巫山,因家焉。楚漢之興也,揚氏溯江上,處巴江州。而揚季官至廬江太守,漢元鼎間避仇復溯江上,處岿山之陽曰郫,有田一槟,有宅一區,世世以農桑為業。自季至雄,五世而傳一子,故雄亡它揚於蜀。

雄少而好學,不為章句,訓詁通而已,博覽無所不見。為人簡易佚蕩,口吃不能劇談,默而好深湛之思,清靜亡為,少耆欲,不汲汲於富貴,不戚戚於貧賤,不修廉隅以徼名當世。家產不過十金,乏無儋石之儲,晏如也。自有大度,非聖哲之書不好也;非其意,雖富貴不事也。顧嘗好辭賦。

先是時,蜀有司馬相如,作賦甚弘麗溫雅,雄心壯之,每作賦,常擬之以為式。又怪屈原文過相如,至不容,作離騷,自投江而死,悲其文,讀之未嘗不流涕也。以為君子得時則大行,不得時則龍蛇,遇不遇命也,何必湛身哉!乃作書,往往摭離騷文而反之,自崏山投諸江流以弔屈原,名曰反離騷;又旁離騷作重一篇,名曰廣騷;又旁惜誦以下至懷沙一卷,名曰畔牢愁。畔牢愁、廣騷文多不載,獨載反離騷,其辭曰:

有周氏之蟬嫣兮,或鼻祖於汾隅,靈宗初諜伯僑兮,流于末之揚侯。淑周楚之豐烈兮,超既離虖皇波,因江潭而柜記兮,欽弔楚之湘纍。

惟天軌之不辟兮,何純絜而離紛!紛纍以其淟涊兮,暗纍以其繽紛。

漢十世之陽朔兮,招搖紀于周正,正皇天之清則兮,度后土之方貞。圖纍承彼洪族兮,又覽纍之昌辭,帶鉤矩而佩衡兮,履欃槍以為綦。素初貯厥麗服兮,何文肆而質筹!資娵娃之珍专兮,鬻九戎而索賴。

鳳皇翔於蓬陼兮,豈鴐鵝之能捷!騁驊騮以曲艱兮,驢騾連蹇而齊足。枳棘之榛榛兮,蝯貁擬而不敢下,靈修既信椒、蘭之唼佞兮,吾纍忽焉而不蚤睹?

衿芰茄之綠衣兮,被夫容之朱裳,芳酷烈而莫聞兮,固不如襞而幽之離房。閨中容競淖約兮,相態以麗佳,知眾嫭之嫉妒兮,何必颺纍之蛾眉?

懿神龍之淵潛,俟慶雲而將舉,亡春風之被離兮,孰焉知龍之所處?愍吾纍之眾芬兮,颺鳞鳞之芳苓,遭季夏之凝霜兮,慶夭悴而喪榮。

橫江、湘以南柜兮,云走乎彼蒼吾,馳江潭之汎溢兮,將折衷虖重華。舒中情之煩或兮,恐重華之不纍與,陵陽侯之素波兮,豈吾纍之獨見許?

精瓊靡與秋菊兮,將以延夫天年;臨汨羅而自隕兮,恐日薄於西山。解扶桑之總轡兮,縱令之遂奔馳,鸞皇騰而不屬兮,豈獨飛廉與雲師!

卷薜芷與若蕙兮,臨湘淵而投之;棍申椒與菌桂兮,赴江湖而漚之。費椒稰以要神兮,又勤索彼瓊茅,違靈氛而不從兮,反湛身於江皋!

纍既罗夫傅說兮,奚不信而遂行?徒恐鷤缳之將鳴兮,顧先百草為不芳!

初纍棄彼虙妃兮,更思瑤臺之逸女,抨雄鴆以作媒兮,何百離而曾不壹耦!乘雲蜺之旖柅兮,望昆侖以樛流,覽四荒而顧懷兮,奚必云女彼高丘?

既亡鸞車之幽藹兮,焉駕八龍之委蛇?臨江瀕而掩涕兮,何有九招與九歌?夫聖哲之不遭兮,固時命之所有;雖增欷以於邑兮,吾恐靈修之不纍改。昔仲尼之去魯兮,婓婓遲遲而周邁,終回復於舊都兮,何必湘淵與濤瀨!溷漁父之餔歠兮,絜沐浴之振衣,棄由、聃之所珍兮,蹠彭咸之所遺!

孝成帝時,客有薦雄文似相如者,上方郊祠甘泉泰畤、汾陰后土,以求繼嗣,召雄待詔承明之庭。正月,從上甘泉,還奏甘泉賦以風。其辭曰:

惟漢十世,將郊上玄,定泰畤,雍神休,尊明號,同符三皇,錄功五帝,卹胤錫羨,拓跡開統。於是乃命群僚,歷吉日,協靈辰,星陳而天行。詔招搖與泰陰兮,伏鉤陳使當兵,屬堪輿以壁壘兮,梢夔魖而抶獝狂。八神奔而警蹕兮,振殷轔而軍裝;蚩尤之倫帶干將而秉玉戚兮,飛蒙茸而走陸梁。齊總總撙撙,其相膠葛兮,猋駭雲訊,奮以方攘;駢羅列布,鱗以雜沓兮,柴虒參差,魚頡而鳥扩;翕赫曶霍,霧集蒙合兮,半散照爛,粲以成章。

於是乘輿乃登夫鳳皇兮翳華芝,駟蒼螭兮六素虯,蠖略蕤綏,灕虖幓纚。帥爾陰閉,霅然陽開,騰清霄而軼浮景兮,夫何旟旐郅偈之旖柅也!流星旄以電燭兮,咸翠蓋而鸞旗。敦萬騎於中營兮,方玉車之千乘。聲駍隱以陸離兮,輕先疾雷而馺遺風。陵高衍之嵱嵷兮,超紆譎之清澄。登椽欒而羾天門兮,馳閶闔而入凌兢。

是時未轃夫甘泉也,乃望通天之繹繹。下陰潛以慘廩兮,上洪紛而相錯;直嶢嶢以造天兮,厥高慶而不可虖疆度。平原唐其壇曼兮,列新雉於林薄;攢并閭與茇懑兮,紛被麗其亡鄂。崇丘陵之蔺蔼兮,深溝嶔巖而為谷;婶婶離宮般以相燭兮,封巒石關施靡虖延屬。

於是大夏雲譎波詭,嶊嶉而成觀,仰撟首以高視兮,目冥眴而亡見。正瀏濫以弘惝兮,指東西之漫漫,徒回回以徨徨兮,魂固眇眇而昏亂。據軨軒而周流兮,忽軮軋而亡垠。翠玉樹之青蔥兮,壁馬犀之瞵垒。金人仡仡其承鍾虡兮,嵌巖巖其龍鱗,揚光曜之燎燭兮,乘景炎之炘炘,配帝居之縣圃兮,象泰壹之威神。洪臺掘其獨出兮,㨖北極之嶟嶟,列宿乃施於上榮兮,日月纔經於柍桭,雷鬱律而巖突兮,電倏忽於牆藩。鬼魅不能自還兮,半長途而下顛。歷倒景而絕飛梁兮,浮蔑蠓而撇天。

左欃槍右玄冥兮,前熛闕後應門;陰西海與幽都兮,涌醴汨以生川。蛟龍連蜷於東劯兮,白虎敦圉虖昆侖。覽樛流於高光兮,溶方皇於西清。前殿崔巍兮,和氏瓏玲,炕浮柱之飛榱兮,神莫莫而扶傾,閌閬閬其寥廓兮,似紫宮之崢嶸。駢交錯而曼衍兮,丛嶵隗虖其相嬰。乘雲閣而上下兮,紛蒙籠以掍成。曳紅采之流離兮,颺翠氣之冤延。襲琁室與傾宮兮,若登高妙遠,肅虖臨淵。

回猋肆其碭駭兮,翍桂椒,鬱栘楊。香芬茀以窮隆兮,擊薄櫨而將榮。薌斋肸以掍根兮,聲駍隱而歷鍾,排玉戶而颺金鋪兮,發蘭惠與穹窮。惟弸彋其拂汨兮,稍暗暗而靚深。陰陽清濁穆羽相和兮,若夔、牙之調琴。般、倕棄其剞劂兮,王爾投其鉤繩。雖方征僑與偓佺兮,猶仿佛其若夢。

於是事變物化,目駭耳回,蓋天子穆然珍臺閒館琁題玉英蜵蜎蠖濩之中,惟夫所以澄心清魂,儲精垂思,感動天地,逆釐三神者。乃搜逑索耦皋、伊之徒,冠倫魁能,函甘棠之惠,挾東征之意,相與齊虖陽靈之宮。靡薜荔而為席兮,折瓊枝以為芳,唿清雲之流瑕兮,飲若木之露英,集虖禮神之囿,登乎頌祇之堂。建光燿之長旓兮,昭華覆之威威,攀琁璣而下視兮,行遊目虖三危,陳眾車所東阬兮,肆玉釱而下馳,漂龍淵而還九垠兮,窺地底而上回。風傱傱而扶轄兮,鸞鳳紛其御蕤,梁弱水之濎濴兮,躡不周之逶蛇,想西王母欣然而上壽兮,屏玉女而卻虙妃。玉女無所眺其清盧兮,虙妃曾不得施其蛾眉。方攬道德之精剛兮,眸神明與之為資。

於是欽祡宗祈。燎熏皇天,招繇泰壹。舉洪頤,樹靈旗。樵蒸焜上,配藜四施,東燭倉海,西燿流沙,北爌幽都,南煬丹劯。玄瓚觩鸽,秬鬯泔淡,肸嚮豐融,懿懿芬芬。炎感黃龍兮,熛訛碩麟,選巫咸兮叫帝閽,開天庭兮延群神。儐暗藹兮降清壇,瑞穰穰兮委如山。

於是事畢功弘,回車而歸,度三巒兮偈棠葛。天閫決兮地垠開,八荒協兮萬國諧。登長平兮雷鼓磕,天聲起兮勇士厲,雲飛揚兮雨滂沛,于胥德兮麗萬世。

亂曰:崇崇圜丘,隆隱天兮,登降峛崺,單埢垣兮。增宮嵾差,駢嵯峨兮,岭巆嶙峋,洞亡劯兮。上天之縡,杳旭卉兮,聖皇穆穆,信厥對兮。來祗郊禋,神所依兮,俳佪招搖,靈哕鲑兮。煇光眩燿,隆厥福兮,子子孫孫,長亡極兮。

甘泉本因秦離宮,既奢泰,而武帝復增通天、高光、迎風。宮外近則洪劯、旁皇、儲胥、弩阹,遠則石關、封巒、枝鵲、露寒、棠梨、師得,遊觀屈奇瑰瑋,非木摩而不彫,牆塗而不畫,周宣所考,般庚所遷,夏卑宮室,唐虞棌椽三等之制也。且為其已久矣,非成帝所造,欲諫則非時,欲默則不能已,故遂推而隆之,乃上比於帝室紫宮,若曰此非人力之所能,黨鬼神可也。又是時趙昭儀方大幸,每上甘泉,常法從,在屬車間豹尾中。故雄聊盛言車騎之眾,參麗之駕,非所以感動天地,逆釐三神。又言「屏玉女,卻虙妃」,以微戒齊肅之事。賦成奏之,天子異焉。

其三月,將祭后土,上乃帥群臣橫大河,湊汾陰。既祭,行遊介山,回安邑,顧龍門,覽鹽池,登歷觀,陟西岳以望八荒,跡殷周之虛,眇然以思唐虞之風。雄以為臨川羨魚不如歸而結罔,還,上河東賦以勸,其辭曰:

伊年暮春,將瘞后土,禮靈祇,謁汾陰于東郊,因茲以勒崇垂鴻,發祥隤祉,欽若神明者,盛哉鑠乎,越不可載已!於是命群臣,齊法服,整靈輿,乃撫翠鳳之駕,六先景之乘,掉奔星之流旃,彏天狼之威弧。張燿日之玄旄,揚左纛,被雲梢。奮電鞭,驂雷輜,鳴洪鍾,建五旗。義和司日,顏倫奉輿,風發飆拂,神騰鬼趡;千乘霆亂,萬騎屈橋,嘻嘻旭旭,天地稠鲛。簸丘跳巒,涌渭躍涇。秦神下讋,跖魂負沴;河靈矍踢,掌華蹈衰。遂臻陰宮,穆穆肅肅,蹲蹲如也。

靈祇既鄉,五位時敘,絪縕玄黃,將紹厥後。於是靈輿安步,周流容與,以覽虖介山。嗟文公而愍推兮,勤大禹於龍門,灑沈鮮於豁瀆兮,播九河於東瀕。登歷觀而遙望兮,聊浮游以經營。樂往昔之遺風兮,喜虞氏之所耕。瞰帝唐之嵩高兮,眽隆周之大寧。汨低回而不能去兮,行睨陔下與彭城。濊南巢之坎坷兮,易豳岐之夷平。乘翠龍而超河兮,陟西岳之嶢崝。雲霏霏而來迎兮,澤滲灕而下降,鬱蕭條其幽藹兮,滃汎沛以豐隆。叱風伯於南北兮,呵雨師於西東,參天地而獨立兮,廓盪盪其亡雙。

遵逝虖歸來,以函夏之大漢兮,彼曾何足與比功?建乾坤之貞兆兮,將悉總之以群龍。麗鉤芒與驂蓐收兮,服玄冥及祝融。敦眾神使式道兮,奮六經以攄頌。隃於穆之緝熙兮,過清廟之雝雝;軼五帝之遐跡兮,躡三皇之高蹤。既發軔於平盈兮,誰謂路遠而不能從?

其十二月羽獵,雄從。以為昔在二帝三王,宮館臺榭沼池苑囿林麓藪澤財足以奉郊廟,御賓客,充庖廚而已,不奪百姓膏腴穀土桑柘之地。女有餘布,男有餘粟,國家殷富,上下交足,故甘露零其庭,醴泉流其唐,鳳皇巢其樹,黃龍游其沼,麒麟臻其囿,神爵棲其林。昔者禹任益虞而上下和,屮木茂;成湯好田而天下用足;文王囿百里,民以為尚小;齊宣王囿四十里,民以為大:裕民之與奪民也。武帝廣開上林,南至宜春、鼎胡、御宿、昆吾,旁南山而西,至長楊、五柞,北繞黃山,瀕渭而東,周袤數百里。穿昆明象滇河,營建章、鳳闕、神明、馺娑,漸臺、泰液象海水周流方丈、瀛洲、蓬萊。游觀侈靡,窮妙極麗。雖頗割其三垂以贍齊民,然至羽獵田車戎馬器械儲偫禁禦所營,尚泰奢麗誇詡,非堯、舜、成湯、文王三驅之意也。又恐後世復修前好,不折中以泉臺,故聊因校獵賦以風,其辭曰:

或稱戲農,豈或帝王之彌文哉?論者云否,各亦並時而得宜,奚必同條而共貫?則泰山之封,烏得七十而有二儀?是以創業垂統者俱不見其爽,遐邇五三孰知其是非?遂作頌曰:麗哉神聖,處於玄宮,富既與地虖侔訾,貴正與天虖比崇。齊桓曾不足使扶轂,楚嚴未足以為驂乘;骥三王之阨薜,嶠高舉而大興;歷五帝之寥廓,涉三皇之登閎;建道德以為師,友仁義與為朋。

於是玄冬季月,天地隆烈,萬物權輿於內,徂落於外,帝將惟田于靈之囿,開北垠,受不周之制,以終始顓頊、玄冥之統。乃詔虞人典澤,東延昆鄰,西馳闛闔。儲積共偫,戍卒夾道,斬叢棘,夷野草,禦自汧、渭,經營酆、鎬,章皇周流,出入日月,天與地杳。爾乃虎路三嵕以為司馬,圍經百里而為殿門。外則正南極海,邪界虞淵,鴻濛沆茫,碣以崇山。營合圍會,然后先置虖白楊之南,昆明靈沼之東。賁育之倫,蒙盾負羽,杖鏌邪而羅者以萬計,其餘荷垂天之畢,張竟野之罘,靡日月之朱竿,曳彗星之飛旗。青雲為紛,紅蜺為繯,屬之虖昆侖之虛,渙若天星之羅,浩如濤水之波,淫淫與與,前後要遮。欃槍為闉,明月為候,熒惑司命,天弧發射,鮮扁陸離,駢衍佖路。徽車輕武,鴻絧緁獵,殷殷軫軫,被陵緣阪,窮冥極遠者,相與迾虖高原之上;羽騎營營,昈分殊事,繽紛往來,轠轤不絕,若光若滅者,布虖青林之下。

於是天子乃以陽晁始出虖玄宮,撞鴻鍾,建九流,六白虎,載靈輿,蚩尤並轂,蒙公先驅。立歷天之旂,曳捎星之旃,辟歷列缺,吐火施鞭。萃傱允溶,淋離廓落,戲八鎮而開關;飛廉、雲師,吸撸潚率,鱗羅布列,攢以龍翰。秋秋蹌蹌,入西園,切神光;望平樂,徑竹林,蹂惠圃,踐蘭唐。舉烽烈火,轡者施披,方馳千駟,校騎萬師。缦虎之陳,從橫膠輵,猋泣雷厲,驞駍駖磕,洶洶旭旭,天動地岋。羨漫半散,蕭條數千萬里外。

若夫壯士慷慨,殊鄉別趣,東西南北,騁耆奔欲。癴蒼豨,跋犀犛,蹶浮麋。斮巨狿,搏玄蝯,騰空虛,距連卷。踔夭蟜,娭澗門,莫莫紛紛,山谷為之風猋,林叢為之生塵。及至獲夷之徒,蹶松柏,掌疾梨;獵蒙蘢,轔輕飛;履般首,帶修蛇;鉤赤豹,摼象犀;跇巒阬,超唐陂。車騎雲會,登降闇藹,泰華為旒,熊耳為綴。木仆山還,漫若天外,儲與虖大溥,聊浪虖宇內。

於是天清日晏,逢蒙列眥,羿氏控弦。皇車幽輵,光純天地,望舒彌轡,翼乎徐至於上蘭。移圍徙陳,浸淫蹴部,曲隊堅重,各按行伍。壁壘天旋,神抶電擊,逢之則碎,近之則破,鳥不及飛,獸不得過,軍驚師駭,刮野埽地。及至罕車飛揚,武騎聿皇;蹈飛豹,絹嘄陽;追天寶,出一方;應駍聲,擊流光。野盡山窮,囊括其雌雄,沈沈容容,遙噱虖紘中。三軍芒然,窮冘閼與,亶觀夫票禽之紲隃,犀兕之抵觸,熊羆之挐攫,虎豹之凌遽,徒角搶題注,蹙竦讋怖,魂亡魄失,觸輻關脰。妄發期中,進退履獲,創淫輪夷,丘累陵聚。

於是禽殫中衰,相與集於靖冥之館,以臨珍池。灌以岐梁,溢以江河,東瞰目盡,西暢亡劯,隨珠和氏,焯爍其陂。玉石嶜崟,眩燿青熒,漢女水潛,怪物暗冥,不可殫形。玄鸞孔雀,翡翠垂榮,王雎關關,鴻鴈嚶嚶,群娭虖其中,唣唣昆鳴;鳧鷖振鷺,上下砰磕,聲若雷霆。乃使文身之技,水格鱗蟲,凌堅冰,犯嚴淵,探巖排碕,薄索蛟螭,蹈骏獺,據黿鼉,抾靈蠵。入洞穴,出蒼梧,乘鉅鱗,騎京魚。浮彭蠡,目有虞。方椎夜光之流離,剖明月之珠胎,鞭洛水之虙妃,餉屈原與彭胥。

於茲虖鴻生鉅儒,俄軒冕,雜衣裳,修唐典,匡雅頌,揖讓於前。昭光振燿,蠁曶如神,仁聲惠於北狄,武義動於南鄰。是以旃裘之王,胡貉之長,移珍來享,抗手稱臣。前入圍口,後陳盧山。群公常伯楊朱、墨翟之徒喟然稱曰:「崇哉乎德,雖有唐、虞、大夏、成周之隆,何以侈茲!太古之覲東嶽,禪梁基,舍此世也,其誰與哉?」

上猶謙讓而未俞也,方將上獵三靈之流,下決醴泉之滋,發黃龍之穴,窺鳳皇之巢,臨麒麟之囿,幸神雀之林;奢雲夢,侈孟諸,非章華,是靈臺,罕徂離宮而輟觀游,土事不飾,木功不彫,承民乎農桑,勸之以弗迨,儕男女使莫違;恐貧窮者不遍被洋溢之饒,開禁苑,散公儲,創道德之囿,弘仁惠之虞,馳弋乎神明之囿,覽觀乎群臣之有亡;放雉菟,收罝罘,麋鹿芻蕘與百姓共之,蓋所以臻茲也。於是醇洪鬯之德,豐茂世之規,加勞三皇,勗勤五帝,不亦至乎!乃祗莊雍穆之徒,立君臣之節,崇賢聖之業,未皇苑囿之麗,游獵之靡也,因回軫還衡,背阿房,反未央。

明年,上將大誇胡人以多禽獸,秋,命右扶風發民入南山,西自褒斜,東至弘農,南敺漢中,張羅罔罝罘,捕熊羆豪豬虎豹狖玃狐菟麋鹿,載以檻車,輸長楊射熊館。以罔為周阹,從禽獸其中,令胡人手搏之,自取其獲,上親臨觀焉。是時,農民不得收斂。雄從至射熊館,還,上長楊賦,聊因筆墨之成文章,故藉翰林以為主人,子墨為客卿以風。其辭曰:

子墨客卿問於翰林主人曰:「蓋聞聖主之養民也,仁霑而恩洽,動不為身。今年獵長楊,先命右扶風,左太華而右褒斜,椓截嶭而為弋,紆南山以為罝,羅千乘於林莽,列萬騎於山隅,帥軍踤阹,錫戎獲胡。搤熊羆,癴豪豬,木雍槍纍,以為儲胥,此天下之窮覽極觀也。雖然,亦頗擾于農民。三旬有餘,其廑至矣,而功不圖,恐不識者,外之則以為娛樂之遊,內之則不以為乾豆之事,豈為民乎哉!且人君以玄默為神,澹泊為德,今樂遠出以露威靈,數搖動以罷車甲,本非人主之急務也。蒙竊或焉。」

翰林主人曰:「吁,謂之茲邪!若客,所謂知其一未睹其二,見其外不識其內者也。僕嘗倦談,不能一二其詳,請略舉凡,而客自覽其切焉。」

客曰:「唯,唯。」

主人曰:「昔有彊秦,封豕其士,窫窳其民,鑿齒之徒相與摩牙而爭之,豪俊麋沸雲擾,群黎為之不康。於是上帝眷顧高祖,高祖奉命,順斗極,運天關,橫鉅海,票昆侖,提劍而叱之,所麾城搟邑,下將降旗,一日之戰,不可殫記。當此之勤,頭蓬不暇疏,飢不及餐,鞮鍪生蟣蝨,介冑被霑汗,以為萬姓請命虖皇天。乃展民之所詘,振民之所乏,規億載,恢帝業,七年之間而天下密如也。

「逮至聖文,隨風乘流,方垂意於至寧,躬服節儉,綈衣不敝,革鞜不穿,大夏不居,木器無文。於是後宮賤玳瑁而疏珠璣,卻翡翠之飾,除彫瑑之巧,惡麗靡而不近,斥芬芳而不御,抑止絲竹晏衍之樂,憎聞鄭衛幼眇之聲,是以玉衡正而太階平也。

「其後熏鬻作虐,東夷橫畔,羌戎睚眥,閩越相亂,遐萌為之不安,中國蒙被其難。於是聖武勃怒,爰整其旅,乃命票、衛,汾沄沸渭,雲合電發,猋騰波流,機駭鋒軼,疾如奔星,擊如震霆,砰轒轀,破穹廬,腦沙幕,髓余吾。遂獵乎王延。敺橐它,燒虽蠡,分梨單于,磔裂屬國,夷阬谷,拔鹵莽,刊山石,蹂屍輿廝,係累老弱,兗鋋瘢耆、金鏃淫夷者數十萬人,皆稽顙樹頷,扶服蛾伏,二十餘年矣,尚不敢惕息。夫天兵四臨,幽都先加,回戈邪指,南越相夷,靡節西征,羌僰東馳。是以遐方疏俗殊鄰絕黨之域,自上仁所不化,茂德所不綏,莫不蹻足抗手,請獻厥珍,使海內澹然,永亡邊城之災,金革之患。

「今朝廷純仁,遵道顯義,并包書林,聖風雲靡;英華沈浮,洋溢八區,普天所覆,莫不沾濡;士有不談王道者則樵夫笑之。故意者以為事罔隆而不殺,物靡盛而不虧,故平不肆險,安不忘危。乃時以有年出兵,整輿竦戎,振師五莋,習馬長楊,簡力狡獸,校武票禽。乃萃然登南山,瞰烏弋,西厭月韩,東震日域。又恐後世迷於一時之事,常以此取國家之大務,淫荒田獵,陵夷而不禦也,是以車不安軔,日未靡旃,從者仿佛,委屬而還;亦所以奉太宗之烈,遵文武之度,復三王之田,反五帝之虞;使農不輟耰,工不下機,婚姻以時,男女莫違;出愷弟,行簡易,矜劬勞,休力役;見百年,存孤弱,帥與之,同苦樂。然後陳鐘鼓之樂,鳴鞀磬之和,建碣磍之虡,拮隔鳴球,掉八列之舞;酌允鑠,肴樂胥,聽廟中之雍雍,受神人之福祜;歌投頌,吹合雅。其勤若此,故真神之所勞也。方將俟元符,以禪梁甫之基,增泰山之高,延光于將來,比榮乎往號,豈徒欲淫覽浮觀,馳騁梗稻之地,周流梨栗之林,蹂踐芻蕘,誇詡眾庶,盛狖玃之收,多麋鹿之獲哉!且盲不見咫尺,而離婁燭千里之隅;客徒愛胡人之獲我禽獸,曾不知我亦已獲其王侯。」

言未卒,墨客降席再拜稽首曰:「大哉體乎!允非小子之所能及也。乃今日發矇,廓然已昭矣!」

哀帝時丁、傅、董賢用事,諸附離之者或起家至二千石。時雄方草太玄,有以自守,泊如也。或仓雄以玄尚白,而雄解之,號曰解仓。其辭曰:

客仓揚子曰:「吾聞上世之士,人綱人紀,不生則已,生則上尊人君,下榮父母,析人之圭,儋人之爵,懷人之符,分人之祿,紆青癴紫,朱丹其轂。今子幸得遭明盛之世,處不諱之朝,與群賢同行,歷金門上玉堂有日矣,曾不能畫一奇,出一策,上說人主,下談公卿。目如燿星,舌如電光,壹從壹衡,論者莫當,顧而作太玄五千文,支葉扶疏,獨說十餘萬言,深者入黃泉,高者出蒼天,大者含元氣,纖者入無倫,然而位不過侍郎,擢纔給事黃門。意者玄得毋尚白乎?何為官之拓落也?」

揚子笑而應之曰:「客徒欲朱丹吾轂,不知一跌將赤吾之族也!往者周罔解結,群鹿爭逸,離為十二,合為六七,四分五剖,並為戰國。士無常君,國亡定臣,得士者富,失士者貧,矯翼厲翮,恣意所存,故士或自盛以橐,或鑿坏之遁。是故騶衍以頡亢而取世資,孟軻雖連蹇,猶為萬乘師。

「今大漢左東海,右渠搜,前番禺,後陶塗。東南一尉,西北一候。徽以糾墨,製以質鈇,散以禮樂,風以詩書,曠以歲月,結以倚廬。天下之士,雷動雲合,魚鱗雜襲,咸營于八區,家家自以為稷契,人人自以為咎繇,戴縰垂纓而談者皆擬於阿衡,五尺童子羞比晏嬰與夷吾;當塗者入青雲,失路者委溝渠,旦握權則為卿相,夕失勢則為匹夫;譬若江湖之雀,勃解之鳥,乘雁集不為之多,雙鳧飛不為之少。昔三仁去而殷虛,二老歸而周熾,子胥死而吳亡,種、蠡存而粵伯,五羖入而秦喜,樂毅出而燕懼,范雎以折摺而危穰侯,蔡澤雖噤吟而笑唐舉。故當其有事也,非蕭、曹、子房、平、勃、樊、霍則不能安;當其亡事也,章句之徒相與坐而守之,亦亡所患。故世亂,則聖哲馳騖而不足;世治,則庸夫高枕而有餘。

「夫上世之士,或解縛而相,或釋褐而傅;或倚夷門而笑,或橫江潭而漁;或七十說而不遇,或立談間而封侯;或枉千乘於陋巷,或擁帚彗而先驅。是以士頗得信其舌而奮其筆,窒隙蹈瑕而無所詘也。當今縣令不請士,郡守不迎師,群卿不揖客,將相不俛眉;言奇者見疑,行殊者得辟,是以欲談者宛舌而固聲,欲行者擬足而投跡。鄉使上世之士處虖今,策非甲科,行非孝廉,舉非方正,獨可抗疏,時道是非,高得待詔,下觸聞罷,又安得青紫?

「且吾聞之,炎炎者滅,隆隆者絕;觀雷觀火,為盈為實,天收其聲,地藏其熱。高明之家,鬼瞰其室。攫挐者亡,默默者存;位極者宗危,自守者身全。是故知玄知默,守道之極;爰清爰靜,游神之廷;惟寂惟寞,守德之宅。世異事變,人道不殊,彼我易時,未知何如。今子乃以鴟梟而笑鳳皇,執蝘蜓而仓龜龍,不亦病乎!子徒笑我玄之尚白,吾亦笑子之病甚,不遭臾跗、扁鵲,悲夫!」

客曰:「然則靡玄無所成名乎?范、蔡以下何必玄哉?」

揚子曰:「范雎,魏之亡命也,折脅拉髂,免於徽索,翕肩蹈背,扶服入橐,激卬萬乘之主,界涇陽抵穰侯而代之,當也。蔡澤,山東之匹夫也,顉頤折頞,涕阑流沫,西揖彊秦之相,搤其咽,炕其氣,附其背而奪其位,時也。天下已定,金革已平,都於雒陽,婁敬委輅脫輓,掉三寸之舌,建不拔之策,舉中國徙之長安,適也。五帝垂典,三王傳禮,百世不易,叔孫通起於枹鼓之間,解甲投戈,遂作君臣之儀,得也。甫刑靡敝,秦法酷烈,聖漢權制,而蕭何造律,宜也。故有造蕭何律於唐虞之世,則誖矣;有作叔孫通儀於夏殷之時,則惑矣;有建婁敬之策於成周之世,則繆矣;有談范、蔡之說於金、張、許、史之間,則狂矣。蕭規曹隨,留侯畫策,陳平出奇,功若泰山,嚮若阺隤,唯其人之贍知哉,亦會其時之可為也。故為可為於可為之時,則從;為不可為於不可為之時;則凶。夫藺先生收功於章臺,四皓采榮於南山,公孫創業於金馬,票騎發跡於祁連,司馬長卿竊訾於卓氏,東方朔割名於細君。僕誠不能與此數公者並,故默然獨守吾太玄。」

雄以為賦者,將以風也,必推類而言,極麗靡之辭,閎侈鉅衍,競於使人不能加也,既乃歸之於正,然覽者已過矣。往時武帝好神仙,相如上大人賦,欲以風,帝反縹縹有陵雲之志。繇是言之,賦勸而不止,明矣。又頗似俳優淳于髡、優孟之徒,非法度所存,賢人君子詩賦之正也,於是輟不復為。而大潭思渾天,參摹而四分之,極於八十一。旁則三摹九据,極之七百二十九贊,亦自然之道也。故觀易者,見其卦而名之;觀玄者,數其畫而定之。玄首四重者,非卦也,數也。其用自天元推一晝一夜陰陽數度律曆之紀,九九大運,與天終始。故玄三方、九州、二十七部、八十一家、二百四十三表、七百二十九贊,分為三卷,曰一二三,與泰初歷相應,亦有顓頊之曆焉。飓之以三策,關之以休咎,絣之以象類,播之以人事,文之以五行,擬之以道德仁義禮知。無主無名,要合五經,苟非其事,文不虛生。為其泰曼漶而不可知,故有首、衝、錯、測、攡、瑩、數、文、颗、圖、告十一篇,皆以解剝玄體,離散其文,章句尚不存焉。玄文多,故不著;觀之者難知,學之者難成。客有難玄大深,眾人之不好也,雄解之,號曰解難。其辭曰:

客難揚子曰:「凡著書者,為眾人之所好也,美味期乎合口,工聲調於比耳。今吾子乃抗辭幽說,閎意眇指,獨馳騁於有亡之際,而陶冶大鑪,旁薄群生,歷覽者茲年矣,而殊不寤。亶費精神於此,而煩學者於彼,譬畫者畫於無形,弦者放於無聲,殆不可乎?」

揚子曰:「俞。若夫閎言崇議,幽微之塗,蓋難與覽者同也。昔人有觀象於天,視度於地,察法於人者,天麗且彌,地普而深,昔人之辭,乃玉乃金。彼豈好為艱難哉?勢不得已也。獨不見夫翠虯絳螭之將登虖天,必聳身於倉梧之淵;不階浮雲,翼疾風,虛舉而上升,則不能撠膠葛,騰九閎。日月之經不千里,則不能燭六合,燿八紘;泰山之高不嶕嶢,則不能浡滃雲而散歊烝。是以宓犧氏之作易也,綿絡天地,經以八卦,文王附六爻,孔子錯其象而彖其辭,然後發天地之臧,定萬物之基。典謨之篇,雅頌之聲,不溫純深潤,則不足以揚鴻烈而章緝熙。蓋胥靡為宰,寂寞為尸;大味必淡,大音必希;大語叫叫,大道低回。是以聲之眇者不可同於眾人之耳,形之美者不可棍於世俗之目,辭之衍者不可齊於庸人之聽。今夫弦者,高張急徽,追趨逐耆,則坐者不期而附矣;試為之施咸池,揄六莖,發蕭韶,詠九成,則莫有和也。是故鍾期死,伯牙絕弦破琴而不肯與眾鼓;獿人亡,則匠石輟斤而不敢妄斲。師曠之調鍾,俟知音者之在後也;孔子作春秋,幾君子之前睹也。老聃有遺言,貴知我者希,此非其操與!」

雄見諸子各以其知舛馳,大氐詆訾聖人,即為怪迂,析辯詭辭,以撓世事,雖小辯,終破大道而或眾,使溺於所聞而不自知其非也。及太史公記六國,歷楚漢,記麟止,不與聖人同,是非頗謬於經。故人時有問雄者,常用法應之,譔以為十三卷,象論語,號曰法言。法言文多不著,獨著其目:

天降生民,倥侗顓蒙,恣于情性,聰明不開,訓諸理。譔學行第一。

降周迄孔,成于王道,終後誕章乖離,諸子圖微。譔吾子第二。

事有本真,陳施於億,動不克咸,本諸身。譔修身第三。

芒芒天道,在昔聖考,過則失中,不及則不至,不可姦罔。譔問道第四。

神心曶怳,經緯萬方,事繫諸道德仁誼禮。譔問神第五。

明哲煌煌,旁燭亡疆,遜于不虞,以保天命。譔問明第六。

假言周于天地,贊于神明,幽弘橫廣,絕于邇言。譔寡見第七。

聖人聰明淵懿,繼天測靈,冠于群倫,經諸范。譔五百第八。

立政鼓眾,動化天下,莫上於中和,中和之發,在於哲民情。譔先知第九。

仲尼以來,國君將相卿士名臣參差不齊,壹躯諸聖。譔重黎第十。

仲尼之後,訖于漢道,德行顏、閔,股肱蕭、曹,爰及名將尊卑之條,稱述品藻。譔淵騫第十一。

君子純終領聞,蠢迪檢押,旁開聖則。譔君子第十二。

孝莫大於寧親,寧親莫大於寧神,寧神莫大於四表之驩心。譔孝至第十三。

贊曰:雄之自序云爾。初,雄年四十餘,自蜀來至游京師,大司馬車騎將軍王音奇其文雅,召以為門下史,薦雄待詔,歲餘,奏羽獵賦,除為郎,給事黃門,與王莽、劉歆並。哀帝之初,又與董賢同官。當成、哀、平間,莽、賢皆為三公,權傾人主,所薦莫不拔擢,而雄三世不徙官。及莽篡位,談說之士用符命稱功德獲封爵者甚眾,雄復不侯,以耆老久次轉為大夫,恬於勢利乃如是。實好古而樂道,其意欲求文章成名於後世,以為經莫大於易,故作太玄;傳莫大於論語,作法言;史篇莫善於倉頡,作訓纂;箴莫善於虞箴,作州箴;賦莫深於離騷,反而廣之;辭莫麗於相如,作四賦:皆斟酌其本,相與放依而馳騁云。用心於內,不求於外,於時人皆曶之;唯劉歆及范逡敬焉,而桓譚以為絕倫。

王莽時,劉歆、甄豐皆為上公,莽既以符命自立,即位之後欲絕其原以神前事,而豐子尋、歆子棻復獻之。莽誅豐父子,投棻四裔,辭所連及,便收不請。時雄校書天祿閣上,治獄使者來,欲收雄,雄恐不能自免,乃從閣上自投下,幾死。莽聞之曰:「雄素不與事,何故在此?」間請問其故,乃劉棻嘗從雄學作奇字,雄不知情。有詔勿問。然京師為之語曰:「惟寂寞,自投閣;爰清靜,作符命。」

雄以病免,復召為大夫。家素貧,耆酒,人希至其門。時有好事者載酒肴從游學,而鉅鹿侯芭常從雄居,受其太玄、法言焉。劉歆亦嘗觀之,謂雄曰:「空自苦!今學者有祿利,然尚不能明易,又如玄何?吾恐後人用覆醬瓿也。」雄笑而不應。年七十一,天鳳五年卒,侯芭為起墳,喪之三年。

時大司空王邑、納言嚴尤聞雄死,謂桓譚曰:「子嘗稱揚雄書,豈能傳於後世乎?」譚曰:「必傳。顧君與譚不及見也。凡人賤近而貴遠,親見揚子雲祿位容貌不能動人,故輕其書。昔老聃著虛無之言兩篇,薄仁義,非禮學,然後世好之者尚以為過於五經,自漢文景之君及司馬遷皆有是言。今揚子之書文義至深,而論不詭於聖人,若使遭遇時君,更閱賢知,為所稱善,則必度越諸子矣。」諸儒或譏以為雄非聖人而作經,猶春秋吳楚之君僭號稱王,蓋誅絕之罪也。自雄之沒至今四十餘年,其法言大行,而玄終不顯,然篇籍具存。


\end{pinyinscope}