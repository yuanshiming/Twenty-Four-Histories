\article{高惠高后文功臣表}

\begin{pinyinscope}
自古帝王之興,曷嘗不建輔弼之臣所與共成天功者乎!漢興自秦二世元年之秋,楚陳之歲,初以沛公總帥雄俊,三年然後西滅秦,立漢王之號,五年東克項羽,即皇帝位,八載而天下乃平,始論功而定封。訖十二年,侯者百四十有三人。時大城名都民人散亡,戶口可得而數裁什二三,是以大侯不過萬家,小者五六百戶。封爵之誓曰:「使黃河如帶,泰山若厲,國以永存,爰及苗裔。」於是申以丹書之信,重以白馬之盟,又作十八侯之位次。高后二年,復詔丞相陳平盡差列侯之功,錄弟下竟,臧諸宗廟,副在有司。始未嘗不欲固根本,而枝葉稍落也。

故逮文、景四五世間,流民既歸,戶口亦息,列侯大者至三四萬戶,小國自倍,富厚如之。子孫驕逸,忘其先祖之艱難,多陷法禁,隕命亡國,云子孫。訖於孝武後元之年,靡有孑遺,耗矣。罔亦少密焉。故孝宣皇帝愍而錄之,乃開廟臧,覽舊籍,詔令有司求其子孫,咸出庸保之中,並受復除,或加以金帛,用章中興之德。

降及孝成,復加卹問,稍益衰微,不絕如悋。善乎,杜業之納說也!曰:「昔唐以萬國致時雍之政,虞、夏以之多群后饗共己之治。湯法三聖,殷氏太平。周封八百,重譯來賀。是以內恕之君樂繼絕世,隆名之主安立亡國,至於不及下車,德念深矣。成王察牧野之克,顧群后之勤,知其恩結於民心,功光於王府也,故追述先父之志,錄遺老之策,高其位,大其驱,愛敬飭盡,命賜備厚。大孝之隆,於是為至。至其沒也,世主歎其功,無民而不思。所息之樹且猶不伐,況其廟乎?是以燕、齊之祀與周並傳,子繼弟及,歷載不墮。豈無刑辟,繇祖之竭力,故支庶賴焉。跡漢功臣,亦皆割符世爵,受山河之誓,存以著其號,亡以顯其魂,賞亦不細矣。百餘年間而襲封者盡,或絕失姓,或乏無主,朽骨孤於墓,苗裔流於道,生為愍隸,死為轉屍。以往況今,甚可悲傷。聖朝憐閔,詔求其後,四方忻忻,靡不歸心。出入數年而不省察,恐議者不思大義,設言虛亡,則厚德掩息,遴柬布章,非所以視化勸後也。三人為眾,雖難盡繼,宜從尤功。」於是成帝復紹蕭何。

哀、平之世,增修曹參、周勃之屬,得其宜矣。以綴續前記,究其本末,并序位次,盡于孝文,以昭元功之候籍云。

號諡姓名侯狀戶數始封位次子孫曾孫玄孫


\end{pinyinscope}