\article{趙充國辛慶忌傳}

\begin{pinyinscope}
趙充國字翁孫,隴西上邽人也,後徙金城令居。始為騎士,以六郡良家子善騎射補羽林。為人沈勇有大略,少好將帥之節,而學兵法,通知四夷事。

武帝時,以假司馬從貳師將軍擊匈奴,大為虜所圍。漢軍乏食數日,死傷者多,充國乃與壯士百餘人潰圍陷陳,貳師引兵隨之,遂得解。身被二十餘創,貳師奏狀,詔徵充國詣行在所。武帝親見視其創,嗟歎之,拜為中郎,遷車騎將軍長史。

昭帝時,武都氐人反,充國以大將軍護軍都尉將兵擊定之,遷中郎將,將屯上谷,還為水衡都尉。擊匈奴,獲西祁王,擢為後將軍,兼水衡如故。

與大將軍霍光定冊尊立宣帝,封營平侯。本始中,為蒲類將軍征匈奴,斬虜數百級,還為後將軍、少府。匈奴大發十餘萬騎,南旁塞,至符奚廬山,欲入為寇。亡者題除渠堂降漢言之,遣充國將四萬騎屯緣邊九郡。單于聞之,引去。

是時,光祿大夫義渠安國使行諸羌,先零豪言願時渡湟水北,逐民所不田處畜牧。安國以聞。充國劾安國奉使不敬。是後,羌人旁緣前言,抵冒渡湟水,郡縣不能禁。元康三年,先零遂與諸羌種豪二百餘人解仇交質盟詛。上聞之,以問充國,對曰:「羌人所以易制者,以其種自有豪,數相攻擊,勢不壹也。往三十餘歲,西羌反時,亦先解仇合約攻令居,與漢相距,五六年乃定。至征和五年,先零豪封煎等通使匈奴,匈奴使人至小月氏,傳告諸羌曰:『漢貳師將軍眾十餘萬人降匈奴。羌人為漢事苦。張掖、酒泉本我地,地肥美,可共擊居之。』以此觀匈奴欲與羌合,非一世也。間者匈奴困於西方,聞烏桓來保塞,恐兵復從東方起,數使使尉黎、危須諸國,設以子女貂裘,欲沮解之。其計不合。疑匈奴更遣使至羌中,道從沙陰地,出鹽澤,過長阬,入窮水塞,南抵屬國,與先零相直。臣恐羌變未止此,且復結聯他種,宜及未然為之備。」後月餘,羌侯狼何果遣使至匈奴藉兵,欲擊鄯善、敦煌以絕漢道。充國以為「狼何,小月氏種,在陽關西南,勢不能獨造此計,疑匈奴使已至羌中,先零、罕、腾乃解仇作約。到秋馬肥,變必起矣。宜遣使者行邊兵豫為備,敕視諸羌,毋令解仇,以發覺其謀。」於是兩府復白遣義渠安國行視諸羌,分別善惡。安國至,召先零諸豪三十餘人,以尤桀黠,皆斬之。縱兵擊其種人,斬首千餘級。於是諸降羌及歸義羌侯楊玉等恐怒,亡所信鄉,遂劫略小種,背畔犯塞,攻城邑,殺長吏。安國以騎都尉將騎三千屯備羌,至浩亹,為虜所擊,失亡車重兵器甚眾。安國引還,至令居,以聞。是歲,神爵元年春也。

時充國年七十餘,上老之,使御史大夫丙吉問誰可將者,充國對曰:「亡踰於老臣者矣。」上遣問焉,曰:「將軍度羌虜何如,當用幾人?」充國曰:「百聞不如一見。兵難隃度,臣願馳至金城,圖上方略。然羌戎小夷,逆天背畔,滅亡不久,願陛下以屬老臣,勿以為憂。」上笑曰:「諾。」

充國至金城,須兵滿萬騎,欲渡河,恐為虜所遮,即夜遣三校銜枚先渡,渡輒營陳,會明,畢,遂以次盡渡。虜數十百騎來,出入軍傍。充國曰:「吾士馬新倦,不可馳逐。此皆驍騎難制,又恐其為誘兵也。擊虜以殄滅為期,小利不足貪。」令軍勿擊。遣騎候四望骥中,亡虜。夜引兵上至落都,召諸校司馬,謂曰:「吾知羌虜不能為兵矣。使虜發數千人守杜四望骥中,兵豈得入哉!」充國常以遠斥候為務,行必為戰備,止必堅營壁,尤能持重,愛士卒,先計而後戰。遂西至西部都尉府,日饗軍士,士皆欲為用。虜數挑戰,充國堅守。捕得生口,言羌豪相數責曰:「語汝亡反,今天子遣趙將軍來,年八九十矣,善為兵。今請欲一鬥而死,可得邪!」

充國子右曹中郎將卬,將期門佽飛、羽林孤兒、胡越騎為支兵,至令居。虜並出絕轉道,卬以聞。有詔將八校尉與驍騎都尉、金城太守合疏捕山間虜,通轉道津渡。

初,罕、腾豪靡當兒使弟雕庫來告都尉曰先零欲反,後數日果反。雕庫種人頗在先零中,都尉即留雕庫為質。充國以為亡罪,乃遣歸告種豪:「大兵誅有罪者,明白自別,毋取并滅。天子告諸羌人,犯法者能相捕斬,除罪。斬大豪有罪者一人,賜錢四十萬,中豪十五萬,下豪二萬,大男三千,女子及老小千錢,又以其所捕妻子財物盡與之。」充國計欲以威信招降罕腾及劫略者,解散虜謀,徼極乃擊之。

時上已發三輔、太常徒弛刑,三河、潁川、沛郡、淮陽、汝南材官,金城、隴西、天水、安定、北地、上郡騎士、羌騎,與武威、張掖、酒泉太守各屯其郡者,合六萬人矣。酒泉太守辛武賢奏言:「郡兵皆屯備南山,北邊空虛,勢不可久。或曰至秋冬乃進兵,此虜在竟外之冊。今虜朝夕為寇,土地寒苦,漢馬不能冬,屯兵在武威、張掖、酒泉萬騎以上,皆多羸瘦。可益馬食,以七月上旬齎三十日糧,分兵並出張掖、酒泉合擊罕、腾在鮮水上者。虜以畜產為命,今皆離散,兵即分出,雖不能盡誅,亶奪其畜產,虜其妻子,復引兵還,冬復擊之,大兵仍出,虜必震壞。」

天子下其書充國,令與校尉以下吏士知羌事者博議。充國及長史董通年以為「武賢欲輕引萬騎,分為兩道出張掖,回遠千里。以一馬自佗負三十日食,為米二斛四斗,麥八斛,又有衣裝兵器,難以追逐。勤勞而至,虜必商軍進退,稍引去,逐水屮,入山林。隨而深入,虜即據前險,守後阨,以絕糧道,必有傷危之憂,為夷狄笑,千載不可復。而武賢以為可奪其畜產,虜其妻子,此殆空言,非至計也。又武威縣、張掖日勒皆當北塞,有通谷水草。臣恐匈奴與羌有謀,且欲大入,幸能要杜張掖、酒泉以絕西域,其郡兵尤不可發。先零首為畔逆,它種劫略。故臣愚冊,欲捐罕、腾闇昧之過,隱而勿章,先行先零之誅以震動之,宜悔過反善,因赦其罪,選擇良吏知其俗者撫循和輯,此全師保勝安邊之冊。」天子下其書。公卿議者咸以為先零兵盛,而負罕、腾之助,不先破罕、腾,則先零未可圖也。

上乃拜侍中樂成侯許延壽為強弩將軍,即拜酒泉太守武賢為破羌將軍,賜璽書嘉納其冊。以書敕讓充國曰:

皇帝問後將軍,甚苦暴露。將軍計欲至正月乃擊罕羌,羌人當獲麥,已遠其妻子,精兵萬人欲為酒泉、敦煌寇。邊兵少,民守保不得田作。今張掖以東粟石百餘,芻槁束數十。轉輸並起,百姓煩擾。將軍將萬餘之眾,不早及秋共水草之利爭其畜食,欲至冬,虜皆當畜食,多藏匿山中依險阻,將軍士寒,手足皸瘃,寧有利哉?將軍不念中國之費,欲以歲數而勝微,將軍誰不樂此者!

今詔破羌將軍武賢將兵六千一百人,敦煌太守快將二千人,長水校尉富昌、酒泉侯奉世將婼、月氏兵四千人,亡慮萬二千人。齎三十日食,以七月二十二日擊罕羌,入鮮水北句廉上,去酒泉八百里,去將軍可千二百里。將軍其引兵便道西並進,雖不相及,使虜聞東方北方兵並來,分散其心意,離其黨與,雖不能殄滅,當有瓦解者。已詔中郎將卬將胡越佽飛射士步兵二校,益將軍兵。

今五星出東方,中國大利,蠻夷大敗。太白出高,用兵深入敢戰者吉,弗敢戰者凶。將軍急裝,因天時,誅不義,萬下必全,勿復有疑。

充國既得讓,以為將任兵在外,便宜有守,以安國家。乃上書謝罪,因陳兵利害,曰:

臣竊見騎都尉安國前幸賜書,擇羌人可使使颍,諭告以大軍當至,漢不誅罕,以解其謀。恩澤甚厚,非臣下所能及。臣獨私美陛下盛德至計亡已,故遣腾豪雕庫宣天子至德,罕、腾之屬皆聞知明詔。今先零羌楊玉此羌之首帥名王將騎四千及煎鞏騎五千,阻石山木,候便為寇,罕羌未有所犯。今置先零,先擊罕,釋有罪,誅亡辜,起壹難,就兩害,誠非陛下本計也。

臣聞兵法「攻不足者守有餘」,又曰「善戰者致人,不致於人」。今罕羌欲為敦煌、酒泉寇,飭兵馬,練戰士,以須其至,坐得致敵之術,以逸擊勞,取勝之道也。今恐二郡兵少不足以守,而發之行攻,釋致虜之術而從為虜所致之道,臣愚以為不便。先零羌虜欲為背畔,故與罕、腾解仇結約,然其私心不能亡恐漢兵至而罕、腾背之也。臣愚以為其計常欲先赴罕、腾之急,以堅其約,先擊罕羌,先零必助之。今虜馬肥,糧食方饒,擊之恐不能傷害,適使先零得施德於罕羌,堅其約,合其黨。虜交堅黨合,精兵二萬餘人,迫脅諸小種,附著者稍眾,莫須之屬不輕得離也。如是,虜兵寖多,誅之用力數倍,臣恐國家憂累繇十年數,不二三歲而已。

臣得蒙天子厚恩,父子俱為顯列。臣位至上卿,爵為列侯,犬馬之齒七十六,為明詔填溝壑,死骨不朽,亡所顧念。獨思惟兵利害至孰悉也,於臣之計,先誅先零已,則罕、腾之屬不煩兵而服矣。先零已誅而罕、腾不服,涉正月擊之,得計之理,又其時也。

以今進兵,誠不見其利,唯陛下裁察。

六月戊申奏,七月甲寅璽書報從充國計焉。

充國引兵至先零在所。虜久屯聚,解弛,望見大軍,棄車重,欲渡湟水,道阨狹,充國徐行驅之。或曰逐利行遲,充國曰:「此窮寇不可迫也。緩之則走不顧,急之則還致死。」諸校皆曰:「善。」虜赴水溺死者數百,降及斬首五百餘人,鹵馬牛羊十萬餘頭,車四千餘兩。兵至罕地,令軍毋燔聚落芻牧田中。罕羌聞之,喜曰:「漢果不擊我矣!」豪靡忘使人來言:「願得還復故地。」充國以聞,未報。靡忘來自歸,充國賜飲食,遣還諭種人。護軍以下皆爭之,曰:「此反虜,不可擅遣。」充國曰:「諸君但欲便文自營,非為公家忠計也。」語未卒,璽書報,令靡忘以贖論。後罕竟不煩兵而下。

其秋,充國病,上賜書曰:「制詔後將軍:聞苦腳脛、寒泄,將軍年老加疾,一朝之變不可諱,朕甚憂之。今詔破羌將軍詣屯所,為將軍副,急因天時大利,吏士銳氣,以十二月擊先零羌。即疾劇,留屯毋行,獨遣破羌、彊弩將軍。」時羌降者萬餘人矣。充國度其必壞,欲罷騎兵屯田,以待其敝。作奏未上,會得進兵璽書,中郎將卬懼,使客諫充國曰:「誠令兵出,破軍殺將以傾國家,將軍守之可也。即利與病,又何足爭?一旦不合上意,遣繡衣來責將軍,將軍之身不能自保,何國家之安?」充國歎曰:「是何言之不忠也!本用吾言,羌虜得至是邪?往者舉可先行羌者,吾舉辛武賢,丞相御史復白遣義渠安國,竟沮敗羌。金城、湟中穀斛八錢,吾謂耿中丞,糴二百萬斛穀,羌人不敢動矣。耿中丞請糴百萬斛,乃得四十萬斛耳。義渠再使,且費其半。失此二冊,羌人故敢為逆。失之毫釐,差之千里,是既然矣。今兵久不決,四夷卒有動搖,相因而起,雖有知者不能善其後,羌獨足憂邪!吾固以死守之,明主可為忠言。」遂上屯田奏曰:

臣聞兵者,所以明德除害也,故舉得於外,則福生於內,不可不慎。臣所將吏士馬牛食,月用糧穀十九萬九千六百三十斛,鹽千六百九十三斛,茭卧二十五萬二百八十六石。難久不解,繇役不息。又恐它夷卒有不虞之變,相因並起,為明主憂,誠非素定廟勝之冊。且羌虜易以計破,難用兵碎也,故臣愚以為擊之不便。

計度臨羌東至浩亹,羌虜故田及公田,民所未墾,可二千頃以上,其間郵亭多壞敗者。臣前部士入山,伐材木大小六萬餘枚,皆在水次。願罷騎兵,留弛刑應募,及淮陽、汝南步兵與吏士私從者,合凡萬二百八十一人,用穀月二萬七千三百六十三斛,鹽三百八斛,分屯要害處。冰解漕下,繕鄉亭,浚溝渠,治湟骥以西道橋七十所,令可至鮮水左右。田事出,賦人二十畝。至四月草生,發郡騎及屬國胡騎伉健各千,倅馬什二,就草,為田者遊兵。以充入金城郡,益積畜,省大費。今大司農所轉穀至者,足支萬人一歲食。謹上田處及器用簿,唯陛下裁許。

上報曰:「皇帝問後將軍,言欲罷騎兵萬人留田,即如將軍之計,虜當何時伏誅,兵當何時得決?孰計其便,復奏。」充國上狀曰:

臣聞帝王之兵,以全取勝,是以貴謀而賤戰。戰而百勝,非善之善者也,故先為不可勝,以待敵之可勝。蠻夷習俗雖殊於禮義之國,然其欲避害就利,愛親戚,畏死亡,一也。今虜亡其美地薦草,愁於寄託遠遯,骨肉離心,人有畔志,而明主般師罷兵,萬人留田,順天時,因地利,以待可勝之虜,雖未即伏辜,兵決可期月而望。羌虜瓦解,前後降者萬七百餘人,及受言去者凡七十輩,此坐支解羌虜之具也。

臣謹條不出兵留田便宜十二事。步兵九校,吏士萬人,留屯以為武備,因田致穀,威德並行,一也。又因排折羌虜,令不得歸肥饒之墬,貧破其眾,以成羌虜相畔之漸,二也。居民得並田作,不失農業,三也。軍馬一月之食,度支田士一歲,罷騎兵以省大費,四也。至春省甲士卒,循河湟漕穀至臨羌,以斈羌虜,揚威武,傳世折衝之具,五也。以閒暇時下所伐材,繕治郵亭,充入金城,六也。兵出,乘危徼幸,不出,令反畔之虜竄於風寒之地,離霜露疾疫瘃墯之患,坐得必勝之道,七也。亡經阻遠追死傷之害,八也。內不損威武之重,外不令虜得乘間之勢,九也。又亡驚動河南大腾、小腾使生它變之憂,十也。治湟骥中道橋,令可至鮮水,以制西域,信威千里,從枕席上過師,十一也。大費既省,繇役豫息,以戒不虞,十二也。留屯田得十二便,出兵失十二利。臣充國材下,犬馬齒衰,不識長冊,唯明詔博詳公卿議臣採擇。

上復賜報曰:「皇帝問後將軍,言十二便,聞之。虜雖未伏誅,兵決可期月而望,期月而望者,謂今冬邪,謂何時也?將軍獨不計虜聞兵頗罷,且丁壯相聚,攻擾田者及道上屯兵,復殺略人民,將何以止之?又大腾、小腾前言曰:『我告漢軍先零所在,兵不往擊,久留,得亡效五年時不分別人而并擊我?』其意常恐。今兵不出,得亡變生,與先零為一?將軍孰計復奏。」充國奏曰:

臣聞兵以計為本,故多算勝少算。先零羌精兵今餘不過七八千人,失地遠客,分散飢凍。罕、腾、莫須又頗暴略其羸弱畜產,畔還者不絕,皆聞天子明令相捕斬之賞。臣愚以為虜破壞可日月冀,遠在來春,故曰兵決可期月而望。竊見北邊自敦煌至遼東萬一千五百餘里,乘塞列隧有吏卒數千人,虜數大眾攻之而不能害。今留步士萬人屯田,地勢平易,多高山遠望之便,部曲相保,為塹壘木樵,校聯不絕,便兵弩,飭鬥具。烽火幸通,勢及并力,以逸待勞,兵之利者也。臣愚以為屯田內有亡費之利,外有守禦之備。騎兵雖罷,虜見萬人留田為必禽之具,其土崩歸德,宜不久矣。從今盡三月,虜馬羸瘦,必不敢捐其妻子於他種中,遠涉河山而來為寇。又見屯田之士精兵萬人,終不敢復將其累重還歸故地。是臣之愚計,所以度虜且必瓦解其處,不戰而自破之冊也。至於虜小寇盜,時殺人民,其原未可卒禁。臣聞戰不必勝,不苟接刃;攻不必取,不苟勞眾。誠令兵出,雖不能滅先零,亶能令虜絕不為小寇,則出兵可也。即今同是而釋坐勝之道,從乘危之勢,往終不見利,空內自罷敝,貶重而自損,非所以視蠻夷也。又大兵一出,還不可復留,湟中亦未可空,如是,繇役復發也。且匈奴不可不備,烏桓不可不憂。今久轉運煩費,傾我不虞之用以澹一隅,臣愚以為不便。校尉臨眾幸得承威德,奉厚幣,拊循眾羌,諭以明詔,宜皆鄉風。雖其前辭嘗曰「得亡效五年」,宜亡它心,不足以故出兵。臣竊自惟念,奉詔出塞,引軍遠擊,窮天子之精兵,散車甲於山野,雖亡尺寸之功,媮得避慊之便,而亡後咎餘責,此人臣不忠之利,非明主社稷之福也。臣幸得奮精兵,討不義,久留天誅,罪當萬死。陛下寬仁,未忍加誅,今臣數得孰計。愚臣伏計孰甚,不敢避斧鉞之誅,昧死陳愚,唯陛下省察。

充國奏每上,輒下公卿議臣。初是充國計者什三,中什五,最後什八。有詔詰前言不便者,皆頓首服。丞相魏相曰:「臣愚不習兵事利害,後將軍數畫軍冊,其言常是,臣任其計可必用也。」上於是報充國曰:「皇帝問後將軍,上書言羌虜可勝之道,今聽將軍,將軍計善。其上留屯田及當罷者人馬數。將軍強食,慎兵事,自愛!」上以破羌、強弩將軍數言當擊,又用充國屯田處離散,恐虜犯之,於是兩從其計,詔兩將軍與中郎將卬出擊。強弩出,降四千餘人,破羌斬首二千級,中郎將卬斬首降者亦二千餘級,而充國所降復得五千餘人。詔罷兵,獨充國留屯田。

明年五月,充國奏言:「羌本可五萬人軍,凡斬首七千六百級,降者三萬一千二百人,溺河湟飢餓死者五六千人,定計遺脫與煎鞏、黃羝俱亡者不過四千人。羌靡忘等自詭必得,請罷屯兵。」奏可,充國振旅而還。

所善浩星賜迎說充國,曰:「眾人皆以破羌、強弩出擊,多斬首獲降,虜以破壞。然有識者以為虜勢窮困,兵雖不出,必自服矣。將軍即見,宜歸功於二將軍出擊,非愚臣所及。如此,將軍計未失也。」充國曰:「吾年老矣,爵位已極,豈嫌伐一時事以欺明主哉!兵勢,國之大事,當為後法。老臣不以餘命壹為陛下明言兵之利害,卒死,誰當復言之者?」卒以其意對。上然其計,罷遣辛武賢歸酒泉太守官,充國復為後將軍衛尉。

其秋,羌若零、離留、且種、兒庫共斬先零大豪猶非、楊玉首,及諸豪弟澤、陽雕、良兒、靡忘皆帥煎鞏、黃羝之屬四千餘人降漢。封若零、弟澤二人為帥眾王,離留、且種二人為侯,兒庫為君,陽雕為言兵侯,良兒為君,靡忘為獻牛君。初置金城屬國以處降羌。

詔舉可護羌校尉者,時充國病,四府舉辛武賢小弟湯。充國遽起奏:「湯使酒,不可典蠻夷。不如湯兄臨眾。」時湯已拜受節,有詔更用臨眾。後臨眾病免,五府復舉湯,湯數醉濑羌人,羌人反畔,卒如充國之言。

初,破羌將軍武賢在軍中時與中郎將卬宴語,卬道:「車騎將軍張安世始嘗不快上,上欲誅之,卬家將軍以為安世本持橐簪筆事孝武帝數十年,見謂忠謹,宜全度之。安世用是得免。」及充國還言兵事,武賢罷歸故官,深恨,上書告卬泄省中語。卬坐禁止而入至充國莫府司馬中亂屯兵下吏,自殺。

充國乞骸骨,賜安車駟馬、黃金六十斤,罷就第。朝庭每有四夷大議,常與參兵謀,問籌策焉。年八十六,甘露二年薨,諡曰壯侯。傳子至孫欽,欽尚敬武公主。主亡子,主教欽良人習詐有身,名它人子。欽薨,子岑嗣侯,習為太夫人。岑父母求錢財亡已,忿恨相告。岑坐非子免,國除。元始中,修功臣後,復封充國曾孫伋為營平侯。

初,充國以功德與霍光等列,畫未央宮。成帝時,西羌嘗有警,上思將帥之臣,追美充國,乃召黃門郎楊雄即充國圖畫而頌之,曰:

明靈惟宣,戎有先零。先零昌狂,侵漢西疆。漢命虎臣,惟後將軍,整我六師,是討是震。既臨其域,諭以威德,有守矜功,謂之弗克。請奮其旅,于罕之羌,天子命我,從之鮮陽。營平守節,婁奏封章,料敵制勝,威謀靡亢。遂克西戎,還師於京,鬼方賓服,罔有不庭。昔周之宣,有方有虎,詩人歌功,乃列于雅。在漢中興,充國作武,赳赳桓桓,亦紹厥後。

充國為後將軍,徙杜陵。辛武賢自羌軍還後七年,復為破羌將軍,征烏孫至敦煌,後不出,徵未到,病卒。子慶忌至大官。

辛慶忌字子真,少以父任為右校丞,隨長羅侯常惠屯田烏孫赤谷城,與歙侯戰,陷陳卻敵。惠奏其功,拜為侍郎,遷校尉,將吏士屯焉耆國。還為謁者,尚未知名。元帝初,補金城長史,舉茂材,遷郎中車騎將軍,朝庭多重之者。轉為校尉,遷張掖太守,徙酒泉,所在著名。

成帝初,徵為光祿大夫,遷左曹中郎將,至執金吾。始武賢與趙充國有隙,後充國家殺辛氏,至慶忌為執金吾,坐子殺趙氏,左遷酒泉太守。歲餘,大將軍王鳳薦慶忌「前在兩郡著功跡,徵入,歷位朝廷,莫不信鄉。質行正直,仁勇得眾心,通於兵事,明略威重,任國柱石。父破羌將軍武賢顯名前世,有威西夷。臣鳳不宜久處慶忌之右。」乃復徵為光祿大夫、執金吾。數年,坐小法左遷雲中太守,復徵為光祿勳。

時數有災異,丞相司直何武上封事曰:「虞有宮之奇,晉獻不寐;衛青在位,淮南寢謀。故賢人立朝,折衝厭難,勝於亡形。司馬法曰:『天下雖安,忘戰必危。』夫將不豫設,則亡以應卒;士不素厲,則難使死敵。是以先帝建列將之官,近戚主內,異姓距外,故姦軌不得萌動而破滅,誠萬世之長冊也。光祿勳慶忌行義修正,柔毅敦厚,謀慮深遠。前在邊郡,數破敵獲虜,外夷莫不聞。乃者大異並見,未有其應。加以兵革久寢。春秋大災未至而豫禦之,慶忌宜在爪牙官以備不虞。」其後拜為右將軍諸吏散騎給事中,歲餘徙為左將軍。

慶忌居處恭儉,食飲被服尤節約,然性好輿馬,號為鮮明,唯是為奢。為國虎臣,遭世承平,匈奴、西域親附,敬其威信。年老卒官。長子通為護羌校尉,中子遵函谷關都尉,少子茂水衡都尉出為郡守,皆有將帥之風。宗族支屬至二千石者十餘人。

元始中,安漢公王莽秉政,見慶忌本大將軍鳳所成,三子皆能,欲親厚之。是時莽方立威柄,用甄豐、甄邯以自助,豐、邯新貴,威震朝廷。水衡都尉茂自見名臣子孫,兄弟並列,不甚詘事兩甄。時平帝幼,外家衛氏不得在京師,而護羌校尉通長子次兄素與帝從舅衛子伯相善,兩人俱游俠,賓客甚盛。及呂寬事起,莽誅衛氏。兩甄搆言諸辛陰與衛子伯為心腹,有背恩不說安漢公之謀。於是司直陳崇舉奏其宗親隴西辛興等侵陵百姓,威行州郡。莽遂按通父子、遵茂兄弟及南郡太守辛伯等,皆誅殺之。辛氏繇是廢。慶忌本狄道人,為將軍,徙昌陵。昌陵罷,留長安。

贊曰:秦漢已來,山東出相,山西出將。秦將軍白起,郿人;王翦,頻陽人。漢興,郁郅王圍、甘延壽,義渠公孫賀、傅介子,成紀李廣、李蔡,杜陵蘇建、蘇武,上邽上官桀、趙充國,襄武廉褒,狄道辛武賢、慶忌,皆以勇武顯聞。蘇、辛父子著節,此其可稱列者也,其餘不可勝數。何則?山西天水、隴西、安定、北地處勢迫近羌胡,民俗修習戰備,高上勇力鞍馬騎射。故《秦詩》曰:「王于興師,修我甲兵,與子皆行。」其風聲氣俗自古而然,今之歌謠慷慨,風流猶存耳。


\end{pinyinscope}