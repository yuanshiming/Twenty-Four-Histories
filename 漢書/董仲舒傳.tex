\article{董仲舒傳}

\begin{pinyinscope}
董仲舒,廣川人也。少治春秋,孝景時為博士。下帷講誦,弟子傳以久次相授業,或莫見其面。蓋三年不窺園,其精如此。進退容止,非禮不行,學士皆師尊之。

武帝即位,舉賢良文學之士前後百數,而仲舒以賢良對策焉。

制曰:朕獲承至尊休德,傳之亡窮,而施之罔極,任大而守重,是以夙夜不皇康寧,永惟萬事之統,猶懼有闕。故廣延四方之豪俊,郡國諸侯公選賢良修絜博習之士,欲聞大道之要,至論之極。今子大夫褎然為舉首,朕甚嘉之。子大夫其精心致思,朕垂聽而問焉。

蓋聞五帝三王之道,改制作樂而天下洽和,百王同之。當虞氏之樂莫盛於韶,於周莫盛於勺。聖王已沒,鐘鼓筦絃之聲未衰,而大道微缺,陵夷至虖桀紂之行,王道大壞矣。夫五百年之間,守文之君,當塗之士,欲則先王之法以戴翼其世者甚眾,然猶不能反,日以仆滅,至後王而後止,豈其所持操或誖繆而失其統與?固天降命不可復反,必推之於大衰而後息與?烏虖!凡所為屑屑,夙興夜寐,務法上古者,又將無補與?三代受命,其符安在?災異之變,何緣而起?性命之情,或夭或壽,或仁或鄙,習聞其號,未燭厥理。伊欲風流而令行,刑輕而姦改,百姓和樂,政事宣昭,何脩何飭而膏露降,百穀登,德潤四海,澤臻屮木,三光全,寒暑平,受天之祜,享鬼神之靈,德澤洋溢,施虖方外,延及群生?

子大夫明先聖之業,習俗化之變,終始之序,講聞高誼之日久矣,其明以諭朕。科別其條,勿猥勿并,取之於術,慎其所出。乃其不正不直,不忠不極,枉于執事,書之不泄,興于朕躬,毋悼後害。子大夫其盡心,靡有所隱,朕將親覽焉。

仲舒對曰:

陛下發德音,下明詔,求天命與情性,皆非愚臣之所能及也。臣謹案春秋之中,視前世已行之事,以觀天人相與之際,甚可畏也。國家將有失道之敗,而天乃先出災害以譴告之,不知自省,又出怪異以警懼之,尚不知變,而傷敗乃至。以此見天心之仁愛人君而欲止其亂也。自非大亡道之世者,天盡欲扶持而全安之,事在彊勉而已矣。彊勉學問,則聞見博而知益明;彊勉行道,則德日起而大有功:此皆可使還至而立有效者也。《詩》曰「夙夜匪解」,《書》云「茂哉茂哉!」皆彊勉之謂也。

道者,所繇適於治之路也,仁義禮樂皆其具也。故聖王已沒,而子孫長久安寧數百歲,此皆禮樂教化之功也。王者未作樂之時,乃用先王之樂宜於世者,而以深入教化於民。教化之情不得,雅頌之樂不成,故王者功成作樂,樂其德也。樂者,所以變民風,化民俗也;其變民也易,其化人也著。故聲發於和而本於情,接於肌膚,臧於骨髓。故王道雖微缺,而筦絃之聲未衰也。夫虞氏之不為政久矣,然而樂頌遺風猶有存者,是以孔子在齊而聞韶也。夫人君莫不欲安存而惡危亡,然而政亂國危者甚眾,所任者非其人,而所繇者非其道,是以政日以仆滅也。夫周道衰於幽厲,非道亡也,幽厲不繇也。至於宣王,思昔先王之德,興滯補弊,明文武之功業,周道粲然復興,詩人美之而作,上天祐之,為生賢佐,後世稱誦,至今不絕。此夙夜不解行善之所致也。孔子曰「人能弘道,非道弘人」也。故治亂廢興在於己,非天降命不可得反,其所操持誖謬失其統也。

臣聞天之所大奉使之王者,必有非人力所能致而自至者,此受命之符也。天下之人同心歸之,若歸父母,故天瑞應誠而至。書曰「白魚入于王舟,有火復于王屋,流為烏」,此蓋受命之符也。周公曰「復哉復哉」,孔子曰「德不孤,必有鄰」,皆積善絫德之效也。及至後世,淫佚衰微,不能統理群生,諸侯背畔,殘賊良民以爭壤土,廢德教而任刑罰。刑罰不中,則生邪氣;邪氣積於下,怨惡畜於上。上下不和,則陰陽繆盭而妖孽生矣。此災異所緣而起也。

臣聞命者天之令也,性者生之質也,情者人之欲也。或夭或壽,或仁或鄙,陶冶而成之,不能粹美,有治亂之所生,故不齊也。孔子曰:「君子之德風也,小人之德草也,屮上之風必偃。」故堯舜行德則民仁壽,桀紂行暴則民鄙夭。夫上之化下,下之從上,猶泥之在鈞,唯甄者之所為;猶金之在鎔,唯冶者之所鑄。「綏之斯來,動之斯和」,此之謂也。

臣謹案春秋之文,求王道之端,得之於正。正次王,王次春。春者,天之所為也;正者,王之所為也。其意曰,上承天之所為,而下以正其所為,正王道之端云爾。然則王者欲有所為,宜求其端於天。天道之大者在陰陽。陽為德,陰為刑;刑主殺而德主生。是故陽常居大夏,而以生育養長為事;陰常居大冬,而積於空虛不用之處。以此見天之任德不任刑也。天使陽出布施於上而主歲功,使陰入伏於下而時出佐陽;陽不得陰之助,亦不能獨成歲。終陽以成歲為名,此天意也。王者承天意以從事,故任德教而不任刑。刑者不可任以治世,猶陰之不可任以成歲也。為政而任刑,不順於天,故先王莫之肯為也。今廢先王德教之官,而獨任執法之吏治民,毋乃任刑之意與!孔子曰:「不教而誅謂之虐。」虐政用於下,而欲德教之被四海,故難成也。

臣謹案春秋謂一元之意,一者萬物之所從始也,元者辭之所謂大也。謂一為元者,視大始而欲正本也。春秋深探其本,而反自貴者始。故為人君者,正心以正朝廷,正朝廷以正百官,正百官以正萬民,正萬民以正四方。四方正,遠近莫敢不壹於正,而亡有邪氣奸其間者。是以陰陽調而風雨時,群生和而萬民殖,五穀孰而草木茂,天地之間被潤澤而大豐美,四海之內聞盛德而皆徠臣,諸福之物,可致之祥,莫不畢至,而王道終矣。

孔子曰:「鳳鳥不至,河不出圖,吾已矣夫!」自悲可致此物,而身卑賤不得致也。今陛下貴為天子,富有四海,居得致之位,操可致之勢,又有能致之資,行高而恩厚,知明而意美,愛民而好士,可謂誼主矣。然而天地未應而美祥莫至者,何也?凡以教化不立而萬民不正也。夫萬民之從利也,如水之走下,不以教化隄防之,不能止也。是故教化立而姦邪皆止者,其隄防完也;教化廢而姦邪並出,刑罰不能勝者,其隄防壞也。古之王者明於此,是故南面而治天下,莫不以教化為大務。立大學以教於國,設庠序以化於邑,漸民以仁,摩民以誼,節民以禮,故其刑罰甚輕而禁不犯者,教化行而習俗美也。

聖王之繼亂世也,埽除其跡而悉去之,復修教化而崇起之。教化已明,習俗已成,子孫循之,行五六百歲尚未敗也。至周之末世,大為亡道,以失天下。秦繼其後,獨不能改,又益甚之,重禁文學,不得挾書,棄捐禮誼而惡聞之,其心欲盡滅先王之道,而顓為自恣苟簡之治,故立為天子十四歲而國破亡矣。自古以來,未嘗有以亂濟亂,大敗天下之民如秦者也。其遺毒餘烈,至今未滅,使習俗薄惡,人民嚚頑,抵冒殊扞,孰爛如此之甚者也。孔子曰:「腐朽之木不可彫也,糞土之牆不可圬也。」

今漢繼秦之後,如朽木糞牆矣,雖欲善治之,亡可柰何。法出而姦生,令下而詐起,如以湯止沸,抱薪救火,愈甚亡益也。竊譬之琴瑟不調,甚者必解而更張之,乃可鼓也;為政而不行,甚者必變而更化之,乃可理也。當更張而不更張,雖有良工不能善調也;當更化而不更化,雖有大賢不能善治也。故漢得天下以來,常欲善治而至今不可善治者,失之於當更化而不更化也。古人有言曰:「臨淵羨魚,不如蛛而結網。」今臨政而願治七十餘歲矣,不如退而更化;更化則可善治,善治則災害日去,福祿日來。《詩》云:「宜民宜人,受祿于天。」為政而宜於民者,固當受祿于天。夫仁誼禮知信五常之道,王者所當脩飭也;五者脩飭,故受天之祐,而享鬼神之靈,德施于方外,延及群生也。

天子覽其對而異焉,乃復冊之曰:

制曰:蓋聞虞舜之時,游於巖郎之上,垂拱無為,而天下太平。周文王至於日昃不暇食,而宇內亦治。夫帝王之道,豈不同條共貫與?何逸勞之殊也?

蓋儉者不造玄黃旌旗之飾。及至周室,設兩觀,乘大路,朱干玉戚,八佾陳於庭,而頌聲興。夫帝王之道豈異指哉?或曰良玉不瑑,又曰非文無以輔德,二端異焉。

殷人執五刑以督姦,傷肌膚以懲惡。成康不式,四十餘年天下不犯,囹圄空虛。秦國用之,死者甚眾,刑者相望,秏矣哀哉!

烏虖!朕夙寤晨興,惟前帝王之憲,永思所以奉至尊,章洪業,皆在力本任賢。今朕親耕藉田以為農先,勸孝弟,崇有德,使者冠蓋相望,問勤勞,恤孤獨,盡思極神,功烈休德未始云獲也。今陰陽錯繆,氛氣充塞,群生寡遂,黎民未濟,廉恥貿亂,賢不肖渾淆,未得其真,故詳延特起之士,意庶幾乎!今子大夫待詔百有餘人,或道世務而未濟,稽諸上古之不同,考之于今而難行,毋乃牽於文繫而不得騁歟?將所繇異術,所聞殊方與?各悉對,著于篇,毋諱有司。明其指略,切磋究之,以稱朕意。

仲舒對曰:

臣聞堯受命,以天下為憂,而未以位為樂也,故誅逐亂臣,務求賢聖,是以得舜、禹、稷、镨、咎繇。眾聖輔德,賢能佐職,教化大行,天下和洽,萬民皆安仁樂誼,各得其宜,動作應禮,從容中道。故孔子曰「如有王者,必世而後仁」,此之謂也。堯在位七十載,乃遜于位以禪虞舜。堯崩,天下不歸堯子丹朱而歸舜。舜知不可辟,乃即天子之位,以禹為相,因堯之輔佐,繼其統業,是以垂拱無為而天下治。孔子曰「韶盡美矣,又盡善也」,此之謂也。至於殷紂,逆天暴物,殺戮賢知,殘賊百姓。伯夷、太公皆當世賢者,隱處而不為臣。守職之人皆奔走逃亡,入于河海。天下秏亂,萬民不安,故天下去殷而從周。文王順天理物,師用賢聖,是以閎夭、大顛、散宜生等亦聚於朝廷。受施兆民,天下歸之,故太公起海濱而即三公也。當此之時,紂尚在上,尊卑昏亂,百姓散亡,故文王悼痛而欲安之,是以日昃而不暇食也。孔子作春秋,先正王而繫萬事,見素王之文焉。繇此觀之,帝王之條貫同,然而勞逸異者,所遇之時異也。孔子曰「武盡美矣,未盡善也」,此之謂也。

臣聞制度文采玄黃之飾,所以明尊卑,異貴賤,而勸有德也。故春秋受命所先制者,改正朔,易服色,所以應天也。然則宮室旌旗之制,有法而然者也。故孔子曰:「奢則不遜,儉則固。」儉非聖人之中制也。臣聞良玉不瑑,資質潤美,不待刻瑑,此亡異於達巷黨人不學而自知也。然則常玉不瑑,不成文章;君子不學,不成其德。

臣聞聖王之治天下也,少則習之學,長則材諸位,爵祿以養其德,刑罰以威其惡,故民曉於禮誼而恥犯其上。武王行大誼,平殘賊,周公作禮樂以文之,至於成康之隆,囹圄空虛四十餘年,此亦教化之漸而仁誼之流,非獨傷肌膚之效也。至秦則不然。師申商之法,行韓非之說,憎帝王之道,以貪狼為俗,非有文德以教訓於天下也。誅名而不察實,為善者不必免,而犯惡者未必刑也。是以百官皆飾空言虛辭而不顧實,外有事君之禮,內有背上之心,造偽飾詐,趣利無恥;又好用憯酷之吏,賦斂亡度,竭民財力,百姓散亡,不得從耕織之業,群盜並起。是以刑者甚眾,死者相望,而姦不息,俗化使然也。故孔子曰「導之以政,齊之以刑,民免而無恥」,此之謂也。

今陛下并有天下,海內莫不率服,廣覽兼聽,極群下之知,盡天下之美,至德昭然,施於方外。夜郎、康居,殊方萬里,說德歸誼,此太平之致也。然而功不加於百姓者,殆王心未加焉。曾子曰:「尊其所聞,則高明矣;行其所知,則光大矣。高明光大,不在於它,在乎加之意而已。」願陛下因用所聞,設誠於內而致行之,則三王何異哉!

陛下親耕藉田以為農先,夙寤晨興,憂勞萬民,思惟往古,而務以求賢,此亦堯舜之用心也,然而未云獲者,士素不厲也。夫不素養士而欲求賢,譬猶不瑑玉而求文采也。故養士之大者,莫大虐太學;太學者,賢士之所關也,教化之本原也。今以一郡一國之眾,對亡應書者,是王道往往而絕也。臣願陛下興太學,置明師,以養天下之士,數考問以盡其材,則英俊宜可得矣。今之郡守、縣令,民之師帥,所使承流而宣化也;故師帥不賢,則主德不宣,恩澤不流。今吏既亡教訓於下,或不承用主上之法,暴虐百姓,與姦為市,貧窮孤弱,冤苦失職,甚不稱陛下之意。是以陰陽錯繆,氛氣充塞,群生寡遂,黎民未濟,皆長吏不明,使至於此也。

夫長吏多出於郎中、中郎,吏二千石子弟選郎吏,又以富訾,未必賢也。且古所謂功者,以任官稱職為差,非所謂積日絫久也。故小材雖絫日,不離於小官;賢材雖未久,不害為輔佐。是以有司竭力盡知,務治其業而以赴功。今則不然。

累日以取貴,積久以致官,是以廉恥貿亂,賢不肖渾殽,未得其真。臣愚以為使諸列侯、郡守、二千石各擇其吏民之賢者,歲貢各二人以給宿衛,且以觀大臣之能;所貢賢者有賞,所貢不肖者有罰。夫如是,諸侯、吏二千石皆盡心於求賢,天下之士可得而官使也。遍得天下之賢人,則三王之盛易為,而堯舜之名可及也。毋以日月為功,實試賢能為上,量材而授官,錄德而定位,則廉恥殊路,賢不肖異處矣。陛下加惠,寬臣之罪,令勿牽制於文,使得切磋究之,臣敢不盡愚!

於是天子復冊之。

制曰:蓋聞「善言天者必有徵於人,善言古者必有驗於今」。故朕垂問乎天人之應,上嘉唐虞,下悼桀紂,寖微寖滅寖明寖昌之道,虛心以改。今子大夫明於陰陽所以造化,習於先聖之道業,然而文采未極,豈惑虖當世之務哉?條貫靡竟,統紀未終,意朕之不明與?聽若眩與?夫三王之教所祖不同,而皆有失,或謂久而不易者道也,意豈異哉?今子大夫既已著大道之極,陳治亂之端矣,其悉之究之,孰之復之。詩不云虖?「嗟爾君子,毋常安息,神之聽之,介爾景福。」朕將親覽焉,子大夫其茂明之。

仲舒復對曰:

臣聞論語曰:「有始有卒者,其唯聖人虖!」今陛下幸加惠,留聽於承學之臣,復下明冊,以切其意,而究盡聖德,非愚臣之所能具也。前所上對,條貫靡竟,統紀不終,辭不別白,指不分明,此臣淺陋之罪也。

冊曰:「善言天者必有徵於人,善言古者必有驗於今。」臣聞天者群物之祖也,故遍覆包函而無所殊,建日月風雨以和之,經陰陽寒暑以成之。故聖人法天而立道,亦溥愛而亡私,布德施仁以厚之,設誼立禮以導之。春者天之所以生也,仁者君之所以愛也;夏者天之所以長也,德者君之所以養也;霜者天之所以殺也,刑者君之所以罰也。繇此言之,天人之徵,古今之道也。孔子作春秋,上揆之天道,下質諸人情,參之於古,考之於今。故春秋之所譏,災害之所加也;春秋之所惡,怪異之所施也。書邦家之過,兼災異之變,以此見人之所為,其美惡之極,乃與天地流通而往來相應,此亦言天之一端也。古者修教訓之官,務以德善化民,民已大化之後,天下常亡一人之獄矣。今世廢而不脩,亡以化民,民以故棄行誼而死財利,是以犯法而罪多,一歲之獄以萬千數。以此見古之不可不用也,故春秋變古則譏之。天令之謂命,命非聖人不行;質樸之謂性,性非教化不成;人欲之謂情,情非度制不節。是故王者上謹於承天意,以順命也;下務明教化民,以成性也;正法度之宜,別上下之序,以防欲也:脩此三者,而大本舉矣。人受命於天,固超然異於群生,入有父子兄弟之親,出有君臣上下之誼,會聚相遇,則有耆老長幼之施;粲然有文以相接,驩然有恩以相愛,此人之所以貴也。生五穀以食之,桑麻以衣之,六畜以養之,服牛乘馬,圈豹檻虎,是其得天之靈,貴於物也。故孔子曰:「天地之性人為貴。」明於天性,知自貴於物;知自貴於物,然後知仁誼;知仁誼,然後重禮節;重禮節,然後安處善;安處善,然後樂循理;樂循理,然後謂之君子。故孔子曰「不知命,亡以為君子」,此之謂也。

冊曰:「上嘉唐虞,下悼桀紂,寖微寖滅寖明寖昌之道,虛心以改。」臣聞眾少成多,積小致鉅,故聖人莫不以晻致明,以微致顯。是以堯發於諸侯,舜興虖深山,非一日而顯也,蓋有漸以致之矣。言出於己,不可塞也;行發於身,不可掩也。言行,治之大者,君子之所以動天地也。故盡小者大,慎微者著。《詩》云:「惟此文王,小心翼翼。」故堯兢兢日行其道,而舜業業日致其孝,善積而名顯,德章而身尊,此其寖明寖昌之道也。積善在身,猶長日加益,而人不知也;積惡在身,猶火之銷膏,而人不見也。非明虖情性察虖流俗者,孰能知之?此唐虞之所以得令名,而桀紂之可為悼懼者也。夫善惡之相從,如景鄉之應形聲也。故桀紂暴謾,讒賊並進,賢知隱伏,惡日顯,國日亂,晏然自以如日在天,終陵夷而大壞。夫暴逆不仁者,非一日而亡也,亦以漸至,故桀、紂雖亡道,然猶享國十餘年,此其寖微寖滅之道也。

冊曰:「三王之教所祖不同,而皆有失,或謂久而不易者道也,意豈異哉?」臣聞夫樂而不亂復而不厭者謂之道;道者萬世亡弊,弊者道之失也。先王之道必有偏而不起之處,故政有眊而不行,舉其偏者以補其弊而已矣。三王之道所祖不同,非其相反,將以捄溢扶衰,所遭之變然也。故孔子曰:「亡為而治者,其舜虖!」改正朔,易服色,以順天命而已;其餘盡循堯道,何更為哉!故王者有改制之名,亡變道之實。然夏上忠,殷上敬,周上文者,所繼之捄,當用此也。孔子曰:「殷因於夏禮,所損益可知也;周因於殷禮,所損益可知也;其或繼周者,雖百世可知也。」此言百王之用,以此三者矣。夏因於虞,而獨不言所損益者,其道如一而所上同也。道之大原出於天,天不變,道亦不變,是以禹繼舜,舜繼堯,三聖相受而守一道,亡救弊之政也,故不言其所損益也。繇是觀之,繼治世者其道同,繼亂世者其道變。今漢繼大亂之後,若宜少損周之文致,用夏之忠者。

陛下有明德嘉道,愍世俗之靡薄,悼王道之不昭,故舉賢良方正之士,論誼考問,將欲興仁誼之休德,明帝王之法制,建太平之道也。臣愚不肖,述所聞,誦所學,道師之言,势能勿失耳。若乃論政事之得失,察天下之息秏,此大臣輔佐之職,三公九卿之任,非臣仲舒所能及也。然而臣竊有怪者。夫古之天下亦今之天下,今之天下亦古之天下,共是天下,古

亦大治,上下和睦,習俗美盛,不令而行,不禁而止,吏亡姦邪,民亡盜賊,囹圄空虛,德潤草木,澤被四海,鳳皇來集,麒麟來游,以古準今,壹何不相逮之遠也!安所繆盭而陵夷若是?意者有所失於古之道與?有所詭於天之理與?試跡之古,返之於天,黨可得見乎。

夫天亦有所分予,予之齒者去其角,傅其翼者兩其足,是所受大者不得取小也。古之所予祿者,不食於力,不動於末,是亦受大者不得取小,與天同意者也。夫已受大,又取小,天不能足,而況人乎!此民之所以囂囂苦不足也。身寵而載高位,家溫而食厚祿,因乘富貴之資力,以與民爭利於下,民安能如之哉!是故眾其奴婢,多其牛羊,廣其田宅,博其產業,畜其積委,務此而亡已,以迫蹴民,民日削月朘,寖以大窮。富者奢侈羨溢,貧者窮急愁苦;窮急愁苦而上不救,則民不樂生;民不樂生,尚不避死,安能避罪!此刑罰之所以蕃而姦邪不可勝者也。故受祿之家,食祿而已,不與民爭業,然後利可均布,而民可家足。此上天之理,而亦太古之道,天子之所宜法以為制,大夫之所當循以為行也。故公儀子相魯,之其家見織帛,怒而出其妻,食於舍而茹葵,慍而拔其葵,曰:「吾已食祿,又奪園夫紅女利虖!」古之賢人君子在列位者皆如是,是故下高其行而從其教,民化其廉而不貪鄙。及至周室之衰,其卿大夫緩於誼而急於利,亡推讓之風而有爭田之訟。故詩人疾而刺之,曰:「節彼南山,惟石巖巖,赫赫師尹,民具爾瞻。」爾好誼,則民鄉仁而俗善;爾好利,則民好邪而俗敗。由是觀之,天子大夫者,下民之所視效,遠方之所四面而內望也。近者視而放之,遠者望而效之,豈可以居賢人之位而為庶人行哉!夫皇皇求財利常恐乏匱者,庶人之意也;皇皇求仁義常恐不能化民者,大夫之意也。《易》曰:「負且乘,致寇至。」乘車者君子之位也,負擔者小人之事也,此言居君子之位而為庶人之行者,其患禍必至也。若居君子之位,當君子之行,則舍公儀休之相魯,亡可為者矣。

春秋大一統者,天地之常經,古今之通誼也。今師異道,人異論,百家殊方,指意不同,是以上亡以持一統;法制數變,下不知所守。臣愚以為諸不在六藝之科孔子之術者,皆絕其道,勿使並進。邪辟之說滅息,然後統紀可一而法度可明,民知所從矣。

對既畢,天子以仲舒為江都相,事易王。易王,帝兄,素驕,好勇。仲舒以禮誼匡正,王敬重焉。久之,王問仲舒曰:「粵王句踐與大夫泄庸、種、蠡謀伐吳,遂滅之。孔子稱殷有三仁,寡人亦以為粵有三仁。桓公決疑於管仲,寡人決疑於君。」仲舒對曰:「臣愚不足以奉大對。聞昔者魯君問柳下惠:『吾欲伐齊,何如?』柳下惠曰:『不可。』歸而有憂色,曰:『吾聞伐國不問仁人,此言何為至於我哉!』徒見問耳,且猶羞之,況設詐以伐吳虖?繇此言之,粵本無一仁。夫仁人者,正其誼不謀其利,明其道不計其功,是以仲尼之門,五尺之童羞稱五伯,為其先詐力而後仁誼也。苟為詐而已,故不足稱於大君子之門也。五伯比於他諸侯為賢,其比三王,猶武夫之與美玉也。」王曰:「善。」

仲舒治國,以春秋災異之變推陰陽所以錯行,故求雨,閉諸陽,縱諸陰,其止雨反是;行之一國,未嘗不得所欲。中廢為中大夫。先是遼東高廟、長陵高園殿災,仲舒居家推說其意,草稿未上,主父偃候仲舒,私見,嫉之,竊其書而奏焉。上召視諸儒,仲舒弟子呂步舒不知其師書,以為大愚。於是下仲舒吏,當死,詔赦之。仲舒遂不敢復言災異。

仲舒為人廉直。是時方外攘四夷,公孫弘治春秋不如仲舒,而弘希世用事,位至公卿。仲舒以弘為從諛,弘嫉之。膠西王亦上兄也,尤縱恣,數害吏二千石。弘乃言於上曰:「獨董仲舒可使相膠西王。」膠西王聞仲舒大儒,善待之,仲舒恐久獲罪,病免。凡相兩國,輒事驕王,正身以率下,數上疏諫爭,教令國中,所居而治。及去位歸居,終不問家產業,以修學著書為事。

仲舒在家,朝廷如有大議,使使者及廷尉張湯就其家而問之,其對皆有明法。自武帝初立,魏其、武安侯為相而隆儒矣。及仲舒對冊,推明孔氏,抑黜百家。立學校之官,州郡舉茂材孝廉,皆自仲舒發之。年老,以壽終於家。家徙茂陵,子及孫皆以學至大官。

仲舒所著,皆明經術之意,及上疏條教,凡百二十三篇。而說春秋事得失,聞舉、玉杯、蕃露、清明、竹林之屬,復數十篇,十餘萬言,皆傳於後世。掇其切當世施朝廷者著于篇。

贊曰:劉向稱「董仲舒有王佐之材,雖伊呂亡以加,筦晏之屬,伯者之佐,殆不及也。」至向子歆以為「伊呂乃聖人之耦,王者不得則不興。故顏淵死,孔子曰『噫!天喪余。』唯此一人為能當之,自宰我、子贛、子游、子夏不與焉。仲舒遭漢承秦滅學之後,六經離析,下帷發憤,潛心大業,令後學者有所統壹,為群儒首,然考其師友淵源所漸,猶未及乎游夏,而曰筦晏弗及,伊呂不加,過矣。」至向曾孫龔,篤論君子也,以歆之言為然。


\end{pinyinscope}