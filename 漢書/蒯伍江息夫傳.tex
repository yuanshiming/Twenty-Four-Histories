\article{蒯伍江息夫傳}

\begin{pinyinscope}
蒯通,范陽人也,本與武帝同諱。楚漢初起,武臣略定趙地,號武信君。通說范陽令徐公曰:「臣,范陽百姓蒯通也,竊閔公之將死,故弔之。雖然,賀公得通而生也。」徐公再拜曰:「

何以弔之?」通曰:「足下為令十餘年矣,殺人之父,孤人之子,斷人之足,黥人之首,甚眾。慈父孝子所以不敢事刃於公之腹者,畏秦法也。今天下大亂,秦政不施,然則慈父孝子將爭接刃於公之腹,以復其怨而成其功名。此通之所以弔者也。」曰:「何以賀得子而生也?」曰:「趙武信君不知通不肖,使人候問其死生,通且見武信君而說之,曰:『必將戰勝而後略地,攻得而後下城,臣竊以為殆矣。用臣之計,毋戰而略地,不攻而下城,傳檄而千里定,可乎?』彼將曰:『何謂也?』臣因對曰:『范陽令宜整頓其士卒以守戰者也,怯而畏死,貪而好富貴,故欲以其城先下君。先下君而君不利,則邊地之城皆將相告曰「

范陽令先降而身死」,必將嬰城固守,皆為金城湯池,不可攻也。為君計者,莫若以黃屋朱輪迎范陽令,使馳騖於燕趙之郊,則邊城皆將相告曰「范陽令先下而身富貴」,必相率而降,猶如阪上走丸也。此臣所謂傳檄而千里定者也。』」徐公再拜,具車馬遣通。通遂以此說武臣。武臣以車百乘,騎二百,侯印迎徐公。燕趙聞之,降者三十餘城,如通策焉。

後漢將韓信虜魏王,破趙、代,降燕,定三國,引兵將東擊齊。未度平原,聞漢王使酈食其說下齊,信欲止。通說信曰:「將軍受詔擊齊,而漢獨發間使下齊,寧有詔止將軍乎?何以得無行!且酈生一士,伏軾掉三寸舌,下齊七十餘城,將軍將數萬之眾,乃下趙五十餘城。為將數歲,反不如一豎儒之功乎!」於是信然之,從其計,遂度河。齊巳聽酈生,即留之縱酒,罷備漢守禦。信因襲歷下軍,遂至臨菑。齊王以酈生為欺己而亨之,因敗走。信遂定齊地,自立為齊假王。漢方困於滎陽,遣張良即立信為齊王,以安固之。項王亦遣武涉說信,欲與連和。

蒯通知天下權在信,欲說信令背漢,乃先微感信曰:「僕嘗受相人之術,相君之面,不過封侯,又危而不安;相君之背,貴而不可言。」信曰:「何謂也?」通因請間,曰:「天下初作難也,俊雄豪桀建號壹呼,天下之士雲合霧集,魚鱗雜襲,飄至風起。當此之時,憂在亡秦而已。今劉、項分爭,使人肝腦塗地,流離中野,不可勝數。漢王將數十萬眾,距鞏、雒,岨山河,一日數戰,無尺寸之功,折北不救,敗滎陽,傷成皋,還走宛、葉之間,此所謂智勇俱困者也。楚人起彭城,轉鬥逐北,至滎陽,乘利席勝,威震天下,然兵困於京、索之間,迫西山而不能進,三年於此矣。銳氣挫於嶮塞,砕食盡於內藏,百姓罷極,無所歸命。以臣料之,非天下賢聖,其勢固不能息天下之禍。當今之時,兩主縣命足下。足下為漢則漢勝,與楚則楚勝。臣願披心腹,墮肝膽,效愚忠,恐足下不能用也。方今為足下計,莫若兩利而俱存之,參分天下,鼎足而立,其勢莫敢先動。夫以足下之賢聖,有甲兵之眾,據彊齊,從燕、趙,出空虛之地以制其後,因民之欲,西鄉為百姓請命,天下孰敢不聽!足下按齊國之故,有淮泗之地,懷諸侯以德,深拱揖讓,則天下君王相率而朝齊矣。蓋聞『天與弗取,反受其咎;時至弗行,反受其殃。』願足下孰圖之。」

信曰:「漢遇我厚,吾豈可見利而背恩乎!」通曰:「始常山王、成安君故相與為刎頸之交,及爭張黶、陳釋之事,常山王奉頭鼠竄,以歸漢王。借兵東下,戰於鄗北,成安君死於泜水之南,頭足異處。此二人相與,天下之至驩也,而卒相滅亡者,何也?患生於多欲而人心難測也。今足下行忠信以交於漢王,必不能固於二君之相與也,而事多大於張黶、陳釋之事者,故臣以為足下必漢王之不危足下,過矣。大夫種存亡越,伯句踐,立功名而身死。語曰:『野禽殫,走犬亨;敵國破,謀臣亡。』故以交友言之,則不過張王與成安君;以忠臣言之,則不過大夫種。此二者,宜足以觀矣。願足下深慮之。且臣聞之,勇略震主者身危,功蓋天下者不賞。足下涉西河,虜魏王,禽夏說,下井陘,誅成安君之罪,以令於趙,脅燕定齊,南摧楚人之兵數十萬眾,遂斬龍且,西鄉以報,此所謂功無二於天下,略不世出者也。今足下挾不賞之功,戴震主之威,歸楚,楚人不信;歸漢,漢人震恐。足下欲持是安歸乎?夫勢在人臣之位,而有高天下之名,切為足下危之。」信曰:「生且休矣,吾將念之。」

數日,通復說曰:「聽者,事之候也;計者,存亡之機也。夫隨廝養之役者,失萬乘之權;守儋石之祿者,闕卿相之位。計誠知之,而決弗敢行者,百事之禍也。故猛虎之猶與,不如蜂蠆之致酿;孟賁之狐疑,不如童子之必至。此言貴能行之也。夫功者難成而易敗,時者難值而易失。『時乎時,不再來。』願足下無疑臣之計。」信猶與不忍背漢,又自以功多,漢不奪我齊,遂謝通。通說不聽,惶恐,乃陽狂為巫。

天下既定,後信以罪廢為淮陰侯,謀反被誅,臨死歎曰:「悔不用蒯通之言,死於女子之手!」高帝曰:「是齊辯士蒯通。」乃詔齊召蒯通。通至,上欲亨之,曰:「若教韓信反,何也?」通曰:「狗各吠非其主。當彼時,臣獨知齊王韓信,非知陛下也。且秦失其鹿,天下共逐之,高材者先得。天下匈匈,爭欲為陛下所為,顧力不能,可殫誅邪!」上乃赦之。

至齊悼惠王時,曹參為相,禮下賢人,請通為客。

初,齊王田榮怨項羽,謀舉兵畔之,劫齊士,不與者死。齊處士東郭先生、梁石君在劫中,強從。及田榮敗,二人醜之,相與入深山隱居。客謂通曰:「先生之於曹相國,拾遺舉過,顯賢進能,齊國莫若先生者。先生知梁石君、東郭先生世俗所不及,何不進之於相國乎?」通曰:「諾。臣之里婦,與里之諸母相善也。里婦夜亡肉,姑以為盜,怒而逐之。婦晨去,過所善諸母,語以事而謝之。里母曰:『女安行,我今令而家追女矣。』即束縕請火於亡肉家,曰:『昨暮夜,犬得肉,爭鬥相殺,請火治之。』亡肉家遽追呼其婦。故里母非談說之士也,束縕乞火非還婦之道也,然物有相感,事有適可。臣請乞火於曹相國。」乃見相國曰:「婦人有夫死三日而嫁者,有幽居守寡不出門者,足下即欲求婦,何取?」曰:「取不嫁者。」通曰:「然則求臣亦猶是也,彼東郭先生、梁石君,齊之俊士也,隱居不嫁,未嘗卑節下意以求仕也。願足下使人禮之。」曹相國曰:「敬受命。」皆以為上賓。

通論戰國時說士權變,亦自序其說,凡八十一首,號曰雋永。

初,通善齊人安其生,安其生嘗干項羽,羽不能用其策。而項羽欲封此兩人,兩人卒不肯受。

伍被,楚人也。或言其先伍子胥後也。被以材能稱,為淮南中郎。是時淮南王安好術學,折節下士,招致英雋以百數,被為冠首。

久之,淮南王陰有邪謀,被數微諫。後王坐東宮,召被欲與計事,呼之曰:「將軍上。」被曰:「王安得亡國之言乎?昔子胥諫吳王,吳王不用,乃曰『臣今見麋鹿游姑蘇之臺也。』今臣亦將見宮中生荊棘,露霑衣也。」於是王怒,繫被父母,囚之三月。

王復召被曰:「將軍許寡人乎?」對曰:「不,臣將為大王畫計耳。臣聞聰者聽於無聲,明者見於未形,故聖人萬舉而萬全。文王壹動而功顯萬世,列為三王,所謂因天心以動作者也。」王曰:「方今漢庭治乎?亂乎?」被曰:「天下治。」王不說曰:「公何以言治也?」被對曰:「被竊觀朝廷,君臣父子夫婦長幼之序皆得其理,上之舉錯遵古之道,風俗紀綱未有所缺。重裝富賈周流天下,道無不通,交易之道行。南越賓服,羌、僰貢獻,東甌入朝,廣長榆,開朔方,匈奴折傷。雖未及古太平時,然猶為治。」王怒,被謝死罪。

王又曰:「山東即有變,漢必使大將軍將而制山東,公以為大將軍何如人也?」被曰:「臣所善黃義,從大將軍擊匈奴,言大將軍遇士大夫以禮,與士卒有恩,眾皆樂為用。騎上下山如飛,材力絕人如此,數將習兵,未易當也。及謁者曹梁使長安來,言大將軍號令明,當敵勇,常為士卒先;須士卒休,乃舍;穿井得水,乃敢飲;軍罷,士卒已踰河,乃度。皇太后所賜金錢,盡以賞賜。雖古名將不過也。」王曰:「夫蓼太子知略不世出,非常人也,以為漢廷公卿列侯皆如沐猴而冠耳。」被曰:「獨先刺大將軍,乃可舉事。」

王復問被曰:「公以為吳舉兵非邪?」被曰:「非也。夫吳王賜號為劉氏祭酒,受几杖而不朝,王四郡之眾,地方數千里,采山銅以為錢,煮海水以為鹽,伐江陵之木以為船,國富民眾,行珍寶,賂諸侯,與七國合從,舉兵而西,破大梁,敗狐父,奔走而還,為越所禽,死於丹徒,頭足異處,身滅祀絕,為天下戮。夫以吳眾不能成功者,何也?誠逆天違眾而不見時也。」王曰:「男子之所死者,一言耳。且吳何知反?漢將一日過成皋者四十餘人。今我令緩先要成皋之口,周被下潁川兵塞轘轅、伊闕之道,陳定發南陽兵守武關。河南太守獨有雒陽耳,何足憂?然此北尚有臨晉關、河東、上黨與河內、趙國界者通谷數行。人言『絕成皋之道,天下不通』。據三川之險,招天下之兵,公以為何如?」被曰:「臣見其禍,未見其福也。」

後漢逮淮南王孫建,繫治之。王恐陰事泄,謂被曰:「事至,吾欲遂發。天下勞苦有間矣,諸侯頗有失行,皆自疑,我舉兵西鄉,必有應者;無應,即還略衡山。勢不得不發。」被曰:「

略衡山以擊盧江,有尋陽之船,守下雉之城,結九江之浦,絕豫章之口,強弩臨江而守,以禁南郡之下,東保會稽,南通勁越,屈強江淮間,可以延歲月之壽耳,未見其福也。」王曰:「左吳、趙賢、朱驕如皆以為什八九成,公獨以為無福,何?」被曰:「大王之群臣近幸素能使眾者,皆前繫詔獄,餘無可用者。」王曰:「陳勝、吳廣無立錐之地,百人之聚,起於大澤,奮臂大呼,天下嚮應,西至於戲而兵百二十萬。今吾國雖小,勝兵可得二十萬,公何以言有禍無福?」被曰:「臣不敢避子胥之誅,願大王無為吳王之聽。往者秦為無道,殘賊天下,殺術士,燔詩書,滅聖跡,棄禮義,任刑法,轉海濱之粟,致于西河。當是之時,男子疾耕不足於糧餽,女子紡績不足於蓋形。遣蒙恬築長城,東西數千里。暴兵露師,常數十萬,死者不可勝數,僵尸滿野,流血千里。於是百姓力屈,欲為亂者十室而五。又使徐福入海求仙藥,多齎珍寶,童男女三千人,五種百工而行。徐福得平原大澤,止王不來。於是百姓悲痛愁思,欲為亂者十室而六。又使尉佗踰五嶺,攻百越,尉佗知中國勞極,止王南越。行者不還,往者莫返,於是百姓離心瓦解,欲為亂者十室而七。興萬乘之駕,作阿房之宮,收太半之賦,發閭左之戍。父不寧子,兄不安弟,政苛刑慘,民皆引領而望,傾耳而聽,悲號仰天,叩心怨上,欲為亂者,十室而八。客謂高皇帝曰:『時可矣。』高帝曰:『待之,聖人當起東南。』間不一歲,陳、吳大呼,劉、項並和,天下嚮應,所謂蹈瑕釁,因秦之亡時而動,百姓願之,若枯旱之望雨,故起於行陳之中,以成帝王之功。今大王見高祖得天下之易也,獨不觀近世之吳楚乎!當今陛下臨制天下,壹齊海內,氾愛蒸庶,布德施惠。口雖未言,聲疾雷震;令雖未出,化馳如神。心有所懷,威動千里;下之應上,猶景嚮也。而大將軍材能非直章邯、楊熊也。王以陳勝、吳廣論之,被以為過矣。且大王之兵眾不能什分吳楚之一,天下安寧又萬倍於秦時。願王用臣之計。臣聞箕子過故國而悲,作麥秀之歌,痛紂之不用王子比干之言也。故孟子曰,紂貴為天子,死曾不如匹夫。是紂先自絕久矣,非死之日天去之也。今臣亦竊悲大王棄千乘之君,將賜絕命之書,為群臣先,身死于東宮也。」被因流涕而起。

後王復召問被:「苟如公言,不可以徼幸邪?」被曰:「

必不得已,被有愚計。」王曰:「柰何?」被曰:「當今諸侯無異心,百姓無怨氣。朔方之郡土地廣美,民徙者不足以實其地。可為丞相、御史請書,徙郡國豪桀及耐罪以上,以赦令除,家產五十萬以上者,皆徙其家屬朔方之郡,益發甲卒,急其會日。又偽為左右都司空上林中都官詔獄書,逮諸侯太子及幸臣。如此,則民怨,諸侯懼,即使辯士隨而說之,黨可以徼幸。」王曰:「此可也。雖然,吾以不至若此,專發而已。」後事發覺,被詣吏自告與淮南王謀反縱跡如此。天子以伍被雅辭多引漢美,欲勿誅。張湯進曰:「被首為王畫反計,罪無赦。」遂誅被。

江充字次倩,趙國邯鄲人也。充本名齊,有女弟善鼓琴歌舞,嫁之趙太子丹。齊得幸於敬肅王,為上客。

久之,太子疑齊以己陰私告王,與齊忤,使吏逐捕齊,不得,收繫其父兄,按驗,皆棄市。齊遂絕跡亡,西入關,更名充。詣闕告太子丹與同產姊及王後宮姦亂,交通郡國豪猾,攻剽為姦,吏不能禁。書奏,天子怒,遣使者詔郡發吏卒圍趙王宮,收捕太子丹,移繫魏郡詔獄,與廷尉雜治,法至死。

趙王彭祖,帝異母兄也,上書訟太子罪,言「充逋逃小臣,苟為姦訛,激怒聖朝,欲取必於萬乘以復私怨。後雖亨醢,計猶不悔。臣願選從趙國勇敢士,從軍擊匈奴,極盡死力,以贖丹罪。」上不許,竟敗趙太子。

初,充召見犬臺宮,自請願以所常被服冠見上。上許之。充衣紗縠襌衣,曲裾後垂交輸,冠襌纚步搖冠,飛翮之纓。充為人魁岸,容貌甚壯。帝望見而異之,謂左右曰:「燕趙固多奇士。」既至前,問以當世政事,上說之。

充因自請,願使匈奴。詔問其狀,充對曰:「因變制宜,以敵為師,事不可豫圖。」上以充為謁者,使匈奴還,拜為直指繡衣使者,督三輔盜賊,禁察踰侈。貴戚近臣多奢僭,充皆舉劾,奏請沒入車馬,令身待北軍擊匈奴。奏可。充即移書光祿勳中黃門,逮名近臣侍中諸當詣北軍者,移劾門衛,禁止無令得出入宮殿。於是貴戚子弟惶恐,皆見上叩頭求哀,願得入錢贖罪。上許之,令各以秩次輸錢北軍,凡數千萬。上以充忠直,奉法不阿,所言中意。

充出,逢館陶長公主行馳道中。充呵問之,公主曰:「有太后詔。」充曰:「獨公主得行,車騎皆不得。」盡劾沒入官。

後充從上甘泉,逢太子家使乘車馬行馳道中,充以屬吏。太子聞之,使人謝充曰:「非愛車馬,誠不欲令上聞之,以教敕亡素者。唯江君寬之!」充不聽,遂白奏。上曰:「

人臣當如是矣。」大見信用,威震京師。

遷為水衡都尉,宗族知友多得其力者。久之,坐法免。

會陽陵朱安世告丞相公孫賀子太僕敬聲為巫蠱事,連及陽石、諸邑公主,賀父子皆坐誅。語在賀傳。後上幸甘泉,疾病,充見上年老,恐晏駕後為太子所誅,因是為姦,奏言上疾祟在巫蠱。於是上以充為使者治巫蠱。充將胡巫掘地求偶人,捕蠱及夜祠,視鬼,染汙令有處,輒收捕驗治,燒鐵鉗灼,強服之。民轉相誣以巫蠱,吏輒劾以大逆亡道,坐而死者前後數萬人。

是時,上春秋高,疑左右皆為蠱祝詛,有與亡,莫敢訟其冤者。充既知上意,因言宮中有蠱氣,先治後宮希幸夫人,以次及皇后,遂掘蠱於太子宮,得桐木人。太子懼,不能自明,收充,自臨斬之。罵曰:「趙虜!亂乃國王父子不足邪!乃復亂吾父子也!」太子繇是遂敗。語在戾園傳。後武帝知充有詐,夷充三族。

息夫躬字子微,河內河陽人也。少為博士弟子,受春秋,通覽記書。容貌壯麗,為眾所異。

哀帝初即位,皇后父特進孔鄉侯傅晏與躬同郡,相友善,躬繇是以為援,交游日廣。先是,長安孫寵亦以游說顯名,免汝南太守,與躬相結,俱上書,召待詔。是時哀帝被疾,始即位,而人有告中山孝王太后祝詛上,太后及弟宜鄉侯馮參皆自殺,其罪不明。是後無鹽危山有石自立,開道。躬與寵謀曰:「上亡繼嗣,體久不平,關東諸侯,心爭陰謀。今無鹽有大石自立,聞邪臣託往事,以為大山石立而先帝龍興。東平王雲以故與其后日夜祠祭祝詛上,欲求非望。而后舅伍宏反因方術以醫技得幸,出入禁門。霍顯之謀將行於杯杓,荊軻之變必起於帷幄。事勢若此,告之必成;發國姦,誅主讎,取封侯之計也。」躬、寵乃與中郎右師譚,共因中常侍宋弘上變事告焉。上惡之,下有司案驗,東平王雲、雲后謁及伍宏等皆坐誅。上擢寵為南陽太守,譚穎川都尉,弘、躬皆光祿大夫左曹給事中。是時侍中董賢愛幸,上欲侯之,遂下詔云:「躬、寵因賢以聞,封賢為高安侯,寵為方陽侯,躬為宜陵侯,食邑各千戶。賜譚爵關內侯,食邑。」丞相王嘉內疑東平獄事,爭不欲侯賢等,語在嘉傳。嘉固言董賢泰盛,寵、躬皆傾覆有佞邪材,恐必撓亂國家,不可任用。嘉以此得罪矣。

躬既親近,數進見言事,論議亡所避。眾畏其口,見之仄目。躬上疏歷詆公卿大臣,曰:「方今丞相王嘉健而蓄縮,不可用。御史大夫賈延墮弱不任職。左將軍公孫祿、司隸鮑宣皆外有直項之名,內實騃不曉政事。諸曹以下僕速不足數。卒有彊弩圍城,長戟指闕,陛下誰與備之?如使狂夫嘄謼於東崖,匈奴飲馬於渭水,邊竟雷動,四野風起,京師雖有武蜂精兵,未有能窺左足而先應者也。軍書交馳而輻湊,羽檄重跡而押至,小夫懦臣之徒憒眊不知所為。其有犬馬之決者,仰藥而伏刃,雖加夷滅之誅,何益禍敗之至哉!」

躬又言:「秦開鄭國渠以富國彊兵,今為京師,土地肥饒,可度地勢水泉,廣溉灌之利。」天子使躬持節領護三輔都水。躬立表,欲穿長安城,引漕注太倉下以省轉輸。議不可成,乃止。

董賢貴幸日盛,丁、傅害其寵,孔鄉侯晏與躬謀,欲求居位輔政。會單于當來朝,遣使言病,願朝明年。躬因是而上奏,以為「單于當以十一月入塞,後以病為解,疑有他變。烏孫兩昆彌弱,卑爰疐強盛,居彊煌之地,擁十萬之眾,東結單于,遣子往侍。如因素彊之威,循烏孫就屠之跡,舉兵南伐,并烏孫之勢也。烏孫并,則匈奴盛,而西域危矣。可令降胡詐為卑爰疐使者來上書曰:『所以遣子侍單于者,非親信之也,實畏之耳。唯天子哀,告單于歸臣侍子。願助戊己校尉保惡都奴之界。』因下其章諸將軍,令匈奴客聞焉。則是所謂『上兵伐謀,其次伐交』者也。」

書奏,上引見躬,召公卿將軍大議。左將軍公孫祿以為「中國常以威信懷伏夷狄,躬欲逆詐造不信之謀,不可許。且匈奴賴先帝之德,保塞稱蕃。今單于以疾病不任奉朝賀,遣使自陳,不失臣子之禮。臣祿自保沒身不見匈奴為邊竟憂也。」躬掎祿曰:「臣為國家計幾先,謀將然,豫圖未形,為萬世慮。而左將軍公孫祿欲以其犬馬齒保目所見。臣與祿異議,未可同日語也。」上曰:「善。」乃罷群臣,獨與躬議。

因建言:「往年熒惑守心,太白高而芒光,又角星茀於河鼓,其法為有兵亂。是後訛言行詔籌,經歷郡國,天下騷動,恐必有非常之變。可遣大將軍行邊兵,敕武備,斬一郡守,以立威,震四夷,因以厭應變異。」上然之,以問丞相。丞相嘉對曰:「臣聞動民以行不以言,應天以實不以文。下民微細,猶不可詐,況於上天神明而可欺哉!天之見異,所以敕戒人君,欲令覺悟反正,推誠行善。民心說而天意得矣。辯士見一端,或妄以意傅著星曆,虛造匈奴、烏孫、西羌之難,謀動干戈,設為權變,非應天之道也。守相有罪,車馳詣闕,交臂就死,恐懼如此,而談說者云,動安之危,辯口快耳,其實未可從。夫議政者,苦其諂諛傾險辯慧深刻也。諂諛則主德毀,傾險則下怨恨,辯慧則破正道,深刻則傷恩惠。昔秦繆公不從百里奚、蹇叔之言,以敗其師,悔過自責,疾詿誤之臣,思黃髮之言,名垂於後世。唯陛下觀覽古戒,反覆參考,無以先入之語為主。」

上不聽,遂下詔曰:「間者災變不息,盜賊眾多,兵革之徵,或頗著見。未聞將軍惻然深以為意,簡練戎士,繕修干戈。器用盬惡,孰當督之!天下雖安,忘戰必危。將軍與中二千石舉明習兵法有大慮者各一人,將軍二人,詣公車。」就拜孔鄉侯傅晏為大司馬衛將軍,陽安侯丁明又為大司馬票騎將軍。

是日,日有食之,董賢因此沮躬、晏之策。後數日,收晏衛將軍印綬,而丞相御史奏躬罪過。上繇是惡躬等,下詔曰:「南陽太守方陽侯寵,素亡廉聲,有酷惡之資,毒流百姓。左曹光祿大夫宜陵侯躬,虛造詐諼之策,欲以詿誤朝廷。皆交遊貴戚,趨權門,為名。其免躬、寵官,遣就國。」

躬歸國,未有第宅,寄居丘亭。姦人以為侯家富,常夜守之。躬邑人河內掾賈惠往過躬,教以祝盜方,以桑東南指枝為匕,畫北斗七星其上,躬夜自被髮,立中庭,向北斗,持匕招指祝盜。人有上書言躬懷怨恨,非笑朝廷所進,侯星宿,視天子吉凶,與巫同祝詛。上遣侍御史、廷尉監逮躬,繫雒陽詔獄。欲掠問,躬仰天大謼,因僵仆。吏就問,云咽已絕,血從鼻耳出。食頃,死。黨友謀議相連下獄百餘人。躬母聖,坐祠灶祝詛上,大逆不道。聖棄市,妻充漢與家屬徙合浦。躬同族親屬素所厚者,皆免廢錮。哀帝崩,有司奏:「方陽侯寵及右師譚等,皆造作姦謀,罪及王者骨肉,雖蒙赦令,不宜處爵位,在中土。」皆免寵等,徙合浦郡。

初,躬待詔,數危言高論,自恐遭害,著絕命辭曰:「玄雲泱鬱,將安歸兮!鷹隼橫厲,鸞俳佪兮!矰若浮猋,動則機兮!藂棘荠荠,曷可棲兮!發忠忘身,自繞罔兮!冤頸折翼,庸得往兮!涕泣流兮萑蘭,心結愲兮傷肝。虹蜺曜兮日微,孽杳冥兮未開。痛入天兮鳴謼,冤際絕兮誰語!仰天光兮自列,招上帝兮我察。秋風為我吟,浮雲為我陰。嗟若是兮欲何留,撫神龍兮髓其須。游曠迥兮反亡期,雄失據兮世我思。」後數年乃死,如其文。

贊曰:仲尼「惡利口之覆邦家」,蒯通一說而喪三侊,其得不亨者,幸也。伍被安於危國,身為謀主,忠不終而詐讎,誅夷不亦宜乎!書放四罪,詩歌青蠅,春秋以來,禍敗多矣。昔子翬謀桓而魯隱危,欒書搆郤而晉厲弒。豎牛奔仲,叔孫卒;郈伯毀季,昭公逐;費忌納女,楚建走;宰嚭讒胥,夫差喪;李園進妹,春申斃;上官訴屈,懷王執;趙高敗斯,二世縊;伊戾坎盟,宋痤死;江充造蠱,太子殺;息夫作姦,東平誅:皆自小覆大,繇疏陷親,可不懼哉!可不懼哉!


\end{pinyinscope}