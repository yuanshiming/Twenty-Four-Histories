\article{刑法志}

\begin{pinyinscope}
夫人宵天地之䫉,懷五常之性,聰明精粹,有生之最靈者也。爪牙不足以供耆欲,趨走不足以避利害,無毛羽以禦寒暑,必將役物以為養,任智而不恃力,此其所以為貴也。故不仁愛則不能群,不能群則不勝物,不勝物則養不足。群而不足,爭心將作,上聖卓然先行敬讓博愛之德者,眾心說而從之。從之成群,是為君矣;歸而往之,是為王矣。洪範曰:「天子作民父母,為天下王。」聖人取類以正名,而謂君為父母,明仁愛德讓,王道之本也。愛待敬而不敗,德須威而久立,故制禮以崇敬,作刑以明威也。聖人既躬明悊之性,必通天地之心,制禮作教,立法設刑,動緣民情,而則天象地。故曰先王立禮,「則天之明,因地之性」也。刑罰威獄,以類天之震曜殺戮也;溫慈惠和,以效天之生殖長育也。《書》云「天秩有禮」,「天討有罪」。故聖人因天秩而制五禮,因天討而作五刑。大刑用甲兵,其次用斧鉞;中刑用刀鋸,其次用鑽鑿;薄刑用鞭扑。大者陳諸原野,小者致之市朝,其所繇來者上矣。

自黃帝有涿鹿之戰以定火災,顓頊有共工之陳以定水害。唐虞之際,至治之極,猶流共工,放讙兜,竄三苗,殛鯀,然後天下服。夏有甘扈之誓,殷、周以兵定天下矣。天下既定,戢臧干戈,教以文德,而猶立司馬之官,設六軍之眾,因井田而制軍賦。地方一里為井,井十為通,通十為成,成方十里;成十為終,終十為同,同方百里;同十為封,封十為畿,畿方千里。有稅有租。稅以足食,賦以足兵。故四井為邑,四邑為丘。丘,十六井也,有戎馬一匹,牛三頭。四丘為甸。甸,六十四井也,有戎馬四匹,兵車一乘,牛十二頭,甲士三人,卒七十二人,干戈備具,是謂乘馬之法。一同百里,提封萬井,除山川沈斥,城池邑居,園囿術路,三千六百井,定出賦六千四百井,戎馬四百匹,兵車百乘,此卿大夫采地之大者也,是謂百乘之家。一封三百一十六里,提封十萬井,定出賦六萬四千井,戎馬四千匹,兵車千乘,此諸侯之大者也,是謂千乘之國。天子畿方千里,提封百萬井,定出賦六十四萬井,戎馬四萬匹,兵車萬乘,故稱萬乘之主。戎馬車徒干戈素具,春振旅以搜,夏拔舍以苗,秋治兵以獮,冬大閱以狩,皆於農隙以講事焉。五國為屬,屬有長;十國為連,連有帥;三十國為卒,卒有正;二百一十國為州,州有牧。連帥比年簡車,卒正三年簡徒,群牧五載大簡車徒,此先王為國立武足兵之大略也。

周道衰,法度颦,至齊桓公任用管仲,而國富民安。公問行伯用師之道,管仲曰:「公欲定卒伍,修甲兵,大國亦將修之,而小國設備,則難以速得志矣。」於是乃作內政而寓軍令焉,故卒伍定虖里,而軍政成虖郊。連其什伍,居處同樂,死生同憂,禍福共之,故夜戰則其聲相聞,晝戰則其目相見,緩急足以相死。其教已成,外攘夷狄,內尊天子,以安諸夏。齊威既沒,晉文接之,亦先定其民,作被廬之法,總帥諸侯,迭為盟主。然其禮已頗僭差,又隨時苟合以求欲速之功,故不能充王制。二伯之後,寖以陵夷,至魯成公作丘甲,哀公用田賦,搜狩治兵大閱之事皆失其正。春秋書而譏之,以存王道。於是師旅亟動,百姓罷敝,無伏節死難之誼。孔子傷焉,曰:「以不教民戰,是謂棄之。」

故稱子路曰:「由也,千乘之國,可使治其賦也。」而子路亦曰:「千乘之國,攝虖大國之間,加之以師旅,因之以饑饉,由也為之,比及三年,可使有勇,且知方也。」治其賦兵教以禮誼之謂也。

春秋之後,滅弱吞小,並為戰國,稍增講武之禮,以為戲樂,用相夸視。而秦更名角抵,先王之禮沒於淫樂中矣。雄桀之士因勢輔時,作為權詐以相傾覆,吳有孫武,齊有孫臏,魏有吳起,秦有商鞅,皆禽敵立勝,垂著篇籍。當此之時,合從連衡,轉相攻伐,代為雌雄。齊愍以技擊彊,魏惠以武卒奮,秦昭以銳士勝。世方爭於功利,而馳說者以孫、吳為宗。時唯孫卿明於王道,而非之曰:「彼孫、吳者,上勢利而貴變詐;施於暴亂昏嫚之國,君臣有間,上下離心,政謀不良,故可變而詐也。夫仁人在上,為下所卬,猶子弟之衛父兄,若手足之扞頭目,何可當也?鄰國望我,歡若親戚,芬若椒蘭,顧視其上,猶焚灼仇讎。人情豈肯為其所惡而攻其所好哉?故以桀攻桀,猶有巧拙;以桀詐堯,若卵投石,夫何幸之有!《詩》曰:『武王載旆,有虔秉鉞,如火烈烈,則莫我敢遏。』言以仁誼綏民者,無敵於天下也。若齊之技擊,得一首則受賜金。事小敵鲧,則媮可用也;事鉅敵堅,則渙然離矣。是亡國之兵也。魏氏武卒,衣三屬之甲,操十二石之弩,負矢五十丢,置戈其上,冠冑帶劍,贏三日之糧,日中而趨百里,中試則復其戶,利其田宅。如此,則其地雖廣,其稅必寡,其氣力數年而衰。是危國之兵也。秦人,其生民也骥阨,其使民也酷烈。劫之以勢,隱之以阨,狃之以賞慶,道之以刑罰,使其民所以要利於上者,非戰無由也。功賞相長,五甲首而隸五家,是最為有數,故能四世有勝於天下。然皆干賞蹈利之兵,庸徒鬻賣之道耳,未有安制矜節之理也。故雖地廣兵彊,鰓鰓常恐天下之一合而共軋己也。至乎齊桓、晉文之兵,可謂入其域而有節制矣,然猶未本仁義之統也。故齊之技擊不可以遇魏之武卒,魏之武卒不可以直秦之銳士,秦之銳士不可以當桓、文之節制,桓、文之節制不可以敵湯、武之仁義。」

故曰:「善師者不陳,善陳者不戰,善戰者不敗,善敗者不亡。」若夫舜修百僚,咎繇作士,命以「蠻夷猾夏,寇賊姦軌」,而刑無所用,所謂善師不陳者也。湯、武征伐,陳師誓眾,而放禽桀、紂,所謂善陳不戰者也。齊桓南服彊楚,使貢周室,北伐山戎,為燕開路,存亡繼絕,功為伯首,所謂善戰不敗者也。楚昭王遭闔廬之禍,國滅出亡,父老送之。王曰:「父老反矣!何患無君?」父老曰:「有君如是其賢也!」相與從之。或奔走赴秦,號哭請救,秦人憐之謂之出兵。二國并力,遂走吳師,昭王返國,所謂善敗不亡者也。若秦因四世之勝,據河山之阻,任用白起、王翦豺狼之徒,奮其爪牙,禽獵六國,以并天下。窮武極詐,士民不附,卒隸之徒,還為敵讎,猋起雲合,果共軋之。斯為下矣。凡兵,所以存亡繼絕,救亂除害也。故伊、呂之將,子孫有國,與商周並。至於末世,苟任詐力,以快貪殘,爭城殺人盈城,爭地殺人滿野。孫、吳、商、白之徒,皆身誅戮於前,而功滅亡於後。報應之勢,各以類至,其道然矣。

漢興,高祖躬神武之材,行寬仁之厚,總攬英雄,以誅秦、項。任蕭、曹之文,用良、平之謀,騁陸、酈之辯,明叔孫通之儀,文武相配,大略舉焉。天下既定,踵秦而置材官於郡國,京師有南北軍之屯。至武帝平百粵,內增七校,外有樓船,皆歲時講肄,修武備云。至元帝時,以貢禹議,始罷角抵,而未正治兵振旅之事也。

古人有言:「天生五材,民並用之,廢一不可,誰能去兵?」鞭扑不可弛於家,刑罰不可廢於國,征伐不可偃於天下;用之有本末,行之有逆順耳。孔子曰:「工欲善其事,必先利其器。」文德者,帝王之利器;威武者,文德之輔助也。夫文之所加者深,則武之所服者大;德之所施者博,則威之所制者廣。三代之盛,至於刑錯兵寢者,其本末有序,帝王之極功也。

昔周之法,建三典以刑邦國,詰四方:一曰,刑新邦用輕典;二曰,刑平邦用中典;三曰,刑亂邦用重典。五刑,墨罪五百,劓罪五百,宮罪五百,刖罪五百,殺罪五百,所謂刑平邦用中典者也。凡殺人者踣諸市,墨者使守門,劓者使守關,宮者使守內,刖者使守囿,完者使守積。其奴,男子入于罪隸,女子入舂槁。凡有爵者,與七十者,與未齔者,皆不為奴。

周道既衰,穆王眊荒,命甫侯度時作刑,以詰四方。墨罰之屬千,劓罰之屬千,髕罰之屬五百,宮罰之屬三百,大辟之罰其屬二百。五刑之屬三千,蓋多於平邦中典五百章,所謂刑亂邦用重典者也。

春秋之時,王道寖壞,教化不行,子產相鄭而鑄刑書。晉叔嚮非之曰:「昔先王議事以制,不為刑辟。懼民之有爭心也,猶不可禁禦,是故閑之以誼,糾之以政,行之以禮,守之以信,奉之以仁;制為祿位以勸其從,嚴斷刑罰以威其淫。懼其未也,故誨之以忠,霭之以行,教之以務,使之以和,臨之以敬,蒞之以彊,斷之以剛。猶求聖哲之上,明察之官,忠信之長,慈惠之師。民於是乎可任使也,而不生禍亂。民知有辟,則不忌於上,並有爭心,以徵於書,而徼幸以成之,弗可為矣。夏有亂政而作禹刑,商有亂政而作湯刑,周有亂政而作九刑。三辟之興,皆叔世也。今吾子相鄭國,制參辟,鑄刑書,將以靖民,不亦難乎!《詩》曰:『儀式刑文王之德,日靖四方。』又曰:『儀刑文王,萬邦作孚。』如是,何辟之有?民知爭端矣,將棄禮而徵於書。錐刀之末,將盡爭之,亂獄滋豐,貨賂並行。終子之世,鄭其敗虖!」子產報曰:「若吾子之言,僑不材,不能及子孫,吾以救世也。」媮薄之政,自是滋矣。孔子傷之,曰:「導之以德,齊之以禮,有恥且格;導之以政,齊之以刑,民免而無恥。」「禮樂不興,則刑罰不中;刑罰不中,則民無所錯手足。」孟氏使陽膚為士師,問於曾子,亦曰:「上失其道,民散久矣。如得其情,則哀矜而勿喜。」

陵夷至於戰國,韓任申子,秦用商鞅,連相坐之法,造參夷之誅;增加肉刑、大辟,有鑿顛、抽脅、鑊亨之刑。

至於秦始皇,兼吞戰國,遂毀先王之法,滅禮誼之官,專任刑罰,躬操文墨,晝斷獄,夜理書,自程決事,日縣石之一。而姦邪並生,赭衣塞路,囹圄成市,天下愁怨,潰而叛之。

漢興,高祖初入關,約法三章曰:「殺人者死,傷人及盜抵罪。」蠲削煩苛,兆民大說。其後四夷未附,兵革未息,三章之法不足以禦姦,於是相國蕭何雳摭秦法,取其宜於時者,作律九章。

當孝惠、高后時,百姓新免毒酿,人欲長幼養老。蕭、曹為相,填以無為,從民之欲,而不擾亂,是以衣食滋殖,刑罰用稀。

及孝文即位,躬脩玄默,勸趣農桑,減省租賦。而將相皆舊功臣,少文多質,懲惡亡秦之政,論議務在寬厚,恥言人之過失。化行天下,告訐之俗易。吏安其官,民樂其業,畜積歲增,戶口寖息。風流篤厚,禁罔疏闊。選張釋之為廷尉,罪疑者予民,是以刑罰大省,至於斷獄四百,有刑錯之風。

即位十三年,齊太倉令淳于公有罪當刑,詔獄逮繫長安。淳于公無男,有五女,當行會逮,罵其女曰:「生子不生男,緩急非有益也!」其少女緹縈,自傷悲泣,乃隨其父至長安,上書曰:「妾父為吏,齊中皆稱其廉平,今坐法當刑。妾傷夫死者不可復生,刑者不可復屬,雖後欲改過自新,其道亡繇也。妾願沒入為官婢,以贖父刑罪,使得自新。」書奏天子,天子憐悲其意,遂下令曰:「制詔御史:蓋聞有虞氏之時,畫衣冠異章服以為戮,而民弗犯,何治之至也!今法有肉刑三,而姦不止,其咎安在?非乃朕德之薄,而教不明與!吾甚自愧。故夫訓道不純而愚民陷焉。《詩》曰:『愷弟君子,民之父母。』今人有過,教未施而刑已加焉,或欲改行為善,而道亡繇至,朕甚憐之。夫刑至斷支體,刻肌膚,終身不息,何其刑之痛而不德也!豈稱為民父母之意哉?其除肉刑,有以易之;及令罪人各以輕重,不亡逃,有年而免。具為令。」

丞相張蒼、御史大夫馮敬奏言:「肉刑所以禁姦,所由來者久矣。陛下下明詔,憐萬民之一有過被刑者終身不息,及罪人欲改行為善而道亡繇至,於盛德,臣等所不及也。臣謹議請定律曰:諸當完者,完為城旦舂;當黥者,髡鉗為城旦舂;當劓者,笞三百;當斬左止者,笞五百;當斬右止,及殺人先自告,及吏坐受賕枉法,守縣官財物而即盜之,已論命復有笞罪者,皆棄市。罪人獄已決,完為城旦舂,滿三歲為鬼薪白粲。鬼薪白粲一歲,為隸臣妾。隸臣妾一歲,免為庶人。隸臣妾滿二歲,為司寇。司寇一歲,及作如司寇二歲,皆免為庶人。其亡逃及有罪耐以上,不用此令。前令之刑城旦舂歲而非禁錮者,如完為城旦舂歲數以免。臣昧死請。」制曰:「可。」是後,外有輕刑之名,內實殺人。斬右止者又當死。斬左止者笞五百,當劓者笞三百,率多死。

景帝元年,下詔曰:「加笞與重罪無異,幸而不死,不可為人。其定律:笞五百曰三百,笞三百曰二百。」猶尚不全。至中六年,又下詔曰:「加笞者,或至死而笞未畢,朕甚憐之。其減笞三百曰二百,笞二百曰一百。」又曰:「笞者,所以教之也,其定箠令。」丞相劉舍、御史大夫衛綰請:「笞者,箠長五尺,其本大一寸,其竹也,末薄半寸,皆平其節。當笞者笞臀。毋得更人,畢一罪乃更人。」自是笞者得全,然酷吏猶以為威。死刑既重,而生刑又輕,民易犯之。

及至孝武即位,外事四夷之功,內盛耳目之好,徵發煩數,百姓貧耗,窮民犯法,酷吏擊斷,姦軌不勝。於是招進張湯、趙禹之屬,條定法令,作見知故縱、監臨部主之法,緩深故之罪,急縱出之誅。其後姦猾巧法,轉相比況,禁罔寖密。律令凡三百五十九章,大辟四百九條,千八百八十二事,死罪決事比萬三千四百七十二事。文書盈於几閣,典者不能遍睹。是以郡國承用者駮,或罪同而論異。姦吏因緣為市,所欲活則傅生議,所欲陷則予死比,議者咸冤傷之。

宣帝自在閭閻而知其若此,及即尊位,廷史路溫舒上疏,言秦有十失,其一尚存,治獄之吏是也。語在溫舒傳。上深愍焉,乃下詔曰:「間者吏用法,巧文寖深,是朕之不德也。夫決獄不當,使有罪興邪,不辜蒙戮,父子悲恨,朕甚傷之。今遣廷史與郡鞠獄,任輕祿薄,其為置廷平,秩六百石,員四人。其務平之,以稱朕意。」於是選于定國為廷尉,求明察寬恕黃霸等以為廷平,季秋後請讞。時上常幸宣室,齋居而決事,獄刑號為平矣。時涿郡太守鄭昌上疏言:「聖王置諫爭之臣者,非以崇德,防逸豫之生也;立法明刑者,非以為治,救衰亂之起也。今明主躬垂明聽,雖不置廷平,獄將自正;若開後嗣,不若刪定律令。律令一定,愚民知所避,姦吏無所弄矣。今不正其本,而置廷平以理其末也,政衰聽怠,則廷平將招權而為亂首矣。」宣帝未及修正。

至元帝初立,乃下詔曰:「夫法令者,所以抑暴扶弱,欲其難犯而易避也。今律令煩多而不約,自典文者不能分明,而欲羅元元之不逮,斯豈刑中之意哉!其議律令可蠲除輕減者,條奏,唯在便安萬姓而已。」

至成帝河平中,復下詔曰:「甫刑云『五刑之屬三千,大辟之罰其屬二百』,今大辟之刑千有餘條,律令煩多,百有餘萬言,奇請它比,日以益滋,自明習者不知所由,欲以曉喻眾庶,不亦難乎!於以羅元元之民,夭絕亡辜,豈不哀哉!其與中二千石、二千石、博士及明習律令者議減死刑及可蠲除約省者,令較然易知,條奏。書不云乎?『惟刑之恤哉!』其審核之,務準古法,朕將盡心覽焉。」有司無仲山父將明之材,不能因時廣宣主恩,建立明制,為一代之法,而徒鉤摭微細,毛舉數事,以塞詔而已。是以大議不立,遂以至今。議者或曰,法難數變,此庸人不達,疑塞治道,聖智之所常患者也。故略舉漢興以來,法令稍定而合古便今者。

漢興之初,雖有約法三章,網漏吞舟之魚,然其大辟,尚有夷三族之令。令曰:「當三族者,皆先黥,劓,斬左右止,笞殺之,梟其首,菹其骨肉於市。其誹謗詈詛者,又先斷舌。」故謂之具五刑。彭越、韓信之屬皆受此誅。至高后元年,乃除三族罪、祅言令。孝文二年,又詔丞相、太尉、御史:「法者,治之正,所以禁暴而衛善人也。今犯法者已論,而使無罪之父母妻子同產坐之及收,朕甚弗取。其議。」左右丞相周勃、陳平奏言:「父母妻子同產相坐及收,所以累其心,使重犯法也。收之之道,所由來久矣。臣之愚計,以為如其故便。」文帝復曰:「朕聞之,法正則民愨,罪當則民從。且夫牧民而道之以善者,吏也;既不能道,又以不正之法罪之,是法反害於民,為暴者也。朕未見其便,宜孰計之。」平、勃乃曰:「陛下幸加大惠於天下,使有罪不收,無罪不相坐,甚盛德,臣等所不及也。臣等謹奉詔,盡除收律、相坐法。」其後,新垣平謀為逆,復行三族之誅。由是言之,風俗移易,人性相近而習相遠,信矣。夫以孝文之仁,平、勃之知,猶有過刑謬論如此甚也,而況庸材溺於末流者乎?

周官有五聽、八議、三刺、三宥、三赦之法。五聽:一曰辭聽,二曰色聽,三曰氣聽,四曰耳聽,五曰目聽。八議:一曰議親,二曰議故,三曰議賢,四曰議能,五曰議功,六曰議貴,七曰議勤,八曰議賓。三刺:一曰訊群臣,二曰訊群吏,三曰訊萬民。三宥:一曰弗識,二曰過失,三曰遺忘。三赦:一曰幼弱,二曰老眊,三曰憃愚。凡囚,「上罪梏拲而桎,中罪梏桎,下罪梏;王之同族拲,有爵者桎,以待弊。」高皇帝七年,制詔御史:「獄之疑者,吏或不敢決,有罪者久而不論,無罪者久繫不決。自今以來,縣道官獄疑者,各讞所屬二千石官,二千石官以其罪名當報之。所不能決者,皆移廷尉,廷尉亦當報之。廷尉所不能決,謹具為奏,傅所當比律令以聞。」上恩如此,吏猶不能奉宣。故孝景中五年復下詔曰:「諸獄疑,雖文致於法而於人心不厭者,輒讞之。」其後獄吏復避微文,遂其愚心。至後元年,又下詔曰:「獄,重事也。人有愚智,官有上下。獄疑者讞,有令讞者已報讞而後不當,讞者不為失。」自此之後,獄刑益詳,近於五聽三宥之意。三年復下詔曰:「高年老長,人所尊敬也;鰥寡不屬逮者,人所哀憐也。其著令:年八十以上,八歲以下,及孕者未乳,師、朱儒當鞠繫者,頌繫之。」至孝宣元康四年,又下詔曰:「

朕念夫耆老之人,髮齒墮落,血氣既衰,亦無暴逆之心,今或羅于文法,執于囹圄,不得終其年命,朕甚憐之。自今以來,諸年八十非誣告殺傷人,它皆勿坐。」至成帝鴻嘉元年,定令:「年未滿七歲,賊鬥殺人及犯殊死者,上請廷尉以聞,得減死。」合於三赦幼弱老眊之人。此皆法令稍定,近古而便民者也。

孔子曰:「如有王者,必世而後仁;善人為國百年,可以勝殘去殺矣。」言聖王承衰撥亂而起,被民以德教,變而化之,必世然後仁道成焉;至於善人,不入於室,然猶百年勝殘去殺矣。此為國者之程式也。今漢道至盛,歷世二百餘載,考自昭、宣、元、成、哀、平六世之間,斷獄殊死,率歲千餘口而一人,耐罪上至右止,三倍有餘。古人有言:「滿堂而飲酒,有一人鄉隅而悲泣,則一堂皆為之不樂。」王者之於天下,譬猶一堂之上也,故一人不得其平,為之悽愴於心。今郡國被刑而死者歲以萬數,天下獄二千餘所,其冤死者多少相覆,獄不減一人,此和氣所以未洽者也。

原獄刑所以蕃若此者,禮教不立,刑法不明,民多貧窮,豪桀務私,姦不輒得,獄豻不平之所致也。《書》云「伯夷降典,悊民惟刑」,言制禮以止刑,猶隄之防溢水也。今隄防凌遲,禮制未立;死刑過制,生刑易犯;饑寒並至,窮斯濫溢;豪桀擅私,為之囊橐,姦有所隱,則狃而寖廣:此刑之所以蕃也。孔子曰:「古之知法者能省刑,本也;今之知法者不失有罪,末矣。」又曰:「今之聽獄者,求所以殺之;古之聽獄者,求所以生之。」與其殺不辜,寧失有罪。今之獄吏,上下相驅,以刻為明,深者獲功名,平者多後患。諺曰:「鬻棺者欲歲之疫。」非憎人欲殺之,利在於人死也。今治獄吏欲陷害人,亦猶此矣。凡此五疾,獄刑所以尤多者也。

自建武、永平,民亦新免兵革之禍,人有樂生之慮,與高、惠之間同,而政在抑彊扶弱,朝無威福之臣,邑無豪桀之俠。以口率計,斷獄少於成、哀之間什八,可謂清矣。然而未能稱意比隆於古者,以其疾未盡除,而刑本不正。

善乎!孫卿之論刑也,曰:「世俗之為說者,以為治古者無肉刑,有象刑墨黥之屬,菲履赭衣而不純,是不然矣。以為治古,則人莫觸罪邪,豈獨無肉刑哉,亦不待象刑矣。以為人或觸罪矣,而直輕其刑,是殺人者不死,而傷人者不刑也。罪至重而刑至輕,民無所畏,亂莫大焉。凡制刑之本,將以禁暴惡,且懲其

末也。殺人者不死,傷人者不刑,是惠暴而寬惡也。故象刑非生治古,方起於亂今也。凡爵列官職,賞慶刑罰,皆以類相從者也。一物失稱,亂之端也。德不稱位,能不稱官,賞不當功,刑不當罪,不祥莫大矣焉。夫征暴誅悖,治之威也。殺人者死,傷人者刑,是百王之所同也,未有知其所由來者也。故治則刑重,亂則刑輕,犯治之罪固重,犯亂之罪固輕也。《書》云『刑罰世重世輕』,此之謂也。」所謂「象刑惟明」者,言象天道而作刑,安有菲屨赭衣者哉?

孫卿之言既然,又因俗說而論之曰:禹承堯舜之後,自以德衰而制肉刑,湯武順而行之者,以俗薄於唐虞故也。今漢承衰周暴秦極敝之流,俗已薄於三代,而行堯舜之刑,是猶以鞿而御駻突,違救時之宜矣。且除肉刑者,本欲以全民也,今去髡鉗一等,轉而入於大辟。以死罔民,失本惠矣。故死者歲以萬數,刑重之所致也。至乎穿窬之盜,忿怒傷人,男女淫佚,吏為姦臧,若此之惡,髡鉗之罰又不足以懲也。故刑者歲十萬數,民既不畏,又曾不恥,刑輕之所生也。故俗之能吏,公以殺盜為威,專殺者勝任,奉法者不治,亂名傷制,不可勝條。是以罔密而姦不塞,刑蕃而民愈嫚。必世而未仁,百年而不勝殘,誠以禮樂闕而刑不正也。豈宜惟思所以清原正本之論,刪定律令,篹二百章,以應大辟。其餘罪次,於古當生,今觸死者,皆可募行肉刑。及傷人與盜,吏受賕枉法,男女淫亂,皆復古刑,為三千章。詆欺文致微細之法,悉蠲除。如此,則刑可畏而禁易避,吏不專殺,法無二門,輕重當罪,民命得全,合刑罰之中,殷天人之和,順稽古之制,成時雍之化。成康刑錯,雖未可致,孝文斷獄,庶幾可及。《詩》云「宜民宜人,受祿于天」。《書》曰「立功立事,可以永年」。言為政而宜於民者,功成事立,則受天祿而永年命,所謂「一人有慶,萬民賴之」者也。


\end{pinyinscope}