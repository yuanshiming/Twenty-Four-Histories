\article{酷吏傳}

\begin{pinyinscope}
孔子曰:「導之以政,齊之以刑,民免而無恥;導之以德,齊之以禮,有恥且格。」老氏稱:「上德不德,是以有德;下德不失德,是以無德。法令滋章,盜賊多有。」信哉是言也!法令者,治之具,而非制治清濁之原也。昔天下之罔嘗密矣,然不軌愈起,其極也,上下相遁,至於不振。當是之時,吏治若救火揚沸,非武健嚴酷,惡能勝其任而媮快乎?言道德者,溺於職矣。故曰:「聽訟吾猶人也,必也使無訟乎!」「下士聞道大笑之。」非虛言也。

漢興,破觚而為圜,斲琱而為樸,號為罔漏吞舟之魚。而吏治蒸蒸,不至於姦,黎民艾安。由是觀之,在彼不在此。高后時,酷吏獨有侯封,刻轢宗室,侵辱功臣。呂氏已敗,遂夷侯封之家。孝景時,晁錯以刻深頗用術輔其資,而七國之亂發怒於錯,錯卒被戮。其後有郅都、甯成之倫。

郅都,河東大陽人也。以郎事文帝。景帝時為中郎將,敢直諫,面折大臣於朝。嘗從入上林,賈姬在廁,野彘入廁,上目都,都不行。上欲自持兵救賈姬,都伏上前曰:「亡一姬復一姬進,天下所少寧姬等邪?陛下縱自輕,奈宗廟太后何?」上還,彘亦不傷賈姬。太后聞之,賜都金百斤,上亦賜金百斤,由此重都。

濟南瞷氏宗人三百餘家,豪猾,二千石莫能制,於是景帝拜都為濟南守。至則誅瞷氏首惡,餘皆股栗。居歲餘,郡中不拾遺,旁十餘郡守畏都如大府。

都為人,勇有氣,公廉,不發私書,問遺無所受,請寄無所聽。常稱曰:「己背親而出,身固當奉職死節官下,終不顧妻子矣。」

都遷為中尉,丞相條侯至貴居也,而都揖丞相。是時民樸,畏罪自重,而都獨先嚴酷,致行法不避貴戚,列侯宗室見都側目而視,號曰「蒼鷹」。

臨江王徵詣中尉府對簿,臨江王欲得刀筆為書謝上,而都禁吏弗與。魏其侯使人間予臨江王。臨江王既得,為書謝上,因自殺。竇太后聞之,怒,以危法中都,都免歸家。景帝乃使使即拜都為鴈門太守,便道之官,得以便宜從事。匈奴素聞郅都節,舉邊為引兵去,竟都死不近鴈門。匈奴至為偶人象都,令騎馳射,莫能中,其見憚如此。匈奴患之。乃中都以漢法。景帝曰:「都忠臣。」欲釋之。竇太后曰:「臨江王獨非忠臣乎?」於是斬都也。

甯成,南陽穰人也。以郎謁者事景帝。好氣,為少吏,必陵其長吏;為人上,操下急如束溼。猾賊任威。稍遷至濟南都尉,而郅都為守。始前數都尉步入府,因吏謁守如縣令,其畏都如此。及成往,直凌都出其上。都素聞其聲,善遇,與結驩。久之,都死,後長安左右宗室多犯法,上召成為中尉。其治效郅都,其廉弗如,然宗室豪桀人皆惴恐。

武帝即位,徙為內史,外戚多毀成之短,抵罪髡鉗。是時九卿死即死,少被刑,而成刑極,自以為不復收,乃解脫,詐刻傳出關歸家。稱曰:「仕不至二千石,賈不至千萬,安可比人乎!」乃貰貣陂田千餘頃,假貧民,役使數千家。數年,會赦,致產數千萬,為任俠,持吏長短,出從數十騎。其使民,威重於郡守。

周陽由,其父趙兼以淮南王舅侯周陽,故因氏焉。由以宗家任為郎,事文帝。景帝時,由為郡守。武帝即位,吏治尚脩謹,然由居二千石中最為暴酷驕恣。所愛者,撓法活之;所憎者,曲法滅之。所居郡,必夷其豪。為守,視都尉如令;為都尉,陵太守,奪之治。汲黯為忮,司馬安之文惡,俱在二千石列,同車未嘗敢均茵馮。後由為河東都尉,與其守勝屠公爭權,相告言,勝屠公當抵罪,議不受刑,自殺,而由棄市。

自甯成、周陽由之後,事益多,民巧法,大抵吏治類多成、由等矣。

趙禹,斄人也。以佐史補中都官,用廉為令史,事太尉周亞夫。亞夫為丞相,禹為丞相史,府中皆稱其廉平。然亞夫弗任,曰:「極知禹無害,然文深,不可以居大府。」武帝時,禹以刀筆吏積勞,遷為御史。上以為能,至中大夫。與張湯論定律令,作見知,吏傳相監司以法,盡自此始。

禹為人廉裾,為吏以來,舍無食客。公卿相造請,禹終不行報謝,務在絕知友賓客之請,孤立行一意而已。見法輒取,亦不覆案求官屬陰罪。嘗中廢,已為廷尉。始條侯以禹賊深,及禹為少府九卿,酷急。至晚節,事益多。吏務為嚴峻,而禹治加緩,名為平。王溫舒等後起,治峻禹。禹以老,徙為燕相。數歲,誖亂有罪,免歸。後十餘年,以壽卒于家。

義縱,河東人也。少年時嘗與張次公俱攻剽,為群盜。縱有姊,以醫幸王太后。太后問:「有子兄弟為官者乎?」姊曰:「有弟無行,不可。」太后乃告上,上拜義姁弟縱為中郎,補上黨郡中令。治敢往,少溫籍,縣無逋事,舉第一。遷為長陵及長安令,直法行治,不避貴戚。以捕桉太后外孫脩成子中,上以為能,遷為河內都尉。至則族滅其豪穰氏之屬,河內道不拾遺。而張次公亦為郎,以勇悍從軍,敢深入,有功,封為岸頭侯。

甯成家居,上欲以為郡守,御史大夫弘曰:「臣居山東為小吏時,甯成為濟南都尉,其治如狼牧羊。成不可令治民。」上乃拜成為關都尉。歲餘,關吏稅肄郡國出入關者,號曰:「寧見乳虎,無直甯成之怒。」其暴如此。義縱自河內遷為南陽太守,聞甯成家居南陽,及至關,甯成側行送迎,然縱氣盛,弗為禮。至郡,逐桉甯氏,破碎其家。成坐有罪,及孔、暴之屬皆奔亡,南陽吏民重足一跡。而平氏朱彊、杜衍杜周為縱爪牙之吏,任用,遷為廷尉史。

軍數出定襄,定襄吏民亂敗,於是徙縱為定襄太守。縱至,掩定襄獄中重罪二百餘人,及賓客昆弟私入相視者亦二百餘人。縱壹切捕鞠,曰「為死罪解脫」。是日皆報殺四百餘人。郡中不寒而栗,猾民佐吏為治。

是時趙禹、張湯為九卿矣,然其治尚寬,輔法而行,縱以鷹擊毛摯為治。後會更五銖錢白金起,民為姦,京師尤甚,乃以縱為右內史,王溫舒為中尉。溫舒至惡,所為弗先言縱,縱必以氣陵之,敗壞其功。其治,所誅殺甚多,然取為小治,姦益不勝,直指始出矣。吏之治以斬殺縛束為務,閻奉以惡用矣。縱廉,其治效郅都。上幸鼎湖,病久,已而卒起幸甘泉,道不治。上怒曰:「縱以我為不行此道乎?」銜之。至冬,楊可方受告緡,縱以為此亂民,部吏捕其為可使者。天子聞,使杜式治,以為廢格沮事,棄縱市。後一歲,張湯亦死。

王溫舒,陽陵人也。少時椎埋為姦。已而試縣亭長,數廢。數為吏,以治獄至廷尉史。事張湯,遷為御史,督盜賊,殺傷甚多。稍遷至廣平都尉,擇郡中豪敢往吏十餘人為爪牙,皆把其陰重罪,而縱使督盜賊,快其意所欲得。此人雖有百罪,弗法;即有避回,夷之,亦滅宗。以故齊趙之郊盜不敢近廣平,廣平聲為道不拾遺。上聞,遷為河內太守。

素居廣平時,皆知河內豪姦之家。及往,以九月至,令郡具私馬五十疋,為驛自河內至長安,部吏如居廣平時方略,捕郡中豪猾,相連坐千餘家。上書請,大者至族,小者乃死,家盡沒入償臧。奏行不過二日,得可,事論報,至流血十餘里。河內皆怪其奏,以為神速。盡十二月,郡中無犬吠之盜。其頗不得,失之旁郡,追求,會春,溫舒頓足歎曰:「嗟乎,令冬月益展一月,卒吾事矣!」其好殺行威不愛人如此。

上聞之,以為能,遷為中尉。其治復放河內,徒請召猜禍吏與從事,河內則楊皆、麻戊,關中揚贛、成信等。義縱為內史,憚之,未敢恣治。及縱死,張湯敗後,徙為延尉。而尹齊為中尉坐法抵罪,溫舒復為中尉。為人少文,居它惛惛不辯,至於中尉則心開。素習關中俗,知豪惡吏,豪惡吏盡復為用。吏苛察淫惡少年,投缿購告言姦,置伯落長以收司姦。溫舒多諂,善事有勢者;即無勢,視之如奴。有勢家,雖有姦如山,弗犯;無勢,雖貴戚,必侵辱。舞文巧,請下戶之猾,以動大豪。其治中尉如此。姦猾窮治,大氐盡靡爛獄中,行論無出者。其爪牙吏虎而冠。於是中尉部中中猾以下皆伏,有勢者為遊聲譽,稱治。數歲,其吏多以權貴富。

溫舒擊東越還,議有不中意,坐以法免。是時上方欲作通天臺而未有人,溫舒請覆中尉脫卒,得數萬人作。上說,拜為少府。徙右內史,治如其故,姦邪少禁。坐法失官,復為右輔,行中尉,如故操。

歲餘,會宛軍發,詔徵豪吏。溫舒匿其吏華成,及人有變告溫舒受員騎錢,它姦利事,罪至族,自殺。其時兩弟及兩婚家亦各自坐它罪而族。光祿勳徐自為曰:「悲夫!夫古有三族,而王溫舒罪至同時而五族乎!」溫舒死,家絫千金。

尹齊,東郡茌平人也。以刀筆吏稍遷至御史。事張湯,湯數稱以為廉。武帝使督盜賊,斬伐不避貴勢。遷關都尉,聲甚於甯成。上以為能,拜為中尉。吏民益彫敝,輕齊木彊少文,豪惡吏伏匿而善吏不能為治,以故事多廢,抵罪。後復為淮陽都尉。王溫舒敗後數年,病死,家直不滿五十金。所誅滅淮陽甚多,及死,仇家欲燒其尸,妻亡去,歸葬。

楊僕,宜陽人也。以千夫為吏。河南守舉為御吏,使督盜賊關東,治放尹齊,以敢擊行。稍遷至主爵都尉,上以為能。南越反,拜為樓船將軍,有功,封將梁侯。東越反,上欲復使將,為其伐前勞,以書敕責之曰:「將軍之功,獨有先破石門、尋骥,非有斬將騫旗之實也,烏足以驕人哉!前破番禺,捕降者以為虜,掘死人以為獲,是一過也。建德、呂嘉逆罪不容於天下,將軍擁精兵不窮追,超然以東越為援,是二過也。士卒暴露連歲,為朝會不置酒,將軍不念其勤勞,而造佞巧,請乘傳行塞,因用歸家,懷銀黃,垂三組,夸鄉里,是三過也。失期內顧,以道惡為解,失尊尊之序,是四過也。欲請蜀刀,問君賈幾何,對曰率數百,武庫日出兵而陽不知,挾偽干君,是五過也。受詔不至蘭池宮,明日又不對。假令將軍之吏問之不對,令之不從,其罪何如?推此心以在外,江海之間可得信乎!今東越深入,將軍能率眾以掩過不?」僕惶恐,對曰:「願盡死贖罪!」與王溫舒俱破東越。後復與左將軍荀彘俱擊朝鮮,為彘所縛,語在朝鮮傳。還,免為庶人,病死。

咸宣,楊人也。以佐史給事河東守。衛將軍青使買馬河東,見宣無害,言上,徵為廄丞。官事辦,稍遷至御史及丞,使治主父偃及淮南反獄,所以微文深詆殺者甚眾,稱為敢決疑。數廢數起,為御史及中丞者幾二十歲。王溫舒為中尉,而宣為左內史。其治米鹽,事小大皆關其手,自部署縣名曹寶物,官吏令丞弗得擅搖,痛以重法繩之。居官數年,壹切為小治辯,然獨宣以小至大,能自行之,難以為經。中廢為右扶風,坐怒其吏成信,信亡藏上林中,宣使郿令將吏卒,闌入上林中蠶室門攻亭格殺信,射中苑門,宣下吏,為大逆當族,自殺。而杜周任用。

是時郡守尉諸侯相二千石欲為治者,大抵盡效王溫舒等,而吏民益輕犯法,盜賊滋起。南陽有梅免、百政,楚有段中、杜少,齊有徐勃,燕趙之間有堅盧、范主之屬。大群至數千人,擅自號,攻城邑,取庫兵,釋死罪,縛辱郡守都尉,殺二千石,為檄告縣趨具食;小群以百數,掠鹵鄉里者不可稱數。於是上始使御史中丞、丞相長史使督之,猶弗能禁,乃使光祿大夫范昆、諸部都尉及故九卿張德等衣繡衣持節,虎符發兵以興擊,斬首大部或至萬餘級。及以法誅通行飲食,坐相連郡,甚者數千人。數歲,乃頗得其渠率。散卒失亡,復聚黨阻山川。往往而群,無可奈何。於是作沈命法,曰:「群盜起不發覺,發覺而弗捕滿品者,二千石以下至小吏主者皆死。」其後小吏畏誅,雖有盜弗敢發,恐不能得,坐課累府,府亦使不言。故盜賊寖多,上下相為匿,以避文法焉。

田廣明字子公,鄭人也。以郎為天水司馬。功次遷河南都尉,以殺伐為治。郡國盜賊並起,遷廣明為淮陽太守。歲餘,故城父令公孫勇與客胡倩等謀反,倩詐稱光祿大夫,從車騎數十,言使督盜賊,止陳留傳舍,太守謁見,欲收取之。廣明覺知,發兵皆捕斬焉。而公孫勇衣繡衣,乘駟馬車至圉,圉使小史侍之,亦知其非是,守尉魏不害與廄嗇夫江德、尉史蘇昌共收捕之。上封不害為當塗侯,德轑陽侯,昌蒲侯。初,四人俱拜於前,小史竊言。武帝問:「言何?」對曰:「為侯者得東歸不?」上曰:「女欲不?貴矣。女鄉名為何?」對曰:「名遺鄉。」上曰:「用遺汝矣。」於是賜小史爵關內侯,食遺鄉六百戶。

上以廣明連禽大姦,徵入為大鴻臚,擢廣明兄雲中代為淮陽太守。昭帝時,廣明將兵擊益州,還,賜爵關內侯,徙衛尉。後出為左馮翊,治有能名。宣帝初立,代蔡義為御史大夫,以前為馮翊與議定策,封昌水侯。歲餘,以祁連將軍將兵擊匈奴,出塞至受降城。受降都尉前死,喪柩在堂,廣明召其寡妻與姦。既出不至質,引軍空還。下太守杜延年簿責,廣明自殺闕下,國除。兄雲中為淮陽守,亦敢誅殺,吏民守闕告之,竟坐棄市。

田延年字子賓,先齊諸田也,徙陽陵。延年以材略給事大將軍莫府,霍光重之,遷為長史。出為河東太守,選拔尹翁歸等以為爪牙,誅鉏豪彊,姦邪不敢發。以選入為大司農。會昭帝崩,昌邑王嗣位,淫亂,霍將軍憂懼,與公卿議廢之,莫敢發言。延年按劍,延叱群臣,即日議決,語在光傳。宣帝即位,延年以決疑定策封陽成侯。

先是,茂陵富人焦氏、賈氏以數千萬陰積貯炭葦諸下里物。昭帝大行時,方上事暴起,用度未辦,延年奏言「商賈或豫收方上不祥器物,冀其疾用,欲以求利,非民臣所當為。請沒入縣官。」奏可。富人亡財者皆怨,出錢求延年罪。初,大司農取民牛車三萬兩為僦,載沙便橋下,送致方上,車直千錢,延年上簿詐增僦直車二千,凡六千萬,盜取其半。焦、賈兩家告其事,下丞相府。丞相議奏延年「主守盜三千萬,不道」。霍將軍召問延年,欲為道地,延年抵曰:「本出將軍之門,蒙此爵位,無有是事。」光曰:「即無事,當窮竟。」御史大夫田廣明謂太僕杜延年:「春秋之義,以功覆過。當廢昌邑王時,非田子賓之言大事不成。今縣官出三千萬自乞之何哉?願以愚言白大將軍。」延年言之大將軍,大將軍曰:「誠然,實勇士也!當發大議時,震動朝廷。」光因舉手自撫心曰:「使我至今病悸!謝田大夫曉大司農,通往就獄,得公議之。」田大夫使人語延年,延年曰:「幸縣官寬我耳,何面目入牢獄,使眾人指笑我,卒徒唾吾背乎!」即閉閣獨居齊舍,偏袒持刀東西步。數日,使者召延年詣廷尉。聞鼓聲,自刎死,國除。

嚴延年字次卿,東海下邳人也。其父為丞相掾,延年少學法律丞相府,歸為郡吏。以選除補御史掾,舉侍御史。是時大將軍霍光廢昌邑王,尊立宣帝。宣帝初即位,延年劾奏光「擅廢立,亡人臣禮,不道」。奏雖寢,然朝廷肅焉敬憚。延年後復劾大司農田延年持兵干屬車,大司農自訟不干屬車。事下御史中丞,譴責延年何以不移書宮殿門禁止大司農,而令得出入宮。於是覆劾延年闌內罪人,法至死。延年亡命。會赦出,丞相御史府徵書同日到,延年以御史書先至,詣御史府,復為掾。宣帝識之,拜為平陵令,坐殺不辜,去官。後為丞相掾,復擢好畤令。神爵中,西羌反,彊弩將軍許延壽請延年為長史,從軍敗西羌,還為涿郡太守。

時郡比得不能太守,涿人畢野白等由是廢亂。大姓西高氏、東高氏,自郡吏以下皆畏避之,莫敢與啎,咸曰:「寧負二千石,無負豪大家。」賓客放為盜賊,發,輒入高氏,吏不敢追。浸浸日多,道路張弓拔刃,然後敢行,其亂如此。延年至,遣掾蠡吾趙繡桉高氏得其死罪。繡見延年新將,心內懼,即為兩劾,欲先白其輕者,觀延年意怒,乃出其重劾。延年已知其如此矣。趙掾至,果白其輕者,延年索懷中,得重劾,即收送獄。夜入,晨將至市論殺之,先所桉者死,吏皆股弁。更遣吏分考兩高,窮竟其姦,誅殺各數十人。郡中震恐,道不拾遺。

三歲,遷河南太守,賜黃金二十斤。豪彊脅息,野無行盜,威震旁郡。其治務在摧折豪彊,扶助貧弱。貧弱雖陷法,曲文以出之;其豪桀侵小民者,以文內之。眾人所謂當死者,一朝出之;所謂當生者,詭殺之。吏民莫能測其意深淺,戰栗不敢犯禁。桉其獄,皆文致不可得反。

延年為人短小精悍,敏捷於事,雖子貢、冉有通藝於政事,不能絕也。吏忠盡節者,厚遇之如骨肉,皆親鄉之,出身不顧,以是治下無隱情。然疾惡泰甚,中傷者多,尤巧為獄文,善史書,所欲誅殺,奏成於手,中主簿親近史不得聞知。奏可論死,奄忽如神。冬月,傳屬縣囚,會論府上,流血數里,河南號曰「屠伯」。令行禁止,郡中正清。

是時張敞為京兆尹,素與延年善。敞治雖嚴,然尚頗有縱舍,聞延年用刑刻急,乃以書諭之曰:「昔韓盧之取菟也,上觀下獲,不甚多殺。願次卿少緩誅罰,思行此術。」延年報曰:「河南天下喉咽,二周餘斃,莠甚苗穢,何可不鉏也?」自矜伐其能,終不衰止。時黃霸在潁川以寬恕為治,郡中亦平,婁蒙豐年,鳳皇下,上賢焉,下詔稱揚其行,加金爵之賞。延年素輕霸為人,及比郡為守,褒賞反在己前,心內不服。河南界中又有蝗蟲,府丞義出行蝗,還見延年,延年曰:「此蝗豈鳳皇食邪?」義又道司農中丞耿壽昌為常平倉,利百姓,延年曰:「丞相御史不知為也,當避位去。壽昌安得權此?」後左馮翊缺,上欲徵延年,符已發,為其名酷復止。延年疑少府梁丘賀毀之,心恨。會琅邪太守以視事久病,滿三月免,延年自知見廢,謂丞曰:「此人尚能去官,我反不能去邪?」又延年察獄史廉,有臧不入身,延年坐選舉不實貶秩,笑曰:「後敢復有舉人者矣!」丞義年老頗悖,素畏延年,恐見中傷。延年本嘗與義俱為丞相史,實親厚之,無意毀傷也,饋遺之甚厚。義愈益恐,自筮得死卦,忽忽不樂,取告至長安,上書言延年罪名十事。已拜奏,因飲藥自殺,以明不欺。事下御史丞按驗,有此數事,以結延年,坐怨望非謗政治不道棄市。

初,延年母從東海來,欲從延年臘,到雒陽,適見報囚。母大驚,便止都亭,不肯入府。延年出至都亭謁母,母閉閤不見。延年免冠頓首閤下,良久,母乃見之,因數責延年:幸得備郡守,專治千里,不聞仁愛教化,有以全安愚民,顧乘刑罰多刑殺人,欲以立威,豈為民父母意哉!」延年服罪,重頓首謝,因自為母御,歸府舍。母畢正臘,謂延年:「天道神明,人不可獨殺。我不意當老見壯子被刑戮也!行矣!去女東歸,埽除墓地耳。」遂去。歸郡,見昆弟宗人,復為言之。後歲餘,果敗。東海莫不賢知其母。延年兄弟五人皆有吏材,至大官,東海號曰「萬石嚴嫗」。次弟彭祖,至太子太傅,在儒林傳。

尹賞字子心,鉅鹿楊氏人也。以郡吏察廉為樓煩長。舉茂材,粟邑令。左馮翊薛宣奏賞能治劇,徙為頻陽令,坐殘賊免。後以御史舉為鄭令。

永治、元延間,上怠於政,貴戚驕恣,紅陽長仲兄弟交通輕俠,臧匿亡命。而北地大豪浩商等報怨,殺義渠長妻子六人,往來長安中。丞相御史遣掾求逐黨與,詔書召捕,久之乃得。長安中姦滑浸多,閭里少年群輩殺吏,受賕報仇,相與探丸為彈,得赤丸者斫武吏,得黑丸者斫文吏,白者主治喪;城中薄暮塵起,剽劫行者,死傷橫道,枹鼓不絕。賞以三輔高第選守長安令,得壹切便宜從事。賞至,修治長安獄,穿地方深各數丈,致令辟為郭,以大石覆其口,名為「虎穴」。乃部戶曹掾史,與鄉吏、亭長、里正、父老、伍人,雜舉長安中輕薄少年惡子,無市籍商販作務,而鮮衣凶服被鎧扞持刀兵者,悉籍記之,得數百人。賞一朝會長安吏,車數百兩,分行收捕,皆劾以為通行飲食群盜。賞親閱,見十置一,其餘盡以次內虎穴中,百人為輩,覆以大石。數日壹發視,皆相枕藉死,便輿出,瘞寺門桓東,楬著其姓名,百日後,乃令死者家各自發取其尸。親屬號哭,道路皆歔欷。長安中歌之曰:「安所求子死?桓東少年場。生時諒不謹,枯骨後何葬?」賞所置皆其魁宿,或故吏善家子失計隨輕黠願自改者,財數十百人,皆貰其罪,詭令立功以自贖。盡力有效者,因親用之為爪牙,追捕甚精,甘耆姦惡,甚於凡吏。賞視事數月,盜賊止,郡國亡命散走,各歸其處,不敢闚長安。

江湖中多盜賊,以賞為江夏太守,捕格江賊及所誅吏民甚多,坐殘賊免。南山群盜起,以賞為右輔都尉,遷執金吾,督大姦猾。三輔吏民甚畏之。

數年卒官。疾病且死,戒其諸子曰:「丈夫為吏,正坐殘賊免,追思其功效,則復進用矣。一坐軟弱不勝任免,終身廢棄無有赦時,其羞辱甚於貪汙坐臧。慎毋然!」賞四子皆至郡守,長子立為京兆尹,皆尚威嚴,有治辦名。

贊曰:自郅都以下皆以酷烈為聲,然都抗直,引是非,爭大體。張湯以知阿邑人主,與俱上下,時辯當否,國家賴其便。趙禹据法守正。杜周從諛,以少言為重。張湯死後,罔密事叢,以寖耗廢,九卿奉職,救國不給,何暇論繩墨之外乎!自是以至哀、平,酷吏眾多,然莫足數,此其知名見紀者也。其廉者足以為儀表,其汙者方略教道,壹切禁姦,亦質有文武焉。雖酷,稱其位矣。湯、周子孫貴盛,故別傳。


\end{pinyinscope}