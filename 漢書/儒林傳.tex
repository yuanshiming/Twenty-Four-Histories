\article{儒林傳}

\begin{pinyinscope}
古之儒者,博學虖六藝之文。六學者,王教之典籍,先聖所以明天道,正人倫,致至治之成法也。周道既衰,壞於幽厲,禮樂征伐自諸侯出,陵夷二百餘年而孔子興,以聖德遭季世,知言之不用而道不行,乃歎曰:「鳳鳥不至,河不出圖,吾已矣夫!」「文王既沒,文不在茲乎?」於是應聘諸侯,以答禮行誼。西入周,南至楚,畏匡厄陳,奸七十餘君。適齊聞韶,三月不知肉味;自衛反魯,然後樂正,雅頌各得其所。究觀古今之篇籍,乃稱曰:「大哉,堯之為君也!唯天為大,唯堯則之。巍巍乎其有成功也,煥乎其有文章也!」又云:「周監於二世,郁郁乎文哉!吾從周。」於是敘書則斷堯典,稱樂則法韶舞,論詩則首周南。綴周之禮,因魯春秋,舉十二公行事,繩之以文武之道,成一王法,至獲麟而止。蓋晚而好易,讀之韋編三絕,而為之傳。皆因近聖之事,鲈立先王之教,故曰:「述而不作,信而好古;」「下學而上達,知我者其天乎!」

仲尼既沒,七十子之徒散遊諸侯,大者為卿相師傅,小者友教士大夫,或隱而不見。故子張居陳,澹臺子羽居楚,子夏居西河,子貢終於齊。如田子方、段干木、吳起、禽滑氂之屬,皆受業於子夏之倫,為王者師。是時,獨魏文侯好學。天下並爭於戰國,儒術既黜焉,然齊魯之間學者猶弗廢,至於威、宣之際,孟子、孫卿之列咸遵夫子之業而潤色之,以學顯於當世。

及至秦始皇兼天下,燔詩書,殺術士,六學從此缺矣。陳涉之王也,魯諸儒持孔氏禮器而歸之,於是孔甲為涉博士,卒與俱死。陳涉起匹夫,敺適戍以立號,不滿歲而滅亡,其事至微淺,然而搢紳先生負禮器往委質為臣者何也?以秦禁其業,積怨而發憤於陳王也。

及高皇帝誅項籍,引兵圍魯,魯中諸儒尚講誦習禮,弦歌之音不絕,豈非聖人遺化好學之國哉?於是諸儒始得修其經學,講習大射鄉飲之禮。叔孫通作漢禮儀,因為奉常,諸弟子共定者,咸為選首,然後喟然興於學。然尚有干戈,平定四海,亦未皇庠序之事也。孝惠、高后時,公卿皆武力功臣。孝文時頗登用,然孝文本好刑名之言。及至孝景,不任儒,竇太后又好黃老術,故諸博士具官待問,未有進者。

漢興,言易自淄川田生;言書自濟南伏生;言詩,於魯則申培公,於齊則轅固生,燕則韓太傅;言禮,則魯高堂生;言春秋,於齊則胡毋生,於趙則董仲舒。及竇太后崩,武安君田蚡為丞相,黜黃老、刑名百家之言,延文學儒者以百數,而公孫弘以治春秋為丞相封侯,天下學士靡然鄉風矣。

弘為學官,悼道之鬱滯,乃請曰:「丞相、御史言:制曰『蓋聞導民以禮,風之以樂。婚姻者,居室之大倫也。今禮廢樂崩,朕甚愍焉,故詳延天下方聞之士,咸登諸朝。其令禮官勸學,講議洽聞,舉遺興禮,以為天下先。太常議,予博士弟子,崇鄉里之化,以厲賢材焉。』謹與太常臧、博士平等議,曰:聞三代之道,鄉里有教,夏曰校,殷曰庠,周曰序。其勸善也,顯之朝廷;其懲惡也,加之刑罰。故教化之行也,建首善自京師始,繇內及外。今陛下昭至德,開大明,配天地,本人倫,勸學興禮,崇化厲賢,以風四方,太平之原也。古者政教未洽,不備其禮,請因舊官而興焉。為博士官置弟子五十人,復其身。太常擇民年十八以上儀狀端正者,補博士弟子。郡國縣官有好文學,敬長上,肅政教,順鄉里,出入不悖,所聞,令相長丞上屬所二千石。二千石謹察可者,常與計偕,詣太常,得受業如弟子。一歲皆輒課,能通一藝以上,補文學掌故缺;其高第可以為郎中,太常籍奏。即有秀才異等,輒以名聞。其不事學若下材,及不能通一藝,輒罷之,而請諸能稱者。臣謹案詔書律令下者,明天人分際,通古今之誼,文章爾雅,訓辭深厚,恩施甚美。小吏淺聞,弗能究宣,亡以明布諭下。以治禮掌故以文學禮義為官,遷留滯。請選擇其秩比二百石以上及吏百石通一藝以上補左右內史、大行卒史,比百石以下補郡太守卒史,皆各二人,邊郡一人。先用誦多者,不足,擇掌故以補中二千石屬,文學掌故補郡屬,備員。請著功令。它如律令。」

制曰:「可。」自此以來,公卿大夫士吏彬彬多文學之士矣。

昭帝時舉賢良文學,增博士弟子員滿百人,宣帝末增倍之。元帝好儒,能通一經者皆復。數年,以用度不足,更為設員千人,郡國置五經百石卒史。成帝末,或言孔子布衣養徒三千人,今天子太學弟子少,於是增弟子員三千人。歲餘,復如故。平帝時王莽秉政,增元士之子得受業如弟子,勿以為員,歲課甲科四十人為郎中,乙科二十人為太子舍人,丙科四十人補文學掌故云。

自魯商瞿子木受易孔子,以授魯橋庇子庸。子庸授江東馯臂子弓。子弓授燕周醜子家。子家授東武孫虞子乘。子乘授齊田何子裝。及秦禁學,易為筮卜之書,獨不禁,故傳受者不絕也。漢興,田何以齊田徙杜陵,號杜田生,授東武王同子中、雒陽周王孫、丁寬、齊服生,皆著易傳數篇。同授淄川楊何,字叔元,元光中徵為太中大夫。齊即墨成,至城陽相。廣川孟但,為太子門大夫。魯周霸、莒衡胡、臨淄主父偃,皆以易至大官。要言易者本之田何。

丁寬字子襄,梁人一王梁項生從田何受易,時寬為項生從者,讀易精敏,材過項生,遂事何。學成,何謝寬。寬東歸,何謂門人曰:「易以東矣。」寬至雒陽,復從周王孫受古義,號周氏傳。景帝時,寬為梁孝王將軍距吳楚,號丁將軍,作易說三萬言,訓故舉大誼而已,今小章句是也。寬授同郡碭田王孫。王孫授施讎、孟喜、梁丘賀。繇是易有施、孟、梁丘之學。

施讎字長卿,沛人也。沛與碭相近。讎為童子,從田王孫受易。後讎徙長陵,田王孫為博士,復從卒業,與孟喜、梁丘賀並為門人。謙讓,常稱學廢,不教授。及梁丘賀為少府,事多,及遣子臨分將門人張禹等從讎問。讎自匿不肯見,賀固請,不得已乃授臨等。於是賀薦讎:「結髮事師數十年,賀不能及。」詔拜讎為博士。甘露中與五經諸儒雜論同異於石渠閣。讎授張禹、琅邪魯伯。伯為會稽太守,禹至丞相。禹授淮陽彭宣、沛戴崇子平。崇為九卿,宣大司空。禹、宣皆有傳。魯伯授太山毛莫如少路、琅邪邴丹曼容,著清名。莫如至常山太守。此其知名者也。繇是施家有張、彭之學。

孟喜字長卿,東海蘭陵人也。父號孟卿,善為禮、春秋,授后蒼、疏廣。世所傳后氏禮、疏氏春秋,皆出孟卿。孟卿以禮經多,春秋煩雜,乃使喜從田王孫受易。喜好自稱譽,得易家候陰陽災變書,詐言師田生且死時枕喜錾,獨傳喜,諸儒以此耀之。同門梁丘賀疏通證明之,曰:「田生絕於施讎手中,時喜歸東海,安得此事?」又蜀人趙賓好小數書,後為易,飾易文,以為「箕子明夷,陰陽氣亡箕子;箕子者,萬物方荄茲也。」賓持論巧慧,易家不能難,皆曰「非古法也」。云受孟喜,喜為名之。後賓死,莫能持其說。喜因不肯仞,以此不見信。喜舉孝廉為郎,曲臺署長,病免,為丞相掾。博士缺,眾人薦喜。上聞喜改師法,遂不用喜。喜授同郡白光少子、沛翟牧子兄,皆為博士。繇是有翟、孟、白之學。

梁丘賀字長翁,琅邪諸人也。以能心計,為武騎。從太中大夫京房受易。房者,淄川楊何弟子也。房出為齊郡太守,賀更事田王孫。宣帝時,聞京房為易明,求其門人,得賀。賀時為都司空令,坐事,論免為庶人。待詔黃門數入說教侍中,以召賀。賀入說,上善之,以賀為郎。會八月飲酎,行祠孝昭廟,先敺旄頭劍挺墮墬,首垂泥中,刃鄉乘輿車,馬驚。於是召賀筮之,有兵謀,不吉。上還,使有司侍祠。是時霍氏外孫代郡太守任宣坐謀反誅,宣子章為公車丞,亡在渭城界中,夜玄服入廟,居郎間,執戟立廟門,待上至,欲為逆。發覺,伏誅。故事,上常夜入廟,其後待明而入,自此始也。賀以筮有應,繇是近幸,為太中大夫,給事中,至少府。為人小心周密,上信重之。年老終官。傳子臨,亦入說,為黃門郎。甘露中,奉使問諸儒於石渠。臨學精孰,專行京房法。琅邪王吉通五經,聞臨說,善之。時宣帝選高材郎十人從臨講,吉乃使其子郎中駿上疏從臨受易。臨代五鹿充宗君孟為少府,駿御史大夫,自有傳。充宗授平陵士孫張仲方、沛鄧彭祖子夏、齊衡咸長賓。張為博士,至揚州牧,光祿大夫給事中,家世傳業;彭祖,真定太傅;咸,王莽講學大夫。繇是粱丘有士孫、鄧、衡之學。

京房受易梁人焦延壽。延壽云嘗從孟喜問易。會喜死,房以為延壽易即孟氏學,翟牧、白生不肯,皆曰非也。至成帝時,劉向校書,考易說,以為諸易家說皆祖田何、楊叔、丁將軍,大誼略同,唯京氏為異,黨焦延壽獨得隱士之說,託之孟氏,不相與同。房以明災異得幸,為石顯所譖誅,自有傳。房授東海殷嘉、河東姚平、河南乘弘,皆為郎、博士。繇是易有京氏之學。

費直字長翁,東萊人也。治易為郎,至單父令。長於卦筮,亡章句,徒以彖象系辭十篇文言解說上下經。琅邪王璜平中能傳之。璜又傳古文尚書。

高相,沛人也。治易與費公同時,其學亦亡章句,專說陰陽災異,自言出於丁將軍。傳至相,相授子康及蘭陵毋將永。康以明易為郎,永至豫章都尉。及王莽居攝,東郡太守翟誼謀舉兵誅莽,事未發,康候知東郡有兵,私語門人,門人上書言之。後數月,翟誼兵起,莽召問,對受師高康。莽惡之,以為惑眾,斬康。繇是易有高氏學。高、費皆未嘗立於學官。

伏生,濟南人也,故為秦博士。孝文時,求能治尚書者,天下亡有,聞伏生治之,欲召。時伏生年九十餘,老不能行,於是詔太常,使掌故朝錯往受之。秦時禁書,伏生壁藏之,其後大兵起,流亡。漢定,伏生求其書,亡數十篇,獨得二十九篇,即以教於齊、魯之間。齊學者由此頗能言尚書,山東大師亡不涉尚書以教。伏生教濟南張生及歐陽生。張生為博士,而伏生孫以治尚書徵,弗能明定。是後魯周霸、雒陽賈嘉頗能言尚書云。

歐陽生字和伯,千乘人也。事伏生,授倪寬。寬又受業孔安國,至御史大夫,自有傳。寬有俊材,初見武帝,語經學。上曰:「吾始以尚書為樸學,弗好,及聞寬說,可觀。」乃從寬問一篇。歐陽、大小夏侯氏學皆出於寬。寬授歐陽生子,世世相傳,至曾孫高子陽,為博士。高孫地餘長賓以太子中庶子授太子,後為博士,論石渠。元帝即位,地餘侍中,貴幸,至少府。戒其子曰:「我死,官屬即送汝財物,慎毋受。汝九卿儒者子孫,以廉絜著,可以自成。」及地餘死,少府官屬共送數百萬,其子不受。天子聞而嘉之,賜錢百萬。地餘少子政為王莽講學大夫。由是尚書世有歐陽氏學。

林尊字長賓,濟南人也。事歐陽高,為博士,論石渠。後至少府、太子太傅,授平陵平當、梁陳翁生。當至丞相,自有傳。翁生信都太傅,家世傳業。由是歐陽有平、陳之學。翁生授琅邪殷崇、楚國龔勝。崇為博士,勝右扶風,自有傳。而平當授九江朱普公文、上黨鮑宣。普為博士,宣司隸校尉,自有傳。徒眾尤盛,知名者也。

夏侯勝,其先夏侯都尉,從濟南張生受尚書,以傳族子始昌。始昌傳勝,勝又事同郡蕑卿。蕑卿者,倪寬門人。勝傳從兄子建,建又事歐陽高。勝至長信少府,建太子太傅,自有傳。由是尚書有大小夏侯之學。

周堪字少卿,齊人也。與孔霸俱事大夏侯勝。霸為博士。堪譯官令,論於石渠,經為最高,後為太子少傅,而孔霸以太中大夫授太子。及元帝即位,堪為光祿大夫,與蕭望之並領尚書事,為石顯等所譖,皆免官。望之自殺,上愍之,乃擢堪為光祿勳,語在劉向傳。堪授牟卿及長安許商長伯。牟卿為博士。霸以帝師賜爵號褒成君,傳子光,亦事牟卿,至丞相,自有傳。由是大夏侯有孔、許之學。商善為算,著五行論曆,四至九卿,號其門人沛唐林子高為德行,平陵吳章偉君為言語,重泉王吉少音為政事,齊炔欽幼卿為文學。王莽時,林、吉為九卿,自表上師冢,大夫博士郎吏為許氏學者,各從門人,會車數百兩,儒者榮之。欽、章皆為博士,徒眾尤盛。章為王莽所誅。

張山拊字長賓,平陵人也。事小夏侯建,為博士,論石渠,至少府。授同縣李尋、鄭寬中少君、山陽張無故子儒、信都秦恭延君、陳留假倉子驕。無故善修章句,為廣陵太傅,守小夏侯說文。恭增師法至百萬言,為城陽內史。倉以謁者論石渠,至膠東相。尋善說災異,為騎都尉,自有傳。寬中有俊材,以博士授太子,成帝即位,賜爵關內侯,食邑八百戶,遷光祿大夫,領尚書事,甚尊重。會疾卒,谷永上疏曰:「臣聞聖王尊師傅,褒賢俊,顯有功,生則致其爵祿,死則異其禮諡。昔周公薨,成王葬以變禮,而當天心。公叔文子卒,衛侯加以美諡,著為後法。近事,大司空朱邑、右扶風翁歸德茂夭年,孝宣皇帝愍冊厚賜,贊命之臣靡不激揚。關內侯鄭寬中有顏子之美質,包商、偃之文學,嚴然總五經之眇論,立師傅之顯位,入則鄉唐虞之閎道,王法納乎聖聽,出則參冢宰之重職,功列施乎政事,退食自公,私門不開,散賜九族,田畝不益,德配周召,忠合羔羊,未得登司徒,有家臣,卒然早終,尤可悼痛!臣愚以為宜加其葬禮,賜之令諡,以章尊師褒賢顯功之德。」上弔贈寬中甚厚。由是小夏侯有鄭、張、秦、假、李氏之學。寬中授東郡趙玄,無故授沛唐尊,恭授魯馮賓。賓為博士,尊王莽太傅,玄哀帝御史大夫,至大官,知名者也。

孔氏有古文尚書,孔安國以今文字讀之,因以起其家逸書,得十餘篇,蓋尚書茲多於是矣。遭巫蠱,未立於學官。安國為諫大夫,授都尉朝,而司馬遷亦從安國問故。遷書載堯典、禹貢、洪範、微子、金縢諸篇,多古文說。都尉朝授膠東庸生。庸生授清河胡常少子,以明穀梁春秋為博士、部刺史,又傳左氏。常授虢徐敖。敖為右扶風掾,又傳毛詩,授王璜、平陵塗惲子真。子真授河南桑欽君長。王莽時,諸學皆立。劉歆為國師,璜、惲等皆貴顯。世所傳百兩篇者,出東萊張霸,分析合二十九篇以為數十,又采左氏傳、書敘為作首尾,凡百二篇。篇或數簡,文意淺陋。成帝時求其古文者,霸以能為百兩徵,以中書校之,非是。霸辭受父,父有弟子尉氏樊並。時太中大夫平當、侍御史周敞勸上存之。後樊並謀反,乃黜其書。

申公,魯人也。少與楚元王交俱事齊人浮丘伯受詩。漢興,高祖過魯,申公以弟子從師人見于魯南宮。呂太后時,浮丘伯在長安,楚元王遣子郢與申公俱卒學。元王薨,郢嗣立為楚王,令申公傅太子戊。戊不好學,病申公。及戊立為王,胥靡申公。申公愧之,歸魯退居家教,終身不出門。復謝賓客,獨王命召之乃往。弟子自遠方至受業者千餘人,申公獨以詩經為訓故以教,亡傳,疑者則闕弗傳。蘭陵王臧既從受詩,已通,事景帝為太子少傅,免去。武帝初即位,臧乃上書宿衛,累遷,一歲至郎中令。及代趙綰亦嘗受詩申公,為御史大夫。綰、臧請立明堂以朝諸侯,不能就其事,乃言師申公。於是上使使束帛加璧,安車以蒲裹輪,駕駟迎申公,弟子兩人乘軺傳從。至,見上,上問治亂之事。申公時已八十餘,老,對曰:「為治者不至多言,顧力行何如耳。」是時上方好文辭,見申公對,默然。然已招致,既以為太中大夫,舍魯邸,議明堂事。太皇竇太后喜老子言,不說儒術,得綰、臧之過,以讓上曰:「此欲復為新垣平也!」上因廢明堂事,下綰、臧吏,皆自殺。申公亦病免歸,數年卒。弟子為博士十餘人,孔安國至臨淮太守,周负膠西內史,夏寬城陽內史,碭魯賜東海太守,蘭陵繆生長沙內史,徐偃膠西中尉,鄒人闕門慶忌膠東內史,其治官民皆有廉節稱。其學官弟子行雖不備,而至於大夫、郎、掌故以百數。申公卒以詩、春秋授,而瑕丘江公盡能傳之,徒眾最盛。及魯許生、免中徐公,皆守學教授。韋賢治詩,事博士大江公及許生,又治禮,至丞相。傳子玄成,以淮陽中尉論石渠,後亦至丞相。玄成及兄子賞以詩授哀帝,至大司馬車騎將軍,自有傳。由是魯詩有韋氏學。

王式字翁思,東平新桃人也。事免中徐公及許生。式為昌邑王師。昭帝崩,昌邑王嗣立,以行淫亂廢,昌邑群臣皆下獄誅,唯中尉王吉、郎中令龔遂以數諫減死論。式繫獄當死,治事使者責問曰:「師何以亡諫書?」式對曰:「臣以詩三百五篇朝夕授王,至於忠臣孝子之篇,未嘗不為王反復誦之也;至於危亡失道之君,未嘗不流涕為王深陳之也。臣以三百五篇諫,是以亡諫書。」使者以聞,亦得減死論,歸家不教授。山陽張長安幼君先事式,後東平唐長賓、沛褚少孫亦來事式,問經數篇,式謝曰:「聞之於師俱是矣,自潤色之。」不肯復授。唐生、褚生應博士弟子選,詣博士,摳衣登堂,頌禮甚嚴,試誦說,有法,疑者丘蓋不言。諸博士驚問何師,對曰事式。皆素聞其賢,共薦式。詔除下為博士。式徵來,衣博士衣而不冠,曰:「刑餘之人,何宜復充禮官?」既至,止舍中,會諸大夫博士,共持酒肉勞式,皆注意高仰之。博士江公世為魯詩宗,至江公著孝經說,心嫉式,謂歌吹諸生曰:「歌驪駒。」式曰:「聞之於師:客歌驪駒,主人歌客毋庸歸。今日諸君為主人,日尚早,未可也。」江翁曰:「經何以言之?」式曰:「在曲禮。」江翁曰:「何狗曲也!」式恥之,陽醉逿墬。式客罷,讓諸生曰:「我本不欲來,諸生彊勸我,竟為豎子所辱!」遂謝病免歸,終於家。張生、唐生、褚生皆為博士。張生論石渠,至淮陽中尉。唐生楚太傅。由是魯詩有張、唐、褚氏之學。張生兄子游卿為諫大夫,以詩授元帝。其門人琅邪王扶為泗水中尉,陳留許晏為博士。由是張家有許氏學。初,薛廣德亦事王式,以博士論石渠,授龔舍。廣德至御史大夫,舍泰山太守,皆有傳。

轅固,齊人也。以治詩孝景時為博士,與黃生爭論於上前。黃生曰:「湯武非受命,乃殺也。」固曰:「不然。夫桀紂荒亂,天下之心皆歸湯武,湯武因天下之心而誅桀紂,桀紂之民弗為使而歸湯武,湯武不得已而立,非受命而何?」黃生曰:「『冠雖敝必加於首,履雖新必貫於足。』何者?上下之分也。今桀紂雖失道,然君上也;湯武雖聖,臣下也。夫主有失行,臣不正言匡過以尊天子,反因過而誅之,代立南面,非殺而何?」固曰:「必若云,是高皇帝代秦即天子之位,非邪?」於是上曰:「

食肉毋食馬肝,未為不知味也;言學者毋言湯武受命,不為愚。」遂罷。竇太后好老子書,召問固。固曰:「此家人言耳。」太后怒曰:「安得司空城旦書乎!」乃使固入圈擊彘。上知太后怒,而固直言無罪,乃假固利兵。下,固刺彘正中其心,彘應手而倒。太后默然,亡以復罪。後上以固廉直,拜為清河太傅,疾免。武帝初即位,復以賢良徵。諸儒多嫉毀曰固老,罷歸之。時固己九十餘矣。公孫弘亦徵,仄目而事固。固曰:「公孫子,務正學以言,無曲學以阿世!」諸齊以詩顯貴,皆固之弟子也。昌邑太傅夏侯始昌最明,自有傳。

后蒼字近君,東海郯人也。事夏侯始昌。始昌通五經,蒼亦通詩禮,為博士,至少府,授翼奉、蕭望之、匡衡。奉為諫大夫,望之前將軍,衡丞相,皆有傳。衡授琅邪師丹、伏理斿君、潁川滿昌君都。君都為詹事,理高密太傅,家世傳業。丹大司空,自有傳。由是齊詩有翼、匡、師、伏之學。滿昌授九江張邯、琅邪皮容,皆至大官,徒眾尤盛。

韓嬰,燕人也。孝文時為博士,景帝時至常山太傅。嬰推詩人之意,而作內外傳數萬言,其語頗與齊、魯間殊,然歸一也。淮南賁生受之。燕趙間言詩者由韓生。韓生亦以易授人,推易意而為之傳。燕趙間好詩,故其易微,唯韓氏自傳之。武帝時,嬰嘗與董仲舒論於上前,其人精悍,處事分明,仲舒不能難也。後其孫商為博士。孝宣時,涿郡韓生其後也,以易徵,待詔殿中,曰:「所受易即先太傅所傳也。嘗受韓詩,不如韓氏易深,太傅故專傳之。」司隸校尉蓋寬饒本受易於孟喜,見涿韓生說易而好之,即更從受焉。

趙子,河內人也。事燕韓生,授同郡蔡誼。誼至丞相,自有傳。誼授同郡食子公與王吉。吉為昌邑中尉,自有傳。食生為博士,授泰山栗豐。吉授淄川長孫順。順為博士,豐部刺史。由是韓詩有王、食、長孫之學。豐授山陽張就,順授東海髮福,皆至大官,徒眾尤盛。

毛公,趙人也。治詩,為河間獻王博士,授同國貫長卿。長卿授解延年。延年為阿武令,授徐敖。敖授九江陳俠,為王莽講學大夫。由是言毛詩者,本之徐敖。

漢興,魯高堂生傳士禮十七篇,而魯徐生善為頌。孝文時,徐生以頌為禮官大夫,傳子至孫延、襄。襄,其資性善為頌,不能通經;延頗能,未善也。襄亦以頌為大夫,至廣陵內史。延及徐氏弟子公戶滿意、柏生、單次皆為禮官大夫。而瑕丘蕭奮以禮至淮陽太守。諸言禮為頌者由徐氏。

孟卿,東海人也。事蕭奮,以授后倉、魯閭丘卿。倉說禮數萬言,號曰后氏曲臺記,授沛聞人通漢子方、梁戴德延君、戴聖次君、沛慶普孝公。孝公為東平太傅。德號大戴,為信都太傅;聖號小戴,以博士論石渠,至九江太守。由是禮有大戴、小戴、慶氏之學。通漢以太子舍人論石渠,至中山中尉。普授魯夏侯敬,又傳族子咸,為豫章太守。大戴授琅邪徐良斿卿,為博士、州牧、郡守,家世傳業。小戴授梁人橋仁季卿、楊榮子孫。仁為大鴻臚,家世傳業,榮琅邪太守。由是大戴有徐氏,小戴有橋、楊氏之學。

胡母生字子都,齊人也。治公羊春秋,為景帝博士。與董仲舒同業,仲舒著書稱其德。年老,歸教於齊,齊之言春秋者宗事之,公孫弘亦頗受焉。而董生為江都相,自有傳。弟子遂之者,蘭陵褚大,東平嬴公,廣川段仲,溫呂步舒。大至梁相,步舒丞相長史,唯嬴公守學不失師法,為昭帝諫大夫,授東海孟卿、魯眭孟。孟為符節令,坐說災異誅,自有傳。

嚴彭祖字公子,東海下邳人也。與顏安樂俱事眭孟。孟弟子百餘人,唯彭祖、安樂為明,質問疑誼,各持所見。孟曰:「春秋之意,在二子矣!」孟死,彭祖、安樂各顓門教授。由是公羊春秋有顏、嚴之學。彭祖為宣帝博士,至河南、東郡太守。以高第入為左馮翊,遷太子太傅,廉直不事權貴。或說曰:「天時不勝人事,君以不修小禮曲意,亡貴人左右之助,經誼雖高,不至宰相。願少自勉強!」彭祖曰:「凡通經術,固當修行先王之道,何可委曲從俗,苟求富貴乎!」彭祖竟以太傅官終。授琅邪王中,為元帝少府,家世傳業。中授同郡公孫文、東門雲。雲為荊州刺史,文東平太傅,徒眾尤盛。雲坐為江賊拜辱命,下獄誅。

顏安樂字公孫,魯國薛人,眭孟姊子也。家貧,為學精力,官至齊郡太守丞,後為仇家所殺。安樂授淮陽泠豐次君、淄川任公。公為少府,豐淄川太守。由是顏家有泠、任之學。始貢禹事嬴公,成於眭孟,至御史大夫,疏廣事孟卿,至太子太傅,皆自有傳。廣授琅邪筦路,路為御史中丞。禹授潁川堂谿惠,惠授泰山冥都,都為丞相史。都與路又事顏安樂,故顏氏復有筦、冥之學。路授孫寶,為大司農,自有傳。豐授馬宮、琅邪左咸。咸為郡守九卿,徒眾尤盛。官至大司徒,自有傳。

瑕丘江公受穀梁春秋及詩於魯申公,傳子至孫為博士。武帝時,江公與董仲舒並。仲舒通五經,能持論,善屬文。江公吶於口,上使與仲舒議,不如仲舒。而丞相公孫弘本為公羊學,比輯其議,卒用董生。於是上因尊公羊家,詔太子受公羊春秋,由是公羊大興。太子既通,復私問穀梁而善之。其後浸微,唯魯榮廣王孫、皓星公二人受焉。廣盡能傳其詩、春秋,高材捷敏,與公羊大師眭孟等論,數困之,故好學者頗復受穀梁。沛蔡千秋少君、梁周慶幼君、丁姓子孫皆從廣受。千秋又事皓星公,為學最篤。宣帝即位,聞衛太子好穀梁春秋,以問丞相韋賢、長信少府夏侯勝及侍中樂陵侯史高,皆魯人也,言穀梁子本魯學,公羊氏乃齊學也,宜興穀梁。時千秋為郎,召見,與公羊家並說,上善穀梁說,擢千秋為諫大夫給事中,後有過,左遷平陵令。復求能為穀梁者,莫及千秋。上愍其學且絕,乃以千秋為郎中戶將,選郎十人從受。汝南尹更始翁君本自事千秋,能說矣,會千秋病死,徵江公孫為博士。劉向以故諫大夫通達待詔,受穀梁,欲令助之。江博士復死,乃徵周慶、丁姓待詔保宮,使卒授十人。自元康中始講,至甘露元年,積十餘歲,皆明習。乃召五經名儒太子太傅蕭望之等大議殿中,平公羊、穀梁同異,各以經處是非。時公羊博士嚴彭祖、侍郎申輓、伊推、宋顯,穀梁議郎尹更始、待詔劉向、周慶、丁姓並論。公羊家多不見從,願請內侍郎許廣,使者亦並內穀梁家中郎王亥,各五人,議三十餘事。望之等十一人各以經誼對,多從穀梁。由是穀梁之學大盛。慶、姓皆為博士。姓至中山太傅,授楚申章昌曼君,為博士,至長沙太傅,徒眾尤盛。尹更始為諫大夫、長樂戶將,又受左氏傳,取其變理合者以為章句,傳子咸及翟方進、琅邪房鳳。咸至大司農,方進丞相,自有傳。

房鳳字子元,不其人也。以射策乙科為太史掌故。太常舉方正,為縣令都尉,失官。大司馬票騎將軍王根奏除補長史,薦鳳明經通達,擢為光祿大夫,遷五官中郎將。時光祿勳王龔以外屬內卿,與奉車都尉劉歆共校書,三人皆侍中。歆白左氏春秋可立,哀帝納之,以問諸儒,皆不對。歆於是數見丞相孔光,為言左氏以求助,光卒不肯。唯鳳、龔許歆,遂共移書責讓太常博士,語在歆傳。大司空師丹奏歆非毀先帝所立,上於是出龔等補吏,龔為弘農,歆河內,鳳九江太守,至青州牧。始江博士授胡常,常授梁蕭秉君房,王莽時為講學大夫。由是穀梁春秋有尹、胡、申章、房氏之學。

漢興,北平侯張蒼及梁太傅賈誼、京兆尹張敞、太中大夫劉公子皆修春秋左氏傳。誼為左氏傳訓故,授趙人貫公,為河間獻王博士,子長卿為蕩陰令,授清河張禹長子。禹與蕭望之同時為御史,數為望之言左氏,望之善之,上書數以稱說。後望之為太子太傅,薦禹於宣帝,徵禹待詔,未及問,會疾死。授尹更始,更始傳子咸及翟方進、胡常。常授黎陽賈護季君,哀帝時待詔為郎,授蒼梧陳欽子佚,以左氏授王莽,至將軍。而劉歆從尹咸及翟方進受。由是言左氏者本之賈護、劉歆。

贊曰:自武帝立五經博士,開弟子員,設科射策,勸以官祿,訖於元始,百有餘年,傳業者寖盛,支葉藩滋,一經說至百餘萬言,大師眾至千餘人,蓋祿利之路然也。初,書唯有歐陽,禮后,易楊,春秋公羊而已。至孝宣世,復立大小夏侯尚書,大小戴禮,施、孟、梁丘易,穀梁春秋。至元帝世,復立京氏易。平帝時,又立左氏春秋、毛詩、逸禮、古文尚書,所以罔羅遺失,兼而存之,是在其中矣。


\end{pinyinscope}