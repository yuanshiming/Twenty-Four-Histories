\article{匡張孔馬傳}

\begin{pinyinscope}
匡衡字稚圭,東海承人也。父世農夫,至衡好學,家貧,庸作以供資用,尤精力過絕人。諸儒為之語曰:「無說詩,匡鼎來;匡說詩,解人頤。」

衡射策甲科,以不應令除為太常掌故,調補平原文學。學者多上書薦衡經明,當世少雙,令為文學就官京師;後進皆欲從衡平原,衡不宜在遠方。事下太子太傅蕭望之、少府梁丘賀問,衡對詩諸大義,其對深美。望之奏衡經學精習,說有師道,可觀覽。宣帝不甚用儒,遣衡歸官。而皇太子見衡對,私善之。

會宣帝崩,元帝初即位,樂陵侯史高以外屬為大司馬車騎將軍,領尚書事,前將軍蕭望之為副。望之名儒,有師傅舊恩,天子任之,多所貢薦。高充位而已,與望之有隙。長安令楊興說高曰:「

將軍以親戚輔政,貴重於天下無二,然眾庶論議令問休譽不專在將軍者何也?彼誠有所聞也。以將軍之莫府,海內莫不卬望,而所舉不過私門賓客,乳母子弟,人情以不自知,然一夫竊議,語流天下。夫富貴在身而列士不譽,是有狐白之裘而反衣之也。古人病其若此,故卑體勞心,以求賢為務。傳曰:以賢難得之故因曰事不待賢,以食難得之故而曰飽不待食,或之甚者也。平原文學匡衡材智有餘,經學絕倫,但以無階朝廷,故隨牒在遠方。將軍誠召置莫府,學士歙然歸仁,與參事議,觀其所有,貢之朝廷,必為國器,以此顯示眾庶,名流於世。」高然其言,辟衡為議曹史,薦衡於上,上以為郎中,遷博士,給事中。

是時,有日蝕地震之變,上問以政治得失,衡上疏曰:

臣聞五帝不同樂,三王各異教,民俗殊務,所遇之時異也。陛下躬聖德,開太平之路,閔愚吏民觸法抵禁,比年大赦,使百姓得改行自新,天下幸甚。臣竊見大赦之後,姦邪不為衰止,今日大赦,明日犯法,相隨入獄,此殆導之未得其務也。蓋保民者,「陳之以德義」,「示之以好惡」,觀其失而制其宜,故動之而和,綏之而安。今天下俗貪財賤義,好聲色,上侈靡,廉恥之節薄,淫辟之意縱,綱紀失序,疏者踰內,親戚之恩薄;婚姻之黨隆,苟合徼幸,以身設利。不改其原,雖歲赦之,刑猶難使錯而不用也。

臣愚以為宜壹曠然大變其俗。孔子曰:「能以禮讓為國乎,何有?」朝廷者,天下之楨幹也。公卿大夫相與循禮恭讓,則民不爭;好仁樂施,則下不暴;上義高節,則民興行;寬柔和惠,則眾相愛。四者,明王之所以不嚴而成化也。何者?朝有變色之言,則下有爭鬥之患;上有自專之士,則下有不讓之人;上有克勝之佐,則下有傷害之心;上有好利之臣,則下有盜竊之民:此其本也。今俗吏之治,皆不本禮讓,而上克暴,或忮害好陷人於罪,貪財而慕勢,故犯法者眾,姦邪不正,雖嚴刑峻法,猶不為變。此非其天性,有由然也。

臣竊考國風之詩,周南、召南被賢聖之化深,故篤於行而廉於色。鄭伯好勇,而國人暴虎;秦穆貴信,而士多從死;陳夫人好巫,而民淫祀;晉侯好儉,而民畜聚;太王躬仁,邠國貴恕。由此觀之,治天下者審所上而已。今之偽薄忮害,不讓極矣。臣聞教化之流,非家至而人說之也。賢者在位,能者布職,朝廷崇禮,百僚敬讓。道德之行,由內及外,自近者始,然後民知所法,遷善日進而不自知。是以百姓安,陰陽和,神靈應,而嘉祥見。《詩》曰:「商邑翼翼,四方之極;壽考且寧,以保我後生。」此成湯所以建至治,保子孫,化異俗而懷鬼方也。今長安天子之都,親承聖化,然其習俗無以異於遠方,郡國來者無所法則,或見侈靡而放效之。此教化之原本,風俗之樞機,宜先正者也。

臣聞天人之際,精祲有以相盪,善惡有以相推,事作乎下者象動乎上,陰陽之理各應其感,陰變則靜者動,陽蔽則明者晻,水旱之災隨類而至。今關東連年饑饉,百姓乏困,或至相食,此皆生於賦斂多,民所共者大,而吏安集之不稱之效也。陛下祗畏天戒,哀閔元元,大自減損,省甘泉、建章宮衛,罷珠崖,偃武行文,將欲度唐虞之隆,絕殷周之衰也。諸見罷珠崖詔書者,莫不欣欣,人自以將見太平也。宜遂減宮室之度,省靡麗之飾,考制度,修外內,近忠正,遠巧佞,放鄭衛,進雅頌,舉異材,開直言,任溫良之人,退刻薄之吏,顯絜白之士,昭無欲之路,覽六藝之意,察上世之務,明自然之道,博和睦之化,以崇至仁,匡失俗,易民視,令海內昭然咸見本朝之所貴,道德弘於京師,淑問揚乎疆外,然後大化可成,禮讓可興也。

上說其言,遷衡為光祿大夫、太子少傅。

時,上好儒術文辭,頗改宣帝之政,言事者多進見,人人自以為得上意。又傅昭儀及子定陶王愛幸,寵於皇后、太子。衡復上疏曰:

臣聞治亂安危之機,在乎審所用心。蓋受命之王務在創業垂統傳之無窮,繼體之君心存於承宣先王之德而褒大其功。昔者成王之嗣位,思述文武之道以養其心,休烈盛美皆歸之二后而不敢專其名,是以上天歆享,鬼神祐焉。其《詩》曰:「念我皇祖,陟降廷止。」言成王常思祖考之業,而鬼神祐助其治也。

陛下聖德天覆,子愛海內,然陰陽未和,姦邪未禁者,殆論議者未丕揚先帝之盛功,爭言制度不可用也,務變更之,所更或不可行,而復復之,是以群下更相是非,吏民無所信。臣竊恨國家釋樂成之業,而虛為此紛紛也。願陛上詳覽統業之事,留神於遵制揚功,以定群下之心。大雅曰:「無念爾祖,聿修厥德。」孔子著之孝經首章,蓋至德之本也。傳曰:「

審好惡,理情性,而王道畢矣。」能盡其性,然後能盡人物之性;能盡人物之性,可以贊天地之化。治性之道,必審己之所有餘,而強其所不足。蓋聰明疏通者戒於大察,寡聞少見者戒於雍蔽,勇猛剛強者戒於大暴,仁愛溫良者戒於無斷,湛靜安舒者戒於後時,廣心浩大者戒於遺忘。必審己之所當戒,而齊之以義,然後中和之化應,而巧偽之徒不敢比周而望進。唯陛下戒所以崇聖德。

臣又聞室家之道修,則天下之理得,故詩始國風,禮本冠婚。始乎國風,原情性而明人倫也;本乎冠婚,正基兆而防未然也。福之興莫不本乎室家,之道衰莫不始乎梱內。故聖王必慎妃后之際,別適長之位。禮之於內也,卑不隃尊,新不先故,所以統人情而理陰氣也。其尊適而卑庶也,適子冠乎阼,禮之用醴,眾子不得與列,所以貴正體而明嫌疑也。非虛加其禮文而已,乃中心與之殊異,故禮探其情而見之外也。聖人動靜游燕,所親物得其序;得其序,則海內自修,百姓從化。如當親者疏,當尊者卑,則佞巧之姦因時而動,以亂國家。故聖人慎防其端,禁於未然,不以私恩害公義。陛下聖德純備,莫不修正,則天下無為而治。《詩》云:「于以四方,克定厥家。」傳曰:「正家而天下定矣。」

衡為少傅數年,數上疏陳便宜,及朝廷有政議,傅經以對,言多法義。上以為任公卿,由是為光祿勳、御史大夫。建昭三年,代韋玄成為丞相,封樂安侯,食邑六百戶。

元帝崩,成帝即位,衡上疏戒妃匹,勸經學威儀之則,曰:

陛下秉至孝,哀傷思慕不絕於心,未有游虞弋射之宴,誠隆於慎終追遠,無窮已也。竊願陛下雖聖性得之,猶復加聖心焉。《詩》云「煢煢在疚」,言成王喪畢思慕,意氣未能平也,蓋所以就文武之業,崇大化之本也。

臣又聞之師曰:「妃匹之際,生民之始,萬福之原。」婚姻之禮正,然後品物遂而天命全。孔子論詩以關雎為始,言太上者民之父母,后夫人之行不侔乎天地,則無以奉神靈之統而理萬物之宜。故《詩》曰:「窈窕淑女,君子好仇。」言能致其貞淑,不貳其操,情欲之感無介乎容儀,宴私之意不形乎動靜,夫然後可以配至尊而為宗廟主。此綱紀之首,王教之端也,自上世已來,三代興廢,未有不由此者也。願陛下詳覽得失盛衰之效以定大基,采有德,戒聲色,近嚴敬,遠技能。

竊見聖德純茂,專精詩書,好樂無厭。臣衡材駑,無以輔相善義,宣揚德音。臣聞六經者,聖人所以統天地之心,著善惡之歸,明吉凶之分,通人道之正,使不悖於其本性者也。故審六藝之指,則人天之理可得而和,草木昆蟲可得而育,此永永不易之道也。及論語、孝經,聖人言行之要,宜究其意。

臣又聞聖王之自為動靜周旋,奉天承親,臨朝享臣,物有節文,以章人倫。蓋欽翼祗栗,事天之容也;溫恭敬遜,承親之禮也;正躬嚴恪,臨眾之儀也;嘉惠和說,饗下之顏也。舉錯動作,物遵其儀,故形為仁義,動為法則。孔子曰:「德義可尊,容止可觀,進退可度,以臨其民,是以其民畏而愛之,則而象之。」大雅云:「敬慎威儀,惟民之則。」諸侯正月朝覲天子,天子惟道德,昭穆穆以視之,又觀以禮樂,饗醴乃歸。故萬國莫不獲賜祉福,蒙化而成俗。今正月初幸路寢,臨朝賀,置酒以饗萬方,傳曰「君子慎始」,願陛下留神動靜之節,使群下得望盛德休光,以立基楨,天下幸甚!

上敬納其言。頃之,衡復奏正南北郊,罷諸淫祀,語在郊祀志。

初,元帝時,中書令石顯用事,自前相韋玄成及衡皆畏顯,不敢失其意。至成帝初即位,衡乃與御史大夫甄譚共奏顯,追條其舊惡,并及黨與。於是司隸校尉王尊劾奏:「衡、譚居大臣位,知顯等專權勢,作威福,為海內患害,不以時白奏行罰,而阿諛曲從,附下罔上,無大臣輔政之義。既奏顯等,不自陳不忠之罪,而反揚著先帝任用傾覆之徒,罪至不道。」有詔勿劾。衡慚懼,上疏謝罪,因稱病乞骸骨,上丞相樂安侯印綬。上報曰:「君以道德修明,位在三公,先帝委政,遂及朕躬。君遵修法度,勤勞公家,朕嘉與君同心合意,庶幾有成。今司隸校尉尊妄詆欺,加非於君,朕甚閔焉。方下有司問狀,君何疑而上書歸侯乞骸骨,是章朕之未燭也。傳不云乎?『禮義不愆,何恤人之言!』君其察焉。專精神,近醫藥,強食自愛。」因賜上尊酒、養牛。衡起視事。上以新即位,褒優大臣,然群下多是王尊者。衡嘿嘿不自安,每有水旱,風雨不時,連乞骸骨讓位。上輒以詔書慰撫,不許。

久之,衡子昌為越騎校尉,醉殺人,繫詔獄。越騎官屬與昌弟且謀篡昌。事發覺,衡免冠徒跣待罪,天子使謁者詔衡冠履。而有司奏衡專地盜土,衡竟坐免。

初,衡封僮之樂安鄉,鄉本田隄封三千一百頃,南以閩佰為界。初元元年,郡圖誤以閩佰為平陵佰。積十餘歲,衡封臨淮郡,遂封真平陵佰以為界,多四百頃。至建始元年,郡乃定國界,上計簿,更定圖,言丞相府。衡謂所親吏趙殷曰:「主簿陸賜故居奏曹,習事曉知國界,署集曹掾。」明年治計時,衡問殷國界事:「曹欲柰何?」殷曰:「賜以為舉計,令郡實之。恐郡不肯從實,可令家丞上書。」衡曰:「顧當得不耳,何至上書?」亦不告曹使舉也,聽曹為之。後賜與屬明舉計曰:「案故圖,樂安鄉南以平陵佰為界,不足故而以閩佰為界,解何?」郡即復以四百頃付樂安國。衡遣從史之僮,收取所還田租穀千餘石入衡家。司隸校尉駿、少府忠行廷尉事劾奏「衡監臨盜所主守直十金以上。春秋之義,諸侯不得專地,所以壹統尊法制也。衡位三公,輔國政,領計簿,知郡實,正國界,計簿已定而背法制,專地盜土以自益,及賜、明阿承衡意,猥舉郡計,亂減縣界,附下罔上,擅以地附益大臣,皆不道。」於是上可其奏,勿治,丞相免為庶人,終於家。

子咸亦明經,歷位九卿。家世多為博士者。

張禹字子文,河內軹人也,至禹父徙家蓮白。禹為兒,數隨家至市,喜觀於卜相者前。久之,頗曉其別蓍布卦意,時從旁言。卜者愛之,又奇其面貌,謂禹父:「是兒多知,可令學經。」及禹壯,至長安學,從沛郡施讎受易,琅邪王陽、膠東庸生問論語,既皆明習,有徒眾,舉為郡文學。甘露中,諸儒薦禹,有詔太子太傅蕭望之問。禹對易及論語大義,望之善焉,奏禹經學精習,有師法,可試事。奏寑,罷歸故官。久之,試為博士。初元中,立皇太子,而博士鄭寬中以尚書授太子,薦言禹善論語。詔令禹授太子論語,由是遷光祿大夫。數歲,出為東平內史。

元帝崩,成帝即位,徵禹、寬中,皆以師賜爵關內侯,寬中食邑八百戶,禹六百戶。拜為諸吏光祿大夫,秩中二千石,給事中,領尚書事。是時,帝舅陽平侯王鳳為大將軍輔政專權,而上富於春秋,謙讓,方鄉經學,敬重師傅。而禹與鳳並領尚書,內不自安,數病上書乞骸骨,欲退避鳳。上報曰:「朕以幼年執政,萬機懼失其中,君以道德為師,故委國政。君何疑而數乞骸骨,忽忘雅素,欲避流言?朕無聞焉。君其固心致思,總秉諸事,推以孳孳,無違朕意。」加賜黃金百斤、養牛、上尊酒,太官致餐,侍醫視疾,使者臨問。禹惶恐,復起視事,河平四年代王商為丞相,封安昌侯。

為相六歲,鴻嘉元年以老病乞骸骨,上加優再三,乃聽許。賜安車駟馬,黃金百斤,罷就第,以列侯朝朔望,位特進,見禮如丞相,置從事史五人,益封四百戶。天子數加賞賜,前後數千萬。

禹為人謹厚,內殖貨財,家以田為業。及富貴,多買田至四百頃,皆涇、渭溉灌,極膏腴上賈。它財物稱是。禹性習知音聲,內奢淫,身居大第,後堂理絲竹筦弦。

禹成就弟子尤著者,淮陽彭宣至大司空,沛郡戴崇至少府九卿。宣為人恭儉有法度,而崇愷弟多智,二人異行。禹心親愛崇,敬宣而疏之。崇每候禹,常責師宜置酒設樂與弟子相娛。禹將崇入後堂飲食,婦女相對,優人筦弦鏗鏘極樂,昏夜乃罷。而宣之來也,禹見之於便坐,講論經義,日晏賜食,不過一肉卮酒相對。宣未嘗得至後堂。及兩人皆聞知,各自得也。

禹年老,自治冢塋,起祠室,好平陵肥牛亭部處地,又近延陵,奏請求之,上以賜禹,詔令平陵徙亭它所。曲陽侯根聞而爭之:「此地當平陵寢廟衣冠所出游道,禹為師傅,不遵謙讓,至求衣冠所游之道,又徙壞舊亭,重非所宜。孔子稱『賜愛其羊,我愛其禮』,宜更賜禹它地。」根雖為舅,上敬重之不如禹,根言雖切,猶不見從,卒以肥牛亭地賜禹。根由是害禹寵,數毀惡之。天子愈益敬厚禹。禹每病,輒以起居聞,車駕自臨問之。上親拜禹床下,禹頓首謝恩,歸誠,言「老臣有四男一女,愛女甚於男,遠嫁為張掖太守蕭咸妻,不勝父子私情,思與相近。」上即時徙咸為弘農太守。又禹小子未有官,上臨候禹,禹數視其小子,上即禹床下拜為黃門郎,給事中。

禹雖家居,以特進為天子師,國家每有大政,必與定議。永始、元延之間,日蝕地震尤數,吏民多上書言災異之應,譏切王氏專政所致。上懼變異數見,意頗然之,未有以明見,乃車駕至禹弟,辟左右,親問禹以天變,因用吏民所言王氏事示禹。禹自見年老,子孫弱,又與曲陽侯不平,恐為所怨。禹則謂上曰:「春秋二百四十二年間,日蝕三十餘,地震五十六,或為諸侯相殺,或夷狄侵中國。災變之異深遠難見,故聖人罕言命,不語怪神。性與天道,自子贛之屬不得聞,何況淺見鄙儒之所言!陛下宜修政事以善應之,與下同其福喜,此經義意也。新學小生,亂道誤人,宜無信用,以經術斷之。」上雅信愛禹,由此不疑王氏。後曲陽侯根及諸王子弟聞知禹言,皆喜說,遂親就禹。禹見時有變異,若上體不安,擇日絜齋露蓍,正衣冠立筮,得吉卦則獻其占,如有不吉,禹為感動憂色。

成帝崩,禹及事哀帝,建平二年薨,諡曰節侯。禹四子,長子宏嗣侯,官至太常,列於九卿。三弟皆為校尉散騎諸曹。

初,禹為師,以上難數對己問經,為論語章句獻之。始魯扶卿及夏侯勝、王陽、蕭望之、韋玄成皆說論語,篇第或異。禹先事王陽,後從庸生,采獲所安,最後出而尊貴。諸儒為之語曰:「欲為論,念張文。」由是學者多從張氏,餘家寖微。

孔光字子夏,孔子十四世之孫也。孔子生伯魚鯉,鯉生子思伋,伋生子上帛,帛生子家求,求生子真箕,箕生子高穿。穿生順,順為魏相。順生鮒,鮒為陳涉博士,死陳下。鮒弟子襄為孝惠博士,長沙太傅。襄生忠,忠生武及安國,武生延年。延年生霸,字次儒。霸生光焉。安國、延年皆以治尚書為武帝博士。安國至臨淮太守。霸亦治尚書,事太傅夏侯勝,昭帝末年為博士,宣帝時為太中大夫,以選授皇太子經,遷詹事,高密相。是時,諸侯王相在郡守上。

元帝即位,徵霸,以師賜爵關內侯,食邑八百戶,號褒成君,給事中,加賜黃金二百斤,第一區,徙名數于長安。霸為人謙退,不好權勢,常稱爵位泰過,何德以堪之!上欲致霸相位,自御史大夫貢禹卒,及薛廣德免,輒欲拜霸。霸讓位,自陳至三,上深知其至誠,乃弗用。以是敬之,賞賜甚厚。及霸薨,上素服臨弔者再,至賜東園祕器錢帛,策贈以列侯禮,諡曰烈君。

霸四子,長子福嗣關內侯。次子捷、捷弟喜皆列校尉諸曹。光,最少子也,經學尤明,年未二十,舉為議郎。光祿勳匡衡舉光方正,為諫大夫。坐議有不合,左遷虹長,自免歸教授。成帝初即位,舉為博士,數使錄冤獄,行風俗,振贍流民,奉使稱旨,由是知名。是時,博士選三科,高第為尚書,次為刺史,其不通政事,以久次補諸侯太傅。光以高第為尚書,觀故事品式,數歲明習漢制及法令。上甚信任之,轉為僕射,尚書令。有詔光周密謹慎,未嘗有過,加諸吏官,以子男放為侍郎,給事黃門。數年,遷諸吏光祿大夫,秩中二千石,給事中,賜黃金百斤,領尚書事。後為光祿勳,復領尚書,諸吏給事中如故。凡典樞機十餘年,守法度,修故事。上有所問,據經法以心所安而對,不希指苟合;如或不從,不敢強諫爭,以是久而安。時有所言,輒削草稿,以為章主之過,以奸忠直,人臣大罪也。有所薦舉,唯恐其人之聞知。沐日歸休,兄弟妻子燕語,終不及朝省政事。或問光:「溫室省中樹皆何木也?」光嘿不應,更答以它語,其不泄如是。光帝師傅子,少以經行自著,進官蚤成。不結黨友,養游說,有求於人。既性自守,亦其勢然也。徙光祿勳為御史大夫。

綏和中,上即位二十五年,無繼嗣,至親有同產弟中山孝王及同產弟子定陶王在。定陶王好學多材,於帝子行。而王祖母傅太后陰為王求漢嗣,私事趙皇后、昭儀及帝舅大司馬驃騎將軍王根,故皆勸上。上於是召丞相翟方進、御史大夫光、右將軍廉褒、後將軍朱博,皆引入禁中,議中山、定陶王誰宜為嗣者。方進、根以為定陶王帝弟之子,禮曰「昆弟之子猶子也」,「為其後者為之子也」,定陶王宜為嗣。褒、博皆如方進、根議。光獨以為禮立嗣以親,中山王先帝之子,帝親弟也,以尚書盤庚殷之及王為比,中山王宜為嗣。上以禮兄弟不相入廟,又皇后、昭儀欲立定陶王,故遂立為太子。光以議不中意,左遷廷尉。

光久典尚書,練法令,號稱詳平。時定陵侯淳于長坐大逆誅,長小妻迺始等六人皆以長事未發覺時棄去,或更嫁。及長事發,丞相方進、大司空武議,以為「令,犯法者各以法時律令論之,明有所訖也。長犯大逆時,迺始等見為長妻,已有當坐之罪,與身犯法無異。後乃棄去,於法無以解。請論。」光議以為「大逆無道,父母妻子同產無少長皆棄市,欲懲後犯法者也。夫婦之道,有義則合,無義則離。長未自知當坐大逆之法,而棄去迺始等,或更嫁,義已絕,而欲以為長妻論殺之,名不正,不當坐。」有詔光議是。

是歲,右將軍褒、後將軍博坐定陵、紅陽侯皆免為庶人。以光為左將軍,居右將軍官職,執金吾王咸為右將軍,居後將軍官職。罷後將軍官。數月,丞相方進薨,召左將軍光,當拜,已刻侯印書贊,上暴崩,即其夜於大行前拜受丞相博山侯印綬。

哀帝初即位,躬行儉約,省減諸用,政事由己出,朝廷翕然,望至治焉。褒賞大臣,益封光千戶。時成帝母太皇太后自居長樂宮,而帝祖母定陶傅太后在國邸,有詔問丞相、大司空:「定陶共王太后宜當何居?」光素聞傅太后為人剛暴,長於權謀,自帝在襁褓而養長教道至於成人,帝之立又有力。光心恐傅太后與政事,不欲令與帝旦夕相近,即議以為定陶太后宜改築宮。大司空何武曰:「可居北宮。」上從武言。北宮有紫房復道通未央宮,傅太后果從復道朝夕至帝所,求欲稱尊號,貴寵其親屬,使上不得直道而行。頃之,太后從弟子傅遷在左右尤傾邪,上免官遣歸故郡。傅太后怒,上不得已復留遷。光與大司空師丹奏言:「詔書『侍中駙馬都尉遷巧佞無義,漏泄不忠,國之賊也,免歸故郡。』復有詔止。天下疑惑,無所取信,虧損聖德,誠不小愆。陛下以變異連見,避正殿,見群臣,思求其故,至今未有所改。臣請歸遷故郡,以銷姦黨,應天戒。」卒不得遣,復為侍中。脅於傅太后,皆此類也。

又傅太后欲與成帝母俱稱尊號,群下多順指,言母以子貴,宜立尊號以厚孝道。唯師丹與光持不可。上重違大臣正議,又內迫傅太后,猗違者連歲。丹以罪免,而朱博代為大司空。光自先帝時議繼嗣有持異之隙矣,又重忤傅太后指,由是傅氏在位者與朱博為表裏,共毀譖光。後數月遂策免光曰:「丞相者,朕之股肱,所與共承宗廟,統理海內,輔朕之不逮以治天下也。朕既不明,災異重仍,日月無光,山崩河決,五星失行,是章朕之不德而股肱之不良也。君前為御史大夫,輔翼先帝,出入八年,卒無忠言嘉謀,今相朕,出入三年,憂國之風復無聞焉。陰陽錯謬,歲比不登,天下空虛,百姓饑饉,父子分散,流離道路,以十萬數。而百官群職曠廢,姦軌放縱,盜賊並起,或攻官寺,殺長吏。數以問君,君無怵惕憂懼之意,對毋能為。是以群卿大夫咸惰哉莫以為意,咎由君焉。君秉社稷之重,總百僚之任,上無以匡朕之闕,下不能綏安百姓。書不云乎?『毋曠庶官,天工人其代之。』於虖!君其上丞相博山侯印綬,罷歸。」

光退閭里,杜門自守。而朱博代為丞相,數月,坐承傅太后指妄奏事自殺。平當代為丞相,數月薨。王嘉復為丞相,數諫爭忤指。旬歲間閱三相,議者皆以為不及光。上由是思之。

會元壽元年正月朔日有蝕之,後十餘日傅太后崩。是月徵光詣公車,問日蝕事。光對曰:「臣聞日者,眾陽之宗,人君之表,至尊之象。君德衰微,陰道盛彊,侵蔽陽明,則日蝕應之。書曰『羞用五事』,『建用皇極』。如貌、言、視、聽、思失,大中之道不立,則咎徵薦臻,六極屢降。皇之不極,是為大中不立,其傳曰『時則有日月亂行』,謂朓、側匿,甚則薄蝕是也。又曰『六沴之作』,歲之朝曰三朝,其應至重。乃正月辛丑朔日有蝕之,變見三朝之會。上天聰明,苟無其事,變不虛生。《書》曰『惟先假王正厥事』,言異變之來,起事有不正也。臣聞師曰,天右與王者,故災異數見,以譴告之,欲其改更。若不畏懼,有以塞除,而輕忽簡誣,則凶罰加焉,其至可必。《詩》曰:『敬之敬之,天惟顯思,命不易哉!』又曰:『畏天之威,于時保之。』皆謂不懼者凶,懼之則吉也。陛下聖德聰明,兢兢業業,承順天戒,敬畏變異,勤心虛己,延見群臣,思求其故,然後敕躬自約,總正萬事,放遠讒說之黨,援納斷斷之介,退去貪殘之徒,進用賢良之吏,平刑罰,薄賦斂,恩澤加於百姓,誠為政之大本,應變之至務也。天下幸甚。書曰『天既付命正厥德』,言正德以順天也。又曰『天棐諶辭』,言有誠道,天輔之也。明承順天道在於崇德博施,加精致誠,孳孳而已。俗之祈禳小數,終無益於應天塞異,銷禍興福,較然甚明,無可疑惑。」

書奏,上說,賜光束帛,拜為光祿大夫,秩中二千石,給事中,位次丞相。詔光舉可尚書令者封上,光謝曰:「臣以朽材,前比歷位典大職,卒無尺寸之效,幸免罪誅,全保首領,今復拔擢,備內朝臣,與聞政事。臣光智謀淺短,犬馬齒臷,誠恐一旦顛仆,無以報稱。竊見國家故事,尚書以久次轉遷,非有踔絕之能,不相踰越。尚書僕射敞,公正勤職,通敏於事,可尚書令。謹封上。」敞以舉故,為東平太守。敞姓成公,東海人也。

光為大夫月餘,丞相嘉下獄死,御史大夫賈延免。光復為御史大夫,二月復丞相,復故國博山侯。上乃知光前免非其罪,以過近臣毀短光者,復免傅嘉,曰:「前為侍中,毀譖仁賢,誣愬大臣,令俊艾者久失其位。嘉傾覆巧偽,挾姦以罔上,崇黨以蔽朝,傷善以肆意。詩不云乎?『讒人罔極,交亂四國。』其免嘉為庶人,歸故郡。」

明年,定三公官,光更為大司徒。會哀帝崩,太皇太后以新都侯王莽為大司馬,徵立中山王、是為平帝。帝年幼,太后稱制,委政於莽。初,哀帝罷黜王氏,故太后與莽怨丁、傅、董賢之黨。莽以光為舊相名儒,天下所信,太后敬之,備禮事光。所欲搏擊,輒為草,以太后指風光令上之,劯眥莫不誅傷。莽權日盛,光憂懼不知所出,上書乞骸骨。莽白太后:「帝幼少,宜置師傅。」徙光為帝太傅,位四輔,給事中,領宿衛供養,行內署門戶,省服御食物。明年,徙為太師,而莽為太傅。光常稱疾,不敢與莽並。有詔朝朔望,領城門兵。莽又風群臣奏莽功德,稱宰衡,位在諸侯王上,百官統焉。光愈恐,固稱疾辭位。太后詔曰:「太師光,聖人之後,先師之子,德行純淑,道術通明,居四輔職,輔道于帝。今年耆有疾,俊艾大臣,惟國之重,其猶不可以闕焉。書曰『無遺耇老』,國之將興,尊師而重傅。其令太師毋朝,十日一賜餐。賜太師靈壽杖,黃門令為太師省中坐置几,太師入省中用杖,賜餐十七物,然後歸老于第,官屬按職如故。」

光凡為御史大夫、丞相各再,壹為大司徒、太傅、太師,歷三世,居公輔位前後十七年。自為尚書,止不教授,後為卿,時會門下大生講問疑難,舉大義云。其弟子多成就為博士大夫者,見師居大位,幾得其助力,光終無所薦舉,至或怨之。其公如此。

光年七十,元始五年薨。莽白太后,使九卿策贈以太師博山侯印綬,賜乘輿祕器,金錢雜帛。少府供張,諫大夫持節與謁者二人使護喪事,博士護行禮。太后亦遣中謁者持節視喪。公卿百官會弔送葬。載以乘輿轀輬及副各一乘,羽林孤兒諸生合四百人輓送,車萬餘兩,道路皆舉音以過喪。將作穿復土,可甲卒五百人,起墳如大將軍王鳳制度。諡曰簡烈侯。

初,光以丞相封,後益封,凡食邑萬一千戶。病甚,上書讓還七千戶,及還所賜一弟。

子放嗣。莽篡位後,以光兄子永為大司馬,封侯。昆弟子至卿大夫四五人。始光父霸以初元元年為關內侯食邑。霸上書求奉孔子祭祀,元帝下詔曰:「其令師褒成君關內侯霸以所食邑八百戶祀孔子焉。」故霸還長子福名數於魯,奉夫子祀。霸薨,子福嗣。福薨,子房嗣。房薨,子莽嗣。元始元年,封周公、孔子後為列侯,食邑各二千戶。莽更封為褒成侯,後避王莽,更名均。

馬宮字游卿,東海戚人也。治春秋嚴氏,以射策甲科為郎,遷楚長史,免官。後為丞相史司直。師丹薦宮行能高絜,遷廷尉平,青州刺史,汝南、九江太守,所在見稱。徵為詹事,光祿勳,右將軍,代孔光為大司徒,封扶德侯。光為太師薨,宮復代光為太師,兼司徒官。

初,宮哀帝時與丞相御史雜議帝祖母傅太后諡,及元始中,王莽發傅太后陵徙歸定陶,以民葬之,追誅前議者。宮為莽所厚,獨不及,內慚懼,上書謝罪乞骸骨。莽以太皇太后詔賜宮策曰:「太師大司徒扶德侯上書言『前以光祿勳議故定陶共王母諡,曰「婦人以夫爵尊為號,諡宜曰孝元傅皇后,稱渭陵東園。」臣知妾不得體君,卑不得敵尊,而希指雷同,詭經辟說,以惑誤上。為臣不忠,當伏斧鉞之誅,幸蒙洒心自新,又令得保首領。伏自惟念,入稱四輔,出備三公,爵為列侯,誠無顏復望闕廷,無心復居官府,無宜復食國邑。願上太師大司徒扶德侯印綬,避賢者路。』下君章有司,皆以為四輔之職為國維綱,三公之任鼎足承君,不有鮮明固守,無以居位。如君言至誠可聽,惟君之惡在洒心前,不敢文過,朕甚多之,不奪君之爵邑,以著『自古皆有死』之義。其上太師大司徒印綬使者,以侯就弟。」王莽篡位,以宮為太子師,卒官。

本姓馬矢,宮仕學,稱馬氏云。

贊曰:自孝武興學,公孫弘以儒相,其後蔡義、韋賢、玄成、匡衡、張禹、翟方進、孔光、平當、馬宮及當子晏咸以儒宗居宰相位,服儒衣冠,傳先王語,其醞藉可也,然皆持祿保位,被阿諛之譏。彼以古人之跡見繩,烏能勝其任乎!


\end{pinyinscope}