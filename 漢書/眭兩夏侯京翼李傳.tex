\article{眭兩夏侯京翼李傳}

\begin{pinyinscope}
眭弘字孟,魯國蕃人也。少時好俠,鬥雞走馬,長乃變節,從嬴公受春秋。以明經為議郎,至符節令。

孝昭元鳳三年正月,泰山萊蕪山南匈匈有數千人聲,民視之,有大石自立,高丈五尺,大四十八圍,入地深八尺,三石為足。石立後有白烏數千下集其旁。是時昌邑有枯社木臥復生,又上林苑中大柳樹斷枯臥地,亦自立生,有蟲食樹葉成文字,曰「公孫病已立」,孟推春秋之意,以為「石柳皆陰類,下民之象,而泰山者岱宗之嶽,王者易姓告代之處。今大石自立,僵柳復起,非人力所為,此當有從匹夫為天子者。枯社木復生,故廢之家公孫氏當復興者也。」孟意亦不知其所在,即說曰:「先師董仲舒有言,雖有繼體守文之君,不害聖人之受命。漢家堯後,有傳國之運。漢帝宜誰差天下,求索賢人,襢以帝位,而退自封百里,如殷周二王後,以承順天命。」孟使友人內官長賜上此書。時,昭帝幼,大將軍霍光秉政,惡之,下其書廷尉。奏賜、孟妄設祅言惑眾,大逆不道,皆伏誅。後五年,孝宣帝興於民間,即位,徵孟子為郎。

夏侯始昌,魯人也。通五經,以齊詩、尚書教授。自董仲舒、韓嬰死後,武帝得始昌,甚重之。始昌明於陰陽,先言柏梁臺災日,至期日果災。時昌邑王以少子愛,上為選師,始昌為太傅。年老,以壽終。族子勝亦以儒顯名。

夏侯勝字長公。初,魯共王分魯西寧鄉以封子節侯,別屬大河,大河後更名東平,故勝為東平人。勝少孤,好學,從始昌受尚書及洪範五行傳,說災異。後事蕑卿,又從歐陽氏問。為學精孰,所問非一師也。善說禮服。徵為博士、光祿大夫。會昭帝崩,昌邑王嗣立,數出。勝當乘輿前諫曰:「天久陰而不雨,臣下有謀上者,陛下出欲何之?」王怒,謂勝為祅言,縛以屬吏。吏白大將軍霍光,光不舉法。是時,光與車騎將軍張安世謀欲廢昌邑王。光讓安世以為泄語,安世實不言。乃召問勝,勝對言:「在洪範傳曰『皇之不極,厥罰常陰,時則下人有伐上者』,惡察察言,故云臣下有謀。」光、安世大驚,以此益重經術士。後十餘日,光卒與安世共白太后,廢昌邑王,尊立宣帝。光以為群臣奏事東宮,太后省政,宜知經術,白令勝用尚書授太后。遷長信少府,賜爵關內侯,以與謀廢立,定策安宗廟,益千戶。

宣帝初即位,欲褒先帝,詔丞相御史曰:「朕以眇身,蒙遺德,承聖業,奉宗廟,夙夜惟念。孝武皇帝躬仁誼,厲威武,北征匈奴,單于遠遁,南平氐羌、昆明、甌駱兩越,東定薉、貉、朝鮮,廓地斥境,立郡縣,百蠻率服,款塞自至,珍貢陳於宗廟;協音律,造樂歌,薦上帝,封太山,立明堂,改正朔,易服色;明開聖緒,尊賢顯功,興滅繼絕,褒周之後;備天地之禮,廣道術之路。上天報況,符瑞並應,寶鼎出,白麟獲,海效鉅魚,神人並見,山稱萬歲。功德茂盛,不能盡宣,而廟樂未稱,朕甚悼焉。其與列侯、二千石、博士議。」於是群臣大議廷中,皆曰:「宜如詔書。」長信少府勝獨曰:「武帝雖有攘四夷廣土斥境之功,然多殺士眾,竭民財力,奢泰亡度,天下虛耗,百姓流離,物故者過半。蝗蟲大起,赤地數千里,或人民相食,畜積至今未復。亡德澤於民,不宜為立廟樂。」公卿共難勝曰:「此詔書也。」勝曰:「詔書不可用也。人臣之誼,宜直言正論,非苟阿意順指。議已出口,雖死不悔。」於是丞相義、御史大夫廣明劾奏勝非議詔書,毀先帝,不道,及丞相長史黃霸阿縱勝,不舉劾,俱下獄。有司遂請尊孝武帝廟為世宗廟,奏盛德、文始、五行之舞,天下世世獻納,以明盛德。武帝巡狩所幸郡國凡四十九,皆立廟,如高祖、太宗焉。

勝、霸既久繫,霸欲從勝受經,勝辭以罪死。霸曰:「『朝聞道,夕死可矣』。」勝賢其言,遂授之。繫再更冬,講論不怠。

至四年夏,關東四十九郡同日地動,或山崩,壞城郭室屋,殺六千餘人。上乃素服,避正殿,遣使者弔問吏民,賜死者棺錢。下詔曰:「蓋災異者,天地之戒也。朕承洪業,託士民之上,未能和群生。曩者地震北海、琅邪,壞祖宗廟,朕甚懼焉。其與列侯、中二千石博問術士,有以應變,補朕之闕,毋有所諱。」因大赦,勝出為諫大夫給事中,霸為揚州刺史。

勝為人質樸守正,簡易亡威儀。見時謂上為君,誤相字於前,上亦以是親信之。嘗見,出道上語,上聞而讓勝,勝曰:「陛下所言善,臣故揚之。堯言布於天下,至今見誦。臣以為可傳,故傳耳。」朝廷每有大議,上知勝素直,謂曰:「先生通正言,無懲前事。」

勝復為長信少府,遷太子太傅。受詔撰尚書、論語說,賜黃金百斤。年九十卒官,賜冢塋,葬平陵。太后賜錢二百萬,為勝素服五日,以報師傅之恩,儒者以為榮。

始,勝每講授,常謂諸生曰:「士病不明經術;經術苟明,其取青紫如俛拾地芥耳。學經不明,不如歸耕。」

勝從父子建字長卿,自師事勝及歐陽高,左右采獲,又從五經諸儒問與尚書相出入者,牽引以次章句,具文飾說。勝非之曰:「建所謂章句小儒,破碎大道。」建亦非勝為學疏略,難以應敵。建卒自顓門名經,為議郎博士,至太子少傅。勝子兼為左曹太中大夫,孫堯至長信少府、司農、鴻臚,曾孫蕃郡守、州牧、長樂少府。勝同產弟子賞為梁內史,梁內史子定國為豫章太守。而建子千秋亦為少府、太子少傅。

京房字君明,東郡頓丘人也。治易,事梁人焦延壽。延壽字贛。贛貧賤,以好學得幸梁王,王共其資用,令極意學。既成,為郡史,察舉補小黃令。以候司先知姦邪,盜賊不得發。愛養吏民,化行縣中。舉最當遷,三老官屬上書願留贛,有詔許增秩留,卒於小黃。贛常曰:「得我道以亡身者,必京生也。」其說長於災變,分六十四卦,更直日用事,以風雨寒溫為候:各有占驗。房用之尤精。好鍾律,知音聲。初元四年以孝廉為郎。

永光、建昭間,西羌反,日蝕,又久青亡光,陰霧不精。房數上疏,先言其將然,近數月,遠一歲,所言屢中,天子說之。數召見問,房對曰:「古帝王以功舉賢,則萬化成,瑞應著,末世以毀譽取人,故功業廢而致災異。宜令百官各試其功,災異可息。」詔使房作其事,房奏考功課吏法。上令公卿朝臣與房會議溫室,皆以房言煩碎,令上下相司,不可許。上意鄉之。時部刺史奏事京師,上召見諸刺史,令房曉以課事,刺史復以為不可行。唯御史大夫鄭弘、光祿大夫周堪初言不可,後善之。

是時中書令石顯顓權,顯友人五鹿充宗為尚書令,與房同經,論議相非。二人用事,房嘗宴見,問上曰:「幽厲之君何以危?所任者何人也?」上曰:「君不明,而所任者巧佞。」房曰:「知其巧佞而用之邪,將以為賢也?」上曰:「賢之。」房曰:「然則今何以知其不賢也?」上曰:「以其時亂而君危知之。」房曰:「

若是,任賢必治,任不肖必亂,必然之道也。幽厲何不覺寤而更求賢,曷為卒任不肖以至於是?」上曰:「臨亂之君各賢其臣,令皆覺寤,天下安得危亡之君?」房曰:「齊桓公、秦二世亦嘗聞此君而非笑之,然則任豎刁、趙高,政治日亂,盜賊滿山,何不以幽厲卜之而覺寤乎?」上曰:「唯有道者能以往知來耳。」房因免冠頓首,曰:「春秋紀二百四十二年災異,以視萬世之君。今陛下即位已來,日月失明,星辰逆行,山崩泉涌,地震石隕,夏霜冬雷,春凋秋榮,隕霜不殺,水旱螟蟲,民人飢疫,盜賊不禁,刑人滿市,春秋所記災異盡備。陸下視今為治邪,亂邪?」上曰:「亦極亂耳。尚何道!」房曰:「今所任用者誰與?」上曰:「然幸其瘉於彼,又以為不在此人也。」房曰:「夫前世之君亦皆然矣。臣恐後之視今,猶今之視前也。」上良久乃曰:「今為亂者誰哉?」房曰:「明主宜自知之。」上曰:「不知也;如知之,何故用之?」房曰:「上最所信任,與圖事帷幄之中進退天下之士者是矣。」房指謂石顯,上亦知之,謂房曰:「已諭。」

房罷出,後上令房上弟子曉知考功課吏事者,欲試用之。房上中郎任良、姚平,「願以為刺史,試考功法,臣得通籍殿中,為奏事,以防雍塞。」石顯、五鹿充宗皆疾房,欲遠之,建言宜試以房為郡守。元帝於是以房為魏郡太守,秩八百石,居得以考功法治郡。房自請,願無屬刺史,得除用它郡人,自第吏千石已下,歲竟乘傳奏事。天子許焉。

房自知數以論議為大臣所非,內與石顯、五鹿充宗有隙,不欲遠離左右,及為太守,憂懼。房以建昭二年二月朔拜,上封事曰:「辛酉以來,蒙氣衰去,太陽精明,臣獨欣然,以為陛下有所定也。然少陰倍力而乘消息。臣疑陛下雖行此道,猶不得如意,臣竊悼懼。守陽平侯鳳欲見未得,至己卯,臣拜為太守,此言上雖明下猶勝之效也。臣出之後,恐必為用事所蔽,身死而功不成,故願歲盡乘傳奏事,蒙哀見許。乃辛巳,蒙氣復乘卦,太陽侵色,此上大夫覆陽而上意疑也。己卯、庚辰之間,必有欲隔絕臣令不得乘傳奏事者。」

房未發,上令陽平侯鳳承制詔房,止無乘傳奏事。房意愈恐,去至新豐,因郵上封事曰:「臣以六月中言遯卦不效,法曰:『道人始去,寒,涌水為災。』至其七月,涌水出。臣弟子姚平謂臣曰:『房可謂知道,未可謂信道也。房言災異,未嘗不中,今涌水已出,道人當逐死,尚復何言?』臣曰:『陛下至仁,於臣尤厚,雖言而死,臣猶言也。』平又曰:『房可謂小忠,未可謂大忠也。昔秦時趙高用事,有正先者,非刺高而死,高威自此成,故秦之亂,正先趣之。』今臣得出守郡,自詭效功,恐未效而死。惟陛下毋使臣塞涌水之異,當正先之死,為姚平所笑。」

房至陝,復上封事曰:「乃丙戌小雨,丁亥蒙氣去,然少陰并力而乘消息,戊子益甚,到五十分,蒙氣復起。此陛下欲正消息,雜卦之黨并力而爭,消息之氣不勝。彊弱安危之機不可不察。己丑夜,有還風,盡辛卯,太陽復侵色,至癸巳,日月相薄,此邪陰同力而太陽為之疑也。臣前白九年不改,必有星亡之異。臣願出任良試考功,臣得居內,星亡之異可去。議者知如此於身不利,臣不可蔽,故云使弟子不若試師。臣為刺史又當奏事,故復云為刺史恐太守不與同心,不若以為太守,此其所以隔絕臣也。陛下不違其言而遂聽之,此乃蒙氣所以不解,太陽亡色者也。臣去朝稍遠,太陽侵色益甚,唯陛下毋難還臣而易逆天意。邪說雖安于人,天氣必變,故人可欺,天不可欺也。願陛下察焉。」房去月餘,竟徵下獄。

初,淮陽憲王舅張博從房受學,以女妻房。房與相親,每朝見,輒為博道其語,以為上意欲用房議,而群臣惡其害己,故為眾所排。博曰:「淮陽王上親弟,敏達好政,欲為國忠。今欲令王上書求入朝,得佐助房。」房曰:「得無不可?」博曰:「

前楚王朝薦士,何為不可?」房曰:「中書令石顯、尚書令五鹿君相與合同,巧佞之人也,事縣官十餘年;及丞相韋侯,皆久亡補於民,可謂亡功矣。此尤不欲行考功者也。淮陽王即朝見,勸上行考功,事善;不然,但言丞相、中書令任事久而不治,可休丞相,以御史大夫鄭弘代之,遷中書令置他官,以鉤盾令徐立代之,如此,房考功事得施行矣。」博具從房記諸所說災異事,固令房為淮陽王作求朝奏草,皆持柬與淮陽王。石顯微司具知之,以房親近,未敢言。及房出守郡,顯告房與張博通謀,非謗政治,歸惡天子,詿誤諸侯王,語在憲王傳。初,房見道幽厲事,出為御史大夫鄭弘言之。房、博皆棄巿,弘坐免為庶人。房本姓李,推律自定為京氏,死時年四十一。

翼奉字少君,東海下邳人也。治齊詩,與蕭望之、匡衡同師。三人經術皆明,衡為後進,望之施之政事,而奉惇學不仕,好律曆陰陽之占。元帝初即位,諸儒薦之,徵待詔宦者署,數言事宴見,天子敬焉。

時,平昌侯王臨以宣布外屬侍中,稱詔欲從奉學其術。奉不肯與言,而上封事曰:「臣聞之於師,治道要務,在知下之邪正。人誠鄉正,雖愚為用;若乃懷邪,知益為害。知下之術,在於六情十二律而已。北方之情,好也;好行貪狼,申子主之。東方之情,怒也;怒行陰賊,亥卯主之。貪狼必待陰而後動,陰賊必待貪狼而後用,二陰並行,是以王者忌子卯也。禮經避之,春秋諱焉。南方之情,惡也;惡行廉貞,寅午主之。西方之情,喜也;喜行寬大,巳酉主之。二陽並行,是以王者吉午酉也。《詩》曰:『吉日庚午。』上方之情,樂也;樂行姦邪,辰未主之。下方之情,哀也;哀行公正,戌丑主之。九辰未屬陰,戌丑屬陽,萬物各以其類應。今陛下明聖虛靜以待物至,萬事雖眾,何聞而不諭,豈況乎執十二律而御六情!於以知下參實,亦甚優矣,萬不失一,自然之道也。乃正月癸未日加申,有暴風從西南來。未主姦邪,申主貪狼,風以大陰下抵建前,是人主左右邪臣之氣也。平昌侯比三來見臣,皆以正辰加邪時。辰為客,時為主人。以律知人情,王者之祕道也,愚臣誠不敢以語邪人。」

上以奉為中郎,召問奉:「來者以善日邪時,孰與邪日善時?」奉對曰:「師法用辰不用日。辰為客,時為主人。見於明主,侍者為主人。辰正時邪,見者正,侍者邪;辰邪時正,見者邪,侍者正。忠正之見,侍者雖邪,辰時俱正;大邪之見,侍者雖正,辰時俱邪。即以自知侍者之邪,而時邪辰正,見者反邪;即以自知侍者之正,而時正辰邪,見者反正。辰為常事,時為一行。辰疏而時精,其效同功,必參五觀之,然後可知。故曰:察其所繇,省其進退,參之六合五行,則可以見人性,知人情。難用外察,從中甚明,故詩之為學,情性而已。五性不相害,六情更興廢。觀性以曆,觀情以律,明主所宜獨用,難與二人共也。故曰:『顯諸仁,臧諸用。』露之則不神,獨行則自然矣,唯奉能用之,學者莫能行。」

是歲,關東大水,郡國十一飢,疫尤甚。上乃下詔江海陂湖園池屬少府者以假貧民,勿租稅;損大官膳,減樂府員,省苑囿,諸宮館稀御幸者勿繕治;太僕少府減食穀馬,水衡省食肉獸。明年二月戊午,地震。其夏,齊地人相食。七月己酉,地復震。上曰:「蓋聞賢聖在位,陰陽和,風雨時,日月光,星辰靜,黎庶康寧,考終厥命。今朕共承天地,託于公侯之上,明不能燭,德不能綏,災異並臻,連年不息。乃二月戊午,地大震于隴西郡,毀落太上廟殿壁木飾,壞敗铠道縣城郭官寺及民室屋,厭殺人眾,山崩地裂,水泉涌出。一年地再動,天惟降災,震驚朕躬。治有大虧,咎至於此。夙夜兢兢,不通大變,深懷鬱悼,未知其序。比年不登,元元困乏,不勝飢寒,以陷刑辟,朕甚閔焉。憯怛於心。已詔吏虛倉廩,開府臧,振捄貧民。群司其茂思天地之戒,有可蠲除減省以便萬姓者,各條奏。悉意陳朕過失,靡有所諱。」因赦天下,舉直言極諫之士。奉奏封事曰:

臣聞之於師曰,天地設位,懸日月,布星辰,分陰陽,定四時,列五行,以視聖人,名之曰道。聖人見道,然後知王治之象,故畫州土,建君臣,立律曆,陳成敗,以視賢者,名之曰經。賢者見經,然後知人道之務,則詩、書、易、春秋、禮、樂是也。易有陰陽,詩有五際,春秋有災異,皆列終始,推得失,考天心,以言王道之安危。至秦乃不說,傷之以法,是以大道不通,至於滅亡。今陛下明聖,深懷要道,燭臨萬方,布德流惠,靡有闕遺。罷省不急之用,振救困貧,賦醫藥,賜棺錢,恩澤甚厚。又舉直言,求過失,盛德純備,天下幸甚。

臣奉竊學齊詩,聞五際之要十月之交篇,知日蝕地震之效昭然可明,猶巢居知風,穴處知雨,亦不足多,適所習耳。臣聞人氣內逆,則感動天地;天變見於星氣日蝕,地變見於奇物震動。所以然者,陽用其精,陰用其形,猶人之有五臧六體,五臧象天,六體象地。故臧病則氣色發於面,體病則欠申動於貌。今年太陰建於甲戌,律以庚寅初用事,曆以甲午從春。曆中甲庚,律得參陽,性中仁義,情得公正貞廉,百年之精歲也。正以精歲,本首王位,日臨中時接律而地大震,其後連月久陰,雖有大令,猶不能復,陰氣盛矣。古者朝廷必有同姓以明親親,必有異姓以明賢賢,此聖王之所以大通天下也。同姓親而易進,異姓疏而難通,故同姓一,異姓五,乃為平均。今左右亡同姓,獨以舅后之家為親,異姓之臣又疏。二后之黨滿朝,非特處位,勢尤奢僭過度,呂、霍、上官足以卜之,甚非愛人之道,又非後嗣之長策也。陰氣之盛,不亦宜乎!

臣又聞未央、建章、甘泉宮才人各以百數,皆不得天性。若杜陵園,其已御見者,臣子不敢有言,雖然,太皇太后之事也。及諸侯王園,與其後宮,宜為設員,出其過制者,此損陰氣應天救邪之道也。今異至不應,災將隨之。其法大水,極陰生陽,反為大旱,甚則有火災,春秋宋伯姬是矣。唯陛下財察。

明年夏四月乙未,孝武園白鶴館災。奉自以為中,上疏曰:「臣前上五際地震之效,曰極陰生陽,恐有火災。不合明聽,未見省答,臣竊內不自信。今白鶴館以四月乙未,時加於卯,月宿亢災,與前地震同法。臣奉乃深知道之可信也。不勝拳拳,願復賜間,卒其終始。」

上復延問以得失。奉以為祭天地於雲陽汾陰,及諸寢廟不以親疏迭毀,皆煩費,違古制。又宮室苑囿,奢泰難供,以故民困國虛,亡累年之畜。所繇來久,不改其本,難以末正,乃上疏曰:

臣聞昔者盤庚改邑以興殷道,聖人美之。竊聞漢德隆盛,在於孝文皇帝躬行節儉,外省繇役。其時未有甘泉、建章及上林中諸離宮館也。未央宮又無高門、武臺、麒麟、凰皇、白虎、玉堂、金華之殿,獨有前殿、曲臺、漸臺、宣室、溫室、承明耳。孝文欲作一臺,度用百金,重民之財,廢而不為,其積土基,至今猶存,又下遺詔,不起山墳。故其時天下大和,百姓洽足,德流後嗣。

如令處於當今,因此制度,必不能成功名。天道有常,王道亡常,亡常者所以應有常也。必有非常之主,然後能立非常之功。臣願陛下徙都於成周,左據成皋,左阻黽池,前鄉崧高,後介大河,建滎陽,扶河東,南北千里以為關,而入敖倉;地方百里者八九,足以自娛;東厭諸侯之權,西遠羌胡之難,陛下共己亡為,按成周之居,兼盤庚之德,萬歲之後,長為高宗。漢家郊兆寢廟祭祀之禮多不應古,臣奉誠難亶居而改作,故願陛下遷都正本。眾制皆定,亡復繕治宮館不急之費,歲可餘一年之畜。

臣聞三代之祖積德以王,然皆不過數百年而絕。周至成王,有上賢之材,因文武之業,以周召為輔,有司各敬其事,在位莫非其人。天下甫二世耳,然周公猶作詩書深戒成王,以恐失天下。書則曰:「王毋若殷王紂。」其詩則曰:「殷之未喪師,克配上帝;宜監于殷,駿命不易。」今漢初取天下,起於豐沛,以兵征伐,德化未洽,後世奢侈,國家之費當數代之用,非直費財,又乃費士。孝武之世,暴骨四夷,不可勝數。有天下雖未久,至於陛下八世九主矣,雖有成王之明,然亡周召之佐。今東方連年飢饉,加之以疾疫,百姓菜色,或至相食。地比震動,天氣溷濁,日光侵奪。繇此言之,執國政者豈可以不懷怵惕而戒萬分之一乎!故臣願陛下因天變而徙都,所謂與天下更始者也。天道終而復始,窮則反本,故能延長而亡窮也。今漢道未終,陛下本而始之,於以永世延祚,不亦優乎!如因丙子之孟夏,順太陰以東行,到後七年之明歲,必有五年之餘蓄,然後大行考室之禮,雖周之隆盛,亡以加此。唯陛下留神,詳察萬世之策。

書奏,天子異其意,答曰:「問奉:今園廟有七,云東徙,狀何如?」奉對曰:「昔成王徙洛,般庚遷殷,其所避就,皆陛下所明知也。非有聖明,不能一變天下之道。臣奉愚戇狂惑,唯陛下裁赦。」

其後,貢禹亦言當定迭毀禮,上遂從之。及匡衡為丞相,奏徙南北郊,其議皆自奉發之。

奉以中郎為博士、諫大夫,年老以壽終。子及孫,皆以學在儒官。

李尋字子長,平陵人也。治尚書,與張孺、鄭寬中同師。寬中等守師法教授,尋獨好洪範災異,又學天文月令陰陽。事丞相翟方進,方進亦善為星曆,除尋為吏,數為翟侯言事。帝舅曲陽侯王根為大司馬票騎將軍,厚遇尋。是時多災異,根輔政,數虛己問尋。尋見漢家有中衰阨會之象,其意以為且有洪水為災,乃說根曰:

《書》云「天聰明,」蓋言紫宮極樞,通位帝紀,太微四門,廣開大道,五經六緯,尊術顯士,翼張舒布,燭臨四海,少微處士,為比為輔,故次帝廷,女宮在後。聖人承天,賢賢易色,取法於此。天官上相上將,皆顓面正朝,憂責甚重,要在得人。得人之效,成敗之機,不可不勉也。昔秦穆公說諓諓之言,任仡仡之勇,身受大辱,社稷幾亡。悔過自責,思惟黃髮,任用百里奚,卒伯西域,德列王道。二者禍福如此,可不慎哉!

夫士者,國家之大寶,功名之本也。將軍一門九侯,二十朱輪,漢興以來,臣子貴盛,未嘗至此。夫物盛必衰,自然之理,唯有賢友彊輔,庶幾可以保身命,全子孫,安國家。

書曰「曆象日月星辰」,此言仰視天文,俯察地理,觀日月消息,候星辰行伍,揆山川變動,參人民繇俗,以制法度,考禍福。舉錯誖逆,咎敗將至,徵兆為之先見。明君恐懼修正,側身博問,轉禍為福;不可救者,即蓄備以待之,故社稷亡憂。

竊見往者赤黃四塞,地氣大發,動土竭民,天下擾亂之徵也。彗星爭明,庶雄為桀,大寇之引也。此二者已頗效矣。城中訛言大水,奔走上城,朝廷驚駭,女孽入宮,此獨未效。間者重以水泉涌溢,旁宮闕仍出。月、太白入東井,犯積水,缺天淵。日數湛於極陽之色。羽氣乘宮,起風積雲。又錯以山崩地動,河不用其道。盛冬雷電,潛龍為孽。繼以隕星流彗,維、填上見,日蝕有背鄉。此亦高下易居,洪水之徵也。不憂不改,洪水乃欲盪滌,流彗乃欲埽除;改之,則有年亡期。故屬者頗有變改,小貶邪猾,日月光精,時雨氣應,此皇天右漢亡已也,何況致大改之!

宜急博求幽隱,拔擢天士,任以大職。諸闒茸佞諂,抱虛求進,及用殘賊酷虐聞者,若此之徒,皆嫉善憎忠,壞天文,敗地理,涌趯邪陰,湛溺太陽,為主結怨於民,宜以時廢退,不當得居位。誠必行之,凶災銷滅,子孫之福不旋日而至。政治感陰陽,猶鐵炭之低卬,見效可信者也。乃諸蓄水連泉,務通利之。修舊隄防,省池澤稅,以助損邪陰之盛。案行事,考變易,訛言之效,未嘗不至。請徵韓放,掾周敞、王望可與圖之。

根於是薦尋。哀帝初即位,召尋待詔黃門,使侍中衛尉傅喜問尋曰:「間者水出地動,日月失度,星辰亂行,災異仍重,極言毋有所諱。」尋對曰:

陛下聖德,尊天敬地,畏命重民,悼懼變異,不忘疏賤之臣,幸使重臣臨問,愚臣不足以奉明詔。竊見陛下新即位,開大明,除忌諱,博延名士,靡不並進。臣尋位卑術淺,過隨眾賢待詔,食太官,衣御府,久汙玉堂之署。比得召見,亡以自效。復特見延問至誠,自以逢不世出之命,願竭愚心,不敢有所避,庶幾萬分有一可采。唯棄須臾之間,宿留瞽言,考之文理,稽之五經,揆之聖意,以參天心。夫變異之來,各應象而至,臣謹條陳所聞。

《易》曰:「縣象著明,莫大乎日月。」夫日者,眾陽之長,輝光所燭,萬里同晷,人君之表也。故日將旦,清風發,群陰伏,君以臨朝,不牽於色。日初出,炎以陽,君登朝,佞不行,忠直進,不蔽障。日中輝光,君德盛明,大臣奉公。日將入,專以壹,君就房,有常節。君不修道,則日失其度,晻昧亡光。各有云為。其於東方作,日初出時,陰雲邪氣起者,法為牽於女謁,有所畏難;日出後,為近臣亂政;日中,為大臣欺誣;日且入,為妻妾役使所營。間者日尤不精,光明侵奪失色,邪氣珥蜺數作。本起於晨,相連至昏,其日出後至日中間差瘉。小臣不知內事,竊以日視陛下志操,衰於始初多矣。其咎恐有以守正直言而得罪者,傷嗣害世,不可不慎也。唯陛下執乾剛之德,彊志守度,毋聽女謁邪臣之態。諸保阿乳母甘言悲辭之託,斷而勿聽。勉強大誼,絕小不忍;良有不得已,可賜以財貨,不可私以官位,誠皇天之禁也。日失其光,則星辰放流。陽不能制陰,陰桀得作。間者太白正晝經天。宜隆德克躬,以執不軌。

臣聞月者,眾陰之長,銷息見伏,百里為品,千里立表,萬里連紀,妃后大臣諸侯之象也。朔晦正終始,弦為繩墨,望成君德,春夏南,秋冬北。間者,月數以春夏與日同道,過軒轅上后受氣,入太微帝廷楊光輝,犯上將近臣,列星皆失色,厭厭如滅,此為母后與政亂朝,陰陽俱傷,兩不相便。外臣不知朝事,竊信天文即如此,近臣已不足杖矣。屋大柱小,可為寒心。唯陛下親求賢士,無彊所惡,以崇社稷,尊彊本朝。

臣聞五星者,五行之精,五帝司命,應王者號令為之節度。歲星主歲事,為統首,號令所紀,今失度而盛,此君指意欲有所為,未得其節也。又填星不避歲星者,后帝共政,相留於奎、婁,當以義斷之。營惑往來亡常,周歷兩宮,作態低卬,入天門,上明堂,貫尾亂宮。太白發越犯庫,兵寇之應也。貫黃龍,入帝庭,當門而出,隨熒惑入天門,至房而分,欲與熒惑為患,不敢當明堂之精。此陛下神靈,故禍亂不成也。熒惑厥弛,佞巧依勢,微言毀譽,進類蔽善。太白出端門,臣有不臣者。火入室,金上堂,不以時解,其憂凶。填、歲相守,又主內亂。宜察蕭牆之內,毋忽親疏之微,誅放佞人,防絕萌牙,以盪滌濁濊,消散積惡,毋使得成禍亂。辰星主正四時,當效於四仲;四時失序,則辰星作異。今出於歲首之孟,天所以譴告陛下也。政急則出蚤,政緩則出晚,政絕不行則伏不見而為彗茀。四孟皆出,為易王命;四季皆出,星家所諱。今幸獨出寅孟之月,蓋皇天所以篤右陛下也,宜深自改。

治國故不可以戚戚,欲速則不達。經曰:「三載考績,三考黜陟。」加以號令不順四時,既往不咎,來事之師也。間者春三月治大獄,時賊陰立逆,恐歲小收;季夏舉兵法,時寒氣應,恐後有霜雹之災;秋月行封爵,其月土濕奧,恐後有雷雹之變。夫以喜怒賞罰,而不顧時禁,雖有堯舜之心,猶不能致和。善言天者,必有效於人。設上農夫而欲冬田,肉袒深耕,汗出種之,然猶不生者,非人心不至,天時不得也。《易》曰:「時止則止,時行則行,動靜不失其時,其道光明。」《書》曰:「敬授民時。」故古之王者,尊天地,重陰陽,敬四時,嚴月令。順之以善政,則和氣可立致,猶枹鼓之相應也。今朝廷忽於時月之令,諸侍中尚書近臣宜皆令通知月令之意,設群下請事;若陛下出令有謬於時者,當知爭之,以順時氣。

臣聞五行以水為本,其星玄武婺女,天地所紀,終始所生。水為準平,王道公正修明,則百川理,落脈通;偏黨失綱,則踊溢為敗。《書》云「水曰潤下」,陰動而卑,不失其道。天下有道,則河出圖,洛出書,故河、洛決溢,所為最大。今汝、潁畎澮皆川水漂踊,與雨水並為民害,此詩所謂「㷸㷸震電,不寧不令,百川沸騰」者也。其咎在於皇甫卿士之屬。唯陛下留意詩人之言,少抑外親大臣。

臣聞地道柔靜,陰之常義也。地有上中下,其上位震,應妃后不順,中位應大臣作亂,下位應庶民離畔。震或於其國,國君之咎也。四方中央連國歷州俱動者,其異最大。間者關東地數震,五星作異,亦未大逆,宜務崇陽抑陰,以救其咎;固志建威,閉絕私路,拔進英雋,退不任職,以彊本朝。夫本彊則精神折衝,本弱則招殃致凶,為邪謀所陵。聞往者淮南王作謀之時,其所難者,獨有汲黯,公孫弘等不足言也。弘,漢之名相,於今亡比,而尚見輕,何況亡弘之屬乎?故曰朝廷亡人,則為賊亂所輕,其道自然也。天下未聞陛下奇策固守之臣也。語曰,何以知朝廷之衰?人人自賢,不務於通人,故世陵夷。

馬不伏歷,不可以趨道;士不素養,不可以重國。《詩》曰「濟濟多士,文王以寧」,孔子曰「十室之邑,必有忠信」,非虛言也。陛下秉四海之眾,曾亡柱幹之固守聞於四境,殆開之不廣,取之不明,勸之不篤。傳曰:「土之美者善養禾,君之明者善養士。」中人皆可使為君子。詔書進賢良,赦小過,無求備,以博聚英俊。如近世貢禹,以言事忠切蒙尊榮,當此之時,士厲身立名者多。禹死之後,日日以衰。及京兆尹王章坐言事誅滅,智者結舌,邪偽並興,外戚顓命,君臣隔塞,至絕繼嗣,女宮作亂。此行事之敗,誠可畏而悲也。

本在積任母后之家,非一日之漸,往者不可及,來者猶可追也。先帝大聖,深見天意昭然,使陛下奉承天統,欲矯正之也。宜少抑外親,選練左右,舉有德行道術通明之士充備天官,然後可以輔聖德,保帝位,承大宗。下至郎吏從官,行能亡以異,又不通一藝,及博士無文雅者,宜皆使就南畝,以視天下,明朝廷皆賢材君子,於以重朝尊君,滅凶致安,此其本也。臣自知所言害身,不辟死亡之誅,唯財留神,反覆覆愚臣之言。

是時哀帝初立,成帝外家王氏未甚抑黜,而帝外家丁、傅新貴,祖母傅太后尤驕恣,欲稱尊號。丞相孔光、大司空師丹執政諫爭,久之,上不得已,遂免光、丹而尊傅太后。語在丹傳。上雖不從尋言,然采其語,每有非常,輒問尋。尋對屢中,遷黃門侍郎。以尋言且有水災,故拜尋為騎都尉,使護河隄。

初,成帝時,齊人甘忠可詐造天官曆、包元太平經十二卷,以言「漢家逢天地之大終,當更受命於天,天帝使真人赤精子,下教我此道。」忠可以教重平夏賀良、容丘丁廣世、東郡郭昌等,中壘校尉劉向奏忠可假鬼神罔上惑眾,下獄治服,未斷病死。賀良等坐挾學忠可書以不敬論,後賀良等復私以相教。哀帝初立,司隸校尉解光亦以明經通災異得幸,白賀良等所挾忠可書。事下奉車都尉劉歆,歆以為不合五經,不可施行。而李尋亦好之。光曰:「前歆父向奏忠可下獄,歆安肯通此道?」時郭昌為長安令,勸尋宜助賀良等。尋遂白賀良等皆待詔黃門,數召見,陳說「漢曆中衰,當更受命。成帝不應天命,故絕嗣。今陛下久疾,變異屢數,天所以譴告人也。宜急改元易號,乃得延年益壽,皇子生,災異息矣。得道不得行,咎殃且亡,不有洪水將出,災火且起,滌盪人民。」

哀帝久寢疾,幾其有益,遂從賀良等議。於是詔制丞相御史:「蓋聞尚書『五曰考終命』,言大運壹終,更紀天元人元,考文正理,推曆定紀,數如甲子也。朕以眇身入繼太祖,承皇天,總百僚,子元元,未有應天心之效。即位出入三年,災變數降,日月失度,星辰錯謬,高下貿易,大異連仍,盜賊並起。朕甚懼焉,戰戰兢兢,唯恐陵夷。惟漢興至今二百載,曆紀開元,皇天降非材之右,漢國再獲受命之符,朕之不德,曷敢不通夫受天之元命,必與天下自新。其大赦天下,以建平二年為太初元將元年,號曰陳聖劉太平皇帝。漏刻以百二十為度。布告天下,使明知之。」後月餘,上疾自若。賀良等復欲妄變政事,大臣爭以為不可許。賀良等奏言大臣皆不知天命,宜退丞相御史,以解光、李尋輔政。上以其言亡驗,遂下賀良等吏,而下詔曰:「朕獲保宗廟,為政不德,變異屢仍,恐懼戰栗,未知所繇。待詔賀良等建言改元易號,增益漏刻,可以永安國家。朕信道不篤,過聽其言,幾為百姓獲福。卒無嘉應,久旱為災。以問賀良等,對當復改制度,皆背經誼,違聖制,不合時宜。夫過而不改,是為過矣。六月甲子詔書,非赦令也,皆蠲除之。賀良等反道惑眾,姦態當窮竟。」皆下獄,光祿勳平當、光祿大夫毛莫如與御史中丞、廷尉雜治,當賀良等執左道,亂朝政,傾覆國家,誣罔主上,不道。賀良等皆伏誅。尋及解光減死一等,徙敦煌郡。

贊曰:幽贊神明,通合天人之道者,莫著乎易、春秋。然子贛猶云「夫子之文章可得而聞,夫子之言性與天道不可得而聞」已矣。漢興推陰陽言災異者,孝武時有董仲舒、夏侯始昌,昭、宣則眭孟、夏侯勝,元、成則京房、翼奉、劉向、谷永,哀、平則李尋、田終術。此其納說時君著明者也。察其所言,仿佛一端。假經設誼,依託象類,或不免乎「億則屢中」。仲舒下吏,夏侯囚執,眭孟誅戮,李尋流放,此學者之大戒也。京房區區,不量淺深,危言刺譏,構怨彊臣,罪辜不旋踵,亦不密以失身,悲夫!


\end{pinyinscope}