\article{宣帝紀}

\begin{pinyinscope}
孝宣皇帝,武帝曾孫,戾太子孫也。太子納史良娣,生史皇孫。皇孫納王夫人,生宣帝,號曰皇曾孫。生數月,遭巫蠱事,太子、良娣、皇孫、王夫人皆遇害。語在太子傳。曾孫雖在襁褓,猶坐收繫郡邸獄。而邴吉為廷尉監,治巫蠱於郡邸,憐曾孫之亡辜,使女徒復作淮陽趙徵卿、渭城胡組更乳養,私給衣食,視遇甚有恩。

巫蠱事連歲不決。至後元二年,武帝疾,往來長楊、五柞宮,望氣者言長安獄中有天子氣,上遣使者分條中都官獄繫者,輕重皆殺之。內謁者令郭穰夜至郡邸獄,吉拒閉,使者不得入,曾孫賴吉得全。因遭大赦,吉乃載曾孫送祖母史良娣家。語在吉及外戚傳。

後有詔掖庭養視,上屬籍宗正。時掖庭令張賀嘗事戾太子,思顧舊恩,哀曾孫,奉養甚謹,以私錢供給教書。既壯,為取暴室嗇夫許廣漢女,曾孫因依倚廣漢兄弟及祖母家史氏。受詩於東海澓中翁,高材好學,然亦喜游俠,鬥雞走馬,具知閭里奸邪,吏治得失。數上下諸陵,周遍三輔,常困於蓮勺鹵中。尤樂杜、鄠之間,率常在下杜。時會朝請,舍長安尚冠里,身足下有毛,臥居數有光燿。每買餅,所從買家輒大讎,亦以自是怪。

元平元年四月,昭帝崩,毋嗣。大將軍霍光請皇后徵昌邑王。六月丙寅,王受皇帝璽綬,尊皇后曰皇太后。癸巳,光奏王賀淫亂,請廢。語在賀及光傳。

秋七月,光奏議曰:「禮,人道親親故尊祖,尊祖故敬宗。大宗毋嗣,擇支子孫賢者為嗣。孝武皇帝曾孫病已,有詔掖庭養視,至今年十八,師受詩、論語、孝經,操行節儉,慈仁愛人,可以嗣孝昭皇帝後,奉承祖宗,子萬姓。」奏可。遣宗正德至曾孫尚冠里舍,洗沐,賜御府衣。太僕以軨獵車奉迎曾孫,就齊宗正府。庚申,入未央宮,見皇太后,封為陽武侯。已而群臣奉上璽綬,即皇帝位,謁高廟。

八月己巳,丞相敞薨。

九月,大赦天下。

十一月壬子,立皇后許氏。賜諸侯王以下金錢,至吏民鰥寡孤獨各有差。皇太后歸長樂宮。初置屯衛。

本始元年春正月,募郡國吏民訾百萬以上徙平陵。遣使者持節詔郡國二千石謹牧養民而風德化。

大將軍光稽首歸政,上謙讓委任焉。論定策功,益封大將軍光萬七千戶,車騎將軍光祿勳富平侯安世萬戶。詔曰:「故丞相安平侯敞等居位守職,與大將軍光、車騎將軍安世建議定策,以安宗廟,功賞未加而薨。其益封敞嗣子忠及丞相陽平侯義、度遼將軍平陵侯明友、前將軍龍雒侯增、太僕建平侯延年、太常蒲侯昌、諫大夫宜春侯譚、當塗侯平、杜侯屠耆堂、長信少府關內侯勝邑戶各有差。封御史大夫廣明為昌水侯,後將軍充國為營平侯,大司農延年為陽城侯,少府樂成為爰氏侯,光祿大夫遷為平丘侯。賜右扶風德、典屬國武、廷尉光、宗正德、大鴻臚賢、詹事畸、光祿大夫吉、京輔都尉廣漢爵皆關內侯。德、武食邑。」

夏四月庚午,地震。詔內郡國舉文學高第各一人。

五月,鳳皇集膠東、千乘。赦天下。賜吏二千石、諸侯相、下至中都官、宦吏、六百石爵,各有差,自左更至五大夫。賜天下人爵各一級,孝者二級,女子百戶牛酒。租稅勿收。

六月,詔曰:「故皇太子在湖,未有號諡。歲時祠,其議諡,置園邑。」語在太子傳。

秋七月,詔立燕剌王太子建為廣陽王,立廣陵王胥少子弘為高密王。

二年春,以水衡錢為平陵,徙民起第宅。

大司農陽城侯田延年有罪,自殺。

夏五月,詔曰:「朕以眇身奉承祖宗,夙夜惟念孝武皇帝躬履仁義,選明將,討不服,匈奴遠遁,平氐、羌、昆明、南越,百蠻鄉風,款塞來享;建太學,修郊祀,定正朔,協音律;封泰山,塞宣房,符瑞應,寶鼎出,白麟獲。功德茂盛,不能盡宣,而廟樂未稱,其議奏。」有司奏請宜加尊號。六月庚午,尊孝武廟為世宗廟,奏盛德、文始、五行之舞,天子世世獻。武帝巡狩所幸之郡國,皆立廟。賜民爵一級,女子百戶牛酒。

匈奴數侵邊,又西伐烏孫。烏孫昆彌及公主因國使者上書,言昆彌願發國精兵擊匈奴,唯天子哀憐,出兵以救公主。秋,大發興調關東輕車銳卒,選郡國吏三百石伉健習騎射者,皆從軍。御史大夫田廣明為祁連將軍,後將軍趙充國為蒲類將軍,雲中太守田順為虎牙將軍,及度遼將軍范明友、前將軍韓增,凡五將軍,兵十五萬騎,校尉常惠持節護烏孫兵,咸擊匈奴。

三年春正月癸亥,皇后許氏崩。戊辰,五將軍師發長安。夏五月,軍罷。祁連將軍廣明、虎牙將軍順有罪,下有司,皆自殺。校尉常惠將烏孫兵入匈奴右地,大克獲,封列侯。

大旱。郡國傷旱甚者,民毋出租賦。三輔民就賤者,且毋收事,盡四年。

六月己丑,丞相義薨。

四年春正月,詔曰:「蓋聞農者興德之本也,今歲不登,已遣使者振貸困乏。其令太官損膳省宰,樂府減樂人,使歸就農業。丞相以下至都官令丞上書入穀,輸長安倉,助貸貧民。民以車船載穀入關者,得毋用傳。」

三月乙卯,立皇后霍氏。賜丞相以下至郎吏從官金錢帛各有差。赦天下。

夏四月壬寅,郡國四十九地震,或山崩水出。詔曰:「蓋災異者,天地之戒也。朕承洪業,奉宗廟,託于士民之上,未能和群生。乃者地震北海、琅邪,壞祖宗廟,朕甚懼焉。丞相、御史其與列侯、中二千石博問經學之士,有以應變,輔朕之不逮,毋有所諱。令三輔、太常、內郡國舉賢良方正各一人。律令有可蠲除以安百姓,條奏。被地震壞敗甚者,勿收租賦。」大赦天下。上以宗廟墮,素服,避正殿五日。

五月,鳳皇集北海安丘、淳于。

秋,廣川王吉有罪,廢遷上庸,自殺。

地節元年春正月,有星孛于西方。

三月,假郡國貧民田。

夏六月,詔曰:「蓋聞堯親九族,以和萬國。朕蒙遺德,奉承聖業,惟念宗室屬未盡而以罪絕,若有賢材,改行勸善,其復屬,使得自新。」

冬十一月,楚王延壽謀反,自殺。

十二月癸亥晦,日有蝕之。

二年春三月庚午,大司馬大將軍光薨。詔曰:「大司馬大將軍博陸侯宿衛孝武皇帝三十餘年,輔孝昭皇帝十有餘年,遭大難,躬秉義,率三公、諸侯、九卿、大夫定萬世策,以安宗廟。天下蒸庶,咸以康寧,功德茂盛,朕甚嘉之。復其後世,疇其爵邑,世世毋有所與。功如蕭相國。」

夏四月,鳳皇集魯郡,群鳥從之。大赦天下。

五月,光祿大夫平丘侯王遷有罪,下獄死。

上始親政事,又思報大將軍功德,乃復使樂平侯山領尚書事,而令群臣得奏封事,以知下情。五日一聽事,以下各奉職奏事,以傅奏其言,考試功能。侍中尚書功勞當遷及有異善,厚加賞賜,至于子孫,終不改易。樞機周密,品式備具,上下相安,莫有苟且之意也。

三年春三月,詔曰:「蓋聞有功不賞,有罪不誅,雖唐虞猶不能以化天下。今膠東相成勞來不怠,流民自占八萬餘口,治有異等。其秩成中二千石,賜爵關內侯。」

又曰:「鰥寡孤獨高年貧困之民,朕所憐也。前下詔假公田,貨種、食。其加賜鰥寡孤獨高年帛。二千石嚴教吏謹視遇,毋令失職。」

令內郡國舉賢良方正可親民者。

夏四月戊申,立皇太子,大赦天下。賜御史大夫爵關內侯,中二千石爵右庶長,天下當為父後者爵一級。賜廣陵王黃金千斤,諸侯王十五人黃金各百斤,列侯在國者八十七人黃金各二十斤。

冬十月,詔曰:「乃者九月壬申地震,朕甚懼焉。有能箴朕過失,及賢良方正直言極諫之士以匡朕之不逮,毋諱有司。朕既不德,不能附遠,是以邊境屯戍未息。今復飭兵重屯,久勞百姓,非所以綏天下也。其罷車騎將軍、右將軍屯兵。」又詔曰:「池烃未御幸者,假與貧民。郡國宮館,勿復修治。流民還歸者,假公田,貸種、食,且勿算事。」

十一月,詔曰:「朕既不逮,導民不明,反側晨興,念慮萬方,不忘元元。唯恐羞先帝聖德,故並舉賢良方正以親萬姓,歷載臻茲,然而俗化闕焉。傳曰:『孝弟也者,其為仁之本與!』其令郡國舉孝弟有行義聞于鄉里者各一人。」

十二月,初置廷尉平四人,秩六百石。

省文山郡,并蜀。

四年春二月,封外祖母博平君,故酇侯蕭何曾孫建世為侯。

詔曰:「導民以孝,則天下順。今百姓或遭衰絰凶災,而吏繇事,使不得葬,傷孝子之心,朕甚憐之。自今諸有大父母,父母喪者勿繇事,使得收斂送終,盡其子道。」

夏五月,詔曰:「父子之親,夫婦之道,天性也。雖有患禍,猶蒙死而存之。誠愛結于心,仁厚之至也,豈能違之哉!自今子首匿父母,妻匿夫,孫匿大父母,皆勿坐。其父母匿子,夫匿妻,大父母匿孫,罪殊死,皆上請廷尉以聞。」

立廣川惠王孫文為廣川王。

秋七月,大司馬霍禹謀反。詔曰:「乃者,東織室令史張赦使魏郡豪李竟報冠陽侯霍雲謀為大逆,朕以大將軍故,抑而不揚,冀其自新。今大司馬博陸侯禹與母宣成侯夫人顯及從昆弟冠陽侯雲、樂平侯山、諸姊妹婿度遼將軍范明友、長信少府鄧廣漢、中郎將任勝、騎都尉趙平、長安男子馮殷等謀為大逆。顯前又使女侍醫淳于衍進藥殺共哀后,謀毒太子,欲危宗廟。逆亂不道,咸服其辜。諸為霍氏所詿誤未發覺在吏者,皆赦除之。」八月己酉,皇后霍氏廢。

九月,詔曰:「朕惟百姓失職不贍,遣使者循行郡國問民所疾苦。吏或營私煩擾,不顧厥咎,朕甚閔之。今年郡國頗被水災,已振貸。鹽,民之食,而賈咸貴,眾庶重困。其減天下鹽賈。」

又曰:「令甲,死者不可生,刑者不可息。此先帝之所重,而吏未稱。今繫者或以掠辜若飢寒瘐死獄中,何用心逆人道也!朕甚痛之。其令郡國歲上繫囚以掠笞若瘐死者所坐名、縣、爵、里,丞相御史課殿最以聞。」

十二月,清河王年有罪,廢遷房陵。

元康元年春,以杜東原上為初陵,更名杜縣為杜陵。徙丞相、將軍、列侯、吏二千石、訾百萬者杜陵。

三月,詔曰:「乃者鳳皇集泰山、陳留,甘露降未央宮。朕未能章先帝休烈,協寧百姓,承天順地,調序四時,獲蒙嘉瑞,賜茲祉福,夙夜兢兢,靡有驕色,內省匪解,永惟罔極。書不云乎?『鳳皇來儀,庶不允諧。』其赦天下徒,賜勤事吏中二千石以下至六百石爵,自中郎吏至五大夫,佐史以上二級,民一級,女子百戶牛酒。加賜鰥寡孤獨、三老、孝弟力田帛。所振貸勿收。」

夏五月,立皇考廟。益奉明園戶為奉明縣。

復高皇帝功臣絳侯周勃等百三十六人家子孫,令奉祭祀,世世勿絕。其毋嗣者,復其次。

秋八月,詔曰:「朕不明六藝,鬱于大道,是以陰陽風雨未時。其博舉吏民,厥身修正,通文學,明於先王之術,宣究其意者,各二人,中二千石各一人。」

冬,置建章衛尉。

二年春正月,詔曰:「《書》云『文王作罰,刑茲無赦』,今吏修身奉法,未有能稱朕意,朕甚愍焉。其赦天下,與士大夫厲精更始。」

二月乙丑,立皇后王氏。賜丞相以下至郎從官錢帛各有差。

三月,以鳳皇甘露降集,賜天下吏爵二級,民一級,女子百戶牛酒,鰥寡孤獨高年帛。

夏五月,詔曰:「獄者萬民之命,所以禁暴止邪,養育群生也。能使生者不怨,死者不恨,則可謂文吏矣。今則不然。用法或持巧心,析律貳端,深淺不平,增辭飾非,以成其罪。奏不如實,上亦亡繇知。此朕之不明,吏之不稱,四方黎民將何仰哉!二千石各察官屬,勿用此人。吏務平法。或擅興繇役,飾廚傳,稱過使客,越職踰法,以取名譽,譬猶踐薄冰以待白日,豈不殆哉!今天下頗被疾疫之災,朕甚愍之。其令郡國被災甚者,毋出今年租賦。」

又曰:「聞古天子之名,難知而易諱也。今百姓多上書觸諱以犯罪者,朕甚憐之。其更諱詢。諸觸諱在令前者,赦之。」

冬,京兆尹趙廣漢有罪,要斬。

三年春,以神爵數集泰山,賜諸侯王、丞相、將軍、列侯、二千石金,郎從官帛,各有差。賜天下吏爵二級,民一級,女子百戶牛酒,鰥寡孤獨高年帛。

三月,詔曰:「蓋聞象有罪,舜封之。骨肉之親粲而不殊。其封故昌邑王賀為海昏侯。」

又曰:「朕微眇時,御史大夫丙吉、中郎將史曾、史玄、長樂衛尉許舜、侍中光祿大夫許延壽皆與朕有舊恩。及故掖庭令張賀輔導朕躬,修文學經術,恩惠卓異,厥功茂焉。詩不云乎?『無德不報。』封賀所子弟子侍中中郎將彭祖為陽都侯,追賜賀諡曰陽都哀侯。吉、曾、玄、舜、延壽皆為列侯。故人下至郡邸獄復作嘗有阿保之功,皆受官祿田宅財物,各以恩深淺報之。」

夏六月,詔曰:「前年夏,神爵集雍。今春,五色鳥以萬數飛過屬縣,翱翔而舞,欲集未下。其令三輔毋得以春夏擿巢探卵,彈射飛鳥。具為令。」

立皇子欽為淮陽王。

遣大中大夫彊等十二人循行天下,存問鰥寡,覽觀風俗,察吏治得失,舉茂材異倫之士。

二月,河東霍徵史等謀反,誅。

三月,詔曰:「乃者,神爵五采以萬數集長樂、未央、北宮、高寢、甘泉泰畤殿中及上林苑。朕之不逮,寡于德厚,屢獲嘉祥,非朕之任。其賜天下吏爵二級,民一級,女子百戶牛酒。加賜三老、孝弟力田帛,人二匹,鰥寡孤獨各一匹。」

秋八月,賜故右扶風尹翁歸子黃金百斤,以奉其祭祀。又賜功臣適後黃金,人二十斤。

丙寅,大司馬衛將軍安世薨。

比年豐,穀石五錢。

神爵元年春正月,行幸甘泉,郊泰畤。三月,行幸河東,祠后土。詔曰:「朕承宗廟,戰戰栗栗,惟萬事統,未燭厥理。乃元康四年嘉穀玄稷降于郡國,神爵仍集,金芝九莖產于函德殿銅池中,九真獻奇獸,南郡獲白虎威鳳為寶。朕之不明,震于珍物,飭躬齋精,祈為百姓。東濟大河,天氣清靜,神魚舞河。幸萬歲宮,神爵翔集。朕之不德,懼不能任。其以五年為神爵元年。賜天下勤事吏爵二級,民一級,女子百戶牛酒,鰥寡孤獨高年帛。所振貸物勿收。行所過毋出田租。」

西羌反,發三輔、中都官徒弛刑,及應募佽飛射士、羽林孤兒,胡、越騎,三河、潁川、沛郡、淮陽、汝南材官,金城、隴西、天水、安定、北地、上郡騎士、羌騎,詣金城。夏四月,遣後將軍趙充國、彊弩將軍許延壽擊西羌。

六月,有星孛于東方。

即拜酒泉太守辛武賢為破羌將軍,與兩將軍並進。詔曰:「軍旅暴露,轉輸煩勞,其令諸侯王、列侯、蠻夷王侯君長當朝二年者,皆毋朝。」

秋,賜故大司農朱邑子黃金百斤,以奉祭祀。後將軍充國言屯田之計,語在充國傳。

二年春二月,詔曰:「乃者正月乙丑,鳳皇甘露降集京師,群鳥從以萬數。朕之不德,屢獲天福,祗事不怠,其赦天下。」

夏五月,羌虜降服,斬其首惡大豪楊玉、酋非首。置金城屬國以處降羌。

秋,匈奴日逐王先賢撣將人眾萬餘來降。使都護西域騎都尉鄭吉迎日逐,破車師,皆封列侯。

九月,司隸校尉蓋寬饒有罪,下有司,自殺。

匈奴單于遣名王奉獻,賀正月,始和親。

三年春,起樂游苑。

三月丙午,丞相相薨。

秋八月,詔曰:「吏不廉平則治道衰。今小吏皆勤事,而奉祿薄,欲其毋侵漁百姓,難矣。其益吏百石以下奉十五。」

四年春二月,詔曰:「乃者鳳皇甘露降集京師,嘉瑞並見。修興泰一、五帝、后土之祠,祈為百姓蒙祉福。鸞鳳萬舉,蜚覽翱翔,集止于旁。齋戒之暮,神光顯著。薦鬯之夕,神光交錯。或降于天,或登于地,或從四方來集于壇。上帝嘉嚮,海內承福。其赦天下,賜民爵一級,女子百戶牛酒,鰥寡孤獨高年帛。」

夏四月,潁川太守黃霸以治行尤異秩中二千石,賜爵關內侯,黃金百斤。及潁川吏民有行義者爵,人二級,力田一級,貞婦順女帛。

令內郡國舉賢良可親民者各一人。

五月,匈奴單于遣弟呼留若王勝之來朝。

冬十月,鳳皇十一集杜陵。

十一月,河南太守嚴延年有罪,棄巿。

十二月,鳳皇集上林。

五鳳元年春正月,行幸甘泉,郊泰畤。

皇太子冠。皇太后賜丞相、將軍、列侯、中二千石帛,人百匹,大夫人八十匹。又賜列侯嗣子爵五大夫,男子為父後者爵一級。

夏,赦徒作杜陵者。

冬十二月乙酉朔,日有蝕之。

左馮翊韓延壽有罪,棄巿。

二年春三月,行幸雍,祠五畤。

夏四月己丑,大司馬車騎將軍增薨。

秋八月,詔曰:「夫婚姻之禮,人倫之大者也;酒食之會,所以行禮樂也。今郡國二千石或擅為苛禁,禁民嫁娶不得具酒食相賀召。由是廢鄉黨之禮,令民亡所樂,非所以導民也。詩不云乎?『民之失德,乾餱以愆。』勿行苛政。」

冬十一月,匈奴呼鸱累單于帥眾來降,封為列侯。

十二月,平通侯陽惲坐前為光祿勳有罪,免為庶人。不悔過,怨望,大逆不道,要斬。

三年春正月癸卯,丞相吉薨。

三月,行幸河東,祠后土。詔曰:「往者匈奴數為邊寇,百姓被其害。朕承至尊,未能綏定匈奴。虛閭權渠單于請求和親,病死。右賢王屠耆堂代立。骨肉大臣立虛閭權渠單于子為呼韓邪單于,擊殺屠耆堂。諸王並自立,分為五單于,更相攻擊,死者以萬數,畜產大耗什八九,人民飢餓,相燔燒以求食,因大乖亂。單于閼氏子孫昆弟及呼鸱累單于、名王、右伊秩訾、且渠、當戶以下將眾五萬餘人來降歸義。單于稱臣,使弟奉珍朝賀正月,北邊晏然,靡有兵革之事。朕飭躬齊戒,郊上帝,祠后土,神光並見,或興于谷,燭燿齊宮,十有餘刻。甘露降,神爵集。已詔有司告祠上帝、宗廟。三月辛丑,鸞鳳又集長樂宮東闕中樹上,飛下止地,文章五色,留十餘刻,吏民並觀。朕之不敏,懼不能任,婁蒙嘉瑞,獲茲祉福。書不云乎?『雖休勿休,祗事不怠。』公卿大夫其勗焉。減天下口錢。赦殊死以下。賜民爵一級,女子百戶牛酒。大酺五日。加賜鰥寡孤獨高年帛。」

置西河、北地屬國以處匈奴降者。

四年春正月,廣陵王胥有罪,自殺。

匈奴單于稱臣,遣弟谷蠡王入侍。以邊塞亡寇,減戍卒什二。

大司農中丞耿壽昌奏設常平倉,以給北邊,省轉漕。賜爵關內侯。

夏四月辛丑晦,日有蝕之。詔曰:「皇天見異,以戒朕躬,是朕之不逮,吏之不稱也。以前使使者問民所疾苦,復遣丞相、御史掾二十四人循行天下,舉冤獄,察擅為苛禁深刻不改者。」

甘露元年春正月,行幸甘泉,郊泰畤。

匈奴呼韓邪單于遣子右賢王銖婁渠堂入侍。

二月丁巳,大司馬車騎將軍延壽薨。

夏四月,黃龍見新豐。

丙申,太上皇廟火。甲辰,孝文廟火。上素服五日。

冬,匈奴單于遣弟左賢王來朝賀。

二年春正月,立皇子囂為定陶王。

詔曰:「乃者鳳皇甘露降集,黃龍登興,醴泉滂流,枯槁榮茂,神光並見,咸受禎祥。其赦天下。減民算三十。賜諸侯王、丞相、將軍、列侯、中二千石金錢各有差。賜民爵一級,女子百戶牛酒,鰥寡孤獨高年帛。」

夏四月,遣護軍都尉祿將兵擊珠崖。

秋九月,立皇子宇為東平王。

冬十二月,行幸萯陽宮屬玉觀。

匈奴呼韓邪單于款五原塞,願奉國珍朝三年正月。詔有司議。咸曰:「聖王之制,施德行禮,先京師而後諸夏,先諸夏而後夷狄。《詩》云:『率禮不越,遂視既發。相土烈烈,海外有涞。』陛下聖德,充塞天地,光被四表。匈奴單于鄉風慕義,舉國同心,奉珍朝賀,自古未之有也。單于非正朔所加,王者所客也,禮儀宜如諸侯王,稱臣昧死再拜,位次諸侯王下。」詔曰:「蓋聞五帝三王,禮所不施,不及以政。今匈奴單于稱北藩臣,朝正月,朕之不逮,德不能弘覆。其以客禮待之,位在諸侯王上。」

三年春正月,行幸甘泉,郊泰畤。

匈奴呼韓邪單于稽侯蛳來朝,贊謁稱藩臣而不名。賜以璽綬、冠帶、衣裳、安車、駟馬、黃金、錦繡、繒絮。使有司道單于先行就邸長安,宿長平。上自甘泉宿池陽宮。上登長平阪,詔單于毋謁。其左右當戶之群皆列觀,蠻夷君長王侯迎者數萬人,夾道陳。上登渭橋,咸稱萬歲。單于就邸。置酒建章宮,饗賜單于,觀以珍寶。二月,單于罷歸。之長樂衛尉高昌侯忠、車騎都尉昌、騎都尉虎將萬六千騎送單于。單于居幕南,保光祿城。詔北邊振穀食。郅支單于遠遁,匈奴遂定。

詔曰:「乃者鳳皇集新蔡,群鳥四面行列,皆鄉鳳皇立,以萬數。其賜汝南太守帛百匹,新蔡長吏、三老、孝弟力田、鰥寡孤獨各有差。賜民爵二級。毋出今年租。」

三月己丑,丞相霸薨。

詔諸儒講五經同異,太子太傅蕭望之等平奏其議,上親稱制臨決焉。乃立梁丘易、大小夏侯尚書、穀梁春秋博士。

冬,烏孫公主來歸。

四年夏,廣川王海陽有罪,廢遷房陵。

冬十月丁卯,未央宮宣室閣火。

黃龍元年春正月,行幸甘泉,郊泰畤。

匈奴呼韓邪單于來朝,禮賜如初。二月,單于歸國。

詔曰:「蓋聞上古之治,君臣同心,舉措曲直,各得其所。是以上下和洽,海內康平,其德弗可及已。朕既不明,數申詔公卿大夫務行寬大,順民所疾苦,將欲配三王之隆,明先帝之德也。今吏或以不禁姦邪為寬大,縱釋有罪為不苛,或以酷惡為賢,皆失其中。奉詔宣化如此,豈不繆哉!方今天下少事,繇役省減,兵革不動,而民多貧,盜賊不止,其咎安在?上計簿,具文而已,務為欺謾,以避其課。三公不以為意,朕將何任?諸請詔省卒徒自給者皆止。御史察計簿,疑非實者,按之,使真偽毋相亂。」

三月,有星孛于王良、閣道,入紫宮。

夏四月,詔曰:「舉廉吏,誠欲得其真也。吏六百石位大夫,有罪先請,秩祿上通,足以效其賢材,自今以來毋得舉。」

冬十二月甲戌,帝崩于未央宮。癸巳,尊皇太后曰太皇太后。

贊曰:孝宣之治,信賞必罰,綜核名實,政事文學法理之士咸精其能,至于技巧工匠器械,自元、成間鮮能及之,亦足以知吏稱其職,民安其業也。遭值匈奴乖亂,推亡固存,信威北夷,單于慕義,稽首稱藩。功光祖宗,業垂後嗣,可謂中興,侔德殷宗、周宣矣。


\end{pinyinscope}