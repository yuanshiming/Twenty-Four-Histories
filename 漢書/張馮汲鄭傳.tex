\article{張馮汲鄭傳}

\begin{pinyinscope}
張釋之字季,南陽堵陽人也。與兄仲同居,以貲為騎郎,事文帝,十年不得調,亡所知名。釋之曰:「久宦減仲之產,不遂。」欲免歸。中郎將爰盎知其賢,惜其去,乃請徙釋之補謁者。釋之既朝畢,因前言便宜事。文帝曰:「卑之,毋甚高論,令今可行也。」於是釋之言秦漢之間事,秦所以失,漢所以興者。文帝稱善,拜釋之為謁者僕射。

從行,上登虎圈,問上林尉禽獸簿,十餘問,尉左右視,盡不能對。虎圈嗇夫從旁代尉對上所問禽獸簿甚悉,欲以觀其能口對嚮應亡窮者。文帝曰:「吏不當如此邪?尉亡賴!」詔釋之拜嗇夫為上林令。釋之前曰:「陛下以絳侯周勃何如人也?」上曰:「長者。」又復問:「東陽侯張相如何如人也?」上復曰:「長者。」釋之曰:「夫絳侯、東陽侯稱為長者,此兩人言事曾不能出口,豈效此嗇夫喋喋利口捷給哉!且秦以任刀筆之吏,爭以亟疾苛察相高,其敝徒文具,亡惻隱之實。以故不聞其過,陵夷至於二世,天下土崩。今陛下以嗇夫口辯而超遷之,臣恐天下隨風靡,爭口辯,亡其實。且下之化上,疾於景嚮,舉錯不可不察也。」文帝曰:「善。」乃止不拜嗇夫。

就車,召釋之驂乘,徐行,行問釋之秦之敝。具以質言。至宮,上拜釋之為公車令。

頃之,太子與梁王共車入朝,不下司馬門,於是釋之追止太子、梁王毌入殿門。遂劾不下公門不敬,奏之。薄太后聞之,文帝免冠謝曰:「教兒子不謹。」薄太后使使承詔赦太子、梁王,然後得入。文帝繇是奇釋之,拜為中大夫。

頃之,至中郎將。從行至霸陵,上居外臨廁。時慎夫人從,上指視慎夫人新豐道,曰:「此走邯鄲道也。」使慎夫人鼓瑟,上自倚瑟而歌,意悽愴悲懷,顧謂群臣曰:「嗟乎!以北山石為槨,用紵絮斮陳漆其間,豈可動哉!」左右皆曰:「善。」釋之前曰:「使其中有可欲,雖錮南山猶有隙;使其中亡可欲,雖亡石槨,又何戚焉?」文帝稱善。其後,拜釋之為廷尉。

頃之,上行出中渭橋,有一人從橋下走,乘輿馬驚。於是使騎捕之,屬廷尉。釋之治問。曰:「縣人來,聞蹕,匿橋下。久,以為行過,既出,見車騎,即走耳。」釋之奏當:「此人犯蹕,當罰金。」上怒曰:「此人親驚吾馬,馬賴和柔,令它馬,固不敗傷我乎?而廷尉乃當之罰金!」釋之曰:「法者天子所與天子公共也。今法如是,更重之,是法不信於民也。且方其時,上使使誅之則已。今已下廷尉,廷尉,天下之平也,壹傾,天下用法皆為之輕重,民安所錯其手足?唯陛下察之。」上良久曰:「廷尉當是也。」

其後人有盜高廟座前玉環,得,文帝怒,下廷尉治。案盜宗廟服御物者為奏,當棄市。上大怒曰:「人亡道,乃盜先帝器!吾屬廷尉者,欲致之族,而君以法奏之,非吾所以共承宗廟意也。」釋之免冠頓首謝曰:「法如是足也。且罪等,然以逆順為基。今盜宗廟器而族之,有如萬分一,假令愚民取長陵一抔土,陛下且何以加其法虖?」文帝與太后言之,乃許廷尉當。是時,中尉條侯周亞夫與梁相山都侯王恬咸見釋之持議平,乃結為親友。張廷尉繇此天下稱之。

文帝崩,景帝立,釋之恐,稱疾。欲免去,懼大誅至;欲見,則未知何如。用王生計,卒見謝,景帝不過也。

王生者,善為黃老言,處士。嘗召居廷中,公卿盡會立,王生老人,曰「吾狲解」,顧謂釋之:「為我結狲!」釋之跪而結之。既已,人或讓王生:「獨柰何廷辱張廷尉如此?」王生曰:「吾老且賤,自度終亡益於張廷尉。廷尉方天下名臣,吾故聊使結狲,欲以重之。」諸公聞之,賢王生而重釋之。

釋之事景帝歲餘,為淮南相,猶尚以前過也。年老病卒。其子摯,字長公,官至大夫,免。以不能取容當世,故終身不仕。

馮唐,祖父趙人也。父徙代。漢興徙安陵。唐以孝著,為郎中署長,事文帝。帝輦過,問唐曰:「父老何自為郎?家安在?」具以實言。文帝曰:「吾居代時,吾尚食監高祛數為我言趙將李齊之賢,戰於鉅鹿下。吾每飲食,意未嘗不在鉅鹿也。父老知之乎?」唐對曰:「齊尚不如廉頗、李牧之為將也。」上曰:「何已?」唐曰:「臣大父在趙時,為官帥將,善李牧。臣父故為代相,善李齊,知其為人也。」上既聞廉頗、李牧為人,良說,乃拊髀曰:「嗟乎!吾獨不得廉頗、李牧為將,豈憂匈奴哉!」唐曰:「主臣!陛下雖有廉頗、李牧,不能用也。」上怒,起入禁中。良久,召唐讓曰:「公眾辱我,獨亡間處虖?」唐謝曰:「鄙人不知忌諱。」

當是時,匈奴新大入朝那,殺北地都尉卬。上以胡寇為意,乃卒復問唐曰:「公何以言吾不能用頗、牧也?」唐對曰:「臣聞上古王者遣將也,跪而推轂,曰:『闑以內寡人制之,闑以外將軍制之;軍功爵賞,皆決於外,歸而奏之。』此非空言也。臣大父言李牧之為趙將居邊,軍市之租皆自用饗士,賞賜決於外,不從中覆也。委任而責成功,故李牧乃得盡其知能,選車千三百乘,彀騎萬三千匹,百金之士十萬,是以北逐單于,破東胡,滅澹林,西抑彊秦,南支韓、魏。當是時,趙幾伯。後會趙王遷立,其母倡也,用郭開讒,而誅李牧,令顏聚代之。是以為秦所滅。今臣竊聞魏尚為雲中守,軍市租盡以給士卒,出私養錢,五日壹殺牛,以饗賓客軍吏舍人,是以匈奴遠避,不近雲中之塞。虜嘗一入,尚帥車騎擊之,所殺甚眾。夫士卒盡家人子,起田中從軍,安知尺籍伍符?終日力戰,斬首捕虜,上功莫府,一言不相應,文吏以法繩之。其賞不行,吏奉法必用。愚以為陛下法太明,賞太輕,罰太重。且雲中守尚坐上功首虜差六級,陛下下之吏,削其爵,罰作之。繇此言之,陛下雖得李牧,不能用也。臣誠愚,觸忌諱,死罪!」文帝說。是日,令唐持節赦魏尚,復以為雲中守,而拜唐為車騎都尉,主中尉及郡國車士。

十年,景帝立,以唐為楚相。武帝即位,求賢良,舉唐。唐時年九十餘,不能為官,乃以子遂為郎。遂字王孫,亦奇士。魏尚,槐里人也。

汲黯字長孺,濮陽人也。其先有寵於古之衛君也。至黯十世,世為卿大夫。以父任,孝景時為太子洗馬,以嚴見憚。

武帝即位,黯為謁者。東粵相攻,上使黯往視之。至吳而還,報曰:「粵人相攻,固其俗,不足以辱天子使者。」河內失火,燒千餘家,上使黯往視之。還報曰:「家人失火,屋比延燒,不足憂。臣過河內,河內貧人傷水旱萬餘家,或父子相食,臣謹以便宜,持節發河內倉粟以振貧民。請歸節,伏矯制罪。」上賢而釋之,遷為滎陽令。黯恥為令,稱疾歸田里。上聞,乃召為中大夫。以數切諫,不得久留內,遷為東海太守。

黯學黃老言,治官民,好清靜,擇丞史任之,責大指而已,不細苛。黯多病,臥閤內不出。歲餘,東海大治,稱之。上聞,召為主爵都尉,列於九卿。治務在無為而已,引大體,不拘文法。

為人性倨,少禮,面折,不能容人之過。合己者善待之,不合者弗能忍見,士亦以此不附焉。然好游俠,任氣節,行修潔。其諫,犯主之顏色。常慕傅伯、爰盎之為人。善灌夫、鄭當時及宗正劉棄疾。亦以數直諫,不得久居位。

是時,太后弟武安侯田蚡為丞相,中二千石拜謁,蚡弗為禮。黯見蚡,未嘗拜,揖之。上方招文學儒者,上曰吾欲云云,黯對曰:「陛下內多欲而外施仁義,柰何欲效唐虞之治乎!」上怒,變色而罷朝。公卿皆為黯懼。上退,謂人曰:「甚矣,汲黯之戇也!」群臣或數黯,黯曰:「天子置公卿輔弼之臣,寧令從諛承意,陷主於不誼虖?且已在其位,縱愛身,柰辱朝廷何!」

黯多病,病且滿三月,上常賜告者數,終不瘉。最後,嚴助為請告。上曰:「汲黯何如人也?」曰:「使黯任職居官,亡以瘉人,然至其輔少主守成,雖自謂賁育弗能奪也。」上曰:「然。古有社稷之臣,至如汲黯,近之矣。」

大將軍青侍中,上踞廁視之。丞相弘宴見,上或時不冠。至如見黯,不冠不見也。上嘗坐武帳,黯前奏事,上不冠,望見黯,避帷中,使人可其奏。其見敬禮如此。

張湯以更定律令為廷尉,黯質責湯於上前,曰:「

公為正卿,上不能褒先帝之功業,下不能化天下之邪心,安國富民,使囹圄空虛,何空取高皇帝約束紛更之為?而公以此無種矣!」黯時與湯論議,湯辯常在文深小苛,黯憤發,罵曰:「天下謂刀筆吏不可謂公卿,果然。必湯也,令天下重足而立,仄目而視矣!」

是時,漢方征匈奴,招懷四夷,黯務少事,間常言與胡和親,毋起兵。上方鄉儒術,尊公孫弘,及事益多,吏民巧。上分別文法,湯等數奏決讞以幸。而黯常毀儒,面觸弘等徒懷詐飾智以阿人主取容,而刀筆之吏專深文巧詆,陷人於罔,以自為功。上愈益貴弘、湯,弘、湯心疾黯,雖上亦不說也,欲誅之以事。弘為丞相,乃言上曰:「右內史界部中多貴人宗室,難治,非素重臣弗能任,請徙黯為右內史。」數歲,官事不廢。

大將軍青既益尊,姊為皇后,然黯與亢禮。或說黯曰:「自天子欲令群臣下大將軍,大將軍尊貴,誠重,君不可以不拜。」黯曰:「夫以大將軍有揖客,反不重耶?」大將軍聞,愈賢黯,數請問以朝廷所疑,遇黯加於平日。

淮南王謀反,憚黯,曰:「黯好直諫,守節死義;至說公孫弘等,如發蒙耳。」

上既數征匈奴有功,黯言益不用。

始黯列九卿矣,而公孫弘、張湯為小吏。及弘、湯稍貴,與黯同位,黯又非毀弘、湯。已而弘至丞相封侯,湯御史大夫,黯時丞史皆與同列,或尊用過之。黯褊心,不能無少望,見上,言曰:「

陛下用群臣如積薪耳,後來者居上。」黯罷,上曰:「人果不可以無學,觀汲黯之言,日益甚矣。」

居無何,匈奴渾邪王帥眾來降,漢發車二萬乘。縣官亡錢,從民貰馬。民或匿馬,馬不具。上怒,欲斬長安令。黯曰:「長安令亡罪,獨斬臣黯,民乃肯出馬。且匈奴畔其主而降漢,徐以縣次傳之,何至令天下騷動,罷中國,甘心夷狄之人乎!」上默然。後渾邪王至,賈人與市者,坐當死五百餘人。黯入,請間,見高門,曰:「夫匈奴攻當路塞,絕和親,中國舉兵誅之,死傷不可勝計,而費以鉅萬百數。臣愚以為陛下得胡人,皆以為奴婢,賜從軍死者家;鹵獲,因與之,以謝天下,塞百姓之心。今縱不能,渾邪帥數萬之眾來,虛府庫賞賜,發良民侍養,若奉驕子。愚民安知市買長安中而文吏繩以為闌出財物如邊關乎?陛下縱不能得匈奴之贏以謝天下,又以微文殺無知者五百餘人,臣竊為陛下弗取也。」上弗許,曰:「吾久不聞汲黯之言,今又復妄發矣。」後數月,黯坐小法,會赦,免官。於是黯隱於田園者數年。

會更立五銖錢,民多盜鑄錢者,楚地尤甚。上以為淮陽,楚地之郊也,召黯拜為淮陽太守。黯伏謝不受印綬,詔數強予,然後奉詔。召上殿,黯泣曰:「臣自以為填溝壑,不復見陛下,不意陛下復收之。臣常有狗馬之心,今病,力不能任郡事。臣願為中郎,出入禁闥,補過拾遺,臣之願也。」上曰:「君薄淮陽邪?吾今召君矣。顧淮陽吏民不相得,吾徒得君重,臥而治之。」黯既辭,過大行李息,曰:「黯棄逐居郡,不得與朝廷議矣。然御史大夫湯智足以距諫,詐足以飾非,非肯正為天下言,專阿主意。主意所不欲,因而毀之;主意所欲,因而譽之。好興事,舞文法,內懷詐以御主心,外挾賊吏以為重。公列九卿不早言之何?公與之俱受其戮矣!」息畏湯,終不敢言。黯居郡如其故治,淮陽政清。後張湯敗,上聞黯與息言,抵息罪。令黯以諸侯相秩居淮陽。居淮陽十歲而卒。

卒後,上以黯故,官其弟仁至九卿,子偃至諸侯相。黯姊子司馬安亦少與黯為太子洗馬。安文深巧善宦,四至九卿,以河南太守卒。昆弟以安故,同時至二千石十人。濮陽段宏始事蓋侯信,信任宏,官亦再至九卿。然衛人仕者皆嚴憚汲黯,出其下。

鄭當時字莊,陳人也。其先鄭君嘗事項籍,籍死而屬漢。高祖令諸故項籍臣名籍,鄭君獨不奉詔。詔盡拜名籍者為大夫,而逐鄭君。鄭君死孝文時。

當時以任俠自喜,脫張羽於阨,聲聞梁楚間。孝景時,為太子舍人。每五日洗沐,常置驛馬長安諸郊,請謝賓客,夜以繼日,至明旦,常恐不遍。當時好黃老言,其慕長者,如恐不稱。自見年少官薄,然其知友皆大父行,天下有名之士也。

武帝即位,當時稍遷為魯中尉,濟南太守,江都相,至九卿為右內史。以武安魏其時議,貶秩為詹事,遷為大司農。

當時為大吏,戒門下:「客至,亡貴賤亡留門下者。」執賓主之禮,以其貴下人。性廉,又不治產,卬奉賜給諸公。然其餽遺人,不過具器食。每朝,候上間說,未嘗不言天下長者。其推轂士及官屬丞史,誠有味其言也。常引以為賢於己。未嘗名吏,與官屬言,若恐傷之。聞人之善言,進之上,唯恐後。山東諸公以此翕然稱鄭莊。

使視決河,自請治行五日。上曰:「吾聞鄭莊行,千里不齎糧,治行者何也?」然當時在朝,常趨和承意,不敢甚斥臧否。漢征匈奴,招四夷,天下費多,財用益屈。當時為大司農,任人賓客僦,入多逋負。司馬安淮陽太守,發其事,當時以此陷罪,贖為庶人。頃之,守長史。遷汝南太守,數歲,以官卒。昆弟以當時故,至二千石者六七人。

當時始與汲黯列為九卿,內行修。兩人中廢,賓客益落。當時死,家亡餘財。

先是下邽翟公為廷尉,賓客亦填門,及廢,門外可設爵羅。後復為廷尉,客欲往,翟公大署其門曰:「一死一生,乃知交情;一貧一富,乃知交態;一貴一賤,交情乃見。」

贊曰:張釋之之守法,馮唐之論將,汲黯之正直,鄭當時之推士,不如是,亦何以成名哉!揚子以為孝文親詘帝尊以信亞夫之軍,曷為不能用頗、牧?彼將有激云爾。


\end{pinyinscope}