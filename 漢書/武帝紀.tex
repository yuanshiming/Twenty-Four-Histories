\article{武帝紀}

\begin{pinyinscope}
孝武皇帝,景帝中子也,母曰王美人。年四歲立為膠東王。七歲為皇太子,母為皇后。十六歲,後三年正月,景帝崩。甲子,太子即皇帝位,尊皇太后竇氏曰太皇太后,皇后曰皇太后。三月,封皇太后同母弟田蚡、勝皆為列侯。

建元元年冬十月,詔丞相、御史、列侯、中二千石、二千石、諸侯相舉賢良方正直言極諫之士。丞相綰奏:「所舉賢良,或治申、商、韓非、蘇秦、張儀之言,亂國政,請皆罷。」奏可。

春二月,赦天下,賜民爵一級。年八十復二算,九十復甲卒。行三銖錢。

夏四月己巳,詔曰:「古之立教,鄉里以齒,朝廷以爵,扶世導民,莫善於德。然則於鄉里先耆艾,奉高年,古之道也。今天下孝子順孫願自竭盡以承其親,外迫公事,內乏資財,是以孝心闕焉。朕甚哀之。民年九十以上,已有受鬻法,為復子若孫,令得身帥妻妾遂其供養之事。」

五月,詔曰:「河海潤千里,其令祠官修山川之祠,為歲事,曲加禮。」

赦吳楚七國帑輸在官者。

秋七月,詔曰:「衛士轉置送迎二萬人,其省萬人。罷苑馬,以賜貧民。」

議立明堂。遣使者安車蒲輪,束帛加璧,徵魯申公。

二年冬十月,御史大夫趙綰坐請毋奏事太皇太后,及郎中令王臧皆下獄,自殺。丞相嬰、太尉蚡免。

春二月丙戌朔,日有蝕之。夏四月戊申,有如日夜出。

初置茂陵邑。

三年春,河水溢于平原,大飢,人相食。

賜徙茂陵者戶錢二十萬,田二頃。初作便門橋。

秋七月,有星孛于西北。

濟川王明坐殺太傅、中傅廢遷防陵。

閩越圍東甌,東甌告急。遣中大夫嚴助持節發會稽兵,浮海救之。未至,閩越走,兵還。

九月丙子晦,日有蝕之。

四年夏,有風赤如血。六月,旱。秋九月,有星孛于東北。

五年春,罷三銖錢,行半兩錢。

置五經博士。

夏四月,平原君薨。

五月,大蝗。

秋八月,廣川王越、清河王乘皆薨。

六年春二月乙未,遼東高廟災。夏四月壬子,高園便殿火。上素服五日。

五月丁亥,太皇太后崩。

秋八月,有星孛于東方,長竟天。

閩越王郢攻南越。遣大行王恢將兵出豫章,大司農韓安國出會稽,擊之。未至,越人殺郢降,兵還。

元光元年冬十一月,初令郡國舉孝廉各一人。

衛尉李廣為驍騎將軍屯雲中,中尉程不識為車騎將軍屯鴈門,六月罷。

夏四月,赦天下,賜民長子爵一級。復七國宗室前絕屬者。

五月,詔賢良曰:「朕聞昔在唐虞,畫象而民不犯,日月所燭,莫不率俾。周之成康,刑錯不用,德及鳥獸,教通四海。海外肅凫,北發渠搜,氐羌徠服。星辰不孛,日月不蝕,山陵不崩,川谷不塞;麟鳳在郊藪,河洛出圖書。嗚虖,何施而臻此與!今朕獲奉宗廟,夙興以求,夜寐以思,若涉淵水,未知所濟。猗與偉與!何行而可以章先帝之洪業休德,上參堯舜,下配三王!朕之不敏,不能遠德,此子大夫之所睹聞也。賢良明於古今王事之體,受策察問,咸以書對,著之於篇,朕親覽焉。」於是董仲舒、公孫弘等出焉。

秋七月癸未,日有蝕之。

二年冬十月,行幸雍,祠五畤。

春,詔問公卿曰:「朕飾子女以配單于,金幣文繡賂之甚厚,單于待命加嫚,侵盜亡已。邊境被害,朕甚閔之。今欲舉兵攻之,何如?」大行王恢建議宜擊。夏六月,御史大夫韓安國為護軍將軍,衛尉李廣為驍騎將軍,太僕公孫賀為輕車將軍,大行王恢為將屯將軍,大中大夫李息為材官將軍,將三十萬眾屯馬邑谷中,誘致單于,欲襲擊之。單于入塞,覺之,走出。六月,軍罷。將軍王恢坐首謀不進,下獄死。

秋九月,令民大酺五日。

三年春,河水徙,從頓丘東南流入勃海。

夏五月,封高祖功臣五人後為列侯。

河水決濮陽,氾郡十六。發卒十萬救決河。起龍淵宮。

四年冬,魏其侯竇嬰有罪,棄巿。

春三月乙卯,丞相蚡薨。

夏四月,隕霜殺草。五月,地震。赦天下。

五年春正月,河間王德薨。

夏,發巴蜀治南夷道,又發卒萬人治雁門阻險。

秋七月,大風拔木。

乙巳,皇后陳氏廢。捕為巫蠱者,皆梟首。

八月,螟。

徵吏民有明當時之務習先聖之術者,縣次續食,令與計偕。

六年冬,初算商車。

春,穿漕渠通渭。

匈奴入上谷,殺略吏民。遣車騎將軍衛青出上谷,騎將軍公孫敖出代,輕車將軍公孫賀出雲中,驍騎將軍李廣出雁門。青至龍城,獲首虜七百級。廣、敖失師而還。詔曰:「夷狄無義,所從來久。間者匈奴數寇邊境,故遣將撫師。古者治兵振旅,因遭虜之方入,將吏新會,上下未輯,代郡將軍敖、雁門將軍廣所任不肖,校尉又背義妄行,棄軍而北,少吏犯禁。用兵之法:不勤不教,將率之過也;教令宣明,不能盡力,士卒之罪也。將軍已下廷尉,使理正之,而又加法於士卒,二者並行,非仁聖之心。朕閔眾庶陷害,欲刷恥改行,復奉正議,厥路亡繇。其赦雁門、代郡軍士不循法者。」

夏,大旱,蝗。

六月,行幸雍。

秋,匈奴盜邊。遣將軍韓安國屯漁陽。

元朔元年冬十一月,詔曰:「公卿大夫,所使總方略,壹統類,廣教化,美風俗也。夫本仁祖義,褒德祿賢,勸善刑暴,五帝三王所繇昌也。朕夙興夜寐,嘉與宇內之士臻於斯路。故旅耆老,復孝敬,選豪俊,講文學,稽參政事,祈進民心,深詔執事,興廉舉孝,庶幾成風,紹休聖緒。夫十室之邑,必有忠信;三人並行,厥有我師。今或至闔郡而不薦一人,是化不下究,而積行之君子雍於上聞也。二千石官長紀綱人倫,將何以佐朕燭幽隱,勸元元,厲蒸庶,崇鄉黨之訓哉?且進賢受上賞,蔽賢蒙顯戮,古之道也。其與中二千石、禮官、博士議不舉者罪。」有司奏議曰:「古者,諸侯貢士,壹適謂之好德,再適謂之賢賢,三適謂之有功,乃加九錫;不貢士,壹則黜爵,再則黜地,三而黜爵地畢矣。夫附下罔上者死,附上罔下者刑,與聞國政而無益於民者斥,在上位而不能進賢者退,此所以勸善黜惡也。今詔書昭先帝聖緒,令二千石舉孝廉,所以化元元,移風易俗也。不舉孝,不奉詔,當以不敬論。不察廉,不勝任也,當免。」奏可。

十二月,江都王非薨。

春三月甲子,立皇后衛氏。詔曰:「朕聞天地不變,不成施化;陰陽不變,物不暢茂。《易》曰『通其變,使民不倦』。《詩》云『九變復貫,知言之選』。朕嘉唐虞而樂殷周,據舊以鑒新。其赦天下,與民更始。諸逋貸及辭訟在孝景後三年以前,皆勿聽治。」

秋,匈奴入遼西,殺太守;入漁陽、雁門,敗都尉,殺略三千餘人。遣將軍衛青出雁門,將軍李息出代,獲首虜數千級。

東夷薉君南閭等口二十八萬人降,為蒼海郡。

魯王餘、長沙王發皆薨。

二年冬,賜淮南王、菑川王几杖,毋朝。

春正月,詔曰:「梁王、城陽王親慈同生,願以邑分弟,其許之。諸侯王請與子弟邑者,朕將親覽,使有列位焉。」於是藩國始分,而子弟畢侯矣。

匈奴入上谷、漁陽,殺略吏民千餘人。遣將軍衛青、李息出雲中,至高闕,遂西至符離,獲首虜數千級。數河南地,置朔方、五原郡。

三月乙亥晦,日有蝕之。

夏,募民徙朔方十萬口。又徙郡國豪傑及訾三百萬以上于茂陵。

秋,燕王定國有罪,自殺。

三年春,罷蒼海郡。三月,詔曰:「夫刑罰所以防姦也,內長文所以見愛也;以百姓之未洽于教化,朕嘉與士大夫日新厥業,祗而不解。其赦天下。」

夏,匈奴入代,殺太守;入雁門,殺略千餘人。

六月庚午,皇太后崩。

秋,罷西南夷,城朔方城。令民大酺五日。

四年冬,行幸甘泉。

夏,匈奴入代、定襄、上郡,殺略數千人。

五年春,大旱。大將軍衛青將六將軍兵十餘萬人出朔方、高闕,獲首虜萬五千級。

夏六月,詔曰:「蓋聞導民以禮,風之以樂,今禮壞樂崩,朕甚閔焉。故詳延天下方聞之士,咸薦諸朝。其令禮官勸學,講議洽聞,舉遺興禮,以為天下先。太常其議予博士弟子,崇鄉黨之化,以厲賢材焉。」丞相弘請為博士置弟子員,學者益廣。

秋,匈奴入代,殺都尉。

六年春二月,大將軍衛青將六將軍兵十餘萬騎出定襄,斬首三千餘級。還,休士馬于定襄、雲中、鴈門。赦天下。

夏四月,衛青復將六將軍絕幕,大克獲。前將軍趙信軍敗,降匈奴。右將軍蘇建亡軍,獨身脫還,贖為庶人。

六月,詔曰:「朕聞五帝不相復禮,三代不同法,所繇殊路而建德一也。蓋孔子對定公以徠遠,哀公以論臣,景公以節用,非期不同,所急異務也。今中國一統而北邊未安,朕甚悼之。日者大將軍巡朔方,征匈奴,斬首虜萬八千級,諸禁錮及有過者,咸蒙厚賞,得免減罪。今大將軍仍復克獲,斬首虜萬九千級,受爵賞而欲移賣者,無所流貤。其議為令。」有司奏請置武功賞官,以寵戰士。

元狩元年冬十月,行幸雍,祠五畤。獲白麟,作白麟之歌。

十一月,淮南王安、衡山王賜謀反,誅。黨與死者數萬人。

十二月,大雨雪,民凍死。

夏四月,赦天下。

丁卯,立皇太子。賜中二千石爵右庶長,民為父後者一級。詔曰:「朕聞咎繇對禹,曰在知人,知人則哲,惟帝難之。蓋君者心也,民猶支體,支體傷則心憯怛。日者淮南、衡山修文學,流貨賂,兩國接壤,怵於邪說,而造篡弒,此朕之不德。《詩》云:『憂心慘慘,念國之為虐。』已赦天下,滌除與之更始。朕嘉孝弟力田,哀夫老眊孤寡鰥獨或匱於衣食,甚憐愍焉。其遣謁者巡行天下,存問致賜。曰『皇帝使謁者賜縣三老、孝者帛,人五匹;鄉三老、弟者、力田帛,人三匹;年九十以上及鰥寡孤獨帛,人二匹,絮三斤;八十以上米,人三石。有冤失職,使者以聞。縣鄉即賜,毋贅聚』。」

五月乙巳晦,日有蝕之。

匈奴入上谷,殺數百人。

二年冬十月,行幸雍,祠五畤。

春三月戊寅,丞相弘薨。

遣驃騎將軍霍去病出隴西,至皋蘭,斬首八千餘級。

夏,馬生余吾水中。南越獻馴象、能言鳥。

將軍去病、公孫敖出北地二千餘里,過居延,斬首虜三萬餘級。

匈奴入鴈門,殺略數百人。遣衛尉張騫、郎中令李廣皆出右北平。廣殺匈奴三千餘人,盡亡其軍四千人,獨身脫還,及公孫敖、張騫皆後期,當斬,贖為庶人。

江都王建有罪,自殺。膠東王寄薨。

秋,匈奴昆邪王殺休屠王,并將其眾合四萬餘人來降,置五屬國以處之。以其地為武威、酒泉郡。

三年春,有星孛于東方。夏五月,赦天下。立膠東康王少子慶為六安王。封故相國蕭何曾孫慶為列侯。

秋,匈奴入右北平、定襄,殺略千餘人。

遣謁者勸有水災郡種宿麥。舉吏民能假貸貧民者以名聞。

減隴西、北地、上郡戍卒半。

發謫吏穿昆明池。

四年冬,有司言關東貧民徙隴西、北地、西河、上郡、會稽凡七十二萬五千口,縣官衣食振業,用度不足,請收銀錫造白金及皮幣以足用。初算緡錢。

春,有星孛于東北。

夏,有長星出于西北。

大將軍衛青將四將軍出定襄,將軍去病出代,各將五萬騎。步兵踵軍後數十萬人。青至幕北圍單于,斬首萬九千級,至闐顏山乃還。去病與左賢王戰,斬獲首虜七萬餘級,封狼居胥山乃還。兩軍士戰死者數萬人。前將軍廣、後將軍食其皆後期。廣自殺,食其贖死。

五年春三月甲午,丞相李蔡有罪,自殺。

天下馬少,平牡馬匹二十萬。

罷半兩錢,行五銖錢。

徙天下姦猾吏民於邊。

六年冬十月,賜丞相以下至吏二千石金,千石以下至乘從者帛,蠻夷錦各有差。

雨水亡冰。

夏四月乙巳,廟立皇子閎為齊王,旦為燕王,胥為廣陵王。初作誥。

六月,詔曰:「日者有司以幣輕多姦,農傷而末眾,又禁以并之塗,故改幣以約之。稽諸往古,制宜於今。廢期有月,而山澤之民未諭。夫仁行而從善,義立則俗易,意奉憲者所以導之未明與?將百姓所安殊路,而撟虔吏因乘勢以侵蒸庶邪?何紛然其擾也!今遣博士大等六人分循行天下,存問鰥寡廢疾,無以自振業者貸與之。諭三老孝弟以為民師,舉獨行之君子,徵詣行在所。朕嘉賢者,樂知其人。廣宣厥道,士有特招,使者之任也。詳問隱處亡位,及冤失職,姦猾為害,野荒治苛者,舉奏。郡國有所以為便者,上丞相、御史以聞。」

秋九月,大司馬驃騎將軍去病薨。

元鼎元年夏五月,赦天下,大酺五日。

得鼎汾水上。

濟東王彭離有罪,廢徙上庸。

二年冬十一月,御史大夫張湯有罪,自殺。十二月,丞相青翟下獄死。

春,起柏梁臺。

三月,大雨雪。夏,大水,關東餓死者以千數。

秋九月,詔曰:「仁不異遠,義不辭難。今京師雖未為豐年,山林池澤之饒與民共之。今水潦移於江南,迫隆冬至,朕懼其飢寒不活。江南之地,火耕水耨,方下巴蜀之粟致之江陵,遣博士中等分循行,諭告所抵,無令重困。吏民有振救飢民免其厄者,具舉以聞。」

三年冬,徙函谷關於新安。以故關為弘農縣。

十一月,令民告緡者以其半與之。

正月戊子,陽陵園火。夏四月,雨雹,關東郡國十餘飢,人相食。

常山王舜薨。子绗嗣立,有罪,廢徙房陵。

四年冬十月,行幸雍,祠五畤。賜民爵一級,女子百戶牛酒。行自夏陽,東幸汾陰。十一月甲子,立后土祠于汾陰脽上。禮畢,行幸滎陽。還至洛陽,詔曰:「祭地冀州,瞻望河洛,巡省豫州,觀于周室,邈而無祀。詢問耆老,乃得孽子嘉。其封嘉為周子南君,以奉周祀。」

春二月,中山王勝薨。

夏,封方士欒大為樂通侯,位上將軍。

六月,得寶鼎后土祠旁。秋,馬生渥洼水中。作寶鼎、天馬之歌。

立常山憲王子商為泗水王。

五年冬十月,行幸雍,祠五畤。遂踰隴,登空同,西臨祖厲河而還。

十一月辛巳朔旦,冬至。立泰畤于甘泉。天子親郊見,朝日夕月。詔曰:「朕以眇身託于王侯之上,德未能綏民,民或飢寒,故巡祭后土以祈豐年。冀州脽壤乃顯文鼎,獲

祭於廟。渥洼水出馬,朕其御焉。戰戰兢兢,懼不克任,思昭天地,內惟自新。《詩》云:『四牡翼翼,以征不服。』親省邊垂,用事所極。望見泰一,修天文浈。辛卯夜,若景光十有二明。《易》曰:『先甲三日,後甲三日。』朕甚念年歲未咸登,飭躬齋戒,丁酉,拜況于郊。」

夏四月,南越王相呂嘉反,殺漢使者及其王、王太后。赦天下。

丁丑晦,日有蝕之。

秋,杀、蝦蟆鬥。

遣伏波將軍路博德出桂陽,下湟水;樓船將軍楊僕出豫章,下湞水;歸義越侯嚴為戈船將軍,出零陵,下離水;甲為下瀨將軍,下蒼梧。皆將罪人,江淮以南樓船十萬人。越馳義侯遺別將巴蜀罪人,發夜郎兵,下牂柯江,咸會番禺。

九月,列侯坐獻黃金酎祭宗廟不如法奪爵者百六人,丞相趙周下獄死。樂通侯欒大坐誣罔要斬。

西羌眾十萬人反,與匈奴通使,攻故安,圍枹罕。匈奴入五原,殺太守。

六年冬十月,發隴西、天水、安定騎士及中尉,河南、河內卒十萬人,遣將軍李息、郎中令一自為征西羌,平之。

行東,將幸緱氏,至左邑桐鄉,聞南越破,以為聞喜縣。春,至汲新中鄉,得呂嘉首,以為獲嘉縣。馳義侯遺兵未及下,上便令征西南夷,平之。遂定越地,以為南海、蒼梧、鬱林、合浦、交阯、九真、日南、珠崖、儋耳郡。定西南夷,以為武都、牂柯、越嶲、沈黎、文山郡。

秋,東越王餘善反,攻殺漢將吏。遣橫海將軍韓說、中尉王溫舒出會稽,樓船將軍楊僕出豫章,擊之。又遣浮沮將軍公孫賀出九原,匈河將軍趙破奴出令居,皆二千餘里,不見虜而還。乃分武威、酒泉地置張掖、敦煌郡,徙民以實之。

元封元年冬十月,詔曰:「南越、東甌咸伏其辜,西蠻北夷頗未輯睦,朕將巡邊垂,擇兵振旅,躬秉武節,置十二部將軍,親帥師焉。」行自雲陽,北歷上郡、西河、五原,出長城,北登單于臺,至朔方,臨北河。勒兵十八萬騎,旌旗徑千餘里,威震匈奴。遣使者告單于曰:「南越王頭已縣於漢北闕矣。單于能戰,天子自將待邊;不能,亟來臣服。何但亡匿幕北寒苦之地為!」匈奴讋焉。還,祠黃帝於橋山,乃歸甘泉。

東越殺王餘善降。詔曰:「東越險阻反覆,為後世患,遷其民於江淮間。」遂虛其地。

春正月,行幸緱氏。詔曰:「朕用事華山,至於中嶽,獲駮麃,見夏后啟母石。翌日親登嵩高,御史乘屬,在廟旁吏卒咸聞呼萬歲者三。登禮罔不答。其令祠官加增太室祠,禁無伐其草木。以山下戶三百為之奉邑,名曰崇高,獨給祠,復亡所與。」行,遂東巡海上。

夏四月癸卯,上還,登封泰山,降坐明堂。詔曰:「朕以眇身承至尊,兢兢焉惟德菲薄,不明于禮樂,故用事八神。遭天地況施,著見景象,饩然如有聞。震于怪物,欲止不敢,遂登封泰山,至於梁父,然後升襢肅然。自新,嘉與士大夫更始,其以十月為元封元年。行所巡至,博、奉高、蛇丘,歷城、梁父,民田租逋賦貸,已除。加年七十以上孤寡帛,人二匹。四縣無出今年算。賜天下民爵一級,女子百戶牛酒。」

行自泰山,復東巡海上,至碣石。自遼西歷北邊九原,歸于甘泉。

秋,有星孛于東井,又孛于三台。

齊王閎薨。

二年冬十月,行幸雍,祠五畤。春,幸緱氏,遂至東萊。夏四月,還祠泰山。至瓠子,臨決河,命從臣將軍以下皆負薪塞河隄,作瓠子之歌。赦所過徒,賜孤獨高年米,人四石。還,作甘泉通天臺、長安飛廉館。

朝鮮王攻殺遼東都尉,乃募天下死罪擊朝鮮。

六月,詔曰:「甘泉宮內中產芝,九莖連葉。上帝博臨,不異下房,賜朕弘休。其赦天下,賜雲陽都百戶牛酒。」作芝房之歌。

秋,作明堂于泰山下。

遣樓船將軍楊僕、左將軍荀彘將應募罪人擊朝鮮。又遣將軍郭昌、中郎將衛廣發巴蜀兵平西南夷未服者,以為益州郡。

三年春,作角抵戲,三百里內皆來觀。

夏,朝鮮斬其王右渠降,以其地為樂浪、臨屯、玄菟、真番郡。

樓船將軍楊僕坐失亡多免為庶民,左將軍荀彘坐爭功棄市。

秋七月,膠西王端薨。

武都氐人反,分徙酒泉郡。

四年冬十月,行幸雍,祠五畤。通回中道,遂北出蕭關,歷獨鹿、鳴澤,自代而還,幸河東。春三月,祠后土。詔曰:「朕躬祭后土地祇,見光集于靈壇,一夜三燭。幸中都宮,殿上見光。其赦汾陰、夏陽、中都死罪以下,賜三縣及楊氏皆無出今年租賦。」

夏,大旱,民多暍死。

秋,以匈奴弱,可遂臣服,乃遣使說之。單于使來,死京師。匈奴寇邊,遣拔胡將軍郭昌屯朔方。

五年冬,行南巡狩,至于盛唐,望祀虞舜于九嶷。登灊天柱山,自尋陽浮江,親射蛟江中,獲之。舳艫千里,薄樅陽而出,作盛唐樅陽之歌。遂北至琅邪,並海,所過禮祠其名山大川。春三月,還至泰山,增封。甲子,祠高祖于明堂,以配上帝,因朝諸侯王列侯,受郡國計。夏四月,詔曰:「朕巡荊揚,輯江淮物,會大海氣,以合泰山。上天見象,增修封禪。其赦天下。所幸縣毋出今年租賦,賜鰥寡孤獨帛,貧窮者粟。」還幸甘泉,郊泰畤。

大司馬大將軍青薨。

初置刺史部十三州。名臣文武欲盡,詔曰:「蓋有非常之功,必待非常之人,故馬或奔踶而致千里,士或有負俗之累而立功名。夫泛駕之馬,跅弛之士,亦在御之而已。其令州郡察吏民有茂材異等可為將相及使絕國者。」

六年冬,行幸回中。春,作首山宮。

三月,行幸河東,祠后土。詔曰:「朕禮首山,昆田出珍物,化或為黃金。祭后土,神光三燭。其赦汾陰殊死以下,賜天下貧民布帛,人一匹。」

益州、昆明反,赦京師亡命令從軍,遣拔胡將軍郭昌將以擊之。

夏,京師民觀角抵于上林平樂館。

秋,大旱,蝗。

太初元年冬十月,行幸泰山。

十一月甲子朔旦,冬至,祀上帝于明堂。

乙酉,柏梁臺災。

十二月,禪高里,祠后土。東臨勃海,望祠蓬萊。春還,受計于甘泉。

二月,起建章宮。

夏五月,正曆,以正月為歲首。色上黃,數用五,定官名,協音律。

遣因杅將軍公孫敖築塞外受降城。

秋八月,行幸安定。遣貳師將軍李廣利發天下謫民西征大宛。

蝗從東方飛至敦煌。

二年春正月戊申,丞相慶薨。

三月,行幸河東,祠后土。令天下大酺五日,膢五日,祠門戶,比臘。

夏四月,詔曰:「朕用事介山,祭后土,皆有光應。其赦汾陰、安邑殊死以下。」

五月,籍吏民馬,補車騎馬。

秋,蝗。遣浚稽將軍趙破奴二萬騎出朔方擊匈奴,不還。

冬十二月,御史大夫兒寬卒。

三年春正月,行東巡海上。夏四月,還,修封泰山,禪石閭。

遣光祿勳徐自為築五原塞外列城,西北至盧朐,游擊將軍韓說將兵屯之。強弩都尉路博德築居延。

秋,匈奴入定襄、雲中,殺略數千人,行壞光祿諸亭障;又入張掖、酒泉,殺都尉。

四年春,貳師將軍廣利斬大宛王首,獲汗血馬來。作西極天馬之歌。

秋,起明光宮。

冬,行幸回中。

徙弘農都尉治武關,稅出入者以給關吏卒食。

天漢元年春正月,行幸甘泉,郊泰畤。三月,行幸河東,祠后土。

匈奴歸漢使者,使使來獻。

夏五月,赦天下。

秋,閉城門大搜。發謫戍屯五原。

二年春,行幸東海。還幸回中。

夏五月,貳師將軍三萬騎出酒泉,與右賢王戰于天山,斬首虜萬餘級。又遣因杅將軍出西河,騎都尉李陵將步兵五千人出居延北,與單于戰,斬首虜萬餘級。陵兵敗,降匈奴。

秋,止禁巫祠道中者。大搜。

渠黎六國使使來獻。

泰山、琅邪群盜徐绗等阻山攻城,道路不通。遣直指使者暴勝之等衣繡衣杖斧分部逐捕。刺史郡守以下皆伏誅。

冬十一月,詔關都尉曰:「今豪傑多遠交,依東方群盜。其謹察出入者。」

三年春二月,御史大夫王卿有罪,自殺。

初榷酒酤。

三月,行幸泰山,修封,祀明堂,因受計。還幸北地,祠常山,瘞玄玉。夏四月,赦天下。行所過毋出田租。

秋,匈奴入鴈門,太守坐畏飨棄市。

四年春正月,朝諸侯王于甘泉宮。發天下七科謫及勇敢士,遣貳師將軍李廣利將六萬騎、步兵七萬人出朔方,因杅將軍公孫敖萬騎、步兵三萬人出鴈門,游擊將軍韓說步兵三萬人出五原,強弩都尉路博德步兵萬餘人與貳師會。廣利與單于戰余吾水上連日,敖與左賢王戰不利,皆引還。

夏四月,立皇子髆為昌邑王。

秋九月,令死罪人贖錢五十萬減死一等。

太始元年春正月,因杅將軍敖有罪,要斬。

徙郡國吏民豪桀于茂陵、雲陵。

夏六月,赦天下。

二年春正月,行幸回中。

三月,詔曰:「有司議曰,往者朕郊見上帝,西登隴首,獲白麟以饋宗廟,渥洼水出天馬,泰山見黃金,宜改故名。今更黃金為麟趾褭蹄以協瑞焉。」因以班賜諸侯王。

秋,旱。九月,募死罪人贖錢五十萬減死一等。

御史大夫杜周卒。

三年春正月,行幸甘泉宮,饗外國客。

二月,令天下大酺五日。行幸東海,獲赤鴈,作朱鴈之歌。幸琅邪,禮日成山。登之罘,浮大海。山稱萬歲。冬,賜行所過戶五千錢,鰥寡孤獨帛人一匹。

四年春三月,行幸泰山。壬午,祀高祖于明堂,以配上帝,因受計。癸未,祀孝景皇帝于明堂。甲申,修封。丙戌,禪石閭。夏四月,幸不其,祠神人于交門宮,若有鄉坐拜者。作交門之歌。夏五月,還幸建章宮,大置酒,赦天下。

秋七月,趙有蛇從郭外入邑,與邑中蛇群鬥孝文廟下,邑中蛇死。

冬十月甲寅晦,日有蝕之。

十二月,行幸雍,祠五畤,西至安定、北地。

征和元年春正月,還,行幸建章宮。

三月,趙王彭祖薨。

冬十一月,發三輔騎士大搜上林,閉長安城門索,十一日乃解。巫蠱起。

二年春正月,丞相賀下獄死。

夏四月,大風發屋折木。

閏月,諸邑公主、陽石公主皆坐巫蠱死。

夏,行幸甘泉。

秋七月,桉道侯韓說、使者江充等掘蠱太子宮。壬午,太子與皇后謀斬充,以節發兵與丞相劉屈氂大戰長安,死者數萬人。庚寅,太子亡,皇后自殺。初置城門屯兵。更節加黃旄。御史大夫暴勝之、司直田仁坐失縱,勝之自殺,仁要斬。八月辛亥,太子自殺于湖。

癸亥,地震。

九月,立趙敬肅王子偃為平王。

匈奴入上谷、五原,殺略吏民。

三年春正月,行幸雍,至安定、北地。匈奴入五原、酒泉,殺兩都尉。三月,遣貳師將軍廣利將七萬人出五原,御史大夫商丘成二萬人出西河,重合侯馬通四萬騎出酒泉。成至浚稽山與虜戰,多斬首。通至天山,虜引去,因降車師。皆引兵還。廣利敗,降匈奴。

夏五月,赦天下。

六月,丞相屈氂下獄要斬,妻子梟首。

秋,蝗。

九月,反者公孫勇、胡倩發覺,皆伏辜。

四年春正月,行幸東萊,臨大海。

二月丁酉,隕石于雍,二,聲聞四百里。

三月,上耕于鉅定。還幸泰山,修封。庚寅,祀于明堂。癸巳,浈石閭。夏六月,還幸甘泉。

秋八月辛酉晦,日有蝕之。

後元元年春正月,行幸甘泉,郊泰畤,遂幸安定。

昌邑王髆薨。

二月,詔曰:「朕郊見上帝,巡于北邊,見群鶴留止,以不羅罔,靡所獲獻。薦于泰畤,光景並見。其赦天下。」

夏六月,御史大夫商丘成有罪自殺。侍中僕射莽何羅與弟重合侯通謀反,侍中駙馬都尉金日磾、奉車都尉霍光、騎都尉上官桀討之。

秋七月,地震,往往湧泉出。

二年春正月,朝諸侯王于甘泉宮,賜宗室。

二月,行幸盩厔五柞宮。乙丑,立皇子弗陵為皇太子。丁卯,帝崩于五柞宮,入殯于未央宮前殿。三月甲申,葬茂陵。

贊曰:漢承百王之弊,高祖撥亂反正,文景務在養民,至于稽古禮文之事,猶多闕焉。孝武初立,卓然罷黜百家,表章六經。遂疇咨海內,舉其俊茂,與之立功。興太學,修郊祀,改正朔,定曆數,協音律,作詩樂,建封禪,禮百神,紹周後,號令文章,煥焉可述。後嗣得遵洪業,而有三代之風。如武帝之雄材大略,不改文景之恭儉以濟斯民,雖詩書所稱何有加焉!


\end{pinyinscope}