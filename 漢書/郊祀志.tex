\article{郊祀志}

\begin{pinyinscope}
洪範八政,三曰祀。祀者,所以昭孝事祖,通神明也。旁及四夷,莫不修之;下至禽獸,豺獺有祭。是以聖王為之典禮。民之精爽不貳,齊肅聰明者,神或降之,在男曰覡,在女曰巫,使制神之處位,為之牲器。使先聖之後,能知山川,敬於禮儀,明神之事者,以為祝;能知四時犧牲,壇場上下,氏姓所出者,以為宗。故有神民之官,各司其序,不相亂也。民神異業,敬而不黷,故神降之嘉生,民以物序,災禍不至,所求不匱。

及少昊之衰,九黎亂德,民神雜擾,不可放物。家為巫史,享祀無度,黷齊明而神弗蠲。嘉生不降,禍災荐臻,莫盡其氣。顓頊受之,乃命南正重司天以屬神,命火正黎司地以屬民,使復舊常,亡相侵黷。

自共工氏霸九州,其子曰句龍,能平水土,死為社祠。有烈山氏王天下,其子曰柱,能殖百穀,死為稷祠。故郊祀社稷,所從來尚矣。

《虞書》曰,舜在璿璣玉衡,以齊七政。遂類于上帝,禋于六宗,望秩于山川,遍于群神。揖五瑞,擇吉月日,見四嶽諸牧,班瑞。歲二月,東巡狩,至于岱宗。岱宗,泰山也。柴,望秩于山川。遂見東后。東后者,諸侯也。合時月正日,同律度量衡,修五禮五樂,三帛二生一死為贄。五月,巡狩至南嶽。南嶽者,衡山也。八月,巡狩至西嶽。西嶽者,華山也。十一月,巡狩至北嶽。北嶽者,恆山也。皆如岱宗之禮。中嶽,嵩高也。五載一巡狩。

禹遵之。後十三世,至帝孔甲,淫德好神,神黷,二龍去之。其後十三世,湯伐桀,欲讓夏社,不可,作夏社。乃讓烈山子柱,而以周棄代為稷祠。後八世,帝太戊有桑穀生於廷,一暮大拱,懼。伊陟曰:「祅不勝德。」太戊修德,桑穀死。伊陟贊巫咸。後十三世,帝武丁得傅說為相,殷復興焉,稱高宗。有雉登鼎耳而雊,武丁懼。祖己曰:「修德。」武丁從之,位以永寧。後五世,帝乙嫚神而震死。後三世,帝紂淫亂,武王伐之。由是觀之,始未嘗不肅祇,後稍怠嫚也。

周公相成王,王道大洽,制禮作樂,天子曰明堂辟雍,諸侯曰泮宮。郊祀后稷以配天,宗祀文王於明堂以配上帝。四海之內各以其職來助祭。天子祭天下名山大川,懷柔百神,咸秩無文。五嶽視三公,四瀆視諸侯。而諸侯祭其疆內名山大川,大夫祭門、戶、井、灶、中霤五祀。士庶人祖考而已。各有典禮,而淫祀有禁。

後十三世,世益衰,禮樂廢。幽王無道,為犬戎所敗,平王東徙雒邑。秦襄公攻戎救周,列為諸侯,而居西,自以為主少昊之神,作西畤,祠白帝,其牲用騮駒黃牛羝羊各一云。

其後十四年,秦文公東獵汧渭之間,卜居之而吉。文公

薨黃蛇自天下屬地,其口止於鄜衍。文公問史敦,敦曰:「此上帝之徵,君其祠之。」於是作鄜畤,用三牲郊祭白帝焉。

自未作鄜畤,而雍旁故有吳陽武畤,雍東有好畤,皆廢無祀。或曰:「自古以雍州積高,神明之隩,故立畤郊上帝,諸神祠皆聚云。蓋黃帝時嘗用事,雖晚周亦郊焉。」其語不經見,縉紳者弗道。

作鄜畤後九年,文公獲若石云,于陳倉北阪城祠之。其神或歲不至,或歲數。來也常以夜,光輝若流星,從東方來,集於祠城,若雄雉,其聲殷殷云,野雞夜鳴。以一牢祠之,名曰陳寶。

作陳寶祠後七十一年,秦德公立,卜居雍。子孫飲馬於河,遂都雍。雍之諸祠自此興。用三百牢於鄜畤。作伏祠。磔狗邑四門,以御蠱災。

後四年,秦宣公作密畤於渭南,祭青帝。

後十三年,秦穆公立,病臥五日不寤;寤,乃言夢見上帝,上帝命穆公平晉亂。史書而藏之府。而後世皆曰上天。

穆公立九年,齊桓公既霸,會諸侯於葵丘,而欲封禪。管仲曰:「古者封泰山禪梁父者七十二家,而夷吾所記者十有二焉。昔無懷氏封泰山,禪云云;虙羲封泰山,禪云云;神農氏封泰山,禪云云;炎帝封泰山,禪云云;黃帝封泰山,禪亭亭;顓頊封泰山,禪云云;帝嚳封泰山,禪云云;堯封泰山,禪云云;舜封泰山,禪云云;禹封泰山,禪會稽;湯封泰山,禪云云;周成王封泰山,禪於社首:皆受命然後得封禪。」桓公曰:「寡人北伐山戎,過孤竹;西伐,束馬縣車,上卑耳之山;南伐至召陵,登熊耳山,以望江漢。兵車之會三,乘車之會六,九合諸侯,一匡天下,諸侯莫違我。昔三代受命,亦何以異乎?」於是管仲睹桓公不可窮以辭,因設之以事,曰:「古之封禪,鄗上黍,北里禾,所以為盛;江淮間一茅三脊,所以為藉也。東海致比目之魚,西海致比翼之鳥。然後物有不召而自至者十有五焉。今鳳皇麒麟不至,嘉禾不生,而蓬蒿藜莠茂,鴟梟群翔,而欲封禪,無乃不可乎?」於是桓公乃止。

是歲,秦穆公納晉君夷吾。其後三置晉國之君,平其亂。穆公立三十九年而卒。

後五十年,周靈王即位。時諸侯莫朝周,萇弘乃明鬼神事,設射不來。不來者,諸侯之不來朝者也。依物怪,欲以致諸侯。諸侯弗從,而周室愈微。後二世,至敬王時,晉人殺萇弘。

是時,季氏專魯,旅於泰山,仲尼譏之。

自秦宣公作密畤後二百五十年,而秦靈公於吳陽作上畤,祭黃帝;作下畤,祭炎帝。

後四十八年,周太史儋見秦獻公曰:「周始與秦國合而別,別五百載當復合,合七十年而伯王出焉。」儋見後七年,櫟陽雨金,獻公自以為得金瑞,故作畦畤櫟陽,而祀白帝。

後百一十歲,周赧王卒,九鼎入於秦。或曰,周顯王之四十二年,宋大丘社亡,而鼎淪沒於泗水彭城下。

自赧王卒後七年,秦莊襄王滅東周,周祀絕。後二十八年,秦并天下,稱皇帝。

秦始皇帝既即位,或曰:「黃帝得土德,黃龍地螾見。夏得木德,青龍止於郊,草木鬯茂。殷得金德,銀自山溢。周得火德,有赤烏之符。今秦變周,水德之時。昔文公出

臘,獲黑龍,此其水德之瑞。」於是秦更名河曰「德水」,以冬十月為年首,色尚黑,度以六為名,音上大呂,事統上法。

即帝位三年,東巡狩郡縣,祠騶嶧山,頌功業。於是從齊魯之儒生博士七十人,至於泰山下。諸儒生或議曰:「古者封禪為蒲車,惡傷山之土石草木;掃地而祠,席用苴峵,言其易遵也。」始皇聞此議各乖異,難施用,由此黜儒生。而遂除車道,上自泰山陽。至顛,立石頌德,明其得封也。從陰道下,禪於梁父。其禮頗采泰祝之祀雍上帝所用,而封臧皆祕之,世不得而記也。

始皇之上泰山,中阪遇暴風雨,休於大樹下。諸儒既黜,不得與封禪,聞始皇遇風雨,即譏之。

於是始皇遂東遊海上,行禮祠名山川及八神,來僊人羨門之屬。八神將自古而有之;或曰太公以來作之。齊所以為齊,以天齊也。其祀絕,莫知起時。八神,一曰天主,祠天齊。天齊淵水,居臨菑南郊山下下者。二曰地主,祠泰山梁父。蓋天好陰,祠之必於高山之下畤,命曰「畤」;地貴陽,祭之必於澤中圜丘云。三曰兵主,祠蚩尤。蚩尤在東平陸監鄉,齊之西竟也。四曰陰主,祠三山;五曰陽主,祠之罘山;六曰月主,祠之萊山:皆在齊北,並勃海。七曰日主,祠盛山。盛山斗入海,最居齊東北陽,以迎日出云。八曰四時主,祠琅邪。琅邪在齊東北,蓋歲之所始。皆各用牢具祠,而巫祝所損益,圭幣雜異焉。

自齊威、宣時,騶子之徒論著終始五德之運,及秦帝而齊人奏之,故始皇采用之。而宋毋忌、正伯僑、元尚、羨門高最後,皆燕人,為方僊道,形解銷化,依於鬼神之事。騶衍以陰陽主運顯於諸侯,而燕齊海上之方士傳其術不能通,然則怪迂阿諛苟合之徒自此興,不可勝數也。

自威、宣、燕昭使人入海求蓬萊、方丈、瀛洲。此三神山者,其傳在勃海中,去人不遠。蓋嘗有至者,諸僊人及不死之藥皆在焉。其物禽獸盡白,而黃金銀為宮闕。未至,望之如雲;及到,三神山反居水下,水臨之。患且至,則風輒引船而去,終莫能至云。世主莫不甘心焉。

及秦始皇至海上,則方士爭言之。始皇如恐弗及,使人齎童男女入海求之。船交海中,皆以風為解,曰未能至,望見之焉。其明年,始皇復游海上,至琅邪,過恆山,從上黨歸。後三年,游碣石,考入海方士,從上郡歸。後五年,始皇南至湘山,遂登會稽,並海上,幾遇海中三神山之奇藥。不得,還到沙丘崩。

二世元年,東巡碣石,並海,南歷泰山,至會稽,皆禮祠之,而刻勒始皇所立石書旁,以章始皇之功德。其秋,諸侯叛秦。三年而二世弒死。

始皇封禪之後十二年而秦亡。諸儒生疾秦焚詩書,誅滅文學,百姓怨其法,天下叛之,皆說曰:「始皇上泰山,為風雨所擊,不得封禪云。」此豈所謂無其德而用其事者邪?

昔三代之居皆河洛之間,故嵩高為中嶽,而四嶽各如其方,四瀆咸在山東。至秦稱帝,都咸陽,則五嶽、四瀆皆并在東方。自五帝以至秦,迭興迭衰,名山大川或在諸侯,或在天子,其禮損益世殊,不可勝記。及秦并天下,令祠官所常奉天地名山大川鬼神可得而序也。

於是自崤以東,名山五,大川祠二。曰太室。太室,嵩高也。恆山,泰山,會稽,湘山。水曰泲,曰淮。春以脯酒為歲禱,因泮凍;秋涸凍;冬塞禱祠。其牲用牛犢各一,牢具圭幣各異。自華以西,名山七,名川四。曰華山,薄山。薄山者,襄山也。岳山,岐山,吳山,鴻冢,瀆山。瀆山,蜀之岷山也。水曰河,祠臨晉;沔,祠漢中;湫淵,祠朝那;江水,祠蜀。亦春秋泮涸禱塞如東方山川;而牲亦牛犢牢具圭幣各異。而四大冢鴻、岐、吳、嶽,皆有嘗禾。陳寶節來祠,其河加有嘗醪。此皆雍州之域,近天子都,故加車一乘,騮駒四。霸、產、豐、澇、涇、渭、長水,皆不在大山川數,以近咸陽,盡得比山川祠,而無諸加。汧、洛二淵,鳴澤、蒲山、嶽婿山之屬,為小山川,亦皆禱塞泮涸祠,禮不必同。而雍有日、月、參、辰、南北斗、熒惑、太白、歲星、填星、辰星、二十八宿、風伯、雨師、四海、九臣、十四臣、諸布、諸嚴、諸逐之屬,百有餘廟。西亦有數十祠。於湖有周天子祠。於下邽有天神。豐、鎬有昭明、天子辟池。於杜、亳有五杜主之祠、壽星祠;而雍、菅廟祠亦有杜主。杜主,故周之右將軍,其在秦中最小鬼之神者也。各以歲時奉祠。

唯雍四時上帝為尊,其光景動人民,唯陳寶。故雍四畤,春以為歲祠禱,因泮凍,秋涸凍,冬賽祠,五月嘗駒,及四中之月月祠,若陳寶節來一祠。春夏用騂,秋冬用騮。畤駒四匹,木寓龍一駟,木寓車馬一駟,各如其帝色。黃犢羔各四,圭幣各有數,皆生瘞埋,無俎豆之具。三年一郊。秦以十月為歲首,故常以十月上宿郊見,通權火,拜於咸陽之旁,而衣上白,其用如經祠云。西畤、畦畤,祠如其故,上不親往。諸此祠皆太祝常主,以歲時奉祠之。至如它名山川諸神及八神之屬,上過則祠,去則已。郡縣遠方祠者,民各自奉祠,不領於天子之祝官。祝官有祕祝,即有災祥,輒祝祠移過於下。

漢興,高祖初起,殺大蛇,有物曰:「蛇,白帝子,而殺者赤帝子也。」及高祖禱豐枌榆社,侚沛,為沛公,則祀蚩尤,釁鼓旗。遂以十月至霸上,立為漢王。因以十月為年首,色上赤。

二年冬,東擊項籍而還入關,問:「故秦時上帝祠何帝也?」對曰:「四帝,有白、青、黃、赤帝之祠。」高祖曰:「吾聞天有五帝,而四,何也?」莫知其說。於是高祖曰:「吾知之矣,乃待我而具五也。」乃立黑帝祠,名曰北畤。有司進祠,上不親往。悉召故秦祀官,復置太祝、太宰,如其故儀禮。因令縣為公社。下詔曰:「吾甚重祠而敬祭。今上帝之祭及山川諸神當祠者,各以其時禮祠之如故。」

後四歲,天下已定,詔御史令豐治枌榆社,常以時,春以羊彘祠之。令祝立蚩尤之祠於長安。長安置祠祀官、女巫。其梁巫祠天、地、天社、天水、房中、當上之屬;晉巫祠五帝、東君、雲中君、巫社、巫祠、族人炊之屬;秦巫祠杜主、巫保、族纍之屬;荊巫祠堂下、巫先、司命、施糜之屬;九天巫祠九天:皆以歲時祠宮中。其河巫祠河於臨晉,而南山巫祠南山、秦中。秦中者,二世皇帝也。各有時日。

其後二歲,或言曰周興而邑立后稷之祠,至今血食天下。於是高祖制詔御史:「其令天下立靈星祠,常以歲時祠以牛。」

高祖十年春,有司請令縣常以春二月及臘祠稷以羊彘,民里社各自裁以祠。制曰:「可。」

文帝即位十三年,下詔曰:「祕祝之官移過於下,朕甚弗取,其除之。」

始名山大川在諸侯,諸侯祝各自奉祠,天子官不領。及齊、淮南國廢,令太祝盡以歲時致禮如故。

明年,以歲比登,詔有司增雍五畤路車各一乘,駕被具;西畤、畦畤寓車各一乘,寓馬四匹,駕被具;河、湫、漢水,玉加各二;及諸祀皆廣壇場,圭幣俎豆以差加之。

魯人公孫臣上書曰:「始秦得水德,及漢受之,推終始傳,則漢當土德,土德之應黃龍見。宜改正朔,服色上黃。」時丞相張蒼好律曆,以為漢乃水德之時,河決金隄,其符也。年始冬十月,色外黑內赤,與德相應。公孫臣言非是,罷之。明年,黃龍見成紀。文帝召公孫臣,拜為博士,與諸生申明土德,草改曆服色事。其夏,下詔曰:「有異物之神見於成紀,毋害於民,歲以有年。朕幾郊祀上帝諸神,禮官議,毋諱以朕勞。」有司皆曰:「古者天子夏親郊祀上帝於郊,故曰郊。」於是夏四月,文帝始幸雍郊見五畤,祠衣皆上赤。

趙人新垣平以望氣見上,言「長安東北有神氣,成五采,若人冠冕焉。或曰東北神明之舍,西方神明之墓也。天瑞下,宜立祠上帝,以合符應。」於是作渭陽五帝廟,同宇,帝一殿,面五門,各如其帝色。祠所用及儀亦如雍五畤。

明年夏四月,文帝親拜霸渭之會,以郊見渭陽五帝。五帝廟臨渭,其北穿蒲池溝水。權火舉而祠,若光煇然屬天焉。於是貴平至上大夫,賜累千金。而使博士諸生刺六經中作王制,謀議巡狩封禪事。

文帝出長門,若見五人於道北,遂因其直立五帝壇,祠以五牢。

其明年,平使人持玉杯,上書闕下獻之。平言上曰:「闕下有寶玉氣來者。」已視之,果有獻玉杯者,刻曰「人主延壽」。平又言「

臣候日再中」。居頃之,日卻復中。於是始更以十七年為元年,令天下大酺。平言曰:「周鼎亡在泗水中,今河決通於泗,臣望東北汾陰直有金寶氣,意周鼎其出乎?兆見不迎則不至。」於是上使使治廟汾陰南,臨河,欲祠出周鼎。人有上書告平所言皆詐也。下吏治,誅夷平。是後,文帝怠於改正服鬼神之事,而渭陽、長門五帝使祠官領,以時致禮,不往焉。

明年,匈奴數入邊,興兵守御。後歲少不登。數歲而孝景即位。十六年,祠官各以歲時祠如故,無有所興。

武帝初即位,尤敬鬼神之祀。漢興已六十餘歲矣,天下艾安,縉紳之屬皆望天子封禪改正度也,而上鄉儒術,招賢良。趙綰、王臧等以文學為公卿,欲議古立明堂城南,以朝諸侯,草巡狩封禪改曆服色事未就。竇太后不好儒術,使人微伺趙綰等姦利事,按綰、臧,綰、臧自殺,諸所興為皆廢。六年,竇太后崩。其明年,徵文學之士。

明年,上初至雍,郊見五畤。後常三歲一郊。是時上求神君,舍之上林中磃氏館。神君者,長陵女子,以乳死,見神於先後宛若。宛若祠之其室,民多往祠。平原君亦往祠,其後子孫以尊顯。及上即位,則厚禮置祠之內中。聞其言,不見其人云。

是時,李少君亦以祠灶、穀道、卻老方見上,上尊之。少君者,故深澤侯人,主方。匿其年及所生長。常自謂七十,能使物,卻老。其游以方遍諸侯。無妻子。人聞其能使物及不死,更餽遺之,常餘金錢衣食。人皆以為不治產業而饒給,又不知其何所人,愈信,爭事之。少君資好方,善為巧發奇中。常從武安侯宴,坐中有年九十餘老人,少君乃言與其大父游射處,老人為兒從其大父,識其處,一坐盡驚。少君見上,上有故銅器,問少君。少君曰:「此器齊桓公十年陳於柏寢。」已而按其刻,果齊桓公器。一宮盡駭,以為少君神,數百歲人也。少君言上:「祠灶皆可致物,致物而丹沙可化為黃金,黃金成以為飲食器則益壽,益壽而海中蓬萊僊者乃可見之,以封禪則不死,黃帝是也。臣嘗游海上,見安期生,安期生食臣棗,大如瓜。安期生僊者,通蓬萊中,合則見人,不合則隱。」於是天子始親祠灶,遣方士入海求蓬萊安期生之屬,而事化丹沙諸藥齊為黃金矣。久之,少君病死。天子以為化去不死也,使黃錘史寬舒受其方,而海上燕齊怪迂之方士多更來言神事矣。

亳人謬忌奏祠泰一方,曰:「天神貴者泰一,泰一佐曰五帝。古者天子以春秋祭泰一東南郊,日一太牢,七日,為壇開八通之鬼道。」於是,天子令太祝立其祠長安城東南郊,常奉祠如忌方。其後,人上書言「古者天子三年一用太牢祠三一:天一、地一、泰一。」天子許之,令太祝領祠之於忌泰一壇上,如其方。後人復有言「古天子常以春解祠,祠黃帝用一梟、破鏡;冥羊用羊祠;馬行用一青牡馬;泰一、皋山山君用牛;武夷君用乾魚;陰陽使者以一牛。」令祠官領之如其方,而祠泰一於忌泰一壇旁。

後二年,郊雍,獲一角獸,若麃然。有司曰:「陛下肅祗郊祀,上帝報享,錫一角獸,蓋麟云。」於是以薦五畤,畤加一牛以燎。賜諸侯白金,以風符應合於天也。於是濟北王以為天子且封禪,上書獻泰山及其旁邑,天子以它縣償之。常山王有罪,俣,天子封其弟真定,以續先王祀,而以常山為郡。然後五嶽皆在天子之郡。

明年,齊人少翁以方見上。上有所幸李夫人,夫人卒,少翁以方蓋夜致夫人及灶鬼之貌云,天子自帷中望見焉。乃拜少翁為文成將軍,賞賜甚多,以客禮禮之。文成言:「上即欲與神通,宮室被服非象神,神物不至。」乃作畫雲氣車,及各以勝日駕車辟惡鬼。又作甘泉宮,中為臺室,畫天地泰一諸鬼神,而置祭具以致天神。居歲餘,其方益衰,神不至。乃為帛書以飯牛,陽不知,言此牛腹中有奇書。殺視得書,書言甚怪。天子識其手,問之,果為書。於是誅文成將軍,隱之。

其後又作柏梁、銅柱、承露僊人掌之屬矣。

文成死明年,天子病鼎湖甚,巫醫無所不致。游水發根言上郡有巫,病而鬼下之。上召置祠之甘泉。及病,使人問神君,神君言曰:「天子無憂病。病少瘉,強與我會甘泉。」於是上病瘉,遂起,幸甘泉,病良已。大赦,置壽宮神君。神君最貴者曰太一,其佐曰太禁、司命之屬,皆從之。非可得見,聞其言,言與人音等。時去時來,來則風肅然。居室帷中,時晝言,然常以夜。天子祓,然後入。因巫為主人,關飲食,所欲言,行下。又置壽宮、北宮,張羽旗,設共具,以禮神君。神君所言,上使受書,其名曰「畫法」。其所言,世俗之所知也,無絕殊者,而天子心獨憙。其事祕,世莫知也。

後三年,有司言元宜以天瑞,不宜以一二數。一元曰「建」,二元以長星曰「光」,今郊得一角獸曰「狩」云。

其明年,天子郊雍,曰:「今上帝朕親郊,而后土無祀,則禮不答也。」有司與太史令談、祠官寬舒議:「天地牲,角繭栗。今陛下親祠后土,后土宜於澤中圜丘為五壇,壇一黃犢牢具。已祠盡瘞,而從祠衣上黃。」於是天子東幸汾陰。汾陰男子公孫滂洋等見汾旁有光如絳,上遂立后土祠於汾陰脽上,如寬舒等議。上親望拜,如上帝禮。禮畢,天子遂至滎陽。還過雒陽,下詔封周後,令奉其祀。語在武紀。上始巡幸郡縣,寖尋於泰山矣。

其春,樂成侯登上書言欒大。欒大,膠東宮人,故嘗與文成將軍同師,已而為膠東王尚方。而樂成侯姊為康王后,無子。王死,它姬子立為王,而康后有淫行,與王不相中,相危以法。康后聞文成死,而欲自媚於上,乃遣欒大入,因樂成侯求見言方。天子既誅文成,後悔其方不盡,及見欒大,大說。大為人長美,言多方略,而敢為大言,處之不疑。大言曰:「臣常往來海中,見安期、羨門之屬,顧以臣為賤,不信臣。又以為康王諸侯耳,不足與方。臣數以言康王,康王又不用臣。臣之師曰:『黃金可成,而河決可塞,不死之藥可得,僊人可致也。』然臣恐效文成,則方士皆掩口,惡敢言方哉!」上曰:「

文成食馬肝死耳。子誠能修其方,我何愛乎!」大曰:「臣師非有求人,人者求之。陛下必欲致之,則貴其使者,令為親屬,以客禮待之,勿卑,使各佩其信印,乃可使通言於神人。神人尚肯邪不邪,尊其使然後可致也。」於是上使驗小方,鬥棋,棋自相觸擊。

是時,上方憂河決而黃金不就,乃拜大為五利將軍。居月餘,得四印;得天士將軍、地士將軍、大通將軍印。制詔御史:「昔禹疏九河,決四瀆。間者,河溢皋陸,隄繇不息。朕臨天下二十有八年,天若遺朕士而大通焉。乾稱『飛龍』,『鴻漸于般』,朕意庶幾與焉。其以二千戶封地士將軍大為樂通侯。」賜列侯甲第,童千人。乘輿斥車馬帷帳器物以充其家。又以衛長公主妻之,齎金十萬斤,更名其邑曰當利公主。天子親如五利之弟,使者存問共給,相屬於道。自大主將相以下,皆置酒其家,獻遺之。天子又刻玉印曰「天道將軍」,使使衣羽衣,夜立白茅上,五利將軍亦衣羽衣,立白茅上受印,以視不臣也。而佩「天道」者,且為天子道天神也。於是五利常夜祠其家,欲以下神。後裝治行,東入海求其師云。大見數月,佩六印,貴震天下,而海上燕齊之間,莫不搤掔而自言有禁方能神僊矣。

其夏六月,汾陰巫錦為民祠魏脽后土營旁,見地如鉤狀,掊視得鼎。鼎大異於眾鼎,文鏤無款識,怪之,言吏。吏告河東太守勝,勝以聞。天子使驗問巫得鼎無姦詐,乃以禮祠,迎鼎至甘泉,從上行,薦之。至中山,晏溫,有黃雲焉。有鹿過,上自射之,因之以祭云。至長安,公卿大夫皆議尊寶鼎。天子曰:「間者河溢,歲數不登,故巡祭后土,祈為百姓育穀。今年豐楙未報,鼎曷為出哉?」有司皆言:「聞昔泰帝興神鼎一,一者一統,天地萬物所繫象也。黃帝作寶鼎三,象天地人。禹收九牧之金,鑄九鼎,象九州。皆嘗鬺享上帝鬼神。其空足曰鬲,以象三德,饗承天祜。夏德衰,鼎遷於殷;殷德衰,鼎遷於周;周德衰,鼎遷於秦;秦德衰,宋之社亡,鼎乃淪伏而不見。周頌曰:『自堂徂基,自羊徂牛,鼐鼎及鼒;不羁不敖,胡考之休。』今鼎至甘泉,以光潤龍變,承休無疆。合茲中山,有黃白雲降,蓋若獸為符,路弓乘矢,集獲壇下,報祠大亨。唯受命而帝者心知其意而合德焉。鼎宜視宗禰廣,臧於帝庭,以合明應。」制曰:「可。」

入海求蓬萊者,言蓬萊不遠,而不能至者,殆不見其氣。上乃遣望氣佐候其氣云。

其秋,上雍,且郊。或曰「五帝,泰一之佐也,宜立泰一而上親郊之」。上疑未定。

齊人公孫卿曰:「今年得寶鼎,其冬辛巳朔旦冬至,與黃帝時等。」卿有札書曰:「黃帝得寶鼎冕候,問於鬼臾區,鬼臾區對曰:『黃帝得寶鼎神策,是歲己酉朔旦冬至,得天之紀,終而復始。』於是黃帝迎日推策,後率二十歲復朔旦冬至,凡二十推,三百八十年,黃帝僊登于天。」卿因所忠欲奏之。所忠視其書不經,疑其妄言,謝曰:「寶鼎事已決矣。尚何以為!」卿因嬖人奏之。上大說,乃召問卿。對曰:「受此書申公,申公已死。」上曰:「申公何人也?」卿曰:「齊人,與安期生通,受黃帝言,無書,獨有此鼎書。曰『漢興復當黃帝之時。』曰『漢之聖者,在高祖之孫且曾孫也。寶鼎出而與神通,封禪。封禪七十二王,唯黃帝得上泰山封。』申公曰:『漢帝亦當上封禪,封禪則能僊登天矣。黃帝萬諸侯,而神靈之封君七千。天下名山八,而三在蠻夷,五在中國。中國華山、首山、太室山、泰山、東萊山,此五山黃帝之所常游,與神會。黃帝且戰且學僊,患百姓非其道,乃斷斬非鬼神者。百餘歲然後得與神通。黃帝郊雍上帝,宿三月。鬼臾區號大鴻,死葬雍,故鴻冢是也。其後黃帝接萬靈明庭。明庭者,甘泉也。所謂寒門者,谷口也。黃帝采首山銅,鑄鼎於荊山下。鼎既成,有龍垂胡敘下迎黃帝。黃帝上騎,群臣後宮從上龍七十餘人,龍乃去。餘小臣不得上,乃悉持龍敘,龍敘拔,墯,墯黃帝之弓。百姓卬望黃帝既上天,乃抱其弓與龍敘號,故後世因名其處曰鼎湖,其弓曰烏號。』」於是天子曰:「嗟乎!誠得如黃帝,吾視去妻子如脫屣耳。」拜卿為郎,使東候神於太室。

上遂郊雍,至隴西,登空桐,幸甘泉。令祠官寬舒等具泰一祠壇,祠壇放亳忌泰一壇,三陔。五帝壇環居其下,各如其方。黃帝西南,除八通鬼道。泰一所用,如雍一畤物,而加醴棗脯之屬,殺一氂牛以為俎豆牢具。而五帝獨有俎豆醴進。其下四方地,為腏,食群神從者及北斗云。已祠,胙餘皆燎之。其牛色白,白鹿居其中,彘在鹿中,鹿中水而酒之。祭日以牛,祭月以羊彘特。泰一祝宰則衣紫及繡。五帝各如其色,日赤,月白。

十一月辛巳朔旦冬至,昒爽,天子始郊拜泰一。朝朝日,夕夕月,則揖;而見泰一如雍郊禮。其贊饗曰:「天始以寶鼎神策授皇帝,朔而又朔,終而復始,皇帝敬拜見焉。」而衣上黃。其祠列火滿壇,壇旁亨炊具。有司云「祠上有光」。公卿言「皇帝始郊見泰一雲陽,有司奉瑄玉嘉牲薦饗,是夜有美光,及晝,黃氣上屬天。」太史令談、祠官寬舒等曰:「神靈之休,祐福兆祥,宜因此地光域立泰畤壇以明應。令太祝領,秋及臘間祠。二歲天子壹郊見。」

其秋,為伐南越,告禱泰一,以牡荊畫幡日月北斗登龍,以象太一三星,為泰一ⓑ旗,命曰「靈旗」。為兵禱,則太史奉以指所伐國。而五利將軍使不敢入海,之泰山祠。上使人隨驗,實無所見。五利妄言見其師,其方盡,多不讎。上乃誅五利。

其冬,公孫卿候神河南,言見僊人跡緱氏城上,有物如雉,往來城上。天子親幸緱氏視跡,問卿:「得毋效文成、五利乎?」卿曰:「僊者非有求人主,人主者求之。其道非少寬暇,神不來。言神事,如迂誕,積以歲,乃可致。」於是郡國各除道,繕治宮館名山神祠所,以望幸矣。

其春,既滅南越,嬖臣李延年以好音見。上善之,下公卿議,曰:「民間祠有鼓舞樂,今郊祀而無樂,豈稱乎?」公卿曰:「古者祠天地皆有樂,而神祇可得而禮。」或曰:「泰帝使素女鼓五十絃瑟,悲,帝禁不止,故破其瑟為二十五絃。」於是塞南越,禱祠泰一、后土,始用樂舞。益召歌兒,作二十五絃及空侯瑟自此起。

其來年冬,上議曰:「古者先振兵釋旅,然後封禪。」乃遂北巡朔方,勒兵十餘萬騎,還祭黃帝冢橋山,釋兵敘如。上曰:「

吾聞黃帝不死,有冢,何也?」或對曰:「黃帝以僊上天,群臣葬其衣冠。」既至甘泉,為且用事泰山,先類祠泰一。

自得寶鼎,上與公卿諸生議封禪。封禪用希曠絕,莫知其儀體,而群儒采封禪尚書、周官、王制之望祀射牛事。齊人丁公年九十餘,曰:「封禪者,古不死之名也。秦皇帝不得上封。陛下必欲上,稍上即無風雨,遂上封矣。」上於是乃令諸儒習射牛,草封禪儀。數年,至且行。天子既聞公孫卿及方士之言,黃帝以上封禪皆致怪物與神通,欲放黃帝以接神人蓬萊,高世比德於九皇,而頗采儒術以文之。群儒既已不能辯明封禪事,又拘於詩書古文而不敢騁。上為封祠器視群儒,群儒或曰「不與古同」,徐偃又曰「太常諸生行禮不如魯善」,周霸屬圖封事,於是上黜偃、霸,而盡罷諸儒弗用。

三月,乃東幸緱氏,禮登中嶽太室。從官在山上聞若有言「萬歲」云。問上,上不言;問下,下不言。乃令祠官加增太室祠,禁毋伐其山木,以山下戶凡三百封镯高,為之奉邑,獨給祠,復無有所與。上因東上泰山,泰山草木未生,乃令人上石立之泰山顛。

上遂東巡海上,行禮祠八神。齊人之上疏言神怪奇方者以萬數,乃益發船,令言海中神山者數千人求蓬萊神人。公孫卿持節常先行候名山,至東萊,言夜見大人,長數丈,就之則不見,見其跡甚大,類禽獸云。群臣有言見一老父牽狗,言「吾欲見鉅公」,已忽不見。上既見大跡,未信,及群臣又言老父,則大以為僊人也。宿留海上,與方士傳車及間使求神僊人以千數。

四月,還至奉高。上念諸儒及方士言封禪人殊,不經,難施行。天子至梁父,禮祠地主。至乙卯,令侍中儒者皮弁縉紳,射牛行事。封泰山下東方,如郊祠泰一之禮。封廣丈二尺,高九尺,其下則有玉牒書,書祕。禮畢,天子獨與侍中奉車子侯上泰山,亦有封。其事皆禁。明日,下陰道。丙辰,禪泰山下阯東北肅然山,如祭后土禮。天子皆親拜見,衣上黃而盡用樂焉。江淮間一茅三脊為神藉。五色土益雜封。縱遠方奇獸飛禽及白雉諸物,頗以加祠。兕牛象犀之屬不用。皆至泰山,然後去。封禪祠,其夜若有光,晝有白雲出封中。

天子從禪還,坐明堂,群臣更上壽。下詔改元為元封。語在武紀。又曰:「古者天子五載一巡狩,用事泰山,諸侯有朝宿地。其令諸侯各治邸泰山下。」

天子既已封泰山,無風雨,而方士更言蓬萊諸神若將可得,於是上欣然庶幾遇之,復東至海上望焉。奉車子侯暴病,一日死。上乃遂去,並海上,北至碣石,巡自遼西,歷北邊至九原。五月,乃至甘泉,周萬八千里云。

其秋,有星孛於東井。後十餘日,有星孛於三能。望氣王朔言:「候獨見填星出如瓜,食頃,復入。」有司皆曰:「陛下建漢家封禪,天其報德星云。」

其來年冬,郊雍五帝。還,拜祝祠泰一。贊饗曰:「德星昭衍,厥維休祥。壽星仍出,淵燿光明。信星昭見,皇帝敬拜泰祝之享。」

其春,公孫卿言見神人東萊山,若云「欲見天子」。天子於是幸緱氏城,拜卿為中大夫。遂至東萊,宿,留之數日,毋所見,見大人跡云。復遣方士求神人采藥以千數。是歲旱。天子既出亡名,乃禱萬里沙,過祠泰山。還至瓠子,自臨塞決河,留二日,湛祠而去。

是時既滅兩粵,粵人勇之乃言「粵人俗鬼,而其祠皆見鬼,數有效。昔東甌王敬鬼,壽百六十歲。後世怠嫚,故衰耗。」乃命粵巫立粵祝祠,安臺無壇,亦祠天神帝百鬼,而以雞卜。上信之,粵祠雞卜自此始用。

公孫卿曰:「僊人可見,上往常遽,以故不見。今陛下可為館如緱氏城,置脯棗,神人宜可致。且僊人好樓居。」於是上令長安則作飛廉、桂館,甘泉則作益壽、延壽館,使卿持節設具而候神人。乃作通天臺,置祠具其下,將招來神僊之屬。於是甘泉更置前殿,始廣諸宮室。夏,有芝生甘泉殿房內中。天子為塞河,興通天,若有光云,乃下詔赦天下。

其明年,伐朝鮮。夏,旱。公孫卿曰:「黃帝時封則天旱,乾封三年。」上乃下詔:「天旱,意乾封乎?其令天下尊祠靈星焉。」

明年,上郊雍五畤,通回中道,遂北出蕭關,歷獨鹿、鳴澤,自西河歸,幸河東祠后土。

明年冬,上巡南郡,至江陵而東。登禮灊之天柱山,號曰南嶽。浮江,自潯陽出樅陽,過彭蠡,禮其名山川。北至琅邪,並海上。四月,至奉高修封焉。

初,天子封泰山,泰山東北阯古時有明堂處,處險不敞。上欲治明堂奉高旁,未曉其制度。濟南人公玉帶上黃帝時明堂圖。明堂中有一殿,四面無壁,以茅蓋,通水,水圜宮垣,為復道,上有樓,從西南入,名曰昆侖,天子從之入,以拜祀上帝焉。於是上令奉高作明堂汶上,如帶圖。及是歲修封,則祠泰一、五帝於明堂上坐,合高皇帝祠坐對之。祠后土於下房,以二十太牢。天子從昆侖道入,始拜明堂如郊禮。畢,抠堂下。而上又上泰山,自有祕祠其顛。而泰山下祠五帝,各如其方,黃帝并赤帝所,有司侍祠焉。山上舉火,下悉應之。還幸甘泉,郊泰畤。春幸汾陰,祠后土。

明年,幸泰山,以十一月甲子朔旦冬至日祀上帝於明堂,後每修封。其贊饗曰:「天增授皇帝泰元神策,周而復始。皇帝敬拜泰一。」東至海上,考入海及方士求神者,莫驗,然益遣,幾遇之。乙酉,柏梁災。十二月甲午朔,上親禪高里,祠后土。臨勃海,將以望祀蓬萊之屬,幾至殊庭焉。

上還,以柏梁災故,受計甘泉。公孫卿曰:「黃帝就青靈臺,十二日燒,黃帝乃治明庭。明庭,甘泉也。」方士多言古帝王有都甘泉者。其後天子又朝諸侯甘泉,甘泉作諸侯邸。勇之乃曰:「粵俗有火災,復起屋,必以大,用勝服之。」於是作建章宮,度為千門萬戶。前殿度高未央。其東則鳳闕,高二十餘丈。其西則商中,數十里虎圈。其北治大池,漸臺高二十餘丈,名曰泰液,池中有蓬萊、方丈、瀛州、壺梁,象海中神山龜魚之屬。其南有玉堂璧門大鳥之屬。立神明臺、井幹樓,高五十丈,輦道相屬焉。

夏,漢改曆,以正月為歲首,而色上黃,官更印章以五字,因為太初元年。是歲,西伐大宛,蝗大起。丁夫人、雒陽虞初等以方祠詛匈奴、大宛焉。

明年,有司言雍五畤無牢孰具,芬芳不備。乃令祠官進畤犢牢具,色食所勝,而以木寓馬代駒云。及諸名山川用駒者,悉以木寓馬代。獨行過親祠,乃用駒,它禮如故。

明年,東巡海上,考神僊之屬,未有驗者。方士有言黃帝時為五城十二樓,以候神人於執期,名曰迎年。上許作之如方,名曰明年。上親禮祠,上犢黃焉。

公玉帶曰:「黃帝時雖封泰山,然風后、封鉅、岐伯令黃帝封東泰山,禪凡山,合符,然後不死。」天子既令設祠具,至東泰山,東泰山卑小,不稱其聲,乃令祠官禮之,而不封焉。其後令帶奉祠候神物。復還泰山,修五年之禮如前,而加禪祠石閭。石閭者,在泰山下阯南方,方士言僊人閭也,故上親禪焉。

其後五年,復至泰山修封,還過祭恆山。

自封泰山後,十三歲而周遍於五嶽、四瀆矣。

後五年,復至泰山修封。東幸琅邪,禮日成山,登之罘,浮大海,用事八神延年。又祠神人於交門宮,若有鄉坐拜者云。

後五年,上復修封於泰山。東游東萊,臨大海。是歲,雍縣無雲如剨者三,或如虹氣蒼黃,若飛鳥集棫陽宮南,聲聞四百里。隕石二,黑如黳,有司以為美祥,以薦宗廟。而方士之候神入海求蓬萊者終無驗,公孫卿猶以大人之跡為解。天子猶羈縻不絕,幾遇其真。

諸所興,如薄忌泰一及三一、冥羊、馬行、赤星,五床。寬舒之祠宮以歲時致禮。凡六祠,皆大祝領之。至如八神,諸明年、凡山它名祠,行過則祠,去則已。方士所興祠,各自主,其人終則已,祠官不主。它祠皆如故。甘泉泰一、汾陰后土,三年親郊祠,而泰山五年一修封。武帝凡五修封。昭帝即位,富於春秋,未嘗親巡祭云。

宣帝即位,由武帝正統興,故立三年,尊孝武廟為世宗,行所巡狩郡國皆立廟。告祠世宗廟日,有白鶴集後庭。以立世宗廟告祠孝昭寢,有鴈五色集殿前。西河築世宗廟,神光興於殿旁,有鳥如白鶴,前赤後青。神光又興於房中,如燭狀。廣川國世宗廟殿上有鍾音,門戶大開,夜有光,殿上盡明。上乃下詔赦天下。

時,大將軍霍光輔政,上共己正南面,非宗廟之祀不出。十二年,乃下詔曰:「蓋聞天子尊事天地,修祀山川,古今通禮也。間者,上帝之祠闕而不親十有餘年,朕甚懼焉。朕親飭躬齊戒,親奉祀,為百姓蒙嘉氣,獲豐年焉。」

明年正月,上始幸甘泉,郊見泰畤,數有美祥。修武帝故事,盛車服,敬齊祠之禮,頗作詩歌。

其三月,幸河東,祠后土,有神爵集,改元為神爵。制詔太常:「夫江海,百川之大者也,今闕焉無祠。其令祠官以禮為歲事,以四時祠江海雒水,祈為天下豐年焉。」自是五嶽、四瀆皆有常禮。東嶽泰山於博,中嶽泰室於嵩高,南嶽灊山於灊,西嶽華山於華陰,北嶽常山於上曲陽,河於臨晉,江於江都,淮於平氏,濟於臨邑界中,皆使者持節侍祠。唯泰山與河歲五祠,江水四,餘皆一禱而三祠云。

時,南郡獲白虎,獻其皮牙爪,上為立祠。又以方士言,為隨侯、劍寶、玉寶璧、周康寶鼎立四祠於未央宮中。又祠太室山於即墨,三戶山於下密,祠天封苑火井於鴻門。又立歲星、辰星、太白、熒惑、南斗祠於長安城旁。又祠參山八神於曲城,蓬山石社石鼓於臨朐,之罘山於腄,成山於不夜,萊山於黃。成山祠日,萊山祠月。又祠四時於琅邪,蚩尤於壽良。京師近縣鄠,則有勞谷、五床山、日月、五帝、僊人、玉女祠。雲陽有徑路神祠,祭休屠王也。又立五龍山僊人祠及黃帝、天神、帝原水,凡四祠於膚施。

或言益州有金馬碧雞之神,可醮祭而致,於是遣諫大夫王褒使持節而求之。

大夫劉更生獻淮南枕中洪寶苑祕之方,令尚方鑄作。事不驗,更生坐論。京兆尹張敞上疏諫曰:「願明主時忘車馬之好,斥遠方士之虛語,游心帝王之術,太平庶幾可興也。」後尚方待詔皆罷。

是時,美陽得鼎,獻之。下有司議,多以為宜薦見宗廟,如元鼎時故事。張敞好古文字,桉鼎銘勒而上議曰:「臣聞周祖始乎后稷,后稷封於斄,公劉發跡於豳,大王建國於廄梁,文武興於酆鎬。由此言之,則廄梁豐鎬之間周舊居也,固宜有宗廟壇場祭祀之臧。今鼎出於廄東,中有刻書曰:『王命尸臣:「官此栒邑,賜爾旂鸞黼黻琱戈。」尸臣拜手稽首曰:「敢對揚天子丕顯休命。」』臣愚不足以跡古文,竊以傳記言之,此鼎殆周之所以褒賜大臣,大臣子孫刻銘其先功,臧之於宮廟也。昔寶鼎之出於汾脽也,河東太守以聞,詔曰:『朕巡祭后土,祈為百姓蒙豐年,今穀嗛未報,鼎焉為出哉?』博問耆老,意舊臧與?誠欲考得事實也。有司驗脽上非舊臧處,鼎大八尺一寸,高三尺六寸,殊異於眾鼎。今此鼎細小,又有款識,不宜薦見於宗廟。」制曰:「京兆尹議是。」

上自幸河東之明年正月,鳳皇集祋祤,於所集處得玉寶,起步壽宮,乃下詔赦天下。後間歲,鳳皇神爵甘露降集京師,赦天下。其冬,鳳皇集上林,乃作鳳皇殿,以答嘉瑞。明年正月,復幸甘泉,郊泰畤,改元曰五鳳。明年,幸雍祠五畤。其明年春,幸河東,祠后土,赦天下。後間歲,改元為甘露。正月,上幸甘泉,郊泰畤。其夏,黃龍見新豐。建章、未央、長樂宮鍾虡銅人皆生毛,長一寸所,時以為美祥。後間歲正月,上郊泰畤,因朝單于於甘泉宮。後間歲,改元為黃龍。正月,復幸甘泉,郊泰畤,又朝單于於甘泉宮。至冬而崩。鳳皇下郡國凡五十餘所。

元帝即位,遵舊儀,間歲正月,一幸甘泉郊泰畤,又東至河東祠后土,西至雍祠五畤。凡五奉泰畤、后土之祠。亦施恩澤,時所過毋出田租,賜百戶牛酒,或賜爵,赦罪人。

元帝好儒,貢禹、韋玄成、匡衡等相繼為公卿。禹建言漢家宗廟祭祀多不應古禮,上是其言。後韋玄成為丞相,議罷郡國廟,自太上皇、孝惠帝諸園寢廟皆罷。後元帝寢疾,夢神靈譴罷諸廟祠,上遂復焉。後或罷或復,至哀、平不定。語在韋玄成傳。

成帝初即位,丞相衡、御史大夫譚奏言:「帝王之事莫大乎承天之序,承天之序莫重於郊祀,故聖王盡心極慮以建其制。祭天於南郊,就陽之義也;瘞地於北郊,即陰之象也。天之於天子也,因其所都而各饗焉。往者,孝武皇帝居甘泉宮,即於雲陽立泰畤,祭於宮南。今行常幸長安,郊見皇天反北之泰陰,祠后土反東之少陽,事與古制殊。又至雲陽,行谿谷中,阨陝且百里,汾陰則渡大川,有風波舟楫之危,皆非聖主所宜數乘。郡縣治道共張,吏民困苦,百官煩費。勞所保之民,行危險之地,難以奉神靈而祈福祐,殆未合於承天子民之意。昔者周文武郊於豐鄗,成王郊於雒邑。由此觀之,天隨王者所居而饗之,可見也。甘泉泰畤、河東后土之祠宜可徙置長安,合於古帝王。願與群臣議定。」奏可。大司馬車騎將軍許嘉等八人以為所從來久遠,宜如故。右將軍王商、博士師丹、議郎翟方進等五十人以為禮記曰「燔柴於太壇,祭天也;瘞薶於大折,祭地也。」兆於南郊,所以定天位也。祭地於大折,在北郊,就陰位也。郊處各在聖王所都之南北。書曰「越三日丁巳,用牲於郊,牛二。」周公加牲,告徙新邑,定郊禮於雒。明王聖主,事天明,事地察。天地明察,神明章矣。天地以王者為主,故聖王制祭天地之禮必於國郊。長安,聖主之居,皇天所觀視也。甘泉、河東之祠非神靈所饗,宜徙就正陽大陰之處。違俗復古,循聖制,定天位,如禮便。於是衡、譚奏議曰:「陛下聖德,璴明上通,承天之大,典覽群下,使各悉心盡慮,議郊祀之處,天下幸甚。臣聞廣謀從眾,則合於天心,故洪範曰『三人占,則從二人言』,言少從多之義也。論當往古,宜於萬民,則依而從之;違道寡與,則廢而不行。今議者五十八人,其五十人言當徙之義,皆著於經傳,同於上世,便於吏民;八人不案經藝,考古制,而以為不宜,無法之議,難以定吉凶。太誓曰:『正稽古立功立事,可以永年,丕天之大律。』《詩》曰『毋曰高高在上,陟降厥士,日監在茲』,言天之日監王者之處也。又曰『乃眷西顧,此維予宅』,言天以文王之都為居也。宜於長安定南北郊,為萬世基。」天子從之。

既定,衡言:「甘泉泰畤紫壇,八觚宣通象八方。五帝壇周環其下,又有群神之壇。以尚書禋六宗、望山川、遍群神之義,紫壇有文章采鏤黼黻之飾及玉、女樂,石壇、僊人祠,瘞鸞路、騂駒、寓龍馬,不能得其象於古。臣聞郊紫壇饗帝之義,埽地而祭,上質也。歌大呂舞雲門以俟天神,歌太蔟舞咸池以俟地祇,其牲用犢,其席槁峵,其器陶匏,皆因天地之性,貴誠上質,不敢修其文也。以為神祇功德至大,雖修精微而備庶物,猶不足以報功,唯至誠為可,致上質不飾,以章天德。紫壇偽飾女樂、鸞路、騂駒、龍馬、石壇之屬,宜皆勿修。」

衡又言:「王者各以其禮制事天地,非因異世所立而繼之。今雍鄜、密、上下畤,本秦侯各以其意所立,非禮之所載術也。漢興之初,儀制未及定,即且因秦故祠,復立北畤。今既稽古,建定天地之大禮,郊見上帝,青赤白黃黑五方之帝皆畢陳,各有位饌,祭祀備具。諸侯所妄造,王者不當長遵。及北畤,未定時所立,不宜復修。」天子皆從焉。及陳寶祠,由是皆罷。

明年,上始祀南郊,赦奉郊之縣及中都官耐罪囚徒。是歲衡、譚復條奏:「長安廚官縣官給祠郡國候神方士使者所祠,凡六百八十三所,其二百八所應禮,及疑無明文,可奉祠如故。其餘四百七十五所不應禮,或復重,請皆罷。」奏可。本雍舊祠二百三所,唯山川諸星十五所為應禮云。若諸布、諸嚴、諸逐,皆罷。杜主有五祠,置其一。又罷高祖所立梁、晉、秦、荊巫、九天、南山、萊中之屬,及孝文渭陽、孝武薄忌泰一、三一、黃帝、冥羊、馬行、泰一、皋山山君、武夷、夏后啟母石、萬里沙、八神、延年之屬,及孝宣參山、蓬山、之罘、成山、萊山、四時、蚩尤、勞谷、五床、僊人、玉女、徑路、黃帝、天神、原水之屬,皆罷。候神方士使者副佐、本草待詔七十餘人皆歸家。

明年,匡衡坐事免官爵。眾庶多言不當變動祭祀者。又初罷甘泉泰畤作南郊日,大風壞甘泉竹宮,折拔畤中樹大十圍以上百餘。天子異之,以問劉向。對曰:「家人尚不欲絕種祠,況於國之神寶舊畤!且甘泉、汾陰及雍五畤始立,皆有神祇感應,然後營之,非苟而已也。武、宣之世,奉此三神,禮敬敕備,神光尤著。祖宗所立神祇舊位,誠未易動。及陳寶祠,自秦文公至今七百餘歲矣,漢興世世常來,光色赤黃,長四五丈,直祠而息,音聲砰隱,野雞皆雊。每見雍太祝祠以太牢,遣候者乘一乘傳馳詣行在所,以為福祥。高祖時五來,文帝二十六來,武帝七十五來,宣帝二十五來,初元元年以來亦二十來,此陽氣舊祠也。及漢宗廟之禮,不得擅議,皆祖宗之君與賢臣所共定。古今異制,經無明文,至尊至重,難以疑說正也。前始納貢禹之議,後人相因,多所動搖。易大傳曰:『

誣神者殃及三世。』恐其咎不獨止禹等。」上意恨之。

後上以無繼嗣故,今皇太后詔有司曰:「蓋聞王者承事天地,交接泰一,尊莫著於祭祀。孝武皇帝大聖通明,始建上下之祀,營泰畤於甘泉,定后土於汾陰,而神祇安之,饗國長久,子孫蕃滋,累世遵業,福流於今。今皇帝寬仁孝順,奉循聖緒,靡有大愆,而久無繼嗣。思其咎職,殆在徙南北郊,違先帝之制,改神祇舊位,失天地之心,以妨繼嗣之福。春秋六十,未見皇孫,食不甘味,寢不安席,朕甚悼焉。春秋大復古,善順祀。其復甘泉泰畤,汾陰后土如故,及雍五畤、陳寶祠在陳倉者。」天子復親郊禮如前。又復長安、雍及郡國祠著明者且半。

成帝末年頗好鬼神,亦以無繼嗣故,多上書言祭祀方術者,皆得待詔,祠祭上林苑中長安城旁,費用甚多,然無大貴盛者,谷永說上曰:「臣聞明於天地之性,不可或以神怪;知萬物之情,不可罔以非類。諸背仁義之正道,不遵五經之法言,而盛稱奇怪鬼神,廣崇祭祀之方,求報無福之祠,及言世有僊人,服食不終之藥,揽興輕舉,登遐倒景,覽觀縣圃,浮游蓬萊,耕耘五德,朝種暮穫,與山石無極,黃冶變化,堅冰淖溺,化色五倉之術者,皆姦人惑眾,挾左道,懷詐偽,以欺罔世主。聽其言,洋洋滿耳,若將可遇;求之,盪盪如係風捕景,終不可得。是以明王距而不聽,聖人絕而不語。昔周史萇弘欲以鬼神之術輔尊靈王會朝諸侯,而周室愈微,諸侯愈叛。楚懷王隆祭祀,事鬼神,欲以獲福助,卻秦師,而兵挫地削,身辱國危。秦始皇初并天下,甘心於神僊之道,遣徐福、韓終之屬多齎童男童女入海求神采藥,因逃不還,天下怨恨。漢興,新垣平、齊人少翁、公孫卿、欒大等,皆以僊人黃冶祭祠事鬼使物入海求神采藥貴幸,賞賜累千金。大尤尊盛,至妻公主,爵位重絫,震動海內。元鼎、元封之際,燕齊之間方士瞋目扼掔,言有神僊祭祀致福之術者以萬數。其後,平等皆以術窮詐得,誅夷伏辜。至初元中,有天淵玉女、鉅鹿神人、轑陽侯師張宗之姦,紛紛復起。夫周秦之末,三五之隆,已嘗專意散財,厚爵祿,竦精神,舉天下以求之矣。曠日經年,靡有毫氂之驗,足以揆今。經曰:『享多儀,儀不及物,惟曰不享。』論語說曰:『子不語怪神。』唯陛下距絕此類,毋令姦人有以窺朝者。」上善其言。

後成都侯王商為大司馬衛將軍輔政,杜鄴說商曰:「『東鄰殺牛,不如西鄰之瀹祭』,言奉天之道,貴以誠質大得民心也。行穢祀豐,猶不蒙祐;德修薦薄,吉必大來。古者壇場有常處,抠禋有常用,贊見有常禮;犧牲玉帛雖備而財不匱,車輿臣役雖動而用不勞。是故每奉其禮,助者歡說,大路所歷,黎元不知。今甘泉、河東天地郊祀,咸失方位,違陰陽之宜。及雍五畤皆曠遠,奉尊之役休而復起,繕治共張無解已時,皇天著象殆可略知。前上甘泉,先敺失道;禮月之夕,奉引復迷。祠后土還,臨河當渡,疾風起波,船不可御。又雍大雨,壞平陽宮垣。乃三月甲子,震電災林光宮門。祥瑞未著,咎徵仍臻。跡三郡所奏,皆有變故。不答不饗,何以甚此!《詩》曰『率由舊章』。舊章,先王法度,文王以之,交神于祀,子孫千億。宜如異時公卿之議,復還長安南北郊。」

後數年,成帝崩,皇太后詔有司曰:「皇帝即位,思順天心,遵經義,定郊禮,天下說憙。懼未有皇孫,故復甘泉泰畤、汾陰后土,庶幾獲福。皇帝恨難之,卒未得其祐。其復南北郊長安如故,以順皇帝之意也。」

哀帝即位,寢疾,博徵方術士,京師諸縣皆有侍祠使者,盡復前世所常興諸神祠官,凡七百餘所,一歲三萬七千祠云。

明年,復令太皇太后詔有司曰:「皇帝孝順,奉承聖業,靡有解怠,而久疾未瘳。夙夜唯思,殆繼體之君不宜改作。其復甘泉泰畤、汾陰后土祠如故。」上亦不能親至,遣有司行事而禮祠焉。後三年,哀帝崩。

平帝元始五年,大司馬王莽奏言:「王者父事天,故爵稱天子。孔子曰:『人之行莫大於孝,孝莫大於嚴父,嚴父莫大於配天。』王者尊其考,欲以配天,緣考之意,欲尊祖,推而上之,遂及始祖。是以周公郊祀后稷以配天,宗祀文王於明堂以配上帝。禮記天子祭天地及山川,歲遍。春秋穀梁傳以十二月下辛卜,正月上辛郊。高皇帝受命,因雍四畤起北畤,而備五帝,未共王地之祀。孝文十六年用新垣平,初起渭陽五帝廟,祭泰一、地祇,以太祖高皇帝配。日冬至祠泰一,夏至祠地祇,皆并祠五帝,而共一牲,上親郊拜。後平伏誅,乃不復自親,而使有司行事。孝武皇帝祠雍,曰:『今上帝朕親郊,而后土無祠,則禮不答也。』於是元鼎四年十一月甲子始立后土祠於汾陰。或曰,五帝,泰一之佐,宜立泰一。五年十一月癸未始立泰一祠於甘泉,二歲一郊,與雍更祠,亦以高祖配,不歲事天,皆未應古制。建始元年,徙甘泉泰畤、河東后土於長安南北郊。永始元年三月,以未有皇孫,復甘泉、河東祠。綏和二年,以卒不獲祐,復長安南北郊。建平三年,懼孝哀皇帝之疾未瘳,復甘泉、汾陰祠,竟復無福。臣謹與太師孔光、長樂少府平晏、大司農左咸、中壘校尉劉歆、太中大夫朱陽、博士薛順、議郎國由等六十七人議,皆曰宜如建始時丞相衡等議,復長安南北郊如故。」

莽又頗改其祭禮,曰:「周官天墬之祀,樂有別有合。其合樂曰『以六律、六鐘、五聲、八音、六舞大合樂』,祀天神,祭墬祇,祀四望,祭山川,享先妣先祖。凡六樂,奏六歌,而天墬神祇之物皆至。四望,蓋謂日月星海也。三光高而不可得親,海廣大無限界,故其樂同。祀天則天文從。祭墬則墬理從。三光,天文也。山川,地理也。天地合祭,先祖配天,先妣配墬,其誼一也。天墬合精,夫婦判合。祭天南郊,則以墬配,一體之誼也。天墬位皆南鄉,同席,墬在東,共牢而食。高帝、高后配於壇上,西鄉,后在北,亦同席共牢。牲用繭栗,玄酒陶匏。禮記曰天子籍田千具以事天墬,繇是言之,宜有黍稷。天地用牲一,燔抠瘞薶用牲一,高帝、高后用牲一。天用牲左,及黍稷燔抠南郊;墬用牲右,及黍稷瘞於北郊。其旦,東鄉再拜朝日;其夕,西鄉再拜夕月。然後孝弟之道備,而神衹嘉享,萬福降輯。此天墬合祀,以祖妣配者也。其別樂曰『冬日至,於墬上之圜丘奏樂六變,則天神皆降;夏日至,於澤中之方丘奏樂八變,則墬衹皆出。』天墬有常位,不得常合,此其各特祀者也。陰陽之別於日冬夏至,其會也以孟春正月上辛若丁。天子親合祀天墬於南郊,以高帝、高后配。陰陽有離合,《易》曰『分陰分陽,迭用柔剛』。以日冬至使有司奉祠南郊,高帝配而望群陽,日夏至使有司奉祭北郊,高后配而望群陰,皆以助致微氣,通道幽弱。當此之時,后不省方,故天子不親而遣有司,所以正承天順地,復聖王之制,顯太祖之功也。渭陽祠勿復修。群望未定悉定,定復奏。」奏可。三十餘年間,天地之祠五徙焉。

後莽又奏言:「書曰『類於上帝,禋于六宗』。歐陽、大小夏侯三家說六宗,皆曰上不及天,下不及墬,旁不及四方,在六者之間,助陰陽變化,實一而名六,名實不相應。禮記祀典,功施於民則祀之。天文日月星辰,所昭仰也;地理山川海澤,所生殖也。易有八卦,乾坤六子,水火不相逮,雷風不相誖,山澤通氣,然後能變化,既成萬物也。臣前奏徙甘泉泰畤、汾陰后土皆復於南北郊。謹案周官『兆五帝於四郊』,山川各因其方,今五帝兆居在雍五畤,不合於古。又日月雷風山澤,易卦六子之尊氣,所謂六宗也。星辰水火溝瀆,皆六宗之屬也。今或未特祀,或無兆居。謹與太師光、大司徒宮、羲和歆等八十九人議,皆曰天子父事天,母事墬,今稱天神曰皇天上帝,泰一兆曰泰畤,而稱地祇曰后土,與中央黃靈同,又兆北郊未有尊稱。宜令地祇稱皇墬后祇,兆曰廣畤。《易》曰『方以類聚,物以群分』。分群神以類相從為五部,兆天墬之別神:中央帝黃靈后土畤及日廟、北辰、北斗、填星、中宿中宮於長安城之未墬兆;東方帝太昊青靈勾芒畤及雷公、風伯廟、歲星、東宿東宮於東郊兆;南方炎帝赤靈祝融畤及熒惑星、南宿南宮於南郊兆;西方帝少皞白靈蓐收畤及太白星、西宿西宮於西郊兆;北方帝顓頊黑靈玄冥畤及月廟、雨師廟、辰星、北宿北宮於北郊兆。」奏可。於是長安旁諸廟兆畤甚盛矣。

莽又言:「帝王建立社稷,百王不易。社者,土也。宗廟,王者所居。稷者,百穀之王,所以奉宗廟,共粢盛,人所食以生活也。王者莫不尊重親祭,自為之主,禮如宗廟。《詩》曰『乃立冢土』。又曰『以御田祖,以祈甘雨』。《禮記》曰『唯祭宗廟社稷,為越紼而行事』。聖漢興,禮儀稍定,已有官社,未立官稷。」遂於官社後立官稷,以夏禹配食官社,后稷配食官稷。稷種穀樹。徐州牧歲貢五色土各一斗。

莽篡位二年,興神僊事,以方士蘇樂言,起八風臺於宮中。臺成萬金,作樂其上,順風作液湯。又種五粱禾於殿中,各順色置其方面,先煮鶴齔、毒冒、犀玉二十餘物漬種,計粟斛成一金,言此黃帝穀僊之術也。以樂為黃門郎,令主之。莽遂崇鬼神淫祀,至其末年,自天地六宗以下至諸小鬼神,凡千七百所,用三牲鳥獸三千餘種。後不能備,乃以雞當鶩鴈,犬當麋鹿。數下詔自以當僊,語在其傳。

贊曰:漢興之初,庶事草創,唯一叔孫生略定朝廷之儀。若乃正朔服色郊望之事,數世猶未章焉。至於孝文,始以夏郊,而張倉據水德,公孫臣、賈誼更以為土德,卒不能明。孝武之世,文章為盛,太初改制,而兒寬、司馬遷等猶從臣、誼之言,服色數度,遂順黃德。彼以五德之傳從所不勝,秦在水德,故謂漢據土而克之。劉向父子以為帝出於震,故包羲氏始受木德,其後以母傳子,終而復始,自神農、黃帝下歷唐虞三代而漢得火焉。故高祖始起,神母夜號,著赤帝之符,旗章遂赤,自得天統矣。昔共工氏以水德間於木火,與秦同運,非其次序,故皆不永。由是言之,祖宗之制蓋有自然之應,順時宜矣。究觀方士祠官之變,谷永之言,不亦正乎!不亦正乎!


\end{pinyinscope}