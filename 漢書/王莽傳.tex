\article{王莽傳}

\begin{pinyinscope}
王莽字巨君,孝元皇后之弟子也。元后父及兄弟皆以元、成世封侯,居位輔政,家凡九侯、五大司馬,語在元后傳。唯莽父曼蚤死,不侯。莽群兄弟皆將軍五侯子,乘時侈靡,以輿馬聲色佚游相高,莽獨孤貧,因折節為恭儉。受禮經,師事沛郡陳參,勤身博學,被服如儒生。事母及寡嫂,養孤兄子,行甚敕備。又外交英俊,內事諸父,曲有禮意。陽朔中,世父大將軍鳳病,莽侍疾,親嘗藥,亂首垢面,不解衣帶連月。鳳且死,以託太后及帝,拜為黃門郎,遷射聲校尉。

久之,叔父成都侯商上書,願分戶邑以封莽,及長樂少府戴崇、侍中金涉、胡騎校尉箕閎、上谷都尉陽並、中郎陳湯,皆當世名士,咸為莽言,上由是賢莽。永始元年,封莽為新都侯,國南陽新野之都鄉,千五百戶。遷騎都尉光祿大夫侍中,宿衛謹敕,爵位益尊,節操愈謙。散輿馬衣裘,振施賓客,家無所餘。收贍名士,交結將相卿大夫甚眾。故在位更推薦之,游者為之談說,虛譽隆洽,傾其諸父矣。敢為激發之行,處之不慚恧。

莽兄永為諸曹,蚤死,有子光,莽使學博士門下。莽休沐出,振車騎,奉羊酒,勞遺其師,恩施下竟同學。諸生縱觀,長老嘆息。光年小於莽子宇,莽使同日內婦,賓客滿堂。須臾,一人言太夫人苦某痛,當飲某藥,比客罷者數起焉。為私買侍婢,昆弟或頗聞知,莽因曰:「後將軍朱子元無子,莽聞此兒種宜子,為買之。」即日以婢奉子元。其匿情求名如此。

是時,太后姊子淳于長以材能為九卿,先進在莽右。莽陰求其罪過,因大司馬曲陽侯根白之,長伏誅,莽以獲忠直,語在長傳。根因乞骸骨,薦莽自代,上遂擢為大司馬。是歲,綏和元年也,年三十八矣。莽既拔出同列,繼四父而輔政,欲令名譽過前人,遂克己不倦,聘諸賢良以為掾史,賞賜邑錢悉以享士,愈為儉約。母病,公卿列侯遣夫人問疾,莽妻迎之,衣不曳地,布蔽膝。見之者以為僮使,問知其夫人,皆驚。

輔政歲餘,成帝崩,哀帝即位,尊皇太后為太皇太后。太后詔莽就第,避帝外家。莽上疏乞骸骨,哀帝遣尚書令詔莽曰:「先帝委政於君而棄群臣,朕得奉宗廟,誠嘉與君同心合意。今君移病求退,以著朕之不能奉順先帝之意,朕甚悲傷焉。已詔尚書待君奏事。」又遣丞相孔光、大司空何武、左將軍師丹、衛尉傅喜白太后曰:「皇帝聞太后詔,甚悲。大司馬即不起,皇帝即不敢聽政。」太后復令莽視事。

時哀帝祖母定陶傅太后、母丁姬在,高昌侯董宏上書言:「春秋之義,母以子貴,丁姬宜上尊號。」莽與師丹共劾宏誤朝不道,語在丹傳。後日,未央宮置酒,內者令為傅太后張幄,坐於太皇太后坐旁。莽案行,責內者令曰:「定陶太后藩妾,何以得與至尊並!」徹去,更設坐。傅太后聞之,大怒,不肯會,重怨恚莽。莽復乞骸骨,哀帝賜莽黃金五百斤,安車駟馬,罷就第。公卿大夫多稱之者,上乃加恩寵,置使家,中黃門十日一賜餐。下詔曰:「

新都侯莽憂勞國家,執義堅固,朕庶幾與為治。太皇太后詔莽就第,朕甚閔焉。其以黃郵聚戶三百五十益封莽,位特進,給事中,朝朔望見禮如三公,車駕乘綠車從。」後二歲,傅太后、丁姬皆稱尊號,丞相朱博奏:「莽前不廣尊尊之義,抑貶尊號,虧損孝道,當伏顯戮,幸蒙赦令,不宜有爵土,請免為庶人。」上曰:「以莽與太皇太后有屬,勿免,遣就國。」

莽杜門自守,其中子獲殺奴,莽切責獲,令自殺。在國三歲,吏上書冤訟莽者以百數。元壽元年,日食,賢良周護、宋崇等對策深頌莽功德,上於是徵莽。

始莽就國,南陽太守以莽貴重,選門下掾宛孔休守新都相。休謁見莽,莽盡禮自納,休亦聞其名,與相答。後莽疾,休候之,莽緣恩意,進其玉具寶劍,欲以為好。休不肯受,莽因曰:「

誠見君面有瘢,美玉可以滅瘢,欲獻其瑑耳。」即解其瑑,休復辭讓。莽曰:「君嫌其賈邪?」遂椎碎之,自裹以進休,休乃受。及莽徵去,欲見休,休稱疾不見。

莽還京師歲餘,哀帝崩,無子,而傅太后、丁太后皆先薨,太皇太后即日駕之未央宮收取璽綬,遣使者馳召莽。詔尚書,諸發兵符節,百官奏事,中黃門、期門兵皆屬莽。莽白:「大司馬高安侯董賢年少,不合眾心,收印綬。」賢即日自殺。太后詔公卿舉可大司馬者,大司徒孔光,大司空彭宣舉莽,前將軍何武、後將軍公孫祿互相舉。太后拜莽為大司馬,與議立嗣。安陽侯王舜莽之從弟,其人修飭,太后所信愛也,莽白以舜為車騎將軍,使迎中山王奉成帝後,是為孝平皇帝。帝年九歲,太后臨朝稱制,委政於莽。莽白趙氏前害皇子,傅氏驕僭,遂廢孝成趙皇后、孝哀傅皇后,皆令自殺,語在外戚傳。

莽以大司徒孔光名儒,相三主,太后所敬,天下信之,於是盛尊事光,引光女婿甄邯為侍中奉車都尉。諸哀帝外戚及大臣居位素所不說者,莽皆傅致其罪,為請奏,令邯持與光。光素畏慎,不敢不上之,莽白太后,輒可其奏。於是前將軍何武、後將軍公孫祿坐互相舉免,丁、傅及董賢親屬皆免官爵,徙遠方。紅陽侯立太后親弟,雖不居位,莽以諸父內敬憚之,畏立從容言太后,令己不得肆意,乃復令光奏立舊惡:「前知定陵侯淳于長犯大逆罪,多受其賂,為言誤朝;後白以官婢楊寄私子為皇子,眾言曰呂氏、少帝復出,紛紛為天下所疑,難以示來世,成襁褓之功。請遣立就國。」太后不聽。莽曰:「今漢家衰,比世無嗣,太后獨代幼主統政,誠可畏懼,力用公正先天下,尚恐不從,今以私恩逆大臣議如此,群下傾邪,亂從此起!宜可且遣就國,安後復徵召之。」太后不得已,遣立就國。莽之所以脅持上下,皆此類也。

於是附順者拔擢,忤恨者誅滅。王舜、王邑為腹心,甄豐、甄邯主擊斷,平晏領機事,劉歆典文章,孫建為爪牙。豐子尋、歆子棻、涿郡崔發、南陽陳崇皆以材能幸於莽。莽色厲而言方,欲有所為,微見風采,黨與承其指意而顯奏之,莽稽首涕泣,固推讓焉,上以惑太后,下用示信於眾庶。

始,風益州令塞外蠻夷獻白雉,元始元年正月,莽白太后下詔,以白雉薦宗廟。群臣因奏言太后「委任大司馬莽定策安宗廟。故大司馬霍光有安宗廟之功,益封三萬戶,疇其爵邑,比蕭相國。莽宜如光故事。」太后問公卿曰:「誠以大司馬有大功當著之邪?將以骨肉故欲異之也?」於是群臣乃盛陳「莽功德致周成白雉之瑞,千載同符。聖王之法,臣有大功則生有美號,故周公及身在而託號於周。莽有定國安漢家之大功,宜賜號曰安漢公,益戶,疇爵邑,上應古制,下準行事,以順天心。」太后詔尚書具其事。

莽上書言:「臣與孔光、王舜、甄豐、甄邯共定策,今願獨條光等功賞,寑置臣莽,勿隨輩列。」甄邯白太后下詔曰:「『無偏無黨,王道蕩蕩。』屬有親者,義不得阿。君有安宗廟之功,不可以骨肉故蔽隱不揚。君其勿辭。」莽復上書讓。太后詔謁者引莽待殿東箱,莽稱疾不肯入。太后使尚書令恂詔之曰:「君以選故而辭以疾,君任重,不可闕,以時亟起。」莽遂固辭。太后復使長信太僕閎承制召莽,莽固稱疾。左右白太后,宜勿奪莽意,但條孔光等,莽乃肯起。太后下詔曰:「太傅博山侯光宿衛四世,世為傅相,忠孝仁篤,行義顯著,建議定策,益封萬戶,以光為太師,與四輔之政。車騎將軍安陽侯舜積累仁孝,使迎中山王,折衝萬里,功德茂著,益封萬戶,以舜為太保。左將軍光祿勳豐宿衛三世,忠信仁篤,使迎中山王,輔導共養,以安宗廟,封豐為廣陽侯,食邑五千戶,以豐為少傅。皆授四輔之職,疇其爵邑,各賜第一區。侍中奉車都尉邯宿衛勤勞,建議定策,封邯為承陽侯,食邑二千四百戶。」四人既受賞,莽尚未起,群臣復上言:「莽雖克讓,朝所宜章,以時加賞,明重元功,無使百僚元元失望。」太后乃下詔曰:「大司馬新都侯莽三世為三公,典周公之職,建萬世策,功

能為忠臣宗,化流海內,遠人慕義,越裳氏重譯獻白雉。其以召陵、新息二縣戶二萬八千益封莽,復其後嗣,疇其爵邑,封功如蕭相國。以莽為太傅,幹四輔之事,號曰安漢公。以故蕭相國甲第為安漢公第,定著於令,傳之無窮。」

於是莽為惶恐,不得已而起受策。策曰:「漢危無嗣,而公定之;四輔之職,三公之任,而公幹之;群僚眾位,而公宰之:功德茂著,宗廟以安,蓋白雉之瑞,周成象焉。故賜嘉號曰安漢公,輔翼于帝,期於致平,毋違朕意。」莽受太傅安漢公號,讓還益封疇爵邑事,云願須百姓家給,然後加賞。群公復爭,太后詔曰:「公自期百姓家給,是以聽之。其令公奉、舍人、賞賜皆倍故。百姓家給人足,大司徒、大司空以聞。」莽復讓不受,而建言宜立諸侯王後及高祖以來功臣子孫,大者封侯,或賜爵關內侯食邑,然後及諸在位,各有第序。上尊宗廟,增加禮樂;下惠士民鰥寡,恩澤之政無所不施。語在平紀。

莽既說眾庶,又欲專斷,知太后猒政,乃風公卿奏言:「往者,吏以功次遷至二千石,及州部所舉茂材異等吏,率多不稱,宜皆見安漢公。又太后不宜親省小事。」令太后下詔曰:「皇帝幼年,朕且統政,比加元服。今眾事煩碎,朕春秋高,精氣不堪,殆非所以安躬體而育養皇帝者也。故選忠賢,立四輔,群下勸職,永以康寧。孔子曰:『巍巍乎,舜禹之有天下而不與焉!』自今以來,非封爵乃以聞。他事,安漢公、四輔平決。州牧、二千石及茂材吏初除奏事者,輒引入至近署對安漢公,考故官,問新職,以知其稱否。」於是莽人人延問,致密恩意,厚加贈送,其不合指,顯奏免之,權與人主侔矣。

莽欲以虛名說太后,白言「親承前孝哀丁、傅奢侈之後,百姓未贍者多,太后宜且衣繒練,頗損膳,以視天下。」莽因上書,願出錢百萬,獻田三十頃,付大司農助給貧民。於是公卿皆慕效焉。莽帥群臣奏言:「陛下春秋尊,久衣重練,減御膳,誠非所以輔精氣,育皇帝,安宗廟也。臣莽數叩頭省戶下,白爭未見許。今幸賴陛下德澤,間者風雨時,甘露降,神芝生,蓂莢、朱草、嘉禾,休徵同時並至。臣莽等不勝大願,願陛下愛精休神,闊略思慮,遵帝王之常服,復太官之法膳,使臣子各得盡驩心,備共養。惟哀省察!」莽又令太后下詔曰:「蓋聞母后之義,思不出乎門閾。國不蒙佑,皇帝年在襁褓,未任親政,戰戰兢兢,懼於宗廟之不安。國家之大綱,微朕孰當統之?是以孔子見南子,周公居攝,蓋權時也。勤身極思,憂勞未綏,故國奢則視之以儉,矯枉者過其正,而朕不身帥,將謂天下何!夙夜夢想,五穀豐孰,百姓家給,比皇帝加元服,委政而授焉。今誠未皇于輕靡而備味,庶幾與百僚有成,其勗之哉!」每有水旱,莽輒素食,左右以白。太后遣使者詔莽曰:「聞公菜食,憂民深矣。今秋幸孰,公勤於職,以時食肉,愛身為國。」

莽念中國已平,唯四夷未有異,乃遣使者齎黃金幣帛,重賂匈奴單于,使上書言:「聞中國譏二名,故名囊知牙斯今更名知,慕從聖制。」又遣王昭君女須卜居次入侍。所以誑耀媚事太后,下至旁側長御,方故萬端。

莽既尊重,欲以女配帝為皇后,以固其權,奏言:「皇帝即位三年,長秋宮未建,液廷媵未充。乃者,國家之難,本從亡嗣,配取不正。請考論五經,定取禮,正十二女之義,以廣繼嗣。博采二王後及周公孔子世列侯在長安者適子女。」事下有司,上眾女名,王氏女多在選中者。莽恐其與己女爭,即上言:「身亡德,子材下,不宜與眾女並采。」太后以為至誠,乃下詔曰:「王氏女,朕之外家,其勿采。」庶民、諸生、郎吏以上守闕上書者日千餘人,公卿大夫或詣廷中,或伏省戶下,咸言:「明詔聖德巍巍如彼,安漢公盛勳堂堂若此,今當立后,獨奈何廢公女?天下安所歸命!願得公女為天下母。」莽遣長史以下分部曉止公卿及諸生,而上書者愈甚。太后不得已,聽公卿采莽女。莽復自白:「宜博選眾女。」公卿爭曰:「不宜采諸女以貳正統。」莽白:「願見女。」太后遣長樂少府、宗正、尚書令納采見女,還奏言:「公女漸漬德化,有窈窕之容,宜承大序,奉祭祀。」有詔遣大司徒、大司空策告宗廟,雜加卜筮,皆曰:「兆遇金水王相,卦遇父母得位,所謂『康強』之占,『逢吉』之符也。」信鄉侯佟上言:「春秋,天子將娶於紀,則褒紀子稱侯,安漢公國未稱古制。」事下有司,皆白:「古者天子封后父百里,尊而不臣,以重宗廟,孝之至也。佟言應禮,可許。請以新野田二萬五千六百頃益封莽,滿百里。」莽謝曰:「臣莽子女誠不足以配至尊,復聽眾議,益封臣莽。伏自惟念,得託肺腑,獲爵土,如使子女誠能奉稱聖德,臣莽國邑足以共朝貢,不須復加益地之寵。願歸所益。」太后許之。有司奏「故事,聘皇后黃金二萬斤,為錢二萬萬。」莽深辭讓,受四千萬,而以其三千三百萬予十一媵家。群臣復言:「今皇后受聘,踰群妾亡幾。」有詔,復益二千三百萬,合為三千萬。莽復以其千萬分予九族貧者。

陳崇時為大司徒司直,與張敞孫竦相善。竦者博通士,為崇草奏,稱莽功德,崇奏之,曰:

竊見安漢公自初束脩,值世俗隆奢麗之時,蒙兩宮厚骨肉之寵,被諸父赫赫之光,財饒勢足,亡所啎意,然而折節行仁,克心履禮,拂世矯俗,確然特立;惡衣惡食,陋車駑馬,妃匹無二,閨門之內,孝友之德,眾莫不聞;清靜樂道,溫良下士,惠于故舊,篤于師友。孔子曰「未若貧而樂,富而好禮」,公之謂矣。

及為侍中,故定陵侯淳于長有大逆罪,公不敢私,建白誅討。周公誅管蔡,季子鴆叔牙,公之謂矣。

是以孝成皇帝命公大司馬,委以國統。孝哀即位,高昌侯董宏希指求美,造作二統,公手劾之,以定大綱。建白定陶太后不宜在乘輿幄坐,以明國體。《詩》曰「柔亦不茹,剛亦不吐,不侮鰥寡,不畏強圉」,公之謂矣。

深執謙退,推誠讓位。定陶太后欲立僭號,憚彼面刺幄坐之義,佞惑之雄,朱博之疇,懲此長、宏手劾之事,上下壹心,讒賊交亂,詭辟制度,遂成篡號,斥逐仁賢,誅殘戚屬,而公被胥、原之訴,遠去就國,朝政崩壞,綱紀廢弛,危亡之禍,不隧如髮。《詩》云「人之云亡,邦國殄悴」,公之謂矣。

當此之時,宮亡儲主,董賢據重,加以傅氏有女之援,皆自知得罪天下,結讎中山,則必同憂,斷金相翼,藉假遺詔,頻用賞誅,先除所憚,急引所附,遂誣往冤,更徵遠屬,事勢張見,其不難矣!賴公立入,即時退賢,及其黨親。當此之時,公運獨見之明,奮亡前之威,盱衡厲色,振揚武怒,乘其未堅,厭其未發,震起機動,敵人摧折,雖有賁育不及持刺,雖有樗里不及回知,雖有鬼谷不及造次,是故董賢喪其魂魄,遂自絞殺。人不還踵,日不移晷,霍然四除,更為寧朝。非陛下莫引立公,非公莫克此禍。《詩》云「惟師尚父,時惟鷹揚,亮彼武王」,孔子曰「敏則有功」,公之謂矣。

於是公乃白內故泗水相豐、斄令邯,與大司徒光、車騎將軍舜建定社稷,奉節東迎,皆以功德受封益土,為國名臣。書曰「知人則哲」,公之謂也。

公卿咸歎公德,同盛公勳,皆以周公為比,宜賜號安漢公,益封二縣,公皆不受。傳曰申包胥不受存楚之報,晏平仲不受輔齊之封,孔子曰「能以禮讓為國乎?何有?」,公之謂也。

將為皇帝定立妃后,有司上名,公女為首,公深辭讓,迫不得已然後受詔。父子之親天性自然,欲其榮貴甚於為身,皇后之尊侔於天子,當時之會千載希有,然而公惟國家之統,揖大福之恩,事事謙退,動而固辭。書曰「舜讓于德不嗣」,公之謂矣。

自公受策,以至于今,斖斖翼翼,日新其德,增修雅素以命下國,悬儉隆約以矯世俗,割財損家以帥群下,彌躬執平以逮公卿,教子尊學以隆國化。僮奴衣布,馬不秣穀,食飲之用,不過凡庶。《詩》云「溫溫恭人,如集于木」,孔子曰「食無求飽,居無求安」,公之謂矣。

克身自約,糴食逮給,物物卬市,日闋亡儲。又上書歸孝哀皇帝所益封邑,入錢獻田,殫盡舊業,為眾倡始。於是小大鄉和,承風從化,外則王公列侯,內則帷幄侍御,翕然同時,各竭所有,或入金錢,或獻田畝,以振貧窮,收贍不足者。昔令尹子文朝不及夕,魯公儀子不茹園葵,公之謂矣。

開門延士,下及白屋,婁省朝政,綜管眾治,親見牧守以下,考跡雅素,審知白黑。《詩》云「夙夜匪解,以事一人」,《易》曰「終日乾乾,夕惕若厲」,公之謂矣。

比三世為三公,再奉送大行,秉冢宰職,填安國家,四海輻奏,靡不得所。書曰「納于大麓,列風雷雨不迷」,公之謂矣。

此皆上世之所鮮,禹稷之所難,而公包其終始,一以貫之,可謂備矣!是以三年之間,化行如神,嘉瑞疊累,豈非陛下知人之效,得賢之致哉!故非獨君之受命也,臣之生亦不虛矣。是以伯禹錫玄圭,周公受郊祀,蓋以達天之使,不敢擅天之功也。揆公德行,為天下紀;觀公功勳,為萬世基。基成而賞不配,紀立而褒不副,誠非所以厚國家,順天心也。

高皇帝褒賞元功,相國蕭何邑戶既倍,又蒙殊禮,奏事不名,入殿不趨,封其親屬十有餘人。樂善無厭,班賞亡遴,苟有一策,即必爵之,是故公孫戎位在充郎,選繇旄頭,壹明樊噲,封二千戶。孝文皇帝褒賞絳侯,益封萬戶,賜黃金五千斤。孝武皇帝卹錄軍功,裂三萬戶以封衛青,青子三人,或在繈褓,皆為通侯。孝宣皇帝顯著霍光,增戶命疇,封者三人,延及兄孫。夫絳侯即因漢藩之固,杖朱虛之鯁,依諸將之遞,據相扶之勢,其事雖醜,要不能遂。霍光即席常任之重,乘大勝之威,未嘗遭時不行,陷假離朝,朝之執事,亡非同類,割斷歷久,統政曠世,雖曰有功,所因亦易,然猶有計策不審過徵之累。及至青、戎,摽末之功,一言之勞,然猶皆蒙丘山之賞。課功絳、霍,造之與因也;比於青、戎,地之與天也。而公又有宰治之效,乃當上與伯禹、周公等盛齊隆,兼其褒賞,豈特與若云者同日而論哉?然曾不得蒙青等之厚,臣誠惑之!

臣聞功亡原者賞不限,德亡首者褒不檢。是故成王之與周公也,度百里之限,越九錫之檢,開七百里之宇,兼商、奄之民,賜以附庸殷民六族,大路大旂,封父之繁弱,夏后之璜,祝宗卜史,備物典策,官司彝器,白牡之牲,郊望之禮。王曰:「叔父,建爾元子。」子父俱延拜而受之。可謂不檢亡原者矣。非特止此,六子皆封。《詩》曰:「亡言不讎,亡德不報。」報當如之,不如非報也。近觀行事,高祖之約非劉氏不王,然而番君得王長沙,下詔稱忠,定著於令,明有大信不拘於制也。春秋晉悼公用魏絳之策,諸夏服從。鄭伯獻樂,悼公於是以半賜之。絳深辭讓,晉侯曰:「微子,寡人不能濟河。夫賞,國之典,不可廢也。子其受之。」魏絳於是有金石之樂,春秋善之,取其臣竭忠以辭功,君知臣以遂賞也。今陛下既知公有周公功德,不行成王之褒賞,遂聽公之固辭,不顧春秋之明義,則民臣何稱,萬世何述?誠非所以為國也。臣愚以為宜恢公國,令如周公,建立公子,令如伯禽。所賜之品,亦皆如之。諸子之封,皆如六子。即群下較然輸忠,黎庶昭然感德。臣誠輸忠,民誠感德,則於王事何有?唯陛下深惟祖宗之重,敬畏上天之戒,儀形虞、周之盛,敕盡伯禽之賜,無遴周公之報,今天法有設,後世有祖,天下幸甚!

太后以視群公,群公方議其事,會呂寬事起。

初,莽欲擅權,白太后:「前哀帝立,背恩義,自貴外家丁、傅,撓亂國家,幾危社稷。今帝以幼年復奉大宗,為成帝後,宜明一統之義,以戒前事,為後代法。」於是遣甄豐奉璽綬,即拜帝母衛姬為中山孝王后,賜帝舅衛寶、寶弟玄爵關內侯,皆留中山,不得至京師。莽子宇,非莽隔絕衛氏,恐帝長大後見怨。宇即私遣人與寶等通書,教令帝母上書求入。語在衛后傳。莽不聽。宇與師吳章及婦兄呂寬議其故,章以為莽不可諫,而好鬼神,可為變怪以驚懼之,章因推類說令歸政於衛氏。宇即使寬夜持血灑莽第,門吏發覺之,莽執宇送獄,飲藥死。宇妻焉懷子,繫獄,須產子已,殺之。莽奏言:「宇為呂寬等所詿誤,流言惑眾,惡與管蔡同罪,臣不敢隱,其誅。」甄邯等白太后下詔曰:「夫唐堯有丹朱,周文王有管蔡,此皆上聖亡奈下愚子何,以其性不可移也。公居周公之位,輔成王之主,而行管蔡之誅,不以親親害尊尊,朕甚嘉之。昔周公誅四國之後,大化乃成,至於刑錯。公其專意翼國,期於致平。」莽因是誅滅衛氏,窮治呂寬之獄,連引郡國豪桀素非議己者,內及敬武公主、梁王立、紅陽侯立、平阿侯仁,使者迫守,皆自殺。死者以百數,海內震焉。大司馬護軍褒奏言:「安漢公遭子宇陷於管蔡之辜,子愛至深,為帝室故不敢顧私。惟宇遭罪,喟然憤發作書八篇,以戒子孫。宜班郡國,令學官以教授。」事下群公,請令天下吏能誦公戒者,以著官簿,比孝經。

四年春,郊祀高祖以配天,宗祀孝文皇帝以配上帝。四月丁未,莽女立為皇后,大赦天下。遣大司徒司直陳崇等八人分行天下,覽觀風俗。

太保舜等奏言:「春秋列功德之義,太上有立德,其次有立功,其次有立言,唯至德大賢然後能之。其在人臣,則生有大賞,終為宗臣,殷之伊尹,周之周公是也。」及民上書者八千餘人,咸曰:「伊尹為阿衡,周公為太宰,周公享七子之封,有過上公之賞。宜如陳崇言。」章下有司,有司請「還前所益二縣及黃郵聚、新野田,采伊尹、周公稱號,加公為宰衡,位上公。掾史秩六百石。三公言事,稱『

敢言之』。群吏毋得與公同名。出從期門二十人,羽林三十人,前後大車十乘。賜公太夫人號曰功顯君,食邑二千戶,黃金印赤韍。封公子男二人,安為褒新侯,臨為賞都侯。加后聘三千七百萬,合為一萬萬,以明大禮。」太后臨前殿,親封拜。安漢公拜前,二子拜後,如周公故事。莽稽首辭讓,出奏封事,願獨受母號,還安、臨印韍及號位戶邑。事下太師光等,皆曰:「賞未足以直功,謙約退讓,公之常節,終不可聽。」莽求見固讓。太后下詔曰:「公每見,叩頭流涕固辭,今移病,固當聽其讓,令視事邪?將當遂行其賞,遣歸就第也?」光等曰:「安、臨親受印韍,策號通天,其義昭昭。黃郵、召陵、新野之田為入尤多,皆止於公,公欲自損以成國化,宜可聽許。治平之化當以時成,宰衡之官不可世及。納徵錢,乃以尊皇后,非為公也。功顯君戶,止身不傳。褒新、賞都兩國合三千戶,甚少矣。忠臣之節,亦宜自屈,而信主上之義。宜遣大司徒、大司空持節承制,詔公亟入視事。詔尚書勿復受公之讓奏。」奏可。

莽乃起視事,上書言:「臣以元壽二年六月戊午倉卒之夜,以新都侯引入未央宮;庚申拜為大司馬,充三公位;元始元年正月丙辰拜為太傅,賜號安漢公,備四輔官;今年四月甲子復拜為宰衡,位上公。臣莽伏自惟,爵為新都侯,號為安漢公,官為宰衡、太傅、大司馬,爵貴號尊官重,一身蒙大寵者五,誠非鄙臣所能堪。據元始三年,天下歲已復,官屬宜皆置。穀梁傳曰:『天子之宰,通于四海。』臣愚以為,宰衡官以正百僚平海內為職,而無印信,名實不副。臣莽無兼官之材,今聖朝既過誤而用之,臣請御史刻宰衡印章曰『宰衡太傅大司馬印』,成,授臣莽,上太傅與大司馬之印。」太后詔曰:「可。韍如相國,朕親臨授焉。」莽乃復以所益納徵錢千萬,遺與長樂長御奉共養者。太保舜奏言:「天下聞公不受千乘之土,辭萬金之幣,散財施予千萬數,莫不鄉化。蜀郡男子路建等輟訟慚怍而退,雖文王卻虞芮何以加!宜報告天下。」奏可。宰衡出,從大車前後各十乘,直事尚書郎、侍御史、謁者、中黃門、期門羽林。宰衡常持節,所止,謁者代持之。宰衡掾史秩六百石,三公稱「敢言之」。

是歲,莽奏起明堂、辟雍、靈臺,為學者築舍萬區,作市、常滿倉,制度甚盛。立樂經,益博士員,經各五人。徵天下通一藝教授十一人以上,及有逸禮、古書、毛詩、周官、爾雅、天文、圖讖、鍾律、月令、兵法、史篇文字,通知其意者,皆詣公車。網羅天下異能之士,至者前後千數,皆令記說廷中,將令正乖繆,壹異說云。群臣奏言:「昔周公奉繼體之嗣,據上公之尊,然猶七年制度乃定。夫明堂、辟雍,墮廢千載莫能興,今安漢公起于第家,輔翼陛下,四年于茲,功德爛然。公以八月載生魄庚子奉使,朝用書臨賦營築,越若翊辛丑,諸生、庶民大和會,十萬眾並集,平作二旬,大功畢成。唐虞發舉,成周造業,誠亡以加。宰衡位宜在諸侯王上,賜以束帛加璧,大國乘車、安車各一,驪馬二駟。」詔曰:「可。其議九錫之法。」

冬,大風吹長安城東門屋瓦且盡。

五年正月,祫祭明堂,諸侯王二十八人,列侯百二十人,宗室子九百餘人,徵助祭。禮畢,封孝宣曾孫信等三十六人為列侯,餘皆益戶賜爵,金帛之賞各有數。是時,吏民以莽不受新野田而上書者前後四十八萬七千五百七十二人,及諸侯、王公、列侯、宗室見者皆叩頭言,宜亟加賞於安漢公。於是莽上書曰:「臣以外屬,越次備位,未能奉稱。伏念聖德純茂,承天當古,制禮以治民,作樂以移風,四海奔走,百蠻並轃,辭去之日,莫不隕涕。非有款誠,豈可虛致?自諸侯王已下至於吏民,咸知臣莽上與陛下有葭莩之故,又得典職,每歸功列德者,輒以臣莽為餘言。臣見諸侯面言事於前者,未嘗不流汗而慚愧也。雖性愚鄙,至誠自知,德薄位尊,力少任大,夙夜悼栗,常恐污辱聖朝。今天下治平,風俗齊同,百蠻率服,皆陛下聖德所自躬親,太師光、太保舜等輔政佐治,群卿大夫莫不忠良,故能以五年之間至致此焉。臣莽實無奇策異謀。奉承太后聖詔,宣之于下,不能得什一;受群賢之籌畫,而上以聞,不能得什伍。當被無益之辜,所以敢且保首領須臾者,誠上休陛下餘光,而下依群公之故也。陛下不忍眾言,輒下其章於議者。臣莽前欲立奏止,恐其遂不肯止。今大禮已行,助祭者畢辭,不勝至願,願諸章下議者皆寢勿上,使臣莽得盡力畢制禮作樂事。事成,以傳示天下,與海內平之。即有所間非,則臣莽當被詿上誤朝之罪;如無他譴,得全命賜骸骨歸家,避賢者路,是臣之私願也。惟陛下哀憐財幸!」甄邯等白太后,詔曰:「可。唯公功德光於天下,是以諸侯、王公、列侯、宗室、諸生、吏民翕然同辭,連守闕庭,故下其章。諸侯、宗室辭去之日,復見前重陳,雖曉喻罷遣,猶不肯去。告以孟夏將行厥賞,莫不驩悅,稱萬歲而退。今公每見,輒流涕叩頭言願不受賞,賞即加不敢當位。方制作未定,事須公而決,故且聽公。制作畢成,群公以聞。究于前議,其九錫禮儀亟奏。」

於是公卿大夫、博士、議郎、列侯富平侯張純等九百二人皆曰:「聖帝明王招賢勸能,德盛者位高,功大者賞厚。故宗臣有九命上公之尊,則有九錫登等之寵。今九族親睦,百姓既章,萬國和協,黎民時雍,聖瑞畢溱,太平已洽。帝者之盛莫隆於唐虞,而陛下任之;忠臣茂功莫著於伊周,而宰衡配之。所謂異時而興,如合符者也。謹以六藝通義,經文所見,周官、禮記宜於今者,為九命之錫。臣請命錫。」奏可。策曰:

惟元始五年五月庚寅,太皇太后臨于前殿,延登,請詔之曰:公進,虛聽朕言。前公宿衛孝成皇帝十有六年,納策盡忠,白誅故定陵侯淳于長,以彌亂發姦,登大司馬,職在內輔。孝哀皇帝即位,驕妾窺欲,姦臣萌亂,公手劾高昌侯董宏,改正故定陶共王母之僭坐。自是之後,朝臣論議,靡不據經。以病辭位,歸于第家,為賊臣所陷。就國之後,孝哀皇帝覺寤,復還公長安,臨病加劇,猶不忘公,復特進位。是夜倉卒,國無儲主,姦臣充朝,危殆甚矣。朕惟定國之計莫宜于公,引納于朝,即日罷退高安侯董賢,轉漏之間,忠策輒建,綱紀咸張。綏和、元壽,再遭大行,萬事畢舉,禍亂不作。輔朕五年,人倫之本正,天地之位定。欽承神祇,經緯四時,復千載之廢,矯百世之失,天下和會,大眾方輯。詩之靈臺,書之作雒,鎬京之制,商邑之度,於今復興。昭章先帝之元功,明著祖宗之令德,推顯嚴父配天之義,修立郊禘宗祀之禮,以光大孝。是以四海雍雍,萬國慕義,蠻夷殊俗,不召自至,漸化端冕,奉珍助祭。尋舊本道,遵術重古,動而有成,事得厥中。至德要道,通於神明,祖考嘉享。光耀顯章,天符仍臻,元氣大同。麟鳳龜龍,眾祥之瑞,七百有餘。遂制禮作樂,有綏靖宗廟社稷之大勳。普天之下,惟公是賴,官在宰衡,位在上公。今加九命之錫,其以助祭,共文武之職,乃遂及厥祖。於戲,豈不休哉!

於是莽稽首再拜,受綠韍袞冕衣裳,瑒琫瑒珌,句履,鸞路乘馬,龍旂九旒,皮弁素積,戎路乘馬,彤弓矢,盧弓矢,左建朱鉞,右建金戚,甲冑一具,秬鬯二卣,圭瓚二,九命青玉珪二,朱戶納陛。署宗官、祝官、卜官、史官,虎賁三百人,家令丞各一人,宗、祝、卜、史官皆置嗇夫,佐安漢公。在中府外第,虎賁為門衛,當出入者傅籍。自四輔、三公有事府第,皆用傳。以楚王邸為安漢公第,大繕治,通周衛。祖禰廟及寢皆為朱戶納陛。陳崇又奏:「安漢公祠祖禰,出城門,城門校尉宜將騎士從。入有門衛,出有騎士,所以重國也。」奏可。

其秋,莽以皇后有子孫瑞,通子午道。子午道從杜陵直絕南山,徑漢中。

風俗使者八人還,言天下風俗齊同,詐為郡國造歌謠,頌功德,凡三萬言。莽奏定著令。又奏為市無二賈,官無獄訟,邑無盜賊,野無飢民,道不拾遺,男女異路之制,犯者象刑。劉歆、陳崇等十二人皆以治明堂,宣教化,封為列侯。

莽既致太平,北化匈奴,東致海外,南懷黃支,唯西方未有加。乃遣中郎將平憲等多持金幣誘塞外羌,使獻地,願內屬。憲等奏言:「羌豪良願等種,人口可萬二千人,願為內臣,獻鮮水海、允谷鹽池,平地美草皆予漢民,自居險阻處為藩蔽。問良願降意,對曰:『太皇太后聖明,安漢公至仁,天下太平,五穀成孰,或禾長丈餘,或一粟三米,或不種自生,或嘤不蠶自成,甘露從天下,醴泉自地出,鳳皇來儀,神爵降集。從四歲以來,羌人無所疾苦,故思樂內屬。』宜以時處業,置屬國領護。」事下莽,莽復奏曰:「太后秉統數年,恩澤洋溢,和氣四塞,絕域殊俗,靡不慕義。越裳氏重譯獻白雉,黃支自三萬里貢生犀,東夷王度大海奉國珍,匈奴單于順制作,去二名,今西域良願等復舉地為臣妾,昔唐堯橫被四表,亦亡以加之。今謹案已有東海、南海、北海郡,未有西海郡,請受良願等所獻地為西海郡。臣又聞聖王序天文,定地理,因山川民俗以制州界。漢家地廣二帝三王,凡十三州,州名及界多不應經。堯典十有二州界,後定為九州。漢家廓地遼遠,州牧行部,遠者三萬餘里,不可為九。謹以經義正十二州名分界,以應正始。」奏可。又增法五十條,犯者徙之西海。徙者以千萬數,民始怨矣。

泉陵侯劉慶上書言:「周成王幼少,稱孺子,周公居攝。今帝富於春秋,宜令安漢公行天子事,如周公。」群臣皆曰:「宜如慶言。」

冬,熒惑入月中。

平帝疾,莽作策,請命於泰畤,戴璧秉圭,願以身代。藏策金縢,置于前殿,敕諸公勿敢言。十二月平帝崩,大赦天下。莽徵明禮者宗伯鳳等與定天下吏六百石以上皆服喪三年。奏尊孝成廟曰統宗,孝平廟曰元宗。時元帝世絕,而宣帝曾孫有見王五人,列侯廣戚侯顯等四十八人,莽惡其長大,曰:「兄弟不得相為後。」乃選玄孫中最幼廣戚侯子嬰,年二歲,託以為卜相最吉。

是月,前煇光謝囂奏武功長孟通浚井得白石,上圓下方,有丹書著石,文曰「告安漢公莽為皇帝」。符命之起,自此始矣。莽使群公以白太后,太后曰:「此誣罔天下,不可施行!」太保舜謂太后:「事已如此,無可奈何,沮之力不能止。又莽非敢有它,但欲稱攝以重其權,填服天下耳。」太后聽許,舜等即共令太后下詔曰:「蓋聞天生眾民,不能相治,為之立君以統理之。君年幼稚,必有寄託而居攝焉,然後能奉天施而成地化,群生茂育。書不云乎?『天工,人其代之。』朕以孝平皇帝幼年,且統國政,幾加元服,委政而屬之。今短命而崩,嗚呼哀哉!已使有司徵孝宣皇帝玄孫二十三人,差度宜者,以嗣孝平皇帝之後。玄孫年在繈褓,不得至德君子,孰能安之?安漢公莽輔政三世,比遭際會,安光漢室,遂同殊風,至于制作,與周公異世同符。今前煇光囂、武功長通上言丹石之符,朕深思厥意,云『為皇帝』者,乃攝行皇帝之事也。夫有法成易,非聖人者亡法。其令安漢公居攝踐祚,如周公故事,以武功縣為安漢公采地,名曰漢光邑。具禮儀奏。」

於是群臣奏言:「太后聖德昭然,深見天意,詔令安漢公居攝。臣聞周成王幼少,周道未成,成王不能共事天地,修文武之烈。周公權而居攝,則周道成,王室安;不居攝,則恐周隊失天命。書曰:『我嗣事子孫,大不克共上下,遏失前人光,在家不知命不易。天應棐諶,乃亡隊命。』說曰:周公服天子之冕,南面而朝群臣,發號施令,常稱王命。召公賢人,不知聖人之意,故不說也。禮明堂記曰:『周公朝諸侯於明堂,天子負斧依南面而立。』謂『周公踐天子位,六年朝諸侯,制禮作樂,而天下大服』也。召公不說。時武王崩,縗麤未除。由是言之,周公始攝則居天子之位,非乃六年而踐阼也。書逸嘉禾篇曰:『周公奉鬯立于阼階,延登,贊曰:「假王蒞政,勤和天下。」』此周公攝政,贊者所稱。成王加元服,周公則致政。書曰『朕復子明辟』,周公常稱王命,專行不報,故言我復子明君也。臣請安漢公居攝踐祚,服天子韍冕,背斧依于戶牖之間,南面朝群臣,聽政事。車服出入警蹕,民臣稱臣妾,皆如天子之制。郊祀天地,宗祀明堂,共祀宗廟,享祭群神,贊曰『假皇帝』,民臣謂之『攝皇帝』,自稱曰『予』。平決朝事,常以皇帝之詔稱『制』,以奉順皇天之心,輔翼漢室,保安孝平皇帝之幼嗣,遂寄託之義,隆治平之化。其朝見太皇太后、帝皇后,皆復臣節。自施政教於其宮家國采,如諸侯禮故事。臣昧死請。」太后詔曰:「可。」明年,改元曰居攝。

居攝元年正月,莽祀上帝於南郊,迎春於東郊,行大射禮於明堂,養三老五更,成禮而去。置柱下五史,秩如御史,聽政事,侍旁記疏言行。

三月己丑,立宣帝玄孫嬰為皇太子,號曰孺子。以王舜為太傅左輔,甄豐為太阿右拂,甄邯為太保後承。又置四少,秩皆二千石。

四月,安眾侯劉崇與相張紹謀曰:「安漢公莽專制朝政,必危劉氏。天下非之者,乃莫敢先舉,此宗室恥也。吾帥宗族為先,海內必和。」紹等從者百餘人,遂進攻宛,不得入而敗。紹者,張竦之從兄也。竦與崇族父劉嘉詣闕自歸,莽赦弗罪。竦因為嘉作奏曰:

建平、元壽之間,大統幾絕,宗室幾棄。賴蒙陛下聖德,扶服振救,遮扞匡衛,國命復延,宗室明目。臨朝統政,發號施令,動以宗室為始,登用九族為先。並錄支親,建立王侯,南面之孤,計以百數。收復絕屬,存亡續廢,得比肩首,復為人者,嬪然成行,所以藩漢國,輔漢宗也。建辟雍,立明堂,班天法,流聖化,朝群后,昭文德,宗室諸侯,咸益土地。天下喁喁,引領而歎,頌聲洋洋,滿耳而入。國家所以服此美,膺此名,饗此福,受此榮者,豈非太皇太后日昃之思,陛下夕惕之念哉!何謂?亂則統其理,危則致其安,禍則引其福,絕則繼其統,幼則代其任,晨夜屑屑,寒暑勤勤,無時休息,孳孳不已者,凡以為天下,厚劉氏也。臣無愚智,民無男女,皆諭至意。

而安眾侯崇乃獨懷悖惑之心,操畔逆之慮,興兵動眾,欲危宗廟,惡不忍聞,罪不容誅,誠臣子之仇,宗室之讎,國家之賊,天下之害也。是故親屬震落而告其罪,民人潰畔而棄其兵,進不跬步,退伏其殃。百歲之母,孩提之子,同時斷斬,懸頭竿杪,珠珥在耳,首飾猶存,為計若此,豈不誖哉!

臣聞古者畔逆之國,既以誅討,而豬其宮室以為汙池,納垢濁焉,名曰凶虛,雖生菜茹,而人不食。四牆其社,覆上棧下,示不得通。辨社諸侯,出門見之,著以為戒。方今天下聞崇之反也,咸欲騫衣手劍而叱之。其先至者,則拂其頸,衝其匈,刃其軀,切其肌;後至者,欲撥其門,仆其牆,夷其屋,焚其器,應聲滌地,則時成創。而宗室尤甚,言必切齒焉。何則?以其背畔恩義,而不知重德之所在也。宗室所居或遠,嘉幸得先聞,不勝憤憤之願,願為宗室倡始,父子兄弟負籠荷鍤,馳之南陽,豬崇宮室,令如古制。及崇社宜如亳社,以賜諸侯,用永監戒。願下四輔公卿大夫議,以明好惡,視四方。

於是莽大說。公卿曰:「皆宜如嘉言。」莽白太后下詔曰:「

惟嘉父子兄弟,雖與崇有屬,不敢阿私,或見萌牙,相率告之,及其禍成,同共讎之,應合古制,忠孝著焉。其以杜衍戶千封嘉為師禮侯,嘉子七人皆賜爵關內侯。」後又封竦為淑德侯。長安

謂之語曰:「欲求封,過張伯松;力戰鬥,不如巧為奏。」莽又封南陽吏民有功者百餘人,汙池劉崇室宅。後謀反者,皆汙池云。

群臣復白:「劉崇等謀逆者,以莽權輕也。宜尊重以填海內。」五月甲辰,太后詔莽朝見太后稱「假皇帝」。

冬十月丙辰朔,日有食之。

十二月,群臣奏請:「益安漢公宮及家吏,置率更令,廟、廄、廚長丞,中庶子,虎賁以下百餘人,又置衛士三百人。安漢公廬為攝省,府為攝殿,第為攝宮。」奏可。

莽白太后下詔曰:「故太師光雖前薨,功效已列。太保舜、大司空豐、輕車將軍邯、步兵將軍建皆為誘進單于籌策,又典靈臺、明堂、辟雍、四郊,定制度,開子午道,與宰衡同心說德,合意并力,功德茂著。封舜子匡為同心侯,林為說德侯,光孫壽為合意侯,豐孫匡為并力侯。益邯、建各三千戶。

是歲,西羌龐恬、傅幡等怨莽奪其地作西海郡,反攻西海太守程永,永奔走。莽誅永,遣護羌校尉竇況擊之。

二年春,竇況等擊破西羌。

五月,更造貨:錯刀,一直五千;契刀,一直五百;大錢,一直五十,與五銖錢並行。民多盜鑄者。禁列侯以下不得挾黃金,輸御府受直,然卒不與直。

九月,東郡太守翟義都試,勒車騎,因發奔命,立嚴鄉侯劉信為天子,移檄郡國,言莽「毒殺平帝,攝天子位,欲絕漢室,今共行天罰誅莽。」郡國疑惑,眾十餘萬。莽惶懼不能食,晝夜抱孺子告禱郊廟,放大誥作策,遣諫大夫桓譚等班於天下,諭以攝位當反政孺子之意。遣王邑、孫建等八將軍擊義,分屯諸關,守阨塞。槐里男子趙明、霍鴻等起兵,以和翟義,相與謀曰:「諸將精兵悉東,京師空,可攻長安。」眾稍多,至且十萬人,莽恐,遣將軍王奇、王級將兵拒之。以太保甄邯為大將軍,受鉞高廟,領天下兵,左杖節,右把鉞,屯城外。王舜、甄豐晝夜循行殿中。

十二月,王邑等破翟義於圉。司威陳崇使監軍上書言:「

陛下奉天洪範,心合寶龜,膺受元命,豫知成敗,感應兆占,是謂配天。配天之主,慮則移氣,言則動物,施則成化。臣崇伏讀詔書下日,竊計其時,聖思始發,而反虜仍破;詔文始書,反虜大敗;制書始下,反虜畢斬,眾將未及齊其鋒芒。臣崇未及盡其愚慮,而事已決矣。」莽大說。

三年春,地震。大赦天下。

王邑等還京師,西與王級等合擊明、鴻,皆破滅,語在翟義傳。莽大置酒未央宮白虎殿,勞賜將帥。詔陳崇治校軍功,第其高下。莽乃上奏曰:「明聖之世,國多賢人,故唐虞之時,可比屋而封,至功成事就,則加賞焉。至於夏后塗山之會,執玉帛者萬國,諸侯執玉,附庸執帛。周武王孟津之上,尚有八百諸侯。周公居攝,郊祀后稷以配天,宗祀文王於明堂以配上帝,是以四海之內各以其職來祭,蓋諸侯千八百矣。禮記王制千七百餘國,是以孔子著孝經曰:『不敢遺小國之臣,而況於公侯伯子男乎?故得萬國之歡心以事其先王。』此天子之孝也。秦為亡道,殘滅諸侯以為郡縣,欲擅天下之利,故二世而亡。高皇帝受命除殘,考功施賞,建國數百,後稍衰微,其餘僅存。太皇太后躬統大綱,廣封功德以勸善,興滅繼絕以永世,是以大化流通,旦暮且成。遭羌寇害西海郡,反虜流言東郡,逆賊惑眾西土,忠臣孝子莫不奮怒,所征殄滅,盡備厥辜,天子咸寧。今制禮作樂,實考周爵五等,地四等,有明文;殷爵三等,有其說,無其文。孔子曰:『周監於二代,郁郁乎文哉!吾從周。』臣請諸將帥當受爵邑者爵五等,地四等。」奏可。於是封者高為侯伯,次為子男,當賜爵關內侯者更名曰附城,凡數百人。擊西海者以「羌」為號,槐里以「武」為號,翟義以「虜」為號。

群臣復奏言:「太后修功錄德,遠者千載,近者當世,或以文封,或以武爵,深淺大小。靡不畢舉。今攝皇帝背依踐祚,宜異於宰國之時,制作雖未畢已,宜進二子爵皆為公。春秋『善善及子孫』,『賢者之後,宜有土地』。成王廣封周公庶子六子,皆有茅土。及漢家名相大將蕭、霍之屬,咸及支庶。兄子光,可先封為列侯;諸孫,制度畢已,大司徒、大司空上名,如前詔書。」太后詔曰:「進攝皇帝子褒新侯安為新舉公,賞都侯臨為褒新公,封光為衍功侯。」是時,莽還歸新都國,群臣復白以封莽孫宗為新都侯。莽既滅翟義,自謂威德日盛,獲天人助,遂謀即真之事矣。

九月,莽母功顯君死,意不在哀,令太后詔議其服。少阿、羲和劉歆與博士諸儒七十八人皆曰:「居攝之義,所以統立天功,興崇帝道,成就法度,安輯海內也。昔殷成湯既沒,而太子蚤夭,其子太甲幼少不明,伊尹放諸桐宮而居攝,以興殷道。周武王既沒,周道未成,成王幼少,周公屏成王而居攝,以成周道。是以殷有翼翼之化,周有刑錯之功。今太皇太后比遭家之不造,委任安漢公宰尹群僚,衡平天下。遭孺子幼少,未能共上下,皇天降瑞,出丹石之符,是以太皇太后則天明命,詔安漢公居攝踐祚,將以成聖漢之業,與唐虞三代比隆也。攝皇帝遂開祕府,會群儒,制禮作樂,卒定庶官,茂成天功。聖心周悉,卓爾獨見,發得周禮,以明因監,則天稽古,而損益焉,猶仲尼之聞韶,日月之不可階,非聖哲之至,孰能若茲!綱紀咸張,成在一匱,此其所以保佑聖漢,安靖元元之效也。今功顯君薨,禮『庶子為後,為其母緦。』傳曰『與尊者為體,不敢服其私親也。』攝皇帝以聖德承皇天之命,受太后之詔居攝踐祚,奉漢大宗之後,上有天地社稷之重,下有元元萬機之憂,不得顧其私親。故太皇太后建厥元孫,俾侯新都,為哀侯後。明攝皇帝與尊者為體,承宗廟之祭,奉共養太皇太后,不得服其私親也。周禮曰『王為諸侯緦縗』,『弁而加環絰』,同姓則麻,異姓則葛。攝皇帝當為功顯君緦縗,弁而加麻環絰,如天子弔諸侯服,以應聖制。」莽遂行焉,凡壹弔再會,而令新都侯宗為主,服喪三年云。

司威陳崇奏,衍功侯光私報執金吾竇況,令殺人,況為收繫,致其法。莽大怒,切責光。光母曰:「女自視孰與長孫、中孫?」遂母子自殺,及況皆死。初,莽以事母、養嫂、撫兄子為名,及後悖虐,復以示公義焉。令光子嘉嗣爵為侯。

莽下書曰:「遏密之義,訖于季冬,正月郊祀,八音當奏。王公卿士,樂凡幾等?五聲八音,條各云何?其與所部儒生各盡精思,悉陳其義。」

是歲廣饒侯劉京、車騎將軍千人扈雲、大保屬臧鴻奏符命。京言齊郡新井,雲言巴郡石牛,鴻言扶風雍石,莽皆迎受。十一月甲子,莽上奏太后曰:「陛下至聖,遭家不造,遇漢十二世三七之阨,承天威命,詔臣莽居攝,受孺子之託,任天下之寄。臣莽兢兢業業,懼於不稱。宗室廣饒侯劉京上書言:『七月中,齊郡臨淄縣昌興亭長辛當一暮數夢,曰:「吾,天公使也。天公使我告亭長曰:『攝皇帝當為真。』即不信我,此亭中當有新井。」亭長晨起視亭中,誠有新井,入地且百尺。』十一月壬子,直建冬至,巴郡石牛,戊午,雍石文,皆到于未央宮之前殿。臣與太保安陽侯舜等視,天風起,塵冥,風止,得銅符帛圖於石前,文曰:『天告帝符,獻者封侯。承天命,用神令。』騎都尉崔發等視說。及前孝哀皇帝建平二年六月甲子下詔書,更為太初元將元年,案其本事,甘忠可、夏賀良讖書臧蘭臺。臣莽以為元將元年者,大將居攝改元之文也,於今信矣。尚書康誥『王若曰:「孟侯,朕其弟,小子封。」』此周公居攝稱王之文也。春秋隱公不言即位,攝也。此二經周公、孔子所定,蓋為後法。孔子曰:『畏天命,畏大人,畏聖人之言。』臣莽敢不承用!臣請共事神祇宗廟,奏言太皇太后、孝平皇后,皆稱假皇帝。其號令天下,天下奏言事,毋言『

攝』。以居攝三年為初始元年,漏刻以百二十為度,用應天命。臣莽夙夜養育隆就孺子,令與周之成王比德,宣明太皇太后威德於萬方,期於富而教之。孺子加元服,復子明辟,如周公故事。」奏可。眾庶知其奉符命,指意群臣博議別奏,以視即真之漸矣。

期門郎張充等六人謀共劫莽,立楚王。發覺,誅死。

梓潼人哀章學問長安,素無行,好為大言。見莽居攝,即作銅匱,為兩檢,署其一曰「天帝行璽金匱圖」,其一署曰「赤帝行璽某傳予黃帝金策書」。某者,高皇帝名也。書言王莽為真天子,皇太后如天命。圖書皆書莽大臣八人,又取令名王興、王盛,章因自竄姓名,凡為十一人,皆署官爵,為輔佐。章聞齊井、石牛事下,即日昏時,衣黃衣,持匱至高廟,以付僕射。僕射以聞。戊辰,莽至高廟拜受金匱神嬗。御王冠,謁太后,還坐未央宮前殿,下書曰:「予以不德,託于皇初祖考黃帝之後,皇始祖考虞帝之苗裔,而太皇太后之末屬。皇天上帝隆顯大佑,成命統序,符契圖文,金匱策書,神明詔告,屬予以天下兆民。赤帝漢氏高皇帝之靈,承天命,傳國金策之書,予甚祗畏,敢不欽受!以戊辰直定,御王冠,即真天子位,定有天下之號曰新。其改正朔,易服色,變犧牲,殊徽幟,異器制。以十二月朔癸酉為建國元年正月之朔,以雞鳴為時。服色配德上黃,犧牲應正用白,使節之旄旛皆純黃,其署曰『新使五威節』,以承皇天上帝威命也。」

始建國元年正月朔,莽帥公侯卿士奉皇太后璽韍,上太皇太后,順符命,去漢號焉。

初,莽妻宜春侯王氏女,立為皇后。本生四男:宇、獲、安、臨。二子前誅死,安頗荒忽,乃以臨為皇太子,安為新嘉辟。封宇子六人:千為功隆公,壽為功明公,吉為功成公,宗為功崇公,世為功昭公,利為功著公。大赦天下。

莽乃策命孺子曰:「咨爾嬰,昔皇天右乃太祖,歷世十二,享國二百一十載,曆數在于予躬。詩不云乎?『侯服于周,天命靡常。』封爾為定安公,永為新室賓。於戲!敬天之休,往踐乃位,毋廢予命。」又曰:「其以平原、安德、漯陰、鬲、重丘,凡戶萬,地方百里,為定安公國。立漢祖宗之廟於其國,與周後並,行其正朔、服色。世世以事其祖宗,永以命德茂功,享歷代之祀焉。以孝平皇后為定安太后。」讀策畢,莽親執孺子手,流涕歔欷,曰:「昔周公攝位,終得復子明辟,今予獨迫皇天威命,不得如意!」哀歎良久。中傅將孺子下殿,北面而稱臣。百僚陪位,莫不感動。

又按金匱,輔臣皆封拜。以太傅、左輔、驃騎將軍安陽侯王舜為太師,封安新公;大司徒就德侯平晏為太傅,就新公;少阿、羲和、京兆尹紅休侯劉歆為國師,嘉新公;廣漢梓潼哀章為國將,美新公:是為四輔,位上公。太保、後承承陽侯甄邯為大司馬,承新公;丕進侯王尋為大司徒,章新公;步兵將軍成都侯王邑為大司空,隆新公:是為三公。大阿、右拂、大司空、衛將軍廣陽侯甄豐為更始將軍,廣新公;京兆王興為衛將軍,奉新公;輕車將軍成武侯孫建為立國將軍,成新公;京兆王盛為前將軍,崇新公:是為四將。凡十一公。王興者,故城門令史。王盛者,賣餅。莽按符命求得此姓名十餘人,兩人容貌應卜相,徑從布衣登用,以視神焉。餘皆拜為郎。是日,封拜卿大夫、侍中、尚書官凡數百人。諸劉為郡守,皆徙為諫大夫。

改明光宮為定安館,定安太后居之。以故大鴻臚府為定安公第,皆置門衛使者監領。敕阿乳母不得與語,常在四壁中,至於長大,不能名六畜。後莽以女孫宇子妻之。

莽策群司曰:「歲星司肅,東獄太師典致時雨,青煒登平,考景以晷。熒惑司悊,南嶽太傅典致時奧,赤煒頌平,考聲以律。太白司艾,西嶽國師典致時陽,白煒象平,考量以銓。辰星司謀,北嶽國將典致時寒,玄煒和平,考星以漏。月刑元股左,司馬典致武應,考方法矩,主司天文,欽若昊天,敬授民時,力來農事,以豐年穀。日德元锾右,司徒典致文瑞,考圜合規,主司人道,五教是輔,帥民承上,宣美風俗,五品乃訓。斗平元心中,司空典致物圖,考度以繩,主司地里,平治水土,掌名山川,眾殖鳥獸,蕃茂草木。」各策命以其職,如典誥之文。

置大司馬司允,大司徒司直,大司空司若,位皆孤卿。更名大司農曰羲和,後更為納言,大理曰作士,太常曰秩宗,大鴻臚曰典樂,少府曰共工,水衡都尉曰予虞,與三公司卿凡九卿,分屬三公。每一卿置大夫三人,一大夫置元士三人,凡二十七大夫,八十一元士,分主中都官諸職。更名光祿勳曰司中,太僕曰太御,衛尉曰太衛,執金吾曰奮武,中尉曰軍正,又置大贅官,主乘輿服御物,後又典兵秩,位皆上卿,號曰六監。改郡太守曰大尹,都尉曰太尉,縣令長曰宰,御史曰執法,公車司馬曰王路四門,長樂宮曰常樂室,未央宮曰壽成室,前殿曰王路堂,長安曰常安。更名秩百石曰庶士,三百石曰下士,四百石曰中士,五百石曰命士,六百石曰元士,千石曰下大夫,比二千石曰中大夫,二千石曰上大夫,中二千石曰卿。車服黻冕,各有差品。又置司恭、司徒、司明、司聰、司中大夫及誦詩工、徹膳宰,以司過。策曰:「予聞上聖欲昭厥德,罔不慎修厥身,用綏于遠,是用建爾司于五事。毋隱尤,毋將虛,好惡不愆,立于厥中。於戲,勗哉!」令王路設進善之旌,非謗之木,欲諫之鼓。諫大夫四人常坐王路門受言事者。

封王氏齊縗之屬為侯,大功為伯,小功為子,緦麻為男,其女皆為任。男以「睦」、女以「隆」為號焉,皆授印韍。令諸侯立太夫人、夫人、世子,亦受印韍。

又曰:「天無二日,土無二王,百王不易之道也。漢氏諸侯或稱王,至于四夷亦如之,違於古典,繆於一統。其定諸侯王之號皆稱公,及四夷僭號稱王者皆更為侯。」

又曰:「帝王之道,相因而通;盛德之祚,百世享祀。予惟黃帝、帝少昊、帝顓頊、帝嚳、帝堯、帝舜、帝夏禹、皋陶、伊尹咸有聖德,假于皇天,功烈巍巍,光施于遠。予甚嘉之,營求其後,將祚厥祀。」惟王氏,虞帝之後也,出自帝嚳;劉氏,堯之後也,出自顓頊。於是封姚恂為初睦侯,奉黃帝後;梁護為脩遠伯,奉少昊後;皇孫功隆公千,奉帝嚳後;劉歆為祁烈伯,奉顓頊後;國師劉歆子疊為伊休侯,奉堯後;媯昌為始睦侯,奉虞帝後;山遵為褒謀子,奉皋陶後;伊玄為褒衡子,奉伊尹後。漢後定安公劉嬰,位為賓。周後衛公姬黨,更封為章平公,亦為賓。殷後宋公孔弘,運轉次移,更封為章昭侯,位為恪。夏後遼西姒豐,封為章功侯,亦為恪。四代古宗,宗祀于明堂,以配皇始祖考虞帝。周公後褒魯子姬就,宣尼公後褒成子孔鈞,已前定焉。

莽又曰:「予前在攝時,建郊宮,定祧廟,立社稷,神祇報況,或光自上復于下,流為烏,或黃氣熏烝,昭燿章明,以著黃、虞之烈焉。自黃帝至于濟南伯王,而祖世氏姓有五矣。黃帝二十五子,分賜厥姓十有二氏。虞帝之先,受姓曰姚,其在陶唐曰媯,在周曰陳,在齊曰田,在濟南曰王。予伏念皇初祖考黃帝,皇始祖考虞帝,以宗祀于明堂,宜序於祖宗之親廟。其立祖廟五,親廟四,后夫人皆配食。郊祀黃帝以配天,黃后以配地。以新都侯東弟為大禖,歲時以祀。家之所尚,種祀天下。姚、媯、陳、田、王氏凡五姓者,皆黃、虞苗裔,予之同族也。書不云乎?『惇序九族。』其令天下上此五姓名籍于秩宗,皆以為宗室。世世復,無有所與。其元城王氏,勿令相嫁娶,以別族理親焉。」封陳崇為統睦侯,奉胡王後;田豐為世睦侯,奉敬王後。

天下牧守皆以前有翟義、趙明等領州郡,懷忠孝,封牧為男,守為附城。又封舊恩戴崇、金涉、箕閎、楊並等子皆為男。

遣騎都尉囂等分治黃帝園位於上都橋畤,虞帝於零陵九疑,胡王於淮陽陳,敬王於齊臨淄,愍王於城陽莒,伯王於濟南東平陵,孺王於魏郡元城,使者四時致祠。其廟當作者,以天下初定,且祫祭於明堂太廟。

以漢高廟為文祖廟。莽曰:「予之皇始祖考虞帝受嬗于唐,漢氏初祖唐帝,世有傳國之象,予復親受金策於漢高皇帝之靈。惟思褒厚前代,何有忘時?漢氏祖宗有七,以禮立廟于定安國。其園寢廟在京師者,勿罷,祠薦如故。予以秋九月親入漢氏高、元、成、平之廟。諸劉更屬籍京兆大尹,勿解其復,各終厥身,州牧數存問,勿令有侵冤。」

又曰:「予前在大麓,至于攝假,深惟漢氏三七之阨,赤德氣盡,思索廣求,所以輔劉延期之述,靡所不用。以故作金刀之利,幾以濟之。然自孔子作春秋以為後王法,至于哀之十四而一代畢,協之於今,亦哀之十四也。赤世計盡,終不可強濟。皇天明威,黃德當興,隆顯大命,屬予以天下。今百姓咸言皇天革漢而立新,廢劉而興王。夫『劉』之為字『卯、金、刀』也,正月剛卯,金刀之利,皆不得行。博謀卿士,僉曰天人同應,昭然著明。其去剛卯莫以為佩,除刀錢勿以為利,承順天心,快百姓意。」乃更作小錢,徑六分,重一銖,文曰「小錢直一」,與前「大錢五十」者為二品,並行。欲防民盜鑄,乃禁不得挾銅炭。

是歲四月,徐鄉侯劉快結黨數千人起兵於其國。快兄殷,故漢膠東王,時改為扶崇公。快舉兵攻即墨,殷閉城門,自繫獄。吏民距快,快敗走,至長廣死。莽曰:「昔予之祖濟南愍王困於燕寇,自齊臨淄出保于莒。宗人田單廣設奇謀,獲殺燕將,復定齊國。今即墨士大夫復同心殄滅反虜,予甚嘉其忠者,憐其無辜。其赦殷等,非快之妻子它親屬當坐者皆勿治。弔問死傷,賜亡者葬錢,人五萬。殷知大命,深疾惡快,以故輒伏厥辜。其滿殷國戶萬,地方百里。」又封符命臣十餘人。

莽曰:「古者,設廬井八家,一夫一婦田百畝,什一而稅,則國給民富而頌聲作。此唐虞之道,三代所遵行也。秦為無道,厚賦稅以自供奉,罷民力以極欲,壞聖制,廢井田,是以兼并起,貪鄙生,強者規田以千數,弱者曾無立錐之居。又置奴婢之市,與牛馬同蘭,制於民臣,顓斷其命。姦虐之人因緣為利,至略賣人妻子,逆天心,誖人倫,繆於『天地之性人為貴』之義。書曰『予則奴戮女』,唯不用命者,然後被此辜矣。漢氏減輕田租,三十而稅一,常有更賦,罷癃咸出,而豪民侵陵,分田劫假。厥名三十稅一,實什稅五也。父子夫婦終年耕芸,所得不足以自存。故富者犬馬餘菽粟,驕而為邪;貧者不厭糟糠,窮而為姦。俱陷于辜,刑用不錯。予前在大麓,始令天下公田口井,時則有嘉禾之祥,遭反虜逆賊且止。今更名天下田曰『王田』,奴婢曰『私屬』,皆不得賣買。其男口不盈八,而田過一井者,分餘田予九族鄰里鄉黨。故無田,今當受田者,如制度。敢有非井田聖制,無法惑眾者,投諸四裔,以禦魑魅,如皇始祖考虞帝故事。」

是時百姓便安漢五銖錢,以莽錢大小兩行難知,又數變改不信,皆私以五銖錢市買。訛言大錢當罷,莫肯挾。莽患之,復下書:「諸挾五銖錢,言大錢當罷者,比非井田制,投四裔。」於是農商失業,食貨俱廢,民人至涕泣於市道。及坐賣買田宅奴婢,鑄錢,自諸侯卿大夫至于庶民,抵罪者不可勝數。

秋,遣五威將王奇等十二人班符命四十二篇於天下。德祥五事,符命二十五,福應十二,凡四十二篇。其德祥言文、宣之世黃龍見於成紀、新都,高祖考王伯墓門梓柱生枝葉之屬。符命言井石、金匱之屬。福應言雌雞化為雄之屬。其文爾雅依託,皆為作說,大歸言莽當代漢有天下云。總而說之曰:「帝王受命,必有德祥之符瑞,協成五命,申以福應,然後能立巍巍之功,傳于子孫,永享無窮之祚。故新室之興也,德祥發於漢三七九世之後。肇命於新都,受瑞於黃支,開王於武功,定命於子同,成命於巴宕,申福於十二應,天所以保祐新室者深矣,固矣!武功丹石出於漢氏平帝末年,火德銷盡,土德當代,皇天眷然,去漢與新,以丹石始命於皇帝。皇帝謙讓,以攝居之,未當天意,故其秋七月,天重以三能文馬。皇帝復謙讓,未即位,故三以鐵契,四以石龜,五以虞符,六以文圭,七以玄印,八以茂陵石書,九以玄龍石,十以神井,十一以大神石,十二以銅符帛圖。申命之瑞,寖以顯著,至于十二,以昭告新皇帝。皇帝深惟上天之威不可不畏,故去攝號,猶尚稱假,改元為初始,欲以承塞天命,克厭上帝之心。然非皇天所以鄭重降符命之意。故是日天復決其以勉書。又侍郎王盱見人衣白布單衣,赤繢方領,冠小冠,立于王路殿前,謂盱曰:『今日天同色,以天下人民屬皇帝。』盱怪之,行十餘步,人忽不見。至丙寅暮,漢氏高廟有金匱圖策:『高帝承天命,以國傳新皇帝。』明旦,宗伯忠孝侯劉宏以聞,乃召公卿議,未決,而大神石人談曰:『趣新皇帝之高廟受命,毋留!』於是新皇帝立登車,之漢氏高廟受命。受命之日,丁卯也。丁,火,漢氏之德也。卯,劉姓所以為字也。明漢劉火德盡,而傳於新室也。皇帝謙謙,既備固讓,十二符應迫著,命不可辭,懼然祗畏,葦然閔漢氏之終不可濟,斖斖在左右之不得從意,為之三夜不御寢,三日不御食,延問公侯卿大夫,僉曰:『宜奉如上天威命。』於是乃改元定號,海內更始。新室既定,神祇懽喜,申以福應,吉瑞累仍。《詩》曰:『宜民宜人,受祿于天;保右命之,自天申之。』此之謂也。」五威將奉符命,齎印綬,王侯以下及吏官名更者,外及匈奴、西域,徼外蠻夷,皆即授新室印綬,因收故漢印綬。賜吏爵人二級,民爵人一級,女子百戶羊酒,蠻夷幣帛各有差。大赦天下。

五威將乘乾文車,駕坤六馬,背負鷩鳥之毛,服飾甚偉。每一將各置左右前後中帥,凡五帥。衣冠車服駕馬,各如其方面色數。將持節,稱太一之使;帥持幢,稱五帝之使。莽策命曰:「普天之下,迄于四表,靡所不至。」其東出者,至玄菟、樂浪、高句驪、夫餘;南出者,隃徼外,歷益州,貶句町王為侯;西出者,至西域,盡改其王為侯;北出者,至匈奴庭,授單于印,改漢印文,去「璽」曰「章」。單于欲求故印,陳饒椎破之,語在匈奴傳。單于大怒,而句町、西域後卒以此皆畔。饒還,拜為大將軍,封威德子。

冬,雷,桐華。

置五威司命,中城四關將軍。司命司上公以下,中城主十二城門。策命統睦侯陳崇曰:「咨爾崇。夫不用命者,亂之原也;大姦猾者,賊之本也;鑄偽金錢者,妨寶貨之道也;驕奢踰制者,兇害之端也;漏泄省中及尚書事者,『機事不密則害成』也;拜爵王庭,謝恩私門者,祿去公室,政從亡矣:凡此六條,國之綱紀。是用建爾作司命,『柔亦不茹,剛亦不吐,不侮鰥寡,不畏強圉』,帝命帥繇,統睦于朝。」命說符侯崔發曰:「『重門擊纤,以待暴客。』女作五威中城將軍,中德既成,天下說符。」命明威侯王級曰:「繞霤之固,南當荊楚。女作五威前關將軍,振武奮衛,明威于前。」命尉睦侯王嘉曰:「羊頭之阨,北當趙燕。女作五威後關將軍,壼口捶扼,尉睦于後。」命堂威侯王奇曰:「肴黽之險,東當鄭衛。女作五威左關將軍,函谷批難,掌威于左。」命懷羌子王福曰:「汧隴之阻,西當戎狄。女作五威右關將軍,成固據守,懷羌于右。」

又遣諫大夫五十人分鑄錢於郡國。

是歲長安狂女子碧呼道中曰:「高皇帝大怒,趣歸我國。不者,九月必殺汝!」莽收捕殺之。治者掌寇大夫陳成自免去官。真定劉都等謀舉兵,發覺,皆誅。真定、常山大雨雹。

二年二月,赦天下。

五威將帥七十二人還奏事,漢諸侯王為公者,悉上璽綬為民,無違命者。封將為子,帥為男。

初設六筦之令。命縣官酤酒,賣鹽鐵器,鑄錢,諸采取名山大澤眾物者稅之。又令市官收賤賣貴,賒貸予民,收息百月三。犧和置酒士,郡一人,乘傳督酒利。禁民不得挾弩鎧,徙西海。

匈奴單于求故璽,莽不與,遂寇邊郡,殺略吏民。

十一月,立國將軍建奏:「西域將欽上言,九月辛巳,戊己校尉史陳良、終帶共賊殺校尉刁護,劫略吏士,自稱廢漢大將軍,亡入匈奴。又今月癸酉,不知何一男子遮臣建車前,自稱『漢氏劉子輿,成帝下妻子也。劉氏當復,趣空宮。』收繫男子,即常安姓武字仲。皆逆天違命,大逆無道。請論仲及陳良等親屬當坐著。奏可。漢氏高皇帝比著戒云,罷吏卒,為賓食,誠欲承天心,全子孫也。其宗廟不當在常安城中,及諸劉為諸侯者當與漢俱廢。陛下至仁,久未定。前故安眾侯劉崇、徐鄉侯劉快、陵鄉侯劉曾、扶恩侯劉貴等更聚眾謀反。今狂狡之虜或妄自稱亡漢將軍,或稱成帝子子輿,至犯夷滅,連未止者,此聖恩不蚤絕其萌牙故也。臣愚以為漢高皇帝為新室賓,享食明堂。成帝,異姓之兄弟,平帝,婿也,皆不宜復入其廟。元帝與皇太后為體,聖恩所隆,禮亦宜之。臣請漢氏諸廟在京師者皆罷。諸劉為諸侯者,以戶多少就五等之差;其為吏者皆罷,待除於家。上當天心,稱高皇帝神靈,塞狂狡之萌。」莽曰:「可。嘉新公國師以符命為予四輔,明德侯劉龔、率禮侯劉嘉等凡三十二人皆知天命,或獻天符,或貢昌言,或捕告反虜,厥功茂焉。諸劉與三十二人同宗共祖者勿罷,賜姓曰王。」唯國師以女配莽子,故不賜姓。改定安太后號曰黃皇室主,絕之於漢也。

冬十二月,雷。

更名匈奴單于曰降奴服于。莽曰:「降奴服于知威侮五行,背畔四條,侵犯西域,廷及邊垂,為元元害,罪當夷滅。命遣立國將軍孫建等凡十二將,十道並出,共行皇天之威,罰于知之身。惟知先祖故呼韓邪單于稽侯蛳累世忠孝,保塞守徼,不忍以一知之罪,滅稽侯蛳之世。今分匈奴國土人民以為十五,立稽侯狦子孫十五人為單于。遣中郎將藺苞、戴級馳之塞下,召拜當為單于者。諸匈奴人當坐虜知之法者,皆赦除之。」遣五威將軍苗訢、虎賁將軍王況出五原,厭難將軍陳欽、震狄將軍王巡出雲中,振武將軍王嘉、平狄將軍王萌出代郡,相威將軍李棽、鎮遠將軍李翁出西河,誅貉將軍陽俊、討穢將軍嚴尤出漁陽,奮武將軍王駿、定胡將軍王晏出張掖,及褊裨以下百八十人。募天下囚徒、丁男、甲卒三十萬人,轉眾郡委輸五大夫衣裘、兵器、糧食,長吏送自負海江淮至北邊,使者馳傳督趣,以軍興法從事,天下騷動。先至者屯邊郡,須畢具乃同時出。

莽以錢幣訖不行,復下書曰:「民以食為命,以貨為資,是以八政以食為首。寶貨皆重則小用不給,皆輕則僦載煩費,輕重大小各有差品,則用便而民樂。」於是造寶貨五品,語在食貨志。百姓不從,但行小大錢二品而已。盜鑄錢者不可禁,乃重其法,一家鑄錢,五家坐之,沒入為奴婢。吏民出入,持布錢以副符傳,不持者,廚傳勿舍,關津苛留。公卿皆持以入宮殿門,欲以重而行之。

是時爭為符命封侯,其不為者相戲曰:「獨無天帝除書乎?」司令陳崇白莽曰:「此開姦臣作福之路而亂天命,宜絕其原。」莽亦厭之,遂使尚書大夫趙並驗治,非五威將率所班,皆下獄。

初,甄豐、劉歆、王舜為莽腹心,倡導在位,褒揚功德;「安漢」、「宰衡」之號及封莽母、兩子、兄子,皆豐等所共謀,而豐、舜、歆亦受其賜,並富貴矣,非復欲令莽居攝也。居攝之萌,出於泉陵侯劉慶、前煇光謝囂、長安令田終術。莽羽翼已成,意欲稱攝。豐等承順其意,莽輒復封舜、歆兩子及豐孫。豐等爵位已盛,心意既滿,又實畏漢宗室、天下豪桀。而疏遠欲進者,並作符命,莽遂據以即真,舜、歆內懼而已。豐素剛強,莽覺其不說,故徙大阿、右拂、大司空豐,託符命文,為更始將軍,與賣餅兒王盛同列。豐父子默默。時子尋為侍中京兆大尹茂德侯,即作符命,言新室當分陝,立二伯,以豐為右伯,太傅平晏為左伯,如周召故事。莽即從之,拜豐為右伯。當述職西出,未行,尋復作符命,言故漢氏平帝后黃皇室主為尋之妻。莽以詐立,心疑大臣怨謗,欲震威以懼下,因是發怒曰:「黃皇室主天下母,此何謂也!」收捕尋。尋亡,豐自殺。尋隨方士入華山,歲餘捕得,辭連國師公歆子侍中東通靈將、五司大夫隆威侯棻,棻弟右曹長水校尉伐虜侯泳,大司空邑弟左

闕將軍堂威侯奇,及歆門人侍中騎都尉丁隆等,牽引公卿黨親列侯以下,死者數百人。尋手理有「天子」字,莽解其臂入視之,曰:「此一大子也,或曰一六子也。六者,戮也。明尋父子當戮死也。」乃流棻于幽州,放尋于三危,殛隆于羽山,皆驛車載其屍傳致云。

莽為人侈口蹶顄,露眼赤精,大聲而嘶。長七尺五寸,好厚履高冠,以氂裝衣,反膺高視,瞰臨左右。是時有用方技待詔黃門者,或問以莽形貌,待詔曰:「莽所謂鴟目虎吻豺狼之聲者也,故能食人,亦當為人所食。」問者告之,莽誅滅待詔,而封告者。後常翳雲母屏面,非親近莫得見也。

是歲,以初睦侯姚恂為寧始將軍。

三年,莽曰:「百官改更,職事分移,律令儀法,未及悉定,且因漢律令儀法以從事。令公卿大夫諸侯二千石舉吏民有德行通政事能言語明文學者各一人,詣王路四門。」

遣尚書大夫趙並使勞北邊,還言五原北假膏壤殖穀,異時常置田官。乃以並為田禾將軍,發戍卒屯田北假,以助軍糧。

是時諸將在邊,須大眾集,吏士放縱,而內郡愁於徵發,民棄城郭流亡為盜賊,并州、平州尤甚。莽令七公六卿號皆兼稱將軍,遣著武將軍逯並等填名都,中郎將、繡衣執法各五十五人,分填緣邊大郡,督大姦猾擅弄兵者,皆便為姦於外,撓亂州郡,貨賂為市,侵漁百姓。莽下書曰:「虜知罪當夷滅,故遣猛將分十二部,將同時出,一舉而決絕之矣。內置司命軍正,外設軍監十有二人,誠欲以司不奉命,令軍人咸正也。今則不然,各為權勢,恐猲良民,妄封人頸,得錢者去。毒酿並作,農民離散。司監若此,可謂稱不?自今以來,敢犯此者,輒捕繫,以名聞。」然猶放縱自若。

而藺苞、戴級到塞下,招誘單于弟咸、咸子登入塞,脅拜咸為孝單于,賜黃金千斤,錦繡甚多,遣去;將登至長安,拜為順單于,留邸。

太師王舜自莽篡位後病悸,寖劇,死。莽曰:「昔齊太公以淑德累世,為周氏太師,蓋予之所監也。其以舜子延襲父爵,為安新公,延弟褒新侯匡為太師將軍,永為新室輔。」

為太子置師友各四人,秩以大夫。以故大司徒馬宮為師疑,故少府宗伯鳳為傅丞,博士袁聖為阿輔,京兆尹王嘉為保拂,是為四師;故尚書令唐林為胥附,博士李充為奔走,諫大夫趙襄為先後,中郎將廉丹為禦侮,是為四友。又置師友祭酒及侍中、諫議、六經祭酒各一人,凡九祭酒,秩上卿。琅邪左咸為講春秋、潁川滿昌為講詩、長安國由為講易、平陽唐昌為講書、沛郡陳咸為講禮、崔發為講樂祭酒。遣謁者持安車印綬,即拜楚國龔勝為太子師友祭酒,勝不應徵,不食而死。

寧始將軍姚恂免,侍中崇祿侯孔永為寧始將軍。

是歲,池陽縣有小人景,長尺餘,或乘車馬,或步行,據持萬物,小大各相稱,三日止。

瀕河郡蝗生。

河決魏郡,泛清河以東數郡。先是,莽恐河決為元城冢墓害。及決東去,元城不憂水,故遂不隄塞。

四年二月,赦天下。

夏,赤氣出東南,竟天。

厭難將軍陳歆言捕虜生口,虜犯邊者皆孝單于咸子角所為。莽怒,斬其子登於長安,以視諸蠻夷。

大司馬甄邯死,寧始將軍孔永為大司馬,侍中大贅侯輔為寧始將軍。

莽每當出,輒先鹑索城中,名曰「橫鹑」。是月,橫鹑五日。

莽至明堂,授諸侯茅土。下書曰:「予以不德,襲于聖祖,為萬國主。思安黎元,在于建侯,分州正域,以美風俗。追監前代,爰綱爰紀。惟在堯典,十有二州,衛有五服。詩國十五,抪遍九州。殷頌有『奄有九有』之言。禹貢之九州無并、幽,周禮司馬則無徐、梁。帝王相改,各有云為。或昭其事,或大其本,厥義著明,其務一矣。昔周二后受命,故有東都、西都之居。予之受命,蓋亦如之。其以洛陽為新室東都,常安為新室西都。邦畿連體,各有采任。州從禹貢為九,爵從周氏有五。諸侯之員千有八百,附城之數亦如之,以俟有功。諸公一同,有眾萬戶,土方百里。侯伯一國,眾戶五千,土方七十里。子男一則,眾戶二千有五百,土方五十里。附城大者食邑九成,眾戶九百,土方三十里。自九以下,降殺以兩,至於一成。五差備具,合當一則。今已受茅土者,公十四人,侯九十三人,伯二十一人,子百七十一人,男四百九十七人,凡七百九十六人。附城千五百一十一人。九族之女為任者,八十三人。及漢氏女孫中山承禮君、遵德君、修義君更以為任。十有一公,九卿,十二大夫,二十四元士。定諸國邑采之處,使侍中講禮大夫孔秉等與州部眾郡曉知地理圖籍者,共校治于壽成朱鳥堂。予數與群公祭酒上卿親聽視,咸已通矣。夫褒德賞功,所以顯仁賢也;九族和睦,所以褒親親也。予永惟匪解,思稽前人,將章黜陟,以明好惡,安元元焉。」以圖簿未定,未授國邑,且令受奉都內,月錢數千。諸侯皆困乏,至有庸作者。

中郎區博諫莽曰:「井田雖聖王法,其廢久矣。周道既衰,而民不從。秦知順民之心,可以獲大利也,故滅廬井而置阡陌,遂王諸夏,訖今海內未厭其敝。今欲違民心,追復千載絕跡,雖堯舜復起,而無百年之漸,弗能行也。天下初定,萬民新附,誠未可施行。」莽知民怨,乃下書曰:「諸名食王田,皆得賣之,勿拘以法。犯私買賣庶人者,且一切勿治。」

初,五威將帥出,改句町王以為侯,王邯怨怒不附。莽諷牂柯大尹周歆詐殺邯。邯弟承起兵攻殺歆。先是,莽發高句驪兵,當伐胡,不欲行,郡強迫之,皆亡出塞,因犯法為寇。遼西大尹田譚追擊之,為所殺。州郡歸咎於高句驪侯騶。嚴尤奏言:「貉人犯法,不從騶起,正有它心,宜令州郡且尉安之。今猥被以大罪,恐其遂畔,夫餘之屬必有和者。匈奴未克,夫餘、穢貉復起,此大憂也。」莽不尉安,穢貉遂反,詔尤擊之。尤誘高句驪侯騶至而斬焉,傳首長安。莽大說,下書曰:「乃者,命遣猛將,共行天罰,誅滅虜知,分為十二部,或斷其右臂,或斬其左腋,或潰其胸腹,或紬其兩脅。今年刑在東方,誅貉之部先縱焉。捕斬虜騶,平定東域,虜知殄滅,在于漏刻。此乃天地群神社稷宗廟佑助之福,公卿大夫士民同心將率虓虎之力也。予甚嘉之。其更名高句驪為下句驪,布告天下,令咸知焉。」於是貉人愈犯邊,東北與西南夷皆亂云。

莽志方盛,以為四夷不足吞滅,專念稽古之事,復下書曰:「伏念予之皇始祖考虞帝,受終文祖,在璇璣玉衡以齊七政,遂類于上帝,禋于六宗,望秩于山川,遍于群神,巡狩五嶽,群后四朝,敷奏以言,明試以功。予之受命即真,到于建國五年,已五載矣。陽九之阨既度,百六之會已過。歲在壽星,填在明堂,倉龍癸酉,德在中宮。觀晉掌歲,龜策告從,其以此年二月建寅之節東巡狩,具禮儀調度。」群公奏請募吏民人馬布帛綿,又請內郡國十二買馬,發帛四十五萬匹,輸常安,前後毋相須。至者過半,莽下書曰:「文母太后體不安,其且止待後。」

是歲,改十一公號,以「新」為「心」,後又改「心」為「信」。

五年二月,文母皇太后崩,葬渭陵,與元帝合而溝絕之。立廟於長安,新室世世獻祭。元帝配食,坐於床下。莽為太后服喪三年。

大司馬孔永乞骸骨,賜安車駟馬,以特進就朝位。同風侯逯並為大司馬。

是時,長安民聞莽欲都雒陽,不肯繕治室宅,或頗徹之。莽曰:「玄龍石文曰『定帝德,國雒陽』。符命著明,敢不欽奉!以始建國八年,歲纏星紀,在雒陽之都。其謹繕脩常安之都,勿令壞敗。敢有犯者,輒以名聞,請其罪。」

是歲,烏孫大小昆彌遣使貢獻。大昆彌者,中國外孫也。其胡婦子為小昆彌,而烏孫歸附之。莽見匈奴諸邊並侵,意欲得烏孫心,乃遣使者引小昆彌使置大昆彌使上。保成師友祭酒滿昌劾奏使者曰:「

夷狄以中國有禮誼,故詘而服從。大昆彌,君也,今序臣使於君使之上,非所以有夷狄也。奉使大不敬!」莽怒,免昌官。

西域諸國以莽積失恩信,焉耆先畔,殺都護但欽。

十一月,彗星出,二十餘日,不見。

是歲,以犯挾銅炭者多,除其法。

明年改元曰天鳳。

天鳳元年正月,赦天下。

莽曰:「予以二月建寅之節行巡狩之禮,太官齎糒乾肉,內者行張坐臥,所過毋得有所給。予之東巡,必躬載耒,每縣則耕,以勸東作。予之南巡,必躬載耨,每縣則薅,以勸南偽。予之西巡,必躬載銍,每縣則穫,以勸西成。予之北巡,必躬載拂,每縣則粟,以勸蓋藏。畢北巡狩之禮,即于土中居雒陽之都焉。敢有趨讙犯法,輒以軍法從事。」群公奏言:「皇帝至孝,往年文母聖體不豫,躬親供養,衣冠稀解。因遭棄群臣悲哀,顏色未復,飲食損少。今一歲四巡,道路萬里,春秋尊,非糒乾肉之所能堪。且無巡狩,須闋大服,以安聖體。臣等盡力養牧兆民,奉稱明詔。」莽曰:「群公、群牧、群司、諸侯、庶尹願盡力相帥養牧兆民,欲以稱予,繇此敬聽,其勗之哉!毋食言焉。更以天鳳七年,歲在大梁,倉龍庚辰,行巡狩之禮。厥明年,歲在實沈,倉龍辛巳,即土之中雒陽之都。」乃遣太傅平晏、大司空王邑之雒陽,營相宅兆,圖起宗廟、社稷、郊兆云。

三月壬申晦,日有食之。大赦天下。策大司馬逯並曰:「日食無光,干戈不戢,其上大司馬印韍,就侯氏朝位。太傅平晏勿領尚書事,省侍中諸曹兼官者。以利苗男訢為大司馬。」

莽即真,尤備大臣,抑奪下權,朝臣有言其過失者,輒拔擢。孔仁、趙博、費興等以敢擊大臣,故見信任,擇名官而居之。公卿入宮,吏有常數,太傅平晏從吏過例,掖門僕射苛問不遜,戊曹士收繫僕射。莽大怒,使執法發車騎數百圍太傅府,捕士,即時死。大司空士夜過奉常亭,亭長苛之,告以官名,亭長醉曰:「寧有符傳邪?」士以馬箠擊亭長,亭長斬士,亡,郡縣逐之。家上書,莽曰:「亭長奉公,勿逐。」大司空邑斥士以謝。國將哀章頗不清,莽為選置和叔,敕曰:「非但保國將閨門,當保親屬在西州者。」諸公皆輕賤,而章尤甚。

四月,隕霜,殺屮木,海瀕尤甚。六月,黃霧四塞。七月,大風拔樹,飛北闕直城門屋瓦。雨雹,殺牛羊。

莽以周官、王制之文,置卒正、連率、大尹,職如太守;屬令、屬長,職如都尉。置州牧、部監二十五人。見禮如三公。監位上大夫,各主五郡。公氏作牧,侯氏卒正,伯氏連率,子氏屬令,男氏屬長,皆世其官,其無爵者為尹。分長安城旁六鄉,置帥各一人。分三輔為六尉郡,河東、河內、弘農、河南、潁川、南陽為六隊郡,置大夫,職如太守;屬正,職如都尉。更名河南大尹曰保忠信卿。益河南屬縣滿三十。置六郊州長各一人,人主五縣。及它官名悉改。大郡至分為五。郡縣以亭為名者三百六十,以應符命文也。緣邊又置竟尉,以男為之。諸侯國閒田,為黜陟增減云。莽下書曰:「常安西都曰六鄉,眾縣曰六尉。義陽東都曰六州,眾縣曰六隊。粟米之內曰內郡,其外曰近郡。有鄣徼者曰邊郡。合百二十有五郡。九州之內,縣二千二百有三。公作甸服,是為惟城;諸在侯服,是為惟寧;在采、任諸侯,是為惟翰;在賓服,是為惟屏;在揆文教,奮武衛,是為惟垣;在九州之外,是為惟藩:各以其方為稱,總為萬國焉。」其後,歲復變更,一郡至五易名,而還復其故。吏民不能紀,每下詔書,輒繫其故名,曰:「制詔陳留大尹、太尉:其以益歲以南付新平。新平,故淮陽。以雍丘以東付陳定。陳定,故梁郡。以封丘以東付治亭。治亭,故東郡。以陳留以西付祈隧。祈隧,故滎陽。陳留已無復有郡矣。大尹、太尉,皆詣行在所。」其號令變易,皆此類也。

令天下小學,戊子代甲子為六旬首。冠以戊子為元日,昏以戊寅之旬為忌日。百姓多不從者。

匈奴單于知死,弟咸立為單于,求和親。莽遣使者厚賂之,詐許還其侍子登,因購求陳良、終帶等。單于即執良等付使者,檻車詣長安。莽燔燒良等於城北,令吏民會觀之。

緣邊大飢,人相食。諫大夫如普行邊兵,還言「軍士久屯塞苦,邊郡無以相贍。今單于新和,宜因是罷兵。」校尉韓威進曰:「以新室之威而吞胡虜,無異口中蚤蝨。臣願得勇敢之士五千人,不齎斗糧,飢食虜肉,渴飲其血,可以橫行。」莽壯其言,以威為將軍。然采普言,徵還諸將在邊者。免陳欽等十八人,又罷四關填都尉諸屯兵。會匈奴使還,單于知侍子登前誅死,發兵寇邊,莽復發軍屯。於是邊民流入內郡,為人奴婢,乃禁吏民敢挾邊民者棄市。

益州蠻夷殺大尹程隆,三邊盡反。遣平蠻將軍馬茂將兵擊之。

寧始將軍侯輔免,講易祭酒戴參為寧始將軍。

二年二月,置酒王路堂,公卿大夫皆佐酒。大赦天下。

是時,日中見星。

大司馬苗訢左遷司命,以延德侯陳茂為大司馬。

訛言黃龍墮死黃山宮中,百姓奔走往觀者有萬數。莽惡之,捕繫問語所從起,不能得。

單于咸既和親,求其子登屍,莽欲遣使送致,恐咸怨恨害使者,乃收前言當誅侍子者故將軍陳欽,以他罪繫獄。欽曰:「是欲以我為說於匈奴也。」遂自殺。莽選儒生能顓對者濟南王咸為大使,五威將琅邪伏黯等為帥,使送登屍。敕令掘單于知墓,棘鞭其屍。又令匈奴卻塞於漠北,責單于馬萬匹,牛三萬頭,羊十萬頭,及稍所略邊民生口在者皆還之。莽好為大言如此。咸到單于庭,陳莽威德,責單于背畔之罪,應敵從橫,單于不能詘,遂致命而還之。入塞,咸病死,封其子為伯,伏黯等皆為子。

莽意以為制定則天下自平,故銳思於地里,制禮作樂,講合六經之說。公卿旦入暮出,議論連年不決,不暇省獄訟冤結民之急務。縣宰缺者,數年守兼,一切貪殘日甚。中郎將、繡衣執法在郡國者,並乘權勢,傳相舉奏。又十一公士分布勸農桑,班時令,案諸章,冠蓋相望,交錯道路,召會吏民,逮捕證左,郡縣賦斂,遞相賕賂,白黑紛然,守闕告訴者多。莽自見前顓權以得漢政,故務自髓眾事,有司受成苟免。諸寶物名、帑藏、錢穀官,皆宦者領之;吏民上封事書,宦官左右開發,尚書不得知。其畏備臣下如此。又好變改制度,政令煩多,當奏行者,輒質問乃以從事,前後相乘,憒眊不渫。莽常御燈火至明,猶不能勝。尚書因是為姦寢事,上書待報者連年不得去,拘繫郡縣者逢赦而後出,衛卒不交代三歲矣。穀常貴,邊兵二十餘萬人仰衣食,縣官愁苦。五原、代郡尤被其毒,起為盜賊,數千人為輩,轉入旁郡。莽遣捕盜將軍孔仁將兵與郡縣合擊,歲餘乃定,邊郡亦略將盡。

邯鄲以北大雨霧,水出,深者數丈,流殺數千人。

立國將軍孫建死,司命趙閎為立國將軍。寧始將軍戴參歸故官,南城將軍廉丹為寧始將軍。

三年二月乙酉,地震,大雨雪,關東尤甚,深者一丈,竹柏或枯。大司空王邑上書言:「視事八年,功業不效,司空之職尤獨廢頓,至乃有地震之變。願乞骸骨。」莽曰:「夫地有動有震,震者有害,動者不害。春秋記地震,易繫坤動,動靜辟脅,萬物生焉。災異之變,各有云為。天地動威,以戒予躬,公何辜焉,而乞骸骨,非所以助予者也。使諸吏散騎司祿大衛脩寧男遵諭予意焉。」

五月,莽下吏祿制度,曰:「予遭陽九之阨,百六之會,國用不足,民人騷動,自公卿以下,一月之祿十愑布二匹,或帛一匹。予每念之,未嘗不戚焉。今阨會已度,府帑雖未能充,略頗稍給,其以六月朔庚寅始,賦吏祿皆如制度。」四輔公卿大夫士,下至輿僚,凡十五等。僚祿一歲六十六斛,稍以差增,上至四輔而為萬斛云。莽又曰:「『普天之下,莫非王土;率土之賓,莫非王臣。』蓋以天下養焉。周禮膳羞百有二十品,今諸侯各食其同、國、則;辟、任、附城食其邑;公、卿、大夫、元士食其采。多少之差,咸有條品。歲豐穰則充其禮,有災害則有所損,與百姓同憂喜也。其用上計時通計,天下幸無災害者,太官膳羞備其品矣;即有災害,以什率多少而損膳焉。東嶽太師立國將軍保東方三州一部二十五郡;南嶽太傅前將軍保南方二州一部二十五郡;西嶽國師寧始將軍保西方一州二部二十五郡;北嶽國將衛將軍保北方二州一部二十五郡;大司馬保納卿、言卿、仕卿、作卿、京尉、扶尉、兆隊、右隊、中部左洎前七部;大司徒保樂卿、典卿、宗卿、秩卿、翼尉、光尉、左隊、前隊、中部、右部,有五郡;大司空保予卿、虞卿、共卿、工卿、師尉、列尉、祈隊、後隊、中部洎後十郡;及六司,六卿,皆隨所屬之公保其災害,亦以十率多少而損其祿。郎、從官、中都官吏食祿都內之委者,以太官膳羞備損而為節。諸侯、辟、任、附城、群吏亦各保其災害。幾上下同心,勸進農業,安元元焉。」莽之制度煩碎如此,課計不可理,吏終不得祿,各因官職為姦,受取賕賂以自共給。

是月戊辰,長平館西岸崩,邕涇水不流,毀而北行。遣大司空王邑行視,還奏狀,群臣上壽,以為河圖所謂「以土填水」,匈奴滅亡之祥也。乃遣并州牧宋弘、游擊都尉任萌等將兵擊匈奴,至邊止屯。

七月辛酉,霸城門災,民間所謂青門也。

戊子晦,日有食之。大赦天下。復令公卿大夫諸侯二千石舉四行各一人。大司馬陳茂以日食免,武建伯嚴尤為大司馬。

十月戊辰,王路朱鳥門鳴,晝夜不絕,崔發等曰:「虞帝闢四門,通四聰。門鳴者,明當修先聖之禮,招四方之士也。」於是令群臣皆賀,所舉四行從朱鳥門入而對策焉。

平蠻將軍馮茂擊句町,士卒疾疫,死者什六七,賦斂民財什取五,益州虛耗而不克,徵還下獄死。更遣寧始將軍廉丹與庸部牧史熊擊句町,頗斬首,有勝。莽徵丹、熊,丹、熊願益調度,必克乃還。復大賦斂,就都大尹馮英不肯給,上言「自越巂遂久仇牛、同亭邪豆之屬反畔以來,積且十年,郡縣距擊不已。續用馮茂,苟施一切之政。僰道以南,山險高深,茂多敺眾遠居,費以億計,吏士離毒氣死者什七。今丹、熊懼於自詭期會,調發諸郡兵穀,復訾民取其十四,空破梁州,功終不遂。宜罷兵屯田,明設購賞。」莽怒,免英官。後頗覺寤,曰:「英亦未可厚非。」復以英為長沙連率。

翟義黨王孫慶捕得,莽使太醫、尚方與巧屠共刳剝之,量度五藏,以竹筳導其脈,知所終始,云可以治病。

是歲,遣大使五威將王駿、西域都護李崇將戊己校尉出西域,諸國皆郊迎貢獻焉。諸國前殺都護但欽,駿欲襲之,命佐帥何封、戊己校尉郭欽別將。焉耆詐降,伏兵擊駿等,皆死。欽、封後到,襲擊老弱,從車師還入塞。莽拜欽為填外將軍,封劋胡子,何封為集胡男。西域自此絕。

四年五月,莽曰:「保成師友祭酒唐林、故諫議祭酒琅邪紀逡,孝弟忠恕,敬上愛下,博通舊聞,德行醇備,至於黃髮,靡有愆失。其封林為建德侯,逡為封德侯,位皆特進,見禮如三公。賜弟一區,錢三百萬,授几杖焉。」

六月,更授諸侯茅土於明堂,曰:「予制作地理,建封五等,考之經藝,合之傳記,通於義理,論之思之,至於再三,自始建國之元以來九年于茲,乃今定矣。予親設文石之平,陳菁茅四色之土,欽告于岱宗泰社后土、先祖先妣,以班授之。各就厥國,養牧民人,用成功業。其在緣邊,若江南,非詔所召,遣侍于帝城者,納言掌貨大夫且調都內故錢,予其祿,公歲八十萬,侯伯四十萬,子男二十萬。」然復不能盡得。莽好空言,慕古法,多封爵人,性實遴嗇,託以地理未定,故且先賦茅土,用慰喜封者。

是歲,復明六筦之令。每一筦下,為設科條防禁,犯者罪至死,吏民抵罪者浸眾。又一切調上公以下諸有奴婢者,率一口出錢三千六百,天下愈愁,盜賊起。納言馮常以六筦諫,莽大怒,免常官。置執法左右刺姦。選用能吏侯霸等分督六尉、六隊,如漢刺史,與三公士郡一人從事。

臨淮瓜田儀等為盜賊,依阻會稽長州,琅邪女子呂母亦起。初,呂母子為縣吏,為宰所冤殺。母散家財,以酤酒買兵弩,陰厚貧窮少年,得百餘人,遂攻海曲縣,殺其宰以祭子墓。引兵入海,其眾浸多,後皆萬數。莽遣使者即赦盜賊,還言「盜賊解,輒復合。問其故,皆曰愁法禁煩苛,不得舉手。力作所得,不足以給貢稅。閉門自守,又坐鄰伍鑄錢挾銅,姦吏因以愁民。民窮,悉起為盜賊。」莽大怒,免之。其或順指,言「民驕黠當誅」,及言「時運適然,且滅不久」,莽說,輒遷之。

是歲八月,莽親之南郊,鑄作威斗。威斗者,以五石銅為之,若北斗,長二尺五寸,欲以厭勝眾兵。既成,令司命負之,莽出在前,入在御旁。鑄斗日,大寒,百官人馬有凍死者。

五年正月朔,北軍南門災。

以大司馬司允費興為荊州牧,見,問到部方略,興對曰:「荊、揚之民率依阻山澤,以漁采為業。間者,國張六筦,稅山澤,妨奪民之利,連年久旱,百姓飢窮,故為盜賊。興到部,欲令明曉告盜賊歸田里,假貸犁牛種食,闊其租賦,幾可以解釋安集。」莽怒,免興官。

天下吏以不得奉祿,並為姦利,郡尹縣宰家累千金。莽下詔曰:「詳考始建國二年胡虜猾夏以來,諸軍吏及緣邊吏大夫以上為姦利增產致富者,收其家所有財產五分之四,以助邊急。」公府士馳傳天下,考覆貪饕,開吏告其將,奴婢告其主,幾以禁姦,姦愈甚。

皇孫功崇公宗坐自畫容貌,被服天子衣冠,刻印三:一曰「維祉冠存己夏處南山臧薄冰」,二曰「肅聖寶繼」,三曰「

德封昌圖」。又宗舅呂寬家前徙合浦,私與宗通,發覺按驗,宗自殺。莽曰:「宗屬為皇孫,爵為上公,知寬等叛逆族類,而與交通;刻銅印三,文意甚害,不知厭足,窺欲非望。春秋之義,『君親毋將,將而誅焉。』迷惑失道,自取此辜,烏呼哀哉!宗本名會宗,以制作去二名,今復名會宗。貶厥爵,改厥號,賜諡為功崇繆伯,以諸伯之禮葬于故同穀城郡。」宗姊妨為衛將軍王興夫人,祝詛姑,殺婢以絕口。事發覺,莽使中常侍鹊惲責問妨,并以責興,皆自殺。事連及司命孔仁妻,亦自殺。仁見莽免冠謝,莽使尚書劾仁:「乘乾車,駕巛馬,左蒼龍,右白虎,前朱雀,後玄武,右杖威節,左負威斗,號曰赤星,非以驕仁,乃以尊新室之威命也。仁擅免天文冠,大不敬。」有詔勿劾,更易新冠。其好怪如此。

以直道侯王涉為衛將軍。涉者,曲陽侯根子也。根,成帝世為大司馬,薦莽自代,莽恩之,以為曲陽非令稱,乃追諡根曰直道讓公,涉嗣其爵。

是歲,赤眉力子都、樊崇等以饑饉相聚,起於琅邪,轉鈔掠,眾皆萬數。遣使者發郡國兵擊之,不能克。

六年春,莽見盜賊多,乃令太史推三萬六千歲曆紀,六歲一改元,布天下。下書曰:「紫閣圖曰『太一、黃帝皆僊上天,張樂崑崙虔山之上。後世聖主得瑞者,當張樂秦終南山之上。』予之不敏,奉行未明,乃今諭矣。復以寧始將軍為更始將軍,以順符命。易不云乎?『日新之謂盛德,生生之謂易。』予其饗哉!」欲以誑燿百姓,銷解盜賊。眾皆笑之。

初獻新樂於明堂、太廟。群臣始冠麟韋之弁。或聞其樂聲,曰:「清厲而哀,非興國之聲也。」

是時,關東饑旱數年,力子都等黨眾浸多。更始將軍廉丹擊益州不能克,徵還。更遣復位後大司馬護軍郭興、庸部牧李嘱擊蠻夷若豆等,太傅犧叔士孫喜清潔江湖之盜賊。而匈奴寇邊甚。莽乃大募天下丁男及死罪囚、吏民奴,名曰豬突豨勇,以為銳卒。一切稅天下吏民,訾三十取一,縑帛皆輸長安。令公卿以下至郡縣黃綬皆保養軍馬,多少各以秩為差。又博募有奇技術可以攻匈奴者,將待以不次之位。言便宜者以萬數:或言能度水不用舟楫,連馬接騎,濟百萬師;或言不持斗糧,服食藥物,三軍不飢;或言能飛,一日千里,可窺匈奴。莽輒試之,取大鳥翮為兩翼,頭與身皆著毛,通引環紐,飛數百步墮。莽知其不可用,苟欲獲其名,皆拜為理軍,賜以車馬,待發。

初,匈奴右骨都侯須卜當,其妻王昭君女也,嘗內附。莽遣昭君兄子和親侯王歙誘呼嘗至塞下,脅將詣長安,強立以為須卜善于後安公。始欲誘迎當,大司馬嚴尤諫曰:「當在匈奴右部,兵不侵邊,單于動靜,輒語中國,此方面之大助也。于今迎當置長安槁街,一胡人耳,不如在匈奴有益。」莽不聽。既得當,欲遣尤與廉丹擊匈奴,皆賜姓徵氏,號二徵將軍,當誅單于輿而立當代之。出車城西橫廄,未發。尤素有智略,非莽攻伐西夷,數諫不從,著古名將樂毅、白起不用之意及言邊事凡三篇,奏以風諫莽。及當出廷議,尤固言匈奴可且以為後,先憂山東盜賊。莽大怒,乃策尤曰:「視事四年,蠻夷猾夏不能遏絕,寇賊姦宄不能殄滅,不畏天威,不用詔命,貌佷自臧,持必不移,懷執異心,非沮軍議。未忍致于理,其上大司馬武建伯印韍,歸故郡。」以降符伯董忠為大司馬。

翼平連率田況奏郡縣訾民不實,莽復三十稅一。以況忠言憂國,進爵為伯,賜錢二百萬。眾庶皆詈之。青、徐民多棄鄉里流亡,老弱死道路,壯者入賊中。

夙夜連率韓博上言:「有奇士,長丈,大十圍,來至臣府,曰欲奮擊胡虜。自謂巨毋霸,出於蓬萊東南,五城西北昭如海瀕,軺車不能載,三馬不能勝。即日以大車四馬,建虎旗,載霸詣闕。霸臥則枕鼓,以鐵箸食,此皇天所以輔新室也。願陛下作大甲高車,賁育之衣,遣大將一人與虎賁百人迎之於道。京師門戶不容者,開高大之,以視百蠻,鎮安天下。」博意欲以風莽。莽聞惡之,留霸在所新豐,更其姓曰巨母氏,謂因文母太后而霸王符也。徵博下獄,以非所宜言,棄市。

明年改元曰地皇,從三萬六千歲曆號也。

地皇元年正月乙未,赦天下。下書曰:「方出軍行師,敢有趨讙犯法者,輒論斬,毋須時,盡歲止。」於是春夏斬人都市,百姓震懼,道路以目。

二月壬申,日正黑。莽惡之,下書曰:「乃者日中見昧,陰薄陽,黑氣為變,百姓莫不驚怪。兆域大將軍王匡遣吏考問上變事者,欲蔽上之明,是以適見于天,以正于理,塞大異焉。」

莽見四方盜賊多,復欲厭之,又下書曰:「予之皇初祖考黃帝定天下,將兵為上將軍,建華蓋,立斗獻,內設大將,外置大司馬五人,大將軍二十五人,偏將軍百二十五人,裨將軍千二百五十人,校尉萬二千五百人,司馬三萬七千五百人,候十一萬二千五百人,當百二十二萬五千人,士吏四十五萬人,士千三百五十萬人,應協於易『弧矢之利,以威天下』。予受符命之文,稽前人,將條備焉。」於是置前後左右中大司馬之位,賜諸州牧號為大將軍,郡卒正、連帥、大尹為偏將軍,屬令長裨將軍,縣宰為校尉。乘傳使者經歷郡國,日且十輩,倉無見穀以給,傳車馬不能足,賦取道中車馬,取辦於民。

七月,大風毀王路堂。復下書曰:「乃壬午餔時,有列風雷雨發屋折木之變,予甚弁焉,予甚栗焉,予甚恐焉。伏念一旬,迷乃解矣。昔符命文立安為新遷王,臨國雒陽,為統義陽王。是時予在攝假,謙不敢當,而以為公。其後金匱文至,議者皆曰:『臨國雒陽為統,謂據土中為新室統也,宜為皇太子。』自此後,臨久病,雖瘳不平,朝見挈茵輿行。見王路堂者,張於西廂及後閣更衣中,又以皇后被疾,臨且去本就舍,妃妾在東永巷。壬午,列風毀王路西廂及後閣更衣中室。昭寧堂池東南榆樹大十圍,東僵,擊東閣,閣即東永巷之西垣也。皆破折瓦壞,發屋拔木,予甚驚焉。又候官奏月犯心前星,厥有占,予甚憂之。伏念紫閣圖文,太一、黃帝皆得瑞以僊,後世褒主當登終南山。所謂新遷王者,乃太一新遷之後也。統義陽王乃用五統以禮義登陽上遷之後也。臨有兄而稱太子,名不正。宣尼公曰:『名不正,則言不順,至於刑罰不中,民無錯手足。』惟即位以來,陰陽未和,風雨不時,數遇枯旱蝗螟為災,穀稼鮮耗,百姓苦飢,蠻夷猾夏,寇賊姦宄,人民正營,無所錯手足。深惟厥咎,在名不正焉。其立安為新遷王,臨為統義陽王,幾以保全二子,子孫千億,外攘四夷,內安中國焉。」

是月,杜陵便殿乘輿虎文衣廢臧在室匣中者出,自樹立外堂上,良久乃委地。吏卒見者以聞,莽惡之,下書曰:「寶黃廝赤,其令郎從官皆衣絳。」

望氣為數者多言有土功象,莽又見四方盜賊多,欲視為自安能建萬世之基者,乃下書曰:「予受命遭陽九之厄,百六之會,府帑空虛,百姓匱乏,宗廟未修,且祫祭於明堂太廟,夙夜永念,非敢寧息。深惟吉昌莫良於今年,予乃卜波水之北,郎池之南,惟玉食。予又卜金水之南,明堂之西,亦惟玉食。予將新築焉。」於是遂營長安城南,提封百頃。九月甲申,莽立載行視,親舉築三下。司徒王尋、大司空王邑持節,及侍中常侍執法杜林等數十人將作。崔發、張邯說莽曰:「德盛者文縟,宜崇其制度,宣視海內,且令萬世之後無以復加也。」莽乃博徵天下工匠諸圖畫,以望法度算,及吏民以義入錢穀助作者,駱驛道路。壞徹城西苑中建章、承光、包陽、大臺、儲元宮及平樂、當路、陽祿館,凡十餘所,取其材瓦,以起九廟。是月,大雨六十餘日。令民入米六百斛為郎,其郎吏增秩賜爵至附城。九廟:一曰黃帝太初祖廟,二曰帝虞始祖昭廟,三曰陳胡王統祖穆廟,四曰齊敬王世祖昭廟,五曰濟北愍王王祖穆廟,凡五廟不墮云;六曰濟南伯王尊禰昭廟,七曰元城孺王尊禰穆廟,八曰陽平頃王戚禰昭廟,九曰新都顯王戚禰穆廟。殿皆重屋。太初祖廟東西南北各四十丈,高十七丈,餘廟半之。為銅薄櫨,飾以金銀琱文,窮極百工之巧。帶高增下,功費數百鉅萬,卒徒死者萬數。

鉅鹿男子馬適求等謀舉燕趙兵以誅莽,大司空士王丹發覺以聞。莽遣三公大夫逮治黨與,連及郡國豪傑數千人,皆誅死。封丹為輔國侯。

自莽為不順時令,百姓怨恨,莽猶安之,又下書曰:「惟設此壹切之法以來,常安六鄉巨邑之都,枹鼓稀鳴,盜賊衰少,百姓安土,歲以有年,此乃立權之力也。今胡虜未滅誅,蠻僰未絕焚,江湖海澤麻沸,盜賊未盡破殄,又興奉宗廟社稷之大作,民眾動搖。今復壹切行此令,盡二年止之,以全元元,救愚姦。」

是歲,罷大小錢,更行貨布,長二寸五分,廣一寸,直貨錢二十五。貨錢徑一寸,重五銖,枚直一。兩品並行。敢盜鑄錢及偏行布貨,伍人知不發舉,皆沒入為官奴婢。

太傅平晏死,以予虞唐尊為太傅。尊曰:「國虛民貧,咎在奢泰。」乃身短衣小袖,乘牝馬柴車,藉槁,瓦器,又以歷遺公卿。出見男女不異路者,尊自下車,以象刑赭幡汙染其衣。莽聞而說之,下詔申敕公卿思與厥齊。封尊為平化侯。

是時,南郡張霸、江夏羊牧、王匡等起雲杜綠林,號曰下江兵,眾皆萬餘人。武功中水鄉民三舍墊為池。

二年正月,以州牧位三公,刺舉怠解,更置牧監副,秩元士,冠法冠,行事如漢刺史。

是月,莽妻死,諡曰孝睦皇后,葬渭陵長壽園西,令永侍文母,名陵曰億年。初莽妻以莽數殺其子,涕泣失明,莽令太子臨居中養焉。莽妻旁侍者原碧,莽幸之。後臨亦通焉,恐事泄,謀共殺莽。臨妻愔,國師公女,能為星,語臨宮中且有白衣會。臨喜,以為所謀且成。後貶為統義陽王,出在外第,愈憂恐。會莽妻病困,臨予書曰:「上於子孫至嚴,前長孫、中孫年俱三十而死。今臣臨復適三十,誠恐一旦不保中室,則不知死命所在!」莽侯妻疾,見其書,大怒,疑臨有惡意,不令得會喪。既葬,收原碧等考問,具服姦、謀殺狀。莽欲祕之,使殺案事使者司命從事,埋獄中,家不知所在。賜臨藥,臨不肯飲,自刺死。使侍中票騎將軍同說侯林賜魂衣璽韍,策書曰:「符命文立臨為統義陽王,此言新室即位三萬六千歲後,為臨之後者乃當龍陽而起。前過聽議者,以臨為太子,有烈風之變,輒順符命,立為統義陽王。在此之前,自此之後,不作信順,弗蒙厥佑,夭年隕命,嗚呼哀哉!跡行賜諡,諡曰繆王。」又詔國師公:「臨本不知星,事從愔起。」愔亦自殺。

是月,新遷王安病死。初,莽為侯就國時,幸侍者增秩、懷能、開明。懷能生男興,增秩生男匡、女嘱,開明生女捷,皆留新都國,以其不明故也。及安疾甚,莽自病無子,為安作奏,使上言:「興等母雖微賤,屬猶皇子,不可以棄。」章視群公,皆曰:「安友于兄弟,宜及春夏加封爵。」於是以王車遣使者迎興等,封興為功脩公,匡為功建公,嘱為睦脩任,捷為睦逮任。孫公明公壽病死,旬月四喪焉。莽壞漢孝武、孝昭廟,分葬子孫其中。

魏成大尹李焉與卜者王況謀,況謂焉曰:「新室即位以來,民田奴婢不得賣買,數改錢貨,徵發煩數,軍旅騷動,四夷並侵,百姓怨恨,盜賊並起,漢家當復興。君姓李,李音徵,徵火也,當為漢輔。」因為焉作讖書,言「文帝發忿,居地下趣軍,北告匈奴,南告越人。江中劉信,執敵報怨,復續古先,四年當發軍。江湖有盜,自稱樊王,姓為劉氏,萬人成行,不受赦令,欲動秦、雒陽。十一年當相攻,太白揚光,歲星入東井,其號當行。」又言莽大臣吉凶,各有日期。會合十餘萬言。焉令吏寫其書,吏亡告之。莽遣使者即捕焉,獄治皆死。

三輔盜賊麻起,乃置捕盜都尉官,令執法謁者追擊長安中,建鳴鼓攻賊幡,而使者隨其後。遣太師犧仲景尚、更始將軍護軍王黨將兵擊青、徐,國師和仲曹放助郭興擊句町。轉天下穀幣詣西河、五原、朔方、漁陽,每一郡以百萬數,欲以擊匈奴。

秋,隕霜殺菽,關東大饑,蝗。

民犯鑄錢,伍人相坐,沒入為官奴婢。其男子檻車,兒女子步,以鐵鎖琅當其頸,傳詣鍾官,以十萬數。到者易其夫婦,愁苦死者什六七。孫喜、景尚、曹放等擊賊不能克,軍師放縱,百姓重困。

莽以王況讖言荊楚當興,李氏為輔,欲厭之,乃拜侍中掌牧大夫李棽為大將軍、揚州牧,賜名聖,使將兵奮擊。

上谷儲夏自請願說瓜田儀,莽以為中郎,使出儀。儀文降,未出而死。莽求其尸葬之,為起冢、祠室,諡曰瓜寧殤男,幾以招來其餘,然無肯降者。

閏月丙辰,大赦天下,天下大服民私服在詔書前亦釋除。

郎陽成脩獻符命,言繼立民母,又曰:「黃帝以百二十女致神僊。」莽於是遣中散大夫、謁者各四十五人分行天下,博采鄉里所高有淑女者上名。

莽夢長樂宮銅人五枚起立,莽惡之,念銅人銘有「皇帝初兼天下」之文,即使尚方工鐫滅所夢銅人膺文。又感漢高廟神靈,遣虎賁武士入高廟,拔劍四面提擊,斧壞戶牖,桃湯赭鞭鞭灑屋壁,令輕車校尉居其中,又令中軍北壘居高寢。

或言黃帝時建華蓋以登僊,莽乃造華蓋九重,高八丈一尺,金瑵羽葆,載以祕機四輪車,駕六馬,力士三百人黃衣幘,車上人擊鼓,輓者皆呼「登僊」。莽出,令在前。百官竊言「此似狈車,非僊物也。」

是歲,南郡秦豐眾且萬人。平原女子遲昭平能說經博以八投,亦聚數千人在河阻中。莽召問群臣禽賊方略,皆曰:「此天囚行尸,命在漏刻。」故左將軍公孫祿徵來與議,祿曰:「太史令宗宣典星曆,候氣變,以凶為吉,亂天文,誤朝廷。太傅平化侯飾虛偽以媮名位,『賊夫人之子』。國師嘉信公顛倒五經,毀師法,令學士疑惑。明學男張邯、地理侯孫陽造井田,使民棄土業。犧和魯匡設六筦,以窮工商。說符侯崔發阿諛取容,令下情不上通。宜誅此數子以慰天下!」又言:「匈奴不可攻,當與和親。臣恐新室憂不在匈奴,而在封域之中也。」莽怒,使虎賁扶祿出。然頗采其言,左遷魯匡為五原卒正,以百姓怨非故。六筦非匡所獨造,莽厭眾意而出之。

初,四方皆以飢寒窮愁起為盜賊,稍稍群聚,常思歲熟得歸鄉里。眾雖萬數,亶稱巨人、從事、三老、祭酒,不敢略有城邑,轉掠求食,日闋而已。諸長吏牧守皆自亂鬥中兵而死,賊非敢欲殺之也,而莽終不諭其故。是歲,大司馬士按章豫州,為賊所獲,賊送付縣。士還,上書具言狀。莽大怒,下獄以為誣罔。因下書責七公曰:「夫吏者,理也。宣德明恩,以牧養民,仁之道也。抑強督姦,捕誅盜賊,義之節也。今則不然。盜發不輒得,至成群黨,遮略乘傳宰士。士得脫者,又妄自言『我責數賊「何故為是?」賊曰「以貧窮故耳」。賊護出我。』今俗人議者率多若此。惟貧困飢寒,犯法為非,大者群盜,小者偷穴,不過二科,今乃結謀連黨以千百數,是逆亂之大者,豈飢寒之謂邪?七公其嚴敕卿大夫、卒正、連率、庶尹,謹牧養善民,急捕殄盜賊。有不同心并力,疾惡黜賊,而妄曰飢寒所為,輒捕繫,請其罪。」於是群下愈恐,莫敢言賊情者,亦不得擅發兵,賊由是遂不制。

唯翼平連率田況素果敢,發民年十八以上四萬餘人,授以庫兵,與刻石為約。赤糜聞之,不敢入界。況自劾奏,莽讓況:「未賜虎符而擅發兵,此弄兵也,厥罪乏興。以況自詭必禽滅賊,故且勿治。」後況自請出界擊賊,所嚮皆破。莽以璽書令況領青、徐二州牧事。況上言:「盜賊始發,其原甚微,非部吏、伍人所能禽也。咎在長吏不為意,縣欺其郡,郡欺朝廷,實百言十,實千言百。朝廷忽略,不輒督責,遂至延曼連州,乃遣將率,多發使者,傳相監趣。郡縣力事上官,應塞詰對,共酒食,具資用,以救斷斬,不給復憂盜賊治官事。將率又不能躬率吏士,戰則為賊所破,吏氣寖傷,徒費百姓。前幸蒙赦令,賊欲解散,或反遮擊,恐入山谷轉相告語,故郡縣降賊,皆更驚駭,恐見詐滅,因饑饉易動,旬日之間更十餘萬人,此盜賊所以多之故也。今雒陽以東,米石二千。竊見詔書,欲遣太師、更始將軍,二人爪牙重臣,多從人眾,道上空竭,少則亡以威視遠方。宜急選牧、尹以下,明其賞罰,收合離鄉。小國無城郭者,徙其老弱置大城中,積藏穀食,并力固守。賊來攻城,剛不能下,所過無食,勢不得群聚。如此,招之必降,擊之則滅。今空復多出將率,郡縣苦之,反甚於賊。宜盡徵還乘傳諸使者,以休息郡縣。委任臣況以二州盜賊,必平定之。」莽畏惡況,陰為發代,遣使者賜況璽書。使者至,見況,因令代監其兵。況隨使者西,到,拜為師尉大夫。況去,齊地遂敗。

三年正月,九廟蓋構成,納神主。莽謁見,大駕乘六馬,以五采毛為龍文衣,著角,長三尺。華蓋車,元戎十乘在前。因賜治廟者司徒、大司空錢各千萬,侍中、中常侍以下皆封。封都匠仇延為邯淡里附城。

二月,霸橋災,數千人以水沃救,不滅。莽惡之,下書曰:「夫三皇象春,五帝象夏,三王象秋,五伯象冬。皇王,德運也;伯者,繼空續乏以成曆數,故其道駮。惟常安御道多以所近為名。乃二月癸巳之夜,甲午之辰,火燒霸橋,從東方西行,至甲午夕,橋盡火滅。大司空行視考問,或云寒民舍居橋下,疑以火自燎,為此災也。其明旦即乙未,立春之日也。予以神明聖祖黃虞遺統受命,至于地皇四年為十五年。正以三年終冬絕滅霸駮之橋,欲以興成新室統壹長存之道也。又戒此橋空東方之道。今東方歲荒民飢,道路不通,東岳太師亟科條,開東方諸倉,賑貸窮乏,以施仁道。其更名霸館為長存館,霸橋為長存橋。」

是月,赤眉殺太師犧仲景尚。關東人相食。

四月,遣太師王匡、更始將軍廉丹東,祖都門外,天大雨,霑衣止。長老歎曰:「是為泣軍!」莽曰:「惟陽九之阨,與害氣會,究于去年。枯旱霜蝗,飢饉薦臻,百姓困乏,流離道路,于春尤甚,予甚悼之。今使東嶽太師特進褒新侯開東方諸倉,賑貸窮乏。太師公所不過道,分遣大夫謁者並開諸倉,以全元元。太師公因與廉丹大使五威司命位右大司馬更始將軍平均侯之兗州,填撫所掌,及青、徐故不軌盜賊未盡解散,後復屯聚者,皆清潔之,期於安兆黎矣。」太師、更始合將銳士十餘萬人,所過放縱。東方為之語曰:「寧逢赤眉,不逢太師!太師尚可,更始殺我!」卒如田況之言。

莽又多遣大夫謁者分教民煮草木為酪,酪不可食,重為煩費。莽下書曰:「惟民困乏,雖溥開諸倉以賑贍之,猶恐未足。其且開天下山澤之防,諸能采取山澤之物而順月令者,其恣聽之,勿令出稅。至地皇三十年如故,是王光上戊之六年也。如令豪吏猾民辜而攉之,小民弗蒙,非予意也。易不云乎?『損上益下,民說無疆。』《書》云:『言之不從,是謂不艾。』咨虖群公,可不憂哉!」

是時下江兵盛,新巿朱鮪、平林陳牧等皆復聚眾,攻擊鄉聚。莽遣司命大將軍孔仁部豫州,納言大將軍嚴尤、秩宗大將軍陳茂擊荊州,各從吏士百餘人,乘船從渭入河,至華陰乃出乘傳,到部募士。尤謂茂曰:「遣將不與兵符,必先請而後動,是猶紲韓盧而責之獲也。」

夏,蝗從東方來,蜚蔽天,至長安,入未央宮,緣殿閣。莽發吏民設購賞捕擊。

莽以天下穀貴,欲厭之,為大倉,置衛交戟,名曰「政始掖門」。

流民入關者數十萬人,乃置養贍官稟食之。使者監領,與小吏共盜其稟,飢死者十七八。先是,莽使中黃門王業領長安巿買,賤取於民,民甚患之。業以省費為功,賜爵附城。莽聞城中飢饉,以問業。業曰:「皆流民也。」乃巿所賣粱丽肉羹,持入視莽,曰:「居民食咸如此。」莽信之。

冬,無鹽索盧恢等舉兵反城。廉丹、王匡攻拔之,斬首萬餘級。莽遣中郎將奉璽書勞丹、匡,進爵為公,封吏士有功者十餘人。

赤眉別校董憲等眾數萬人在梁郡,王匡欲進擊之,廉丹以為新拔城罷勞,當且休士養威。匡不聽,引兵獨進,丹隨之。合戰成昌,兵敗,匡走。丹使吏持其印韍符節付匡曰:「小兒可走,吾不可!」遂止,戰死。校尉汝雲、王隆等二十餘人別鬥,聞之,皆曰:「廉公已死,吾誰為生?」馳奔賊,皆戰死。莽傷之,下書曰:「惟公多擁選士精兵,眾郡駿馬倉穀帑藏皆得自調,忽於詔策,離其威節,騎馬呵譟,為狂刃所害,烏呼哀哉!賜諡曰果公。」

國將哀章謂莽曰:「皇祖考黃帝之時,中黃直為將,破殺蚩尤。今臣居中黃直之位,願平山東。」莽遣章馳東,與太師匡并力。又遣大將軍陽浚守敖倉,司徒王尋將十餘萬屯雒陽填南宮,大司馬董忠養士習射中軍北壘,大司空王邑兼三公之職。司徒尋初發長安,宿霸昌廄,亡其黃鉞。尋士房揚素狂直,乃哭曰:「此經所謂『喪其齊斧』者也!」自劾去。莽擊殺揚。

四方盜賊往往數萬人攻城邑,殺二千石以下。太師王匡等戰數不利。莽知天下潰畔,事窮計迫,乃議遣風俗大夫司國憲等分行天下,除井田奴婢山澤六筦之禁,即位以來詔令不便於民者皆收還之。待見未發,會世祖與兄齊武王伯升、宛人李通等帥舂陵子弟數千人,招致新巿平林朱鮪、陳牧等合攻拔棘陽。是時嚴尤、陳茂破下江兵,成丹、王常等數千人別走,入南陽界。

十一月,有星孛于張,東南行,五日不見。莽數召問太史令宗宣,諸術數家皆繆對,言天文安善,群賊且滅。莽差以自安。

四年正月,漢兵得下江王常等以為助兵,擊前隊大夫甄阜、屬正梁丘賜,皆斬之,殺其眾數萬人。初,京師聞青、徐賊眾數十萬人,訖無文號旌旗表識,咸怪異之。好事者竊言:「此豈如古三皇無文書號諡邪?」莽亦心怪,以問群臣,群臣莫對。唯嚴尤曰:「此不足怪也。自黃帝、湯、武行師,必待部曲旌旗號令,今此無有者,直飢寒群盜,犬羊相聚,不知為之耳。」莽大說,群臣盡服。及後漢兵劉伯升起,皆稱將軍,攻城略地,既殺甄阜,移書稱說。莽聞之憂懼。

漢兵乘勝遂圍宛城。初,世祖族兄聖公先在平林兵中。三月辛巳朔,平林、新巿、下江兵將王常、朱鮪等共立聖公為帝,改年為更始元年,拜置百官。莽聞之愈恐。欲外視自安,乃染其須髮,進所徵天下淑女杜陵史氏女為皇后,聘黃金三萬斤,車馬奴婢雜帛珍寶以巨萬計。莽親迎於前殿兩階間,成同牢之禮于上西堂。備和嬪、美御、和人三,位視公;嬪人九,視卿;美人二十七,視大夫;御人八十一,視元士:凡百二十人,皆佩印韍,執弓韣。封皇后父諶為和平侯,拜為寧始將軍,諶子二人皆侍中。是日,大風發屋折木。群臣上壽曰:「乃庚子雨水灑道,辛丑清靚無塵,其夕穀風迅疾,從東北來。辛丑,巽之宮日也。巽為風為順,后誼明,母道得,溫和慈惠之化也。《易》曰:『受茲介福,于其王母。』禮曰:『承天之慶,萬福無疆。』諸欲依廢漢火劉,皆沃灌雪除,殄滅無餘雜矣。百穀豐茂,庶草蕃殖,元元驩喜,兆民賴福,天下幸甚!」莽日與方士涿郡昭君等於後宮考驗方術,縱淫樂焉。大赦天下,然猶曰:「故漢氏舂陵侯群子劉伯升與其族人婚姻黨與,妄流言惑眾,悖畔天命,及手害更始將軍廉丹、前隊大夫甄阜、屬正梁丘賜,及北狄胡虜逆輿泊南僰虜若豆、孟遷,不用此書。有能捕得此人者,皆封為上公,食邑萬戶,賜寶貨五千萬。」

又詔:「太師王匡、國將哀章、司命孔仁、兗州牧壽良、卒正王閎、揚州牧李聖亟進所部州郡兵凡三十萬眾,迫措青、徐盜賊。納言將軍嚴尤、秩宗將軍陳茂、車騎將軍王巡、左隊大夫王吳亟進所部州郡兵凡十萬眾,迫措前隊醜虜。明告以生活丹青之信,復迷惑不解散,皆并力合擊,殄滅之矣!大司空隆新公,宗室戚屬,前以虎牙將軍東指則反虜破壞,西擊則逆賊靡碎,此乃新室威寶之臣也。如黠賊不解散,將遣大司空將百萬之師征伐劋絕之矣!」遣七公幹士隗囂等七十二人分下赦令曉諭云。囂等既出,因逃亡矣。

四月,世祖與王常等別攻潁川,下昆陽、郾、定陵。莽聞之愈恐,遣大司空王邑馳傳之雒陽,與司徒王尋發眾郡兵百萬,號曰「虎牙五威兵」,平定山東。得顓封爵,政決於邑,除用徵諸明兵法六十三家術者,各持圖書,受器械,備軍吏。傾府庫以遣邑,多齎珍寶猛獸,欲視饒富,用怖山東。邑至雒陽,州郡各選精兵,牧守自將,定會者四十二萬人,餘在道不絕,車甲士馬之盛,自古出師未嘗有也。

六月,邑與司徒尋發雒陽,欲至宛,道出潁川,過昆陽。昆陽時已降漢,漢兵守之。嚴尤、陳茂與二公會,二公縱兵圍昆陽。嚴尤曰:「稱尊號者在宛下,宜亟進。彼破,諸城自定矣。」邑曰:「百萬之師,所過當滅,今屠此城,喋血而進,前歌後舞,顧不快邪!」遂圍城數十重。城中請降,不許。嚴尤又曰:「『歸師勿遏,圍城為之闕』,可如兵法,使得逸出,以怖宛下。」邑又不聽。會世祖悉發郾、定陵兵數千人來救昆陽,尋、邑易之,自將萬餘人行陳,敕諸營皆按部毋得動,獨迎,與漢兵戰,不利。大軍不敢擅相救,漢兵乘勝殺尋。昆陽中兵出並戰,邑走,軍亂。天風蜚瓦,雨如注水,大眾崩壞號謼,虎豹股栗,士卒奔走,各還歸其郡。邑獨與所將長安勇敢數千人還雒陽。關中聞之震恐,盜賊並起。

又聞漢兵言,莽鴆殺孝平帝。莽乃會公卿以下於王路堂,開所為平帝請命金縢之策,泣以視群臣。命明學男張邯稱說其德及符命事,因曰:「易言:『伏戎于莽,升其高陵,三歲不興。』『莽』,皇帝之名。『升』謂劉伯升。『高陵』謂高陵侯子翟義也。言劉升、翟義為伏戎之兵於新皇帝世,猶殄滅不興也。」群臣皆稱萬歲。又令東方檻車傳送數人,言「劉伯升等皆行大戮」。臣知其詐也。

先是,衛將軍王涉素養道士西門君惠。君惠好天文讖記,為涉言:「星孛掃宮室,劉氏當復興,國師公姓名是也。」涉信其言,以語大司馬董忠,數俱至國師殿中廬道語星宿,國師不應。後涉特往,對歆涕泣言:「誠欲與公共安宗族,奈何不信涉也!」歆因為言天文人事,東方必成。涉曰:「新都哀侯小被病,功顯君素耆酒,疑帝本非我家子也。董公主中軍精兵,涉領宮衛,伊休侯主殿中,如同心合謀,共劫持帝,東降南陽天子,可以全宗族;不者,俱夷滅矣!」伊休侯者,歆長子也,為侍中五官中郎將,莽素愛之。歆怨莽殺其三子,又畏大禍至,遂與涉、忠謀,欲發。歆曰:「當待太白星出,乃可。」忠以司中大贅起武侯孫伋亦主兵,復與伋謀。伋歸家,顏色變,不能食。妻怪問之,語其狀。妻以告弟雲陽陳邯,邯欲告之。七月,伋與邯俱告,莽遣使者分召忠等。時忠方講兵都肄,護軍王咸謂忠謀久不發,恐漏泄,不如遂斬使者,勒兵入。忠不聽,遂與歆、涉會省戶下。莽令鹊惲責問,皆服。中黃門各拔刃將忠等送廬,忠拔劍欲自刎,侍中王望傳言大司馬反,黃門持劍共格殺之。省中相驚傳,勒兵至郎署,皆拔刃張弩。更始將軍史諶行諸署,告郎吏曰:「大司馬有狂病,發,已誅。」皆令弛兵。莽欲以厭凶,使虎賁以斬馬劍挫忠,盛以竹器,傳曰「反虜出」。下書赦大司馬官屬吏士為忠所詿誤,謀反未發覺者。收忠宗族,以醇醯毒藥、尺白刃叢僰并一坎而埋之。劉歆、王涉皆自殺。莽以二人骨肉舊臣,惡其內潰,故隱其誅。伊休侯疊又以素謹,歆訖不告,但免侍中中郎將,更為中散大夫。後日殿中鉤盾土山僊人掌旁有白頭公青衣,郎吏見者私謂之國師公。衍功侯喜素善卦,莽使筮之,曰:「憂兵火。」莽曰:「小兒安得此左道?是乃予之皇祖叔父子僑欲來迎我也。」

莽軍師外破,大臣內畔,左右亡所信,不能復遠念郡國,欲謼邑與計議。崔發曰:「邑素小心,今失大眾而徵,恐其執節引決,宜有以大慰其意。」於是莽遣發馳傳諭邑:「我年老毋適子,欲傳邑以天下。敕亡得謝,見勿復道。」邑到,以為大司馬。大長秋張邯為大司徒,崔發為大司空,司中壽容苗訢為國師,同說侯林為衛將軍。莽憂懣不能食,亶飲酒,啗鰒魚。讀軍書倦,因馮几寐,不復就枕矣。性好時日小數,及事迫急,亶為厭勝。遣使壞渭陵、延陵園門罘罳,曰:「毋使民復思也。」又以墨洿色其周垣。號將至曰「歲宿」,申水為「助將軍」,右庚「刻木校尉」,前丙「燿金都尉」,又曰:「執大斧,伐枯木;流大水,滅發火。」如此屬不可勝記。

秋,太白星流入太微,燭地如月光。

成紀隗崔兄弟共劫大尹李育,以兄子隗囂為大將軍,攻殺雍州牧陳慶、安定卒正王旬,并其眾,移書郡縣,數莽罪惡萬於桀紂。

是月,析人鄧曄、于匡起兵南鄉百餘人。時析宰將兵數千屯鄡亭,備武關。曄、匡謂宰曰:「劉帝已立,君何不知命也!」宰請降,盡得其眾。曄自稱輔漢左將軍,匡右將軍,拔析、丹水,攻武關,都尉朱萌降。進攻右隊大夫宋綱,殺之,西拔湖。莽愈憂,不知所出。崔發言:「周禮及春秋左氏,國有大災,則哭以厭之。故易稱『先號咷而後笑』。宜呼嗟告天以求救。」莽自知敗,乃率群臣至南郊,陳其符命本末,仰天曰:「皇天既命授臣莽,何不殄滅眾賊?即令臣莽非是,願下雷霆誅臣莽!」因搏心大哭,氣盡,伏而叩頭。又作告天策,自陳功勞,千餘言。諸生小民會旦夕哭,為設飧粥,甚悲哀及能誦策文者除以為郎,至五千餘人。鹊惲將領之。

莽拜將軍九人,皆以虎為號,號曰「九虎」,將北軍精兵數萬人東,內其妻子宮中以為質。時省中黃金萬斤者為一匱,尚有六十匱,黃門、鉤盾、臧府、中尚方處處各有數匱。長樂御府、中御府及都內、平準帑藏錢帛珠玉財物甚眾,莽愈愛之,賜九虎士人四千錢。眾重怨,無鬥意。九虎至華陰回谿,距隘,北從河南至山。于匡持數千弩,乘堆挑戰。鄧曄將二萬餘人從閿鄉南出棗街、作姑,破其一部,北出九虎後擊之。六虎敗走。史熊、王況詣闕歸死,莽使使責死者安在,皆自殺;其四虎亡。三虎郭欽、陳翬、成重收散卒,保京師倉。

鄧曄開武關迎漢,丞相司直李松將二千餘人至湖,與曄等共攻京師倉,未下。曄以弘農掾王憲為校尉,將數百人北度渭,入左馮翊界,降城略地。李松遣偏將軍韓臣等徑西至新豐,與莽波水將軍戰,波水走。韓臣等追奔,遂至長門宮。王憲北至頻陽,所過迎降。大姓櫟陽申碭、下邽王大皆率眾隨憲。屬縣斄嚴春、茂陵董喜、藍田王孟、槐里汝臣、盩厔王扶、陽陵嚴本、杜陵屠門少之屬,眾皆數千人,假號稱漢將。

時李松、鄧曄以為京師小小倉尚未可下,何況長安城,當須更始帝大兵到。即引軍至華陰,治攻具。而長安旁兵四會城下,聞天水隗氏兵方到,皆爭欲先入城,貪立大功鹵掠之利。

莽遣使者分赦城中諸獄囚徒,皆授兵,殺豨飲其血,與誓曰:「

有不為新室者,社鬼記之!」更始將軍史諶將度渭橋,皆散走。諶空還。眾兵發掘莽妻子父祖冢,燒其棺槨及九廟、明堂、辟雍,火照城中。或謂莽曰:「城門卒,東方人,不可信。」莽更發越騎士為衛,門置六百人,各一校尉。

十月戊申朔,兵從宣平城門入,民間所謂都門也。張邯行城門,逢兵見殺。王邑、王林、王巡、鹊惲等分將兵距擊北闕下。漢兵貪莽封力戰者七百餘人。會日暮,官府邸第盡奔亡。二日己酉,城中少年朱弟、張魚等恐見鹵掠,趨讙並和,燒作室門,斧敬法闥,謼曰:「反虜王莽,何不出降?」火及掖廷承明,黃皇室主所居也。莽避火宣室前殿,火輒隨之。宮人婦女謕謼曰:「當奈何!」時莽紺袀服,帶璽韍,持虞帝匕首。天文郎桉栻於前,日時加某,莽旋席隨斗柄而坐,曰:「天生德於予,漢兵其如予何!」莽時不食,少氣困矣。

三日庚戌,晨旦明,群臣扶掖莽,自前殿南下椒除,西出白虎門,和新公王揖奉車待門外。莽就車,之漸臺,欲阻池水,猶抱持符命、威斗,公卿大夫、侍中、黃門郎從官尚千餘人隨之。王邑晝夜戰,罷極,士死傷略盡,馳入宮,間關至漸臺,見其子侍中睦解衣冠欲逃,邑叱之令還,父子共守莽。軍人入殿中,謼曰:「反虜王莽安在?」有美人出房曰:「在漸臺。」眾兵追之,圍數百重。臺上亦弓弩與相射,稍稍落去。矢盡,無以復射,短兵接。王邑父平、鹊惲、王巡戰死,莽入室。下餔時,眾兵上臺,王揖、趙博、苗訢、唐尊、王盛、中常侍王參等皆死臺上。商人杜吳殺莽,取其綬。校尉東海公賓就,故大行治禮,見吳問綬主所在。曰:「室中西北陬間。」就識,斬莽首。軍人分裂莽身,支節肌骨臠分,爭相殺者數十人。公賓就持莽首詣王憲。憲自稱漢大將軍,城中兵數十萬皆屬焉,舍東宮,妻莽後宮,乘其車服。

六日癸丑,李松、鄧曄入長安,將軍趙萌、申屠建亦至,以王憲得璽綬不輒上,多挾宮女,建天子鼓旗,收斬之。傳莽首詣更始,縣宛市,百姓共提擊之,或切食其舌。

莽揚州牧李聖、司命孔仁兵敗山東,聖格死,仁將其眾降,已而歎曰:「吾聞食人食者死其事。」拔劍自刺死。及曹部監杜普、陳定大尹沈意、九江連率賈萌皆守郡不降,為漢兵所誅。賞都大尹王欽及郭欽守京師倉,聞莽死,乃降,更始義之,皆封為侯。太師王匡、國將哀章降雒陽,傳詣宛,斬之。嚴尤、陳茂敗昆陽下,走至沛郡譙,自稱漢將,召會吏民。尤為稱說王莽篡位天時所亡聖漢復興狀,茂伏而涕泣。聞故漢鍾武侯劉聖聚眾汝南稱尊號,尤、茂降之。以尤為大司馬,茂為丞相。十餘日敗,尤、茂并死。郡縣皆舉城降,天下悉歸漢。

初,申屠建嘗事崔發為詩,建至,發降之。後復稱說,建令丞相劉賜斬發以徇。史諶、王延、王林、王吳、趙閎亦降,復見殺。初,諸假號兵人人望封侯。申屠建既斬王憲,又揚言三輔黠共殺其主。吏民惶恐,屬縣屯聚,建等不能下,馳白更始。

二年二月,更始到長安,下詔大赦,非王莽子,他皆除其罪,故王氏宗族得全。三輔悉平,更始都長安,居長樂宮。府藏完具,獨未央宮燒攻莽三日,死則案堵復故。更始至,歲餘政教不行。明年夏,赤眉樊崇等眾數十萬人入關,立劉盆子,稱尊號,攻更始,更始降之。赤眉遂燒長安宮室市里,害更始。民飢餓相食,死者數十萬,長安為虛,城中無人行。宗廟園陵皆發掘,唯霸陵、杜陵完。六月,世祖即位,然後宗廟社稷復立,天下艾安。

贊曰:王莽始起外戚,折節力行,以要名譽,宗族稱孝,師友歸仁。及其居位輔政,成、哀之際,勤勞國家,直道而行,動見稱述。豈所謂「在家必聞,在國必聞」,「色取仁而行違」者邪?莽既不仁而有佞邪之材,又乘四父歷世之權,遭漢中微,國統三絕,而太后壽考為之宗主,故得肆其姦慝,以成篡盜之禍。推是言之,亦天時,非人力之致矣。及其竊位南面,處非所據,顛覆之勢險於桀紂,而莽晏然自以黃、虞復出也。乃始恣睢,奮其威詐,滔天虐民,窮凶惡極,毒流諸夏,亂延蠻貉,猶未足逞其欲焉。是以四海之內,囂然喪其樂生之心,中外憤怨,遠近俱發,城池不守,支體分裂,遂令天下城邑為虛,丘壟發掘,害遍生民,辜及朽骨,自書傳所載亂臣賊子無道之人,考其禍敗,未有如莽之甚者也。昔秦燔詩書以立私議,莽誦六藝以文姦言,同歸誅塗,俱用滅亡,皆炕龍絕氣,非命之運,紫色麴聲,餘分閏位,聖王之驅除云爾!


\end{pinyinscope}