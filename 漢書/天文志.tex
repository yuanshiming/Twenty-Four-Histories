\article{天文志}

\begin{pinyinscope}
凡天文在圖籍昭昭可知者,經星常宿中外官凡百一十八名,積數七百八十三星,皆有州國官宮物類之象。其伏見蚤晚,邪正存亡,虛實闊骥,及五星所行,合散犯守,陵歷鬥食,彗孛飛流,日月薄食,暈適背穴,抱珥鸶蜺,迅雷風祅,怪雲變氣:此皆陰陽之精,其本在地,而上發于天者也。政失於此,則變見於彼,猶景之象形,鄉之應聲。是以明君睹之而寤,飭身正事,思其咎謝,則禍除而福至,自然之符也。

中宮天極星,其一明者,泰一之常居也,旁三星三公,或曰子屬。後句四星,末大星正妃,餘三星後官之屬也。環之匡衛十二星,藩臣。皆曰紫宮。

前列直斗口三星,隨北耑銳,若見若不見,曰陰德,或曰天一。紫宮左三星曰天槍,右四星曰天棓。後十七星絕漢抵營室,曰閣道。

北斗七星,所謂「旋、璣、玉衡以齊七政」。杓攜龍角,衡殷南斗,魁枕參首。用昏建者杓;杓,自華以西南。夜半建者衡;衡,殷中州河、濟之間。平旦建者魁;魁,海岱以東北也。斗為帝車,運于中央,臨制四海。分陰陽,建四時,均五行,移節度,定緒紀,皆繫於斗。

斗魁戴筐六星,曰文昌宮:一曰上將,二曰次將,三曰貴相,四曰司命,五曰司祿,六曰司災。在魁中,貴人之牢。魁下六星兩兩而比者,曰三能。三能色齊,君臣和;不齊,為乖戾。柄輔星,明近,輔臣親彊;斥小,疏弱。

杓端有兩星:一內為矛,招搖;一外為盾,天蜂。有句圜十五星,屬杓,曰賤人之牢。牢中星實則囚多,虛則開出。

天一、槍、棓、矛、盾動搖,角大,兵起。

東宮蒼龍,房、心。心為明堂,大星天王,前後星子屬。不欲直;直,王失計。房為天府,曰天駟。其陰,右驂。旁有兩星曰衿。衿北一星曰鳜。東北曲十二星曰旗。旗中四星曰天市。天市中星眾者實,其中虛則耗。房南眾星曰騎官。

左角,理;右角,將。大角者,天王帝坐廷。其兩旁各有三星,鼎足句之,曰攝提。攝提者,直斗杓所指,以建時節,故曰「

攝提格」。亢為宗廟,主疾。其南北兩大星,曰南門。氐為天根,主疫。尾為九子,曰君臣;斥絕,不和。箕為敖客,后妃之府,曰口舌。火犯守角,則有戟。房、心,王者惡之。

南宮朱鳥,權、衡。衡、太微,三光之廷。筐衛十二星,藩臣:西,將;東,相;南四星,執法;中,端門;左右,掖門。掖門內六星,諸侯。其內五星,五帝坐。後聚十五星,曰哀烏郎位;旁一大星,將位也。月、五星順入,軌道,司其出,所守,天子所誅也。其逆入,若不軌道,以所犯名之;中坐,成形,皆群下不從謀也。金、火尤甚。廷藩西有隨星四,名曰少微,士大夫。權,軒轅,黃龍體。前大星,女主象;旁小星,御者後宮屬。月、五星守犯者,如衡占。

東井為水事。火入之,一星居其左右,天子且以火為敗。東井西曲星曰戉;北,北河;南,南河;兩河、天闕間為關梁。輿鬼,鬼祠事;中白者為質。火守南北河,兵起,穀不登。故德成衡,觀成潢,傷成戉,禍成井,誅成質。

柳為鳥喙,主木草。七星,頸,為員宮,主急事。張,嗉,為廚,主觴客。翼為羽翮,主遠客。

軫為車,主風。其旁有一小星,曰長沙,星星不欲明;明與四星等,若五星入軫中,兵大起。軫南眾星曰天庫,庫有五車。車星角,若益眾,及不具,亡處車馬。

西宮咸池,曰天五潢。五潢,五帝車舍。火入,旱;金,兵;水,水。中有三柱;柱不具,兵起。

奎曰封豨,為溝瀆。婁為聚眾。胃為天倉。其南眾星曰廥積。

昴曰旄頭,胡星也,為白衣會。畢曰罕車,為邊兵,主弋獵。其大星旁小星為附耳。附耳搖動,有讒亂臣在側。昴、畢間為天街。其陰,陰國;陽,陽國。

參為白虎。三星直者,是為衡石。下有三星,銳,曰罰,為斬艾事。其外四星,左右肩股也。小三星隅置,曰觜觿,為虎首,主葆旅事。其南有四星,曰天廁。天廁下一星,曰天矢。矢黃則吉;青、白、黑,凶。其西有句曲九星,三處羅列:一曰天旗,二曰天苑,三曰九斿。其東有大星曰狼,狼角變色,多盜賊。下有四星曰弧,直狼。比地有大星,曰南極老人。老人見,治安;不見,兵起。常以秋分時候之南郊。

北宮玄武,虛、危。危為蓋屋;虛為哭泣之事。其南有眾星,曰羽林天軍。軍西為壘,或曰戉。旁一大星,北落。北落若微亡,軍星動角益稀,及五星犯北落,入軍,軍起。火、金、水尤甚。火入,軍憂;水,水患;木、土,軍吉。危東六星,兩兩而比,曰司寇。

營室為清廟,曰離宮、閣道。漢中四星,曰天駟。旁一星,曰王梁。王梁策馬,車騎滿野。旁有八星,絕漢,曰天橫。天橫旁,江星。江星動,以人涉水。

杵、臼四星,在危南。匏瓜,有青黑星守之,魚鹽貴。

南斗為廟,其北建星。建星者,旗也。牽牛為犧牲,其北河鼓。河鼓大星,上將;左,左將;右,右將。婺女,其北織女。織女,天女孫也。

歲星曰東方春木,於人五常仁也,五事貌也。仁虧貌失,逆春令,傷木氣,罰見歲星。歲星所在,國不可伐,可以伐人。超舍而前為贏,退舍為縮。贏,其國有兵不復;縮,其國有憂,其將死,國傾敗。所去,失地;所之,得地。一曰,當居不居,國亡;所之,國昌;已居之,又東西去之,國凶,不可舉事用兵。安靜中度,吉。出入不當其次,必有天祅見其舍也。

歲星贏而東南,石氏「見彗星」,甘氏「不出三月乃生彗,本類星,末類彗,長二丈」。贏東北,石氏「見覺星」,甘氏「不出三月乃生天棓,本類星,末銳,長四尺」。縮西南,石氏「

見欃雲,如牛」,甘氏「不出三月乃生天槍,左右銳,長數丈」。縮西北,石氏「見槍雲,如馬」,甘氏「不出三月乃生天欃,本類星,末銳,長數丈」。石氏「槍、欃、棓、彗異狀,其殃一也,必有破國亂君,伏死其辜,餘殃不盡,為旱、凶、飢、暴疾」。至日行一尺,出二十餘日乃入,甘氏「其國凶,不可舉事用兵」。出而易,「所當之國,是受其殃」。又曰「祅星,不出三年,其下有軍,及失地,若國君喪」。

熒惑曰南方夏火,禮也,視也。禮虧視失,逆夏令,傷火氣,罰見熒惑。逆行一舍二舍為不祥,居之三月國有殃,五月受兵,七月國半亡地,九月地太半亡。因與俱出入,國絕祀。熒惑為亂為

成,為疾為喪,為飢為兵,所居之宿國受殃。殃還至者,雖大當小;居之久殃乃至者,當小反大。已去復還居之,若居之而角者,若動者,繞環之,及乍前乍後,乍左乍右,殃愈甚。一曰,熒惑出則有大兵,入則兵散。周還止息,乃為其死喪。寇亂在其野者亡地,以戰不勝。東行疾則兵聚于東方,西行疾則兵聚于西方;其南為丈夫喪,北為女子喪。熒惑,天子理也,故曰雖有明天子,必視熒惑所在。

太白曰西方秋金,義也,言也。義虧言失,逆秋令,傷金氣,罰見太白。日方南太白居其南,日方北太白居其北,為贏,侯王不寧,用兵進吉退凶。日方南太白居其北,日方北太白居其南,為縮,侯王有憂,用兵退吉進凶。當出不出,當入不入,為失舍,不有破軍,必有死王之墓,有亡國。一曰,天下匽兵,野有兵者,所當之國大凶。當出不出,未當入而入,天下匽兵,兵在外,入。未當出而出,當入而不入,天下起兵,有至破國。未當出而出,未當入而入,天下舉兵,所當之國亡。當期而出,其國昌。出東為東方,入為北方;出西為西方,入為南方。所居久,其國利;易,其鄉凶。入七日復出,將軍戰死。入十日復出,相死之。入又復出,人君惡之。已出三日而復微入,三日乃復盛出,是為耎而伏,其下國有軍,其眾敗將北。已入三日,又復微出,三日乃復盛入,其下國有憂,帥師雖眾,敵食其糧,用其兵,虜其帥。出西方,失其行,夷狄敗;出東方,失其行,中國敗。一曰,出蚤為月食,晚為天祅及彗星,將發於亡道之國。

太白出而留桑榆間,病其下國。上而疾,未盡期日過參天,病其對國。太白經天,天下革,民更王,是為亂紀,人民流亡。晝見與日爭明,彊國弱,小國彊,女主昌。

太白,兵象也。出而高,用兵深吉淺凶;埤,淺吉深凶。行疾,用兵疾吉遲凶;行遲,用兵遲吉疾凶。角,敢戰吉,不敢戰凶;擊角所指吉,逆之凶。進退左右,用兵進退左右吉,靜凶。圜以靜,用兵靜吉趮凶。出則兵出,入則兵入。象太白吉,反之凶。赤角,戰。

太白者,猶軍也,而熒惑,憂也。故熒惑從太白,軍憂;離之,軍舒。出太白之陰,有分軍;出其陽,有偏將之戰。當其行,太白還之,破軍殺將。

辰星,殺伐之氣,戰鬥之象也。與太白俱出東方,皆赤而角,夷狄敗,中國勝;與太白俱出西方,皆赤而角,中國敗,夷狄勝。

五星分天之中,積于東方,中國大利;積于西方,夷狄用兵者利。

辰星不出,太白為客;辰星出,太白為主人。辰星與太白不相從,雖有軍不戰。辰星出東方,太白出西方。若辰星出西方,太白出東方,為格,野雖有兵,不戰。辰星入太白中,五日乃出,及入而上出,破軍殺將,客勝;下出,客亡地。辰星來抵,太白不去,將死。正其上出,破軍殺將,客勝;下出,客亡地。視其所指,以名破軍。辰星繞環太白,若鬥,大戰,客勝,主人吏死。辰星過太白,間可椷劍,小戰,客勝;居太白前旬三日,軍罷;出太白左,小戰;歷太白右,數萬人戰,主人吏死;出太白右,去三尺,軍急約戰。

凡太白所出所直之辰,其國為得位,得位者戰勝。所直之辰順其色而角者勝,其色害者敗。太白白比狼,赤比心,黃比參右肩,青比參左肩,黑比奎大星。色勝位,行勝色,行得盡勝之。

辰星曰北方冬水,知也,聽也。知虧聽失,逆冬令,傷水氣,罰見辰星。出蚤為月食,晚為彗星及天祅。一時不出,其時不和;四時不出,天下大饑。失其時而出,為當寒反溫,當溫反寒。當出不出,是謂擊卒,兵大起。與它星遇而鬥,天下大亂。出於房、心間,地動。

填星曰中央季夏土,信也,思心也。仁義禮智以信為主,貌言視聽以心為正,故四星皆失,填星乃為之動。填星所居,國吉。未當居而居之,若已去而復還居之,國得土,不乃得女子。當居不居,既已居之,又東西去之,國失土,不乃失女,不,有土事若女之憂。居宿久,國福厚;易,福薄。當居不居,為失填,其下國可伐;得者,不可伐。其贏,為王不寧;縮,有軍不復。一曰,既已居之又東西去之,其國凶,不可舉事用兵。失次而上一舍三舍,有王命不成,不乃大水;失次而下二舍,有后慼,其歲不復,不乃天裂若地動。

凡五星,歲與填合則為內亂,與辰合則為變謀而更事,與熒惑合則為飢,為旱,與太白合則為白衣之會,為水。太白在南,歲在北,名曰牡牡,年穀大孰。太白在北,歲在南,年或有或亡。熒惑與太白合則為喪,不可舉事用兵;與填合則為憂,主孽卿;與辰合則為北軍,用兵舉事大敗。填與辰合則將有覆軍下師;與太白合則為疾,為內兵。辰與太白合則為變謀,為兵憂。凡歲、熒惑、填、太白四星與辰鬥,皆為戰,兵不在外,皆為內亂。一曰,火與水合為淬,與金合為鑠,不可舉事用兵。土與金合國亡地,與木合則國饑,與水合為雍沮,不可舉事用兵。木與金合鬥,國有內亂。同舍為合,相陵為鬥。二星相近者其殃大,二星相遠者殃無傷也,從七寸以內必之。

凡月食五星,其國必亡:歲以飢,熒惑以亂,填以殺,太白彊國以戰,辰以女亂。月食大角,王者惡之。

凡五星所聚宿,其國王天下:從歲以義,從熒惑以禮,從填以重,從太白以兵,從辰以法。以法者,以法致天下也。三星若合,是謂驚立絕行,其國外內有兵與喪,民人乏飢,改立王公。四星若合,是謂大湯,其國兵喪並起,君子憂,小人流。五星若合,是謂易行:有德受慶,改立王者,掩有四方,子孫蕃昌;亡德受罰,離其國家,滅其宗廟,百姓離去,被滿四方。五星皆大,其事亦大;皆小,其事亦小也。

凡五星色:皆圜,白為喪為旱,赤中不平為兵,青為憂為水,黑為疾為多死,黃吉;皆角,赤犯我城,黃地之爭,白哭泣之聲,青有兵憂,黑水。五星同色,天下匽兵,百姓安寧,歌舞以行,不見災疾,五穀蕃昌。

凡五星,歲,緩則不行,急則過分,逆則占。熒惑,緩則不出,急則不入,違道則占。填,緩則不建,急則過舍,逆則占。太白,緩則不出,急則不入,逆則占。辰,緩則不出,急則不入,非時則占。五星不失行,則年穀豐昌。

凡以宿星通下之變者,維星散,句星信,則地動。有星守三淵,天下大水,地動,海魚出。紀星散者山崩,不即有喪。龜、鱉星不居漢中,川有易者。辰星入五車,大水。熒惑入積水,水,兵起;入積薪,旱,兵起;守之,亦然。極後有四星,名曰句星。斗杓後有三星,名曰維星。散者,不相從也。三淵,蓋五車之三柱也。天紀屬貫索。積薪在北戍西北。積水在北戍東北。

角、亢、氐,沇州。房、心,豫州。尾、箕,幽州。斗,江、湖。牽牛、婺女,揚州。虛、危,青州。營室、東壁,并州。奎、婁、胃,徐州。昂、畢,冀州。觜觿、參,益州。東井、輿鬼,雍州。柳、七星、張,三河。翼、軫,荊州。

甲乙,海外,日月不占。丙丁,江、淮、海、岱。戊己,中州河、濟。庚辛,華山以西。壬癸,常山以北。一曰,甲齊,乙東夷,丙楚,丁南夷,戊魏,己韓,庚秦,辛西夷,壬燕、趙,癸北夷。子周,丑翟,寅趙,卯鄭,辰邯鄲,巳衛,午秦,未中山,申齊,酉魯,戌吳、越,亥燕、代。

秦之疆,候太白,占狼、弧。吳、楚之疆,候熒惑,占鳥衡。燕、齊之疆,候辰星,占虛、危。宋、鄭之疆,候歲星,占房、心。晉之疆,亦候辰星,占參、罰。及秦并吞三晉、燕、代,自河、山以南者中國。中國於四海內則在東南,為陽,陽則日、歲星、熒惑、填星,占於街南,畢主之。其西北則胡、貉、月氏旃裘引弓之民,為陰,陰則月、太白、辰星,占於街北,昴主之。故中國山川東北流,其維,首在隴、蜀,尾沒於勃海碣石。是以秦、晉好用兵,復占太白。太白主中國,而胡、貉數侵掠,獨占辰星。辰星出入趮疾,常主夷狄,其大經也。

凡五星,早出為贏,贏為客;晚出為縮,縮為主人。五星贏縮,必有天應見杓。

太歲在寅曰攝提格。歲星正月晨出東方,石氏曰名監德,在斗、牽牛。失次,杓,早水,晚旱。甘氏在建星、婺女。太初曆在營室、東壁。

在卯曰單閼。二月出,石氏曰名降入,在婺女、虛、危。甘氏在虛、危。失次,杓,有水災。太初在奎、婁。

在辰曰執徐。三月出,石氏曰名青章,在營室、東壁。失次,杓,早旱,晚水。甘氏同。太初在胃、昴。

在巳曰大荒落。四月出,石氏曰名路踵,在奎、婁。甘氏同。太初在參、罰。

在午曰敦牂。五月出,石氏曰名啟明,在胃、昴、畢。失次,杓,早旱,晚水。甘氏同。太初在東井、輿鬼。

在未曰協洽。六月出,石氏曰名長烈,在觜觿、參。甘氏在參、罰。太初在注、張、七星。

在申曰涒灘。七月出。石氏曰名天晉,在東井、輿鬼。甘氏在弧。太初在翼、軫。

在酉曰作詻。爾雅作作噩。八月出,石氏曰名長壬,在柳、七星、張。失次,杓,有女喪、民疾。甘氏在注、張。失次,杓,有火。太初在角、亢。

在戌曰掩茂。九月出,石氏曰名天睢,在翼、軫。失次,杓,水。甘氏在七星、翼。太初在氐、房、心。

在亥曰大淵獻。十月出,石氏曰名天皇,在角、亢始。甘氏在軫、角、亢。太初在尾、箕。

在子曰困敦。十一月出,石氏曰名天宗,在氐、房始。甘氏同。太初在建星、牽牛。

在丑曰赤奮若。十二月出,石氏曰名天昊,在尾、箕。甘氏在心、尾。太初在婺女、虛、危。

甘氏、太初曆所以不同者,以星贏縮在前,各錄後所見也。其四星亦略如此。

古曆五星之推,亡逆行者,至甘氏、石氏經,以熒惑、太白為有逆行。夫曆者,正行也。古人有言曰:「天下太平,五星循度,亡有逆行。日不食朔,月不食望。」夏氏日月傳曰:「日月食盡,主位也;不盡,臣位也。」星傳曰:「日者德也,月者刑也,故曰日食修德,月食修刑。」然而曆紀推月食,與二星之逆亡異。熒惑主內亂,太白主兵,月主刑。自周室衰,亂臣賊子師旅數起,刑罰失中,雖其亡亂臣賊子師旅之變,內臣猶不治,四夷猶不服,兵革猶不寢,刑罰猶不錯,故二星與月為之失度,三變常見;及有亂臣賊子伏尸流血之兵,大變乃出。甘、石氏見其常然,因以為紀,皆非正行也。《詩》云:「彼月而食,則惟其常;此日而食,于何不臧?」詩傳曰:「月食非常也,比之日食猶常也,日食則不臧矣。」謂之小變,可也;謂之正行,非也。故熒惑必行十六舍,去日遠而顓恣。太白出西方,進在日前,氣盛乃逆行。及月必食於望,亦誅盛也。

國皇星,大而赤,狀類南極。所出,其下起兵。兵彊,其衝不利。

昭明星,大而白,無角,乍上乍下。所出國,起兵多變。

五殘星,出正東,東方之星。其狀類辰,去地可六丈,大而黃。

六賊星,出正南,南方之星。去地可六丈,大而赤,數動,有光。

司詭星,出正西,西方之星。去地可六丈,大而白,類太白。

咸漢星,出正北,北方之星。去地可六丈,大而赤,數動,察之中青。

此四星所出非其方,其下有兵,衝不利。

四填星,出四隅,去地可四丈。地維臧光,亦出四隅,去地可二丈,若月始出。所見下,有亂者亡,有德者昌。

燭星,狀如太白,其出也不行,見則滅。所燭,城邑亂。

如星非星,如雲非雲,名曰歸邪。歸邪出,必有歸國者。

星者,金之散氣,其本曰人。星眾,國吉,少則凶。漢者,亦金散氣,其本曰水。星多,多水,少則旱,其大經也。

天鼓,有音如雷非雷,音在地而下及地。其所住者,兵發其下。

天狗,狀如大流星,有聲,共下止地,類狗。所墜及,望之如火光炎炎中天。其下圜如數頃田處,上銳見則有黃色,千里破軍殺將。

格澤者,如炎火之狀,黃白,起地而上,下大上銳。其見也,不種而穫。不有土功,必有大客。

蚩尤之旗,類彗而後曲,象旗。見則王者征伐四方。

旬始,出於北斗旁,狀如雄雞。其怒,青黑色,象伏鱉。

枉矢,狀類大流星,蛇行而倉黑,望如有毛目然。

長庚,廣如一匹布著天。此星見,起兵。

星蕴至地,則石也。

天蛲而見景星。景星者,德星也,其狀無常,常出於有道之國。

日有中道,月有九行。

中道者,黃道,一曰光道。光道北至東井,去北極近;南至牽牛,去北極遠;東至角,西至婁,去極中。夏至至於東井,北近極,故晷短;立八尺之表,而晷景長尺五寸八分。冬至至於牽牛,遠極,故晷長;立八尺之表,而晷景長丈三尺一寸四分。春秋分日至婁、角,去極中,而晷中;立八尺之表,而晷景長七尺三寸六分。此日去極遠近之差,晷景長短之制也。去極遠近難知,要以晷景。晷景者,所以知日之南北也。日,陽也。陽用事則日進而北,晝進而長,陽勝,故為溫暑;陰用事則日退而南,晝退而短,陰勝,故為涼寒也。故日進為暑,退為寒。若日之南北失節,晷過而長為常寒,退而短為常燠。此寒燠之表也。故曰為寒暑。一曰,晷長為潦,短為旱,奢為扶。扶者,邪臣進而正臣疏,君子不足,姦人有餘。

月有九行者:黑道二,出黃道北;赤道二,出黃道南;白道二,出黃道西;青道二,出黃道東。立春、春分,月東從青道;立秋,秋分,西從白道;立冬、冬至,北從黑道;立夏、夏至,南從赤道。然用之,一決房中道。青赤出陽道,白黑出陰道。若月失節度而妄行,出陽道則旱風,出陰道則陰雨。

凡君行急則日行疾,君行緩則日行遲。日行不可指而知也,故以二至二分之星為候。日東行,星西轉。冬至昏,奎八度中;夏至,氐十三度中;春分,柳一度中;秋分,牽牛三度七分中:此其正行也。日行疾,則星西轉疾,事勢然也。故過中則疾,君行急之感也;不及中則遲,君行緩之象也。

至月行,則以晦朔決之。日冬則南,夏則北;冬至於牽牛,夏至於東井。日之所行為中道,月、五星皆隨之也。

箕星為風,東北之星也。東北地事,天位也,故易曰「東北喪朋」。及巽在東南,為風;風,陽中之陰,大臣之象也,其星,軫也。月去中道,移而東北入箕,若東南入軫,則多風。西方為雨;雨,少陰之位也。月去中道,移而西入畢,則多雨。故詩云「月離于畢,俾滂沱矣」,言多雨也。星傳曰「月入畢則將相有以家犯罪者」,言陰盛也。書曰「星有好風,星有好雨,月之從星,則以風雨」,言失中道而東西也。故星傳曰「月南入牽牛南戒,民間疾疫;月北入太微,出坐北,若犯坐,則下人謀上。」

一曰月為風雨,日為寒溫。冬至日南極,晷長,南不極則溫為害;夏至日北極,晷短,北不極則寒為害。故書曰「日月之行,則有冬有夏」也。政治變於下,日月運於上矣。日出房北,為雨為陰,為亂為兵;出房南,為旱為夭喪。水旱至衝而應,及五星之變,必然之效也。

兩軍相當,日暈等,力均;厚長大,有勝;薄短小,亡勝。重抱大破亡。抱為和,背為不和,為分離相去。直為自立,立兵破軍,若曰殺將。抱且戴,有喜。圍在中,中勝;在外,外勝。青外赤中,以和相去;赤外青中,以惡相去。氣暈先至而後去,居軍勝。先至先去,前有利,後有病;後至後去,前病後利;後至先去,前後皆病,居軍不勝。見而去,其後發疾,雖勝亡功。見半日以上,功太。白鸶屈短,上下銳,有者下大流血。日暈制勝,近期三十日,遠期六十日。

其食,食所不利;復生,生所利;不然,食盡為主位。以其直及日所躔加日時,用名其國。

凡望雲氣,仰而望之,三四百里;平望,在桑榆上,千餘里,二千里;登高而望之,下屬地者居三千里。雲氣有戰居上者,勝。

自華以南,氣下黑上赤。嵩高、三河之郊,氣正赤。常山以北,氣下黑上青。勃、碣、海、岱之間,氣皆黑。江、淮之間,氣皆白。

徒氣白。土功氣黃。車氣乍高乍下,往往而聚。騎氣卑而布。卒氣摶。前卑而後高者,疾;前方而後高者,銳;後銳而卑者,卻。其氣平者其行徐。前高後卑者,不止而反。氣相遇者,卑勝高,銳勝方。氣來卑而循車道者,不過三四日,去之五六里見。氣來高七八尺者,不過五六日,去之十餘二十里見。氣來高丈餘二丈者,不過三四十日,去之五六十里見。

捎雲精白者,其將悍,其士怯。其大根而前絕遠者,戰。精白,其芒低者,戰勝;其前赤而卬者,戰不勝。陳雲如立垣。杼雲類杼。柚雲摶而耑銳。杓雲如繩者,居前竟天,其半半天。蜺雲者,類鬥旗故。銳鉤雲句曲。諸此雲見,以五色占。而澤摶密,其見動人,乃有占;兵必起。占鬥其直。

王朔所候,決於日旁。日旁雲氣,人主象。皆如其形以占。

故北夷之氣如群畜穹閭,南夷之氣類舟船幡旗。大水處,敗軍場,破國之虛,下有積泉,金寶上,皆有氣,不可不察。海旁蜃氣象樓臺,廣野氣成宮闕然。雲氣各象其山川人民所聚積。故侯息秏者,入國邑,視封疆田疇之整治,城郭室屋門戶之潤澤,次至車服畜產精華。實息者吉,虛秏者凶。

若煙非煙,若雲非雲,郁郁紛紛,蕭索輪囷,是謂慶雲。慶雲見,喜氣也。若霧非霧,衣冠不濡,見則其城被甲而趨。

夫雷電、赮鸶、辟歷、夜明者,陽氣之動者也,春夏則發,秋冬則藏,故候書者亡不司。

天開縣物,地動坼絕。山崩及鲶,川塞谿垘;水澹地長,澤竭見象。城郭門閭,潤息槁枯;宮廟廊第,人民所次。謠俗車服,觀民飲食。五穀草木,觀其所屬。倉府廄庫,四通之路。六畜禽獸,所產去就;魚鱉鳥鼠,觀其所處。鬼哭若謼,與人逢槞。訛言,誠然。

凡候歲美惡,謹候歲始。歲始或冬至日,產氣始萌。臘明日,人眾卒歲,壹會飲食,發陽氣,故曰初歲。正月旦,王者歲首;立春,四時之始也。四始者,候之日。

而漢魏鮮集臘明正月旦決八風。風從南,大旱;西南,小旱;西方,有兵;西北,戎叔為,小雨,趣兵;北方,為中歲;東北,為上歲;東方,大水;東南,民有疾疫,歲惡。故八風各與其衝對,課多者為勝。多勝少,久勝亟,疾勝徐。旦至食,為麥;食至日跌,為疾;跌至晡,為黍;晡至下晡,為叔;下晡至日入,為麻。欲終日有雲,有風,有日,當其時,深而多實;亡雲,有風日,當其時,淺而少實;有雲風,亡日,當其時,深而少實;有日,亡雲,不風,當其時者稼有敗。如食頃,小敗;孰五斗米頃,大敗。風復起,有雲,其稼復起。各以其時用雲色占種所宜。雨雪,寒,歲惡。

是日光明,聽都邑人民之聲。聲宮,則歲美,吉;商,有兵;徵,旱;羽,水;角,歲惡。

或從正月旦比數雨。率日食一升,至七升而極;過之,不占。數至十二日,直其月,占水旱。為其環域千里內占,即為天下候,竟正月。月所離列宿,日、風、雲,占其國。然必察太歲所在。金,穰;水,毀;木,飢;火,旱。此其大經也。

正月上甲,風從東方來,宜蠶;從西方來,若旦有黃雲,惡。

冬至短極,縣土炭,炭動,麋鹿解角,蘭根出,泉水踊,略以知日至,要決晷景。

夫天運三十歲一小變,百年中變,五百年大變,三大變一紀,三紀而大備,此其大數也。

春秋二百四十二年間,日食三十六,彗星三見,夜常星不見,夜中星隕如雨者各一。當是時,禍亂輒應,周室微弱,上下交怨,殺君三十六,亡國五十二,諸侯奔走不得保其社稷者不可勝數。自是之後,眾暴寡,大并小。秦、楚、吳、粵,夷狄也,為彊伯。田氏篡齊,三家分晉,並為戰國,爭於攻取,兵革遞起,城邑數屠,因以飢饉疾疫愁苦,臣主共憂患,其察禨祥候星氣尤急。近世十二諸侯七國相王,言從橫者繼踵,而占天文者因時務論書傳,故其占驗鱗雜米鹽,亡可錄者。

周卒為秦所滅。始皇之時,十五年間彗星四見,久者八十日,長或竟天。後秦遂以兵內兼六國,外攘四夷,死人如亂麻。又熒惑守心,及天市芒角,色赤如雞血。始皇既死,適庶相殺,二世即位,殘骨肉,戮將相,太白再經天。因以張楚並興,兵相跆籍,秦遂以亡。

項羽救鉅鹿,枉矢西流,枉矢所觸,天下之所伐射,滅亡象也。物莫直於矢,今蛇行不能直而枉者,執矢者亦不正,以象項羽執政亂也。羽遂合從,阬秦人,屠咸陽。凡枉矢之流,以亂伐亂也。

漢元年十月,五星聚於東井,以曆推之,從歲星也。此高皇帝受命之符也。故客謂張耳曰:「東并秦地,漢王入秦,五星從歲星聚,當以義取天下。」秦王子嬰降於枳道,漢王以屬吏,寶器婦女亡所取,閉宮封門,還軍次于霸上,以候諸侯。與秦民約法三章,民亡不歸心者,可謂能行義矣,天之所予也。五年遂定天下,即帝位。此明歲星之崇義,東井為秦之地明效也。

三年秋,太白出西方,有光幾中,乍北乍南,過期乃入。辰星出四孟。是時,項羽為楚王,而漢已定三秦,與相距滎陽。太白出西方,有光幾中,是秦地戰將勝,而漢國將興也。辰星出四孟,易主之表也。後二年,漢滅楚。

七年,月暈,圍參、畢七重。占曰:「畢、昴間,天街也;街北,胡也;街南,中國也。昴為匈奴,參為趙,畢為邊兵。」是歲高皇帝自將兵擊匈奴,至平城,為冒頓單于所圍,七日乃解。

十二年春,熒惑守心。四月,宮車晏駕。

孝惠二年,天開東北,廣十餘丈,長二十餘丈。地動,陰有餘;天裂,陽不足:皆下盛彊將害上之變也。其後有呂氏之亂。

孝文後二年正月壬寅,天欃夕出西南。占曰:「為兵喪亂。」其六年十一月,匈奴入上郡、雲中,漢起三軍以衛京師。其四月乙巳,水、木、火三合於東井。占曰:「外內有兵與喪,改立王公。東井,秦也。」八月,天狗下梁野,是歲誅反者周殷長安市。其七年六月,文帝崩。其十一月戊戌,土、水合於危。占曰:「為雍沮,所當之國不可舉事用兵,必受其殃。一曰將覆軍。危,齊也。」其七月,火東行,行畢陽,環畢東北,出而西,逆行至昴,即南乃東行。占曰:「為喪死寇亂。畢、昴,趙也。」

孝景元年正月癸酉,金、水合於婺女。占曰:「為變謀,為兵憂。婺女,粵也,又為齊。」其七月乙丑,金、木、水三合於張。占曰:「外內有兵與喪,改立王公。張,周地,今之河南也,又為楚。」其二年七月丙子,火與水晨出東方,因守斗。占曰:「其國絕祀。」至其十二月,水、火合於斗。占曰:「為淬,不可舉事用兵,必受其殃。」一曰:「為北軍,用兵舉事大敗。斗,吳也,又為粵。」是歲彗星出西南。其三月,立六皇子為王,淮陽、汝南、河間、臨江、長沙、廣川。其三年,吳、楚、膠西、膠東、淄川、濟南、趙七國反。吳、楚兵先至攻梁,膠西、膠東、淄川三國攻圍齊。漢遣大將軍周亞夫等戍止河南,以候吳楚之敝,遂敗之。吳王亡走粵,粵攻而殺之。平陽侯敗三國之師于齊,咸伏其辜,齊王自殺。漢兵以水攻趙城,城壞,王自殺。六月,立皇子二人、楚元王子一人為王,王膠西、中山、楚。徙濟北為淄川王,淮陽為魯王,汝南為江都王。七月,兵罷。天狗下,占為:「破軍殺將。狗又守禦類也,天狗所降,以戒守禦。」吳、楚攻梁,梁堅城守,遂伏尸流血其下。

三年,填星在婁,幾入,還居奎。奎,魯也。占曰:「其國得地為得填。」是歲魯為國。

四年七月癸未,火入東井,行陰,又以九月己未入輿鬼,戊寅出。占曰:「為誅罰,又為火災。」後二年,有栗氏事。其後未央東闕災。

中元年,填星當在觜觿、參,去居東井。占曰:「亡地,不乃有女憂。」其三年正月丁亥,金、木合於觜觿,為白衣之會。三月丁酉,彗星夜見西北,色白,長丈,在觜觿,且去益小,十五日不見。占曰:「必有破國亂君,伏死其辜。觜觿,梁也。」其五月甲午,金、木俱在東井。戉,金去木留,守之二十日。占曰:「傷成於戉。木為諸侯,誅將行於諸侯也。」其六月壬戌,蓬星見西南,在房南,去房可二丈,大如二斗器,色白;癸亥,在心東北,可長丈所;甲子,在尾北,可六丈;丁卯,在箕北,近漢,稍小,且去時,大如桃。壬申去,凡十日。占曰:「蓬星出,必有亂臣。房、心間,天子宮也。」是時梁王欲為漢嗣,使人殺漢爭臣袁盎。漢桉誅梁大臣,斧戉用。梁王恐懼,布車入關,伏斧戉謝罪,然後得免。

中三年十一月庚午夕,金、火合於虛,相去一寸。占曰:「為鑠,為喪。虛,齊也。」

四年四月丙申,金、木合於東井。占曰:「為白衣之會。非,秦也。」其五年四月乙巳,水、火合於參。占曰:「國不吉。參,梁也。」其六年四月,梁孝王死。五月,城陽王、濟陰王死。六月,成陽公主死。出入三月,天子四衣白,臨邸第。

後元年五月壬午,火、金合於輿鬼之東北,不至柳,出輿鬼北可五寸。占曰:「為鑠,有喪。輿鬼,秦也。」丙戌,地大動,鈴鈴然,民大疫死,棺貴,至秋止。

孝武建元三年三月,有星孛於注、張,歷太微,干紫宮,至於天漢。春秋「星孛於北斗,齊、魯、晉之君皆將死亂」。今星孛歷五宿,其後濟東、膠西、江都王皆坐法削黜自殺,淮陽、衡山謀反而誅。

三年四月,有星孛於天紀,至織女。占曰:「織女有女變,天紀為地震。」至四年十月而地動,其後陳皇后廢。

六年,熒惑守輿鬼。占曰:「為火變,有喪。」是歲高園有火災,竇太后崩。

元光元年六月,客星見于房。占曰:「為兵起。」其二年十一月,單于將十萬騎入武州,漢遣兵三十餘萬以待之。

元光中,天星盡搖,上以問候星者。對曰:「星搖者,民勞也。」後伐四夷,百姓勞于兵革。

元鼎五年,太白入于天苑。占曰:「將以馬起兵也。」一曰:「

馬將以軍而死秏。」其後以天馬故誅大宛,馬大死於軍。

元鼎中,熒惑守南斗。占曰:「熒惑所守,為亂賊喪兵;守之久,其國絕祀。南斗,越分也。」其後越相呂嘉殺其王及太后,漢兵誅之,滅其國。

元封中,星孛于河戍。占曰:「南戍為越門,北戍為胡門。」其後漢兵擊拔朝鮮,以為樂浪、玄菟郡。朝鮮在海中,越之象也;居北方,胡之域也。

太初中,星孛于招搖。傳曰:「客星守招搖,蠻夷有亂,民死君。」其後漢兵擊大宛,斬其王。招搖,遠夷之分也。

孝昭始元中,漢宦者梁成恢及燕王候星者吳莫如見蓬星出西方天市東門,行過河鼓,入營室中。恢曰:「蓬星出六十日,不出三年,下有亂臣戮死於市。」後太白出西方,下行一舍,復上行二舍而下去。太白主兵,上復下,將有戮死者。後太白出東方,入咸池,東下入東井。人臣不忠,有謀上者。後太白入太微西藩第一星,北出東藩第一星,北東下去。太微者,天廷也,太白行其中,宮門當閉,大將被甲兵,邪臣伏誅。熒惑在婁,逆行至奎,法曰「當有兵」。後太白入昴。莫如曰:「蓬星出西方,當有大臣戮死者。太白星入東井、太微廷,出東門,漢有死將。」後熒惑出東方,守太白。兵當起,主人不勝。後流星下燕萬載宮極,東去,法曰「國恐,有誅」。其後左將軍桀、票騎將軍安與長公主、燕剌王謀亂,咸伏其辜。兵誅烏桓。

元鳳四年九月,客星在紫宮中斗樞極間。占曰:「為兵。」其五年六月,發三輔郡國少年詣北軍。五年四月,燭星見奎、婁間。占曰「有土功,胡人死,邊城和。」其六年正月,築遼東、玄菟城。二月,度遼將軍范明友擊烏桓還。

元平元年正月庚子,日出時有黑雲,狀如焱風亂鬊,轉出西北,東南行,轉而西,有頃亡。占曰:「有雲如眾風,是謂風師,法有大兵。」其後兵起烏孫,五將征匈奴。

二月甲申,晨有大星如月,有眾星隨而西行。乙酉,牂雲如狗,赤色,長尾三枚,夾漢西行。大星如月,大臣之象,眾星隨之,眾皆隨從也。天文以東行為順,西行為逆,此大臣欲行權以安社稷。占曰:「太白散為天狗,為卒起。卒起見,禍無時,臣運柄。牂雲為亂君。」到其四月,昌邑王賀行淫辟,立二十七日,大將軍霍光白皇太后廢賀。

三月丙戌,流星出翼、軫東北,干太微,入紫宮。始出小,且入大,有光。入有頃,聲如雷,三鳴止。占曰:「流星入紫宮,天下大凶。」其四月癸未,宮車晏駕。

孝宣本始元年四月壬戌甲夜,辰星與參出西方。其二年七月辛亥夕,辰星與翼出,皆為蚤。占曰:「大臣誅。」其後熒惑守房之鉤鈐。鉤鈐,天子之御也。占曰:「不太僕,則奉車,不黜即死也。房、心,天子宮也。房為將相,心為子屬也。其地宋,今楚彭城也。」四年七月甲辰,辰星在翼,月犯之。占曰:「兵起,上卿死,將相也。」是日,熒惑入輿鬼天質。占曰:「大臣有誅者,名曰天賊在大人之側。」

地節元年正月戊午乙夜,月食熒惑,熒惑在角、亢。占曰:「憂在宮中,非賊而盜也。有內亂,讒臣在旁。」其辛酉,熒惑入氐中。氐,天子之宮,熒惑入之,有賊臣。其六月戊戌甲夜,客星又居左右角間,東南指,長可二尺,色白。占曰:「有姦人在宮廷間。」其丙寅,又有客星見貫索東北,南行,至七月癸酉夜入天市,芒炎東南指,其色白。占曰:「有戮卿。」一曰:「有戮王。期皆一年,遠二年。」是時,楚王延壽謀逆自殺。四年,故大將軍霍光夫人顯、將軍霍禹、范明友、奉車霍山及諸昆弟賓婚為侍中、諸曹、九卿、郡守皆謀反,咸伏其辜。

黃龍元年三月,客星居王梁東北可九尺,長丈餘,西指,出閣道間,至紫宮。其十二月,宮車晏駕。

元帝初元元年四月,客星大如瓜,色青白,在南斗第二星東可四尺。占曰:「為水飢。」其五月,勃海水大溢。六月,關東大飢,民多餓死,琅邪郡人相食。

二年五月,客星見昴分,居卷舌東可五尺,青白色,炎長三寸。占曰:「天下有妄言者。」其十二月,鉅鹿都尉謝君男詐為神人,論死,父免官。

五年四月,彗星出西北,赤黃色,長八尺所,後數日長丈餘,東北指,在參分。後二歲餘,西羌反。

孝成建始元年九月戊子,有流星出文昌,色白,光燭地,長可四丈,大一圍,動搖如龍蛇形。有頃,長可五六丈,大四圍所,詘折委曲,貫紫宮西,在斗西北子亥間。後詘如環,北方不合,留一合所。占曰:「文昌為上將貴相。」是時帝舅王鳳為大將軍,其後宣帝舅子王商為丞相,皆貴重任政。鳳妒商,譖而罷之。商自殺,親屬皆廢黜。

四年七月,熒惑隃歲星,居其東北半寸所如連李。時歲星在關星西四尺所,熒惑初從畢口大星東東北往,數日至,往疾去遲。占曰:「熒惑與歲星鬥,有病君飢歲。」至河平元年三月,旱,傷麥,民食榆皮。二年十二月壬申,太皇太后避時昆明東觀。

十一月乙卯,月食填星,星不見,時在輿鬼西北八九尺所。占曰:「月食填星,流民千里。」河平元年三月,流民入函谷關。

河平二年十月下旬,填星在東井軒轅南耑大星尺餘,歲星在其西北尺所,熒惑在其西北二尺所,皆從西方來。填星貫輿鬼,先到歲星次,熒惑亦貫輿鬼。十一月上旬,歲星、熒惑西去填星,皆西北逆行。占曰:「三星若合,是謂驚位,是謂絕行,外內有兵與喪,改立王公。」其十一月丁巳,夜郎王歆大逆不道,牂柯太守立捕殺歆。三年九月甲戌,東郡莊平男子侯母辟兄弟五人群黨為盜,攻燔官寺,縛縣長吏,盜取印綬,自稱將軍。三月辛卯,左將軍千秋卒,右將軍史丹為左將軍。四年四月戊申,梁王賀薨。

陽朔元年七月壬子,月犯心星。占曰:「其國有憂,若有大喪。房、心為宋,今楚地。」十一月辛未,楚王友薨。

四夫閏月庚午,飛星大如缶,出西南,入斗下。占曰:「漢使匈奴。」明年,鴻嘉元年正月,匈奴單于雕陶莫皋死。五月甲午,遣中郎將楊興使弔。

永始二年二月癸未夜,東方有赤色,大三四圍,長二三丈,索索如樹,南方有大四五圍,下行十餘丈,皆不至地滅。占曰:「東方客之變氣,狀如樹木,以此知四方欲動者。」明年十二月己卯,尉氏男子樊並等謀反,賊殺陳留太守嚴普及吏民,出囚徒,取庫兵,劫略令丞,自稱將軍,皆誅死。庚子,山陽鐵官亡徒蘇令等殺傷吏民,篡出囚徒,取庫兵,聚黨數百人為大賊,踰年經歷郡國四十餘。一日有兩氣同時起,並見,而並、令等同月俱發也。

元延元年四月丁酉日餔時,天蛲晏,殷殷如雷聲,有流星頭大如缶,長十餘丈,皎然赤白色,從日下東南去。四面或大如盂,或如雞子,燿燿如雨下,至昏止。郡國皆言星隕。春秋星隕如雨為王者失勢諸侯起伯之異也。其後王莽遂顓國柄。王氏之興萌於成帝,是以有星隕之變。後莽遂篡國。

綏和元年正月辛未,有流星從東南入北斗,長數十丈,二刻所息。占曰:「大臣有繫者。」其年十一月庚子,定陵侯淳于長坐執左道下獄死。

二年春,熒惑守心。二月乙丑,丞相翟方進欲塞災異,自殺。

二月丙戌,宮車晏駕。

哀帝建平元年正月丁未日出時,有著天白氣,廣如一匹布,長十餘丈,西南行,讙如雷,西南行一刻而止,名曰天狗。傳曰:「言之不從,則有犬禍詩妖。」到其四年正月、二月、三月,民相驚動,讙譁奔走,傳行詔籌祠西王母,又曰「從目人當來」。十二月,白氣出西南,從地上至天,出參下,貫天廁,廣如一疋布,長十餘丈,十餘日去。占曰:「天子有陰病。」其三年十一月壬子,太皇太后詔曰:「皇帝寬仁孝順,奉承聖緒,靡有解怠,而久病未瘳。夙夜惟思,殆繼體之君不宜改作。春秋大復古,其復甘泉泰畤、汾陰后土如故。」

二年二月,彗星出牽牛七十餘日。傳曰:「彗所以除舊布新也。牽牛,日、月、五星所從起,曆數之元,三正之始。彗而出之,改更之象也。其出久者,為其事大也。」其六月甲子,夏賀良等建言當改元易號,增漏刻。詔書改建平二年為太初元將元年,號曰陳聖劉太平皇帝,刻漏以百二十為度。八月丁巳,悉復蠲除之,賀良及黨與皆伏誅流放。其後卒有王莽篡國之禍。

元壽元年十一月,歲星入太微,逆行干右執法。占曰:「大臣有憂,執法者誅,若有罪。」二年十月戊寅,高安侯董賢免大司馬位,歸第自殺。


\end{pinyinscope}