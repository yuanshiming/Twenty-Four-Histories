\article{韋賢傳}

\begin{pinyinscope}
韋賢字長孺,魯國鄒人也。其先韋孟,家本彭城,為楚元王傅,傅子夷王及孫王戊。戊荒淫不遵道,孟作詩風諫。後遂去位,徙家於鄒,又作一篇。其諫詩曰:

肅肅我祖,國自豕韋,黼衣朱紱,四牡龍旂。彤弓斯征,撫寧遐荒,總齊群邦,以翼大商,迭彼大彭,勳績惟光。至于有周,歷世會同。王赧聽譖,寔絕我邦。我邦既絕,厥政斯逸,賞罰之行,非繇王室。庶尹群后,靡扶靡衛,五服崩離,宗周以隊。我祖斯微,遷于彭城,在予小子,勤誒厥生,阨此嫚秦,耒耜以耕。悠悠嫚秦,上天不寧,乃眷南顧,授漢于京。

於赫有漢,四方是征,靡適不懷,萬國逌平。乃命厥弟,建侯於楚,俾我小臣,惟傅是輔。兢兢元王,恭儉淨壹,惠此黎民,納彼輔弼。饗國漸世,垂烈于後,乃及夷王,克奉厥緒。咨命不永,唯王統祀,左右陪臣,此惟皇士。

如何我王,不思守保,不惟履冰,以繼祖考!邦事是廢,逸游是娛,犬馬繇繇,是放是驅。務彼鳥獸,忽此稼苗,烝民以匱,我王以媮。所弘非德,所親非俊,唯囿是恢,唯諛是信。睮睮諂夫,咢咢黃髮,如何我王,曾不是察!既藐下臣,追欲從逸,嫚彼顯祖,輕茲削黜。

嗟嗟我王,漢之睦親,曾不夙夜,以休令聞!穆穆天子,臨爾下土,明明群司,執憲靡顧。正遐繇近,殆其怙茲,嗟嗟我王,曷不此思!

非思非鑒,嗣其罔則,彌彌其失,岌岌其國。致冰匪霜,致隊靡嫚,瞻惟我王,昔靡不練。興國救顛,孰違悔過,追思黃髮,秦繆以霸。歲月其徂,年其逮耇,於昔君子,庶顯于後。我王如何,曾不斯覽!黃髮不近,胡不時監!

其在鄒詩曰:

微微小子,既耇且陋,豈不牽位,穢我王朝。王朝肅清,唯俊之庭,顧瞻余躬,懼穢此征。

我之退征,請于天子,天子我恤,矜我髮齒。赫赫天子,明悊且仁,懸車之義,以洎小臣。嗟我小子,豈不懷土?庶我王寤,越遷于魯。

既去禰祖,惟懷惟顧,祁祁我徒,戴負盈路。爰戾于鄒,鬋茅作堂,我徒我環,築室于牆。

我既讓逝,心存我舊,夢我瀆上,立于王朝。其夢如何?夢爭王室。其爭如何?夢王我弼。寤其外邦,歎其喟然,念我祖考,泣涕其漣。微微老夫,咨既遷絕,洋洋仲尼,視我遺烈。濟濟鄒魯,禮義唯恭,誦習弦歌,于異他邦。我雖鄙耇,心其好而,我徒侃爾,樂亦在而。

孟卒于鄒。或曰其子孫好事,述先人之志而作是詩也。

自孟至賢五世。賢為人質朴少欲,篤志於學,兼通禮、尚書,以詩教授,號稱鄒魯大儒。徵為博士,給事中,進授昭帝詩,稍遷光祿大夫詹事,至大鴻臚。昭帝崩,無嗣,大將軍霍光與公卿共尊立孝宣帝。帝初即位,賢以與謀議,安宗廟,賜爵關內侯,食邑。徙為長信少府。以先帝師,甚見尊重。本始三年,代蔡義為丞相,封扶陽侯,食邑七百戶。時賢七十餘,為相五歲,地節三年以老病乞骸骨,賜黃金百斤,罷歸,加賜弟一區。丞相致仕自賢始。年八十二薨,諡曰節侯。

賢四子:長子方山為高寢令,早終;次子弘,至東海太守;次子舜,留魯守墳墓;少子玄成,復以明經歷位至丞相。故鄒魯諺曰:「

遺子黃金滿籯,不如一經。」

玄成字少翁,以父任為郎,常侍騎。少好學,修父業,尤謙遜下士。出遇知識步行,輒下從者,與載送之,以為常。其接人,貧賤者益加敬,繇是名譽日廣。以明經擢為諫大夫,遷大河都尉。

初,玄成兄弘為太常丞,職奉宗廟,典諸陵邑,煩劇多罪過。父賢以弘當為嗣,故敕令自免。弘懷謙,不去官。及賢病篤,弘竟坐宗廟事繫獄,罪未決。室家問賢當為後者,賢恚恨不肯言。於是賢門下生博士義倩等與宗家計議,共矯賢令,使家丞上書言大行,以大河都尉玄成為後。賢薨,玄成在官聞喪,又言當為嗣,玄成深知其非賢雅意,即陽為病狂,臥便利,妄笑語昏亂。徵至長安,既葬,當襲爵,以病狂不應召。大鴻臚奉狀,章下丞相御史案驗。玄成素有名聲,士大夫多疑其欲讓爵辟兄者。案事丞相史乃與玄成書曰:「古之辭讓,必有文義可觀,故能垂榮於後。今子獨壞容貌,蒙恥辱,為狂癡,光曜晻而不宣。微哉!子之所託名也。僕素愚陋,過為宰相執事,願少聞風聲。不然,恐子傷高而僕為小人也。」玄成友人侍郎章亦上疏言:「聖王貴以禮讓為國,宜優養玄成,勿枉其志,使得自安衡門之下。」而丞相御史遂以玄成實不病,劾奏之。有詔勿劾,引拜。玄成不得已受爵。宣帝高其節,以玄成為河南太守。兄弘太山都尉,遷東海太守。

數歲,玄成徵為未央衛尉,遷太常。坐與故平通侯楊惲厚善,惲誅,黨友皆免官。後以列侯侍祀孝惠廟,當晨入廟,天雨淖,不駕駟馬車而騎至廟下。有司劾奏,等輩數人皆削爵為關內侯。玄成自傷貶黜父爵,歎曰:「吾何面目以奉祭祀!」作詩自劾責,曰:

赫矣我祖,侯于豕韋,賜命建伯,有殷以綏。厥績既昭,車服有常,朝宗商邑,四牡翔翔。德之令顯,慶流于裔,宗周至漢,群后歷世。

肅肅楚傅,輔翼元、夷,厥駟有庸,惟慎惟祗。嗣王孔佚,越遷于鄒,五世壙僚,至我節侯。

惟我節侯,顯德遐聞,左右昭、宣,五品以訓。既耇致位,惟懿惟奐,厥賜祁祁,百金洎館。國彼扶陽,在京之東,惟帝是留,政謀是從。繹繹六轡,是列是理,威儀濟濟,朝享天子。天子穆穆,是宗是師,四方遐爾,觀國之煇。

茅土之繼,在我俊兄,惟我俊兄,是讓是形。於休厥德,於赫有聲,致我小子,越留於京。惟我小子,不肅會同,惮彼車服,黜此附庸。

赫赫顯爵,自我隊之;微微附庸,自我招之。誰能忍媿,寄之我顏;誰將遐征,從之夷蠻。於赫三事,匪俊匪作,於蔑小子,終焉其度。誰謂華高,企其齊而;誰謂德難,厲其庶而。嗟我小子,于貳其尤,隊彼令聲,申此擇辭。四方群后,我監我視,威儀車服,唯肅是履!

初,宣帝寵姬張婕妤男淮陽憲王好政事,通法律,上奇其材,有意欲以為嗣,然用太子起於細微,又早失母,故不忍也。久之,上欲感風憲王,輔以禮讓之臣,乃召拜玄成為淮陽中尉。是時王未就國,玄成受詔,與太子太傅蕭望之及五經諸儒雜論同異於石渠閣,條奏其對。及元帝即位,以玄成為少府,遷太子太傅,至御史大夫。永光中,代于定國為丞相。貶黜十年之間,遂繼父相位,封侯故國,榮當世焉。玄成復作詩,自著復玷缺之艱難,因以戒示子孫,曰:

於肅君子,既令厥德,儀服此恭,棣棣其則。咨余小子,既德靡逮,曾是車服,荒嫚以隊。

明明天子,俊德烈烈,不遂我遺,恤我九列。我既茲恤,惟夙惟夜,畏忌是申,供事靡惰。天子我監,登我三事,顧我傷隊,爵復我舊。

我既此登,望我舊階,先后茲度,漣漣孔懷。司直御事,我熙我盛;群公百僚,我嘉我慶。于異卿士,非同我心,三事惟谡,莫我肯矜。赫赫三事,力雖此畢,非吾所度,退其罔日。昔我之隊,畏不此居,今我度茲,戚戚其懼。

嗟我後人,命其靡常,靖享爾位,瞻仰靡荒。慎爾會同,戒爾車服,無惰爾儀,以保爾域。爾無我視,不慎不整;我之此復,惟祿之幸。於戲後人,惟肅惟栗。無忝顯祖,以蕃漢室!

玄成為相七年,守正持重不及父賢,而文采過之。建昭三年薨,諡曰共侯。初,賢以昭帝時徙平陵,玄成別徙杜陵,病且死,因使者自白曰:「不勝父子恩,願乞骸骨,歸葬父墓。」上許焉。

子頃侯寬嗣。薨,子僖侯育嗣。薨,子節侯沈嗣。自賢傳國至玄孫乃絕。玄成兄高寢令方山子安世歷郡守,大鴻臚,長樂衛尉,朝廷稱有宰相之器,會其病終。而東海太守弘子賞亦明詩。哀帝為定陶王時,賞為太傅。哀帝即位,賞以舊恩為大司馬車騎將軍,列為三公,賜爵關內侯,食邑千戶,亦年八十餘,以壽終。宗族至吏二千石者十餘人。

初,高祖時,令諸侯王都皆立太上皇廟。至惠帝尊高帝廟為太祖廟,景帝尊孝文廟為太宗廟,行所嘗幸郡國各立太祖、太宗廟。至宣帝本始二年,復尊孝武廟為世宗廟,行所巡狩亦立焉。凡祖宗廟在郡國六十八,合百六十七所。而京師自高祖下至宣帝,與太上皇、悼皇考各自居陵旁立廟,并為百七十六。又園中各有寢、便殿。日祭於寢,月祭於廟,時祭於便殿。寢,日四上食;廟,歲二十五祠;便殿,歲四祠。又月一游衣冠。而昭靈后、武哀王、昭哀后、孝文太后、孝昭太后、衛思后、戾太子、戾后各有寢園,與諸帝合,凡三十所。一歲祠,上食二萬四千四百五十五,用衛士四萬五千一百二十九人,祝宰樂人萬二千一百四十七人,養犧牲卒不在數中。

至元帝時,貢禹奏言:「古者天子七廟,今孝惠、孝景廟皆親盡,宜毀。及郡國廟不應古禮,宜正定。」天子是其議,未及施行而禹卒。永光四年,乃下詔先議罷郡國廟,曰:「朕聞明王之御世也,遭時為法,因事制宜。往者天下初定,遠方未賓,因嘗所親以立宗廟,蓋建威銷萌,一民之至權也。今賴天地之靈,宗廟之福,四方同軌,蠻貊貢職,久遵而不定,令疏遠卑賤共承尊祀,殆非皇天祖宗之意,朕甚懼焉。傳不云乎?『吾不與祭,如不祭。』其與將軍、列侯、中二千石、二千石、諸大夫、博士、議郎議。」丞相玄成、御史大夫鄭弘、太子太傅嚴彭祖、少府歐陽地餘、諫大夫尹更始等七十人皆曰:「臣聞祭,非自外至者也,繇中出,生於心也。故唯聖人為能饗帝,孝子為能饗親。立廟京師之居,躬親承事,四海之內各以其職來助祭,尊親之大義,五帝三王所共,不易之道也。《詩》云:『有來雍雍,至止肅肅,相維辟公,天子穆穆。』春秋之義,父不祭於支庶之宅,君不祭於臣僕之家,王不祭於下土諸侯。臣等愚以為宗廟在郡國,宜無修,臣請勿復修。」奏可。因罷昭靈后、武哀王、昭哀后、衛思后、戾太子、戾后園,皆不奉祠,裁置吏卒守焉。

罷郡國廟後月餘,復下詔曰:「蓋聞明王制禮,立親廟四,祖宗之廟,萬世不毀,所以明尊祖敬宗,著親親也。朕獲承祖宗之重,惟大禮未備,戰栗恐懼,不敢自顓,其與將軍、列侯、中二千石、二千石、諸大夫、博士議。」玄成等四十四人奏議曰:「禮,王者始受命,諸侯始封之君,皆為太祖。以下,五廟而迭毀,毀廟之主臧乎太祖,五年而再殷祭,言壹禘壹祫也。祫祭者,毀廟與未毀廟之主皆合食於太祖,父為昭,子為穆,孫復為昭,古之正禮也。祭義曰:『王者禘其祖自出,以其祖配之,而立四廟。』言始受命而王,祭天以其祖配,而不為立廟,親盡也。立親廟四,親親也。親盡而迭毀,親疏之殺,示有終也。周之所以七廟者,以后稷始封,文王、武王受命而王,是以三廟不毀,與親廟四而七。非有后稷始封,文、武受命之功者,皆當親盡而毀。成王成二聖之業,制禮作樂,功德茂盛,廟猶不世,以行為諡而已。禮,廟在大門之內,不敢遠親也。臣愚以為高帝受命定天下,宜為帝者太祖之廟,世世不毀,承後屬盡者宜毀。今宗廟異處,昭穆不序,宜入就太祖廟而序昭穆如禮。太上皇、孝惠、孝文、孝景廟皆親盡宜毀,皇考廟親未盡,如故。」大司馬車騎將軍許嘉等二十九人以為孝文皇帝除誹謗,去肉刑,躬節儉,不受獻,罪人不帑,不私其利,出美人,重絕人類,賓賜長老,收恤孤獨,德厚侔天地,利澤施四海,宜為帝者太宗之廟。廷尉忠以為孝武皇帝改正朔,易服色,攘四夷,宜為世宗之廟。諫大夫尹更始等十八人以為皇考廟上序於昭穆,非正禮,宜毀。

於是上重其事,依違者一年,乃下詔曰:「蓋聞王者祖有功而宗有德,尊尊之大義也;存親廟四,親親之至恩也。高皇帝為天下誅暴除亂,受命而帝,功莫大焉。孝文皇帝國為代王,諸呂作亂,海內搖動,然群臣黎庶靡不壹意,北面而歸心,猶謙辭固讓而後即位,削亂秦之跡,興三代之風,是以百姓晏然,咸獲嘉福,德莫盛焉。高皇帝為漢太祖,孝文皇帝為太宗,世世承祀,傳之無窮,朕甚樂之。孝宣皇帝為孝昭皇帝後,於義壹體。孝景皇帝廟及皇考廟皆親盡,其正禮儀。」玄成等奏曰:「祖宗之廟世世不毀,繼祖以下,五廟而迭毀。今高皇帝為太祖,孝文皇帝為太宗,孝景皇帝為昭,孝武皇帝為穆,孝昭皇帝與孝宣皇帝俱為昭。皇考廟親未盡。太上、孝惠廟皆親盡,宜毀。太上廟主宜瘞園,孝惠皇帝為穆,主遷於太祖廟,寢園皆無復修。」奏可。

議者又以為清廟之詩言交神之禮無不清靜,今衣冠出游,有車騎之眾,風雨之氣,非所謂清靜也。「祭不欲數。數則瀆,瀆則不敬。」宜復古禮,四時祭於廟,諸寢園日月間祀皆可勿復修。上亦不改也。明年,玄成復言:「古者制禮,別尊卑貴賤,國君之母非適不得配食,則薦於寢,身沒而已。陛下躬至孝,承天心,建祖宗,定迭毀,序昭穆,大禮既定,孝文太后、孝昭太后寢祠園宜如禮勿復修。」奏可。

後歲餘,玄成薨,匡衡為丞相。上寢疾,夢祖宗譴罷郡國廟,上少弟楚孝王亦夢焉。上詔問衡,議欲復之,衡深言不可。上疾久不平,衡惶恐,禱高祖、孝文、孝武廟曰:「嗣曾孫皇帝恭承洪業,夙夜不敢康寧,思育休烈,以章祖宗之盛功。故動作接神,必因古聖之經。往者有司以為前因所幸而立廟,將以繫海內之心,非為尊祖嚴親也。今賴宗廟之靈,六合之內莫不附親,廟宜一居京師,天子親奉,郡國廟可止毋修。皇帝祗肅舊禮,尊重神明,即告于祖宗而不敢失。今皇帝有疾不豫,乃夢祖宗見戒以廟,楚王夢亦有其序。皇帝悼懼,即詔臣衡復修立。謹案上世帝王承祖禰之大義,皆不敢不自親。郡國吏卑賤,不可使獨承。又祭祀之義以民為本,間者歲數不登,百姓困乏,郡國廟無以修立。禮,凶年則歲事不舉,以祖禰之意為不樂,是以不敢復。如誠非禮義之中,違祖宗之心,咎盡在臣衡,當受其殃,大被其疾,隊在溝瀆之中。皇帝至孝肅慎,宜蒙祐福。唯高皇帝、孝文皇帝、孝武皇帝省察,右饗皇帝之孝,開賜皇帝眉壽亡疆,令所疾日瘳,平復反常,永保宗廟,天下幸甚!」

又告謝毀廟曰:「往者大臣以為在昔帝王承祖宗之休典,取象於天地,天序五行,人親五屬,天子奉天,故率其意而尊其制。是以禘嘗之序,靡有過五。受命之君躬接于天,萬世不墮。繼烈以下,五廟而遷,上陳太祖,間歲而祫,其道應天,故福祿永終。太上皇非受命而屬盡,義則當遷。又以為孝莫大於嚴父,故父之所尊子不敢不承,父之所異子不敢同。禮,公子不得為母信,為後則於子祭,於孫止,尊祖嚴父之義也。寢日四上食,園廟間祠,皆可亡修。皇帝思慕悼懼,未敢盡從。惟念高皇帝聖德茂盛,受命溥將,欽若稽古,承順天心,子孫本支,陳錫亡疆。誠以為遷廟合祭,久長之策,高皇帝之意,乃敢不聽?即以令日遷太上、孝惠廟,孝文太后、孝昭太后寢,將以昭祖宗之德,順天人之序,定無窮之業。今皇帝未受茲福,乃有不能共職之疾。皇帝願復修承祀,臣衡等咸以為禮不得。如不合高皇帝、孝惠皇帝、孝文皇帝、孝武皇帝、孝昭皇帝、孝宣皇帝、太上皇、孝文太后、孝昭太后之意,罪盡在臣衡等,當受其咎。今皇帝尚未平,詔中朝臣具復毀廟之文。臣衡中朝臣咸復以為天子之祀義有所斷,禮有所承,違統背制,不可以奉先祖,皇天不祐,鬼神不饗。六藝所載,皆言不當,無所依緣,以作其文。事如失指,罪乃在臣衡,當深受其殃。皇帝宜厚蒙祉福,嘉氣日興,疾病平復,永保宗廟,與天亡極,群生百神,有所歸息。」諸廟皆同文。

久之,上疾連年,遂盡復諸所罷寢廟園,皆修祀如故。初,上定迭毀禮,獨尊孝文廟為太宗,而孝武廟親未盡,故未毀。上於是乃復申明之,曰:「孝宣皇帝尊孝武廟曰世宗,損益之禮,不敢有與焉。他皆如舊制。」唯郡國廟遂廢云。

元帝崩,衡奏言:「前以上體不平,故復諸所罷祠,卒不蒙福。案衛思后、戾太子、戾后園,親未盡。孝惠、孝景廟親盡,宜毀。及太上皇、孝文、孝昭太后、昭靈后、昭哀后、武哀王祠,請悉罷,勿奉。」奏可。初,高后時患臣下妄非議先帝宗廟寢園官,故定著令,敢有擅議者棄市。至元帝改制,蠲除此令。成帝時以無繼嗣,河平元年復復太上皇寢廟園,世世奉祠。昭靈后、武哀王、昭哀后并食於太上寢廟如故,又復擅議宗廟之命。

成帝崩,哀帝即位。丞相孔光、大司空何武奏言:「永光五年制書,高皇帝為漢太祖,孝文皇帝為太宗。建昭五年制書,孝武皇帝為世宗。損益之禮,不敢有與。臣愚以為迭毀之次,當以時定,非令所為擅議宗廟之意也。臣請與群臣雜議。」奏可。於是,光祿勳彭宣、詹事滿昌、博士左咸等五十三人皆以為繼祖宗以下,五廟而迭毀,後雖有賢君,猶不得與祖宗並列。子孫雖欲褒大顯揚而立之,鬼神不饗也。孝武皇帝雖有功烈,親盡宜毀。

太僕王舜、中壘校尉劉歆議曰:「臣聞周室既衰,四夷並侵,獫狁最彊,於今匈奴是也。至宣王而伐之,詩人美而頌之曰『薄伐獫狁,至于太原』,又曰『嘽嘽推推,如霆如雷,顯允方叔,征伐獫狁,荊蠻來威』,故稱中興。及至幽王,犬戎來伐,殺幽王,取宗器。自是之後,南夷與北夷交侵,中國不絕如悋。春秋紀齊桓南伐楚,北伐山戎,孔子曰:『微管仲,吾其被髮左衽矣。』是故棄桓之過而錄其功,以為伯首。及漢興,冒頓始彊,破東胡,禽月氏,并其土地,地廣兵彊,為中國害。南越尉佗總百粵,自稱帝。故中國雖平,猶有四夷之患,且無寧歲。一方有急,三面救之,是天下皆動而被其害也。孝文皇帝厚以貨賂,與結和親,猶侵暴無已。甚者,興師十餘萬眾,近屯京師及四邊,歲發屯備虜,其為患久矣,非一世之漸也。諸侯郡守連匈奴及百粵以為逆者非一人也。匈奴所殺郡守都尉,略取人民,不可勝數。孝武皇帝愍中國罷勞無安寧之時,乃遣大將軍、驃騎、伏波、樓船之屬,南滅百粵,起七郡;北攘匈奴,降昆邪十萬之眾,置五屬國,起朔方,以奪其肥饒之地;東伐朝鮮,起玄菟、樂浪,以斷匈奴之左臂;西伐大宛,並三十六國,結烏孫,起敦煌、酒泉、張掖,以鬲婼羌,裂匈奴之右肩。單于孤特,遠遁于幕北。四垂無事,斥地遠境,起十餘郡。功業既定,乃封丞相為富民侯,以大安天下,富實百姓,其規跻可見。又招集天下賢俊,與協心同謀,興制度,改正朔,易服色,立天地之祠,建封禪,殊官號,存周後,定諸侯之制,永無逆爭之心,至今累世賴之。單于守藩,百蠻服從,萬世之基也,中興之功未有高焉者也。高帝建大業,為太祖;孝文皇帝德至厚也,為文太宗;孝武皇帝功至著也,為武世宗;此孝宣帝所以發德音也。禮記王制及春秋穀梁傳,天子七廟,諸侯五,大夫三,士二。天子七日而殯,七月而葬;諸侯五日而殯,五月而葬;此喪事尊卑之序也,與廟數相應。其文曰:『天子三昭三穆,與太祖之廟而七;諸侯二昭二穆,與太祖之廟而五。』故德厚者流光,德薄者流卑。春秋左氏傳曰:『名位不同,禮亦異數。』自上以下,降殺以兩,禮也。七者,其正法數,可常數者也。宗不在此數中。宗,變也,苟有功德則宗之,不可預為設數。故於殷,太甲為太宗,大戊曰中宗,武丁曰高宗。周公為毋逸之戒,舉殷三宗以勸成王。繇是言之,宗無數也,然則所以勸帝者之功德博矣。以七廟言之,孝武皇帝未宜毀;以所宗言之,則不可謂無功德。禮記祀典曰:『夫聖王之制祀也,功施於民則祀之,以勞定國則祀之,能救大災則祀之。』竊觀孝武皇帝,功德皆兼而有焉。凡在於異姓,猶將特祀之,況于先祖?或說天子五廟無見文,又說中宗、高宗者,宗其道而毀其廟。名與實異,非尊德貴功之意也。《詩》云:『蔽芾甘棠,勿鬋勿伐,邵伯所茇。』思其人猶愛其樹,況宗其道而毀其廟乎?迭毀之禮自有常法,無殊功異德,固以親疏相推及。至祖宗之序,多少之數,經傳無明文,至尊至重,難以疑文虛說定也。孝宣皇舉公卿之議,用眾儒之謀,既以為世宗之廟,建之萬世,宣布天下。臣愚以為孝武皇帝功烈如彼,孝宣皇帝崇立之如此,不宜毀。」上覽其議而從之。制曰:「太僕舜、中壘校尉歆議可。」

歆又以為「禮,去事有殺,故春秋外傳曰:『日祭,月祀,時享,歲貢,終王。』祖禰則日祭,曾高則月祀,二祧則時享,壇墠則歲貢,大禘則終王。德盛而游廣,親親之殺也;彌遠則彌尊,故禘為重矣。孫居王父之處,正昭穆,則孫常與祖相代,此遷廟之殺也。聖人於其祖,出於情矣,禮無所不順,故無毀廟。自貢禹建迭毀之議,惠、景及太上寢園廢而為虛,失禮意矣。」

至平帝元始中,大司馬王莽奏:「本始元年丞相義等議,諡孝宣皇帝親曰悼園,置邑三百家,至元康元年,丞相相等奏,父為士,子為天子,祭以天子,悼園宜稱尊號曰『皇考』,立廟,益故奉園民滿千六百家,以為縣。臣愚以為皇考廟本不當立,累世奉之,非是。又孝文太后南陵、孝昭太后雲陵園,雖前以禮不復修,陵名未正。謹與大司徒晏等百四十七人議,皆曰孝宣皇帝以兄孫繼統為孝昭皇帝後,以數,故孝元世以孝景皇帝及皇考廟親未盡,不毀。此兩統貳父,違於禮制。案義奏親諡曰『悼』,裁置奉邑,皆應經義。相奏悼園稱『皇考』,立廟,益民為縣,違離祖統,乖繆本義。父為士,子為天子,祭以天子者,乃謂若虞舜、夏禹、殷湯、周文、漢之高祖受命而王者也,非謂繼祖統為後者也。臣請皇高祖考廟奉明園毀勿修,罷南陵、雲陵為縣。」奏可。

司徒掾班彪曰:漢承亡秦絕學之後,祖宗之制因時施宜。自元、成後學者番滋,貢禹毀宗廟,匡衡改郊兆,何武定三公,後皆數復,故紛紛不定。何者?禮文缺微,古今異制,各為一家,未易可偏定也。考觀諸儒之議,劉歆博而篤矣。


\end{pinyinscope}