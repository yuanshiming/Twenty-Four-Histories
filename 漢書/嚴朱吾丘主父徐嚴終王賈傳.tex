\article{嚴朱吾丘主父徐嚴終王賈傳}

\begin{pinyinscope}
嚴助,會稽吳人,嚴夫子子也,或言族家子也。郡舉賢良,對策百餘人,武帝善助對,繇是獨擢助為中大夫。後得朱買臣、吾丘壽王、司馬相如、主父偃、徐樂、嚴安、東方朔、枚皋、膠倉、終軍、嚴蔥奇等,並在左右。是時征伐四夷,開置邊郡,軍旅數發,內改制度,朝廷多事,婁舉賢良文學之士。公孫弘起徒步,數年至丞相,開東閣,延賢人與謀議,朝覲奏事,因言國家便宜。上令助等與大臣辯論,中外相應以義理之文,大臣數詘。其尤親幸者,東方朔、枚皋、嚴助、吾丘壽王、司馬相如。相如常稱疾避事。朔、皋不根持論,上頗俳優畜之。唯助與壽王見任用,而助最先進。

建元三年,閩越舉兵圍東甌,東甌告急於漢。時武帝年未二十,以問太尉田蚡。蚡以為越人相攻擊,其常事,又數反覆,不足煩中國往救也,自秦時棄不屬。於是助詰蚡曰:「特患力不能救,德不能覆,誠能,何故棄之?且秦舉咸陽而棄之,何但越也!今小國以窮困來告急,天子不振,尚安所愬,又何以子萬國乎?」上曰:「太尉不足與計。吾新即位,不欲出虎符發兵郡國。」乃遣助以節發兵會稽。會稽守欲距法,不為發。助乃斬一司馬,諭意指,遂發兵浮海救東甌。未至,閩越引兵罷。

後三歲,閩越復興兵擊南越。南越守天子約,不敢擅發兵,而上書以聞。上多其義,大為發興,遣兩將軍將兵誅閩越。淮南王安上書諫曰:

陛下臨天下,布德施惠,緩刑罰,薄賦斂,哀鰥寡,恤孤獨,養耆老,振匱乏,盛德上隆,和澤下洽,近者親附,遠者懷德,天下攝然,人安其生,自以身不見兵革。今聞有司舉兵將以誅越,臣安竊為陛下重之。越,方外之地,劗髮文身之民也。不可以冠帶之國法度理也。自三代之盛,胡越不與受正朔,非彊弗能服,威弗能制也,以為不居之地,不牧之民,不足以煩中國也。故古者封內甸服,封外侯服,侯衛賓服,蠻夷要服,戎狄荒服,遠近勢異也。自漢初定已來七十二年,吳越人相攻擊者不可勝數,然天子未嘗舉兵而入其地也。

臣聞越非有城郭邑里也,處谿谷之間,篁竹之中,習於水鬥,便於用舟,地深昧而多水險,中國之人不知其勢阻而入其地,雖百不當其一。得其地,不可郡縣也;攻之,不可暴取也。以地圖察其山川要塞,相去不過寸數,而間獨數百千里,阻險林叢弗能盡著。視之若易,行之甚難。天下賴宗廟之靈,方內大寧,戴白之老不見兵革,民得夫婦相守,父子相保,陛下之德也。越人名為藩臣,貢酎之奉,不輸大內,一卒之用不給上事。自相攻擊而陛下發兵救之,是反以中國而勞蠻夷也。且越人愚戇輕薄,負約反覆,其不可用天子之法度,非一日之積也。壹不奉詔,舉兵誅之,臣恐後兵革無時得息也。

間者,數年歲比不登,民待賣爵贅子以接衣食,賴陛下德澤振救之,得毋轉死溝壑。四年不登,五年復蝗,民生未復。今發兵行數千里,資衣糧,入越地,輿轎而隃領,癴舟而入水,行數百千里,夾以深林叢竹,水道上下擊石,林中多蝮蛇猛獸,夏月暑時,歐泄霍亂之病相隨屬也,曾未施兵接刃,死傷者必眾矣。前時南海王反,陛下先臣使將軍間忌將兵擊之,以其軍降,處之上淦。後復反,會天暑多雨,樓船卒水居擊櫂,未戰而疾死者過半。親老涕泣,孤子謕號,破家散業,迎尸千里之外,裹骸骨而歸。悲哀之氣數年不息,長老至今以為記。曾未入其地而禍已至此矣。

臣聞:軍旅之後,必有凶年;言民之各以其愁苦之氣薄陰陽之和,感天地之精,而災氣為之生也。陛下德配天地,明象日月,恩至禽獸,澤及草木,一人有飢寒不終其天年而死者,為之悽愴於心。今方內無狗吠之警,而使陛下甲卒死亡,暴露中原,霑漬山谷,邊境之民為之早閉晏開,晁不及夕,臣安竊為陛下重之。

不習南方地形者,多以越為人眾兵彊,能難邊城。淮南全國之時,多為邊吏,臣竊聞之,與中國異。限以高山,人跡所絕,車道不通,天地所以隔外內也。其入中國必下領水,領水之山峭峻,漂石破舟,不可以大船載食糧下也。越人欲為變,必先田餘干界中,積食糧,乃入伐材治船。邊城守候誠謹,越人有入伐材者,輒收捕,焚其積聚,雖百越,奈邊城何!且越人綿力薄材,不能陸戰,又無車騎弓弩之用,然而不可入者,以保地險,而中國之人不能其水土也。臣聞越甲卒不下數十萬,所以入之,五倍乃足,輓車奉饟者,不在其中。南方暑溼,近夏癉熱,暴露水居,蝮蛇酿生,疾癘多作,兵未血刃而病死者什二三,雖舉越國而虜之,不足以償所亡。

臣聞道路言,閩越王弟甲弒而殺之,甲以誅死,其民未有所屬。陛下若欲來內,處之中國,使重臣臨存,施德垂賞以招致之,此必攜幼扶老以歸聖德。若陛下無所用之,則繼其絕世,存其亡國,建其王侯,以為畜越,此必委質為藩臣,世共貢職。陛下以方寸之印,丈二之組,填撫方外,不勞一卒,不頓一戟,而威德並行。今以兵入其地,此必震恐,以有司為欲屠滅之也,必雉兔逃入山林險阻。背而去之,則復相群聚;留而守之,歷歲經年,則士卒罷倦,食糧乏絕,男子不得耕稼種樹,婦人不得紡績織紝,丁壯從軍,老弱轉餉,居者無食,行者無糧。民苦兵事,亡逃者必眾,隨而誅之,不可勝盡,盜賊必起。

臣聞長老言,秦之時嘗使尉屠睢擊越,又使監祿鑿渠通道。越人逃入深山林叢,不可得攻。留軍屯守空地,曠日

持久,士卒勞倦,越乃出擊之。秦兵大破,乃發適戍以備之。當此之時,外內騷動,百姓靡敝,行者不還,往者果反,皆不聊生,亡逃相從,群為盜賊,於是山東之難始興。此老子所謂「師之所處,荊棘生之」者也。兵者凶事,一方有急,四面皆從。臣恐變故之生,姦邪之作,由此始也。《周易》曰:「高宗伐鬼方,三年而克之。」鬼方,小蠻夷;高宗,殷之盛天子也。以盛天子伐小蠻夷,三年而後克,言用兵之不可不重也。

臣聞天子之兵有征而無戰,言莫敢挍也。如使越人蒙死徼幸以逆執事之顏行,廝輿之卒有一不備而歸者,雖得越王之首,臣猶竊為大漢羞之。陛下以四海為境,九州為家,八蔬為囿,江海為池,生民之屬皆為臣妾。人徒之眾足以奉千官之共,租稅之收足以給乘輿之御。玩心神明,秉執聖道,負黼依,馮玉几,南面而聽斷,號令天下,四海之內莫不嚮應。陛下垂德惠以覆露之,使元元之民安生樂業,則澤被萬世,傳之子孫,施之無窮。天下之安猶泰山而四維之也,夷狄之地何足以為一日之閒,而煩汗馬之勞乎!《詩》云「王猶允塞,徐方既來」,言王道甚大,而遠方懷之也。臣聞之,農夫勞而君子養焉,愚者言而智者擇焉。臣安幸得為陛下守藩,以身為鄣蔽,人臣之任也。邊境有警,愛身之死而不畢其愚,非忠臣也。臣安竊恐將吏之以十萬之師為一使之任也!

是時,漢兵遂出,踰領,適會閩越王弟餘善殺王以降。漢兵罷。上嘉淮南之意,美將卒之功,乃令嚴助諭意風指於南越。南越王頓首曰:「天子乃幸興兵誅閩越,死無以報!」即遣太子隨助入侍。

助還,又諭淮南曰:「皇帝問淮南王:使中大夫玉上書言事,聞之。朕奉先帝之休德,夙興夜昧,明不能燭,重以不德,是以比年凶菑害眾。夫以眇眇之身,託于王侯之上,內有飢寒之民,南夷相攘,使邊騷然不安,朕甚懼焉。今王深惟重慮,明太平以弼朕失,稱三代至盛,際天接地,人跡所及,咸盡賓服,藐然甚慚。嘉王之意,靡有所終,使中大夫助諭朕意,告王越事。」

助諭意曰:「今者大王以發屯臨越事上書,陛下故遣臣助告王其事。王居遠,事薄遽,不與王同其計。朝有闕政,遺王之憂,陛下甚恨之。夫兵固凶器,明主之所重出也,然自五帝三王禁暴止亂,非兵,未之聞也。漢為天下宗,操殺生之柄,以制海內之命,危者望安,亂者卬治。今閩越王狼戾不仁,殺其骨肉,離其親戚,所為甚多不義,又數舉兵侵陵百越,并兼鄰國,以為暴彊,陰計奇策,入燔尋陽樓船,欲招會稽之地,以踐句踐之跡。今者,邊又言閩王率兩國擊南越。陛下為萬民安危久遠之計,使人諭告之曰:『天下安寧,各繼世撫民,禁毋敢相并。』有司疑其以虎狼之心,貪據百越之利,或於逆順,不奉明詔,則會稽、豫章必有長患。且天子誅而不伐,焉有勞百姓苦士卒乎?故遣兩將屯於境上,震威武,揚聲鄉。屯曾未會,天誘其衷,閩王隕命,輒遣使者罷屯,毋後農時。南越王甚嘉被惠澤,蒙休德,願革心易行,身從使者入謝。有狗馬之病,不能勝服,故遣太子嬰齊入侍;病有瘳,願伏北闕,望大廷,以報盛德。閩王以八月舉兵於冶南,士卒罷倦,三王之眾相與攻之,因其弱弟餘善以成其謀。至今國空虛,遣使者上符節,請所立,不敢自立,以待天子之明詔。此一舉,不挫一兵之鋒,不用一卒之死,而閩王伏辜,南越被澤,威震暴王,義存危國,此則陛下深計遠慮之所出也。事效見前,故使臣助來諭王意。」

於是王謝曰:「雖湯伐桀,文王伐崇,誠不過此。臣安妄以愚意狂言,陛下不忍加誅,使使者臨詔臣安以所不聞,臣不勝厚幸!」助由是與淮南王相結而還。上大說。

助侍燕從容,上問助居鄉里時,助對曰:「家貧,為友婿富人所辱。」上問所欲,對願為會稽太守。於是拜為會稽太守。數年,不聞問。賜書曰:「制詔會稽太守:君厭承明之廬,勞侍從之事,懷故土,出為郡吏。會稽東接於海,南近諸越,北枕大江。間者,闊焉久不聞問,具以春秋對,毋以蘇秦從橫。」助恐,上書謝稱:「春秋天王出居于鄭,不能事母,故絕之。臣事君,猶子事父母也,臣助當伏誅。陛下不忍加誅,願奉三年計最。」詔許,因留侍中。有奇異,輒使為文,及作賦頌數十篇。

後淮南王來朝,厚賂遺助,交私論議。及淮南王反,事與助相連,上薄其罪,欲勿誅。廷尉張湯爭,以為助出入禁門,腹心之臣,而外與諸侯交私如此,不誅,後不可治。助竟棄市。

朱買臣字翁子,吳人也。家貧,好讀書,不治產業,常艾薪樵,賣以給食,擔束薪,行且誦書。其妻亦負戴相隨,數止買臣毋歌嘔道中。買臣愈益疾歌,妻羞之,求去。買臣笑曰:「我年五十當富貴,今已四十餘矣。女苦日久,待我富貴報女功。」妻恚怒曰:「如公等,終餓死溝中耳,何能富貴?」買臣不能留,即聽去。其後,買臣獨行歌道中,負薪墓間。故妻與夫家俱上冢,見買臣饑寒,呼飯飲之。

後數歲,買臣隨上計吏為卒,將重車至長安,詣闕上書,書久不報。待詔公車,糧用乏,上計吏卒更乞饨之。會邑子嚴助貴幸,薦買臣。召見,說春秋,言楚詞,帝甚說之,拜買臣為中大夫,與嚴助俱侍中。是時方築朔方,公孫弘諫,以為罷敝中國。上使買臣難詘弘,語在弘傳。後買臣坐事免,久之,召待詔。

是時,東越數反覆,買臣因言:「故東越王居保泉山,一人守險,千人不得上。今聞東越王更徙處南行,去泉山五百里,居大澤中。今發兵浮海,直指泉山,陳舟列兵,席卷南行,可破滅也。」上拜買臣會稽太守。上謂買臣曰:「富貴不歸故鄉,如衣繡夜行,今子何如?」買臣頓首辭謝。詔買臣到郡,治樓船,備糧食、水戰具,須詔書到,軍與俱進。

初,買臣免,待詔,常從會稽守邸者寄居飯食。拜為太守,買臣衣故衣,懷其印綬,步歸郡邸。直上計時,會稽吏方相與群飲,不視買臣。買臣入室中,守邸與共食,食且飽,少見其綬。守邸怪之,前引其綬,視其印,會稽太守章也。守邸驚,出語上計掾吏。皆醉,大呼曰:「妄誕耳!」守邸曰:「試來視之。」其故人素輕買臣者入視之,還走,疾呼曰:「實然!」坐中驚駭,白守丞,相推排陳列中庭拜謁。買臣徐出戶。有頃,長安廄吏乘駟馬車來迎,買臣遂乘傳去。會稽聞太守且至,發民除道,縣吏並送迎,車百餘乘。入吳界,見其故妻、妻夫治道。買臣駐車,呼令後車載其夫妻,到太守舍,置園中,給食之。居一月,妻自經死,買臣乞其夫錢,令葬。悉召見故人與飲食諸嘗有恩者,皆報復焉。

居歲餘,買臣受詔將兵,與橫海將軍韓說等俱擊破東越,有功。徵入為主爵都尉,列於九卿。

數年,坐法免官,復為丞相長史。張湯為御史大夫。始買臣與嚴助俱侍中,貴用事,湯尚為小吏,趨走買臣等前。後湯以廷尉治淮南獄,排陷嚴助,買臣怨湯。及買臣為長史,湯數行丞相事,知買臣素貴,故陵折之。買臣見湯,坐床上弗為禮。買臣深怨,常欲死之。後遂告湯陰事,湯自殺,上亦誅買臣。買臣子山拊官至郡守,右扶風。

吾丘壽王字子贛,趙人也。年少,以善格五召待詔。詔使從中大夫董仲舒受春秋,高材通明。遷侍中中郎,坐法免。上書謝罪,願養馬黃門,上不許。後願守塞扞寇難,復不許。久之,上疏願擊匈奴,詔問狀,壽王對良善,復召為郎。

稍遷,會東郡盜賊起,拜為東郡都尉。上以壽王為都尉,不復置太守。是時,軍旅數發,年歲不熟,多盜賊。詔賜壽王璽書曰:「子在朕前之時,知略輻湊,以為天下少雙,海內寡二。及至連十餘城之守,任四千石之重,職事並廢,盜賊從橫,甚不稱在前時,何也?」壽王謝罪,因言其狀。

後徵入為光祿大夫侍中。丞相公孫弘奏言「民不得挾弓弩。十賊缙弩,百吏不敢前,盜賊不輒伏辜,免脫者眾,害寡而利多,此盜賊所以蕃也。禁民不得挾弓弩,則盜賊執短兵,短兵接則眾者勝。以眾吏捕寡賊,其勢必得。盜賊有害無利,則莫犯法,刑錯之道也。臣愚以為禁民毋得挾弓弩便。」上下其議。壽王對曰:

臣聞古者作五兵,非以相害,以禁暴討邪也。安居則以制猛獸而備非常,有事則以設守衛而施行陣。及至周室衰微,上無明王,諸侯力政,彊侵弱,眾暴寡,海內抏敝,是以巧詐並生。知者陷愚,勇者威怯,苟以得勝為務,不顧義理。故機變械飾,所以相賊害之具不可勝數。於是秦兼天下,廢王道,立私議,滅詩書而首法令,去仁恩而任刑戮,墮名城,殺豪桀,銷甲兵,折鋒刃。其後,民以耰鉏箠梃相撻擊,犯法滋眾,盜賊不勝,至於赭衣塞路,群盜滿山,卒以亂亡。故聖王務教化而省禁防,知其不足恃也。

今陛下昭明德,建太平,舉俊材,興學官,三公有司或由窮巷,起白屋,裂地而封,宇內日化,方外鄉風,然而盜賊猶有者,郡國二千石之罪,非挾弓弩之過也。禮曰男子生,桑弧蓬矢以舉之,明示有事也。孔子曰:「吾何執?執射乎?」大射之禮,自天子降及庶人,三代之道也。《詩》云「大侯既抗,弓矢斯張,射夫既同,獻爾發功」,言貴中也。愚聞聖王合射以明教矣,未聞弓矢之為禁也。且所為禁者,為盜賊之以攻奪也。攻奪之罪死,然而不止者,大姦之於重誅固不避也。臣恐邪人挾之而吏不能止,良民以自備而抵法禁,是擅賊威而奪民救也。竊以為無益於禁姦,而廢先王之典,使學者不得習行其禮,大不便。

書奏,上以難丞相弘。弘詘服焉。

及汾陰得寶鼎,武帝嘉之,薦見宗廟,臧於甘泉宮。群臣皆上壽賀曰:「陛下得周鼎。」壽王獨曰非周鼎。上聞之,召而問之,曰:「今朕得周鼎,群臣皆以為然,壽王獨以為非,何也?有說則可,無說則死。」壽王對曰:「臣安敢無說!臣聞周德始乎后稷,長於公劉,大於大王,成於文武,顯於周公。德澤上昭,天下漏泉,無所不通。上天報應,鼎為周出,故名曰周鼎。今漢自高祖繼周,亦昭德顯行,布恩施惠,六合和同。至於陛下,恢廓祖業,功德愈盛,天瑞並至,珍祥畢見。昔秦始皇親出鼎於彭城而不能得,天祚有德而寶鼎自出,此天之所以與漢,乃漢寶,非周寶。」上曰:「善。」群臣皆稱萬歲。是日,賜壽王黃金十斤。後坐事誅。

主父偃,齊國臨菑人也。學長短從橫術,晚乃學易、春秋、百家之言。游齊諸子間,諸儒生相與排儐,不客於齊。家貧,假貣無所得,北游燕、趙、中山,皆莫能厚,客甚困。以諸侯莫足游者,元光元年,乃西入關見衛將軍。衛將軍數言上,上不省。資用乏,留久,諸侯賓客多厭之,乃上書闕下。朝奏,暮召入見。所言九事,其八事為律令,一事諫伐匈奴,曰:

臣聞明主不惡切諫以博觀,忠臣不避重誅以直諫,是故事無遺策而功流萬世。今臣不敢隱忠避死,以效愚計,願陛下幸赦而少察之。

司馬法曰:「國雖大,好戰必亡;天下雖平,忘戰必危。」天下既平,天子大愷,春蒐秋獮,諸侯春振旅,秋治兵,所以不忘戰也。且怒者逆德也,兵者凶器也,爭者末節也。古之人君一怒必伏尸流血,故聖王重行之。夫務戰勝,窮武事,未有不悔者也。

昔秦皇帝任戰勝之威,蠶食天下,并吞戰國,海內為一,功齊三代。務勝不休,欲攻匈奴,李斯諫曰:「不可。夫匈奴無城郭之居,委積之守,遷徙鳥舉,難得而制。輕兵深入,糧食必絕;運糧以行,重不及事。得其地,不足以為利;得其民,不可調而守也。勝必棄之,非民父母。靡敝中國,甘心匈奴,非完計也。」秦皇帝不聽,遂使蒙恬將兵而攻胡,卻地千里,以河為境。地固澤鹵,不生五穀,然後發天下丁男以守北河。暴兵露師十有餘年,死者不可勝數,終不能踰河而北。是豈人眾之不足,兵革之不備哉?其勢不可也。又使天下飛芻輓粟,起於黃、腄、琅邪負海之郡,轉輸北河,率三十鍾而致一石。男子疾耕不足於糧餉,女子紡績不足於帷幕。百姓靡敝,孤寡老弱不能相養,道死者相望,蓋天下始叛也。

及至高皇帝定天下,略地於邊,聞匈奴聚代谷之外而欲擊之。御史成諫曰:「不可。夫匈奴,獸聚而鳥散,從之如搏景,今以陛下盛德攻匈奴,臣竊危之。」高帝不聽,遂至代谷,果有平城之圍。高帝悔之,乃使劉敬往結和親,然後天下亡干戈之事。

故兵法曰:「興師十萬,日費千金。」秦常積眾數十萬人,雖有覆軍殺將,係虜單于,適足以結怨深讎,不足以償天下之費。夫匈奴行盜侵敺,所以為業,天性固然。上自虞夏殷周,固不程督,禽獸畜之,不比為人。夫不上觀虞夏殷周之統,而下循近世之失,此臣之所以大恐,百姓所疾苦也。且夫兵久則變生,事苦則慮易。使邊境之民靡敝愁苦,將吏相疑而外市,故尉佗、章邯得成其私,而秦政不行,權分二子,此得失之效也。故周書曰:「安危在出令,存亡在所用。」願陛下孰計之而加察焉。

是時,徐樂、嚴安亦俱上書言世務。書奏,上召見三人,謂曰:「公皆安在?何相見之晚也!」乃拜偃、樂、安皆為郎中。偃數上疏言事,遷謁者,中郎,中大夫。歲中四遷。

偃說上曰:「古者諸侯地不過百里,彊弱之形易制。今諸侯或連城數十,地方千里,緩則驕奢易為淫亂,急則阻其彊而合從以逆京師。今以法割削,則逆節萌起,前日朝錯是也。今諸侯子弟或十數,而適嗣代立,餘雖骨肉,無尺地之封,則仁孝之道不宣。願陛下令諸侯得推恩分子弟,以地侯之。彼人人喜得所願,上以德施,實分其國,必稍自銷弱矣。」於是上從其計。又說上曰:「

茂陵初立,天下豪桀兼并之家,亂眾民,皆可徙茂陵,內實京師,外銷姦猾,此所謂不誅而害除。」上又從之。

尊立衛皇后及發燕王定國陰事,偃有功焉。大臣皆畏其口,賂遺累千金。或說偃曰:「大橫!」偃曰:「臣結髮游學四十餘年,身不得遂,親不以為子,昆弟不收,賓客棄我,我阨日久矣。丈夫生不五鼎食,死則五鼎亨耳!吾日暮,故倒行逆施之。」

偃盛言朔方地肥饒,外阻河,蒙恬築城以逐匈奴,內省轉輸戍漕,廣中國,滅胡之本也。上覽其說,下公卿議,皆言不便。公孫弘曰:「秦時嘗發三十萬眾築北河,終不可就,已而棄之。」朱買臣難詘弘,遂置朔方,本偃計也。

元朔中,偃言齊王內有淫失之行,上拜偃為齊相。至齊,遍召昆弟賓客,散五百金予之,數曰:「始吾貧時,昆弟不我衣食,賓客不我內門,今吾相齊,諸君迎我或千里。吾與諸君絕矣,毋復入偃之門!」乃使人以王與姊姦事動王。王以為終不得脫,恐效燕王論死,乃自殺。

偃始為布衣時,嘗游燕、趙,及其貴,發燕事。趙王恐其為國患,欲上書言其陰事,為居中,不敢發。及其為齊相,出關,即使人上書,告偃受諸侯金,以故諸侯子多以得封者。及齊王以自殺聞,上大怒,以為偃劫其王令自殺,乃徵下吏治。偃服受諸侯之金,實不劫齊王令自殺。上欲勿誅,公孫弘爭曰:「齊王自殺無後,國除為郡,入漢,偃本首惡,非誅偃無以謝天下。」乃遂族偃。

偃方貴幸時,客以千數,及族死,無一人視,獨孔車收葬焉。上聞之,以車為長者。

徐樂,燕郡無終人也。上書曰:

臣聞天下之患,在於土崩,不在瓦解,古今一也。

何謂土崩?秦之末世是也。陳涉無千乘之尊,尺土之地,身非王公大人名族之後,鄉曲之譽,非有孔、曾、墨子之賢,陶朱、猗頓之富也。然起窮巷,奮棘矜,偏袒大呼,天下從風,此其故何也?由民困而主不恤,下怨而上不知,俗已亂而政不修,此三者陳涉之所以為資也。此之謂土崩。故曰天下之患在乎土崩。

何謂瓦解?吳、楚、齊、趙之兵是也。七國謀為大逆,號皆稱萬乘之君,帶甲數十萬,威足以嚴其境內,財足以勸其士民,然不能西攘尺寸之地,而身為禽於中原者,此其故何也?非權輕於匹夫而兵弱於陳涉也,當是之時先帝之德未衰,而安土樂俗之民眾,故諸侯無竟外之助。此之謂瓦解。故曰天下之患不在瓦解。

由此觀之,天下誠有土崩之勢,雖布衣窮處之士或首難而危海內,陳涉是也,況三晉之君或存乎?天下雖未治也,誠能無土崩之勢,雖有彊國勁兵,不得還踵而身為禽,吳楚是也,況群臣百姓,能為亂乎?此二體者,安危之明要,賢主之所留意而深察也。

間者,關東五穀數不登,年歲未復,民多窮困,重之以邊境之事,推數循理而觀之,民宜有不安其處者矣。不安故易動,易動者,土崩之勢也。故賢主獨觀萬化之原,明於安危之機,修之廟堂之上,而銷未形之患也。其要,期使天下無土崩之勢而已矣。故雖有彊國勁兵,陛下逐走獸,射飛鳥,弘游燕之囿,淫從恣之觀,極馳騁之樂,自若。金石絲竹之聲不絕於耳,帷幄之私俳優朱儒之笑不乏於前,而天下無宿憂。名何必夏、子,俗何必成、康!雖然,臣竊以為陛下天然之質,寬仁之資,而誠以天下為務,則禹、湯之名不難侔,而成、康之俗未必不復興也。此二體者立,然後處尊安之實,揚廣譽於當世,親天下而服四夷,餘恩遺德為數世隆,南面背依攝袂而揖王公,此陛下之所服也。臣聞圖王不成,其敝足以安。安則陛下何求而不得,何威而不成,奚征而不服哉?

嚴安者,臨菑人也。以故丞相史上書,曰:

臣聞鄒衍曰:「政教文質者,所以云救也,當時則用,過則舍之,有易則易也,故守一而不變者,未睹治之至也。」今天下人民用財侈靡,車馬衣裘宮室皆競修飾,調五聲使有節族,雜五色使有文章,重五味方丈於前,以觀欲天下。彼民之情,見美則願之,是教民以侈也。侈而無節,則不可贍,民離本而徼末矣。末不可徒得,故搢紳者不憚為詐,帶劍者夸殺人以矯奪,而世不知媿,故姦軌浸長。夫佳麗珍怪固順於耳目,故養失而泰,樂失而淫,禮失而采,教失而偽。偽、采、淫、泰,非所以範民之道也。是以天下人民逐利無已,犯法者眾。臣願為民制度以防其淫,使貧富不相燿以和其心。心既和平,其性恬安。恬安不營,則盜賊銷;盜賊銷,則刑罰少;刑罰少,則陰陽和,四時正,風雨時,草木暢茂,五穀蕃孰,六畜遂字,民不夭厲,和之至也。

臣聞周有天下,其治三百餘歲,成康其隆也,刑錯四十餘年而不用。及其衰,亦三百餘年,故五伯更起。伯者,常佐天子興利除害,誅暴禁邪,匡正海內,以尊天子。五伯既沒,賢聖莫續,天子孤弱,號令不行。諸侯恣行,彊陵弱,眾暴寡。田常篡齊,六卿分晉,並為戰國,此民之始苦也。於是彊國務攻,弱國修守,合從連衡,馳車轂擊,介冑生蟣蝨,民無所告愬。

及至秦王,蠶食天下,并吞戰國,稱號皇帝,一海內之政,壞諸侯之城。銷其兵,鑄以為鍾虡,示不復用。元元黎民得免於戰國,逢明天子,人人自以為更生。鄉使秦緩刑罰,薄賦斂,省繇役,貴仁義,賤權利,上篤厚,下佞巧,變風易俗,化於海內,則世世必安矣。秦不行是風,循其故俗,為知巧權利者進,篤厚忠正者退,法嚴令苛,諂諛者眾,曰聞其美,章廣心逸。欲威海外,使蒙恬將兵以北攻彊胡,辟地進境,戍於北河,飛芻輓粟以隨其後。又使尉屠睢將樓船之士攻越,使監祿鑿渠運糧,深入越地,越人遁逃。曠日持久,糧食乏絕,越人擊之,秦兵大敗。秦乃使尉佗將卒以戍越。當是時,秦禍北構於胡,南挂於越,宿兵於無用之地,進而不得退。行十餘年,丁男被甲,丁女轉輸,苦不聊生,自經於道樹,死者相望。及秦皇帝崩,天下大畔。陳勝、吳廣舉陳,武臣、張耳舉趙,項梁舉吳,田儋舉齊,景駒舉郢,周市舉魏,韓廣舉燕,窮山通谷,豪士並起,不可勝載也。然本皆非公侯之後,非長官之吏,無尺寸之勢,起閭巷,杖棘矜,應時而動,不謀而俱起,不約而同會,壤長地進,至乎伯王,時教使然也。秦貴為天子,富有天下,滅世絕祀,窮兵之禍也。故周失之弱,秦失之彊,不變之患也。

今徇南夷,朝夜郎,降羌僰,略薉州,建城邑,深入匈奴,燔其龍城,議者美之。此人臣之利,非天下之長策也。今中國無狗吠之警,而外累於遠方之備,靡敝國家,非所以子民也。行無窮之欲,甘心快意,結怨於匈奴,非所以安邊也。禍挐而不解,兵休而復起,近者愁苦,遠者驚駭,非所以持久也。今天下鍛甲摩劍,矯箭控弦,轉輸軍糧,未見休時,此天下所共憂也。夫兵久而變起,事煩而慮生。今外郡之地或幾千里,列城數十,形束壤制,帶脅諸侯,非宗室之利也。上觀齊晉所以亡,公室卑削,六卿大盛也;下覽秦之所以滅,刑嚴文刻,欲大無窮也。今郡守之權非特六卿之重也,地幾千里非特閭巷之資也,甲兵器械非特棘矜之用也,以逢萬世之變,則不可勝諱也。

後以安為騎馬令。

終軍字子雲,濟南人也。少好學,以辯博能屬文聞於郡中。年十八,選為博士弟子。至府受遣,太守聞其有異材,召見軍,甚奇之,與交結。軍揖太守而去,至長安上書言事。武帝異其文,拜軍為謁者給事中。

從上幸雍祠五畤,獲白麟,一角而五蹄。時又得奇木,其枝旁出,輒復合於木上。上異此二物,博謀群臣。軍上對曰:

臣聞詩頌君德,樂舞后功,異經而同指,明盛德之所隆也。南越竄屏葭葦,與鳥魚群,正朔不及其俗。有司臨境,而東甌內附,閩王伏辜,南越賴救。北胡隨畜薦居,禽獸行,虎狼心,上古未能攝。大將軍秉鉞,單于奔幕;票騎抗旌,昆邪右衽。是澤南洽而威北暢也。若罰不阿近,舉不遺遠,設官俟賢,縣賞待功,能者進以保祿,罷者退而勞力,刑於宇內矣。履眾美而不足,懷聖明而不專,建三宮之文質,章厥職之所宜,封禪之君無聞焉。

夫人命初定,萬事草創,及臻六合同風,九州共貫,必待明聖潤色,祖業傳於無窮。故周至成王,然後制定,而休徵之應見。陛下盛日月之光,垂聖思於勒成,專神明之敬,奉燔瘞於郊宮,獻享之精交神,積和之氣塞明,而異獸來獲,宜矣。昔武王中流未濟,白魚入於王舟,俯取以燎,群公咸曰「休哉!」今郊祀未見於神祇,而獲獸以饋,此天之所以示饗,而上通之符合也。宜因昭時令日,改定告元,苴以白茅於江淮,發嘉號于營丘,以應緝熙,使著事者有紀焉。

蓋六鶂退飛,逆也;白魚登舟,順也。夫明闇之徵,上亂飛鳥,下動淵魚,各以類推。今野獸并角,明同本也;眾支內附,示無外也。若此之應,殆將有解編髮,削左衽,襲冠帶,要衣裳,而蒙化者焉。斯拱而俟之耳!

對奏,上甚異之,由是改元為元狩。後數月,越地及匈奴名王有率眾來降者,時皆以軍言為中。

元鼎中,博士徐偃使行風俗。偃矯制,使膠東、魯國鼓鑄鹽鐵。還,奏事,徙為太常丞。御史大夫張湯劾偃矯制大害,法至死。偃以為春秋之義,大夫出疆,有可以安社稷,存萬民,顓之可也。湯以致其法,不能詘其義。有詔下軍問狀,軍詰偃曰:「古者諸侯國異俗分,百里不通,時有聘會之事,安危之勢,呼吸成變,故有不受辭造命顓己之宜;今天下為一,萬里同風,故春秋『王者無外』。偃巡封域之中,稱以出疆何也?且鹽鐵,郡有餘臧,正二國廢,國家不足以為利害,而以安社稷存萬民為辭,何也?」又詰偃:「膠東南近琅邪,北接北海,魯國西枕泰山,東有東海,受其鹽鐵。偃度四郡口數田地,率其用器食鹽,不足以并給二郡邪?將勢宜有餘,而吏不能也?何以言之?偃矯制而鼓鑄者,欲及春耕種贍民器也。今魯國之鼓,當先具其備,至秋乃能舉火。此言與實反者非?偃已前三奏,無詔,不惟所為不許,而直矯作威福,以從民望,干名采譽,此明聖所必加誅也。『枉尺直尋』,孟子稱其不可;今所犯罪重,所就者小,偃自予必死而為之邪?將幸誅不加,欲以采名也?」偃窮詘,服罪當死。軍奏「偃矯制顓行,非奉使體,請下御史徵偃即罪。」奏可。上善其詰,有詔示御史大夫。

初,軍從濟南當詣博士,步入關,關吏予軍繻。軍問:「

以此何為?」吏曰;「為復傳,還當以合符。」軍曰:「大丈夫西游,終不復傳還。」棄繻而去。軍為謁者,使行郡國,建節東出關,關吏識之,曰:「此使者乃前棄繻生也。」軍行郡國,所見便宜以聞。還奏事,上甚說。

當發使使匈奴,軍自請曰:「軍無橫草之功,得列宿衛,食祿五年。邊境時有風塵之警,臣宜被堅執銳,當矢石,啟前行。駑下不習金革之事,今聞將遣匈奴使者,臣願盡精厲氣,奉佐明使,畫吉凶於單于之前。臣年少材下,孤於外官,不足以亢一方之任,竊不勝憤懣。」詔問畫吉凶之狀,上奇軍對,擢為諫大夫。

南越與漢和親,乃遣軍使南越,說其王,欲令入朝,比內諸侯。軍自請:「願受長纓,必羈南越王而致之闕下。」軍遂往說越王,越王聽許,請舉國內屬。天子大說,賜南越大臣印綬,壹用漢法,以新改其俗,令使者留填撫之。越相呂嘉不欲內屬,發兵攻殺其王,及漢使者皆死。語在南越傳。軍死時年二十餘,故世謂之「終童」。

王褒字子淵,蜀人也。宣帝時修武帝故事,講論六藝群書,博盡奇異之好,徵能為楚辭九江被公,召見誦讀,益召高材劉向、張子僑、華龍、柳褒等待詔金馬門。神爵、五鳳之間,天下殷當,數有嘉應。上頗作歌詩,欲興協律之事,丞相魏相奏言知音善鼓雅琴者渤海趙定、梁國龔德,皆召見待詔。於是益州刺史王襄欲宣風化於眾庶,聞王褒有俊材,請與相見,使褒作中和、樂職、宣布詩,選好事者令依鹿鳴之聲習而歌之。時氾鄉侯何武為僮子,選在歌中。久之,武等學長安,歌太學下,轉而上聞。宣帝召見武等觀之,皆賜帛,謂曰:「此盛德之事,吾何足以當之!」

褒既為刺史作頌,又作其傳,益州刺史因奏褒有軼材。上乃徵褒。既至,詔褎為聖主得賢臣頌其意。褒對曰:

夫荷旃被毳者,難與道純綿之麗密;羹黎唅糗者,不足與論太牢之滋味。今臣辟在西蜀,生於窮巷之中,長於蓬茨之下,無有游觀廣覽之知,顧有至愚極陋之累,不足以塞厚望,應明指。雖然,敢不略陳愚而抒情素!

記曰:共惟春秋五始之要,在乎審己正統而已。夫賢者,國家之器用也。所任賢,則趨舍省而功施普;器用利,則用力少而就效眾。故工人之用鈍器也,勞筋苦骨,終日矻矻。及至巧冶鑄干將之樸,清水焠其鋒,越砥斂其咢,水斷蛟龍,陸剸犀革,忽若彗氾畫塗。如此,則使離婁督繩,公輸削墨,雖崇臺五增,延袤百丈,而不溷者,工用相得也。庸人之御駑馬,亦傷吻敝策而不進於行,匈喘膚汗,人極馬倦。及至駕齧錾,驂乘旦,王良執靶,韓哀附輿,縱馳騁騖,忽如景靡,過都越國,蹶如歷塊;追奔電,逐遺風,周流八極,萬里壹息。何其遼哉?人馬相得也。故服絺綌之涼者,不苦盛暑之鬱燠;襲貂狐之飕者,不憂至寒之悽愴。何則?有其具者易其備。賢人君子,亦聖主之所以易海內也。是以嘔喻受之,開寬裕之路,以延天下英俊也。夫竭知附賢者,必建仁策;索人求士者,必樹伯跡。昔周公躬吐捉之勞,故有圉空之隆;齊桓設庭燎之禮,故有匡合之功。由此觀之,君人者勤於求賢而逸於得人。

人臣亦然。昔賢者之未遭遇也,圖事揆策則君不用其謀,陳見悃誠則上不然其信,進仕不得施效,斥逐又非其愆。是故伊尹勤於鼎俎,太公困於鼓刀,百里自鬻,甯子飯牛,離此患也。及其遇明君遭聖主也,運籌合上意,諫諍即見聽,進退得關其忠,任職得行其術,去卑辱奧渫而升本朝,離疏釋蹻而享膏粱,剖符錫壤而光祖考,傳之子孫,以資說士。故世必有聖知之君,而後有賢明之臣。故虎嘯而冽風,龍興而致雲,蟋蟀俟秋吟,蜉蝤出以陰。《易》曰:「飛龍在天,利見大人。」《詩》曰:「思皇多士,生此王國。」故世平主聖,俊艾將自至,若堯、舜、禹、湯、文、武之君,獲稷、契、皋陶、伊尹、呂望,明明在朝,穆穆列布,聚精會神,相得益章。雖伯牙操遞鍾,逢門子彎烏號,猶未足以喻其意也。

故聖主必待賢臣而弘功業,俊士亦俟明主以顯其德。上下俱欲,驩然交欣,千載壹合,論說無疑,翼乎如鴻毛過順風,沛乎如巨魚縱大壑。其得意若此,則胡禁不止,曷令不行?化溢四表,橫被無窮,遐夷貢獻,萬祥畢溱。是以聖王不遍窺望而視已明,不單頃耳而聽已聰;恩從祥風翱,德與和氣游,太平之責塞,優游之望得;遵遊自然之勢,恬淡無為之場,休徵自至,壽考無疆,雍容垂拱,永永萬年,何必偃卬詘信若彭祖,呴噓呼吸如僑、松,眇然絕俗離世哉!《詩》云「濟濟多士,文王以寧」,蓋信乎其以寧也!

是時,上頗好神僊,故褒對及之。

上令褒與張子僑等並待詔,數從褒等放獵,所幸宮館,輒為歌頌,第其高下,以差賜帛。議者多以為淫靡不急,上曰:「『不有博弈者乎,為之猶賢乎已!』辭賦大者與古詩同義,小者辯麗可喜。辟如女工有綺縠,音樂有鄭衛,今世俗猶皆以此虞說耳目,辭賦比之,尚有仁義風諭,鳥獸草木多聞之觀,賢於倡優博弈遠矣。」頃之,擢褒為諫大夫。

其後太子體不安,苦忽忽善忘,不樂。詔使褒等皆之太子宮虞侍太子,朝夕誦讀奇文及所自造作。疾平復,乃歸。太子喜褒所為甘泉及洞簫頌,令後宮貴人左右皆誦讀之。

後方士言益州有金馬碧雞之寶,可祭祀致也,宣帝使褒往祀焉。褒於道病死,上閔惜之。

賈捐之字君房,賈誼之曾孫也。元帝初即位,上疏言得失,召待詔金馬門。

初,武帝征南越,元封元年立儋耳、珠崖郡,皆在南方海中洲居,廣袤可千里,合十六縣,戶二萬三千餘。其民暴惡,自以阻絕,數犯吏禁,吏亦酷之,率數年壹反,殺吏,漢輒發兵擊定之。自初為郡至昭帝始元元年,二十餘年間,凡六反叛。至其五年,罷儋耳郡并屬珠崖。至宣帝神爵三年,珠崖三縣復反。反後七年,甘露元年,九縣反,輒發兵擊定之。元帝初元元年,珠崖又反,發兵擊之。諸縣更叛,連年不定。上與有司議大發軍,捐之建議,以為不當擊。上使侍中駙馬都尉樂昌侯王商詰問捐之曰:「珠崖內屬為郡久矣,今背畔逆節,而云不當擊,長蠻夷之亂,虧先帝功德,經義何以處之?」捐之對曰:

臣幸得遭明盛之朝,蒙危言之策,無忌諱之患,敢昧死竭卷卷。

臣聞堯舜,聖之盛也,禹入聖域而不優,故孔子稱堯曰「大哉」,韶曰「盡善」,禹曰「無間」。以三聖之德,地方不過數千里,被流沙,東漸于海,朔南暨聲教,迄于四海,欲與聲教則治之,不欲與者不彊治也。故君臣歌德,含氣之物各德其宜。武丁、成王,殷、周之大仁也,然地東不過江、黃,西不過氐、羌,南不過蠻荊,北不過朔方。是以頌聲並作,視聽之類咸樂其生,越裳氏重九譯而獻,此非兵革之所能致。及其衰也,南征不還,齊桓捄其難,孔子定其文。以至乎秦,興兵遠攻,貪外虛內,務欲廣地,不慮其害。然地南不過閩越,北不過太原,而天下潰畔,禍卒在於二世之末,長城之歌至今未絕。

賴聖漢初興,為百姓請命,平定天下。至孝文皇帝,閔中國未安,偃武行文,則斷獄數百,民賦四十,丁男三年而一事。時有獻千里馬者,詔曰:「鸞旗在前,屬車在後,吉行日五十里,師行二十里,朕乘千里之馬,獨先安之?」於是還馬,與道里費,而下詔曰:「朕不受獻也,其令四方毋求來獻。」當此之時,逸游之樂絕,奇麗之賂塞,鄭衛之倡微矣。夫後官盛色則賢者隱處,佞人用事則諍臣杜口,而文帝不行,故諡為孝文,廟稱太宗。至孝武皇帝元狩六年,太倉之粟紅腐而不可食,都內之錢貫朽而不可挍。乃探平城之事,錄冒頓以來數為邊害,籍兵厲馬,因富民以攘服之。西連諸國至于安息,東過碣石以玄菟、樂浪為郡,比卻匈奴萬里,更起營塞,制南海以為八郡,則天下斷獄萬數,民賦數百,造鹽鐵酒榷之利以佐用度,猶不能足。當此之時,寇賊並起,軍旅數發,父戰死於前,子鬥傷於後,女子乘亭鄣,孤兒號於道,老母寡婦飲泣巷哭,遙設虛祭,想魂乎萬里之外。淮南王盜寫虎符,陰聘名士,關東公孫勇等詐為使者,是皆廓地泰大,征伐不休之故也。

今天下獨有關東,關東大者獨有齊楚,民眾久困,連年流離,離其城郭,相枕席於道路。人情莫親父母,莫樂夫婦,至嫁妻賣子,法不能禁,義不能止,此社稷之憂也。今陛下不忍悁悁之忿,欲驅士眾擠之大海之中,快心幽冥之地,非所以救助飢饉,保全元元也。《詩》云「蠢爾蠻荊,大邦為讎」,言聖人起則後服,中國衰則先畔,動為國家難,自古而患之久矣,何況乃復其南方萬里之蠻乎!駱越之人父子同川而浴,相習以鼻飲,與禽獸無異,本不足郡縣置也。顓顓獨居一海之中,霧露氣溼,多毒草蟲蛇水土之害,人未見虜,戰士自死。又非獨珠崖有珠犀玳瑁也,棄之不足惜,不擊不損威。其民譬猶魚鱉,何足貪也!

臣竊以往者羌軍言之,暴師曾未一年,兵出不踰千里,費四十餘萬萬,大司農錢盡,乃以少府禁錢續之。夫一隅為不善,費尚如此,況於勞師遠攻,亡士毋功乎!求之往古則不合,施之當今又不便。臣愚以為非冠帶之國,禹貢所及,春秋所治,皆可且無以為。願遂棄珠崖,專用恤關東為憂。

對奏,上以問丞相御史。御史大夫陳萬年以為當擊;丞相于定國以為「前日興兵擊之連年,護軍都尉、校尉及丞凡十一人,還者二人,卒士及轉輸死者萬人以上,費用三萬萬餘,尚未能盡降。今關東困乏,民難搖動,捐之議是。」上乃從之。遂下詔曰:「珠崖虜殺吏民,背畔為逆,今廷議者或言可擊,或言可守,或欲棄之,其指各殊。朕日夜惟思議者之言,羞威不行,則欲誅之;狐疑辟難,則守屯田;通于時變,則憂萬民。夫萬民之饑餓,與遠蠻之不討,危孰大焉?且宗廟之祭,凶年不備,況乎辟不嫌之辱哉!今關東大困,倉庫空虛,無以相贍,又以動兵,非特勞民,凶年隨之。其罷珠崖郡。民有慕義欲內屬,便處之;不欲,勿彊。」珠崖由是罷。

捐之數召見,言多納用。時中書令石顯用事,捐之數短顯,以故不得官,後稀復見。而長安令楊興新以材能得幸,與捐之相善。捐之欲得召見,謂興曰:「京兆尹缺,使我得見,言君蘭,京兆尹可立得。」興曰:「縣官嘗言興瘉薛大夫,我易助也。君房下筆,言語妙天下,使君房為尚書令,勝五鹿充宗遠甚。」捐之曰:「令我得代充宗,君蘭為京兆,京兆郡國首,尚書百官本,天下真大治,士則不隔矣。捐之前言平恩侯可為將軍,期思侯並可為諸曹,皆如言;又薦謁者滿宣,立為冀州刺史;言中謁者不宜受事,宦者不宜入宗廟,立止。相薦之信,不當如是乎!」興曰:「我復見,言君房也。」捐之復短石顯。興曰:「顯鼎貴,上信用之。今欲進,弟從我計,且與合意,即得入矣。」

捐之即與興共為薦顯奏,曰:「竊見石顯本山東名族,有禮義之家也。持正六年,未嘗有過,明習於事,敏而疾見,出公門,入私門。宜賜爵關內侯,引其兄弟以為諸曹。」又共為薦興奏,曰:「竊見長安令興,幸得以知名數召見。興事父母有曾氏之孝,事師有顏閔之材,榮名聞於四方。明詔舉茂材,列侯以為首。為長安令,吏民敬鄉,道路皆稱能。觀其下筆屬文,則董仲舒;進談動辭,則東方生;置之爭臣,則汲直;用之介冑,則冠軍侯;施之治民,則趙廣漢;抱公絕私,則尹翁歸。興兼此六人而有之,守道堅固,執義不回,臨大節而不可奪,國之良臣也,可試守京兆尹。」

石顯聞知,白之上。乃下興、捐之獄,令皇后父陽平侯禁與顯共雜治,奏「興、捐之懷詐偽,以上語相風,更相薦譽,欲得大位,漏泄省中語,岡上不道。書曰:『讒說殄行,震驚朕師。』王制:『順非而澤,不聽而誅。』請論如法。」

捐之竟坐棄市。興減死罪一等,髡鉗為城旦。成帝時,至部刺史。

贊曰:《詩》稱「戎狄是膺,荊舒是懲」,久矣其為諸夏患也。漢興,征伐胡越,於是為盛。究觀淮南、捐之、主父、嚴安之義,深切著明,故備論其語。世稱公孫弘排主父,張湯陷嚴助,石顯譖捐之,察其行跡,主父求欲鼎亨而得族,嚴、賈出入禁門招權利,死皆其所也,亦何排陷之恨哉!


\end{pinyinscope}