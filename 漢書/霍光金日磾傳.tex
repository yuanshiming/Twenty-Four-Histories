\article{霍光金日磾傳}

\begin{pinyinscope}
霍光字子孟,票騎將軍去病弟也。父中孺,河東平陽人也,以縣吏給事平陽侯家,與侍者衛少兒私通而生去病。中孺吏畢歸家,娶婦生光,因絕不相聞。久之,少兒女弟子夫得幸於武帝,立為皇后,去病以皇后姊子貴幸。既壯大,乃自知父為霍中孺,未及求問。會為票騎將軍擊匈奴,道出河東,何東太守郊迎,負弩矢先驅,至平陽傳舍,遣吏迎霍中孺。中孺趨入拜謁,將軍迎拜,因跪曰:「去病不早自知為大人遺體也。」中孺扶服叩頭,曰:「老臣得託命將軍,此天力也。」去病大為中孺買田宅奴婢而去。還,復過焉,乃將光西至長安,時年十餘歲,任光為郎,稍遷諸曹侍中。去病死後,光為奉常都尉光祿大夫,出則奉車,入侍左右,出入禁闥二十餘年,小心謹慎,未嘗有過,甚見親信。

征和二年,衛太子為江充所敗,而燕王旦、廣陵王胥皆多過失。是時上年老,寵姬鉤弋趙婕妤有男,上心欲以為嗣,命大臣輔之。察群臣唯光任大重,可屬社稷。上乃使黃門畫者畫周公負成王朝諸侯以賜光。後元二年春,上游五柞宮,病篤,光涕泣問曰:「如有不諱,誰當嗣者?」上曰:「君未諭前畫意邪?立少子,君行周公之事。」光頓首讓曰:「臣不如金日磾。」日磾亦曰:「臣外國人,不如光。」上以光為大司馬大將軍,日磾為車騎將軍,及太僕上官桀為左將軍,搜粟都尉桑弘羊為御史大夫,皆拜臥內床下,受遺詔輔少主。明日,武帝崩,太子襲尊號,是為孝昭皇帝。帝年八歲,政事壹決於光。

先是,後元年,侍中僕射莽何羅與弟重合侯通謀為逆,時光與金日磾、上官桀等共誅之,功未錄。武帝病,封璽書曰:「帝崩發書以從事。」遺詔封金日磾為秺侯,上官桀為安陽侯,光為博陸侯,皆以前捕反者功封。時衛尉王莽子男忽侍中,揚語曰:「帝病,忽常在左右,安得遺詔封三子事!群兒自相貴耳。」光聞之,切讓王莽,莽酖殺忽。

光為人沈靜詳審,長財七尺三寸,白皙,疏眉目,美須敘。每出入下殿門,止進有常處,郎僕射竊識視之,不失尺寸,其資性端正如此。初輔幼主,政自己出,天下想聞其風采。殿中嘗有怪,一夜群臣相驚,光召尚符璽郎,郎不肯授光。光欲奪之,郎按劍曰:「臣頭可得,璽不可得也!」光甚誼之。明日,詔增此郎秩二等。眾庶莫不多光。

光與左將軍桀結婚相親,光長女為桀子安妻。有女年與帝相配,桀因帝姊鄂邑蓋主內安女後宮為婕妤,數月立為皇后。父安為票騎將軍,封桑樂侯。光時休沐出,桀輒入代光決事。桀父子既尊盛,而德長公主。公主內行不修,近幸河間丁外人。桀、安欲為外人求封,幸依國家故事以列侯尚公主者,光不許。又為外人求光祿大夫,欲令得召見,又不許。長主大以是怨光。而桀、安數為外人求官爵弗能得,亦慚。自先帝時,桀已為九卿,位在光右。及父子並為將軍,有椒房中宮之重,皇后親安女,光乃其外祖,而顧專制朝事,繇是與光爭權。

燕王旦自以昭帝兄,常懷怨望。及御史大夫桑弘羊建造酒榷鹽鐵,為國興利,伐其功,欲為子弟得官,亦怨恨光。於是蓋主、上官桀、安及弘羊皆與燕王旦通謀,詐令人為燕王上書,言「光出都肄郎羽林,道上稱旧,太官先置。又引蘇武前使匈奴,拘留二十年不降,還乃為典屬國,而大將軍長史敞亡功為搜粟都尉。又擅調益莫府校尉。光專權自恣,疑有非常。臣旦願歸符璽,入宿衛,察姦臣變。」候司光出沐日奏之。桀欲從中下其事,桑弘羊當與諸大臣共執退光。書奏,帝不肯下。

明旦,光聞之,止畫室中不入。上問「大將軍安在?」左將軍桀對曰:「以燕王告其罪,故不敢入。」有詔召大將軍。光入,免冠頓首謝,上曰:「將軍冠。朕知是書詐也,將軍亡罪。」光曰:「陛下何以知之?」上曰:「將軍之廣明,都郎屬耳。調校尉以來未能十日,燕王何以得知之?且將軍為非,不須校尉。」是時帝年十四,尚書左右皆驚,而上書者果亡,捕之甚急。桀等懼,白上小事不足遂,上不聽。

後桀黨與有譖光者,上輒怒曰:「大將軍忠臣,先帝所屬以輔朕身,敢有毀者坐之。」自是桀等不敢復言,乃謀令長公主置酒請光,伏兵格殺之,因廢帝,迎立燕王為天子。事發覺,光盡誅桀、安、弘羊、外人宗族。燕王、蓋主皆自殺。光威震海內。昭帝既冠,遂委任光,訖十三年,百姓充實,四夷賓服。

元平元年,昭帝崩,亡嗣。武帝六男獨有廣陵王胥在,群臣議所立,咸持廣陵王。王本以行失道,先帝所不用。光內不自安。郎有上書言「周太王廢太伯立王季,文王舍伯邑考立武王,唯在所宜,雖廢長立少可也。廣陵王不可以承宗廟。」言合光意。光以其書視丞相敞等,擢郎為九江太守,即日承皇太后詔,遣行大鴻臚事少府樂成、宗正德、光祿大夫吉、中郎將利漢迎昌邑王賀。

賀者,武帝孫,昌邑哀王子也。既至,即位,行淫亂。光憂懣,獨以問所親故吏大司農田延年。延年曰:「將軍為國柱石,審此人不可,何不建白太后,更選賢而立之?」光曰:「

今欲如是,於古嘗有此否?」延年曰:「伊尹相殷,廢太甲以安宗廟,後世稱其忠。將軍若能行此,亦漢之伊尹也。」光乃引延年給事中,陰與車騎將軍張安世圖計,遂召丞相、御史、將軍、列侯、中二千石、大夫、博士會議未央宮。光曰:「昌邑王行昏亂,恐危社稷,如何?」群臣皆驚鄂失色,莫敢發言,但唯唯而已。田延年前,離席按劍,曰:「先帝屬將軍以幼孤,寄將軍以天下,以將軍忠賢能安劉氏也。今群下鼎沸,社稷將傾,且漢之傳諡常為孝者,以長有天下,令宗廟血食也。如令漢家絕祀,將軍雖死,何面目見先帝於地下乎?今日之議,不得旋踵。群臣後應者,臣請劍斬之。」光謝曰:「九卿責光是也。天下匈匈不安,光當受難。」於是議者皆叩頭,曰:「萬姓之命在於將軍,唯大將軍令。」

光即與群臣俱見白太后,具陳昌邑王不可以承宗廟狀。皇太后乃車駕幸未央承明殿,詔諸禁門毋內昌邑群臣。王入朝太后還,乘輦欲歸溫室,中黃門宦者各持門扇,王入,門閉,昌邑群臣不得入。王曰:「何為?」大將軍跪曰:「有皇太后詔,毋內昌邑群臣。」王曰:「徐之,何乃驚人如是!」光使盡驅出昌邑群臣,置金馬門外。車騎將軍安世將羽林騎收縛二百餘人,皆送廷尉詔獄。令故昭帝侍中中臣侍守王。光敕左右:「謹宿衛,卒有物故自裁,令我負天下,有殺主名。」王尚未自知當廢,謂左右:「我故群臣從官安得罪,而大將軍盡繫之乎。」頃之,有太后詔召王。王聞召,意恐,乃曰:「我安得罪而召我哉!」太后被珠襦,盛服坐武帳中,侍御數百人皆持兵,期門武士陛戟,陳列殿下。群臣以次上殿,召昌邑王伏前聽詔。光與群臣連名奏王,尚書令讀奏曰:

丞相臣敞、大司馬大將軍臣光、車騎將軍臣安世、度遼將軍臣明友、前將軍臣增、後將軍臣充國、御史大夫臣誼、宜春侯臣譚、當塗侯臣聖、隨桃侯臣昌樂、杜侯臣屠耆堂、太僕臣延年、太常臣昌、大司農臣延年、宗正臣德、少府臣樂成、廷尉臣光、執金吾臣延壽、大鴻臚臣賢、左馮翊臣廣明、右扶風臣德、長信少府臣嘉、典屬國臣武、京輔都尉臣廣漢、司隸校尉臣辟兵、諸吏文學光祿大夫臣遷、臣畸、臣吉、臣賜、臣管、臣勝、臣梁、臣長幸、臣夏侯勝、太中大夫臣德、臣卬昧死言皇太后陛下:臣敞等頓首死罪。大子所以永保宗廟總壹海內者,以慈孝禮誼賞罰為本。孝昭皇帝早棄天下,亡嗣,臣敞等議,禮曰「為人後者為之子也」,昌邑王宜嗣後,遣宗正、大鴻臚、光祿大夫奉節使徵昌邑王典喪。服斬縗,亡悲哀之心,廢禮誼,居道上不素食,使從官略女子載衣車,內所居傳舍。始至謁見,立為皇太子,常私買雞豚以食。受皇帝信璽、行璽大行前,就次發璽不封。從官更持節,引內昌邑從官騶宰官奴二百餘人,常與居禁闥內敖戲。自之符璽取節十六,朝暮臨,令從官更持節從。為書曰「皇帝問侍中君卿:使中御府令高昌奉黃金千斤,賜君卿取十妻。」大行在前殿,發樂府樂器,引內昌邑樂人,擊鼓歌吹作俳倡。會下還,上前殿,擊鐘磬,召內泰壹宗廟樂人輦道牟首,鼓吹歌舞,悉奏眾樂。發長安廚三太牢具祠閣室中,祀已,與從官飲啗。駕法駕,皮軒鸞旗,驅馳北宮、桂宮,弄彘鬥虎。召皇太后御小馬車,使官奴騎乘,遊戲掖庭中。與孝昭皇帝宮人蒙等淫亂,詔掖庭令敢泄言要斬。

太后曰:「止!為人臣子當悖亂如是邪!」王離席伏。尚書令復讀曰:

取諸侯王列侯二千石綬及墨綬黃綬以并佩昌邑郎官者免奴。變易節上黃旄以赤。發御府金錢刀劍玉器采繒,賞賜所與遊戲者。與從官官奴夜飲,湛沔於酒。詔太官上乘輿食如故。食監奏未釋服未可御故食,復詔太官趣具,無關食監。太官不敢具,即使從官出買雞豚,詔殿門內,以為常。獨夜設九賓溫室,延見姊夫昌邑關內侯。祖宗廟祠未舉,為璽書使使者持節,以三太牢祠昌邑哀王園廟,稱嗣子皇帝。受璽以來二十七日,使者旁午,持節詔諸官署徵發,凡千一百二十七事。文學光祿大夫夏侯勝等及侍中傅嘉數進諫以過失,使人簿責勝,縛嘉繫獄。荒淫迷惑,失帝王禮誼,亂漢制度。臣敞等數進諫,不變更,日以益甚,恐危社稷,天下不安。

臣敞等謹與博士臣霸、臣雋舍、臣德、臣虞舍、臣射、臣倉議,皆曰:「高皇帝建功業為漢太祖,孝文皇帝慈仁節儉為太宗,今陛下嗣孝昭皇帝後,行淫辟不軌。《詩》云:『籍曰未知,亦既抱子。』五辟之屬,莫大不孝。周襄王不能事母,春秋曰『天王出居于鄭』,繇不孝出之,絕之於天下也。宗廟重於君,陛下未見命高廟,不可以承天序,奉祖宗廟,子萬姓,當廢。」臣請有司御史大夫臣誼、宗正臣德、太常臣昌與太祝以一太牢具,告祠高廟。臣敞等昧死以聞。

皇太后詔曰:「可。」光令王起拜受詔,王曰:「聞天子有爭臣七人,雖無道不失天下。」光曰:「皇太后詔廢,安得天子!」乃即持其手,解脫其璽組,奉上太后,扶王下殿,出金馬門,群臣隨送。王西面拜,曰:「愚戇不任漢事。」起就乘輿副車。大將軍光送至昌邑邸,光謝曰:「王行自絕於天,臣等駑怯,不能殺身報德。臣寧負王,不敢負社稷。願王自愛,臣長不復見左右。」光涕泣而去。群臣奏言:「古者廢放之人屏於遠方,不及以政,請徙王賀漢中房陵縣。」太后詔歸賀昌邑,賜湯沐邑二千戶。昌邑群臣坐亡輔導之誼,陷王於惡,光悉誅殺二百餘人。出死,號呼巿中曰:「當斷不斷,反受其亂。」

光坐庭中,會丞相以下議定所立。廣陵王已前不用,及燕剌王反誅,其子不在議中。近親唯有衛太子孫號皇曾孫在民間,咸稱述焉。光遂復與丞相敞等上奏曰:「禮曰『人道親親故尊祖,尊祖故敬宗。』太宗亡嗣,擇支子孫賢者為嗣。孝武皇帝曾孫病已,武帝時有詔掖庭養視,至今年十八,師受詩、論語、孝經,躬行節儉,慈仁愛人,可以嗣孝昭皇帝後,奉承祖宗廟,子萬姓。臣昧死以聞。」皇太后詔曰:「可。」光遣宗正劉德至曾孫家尚冠里,洗沐賜御衣,太僕以軨獵車迎曾孫就齋宗正府,入未央宮見皇太后,封為陽武侯。已而光奉上皇帝璽綬,謁于高廟,是為孝宣皇帝。明年,下詔曰:「夫褒有德,賞元功,古今通誼也。大司馬大將軍光宿衛忠正,宣德明恩,守節秉誼,以安宗廟。其以河北、東武陽益封光萬七千戶。」與故所食凡二萬戶。賞賜前後黃金七千斤,錢六千萬,雜繒三萬疋,奴婢百七十人,馬二千疋,甲第一區。

自昭帝時,光子禹及兄孫雲皆中郎將,雲弟山奉車都尉侍中,領胡越兵。光兩女婿為東西宮衛尉,昆弟諸婿外孫皆奉朝請,為諸曹大夫,騎都尉,給事中。黨親連體,根據於朝廷。光自後元秉持萬機,及上即位,乃歸政。上謙讓不受,諸事皆先關白光,然後奏御天子。光每朝見,上虛己斂容,禮下之已甚。

光秉政前後二十年,地節二年春病篤,車駕自臨問光病,上為之涕泣。光上書謝恩曰:「願分國邑三千戶,以封兄孫奉車都尉山為列侯,奉兄票騎將軍去病祀。」事下丞相御史,即日拜光子禹為右將軍。

光薨,上及皇太后親臨光喪。太中大夫任宣與侍御史五人持節護喪事。中二千石治莫府冢上。賜金錢、繒絮,繡被百領。衣五十篋,璧珠璣玉衣,梓宮、便房、黃腸題湊各一具,樅木外臧槨十五具。東園溫明,皆如乘輿制度。載光尸柩以轀輬車,黃屋左纛,發材官輕車北軍五校士軍陳至茂陵,以送其葬。諡曰宣成侯。發三河卒穿復土,起冢祠堂,置園邑三百家,長丞奉守如舊法。

既葬,封山為樂平侯,以奉車都尉領尚書事。天子思光功德,下詔曰:「故大司馬大將軍博陸侯宿衛孝武皇帝三十有餘年,輔孝昭皇帝十有餘年,遭大難,躬秉誼,率三公九卿大夫定萬世冊以安社稷,天下蒸庶咸以康寧。功德茂盛,朕甚嘉之。復其後世,疇其爵邑,世世無有所與,功如蕭相國。」明年夏,封太子外祖父許廣漢為平恩侯。復下詔曰:「宣成侯光宿衛忠正,勤勞國家。善善及後世,其封光兄孫中郎將雲為冠陽侯。」

禹既嗣為博陸侯,太夫人顯改光時所自造塋制而侈大之。起三山闕,築神道,北臨昭靈,南出承恩,盛飾祠室,輦閣通屬永巷,而幽良人婢妾守之。廣治第室,作乘輿輦,加畫繡絪馮,黃金塗,韋絮薦輪,侍婢以五采絲輓顯,游戲第中。初,光愛幸監奴馮子都,常與計事,及顯寡居,與子都亂。而禹、山亦並繕治第宅,走馬馳逐平樂館。雲當朝請,數稱病私出,多從賓客,張圍獵黃山苑中,使蒼頭奴上朝謁,莫敢譴者。而顯及諸女,晝夜出入長信宮殿中,亡期度。

宣帝自在民間聞知霍氏尊盛日久,內不能善。光薨,上始躬親朝政,御史大夫魏相給事中。顯謂禹、雲、山:「女曹不務奉大將軍餘業,今大夫給事中,他人壹間,女能復自救邪?」後兩家奴爭道,霍氏奴入御史府,欲鸲大夫門,御史為叩頭謝,乃去。人以謂霍氏,顯等始知憂。會魏大夫為丞相,數燕見言事。平恩侯與侍中金安上等徑出入省中。時霍山自若領尚書,上令吏民得奏封事,不關尚書,群臣進見獨往來,於是霍氏甚惡之。

宣帝始立,立微時許妃為皇后。顯愛小女成君,欲貴之,私使乳醫淳于衍行毒藥殺許后,因勸光內成君,代立為后。語在外戚傳。始許后暴崩,吏捕諸醫,劾衍侍疾亡狀不道,下獄。吏簿問急,顯恐事敗,即具以實語光。光大驚,欲自發舉,不忍,猶與。會奏上,因署衍勿論。光薨後,語稍泄。於是上始聞之而未察,乃徙光女婿度遼將軍未央衛尉平陵侯范明友為光祿勳,次婿諸吏中郎將羽林監任勝出為安定太守。數月,復出光姊婿給事中光祿大夫張朔為蜀郡太守,群孫婿中郎將王漢為武威太守。頃之,復徙光長女婿長樂衛尉鄧廣漢為少府。更以禹為大司馬,冠小冠,亡印綬,罷其右將軍屯兵官屬,特使禹官名與光俱大司馬者。又收范明友度遼將軍印綬,但為光祿勳。及光中女婿趙平為散騎騎都尉光祿大夫將屯兵,又收平騎都尉印綬。諸領胡越騎、羽林及兩宮衛將屯兵,悉易以所親信許、史子弟代之。

禹為大司馬,稱病。禹故長史任宣候問,禹曰:「我何病?縣官非我家將軍不得至是,今將軍墳墓未乾,盡外我家,反任許、史,奪我印綬,令人不省死。」宣見禹恨望深,乃謂曰:「大將軍時何可復行!持國權柄,殺生在手中。廷尉李种、王平、左馮翊賈勝胡及車丞相女婿少府徐仁皆坐逆將軍竟下獄死。使樂成小家子得幸將軍,至九卿封侯。百官以下但事馮子都、王子方等,視丞相亡如也。各自有時,今許、史自天子骨肉,貴正宜耳。大司馬欲用是怨恨,愚以為不可。」禹默然。數日,起視事。

顯及禹、山、雲自見日侵削,數相對啼泣,自怨。山曰:「今丞相用事,縣官信之,盡變易大將軍時法令,以公田賦與貧民,發揚大將軍過失。又諸儒生多窶人子,遠客飢寒,喜妄說狂言,不避忌諱,大將軍常讎之,今陛下好與諸儒生語,人人自使書對事,多言我家者。嘗有上書言大將軍時主弱臣強,專制擅權,今其子孫用事,昆弟益驕恣,恐危宗廟,災異數見,盡為是也。其言絕痛,山屏不奏其書。後上書者益黠,盡奏封事,輒使中書令出取之,不關尚書,益不信人。」顯曰:「丞相數言我家,獨無罪乎?」山曰:「丞相廉正,安得罪?我家昆弟諸婿多不謹。又聞民間讙言霍氏毒殺許皇后,寧有是邪?」顯恐急,即具以實告山、雲、禹。山、雲、禹驚曰:「如是,何不早告禹等!縣官離散斥逐諸婿,用是故也。此大事,誅罰不小,柰何?」於是始有邪謀矣。

初,趙平客石夏善為天官,語平曰:「熒惑守御星,御星,太僕奉車都尉也,不黜則死。」平內憂山等。雲舅李竟所善張赦見雲家卒卒,謂竟曰:「今丞相與平恩侯用事,可令太夫人言太后,先誅此兩人。移徙陛下,在太后耳。」長安男子張章告之,事下廷尉。執金吾捕張赦、石夏等,後有詔止勿捕。山等愈恐,相謂曰:「此縣官重太后,故不竟也。然惡端已見,又有弒許后事,陛下雖寬仁,恐左右不聽,久之猶發,發即族矣,不如先也。」遂令諸女各歸報其夫,皆曰:「安所相避?」

會李竟坐與諸侯王交通,辭語及霍氏,有詔雲、山不宜宿衛,免就第。光諸女遇太后無禮,馮子都數犯法,上并以為讓,山、禹等甚恐。顯夢第中井水溢流庭下,灶居樹上,又夢大將軍謂顯曰:「知捕兒不?亟下捕之。」第中鼠暴多,與人相觸,以尾畫地。鴞數鳴殿前樹上。第門自壞。雲尚冠里宅中門亦壞。巷端人共見有人居雲屋上,徹瓦投地,就視,亡有,大怪之。禹夢車騎聲正讙來捕禹,舉家憂愁。山曰:「丞相擅減宗廟羔、菟、杀,可以此罪也。」謀令太后為博平君置酒,召丞相、平恩侯以下,使范明友、鄧廣漢承太后制引斬之,因廢天子而立禹。約定未發,雲拜為玄菟太守,太中大夫任宣為代郡太守。山又坐寫祕書,顯為上書獻城西第,入馬千匹,以贖山罪。書報聞。會事發覺,雲、山、明友自殺,顯、禹、廣漢等捕得。禹要斬,顯及諸女昆弟皆棄市。唯獨霍后廢處昭臺宮。與霍氏相連坐誅滅者數千家。

上乃下詔曰:「乃者東織室令史張赦使魏郡豪李竟報冠陽侯雲謀為大逆,朕以大將軍故,抑而不揚,冀其自新。今大司馬博陸侯禹與母宣成侯夫人顯及從昆弟子冠陽侯雲、樂平侯山諸姊妺婿謀為大逆,欲詿誤百姓。賴祖宗神靈,先發得,咸伏其辜,朕甚悼之。諸為霍氏所詿誤,事在丙申前,未發覺在吏者,皆赦除之。男子張章先發覺,以語期門董忠,忠告左曹楊惲,惲告侍中金安上。惲召見對狀,後章上書以聞。侍中史高與金安上建發其事,言無入霍氏禁闥,卒不得遂其謀,皆讎有功。封章為博成侯,忠高昌侯,惲平通侯,安上都成侯,高樂陵侯。」

初,霍氏奢侈,茂陵徐生曰:「霍氏必亡。夫奢則不遜,不遜必侮上。侮上者,逆道也。在人之右,眾必害之。霍氏秉權日久,害之者多矣。天下害之,而又行以逆道,不亡何待!」乃上疏言「

霍氏泰盛,陛下即愛厚之,宜以時抑制,無使至亡。」書三上,輒報聞。其後霍氏誅滅,而告霍氏者皆封。人為徐生上書曰:「臣聞客有過主人者,見其灶直突,傍有積薪,客謂主人,更為曲突,遠徙其薪,不者且有火患。主人嘿然不應。俄而家果失火,鄰里共救之,幸而得息。於是殺牛置酒,謝其鄰人,灼爛者在於上行,餘各以功次坐,而不錄言曲突者。人謂主人曰:『鄉使聽客之言,不費牛酒,終亡火患。今論功而請賓,曲突徙薪亡恩澤,燋頭爛額為上客耶?』主人乃寤而請之。今茂陵徐福數上書言霍氏且有變,宜防絕之。鄉使福說得行,則國亡裂土出爵之費,臣亡逆亂誅滅之敗。往事既已,而福獨不蒙其功,唯陛下察之,貴徙薪曲突之策,使居焦髮灼爛之右。」上乃賜福帛十疋,後以為郎。

宣帝始立,謁見高廟,大將軍光從驂乘,上內嚴憚之,若有芒剌在背。後車騎將軍張安世代光驂乘,天子從容肆體,甚安近焉。及光身死而宗族竟誅,故俗傳之曰:「威震主者不畜,霍氏之禍萌於驂乘。」

至成帝時,為光置守冢百家,吏卒奉祠焉。元始二年,封光從父昆弟曾孫陽為博陸侯,千戶。

金日磾字翁叔,本匈奴休屠王太子也。武帝元狩中,票騎將軍霍去病將兵擊匈奴右地,多斬首,虜獲休屠王祭天金人。其夏,票騎復西過居延,攻祁連山,大克獲。於是單于怨昆邪、休屠居西方多為漢所破,召其王欲誅之。昆邪、休屠恐,謀降漢。休屠王後悔,昆邪王殺之,并將其眾降漢。封昆邪王為列侯。日磾以父不降見殺,與母閼氏、弟倫俱沒入官,輸黃門養馬,時年十四矣。

久之,武帝游宴見馬,後宮滿側。日磾等數十人牽馬過殿下,莫不竊視,至日磾獨不敢。日磾長八尺二寸,容貌甚嚴,馬又肥好,上異而問之,具以本狀對。上奇焉,即日賜湯沐衣冠,拜為馬監,遷侍中駙馬都尉光祿大夫。日磾既親近,未嘗有過失,上甚信愛之,賞賜累千金,出則驂乘,入侍左右。貴戚多竊怨,曰:「陛下妄得一胡兒,反貴重之!」上聞,愈厚焉。

日磾母教誨兩子,甚有法度,上聞而嘉之。病死,詔圖畫於甘泉宮,署曰「休屠王閼氏。」日磾每見畫常拜,鄉之涕泣,然後乃去。日磾子二人皆愛,為帝弄兒,常在旁側。弄兒或自後擁上項,日磾在前,見而目之。弄兒走且啼曰:「翁怒。」上謂日磾「何怒吾兒為?」其後弄兒壯大,不謹,自殿下與宮人戲,日磾適見之,惡其淫亂,遂殺弄兒。弄兒即日磾長子也。上聞之大怒,日磾頓首謝,具言所以殺弄兒狀。上甚哀,為之泣,已而心敬日磾。

初,莽何羅與江充相善,及充敗衛太子,何羅弟通用誅太子時力戰得封。後上知太子冤,乃夷滅充宗族黨與。何羅兄弟懼及,遂謀為逆。日磾視其志意有非常,心疑之,陰獨察其動靜,與俱上下。何羅亦覺日磾意,以故久不得發。是時上行幸林光宮,日磾小疾臥廬。何羅與通及小弟安成矯制夜出,共殺使者,發兵。明旦,上未起,何羅亡何從外入。日磾奏廁心動,立入坐內戶下。須臾,何羅袖白刃從東箱上,見日磾,色變,走趨臥內欲入,行觸寶瑟,僵。日磾得抱何羅,因傳曰:「

莽何羅反!」上驚起,左右拔刃欲格之,上恐并中日磾,止勿格。日磾捽胡投何羅殿下,得禽縛之,窮治皆伏辜。繇是著忠孝節。

日磾自在左右,目不忤視者數十年。賜出宮女,不敢近。上欲內其女後宮,不肯。其篤慎如此,上尤奇異之。及上病,屬霍光以輔少主,光讓日磾。日磾曰:「臣外國人,且使匈奴輕漢。」於是遂為光副。光以女妻日磾嗣子賞。初,武帝遺詔以討莽何羅功封日磾為秺侯,日磾以帝少不受封。輔政歲餘,病困,大將軍光白封日磾,臥授印綬。一日,薨,賜葬具冢地,送以輕車介士,軍陳至茂陵,諡曰敬侯。

日磾兩子,賞、建,俱侍中,與昭帝略同年,共臥起。賞為奉車、建駙馬都尉。及賞嗣侯,佩兩綬,上謂霍將軍曰:「金氏兄弟兩人不可使俱兩綬邪?」霍光對曰:「賞自嗣父為侯耳。」上笑曰:「侯不在我與將軍乎?」光曰:「先帝之約,有功乃得封侯。」時年俱八九歲。宣帝即位,賞為太僕,霍氏有事萌牙,上書去妻。上亦自哀之,獨得不坐。元帝時為光祿勳,薨,亡子,國除。元始中繼絕世,封建孫當為秺侯,奉日磾後。

初,日磾所將俱降弟倫,字少卿,為黃門郎,早卒。日磾兩子貴,及孫則衰矣,而倫後嗣遂盛,子安上始貴顯封侯。

安上字子侯,少為侍中,惇篤有智,宣帝愛之。頗與發舉楚王延壽反謀,賜爵關內侯,食邑三百戶。後霍氏反,安上傳禁門闥,無內霍氏親屬,封為都成侯,至建章衛尉。薨,賜冢塋杜陵,諡曰敬侯。四子,常、敞、岑、哭。

今、明皆為諸曹中郎將,常光祿大夫。元帝為太子時,敞為中庶子,幸有寵,帝即位,為騎都尉光祿大夫,中郎將侍中。元帝崩,故事,近臣皆隨陵為園郎,敞以世名忠孝,太后詔留侍成帝,為奉車水衡都尉,至衛尉。敞為人正直,敢犯顏色,左右憚之,唯上亦難焉。病甚,上使使者問所欲,以弟岑為託。上召岑,拜為郎使主客。敞子涉本為左曹,上拜涉為侍中,使待幸綠車載送衛尉舍。須臾卒。敞三子,涉、參、饒。

涉明經儉節,諸儒稱之。成帝時為侍中騎都尉,領三輔胡越騎。哀帝即位,為奉車都尉,至長信少府。而參使匈奴,匈奴中郎將,越騎校尉,關都尉,安定、東海太守。饒為越騎校尉。

涉兩子,湯、融,皆侍中諸曹將大夫。而涉之從父弟欽舉明經,為太子門大夫,哀帝即位,為太中大夫給事中,欽從父弟遷為尚書令,兄弟用事。帝祖母傅太后崩,欽使護作,職辦,擢為泰山、弘農太守,著威名。平帝即位,徵為大司馬司直、京兆尹。帝年幼,選置師友,大司徒孔光以明經高行為孔氏師,京兆尹金欽以家世忠孝為金氏友。徙光祿大夫侍中,秩中二千石,封都成侯。

時王莽新誅平帝外家衛氏,召明禮少府宗伯鳳入說為人後之誼,白令公卿、將軍、侍中、朝臣並聽,欲以內厲平帝而外塞百姓之議。欽與族昆弟秺侯當俱封。初,當曾祖父日磾傳子節侯賞,而欽祖父安上傳子夷侯常,皆亡子,國絕,故莽封欽、當奉其後。當母南即莽母功顯君同產弟也。當上南大行為太夫人。欽因緣謂當:「詔書陳日磾功,亡有賞語。當名為以孫繼祖也,自當為父、祖父立廟。賞故國君,使大夫主其祭。」時甄邯在旁,庭叱欽,因劾奏曰:「欽幸得以通經術,超擢侍帷幄,重蒙厚恩,封襲爵號,知聖朝以世有為人後之誼。前遭故定陶太后背本逆天,孝哀不獲厥福,乃者呂寬、衛寶復造姦謀,至於反逆,咸伏厥辜。太皇太后懲艾悼懼,逆天之咎,非聖誣法,大亂之殃,誠欲奉承天心,遵明聖制,專壹為後之誼,以安天下之命,數臨正殿,延見群臣,講習禮經。孫繼祖者,謂亡正統持重者也。賞見嗣日磾,後成為君,持大宗重,則禮所謂『尊祖故敬宗』,大宗不可以絕者也。欽自知與當俱拜同誼,即數揚言殿省中,教當云云。當即如其言,則欽亦欲為父明立廟而不入夷侯常廟矣。進退異言,頗惑眾心,亂國大綱,開禍亂原,誣祖不孝,罪莫大焉。尤非大臣所宜,大不敬。秺侯當上母南為太夫人,失禮不敬。」莽白太后,下四輔、公卿、大夫、博士、議郎,皆曰:「欽宜以時即罪。」謁者召欽詣詔獄,欽自殺。邯以綱紀國體,亡所阿私,忠孝尤著,益封千戶。更封長信少府涉子右曹湯為都成侯。湯受封日,不敢還歸家,以明為人後之誼。益封之後,莽復用欽弟遵,封侯,歷九卿位。

贊曰:霍光以結髮內侍,起於階闥之間,確然秉志,誼形於主。受襁褓之託,任漢室之寄,當廟堂,擁幼君,摧燕王,仆上官,因權制敵,以成其忠。處廢置之際,臨大節而不可奪,遂匡國家,安社稷。擁昭立宣,光為師保,雖周公、阿衡,何以加此!然光不學亡術,闇於大理,陰妻邪謀,立女為后,湛溺盈溢之欲,以增顛覆之禍,死財三年,宗族誅夷,哀哉!昔霍叔封於晉,晉即河東,光豈其苗裔乎?金日磾夷狄亡國,羈虜漢庭,而以篤敬寤主,忠信自著,勒功上將,傳國後嗣,世名忠孝,七世內侍,何其盛也!本以休屠作金人為祭天主,故因賜姓金氏云。


\end{pinyinscope}