\article{五行志}

\begin{pinyinscope}
《易》曰:「天垂象,見吉凶,聖人象之;河出圖,雒出書,聖人則之。」劉歆以為虙羲氏繼天而王,受河圖,則而畫之,八卦是也;禹治洪水,賜雒書,法而陳之,洪範是也。聖人行其道而寶其真。降及于殷,箕子在父師位而典之。周既克殷,以箕子歸,武王親虛己而問焉。故經曰:「惟十有三祀,王訪于箕子,王乃言曰:『烏呼,箕子!惟天陰騭下民,相協厥居,我不知其彝倫逌敘。』箕子乃言曰:『我聞在昔,鯀讯洪水,汨陳其五行,帝乃震怒,弗畀洪範九疇,彝倫逌斁。鯀則殛死,禹乃嗣興,天乃錫禹洪範九疇,彝倫逌敘。』」此武王問雒書於箕子,箕子對禹得雒書之意也。

「初一曰五行;次二曰羞用五事;次三曰農用八政;次四曰渖用五紀;次五曰建用皇極;次六曰艾用三德;次七曰明用稽疑;次八曰念用庶徵;次九曰嚮用五福,畏用六極。」凡此六十五字,皆雒書本文,所謂天乃錫禹大法九章常事所次者也。以為河圖、雒書相為經緯,八卦、九章相為表裏。昔殷道弛,文王演周易;周道敝,孔子述春秋。則乾坤之陰陽,效洪範之咎徵,天人之道粲然著矣。

漢興,承秦滅學之後,景、武之世,董仲舒治公羊春秋,始推陰陽,為儒者宗。宣、元之後,劉向治穀梁春秋,數其禍福,傳以洪範,與仲舒錯。至向子歆治左氏傳,其春秋意亦已乖矣;言五行傳,又頗不同。是以髓仲舒,別向、歆,傳載眭孟、夏侯勝、京房、谷永、李尋之徒所陳行事,訖於王莽,舉十二世,以傅春秋,著於篇。

經曰:「初一曰五行。五行:一曰水,二曰火,三曰木,四曰金,五曰土。水曰潤下,火曰炎上,木曰曲直,金曰從革,土爰稼穡。」

傳曰:「田獵不宿,飲食不享,出入不節,奪民農時,及有姦謀,則木不曲直。」

說曰:木,東方也。於易,地上之木為觀。其於王事,威儀容貌亦可觀者也。故行步有佩玉之度,登車有和鸞之節,田狩有三驅之制,飲食有享獻之禮,出入有名,使民以時,務在勸農桑,謀在安百姓:如此,則木得其性矣。若乃田獵馳騁不反宮室,飲食沈湎不顧法度,妄興繇役以奪民時,作為姦詐以傷民財,則木失其性矣。蓋工匠之為輪矢者多傷敗,及木為變怪,是為木不曲直。

春秋成公十六年「正月,雨,木冰」。劉歆以為上陽施不下通,下陰施不上達,故雨,而木為之冰,雰氣寒,木不曲直也。劉向以為冰者陰之盛而水滯者也,木者少陽,貴臣卿大夫之象也。此人將有害,則陰氣協木,木先寒,故得雨而冰也。是時叔孫喬如出奔,公子偃誅死。一曰,時晉執季孫行父,又執公,此執辱之異。或曰,今之長老名木冰為「木介」。介者,甲。甲,兵象也。是歲晉有鄢陵之戰,楚王傷目而敗。屬常雨也。

傳曰:「棄法律,逐功臣,殺太子,以妾為妻,則火不炎上。」

說曰:火,南方,揚光煇為明者也。其於王者,南面鄉明而治。《書》云:「知人則悊,能官人。」故堯舜舉群賢而命之朝,遠四佞而放諸野。孔子曰:「浸潤之譖、膚受之訴不行焉,可謂明矣。」賢佞分別,官人有序,帥由舊章,敬重功勳,殊別適庶,如此則火得其性矣。若乃信道不篤,或燿虛偽,讒夫昌,邪勝正,則火失其性矣。自上而降,及濫炎妄起,災宗廟,燒宮館,雖興師眾,弗能救也,是為火不炎上。

春秋桓公十四年「八月壬申,御廩災」。董仲舒以為先是四國共伐魯,大破之於龍門。百姓傷者未瘳,怨咎未復,而君臣俱惰,內怠政事,外侮四鄰,非能保守宗廟終其天年者也,故天災御廩以戒之。劉向以為御廩,夫人八妾所舂米之臧以奉宗廟者也,時夫人有淫行,挾逆心,天戒若曰,夫人不可以奉宗廟。桓不寤,與夫人俱會齊,夫人譖桓公於齊侯,齊侯殺桓公。劉歆以為御廩,公所親耕籍田以奉粢盛者也,棄法度亡禮之應也。

嚴公二十年「夏,齊大災」。劉向以為齊桓好色,聽女口,以妾為妻,適庶數更,故致太災。桓公不寤,及死,適庶分爭,九月不得葬。公羊傳曰,大災,疫也。董仲舒以為魯夫人淫於齊,齊桓姊妹不嫁者七人。國君,民之父母;夫婦,生化之本。本傷則末夭,故天災所予也。

釐公二十年「五月己酉,西宮災」。穀梁以為愍公宮也,以諡言之則若疏,故謂之西宮。劉向以為釐立妾母為夫人以入宗廟,故天災愍宮,若曰,去其卑而親者,將害宗廟之正禮。董仲舒以為釐娶於楚,而齊媵之,脅公使立以為夫人。西宮者,小寢,夫人之居也。若曰,妾何為此宮!誅去之意也。以天災之,故大之曰西宮也。左氏以為西宮者,公宮也。言西,知有東。東宮,太子所居。言宮,舉區皆災也。

宣公十六年「夏,成周宣榭火」。榭者,所以臧樂器,宣其名也。董仲舒、劉向以為十五年王札子殺召伯、毛伯,天子不能誅。天戒若曰,不能行政令,何以禮樂為而臧之?左氏經曰:「

成周宣榭火,人火也。人火曰火,天火曰災。」榭者,講武之坐屋。

成公三年「二月甲子,新宮災」。穀梁以為宣宮,不言諡,恭也。劉向以為時魯三桓子孫始執國政,宣公欲誅之,恐不能,使大夫公孫歸父如晉謀。未反,宣公死。三家譖歸父於成公。成公父喪未葬,聽讒而逐其父之臣,使奔齊,故天災宣宮,明不用父命之象也。一曰,三家親而亡禮,猶宣公殺子赤而立。亡禮而親,天災宣廟,欲示去三家也。董仲舒以為成居喪亡哀戚心,數興兵戰伐,故天災其父廟,示失子道,不能奉宗廟也。一曰,宣殺君而立,不當列於群祖也。

襄公九年「春,宋災」。劉向以為先是宋公聽讒,逐其大夫華弱,出奔魯。左氏傳曰,宋災,樂喜為司城,先使火所未至徹小屋,塗大屋,陳畚輂,具綆缶,備水器,畜水潦,積土塗,繕守備,表火道,儲正徒。郊保之民,使奔火所。又飭眾官,各慎其職。晉侯聞之,問士弱曰:「宋災,於是乎知有天道,何故?」對曰:「古之火正,或食於心,或食於咮,以出入火。是故咮為鶉火,心為大火。陶唐氏之火正閼伯,居商丘,祀大火,而火紀時焉。相土因之,故商主大火。商人閱其禍敗之釁必始於火,是以知有天道。」公曰:「可必乎?」對曰:「在道。國亂亡象,不可知也。」說曰:古之火正,謂火官也,掌祭火星,行火政。季春昏,心星出東方,而咮、七星、鳥首正在南方,則用火;季秋,星入,則止火,以順天時,救民疾。帝嚳則有祝融,堯時有閼伯,民賴其德,死則以為火祖,配祭火星,故曰「或食於心,或食於咮也」。相土,商祖契之曾孫,代閼伯後主火星。宋,其後也。世司其占,故先知火災。賢君見變,能修道以除凶;亂君亡象,天不譴告,故不可必也。

三十年「五月甲午,宋災」。董仲舒以為伯姬如宋五年,宋恭公卒,伯姬幽居守節三十餘年,又憂傷國家之患禍,積陰生陽,故火生災也。劉向以為先是宋公聽讒而殺太子痤,應火不炎上之罰也。

左氏傳昭公六年「六月丙戌,鄭災」。是春三月,鄭人鑄刑書。士文伯曰:「火見,鄭其火乎?火未出而作火以鑄刑器,臧爭辟焉。火而象之,不火何為?」說曰:火星出於周五月,而鄭以三月作火鑄鼎,刻刑辟書,以為民約,是為刑器爭辟。故火星出,與五行之火爭明為災,其象然也,又棄法律之占也。不書於經,時不告魯也。

九年「夏四月,陳火」。董仲舒以為陳夏徵舒殺君,楚嚴王託欲為陳討賊,陳國闢門而待之,至因滅陳。陳臣子尤毒恨甚,極陰生陽,故致火災。劉向以為先是陳侯弟招殺陳太子偃師,皆外事,不因其宮館者,略之也。八年十月壬午,楚師滅陳,春秋不與蠻夷滅中國,故復書陳火也。左氏經曰「陳災」。傳曰「鄭屓灶曰:『五年,陳將復封,封五十二年而遂亡。』子產問其故,對曰:『陳,水屬也。火,水妃也,而楚所相也。今火出而火陳,逐楚而建陳也。妃以五陳,故曰五年。歲五及鶉火,而後陳卒亡,楚克有之,天之道也。』」說曰:顓頊以水王,陳其族也。今茲歲在星紀,後五年在大梁。大梁,昴也。金為水宗,得其宗而昌,故曰「五年陳將復封」。楚之先為火正,故曰「楚所相也」。天以一生水,地以二生火,天以三生木,地以四生金,天以五生土。五位皆以五而合,而陰陽易位,故曰「妃以五成」。然則水之大數六,火七,木八,金九,土十。故水以天一為火二牡,木以天三為土十牡,土以天五為水六牡,火以天七為金四牡,金以天九為木八牡。陽奇為牡,陰耦為妃。故曰「水,火之牡也;火,水妃也」。於易,坎為水,為中男,離為火,為中女,蓋取諸此也。自大梁四歲而及鶉火,四周四十八歲,凡五及鶉火,五十二年而陳卒亡。火盛水衰,故曰「天之道也」。哀公十七年七月己卯,楚滅陳。

昭十八年「五月壬午,宋、衛、陳、鄭災」。董仲舒以為象王室將亂,天下莫救,故災四國,言亡四方也。又宋、衛、陳、鄭之君皆荒淫於樂,不恤國政,與周室同行。陽失節則火災出,是以同日災也。劉向以為宋、陳,王者之後,衛、鄭,周同姓也。時周景王老,劉子、單子事王子猛,尹氏、召伯、毛伯事王子晁。子晁,楚之出也。及宋、衛、陳、鄭亦皆外附於楚,亡尊周室之心。後三年,景王崩,王室亂,故天災四國。天戒若曰,不救周,反從楚,廢世子,立不正,以害王室,明同罪也。

定公二年「五月,雉門及兩觀災」。董仲舒、劉向以為此皆奢僭過度者也。先是,季氏逐昭公,昭公死于外。定公即位,既不能誅季氏,又用其邪說,淫於女樂,而退孔子。天戒若曰,去高顯而奢僭者。一曰,門闕,號令所由出也,今舍大聖而縱有罪,亡以出號令矣。京房易傳曰:「君不思道,厥妖火燒宮。」

哀公三年「五月辛卯,桓、釐宮災」。董仲舒、劉向以為此二宮不當立,違禮者也。哀公又以季氏之故不用孔子。孔子在陳聞魯災,曰:「其桓、釐之宮乎!」以為桓,季氏之所出,釐,使季氏世卿者也。

四年「六月辛丑,亳社災」。董仲舒、劉向以為亡國之社,所以為戒也。天戒若曰,國將危亡,不用戒矣。春秋火災,屢於定、哀之間,不用聖人而縱驕臣,將以亡國,不明甚也。一曰,天生孔子,非為定、哀也,蓋失禮不明,火災應之,自然象也。

高后元年五月丙申,趙叢臺災。劉向以為是時呂氏女為趙王后,嫉妒,將為讒口以害趙王。王不寤焉,卒見幽殺。

惠帝四年十月乙亥,未央宮凌室災;丙子,織室災。劉向以為元年呂太后殺趙王如意,殘戮其母戚夫人。是歲十月壬寅,太后立帝姊魯元公主女為皇后。其乙亥,凌室災。明日,織室災。凌室所以供養飲食,織室所以奉宗廟衣服,與春秋御廩同義。天戒若曰,皇后亡奉宗廟之德,將絕祭祀。其後,皇后亡子,後宮美人有男,太后使皇后名之,而殺其母。惠帝崩,嗣子立,有怨言,太后廢之,更立呂氏子弘為少帝。賴大臣共誅諸呂而立文帝,惠后幽廢。

文帝七年六月癸酉,未央宮東闕罘思災。劉向以為東闕所以朝諸侯之門也,罘思在其外,諸侯之象也。漢興,大封諸侯王,連城數十。文帝即位,賈誼等以為違古制度,必將叛逆。先是,濟北、淮南王皆謀反,其後吳楚七國舉兵而誅。

景帝中五年八月己酉,未央宮東闕災。先是,栗太子廢為臨江王,以罪徵詣中尉,自殺。丞相條侯周亞夫以不合旨稱疾免,後二年下獄死。

武帝建元六年六月丁酉,遼東高廟災。四月壬子,高園便殿火。董仲舒對曰:「春秋之道舉往以明來,是故天下有物,視春秋所舉與同比者,精微眇以存其意,通倫類以貫其理,天地之變,國家之事,粲然皆見,亡所疑矣。按春秋魯定公、哀公時,季氏之惡已孰,而孔子之聖方盛。夫以盛聖而易孰惡,季孫雖重,魯君雖輕,其勢可成也。故定公二年五月兩觀災。兩觀,僭禮之物,天災之者,若曰,僭禮之臣可以去。已見罪徵,而後告可去,此天意也。定公不知省。至哀公三年五月,桓宮、釐宮災。二者同事,所為一也,若曰燔貴而去不義云爾。哀公未能見,故四年六月亳社災。兩觀、桓、釐廟、亳社,四者皆不當立,天皆燔其不當立者以示魯,欲其去亂臣而用聖人也。季氏亡道久矣,前是天不見災者,魯未有賢聖臣,雖欲去季孫,其力不能,昭公是也。至定、哀乃見之,其時可也。不時不見,天之道也。今高廟不當居遼東,高園殿不當居陵旁,於禮亦不當立,與魯所災同。其不當立久矣,至於陛下時天乃災之者,殆亦其時可也。昔秦受亡周之敝,而亡以化之;漢受亡秦之敝,又亡以化之。夫繼二敝之後,承其下流,兼受其猥,難治甚矣。又多兄弟親戚骨肉之連,驕揚奢侈恣睢者眾,所謂重難之時者也。陛下正當大敝之後,又遭重難之時,甚可憂也。故天災若語陛下:『當今之世,雖敝而重難,非以太平至公,不能治也。視親戚貴屬在諸侯遠正最甚者,忍而誅之,如吾燔遼高廟乃可;視近臣在國中處旁仄及貴而不正者,忍而誅之,如吾燔高園殿乃可』云爾。在外而不正者,雖貴如高廟,猶災燔之,況諸侯乎!在內不正者,雖貴如高園殿,猶燔災之,況大臣乎!此天意也。罪在外者天災外,罪在內者天災內,燔甚罪當重,燔簡罪當輕,承天意之道也。」

先是,淮南王安入朝,始與帝舅太尉武安侯田蚡有逆言。其後膠西于王、趙敬肅王、常山憲王皆數犯法,或至夷滅人家,藥殺二千石,而淮南、衡山王遂謀反。膠東、江都王皆知其謀,陰治兵弩,欲以應之。至元朔六年,乃發覺而伏辜。時田蚡已死,不及誅。上思仲舒前言,使仲舒弟子呂步舒持斧鉞治淮南獄,以春秋誼顓斷於外,不請。既還奏事,上皆是之。

太初元年十一月乙酉,未央宮柏梁臺災。先是,大風發其屋,夏侯始昌先言其災日。後有江充巫蠱衛太子事。

征和二年春,涿郡鐵官鑄鐵,鐵銷,皆飛上去,此火為變使之然也。其三月,涿郡太守劉屈釐為丞相。後月,巫蠱事興,帝女諸邑公主、陽石公主、丞相公孫賀、子太僕敬聲、平陽侯曹宗等皆下獄死。七月,使者江充掘蠱太子宮,太子與母皇后議,恐不能自明,乃殺充,舉兵與丞相劉屈釐戰,死者數萬人,太子敗走,至湖自殺。明年,屈釐復坐祝验要斬,妻梟首也。成帝河平二年正月,沛郡鐵官鑄鐵,鐵不下,隆隆如雷聲,又如鼓音,工十三人驚走。音止,還視地,地陷數尺,鑪分為十,一鑪中銷鐵散如流星,皆上去,與征和二年同象。其夏,帝舅五人封列侯,號五侯。元舅王鳳為大司馬大將軍秉政。後二年,丞相王商與鳳有隙,鳳譖之,免官,自殺。明年,京兆尹王章訟商忠直,言鳳顓權,鳳誣章以大逆罪,下獄死,妻子徙合浦。後許皇后坐巫蠱廢,而趙飛燕為皇后,妹為昭儀,賊害皇子,成帝遂亡嗣。皇后,昭儀皆伏辜。一曰,鐵飛屬金不從革。

昭帝元鳳元年,燕城南門災。劉向以為時燕王使邪臣通於漢,為讒賊,謀逆亂。南門者,通漢道也。天戒若曰,邪臣往來,為姦讒於漢,絕亡之道也。燕王不寤,卒伏其辜。

元鳳四年五月丁丑,孝文廟正殿災。劉向以為孝文,太宗之君,與成周宣榭火同義。先是,皇后父車騎將軍上官安、安父左將軍桀謀為逆,大將軍霍光誅之。皇后以光外孫,年少不知,居位如故。光欲后有子,因上侍疾醫言,禁內後宮皆不得進,唯皇后顓寢。皇后年六歲而立,十三年而昭帝崩,遂絕繼嗣。光執朝政,猶周公之攝也。是歲正月,上加元服,通詩、尚書,有明悊之性。光亡周公之德,秉政九年,久於周公,上既已冠而不歸政,將為國害。故正月加元服,五月而災見。古之廟皆在城中,孝文廟始出居外,天戒若曰,去貴而不正者。宣帝既立,光猶攝政,驕溢過制,至妻顯殺許皇后,光聞而不討,後遂誅滅。

宣帝甘露元年四月丙申,中山太上皇廟災。甲辰,孝文廟災。元帝初元三年四月乙未,孝武園白鶴館災。劉向以為先是前將軍蕭望之、光祿大夫周堪輔政,為佞臣石顯、許章等所譖,望之自殺,堪廢黜。明年,白鶴館災。園中五里馳逐走馬之館,不當在山陵昭穆之地。天戒若曰,去貴近逸遊不正之臣,將害忠良。後章坐走馬上林下烽馳逐,免官。

永光四年六月甲戌,孝宣杜陵園東闕南方災。劉向以為先是上復徵用周堪為光祿勳,及堪弟子張猛為太中大夫,石顯等復譖毀之,皆出外遷。是歲,上復徵堪領尚書,猛給事中,石顯等終欲害之。園陵小於朝廷,闕在司馬門中,內臣石顯之象也。孝宣,親而貴;闕,法令所從出也。天戒若曰,去法令,內臣親而貴者必為國害。後堪希得進見,因顯言事,事決顯口。堪病不能言。顯誣告張猛,自殺於公車。成帝即位,顯卒伏辜。

成帝建始元年正月乙丑,皇考廟災。初,宣帝為昭帝後而立父廟,於禮不正。是時大將軍王鳳顓權擅朝,甚於田蚡,將害國家,故天於元年正月而見象也。其後寖盛,五將世權,遂以亡道。

鴻嘉三年八月乙卯,孝景廟北闕災。十一月甲寅,許皇后廢。

永始元年正月癸丑,大官凌室災。戊午,戾后園南闕災。是時,趙飛燕大幸,許后既廢,上將立之,故天見象於凌室,與惠帝四年同應。戾后,衛太子妾,遭巫蠱之禍,宣帝既立,追加尊號,於禮不正。又戾后起於微賤,與趙氏同。天戒若曰,微賤亡德之人不可以奉宗廟,將絕祭祀,有凶惡之禍至。其六月丙寅,趙皇后遂立,姊妺驕妒,賊害皇子,卒皆受誅。

永始四年四月癸未,長樂宮臨華殿及未央宮東司馬門災。六月甲午,孝文霸陵園東闕南方災。長樂宮,成帝母王太后之所居也。未央宮,帝所居也。霸陵,太宗盛德園也。是時,太后三弟相續秉政,舉宗居位,充塞朝廷,兩宮親屬將害國家,故天象仍見。明年,成都侯商薨,弟曲陽侯根代為大司馬秉政。後四年,根乞骸骨,薦兄子新都侯莽自代,遂覆國焉。

哀帝建平三年正月癸卯,桂宮鴻寧殿災,帝祖母傅太后之所居也。時,傅太后欲與成帝母等號齊尊,大臣孔光、師丹等執政,以為不可,太后皆免官爵,遂稱尊號。後三年,帝崩,傅氏誅滅。

平帝元始五年七月己亥,高皇帝原廟殿門災盡。高皇帝廟在長安城中,後以叔孫通譏復道,故復起原廟於渭北,非正也。是時平帝幼,成帝母王太后臨朝,委任王莽,將篡絕漢,墮高祖宗廟,故天象見也。其冬,平帝崩。明年,莽居攝,因以篡國,後卒夷滅。

傳曰:「治宮室,飾臺榭,內淫亂,犯親戚,侮父兄,則稼穡不成。」

說曰:土,中央,生萬物者也。其於王者,為內事。宮室、夫婦、親屬,亦相生者也。古者天子諸侯,宮廟大小高卑有制,后夫人媵妾多少進退有度,九族親疏長幼有序。孔子曰:「禮,與其奢也,寧儉。」故禹卑宮室,文王刑于寡妻,此聖人之所以昭教化也。如此則土得其性矣。若乃奢淫驕慢,則土失其性。亡水旱之災而草木百穀不孰,是為稼穡不成。

嚴公二十八年「冬,大水亡麥禾」。董仲舒以為夫人哀姜淫亂,逆陰氣,故大水也。劉向以為水旱當書,不書水旱而曰「

大亡麥禾」者,土氣不養,稼穡不成者也。是時,夫人淫於二叔,內外亡別,又因凶飢,一年而三築臺,故應是而稼穡不成,飾臺榭內淫亂之罰云。遂不改寤,四年而死,禍流二世,奢淫之患也。

傳曰:「好戰攻,輕百姓,飾城郭,侵邊境,則金不從革。」

說曰:金,西方,萬物既成,殺氣之始也。故立秋而鷹隼擊,秋分而微霜降。其於王事,出軍行師,把旄杖鉞,誓士眾,抗威武,所以征畔逆止暴亂也。《詩》云:「有虔秉鉞,如火烈烈。」又曰:「載戢干戈,載櫜弓矢。」動靜應誼,「說以犯難,民忘其死。」金得其性矣。若乃貪欲恣睢,務立威勝,不重民命,則金失其性。蓋工冶鑄金鐵,金鐵冰滯涸堅,不成者眾,及為變怪,是為金不從革。

左氏傳曰昭公八年「春,石言於晉」。晉平公問於師曠,對曰;「石不能言,神或馮焉。作事不時,怨讟動於民,則有非言之物而言。今宮室崇侈,民力彫盡,怨讟並興,莫信其性,石之言不宜乎!」於是晉侯方築虒祁之宮。叔向曰:「君子之言,信而有徵。」劉歆以為金石同類,是為金不從革,失其性也。劉向以為石白色為主,屬白祥。

成帝鴻嘉三年五月乙亥,天水冀南山大石鳴,聲隆隆如雷,有頃止,聞平襄二百四十里,野雞皆鳴。石長丈三尺,廣厚略等,旁著岸脅,去地二百餘丈,民俗名曰石鼓。石鼓鳴,有兵。是歲,廣漢鉗子謀攻牢,篡死罪囚鄭躬等,盜庫兵,劫略吏民,衣繡衣,自號曰山君,黨與娅廣。明年冬,乃伏誅,自歸者三千餘人。後四年,尉氏樊並等謀反,殺陳留太守嚴普,自稱將軍,山陽亡徒蘇令等黨與數百人盜取庫兵,經歷郡國四十餘,皆踰年乃伏誅。是時起昌陵,作者數萬人,徙郡國吏民五千餘戶以奉陵邑。作治五年不成,乃罷昌陵,還徙家。石鳴,與晉石言同應,師曠所謂「民力彫盡」,傳云「輕百姓」者也。虒祁離宮去絳都四十里,昌陵亦在郊野,皆與城郭同占。城郭屬金,宮室屬土,外內之別云。

傳曰:「簡宗廟,不禱祠,廢祭祀,逆天時,則水不潤下。」

說曰:水,北方,終臧萬物者也。其於人道,命終而形臧,精神放越,聖人為之宗廟以收魂氣,春秋祭祀,以終孝道。王者即位,必郊祀天地,禱祈神祇,望秩山川,懷柔百神,亡不宗事。慎其齊戒,致其嚴敬,鬼神歆饗,多獲福助。此聖王所以順事陰氣,和神人也。至發號施令,亦奉天時。十二月咸得其氣,則陰陽調而終始成。如此則水得其性矣。若乃不敬鬼神,致令逆時,則水失其性。霧水暴出,百川逆溢,壞鄉邑,溺人民,及淫雨傷稼穡,是為水不潤下。京房易傳曰:「顓事有知,誅罰絕理,厥災水,其水也,雨殺人以隕霜,大風天黃。飢而不損茲謂泰,厥災水,水殺人。辟遏有德茲謂狂,厥災水,水流殺人,已水則地生蟲。歸獄不解,茲謂追非,厥水寒,殺人。追誅不解,茲謂不理,厥水五穀不收。大敗不解,茲謂皆陰。解,舍也,王者於大敗,誅首惡,赦其眾,不則皆函陰氣,厥水流入國邑,隕霜殺穀。」

桓公元年「秋,大水」。董仲舒、劉向以為桓弒兄隱公,民臣痛隱而賤桓。後宋督弒其君,諸侯會,將討之,桓受宋賂而歸,又背宋。諸侯由是伐魯,仍交兵結讎,伏尸流血,百姓愈怨,故十三年夏復大水。一曰,夫人驕淫,將弒君,陰氣盛,桓不寤,卒弒死。劉歆以為桓易許田,不祀周公,廢祭祀之罰也。

嚴公七年「秋,大水,亡麥苗」。董仲舒、劉向以為嚴母文姜與兄齊襄公淫,共殺威公,嚴釋父讎,復取齊女,未入,先與之淫,一年再出,會於道逆亂,臣下賤之之應也。

十一年「秋,宋大水」。董仲舒以為時魯、宋比年為乘丘、鄑之戰,百姓愁怨,陰氣盛,故二國俱水。劉向以為時宋愍公驕慢,睹災不改,明年與其臣宋萬博戲,婦人在側,矜而罵萬,萬殺公之應。

二十四年,「大水」。董仲舒以為夫人哀姜淫亂不婦,陰氣盛也。劉向以為哀姜初入,公使大夫宗婦見,用幣,又淫於二叔,公弗能禁。臣下賤之,故是歲、明年仍大水。劉歆以為先是嚴飾宗廟,刻桷丹楹,以夸夫人,簡宗廟之罰也。

宣公十年「秋大水,飢」。董仲舒以為時比伐邾取邑,亦見報復,兵讎連結,百姓愁怨。劉向以為宣公殺子赤而立,子赤,齊出也,故懼,以濟西田賂齊。邾子貜且亦齊出也,而宣比與邾交兵。臣下懼齊之威,創邾之禍,皆賤公行而非其正也。

成公五年「秋,大水」。董仲舒、劉向以為時成幼弱,政在大夫,前此一年再用師,明年復城鄆以彊私家,仲孫蔑、叔孫僑如顓會宋、晉,陰勝陽。

襄公二十四年「秋,大水」。董仲舒以為先是一年齊伐晉,襄使大夫帥師救晉,後又侵齊,國小兵弱,數敵彊大,百姓愁怨,陰氣盛。劉向以為先是襄慢鄰國,是以邾伐其南,齊伐其北,莒伐其東,百姓騷動,後又仍犯彊齊也。大水,饑,穀不成,其災甚也。

高后三年夏,漢中、南郡大水,水出流四千餘家。四年秋,河南大水,伊、雒流千六百餘家,汝水流八百餘家。八年夏,漢中、南郡水復出,流六千餘家。南陽沔水流萬餘家。是時女主獨治,諸呂相王。

文帝後三年秋,大雨,晝夜不絕三十五日。藍田山水出,流九百餘家。燕,壞民室八千餘所,殺三百餘人。先是,趙人新垣平以望氣得幸,為上立渭陽五帝廟,欲出周鼎,以夏四月,郊見上帝。歲餘懼誅,謀為逆,發覺,要斬,夷三族。是時,比再遣公主配單于,賂遺甚厚,匈奴愈驕,侵犯北邊,殺略多至萬餘人,漢連發軍征討戍邊。

元帝永光五年夏及秋,大水。潁川、汝南、淮陽、廬江雨,壞鄉聚民舍,及水流殺人。先是一年有司奏罷郡國廟,是歲又定迭毀,罷太上皇、孝惠帝寢廟,皆無復修,通儒以為違古制。刑臣石顯用事。

成帝建始三年夏,大水,三輔霖雨三十餘日,郡國十九雨,山谷水出,凡殺四千餘人,壞官寺民舍八萬三千餘所。元年,有司奏徙甘泉泰畤、河東后土于長安南北郊。二年,又罷雍五畤、郡國諸舊祀,凡六所。

經曰:「羞用五事。五事:一曰貌,二曰言,三曰視,四曰聽,五曰思。貌曰恭,言曰從,視曰明,聽曰聰,思曰睿。恭作肅,從作艾,明作悊,聰作謀,睿作聖。休徵:曰肅,時雨若;艾,時陽若;悊,時奧若;謀,時寒若;聖,時風若。咎徵:曰狂,恆雨若;僭,恆陽若;舒,恆奧若;急,恆寒若;霿,恆風若。」

傳曰:「貌之不恭,是謂不肅,厥咎狂,厥罰恆雨,厥極惡。時則有服妖,時則有龜孽,時則有雞禍,時則有下體生上之痾,時則有青眚青祥。唯金沴水。」

說曰:凡草物之類謂之妖。妖猶夭胎,言尚微。蟲豸之類謂之孽。孽則牙孽矣。及六畜,謂之禍,言其著也。及人,謂之痾。痾,病貌,言浸深也。甚則異物生,謂之眚;自外來,謂之祥。祥猶禎也。氣相傷,謂之沴。沴猶臨蒞,不和意也。每一事云「時則」以絕之,言非必俱至,或有或亡,或在前或在後也。

孝武時,夏侯始昌通五經,善推五行傳,以傳族子夏侯勝,下及許商,皆以教所賢弟子。其傳與劉向同,唯劉歆傳獨異。貌之不恭,是謂不肅。肅,敬也。內曰恭,外曰敬。人君行己,體貌不恭,怠慢驕蹇,則不能敬萬事,失在狂易,故其咎狂也。上嫚下暴,則陰氣勝,故其罰常雨也。水傷百穀,衣食不足,則姦軌並作,故其極惡也。一曰,民多被刑,或形貌醜惡,亦是也。風俗狂慢,變節易度,則為剽輕奇怪之服,故有服妖。水類動,故有龜孽。於易,巽為雞,雞有冠距文武之貌。不為威儀,貌氣毀,故有雞禍。一曰,水歲雞多死及為怪,亦是也。上失威儀,則下有彊臣害君上者,故有下體生於上之痾。木色青,故有青眚青祥。凡貌傷者病木氣,木氣病則金沴之,衝氣相通也。於易,震在東方,為春為木也;兌在西方,為秋為金也;離在南方,為夏為火也;坎在北方,為冬為水也。春與秋,日夜分,寒暑平,是以金木之氣易以相變,故貌傷則致秋陰常雨,言傷則致春陽常旱也。至於冬夏,日夜相反,寒暑殊絕,水火之氣不得相併,故視傷常奧,聽傷常寒者,其氣然也。逆之,其極曰惡;順之,其福曰攸好德。劉歆貌傳曰有鱗蟲之孽,羊禍,鼻痾。說以為於天文東方辰為龍星,故為鱗蟲;於易兌為羊,木為金所病,故致羊禍,與常雨同應。此說非是。春與秋,氣陰陽相敵,木病金盛,故能相并,唯此一事耳。禍與妖痾祥眚同類,不得獨異。

史記成公十六年,公會諸侯于周,單襄公見晉厲公視遠步高,告公曰:「晉將有亂。」魯侯曰:「敢問天道也?抑人故也?」對曰:「吾非瞽史,焉知天道?吾見晉君之容,殆必禍者也。夫君子目以定體,足以從之,是以觀其容而知其心矣。目以處誼,足以步目。晉侯視遠而足高,目不在體,而足不步目,其心必異矣。目體不相從,何以能久?夫合諸侯,民之大事也,於是虖觀存亡。故國將無咎,其君在會,步言視聽必皆無謫,則可以知德矣。視遠,曰絕其誼;足高,曰棄其德;言爽,曰反其信;聽淫,曰離其名。夫目以處誼,足以踐德,口以庇信,耳以聽名者也,故不可不慎。偏喪有咎;既喪,則國從之。晉侯爽二,吾是以云。」後二年,晉人殺厲公。凡此屬,皆貌不恭之咎云。

左氏使桓公十三年,楚屈瑕伐羅,鬥伯比送之,還謂其馭曰:「莫囂必敗,舉止高,心不固矣。」遽見楚子以告。楚子使賴人追之,弗及。莫囂行,遂無次,且不設備。及羅,羅人軍之,大敗。莫囂縊死。

釐公十一年,周使內史過賜晉惠公命,受玉,惰。過歸告王曰:「晉侯其無後乎!王賜之命,而惰於受瑞,先自棄也已,其何繼之有!禮,國之幹也;敬,禮之輿也。不敬則禮不行,禮不行則上下昏,何以長世!」二十一年,晉惠公卒,子懷公立,晉人殺之,更立文公。

成公十三年,晉侯使郤錡乞師于魯,將事不敬。孟獻子曰:「郤氏其亡乎!禮,身之幹也;敬,身之基也。郤子無基。且先君之嗣卿也,受命以求師,將社稷是衛,而惰棄君命也,不亡何為!」十七年,郤氏亡。

成公十三年,諸侯朝王,遂從劉康公伐秦。成肅公受賑于社,不敬。劉子曰:「吾聞之曰,民受天地之中以生,所謂命也。是以有禮義動作威儀之則,以定命也。能者養以之福,不能者敗以取禍,是故君子勤禮,小人盡力。勤禮莫如致敬,盡力莫如惇篤。敬在養神,篤在守業。國之大事,在祀與戎。祀有執膰,戎有受賑,神之大節也。今成子惰,棄其命矣,其不反虖!」五月,成肅公卒。

成公十四年,衛定公享苦成叔,甯惠子相。苦成叔敖,甯子曰:「苦成家其亡虖!古之為享食也,以觀威儀省禍福也。故《詩》曰:『兕觥其觩,旨酒思柔,匪儌匪傲,萬福來求。』今夫子傲,取禍之道也。」後三年,苦成家亡。

襄公七年,衛孫文子聘于魯,君登亦登。叔孫穆子相,趨進曰:「諸侯之會,寡君未嘗後衛君,今吾子不後寡君,寡君未知所過,吾子其少安!」孫子亡辭,亦亡悛容。穆子曰:「孫子必亡。為臣而君,過而不悛,亡之本也。」十四年,孫子逐其君而外叛。

襄公二十八年,蔡景侯歸自晉,入于鄭。鄭伯享之,不敬。子產曰:「蔡君其不免虖!日其過此也,君使子展往勞于東門,而敖。吾曰:『猶將更之。』今還,受享而惰,乃其心也。君小國,事大國,而惰敖以為己心,將得死虖?君若不免,必由其子。淫而不父,如是者必有子禍。」三十年,為世子般所殺。

襄公三十一年,公薨。季武子將立公子裯,穆叔曰:「是人也,居喪而不哀,在慼而有嘉容,是謂不度。不度之人,鮮不為患,若果立,必為季氏憂。」武子弗聽,卒立之。比及葬,三易衰,衰衽如故衰。是為昭公。立二十五年,聽讒攻季氏。兵敗,出奔,死于外。

襄公三十一年,衛北宮文子見楚令尹圍之儀,言於衛侯曰:「令尹似君矣,將有它志;雖獲其志,弗能終也。」公曰:「子何以知之?」對曰:「《詩》云『敬慎威儀,惟民之則』,令尹無威儀,民無則焉。民所不則,以在民上,不可以終。

昭公十一年夏,周單子會於戚,視下言徐。晉叔向曰:「單子其死虖!朝有著定,會有表,衣有襘,帶有結。會朝之言必聞于表著之位,所以昭事序也;視不過結襘之中,所以道容貌也。言以命之,容貌以明之,失則有闕。今單子為王官伯,而命事於會,視不登帶,言不過步,貌不道容而言不昭矣。不道不恭,不昭不從,無守氣矣。」十二月,單成公卒。

昭公二十一年三月,葬蔡平公,蔡太子朱失位,位在卑。魯大夫送葬者歸告昭子。昭子歎曰:「蔡其亡虖!若不亡,是君也必不終。《詩》曰『不解於位,民之攸塈。』今始即位而適卑,身將從之。」十月,蔡侯朱出奔楚。

晉魏舒合諸侯之大夫于翟泉,將以城成周。魏子蒞政,衛彪傒曰:「將建天子,而易位以令,非誼也。大事奸誼,必有大咎。晉不失諸侯,魏子其不免虖!」是行也,魏獻子屬役於韓簡子,而田於大陸,焚焉而死。

定公十五年,邾隱公朝於魯,執玉高,其容仰。公受玉卑,其容俯。子贛觀焉,曰:「以禮觀之,二君者皆有死亡焉。夫禮,死生存亡之體也。將左右周旋,進退俯仰,於是虖取之;朝祀喪戎,於是虖觀之。今正月相朝,而皆不度,心已亡矣。嘉事不體,何以能久?高仰,驕也;卑俯,替也。驕近亂,替近疾。君為主,其先亡虖!」

庶徵之恆雨,劉歆以為春秋大雨也,劉向以為大水。

隱公九年「三月癸酉,大雨,震電;庚辰,大雨雪」。大雨,雨水也;震,雷也。劉歆以為三月癸酉,於曆數春分後一日,始震電之時也,當雨,而不當大雨。大雨,常雨之罰也。於始震電八日之間而大雨雪,常寒之罰也。劉向以為周三月,今正月也,當雨水,雪雜雨,雷電未可以發也。既已發也,則雪不當復降。皆失節,故謂之異。於易,雷以二月出,其卦日豫,言萬物隨雷出地,皆逸豫也。以八月入,其卦曰歸妹,言雷復歸。入地則孕毓根核,保藏蟄蟲,避盛陰之害;出地則養長華實,發揚隱伏,宣盛陽之德。入能除害,出能興利,人君之象也。是時,隱以弟桓幼,代而攝立。公子翬見隱居位已久,勸之遂立。隱既不許,翬懼而易其辭,遂與桓共殺隱。天見其將然,故正月大雨水而雷電。是陽不閉陰,出涉危難而害萬物。天戒若曰,為君失時,賊弟佞臣將作亂矣。後八日大雨雪,陰見間隙而勝陽,篡殺之禍將成也。公不寤,後二年而殺。

昭帝始元元年七月,大水雨,自七月至十月。成帝建始三年秋,大雨三十餘日;四年九月,大雨十餘日。

左氏傳愍公二年,晉獻公使太子申生帥師,公衣之偏衣,佩之金玦。狐突歎曰:「時,事之徵也;衣,身之章也;佩,衷之旗也。故敬其事,則命以始;服其身,則衣之純;用其衷,則佩之度。今命以時卒,閟其事也;衣以尨服,遠其躬也;佩以金玦,棄其衷也。服以遠之,時以閟之,尨涼冬殺,金寒玦離,胡可恃也!」梁餘子養曰:「帥師者,受命于廟,受脤於社,有常服矣。弗獲而尨,命可知也。死而不孝,不如逃之。」罕夷曰:「尨奇無常,金玦不復,君有心矣。」後四年,申生以讒自殺。近服妖也。

左氏傳曰,鄭子臧好聚鷸冠,鄭文公惡之,使盜殺之。劉向以為近服妖者也。一曰,非獨為子臧之身,亦文公之戒也。初,文公不禮晉文,又犯天子命而伐滑,不尊尊敬上。其後晉文伐鄭,幾亡國。

昭帝時,昌邑王賀遣中大夫之長安,多治仄注冠,以賜大臣,又以冠奴。劉向以為近服妖也。時王賀狂悖,聞天子不豫,弋獵馳騁如故,與騶奴宰人游居娛戲,驕嫚不敬。冠者尊服,奴者賤人,賀無故好作非常之冠,暴尊象也。以冠奴者,當自至尊墜至賤也。其後帝崩,無子,漢大臣徵賀為嗣。即位,狂亂無道,縛戮諫者夏侯勝等。於是大臣白皇太后,廢賀為庶人。賀為王時,又見大白狗冠方山冠而無尾,此服妖,亦犬禍也。賀以問郎中令龔遂,遂曰:「此天戒,言在仄者盡冠狗也。去之則存,不去則亡矣。」賀既廢數年,宣帝封之為列侯,復有罪,死不得置後,又犬禍無尾之效也。京房易傳曰:「行不順,厥咎人奴冠,天下亂,辟無適,妾子拜。」又曰:「君不正,臣欲篡,厥妖狗冠出朝門。」

成帝鴻嘉、永始之間,好為微行出游,選從期門郎有材力者,及私奴客,多至十餘,少五六人,皆白衣袒幘,帶持刀劍。或乘小車,御者在茵上,或皆騎,出入市里郊野,遠至旁縣。時,大臣車騎將軍王音及劉向等數以切諫。谷永曰:「易稱『得臣無家』,言王者臣天下,無私家也。今陛下棄萬乘之至貴,樂家人之賤事;厭高美之尊稱,好匹夫之卑字;崇聚票輕無誼之人,以為私客;置私田於民間,畜私奴車馬於北宮;數去南面之尊,離深宮之固,挺身獨與小人晨夜相隨,烏集醉飽吏民之家,亂服共坐,溷肴亡別,閔勉遯樂,晝夜在路。典門戶奉宿衛之臣執干戈守空宮,公卿百寮不知陛下所在,積數年矣。昔虢公為無道,有神降曰『賜爾土田』,言將以庶人受土田也。諸侯夢得土田,為失國祥,而況王者畜私田財物,為庶人之事乎!」

左氏傳曰,周景王時大夫賓起見雄雞自斷其尾。劉向以為近雞禍也。是時,王有愛子子晁,王與賓起陰謀欲立之。田于北山,將因兵眾殺適子之黨,未及而崩。三子爭國,王室大亂。其後,賓起誅死,子晁奔楚而敗。京房易傳曰:「有始無終,厥妖雄雞自齧斷其尾。」

宣帝黃龍元年,未央殿輅軨中雌雞化為雄,毛衣變化而不鳴,不將,無距。元帝初元中,丞相府史家雌雞伏子,漸化為雄,冠距鳴將。永光中,有獻雄雞生角者。京房易傳曰:「雞知時,知時者當死。」房以為己知時,恐當之。劉向以為房失雞占。雞者小畜,主司時,起居人,小臣執事為政之象也。言小臣將秉君威,以害正事,猶石顯也。竟寧元年,石顯伏辜,此其效也。一曰,石顯何足以當此?昔武王伐殷,至于牧野,誓師曰:「古人有言曰『牝雞無晨;牝雞之晨,惟家之索。』今殷王紂惟婦言用。」繇是論之,黃龍、初元、永光雞變,乃國家之占,妃后象也。孝元王皇后以甘露二年生男,立為太子。妃,王禁女也。黃龍元年,宣帝崩,太子立,是為元帝。王妃將為皇后,故是歲未央殿中雌雞為雄,明其占在正宮也。不鳴不將無距,貴始萌而尊未成也。至元帝初元元年,將立王皇后,先以為婕妤。三月癸卯制書曰:「其封婕妤父丞相少史王禁為陽平侯,位特進。」丙午,立王婕妤為皇后。明年正月,立皇后子為太子。故應是,丞相府史家雌雞為雄,其占即丞相少史之女也。伏子者,明已有子也。冠距鳴將者,尊已成也。永光二年,陽平頃侯禁薨,子鳳嗣侯,為侍中衛尉。元帝崩,皇太子立,是為成帝。尊皇后為皇太后,以后弟鳳為大司馬大將軍,領尚書事,上委政,無所與。王氏之權自鳳起,故於鳳始受爵位時,雄雞有角,明視作威顓君害上危國者,從此人始也。其後群弟世權,以至於莽,遂篡天下。即位五年,王太后乃崩,此其效也。京房易傳曰:「賢者居明夷之世,知時而傷,或眾在位,厥妖雞生角。雞生角,時主獨。」又曰:「婦人顓政,國不靜;牝雞雄鳴,主不榮。」故房以為己亦在占中矣。

成公七年「正月,鼷鼠食郊牛角;改卜牛,又食其角」。劉向以為近青祥,亦牛禍也,不敬而傋霿之所致也。昔周公制禮樂,成周道,故成王命魯郊祀天地,以尊周公。至成公時,三家始顓政,魯將從此衰。天愍周公之德,痛其將有敗亡之禍,故於郊祭而見戒云。鼠,小蟲,性盜竊,鼷又其小者也。牛,大畜,祭天尊物也。角,兵象,在上,君威也。小小鼷鼠,食至尊之牛角,象季氏乃陪臣盜竊之人,將執國命以傷君威而害周公之祀也。改卜牛,鼷鼠又食其角,天重語之也。成公怠慢昏亂,遂君臣更執于晉。至于襄公,晉為溴梁之會,天下大夫皆奪君政。其後三家逐昭公,卒死于外,幾絕周公之祀。董仲舒以為鼷鼠食郊牛,皆養牲不謹也。京房易傳曰:「祭天不慎,厥妖鼷鼠齧郊牛角。」

定公十五年「正月,鼷鼠食郊牛,牛死」。劉向以為定公知季氏逐昭公,罪惡如彼,親用孔子為夾谷之會,齊人來歸鄆、讙、龜陰之田,聖德如此,反用季桓子,淫於女樂,而退孔子,無道甚矣。《詩》曰:「人而亡儀,不死何為!」是歲五月,定公薨,牛死之應也。京房易傳曰:「子不子,鼠食其郊牛。」

哀公元年「正月,鼷鼠食郊牛」。劉向以為天意汲汲於用聖人,逐三家,故復見戒也。哀公年少,不親見昭公之事,故見敗亡之異。已而哀不寤,身奔於粵,此其效也。

昭帝元鳳元年九月,燕有黃鼠銜其尾舞王宮端門中,王往視之,鼠舞如故。王使吏以酒脯祠,鼠舞不休,一日一夜死。近黃祥,時燕剌王旦謀反將死之象也。其月,發覺伏辜。京房易傳曰:「誅不原情,厥妖鼠舞門。」

成帝建始四年九月,長安城南有鼠銜黃蒿、柏葉,上民冢柏及榆樹上為巢,桐柏尤多。巢中無子,皆有乾鼠矢數十。時議臣以為恐有水災。鼠,盜竊小蟲,夜出晝匿;今晝去穴而登木,象賤人將居顯貴之位也。桐柏,衛思后園所在也。其後,趙皇后自微賤登至尊,與衛后同類。趙后終無子而為害。明年,有鳶焚巢,殺子之異也。天象仍見,甚可畏也。一曰,皆王莽竊位之象云。京房易傳曰:「臣私祿罔辟,厥妖鼠巢。」

文公十三年,「大室屋壞」。近金沴木,木動也。先是,冬,釐公薨,十六月乃作主。後六月,又吉禘於太廟而致釐公,春秋譏之。經曰:「大事於太廟,躋釐公。」左氏說曰:太廟,周公之廟,饗有禮義者也;祀,國之大事也。惡其亂國之大事於太廟,故言大事也。躋,登也,登釐公於愍公上,逆祀也。釐雖愍之庶兄,嘗為愍臣,臣子一例,不得在愍上。又未三年而吉禘,前後亂賢父聖祖之大禮,內為貌不恭而狂,外為言不從而僭。故是歲自十二月不雨,至于秋七月。後年,若是者三,而太室屋壞矣。前堂曰太廟,中央曰太室;屋,其上重屋尊高者也,象魯自是陵夷,將墮周公之祀也。穀梁、公羊經曰,世室,魯公伯禽之廟也。周公稱太廟,魯公稱世室。大事者,祫祭也。躋釐公者,先禰後祖也。

景帝三年十二月,吳二城門自傾,大船自覆。劉向以為近金沴木,木動也。先是,吳王濞以太子死於漢,稱疾不朝,陰與楚王戊謀為逆亂。城猶國也,其一門名曰楚門,一門曰魚門。吳地以船為家,以魚為食。天戒若曰,與楚所謀,傾國覆家。吳王不寤,正月,與楚俱起兵,身死國亡。京房易傳曰:「上下咸誖,厥妖城門壞。」

宣帝時,大司馬霍禹所居第門自壞。時禹內不順,外不敬,見戒不改,卒受滅亡之誅。

哀帝時,大司馬董賢第門自壞。時賢以私愛居大位,賞賜無度,驕嫚不敬,大失臣道,見戒不改。後賢夫妻自殺,家徙合浦。

傳曰:「言之不從,是謂不艾,厥咎僭,厥罰恆陽,厥極憂。時則有詩妖,時則有介蟲之孽,時則有犬禍,時則有口舌之痾,時則有白眚白祥。惟木沴金。」

「言之不從」,從,順也。「是謂不乂」,乂,治也。孔子曰:「君子居其室,出其言不善,則千里之外違之,況其邇者虖!」《詩》云:「如蜩如螗,如沸如羹。」言上號令不順民心,虛譁憒亂,則不能治海內,失在過差,故其咎僭。僭,差也。刑罰妄加,群陰不附,則陽氣勝,故其罰常陽也。旱傷百穀,則有寇難,上下俱憂,故其極憂也。君炕陽而暴虐,臣畏刑而柑口,則怨謗之氣發於歌謠,故有詩妖。介蟲孽者,謂小蟲有甲飛揚之類,陽氣所生也,於春秋為螽,今謂之蝗,皆其類也。於易,兌為口,犬以吠守,而不可信,言氣毀故有犬禍。一曰,旱歲犬多狂死及為怪,亦是也。及人,則多病口喉欬者,故有口舌痾。金色白,故有白眚白祥。凡言傷者,病金氣;金氣病,則木沴之。其極憂者,順之,其福曰康寧,劉歆言傳曰時有毛蟲之孽。說以為於天文西方參為虎星,故為毛蟲。

史記周單襄公與晉郤錡、郤犨、郤至、齊國佐語,告魯成公曰:「晉將有亂,三郤其當之虖!夫郤氏,晉之寵人也,二卿而五大夫,可以戒懼矣。高位實疾顛,厚味實腊毒。今郤伯之語犯,叔迂,季伐。犯則陵人,迂則誣人,伐則掩人。有是寵也,而益之以三怨,其誰能忍之!雖齊國子亦將與焉。立於淫亂之國,而好盡言以招人過,怨之本也。唯善人能受盡言,齊其有虖?」十七年,晉殺三郤。十八年,齊殺國佐。凡此屬,皆言不從之咎云。

晉穆侯以條之役生太子,名之曰仇;其弟以千畝之戰生,名之曰成師。師服曰:「異哉,君之名子也!夫名以制誼,誼以出禮,禮以體政,政以正民,是以政成而民聽;易則生亂。嘉耦曰妃,怨耦曰仇,古之命也。今君名太子曰仇,弟曰成師,始兆亂矣,兄其替虖!」及仇嗣立,是為文侯。文侯卒,子昭侯立,封成師于曲沃,號桓叔。後晉人殺昭侯而納桓叔,不克。復立昭侯子孝侯,桓叔子嚴伯殺之。晉人立其弟鄂侯。鄂侯生哀侯,嚴伯子武公復殺哀侯及其弟,滅之,而代有晉國。

宣公六年,鄭公子曼滿與王子伯廖語,欲為卿。伯廖告人曰:「無德而貪,其在周易豐之離,弗過之矣。」間一歲,鄭人殺之。

襄公二十九年,齊高子容與宋司徒見晉知伯,汝齊相禮。賓出,汝齊語知伯曰:「二子皆將不免!子容專,司徒侈,皆亡家之主也。專則速及,侈將以其力敝,專則人實敝之,將及矣。」九月,高子出奔燕。

襄公三十一年正月,魯穆叔會晉歸,告孟孝伯曰:「趙孟將死矣!其語偷,不似民主;且年未盈五十,而諄諄焉如八九十者,弗能久矣。若趙孟死,為政者其韓子虖?吾子盍與季孫言之?可以樹善,君子也。」孝伯曰:「民生幾何,誰能毋偷!朝不及夕,將焉用樹!」穆叔告人曰:「孟孫將死矣!吾語諸趙孟之偷也,而又甚焉。」九月,孟孝伯卒。

昭公元年,周使劉定公勞晉趙孟,因曰:「子弁冕以臨諸侯,盍亦遠績禹功,而大庇民乎?」對曰:「老夫罪戾是懼,焉能恤遠?吾儕偷食,朝不謀夕,何其長也?」劉子歸,以語王曰:「諺所謂老將知而耄及之者,其趙孟之謂虖!為晉正卿以主諸侯,而儕于隸人,朝不謀夕,棄神人矣。神怒民畔,何以能久?趙孟不復年矣!」是歲,秦景公弟后子奔晉,趙孟問:「秦君何如?」對曰:「無道。」趙孟曰:「亡虖?」對曰:「何為?一世無道,國未艾也。國于天地,有與立焉,不數世淫,弗能敝也。」趙孟曰:「天虖?」對曰:「

有焉。」趙孟曰:「其幾何?」對曰:「鍼聞國無道而年穀和孰,天贊之也,鮮不五稔。」趙孟視蔭,曰:「朝夕不相及,誰能待五?」后子出而告人曰:「趙孟將死矣!主民玩歲而愒日,其與幾何?」冬,趙孟卒。昭五年,秦景公卒。

昭公元年,楚公子圍會盟,設服離衛。魯叔孫穆子曰:「楚公子美矣君哉!」伯州犁曰:「此行也,辭而假之寡君。」鄭行人子羽曰:「假不反矣。」伯州犁曰:「子姑憂子晢之欲背誕也。」子羽曰:「假而不反,子其無憂虖?」齊國子曰:「吾代二子閔矣。」陳公子招曰:「不憂何成?二子樂矣!」衛齊子曰:「苟或知之,雖憂不害。」退會,子羽告人曰:「齊、衛、陳大夫其不免乎!國子代人憂,子招樂憂,齊子雖憂弗害。夫弗及而憂,與可憂而樂,與憂而弗害,皆取憂之道也。太誓曰:『民之所欲,天必從之。』三大夫兆憂矣,能無至乎!言以知物,其是之謂矣。」

昭公十五年,晉籍談如周葬穆后,既除喪而燕,王曰:「諸侯皆有以填撫王室,晉獨無有,何也?」籍談對曰:「諸侯之封也,皆受明器於王室,故能薦彝器。晉居深山,戎翟之與鄰,拜戎不暇,其何以獻器?」王曰:「叔氏其忘諸乎!叔父唐叔,成王之母弟,其反亡分乎?昔而高祖司晉之典籍,以為大正,故曰籍氏。女,司典之後也,何故忘之?」籍談不能對。賓出,王曰:「籍父其無後乎!數典而忘其祖。」籍談歸,以語叔嚮。叔嚮曰:「王其不終乎!吾聞所樂必卒焉。今王樂憂,若卒以憂,不可謂終。王一歲而有三年之喪二焉,於是乎以喪賓燕,又求彝器,樂憂甚矣。三年之喪,雖貴遂服,禮也。王雖弗遂,燕樂已早。禮,王之大經也;一動而失二禮,無大經矣。言以考典,典以志經。忘經而多言舉典,將安用之!」

哀公十六年,孔丘卒,公誄之曰:「昊天不弔,不憖遺一老,俾屏予一人。」子贛曰:「君其不歿於魯乎?夫子之言曰:『禮失則昏,名失則愆。』失志為昏,失所謂愆。生弗能用,死而誄之,非禮也;稱『予一人』,非名也。君兩失之。」二十七年,公孫于邾,遂死於越。

庶徵之恆陽,劉向以為春秋大旱也。其夏旱雩祀,謂之大雩。不傷二穀,謂之不雨。京房易傳曰:「欲德不用茲謂張,厥災荒。荒,旱也,其旱陰雲不雨,變而赤,因而除。師出過時茲謂廣,其旱不生。上下皆蔽茲謂隔,其旱天赤三月,時有雹殺飛禽。上緣求妃茲謂僭,其旱三月大溫亡雲。居高臺府,茲謂犯陰侵陽,其旱萬物根死,數有火災。庶位踰節茲謂僭,其旱澤物枯,為火所傷。」

釐公二十一年「夏,大旱」。董仲舒、劉向以為齊威既死,諸侯從楚,釐尤得楚心。楚來獻捷,釋宋之執。外倚彊楚,炕陽失眾,又作南門,勞民興役。諸雩旱不雨,略皆同說。

宣公七年「秋,大旱」。是夏,宣與齊侯伐萊。

襄公五年「秋,大雩」。先是宋魚石奔楚,楚伐宋,取彭城以封魚石。鄭畔于中國而附楚,襄與諸侯共圍彭城,城鄭虎牢以禦楚。是歲鄭伯使公子發來聘,使大夫會吳于善道。外結二國,內得鄭聘,有炕陽動眾之應。

八年「九月,大雩」。時作三軍,季氏盛。

二十八年「八月,大雩」。先是,比年晉使荀吳、齊使慶封來聘,是夏邾子來朝。襄有炕陽自大之應。

昭公三年「八月,大雩」。劉歆以為昭公即位年十九矣,猶有童心,居喪不哀,炕陽失眾。

六年「九月,大雩」。先是莒牟夷以二邑來奔,莒怒伐魯,叔弓帥師,距而敗之,昭得入晉。外和大國,內獲二邑,取勝鄰國,有炕陽動眾之應。

十六年「九月,大雩」。先是昭公母夫人歸氏薨,昭不慼,又大蒐于比蒲。晉叔嚮曰:「魯有大喪而不廢蒐。國不恤喪,不忌君也;君亡慼容,不顧親也。殆其失國。」與三年同占。

二十四年「八月,大雩」。劉歆以為左氏傳二十三年邾師城翼,還經魯地,魯襲取邾師,獲其三大夫。邾人愬于晉,晉人執我行人叔孫婼,是春乃歸之。

二十五年「七月上辛大雩,季辛又雩」,旱甚也。劉歆以為時后氏與季氏有隙。又季氏之族有淫妻為讒,使季平子與族人相惡,皆共譖平子。子家駒諫曰:「讒人以君徼幸,不可。」昭公遂伐季氏,為所敗,出奔齊。

定公十年「九月,大雩」。先是定公自將侵鄭,歸而城中城。二大夫帥師圍鄆。

嚴公三十一年「冬,不雨」。是歲,一年而三築臺,奢侈不恤民。

釐公二年「冬十月不雨」,三年「春正月不雨,夏四月不雨」,「六月雨」。先是者,嚴公夫人與公子慶父淫,而殺二君。國人攻之,夫人遜于邾,慶父奔莒。釐公即位,南敗邾,東敗莒,獲其大夫。有炕陽之應。

文公二年,「自十有二月不雨,至于秋七月」。文公即位,天子使叔服會葬,毛伯賜命。又會晉侯于戚。公子遂如齊納幣。又與諸侯盟。上得天子,外得諸侯,沛然自大。躋釐公主。大夫始顓事。

十年,「自正月不雨,至于秋七月」。先是公子遂會四國而救鄭。楚使越椒來聘。秦人歸襚。有炕陽之應。

十三年,「自正月不雨,至于秋七月」。先是曹伯、杞伯、滕子來朝,郕伯來奔,秦伯使遂來聘,季孫行父城諸及鄆。二年之間,五國趨之,內城二邑。炕陽失眾。一曰,不雨而五穀皆孰,異也。文公時,大夫始顓盟會,公孫敖會晉侯,又會諸侯盟于垂隴。故不雨而生者,陰不出氣而私自行,以象施不由上出,臣下作福而私自成。一曰,不雨近常陰之罰,君弱也。

惠帝五年夏,大旱,江河水少,谿谷絕。先是發民男女十四萬六千人城長安,是歲城乃成。

文帝三年秋,天下旱。是歲夏,匈奴右賢王寇侵上郡,詔丞相灌嬰發車騎士八萬五千人詣高奴,擊右賢王走出塞。其秋,濟北王興居反,使大將軍討之,皆伏誅。

後六年春,天下大旱。先是發車騎材官屯廣昌,是歲二月復發材官屯隴西。後匈奴大入上郡、雲中,烽火通長安,三將軍屯邊,又三將軍屯京師。

景帝中三年秋,大旱。

武帝元光六年夏,大旱。是歲,四將軍征匈奴。

元朔五年春,大旱。是歲,六將軍眾十餘萬征匈奴。

元狩三年夏,大旱。是歲發天下故吏伐棘上林,穿昆明池。

天漢元年夏,大旱;其三年夏,大旱。先是貳師將軍征大宛還。天漢元年,發適民。二年夏,三將軍征匈奴,李陵沒不還。

征和元年夏,大旱。是歲發三輔騎士閉長安城門,大搜,始治巫蠱。明年,衛皇后、太子敗。

昭帝始元六年,大旱。先是大鴻臚田廣明征益州,暴師連年。

宣帝本始三年夏,大旱,東西數千里。先是五將軍眾二十萬征匈奴。

神爵元年秋,大旱。是歲,後將軍趙充國征西羌。

成帝永始三年、四年夏,大旱。

左氏傳晉獻公時童謠曰:「丙之晨,龍尾伏辰,袀服振振,取虢之旂。鶉之賁賁,天策焞焞,火中成軍,虢公其奔。」是時虢為小國,介夏陽之阨,怙虞國之助,亢衡于晉,有炕陽之節,失臣下之心。晉獻伐之,問於卜偃曰:「吾其濟乎?」偃以童謠對曰:「克之。十月朔丙子旦,日在尾,月在策,鶉火中,必此時也。」冬十二月丙子朔,晉師滅虢,虢公醜奔周。周十二月,夏十月也。言天者以夏正。

史記晉惠公時童謠曰:「恭太子更葬兮,後十四年,晉亦不昌,昌乃在其兄。」是時,惠公賴秦力得立,立而背秦,內殺二大夫,國人不說。及更葬其兄恭太子申生而不敬,故詩妖作也。後與秦戰,為秦所獲,立十四年而死。晉人絕之,更立其兄重耳,是為文公,遂伯諸侯。

左氏傳文、成之世童謠曰:「鴝之鵒之,公出辱之。鴝鵒之羽,公在外野,往饋之馬。鴝鵒跦跦,公在乾侯,徵褰與襦。鴝鵒之巢,遠哉搖搖,裯父喪勞,宋父以驕。鴝鵒鴝鵒,往歌來哭。」至昭公時,有鴝鵒來巢。公攻季氏,敗,出奔齊,居外野,次乾侯。八年,死于外,歸葬魯。昭公名裯。公子宋立,是為定公。

元帝時童謠曰:「井水溢,滅灶煙,灌玉堂,流金門。」至成帝建始二年三月戊子,北宮中井泉稍上,溢出南流,象春秋時先有鴝鵒之謠,而後有來巢之驗。井水,陰也;灶煙,陽也;玉堂、金門,至尊之居:象陰盛而滅陽,竊有宮室之應也。王莽生於元帝初元四年,至成帝封侯,為三公輔政,因以篡位。

成帝時童謠曰:「燕燕尾龚龚,張公子,時相見。木門倉琅根,燕飛來,啄皇孫,皇孫死,燕啄矢。」其後帝為微行出遊,常與富平侯張放俱稱富平侯家人,過河陽主作樂,見舞者趙飛燕而幸之,故曰「燕燕尾龚龚」,美好貌也。張公子謂富平侯也。「木門倉琅根」,謂宮門銅鍰,言將尊貴也。後遂立為皇后。弟昭儀賊害後宮皇子,卒皆伏辜,所謂「燕飛來,啄皇孫,皇孫死,燕啄矢」者也。

成帝時歌謠又曰:「邪徑敗良田,讒口亂善人。桂樹華不實,黃爵巢其顛。故為人所羨,今為人所憐。」桂,赤色,漢家象。華不實,無繼嗣也。王莽自謂黃象,黃爵巢其顛也。

嚴公十七年「冬,多麋」。劉歆以為毛蟲之孽為災。劉向以為麋色青,近青祥也。麋之為言迷也,蓋牝獸之淫者也。是時,嚴公將取齊之淫女,其象先見,天戒若曰,勿取齊女,淫而迷國。嚴不寤,遂取之。夫人既入,淫於二叔,終皆誅死,幾亡社稷。董仲舒指略同。京房易傳曰:「廢正作淫,大不明,國多麋。」又曰:「震遂泥,厥咎國多麋。」

昭帝時,昌邑王賀聞人聲曰「熊」,視而見大熊。左右莫見,以問郎中令龔遂,遂曰:「熊,山野之獸,而來入宮室,王獨見之,此天戒大王,恐宮室將空,危亡象也。」賀不改寤,後卒失國。

左氏傳襄公十七年十一月甲午,宋國人逐狾狗,狾狗入於華臣氏,國人從之。臣懼,遂奔陳。先是臣兄閱為宋卿,閱卒,臣使賊殺閱家宰,遂就其妻。宋平公聞之,曰:「臣不唯其宗室是暴,大亂宋國之政。」欲逐之。左師向戌曰:「大臣不順,國之恥也,不如蓋之。」公乃止。華臣炕暴失義,內不自安,故犬禍至,以奔亡也。

高后八年三月,祓霸上,還過枳道,見物如倉狗,橶高后掖,忽而不見。卜之,趙王如意為祟。遂病掖傷而崩。先是高后鴆殺如意,支斷其母戚夫人手足,搉其服以為人彘。

文帝後五年六月,齊雍城門外有狗生角。先是帝兄齊悼惠王亡後,帝分齊地,立其庶子七人皆為王。兄弟並彊,有炕陽心,故犬禍見也。犬守御,角兵象,在前而上鄉者也。犬不當生角,猶諸侯不當舉兵鄉京師也。天之戒人蚤矣,諸侯不寤。後六年,吳、楚畔,濟南、膠西、膠東三國應之,舉兵至齊。齊王猶與城守,三國圍之。會漢破吳、楚,因誅四王。故天狗下梁而吳、楚攻梁,狗生角於齊而三國圍齊。漢卒破吳、楚於梁,誅四王於齊。京房易傳曰:「執政失,下將害之,厥妖狗生角。君子苟免,小人陷之,厥妖狗生角。」

景帝三年二月,邯鄲狗與彘交。悖亂之氣,近犬豕之禍也。是時趙王遂悖亂,與吳、楚謀為逆,遣使匈奴求助兵,卒伏其辜。犬,兵革失眾之占;豕,北方匈奴之象。逆言失聽,交於異類,以生害也。京房易傳曰:「夫婦不嚴,厥妖狗與豕交。茲謂反德,國有兵革。」

成帝河平元年,長安男子石良、劉音相與同居,有如人狀在其室中,擊之,為狗,走出。去後有數人被甲持兵弩至良家,良等格擊,或死或傷,皆狗也。自二月至六月乃止。

鴻嘉中,狗與彘交。

左氏昭公二十四年十月癸酉,王子晁以成周之寶圭湛于河,幾以獲神助。甲戌,津人得之河上,陰不佞取將賣之,則為石。是時王子晁篡天子位,萬民不鄉,號令不從,故有玉變,近白祥也。癸酉入而甲戌出,神不享之驗云。玉化為石,貴將為賤也。後二年,子晁奔楚而死。

史記秦始皇帝三十六年,鄭客從關東來,至華陰,望見素車白馬從華山上下,知其非人,道住止而待之。遂至,持璧與客曰:「為我遺鎬池君。」因言「今年祖龍死」。忽不見。鄭客奉璧,即始皇二十八年過江所湛璧也。與周子晁同應。是歲,石隕于東郡,民或刻其石曰:「始皇死而地分。」此皆白祥,炕陽暴虐,號令不從,孤陽獨治,群陰不附之所致也。一曰,石,陰類也,陰持高節,臣將危君,趙高、李斯之象也。始皇不畏戒自省,反夷滅其旁民,而燔燒其石。是歲始皇死,後三年而秦滅。

孝昭元鳳三年正月,泰山萊蕪山南匈匈有數千人聲。民視之,有大石自立,高丈五尺,大四十八圍,入地深八尺,三石為足。石立處,有白烏數千集其旁。眭孟以為石陰類,下民象,泰山岱宗之嶽,王者易姓告代之處,當有庶人為天子者。孟坐伏誅。京房易傳曰:「『

復,崩來無咎。』自上下者為崩,厥應泰山之石顛而下,聖人受命人君虜。」又曰:「石立如人,庶士為天下雄。立於山,同姓;平地,異姓。立於水,聖人;於澤,小人。」

天漢元年三月,天雨白毛;三年八月,天雨白氂。京房易傳曰:「前樂後憂,厥妖天雨羽。」又曰:「邪人進,賢人逃,天雨毛。」

史記周威烈王二十三年,九鼎震。金震,木動之也。是時周室衰微,刑重而虐,號令不從,以亂金氣。鼎者,宗廟之寶器也。宗廟將廢,寶鼎將遷,故震動也。是歲晉三卿韓、魏、趙篡晉君而分其地,威烈王命以為諸侯。天子不恤同姓,而爵其賊臣,天下不附矣。後三世,周致德祚於秦。其後秦遂滅周,而取九鼎。九鼎之震,木沴金,失眾甚。

成帝元延元年正月,長安章城門門牡自亡,函谷關次門牡亦自亡。京房易傳曰:「飢而不損茲謂泰,厥災水,厥咎牡亡。」妖辭曰:「關動牡飛,辟為亡道臣為非,厥咎亂臣謀篡。」故谷永對曰:「章城門通路寢之路,函谷關距山東之險,城門關守國之固,固將去焉,故牡飛也。」

傳曰:「視之不明,是謂不悊,厥咎舒,厥罰恆奧,厥極疾。時則有草妖,時則有蠃蟲之孽,時則有羊禍,時則有目痾,時則有赤眚赤祥。惟水沴火。」

「視之不明,是謂不悊」,悊,知也。《詩》云:「爾德不明,以亡陪亡卿;不明爾德,以亡背亡仄。」言上不明,暗昧蔽惑,則不能知善惡,親近習,長同類,亡功者受賞,有罪者不殺,百官廢亂,失在舒緩,故其咎舒也。盛夏日長,暑以養物,政弛緩,故其罰常奧也。奧則冬溫,春夏不和,傷病民人,故極疾也。誅不行則霜不殺草,繇臣下則殺不以時,故有草妖。凡妖,貌則以服,言則以詩,聽則以聲。視則以色者,五色物之大分也,在於眚祥,故聖人以為草妖,失秉之明者也。溫奧生蟲,故有蠃蟲之孽,謂螟螣之類當死不死,未當生而生,或多於故而為災也。劉歆以為屬思心不容。於易,剛而包柔為離,離為火為目。羊上角下號,剛而包柔,羊大目而不精明,視氣毀故有羊禍。一曰,暑歲羊多疫死,及為怪,亦是也。及人,則多病目者,故有目痾。火色赤,故有赤眚赤祥。凡視傷者病火氣,火氣傷則水沴之。其極疾者,順之,其福曰壽。劉歆視傳曰有羽蟲之孽,雞禍。說以為於天文南方喙為鳥星,故為羽蟲;禍亦從羽,故為雞;雞於易自在巽。說非是。庶徵之恆奧,劉向以為春秋亡冰也。小奧不書,無冰然後書,舉其大者也。京房易傳曰:「祿不遂行茲謂欺,厥咎奧,雨雪四至而溫。臣安祿樂逸茲謂亂,奧而生蟲。知罪不誅茲謂舒,其奧,夏則暑殺人,冬則物華實。重過不誅,茲謂亡徵,其咎當寒而奧六日也。」

桓公十五年「春,亡冰」。劉向以為周春,今冬也。先是連兵鄰國,三戰而再敗也,內失百姓,外失諸侯,不敢行誅罰,鄭伯突篡兄而立,公與相親,長養同類,不明善惡之罰也。董仲舒以為象夫人不正,陰失節也。

成公元年「二月,無冰」。董仲舒以為方有宣公之喪,君臣無悲哀之心,而炕陽,作丘甲。劉向以為時公幼弱,政舒緩也。

襄公二十八年「春,無冰」。劉向以為先是公作三軍,有侵陵用武之意,於是鄰國不和,伐其三鄙,被兵十有餘年,因之以饑饉,百姓怨望,臣下心離,公懼而弛緩,不敢行誅罰,楚有夷狄行,公有從楚心,不明善惡之應。董仲舒指略同。一曰,水旱之災,寒暑之變,天下皆同,故曰「無冰」,天下異也。桓公殺兄弒君,外成宋亂,與鄭易邑,背畔周室。成公時,楚橫行中國,王札子殺召伯、毛伯,晉敗天子之師于貿戎,天子皆不能討。襄公時,天下諸侯之大夫皆執國權,君不能制。漸將日甚,善惡不明,誅罰不行。周失之舒,秦失之急,故周衰亡寒歲,秦滅亡奧年。

武帝元狩六年冬,亡冰。先是,比年遣大將軍衛青、霍去病攻祁連,絕大幕,窮追單于,斬首十餘萬級,還,大行慶賞。乃閔海內勤勞,是歲遣博士褚大等六人持節巡行天下,存賜鰥寡,假與乏困,舉遺逸獨行君子詣行在所。郡國有以為便宜者,上丞相、御史以聞。天下咸喜。

昭帝始元二年冬,亡冰。是時上年九歲,大將軍霍光秉政,始行寬緩,欲以說下。

僖公三十三年「十二月,隕霜不殺草」。劉歆以為草妖也。劉向以為今十月,周十二月。於易,五為天位,為君位,九月陰氣至,五通於天位,其卦為剝,剝落萬物,始大殺矣,明陰從陽命,臣受君令而後殺也。今十月隕霜而不能殺草,此君誅不行,舒緩之應也。是時公子遂顓權,三桓始世官,天戒若曰,自此之後,將皆為亂矣。文公不寤,其後遂殺子赤,三家逐昭公。董仲舒指略同。京房易傳曰:「臣有緩茲謂不順,厥異霜不殺也。」

書序曰:「伊涉相太戊,亳有祥桑穀共生。」傳曰:「俱生乎朝,七日而大拱。伊陟戒以修德,而木枯。」劉向以為殷道既衰,高宗承敝而起,盡涼陰之哀,天下應之,既獲顯榮,怠於政事,國將危亡,故桑穀之異見。桑猶喪也,穀猶生也,殺生之秉失而在下,近草妖也。一曰,野木生朝而暴長,小人將暴在大臣之位,危亡國家,象朝將為虛之應也。

書序又曰:「高宗祭成湯,有蜚雉登鼎耳而雊。」祖己曰:「惟先假王,正厥事。」劉向以為雉雊鳴者雄也,以赤色為主。於易,離為雉,雉,南方,近赤祥也。劉歆以為羽蟲之孽。易有鼎卦,鼎,宗廟之器,主器奉宗廟者長子也。野鳥自外來,入為宗廟器主,是繼嗣將易也。一曰,鼎三足,三公象,而以耳行。野鳥居鼎耳,小人將居公位,敗宗廟之祀。野木生朝,野鳥入廟,敗亡之異也。武丁恐駭,謀於忠賢,修德而正事,內舉傅說,授以國政,外伐鬼方,以安諸夏,故能攘木鳥之妖,致百年之壽,所謂「六沴作見,若是共御,五福乃降,用章于下」者也。一曰,金沴木曰木不曲直。

僖公三十三年「十二月,李梅實」。劉向以為周十二月,今十月也,李梅當剝落,今反華實,近草妖也。先華而後實,不書華,舉重者也。陰成陽事,象臣顓君作威福。一曰,冬當殺,反生,象驕臣當誅,不行其罰也。故冬華華者,象臣邪謀有端而不成,至於實,則成矣。是時僖公死,公子遂顓權,文公不寤,後有子赤之變。一曰,君舒緩甚,奧氣不臧,則華實復生。董仲舒以為李梅實,臣下彊也。記曰:「不當華而華,易大夫;不當實而實,易相室。」冬,水王,木相,故象大臣。劉歆以為庶徵皆以蟲為孽,思心蠃蟲孽也。李梅實,屬草妖。

惠帝五年十月,桃李華,棗實。昭帝時,上林苑中大柳樹斷仆地,一朝起立,生枝葉,有蟲食其葉,成文字,曰「公孫病已立」。又昌邑王國社有枯樹復生枝葉。眭孟以為木陰類,下民象,當有故廢之家公孫氏從民間受命為天子者。昭帝富於春秋,霍光秉政,以孟妖言,誅之。後昭帝崩,無子,徵昌邑王賀嗣位,狂亂失道,光廢之,更立昭帝兄衛太子之孫,是為宣帝。帝本名病已。京房易傳曰:「枯楊生稊,枯木復生,人君亡子。」

元帝初元四年,皇后曾祖父濟南東平陵王伯墓門梓柱卒生枝葉,上出屋。劉向以為王氏貴盛將代漢家之象也。後王莽篡位,自說之曰:「初元四年,莽生之歲也,當漢九世火德之厄,而有此祥興於高祖考之門。門為開通,梓猶子也,言王氏當有賢子開通祖統,起於柱石大臣之位,受命而王之符也。」

建昭五年,兗州刺史浩賞禁民私所自立社。山陽橐茅鄉社有大槐樹,吏伐斷之,其夜樹復立其故處。成帝永始元年二月,河南街郵樗樹生支如人頭,眉目須皆具,亡髮耳。哀帝建平三年十月,汝南西平遂陽鄉柱仆地,生支如人形,身青黃色,面白,頭有务髮,稍長大,凡長六寸一分。京房易傳曰:「王德衰,下人將起,則有木生為人狀。」

哀帝建平三年,零陵有樹僵地,圍丈六尺,長十丈七尺。民斷其本,長九尺餘,皆枯。三月,樹卒自立故處。京房易傳曰:「棄正作淫,厥妖木斷自屬。妃后有顓,木仆反立,斷枯復生。天辟惡之。」

元帝永光二年八月,天雨草,而葉相摎結,大如彈丸。平帝元始三年正月,天雨草,狀如永光時。京房易傳曰:「君吝於祿,信衰賢去,厥妖天雨草。」

昭公二十五年「夏,有鴝鵒來巢」。劉歆以為羽蟲之孽,其色黑,又黑祥也,視不明聽不聰之罰也。劉向以為有蜚有蛊不言來者,氣所生,所謂眚也;鴝鵒言來者,氣所致,所謂祥也。鴝鵒,夷狄穴藏之禽,來至中國,不穴而巢,陰居陽位,象季氏將逐昭公,去宮室而居外野也。鴝鵒白羽,旱之祥也;穴居而好水,黑色,為主急之應也。天戒若曰,既失眾,不可急暴;急暴,陰將持節陽以逐爾,去宮室而居外野矣。昭不寤,而舉兵圍季氏,為季氏所敗,出奔于齊,遂死于外野。董仲舒指略同。

景帝三年十一月,有白頸烏與黑烏群鬥楚國呂縣,白頸不勝,墮泗水中,死者數千。劉向以為近白黑祥也。時楚王戊暴逆無道,刑辱申公,與吳王謀反。烏群鬥者,師戰之象也。白頸者小,明小者敗也。墮於水者,將死水地。王戊不寤,遂舉兵應吳,與漢大戰,兵敗而走,至於丹徒,為越人所斬,墮死於水之效也。京房易傳曰:「逆親親,厥妖白黑烏鬥於國。」

昭帝元鳳元年,有烏與鵲鬥燕王宮中池上,烏墮池死,近黑祥也。時燕王旦謀為亂,遂不改寤,伏辜而死。楚、燕皆骨肉藩臣,以驕怨而謀逆,俱有烏鵲鬥死之祥,行同而占合,此天人之明表也。燕一烏鵲鬥於宮中而黑者死,楚以萬數鬥於野外而白者死,象燕陰謀未發,獨王自殺於宮,故一烏水色者死,楚炕陽舉兵,軍師大敗於野,故眾烏金色者死,天道精微之效也。京房易傳曰:「專征劫殺,厥妖烏鵲鬥。」

昭帝時有鵜鶘或曰禿鶖,集昌邑王殿下,王使人射殺之。劉向以為水鳥色青,青祥也。時王馳騁無度,慢侮大臣,不敬至尊,有服妖之象,故青祥見也。野鳥入處,宮室將空。王不寤,卒以亡。京房易傳曰:「辟退有德,厥咎狂,厥妖水鳥集于國中。」

成帝河平元年二月庚子,泰山山桑谷有觏焚其巢。男子孫通等聞山中群鳥觏鵲聲,往視,見巢萝,盡墮地中,有三觏鷇燒死。樹大四圍,巢去地五丈五尺。太守平以聞。觏色黑,近黑祥,貪虐之類也。《易》曰:「鳥焚其巢,旅人先笑後號咷。」泰山,岱宗,五嶽之長,王者易姓告代之處也。天戒若曰,勿近貪虐之人,聽其賊謀,將生焚巢自害其子絕世易姓之禍。其後趙蜚燕得幸,立為皇后,弟為昭儀,姊妹專寵,聞後宮許美人、曹偉能生皇子也,昭儀大怒,令上奪取而殺之,皆并殺其母。成帝崩,昭儀自殺,事乃發覺,趙后坐誅。此焚巢殺子後號咷之應也。一曰,王莽貪虐而任社稷之重,卒成易姓之禍云。京房易傳曰:人君暴虐,鳥焚其舍。」

鴻嘉二年三月,博士行大射禮,有飛雉集于庭,歷階登堂而雊。後雉又集太常、宗正、丞相、御史大夫、大司馬車騎將軍之府,又集未央宮承明殿屋上。時大司馬車騎將軍王音、待詔寵等上言:「天地之氣,以類相應,譴告人君,甚微而著。雉者聽察,先聞雷聲,故月令以紀氣。經載高宗雊雉之異,以明轉禍為福之驗。今雉以博士行禮之日大眾聚會,飛集於庭,歷階登堂,萬眾睢睢,驚怪連日。徑歷三公之府,太常宗正典宗廟骨肉之官,然後入宮。其宿留告曉人,具備深切,雖人道相戒,何以過是!」後帝使中常侍晁閎詔音曰:「聞捕得雉,毛羽頗摧折,類拘執者,得無人為之?」音復對曰:「陛下安得亡國之語?不知誰主為佞諂之計,誣亂聖德如此者!左右阿諛甚眾,不待臣音復諂而足。公卿以下,保位自守,莫有正言。如令陛下覺寤,懼大禍且至身,深責臣下,繩以聖法,臣音當先受誅,豈有以自解哉!今即位十五年,繼嗣不立,日日駕車而出,泆行流聞,海內傳之,甚於京師。外有微行之害,內有疾病之憂,皇天數見災異,欲人變更,終已不改。天尚不能感動陛下,臣子何望?獨有極言待死,命在朝暮而已。如有不然,老母安得處所,尚何皇太后之有!高祖天下當以誰屬乎!宜謀於賢知,克己復禮,以求天意,繼嗣可立,災變尚可銷也。」

成帝綏和二年三月,天水平襄有燕生爵,哺食至大,俱飛去。京房易傳曰:「賊臣在國,厥咎燕生爵,諸侯銷。」一曰,生非其類,子不嗣世。

史記魯定公時,季桓子穿井,得土缶,中得蟲若羊,近羊禍也。羊者,地上之物,幽於土中,象定公不用孔子而聽季氏,暗昧不明之應也。一曰,羊去野外而拘土缶者,象魯君失其所而拘於季氏,季氏亦將拘於家臣也。是歲季氏家臣陽虎囚季桓子。後三年,陽虎劫公伐孟氏,兵敗,竊寶玉大弓而出亡。

左氏傳魯襄公時,宋有生女子赤而毛,棄之隄下,宋平公母共姬之御者見而收之,因名曰棄。長而美好,納之平公,生子曰佐。後宋臣伊戾讒太子痤而殺之。先是,大夫華元出奔晉,華弱奔魯,華臣奔陳,華合比奔衛。劉向以為時則火災赤眚之明應也。京房易傳曰:「尊卑不別,厥妖女生赤毛。」

惠帝二年,天雨血於宜陽,一頃所,劉向以為赤眚也。時又冬雷,桃李華,常奧之罰也。是時政舒緩,諸呂用事,讒口妄行,殺三皇子,建立非嗣,及不當立之王,退王陵、趙堯、周昌。呂太后崩,大臣共誅滅諸呂,僵尸流血。京房易傳曰:「歸獄不解,茲謂追非,厥咎天雨血;茲謂不親,民有怨心,不出三年,無其宗人。」又曰:「佞人祿,功臣僇,天雨血。」

哀帝建平四年四月,山陽湖陵雨血,廣三尺,長五尺,大者如錢,小者如麻子。後二年,帝崩,王莽擅朝,誅貴戚丁、傅,大臣董賢等皆放徙遠方,與諸呂同眾。誅死者少,雨血亦少。

傳曰:「聽之不聰,是謂不謀,厥咎急,厥罰恆寒,厥極貧。時則有鼓妖,時則有魚孽,時則有豕禍,時則有耳痾,時則有黑眚黑祥。惟火沴水。」

「聽之不聰,是謂不謀」,言上偏聽不聰,下情隔塞,則不能謀慮利害,失在嚴急,故其咎急也。盛冬日短,寒以殺物,政促迫,故其罰常寒也。寒則不生百穀,上下俱貧,故其極貧也。君嚴猛而閉下,臣戰栗而塞耳,則妄聞之氣發於音聲,故有鼓妖。寒氣動,故有魚孽。雨以龜為孽,龜能陸處,非極陰也;魚去水而死,極陰之孽也。於易坎為豕,豕大耳而不聰察,聽氣毀,故有豕禍也。一曰,寒歲豕多死,及為怪,亦是也。及人,則多病耳者,故有耳痾。水色黑,故有黑眚黑祥。凡聽傷者病水氣,水氣病則火沴之。其極貧者,順之,其福曰富。劉歆聽傳曰有介蟲孽也,庶徵之恆寒。劉向以為春秋無其應,周之末世舒緩微弱,政在臣下,奧煖而已,故籍秦以為驗。秦始皇帝即位尚幼,委政太后,太后淫於呂不韋及嫪毐,封毐為長信侯,以太原郡為毐國,宮室苑囿自恣,政事斷焉。故天冬雷,以見陽不禁閉,以涉危害,舒奧迫近之變也。始皇既冠,毐懼誅作亂,始皇誅之,斬首數百級,大臣二十人,皆車裂以徇,夷滅其宗,遷四千餘家於房陵。是歲四月,寒,民有凍死者。數年之間,緩急如此,寒奧輒應,此其效也。劉歆以為大雨雪,及未當雨雪而雨雪,及大雨雹,隕霜殺叔草,皆常寒之罰也。劉向以為常雨屬貌不恭。京房易傳曰:「有德遭險,茲謂逆命,厥異寒。誅過深,當奧而寒,盡六日,亦為雹。害正不誅,茲謂養賊,寒七十二日,殺蜚禽。道人始去茲謂傷,其寒物無霜而死,涌水出。戰不量敵,茲謂辱命,其寒雖雨物不茂。聞善不予,厥咎聾。」

桓公八年「十月,雨雪」。周十月,今八月也,未可以雪,劉向以為時夫人有淫齊之行,而桓有妒媚之心,夫人將殺,其象見也。桓不覺寤,後與夫人俱如齊而殺死。凡雨,陰也,雪又雨之陰也,出非其時,迫近象也。董仲舒以為象大人專恣,陰氣盛也。

釐公十年「冬,大雨雪」。劉向以為先是釐公立妾為夫人,陰居陽位,陰氣盛也。公羊經曰「大雨雹」。董仲舒以為公脅於齊桓公,立妾為夫人,不敢進群妾,故專壹之象見諸雹,皆為有所漸脅也,行專壹之政云。

昭公四年「正月,大雨雪」。劉向以為昭取於吳而為同姓,謂之吳孟子。君行於上,臣非於下。又三家已彊,皆賤公行,慢臢之心生。董仲舒以為季孫宿任政,陰氣盛也。

文帝四年六月,大雨雪。後三歲,淮南王長謀反,發覺,遷,道死。京房易傳曰:「夏雨雪,戒臣為亂。」

景帝中六年三月,雨雪。其六月,匈奴入上郡取苑馬,吏卒戰死者二千餘人。明年,條侯周亞夫下獄死。

武帝元狩元年十二月,大雨雪,民多凍死。是歲淮南、衡山王謀反,發覺,皆自殺。使者行郡國,治黨與,坐死者數萬人。

元鼎二年三月,雪,平地厚五尺。是歲御史大夫張湯有罪自殺,丞相嚴青翟坐與三長史謀陷湯,青翟自殺,三長史皆棄市。

元鼎三年三月水冰,四月雨雪,關東十餘郡人相食。是歲,民不占緡錢有告者,以半畀之。

元帝建昭二年十一月,齊楚地大雪,深五尺。是歲魏郡太守京房為石顯所告,坐與妻父淮陽王舅張博、博弟光勸視淮陽王以不義,博要斬,光、房棄市,御史大夫鄭弘坐免為庶人。成帝即位,顯伏辜,淮陽王上書冤博,辭語增加,家屬徙者復得還。

建昭四年三月,雨雪,燕多死。谷永對曰:「皇后桑蠶以治祭服,共事天地宗廟,正以是日疾風自西北,大寒雨雪,壞敗其功,以章不鄉。宜齊戒辟寢,以深自責,請皇后就宮,鬲閉門戶,毋得擅上。且令眾妾人人更進,以時博施。皇天說喜,庶幾可以得賢明之嗣。即不行臣言,災異俞甚,天變成形,臣雖欲復捐身關策,不及事已。」其後許后坐祝詛廢。

陽朔四年四月,雨雪,燕雀死。後十六年,許皇后自殺。

定公元年「十月,隕霜殺菽」。劉向以為周十月,今八月也,消卦為觀,陰氣未至君位而殺,誅罰不由君出,在臣下之象也。是時季氏逐昭公,公死于外,定公得立,故天見災以視公也。釐公二年「十月,隕霜不殺草」,為嗣君微,失秉事之象也。其後卒在臣下,則災為之生矣。異故言草,災故言菽,重殺穀。一曰菽,草之難殺者也,言殺菽,知草皆死也;言不殺草,知菽亦不死也。董仲舒以為菽,草之彊者,天戒若曰,加誅於彊臣。言菽,以微見季氏之罰也。

武帝元光四年四月,隕霜殺草木。先是二年,遣五將軍三十萬眾伏馬邑下,欲襲單于,單于覺之而去。自是始征伐四夷,師出三十餘年,天下戶口減半。京房易傳曰:「興兵妄誅,茲謂亡法,厥災霜,夏殺五穀,冬殺麥。誅不原情,茲謂不仁,其霜,夏先大雷風,冬先雨,乃隕霜,有芒角。賢聖遭害,其霜附木不下地。佞人依刑,茲謂私賊,其霜在草根土隙間。不教而誅茲謂虐,其霜反在草下。」

元帝永光元年三月,隕霜殺桑;九月二日,隕霜殺稼,天下大飢。是時中書令石顯用事專權,與春秋定公時隕霜同應。成帝即位,顯坐作威福誅。

釐公二十九年「秋,大雨雹」。劉向以為盛陽雨水,溫煖而湯熱,陰氣脅之不相入,則轉而為雹;盛陰雨雪,凝滯而冰寒,陽氣薄之不相入,則散而為霰。故沸湯之在閉器,而湛於寒泉,則為冰,及雪之銷,亦冰解而散,此其驗也。故雹者陰脅陽也,霰者陽脅陰也,春秋不書霰者,猶月食也。釐公末年信用公子遂,遂專權自恣,將至於殺君,故陰脅陽之象見。釐公不寤,遂終專權,後二年殺子赤,立宣公。左氏傳曰:「聖人在上無雹,雖有不為災。」說曰:凡物不為災不書,書大,言為災也。凡雹,皆冬之愆陽,夏之伏陰也。

昭公三年,「大雨雹」。是時季氏專權,脅君之象見。昭公不寤,後季氏卒逐昭公。

元封三年十二月,雷雨雹,大如馬頭。宣帝地節四年五月,山陽濟陰雨雹如雞子,深二尺五寸,殺二十人,蜚鳥皆死。其十月,大司馬霍禹宗族謀反,誅,霍皇后廢。

成帝河平二年四月,楚國雨雹,大如斧,蜚鳥死。

左傳曰釐公三十二年十二月己卯,晉文公卒,庚辰,將殯于曲沃,出絳,柩有聲如牛。劉向以為近鼓妖也。喪,凶事;聲如牛,怒象也。將有急怒之謀,以生兵革之禍。是時,秦穆公遣兵襲鄭而不假道,還,晉大夫先軫謂襄公曰,秦師過不假塗,請擊之。遂要崤阨,以敗秦師,匹馬觭輪無反者,操之急矣。晉不惟舊,而聽虐謀,結怨彊國,四被秦寇,禍流數世,凶惡之效也。

哀帝建平二年四月乙亥朔,御史大夫朱博為丞相,少府趙玄為御史大夫,臨延登受策,有大聲如鍾鳴,殿中郎吏陛者皆聞焉。上以問黃門侍郎揚雄、李尋,尋對曰:「洪範所謂鼓妖者也。師法以為人君不聰,為眾所惑,空名得進,則有聲無形,不知所從生。其傳曰歲月日之中,則正卿受之。今以四月日加辰巳有異,是為中焉。正卿謂執政大臣也。宜退丞相、御史,以應天變。然雖不退,不出期年,其人自蒙其咎。」揚雄亦以為鼓妖,聽失之象也。朱博為人彊毅多權謀,宜將不宜相,恐有凶惡亟疾之怒。八月,博、玄坐為姦謀,博自殺,玄減死論。京房易傳曰:「令不修本,下不安,金毋故自動,若有音。」

史記秦二世元年,天無雲而雷。劉向以為雷當託於雲,猶君託於臣,陰陽之合也。二世不恤天下,萬民有怨畔之心。是歲陳勝起,天下畔,趙高作亂,秦遂以亡。一曰,易震為雷,為貌不恭也。

史記秦始皇八年,河魚大上。劉向以為近魚孽也。是歲,始皇弟長安君將兵擊趙,反,死屯留,軍吏皆斬,遷其民於臨洮。明年有嫪毒之誅。魚陰類,民之象,逆流而上者,民將不從君令為逆行也。其在天文,魚星中河而處,車騎滿野。至于二世,暴虐愈甚,終用急亡。京房易傳曰:「眾逆同志,厥妖河魚逆流上。」

武帝元鼎五年秋,蛙與蝦蟆群鬥。是歲,四將軍眾十萬征南越,開九郡。

成帝鴻嘉四年秋,雨魚于信都,長五寸以下。成帝永始元年春,北海出大魚,長六丈,高一丈,四枚。哀帝建平三年,東萊平度出大魚,長八丈,高丈一尺,七枚,皆死。京房易傳曰:「海數見巨魚,邪人進,賢人疏。」

桓公五年「秋,螽」。劉歆以為貪虐取民則螽,介蟲之孽也,與魚同占。劉向以為介蟲之孽屬言不從。是歲,公獲二國之聘,取鼎易邑,興役起城。諸螽略皆從董仲舒說云。

嚴公二十九年「有蜚」。劉歆以為負蠜也,性不食穀,食穀為災,介蟲之孽。劉向以為蜚色青,近青眚也,非中國所有。南越盛暑,男女同川澤,淫風所生,為蟲臭惡。是時嚴公取齊淫女為夫人,既入,淫於兩叔,故蜚至。天戒若曰,今誅絕之尚及,不將生臭惡,聞於四方。嚴不寤,其後夫人與兩叔作亂,二嗣以殺,卒皆被辜。董仲舒指略同。

釐公十五年「八月,螽」。劉向以為先是釐有鹹之會,後城緣陵,是歲復以兵車為牡丘會,使公孫敖帥師,及諸侯大夫救徐,兵比三年在外。

文公三年「秋,雨螽于宋」。劉向以為先是宋殺大夫而無罪,有暴虐賦斂之應。穀梁傳曰上下皆合,言甚。董仲舒以為宋三世內取,大夫專恣,殺生不中,故螽先死而至。劉歆以為螽為穀災,卒遇賊陰,墜而死也。

八年「十月,螽」。時公伐邾取須朐,城郚。

宣公六年「八月,螽」。劉向以為先是時宣伐莒向,後比再如齊,謀伐萊。

十三年「秋,螽」。公孫歸父會齊伐莒。

十五年「秋,螽」。宣亡熟歲,數有軍旅。

襄公七年「八月,螽」。劉向以為先是襄興師救陳,滕子、郯子、小邾子皆來朝。夏,城費。

哀公十二年「十二月,螽」。是時哀用田賦。劉向以為春用田賦,冬而螽。

十三年「九月,螽;十二月,螽」。比三螽,虐取於民之效也。劉歆以為周十二月,夏十月也,火星既伏,蟄蟲皆畢,天之見變,因物類之宜,不得以螽,是歲再失閏矣。周九月,夏七月,故傳曰「火猶西流,司曆過也」。

宣公十五年「冬,蝝生」。劉歆以為蝝,鐵缨之有翼者,食穀為災,黑眚也。董仲舒、劉向以為蝝,螟始生也,一曰

螟始生。是時民患上力役,解於公田。宣是時初稅畝。稅畝,就民田畝擇美者稅其什一,亂先王制而為貪利,故應是而蝝生,屬蠃蟲之孽。

景帝中三年秋,蝗。先是匈奴寇邊,中尉不害將車騎材官士屯代高柳。

武帝元光五年秋,螟;六年夏,蝗。先是,五將軍眾三十萬伏馬邑,欲襲單于也。是歲,四將軍征匈奴。

元鼎五年秋,蝗。是歲,四將軍征南越及西南夷,開十餘郡。

元封六年秋,蝗。先是,兩將軍征朝鮮,開三郡。

太初元年夏,蝗從東方蜚至敦煌;三年秋,復蝗。元年貳師將軍征大宛,天下奉其役連年。

征和三年秋,蝗;四年夏,蝗。先是一年,三將軍眾十餘萬征匈奴。征和三年,貳師七萬人沒不還。

平帝元始二年秋,蝗,遍天下。是時王莽秉政。

左氏傳曰嚴公八年齊襄公田于貝丘,見豕。從者曰:「公子彭生也。」公怒曰:「射之!」豕人立而啼,公懼,墜車,傷足喪屨。劉向以為近豕禍也。先是,齊襄淫於妹魯桓公夫人,使公子彭生殺威公,又殺彭生以謝魯。公孫無知有寵於先君,襄公絀之,無知帥怨恨之徒攻襄於田所,襄匿其戶間,足見於戶下,遂殺之。傷足喪屨,卒死於足,虐急之效也。

昭帝元鳳元年,燕王宮永巷中豕出圂,壞都灶,銜其釜六七枚置殿前。劉向以為近豕禍也。時燕王旦與長公主、左將軍謀為大逆,誅殺諫者,暴急無道。灶者,生養之本,豕而敗灶,陳釜於庭,釜灶將不用,宮室將廢辱也。燕王不改,卒伏其辜。京房易傳曰:「眾心不安君政,厥妖豕入居室。」

史記魯襄公二十三年,穀、洛水鬥,將毀王宮。劉向以為近火沴水也。周靈王將擁之,有司諫曰:「不可。長民者不崇藪,不墮山,不防川,不竇澤。今吾執政毋乃有所辟,而滑夫二川之神,使至于爭明,以防王宮室,王而飾之,毋乃不可乎!懼及子孫,王室愈卑。」王卒擁之。以傳推之,以四瀆比諸侯,穀、洛其次,卿大夫之象也,為卿大夫將分爭以危亂王室也。是時世卿專權,儋括將有篡殺之謀,如靈王覺寤,匡其失政,懼以承戒,則災禍除矣。不聽諫謀,簡嫚大異,任其私心,塞埤擁下,以逆水勢而害鬼神。後數年有黑如日者五。是歲蚤霜,靈王崩。景王立二年,儋括欲殺王,而立王弟佞夫。佞夫不知,景王并誅佞夫。及景王死,五大夫爭權,或立子猛,或立子朝,王室大亂。京房易傳曰:「天子弱,諸侯力政,厥異水鬥。」

史記曰,秦武王三年渭水赤者三日,昭王三十四年渭水又赤三日。劉向以為近火沴水也。秦連相坐之法,棄灰於道者黥,罔密而刑虐,加以武伐橫出,殘賊鄰國,至於變亂五行,氣色謬亂。天戒若曰,勿為刻急,將致敗亡。秦遂不改,至始皇滅六國,二世而亡。昔三代居三河,河洛出圖書,秦居渭陽,而渭水數赤,瑞異應德之效也。京房易傳曰:「君湎于酒,淫于色,賢人潛,國家危,厥異流水赤也。」

傳曰:「思心之不睿,是謂不聖,厥咎霿,厥罰恆風,厥極凶短折。時則有脂夜之妖,時則有華孽,時則有牛禍,時則有心腹之痾,時則有黃眚黃祥,時則有金木水火沴土。」

「思心之不睿,是謂不聖。」思心者,心思慮也;睿,寬也。孔子曰:「居上不寬,吾何以觀之哉!」言上不寬大包容臣下,則不能居聖位。貌言視聽,以心為主,四者皆失,則區霿無識,故其咎霿也。雨旱寒奧,亦以風為本,四氣皆亂,故其罰常風也。常風傷物,故其極凶短折也。傷人曰凶,禽獸曰短,屮木曰折。一曰,凶,夭也;兄喪弟曰短,父喪子曰折。在人腹中,肥而包裹心者脂也,心區霿則冥晦,故有脂夜之妖。一曰,有脂物而夜為妖,若脂水夜汙人衣,淫之象也。一曰,夜妖者,雲風並起而杳冥,故與常風同象也。溫而風則生螟螣,有裸蟲之孽。劉向以為於易巽為風為木,卦在三月四月,繼陽而治,主木之華實。風氣盛,至秋冬木復華,故有華孽。一曰,地氣盛則秋冬復華。一曰,華者色也,土為內事,為女孽也。於易坤為土為牛,牛大心而不能思慮,思心氣毀,故有牛禍。一曰,牛多死及為怪,亦是也。及人,則多病心腹者,故有心腹之痾。土色黃,故有黃眚黃祥。凡思心傷者病土氣,土氣病則金木水火沴之,故曰「時則有金木水火沴土」。不言「惟」而獨曰「時則有」者,非一衝氣所沴,明其異大也。其極曰凶短折,順之,其福曰考終命。劉歆思心傳曰時則有臝蟲之孽,謂螟螣之屬也。庶徵之常風,劉向以為春秋無其應。

釐公十六年「正月,六鶂退蜚,過宋都」。左氏傳曰「風也」。劉歆以為風發於它所,至宋而高,鶂高蜚而逢之,則退。經以見者為文,故記退蜚;傳以實應著,言風,常風之罰也。象宋襄公區霿自用,不容臣下,逆司馬子魚之諫,而與彊楚爭盟,後六年為楚所執,應六鶂之數云。京房易傳曰:「潛龍勿用,眾逆同志,至德乃潛,厥異風。其風也,行不解物,不長,雨小而傷。政悖德隱茲謂亂,厥風先風不雨,大風暴起,發屋折木。守義不進茲謂耄,厥風與雲俱起,折五穀莖。臣易上政,茲謂不順,厥風大焱發屋。賦斂不理茲謂禍,厥風絕經紀,止即溫,溫即蟲。侯專封茲謂不統,厥風疾,而樹不搖,穀不成。辟不思道利,茲謂無澤,厥風不搖木,旱無雲,傷禾。公常於利茲謂亂,厥風微而溫,生蟲蝗,害五穀。棄正作淫茲謂惑,厥風溫,螟蟲起,害有益人之物。侯不朝茲謂叛,厥風無恆,地變赤而殺人。」

文帝二年六月,淮南王都壽春大風毀民室,殺人。劉向以為是歲南越反,攻淮南邊,淮南王長破之,後年入朝,殺漢故丞相辟陽侯,上赦之,歸聚姦人謀逆亂,自稱東帝,見異不寤,後遷于蜀,道死廱。

文帝五年,吳暴風雨,壞城官府民室。時吳王濞謀為逆亂,天戒數見,終不改寤,後卒誅滅。

五年十月,楚王都彭城大風從東南來,毀巿門,殺人。是月王戊初嗣立,後坐淫削國,與吳王謀反,刑僇諫者。吳在楚東南,天戒若曰,勿與吳為惡,將敗巿朝。王戊不寤,卒隨吳亡。

昭帝元鳳元年,燕王都薊大風雨,拔宮中樹七圍以上十六枚,壞城樓。燕王旦不寤,謀反發覺,卒伏其辜。

釐公十五年「九月己卯晦,震夷伯之廟」。劉向以為晦,暝也;震,雷也。夷伯,世大夫,正書雷,其廟獨冥。天戒若曰,勿使大夫世官,將專事暝晦。明年,公子季友卒,果世官,政在季氏。至成公十六年「六月甲午晦」,正晝皆暝,陰為陽,臣制君也。成公不寤,其冬季氏殺公子偃。季氏萌於釐公,大於成公,此其應也。董仲舒以為夷伯,季氏之孚也,陪臣不當有廟。震者雷也,晦暝,雷擊其廟,明當絕去僭差之類也。向又以為此皆所謂夜妖者也。劉歆以為春秋及朔言朔,及晦言晦,人道所不及,則天震之。展氏有隱慝,故天加誅於其祖夷伯之廟以譴告之也。

成公十六年「六月甲午晦,晉侯及楚子、鄭伯戰于鄢陵」。皆月晦云。

隱公五年「秋,螟」。董仲舒、劉向以為時公觀漁于棠,貪利之應也。劉歆以為又逆臧釐伯之諫,貪利區瓒,以生臝蟲之孽也。

八年「九月,螟」。時鄭伯以邴將易許田,有貪利心。京房易傳曰:「臣安祿茲謂貪,厥災蟲,蟲食根。德無常茲謂煩,蟲食葉。不絀無德,蟲食本。與東作爭,茲謂不時,蟲食節。蔽惡生孽,蟲食心。」

嚴公六年「秋,螟」。董仲舒、劉向以為先是衛侯朔出奔齊,齊侯會諸侯納朔,許諸侯賂。齊人歸衛寶,魯受之,貪利應也。

文帝後六年秋,螟。是歲匈奴大入上郡、雲中,烽火通長安,遣三將軍屯邊,三將軍屯京師。

宣公三年,「郊牛之口傷,改卜牛,牛死」。劉向以為近牛禍也。是時宣公與公子遂謀共殺子赤而立,又以喪娶,區霿昏亂。亂成於口,幸有季文子得免於禍,天猶惡之,生則不饗其祀,死則災燔其廟。董仲舒指略同。

秦孝文王五年,斿朐衍,有獻五足牛者。劉向以為近牛禍也。先是文惠王初都咸陽,廣大宮室,南臨渭,北臨涇,思心失,逆土氣。足者止也,戒秦建止奢泰,將致危亡。秦遂不改,至於離宮三百,復起阿房,未成而亡。一曰,牛以力為人用,足所以行也。其後秦大用民力轉輸,起負海至北邊,天下叛之。京房易傳曰:「興繇役,奪民時,厥妖牛生五足。」

景帝中六年,梁孝王田北山,有獻牛,足上出背上。劉向以為近牛禍。先是孝王驕奢,起苑方三百里,宮館閣道相連三十餘里。納於邪臣羊勝之計,欲求為漢嗣,刺殺議臣爰盎,事發,負斧歸死。既退歸國,猶有恨心,內則思慮霿亂,外則土功過制,故牛禍作。足而出於背,下奸上之象也。猶不能自解,發疾暴死,又凶短之極也。

左氏傳昭公二十一年春,周景王將鑄無挛鍾,叠州鳩曰:「王其以心疾死乎!夫天子省風以作樂,小者不窕,大者不摦。摦則不容,心是以感,感實生疾。今鍾摦矣,王心弗龊,其能久乎?」劉向以為是時景王好聽淫聲,適庶不明,思心霿亂,明年以心疾崩,近心腹之痾,凶短之極者也。

昭二十五年春,魯叔孫昭子聘于宋,元公與燕,飲酒樂,語相泣也。樂祈佐,告人曰:「今茲君與叔孫其皆死乎!吾聞之,哀樂而樂哀,皆喪心也。心之精爽,是謂魂魄;魂魄去之,何以能久?」冬十月,叔孫昭子死;十一月,宋元公卒。

昭帝元鳳元年九月,燕有黃鼠銜其尾舞王宮端門中,往視之,鼠舞如故。王使夫人以酒脯祠,鼠舞不休,夜死。黃祥也。時燕剌王旦謀反將敗,死亡象也。其月,發覺伏辜。京房易傳曰:「誅不原情,厥妖鼠舞門。」

成帝建始元年四月辛丑夜,西北有如火光。壬寅晨,大風從西北起,雲氣赤黃,四塞天下,終日夜下著地者黃土塵也。是歲,帝元舅大司馬大將軍王鳳始用事;又封鳳母弟崇為安成侯,食邑萬戶;庶弟譚等五人賜爵關內侯,食邑三千戶。復益封鳳五千戶,悉封譚等為列侯,是為五侯。哀帝即位,封外屬丁氏、傅氏、周氏、鄭氏凡六人為列侯。楊宣對曰:「五侯封日,天氣赤黃,丁、傅復然。此殆爵土過制,傷亂土氣之祥也。」京房易傳曰:「經稱『

觀其生』,言大臣之義,當觀賢人,知其性行,推而貢之,否則為聞善不與,茲謂不知,厥異黃,厥咎聾,厥災不嗣。黃者,日上黃光不散如火然,有黃濁氣四塞天下。蔽賢絕道,故災異至絕世也。經曰『良馬逐』。逐,進也,言大臣得賢者謀,當顯進其人,否則為下相攘善,茲謂盜明,厥咎亦不嗣,至於身僇家絕。」

史記周幽王二年,周三川皆震。劉向以為金木水火沴土者也。伯陽甫曰:「周將亡矣!天地之氣不過其序;若過其序,民亂之也。陽伏而不能出,陰迫而不能升,於是有地震。今三川實震,是陽失其所而填陰也。陽失而在陰,原必塞;原塞,國必亡。夫水,土演而民用也;土無所演,而民乏財用,不亡何待?昔伊雒竭而夏亡,河竭而商亡,今周德如二代之季,其原又塞,塞必竭;川竭,山必崩。夫國必依山川,山崩川竭,亡之徵也。若國亡,不過十年,數之紀也。」

是歲二川竭,岐山崩。劉向以為陽失在陰者,謂火氣來煎枯水,故川竭也。山川連體,下竭上崩,事勢然也。時幽王暴虐,妄誅伐,不聽諫,迷於褒姒,廢其正后,廢后之父申侯與犬戎共攻殺幽王。一曰,其在天文,水為辰星,辰星為蠻夷。月食辰星,國以女亡。幽王之敗,女亂其內,夷攻其外。京房易傳曰:「君臣相背,厥異名水絕。」

文公九年「九月癸酉,地震」。劉向以為先是時,齊桓、晉文、魯釐二伯賢君新沒,周襄王失道,楚穆王殺父,諸侯皆不肖,權傾於下,天戒若曰,臣下彊盛者將動為害。後宋、魯、晉、莒、鄭、陳、齊皆殺君。諸震,略皆從董仲舒說也。京房易傳曰:「臣事雖正,專必震,其震,於水則波,於木則搖,於屋則瓦落。大經在辟而易臣,茲謂陰動,厥震搖政宮。大經搖政,茲謂不陰,厥震搖山,山出涌水。嗣子無德專祿,茲謂不順,厥震動丘陵,涌水出。」

襄公十六年「五月甲子,地震」。劉向以為先是雞澤之會,諸侯盟,大夫又盟。是歲三月,諸侯為溴梁之會,而大夫獨相與盟,五月地震矣。其後崔氏專齊,欒盈亂晉,良霄傾鄭,閽殺吳子,燕逐其君,楚滅陳、蔡。

昭公十九年「五月己卯,地震」。劉向以為是時季氏將有逐君之變。其後宋三臣、曹會皆以地叛,蔡、莒逐其君,吳敗中國殺二君。

二十三年「八月乙未,地震」。劉向以為是時周景王崩,劉、單立王子猛,尹氏立子朝。其後季氏逐昭公,黑肱叛邾,吳殺其君僚,宋五大夫、晉二大夫皆以地叛。

哀公三年「四月甲午,地震」。劉向以為是時諸侯皆信邪臣,莫能用仲尼,盜殺蔡侯,齊陳乞弒君。

惠帝二年正月,地震隴西,厭四百餘家。武帝征和二年八月癸亥,地震,厭殺人。宣帝本始四年四月壬寅,地震河南以東四十九郡,北海琅邪壞祖宗廟城郭,殺六千餘人。元帝永光三年冬,地震。綏和二年九月丙辰,地震,自京師至北邊郡國三十餘壞城郭,凡殺四百一十五人。

釐公十四年「秋八月辛卯,沙麓崩」。穀梁傳曰:「林屬於山曰麓,沙其名也。」劉向以為臣下背叛,散落不事上之象也。先是,齊桓行伯道,會諸侯,事周室。管仲既死,桓德日衰,天戒若曰,伯道將廢,諸侯散落,政逮大夫,陪臣執命,臣下不事上矣。桓公不寤,天子蔽晦。及齊威死,天下散而從楚。王札子殺二大夫,晉敗天子之師,莫能征討,從是陵遲。公羊以為沙麓,河上邑也。董仲舒說略同。一曰,河,大川象;齊,大國;桓德衰,伯道將移於晉文,故河為徙也。左氏以為沙麓,晉地;沙,山名也;地震而麓崩,不書震,舉重者也。伯陽甫所謂「

國必依山川,山崩川竭,亡之徵也;不過十年,數之紀也。」至二十四年,晉懷公殺於高梁。京房易傳曰:「小人剝廬,厥妖山崩,茲謂陰乘陽,弱勝彊。」

成公五年「夏,梁山崩」。穀梁傳曰廱河三日不流,晉君帥群臣而哭之,乃流。劉向以為山陽,君也,水陰,民也,天戒若曰,君道崩壞,下亂,百姓將失其所矣。哭然後流,喪亡象也。梁山在晉地,自晉始而及天下也。後晉暴殺三卿,厲公以弒。溴梁之會,天下大夫皆執國政,其後孫、甯出衛獻,三家逐魯昭,單、尹亂王室。董仲舒說略同。劉歆以為梁山,晉望也;崩,弛崩也。古者三代命祀,祭不越望,吉凶禍福,不是過也。國主山川,山崩川竭,亡之徵也,美惡周必復。是歲歲在鶉火,至十七年復在鶉火,欒書、中行偃殺厲公而立悼公。

高后二年正月,武都山崩,殺七百六十人,地震至八月乃止。文帝元年四月,齊楚地山二十九所同日俱大發水,潰出,劉向以為近水沴土也。天戒若曰,勿盛齊楚之君,今失制度,將為亂。後十六年,帝庶兄齊悼惠王之孫文王則薨,無子,帝分齊地,立悼惠王庶子六人皆為王。賈誼、晁錯諫,以為違古制,恐為亂。至景帝三年,齊楚七國起兵百餘萬,漢皆破之。春秋四國同日災,漢七國同日眾山潰,咸被其害,不畏天威之明效也。

成帝河平三年二月丙戌,犍為柏江山崩,捐江山崩,皆廱江水,江水逆流壞城,殺十三人,地震積二十一日,百二十四動。元延三年正月丙寅,蜀郡岷山崩,廱江,江水逆流,三日乃通。劉向以為周時岐山崩,三川竭,而幽王亡。岐山者,周所興也。漢家本起於蜀漢,今所起之地山崩川竭,星孛又及攝提、大角,從參至辰,殆必亡矣。其後三世亡嗣,王莽篡位。

傳曰:「皇之不極,是謂不建,厥咎眊,厥罰恆陰,厥極弱。時則有射妖,時則有龍蛇之孽,時則有馬禍,時則有下人伐上之痾,時則有日月亂行,星辰逆行。」

「皇之不極,是謂不建」,皇,君也。極,中;建,立也。人君貌言視聽思心五事皆失,不得其中,則不能立萬事,失在眊悖,故其咎眊也。王者自下承天理物。雲起於山,而彌於天;天氣亂,故其罰常陰也。一曰,上失中,則下彊盛而蔽君明也。《易》曰「亢龍有悔,貴而亡位,高而亡民,賢人在下位而亡輔」,如此,則君有南面之尊,而亡一人之助,故其極弱也。盛陽動進輕疾。

禮,春而大射,以順陽氣。上微弱則下奮動,故有射妖。《易》曰「雲從龍」,又曰「龍蛇之蟄,以存身也」。陰氣動,故有龍蛇之孽。於易,乾為君為馬,馬任用而彊力,君氣毀,故有馬禍。一曰,馬多死及為怪,亦是也。君亂且弱,人之所叛,天之所去,不有明王之誅,則有篡弒之禍,故有下人伐上之痾。凡君道傷者病天氣,不言五行沴天,而曰「日月亂行,星辰逆行」者,為若下不敢沴天,猶春秋曰「王師敗績于貿戎」,不言敗之者,以自敗為文,尊尊之意也。劉歆皇極傳曰有下體生上之痾。說以為下人伐上,天誅已成,不得復為痾云。皇極之常陰,劉向以為春秋亡其應。一曰,久陰不雨是也。劉歆以為自屬常陰。

昭帝元平元年四月崩,亡嗣,立昌邑王賀。賀即位,天陰,晝夜不見日月。賀欲出,光祿大夫夏侯勝當車諫曰:「天久陰而不雨,臣下有謀上者,陛下欲何之?」賀怒,縛勝以屬吏,吏白大將軍霍光。光時與車騎將軍張安世謀欲廢賀。光讓安世,以為泄語,安世實不泄,召問勝。勝上洪範五行傳曰:「『皇之不極,厥罰常陰,時則有下人伐上。』不敢察察言,故云臣下有謀。」光、安世讀之,大驚,以此益重經術士。後數日卒共廢賀,此常陰之明效也。京房易傳曰:「有蜺、蒙、霧。霧,上下合也。蒙如塵雲。蜺,日旁氣也。其占曰:后妃有專,蜺再重,赤而專,至衝旱。妻不壹順,黑蜺四背,又白蜺雙出日中。妻以貴高夫,茲謂擅陽,蜺四方,日光不陽,解而溫。內取茲謂禽,蜺如禽,在日旁。以尊降妃,茲謂薄嗣,蜺直而塞,六辰乃除,夜星見而赤。女不變始,茲謂乘夫,蜺白在日側,黑蜺果之。氣正直。妻不順正,茲謂擅陽,蜺中窺貫而外專。夫妻不嚴茲謂媟,蜺與日會。婦人擅國茲謂頃,蜺白貫日中,赤蜺四背。適不答茲謂不次,蜺直在左,蜺交在右。取於不專,茲謂危嗣,蜺抱日兩未及。君淫外茲謂亡,蜺氣左日交於外。取不達茲謂不知,蜺白奪明而大溫,溫而雨。尊卑不別茲謂媟,蜺三出三已,三辰除,除則日出且雨。臣私祿及親,茲謂罔辟,厥異蒙,其蒙先大溫,已蒙起,日不見。行善不請於上,茲謂作福,蒙一日五起五解。辟不下謀,臣辟異道,茲謂不見,上蒙下霧,風三變而俱解。立嗣子疑,茲謂動欲,蒙赤,日不明。德不序茲謂不聰,蒙,日不明,溫而民病。德不試,空言祿,茲謂主窳臣夭,蒙起而白。君樂逸人茲謂放,蒙,日青,黑雲夾日,左右前後行過日。公不任職,茲謂怙祿,蒙三日,又大風五日,蒙不解。利邪以食,茲謂閉上,蒙大起,白雲如山行蔽日。公懼不言道,茲謂閉下,蒙大起,日不見,若雨不雨,至十二日解,而有大雲蔽日。祿生於下,茲謂誣君,蒙微而小雨,已乃大雨。下相攘善,茲謂盜明,蒙黃濁。下陳功,求於上,茲謂不知,蒙,微而赤,風鳴條,解復蒙。下專刑茲謂分威,蒙而日不得明,大臣厭小臣茲謂蔽,蒙微,日不明,若解不解,大風發,赤雲起而蔽日。眾不惡惡茲謂閉,蒙,尊卦用事,三日而起,日不見。漏言亡喜,茲謂下厝用,蒙微,日無光,有雨雲,雨不降。廢忠惑佞茲謂亡,蒙,天先清而暴,蒙微而日不明。有逸民茲謂不明,蒙濁,奪日光。公不任職,茲謂不絀,蒙白,三辰止,則日青,青而寒,寒必雨。忠臣進善君不試,茲謂遏,蒙,先小雨,雨已蒙起,微而日不明。惑眾在位,茲謂覆國,蒙微而日不明,一溫一寒,風揚塵。知佞厚之茲謂庳,蒙甚而溫。君臣故弼茲謂悖,厥災風雨霧,風拔木,亂五穀,已而大霧。庶正蔽惡,茲謂生孽災,厥異霧。」此皆陰雲之類云。

嚴公十八年「秋,有蜮」。劉向以為蜮生南越。越地多婦人,男女同川,淫女為主,亂氣所生,故聖人名之曰蜮。蜮猶惑也,在水旁,能射人,射人有處,甚者至死。南方謂之短弧,近射妖,死亡之象也。時嚴將取齊之淫女,故蜮至。天戒若曰,勿取齊女,將生淫惑篡弒之禍。嚴不寤,遂取之。入後淫於二叔,二叔以死,兩子見弒,夫人亦誅。劉歆以為蜮,盛暑所生,非自越來也。京房易傳曰:「忠臣進善君不試,厥咎國生蜮。」

史記魯哀公時,有隼集于陳廷而死,楛矢貫之,石砮,長尺有咫。陳閔公使使問仲尼,仲尼曰:「

隼之來遠矣!昔武王克商,通道百蠻,使各以方物來貢,肅慎貢楛矢,石砮長尺有咫。先王分異姓以遠方職,使毋忘服,故分陳以肅慎矢。」試求之故府,果得之。劉向以為隼近黑祥,貪暴類也;矢貫之,近射妖也;死於廷,國亡表也。象陳眊亂,不服事周,而行貪暴,將致遠夷之禍,為所滅也。是時中國齊晉、南夷吳楚為彊,陳交晉不親,附楚不固,數被二國之禍。後楚有白公之亂,陳乘而侵之,卒為楚所滅。

史記夏后氏之衰,有二龍止於夏廷,而言「余,褒之二君也」。夏帝卜殺之,去之,止之,莫吉;卜請其漦而藏之,乃吉。於是布幣策告之。龍亡而漦在,乃櫝去之。其後夏亡,傳櫝於殷周,三代莫發,至厲王末,發而觀之,漦流于廷,不可除也。厲王使婦人臝而譟之,漦化為玄黿,入後宮。處妾遇之而孕,生子,懼而棄之。宣王立,女童謠曰:「览弧萁服,實亡周國。」後有夫婦鬻是器者,宣王使執而僇之。既去,見處妾所棄妖子,聞其夜號,哀而收之,遂亡奔褒。後褒人有罪,入妖子以贖,是為褒姒,幽王見而愛之,生子伯服。王廢申后及太子宜咎,而立褒姒、伯服代之。廢后之父申侯與繒西畎戎共攻殺幽王。《詩》曰:「赫赫宗周,褒姒醤之。」劉向以為夏后季世,周之幽、厲,皆誖亂逆天,故有龍黿之怪,近龍蛇孽也。漦,血也,一曰沫也。览弧,桑弓也。萁服,蓋以萁草為箭服,近射妖也。女童謠者,禍將生於女,國以兵寇亡也。

左氏傳昭公十九年,龍鬥於鄭時門之外洧淵。劉向以為近龍孽也。鄭以小國攝乎晉楚之間,重以彊吳,鄭當其衝,不能修德,將鬥三國,以自危亡。是時子產任政,內惠於民,外善辭令,以交三國,鄭卒亡患,能以德消變之效也。京房易傳曰:「眾心不安,厥妖龍鬥。」

惠帝二年正月癸酉旦,有兩龍見於蘭陵廷東里溫陵井中,至乙亥夜去。劉向以為龍貴象而困於庶人井中,象諸侯將有幽執之禍。其後呂太后幽殺三趙王,諸呂亦終誅滅。京房易傳曰:「有德遭害,厥妖龍見井中。」又曰:「行刑暴惡,黑龍從井出。」

左氏傳魯嚴公時有內蛇與外蛇鬥鄭南門中,內蛇死。劉向以為近蛇孽也。先是鄭厲公劫相祭仲而逐兄昭公代立。後厲公出奔,昭公復入。死,弟子儀代立。厲公自外劫大夫傅瑕,使僇子儀。此外蛇殺內蛇之象也。蛇死六年,而厲公立。嚴公聞之,問申繻曰:「猶有妖乎?」對曰:「人之所忌,其氣炎以取之,妖由人興也。人亡舋焉,妖不自作。人棄常,故有妖。」京房易傳曰:「立嗣子疑,厥妖蛇居國門鬥。」

左氏傳文公十六年夏,有蛇自泉宮出,入于國,如先君之數。劉向以為近蛇孽也。泉宮在囿中,公母姜氏嘗居之,蛇從之出,象宮將不居也。《詩》曰:「維虺維蛇,女子之祥。」又蛇入國,國將有女憂也。如先君之數者,公母將薨象也。秋,公母薨。公惡之,乃毀泉臺。夫妖孽應行而自見,非見而為害也。文不改行循正,共御厥罰,而作非禮,以重其過。後二年薨,公子遂殺文之二子惡、視,而立宣公。文公夫人大歸于齊。

武帝太始四年七月,趙有蛇從郭外入,與邑中蛇鬥孝文廟下,邑中蛇死。後二年秋,有衛太子事,事自趙人江充起。

左氏傳定公十年,宋公子地有白馬駟,公嬖向魋欲之,公取而朱其尾鬣以予之。地怒,使其徒抶魋而奪之。魋懼將走,公閉門而泣之,目盡腫。公弟辰謂地曰:「子為君禮,不過出竟,君必止子。」地出奔陳,公弗止。辰為之請,不聽。辰曰:「是我迋吾兄也,吾以國人出,君誰與處?」遂與其徒出奔陳。明年俱入于蕭以叛,大為宋患,近馬禍也。

史記秦孝公二十一年有馬生人,昭王二十年牡馬生子而死。劉向以為皆馬禍也。孝公始用商君攻守之法,東侵諸侯,至於昭王,用兵彌烈。其象將以兵革抗極成功,而還自害也。牡馬非生類,妄生而死,猶秦恃力彊得天下,而還自滅之象也。曰,諸畜生非其類,子孫必有非其姓者,至於始皇,果呂不韋子。京房易傳曰:「

方伯分威,厥妖牡馬生子。亡天子,諸侯相伐,厥妖馬生人。」

文帝十二年,有馬生角於吳,角在耳前,上鄉。右角長三寸,左角長二寸,皆大二寸。劉向以為馬不當生角,猶吳不當舉兵鄉上也。是時,吳王濞封有四郡五十餘城,內懷驕恣,變見於外,天戒早矣。王不寤,後卒舉兵,誅滅。京房易傳曰:「臣易上,政不順,厥妖馬生角,茲謂賢士不足。」又曰:「天子親伐,馬生角。」

成帝綏和三年二月,大廄馬生角,在左耳前,圍長各二寸。是時王莽為大司馬,害上之萌自此始矣。哀帝建平二年,定襄牡馬生駒,三足,隨群飲食,太守以聞。馬,國之武用,三足,不任用之象也。後侍中董賢年二十二為大司馬,居上公之位,天下不宗。哀帝暴崩,成帝母王太后召弟子新都侯王莽入,收賢印綬,賢恐,自殺,莽因代之,並誅外家丁、傅。又廢哀帝傅皇后,令自殺,發掘帝祖母傅太后、母丁太后陵,更以庶人葬之。辜及至尊,大臣微弱之禍也。

文公十一年,「敗狄于鹹」。穀梁、公羊傳曰,長狄兄弟三人,一者之魯,一者之齊,一者之晉。皆殺之,身橫九畝;斷其首而載之,眉見於軾。何以書?記異也。劉向以為是時周室衰微,三國為大,可責者也。天戒若曰,不行禮義,大為夷狄之行,將至危亡。其後三國皆有篡弒之禍,近下人伐上之痾也。劉歆以為人變,屬黃祥。一曰,屬臝蟲之孽。一曰,天地之性人為貴,凡人為變,皆屬皇極下人伐上之痾云。京房易傳曰:「君暴亂,疾有道,厥妖長狄入國。」又曰:「豐其屋,下獨苦。長狄生,世主虜。」

史記秦始皇帝二十六年,有大人長五丈,足履六尺,皆夷狄服,凡十二人,見于臨洮。天戒若曰,勿大為夷狄之行,將受其禍。是歲始皇初并六國,反喜以為瑞,銷天下兵器,作金人十二以象之。遂自賢聖,燔詩書,阬儒士;奢淫暴虐,務欲廣地;南戍五嶺,北築長城以備胡越,塹山填谷,西起臨洮,東至遼東,徑數千里。故大人見於臨洮,明禍亂之起。後十四年而秦亡,亡自戍卒陳勝發。

史記魏襄王十三年,魏有女子化為丈夫。京房易傳曰:「女子化為丈夫,茲謂陰昌,賤人為王;丈夫化為女子,茲謂陰勝,厥咎亡。」一曰,男化為女,宮刑濫也;女化為男,婦政行也。

哀帝建平中,豫章有男子化為女子,嫁為人婦,生一子。長安陳鳳言此陽變為陰,將亡繼嗣,自相生之象。一曰,嫁為人婦生一子,將復一世乃絕。

哀帝建平四年四月,山陽方與女子田無嗇生子。先未生二月,兒啼腹中,及生,不舉,葬之陌上,三日,人過聞啼聲,母掘收養。

平帝元始元年二月,朔方廣牧女子趙春病死,斂棺積六日,出在棺外,自言見夫死父,曰:「年二十七,不當死。」太守譚以聞。京房易傳曰:「『幹父之蠱,有子,考亡咎』。子三年不改父道,思慕不皇,亦重見先人之非,不則為私,厥妖人死復生。」一曰,至陰為陽,下人為上。

六月,長安女子有生兒,兩頭異頸面相鄉,四臂共匈俱前鄉,犊上有目長二寸所。京房易傳曰:「『睽孤,見豕負塗』,厥妖人生兩頭。下相攘善,妖亦同。人若六畜首目在下,茲謂亡上,正將變更。凡妖之作,以譴失正,各象其類。二首,下不壹也;足多,所任邪也;足少,下不勝任,或不任下也。凡下體生於上,不敬也;上體生於下,媟瀆也;生非其類,淫亂也;人生而大,上速成也;生而能言,好虛也。群妖推此類,不改乃成凶也。」

景帝二年九月,膠東下密人年七十餘,生角,角有毛。時膠東、膠西、濟南、齊四主有舉兵反謀,謀由吳王濞起,連楚、趙,凡七國。下密,縣居四齊之中;角,兵象,上鄉者也;老人,吳王象也;年七十,七國象也。天戒若曰,人不當生角,猶諸侯不當舉兵以鄉京師也;禍從老人生,七國俱敗云。諸侯不寤,明年吳王先起,諸侯從之,七國俱滅。京房易傳曰:「冢宰專政,厥妖人生角。」

成帝建始三年十月丁未,京師相驚,言大水至。渭水虒上小女陳持弓年九歲,走入橫城門,入未央宮尚方掖門,殿門門衛戶者莫見,至句盾禁中而覺得。民以水相驚者,陰氣盛也。小女而入宮殿中者,下人將因女寵而居有宮室之象也。名曰持弓,有似周家览弧之祥。《易》曰:「弧矢之利,以威天下。」是時,帝母王太后弟鳳始為上將,秉國政,天知其後將威天下而入宮室,故象先見也。其後,王氏兄弟父子五侯秉權,至莽卒篡天下,蓋陳氏之後云。京房易傳曰:「妖言動眾,茲謂不信,路將亡人,司馬死。」

成帝綏和二年八月庚申,鄭通里男子王褒衣絳衣小冠,帶劍入北司馬門殿東門,上前殿,入非常室中,解帷組結佩之,招前殿署長業等曰:「天帝令我居此。」業等收縛考問,褒故公車大誰卒,病狂易,不自知入宮狀,下獄死。是時王莽為大司馬,哀帝即位,莽乞骸骨就第,天知其必不退,故因是而見象也。姓名章服甚明,徑上前殿路寢,入室取組而佩之,稱天帝命,然時人莫察。後莽就國,天下冤之,哀帝徵莽還京師。明年帝崩,莽復為大司馬,因是而篡國。

哀帝建平四年正月,民驚走,持稿或棷一枚,傳相付與,曰行詔籌。道中相過逢多至千數,或被髮徒踐,或夜折關,或踰牆入,或乘車騎奔馳,以置驛傳行,經歷郡國二十六,至京師。其夏,京師郡國民聚會里巷仟佰,設祭張博具,歌舞祠西王母,又傳書曰:「母告百姓,佩此書者不死。不信我言,視門樞下,當有白髮。」至秋止。是時帝祖母傅太后驕,與政事,故杜鄴對曰:「春秋災異,以指象為言語。籌,所以紀數。民,陰,水類也。水以東流為順走,而西行,反類逆上。象數度放溢,妄以相予,違忤民心之應也。西王母,婦人之稱。博弈,男子之事。於街巷仟伯,明離闑內,與疆外。臨事盤樂,炕陽之意。白髮,衰年之象,體尊性弱,難理易亂。門,人之所由;樞,其要也。居人之所由,制持其要也。其明甚者。今外家丁、傅並侍帷幄,布於列位,有罪惡者不坐辜罰,亡功能者畢受官爵。皇甫、三桓,詩人所刺,春秋所譏,亡以甚此。指象昭昭,以覺聖朝,柰何不應!」後哀帝崩,成帝母王太后臨朝,王莽為大司馬,誅滅丁、傅。一曰丁、傅所亂者小,此異乃王太后、莽之應云。

隱公三年「二月己巳,日有食之」。穀梁傳曰,言日不言朔,食晦。公羊傳曰,食二日,董仲舒、劉向以為其後戎執天子之使,鄭獲魯隱,滅戴,衛、魯、宋咸殺君。左氏劉歆以為正月二日,燕、越之分野也。凡日所躔而有變,則分野之國失政者受之。人君能修政,共御厥罰,則災消而福至;不能,則災息而禍生。故經書災而不記其故,蓋吉凶亡常,隨行而成禍福也。周衰,天子不班朔,魯曆不正,置閏不得其月,月大小不得其度。史記曰食,或言朔而實非朔,或不言朔而實朔,或脫不書朔與日,皆官失之也。京房易傳曰:「亡師茲謂不御,厥異日食,其食也既,並食不一處。誅眾失理,茲謂生叛,厥食既,光散。縱畔茲謂不明,厥食先大雨三日,雨除而寒,寒即食。專祿不封,茲謂不安,厥食既,先日出而黑,光反外燭。君臣不通茲謂亡,厥蝕三既。同姓上侵,茲謂誣君,厥食四方有雲,中央無雲,其日大寒。公欲弱主位,茲謂不知,厥食中白青,四方赤,已食地震。諸侯相侵,茲謂不承,厥食三毀三復。君疾善,下謀上,茲謂亂,厥食既,先雨雹,殺走獸。弒君獲位茲謂逆,厥食既,先風雨折木,日赤。內臣外鄉茲謂背,厥食食且雨,地中鳴。冢宰專政茲謂因,厥食先大風,食時日居雲中,四方亡雲。伯正越職,茲謂分威,厥食日中分。諸侯爭美於上茲謂泰,厥食日傷月,食半,天營而鳴。賦不得茲謂竭,厥食星隨而下。受命之臣專征云試,厥食雖侵光猶明,若文王臣獨誅紂矣。小人順受命者征其君云殺,厥食五色,至大寒隕霜,若紂臣順武王而誅紂矣。諸侯更制茲謂叛,厥食三復三食,食已而風,地動。適讓庶茲謂生欲,厥食日失位,光晻晻,月形見。酒亡節茲謂荒,厥蝕乍青乍黑乍赤,明日大雨,發霧而寒。」凡食二十占,其形二十有四,改之輒除;不改三年,三年不改六年,六年不改九年。推隱三年之食,貫中央,上下竟而黑,臣弒從中成之形也。後衛州吁弒君而立。

桓公三年「七月壬辰朔,日有食之,既」。董仲舒、劉向以為前事已大,後事將至者又大,則既。先是魯、宋弒君,魯又成宋亂,易許田,亡事天子之心;楚僭稱王。後鄭岠王師,射桓王,又二君相篡。劉歆以為六月,趙與晉分。先是,晉曲沃伯再弒晉侯,是歲晉大亂,滅其宗國。京房易傳以為桓三年日食貫中央,上下竟而黃,臣弒而不卒之形也。後楚嚴稱王,兼地千里。

十七年「十月朔,日有食之」。穀梁傳曰,言朔不言日,食二日也。劉向以為是時衛侯朔有罪出奔齊,天子更立衛君。朔藉助五國,舉兵伐之而自立,王命遂壞。魯夫人淫失於齊,卒殺威公。董仲舒以為言朔不言日,惡魯桓且有夫人之禍,將不終日也。劉歆以為楚、鄭分。

嚴公十八年「三月,日有食之」。穀梁傳曰,不言日,不言朔,夜食。史推合朔在夜,明旦日食而出,出而解,是為夜食。劉向以為夜食者,陰因日明之衰而奪其光,象周天子不明,齊桓將奪其威,專會諸侯而行伯道。其後遂九合諸侯,天子使世子會之,此其效也。公羊傳曰食晦。董仲舒以為宿在東壁,魯象也。後公子慶父、叔牙果通於夫人以劫公。劉歆以為晦魯、衛分。

二十五年「六月辛未朔,日有食之」。董仲舒以為宿在畢,主邊兵夷狄象也。後狄滅邢、衛。劉歆以為五月二日魯、趙分。

二十六年「十二月癸亥朔,日有食之」。董仲舒以為宿在心,心為明堂,文武之道廢,中國不絕若悋之象也。劉向以為時戎侵曹,魯夫人淫於慶父、叔牙,將以弒君,故比年再蝕以見戒。劉歆以為十月二日楚、鄭分。

三十年「九月庚午朔,日有食之」。董仲舒、劉向以為後魯二君弒,夫人誅,兩弟死,狄滅邢,徐取舒,晉殺世子,楚滅弦。劉歆以為八月秦、周分。

僖公五年「九月戊申朔,日有食之」。董仲舒、劉向以為先是齊桓行伯,江、黃自至,南服彊楚。其後不內自正,而外執陳大夫,則陳、楚不附,鄭伯逃盟,諸侯將不從桓政,故天見戒。其後晉滅虢,楚國許,諸侯伐鄭,晉弒二君,狄滅溫,楚伐黃,桓不能救。劉歆以為七月秦、晉分。

十二年「三月庚午朔,日有食之」。董仲舒、劉向以為是時楚滅黃,狄侵衛、鄭,莒滅杞。劉歆以為三月齊、衛分。

十五年「五月,日有食之」。劉向以為象晉文公將行伯道,後遂伐衛,執曹伯,敗楚城濮,再會諸侯,召天王而朝之,此其效也。日食者臣之惡也,夜食者掩其罪也,以為上亡明王,桓、文能行伯道,攘夷狄,安中國,雖不正猶可,蓋春秋實與而文不與之義也。董仲舒以為後秦獲晉侯,齊滅項,楚敗徐于婁林。劉歆以為二月朔齊、越分。

文公元年「二月癸亥,日有食之」。董仲舒、劉向以為先是大夫始執國政,公子遂如京師,後楚世子商臣殺父,齊公子商人弒君,皆自立,宋子哀出奔,晉滅江,楚滅六,大夫公孫敖、叔彭生並專會盟。劉歆以為正月朔燕、越分。

十五年「六月辛丑朔,日有食之」。董仲舒、劉向以為後宋、齊、莒、晉、鄭八年之間五君殺死,夷滅舒蓼。劉歆以為四月二日魯、衛分。

宣公八年「七月甲子,日有食之,既」。董仲舒、劉向以為先是楚商臣弒父而立,至于嚴王遂彊。諸夏大國唯有齊、晉,齊、晉新有篡弒之禍,內皆未安,故楚乘弱橫行,八年之間六侵伐而一滅國;伐陸渾戎,觀兵周室;後又入鄭,鄭伯肉袒謝罪;北敗晉師于邲,流血色水;圍宋九月,析骸而炊之。劉歆以為十月二日楚、鄭分。

十年「四月丙辰,日有食之」。董仲舒、劉向以為後陳夏徵舒弒其君,楚滅蕭,晉滅二國,王札子殺召伯、毛伯。劉歆以為二月魯、衛分。

十七年「六月癸卯,日有食之」。董仲舒、劉向以為後邾支解鄫子,晉敗王師于貿戎,敗齊于鞍。劉歆以為三月晦朓魯、衛分。

成公十六年「六月丙寅朔,日有食之」。董仲舒、劉向以為後晉敗楚、鄭于鄢陵,執魯侯。劉歆以為四月二日魯、衛分。

十七年「十二月丁巳朔,日有食之」。董仲舒、劉向以為後楚滅舒庸,晉弒其君,宋魚石因楚奪君邑,莒滅鄫,齊滅萊,鄭伯弒死。劉歆以為九月周、楚分。

襄公十四年「二月乙未朔,日有食之」。董仲舒、劉向以為後衛大夫孫、甯共逐獻公,立孫剽。劉歆以為前年十二月二日宋、燕分。

十五年「八月丁巳,日有食之」。董仲舒、劉向以為先是晉為雞澤之會,諸侯盟,又大夫盟,後為溴梁之會,諸侯在而大夫獨相與盟,君若綴斿,不得舉手。劉歆以為五月二日魯、趙分。

二十年「十月丙辰朔,日有食之」。董仲舒以為陳慶虎、慶寅蔽君之明,邾庶其有叛心,後庶其以漆、閭丘來奔,陳殺二慶。劉歆以為八月秦、周分。

二十一年「九月庚戌朔,日有食之」。董仲舒以為晉欒盈將犯君,後入于曲沃。劉歆以為七月秦、晉分。

「十月庚辰朔,日有食之」。董仲舒以為宿在軫、角,楚大國象也。後楚屈氏譖殺公子追舒,齊慶封脅君亂國。劉歆以為八月秦、周分。

二十三年「二月癸酉朔,日有食之」。董仲舒以為後衛侯入陳儀,甯喜弒其君剽。劉歆以為前年十二月二日宋、燕分。

二十四年「七月甲子朔,日有食之,既」。劉歆以為五月魯、趙分。

「八月癸巳朔,日有食之」。董仲舒以為比食又既,象陽將絕,夷狄主上國之象也。後六君弒,楚子果從諸侯伐鄭,滅舒鳩,魯往朝之,卒主中國,伐吳討慶封。劉歆以為六月晉、趙分。

二十七年「十二月乙亥朔,日有食之」。董仲舒以為禮義將大滅絕之象也。時吳子好勇,使刑人守門;蔡侯通於世子之妻;莒不早立嗣。後閽戕吳子,蔡世子般弒其父,莒人亦弒君而庶子爭。劉向以為自二十年至此歲,八年間日食七作,禍亂將重起,故天仍見戒也。後齊崔杼弒君,宋殺世子,北燕伯出奔,鄭大夫自外入而篡位,指略如董仲舒。劉歆以為九月周、楚分。

昭公七年「四月甲辰朔,日有食之」。董仲舒、劉向以為先是楚靈王弒君而立,會諸侯,執徐子,滅賴,後陳公子招殺世子,楚因而滅之,又滅蔡,後靈王亦弒死。劉歆以為二月魯、衛分。傳曰晉侯問於士文伯曰:「誰將當日食?」對曰:「魯、衛惡之,衛大魯小。」公曰:「何故?」對曰:「去衛地,如魯地,於是有災,其衛君乎?魯將上卿。」是歲,八月衛襄公卒,十一月魯季孫宿卒。晉侯謂士文伯曰:「吾所問日食從矣,可常乎?」對曰:「不可。六物不同,民心不壹,事序不類,官職不則,同始異終,胡可常也?《詩》曰:『或宴宴居息,或盡悴事國。』其異終也如是。」公曰:「何謂六物?」對曰:「

歲、時、日、月、星、辰是謂。」公曰:「何謂辰?」對曰:「日月之會是謂。」公曰:「詩所謂『此日而食,于何不臧』,何也?」對曰:「不善政之謂也。國無政,不用善,則自取適于日月之災。故政不可不慎也,務三而已:一曰擇人,二曰因民,三曰從時。」此推日食之占循變復之要也。《易》曰:「縣象著明,莫大於日月。」是故聖人重之,載于三經。於易在豐之震曰:「豐其沛,日中見昧,折其右肱,亡咎。」於詩十月之交,則著卿士、司徒,下至趣馬、師氏,咸非其材。同於右肱之所折,協於三務之所擇,明小人乘君子,陰侵陽之原也。

十五年「六月丁巳朔,日有食之」。劉歆以為三月魯、衛分。

十七年「六月甲戌朔,日有食之」。董仲舒以為時宿在畢,晉國象也。晉厲公誅四大夫,失眾心,以弒死。後莫敢復責大夫,六卿遂相與比周,專晉國,君還事之。日比再食,其事在春秋後,故不載於經。劉歆以為魯、趙分。左氏傳平子曰:「唯正月朔,慝未作,日有食之,於是乎天子不舉,伐鼓於社,諸侯用幣於社,伐鼓於朝,禮也。其餘則否。」太史曰:「在此月也,日過分而未至,三辰有災,百官降物,君不舉,避移時,樂奏鼓,祝用幣,史用辭,嗇夫馳,庶人走,此月朔之謂也。當夏四月,是謂孟夏。」說曰:正月謂周六月,夏四月,正陽純乾之月也。慝謂陰爻也,冬至陽爻起初,故曰復。至建巳之月為純乾,亡陰爻,而陰侵陽,為災重,故伐鼓用幣,責陰之禮。降物,素服也。不舉,去樂也。避移時,避正堂,須時移災復也。嗇夫,掌幣吏。庶人,其徒役也。劉歆以為六月二日魯、趙分。

二十一年「七月壬午朔,日有食之」。董仲舒以為周景王老,劉子、單子專權,蔡侯朱驕,君臣不說之象也。後蔡侯朱果出奔,劉子、單子立王猛。劉歆以為五月二日魯、趙分。

二十二年「十二月癸酉朔,日有食之」。董仲舒以為宿在心,天子之象也。後尹氏立王子朝,天王居于狄泉。劉歆以為十月楚、鄭分。

二十四年「五月乙未朔,日有食之」。董仲舒以為宿在胃,魯象也。後昭公為季氏所逐。劉向以為自十五年至此歲,十年間天戒七見,人君猶不寤。後楚殺戎蠻子,晉滅陸渾戎,盜殺衛侯兄,蔡、莒之君出奔,吳滅巢,公子光殺王僚,宋三臣以邑叛其君。它如仲舒。劉歆以為二日魯、趙分。是月斗建辰。左氏傳梓慎曰:「將大水。」昭子曰:「旱也。日過分而陽猶不克,克必甚,能無旱乎!陽不克,莫將積聚也。」是歲秋,大雩,旱也。二至二分,日有食之,不為災。日月之行也,春秋分日夜等,故同道;冬夏至長短極,故相過。相過同道而食輕,不為大災,水旱而已。

三十一年「十二月辛亥朔,日有食之」。董仲舒以為宿在心,天子象也。時京師微弱,後諸侯果相率而城周,宋中幾亡尊天子之心,而不衰城。劉向以為時吳滅徐,而蔡滅沈,楚圍蔡,吳敗楚入郢,昭王走出。劉歆以為二日宋、燕分。

定公五年「三月辛亥朔,日有食之」。董仲舒、劉向以為後鄭滅許,魯陽虎作亂,竊寶玉大弓,季桓子退仲尼,宋三臣以邑叛。劉歆以為正月二日燕、趙分。

十二年「十一月丙寅朔,日有食之」。董仲舒、劉向以為後晉三大夫以邑叛,薛弒其君,楚滅頓、胡,越敗吳,衛逐世子。劉歆以為十二月二日楚、鄭分。

十五年「八月庚辰朔,日有食之」。董仲舒以為宿在柳,周室大壞,夷狄主諸夏之象也。明年,中國諸侯果累累從楚而圍蔡,蔡恐,遷于州來。晉人執戎蠻子歸于楚,京師楚也。劉向以為盜殺蔡侯,齊陳乞弒其君而立陽生,孔子終不用。劉歆以為六月晉、趙分。

哀公十四年「五月庚申朔,日有食之」。在獲麟後。劉歆以為三月二日齊、衛分。

凡春秋十二公,二百四十二年,日食三十六。穀梁以為朔二十六,晦七,夜二,二日一。公羊以為朔二十七,二日七,晦二。左氏以為朔十六,二日十八,晦一,不書日者二。

高帝三年十月甲戌晦,日有食之,在斗二十度,燕地也。後二年,燕王臧荼反,誅,立盧綰為燕王,後又反,敗。

十一月癸卯晦,日有食之,在虛三度,齊地也。後二年,齊王韓信徙為楚王,明年廢為列侯,後又反,誅。

九年六月乙未晦,日有食之,既,在張十三度。

惠帝七年正月辛丑朔,日有食之,在危十三度。谷永以為歲首正月朔日,是為三朝,尊者惡之。

五月丁卯,先晦一日,日有食之,幾盡,在七星初。劉向以為五月微陰始起而犯至陽,其占重。至其八月,宮車晏駕,有呂氏詐置嗣君之害。京房易傳曰:「凡日食不以晦朔者,名曰薄。人君誅將不以理,或賊臣將暴起,日月雖不同宿,陰氣盛,薄日光也。」

高后二年六月丙戌晦,日有食之。

七年正月己丑晦,日有食之,既,在營室九度,為宮室中。時高后惡之,曰:「此為我也!」明年應。

文帝二年十一月癸卯晦,日有食之,在婺女一度。

三年十月丁酉晦,日有食之,在斗二十三度。

十一月丁卯晦,日有食之,在虛八度。

後四年四月丙辰晦,日有食之,在東井十三度。

七年正月辛未朔,日有食之。

景帝三年二月壬午晦,日有食之,在胃二度。

七年十一月庚寅晦,日有食之,在虛九度。

中元年十二月甲寅晦,日有食之。

中二年九月甲戌晦,日有食之。

三年九月戊戌晦,日有食之,幾盡,在尾九度。

六年七月辛亥晦,日有食之,在軫七度。

後元年七月乙巳,先晦一日,日有食之,在翼十七度。

武帝建元二年二月丙戌朔,日有食之,在奎十四度。劉向以為奎為卑賤婦人,後有衛皇后自至微興,卒有不終之害。

三年九月丙子晦,日有食之,在尾二度。

五年正月己巳朔,日有食之。

元光元年二月丙辰晦,日有食之。

七月癸未,先晦一日,日有食之,在翼八度。劉向以為前年高園便殿災,與春秋御廩災後日食於翼、軫同。其占,內有女變,外為諸侯。其後陳皇后廢,江都、淮南、衡山王謀反,誅。日中時食從東北,過半,晡時復。

元朔二年二月乙巳晦,日有食之,在胃三度。

六年十一月癸丑晦,日有食之。

元狩元年五月乙巳晦,日有食之,在柳六度。京房易傳推以為是時日食從旁右,法曰君失臣。明年丞相公孫弘薨。日食從旁左者,亦君失臣;從上者,臣失君;從下者,君失民。

元鼎五年四月丁丑晦,日有食之,在東井二十三度。

元封四年六月己酉朔,日有食之。

太始元年正月乙巳晦,日有食之。

四年十月甲寅晦,日有食之,在斗十九度。

征和四年八月辛酉晦,日有食之,不盡如鉤,在亢二度。晡時食從西北,日下晡時復。

昭帝始元三年十一月壬辰朔,日有食之,在斗九度,燕地也。後四年,燕剌王謀反,誅。

元鳳元年七月己亥晦,日有食之,幾盡,在張十二度。劉向以為己亥而既,其占重。後六年,宮車晏駕,卒以亡嗣。

宣帝地節元年十二月癸亥晦,日有食之,在營室十五度。

五鳳元年十二月乙酉朔,日有食之,在婺女十度。

四年四月辛丑朔,日有食之,在畢十九度。是為正月朔,慝未作,左氏以為重異。

元帝永光二年三月壬戌朔,日有食之,在婁八度。

四年六月戊寅晦,日有食之,在張七度。

建昭五年六月壬申晦,日有食之,不盡如鉤,因入。

成帝建始三年十二月戊申朔,日有食之,其夜未央殿中地震。谷永對曰:「日食婺女九度,占在皇后。地震蕭牆之內,咎在貴妾。二者俱發,明同事異人,共掩制陽,將害繼嗣也。亶日食,則妾不見;亶地震,則后不見。異日而發,則似殊事;亡故動變,則恐不知。是月后妾當有失節之郵,故天因此兩見其變。若曰,違失婦道,隔遠眾妾,妨絕繼嗣者,此二人也。」杜欽對亦曰:「日以戊申食,時加未。戊未,土也,中宮之部。其夜殿中地震,此必適妾將有爭寵相害而為患者。人事失於下,變象見於上。能應之司德,則咎異消;忽而不戒,則禍敗至。應之,非誠不立,非信不行。」

河平元年四月己亥晦,日有食之,不盡如鉤,在東井六度。劉向對曰:「四月交於五月,月同孝惠,日同孝昭。東井,京師地,且既,其占恐害繼嗣。」日蚤食時,從西南起。

三年八月乙卯晦,日有食之,在房。

四年三月癸丑朔,日有食之,在昴。

陽朔元年二月丁未晦,日有食之,在胃。

永始元年九月丁巳晦,日有食之。谷永以京房易占對曰:「元年九月日蝕,酒亡節之所致也。獨使京師知之,四國不見者,若曰,湛湎于酒,君臣不別,禍在內也。」

永始二年二月乙酉晦,日有食之。谷永以京房易占對曰:「今年二月日食,賦斂不得度,民愁怨之所致也。所以使四方皆見,京師陰蔽者,若曰,人君好治宮室,大營墳墓,賦斂茲重,而百姓屈竭,禍在外也。」

三年正月己卯晦,日有食之。

四年七月辛未晦,日有食之。

元延元年正月己亥朔,日有食之。

哀帝元壽元年正月辛丑朔,日有食之,不盡如鉤,在營室十度,與惠帝七年同月日。

二年三月壬辰晦,日有食之。

平帝元始元年五月丁巳朔,日有食之,在東井。

二年九月戊申晦,日有食之,既。

凡漢著紀十二世,二百一十二年,日食五十三,朔十四,晦三十六,先晦一日三。

成帝建始元年八月戊午,晨漏未盡三刻,有兩月重見。京房易傳曰:「『婦貞厲,月幾望,君子征,凶。』言君弱而婦彊,為陰所乘,則月並出。晦而月見西方謂之朓,朔而月見東方謂之仄慝,仄慝則侯王其肅,朓則侯王其舒。」劉向以為朓者疾也,君舒緩則臣驕慢,故日行遲而月行疾也。仄慝者不進之意,君肅急則臣恐懼,故日行疾而月行遲,不敢迫近君也。不舒不急,以正失之者,食朔日。劉歆以為舒者侯王展意顓事,臣下促急,故月行疾也。肅者王侯縮朒不任事,臣下弛縱,故月行遲也。當春秋時,侯王率多縮朒不任事,故食二日仄慝者十八,食晦日朓者一,此其效也。考之漢家,食晦朓者三十六,終亡二日仄慝者,歆說信矣。此皆謂日月亂行者也。

元帝永光元年四月,日色青白,亡景,正中時有景亡光。是夏寒,至九月,日乃有光。京房易傳曰:「美不上人,茲謂上弱,厥異日白,七日不溫。順亡所制茲謂弱,日白六十日,物亡霜而死。天子親伐,茲謂不知,日白,體動而寒。弱而有任,茲謂不亡,日白不溫,明不動。辟鳗公行,茲謂不伸,厥異日黑,大風起,天無雲,日光晻。不難上政,茲謂見過,日黑居仄,大如彈丸。」

成帝河平元年正月壬寅朔,日月俱在營室,時日出赤。二月癸未,日朝赤,且入又赤,夜月赤。甲申,日出赤如血,亡光,漏上四刻半,乃頗有光,燭地赤黃,食後乃復。京房易傳曰:「辟不聞道茲謂亡,厥異日赤。」三月乙未,日出黃,有黑氣大如錢,居日中央。京房易傳曰:「祭天不順茲謂逆,厥異日赤,其中黑。聞善不予,茲謂失知,厥異日黃。」夫大人者,與天地合其德,與日月合其明,故聖王在上,總命群賢,以亮天功,則日之光明,五色備具,燭燿亡主;有主則為異,應行而變也。色不虛改,形不虛毀,觀日之五變,足以監矣。故曰「縣象著明,莫大乎日月」,此之謂也。

嚴公七年「四月辛卯夜,恆星不見,夜中星隕如雨」。董仲舒、劉向以為常星二十八宿者,人君之象也;眾星,萬民之類也。列宿不見,象諸侯微也;眾星隕墜,民失其所也。夜中者,為中國也。不及地而復,象齊桓起而救存之也。鄉亡桓公,星遂至地,中國其良絕矣。劉向以為夜中者,言不得終性命,中道敗也。或曰象其叛也,言當中道叛其上也。天垂象以視下,將欲人君防惡遠非,慎卑省微,以自全安也。如人君有賢明之材,畏天威命,若高宗謀祖己,成王泣金縢,改過修正,立信布德,存亡繼絕,修廢舉逸,下學而上達,裁什一之稅,復三日之役,節用儉服,以惠百姓,則諸侯懷德,士民歸仁,災消而福興矣。遂莫肯改寤,法則古人,而各行其私意,終於君臣乖離,上下交怨。自是之後,齊、宋之君弒,譚、遂、邢、衛之國滅,宿遷於宋,蔡獲於楚,晉相弒殺,五世乃定,此其效也。左氏傳曰:「恆星不見,夜明也;星隕如雨,與雨偕也。」劉歆以為晝象中國,夜象夷狄。夜明,故常見之星皆不見,象中國微也。「星隕如雨」,如,而也,星隕而且雨,故曰「與雨偕也」,明雨與星隕,兩變相成也。洪範曰:「庶民惟星。」《易》曰:「雷雨作,解。」是歲歲在玄枵,齊分野也。夜中而星隕,象庶民中離上也。雨以解過施,復從上下,象齊桓行伯,復興周室也。周四月,夏二月也,日在降婁,魯分野也。先是,衛侯朔奔齊,衛公子黔牟立,齊帥諸侯伐之,天子使使救衛。魯公子溺專政,會齊以犯王命,嚴弗能止,卒從而伐衛,逐天王所立。不義至甚,而自以為功。名去其上,政繇下作,尤著,故星隕於魯,天事常象也。

成帝永始二年二月癸未,夜過中,星隕如雨,長一二丈,繹繹未至地滅,至雞鳴止。谷永對曰:「日月星辰燭臨下土,其有食隕之異,則遐邇幽隱靡不咸睹。星辰附離于天,猶庶民附離王者也。王者失道,綱紀廢頓,下將叛去,故星叛天而隕,以見其象。春秋記異,星隕最大,自魯嚴以來,至今再見。臣聞三代所以喪亡者,皆繇婦人群小,湛湎於酒。《書》云:『乃用其婦人之言,四方之逋逃多罪,是信是使。』《詩》曰:『赫赫宗周、褒姒醤之。』『顛覆厥德,荒沈于酒。』及秦所以二世而亡者,養生大奢,奉終大厚。方今國家兼而有之,社稷宗廟之大憂也。」京房易傳曰:「君不任賢,厥妖天雨星。」

文公十四年「七月,有星孛入于北斗」。董仲舒以為孛者惡氣之所生也。謂之孛者,言其孛孛有所防蔽,闇亂不明之貌也。北斗,大國象。後齊、宋、魯、莒、晉皆弒君。劉向以為君臣亂於朝,政令虧於外,則上濁三光之精,五星贏縮,變色逆行,甚則為孛。北斗,人君象;孛星,亂臣類,篡殺之表也。星傳曰「魁者,貴人之牢」。又曰「孛星見北斗中,大臣諸侯有受誅者」。一曰魁為齊、晉。夫彗星較然在北斗中,天之視人顯矣,史之有占明矣,時君終不改寤。是後,宋、魯、莒、晉、鄭、陳六國咸弒其君,齊再弒焉。中國既亂,夷狄並侵,兵革從橫,楚乘威席勝,深入諸夏,六侵伐,一滅國,觀兵周室。晉外滅二國,內敗王師,又連三國之兵大敗齊師于鞍,追亡逐北,東臨海水,威陵京師,武折大齊。皆孛星炎之所及,流至二十八年。星傳又曰:「彗星入北斗,有大戰。其流入北斗中,得名人;不入,失名人。」宋華元,賢名大夫,大棘之戰,華元獲於鄭,傳舉其效云。左氏傳曰有星孛北斗,周史服曰:「不出七年,宋、齊、晉之君皆將死亂。」劉歆以為北斗有環域,四星入其中也。斗,天之三辰,綱紀星也。宋、齊、晉,天子方伯,中國綱紀。彗所以除舊布新也。斗七星,故曰不出七年。至十六年,宋人弒昭公;十八年,齊人弒懿公;宣公二年,晉趙穿弒靈公。

昭公十七年「冬,有星孛于大辰」。董仲舒以為大辰心也,心

在明堂,天子之象。後王室大亂,三王分爭,此其效也。劉向以為星傳曰「心,大星,天王也。其前星,太子;後星,庶子也。尾為君臣乖離。」孛星加心,象天子適庶將分爭也。其在諸侯,角、亢、氐,陳、鄭也;房、心,宋也。後五年,周景王崩,王室亂,大夫劉子、單子立王猛,尹氏、召伯、毛伯立子晁。子晁,楚出也。時楚彊,宋、衛、陳、鄭皆南附楚。王猛既卒,敬王即位,子晁入王城,天王居狄泉,莫之敢納。五年,楚平王居卒,子晁奔楚,王室乃定。後楚帥六國伐吳,吳敗之于雞父,殺獲其君臣。蔡怨楚而滅沈,楚怒,圍蔡。吳人救之,遂為柏舉之戰,敗楚師,屠郢都,妻昭王母,鞭平王墓。此皆孛彗流炎所及之效也。左氏傳曰:「有星孛于大辰,西及漢。申繻曰:『彗,所以除舊布新也,天事恆象。今除於火,火出必布焉。諸侯其有火災乎?』梓慎曰:『往年吾見,是其徵也。火出而見,今茲火出而章,必火入而伏,其居火也久矣,其與不然乎?火出,於夏為三月,於商為四月,於周為五月。夏數得天,若火作,其四國當之,在宋、衛、陳、鄭乎?宋,大辰之虛;陳,太昊之虛;鄭,祝融之虛:皆火房也。星孛及漢;漢,水祥也。衛,顓頊之虛,其星為大水。水,火之牡也。其以丙子若壬午作乎?水火所以合也。若火入而伏,必以壬午,不過見之月。』」明年「夏五月,火始昏見,丙子風。梓慎曰:『是謂融風,火之始也。七日其火作乎?』戊寅風甚,壬午太甚,宋、衛、陳、鄭皆火。」劉歆以為大辰,房、心、尾也,八月心星在西方,孛從其西過心東及漢也。宋,大辰虛,謂宋先祖掌祀大辰星也。陳,太昊虛,虙羲木德,火所生也。鄭,祝融虛,高辛氏火正也。故皆為火所舍。衛,顓頊虛,星為大水,營室也。天星既然,又四國失政相似,及為王室亂皆同。

哀公十三年「冬十一月,有星孛于東方」。董仲舒、劉向以為不言宿名者,不加宿也。以辰乘日而出,亂氣蔽君明也。明年,春秋事終。一曰,周之十一月,夏九月,日在氐。出東方者,軫、角、亢也。軫,楚;角、亢,陳、鄭也。或曰角、亢大國象,為齊、晉也。其後楚滅陳,田氏篡齊,六卿分晉,此其效也。劉歆以為孛,東方大辰也,不言大辰,旦而見與日爭光,星入而彗猶見。是歲再失閏,十一月實八月也。日在鶉火,周分野也。十四年冬,「有星孛」,在獲麟後。劉歆以為不言所在,官失之也。

高帝三年七月,有星孛于大角,旬餘乃入。劉向以為是時項羽為楚王,伯諸侯,而漢已定三秦,與羽相距滎陽,天下歸心於漢,楚將滅,故彗除王位也。一曰,項羽阬秦卒,燒宮室,弒義帝,亂王位,故彗加之也。

文帝後七年九月,有星孛于西方,其本直尾、箕,末指虛、危,長丈餘,及天漢,十六日不見。劉向以為尾宋地,今楚彭城也。箕為燕,又為吳、越、齊。宿在漢中,負海之國水澤地也。是時景帝新立,信用晁錯,將誅正諸侯王,其象先見。後三年,吳、楚、四齊與趙七國舉兵反,皆誅滅云。

武帝建元六年六月,有星孛于北方。劉向以為明年淮南王安入朝,與太尉武安侯田蚡有邪謀,而陳皇后驕恣,其後陳后廢,而淮南王反,誅。

八月,長星出于東方,長終天,三十日去。占曰:「是為蚩尤旗,見則王者征伐四方。」其後兵誅四夷,連數十年。

元狩四年四月,長星又出西北,是時伐胡尤甚。

元封元年五月,有星孛于東井,又孛于三台。其後江充作亂,京師紛然。此明東井、三台為秦地效也。

宣帝地節元年正月,有星孛于西方,去太白二丈所。劉向以為太白為大將,彗孛加之,掃滅象也。明年,大將軍霍光薨,後二年家夷滅。

成帝建始元年正月,有星孛于營室,青白色,長六七丈,廣尺餘。劉向、谷永以為營室為後宮懷任之象,彗星加之,將有害懷任絕繼嗣者。一曰,後宮將受害也。其後許皇后坐祝詛後宮懷任者廢。趙皇后立妹為昭儀,害兩皇子,上遂無嗣。趙后姊妹卒皆伏辜。

元延元年七月辛未,有星孛于東井,踐五諸侯,出河戍北率行軒轅、太微,後日六度有餘,晨出東方。十三日夕見西方,犯次妃、長秋、斗、填,蜂炎再貫紫宮中。大火當後,達天河,除於妃后之域。南逝度犯大角、攝提,至天市而按節徐行,炎入市,中旬而後西去,五十六日與倉龍俱伏。谷永對曰:「上古以來,大亂之極,所希有也。察其馳騁驟步,芒炎或長或短,所歷奸犯,內為後宮女妾之害,外為諸夏叛逆之禍。」劉向亦曰:「三代之亡,攝提易方;秦、項之滅,星孛大角。」是歲,趙昭儀害兩皇子。後五年,成帝崩,昭儀自殺。哀帝即位,趙氏皆免官爵,徙遼西。哀帝亡嗣。平帝即位,王莽用事,追廢成帝趙皇后、哀帝傅皇后,皆自殺。外家丁、傅皆免官爵,徙合浦,歸故郡。平帝亡嗣,莽遂篡國。

釐公十六年「正月戊申朔,隕石于宋,五,是月六鶂退飛過宋都」。董仲舒、劉向以為象宋襄公欲行伯道將自敗之戒也。石陰類,五陽數,自上而隕,此陰而陽行,欲高反下也。石與金同類,色以白為主,近白祥也。鶂水鳥,六陰數,退飛,欲進反退也。其色青,青祥也,屬於貌之不恭。天戒若曰,德薄國小,勿持炕陽,欲長諸侯,與彊大爭,必受其害。襄公不寤,明年齊威死,伐齊喪,執滕子,圍曹,為盂之會,與楚爭盟,卒為所執。後得反國,不悔過自責,復會諸侯伐鄭,與楚戰于泓,軍敗身傷,為諸侯笑。左氏傳曰:隕石,星也;鶂退飛,風也。宋襄公以問周內史叔興曰:「是何祥也?吉凶何在?」對曰:「今茲魯多大喪,明年齊有亂,君將得諸侯而不終。」退而告人曰:「是陰陽之事,非吉凶之所生也。吉凶繇人,吾不敢逆君故也。」是歲,魯公子季友、鄫季姬、公孫茲皆卒。明年齊威死,適庶亂。宋襄公伐齊行伯,卒為楚所敗。劉歆以為是歲歲在壽星,其衝降婁。降婁,魯分野也,故為魯多大喪。正月,日在星紀,厭在玄枵。玄枵,齊分野也。石,山物;齊,大嶽後。五石象齊威卒而五公子作亂,故為明年齊有亂。庶民惟星,隕於宋,象宋襄將得諸侯之眾,而治五公子之亂。星隕而鶂退飛,故為得諸侯而不終。六鶂象後六年伯業始退,執於盂也。民反德為亂,亂則妖災生,言吉凶繇人,然后陰陽衝厭受其咎。齊、魯之災非君所致,故曰「吾不敢逆君故也」。京房易傳曰:「距諫自彊,茲謂卻行,厥異鶂退飛。適當黜,則鶂退飛。」

惠帝三年,隕石綿諸,一。

武帝征和四年二月丁酉,隕石雍,二,天晏亡雲,聲聞四百里。

元帝建昭元年正月戊辰,隕石梁國,六。

成帝建始四年正月癸卯,隕石槁,四,肥累,一。

陽朔三年二月壬戌,隕石白馬,八。

鴻嘉二年五月癸未,隕石杜衍,三。

元延四年三月,隕石都關,二。

哀帝建平元年正月丁未,隕石北地,十。其九月甲辰,隕石虞,二。

平帝元始二年六月,隕石鉅鹿,二。

自惠盡平,隕石凡十一,皆有光燿雷聲,成、哀尤屢。


\end{pinyinscope}