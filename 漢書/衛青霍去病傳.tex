\article{衛青霍去病傳}

\begin{pinyinscope}
衛青字仲卿。其父鄭季,河東平陽人也,以縣吏給事侯家。平陽侯曹壽尚武帝姊陽信長公主。季與主家僮衛媼通,生青。青有同母兄衛長及姊子夫,子夫自平陽公主家得幸武帝,故青冒姓為衛氏。衛媼長女君孺,次女少兒,次女則子夫。子夫男弟步廣,皆冒衛氏。

青為侯家人,少時歸其父,父使牧羊。民母之子皆奴畜之,不以為兄弟數。青嘗從人至甘泉居室,有一鉗徒相青曰:「

貴人也,官至封侯。」青笑曰:「人奴之生,得無笞罵即足矣,安得封侯事乎!」

青壯,為侯家騎,從平陽主。建元二年春,青姊子夫得入宮幸上。皇后,大長公主女也,無子,妒。大長公主聞衛子夫幸,有身,妒之,乃使人捕青。青時給事建章,未知名。大長公主執囚青,欲殺之。其友騎郎公孫敖與壯士往篡之,故得不死。上聞,乃召青為建章監,侍中。及母昆弟貴,賞賜數日間累千金。君孺為太僕公孫賀妻。少兒故與陳掌通,上召貴掌。公孫敖由此益顯。子夫為夫人。青為太中大夫。

元光六年,拜為車騎將軍,擊匈奴,出上谷;公孫賀為輕車將軍,出雲中;太中大夫公孫敖為騎將軍,出代郡;衛尉李廣為驍騎將軍,出雁門:軍各萬騎。青至籠城,斬首虜數百。騎將軍敖亡七千騎,衛尉廣為虜所得,得脫歸,皆當斬,贖為庶人。賀亦無功。唯青賜爵關內侯。是後匈奴仍侵犯邊。語在匈奴傳。

元朔元年春,衛夫人有男,立為皇后。其秋,青復將三萬騎出雁門,李息出代郡。青斬首虜數千。明年,青復出雲中,西至高闕,遂至於隴西,捕首虜數千,畜百餘萬,走白羊、樓煩王。遂取河南地為朔方郡。以三千八百戶封青為長平侯。青校尉蘇建為平陵侯,張次公為岸頭侯。使建築朔方城。上曰:「匈奴逆天理,亂人倫,暴長虐老,以盜竊為務,行詐諸蠻夷,造謀籍兵,數為邊害。故興師遣將,以征厥罪。詩不云乎?『薄伐獫允,至于太原』;『出車彭彭,城彼朔方』。今車騎將軍青度西河至高闕,獲首二千三百級,車輜畜產畢收為鹵,已封為列侯,遂西定河南地,案榆谿舊塞,絕梓領,梁北河,討蒲泥,破符離,斬輕銳之卒,捕伏聽者三千一十七級。執訊獲醜,敺馬牛羊百有餘萬,全甲兵而還,益封青三千八百戶。」其後匈奴比歲入代郡、雁門、定襄、上郡、朔方,所殺略甚眾。語在匈奴傳。

元朔五年春,令青將三萬騎出高闕,衛尉蘇建為遊擊將軍,左內史李沮為彊弩將軍,太僕公孫賀為騎將軍,代相李蔡為輕車將軍,皆領屬車騎將軍,俱出朔方。大行李息、岸頭侯張次公為將軍,俱出右北平。匈奴右賢王當青等兵,以為漢兵不能至此,飲醉,漢兵夜至,圍右賢王。右賢王驚,夜逃,獨與其愛妾一人騎數百馳,潰圍北去。漢輕騎校尉郭成等追數百里,弗得,得右賢裨王十餘人,眾男女萬五千餘人,畜數十百萬,於是引兵而還。至塞,天子使使者持大將軍印,即軍中拜青為大將軍,諸將皆以兵屬,立號而歸。上曰:「大將軍青躬率戎士,師大捷,獲匈奴王十有餘人,益封青八千七百戶。」而封青子伉為宜春侯,子不疑為陰安侯,子登為發干侯。青固謝曰:「臣幸得待罪行間,賴陛下神靈,軍大捷,皆諸校力戰之功也。陛下幸已益封臣青,臣青子在繈褓中,未有勤勞,上幸裂地封為三侯,非臣待罪行間所以勸士力戰之意也。伉等三人何敢受封!」上曰:「我非忘諸校功也,今固且圖之。」乃詔御史曰:「護軍都尉公孫敖三從大將軍擊匈奴,常護軍傅校獲王,封敖為合騎侯。都尉韓說從大軍出窴渾,至匈奴右賢王庭,為戲下搏戰獲王,封說為龍哣侯。騎將軍賀從大將軍獲王,封賀為南窌侯。輕車將軍李蔡再從大將軍獲王,封蔡為樂安侯。校尉李朔、趙不虞、公孫戎奴各三從大將軍獲王,封朔為陟軹侯,不虞為隨成侯,戎奴為從平侯。將軍李沮、李息及校尉豆如意、中郎將綰皆有功,賜爵關內侯。沮、息、如意食邑各三百戶。」其秋,匈奴入代,殺都尉。

明年春,大將軍青出定襄,合騎侯敖為中將軍,太僕賀為左將軍,翕侯趙信為前將軍,衛尉蘇建為右將軍,郎中令李廣為後將軍,左內史李沮為彊弩將軍,咸屬大將軍,斬首數千級而還。月餘,悉復出定襄,斬首虜萬餘人。蘇建、趙信并軍三千餘騎,獨逢單于兵,與戰一日餘,漢兵且盡。信故胡人,降為翕侯,見急,匈奴誘之,遂將其餘騎可八百奔降單于。蘇建盡亡其軍,獨以身得亡去,自歸青。青問其罪正閎、長史安、議郎周霸等:「建當云何?」霸曰:「自大將軍出,未嘗斬裨將,今建棄軍,可斬,以明將軍之威。」閎、安曰:「不然。兵法『小敵之堅,大敵之禽也。』今建以數千當單于數萬,力戰一日餘,士皆不敢有二心。自歸而斬之,是示後無反意也。不當斬。」青曰:「青幸得以胏附待罪行間,不患無威,而霸說我以明威,甚失臣意。且使臣職雖當斬將,以臣之尊寵而不敢自擅專誅於境外,其歸天子,天子自裁之,於以風為人臣不敢專權,不亦可乎?」軍吏皆曰「善」。遂囚建行在所。

是歲也,霍去病始侯。

霍去病,大將軍青姊少兒子也。其父霍仲孺先與少兒通,生去病。及衛皇后尊,少兒更為詹事陳掌妻。去病以皇后姊子,年十八為侍中。善騎射,再從大將軍。大將軍受詔,予壯士,為票姚校尉,與輕勇騎八百直棄大將軍數百里赴利,斬捕首虜過當。於是上曰:「票姚校尉去病斬首捕虜二千二十八級,得相國、當戶,斬單于大父行藉若侯產,捕季父羅姑比,再冠軍,以二千五百戶封去病為冠軍侯。上谷太守郝賢四從大將軍,捕首虜千三百級,封賢為終利侯。騎士孟已有功,賜爵關內侯,邑二百戶。」

是歲失兩將軍,亡翕侯,功不多,故青不益封。蘇建至,上弗誅,贖為庶人。青賜千金。是時王夫人方幸於上,甯乘說青曰:「將軍所以功未甚多,身食萬戶,三子皆為侯者,以皇后故也。今王夫人幸而宗族未富貴,願將軍奉所賜千金為王夫人親壽。」青以五百金為王夫人親壽。上聞,問青,青以實對。上乃拜甯乘為東海都尉。

校尉張騫從大將軍,以嘗使大夏,留匈奴中久,道軍,知善水草處,軍得以無飢渴,因前使絕國功,封騫為博望侯。

去病侯三歲,元狩三年春為票騎將軍,將萬騎出隴西,有功。上曰:「票騎將軍率戎士隃烏盭,討遫濮,涉狐奴,歷五王國,輜重人眾攝讋者弗取,幾獲單于子。轉戰六日,過焉支山千有餘里,合短兵,鏖皋蘭下,殺折蘭王,斬盧侯王,銳悍者誅,全甲獲醜,執渾邪王子及相國、都尉,捷首虜八千九百六十級,收休屠祭天金人,師率減什七,益封去病二千二百戶。」

其夏,去病與合騎侯敖俱出北地,異道。博望侯張騫、郎中令李廣俱出右北平,異道。廣將四千騎先至,騫將萬騎後。匈奴左賢王將數萬騎圍廣,廣與戰二日,死者過半,所殺亦過當。騫至,匈奴引兵去。騫坐行留,當斬,贖為庶人。而去病出北地,遂深入,合騎侯失道,不相得。去病至祁連山,捕首虜甚多。上曰:「票騎將軍涉鈞耆,濟居延,遂臻小月氏,攻祁連山,揚武乎鱳得,得單于單桓、酋涂王,及相國、都尉以眾降下者二千五百人,可謂能舍服知成而止矣。捷首虜三萬二百,獲五王,王母、單于閼氏、王子五十九人,相國、將軍、當戶、都尉六十三人,師大率減什三,益封去病五千四百戶。賜校尉從至小月氏者爵左庶長。鷹擊司馬破奴再從票騎將軍斬遫濮王,捕稽且王,右千騎將王、王母各一人,王子以下四十一人,捕虜三千三百三十人,前行捕虜千四百人,封破奴為從票侯。校尉高不識從票騎將軍捕呼于耆王王子以下十一人,捕虜千七百六十八人,封不識為宜冠侯。校尉僕多有功,封為煇渠侯。」合騎侯敖坐留不與票騎將軍會,當斬,贖為庶人。諸宿將所將士馬兵亦不如去病,去病所將常選,然亦敢深入,常與壯騎先其大軍,軍亦有天幸,未嘗困絕也。然而諸宿將常留落不耦。由此去病日以親貴,比大將軍。

其後,單于怒渾邪王居西方數為漢所破,亡數萬人,以票騎之兵也,欲召誅渾邪王。渾邪王與休屠王等謀欲降漢,使人先要道邊。是時大行李息將城河上,得渾邪王使,即馳傳以聞。上恐其以詐降而襲邊,乃令去病將兵往迎之。去病既度河,與渾邪眾相望。渾邪裨王將見漢軍而多欲不降者,頗遁去。去病乃馳入,得與渾邪王相見,斬其欲亡者八千人,遂獨遣渾邪王乘傳先詣行在所,盡將其眾度河,降者數萬人,號稱十萬。既至長安,天子所以賞賜數十鉅萬。封渾邪王萬戶,為漯陰侯。封其裨王呼毒尼為下摩侯,雁疪為煇渠侯,禽黎為河綦侯,大當戶調雖為常樂侯。於是上嘉去病之功,曰:「票騎將軍去病率師征匈奴,西域王渾邪王及厥眾萌咸奔於率,以軍糧接食,并將控弦萬有餘人,誅獟悍,捷首虜八千餘級,降異國之王三十二。戰士不離傷,十萬之眾畢懷集服。仍興之勞,爰及河塞,庶幾亡患。以千七百戶益封票騎將軍。減隴西、北地、上郡戍卒之半,以寬天下繇役。」乃分處降者於邊五郡故塞外,而皆在河南,因其故俗為屬國。其明年,匈奴入右北平、定襄,殺略漢千餘人。

其明年,上與諸將議曰:「翕侯趙信為單于畫計,常以為漢兵不能度幕輕留,今大發卒,其勢必得所欲。」是歲元狩四年也。春,上令大將軍青、票騎將軍去病各五萬騎,步兵轉者踵軍數十萬,而敢力戰深入之士皆屬去病。去病始為出定襄,當單于。捕虜,虜言單于東,乃更令去病出代郡,令青出定襄。郎中令李廣為前將軍,太僕公孫賀為左將軍,主爵趙食其為右將軍,平陽侯襄為後將軍,皆屬大將軍。趙信為單于謀曰:「漢兵即度幕,人馬罷,匈奴可坐收虜耳。」乃悉遠北其輜重,皆以精兵待幕北。而適直青軍出塞千餘里,見單于兵陳而待,於是青令武剛車自環為營,而縱五千騎往當匈奴,匈奴亦從萬騎。會日且入,而大風起,沙礫擊面,兩軍不相見,漢益縱左右翼繞單于。單于視漢兵多,而士馬尚彊,戰而匈奴不利,薄莫,單于遂乘六臝,壯騎可數百,直冒漢圍西北馳去。昏,漢匈奴相紛挐,殺傷大當。漢軍左校捕虜,言單于未昏而去,漢軍因發輕騎夜追之,青因隨其後。匈奴兵亦散走。會明,行二百餘里,不得單于,頗捕斬首虜萬餘級,遂至窴顏山趙信城,得匈奴積粟食軍。軍留一日而還,悉燒其城餘粟以歸。

青之與單于會也,而前將軍廣、右將軍食其軍別從東道,或失道。大將軍引還,過幕南,乃相逢。青欲使使歸報,令長史簿責廣,廣自殺。食其贖為庶人。青軍入塞,凡斬首虜萬九千級。

是時匈奴眾失單于十餘日,右谷蠡王自立為單于。單于後得其眾,右王乃去單于之號。

去病騎兵車重與大將軍軍等,而亡裨將。悉以李敢等為大校,當裨將,出代、右北平二千餘里,直左方兵,所斬捕功已多於青。

既皆還,上曰:「票騎將軍去病率師躬將所獲葷允之士,約輕齎,絕大幕,涉獲單于章渠,以誅北車耆,轉擊左大將雙,獲旗鼓,歷度難侯,濟弓盧,獲屯頭王、韓王等三人,將軍、相國、當戶、都尉八十三人,封狼居胥山,禪於姑衍,登臨翰海,執訊獲醜七萬有四百四十三級,師率減什二,取食於敵,卓行殊遠而糧不絕。以五千八百戶益封票騎將軍。右北平太守路博德屬票騎將軍,會興城,不失期,從至檮余山,斬首捕虜二千八百級,封博德為邳離侯。北地都尉衛山從票騎將軍獲王,封山為義陽侯。故歸義侯因淳王復陸支、樓剸王伊即靬皆從票騎將軍有功,封復陸支為杜侯,伊即靬為眾利侯。從票侯破奴、昌武侯安稽從票騎有功,益封各三百戶。漁陽太守解、校尉敢皆獲鼓旗,賜爵關內侯,解食邑三百戶,敢二百戶。校尉自為爵左庶長。」軍吏卒為官,賞賜甚多。而青不得益封,吏卒無封者。唯西河太守常惠、雲中太守遂成受賞,遂成秩諸侯相,賜食邑二百戶,黃金百斤,惠爵關內侯。

兩軍之出塞,塞閱官及私馬凡十四萬匹,而後入塞者不滿三萬匹。乃置大司馬位,大將軍、票騎將軍皆為大司馬。定令,令票騎將軍秩祿與大將軍等。自是後,青日衰而去病日益貴。青故人門下多去事去病,輒得官爵,唯獨任安不肯去。

去病為人少言不泄,有氣敢往。上嘗欲教之吳孫兵法,對曰:「顧方略何如耳,不至學古兵法。」上為治第,令視之,對曰:「匈奴不滅,無以家為也。」由此上益重愛之。然少而侍中,貴不省士。其從軍,上為遣太官齎數十乘,既還,重車餘棄粱肉,而士有飢者。其在塞外,卒乏糧,或不能自振,而去病尚穿域鸲鞠也。事多此類。青仁,喜士退讓,以和柔自媚於上,然於天下未有稱也。

去病自四年軍後三歲,元狩六年薨。上悼之,發屬國玄甲,軍陳自長安至茂陵,為冢象祁連山。諡之并武與廣地曰景桓侯。子嬗嗣。嬗字子侯,上愛之,幸其壯而將之。為奉車都尉,從封泰山而薨。無子,國除。

自去病死後,青長子宜春侯伉坐法失侯。後五歲,伉弟二人,陰安侯不疑、發干侯登,皆坐酎伉失侯。後二歲,冠軍侯國絕。後四年,元封五年,青薨,諡曰烈侯。子伉嗣,六年坐法免。

自青圍單于後十四歲而卒,竟不復擊匈奴者,以漢馬少,又方南誅兩越,東伐朝鮮,擊羌、西南夷,以故久不伐胡。

初,青既尊貴,而平陽侯曹壽有惡疾就國,長公主問:「列侯誰賢者?」左右皆言大將軍。主笑曰:「此出吾家,常騎從我,柰何?」左右曰:「於今尊貴無比。」於是長公主風白皇后,皇后言之,上乃詔青尚平陽主,與主合葬,起冢象廬山云。

最大將軍青凡七出擊匈奴,斬捕首虜五萬餘級。一與單于戰,收河南地,置朔方郡。再益封,凡萬六千三百戶;封三子為侯,侯千三百戶,并之二萬二百戶。其裨將及校尉侯者九人,為特將者十五人,李廣、張騫、公孫賀、李蔡、曹襄、韓說、蘇建皆自有傳。

李息,郁郅人也,事景帝。至武帝立八歲,為材官將軍,軍馬邑;後六歲,為將軍,出代;後三歲,為將軍,從大將軍出朔方:皆無功。凡三為將軍,其後常為大行。

公孫敖,義渠人,以郎事景帝。至武帝立十二歲,為騎將軍,出代,亡卒七千人,當斬,贖為庶人。後五歲,以校尉從大將軍,封合騎侯。後一歲,以中將軍從大將軍再出定襄,無功。後二歲,以將軍出北地,後票騎,失期當斬,贖為庶人。後二歲,以校尉從大將軍,無功。後十四歲,以因杆將軍築受降城。七歲,復以因杅將軍再出擊匈奴,至余吾,亡士多,下吏,當斬,詐死,亡居民間五六歲。後覺,復繫。坐妻為巫蠱,族。凡四為將軍。

李沮,雲中人,事景帝。武帝立十七歲,以左內史為彊弩將軍。後一歲,復為彊弩將軍。

張次公,河東人,以校尉從大將軍,封岸頭侯。其後太后崩,為將軍,軍北軍。後一歲,復從大將軍。凡再為將軍,後坐法失侯。

趙信,以匈奴相國降,為侯。武帝立十八年,為前將軍,與匈奴戰,敗,降匈奴。

趙食其,祋祤人。武帝立十八年,以主爵都尉從大將軍,斬首六百六十級。元狩三年,賜爵關內侯,黃金百斤。明年,為右將軍,從大將軍出定襄,迷失道,當斬,贖為庶人。

郭昌,雲中人,以校尉從大將軍。元封四年,以太中大夫為拔胡將軍,屯朔方。還擊昆明,無功,奪印。

荀彘,太原廣武人,以御見,侍中,用校尉數從大將軍。元封三年,為左將軍擊朝鮮,無功,坐捕樓船將軍誅。

最票騎將軍去病凡六出擊匈奴,其四出以將軍,斬首虜十一萬餘級。渾邪王以眾降數萬,開河西酒泉之地,西方益少胡寇。四益封,凡萬七千七百戶。其校尉吏有功侯者六人,為將軍者二人。

路博德,西河平州人,以右北平太守從票騎將軍,封邳離侯。票騎死後,博德以衛尉為伏波將軍,伐破南越,益封。其後坐法失侯。為彊弩都尉,屯居延,卒。

趙破奴,太原人。嘗亡入匈奴,已而歸漢,為票騎將軍司馬。出北地,封從票侯,坐酎金失侯。後一歲,為匈河將軍,攻胡至匈河水,無功。後一歲,擊虜樓蘭王,後為浞野侯。後六歲,以浚稽將軍將二萬騎擊匈奴左王。左王與戰,兵八萬騎圍破奴,破奴為虜所得,遂沒其軍。居匈奴中十歲,復與其太子安國亡入漢。後坐巫蠱,族。

自衛氏興,大將軍青首封,其後支屬五人為侯。凡二十四歲而五侯皆奪國。征和中,戾太子敗,衛氏遂滅。而霍去病弟光貴盛,自有傳。

贊曰:蘇建嘗說責「大將軍至尊重,而天下之賢士大夫無稱焉,願將軍觀古名將所招選者,勉之哉!」青謝曰:「自魏其、武安之厚賓客,天子常切齒,彼親待士大夫,招賢黜不肖者,人主之柄也。人臣奉法遵職而已,何與招士!」票騎亦方此意,為將如此。


\end{pinyinscope}