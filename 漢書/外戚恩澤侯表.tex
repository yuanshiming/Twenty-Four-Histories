\article{外戚恩澤侯表}

\begin{pinyinscope}
自古受命及中興之君,必興滅繼絕,修廢舉逸,然後天下歸仁,四方之政行焉。傳稱武王克殷,追存賢聖,至乎不及下車。世代雖殊,其揆一也。高帝撥亂誅暴,庶事草創,日不暇給,然猶修祀六國,求聘四皓,過魏則寵無忌之墓,適趙則封樂毅之後。及其行賞而授位也,爵以功為先後,官用能為次序。後嗣共己遵業,舊臣繼踵居位。至乎孝武,元功宿將略盡。會上亦興文學,進拔幽隱,公孫弘自海瀕而登宰相,於是寵以列侯之爵。又疇咨前代,詢問耆老,初得周後,復加爵邑。自是之後,宰相畢侯矣。元、成之間,晚得殷世,以備賓位。

漢興,外戚與定天下,侯者二人。故誓曰:「非劉氏不王,若有亡功非上所置而侯者,天下共誅之。」是以高后欲王諸呂,王陵廷爭;孝景將侯王氏,脩侯犯色。卒用廢黜。是後薄昭、竇嬰、上官、衛、霍之侯,以功受爵。其餘后父據春秋褒紀之義,帝舅緣大雅申伯之意,寖廣博矣。是以別而敘之。

號諡姓名侯狀戶數始封子孫曾孫玄孫


\end{pinyinscope}