\article{匈奴傳}

\begin{pinyinscope}
匈奴,其先夏后氏之苗裔,曰淳維。唐虞以上有山戎、獫允、薰粥,居于北邊,隨草畜牧而轉移。其畜之所多則馬、牛、羊,其奇畜則橐佗、驢、执、駃騠、騊駼、驒奚。逐水草遷徙,無城郭常居耕田之業,然亦各有分地。無文書,以言語為約束。兒能騎羊,引弓射鳥鼠,少長則射狐菟,肉食。士力能彎弓,盡為甲騎。其俗,寬則隨畜田獵禽獸為生業,急則人習戰攻以侵伐,其天性也。其長兵則弓矢,短兵則刀鋋。利則進,不利則退,不羞遁走。苟利所在,不知禮義。自君王以下咸食畜肉,衣其皮革,被旃裘。壯者食肥美,老者飲食其餘。貴壯健,賤老弱。父死,妻其後母;兄弟死,皆取其妻妻之。其俗有名不諱而無字。

夏道衰,而公劉失其稷官,變于西戎,邑于豳。其後三百有餘歲,戎狄攻太王亶父,亶父亡走于岐下,豳人悉從亶父而邑焉,作周。其後百有餘歲,周西伯昌伐畎夷。後十有餘年,武王伐紂而營雒邑,復居于酆鎬,放逐戎夷涇、洛之北,以時入貢,名曰荒服。其後二百有餘年,周道衰,而周穆王伐畎戎,得四白狼四白鹿以歸。自是之後,荒服不至。於是作呂刑之辟。至穆王之孫懿王時,王室遂衰,戎狄交侵,暴虐中國。中國被其苦,詩人始作,疾而歌之,曰:「靡室靡家,獫允之故;」「豈不日戒,獫允孔棘。」至懿王曾孫宣王,興師命將以征伐之,詩人美大其功,曰:「薄伐獫狁,至於太原;」「出車彭彭」,「城彼朔方。」是時四夷賓服,稱為中興。

至于幽王,用寵姬褒姒之故,與申后有隙。申侯怒而與畎戎共攻殺幽王于麗山之下,遂取周之地,鹵獲而居于涇渭之間,侵暴中國。秦襄公救周,於是周平王去酆鎬而東徙于雒邑。當時秦襄公伐戎至廄,始列為諸侯。後六十有五年,而山戎越燕而伐齊,齊釐公與戰于齊郊。後四十四年,而山戎伐燕。燕告急齊,齊桓公北伐山戎,山戎走。後二十餘年,而戎翟至雒邑,伐周襄王,襄王出奔于鄭之氾邑。初,襄王欲伐鄭,故取翟女為后,與翟共伐鄭。已而黜翟后,翟后怨,而襄王繼母曰惠后,有子帶,欲立之,於是惠后與翟后、子帶為內應,開戎翟,戎翟以故得入,破逐襄王,而立子帶為王。於是戎翟或居於陸渾,東至于衛,侵盜尤甚。周襄王既居外四年,乃使使告急於晉。晉文公初立,欲修霸業,乃興師伐戎翟,誅子帶,迎內襄王于洛邑。

當是時,秦晉為強國。晉文公攘戎翟,居于西河圜、洛之間,號曰赤翟、白翟。而秦穆公得由余,西戎八國服於秦。故隴以西有綿諸、畎戎、狄獂之戎,在岐、梁、涇、漆之北有義渠、大荔、烏氏、朐衍之戎,而晉北有林胡、樓煩之戎,燕北有東胡、山戎。各分散谿谷,自有君長,往往而聚者百有餘戎,然莫能相壹。

自是之後百有餘年,晉悼公使魏絳和戎翟,戎翟朝晉。後百有餘年,趙襄子踰句注而破之,并代以臨胡貉。後與韓魏共滅知伯,分晉地而有之,則趙有代、句注以北,而魏有西河、上郡,以與戎界邊。其後,義渠之戎築城郭以自守,而秦稍蠶食之,至於惠王,遂拔義渠二十五城。惠王伐魏,魏盡入西河及上郡于秦。秦昭王時,義渠戎王與宣太后亂,有二子。宣太后詐而殺義渠戎王於甘泉,遂起兵伐滅義渠。於是秦有隴西、北地、上郡,築長城以距胡。而趙武靈王亦變俗胡服,習騎射,北破林胡、樓煩,自代並陰山下至高闕為塞,而置雲中、雁門、代郡。其後燕有賢將秦開,為質於胡,胡甚信之。歸而襲破東胡,卻千餘里。與荊軻刺秦王秦舞陽者,開之孫也。燕亦築長城,自造陽至襄平,置上谷、漁陽、右北平、遼西、遼東郡以距胡。當是時,冠帶戰國七,而三國邊於匈奴。其後趙將李牧時,匈奴不敢入趙邊。後秦滅六國,而始皇帝使蒙恬將數十萬之物北擊胡,悉收河南地,因河為塞,築四十四縣城臨河,徙適戍以充之。而通直道,自九原至雲陽,因邊山險,塹谿谷,可繕者繕之,起臨洮至遼東萬餘里。又度河據陽山北假中。

當是時,東胡強而月氏盛。匈奴單于曰頭曼,頭曼不勝秦,北徙。十有餘年而蒙恬死,諸侯畔秦,中國擾亂,諸秦所徙適邊者皆復去,於是匈奴得寬,復稍度河南與中國界於故塞。

單于有太子,名曰冒頓。後有愛閼氏,生少子,頭曼欲廢冒頓而立少子,乃使冒頓質於月氏。冒頓既質,而頭曼急擊月氏。月氏欲殺冒頓,冒頓盜其善馬,騎亡歸。頭曼以為壯,令將萬騎。冒頓乃作鳴鏑,習勒其騎射,令曰:「鳴鏑所射而不悉射者斬。」行獵獸,有不射鳴鏑所射輒斬之。已而,冒頓以鳴鏑自射善馬,左右或莫敢射,冒頓立斬之。居頃之,復以鳴鏑自射其愛妻,左右或頗恐,不敢射,復斬之。頃之,冒頓出獵,以鳴鏑射單于善馬,左右皆射之。於是冒頓知其左右可用,從其父單于頭曼獵,以鳴鏑射頭曼,其左右皆隨鳴鏑而射殺頭曼,盡誅其後母與弟及大臣不聽從者。於是冒頓自立為單于。

冒頓既立,時東胡強,聞冒頓殺父自立,乃使使謂冒頓曰:「欲得頭曼時號千里馬。」冒頓問群臣,群臣皆曰:「此匈奴寶馬也,勿予。」冒頓曰:「奈何與人鄰國愛一馬乎?」遂與之。頃之,東胡以為冒頓畏之,使使謂冒頓曰:「欲得單于一閼氏。」冒頓復問左右,左右皆怒曰:「東胡無道,乃求閼氏!請擊之。」冒頓曰:「奈何與人鄰國愛一女子乎?」遂取所愛閼氏予東胡。東胡王愈驕,西侵。與匈奴中間有棄地莫居千餘里,各居其邊為甌脫。東胡使使謂冒頓曰:「匈奴所與我界甌脫外棄地,匈奴不能至也,吾欲有之。」冒頓問群臣,或曰:「此棄地,予之。」於是冒頓大怒,曰:「地者,國之本也,奈何予人!」諸言與者,皆斬之。冒頓上馬,令國中有後者斬,遂東襲擊東胡。東胡初輕冒頓,不為備。及冒頓以兵至,大破滅東胡王,虜其民眾畜產。既歸,西擊走月氏,南并樓煩、白羊河南王,悉復收秦所使蒙恬所奪匈奴地者,與漢關故河南塞,至朝那、膚施,遂侵燕、代。是時漢方與項羽相距,中國罷於兵革,以故冒頓得自強,控弦之士三十餘萬。

自淳維以至頭曼千有餘歲,時大時小,別散分離,尚矣,其世傳不可得而次。然至冒頓,而匈奴最強大,盡服從北夷,而南與諸夏為敵國,其世信官號可得而記云。

單于姓攣鞮氏,其國稱之曰「撐犁孤塗單于」。匈奴謂天為「撐犁」,謂子為「孤塗」,單于者,廣大之貌也,言其象天單于然也。置左右賢王,左右谷蠡,左右大將,左右大都尉,左右大當戶,左右骨都侯。匈奴謂賢曰「屠耆」,故常以太子為左屠耆王。自左右賢王以下至當戶,大者萬餘騎,小者數千,凡二十四長,立號曰「萬騎」。其大臣皆世官。呼衍氏,蘭氏,其後有須卜氏,此三姓,其貴種也。諸左王將居東方,直上谷以東,接穢貉、朝鮮;右王將居西方,直上郡以西,接氐、羌;而單于庭直代、雲中。各有分地,逐水草移徙。而左右賢王、左右谷蠡最大國,左右骨都侯輔政。諸二十四長,亦各自置千長、百長、什長、裨小王、相、都尉、當戶、且渠之屬。

歲正月,諸長小會單于庭,祠。五月,大會龍城,祭其先、天地、鬼神。秋,馬肥,大會蹛林,課校人畜計。其法,拔刃尺者死,坐盜者沒入其家;有罪,小者軋,大者死。獄久者不滿十日,一國之囚不過數人。而單于朝出營,拜日之始生,夕拜月。其坐,長左而北向。日上戊已。其送死,有棺槨金銀衣裳,而無封樹喪服;近幸臣妾從死者,多至數十百人。舉事常隨月,盛壯以攻戰,月虧則退兵。其攻戰,斬首虜賜一卮酒,而所得鹵獲因以予之,得人以為奴婢。故其戰,人人自為趨利,善為誘兵以包敵。故其逐利,如鳥之集;其困敗,瓦解雲散矣。戰而扶轝死者,盡得死者家財。

後北服渾窳、屈射、丁零、隔昆、龍新赚之國。於是匈奴貴人大臣皆服,以冒頓為賢。

是時,漢初定,徙韓王信於代,都馬邑。匈奴大攻圍馬邑,韓信降匈奴。匈奴得信,因引兵南踰句注,攻太原,至晉陽下。高帝自將兵往擊之。會冬大寒雨雪,卒之墮指者十二三,於是冒頓陽敗走,誘漢兵。漢兵逐擊冒頓,冒頓匿其精兵,見其羸弱,於是漢悉兵,多步兵,三十二萬,北逐之。高帝先至平城,步兵未盡到,冒頓縱精兵三十餘萬騎圍高帝於白登,七日,漢兵中外不得相救餉。匈奴騎,其西方盡白,東方盡駹,北方盡驪,南方盡騂馬。高帝乃使使間厚遺閼氏,閼氏乃謂冒頓曰:「兩主不相困。今得漢地,單于終非能居之。且漢主有神,單于察之。」冒頓與韓信將王黃、趙利期,而兵久不來,疑其與漢有謀,亦取閼氏之言,乃開圍一角。於是高皇帝令士皆持滿傅矢外鄉,從解角直出,得與大軍合,而冒頓遂引兵去。漢亦引兵罷,使劉敬結和親之約。

是後韓信為匈奴將,及趙利、王黃等數背約,侵盜代、鴈門、雲中。居無幾何,陳豨反,與韓信合謀擊代。漢使樊噲往擊之,復收代、鴈門、雲中郡縣,不出塞。是時匈奴以漢將數率眾往降,故冒頓常往來侵盜代地。於是高祖患之,乃使劉敬奉宗室女翁主為單于閼氏,歲奉匈奴絮繒酒食物各有數,約為兄弟以和親,冒頓乃少止。後燕王盧綰復反,率其黨且萬人降匈奴,往來苦上谷以東,終高祖世。

孝惠、高后時,冒頓寖驕,乃為書,使使遺高后曰:「孤僨之君,生於沮澤之中,長於平野牛馬之域,數至邊境,願遊中國。陛下獨立,孤僨獨居。兩主不樂,無以自虞,願以所有,易其所無。」高后大怒,召丞相平及樊噲、季布等,議斬其使者,發兵而擊之。樊噲曰:「臣願得十萬眾,橫行匈奴中。」問季布,布曰:「噲可斬也!前陳豨反於代,漢兵三十二萬,噲為上將軍,時匈奴圍高帝於平城,噲不能解圍。天下歌之曰:『平城之下亦誠苦!七日不食,不能彀弩。』今歌吟之聲未絕,傷痍者甫起,而噲欲搖動天下,妄言以十萬眾橫行,是面謾也。且夷狄譬如禽獸,得其善言不足喜,惡言不足怒也。」高后曰:「善。」令大謁者張澤報書曰:「單于不忘弊邑,賜之以書,弊邑恐懼。退日自圖,年老氣衰,髮齒墮落,行步失度,單于過聽,不足以自汙。弊邑無罪,宜在見赦。竊有御車二乘,馬二駟,以奉常駕。」冒頓得書,復使使來謝曰:「未嘗聞中國禮義,陛下幸而赦之。」因獻馬,遂和親。

至孝文即位,復修和親。其三年夏,匈奴右賢王入居河南地為寇,於是文帝下詔曰:「漢與匈奴約為昆弟,無侵害邊境,所以輸遺匈奴甚厚。今右賢王離其國,將眾居河南地,非常故。往來入塞,捕殺吏卒,敺侵上郡保塞蠻夷,令不得居其故。陵轢邊吏,入盜,甚驁無道,非約也。其發邊吏車騎八萬詣高奴,遣丞相灌嬰將擊右賢王。」右賢王走出塞,文帝幸太原。是時,濟北王反,文帝歸,罷丞相擊胡之兵。

其明年,單于遺漢書曰:「天所立匈奴大單于敬問皇帝無恙。前時皇帝言和親事,稱書意合驩。漢邊吏侵侮右賢王,右賢王不請,聽後義盧侯難支等計,與漢吏相恨,絕二主之約,離昆弟之親。皇帝讓書再至,發使以書報,不來,漢使不至。漢以其故不和,鄰國不附。今以少吏之敗約,故罰右賢王,使至西方求月氏擊之。以天之福,吏卒良,馬力強,以滅夷月氏,盡斬殺降下定之。樓蘭、烏孫、呼揭及其旁二十六國皆已為匈奴。諸引弓之民并為一家,北州以定。願寢兵休士養馬,除前事,復故約,以安邊民,以應古始,使少者得成其長,老者得安其處,世世平樂。未得皇帝之志,故使郎中係虖淺奉書請,獻橐佗一,騎馬二,駕二駟。皇帝即不欲匈奴近塞,則且詔吏民遠舍。使者至,即遣之。」六月中,來至新望之地。書至,漢議擊與和親孰便,公卿皆曰:「單于新破月氏,乘勝,不可擊也。且得匈奴地,澤鹵非可居也,和親甚便。」漢許之。

孝文前六年,遺匈奴書曰:「皇帝敬問匈奴大單于無恙。使係虖淺遺朕書,云『願寢兵休事,除前事,復故約,以安邊民,世世平樂』,朕甚嘉之。此古聖王之志也。漢與匈奴約為兄弟,所以遺單于甚厚。背約離兄弟之親者,常在匈奴。然右賢王事已在赦前,勿深誅。單于若稱書意,明告諸吏,使無負約,有信,敬如單于書。使者言單于自將并國有功,甚苦兵事。服繡袷綺衣、長襦、錦袍各一,比疏一,黃金飭具帶一,黃金犀毗一,繡十匹,錦二十匹,赤綈、綠繒各四十匹,使中大夫意、謁者令肩遺單于。」

後頃之,冒頓死,子稽粥立,號曰老上單于。

老上稽粥單于初立,文帝復遣宗人女翁主為單于閼氏,使宦者燕人中行說傅翁主。說不欲行,漢強使之。說曰:「必我也,為漢患者。」中行說既至,因降單于,單于愛幸之。

初,單于好漢繒絮食物,中行說曰:「匈奴人眾不能當漢之一郡,然所以強之者,以衣食異,無卬於漢。今單于變俗好漢物,漢物不過什二,則匈奴盡歸於漢矣。其得漢絮繒,以馳草棘中,衣苎皆裂弊,以視不如旃裘堅善也;得漢食物皆去之,以視不如重酪之便美也。」於是說教單于左右疏記,以計識其人眾畜牧。

漢遺單于書,以尺一牘,辭曰「皇帝敬問匈奴大單于無恙」,所以遺物及言語云云。中行說令單于以尺二寸牘,及印封皆令廣長大,倨驁其辭曰「天地所生日月所置匈奴大單于敬問漢皇帝無恙」,所以遺物言語亦云云。

漢使或言匈奴俗賤老,中行說窮漢使曰:「而漢俗屯戍從軍當發者,其親豈不自奪溫厚肥美齎送飲食行者乎?」漢使曰:「然。」說曰:「匈奴明以攻戰為事,老弱不能鬥,故以其肥美飲食壯健以自衛,如此父子各得相保,何以言匈奴輕老也?」漢使曰:「匈奴父子同穹廬臥。父死,妻其後母;兄弟死,盡妻其妻。無冠帶之節,闕庭之禮。」中行說曰:「匈奴之俗,食畜肉,飲其汁,衣其皮;畜食草飲水,隨時轉移。故其急則人習騎射,寬則人樂無事。約束徑,易行;君臣簡,可久。一國之政猶一體也。父兄死,則妻其妻,惡種姓之失也。故匈奴雖亂,必立宗種。今中國雖陽不取其父兄之妻,親屬益疏則相殺,至到易姓,皆從此類也。且禮義之敝,上下交怨,而室屋之極,生力屈焉。夫力耕桑以求衣食,築城郭以自備,故其民急則不習戰攻,緩則罷於作業。嗟土室之人,顧無喋喋佔佔,冠固何當!」自是之後,漢使欲辯論者,中行說輒曰:「漢使毋多言,顧漢所輸匈奴繒絮米糱,令其量中,必善美而已,何以言為乎?且所給備善則已,不備善而苦惡,則候秋孰,以騎馳蹂乃稼穡也。」日夜教單于候利害處。

孝文十四年,匈奴單于十四萬騎入朝那蕭關,殺北地都尉卬,虜人民畜產甚多,遂至彭陽。使騎兵入燒回中宮,候騎至雍甘泉。於是文帝以中尉周舍、郎中令張武為將軍,發車千乘,十萬騎,軍長安旁以備胡寇。而拜昌侯盧卿為上郡將軍,甯侯魏遫為北地將軍,隆慮侯周灶為隴西將軍,東陽侯張相如為大將軍,成侯董赤為將軍,大發車騎往擊胡。單于留塞內月餘,漢逐出塞即還,不能有所殺。匈奴日以驕,歲入邊,殺略人民甚眾,雲中、遼東最甚,郡萬餘人。漢甚患之,乃使使遺匈奴書,單于亦使當戶報謝,復言和親事。

孝文後二年,使使遣匈奴書曰:「皇帝敬問匈奴大單于無恙。使當戶且渠雕渠難、郎中韓遼遺朕馬二匹,已至,敬受。先帝制,長城以北引弓之國受令單于,長城以內冠帶之室朕亦制之,使萬民耕織,射獵衣食,父子毋離,臣主相安,居無暴虐。今聞渫惡民貪降其趨,背義絕約,忘萬民之命,離兩主之驩,然其事已在前矣。《書》云『二國已和親,兩主驩說,寢兵休卒養馬,世世昌樂,翕然更始』,朕甚嘉之。聖者日新,改作更始,使老者得息,幼者得長,各保其首領,而終其天年。朕與單于俱由此道,順天恤民,世世相傳,施之無窮,天下莫不咸嘉。使漢與匈奴鄰敵之國,匈奴處北地,寒,殺氣早降,故詔吏遺單于秫糱金帛綿絮它物歲有數。今天下大安,萬民熙熙,獨朕與單于為之父母。朕追念前事,薄物細故,謀臣計失,皆不足以離昆弟之驩。朕聞天不頗覆,地不偏載。朕與單于皆捐細故,俱蹈大道也,墮壞前惡,以圖長久,使兩國之民若一家子。元元萬民,下及魚鱉,上及飛鳥,跂行喙息蝡動之類,莫不就安利,避危殆。故來者不止,天之道也。俱去前事,朕釋逃虜民,單于毋言章尼等。朕聞古之帝王,約分明而不食言。單于留志,天下大安,和親之後,漢過不先。單于其察之。」

單于既約和親,於是制詔御史:「匈奴大單于遺朕書,和親已定,亡人不足以益眾廣地,匈奴無入塞,漢無出塞,犯今約者殺之,可以久親,後無咎,俱便。朕已許。其布告天下,使明知之。」

後四年,老上單于死,子軍臣單于立,而中行說復事之。漢復與匈奴和親。

軍臣單于立歲餘,匈奴復絕和親,大入上郡、雲中各三萬騎,所殺略甚眾。於是漢使三將軍軍屯北地,代屯句注,趙屯飛狐口,緣邊亦各堅守以備胡寇。又置三將軍,軍長安西細柳、渭北棘門、霸上以備胡。胡騎入代句注邊,烽火通於甘泉、長安。數月,漢兵至邊,匈奴亦遠塞,漢兵亦罷。後歲餘,文帝崩,景帝立,而趙王遂乃陰使於匈奴。吳楚反,欲與趙合謀入邊。漢圍破趙,匈奴亦止。自是後,景帝復與匈奴和親,通關市,給遺單于,遣翁主如故約。終景帝世,時時小入盜邊,無大寇。

武帝即位,明和親約束,厚遇關市,饒給之。匈奴自單于以下皆親漢,往來長城下。

漢使馬邑人聶翁壹間闌出物與匈奴交易,陽為賣馬邑城以誘單于。單于信之,而貪馬邑財物,乃以十萬騎入武州塞。漢伏兵三十餘萬馬邑旁,御史大夫韓安國為護軍將軍,護四將軍以伏單于。單于既入漢塞,未至馬邑百餘里,見畜布野而無人牧者,怪之,乃攻亭。時雁門尉史行徼,見寇,保此亭,單于得,欲刺之。尉史知漢謀,乃下,具告單于。單于大驚,曰:「吾固疑之。」乃引兵還。出曰:「吾得尉史,天也。」以尉史為天王。漢兵約單于入馬邑而縱兵,單于不至,以故無所得。將軍王恢部出代擊胡輜重,聞單于還,兵多,不敢出。漢以恢本建造兵謀而不進,誅恢。自是後,匈奴絕和親,攻當路塞,往往入盜於邊,不可勝數。然匈奴貪,尚樂關市,耆漢財物,漢亦通關市不絕以中之。

自馬邑軍後五歲之秋,漢使四將各萬騎擊胡關市下。將軍衛青出上谷,至龍城,得胡首虜七百人。公孫賀出雲中,無所得。公孫敖出代郡,為胡所敗七千。李廣出雁門,為胡所敗,匈奴生得廣,廣道亡歸。漢囚敖、廣,敖、廣贖為庶人。其冬,匈奴數千人盜邊,漁陽尤甚。漢使將軍韓安國屯漁陽備胡。其明年秋,匈奴二萬騎入漢,殺遼西太守,略二千餘人。又敗漁陽太守軍千餘人,圍將軍安國。安國時千餘騎亦且盡,會燕救之,至,匈奴乃去,又入雁門殺略千餘人。於是漢使將軍衛青將三萬騎出雁門,李息出代郡,擊胡,得首虜數千。其明年,衛青復出雲中以西至隴西,擊胡之樓煩、白羊王於河南,得胡首虜數千,羊百餘萬。於是漢遂取河南地,築朔方,復繕故秦時蒙恬所為塞,因河而為固。漢亦棄上谷之斗辟縣造陽地以予胡。是歲,元朔二年也。

其後冬,軍臣單于死,其弟左谷蠡王伊峺斜自立為單于,攻敗軍臣單于太子於單。於單亡降漢,漢封於單為陟安侯,數月死。

伊峺斜單于既立,其夏,匈奴數萬騎入代郡,殺太守共友,略千餘人。秋,又入鴈門,殺略千餘人。其明年,又入代郡、定襄、上郡,各三萬騎,殺略數千人。匈奴右賢王怨漢奪之河南地而築朔方,數寇盜邊,及入河南,侵擾朔方,殺略吏民甚眾。

其明年春,漢遣衛青將六將軍十餘萬人出朔方高闕。右賢王以為漢兵不能至,飲酒醉。漢兵出塞六七百里,夜圍右賢王。右賢王大驚,脫身逃走,精騎往往隨後去。漢將軍得右賢王人眾男女萬五千人,裨小王十餘人。其秋,匈奴萬騎入代郡,殺都尉朱央,略千餘人。

其明年春,漢復遣大將軍衛青將六將軍,十餘萬騎,仍再出定襄數百里擊匈奴,得首虜前後萬九千餘級,而漢亦亡兩將軍,三千餘騎。右將軍建得以身脫,而前將軍翕侯趙信兵不利,降匈奴。趙信者,故胡小王,降漢,漢封為翕侯,以前將軍與右將軍并軍,介獨遇單于兵,故盡沒。單于既得翕侯,以為自次王,用其姊妻之,與謀漢。信教單于益北絕幕,以誘罷漢兵,徼極而取之,毋近塞。單于從之。其明年,胡數萬騎入上谷,殺數百人。

其明年春,漢復遣大將軍衛青將六將軍,十餘萬騎,仍再出定襄數百里擊匈奴,得……

明年春,漢使票騎將軍去病將萬騎出隴西,過焉耆山千餘里,得胡首虜八千餘級,得休屠王祭天金人。其夏,票騎將軍復與合騎侯數萬騎出隴西、北地二千里,過居延,攻祁連山,得胡首虜三萬餘級,裨小王以下十餘人。是時,匈奴亦來入代郡、鴈門,殺略數百人。漢使博望侯及李將軍廣出右北平,擊匈奴左賢王。左賢王圍李廣,廣軍四千人死者過半,殺虜亦過當。會博望侯軍救至,李將軍得脫,盡亡其軍。合騎侯後票騎將軍期,及博望侯皆當死,贖為庶人。

其秋,單于怒昆邪王、休屠王居西方為漢所殺虜數萬人,欲召誅之。昆邪、休屠王恐,謀降漢,漢使票騎將軍迎之。昆邪王殺休屠王,并將其眾降漢,凡四萬餘人,號十萬。於是漢已得昆邪,則隴西、北地、河西益少胡寇,徙關東貧民處所奪匈奴河南地新秦中以實之,西減北地以西戍卒半。明年春,匈奴入右北平、定襄各數萬騎,殺略千餘人。

其年春,漢謀以為「翕侯信為單于計,居幕北,以為漢兵不能至」。乃粟馬,發十萬騎,私負從馬凡十四萬匹,糧重不與焉。令大將軍青、票騎將軍去病中分軍,大將軍出定襄,票騎將軍出代,咸約絕幕擊匈奴。單于聞之,遠甚輜重,以精兵待於幕北。與漢大將軍接戰一日,會暮,大風起,漢兵縱左右翼圍單于。單于自度戰不能與漢兵,遂獨與壯騎數百潰漢圍西北遁走。漢兵夜追之不得,行捕斬首虜凡萬九千級,北至窴顏山趙信城而還。

單于之走,其兵往往與漢軍相亂而隨單于。單于久不與其大眾相得,右谷蠡王以為單于死,乃自立為單于。真單于復得其眾,右谷蠡乃去號,復其故位。

票騎之出代二千餘里,與左王接戰,漢兵得胡首虜凡七萬餘人,左王將皆遁走。票騎封於狼居胥山,禪姑衍,臨翰海而還。

是後匈奴遠遁,而幕南無王庭。漢度河自朔方以西至令居,往往通渠置田官,吏卒五六萬人,稍蠶食,地接匈奴以北。

初,漢兩將大出圍單于,所殺虜八九萬,而漢士物故者亦萬數,漢馬死者十餘萬匹。匈奴雖病,遠去,而漢馬亦少,無以復往。單于用趙信計,遣使好辭請和親。天子下其議,或言和親,或言遂臣之。丞相長史任敞曰:「匈奴新困,宜使為外臣,朝請於邊。」漢使敞使於單于。單于聞敞計,大怒,留之不遣。先是漢亦有所降匈奴使者,單于亦輒留漢使相當。漢方復收士馬,會票騎將軍去病死,於是漢久不北擊胡。

數歲,伊稚斜單于立十三年死,子烏維立為單于。是歲,元鼎三年也。烏維單于立,而漢武帝始出巡狩郡縣。其後漢方南誅兩越,不擊匈奴,匈奴亦不入邊。

烏維立三年,漢已滅兩越,遣故太僕公孫賀將萬五千騎出九原二千餘里,至浮苴井,從票侯趙破奴萬餘騎出令居數千里,至匈奴河水,皆不見匈奴一人而還。

是時,天子巡邊,親至朔方,勒兵十八萬騎以見武節,而使郭吉風告單于。既至匈奴,匈奴主客問所使,郭吉卑體好言曰:「吾見單于而口言。」單于見吉,吉曰:「南越王頭已縣於漢北闕下。今單于即能前與漢戰,天子自將兵待邊;即不能,亟南面而臣於漢。何但遠走,亡匿於幕北寒苦無水草之地為?」語卒,單于大怒,立斬主客見者,而留郭吉不歸,遷辱之北海上。而單于終不肯為寇於漢邊,休養士馬,習射獵,數使使好辭甘言求和親。

漢使王烏等闚匈奴。匈奴法,漢使不去節,不以墨黥其面,不得入穹廬。王烏,北地人,習胡俗,去其節,黥面入廬。單于愛之,陽許曰:「吾為遣其太子入質於漢,以求和親。」

漢使楊信使於匈奴。是時漢東拔濊貉、朝鮮以為郡,而西置酒泉郡以隔絕胡與羌通之路。又西通月氏、大夏,以翁主妻烏孫王,以分匈奴西方之援國。又北益廣田至眩雷為塞,而匈奴終不敢以為言。是歲,翕侯信死,漢用事者以匈奴已弱,可臣從也。楊信為人剛直屈強,素非貴臣也,單于不親。欲召入,不肯去節,乃坐穹廬外見楊信。楊信說單于曰:「即欲和親,以單于太子為質於漢。」單于曰:「非故約。故約,漢常遣翁主,給繒絮食物有品,以和親,而匈奴亦不復擾邊。今乃欲反古,令吾太子為質,無幾矣。」匈奴俗,見漢使非中貴人,其儒生,以為欲說,折其辭辯;少年,以為欲刺,折其氣。每漢兵入匈奴,匈奴輒報償。漢留匈奴使,匈奴亦留漢使,必得當乃止。

楊信既歸,漢使王烏等如匈奴。匈奴復諂以甘言,欲多得漢財物,紿王烏曰:「吾欲入漢見天子,面相結為兄弟。」王烏歸報漢,漢為單于築邸于長安。匈奴曰:「非得漢貴人使,吾不與誠語。」匈奴使其貴人至漢,病,服藥欲愈之,不幸而死。漢使路充國佩二千石印綬,使送其喪,厚幣直數千金。單于以為漢殺吾貴使者,乃留路充國不歸。諸所言者,單于特空紿王烏,殊無意入漢,遣太子來質。於是匈奴數使奇兵侵犯漢邊。漢乃拜郭昌為拔胡將軍,及浞野侯屯朔方以東,備胡。

烏維單于立十歲死,子詹師廬立,年少,號為兒單于。是歲,元封六年也。自是後,單于益西北。左方兵直雲中,右方兵直酒泉、敦煌。

兒單于立,漢使兩使,一人弔單于,一人弔右賢王,欲以乖其國。使者入匈奴,匈奴悉將致單于。單于怒而悉留漢使。漢使留匈奴者前後十餘輩,而匈奴使來漢,亦輒留之相當。

是歲,漢使貳師將軍西伐大宛,而令因杅將軍築受降城。其冬,匈奴大雨雪,畜多飢寒死,而單于年少,好殺伐,國中多不安。左大都尉欲殺單于,使人間告漢曰:「我欲殺單于降漢,漢遠,漢即來兵近我,我即發。」初漢聞此言,故築受降城,猶以為遠。

其明年春,漢使浞野侯破奴將二萬騎出朔方北二千餘里,期至浚稽山而還。浞野侯既至期,左大都尉欲發而覺,單于誅之,發兵擊浞野侯。浞野侯行捕首虜數千人。還,未至受降城四百里,匈奴八萬騎圍之。浞野侯夜出自求水,匈奴生得浞野侯,因急擊其軍。軍吏畏亡將而誅,莫相勸而歸,軍遂沒於匈奴。單于大喜,遂遣兵攻受降城,不能下,乃侵入邊而去。明年,單于欲自攻受降城,未到,病死。

兒單于立三歲而死。子少,匈奴乃立其季父烏維單于弟右賢王句黎湖為單于。是歲,太初三年也。

句黎湖單于立,漢使光祿徐自為出五原塞數百里,遠者千里,築城障列亭至盧朐,而使游擊將軍韓說、長平侯衛伉屯其旁,使強弩都尉路博德築居延澤上。

其秋,匈奴大入雲中、定襄、五原、朔方,殺略數千人,敗數二千石而去,行壞光祿所築亭障。又使右賢王入酒泉、張掖,略數千人。會任文擊救,盡復失其所得而去。聞貳師將軍破大宛,斬其王還,單于欲遮之,不敢,其冬病死。

句黎湖單于立一歲死,其弟左大都尉且鞮侯立為單于。

漢既誅大宛,威震外國,天子意欲遂困胡,乃下詔曰:「高皇帝遺朕平城之憂,高后時單于書絕悖逆。昔齊襄公復九世之讎,春秋大之。」是歲,太初四年也。

且鞮侯單于初立,恐漢襲之,盡歸漢使之不降者路充國等於漢。單于乃自謂「我兒子,安敢望漢天子!漢天子,我丈人行。」漢遣中郎將蘇武厚幣賂遺單于,單于益驕,禮甚倨,非漢所望也。明年,浞野侯破奴得亡歸漢。

其明年,漢使貳師將軍將三萬騎出酒泉,擊右賢王於天山,得首虜萬餘級而還。匈奴大圍貳師,幾不得脫。漢兵物故什六七。漢又使因杅將軍出西河,與強弩都尉會涿邪山,亡所得。使騎都尉李陵將步兵五千人出居延北千餘里,與單于會,合戰,陵所殺傷萬餘人,兵食盡,欲歸,單于圍陵,陵降匈奴,其兵得脫歸漢者四百人。單于乃貴陵,以其女妻之。

後二歲,漢使貳師將軍六萬騎,步兵七萬,出朔方;強弩都尉路博德將萬餘人,與貳師會;游擊將軍說步兵三萬人,出五原;因杅將軍敖將騎萬,步兵三萬人,出雁門。匈奴聞,悉遠其累重於余吾水北,而單于以十萬待水南,與貳師接戰。貳師解而引歸,與單于連鬥十餘日。游擊亡所得。因杅與左賢王戰,不利,引歸。

明年,且鞮侯單于死,立五年,長子左賢王立為狐鹿姑單于。是歲,太始元年也。

初,且鞮侯兩子,長為左賢王,次為左大將,病且死,言立左賢王。左賢王未至,貴人以為有病,更立左大將為單于。左賢王聞之,不敢進。左大將使人召左賢王而讓位焉。左賢王辭以病,左大將不聽,謂曰:「即不幸死,傳之於我。」左賢王許之,遂立為狐鹿姑單于。

狐鹿姑單于立,以左大將為左賢王,數年病死,其子先賢撣不得代,更以為日逐王。日逐王者,賤於左賢王。單于自以其子為左賢王。

單于既立六年,而匈奴入上谷、五原,殺略吏民。其年,匈奴復入五原、酒泉,殺兩部都尉。於是漢遣貳師將軍七萬人出五原,御史大夫商丘成將三萬餘人出西河,重合侯莽通將四萬騎出酒泉千餘里。單于聞漢兵大出,悉遣其輜重,徙趙信城北邸郅居水。左賢王驅其人民度余吾水六七百里,居兜銜山。單于自將精兵左安侯度姑且水。

御史大夫軍至追斜徑,無所見,還。匈奴使大將與李陵將三萬餘騎追漢軍,至浚稽山合,轉戰九日,漢兵陷陳卻敵,殺傷虜甚眾。至蒲奴水,虜不利,還去。

重合侯軍至天山,匈奴使大將偃渠與左右呼知王將二萬餘騎要漢兵,見漢兵強,引去。重合侯無所得失。是時,漢恐車師兵遮重合侯,乃遣闓陵侯將兵別圍車師,盡得其王民眾而還。

貳師將軍將出塞,匈奴使右大都尉與衛律將五千騎要擊漢軍於夫羊句山狹。貳師遣屬國胡騎二千與戰,虜兵壞散,死傷者數百人。漢軍乘勝追北,至范夫人城,匈奴奔走,莫敢距敵。會貳師妻子坐巫蠱收,聞之憂懼。其掾胡亞夫亦避罪從軍,說貳師曰:「

夫人室家皆在吏,若還不稱意,適與獄會,郅居以北可復得見乎?」貳師由是狐疑,欲深入要功,遂北至郅居水上。虜已去,貳師遣護軍將二萬騎度郅居之水。一日,逢左賢王左大將,將二萬騎與漢軍合戰一日,漢軍殺左大將,虜死傷甚眾。軍長史與決眭都尉煇渠侯謀曰:「將軍懷異心,欲危眾求功,恐必敗。」謀共執貳師。貳師聞之,斬長史,引兵還至速邪烏燕然山。單于知漢軍勞倦,自將五萬騎遮擊貳師,相殺傷甚眾。夜塹漢軍前,深數尺,從後急擊之,軍大亂敗,貳師降。單于素知其漢大將貴臣,以女妻之,尊寵在衛律上。

其明年,單于遣使遺漢書云:「南有大漢,北有強胡。胡者,天之驕子也,不為小禮以自煩。今欲與漢闓大關,取漢女為妻,歲給遺我糱酒萬石,稷米五千斛,雜繒萬匹,它如故約,則邊不相盜矣。」漢遣使者報送其使,單于使左右難漢使者,曰:「漢,禮義國也。貳師道前太子發兵反,何也?」使者曰:「然。乃丞相私與太子爭鬥,太子發兵欲誅丞相,丞相誣之,故誅丞相。此子弄父兵,罪當笞,小過耳。孰與冒頓單于身殺其父代立,常妻後母,禽獸行也!」單于留使者,三歲乃得還。

貳師在匈奴歲餘,衛律害其寵,會母閼氏病,律飭胡巫言先單于怒,曰:「胡攻時祠兵,常言得貳師以社,今何故不用?」於是收貳師,貳師怒曰:「我死必滅匈奴!」遂屠貳師以祠。會連雨雪數月,畜產死,人民疫病,穀稼不孰,單于恐,為貳師立祠室。

自貳師沒後,漢新失大將軍士卒數萬人,不復出兵。三歲,武帝崩。前此者,漢兵深入窮追二十餘年,匈奴孕重墯殰,罷極苦之。自單于以下常有欲和親計。

後三年,單于欲求和親,會病死。初,單于有異母弟為左大都尉,賢,國人鄉之,母閼氏恐單于不立子而立左大都尉也,乃私使殺之。左大都尉同母兄怨,遂不肯復會單于庭。又單于病且死,謂諸貴人:「我子少,不能治國,立弟右谷蠡王。」及單于死,衛律等與顓渠閼氏謀,匿單于死,詐撟單于令,與貴人飲盟,更立子左谷蠡王為壺衍鞮單于。是歲,始元二年也。

壺衍鞮單于既立,風謂漢使者,言欲和親。左賢王、右谷蠡王以不得立怨望,率其眾欲南歸漢。恐不能自致,即脅盧屠王,欲與西降烏孫,謀擊匈奴。盧屠王告之,單于使人驗問,右谷蠡王不服,反以其罪罪盧屠王,國人皆冤之。於是二王去居其所,未嘗肯會龍城。

後二年秋,匈奴入代,殺都尉。單于年少初立,母閼氏不正,國內乖離,常恐漢兵襲之。於是衛律為單于謀「穿井築城,治樓以藏穀,與秦人守之。漢兵至,無奈我何。」即穿井數百,伐材數千。或曰胡人不能守城,是遺漢糧也,衛律於是止,乃更謀歸漢使不降者蘇武、馬宏等。馬宏者,前副光祿大夫王忠使西國,為匈奴所遮,忠戰死,馬宏生得,亦不肯降。故匈奴歸此二人,欲以通善意。是時,單于立三歲矣。

明年,匈奴發左右部二萬騎,為四隊,並入邊為寇。漢兵追之,斬首獲虜九千人,生得甌脫王,漢無所失亡。匈奴見甌脫王在漢,恐以為道擊之,即西北遠去,不敢南逐水草,發人民屯甌脫。明年,復遣九千騎屯受降城以備漢,北橋余吾,令可度,以備奔走。是時,衛律已死。衛律在時,常言和親之利,匈奴不信,及死後,兵數困,國益貧。單于弟左谷蠡王思衛律言,欲和親而恐漢不聽,故不肯先言,常使左右風漢使者。然其侵盜益希,遇漢使愈厚,欲以漸致和親,漢亦羈縻之。其後,左谷蠡王死。明年,單于使犁汙王窺邊,言酒泉、張掖兵益弱,出兵試擊,冀可復得其地。時漢先得降者,聞其計,天子詔邊警備。後無幾,右賢王、犁汙王四千騎分三隊,入日勒、屋蘭、番和。張掖太守、屬國都尉發兵擊,大破之,得脫者數百人。屬國千長義渠王騎士射殺犁汙王,賜黃金二百斤,馬二百匹,因封為犁汙王。屬國都尉郭忠封成安侯。自是後,匈奴不敢入張掖。

其明年,匈奴三千餘騎入五原,略殺數千人,後數萬騎南旁塞獵,行攻塞外亭長,略取吏民去。是時漢邊郡烽火候望精明,匈奴為邊寇者少利,希復犯塞。漢復得匈奴降者,言烏桓嘗發先單于冢,匈奴怨之,方發二萬騎擊烏桓。大將軍霍光欲發兵要擊之,以問護軍都尉趙充國。充國以為「烏桓間數犯塞,今匈奴擊之,於漢便。又匈奴希寇盜,北邊幸無事。蠻夷自相攻擊,而發兵要之,招寇生事,非計也。」光更問中郎將范明友,明友言可擊。於是拜明友為度遼將軍,將二萬騎出遼東。匈奴聞漢兵至,引去。初,光誠明友:「兵不空出,即後匈奴,遂擊烏桓。」烏桓時新中匈奴兵,明友既後匈奴,因乘烏桓敝,擊之,斬首六千餘級,獲三王首,還,封為平陵侯。

匈奴繇是恐,不能出兵。即使使之烏孫,求欲得漢公主。擊烏孫,取車延、惡師地。烏孫公主上書,下公卿議救,未決。昭帝崩,宣帝即位,烏孫昆彌復上書,言「連為匈奴所侵削,昆彌願發國半精兵人馬五萬匹,盡力擊匈奴,唯天子出兵,哀救公主!」本始二年,漢大發關東輕銳士,選郡國吏三百石伉健習騎射者,皆從軍。遣御史大夫田廣明為祈連將軍,四萬餘騎,出西河;度遼將軍范明友三萬餘騎,出張掖;前將軍韓增三萬餘騎,出雲中;後將軍趙充國為蒲類將軍,三萬餘騎,出酒泉;雲中太守田順為虎牙將軍,三萬餘騎,出五原:凡五將軍,兵十餘萬騎,出塞各二千餘里。及校尉常惠使護發兵烏孫西域,昆彌自將翕侯以下五萬餘騎從西方入,與五將軍兵凡二十餘萬眾。匈奴聞漢兵大出,老弱奔走,敺畜產遠遁逃,是以五將少所得。

度遼將軍出塞千二百餘里,至蒲離候水,斬首捕虜七百餘級,鹵獲馬牛羊萬餘。前將軍出塞千二百餘里,至烏員,斬首捕虜,至候山百餘級,鹵馬牛羊二千餘。蒲類將軍兵當與烏孫合擊匈奴蒲類澤,烏孫先期至而去,漢兵不與相及。蒲類將軍出塞千八百餘里,西去候山,斬首捕虜,得單于使者蒲陰王以下三百餘級,鹵馬牛羊七千餘。聞虜已引去,皆不至期還。天子薄其過,寬而不罪。祁連將軍出塞千六百里,至雞秩山,斬首捕虜十九級,獲牛馬羊百餘。逢漢使匈奴還者冉弘等,言雞秩山西有虜眾,祁連即戒弘,使言無虜,欲還兵。御史屬公孫益壽諫,以為不可,祁連不聽,遂引兵還。虎牙將軍出塞八百餘里,至丹余吾水上,即止兵不進,斬首捕虜千九百餘級,鹵馬牛羊七萬餘,引兵還。上以虎牙將軍不至期,詐增鹵獲,而祁連知虜在前,逗遛不進,皆下吏自殺。擢公孫益壽為侍御史。校尉常惠與烏孫兵至右谷蠡庭,獲單于父行及嫂、居次、名王、犁汙都尉、千長、將以下三萬九千萬餘級,虜馬牛羊驢执橐駝七十餘萬。漢封惠為長羅侯。然匈奴民眾死傷而去者,及畜產遠移死

于不可數勝。於是匈奴遂衰耗,怨烏孫。

其冬,單于自將萬騎擊烏孫,頗得老弱,欲還。會天大雨雪,一日深丈餘,人民畜產凍死,還者不能什一。於是丁令乘弱攻其北,烏桓入其東,烏孫擊其西。凡三國所殺數萬級,馬數萬匹,牛羊甚眾。又重以餓死,人民死者什三,畜產什五,匈奴大虛弱,諸國羈屬者皆瓦解,攻盜不能理。其後漢出三千餘騎,為三道,並入匈奴,捕虜得數千人還。匈奴終不敢取當,茲欲鄉和親,而邊境少事矣。

壺衍鞮單于立十七年死,弟左賢王立,為虛閭權渠單于。是歲,地節二年也。

虛閭權渠單于立,以右大將女為大閼氏,而黜前單于所幸顓渠閼氏。顓渠閼氏父左大且渠怨望。是時匈奴不能為邊寇,於是漢罷外城,以休百姓。單于聞之喜,召貴人謀,欲與漢和親。左大且渠心害其事,曰:「前漢使來,兵隨其後,今亦效漢發兵,先使使者入。」乃自請與呼盧訾王各將萬騎南旁塞獵,相逢俱入。行未到,會三騎亡降漢,言匈奴欲為寇。於是天子詔發邊騎屯要害處,使大將軍軍監治眾等四人將五千騎,分三隊,出塞各數百里,捕得虜各數十人而還。時匈奴亡其三騎,不敢入,即引去。是歲也,匈奴飢,人民畜產死十六七。又發兩屯各萬騎以備漢。其秋,匈奴前所得西嗕居左地者,其君長以下數千人皆驅畜產行,與甌脫戰,所戰殺傷甚眾,遂南降漢。

其明年,西域城郭共擊匈奴,取車師國,得其王及人眾而去。單于復以車師王昆弟兜莫為車師王,收其餘民東徙,不敢居故地。而漢益遣屯士分田車師地以實之。其明年,匈奴怨諸國共擊車師,遣左右大將各萬餘騎屯田右地,欲以侵迫烏孫西域。後二歲,匈奴遣左右奧鞬各六千騎,與左大將再擊漢之田車師城者,不能下。其明年,丁令比三歲入盜匈奴,殺略人民數千,驅馬畜去。匈奴遣萬餘騎往擊之,無所得。其明年,單于將十萬餘騎旁塞獵,欲入邊寇。未至,會其民題除渠堂亡降漢言狀,漢以為言兵鹿奚盧侯,而遣後將軍趙充國將兵四萬餘騎屯緣邊九郡備虜。月餘,單于病歐血,因不敢入,還去,即罷兵。乃使題王都犁胡次等入漢,請和親,未報,會單于死。是歲,神爵二年也。

虛閭權渠單于立九年死。自始立而黜顓渠閼氏,顓渠閼氏即與右賢王私通。右賢王會龍城而去,顓渠閼氏語以單于病甚,且勿遠。後數日,單于死。郝宿王刑未央使人召諸王,未至,顓渠閼氏與其弟左大且渠都隆奇謀,立右賢王屠耆堂為握衍朐鞮單于。握衍朐鞮單于者,代父為右賢王,烏維單于耳孫也。

握衍朐鞮單于立,復修和親,遣弟伊酋若王勝之入漢獻見。單于初立,凶惡,盡殺虛閭權渠時用事貴人刑未央等,而任用顓渠閼氏弟都隆奇,又盡免虛閭權渠子弟近親,而自以其子弟代之。虛閭權渠單于子稽侯狦既不得立,亡歸妻父烏禪幕。烏禪幕者,本烏孫、康居間小國,數見侵暴,率其眾數千人降匈奴,狐鹿姑單于以其弟子日逐王姊妻之,使長其眾,居右地。日逐王先賢撣,其父左賢王當為單于,讓狐鹿姑單于,狐鹿姑單于許立之。國人以故頗言日逐王當為單于。日逐王素與握衍朐鞮單于有隙,即率其眾數萬騎歸漢。漢封日逐王為歸德侯。單于更立其從兄薄胥堂為日逐王。

明年,單于又殺先賢撣兩弟。烏禪幕請之,不聽,心恚。其後左奧鞬王死,單于自立其小子為奧鞬王,留庭。奧鞬貴人共立故奧鞬王子為王,與俱東徙。單于遣右丞相將萬騎往擊之,失亡數千人,不勝。時單于已立二歲,暴虐殺伐,國中不附。及太子、左賢王數讒左地貴人,左地貴人皆怨。其明年,烏桓擊匈奴東邊姑夕王,頗得人民,單于怒。姑夕王恐,即與烏禪幕及左地貴人共立稽侯蛳為呼韓邪單于,發左地兵四五萬人,西擊握衍朐鞮單于,至姑且水北。未戰,握衍朐鞮單于兵敗走,使人報其弟右賢王曰:「匈奴共攻我,若肯發兵助我乎?」右賢王曰:「若不愛人,殺昆弟諸貴人。各自死若處,無來汙我。」握衍朐鞮單于恚,自殺。左大且渠都隆奇亡之右賢王所,其民眾盡降呼韓邪單于。是歲,神爵四年也。握衍朐鞮單于立三年而敗。

呼韓邪單于歸庭數月,罷兵使各歸故地,乃收其兄呼屠吾斯在民間者立為左谷蠡王,使人告右賢貴人,欲令殺右賢王。其冬,都隆奇與右賢王共立日逐王薄胥堂為屠耆單于,發兵數萬人東襲呼韓邪單于。呼韓邪單于兵敗走,屠耆單于還,以其長子都塗吾西為左谷蠡王,少子姑瞀樓頭為右谷蠡王,留居單于庭。

明年秋,屠耆單于使日逐王先賢撣兄右奧鞬王為烏藉都尉各二萬騎,屯東方以備呼韓邪單于。是時,西方呼揭王來與唯犁當戶謀,共讒右賢王,言欲自立為烏藉單于。屠耆單于殺右賢王父子,後知其冤,復殺唯犁當戶。於是呼揭王恐,遂畔去,自立為呼揭單于。右奧鞬王聞之,即自立為車犁單于。烏藉都尉亦自立為烏藉單于。凡五單于。屠耆單于自將兵東擊車犁單于,使都隆奇擊烏藉。烏藉、車犁皆敗,西北走,與呼揭單于兵合為四萬人。烏藉、呼揭皆去單于號,共并力尊輔車犁單于。屠耆單于聞之,使左大將、都尉將四萬騎屯東方,以備呼韓邪單于,自將四萬騎西擊車犁單于。車犁單于敗,西北走,屠耆單于即引西南,留闟敦地。

其明年,呼韓邪單于遣其弟右谷蠡王等西襲屠耆單于屯兵,殺略萬餘人。屠耆單于聞之,即自將六萬騎擊呼韓邪單于,行千里,未至嗕姑地,逢呼韓邪單于兵可四萬人,合戰。屠耆單于兵敗,自殺。都隆奇乃與屠耆少子右谷蠡王姑瞀樓頭亡歸漢,車犁單于東降呼韓邪單于。呼韓邪單于左大將烏厲屈與父呼遫累烏厲溫敦皆見匈奴亂,率其眾數萬人南降漢。封烏厲屈為新城侯,烏厲溫敦為義陽侯。是時李陵子復立烏藉都尉為單于,呼韓邪單于捕斬之,遂復都單于庭,然眾裁數萬人。屠耆單于從弟休旬王將所主五六百騎,擊殺左大且渠,并其兵,至右地,自立為閏振單于,在西邊。其後,呼韓邪單于兄左賢王呼屠吾斯亦自立為郅支骨都侯單于,在東邊。其後二年,閏振單于率其眾東擊郅支單于。郅支單于與戰,殺之,并其兵,遂進攻呼韓邪。呼韓邪破,其兵走,郅支都單于庭。

呼韓邪之敗也,左伊秩訾王為呼韓邪計,勸令稱臣入朝事漢,從漢求助,如此匈奴乃定。呼韓邪議問諸大臣,皆曰:「不可。匈奴之俗,本上氣力而下服役,以馬上戰鬥為國,故有威名於百蠻。戰死,壯士所有也。今兄弟爭國,不在兄則在弟,雖死猶有威名,子孫常長諸國。漢雖彊,猶不能兼并匈奴,奈何亂先古之制,臣事於漢,卑辱先單于,為諸國所笑!雖如是而安,何以復長百蠻!」左伊秩訾曰:「不然。彊弱有時,今漢方盛,烏孫城郭諸國皆為臣妾。自且鞮侯單于以來,匈奴日削,不能取復,雖屈彊於此,未嘗一日安也。今事漢則安存,不事則危亡,計何以過此!」諸大人相難久之。呼韓邪從其計,引眾南近塞,遣子右賢王銖婁渠堂入侍。郅支單于亦遣子右大將駒于利受入侍。是歲,甘露元年也。

明年,呼韓邪單于款五原塞,願朝三年正月。漢遣車騎都尉韓昌迎,發過所七郡郡二千騎,為陳道上。單于正月朝天子于甘泉宮,漢寵以殊禮,位在諸侯王上,贊謁稱臣而不名。賜以冠帶衣裳,黃金璽盭綬,玉具劍,佩刀,弓一張,矢四發,谧戟十,安車一乘,鞍勒一具,馬十五匹,黃金二十斤,錢二十萬,衣被七十七襲,錦鏽綺縠雜帛八千匹,絮六千斤。禮畢,使使者道單于先行,宿長平。上自甘泉宿池陽宮。上登長平,詔單于毋謁,其左右當戶之群臣皆得列觀,及諸蠻夷君長王侯數萬,咸迎於渭橋下,夾道陳。上登渭橋,咸稱萬歲。單于就邸,留月餘,遣歸國。單于自請願留居光祿塞下,有急保漢受降城。漢遣長樂衛尉高昌侯董忠、車騎都尉韓昌將騎萬六千,又發邊郡士馬以千數,送單于出朔方雞鹿塞。詔忠等留衛單于,助誅不服,又轉邊穀米糒,前後三萬四千斛,給贍其食。是歲,郅支單于亦遣使奉獻,漢遇之甚厚。明年,兩單于俱遣使朝獻,漢待呼韓邪使有加。明年,呼韓邪單于復入朝,禮賜如初,加衣百一十襲,錦帛九千匹,絮八千斤。以有屯兵,故不復發騎為送。

始郅支單于以為呼韓邪降漢,兵弱不能復自還,即引其眾西,欲攻定右地。又屠耆單于小弟本侍呼韓邪,亦亡之右地,收兩兄餘兵得數千人,自立為伊利目單于,道逢郅支,合戰,郅支殺之,并其兵五萬餘人。聞漢出兵穀助呼韓邪,即遂留居右地。自度力不能定匈奴,乃益西近烏孫,欲與并力,遣使見小昆彌烏就屠。烏就屠見呼韓邪為漢所擁,郅支亡虜,欲攻之以稱漢,乃殺郅支使,持頭送都護在所,發八千騎迎郅支。郅支見烏孫兵多,其使又不反,勒兵逢擊烏孫,破之。因北擊烏揭,烏揭降。發其兵西破堅昆,北降丁令,并三國。數遣兵擊烏孫,常勝之。堅昆東去單于庭七千里,南去車師五千里,郅支留都之。

元帝初即位,呼韓邪單于復上書,言民眾困乏。漢詔雲中、五原郡轉穀二萬斛以給焉。邪支單于自以道遠,又怨漢擁護呼韓邪,遣使上書求侍子。漢遣谷吉送之,郅支殺吉。漢不知吉音問,而匈奴降者言聞甌脫皆殺之。呼韓邪單于使來,漢輒簿責之甚急。明年,漢遣車騎都尉韓昌、光祿大夫張猛送呼韓邪單于侍子,求問吉等,因赦其罪,勿令自疑。昌、猛見單于民眾益盛,塞下禽獸盡,單于足以自衛,不畏郅支。聞其大臣多勸單于北歸者,恐北去後難約束,昌、猛即與為盟約曰:「自今以來,漢與匈奴合為一家,世世毋得相詐相攻。有竊盜者,相報,行其誅,償其物;有寇,發兵相助。漢與匈奴敢先背約者,受天不祥。令其世世子孫盡如盟。」昌、猛與單于及大臣俱登匈奴諾水東山,刑白馬,單于以徑路刀金留犁撓酒,以老上單于所破月氏王頭為飲器者共飲血盟。昌、猛還奏事,公卿議者以為「單于保塞為藩,雖欲北去,猶不能為危害。昌、猛擅以漢國世世子孫與夷狄詛盟,令單于得以惡言上告于天,羞國家,傷威重,不可得行。宜遣使往告祠天,與解盟。昌、猛奉使無狀,罪至不道。」上薄其過,有詔昌、猛以贖論,勿解盟。其後呼韓邪竟北歸庭,人眾稍稍歸之,國中遂定。

郅支既殺使者,自知負漢,又聞呼韓邪益彊,恐見襲擊,欲遠去。會康居王數為烏孫所困,與諸翕侯計,以為匈奴大國,烏孫素服屬之,今郅支單于困阨在外,可迎置東邊,使合兵取烏孫以立之,長無匈奴憂矣。即使使至堅昆通語郅支。郅支素恐,又怨烏孫,聞康居計,大說,遂與相結,引兵而西。康居亦遣貴人,橐它驢馬數千匹,迎郅支。郅支人眾中寒道死,餘財三千人到康居。其後,都護甘延壽與副陳湯發兵即康居誅斬郅支,語在延壽、湯傳。

郅支既誅,呼韓邪單于且喜且懼,上書言曰:「常願謁見天子,誠以郅支在西方,恐其與烏孫俱來擊臣,以故未得至漢。今郅支已伏誅,願入朝見。」竟寧元年,單于復入朝,禮賜如初,加衣服錦帛絮,皆倍於黃龍時。單于自言願婿漢氏以自親。元帝以後宮良家子王牆字昭君賜單于。單于驩喜,上書願保塞上谷以西至敦煌,傳之無窮,請罷邊備塞吏卒,以休天子人民。天子令下有司議,議者皆以為便。郎中侯應習邊事,以為不可許。上問狀,應曰:「周秦以來,匈奴暴桀,寇侵邊境,漢興,尤被其害。臣聞北邊塞至遼東,外有陰山,東西千餘里,草木茂盛,多禽獸,本冒頓單于依阻其中,治作弓矢,來出為寇,是其苑囿也。至孝武世,出師征伐,斥奪此地,攘之於幕北。建塞徼,起亭隧,築外城,設屯戍,以守之,然後邊境得用少安。幕北地平,少草木,多大沙,匈奴來寇,少所蔽隱,從塞以南,徑深山谷,往來差難。邊長老言匈奴失陰山之後,過之未嘗不哭也。如罷備塞戍卒,示夷狄之大利,不可一也。今聖德廣被,天覆匈奴,匈奴得蒙全活之恩,稽首來臣。夫夷狄之情,困則卑順,彊則驕逆,天性然也。前以罷外城,省亭隧,今裁足以候望通烽火而已。古者安不忘危,不可復罷,二也。中國有禮義之教,刑罰之誅,愚民猶尚犯禁,又況單于,能必其眾不犯約哉!三也。自中國尚建關梁以制諸侯,所以絕臣下之覬欲也。設塞徼,置屯戍,非獨為匈奴而已,亦為諸屬國降民,本故匈奴之人,恐其思舊逃亡,四也。近西羌保塞,與漢人交通,吏民貪利,侵盜其畜產妻子,以此怨恨,起而背畔,世世不絕。今罷乘塞,則生嫚易分爭之漸,五也。往者從軍多沒不還者,子孫貧困,一旦亡出,從其親戚,六也。又邊人奴婢愁苦,欲亡者多,曰『聞匈奴中樂,無奈候望急何!』然時有七出塞者,七也。盜賊桀黠,群輩犯法,如其窘急,亡走北出,則不可制,八也。起塞以來百有餘年,非皆以土垣也,或因山巖石,木柴僵落,谿谷水門,稍稍平之,卒徒築治,功費久遠,不可勝計。臣恐議者不深慮其終始,欲以壹切省繇戍,十年之外,百歲之內,卒有它變,障塞破壞,亭隧滅絕,當更發屯繕治,累世之功不可卒復,九也。如罷戍卒,省候望,單于自以保塞守御,必深德漢,請求無已。小失其意,則不可測。開夷狄之隙,虧中國之固,十也。非所以永持至安,威制百蠻之長策也。」

對奏,天子有詔:「勿議罷邊塞事。」使車騎將軍口諭單于曰:「單于上書願罷北邊吏士屯戍,子孫世世保塞。單于鄉慕禮義,所以為民計者甚厚,此長久之策也,朕甚嘉之。中國四方皆有關梁障塞,非獨以備塞外也,亦以防中國姦邪放縱,出為寇害,故明法度以專眾心也。敬諭單于之意,朕無疑焉。為單于怪其不罷,故使大司馬車騎將軍嘉曉單于。」單于謝曰:「愚不知大計,天子幸使大臣告語,甚厚!」

初,左伊秩訾為呼韓邪畫計歸漢,竟以安定。其後或讒伊秩訾自伐其功,常鞅鞅,呼韓邪疑之。左伊秩訾懼誅,將其眾千餘人降漢,漢以為關內侯,食邑三百戶,令佩其王印綬。及竟寧中,呼韓邪來朝,與伊秩訾相見,謝曰:「王為我計甚厚,令匈奴至今安寧,王之力也,德豈可忘!我失王意,使王去不復顧留,皆我過也。今欲白天子,請王歸庭。」伊秩訾曰:「單于賴天命,自歸於漢,得以安寧,單于神靈,天子之祐也,我安得力!既已降漢,又復歸匈奴,是兩心也。願為單于侍史於漢,不敢聽命。」四單于固請不能得而歸。

王昭君號寧胡閼氏,生一男伊屠智牙師,為右日逐王。呼韓邪立二十八年,建始二年死。始呼韓邪嬖左伊秩訾兄呼衍王女二人。長女顓渠閼氏,生二子,長曰且莫車,次曰囊知牙斯。少女為大閼氏,生四子,長曰雕陶莫皋,次曰且麋胥,皆長於且莫車,少子咸、樂二人,皆小於囊知牙斯。又它閼氏子十餘人。顓渠閼氏貴,且莫車愛。呼韓邪病且死,欲立且莫車,其母顓渠閼氏曰:「

匈奴亂十餘年,不絕如髮,賴蒙漢力,故得復安。今平定未久,人民創艾戰鬥,且莫車年少,百姓未附,恐復危國。我與大閼氏一家共子,不如立雕陶莫皋。」大閼氏曰:「且莫車雖少,大臣共持國事,今舍貴立賤,後世必亂。」單于卒從顓渠閼氏計,立雕陶莫皋,約令傳國與弟。呼韓邪死,雕陶莫皋立,為復株絫若鞮單于。

復株絫若鞮單于立,遣子右致盧兒王醯諧屠奴侯入侍,以且麋胥為左賢王,且莫車為左谷蠡王,囊知牙斯為右賢王。復株絫單于復妻王昭君,生二女,長女云為須卜居次,小女為當于居次。

河平元年,單于遣右皋林王伊邪莫演等奉獻朝正月。既罷,遣使者送至蒱反。伊故事,受其降。光祿大夫谷永、議郎杜欽以為「漢興,匈奴數為邊害,故設金爵之賞以待降者。今單于詘體稱臣,列為北藩,遣使朝賀,無有二心,漢家接之,宜異於往時。今既享單于聘貢之質,而更受其逋逃之臣,是貪一夫之得而失一國之心,擁有罪之臣而絕慕義之君也。假令單于初立,欲委身中國,未知利害,私使伊邪莫演詐降以卜吉凶,受之虧德沮善,今單于自疏,不親邊吏;或者設為反間,欲因而生隙,受之適合其策,使得歸曲而直責。此誠邊竟安危之原,師旅動靜之首,不可不詳也。不如勿受,以昭日月之信,抑詐諼之謀,懷附親之心,便」。對奏,天子從之。遣中郎將王舜往問降狀。伊邪莫演曰:「我病狂妄言耳。」遣去。歸到,官位如故,不肯令見漢使。明年,單于上書願朝河平四年正月,遂入朝,加賜錦繡繒帛二萬匹,絮二萬斤,它如竟寧時。

復株絫單于立十歲,鴻嘉元年死。弟且麋胥立,為搜諧若鞮單于。

搜諧單于立,遣子左祝都韓王朐留斯侯入侍,以且莫車為左賢王。搜諧單于立八歲,元延元年,為朝二年發行,未入塞,病死。弟且莫車立,為車牙若鞮單于。

搜諧單于立,遣子左祝都韓王朐留斯侯入侍,以且莫車為左賢王。搜諧單于立八

車牙單于立,遣子右於涂仇撣王烏夷當入侍,以囊知牙斯為左賢王。車牙單于立四歲,綏和元年死。弟囊知牙斯立,為烏珠留若鞮單于。

烏珠留單于立,以第二閼氏子樂為左賢王,以第五閼氏子輿為右賢王,遣子右股奴王烏鞮牙斯入侍。漢遣中郎將夏侯藩、副校尉韓容使匈奴。時帝舅大司馬票騎將軍王根領尚書事,或說根曰:「

匈奴有斗入漢地,直張掖郡,生奇材木,箭竿就羽,如得之,於邊甚饒,國家有廣地之實,將軍顯功,垂於無窮。」根為上言其利,上直欲從單于求之,為有不得,傷命損威。根即但以上指曉藩,令從藩所說而求之。藩至匈奴,以語次說單于曰:「竊見匈奴斗入漢地,直張掖郡。漢三都尉居塞上,士卒數百人寒苦,候望久勞。單于宜上書獻此地,直斷閼之,省兩都尉士卒數百人,以復天子厚恩,其報必大。」單于曰:「此天子詔語邪,將從使者所求也?」藩曰:「詔指也,然藩亦為單于畫善計耳。」單于曰:「孝宣、孝元皇帝哀憐父呼韓邪單于,從長城以北匈奴有之。此溫偶駼王所居地也,未曉其形狀所生,請遣使問之。」藩、容歸漢。從復使匈奴,至則求地。單于曰:「父兄傳五世,漢不求此地,至知獨求,何也?已問溫偶駼王,匈奴西邊諸侯作穹廬及車,皆仰此山材木,且先父地,不敢失也。」藩還,遷為太原太守。單于遣使上書,以藩求地狀聞。詔報單于曰:「

藩擅稱詔從單于求地,法當死,更大赦二,今徙藩為濟南太守,不令當匈奴。」明年,侍子死,歸葬。復遣子左於駼仇撣王稽留昆入侍。

至哀帝建平二年,烏孫庶子卑援疐翕侯人眾入匈奴西界,寇盜牛畜,頗殺其民。單于聞之,遣左大當戶烏夷泠將五千騎擊烏孫,殺數百人,略千餘人,敺牛畜去。卑援疐恐,遣子趨逯為質匈奴。單于受,以狀聞。漢遣中郎將丁野林、副校尉公乘音使匈奴,責讓單于,告令還歸卑援疐質子。單于受詔,遣歸。

建平四年,單于上書願朝五年。時哀帝被疾,或言匈奴從上游來厭人,自黃龍、竟寧時,單于朝中國輒有大故。上由是難之,以問公卿,亦以為虛費府帑,可且勿許。單于使辭去,未發,黃門郎揚雄上書諫曰:

臣聞六經之治,貴於未亂;兵家之勝,貴於未戰。二者皆微,然而大事之本,不可不察也。今單于上書求朝,國家不許而辭之,臣愚以為漢與匈奴從此隙矣。本北地之狄,五帝所不能臣,三王所不能制,其不可使隙甚明。臣不敢遠稱,請引秦以來明之:

以秦始皇之彊,蒙恬之威,帶甲四十餘萬,然不敢窺西河,乃築長城以界之。會漢初興,以高祖之威靈,三十萬眾困於平城,士或七日不食。時奇譎之士石畫之臣甚眾,卒其所以脫者,世莫得而言也。又高皇后嘗忿匈奴,群臣庭議,樊噲請以十萬眾橫行匈奴中,季布曰:「噲可斬也,妄阿順指!」於是大臣權書遺之,然後匈奴之結解,中國之憂平。及孝文時,匈奴侵暴北邊,候騎至雍甘泉,京師大駭,發三將軍屯細柳、棘門、霸上以備之,數月乃罷。孝武即位,設馬邑之權,欲誘匈奴,使韓安國將三十萬眾徼於便墬,匈奴覺之而去,徒費財勞師,一虜不可得見,況單于之面乎!其後深惟社稷之計,規恢萬載之策,乃大興師數十萬,使衛青、霍去病操兵,前後十餘年。於是浮西河,絕大幕,破寘顏,襲王庭,窮極其地,追奔逐北,封狼居胥山,禪於姑衍,以臨翰海,虜名王貴人以百數。自是之後,匈奴震怖,益求和親,然而未肯稱臣也。

且夫前世豈樂傾無量之費,役無罪之人,快心於狼望之北哉?以為不壹勞者不久佚,不蹔費者不永寧,是以忍百萬之師以摧餓虎之喙,運府庫之財填盧山之壑而不悔也。至本始之初,匈奴有桀心,欲掠烏孫,侵公主,乃發五將之師十五萬騎獵其南,而長羅侯以烏孫五萬騎震其西,皆至質而還。時鮮有所獲,徒奮揚威武,明漢兵若雷風耳。雖空行空反,尚誅兩將軍。故北狄不服,中國未得高枕安寢也。逮至元康、神爵之間,大化神明,鴻恩溥洽,而匈奴內亂,五單于爭立,日逐、呼韓邪攜國歸死,扶伏稱臣,然尚羈縻之,計不顓制。自此之後,欲朝者不距,不欲者不彊。何者?外國天性忿鷙,形容魁健,負力怙氣,難化以善,易誊以惡,其彊難詘,其和難得。故未服之時,勞師遠攻,傾國殫貨,伏尸流血,破堅拔敵,如彼之難也;既服之後,慰薦撫循,交接賂遺,威儀俯仰,如此之備也。往時嘗屠大宛之城,蹈烏桓之壘,探姑繒之壁,籍蕩姐之場,艾朝鮮之旃,拔兩越之旗,近不過旬月之役,遠不離二時之勞,固已犁其庭,掃其閭,郡縣而置之,雲徹席卷,後無餘菑。唯北狄為不然,真中國之堅敵也,三垂比之懸矣,前世重之茲甚,未易可輕也。

今單于歸義,懷款誠之心,欲離其庭,陳見於前,此乃上世之遺策,神靈之所想望,國家雖費,不得已者也。奈何距以來厭之辭,疏以無日之期,消往昔之恩,開將來之隙!夫款而隙之,使有恨心,負前言,緣往辭,歸怨於漢,因以自絕,終無北面之心,威之不可,諭之不能,焉得不為大憂乎!夫明者視於無形,聰者聽於無聲,誠先於未然,即蒙恬、樊噲不復施,棘門、細柳不復備,馬邑之策安所設,衛、霍之功何得用,五將之威安所震?不然,壹有隙之後,雖智者勞心於內,辯者轂擊於外,猶不若未然之時也。且往者圖西域,制車師,置城郭都護三十六國,費歲以大萬計者,豈為康居、烏孫能踰白龍堆而寇西邊哉?乃以制匈奴也。夫百年勞之,一日失之,費十而愛一,臣竊為國不安也。唯陛下少留意於未亂未戰,以遏邊萌之禍。

書奏,天子寤焉,召還匈奴使者,更報單于書而許之。賜雄帛五十匹,黃金十斤。單于未發,會病,復遣使願朝明年。故事,單于朝,從名王以下及從者二百餘人。單于又上書言:「蒙天子神靈,人民盛壯,願從五百人入朝,以明天子盛德。」上皆許之。

元壽二年,單于來朝,上以太歲厭勝所在,舍之上林苑蒲陶宮。告之以加敬於單于,單于知之。加賜衣三百七十襲,錦繡繒帛三萬匹,絮三萬斤,它如河平時。既罷,遣中郎將韓況送單于。單于出塞,到休屯井,北度車田盧水,道里回遠。況等乏食,單于乃給其糧,失期不還五十餘日。

初,上遣稽留昆隨單于去,到國,復遣稽留昆同母兄右大且方與婦入侍。還歸,復遣且方同母兄左日逐王都與婦入侍。是時,漢平帝幼,太皇太后稱制,新都侯王莽秉政,欲說太后以威德至盛異於前,乃風單于令遣王昭君女須卜居次云入侍太后,所以賞賜之甚厚。

會西域車師後王句姑、去胡來王唐兜皆怨恨都護校尉,將妻子人民亡降匈奴,語在西域傳。單于受置左谷蠡地,遣使上書言狀曰:「臣謹已受。」詔書中郎將韓隆、王昌、副校尉甄阜、侍中謁者帛敞、長水校尉王歙使匈奴,告單于曰:「西域內屬,不當得受,今遣之。」單于曰:「孝宣、孝元皇帝哀憐,為作約束,自長城以南天子有之,長城以北單于有之。有犯塞,輒以狀聞;有降者,不得受。臣知父呼韓邪單于蒙無量之恩,死遺言曰:『有從中國來降者,勿受,輒送至塞,以報天子厚恩。』此外國也,得受之。」使者曰:「匈奴骨肉相攻,國幾絕,蒙中國大恩,危亡復續,妻子完安,累世相繼,宜有以報厚恩。」單于叩頭謝罪,執二虜還付使者。詔使中郎將王萌待西域惡都奴界上逆受。單于遣使送到國,因請其罪。使者以聞,有詔不聽,會西域諸國王斬以示之。乃造設四條:中國人亡入匈奴者,烏孫亡降匈奴者,西域諸國佩中國印綬降匈奴者,烏桓降匈奴者,皆不得受。遣中郎將王駿、王昌、副校尉甄阜、王尋使匈奴,班四條與單于,雜函封,付單于,令奉行,因收故宣帝所為約束封函還。時,莽奏令中國不得有二名,因使使者以風單于,宜上書慕化,為一名,漢必加厚賞。單于從之,上書言:「幸得備藩臣,竊樂太平聖制,臣故名囊知牙斯,今謹更名曰知。」莽大說,白太后,遣使者答諭,厚賞賜焉。

漢既班四條,後護烏桓使者告烏桓民,毋得復與匈奴皮布稅。匈奴以故事遣使者責烏桓稅,匈奴人民婦女欲賈販者皆隨往焉。烏桓距曰:「奉天子詔條,之當予匈奴稅。」匈奴使怒,收烏桓酋豪,縛到懸之。酋豪昆弟怒,共入匈奴使及其官屬,收略婦女馬牛。單于聞之,遣使發左賢王兵入烏桓責殺使者,因攻擊之。烏桓分散,或走上山,或東保塞。匈奴頗殺人民,敺婦女弱小且千人去,置左地,告烏桓曰:「持馬畜皮布來贖之。」烏桓見略者親屬二千餘人持財畜往贖,匈奴受,留不遣。

王莽之篡位也,建國元年,遣五威將王駿率甄阜、王颯、陳饒、帛敞、丁業六人,多齎金帛,重遺單于,諭曉以受命代漢狀,因易單于故印。故印文曰「匈奴單于璽」,莽更曰「新匈奴單于章」。將率既至,授單于印紱,詔令上故印紱。單于再拜受詔。譯前,欲解取故印紱,單于舉掖授之。左姑夕侯蘇從旁謂單于曰:「未見新印文,宜且勿與。」單于止,不肯與。請使者坐穹廬,單于欲前為壽。五威將曰:「故印紱當以時上。」單于曰:「諾。」復舉掖授譯。蘇復曰:「未見印文,且勿與。」單于曰:「印文何由變更!」遂解故印紱奉上,將率受。著新紱,不解視印,飲食至夜乃罷。右率陳饒謂諸將率曰:「鄉者姑夕侯疑印文,幾令單于不與人。如令視印,見其變改,必求故印,此非辭說所能距也。既得而復失之,辱命莫大焉。不如椎破故印,以絕禍根。」將率猶與,莫有應者。饒,燕士,果悍,即引斧椎壞之。明日,單于果遣右骨都侯當白將率曰:「漢賜單于印,言『璽』不言『章』,又無『

漢』字,諸王已下乃有『漢』言『章』。今印去『璽』加『新』,與臣下無別。願得故印。」將率示以故印,謂曰:「新室順天制作,故印隨將率所自為破壞。單于宜奉天命,奉新室之制。」當還白,單于知已無可奈何,又多得賂遺,即遣弟右賢王輿奉馬牛隨將率入謝,因上書求故印。

將率還到左犁汗王咸所居地,見烏桓民多,以問咸。咸具言狀,將率曰:「前封四條,不得受烏桓降者,亟還之。」咸曰:「請密與單于相聞,得語,歸之。」單于使咸報曰:「當從塞內還之邪,從塞外還之邪?」將率不敢顓決,以聞。詔報,從塞外還之。

單于始用夏侯藩求地有距漢語,後以求稅烏桓不得,因寇略其人民,釁由是生,重以印文改易,故怨恨。乃遣右大且渠蒲呼盧訾等十餘人將兵眾萬騎,以謢送烏桓為名,勒兵朔方塞下。朔方太守以聞。

明年,西域車師後王須置離謀降匈奴,都護但欽誅斬之。置離兄狐蘭支將人眾二千餘人,敺畜產,舉國亡降匈奴,單于受之。狐蘭支與匈奴共入寇,擊車師,殺後成長,傷都護司馬,復還入匈奴。

時戊己校尉史陳良、終帶、司馬丞韓玄、右曲候任商等見西域頗背叛,聞匈奴欲大侵,恐并死,即謀劫略吏卒數百人,共殺戊己校尉刀護,遣人與匈奴南犁汗王南將軍相聞。匈奴南將軍二千騎入西域迎良等,良等盡脅略戊己校尉吏士男女二千餘人入匈奴。玄、商留南將軍所,良、帶徑至單于庭,人眾別置零吾水上田居。單于號良、帶曰烏桓都將軍,留居單于所,數呼與飲食。西域都護但欽上書言匈奴南將軍右伊秩訾將人眾寇擊諸國。莽於是大分匈奴為十五單于,遣中郎將藺苞、副校尉戴級將兵萬騎,多齎珍寶至雲中塞下,招誘呼韓邪單于諸子,欲以次拜之。使譯出塞誘呼右犁汗王咸、咸子登、助三人,至則脅拜咸為孝單于,賜安車鼓車各一,黃金千斤,雜繒千匹,戲戟十;拜助為順單于,賜黃金五百斤;傳送助、登長安。莽封苞為宣威公,拜為虎牙將軍;封級為揚威公,拜為虎賁將軍。單于聞之,怒曰:「先單于受漢宣帝恩,不可負也。今天子非宣帝子孫,何以得立?」遣左骨都侯、右伊秩訾王呼盧訾及左賢王樂將兵入雲中益壽塞,大殺吏民。是歲,建國三年也。

是後,單于歷告左右部都尉、諸邊王,入塞寇盜,大輩萬餘,中輩數千,少者數百,殺鴈門、朔方太守、都尉,略吏民畜產不可勝數,緣邊虛耗。莽新即位,怙府庫之富欲立威,乃拜十二部將率,發郡國勇士,武庫精兵,各有所屯守,轉委輸於邊。議滿三十萬眾,齎三百日糧,同時十道並出,窮追匈奴,內之于丁令,因分其地,立呼韓邪十五子。

莽將嚴尤諫曰:「臣聞匈奴為害,所從來久矣,未聞上世有必征之者也。後世三家周、秦、漢征之,然皆未有得上策者也。周得中策,漢得下策,秦無策焉。當周宣王時,獫允內侵,至于涇陽,命將征之,盡境而還。其視戎狄之侵,譬猶蚊虻之螫,敺之而已。故天下稱明,是為中策。漢武帝選將練兵,約齎輕糧,深入遠戍,雖有克獲之功,胡輒報之,兵連禍結三十餘年,中國罷耗,匈奴亦創艾,而天下稱武,是為下策。秦始皇不忍小恥而輕民力,築長城之固,延袤萬里,轉輸之行,起於負海,疆境既完,中國內竭,以喪社稷,是為無策。今天下遭陽九之阨,比年饑饉,西北邊尤甚。發三十萬眾,具三百日糧,東援海代,南取江淮,然後乃備。計其道里,一年尚未集合,兵先至者聚居暴露,師老械弊,勢不可用,此一難也。邊既空虛,不能奉軍糧,內調郡國,不相及屬,此二難也。計一人三百日食,用糒十八斛,非牛力不能勝;牛又當自齎食,加二十斛,重矣。胡地沙鹵,多乏水草以往事揆之,軍出未滿百日,牛必物故且盡,餘糧尚多,人不能負,此三難也。胡地秋冬甚寒,春夏甚風,多齎釜鍑薪炭,重不可勝,食糒飲水,以歷四時,師有疾疫之憂,是故前世伐胡,不過百日,非不欲久,勢力不能,此四難也。輜重自隨,則輕銳者少,不得疾行,虜徐遁逃,勢不能及,幸而逢虜,又累輜重,如遇險阻,銜尾相隨,虜要遮前後,危殆不測,此五難也。大用民力,功不可必立,臣伏憂之。今既發兵,宜縱先至者,令臣尤等深入霆擊,且以創艾胡虜。」莽不聽尤言,轉兵穀如故,天下騷動。

咸既受莽孝單于之號,馳出塞歸庭,具以見脅狀白單于。單于更以為於粟置支侯,匈奴賤官也。後助病死,莽以登代助為順單于。

厭難將軍陳欽、震狄將軍王巡屯雲中葛邪塞。是時,匈奴數為邊寇,殺將率吏士,略人民,敺畜產去甚眾。捕得虜生口驗問,皆曰孝單于咸子角數為寇。兩將以聞。四年,莽會諸蠻夷,斬咸子登於長安市。

初,北邊自宣帝以來,數世不見煙火之警,人民熾盛,牛馬布野。及莽撓亂匈奴,與之構難,邊民死亡係獲,又十二部兵久屯而不出,吏士罷弊,數年之間,北邊虛空,野有暴骨矣。

烏珠留單于立二十一歲,建國五年死。匈奴用事大臣右骨都侯須卜當,即王昭君女伊墨居次云之婿也。云常欲與中國和親,又素與咸厚善,見咸前後為莽所拜,故遂越輿而立咸為烏累若鞮單于。

烏累單于咸立,以弟輿為左谷蠡王。烏珠留單于子蘇屠胡本為左賢王,以弟屠耆閼氏子盧渾為右賢王。烏珠留單于在時,左賢王數死,以為其號不祥,更易命左賢王曰「護于」。護于之尊最貴,次當為單于,故烏珠留單于授其長子以為護于,欲傳以國。咸怨烏珠留單于貶賤己號,不欲傳國,及立,貶護于為左屠耆王。云、當遂勸咸和親。

天鳳元年,云、當遣人之西河虎猛制虜塞下,告塞吏曰欲見和親侯。和親侯王歙者,王昭君兄子也。中部都尉以聞。莽遣歙、歙弟騎都尉展德侯颯使匈奴,賀單于初立,賜黃金衣被繒帛,紿言侍子登在,因購求陳良、終帶等。單于盡收四人及手殺校尉刀護賊芝音妻子以下二十七人,皆械檻付使者,遣廚唯姑夕王富等四十人送歙、颯。莽作焚如之刑,燒殺陳良等,罷諸將率屯兵,但置游擊都尉。單于貪莽賂遺,故外不失漢故事,然內利寇掠。又使還,知子登前死,怨恨,寇虜從左地入,不絕。使者問單于,輒曰:「烏桓與匈奴無狀黠民共為寇入塞,譬如中國有盜賊耳!咸初立持國,威信尚淺,盡力禁止,不敢有二心。」

天鳳二年五月,莽復遣歙與五威將王咸率伏黯、丁業等六人,使送右廚唯姑夕王,因奉歸前所斬侍子登及諸貴人從者喪,皆載以常車。至塞下,單于遣云、當子男大且渠奢等至塞迎。咸等至,多遺單于金珍,因諭說改其號,號匈奴曰「恭奴」,單于曰「善于」,賜印綬。封骨都侯當為後安公,當子男奢為後安侯。單于貪莽金幣,故曲聽之,然寇盜如故。咸、歙又以陳良等購金付云、當,令自差與之。十二月,還入塞,莽大喜,賜歙錢二百萬,悉封黯等。

單于咸立五歲,天鳳五年死,弟左賢王輿立,為呼都而尸道皋若鞮單于。匈奴謂孝曰「若鞮」。自呼韓邪後,與漢親密,見漢諡帝為「孝」,慕之,故皆為「若鞮」。

呼都而尸單于輿既立,貪利賞賜,遣大且渠奢與云女弟當戶居次子醯櫝王俱奉獻至長安。莽遣和親侯歙與奢等俱至制虜塞下,與云、當會,因以兵迫脅,將至長安。云、當小男從塞下得脫,歸匈奴。當至長安,莽拜為須卜單于,欲出大兵以輔立之。兵調度亦不合,而匈奴愈怒,並入北邊,北邊由是壞敗。會當病死,莽以其庶女陸逯任妻後安公奢,所以尊寵之甚厚,終為欲出兵立之者。會漢兵誅莽,云、奢亦死。

更始二年冬,漢遣中郎將歸德侯颯、大司馬護軍陳遵使匈奴,授單于漢舊制璽綬,王侯以下印綬,因送云、當餘親屬貴人從者。單于輿驕,謂遵、颯曰:「匈奴本與漢為兄弟,匈奴中亂,孝宣皇帝輔立呼韓邪單于,故稱臣以尊漢。今漢亦大亂,為王莽所篡,匈奴亦出兵擊莽,空其邊境,令天下騷動思漢,莽卒以敗而漢復興,亦我力也,當復尊我!」遵與相牚距,單于終持此言。其明年夏,還。會赤眉入長安,更始敗。

贊曰:書戒「蠻夷猾夏」,詩稱「戎狄是膺」,春秋「有道守在四夷」,久矣夷狄之為患也。故自漢興,忠言嘉謀之臣曷嘗不運籌策相與爭於廟堂之上乎?高祖時則劉敬,呂后時樊噲、季布,孝文時賈誼、朝錯,孝武時王恢、韓安國、朱買臣、公孫弘、董仲舒,人持所見,各有同異,然總其要,歸兩科而已。縉紳之儒則守和親,介冑之士則言征伐,皆偏見一時之利害,而未究匈奴之終始也。自漢興以至于今,曠世歷年,多於春秋,其與匈奴,有脩文而和親之矣,有用武而克伐之矣,有卑下而承事之矣,有威服而臣畜之矣,詘伸異變,強弱相反,是故其詳可得而言也。

昔和親之論,發於劉敬。是時天下初定,新遭平城之難,故從其言,約結和親,賂遺單于,冀以救安邊境。孝惠、高后時遵而不違,匈奴寇盜不為衰止,而單于反以加驕倨。逮至孝文,與通關市,妻以漢女,增厚其賂,歲以千金,而匈奴數背約束,邊境屢被其害。是以文帝中年,赫然發憤,遂躬戎服,親御鞍馬,從六郡良家材力之士,馳射上林,講習戰陳,聚天下精兵,軍於廣武,顧問馮唐,與論將帥,喟然歎息,思古名臣,此則和親無益,已然之明效也。

仲舒親見四世之事,猶復欲守舊文,頗增其約。以為「義動君子,利動貪人,如匈奴者,非可以仁義說也。獨可說以厚利,結之於天耳。故與之厚利以沒其意,與盟於天以堅其約,質其愛子以累其心,匈奴雖欲展轉,奈失重利何,奈欺上天何,奈殺愛子何。夫賦斂行賂不足以當三軍之費,城郭之固無以異於貞士之約,而使邊城守境之民父兄緩帶,稚子咽哺,胡馬不窺於長城,而羽檄不行於中國,不亦便於天下乎!」察仲舒之論,考諸行事,乃知其未合於當時,而有闕於後世也。當孝武時,雖征伐克獲,而士馬物故亦略相當;雖開河南之野,建朔方之郡,亦棄造陽之北九百餘里。匈奴人民每來降漢,單于亦輒拘留漢使以相報復,其桀驁尚如斯,安肯以愛子而為質乎?此不合當時之言也。若不置質,空約和親,是襲孝文既往之悔,而長匈奴無已之詐也。夫邊城不選守境武略之臣,脩障隧備塞之具,厲長戟勁弩之械,恃吾所以待邊寇。而務賦斂於民,遠行貨賂,割剝百姓,以奉寇讎。信甘言,守空約,而幾胡馬之不窺,不已過乎!

至孝宣之世,承武帝奮擊之威,直匈奴百年之運,因其壞亂幾亡之阨,權時施宜,覆以威德,然後單于稽首臣服,遣子入侍,二世稱藩,賓於漢庭。是時邊城晏閉。牛馬布野,三世無犬吠之警,菞庶亡干戈之役。

後六十餘載之間,遭王莽篡位,始開邊隙,單于由是歸怨自絕,莽遂斬其侍子,邊境之禍搆矣。故呼韓邪始朝於漢,漢議其儀,而蕭望之曰:「戎狄荒服,言其來服荒忽無常,時至時去,宜待以客禮,讓而不臣。如其後嗣遁逃竄伏,使於中國不為叛臣。」及孝元時,議罷守塞之備,侯應以為不可,可謂盛不忘衰,安必思危,遠見識微之明矣。至單于咸棄其愛子,昧利不顧,侵掠所獲,歲鉅萬計,而和親賂遺,不過千金,安在其不棄質而失重利也?仲舒之言,漏於是矣。

夫規事建議,不圖萬世之固,而媮恃一時之事者,未必以經遠也。若乃征伐之功,秦漢行事,嚴尤論之當矣。故先王度土,中立封畿,分九州,列五服,物土貢,制外內,或脩刑政,或詔文德,遠近之勢異也。是以春秋內諸夏而外夷狄。夷狄之人貪而好利,被髮左衽,人面獸心。其與中國殊章服,異習俗,飲食不同,言語不通,辟居北垂寒露之野,逐草隨畜,射獵為生,隔以山谷,雍以沙幕,天地所以絕外內也。是故聖王禽獸畜之,不與約誓,不就攻伐;約之則費賂而見欺,攻之則勞師而詔寇。其地不可耕而食也,其民不可臣而畜也,是以外而不內,疏而不戚,政教不及其人,正朔不加其國;來則懲而御之,去則備而守之。其慕義而貢獻,則接之以禮讓,羈靡不絕,使曲在彼,蓋聖王制御蠻夷之常道也。


\end{pinyinscope}