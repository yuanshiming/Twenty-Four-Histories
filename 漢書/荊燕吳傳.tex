\article{荊燕吳傳}

\begin{pinyinscope}
荊王劉賈,高帝從父兄也,不知其初起時。漢元年,還定三秦,賈為將軍,定塞地,從東擊項籍。

漢王敗成皋,北度河,得張耳、韓信軍,軍脩武,深溝高壘,使賈將二萬人,騎數百,擊楚,度白馬津入楚地,燒其積聚,以破其業,無以給項王軍食。已而楚兵擊之,賈輒避不肯與戰,而與彭越相保。

漢王追項籍至固陵,使賈南度淮圍壽春。還至,使人間招楚大司馬周殷。周殷反楚,佐賈舉九江,迎英布兵,皆會垓下,誅項籍。漢王因使賈將九江兵,與太尉盧綰西南擊臨江王共尉,尉死,以臨江為南郡。

賈既有功,而高祖子弱,昆弟少,又不賢,欲王同姓以填天下,乃下詔曰:「將軍劉賈有功,及擇子弟可以為王者。」群臣皆曰:「立劉賈為荊王,王淮東。」立六年而淮南王黥布反,東擊荊。賈與戰,弗勝,走富陵,為布軍所殺。

燕王劉澤,高祖從祖昆弟也。高祖三年,澤為郎中。十一年,以將軍擊陳豨將王黃,封為營陵侯。

高后時,齊人田生游乏資,以畫奸澤。澤大說之,用金二百斤為田生壽。田生已得金,即歸齊。二歲,澤使人謂田生曰:「弗與矣。」田生如長安,不見澤,而假大宅,令其子求事呂后所幸大謁者張卿。居數月,田生子請張卿臨,親脩具。張卿往,見田生帷帳具置如列侯。張卿驚。酒酣,乃屏人說張卿曰:「臣觀諸侯邸第百餘,皆高帝一切功臣。今呂氏雅故本推轂高帝就天下,功至大,又有親戚太后之重。太后春秋長,諸呂弱,太后欲立呂產為呂王,王代。呂后又重發之,恐大臣不聽。今卿最幸,大臣所敬,何不風大臣以聞太后,太后必喜。諸呂以王,萬戶侯亦卿之有。太后心欲之,而卿為內臣,不急發,恐過及身矣。」張卿大然之,乃風大臣語太后。太后朝,因問大臣。大臣請立呂產為呂王。太后賜張卿千金,張卿以其半進田生。田生弗受,因說之曰:「呂產王也,諸大臣未大服。今營陵侯澤,諸劉長,為大將軍,獨此尚觖望。今卿言太后,裂十餘縣王之,彼得王喜,於諸呂王益固矣。」張卿入言之。又太后女弟呂須女亦為營陵侯妻,故遂立營陵侯澤為琅邪王。琅邪王與田生之國,急行毋留。出關,太后果使人追之。已出,即還。

澤王琅邪二年,而太后崩,澤乃曰:「帝少,諸呂用事,諸劉孤弱。」引兵與齊王合謀西,欲誅諸呂。至梁,聞漢灌將軍屯滎陽,澤還兵備西界,遂跳驅至長安。代王亦從代至。諸將相與琅邪王共立代王,是為孝文帝。文帝元年,徙澤為燕王,而復以琅邪歸齊。

澤王燕二年,薨,諡曰敬王。子康王嘉嗣,九年薨。子定國嗣。定國與父康王姬姦,生子男一人。奪弟妻為姬。與子女三人姦。定國有所欲誅殺臣肥如令郢人,郢人等告定國。定國使謁者以它法劾捕格殺郢人滅口。至元朔中,郢人昆弟復上書具言定國事。下公卿,皆議曰:「定國禽獸行,亂人倫,逆天道,當誅。」上許之。定國自殺,立四十二年,國除。哀帝時繼絕世,乃封敬王澤玄孫之孫無終公士歸生為營陵侯,更始中為兵所殺。

吳王濞,高帝兄仲之子也。高帝立仲為代王。匈奴攻代,仲不能堅守,棄國間行,走雒陽,自歸,天子不忍致法,廢為合陽侯。子濞,封為沛侯。黥布反,高祖自將往誅之。濞年二十,以騎將從破布軍。荊王劉賈為布所殺,無後。上患吳會稽輕悍,無壯王填之,諸子少,乃立濞於沛,為吳王,王三郡五十三城。已拜受印,高祖召濞相之,曰:「若狀有反相。」獨悔,業已拜,因拊其背,曰:「漢後五十年東南有亂,豈若邪?然天下同姓一家,慎無反!」濞頓首曰:「不敢。」

會孝惠、高后時天下初定,郡國諸侯各務自拊循其民。吳有豫章郡銅山,即招致天下亡命者盜鑄錢,東煮海水為鹽,以故無賦,國用饒足。

孝文時,吳太子入見,得侍皇太子飲博。吳太子師傅皆楚人,輕悍,又素驕。博爭道,不恭,皇太子引博局提吳太子,殺之。於是遣其喪歸葬吳。吳王慍曰:「天下一宗,死長安即葬長安,何必來葬!」復遣喪之長安葬。吳王由是怨望,稍失藩臣禮,稱疾不朝。京師知其以子故,驗問實不病,諸吳使來,輒繫責治之。吳王恐,所謀滋甚。及後使人為秋請,上復責問吳使者。使者曰:「察見淵中魚,不祥。今吳王始詐疾,反覺,見責急,愈益閉,恐上誅之,計乃無聊。唯上與更始。」於是天子皆赦吳使者歸之,而賜吳王几杖,老,不朝。吳得釋,其謀亦益解。然其居國以銅鹽故,百姓無賦。卒踐更,輒予平賈。歲時存問茂材,賞賜閭里。它郡國吏欲來捕亡人者,頌共禁不與。如此者三十餘年,以故能使其眾。

朝錯為太子家令,得幸皇太子,數從容言吳過可削。數上書說之,文帝寬,不忍罰,以此吳王日益橫。及景帝即位,錯為御史大夫,說上曰:「昔高帝初定天下,昆弟少,諸子弱,大封同姓,故孽子悼惠王王齊七十二城,庶弟元王王楚四十城,兄子王吳五十餘城。封三庶孽,分天下半。今吳王前有太子之隙,詐稱病不朝,於古法當誅。文帝不忍,因賜几杖,德至厚也。不改過自新,乃益驕恣,公即山鑄錢,煮海為鹽,誘天下亡人謀作亂逆。今削之亦反,不削亦反。削之,其反亟,禍小;不削之,其反遲,禍大。」三年冬,楚王來朝,錯因言楚王戊往年為薄太后服,私姦服舍,請誅之。詔赦,削東海郡。及前二年,趙王有罪,削其常山郡。膠西王卬以賣爵事有姦,削其六縣。

漢廷臣方議削吳,吳王恐削地無已,因欲發謀舉事。念諸侯無足與計者,聞膠西王勇,好兵,諸侯皆畏憚之,於是乃使中大夫應高口說膠西王曰:「吳王不肖,有夙夜之憂,不敢自外,使使臣諭其愚心。」王曰:「何以教之?」高曰:「今者主上任用邪臣,聽信讒賊,變更律令,侵削諸侯,徵求滋多,誅罰良重,日以益甚。語有之曰:『镉㐄及米。』吳與膠西,知名諸侯也,一時見察,不得安肆矣。吳王身有內疾,不能朝請二十餘年,常患見疑,無以自白,脅肩絫足,猶懼不見釋。竊聞大王以爵事有過,所聞諸侯削地,罪不至此,此恐不止削地而已。」王曰:「有之,子將柰何?」高曰:「同惡相助,同好相留,同情相求,同欲相趨,同利相死。今吳王自以與大王同憂,願因時循理,棄軀以除患於天下,意亦可乎?」膠西王瞿然駭曰:「寡人何敢如是?主上雖急,固有死耳,安得不事?」高曰:「御史大夫朝錯營或天子,侵奪諸侯,蔽忠塞賢,朝廷疾怨,諸侯皆有背叛之意,人事極矣。彗星出,蝗蟲起,此萬世一時,而愁勞,聖人所以起也。吳王內以朝錯為誅,外從大王後車,方洋天下,所向者降,所指者下,莫敢不服。大王誠幸而許之一言,則吳王率楚王略函谷關,守滎陽敖倉之粟,距漢兵,治次舍,須大王。大王幸而臨之,則天下可并,兩主分割,不亦可乎?」王曰:「善。」歸報吳王,猶恐其不果,乃身自為使者,至膠西面約之。

膠西群臣或聞王謀,諫曰:「諸侯地不能為漢十二,為叛逆以憂太后,非計也。今承一帝,尚云不易,假令事成,兩主分爭,患乃益生。」王不聽,遂發使約齊、菑川、膠東、濟南,皆許諾。

諸侯既新削罰,震恐,多怨錯。乃削吳會稽、豫章郡書至,則吳王先起兵,誅漢吏二千石以下。膠西、膠東、菑川、濟南、楚、趙亦皆反,發兵西。齊王後悔,背約城守。濟北王城壞未完,其郎中令劫守王,不得發兵。膠西王、膠東王為渠率,與菑川、濟南共攻圍臨菑。趙王遂亦陰使匈奴與連兵。

七國之發也,吳王悉其士卒,下令國中曰:「寡人年六十二,身自將。少子年十四,亦為士卒先。諸年上與寡人同,下與少子等,皆發。」二十餘萬人。南使閩、東越,閩、東越亦發兵從。

孝景前三年正月甲子,初起兵於廣陵。西涉淮,因并楚兵。發使遺諸侯書曰:「吳王劉濞敬問膠西王、膠東王、菑川王、濟南王、趙王、楚王、淮南王、衡山王、廬江王、故長沙王子:幸教!以漢有賊臣錯,無功天下,侵奪諸侯之地,使吏劾繫訊治,以侵辱之為故,不以諸侯人君禮遇劉氏骨肉,絕先帝功臣,進任姦人,誑亂天下,欲危社稷。陛下多病志逸,不能省察。欲舉兵誅之,謹聞教。敝國雖狹,地方三千里;人民雖少,精兵可具五十萬。寡人素事南越三十餘年,其王諸君皆不辭分其兵以隨寡人,又可得三十萬。寡人雖不肖,願以身從諸王。南越直長沙者,因王子定長沙以北,西走蜀、漢中。告越、楚王、淮南三王,與寡人西面;齊諸王與趙王定河間、河內,或入臨晉關,或與寡人會雒陽;燕王、趙王故與胡王有約,燕王北定代、雲中,轉胡眾入蕭關,走長安,匡正天下,以安高廟。願王勉之。楚元王子、淮南三王或不沐洗十餘年,怨入骨髓,欲壹有所出久矣,寡人未得諸王之意,未敢聽。今諸王苟能存亡繼絕,振弱伐暴,以安劉氏,社稷所願也。吳國雖貧,寡人節衣食用,積金錢,脩兵革,聚糧食,夜以繼日,三十餘年矣。凡皆為此,願諸王勉之。能斬捕大將者,賜金五千斤,封萬戶;列將,三千斤,封五千戶;裨將,二千斤,封二千戶;二千石,千斤,封千戶:皆為列侯。其以軍若城邑降者,卒萬人,邑萬戶,如得大將;人戶五千,如得列將;人戶三千,如得裨將;人戶千,如得二千石;其小吏皆以差次受爵金。它封賜皆倍軍法。其有故爵邑者,更益勿因。願諸王明以令士大夫,不敢欺也。寡人金錢在天下者往往而有,非必取於吳,諸王日夜用之不能盡。有當賜者告寡人,寡人且往遺之。敬以聞。」

七國反書聞,天子乃遣太尉條侯周亞夫將三十六將軍往擊吳楚;遣曲周侯酈寄擊趙,將軍欒布擊齊,大將軍竇嬰屯滎陽監齊趙兵。

初,吳楚反書聞,兵未發,竇嬰言故吳相爰盎。召入見,上問以吳楚之計,盎對曰:「吳楚相遺書,曰『賊臣朝錯擅適諸侯,削奪之地』,以故反,名為西共誅錯,復故地而罷。方今計獨斬錯,發使赦七國,復其故地,則兵可毋血刃而俱罷。」上從其議,遂斬錯。語具在盎傳。以盎為泰常,奉宗廟,使吳王,吳王弟子德侯為宗正,輔親戚。使至吳,吳楚兵已攻梁壁矣。宗正以親故,先入見,諭吳王拜受詔。吳王聞盎來,亦知其欲說,笑而應曰:「我已為東帝,尚誰拜?」不肯見盎而留軍中,欲劫使將。盎不肯,使人圍守,且殺之。盎得夜亡走梁,遂歸報。

條侯將乘六乘傳,會兵滎陽。至雒陽,見劇孟,喜曰:「

七國反,吾乘傳至此,不自意全。又以為諸侯已得劇孟。孟今無動,吾據滎陽,滎陽以東無足憂者。」至淮陽,問故父絳侯客鄧都尉曰:「策安出?」客曰:「吳楚兵銳甚,難與爭鋒。楚兵輕,不能久。方今為將軍計,莫若引兵東北壁昌邑,以梁委吳,吳必盡銳攻之。將軍深溝高壘,使輕兵絕淮泗口,塞吳饟道。使吳、梁相敝而糧食竭,乃以全制其極,破吳必矣。」條侯曰:「善。」從其策,遂堅壁昌邑南,輕兵絕吳饟道。

吳王之初發也,吳臣田祿伯為大將軍。田祿伯曰:「兵屯聚而西,無它奇道,難以立功。臣願得五萬人,別循江淮而上,收淮南、長沙,入武關,與大王會,此亦一奇也。」吳王太子諫曰:「王以反為名,此兵難以藉人,人亦且反王,柰何?且擅兵而別,多它利害,徒自損耳。」吳王即不許田祿伯。

吳少將桓將軍說王曰:「吳多步兵,步兵利險;漢多車騎,車騎利平地。願大王所過城不下,直去,疾西據雒陽武庫,食敖倉粟,阻山河之險以令諸侯,雖無入關,天下固已定矣。大王徐行,留下城邑,漢軍車騎至,馳入梁楚之郊,事敗矣。」吳王問吳老將,老將曰:「此年少椎鋒可耳,安知大慮!」於是王不用桓將軍計。

王專并將其兵,未度淮,諸賓客皆得為將、校尉、行間候、司馬,獨周丘不用。周丘者,下邳人,亡命吳,酤酒無行,王薄之,不任。周丘乃上謁,說王曰:「臣以無能,不得待罪行間。臣非敢求有所將也,願請王一漢節,必有以報。」王乃予之。周丘得節,夜馳入下邳。下邳時聞吳反,皆城守。至傳舍,召令入戶,使從者以罪斬令。遂召昆弟所善豪吏告曰:「吳反兵且至,屠下邳不過食頃。今先下,家室必完,能者封侯至矣。」出乃相告,下邳皆下。周丘一夜得三萬人,使人報吳王,遂將其兵北略城邑。比至城陽,兵十餘萬,破城陽中尉軍。聞吳王敗走,自度無與共成功,即引兵歸下邳。未至,癰發背死。

二月,吳王兵既破,敗走,於是天子制詔將軍:「蓋聞為善者天報以福,為非者天報以殃。高皇帝親垂功德,建立諸侯,幽王、悼惠王絕無後,孝文皇帝哀憐加惠,王幽王子遂,悼惠王子卬等,令奉其先王宗廟,為漢藩國,德配天地,明並日月。而吳王濞背德反義,誘受天下亡命罪人,亂天下幣,稱疾不朝二十餘年。有司數請濞罪,孝文皇帝寬之,欲其改行為善。今乃與楚王戊、趙王遂、膠西王卬、濟南王辟光、菑川王賢、膠東王雄渠約從謀反,為逆無道,起兵以危宗廟,賊殺大臣及漢使者,迫劫萬民,伐殺無罪,燒殘民家,掘其丘壟,甚為虐暴。而卬等又重逆無道,燒宗廟,鹵御物,朕甚痛之。朕素服避正殿,將軍其勸士大夫擊反虜。擊反虜者,深入多殺為功,斬首捕虜比三百石以上皆殺,無有所置。敢有議詔及不如詔者,皆要斬。」

初,吳王之度淮,與楚王遂西敗棘壁,乘勝而前,銳甚。梁孝王恐,遣將軍擊之,又敗梁兩軍,士卒皆還走。梁數使使條侯求救,條侯不許。又使使愬條侯於上,上使告條侯救梁,又守便宜不行。梁使韓安國及楚死事相弟張羽為將軍,乃得頗敗吳兵。吳兵欲西,梁城守,不敢西,即走條侯軍,會下邑。欲戰,條侯壁,不肯戰。吳糧絕,卒飢,數挑戰,遂夜奔條侯壁,驚東南。條侯使備西北,果從西北。不得入,吳大敗,士卒多飢死叛散。於是吳王乃與其戲下壯士千人夜亡去,度淮走丹徒,保東越。東越兵可萬餘人,使人收聚亡卒。漢使人以利啗東越,東越即紿吳王,吳王出勞軍,使人鏦殺吳王,盛其頭,馳傳以聞。吳王太子駒亡走閩越。吳王之棄軍亡也,軍遂潰,往往稍降太尉條侯及梁軍。楚王戊軍敗,自殺。

三王之圍齊臨菑也,三月不能下。漢兵至,膠西、膠東、菑川王各引兵歸國。膠西王徒跣,席稿,飲水,謝太后。王太子德曰:「漢兵還,臣觀之以罷,可襲,願收王餘兵擊之,不勝而逃入海,未晚也。」王曰:「吾士卒皆已壞,不可用。」不聽。漢將弓高侯頹當遺王書曰:「奉詔誅不義,降者赦,除其罪,復故;不降者滅之。王何處?須以從事。」王肉袒叩頭漢軍壁,謁曰:「臣卬奉法不謹,驚駭百姓,乃苦將軍遠道至于窮國,敢請葅醢之罪。」弓高侯執金鼓見之,曰:「王苦軍事,願聞王發兵狀。」王頓首膝行對曰:「今者,朝錯天子用事臣,變更高皇帝法令,侵奪諸侯地。卬等以為不義,恐其敗亂天下,七國發兵,且以誅錯。今聞錯已誅,卬等謹已罷兵歸。」將軍曰:「王苟以錯為不善,何不以聞?及未有詔虎符,擅發兵擊義國。以此觀之,意非徒欲誅錯也。」乃出詔書為王讀之,曰:「王其自圖之。」王曰:「如卬等死有餘罪。」遂自殺。太后、太子皆死。膠東、菑川、濟南王皆伏誅。酈將軍攻趙,十月而下之,趙王自殺。濟北王以劫故,不誅。

初,吳王首反,并將楚兵,連齊、趙。正月起,三月皆破滅。

贊曰:荊王王也,由漢初定,天下未集,故雖疏屬,以策為王,鎮江淮之間。劉澤發於田生,權激呂氏,然卒南面稱孤者三世。事發相重,豈不危哉!吳王擅山海之利,能薄斂以使其眾,逆亂之萌,自其子興。古者諸侯不過百里,山海不以封,蓋防此矣。朝錯為國遠慮,禍反及身。「毋為權首,將受其咎」,豈謂錯哉!


\end{pinyinscope}