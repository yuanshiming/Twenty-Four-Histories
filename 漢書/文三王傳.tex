\article{文三王傳}

\begin{pinyinscope}
孝文皇帝四男:竇皇后生孝景帝、梁孝王武,諸姬生代孝王參、梁懷王揖。

梁孝王武以孝文二年與太原王參、梁王揖同日立。武為代王,四年徙為淮陽王,十二年徙梁,自初王通歷已十一年矣。

孝王十四年,入朝。十七年,十八年,比年入朝,留。其明年,乃之國。二十一年,入朝。二十二年,文帝崩。二十四年,入朝。二十五年,復入朝。是時,上未置太子,與孝王宴飲,從容言曰:「千秋萬歲後傳於王。」王辭謝。雖知非至言,然心內喜。太后亦然。

其春,吳、楚、齊、趙七國反,先擊梁棘壁,殺數萬人。梁王城守睢陽,而使韓安國、張羽等為將軍以距吳、楚。吳、楚以梁為限,不敢過而西,與太尉亞夫等相距三月。吳、楚破,而梁所殺虜略與漢中分。

明年,漢立太子。梁最親,有功,又為大國,居天下膏腴地,北界泰山,西至高陽,四十餘城,多大縣。孝王,太后少子,愛之,賞賜不可勝道。於是孝王築東苑,方三百餘里,廣睢陽城七十里,大治宮室,為復道,自宮連屬於平臺三十餘里。得賜天子旌旗,從千乘萬騎,出稱警,入言旧,儗於天子。招延四方豪桀,自山東游士莫不至:齊人羊勝、公孫詭、鄒陽之屬。公孫詭多奇邪計,初見日,王賜千金,官至中尉,號曰公孫將軍。多作兵弩弓數十萬,而府庫金錢且百鉅萬,珠玉寶器多於京師。

二十九年十月,孝王入朝。景帝使使持乘輿駟,迎梁王於關下。既朝,上疏,因留。以太后故,入則侍帝同輦,出則同車遊獵上林中。梁之侍中、郎、謁者著引籍出入天子殿門,與漢宦官亡異。

十一月,上廢栗太子,太后心欲以梁王為嗣。大臣及爰盎等有所關說於帝,太后議格,孝王不敢復言太后以嗣事。事祕,世莫知,乃辭歸國。

其夏,上立膠東王為太子。梁王怨爰盎及議臣,乃與羊勝、公孫詭之屬謀,陰使人刺殺爰盎及他議臣十餘人。賊未得也。於是天子意梁,逐賊,果梁使之。遣使冠蓋相望於道,覆案梁事。捕公孫詭、羊勝,皆匿王後宮。使者責二千石急,梁相軒丘豹及內史安國皆泣諫王,王乃令勝、詭皆自殺,出之。上由此怨望於梁王。梁王恐,乃使韓安國因長公主謝罪太后,然後得釋。

上怒稍解,因上書請朝。既至關,茅蘭說王,使乘布車,從兩騎入,匿於長公主園。漢使迎王,王已入關,車騎盡居外,外不知王處。太后泣曰:「帝殺吾子!」弟憂恐。於是梁王伏斧質,之闕下謝罪。然後太后、帝皆大喜,相與泣,復如故。悉召王從官入關。然帝益疏王,不與同車輦矣。

三十五年冬,復入朝。上疏欲留,上弗許。歸國,意忽忽不樂。北獵梁山,有獻牛,足上出背上,孝王惡之。六月中,病熱,六日薨。

孝王慈孝,每聞太后病,口不能食,常欲留長安侍太后。太后亦愛之。及聞孝王死,竇太后泣極哀,不食,曰:「帝果殺吾子!」帝哀懼,不知所為。與長公主計之,乃分梁為五國,盡立孝王男五人為王,女五人皆令食湯沐邑。奏之太后,太后乃說,為帝壹餐。

孝王未死時,財以鉅萬計,不可勝數。及死,藏府餘黃金尚四十餘萬斤,他財物稱是。

代孝王參初立為太原王。四年,代王武徙為淮陽王,而參徙為代王,並得復太原,都晉陽如故。五年一朝,凡三朝。十七年薨,子共王登嗣。二十九年薨,子義嗣。元鼎中,漢廣關,以常山為阻,徙代王於清河,是為剛王。並前在代凡立四十年薨,子頃王湯嗣。二十四年薨,子年嗣。

地節中,冀州刺史林奏年為太子時與女弟則私通。及年立為王後,則懷年子,其婿使勿舉。則曰:「自來殺之。」婿怒曰:「

為王生子,自令王家養之。」則送兒頃太后所。相聞知,禁止則,令不得入宮。年使從季父往來送迎則,連年不絕。有司奏年淫亂,年坐廢為庶人,徙房陵,與湯沐邑百戶。立三年,國除。

元始二年,新都侯王莽興滅繼絕,白太皇太后,立年弟子如意為廣宗王,奉代孝王後。莽篡位,國絕。

梁懷王揖,文帝少子也。好詩書,帝愛之,異於他子。五年一朝,凡再入朝。因墮馬死,立十年薨。無子,國除。明年,梁孝王武徙王梁。

梁孝王子五人為王。太子買為梁共王,次子明為濟川王,彭離為濟東王,定為山陽王,不識為濟陰王,皆以孝景中六年同日立。

梁共王買立十年薨,子平王襄嗣。

濟川王明以垣邑侯立。七年,坐射殺其中尉,有司請誅,武帝弗忍,廢為庶人,徙房陵,國除。

濟東王彭離立二十九年。彭離驕悍,昏莫私與其奴亡命少年數十人行剽,殺人取財物以為好。所殺發覺者百餘人,國皆知之,莫敢夜行。所殺者子上書告言,有司請誅,武帝弗忍,廢為庶人,徙上庸,國除,為大河郡。

山陽哀王定立九年薨。亡子,國除。

濟陰哀王不識立一年薨。亡子,國除。

孝王支子四王,皆絕於身。

梁平王襄,母曰陳太后。共王母曰李太后。李太后,親平王之大母也。而平王之后曰任后,任后甚有寵於襄。

初,孝王有膑尊,直千金,戒後世善寶之,毋得以與人。任后聞而欲得之。李太后曰:「先王有命,毋得以尊與人。他物雖百鉅萬,猶自恣。」任后絕欲得之。王襄直使人開府取尊賜任后,又王及母陳太后事李太后多不順。有漢使者來,李太后欲自言,王使謁者中郎胡等遮止,閉門。李太后與爭門,措指,太后啼謼,不得見漢使者。李太后亦私與食官長及郎尹霸等姦亂,王與任后以此使人風止李太后。李太后亦已,後病薨。病時,任后未嘗請疾;薨,又不侍喪。

元朔中,睢陽人犴反,人辱其父,而與睢陽太守客俱出同車。犴反殺其仇車上,亡去。睢陽太守怒,以讓梁二千石。二千石以下求反急,執反親戚。反知國陰事,乃上變告梁王與大母爭尊狀。時相以下具知之,欲以傷梁長吏,書聞。天子下吏驗問,有之。公卿治,奏以為不孝,請誅王及太后。天子曰:「首惡失道,任后也。朕置相吏不逮,無以輔王,故陷不誼,不忍致法。」削梁王五縣,奪王太后湯沐成陽邑,梟任后首于市,中郎胡等皆伏誅。梁餘尚有八城。

襄立四十年薨,子頃王無傷嗣。十一年薨,子敬王定國嗣。四十年薨,子夷王遂嗣。六年薨,子荒王嘉嗣。十五年薨,子立嗣。

鴻嘉中,太傅輔奏:「立一日至十一犯法,臣下愁苦,莫敢親近,不可諫止。願令王,非耕、祠,法駕毋得出宮,盡出馬置外苑,收兵杖藏私府,毋得以金錢財物假賜人。」事下丞相、御史,請許。奏可。後數復敺傷郎,夜私出宮。傅相連奏,坐削或千戶或五百戶,如是者數焉。

荒王女弟園子為立舅任寶妻,寶兄子昭為立后。數過寶飲食,報寶曰:「我好翁主,欲得之。」寶曰:「翁主,姑也,法重。」立曰:「何能為!」遂與園子姦。

積數歲,永始中,相禹奏立對外家怨望,有惡言。有司案驗,因發淫亂事,奏立禽獸行,請誅。太中大夫谷永上疏曰:「臣聞『禮,天子外屏,不欲見外』也。是故帝王之意,不窺人閨門之私,聽聞中冓之言。春秋為親者諱。《詩》云『戚戚兄弟,莫遠具爾』。今梁王年少,頗有狂病,始以惡言按驗,既亡事實,而發閨門之私,非本章所指。王辭又不服,猥強劾立,傅致難明之事,獨以偏辭成罪斷獄,亡益於治道。汙衊宗室,以內亂之惡披布宣揚於天下,非所以為公族隱諱,增朝廷之榮華,昭聖德之風化也。臣愚以為王少,而父同產長,年齒不倫;梁國之富,足以厚聘美女,招致妖麗;父同產亦有恥辱之心。案事者乃驗問惡言,何故猥自發舒?以三者揆之,殆非人情,疑有所迫切,過誤失言,文吏躡尋,不得轉移。萌牙之時,加恩勿治,上也。既已案驗舉憲,宜及王辭不服,詔廷尉選上德通理之吏,更審考清問,著不然之效,定失誤之法,而反命於下吏,以廣公族附疏之德,為宗室刷汙亂之恥,甚得治親之誼。」天子由是寢而不治。

居數歲,元延中,立復以公事怨相掾及睢陽丞,使奴殺之,殺奴以滅口。凡殺三人,傷五人,手敺郎吏二十餘人。上書不拜奏。謀篡死罪囚。有司請誅,上不忍,削立五縣。

哀帝建平中,立復殺人。天子遣廷尉賞、大鴻臚由持節即訊。至,移書傅、相、中尉曰:「王背策戒,誖暴妄行,連犯大辟,毒流吏民。比比蒙恩,不伏重誅,不思改過,復賊殺人。幸得蒙恩,丞相長史、大鴻臚丞即問。王陽病抵讕,置辭驕嫚,不首主令,與背畔亡異。丞相、御史請收王璽綬,送陳留獄。明詔加恩,復遣廷尉、大鴻臚雜問。今王當受詔置辭,恐復不首實對。書曰:『至于再三,有不用,我降爾命。』傅、相、中尉皆以輔正為職,『虎兕出於匣,龜玉毀於匱中,是誰之過也?』書到,明以誼曉王。敢復懷詐,罪過益深。傅、相以下,不能輔導,有正法。」

立惶恐,免冠對曰:「立少失父母,孤弱處深宮中,獨與宦者婢妾居,漸漬小國之俗,加以質性下愚,有不可移之姿。往者傅相亦不純以仁誼輔翼立,大臣皆尚苛刻,刺求微密。讒臣在其間,左右弄口,積使上下不和,更相脐伺。宮殿之裏,毛氂過失,亡不暴陳。當伏重誅,以視海內,數蒙聖恩,得見貰赦。今立自知賊殺中郎曹將,冬月迫促,貪生畏死,即詐僵仆陽病,徼幸得踰於須臾。謹以實對,伏須重誅。」時冬月盡,其春大赦,不治。

元始中,立坐與平帝外家中山衛氏交通,新都侯王莽奏廢立為庶人,徙漢中。立自殺。二十七年,國除。後二歲,莽白太皇太后立孝王玄孫之曾孫沛郡卒史音為梁王,奉孝王後。莽篡,國絕。

贊曰:梁孝王雖以愛親故王膏腴之地,然會漢家隆盛,百姓殷富,故能殖其貨財,廣其宮室車服。然亦僭矣。怙親亡厭,牛禍告罰,卒用憂死,悲夫!


\end{pinyinscope}