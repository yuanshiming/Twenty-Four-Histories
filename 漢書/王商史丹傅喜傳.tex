\article{王商史丹傅喜傳}

\begin{pinyinscope}
王商字子威,涿郡蠡吾人也,徙杜陵。商父武,武兄無故,皆以宣帝舅封。無故為平昌侯,武為樂昌侯。語在外戚傳。

商少為太子中庶子,以肅敬敦厚稱。父薨,商嗣為侯,推財以分異母諸弟,身無所受,居喪哀轉。於是大臣薦商行可以厲群臣,義足以厚風俗,宜備近臣。繇是擢為諸曹侍中中郎將。元帝時,至右將軍、光祿大夫。是時,定陶共王愛幸,幾代太子。商為外戚重臣輔政,擁佑太子,頗有力焉。

元帝崩,成帝即位,甚敬重商,徙為左將軍。而帝元舅大司馬大將軍王鳳顓權,行多驕僭。商論議不能平鳳,鳳知之,亦疏商。建始三年秋,京師民無故相驚,言大水至,百姓奔走相蹂躪,長安中大亂。天子親御前殿,召公卿議。大將軍鳳以為太后與上及後宮可御船,令吏民上長安城以避水。群臣皆從鳳議。左將軍商獨曰:「自古無道之國,水猶不冒城郭。今政治和平,世無兵革,上下相安,何因當有大水一日暴至?此必訛言也,不宜令上城,重驚百姓。」上乃止。有頃,長安中稍定,問之,果訛言。上於是美壯商之固守,數稱其議。而鳳大慚,自恨失言。

明年,商代匡衡為丞相,益封千戶,天子甚尊任之。為人多質有威重,長八尺餘,身體鴻大,容貌甚過絕人。河平四年,單于來朝,引見白虎殿。丞相商坐未央廷中,單于前,拜謁商。商起,離席與言,單于仰視商貌,大畏之,遷延卻退。天子聞而歎曰:「此真漢相矣!」

初,大將軍鳳連昏楊肜為琅邪太守,其郡有災害十四,已上。商部屬按問,鳳以曉商曰:「災異天事,非人力所為。肜素善吏,宜以為後。」商不聽,竟奏免肜,奏果寑不下,鳳重以是怨商,陰求其短,使人上書言商閨門內事。天子以為暗昧之過,不足以傷大臣,鳳固爭,下其事司隸。

先是皇太后嘗詔問商女,欲以備後宮。時女病,商意亦難之,以病對,不入。及商以閨門事見考,自知為鳳所中,惶怖,更欲內女為援,乃因新幸李婕妤家白見其女。

會日有蝕之,太中大夫蜀郡張匡,其人佞巧,上書願對近臣陳日蝕咎。下朝者左將軍丹等問匡,對曰:「竊見丞相商作威作福,從外制中,取必於上,性殘賊不仁,遣票輕吏微求人罪,欲以立威,天下患苦之。前頻陽耿定上書言商與父傅通,及女弟淫亂,奴殺其私夫,疑商教使。章下有司,商私怨懟。商子俊欲上書告商,俊妻左將軍丹女,持其書以示丹,丹惡其父子乖迕,為女求去。商不盡忠納善以輔至德,知聖主崇孝,遠別不親,後庭之事皆受命皇太后,太后前聞商有女,欲以備後宮,商言有固疾,後有耿定事,更詭道因李貴人家內女。執左道以亂政,誣罔誖大臣節,故應是而日蝕。《周書》曰:『以左道事君者誅。』《易》曰:『日中見昧,則折其右肱。』往者丞相周勃再建大功,及孝文時纖介怨恨,而日為之蝕,於是退勃使就國,卒無怵悐憂。今商無尺寸之功,而有三世之寵,身位三公,宗族為列侯、吏二千石、侍中諸曹,給事禁門內,連昏諸侯王,權寵至盛。審有內亂殺人怨懟之端,宜窮意考問。臣聞秦丞相呂不韋見王無子,意欲有秦國,即求好女以為妻,陰知其有身而獻之王,產始皇帝。及楚相春申君亦見王無子,心利楚國,即獻有身妻而產懷王。自漢興幾遭呂、霍之患,今商有不仁之性,乃因怨以內女,其姦謀未可測度。前孝景世七國反,將軍周亞夫以為即得雒陽劇孟,關東非漢之有。今商宗族權勢,合貲鉅萬計,私奴以千數,非特劇孟匹夫之徒也。且失道之至,親戚畔之,閨門內亂,父子相訐,而欲使之宣明聖化,調和海內,豈不謬哉!商視事五年,官職陵夷而大惡著於百姓,甚虧損盛德,有鼎折足之凶。臣愚以為聖主富於春秋,即位以來,未有懲姦之威,加以繼嗣未立,大異並見,尤宜誅討不忠,以遏未然。行之一人,則海內震動,百姦之路塞矣。」

於是左將軍丹等奏:「商位三公,爵列侯,親受詔策為天下師,不遵法度以翼國家,而回辟下媚以進其私,執左道以亂政,為臣不忠,罔上不道,甫刑之辟,皆為上戮,罪名明白。臣請詔謁者召商詣若盧詔獄。」上素重商,知匡言多險,制曰「弗治」。鳳固爭之,於是制詔御史:「蓋丞相以德輔翼國家,典領百寮,協和萬國,為職任莫重焉。今樂昌侯商為丞相,出入五年,未聞忠言嘉謀,而有不忠執左道之辜,陷于大辟。前商女弟內行不修,奴賊殺人,疑商教使,為商重臣,故抑而不窮。今或言商不以自悔而反怨懟,朕甚傷之。惟商與先帝有外親,未忍致于理。其赦商罪。使者收丞相印綬。」

商免相三日,發病蓝血薨,諡曰戾侯。而商子弟親屬為駙馬都尉、侍中、中常侍、諸曹大夫郎吏者,皆出補吏,莫得留給事宿衛者。有司奏商罪過未決,請除國邑。有詔長子安嗣爵為樂昌侯,至長樂衛尉、光祿勳。

商死後,連年日蝕地震,直臣京兆尹王章上封事召見,訟商忠直無罪,言鳳顓權蔽主。鳳竟以法誅章,語在元后傳。至元始中,王莽為安漢公,誅不附己者,樂昌侯安見被以罪,自殺,國除。

史丹字君仲,魯國人也,徙杜陵。祖父恭有女弟,武帝時為衛太子良娣,產悼皇考。皇考者,孝宣帝父也。宣帝微時依倚史氏。語在史良娣傳。及宣帝即尊位,恭已死,三子,高、曾、玄。曾、玄皆以外屬舊恩封,曾為將陵侯,玄平臺侯。高侍中貴幸,以發舉反者大司馬霍禹功封樂陵侯。宣帝疾病,拜高為大司馬車騎將軍,領尚書事。帝崩,太子襲尊號,是為孝元帝。高輔政五年,乞骸骨,賜安車駟馬黃金,罷就第。薨,諡曰安侯。

自元帝為太子時,丹以父高任為中庶子,侍從十餘年。元帝即位,為駙馬都尉侍中,出常驂乘,甚有寵。上以丹舊臣,皇考外屬,親信之,詔丹護太子家。是時,傅昭儀子定陶共王有材藝,子母俱愛幸,而太子頗有酒色之失,母王皇后無寵。

建昭之間,元帝被疾,不親政事,留好音樂。或置鼙鼓殿下,天子自臨軒檻上,隤銅丸以擿鼓,聲中嚴鼓之節。後宮及左右習知音者莫能為,而定陶王亦能之,上數稱其材。丹進曰:「凡所謂材者,敏而好學,溫故知新,皇太子是也。若乃器人於絲竹鼓鼙之間,則是陳惠、李微高於匡衡,可相國也。」於是上嘿然而笑。其後,中山哀王薨,太子前弔。哀王者,帝之少弟,與太子遊學相長大。上望見太子,感念哀王,悲不能自止。太子既至前,不哀。上大恨曰:「安有人不慈仁而可奉宗廟為民父母者乎!」上以責謂丹。丹免冠謝上曰:「臣誠見陛下哀痛中山王,至以感損。向者太子當進見,臣竊戒屬毋涕泣,感傷陛下。罪乃在臣,當死。」上以為然,意乃解。丹之輔相,皆此類也。

竟寧元年,上寑疾,傅昭儀及定陶王常在左右,而皇后太子希得進見。上疾稍侵,意忽忽不平,數問尚書以景帝時立膠東王故事。是時,太子長舅陽平侯王鳳為衛尉侍中,與皇后太子皆憂,不知所出。丹以親密臣得侍視疾,候上間獨寢時,丹直入臥內,頓首伏青蒲上,涕泣言曰:「皇太子以適長立,積十餘年,名號繫於百姓,天下莫不歸心臣子。見定陶王雅素愛幸,今者道路流言,為國生意,以為太子有動搖之議。審若此,公卿以下必以死爭,不奉詔。臣願先賜死以示群臣!」天子素仁,不忍見丹涕泣,言又切至,上意大感,喟然太息曰:「吾日困劣,而太子兩王幼少,意中戀戀,亦何不念乎!然無有此議。且皇后謹慎,先帝又愛太子,吾豈可違指!駙馬都尉安所受此語?」丹即卻,頓首曰:「

愚臣妄聞,罪當死!」上因納,謂丹曰:「吾病寖加,恐不能自還。善輔道太子,毋違我意!」丹噓唏而起。太子由是遂為嗣矣。

元帝竟崩,成帝初即位,擢丹為長樂衛尉,遷右將軍,賜爵關內侯,食邑三百戶,給事中,後徙左將軍、光祿大夫。鴻嘉元年,上遂下詔曰:「夫褒有德,賞元功,古今通義也。左將軍丹往時導朕以忠正,秉義醇壹,舊德茂焉。其封丹為武陽侯,國東海郯之武彊聚,戶千一百。」

丹為人足知,愷弟愛人,貌若儻蕩不備,然心甚謹密,故尤得信於上。丹兄嗣父爵為侯,讓不受分。丹盡得父財,身又食大國邑,重以舊恩,數見褒賞,賞賜累千金,僮奴以百數,後房妻妾數十人,內奢淫,好飲酒,極滋味聲色之樂。為將軍前後十六年,永始中病乞骸骨,上賜策曰:「左將軍寑病不衰,願歸治疾,朕愍以官職之事久留將軍,使躬不瘳。使光祿勳賜將軍黃金五十斤,安車駟馬,其上將軍印綬。宜專精神,務近醫藥,以輔不衰。」

丹歸第數月薨,諡曰頃侯。有子男女二十人,九男皆以丹任並為侍中諸曹,親近在左右。史氏凡四人侯,至卿大夫二千石者十餘人,皆訖王莽乃絕,唯將陵侯曾無子,絕於身云。

傅喜字稚游,河內溫人也,哀帝祖母定陶傅太后從父弟。少好學問,有志行。哀帝立為太子,成帝選喜為太子庶子。哀帝初即位,以喜為衛尉,遷右將軍。是時,王莽為大司馬,乞骸骨,避帝外家。上既聽莽退,眾庶歸望於喜。喜從弟孔鄉侯晏親與喜等,而女為皇后。又帝舅陽安侯丁明,皆親以外屬封。喜執謙稱疾。傅太后始與政事,喜數諫之,由是傅太后不欲令喜輔政。上於是用左將軍師丹代王莽為大司馬,賜喜黃金百斤,上將軍印綬,以光祿大夫養病。

大司空何武、尚書令唐林皆上書言:「喜行義修絜,忠誠憂國,內輔之臣也,今以寑病,一旦遣歸,眾庶失望,皆曰傅氏賢子,以論議不合於定陶太后故退,百寮莫不為國恨之。忠臣,社稷之衛,魯以季友治亂,楚以子玉輕重,魏以無忌折衝,項以范增存亡。故楚跨有南土,帶甲百萬,鄰國不以為難,子玉為將,則文公側席而坐,及其死也,君臣相慶。百萬之眾,不如一賢,故秦行千金以間廉頗,漢散萬金以疏亞父。喜立於朝,陛下之光煇,傅氏之廢興也。」上亦自重之。明年正月,乃徙師丹為大司空,而拜喜為大司馬,封高武侯。

丁、傅驕奢,皆嫉喜之恭儉。又傅太后欲求稱尊號,與成帝母齊尊,喜與丞相孔光、大司空師丹共執正議。傅太后大怒,上不得已,先免師丹以感動喜,喜終不順。後數月,遂策免喜曰:「君輔政出入三年,未有昭然匡朕不逮,而本朝大臣遂其姦心,咎由君焉。其上大司馬印綬,就第。」傅太后又自詔丞相御史曰:「高武侯喜無功而封,內懷不忠,附下罔上,與故大司空丹同心背畔,放命圮族,虧損德化,罪惡雖在赦前,不宜奉朝請,其遣就國。」後又欲奪喜侯,上亦不聽。

喜在國三歲餘,哀帝崩,平帝即位,王莽用事,免傅氏官爵歸故郡,晏將妻子徙合浦。莽白太后下詔曰:「高武侯喜姿性端瓓,論議忠直,雖與故定陶太后有屬,終不順指從邪,介然守節,以故斥逐就國。傳不云乎?『歲寒然後知松柏之後凋也。』其還喜長安,以故高安侯莫府賜喜,位特進,奉朝請。」喜雖外見褒賞,孤立憂懼,後復遣就國,以壽終。莽賜諡曰貞侯。子嗣,莽敗乃絕。

贊曰:自宣、元、成、哀外戚興者,許、史、三王、丁、傅之家,皆重侯累將,窮貴極富,見其位矣,未見其人也。陽平之王多有材能,好事慕名,其勢尤盛,曠貴最久。然至於莽,亦以覆國。王商有剛毅節,廢黜以憂死,非其罪也。史丹父子相繼,高以重厚,位至三公。丹之輔道副主,掩惡揚美,傅會善意,雖宿儒達士無以加焉。及其歷房闥,入臥內,推至誠,犯顏色,動寤萬乘,轉移大謀,卒成太子,安母后之位。「無言不讎」,終獲忠貞之報。傅喜守節不傾,亦蒙後凋之賞。哀、平際會,禍福速哉!


\end{pinyinscope}