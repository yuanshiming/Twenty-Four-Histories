\article{東方朔傳}

\begin{pinyinscope}
東方朔字曼倩,平原厭次人也。武帝初即位,徵天下舉方正賢良文學材力之士,待以不次之位,四方士多上書言得失,自衒鬻者以千數,其不足采者輒報聞罷。朔初來,上書曰:「臣朔少失父母,長養兄嫂。年十三學書,三冬文史足用。十五學擊劍。十六學詩書,誦二十二萬言。十九學孫吳兵法,戰陣之具,鉦鼓之教,亦誦二十二萬言。凡臣朔固已誦四十四萬言。又常服子路之言。臣朔年二十二,長九尺三寸,目若懸珠,齒若編貝,勇若孟賁,捷若慶忌,廉若鮑叔,信若尾生。若此,可以為天子大臣矣。臣朔昧死再拜以聞。」

朔文辭不遜,高自稱譽,上偉之,令待詔公車,奉祿薄,未得省見。

久之,朔紿騶朱儒,曰:「上以若曹無益於縣官,耕田力作固不及人,臨眾處官不能治民,從軍擊虜不任兵事,無益於國用,徒索衣食,今欲盡殺若曹。」朱儒大恐,啼泣。朔教曰:「上即過,叩頭請罪。」居有頃,聞上過,朱儒皆號泣頓首。上問:「何為?」對曰:「東方朔言上欲盡誅臣等。」上知朔多端,召問朔:「何恐朱儒為?」對曰:「臣朔生亦言,死亦言。朱儒長三尺餘,奉一囊粟,錢二百四十。臣朔長九尺餘,亦奉一囊粟,錢二百四十。朱儒飽欲死,臣朔飢欲死。臣言可用,幸異其禮;不可用,罷之,無令但索長安米。」上大笑,因使待詔金馬門,稍得親近。

上嘗使諸數家射覆,置守宮盂下,射之,皆不能中。朔自贊曰:「臣嘗受易,請射之。」乃別蓍布卦而對曰:「臣以為龍又無角,謂之為蛇又有足,跂跂脈脈善緣壁,是非守宮即蜥蜴。」上曰:「善。」賜帛十匹。復使射他物,連中,輒賜帛。

時有幸倡郭舍人,滑稽不窮,常侍左右,曰:「朔狂,幸中耳,非至數也。臣願令朔復射,朔中之,臣榜百,不能中,臣賜帛。」乃覆樹上寄生,令朔射之。朔曰:「是寠藪也。」舍人曰:「果知朔不能中也。」朔曰:「生肉為膾,乾肉為脯;著樹為寄生,盆下為寠數。」上令倡監榜舍人,舍人不勝痛,呼謈。朔笑之曰:「咄!口無毛,聲謷謷,犊益高。」舍人恚曰:「朔擅詆欺天子從官,當棄巿。」上問朔:「何故詆之?」對曰:「臣非敢詆之,乃與為隱耳。」上曰:「隱云何?」朔曰:「夫口無毛者,狗竇也;聲謷謷者,鳥哺鷇也;尻益高者,鶴俛啄也。」舍人不服,因曰:「臣願復問朔隱語,不知,亦當榜。」即妄為諧語曰:「令壺齟,老柏塗,伊優亞,狋吽牙。何謂也?」朔曰:「令者,命也。壺者,所以盛也。齟者,齒不正也。老者,人所敬也。柏者,鬼之廷也。塗者,漸洳徑也。伊優亞者,辭未定也。狋吽牙者,兩犬爭也。」舍人所問,朔應聲輒對,變詐鋒出,莫能窮者,左右大驚。上以朔為常侍郎,遂得愛幸。

久之,伏日,詔賜從官肉。大官丞日晏不來,朔獨拔劍割肉,謂其同官曰:「伏日當蚤歸,請受賜。」即懷肉去。大官奏之。朔入,上曰:「昨賜肉,不待詔,以劍割肉而去之,何也?」朔免冠謝。上曰:「先生起自責也。」朔再拜曰:「朔來!朔來!受賜不待詔,何無禮也!拔劍割肉,壹何壯也!割之不多,又何廉也!歸遺細君,又何仁也!」上笑曰:「使先生自責,乃反自譽!」復賜酒一石,肉百斤,歸遺細君。

初,建元三年,微行始出,北至池陽,西至黃山,南獵長楊,東游宜春。微行常用飲酎已。八九月中,與侍中常侍武騎及待詔隴西北地良家子能騎射者期諸殿門,故有「期門」之號自此始。微行以夜漏下十刻乃出,常稱平陽侯。旦明,入山下馳射鹿豕狐兔,手格熊羆,馳騖禾稼稻岻之地。民皆號呼罵詈,相聚會,自言鄠杜令。令往,欲謁平陽侯,諸騎欲擊鞭之。令大怒,使吏呵止,獵者數騎見留,乃示以乘輿物,久之乃得去。時夜出夕還,後齎五日糧,會朝長信宮,上大驩樂之。是後,南山下乃知微行數出也,然尚迫於太后,未敢遠出。丞相御史知指,乃使右輔都尉徼循長楊以東,右內史發小民共待會所。後乃私置更衣,從宣曲以南十二所,中休更衣,投宿諸宮,長楊、五柞、倍陽、宣曲尤幸。於是上以為道遠勞苦,又為百姓所患,乃使太中大夫吾丘壽王與待詔能用算者二人,舉籍阿城以南,盩厔以東,宜春以西,提封頃畝,及其賈直,欲除以為上林苑,屬之南山。又詔中尉、左右內史表屬縣草田,欲以償鄠杜之民。吾丘壽王奏事,上大說稱善。時朔在傍,進諫曰:

臣聞謙遜靜愨,天表之應,應之以福;驕溢靡麗,天表之應,應之以異。今陛下累郎臺,恐其不高也;弋獵之處,恐其不廣也。如天不為變,則三輔之地盡可以為苑,何必盩厔、鄠、杜乎!奢侈越制,天為之變,上林雖小,臣尚以為大也。

夫南山,天下之阻也,南有江淮,北有河渭,其地從汧隴以東,商雒以西,厥壤肥饒。漢興,去三河之地,止霸產以西,都涇渭之南,此所謂天下陸海之地,秦之所以虜西戎兼山東者也。其山出玉石,金、銀、銅、鐵,豫章、檀、柘,異類之物,不可勝原,此百工所取給,萬民所卬足也。又有岻稻梨栗桑麻竹箭之饒,土宜薑芋,水多麴魚,貧者得以人給家足,無飢寒之憂。故酆鎬之間號為土膏,其賈畝一金。今規以為苑,絕陂池水澤之利,而取民膏腴之地,上乏國家之用,下奪農桑之業,棄成功,就敗事,損耗五穀,是其不可一也。且盛荊棘之林,而長養麋鹿,廣狐兔之苑,大虎狼之虛,又壞人冢墓,發人室廬,令幼弱懷土而思,耆老泣涕而悲,是其不可二也。斥而營之,垣而囿之,騎馳東西,車騖南北,又有深溝大渠,夫一日之樂不足以危無隄之輿,是其不可三也。故務苑囿之大,不恤農時,非所以彊國富人也。

夫殷作九巿之宮而諸侯畔,靈王起章華之臺而楚民散,秦興阿房之殿而天下亂。糞土愚臣,忘生觸死,逆盛意,犯隆指,罪當萬死,不勝大願,願陳泰階六符,以觀天變,不可不省。

是日因奏泰階之事,上乃拜朔為太中大夫給事中,賜黃金百斤。然遂起上林苑,如壽王所奏云。

久之,隆慮公主子昭平君尚帝女夷安公主,隆慮主病困,以金千斤錢千萬為昭平君豫贖死罪,上許之。隆慮主卒,昭平君日驕,醉殺主傅,獄繫內官。以公主子,廷尉上請請論。左右人人為言:「前又入贖,陛下許之。」上曰:「吾弟老有是一子,死以屬我。」於是為之垂涕歎息,良久曰:「法令者,先帝所造也,用弟故而誣先帝之法,吾何面目入高廟乎!又下負萬民。」乃可其奏,哀不能自止,左右盡悲。朔前上壽,曰:「臣聞聖王為政,賞不避仇讎,誅不擇骨肉。書曰:『不偏不黨,王道蕩蕩。』此二者,五帝所重,三王所難也。陛下行之,是以四海之內元元之民各得其所,天下幸甚!臣朔奉觴,昧死再拜上萬歲壽。」上乃起,入省中,夕時召讓朔,曰:「傳曰『時然後言,人不厭其言』。今先生上壽,時乎?」朔免冠頓首曰:「臣聞樂太甚則陽溢,哀太甚則陰損,陰陽變則心氣動,心氣動則精神散,精神散而邪氣及。銷憂者莫若酒,臣朔所以上壽者,明陛下正而不阿,因以止哀也。愚不知忌諱,當死。」先是,朔嘗醉入殿中,小遺殿上,劾不敬。有詔免為庶人,待詔宦者署,因此時復為中郎,賜帛百匹。

初,帝姑館陶公主號竇太主,堂邑侯陳午尚之。午死,主寡居,年五十餘矣,近幸董偃。始偃與母以賣珠為事,偃年十三,隨母出入主家。左右言其姣好,主召見,曰:「吾為母養之。」因留第中,教書計相馬御射,頗讀傳記。至年十八而冠,出則執轡,入則侍內。為人溫柔愛人,以主故,諸公接之,名稱城中,號曰董君。主因推令散財交士,令中府曰:「董君所發,一日金滿百斤,錢滿百萬,帛滿千匹,乃白之。」安陵爰叔者,爰盎兄子也,與偃善,謂偃曰:「足下私侍漢主,挾不測之罪,將欲安處乎?」偃懼曰:「憂之久矣,不知所以。」爰叔曰:「

顧城廟遠無宿宮,又有萩竹籍田,足下何不白主獻長門園?此上所欲也。如是,上知計出於足下也,則安枕而臥,長無慘怛之憂。久之不然,上且請之,於足下何如?」偃頓首曰:「敬奉教。」入言之主,主立奏書獻之。上大說,更名竇太主園為長門宮。主大喜,使偃以黃金百斤為爰叔壽。

叔因是為董君畫求見上之策,令主稱疾不朝。上往臨疾,問所欲,主辭謝曰:「妾幸蒙陛下厚恩,先帝遺德,奉朝請之禮,備臣妾之儀,列為公主,賞賜邑入,隆天重地,死無以塞責。一日卒有不勝洒掃之職,先狗馬填溝壑,竊有所恨,不勝大願,願陛下時忘萬事,養精游神,從中掖庭回輿,枉路臨妾山林,得獻觴上壽,娛樂左右。如是而死,何恨之有!」上曰:「主何憂?幸得愈。恐群臣從官多,大為主費。」上還。有頃,主疾愈,起謁,上以錢千萬從主飲。後數日,上臨山林,主自執宰敝膝,道入登階就坐。坐未定,上曰:「願謁主人翁。」主乃下殿,去簪珥,徒跣頓首謝曰:「妾無狀,負陛下,身當伏誅。陛下不致之法,頓首死罪。」有詔謝。主簪履起,之東箱自引董君。董君綠幘傅韝,隨主前,伏殿下。主乃贊:「館陶公主胞人臣偃昧死再拜謁。」因叩頭謝,上為之起。有詔賜衣冠上。偃起,走就衣冠。主自奉食進觴。當是時,董君見尊不名,稱為「主人翁」,飲大驩樂。主乃請賜將軍列侯從官金錢雜繒各有數。於是董君貴寵,天下莫不聞。郡國狗馬蹴鞠劍客輻湊董氏。常從游戲北宮,馳逐平樂,觀雞鞠之會,角狗馬之足,上大歡樂之。於是上為竇太主置酒宣室,使謁者引內董君。

是時,朔陛戟殿下,辟戟而前曰:「董偃有斬罪三,安得入乎?」上曰:「何謂也?」朔曰:「偃以人臣私侍公主,其罪一也。敗男女之化,而亂婚姻之禮,傷王制,其罪二也。陛下富於春秋,方積思於六經,留神於王事,馳騖於唐虞,折節於三代,偃不遵經勸學,反以靡麗為右,奢侈為務,盡狗馬之樂,極耳目之欲,行邪枉之道,徑淫辟之路,是乃國家之大賊,人主之大蜮。偃為淫首,其罪三也。昔伯姬燔而諸侯憚,奈何乎陛下?」上默然不應,良久曰:「吾業以設飲,後而自改。」朔曰:「

不可。夫宣室者,先帝之正處也,非法度之政不得入焉。故淫亂之漸,其變為篡,是以豎貂為淫而易牙作患,慶父死而魯國全,管蔡誅而周室安。」上曰:「善。」有詔止,更置酒北宮,引董君從東司馬門。東司馬門更名東交門。賜朔黃金三十斤。董君之寵由是日衰,至年三十而終。後數歲,竇太主卒,與董君會葬於霸陵。是後,公主貴人多踰禮制,自董偃始。

時天下侈靡趨末,百姓多離農畝。上從容問朔:「吾欲化民,豈有道乎?」朔對曰:「堯舜禹湯文武成康上古之事,經歷數千載,尚難言也,臣不敢陳。願近述孝文皇帝之時,當世耆老皆聞見之。貴為天下,富有四海,身衣弋綈,足履革舄,以韋帶劍,莞蒲為席,兵木無刃,衣縕無文,集上書囊以為殿帷;以道德為麗,以仁義為準。於是天下望風成俗,昭然化之。今陛下以城中為小,圖起建章,左鳳闕,右神明,號稱千門萬戶;木土衣綺繡,狗馬被繢罽;宮人簪玳瑁,垂珠璣;設戲車,教馳逐,飾文采,藂珍怪;撞萬石之鐘,趋雷霆之鼓,作俳優,舞鄭女。上為淫侈如此,而欲使民獨不奢侈失農,事之難者也。陛下誠能用臣朔之計,推甲乙之帳燔之於四通之衢,卻走馬示不復用,則堯舜之隆宜可與比治矣。《易》曰:『正其本,萬事理;失之豪氂,差以千里。』願陛下留意察之。」

朔雖詼笑,然時觀察顏色,直言切諫,上常用之。自公卿在位,朔皆敖弄,無所為屈。

上以朔口諧辭給,好作問之。嘗問朔曰:「先生視朕何如主也?」朔對曰:「自唐虞之隆,成康之際,未足以諭當世。臣伏觀陛下功德,陳五帝之上,在三王之右。非若此而已,誠得天下賢士,公卿在位咸得其人矣。譬若以周邵為丞相,孔丘為御史大夫,太公為將軍,畢公高拾遺於後,弁嚴子為衛尉,皋陶為大理,后稷為司農,伊尹為少府,子贛使外國,顏閔為博士,子夏為太常,益為右扶風,季路為執金吾,契為鴻臚,龍逢為宗正,伯夷為京兆,管仲為馮翊,魯般為將作,仲山甫為光祿,申伯為太僕,延陵季子為水衡,百里奚為典屬國,柳下惠為大長秋,史魚為司直,蘧伯玉為太傅,孔父為詹事,孫叔敖為諸侯相,子產為郡守,王慶忌為期門,夏育為鼎官,羿為旄頭,宋萬為式道候。」上乃大笑。

是時朝廷多賢材,上復問朔:「方今公孫丞相、兒大夫、董仲舒、夏侯始昌、司馬相如、吾丘壽王、主父偃、朱買臣、嚴助、汲黯、膠倉、終軍、嚴安、徐樂、司馬遷之倫,皆辯知閎達,溢于文辭,先生自視,何與比哉?」朔對曰:「臣觀其臿齒牙,樹頰胲,吐脣吻,擢項頤,結股腳,連脽尻,遺蛇其跡,行步偊旅,臣朔雖不肖,尚兼此數子者。」朔之進對澹辭,皆此類也。

武帝既招英俊,程其器能,用之如不及。時方外事胡越,內興制度,國家多事,自公孫弘以下至司馬遷皆奉使方外,或為郡國守相至公卿,而朔嘗至太中大夫,後常為郎,與枚皋、郭舍人俱在左右,詼啁而已。久之,朔上書陳農戰彊國之計,因自訟獨不得大官,欲求試用。其言專商鞅、韓非之語也,指意放蕩,頗復詼諧,辭數萬言,終不見用。朔因著論,設客難己,用位卑以自慰諭。其辭曰:

客難東方朔曰:「蘇秦、張儀一當萬乘之主,而都卿相之位,澤及後世。今子大夫修先王之術,慕聖人之義,諷誦詩書百家之言,不可勝數,著於竹帛,脣腐齒落,服膺而不釋,好學樂道之效,明白甚矣;自以智能海內無雙,則可謂博聞辯智矣。然悉力盡忠以事聖帝,曠日持久,官不過侍郎,位不過執戟,意者尚有遺行邪?同胞之徒無所容居,其故何也?」

東方先生喟然長息,仰而應之曰:「是固非子之所能備也。彼一時也,此一時也,豈可同哉?夫蘇秦、張儀之時,周室大壞,諸侯不朝,力政爭權,相禽以兵,并為十二國,未有雌雄,得士者彊,失士者亡,故談說行焉。身處尊位,珍寶充內,外有廩倉,澤及後世,子孫長享。今則不然。聖帝流德,天下震懾,諸侯賓服,連四海之外以為帶,安於覆盂,動猶運之掌,賢不肖何以異哉?遵天之道,順地之理,物無不得其所;故綏之則安,動之則苦;尊之則為將,卑之則為虜;抗之則在青雲之上,抑之則在深泉之下;用之則為虎,不用則為鼠;雖欲盡節效情,安知前後?夫天地之大,士民之眾,竭精談說,並進輻湊者不可勝數,悉力募之,困於衣食,或失門戶。使蘇秦、張儀與僕並生於今之世,曾不得掌故,安敢望常侍郎乎!故曰時異事異。

「雖然,安可以不務修身乎哉!《詩》云:『鼓鐘于宮,聲聞于外。』『鶴鳴于九皋,聲聞于天。』苟能修身,何患不榮!太公體行仁義,七十有二延設用於文武,得信厥說,封於齊,七百歲而不絕。此士所以日夜孳孳,敏行而不敢怠也。辟若鶺鴒,飛且鳴矣。傳曰:『天不為人之惡寒而輟其冬,地不為人之惡險而輟其廣,君子不為小人之匈匈而易其行。』『天有常度,地有常形,君子有常行;君子道其常,小人計其功。』《詩》云:『禮義之不愆,何恤人之言?』故曰:『水至清則無魚,人至察則無徒,冕而前旒,所以蔽明;黈纊充耳,所以塞聰。』明有所不見,聰有所不聞,舉大德,赦小過,無求備於一人之義也。枉而直之,使自得之;優而柔之,使自求之;揆而度之,使自索之。蓋聖人教化如此,欲自得之;自得之,則敏且廣矣。

「今世之處士,魁然無徒,廓然獨居,上觀許由,下察接輿,計同范蠡,忠合子胥,天下和平,與義相扶,寡耦少徒,固其宜也,子何疑於我哉?若夫燕之用樂毅,秦之任李斯,酈食其之下齊,說行如流,曲從如環,所欲必得,功若丘山,海內定,國家安,是遇其時也,子又何怪之邪!語曰『以筦闚天,以蠡測海,以莛撞鐘』,豈能通其條貫,考其文理,發其音聲哉!繇是觀之,譬猶鼱鼩之襲狗,孤豚之咋虎,至則靡耳,何功之有?今以下愚而非處士,雖欲勿困,固不得已,此適足以明其不知權變而終或於大道也。」

又設非有先生之論,其辭曰:

非有先生仕於吳,進不稱往古以厲主意,退不能揚君美以顯其功,默默無言者三年矣。吳王怪而問之,曰:「寡人獲先人之功,寄於眾賢之上,夙興夜寐,未嘗敢怠也。今先生率然高舉,遠集吳地,將以輔治寡人,誠竊嘉之,體不安席,食不甘味,目不視靡曼之色,耳不聽鐘鼓之音,虛心定志欲聞流議者三年于茲矣。今先生進無以輔治,退不揚主譽,竊不為先生取之也。蓋懷能而不見,是不忠也;見而不行,主不明也。意者寡人殆不明乎?」非有先生伏而唯唯。吳王曰:「可以談矣,寡人將竦意而覽焉。」先生曰:「於戲!可乎哉?可乎哉?談何容易!夫談有悖於目拂於耳謬於心而便於身者,或有說於目順於耳快於心而毀於行者,非有明王聖主,孰能聽之?」吳王曰:「何為其然也?『中人已上可以語上也。』先生試言,寡人將聽焉。」

先生對曰:「昔者關龍逢深諫於桀,而王子比干直言於紂,此二臣者,皆極慮盡忠,閔王澤不下流,而萬民騷動,故直言其失,切諫其邪者,將以為君之榮,除主之禍也。今則不然,反以為誹謗君之行,無人臣之禮,果紛然傷於身,蒙不辜之名,戮及先人,為天下笑,故曰談何容易!是以輔弼之臣瓦解,而邪諂之人並進,及蜚廉、惡來輩等。二人皆詐偽,巧言利口以進其身,陰奉琱瑑刻鏤之好以納其心。務快耳目之欲,以苟容為度。遂往不戒,身沒被戮,宗廟崩阤,國家為虛,放戮聖賢,親近讒夫。詩不云乎?『讒人罔極,交亂四國』,此之謂也。故卑身賤體,說色微辭,愉愉咰咰,終無益於主上之治,則志士仁人不忍為也。將儼然作矜嚴之色,深言直諫,上以拂主之邪,下以損百姓之害,則忤於邪主之心,歷於衰世之法。故養壽命之士莫肯進也,遂居家山之間,積土為室,編蓬為戶,彈琴其中,以詠先王之風,亦可以樂而忘死矣。是以伯夷叔齊避周,餓于首陽之下,後世稱其仁。如是,邪主之行固足畏也,故曰談何容易!」

於是吳王懼然易容,捐薦去几,危坐而聽。先生曰:「接輿避世,箕子被髮陽狂,此二人者,皆避濁世以全其身者也。使遇明王聖主,得清燕之閒,寬和之色,發憤畢誠,圖畫安危,揆度得失,上以安主體,下以便萬民,則五帝三王之道可幾而見也。故伊尹蒙恥辱負鼎俎和五味以干湯,太公釣於渭之陽以見文王。心合意同,謀無不成,計無不從,誠得其君也。深念遠慮,引義以正其身,推恩以廣其下,本仁祖義,褒有德,祿賢能,誅惡亂,總遠方,一統類,美風俗,此帝王所由昌也。上不變天性,下不奪人倫,則天地和洽,遠方懷之,故號聖王。臣子之職既加矣,於是裂地定封,爵為公侯,傳國子孫,名顯後世,民到于今稱之,以遇湯與文王也。太公、伊尹以如此,龍逢、比干獨如彼,豈不哀哉!故曰談何容易!」

於是吳王穆然,俛而深惟,仰而泣下交頤,曰:「嗟乎!余國之不亡也,綿綿連連,殆哉,世不絕也!」於是正明堂之朝,齊君之位,舉賢材,布德惠,施仁義,賞有功;躬節儉,減後宮之費,損車馬之用;放鄭聲,遠佞人,省庖廚,去侈靡;卑宮館,壞苑囿,填池塹,以予貧民無產業者;開內藏,振貧窮,存耆老,卹孤獨;薄賦斂,省刑辟。行此三年,海內晏然,天下大洽,陰陽和調,萬物咸得其宜;國無災害之變,民無飢寒之色,家給人足,畜積有餘,囹圄空虛;鳳凰來集,麒麟在郊,甘露既降,朱草萌牙;遠方異俗之人鄉風慕義,各奉其職而來朝賀。故治亂之道,存亡之端,若此易見,而君人者莫肯為也,臣愚竊以為過。故《詩》云:「王國克生,惟周之楨,濟濟多士,文王以寧。」此之謂也。

朔之文辭,此二篇最善。其餘有封泰山,責和氏璧及皇太子生禖,屏風,殿上柏柱,平樂觀賦獵,八言、七言上下,從公孫弘借車,凡向所錄朔書具是矣。世所傳他事皆非也。

贊曰:劉向言少時數問長老賢人通於事及朔時者,皆曰朔口諧倡辯,不能持論,喜為庸人誦說,故令後世多傳聞者。而楊雄亦以為朔言不純師,行不純德,其流風遺書蔑如也。然朔名過實者,以其詼達多端,不名一行,應諧似優,不窮似智,正諫似直,穢德似隱。非夷齊而是柳下惠,戒其子以上容:「首陽為拙,柱下為工;飽食安步,以仕易農;依隱玩世,詭時不逢」。其滑稽之雄乎!朔之詼諧,逢占射覆,其事浮淺,行於眾庶,童兒牧豎莫不眩燿。而後世好事者因取奇言怪語附著之朔,故詳錄焉。


\end{pinyinscope}