\article{敘傳}

\begin{pinyinscope}
班氏之先,與楚同姓,令尹子文之後也。子文初生,棄於瞢中,而虎乳之。楚人謂乳「穀」,謂虎「於檡」,故名穀於檡,字子文。楚人謂虎「班」,其子以為號。秦之滅楚,遷晉、代之間,因氏焉。

始皇之末,班壹避墬於樓煩,致馬牛羊數千群。值漢初定,與民無禁,當孝惠、高后時,以財雄邊,出入弋獵,旌旗鼓吹,年百餘歲,以壽終,故北方多以「壹」為字者。

壹生孺。孺為任俠,州郡歌之。孺生長,官至上谷守。長生回,以茂材為長子令。回生況,舉孝廉為郎,積功勞,至上河農都尉,大司農奏課連最,入為左曹越騎校尉。成帝之初,女為婕妤,致仕就第,貲累千金,徙昌陵。昌陵後罷,大臣名家皆占數于長安。

況生三子:伯、斿、稚。伯少受詩於師丹。大將軍王鳳薦伯宜勸學,召見宴昵殿,容貌甚麗,誦說有法,拜為中常侍。時上方鄉學,鄭寬中、張禹朝夕入說尚書、論語於金華殿中,詔伯受焉。既通大義,又講異同於許商,遷奉車都尉。數年,金華之業絕,出與王、許子弟為群,在於綺襦紈褲之間,非其好也。

家本北邊,志節忼慨,數求使匈奴。河平中,單于來朝,上使伯持節迎於塞下。會定襄大姓石、李群輩報怨,殺追捕吏,伯上狀,因自請願試守期月。上遣侍中中郎將王舜馳傳代伯護單于,并奉璽書印綬,即拜伯為定襄太守。定襄聞伯素貴,年少,自請治劇,畏其下車作威,吏民竦息。伯至,請問耆老父祖故人有舊恩者,迎延滿堂,日為供具,執子孫禮。郡中益弛。諸所賓禮皆名豪,懷恩醉酒,共諫伯宜頗攝錄盜賊,具言本謀亡匿處。伯曰:「是所望於父師矣。」乃召屬縣長吏,選精進掾史,分部收捕,及它隱伏,旬日盡得。郡中震栗,咸稱神明。歲餘,上徵伯。伯上書願過故郡上父祖冢。有詔,太守都尉以下會。因召宗族,各以親疏加恩施,散數百金。北州以為榮,長老紀焉。道病中風,既至,以侍中光祿大夫養病,賞賜甚厚,數年未能起。

會許皇后廢,班婕妤供養東宮,進侍者李平為婕妤,而趙飛燕為皇后,伯遂稱篤。久之,上出過臨候伯,伯惶恐,起視事。

自大將軍薨後,富平、定陵侯張放、淳于長等始愛幸,出為微行,行則同輿執轡;入侍禁中,設宴飲之會,及趙、李諸侍中皆引滿舉白,談笑大噱。時乘輿幄坐張畫屏風,畫紂醉踞妲己作長夜之樂。上以伯新起,數目禮之,因顧指畫而問伯:「紂為無道,至於是虖?」伯對曰:「《書》云『乃用婦人之言』,何有踞肆於朝?所謂眾惡歸之,不如是之甚者也。」上曰:「苟不若此,此圖何戒?」伯曰:「『沈湎于酒』,微子所以告去也;『式號式謼』,大雅所以流連也。詩書淫亂之戒,其原皆在於酒。」上乃謂然歎曰:「吾久不見班生,今日復聞讜言!」放等不懌,稍自引起更衣,因罷出。時長信庭林表適使來,聞見之。

後上朝東宮,太后泣曰:「帝間顏色瘦黑,班侍中本大將軍所舉,宜寵異之,益求其比,以輔聖德。宜遣富平侯且就國。」上曰:「諾。」車騎將軍王音聞之,以風丞相御史奏富平侯罪過,上乃出放為邊都尉。後復徵入,太后與上書曰:「前所道尚未效,富平侯反復來,其能默虖?」上謝曰:「請今奉詔。」是時許商為少府,師丹為光祿勳,上於是引商、丹入為光祿大夫,伯遷水衡都尉,與兩師並侍中,皆秩中二千石。每朝東宮,常從;及有大政,俱使諭指於公卿。上亦稍厭游宴,復修經書之業,太后甚悅。丞相方進復奏,富平侯竟就國。會伯病卒,年三十八,朝廷愍惜焉。

斿博學有俊材,左將軍師丹舉賢良方正,以對策為議郎,遷諫大夫、右曹中郎將,與劉向校祕書。每奏事,斿以選受詔進讀群書。上器其能,賜以祕書之副。時書不布,自東平思王以叔父求太史公、諸子書,大將軍白不許。語在東平王傳。斿亦早卒,有子曰嗣,顯名當世。

峺少為黃門郎中常侍,方直自守。成帝季年,立定陶王為太子,數遣中盾請問近臣,峺獨不敢答。哀帝即位,出峺為西河屬國都尉,遷廣平相。

王莽少與峺兄弟同列友善,兄事斿而弟畜峺。斿之卒也,修緦麻,賻賵甚厚。平帝即位,太后臨朝,莽秉政,方欲文致太平,使使者分行風俗,采頌聲,而峺無所上。琅邪太守公孫閎言災害於公府,大司空甄豐遣屬馳至兩郡諷吏民,而劾閎空造不祥,峺絕嘉應,嫉害聖政,皆不道。太后曰:「不宣德美,宜與言災害者異罰。且後宮賢家,我所哀也。」閎獨下獄誅。峺懼,上書陳恩謝罪,願歸相印,入補延陵園郎,太后許焉。食故祿終身。由是班氏不顯莽朝,亦不罹咎。

初,成帝性寬,進入直言,是以王音、翟方進等繩法舉過,而劉向、杜鄴、王章、朱雲之徒肆意犯上,故自帝師安昌侯,諸舅大將軍兄弟及公卿大夫、後宮外屬史許之家有貴寵者,莫不被文傷詆。唯谷永嘗言「建始、河平之際,許、班之貴,傾動前朝,熏灼四方,賞賜無量,空虛內臧,女寵至極,不可尚矣;今之後起,天所不饗,什倍於前。」永指以駮譏趙、李,亦無間云。

峺生彪。彪字叔皮,幼與從兄嗣共遊學,家有賜書,內足於財,好古之士自遠方至,父黨揚子雲以下莫不造門。

嗣雖修儒學,然貴老嚴之術。桓生欲借其書,嗣報曰:「若夫嚴子者,絕聖棄智,修生保真,清虛澹泊,歸之自然,獨師友造化,而不為世俗所役者也。漁釣於一壑,則萬物不奸其志;栖遲於一丘,則天下不易其樂。不絓聖人之罔,不泪驕君之餌,蕩然肆志,談者不得而名焉,故可貴也。今吾子已貫仁誼之羈絆,繫名聲之韁鎖,伏周、孔之軌躅,馳顏、閔之極摯,既繫攣於世教矣,何用大道為自眩曜?昔有學步於邯鄲者。曾未得其髣彿,又復失其故步,遂匍匐而歸耳!恐似此類,故不進。」嗣之行己持論如此。

叔皮唯聖人之道然後盡心焉。年二十,遭王莽敗,世祖即位於冀州。時隗囂據壟擁眾,招輯英俊,而公孫述稱帝於蜀漢,天下雲擾,大者連州郡,小者據縣邑。囂問彪曰:「往者周亡,戰國並爭,天下分裂,數世然後乃定,其抑者從橫之事復起於今乎?將承運迭興在於一人也?願先生論之。」對曰:「

周之廢興與漢異。昔周立爵五等,諸侯從政,本根既微,枝葉強大,故其末流有從橫之事,其勢然也。漢家承秦之制,並立郡縣,主有專己之威,臣無百年之柄,至於成帝,假借外家,哀、平短祚,國嗣三絕,危自上起,傷不及下。故王氏之貴,傾擅朝廷,能竊號位,而不根於民。是以即真之後,天下莫不引領而歎,十餘年間,外內騷擾,遠近俱發,假號雲合,咸稱劉氏,不謀而同辭。方今雄桀帶州城者,皆無七國世業之資。《詩》云:『皇矣上帝,臨下有赫,鑒觀四方,求民之莫。』今民皆謳吟思漢,鄉仰劉氏,已可知矣。」囂曰:「先生言周、漢之勢,可也,至於但見愚民習識劉氏姓號之故,而謂漢家復興,疏矣!昔秦失其鹿,劉季逐而掎之,時民復知漢虖!」既感囂言,又愍狂狡之不息,乃著王命論以救時難。其辭曰:

昔在帝堯之禪曰:「咨爾舜,天之曆數在爾躬。」舜亦以命禹。臮于稷鲭,咸佐唐虞,光濟四海,奕世載德,至于湯武,而有天下。雖其遭遇異時,禪代不同,至于應天順民,其揆一也。是故劉氏承堯之祚,氏族之世,著乎春秋。唐據火德,而漢紹之,始起沛澤,則神母夜號,以章赤帝之符。由是言之,帝王之祚,必有明聖顯懿之德,豐功厚利積絫之業,然後精誠通於神明,流澤加於生民,故能為鬼神所福饗,天下所歸往,未見運世無本,功德不紀,而得屈起在此位者也。世俗見高祖興於布衣,不達其故,以為適遭暴亂,得奮其劍,游說之士至比天下於逐鹿,幸捷而得之,不知神器有命,不可以智力求也。悲夫!此世所以多亂臣賊子者也。若然者,豈徒闇於天道哉?又不睹之於人事矣!

夫餓饉流隸,飢寒道路,思有裋褐之褻,儋石之畜,所願不過一金,然終於轉死溝壑。何則?貧窮亦有命也。況虖天子之貴,四海之富,神明之祚,可得而妄處哉?故雖遭罹阨會,竊其權柄,勇如信、布,彊如梁、籍,成如王莽,然卒潤鑊伏質,亨醢分裂,又況幺鲸,尚不及數子,而欲闇奸天位者虖!是故駑蹇之乘不騁千里之塗,燕雀之疇不奮六翮之用,楶梲之材不荷棟梁之任,斗筲之子不秉帝王之重。《易》曰「鼎折足,覆公餗」,不勝其任也。

當秦之末,豪桀共推陳嬰而王之,嬰母止之曰:「自吾為子家婦,而世貧賤,卒富貴不祥,不如以兵屬人,事成少受其利,不成禍有所歸。」嬰從其言,而陳氏以寧。王陵之母亦見項氏之必亡,而劉氏之將興也。是時陵為漢將,而母獲於楚,有漢使來,陵母見之,謂曰:「願告吾子,漢王長者,必得天下,子謹事之,無有二心。」遂對漢使伏劍而死,以固勉陵。其後果定於漢,陵為宰相封侯。夫以匹婦之明,猶能推事理之致,探禍福之機,而全宗祀於無窮,垂策書於春秋,而況大丈夫之事虖!是故窮達有命,吉凶由人,嬰母知廢,陵母知興,審此四者,帝王之分決矣。

蓋在高祖,其興也有五:一曰帝堯之苗裔,二曰體貌多奇異,三曰神武有徵應,四曰寬明而仁恕,五曰知人善任使。加之以信誠好謀,達於聽受,見善如不及,用人如由己,從諫如順流,趣時如嚮赴;當食吐哺,納子房之策;拔足揮洗,揖酈生之說;寤戍卒之言,斷懷土之情;高四皓之名,割肌膚之愛;舉韓信於行陳,收陳平於亡命,英雄陳力,群策畢舉:此高祖之大略,所以成帝業也。若乃靈瑞符應,又可略聞矣。初劉媼任高祖而夢與神遇,震電晦冥,有龍蛇之怪。及其長而多靈,有異於眾,是以王、武感物而折券,呂公睹形而進女;秦皇東游以厭其氣,呂后望雲而知所處;始受命則白蛇分,西入關則五星聚。故淮陰、留侯謂之天授,非人力也。

歷古今之得失,驗行事之成敗,稽帝王之世運,考五者之所謂,取舍不厭斯位,符瑞不同斯度,而苟昧於權利,越次妄據,外不量力,內不知命,則必喪保家之主,失天年之壽,遇折足之凶,伏鈇鉞之誅。英雄誠知覺寤,畏若禍戒,超然遠覽,淵然深識,收陵、嬰之明分,絕信、布之覬覦,距逐鹿之瞽說,審神器之有授,毋貪不可幾,為二母之所沧,則福祚流于子孫,天祿其永終矣。

知隗囂終不寤,乃避墬於河西。河西大將軍竇融嘉其美德,訪問焉。舉茂材,為徐令,以病去官。後數應三公之召。仕不為祿,所如不合;學不為人,博而不俗;言不為華,述而不作。

有子曰固,弱冠而孤,作幽通之賦,以致命遂志。其辭曰:

系高頊之玄冑兮,氏中葉之炳靈,繇凱風而蟬蛻兮,雄朔野以颺聲。皇十紀而鴻漸兮,有羽儀於上京。巨滔天而泯夏兮,考遘愍以行謠,終保己而貽則兮,里上仁之所廬。懿前烈之純淑兮,窮與達其必濟,咨孤矇之眇眇兮,將圮絕而罔階,豈余身之足殉兮?骇世業之可懷。

靖潛處以永思兮,經日月而彌遠,匪黨人之敢拾兮,庶斯言之不玷。魂煢煢與神交兮,精誠發於宵寐,夢登山而迥眺兮,覿幽人之髣彿,髓葛藟而授余兮,眷峻谷曰勿隧。昒昕寤而仰思兮,心蒙蒙猶未察,黃神邈而靡質兮,儀遺讖以臆對。曰乘高而骗神兮,道遐通而不迷,葛綿綿於樛木兮,詠南風以為綏,蓋惴惴之臨深兮,乃二雅之所祗。既誶爾以吉象兮,又申之以炯戒:盍孟晉以迨群兮?辰焂忽其不再。

承靈訓其虛徐兮,嵜盤桓而且俟,惟天墬之無窮兮,嫤生民之脢在。紛屯亶與蹇連兮,何艱多而智寡!上聖寤而後拔兮,豈群黎之所御!昔衛叔之御昆兮,昆為寇而喪予。管彎弧欲斃讎兮,讎作后而成己。變化故而相詭兮,孰云豫其終始!雍造怨而先賞兮,丁繇惠而被戮;莺取弔于逌吉兮,王膺慶於所慼。畔回缈其若茲兮,北叟頗識其倚伏。單治裏而外凋兮,張修襮而內逼,欥中龢為庶幾兮,顏與冉又不得。溺招路以從己兮,謂孔氏猶未可,安慆慆而不萉兮,卒隕身虖世禍。游聖門而靡救兮,顧覆醢其何補?固行行其必凶兮,免盜亂為賴道;形氣發于根柢兮,柯葉彙而靈茂。恐网锹之責景兮,慶未得其云已。

黎淳耀于高辛兮,羋彊大於南汜;嬴取威於百儀兮,姜本支虖三止:既仁得其信然兮,卬天路而同軌。東馒虐而殲仁兮,王合位虖三五;戎女烈而喪孝兮,伯徂歸於龍虎:發還師以成性兮,重醉行而自耦。震鱗漦于夏庭兮,匝三正而滅周;巽羽化于宣宮兮,彌五辟而成災。

道悠長而世短兮,夐冥默而不周,胥仍物而鬼諏兮,乃窮宙而達幽。媯巢姜於孺筮兮,旦算祀于挈龜。宣、曹興敗於下夢兮,魯、衛名諡於銘謠。妣聆呱而刻石兮,許相理而鞠條。道混成而自然兮,術同原而分流。神先心以定命兮,命隨行以消息。斡流遷其不濟兮,故遭罹而贏縮。三欒同於一體兮,雖移盈然不忒。洞參差其紛錯兮,斯眾兆之所惑。周、賈盪而貢憤兮,齊死生與禍福,抗爽言以矯情兮,信畏犧而忌服。

所貴聖人之至論兮,順天性而斷誼。物有欲而不居兮,亦有惡而不避,守孔約而不貳兮,乃輶德而無累。三仁殊而一致兮,夷、惠舛而齊聲。木偃息以蕃魏兮,申重繭以存荊。紀焚躬以衛上兮,皓頤志而弗營。侯屮木之區別兮,苟能實而必榮。要沒世而不朽兮,乃先民之所程。

觀天罔之紘覆兮,實棐諶而相順,謨先聖之大繇兮,亦馒德而助信。虞韶美而儀鳳兮,孔忘味於千載。素文信而底麟兮,漢賓祚于異代。精通靈而感物兮,神動氣而入微。養游睇而猿號兮,李虎發而石開。非精誠其焉通兮,苟無實其孰信!操末技猶必然兮,矧湛躬於道真!

登孔、顥而上下兮,緯群龍之所經,朝貞觀而夕化兮,猶諠己而遺形,若胤彭而偕老兮,訴來哲以通情。

亂曰:天造屮昧,立性命兮,復心弘道,惟賢聖兮。渾元運物,流不處兮,

保身遺名,民之表兮。舍生取誼,亦道用兮,憂傷夭物,忝莫痛兮!昊爾太素,

曷渝色兮?尚粵其幾,淪神域兮!

永平中為郎,典校祕書,專篤志於博學,以著述為業。或譏以無功,又感東方朔、揚雄自諭以不遭蘇、張、范、蔡之時,曾不折之以正道,明君子之所守,故聊復應焉。其辭曰:

賓戲主人曰:「蓋聞聖人有壹定之論,列士有不易之分,亦云名而已矣。故太上有立德,其次有立功。夫德不得後身而特盛,功不得背時而獨章,是以聖咝之治,棲棲皇皇,孔席不飕,墨突不黔。由此言之,取舍者昔人之上務,著作者前列之餘事耳。今吾子幸游帝王之世,躬帶冕之服,浮英華,湛道德,矕龍虎之文,舊矣。卒不能攄首尾,奮翼鱗,振拔洿塗,跨騰風雲,使見之者景駭,聞之者嚮震。徒樂枕經籍書,紆體衡門,上無所蔕,下無所根。獨攄意虖宇宙之外,銳思於豪芒之內,潛神默記,恆以年歲。然而器不賈於當己,用不效於一世,雖馳辯如濤波,摛藻如春華,猶無益於殿最。意者,且運朝夕之策,定合會之計,使存有顯號,亡有美諡,不亦優虖?」

主人逌爾而沧曰:「若賓之言,斯所謂見勢利之華,闇道德之實,守穾奧之熒燭,未卬天庭而睹白日也。曩者王塗蕪穢,周失其御,侯伯方軌,戰國橫騖,於是七雄虓闞,分裂諸夏,龍戰而虎爭。游說之徒,風颺電激,並起而救之,其餘猋飛景附,煜霅其間者,蓋不可勝載。當此之時,搦朽摩鈍,鈆刀皆能壹斷,是故魯連飛一矢而蹶千金,虞卿以顧眄而捐相印也。夫啾發投曲,感耳之聲,合之律度,淫麴而不可聽者,非韶、夏之樂也;因勢合變,偶時之會,風移俗易,乖忤而不可通者,非君子之法也。及至從人合之,衡人散之,亡命漂說,羇旅騁辭,商鞅挾三術以鑽孝公,李斯奮時務而要始皇,彼皆躡風雲之會,履顛沛之勢,據徼乘邪以求一日之富貴,朝為榮華,夕而焦瘁,福不盈眥,禍益於世,凶人且以自悔,況吉士而是賴虖!且功不可以虛成,名不可以偽立,韓設辯以徼君,呂行詐以賈國。說難既酋,其身乃囚;秦貨既貴,厥宗亦隧。是故仲尼抗浮雲之志,孟軻養浩然之氣,彼豈樂為迂闊哉?道不可以貳也。方今大漢洒埽群穢,夷險芟荒,廓帝紘,恢皇綱,基隆於羲、農,規廣於黃、唐;其君天下也,炎之如日,威之如神,函之如海,養之如春。是以六合之內,莫不同原共流,沐浴玄德,稟卬太和,枝附葉著,譬猶屮木之殖山林,鳥魚之毓川澤,得氣者蕃滋,失時者苓落,參天墬而施化,豈云人事之厚薄哉?今子處皇世而論戰國,耀所聞而疑所覿,欲從旄敦而度高虖泰山,懷氿濫而測深虖重淵,亦未至也。」

賓曰:「若夫鞅、斯之倫,衰周之凶人,既聞命矣。敢問上古之士,處身行道,輔世成名,可述於後者,默而已虖?」

主人曰:「何為其然也!昔咎繇謨虞,箕子訪周,言通帝王,謀合聖神;殷說夢發於傅巖,周望兆動於渭濱,齊甯激聲於康衢,漢良受書於邳沂,皆俟命而神交,匪詞言之所信,故能建必然之策,展無窮之勳也。近者陸子優繇,新語以興;董生下帷,發藻儒林;劉向司籍,辯章舊聞;揚雄覃思,法言、大玄:皆及時君之門闈,究先聖之壼奧,婆娑虖術藝之場,休息虖篇籍之囿,以全其質而發其文,用納虖聖聽,列炳於後人,斯非其亞與!若乃夷抗行於首陽,惠降志於辱仕,顏耽樂於簞瓢,孔終篇於西狩,聲盈塞於天淵,真吾徒之師表也。且吾聞之:壹陰壹陽,天墬之方;乃文乃質,王道之綱;有同有異,聖咝之常。故曰:慎修所志,守爾天符,委命共己,味道之腴,神之聽之,名其舍諸!賓又不聞龢氏之璧韞於荊石,隨侯之珠藏於馋蛤虖?歷世莫严,不知其將含景耀,吐英精,曠千載而流夜光也。應龍潛於潢汙,魚黿媟之,不睹其能奮靈德,合風雲,超忽荒,而躆顥蒼也。故夫泥蟠而天飛者,應龍之神也;先賤而後貴者,龢、隨之珍也;時闇而久章者,君子之真也。若乃牙、曠清耳於管絃,離婁眇目於豪分;逢蒙絕技於弧矢,班輸榷巧於斧斤;良樂軼能於相馭,烏獲抗力於千鈞;龢、鵲發精於鍼石,研、桑心計於無垠。僕亦不任廁技於彼列,故密爾自娛於斯文。」

固以為唐虞三代,詩書所及,世有典籍,故雖堯舜之盛,必有典謨之篇,然後揚名於後世,冠德於百王,故曰「巍巍乎其有成功,煥乎其有文章也!」漢紹堯運,以建帝業,至於六世,史臣乃追述功德,私作本紀,編於百王之末,廁於秦、項之列。太初以後,闕而不錄,故探篹前記,綴輯所聞,以述漢書,起元高祖,終于孝平王莽之誅,十有二世,二百三十年,綜其行事,旁貫五經,上下洽通,為春秋考紀、表、志、傳,凡百篇。其敘曰:

皇矣漢祖,纂堯之緒,實天生德,聰明神武。秦人不綱,罔漏于楚,爰茲發跡,斷蛇奮旅。神母告符,朱旗乃舉,粵蹈秦郊,嬰來稽首。革命創制,三章是紀,應天順民,五星同晷。項氏畔換,黜我巴、漢,西土宅心,戰士憤怨。乘釁而運,席卷三秦,割據河山,保此懷民。股肱蕭、曹,社稷是經,爪牙信、布,腹心良、平,龔行天罰,赫赫明明。述高紀第一。

孝惠短世,高后稱制,罔顧天顯,呂宗以敗。述惠紀第二,高后紀第三。

太宗穆穆,允恭玄默,化民以躬,帥下以德。農不供貢,罪不收孥,宮不新館,陵不崇墓。我德如風,民應如屮,國富刑清,登我漢道。述文紀第四。

孝景蒞政,諸侯方命,克伐七國,王室以定。匪怠匪荒,務在農桑,著于甲令,民用寧康。述景紀第五。

世宗曄曄,思弘祖業,疇咨熙載,髦俊並作。厥作伊何?百蠻是攘,恢我疆宇,外博四荒。武功既抗,亦迪斯文,憲章六學,統壹聖真。封禪郊祀,登秩百神;協律改正,饗茲永年。述武紀第六。

孝昭幼沖,冢宰惟忠。燕、蓋譸張,實叡實聰,罪人斯得,邦家和同。述昭紀第七。

中宗明明,夤用刑名,時舉傅納,聽斷惟精。柔遠能邇,燀燿威靈,龍荒幕朔,莫不來庭。丕顯祖烈,尚於有成。述宣紀第八。

孝元翼翼,高明柔克,賓禮故老,優繇亮直。外割禁囿,內損御服,離宮不衛,山陵不邑。閹尹之髁,穢我明德。述元紀第九。

孝成煌煌,臨朝有光,威儀之盛,如圭如璋。壼闈恣趙,朝政在王,炎炎燎火,亦允不陽。述成紀第十。

孝哀彬彬,克髓威神,彫落洪支,底剭鼎臣。婉孌董公,惟亮天功,大過之困,實橈實凶。述哀紀第十一。

孝平不造,新都作宰,不周不伊,喪我四海。述平紀第十二。

漢初受命,諸侯並政,制自項氏,十有八姓。述異姓諸侯王表第一。

太祖元勳,啟立輔臣,支庶藩屏,侯王並尊。述諸侯王表第二。

侯王之祉,祚及宗子,公族蕃滋,支葉碩茂。述王子侯表第三。

受命之初,贊功剖符,奕世弘業,爵土乃昭。述高惠高后孝文功臣侯表第四。

景征吳楚,武興師旅,後昆承平,亦有紹土。述景武昭宣元成哀功臣侯表第五。

亡德不報,爰存二代,宰相外戚,昭韙見戒。述外戚恩澤侯表第六。

漢迪於秦,有革有因,觕舉僚職,並列其人。述百官公卿表第七。

篇章博舉,通于上下,略差名號,九品之敘。述古今人表第八。

元元本本,數始於一,產氣黃鍾,造計秒忽。八音七始,五聲六律,度量權衡,曆算逌出。官失學微,六家分乖,壹彼壹此,庶研其幾。述律曆志第一。

上天下澤,春雷奮作,先王觀象,爰制禮樂。厥後崩壞,鄭衛荒淫,風流民化,湎湎紛紛。略存大綱,以統舊文。述禮樂志第二。

剨電皆至,天威震耀,五刑之作,是則是效,威實輔德,刑亦助教。季世不詳,背本爭末,吳、孫狙詐,申、商酷烈。漢章九法,太宗改作,輕重之差,世有定籍。述刑法志第三。

厥初生民,食貨惟先。割制廬井,定爾土田,什一供貢,下富上尊。商以足用,茂遷有無,貨自龜貝,至此五銖。揚搉古今,監世盈虛。述食貨志第四。

昔在上聖,昭事百神,類帝禋宗,望秩山川,明德惟馨,永世豐年。季末淫祀,營信巫史,大夫臚岱,侯伯僭畤,放誕之徒,緣間而起。瞻前顧後,正其終始。述郊祀志第五。

炫炫上天,縣象著明,日月周輝,星辰垂精。百官立法,宮室混成,降應王政,景以燭形。三季之後,厥事放紛,舉其占應,覽故考新。述天文志第六。

河圖命庖,洛書賜禹,八卦成列,九疇逌敘。世代寔寶,光演文武,春秋之占,咎徵是舉。告往知來,王事之表。述五行志第七。

坤作墬勢,高下九則,自昔黃、唐,經略萬國,變定東西,疆理南北。三代損益,降及秦、漢,革凛五等,制立郡縣。略表山川,彰其剖判。述地理志第八。

夏乘四載,百川是導。唯河為谡,災及後代。商竭周移,秦決南涯,自茲韵漢,北亡八支。文讯棗野,武作瓠歌,成有平年,後遂滂沱。爰及溝渠,利我國家。述溝洫志第九。

虙羲畫卦,書契後作,虞夏商周,孔纂其業,篹書刪詩,綴禮正樂,彖系大易,因史立法。六學既登,遭世罔弘,群言紛亂,諸子相騰。秦人是滅,漢修其缺,劉向司籍,九流以別。爰著目錄,略序洪烈。述藝文志第十。

上嫚下暴,惟盜是伐,勝、廣熛起,梁、籍扇烈。赫赫炎炎,遂焚咸陽,宰割諸夏,命立侯王,誅嬰放懷,詐虐以亡。述陳勝項籍傳第一。

張、陳之交,斿如父子,攜手韬秦,拊翼俱起。據國爭權,還為豺虎,耳諫甘公,作漢藩輔。述張耳陳餘傳第二。

三辞之起,本根既朽,枯楊生華,曷惟其舊!橫雖雄材,伏于海闯,沐浴尸鄉,北面奉首,旅人慕殉,義過黃鳥。述魏豹田儋韓信傳第三。

信惟餓隸,布實黥徒,越亦狗盜,芮尹江湖。雲起龍襄,化為侯王,割有齊、楚,跨制淮、梁。綰自同閈,鎮我北疆,德薄位尊,非胙惟殃。吳克忠信,胤嗣乃長。述韓彭英盧吳傳第四。

賈廑從旅,為鎮淮、楚。澤王琅邪,權激諸呂。濞之受吳,疆土踰矩,雖戒東南,終用齊斧。述荊燕吳傳第五。

太上四子:伯兮早夭,仲氏王代,斿宅于楚。戊實淫雾,平陸乃紹。其在于京,奕世宗正,劬勞王室,用侯陽成。子政博學,三世成名,述楚元王傳第六。

季氏之詘,辱身毀節,信于上將,議臣震栗。欒公哭梁,田叔殉趙,見危授命,誼動明主。布歷燕、齊,叔亦相魯,民思其政,或金或社。述季布欒布田叔傳第七。

高祖八子,二帝六王。三趙不辜,淮厲自亡,燕靈絕嗣,齊悼特昌。掩有東土,自岱徂海,支庶分王,前後九子。六國誅斃,適齊亡祀。城陽、濟北,後承我國。赳赳景王,匡漢社稷。述高五王傳第八。

猗與元勳,包漢舉信,鎮守關中,足食成軍,營都立宮,定制修文。平陽玄默,繼而弗革,民用作歌,化我淳德。漢之宗臣,是謂相國。述蕭何曹參傳第九。

留侯襲秦,作漢腹心,圖折武關,解阨鴻門。推齊銷印,敺致越、信;招賓四老,惟寧嗣君。陳公擾攘,歸漢乃安,斃范亡項,走狄擒韓,六奇既設,我罔艱難。安國廷爭,致仕杜門。絳侯矯矯,誅呂尊文。亞夫守節,吳楚有勳。述張陳王周傳第十。

舞陽鼓刀,滕公廄騶,潁陰商販,曲周庸夫,攀龍附鳳,並乘天衢。述樊酈滕灌傅靳周傳第十一。

北平志古,司秦柱下,定漢章程,律度之緒。建平質直,犯上干色;廣阿之廑,食厥舊德。故安執節,責通請錯,蹇蹇帝臣,匪躬之故。述張周趙任申屠傳第十二。

食其監門,長揖漢王,畫襲陳留,進收敖倉,塞隘杜津,王基以張。賈作行人,百越來賓,從容風議,博我以文。敬繇役夫,遷京定都,內強關中,外和匈奴。叔孫奉常,與時抑揚,稅介免冑,禮義是創。或悊或謀,觀國之光。述酈陸朱婁叔孫傳第十三。

淮南僭狂,二子受殃。安辯而邪,賜頑以荒,敢行稱亂,窘世薦亡。述淮南衡山濟北傳第十四。

蒯通壹說,三雄是敗,覆酈驕韓,田橫顛沛。被之拘係,乃成患害。充、躬罔極,交亂弘大。述蒯伍江息夫傳第十五。

萬石溫溫,幼寤聖君,宜爾子孫,夭夭伸伸,慶社于齊,不言動民。衛、直、周、張,淑慎其身。述萬石衛直周張傳第十六。

孝文三王,代孝二梁,懷折亡嗣,孝乃尊光。內為母弟,外扞吳楚,怙寵矜功,僭欲失所,思心既霿,牛禍告妖。帝庸親親,厥國五分,德不堪寵,四支不傳。述文三王傳第十七。

賈生矯矯,弱冠登朝。遭文叡聖,屢抗其疏,暴秦之戒,三代是據。建設藩屏,以強守圉,吳楚合從,賴誼之慮。述賈誼傳第十八。

子絲慷慨,激辭納說,髓轡正席,顯陳成敗。錯之瑣材,智小謀大,禍如發機,先寇受害。述爰盎朝錯傳第十九。

釋之典刑,國憲以平。馮公矯魏,增主之明。長孺剛直,義形於色,下折淮南,上正元服。莊之推賢,於茲為德。述張馮汲鄭傳第二十。

榮如辱如,有機有樞,自下摩上,惟德之隅。賴依忠正,君子采諸。述賈鄒枚路傳第二十一。

魏其翩翩,好節慕聲,灌夫矜勇,武安驕盈,凶德相挻,禍敗用成。安國壯趾,王恢兵首,彼若天命,此近人咎。述竇田灌韓傳第二十二。

景十三王,承文之慶。魯恭館室,江都訬輕;趙敬險詖,中山淫醟;長沙寂漠,廣川亡聲;膠東不亮,常山驕盈。四國絕祀,河間賢明,禮樂是修,為漢宗英。述景十三王傳第二十三。

李廣恂恂,實獲士心,控弦貫石,威動北鄰,躬戰七十,遂死于軍。敢怨衛青,見討去病。陵不引決,忝世滅姓。蘇武信節,不詘王命。述李廣蘇建傳第二十四。

長平桓桓,上將之元,薄伐獫允,恢我朔邊,戎車七征,衝輣閑閑,合圍單于,北登闐顏。票騎冠軍,猋勇紛紜,長驅六舉,電擊雷震,飲馬翰海,封狼居山,西規大河,列郡祈連。述衛青霍去病傳第二十五。

抑抑仲舒,再相諸侯,身修國治,致仕縣車,下帷覃思,論道屬書,讜言訪對,為世純儒。述董仲舒傳第二十六。

文豔用寡,子虛烏有,寓言淫麗,託風終始,多識博物,有可觀采,蔚為辭宗,賦頌之首。述司馬相如傳第二十七。

平津斤斤,晚躋金門,既登爵位,祿賜頤賢,布衾疏食,用儉飭身。卜式耕牧,以求其志,忠寤明君,乃爵乃試。兒生亹亹,束髮修學,偕列名臣,從政輔治。述公孫弘卜式兒寬傳第二十八。

張湯遂達,用事任職,媚茲一人,日旰忘食,既成寵祿,亦羅咎慝。安世溫良,塞淵其德,子孫遵業,全祚保國。述張湯傳第二十九。

杜周治文,唯上淺深,用取世資,幸而免身。延年寬和,列于名臣。欽用材謀,有異厥倫。述杜周傳第三十。

博望杖節,收功大夏;貳師秉鉞,身釁胡社。致死為福,每生作禍。述張騫李廣利傳第三十一。

烏呼史遷,薰胥以刑!幽而發憤,乃思乃精,錯綜群言,古今是經,勒成一家,大略孔明。述司馬遷傳第三十二。

孝武六子,昭、齊亡嗣。燕剌謀逆,廣陵祝詛。昌邑短命,昏賀失據。戾園不幸,宣承天序。述武五子傳第三十三。

六世耽耽,其欲浟浟,文武方作,是庸四克。助、偃、淮南,數子之德,不忠其身,善謀於國。述嚴朱吾丘主父徐嚴終王賈傳第三十四。

東方贍辭,詼諧倡優,譏苑扞偃,正諫舉郵,懷肉汙殿,弛張沈浮。述東方朔傳第三十五。

葛繹內寵,屈氂王子。千秋時發,宜春舊仕。敞、義依霍,庶幾云已。弘惟政事,萬年容己。咸睡厥誨,孰為不子?述公孫劉田楊王蔡陳鄭傳第三十六。

王孫臝葬,建乃斬將。雲廷訐禹,福逾刺鳳,是謂狂狷,敞近其衷。述楊胡朱梅云傳第三十七。

博陸堂堂,受遺武皇,擁毓孝昭,末命導揚。遭家不造,立帝廢王,權定社稷,配忠阿衡。懷祿耽寵,漸化不詳,陰妻之逆,至子而亡。秺侯狄孥,虔恭忠信,奕世載德,貤于子孫。述霍光金日磾傳第三十八。

兵家之策,惟在不戰。營平皤皤,立功立論,以不濟可,上諭其信。武賢父子,虎臣之俊。述趙充國辛慶忌傳第三十九。

義陽樓蘭,長羅昆彌,安遠日逐,義成郅支。陳湯誕節,救在三悊;會宗勤事,疆外之桀。述傅常鄭甘陳段傳第四十。

不疑膚敏,應變當理,辭霍不婚,逡遁致仕。疏克有終,散金娛老。定國之祚,于其仁考。廣德、當、宣,近於知恥。述雋疏于薛平彭傳第四十一。

四皓遯秦,古之逸民,不營不拔,嚴平、鄭真。吉困于賀,涅而不緇;禹既黃髮,以德來仕。舍惟正身,勝死善道;郭欽、蔣詡,近遯之好。述王貢兩龔鮑傳第四十二。

扶陽濟濟,聞詩聞禮。玄成退讓,仍世作相。漢之宗廟,叔孫是謨,革自孝元,諸儒變度。國之誕章,博載其路。述韋賢傳第四十三。

高平師師,惟辟作威,圖黜凶害,天子是毗。博陽不伐,含弘光大,天誘其衷,慶流苗裔。述魏相丙吉傳第四十四。

占往知來,幽贊神明,苟非其人,道不虛行。學微術昧,或見仿佛,疑殆匪闕,違眾迕世,淺為尤悔,深作敦害。述眭兩夏侯京翼李傳第四十五。

廣漢尹京,克聰克明;延壽作翊,既和且平。矜能訐上,俱陷極刑。翁歸承風,帝揚厥聲。敞;亦平平,文雅自贊;尊實赳赳,邦家之彥;章死非罪,士民所歎。述趙尹韓張兩王傳第四十六。

寬饒正色,國之司直。豐繄好剛,輔亦慕直。皆陷狂狷,不典不式。崇執言責,隆持官守。寶曲定陵,並有立志。述蓋諸葛劉鄭毋將孫何傳第四十七。

長倩陇陇,覿霍不舉,遇宣乃拔,傅元作輔,不圖不慮,見躓石、許。述蕭望之傳第四十八。

子明光光,發跡西疆,列於禦侮,厥子亦良。述馮奉世傳第四十九。

宣之四子,淮陽聰敏,舅氏蘧蒢,幾陷大理。楚孝惡疾,東平失軌,中山凶短,母歸戎里。元之二王,孫後大宗,昭而不穆,大命更登。述宣元六王傳第五十。

樂安铖铖,古之文學,民具爾瞻,困于二司。安昌貨殖,朱雲作娸。博山惇慎,受莽之疚。述匡張孔馬傳第五十一。

樂昌篤實,不橈不詘,遘閔既多,是用廢黜。武陽殷勤,輔導副君,既忠且謀,饗茲舊勳。高武守正,因用濟身。述王商、史丹、傅喜傳第五十二。

高陽文法,揚鄉武略,政事之材,道德惟薄,位過厥任,鮮終其祿。博之翰音,鼓妖先作。述薛宣朱博傳第五十三。

高陵修儒,任刑養威,用合時宜,器周世資。義得其勇,如虎如貔,進不跬步,宗為鯨鯢。述翟方進傳第五十四。

統微政缺,災眚屢發。永陳厥咎,戒在三七。鄴指丁、傅,略窺占術。述谷永杜鄴傳第五十五。

哀、平之卹,丁、傅、莽、賢。武、嘉戚之,乃喪厥身。高樂廢黜,咸列貞臣。述何武王嘉師丹傳第五十六。

淵哉若人!實好斯文。初擬相如,獻賦黃門,輟而覃思,草法篹玄,斟酌六經,放易象論,潛于篇籍,以章厥身。述揚雄傳第五十七。

獷獷亡秦,滅我聖文,漢存其業,六學析分。是綜是理,是綱是紀,師徒彌散,著其終始。述儒林傳第五十八。

誰毀誰譽,譽其有試。泯泯群黎,化成良吏。淑人君子,時同功異。沒世遺愛,民有餘思。述循吏傳第五十九。

上替下陵,姦軌不勝,猛政橫作,刑罰用興。曾是強圉,掊克為雄,報虐以威,殃亦凶終。述酷吏傳第六十。

四民食力,罔有兼業,大不淫侈,細不匱乏,蓋均無貧,遵王之法。靡法靡度,民肆其詐,偪上并下,荒殖其貨。侯服玉食,敗俗傷化。述貨殖傳第六十一。

開國承家,有法有制,家不臧甲,國不專殺。矧乃齊民,作威作惠,如台不匡,禮法是謂!述游俠傳第六十二。

彼何人斯,竊此富貴!營損高明,作戒後世。述佞幸傳第六十三。

於惟帝典,戎夷猾夏;周宣攘之,亦列風雅。宗幽既昏,淫於褒女,戎敗我驪,遂亡酆鄗。大漢初定,匈奴強盛,圍我平城,寇侵邊境。至于孝武,爰赫斯怒,王師雷起,霆擊朔野。宣承其末,乃施洪德,震我威靈,五世來服。王莽竊命,是傾是覆,備其變理,為世典式。述匈奴傳第六十四。

西南外夷,種別域殊。南越尉佗,自王番禺,攸攸外寓,閩越、東甌。爰洎朝鮮,燕之外區。漢興柔遠,與爾剖符。皆恃其岨,乍臣乍驕,孝武行師,誅滅海隅。述西南夷兩越朝鮮傳第六十五。

西戎即序,夏后是表。周穆觀兵,荒服不旅。漢武勞神,圖遠甚勤。王師驒驒,致誅大宛。姼姼公主,乃女烏孫,使命乃通,條支之瀕。昭、宣承業,都護是立,總督城郭,三十有六,修奉朝貢,各以其職。述西城傳第六十六。

詭矣禍福,刑于外戚。高后首命,呂宗顛覆。薄姬蕴魏,宗文產德。竇后違意,考盤于代。王氏仄微,世武作嗣。子夫既興,扇而不終。鉤弋憂傷,孝昭以登。上官幼尊,類禡厥宗。史娣、王悼,身遇不祥,及宣饗國,二族後光。恭哀產元,夭而不遂。邛成乘序,履尊三世。飛燕之妖,禍成厥妹。丁、傅僭恣,自求凶害。中山無辜,乃喪馮、衛。惠張、景薄,武陳、宣霍,成許、哀傅,平王之作,事雖歆羨,非天所度。怨咎若茲,如何不恪!述外戚傳第六十七。

元后娠母,月精見表。遭成之逸,政自諸舅。陽平作威,誅加卿宰。成都煌煌,假我明光。曲陽攜攜,亦朱其堂。新都亢極,作亂以亡。述元后傳第六十八。

咨爾賊臣,篡漢滔天,行驕夏癸,虐烈商辛。偽稽黃、虞,繆稱典文,眾怨神怒,惡復誅臻。百王之極,究其姦昏。述王莽傳第六十九。

凡漢書,敘帝皇,列官司,建侯王。準天地,統陰陽,闡元極,步三光。分州域,物土疆,窮人理,該萬方。緯六經,綴道綱,總百氏,贊篇章。函雅故,通古今,正文字,惟學林。述敘傳第七十。


\end{pinyinscope}