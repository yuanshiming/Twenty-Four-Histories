\article{食貨志}

\begin{pinyinscope}
洪範八政,一曰食,二曰貨。食謂農殖嘉穀可食之物,貨謂布帛可衣,及金刀龜貝,所以分財布利通有無者也。二者,生民之本,興自神農之世。「斲木為耜,煣木為耒,耒鲈之利以教天下」,而食足;「日中為市,致天下之民,聚天下之貨,交易而退,各得其所」,而貨通。食足貨通,然後國實民富,而教化成。黃帝以下「通其變,使民不倦」。堯命四子以「敬授民時」,舜命后稷以「黎民祖飢」,是為政首。禹平洪水,定九州,制土田,各因所生遠近,賦入貢棐,楙遷有無,萬國作乂。殷周之盛,詩書所述,要在安民,富而教之。故易稱「天地之大德曰生,聖人之大寶曰位;何以守位曰仁,何以聚人曰財。」財者,帝王所以聚人守位,養成群生,奉順天德,治國安民之本也。故曰:「不患寡而患不均,不患貧而患不安;蓋均亡貧,和亡寡,安亡傾。」是以聖王域民,築城郭以居之,制廬井以均之,開市肆以通之,設庠序以教之;士農工商,四民有業。學以居位曰士,闢土殖穀曰農,作巧成器曰工,通財鬻貨曰商。聖王量能授事,四民陳力受職,故朝亡廢官,邑亡敖民,地亡曠土。

理民之道,地著為本。故必建步立畝,正其經界。六尺為步,步百為畝,畝百為夫,夫三為屋,屋三為井,井方一里,是為九夫。八家共之,各受私田百畝,公田十畝,是為八百八十畝,餘二十畝以為廬舍。出入相友,守望相助,疾病則救,民是以和睦,而教化齊同,力役生產可得而平也。

民受田,上田夫百畝,中田夫二百畝,下田夫三百畝。歲耕種者為不易上田;休一歲者為一易中田;休二歲者為再易下田,三歲更耕之,自爰其處。農民戶人己受田,其家眾男為餘夫,亦以口受田如比。士工商家受田,五口乃當農夫一人。此謂平土可以為法者也。若山林藪澤原陵淳鹵之地,各以肥磽多少為差。有賦有稅。稅謂公田什一及工商衡虞之入也。賦共車馬甲兵士徒之役,充實府庫賜予之用。稅給郊社宗廟百神之祀,天子奉養百官祿食庶事之費。民年二十受田,六十歸田。七十以上,上所養也;十歲以下,上所長也;十一以上,上所強也。種穀必雜五種,以備災害。田中不得有樹,用妨五穀。力耕數耘,收穫如寇盜之至。還廬樹桑,菜茹有畦,瓜瓠果蓏殖於疆易。雞豚狗彘毋失其時,女修蠶織,則五十可以衣帛,七十可以食肉。

在野曰廬,在邑曰里。五家為鄰,五鄰為里,四里為族,五族為黨,五黨為州,五州為鄉。鄉,萬二千五百戶也。鄰長位下士,自此以上,稍登一級,至鄉而為卿也。於里有序而鄉有庠。序以明教,庠則行禮而視化焉。春令民畢出在野,冬則畢入於邑。其詩曰:「四之日舉止,同我婦子,饁彼南具。」又曰:「十月蟋蟀,入我床下,嗟我婦子,聿為改歲,入此室處。」所以順陰陽,備寇賊,習禮文也。春,秋出民,里胥平旦坐於右塾,鄰長坐於右塾,畢出然後歸,夕亦如之。入者必持薪樵,輕重相分,班白不提挈。冬,民既入,婦人同巷,相從夜績,女工一月得四十五日。必相從者,所以省費燎火,同巧拙而合習俗也。男女有不得其所者,因相與歌詠,各言其傷。

是月,餘子亦在于序室。八歲入小學,學六甲五方書計之事,始知室家長幼之節。十五入大學,學先聖禮樂,而知朝廷君臣之禮。其有秀異者,移鄉學于庠序;庠序之異者,移國學于少學。諸侯歲貢少學之異者於天子,學于大學,命曰造士。行同能偶,則別之以射,然後爵命焉。

孟春之月,群居者將散,行人振木鐸徇于路,以采詩,獻之大師,比其音律,以聞於天子。故曰王者不窺牖戶而知天下。

此先王制土處民富而教之之大略也。故孔子曰:「道千乘之國,敬事而信,節用而愛人,使民以時。」故民皆勸功樂業,先公而後私。其詩曰:「有渰淒淒,興雲祁祁,雨我公田,遂及我私。」民三年耕,則餘一年之畜。衣食足而知榮辱,廉讓生而爭訟息,故三載考績。孔子曰「苟有用我者,期月而已可也,三年有成」,成此功也。三考黜陟,餘三年食,進業曰登;再登曰平,餘六年食;三登曰泰平,二十七歲,遺九年食。然後鲈德流洽,禮樂成焉。故曰「如有王者,必世而後仁」,繇此道也。」

周室既衰,暴君污吏慢其經界,繇役橫作,政令不信,上下相詐,公田不治。故魯宣公「初稅畝」,春秋譏焉。於是上貪民怨,災害生而禍亂作。

陵夷至於戰國,貴詐力而賤仁誼,先富有而後禮讓。是時,李悝為魏文侯作盡地力之教,以為地方百里,提封九萬頃,除山澤邑居參分去一,為田六百萬畝,治田勤謹則畝益三升,不勤則損亦如之。地方百里之增減,輒為粟百八十萬石矣。又曰糴其貴傷民,甚賤傷農;民傷則離散,農傷則國貧。故甚貴與甚賤,其傷一也。善為國者,使民毋傷而農益勸。今一夫挾五口,治田百畝,歲收畝一石半,為粟百五十石,除十一之稅十五石,餘百三十五石。食,人月一石半,五人終歲為粟九十石,餘有四十五石。石三十,為錢千三百五十,除社閭嘗新春秋之祠,用錢三百,餘千五十。衣,人率用錢三百,五人終歲用千五百,不足四百五十。不幸疾病死喪之費,及上賦斂,又未與此。此農夫所以常困,有不勸耕之心,而令糴至於甚貴者也。是故善平糴者,必謹觀歲有上中下孰。上孰其收自四,餘四百石;中孰自三,餘三百石;下孰自倍,餘百石。小飢則收百石,中飢七十石,大飢三十石。故大孰則上糴三而舍一,中孰則糴二,下孰則糴一,使民適足,賈平則止。小飢則發小孰之所斂,中飢則發中孰之所斂,大飢則發大孰之所斂,而糶之。故雖遇饑饉水旱,糴不貴而民不散,取有餘以補不足也。行之魏國,國以富彊。

及秦孝公用商君,壞井田,開仟伯,急耕戰之賞,雖非古道,猶以務本之故,傾鄰國而雄諸侯。然王制遂滅,僭差亡度。庶人之富者累鉅萬,而貧者食糟糠;有國彊者兼州域,而弱者喪社稷。至於始皇,遂并天下,內興功作,外攘夷狄,收泰半之賦,發閭左之戍。男子力耕不足糧饟,女子紡績不足衣服。竭天下之資財以奉其政,猶未足以澹其欲也。海內愁怨,遂用潰畔。

漢興,接秦之敝,諸侯並起,民失作業,而大饑饉。凡米石五千,人相食,死者過半。高祖乃令民得賣子,就食蜀漢。天下既定,民亡蓋臧,自天子不能具醇駟,而將相或乘牛車。上於是約法省禁,輕田租,什五而稅一,量吏祿,度官用,以賦於民。而山川園池市肆租稅之入,自天子以至封君湯沐邑,皆各為私奉養,不領於天子之經費。漕轉關東粟以給中都官,歲不過數十萬石。孝惠、高后之間,衣食滋殖。文帝即位,躬修儉節,思安百姓。時民近戰國,皆背本趨末,賈誼說上曰:

筦子曰「倉廩實而知禮節」。民不足而可治者,自古及今,未之嘗聞。古之人曰:「一夫不耕,或受之飢;一女不織,或受之寒。」生之有時,而用之亡度,則物力必屈。古之治天下,至孅至悉也,故其畜積足恃。今背本而趨末,食者甚眾,是天下之大殘也;淫侈之俗,日日以長,是天下之大賊也。殘賊公行,莫之或止;大命將泛,莫之振救。生之者甚少而靡之者甚多,天下財產何得不蹶!漢之為漢幾四十年矣,公私之積猶可哀痛。失時不雨,民且狼顧;歲惡不入,請賣爵、子。既聞耳矣,安有為天下阽危者若是而上不驚者!

世之有飢穰,天之行也,禹、湯被之矣。即不幸有方二三千里之旱,國胡以相恤?卒然邊境有急,數十百萬之眾,國胡以餽之?兵旱相乘,天下大屈,有勇力者聚徒而衡擊,罷夫羸老易子而沦其骨。政治未畢通也,遠方之能疑者並舉而爭起矣,乃駭而圖之,豈將有及乎?

夫積貯者,天下之大命也。苟粟多而財有餘,何為而不成?以攻則取,以守則固,以戰則勝。懷敵附遠,何招而不至?今毆民而歸之農,皆著於本,使天下各食其力,末技游食之民轉而緣南畝,則畜積足而人樂其所矣。可以為富安天下,而直為此廩廩也,竊為陛下惜之!

於是上感誼言,始開籍田,躬耕以勸百姓。晁錯復說上曰:

聖王在上而民不凍飢者,非能耕而食之,織而衣之也,為開其資財之道也。故堯、禹有九年之水,湯有七年之旱,而國亡捐瘠者,以畜積多而備先具也。今海內為一,土地人民之眾不避湯、禹,加以亡天災數年之水旱,而畜積未及者,何也?地有遺利,民有餘力,生穀之土未盡墾,山澤之利未盡出也,游食之民未盡歸農也。民貧,則姦邪生。貧生於不足,不足生於不農,不農則不地著,不地著則離鄉輕家,民如鳥獸,雖有高城深池,嚴法重刑,猶不能禁也。

夫寒之於衣,不待輕煖;飢之於食,不待甘旨;飢寒至身,不顧廉恥。人情,一日不再食則飢,終歲不製衣則寒。夫腹飢不得食,膚寒不得衣,雖慈母不能保其子,君安能以有其民哉!明主知其然也,故務民於農桑,薄賦斂,廣畜積,以實倉廩,備水旱,故民可得而有也。

民者,在上所以牧之,趨利如水走下,四方亡擇也。夫珠玉金銀,飢不可食,寒不可衣,然而眾貴之者,以上用之故也。其為物輕微易臧,在於把握,可以周海內而亡飢寒之患。此令臣輕背其主,而民易去其鄉,盜賊有所勸,亡逃者得輕資也。粟米布帛生於地,長於時,聚於力,非可一日成也;數石之重,中人弗勝,不為姦邪所利,一日弗得而飢寒至。是故明君貴五穀而賤金玉。

今農夫五口之家,其服役者不下二人,其能耕者不過百畝,百畝之收不過百石。春耕夏耘,秋穫冬臧,伐薪樵,治官府,給繇役;春不得避風塵,夏不得避暑熱,秋不得避陰雨,冬不得避寒凍,四時之間亡日休息;又私自送往迎來,弔死問疾,養孤長幼在其中。勤苦如此,尚復被水旱之災,急政暴虐,賦斂不時,朝令而暮改。當具有者半賈而賣,亡者取倍稱之息,於是有賣田宅鬻子孫以償責者矣。而商賈大者積貯倍息,小者坐列販賣,操其奇贏,日游都市,乘上之急,所賣必倍。故其男不耕耘,女不蠶織,衣必文采,食必梁肉;亡農夫之苦,有仟伯之得。因其富厚,交通王侯,力過吏勢,以利相傾;千里游敖,冠蓋相望,乘堅策肥,履絲曳縞。此商人所以兼并農人,農人所以流亡者也。

今法律賤商人,商人已富貴矣;尊農夫,農夫已貧賤矣。故俗之所貴,主之所賤也;吏之所卑,法之所尊也。上下相反,好惡乖迕,而欲國富法立,不可得也。方今之務,莫若使民務農而已矣。欲民務農,在於貴粟;貴粟之道,在於使民以粟為賞罰。今募天下入粟縣官,得以拜爵,得以除罪。如此,富人有爵,農民有錢,粟有所渫。夫能入粟以受爵,皆有餘者也;取於有餘,以供上用,則貧民之賦可損,所謂損有餘補不足,令出而民利者也。順於民心,所補者三:一曰主用足,二曰民賦少,三曰勸農功。今令民有車騎馬一匹者,復卒三人。車騎者,天下武備也,故為復卒。神農之教曰:「有石城十仞,湯池百步,帶甲百萬,而亡粟,弗能守也。」以是觀之,粟者,王者大用,政之本務。令民入粟受爵至五大夫以上,乃復一人耳,此其與騎馬之功相去遠矣。爵者,上之所擅,出於口而亡窮;粟者,民之所種,生於地而不乏。夫得高爵與免罪,人之所甚欲也。使天下入粟於邊,以受爵免罪,不過三歲,塞下之粟必多矣。

於是文帝從錯之言,令民入粟邊,六百石爵上造,稍增至四千石為五大夫,萬二千石為大庶長,各以多少級數為差。錯復奏言:「陛下幸使天下入粟塞下以拜爵,甚大惠也。竊恐塞卒之食不足用大渫天下粟。邊食足以支五歲,可令入粟郡縣矣;足支一歲以上,可時赦,勿收農民租。如此,德澤加於萬民,民俞勤農。時有軍役,若遭水旱,民不困乏,天下安寧;歲孰且美,則民大富樂矣。」上復從其言,乃下詔賜民十二年租稅之半。明年,遂除民田之租稅。

後十三歲,孝景二年,令民半出田租,三十而稅一也。其後,上郡以西旱,復修賣爵令,而裁其賈以招民;及徒復作,得輸粟於縣官以除罪。始造苑馬以廣用,宮室列館車馬益增修矣。然婁敕有司以農為務,民遂樂業。至武帝之初七十年間,國家亡事,非遇水旱,則民人給家足,都鄙廩庾盡滿,而府庫餘財。京師之錢累百鉅萬,貫朽而不可校。太倉之粟陳陳相因,充溢露積於外,腐敗不可食。眾庶街巷有馬,仟伯之間成群,乘牸牝者擯而不得會聚。守閭閻者食粱肉;為吏者長子孫;居官者以為姓號。人人自愛而重犯法,先行誼而黜媿辱焉。於是罔疏而民富,役財驕溢,或至并兼豪黨之徒以武斷於鄉曲。宗室有土,公卿大夫以下爭於奢侈,室廬車服僭上亡限。物盛而衰,固其變也。

是後,外事四夷,內興功利,役費並興,而民去本。董仲舒說上曰:「春秋它穀不書,至於麥禾不成則書之,以此見聖人於五穀最重麥與禾也。今關中俗不好種麥,是歲失春秋之所重,而損生民之具也。願陛下幸詔大司農,使關中民益種宿麥,令毋後時。」又言:「古者稅民不過什一,其求易共;使民不過三日,其力易足。民財內足以養老盡孝,外足以事上共稅,下足以畜妻子極愛,故民說從上。至秦則不然,用商鞅之法,改帝王之制,除井田,民得賣買,富者田連仟伯,貧者亡立錐之地。又顓川澤之利,管山林之饒,荒淫越制,踰侈以相高;邑有人君之尊,里有公侯之富,小民安得不困?又加月為更卒,已復為正,一歲屯戍,一歲力役,三十倍於古;田租口賦,鹽鐵之利,二十倍於古。或耕豪民之田,見稅什五。故貧民常衣牛馬之衣,而食犬彘之食。重以貪暴之吏,刑戮妄加,民愁亡聊,亡逃山林,轉為盜賊,赭衣半道,斷獄歲以千萬數。漢興,循而未改。古井田法雖難卒行,宜少近古,限民名田,以澹不足,塞并兼之路。鹽鐵皆歸於民。去奴婢,除專殺之威。薄賦斂,省繇役,以寬民力。然後可善治也。」仲舒死後,功費愈甚,天下虛耗,人復相食。

武帝末年,悔征伐之事,乃封丞相為富民侯。下詔曰:「

方今之務,在於力農。」以趙過為搜粟都尉。過能為代田,一畝三甽。歲代處,故曰代田,古法也。后稷始甽田,以二耜為耦,廣尺深尺曰甽,長終畝。一畝三甽,一夫三百甽,而播種於甽中。苗生葉以上,稍耨隴草,因隤其土以附根苗。故其詩曰:「或芸或芓,黍稷儗儗。」芸,除草也。刭,附根也。言苗稍壯,每耨輒附根,比盛暑,隴盡而根深,能風與旱,故儗儗而盛也。其耕耘下種田器,皆有便巧。率十二夫為田一井一屋,故具五頃,用耦犁,二牛三人,一歲之收常過縵田具一斛以上,善者倍之。過使教田太常、三輔,大農置工巧奴與從事,為作田器。二千石遣令長、三老、力田及里父老善田者受田器,學耕種養苗狀。民或苦少牛,亡以趨澤,故平都令光教過以人輓犁。過奏光以為丞,教民相與庸輓犁。率多人者田日三十畝,少者十三畝,以故田多墾闢。過試以離宮卒田其宮壖地,課得穀皆多其旁田畝一斛以上。令命家田三輔公田,又教邊郡及居延城。是後邊城、河東、弘農、三輔、太常民皆便代田,用力少而得穀多。

至昭帝時,流民稍還,田野益闢,頗有畜積。宣帝即位,用吏多選賢良,百姓安土,歲數豐穰,穀至石五錢,農人少利。時大司農中丞耿壽昌以善為算能商功利得幸於上,五鳳中奏言:「

故事,歲漕關東穀四百萬斛以給京師,用卒六萬人。宜糴三輔、弘農、河東、上黨、太原郡穀足供京師,可以省關東漕卒過半。」又白增海租三倍,天子皆從其計。御史大夫蕭望之奏言:「故御史屬徐宮家在東萊,言往年加海租,魚不出。長老皆言武帝時縣官嘗自漁,海魚不出,後復予民,魚乃出。夫陰陽之感,物類相應,萬事盡然。今壽昌欲近糴漕關內之穀,築倉治船,費直二萬萬餘,有動眾之功,恐生旱氣,民被其災。壽昌習於商功分銖之事,其深計遠慮,誠未足任,宜且如故。」上不聽。漕事果便,壽昌遂白令邊郡皆築倉,以穀賤時增其賈而糴,以利農,穀貴時減賈而糶,名曰常平倉。民便之。上乃下詔,賜壽昌爵關內侯。而蔡癸以好農使勸郡國,至大官。

元帝即位,天下大水,關東郡十一尤甚。二年,齊地飢,穀石三百餘,民多餓死,琅邪郡人相食。在位諸儒多言鹽鐵官及北假田官、常平倉可罷,毋與民爭利。上從其議,皆罷之。又罷建章、甘泉宮衛,角抵,齊三服官,省禁苑以予貧民,減諸侯王廟衛卒半。又減關中卒五百人,轉穀振貸窮乏。其後用度不足,獨復鹽鐵官。

成帝時,天下亡兵革之事,號為安樂,然俗奢侈,不以畜聚為意。永始二年,梁國、平原郡比年傷水災,人相食,刺史守相坐免。

哀帝即位,師丹輔政,建言:「古之聖王莫不設井田,然後治乃可平。孝文皇帝承亡周亂秦兵革之後,天下空虛,故務勸農桑,帥以節儉。民始充實,未有并兼之害,故不為民田及奴婢為限。今累世承平,豪富吏民訾數鉅萬,而貧弱俞困。蓋君子為政,貴因循而重改作,然所以有改者,將以救急也。亦未可詳,宜略為限。」天子下其議。丞相孔光、大司空何武奏請:「諸侯王、列侯皆得名田國中。列侯在長安,公主名田縣道,及關內侯、吏民名田皆毋過三十頃。諸侯王奴婢二百人,列侯、公主百人,關內侯、吏民三十人。期盡三年,犯者沒入官。」時田宅奴婢賈為減賤,丁、傅用事,董賢隆貴,皆不便也。詔書且須後,遂寢不行。宮室苑囿府庫之臧已侈,百姓訾富雖不及文景,然天下戶口最盛矣。

平帝崩,王莽居攝,遂篡位。王莽因漢承平之業,匈奴稱藩,百蠻賓服,舟車所通,盡為臣妾,府庫百官之富,天下晏然。莽一朝有之,其心意未滿,骥小漢家制度,以為疏闊。宣帝始賜單于印璽,與天子同,而西南夷鉤町稱王。莽乃遣使易單于印,貶鉤町王為侯。二方始怨,侵犯邊境。莽遂興師,發三十萬眾,欲同時十道並出,一舉滅匈奴;募發天下囚徒丁男甲卒轉委輸兵器,自負海江淮而至北邊,使者馳傳督趣,海內擾矣。又動欲慕古,不度時宜,分裂州郡,改職作官,下令曰:「漢氏減輕田租,三十而稅一,常有更賦,罷癃咸出,而豪民侵陵,分田劫假,厥名三十,實什稅五也。富者驕而為邪,貧者窮而為姦,俱陷於辜,刑用不錯。今更名天下田曰王田,奴婢曰私屬,皆不得賣買。其男口不滿八,而田過一井者,分餘田與九族鄉黨。」犯令,法至死,制度又不定,吏緣為姦,天下謷謷然,陷刑者眾。

後三年,莽知民愁,下詔諸食王田及私屬皆得賣買,勿拘以法。然刑罰深刻,它政誖亂。邊兵二十餘萬人仰縣官衣食,用度不足,數橫賦歛,民俞貧困。常苦枯旱,亡有平歲,穀賈翔貴。

末年,盜賊群起,發軍擊之,將吏放縱於外。北邊及青徐地人相食,雒陽以東米石二千。莽遣三公將軍開東方諸倉振貸窮乏,又分遣大夫謁者教民煮木為酪;酪不可食,重為煩擾。流民入關者數十萬人,置養澹官以稟之,吏盜其稟,飢死者什七八。莽恥為政所致,乃下詔曰:「予遭陽九之阨,百六之會,枯旱霜蝗,饑饉荐臻,蠻夷猾夏,寇賊姦軌,百姓流離。予甚悼之,害氣將究矣。」歲為此言,以至於亡。

凡貨,金錢布帛之用,夏殷以前其詳靡記云。太公為周立九府圜法:黃金方寸,而重一斤;錢圜函方,輕重以銖;布帛廣二尺二寸為幅,長四丈為匹。故貨寶於金,利於刀,流於泉,布於布,束於帛。

太公退,又行之于齊。至管仲相桓公,通輕重之權,曰:「歲有凶穰,故穀有貴賤;令有緩急,故物有輕重。人君不理,則畜賈游於市,乘民之不給,百倍其本矣。故萬乘之國必有萬金之賈,千乘之國必有千金之賈者,利有所并也。計本量委則足矣。然而民有飢餓者,穀有所臧也。民有餘則輕之,故人君斂之以輕;民不足則重之,故人君散之以重。凡輕重斂散之以時,則準平。使萬室之邑必有萬鍾之臧,臧繈千萬;千室之邑必有千鍾之臧,臧繈百萬。春以奉耕,夏以奉耘,耒耜器械,種饟糧食,必取澹焉。故大賈畜家不得豪奪吾民矣。」桓公遂用區區之齊合諸侯,顯伯名。

其後百餘年,周景王時患錢輕,將更鑄大錢,單穆公曰:「不可。古者天降災戾,於是乎量資幣,權輕重,以救民。民患輕,則為之作重幣以行之,於是有母權子而行,民皆得焉。若不堪重,則多作輕而行之,亦不廢重,於是乎有子權母而行,小大利之。今王廢輕而作重,民失其資,能無匱乎?民若匱,王用將有所乏;乏將厚取於民;民不給,將有遠志,是離民也。且絕民以實王府,猶塞川原為潢洿也,竭亡日矣。王其圖之。」弗聽,卒鑄大錢,文曰「寶貨」,肉好皆有周郭,以勸農澹不足,百姓蒙利焉。

秦兼天下,幣為二等:黃金以溢為名,上幣;銅錢質如周錢,文曰「半兩」,重如其文。而珠玉龜貝銀錫之屬為器飾寶臧,不為幣,然各隨時而輕重無常。

漢興,以為秦錢重難用,更令民鑄莢錢。黃金一斤。而不軌逐利之民畜積餘贏以稽市物,痛騰躍,米至石萬錢,馬至匹百金。天下已平,高祖乃令賈人不得衣絲乘車,重稅租以困辱之。孝惠、高后時,為天下初定,復弛商賈之律,然市井子孫亦不得宦為吏。孝文五年,為錢益多而輕,乃更鑄四銖錢,其文為「半兩」。除盜鑄錢令,使民放鑄。賈誼諫曰:

法使天下公得顧租鑄銅錫為錢,敢雜以鉛鐵為它巧者,其罪黥。然鑄錢之情,非殽雜為巧,則不可得贏;而殽之甚微,為利甚厚。夫事有召禍而法有起姦,今令細民人操造幣之勢,各隱屏而鑄作,因欲禁其厚利微姦,雖黥罪日報,其勢不止。乃者,民人抵罪,多者一縣百數,及吏之所疑,榜笞奔走者甚眾。夫縣法以誘民,使入陷阱,孰積於此!曩禁鑄錢,死罪積下;今公鑄錢,黥罪積下。為法若此,上何賴焉?

又民用錢,郡縣不同:或用錢輕,百加若干;或用重錢,平稱不受。法錢不立,吏急而壹之虖,則大為煩苛,而力不能勝;縱而弗呵虖,則市肆異用,錢文大亂。苟非其術,何鄉而可哉!

今農事棄捐而采銅者日蕃,釋其耒耨,冶鎔炊炭,姦錢日多,五穀不為多。善人怵而為姦邪,愿民陷而之刑戮,刑戮將甚不詳,奈何而忽!國知患此,吏議必曰禁之。禁之不得其術,其傷必大。令禁鑄錢,則錢必重;重則其利深,盜鑄如雲而起,棄市之罪又不足以禁矣。姦數不勝而法禁數潰,銅使之然也。故銅布於天下,其為禍博矣。

今博禍可除,而七福可致也。何謂七福?上收銅勿令布,則民不鑄錢,黥罪不積,一矣。偽錢不蕃,民不相疑,二矣。采銅鑄作者反於耕田,三矣。銅畢歸於上,上挾銅積以御輕重,錢輕則以術斂之,重則以術散之,貨物必平,四矣。以作兵器,以假貴臣,多少有制,用別貴踐,五矣。以臨萬貨,以調盈虛,以收奇羨,則官富實而末民困,六矣。制吾棄財,以與匈奴逐爭其民,其敵必懷,七矣。故善為天下者,因禍而為福,轉敗而為功。今久退七福而行博禍,臣誠傷之。

上不聽。是時,吳以諸侯即山鑄錢,富埒天子,後卒叛逆。鄧通,大夫也,以鑄錢財過王者。故吳、鄧錢布天下。

武帝因文、景之畜,忿胡、粵之害,即位數年,嚴助、朱買臣等招徠東甌,事兩粵,江淮之間蕭然煩費矣。唐蒙、司馬相如始開西南夷,鑿山通道千餘里,以廣巴蜀,巴蜀之民罷焉。彭吳穿穢貊、朝鮮,置滄海郡,則燕齊之間靡然發動。及王恢謀馬邑,匈奴絕和親,侵優北邊,兵連而不解,天下共其勞。干戈日滋,行者齎,居者送,中外騷擾相奉,百姓抏敝以巧法,財賂衰耗而不澹。入物者補官,出貨者除罪,選舉陵夷,廉恥相冒,武力進用,法嚴令具,興利之臣自此而始。

其後,衛青歲以數萬騎出擊匈奴,遂取河南地,築朔方。時又通西南夷道,作者數萬人,千里負擔餽饟,率十餘鍾致一石,散幣於邛僰以輯之。數歲而道不通,蠻夷因以數攻吏,吏發兵誅之。悉巴蜀租賦不足以更之,乃募豪民田南夷,入粟縣官,而內受錢於都內。東置滄海郡,人徒之費疑於南夷。又興十餘萬築衛朔方,轉漕甚遠,自山東咸被其勞,費數十百鉅萬,府庫並虛。乃募民能入奴婢得以終身復,為郎增秩,及入羊為郎,始於此。

此後四年,衛青比歲十餘萬眾擊胡,斬捕首虜之士受賜黃金二十餘萬斤,而漢軍士馬死者十餘萬,兵甲轉漕之費不與焉。於是大司農陳臧錢經用,賦稅既竭,不足以奉戰士。有司請令民得買爵及贖禁錮免臧罪;請置賞官,名曰武功爵。級十七萬,凡直三十餘萬金。諸買武功爵官首者試補吏,先除;千夫如五大夫;其有罪又減二等;爵得至樂卿,以顯軍功。軍功多用超等,大者封侯卿大夫,小者郎。吏道雜而多端,則官職秏廢。

自孫弘以春秋之義繩臣下取漢相,張湯以峻文決理為廷尉,於是見知之法生,而廢格沮誹窮治之獄用矣。其明年,淮南、衡山、江都王謀反跡見,而公卿尋端治之,竟其黨與,坐而死者數萬人,吏益慘急而法令察。當是時,招尊方正賢良文學之士,或至公卿大夫。公孫弘以宰相,布被,食不重味,為下先,然而無益於俗,稍務於功利矣。

其明年,票騎仍再出擊胡,大克獲。渾邪王率數萬眾來降,於是漢發車三萬兩迎之。既至,受賞,賜及有功之士。是歲費凡百餘鉅萬。

先是十餘歲,河決,灌梁、楚地,固已數困,而緣河之郡隄塞河,輒壞決,費不可勝計。其後番係欲省底柱之漕,穿汾、河渠以為溉田;鄭當時為渭漕回遠,鑿漕直渠自長安至華陰;而朔方亦穿溉渠。作者各數萬人,歷二三期而功未就,費亦各以鉅萬十數。

天子為伐胡故,盛養馬,馬之往來食長安者數萬匹,卒掌者關中不足,乃調旁近郡。而胡降者數萬人皆得厚賞,衣食仰給縣官,縣官不給,天子乃損膳,解乘輿駟,出御府禁臧以澹之。

其明年,山東被水災,民多飢乏,於是天子遣使虛郡國倉廩以振貧。猶不足,又募豪富人相假貸。尚不能相救,乃徙貧民於關以西,及充朔方以南新秦中,七十餘萬口,衣食皆仰給於縣官。數歲,貸與產業,使者分部護,冠蓋相望,費以億計,縣官大空。而富商賈或墆財役貧,轉轂百數,廢居居邑,封君皆氐首仰給焉。冶鑄煮鹽,財或累萬金,而不佐公家之急,黎民重困。

於是天子與公卿議,更造錢幣以澹用,而摧浮淫并兼之徒。是時禁苑有白鹿而少府多銀錫。自孝文更造四銖錢,至是歲四十餘年,從建元以來,用少,縣官往往即多銅山而鑄錢,民亦盜鑄,不可勝數。錢益多而輕,物益少而貴。有司言曰:「

古者皮幣,諸侯以聘享。金有三等,黃金為上,白金為中,赤金為下。今半兩錢法重四銖,而姦或盜摩錢質而取鋊,錢益輕薄而物貴,則遠方用幣煩費不省。」乃以白鹿皮方尺,緣以繢,為皮幣,直四十萬。王侯宗室朝覲聘享,必以皮幣薦璧,然後得行。

又造銀錫白金。以為天用莫如龍,地用莫如馬,人用莫如龜,故白金三品:其一曰重八兩,圜之,其文龍,名「白撰」,直三千;二曰以重差小,方之,其文馬,直五百;三曰復小,橢之,其文龜,直三百。令縣官銷半兩錢,更鑄三誅錢,重如其文。盜鑄諸金錢罪皆死,而吏民之犯者不可勝數。

於是以東郭咸陽、孔僅為大農丞,領鹽鐵事,而桑弘羊貴幸。咸陽,齊之大煮鹽,孔僅,南陽大冶,皆致產累千金,故鄭當時進言之。弘羊,洛陽賈人之子,以心計,年十三侍中。故三人言利事析秋豪矣。

法既益嚴,吏多廢免。兵革數動,民多買復及五大夫、千夫,徵發之士益鮮。於是除千夫、五大夫為吏,不欲者出馬;故吏皆適令伐棘上林,作昆明池。

其明年,大將軍、票騎大出擊胡,賞賜五十萬金,軍馬死者十餘萬匹,轉漕車甲之費不與焉。是時財匱,戰士頗不得祿矣。

有司言三銖錢輕,輕錢易作姦詐,乃更請郡國鑄五銖錢,周郭其質,令不可得摩取鉛。

大農上鹽鐵丞孔僅、咸陽言:「山海,天地之臧,宜屬少府,陛下弗私,以屬大農佐賦。願募民自給費,因官器作煮鹽,官與牢盆。浮食奇民欲擅斡山海之貨,以致富羨,役利細民。其沮事之議,不可勝聽。敢私鑄鐵器煮鹽者,釱左趾,沒入其器物。郡不出鐵者,置小鐵官,使屬在所縣。」使僅、咸陽乘傳舉行天下鹽鐵,作官府,除故鹽鐵家富者為吏。吏益多賈人矣。

商賈以幣之變,多積貨逐利。於是公卿言:「郡國頗被災害,貧民無產業者,募徙廣饒之地。陛下損膳省用,出禁錢以振元元,寬貸,而民不齊出南畝,商賈滋眾。貧者畜積無有,皆仰縣官。異時算軺車賈人之嬢錢皆有差,請算如故。諸賈人末作貰貸賣買,居邑貯積諸物,及商以取利者,雖無市籍,各以其物自占,率嬢錢二千而算一。諸作有租及鑄,率嬢錢四千算一。非吏比者、三老、北邊騎士,軺車一算;商賈人軺車二算;船五丈以上一算。匿不自占,占不悉,戍邊一歲,沒入嬢錢。有能告者,以其半畀之。賈人有市籍,及家屬,皆無得名田,以便農。敢犯令,沒入田貨。」

是時,豪富皆爭匿財,唯卜式數求入財以助縣官。天子乃超拜式為中郎,賜爵左庶長,田十頃,布告天下,以風百姓。初,式不願為官,上強拜之,稍遷至齊相。語自在其傳。孔僅使天下鑄作器,三年中至大司農,列於九卿。而桑弘羊為大司農中丞,管諸會計事,稍稍置均輸以通貨物。始令吏得入穀補官,郎至六百石。

自造白金五銖錢後五歲,而赦吏民之坐盜鑄金錢死者數十萬人。其不發覺相殺者,不可勝計。赦自出者百餘萬人。然不能半自出,天下大氐無慮皆鑄金錢矣。犯法者眾,吏不能盡誅,於是遣博士褚大、徐偃等分行郡國,舉并兼之徒守相為利者。而御史大夫張湯方貴用事,減宣、杜周等為中丞,義縱、尹齊、王溫舒等用急刻為九卿,直指夏蘭之屬始出。而大農顏異誅矣。初,異為濟南亭長,以廉直稍遷至九卿。上與湯既造白鹿皮幣,問異。異曰:「今王侯朝賀以倉璧,直數千,而其皮薦反四十萬,本末不相稱。」天子不說。湯又與異有隙,及人有告異以它議,事下湯治。異與客語,客語初令下有不便者,異不應,微反脣。湯奏當異九卿見令不便,不入言而腹非,論死。自是後有腹非之法比,而公卿大夫多諂諛取容。

天下既下嬢錢令而尊卜式,百姓終莫分財佐縣官,於是告嬢錢縱矣。

郡國鑄錢,民多姦鑄,錢多輕,而公卿請令京師鑄官赤仄,一當五,賦官用非赤仄不得行。白金稍賤,民弗寶用,縣官以令禁之,無益,歲餘終廢不行。是歲,湯死而民不思。其後二歲,赤仄錢賤,民巧法用之,不便,又廢。於是悉禁郡國毋鑄錢,專令上林三官鑄。錢既多,而令天下非三官錢不得行,諸郡國前所鑄錢皆廢銷之,輸入其銅三官。而民之鑄錢益少,計其費不能相當,唯真工大姦乃盜為之。

楊可告嬢遍天下,中家以上大氐皆遇告。杜周治之,獄少反者。乃分遣御史廷尉正監分曹往,往即治郡國嬢錢,得民財物以億計,奴婢以千萬數,田大縣數百頃,小縣百餘頃,宅亦如之。於是商賈中家以上大氐破,民媮甘食好衣,不事畜臧之業,而縣官以鹽鐵嬢錢之故,用少饒矣。益廣開,置左右輔。

初,大農幹鹽鐵官布多,置水衡,欲以主鹽鐵;及楊可告嬢,上林財物眾,乃令水衡主上林。上林既充滿,益廣。是時粵欲與漢用船戰逐,乃大修昆明池,列館環之。治樓船,高十餘丈,旗織加其上,甚壯。於是天子感之,乃作柏梁臺,高數十丈。宮室之修,繇此日麗。

乃分嬢錢諸官,而水衡、少府、太僕、大農各置農官,往往即郡縣比沒入田田之。其沒入奴婢,分諸苑養狗馬禽獸,及與諸官。官益雜置多,徒奴婢眾,而下河漕度四百萬石,及官自糴乃足。

所忠言:「世家子弟富人或鬥雞走狗馬,弋獵博戲,亂齊民。」乃徵諸犯令,相引數千人,名曰「株送徒」。入財者得補郎,郎選衰矣。

是時山東被河災,及歲不登數年,人或相食,方二三千里。天子憐之,令飢民得流就食江淮間,欲留,留處。使者冠蓋相屬於道護之,下巴蜀粟以振焉。

明年,天子始出巡郡國。東度河,河東守不意行至,不辯,自殺。行西踰隴,卒,從官不得食,隴西守自殺。於是上北出蕭關,從數萬騎行獵新秦中,以勒邊兵而歸。新秦中或千里無亭徼,於是誅北地太守以下,而令民得畜邊縣,官假馬母,三歲而歸,及息什一,以除告嬢,用充入新秦中。

既得寶鼎,立后土、泰一祠,公卿白議封禪事,而郡國皆豫治道,修繕故宮,及當馳道縣,縣治宮儲,設共具,而望幸。

明年,南粵反,西羌侵邊。天子為山東不澹,赦天下囚,因南方樓船士二十餘萬人擊粵,發三河以西騎擊羌,又數萬人度河築令居。初置張掖、酒泉郡,而上郡、朔方、西河、河西開田官,斥塞卒六十萬人戍田之。中國繕道餽糧,遠者三千,近者千餘里,皆仰給大農。邊兵不足,乃發武庫工官兵器以澹之。車騎馬乏,縣官錢少,買馬難得,乃著令,令封君以下至三百石吏以上差出

牡馬天下亭,亭有畜字馬,歲課息。

齊相卜式上書,願父子死南粵。天子下詔褒揚,賜爵關內侯,黃金四十斤,田十頃。布告天下,天下莫應。列侯以百數,皆莫求從軍。至飲酎,少府省金,而列侯坐酎金失侯者百餘人。乃拜卜式為御史大夫。式既在位,見郡國多不便縣官作鹽鐵,器苦惡,賈貴,或彊令民買之。而船有算,商者少,物貴,乃因孔僅言船算事。上不說。

漢連出兵三歲,誅羌,滅兩粵,番禺以西至蜀南者置初郡十七,且以其故俗治,無賦稅。南陽、漢中以往,各以地比給初郡吏卒奉食幣物,傳車馬被具。而初郡又時時小反,殺吏,漢發南方吏卒往誅之,間歲萬餘人,費皆仰大農。大農以均輸調鹽鐵助賦,故能澹之。然兵所過縣,縣以為訾給毋乏而已,不敢言輕賦法矣。

其明年,元封元年,卜式貶為太子太傅。而桑弘羊為治粟都尉,領大農,盡代僅斡天下鹽鐵。弘羊以諸官各自市相爭,物以故騰躍,而天下賦輸或不償其僦費,乃請置大農部丞數十人,分部主郡國,各往往置均輸鹽鐵官,令遠方各以其物如異時商賈所轉

貶者為賦,而相灌輸。置平準於京師,都受天下委輸。召工官治車諸器,皆仰給大農。大農諸官盡籠天下之貨物,貴則賣之,賤則買之。如此,富商大賈亡所牟大利,則反本,而萬物不得騰躍。故抑天下之物,名曰「平準」。天子以為然而許之。於是天子北至朔方,東封泰山,巡海上,旁北邊以歸。所過賞賜,用帛百餘萬匹,錢金以鉅萬計,皆取足大農。

弘羊又請令民得入粟補吏,及罪以贖。令民入粟甘泉各有差,以復終身,不復告嬢。它郡各輸急處,而諸農各致粟,山東漕益歲六百萬石。一歲之中,太倉、甘泉倉滿。邊餘穀,諸均輸帛五百萬匹。民不益賦而天下用饒。於是弘羊賜爵左庶長,黃金者再百焉。

是歲小旱,上令百官求雨。卜式言曰:「縣官當食租衣稅而已,今弘羊令吏坐市列,販物求利。亨弘羊,天乃雨。」久之,武帝疾病,拜弘羊為御史大夫。

昭帝即位六年,詔郡國舉賢良文學之士,問以民所疾苦,教化之要。皆對願罷鹽鐵酒鸾均輸官,毋與天下爭利,視以儉節,然後教化可興。弘羊難,以為此國家大業,所以制四夷,安邊足用之本,不可廢也。乃與丞相千秋共奏罷酒酤。弘羊自以為國興大利,伐其功,欲為子弟得官,怨望大將軍霍光,遂與上官桀等謀反,誅滅。

宣、元、成、哀、平五世,亡所變改。元帝時嘗罷鹽鐵官,三年而復之。貢禹言:「鑄錢采銅,一歲十萬人不耕,民坐盜鑄陷刑者多。富人臧錢滿室,猶無厭足。民心動搖,棄本逐末,耕者不能半,姦邪不可禁,原起於錢。疾其末者絕其本,宜罷采珠玉金銀鑄錢之官,毋復以為幣,除其販賣租銖之律,租稅祿賜皆以布帛及穀,使百姓壹意農桑。」議者以為交易待錢,布帛不可尺寸分裂。禹議亦寢。

自孝武元狩五年三官初鑄五銖錢,至平帝元始中,成錢二百八十億萬餘云。

王莽居攝,變漢制,以周錢有子母相權,於是更造大錢,徑寸二分,重十二銖,文曰「大錢五十」。又造契刀、錯刀。契刀,其環如大錢,身形如刀,長二寸,文曰「契刀五百」。錯刀,以黃金錯其文,曰「一刀直五千」。與五銖錢凡四品,並行。

莽即真,以為書「劉」字有金刀,乃罷錯刀、契刀及五銖錢,而更作金、銀、龜、貝、錢、布之品,名曰「寶貨」。

小錢徑六分,重一銖,文曰「小錢直一」。次七分,三銖,曰「

錢一十」。次八分,五銖,曰「幼錢二十」。次九分,七銖,曰「中錢三十」。次一寸,九銖,曰「壯錢四十」。因前「大錢五十」,是為錢貨六品,直各如其文。

黃金重一斤,直錢萬。朱提銀重八兩為一流,直一千五百八十。它銀一流直千。是為銀貨二品。

元龜岠冉長尺二寸,直二千一百六十,為大貝十朋。公龜九寸,直五百,為壯貝十朋。侯龜七寸以上,直三百,為幺貝十朋。子龜五寸以上,直百,為小貝十朋。是為龜寶四品。

大貝四寸八分以上,二枚為一朋,直二百一十六。壯貝三寸六分以上,二枚為一朋,直五十。幺貝二寸四分以上,二枚為一朋,直三十。小貝寸二分以上,二枚為一朋,直十。不盈寸二分,漏度不得為朋,率枚直錢三。是為貝貨五品。

大布、次布、弟布、壯布、中布、差布、厚布、幼布、幺布、小布。小布長寸五分,重十五銖,文曰「小布一百」。自小布以上,各相長一分,相重一銖,文各為其布名,直各加一百。上至大布,長二寸四分,重一兩,而直千錢矣。是為布貨十品。

凡寶貨五物,六名,二十八品。

鑄作錢布皆用銅,殽以連錫,文質周郭放漢五銖錢云。其金銀與它物雜,色不純好,龜不盈五寸,貝不盈六分,皆不得為寶貨。元龜為蔡,非四民所得居,有者,入大卜受直。

百姓憒亂,其貨不行。民私以五銖錢市買。莽患之,下詔:「敢非井田挾五銖錢者為惑眾,投諸四裔以御魑魅。」於是農商失業,食貨俱廢,民涕泣於市道。坐賣買田宅奴婢鑄錢抵罪者,自公卿大夫至庶人,不可稱數。莽知民愁,乃但行小錢直一,與大錢五十,二品並行,龜貝布屬且寢。

莽性躁擾,不能無為,每有所興造,必欲依古得經文。國師公劉歆言周有泉府之官,收不讎,與欲得,即易所謂「理財正辭,禁民為非」者也。莽乃下詔曰:「夫周禮有賒貸,樂語有五均,傳記各有斡焉。今開賒貸,張五均,設諸斡者,所以齊眾庶,抑并兼也。」遂於長安及五都立五均官,更名長安東西市令及洛陽、邯鄲、臨甾、宛、成都市長皆為五均司市稱師。東市稱京,西市稱畿,洛陽稱中,餘四都各用東西南北為稱,皆置交易丞五人,錢府丞一人。工商能采金銀銅連錫登龜取貝者,皆自占司市錢府,順時氣而取之。

又以周官稅民:凡田不耕為不殖,出三夫之稅;城郭中宅不樹藝者為不毛,出三夫之布;民浮游無事,出夫布一匹。其不能出布者,缈作,縣官衣食之。諸取眾物鳥獸魚鱉百蟲於山林水澤及畜牧者,嬪婦桑蠶織紝紡績補縫,工匠醫巫卜祝及它方技商販賈人坐肆列里區謁舍,皆各自占所為於其在所之縣官,除其本,計其利,十一分之,而以其一為貢。敢不自占,自占不以實者,盡沒入所采取,而作縣官一歲。

諸司市常以四時中月實定所掌,為物上中下之賈,各自用為其市平,毋拘它所。眾民賣買五穀布帛絲綿之物,周於民用而不讎者,均官有以考檢厥實,用其本賈取之,毋令折錢。萬物卬貴,過平一錢,則以平賈賣與民。其賈氐賤減平者,聽民自相與市,以防貴庾者。民欲祭祀喪紀而無用者,錢府以所入工商之貢但賒之,祭祀無過旬日,喪紀毋過三月。民或乏絕,欲貸以治產業者,均授之,除其費,計所得受息,毋過歲什一。

羲和魯匡言:「名山大澤,鹽鐵錢布帛,五均賒貸,斡在縣官,唯酒酤獨未斡。酒者,天之美祿,帝王所以頤養天下,享祀祈福,扶衰養疾。百禮之會,非酒不行。故《詩》曰『無酒酤我』,而論語曰『酤酒不食』,二者非相反也。夫詩據承平之世,酒酤在官,和旨便人,可以相御也。論語孔子當周衰亂,酒酤在民,薄惡不誠,是以疑而弗食。今絕天下之酒,則無以行禮相養;放而亡限,則費財傷民。請法古,令官作酒,以二千五百石為一均,率開一盧以賣,讎五十釀為準。一釀用麤米二斛,麴一斛,得成酒六斛六斗。各以其市月朔米麴三斛,并計其賈而參分之,以其一為酒一斛之平。除米麴本賈,計其利而什分之,以其七入官,其三及纸酨灰炭給工器薪樵之費。」

羲和置命士督五均六斡,郡有數人,皆用富賈。洛陽薛子仲、張長叔、臨菑姓偉等,乘傳求利,交錯天下。因與郡縣通姦,多張空簿,府臧不實,百姓俞病。莽知民苦之,復下詔曰:「夫鹽,食肴之將;酒,百藥之長,嘉會之好;鐵,曰農之本;名山大澤,饒衍之臧;五均賒貸,百姓所取平,卬以給澹;鐵布銅冶,通行有無,備民用也。此六者,非編戶齊民所能家作,必卬於市,雖貴數倍,不得不買。豪民富賈,即要貧弱,先聖知其然也,故斡之。每一斡為設科條防禁,犯者罪至死。」姦吏猾民並侵,眾庶各不安生。

後五歲,天鳳元年,復申下金銀龜貝之貨,頗增減其賈直。而罷大小錢,改作貨布,長二寸五分,廣一寸,首長八分有奇,廣八分,其圜好徑二分半,足枝長八分,間廣二分,其文右曰「

貨」,左曰「布」,重二十五銖,直貨泉二十五。貨泉徑一寸,重五銖,文右曰「貨」,左曰「泉」,枚直一,與貨布二品並行。又以大錢行久,罷之,恐民挾不止,乃令民且獨行大錢,與新貨泉俱枚直一,並行盡六年,毋得復挾大錢矣。每壹易錢,民用破業,而大陷刑。莽以私鑄錢死,及非沮寶貨投四裔,犯法者多,不可勝行,乃更輕其法:私鑄作泉布者,與妻子沒入為官奴婢;吏及比伍,知而不舉告,與同罪;非沮寶貨,民罰作一歲,吏免官。犯者俞眾,及五人相坐皆沒入,郡國檻車鐵鎖,傳送長安鍾官,愁苦死者什六七。

作貨布六年後,匈奴侵寇甚,莽大募天下囚徒人奴,名曰豬突豨勇,壹切稅吏民,訾三十而取一。又令公卿以下至郡縣黃綬吏,皆保養軍馬,吏盡復以與民。民搖手觸禁,不得耕桑,繇役煩劇,而枯旱蝗蟲相因。又用制作未定,上自公侯,下至小吏,皆不得奉祿,而私賦斂,貨賂上流,獄訟不決。吏用苛暴立威,旁緣莽禁,侵刻小民。富者不得自保,貧者無以自存,起為盜賊,依阻山澤,吏不能禽而覆蔽之,浸淫日廣,於是青、徐、荊楚之地往往萬數。戰鬥死亡,緣邊四夷所係虜,陷罪,飢疫,人相食,及莽未誅,而天下戶口減半矣。

自發豬突豨勇後四年,而漢兵誅莽。後二年,世祖受命,盪滌煩苛,復五銖錢,與天下更始。

贊曰:《易》稱「裒多益寡,稱物平施」,《書》云「楙遷有無」,周有泉府之官,而孟子亦非「狗彘食人之食不知斂,野有餓驿而弗知發」。故管氏之輕重,李悝之平糴,弘羊均輸,壽昌常平,亦有從徠。顧古為之有數,吏良而令行,故民賴其利,萬國作乂。及孝武時,國用饒給,而民不益賦,其次也。至于王莽,制度失中,姦軌弄權,官民俱竭,亡次矣。


\end{pinyinscope}