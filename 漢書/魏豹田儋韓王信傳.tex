\article{魏豹田儋韓王信傳}

\begin{pinyinscope}
魏豹,故魏諸公子也。其兄魏咎,故魏時封為甯陵君,秦滅魏,為庶人。陳勝之王也,咎往從之。勝使魏人周市徇魏地,魏地已下,欲立周市為魏王。市曰:「天下昏亂,忠臣乃見。今天下共畔秦,其誼必立魏王後乃可。」齊、趙使車各五十乘,立市為王。市不受,迎魏咎於陳,五反,陳王乃遣立咎為魏王。

章邯已破陳王,進兵擊魏王於臨濟。魏王使周市請救齊、楚。齊、楚遣項它、田巴將兵,隨市救魏。章邯遂擊破殺周市等軍,圍臨濟。咎為其民約降。約降定,咎自殺。

魏豹亡走楚。楚懷王予豹數千人,復徇魏地。項羽己破秦兵,降章邯,豹下魏二十餘城,立為魏王。豹引精兵從項羽入關。羽封諸侯,欲有梁地,乃徙豹於河東,都平陽,為西魏王。

漢王還定三秦,渡臨晉,豹以國屬焉,遂從擊楚於彭城。漢王敗,還至滎陽,豹請視親病,至國,則絕河津畔漢。漢王謂酈生曰:「緩頰往說之。」酈生至,豹謝曰:「人生一世間,如白駒過隙。今漢王嫚侮人,罵詈諸侯群臣如奴耳,非有上下禮節,吾不忍復見也。」漢王遣韓信擊豹,遂虜之,傳豹詣滎陽,以其地為河東、太原、上黨郡。漢王令豹守滎陽。楚圍之急,周苛曰:「反國之王,難與共守。」遂殺豹。

田儋,狄人也,故齊王田氏之族也。儋從弟榮,榮弟橫,皆豪桀,宗彊,能得人。陳涉使周市略地,北至狄,狄城守。儋陽為縛其奴,從少年之廷,欲謁殺奴。見狄令,因擊殺令,而召豪吏子弟曰:「諸侯皆反秦自立,齊,古之建國,儋,田氏,當王。」遂自立為齊王,發兵擊周市。市軍還去,儋因率兵東略定齊地。

秦將章邯圍魏王咎於臨濟,急。魏王請救於齊,儋將兵救魏。章邯夜銜枚擊,大破齊、楚軍,殺儋於臨濟下。儋從弟榮收儋餘兵東走東阿。

齊人聞儋死,乃立故齊王建之弟田假為王,田角為相,田閒為將,以距諸侯。

榮之走東阿,章邯追圍之。項梁聞榮急,乃引兵擊破章邯東阿下。章邯走而西,項梁因追之。而榮怒齊之立假,乃引兵歸,擊逐假。假亡走楚。相角亡走趙。角弟閒前救趙,因不敢歸。榮乃立儋子市為王,榮相之,橫為將,平齊地。

項梁既追章邯,章邯兵益盛,項梁使使趣齊兵共擊章邯。榮曰:「楚殺田假,趙殺角、閒,乃出兵。」楚懷王曰:「田假與國之王,窮而歸我,殺之不誼。」趙亦不殺田角、田閒以市於齊。齊王曰:「蝮酿手則斬手,酿足則斬足。何者?為害於身也。田假、田角、田閒於楚、趙,非手足戚,何故不殺?且秦復得志於天下,則齮齕首用事者墳墓矣。」楚、趙不聽齊,齊亦怒,終不肯出兵。章邯果敗殺項梁,破楚兵。楚兵東走,而章邯渡河圍趙於鉅鹿。項羽由此怨榮。

羽既存趙,降章邯,西滅秦,立諸侯王,乃徙齊王市更王膠東,治即墨。齊將田都從共救趙,因入關,故立都為齊王,治臨菑。故齊王建孫田安,項羽方渡河救趙,安下濟北數城,引兵降項羽,羽立安為濟北王,治博陽。榮以負項梁,不肯助楚攻秦,故不得王。趙將陳餘亦失職,不得王。二人俱怨項羽。

榮使人將兵助陳餘,令反趙地,而榮亦發兵以距擊田都,都亡走楚。榮留齊王市毋之膠東。市左右曰:「項王強暴,王不就國,必危。」市懼,乃亡就國。榮怒,追擊殺市於即墨,還攻殺濟北王安,自立為王,盡并三齊之地。

項王聞之,大怒,乃北伐齊。榮發兵距之城陽。榮兵敗,走平原,平原民殺榮。項羽遂燒夷齊城郭,所過盡屠破。齊人相聚畔之。榮弟橫收齊散兵,得數萬人,反擊項羽於城陽。而漢王帥諸侯敗楚,入彭城。項羽聞之,乃釋齊而歸擊漢於彭城,因連與漢戰,相距滎陽。以故橫復收齊城邑,立榮子廣為王,而橫相之,政事無巨細皆斷於橫。

定齊三年,聞漢將韓信引兵且東擊齊,齊使華毋傷、田解軍歷下以距漢。會漢使酈食其往說王廣及相橫,與連和。橫然之,乃罷歷下守備,縱酒,且遣使與漢平。韓信乃渡平原,襲破齊歷下軍,因入臨菑。王廣、相橫以酈生為賣己而亨之。廣東走高密,橫走博,守相田光走城陽,將軍田既軍於膠東。楚使龍且救齊,齊王與合軍高密。漢將韓信、曹參破殺龍且,虜齊王廣。漢將灌嬰追得守相光,至博。而橫聞王死,自立為王,還擊嬰,嬰敗橫軍於贏下。橫亡走梁,歸彭越。越時居梁地,中立,且為漢,且為楚。韓信已殺龍且,因進兵破殺田既於膠東,灌嬰破殺齊將田吸於千乘,遂平齊地。

漢滅項籍,漢王立為皇帝,彭越為梁王。橫懼誅,而與其徒屬五百餘人入海,居闯中。高帝聞之,以橫兄弟本定齊,齊人賢者多附焉,今在海中不收,後恐有亂,乃使使赦橫罪而召之。橫謝曰:「臣亨陛下之使酈食其,今聞其弟商為漢將而賢,臣恐懼,不敢奉詔,請為庶人,守海闯中。」使還報,高帝乃詔衛尉酈商曰:「齊王橫即至,人馬從者敢動搖者致族夷!」乃復使使持節具告以詔意,曰:「橫來,大者王,小者乃侯耳;不來,且發兵加誅。」橫乃與其客二人乘傳詣雒陽。

至尸鄉廄置,橫謝使者曰:「人臣見天子,當洗沐。」止留。謂其客曰:「橫始與漢王俱南面稱孤,今漢王為天子,而橫乃為亡虜,北面事之,其媿固已甚矣。又吾亨人之兄,與其弟併肩而事主,縱彼畏天子之詔,不敢動搖,我獨不媿於心乎?且陛下所以欲見我,不過欲壹見我面貌耳。陛下在雒陽,今斬吾頭,馳三十里間,形容尚未能敗,猶可知也。」遂自剄,令客奉其頭,從使者馳奏之高帝。高帝曰:「嗟乎,有以!起布衣,兄弟三人更王,豈非賢哉!」為之流涕,而拜其二客為都尉,發卒二千,以王者禮葬橫。

既葬,二客穿其冢旁,皆自剄從之。高帝聞而大驚,以橫之客皆賢者,吾聞其餘尚五百人在海中,使使召至,聞橫死,亦皆自殺。於是乃知田橫兄弟能得士也。

韓王信,故韓襄王孽孫也,長八尺五寸。項梁立楚懷王,燕、齊、趙、魏皆已前王,唯韓無有後,故立韓公子橫陽城君為韓王,欲以撫定韓地。項梁死定陶,成奔懷王。沛公引兵擊陽城,使張良以韓司徒徇韓地,得信,以為韓將,將其兵從入武關。

沛公為漢王,信從入漢中,乃說漢王曰:「項王王諸將,王獨居此,遷也。士卒皆山東人,竦而望歸,及其蠭東鄉,可以爭天下。」漢王還定三秦,乃許王信,先拜為韓太尉,將兵略韓地。

項籍之封諸王皆就國,韓王成以不從無功,不遣之國,更封為穰侯,後又殺之。聞漢遣信略韓地,乃令故籍游吳時令鄭昌為韓王距漢。漢二年,信略定韓地十餘城。漢王至河南,信急擊韓王昌,昌降漢。乃立信為韓王,常將韓兵從。漢王使信與周苛等守滎陽,楚拔之,信降楚。已得亡歸漢,漢復以為韓王,竟從擊破項籍。五年春,與信剖符,王潁川。

六年春,上以為信壯武,北近鞏、雒,南迫宛、葉,東有淮陽,皆天下勁兵處也,乃更以太原郡為韓國,徙信以備胡,都晉陽。信上書曰:「國被邊,匈奴數入,晉陽去塞遠,請治馬邑。」上許之。秋,匈奴冒頓大入圍信,信數使使胡求和解。漢發兵救之,疑信數間使,有二心。上賜信書責讓之曰:「專死不勇,專生不任,寇攻馬邑,君王力不足以堅守乎?安危存亡之地,此二者朕所以責於君王。」信得書,恐誅,因與匈奴約共攻漢,以馬邑降胡,擊太原。

七年冬,上自往擊破信軍銅鞮,斬其將王喜。信亡走匈奴。與其將白土人曼丘臣、王黃立趙苗裔趙利為王,復收信散兵,而與信及冒頓謀攻漢。匈奴使左右賢王將萬餘騎與王黃等屯廣武以南,至晉陽,與漢兵戰,漢兵大破之,追至于離石,復破之。匈奴復聚兵樓煩西北。漢令車騎擊匈奴,常敗走,漢乘勝追北。聞冒頓居代谷,上居晉陽,使人視冒頓,還報曰「可擊」。上遂至平城,上白登。匈奴騎圍上,上乃使人厚遺閼氏。閼氏說冒頓曰:「今得漢地,猶不能居,且兩主不相厄。」居七日,胡騎稍稍引去。天霧,漢使人往來,胡不覺。護軍中尉陳平言上曰:「胡者全兵,請令彊弩傅兩矢外鄉,徐行出圍。」入平城,漢救兵亦至。胡騎遂解去,漢亦罷兵歸。信為匈奴將兵往來擊邊,令王黃等說誤陳豨。

十一年春,信復與胡騎入居參合。漢使柴將軍擊之,遺信書曰:「陛下寬仁,諸侯雖有叛亡,而後歸,輒復故位號,不誅也。大王所知。今王以敗亡走胡,非有大罪,急自歸。」信報曰:「陛下擢僕閭巷,南面稱孤,此僕之幸也。滎陽之事,僕不能死,囚於項籍,此一罪也。寇攻馬邑,僕不能堅守,以城降之,此二罪也。今為反寇,將兵與將軍爭一旦之命,此三罪也。夫種、蠡無一罪,身死亡;僕有三罪,而欲求活,此伍子胥所以僨於吳世也。今僕亡匿山谷間,旦暮乞貣蠻夷,僕之思歸,如痿人不忘起,盲者不忘視,勢不可耳。」遂戰。柴將軍屠參合,斬信。

信之入匈奴,與太子俱,及至頹當城,生子,因名曰頹當。韓太子亦生子嬰。至孝文時,頹當及嬰率其眾降。漢封頹當為弓高侯,嬰為襄城侯。吳楚反時,弓高侯功冠諸將。傳子至孫,孫無子,國絕。嬰孫以不敬失侯。穨當孽孫嫣,貴幸,名顯當世。嫣弟說,以校尉擊匈奴,封龍哣侯。後坐酎金失侯,復以待詔為橫海將軍,擊破東越,封按道侯。太初中,為游擊將軍屯五原外列城,還為光祿勳,掘蠱太子宮,為太子所殺。子興嗣,坐巫蠱誅。上曰:「游擊將軍死事,無論坐者。」乃復封興弟增為龍镪侯。增少為郎,諸曹侍中光祿大夫,昭帝時至前將軍,與大將軍霍光定策立宣帝,益封千戶。本始二年,五將征匈奴,增將三萬騎出雲中,斬首百餘級,至期而還。神爵元年,代張安世為大司馬車騎將軍,領尚書事。增世貴,幼為忠臣,事三主,重於朝廷。為人寬和自守,以溫顏遜辭承上接下,無所失意,保身固寵,不能有所建明。五鳳二年薨,諡曰安侯。子寶嗣,亡子,國除。成帝時,繼功臣後,封增兄子岑為龍镪侯。薨,子持弓嗣。王莽敗,乃絕。

贊曰:周室既壞,至春秋末,諸侯秏盡,而炎黃唐虞之苗裔尚猶頗有存者。秦滅六國,而上古遺烈埽地盡矣。楚漢之際,豪桀相王,唯魏豹、韓信、田儋兄弟為舊國之後,然皆及身而絕。橫之志節,賓客慕義,猶不能自立,豈非天虖!韓氏自弓高後貴顯,蓋周烈近與!


\end{pinyinscope}