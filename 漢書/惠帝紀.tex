\article{惠帝紀}

\begin{pinyinscope}
孝惠皇帝,高祖太子也,母曰呂皇后。帝年五歲,高祖初為漢王。二年,立為太子。十二年四月,高祖崩。五月丙寅,太子即皇帝位,尊皇后曰皇太后。賜民爵一級。中郎、郎中滿六歲爵三級,四歲二級。外郎滿六歲二級。中郎不滿一歲一級。外郎不滿二歲賜錢萬。宦官尚食比郎中。謁者、執楯、執戟、武士、騶比外郎。太子御驂乘賜爵五大夫,舍人滿五歲二級。賜給喪事者,二千石錢二萬,六百石以上萬,五百石、二百石以下至佐史五千。視作斥上者,將軍四十金,二千石二十金,六百石以上六金,五百石以下至佐史二金。減田租,復十五稅一。爵五大夫、吏六百石以上及宦皇帝而知名者有罪當盜械者,皆頌繫。上造以上及內外公孫耳孫有罪當刑及當為城旦舂者,皆耐為鬼薪白粲。民年七十以上若不滿十歲有罪當刑者,皆完之。又曰:「吏所以治民也,能盡其治則民賴之,故重其祿,所以為民也。今吏六百石以上父母妻子與同居,及故吏嘗佩將軍都尉印將兵及佩二千石官印者,家唯給軍賦,他無有所與。

令郡諸侯王立高廟。

元年冬十二月,趙隱王如意薨。民有罪,得買爵三十級以免死罪。賜民爵,戶一級。

春正月,城長安。

二年冬十月,齊悼惠王來朝,獻城陽郡以益魯元公主邑,尊公主為太后。

春正月癸酉,有兩龍見蘭陵家人井中,乙亥夕而不見。隴西地震。

夏旱。郃陽侯仲薨。秋七月辛未,相國何薨。

三年春,發長安六百里內男女十四萬六千人城長安,三十日罷。

以宗室女為公主,嫁匈奴單于。

夏五月,立閩越君搖為東海王。

六月,發諸侯王、列侯徒隸二萬人城長安。

秋七月,都廄災。南越王趙佗稱臣奉貢。

四年冬十月壬寅,立皇后張氏。

春正月,舉民孝弟力田者復其身。

三月甲子,皇帝冠,赦天下。省法令妨吏民者;除挾書律。長樂宮鴻臺災。宜陽雨血。

秋七月乙亥,未央宮凌室災;丙子,織室災。

五年冬十月,剨;桃李華,棗實。

春正月,復發長安六百里內男女十四萬五千人城長安,三十日罷。

夏,大旱。

秋八月己丑,相國參薨。

九月,長安城成。賜民爵,戶一級。

六年冬十月辛丑,齊王肥薨。

令民得賣爵。女子年十五以上至三十不嫁,五算。

夏六月,舞陽侯噲薨。

起長安西巿,修敖倉。

七年冬十月,發車騎、材官詣滎陽,太尉灌嬰將。

春正月辛丑朔,日有蝕之。夏五月丁卯,日有蝕之,既。

秋八月戊寅,帝崩于未央宮。九月辛丑,葬安陵。

贊曰:孝惠內修親親,外禮宰相,優寵齊悼、趙隱,恩敬篤矣。聞叔孫通之諫則懼然,納曹相國之對而心說,可謂寬仁之主。遭呂太后虧損至德,悲夫!


\end{pinyinscope}