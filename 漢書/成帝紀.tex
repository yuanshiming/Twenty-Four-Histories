\article{成帝紀}

\begin{pinyinscope}
孝成皇帝,元帝太子也。母曰王皇后,元帝在太子宮生甲觀畫堂,為世嫡皇孫。宣帝愛之,字曰太孫,常置左右。年三歲而宣帝崩,元帝即位,帝為太子。壯好經書,寬博謹慎。初居桂宮,上嘗急召,太子出龍樓門,不敢絕馳道,西至直城門,得絕乃度,還入作室門。上遲之,問其故,以狀對。上大說,乃著令,令太子得絕馳道云。其後幸酒,樂燕樂,上不以為能。而定陶恭王有材藝,母傅昭儀又愛幸,上以故常有意欲以恭王為嗣。賴侍中史丹護太子家,輔助有力,上亦以先帝尤愛太子,故得無廢。

竟寧元年五月,元帝崩。六月己未,太子即皇帝位,謁高廟。尊皇太后曰太皇太后,皇后曰皇太后。以元舅侍中衛尉陽平侯王鳳為大司馬大將軍,領尚書事。

乙未,有司言:「乘輿車、牛馬、禽獸皆非禮,不宜以葬。」奏可。

七月,大赦天下。

建始元年春正月乙丑,皇曾祖悼考廟災。

立故河間王弟上郡庫令良為王。

有星孛于營室。

罷上林詔獄。

二月,右將軍長史姚尹等使匈奴還,去塞百餘里,暴風火發,燒殺尹等七人。

賜諸侯王、丞相、將軍、列侯、王太后、公主、王主、吏二千石黃金,宗室諸官吏千石以下至二百石及宗室子有屬籍者、三老、孝弟力田、鰥寡孤獨錢帛,各有差,吏民五十戶牛酒。

詔曰:「乃者火災降於祖廟,有星孛于東方,始正而虧,咎孰大焉!《書》云:『惟先假王正厥事。』群公孜孜,帥先百寮,輔朕不逮。崇寬大,長和睦,凡事恕己,毋行苛刻。其大赦天下,使得自新。」

封舅諸吏光祿大夫關內侯王崇為安成侯。賜舅王譚、商、立、根、逢時爵關內侯。

夏四月,黃霧四塞,博問公卿大夫,無有所諱。六月,有青蠅無萬數集未央宮殿中朝者坐。

秋,罷上林宮館希御幸者二十五所。

八月,有兩月相承,晨見東方。

九月戊子,流星光燭地,長四五丈,委曲蛇形,貫紫宮。

十二月,作長安南北郊,罷甘泉、汾陰祠。是日大風,拔甘泉畤中大木十韋以上。郡國被災什四以上,毋收田租。

二年春正月,罷雍五畤。辛巳,上始郊祀長安南郊。詔曰:「乃者徙泰畤、后土于南郊、北郊,朕親飭躬,郊祀上帝。皇天報應,神光並見。三輔長無共張繇役之勞,赦奉郊縣長安、長陵及中都官耐罪徒。減天下賦錢,算四十。」

閏月,以渭城延陵亭部為初陵。

二月,詔三輔內郡舉賢良方正各一人。

三月,北宮井水溢出。

辛丑,上始祠后土于北郊。

丙午,立皇后許氏。

罷六廄、技巧官。

夏,大旱。

東平王宇有罪,削樊、亢父縣。

秋,罷太子博望苑,以賜宗室朝請者。減乘輿廄馬。

三年春三月,赦天下徒。賜孝弟力田爵二級。諸逋租賦所振貸勿收。

秋,關內大水。七月,虒上小女陳持弓聞大水至,走入橫城門,闌入尚方掖門,至未央宮鉤盾中。吏民驚上城。九月,詔曰:「乃者郡國被水災,流殺人民,多至千數。京師無故訛言大水至,吏民驚恐,奔走乘城。殆苛暴深刻之吏未息,元元冤失職者眾。遣諫大夫林等循行天下。」

冬十二月戊申朔,日有蝕之。夜,地震未央宮殿中。詔曰:「蓋聞天生眾民,不能相治,為之立君以統理之。君道得,則草木昆蟲咸得其所;人君不德,謫見天地,災異婁發,以告不治。朕涉道日寡,舉錯不中,乃戊申日蝕地震,朕甚懼焉。公卿其各思朕過失,明白陳之。『女無面從,退有後言。』丞相、御史與將軍、列侯、中二千石及內郡國舉賢良方正能直言極諫之士,詣公車,朕將覽焉。」

越嶲山崩。

四年春,罷中書宦官,初置尚書員五人。

夏四月,雨雪。

五月,中謁者丞陳臨殺司隸校尉轅豐於殿中。

秋,桃李實。大水,河決東郡金隄。冬十月,御史大夫尹忠以河決不憂職,自殺。

河平元年春三月,詔曰:「河決東郡,流漂二州,校尉王延世隄塞輒平,其改元為河平。賜天下吏民爵,各有差。」

夏四月己亥晦,日有蝕之,既。詔曰:「朕獲保宗廟,戰戰栗栗,未能奉稱。傳曰:『男教不修,陽事不得,則日為之蝕。』天著厥異,辜在朕躬。公卿大夫其勉悉心,以輔不逮。百寮各修其職,惇任仁人,退遠殘賊。陳朕過失,無有所諱。」大赦天下。

六月,罷典屬國并大鴻臚。

秋九月,復太上皇寢廟園。

二年春正月,沛郡鐵官冶鐵飛。語在五行志。

夏六月,封舅譚、商、立、根、逢時皆為列侯。

三年春二月丙戌,犍為地震山崩,雍江水,水逆流。

秋八月乙卯晦,日有蝕之。

光祿大夫劉向校中祕書。謁者陳農使,使求遺書於天下。

四年春正月,匈奴單于來朝。

赦天下徒,賜孝弟力田爵二級,諸逋租賦所振貸勿收。

二月,單于罷歸國。

三月癸丑朔,日有蝕之。

遣光祿大夫博士嘉等十一人行舉瀕河之郡水所毀傷困乏不能自存者,財振貸。其為水所流壓死,不能自葬,令郡國給槥櫝葬埋。已葬者與錢,人二千。避水它郡國,在所冗食之,謹遇以文理,無令失職。舉惇厚有行能直言之士。

壬申,長陵臨涇岸崩,雍涇水。

夏六月庚戌,楚王囂薨。

山陽火生石中,改元為陽朔。

陽朔元年。

春二月丁未晦,日有蝕之。

三月,赦天下徒。

冬,京兆尹王章有罪,下獄死。

二年春,寒。詔曰:「昔在帝堯立羲、和之官,命以四時之事,令不失其序。故《書》云『黎民於蕃時雍』,明以陰陽為本也。今公卿大夫或不信陰陽,薄而小之,所奏請多違時政。傳以不知,周行天下,而欲望陰陽和調,豈不謬哉!其務順四時月令。」

三月,大赦天下。

夏五月,除吏八百石、五百石秩。

秋,關東大水,流民欲入函谷、天井、壺口、五阮關者,勿苛留。遣諫大夫博士分行視。

八月甲申,定陶王康薨。

九月,奉使者不稱。詔曰:「古之立太學,將以傳先王之業,流化於天下也。儒林之官,四海淵原,宜皆明於古今,溫故知新,通達國體,故謂之博士。否則學者無述焉,為下所輕,非所以尊道德也。『工欲善其事,必先利其器。』丞相、御史其與中二千石、二千石雜舉可充博士位者,使卓然可觀。」

是歲,御史大夫張忠卒。

三年春三月壬戌,隕石東郡,八。

夏六月,潁川鐵官徒申屠聖等百八十人殺長吏,盜庫兵,自稱將軍,經歷九郡。遣丞相長史、御史中丞逐捕,以軍興從事,皆伏辜。

秋八月丁巳,大司馬大將軍王鳳薨。

四年春正月,詔曰:「夫洪範八政,以食為首,斯誠家給刑錯之本也。先帝劭農,薄其租稅,寵其彊力,令與孝弟同科。間者,民彌惰怠,鄉本者少,趨末者眾,將何以矯之?方東作時,其令二千石勉勸農桑,出入阡陌,致勞來之。書不云乎?『服田力嗇,乃亦有秋。』其勗之哉!」

二月,赦天下。

秋九月壬申,東平王宇薨。

閏月壬戌,御史大夫于永卒。

鴻嘉元年春二月,詔曰:「朕承天地,獲保宗廟,明有所蔽,德不能綏,刑罰不中,眾冤失職,趨闕告訴者不絕。是以陰陽錯謬,寒暑失序,日月不光,百姓蒙辜,朕甚閔焉。書不云乎?『即我御事,罔克耆壽,咎在厥躬。』方春生長時,臨遣諫大夫理等舉三輔、三河、弘農冤獄。公卿大夫、部刺史明申敕守相,稱朕意焉。其賜天下民爵一級,女子百戶牛酒,加賜鰥寡孤獨高年帛。逋貸未入者勿收。」

壬午,行幸初陵,赦作徒。以新豐戲鄉為昌陵縣,奉初陵,賜百戶牛酒。

上始為微行出。

冬,黃龍見真定。

二年春,行幸雲陽。

三月,博士行飲酒禮,有雉蜚集于庭,歷階升堂而雊,後集諸府,又集承明殿。

詔曰:「古之選賢,傅納以言,明試以功,故官無廢事,下無逸民,教化流行,風雨和時,百穀用成,眾庶樂業,咸以康寧。朕承鴻業十有餘年,數遭水旱疾疫之災,黎民婁困於飢寒,而望禮義之興,豈不難哉!朕既無以率道,帝王之道日以陵夷,意乃招賢選士之路鬱滯而不通與,將舉者未得其人也?其舉敦厚有行義能直言者,冀聞切言嘉謀,匡朕之不逮。」

夏,徙郡國豪傑貲五百萬以上五千戶于昌陵。賜丞相、御史、將軍、列侯、公主、中二千石冢地、第宅。

六月,立中山憲王孫雲客為廣德王。

三年夏四月,赦天下。令吏民得買爵,賈級千錢。

大旱。

秋八月乙卯,孝景廟闕災。

冬十一月甲寅,皇后許氏廢。

廣漢男子鄭躬等六十餘人攻官寺,篡囚徒,盜庫兵,自稱山君。

四年春正月,詔曰:「數敕有司,務行寬大,而禁苛暴,訖今不改。一人有辜,舉宗拘繫,農民失業,怨恨者眾,傷害和氣,水旱為災,關東流冗者眾,青、幽、冀部尤劇,朕甚痛焉。未聞在位有惻然者,孰當助朕憂之!已遣使者循行郡國。被災害什四以上,民貲不滿三萬,勿出租賦。逋貸未入,皆勿收。流民欲入關,輒籍內。所之郡國,謹遇以理,務有以全活之,思稱朕意。」

秋,勃海、清河河溢,被災者振貸之。

冬,廣漢鄭躬等黨與寖廣,犯歷四縣,眾且萬人。拜河東都尉趙護為廣漢太守,發郡中及蜀郡合三萬人擊之。或相捕斬,除罪。旬月平,遷護為執金吾,賜黃金百斤。

永始元年春正月癸丑,太官凌室火。戊午,戾后園闕火。

夏四月,封婕妤趙氏父臨為成陽侯。五月,封舅曼子侍中騎都尉光祿大夫王莽為新都侯。六月丙寅,立皇后趙氏。大赦天下。

秋七月,詔曰:「朕執德不固,謀不盡下,過聽將作大匠萬年言昌陵三年可成。作治五年,中陵、司馬殿門內尚未加功。天下虛耗,百姓罷勞,客土疏惡,終不可成。朕惟其難,怛然傷心。夫『過而不改,是謂過矣。』其罷昌陵,及故陵勿徙吏民,令天下毋有動搖之心。」立城陽孝王子俚為王。

八月丁丑,太皇太后王氏崩。

二年春正月己丑,大司馬車騎將軍王音薨。

二月癸未夜,星隕如雨。乙酉晦,日有蝕之。詔曰:「乃者,龍見于東萊,日有蝕之。天著變異,以顯朕郵,朕甚懼焉。公卿申敕百寮,深思天誡,有可省減便安百姓者,條奏。所振貸貧民,勿收。」又曰:「關東比歲不登,吏民以義收食貧民、入穀物助縣官振贍者,已賜直,其百萬以上,加賜爵右更,欲為吏補三百石,其吏也遷二等。三十萬以上,賜爵五大夫,吏亦遷二等,民補郎。十萬以上,家無出租賦三歲。萬錢以上,一年。」

冬十一月,行幸雍,祠五畤。

十二月,詔曰:「前將作大匠萬年知昌陵卑下,不可為萬歲居,奏請營作,建置郭邑,妄為巧詐,積土增高,多賦斂繇役,興卒暴之作。卒徒蒙辜,死者連屬,百姓罷極,天下匱竭。常侍閎前為大司農中丞,數奏昌陵不可成。侍中衛尉長數白宜早止,徙家反故處。朕以長言下閎章,公卿議者皆合長計。首建至策,閎典主省大費,民以康寧。閎前賜爵關內侯,黃金百斤。其賜長爵關內侯,食邑千戶,閎五百戶。萬年佞邪不忠,毒流眾庶,海內怨望,至今不息,雖蒙赦令,不宜居京師。其徙萬年敦煌郡。」

是歲,御史大夫王駿卒。

三年春正月己卯晦,日有蝕之。詔曰:「天災仍重,朕甚懼焉。惟民之失職,臨遣大中大夫嘉等循行天下,存問耆老,民所疾苦。其與部刺史舉惇樸遜讓有行義者各一人。」

冬十月庚辰,皇太后詔有司復甘泉泰畤、汾陰后土、雍五畤、陳倉陳寶祠。語在郊祀志。

十一月,尉氏男子樊並等十三人謀反,殺陳留太守,劫略吏民,自稱將軍。徒李譚等五人共格殺並等,皆封為列侯。

十二月,山陽鐵官徒蘇令等二百二十八人攻殺長吏,盜庫兵,自稱將軍,經歷郡國十九,殺東郡太守、汝南都尉。遣丞相長史、御史中丞持節督趣逐捕。汝南太守嚴訢捕斬令等。遷訢為大司農,賜黃金百斤。

四年春正月,行幸甘泉,郊泰畤,神光降集紫殿。大赦天下。賜雲陽吏民爵,女子百戶牛酒,鰥寡孤獨高年帛。三月,行幸河東,祠后土,賜吏民如雲陽,行所過無出田租。

夏四月癸未,長樂臨華殿、未央宮東司馬門皆災。

六月甲午,霸陵園門闕災。出杜陵諸未嘗御者歸家。詔曰:「乃者,地震京師,火災婁降,朕甚懼之。有司其悉心明對厥咎,朕將親覽焉。」

又曰:「聖王明禮制以序尊卑,異車服以章有德,雖有其財,而無其尊,不得踰制,故民興行,上義而下利。方今世俗奢僭罔極,靡有厭足。公卿列侯親屬近臣,四方所則,未聞修身遵禮,同心憂國者也。或乃奢侈逸豫,務廣第宅,治園池,多畜奴婢,被服綺穀,設鐘鼓,備女樂,車服嫁娶葬埋過制。吏民慕效,娅以成俗,而欲望百姓儉節,家給人足,豈不難哉!詩不云乎?『赫赫師尹,民具爾瞻。』其申敕有司,以漸禁之。青綠民所常服,且勿止。列侯近臣,各自省改。司隸校尉察不變者。」

秋七月辛未晦,日有蝕之。

元延元年春正月己亥朔,日有蝕之。

三月,行幸雍,祠五畤。

夏四月丁酉,無雲有雷,聲光耀耀,四面下至地,昏止。赦天下。

秋七月,有星孛于東井。詔曰:「乃者,日蝕星隕,謫見于天,大異重仍。在位默然,罕有忠言。今孛星見于東井,朕甚懼焉。公卿大夫、博士、議郎其各悉心,惟思變意,明以經對,無有所諱;與內郡國舉方正能直言極諫者各一人,北邊二十二郡舉勇猛知兵法者各一人。」

封蕭相國後喜為酇侯。

冬十二月辛亥,大司馬大將軍王商薨。

是歲,昭儀趙氏害後宮皇子。

二年春正月,行幸甘泉,郊泰畤。

三月,行幸河東,祠后土。

夏四月,立廣陵孝王子守為王。

冬,行幸長楊宮,從胡客大校獵。宿萯陽宮,賜從官。

三年春正月丙寅,蜀郡岷山崩,雍江三日,江水竭。

二月,封侍中衛尉淳于長為定陵侯。

三月,行幸雍,祠五畤。

四年春正月,行幸甘泉,郊泰畤。

二月,罷司隸校尉官。

三月,行幸河東,祠后土。

甘露降京師,賜長安民牛酒。

綏和元年春正月,大赦天下。

二月癸丑,詔曰:「朕承太祖鴻業,奉宗廟二十五年,德不能綏理宇內,百姓怨望者眾。不蒙天祐,至今未有繼嗣,天下無所係心。觀于往古近事之戒,禍亂之萌,皆由斯焉。定陶王欣於朕為子,慈仁孝順,可以承天序,繼祭祀。其立欣為皇太子。封中山王舅諫大夫馮參為宜鄉侯,益中山國三萬戶,以慰其意。賜諸侯王、列侯金,天下當為父後者爵,三老、孝弟力田帛,各有差。」

又曰:「蓋聞王者必存二王之後,所以通三統也。昔成湯受命,列為三代,而祭祀廢絕。考求其後,莫正孔吉。其封吉為殷紹嘉侯。」三月,進爵為公,及周承休侯皆為公,地各百里。

行幸雍,祠五畤。

夏四月,以大司馬票騎大將軍根為大司馬,罷將軍官。御史大夫為大司空,封為列侯。益大司馬、大司空奉如丞相。

秋八月庚戌,中山王興薨。

冬十一月,立楚孝王孫景為定陶王。

定陵侯淳于長大逆不道,下獄死。廷尉孔光使持節賜貴人許氏藥,飲藥死。

十二月,罷部刺史,更置州牧,秩二千石。

二年春正月,行幸甘泉,郊泰畤。

二月壬子,丞相翟方進薨。

三月,行幸河東,祠后土。

丙戌,帝崩于未央宮。皇太后詔有司復長安南北郊。四月己卯,葬延陵。

贊曰:臣之姑充後宮為婕妤,父子昆弟侍帷幄,數為臣言成帝善修容儀,升車正立,不內顧,不疾言,不親指,臨朝淵嘿,尊嚴若神,可謂穆穆天子之容者矣!博覽古今,容受直辭。公卿稱職,奏議可述。遭世承平,上下和睦。然湛于酒色,趙氏亂內,外家擅朝,言之可為於邑。建始以來,王氏始執國命,哀、平短祚,莽遂篡位,蓋其威福所由來者漸矣!


\end{pinyinscope}